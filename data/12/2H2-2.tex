\setcounter{section}{1}
\Opensolutionfile{ans}[ans/ans2H2-2]
\section{MẶT CẦU}
\subsection{Kiến thức sách giáo khoa cần cần nắm}
\subsubsection{Định nghĩa.}
\begin{dn}
	Tập hợp các điểm $M$ trong không gian cách điểm $O$ cố định một khoảng $R$ gọi là mặt cầu tâm $O$, bán kính $R$, kí hiệu là: $S(O;R)$. Khi đó $S(O;R)=\{M|OM=R\}$.
\end{dn}
\subsubsection{Vị trí tương đối của một điểm đối với mặt cầu.}
Cho mặt cầu $S(O;R)$ và một điểm $A$ bất kì, khi đó:
\immini{
	\begin{itemize}
		\item Nếu $OA=R\Leftrightarrow A\in S(O;R)$. Khi đó $OA$ gọi là bán kính mặt cầu. Nếu $OA$ và $OB$ là hai bán kính sao cho $\overrightarrow{OA}=-\overrightarrow{OB}$ thì đoạn thẳng $AB$ gọi là một đường kính của mặt cầu.
		\item Nếu $OA<R\Leftrightarrow A$ nằm trong mặt cầu.
		\item Nếu $OA>R\Leftrightarrow A$ nằm ngoài mặt cầu.\\
		$ \Rightarrow $ Khối cầu $S(O;R)$ là tập hợp tất cả các điểm $M$ sao cho $OM\leq R$.\\
	\end{itemize}
}
{
	\begin{tikzpicture}[scale=1, font=\footnotesize, line join=round, line cap=round, >=stealth]
	\def\R{2}
	\def\r{\R/2}
	\coordinate[label=above left:$O$] (O) at (0,0);
	\coordinate (A') at (\R,0);
	\draw (O) circle (\R) (A') arc (0:-180: {\R} and {\r});
	\draw[dashed] (A') arc (0:180: {\R} and {\r});
	\coordinate (B) at ($(0,0)+(60:{\R} and {\r})$);
	\coordinate (A1) at ($(0,0)+(240:{\R} and {\r})$);
	\coordinate (A2) at (0.75,-0.4);
	\coordinate (A3) at (-1,-1.4);
	\draw[dashed] (A1)--(B);
	\foreach \x/\g in {A1/90,A2/-45,A3/-45}\fill[black] (\x) circle (1pt)+(\g:3mm) node{$A$};
	\fill[black] (B) circle (1pt) node[above]{$B$};
	\fill[black] (O) circle (1pt);
	\end{tikzpicture}
}
\subsubsection{Vị trí tương đối của mặt phẳng và mặt cầu.}
Cho mặt cầu $S(O;R)$ và một mp $(P)$. Gọi $d$ là khoảng cách từ tâm $O$ của mặt cầu đến mp $(P)$ và $H$ là hình chiếu của $O$ trên mp $(P)\Rightarrow \mathrm{d}=OH$.
\begin{itemize}
	\item Nếu $d<R\Leftrightarrow \text{mp }(P)$ cắt mặt cầu $S(O;R)$ theo giao tuyến là đường tròn nằm trên mp $(P)$ có tâm là $H$ và bán kính $r=HM=\sqrt{R^2-d^2}=\sqrt{R^2-OH^2}$ (hình a).
	\item Nếu $d>R\Leftrightarrow \text{mp }(P)$ không cắt mặt cầu $S(O;R)$ (hình b).
	\item Nếu $d=R\Leftrightarrow \text{mp }(P)$ có một điểm chung duy nhất. Ta nói mặt cầu $S(O;R)$ tiếp xúc mp $(P)$. Do đó, điều kiện cần và đủ để mp $(P)$ tiếp xúc với mặt cầu $S(O;R)$ là $\mathrm{d}\left(O,(P)\right)=R$ (hình c).
\end{itemize}
\begin{center}
	\begin{tabular}{c c c }
		\begin{tikzpicture}[>=stealth,line join=round,line cap=round,font=\footnotesize,scale=1]
		\def\a{5}
		\def\b{0.75}
		\def\h{2}
		\def\r{2}
		\def\hdt{2}
		\path
		(0,0) coordinate (A)
		(\a,0) coordinate (B)
		(\b,\h) coordinate (C)
		($(B)+(C)-(A)$) coordinate (D)
		($(B)!.5!(C)$) coordinate (I)
		($(I)+(0,\hdt)$) coordinate (O)
		($(O)+(-40:\r)$) coordinate (M)
		($(O)+(-140:\r)$) coordinate (N)
		($(M)!.5!(N)$) coordinate (H)
		;
		\draw (O) circle[radius=\r cm];
		\tkzCalcLength[cm](M,N) \tkzGetLength{dtMN}
		\draw[dashed] (M) arc (0:180:{\dtMN/2} and {\dtMN/8});
		\draw (M) arc (0:-180:{\dtMN/2} and {\dtMN/8});
		\draw[dashed]
		(O)--(H)--(M)--cycle
		;
		\draw (B)--(A)--(C);
		\draw pic[draw,angle radius=6mm] {angle = B--A--C};
		\path
		(A) node[above right] {$P$}
		($(O)!.5!(H)$) node[left] {$d$}
		($(O)!.5!(M)$) node[above] {$R$}
		;
		\foreach \x/\g in {O/90,H/-135,M/-45}\fill[black] (\x) circle (1pt)+(\g:3mm) node{$ \x $};
		\end{tikzpicture}
		&
		\begin{tikzpicture}[>=stealth,line join=round,line cap=round,font=\footnotesize,scale=1]
		\def\a{5}
		\def\b{0.75}
		\def\h{2}
		\def\r{2}
		\def\hdt{3.5}
		\path
		(0,0) coordinate (A)
		(\a,0) coordinate (B)
		(\b,\h) coordinate (C)
		($(B)+(C)-(A)$) coordinate (D)
		($(B)!.5!(C)$) coordinate (I)
		($(I)+(0,\hdt)$) coordinate (O)
		($(O)+(-40:\r)$) coordinate (M)
		($(O)+(-140:\r)$) coordinate (N)
		($(M)!.5!(N)$) coordinate (H)
		;
		\draw (O) circle[radius=\r cm];
		\draw
		(B)--(A)--(C)--(D)--cycle
		(I)--(O)--(N)
		;
		\draw pic[draw,angle radius=6mm] {angle = B--A--C};
		\path
		(A) node[above right] {$P$}
		;
		\fill[black] (N);
		\fill[black] (O) circle (1pt)+(90:3mm) node{$O$};
		\fill[black] (I) circle (1pt)+(-90:3mm) node{$H$};
		\path
		($(O)!.5!(I)$) node[right] {$d$}
		($(O)!.5!(N)$) node[above left] {$R$}
		;
		\end{tikzpicture}
		&
		\begin{tikzpicture}[>=stealth,line join=round,line cap=round,font=\footnotesize,scale=1]
		\def\a{5}
		\def\b{0.75}
		\def\h{2}
		\def\r{2}
		\def\hdt{2}
		\path
		(0,0) coordinate (A)
		(\a,0) coordinate (B)
		(\b,\h) coordinate (C)
		($(B)+(C)-(A)$) coordinate (D)
		($(B)!.5!(C)$) coordinate (I)
		($(I)+(0,\hdt)$) coordinate (O)
		($(O)+(-40:\r)$) coordinate (M)
		($(O)+(-140:\r)$) coordinate (N)
		($(M)!.5!(N)$) coordinate (H)
		;
		\draw (O) circle[radius=\r cm];
		\draw
		(I)--(O)
		;
		\draw pic[draw,angle radius=6mm] {angle = B--A--C};
		\path
		(A) node[above right] {$P$}
		;
		\path[name path=cd] (C)--(D);
		\path[name path=circleO] (O)circle(\r cm);
		\path[name intersections={of= cd and circleO}] coordinate (E) at (intersection-1) coordinate (F) at (intersection-2);
		\draw (C)--(A)--(B)--(D) (C)--(E) (F)--(D);
		\draw[dashed] (E)--(F);
		\fill[black] (N);
		\fill[black] (O) circle (1pt)+(90:3mm) node{$O$};
		\fill[black] (I) circle (1pt)+(-90:3mm) node{$H$};
		\path
		($(O)!.5!(I)$) node[above left] {$d=R$}
		;
		\end{tikzpicture}
		\\
		Hình a& Hình b& Hình c\\
	\end{tabular}
\end{center}
\subsubsection{Vị trí tương đối của đường thẳng và mặt cầu.}
\begin{dl}
	Cho mặt cầu $S(O;R)$ và một đường thẳng $\Delta$. Gọi $H$ là hình chiếu của $O$ trên đường thẳng $\Delta$ và $d=OH$ là khoảng cách từ tâm $O$ của mặt cầu đến đường thẳng $\Delta$. Khi đó
	\begin{itemize}
		\item Nếu $d>R\Leftrightarrow\Delta$ không cắt mặt cầu $S(O;R)$.
		\item Nếu $d<R\Leftrightarrow\Delta$ cắt mặt cầu $S(O;R)$ tại hai điểm phân biệt.
		\item Nếu $d=R\Leftrightarrow\Delta$ và mặt cầu tiếp xúc nhau (tại một điểm duy nhất). Do đó: điều kiện cần và đủ để đường thẳng $\Delta$ tiếp xúc với mặt cầu là $d=\mathrm{d}(O,\Delta)=R$.
	\end{itemize}
\end{dl}
\begin{dl}
	Nếu điểm $A$ nằm ngoài mặt cầu $S(O;R)$ thì:
	\begin{itemize}
		\item Qua $A$ có vô số tiếp tuyến với mặt cầu $S(O;R)$.
		\item Độ dài đoạn thẳng nối $A$ với các tiếp điểm đều bằng nhau.
		\item Tập hợp các điểm này là một đường tròn nằm trên mặt cầu $S(O;R)$.
	\end{itemize}
\end{dl}
\subsubsection{Diện tích và thể tích mặt cầu.}
\begin{itemize}
	\item Diện tích mặt cầu: $S_C=4\pi R^2$.
	\item Thể tích mặt cầu: $V_C=\dfrac{4}{3}\pi R^3$.
\end{itemize}
\subsubsection{Mặt cầu ngoại tiếp khối đa diện}
\begin{enumerate}
\item Các khái niệm cơ bản
\begin{itemize}
	\item \textbf{Trục của đa giác đáy:} là đường thẳng đi qua tâm đường tròn ngoại tiếp của đa giác đáy và vuông góc với mặt phẳng chứa đa giác đáy.\\
	$ \Rightarrow $ Bất kì một điểm nào nằm trên trục của đa giác thì cách đều các đỉnh của đa giác đó.
	\item \textbf{Đường trung trực của đoạn thẳng:} là đường thẳng đi qua trung điểm của đoạn thẳng và vuông góc với đoạn thẳng đó.\\
	$ \Rightarrow $ Bất kì một điểm nào nằm trên đường trung trực thì cách đều hai đầu mút của đoạn thẳng.
	\item \textbf{Mặt trung trực của đoạn thẳng:} là mặt phẳng đi qua trung điểm của đoạn thẳng và vuông góc với đoạn thẳng đó.\\
	$ \Rightarrow $ Bất kì một điểm nào nằm trên mặt trung trực thì cách đều hai đầu mút của đoạn thẳng.
\end{itemize}
\item Tâm và bán kính mặt cầu ngoại tiếp hình chóp
\begin{itemize}
	\item \textbf{Tâm mặt cầu ngoại tiếp hình chóp:} là điểm cách đều các đỉnh của hình chóp. Hay nói cách khác, nó chính là giao điểm $ I $ của trục \textit{đường tròn ngoại tiếp mặt phẳng đáy} và \textit{mặt phẳng trung trực của một cạnh bên} hình chóp.
	\item \textbf{Bán kính:} là khoảng cách từ $ I $ đến các đỉnh của hình chóp.
\end{itemize}
\item Cách xác định tâm và bán kính mặt cầu của một số hình đa diện cơ bản
\begin{enumerate}[1)]
	\item \textbf{Hình hộp chữ nhật, hình lập phương.}
	\immini
	{ \begin{itemize}
			\item \textbf{Tâm:} trùng với tâm đối xứng của hình hộp chữ nhật (hình lập phương)
			$ \Rightarrow $ Tâm là $I$, là trung điểm của $AC’$.
			\item \textbf{Bán kính:} bằng nửa độ dài đường chéo hình hộp chữ nhật (hình lập phương)\\
			$ \Rightarrow $ Bán kính: $R=\dfrac{AC’}{2}$.
		\end{itemize}
	}
	{
		\begin{tikzpicture}[>=stealth,line join=round,line cap=round,font=\footnotesize,scale=.8]
		\def\ab{3}
		\def\ad{2}
		\def\gd{-130}
		\def\aa{2.5}
		\path
		(0,0) coordinate (A')
		(0:\ab) coordinate (B')
		(\gd:\ad) coordinate (D')
		($ (B')+(D')-(A') $) coordinate (C')
		($ (A')+(90:\aa) $) coordinate (A)
		($ (B')+(90:\aa) $) coordinate (B)
		($ (C')+(90:\aa) $) coordinate (C)
		($ (D')+(90:\aa) $) coordinate (D)
		($ (A)!.5!(C') $) coordinate (I);
		\draw (D)--(D')--(C')--(B')--(B)--(A)--(D)--(C)--(B) (C)--(C');
		\draw[dashed] (D')--(A')--(B') (A')--(A)--(C');
		\foreach \x/\g in {A/90,B/0,C/-30,D/180,A'/180,B'/0,C'/-90,D'/-90,I/200}\fill[black] (\x) circle (1pt)+(\g:3mm) node{$ \x $};
		\end{tikzpicture}
	}
	\item \textbf{Hình lăng trụ đứng có đáy nội tiếp đường tròn.}
	\immini
	{
		Xét hình lăng trụ đứng $A_1A_2A_3\cdots A_n\cdot A_1’A_2’A_3’\cdots A_n’$, trong đó có 2 đáy $A_1A_2A_3\cdots A_n$ và $A_1’A_2’A_3’\cdots A_n’$ nội tiếp đường tròn $(O)$ và $(O’)$. Lúc đó, mặt cầu nội tiếp hình lăng trụ đứng có:
		\begin{itemize}
			\item \textbf{Tâm:} $I$ với $I$ là trung điểm của $OO’$.
			\item \textbf{Bán kính:} $R=IA_1=IA_2=\cdots =IA_n’$.
		\end{itemize}
	}
	{
		\begin{tikzpicture}[scale=.6, font=\footnotesize, line join=round, line cap=round, >=stealth]
		\coordinate[label=left:$O$] (O) at (0,0);
		\coordinate[label=above left:$A_1$] (A1) at ($(O)+(160:3)$);
		\coordinate[label=left:$A_2$] (A2) at ($(O)+(-150:2)$);
		\coordinate[label=below right:$A_3$] (A3) at ($(O)+(-50:2)$);
		\coordinate[label=above right:$A_n$] (An) at ($(O)+(30:3)$);
		
		\coordinate[label=above left:$O'$] (O') at ($(O)+(-90:4)$);
		\coordinate[label=above left:$A'_1$] (A'1) at ($(A1)+(-90:4)$);
		\coordinate[label=left:$A'_2$] (A'2) at ($(A2)+(-90:4)$);
		\coordinate[label=below right:$A'_3$] (A'3) at ($(A3)+(-90:4)$);
		\coordinate[label=above right:$A'_n$] (A'n) at ($(An)+(-90:4)$);
		\coordinate[label=above left:$I$] (I) at ($(O)!.5!(O')$);
		
		\draw (A1)--(A2)--(A'2)--(A'1)--cycle (A3)--(An)--(A'n)--(A'3)--cycle (A2)--(A3) (A'2)--(A'3);
		\draw[dashed] (A1)--(An) (A'1)--(A'n) (O)--(O');
		\foreach \d in {O,A1,A2,A3,An,O',A'1,A'2,A'3,A'n,I} {\draw[fill=black] (\d)circle(1pt);}
		\end{tikzpicture}
	}
	
	\item \textbf{Hình chóp có các đỉnh nhìn đoạn thẳng nối 2 đỉnh còn lại dưới 1 góc vuông.}\\
	\immini
	{
		Hình chóp $S.ABC$ có $\widehat{SAC}=\widehat{SBC}=90^{\circ}$.
		\begin{itemize}
			\item \textbf{Tâm:} $I$ là trung điểm của $SC$.
			\item \textbf{Bán kính:} $R=\dfrac{SC}{2}=IA=IB=IC$.
		\end{itemize}
		
	}
	{
		\begin{tikzpicture}[scale=.8, font=\footnotesize, line join=round, line cap=round, >=stealth]
		\coordinate[label=left:$A$] (A) at (0,0);
		\coordinate[label=above left:$S$] (S) at ($(A)+(90:4)$);
		\coordinate[label=below left:$B$] (B) at ($(A)+(-60:1.7)$);
		\coordinate[label=right:$C$] (C) at ($(A)+(0:4.5)$);
		\coordinate[label=above right:$I$] (I) at ($(S)!.5!(C)$);
		
		\draw (S)--(A)--(B)--cycle (S)--(C)--(B)--(I);
		\draw[dashed] (I)--(A)--(C);
		\foreach \d in {S,A,B,C,I} {\draw[fill=black] (\d)circle(1pt);}
		\end{tikzpicture}
	}
	\immini
	{
		-Hình chóp $S.ABCD$ có
		$\widehat{SAC}=\widehat{SBC}=\widehat{SDC}=90^{\circ}$.
		\begin{itemize}
			\item \textbf{Tâm:} $I$ là trung điểm của $SC$.
			\item \textbf{Bán kính:} $R=\dfrac{SC}{2}=IA=IB=IC=ID$.
		\end{itemize}
		
	}
	{
		\begin{tikzpicture}[scale=.8, font=\footnotesize, line join=round, line cap=round, >=stealth]
		\coordinate[label=left:$A$] (A) at (0,0);
		\coordinate[label=above left:$S$] (S) at ($(A)+(90:4)$);
		\coordinate[label=below left:$B$] (B) at ($(A)+(-120:2)$);
		\coordinate[label=right:$D$] (D) at ($(A)+(0:5)$);
		\coordinate[label=below right:$C$] (C) at ($(B)+(D)-(A)$);
		\coordinate[label=above right:$I$] (I) at ($(S)!.5!(C)$);
		
		\draw (S)--(B)--(C)--(D)--(S)--(C) (B)--(I);
		\draw[dashed] (S)--(A)--(B) (D)--(A)--(I);
		\foreach \d in {S,A,B,C,D,I} {\draw[fill=black] (\d)circle(1pt);}
		\end{tikzpicture}
	}
	\item \textbf{Hình chóp đều.}\\
	Cho hình chóp đều $S.ABC$.
	\immini
	{
		\begin{itemize}
			\item Gọi $O$ là tâm của đáy $\Rightarrow SO$ là trục của đáy.
			\item Trong mặt phẳng xác định bởi $SO$ và một cạnh bên,\\
			chẳng hạn như $mp(SAO)$, ta vẽ đường trung trực của cạnh $SA$.
			là $\triangle$ cắt $SA$ tại $M$ và cắt $SO$ tại $I\Rightarrow I$ là tâm của mặt cầu.\\
			\item \textbf{Bán kính:}\\
			Ta có: $\triangle SMI\sim\triangle SOA\Rightarrow\dfrac{SM}{SO}=\dfrac{SI}{SA}\Rightarrow$ Bán kính là:\\
			$R=IS=\dfrac{SM\cdot SA}{SO}=\dfrac{SA^2}{2SO}=IA=IB=IC=\cdots$.
		\end{itemize}
	}
	{
		\begin{tikzpicture}[scale=1, font=\footnotesize, line join=round, line cap=round, >=stealth]
		\coordinate[label=right:$O$] (O) at (0,0);
		\coordinate[label=above:$S$] (S) at ($(O)+(90:4)$);
		\coordinate[label=left:$A$] (A) at ($(O)+(165:2)$);
		\coordinate[label=left:$B$] (B) at ($(O)+(-160:1.5)$);
		\coordinate[label=below right:$C$] (C) at ($(O)+(-50:1.5)$);
		\coordinate[label=right:$D$] (D) at ($(O)+(0:2)$);
		\coordinate (E) at ($(O)+(30:1.5)$);
		\coordinate (F) at ($(O)+(140:1.5)$);
		\coordinate[label=left:$M$] (M) at ($(S)!0.5!(A)$);
		\coordinate[label=below left:$I$] (I) at ($(S)!0.6!(O)$);
		\coordinate[label=above:$\Delta$] (N) at ($(I)!2!(M)$);
		
		\draw (S)--(A)--(B)--(C)--(D)--(S)--(C) (S)--(B) (M)--(N);
		\draw[dashed] (F)--(A)--(O) (D)--(E) (S)--(O) (I)--(M);
		\foreach \d in {S,A,B,C,D,O,M,I} {\draw[fill=black] (\d)circle(1pt);}
		\def\gv(#1,#2,#3,#4){\path
			($(#2)!#4 mm!(#1)$) coordinate (#2#1)
			($(#2)!#4 mm!(#3)$) coordinate (#2#3);
			\draw (#2#1)--($(#2#1)+(#2#3)-(#2)$)--(#2#3);}
		\gv(S,O,A,2) \gv(S,M,I,1.5)
		\end{tikzpicture}
	}
\end{enumerate}
\item Kỹ thuật xác định mặt cầu ngoại tiếp hình chóp
Cho hình chóp $S.A_1A_2\ldots A_n$ (thoả mãn điều kiện tồn tại mặt cầu ngoại tiếp). Thông thường, để xác định mặt cầu ngoại tiếp hình chóp ta thực hiện theo ba bước
\immini
{
	\begin{enumerate}[\bf Bước 1.]
		\item Xác định tâm của đường tròn ngoại tiếp đa giác đáy.
		\item Dựng $\triangle$: trục đường tròn ngoại tiếp đa giác đáy.
		\item Lập mặt phẳng trung trực $(\alpha)$ của một cạnh bên.
	\end{enumerate}
	Lúc đó
	\begin{itemize}
		\item Tâm O của mặt cầu: $\triangle\cap mp(\alpha)=\{O\}$.
		\item Bán kính: $R=SA(=SO)$. Tuỳ vào từng trường hợp.
	\end{itemize}
}
{
	\begin{tikzpicture}[>=stealth,line join=round,line cap=round,font=\footnotesize,scale=.8]
	\def\h{4}
	\def\a{3}
	\pgfmathsetmacro{\b}{\a/6}
	\path
	(0,0) coordinate (H)
	(90:\h) coordinate (S)
	(0:\a) coordinate (C)
	(180:\a) coordinate (A)
	($ (H)+(50:\a cm and \b cm) $) coordinate (D)
	($ (H)+(-70:\a cm and \b cm) $) coordinate (B)
	($ (S)!.5!(A) $) coordinate (I)
	($ (I)!1!-90:(S) $) coordinate (K)
	(intersection of I--K and S--H) coordinate (O);
	\fill[black!10] (A)--(B)--(C)--(D)--cycle;
	\draw[dashed] (A) arc (-180:-360:\a cm and \b cm) (H)--(S)--(D)--(C)--(B)--(A)--(D) (A)--(O)--(I);
	\draw (A) arc (-180:0:\a cm and \b cm) (B)--(S)--(A) (C)--(S);
	\draw pic[draw,angle radius=3mm] {right angle = S--H--C};
	\foreach \x/\g in {A/180,B/-90,C/0,D/60,I/180,O/0,H/200,S/90}\fill[black] (\x) circle (1pt)+(\g:3mm) node{$ \x $};
	\end{tikzpicture}
}
\end{enumerate}
\begin{note}{\textit{Lưu ý: Kỹ năng xác định trục đường tròn ngoại tiếp đa giác đáy.}}
	\begin{enumerate}[1)]
		\item \textbf{Trục đường tròn ngoại tiếp đa giác đáy:} là đường thẳng đi qua tâm đường tròn ngoại tiếp đáy và vuông góc với mặt phẳng đáy.
		\immini
		{Tính chất $\forall M\in\triangle\colon MA=MB=MC$.\\
			Suy ra $MA=MB=MC\Leftrightarrow M \in \triangle$.
			\item \textbf{Các bước xác định trục}
			\begin{enumerate}[\bf Bước 1.]
				\item Xác định tâm $ H $ của đường tròn ngoại tiếp đa giác đáy.
				\item Qua $ H $ dựng $\triangle$ vuông góc với mặt phẳng đáy.
		\end{enumerate}}
		{\begin{tikzpicture}[>=stealth,line join=round,line cap=round,font=\footnotesize,scale=.8]
			\def\h{4}
			\def\a{3}
			\pgfmathsetmacro{\b}{\a/4}
			\path
			(0,0) coordinate (H)
			(90:\h) coordinate (M)
			(0:\a) coordinate (C)
			($ (H)+(130:\a cm and \b cm) $) coordinate (A)
			($ (H)+(-110:\a cm and \b cm) $) coordinate (B);
			\foreach \x/\y in {A/M,B/M,C/M}\path (\x)--(\y)node[midway,sloped]{$ | $};
			\fill[black!10] (A)--(B)--(C)--cycle;
			\draw (H) ellipse (\a cm and \b cm) (A)--(M)--(B) (C)--(M);
			\draw[dashed] (A)--(B)--(C)--cycle (A)--(H)--(M) (B)--(H)--(C);
			\draw pic[draw,angle radius=3mm] {right angle = M--H--C};
			\foreach \x/\g in {A/150,B/-120,C/0,M/90,H/-60}\fill[black] (\x) circle (1pt)+(\g:3mm) node{$ \x $};
			\end{tikzpicture}
		}
		\textbf{Ví dụ 1.} Một số trường hợp đặc biệt.
		\begin{enumEX}{3}
			\item Tam giác vuông\\
			\begin{tikzpicture}[>=stealth,line join=round,line cap=round,font=\footnotesize,scale=.6]
			\def\r{3}
			\path
			(0,0) coordinate (H)
			(0:\r) coordinate (C)
			(180:\r) coordinate (B)
			(-40:\r) coordinate (A)
			(90:\r-2) coordinate (E)
			(intersection of H--E and A--B) coordinate (F)
			($ (F)+(-90:\r-2.5) $) coordinate (K);
			\draw (A)--(B)--(C)--cycle (H)--(A) (F)--(K);
			\draw[line width=.6pt] (F)--(K) (E)--(H);
			\draw[dashed,line width=.6pt] (H)--(F);
			\foreach \x/\y/\z in {C/H/E,C/A/B}\draw pic[draw,angle radius =2mm] {right angle = \x--\y--\z};
			\foreach \x/\g in {A/-30,B/90,C/90,H/140}\fill[black] (\x) circle (1pt)+(\g:3mm) node{\tiny $ \x $};
			\end{tikzpicture}
			\item Tam giác đều\\
			\begin{tikzpicture}[>=stealth,line join=round,line cap=round,font=\footnotesize,scale=.6]
			\def\r{3}
			\def\gc{10}
			\path
			(0,0) coordinate (H)
			(\gc:\r) coordinate (C)
			(180-\gc:\r) coordinate (B)
			(-130:\r-.5) coordinate (A)
			(intersection of A--H and C--B) coordinate (K)
			($ (H)+(90:\r-1.5) $) coordinate (E)
			(intersection of E--H and A--C) coordinate (F);
			\draw (A)--(B)--(C)--cycle (B)--(H) (A)--(K) (H)--(C);
			\draw[line width=.6pt] (H)--(E)node[below right]{$\triangle$} (F)--++(-90:1);
			\draw[dashed,line width=.6pt] (H)--(F);
			\foreach \x/\y/\z in {B/H/E,A/K/C}\draw pic[draw,angle radius =2mm] {right angle = \x--\y--\z};
			\foreach \x/\g in {A/-30,B/90,C/90,H/-30}\fill[black] (\x) circle (1pt)+(\g:3mm) node{\tiny $ \x $};
			\end{tikzpicture}
			\item Tam giác bất kì\\
			\begin{tikzpicture}[>=stealth,line join=round,line cap=round,font=\footnotesize,scale=.6]
			\path
			(0,0) coordinate (H)
			(140:2.5) coordinate (B)
			(0:3) coordinate (C)
			(-120:2) coordinate (A)
			($ (B)!.5!(C) $) coordinate (M)
			($ (A)!.5!(C) $) coordinate (N)
			($ (H)+(90:2) $) coordinate (F)
			(intersection of H--F and A--C) coordinate (E);
			\draw (A)--(B)--(C)--cycle (H)--(F)node[left]{$\triangle$};
			\draw[shorten <= -0.5cm, shorten >= -0.5cm] (M)--(H);
			\draw[shorten <= -0.5cm, shorten >= -0.5cm] (N)--(H);
			\draw (E)--++(-90:1);
			\draw[dashed] (H)--(E);
			\foreach \x/\y/\z in {C/M/H,H/N/C}\draw pic[draw,angle radius =1mm] {right angle = \x--\y--\z};
			\foreach \x/\y/\z in {B/M/||,M/C/||,C/N/|,N/A/|}\path (\x)--(\y) node[midway,sloped]{$\z$};
			\foreach \x/\g in {A/-30,B/90,C/90,H/10}\fill[black] (\x) circle (1pt)+(\g:3mm) node{\tiny $ \x $};
			\end{tikzpicture}
		\end{enumEX}
		\immini
		{
			\item \textbf{Lưu ý:} Kỹ năng tam giác đồng dạng\\
			$\triangle SMO$ đồng dạng với $\triangle SIA \Rightarrow \dfrac{SO}{SA}=\dfrac{SM}{SI}$.
		}
		{
			\begin{tikzpicture}[>=stealth,line join=round,line cap=round,font=\footnotesize,scale=.6]
			\path
			(0,0) coordinate (I)
			(0:2) coordinate (A)
			(90:4) coordinate (S)
			($ (A)!.5!(S) $) coordinate (M)
			($ (M)!1!90:(S) $) coordinate (K)
			(intersection of M--K and S--I) coordinate (O);
			\draw (A)--(I)--(S)--cycle (M)--(O);
			\foreach \x/\y in {A/M,S/M}\path (\x)--(\y)node[midway,sloped]{$ || $};
			\foreach \x/\y/\z in {A/M/O,A/I/O}\draw pic[draw,angle radius =2mm] {right angle = \x--\y--\z};
			\foreach \x/\g in {A/0,S/90,I/180,M/60,O/180}\fill[black] (\x) circle (1pt)+(\g:3mm) node{\tiny $ \x $};
			\end{tikzpicture}
		}
		\item \textbf{Nhận xét quan trọng}
		$\exists M$, $S\colon\heva{&MA=MB=MC\\&SA=SB=SC}\Rightarrow SM$ là trục đường tròn ngoại tiếp $\triangle ABC$.
		\item \textbf{Ví dụ:} Tìm tâm và bán kính mặt cầu ngoại tiếp hình chóp.
	\end{enumerate}
\end{note}
\subsection{PHÂN LOẠI VÀ PHƯƠNG PHÁP GIẢI BÀI TẬP}

\begin{dang}{Chóp có các điểm cùng nhìn một đoạn dưới một góc vuông}
\end{dang}
\begin{vd}%Ví dụ 1.%[2H2B2-2]
	Cho $S.ABC$ có $SA \perp (ABC)$ và $\triangle ABC$ vuông tại $B$. Xác định tâm mặt cầu ngoại tiếp hình chóp.
	\loigiai{
		\immini
		{
			$\heva{&BC \perp AB &\text{(giả thuyết)} \\& BC \perp SA &(SA \perp (ABC))}$\\
			$ \Rightarrow BC \perp (SAB) \Rightarrow BC \perp SB$.
			Ta có $B$ và $A$ nhìn $SC$ dưới một góc vuông.\\
			Suy ra nên $B$ và $A$ cùng nằm trên một mặt cầu có đường kính là $SC$.\\
			Gọi $I$ là trung điểm $SC\Rightarrow I$ là tâm mặt cầu ngoại tiếp chóp khối chóp $S.ABC$ và bán kính $R=SI$.
			
		}
		{
			\begin{tikzpicture}[>=stealth,line join=round,line cap=round,font=\footnotesize,scale=1]
			\path
			(0,0) coordinate (A)
			(0:3) coordinate (C)
			(-50:1.5) coordinate (B)
			(90:2) coordinate (S)
			($ (S)!.5!(C) $) coordinate (I)
			;
			\draw (S)--(A)--(B)--(C)--cycle (I)--(B)--(S);
			\draw[dashed] (C)--(A)--(I);
			\foreach \x/\y in {S/I,I/C,I/B,I/A}\path (\x)--(\y)node[midway,sloped]{$ | $};
			\foreach \x/\g in {A/180,B/-90,C/0,S/90,I/90}\fill[black] (\x) circle (1pt)+(\g:3mm) node{$ \x $};
			\end{tikzpicture}}}
%<MyLT>
\end{vd}
\begin{vd}%Ví dụ 2.%[2H2B2-2]
	Cho hình chóp $S.ABCD$ có đáy là hình chữ nhật, $SA$ vuông góc với đáy, $SA=a, AD=5a, AB=2a$. Điểm $E$ thuộc cạnh $BC$ sao cho $CE=a$. Tính theo $a$ bán kính mặt cầu ngoại tiếp tứ diện $SAED$.
	\loigiai{
		\immini
		{
			Ta có $AE^2=AB^2+BE^2=4a^2+(4a)^2=20a^2$,\\
			$ DE^2=DC^2+CE^2=4a^2+a^2=5a^2$.\\
			Do đó $AE^2+DE^2=AD^2=25a^2$.\\
			Suy ra tam giác $AED$ suy ra tam giác $AED$ vuông ở $E$.\\
			Suy ra $ED\perp(SAE)\Rightarrow ED\perp SE$.\\
			Vậy $A$ và $E$ đều nhìn $SD$ dưới một góc vuông.\\
			Do đó mặt cầu ngoại tiếp tứ diện $SAED$ có bán kính là\\
			$R=\dfrac{SD}{2}=\dfrac{1}{2}\sqrt{SA^2+AD^2}=\dfrac{a\sqrt{26}}{2}$.
		}
		{
			\begin{tikzpicture}[>=stealth,line join=round,line cap=round,font=\footnotesize,scale=.9]
			\def\ad{5}
			\def\ab{2}
			\def\gb{-120}
			\def\hh{1}
			\path
			(90:\hh) coordinate (S)
			(0,0) coordinate (A)
			(0:\ad) coordinate (D)
			(\gb:\ab) coordinate (B)
			($ (D)+(B)-(A) $) coordinate (C)
			(90:3) coordinate (S)
			($ (B)+(0:\ad-1) $) coordinate (E);
			\draw (C)--(S)--(B)--(C)--(D)--(S)--(E);
			\draw[dashed] (S)--(A)--(D)--(E)--(A)--(B);
			\foreach \x/\g in {A/180,B/-90,C/-90,D/00,S/90,E/-90}\fill[black] (\x) circle (1pt)+(\g:3mm) node{$ \x $};
			\end{tikzpicture}
		}
	}
%<MyLT>
\end{vd}
\begin{vd}%Ví dụ 3%[2H2B2-2]
	Cho hình chóp $S.ABCD$ có đáy là hình thang vuông tại $A$, $B$. Biết $SA \perp (ABCD)$, $AB=BC=a$, $AD=2a$, $SA=a\sqrt{2}$. Gọi $E$ là trung điểm của $AD$. Tính bán kính mặt cầu đi qua các điểm $S$, $A$, $B$, $C$, $E$.
	\loigiai{\immini{
			$SA\perp(ABCD)\Rightarrow SA\perp AC\Rightarrow\widehat{SAC}=90^{\circ}$.\\
			$BC\perp(SAB)\Rightarrow BC\perp SC\Rightarrow\widehat{SBC}=90^{\circ}$.\\
			$CE\parallel AB\Rightarrow CE\perp(SAD)\Rightarrow CE\perp SE\Rightarrow\widehat{SEC}=90^{\circ}$.\\
			Suy ra các điểm $A$, $B$, $E$ cùng nhìn đoạn $SC$ dưới một góc vuông nên mặt cầu đi qua các điểm $S$, $A$, $B$, $C$, $E$ là mặt cầu đường kính $SC$.\\
			Bán kính mặt cầu đi qua các điểm $S$, $A$, $B$, $C$, $E$ là $R=\dfrac{SC}{2}$.\\
			Xét tam giác $SAC$ vuông tại $A$ ta có:\\ $AC=AB\sqrt{2}=a\sqrt{2}\Rightarrow SC=AC\sqrt{2}=2a$ \\
			$ \Rightarrow R=\dfrac{SC}{2}=a $.}{
			\begin{tikzpicture}[scale=.8, font=\footnotesize, line join=round, line cap=round, >=stealth]
			\def\bc{3}
			\def\ba{2}
			\def\sa{4}
			\def\gocB{50}
			\coordinate[label=below left:$B$] (B) at (0,0);
			\coordinate[label=above left:$A$] (A) at (\gocB:\ba);
			\coordinate[label=below:$C$] (C) at (\bc,0);
			\coordinate[label=right:$D$] (D) at ($2*(C)-2*(B)+(A)$);
			\coordinate[label=above:$S$] (S) at ($(A)+(90:\sa)$);
			\coordinate[label=above:$E$] (E) at ($(A)!.5!(D)$);
			\draw (B)--(C)--(D)--(S)--cycle (S)--(C);
			\draw[dashed] (A)--(D) (S)--(A)--(B) (A)--(C)--(E)--(S);
			\foreach \diem in {A,B,C,D,S,E}	\fill (\diem)circle(1.5pt);
			\newcommand{\gocv}[4][black]{\draw[#1] ($(#3)!5pt!(#2)$)--($(#3)!2!($($(#3)!5pt!(#2)$)!.5!($(#3)!5pt!(#4)$)$)$)--($(#3)!5pt!(#4)$);}
			\gocv{S}{A}{C} \gocv{S}{B}{C} \gocv{S}{E}{C}
			\end{tikzpicture}
		}
	}
\end{vd}
\begin{vd}%Ví dụ 4.%[2H2B2-2]
	Cho hình chóp $S.ABCD$ có đáy $ABCD$ là hình chữ nhật với độ dài đường chéo bằng $\sqrt{2}a$, cạnh $SA$ có độ dài bằng $2a$ và vuông góc với mặt phẳng đáy. Tính bán kính mặt cầu ngoại tiếp hình chóp $S.ABCD$.
	\loigiai{
		\immini{
			Ta có $SA \perp (ABCD)$ nên $SA \perp AC$ hay $\triangle SAC$ vuông tại $A$ và $SA \perp BC$, $SA \perp CD$.\\
			$BC \perp AB$ nên $BC \perp SB$ hay $\triangle SBC$ vuông tại $B$.\\
			$CD \perp AD$ nên $CD \perp SD$ hay $\triangle SCD$ vuông tại $D$.\\
			Khi đó $\triangle SAC$, $\triangle SBC$, $\triangle SCD$ cùng nhìn cạnh huyền $SC$ dưới một góc vuông nên các đỉnh $S$, $A$, $B$, $C$, $D$ cùng nằm trên mặt cầu đường kính $SC$.\\
			Bán kính mặt cầu ngoại tiếp hình chóp $S.ABCD$ là\\
			$R=\dfrac{1}{2}SC=\dfrac{1}{2}\sqrt{SA^2+AC^2}=\dfrac{1}{2}\sqrt{4a^2+2a^2}=\dfrac{a\sqrt{6}}{2}$.
		}{
			\begin{tikzpicture}[scale=.8, font=\footnotesize, line join=round, line cap=round, >=stealth]
			\def\bc{4}
			\def\ba{2}
			\def\sa{4}
			\def\gocB{30}
			\coordinate[label=below left:$B$] (B) at (0,0);
			\coordinate[label=above left:$A$] (A) at (\gocB:\ba);
			\coordinate[label=below:$C$] (C) at (\bc,0);
			\coordinate[label=right:$D$] (D) at ($(C)-(B)+(A)$);
			\coordinate[label=above:$S$] (S) at ($(A)+(90:\sa)$);
			\coordinate[label=above:$I$] (I) at ($(S)!.5!(C)$);
			\draw (B)--(C)--(D)--(S)--cycle (S)--(C);
			\draw[dashed] (A)--(D) (S)--(A)--(B) (I)--(A)--(C);
			\foreach \diem in {A,B,C,D,S,I}	\fill (\diem)circle(1.5pt);
			\newcommand{\gocv}[4][black]{\draw[#1] ($(#3)!5pt!(#2)$)--($(#3)!2!($($(#3)!5pt!(#2)$)!.5!($(#3)!5pt!(#4)$)$)$)--($(#3)!5pt!(#4)$);}
			\gocv{S}{A}{C}
			\end{tikzpicture}}
	}
\end{vd}
\begin{vd}%Ví dụ 5.%[2H2B2-2]
	Cho hình chóp $S.ABCD$ có $\widehat{ABC}=\widehat{ADC}=90^\circ$, cạnh bên $SA$ vuông góc với $(ABCD)$, góc tạo bởi $SC$ và đáy $ABCD$ bằng $60^\circ$, $CD=a$ và tam giác $ACD$ có diện tích bằng $\dfrac{a^2 \sqrt{3}}{2}$. Tính diện tích mặt cầu $S_{mc}$ ngoại tiếp hình chóp $S.ABCD$.
	\loigiai{
		\immini{
			Giả thiết $SA \perp (ABCD) \Rightarrow AC$ là hình chiếu của $SC$ lên $(ABCD)$.\\
			Do đó: $(SC,(ABCD))=(SC,AC)=\widehat{SCA}=60^\circ$.\\
			Xét tam giác $ADC$ vuông tại $D$, có:\\
			$S_{ADC}=\dfrac{1}{2} \cdot AD \cdot DC =\dfrac{a^2 \sqrt{3}}{2} \Leftrightarrow AD=a\sqrt{3}$.\\
			Khi đó: $AC=\sqrt{AD^2+DC^2}=\sqrt{\left( a\sqrt{3} \right)^2+a^2}=2a$.\\
			$\triangle SAC$ vuông tại $A$, có: $\tan \widehat{SAC}=\dfrac{SA}{AC} \Rightarrow SA=AC \cdot \tan 60^\circ=2a\sqrt{3}$.\\
			Gọi $I$ là trung điểm $SC$, $H$ là trung điểm $AC$. \hfill (1)\\
			Khi đó $IH \parallel SA \Rightarrow IH \perp (ABCD)$.\\
			Tứ giác $ABCD$ có $\widehat{B}=\widehat{D}=90^\circ$, $H$ là trung điểm $AC$ nên $H$ là tâm đường tròn ngoại tiếp tứ giác $ABCD$.\\
			Suy ra $LA=IB=IC=ID$. \hfill (2)\\
			Từ (1) và (2) suy ra $I$ là tâm mặt cầu ngoại tiếp hình chóp $S.ABCD$. Bán kính mặt cầu: $R=\dfrac{1}{2} SC=\dfrac{1}{2}\sqrt{4a^2+12a^2}=2a$.\\
			Diện tích mặt càu: $S=4\pi R^2=16 \pi a^2$.
		}{
			\begin{tikzpicture}[scale=1, font=\footnotesize, line join=round, line cap=round, >=stealth]
			\coordinate[label=left:$A$] (A) at (0,0);
			\coordinate[label=above left:$ S$] (S) at ($(A)+(90:3.5)$);
			\coordinate[label=right:$D$] (D) at ($(A)+(0:4)$);
			\coordinate[label=below left:$B$] (B) at ($(A)+(-60:2.1)$);
			\coordinate[label=below right:$C$] (C) at ($(A)+(-25:3.2)$);
			\coordinate[label=above:$I$] (I) at ($(S)!.5!(C)$);
			\coordinate[label=below:$H$] (H) at ($(A)!.5!(C)$);
			\draw (A)--(B)--(C)--(D)--(S)--cycle (B)--(S)--(C) (B)--(I)--(D);
			\draw[dashed] (D)--(A)--(C) (H)--(I)--(A);
			\foreach \diem in {A,B,C,D,S,I,H}	\fill (\diem)circle(1.5pt);
			\newcommand{\gocv}[4][black]{\draw[#1] ($(#3)!5pt!(#2)$)--($(#3)!2!($($(#3)!5pt!(#2)$)!.5!($(#3)!5pt!(#4)$)$)$)--($(#3)!5pt!(#4)$);}
			\gocv{S}{A}{C} \gocv{A}{B}{C} \gocv{A}{D}{C}
			\end{tikzpicture}
		}
	}
\end{vd}
\subsubsection{Câu hỏi trắc nghiệm}
\begin{ex}%Câu 1%[2H2Y2-2]
	Trong các mệnh đề sau, mệnh đề nào đúng?
	\choice
	{Hình chóp có đáy là hình thang vuông thì luôn có mặt cầu ngoại tiếp}
	{Hình chóp có đáy là hình thoi thì luôn có mặt cầu ngoại tiếp}
	{Hình chóp có đáy là hình tứ giác thì luôn có mặt cầu ngoại tiếp}
	{\True Hình chóp có đáy là hình tam giác thì luôn có mặt cầu ngoại tiếp}
	\loigiai{
		Điều kiện để một hình chóp có mặt cầu ngoại tiếp là đa giác đáy là đa giác nội tiếp đường tròn. Do đó: Đáy là tam giác thì luôn có tâm đường tròn ngoại tiếp.}
\end{ex}
\begin{ex}%Câu 2%[2H2Y2-2]
	Trong các hình đa diện sau, hình nào \textbf{không} nội tiếp được trong một mặt cầu?
	\choice
	{Hình tứ diện}
	{Hình hộp chữ nhật}
	{Hình chóp ngũ giác đều}
	{\True Hình chóp có đáy là hình thang vuông}
	\loigiai{
		Vì hình thang vuông không nội tiếp được trong một đường tròn nên hình chóp có đáy là hình thang vuông không nội tiếp được trong một mặt cầu.}
\end{ex}
\begin{ex}%Câu 3%[2H2Y2-2]
	Chọn mệnh đề đúng trong các mệnh đề sau.
	\choice
	{Hình có đáy là hình bình hành thì có mặt cầu ngoại tiếp}
	{Hình có đáy là hình tứ giác thì có mặt cầu ngoại tiếp}
	{Hình có đáy là hình thang thì có mặt cầu ngoại tiếp}
	{\True Hình có đáy là hình thang cân thì có mặt cầu ngoại tiếp}
	\loigiai{
		Một hình chóp có mặt cầu ngoại tiếp khi và chỉ khi đáy của nó là một đa giác nội tiếp được đường tròn.\\
		Như vậy đáy là hình bình hành, hình tứ giác, hình thang bất kỳ chưa chắc đã nội tiếp được một mặt cầu nên đáp án $ A $, $ B $,$ C $ (loại).}
\end{ex}
\begin{ex}%Câu 4%[2H2B2-2]
	Cho hình chóp $S.ABCD$ có đáy $ABCD$ là hình chữ nhật, $SA$ vuông góc với đáy, $I$ là tâm mặt cầu ngoại tiếp hình chóp. Khẳng định nào sau đây là \textbf{đúng}?
	\choice
	{\True $I$ là trung điểm $SC$}
	{$I$ là tâm đường tròn ngoại tiếp tam giác $SBD$}
	{$I$ là giao điểm của $AC$ và $BD$}
	{$I$ trung điêm $SA$}
	\loigiai{
		\immini{
			Gọi $I$ là trung điểm $SC$.\\
			Ta có $SA \perp(ABCD) \Rightarrow SA \perp AC$.\\
			Suy ra tam giác $SAC$ vuông tại $A$.\\
			Do đó $\Rightarrow IA=IC=IS$.\hfill$(1)$\\
			Lại có: $AB, AD$ là hình chiếu vuông góc của $SB, SD$ lên mặt phằng $(ABCD)$.\\
			Mà $BC \perp AB$, $CD \perp AD$ nên $BC \perp SB$, $CD \perp SD$ (định lí ba đường vuông góc).\\
			$\Rightarrow$ các tam giác $SBC$ và $SAD$ vuông tại $B$ và $D$\\
			$\Rightarrow\heva{&IB=IC=IS \\& IC=ID=IS.}$ \hfill$(2)$\\
			Tù $(1)$ và $(2)$ suy ra $I$ là tâm mặt cầu ngoại tiếp hình chóp.\\
			Vậy tâm của mặt cầu ngoại tiếp hình chóp là trung điểm $I$ của $SC$.
		}{
			\begin{tikzpicture}[scale=1, font=\footnotesize, line join=round, line cap=round, >=stealth]
			\coordinate[label=left:$A$] (A) at (0,0);
			\coordinate[label=above left:$ S$] (S) at ($(A)+(90:3.5)$);
			\coordinate[label=right:$D$] (D) at ($(A)+(0:4)$);
			\coordinate[label=below left:$B$] (B) at ($(A)+(-150:2.5)$);
			\coordinate[label=below right:$C$] (C) at ($(B)+(D)-(A)$);
			\coordinate[label=above:$I$] (I) at ($(S)!.5!(C)$);
			\draw (B)--(C)--(D)--(S)--cycle (S)--(C);
			\draw[dashed] (S)--(A) (D)--(A)--(B);
			\foreach \diem in {A,B,C,D,S,I}	\fill (\diem)circle(1.5pt);
			\newcommand{\gocv}[4][black]{\draw[#1] ($(#3)!5pt!(#2)$)--($(#3)!2!($($(#3)!5pt!(#2)$)!.5!($(#3)!5pt!(#4)$)$)$)--($(#3)!5pt!(#4)$);}
			\gocv{S}{A}{C} \gocv{A}{B}{C} \gocv{A}{D}{C}
			\end{tikzpicture}}
	}
\end{ex}
\begin{ex}%Câu 5%[2H2Y2-2]
	Chọn mệnh đề đúng trong các mệnh đề sau.
	\choice
	{Hình có đáy là hình bình hành thì có mặt cầu ngoại tiếp}
	{\True Hình chóp có đáy là hình thang cân thì có mặt cầu ngoại tiếp}
	{Hình chóp có đáy là hình thang vuông thì có mặt cầu ngoại tiếp}
	{Hình chóp có đáy là tứ giác thì có mặt cầu ngoại tiếp}
	\loigiai{
		Trong các đáp án chỉ có đáp án B có đáy là hình thang cân mới có đường tròn ngoại tiếp đáy, suy ra có mặt cầu ngoại tiếp.}
\end{ex}
\begin{ex}%Câu 6%[2H2Y2-2]
	Trong các mệnh đề sau, mệnh đề nào \textbf{sai}?
	\choice
	{\True Bất kì một hình hộp nào cũng có một mặt cầu ngoại tiếp}
	{Bất kì một hình tứ diện nào cũng có một mặt cầu ngoại tiếp}
	{Bất kì một hình chóp đều nào cũng có một mặt cầu ngoại tiếp}
	{Bất kì một hình hộp chữ nhật nào cũng có một mặt cầu ngoại tiếp}
	\loigiai{
		Điều kiện cần để một hình hộp có một mặt cầu ngoại tiếp là đáy của hình hộp là đa giác nội tiếp.}
\end{ex}
%--------------------------------------------------------------------------------------------------------
%\Opensolutionfile{ans}[ans/ans2H2]
%\setcounter{ex}{6}
\begin{ex}%Câu 7.%[Paul Hieu Nguyen]%[2H2B2-2]
	Cho hình chóp $S.ABC$ có tam giác $ABC$ vuông tại $B$, $SA$ vuông góc với mặt phẳng $(ABC)$, $SA=5$, $AB=3$, $BC=4$. Tính bán kính $R$ của mặt cầu ngoại tiếp hình chóp $S.ABC$.
	\choice
	{$R=\dfrac{5 \sqrt{2}}{2}$}
	{$R=\dfrac{5 \sqrt{2}}{3}$}
	{\True $R=\dfrac{5 \sqrt{3}}{3}$}
	{$R=\dfrac{5 \sqrt{3}}{2}$}
	\loigiai{
		\immini{
			Ta có $BC \perp SA$ và $BC \perp AB$ nên $BC \perp(SAB) \Rightarrow BC \perp SB$. Vậy hai điểm $A$, $B$ cùng nhìn cạnh $S C$ dưới một góc vuông. Điều đó chứng tỏ $SC$ là đường kính của mặt cầu ngoại tiếp hình chóp $S.ABC$. Do đó bán kính
			$$R=\dfrac{SC}{2}=\dfrac{1}{2} \sqrt{SA^2+AC^2}=\dfrac{1}{2}\sqrt{SA^2+AB^2+BC^2}=\dfrac{1}{2} \sqrt{5^2+3^2+4^2}=\dfrac{5\sqrt{2}}{2}.
			$$}
		{
			\begin{tikzpicture}[>=stealth,line join=round,line cap=round,font=\footnotesize,scale=1]
			\coordinate (A) at (0,0);
			\coordinate (B) at (2,-1);
			\coordinate (C) at (3,0);
			\coordinate (S) at ($(A)+(0,2.5)$);
			\coordinate (I) at ($(S)!.5!(C)$);
			\draw (S)--(A)--(B)--(C)--(S)--(B)--(I);
			\draw[dashed] (I)--(A)--(C);
			\foreach \d/\g in {A/180,B/-90,C/0,S/90,I/45}
			\fill[black](\d) circle (1pt)+(\g:.3)node{$\d$};
			\foreach \a/\b/\c/\t in {A/B/C/6,S/A/B/5,S/A/C/6}{\def\k{\t pt}\draw ($(\b)!\k!(\a)$)--($(\b)!2!($($(\b)!\k!(\a)$)!.5!($(\b)!\k!(\c)$)$)$)--($(\b)!\k!(\c)$);}
			\end{tikzpicture}
		}
	}
\end{ex}
\begin{ex}%Câu 8.%[Paul Hieu Nguyen]%[2H2K2-2]
	Cho hình chóp $S.ABCD$ có đáy là hình vuông cạnh bằng $a$. Cạnh bên $SA$ vuông góc với mặt đáy và $SA=a\sqrt{2}$. Tính thể tích khối cầu ngoại tiếp hình chóp $S.ABCD$ theo $a$. 
	\choice
	{$\dfrac{8\pi a^3\sqrt{2}}{3}$}
	{$4\pi a^3$}
	{\True $\dfrac{4}{3}\pi a^3$}
	{$8\pi a^3$}
	\loigiai{
		\immini{
			Ta chứng minh được các tam giác $SBC$, $SAC$ và $SCD$ là các tam giác vuông lần lượt tại $B$, $A$, $D$.\\
			Suy ra các điểm $B$, $A$, $D$ nhìn cạnh $SC$ dưới một góc vuông.\\
			Gọi $I$ là trung điểm $SC$ $\Rightarrow I$ là tâm mặt cầu ngoại tiếp hình chóp $S.ABCD$.\\
			Khi đó bán kính mặt cầu ngoại tiếp hình chóp $S.ABCD$ là\\
			$R=AI=\dfrac{1}{2} \sqrt{SA^2+AC^2}=\dfrac{1}{2}\sqrt{(a \sqrt{2})^2+(a \sqrt{2})^2}=a$.\\
			Vậy thể tích khối cầu ngoại tiếp hình chóp $S.ABCD$ là: $$V =\dfrac{4}{3}\pi R^3=\dfrac{4}{3}\pi \cdot a^3=\dfrac{4\pi a^3}{3}.$$}
		{
			\begin{tikzpicture}[>=stealth,line join=round,line cap=round,font=\footnotesize,scale=.8]
			\coordinate (A) at (0,0);
			\coordinate (B) at (3,0);
			\coordinate (D) at (-1.5,-1.5);
			\coordinate (C) at ($(B)+(D)-(A)$);
			\coordinate (S) at ($(A)+(0,2.5)$);
			\coordinate (I) at ($(S)!.5!(C)$);
			\draw (S)--(D)--(C)--(B)--(S)--(C);
			\draw[dashed] (A)--(D)--(B)--(A)--(C) (S)--(A)--(I);
			\foreach \d/\g in {A/180,B/0,C/-80,D/200,S/90,I/45}
			\fill[black](\d) circle (1pt)+(\g:.3)node{$\d$};
			\foreach \a/\b/\c/\t in {S/A/C/6,S/B/C/5,S/D/C/6}{\def\k{\t pt}\draw ($(\b)!\k!(\a)$)--($(\b)!2!($($(\b)!\k!(\a)$)!.5!($(\b)!\k!(\c)$)$)$)--($(\b)!\k!(\c)$);}
			\end{tikzpicture}
		}
	}
\end{ex}
\begin{ex}%Câu 9.%[Paul Hieu Nguyen]%[2H2K2-2]
	Cho hình chóp tam giác đều $S.ABC$ có cạnh đáy bằng $a$ và mỗi cạnh bên bằng $a\sqrt{2}$. Khi đó bán kính mặt cầu ngoại tiếp hình chóp $S.ABC$ là 
	\choice
	{\True $\dfrac{a \sqrt{15}}{5}$}
	{$\dfrac{3a}{5}$}
	{$\dfrac{a \sqrt{3}}{5}$}
	{$\dfrac{a \sqrt{6}}{4}$}	
	\loigiai{
		\immini{
			Gọi $H$ là trọng tâm tam giác đều $ABC$, khi đó $SH\perp(ABC)$ và là trục đường tròn ngoại tiếp
			mặt đáy.\\
			Gọi $N$ là trung điểm $SA$, mặt phẳng trung trực của cạnh $SA$ cắt $SH$ tại $I$. Khi đó $IS=IA=IB=IC$ nên $I$ là tâm mặt cầu ngoại tiếp hình chóp $S.ABC$.\\
			Bán kính mặt cầu là $$R=SI=\dfrac{SN\cdot SA}{SH}=\dfrac{\dfrac{1}{2}SA^2}{\sqrt{SA^2-AH^2}}=\dfrac{1}{2}\dfrac{(a\sqrt{2})^2}{\sqrt{(a\sqrt{2})^2-\left(\dfrac{2}{3}\dfrac{a\sqrt{3}}{2}\right)^2}}=\dfrac{a\sqrt{15}}{5}.$$}
		{
			\begin{tikzpicture}[>=stealth,line join=round,line cap=round,font=\footnotesize,scale=1]
			\coordinate (A) at (0,0);
			\coordinate (C) at (3,0);
			\coordinate (B) at (1,-1);
			\coordinate (M) at ($(B)!.5!(C)$);
			\coordinate (H) at ($(A)!2/3!(M)$);
			\coordinate (S) at ($(H)+(0,2.5)$);
			\coordinate (N) at ($(S)!.5!(A)$);
			\coordinate (I) at ($(S)!.7!(H)$);
			\draw (S)--(A)--(B)--(C)--(S)--(B);
			\draw[dashed] (C)--(A)--(M) (S)--(H) (N)--(I);
			\foreach \d/\g in {A/180,B/-90,C/0,H/-90,S/90,M/-45,N/150,I/0}
			\fill[black](\d) circle (1pt)+(\g:.3)node{$\d$};
			\end{tikzpicture}
		}
	}
\end{ex}
\begin{ex}%Câu 10.%[Paul Hieu Nguyen]%[2H2K2-2]
	Hình chóp $S.ABCD$ có đáy là hình vuông cạnh $a$, $SA$ vuông góc với mặt phẳng $(ABCD)$ và $SA=2a$. Diện tích mặt cầu ngoại tiếp hình chóp $S.ABCD$ bằng
	\choice
	{$2 \pi a^2$}
	{$\pi a^2$}
	{$3\pi a^2$}
	{\True $6 \pi a^2$}	
	\loigiai{
		\immini{
			Ta chứng minh được:\\
			$\bullet$ $BC \perp(SAB) \Rightarrow BC \perp SB \Rightarrow \Delta SBC$ vuông tại $B$.\\
			$\bullet$ $CD \perp(SAD) \Rightarrow CD \perp SD \Rightarrow \Delta SCD$ vuông tại $D$.\\
			$\bullet$ $SA \perp(ABCD) \Rightarrow SA\perp AC \Rightarrow \Delta SAC$ vuông tại $A$.\\
			Gọi $O$ là trung điểm cạnh $SC$.\\
			Khi đó $OA=OC=OD=OB=OS=\dfrac{1}{2} SC$. Do đó $O$ là tâm mặt cầu ngoại tiếp khối chóp $S.ABCD$.\\
			Bán kính mặt cầu là: $R=\dfrac{1}{2} SC=\dfrac{1}{2} \sqrt{SA^2+AC^2}=\dfrac{1}{2} \sqrt{4a^2+2a^2}=\dfrac{a \sqrt{6}}{2}$.\\
			Diện tích mặt cầu: $S=4 \pi R^2=4\pi \cdot \dfrac{3a^2}{2}=6\pi a^2$.}
		{
			\begin{tikzpicture}[>=stealth,line join=round,line cap=round,font=\footnotesize,scale=.8]
			\coordinate (A) at (0,0);
			\coordinate (B) at (3,0);
			\coordinate (D) at (-1.5,-1.5);
			\coordinate (C) at ($(B)+(D)-(A)$);
			\coordinate (S) at ($(A)+(0,2.5)$);
			\coordinate (O) at ($(S)!.5!(C)$);
			\draw (S)--(D)--(C)--(B)--(S)--(C);
			\draw[dashed] (B)--(A)--(D) (S)--(A)--(C);
			\foreach \d/\g in {A/180,B/0,C/-80,D/200,S/90,O/45}
			\fill[black](\d) circle (1pt)+(\g:.3)node{$\d$};
			\end{tikzpicture}
		}
	}
\end{ex}
\begin{ex}%Câu 11.%[Paul Hieu Nguyen]%[2H2K2-2]
	Cho hình chóp $S.ABCD$ có đáy là hình chữ nhật. Biết $SA=AB=a$, $AD=2a$, $SA\perp(ABCD)$. Tính bán kính mặt cầu ngoại tiếp hình chóp $S.ABCD$.
	\choice
	{$\dfrac{2 a \sqrt{39}}{13}$}
	{$\dfrac{a \sqrt{3}}{2}$}
	{$\dfrac{3 a \sqrt{3}}{4}$}
	{\True $\dfrac{a \sqrt{6}}{2}$}
	\loigiai{
		\immini{
			Dễ thấy $\widehat{SAC}=\widehat{SBC}=\widehat{SDC}=90^{\circ}$.\\
			Suy ra mặt cầu ngoại tiếp hình chóp $S.ABCD$ có đường kính là cạnh $SC=\sqrt{SA^2+AC^2}=a\sqrt{6}$.\\
			Vậy bán kính $R=\dfrac{a\sqrt{6}}{2}$.}
		{
			\begin{tikzpicture}[>=stealth,line join=round,line cap=round,font=\footnotesize,scale=.8]
			\coordinate (A) at (0,0);
			\coordinate (B) at (3,0);
			\coordinate (D) at (-1.5,-1.5);
			\coordinate (C) at ($(B)+(D)-(A)$);
			\coordinate (S) at ($(A)+(0,2.5)$);
			\coordinate (I) at ($(S)!.5!(C)$);
			\draw (S)--(D)--(C)--(B)--(S)--(C);
			\draw[dashed] (B)--(A)--(D) (S)--(A)--(C);
			\foreach \d/\g in {A/180,B/0,C/-80,D/200,S/90,I/45}
			\fill[black](\d) circle (1pt)+(\g:.3)node{$\d$};
			\foreach \a/\b/\c/\t in {A/D/C/5,S/A/B/5,S/A/D/5}{\def\k{\t pt}\draw ($(\b)!\k!(\a)$)--($(\b)!2!($($(\b)!\k!(\a)$)!.5!($(\b)!\k!(\c)$)$)$)--($(\b)!\k!(\c)$);}
			\end{tikzpicture}
		}
	}
\end{ex}
\begin{ex}%Câu 12.%[Paul Hieu Nguyen]%[2H2K2-2]
	Cho hình chóp tứ giác $S.ABCD$ có đáy $ABCD$ là hình thang vuông tại $A$ và $B$, $AB = BC =a$, $A D=2 a$, $SA \perp(ABCD)$ và $SA=a\sqrt{2}$. Gọi $E$ là trung điểm của $AD$. Kẻ $EK\perp SD$ tại $K$. Bán kính mặt cầu đi qua sáu điểm $S$, $A$, $B$, $C$, $E$, $K$ là
	\choice
	{$R=\dfrac{1}{2} a$}
	{$R=\dfrac{\sqrt{3}}{2} a$}
	{$\True R=a$}
	{$R=\dfrac{\sqrt{6}}{2} a$}	
	\loigiai{
		\immini{
			Vì $E$ là trung điểm của $AD$, $ABCD$ là hình thang vuông tại $A$ và $B$ và $AB=B C=a$, $AD=2a$ nên $AB=BC=CE=AE=ED=a$ và $CE \parallel AB$.\\
			Khi đó $CE \perp AD$, $CE\perp SA$ (do $SA \perp(ABCD)$) nên $CE \perp SE$ hay $\widehat{SEC}=90^{\circ}$ và $CE \perp SD$.\\
			Mặt khác $EK \perp SD$, do đó $SD \perp(CEK)$ suy ra $CK\perp SD$ hay $\widehat{SCK}=90^{\circ}$.\\
			Ta có $CB \perp AB$, $CB\perp SA$ nên $CB\perp SB$ hay $\widehat{SBC}=90^{\circ}$.\\
			Ta cũng có $CA\perp SA$ nên $\widehat{SAC}=90^{\circ}$.\\
			Vậy các góc $\widehat{SEC}$, $\widehat{SCK}$, $\widehat{SBC}$, $\widehat{SAC}$ cùng nhìn cạnh $S C$ dưới một góc không đổi $90^{\circ}$ nên các điểm $S$, $A$, $B$, $C$, $E$, $K$ nằm trên mặt cầu đường kính $SC$, bán kính $R=\dfrac{S C}{2}$.\\
			Ta có $AC=\sqrt{AB^2+BC^2}=a\sqrt{2}$; $SC=\sqrt{AC^2+SA^2}=2a$, suy ra $R=a$.}
		{
			\begin{tikzpicture}[>=stealth,line join=round,line cap=round,font=\footnotesize,scale=.8]
			\coordinate (A) at (0,0);
			\coordinate (B) at (-1.5,-1.5);
			\coordinate (D) at (5,0);
			\coordinate (E) at ($(A)!.5!(D)$);
			\coordinate (C) at ($(B)+(E)-(A)$);
			\coordinate (S) at ($(A)+(0,3)$);
			\coordinate (K) at ($(S)!.6!(D)$);
			\draw (S)--(B)--(C)--(D)--(S)--(C)--(K);
			\draw[dashed] (B)--(A)--(D) (S)--(A)--(K)--(E)--(C);
			\foreach \d/\g in {A/180,B/200,C/-80,D/0,S/90,K/45,E/-90}
			\fill[black](\d) circle (1pt)+(\g:.3)node{$\d$};
			\foreach \a/\b/\c/\t in {E/K/D/5}{\def\k{\t pt}\draw ($(\b)!\k!(\a)$)--($(\b)!2!($($(\b)!\k!(\a)$)!.5!($(\b)!\k!(\c)$)$)$)--($(\b)!\k!(\c)$);}
			\end{tikzpicture}
		}
	}
\end{ex}
\begin{ex}%Câu 13.%[Paul Hieu Nguyen]%[2H2K2-2]
	Cho khối tứ diện $OABC$ với $OA$, $OB$, $OC$ từng đôi một vuông góc và $OA=OB=OC =6$. Tính bán kính $R$ của mặt cầu ngoại tiếp tứ diện $OABC$.
	\choice
	{$R=4 \sqrt{2}$}
	{$R=2$}
	{$R=3$}
	{\True $R=3 \sqrt{3}$}	
	\loigiai{
		\immini{
			Gọi $M$ là trung điểm của $BC$, do tam giác $OBC$ vuông tại $O$ nên $M$ là tâm đường tròn ngoại tiếp tam giác $OBC$.\\
			Qua $M$ dựng đường thẳng $d$ song song với $OA$. Khi đó $d$ là trục đường tròn ngoại tiếp tam giác $OBC$.\\
			Gọi $\Delta$ là đường trung trực của cạnh $OA$ và $I$ là giao điểm của $\Delta$ và $d$.\\
			Khi đó $I$ là tâm mặt cầu ngoại tiếp tứ diện $OABC$.\\
			Ta có $OM=\dfrac{1}{2} BC=\dfrac{1}{2}\sqrt{OB^2+OC^2}=3\sqrt{2}$; $ON=IM=\dfrac{1}{2} OA=3$.\\
			Tam giác $OMI$ vuông tại $M$ nên 
			$$IM=\sqrt{OM^2+IM^2}=\sqrt{(3\sqrt{2})^2+3^2}=3\sqrt{3}.$$
			Vậy bán kính mặt cầu ngoại tiếp tứ diện $OABC$ là $R=3\sqrt{3}$.}
		{
			\begin{tikzpicture}[>=stealth,line join=round,line cap=round,font=\footnotesize,scale=1]
			\coordinate (O) at (0,0);
			\coordinate (B) at (1,-1);
			\coordinate (C) at (3,0);
			\coordinate (A) at ($(O)+(0,2.5)$);
			\coordinate (M) at ($(B)!.5!(C)$);
			\coordinate (N) at ($(A)!.5!(O)$);
			\coordinate (M1) at ($(M)-(0,.5)$);
			\coordinate (M2) at ($(M)+(0,2.5)$);
			\coordinate (I) at ($(M)+(N)-(O)$);
			\coordinate (K) at (intersection of N--I and A--M);
			\coordinate (G) at (intersection of O--I and A--M);
			\draw (M)--(A)--(O)--(B)--(C)--(A)--(B) (K)--(I)--(G) (M1)--(M2);
			\draw[dashed] (C)--(O)--(M) (O)--(G) (N)--(K);
			\foreach \d/\g in {O/180,B/-90,C/0,A/90,I/0,M/-45,N/180,I/0}
			\fill[black](\d) circle (1pt)+(\g:.3)node{$\d$};
			\end{tikzpicture}
		}
	}
\end{ex}
\begin{ex}%Câu 14.%[Paul Hieu Nguyen]%[2H2K2-2]
	Cho bốn điểm $A$, $B$, $C$, $D$ cùng thuộc một mặt cầu và $DA$, $DB$, $DC$ đôi một vuông góc, $G$ là trọng tâm tam giác $ABC$, $D'$ là điểm thỏa mãn $\overrightarrow{DD'}=3\overrightarrow{DG}$. Một đường kính của mặt cầu đó là
	\choice
	{$AB$}
	{$AC$}
	{\True $DD’$}
	{$BC$}	
	\loigiai{
		\immini{
			Gọi $M$ là trung điểm của $BC$. Dựng $d$ qua $M$ và vuông góc với mặt phẳng $(BCD)$.\\
			Khi đó $d\parallel AD$\\
			Gọi $J$ là trung điểm $AD$. Dựng mặt phẳng trung trực của đoạn thẳng $AD$ cắt $d$ tại $I$, khi đó $I$ là tâm mặt cầu ngoại tiếp tứ diện $ABCD$. Ta có:\\
			$\overrightarrow{ID}=\overrightarrow{IJ}+\overrightarrow{IM}.\quad (1)$
			\begin{eqnarray*}
				\overrightarrow{IG}&=&\overrightarrow{IM}+\overrightarrow{MG}=\overrightarrow{IM}+\dfrac{1}{3} \overrightarrow{MA}= \overrightarrow{IM}+\dfrac{1}{3}(\overrightarrow{MD}+\overrightarrow{DA})\\
				&=&\overrightarrow{IM}+\dfrac{1}{3}\overrightarrow{MD}-\dfrac{1}{3}\cdot 2\overrightarrow{IM}=\dfrac{1}{3} \overrightarrow{IM}+\dfrac{1}{3} \overrightarrow{MD}=\dfrac{1}{3}\overrightarrow{IM}+\dfrac{1}{3}\overrightarrow {IJ}\\
				&=&\dfrac{1}{3}(\overrightarrow{IM}+\overrightarrow{IJ})
			\end{eqnarray*}
			$\Rightarrow \overrightarrow{IM}+\overrightarrow{IJ}=3 \overrightarrow{IG}.\quad(2)$\\
			Từ $(1)$, $(2)$ suy ra: $\overrightarrow{ID}=3 \overrightarrow{IG}$ hay ba điểm $D$, $I$, $G$ thẳng hàng.\\
			Mặt khác: $IM \parallel AD$ (cùng vuông góc với mặt phẳng đáy)\\
			$\Rightarrow \dfrac{DG}{GI}=\dfrac{AG}{GM}=2 \Rightarrow \overrightarrow{DG}=2 \overrightarrow{GI} \Leftrightarrow \overrightarrow{DG}=2(\overrightarrow{DI}-\overrightarrow{DG})$\\
			$\Leftrightarrow 3 \overrightarrow{DG}=2 \overrightarrow{DI} \Leftrightarrow \overrightarrow{DD’}=2 \overrightarrow{DI}$
			$\Rightarrow I$ là trung điểm của $DD’$.}
		{
			\begin{tikzpicture}[>=stealth,line join=round,line cap=round,font=\footnotesize,scale=1]
			\coordinate (D) at (0,0);
			\coordinate (B) at (2,-1);
			\coordinate (C) at (3,0);
			\coordinate (A) at ($(D)+(0,2.5)$);
			\coordinate (M) at ($(B)!.5!(C)$);
			\coordinate (J) at ($(A)!.5!(D)$);
			\coordinate (M1) at ($(M)-(0,.5)$);
			\coordinate (M2) at ($(M)+(0,2.5)$);
			\coordinate (I) at ($(M)+(J)-(D)$);
			\coordinate (K) at (intersection of J--I and A--M);
			\coordinate (G) at ($(A)!2/3!(M)$);
			\coordinate (D') at ($(D)!3!(G)$);
			\draw (M)--(A)--(D)--(B)--(C)--(A)--(B) (K)--(I) (M1)--(M2) (G)--(D');
			\draw[dashed] (C)--(D)--(M) (J)--(K) (D)--(G);
			\foreach \d/\g in {D/180,B/-90,C/0,A/90,I/45,M/-45,J/180,G/-90,D'/30}
			\fill[black](\d) circle (1pt)+(\g:.3)node{$\d$};
			\end{tikzpicture}
		}
	}
\end{ex}
\begin{ex}%Câu 15.%[Paul Hieu Nguyen]%[2H2K2-2]
	Tính theo $a$ bán kính của mặt cầu ngoại tiếp hình chóp tam giác đều $S.ABC$, biết các cạnh đáy có độ dài bằng $a$, cạnh bên $SA=a \sqrt{3}$.
	\choice
	{\True $\dfrac{3a\sqrt{6}}{8}$}
	{$\dfrac{3a\sqrt{3}}{2\sqrt{2}}$}
	{$\dfrac{2a\sqrt{3}}{\sqrt{2}}$}
	{$\dfrac{a\sqrt{3}}{8}$}	
	\loigiai{
		\immini{
			Gọi $H$ là trung điểm của $SA$. Trong mặt phẳng ($SAO$) kẻ đường thẳng qua $H$ và vuông góc với $SA$ cắt $SO$ tại $I$. Khi đó $IS=IA=IB=IC$.\\
			Ta có: $AM=\dfrac{a\sqrt{3}}{2}$; $AO=\dfrac{a\sqrt{3}}{3}$; $SO=\sqrt{SA^2-OA^2}=\dfrac{2\sqrt{6}a}{3}$.\\
			Do $\triangle SHI\backsim\triangle SOA$ ta có: $\dfrac{SI}{SA}=\dfrac{SH}{SO}\Rightarrow SI=\dfrac{SH\cdot SA}{SO}=\dfrac{3\sqrt{6}a}{8}$.}
		{
			\begin{tikzpicture}[>=stealth,line join=round,line cap=round,font=\footnotesize,scale=1]
			\coordinate (A) at (0,0);
			\coordinate (C) at (3,0);
			\coordinate (B) at (1,-1);
			\coordinate (M) at ($(B)!.5!(C)$);
			\coordinate (O) at ($(A)!2/3!(M)$);
			\coordinate (S) at ($(O)+(0,2.5)$);
			\coordinate (H) at ($(S)!.5!(A)$);
			\coordinate (I) at ($(S)!.7!(O)$);
			\draw (S)--(A)--(B)--(C)--(S)--(B);
			\draw[dashed] (C)--(A)--(M) (S)--(O) (H)--(I);
			\foreach \d/\g in {A/180,B/-90,C/0,O/-90,S/90,M/-45,H/150,I/0}
			\fill[black](\d) circle (1pt)+(\g:.3)node{$\d$};
			\end{tikzpicture}
		}
	}
\end{ex}
\begin{ex}%Câu 16.%[Paul Hieu Nguyen]%[2H2K2-2]
	Cho hình chóp $S.ABC$ có đáy $ABC$ là tam giác vuông tại $B$ và $BA=BC=a$. Cạnh bên $SA=2a$ và vuông góc với mặt phẳng $(ABC)$. Bán kính mặt cầu ngoại tiếp khối chóp $S.ABC$ là
	\choice
	{$3a$}
	{$\dfrac{a\sqrt{2}}{2}$}
	{$a\sqrt{6}$}
	{\True $\dfrac{a\sqrt{6}}{2}$}	
	\loigiai{
		\immini{
			Gọi $I$ là trung điểm cạnh $SC$.\\
			$SA\perp(ABC)\Rightarrow SA\perp AC\Rightarrow\triangle SAC$ vuông tại $A$. Suy ra: $LA=IC=IS$\\
			$SA\perp(ABC)\Rightarrow SA\perp BC$ và $BC\perp AB$ (do $\triangle ABC$ vuông tại $B$).\\
			Suy ra: $BC\perp(SAB)$ nên $BC\perp SB\Rightarrow\triangle SBC$ vuông tại $B$.\\
			Do đó $IB=IC=IS$.\\
			Vậy $I$ là tâm mặt cầu ngoại tiếp hình chóp $S.ABC$. Khi đó 
			\begin{eqnarray*}
				R=IS&=&\dfrac{1}{2}SC=\dfrac{1}{2}\sqrt{SA^2+AC^2}=\dfrac{1}{2}\sqrt{SA^2+AB^2+BC^2}\\
				&=&\dfrac{1}{2}\sqrt{4a^2+a^2+a^2}=\dfrac{a\sqrt{6}}{2}.
			\end{eqnarray*}
		}
		{
			\begin{tikzpicture}[>=stealth,line join=round,line cap=round,font=\footnotesize,scale=1]
			\coordinate (A) at (0,0);
			\coordinate (B) at (2,-1);
			\coordinate (C) at (3,0);
			\coordinate (S) at ($(A)+(0,2.5)$);
			\coordinate (I) at ($(S)!.5!(C)$);
			\draw (S)--(A)--(B)--(C)--(S)--(B);
			\draw[dashed] (A)--(C);
			\foreach \d/\g in {A/180,B/-90,C/0,S/90,I/45}
			\fill[black](\d) circle (1pt)+(\g:.3)node{$\d$};
			\end{tikzpicture}
		}
	}
\end{ex}
\begin{ex}%Câu 17.%[Paul Hieu Nguyen]%[2H2K2-2]
	Cho hình chóp $S.ABC$ có cạnh bên $SA$ vuông góc với đáy, $AB=a\sqrt{2}$, $BC=a$, $SC=2a$ và $\widehat{SCA}=30^{\circ}$. Tính bán kính $R$ của mặt cầu ngoại tiếp tứ diện $S.ABC$.
	\choice
	{$R=a\sqrt{3}$}
	{$R=\dfrac{a\sqrt{3}}{2}$}
	{\True $R=a$}
	{$R=\dfrac{a}{2}$}	
	\loigiai{
		\immini{
			Ta có:\\
			$AC=SC\cdot\cos30^{\circ}=a\sqrt{3}$\\
			$AB^2+BC^2=2a^2+a^2=3a^2=AC^2\Rightarrow\triangle ABC$ là tam giác vuông ở $B$.\\
			Gọi $H,I$ lần lượt là trung điểm của $AC,SC$. Khi đó ta có:
			$H$ là tâm đường tròn ngoại tiếp $\triangle ABC$.\\
			$IH\perp (ABC)$\\
			Do đó $I$ là tâm mặt cầu ngoại tiếp hình chóp $S.ABC$. Suy ra $R=\dfrac{1}{2}SC=a$.\\
			Vậy $R=a$.}
		{
			\begin{tikzpicture}[>=stealth,line join=round,line cap=round,font=\footnotesize,scale=1]
			\coordinate (A) at (0,0);
			\coordinate (B) at (1,-1);
			\coordinate (C) at (3,0);
			\coordinate (S) at ($(A)+(0,2.5)$);
			\coordinate (I) at ($(S)!.5!(C)$);
			\coordinate (H) at ($(A)!.5!(C)$);
			\draw (S)--(A)--(B)--(C)--(S)--(B)--(I);
			\draw[dashed] (A)--(C) (I)--(H)--(B);
			\foreach \d/\g in {A/180,B/-90,C/0,S/90,I/45,H/-60}
			\fill[black](\d) circle (1pt)+(\g:.3)node{$\d$};
			\draw[blue](C)+(140:.45)arc(150:188:.45);
			\draw (2.3,-.07) node[above]{$30^\circ$};
			\end{tikzpicture}
		}
	}
\end{ex}
\begin{ex}%Câu 18.%[Paul Hieu Nguyen]%[2H2K2-2]
	Cho hình chóp $S.ABCD$ đều có đáy $ABCD$ là hình vuông cạnh $a$, cạnh bên hợp với đáy một góc bằng $60^{\circ}$. Gọi $(S)$ là mặt cầu ngoại tiếp hình chóp $S.ABCD$. Tính thể tích $V$ của khối cầu $(S)$.
	\choice
	{\True $V=\dfrac{8 \sqrt{6} \pi a^{3}}{27}$}
	{$V=\dfrac{4 \sqrt{6} \pi a^{3}}{9}$}
	{$V=\dfrac{4 \sqrt{3} \pi a^{3}}{27}$}
	{$V=\dfrac{8 \sqrt{6} \pi a^{3}}{9}$}	
	\loigiai{
		\immini{
			Gọi $O$ là tâm của hình vuông $ABCD$. Do $S.ABCD$ là hình chóp đều nên $SO\perp (ABCD)$ hay $SO$ là trục của đường tròn ngoại tiếp đáy.\\
			Trong mặt phẳng $(SBO)$ kẻ đường trung trực $\Delta$ của cạnh $SB$ và gọi $I =\triangle\cap SO$ khi đó ta có\\
			$I$ là tâm mặt cầu ngoại tiếp hình chóp $S.ABCD$.\\
			Theo giả thiết ta có $S.ABCD$ là hình chóp đều và góc giữa cạnh bên với mặt phẳng đáy bằng $60^{\circ}$ nên $\widehat{SBO}=60^{\circ}$\\
			Ta có $\triangle SMI\backsim\triangle SOB$ nên $\dfrac{SM}{SO}=\dfrac{SI}{SB}\Leftrightarrow SI=\dfrac{SM\cdot SB}{SO}$.\\
			Với $SO=OB\tan60^{\circ}\Leftrightarrow SO=\dfrac{a\sqrt{6}}{3}$;\\
			$SB=OB\cos60^{\circ}\Leftrightarrow SB=a\sqrt{2};SM=\dfrac{a\sqrt{2}}{2}$.\\
			Vậy $SI=\dfrac{SM\cdot SB}{SO}=\dfrac{a\sqrt{6}}{2}$.\\
			Thể tích khối cầu ngoại tiếp hình chóp $S.ABCD$ là $$V=\dfrac{4}{3}\pi R^3=\dfrac{4}{3}\pi\left(\dfrac{a\sqrt{6}}{2}\right)^3=\dfrac{8\sqrt{6}\pi a^3}{27}.$$}
		{
			\begin{tikzpicture}[>=stealth,line join=round,line cap=round,font=\footnotesize,scale=.8]
			\coordinate (A) at (0,0);
			\coordinate (D) at (3,0);
			\coordinate (B) at (-1.5,-1.5);
			\coordinate (C) at ($(B)+(D)-(A)$);
			\coordinate (O) at ($(A)!.5!(C)$);
			\coordinate (S) at ($(O)+(0,3.5)$);
			\coordinate (M) at ($(S)!.5!(B)$);
			\coordinate (I) at ($(S)!.7!(O)$);
			\draw (S)--(B)--(C)--(D)--(S)--(C);
			\draw[dashed] (B)--(A)--(D)--(B) (O)--(S)--(A)--(C) (M)--(I);
			\foreach \d/\g in {A/160,B/200,C/-60,D/0,S/90,O/-105,M/150,I/0}
			\fill[black](\d) circle (1pt)+(\g:.25)node{$\d$};
			\end{tikzpicture}
		}
	}
\end{ex}
\begin{ex}%Câu 19.%[Paul Hieu Nguyen]%[2H2K2-2]
	Cho hình chóp đều $S.ABCD$ có cạnh đáy $2a$ và cạnh bên $a\sqrt{6}$. Tính diện tích của mặt cầu ngoại tiếp hình chóp $S.ABCD$.
	\choice
	{$8\pi a^2$}
	{$18a^2$}
	{$9a^2$}
	{\True $9\pi a^2$}	
	\loigiai{
		\immini{
			Gọi $O$ là tâm hình vuông $ABCD$, $M$ là trung điểm của $SC$.\\
			Trong mặt phẳng $(SOC)$ dựng đường thẳng qua $M$ và vuông góc với $SC$ cắt $SO$ tại $I$.\\
			Khi đó $I$ là tâm mặt cầu ngoại tiếp hình chóp $S.ABCD$ và bán kính $r=SI$.\\
			Xét tam giác vuông $ABC$ ta có: $AC=2\sqrt{2}a$.\\
			Xét tam giác vuông $SOC$ ta có: $SO = \sqrt{SC^2-OC^2}=2a$.\\
			Xét $\Delta SMI \backsim \Delta SOC$ ta có: $\dfrac{SM}{SO}=\dfrac{SI}{SC} \Rightarrow SI=\dfrac{SM \cdot SC}{SO}=\dfrac{3a}{2}$.\\
			Vậy diện tích mặt cầu cần tìm là $S=4\pi\left(\dfrac{3a}{2}\right)^2=9\pi a^2$.}
		{
			\begin{tikzpicture}[>=stealth,line join=round,line cap=round,font=\footnotesize,scale=.8]
			\coordinate (A) at (0,0);
			\coordinate (D) at (3,0);
			\coordinate (B) at (-1.3,-1.5);
			\coordinate (C) at ($(B)+(D)-(A)$);
			\coordinate (O) at ($(A)!.5!(C)$);
			\coordinate (S) at ($(O)+(0,3.5)$);
			\coordinate (M) at ($(S)!.5!(C)$);
			\coordinate (I) at ($(S)!.65!(O)$);
			\draw (S)--(B)--(C)--(D)--(S)--(C);
			\draw[dashed] (B)--(A)--(D)--(B) (O)--(S)--(A)--(C) (M)--(I);
			\foreach \d/\g in {A/160,B/200,C/-60,D/0,S/90,O/-105,M/35,I/180}
			\fill[black](\d) circle (1pt)+(\g:.25)node{$\d$};
			\end{tikzpicture}
		}
	}
\end{ex}
\begin{ex}%Câu 20.%[Paul Hieu Nguyen]%[2H2K2-2]
	Cho hình chóp $S.ABC$ có đáy $ABC$ là tam giác vuông tại $B$ với $AB = a$, $BC = a\sqrt{3}$. Cạnh $SA$ vuông góc với mặt phẳng đáy và $SA=a\sqrt{3}$. Tính bán kính $R$ của mặt cầu ngoại tiếp hình chóp $S.ABC$.
	\choice
	{$R=a$}
	{$R=3a$}
	{$R=4a$}
	{\True $R=2a$}	
	\loigiai{
		\immini{
			Ta có $SA\perp(ABC)$ nên tam giác $SAC$ vuông tại $A\Rightarrow$ điểm $A$ thuộc mặt cầu tâm $I$ đường kính $SC.\quad(1)$\\
			Mặt khác ta lại có:
			$\heva{&BC\perp AB\\&BC\perp SA}\Rightarrow BC\perp(SAB)\Rightarrow BC\perp SB$ hay tam giác $SBC$ vuông tại $B\Rightarrow$ điểm $B$ thuộc mặt cầu tâm $I$ đường kính $SC.\quad(2)$\\
			Từ $(1)$ và $(2)$ ta có bốn điểm $A$, $B$, $S$, $C$ cùng thuộc mặt cầu tâm $I$ đường kính $SC$.\\
			Xét tam giác vuông $ABC$ ta có $AC^2=AB^2+BC^2=2a$.\\
			Xét tam giác vuông $SAC$ có $SC^2=SA^2+AC^2=16a^2\Rightarrow SC=4a$.\\
			Vậy bán kính mặt cầu ngoại tiếp hình chóp $S.ABC$ là $R=\dfrac{SC}{2}=2a$.}
		{
			\begin{tikzpicture}[>=stealth,line join=round,line cap=round,font=\footnotesize,scale=1]
			\coordinate (A) at (0,0);
			\coordinate (B) at (1,-1);
			\coordinate (C) at (3,0);
			\coordinate (S) at ($(A)+(0,2.5)$);
			\coordinate (I) at ($(S)!.5!(C)$);
			\draw (S)--(A)--(B)--(C)--(S)--(B)--(I);
			\draw[dashed] (I)--(A)--(C);
			\foreach \d/\g in {A/180,B/-90,C/0,S/90,I/45}
			\fill[black](\d) circle (1pt)+(\g:.3)node{$\d$};
			\foreach \a/\b/\c/\t in {A/B/C/6,S/A/B/5,S/A/C/6}{\def\k{\t pt}\draw ($(\b)!\k!(\a)$)--($(\b)!2!($($(\b)!\k!(\a)$)!.5!($(\b)!\k!(\c)$)$)$)--($(\b)!\k!(\c)$);}
			\end{tikzpicture}
		}
	}
\end{ex}
\begin{ex}%Câu 21.%[Paul Hieu Nguyen]%[2H2K2-2]
	Cho hình chóp đều $S.ABCD$ có $AB=2$ và $SA=3\sqrt{2}$. Bán kính của mặt cầu ngoại tiếp hình chóp đã cho bằng
	\choice
	{$\dfrac{\sqrt{33}}{4}$}
	{$\dfrac{7}{4}$}
	{$2$}
	{\True $\dfrac{9}{4}$}
	\loigiai{
		\immini{
			Gọi $H$ là tâm hình vuông $ABCD$ thì $SH$ là trục của $ABCD$.\\
			Gọi $M$ là trung điểm của $SD$, trong mp $(SDH)$ kẻ đường trung trực của đoạn $SD$ cắt $SH$ tại $O$ thì $OS=OA=OB=OC=OD$ nên $O$ chính là tâm của mặt cầu ngoại tiếp hình chóp $S.ABCD$. Bán kính mặt cầu là $R=SO$.
			Ta có $\triangle SMO\backsim\triangle SHD$ nên $\dfrac{SO}{SD}=\dfrac{SM}{SH}\Leftrightarrow R=SO=\dfrac{SD\cdot SN}{SH}=\dfrac{SD^2}{2SH}$.\\
			Với $SH^2=SD^2-HD^2=16\Rightarrow SH=4$.\\
			Vậy $R=\dfrac{SD^2}{2SH}=\dfrac{9}{4}$.}
		{
			\begin{tikzpicture}[>=stealth,line join=round,line cap=round,font=\footnotesize,scale=.8]
			\coordinate (A) at (0,0);
			\coordinate (B) at (3,0);
			\coordinate (D) at (-1.5,-1.5);
			\coordinate (C) at ($(B)+(D)-(A)$);
			\coordinate (H) at ($(A)!.5!(C)$);
			\coordinate (S) at ($(H)+(0,3.7)$);
			\coordinate (M) at ($(S)!.5!(D)$);
			\coordinate (O) at ($(S)!.72!(H)$);
			\draw (S)--(B)--(C)--(D)--(S)--(C);
			\draw[dashed] (B)--(A)--(D)--(B) (H)--(S)--(A)--(C) (M)--(O);
			\foreach \d/\g in {A/160,B/0,C/-60,D/200,S/90,H/-105,M/150,O/30}
			\fill[black](\d) circle (1pt)+(\g:.25)node{$\d$};
			\foreach \a/\b/\c/\t in {S/M/O/6}{\def\k{\t pt}\draw ($(\b)!\k!(\a)$)--($(\b)!2!($($(\b)!\k!(\a)$)!.5!($(\b)!\k!(\c)$)$)$)--($(\b)!\k!(\c)$);}
			\end{tikzpicture}
		}
	}
\end{ex}

%\Closesolutionfile{ans}
%\begin{indapan}
%	{10}{ans/ansCD2D3-3.1-2}
%\end{indapan}
%--------------------------------------------------------------------------------------------------------
%\Opensolutionfile{ans}[ans/ansGV19]
\begin{ex}%Câu 22.%[2H2K2-2]
	Cho hình chóp $S.ABCD$ có đáy $ABCD$ là hình chữ nhật, $AB=3, AD=4$ và các cạnh bên của hình chóp tạo với đáy một góc $60^\circ$. Tính thể tích của khối cầu ngoại tiếp hình chóp đã cho. 
	\choice
	{$V=\dfrac{250\sqrt 3}3\pi$}
	{$V=\dfrac{125\sqrt 3}6\pi$}
	{\True $V=\dfrac{500\sqrt 3}{27}\pi$}
	{$V=\dfrac{50\sqrt 3}{27}\pi$}
	\loigiai{
		\immini{
			Gọi $O=A C\cap B D$. Do các cạnh bên của hình chóp tạo với đáy một góc $60^\circ$ nên $SO\perp(ABCD)$ hay $SO$ là trục của đường tròn ngoại tiếp đa giác đáy.\\
			Gọi $M$ là trung điểm của cạnh $SB$, trong mặt phẳng $(SBO)$ kẻ đường thẳng qua $M$ và vuông góc với $SB$, cắt $SO$ tại $I$. Khi đó ta có $I A=I B=I C=I D=I S $ hay $I$ là tâm mặt cầu ngoại tiếp hình chóp $S.ABCD$.\\
			Theo giả thiết ta có $A B=3, A D=4 $ nên $B O=\dfrac 52$. Mà góc giữa $SB$ và mặt phẳng $(ABCD)$ bằng $60^\circ$ hay $\widehat{S B O}=60^{\circ}$. Suy ra $SB=\dfrac{B O}{\cos 60^{\circ}}=5 \Rightarrow S O=\dfrac{5\sqrt 3}2$.\\
			Ta có $\Delta S M I\backsim\Delta S O B$ nên $S I=\dfrac{S M\cdot S B}{S O}=\dfrac{5\cdot\dfrac 52}{5\sqrt 3}=\dfrac{5\sqrt 3}3$.\\
			Vậy thể tích của khối cầu ngoại tiếp hình chóp $S.ABCD$ là $V=\dfrac 43\pi\left(\dfrac{5\sqrt 3}3\right)^3=\dfrac{500\sqrt 3}{27}\pi$.}
		{\begin{tikzpicture}[scale=1, font=\footnotesize, line join=round, line cap=round, >=stealth]
			\def\bc{4} % cạnh BC
			\def\ba{2} % cạnh BA
			\def\h{4} % đường cao
			\def\gocB{30} % góc B của đáy
			\coordinate[label=below left:$B$] (B) at (0,0);
			\coordinate[label=above right:$A$] (A) at (\gocB:\ba);
			\coordinate[label=below:$C$] (C) at (\bc,0);
			\coordinate[label=right:$D$] (D) at ($(C)-(B)+(A)$);
			\coordinate[label=below:$O$] (O) at ($(A)!.5!(C)$);
			\coordinate[label=above:$S$] (S) at ($(O)+(90:\h)$);
			\coordinate[label=left:$M$] (M) at ($(S)!0.5!(B)$);
			\coordinate[label=right:$I$] (I) at ($(S)!0.7!(O)$);
			\draw (B)--(C)--(D)--(S)--cycle (S)--(C);
			\draw[dashed] (C)--(A)--(D)--(B) (O)--(S)--(A)--(B) (M)--(I);
			\foreach \diem in {A,B,C,D,S,O,M,I}	\fill (\diem)circle(1.5pt);
			\end{tikzpicture}
		}
	}
\end{ex}
\begin{ex}%Câu 23.%[2H2B2-2]
	Cho khối chóp $S.ABC$ có đáy là tam giác vuông tại $B$, $AB=1$, $BC=\sqrt{2}$, cạnh bên $SA$ vuông góc với đáy và $SA=\sqrt 3$. Diện tích mặt cầu ngoại tiếp hình chóp $S.ABC$ bằng
	\choice
	{\True $6\pi$}
	{$\dfrac{3\pi}{2}$}
	{$12\pi$}
	{$2\pi$}
	\loigiai{
		\immini{
			Gọi $I$ là trung điểm của $SC$. Tam giác $SAC$ vuông tại $A$.\\
			Suy ra $I A=I S=I C=\dfrac 12S C$ $(1)$.\\
			Dễ dàng chứng minh được $B C\perp(SAB)\Rightarrow BC\perp SB $ hay tam giác $SBC$ vuông tại $B$.\\
			$\Rightarrow I B=I S=I C=\dfrac 12S C$ $(2)$.\\
			Từ $(1)$ và $(2)$ suy ra $I A=I B=I C=I S=\dfrac 12S C$ hay $I$ là tâm mặt cầu ngoại tiếp hình chóp $S.ABC$ có bán kính $R=\dfrac 12S C=\dfrac 12\sqrt{S A^2+A C^2}=\dfrac 12\sqrt{S A^2+A B^2+B C^2}=\dfrac{\sqrt 6}2$.\\
			Vậy diện tích mặt cầu cần tìm là $S=4\pi R^2=6\pi$.}
		{\begin{tikzpicture}[scale=1, font=\footnotesize, line join=round, line cap=round, >=stealth]
			\def\ac{4} % cạnh AC
			\def\ab{2} % cạnh AB
			\def\h{3.5} % chiều cao
			\def\gocA{50} % góc A của đáy
			\coordinate[label=left:$A$] (A) at (0,0);
			\coordinate[label=right:$C$] (C) at (\ac,0);
			\coordinate[label=below left:$B$] (B) at (-\gocA:\ab);
			\coordinate[label=above:$S$] (S) at ($(A)+(90:\h)$);
			\coordinate[label=above right:$I$] (I) at ($(S)!0.5!(C)$);
			\draw (A)--(B)--(C)--(S)--cycle (S)--(B)--(I);
			\draw[dashed] (I)--(A)--(C);
			\foreach \diem in {A,B,C,S,I}	\fill (\diem)circle(1.5pt);
			\newcommand{\gocv}[4][black]{\draw[#1] ($(#3)!5pt!(#2)$)--($(#3)!2!($($(#3)!5pt!(#2)$)!.5!($(#3)!5pt!(#4)$)$)$)--($(#3)!5pt!(#4)$);}
			\gocv{S}{A}{C} 
			\gocv{A}{B}{C} 
			\end{tikzpicture}
		}
	}
\end{ex}
\begin{ex}%Câu 24.%[2H2B2-2]
	Cho hình chóp $S.ABCD$ có $\triangle ABC$ vuông tại $B$, $BA=a$, $BC=a\sqrt 3$. Cạnh bên $SA$ vuông góc với đáy và $SA=a$. Tính bán kính của mặt cầu ngoại tiếp hình chóp $S.ABC$. 
	\choice
	{\True $R=\dfrac{a\sqrt 5}2$}
	{$R=\dfrac{a\sqrt 5}4$}
	{$R=2a\sqrt 5$}
	{$R=a\sqrt 5$}
	\loigiai{
		\immini{
			Tâm của mặt cầu ngoại tiếp chóp $S.ABC$ là trung điểm $I$ của $SC$.\\ 
			Ta có $A C=\sqrt{A B^2+B C^2}=2a$.\\
			Khi đó $S C=\sqrt{S A^2+A C^2}=\sqrt{a^2+4a^2}=a\sqrt 5$.\\
			Vậy $R=S I=\dfrac{S C}2=\dfrac{a\sqrt 5}2$.}
		{\begin{tikzpicture}[scale=1, font=\footnotesize, line join=round, line cap=round, >=stealth]
			\def\ac{4} % cạnh AC
			\def\ab{2} % cạnh AB
			\def\h{3.5} % chiều cao
			\def\gocA{50} % góc A của đáy
			\coordinate[label=left:$A$] (A) at (0,0);
			\coordinate[label=right:$C$] (C) at (\ac,0);
			\coordinate[label=below left:$B$] (B) at (-\gocA:\ab);
			\coordinate[label=above:$S$] (S) at ($(A)+(90:\h)$);
			\coordinate[label=above right:$I$] (I) at ($(S)!0.5!(C)$);
			\draw (A)--(B)--(C)--(S)--cycle (S)--(B)--(I);
			\draw[dashed] (I)--(A)--(C);
			\foreach \diem in {A,B,C,S,I}	\fill (\diem)circle(1.5pt);
			\newcommand{\gocv}[4][black]{\draw[#1] ($(#3)!5pt!(#2)$)--($(#3)!2!($($(#3)!5pt!(#2)$)!.5!($(#3)!5pt!(#4)$)$)$)--($(#3)!5pt!(#4)$);}
			\gocv{S}{A}{C} 
			\gocv{A}{B}{C} 
			\end{tikzpicture}}
	}
\end{ex}
\begin{ex}%Câu 25.%[2H2B2-2]
	Cho tứ diện $ABCD$ có $AD$ vuông góc với mặt phẳng $(ABC)$, tam giác $ABC$ vuông cân tại $A$, $AD=2a$, $AB=a$. Bán kính mặt cầu ngoại tiếp tứ diện $ABCD$ bằng 
	\choice
	{$\dfrac{a\sqrt 6}3$}
	{\True $\dfrac{a\sqrt 6}2$}
	{$\dfrac{a\sqrt 6}4$}
	{$\dfrac{a\sqrt 2}2$}
	\loigiai{
		\immini{
			Bán kính đường tròn ngoại tiếp tam giác $ABC$ là $r=\dfrac{B C}2=\dfrac{a\sqrt 2}2$.\\
			Bán kính mặt cầu ngoại tiếp tứ diện $ABCD$
			\[R=\sqrt{\left(\dfrac{A D}2\right)^2+r^2}=\sqrt{a^2+\dfrac{a^2}2}=\dfrac{a\sqrt 6}2.\]}
		{\begin{tikzpicture}[scale=1, font=\footnotesize, line join=round, line cap=round, >=stealth]
			\def\ac{4} % cạnh AC
			\def\ab{2} % cạnh AB
			\def\h{3} % chiều cao
			\def\gocA{50} % góc A của đáy
			\coordinate[label=left:$A$] (A) at (0,0);
			\coordinate[label=right:$C$] (C) at (\ac,0);
			\coordinate[label=below left:$B$] (B) at (-\gocA:\ab);
			\coordinate[label=above:$D$] (D) at ($(A)+(90:\h)$);
			\draw (A)--(B)--(C)--(D)--cycle (D)--(B);
			\draw[dashed] (A)--(C);
			\foreach \diem in {A,B,C,D}	\fill (\diem)circle(1.5pt);
			\newcommand{\gocv}[4][black]{\draw[#1] ($(#3)!5pt!(#2)$)--($(#3)!2!($($(#3)!5pt!(#2)$)!.5!($(#3)!5pt!(#4)$)$)$)--($(#3)!5pt!(#4)$);}
			\gocv{D}{A}{C}
			\gocv{B}{A}{C}
			\end{tikzpicture}
		}
	}
\end{ex}
\begin{ex}%Câu 26.%[2H2K2-2]
	\immini{
		Cho khối chóp $S.ABC$ có $SA$ vuông góc với $(ABC)$ và $SA=a$. Đáy $ABC$ nội tiếp trong đường tròn tâm $I$ có bán kính bằng $2a$ (hình vẽ). Tính bán kính mặt cầu ngoại tiếp khối chóp $S.ABC$. 	
		\choice
		{$\dfrac{a\sqrt 5}2$}
		{\True $\dfrac{a\sqrt{17}}2$}
		{$a\sqrt 5$}
		{$\dfrac{a\sqrt 5}3$}}
	{\begin{tikzpicture}[line cap=round,line join=round]
		\def\a{2}
		\def\b{0.8}
		\def\c{3}
		\coordinate[label=below:{$I$}] (I) at (0,0);
		\coordinate[label=above right:{$C$}] (C) at (20:\a cm and \b cm);
		\coordinate[label=left:{$A$}] (A) at (180:\a cm and \b cm);
		\coordinate[label=below:{$B$}] (B) at (260:\a cm and \b cm);
		\coordinate[label=above:{$S$}] (S) at ($(A)+(0,\c)$);
		\tkzDefPointBy[rotation = center I angle 180](A)\tkzGetPoint{M}
		\draw[dashed] (A)--(C) ($(I)-(\a,0)$) arc (180:20:\a cm and \b cm);
		\draw(S)--(A)--(B)--(C)--(S)--(B) ($(I)-(\a,0)$) arc (-180:20:\a cm and \b cm);
		\foreach \diem in {A,B,C,S,I}	\fill (\diem)circle(1.5pt);
		\end{tikzpicture}}
	\loigiai{
		\immini{
			Gọi $\Delta$ là đường thẳng qua $I$ và $\Delta\perp(A B C)$. Gọi $M$ là trung điểm của $SA$, mặt phẳng trung trực của đoạn thẳng $SA$ cắt $\Delta$ tại $O$.\\
			Khi đó $O$ là tâm mặt cầu ngoại tiếp hình chóp $S.ABC$, bán kính $R=OA$.\\ 
			\[O A=\sqrt{A I^2+O I^2}=\sqrt{4a^2+\dfrac{a^2}4}=\dfrac{a\sqrt{17}}2.\]}
		{\begin{tikzpicture}[line cap=round,line join=round]
			\def\a{2}
			\def\b{0.8}
			\def\c{3}
			\coordinate[label=below:{$I$}] (I) at (0,0);
			\coordinate(D) at (0,1);
			\coordinate(E) at (0,2.5);
			\coordinate(F) at (-0.4,1.44);
			\coordinate(G) at (0.5,1.42);
			\coordinate[label=left:{$M$}](M) at (-2,1.5);
			\coordinate[label=left:{$\Delta$}](K) at (0.6,2.4);
			\coordinate[label=above right:{$C$}] (C) at (20:\a cm and \b cm);
			\coordinate[label=left:{$A$}] (A) at (180:\a cm and \b cm);
			\coordinate[label=below:{$B$}] (B) at (260:\a cm and \b cm);
			\coordinate[label=above:{$S$}] (S) at ($(A)+(0,\c)$);
			\coordinate[label=above right:{$O$}](O) at (0,1.42);
			\draw[dashed] (M)--(F) (A)--(I)--(D) (A)--(C) ($(I)-(\a,0)$) arc (180:20:\a cm and \b cm);
			\draw (F)--(G) (E)--(D) (S)--(A)--(B)--(C)--(S)--(B) ($(I)-(\a,0)$) arc (-180:20:\a cm and \b cm);
			\foreach \diem in {A,B,C,S,I,M,O}	\fill (\diem)circle(1.5pt);
			\end{tikzpicture}}
	}
\end{ex}
\begin{ex}%Câu 27.%[2H2K2-2]
	Cho hình chóp $S.ABCD$ có đáy $ABCD$ là hình thoi cạnh $a$, góc $\widehat{B A D}=120^{\circ}$. Cạnh bên $SA$ vuông góc với đáy và $SA=3a$. Tính bán kính $R$ của mặt cầu ngoại tiếp khối chóp $S.BCD$. 
	\choice
	{$R=\dfrac{\sqrt 3a}3$}
	{$R=\dfrac{\sqrt 5a}3$}
	{\True $R=\dfrac{5a}3$}
	{$R=\dfrac{4a}3$}
	\loigiai{
		\immini{
			Xét hình thoi $ABCD$ có $\widehat{B A D}=120^{\circ}$ nên $A D=A C=A B$, suy ra $A$ là tâm đường tròn ngoại tiếp đa giác đáy $BCD$.\\
			Theo giả thiết $SA$ vuông góc với đáy $(ABCD)$ nên đường thẳng $SA$ là trục của đáy $(BCD)$.\\
			Gọi $M$ là trung điểm $SD$, trong mặt phẳng $(SAD)$ kẻ đường thẳng $d$ vuông góc với $SD$ tại $M$, $d$ cắt $SA$ tại $I$.\\
			Ta có $I$ là tâm mặt cầu ngoại tiếp khối chóp $S.BCD$. Khi đó $R=IS$.\\
			Ta có $\Delta I S M\backsim\Delta D S A\Rightarrow\dfrac{I S}{D S}=\dfrac{S M}{S A}$\\
			$\Rightarrow I S=\dfrac{S M\cdot D S}{S A}=\dfrac{\dfrac{a\sqrt{10}}2\cdot a\sqrt{10}}{3a}=\dfrac{5a}3$.}
		{\begin{tikzpicture}[scale=1, font=\footnotesize, line join=round, line cap=round, >=stealth]
			\def\bc{4} % cạnh BC
			\def\ba{2} % cạnh BA
			\def\h{4} % đường cao
			\def\gocB{30} % góc B của đáy
			\coordinate[label=below left:$B$] (B) at (0,0);
			\coordinate[label=above left:$A$] (A) at (\gocB:\ba);
			\coordinate[label=below:$C$] (C) at (\bc,0);
			\coordinate[label=right:$D$] (D) at ($(C)-(B)+(A)$);
			\coordinate[label=above:$S$] (S) at ($(A)+(90:\h)$);
			\coordinate[label=above right:$M$] (M) at ($(S)!0.5!(D)$);
			\coordinate[label=above left:$I$] (I) at ($(S)!0.7!(A)$);
			\draw (B)--(C)--(D)--(S)--cycle (S)--(C);
			\draw[dashed] (A)--(D) (S)--(A)--(B)--(D) (M)--(I);
			\foreach \diem in {A,B,C,D,S,M,I}	\fill (\diem)circle(1.5pt);
			\newcommand{\gocv}[4][black]{\draw[#1] ($(#3)!5pt!(#2)$)--($(#3)!2!($($(#3)!5pt!(#2)$)!.5!($(#3)!5pt!(#4)$)$)$)--($(#3)!5pt!(#4)$);}
			\gocv{S}{A}{D}
			\gocv{S}{M}{I}
			\end{tikzpicture}
		}
	}
\end{ex}
\begin{ex}%Câu 28.%[2H2K2-2]
	Cho hình chóp $S.ABC$ có $SA$ vuông góc với $(ABC)$, $AB=a$, $AC=a\sqrt 2$, $\widehat{BAC}=45^\circ$. Gọi $B_1$, $C_1$ lần lượt là hình chiếu vuông góc của $A$ lên $SB$, $SC$. Tính thể tích mặt cầu ngoại tiếp hình chóp $A.BCC_1B_1$. 
	\choice
	{\True $V=\dfrac{\pi a^3\sqrt 2}3$}
	{$V=\pi a^3\sqrt 2$}
	{$V=\dfrac 43\pi a^3$}
	{$V=\dfrac{\pi a^3}{\sqrt 2}$}
	\loigiai{
		\immini{
			Ta có $B C^2=a^2+2a^2-2\cdot a\cdot a\sqrt 2\cdot\cos 45^{\circ}=a^2\Rightarrow B C=a$.\\
			Suy ra tam giác $A B C $ vuông tại $B$. \\
			Ta có \\
			$\heva{&BC\perp AB\\ &BC\perp SA}\Rightarrow BC\perp(SAB)\Rightarrow BC\perp AB_1$.\\
			$\heva{&AB_1\perp CB\\ &AB_1\perp SB}\Rightarrow AB_1\perp(SBC)\Rightarrow AB_1\perp CB_1$.\\
			Gọi $I$ là trung điểm $AC$, suy ra $IC=IA=IB$.\\
			Tam giác $AB_1C$ vuông tại $B_1$ suy ra $IC=IA=IB_1$.\\
			Tam giác $AC_1C$ vuông tại $C_1$ suy ra $IC=IA=IC_1$.\\
		}
		{\begin{tikzpicture}[scale=1, font=\footnotesize, line join=round, line cap=round, >=stealth]
			\def\ac{4} % cạnh AC
			\def\ab{2.5} % cạnh AB
			\def\h{3.5} % chiều cao
			\def\gocA{45} % góc A của đáy
			\coordinate[label=left:$A$] (A) at (0,0);
			\coordinate[label=right:$C$] (C) at (\ac,0);
			\coordinate[label=below left:$B$] (B) at (-\gocA:\ab);
			\coordinate[label=above:$S$] (S) at ($(A)+(90:\h)$);
			\coordinate[label=above left:$B_1$] (B_1) at ($(S)!0.6!(B)$);
			\coordinate[label=above right:$C_1$] (C_1) at ($(S)!0.4!(C)$);
			\coordinate[label=below right:$I$] (I) at ($(A)!0.5!(C)$);
			\draw (C_1)--(B_1)--(A) (A)--(B)--(C)--(S)--cycle (S)--(B) (B_1)--(C);
			\draw[dashed] (C_1)--(A)--(C) (B_1)--(I)--(C_1) (I)--(B);
			\foreach \diem in {A,B,C,S,B_1,C_1,I}	\fill (\diem)circle(1.5pt);
			\newcommand{\gocv}[4][black]{\draw[#1] ($(#3)!5pt!(#2)$)--($(#3)!2!($($(#3)!5pt!(#2)$)!.5!($(#3)!5pt!(#4)$)$)$)--($(#3)!5pt!(#4)$);}
			\gocv{S}{A}{C}
			\gocv{B}{B_1}{A}
			\gocv{S}{C_1}{A}
			\end{tikzpicture}
		}
		\noindent Do đó hình chóp $A.BCC_1B_1$ nội tiếp mặt cầu tâm $I$, bán kính $R=IA=\dfrac{a\sqrt 2}{2}$.\\
		Vậy thể tích mặt cầu ngoại tiếp hình chóp $A.BCC_1B_1$ là \[V=\dfrac 43\pi\left(\dfrac{a\sqrt 2}2\right)^3=\dfrac{a^3\pi\sqrt 2}3.\]
	}
\end{ex}
\begin{ex}%Câu 29.%[2H2K2-2]
	Cho hình chóp $S.ABCD$ có đáy $ABCD$ là hình vuông cạnh $a$. Cạnh bên $SA$ vuông góc với mặt đáy và $SA=a$. Gọi $E$ là trung điểm của cạnh $CD$. Mặt cầu đi qua bốn điểm $S$, $A$, $B$, $E$ có bán kính là
	\choice
	{\True $\dfrac{a\sqrt{41}}8$}
	{$\dfrac{a\sqrt{41}}{24}$}
	{$\dfrac{a\sqrt{41}}{16}$}
	{$\dfrac{a\sqrt 2}{16}$}
	\loigiai{
		\immini{
			Gọi $H$, $M$ lần lượt là trung điểm của $AB$, $SA$.\\
			Gọi $I$ là tâm của đường tròn ngoại tiếp $\triangle ABE$, $d$ là trục của đường tròn ngoại tiếp $\triangle ABE$.\\
			Suy ra $d\parallel SA$, $A I\cap B C=K $.\\
			Giả sử $d\cap S K=O $ thì $O$ là giao của mặt phẳng trung trực của cạnh bên $SA$ và trục $d$ của đáy $ABE$ nên $O$ là trung điểm của $SK$ và là tâm mặt cầu ngoại tiếp hình chóp $S.ABE$.\\
			Do đó tứ giác $MOIA$ là hình chữ nhật nên $IA=MO$.\\
			Ta có $A E=B E=\sqrt{a^2+\left(\dfrac a2\right)^2}=\dfrac{a\sqrt 5}2$ nên trong $\triangle ABE$, $S_{\Delta A B E}=\dfrac 12a^2=\dfrac{A B\cdot A E\cdot B E}{4R}$ với $R$ là bán kính đường tròn ngoại tiếp tam giác $ABE$.\\
		}
		{\begin{tikzpicture}[scale=1, font=\footnotesize, line join=round, line cap=round, >=stealth]
			\def\bc{4} % cạnh BC
			\def\ba{2} % cạnh BA
			\def\h{3.5} % đường cao
			\def\gocB{30} % góc B của đáy
			\coordinate[label=below left:$B$] (B) at (0,0);
			\coordinate[label=above left:$A$] (A) at (\gocB:\ba);
			\coordinate[label=below:$C$] (C) at (\bc,0);
			\coordinate[label=right:$D$] (D) at ($(C)-(B)+(A)$);
			\coordinate[label=above:$S$] (S) at ($(A)+(90:\h)$);
			\coordinate[label=left:$H$] (H) at ($(A)!0.5!(B)$);
			\coordinate[label=left:$M$] (M) at ($(A)!0.5!(S)$);
			\coordinate[label=below right:$E$] (E) at ($(D)!0.5!(C)$);
			\coordinate[label=below:$K$] (K) at ($(B)!0.7!(C)$);			
			\coordinate[label=right:$O$] (O) at ($(K)!0.5!(S)$);
			\tkzInterLL(A,K)(H,E) \tkzGetPoint{I} \tkzLabelPoints[below](I)	
			\draw (B)--(C)--(D)--(S)--cycle (K)--(S)--(C);
			\draw[dashed] (A)--(D) (S)--(A)--(B)--(E)--(H) (E)--(A)--(K) (I)--(O)--(M);
			\foreach \diem in {A,B,C,D,S,H,M,E,K,I,O}	\fill (\diem)circle(1.5pt);
			\newcommand{\gocv}[4][black]{\draw[#1] ($(#3)!5pt!(#2)$)--($(#3)!2!($($(#3)!5pt!(#2)$)!.5!($(#3)!5pt!(#4)$)$)$)--($(#3)!5pt!(#4)$);}
			\gocv{S}{A}{D}
			\end{tikzpicture}
		}
		\noindent Suy ra $R=\dfrac{A B\cdot A E\cdot B E}{2a^2}=\dfrac{5a}8=I A$.\\
		Xét $SMO$ vuông tại $M$ có 
		$r=S O=\sqrt{S M^2+M O^2}=\sqrt{\left(\dfrac{S A}2\right)^2+IA^2}=\sqrt{\dfrac{a^2}4+\dfrac{25a^2}{64}}=\dfrac{a\sqrt{41}}8$ là bán kính mặt cầu ngoại tiếp hình chóp $S.ABE$.
	}
\end{ex}
\begin{ex}%Câu 30.%[2H2G2-2]
	Cho tứ diện $ABCD$ có $AB=4a$, $CD=6a$, các cạnh còn lại có độ dài $a\sqrt{22}$. Tính bán kính $R$ của mặt cầu ngoại tiếp tứ diện $ABCD$. 
	\choice
	{$R=\dfrac{a\sqrt{79}}3$}
	{$R=\dfrac{5a}2$}
	{\True $R=\dfrac{a\sqrt{85}}3$}
	{$R=3a$}
	\loigiai{
		\immini{
			Gọi $M$, $N$ lần lượt là trung điểm của $CD$ và $AB$.\\
			Ta có $\heva{&AB\perp CN\\ &AB\perp DN}\Rightarrow AB\perp (NCD) \Rightarrow AB\perp MN$. \\
			Tương tự $CD\perp MN$.\\
			Suy ra $MN$ là đoạn vuông góc chung của $AB$ và $CD$.\\
			Gọi $I$ là tâm mặt cầu ngoại tiếp tứ diện $ABCD$ thì $I$ thuộc $MN$.\\
			Xét tam giác $NAC$ vuông tại $N$ có \\$C N=\sqrt{A C^2-N A^2}=\sqrt{22a^2-4a^2}=3\sqrt 2a$.\\
			Xét tam giác $CMN$ vuông tại $M$ có \\$M N=\sqrt{C N^2-C M^2}=\sqrt{18a^2-9a^2}=3a$.\\
		}
		{\begin{tikzpicture}[scale=1, font=\footnotesize, line join=round, line cap=round, >=stealth]
			\def\ac{4} % cạnh AC
			\def\ab{2} % cạnh AB
			\def\as{4} % cạnh AS
			\def\gocA{50} % góc A của đáy
			\coordinate[label=left:$A$] (A) at (0,0);
			\coordinate[label=right:$C$] (C) at (\ac,0);
			\coordinate[label=below left:$B$] (B) at (-\gocA:\ab);
			\coordinate[label=above:$D$] (D) at (60:\as);
			\coordinate[label=above right:$M$] (M) at ($(C)!0.5!(D)$);
			\coordinate[label=left:$N$] (N) at ($(A)!0.5!(B)$);
			\coordinate[label=left:$I$] (I) at ($(M)!0.2!(N)$);
			\draw (A)--(B)--(C)--(D)--cycle (D)--(B);
			\draw[dashed] (A)--(C) (M)--(N);
			\foreach \diem in {A,B,C,D,M,N,I}\fill (\diem)circle(1.5pt);
			\end{tikzpicture}
		}
		\noindent Lại có
		\allowdisplaybreaks
		\begin{eqnarray*}
			\heva{&I M+I N=3a\\ &I M^2+M C^2=I N^2+N A^2}&\Leftrightarrow&\heva{&I M+I N=3a\\ &I M^2-I N^2=N A^2-M C^2}\\
			&\Leftrightarrow& \heva{&I M+I N=3a\\ &(I M+I N)(I M-I N)=-5a^2}\\
			&\Leftrightarrow& \heva{&I M+I N=3a\\ &I M-I N=-\dfrac 53a}\Leftrightarrow\heva{&I M=\dfrac 23a\\ &I N=\dfrac 73a.}
		\end{eqnarray*}
		Vậy bán kính cần tìm là $R=IC=\sqrt{I M^2+M C^2}=\sqrt{\dfrac 49a^2+9a^2}=\dfrac{\sqrt{85}}3a$.
	}
\end{ex}
\begin{ex}%Câu 31.%[2H2K2-2]
	Cho hình chóp $S.ABCD$ có $S A\perp(ABC)$ và $SA=2a$. Biết tam giác $ABC$ cân tại $A$ có $BC=2a\sqrt 2$, $\cos\widehat{ACB}=\dfrac 13$. Tính diện tích mặt cầu ngoại tiếp hình chóp $S.ABCD$. 
	\choice
	{$S=\dfrac{65\pi a^2}4$}
	{$S=13\pi a^2$}
	{\True $S=\dfrac{97\pi a^2}4$}
	{$S=4\pi a^2$}
	\loigiai{
		\immini{
			Gọi $M$, $N$ lần lượt là trung điểm $BC$ và $SA$, $O$ là tâm đường tròn ngoại tiếp tam giác $ABC$.\\
			Do $\triangle ABC$ cân tại $A$ nên $O\in AM$.\\
			Qua $O$ dựng $\Delta$ là trục đường tròn ngoại tiếp tam giác $\triangle ABC$ $(\Delta\parallel SA)$.\\
			Trong $(SAM)$, kẻ đường thẳng qua $N$ vuông góc với $SA$ cắt $\Delta$ tại $I$. Khi đó $IS=IA=IB=IC$ nên $I$ là tâm mặt cầu ngoại tiếp hình chóp $S.ABC$.\\
			$\triangle AMC$ có $\cos\overrightarrow{A C M}=\dfrac{M C}{A C}\Rightarrow A B=A C=3a\sqrt 2$. \\
			$S_{\triangle A B C}=\dfrac 12\cdot C A\cdot C B\cdot\sin\widehat{A C B}=\dfrac 12\cdot 3a\sqrt 2\cdot 2a\sqrt 2\cdot\sqrt{1-\left(\dfrac 13\right)^2}=4a^2\sqrt 2$.\\
			Mà $S_{A B C}=\dfrac{A B\cdot A C\cdot B C}{4\cdot O A}\Rightarrow O A=\dfrac 94a$.\\
		}
		{\begin{tikzpicture}[scale=0.7, line join = round, line cap = round]
			\tikzset{label style/.style={font=\footnotesize}}
			\tkzDefPoints{0/0/A,7/0/C,3/-3/B,3/1.5/I}
			\coordinate (S) at ($(A)+(0,5)$);
			\coordinate (M) at ($(B)!0.5!(C)$);
			\coordinate (N) at ($(A)!0.5!(S)$);
			\coordinate (O) at ($(A)!0.6!(M)$);			
			\tkzDrawPolygon(S,A,B,C)
			\tkzDrawSegments(S,B)
			\tkzDrawSegments[dashed](A,C N,I O,I A,M)
			\tkzDrawPoints(A,B,C,S,O,I,M,N)
			\tkzLabelPoints[above](S,I)
			\tkzLabelPoints[below](B,O)
			\tkzLabelPoints[left](A)
			\tkzLabelPoints[right](C)
			\tkzLabelPoints[below right](M)
			\tkzLabelPoints[left](N)
			\end{tikzpicture}}
		\noindent Tứ giác $NAOI$ là hình chữ nhật nên $AI=\sqrt{N A^2+A O^2}=\dfrac{\sqrt{97}a}4$.\\
		Suy ra bán kính mặt cầu $R=\dfrac{\sqrt{97}a}4$.\\
		Vậy diện tích mặt cầu là $S=4\pi R^2=\dfrac{97\pi a^2}4$.
	}
\end{ex}
\begin{ex}%Câu 32.%[2H2G2-2]
	Cho hình chóp $S.ABC$ có đáy là tam giác vuông tại $A$, $AB=a$, $AC=2a$. Mặt bên $SAB$, $SAC$ lần lượt là các tam giác vuông tại $B$, $C$. Biết thể tích khối chóp $S.ABC$ bằng $\dfrac 23a^3$. Bán kính mặt cầu ngoại tiếp hình chóp $S.ABC$ là
	\choice
	{$R=a\sqrt 2$}
	{$R=a$}
	{\True $R=\dfrac{3a}2$}
	{$R=\dfrac{\sqrt 3a}2$}
	\loigiai{
		\immini{
			Gọi $H$ là hình chiếu của $S$ trên mặt phẳng $(ABC)$ thì $SH$ là đường cao của hình chóp.\\
			Do $H,\,B,\,C$ cùng nhìn $SA$ dưới một góc vuông nên năm điểm $A,\, B,\, H,\, C,\, S $ cùng thuộc mặt cầu ngoại tiếp hình chóp $S.ABC$ có tâm $I$ là trung điểm của $SA$.\\
			Mặt khác thể tích khối chóp $S.ABC$ bằng $\dfrac 23a^3$ nên ta có\\
			$\dfrac 13\cdot \dfrac 12\cdot AB\cdot AC\cdot SH=\dfrac 23a^3\Leftrightarrow S H=2a$.\\
			
			Mặt khác $A,\,B,\,H,\,C$ cùng thuộc một mặt phẳng nên tứ giác $ABHC$ nội tiếp đường tròn.\\
			Mà $\widehat{B A C}=90^{\circ}\Rightarrow\widehat{B H C}=90^{\circ}\Rightarrow H M=\dfrac{B C}2=\dfrac{a\sqrt 5}2$\\
			$\Rightarrow S M=\sqrt{H M^2+S H^2}=\dfrac{a\sqrt{21}}2$.\\
		}
		{\begin{tikzpicture}[scale=1, font=\footnotesize, line join=round, line cap=round, >=stealth]
			\def\ad{4} % cạnh AD
			\def\ab{2} % cạnh AB
			\def\bc{2} % chéo AC
			\def\as{3.5} % cạnh AS
			\def\gocA{50} % góc A của đáy
			\def\gocB{120} % góc B của đáy
			\coordinate[label=left:$H$] (H) at (0,0);
			\coordinate[label=below left:$B$] (B) at (-\gocA:\ab);
			\coordinate[label=below right:$A$] (A) at ($(B)+(180-\gocA-\gocB:\bc)$);
			\coordinate[label=right:$C$] (C) at (\ad,0);
			\coordinate[label=above:$S$] (S) at (90:\as); 
			\coordinate[label=below:$M$] (M) at ($(B)!0.5!(C)$);
			\coordinate[label=above:$I$] (I) at ($(S)!0.5!(A)$);
			\draw (H)--(B)--(A)--(C)--(S)--cycle (B)--(S)--(A) (I)--(C) (I)--(B);
			\draw[dashed] (H)--(C)--(B) (S)--(M)--(H);
			\foreach \diem in {H,B,C,A,S,M,I}	\fill (\diem)circle(1.5pt);
			\newcommand{\gocv}[4][black]{\draw[#1] ($(#3)!5pt!(#2)$)--($(#3)!2!($($(#3)!5pt!(#2)$)!.5!($(#3)!5pt!(#4)$)$)$)--($(#3)!5pt!(#4)$);}
			\gocv{B}{A}{C}
			\gocv{B}{H}{C}
			\gocv{S}{H}{C}	\gocv{S}{B}{A} \gocv{S}{C}{A}
			\end{tikzpicture}
		}
		\noindent Áp dụng công thức đường trung tuyến ta có\\
		$S M^2=\dfrac{S B^2+S C^2}2-\dfrac{B C^2}4\Rightarrow\dfrac{S B^2+S C^2}2=S M^2+\dfrac{B C^2}4=\dfrac{13a^2}2$.\\
		$R^2=C I^2=\dfrac{C A^2+S C^2}2-\dfrac{S A^2}4\Leftrightarrow R^2=\dfrac{4a^2+S C^2}2-R^2$.\\
		$R^2=B I^2=\dfrac{B A^2+S B^2}2-\dfrac{S A^2}4\Leftrightarrow R^2=\dfrac{a^2+S B^2}2-R^2$.\\
		Suy ra $4R^2=\dfrac{a^2+S B^2}2+\dfrac{4a^2+S C^2}2=\dfrac{5a^2}2+\dfrac{S B^2+S C^2}2=\dfrac{5a^2}2+\dfrac{13a^2}2=9a^2
		\Rightarrow R=\dfrac{3a}2$.
	}
\end{ex}
\begin{ex}%Câu 33.%[2H2K2-2]
	Cho hình chóp $S.ABC$ có đáy $ABC$ là tam giác vuông cân tại đỉnh $B$. Biết $AB=BC=a\sqrt 3$, $\widehat{S A B}=\widehat{S C B}=90^{\circ}$ và khoảng cách từ $A$ đến mặt phẳng $(SBC)$ bằng $a\sqrt 2$. Tính diện tích mặt cầu ngoại tiếp hình chóp $S.ABC$. 
	\choice
	{$16\pi a^2$}
	{\True $12\pi a^2$}
	{$8\pi a^2$}
	{$2\pi a^2$}
	\loigiai{
		\immini{
			Ta có $\widehat{S A B}=\widehat{S C B}=90^{\circ}$ nên $A,$ $C$ cùng nhìn $SB$ dưới một góc vuông.\\
			Suy ra $S,\,A,\,B,\,C$ cùng nội tiếp mặt cầu đường kính $SB$.\\
			Gọi $D$ là hình chiếu của $S$ trên $(ABCD)$.\\
			Ta có $\heva{&A B\perp S A\\ &A B\perp S D}\Rightarrow A B\perp A D$\\ và $\heva{&B C\perp S C\\ &B C\perp S D}\Rightarrow B C\perp C D$.\\
			Tam giác $ABC$ vuông cân tại $B$, suy ra $ABCD$ là hình vuông.\\
			Trong $\triangle SCD$, kẻ $DH\perp S C$ $(H\in S C)$.\\
			Ta có $B C\perp S D$ và $B C\perp S C$ nên $B C\perp D H$.\\
			Suy ra $D H\perp(S B C)$ hay $\mathrm{d}(D,(S B C))=D H$.\\
		}
		{\begin{tikzpicture}[scale=1, font=\footnotesize, line join=round, line cap=round, >=stealth]
			\def\bc{4} % cạnh BC
			\def\ba{2} % cạnh BA
			\def\h{4} % đường cao
			\def\gocB{30} % góc B của đáy
			\coordinate[label=below left:$C$] (C) at (0,0);
			\coordinate[label=above right:$D$] (D) at (\gocB:\ba);
			\coordinate[label=below:$B$] (B) at (\bc,0);
			\coordinate[label=right:$A$] (A) at ($(B)-(C)+(D)$);
			\coordinate[label=above:$S$] (S) at ($(D)+(90:\h)$);
			\coordinate[label=left:$H$] (H) at ($(S)!0.7!(C)$);
			\draw (C)--(B)--(A)--(S)--cycle (S)--(B);
			\draw[dashed] (A)--(D) (S)--(D)--(C) (D)--(H);
			\foreach \diem in {A,B,C,D,S,H}	\fill (\diem)circle(1.5pt);
			\newcommand{\gocv}[4][black]{\draw[#1] ($(#3)!5pt!(#2)$)--($(#3)!2!($($(#3)!5pt!(#2)$)!.5!($(#3)!5pt!(#4)$)$)$)--($(#3)!5pt!(#4)$);}
			\gocv{S}{D}{A} \gocv{S}{A}{B} \gocv{S}{C}{B}
			\gocv{S}{H}{D}
			\end{tikzpicture}
			
		}
		\noindent Ta có $AD\parallel (SBC)$ suy ra $\mathrm{d}(A,(S B C))=\mathrm{d}(D,(S B C))=DH=a\sqrt 2$.\\ 
		$\dfrac 1{D H^2}=\dfrac 1{D C^2}+\dfrac 1{S D^2}\Rightarrow S D=a\sqrt 6$, $S C=\sqrt{S D^2+C D^2}=\sqrt{6a^2+3a^2}=3a$.\\
		$S B=\sqrt{S C^2+B C^2}=\sqrt{9a^2+3a^2}=2a\sqrt 3$.\\
		Khi đó bán kính mặt cầu ngoài tiếp hình chóp $S.ABC$ là $R=\dfrac{S B}2=a\sqrt 3$.\\
		Vậy diện tích mặt cầu bằng $S=4\pi R^2=12\pi a^2$.
	}
\end{ex}
%\Closesolutionfile{ans}
%--------------------------------------------------------------------------------------------------------
%\Opensolutionfile{ans}[ans/ansCD2D2-6.1]
%%\subsubsection{Câu hỏi trắc nghiệm}
\begin{ex}%Câu 35.%[Phạm Văn Long]%[2H2B2-2]
	\textbf{(THPT Yên Lạc-Vĩnh Phúc-lần 4 - năm 2017-2018)} \\Cho hình chóp S.ABC có $AB=3$. Hình chiếu của $S$ lên mặt phẳng $(ABC)$ là điểm $H$ thuộc miền trong tam giác $ABC$ sao cho $\widehat{AHB}=120^{\circ}$. Tính bán kính $R$ của mặt cầu ngoại tiếp hình chóp $S.HAB$, biết $SH=4 \sqrt{3}$. 
	\choice
	{$R=\sqrt{5}$}
	{$R=3 \sqrt{5}$}
	{\True $R=\sqrt{15}$}
	{$R=2 \sqrt{3}$}
	\loigiai{
		\immini{
			Gọi $O$ là tâm đường tròn ngoại tiếp tam giác $AHB$ và $r$ là bán kính đường tròn ngoại tiếp tam giác $AHB$.\\
			Áp dụng định lí sin trong tam giác $AHB$ ta có \\$\dfrac{AB}{\sin \widehat{AHB}}=2r \Rightarrow r=\dfrac{AB}{2 \sin \widehat{AHB}}=\sqrt{3}$.\\
			Qua $O$ dựng đường thẳng $d$ vuông góc với mặt phẳng $(AHB)$. Gọi $M$ là trung điểm của $SH$. Trong mặt phẳng $(SHO)$ dựng đường trung trực của đoạn SH cắt $d$ tại $I$. \\
			Khi đó $I$ là tâm của mặt cầu ngoại tiếp hình chóp $S.HAB$ và có bán kính $R=HI=\sqrt{OI^2+HO^2}=\sqrt{(2 \sqrt{3})^{2}+(\sqrt{3})^2}=\sqrt{15}$.}
		{\begin{tikzpicture}[scale=0.7, font=\footnotesize, line join=round, line cap=round, >=stealth]
			\def\ac{6} 
			\def\ab{3} 
			\def\h{5} 
			\def\gocA{50} 
			\coordinate[label=left:$H$] (A) at (0,0);
			\coordinate[label=right:$B$] (C) at (\ac,0);
			\coordinate[label=below left:$A$] (B) at (-\gocA:\ab);
			\coordinate[label=above:$S$] (S) at ($(A)+(90:\h)$);
			\coordinate[label=right:$O$] (O) at (3,-0.6);
			\coordinate[label=right:$d$] (E) at ($(O)+(90:\h)$);
			\coordinate[label=left:$M$] (M) at ($(S)!0.5!(A)$);
			\coordinate[label=right:$I$] (I) at ($(O)!0.5!(E)$);
			\path
			(intersection of I--E and C--S) coordinate (K);
			\draw (A)--(B)--(C)--(S)--cycle (S)--(B) (K)--(E);
			\draw[dashed] (A)--(C) (A)--(O) (M)--(I) (O)--(I) (I)--(K);
			\foreach \diem in {A,B,C,S,O,M,I}	\fill (\diem)circle(1.5pt);
			\newcommand{\gocv}[4][black]{\draw[#1] ($(#3)!5pt!(#2)$)--($(#3)!2!($($(#3)!5pt!(#2)$)!.5!($(#3)!5pt!(#4)$)$)$)--($(#3)!5pt!(#4)$);}
			\gocv{S}{A}{C}
			\gocv{I}{M}{A}
			\gocv{I}{O}{A}
			\end{tikzpicture}}}
\end{ex}
\begin{ex}%Câu 36.%[Phạm Văn Long]%[2H2B2-2]
	\textbf{(THPT Hồng Bàng - Hải Phòng - năm 2017 - 2018)}\\
	Cho hình chóp $S.ABC$ có đáy là tam giác đều cạnh bằng $1$, $SA$ vuông góc với đáy, góc giữa mặt bên $(SBC)$ và đáy bằng $60^{\circ}$. Diện tích mặt cầu ngoại tiếp hình chóp $S.ABC$ bằng
	\choice
	{$\dfrac{4 \pi a^3}{12}$ }
	{$\dfrac{43 \pi}{36}$ }
	{$\dfrac{43 \pi}{4}$ }
	{\True $\dfrac{43 \pi}{12}$}
	\loigiai{
		\immini{
			Gọi $M$ là trung điềm của $BC$ thì $A M \perp BC$ (1). Mặt khác $SA \perp(ABC)$ nên $SA \perp BC$ (2).\\
			Từ (1) và (2) suy ra $BC \perp(SAM)$. Do đó góc giữa $(SBC)$ và $(ABC)$ là góc $SMA \Rightarrow \widehat{SMA}=60^{\circ}$.\\
			Trong tam giác vuông $SAM$ có $SA=AM \cdot \tan \widehat{SMA}=\dfrac{\sqrt{3}}{2} \cdot \sqrt{3}=\dfrac{3}{2}$.\\
			Gọi $G$ là trọng tâm của tam giác $ABC$. Qua $G$ dựng đường thẳng song song với $SA$ cắt mặt phẳng trung trực của đoạn $SA$ tại $I$ thì $I$ là tâm mặt cầu ngoại tiếp hình chóp $S.ABC$. \\
			Ta có $IG=\dfrac{1}{2} SA=\dfrac{3}{4}$.\\
			Trong tam giác vuông $AIG$ ta có\\ $IA^2=IG^2+GA^2=\left(\dfrac{3}{4}\right)^{2}+\left(\dfrac{2}{3} \cdot \dfrac{\sqrt{3}}{2}\right)^{2}=\dfrac{9}{16}+\dfrac{1}{3}=\dfrac{43}{48}$.\\
			Vậy mặt cầu ngoại tiếp hinh chóp $S.ABC$ có diện tích \\
			$S=4 \pi \cdot IA^{2}=4 \pi\cdot \dfrac{43}{48}=\dfrac{43 \pi}{12}$.}
		{\begin{tikzpicture}[scale=0.7, font=\footnotesize, line join=round, line cap=round, >=stealth]
			\def\ac{6} 
			\def\ab{3} 
			\def\h{5} 
			\def\gocA{50} 
			\coordinate[label=left:$A$] (A) at (0,0);
			\coordinate[label=right:$C$] (C) at (\ac,0);
			\coordinate[label=below left:$B$] (B) at (-\gocA:\ab);
			\coordinate[label=above:$S$] (S) at ($(A)+(90:\h)$);
			\coordinate[label= below right:$M$] (M) at ($(B)!0.5!(C)$);
			\coordinate[label=below:$G$] (G) at ($(A)!2/3!(M)$);
			\coordinate (F) at ($(S)!0.5!(A)$);
			\coordinate (E) at ($(G)+(90:\h)$);
			\coordinate[label=right:$I$] (I) at ($(G)!0.5!(E)$);
			\draw (A)--(B)--(C)--(S)--cycle (S)--(B) (S)--(M);
			\draw[dashed] (A)--(C) (G)--(I) (F)--(I) (A)--(M);
			\foreach \diem in {A,B,C,S,M,G,I}	\fill (\diem)circle(1.5pt);
			\newcommand{\gocv}[4][black]{\draw[#1] ($(#3)!5pt!(#2)$)--($(#3)!2!($($(#3)!5pt!(#2)$)!.5!($(#3)!5pt!(#4)$)$)$)--($(#3)!5pt!(#4)$);}
			\gocv{S}{A}{C}
			\gocv{I}{F}{A}
			\gocv{I}{G}{A}
			\gocv{A}{M}{B}
			\end{tikzpicture}}
	}
\end{ex}
\begin{ex}%Câu 37.%[Phạm Văn Long]%[2H2B1-1]
	Cho hình chóp $S.ABCD$ có đáy là hình vuông cạnh bằng $2$, cạnh bên $SA$ vuông góc với đáy, góc giữa cạnh bên $SC$ và đáy bằng $60 ^{\circ}$. Tính thể tích của khối trụ có một đáy là đường tròn ngoại tiếp hình vuông $ABCD$ và chiều cao bằng chiều cao của khối chóp $S.ABCD$.
	\choice
	{\True $V=4 \sqrt{6} \pi$}
	{$V=\dfrac{2 \sqrt{6} \pi}{3}$}
	{$V=2 \sqrt{6} \pi$}
	{$V=\dfrac{4 \sqrt{3} \pi}{3}$}
	\loigiai{
		\immini{
			Ta có $AC=2 \sqrt{2}$ và góc giữa $SC$ với $(ABCD)$ là góc $SCA$.\\ $\Rightarrow \widehat{SCA}=60^{\circ}$.\\ 
			Suy ra $SA=AC \cdot \tan 60^{\circ}=2 \sqrt{2} \cdot \sqrt{3}=2 \sqrt{6}$.\\
			Bán kính đường tròn ngoại tiếp hình vuông $ABCD$ bằng $\dfrac{AC}{2}=\sqrt{2}$. \\
			$\Rightarrow V=\pi(\sqrt{2})^{2} \cdot 2 \sqrt{6}=4 \sqrt{6} \pi$.}
		{\begin{tikzpicture}[scale=.7, font=\footnotesize, line join=round, line cap=round, >=stealth]
			\def\bc{4} % cạnh BC
			\def\ba{2} % cạnh BA
			\def\h{3.5} % đường cao
			\def\gocB{30} % góc B của đáy
			\coordinate[label=below left:$B$] (B) at (0,0);
			\coordinate[label=above right:$A$] (A) at (\gocB:\ba);
			\coordinate[label=below:$C$] (C) at (\bc,0);
			\coordinate[label=right:$D$] (D) at ($(C)-(B)+(A)$);
			\coordinate[label=above left:$S$] (S) at ($(A)+(90:\h)$);
			\draw (B)--(C)--(D)--(S)--cycle (S)--(C);
			\draw[dashed] (A)--(D)--(S)--(A)--(B) (A)--(C);
			\foreach \diem in {A,B,C,D,S}	\fill (\diem)circle(1.5pt);
			\newcommand{\gocv}[4][black]{\draw[#1] ($(#3)!5pt!(#2)$)--($(#3)!2!($($(#3)!5pt!(#2)$)!.5!($(#3)!5pt!(#4)$)$)$)--($(#3)!5pt!(#4)$);}
			\gocv{S}{A}{B}
			\draw pic [draw, angle radius = 6 mm] {angle = S--C--A}; 
			\end{tikzpicture}}
	}
\end{ex}
\begin{ex}%Câu 38.%[Phạm Văn Long]%[2H2K2-2]
	Cho hình chóp tứ giác $S.ABCD$ có đáy $ABCD$ là hình vuông cạnh $a$. Tam giác $SAB$ vuông cân tại $S$ và tam giác $SCD$ đều. Tính bán kính mặt cầu ngoại tiếp hình chóp $S.ABCD$.
	\choice
	{$R=\dfrac{a}{2}$}
	{\True $R=a \sqrt{\dfrac{7}{12}}$}
	{$R=\dfrac{a}{\sqrt{3}}$}
	{$R=a \sqrt{\dfrac{3}{4}}$}
	\loigiai{
		\immini{
			Gọi $I$, $K$ lần lượt là trung điểm của $AB$ và $CD$, $O$ là tâm của hình vuông $ABCD$, $H$ là hình chiếu của $S$ trên $IK$. Ta có:\\
			$\left.\begin{array}{l}A B \perp S I \\ A B \perp I K\end{array}\right\} \Rightarrow A B \perp(S I K)$.\\
			$\left.\begin{array}{l}S H \perp A B \\ S H \perp I K\end{array}\right\} \Rightarrow S H \perp(A B C D)$.\\
			Qua $O$ dựng đường thẳng song song với $SH$ cắt $SK$ tại $J$.\\
			Mặt khác ta có:\\
			$SI=\dfrac{1}{2} AB=\dfrac{a}{2}, SK=\dfrac{a \sqrt{3}}{2} \Rightarrow SK^2+SI^2=a^2=KI^2.$\\ $\Rightarrow \triangle SIK$ vuông ở $S$ $\Rightarrow SK \perp(SAB)$.\\
			Qua $I$ dựng đường thẳng song song với $SK$ cắt $OJ$ tại $M$. Khi đó, điểm $M$ là tâm của mặt cầu ngoại tiếp hình chóp $S.ABCD$. Theo cách dựng ở trên thì tứ giác $IJKM$ là hình binh hành $\Rightarrow MB=JB$.\\
			Lại có $\tan \widehat{OKJ}=\dfrac{SI}{SK}=\dfrac{1}{\sqrt{3}}$.\\ $\Rightarrow JO=OK \cdot \tan \widehat{OKJ}=\dfrac{a}{2 \sqrt{3}}$.\\
			$\Rightarrow JB^2=JO^2+OB^2=\dfrac{7 a^2}{12} \Rightarrow JB=a \sqrt{\dfrac{7}{12}}$.\\
			Vậy bán kính mặt cầu ngoại tiếp hình chóp bằng $a \sqrt{\dfrac{7}{12}}$.}
		{\begin{tikzpicture}[scale=.8, font=\footnotesize, line join=round, line cap=round, >=stealth]
			\def\bc{5} % cạnh BC
			\def\ba{3} % cạnh BA
			\def\h{5} % đường cao
			\def\gocB{30} % góc B của đáy
			\coordinate[label=below left:$D$] (B) at (0,0);
			\coordinate[label=above left:$A$] (A) at (\gocB:\ba);
			\coordinate[label=below:$C$] (C) at (\bc,0);
			\coordinate[label=right:$B$] (D) at ($(C)-(B)+(A)$);
			\path
			(intersection of A--C and B--D) coordinate[label=left:$O$] (O);
			\coordinate[label=below:$K$] (K) at ($(B)!0.5!(C)$);
			\coordinate[label=above right:$I$] (I) at ($(A)!0.5!(D)$);
			\coordinate[label=left:$H$] (H) at ($(O)!0.62!(I)$);
			\coordinate[label=above left:$S$] (S) at ($(H)+(90:\h)$);
			\coordinate[label=right:$J$] (J) at ($(K)!0.62!(S)$);
			\coordinate[label= left:$M$] (M) at ($(J)!2!(O)$);
			\path
			(intersection of O--M and B--C) coordinate (T);
			\path
			(intersection of I--M and B--C) coordinate (R);
			\draw (B)--(C)--(D)--(S)--cycle (S)--(C) (S)--(K) (T)--(M) (M)--(R);
			\draw[dashed] (A)--(D)--(S)--(A)--(B) (A)--(C) (B)--(D) (S)--(H) (I)--(K) (J)--(O) (S)--(I) (T)--(O) (R)--(I);
			\foreach \diem in {A,B,C,D,S,O,I,K,H,J,M}	\fill (\diem)circle(1.5pt);
			\newcommand{\gocv}[4][black]{\draw[#1] ($(#3)!5pt!(#2)$)--($(#3)!2!($($(#3)!5pt!(#2)$)!.5!($(#3)!5pt!(#4)$)$)$)--($(#3)!5pt!(#4)$);}
			\gocv{S}{H}{I}
			\end{tikzpicture}}
	}
\end{ex}
\begin{ex}%Câu 39.%[Phạm Văn Long]%[2H2K2-2]
	Cho tứ diện đều $ABCD$ cạnh bằng $a$. Gọi $H$ là hình chiếu vuông góc của $A$ trên mặt phẳng $(BCD)$ và $I$ là trung điểm của $AH$. Tính bán kính $R$ của mặt cầu ngoại tiếp tứ diện $IBCD$.
	\choice
	{\True $R=\dfrac{a \sqrt{6}}{4}$}
	{$R=\dfrac{a \sqrt{3}}{4}$}
	{$R=\dfrac{a \sqrt{6}}{2}$}
	{$R=\dfrac{a \sqrt{3}}{2}$}
	\loigiai{
		\immini{
			Tứ diện đều $ABCD$ cạnh bằng $a$ nên $AH=\sqrt{AB^2-BH^2}=\sqrt{a^2-\left(\dfrac{a \sqrt3}{3}\right)^{2}}=\dfrac{a \sqrt{6}}{3}$.\\
			$I$ là trung điểm của $AH$ nên $BI=\sqrt{IH^2+BH^2}=\dfrac{a \sqrt2}{2}$.\\
			Do $I \in AH \Rightarrow IB=IC=ID=\dfrac{a \sqrt{2}}{2}$ nên $IBCD$ là hình chóp tam giác đều. Do đó bán kính $R$ của mặt cầu ngoại tiểp tứ diện $IBCD$ là $R=\dfrac{IB^{2}}{2\cdot IH}=\dfrac{a \sqrt{6}}{4}$.}
		{\begin{tikzpicture}[scale=0.7, font=\footnotesize, line join=round, line cap=round, >=stealth]
			\def\ac{7} 
			\def\ab{3} 
			\def\h{5} 
			\def\gocA{50} 
			\coordinate[label=left:$B$] (A) at (0,0);
			\coordinate[label=right:$D$] (C) at (\ac,0);
			\coordinate[label=below left:$C$] (B) at (-\gocA:\ab);
			\coordinate (M) at ($(B)!0.5!(C)$);
			\coordinate[label=below:$H$] (H) at ($(A)!2/3!(M)$);
			\coordinate[label=above:$A$] (S) at ($(H)+(90:\h)$);
			\coordinate[label=above right:$I$] (I) at ($(H)!0.5!(S)$);
			\draw (A)--(B)--(C)--(S)--cycle (S)--(B);
			\draw[dashed] (A)--(C) (S)--(H) (A)--(M) (I)--(A) (I)--(B) (I)--(C);
			\foreach \diem in {A,B,C,S,H,I}	\fill (\diem)circle(1.5pt);
			\newcommand{\gocv}[4][black]{\draw[#1] ($(#3)!5pt!(#2)$)--($(#3)!2!($($(#3)!5pt!(#2)$)!.5!($(#3)!5pt!(#4)$)$)$)--($(#3)!5pt!(#4)$);}
			\gocv{S}{H}{A}
			\end{tikzpicture}}
	}
\end{ex}
\begin{ex}%Câu 40.%[Phạm Văn Long]%[2H2G2-2]
	\textbf{(THPT Chuyên Lam-Thanh Hóa-lần 1-năm 2017-2018)}\\
	Tính thể tích $V$ của khối cầu tiếp xúc với tất cả các cạnh của tứ diện đều $ABCD$ cạnh bằng $1$.
	\choice
	{\True$V=\dfrac{\sqrt{2} \pi}{24}$}
	{$V=\dfrac{\sqrt{2} \pi}{12}$}
	{$V=\dfrac{\sqrt{2} \pi}{8}$ }
	{$V=\dfrac{\sqrt{2} \pi}{3}$}
	\loigiai{
		\immini{
			Gọi $I$ là tâm mặt cầu ngoại tiếp tứ diện đều $ABCD$. Ta có các tam giác cân $IAD$, $IAB$, $IAC$, $IBD$, $IBC$, $ICD$ bằng nhau nên khoảng cách từ $I$ đến các cạnh của tứ diện đều cũng bằng nhau. Ta được tâm mặt cầu tiếp xúc với tất cả các cạnh của tứ diện trùng với I.\\
			Ta có $HA=\dfrac{\sqrt3}{3}$, $HD=\sqrt{1-\dfrac{1}{3}}=\dfrac{\sqrt{6}}{3}$.\\
			Từ hai tam giác $DMI$ và $DHA$ đồng dạng, ta có:\\
			$\dfrac{MI}{AH}=\dfrac{DM}{DH} \Rightarrow MI=\dfrac{AH \cdot DM}{DH}=\dfrac{\sqrt{3}}{3} \cdot \dfrac{1}{2} \cdot \dfrac{3}{\sqrt{6}}=\dfrac{\sqrt{2}}{4}$.\\
			Thể tích mặt cầu tiếp xúc với tất cả các cạnh của tứ diện là \\$V=\dfrac{4}{3} \pi R^{3}=\dfrac{\sqrt{2} \pi}{24}$.}
		{\begin{tikzpicture}[scale=0.7, font=\footnotesize, line join=round, line cap=round, >=stealth]
			\def\ac{7} 
			\def\ab{3} 
			\def\h{5} 
			\def\gocA{50} 
			\coordinate[label=left:$A$] (A) at (0,0);
			\coordinate[label=right:$C$] (C) at (\ac,0);
			\coordinate[label=below left:$B$] (B) at (-\gocA:\ab);
			\coordinate[label=below right:$N$] (N) at ($(B)!0.5!(C)$);
			\coordinate[label=below:$H$] (H) at ($(A)!2/3!(N)$);
			\coordinate[label=above:$D$] (S) at ($(H)+(90:\h)$);
			\coordinate[label=left:$M$] (M) at ($(A)!0.5!(S)$);
			\coordinate[label=above right:$I$] (I) at ($(S)!0.7!(H)$);
			\draw (A)--(B)--(C)--(S)--cycle (S)--(B) (S)--(N);
			\draw[dashed] (A)--(C) (S)--(H) (A)--(N) (I)--(M);
			\foreach \diem in {A,B,C,S,H,M,I,N}	\fill (\diem)circle(1.5pt);
			\newcommand{\gocv}[4][black]{\draw[#1] ($(#3)!5pt!(#2)$)--($(#3)!2!($($(#3)!5pt!(#2)$)!.5!($(#3)!5pt!(#4)$)$)$)--($(#3)!5pt!(#4)$);}
			\gocv{S}{H}{A}
			\gocv{I}{M}{A}
			\end{tikzpicture}}
	}
\end{ex}
\begin{ex}%Câu 41.%[Phạm Văn Long]%[2H2G2-2]
	\textbf{(SGD Ninh Bình năm 2017-2018)}\\
	Cho hình chóp $S.ABC$ có đáy $ABC$ là tam giác đều cạnh $a$, cạnh bên $SA$ vuông góc với mặt phẳng đáy. Gọi $B_1$, $C_1$ lần lượt là hình chiếu của $A$ trên $SB, SC$. Tính theo $a$ bán kính $R$ của mặt cầu đi qua năm điểm $A$, $B$, $C$, $B_1$, $C_1$.
	\choice
	{$R=\dfrac{a \sqrt{3}}{6}$}
	{$R=\dfrac{a \sqrt{3}}{2}$}
	{$R=\dfrac{a \sqrt{3}}{4}$ }
	{\True $R=\dfrac{a \sqrt{3}}{3}$}
	\loigiai{
		\immini{
			Đặt $SA=x$, gọi $I$ là tâm đường tròn ngoại tiếp tam giác $ABC$, $H$ là hình chiếu của $B_1$ trên cạnh $AB$, $M$ là trung điểm của $AB$.\\
			Ta có $SA^2=SB_1 \cdot SB \Rightarrow \dfrac{SB_1}{SB}=\dfrac{SA^2}{SB^2}=\dfrac{x^2}{a^2+x^2}$.\\ Tương tự ta cũng có $\dfrac{SC_1}{SC}=\dfrac{SA^2}{SC^2}=\dfrac{x^2}{a^2+x^2}$.\\
			Suy ra $B_1C_1 \parallel BC$ mà $B_1H \parallel SA$ nên $\dfrac{BB_1}{SB}=\dfrac{HB_{1}}{SA}=\dfrac{BH}{AB}=\dfrac{a^2}{x^2+a^2}$\\
			$\Rightarrow HB_{1}=\dfrac{x a^2}{x^2+a^2}, HB=\dfrac{a x^2}{x^2+a^2}$.\\
			Ta chỉ cần chứng minh $IA=IB_1=\dfrac{a \sqrt{3}}{3}$. \\
			Giả sử $x>a$ ($x \leq a$ ta làm tương tự).\\
			Khi đó $HB=\dfrac{a x^2}{x^2+a^2}>BM=\dfrac{a}{2}$. \\
			Suy ra $HM=\dfrac{a x^2}{x^2+a^2}-\dfrac{a}{2}=\dfrac{a\left(x^2-a^2\right)}{2\left(x^2+a^2\right)}$.\\
			$IB_1^2=HI^2+B_1H^2=HM^2+IM^2+B_1H^2=\dfrac{a^2}{3}.$\\
			$\Rightarrow IB_1=IA=\dfrac{a \sqrt{3}}{3}$.\\
			Vậy $IA=IB=IC=IB_{1}=IC_{1}=\dfrac{a \sqrt{3}}{3}$ là bán kính mặt cầu đi qua năm điểm $A$, $B$, $C$, $B_1$, $C_{1}$.}
		{\begin{tikzpicture}[scale=0.7, font=\footnotesize, line join=round, line cap=round, >=stealth]
			\def\ac{8} 
			\def\ab{4} 
			\def\h{5} 
			\def\gocA{40} 
			\coordinate[label=left:$A$] (A) at (0,0);
			\coordinate[label=right:$C$] (C) at (\ac,0);
			\coordinate[label=below left:$B$] (B) at (-\gocA:\ab);
			\coordinate[label=above:$S$] (S) at ($(A)+(90:\h)$);
			\coordinate (N) at ($(B)!0.5!(C)$);
			\coordinate[label=left:$M$] (M) at ($(B)!0.5!(A)$);
			\coordinate[label=below:$I$] (I) at ($(A)!2/3!(N)$);
			\coordinate[label=left:$B_1$] (B_1) at ($(S)!1/2.8!(B)$);
			\coordinate[label=right:$C_1$] (C_1) at ($(S)!1/2.8!(C)$);
			\coordinate[label=left:$H$] (H) at ($(A)!2/3!(M)$);
			\draw (A)--(B)--(C)--(S)--cycle (S)--(B) (B_1)--(C_1) (A)--(B_1) (B_1)--(H);
			\draw[dashed] (A)--(C) (A)--(N) (C)--(M) (A)--(C_1) (I)--(B_1) (I)--(C_1) (H)--(I);
			\foreach \diem in {A,B,C,S,N,I,M,B_1,C_1,H}	\fill (\diem)circle(1.5pt);
			\newcommand{\gocv}[4][black]{\draw[#1] ($(#3)!5pt!(#2)$)--($(#3)!2!($($(#3)!5pt!(#2)$)!.5!($(#3)!5pt!(#4)$)$)$)--($(#3)!5pt!(#4)$);}
			\gocv{S}{A}{C}
			\gocv{S}{A}{B}
			\gocv{C}{M}{B}
			\gocv{A}{N}{B}
			\gocv{A}{B_1}{B}
			\gocv{A}{C_1}{C}
			\gocv{B_1}{H}{B}
			\end{tikzpicture}}
	}
\end{ex}
\begin{ex}%Câu 42.%[Phạm Văn Long]%[2H2G2-2]
	\textbf{(THPT Chuyên Biên Hòa-Hà Nam-lần 1 năm 2017-2018)}\\
	Cho hình chóp $S.ABC$ có đáy là tam giác $ABC$ đều, đường cao $SH$ với $H$ nằm trong $\triangle ABC$ và $2SH=BC$, mặt phẳng $(SBC)$ tạo với mặt phẳng $(ABC)$ một góc $60^\circ$. Biết có một điểm $O$ nằm trên đường cao $SH$ sao cho $\mathrm{d}(O; AB)=\mathrm{d}(O;AC)=\mathrm{d}(O;(SBC))=1$. Tính thể tích khối cầu ngoại tiếp hình chóp đã cho.
	\choice
	{$\dfrac{256 \pi}{81}$}
	{$\dfrac{125 \pi}{162}$}
	{$\dfrac{500 \pi}{81}$ }
	{\True$\dfrac{343 \pi}{48}$}
	\loigiai{
		\immini{
			Giả sử $E$, $F$ là chân đường vuông góc hạ từ $O$ xuống $AB$, $AC$. Khi đó ta có $HE \perp AB$, $HF \perp AC$. \\
			Do $OE=OF=1$ nên $HE=HF$. Do đó $AH$ là phân giác của góc $\widehat{BAC}$. Khi đó $AH \cap BC=D$ là trung điểm của $BC$.\\ 
			Do $BC \perp AD \Rightarrow BC \perp(SAD)$. \\
			Trong $(SAD)$ kẻ $OK \perp SD$ thì $OK \perp(SBC)$. Do đó $OK=1$ và $\widehat{SDA}=60^\circ$.\\
			Đặt $AB=BC=CA=2a$ $(a>0)$ thì $SH=a$, $HD=a \cdot \cot 60^\circ=\dfrac{a}{\sqrt{3}}$.\\
			Ta có $AD$ là đường cao trong tam giác đều $ABC$.
			Do đó $AD=a \sqrt{3}=3 HD$ nên $H$ là tâm tam giác đều $ABC \Rightarrow S.ABC$ là hình chóp tam giác đều và $E$, $F$ là trung điềm $AB$, $AC$.\\
			Mặt khác trong tam giác $SOK$ có $SO=\dfrac{OK}{\sin 30^\circ}=2$. Do $\triangle DEF$ đều có $OH \perp(DFE)$ nên $OE=OF=OD=1 \Rightarrow K \equiv D$.
		}
		{\begin{tikzpicture}[scale=0.7, font=\footnotesize, line join=round, line cap=round, >=stealth]
			\def\ac{8} 
			\def\ab{4} 
			\def\h{6} 
			\def\gocA{60} 
			\coordinate[label=left:$A$] (A) at (0,0);
			\coordinate[label=right:$C$] (C) at (\ac,0);
			\coordinate[label=below left:$B$] (B) at (-\gocA:\ab);
			\coordinate[label=below right:$D$] (D) at ($(B)!0.5!(C)$);
			\coordinate[label=above left:$H$] (H) at ($(A)!2/3!(D)$);
			\coordinate[label=above:$S$] (S) at ($(H)+(90:\h)$);
			\coordinate[label=left:$E$] (E) at ($(A)!0.5!(B)$);
			\coordinate[label=above:$F$] (F) at ($(A)!0.5!(C)$);
			\coordinate[label=below:$O$] (O) at ($(S)!1.5!(H)$);
			\coordinate[label=right:$K$] (K) at ($(S)!0.8!(D)$);
			\draw (A)--(B)--(C)--(S)--cycle (S)--(B) (S)--(D) (O)--(B) (O)--(C) (O)--(D);
			\draw[dashed] (A)--(C) (S)--(H) (A)--(D) (C)--(E) (B)--(F) (H)--(O) (O)--(F) (O)--(E) (O)--(K);
			\foreach \diem in {A,B,C,S,H,D,E,F,O,K}	\fill (\diem)circle(1.5pt);
			\newcommand{\gocv}[4][black]{\draw[#1] ($(#3)!5pt!(#2)$)--($(#3)!2!($($(#3)!5pt!(#2)$)!.5!($(#3)!5pt!(#4)$)$)$)--($(#3)!5pt!(#4)$);}
			\gocv{S}{H}{A}
			\gocv{O}{K}{D}
			\gocv{O}{E}{B}
			\gocv{O}{F}{C}
			\gocv{H}{E}{A}
			\end{tikzpicture}}
		Khi đó $\triangle DSO$ vuông tại $D$ và có $DH \perp SO$.\\
		Từ đó $DH^2=HS \cdot HO \Rightarrow \dfrac{a^2}{3}=a\cdot (2-a) \Rightarrow a=\dfrac{3}{2} \Rightarrow AB=3$, $SH=\dfrac{3}{2}$.\\
		Gọi $R$ là bán kính mặt cầu ngoại tiếp hình chóp $S.ABC$ thì $R=\dfrac{SA^2}{2SH}=\dfrac{7}{4}$. \\
		Vậy $V=\dfrac{4}{3} \pi \cdot\left(\dfrac{7}{4}\right)^{3}=\dfrac{343}{48} \pi$.}
\end{ex}
\begin{ex}%Câu 43.%[Phạm Văn Long]%[2H2K2-2]
	Trong không gian cho hai đường thẳng $d$ và $\Delta$ chéo nhau và vuông góc nhau, nhận $AB=a$ làm đoạn vuông góc chung, $A \in d$; $B \in \Delta$. Trên $d$ lấy điểm $M$, trên $\Delta$ lấy điểm $N$ sao cho $AM=2a$, $BN=4a$. Goi $I$ là tâm mặt cầu ngoại tiếp tứ diện $ABMN$. Khoảng cách giữa hai đường thẳng $AM$ và $BI$ là
	\choice
	{\True$\dfrac{4 a}{\sqrt{17}}$}
	{$a$}
	{$\dfrac{4 a}{5}$ }
	{$\dfrac{2 a \sqrt{2}}{3}$}
	\loigiai{
		\immini{
			Ta có $MA \perp(ABN)$ suy ra $MA \perp AN$ và $NB \perp(ABM)$ suy ra $NB \perp BM$.\\
			Do đó, tâm mặt cầu ngoại tiếp tứ diện $ABMN$ là trung điểm $I$ của $MN$.\\
			Gọi $F$ là trung điểm của $AN$ suy ra $IF \parallel AM$.\\
			Do đó $\mathrm{d}(AM, BI)=\mathrm{d}(AM,(BIF))=\mathrm{d}(A,(BIF))$
			và $IF \perp(A B N)$.\\
			Gọi $H$ là hình chiếu của $A$ lên $BF$, $P$ đối xứng với $B$ qua $F$ suy ra $ABNP$ là hình chữ nhật. \\
			Ta có $\left\{\begin{array}{l}AH \perp BF \\ 
			AH \perp IF\end{array} \Rightarrow AH \perp(BIF) \Rightarrow \mathrm{d}(AM, BI)=AH\right.$.\\
			Xét tam giác $ABP$ vuông tại $A$ có $AH$ là đường cao nên\\ $\mathrm{d}(AM, BI)=AH=\sqrt{\dfrac{AB^2 \cdot AP^2}{AB^2+AP^2}}=\sqrt{\dfrac{a^2 \cdot 16 a^2}{a^2+4 a^2}}=\dfrac{4a}{\sqrt{17}}$.}
		{\begin{tikzpicture}[scale=0.7, font=\footnotesize, line join=round, line cap=round, >=stealth]
			\def\ac{9} 
			\def\ab{4} 
			\def\h{5} 
			\def\gocA{40} 
			\coordinate[label=left:$A$] (A) at (0,0);
			\coordinate[label=right:$N$] (C) at (\ac,0);
			\coordinate[label=below left:$B$] (B) at (-\gocA:\ab);
			\coordinate[label=left:$M$] (S) at ($(A)+(90:\h)$);
			\coordinate[label=left:$d$] (S_1) at ($(A)+(90:\h+1)$);
			\coordinate[label= below right:$F$] (F) at ($(A)!0.5!(C)$);
			\coordinate[label= right:$H$] (H) at ($(F)!0.6!(B)$);
			\coordinate[label= right:$P$] (P) at ($(B)!2!(F)$);
			\coordinate[label=below:$\Delta $] (K) at ($(B)!0.5!(C)$);
			%
			\path
			(intersection of A--P and C--S) coordinate (A_1);
			\path
			(intersection of B--P and C--S) coordinate (B_1);
			\coordinate[label= above right:$I$] (I) at ($(S)!0.5!(C)$);
			\draw (A)--(B)--(C)--(S)--cycle (S)--(B) (C)--(P) (A_1)--(P) (B_1)--(P) (B)--(I) (S)--(S_1);
			\draw[dashed] (A)--(C) (B)--(B_1) (A)--(H) (A)--(A_1) (I)--(F);
			\foreach \diem in {A,B,C,S,F,P,I}	\fill (\diem)circle(1.5pt);
			\newcommand{\gocv}[4][black]{\draw[#1] ($(#3)!5pt!(#2)$)--($(#3)!2!($($(#3)!5pt!(#2)$)!.5!($(#3)!5pt!(#4)$)$)$)--($(#3)!5pt!(#4)$);}
			\gocv{S}{A}{C}
			\gocv{A}{B}{C}
			\gocv{A}{H}{F}
			\gocv{I}{F}{A}
			\end{tikzpicture}}
	}
\end{ex}
\begin{dang}{Chóp có các cạnh bên bằng nhau}.
\end{dang}
\subsubsection{Các ví dụ}
\begin{vd}%VD1.%[Phạm Văn Long]%[2H2B2-2]
	Cho hình chóp tam giác đều $S.ABC$. Xác định tâm và bán kính mặt cầu ngoại tiếp khối chóp $S.ABC$.
	\loigiai{
		\immini{
			\begin{itemize}
				\item Vẽ $SH \perp(ABC)$ thì $H$ là tâm đường tròn ngoại tiếp $ \triangle ABC$.
				\item Trên mặt phẳng $(SHA)$, vẽ đường trung trực của $SA$, đường này cắt $SH$ tại $I$ thì $I$ là tâm mặt cầu ngoại tiếp $S.ABC$ và bán kính $R=IS$. 
				\item Ta có $\triangle SHA \sim \triangle SMI$ (g-g) $\Rightarrow \dfrac{SH}{SM}=\dfrac{SA}{SI}$.\\ $\Rightarrow R=\dfrac{SA\cdot SM}{SH}=\dfrac{SA^2}{2\cdot SH}$.
		\end{itemize}}
		{\begin{tikzpicture}[scale=0.7, font=\footnotesize, line join=round, line cap=round, >=stealth]
			\def\ac{7} 
			\def\ab{3} 
			\def\h{5} 
			\def\gocA{50} 
			\coordinate[label=left:$A$] (A) at (0,0);
			\coordinate[label=right:$C$] (C) at (\ac,0);
			\coordinate[label=below left:$B$] (B) at (-\gocA:\ab);
			\coordinate[label=below right:$N$] (N) at ($(B)!0.5!(C)$);
			\coordinate[label=below:$H$] (H) at ($(A)!2/3!(N)$);
			\coordinate[label=above:$S$] (S) at ($(H)+(90:\h)$);
			\coordinate[label=left:$M$] (M) at ($(A)!0.5!(S)$);
			\coordinate[label=above right:$I$] (I) at ($(S)!0.7!(H)$);
			\draw (A)--(B)--(C)--(S)--cycle (S)--(B) (S)--(N);
			\draw[dashed] (A)--(C) (S)--(H) (A)--(N) (I)--(M);
			\foreach \diem in {A,B,C,S,H,M,I,N}	\fill (\diem)circle(1.5pt);
			\newcommand{\gocv}[4][black]{\draw[#1] ($(#3)!5pt!(#2)$)--($(#3)!2!($($(#3)!5pt!(#2)$)!.5!($(#3)!5pt!(#4)$)$)$)--($(#3)!5pt!(#4)$);}
			\gocv{S}{H}{A}
			\gocv{I}{M}{A}
			% \gocv{I}{G}{A}
			\end{tikzpicture}}}	
\end{vd}
\begin{vd}%VD2%[Phạm Văn Long]%[2H2B2-2]
	Cho hình chóp tam giác đều $S.ABC$ có cạnh đáy bằng $3a$, góc giữa cạnh bên và mặt đáy bằng $45^\circ$. Tính thể tích khối cầu ngoại tiếp khối chóp $S.ABC$?
	\loigiai{
		\immini{
			Ta có: $AH=\dfrac{2}{3} \cdot \dfrac{3a \sqrt3}{2}=a \sqrt{3}$.\\
			$\triangle SAH$ vuông cân $\Rightarrow SH=AH=a \sqrt{3}$.\\	
			Bán kính mặt cầu ngoại tiếp khối chóp $S.ABC$ là\\ $R=\dfrac{SA^2}{2\cdot SH}=\dfrac{6a^2}{2a \sqrt3}=a \sqrt{3}$.\\
			Vậy $V=\dfrac{4}{3} \pi R^3=\dfrac{4}{3} \pi(a \sqrt3)^{3}=4 \pi a^{3} \sqrt{3}$.}
		{\begin{tikzpicture}[scale=0.7, font=\footnotesize, line join=round, line cap=round, >=stealth]
			\def\ac{7} 
			\def\ab{3} 
			\def\h{5} 
			\def\gocA{50} 
			\coordinate[label=left:$A$] (A) at (0,0);
			\coordinate[label=right:$C$] (C) at (\ac,0);
			\coordinate[label=below left:$B$] (B) at (-\gocA:\ab);
			\coordinate[label=below right:$N$] (N) at ($(B)!0.5!(C)$);
			\coordinate[label=below:$H$] (H) at ($(A)!2/3!(N)$);
			\coordinate[label=above:$S$] (S) at ($(H)+(90:\h)$);
			\coordinate[label=left:$M$] (M) at ($(A)!0.5!(S)$);
			\coordinate[label=above right:$I$] (I) at ($(S)!0.7!(H)$);
			\draw (A)--(B)--(C)--(S)--cycle (S)--(B) (S)--(N);
			\draw[dashed] (A)--(C) (S)--(H) (A)--(N) (I)--(M);
			\foreach \diem in {A,B,C,S,H,M,I,N}	\fill (\diem)circle(1.5pt);
			\newcommand{\gocv}[4][black]{\draw[#1] ($(#3)!5pt!(#2)$)--($(#3)!2!($($(#3)!5pt!(#2)$)!.5!($(#3)!5pt!(#4)$)$)$)--($(#3)!5pt!(#4)$);}
			\gocv{S}{H}{A}
			\gocv{I}{M}{A}
			% \gocv{I}{G}{A}
			\end{tikzpicture}}
	}
\end{vd}
\begin{vd}%VD3%[Phạm Văn Long]%[2H2B2-2]
	Cho hình chóp đều $S.ABCD$ có cạnh đáy bằng $a \sqrt2$, góc giữa cạnh bên và mặt đáy bằng $45^\circ$. Tính diện tích mặt cầu ngoại tiếp hình chóp?
	\loigiai{
		\immini{
			Gọi $O$ là tâm hình vuông $ABCD$. \\
			Ta có góc giữa cạnh bên và mặt đáy là góc $\widehat{SAO}=45^{\circ}$.\\ $\Rightarrow SO=OA=\dfrac{AC}{2}=\dfrac{AB \sqrt{2}}{2}=a$.\\
			Suy ra $OA=OB=OC=OD=OS$ nên $O$ là tâm mặt cầu ngoại tiếp hình chóp $S.ABCD$ có bán kính $R=OA=a$. \\
			Vậy diện tích mặt cầu ngoại tiếp hình chóp $S.ABCD$ là\\ $S=4 \pi R^{2}=4 \pi a^{2}$.}
		{\begin{tikzpicture}[scale=.9, font=\footnotesize, line join=round, line cap=round, >=stealth]
			\def\bc{4} % cạnh BC
			\def\ba{2} % cạnh BA
			\def\h{3.5} % đường cao
			\def\gocB{30} % góc B của đáy
			\coordinate[label=below left:$B$] (B) at (0,0);
			\coordinate[label=above left:$A$] (A) at (\gocB:\ba);
			\coordinate[label=below:$C$] (C) at (\bc,0);
			\coordinate[label=right:$D$] (D) at ($(C)-(B)+(A)$);
			\path
			(intersection of A--C and B--D) coordinate[label=below:$O$] (O);
			\coordinate[label=above left:$S$] (S) at ($(O)+(90:\h)$);
			\draw (B)--(C)--(D)--(S)--cycle (S)--(C);
			\draw[dashed] (A)--(D)--(S)--(A)--(B) (A)--(C) (B)--(D) (S)--(O);
			\foreach \diem in {A,B,C,D,S,O}	\fill (\diem)circle(1.5pt);
			\end{tikzpicture}}
	}
\end{vd}
\Closesolutionfile{ans}
% \DAPAN
\inputansbox{10}{ans/ans2H2-2}
%--------------------------------------------------------------------------------------------------------
\Opensolutionfile{ans}[ans/ans2H2-2]
\begin{dang}{Chóp có các cạnh bên bằng nhau}
\end{dang}
\subsubsection{Các ví dụ}		
\begin{vd}%Ví dụ 4.%[2H2B2-2]%[Đặng Tấn Phát]
	Cho hình chóp tam giác đều có cạnh đáy bằng $a$, góc giữa cạnh bên và mặt đáy bằng $60^\circ$. Tính bán kính mặt cầu ngoại tiếp hình chóp đã cho. 
	\loigiai{
		\immini{
			Gọi hình chóp đều đó là $ABCD$.\\
			Gọi $H$ là trung điểm $BC$, $G$ là tâm đường tròn ngoại tiếp $\triangle ABC$. Suy ra $SG\perp (ABC)$ và $SG$ là trục của đường tròn ngoại tiếp tam giác $ABC$.\\
			Gọi $I$ là trung điểm $SA$, đường trung trực của $SA$ cắt $SG$ tại $O$.\\
			$\Rightarrow$ $O$ là tâm mặt cầu ngoại tiếp hình chóp và bán kính mặt cầu $R=SO$.\\
			Ta có $\widehat{(SA,(ABC))}=\widehat{SAG}=60^\circ;\quad AG=\dfrac{2}{3}AH=\dfrac{a\sqrt{3}}{3}$.\\
			Khi đó \[SG=\tan 60^\circ\cdot AG=\dfrac{a\sqrt{3}}{3}\cdot\sqrt{3}=a;\quad SA=\dfrac{SG}{\sin 60^\circ}=\dfrac{2a}{\sqrt{3}}.\]
			Ta có $\triangle SIO\backsim \triangle SGA$ nên suy ra
			\[\dfrac{SI}{SO}=\dfrac{SG}{SA}\Rightarrow SO=\dfrac{SI\cdot SA}{SG}\Rightarrow SO=\dfrac{\dfrac{a}{\sqrt{3}}\cdot \dfrac{2a}{\sqrt{3}}}{a}=\dfrac{2a}{3}\]
			Vậy bán kính mặt cầu ngoại tiếp hình chóp đã cho là $R=\dfrac{2a}{3}$.}
		{
			\begin{tikzpicture}[>=stealth,line join=round,line cap=round,font=\footnotesize,scale=1.2]
			\def\h{3.5}
			\def\ac{3.25}
			\def\xb{2.5}
			\def\yb{-1}
			\coordinate[label=left:{$A$}] (A) at (0,0);
			\coordinate[label=below:{$B$}] (B) at (\xb,\yb);
			\coordinate[label=right:{$C$}] (C) at (\ac,0);
			\coordinate[label=below:{$G$}] (G) at ($($(B)!0.5!(C)$)!1/3!(A)$);
			\coordinate[label=above:{$S$}] (S) at ($(G)+(0,\h)$);
			\coordinate[label=left:{$I$}] (I) at ($(S)!0.5!(A)$);
			\coordinate[rotate around={-90:(I)}] (x) at (0,0);
			\coordinate[label=below right:{$H$}] (H) at ($(B)!0.5!(C)$);
			\coordinate[label=below left:{$O$}] (O) at (intersection of I--x and S--G);
			\draw (S)--(A)--(B)--(C)--cycle
			(S)--(B);
			\draw[dashed] (A)--(C) (A)--(H) (S)--(G) (I)--(O);
			\pic[draw,pic text=$60^\circ$,angle radius=4mm,angle eccentricity=1.8] {angle = G--A--S};
			\pic[draw,angle radius=2mm,angle eccentricity=1.5] {right angle = S--G--A};
			\pic[draw,angle radius=2mm,angle eccentricity=1.5] {right angle = O--I--S};
			
			\foreach \point in {A,B,C,S,G,O,I,H}
			\fill (\point) circle (1pt);
			
			\end{tikzpicture}
		}
	}
\end{vd}
\begin{vd}%Ví dụ 5.%[2H2K2-3]%[Đặng Tấn Phát]
	Tính thể tích khối cầu nội tiếp tứ diện đều có cạnh bằng $a$.
	\loigiai{
		\immini{
			Gọi tứ diện đều đó là $ABCD$.\\
			Gọi $H$ là trọng tâm tam giác $BCD$ và $G$ là tâm mặt cầu nội tiếp tứ diện $ABCD$.\\
			Khi đó bán kính mặt cầu nội tiếp tứ diện $ABCD$ là
			\[r=\mathrm{d}\left(G,(ABC)\right)=\mathrm{d}\left(G,(BCD)\right)=\mathrm{d}\left(G,(ACD)\right)=\mathrm{d}\left(G,(ABD)\right)\]
			Ta có \[V_{G.BCD}=\dfrac{1}{3}\cdot S_{BCD}\cdot \mathrm{d}\left(G,(BCD)\right)\Rightarrow \mathrm{d}\left(G,(BCD)\right)=\dfrac{3V_{G.BCD}}{S_{BCD}}.\]
			Vì $ABCD$ là tứ diện đều nên $S_{BCD}=S_{ABC}=S_{ABD}=S_{ACD}$.\\
			$\Rightarrow V_{G.BCD}=V_{G.ABC}=V_{G.ABD}=V_{G.ACD}$
		}
		{
			\begin{tikzpicture}[>=stealth,line join=round,line cap=round,font=\footnotesize,scale=1.1]
			\def\h{4}
			\def\cd{4}
			\def\xc{1.5}
			\def\yc{-1.25}
			
			\coordinate[label=left:{$B$}] (B) at (0,0);
			\coordinate[label=below:{$C$}] (C) at (\xc,\yc);
			\coordinate[label=right:{$D$}] (D) at (\cd,0);
			\coordinate (M) at ($(C)!0.5!(D)$);
			\coordinate (N) at ($(B)!0.5!(C)$);
			\coordinate[label=below:{$H$}] (H) at (intersection of B--M and D--N);
			\coordinate[label=above:{$A$}] (A) at ($(H)+(0,\h)$);
			\draw (A)--(B)--(C)--(D)--cycle
			(A)--(C);
			\draw[dashed] (B)--(M) (D)--(N) (B)--(D)
			(A)--(H);
			
			\pic[draw,angle radius=2mm,angle eccentricity=1.5] {right angle = D--H--A};
			\pic[draw,angle radius=2mm,angle eccentricity=1.5] {right angle = H--D--B};
			\pic[draw,angle radius=2mm,angle eccentricity=1.5] {right angle = D--N--B};
			\pic[draw,angle radius=2mm,angle eccentricity=1.5] {right angle = B--M--D};
			\foreach \point in {A,B,C,D,H}
			\fill (\point) circle (1pt);
			\end{tikzpicture}
		}
		\noindent Mặt khác vì $V_{G.BCD}+V_{G.ABC}+V_{G.ABD}+V_{G.ACD}=V_{ABCD}$ nên suy ra $V_{G.BCD}=\dfrac{1}{4}V_{ABCD}$.\\
		Ta có $BH=\dfrac{a\sqrt{3}}{3};\quad AH=\sqrt{AB^2-BH^2}=\dfrac{a\sqrt{6}}{3}$.\\
		Khi đó
		\[V_{ABCD}=\dfrac{1}{3}\cdot \dfrac{a^2\sqrt{3}}{4}\cdot\dfrac{a\sqrt{6}}{3}=\dfrac{a^3\sqrt{2}}{12}\Rightarrow V_{G.BCD}=\dfrac{1}{4}\cdot V_{ABCD}=\dfrac{a^3\sqrt{2}}{48}.\]
		Suy ra
		\[r=\mathrm{d}\left(G,(BCD)\right)=\dfrac{3V_{G.BCD}}{S_{BCD}}=\dfrac{3\cdot \dfrac{a^3\sqrt{2}}{48}}{\dfrac{a^2\sqrt{3}}{4}}=\dfrac{a\sqrt{6}}{12}.\]
		Vậy thể tích khối cầu nội tiếp tứ diện là $V=\dfrac{4}{3}\pi r^3=\dfrac{a^3\pi\sqrt{6}}{216}$.
	}
\end{vd}

\subsubsection{Câu hỏi trắc nghiệm}
\begin{ex}%Câu 1.%[2H2K2-2]
	Cho hình chóp $S.ABC$ có $SA=SB=SC=a$ và $\widehat{ASB}=90^{\circ}$, $\widehat{BSC}=60^{\circ}$, $ \widehat{CSA}=120^{\circ}$. Diện tích mặt cầu ngoại tiếp của hình chóp $S.ABC$ là 
	\choice
	{\True $4\pi a^2$}
	{$2\pi a^2$}
	{$\pi a^2$}
	{$\dfrac{4}{3}\pi a^3$}
	\loigiai{
		\immini{Áp dụng định lí cosin cho tam giác $SAB$ ta được
			\begin{eqnarray*}
				AB^2&=&SA^2+SB^2-2SA\cdot SB\cdot \cos\widehat{ASB}\\
				&=&a^2+a^2-2a\cdot a\cdot\cos90^\circ=2a^2\\
				\Rightarrow AB&=&a\sqrt{2}.
			\end{eqnarray*}
			Áp dụng định lí cosin cho tam giác $SAC$ ta được
			\begin{eqnarray*}
				AC^2&=&SA^2+SC^2-2SA\cdot SC\cdot \cos\widehat{ASC}\\
				&=&a^2+a^2-2a\cdot a\cdot\cos120^\circ=3a^2\\
				\Rightarrow AC&=&a\sqrt{3}.
			\end{eqnarray*}
			Áp dụng định lí cosin cho tam giác $SBC$ ta được
			\begin{eqnarray*}
				BC^2&=&SB^2+SC^2-2SB\cdot SC\cdot \cos\widehat{BSC}\\
				&=&a^2+a^2-2a\cdot a\cdot\cos60^\circ=a^2\\
				\Rightarrow BC&=&a.
		\end{eqnarray*}}
		{\begin{tikzpicture}[>=stealth,line join=round,line cap=round,font=\footnotesize,scale=1.2]
			\def\h{3}
			\def\ac{4}
			\def\xb{3}
			\def\yb{-1}
			\coordinate[label=left:{$A$}] (A) at (0,0);
			\coordinate[label=below:{$B$}] (B) at (\xb,\yb);
			\coordinate[label=right:{$C$}] (C) at (\ac,0);
			\coordinate[label=below:{$O$}] (O) at ($(A)!0.5!(C)$);
			\coordinate[label=above:{$S$}] (S) at ($(O)+(0,\h)$);
			\coordinate[label=above right:{$E$}] (E) at ($(S)!0.5!(C)$);
			\coordinate[rotate around={-90:(A)}] (x) at (0,0);
			\coordinate[label=below left:{$I$}] (I) at (intersection of S--O and E--x);
			\draw (S)--(A)--(B)--(C)--cycle (S)--(B);
			\draw[dashed] (S)--(O) (A)--(C) (E)--(I);
			
			\pic[draw,angle radius=2mm,angle eccentricity=1.5] {right angle = I--E--S};
			\pic[draw,angle radius=2mm,angle eccentricity=1.5] {right angle = S--O--C};
			\pic[draw,angle radius=2.5mm,angle eccentricity=1.5] {right angle = B--S--A};
			\foreach \point in {A,B,C,S,E,I,O}
			\fill (\point) circle (1pt);
			\end{tikzpicture}
		}
		\noindent Từ đó ta có $AB^2+BC^2=AC^2$ nên suy ra $\triangle ABC$ vuông tại $B$.\\
		Gọi $O$ là trung điểm của $AC$. Ta có $O$ là tâm đường tròn ngoại tiếp $\triangle ABC$.\\
		Vì $SA=SB=SC$ và $OA=OB=OC$ nên $SO$ là trục đường tròn ngoại tiếp tam giác $ABC$.\\
		$\Rightarrow SO\perp (ABC)$.\\
		Dựng mặt phẳng trung trực của $SC$ cắt $SO$ tại $I$. Khi đó $I$ là tâm mặt cầu ngoại tiếp hình chóp $S.ABC$.\\
		Với $E$ là trung điểm AC ta có $\triangle SEI\backsim\triangle SOC$ (góc-góc) nên suy ra $\dfrac{SI}{SC}=\dfrac{SE}{SO}.\qquad (1)$\\
		Áp dụng định lí Pi-ta-go cho tam giác $SOC$ vuông tại $O$ ta được
		\[SO^2=SB^2-BO^2=\dfrac{a^2}{4}\Rightarrow SO=\dfrac{a}{2}.\]
		Mặt khác ta có $SE=\dfrac{a}{2}$ và $SC=a$ nên từ $(1)$ suy ra $SI=a$.\\
		Vậy bán kính mặt cầu ngoại tiếp hình chóp $S.ABC$ là $a$.\\
		Suy ra diện tích mặt cầu ngoại tiếp hình chóp $S.ABC$ là $4\pi a^2$.\\
		\textbf{Chú ý:} Sau khi chứng minh $SO\perp (ABC)$ tại $O$ ta có thể áp dụng $R=\dfrac{SA^2}{2\cdot SO}=\dfrac{a^2}{2\dfrac{AC}{2}}=a$.
	}
\end{ex}
\begin{ex}%Câu 2.%[2H2B2-2]%[Đặng Tấn Phát]
	Cho hình chóp tam giác đều có cạnh đáy bằng $\sqrt{6}$ và chiều cao $h=1$. Diện tích của mặt cầu ngoại tiếp của hình chóp đó là 
	\choice
	{\True $S=9\pi$}
	{$=6\pi$}
	{$S=5\pi$}
	{$S=27\pi$}
	\loigiai{
		\immini{
			Gọi $O$ là tâm của tam giác $ABC$. Suy ra $SO\perp (ABC)$.\\
			Khi đó
			\[SO=h=1;\quad OA=\dfrac{2}{3}\cdot\sqrt{6}\cdot\dfrac{\sqrt{3}}{2}=\sqrt{2}.\]
			Xét tam giác vuông $SAO$ ta có $SA=\sqrt{SO^2+OA^2}=\sqrt{1+2}=\sqrt{3}$.\\
			Trong $(SAO)$ dựng đường trung trực của đoạn $SA$ cắt $SO$ tại $I$.\\
			$\Rightarrow IS=IA=IB=IC$ nên $I$ là tâm mặt cầu ngoại tiếp hình chóp $S.ABC$.\\
			Gọi $H$ là trung điểm $SA$, ta có $\triangle SHI$ đồng dạng với $\triangle SOA$ nên
			\[R=IS=\dfrac{SH\cdot SA}{SO}=\dfrac{\dfrac{\sqrt{3}}{2}\cdot\sqrt{3}}{1}=\dfrac{3}{2}\]
			Vậy diện tích mặt cầu ngoại tiếp hình chóp là $S=4\pi R^2=9\pi$.}{\begin{tikzpicture}[>=stealth,line join=round,line cap=round,font=\footnotesize,scale=1]
			\def\h{1.75}
			\def\ac{4.5}
			\def\xb{3.5}
			\def\yb{-1}
			\coordinate[label=left:{$A$}] (A) at (0,0);
			\coordinate[label=below:{$B$}] (B) at (\xb,\yb);
			\coordinate[label=right:{$C$}] (C) at (\ac,0);
			\coordinate[label=right:{$O$}] (O) at ($($(B)!0.5!(C)$)!1/3!(A)$);
			\coordinate[label=above:{$S$}] (S) at ($(O)+(0,\h)$);
			\coordinate[label=above:{$H$}] (H) at ($(S)!0.5!(A)$);
			\coordinate[rotate around={-90:(H)}] (x) at (0,0);
			\coordinate[label=below:{$I$}] (I) at (intersection of S--O and H--x);
			\coordinate (gd1) at (intersection of H--I and A--B); % giao điểm HI và BC để hiển thị nét đứt-nét liền
			\coordinate (gd2) at (intersection of S--O and A--B); % giao điểm SO và BC để hiển thị nét đứt-nét liền
			\draw 
			(S)--(A)--(B)--(C)--cycle
			(S)--(B)
			(gd1)--(I) (gd2)--(I);
			
			\draw[densely dashed] (S)--(O) (A)--(C) (H)--(gd1) (S)--(gd2) (A)--(O);
			\pic[draw,angle radius=2mm,angle eccentricity=1.5] {right angle = A--O--S};
			\pic[draw,angle radius=2mm,angle eccentricity=1.5] {right angle = I--H--S};
			
			\foreach \point in {A,B,C,S,O,I,H}
			\fill (\point) circle (1pt);
			\end{tikzpicture}
		}
	}
\end{ex}
\begin{ex}%Câu 3.%[2H2K2-2]%[Đặng Tấn Phát]
	Cho tứ diện $ABCD$ có $ABC$ và $DBC$ là các tam giác đều cạnh $a$, $AD=\dfrac{4}{3}a$. Tính bán kính mặt cầu ngoại tiếp tứ diện $ABCD$. 
	\choice
	{\True $\dfrac{\sqrt{55}}{11}a$}
	{$\dfrac{\sqrt{57}}{11}a$}
	{$\dfrac{\sqrt{59}}{11}a$}
	{$\dfrac{\sqrt{61}}{11}a$}
	\loigiai{
		\immini{
			Gọi $M$ là trung điểm $BC$.\\
			Vì $\triangle ABC$ và $\triangle DBC$ là tam giác đều nên suy ra $BC\perp AM, BC\perp DM, AM=DM$.\\
			Do đó $BC\perp (AMD)$. Suy ra $(AMD)$ là mặt phẳng trung trực của $BC$.\\
			Trong $(AMD)$ dựng $AH\perp MD$ thì $AH\perp (BCD)$.\\
			Dựng $d\perp (BCD)$ tại $G$ là trọng tâm của $\triangle ABC$. Khi đó $d$ là trục của đáy $BCD$.\\
			Trong $(AMD)$ gọi $O$ là giao điểm của $d$ và $MK$ với $K$ là trung điểm $AD$.\\
			Vì $\triangle AMD$ cân tại $M$ nên $MK$ là đường trung trực của $AD$.\\
			$\Rightarrow OA=OB=OC=OD$ hay $O$ là tâm mặt cầu ngoại tiếp tứ diện $ABCD$.\\
			Ta có $AM=DM=\dfrac{a\sqrt{3}}{2};\quad DK=\dfrac{1}{2}AD=\dfrac{2}{3}a$ nên 
			\[MK^2=MD^2-DK^2=\left(\dfrac{a\sqrt{3}}{2}\right)^2-\left(\dfrac{2}{3}a\right)^2\Rightarrow MK=\dfrac{a\sqrt{11}}{6}.\]
			Ta có $\tan\widehat{KMD}=\dfrac{DK}{MK}=\dfrac{OG}{MG}\Rightarrow OG=\dfrac{DK\cdot MG}{MK}=\dfrac{2a\sqrt{33}}{33}$.\\
			Vì $GD=\dfrac{2}{3}MD=\dfrac{a\sqrt{3}}{3}$ nên $R=OD=\sqrt{OG^2+GD^2}=\dfrac{a\sqrt{55}}{11}$.
		}{
			\begin{tikzpicture}[>=stealth,line join=round,line cap=round,font=\footnotesize,scale=1]
			\def\h{3.5}
			\def\bd{3.75}
			\def\xc{0.75}
			\def\yc{-1.5}
			\coordinate[label=left:{$B$}] (B) at (0,0);
			\coordinate[label=below:{$C$}] (C) at (\xc,\yc);
			\coordinate[label=below:{$D$}] (D) at (\bd,0);
			\coordinate[label=below left:{$M$}] (M) at ($(B)!0.5!(C)$);
			\coordinate[label=below:{$H$}] (H) at ($(M)!0.2!(D)$);
			\coordinate[label=below:{$G$}] (G) at ($(M)!1/3!(D)$);
			\coordinate[label=above:{$A$}] (A) at ($(H)+(0,\h)$);
			\coordinate[label=above right:{$K$}] (K) at ($(A)!0.5!(D)$);
			
			\coordinate (x) at ($(G)+(A)-(H)$);
			\coordinate[label=above:{$O$}] (O) at (intersection of G--x and M--K);
			
			\draw (A)--(B)--(C)--(D)--cycle
			(A)--(M) (A)--(C);
			\draw[dashed] (B)--(D) (M)--(D) (O)--(G) (M)--(K) (A)--(H) 
			(O)--(D);
			
			\pic[draw,angle radius=2mm,angle eccentricity=1.5] {right angle = D--G--O};
			\pic[draw,angle radius=2mm,angle eccentricity=1.5] {right angle = O--K--D};
			\pic[draw,angle radius=2mm,angle eccentricity=1.5] {right angle = D--H--A};
			
			\foreach \point in {A,B,C,D,O,G,H,K,M}
			\fill (\point) circle (1pt);
			\end{tikzpicture}
		}
	}
\end{ex}
\begin{ex}%Câu 4.%[2H2K2-3]%[Đặng Tấn Phát]
	Tính thể tích $V$ của khối chóp tứ giác đều có chiều cao là $h$ và bán kính mặt cầu nội tiếp là $r$ ($h>2r>0$). 
	\choice
	{$V=\dfrac{4r^2h^2}{3(h+2r)}$}
	{$V=\dfrac{4r^2h^2}{h+2r}$}
	{\True $V=\dfrac{4r^2h^2}{3(h-2r)}$}
	{$V=\dfrac{3r^2h^2}{4(h-2r)}$}
	\loigiai{
		\immini{
			Gọi $M$ và $M'$ lần lượt là trung điểm của $AB$ và $CD$. \\
			Gọi $O$ là tâm của hình vuông $ABCD$ và $I$ là giao điểm ba đường phân giác trong của tam giác $SMM'$. Khi đó $I$ là tâm đường tròn nội tiếp tam giác $SMM'$.\\
			Mặt khác vì $S.ABCD$ là hình chóp tứ giác đều nên $I$ là tâm mặt cầu nội tiếp hình chóp.\\
			Xét $\triangle SMO$ có $MI$ là đường phân giác, ta có
			\[\dfrac{SM}{MO}=\dfrac{SI}{IO}\Rightarrow \dfrac{\sqrt{h^2+x^2}}{x}=\dfrac{h-r}{r}\Rightarrow x^2=\dfrac{hr^2}{h-2r}.\quad(\text{Với } x=OM)\]
			Vậy thể tích cần tìm là $V=\dfrac{1}{3}h\cdot4x^2=\dfrac{h^2r^2}{3(h-2r)}$.}{
			\begin{tikzpicture}[>=stealth,line join=round,line cap=round,font=\footnotesize,scale=1]
			\def\h{4}
			\def\ad{3.5}
			\def\xb{-2.5}
			\def\yb{-0.5}
			\coordinate[label=below:{$O$}] (O) at (0,0);
			\coordinate[label=left:{$B$}] (B) at (\xb,\yb);
			\coordinate[label=below:{$C$}] (C) at ($(B)+(\ad,0)$);
			\coordinate[label=right:{$D$}] (D) at ($-1*(B)$);
			\coordinate[label=above left:{$A$}] (A) at ($-1*(C)$);
			\coordinate[label=above:{$S$}] (S) at (0,\h);
			
			\coordinate[label={[label distance=-0.2cm]above left:$M'$}] (M') at ($(A)!0.5!(B)$);
			\coordinate[label=below right:{$M$}] (M) at ($(C)!0.5!(D)$);
			\coordinate[label=above right:{$I$}] (I) at ($(S)!0.75!(O)$);
			\coordinate (x) at (intersection of M--I and S--M');
			
			\draw (S)--(B)--(C)--(D)--cycle
			(S)--(C) (S)--(M);
			\draw[dashed] (D)--(A)--(B) (M')--(M)--(x) (A)--(C) (B)--(D) (S)--(O) (S)--(A) (S)--(M');
			
			\foreach \point in {A,B,C,D,S,M,M',I,O}
			\fill (\point) circle (1pt);
			
			\end{tikzpicture}}
	}
\end{ex}
\begin{ex}%Câu 5.%[2H2K2-2]%[Đặng Tấn Phát]
	Cho hình chóp $S.ABC$ có $\widehat{BSC}=120^{\circ}$, $\widehat{CSA}=60^{\circ}$, $\widehat{ASB}=90^{\circ}$ và $SA=SB=SC$. Gọi $I$ là hình chiếu vuông góc của $S$ lên mặt phẳng $(ABC)$. Khẳng định nào sau đây đúng?
	\choice
	{$I$ là trung điểm $AB$}
	{$I$ là trọng tâm tam giác $ABC$}
	{$I$ là trung điểm $AC$}
	{\True $I$ là trung điểm $BC$}
	\loigiai{
		\immini{
			Ta có $SA=SB=SC$ nên hình chiếu của $S$ lên $(ABC)$ là tâm đường tròn ngoại tiếp $\triangle ABC$.\\
			Đặt $SA=SB=SC=a$.\\
			Theo giả thiết ta có tam giác $SAC$ đều cạnh $a$, tam giác $SAB$ vuông cân tại $S$ nên $AC=a$ và $AB=a\sqrt{2}$.\\
			Xét tam giác $SBC$ ta có\\
			$BC^2=SB^2+SC^2-2SB\cdot SC\cdot\cos\widehat{BSC} =a^2+a^2-2\cdot a\cdot a\cdot\cos 120^{\circ}=3a^2$.\\
			Do đó $AB^2+AC^2 =3a^2=BC^2$ nên tam giác $ABC$ vuông tại $A$.\\
			Khi đó tâm đường tròn ngoại tiếp $\triangle ABC$ là trung điểm $BC$.\\
			Vậy $I$ là trung điểm $BC$.}
		{\begin{tikzpicture}[>=stealth,line join=round,line cap=round,font=\footnotesize,scale=1]
			\def\h{3}
			\def\bc{4}
			\def\xa{1}
			\def\ya{-1.25}
			\coordinate[label=left:{$B$}] (B) at (0,0);
			\coordinate[label=right:{$C$}] (C) at (\bc,0);
			\coordinate[label=below:{$A$}] (A) at (\xa,\ya);
			\coordinate[label=below:{$I$}] (I) at ($(B)!0.7!($(A)!0.5!(C)$)$);
			\coordinate[label=above:{$S$}] (S) at ($(I)+(0,\h)$);
			
			\draw (S)--(B)--(A)--(C)--cycle 
			(S)--(A);
			\draw[dashed] (S)--(I) (B)--(C);
			
			\pic[draw,angle radius=2mm,angle eccentricity=1.5] {right angle = C--I--S};
			\pic[draw,angle radius=2mm,angle eccentricity=1.5] {right angle = B--S--A};
			\pic[draw,angle radius=2mm,angle eccentricity=1.5] {right angle = B--A--C};
			\foreach \point in {S,A,B,C,I}
			\fill (\point) circle (1pt);
			\end{tikzpicture}}
	}
\end{ex}
\begin{ex}%Câu 6.%[2H2K2-2]%[Đặng Tấn Phát]
	Cho hình chóp tứ giác đều có góc giữa mặt bên và mặt đáy bằng $60^{\circ}$. Biết rằng mặt cầu ngoại tiếp hình chóp đó có bán kính $R=a\sqrt{3}$. Tính độ dài cạnh đáy của hình chóp tứ giác đều nói trên. 
	\choice
	{$\dfrac{9}{4}a$}
	{$2a$}
	{$\dfrac{3}{2}$a}
	{\True $\dfrac{12}{5}a$}
	\loigiai{
		\immini{
			Gọi $N$ là trung điểm của $BC$. Suy ra góc giữa hai mặt phẳng $(SBC)$ và $(ABCD)$ là $\widehat{SNO}$. Vậy $\widehat{SNO}=60^\circ$.\\
			Gọi $M$ là trung điểm $SB$, dựng $MI\perp SB$ ($I\in SO$).\\
			Suy ra $I$ là tâm mặt cầu ngoại tiếp hình chóp.\\
			Đặt $CD=2x \Rightarrow SO=x\sqrt{3}, SB=x\sqrt{5}$.\\
			Ta có $\triangle SMI$ đồng dạng với $\triangle SOB$ nên suy ra
			\[SI=\dfrac{SM\cdot SB}{SO}=\dfrac{\dfrac{SB}{2}\cdot SB}{SO}=\dfrac{5x\sqrt{3}}{6}=a\sqrt{3}\Rightarrow x=\dfrac{6a}{5}.\]
			Vậy độ dài cạnh đáy là $2x=\dfrac{12a}{5}$.}
		{
			\begin{tikzpicture}[>=stealth,line join=round,line cap=round,font=\footnotesize,scale=1]
			\def\h{4}
			\def\ab{3.5}
			\def\xd{-2.5}
			\def\yd{-0.5}
			\coordinate[label=below:{$O$}] (O) at (0,0);
			\coordinate[label=left:{$D$}] (D) at (\xd,\yd);
			\coordinate[label=below:{$C$}] (C) at ($(D)+(\ab,0)$);
			\coordinate[label=right:{$B$}] (B) at ($-1*(D)$);
			\coordinate[label=above left:{$A$}] (A) at ($-1*(C)$);
			\coordinate[label=above:{$S$}] (S) at (0,\h);
			
			\coordinate[label=above right:{$M$}] (M) at ($(S)!0.5!(B)$);
			\coordinate[label=below right:{$N$}] (N) at ($(B)!0.5!(C)$);
			\coordinate[label=left:{$I$}] (I) at ($(S)!0.65!(O)$);
			
			\draw (S)--(D)--(C)--(B)--cycle
			(S)--(C) (S)--(N);
			\draw[dashed] (D)--(A)--(B) (A)--(C) (B)--(D) (A)--(S)--(O)--(N) (I)--(M);
			
			\pic[draw,angle radius=2mm,angle eccentricity=1.5] {right angle = I--M--S};
			\pic[draw,angle radius=3mm,angle eccentricity=1.5] {angle = S--N--O};
			\pic[draw,angle radius=2mm,angle eccentricity=1.5] {right angle = B--O--S};
			\foreach \point in {A,B,C,D,S,M,N,I,O}
			\fill (\point) circle (1pt);	
			\end{tikzpicture}
		}
	}
\end{ex}
\begin{ex}%Câu 7.%[2H2G2-2]%[Đặng Tấn Phát]
	Cho hình chóp $S.ABC$ có $SA=SB=SC=2a$, tam giác $ABC$ có góc $A$ bằng $120^\circ, BC=2a$. Tính bán kính mặt cầu ngoại tiếp hình chóp tho $a$. 
	\choice
	{$\dfrac{a\sqrt{3}}{2}$}
	{$\dfrac{2a\sqrt{3}}{3}$}
	{$\dfrac{a\sqrt{6}}{6}$}
	{\True $\dfrac{a\sqrt{6}}{2}$}
	\loigiai{
		\immini{
			Gọi $I$ là tâm đường tròn ngoại tiếp $\triangle ABC$.\\
			Do $SA=SB=SC$ nên ta có $SI\perp (ABC)$.\\
			Gọi $K$ là trung điểm $SA$. Dựng $OK$ là đường trung trực của $SA$ với $O\in SI$.\\
			Khi đó $O$ là tâm của mặt cầu ngoại tiếp hình chóp $S.ABC$.\\
			Ta có $S_{ABC}=\dfrac{1}{2}AB\cdot AC\cdot\sin A$ và $S_{ABC}=\dfrac{AB\cdot AC \cdot BC}{4IA}$. Khi đó
			\[\dfrac{1}{2}AB\cdot AC\cdot\sin 120^\circ=\dfrac{AB\cdot AC\cdot 2a}{4IA}\Rightarrow IA=\dfrac{4a}{4\sin 120^\circ}=\dfrac{2a\sqrt{3}}{3}.\]
			Xét tam giác $SIA$ vuông tại $I$ ta có
			\[SI=\sqrt{SA^2-IA^2}=\sqrt{4a^2-\dfrac{4a^2}{3}}=\dfrac{2a\sqrt{6}}{3}.\]
			Vì $\triangle SKO$ đồng dạng với $\triangle SIA$ nên ta có
			\[\dfrac{SK}{SI}=\dfrac{SO}{SA}\Rightarrow SO=\dfrac{SK\cdot SA}{SI}=\dfrac{SA^2}{2SI}=\dfrac{4a^2}{2\cdot \dfrac{2a\sqrt{6}}{3}}=\dfrac{a\sqrt{6}}{2}\]}
		{\begin{tikzpicture}[>=stealth,line join=round,line cap=round,font=\footnotesize,scale=1]
			\def\h{4}
			\def\bc{4}
			\def\xa{0.75}
			\def\ya{-1.25}
			\coordinate[label=left:{$B$}] (B) at (0,0);
			\coordinate[label=right:{$C$}] (C) at (\bc,0);
			\coordinate[label=below:{$A$}] (A) at (\xa,\ya);
			\coordinate[label=below:{$I$}] (I) at ($(B)!0.7!($(A)!0.5!(C)$)$);
			\coordinate[label=above:{$S$}] (S) at ($(I)+(0,\h)$);
			\coordinate[label=left:{$K$}] (K) at ($(S)!0.5!(A)$);
			\coordinate[label=right:{$O$}] (O) at ($(S)!0.65!(I)$);
			\draw (S)--(B)--(A)--(C)--cycle 
			(S)--(A);
			\draw[dashed] (S)--(I) (B)--(C)
			(O)--(K) (I)--(A);
			
			\pic[draw,angle radius=2mm,angle eccentricity=1.5] {right angle = A--I--S};
			\pic[draw,angle radius=2mm,angle eccentricity=1.5] {right angle = O--K--S};
			\foreach \point in {S,A,B,C,I,O}
			\fill (\point) circle (1pt);
			\end{tikzpicture}
		}
	}
\end{ex}
\Closesolutionfile{ans}
% \DAPAN
\inputansbox{10}{ans/ans2H2-2}
\Opensolutionfile{ans}[ans/ans2H2-2]	
\begin{dang}{Chóp có một mặt bên vuông góc với đáy}
\end{dang}
\subsubsection{Các ví dụ}		
\begin{vd}%Ví dụ 1.%[2H2B2-2]%[Đặng Tấn Phát]
	Cho hình chóp $S.ABC$ có đáy $ABC$ là tam giác vuông tại $A$. Mặt bên $(SAB)\perp (ABC)$ và $\triangle SAB$ đều. Tìm tâm và tính bán kính khối cầu ngoại tiếp hình chóp.
	\loigiai{
		\immini{Gọi $H,M$ lần lượt là trung điểm của $AB,AC$.\\
			Ta có $M$ là tâm đường tròn ngoại tiếp $\triangle ABC$ (do $MA=MB=MC$).\\
			Dựng $d_1$ là trục đường tròn ngoại tiếp $\triangle ABC$ ($d_1$ qua $M$ và song song $SH$).\\
			Gọi $G$ là tâm đường tròn ngoại tiếp $\triangle SAB$ và $d_2$ là trục đường tròn ngoại tiếp $\triangle SAB$. $d_1$cắt $d_2$ tại $I$.\\
			Khi đó $I$ là tâm mặt cầu ngoại tiếp khối chóp $S.ABC$.\\
			Bán kính $R=SI$. Xét $\triangle SGI$ suy ra $SI=\sqrt{GI^2+SG^2}$.}{
			\begin{tikzpicture}[>=stealth,line join=round,line cap=round,font=\footnotesize,scale=1.2]
			\def\h{4}
			\def\ac{4}
			\def\xb{0.6}
			\def\yb{-0.6}
			\coordinate[label=below left:{$H$}] (H) at (0,0);
			\coordinate[label=below:{$B$}] (B) at (\xb,\yb);
			\coordinate[label=left:{$A$}] (A) at ($-1*(B)$);
			\coordinate[label=right:{$C$}] (C) at ($(A)+(\ac,0)$);
			\coordinate[label=above:{$S$}] (S) at (0,\h);
			
			\coordinate[label=below right:{$M$}] (M) at ($(B)!0.5!(C)$);
			\coordinate[label=left:{$G$}] (G) at ($(S)!2/3!(H)$);
			\coordinate[label=above right:{$I$}] (I) at ($(G)+(M)-(H)$);
			\coordinate (x) at ($(G)+2/3*(M)-2/3*(H)$);
			\draw (S)--(A)--(B)--cycle
			(B)--(C)--(S) (S)--(I) (S)--(H);
			\draw[dashed] (A)--(M)--(H) (A)--(C) (G)--(x); 
			\draw[shorten >=-1.5cm] (x)--(I)node[above,xshift=1cm]{$d_2$};
			\draw[shorten >=-1.5cm] (M)--(I)node[right,yshift=1cm]{$d_1$};
			
			\pic[draw,angle radius=2mm,angle eccentricity=1.5] {right angle = B--A--C};
			\pic[draw,angle radius=2mm,angle eccentricity=1.5] {right angle = M--H--S};
			\pic[draw,angle radius=2mm,angle eccentricity=1.5] {right angle = H--M--I};
			\pic[draw,angle radius=2mm,angle eccentricity=1.5] {right angle = B--H--S};
			\foreach \point in {S,A,B,C,H,M,I,G}
			\fill (\point) circle (1pt);
			\end{tikzpicture}
		}
	}
\end{vd}
\begin{vd}%Ví dụ 2.%[2H2B2-2]%[Đặng Tấn Phát]]
	Cho hình chóp $S.ABCD$ có đáy là hình vuông và $BD=2a$. Tam giác $SAC$ vuông cân tại $S$ và nằm trong mặt phẳng vuông góc với đáy. Tính thể tích của khối cầu ngoại tiếp hình chóp. 
	\loigiai{
		\immini{
			Vì $\triangle SAC$ vuông cân tại $S$ nên $OS=OA=OC \quad (1)$\\
			Mặt khác vì $ABCD$ là hình vuông nên \[OA=OC=OB=OD=\dfrac{BD}{2}=a\quad (2)\]
			Từ $(1)$ và $(2)$ suy ra $O$ là tâm mặt cầu ngoại tiếp hình chóp $S.ABCD$. Do đó $R=OA=a$.\\
			Thể tích khối cầu $V=\dfrac{4}{3}\pi R^3=\dfrac{4}{3}\pi a^3$. }
		{\begin{tikzpicture}[>=stealth,line join=round,line cap=round,font=\footnotesize,scale=.8]
			\def\h{4}
			\def\ad{4}
			\def\xb{-3}
			\def\yb{-0.75}
			\coordinate[label=below:{$O$}] (O) at (0,0);
			\coordinate[label=below:{$B$}] (B) at (\xb,\yb);
			\coordinate[label=below:{$C$}] (C) at ($(B)+(\ad,0)$);
			\coordinate[label=right:{$D$}] (D) at ($-1*(B)$);
			\coordinate[label=left:{$A$}] (A) at ($-1*(C)$);
			\coordinate[label=above:{$S$}] (S) at (0,\h);
			
			\draw (S)--(B)--(C)--(D)--cycle
			(S)--(C);
			\draw[dashed] (S)--(A)--(B)--(D)--(A)--(C)
			(S)--(O);
			
			\foreach \point in {S,A,B,C,D,O}
			\fill (\point) circle (1pt);
			\end{tikzpicture}
		}
	}
\end{vd}
\begin{vd}%Ví dụ 3.%[2H2K2-2]%[Đặng Tấn Phát]
	Cho hình chóp tam giác đều $S.ABC$ có $AB=a$, góc tạo bởi $(SAB)$ và $(ABC)$ bằng $60^\circ$. Tính diện tích xung quanh của hình nón đỉnh $S$ và có đường tròn đáy ngoại tiếp tam giác $ABC$.
	\loigiai{
		\immini{
			Gọi $M$ là trung điểm $AB$ và $O$ là tâm của $\triangle ABC$.\\
			Khi đó $\heva{&AB\perp CM\\&AB\perp SO}\Rightarrow AB\perp (SCM)\Rightarrow AB\perp SM$ và $AB\perp CM$.\\
			Do đó góc giữa $(SAB)$ và $(ABC)$ là $\widehat{SMO}=60^\circ$.\\
			Mặt khác tam giác $ABC$ đều cạnh $a$ nên $CM=\dfrac{a\sqrt{3}}{2}$.\\
			Suy ra $OM=\dfrac{1}{3}CM=\dfrac{a\sqrt{3}}{6}$.\\
			Khi đó
			\[SO=OM\cdot\tan60^\circ=\dfrac{a\sqrt{3}}{6}\cdot\sqrt{3}=\dfrac{a}{2}.\]
		}
		{\begin{tikzpicture}[>=stealth,line join=round,line cap=round,font=\footnotesize,scale=1]
			\def\h{4}
			\def\ac{4}
			\def\xb{1.5}
			\def\yb{-1}
			\coordinate[label=left:{$A$}] (A) at (0,0);
			\coordinate[label=below:{$B$}] (B) at (\xb,\yb);
			\coordinate[label=right:{$C$}] (C) at (\ac,0);
			\coordinate[label=below:{$M$}] (M) at ($(A)!0.5!(B)$);
			\coordinate (x) at ($(B)!0.5!(C)$);
			\coordinate[label=below:{$O$}] (O) at ($(M)!1/3!(C)$);
			\coordinate[label=above:{$S$}] (S) at ($(O)+(0,\h)$);
			
			\draw (S)--(A)--(B)--(C)--cycle
			(S)--(M) (S)--(B);
			\draw[dashed] (S)--(O) (A)--(x) (A)--(C) (C)--(M);
			
			\foreach \point in {A,B,C,M,O,S}
			\fill (\point) circle (1pt);
			\end{tikzpicture}}
		\noindent Hình nón đã cho có chiều cao $h=SO=\dfrac{a}{2}$, bán kính đáy $R=OA=\dfrac{a\sqrt{3}}{3}$ và độ dài đường sinh $l=\sqrt{h^2+R^2}=\dfrac{a\sqrt{21}}{6}$.\\
		Vậy diện tích xung quanh của hình nón là $S_{xq}=\pi Rl=\pi\cdot \dfrac{a\sqrt{3}}{3}\cdot\dfrac{a\sqrt{21}}{6}=\dfrac{\sqrt{7}\pi a^2}{6}$.
	}
\end{vd}
\begin{vd}%Ví dụ 4.%[2H2K2-2]%[Đặng Tấn Phát]
	Cho hình chóp $S.ABC$ có đáy là tam giác đều cạnh $1$. Tam giác $SAB$ đều và nằm trong mặt phẳng vuông góc với mặt phẳng đáy. Tính thể tích của khối cầu ngoại tiếp hình chóp đã cho. 
	\loigiai{
		\immini{
			Gọi $M$ là trung điểm $AB$.\\
			Vì $\triangle SAB$ đều và nằm trong mặt phẳng vuông góc với mặt phẳng đáy nên $SM\perp (ABC)$.\\
			Gọi $I$ là trọng tâm của tam giác $ABC$.\\
			Vì $\triangle ABC$ đều nên $I$ là tâm đường tròn ngoại tiếp $\triangle ABC$.\\
			Dựng đường thẳng $d$ qua $I$ và vuông góc với $(ABC)$.\\
			Gọi $J$ là tâm đường tròn ngoại tiếp $\triangle SAB$. Dựng đường thẳng $d'$ qua $J$ và vuông góc với $(SAB)$.\\
			Gọi $O$ là giao điểm của $d$ và $d'$. Khi đó $O$ là tâm mặt cầu ngoại tiếp hình chóp $S.ABC$ với $R=OC$.\\
			Vì $\triangle SAB$ và $\triangle ABC$ là những tam giác đều cạnh bằng 1 nên 
			\[JM=\dfrac{1}{3}\cdot\dfrac{\sqrt{3}}{2}=\dfrac{\sqrt{3}}{6};\quad IC=\dfrac{2}{3}\cdot\dfrac{\sqrt{3}}{2}=\dfrac{\sqrt{3}}{3}.\]
		}
		{
			\begin{tikzpicture}[>=stealth,line join=round,line cap=round,font=\footnotesize,scale=1.2]
			\def\h{4}
			\def\ac{4}
			\def\xb{0.6}
			\def\yb{-0.6}
			\coordinate[label=below left:{$M$}] (M) at (0,0);
			\coordinate[label=below:{$B$}] (B) at (\xb,\yb);
			\coordinate[label=left:{$A$}] (A) at ($-1*(B)$);
			\coordinate[label=right:{$C$}] (C) at ($(A)+(\ac,0)$);
			\coordinate[label=above:{$S$}] (S) at (0,\h);
			
			\coordinate[label=left:{$J$}] (J) at ($(S)!2/3!(H)$);
			\coordinate[label=above right:{$I$}] (I) at ($(M)!1/3!(C)$);
			\coordinate (x) at ($(G)+2/3*(I)-2/3*(M)$);
			\coordinate (y) at ($(I)+2/3*(J)-2/3*(M)$);
			\coordinate[label=below right:{$O$}] (O) at ($(J)+(I)-(M)$);
			
			\draw (S)--(A)--(B)--cycle
			(B)--(C)--(S) (S)--(M);
			\draw[dashed] (A)--(C)--(M) (J)--(x) (I)--(y); 
			\draw[shorten >=-1.5cm] (y)--($3/2*(O)-1/2*(I)$)node[above,xshift=0.8cm,yshift=-0.7cm]{$d'$};
			\draw[shorten >=-1.5cm] (x)--(O)node[right,yshift=1cm]{$d$};
			
			
			\pic[draw,angle radius=2mm,angle eccentricity=1.5] {right angle = B--M--S};
			\pic[draw,angle radius=2mm,angle eccentricity=1.5] {right angle = C--M--S};
			\pic[draw,angle radius=2mm,angle eccentricity=1.5] {right angle = M--I--O};
			\pic[draw,angle radius=2mm,angle eccentricity=1.5] {right angle = M--J--O};
			\foreach \point in {S,A,B,C,J,M,I,O}
			\fill (\point) circle (1pt);
			\end{tikzpicture}
		}
		\noindent Xét $\triangle ABC$ vuông tại $I$ có $OC=\sqrt{IC^2+OC^2}=\sqrt{\left(\dfrac{\sqrt{3}}{3}\right)^2+\left(\dfrac{\sqrt{3}}{6}\right)^2}=\dfrac{\sqrt{15}}{6}$.\\
		Vậy thể tích khối cầu ngoại tiếp hình chóp đã cho là
		\[V=\dfrac{4}{3}\pi\cdot\left(\dfrac{\sqrt{15}}{6}\right)^3=\dfrac{5\sqrt{15}\pi}{54}.\]
	}
\end{vd}
\begin{vd}%Ví dụ 5.%[2H2G2-2]%[Đặng Tấn Phát]
	Cho hình chóp $S.ABCD$ có đáy là hình vuông cạnh $a$, $SAD$ là tam giác đều và nằm trong mặt phẳng vuông góc với đáy. Gọi $M,N$ lần lượt là trung điểm của $BC$ và $CD$. Tính bán kính $R$ của khối cầu ngoại tiếp khối chóp $S.CMN$.
	\loigiai{
		\immini{
			Gọi $I$ là trung điểm $MN$. Khi đó $I$ là tâm đường tròn ngoại tiếp tam giác $CMN$.\\
			Dựng đường thẳng $d$ qua $I$ và vuông góc với mặt đáy.\\
			Gọi $O$ là tâm mặt cầu ngoại tiếp hình chóp. Suy ra $O\in d$.\\
			Gọi $H$ là trung điểm $AD$. Khi đó $SH\perp (ABCD)$, suy ra $SH\parallel d$.\\
			Trong mặt phẳng $(SHOI)$ gọi $K$ là hình chiếu của $O$ lên $SH$.\\
			Đặt $OI=x$. Vì $\triangle CMN$ vuông cân tại $C$, $\triangle OIC$ vuông tại I nên ta có:
			\[CI=\dfrac{1}{2}MN=\dfrac{a\sqrt{2}}{4};\quad OC=\sqrt{IC^2+IO^2}=\sqrt{\dfrac{a^2}{8}+x^2}.\]
			Gọi $E$ là hình chiếu của $I$ lên $AD$, ta có \[KO=HI=\sqrt{IE^2+EH^2}=\sqrt{\left(\dfrac{3a}{4}\right)^2+\left(\dfrac{a}{4}\right)^2}=\dfrac{a\sqrt{10}}{4}.\]
		}
		{\begin{tikzpicture}[>=stealth,line join=round,line cap=round,font=\footnotesize,scale=1]
			\def\h{4.5}
			\def\ab{4}
			\def\xa{-0.75}
			\def\ya{-0.75}
			\coordinate[label=left:{$H$}] (H) at (0,0);
			\coordinate[label=below:{$A$}] (A) at (\xa,\ya);
			\coordinate[label=above right:{$D$}] (D) at ($-1*(A)$);
			\coordinate[label=below:{$B$}] (B) at ($(A)+(\ab,0)$);
			\coordinate[label=right:{$C$}] (C) at ($(D)+(\ab,0)$);
			\coordinate[label=above:{$S$}] (S) at (0,\h);
			\coordinate[label=below:{$M$}] (M) at ($(B)!0.5!(C)$);
			\coordinate[label=above:{$N$}] (N) at ($(D)!0.5!(C)$);
			\coordinate[label=below:{$I$}] (I) at ($(M)!0.5!(N)$);
			\coordinate[label=above right:{$K$}] (K) at ($(S)!3/5!(H)$);
			\coordinate (x) at ($(I)+(H)-(M)$); %%% đường thẳng qua I song song HM
			\coordinate[label=above:{$E$}] (E) at (intersection of A--D and I--x);
			\coordinate[label=above right:{$O$}] (O) at ($(K)+(I)-(H)$);
			\coordinate (gd1) at ($(K)+2/3*(I)-2/3*(H)$);
			\coordinate (gd2) at ($(I)+2/3*(K)-2/3*(H)$);
			
			\draw (S)--(A)--(B)--(C)--cycle
			(S)--(B) (gd1)--(O) (gd2)--($(O)+1/3*(K)-1/3*(H)$)node[above]{$d$};
			\draw[dashed] (A)--(D)--(C) (S)--(H)--(M) 
			(S)--(D) (K)--(gd1) (M)--(N) (C)--(I)--(H) (I)--(gd2) 
			(I)--(E);
			
			\pic[draw,angle radius=2mm,angle eccentricity=1.5] {right angle = H--K--O};
			\pic[draw,angle radius=1.5mm,angle eccentricity=1.5] {right angle = I--E--D};
			\pic[draw,angle radius=2mm,angle eccentricity=1.5] {right angle = H--I--O};
			\pic[draw,angle radius=2mm,angle eccentricity=1.5] {right angle = I--H--S};
			\foreach \point in {A,B,C,D,S,O,I,M,N,H,K,E}
			\fill (\point) circle (1pt);
			\end{tikzpicture}
		}
		\noindent Khi đó
		\[SO=\sqrt{SK^2+KO^2}=\sqrt{\left(\dfrac{a\sqrt{3}}{2}-x\right)^2+\left(\dfrac{a\sqrt{10}}{4}\right)^2}=\sqrt{x^2-\sqrt{3}ax+\dfrac{22a^2}{16}}.\]
		Vì $O$ là tâm mặt cầu ngoại tiếp khối chóp $S.CMN$ nên $SO=OC$ nên ta có
		\[\sqrt{\dfrac{a^2}{8}+x^2}=\sqrt{x^2-\sqrt{3}ax+\dfrac{22a^2}{16}}\Leftrightarrow \sqrt{3}ax=\dfrac{5}{4}a^2\Leftrightarrow x=\dfrac{5\sqrt{3}a}{12}.\]
		Vậy $R=OC=\sqrt{\dfrac{a^2}{8}+\dfrac{25a^2}{48}}=\dfrac{\sqrt{93}}{12}a$.
	}
\end{vd}
\subsubsection{Câu hỏi trắc nghiệm}
\begin{ex}%Câu 1.%[2H2K2-2]%[Đặng Tấn Phát]
	Cho hình chóp $S.ABC$ có đáy $ABC$ là tam giác đều cạnh $6a$, tam giác $SBC$ vuông tại $S$ và mặt phẳng $(SBC)$ vuông góc với mặt phẳng $(ABC)$. Tính thể tích $V$ của khối cầu ngoại tiếp hình chóp $S.ABC$.
	\choice
	{$V=96\sqrt{3}\pi a^3$}
	{\True $V=32\sqrt{3}\pi a^3$}
	{$V=\dfrac{4\sqrt{3}}{27}\pi a^3$}
	{$V=\dfrac{4\sqrt{3}}{9}\pi a^3$}
	\loigiai{
		\immini{
			Gọi $H$ là trung điểm của $BC$.\\
			Vì $\triangle ABC$ đều nên $AH\perp BC$.\\
			Ta có $(SBC)\perp (ABC)$ và $(SBC)\cap(ABC)=BC$ nên $AH\perp (SBC)$.\\
			Do $H$ là tâm đường tròn ngoại tiếp tam giác $SBC$ nên $AH$ là trục đường tròn ngoại tiếp $\triangle SBC$.\\
			Vì $\triangle ABC$ đều nên có trọng tâm $G$ chính là tâm đường tròn ngoại tiếp.\\
			Vậy ta có $GA=GB=GC$. Mà $G\in AH$ nên $GS=GB=GC$.\\
			Suy ra $GS=GA=GB=GC$ hay $G$ là tâm mặt cầu ngoại tếp khối chóp $S.ABC$.\\
			Bán kính $R=GA=\dfrac{2}{3}\cdot6a\cdot \dfrac{\sqrt{3}}{2}=2\sqrt{3}a$.\\
			Thể tích khối cầu ngoại tiếp hình chóp là $V=\dfrac{4}{3}\pi\left(2\sqrt{3}a\right)^3=32\sqrt{3}\pi a^3$.
		}
		{\begin{tikzpicture}[>=stealth,line join=round,line cap=round,font=\footnotesize,scale=1]
			\def\h{3}
			\def\ac{4}
			\def\xb{3}
			\def\yb{-1}
			\coordinate[label=left:{$A$}] (A) at (0,0);
			\coordinate[label=right:{$C$}] (C) at (\ac,0);
			\coordinate[label=below:{$B$}] (B) at (\xb,\yb);
			\coordinate[label=below:{$H$}] (H) at ($(B)!0.5!(C)$);
			\coordinate[label=above:{$S$}] (S) at ($(H)+(0,\h)$);
			\coordinate[label=below:{$G$}] (G) at ($(A)!2/3!(H)$);
			
			\draw (S)--(A)--(B)--(C)--cycle
			(S)--(B) (S)--(H);
			\draw[dashed] (S)--(G) (A)--(C) (A)--(H) (G)--(B) (G)--(C);
			
			\pic[draw,angle radius=2mm,angle eccentricity=1.5] {right angle = B--H--A};
			\pic[draw,angle radius=2mm,angle eccentricity=1.5] {right angle = A--H--S};
			\foreach \point in {A,B,C,S,H,G}
			\fill (\point) circle (1pt);
			\end{tikzpicture}
		}
	}
\end{ex}

%--------------------------------------------------------------------------------------------------------

%\Opensolutionfile{ans}[ans/ans2H2-2]
\begin{ex}%[Phạm Thế Sinh]%[2H2K2-2]
	Cho tứ diện $ABCD$ có tam giác $ABC$ là tam giác cân với $\widehat{BAC}=120^\circ$, $AB=AC=a$. Hình chiếu của $D$ trên mặt phẳng $(ABC)$ là trung điểm $BC$. Tính bán kính $R$ của mặt cầu ngoại tiếp tứ diện $ABCD$ biết thể tích của tứ diện $ABCD$ là $V=\dfrac{a^3}{16}$.
	\choice
	{\True $R=\dfrac{\sqrt{91a}}{8}$}
	{$R=\dfrac{a\sqrt{13}}{4}$}
	{$R=\dfrac{13a}{2}$}
	{$R=6a$}
	\loigiai{
		\immini{
		Gọi $H$ là trung điểm $BC$.\\
		Có $AB=a,\widehat{BAH}=60^\circ\Rightarrow AH=\dfrac{a}{2}; BH=\dfrac{a\sqrt{3}}{2}$ và $BC=a\sqrt{3}$.\\
		$V_{ABCD}=\dfrac{1}{3}DH\cdot S_{ABC}\Leftrightarrow\dfrac{a^3}{16}=\dfrac{1}{3}DH\cdot\dfrac{1}{2}a^2\dfrac{\sqrt{3}}{2}\Leftrightarrow DH=\dfrac{a\sqrt{3}}{4}$.\\
		Do đó $DA=\sqrt{AH^2+DH^2}=\dfrac{a\sqrt{7}}{4}$.\\
		Gọi $O$ là tâm đường tròn ngoại tiếp tam giác $ABC$ thì bán kính đường tròn đó là $R=AO=\dfrac{BC}{2\cdot \sin A}=a$. Vậy $H$ là trung điểm $AO$.\\
		Kẻ trục đường tròn ngoại tiếp tam giác $ABC$, đường thẳng này cắt $AD$ tại $S \Rightarrow D$ là trung điểm $SA$.\\
		Suy ra $SO=2DH=\dfrac{a\sqrt{3}}{2},SA=2DA=\dfrac{a\sqrt{7}}{2}$
		và $SM=\dfrac{3}{4}SA=\dfrac{3a\sqrt{7}}{8}$.	
		}{
		\begin{tikzpicture}[line join=round,line width=.6pt,font=\small,scale=1]
		\def\gvg[size=#1](#2,#3,#4){\draw[red,very thin]($(#3)!#1!(#2)$)--($($(#3)!#1!(#2)$)+($(#3)!#1!(#4)$)-(#3)$)--($(#3)!#1!(#4)$);}
		\coordinate (A) at (0,0);
		\coordinate (B) at (1,-1);
		\coordinate (C) at (3,0);
		\coordinate (H) at ($(B)!.5!(C)$);
		\coordinate (D) at ($ (H)+(0,3) $);
		\coordinate (O) at ($ (A)!2!(H)$);
		\coordinate (S) at ($ (A)!2!(D)$);
		\coordinate (M) at ($(A)!.5!(D)$);
		\coordinate (I) at ($ (S)-(0,5) $);
		\coordinate(N) at (intersection of M--I and D--H);
		\draw (H)--(D)--(B)--(C)--(D)--(A) (H)--(O)--(S)--(D) (B)--(A) (N)--(I);
		\draw[dashed] (H)--(A)--(C) (M)--(N);
		\foreach \p/\g in {A/180,B/-90,C/0,D/120,H/-90,I/0,M/120,O/0,S/90}\draw[fill=gray](\p)circle (1pt)node[shift={(\g:.25)}]{$\p$};
		\draw[fill=gray](N)circle (1pt);
		\gvg[size=5pt](I,M,S);
		\end{tikzpicture}
		}
		\noindent Từ trung điểm $M$ của đoạn $AD$ kẻ đường vuông góc với $AD$, cắt $SO$ tại $I$.\\
		Dễ dàng có $I$ là tâm mặt cầu ngoại tiếp tứ diện $ABCD$.\\
		Hai tam giác vuông $SAO$ và $SIM$ đồng dạng nên $\dfrac{MI}{OA}=\dfrac{SM}{SO}\Rightarrow MI=a\cdot \dfrac{3\sqrt{7}a}{8\cdot \dfrac{a\sqrt{3}}{2}}=\dfrac{a\sqrt{21}}{4}$.\\
		Bán kính mặt cầu bằng $R=ID=\sqrt{MI^2+MD^2}=\dfrac{a\sqrt{91}}{8}$.
	}
\end{ex}
\begin{ex}%[Phạm Thế Sinh]%[2H2K2-2]
	Cho hình chóp $S.ABCD$ có đáy là hình chữ nhật $AB=3,AD=2$. Mặt bên $(SAB)$ là tam giác đều và nằm trong mặt phẳng vuông góc với đáy. Tính thể tích $V$ của khối cầu ngoại tiếp hình chóp đã cho. 
	\choice
	{\True $V=\dfrac{32\pi}{3}$}
	{$V=\dfrac{20\pi}{3}$}
	{$V=\dfrac{16\pi}{3}$}
	{$V=\dfrac{10\pi}{3}$}
	\loigiai{
		\immini{
		Gọi $E$ là trung điểm $AB$. 
		Dễ thấy $SE \perp (ABCD)$.\\
		Dựng trục $d$ của $ (ABCD) $qua $O$ và song song với $SE$.\\
		Gọi $G$ là trọng tâm tam giác $ABC$. Đường thẳng đi qua $G$ vuông góc với mặt phẳng $(ABC)$ cắt $d$ tại $I$. $I$ là tâm mặt cầu ngoại tiếp hình chóp $S.ABCD$.\\
		Ta có $ \triangle SEB $ đều cạnh $ AB=2 \Rightarrow SE=\dfrac{3\sqrt{3}}{2}\Rightarrow SG=\dfrac{2}{3}SE$=$\sqrt{3}$.\\
		$GI=EO=\dfrac{1}{2}AD=1$.\\
		$R=SI=\sqrt{SG^2+GI^2}=\sqrt{4}=2$.\\
		Suy ra thể tích khối cầu ngoại tiếp là\\
		$V=\dfrac{4}{3}\pi R^3=\dfrac{4}{3}\pi \cdot 8=\dfrac{32\pi }{3}$.	
		}{
		\begin{tikzpicture}[line join=round,line width=.6pt,font=\small,scale=1.5]
		\def\gvg[size=#1](#2,#3,#4){\draw[red,very thin]($(#3)!#1!(#2)$)--($($(#3)!#1!(#2)$)+($(#3)!#1!(#4)$)-(#3)$)--($(#3)!#1!(#4)$);}
		\coordinate (B) at (0,0);
		\coordinate (A) at (1,1);
		\coordinate (C) at (3,0);
		\coordinate (D) at ($(A)+(C)$);
		\coordinate (E) at ($(A)!.5!(B)$);
		\coordinate (S) at ($(E)+(0,3)$);
		\coordinate (F) at ($(S)!.5!(B)$);
		\coordinate(G) at (intersection of A--F and S--E);
		\coordinate (O) at ($(A)!.5!(C)$);
		\coordinate (I) at ($(G)-(E)+(O)$);
		\draw (S)--(B)--(C)--(D)--(S)--(C) (I)--($(I)!-2!(O)$) node[pos=.8,right]{$d$};
		\draw[dashed] (E)--(S)--(A)--(B)--(I)--(G) (B)--(D)--(A)--(C) (A)--(F) (I)--($(I)!1.4!(O)$);
		\foreach \p/\g in {A/180,B/-135,C/-45,D/0,E/170,F/160,G/-135,I/0,O/-125,S/90}\draw[fill=gray](\p)circle (1pt)node[shift={(\g:.25)}]{$\p$};
		\gvg[size=5pt](S,E,A);
		\end{tikzpicture}
		}}
\end{ex}
\begin{ex}%[Phạm Thế Sinh]%[2H2G2-2]
	Cho hình chóp $S.ABC$ có đáy $ABC$ là tam giác cân tại $A$, mặt bên $(SBC)$ vuông góc với mặt phẳng $(ABC)$ và $SA=SB=AB=AC=a$; $SC=a\sqrt{2}$. Diện tích mặt cầu ngoại tiếp hình chóp $S.ABC$ bằng 
	\choice
	{$2\pi a^2$}
	{$\pi a^2$}
	{$8\pi a^2$}
	{\True $4\pi a^2$}
	\loigiai{
		\immini{
		Gọi $D$ là trung điểm $BC$. Đặt $BC=x\;(x>0)$.\\
		Kẻ $SH \perp BC,\;(H\in BC) \Rightarrow SH\perp (ABC)$.\\
		Mà $SA=SB\Rightarrow HA=HB$.\\
		Gọi $E$ là trung điểm $AB$.\\
		Ta có $\triangle BHE$ đồng dạng $\triangle BAD$, suy ra $\dfrac{BH}{BA}=\dfrac{BE}{BD}\\
		\Rightarrow BH=\dfrac{BA\cdot BE}{BD}=\dfrac{a^2}{x} \Rightarrow CH=x-\dfrac{a^2}{x}$.\\
		Trong tam giác vuông $SBH$ có: $SH^2=SB^2-HB^2=a^2-\dfrac{a^4}{x^2}$.\\
		Trong tam giác vuông $SHC$ có: $SC^2=SH^2+HC^2\\
		\Rightarrow 2a^2=a^2-\dfrac{a^4}{x^2}+\left(x-\dfrac{a^2}{x}\right)^2\Rightarrow x=a\sqrt{3}$.\\
		Do $SB=a;\;SC=a\sqrt{2};\;BC=a\sqrt{3} \Rightarrow \triangle SBC$ vuông tại S.\\
		Mặt khác $\heva{&AD\perp BC\\&AD\perp SH}\Rightarrow AD\perp (SBC)$.
		}{
		\begin{tikzpicture}[line join=round,line width=.6pt,font=\small,scale=1.2]
		\def\gvg[size=#1](#2,#3,#4){\draw[red,very thin]($(#3)!#1!(#2)$)--($($(#3)!#1!(#2)$)+($(#3)!#1!(#4)$)-(#3)$)--($(#3)!#1!(#4)$);}
		\coordinate (A) at (0,0);
		\coordinate (B) at (1,-1.8);
		\coordinate (C) at (4.2,0);
		\coordinate (S) at (2,3);
		\coordinate (D) at ($(B)!.5!(C)$);
		\coordinate (E) at ($ (A)!.5!(B)$);
		\coordinate(H) at (intersection of B--C and S--{(2,-4)});
		\draw (H)--(S)--(B)--(A)--(S)--(C)--(B);
		\draw[dashed] (C)--(A)--(H)--(E) (A)--(D);
		\foreach \p/\g in {A/180,B/-100,C/0,D/-30,E/170,H/-60,S/90}\draw[fill=gray](\p)circle (1pt)node[shift={(\g:.25)}]{$\p$};
		\gvg[size=5pt](A,D,B);\gvg[size=5pt](B,E,H);
		\tkzMarkSegments[mark=|](H,A H,B)
		\tkzMarkSegments[mark=||](A,E A,C A,S B,S)
		\end{tikzpicture}}
		\noindent Suy ra $AD$ là trục đường tròn ngoại tiếp tam giác $SBC$.\\
		Gọi $I$ là tâm đường tròn ngoại tiếp tam giác $ABC$, vì $ I \in AD \Rightarrow IA=IB=IC=IS$. Do đó $I$ là tâm mặt cầu ngoại tiếp hình chóp $S.ABC$.\\
		Ta có $AD=\sqrt{a^2-\left(\dfrac{a\sqrt{3}}{2}\right)^2}=\dfrac{a}{2}$, suy ra $S_{\triangle ABC}=\dfrac{1}{2}AD\cdot BC=\dfrac{1}{2}\cdot \dfrac{a}{2}\cdot a\sqrt{3}=\dfrac{a^2\sqrt{3}}{4}$.\\
		Suy ra $IA=\dfrac{AB\cdot BC\cdot AC}{4\cdot S_{\triangle ABC}}=\dfrac{a\cdot a\cdot a\sqrt{3}}{a^2\sqrt{3}}=a$.\\
		Vậy diện tích mặt cầu là $S_{mc}=4\pi \cdot IA^2=4\pi a^2$.
		\immini{
		\textbf{Cách khác:}\\
		Do $AS=AB=AC$ nên $A$ thuộc trục đường tròn ngoại tiếp tam giác $SBC$.\\
		Do $(ABC)\perp (SBC)$ nên hạ $AH\perp BC$ thì $AH\perp (SBC)$.\\
		Vậy $AH$ là trục đường tròn ngoại tiếp đáy $(SBC)$, nên $H$ là tâm đường tròn ngoại tiếp tam giác $SBC$.\\
		Suy ra $H$ là trung điểm $BC$ và $\triangle SBC$ vuông tại $S$,\\
		suy ra $BC=a\sqrt{3}$ và $AH=\dfrac{a}{2}$.\\
		Kẻ trung trực $MI$ của đoạn $AB$ thì $I$ chính là tâm mặt cầu ngoại tiếp $SABC$ và bán kính của nó bằng $R=\dfrac{AB^2}{2AH}=a$.\\
		Vậy diện tích mặt cầu là $S_{mc}=4\pi \cdot IA^2=4\pi a^2$.	
		}{
		\begin{tikzpicture}[line join=round,line width=.6pt,font=\small,scale=1]
		\def\gvg[size=#1](#2,#3,#4){\draw[red,very thin]($(#3)!#1!(#2)$)--($($(#3)!#1!(#2)$)+($(#3)!#1!(#4)$)-(#3)$)--($(#3)!#1!(#4)$);}
		\coordinate (S) at (0,0);
		\coordinate (B) at (5,0);
		\coordinate (C) at (1,-2);
		\coordinate (H) at ($(B)!.5!(C)$);
		\coordinate (A) at ($(H)+(0,3)$);
		\coordinate (M) at ($ (A)!.5!(B)$);
		\coordinate(I) at ($ (A)!2!(H)$);
		\draw (M)--(I)--(A)--(C)--(B)node[pos=.75,right]{$a\sqrt{3}$}--(A)node[pos=.75,right]{$a$}--(S)--(C)node[pos=.5,left]{$a\sqrt{2}$};
		\draw[dashed] (S)--(B);
		\foreach \p/\g in {A/90,B/0,C/-100,H/-45,I/170,M/60,S/180}\draw[fill=gray](\p)circle (1.2pt)node[shift={(\g:.25)}]{$\p$};
		\gvg[size=5pt](B,S,C);\gvg[size=5pt](A,H,B);\gvg[size=5pt](I,M,B);
		\tkzMarkSegments[mark=|](M,A M,B)
		\end{tikzpicture}	
		}}
\end{ex}
\begin{ex}%[Phạm Thế Sinh]%[2H2B2-2]
	Cho hình chóp $S.ABCD$ có đáy $ABCD$ là hình vuông cạnh $a$, tam giác $SAB$ đều và nằm trong mặt phẳng vuông góc với đáy. Tính thể tích khối cầu ngoại tiếp khối chóp $S.ABCD$. 
	\choice
	{\True $\dfrac{7\sqrt{21}}{54}\pi a^3$}
	{$\dfrac{7\sqrt{21}}{162}\pi a^3$}
	{$\dfrac{7\sqrt{21}}{216}\pi a^3$}
	{$\dfrac{49\sqrt{21}}{36}\pi a^3$}
	\loigiai{
		\immini{
		Gọi $H$ là trung điểm của $AB$, suy ra $AH \perp (ABCD)$.\\
		Gọi $G$ là trọng tâm tam giác $\triangle SAB$ và $O$ là tâm hình vuông $ABCD$.\\
		Từ $G$ kẻ $GI \parallel HO$ suy ra $GI$ là trục đường tròn ngoại tiếp tam giác $\triangle SAB$ và từ $O$ kẻ $OI \parallel SH$ thì $OI$ là trục đường tròn ngoại tiếp hình vuông $ABCD$.\\
		Ta có hai đường này cùng nằm trong mặt phẳng và cắt nhau tại $I$.\\
		Suy ra $I$ là tâm mặt cầu ngoại tiếp hình chóp $S.ABCD$.\\[3pt]
		$R=SI=\sqrt{SG^2+GI^2}=\dfrac{a\sqrt{21}}{6}$.	
		}{
		\begin{tikzpicture}[line join=round,line width=.6pt,font=\small,scale=1.5]
		\def\gvg[size=#1](#2,#3,#4){\draw[red,very thin]($(#3)!#1!(#2)$)--($($(#3)!#1!(#2)$)+($(#3)!#1!(#4)$)-(#3)$)--($(#3)!#1!(#4)$);}
		\coordinate (B) at (0,0);
		\coordinate (A) at (1,1);
		\coordinate (C) at (3,0);
		\coordinate (D) at ($(A)+(C)$);
		\coordinate (H) at ($(A)!.5!(B)$);
		\coordinate (S) at ($(H)+(0,2.5)$);
		\coordinate (E) at ($(S)!.5!(B)$);
		\coordinate (G) at (intersection of S--H and A--E);
		\coordinate (O) at ($(A)!.5!(C)$);
		\coordinate (I) at ($(G)-(H)+(O)$);
		\coordinate (K) at ($(C)!.5!(D)$);
		\draw (C)--(S)--(B)--(C)--(D)--(S)--(K);
		\draw[dashed] (K)--(H)--(S)--(A)--(C) (B)--(A)--(D) (G)--(I)--(O);
		\foreach \p/\g in {A/180,B/-135,C/-45,D/0,G/170,H/160,I/-10,K/-10,O/-100,S/90}\draw[fill=gray](\p)circle (1pt)node[shift={(\g:.25)}]{$\p$};
		\gvg[size=4pt](S,G,I);\gvg[size=4pt](S,H,O);\gvg[size=4pt](S,H,B);\gvg[size=4pt](I,O,H);
		\end{tikzpicture}	
		}
		\vspace{-10pt}\noindent Suy ra thể tích khối cầu ngoại tiếp khối chóp $S.ABCD$ là $V=\dfrac{4}{3}\pi R^3=\dfrac{7\sqrt{21}}{54}\pi a^3$.
	}
\end{ex}
\begin{ex}%[Phạm Thế Sinh]%[2H2K2-2]
	Cho hình chóp $S.ABC$ có đáy là tam giác $ABC$ vuông tại $B,AB=3,BC=4$. Hai mặt phẳng $(SAB),(SAC)$ cùng vuông góc với mặt phẳng đáy, đường thẳng $SC$ hợp với mặt phẳng đáy một góc $45^\circ$. Thể tích mặt cầu ngoại tiếp hình chóp $S.ABC$ là 
	\choice
	{$V=\dfrac{5\pi\sqrt{2}}{3}$}
	{$V=\dfrac{25\pi\sqrt{2}}{3}$}
	{$V=\dfrac{125\pi\sqrt{3}}{3}$}
	{\True $V=\dfrac{125\pi\sqrt{2}}{3}$}
	\loigiai{
		\immini{
		Hai mặt phẳng $(SAB),(SAC)$ cùng vuông góc với mặt phẳng đáy nên $SA\perp (ABC)$.\\
		Ta cũng có $\heva{&BC\perp AB\\&BC\perp SA}\Rightarrow BC\perp SB$.\\
		Suy ra $SAC$ và $SBC$ là hai tam giác vuông tại $A$ và $B$.\\
		Gọi $I$ là trung điểm của $SC$ thì $\heva{&IA=IC=IS\\&IB=IC=IS}\\
		\Rightarrow IA=IB=IC=IS\Rightarrow I$ là tâm mặt cầu $(S)$ ngoại tiếp hình chóp $S.ABC$.\\
		Vì $SA\perp (ABC)$ nên $(SC,(ABC))=\widehat{SCA}=45^\circ$.\\
		Ta lần lượt tính được: $AC=\sqrt{AB^2+BC^2}=5;\;SA=AC=5;\;SC=AC\sqrt{2}=5\sqrt{2}$.\\
		Suy ra bán kính mặt cầu $(S)$ là $R=\dfrac{SC}{2}=\dfrac{5\sqrt{2}}{2}$.\\
		Vậy thể tích khối cầu $(S)$ là $V=\dfrac{4}{3}\cdot\pi\cdot \left(\dfrac{5\sqrt{2}}{2}\right)^3=\dfrac{125\pi\sqrt{2}}{3}$.	
		}{
		\begin{tikzpicture}[line join=round,line width=.6pt,font=\small,scale=1.2]
		\def\gvg[size=#1](#2,#3,#4){\draw[red,very thin]($(#3)!#1!(#2)$)--($($(#3)!#1!(#2)$)+($(#3)!#1!(#4)$)-(#3)$)--($(#3)!#1!(#4)$);}
		\coordinate (A) at (0,0);
		\coordinate (B) at (1.5,-1);
		\coordinate (C) at (4,0);
		\coordinate (S) at (0,3);
		\coordinate (I) at ($(S)!.5!(C)$);
		\draw (I)--(B)--(S)--(A) (A)--(B)--(C)--(S);
		\draw[dashed] (I)--(A)--(C);
		\foreach \p/\g in {A/180,B/-90,C/0,I/45,S/90}\draw[fill=gray](\p)circle (1pt)node[shift={(\g:.25)}]{$\p$};
		\gvg[size=4pt](S,A,C);
		\end{tikzpicture}
		}}
\end{ex}
\begin{ex}%[Phạm Thế Sinh]%[2H2G2-5]
	Trong tất cả các hình chóp tứ giác đều nội tiếp hình cầu có bán kính bằng $9$. Tính thể tích $V$ của khối chóp có thể tích lớn nhất. 
	\choice
	{$576\sqrt{2}$}
	{\True $576$}
	{$144\sqrt{2}$}
	{$144$}
	\loigiai{
		\immini{
		Gọi $(S)$ là mặt cầu có tâm $I$ và bán kính $R=9$.\\
		Xét hình chóp tứ giác đều $S.ABCD$ có đáy $ABCD$ là hình vuông tâm $O$, cạnh $a$, $(0<a\leq 9\sqrt{2})$.\\
		Ta có $OA=\dfrac{AC}{2}=\dfrac{a\sqrt{2}}{2}\Rightarrow OI=\sqrt{IA^2-OA^2}=\sqrt{81-\dfrac{a^2}{2}}$.\\
		Mặt khác ta lại có $SO=SI+IO=9+\sqrt{81-\dfrac{a^2}{2}}$.\\
		Thể tích của khối chóp $S.ABCD$ là\\ $V=\dfrac{1}{3}a^2\left(9+\sqrt{81-\dfrac{a^2}{2}}\right)=3a^2+\dfrac{1}{3}a^2\sqrt{81-\dfrac{a^2}{2}}$.\\
		Đặt $a^2=t$, do $0<a\leq 9\sqrt{2}$ nên $0<t\leq 162$.\\
		Xét hàm số $f(t)=3t+\dfrac{1}{3}t\left(9+\sqrt{81-\dfrac{t}{2}}\right)$, với $0<t\leq 162$. 	
		}{
		\begin{tikzpicture}[line join=round,line width=.6pt,font=\small,scale=1.2]
		\def\r{2.5}\def\alpha{25}
		\pgfmathsetmacro{\a}{\r*cos(\alpha)}
		\pgfmathsetmacro{\b}{\a/4}
		\coordinate (I) at (0,0);
		\coordinate (A) at (180+\alpha:\r);
		\coordinate (C) at (-\alpha:\r);
		\coordinate (S) at (90:\r);
		\coordinate (O) at ($(A)!.5!(C)$);
		\draw (A) arc (-180:0:{\a} and \b) coordinate[pos=0.35] (B);
		\draw[dashed] (A) arc (180:0:{\a} and \b) coordinate[pos=0.65] (D);
		\draw (0,0) circle (\r); 
		\draw[dashed] (S)--(O) (I)--(A)--(B)--(C)--(D)--(B) (D)--(A)--(C) (A)--(S)--(C) (B)--(S)--(D);
		\draw ($(I) + (-\alpha:\r)$) arc(0:-180:{\a} and {\b});
		\foreach \p/\g in {A/180,B/-120,C/0,D/45,I/0,O/-90,S/90}\draw[fill=gray](\p)circle (1pt)node[shift={(\g:.25)}]{$\p$};
		\end{tikzpicture}
		}
		\noindent Ta có $f'(t)=3+\dfrac{324-3t}{12\sqrt{81-\dfrac{t}{2}}}$;\\
		$f'(t)=0\Leftrightarrow 
		\sqrt{81-\dfrac{t}{2}}=\dfrac{t}{12}-9
		\Leftrightarrow \heva{&t\geq 108\\&81-\dfrac{t}{2}=\left(\dfrac{t}{12}-9\right)^2}\Leftrightarrow \heva{&t\geq 108\\&\hoac{&t=0\\&t=144}}\Leftrightarrow t=144$.\\
		Ta có bảng biến thiên.	
		\begin{center}
			\begin{tikzpicture}[yscale=.5,xscale=1.5]
			\begin{scope}[shift={(-.5,.5)}]
			\draw (0,0) rectangle +(5.9,-5) (0,-1)--+(0:5.9) (0,-2)--+(0:5.9) (1,0)--+(-90:5);
			\end{scope}
			\path(0,0) node{$t$} ++(0:1) node[left]{$0$}++(0:2) node{$144$}++(0:2) node{$162$} % <<< dòng 1
			(0,-1) node{$f'(t)$} ++(0:2) node{$+$} ++(1,0) node{$0$} ++(1,0) node{$-$} ++(0:1) % <<< dòng 2
			(0,-3) node{$f(t)$} ++(1,-.9) node[left] (A) {} ++(2,1.7)node (B) {$576$} ++(2,-1.8) node (C) {}; % <<< dòng 3
			\draw[-stealth] (A)--(B);
			\draw[-stealth] (B)--(C);
			\end{tikzpicture}
		\end{center}		
		Từ bảng biến thiên ta có $V_{max}=576$ khi $t=144$ hay $a=12$.}
\end{ex}
\begin{ex}%[Phạm Thế Sinh]%[2H2K2-2]
	Cho hình chóp $S.ABCD$ có đáy $ABCD$ là hình thoi cạnh $a$, $\widehat{ABC}=60^\circ$. Mặt bên $SAB$ là tam giác đều và nằm trong mặt phẳng vuông góc với mặt phẳng đáy. Tính diện tích $S$ của mặt cầu ngoại tiếp hình chóp $S.ABCD$. 
	\choice
	{$S=\dfrac{13\pi a^2}{12}$}
	{\True $S=\dfrac{5\pi a^2}{3}$}
	{$S=\dfrac{13\pi a^2}{36}$}
	{$S=\dfrac{5\pi a^2}{9}$}
	\loigiai{
		\immini{
		Gọi $H$ là trung điểm của cạnh $AB$.\\
		Vì $\triangle SAB$ là tam giác đều và nằm trong mặt phẳng vuông góc với mặt phẳng đáy nên $SH\perp (ABCD)$.\\
		Gọi $O, G$ lần lượt là tâm đường tròn ngoại tiếp các tam giác $ABC$ và $SAB$.\\
		Ta có $\heva{&CH\perp AB\\&CH\perp SH}\Rightarrow CH\perp (SAB)$.\\
		Từ $O$ kẻ đường thẳng $\Delta _1 \perp (ABC)\Rightarrow \Delta _1\parallel SH$.\\
		Trong mặt phẳng $(\Delta _1;SH)$ từ G kẻ đường thẳng $\Delta _2\parallel CH$ và $\Delta _2 \cap \Delta _1 =I$.\\
		Do $\Delta _2 \parallel CH\Rightarrow \Delta _2\perp (SAB)$.
		}{
		\begin{tikzpicture}[line join=round,line width=.6pt,font=\small,scale=1.4]
		\coordinate (B) at (0,0);
		\coordinate (A) at (1,1);
		\coordinate (C) at (3,0);
		\coordinate (D) at ($(A)+(C)$);
		\coordinate (H) at ($(A)!.5!(B)$);
		\coordinate (S) at ($(H)+(0,2.5)$);
		\coordinate (G) at ($(H)!1/3!(S)$);
		\coordinate (E) at ($(S)!.5!(B)$);
		\coordinate (G) at (intersection of S--H and A--E);
		\coordinate (O) at ($(H)!.4!(C)$);
		\coordinate (I) at ($(G)-(H)+(O)$);
		\draw (C)--(S)--(B)--(C)--(D)--(S);
		\draw[dashed] (H)--(S)--(A)--(C)--(H) (B)--(A)--(D) (G)--(I)--(O) (S)--(I);
		\foreach \p/\g in {A/180,B/-135,C/-45,D/0,G/170,H/160,I/10,O/-100,S/90}\draw[fill=gray](\p)circle (1pt)node[shift={(\g:.25)}]{$\p$};
		\end{tikzpicture}	
		}
		\noindent Vì $I \in \Delta _1 \Rightarrow IA=IB=IC\;(1)$. Vì $I \in \Delta _2 \Rightarrow IA=IB=IS\;(2)$. Từ $(1)$, $(2)$ có $I$ là tâm của mặt cầu ngoại tiếp hình chóp $S.ABC$.\\
		Các tam giác $ABC$ và $SAB$ đều cạnh $a$ nên $SG=\dfrac{a\sqrt{3}}{3}$ và $GI=OH=\dfrac{a\sqrt{3}}{6}$.\\
		Bán kính của mặt cầu là $R=SI=\sqrt{SG^2+GI^2}=\sqrt{\dfrac{3a^2}{9}+\dfrac{3a^2}{36}}=\dfrac{a\sqrt{15}}{6}$.\\
		Do đó diện tích $S$ của mặt cầu ngoại tiếp hình chóp $S.ABC$ là $S=4\pi R^2=\dfrac{5\pi a^2}{3}$.
	}
\end{ex}
\begin{ex}%[Phạm Thế Sinh]%[2H2K2-2]
	Cho hình chóp $S.ABCD$ có đáy $ABCD$ là hình chữ nhật, $AB=\sqrt{3}a,AD=a,\triangle SAB$ là tam giác đều và nằm trong mặt phẳng vuông góc với đáy. Tính theo $a$ diện tích $S$ của mặt cầu ngoại tiếp hình chóp $S.ABCD$. 
	\choice
	{\True $S=5\pi a^2$}
	{$S=10\pi a^2$}
	{$S=4\pi a^2$}
	{$S=2\pi a^2$}
	\loigiai{
		\immini{
		Gọi $H$ là trung điểm $AB\Rightarrow SH\perp AB$ (vì $\triangle SAB$ đều).\\
		Mặt khác $(SAB)\perp (ABCD)\Rightarrow SH \perp (ABCD)$.\\
		Gọi $O$ là giao điểm của $AC,BD \Rightarrow O$ là tâm đường tròn ngoại tiếp hình chữ nhật $ABCD$.\\
		Gọi $G$ là trọng tâm $\triangle SBC\Rightarrow G$ là tâm đường tròn ngoại tiếp tam giác đều $SBC$.\\
		Qua $O$ dựng đường thẳng $d\parallel SH\Rightarrow d$ là trục của đường tròn $(O)$, qua $ G $ dựng đường thẳng $\Delta \parallel OH\Rightarrow \Delta$ là trục của đường tròn $(H)$.\\
		$d\cap \Delta =I\Rightarrow IA=IB=IC=ID=IS\Rightarrow I$ là tâm của mặt cầu ngoại tiếp hình chóp $S.ABCD$.\\
		}{
		\begin{tikzpicture}[line join=round,line width=.6pt,font=\small,scale=1.5]
		\coordinate (B) at (0,0);
		\coordinate (A) at (1,1);
		\coordinate (C) at (2.5,0);
		\coordinate (D) at ($(A)+(C)$);
		\coordinate (H) at ($(A)!.5!(B)$);
		\coordinate (S) at ($(H)+(0,2.5)$);
		\coordinate (G) at ($(H)!1/3!(S)$);
		\coordinate (O) at ($(A)!.5!(C)$);
		\coordinate (I) at ($(G)-(H)+(O)$);
		\draw (S)--(B)--(C)--(D)--(S)--(C);
		\draw[dashed] (O)--(H)--(S)--(A)--(B) (B)--(D)--(A)--(C)--(H) (G)--(I)--(O);
		\foreach \p/\g in {A/180,B/-135,C/-45,D/0,H/170,G/-160,I/0,O/-100,S/90}\draw[fill=gray](\p)circle (1pt)node[shift={(\g:.25)}]{$\p$};
		\end{tikzpicture}	
		}
		\noindent Xét tam giác đều $SAB$ có cạnh là $a\sqrt{3}\Rightarrow SH=\dfrac{3a}{2}\Rightarrow SG=a$.\\
		Mặt khác $IG=OH=\dfrac{AD}{2}=\dfrac{a}{2}$.\\
		Xét tam giác vuông $SIG$: $IS^2=SG^2+IG^2=a^2+\dfrac{a^2}{4}=\dfrac{5a^2}{4}\Rightarrow IS=\dfrac{a\sqrt{5}}{2}$.\\
		Vậy diện tích mặt cầu ngoại tiếp hình chóp $S.ABCD$ là $S=4\pi R^2=5\pi a^2$.
	}
\end{ex}
\begin{ex}%[Phạm Thế Sinh]%[2H2K2-2]
	Cho hình chóp $S.ABC$ có đáy $ABC$ là tam giác đều cạnh bằng $1$, mặt bên $SAB$ là tam giác cân tại $S$ và nằm trong mặt phẳng vuông góc với mặt phẳng đáy. Tính thể tích $V$ của khối cầu ngoại tiếp hình chóp đã cho biết $\widehat{ASB}=120^\circ$. 
	\choice
	{\True $V=\dfrac{5\sqrt{15}\pi}{54}$}
	{$V=\dfrac{4\sqrt{3}\pi}{27}$}
	{$V=\dfrac{5\pi}{3}$}
	{$V=\dfrac{13\sqrt{78}\pi}{27}$}
	\loigiai{
		\immini{
		Gọi $H$ là trung điểm $AB$, do $(SAB)\perp (ABC)$, tam giác $ABC$ đều và tam giác $SAB$ cân tại $S$ nên $SH\perp (ABC)$ và $CH\perp (SAB)$.\\
		Gọi $I$ và $J$ là tâm đường tròn ngoại tiếp các tam giác $ABC$ và tam giác $SAB$.\\
		Dựng đường thẳng $Ix\parallel SH$ và $Jy\parallel CH$ thì $Ix\perp (ABC)$ và $Jy\perp (SAB)$ nên $Ix$ là trục của đường tròn ngoại tiếp tam giác $ABC$ và $Jy$ là trục đường tròn ngoại tiếp tam giác $SAB$. Khi đó $Ix\cap Jy=O$ thì $O$ là tâm mặt cầu ngoại tiếp hình chóp.\\
		Ta có 
		\begin{eqnarray*}
			OJ&=&IH=\dfrac{\sqrt{3}}{6}\cdot R_{(SAB)}=SJ\\
			&=&\dfrac{SA\cdot SB\cdot AB}{4\cdot\dfrac{1}{2}\cdot SA\cdot SB\cdot \sin 120^\circ}=\dfrac{AB}{\sqrt{3}}=\dfrac{\sqrt{3}}{3}. 
		\end{eqnarray*}	
		}{
		\begin{tikzpicture}[line join=round,line width=.6pt,font=\small,scale=1]
		\coordinate (A) at (0,0);
		\coordinate (B) at (2,-2);
		\coordinate (C) at (5,0);
		\coordinate (H) at ($(A)!.5!(B)$);
		\coordinate (S) at ($(H)+(0,5)$);
		\coordinate (M) at ($ (B)!.5!(C)$);
		\coordinate (I) at (intersection of A--M and C--H);
		\coordinate (J) at ($(H)-(0,2)$);
		\coordinate (O) at ($(I)-(H)+(J)$);
		\coordinate (P) at (intersection of O--S and B--C);
		\coordinate (Q) at (intersection of O--I and B--C);
		\draw (Q)--(O)--(J)--(S)--(B) (O)--(P)--(B)--(A)--(S)--(C)--(P);
		\draw[dashed] (M)--(A)--(C)--(H) (S)--(P) (I)--(Q);
		\foreach \p/\g in {A/180,B/-100,C/0,H/-160,I/90,J/-100,M/-30,O/-60,S/90}\draw[fill=gray](\p)circle (1pt)node[shift={(\g:.25)}]{$\p$};
		\end{tikzpicture}
		}
		\noindent Vậy $R=SO=\sqrt{\dfrac{1}{3}+\dfrac{1}{12}}=\dfrac{\sqrt{15}}{6}$ nên $V=\dfrac{4}{3}\pi R^3=\dfrac{4}{3}\pi\left(\dfrac{\sqrt{15}}{6}\right)^3=\dfrac{5\sqrt{15}\pi}{54}$.
	}
\end{ex}
\begin{ex}%[Phạm Thế Sinh]%[2H2K2-2]
	Cho hình chóp $S.ABC$ có đáy $ABC$ là tam giác vuông tại $B$, $BC=2a$. Mặt bên $(SAB)$ vuông góc với đáy, $\widehat{ASB}=60^\circ$, $SB=a$. Gọi $(S)$ là mặt cầu tâm $B$ và tiếp xúc với $(SAC)$. Tính bán kính $r$ của mặt cầu $(S)$. 
	\choice
	{$r=2a$}
	{\True $r=2a\sqrt{\dfrac{3}{19}}$}
	{$r=2a\sqrt{3}$}
	{$r=a\sqrt{\dfrac{3}{19}}$}
	\loigiai{
		\immini{
		Ta có $(SAB)\perp (ABC)$, $(SAB)\cap (ABC)=AB$, $BC\perp AB$ \\ 
		$\Rightarrow BC\perp (SAB)$.\\
		Vẽ $BM\perp SA$ tại $M \Rightarrow SA\perp (BMC)\Rightarrow (SAC)\perp (BMC)$, \\vẽ $BH\perp MC$ tại $H \Rightarrow BH\perp (SAC)\Rightarrow r=BH$.\\
		Ta có $BM=\sin 60^\circ\cdot SB\Rightarrow BM=\dfrac{a\sqrt{3}}{2}$, \\
		$BH=\dfrac{BC\cdot BM}{\sqrt{BC^2+BM^2}}=\dfrac{2a\cdot\dfrac{a\sqrt{3}}{2}}{\sqrt{4a^2+\dfrac{3a^2}{4}}}=2a\sqrt{\dfrac{3}{19}}$.\\
		Vậy bán kính của mặt cầu $(S)$ bằng $2a\sqrt{\dfrac{3}{19}}$.	
		}{
		\begin{tikzpicture}[line join=round,line width=.6pt,font=\small,scale=1.2]
		\def\gvg[size=#1](#2,#3,#4){\draw[red,very thin]($(#3)!#1!(#2)$)--($($(#3)!#1!(#2)$)+($(#3)!#1!(#4)$)-(#3)$)--($(#3)!#1!(#4)$);}
		\coordinate (B) at (0,0);
		\coordinate (A) at (1.5,-1.5);
		\coordinate (S) at (4,0);
		\coordinate (C) at (0,3);
		\coordinate (M) at ($(A)!.3!(S)$);
		\coordinate (H) at ($(C)!.6!(M)$);
		\draw (M)--(C)--(A)--(B)--(C)node[left,pos=.5]{$2a$}--(S)--(A);
		\draw[dashed] (B)--(S)node[above,pos=.5]{$a$} (H)--(B)--(M);
		\foreach \p/\g in {A/-90,B/180,C/120,H/45,M/-45,S/0}\draw[fill=gray](\p)circle (1pt)node[shift={(\g:.25)}]{$\p$};
		\gvg[size=4pt](C,B,S);
		\end{tikzpicture}	
		}}
\end{ex}
\begin{ex}%[Phạm Thế Sinh]%[2H2K2-2]
	Cho hình chóp $S.ABCD$ có đáy $ABCD$ là hình vuông cạnh $a$. Mặt bên $SAB$ là tam giác đều và nằm trong mặt phẳng vuông góc với mặt đáy. Bán kính của mặt cầu ngoại tiếp hình chóp $S.ABCD$ bằng
	\choice
	{\True $\dfrac{a\sqrt{21}}{6}$}
	{$\dfrac{a\sqrt{11}}{6}$}
	{$\dfrac{a\sqrt{3}}{6}$}
	{$\dfrac{a\sqrt{7}}{3}$}
	\loigiai{
		\immini{
		\textbf{Cách $1$:} Đây là mặt cầu ngoại tiếp hình chóp có mặt bên vuông góc với đáy nên:
		\[R_{mc}=\sqrt{R_1^2+R_2^2-\dfrac{AB^2}{4}}=\sqrt{\left(\dfrac{a\sqrt{2}}{2}\right)^2+\left(\dfrac{a}{\sqrt{3}}\right)^2-\dfrac{a^2}{4}}=\dfrac{a\sqrt{21}}{6}.\]
		(Với $R_1$ là bán kính đường tròn ngoại tiếp hình vuông $ABCD$ cạnh $a$, $R_2$ là bán kính đường tròn ngoại tiếp tam giác $SAB$ đều cạnh $a$ và $AB=(ABCD)\cap (SAB)$.
		}{
		\begin{tikzpicture}[line join=round,line width=.6pt,font=\small,scale=1.5]
		\coordinate (B) at (0,0);
		\coordinate (A) at (1,1);
		\coordinate (C) at (2.5,0);
		\coordinate (D) at ($(A)+(C)$);
		\coordinate (H) at ($(A)!.5!(B)$);
		\coordinate (S) at ($(H)+(0,2.5)$);
		\coordinate (E) at ($(S)!.5!(B)$);
		\coordinate (G) at (intersection of S--H and A--E);
		\coordinate (O) at ($(A)!.5!(C)$);
		\coordinate (I) at ($(G)-(H)+(O)$);
		\coordinate (F) at ($(S)!.5!(B)$);
		\draw (S)--(B)--(C)--(D)--(S)--(C) (I)--($(I)!-2!(O)$) node[pos=.8,right]{$x$};
		\draw[dashed] (O)--(H)--(S)--(A)--(B)--(I)--(G) (B)--(D)--(A)--(C) (A)--(F) (I)--($(I)!1.4!(O)$);
		\foreach \p/\g in {A/180,B/-135,C/-45,D/0,H/170,F/160,G/-135,I/0,O/-125,S/90}\draw[fill=gray](\p)circle (1pt)node[shift={(\g:.25)}]{$\p$};
		\end{tikzpicture}
		}
		\noindent\textbf{Cách $2$:} 
		Gọi $H$ là trung điểm $AB$, mặt bên $SAB$ là tam giác đều và nằm trong mặt phẳng vuông góc với mặt đáy nên $SH\perp (ABCD)$.\\
		Gọi $O=AC\cap BD$, dựng $Ox \perp (ABCD)\Rightarrow Ox\parallel SH$.\\
		Gọi $G$ là trọng tâm tam giác $SAB$. Dựng $Gy\perp (SAB)\Rightarrow Gy\parallel HO$, $Gy\cap Ox=I$.\\
		Ta có $\heva{&I \in Ox\Rightarrow IA=IB=IC\\&I \in Gy\Rightarrow IS=IA=IB}$ suy ra $I$ là tâm mặt cầu ngoại tiếp hình chóp $S.ABCD$.\\
		Bán kính mặt cầu là $SI=\sqrt{SG^2+GI^2}=\sqrt{SG^2+HO^2}=\sqrt{\left(\dfrac{2}{3}\cdot\dfrac{a\sqrt{3}}{2}\right)^2+\left(\dfrac{a}{2}\right)^2}=\dfrac{a\sqrt{21}}{6}$.	
		}
\end{ex}
\begin{ex}%[Phạm Thế Sinh]%[2H2G2-6]
	Cho khối chóp $S.ABCD$ có đáy là hình vuông, tam giác $SAB$ đều và nằm trong mặt phẳng vuông góc với đáy. Mặt cầu ngoại tiếp khối chóp $S.ABCD$ có diện tích $84\pi$ (cm$^2$). Khoảng cách giữa hai đường thẳng $SA$ và $BD$ bằng 
	\choice
	{$\dfrac{2\sqrt{21}}{7}$ (cm)}
	{$\dfrac{3\sqrt{21}}{7}$ (cm)}
	{$\dfrac{\sqrt{21}}{7}$ (cm)}
	{\True $\dfrac{6\sqrt{21}}{7}$ (cm)}
	\loigiai{
		\immini{
		Gọi $M$ là trung điểm $ AB $ và $ G $ là tâm đường tròn ngoại tiếp tam giác đều $SAB$, $O$ là tâm của hình vuông $ABCD$.\\
		Ta có $OM \perp (SAB)$.\\
		Dựng trục của hình vuông $ABCD$ và trục tam giác $SAB$, khi đó chúng đồng phẳng và cắt nhau tại $I$ tức là $OI$, $GI$ là các trục hình vuông $ABCD$ và trục tam giác $SAB$.\\
		Bán kính mặt cầu cần tìm là $R=SI$. Ta có $4\pi R^2=84\pi $ (cm$^2$) $\Rightarrow R=\sqrt{21}$ (cm).\\
		Đặt $AB=x$ (cm).
		}{
		\begin{tikzpicture}[line join=round,line width=.6pt,font=\small,scale=1.4]
		\coordinate (B) at (0,0);
		\coordinate (A) at (1,1);
		\coordinate (C) at (2.3,0);
		\coordinate (D) at ($(A)+(C)$);
		\coordinate (M) at ($(A)!.5!(B)$);
		\coordinate (S) at ($(M)+(0,2.5)$);
		\coordinate (F) at ($(S)!.5!(B)$);
		\coordinate (G) at (intersection of S--M and A--F);
		\coordinate (O) at ($(A)!.5!(C)$);
		\coordinate (I) at ($(G)-(M)+(O)$);
		\coordinate (E) at ($(C)!2!(B)$);
		\coordinate (M') at ($(O)-(A)+(M)$);
		\coordinate (M'') at (intersection of M--M' and O--B);
		\coordinate (K) at ($(M'')!2!(M)$);
		\coordinate (J) at (intersection of S--B and A--E);
		\draw (S)--(B)--(C)--(D)--(S)--(C) (J)--(E)--(B);
		\draw[dashed] (I)--(O)--(M)--(S)--(A)--(B)--(I)--(G) (B)--(D)--(A)--(C) (A)--(J) (M'')--(K);
		\foreach \p/\g in {A/150,B/-90,C/-45,D/10,E/-135,G/-160,I/20,O/-100,K/160,M/180,S/90}\draw[fill=gray](\p)circle (1pt)node[shift={(\g:.25)}]{$\p$};
		\draw[fill=gray](M'')circle (1pt);
		\end{tikzpicture}	
		}
		\noindent Trong tam giác vuông $SGI$, ta có $SI^2=SG^2+GI^2$, tính ra $x=6$.\\
		Dựng hình bình hành $ABDE$. Khoảng cách $d$ giữa $BD$ và $SA$ là $d=d(BD,(SAE))$.\\
		$d=d(B,(SAE))=2d(M,(SAE))$.\\
		Kẻ $MK \perp AE$ ta có $(SAE) \perp (SMK)$.\\		$d(M,(SAE))=d(M,(SK))=\dfrac{SM\cdot MK}{\sqrt{SM^2+MK^2}}\;(1)$.\\
		Ta có $SM=\dfrac{x\sqrt{3}}{2}=3\sqrt{3}$, $MK=\dfrac{x\sqrt{2}}{4}=\dfrac{3\sqrt{2}}{2}$.\\
		Thay các giá trị vào $(1)$ tính được $d(M,(SAE))=\dfrac{3\sqrt{21}}{7}$.\\
		Vậy khoảng cách giữa $SA$ và $BD$ là $\dfrac{6\sqrt{21}}{7}$.
	}
\end{ex}
\Closesolutionfile{ans}
% \DAPAN
\inputansbox{10}{ans/ans2H2-2}

