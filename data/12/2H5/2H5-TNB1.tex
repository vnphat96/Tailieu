\subsection{BÀI TẬP TRẮC NGHIỆM TỰ LUYỆN}
\TN
	\setcounter{ex}{0}
	\Opensolutionfile{ans}[ans/B1-De2-1]

\begin{ex}%[2H5N1-4]
	Trong không gian với hệ tọa độ $Oxyz$, cho mặt phẳng $(P)\colon x-z+3=0$. Mặt phẳng nào sau đây vuông góc với mặt phẳng $(P)$?
	\choice
	{\True $(\alpha)\colon 2x-y+2z=0$}
	{$(\beta)\colon 2x-y-2z=0$}
	{$(Q)\colon -2x-y+2z=0$}
	{$(R)\colon 2x+y-2z=0$}
	\loigiai{
		Mặt phẳng $(P)$ có véc-tơ pháp tuyến là $\overrightarrow{n}_{(P)}=(1;0;-1)$.\\
		Mặt phẳng $(\alpha)$ có véc-tơ pháp tuyến là $\overrightarrow{n}_{(\alpha)}=(2;-1;2)$.\\
		Ta có $\overrightarrow{n}_{(P)}\cdot \overrightarrow{n}_{(\alpha)}=1\cdot 2+0\cdot (-1)+(-1)\cdot 2=0$. Do đó, $(P)\perp (\alpha)$.
	}
\end{ex}

%G:\My Drive\CODE12-2024\DE-ON-THEO BAI\2H5-TACH DE\Bai1-De2.tex
\begin{ex}%[2H5N1-5]
	Trong không gian với hệ tọa độ $Oxyz$, cho điểm $A(1;3;-2)$ và mặt phẳng $(P)\colon 2x+y-2z-3=0$. Khoảng cách từ điểm $A$ đến mặt phẳng $(P)$ bằng
	\choice
	{\True $2$}
	{$1$}
	{$\dfrac{2}{3}$}
	{$3$}
	\loigiai{
		Ta có $\mathrm{d}\big(A,(P)\big)=\dfrac{\left|2\cdot 1+3-2\cdot (-2)-3\right|}{\sqrt{2^2+1^2+(-2)^2}}=2$.
	}
\end{ex}

%G:\My Drive\CODE12-2024\DE-ON-THEO BAI\2H5-TACH DE\Bai1-De2.tex
\begin{ex}%[2H5N1-2]
	Trong không gian với hệ tọa độ $Oxyz$, cho mặt phẳng $(P)\colon x-y+3=0$. Véc-tơ nào sau đây \textbf{không phải} là véc-tơ pháp tuyến của mặt phẳng $(P)$?
	\choice
	{$\overrightarrow{a}=(3;-3;0)$}
	{$\overrightarrow{a}=(1;-1;0)$}
	{\True $\overrightarrow{a}=(1;-1;3)$}
	{$\overrightarrow{a}=(-1;1;0)$}
	\loigiai{
		Véc-tơ pháp tuyến của mặt phẳng $(P)$ là $\overrightarrow{n}=(1;-1;0)$.\\
		Ta có $\overrightarrow{a}=(-1;1;0)=-(1;-1;0)=-\overrightarrow{n}$. Vậy $\overrightarrow{a}=(-1;1;0)$ là một véc-tơ pháp tuyến của mặt phẳng $(P)$.\\
		Tương tự $\overrightarrow{a}=(3;-3;0)=3(1;-1;0)=3\overrightarrow{n}$. Vậy $\overrightarrow{a}=(3;-3;0)$ là một véc-tơ pháp tuyến của mặt phẳng $(P)$.\\
		Do véc-tơ $\overrightarrow{a}=(1;-1;3)$ không cùng phương với véc-tơ $\overrightarrow{n}=(1;-1;0)$. Nên $\overrightarrow{a}=(1;-1;3)$ không là véc-tơ pháp tuyến của mặt phẳng $(P)$.
	}
\end{ex}

%G:\My Drive\CODE12-2024\DE-ON-THEO BAI\2H5-TACH DE\Bai1-De2.tex
\begin{ex}%[2H5H1-3]
	Trong không gian với hệ tọa độ $Oxyz$ với điểm $M(-3;1;4)$ và gọi $A$, $B$, $C$ lần lượt là hình chiếu của $M$ lên các trục $Ox$, $Oy$, $Oz$. Phương trình nào dưới đây là phương trình mặt phẳng song song với mặt phẳng $(ABC)$?
	\choice
	{$4x-12y+3z-12=0$}
	{$4x-12y-3z+12=0$}
	{$4x+12y-3z-12=0$}
	{\True $4x-12y-3z-12=0$}
	\loigiai{
		Vì $A$, $B$, $C$ lần lượt là hình chiếu của $M(-3;1;4)$ các trục $Ox$, $Oy$, $Oz$ nên $A(-3;0;0)$, $B(0;1;0)$, $C(0;0;4)$.\\
		Phương trình mặt phẳng $(ABC)\colon \dfrac{x}{-3}+\dfrac{y}{1}+\dfrac{z}{4}=1$ $\Leftrightarrow$ $4x-12y-3z+12=0$.\\
		Vậy mặt phẳng $4x-12y-3z-12=0$ song song với mặt phẳng $(ABC)$.
	}
\end{ex}

%G:\My Drive\CODE12-2024\DE-ON-THEO BAI\2H5-TACH DE\Bai1-De2.tex
\begin{ex}%[2H5H1-2]
	Trong không gian với hệ tọa độ $Oxyz$, cho ba điểm $A(2;0;0)$, $B(0;-3;0)$, $C(0;0;1)$. Một véc-tơ pháp tuyến của mặt phẳng $(ABC)$ là
	\choice
	{\True $\overrightarrow{n}=(3;-2;6)$}
	{$\overrightarrow{n}=(2;-3;-1)$}
	{$\overrightarrow{n}=(2;3;1)$}
	{$\overrightarrow{n}=(2;-3;1)$}
	\loigiai{
		Phương trình mặt phẳng $(ABC)\colon \dfrac{x}{2}+\dfrac{y}{-3}+\dfrac{z}{1}=1$ $\Leftrightarrow$ $3x-2y+6z-6=0$.\\
		Vậy mặt phẳng $(ABC)$ có một véc-tơ pháp tuyến $\overrightarrow{n}=(3;-2;6)$.
	}
\end{ex}

%G:\My Drive\CODE12-2024\DE-ON-THEO BAI\2H5-TACH DE\Bai1-De2.tex
\begin{ex}%[2H2N2-2]
	Trong không gian với hệ tọa độ $Oxyz$, cho điểm $M(2024;0;-1)$. Mệnh đề nào dưới đây đúng?
	\choice
	{\True $M\in(Oxz)$}
	{$M\in Oy$}
	{$M\in(Oxy)$}
	{$M\in(Oyz)$}
	\loigiai{
		Do tung độ của điểm $M(2024;0;-1)$ bằng $0$ nên $M\in(Oxz)$.
	}
\end{ex}

%G:\My Drive\CODE12-2024\DE-ON-THEO BAI\2H5-TACH DE\Bai1-De2.tex
\begin{ex}%[2H5H1-3]
	Trong không gian với hệ tọa độ $Oxyz$, cho ba điểm $A(-2;0;0)$, $B(0;3;0)$ và $C(0;0;4)$. Mặt phẳng $(ABC)$ có phương trình là
	\choice
	{\True $\dfrac{x}{-2}+\dfrac{y}{3}+\dfrac{z}{4}=1$}
	{$\dfrac{x}{2}+\dfrac{y}{3}+\dfrac{z}{-4}=1$}
	{$\dfrac{x}{2}+\dfrac{y}{-3}+\dfrac{z}{4}=1$}
	{$\dfrac{x}{2}+\dfrac{y}{3}+\dfrac{z}{4}=1$}
	\loigiai{
		Vì ba điểm $A(-2;0;0)$, $B(0;3;0)$ và $C(0;0;4)$ lần lượt nằm trên các trục tọa độ $Ox$, $Oy$, $Oz$ và không trùng với gốc tọa độ $O$ nên mặt phẳng $(ABC)$ có phương trình là
		\begin{align*}
			\dfrac{x}{-2}+\dfrac{y}{3}+\dfrac{z}{4}=1.
		\end{align*}
	}
\end{ex}

%G:\My Drive\CODE12-2024\DE-ON-THEO BAI\2H5-TACH DE\Bai1-De2.tex
\begin{ex}%[2H5H1-5]
	Trong không gian với hệ tọa độ $Oxyz$, cho mặt phẳng $(\alpha)\colon 2x+2y-z+m=0$ ($m$ là tham số). Tìm giá trị của tham số $m$ dương để khoảng cách từ gốc tọa độ đến mặt phẳng $(\alpha)$ bằng $1$.
	\choice
	{$-6$}
	{$-3$}
	{\True $3$}
	{$6$}
	\loigiai{
		Ta có $\mathrm{d}(O,(\alpha))=\dfrac{\left| m\right|}{3}=1$ $\Leftrightarrow$ $\left| m\right|=3$ $\Leftrightarrow$ $m=\pm 3$.\\
		Do $m>0$ nên $m=3$.
	}
\end{ex}

%G:\My Drive\CODE12-2024\DE-ON-THEO BAI\2H5-TACH DE\Bai1-De2.tex
\begin{ex}%[2H2N2-2]
	Trong không gian với hệ tọa độ $Oxyz$, hình chiếu vuông góc của điểm $M(2;3;-4)$ trên mặt phẳng $(Oyz)$ có tọa độ là
	\choice
	{$(2;0;-4)$}
	{\True $(0;3;-4)$}
	{$(2;3;0)$}
	{$(0;3;0)$}
	\loigiai{
		Trong không gian với hệ tọa độ $Oxyz$, hình chiếu vuông góc của điểm $M_0(x_0;y_0;z_0)$ trên mặt phẳng $(Oyz)$ có tọa độ là $(0;y_0;z_0)$.\\
		Do đó, hình chiếu vuông góc của điểm $M(2;3;-4)$ trên mặt phẳng $(Oyz)$ có tọa độ là $(0;3;-4)$.
	}
\end{ex}

%G:\My Drive\CODE12-2024\DE-ON-THEO BAI\2H5-TACH DE\Bai1-De2.tex
\begin{ex}%[2H5H1-3]
	Trong không gian với hệ tọa độ $Oxyz$, cho hai điểm $A(1;2;-1)$, $B(-1;0;1)$ và mặt phẳng $(P)\colon x+2y-z+1=0$. Viết phương trình mặt phẳng $(Q)$ qua $A$, $B$ và vuông góc với $(P)$.
	\choice
	{$(Q)\colon 3x-y+z=0$}
	{\True $(Q)\colon x+z=0$}
	{$(Q)\colon 2x-y+3=0$}
	{$(Q)\colon -x+y+z=0$}
	\loigiai{
		Ta có $\overrightarrow{AB}=(-2;-2;2)$, $\overrightarrow{n}_P=(1;2;-1)$.\\
		$\left[\overrightarrow{AB},\overrightarrow{n}_P\right]=(-2;0;-2)$.\\
		Mặt phẳng $(Q)$ đi qua $A$, $B$ và vuông góc với $(P)$ nhận véc-tơ $\overrightarrow{n}_Q=\left[\overrightarrow{AB},\overrightarrow{n}_P\right]$ là véc-tơ pháp tuyến.\\
		Mặt phẳng $(Q)$ có phương trình là
		\begin{align*}
			-2(x-1)+0(x-2)-2(z+1)=0 \Leftrightarrow -2x-2z=0 \Leftrightarrow x+z=0.
		\end{align*}
	}
\end{ex}

%G:\My Drive\CODE12-2024\DE-ON-THEO BAI\2H5-TACH DE\Bai1-De2.tex
\begin{ex}%[2H5N1-2]
	Trong không gian với hệ tọa độ $Oxyz$, cho hai véc-tơ $\overrightarrow{u}=(1;2;3)$, $\overrightarrow{v}=(0;-1;1)$. Mặt phẳng $(\alpha)$ đi qua điểm $A(1;2;5)$ và song song với giá của hai véc-tơ $\overrightarrow{u}$ và $\overrightarrow{v}$. Véc-tơ nào dưới đây là một véc-tơ pháp tuyến của mặt phẳng $(\alpha)$?
	\choice
	{$\overrightarrow{n}_3=(-1;-1;-1)$}
	{\True $\overrightarrow{n}_2=(5;-1;-1)$}
	{$\overrightarrow{n}_1=(5;1;-1)$}
	{$\overrightarrow{n}_4=(-1;-1;5)$}
	\loigiai{
		Vì mặt phẳng $(\alpha)$ song song với giá của hai véc-tơ $\overrightarrow{u}$ và $\overrightarrow{v}$ nên có một véc-tơ pháp tuyến là $\overrightarrow{n}_2=\left[\overrightarrow{u},\overrightarrow{v}\right]=(5;-1;-1)$.
	}
\end{ex}

%G:\My Drive\CODE12-2024\DE-ON-THEO BAI\2H5-TACH DE\Bai1-De2.tex
\begin{ex}%[2H5H1-3]
	Trong không gian với hệ tọa độ $Oxyz$, cho ba điểm $A(1;-1;0)$, $B(-1;0;1)$, $C(2;1;-1)$. Phương trình mặt phẳng $(ABC)$ là
	\choice
	{\True $3x+y+5z-2=0$}
	{$x+3y+z+2=0$}
	{$3x-y+5z-2=0$}
	{$3x+y+5z+2=0$}
	\loigiai{
		Mặt phẳng $(ABC)$ đi qua $A(1;-1;0)$ và nhận $\overrightarrow{n}=\left[\overrightarrow{AB},\overrightarrow{AC}\right]=(-3;-1;-5)=-(3;1;5)$ làm véc-tơ pháp tuyến có phương trình: $3(x-1)+1(y+1)+5(z-0)=0$.\\
		Do đó, $(ABC)\colon 3x+y+5z-2=0$.
	}
\end{ex}
	\Closesolutionfile{ans}

\TNTF
	\setcounter{ex}{0}
	\Opensolutionfile{ans}[ans/B1-De2-2]
\begin{ex}%[2H5V1-5]
	Trong không gian với hệ trục tọa độ $Oxyz$, cho hai điểm $A(1;2;3)$, $B(3;4;4)$ và mặt phẳng $(\alpha)\colon 2x+y+mz-1=0$. Các mệnh đề sau đúng hay sai?
	\choiceTF
	{\True Mặt phẳng đi qua $3$ điểm là hình chiếu vuông góc của $A(1;2;3)$ lên ba trục tọa độ có phương trình là $6x+3y+2z-6=0$}
	{Điểm $A$ cách đều mặt phẳng $(\gamma)\colon 2x+y+mz-1=0$ và điểm $B$ khi $m=-2$}
	{\True Biết mặt phẳng $(\beta)\colon 4x+(n-2)y+z-3=0$ song song với mặt phẳng $(\alpha)$. Khi đó, $2m+n=5$}
	{Khi $B\in (\alpha)\colon 2x+y+mz-1=0$ thì $m=-2$}
	\loigiai{
		\begin{itemchoice}
			\itemch \textbf{Đúng}.\\
			Ta có hình chiếu vuông góc của $A(1;2;3)$ lên ba trục tọa độ lần lượt $M(1;0;0)$, $N(0;2;0)$, $P(0;0;3)$.\\
			Do đó, phương trình mặt phẳng cần tìm là $\dfrac{x}{1}+\dfrac{y}{2}+\dfrac{z}{3}=1$ $\Leftrightarrow$ $6x+3y+2z-6=0$.
			\itemch \textbf{Sai}.\\
			Ta có $\overrightarrow{AB}=(2;2;1)$ $\Rightarrow$ $AB=\sqrt{2^2+2^2+1^2}=3$. \hfill$(1)$\\
			Khoảng cách từ $A$ đến mặt phẳng $(P)$ là
			\begin{align*}
				\mathrm{d}\big(A,(P)\big)=\dfrac{\left|2\cdot 1+2+m\cdot 3-1\right|}{\sqrt{2^2+1^2+ m^2}}=\dfrac{\left|3m+3\right|}{\sqrt{5+m^2}}. \tag{2}
			\end{align*}
			Do đó, $AB=\mathrm{d}\big(A,(P)\big)$ khi và chỉ khi
			\begin{align*}
				3=\dfrac{\left|3m+3\right|}{\sqrt{5+ m^2}} \Leftrightarrow 9(5+m^2)=9(m+1)^2 \Leftrightarrow m=2.
			\end{align*}
			\itemch \textbf{Đúng}.\\
			Vì $(\alpha)\parallel (\beta)$ nên $\dfrac{2}{4}=\dfrac{1}{n-2}=\dfrac{m}{1}\ne\dfrac{-1}{-3}$ $\Leftrightarrow$ $\heva{
				& m=\dfrac{1}{2}\\
				& n=4
			}$ $\Rightarrow$ $2m+n=5$.
			\itemch \textbf{Sai}.\\
			Ta có $B\in (\alpha)\colon 2x+y+mz-1=0$ $\Leftrightarrow$ $2\cdot 3+4+m\cdot 4-1=0$ $\Leftrightarrow$ $m=-\dfrac{9}{4}$.
		\end{itemchoice}
	}
\end{ex}

%G:\My Drive\CODE12-2024\DE-ON-THEO BAI\2H5-TACH DE\Bai1-De2.tex
\begin{ex}%[2H5H1-5]
	Trong không gian với hệ tọa độ $Oxyz$, cho $A(1;2;-1)$, $B(-1;0;1)$ và mặt phẳng $(P)\colon x+2y-z+1=0$. Các mệnh đề sau đúng hay sai?
	\choiceTF
	{\True Biết điểm $M$ nằm trên tia $Ox$ mà khoảng cách từ $M$ đến mặt phẳng $(P)$ bằng $\sqrt{6}$. Khi đó, hoành độ điểm $M$ là $x_M=5$}
	{\True Mặt phẳng $(Q)$ qua $A$, $B$ và vuông góc với $(P)$ có phương trình là $x+z=0$}
	{\True Mặt phẳng $(P)$ có một véc-tơ pháp tuyến là $(1;2;-1)$}
	{Khi $m=-4$ thì mặt phẳng $(R)\colon 2x-my+3=0$ vuông góc với mặt phẳng $(P)$}
	\loigiai{
		\begin{itemchoice}
			\itemch \textbf{Đúng}.\\
			$M$ nằm trên tia $Ox$ $\Rightarrow$ $M(x;0;0)$, $x>0$.\\
			Khi đó, khoảng cách từ $M$ đến mặt phẳng $(P)$ bằng $\sqrt{6}$ khi và chỉ khi
			\begin{align*}
				\dfrac{\left| x+1\right|}{\sqrt{6}}=\sqrt{6} \Leftrightarrow x=5, (x>0).
			\end{align*}
			\itemch \textbf{Đúng}.\\
			Ta có $\overrightarrow{AB}=(-2;-2;2)$, $\overrightarrow{u}=-\dfrac{1}{2}\overrightarrow{AB}=(1;1;-1)$; $\overrightarrow{n}_P=(1;2;-1)$.\\
			Do đó, $\overrightarrow{n}_Q=\left[\overrightarrow{u},\overrightarrow{n}_{(P)}\right]=(1;0;1)$ là một véc-tơ pháp tuyến của mặt phẳng $(Q)$.\\
			Vậy $(Q)\colon 1\cdot(x-1)+0\cdot(y-2)+1\cdot(z+1)=0 \Leftrightarrow x+z=0$.
			\itemch \textbf{Đúng}.\\
			Mặt phẳng $(P)\colon x+2y-z+1=0$ có một véc-tơ pháp tuyến là $(1;2;-1)$.
			\itemch \textbf{Sai}.\\
			$(P)$ có một véc-tơ pháp tuyến là $\overrightarrow{n}_P=(1;2;-1)$; $(R)$ có một véc-tơ pháp tuyến là $\overrightarrow{n}_R=(2;-m;0)$.\\
			Hai mặt phẳng $(P)$ và $(R)$ vuông góc với nhau khi và chỉ khi
			\begin{align*}
				\overrightarrow{n}_P\cdot \overrightarrow{n}_R=0 \Leftrightarrow 1\cdot 2+2\cdot (-m)+(-1)\cdot 0=0 \Leftrightarrow m=1.
			\end{align*}
		\end{itemchoice}
	}
\end{ex}

%G:\My Drive\CODE12-2024\DE-ON-THEO BAI\2H5-TACH DE\Bai1-De2.tex
\begin{ex}%[2H5V1-3]
	Trong không gian với hệ tọa độ $Oxyz$, cho hai điểm $A(1;2;1)$ và $B(3;-1;5)$. Các mệnh đề sau đúng hay sai?
	\choiceTF
	{\True Phương trình mặt phẳng trung trực của đoạn thẳng $AB$ là $2x-3y+4z-\dfrac{29}{2}=0$}
	{Điểm $N(1;2;-1)$ đối xứng với $A(1;2;1)$ qua mặt phẳng $(Oyz)$}
	{\True Mặt phẳng $(P)$ vuông góc với đường thẳng $AB$ và cắt các trục $Ox$, $Oy$ và $Oz$ lần lượt tại các điểm $D$, $E$ và $F$. Khi thể tích của tứ diện $ODEF$ bằng $\dfrac{3}{2}$, phương trình mặt phẳng $(P)$ là $2x-3y+4z\pm 6=0$}
	{Véc-tơ $\overrightarrow{AB}$ là một véc-tơ pháp tuyến của mặt phẳng $(\alpha)\colon 2x+3y+4z-2=0$}
	\loigiai{
		\begin{itemchoice}
			\itemch \textbf{Đúng}.\\
			Trung điểm của đoạn thẳng $AB$ là điểm $I\left(2;\dfrac{1}{2};3\right)$.\\
			Vậy mặt phẳng trung trực của $AB$ đi qua $I\left(2;\dfrac{1}{2};3\right)$ và nhận $\overrightarrow{AB}=(2;-3;4)$ làm véc-tơ pháp tuyến nên có phương trình là
			\begin{align*}
				2(x-2)-3\left(y-\dfrac{1}{2}\right)+4(z-3)=0 \Leftrightarrow 2x-3y+4z-\dfrac{29}{2}=0.
			\end{align*}
			\itemch \textbf{Sai}.\\
			Điểm đối xứng với $A(1;2;1)$ qua mặt phẳng $(Oyz)$ có tọa độ là $(-1;2;1)$.
			\itemch \textbf{Đúng}.\\
			Vì $AB\perp (P)$ nên mặt phẳng $(P)$ có một véc-tơ pháp tuyến là $\overrightarrow{AB}=(2;-3;4)$. Do đó, phương trình mặt phẳng $(P)$ có dạng $2x-3y+4z+d=0$.\\
			Từ đây tìm được $D\left(-\dfrac{d}{2};0;0\right)$, $E\left(0;\dfrac{d}{3};0\right)$, $F\left(0;0;-\dfrac{d}{4}\right)$ suy ra $OD=\dfrac{\left|d\right|}{2}$, $OE=\dfrac{\left|d\right|}{3}$, $OF=\dfrac{\left|d\right|}{4}$.\\
			Mặt khác, tứ diện $ODEF$ có $OD$, $OE$, $OF$ đôi một vuông góc nên thể tích của tứ diện $ODEF$ là $V_{ODEF}=\dfrac{1}{6}OD. OE. OF=\dfrac{(\left|d\right|)^3}{144}$.\\
			Do đó, $V_{ODEF}=\dfrac{3}{2}$ $\Leftrightarrow$ $\dfrac{(\left|d\right|)^3}{144}=\dfrac{3}{2}$ $\Leftrightarrow$ $\left|d\right|=6$ $\Leftrightarrow$ $d=\pm 6$.\\
			Vậy phương trình mặt phẳng $(P)$ là $2x-3y+4z\pm 6=0$.
			\itemch \textbf{Sai}.\\
			Véc-tơ $\overrightarrow{AB}=(2;-3;4)$ không cùng phương với $\overrightarrow{n}=(2;3;4)$ là véc-tơ pháp tuyến của mặt phẳng $(\alpha)$ nên $\overrightarrow{AB}$ không là một véc-tơ pháp tuyến của mặt phẳng $(\alpha)$.
		\end{itemchoice}
	}
\end{ex}

%G:\My Drive\CODE12-2024\DE-ON-THEO BAI\2H5-TACH DE\Bai1-De2.tex
\begin{ex}%[2H5V1-4]
	Trong không gian với hệ tọa độ $Oxyz$, cho hai mặt phẳng $(\alpha)\colon 3x-2y+2z+7=0$ và $(\beta)\colon 5x-4y+3z+1=0$. Các mệnh đề sau đúng hay sai?
	\choiceTF
	{Hai mặt phẳng $(\alpha)$, $(\beta)$ song song với nhau}
	{Điểm $A(1;2;-1)$ nằm trên mặt phẳng $(\alpha)\colon 3x-2y+2z+7=0$}
	{\True Phương trình mặt phẳng qua $O$, đồng thời vuông góc với cả $(\alpha)$ và $(\beta)$ có phương trình là $2x+y-2z=0$}
	{\True Mặt phẳng $(\gamma)$ đi qua điểm $I(1;0;-1)$ và song song với $(\alpha)\colon 3x-2y+2z+7=0$ có phương trình là $(\gamma)\colon 3x-2y+2z-1=0$}
	\loigiai{
		\begin{itemchoice}
			\itemch \textbf{Sai}.\\
			Ta có $\dfrac{3}{5}\ne\dfrac{-2}{-4}$ nên hai mp $(\alpha)$, $(\beta)$ cắt nhau. Vậy chúng không song song.
			\itemch \textbf{Sai}.\\
			Ta có $3\cdot 1-2\cdot 2+2\cdot (-1)+7\ne 0$ nên $A(1;2;-1)$ không nằm trên mặt phẳng $(\alpha)$.
			\itemch \textbf{Đúng}.\\
			Mặt phẳng $(\alpha)$ có một véc-tơ pháp tuyến là $\overrightarrow{n}_1=(3;-2;2)$.\\
			Mặt phẳng $(\beta)$ có một véc-tơ pháp tuyến là $\overrightarrow{n}_2=(5;-4;3)$.\\
			Do mặt phẳng $(Q)$ vuông góc với cả $(\alpha)$ và $(\beta)$ nên $\overrightarrow{n}=\left[\overrightarrow{n}_1,\overrightarrow{n}_2\right]=(2;1;-2)$ là một véc-tơ pháp tuyến của mặt phẳng $(Q)$.\\
			Mặt phẳng $(Q)$ đi qua $O(0;0;0)$ và có véc-tơ pháp tuyến $\overrightarrow{n}=(2;1;-2)$ có phương trình là $2x+y-2z=0$.
			\itemch \textbf{Đúng}.\\
			Mặt phẳng $(\gamma)$ song song với mặt phẳng $(\alpha)\colon 3x-2y+2z+7=0$ nên phương trình của mặt phẳng $(\gamma)$ có dạng $3x-2y+2z+d=0$, với $d\ne7$.\\
			Vì $I(1;0;-1)\in(\gamma)$ nên $3\cdot 1-2\cdot 0+2\cdot(-1)+d=0$ $\Leftrightarrow$ $d=-1$ (thỏa mãn điều kiện $d\ne7$).\\
			Vậy $(\gamma)\colon 3x-2y+2z-1=0$.
		\end{itemchoice}
	}
\end{ex}
	\Closesolutionfile{ans}

\TNSA
	\setcounter{ex}{0}
	\Opensolutionfile{ans}[ans/B1-De2-3]
\begin{ex}%[2H5V1-7]
	Từ mặt nước trong một bể nước, tại ba vị trí đôi một cách nhau $2$ m, người ta lần lượt thả dây dọi để quả dọi chạm đáy bể. Phần dây dọi (thẳng) nằm trong nước tại ba vị trí đó lần lượt có độ dài $4$ m; $4{,}4$ m; $4{,}8$ m. Biết đáy bể là phẳng. Hỏi đáy bể nghiêng so với mặt phẳng nằm ngang một góc bao nhiêu độ (làm tròn đến hàng phần chục)?\\
	\shortans[oly]{$21{,}8$}
	\loigiai{
		Gọi ba vị trí trên mặt nước là $A$, $B$, $C$ thì tam giác $ABC$ là tam giác đều cạnh bằng $2$ m. Gọi dây dọi lần lượt là $AA'$, $BB'$, $CC'$ có độ dài lần lượt là $4$ m; $4{,}4$ m; $4{,}8$ m.\\
		Chọn hệ trục toạ độ $Oxyz$ sao cho $O$ là trung điểm của $BC$, tia $Ox$ chứa điểm $A$, tia $Oy$ chứa điểm $B$, tia $Oz$ đi qua trung điểm của $B'C'$ và đơn vị trên các trục là mét.\\
		Ta có $OB=OC=1$, $OA=\sqrt{3}$ $\Rightarrow$ $A'\left(\sqrt{3};0;4\right)$, $B'(0;1;4{,}4)$, $C'(0;-1;4{,}8)$.\\
		Khi đó, $\overrightarrow{A'B'}=\left(-\sqrt{3};1;0{,}4\right)$, $\overrightarrow{A'C'}=\left(-\sqrt{3};-1;0{,}8\right)$.\\
		Mặt phẳng $(A'B'C')$ có một véc-tơ pháp tuyến là $\overrightarrow{n}=\left[\overrightarrow{A'B'},\overrightarrow{A'C'}\right]=0{,}4\sqrt{3}\left(\sqrt{3};1;5\right)$.\\
		Mặt phẳng $(ABC)$ có một véc-tơ pháp tuyến là $\overrightarrow{k}=(0;0;1)$.\\
		Do đó, $\cos\big((ABC),(A'B'C')\big)=\left|\cos\left(\overrightarrow{n},\overrightarrow{k}\right)\right|=\dfrac{5}{\sqrt{29}}$. Góc cần tìm gần bằng $21{,}8^\circ$.
	}
\end{ex}

\begin{ex}%[2H5V1-3]
	Trong không gian với hệ tọa độ $Oxyz$, cho mặt cầu $(S)\colon (x-1)^2+(y+1)^2+z^2=11$ và hai véc-tơ $\overrightarrow{u}_1=(1;1;2)$, $\overrightarrow{u}_2=(1;2;1)$. Gọi $(P)$ là mặt phẳng tiếp xúc với mặt cầu $(S)$ đồng thời song song với giá của hai véc-tơ $\overrightarrow{u}_1$, $\overrightarrow{u}_2$. Phương trình mặt phẳng $(P)$ có dạng $3x+by+cz+d=0$, với $b,\,c,\,d\in\mathbb{Z}$ và $d\ne -15$. Khi đó, $b+c+d$ bằng bao nhiêu?\\
	\shortans[oly]{$5$}
	\loigiai{
		Mặt cầu $(S)$ có tâm $I(1;-1;0)$, bán kính $R=\sqrt{11}$.\\
		Mặt phẳng $(P)$ song song với của hai véc-tơ $\overrightarrow{u}_1$, $\overrightarrow{u}_2$ nên $(P)$ có véc-tơ pháp tuyến là $\overrightarrow{n}=\left[\overrightarrow{u}_1,\overrightarrow{u}_2\right]=(-3;1;1)$.\\
		Phương trình mặt phẳng $(P)$ có dạng $-3x+y+z+d=0$$\Leftrightarrow$ $3x-y-z-d=0$, $d\ne 15$.\\
		Mặt khác, mặt phẳng $(P)$ tiếp xúc với mặt cầu $(S)$ nên ta có
		\begin{align*}
			\mathrm{d}\big(I,(P)\big)=R \Leftrightarrow \dfrac{\left|3+1-0-d\right|}{\sqrt{9+1+1}}=\sqrt{11} \Leftrightarrow \left|-d+4\right|=11 \Leftrightarrow \hoac{&d=15& (\text{loại})\\ &d=-7.&}
		\end{align*}
		Với $d=-7$, ta có phương trình mặt phẳng $(P)$ là $-3x+y+z-7=0$ $\Leftrightarrow$ $3x-y-z+7=0$.\\
		Vậy $b+c+d=-1-1+7=5$.
	}
\end{ex}

\begin{ex}%[2H5C1-3]
	Trong không gian với hệ tọa độ $Oxyz$, có hai mặt phẳng $(P)$ và $(Q)$ cùng thỏa mãn các điều kiện sau: đi qua hai điểm $A(1;1;1)$ và $B(0;-2;2)$, đồng thời cắt các trục tọa độ $Ox$, $Oy$ tại hai điểm cách đều $O$. Giả sử $(P)$ có phương trình $x+b_1y+c_1z+d_1=0$ và $(Q)$ có phương trình $x+b_2y+c_2z+d_2=0$. Tính giá trị biểu thức $b_1b_2+c_1c_2$.\\
	\shortans[oly]{$-9$}
	\loigiai{
		Xét mặt phẳng $(\alpha)$ có phương trình $x+by+cz+d=0$ thỏa mãn các điều kiện: đi qua hai điểm $A(1;1;1)$ và $B(0;-2;2)$, đồng thời cắt các trục tọa độ $Ox,Oy$ tại hai điểm cách đều $O$.\\
		Vì $(\alpha)$ đi qua $A(1;1;1)$ và $B(0;-2;2)$ nên ta có hệ phương trình
		\begin{align*}
			\heva{& 1+b+c+d=0\\ & -2b+2c+d=0.} \tag{*}
		\end{align*}
		Mặt phẳng $(\alpha)$ cắt các trục tọa độ $Ox$, $Oy$ lần lượt tại $M(-d;0;0)$, $N\left(0;\dfrac{-d}{b};0\right)$.\\
		Vì $M$, $N$ cách đều $O$ nên $OM=ON$. Suy ra $\left|d\right|=\left|\dfrac{d}{b}\right|$.\\
		Nếu $d=0$ thì chỉ tồn tại duy nhất một mặt phẳng thỏa mãn yêu cầu bài toán (mặt phẳng này sẽ đi qua điểm $O$).\\
		Do đó, để tồn tại hai mặt phẳng thỏa mãn yêu cầu bài toán thì $\left|d\right|=\left|\dfrac{d}{b}\right|$ $\Leftrightarrow$ $b=\pm 1$.
		\begin{itemize}
			\item Với $b=1$, $(*)$ $\Leftrightarrow$ $\heva{
				& c+d=-2\\
				& 2c+d=2
			}$ $\Leftrightarrow$ $\heva{
				& c=4\\
				& d=-6
			}$. Ta được $(P)\colon x+y+4z-6=0$.
			\item Với $b=-1$, $(*)$ $\Leftrightarrow$ $\heva{
				& c+d=0\\
				& 2c+d=-2
			}$ $\Leftrightarrow$ $\heva{
				& c=-2\\
				& d=2
			}$. Ta được $(Q)\colon x-y-2z+2=0$.
		\end{itemize}
		Vậy $b_1 b_2+ c_1 c_2=1\cdot (-1)+4\cdot (-2)=-9$.
	}
\end{ex}

\begin{ex}%[2H5V1-3]
	Trong không gian với hệ tọa độ $Oxyz$, cho mặt cầu $(S)\colon (x+1)^2+(y-2)^2+(z-3)^2=8$ và điểm $A(1;3;2)$.
	Mặt phẳng $(P)$ đi qua $A$ và cắt $(S)$ theo giao tuyến là đường tròn có bán kính nhỏ nhất. Biết phương trình của $(P)$ có dạng $ax+by+cz+6=0$. Tính $a+b+c$.\\
	\shortans[oly]{$-4$}
	\loigiai{
		Mặt cầu $(S)$ có tâm $I(-1;2;3)$, bán kính $R=2\sqrt{2}$.\\
		Ta có $\overrightarrow{IA}=(2;1;-1)$; $AI=\sqrt{6}<R$, suy ra điểm $A$ nằm trong mặt cầu $(S)$.\\
		Gọi $H$ là hình chiếu vuông góc của $I$ trên mặt phẳng $(P)$. Khi đó mặt phẳng $(P)$ đi qua $A$ và cắt $(S)$ theo giao tuyến là đường tròn có bán kính $r=\sqrt{R^2-IH^2}$. Do đó, $r$ nhỏ nhất khi và chỉ khi $IH$ lớn nhất.\\
		Mặt khác, ta luôn có $IH\le IA$, dấu bằng xảy ra khi và chỉ khi $H$ trùng với $A$, hay $(P)\perp IA$.\\
		Mặt phẳng $(P)$ có véc-tơ pháp tuyến $\overrightarrow{IA}=(2;1;-1)$ và qua $A(1;3;2)$ có phương trình $2(x-1)+(y-3)-1(z-2)=0$ $\Leftrightarrow$ $2x+y-z-3=0$ $\Leftrightarrow$ $-4x-2y+2z+6=0$.\\
		Vậy $a+b+c=-4$.
	}
\end{ex}

\begin{ex}%[2H5V1-3]
	Trong không gian với hệ tọa độ $Oxyz$, cho hai điểm $A(2;-3;1)$, $B(-1;1;0)$ và mặt phẳng $(P)\colon x-y+z-2=0$. Một mặt phẳng $(Q)$ đi qua hai điểm $A$, $B$ và vuông góc với $(P)$ có dạng là $ax+by+cz+2=0$. Tính $a^2+ b^2+ c^2$.\\
	\shortans[oly]{$56$}
	\loigiai{
		$\overrightarrow{AB}=(-3;4;-1)$, $(P)$ có một véc-tơ pháp tuyến là $\overrightarrow{n}_{(P)}=(1;-1;1)$.\\
		$\left[\overrightarrow{AB},\overrightarrow{n}_{(P)}\right]=(3;2;-1)$.\\
		$(Q)$ đi qua $B(-1;1;0)$ và có một véc-tơ pháp tuyến $\overrightarrow{n}_{(Q)}=(3;2;-1)$ nên có phương trình
		\begin{align*}
			3(x+1)+2(y-1)-z=0 \Leftrightarrow 3x+2y-z+1=0 \Leftrightarrow 6x+4y-2z+2=0.
		\end{align*}
		Suy ra $a=6$, $b=4$, $c=-2$ hay $a^2+b^2+c^2=56$.
	}
\end{ex}

\begin{ex}%[2H5H1-3]
	Trong không gian với hệ trục $Oxyz$, cho ba điểm $A(1;2;1)$, $B(2;-1;0)$, $C(1;1;3)$. Phương trình mặt phẳng đi qua ba điểm $A$, $B$, $C$ có dạng $ax+by+cz-12=0$. Khi đó, $a-b-2c$ bằng\\
	\shortans[oly]{$3$}
	\loigiai{
		Ta có $\overrightarrow{AB}=(1;-3;-1)$, $\overrightarrow{AC}=(0;-1;2)$ suy ra $\left[\overrightarrow{AB},\overrightarrow{AC}\right]=(-7;-2;-1)=-1(7;2;1)$.\\
		Mặt phẳng $(ABC)$ đi qua điểm $A(1;2;1)$ có véc-tơ pháp tuyến $\overrightarrow{n}=(7;2;1)$ có phương trình là $7x+2y+z-12=0$. Khi đó, $a-b-2c=3$.
	}
\end{ex}

\centerline{---HẾT---}
\Closesolutionfile{ans}
%\newpage
%%=====================
%\begin{center}
%\textbf{\large BẢNG ĐÁP ÁN}
%\end{center}
%\noindent\textbf{ĐÁP ÁN PHẦN I}
%\inputansbox{10}{ans/B1-De2-1}
	
%\noindent\textbf{ĐÁP ÁN PHẦN II}
%\inputansbox[2]{2}{ans/B1-De2-2}
	
%\noindent\textbf{ĐÁP ÁN PHẦN III}
%\inputansbox[3]{6}{ans/B1-De2-3}


