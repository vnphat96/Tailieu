\subsection{BÀI TẬP TRẮC NGHIỆM TỰ LUYỆN}
\TN
	\setcounter{ex}{0}
	\Opensolutionfile{ans}[ans/B3-De2-1]

\begin{ex}%[2H2N2-4]
	Trong không gian $Oxyz$, cho đường thẳng $d\colon\heva{
		& x=6+5t \\
		& y=2+t \\
		& z=1
	}$ và mặt phẳng $\left( P \right)\colon3x-2y+1=0$. Tính góc hợp bởi đường thẳng $d$ và mặt phẳng $\left( P \right)$.
	\choice
	{$30^\circ $}
	{\True $45^\circ $}
	{$60^\circ $}
	{$90^\circ $}
	\loigiai{
		Đường thẳng $d\colon\heva{
			& x=6+5t \\
			& y=2+t \\
			& z=1 \\
		}$ có vectơ chỉ phương $\overrightarrow{u}=\left( 5;1;0 \right)$.\\
		Mặt phẳng $\left( P \right)\colon3x-2y+1=0$ có vectơ pháp tuyến $\overrightarrow{n}=\left( 3;-2;0 \right)$.\\
		Gọi $\alpha $ là góc hợp bởi đường thẳng $d$ và mặt phẳng $\left( P \right)$.\\
		Khi đó: $\sin \alpha =\dfrac{\left| \overrightarrow{u}\cdot\overrightarrow{n} \right|}{\left| {\overrightarrow{u}} \right|\cdot\left| {\overrightarrow{n}} \right|}=\dfrac{\left| 5\cdot3+1\cdot\left( -2 \right)+0\cdot0 \right|}{\sqrt{5^2+1^2}\cdot\sqrt{3^2+{{\left( -2 \right)}^2}}}=\dfrac{\sqrt{2}}{2}$.\\ Suy ra $\alpha =45^\circ $.}
\end{ex}

%G:\My Drive\CODE12-2024\DE-ON-THEO BAI\2H5-TACH DE\Bai3-De2.tex
\begin{ex}%[2H2N2-4]
	Trong không gian $Oxyz$, cho ba điểm $M\left( 2; 3; -1 \right)$, $N\left( -1; 1; 1 \right)$ và $P\left( 1; m-1; 2 \right)$. Tìm $m$ để tam giác $MNP$ vuông tại $N$.
	\choice
	{\True $m=0$}
	{$m=-4$}
	{$m=2$}
	{$m=-6$}
	\loigiai{
		Ta có\\
		$\overrightarrow{NM}=\left( 3; 2; -2 \right)$, $\overrightarrow{NP}=\left( 2; m-2; 1 \right)$.\\
		Tam giác $MNP$ vuông tại $N$ khi và chỉ khi
		\begin{eqnarray*}
			& & \overrightarrow{NM}\cdot\overrightarrow{NP}=0\\
			&\Leftrightarrow & 3\cdot2+2\cdot\left( m-2 \right)-2\cdot1=0\\
			&\Leftrightarrow & m=0.
		\end{eqnarray*}
		Vậy giá trị cần tìm của $m$ là $m=0$.}
\end{ex}

%G:\My Drive\CODE12-2024\DE-ON-THEO BAI\2H5-TACH DE\Bai3-De2.tex
\begin{ex}%[2H5N2-7]
	Trong không gian $Oxyz$, tính góc giữa hai đường thẳng $d_1\colon\dfrac{x}{1}=\dfrac{y+1}{-1}=\dfrac{z-1}{2}$ và $d_2\colon\dfrac{x+1}{-1}=\dfrac{y}{1}=\dfrac{z-3}{1}$.
	\choice
	{$60^\circ $}
	{$30^\circ $}
	{$45^\circ $}
	{\True $90^\circ $}
	\loigiai{
		Ta có $\overrightarrow{u}_{d_1}=\left( 1;-1;2 \right)$ và
		$\overrightarrow{u}_{d_2}=\left( -1;1;1 \right)$ lần lượt là véc tơ chỉ phương của $d_1$ và $d_2$.\\
		$\overrightarrow{u}_{d_1}\cdot\overrightarrow{u}_{d_2}=1\cdot\left( -1 \right)+\left( -1 \right)\cdot1+2\cdot1=0\Rightarrow {d_1}\bot {d_2}\Rightarrow \left( \widehat{d_1,d_2} \right)=90^\circ $.}
\end{ex}

%G:\My Drive\CODE12-2024\DE-ON-THEO BAI\2H5-TACH DE\Bai3-De2.tex
\begin{ex}%[2H5N2-7]
	Trong không gian $Oxyz$, cho đường thẳng $d\colon\dfrac{x-4}{1}=\dfrac{y-5}{2}=\dfrac{z}{3}$ mặt phẳng $\left( \alpha \right)$ chứa đường thẳng $d$ sao cho khoảng cách từ $O$ đến $\left( \alpha \right)$ đạt giá trị lớn nhất. Khi đó góc giữa mặt phẳng $\left( \alpha \right)$ và trục $Ox$ là $\varphi $ thỏa mãn.
	\choice
	{$\sin \varphi =\dfrac{2}{3\sqrt{3}}$}
	{$\sin \varphi =\dfrac{1}{3\sqrt{3}}$}
	{$\sin \varphi =\dfrac{1}{2\sqrt{3}}$}
	{\True $\sin \varphi =\dfrac{1}{\sqrt{3}}$}
	\loigiai{
		Đường thẳng $d$ có véc tơ chỉ phương $\overrightarrow{u}=\left( 1; 2; 3 \right)$.\\
		Gọi $H$ là hình chiếu của $O$ lên $d$, $K$ là hình chiếu của $O$ lên $\left( \alpha \right)$.\\
		Ta có
		$d\left( O,\left( \alpha \right) \right)=OK\le OH\Rightarrow d\left( O,\left( \alpha \right) \right)$ lớn nhất bằng $OH$ khi $K\equiv H$.\\
		Khi đó $\left( \alpha \right)$ chứa $d$ và nhận $\overrightarrow{n}=\overrightarrow{OH}$ làm véc tơ pháp tuyến.\\
		$H\in d\Rightarrow H\left( 4+t; 5+2t; 3t \right)\Rightarrow \overrightarrow{OH}=\left( 4+t; 5+2t; 3t \right)$.\\
		Vì $OH\bot d\Rightarrow \overrightarrow{OH}\cdot\overrightarrow{u}=0\Leftrightarrow 4+t+2\left( 5+2t \right)+3\cdot3t=0\Leftrightarrow 14t+14=0\Leftrightarrow t=-1$.\\
		$\Rightarrow H\left( 3; 3; -3 \right)$, $\overrightarrow{OH}=\left( 3; 3; -3 \right)$.\\
		Trục $Ox$ có véc tơ chỉ phương  $\overrightarrow{i}=\left( 1; 0; 0 \right)$.\\
		$\sin \varphi =\dfrac{\left| \overrightarrow{i}\cdot\overrightarrow{n} \right|}{\left| \overrightarrow{i} \right|\cdot\left| \overrightarrow{n} \right|}=\dfrac{\left| 3 \right|}{\sqrt{1}\cdot\sqrt{3^2+3^2+{{\left( -3 \right)}^2}}}=\dfrac{1}{\sqrt{3}}$.}
\end{ex}

%G:\My Drive\CODE12-2024\DE-ON-THEO BAI\2H5-TACH DE\Bai3-De2.tex
\begin{ex}%[2H5N1-3]
	Trong không gian $Oxyz$, gọi $\left( P \right)$ là mặt phẳng chứa trục $Oy$ và tạo với mặt phẳng $y+z+1=0$ một góc $60^\circ$. Phương trình mặt phẳng $\left( P \right)$ là
	\choice
	{$\hoac{
			& x-z-1=0 \\
			& x-z=0 \\
		}$}
	{$\hoac{
			& x-2z=0 \\
			& x+z=0 \\
		}$}
	{$\hoac{
			& x-y=0 \\
			& x+y=0 \\
		}$}
	{\True $\hoac{
			& x-z=0 \\
			& x+z=0 \\
		}$}
	\loigiai{
		+) Do $\left( P \right)$ chứa trục $Oy$ nên phương trình $\left( P \right)$ có dạng $\left( P \right)\colon ax+cz=0$, $\left( {a^2}+c^2>0 \right)$.\\
		+) Gọi $\left(Q\right)\colon y+z+1=0$.\\ Ta có $\cos \left( \left( P \right),\,\left( Q \right) \right)=\cos 60^\circ \Leftrightarrow \dfrac{\left| c \right|}{\sqrt{a^2+c^2}\cdot\sqrt{2}}=\dfrac{1}{2}\Leftrightarrow c=\pm a$.\\
		Khi đó $\hoac{
			& \left( P \right)\colon ax+az=0 \\
			& \left( P \right)\colon ax-az=0 \\
		}\Leftrightarrow \hoac{
			& \left( P \right)\colon x+z=0 \\
			& \left( P \right)\colon x-z=0 \\
		}$ .}
\end{ex}

%G:\My Drive\CODE12-2024\DE-ON-THEO BAI\2H5-TACH DE\Bai3-De2.tex
\begin{ex}%[2H5N1-4]
	Với giá trị nào của $m$ thì đường thẳng $\left( D \right)\colon\dfrac{x+1}{2}=\dfrac{y-3}{m}=\dfrac{z-1}{m-2}$ vuông góc với mặt phẳng $\left( P \right)\colon x+3y+2z=2$.
	\choice
	{\True $6$}
	{$5$}
	{$-7$}
	{$1$}
	\loigiai{
		Vectơ chỉ phương của $\left( D \right)$ là $\overrightarrow{a}=\left( 2;m;m-2 \right)$.\\
		Vectơ pháp tuyến của $\left( P \right)$ là $\overrightarrow{n}=\left( 1;3;2 \right)$.\\
		$\left( D \right)\bot \left( P \right)\Leftrightarrow $ $\overrightarrow{a}$ và $\overrightarrow{n}$ cùng phương.\\ $2=\dfrac{m}{3}=\dfrac{m-2}{2}\Leftrightarrow m=6$.}
\end{ex}

%G:\My Drive\CODE12-2024\DE-ON-THEO BAI\2H5-TACH DE\Bai3-De2.tex
\begin{ex}%[2H5N1-4]
	Trong không gian $Oxyz$, cho mặt phẳng $\left( P \right)\colon mx+ny-2z-1=0$ và đường thẳng $\dfrac{x}{n+1}=\dfrac{y}{m}=\dfrac{1-z}{1}$ với $m\ne 0$, $m\ne -1$. Khi $\left( P \right)\bot d$ thì tổng $m+n$ bằng bao nhiêu?
	\choice
	{$m+n=2$}
	{\True $m+n=-2$}
	{$m+n=-\dfrac{1}{2}$}
	{$m+n=-\dfrac{2}{3}$}
	\loigiai{
		Sử dụng tỷ lệ thức $\dfrac{m}{n+1}=\dfrac{n}{m}=\dfrac{-2}{-1}\Rightarrow \dfrac{m+n}{n+1+m}=2\Rightarrow m+n=-2$.}
\end{ex}

%G:\My Drive\CODE12-2024\DE-ON-THEO BAI\2H5-TACH DE\Bai3-De2.tex
\begin{ex}%[2H5N2-7]
	Trong không gian $Oxyz$, cho hai đường thẳng $d_1:\dfrac{x}{1}=\dfrac{y+1}{-1}=\dfrac{z-1}{2}$ và $d_2:\dfrac{x+1}{-1}=\dfrac{y}{1}=\dfrac{z-3}{1}$. Góc giữa hai đường thẳng đó bằng
	\choice
	{\True $90^\circ $}
	{$60^\circ $}
	{$30^\circ $}
	{$45^\circ $}
	\loigiai{
		Đường thẳng $d_1$ có véctơ chỉ phương $\overrightarrow{u}_1=\left( 1;-1;2 \right)$.\\
		Đường thẳng $d_2$ có véctơ chỉ phương $\overrightarrow{u}_2=\left( -1;1;1 \right)$.\\
		Gọi $\alpha$ là góc giữa hai đường thẳng trên.\\
		Khi đó ta có $\cos \alpha =\left| \cos \left( \overrightarrow{u}_1,\overrightarrow{u}_2 \right) \right|=\dfrac{\left| 1\cdot\left( -1 \right)+\left( -1 \right)\cdot1+2\cdot1 \right|}{\sqrt{1^2+{{\left( -1 \right)}^2}+2^2}\cdot\sqrt{{{\left( -1 \right)}^2}+1^2+1^2}}=0$.\\$\Rightarrow \left( \widehat{d_1,d_2} \right)=90^\circ $.}
\end{ex}

%G:\My Drive\CODE12-2024\DE-ON-THEO BAI\2H5-TACH DE\Bai3-De2.tex
\begin{ex}%[2H5N2-7]
	Trong không gian $Oxyz$, cho đường thẳng $\Delta \colon x=\dfrac{y}{2}=\dfrac{z-1}{3}$ và mặt phẳng $\left( P \right)\colon4x+2y+z-1=0$. Khi đó khẳng định nào sau đây là đúng?
	\choice
	{\True Góc tạo bởi $\left( \Delta \right)$ và $\left( P \right)$ lớn hơn $30^\circ $}
	{$ \Delta \parallel\left( P \right)$}
	{$ \Delta\bot \left( P \right)$}
	{$ \Delta \subset \left( P \right)$}
	\loigiai{
		Ta có $\sin \left( \widehat{\Delta ,\left( P \right)} \right)=\dfrac{11}{7\sqrt{6}}>\dfrac{1}{2}$.\\ Suy ra góc tạo bởi $ \Delta $ và $\left( P \right)$ lớn hơn $30^\circ $.}
\end{ex}

%G:\My Drive\CODE12-2024\DE-ON-THEO BAI\2H5-TACH DE\Bai3-De2.tex
\begin{ex}%[2H2N2-4]
	Trong không gian $Oxyz$, cho mặt phẳng $\left( P \right)\colon3x+4y+5z-8=0$ và đường thẳng $d\colon\heva{
		& x=2-3t \\
		& y=-1-4t \\
		& z=5-5t \\
	}$. Góc giữa đường thẳng $d$ và mặt phẳng $\left( P \right)$ là
	\choice
	{\True $90^\circ $}
	{$45^\circ $}
	{$60^\circ $}
	{$30^\circ $}
	\loigiai{
		Mặt phẳng $\left( P \right)$ có một véc tơ pháp tuyến là $\overrightarrow{n}=\left( 3;4;5 \right)$.\\
		Đường thẳng $d$ có một véc tơ chỉ phương là $\overrightarrow{u}=\left( -3;-4;-5 \right)$.\\
		Ta có $\overrightarrow{n}=-\overrightarrow{u}\Rightarrow d\bot \left( P \right)$ nên góc $90^\circ $}
\end{ex}

%G:\My Drive\CODE12-2024\DE-ON-THEO BAI\2H5-TACH DE\Bai3-De2.tex
\begin{ex}%[2H5N2-7]
	Trong không gian $Oxyz$, cho hai đường thẳng $d_1\colon\heva{
		& x=-1-t \\
		& y=3+4t \\
		& z=3+3t \\
	}$ và $d_2\colon\dfrac{x}{1}=\dfrac{y+8}{-4}=\dfrac{z+3}{-3}$. Tính góc hợp bởi đường thẳng $d_1$ và $d_2$.\\
	\choice
	{$90^\circ $}
	{$60^\circ $}
	{\True $0^\circ $}
	{$30^\circ $}
	\loigiai{
		Ta có đường thẳng $d_1\colon \heva{
			& x=-1-t \\
			& y=3+4t \\
			& z=3+3t \\
		}$ có vectơ chỉ phương là $\overrightarrow{u_1}=\left( -1; 4; 3 \right)$,\\
		đường thẳng $d_2\colon\dfrac{x}{1}=\dfrac{y+8}{-4}=\dfrac{z+3}{-3}$ có vectơ chỉ phương là $\overrightarrow{u_2}=\left( 1;-4;-3 \right)$.\\
		Vì $\overrightarrow{u_1}=\left( -1;4;3 \right)$ và $\overrightarrow{u_2}=\left( 1;-4;-3 \right)$ cùng phương với nhau nên góc hợp bởi đường thẳng $d_1$ và $d_2$ bằng $0^\circ $.}
\end{ex}

%G:\My Drive\CODE12-2024\DE-ON-THEO BAI\2H5-TACH DE\Bai3-De2.tex
\begin{ex}%[2H5N2-7]
	Trong không gian $Oxyz$, cho mặt phẳng $(P)\colon -\sqrt{3}x+y+1=0$. Tính góc tạo bởi $(P)$ với trục $Ox$?
	\choice
	{$120^\circ$}
	{$30^\circ$}
	{$150^\circ$}
	{\True $60^\circ$}
	\loigiai{
		Mặt phẳng $(P)$ có véc tơ pháp tuyến $\overrightarrow{n}=(-\sqrt{3};1;0)$\\
		Trục $Ox$ có có véc tơ pháp tuyến $\overrightarrow{i}=(1;0;0)$.\\
		Góc tạo bởi $(P)$ với trục $Ox$\\
		$\sin((P),Ox)=\left| \cos((P), Ox) \right|=\dfrac{\left| \overrightarrow{n}\cdot\overrightarrow{i} \right|}{\left| \overrightarrow{n} \right|\cdot\left| \overrightarrow{i} \right|}=\dfrac{\left| -\sqrt{3}\cdot1+1\cdot0+0\cdot0 \right|}{\sqrt{3+1}\cdot\sqrt{1}}=\dfrac{\sqrt{3}}{2}$.\\
		Vậy góc tạo bởi $(P)$ với trục $Ox$ bằng $60^\circ$.}
\end{ex}
	\Closesolutionfile{ans}

\TNTF
	\setcounter{ex}{0}
	\Opensolutionfile{ans}[ans/B3-De2-2]

\begin{ex}%[2H5N1-3]
	Trong không gian $Oxyz$, cho hai đường thẳng $d_1\colon\heva{
		& x=2+t \\
		& y=-1+t \\
		& z=3 \\
	}$ và $d_2\colon\heva{
		& x=1-t \\
		& y=2 \\
		& z=-2+t \\
	}$. Xét tính đúng sai của các khẳng định sau
	\choiceTF
	{Đường thẳng $d_1$ có một vectơ chỉ phương là $\overrightarrow{u_1}=\left( 1; 1; 3 \right)$}
	{\True Góc giữa hai đường thẳng $d_1$ và $d_2$ bằng $60^\circ $}
	{\True Đường thẳng $d_2$ có một vectơ chỉ phương là $\overrightarrow{u_2}=\left( -1; 0; 1 \right)$}
	{Giá trị cosin của góc giữa hai đường thẳng $d_1$ và $d_2$ bằng $-\dfrac{1}{2}$}
	
	\loigiai{
		\begin{itemchoice}
			\itemch Sai. Đường thẳng $d_1$ có một vectơ chỉ phương là $\overrightarrow{u_1}=\left( 1; 1; 0 \right)$.
			\itemch Đúng. Ta có $\cos \left( {d_1}, d_2 \right)=\dfrac{\left| 1\cdot\left( -1 \right)+0+0 \right|}{\sqrt{2}\cdot\sqrt{2}}=\dfrac{1}{2}\Rightarrow \left( d_1, d_2 \right)=60^\circ $.
			\itemch Đúng. Đường thẳng $d_2$ có một vectơ chỉ phương là $\overrightarrow{u_2}=\left( -1; 0; 1 \right)$.
			\itemch Sai. Ta có $\cos \left( {d_1},\,d_2 \right)=\dfrac{\left| 1\cdot\left( -1 \right)+0+0 \right|}{\sqrt{2}\cdot\sqrt{2}}=\dfrac{1}{2}$.
		\end{itemchoice}
	}
\end{ex}

%G:\My Drive\CODE12-2024\DE-ON-THEO BAI\2H5-TACH DE\Bai3-De2.tex
\begin{ex}%%[2H5N2-7]
	Trong không gian $Oxyz$, cho mặt phẳng $\left( P \right)\colon2x+y+2z-1=0$ và hai điểm $A\left( 1; -1; 2 \right)$, $B\left( 0; 1; -1 \right)$. Xét tính đúng sai của các khẳng định sau
	\choiceTF
	{Giá trị cosin của góc giữa đường thẳng $AB$ và mặt phẳng $\left( P \right)$ bằng $\dfrac{2}{\sqrt{21}}$}
	{Đường thẳng $AB$ vuông góc với mặt phẳng $\left( P \right)$}
	{\True Mặt phẳng $\left( OAB \right)$ có một vectơ pháp tuyến là $\overrightarrow{n}=\left( -1; 1; 1 \right)$}
	{\True Giá trị cosin của góc giữa mặt phẳng $\left( OAB \right)$ và mặt phẳng $\left( P \right)$ bằng $\dfrac{\sqrt{3}}{9}$}
	\loigiai{
		\begin{itemchoice}
			\itemch Sai. Ta có $\sin \left( AB, \left( P \right) \right)=\dfrac{\left| -1\cdot2+2\cdot1+\left( -3 \right)\cdot2 \right|}{\sqrt{14}\cdot3}=\dfrac{\sqrt{7}}{7}$.\\ $\Rightarrow \cos \left( d, \left( P \right) \right)=\sqrt{1-\dfrac{1}{7}}=\dfrac{\sqrt{42}}{7}$.
			\itemch Sai. Đường thẳng $AB$ có một véctơ chỉ phương $\overrightarrow{u}=\overrightarrow{AB}=\left( -1; 2; -3 \right)$, ta thấy véctơ $\overrightarrow{u}$ không cùng phương với véctơ $\overrightarrow{n}$ nên $AB$ không vuông góc với mặt phẳng $\left( P \right)$.
			\itemch Đúng. Ta có $A\left( 1; -1; 2 \right)\Rightarrow \overrightarrow{OA}=\left( 1; -1; 2 \right)$; $B\left( 0; 1; -1 \right)\Rightarrow \overrightarrow{OB}=\left( 0; 1; -1 \right).$\\
			Do đó, mặt phẳng $\left( OAB \right)$ có một vectơ pháp tuyến là $\overrightarrow{n}=\left[ \overrightarrow{OA},\overrightarrow{OB} \right]=\left( -1; 1;  \right)$.
			\itemch Đúng. Mặt phẳng $\left( OAB \right)$ có một vectơ pháp tuyến là $\overrightarrow{n}=\left( -1; 1 ; 1 \right)$ và mặt phẳng $\left( P \right)$ có một vectơ pháp tuyến là $\overrightarrow{n_1}=\left( 2; 1; 2 \right)$.\\ $\Rightarrow \cos \left( \left( OAB \right),\left( P \right) \right)=\dfrac{\left| 2\cdot\left( -1 \right)+1\cdot1+2\cdot1 \right|}{3\cdot\sqrt{3}}=\dfrac{\sqrt{3}}{9}.$
		\end{itemchoice}
	}
\end{ex}

%G:\My Drive\CODE12-2024\DE-ON-THEO BAI\2H5-TACH DE\Bai3-De2.tex
\begin{ex}%[2H5N2-7]
	Trong không gian $Oxyz$, cho đường thẳng $d\colon\dfrac{x}{1}=\dfrac{y}{-2}=\dfrac{z}{1}$ và mặt phẳng $\left( P \right)\colon5x+11y+2z-4=0$. Xét tính đúng sai của các khẳng định sau
	\choiceTF
	{\True Đường thẳng $d$ có một vectơ chỉ phương là $\overrightarrow{u}=\left( 1; -2; 1 \right)$}
	{Mặt phẳng $\left( P \right)$ có một vectơ pháp tuyến là $\overrightarrow{n}=\left( -5; -11; 2 \right)$}
	{Giá trị cosin của góc giữa đường thẳng $d$ và mặt phẳng $\left( P \right)$ bằng $\dfrac{1}{2}$}
	{\True Góc giữa đường thẳng $d$ và mặt phẳng bằng $30^\circ $}
	\loigiai{
		\begin{itemchoice}
			\itemch Đúng. Đường thẳng $d$ có một vectơ chỉ phương là $\overrightarrow{u}=\left( 1; -2; 1 \right)$.
			\itemch Sai. Mặt phẳng $\left( P \right)$ có một vectơ pháp tuyến là $\overrightarrow{n}=\left( 5; 11; 2 \right)$.
			\itemch Sai. Ta có $\sin \left( d, \left( P \right) \right)=\dfrac{\left| 1\cdot5+\left( -2 \right)\cdot11+1\cdot2 \right|}{\sqrt{6}\cdot5\sqrt{6}}=\dfrac{1}{2}\Rightarrow \cos \left( d, \left( P \right) \right)=\dfrac{\sqrt{3}}{2}$.
			\itemch Đúng. Ta có $\sin \left( d, \left( P \right) \right)=\dfrac{\left| 1\cdot5+\left( -2 \right)\cdot11+1\cdot2 \right|}{\sqrt{6}\cdot5\sqrt{6}}=\dfrac{1}{2}\Rightarrow \left( d, \left( P \right) \right)=30^\circ $.
		\end{itemchoice}
	}
\end{ex}

%G:\My Drive\CODE12-2024\DE-ON-THEO BAI\2H5-TACH DE\Bai3-De2.tex
\begin{ex}%[2H5N2-7]
	Trong không gian $Oxyz$, cho mặt phẳng $\left( P \right)\colon3x+4y+5z+2=0$ và đường thẳng $d$ là giao tuyến của hai mặt phẳng $\left( \alpha \right)\colon x-2y+1=0$ và $\left( \beta \right)\colon x-2z-3=0$. Xét tính đúng sai của các khẳng định sau
	\choiceTF
	{\True Mặt phẳng $\left( P \right)$ có một vectơ pháp tuyến là $\overrightarrow{n}=\left( 3; 4; 5 \right)$}
	{Góc giữa đường thẳng $d$ và mặt phẳng $\left( P \right)$ bằng $30^\circ $}
	{Đường thẳng $d$ có một vectơ chỉ phương là $\overrightarrow{u}=\left( 2\,;-1\,;1 \right)$}
	{\True Đường thẳng $d$ cắt mặt phẳng $\left( P \right)$ tại $A\left( \dfrac{7}{15}; \dfrac{11}{15}; -\dfrac{19}{15} \right)$}
	\loigiai{
		\begin{itemchoice}
			\itemch Đúng. Mặt phẳng $\left( P \right)$ có một vectơ pháp tuyến là $\overrightarrow{n}=\left( 3; 4; 5 \right)$.
			\itemch Sai. Ta có $\sin \left( d, \left( P \right) \right)=\dfrac{\left| 4\cdot3+2\cdot4+2\cdot5 \right|}{2\sqrt{6}\cdot5\sqrt{2}}=\dfrac{\sqrt{3}}{2}\Rightarrow \sin \left( d, \left( P \right) \right)=60^\circ $. 
			\itemch Sai.  Mặt phẳng $\left( \alpha \right)$ có một vectơ pháp tuyến là $\overrightarrow{n_1}=\left( 1; -2; 0 \right)$.\\
			Mặt phẳng $\left( \beta \right)$ có một vectơ pháp tuyến là $\overrightarrow{n_2}=\left( 1; 0; -2 \right)$.\\
			$\Rightarrow d$ có một vectơ chỉ phương là $\overrightarrow{u}=\left[ \overrightarrow{n_1}, \overrightarrow{n_2} \right]=\left( 4; 2; 2 \right)$.
			\itemch Đúng. Ta có\\
			$\heva{
				& x-2y+1=0 \\
				& x-2z-3=0 \\
				& 3x+4y+5z+2=0 \\
			}\Leftrightarrow \heva{
				& x=\dfrac{7}{15} \\
				& y=\dfrac{11}{15} \\
				& z=-\dfrac{19}{15} \\
			}$.\\ Suy ra đường thẳng $d$ cắt mặt phẳng $\left( P \right)$ tại $A\left( \dfrac{7}{15}; \dfrac{11}{15}; -\dfrac{19}{15} \right)$.
		\end{itemchoice}
	}
\end{ex}

\Closesolutionfile{ans}
\TNSA

	\setcounter{ex}{0}
	\Opensolutionfile{ans}[ans/B3-De2-3]
	
\begin{ex}%[2H5H2-7]
	Trong hệ tọa độ $Oxyz$, một vật chuyển động theo quĩ đạo là một đường thẳng. Tại thời điểm ban đầu, vật ở vị trí điểm $A(1;5;0)$, sau $10$ phút vật ở vị trí điểm $B(101;205;1250)$. Hỏi vật chuyển động theo phương hợp với mặt đất góc bao nhiêu độ ( giả sử mặt đất là mặt phẳng $Oxy$, kết quả làm tròn đến hàng phần chục).\\
	\shortans[oly]{$79{,}9$}
	\loigiai{
		Ta có $\overrightarrow{AB}=(100;200;1250)$.\\
		$\sin\left( AB,\left( Oxy \right) \right)=\dfrac{\left| \overrightarrow{AB}\cdot{{\overrightarrow{n}}_{Oxy}} \right|}{\left| \overrightarrow{AB} \right|\cdot\left| {{\overrightarrow{n}}_{Oxy}} \right|}=\dfrac{1250}{\sqrt{{{100}^2}+{{200}^2}+{{1250}^2}}}$ nên $\widehat{\left( AB;(Oxy) \right)}\approx 79,9^0$.}
\end{ex}

\begin{ex}%[2H5H2-7]
	Cho hình lăng trụ tam giác đều $ABC.A^\prime B^\prime C^\prime $ có cạnh bên $2a$, góc tạo bởi $A^\prime B$ và mặt đáy là $60^\circ$. Gọi $M$ là trung điểm $BC$. Ta có $\cos\left( A^\prime C, AM \right)=\dfrac{\sqrt{a}}{b}$ với $\dfrac{a}{b}$ là phân số tối giản, $a,b\in N$. Tổng $a+b$ bằng bao nhiêu?\\
	\shortans[oly]{$7$}
	\loigiai{
		\immini{	Chọn hệ trục tọa độ như hình vẽ.\\
			Ta có $AB=AC=BC=\dfrac{2a}{\tan 60^\circ}=\dfrac{2a}{\sqrt{3}}$.\\$\Rightarrow MC=\dfrac{BC}{2}=\dfrac{a}{\sqrt{3}}$.\\
			$AM=\dfrac{AB\sqrt{3}}{2}=a$.\\Khi đó: $M\left( 0; 0; 0 \right)$, $A\left( 0; a; 0 \right)$, $C\left( \dfrac{a}{\sqrt{3}}; 0; 0 \right)$, $A^\prime \left( 0; a; 2a \right)$.\\
			Ta có $\overrightarrow{A^\prime C}=\left( \dfrac{a}{\sqrt{3}} ;-a; -2a \right)$ $\Rightarrow A^\prime C=\dfrac{4a}{\sqrt{3}}$.\\
			$\overrightarrow{AM}=\left( 0; -a; 0 \right)$ $\Rightarrow AM=a$.\\
			Khi đó có $\cos \left( A^\prime C, AM \right)=\dfrac{\left| \overrightarrow{A^\prime C}\cdot\overrightarrow{AM} \right|}{\left| \overrightarrow{A^\prime C} \right|\cdot\left| \overrightarrow{AM} \right|}=\dfrac{\sqrt{3}}{4}$.\\ Vậy $a+b=7$.}{	\begin{tikzpicture}[scale=0.7, font=\footnotesize,line join=round, line cap=round, >=stealth]
				\path
				(0:0) coordinate (B)
				(-45:3) coordinate (C)
				(0:5) coordinate (A)
				\foreach \x in {A,B,C}{(\x)+(90:5.5) coordinate (\x')}
				($(B)!0.5!(C)$) coordinate (M)
				($(B')!0.5!(C')$) coordinate (M')
				($(M)!1.5!(M')$) coordinate (T1)
				($(M)!1.5!(C)$) coordinate (T2)
				($(M)!1.5!(A)$) coordinate (T3)
				;
				\draw[dashed] (A)--(B) (A)--(M) (C)--(A')--(B);
				\draw 
				(A)--(C) (B')--(B)--(C)
				(A')--(C')--(B')--cycle  (C)--(C') (A)--(A') 
				;
				\draw[thick,->] (M)--(T1) ;
				\draw[thick,->] (A)--(T3) ;
				\draw[thick,->] (C)--(T2) ;
				\foreach \x/\g in {A/-10,B/-90,C/180,A'/0,B'/180,C'/90,M/180}\draw[fill=white] (\x) circle (.03) +(\g:.3) node{$\x$};
				\draw[right]  (T1) node{$z$} (T3) node{$y$} (T2) node{$x$};
		\end{tikzpicture}}
	}
\end{ex}

\begin{ex}%[2H5H2-7]
	Trong không gian $Oxyz$, cho đường thẳng $d\colon\dfrac{x+1}{2}=\dfrac{y-1}{2}=\dfrac{z+2}{1}$ và mặt phẳng $(P)\colon3x+my-1=0$ ($m$ là tham số ).Tìm $m$ để đường thẳng $d$ tạo với mặt phẳng $\left( P \right)$ góc $\alpha $ thỏa mãn $\sin\alpha =\dfrac{2}{3}$.\\
	\shortans[oly]{$0$}
	\loigiai{
		Ta có đường thẳng $d$ có véc tơ chỉ phương $\overrightarrow{u_d}=(2; 2; 1)$.\\Mặt phẳng $\left(P\right)$ có véc tơ pháp tuyến  $\overrightarrow{n}_P=(3; m; 0)$.\\ Suy ra $\sin\left( d,\left( P \right) \right)=\dfrac{\left| 6+2m \right|}{3\cdot\sqrt{9+m^2}}$.\\
		Theo giả thiết ta có
		\begin{eqnarray*}
			& & \sin\left( d,\left( P \right) \right)=\dfrac{2}{3}\\
			&\Leftrightarrow & \dfrac{\left| 6+2m \right|}{3\cdot\sqrt{9+m^2}}=\dfrac{2}{3}\\
			&\Leftrightarrow & 4m^2+24m+36=4m^2+36 \\
			&\Leftrightarrow & m=0.
		\end{eqnarray*}
	}
\end{ex}

\begin{ex}%[2H5H2-7]
	Trong không gian, cho mặt phẳng $(P)$ có phương trình $ax+by+cz-1=0$ với $c<0$ đi qua hai điểm $A(0; 1; 0)$, $B(1; 0; 0)$ tạo với $\left( Oyz \right)$ một góc $60^\circ$. Khi đó $a+b-\sqrt{2}c$ bằng\\
	\shortans[oly]{$4$}
	\loigiai{
		Mặt phẳng $\left( P \right)$ đi qua $A$, $B$ nên $a=b=1\quad (1)$ .\\
		Ta có $\cos ((P),(Oyz))=\dfrac{\left| a \right|}{\sqrt{a^2+b^2+c^2}\cdot\sqrt{1}}=\dfrac{1}{2}\quad(2)$.\\
		Thay (1) vào (2) ta được:
		\begin{eqnarray*}
			& & \dfrac{1}{\sqrt{2+c^2}}=\dfrac{1}{2}\\
			&\Leftrightarrow & \sqrt{2+c^2}=2\\
			&\Leftrightarrow & c^2=2\\
			&\Leftrightarrow & \hoac{&c=\sqrt{2} \quad\left(\text{Loại}\right)\\&c=-\sqrt{2}}\\
			&\Leftrightarrow & c=-\sqrt{2}
		\end{eqnarray*}
		Vậy $a+b-\sqrt{2}c=4$.}
\end{ex}

\begin{ex}%[2H5H2-7]
	Có hai bức tường hình vuông cạnh $5m$, vuông góc với nhau và cùng vuông góc với mặt đất, hai mặt tường giao nhau tại cột $d$. Trên cột $d$ có một điểm $A$ cách mặt đất $2m$. Có một chiếc cột cao $1m$ đặt vuông góc với mặt đất, khoảng cách từ chân cột đến mỗi bức tường là $1$m. Người ta muốn căng một chiếc bạt phẳng hình tam giác đi qua điểm $A$ và đầu cột, hai đầu mút $M$, $N$ thuộc hai chân tường sao cho diện tích bạt bé nhất. Hỏi phải căng chiếc bạt hợp với mặt đất góc bao nhiêu độ ( Kết quả làm tròn đến hàng phần chục).\\
	\shortans[oly]{$54{,}7$}
	\loigiai{
		\immini{	Đặt hệ trục $Oxyz$ như hình vẽ.\\
			Ta có $A(0;0;2)$, $I(1;1;1)$. Gọi $M(m;0;0)$; $N(0;n;0)$ với $m>0$, $n>0$.\\
			Phương trình mặt phẳng $(AMN)\colon\dfrac{x}{m}+\dfrac{y}{n}+\dfrac{z}{2}=1$.\\
			Vì mặt phẳng $\left( AMN \right)$ đi qua điểm $I(1;1;1)$ nên ta có $\dfrac{1}{m}+\dfrac{1}{n}+\dfrac{1}{2}=1\Rightarrow \dfrac{1}{m}+\dfrac{1}{n}=\dfrac{1}{2}$.\\
			Ta có mặt phẳng $\left(AMN\right)$ có véc tơ pháp tuyến ${\overrightarrow{n}}_{\left( AMN \right)}=\left( \dfrac{1}{m};\dfrac{1}{n};\dfrac{1}{2} \right)$,\\mặt phẳng $\left(OMN\right)$ có véc tơ pháp tuyến${{\overrightarrow{n}}_{\left( OMN \right)}}=\left( 0;0;1 \right)$.\\$\Rightarrow \cos\left( \left( AMN \right),\left( OMN \right) \right)=\dfrac{\dfrac{1}{2}}{\sqrt{\dfrac{1}{m^2}+\dfrac{1}{n^2}+\dfrac{1}{4}}}$.}{	\begin{tikzpicture}[scale=0.7, font=\footnotesize,line join=round, line cap=round, >=stealth]
				\path
				(0:0) coordinate (D)
				(0:4) coordinate (C)
				(-135:4) coordinate (A)
				+(C) coordinate (B)
				\foreach \x in {A,B,C,D}{(\x)+(90:5.5) coordinate (\x')}
				($(D)!0.8!(C)$) coordinate (N)
				($(D)!0.5!(C)$) coordinate (T1)
				($(D)!0.8!(A)$) coordinate (M)
				($(D)!0.5!(A)$) coordinate (T2)+(T1) coordinate (E)
				(E)+(90:2.5) coordinate (I) (D)+(90:4) coordinate (T3)
				;
				\foreach \x/\g in {M/135,N/90,I/90}\draw[fill=white] (\x) circle (.03) +(\g:.4) node{$\x$};
				\draw (T1)--(E)--(T2) (I)--(E);
				\draw [->](D)--(C);
				\draw [->](D)--(A);
				\draw [->](D)--(D');
				\draw[right] (C) node{$y$} (D') node{$z$} (A) node{$x$} ;
				\draw[fill=white] (T3) circle (.03) +(180:.4) node{$A$};
		\end{tikzpicture}			}
		Mà ${S_{AMN}}\cdot \cos\left( \left( AMN \right),\left( OMN \right) \right)={S_{OMN}}\Rightarrow {S_{AMN}}=mn\sqrt{\dfrac{1}{m^2}+\dfrac{1}{n^2}+\dfrac{1}{4}}=\dfrac{\sqrt{\dfrac{1}{m^2}+\dfrac{1}{n^2}+\dfrac{1}{4}}}{\dfrac{1}{m}\cdot\dfrac{1}{n}}$.
		Đặt $\dfrac{1}{m}=a$, $\dfrac{1}{n}=b$ thì $a+b=\dfrac{1}{2}$ và ${S_{AMN}}=\sqrt{\dfrac{a^2+b^2+\dfrac{1}{4}}{a^2b^2}}=\sqrt{\dfrac{2a^2+2b^2+2ab}{a^2{b^2}}}$.\\
		Theo bất đẳng thức Cosi ta có $2a^2+2b^2+2ab\ge 3\sqrt[3]{8a^3{b^3}}=6ab$.\\
		Suy ra ${S_{AMN}}\ge \sqrt{\dfrac{6}{ab}}$.\\ Mà $a+b\ge 2\sqrt{ab}\Rightarrow \dfrac{1}{2}\ge 2\sqrt{ab}\Rightarrow ab\le \dfrac{1}{16}$ nên ${S_{AMN}}\ge \sqrt{96}$.\\
		Dấu bằng xảy ra khi $a=b=\dfrac{1}{4}$.\\ Thay vào (*) tc có $\cos\left( \left( AMN \right),\left( OMN \right) \right)=\dfrac{1}{\sqrt{3}}$ nên $\widehat{\left( \left( AMN \right),\left( OMN \right) \right)}\approx 54,7^\circ$\\
	}
\end{ex}

\begin{ex}%[2H5H2-7]
	Trong không gian, tìm $m$ để số đo góc giữa hai đường thẳng $d_1$, $d_2$ bằng $60^\circ$ biết $d_1\colon\heva{
		& x=1+t \\
		& y=1-t \\
		& z=-3+\sqrt{2}t \\
	}$, $d_2 \colon\heva{
		& x=2+mt \\
		& y=3+t \\
		& z=\sqrt{2}t \\
	}$.\\
	\shortans[oly]{$1$}
	\loigiai{
		Ta có $\cos \alpha =\cos 60^\circ=\dfrac{\left| 1\cdot m-1\cdot1+2 \right|}{\sqrt{1+1+2}\sqrt{m^2+1+2}}\Leftrightarrow \left| m-1 \right|=\sqrt{m^2+1}\Leftrightarrow m=1$.}
\end{ex}

\centerline{---HẾT---}
\Closesolutionfile{ans}
%\newpage
%%=====================
%\begin{center}
%\textbf{\large BẢNG ĐÁP ÁN}
%\end{center}
%\noindent\textbf{ĐÁP ÁN PHẦN I}
%\inputansbox{10}{ans/B3-De2-1}
	
%\noindent\textbf{ĐÁP ÁN PHẦN II}
%\inputansbox[2]{2}{ans/B3-De2-2}
	
%\noindent\textbf{ĐÁP ÁN PHẦN III}
%\inputansbox[3]{6}{ans/B3-De2-3}



