\subsection{BÀI TẬP TRẮC NGHIỆM TỰ LUYỆN}
\subsection*{\indam{PHẦN I. Câu trắc nghiệm nhiều phương án lựa chọn. Thí sinh trả lời từ câu 1 đến câu 12. Mỗi câu hỏi thí sinh chỉ chọn một phương án.}} 
	\setcounter{ex}{0}
	\Opensolutionfile{ans}[ans/B2-De2-1]
\begin{ex}%[2H5N2-2]
	Trong không gian với hệ trục tọa độ $O x y z$, cho hai điểm $A(0 ;-1 ;-2)$ và $B(2 ; 2 ; 2)$. Véc-tơ $\overrightarrow{a}$ nào dưới đây là một véc-tơ chỉ phương của đường thẳng $A B$?
	\choice
	{$\overrightarrow{a}=(-2 ; 1 ; 0)$}
	{$\overrightarrow{a}=(2 ; 3 ; 0)$}
	{$\overrightarrow{a}=(2 ; 1 ; 0)$}
	{\True $\overrightarrow{a}=(2 ; 3 ; 4)$}
	\loigiai{
		Ta có $\overrightarrow{A B}=(2 ; 3 ; 4)$ nên đường thẳng $A B$ có một véc-tơ chỉ phương là $\overrightarrow{a}=(2 ; 3 ; 4)$.}
\end{ex}

%G:\My Drive\CODE12-2024\DE-ON-THEO BAI\2H5-TACH DE\Bai2-De2.tex
\begin{ex}%[2H5H2-1]
	Đường thẳng $(\Delta)\colon \dfrac{x-1}{2}=\dfrac{y+2}{1}=\dfrac{z}{-1}$ \textbf{không} đi qua điểm nào dưới đây?
	\choice
	{$C(3 ;-1 ;-1)$}
	{$D(1 ;-2 ; 0)$}
	{\True $A(-1 ; 2 ; 0)$}
	{$B(-1 ;-3 ; 1)$}
	\loigiai{
		Ta có $\dfrac{-1-1}{2} \neq \dfrac{2+2}{1} \neq \dfrac{0}{-1}$ nên điểm $A(-1 ; 2 ; 0)$ không thuộc đường thẳng $(\Delta)$.
	}
\end{ex}

%G:\My Drive\CODE12-2024\DE-ON-THEO BAI\2H5-TACH DE\Bai2-De2.tex
\begin{ex}%[2H5H2-3]
	Cho đường thẳng $\Delta$ đi qua điểm $M(2 ; 0 ;-1)$ và có một véc-tơ chỉ phương $\overrightarrow{a}=(4 ;-6 ; 2)$. Phương trình tham số của đường thẳng $\Delta$ là
	\choice
	{\True $\heva{&x=2+2 t \\& y=-3 t \\& z=-1+t}$}
	{$\heva{&x=-2+4 t \\& y=-6 t \\& z=1+2 t}$}
	{$\heva{&x=4+2 t \\& y=-3 t \\& z=2+t}$}
	{$\heva{&x=-2+2 t \\& y=-3 t \\& z=1+t}$}
	\loigiai{
		Véc-tơ chỉ phương $\overrightarrow{a}=(4 ;-6 ; 2)=2(2 ;-3 ; 1)$ nên đường thẳng $\Delta$ có phương trình tham số là $\heva{&x=2+2 t \\& y=-3 t \\& z=-1+t.}$
	}
\end{ex}

%G:\My Drive\CODE12-2024\DE-ON-THEO BAI\2H5-TACH DE\Bai2-De2.tex
\begin{ex}%[2H5N2-2]
	Trong không gian $Oxyz$, cho đường thẳng $d\colon \dfrac{x-1}{2}=\dfrac{y}{-1}=\dfrac{z-1}{-3}$. Một véc-tơ chỉ phương của đường thẳng $d$ là
	\choice
	{\True $\overrightarrow{u}_3=(2 ; -1 ; -3)$}
	{$\overrightarrow{u}_4=(-2 ;-1 ; 3)$}
	{$\overrightarrow{u}_1=(2 ;-1 ; 3)$}
	{ $\overrightarrow{u}_2=(1 ; 0 ; 1)$}
	\loigiai{
		Một véc-tơ chỉ phương của đường thẳng $d$ là $\overrightarrow{u}_3=(2 ;-1 ;-3)$.
	}
\end{ex}

%G:\My Drive\CODE12-2024\DE-ON-THEO BAI\2H5-TACH DE\Bai2-De2.tex
\begin{ex}%[2H5H2-3]
	Trong không gian với hệ trục tọa độ $O x y z$, cho mặt phẳng $(P)$ có phương trình là $2 x+y-5 z+6=0$ . Phương trình đường thẳng $d$ đi qua điểm $M(1 ;-2 ; 7)$ và  vuông góc với $(P)$ là
	\choice
	{$d\colon  \dfrac{x+1}{2}=\dfrac{y-2}{-1}=\dfrac{z+7}{-5}$}
	{$d\colon  \dfrac{x-1}{2}=\dfrac{y-2}{1}=\dfrac{z-7}{-5}$}
	{\True $d\colon  \dfrac{x-1}{2}=\dfrac{y+2}{1}=\dfrac{z-7}{-5}$}
	{$d\colon  \dfrac{x-2}{1}=\dfrac{y-1}{-2}=\dfrac{z+5}{7}$}
	\loigiai{
		Ta có $d$ vuông góc với $(P)$ nên có véc-tơ chỉ phương là $\overrightarrow{u}=(2 ; 1 ;-5)$.\\
		Kết hợp với $d$ đi qua điểm $M(1 ;-2 ; 7)$ nên $d\colon  \dfrac{x-1}{2}=\dfrac{y+2}{1}=\dfrac{z-7}{-5}$.
	}
\end{ex}

%G:\My Drive\CODE12-2024\DE-ON-THEO BAI\2H5-TACH DE\Bai2-De2.tex
\begin{ex}%[2H5H2-4]
	Trong không gian với hệ trục tọa độ $O x y z$, cho hai đường thẳng $d_1\colon \dfrac{x-1}{2}=\dfrac{y-2}{3}=\dfrac{z-3}{4}$ và $d_2\colon\heva{&x=1+t \\& y=2+2 t \\& z=3-2 t}$. Mệnh đề nào sau đây đúng?
	\choice
	{\True $d_1$ và $d_2$  vừa cắt nhau vừa vuông góc}
	{$d_1$ và $d_2$ không vuông góc và không cắt nhau}
	{$d_1$ và $d_2$ cắt nhau nhưng không vuông góc}
	{$d_1$ và $d_2$ vuông góc nhưng không cắt nhau}
	\loigiai{
		Chọn $M(1 ; 2 ; 3), $ $N(0 ; 0 ; 5)$ là hai điểm lần lượt thuộc đường thẳng $d_1$ và $d_2$.\\
		Ta có $\overrightarrow{u}_{d_1}=(2 ; 3 ; 4)$ và $\overrightarrow{u}_{d_2}=(1 ; 2 ;-2)$ nên $\overrightarrow{u}_{d_1} \cdot \overrightarrow{u}_{d_2}=0$ nên $d_1 \perp d_2$.\\
		Mặt khác, ta có $\left[\overrightarrow{u}_{d_1} ; \overrightarrow{u}_{d_2}\right] \overrightarrow{M N}=0$ nên $d_1$ cắt $d_2$.\\
		Vậy hai đường thẳng vừa vuông góc, vừa cắt nhau.
	}
\end{ex}

%G:\My Drive\CODE12-2024\DE-ON-THEO BAI\2H5-TACH DE\Bai2-De2.tex
\begin{ex}%[2H5N2-2]
	Trong không gian với hệ trục tọa độ $Oxyz$, véc-tơ nào dưới đây là véc-tơ chỉ phương của trục $Oz$?
	\choice
	{$\overrightarrow{m}=(1 ; 1 ; 1)$}
	{\True $\overrightarrow{k}=(0 ; 0 ; 1)$}
	{$\overrightarrow{i}=(1 ; 0 ; 0)$}
	{$\overrightarrow{j}=(0 ; 1 ; 0)$}
	\loigiai{
		Trục $O z$ có một vectơ chỉ phương là $\overrightarrow{k}=(0 ; 0 ; 1)$.}
\end{ex}

%G:\My Drive\CODE12-2024\DE-ON-THEO BAI\2H5-TACH DE\Bai2-De2.tex
\begin{ex}%[2H5N2-1]
	Đường thẳng $(\Delta)\colon \dfrac{x-1}{2}=\dfrac{y+2}{1}=\dfrac{z}{-1}$ đi qua điểm nào dưới đây?
	\choice
	{$Q(-1 ;-2 ; 0)$}
	{$N(-1 ; 2 ; 0)$}
	{$P(3 ; 1 ;-1)$}
	{\True $M(1 ;-2 ; 0)$}
	\loigiai{
		Ta có $\dfrac{1-1}{2}=\dfrac{2-2}{1}=\dfrac{0}{-1}$ nên điểm $M(1 ;-2 ; 0)$ thuộc đường thẳng $(\Delta)$.
	}
\end{ex}

%G:\My Drive\CODE12-2024\DE-ON-THEO BAI\2H5-TACH DE\Bai2-De2.tex
\begin{ex}%[2H5H2-7]
	Đường thẳng $d\colon \dfrac{x-1}{2}=\dfrac{y+1}{-1}=\dfrac{z+3}{-1}$ vuông góc với đường thẳng nào dưới đây?
	\choice
	{$d_1\colon \heva{&x=2-3 t \\& y=-2 t \\& z=1+5 t}$}
	{\True $d_4\colon \heva{&x=1-3 t \\& y=2-t \\& z=5-5 t}$}
	{$d_3\colon \heva{&x=2+3 t \\& y=3-t \\& z=5 t}$}
	{$d_2\colon \heva{&x=2 \\& y=3-3 t \\& z=1+t}$}
	\loigiai{
		Đường thẳng $d$ có véc-tơ chỉ phương $\overrightarrow{u}=(2 ;-1 ;-1)$.\\
		Các đường thẳng $d_1,$ $ d_2, $ $d_3,$ $ d_4$ lần lượt có véc-tơ chỉ phương là
		$\overrightarrow{u}_1=(-3 ;-2 ; 5),$ $ \overrightarrow{u}_2=(0 ;-3 ; 1),$ $ \overrightarrow{u}_3=(3 ;-1 ; 5)$ và $\overrightarrow{u}_4=(-3 ;-1 ;-5)$.\\
		Vì $\overrightarrow{u} \cdot \overrightarrow{u}_4=0$ nên $d \perp d_4$.
	}
\end{ex}

%G:\My Drive\CODE12-2024\DE-ON-THEO BAI\2H5-TACH DE\Bai2-De2.tex
\begin{ex}%[2H5H2-5]
	Trong không gian với hệ trục tọa độ $Oxyz$, cho đường thẳng $d$ có véc-tơ chỉ phương $\overrightarrow{u}$ và mặt phẳng $(P)$ có véc-tơ pháp tuyến $\overrightarrow{n}$. Mệnh đề nào dưới đây đúng?
	\choice
	{$d$ song song với $(P)$ thì $\overrightarrow{u}$ cùng phương với $\overrightarrow{n}$}
	{$\overrightarrow{u}$ vuông góc với $\overrightarrow{n}$ thì $d$ song song với $(P)$}
	{\True $\overrightarrow{u}$ không vuông góc với $\overrightarrow{n}$ thì $d$ cắt $(P)$}
	{$d$ vuông góc với $(P)$ thì $\overrightarrow{u}$ vuông góc với $\overrightarrow{n}$}
	\loigiai{
		Ta có $\overrightarrow{u}$ không vuông góc với $\overrightarrow{n}$ thì $d$ cắt $(P)$.}
\end{ex}

%G:\My Drive\CODE12-2024\DE-ON-THEO BAI\2H5-TACH DE\Bai2-De2.tex
\begin{ex}%[2H5H2-3]
	Cho đường thẳng $d$ có phương trình tham số $\heva{&x=1+2 t \\& y=2-t \\& z=-3+t}$. Viết phương trình chính tắc của đường thẳng $d$.
	\choice
	{$d\colon  \dfrac{x-1}{2}=\dfrac{y-2}{-1}=\dfrac{z-3}{1}$}
	{$d\colon  \dfrac{x+1}{2}=\dfrac{y+2}{-1}=\dfrac{z-3}{1}$}
	{$d\colon \dfrac{x-1}{2}=\dfrac{y-2}{1}=\dfrac{z+3}{1}$}
	{\True $d\colon  \dfrac{x-1}{2}=\dfrac{y-2}{-1}=\dfrac{z+3}{1}$}
	\loigiai{
		Từ phương trình tham số ta thấy đường thẳng $d$ đi qua điểm tọa độ $(1 ; 2 ;-3)$ và có véc-tơ chỉ phương  $\overrightarrow{u}=(2 ;-1 ; 1)$.\\
		Suy ra phương trình chính tắc của $d$ là $ \dfrac{x-1}{2}=\dfrac{y-2}{-1}=\dfrac{z+3}{1}$.
	}
\end{ex}

%G:\My Drive\CODE12-2024\DE-ON-THEO BAI\2H5-TACH DE\Bai2-De2.tex
\begin{ex}%[2H5V2-3]
	Trong không gian $O x y z$, cho đường thẳng $d\colon \dfrac{x+3}{2}=\dfrac{y+1}{1}=\dfrac{z}{-1}$ và mặt phẳng $(P)\colon x+y-3 z-2=0$. Gọi $d'$ là đường thẳng nằm trong mặt phẳng $(P)$, cắt và vuông góc với $d$. Đường thẳng $d'$ có phương trình là $\dfrac{x+1}{a}=\dfrac{y}{5}=\dfrac{z+1}{c}$. Tính $S=a-c$.
	\choice
	{$3$}
	{$-7 $}
	{\True $-3$}
	{$4$}
	\loigiai{
		Phương trình tham số của $d\colon\heva{&x=-3+2 t \\& y=-1+t \\& z=-t.}$\\
		Tọa độ giao điểm của $d$ và $(P)$ là nghiệm của hệ
		\allowdisplaybreaks
		\begin{eqnarray*}
			&&\heva{&x=-3+2 t \\& y=-1+t \\& z=-t \\& x+y-3 z-2=0} \Leftrightarrow \heva{&x=-3+2 t \\& y=-1+t \\& z=-t \\& -3+2 t-1+t+3 t-2=0} \\&\Leftrightarrow&\heva{&t=1 \\& x=-1 \\& y=0 \\& z=-1} \Rightarrow d \cap(P)=M(-1 ; 0 ;-1).
		\end{eqnarray*}
		Theo đề bài, đường thẳng $d$ có véc-tơ chỉ phương $u_d=(2;1;-1)$, mặt phẳng $(P)$ có véc-tơ pháp tuyến $n_{(P)}=(1;1;-3)$.\\
		Vì $d'$ nằm trong mặt phẳng $(P)$, cắt và vuông góc với $d$ nên $d'$ đi qua $M$ và có véc-tơ chỉ phương $\overrightarrow{u}_{d'}=\left[\overrightarrow{n}_{(P)}, \overrightarrow{u}_d\right]=(2 ;-5 ;-1)$ hay $d'$ nhận véc-tơ $\overrightarrow{v}=(-2 ; 5 ; 1)$ làm véc tơ chỉ phương.\\
		Phương trình của $d'\colon  \dfrac{x+1}{-2}=\dfrac{y}{5}=\dfrac{z+1}{1}$.\\
		Do đó $S=a-c=-2-1=-3$.
} \end{ex}

	\Closesolutionfile{ans}

\subsection*{\indam{PHẦN II. Câu trắc nghiệm đúng sai. Thí sinh trả lời từ câu 1 đến câu 4. Trong mỗi ý a), b), c), d) ở mỗi câu, thí sinh chọn đúng hoặc sai.}}
	\setcounter{ex}{0}
	\Opensolutionfile{ans}[ans/B2-De2-2]

\begin{ex}%[2H5H2-5]
	Trong không gian $O x y z$ cho đường thẳng $d$ có phương trình tham số $\heva{&x=-1+2 t \\& y=1+t \\& z=3-2 t.}$
	\choiceTF
	{Giao điểm của đt $d$ và mặt phẳng $(P)\colon x+2 y-3 z+2=0$  là $I(0 ; 1 ; 2)$}
	{Véc-tơ $\overrightarrow{a}=(4 ; 2 ;-3)$ là một véc-tơ chỉ phương   của đường thẳng $d$}
	{\True  Đường thẳng $d$ đi qua điểm $A(-1 ; 1 ; 3)$}
	{\True Phương trình chính tắc của đường thẳng $d$ là $\dfrac{x+1}{2}=\dfrac{y-1}{1}=\dfrac{z-3}{-2}$}
	\loigiai{
		\begin{itemchoice}
			\itemch \textbf{Sai.} Vì  ta có $-1+2 t+2(1+t)-3(3-2 t)+2=0$ $\Leftrightarrow 10 t-6=0 \Leftrightarrow t=\dfrac{3}{5}$.\\
			Giao điểm của $d$ và $(P)$ là $B\left(\dfrac{1}{5} ; \dfrac{8}{5} ; \dfrac{9}{5}\right)$.\itemch \textbf{Sai.} Vì véc-tơ chỉ phương của đường thẳng $d$ là $\overrightarrow{u}_d=(2 ; 1 ;-2)$. \\Xét hai véc-tơ $\overrightarrow{a}=(4 ; 2 ;-3)$ và $\overrightarrow{u}_d=(2 ; 1 ;-2)$.\\
			Vì $\dfrac{1}{2} \neq \dfrac{-2}{-3}$ nên $\overrightarrow{a}=(4 ; 2 ;-3)$ và $\overrightarrow{u}_d$ không cùng phương. \\Do đó $\overrightarrow{a}$ không là véc-tơ chỉ phương của đường thẳng $d$.
			\itemch \textbf{Đúng.} Vì theo phương trình tham số của đường thẳng $d$ thì $d$ đi qua điểm $A(-1 ; 1 ; 3)$.
			\itemch \textbf{Đúng.} Vì  đường thẳng $d$ đi qua điểm $M(-1 ; 1 ; 3)$ và có một  véc-tơ chỉ phương   là $\overrightarrow{u}_d=(2 ; 1 ;-2)$ nên có phương trình chính tắc là $\dfrac{x+1}{2}=\dfrac{y-1}{1}=\dfrac{z-3}{-2}$.
		\end{itemchoice}
	}
\end{ex}

%G:\My Drive\CODE12-2024\DE-ON-THEO BAI\2H5-TACH DE\Bai2-De2.tex
\begin{ex}%[2H5H2-7]
	Trong không gian $O x y z$ cho đường thẳng $d\colon \dfrac{x-1}{2}=\dfrac{y+1}{-1}=\dfrac{z}{1}$ và mặt phẳng $(P)\colon x+y+2 z-3=0$.
	\choiceTF
	{\True Đường thẳng $d'$ đi qua điểm $A(1 ; 0 ;-1)$ và vuông góc với mặt phẳng $(P)$. Phương trình tham số của đường thẳng $d'$ là $\heva{&x=1+t \\& y=t \\& z=-1+2 t}$}
	{ Đường thẳng $d$ có một véc-tơ chỉ phương là $\overrightarrow{a}=(1 ;-1 ; 0)$}
	{ Đường thẳng $d$ đi qua điểm $M(2 ;-1 ; 1)$}
	{\True Góc giữa đường thẳng $d$ và mặt phẳng $(P)$ bằng $30^{\circ}$}
	\loigiai{
		\begin{itemchoice}
			\itemch \textbf{Đúng.} Vì đường thẳng $d'$ vuông góc với mặt phẳng $(P)$ nên có một véc-tơ chỉ phương là $\overrightarrow{u}_{d'}=\overrightarrow{n}_P=(1 ; 1 ; 2)$.\\
			Kết hợp với $d'$
			đi qua $A(1;0;-1)$ nên phương trình tham số của đường thẳng $d'$ là $\heva{&x=1+t \\& y=t \\& z=-1+2 t.}$
			\itemch \textbf{Sai.} Vì đường thẳng $d$ có một véc-tơ chỉ phương là $\overrightarrow{u}=(2 ;-1 ; 1)$. Mà $\overrightarrow{a},$ $ \overrightarrow{u}$ không cùng phương nên $\overrightarrow{a}$ không là véc-tơ chỉ phương  của $d$.
			\itemch \textbf{Sai.} Vì lấy $M\in d \Rightarrow M(1+2t; -1- t;  t).$ Không có giá trị nào của $t$ thỏa hệ $\heva{&1+2t=2 \\& -1- t=-1 \\&  t=1}$ nên đường thẳng $d$ không đi qua điểm $M(2 ;-1 ; 1)$.
			\itemch \textbf{Đúng.} Vì gọi $\alpha$ là góc giữa đường thẳng $d$ và mặt phẳng $(P)$. Ta có
			\allowdisplaybreaks
			\begin{eqnarray*}
				\sin \alpha&=&\bigg|\cos \left(\overrightarrow{u}_d, \overrightarrow{n}_P\right)\bigg|=\dfrac{\left|\overrightarrow{u}_d \cdot \overrightarrow{n}_P\right|}{\left|\overrightarrow{u}_d\right|\left|\overrightarrow{n}_P\right|}\\&=&\dfrac{|2 \cdot 1+(-1) \cdot 1+1 \cdot 2|}{\sqrt{2^{2}+(-1)^{2}+1^{2}} \cdot \sqrt{1^{2}+1^{2}+2^{2}}}=\dfrac{1}{2} \Rightarrow \alpha=30^{\circ}.
			\end{eqnarray*}
		\end{itemchoice}
	}
\end{ex}

%G:\My Drive\CODE12-2024\DE-ON-THEO BAI\2H5-TACH DE\Bai2-De2.tex
\begin{ex}%[2H5H2-6] 
	Trong không gian $Oxyz$ cho đường thẳng $d$ có phương trình tham số $\heva{&x=2+t \\& y=3-2 t \\& z=3 t.}$
	\choiceTF
	{\True Đường thẳng $d$ có một véc-tơ chỉ phương là $\overrightarrow{u}=(1 ;-2 ; 3)$}
	{ Đường thẳng $d$ đi qua điểm $M(1 ;-2 ; 3)$}
	{\True  Đường thẳng $d'$ đi qua điểm $A(1 ; 2 ;-2)$ và song song với đường thẳng $d$. Phương trình tham số của đường thẳng $d'$ là $\heva{&x=1+t \\& y=2-2 t \\& z=-2+3 t}$}
	{ Khoảng cách từ điểm $B(0 ; 1 ; 2)$ đến đường thẳng $d$ bằng $3$}
	\loigiai{
		\begin{itemchoice}
			\itemch \textbf{Đúng.} Vì  đường thẳng $d$ có một véc-tơ chỉ phương là $\overrightarrow{u}=(1 ;-2 ; 3)$.
			\itemch \textbf{Sai.} Vì lấy $M\in d \Rightarrow M(2+t; 3-2 t; 3 t).$ Không có giá trị nào của $t$ thỏa hệ $\heva{&2+t=1 \\& 3-2 t=-2 \\& 3 t=3}$ nên đường thẳng $d$ không đi qua điểm $M(1 ;-2 ; 3)$.
			\itemch \textbf{Đúng.} Vì $d'\parallel d$ nên $d'$ có một véc-tơ chỉ phương  là $\overrightarrow{u}_{d'}=\overrightarrow{u}_d=(1 ;-2 ; 3)$. Vậy đường thẳng $d'$ có phương trình tham số là $\heva{&x=1+t \\& y=2-2 t \\& z=-2+3 t.}$
			\itemch \textbf{Sai.} Vì lấy điểm $C (3 ; 1 ; 3)\in d,$ ta có $\overrightarrow{u}_d=(1 ;-2 ; 3),$ $ \overrightarrow{B C}=(3 ; 0 ; 1)$.\\
			Khoảng cách từ điểm $B$ đến đường thẳng $d$ là $h=\dfrac{\left[\overrightarrow{B C} , \overrightarrow{u}_d\right]}{\left|\overrightarrow{u}_d\right|}=\dfrac{\sqrt{364}}{7}$.\end{itemchoice}
	}
\end{ex}

%G:\My Drive\CODE12-2024\DE-ON-THEO BAI\2H5-TACH DE\Bai2-De2.tex
\begin{ex}%[2H5V2-5] 
	Trong không gian $O x y z$ cho đường thẳng $d$ có phương trình tham số $\heva{&x=2+2 t \\& y=1-t \\& z=1+2 t.}$
	\choiceTF
	{ Đường thẳng $d$ có một  véc-tơ chỉ phương là $\overrightarrow{a}=(2 ; 1 ; 1)$}
	{\True Điểm $B(4 ; 0 ; 3)$ thuộc đường thẳng $d$}
	{ Khoảng cách giữa đường thẳng $d$ và mặt phẳng $(P)\colon x+2 y-3=0$ bằng $1$}
	{\True  Đường thẳng $d$ và đường thẳng $d'\colon \dfrac{x}{4}=\dfrac{y-2}{-2}=\dfrac{z+1}{4}$ trùng nhau}
	\loigiai{
		\begin{itemchoice}
			\itemch \textbf{Sai.} Vì  véc-tơ chỉ phương của đường thẳng $d$ là $\overrightarrow{u}_d=(2 ;-1 ; 2)$.\\ Xét hai  véc-tơ $\overrightarrow{a}=(2 ; 1 ; 1)$ và $\overrightarrow{u}_d=(2 ;-1 ; 2)$.\\
			Vì $\dfrac{2}{2} \neq \dfrac{1}{-1}$ nên $\overrightarrow{a}=(2 ; 1 ; 1)$ và $\overrightarrow{u}_d$ không cùng phương. \\Do đó $\overrightarrow{a}$ không là  véc-tơ chỉ phương của đường thẳng $d$.
			\itemch \textbf{Đúng.} Vì thế tọa độ điểm $B(4 ; 0 ; 3)$ vào phương trình của đường thẳng $d$, ta có $\heva{&4=2+2 t \\& 0=1-t \\& 3=1+2 t}\Leftrightarrow t=1$ nên điểm $B(4 ; 0 ; 3)$ thuộc đường thẳng $d$. 
			\itemch \textbf{Sai.} Vì  ta có $\overrightarrow{u}_d=(2 ;-1 ; 2),$ $\overrightarrow{n}_{(P)}=(1 ; 2 ; 0)$ thỏa mãn $\overrightarrow{u}_d \cdot \overrightarrow{n}_{(P)}=0$. \\Suy ra $\overrightarrow{u}_d \perp \overrightarrow{n}_{(P)} \Rightarrow \hoac{&d \parallel (P) \\& d \subset(P).}$\\
			Mặt khác điểm $A(2 ; 1 ; 1) \in d$ nhưng $A \notin(P)$ nên $d \parallel(P)$.\\Theo câu \textbf{b)}, ta có $$B(4 ; 0 ; 3)\in d\Rightarrow \mathrm{d}(d,(P))=\mathrm{d}(B,(P))=\dfrac{4+2 \cdot 0-3}{\sqrt{1^2+2^2}}=\dfrac{\sqrt{5}}{5}.$$ 
			\itemch \textbf{Đúng.} Vì  ta có $\overrightarrow{u}_d=(2 ;-1 ; 2),$ $ \overrightarrow{u}_d'=(4 ;-2 ; 4)$ cùng phương.\hfill $(1)$\\
			Điểm $A(2 ; 1 ; 1) \in d$ và $A(2 ; 1 ; 1) \in d'$. \hfill $(2)$\\
			Từ $(1)$ và $(2)$ chứng tỏ $d,$ $ d'$ trùng nhau.
		\end{itemchoice}
	}
\end{ex}
\Closesolutionfile{ans}
\subsection*{\indam{PHẦN III. Câu trắc nghiệm trả lời ngắn. Thí sinh trả lời từ câu 1 đến câu 6 vào ô kết quả.}}
\setcounter{ex}{0}
\Opensolutionfile{ans}[ans/B2-De2-3]
\begin{ex}%[2H5V2-5] 
	Trong không gian $O x y z$, cho ba điểm $A(1 ;-2 ; 1),$ $ B(5 ; 0 ;-1),$ $ C(3 ; 1 ; 2)$ và mặt phẳng $(Q)\colon 3 x+y-z+3=0$. Gọi $M(a ; b ; c)$ là điểm thuộc $(Q)$ thỏa mãn $M A^2+M B^2+2 M C^2$ nhỏ nhất. Khi đó tổng $a+b+3 c$ bằng bao nhiêu?\\
	\shortans[oly]{$5$}
	\loigiai{
		Gọi $E$ là điểm thỏa mãn $\overrightarrow{E A}+\overrightarrow{E B}+2 \overrightarrow{E C}=\overrightarrow{0} \Rightarrow E(3 ; 0 ; 1)$.\\
		Ta có 
		\allowdisplaybreaks
		\begin{eqnarray*}
			S&=&M A^2+M B^2+2 M C^2=\overrightarrow{M A}^2+\overrightarrow{M B}^2+2 \overrightarrow{M C}^2\\&=&\left(\overrightarrow{M E}+\overrightarrow{E A}\right)^2+\left(\overrightarrow{M E}+\overrightarrow{E B}\right)^2+2\left(\overrightarrow{M E}+\overrightarrow{E C}\right)^2\\&=&4 M E^2+E A^2+E B^2+2 E C^2.
		\end{eqnarray*}
		Vì $E A^2+E B^2+2 E C^2$ không đổi nên $S$ nhỏ nhất khi và chỉ khi $M E$ nhỏ nhất.\\
		Suy ra $ M$ là hình chiếu vuông góc của $E$ lên $(Q)$. \\Do đó $ME\perp (Q)$ nên $\overrightarrow{u}_{ME}=\overrightarrow{n}_{(Q)}=(3;1;-1)$, và $E(3 ; 0 ; 1)$.\\
		Suy ra phương trình đường thẳng $M E\colon \heva{&x=3+3 t \\& y=t \\& z=1-t.}$\\
		Tọa độ điểm $M$ là nghiệm của hệ phương trình $\heva{&x=3+3 t \\& y=t \\& z=1-t \\& 3 x+y-z+3=0} \Leftrightarrow\heva{&x=0 \\& y=-1 \\& z=2 \\& t=-1.}$\\
		Vậy $M(0 ;-1 ; 2) \Rightarrow a=0,$ $ b=-1,$ $ c=2 \Rightarrow a+b+3 c=5$.
	}
\end{ex}

\begin{ex}%[2H5H2-8] 
	Trong không gian $O x y z$, một viên đạn được bắn ra từ điểm $A(3 ; 4 ; 2)$ và trong $4$ giây đầu đạn đi với vận tốc không đổi, véc-tơ vận tốc (trên giây) là $\overrightarrow{v}=(4 ; 5 ; 1)$. Biết viên đạn trúng mục tiêu tại điểm $M(13 ; b ; c)$, tính $b+2 c$.\\
	\shortans[oly]{$25{,}5$}
	\loigiai{
		Phương trình đường đi của viên đạn $\heva{&x=3+4 t \\& y=4+5 t \\& z=2+t}$ với $0 \leq t \leq 4$.\\
		Viên đạn trúng mục tiêu tại điểm $M(13 ; b ; c)$ khi $M$ nằm trên đường đi của viên đạn
		$$
		\Rightarrow\heva{& 1 3 = 3 + 4 t \\&
			b = 4 + 5 t  \\&
			c = 2 + t } \Leftrightarrow \heva{&
			t=\dfrac{5}{2} \\&
			b=\dfrac{33}{2} \\&
			c=\dfrac{9}{2}} \Rightarrow b+2 c=\dfrac{33}{2}+9=\dfrac{51}{2}=25{,}5.
		$$}
\end{ex}

\begin{ex}%[2H5V2-4]
	Trong không gian với hệ trục tọa độ $O x y z$, cho điểm $M(3 ; 3 ;-2)$ và hai đường thẳng $d_1\colon \dfrac{x-1}{1}=\dfrac{y-2}{3}=\dfrac{z}{1} ;$ $ d_2\colon \dfrac{x+1}{-1}=\dfrac{y-1}{2}=\dfrac{z-2}{4}$. Đường thẳng $d$ đi qua $M$ cắt $d_1,$ $ d_2$ lần lượt tại $A$ và $B$. Khi đó độ dài đoạn thẳng $A B$ bằng bao nhiêu?\\
	\shortans[oly]{$3$}
	\loigiai{
		Ta có
		\begin{itemize}
			\item Phương trình tham số của $d_1\colon \heva{&x=1+t_1 \\& y=2+3 t_1 \\& z=t_1} ;$ $ t_1 \in \mathbb{R},$\\ Do $A \in d_1$ nên $ A\left(1+t_1 ; 2+3 t_1 ; t_1\right).$
			\item Phương trình tham số của $d_2\colon \heva{&x=-1-t_2 \\& y=1+2 t_2 \\& z=2+4 t_2} ;$ $t_2 \in \mathbb{R}.$ \\Do $ B \in d_2$ nên $ B\left(-1-t_2 ; 1+2 t_2 ; 2+4 t_2\right)$.
		\end{itemize}
		$\overrightarrow{M A}=\left(t_1-2 ; 3 t_1-1 ; t_1+2\right) ;$ $ \overrightarrow{M B}=\left(-4-4 t_2 ;-2+2 t_2 ; 4+4 t_2\right)$.\\
		Vì $A,$ $ B,$ $ M$ thẳng hàng nên 
		\allowdisplaybreaks
		\begin{eqnarray*}
			&&\overrightarrow{M A}=k \overrightarrow{M B}, k \in \mathbb{R}
			\\&\Leftrightarrow&\heva{&t_1-2=-4 k-k t_2 \\& 3 t_1-1=-2 k+2 k t_2 \\& t_1+2=4 k+4 k t_2} \Leftrightarrow\heva{&t_1+4 k+k t_2=2 \\& 3 t_1+2 k-2 k t_2=1 \\& t_1-4 k-4 k t_2=-2}\\& \Leftrightarrow&\heva{&t_1=0 \\& k=\dfrac{1}{2} \\& k t_2=0} \Leftrightarrow\heva{&t_1=0 \\& k=\dfrac{1}{2} \\& t_2=0.}
		\end{eqnarray*}
		Vậy $A(1 ; 2 ; 0)$ và $B(-1 ; 1 ; 2) \Rightarrow \overrightarrow{A B}=(-2 ;-1 ; 2)$.\\ Độ dài đoạn thẳng $A B=\left|\overrightarrow{A B}\right|=3$.
	}
\end{ex}

\begin{ex}%[2H5V2-8] 
	Hình vẽ dưới đây là hình ảnh Cầu Cổng Vàng (The Golden Gate Bridge) ở Mỹ. Xét hệ trục toạ độ $O x y z$ với $O$ là bệ của chân cột trụ tại mặt nước, trục $O z$ trùng với cột trụ, mặt phẳng $O x y$ là mặt nước và xem như trục $O y$ cùng phương với cầu như hình vẽ. Dây cáp $A D$ (xem như là một đoạn thẳng) đi qua đỉnh $D$ thuộc trục $O z$ và điểm $A$ thuộc mặt phẳng $O y z$, trong đó điểm $D$ là đỉnh cột trụ cách mặt nước $227$ m, điểm $A$ cách mặt nước $75$ m và cách trục $O z$ khoảng $343$ m. \begin{flushright}
		\textit{(Nguồn: https://www.goldengate.org/assets/1/6/ggb-exhibit-chapter-statistics.pdf)}
	\end{flushright}
	\begin{center}
		\includegraphics[scale=0.7]{image/Cau-cong-vang}
	\end{center}
	Giả sử ta dùng một đoạn dây nối điểm $N$ trên dây cáp $A D$ và điểm $M$ trên thành cầu, biết $M$ cách mặt nước $75$ m và $M N$ song song với cột trụ. Tính độ dài $M N$ (đơn vị mét) biết điểm $M$ cách trục $O z$ một khoảng bằng $230$ m (kết quả làm tròn đến hàng phần mười).\\
	\shortans[oly]{$50{,}1$}
	\loigiai{
		Chọn một đơn vị trên các trục bằng $1$ m.\\
		Ta có $D(0 ; 0 ; 227),$ $ A(0 ;-343 ; 75),$ $ M(0 ;-230 ; 75)$, $\overrightarrow{A D}=(0 ; 343 ; 152)$.\\ Phương trình đường thẳng $A D\colon \heva{&x=0 \\& y=343 t \\& z=227+152 t} \Rightarrow N(0 ; 343 t ; 227+152 t)$.\\
		Ta có $\overrightarrow{M N}=(0 ; 343 t+230 ; 152+152 t)$, $M N$ song song với trục $O z$, suy ra $$ 343 t+230=0 \Rightarrow t=-\dfrac{230}{343} \Rightarrow M N=152+152 \cdot \left(-\dfrac{230}{343}\right) \approx 50,1(m).$$
	}
\end{ex}

\begin{ex}%[2H5V2-5] 
	Trong không gian với hệ trục tọa độ $O x y z$, cho hai điểm $A(-2 ;-1 ; 2)$ và $B(5 ;-1 ; 1)$. Đường thẳng $d'$ là hình chiếu của đường thẳng $A B$ lên mặt phẳng $(P)\colon x+2 y+z+2=0$ có một véc-tơ chỉ phương $\overrightarrow{u}=(a ; b ; 2)$. Tính $S=a+b$.\\
	\shortans[oly]{$-4$}
	\loigiai{
		Gọi $(Q)$ là mặt phẳng chứa đường thẳng $A B$ và vuông góc $(P)$. 
		\\Khi đó, đường thẳng $d'=(P) \cap(Q)$.\\
		Có $\heva{&\overrightarrow{n}_{(Q)} \perp \overrightarrow{A B}=(7 ; 0 ;-1) \\& \overrightarrow{n}_{(Q)} \perp \overrightarrow{n}_{(P)}=(1 ; 2 ; 1)}$. Suy ra chọn $\overrightarrow{n}_{(Q)}=\left[\overrightarrow{A B} ; \overrightarrow{n}_{(P)}\right]=(2 ;-8 ; 14)$.\\
		Mặt khác $\heva{&\overrightarrow{u}_{\left(d'\right)} \perp \overrightarrow{n}_{(P)} \\& \overrightarrow{u}_{\left(d'\right)} \perp \overrightarrow{n}_{(Q)}} $. Suy ra chọn $\overrightarrow{u}_{\left(d'\right)}=\left[\overrightarrow{n}_{(P)} ; \overrightarrow{n}_{(Q)}\right]=(36 ;-12 ;-12)$ cùng phương với $\overrightarrow{u}(-6 ; 2 ; 2)$.\\
		Như vậy $a=-6,$ $ b=2 \Rightarrow a+b=-4$.
	}
\end{ex}

\begin{ex}%[2H5V2-4]
	Trong không gian $O x y z$, cho điểm $A(1 ; 0 ; 2)$ và đường thẳng $d\colon \dfrac{x-1}{1}=\dfrac{y}{1}=\dfrac{z+1}{2}$. Đường thẳng $\Delta$ đi qua $A$, vuông góc và cắt $d$ đi qua điểm $M(a ; b ; 0)$. Tính $\dfrac{a}{b}$.\\
	\shortans[oly]{$1{,}5$}
	\loigiai{
		Đường thẳng $d$ có véc-tơ chỉ phương là $\overrightarrow{u}_d=(1 ; 1 ; 2)$.\\
		Gọi giao điểm của đường thẳng $\Delta$ và $d$ là $B$.\\
		Vì $B\in d$ nên $B(1+t ; t ;-1+2 t)\Rightarrow \overrightarrow{A B}=(t ; t ;-3+2 t)$.\\
		Vì đường thẳng $\Delta$ vuông góc với đường thẳng $d$ nên $$\overrightarrow{A B} \perp \overrightarrow{u}_d \Leftrightarrow \overrightarrow{A B} \cdot \overrightarrow{u}_d=0 \Leftrightarrow 1 \cdot t+1  \cdot  t+2  \cdot (-3+2 t)=0 \Leftrightarrow t=1.$$
		Do đó $B(2 ; 1 ; 1),$ $ \overrightarrow{A B}=(1 ; 1 ;-1)$.\\
		Đường thẳng $\Delta$ đi qua điểm $A(1 ; 0 ; 2)$ và có vectơ chỉ phương là $\overrightarrow{A B}=(1 ; 1 ;-1)$ nên có phương trình tham số $\heva{&x=1+t \\& y=0+t  \\& z=2-t.}$\\ $M(a ; b ; 0) \in \Delta \Rightarrow M(3 ; 2 ; 0) \Rightarrow a=3 ;$ $ b=2 ;$ $ \dfrac{a}{b}=1{,}5.$
	}
\end{ex}
\centerline{---HẾT---}
\Closesolutionfile{ans}
%\newpage
%%=====================
%\begin{center}
%\textbf{\large BẢNG ĐÁP ÁN}
%\end{center}
%\noindent\textbf{ĐÁP ÁN PHẦN I}
%\inputansbox{10}{ans/B2-De2-1}
	
%\noindent\textbf{ĐÁP ÁN PHẦN II}
%\inputansbox[2]{2}{ans/B2-De2-2}
	
%\noindent\textbf{ĐÁP ÁN PHẦN III}
%\inputansbox[3]{6}{ans/B2-De2-3}



