\section{TÍCH PHÂN}
\subsection{Tóm tắt lý thuyết}
\begin{tomtat} 
	\subsubsection{Định nghĩa}%[Võ Thanh Phong, Dự án TeX hóa Lý Thuyết] 

\begin{center}
	\boxed{\displaystyle\int\limits_a^bf(x)\mathrm{\,d}x=\left.F(x)\right|_a^b=F(b)-F(a)}.
\end{center}
\begin{nx}
	Tích phân của hàm số $f$ từ $a$ đến $b$  được kí hiệu là $\displaystyle\int\limits_a^bf(x)\mathrm{\,d}x$ hay $\displaystyle\int\limits_a^bf(t)\mathrm{\,d}t$.  Tích phân
	đó chỉ phụ thuộc vào $f$ và các cận $a$, $b$ mà không phụ thuộc vào cách ghi biến số. 
\end{nx}
\subsubsection{Tính chất của tích phân}
Cho hàm số $f(x)$ và $g(x)$ liên tục trên $K$, $a,b,c$ là ba số thuộc $K$. Khi đó ta có:
\begin{enumerate}
	\item $\displaystyle\int\limits_a^af(x)\mathrm{\,d}x = 0$.
	\item $\displaystyle\int\limits_a^bf(x)\mathrm{\,d}x = -\displaystyle\int\limits_b^af(x)\mathrm{\,d}x$.
	\item $\displaystyle\int\limits_a^bf(x)\mathrm{\,d}x = \displaystyle\int\limits_a^cf(x)\mathrm{\,d}x + \displaystyle\int\limits_c^bf(x)\mathrm{\,d}x$.
	\item $\displaystyle\int\limits_a^b\left[ f(x) \pm g(x) \right] \mathrm{\,d}x = \displaystyle\int\limits_a^bf(x)\mathrm{\,d}x + \pm \displaystyle\int\limits_a^bg(x)\mathrm{\,d}x $.
	\item $\displaystyle\int\limits_a^bk\cdot f(x)\mathrm{\,d}x = k\cdot \displaystyle\int\limits_a^bf(x)\mathrm{\,d}x$.
	\item Nếu $f(x) \ge 0, \forall x \in [a; b]$ thì $\displaystyle\int\limits_a^bf(x)\mathrm{\,d}x \ge 0, \forall x \in [a; b]$.
	\item $\forall x \in [a;b]: f(x) \le g(x) \Rightarrow \displaystyle\int\limits_a^bf(x)\mathrm{\,d}x \le \displaystyle\int\limits_a^bg(x)\mathrm{\,d}x$.
	\item $\forall x \in [a;b]$, nếu $M \ge f(x) \ge N \Rightarrow M(b - a) \le \displaystyle\int\limits_a^bf(x)\mathrm{\,d}x \le N(b - a)$.
\end{enumerate}
\end{tomtat}
\subsection{Các dạng toán cơ bản và phương pháp giải}
\begin{dang}{Tích phân cơ bản và tính chất tính phân}
	Dùng định nghĩa tích phân và các tính chất để giải bài toán. 
\end{dang}
\begin{bt}%[Nguyễn Anh Tuấn]%[12EX1, DCHT]%[2D3B2-1]
	\begin{enumEX}{1}
		\item Cho hàm số $ f(x) $ liên tục trên đoạn $ [0;10] $ thỏa mãn $ \displaystyle\int\limits_0^{10} f(x)\mathrm{\,d}x=7 $ và $ \displaystyle\int\limits_2^6 f(x)\mathrm{\,d}x =3$. Tính $ \displaystyle\int\limits_0^2 f(x)\mathrm{\,d}x +\displaystyle\int\limits_6^{10} f(x)\mathrm{\,d}x$.
		\dapso{$ 4 $}
		\item Cho hàm số $ y=f(x) $ liên tục trên $ \mathbb{R} $ thỏa mãn $ \displaystyle\int\limits_a^b f(x)\mathrm{\,d}x =2$ và $ \displaystyle\int\limits_c^b f(x)\mathrm{\,d}x =3$ với $ a<b<c $. Tính $ \displaystyle\int\limits_a^c f(x)\mathrm{\,d}x $
		\dapso{$ -1 $}
		\item Cho hàm số $ y=f(x) $ liên tục trên $ \mathbb{R} $ thỏa mãn $ \displaystyle\int\limits_1^3 f(x)\mathrm{\,d}x =2017$ và $ \displaystyle\int\limits_4^3 f(x)\mathrm{\,d}x =2018$. Tính $ \displaystyle\int\limits_1^4 f(x)\mathrm{\,d}x $.
		\dapso{$ -1 $}
	\end{enumEX}
	\loigiai{
		\begin{enumEX}{1}
			\item Ta có 
			$$7=\displaystyle\int\limits_0^{10} f(x)\mathrm{\,d}x=\displaystyle\int\limits_0^2 f(x)\mathrm{\,d}x +\displaystyle\int\limits_2^6 f(x)\mathrm{\,d}x+\displaystyle\int\limits_6^{10} f(x)\mathrm{\,d}x.$$
			Hay là 
			$$7=\displaystyle\int\limits_0^{10} f(x)\mathrm{\,d}x=\displaystyle\int\limits_0^2 f(x)\mathrm{\,d}x+ 3+\displaystyle\int\limits_6^{10} f(x)\mathrm{\,d}x \Rightarrow P=\displaystyle\int\limits_0^2 f(x)\mathrm{\,d}x +\displaystyle\int\limits_6^{10} f(x)\mathrm{\,d}x=7-3=4.$$
			\item Ta có
			$$\displaystyle\int\limits_a^c f(x)\mathrm{\,d}x=\displaystyle\int\limits_a^b f(x)\mathrm{\,d}x+\displaystyle\int\limits_b^c f(x)\mathrm{\,d}x=\displaystyle\int\limits_a^b f(x)\mathrm{\,d}x-\displaystyle\int\limits_c^b f(x)\mathrm{\,d}x=2-3=-1.$$
			\item Ta có
			$$\displaystyle\int\limits_1^4 f(x)\mathrm{\,d}x =\displaystyle\int\limits_1^3 f(x)\mathrm{\,d}x +\displaystyle\int\limits_3^4 f(x)\mathrm{\,d}x=\displaystyle\int\limits_1^3 f(x)\mathrm{\,d}x-\displaystyle\int\limits_4^3 f(x)\mathrm{\,d}x =2017-2018=-1.$$
	\end{enumEX}}
\end{bt}
\begin{bt}%[Nguyễn Anh Tuấn]%[12EX1, DCHT]%[2D3B2-1]
	\begin{enumEX}{1}
		\item Cho hàm số $ y=f(x) $ liên tục trên $ \mathbb{R} $ thỏa mãn $ \displaystyle\int\limits_2^5 f(x)\mathrm{\,d}x =3$ và $ \displaystyle\int\limits_5^7 f(x)\mathrm{\,d}x =9$. Tính $ \displaystyle\int\limits_2^7 f(x)\mathrm{\,d}x $.
		\dapso{$ 12 $}
		\item Cho hàm số $ y=f(x) $ liên tục trên $ \mathbb{R} $ thỏa mãn $ \displaystyle\int\limits_0^6 f(x)\mathrm{\,d}x =4$ và $ \displaystyle\int\limits_2^6 f(t)\mathrm{\,d}t =-3$. Tính $ \displaystyle\int\limits_0^2 [f(v)-3]\mathrm{\,d}v $.
		\dapso{$ 1 $}
	\end{enumEX}
	\loigiai{
		\begin{enumEX}{1}
			\item Ta có
			$$\displaystyle\int\limits_2^7 f(x)\mathrm{\,d}x =\displaystyle\int\limits_2^5 f(x)\mathrm{\,d}x +\displaystyle\int\limits_5^7 f(x)\mathrm{\,d}x=3+9=12.$$
			\item Ta có
			$$\displaystyle\int\limits_0^2 f(v)\mathrm{\,d}v =\displaystyle\int\limits_0^6 f(v)\mathrm{\,d}v -\displaystyle\int\limits_2^6 f(v)\mathrm{\,d}v=\displaystyle\int\limits_0^6 f(x)\mathrm{\,d}x -\displaystyle\int\limits_2^6 f(x)\mathrm{\,d}x=4-(-3)=7.$$
			Hay là $$\displaystyle\int\limits_0^2 f(v)\mathrm{\,d}v=7\Rightarrow \displaystyle\int\limits_0^2 [f(v)-3]\mathrm{\,d}v=\displaystyle\int\limits_0^2 f(v)\mathrm{\,d}v-\displaystyle\int\limits_0^2 3\mathrm{\,d}v=7-3v|_0^2=1.$$
	\end{enumEX}}
\end{bt}
\begin{bt}%[Nguyễn Anh Tuấn]%[12EX1, DCHT]%[2D3B2-1]
	\begin{enumEX}{1}
		\item Cho hàm số $ f(x) $ có đạo hàm trên đoạn $ [1;2] $, $ f'(1) =1$ và $ f(2)=2 $. Tính $ \displaystyle\int\limits_1^2 f'(x)\mathrm{\,d}x $.
		\dapso{$ 1 $}
		\item Cho hàm số $ f(x) $ có đạo hàm trên đoạn $ [1;4] $, $ f(1) =1$ và $ \displaystyle\int\limits_1^4 f'(x)\mathrm{\,d}x =2$. Tính $ f(4) $.
		\dapso{$ 3 $}
		\item Cho hàm số $ f(x) $ có đạo hàm trên đoạn $ [1;3] $, $ f(3) =5$ và $ \displaystyle\int\limits_1^3 f'(x)\mathrm{\,d}x =6$. Tính $ f(1) $.
		\dapso{$ -1 $}
		
	\end{enumEX}
	\loigiai{
		\begin{enumEX}{1}
			\item 
			Ta có $ \displaystyle\int\limits_1^2 f'(x)\mathrm{\,d}x =\left.f(x)\right|_1^2=f(2)-f(1)=2-1=1 $.
			\item 	Ta có $2= \displaystyle\int\limits_1^4 f'(x)\mathrm{\,d}x =\left.f(x)\right|_1^4=f(4)-f(1)=f(4)-1 \Rightarrow f(4)=3$.
			\item Ta có $6= \displaystyle\int\limits_1^3 f'(x)\mathrm{\,d}x =\left.f(x)\right|_1^3=f(3)-f(1)=5-f(1) \Rightarrow f(1)=-1$.
			
	\end{enumEX}}	
\end{bt}
\begin{bt}%[Lê Quốc Hiệp,dự án 12-EX-1-DCHT]%[2D3B2-1]
	Tính các tích phân sau
		\begin{enumerate}
		\item Tính $\displaystyle\int\limits_{-2}^3 (4x^3-3x^2+10)\mathrm{\,d}x$
		
		\dapso{$80$}
		\loigiai{
			$\displaystyle\int\limits_{-2}^3 (4x^3-3x^2+10)\mathrm{\,d}x=(x^4-x^3+10x)\bigg|_{-2}^3=84-4=80$.
		}
		\item Tính $\displaystyle\int\limits_1^4 (x^2+3\sqrt{x})\mathrm{\,d}x$
		
		\dapso{$35$}
		\loigiai{
			$\displaystyle\int\limits_1^4 \dfrac{4}{(1-2x)^2}\mathrm{\,d}x=\left(\dfrac{x^3}{3}+2x\sqrt{x}\right)\bigg|_1^4=\dfrac{112}{3}-\dfrac{7}{3}=35$.
		}
		\item Tính $\displaystyle\int\limits_0^2 x(x+1)^2\mathrm{\,d}x$
		
		\dapso{$\dfrac{34}{3}$}
		\loigiai{
			$\displaystyle\int\limits_0^2 x(x+1)^2\mathrm{\,d}x=\displaystyle\int\limits_0^2(x^3+2x^2+x)\mathrm{\,d}x=\left(\dfrac{x^4}{4}+\dfrac{2x^3}{3}+\dfrac{x^2}{2}\right)\bigg|_0^2=\dfrac{34}{3}-0=\dfrac{34}{3}$.
		}
		\item Tính $\displaystyle\int\limits_2^4 \left(x+\dfrac{1}{x}\right)\mathrm{\,d}x$
		
		\dapso{$6+\ln 2$}
		\loigiai{
			$\displaystyle\int\limits_2^4 \left(x+\dfrac{1}{x}\right)\mathrm{\,d}x=\left(\dfrac{x^2}{2}+\ln x\right)\bigg|_2^4=8+\ln 4-2-\ln 2=6+\ln 2$.
		}
		\item Tính $\displaystyle\int\limits_1^3 \left(\dfrac{3}{x}-\dfrac{1}{x^2}\right)\mathrm{\,d}x$
		
		\dapso{$3\ln 3-\dfrac{2}{3}$}
		\loigiai{
			$\displaystyle\int\limits_1^3 \left(\dfrac{3}{x}-\dfrac{1}{x^2}\right)\mathrm{\,d}x=\left(3\ln x+\dfrac{1}{x}\right)\bigg|_1^3=3\ln 3+\dfrac{1}{3}-0-1=3\ln 3-\dfrac{2}{3}$.
		}
		\item Tính $\displaystyle\int\limits_0^1 \mathrm{e}^{3x}\mathrm{\,d}x$
		
		\dapso{$\dfrac{1}{3}\mathrm{e}^{3}-\dfrac{1}{3}$}
		\loigiai{
			$\displaystyle\int\limits_0^1 \mathrm{e}^{3x}\mathrm{\,d}x=\dfrac{1}{3}\mathrm{e}^{3x}\bigg|_0^1=\dfrac{1}{3}\mathrm{e}^{3}-\dfrac{1}{3}$.
		}
		\item Tính $\displaystyle\int\limits_0^{2018} 7^x\mathrm{\,d}x$
		
		\dapso{$\dfrac{7^{2018}-1}{\ln 7}$}
		\loigiai{
			$\displaystyle\int\limits_0^{2018} 7^x\mathrm{\,d}x=\dfrac{7^x}{\ln 7}\bigg|_0^{2018}=\dfrac{7^{2018}}{\ln 7}-\dfrac{1}{\ln 7}=\dfrac{7^{2018}-1}{\ln 7}$.
		}
		\item Tính $\displaystyle\int\limits_0^6 \dfrac{\mathrm{\,d}x}{x+6}$
		
		\dapso{$\ln 2$}
		\loigiai{
			$\displaystyle\int\limits_0^6 \dfrac{\mathrm{\,d}x}{x+6}=\ln (x+6)\bigg|_0^6=\ln 12-\ln 6=\ln 2$.
		}
		\item Tính $\displaystyle\int\limits_1^3 \dfrac{\mathrm{\,d}x}{1-3x}$
		
		\dapso{$-\dfrac{1}{3}\ln 4$}
		\loigiai{
			$\displaystyle\int\limits_1^3 \dfrac{\mathrm{\,d}x}{1-3x}=-\dfrac{1}{3}\ln (3x-1)\bigg|_1^3=-\dfrac{1}{3}\ln 8+\dfrac{1}{3}\ln 2=-\dfrac{1}{3}\ln 4$.
		}
		\item Tính $\displaystyle\int\limits_1^2 \dfrac{\mathrm{\,d}x}{(4x-1)^2}$
		
		\dapso{$\dfrac{1}{21}$}
		\loigiai{
			$\displaystyle\int\limits_1^2 \dfrac{\mathrm{\,d}x}{(4x-1)^2}=-\dfrac{1}{4}\cdot\dfrac{1}{4x-1}\bigg|_1^2=-\dfrac{1}{28}+\dfrac{1}{12}=\dfrac{1}{21}$.
		}
		\item Tính $\displaystyle\int\limits_1^4 \dfrac{4}{(1-2x)^2}\mathrm{\,d}x$
		\dapso{$\dfrac{12}{7}$}
		\loigiai{
			$\displaystyle\int\limits_1^4 \dfrac{4}{(1-2x)^2}\mathrm{\,d}x=2\cdot\dfrac{1}{1-2x}\bigg|_1^4=-\dfrac{2}{7}+2=\dfrac{12}{7}$.
		}
\end{enumerate}
\end{bt}
\begin{bt}%[Lê Quốc Hiệp,dự án 12-EX-1-DCHT]%[2D3B2-1]
	Tính các tích phân sau
	\begin{enumerate}
		\item $\displaystyle\int\limits_{\tfrac{\pi}{3}}^{\tfrac{\pi}{2}} \sin x\mathrm{\,d}x$. \dapso{ $\dfrac{1}{2}$}
		\loigiai{
			$\displaystyle\int\limits_{\tfrac{\pi}{3}}^{\tfrac{\pi}{2}} \sin x\mathrm{\,d}x=-\cos x\bigg|_{\tfrac{\pi}{3}}^{\tfrac{\pi}{2}}=0+\dfrac{1}{2}=\dfrac{1}{2}$.
		}
		\item $\displaystyle\int\limits_{\tfrac{\pi}{4}}^{\tfrac{\pi}{3}}\dfrac{\mathrm{\,d}x}{\cos^2x}$. \dapso{ $\sqrt{3}-1$}
		\loigiai{
			$\displaystyle\int\limits_{\tfrac{\pi}{4}}^{\tfrac{\pi}{3}}\dfrac{\mathrm{\,d}x}{\cos^2x}=\tan x\bigg|_{\tfrac{\pi}{4}}^{\tfrac{\pi}{3}}=\sqrt{3}-1$.
		}
\end{enumerate}
\end{bt}
\begin{bt}%[Lê Quốc Hiệp,dự án 12-EX-1-DCHT]%[2D3K2-1]
	Tính các tính phân
\begin{enumerate}
		\item $\displaystyle\int\limits_{-2}^{-5} \dfrac{\mathrm{\,d}x}{\sqrt{1-3x}}$. \dapso{ $\dfrac{2\sqrt{7}-8}{3}$}
		\loigiai{
			$\displaystyle\int\limits_{-2}^{-5} \dfrac{\mathrm{\,d}x}{\sqrt{1-3x}}=-\dfrac{2}{3}\sqrt{1-3x}\bigg|_{-2}^{-5}=-\dfrac{8}{3}+\dfrac{2\sqrt{7}}{3}=\dfrac{2\sqrt{7}-8}{3}$.
		}
		\item $\displaystyle\int\limits_2^7 \dfrac{4\mathrm{\,d}x}{\sqrt{x+1}+\sqrt{x-1}}$. \dapso{ $\dfrac{64\sqrt{2}-12\sqrt{3}-24\sqrt{6}+4}{3}$}
		\loigiai{
			$\displaystyle\int\limits_2^7 \dfrac{4\mathrm{\,d}x}{\sqrt{x+1}+\sqrt{x-1}}=\displaystyle\int\limits_2^7\dfrac{4\left(\sqrt{x+1}-\sqrt{x-1}\right)\mathrm{\,d}x}{x+1-x+1}\\=\displaystyle\int\limits_2^7 2\sqrt{x+1}\mathrm{\,d}x-\displaystyle\int\limits_2^7 2\sqrt{x-1}\mathrm{\,d}x$.\\
			Ta có $\displaystyle\int\limits_2^7 2\sqrt{x+1}\mathrm{\,d}x=\dfrac{4}{3}(x+1)^{\tfrac{3}{2}}\bigg|_2^7=\dfrac{64\sqrt{2}}{3}-4\sqrt{3}$,\\
			$\displaystyle\int\limits_2^7 2\sqrt{x-1}\mathrm{\,d}x=\dfrac{4}{3}(x-1)^{\tfrac{3}{2}}\bigg|_2^7=\dfrac{24\sqrt{6}}{3}-\dfrac{4}{3}$.\\
			Vậy $\displaystyle\int\limits_2^7 \dfrac{4\mathrm{\,d}x}{\sqrt{x+1}+\sqrt{x-1}}=\dfrac{64\sqrt{2}-12\sqrt{3}-24\sqrt{6}+4}{3}$.
		}
\end{enumerate}
\end{bt}
\begin{bt}%[Lê Quốc Hiệp,dự án 12-EX-1-DCHT]%[2D3K2-1]
	Tìm số thực $m$ thỏa mãn
	\begin{enumerate}
		\item $\displaystyle\int\limits_{-1}^m \mathrm{e}^{x+1}\mathrm{\,d}x=\mathrm{e}^2-1$. \dapso{ $m=1$}
		\item $\displaystyle\int\limits_0^m (2x+5)\mathrm{\,d}x=6$. \dapso{ $m=1,~m=-6$}
\end{enumerate}
\loigiai{
		\begin{enumerate}
		\item  $\displaystyle\int\limits_{-1}^m \mathrm{e}^{x+1}\mathrm{\,d}x=\mathrm{e}^{x+1}\bigg|_{-1}^m=\mathrm{e}^{m+1}-1$.\\
		Theo đề bài ta suy ra
		\[\mathrm{e}^{2}-1=\mathrm{e}^{m+1}-1\Leftrightarrow m=1.\]
		Vậy $m=1$.
		\item  $\displaystyle\int\limits_0^m (2x+5)\mathrm{\,d}x=\left(x^2+5x\right)\bigg|_0^m=m^2+5m$.\\
		Theo đề bài ta suy ra
		\[m^2+5m=6\Leftrightarrow m=1 \text{ hoặc }m=-6.\]
		Vậy $m=1$ hoặc $m=-6$.
\end{enumerate}
}
\end{bt}
\begin{bt}%[Lê Quốc Hiệp,dự án 12-EX-1-DCHT]%[2D3K2-1]
		\begin{enumerate}
		\item Biết $\displaystyle\int\limits_{1}^2 \sqrt{2x-1}\mathrm{\,d}x=\dfrac{\sqrt{a}-1}{b}$ với $a,~b$ là số nguyên dương. Tính $a-b^3$. 
		
		\dapso{$0$}
		\loigiai{
			$\displaystyle\int\limits_{1}^2 \sqrt{2x-1}\mathrm{\,d}x=\displaystyle\int\limits_{1}^2 (2x-1)^{\tfrac{1}{2}}\mathrm{\,d}x=\dfrac{1}{3}(2x-1)^{\tfrac{3}{2}}\bigg|_1^2=\sqrt{3}-\dfrac{1}{3}=\dfrac{\sqrt{27}-1}{3}$.\\
			Suy ra $a=27,~b=3$ và $a-b^3=0$.
		}
		\item Biết $\displaystyle\int\limits_{1}^2 \dfrac{\mathrm{\,d}x}{(x+1)\sqrt{x}+x\sqrt{x+1}}=\sqrt{a}-\sqrt{b}-c$ với $a,~b,~c$ là các số nguyên dương. Tính $P=a+b+c$. 
		
		\dapso{$46$}
		\loigiai{
		\begin{eqnarray*}
			\displaystyle\int\limits_{1}^2 \dfrac{\mathrm{\,d}x}{(x+1)\sqrt{x}+x\sqrt{x+1}}&=& \displaystyle\int\limits_{1}^2 \dfrac{\mathrm{\,d}x}{\sqrt{x(x+1)}\left(\sqrt{x+1}+\sqrt{x}\right)}\\
			&=& \displaystyle\int\limits_{1}^2 \dfrac{\sqrt{x+1}-\sqrt{x}}{\sqrt{x(x+1)}}\mathrm{\,d}x\\
			&=&\displaystyle\int\limits_{1}^2 \left(\dfrac{1}{\sqrt{x}}-\dfrac{1}{\sqrt{x+1}}\right)\mathrm{\,d}x\\
			&=&2\left(\sqrt{x}-\sqrt{x+1}\right)\bigg|_1^2\\
			&=&4\sqrt{2}-2\sqrt{3}-2\\
			&=&\sqrt{32}-\sqrt{12}-2.
		\end{eqnarray*}\
	Suy ra $a=32,~b=12,~c=2$ và $P=a+b+c=46$. 
		}
\end{enumerate}
\end{bt}
\begin{dang}{Tích phân hàm số phân thức hữu tỉ}
	%Phương pháp giải: 
	Chú ý nguyên hàm của một số hàm phân thức hữu tỉ thường gặp.
	\begin{enumEX}{1}
		\item $\displaystyle \int \dfrac{1}{ax+b}\mathrm{\,d}x= \dfrac{1}{a}\ln |ax+b|+C$, với $ a  e 0 $.
		\item  $\displaystyle \int \dfrac{1}{(ax+b)^n}\mathrm{\,d}x= \dfrac{1}{a}\cdot \dfrac{-1}{(n-1)(ax+b)^{n-1}}+C$, với $ a  e 0 $, $ n \in \mathbb{N}, n \geq 2 $.
		\item $\displaystyle \int \dfrac{1}{(x-a)(x-b)}\mathrm{\,d}x=\dfrac{1}{a-b}\ln {\left| \dfrac{x-a}{x-b}\right|}+C$, với $ a  e b $.
	\end{enumEX}	
\end{dang}
\begin{bt}%[Nguyễn Anh Tuấn]%[12EX1, DCHT]%[2D3K2-1]
	Tính các tích phân sau
		\begin{enumerate}
		\item Tính $ \displaystyle\int\limits_0^1 \dfrac{x}{(x+1)^2}\mathrm{\,d}x $.
		\dapso{$ \ln 2-\dfrac{1}{2} $}
		\item $ \displaystyle\int\limits_0^1 \dfrac{x}{(x+2)^3}\mathrm{\,d}x$.
		\dapso{$\ln \dfrac{3}{2}-\dfrac{5}{36} $}
		\item Biết $ \displaystyle\int\limits_{-1}^0 \dfrac{3x^2+5x-1}{x-2}\mathrm{\,d}x =a \ln \dfrac{2}{3}+b $ với $ a, b \in \mathbb{Q} $. Tính giá trị của $ S=a+4b $.
		\dapso{$ -29 $}
		\item Tính $ \displaystyle\int\limits_0^1 \dfrac{x-3}{2x+1}\mathrm{\,d}x $.
\end{enumerate}
	\loigiai{
		\begin{enumerate}
			\item 	
			Ta có $$\displaystyle\int\limits_0^1 \dfrac{x}{(x+1)^2}\mathrm{\,d}x=\displaystyle\int\limits_0^1 \dfrac{x+1-1}{(x+1)^2}\mathrm{\,d}x=\displaystyle\int\limits_0^1 \left[\dfrac{1}{x+1}-\dfrac{1}{(x+1)^2}\right]\mathrm{\,d}x=\left.\left[\ln |x+1|+\dfrac{1}{x+1}\right]\right|_0^1=\ln 2-\dfrac{1}{2}.$$
			\item  Ta có $$\displaystyle\int\limits_0^1 \dfrac{x}{(x+2)^3}\mathrm{\,d}x=\displaystyle\int\limits_0^1 \dfrac{x+2-2}{(x+2)^3}\mathrm{\,d}x=\displaystyle\int\limits_0^1 \left[\dfrac{1}{x+2}-\dfrac{2}{(x+2)^3}\right]\mathrm{\,d}x=\left.\left[\ln |x+2|+\dfrac{1}{(x+2)^2}\right]\right|_0^1=\ln \dfrac{3}{2}-\dfrac{5}{36}.$$
			\item Ta có $ \dfrac{3x^2+5x-1}{x-2}=\dfrac{3(x^2-4)+5(x-2)+21}{x-2}=(3x+11)+\dfrac{21}{x-2} $.\\
				Do đó: $  \displaystyle\int\limits_{-1}^0 \dfrac{3x^2+5x-1}{x-2}\mathrm{\,d}x = \displaystyle\int\limits_-1^0 \left[3x+11+\dfrac{21}{x-2}\right]\mathrm{\,d}x =\left.\left[\dfrac{3}{2}x^2+11x+21\ln |x-2|\right]\right|_{-1}^0 =-\dfrac{25}{2}+21\ln \dfrac{2}{3}$.\\
				Khi đó $ a=21, b=-\dfrac{25}{2} $ nên $ S=21+4 \cdot \left(-\dfrac{25}{2}\right)=-29 $.
	\end{enumerate}}
\end{bt}
\begin{bt}%[Nguyễn Anh Tuấn]%[12EX1, DCHT]%[2D3K2-1]
	Tính các tích phân sau
	\begin{enumEX}{1}
		\item Biết $\displaystyle\int\limits_1^2 \dfrac{\mathrm{\,d}x}{3x-1}=\dfrac{1}{a} \ln b$ với $ b>0 $. Tính $ S=a^2+b $. \dapso{$ \dfrac{47}{18} $}
		\item Biết $ \displaystyle\int\limits_0^2 \dfrac{x^2}{x+1}\mathrm{\,d}x=a+\ln b $ với $ a, b \in \mathbb{Q} $. Tính $ S=2a+b+2^b $. \dapso{$ 11 $}
	\end{enumEX}
	\loigiai{
		\begin{enumEX}{1}
			\item Ta có  $\displaystyle\int\limits_1^2 \dfrac{\mathrm{\,d}x}{3x-1}=\dfrac{1}{3} \displaystyle\int\limits_1^2 \dfrac{\mathrm{\,d}(3x-1)}{3x-1} =\dfrac{1}{3} \left. \ln |3x-1|\right|_1^2=\dfrac{1}{3} \left(\ln 5-\ln 2\right)=\dfrac{1}{3}\ln \dfrac{5}{2} $.\\
			Suy ra $ a=3, b=\dfrac{5}{2} $. Do đó $ S=\dfrac{1}{9}+\dfrac{5}{2}=\dfrac{47}{18}$.
			\item Ta có $$ \displaystyle\int\limits_0^2 \dfrac{x^2}{x+1}\mathrm{\,d}x=\displaystyle\int\limits_0^2 \left(x-1+\dfrac{1}{x+1}\right)\mathrm{\,d}x=\left.\left[\dfrac{x^2}{2}-x+\ln |x+1|\right]\right|_0^2=\ln 3=0+ \ln 3. $$
			Suy ra $ a=0, b=3 $ nên $ S=2\cdot 0+3+2^3=11 $.
	\end{enumEX}}
\end{bt}
%\setcounter{bt}{0}
\begin{bt}%[Nguyễn Anh Tuấn]%[12EX1, DCHT]%[2D3K2-1]
	Tính các tích phân sau
	\begin{enumEX}{1}
		\item Biết $ \displaystyle\int\limits_0^1 \dfrac{2x+3}{2-x}\mathrm{\,d}x=a\ln 2+b$ với $ a, b \in \mathbb{Q} $. Tính $ P=a+2b+2^a-2^b $.	
		\dapso{$\dfrac{523}{4} $}
		\loigiai{Ta có $ \dfrac{2x+3}{2-x}=-\dfrac{2(x-2)+7}{x-2} =-2-\dfrac{7}{x-2}$. Do đó
			$$\displaystyle\int\limits_0^1 \dfrac{2x+3}{2-x}\mathrm{\,d}x=\displaystyle\int\limits_0^1 \left(-2-\dfrac{7}{x-2}\right)\mathrm{\,d}x=\left. \left[-2x-7\ln |x-2|\right]\right|_0^1=-2+7\ln 2=7\ln 2-2.$$
			Do đó, $ a=7, b=-2 $. Vậy $ P=7+2 \cdot (-2)+2^7-2^{-2}=\dfrac{523}{4} $.
		}
		\item Biết $ \displaystyle\int\limits_0^1 \dfrac{2x-1}{x+1}\mathrm{\,d}x=a+b \ln 2 $ với $ a, b \in \mathbb{Q} $. Tính $ P=ab-a+b $. 
		\dapso{$ -11 $}
		\loigiai{Ta có $ \displaystyle\int\limits_0^1 \dfrac{2x-1}{x+1}\mathrm{\,d}x =\displaystyle\int\limits_0^1 \dfrac{2(x+1)-3}{x+1}\mathrm{\,d}x=\displaystyle\int\limits_0^1 \left(2-\dfrac{3}{x+1}\right)\mathrm{\,d}x=\left. \left(2x-3\ln |x+1|\right)\right|_0^1=2-3\ln 2$.\\
			Vậy $ a=2, b=-3 $, suy ra $ P=-6-2-3=-11 $.}
		
	\end{enumEX}
\end{bt}
\begin{bt}%[Nguyễn Anh Tuấn]%[12EX1, DCHT]%[2D3K2-1]
	Tính các tích phân sau
	\begin{enumEX}{1}
		\item Tính $ \displaystyle\int\limits_0^1 	\dfrac{3x-1}{x^2+6x+9}\mathrm{\,d}x=3\ln \dfrac{a}{b}-\dfrac{5}{6} $ với $ a, b \in \mathbb{Z^+} $ và $\dfrac{a}{b} $ là phân số tối giản. Tính giá trị của biểu thức $ P=2^a+2^b-ab $.
			\dapso{$ 12 $}
		\loigiai{Ta có $$	\dfrac{3x-1}{x^2+6x+9}=\dfrac{3(x+3)-10}{(x+3)^2}=\dfrac{3}{x+3}-\dfrac{10}{(x+3)^2}.$$
			Do đó 
			$$ \displaystyle\int\limits_0^1 	\dfrac{3x-1}{x^2+6x+9}\mathrm{\,d}x= \displaystyle\int\limits_0^1 	\left[\dfrac{3}{x+3}-\dfrac{10}{(x+3)^2}\right]\mathrm{\,d}x=
			\left. \left[3\ln |x+3|+\dfrac{10}{x+3}\right]\right|_0^1=3\ln \dfrac{4}{3}-\dfrac{5}{6}.$$
			Suy ra $ a=4, b=3 $ nên $ P=2^4+2^3-4\cdot 3=12 $.
		}
		\item Biết $ \displaystyle\int\limits_0^1 \left(\dfrac{1}{x+1}-\dfrac{1}{x+2}\right)\mathrm{\,d}x=a\ln 2+b\ln 3 $ với $ a, b \in \mathbb{Z}$. Tính $ S=a+b-ab^2 $.
		\dapso{$-1 $}
		\loigiai{Ta có $$\displaystyle\int\limits_0^1 \left(\dfrac{1}{x+1}-\dfrac{1}{x+2}\right)\mathrm{\,d}x=\left.\left[\ln |x+1|-\ln |x+2|\right]\right|_0^1=\left.\ln \left|\dfrac{x+1}{x+2}\right|\right|_0^1=\ln \dfrac{2}{3}-\ln \dfrac{1}{2}=\ln \dfrac{4}{3}=2\ln 2-\ln 3.$$
			Suy ra $ a=2, b=-1 $ nên $ S=2+(-1)-2\cdot(-1)^2=-1 $.}
\end{enumEX}
\end{bt}
%%%%%%%%%%%%%%%%%%%%%%%%%%%%%%%%%%%%%%%%%%%%%%%%%%%%%%%%%%%%%%%%%%%%%
\begin{dang}{Tích phân hàm số chứa dấu giá trị tuyệt đối $\displaystyle\int\limits_{a}^{b} \mid f(x) \mid \mathrm{\,d}x$}
	Phương pháp giải\\
	Sử dụng tính chất của tích phân \\
	\begin{displaymath}
		\int\limits_{a}^{b} \mid f(x)\mid  \mathrm{\,d}x=\int\limits_{a}^{c} \mid f(x)\mid  \mathrm{\,d}x+\int\limits_{c}^{b} \mid f(x)\mid  \mathrm{\,d}x
	\end{displaymath}
	đến đây ta có 2 cách để phá dấu giá trị tuyệt đối
	\begin{listEX}
		\item [$\bullet$] Cách 1. Xét dấu biểu thức $f(x)$ để khử dấu trị tuyệt đối.
		\item [$\bullet$] Cách 2. Giải phương trình $f(x)=0$ trên  $\left(a;b\right)$. Giả sử trên khoảng $\left(a;b\right)$ phương trình có nghiệm $a<x_1<x_2<\ldots <x_n<b$. Do hàm số $f(x)$ không đổi dấu trên mỗi khoảng $\left(x_i;x_{i+1}\right)$ nên ta có
		\begin{eqnarray*}
			\int\limits_{a}^{b} \mid f(x)\mid  \mathrm{\,d}x&=&\int\limits_{a}^{x_1} \mid f(x)\mid  \mathrm{\,d}x+\int\limits_{x_1}^{x_2} \mid f(x)\mid  \mathrm{\,d}x+\ldots +\int\limits_{x_n}^{b} \mid f(x)\mid  \mathrm{\,d}x\\
			&=&\bigg{|} \int\limits_{a}^{x_1}  f(x) \mathrm{\,d}x\bigg{|}+\bigg{|} \int\limits_{x_1}^{x_2}  f(x) \mathrm{\,d}x\bigg{|}+\ldots +\bigg{|} \int\limits_{x_n}^{b}  f(x) \mathrm{\,d}x\bigg{|}
		\end{eqnarray*}
	\end{listEX}
\end{dang}
\begin{bt}%[Nguyễn Thành Tiến - Dự án 12EX-DCHT] %[2D3B2-4]
	Tính tích phân $I=\displaystyle\int\limits_{0}^{2} |1-x| \mathrm{\,d}x$.
	\dapso{$1$}
	\loigiai{
		Cách 1. Ta có $1-x=0\Leftrightarrow x=1$ \\
		Và $1-x\geq 0,\, \forall x\in \left(0;1\right)$ \\
		Do đó
		$I=\displaystyle\int\limits_{0}^{1} (1-x) \mathrm{\,d}x+\displaystyle\int\limits_{1}^{2} \left(x-1\right) \mathrm{\,d}x=\left(x-\dfrac{x^2}{2}\right)\bigg{|}_0^1+\left(\dfrac{x^2}{2}-x\right)\bigg{|}_1^2=1$.\\	
		Cách 2. phương trình $1-x=0\Leftrightarrow x=1 \in (0;2)$, nên ta có
		\begin{eqnarray*}
			I&=&\int\limits_{0}^{2} |1-x| \mathrm{\,d}x=\int\limits_{0}^{1} |1-x| \mathrm{\,d}x+\int\limits_{1}^{2} |1-x| \mathrm{\,d}x=\bigg{|} \int\limits_{0}^{1} \left(1-x\right) \mathrm{\,d}x \bigg{|}+\bigg{|} \int\limits_{1}^{2} \left(1-x\right) \mathrm{\,d}x \bigg{|}\\
			&=&\bigg{|} 1-\dfrac{1}{2}\bigg{|}+\bigg{|}\dfrac{1}{2}-1\bigg{|}=1
		\end{eqnarray*}
	}
\end{bt}
\begin{bt}%[Nguyễn Thành Tiến - Dự án 12EX-DCHT] %[2D3B2-4]
	Tính tích phân $I=\displaystyle\int\limits_{0}^{2} \mid x^2-x \mid \mathrm{\,d}x$.
	\dapso{$1$}
	\loigiai{
		Ta có $x^2-x=0\Leftrightarrow \hoac{& x=0 \\ & x=1.}$\\
		Do đó
		\begin{eqnarray*}
			I&=&\int\limits_{0}^{2} \mid x^2-x \mid \mathrm{\,d}x=\int\limits_{0}^{1} \mid x^2-x \mid \mathrm{\,d}x+\int\limits_{1}^{2} \mid x^2-x \mid \mathrm{\,d}x\\
			&=&\bigg{|} \int\limits_{0}^{1} (x^2-x) \mathrm{\,d}x \bigg{|}+\bigg{|} \int\limits_{1}^{2} (x^2-x ) \mathrm{\,d}x \bigg{|}=\dfrac{1}{6}+\dfrac{5}{6}=1
		\end{eqnarray*}
		
	}
\end{bt}
%\setcounter{bt}{0}% Reset lại số đếm câu hỏi
\begin{dang}{Phương pháp đổi biến số}
\subsubsection{Phương pháp đổi biến số dạng 1}
\subsubsection*{Định lý}
Nếu 
\begin{enumerate}
	\item Hàm $x = u(t)$ có đạo hàm liên tục trên $[\alpha; \beta]$.
	\item Hàm hợp $f(u(t))$ được xác định trên $[\alpha; \beta]$.
	\item $u(\alpha) = a, u(\beta) = b$.
\end{enumerate}
Khi đó $\displaystyle\int\limits_a^bf(x)\mathrm{\,d}x = \displaystyle\int\limits_\alpha^\beta f(u(t))u'(t)\mathrm{\,d}t.$
\subsubsection*{Phương pháp chung}
\begin{itemize}
	\item \textbf{Bước 1:} Biến đổi để chọn phép đặt $x=u(t)\Rightarrow\mathrm{d}x=u'(t)\mathrm{d}t$.
	\item \textbf{Bước 2:} Đổi cận $\heva{&x=b\Rightarrow t=\beta\\&x=a\Rightarrow t=\alpha}.$
	\item\textbf{Bước 3:} Đưa về dạng $I=\displaystyle\int\limits_{\alpha}^{\beta} g(t)\mathrm{\,d}t$ đơn giản hơn và dễ tính toán.
\end{itemize}
Vậy  $I = \displaystyle\int\limits_a^bf(x)\mathrm{\,d}x = \displaystyle\int\limits_\alpha^\beta f(u[t])u'(t)\mathrm{\,d}t= 	\displaystyle\int\limits_\alpha^\beta g(t)\mathrm{\,d}t=	\left.G(t)\right|_\alpha^\beta=G(\beta)-G(\alpha).$	
\subsubsection{Phương pháp đổi biến số dạng 2}
\subsubsection*{Định lý}
Nếu hàm số $u=u(x)$ đơn điệu và có đạo hàm liên tục trên đoạn $\left[a;b\right]$ sao cho $f(x)\,\mathrm{d}x=g\left(u(x)\right)u'(x)\,\mathrm{d}x=g(u)\,\mathrm{d}u$ thì $I=\displaystyle\int\limits_{u(a)}^{u(b)}g(u)\,\mathrm{d}u$.
\subsubsection*{Phương pháp chung}
$$\displaystyle\int\limits_a^b \left [f(x) \right ]u'(x)\mathrm{\,d}x=F\left [u(x) \right ]\bigg|_{a}^b=F\left [u(b) \right]-F\left [u(a) \right ].$$
\begin{itemize}
	\item \textbf{Bước 1:} Biến đổi để chọn phép đặt $t=u(x)\Rightarrow\mathrm{d}t=u'(x)\mathrm{d}x$.
	\item \textbf{Bước 2:} Đổi cận $\heva{&x=b\Rightarrow t=u(b)\\&x=a\Rightarrow t=u(a)}.$
	\item\textbf{Bước 3:} Đưa về dạng $I=\displaystyle\int\limits_{u(a)}^{u(b)} g(t)\mathrm{\,d}t$ đơn giản hơn và dễ tính toán.
\end{itemize}
Vậy $I=\displaystyle\int\limits_{a}^{b}f(x)\,\mathrm{d}x=\displaystyle\int\limits_{a}^{b}g\left[u(x)\right].u'(x)\,\mathrm{d}x=\displaystyle\int\limits_{u(b)}^{u(a)}g(t)\,\mathrm{d}t$.

\end{dang}
\begin{bt}%[Lê Nguyễn Viết Tường, 12-EX1-DCHT]%[2D3B2-2]
	Tính các tích phân sau:
	\begin{enumerate}
		\item $I=\displaystyle\int\limits_{1}^{2} x(1-x)^{50}\mathrm{\,d}x$
		\dapso{$\dfrac{103}{2652}$}
		\loigiai{
			Đặt $t=1-x\Rightarrow x=1-t\Rightarrow\mathrm{d}x=-\mathrm{d}t$.\\
			Đổi cận $\heva{&x=1\Rightarrow t=0\\&x=2\Rightarrow t=-1}.$\\
			Khi đó $I=-\displaystyle\int\limits_{0}^{-1} (1-t)t^{50}\mathrm{\,d}t=\displaystyle\int\limits_{-1}^{0} \left(t^{50}-t^{51}\right) \mathrm{\,d}t=\left(\dfrac{t^{51}}{51}-\dfrac{t^{52}}{52}\right) \bigg|_{-1}^{0}=\dfrac{103}{2652}$.
		}
		\item $I=\displaystyle\int\limits_{0}^{1} x\left(1+x^2\right)^4\mathrm{\,d}x$
		
		\dapso{$\dfrac{31}{10}$}
		\loigiai{
			Đặt $t=1+x^2\Rightarrow x^2=t-1\Rightarrow x\mathrm{d}x=\dfrac{1}{2}\mathrm{d}t$.\\
			Đổi cận $\heva{&x=0\Rightarrow t=1\\&x=1\Rightarrow t=2}.$\\
			Khi đó $I=\dfrac{1}{2}\displaystyle\int\limits_{1}^{2} t^4\mathrm{\,d}t=\dfrac{1}{2}\cdot\dfrac{t^5}{5}\bigg|_{1}^{2}=\dfrac{1}{2}\left(\dfrac{2^5}{5}-\dfrac{1}{5}\right) =\dfrac{31}{10}$.
		}
		\item $I=\displaystyle\int\limits_{0}^{1} \dfrac{x^5}{x^2+1}\mathrm{\,d}x$
		
		\dapso{$\dfrac{1}{2}\left(\ln 2-\dfrac{1}{2}\right)$}
		\loigiai{
			Đặt $t=x^2+1\Rightarrow x^2=t-1\Rightarrow x\mathrm{d}x=\dfrac{1}{2}\mathrm{d}t$.\\
			Đổi cận $\heva{&x=0\Rightarrow t=1\\&x=1\Rightarrow t=2}.$\\
			Khi đó $I=\displaystyle\int\limits_{0}^{1} \dfrac{x^4}{x^2+1}x\mathrm{\,d}x=\dfrac{1}{2}\displaystyle\int\limits_{1}^{2} \dfrac{(t-1)^2}{t}\mathrm{\,d}t=\dfrac{1}{2}\displaystyle\int\limits_{1}^{2} \left(t-2+\dfrac{1}{t}\right) \mathrm{\,d}t=\dfrac{1}{2}\left(\dfrac{t^2}{2}-2t+\ln |t|\right) \bigg|_{1}^{2}=\dfrac{1}{2}\left(\ln 2-\dfrac{1}{2}\right)$.
		}
		\item $I=\displaystyle\int\limits_{0}^{1} \dfrac{x^3}{\left(1+x^2\right) ^3}\mathrm{\,d}x$
		
		\dapso{$\dfrac{1}{16}$}
		\loigiai{
			Đặt $t=1+x^2\Rightarrow x^2=t-1\Rightarrow x\mathrm{d}x=\dfrac{1}{2}\mathrm{d}t$.\\
			Đổi cận $\heva{&x=0\Rightarrow t=1\\&x=1\Rightarrow t=2}.$\\
			Khi đó $I=\displaystyle\int\limits_{0}^{1} \dfrac{x^2}{\left(1+x^2\right) ^3}x\mathrm{\,d}x=\dfrac{1}{2}\displaystyle\int\limits_{1}^{2} \dfrac{t-1}{t^3}\mathrm{\,d}t=\dfrac{1}{2}\displaystyle\int\limits_{1}^{2} \left(\dfrac{1}{t^2}-\dfrac{1}{t^3}\right) \mathrm{\,d}t=\dfrac{1}{2}\left(-\dfrac{1}{t}+\dfrac{1}{2t^2}\right) \bigg|_{1}^{2}=\dfrac{1}{16}$.
		}
		\item $I=\displaystyle\int\limits_{-1}^{1} \dfrac{2x+1}{\sqrt{x^2+x+1}}\mathrm{\,d}x$
		
		\dapso{$2\left(\sqrt{3}-1\right)$}
		\loigiai{
			Đặt $t=\sqrt{x^2+x+1}\Rightarrow\mathrm{d}t=\dfrac{2x+1}{2\sqrt{x^2+x+1}}\mathrm{d}x.$\\
			Đổi cận $\heva{&x=-1\Rightarrow t=1\\&x=1\Rightarrow t=\sqrt{3}}.$\\
			Khi đó $I=\displaystyle\int\limits_{1}^{\sqrt{3}} 2\mathrm{\,d}t=2t\bigg|_{1}^{\sqrt{3}}=2\left(\sqrt{3}-1\right) $.
		}
		\item $I=\displaystyle\int\limits_{0}^{1} x\sqrt{1-x}\mathrm{\,d}x$
		
		\dapso{$\dfrac{4}{15}$}
		\loigiai{
			Đặt $t=\sqrt{1-x}\Rightarrow x=1-t^2\Rightarrow\mathrm{d}x=-2t\mathrm{d}t$.\\
			Đổi cận $\heva{&x=0\Rightarrow t=1\\&x=1\Rightarrow t=0}.$\\
			Khi đó $I=-2\displaystyle\int\limits_{1}^{0} (1-t^2)t^2\mathrm{\,d}t=2\displaystyle\int\limits_{0}^{1} (t^2-t^4)\mathrm{\,d}t=2\left(\dfrac{t^3}{3}-\dfrac{t^5}{5}\right) \bigg|_{0}^{1}=\dfrac{4}{15}$.
		}
		\item $I=\displaystyle\int\limits_{0}^{1} x^3\sqrt{1+x^2}\mathrm{\,d}x$
		
		\dapso{$\dfrac{2\sqrt{2}+2}{15}$}
		\loigiai{
			Đặt $t=\sqrt{1+x^2}\Rightarrow x^2=t^2-1\Rightarrow x\mathrm{d}x=t\mathrm{d}t$.\\
			Đổi cận $\heva{&x=0\Rightarrow t=1\\&x=1\Rightarrow t=\sqrt{2}}.$\\
			Khi đó $I=\displaystyle\int\limits_{1}^{\sqrt{2}} \left(t^2-1\right) t^2\mathrm{\,d}t=\displaystyle\int\limits_{1}^{\sqrt{2}} \left(t^4-t^2\right) \mathrm{\,d}t=\left(\dfrac{t^5}{5}-\dfrac{t^3}{3}\right) \bigg|_{1}^{\sqrt{2}}=\dfrac{2\sqrt{2}+2}{15}$.
		}	
		\item $I=\displaystyle\int\limits_{\sqrt{7}}^{4} \dfrac{1}{x\sqrt{x^2+9}}\mathrm{\,d}x$
		
		\dapso{$\dfrac{1}{6}\ln \left(\dfrac{7}{4}\right) $}
		\loigiai{
			Đặt $t=\sqrt{x^2+9}\Rightarrow x=\sqrt{t^2-9}\Rightarrow\mathrm{d}x=\dfrac{t}{\sqrt{t^2-9}}\mathrm{d}t$.\\
			Đổi cận $\heva{&x=\sqrt{7}\Rightarrow t=4\\&x=4\Rightarrow t=5}.$\\
			Khi đó $I=\displaystyle\int\limits_{4}^{5} \dfrac{1}{(t-3)(t+3)}\mathrm{\,d}t=\dfrac{1}{6}\displaystyle\int\limits_{4}^{5} \left(\dfrac{1}{t-3}-\dfrac{1}{t+3}\right) \mathrm{\,d}t=\dfrac{1}{6}\left(\ln |t-3|-\ln |t+3|\right) \bigg|_{4}^{5}=\dfrac{1}{6}\ln \left(\dfrac{7}{4}\right) $.
		}
		\item $I=\displaystyle\int\limits_{1}^{\mathrm{e}}\dfrac{1+\ln ^{2}x}{x}\mathrm{\,d}x$ \dapso{$ \dfrac{4}{3} $}
		\loigiai{Đặt $ t=\ln x\Rightarrow \mathrm{\,d}t=\dfrac{1}{x}\mathrm{\,d}x$.\\
			Đổi cận: $ x=\mathrm{e}\Rightarrow t=1;x=1\Rightarrow t=0$.\\
			Khi đó $ I=\displaystyle\int\limits_0^1\left(1+t^2\right)\mathrm{\,d}t=\left.\left(t+\dfrac{t^3}{3}\right)\right|_0^1=1+\dfrac{1}{3}=\dfrac{4}{3}$.}
		\item Tính $I=\displaystyle\int\limits_0^{\tfrac{\pi}{2}} \sin^2x\cos x\mathrm{\,d}x$. \dapso{$I=\dfrac{1}{3}$}
		\loigiai{
			\begin{itemize}
				\item Đặt $t=\sin x\Rightarrow \mathrm{\,d}t=\cos x\mathrm{\,d}x$.
				\item Có $\heva{&x=0\Rightarrow t=0\\&x=\dfrac{\pi}{2}\Rightarrow t=1.}$
				\item Khi đó $I=\displaystyle\int\limits_0^1 t^2\mathrm{\,d}t=\dfrac{t^3}{3}\Biggl|_0^1=\dfrac{1}{3}$.
			\end{itemize}
		}
		\item 
		$I=\displaystyle\int\limits_0^{\tfrac{\pi}{3}} \cos^5x\mathrm{\,d}x$. 
		
		\dapso{$I=\dfrac{49\sqrt{3}}{160}$}
		\loigiai{
			\begin{itemize}
				\item Ta có $I=\displaystyle\int\limits_0^{\tfrac{\pi}{3}} \cos^4x\cos x\mathrm{\,d}x=\displaystyle\int\limits_0^{\tfrac{\pi}{3}} \left(1-\sin^2x\right)^2\cos x\mathrm{\,d}x$.
				\item Đặt $t=\sin x\Rightarrow \mathrm{\,d}t=\cos x\mathrm{\,d}x$.
				\item Có $\heva{&x=0\Rightarrow t=0\\&x=\dfrac{\pi}{3}\Rightarrow t=\dfrac{\sqrt{3}}{2}.}$
				\item Khi đó $I=\displaystyle\int\limits_0^{\tfrac{\sqrt{3}}{2}} (1-t^2)^2\mathrm{\,d}t=\displaystyle\int\limits_0^{\tfrac{\sqrt{3}}{2}} \left(1-2t^2+t^4 \right) \mathrm{\,d}t=\left( t-2\dfrac{t^3}{3}+\dfrac{t^5}{5}\right) \Biggl|_0^{\tfrac{\sqrt{3}}{2}}=\dfrac{49\sqrt{3}}{160}$.
			\end{itemize}
		}
		\item 
		$I = \displaystyle\int\limits_{0}^{\frac{\pi}{6}}\frac{\cos x}{6 - 5 \sin x + \sin^2x}\mathrm{\,d} x.$
		
		\dapso{$I=\ln \dfrac{10}{9}$}
		\loigiai{
			Đặt $t=\sin x \Rightarrow \mathrm{\,d} t=\cos x \mathrm{\,d} x.$\\
			Đổi cận $x=0\Rightarrow t=0; x=\dfrac{\pi}{6}\Rightarrow t= \dfrac{1}{2}.$\\
			Khi đó $I=\displaystyle\int\limits_0^{\frac{1}{2}}\frac{1}{t^2- 5t + 6}\mathrm{\,d} t=\displaystyle\int\limits_0^{\frac{1}{2}}\left(\dfrac{1}{t-3}-\dfrac{1}{t-2}\right)\mathrm{\,d} t=\ln \left|\dfrac{t-3}{t-2}\right|\Bigg|_0^{\frac{1}{2}}=\ln \dfrac{10}{9}.$
		}
		\item Tính $I = \displaystyle\int\limits_{0}^{\frac{\pi}{3}} \sin x \cos^4x \mathrm{\,d} x.$
		\dapso{$I=\dfrac{31}{160}$}
		\loigiai{
			Đặt $\cos x = t \Rightarrow - \sin x \mathrm{\,d}x = \mathrm{\,d}t$.\\
			Đổi cận $x = \dfrac{\pi}{3}\Rightarrow t = \dfrac{1}{2}; x = 0 \Rightarrow t = 1$.\\
			Khi đó $I =\displaystyle\int\limits_{1}^{\frac{1}{2}} - t ^{4}\mathrm{\,d}t = \left(- \dfrac{t ^{5}}{5}\right)\Big|_{1}^{\frac{1}{2}} = \dfrac{31}{160}$.
		}
		\item  $I=\displaystyle\int\limits_{\frac{\pi}{4}}^{\frac{\pi}{2}}\dfrac{\sin x -\cos x}{\sin x +\cos x} \mathrm{\,d}x$.   \dapso{$\dfrac{1}{2}\ln 2$}
		\loigiai{
			
			Đặt $t=\sin x +\cos x \Rightarrow   \mathrm{\,d}t =-\left(\sin x -\cos x\right) \mathrm{\,d}x$\\
			Đổi cận $x=\dfrac{\pi}{4}\Rightarrow t = \sqrt{2}$, $x=\dfrac{\pi}{2}\Rightarrow t = 1$\\
			Khi đó $I = \displaystyle\int\limits_1^{\sqrt{2}}\dfrac{1}{t}  \mathrm{\,d}t = \left. \ln \left| t\right|\right|_1^{\sqrt{2}}=\dfrac{1}{2}\ln 2$.}
		\item $I=\displaystyle \int\limits_{0}^{\ln 2}\mathrm{e}^{x}\sqrt{5-\mathrm{e}^{x}}\mathrm{\,d}x$
		
		\dapso{$\dfrac{16-6\sqrt{3}}{3}$}
		\loigiai{
			Đặt $t=\sqrt{5-\mathrm{e}^x}\Rightarrow t^2=5-\mathrm{e}^x \Rightarrow \heva{& \mathrm{e}^x =5-t^2\\& 2t\mathrm{\,d}t=-\mathrm{e}^x\mathrm{\,d}x.}$\\
			Đổi cận: $\heva{& x=0\Rightarrow t=2\\& x=\ln 2\Rightarrow t=\sqrt{3}}$.\\
			$I=\displaystyle \int\limits_{2}^{\sqrt{3}}t\cdot (-2t)\mathrm{\,d}t=2\displaystyle \int\limits_{\sqrt{3}}^{2}t^2\mathrm{\,d}t=2\left( \dfrac{t^3}{3}\right)\bigg|_{\sqrt{3}}^{2}=2\left( \dfrac{8}{3}-\dfrac{\sqrt{27}}{3}\right) =\dfrac{16-6\sqrt{3}}{3}$.	
		}
		\item $I=\displaystyle \int\limits_{1}^{\mathrm{e}\sqrt{\mathrm{e}}}\dfrac{3-2\ln x}{x\sqrt{1+2\ln x}}\mathrm{\,d}x$
	
	\dapso{$\dfrac{5}{3}$}
	\loigiai{
		Đặt $t=\sqrt{1+2\ln x}\Rightarrow t^2=1+2\ln x\Rightarrow \heva{& \ln x = \dfrac{t^2-1}{2}\\
			& 2t\mathrm{\,d}t=\dfrac{2\mathrm{\,d}x}{x}.}$\\
		Đổi cận: $\heva{& x=1\Rightarrow t=1\\& x=\mathrm{e}\sqrt{\mathrm{e}}\Rightarrow t=2}$.\\
		$I=\displaystyle \int\limits_{1}^{2}\dfrac{ 3-\left( t^2-1\right) }{t}\cdot t\mathrm{\,d}t=\displaystyle \int\limits_{1}^{2}\left( 4-t^2\right)\mathrm{\,d}t=\left( 4t-\dfrac{t^3}{3}\right)\bigg|_1^{2}=8-\dfrac{8}{3}-4+\dfrac{1}{3}=\dfrac{5}{3}$.
	}
	\end{enumerate}
\end{bt}
\begin{bt}%[Lê Xuân Hòa, dự án(12EX-1-DCHT-2019)]%[2D3B2-2]
	Tính các tích phân
	\begin{enumerate}
		\item  $I=\displaystyle\int\limits_0^1\sqrt{1-x^2} \mathrm{\,d}x$. \dapso{$ \dfrac{\pi}{4}$}
		\item   $I=\displaystyle\int\limits_{-\frac{1}{2}}^1\sqrt{1-x^2}\mathrm{\,d}x$.	\dapso{$\ \dfrac{\pi}{3}+\dfrac{\sqrt{3}}{8}$}
		\item  $I=\displaystyle\int\limits_0^2 x^2\sqrt{4-x^2}\mathrm{\,d}x$ \dapso{$\pi$}
		\item $I=\displaystyle \int \limits_{0}^{1} \dfrac{1}{1+x^2} \mathrm{d}x$ \dapso{$\dfrac{\pi}{4}$}
		\item $I= \displaystyle \int \limits_2^{4} \dfrac{1}{x^2-2x+4} \mathrm{d} x$ 
		
		\dapso{$I=\dfrac{\pi}{6\sqrt{3}}$}
		\item $I=\displaystyle \int \limits_{0}^{3} \dfrac{x+1}{9+x^2} \mathrm{d}x$ 
	\end{enumerate}

\loigiai{
	\begin{enumerate}
		\item Đặt $x=\sin t \Rightarrow  \mathrm{\,d}x = \cos t  \mathrm{\,d}t$.\\
		Đổi cận $x =0 \Rightarrow t=0; x= 1\Rightarrow t=\dfrac{\pi}{2}$.\\
		Khi đó $I = \displaystyle\int\limits_0^{\frac{\pi}{2}}\cos^2 t \mathrm{\,d}t   = \displaystyle\int\limits_0^{\frac{\pi}{2}}\dfrac{1}{2}\left(1+\cos 2t\right) \mathrm{\,d}t  =\left. \dfrac{1}{2}\left( t +\dfrac{1}{2}\sin 2t\right)\right| _0^{\frac{\pi}{2}}=\dfrac{\pi}{4}$.\\
		Vậy $I= \dfrac{\pi}{4}$.
		\item Đặt $x=\sin t \Rightarrow  \mathrm{\,d}x = \cos t  \mathrm{\,d}t$.\\
		Đổi cận $x =-\dfrac{1}{2} \Rightarrow t=-\dfrac{\pi}{6}; x= 1\Rightarrow t=\dfrac{\pi}{2}$.\\
		Khi đó $I = \displaystyle\int\limits_{-\frac{\pi}{6}}^{\frac{\pi}{2}}\cos^2 t \mathrm{\,d}t   = \displaystyle\int\limits_{-\frac{\pi}{6}}^{\frac{\pi}{2}}\dfrac{1}{2}\left(1+\cos 2t\right) \mathrm{\,d}t  =\left. \dfrac{1}{2}\left( t +\dfrac{1}{2}\sin 2t\right)\right| _{-\frac{\pi}{6}}^{\frac{\pi}{2}}=\dfrac{\pi}{3}+\dfrac{\sqrt{3}}{8}$.\\
		Vậy $I= \dfrac{\pi}{3}+\dfrac{\sqrt{3}}{8}$.
		\item Đặt $x=2\sin t \Rightarrow  \mathrm{\,d}x =2 \cos t  \mathrm{\,d}t$.\\
		Đổi cận $x =0 \Rightarrow t=0; x= 2\Rightarrow t=\dfrac{\pi}{2}$.\\
		Khi đó $$I = \displaystyle\int\limits_0^{\frac{\pi}{2}}4\sin^2 t\cdot  \sqrt{4-4\sin^2t}\cdot 2\cos t\mathrm{\,d}t = \displaystyle\int\limits_0^{\frac{\pi}{2}}4\sin^22t\mathrm{\,d}t  =2\displaystyle\int\limits_0^{\frac{\pi}{2}}(1-\cos 4t)\mathrm{\,d}t=\left. 2\left( t -\dfrac{1}{4}\sin 4t\right)\right| _0^{\frac{\pi}{2}}=\pi$$
		Vậy $I= \pi$.
		\item Đặt $x=\tan t \Rightarrow \mathrm{\, d} x = (1+\tan^2 t ) \mathrm{\, d}t $.\\
		\begin{tabular}{ll}
			Đổi cận: \quad & $x=0\Rightarrow t=0$ \\
			&  $x=1 \Rightarrow t= \dfrac{\pi}{4}$
		\end{tabular}
		$$I= \displaystyle \int \limits_0^{\frac{\pi}{4}} \dfrac{1}{1+\tan^2 t} \cdot (1+\tan^2 t ) \mathrm{\, d}t  = \displaystyle \int \limits_0^{\frac{\pi}{4}}  \mathrm{d}t =  t \bigg|_0^{\frac{\pi}{4}} = \dfrac{\pi}{4}.$$
		$I= \displaystyle \int \limits_2^{4} \dfrac{1}{x^2-2x+4} \mathrm{d} x =\displaystyle \int \limits_2^{4} \dfrac{1}{(x-1)^2+3} \mathrm{d} x $. 
		\item Đặt $x-1=\sqrt{3} \tan t \Rightarrow \mathrm{\, d}x = \sqrt{3} (1+\tan^2 t)\mathrm{\, d}t$.\\
		\begin{tabular}{ll}
			Đổi cận: \hfill &$x=4 \Rightarrow t = \dfrac{\pi}{3}$ \\ \\
			&$x=2 \Rightarrow t = \dfrac{\pi}{6}$.
		\end{tabular}
		$$I= \displaystyle \int \limits_{\frac{\pi}{6}}^{\frac{\pi}{3}} \dfrac{1}{3\tan^2 t +3}\sqrt{3} (1+\tan^2 t)\mathrm{\, d}t = \displaystyle \int \limits_{\frac{\pi}{6}}^{\frac{\pi}{3}} \dfrac{1}{\sqrt{3}} \mathrm{d}t = \dfrac{\pi}{6\sqrt{3}} .$$
\end{enumerate}
}
\end{bt}
\begin{dang}{Phương pháp tích phân từng phần}
	\textbf{1. Định lý}
	Nếu $u(x)$ và $v(x)$ là các hàm số có đạo hàm liên tục trên $\left[a;b\right]$ thì
	\[\displaystyle\int\limits_{a}^{b}u(x)v'(x)\,\mathrm{d}x =\left(u(x)v(x)\right)\Big|^{b}_a -\displaystyle\int\limits_{a}^{b}v(x)u'(x)\,\mathrm{d}x,\]
	hay 
	\[\displaystyle\int\limits_{a}^{b}u\,\mathrm{d}v = uv\Big|^b_a - \displaystyle\int\limits_{a}^{b}v\,\mathrm{d}u.\]
	
	\textbf{2. Phương pháp chung}
	\begin{itemize}
		\item \textbf{Nhận dạng} Tích hai hàm khác loại nhân nhau, chẳng hạn: mũ nhân lượng giác, $\ldots$\\
		\item \textbf{Đặt:} $\heva{&u=\cdots\cdots\,\,\overset{VP}\longrightarrow\,\mathrm{\, d}u=\cdots\cdots\mathrm{\, d}x\\
			&\mathrm{\, d}v=\cdots\mathrm{\, d}x\,\,\overset{NH}{\longrightarrow}\, v=\cdots\cdots}$.
		
		\item Suy ra $I=\displaystyle\int\limits_a^bu\mathrm{\, d}v=\left.uv\right|_a^b-\displaystyle\int\limits_a^bv\mathrm{\, d}u.$
	\end{itemize}
	{\bf Cách đặt $u$ và $dv$ trong phương pháp tích phân từng phần}
	
	\begin{longtable}{|p{4cm}|p{1.5cm}|p{1.8cm}|p{1.8cm}|p{1.5cm}|}
		\hline
		Đặt $u$ theo thứ tự ưu tiên \textbf{Log - đa - lũy-mũ - lượng}& \footnotesize $ \displaystyle\int\limits_{a}^{b}P(x)e^x\,\mathrm{d}x$&\footnotesize $\displaystyle\int\limits_{a}^{b}P(x)\ln (x)\,\mathrm{d}x$ & \footnotesize$\displaystyle\int\limits_{a}^{b}P(x)\cos x\,\mathrm{d}x$&\footnotesize $\displaystyle\int\limits_{a}^{b}e^x\cos x\,\mathrm{d}x$ \\
		\hline
		$u$& $P(x)$&$\ln x$& $P(x)$&$e^x$ \\
		\hline
		$dv$& $e^x\,\mathrm{d}x$& $P(x)\,\mathrm{d}x$ & $\cos x\,\mathrm{d}x$& $\cos x\,\mathrm{d}x$\\
		\hline
	\end{longtable}
	\textbf{Chú ý:} \textbf{Thứ tự ưu tiên chọn $u$: \emph{loga - đa -lũy- lượng - mũ}} và $\mathrm{\, d}v=$ \textbf{\emph{phần còn lại}}. Nghĩa là nếu có $\ln x$ hay $\log_a x$ thì chọn $u=\ln x$ hay $u=\log_a x=\dfrac{1}{\ln a}\cdot\ln x$ và $\mathrm{\, d}v=$ còn lại. Nếu không có $\ln $, $\log$ thì chọn $u=$ đa thức và $\mathrm{\, d}v=$ còn lại. Nếu không có $\log$, đa thức, ta chọn $u=$ lượng giác, $\ldots$\\
	- \textbf{Lưu ý} rằng bậc của đa thức và bậc của $\ln $ tương ứng với số lần lấy nguyên hàm.\\
	- Dạng \emph{\textbf{mũ nhân lượng giác}} là dạng nguyên hàm từng phần luân hồi.	
\end{dang}

\subsubsection{Ví dụ - Bài tập áp dụng}
\begin{vd}%[Nguyện Ngô]%[DCHT]%[2D3K2-3]
    Tính $I=\displaystyle\int\limits_0^{1} (x-3) \mathrm{e}^x \mathrm{\, d}x$.
        \dapso{$I=4-3\mathrm{e}$}
    \loigiai
    {
    Chọn $\heva{&u=x-3 \Rightarrow \mathrm{\, d}u =\mathrm{\, d}x \\ &\mathrm{\, d}v = \mathrm{e}^x \mathrm{\, d}x \Rightarrow v=\mathrm{e}^x.}$\\
    Khi đó $I= (x-3)\mathrm{e}^x\Big|_0^1 - \displaystyle\int\limits_0^{1} \mathrm{e}^x \mathrm{\, d}x = -2\mathrm{e}+3 - \mathrm{e}^x \Big|_0^1 = 4-3\mathrm{e}$.
    }
    \end{vd}
    
     \begin{vd}%[Nguyện Ngô]%[DCHT]%[2D3K2-3]
    Tính $I=\displaystyle\int\limits_0^{1} (x^2+2x)\mathrm{e}^{x} \mathrm{\, d}x$.
        \dapso{$I=\mathrm{e}$}
    \loigiai
    {
    Chọn $\heva{&u=x^2+2x \Rightarrow \mathrm{\, d}u =2(x+1)\mathrm{\, d}x \\ &\mathrm{\, d}v = \mathrm{e}^{x} \mathrm{\, d}x \Rightarrow v= \mathrm{e}^{x}.}$\\
    Khi đó $I=(x^2+2x)\mathrm{e}^{x} \Big|_0^1 -2\displaystyle\int\limits_0^{1} (x+1)\mathrm{e}^{x} \mathrm{\, d}x = 3\mathrm{e}-
    2\displaystyle\int\limits_0^{1} (x+1)\mathrm{e}^{x} \mathrm{\, d}x =3\mathrm{e}-2J$.\\
    Tính $J$:
    Chọn $\heva{&u_1=x+1 \Rightarrow \mathrm{\, d}u_1 =\mathrm{\, d}x \\ &\mathrm{\, d}v = \mathrm{e}^{x} \mathrm{\, d}x \Rightarrow v=\mathrm{e}^{x}.}$\\
    Khi đó $J=(x+1)\mathrm{e}^{x} \Big|_0^1 -\displaystyle\int\limits_0^{1} \mathrm{e}^{x} \mathrm{\, d}x = 2\mathrm{e}-1-\mathrm{e}^{x} \Big|_0^1 = \mathrm{e}$.\\
    Vậy $I=3\mathrm{e}-2\mathrm{e}=\mathrm{e}$.
    }
    \end{vd}
    
    \begin{vd}%[Nguyện Ngô]%[DCHT]%[2D3K2-3]
    Tính $I=\displaystyle\int\limits_0^{\pi}  \mathrm{e}^{x} \cos x \mathrm{\, d}x$.
        \dapso{$I=-\dfrac{1}{2} \left({\mathrm{e}^{\pi}}+1\right)$}
    \loigiai
    {
    Chọn $\heva{&u=\cos x \Rightarrow \mathrm{\, d}u =-\sin x\mathrm{\, d}x \\ &\mathrm{\, d}v = \mathrm{e}^{x} \mathrm{\, d}x \Rightarrow v= \mathrm{e}^{x}.}$\\
    Khi đó $I=\mathrm{e}^{x} \cos x \Big|_0^{\pi} +\displaystyle\int\limits_0^{\pi} \sin x \mathrm{e}^{x} \mathrm{\, d}x =  -\mathrm{e}^{\pi}-1+J$.\\
    Tính $J$.
    Chọn $\heva{&u_1=\sin x \Rightarrow \mathrm{\, d}u_1 =\cos x \mathrm{\, d}x \\ &\mathrm{\, d}v = \mathrm{e}^{x} \mathrm{\, d}x \Rightarrow v=\mathrm{e}^{x}.}$\\
    Khi đó $J=\sin x\mathrm{e}^{x} \Big|_0^{\pi} -I=-I$.\\
    Vậy $I=-\dfrac{1}{2} \left({\mathrm{e}^{\pi}}+1\right)$.
    }
    \end{vd}
    \textbf{Bài tập áp dụng}
    \begin{bt}%[Nguyện Ngô]%[DCHT]%[2D3K2-3]
    Tính các tích phân sau:
    \begin{enumerate}
    \item $I=\displaystyle\int\limits_0^{1} x\mathrm{e}^x \mathrm{\, d}x$.
        \dapso{$I=1$}
    \loigiai
    {
    Chọn $\heva{&u=x \Rightarrow \mathrm{\, d}u =\mathrm{\, d}x \\ &\mathrm{\, d}v = \mathrm{e}^x \mathrm{\, d}x \Rightarrow v=\mathrm{e}^x.}$\\
    Khi đó $I= x\mathrm{e}^x \Big|_0^1 - \displaystyle\int\limits_0^{1} \mathrm{e}^x \mathrm{\, d}x = \mathrm{e} - \mathrm{e}^x \Big|_0^1 =1$.
    }
    \item $I=\displaystyle\int\limits_0^{2} (2x-1)\mathrm{e}^x \mathrm{\, d}x$.
        \dapso{$I=\mathrm{e^2}+3$}
    \loigiai
    {
    Chọn $\heva{&u=2x-1 \Rightarrow \mathrm{\, d}u =2\mathrm{\, d}x \\ &\mathrm{\, d}v = \mathrm{e}^x \mathrm{\, d}x \Rightarrow v=\mathrm{e}^x.}$\\
    Khi đó $I= (2x-1)\mathrm{e}^x \Big|_0^2 - 2\displaystyle\int\limits_0^{2} \mathrm{e}^x \mathrm{\, d}x = 3\mathrm{e}^2+1 - 2\mathrm{e}^x \Big|_0^2 = \mathrm{e}^2+3$.
    }
    \item $I=\displaystyle\int\limits_0^{1} (2x+1)\mathrm{e}^x \mathrm{\, d}x$.
        \dapso{$I=e+1$}
    \loigiai
    {
    Chọn $\heva{&u=2x+1 \Rightarrow \mathrm{\, d}u =2\mathrm{\, d}x \\ &\mathrm{\, d}v = \mathrm{e}^x \mathrm{\, d}x \Rightarrow v=\mathrm{e}^x.}$\\
    Khi đó $I= (2x+1)\mathrm{e}^x \Big|_0^1 - 2\displaystyle\int\limits_0^{1} \mathrm{e}^x \mathrm{\, d}x = 3\mathrm{e}-1 - 2\mathrm{e}^x \Big|_0^1 = \mathrm{e}+1$.
    }
    \item $I=\displaystyle\int\limits_0^{1} (4x-1)\mathrm{e}^{2x} \mathrm{\, d}x$.
        \dapso{$I=\dfrac{1}{2}\mathrm{e}^2+\dfrac{3}{2}$}
    \loigiai
    {
    Chọn $\heva{&u=4x-1 \Rightarrow \mathrm{\, d}u =4\mathrm{\, d}x \\ &\mathrm{\, d}v = \mathrm{e}^{2x} \mathrm{\, d}x \Rightarrow v=\dfrac {1}{2} \mathrm{e}^{2x}.}$\\
    Khi đó $I=\dfrac {1}{2} (4x-1)\mathrm{e}^{2x} \Big|_0^1 - 2\displaystyle\int\limits_0^{1} \mathrm{e}^{2x} \mathrm{\, d}x
    = \dfrac {3}{2}\mathrm{e}^2+\dfrac{1}{2}- \mathrm{e}^{2x} \Big|_0^1 = \dfrac{1}{2}\mathrm{e}^2+\dfrac{3}{2}$.
    }
    \item $I=\displaystyle\int\limits_0^{1} (x-1)\mathrm{e}^{2x} \mathrm{\, d}x$.
        \dapso{$I=\dfrac{3}{4}-\dfrac {1}{4} \mathrm{e}^2$}
    \loigiai
    {
    Chọn $\heva{&u=x-1 \Rightarrow \mathrm{\, d}u =\mathrm{\, d}x \\ &\mathrm{\, d}v = \mathrm{e}^{2x} \mathrm{\, d}x \Rightarrow v=\dfrac {1}{2} \mathrm{e}^{2x}.}$\\
    Khi đó $I=\dfrac {1}{2} (x-1)\mathrm{e}^{2x}\Big|_0^1 - \dfrac {1}{2} \displaystyle\int\limits_0^{1} \mathrm{e}^{2x} \mathrm{\, d}x
    = \dfrac {1}{2}-\dfrac {1}{4} \mathrm{e}^{2x} \Big|_0^1 = \dfrac{3}{4}-\dfrac {1}{4} \mathrm{e}^2$.
    }
    \item $I=\displaystyle\int\limits_1^{3} x\mathrm{e}^{-x} \mathrm{\, d}x$.
        \dapso{$I=-\dfrac{4}{\mathrm{e}^{3}}+\dfrac {2}{\mathrm{e}}$}
    \loigiai
    {
    Chọn $\heva{&u=x \Rightarrow \mathrm{\, d}u =\mathrm{\, d}x \\ &\mathrm{\, d}v = \mathrm{e}^{-x} \mathrm{\, d}x \Rightarrow v=- \mathrm{e}^{-x}.}$\\
    Khi đó $I=-x\mathrm{e}^{-x}\Big|_1^3 + \displaystyle\int\limits_1^{3} \mathrm{e}^{-x} \mathrm{\, d}x
    = -3 \mathrm{e}^{-3}+\mathrm{e}^{-1}-\mathrm{e}^{-x}\Big|_1^3 = -\dfrac{3}{\mathrm{e}^{3}}+\dfrac{1}{\mathrm{e}}-\dfrac{1}{\mathrm{e}^{3}}+\dfrac{1}{\mathrm{e}}=-\dfrac{4}{\mathrm{e}^{3}}+\dfrac {2}{\mathrm{e}}$.
    }
    \item $I=\displaystyle\int\limits_0^{2} (1-2x)\mathrm{e}^{-x} \mathrm{\, d}x$.
        \dapso{$I=\dfrac{5}{\mathrm{e}^{2}}-1$}
    \loigiai
    {
    Chọn $\heva{&u=1-2x \Rightarrow \mathrm{\, d}u =-2\mathrm{\, d}x \\ &\mathrm{\, d}v = \mathrm{e}^{-x} \mathrm{\, d}x \Rightarrow v=- \mathrm{e}^{-x}.}$\\
    Khi đó $I=-(1-2x)\mathrm{e}^{-x} \Big|_0^2 -2\displaystyle\int\limits_0^{2} \mathrm{e}^{-x} \mathrm{\, d}x = 3 \mathrm{e}^{-2}+1+2\mathrm{e}^{-x} \Big|_0^2 = \dfrac{3}{\mathrm{e}^{2}}+1+\dfrac{2}{\mathrm{e}^{2}}-2=\dfrac{5}{\mathrm{e}^{2}}-1$.
    }
    \item $I=\displaystyle\int\limits_1^{3} x^2 \mathrm{e}^{-x} \mathrm{\, d}x$.
        \dapso{$I=-\dfrac{17}{\mathrm{e}^3}+\dfrac{5}{\mathrm{e}}$}
    \loigiai
    {
    Chọn $\heva{&u=x^2 \Rightarrow \mathrm{\, d}u =2x\mathrm{\, d}x \\ &\mathrm{\, d}v = \mathrm{e}^{-x} \mathrm{\, d}x \Rightarrow v= -\mathrm{e}^{-x}.}$\\
    Khi đó $I=-x^2\mathrm{e}^{-x} \Big|_1^3 +2\displaystyle\int\limits_1^{3} x\mathrm{e}^{-x} \mathrm{\, d}x = -\dfrac{9}{\mathrm{e}^3}+\dfrac{1}{\mathrm{e}}+2J$.\\
    Tính $J$.
    Chọn $\heva{&u=x \Rightarrow \mathrm{\, d}u =\mathrm{\, d}x \\ &\mathrm{\, d}v = \mathrm{e}^{-x} \mathrm{\, d}x \Rightarrow v=-\mathrm{e}^{-x}.}$\\
    Khi đó $J=-x\mathrm{e}^{-x} \Big|_1^3 +\displaystyle\int\limits_1^{3} \mathrm{e}^{-x} \mathrm{\, d}x = -3\mathrm{e}^{-3}+\mathrm{e}^{-1}-\mathrm{e}^{-x} \Big|_1^3 = -3\mathrm{e}^{-3}+\mathrm{e}^{-1}-\mathrm{e}^{-3}+\mathrm{e}^{-1}=-\dfrac{4}{\mathrm{e}^3}+\dfrac{2}{\mathrm{e}}$.\\
    Vậy $I=-\dfrac{17}{\mathrm{e}^3}+\dfrac{5}{\mathrm{e}}$.
    }
    \item $I=\displaystyle\int\limits_0^{\tfrac{\pi}{4}}  5\mathrm{e}^{x} \sin 2x \mathrm{\, d}x$.
        \dapso{$I=\mathrm{e}^{\tfrac{\pi}{4}}+2$}
    \loigiai
    {
    $I=\displaystyle\int\limits_0^{\tfrac{\pi}{4}}  5\mathrm{e}^{x} \sin 2x \mathrm{\, d}x =5\displaystyle\int\limits_0^{\tfrac{\pi}{4}}  \mathrm{e}^{x} \sin 2x \mathrm{\, d}x =5J$.\\
    Chọn $\heva{&u=\sin 2x \Rightarrow \mathrm{\, d}u =2\cos 2x\mathrm{\, d}x \\ &\mathrm{\, d}v = \mathrm{e}^{x} \mathrm{\, d}x \Rightarrow v= \mathrm{e}^{x}.}$\\
    Khi đó $J=\mathrm{e}^{x} \sin 2x \Big|_0^{\tfrac{\pi}{4}} -2\displaystyle\int\limits_0^{\tfrac{\pi}{4}} \cos 2x \mathrm{e}^{x} \mathrm{\, d}x= \mathrm{e}^{\tfrac{\pi}{4}}-2K$.\\
    Tính $K$.
    Chọn $\heva{&u=\cos 2x \Rightarrow \mathrm{\, d}u =-2\sin 2x \mathrm{\, d}x \\ &\mathrm{\, d}v = \mathrm{e}^{x} \mathrm{\, d}x \Rightarrow v=\mathrm{e}^{x}.}$\\
    Khi đó $K=\cos 2xe^{x} \Big|_0^{\tfrac{\pi}{4}}+2J=-1+2J$.\\
    Vậy $I=\mathrm{e}^{\tfrac{\pi}{4}}+2$.
    }
    \item $I=\displaystyle\int\limits_0^{\tfrac{\pi}{2}}  \mathrm{e}^{-x} \cos x \mathrm{\, d}x$.
        \dapso{$I=\dfrac{\mathrm{e}^{-\tfrac{\pi}{2}}+1}{2}$}
    \loigiai
    {
    Chọn $\heva{&u=\cos x \Rightarrow \mathrm{\, d}u =-\sin x\mathrm{\, d}x \\ &\mathrm{\, d}v = \mathrm{e}^{-x} \mathrm{\, d}x \Rightarrow v= -\mathrm{e}^{-x}.}$\\
    Khi đó $I=-\mathrm{e}^{-x} \cos x \Big|_0^{\tfrac{\pi}{2}} -\displaystyle\int\limits_0^{\tfrac{\pi}{2}} \sin x \mathrm{e}^{-x} \mathrm{\, d}x= 1-J$.\\
    Tính $J$.
    Chọn $\heva{&u=\sin x \Rightarrow \mathrm{\, d}u =\cos x \mathrm{\, d}x \\ &\mathrm{\, d}v = \mathrm{e}^{-x} \mathrm{\, d}x \Rightarrow v=-\mathrm{e}^{-x}.}$\\
    Khi đó $J=-\mathrm{e}^{-x}\sin x \Big|_0^{\tfrac{\pi}{2}}+I=-\mathrm{e}^{-\tfrac{\pi}{2}}+I$.\\
    Vậy $2I=1+\mathrm{e}^{-\tfrac{\pi}{2}} \Rightarrow I=\dfrac{\mathrm{e}^{-\tfrac{\pi}{2}}+1}{2}$.
    }
    \item $I=\displaystyle\int\limits_0^{\dfrac{\pi}{4}}  \mathrm{e}^{3x} \sin 4x \mathrm{\, d}x$.
        \dapso{$I=\dfrac{4\mathrm{e}^{\tfrac{3\pi}{4}}+4}{25}$}
    \loigiai
    {
    Chọn $\heva{&u=\sin 4x \Rightarrow \mathrm{\, d}u =4\cos 4x\mathrm{\, d}x \\ &\mathrm{\, d}v = \mathrm{e}^{3x} \mathrm{\, d}x \Rightarrow v= \dfrac {1}{3}\mathrm{e}^{3x}.}$\\
    Khi đó $I=\dfrac {1}{3} \sin 4x\mathrm{e}^{3x} \Big|_0^{\dfrac{\pi}{4}} -\dfrac {4}{3}\displaystyle\int\limits_0^{\dfrac{\pi}{4}} \cos 4x \mathrm{e}^{3x} \mathrm{\, d}x= 0-\dfrac {4}{3}J =-\dfrac {4}{3}J$.\\
    Tính $J$.
    Chọn $\heva{&u=\cos 4x \Rightarrow \mathrm{\, d}u =-4\sin 4x \mathrm{\, d}x \\ &\mathrm{\, d}v = \mathrm{e}^{3x} \mathrm{\, d}x \Rightarrow v= \dfrac {1}{3}\mathrm{e}^{3x}.}$\\
    Khi đó $J=\dfrac{1}{3}\mathrm{e}^{3x}\cos4x\Big|_0^{\tfrac{\pi}{4}}+\dfrac{4}{3}I=-\dfrac{1}{3}e^{\tfrac{3\pi}{4}}-\dfrac{1}{3}+\dfrac{4}{3}I$.\\
    Suy ra $I=\dfrac{4\mathrm{e}^{\tfrac{3\pi}{4}}+4}{25}$.
    }
    \item $I=\displaystyle\int\limits_0^{\tfrac{\pi}{2}}  \mathrm{e}^{x} \cos 2x \mathrm{\, d}x$.
        \dapso{$I=-\dfrac{\mathrm{e}^{\tfrac{\pi}{2}}+1}{5}$}
    \loigiai
    {
    Chọn $\heva{&u=\cos 2x \Rightarrow \mathrm{\, d}u =-2\sin 2x\mathrm{\, d}x \\ &\mathrm{\, d}v = \mathrm{e}^x \mathrm{\, d}x \Rightarrow v= \mathrm{e}^x.}$\\
    Khi đó $I=\cos 2x\mathrm{e}^x \Big|_0^{\tfrac{\pi}{2}} +2\displaystyle\int\limits_0^{\tfrac{\pi}{2}} \sin 2x \mathrm{e}^x \mathrm{\, d}x= -\mathrm{e}^{\tfrac{\pi}{2}}-1+2J$.\\
    Tính $J$.
    Chọn $\heva{&u=\sin 2x \Rightarrow \mathrm{\, d}u =2\cos 2x \mathrm{\, d}x \\ &\mathrm{\, d}v = \mathrm{e}^x \mathrm{\, d}x \Rightarrow v= \mathrm{e}^x.}$\\
    Khi đó $J=\mathrm{e}^x \sin 2x \Big|_0^{\tfrac{\pi}{2}}-2I=-2I$.\\
    Vậy $I=-1-\mathrm{e}^{\tfrac{\pi}{2}} -4I \Rightarrow I=-\dfrac {\mathrm{e}^{\tfrac{\pi}{2}}+1}{5}$.
    }
    \item $I=\displaystyle\int\limits_0^{1} \dfrac {3x+1}{\mathrm{e}^{2x}} \mathrm{\, d}x$.
        \dapso{$I=-\dfrac {11}{4\mathrm{e}^2}+\dfrac {5}{4}$}
    \loigiai
    {
    Chọn $\heva{&u=3x+1\Rightarrow \mathrm{\, d}u =3\mathrm{\, d}x \\ &\mathrm{\, d}v = \mathrm{e}^{-2x} \mathrm{\, d}x
    \Rightarrow v=-\dfrac {1}{2}\mathrm{e}^{-2x}.}$\\
    Khi đó $I=-\dfrac {1}{2}(3x+1)\mathrm{e}^{-2x} \Big|_0^1 +\dfrac {3}{2}\displaystyle\int\limits_0^{1} \mathrm{e}^{-2x} \mathrm{\, d}x
    = -2\mathrm{e}^{-2}+\dfrac {1}{2}-\dfrac {3}{4}\mathrm{e}^{-2x} \Big|_0^1 = -\dfrac {11}{4\mathrm{e}^2}+\dfrac {5}{4}$.
    }
    \end{enumerate}
    \end{bt}
    \subsubsection{Ví dụ - Bài tập áp dụng}
    \begin{vd}%[Nguyện Ngô]%[DCHT]%[2D3K2-3]
    Tính $I=\displaystyle\int\limits_1^{3} \ln x \mathrm{\, d}x$.
        \dapso{$I=3\ln 3-2$}
    \loigiai{
    Chọn $\heva{&u=\ln x \Rightarrow \mathrm{\, d}u = \dfrac {1}{x}\mathrm{\, d}x \\ &\mathrm{\, d}v =\mathrm{\, d}x \Rightarrow v=x.}$\\
    Khi đó $I= x\ln x \Big|_1^3 - \displaystyle\int\limits_0^{1}  \mathrm{\, d}x = 3\ln 3-x\Big|_1^3 =3\ln 3-2$.
    }
    \end{vd}
    
    \begin{vd}%[Nguyện Ngô]%[DCHT]%[2D3K2-3]
    Tính $I=\displaystyle\int\limits_1^{\mathrm{e}} x^2 \ln x \mathrm{\, d}x$.
        \dapso{$I= \dfrac {2\mathrm{e}^3}{9}+\dfrac {1}{9}$}
    \loigiai{
    Chọn $\heva{&u=\ln x \Rightarrow \mathrm{\, d}u = \dfrac {1}{x}\mathrm{\, d}x \\ &\mathrm{\, d}v =x^2\mathrm{\, d}x \Rightarrow v=\dfrac {x^3}{3}.}$\\
    Khi đó $I= \dfrac {x^3}{3}\ln x \Big|_1^{\mathrm{e}} -\dfrac {1}{3} \displaystyle\int\limits_1^{\mathrm{e}}x^2\mathrm{\, d}x
     = \dfrac {\mathrm{e}^3}{3}-\dfrac {1}{9} x^3\Big|_1^{\mathrm{e}} =\dfrac {\mathrm{e}^3}{3}-\dfrac {\mathrm{e}^3}{9}+\dfrac {1}{9}=
     \dfrac {2\mathrm{e}^3}{9}+\dfrac {1}{9}$.
    }
    \end{vd}
    
    \begin{vd}%[Nguyện Ngô]%[DCHT]%[2D3K2-3]
    Tính $I=\displaystyle\int\limits_1^{\mathrm{e}} x\ln ^2 x \mathrm{\, d}x$.
        \dapso{$I=\dfrac {\mathrm{e}^2}{4} -\dfrac {1}{4}$}
    \loigiai{
    Chọn $\heva{&u=\ln ^2 x \Rightarrow \mathrm{\, d}u = \dfrac {2}{x} \ln x \mathrm{\, d}x \\ &\mathrm{\, d}v =x\mathrm{\, d}x \Rightarrow v=\dfrac {x^2}{2}.}$\\
    Khi đó $I= \dfrac {x^2}{2}\ln ^2 x \Big|_1^e - \displaystyle\int\limits_1^{e} x\ln x \mathrm{\, d}x = \dfrac {\mathrm{e}^2}{2} -J$.\\
    Tính $J$.
    Chọn $\heva{&u=\ln x \Rightarrow \mathrm{\, d}u = \dfrac {1}{x}\mathrm{\, d}x \\ &dv =x\mathrm{\, d}x \Rightarrow v=\dfrac {x^2}{2}.}$\\
    Khi đó $J= \dfrac {x^2}{2}\ln x \Big|_1^e- \dfrac {1}{2} \displaystyle\int\limits_1^{e} x\mathrm{\, d}x = \dfrac {\mathrm{e}^2}{2} - \dfrac {1}{4} x^2 \Big|_1^e= \dfrac {\mathrm{e}^2}{2} -\dfrac {\mathrm{e}^2}{4}+\dfrac {1} {4}= \dfrac {\mathrm{e}^2}{4} +\dfrac {1}{4}$.\\
    Vậy $I=\dfrac {\mathrm{e}^2}{2}-\dfrac {\mathrm{e}^2}{4} -\dfrac {1}{4} \Rightarrow I=\dfrac {\mathrm{e}^2}{4} -\dfrac {1}{4}$..
    }
    \end{vd}
    
    \begin{vd}%[Nguyện Ngô]%[DCHT]%[2D3K2-3]
    Tính $I=\displaystyle\int\limits_0^{1} (2x-1)\ln (x+1) \mathrm{\, d}x$.
        \dapso{$I=\dfrac{3}{2}-\ln 4$}
    \loigiai{
    Chọn $\heva{&u=\ln (x+1) \Rightarrow \mathrm{\, d}u = \dfrac {1}{x+1}\mathrm{\, d}x \\ &\mathrm{\, d}v =(2x-1)\mathrm{\, d}x
    \Rightarrow v=x^2-x.} $\\
    Khi đó $I= (x^2-x)\ln (x+1)\Big|_0^{1} - \displaystyle\int\limits_0^{1} \dfrac {x^2-x}{x+1}\mathrm{\, d}x
    = -\displaystyle\int\limits_0^{1} \left(x-2 + \dfrac {2}{x+1}\right)\mathrm{\, d}x = -\left(\dfrac {x^2}{2} -2x+2\ln |x+1|\right)\Big|_0^{1}=\dfrac{3}{2}-\ln 4$.
    }
    \end{vd}
    
    \begin{vd}%[Nguyện Ngô]%[DCHT]%[2D3G2-3]
    Tính $I=\displaystyle\int\limits_0^{\tfrac{\pi}{4}}\dfrac{\ln (\sin x +2\cos x)}{\cos^2 x} \mathrm{\, d}x$.
        \dapso{$I=\ln \dfrac{27\sqrt{2}}{8}-\dfrac{\pi}{4}$}
    \loigiai{
    Với mọi $x\in\left[0;\dfrac{\pi}{4}\right]$, ta có
    \[\dfrac{\ln (\sin x +2\cos x)}{\cos^2 x} =\dfrac{\ln [\cos x(\tan x+2)]}{\cos^2x}=\dfrac{\ln \cos x}{\cos^2x}+\dfrac{\ln (\tan x+2)}{\cos^2x}.\]
    Tính $I_1=\displaystyle\int\limits_0^{\tfrac{\pi}{4}} \dfrac{\ln \cos x}{\cos^2 x} \mathrm{\, d}x$.\\
    Chọn $\heva{&u=\ln \cos x  \Rightarrow \mathrm{\, d}u = \dfrac {1}{\cos x}\cdot(-\sin x)\mathrm{\, d}x \\
    &\mathrm{\, d}v = \dfrac{1}{\cos^2x}\mathrm{\, d}x \Rightarrow v=\tan x.}$\\
    Khi đó
    \begin{eqnarray*}I_1&=& \tan x\ln \cos x\Big|_{0}^{\tfrac{\pi}{4}}
     + \displaystyle\int\limits_{0}^{\tfrac{\pi}{4}} \tan^2x\mathrm{\, d}x\\
    &=& \ln \dfrac{\sqrt{2}}{2}+\displaystyle\int\limits_{0}^{\tfrac{\pi}{4}}\left(\dfrac{1}{\cos^2x}-1\right)\mathrm{\, d}x\\
    &=&  \ln \dfrac{\sqrt{2}}{2}+\left(\tan x-x\right)\Big|_{0}^{\tfrac{\pi}{4}}\\
    &=&\ln \dfrac{\sqrt{2}}{2}+1-\dfrac{\pi}{4}.
    \end{eqnarray*}
    Tính $I_2=\displaystyle\int\limits_{0}^{\tfrac{\pi}{4}} \dfrac{\ln (\tan x+2)}{\cos^2 x} \mathrm{\, d}x$.\\
    Chọn $\heva{&u=\ln (\tan x+2)  \Rightarrow \mathrm{\, d}u = \dfrac {1}{\tan x+2}\mathrm{\, d}(\tan x) \\
    &\mathrm{\, d}v = \dfrac{1}{\cos^2x}\mathrm{\, d}x \Rightarrow v=\tan x.}$\\
    Khi đó
    \begin{eqnarray*}I_2&=& \tan x\ln (\tan x+2)\Big|_{0}^{\tfrac{\pi}{4}}
     - \displaystyle\int\limits_{0}^{\tfrac{\pi}{4}} \dfrac{\tan x}{\tan x+2}\mathrm{\, d}(\tan x)\\
    &=& \ln 3-\displaystyle\int\limits_{0}^{\tfrac{\pi}{4}}\left(1-\dfrac{2}{\tan x+2}\right)\mathrm{\, d}(\tan x)\\
    &=&  \ln 3-\left(\tan x-2\ln |\tan x+2|\right)\Big|_{0}^{\tfrac{\pi}{4}}\\
    &=&3\ln 3-2\ln 2-1.
    \end{eqnarray*}
    Vậy $I=I_1+I_2=\ln \dfrac{\sqrt{2}}{2}+1-\dfrac{\pi}{4}+3\ln 3-2\ln 2-1=\ln \dfrac{27\sqrt{2}}{8}-\dfrac{\pi}{4}$.
    }
    \end{vd}
    \textbf{Bài tập áp dụng}
    \begin{bt}%[Nguyện Ngô]%[DCHT]%[2D3K2-3]
    Tính các tích phân sau:
    \begin{enumerate}
    \item $I=\displaystyle\int\limits_1^{2} x\ln x \mathrm{\, d}x$.
        \dapso{$I=2\ln 2-\dfrac {3}{4}$}
    \loigiai{
    Chọn $\heva{&u=\ln x \Rightarrow \mathrm{\, d}u = \dfrac {1}{x}\mathrm{\, d}x \\ &\mathrm{\, d}v =x\mathrm{\, d}x \Rightarrow v=\dfrac {x^2}{2}.}$\\
    Khi đó $I= \dfrac {x^2}{2}\ln x \Big|_1^2 -\dfrac {1}{2} \displaystyle\int\limits_1^{2} x \mathrm{\, d}x = 2\ln 2-\dfrac {x^2}{4}\Big|_1^2 =2\ln 2-\dfrac {3}{4}$.
    }
    \item $I=\displaystyle\int\limits_1^{2} (2x-1)\ln x \mathrm{\, d}x$.
        \dapso{$I=2\ln 2-\dfrac {1}{2}$}
    \loigiai{
    Chọn $\heva{&u=\ln x \Rightarrow \mathrm{\, d}u = \dfrac {1}{x}\mathrm{\, d}x \\ &\mathrm{\, d}v =(2x-1)\mathrm{\, d}x \Rightarrow v=x^2-x.}$\\
    Khi đó $I= (x^2-x)\ln x \Big|_1^2 -\displaystyle\int\limits_1^{2} (x-1) \mathrm{\, d}x = 2\ln 2-\left(\dfrac {x^2}{2}-x\right)\Big|_1^2 =2\ln 2-\dfrac {1}{2}$.
    }
    \item $I=\displaystyle\int\limits_1^{\mathrm{e}} (1+x)\ln x \mathrm{\, d}x$.
        \dapso{$I=\dfrac{\mathrm{e}^2}{4}+\dfrac {5}{4}$}
    \loigiai{
    Chọn $\heva{&u=\ln x \Rightarrow \mathrm{\, d}u = \dfrac {1}{x}\mathrm{\, d}x \\ &\mathrm{\, d}v =(1+x)\mathrm{\, d}x \Rightarrow v=\dfrac {x^2}{2}+x.}$\\
    Khi đó $I= \left(\dfrac{x^2}{2}+x\right)\ln x \Big|_1^{\mathrm{e}} - \displaystyle\int\limits_1^{\mathrm{e}} \left(\dfrac {x}{2}+1\right) \mathrm{\, d}x =
    \dfrac{\mathrm{e}^2}{2}+\mathrm{e}-\left(\dfrac {x^2}{4}+x\right)\Big|_1^{\mathrm{e}}=\dfrac {\mathrm{e}^2}{2}+\mathrm{e}-
    \dfrac{\mathrm{e}^2}{4}-\mathrm{e}+\dfrac {5}{4}=\dfrac{\mathrm{e}^2}{4}+\dfrac {5}{4}$.
    }
    \item $I=\displaystyle\int\limits_1^{\mathrm{e}} (x+2)\ln x \mathrm{\, d}x$.
        \dapso{$I=\dfrac{\mathrm{e}^2}{4}+\dfrac {9}{4}$}
    \loigiai{
    Chọn $\heva{&u=\ln x \Rightarrow \mathrm{\, d}u = \dfrac {1}{x}\mathrm{\, d}x \\ &\mathrm{\, d}v =(x+2)\mathrm{\, d}x \Rightarrow v=\dfrac {x^2}{2}+2x.}$\\
    Khi đó $I= \left(\dfrac{x^2}{2}+2x\right)\ln x \Big|_1^{\mathrm{e}} - \displaystyle\int\limits_1^{\mathrm{e}} \left(\dfrac {x}{2}+2\right) \mathrm{\, d}x
    = \dfrac{\mathrm{e}^2}{2}+2\mathrm{e}-\left(\dfrac{x^2}{4}+2x\right)\Big|_1^{\mathrm{e}}\\
    =\dfrac{\mathrm{e}^2}{2}+2\mathrm{e}-\dfrac {\mathrm{e}^2}{4}-2\mathrm{e}+\dfrac {9}{4}=\dfrac{\mathrm{e}^2}{4}+\dfrac {9}{4}$.
    }
    \item $I=\displaystyle\int\limits_1^{\mathrm{e}} x(\ln x+1) \mathrm{\, d}x$.
        \dapso{$I=\dfrac {3\mathrm{e}^2}{4} -\dfrac {1}{4}$}
    \loigiai{
    $I=\displaystyle\int\limits_1^{\mathrm{e}} x(\ln x+1) \mathrm{\, d}x =\displaystyle\int\limits_1^{\mathrm{e}} x\ln x \mathrm{\, d}x
    +\displaystyle\int\limits_1^{\mathrm{e}} x \mathrm{\, d}x=I_1+I_2$.\\
    Tính $I_1$. Chọn $\heva{&u=\ln x \Rightarrow \mathrm{\, d}u = \dfrac {1}{x}\mathrm{\, d}x \\ &dv =x\mathrm{\, d}x \Rightarrow v=\dfrac {x^2}{2}.}$\\
    Khi đó $I_1= \dfrac {x^2}{2}\ln x \Big|_1^e- \dfrac {1}{2} \displaystyle\int\limits_1^{e} x\mathrm{\, d}x = \dfrac {\mathrm{e}^2}{2} - \dfrac {1}{4} x^2 \Big|_1^e= \dfrac {\mathrm{e}^2}{2} -\dfrac {\mathrm{e}^2}{4}+\dfrac {1} {4}= \dfrac {\mathrm{e}^2}{4} +\dfrac {1}{4}$.\\
    Tính $I_2=\dfrac {x^2}{2} \Big|_1^e=\dfrac {\mathrm{e}^2}{2}-\dfrac {1}{2}$.\\
    Vậy $I=I_1+I_2=\dfrac {3\mathrm{e}^2}{4} -\dfrac {1}{4}$.
    }
    \item $I=\displaystyle\int\limits_1^{2} \dfrac {x^3-2\ln x}{x^2} \mathrm{\, d}x$.
        \dapso{$I=\ln 2+\dfrac {1}{2}$}
    \loigiai{
    $I=\displaystyle\int\limits_1^{2} \dfrac {x^3-2\ln x}{x^2} \mathrm{\, d}x
    = \displaystyle\int\limits_1^{2} x \mathrm{\, d}x -2 \displaystyle\int\limits_1^{2} \dfrac {\ln x}{x^2} \mathrm{\, d}x
    = \dfrac {x^2}{2}\Big|_1^2-2J=\dfrac {3}{2}-2J$.\\
    Tính $J$.
    Chọn $\heva{&u=\ln x \Rightarrow \mathrm{\, d}u = \dfrac {1}{x}\mathrm{\, d}x \\ &\mathrm{\, d}v =x^{-2}\mathrm{\, d}x \Rightarrow v=-\dfrac {1}{x}.}$\\
    Khi đó $J= -\dfrac {1}{x}\ln x \Big|_1^2+ \displaystyle\int\limits_1^2 \dfrac {1}{x^2} \mathrm{\, d}x = -\dfrac {1}{2}\ln 2 - \dfrac {1}{x} \Big|_1^2
    = -\dfrac {1}{2} \ln 2 -\dfrac {1}{2} +1=\ln 2+\dfrac{1}{2}$.\\
    Vậy $I=\ln 2+\dfrac {1}{2}$.
    }
    \item $I=\displaystyle\int\limits_1^{2} \dfrac {\ln (x\mathrm{e}^x)}{(x+2)^2} \mathrm{\, d}x$.
        \dapso{$I=\dfrac{5}{4}\ln 2-\dfrac{1}{2}\ln 3-\dfrac{1}{6}$}
    \loigiai{
    Chọn $\heva{&u=\ln (xe^x) \Rightarrow \mathrm{\, d}u = \dfrac {x+1}{x}\mathrm{\, d}x \\ &\mathrm{\, d}v =(x+2)^{-2}\mathrm{\, d}x
    \Rightarrow v=-\dfrac {1}{x+2}.}$\\
    Khi đó $I= -\dfrac {1}{x+2}\ln (xe^x)\Big|_1^{2} + \displaystyle\int\limits_1^{2} \dfrac {x+1}{x(x+2)}\mathrm{\, d}x
    = -\dfrac {1}{4} \ln (2\mathrm{e}^2)+\dfrac{1}{3}+J$.\\
    Tính $J$. Ta có $J=\displaystyle\int\limits_1^{2} \dfrac {x+1}{x(x+2)}
    = \dfrac {1}{2} \displaystyle\int\limits_1^{2} \left(\dfrac {1}{x}+\dfrac {1}{x+2}\right)\mathrm{\, d}x= \dfrac {1}{2} \ln |x(x+2)|\Big|_1^2=
    \dfrac{1}{2}\left(\ln 8-\ln 3\right)$.\\
    Vậy $I=\dfrac{5}{4}\ln 2-\dfrac{1}{2}\ln 3-\dfrac{1}{6}$.
    }
    \item $I=\displaystyle\int\limits_1^{\mathrm{e}} 2x(1-\ln x) \mathrm{\, d}x$.
        \dapso{$I=\dfrac{\mathrm{e}^2-3}{2}$}
    \loigiai{
    $I=\displaystyle\int\limits_1^{\mathrm{e}} 2x(1-\ln x) \mathrm{\, d}x= \displaystyle\int\limits_1^{\mathrm{e}} 2x \mathrm{\, d}x
    -2\displaystyle\int\limits_1^{\mathrm{e}} x\ln x \mathrm{\, d}x=x^2 \Big|_1^{\mathrm{e}}-2J=\mathrm{e}^2-1-2J$.\\
    Tính  $J$. Chọn $\heva{&u=\ln x \Rightarrow \mathrm{\, d}u = \dfrac {1}{x}\mathrm{\, d}x \\ &\mathrm{\, d}v =x\mathrm{\, d}x \Rightarrow v=\dfrac {x^2}{2}.}$\\
    Khi đó $J= \dfrac {x^2}{2}\ln x \Big|_1^e- \dfrac {1}{2} \displaystyle\int\limits_1^{e} x\mathrm{\, d}x = \dfrac {\mathrm{e}^2}{2} - \dfrac {1}{4} x^2 \Big|_1^e
    = \dfrac{\mathrm{e}^2}{2} -\dfrac{\mathrm{e}^2}{4}+\dfrac{1} {4}= \dfrac {\mathrm{e}^2}{4} +\dfrac {1}{4}$.\\
    Vậy $I=\mathrm{e}^2-1-\dfrac{\mathrm{e}^2}{2}-\dfrac{1}{2}=\dfrac{\mathrm{e}^2}{2}-\dfrac {3}{2}$.
    }
    \item $I=\displaystyle\int\limits_\mathrm{e}^{\mathrm{e}^2} (1+\ln x)x \mathrm{\, d}x$.
        \dapso{$I=\dfrac{5\mathrm{e}^4}{4}-\dfrac{3\mathrm{e}^2}{4}$}
    \loigiai{
    $I=\displaystyle\int\limits_\mathrm{e}^{\mathrm{e}^2} (1+\ln x)x \mathrm{\, d}x
    = \displaystyle\int\limits_\mathrm{e}^{\mathrm{e}^2} x\ln x \mathrm{\, d}x+\displaystyle\int\limits_\mathrm{e}^{\mathrm{e}^2} x \mathrm{\, d}x
    =J+\dfrac{x^2}{2}\Big|_\mathrm{e}^{\mathrm{e}^2} =J+ \dfrac{\mathrm{e}^4}{2}-\dfrac{\mathrm{e}^2}{2}$.\\
    Tính $J$. Chọn $\heva{&u=\ln x \Rightarrow \mathrm{\, d}u = \dfrac {1}{x}\mathrm{\, d}x \\ &\mathrm{\, d}v =x\mathrm{\, d}x \Rightarrow v=\dfrac {x^2}{2}.}$\\
    Khi đó $J= \dfrac {x^2}{2}\ln x \Big|_\mathrm{e}^{\mathrm{e}^2}- \dfrac {1}{2} \displaystyle\int\limits_\mathrm{e}^{\mathrm{e}^2} x\mathrm{\, d}x
     =\mathrm{e}^4- \dfrac{\mathrm{e}^2}{2}- \dfrac{1}{4} x^2 \Big|_\mathrm{e}^{\mathrm{e}^2}= \dfrac {3\mathrm{e}^4}{4}-\dfrac{\mathrm{e}^2}{4}$.\\
     Vậy $I= \dfrac {3\mathrm{e}^4}{4}-\dfrac{\mathrm{e}^2}{4}+\dfrac{\mathrm{e}^4}{2}-\dfrac{\mathrm{e}^2}{2}
     =\dfrac{5\mathrm{e}^4}{4}-\dfrac{3\mathrm{e}^2}{4}$.
    }
    \item $I=\displaystyle\int\limits_1^{3} \dfrac {1+\ln (x+1)}{x^2} \mathrm{\, d}x$.
        \dapso{$I=\ln 3+\dfrac {2}{3}-\dfrac {2}{3}\ln 2$}
    \loigiai{
    Chọn $\heva{&u=\ln (x+1)+1 \Rightarrow \mathrm{\, d}u = \dfrac {1}{x+1}\mathrm{\, d}x \\
    &\mathrm{\, d}v =x^{-2}\mathrm{\, d}x \Rightarrow v=-\dfrac {1}{x}.}$\\
    Khi đó
    \begin{eqnarray*}I&=& -\dfrac {1}{x} (\ln (x+1)+1)\Big|_1^{3} + \displaystyle\int\limits_1^{3} \dfrac {1}{x(x+1)}\mathrm{\, d}x\\
    &=&\dfrac{1}{3}\ln 2+\dfrac{2}{3}+ \displaystyle\int\limits_1^{3}\left( \dfrac{1}{x}-\dfrac{1}{(x+1)}\right)\mathrm{\, d}x\\
    &=&\dfrac{1}{3}\ln 2 +\dfrac{2}{3}+(\ln |x|-\ln |x+1|)\Big|_1^3\\
    &=&\dfrac {1}{3}\ln 2+\dfrac {2}{3}+\ln 3-\ln 4+\ln 2=-\dfrac{2}{3}\ln 2+\ln 3+\dfrac{2}{3}.
    \end{eqnarray*}
    }
    \item $I=\displaystyle\int\limits_2^{3} 2x \ln (x-1) \mathrm{\, d}x$.
        \dapso{$I=8\ln 2- \dfrac{7}{2}$}
    \loigiai{
    Chọn $\heva{&u=\ln (x-1) \Rightarrow \mathrm{\, d}u = \dfrac{1}{x-1}\mathrm{\, d}x \\ &\mathrm{\, d}v =2x\mathrm{\, d}x \Rightarrow v=x^2.}$\\
    Khi đó
    $I=x^2 \ln (x-1)\Big|_2^{3} - \displaystyle\int\limits_2^{3} \dfrac {x^2}{x-1}\mathrm{\, d}x= 9\ln 2-J$.
    Tính $J=\displaystyle\int\limits_2^{3}\left(x+1+\dfrac{1}{x-1}\right)\mathrm{\, d}x
    = \left(\dfrac{x^2}{2}+x+\ln |x-1|\right)\Big|_2^{3}=\dfrac{7}{2}+\ln 2$.\\
    Vậy $I=8\ln 2-\dfrac{7}{2}$.
    }
    \item $I=\displaystyle\int\limits_{-1}^{1} (4x-5)\ln (2x+3) \mathrm{\, d}x$.
        \dapso{$I=16-15 \ln 5$}
    \loigiai{
    Chọn $\heva{&u=\ln (2x+3) \Rightarrow \mathrm{\, d}u = \dfrac {2}{2x+3}\mathrm{\, d}x \\ &\mathrm{\, d}v =(4x-5)\mathrm{\, d}x \Rightarrow v=2x^2-5x.}$\\
    Khi đó
    \begin{eqnarray*}I&=& ( 2x^2-5x)\ln (2x+3)\Big|_{-1}^{1} - 2\displaystyle\int\limits_{-1}^{1} \dfrac {2x^2-5x}{2x+3}\mathrm{\, d}x\\
     &=& -3\ln 5- 2\displaystyle\int\limits_{-1}^{1} \left(x-4 + \dfrac {12}{2x+3}\right)\mathrm{\, d}x
     = -3\ln 5-2\left(\dfrac {x^2}{2}-4x+6\ln |2x+3|\right)\Big|_{-1}^{1}\\
    &=&16-15 \ln 5.
    \end{eqnarray*}
    }
    \item $I=\displaystyle\int\limits_0^{1} x\ln (2+x^2) \mathrm{\, d}x$.
        \dapso{$I=\ln \dfrac {3\sqrt {3}}{2}-\dfrac {1}{2}$}
    \loigiai{
    Chọn
    $\heva{&u=\ln (2+x^2) \Rightarrow \mathrm{\, d}u = \dfrac {2x}{x^2+2}\mathrm{\, d}x \\ &\mathrm{\, d}v =x\mathrm{\, d}x \Rightarrow v=\dfrac {x^2}{2}.} $\\
    Khi đó
    \begin{eqnarray*}I&=& \dfrac {x^2}{2}\ln (x^2+2)\Big|_0^{1} - \displaystyle\int\limits_0^{1} \dfrac {x^3}{x^2+2}\mathrm{\, d}x
    = \dfrac {1}{2} \ln 3- \displaystyle\int\limits_0^{1} \left(x- \dfrac {2x}{x^2+2}\right)\mathrm{\, d}x \\
    &=& \dfrac {1}{2}\ln 3- \left(\dfrac {x^2}{2} -\ln (x^2+2)\right)\Big|_0^{1}
    = \dfrac {1}{2}\ln 3-\dfrac {1}{2}+\ln 3 - \ln 2=\ln \dfrac {3\sqrt {3}}{2}-\dfrac {1}{2}.
    \end{eqnarray*}
    }
    \item $I=\displaystyle\int\limits_0^{1} (x-5)\ln (2x+1) \mathrm{\, d}x$.
        \dapso{$I=5-\dfrac {57}{8}\ln 3$}
    \loigiai{
    Chọn
    $\heva{&u=\ln (2x+1) \Rightarrow \mathrm{\, d}u = \dfrac {2}{2x+1}\mathrm{\, d}x \\ &\mathrm{\, d}v =(x-5)\mathrm{\, d}x \Rightarrow v=\dfrac {x^2}{2}-5x.} $\\
    Khi đó $I= \left(\dfrac {x^2}{2}-5x\right)\ln (2x+1)\Big|_0^{1} - \displaystyle\int\limits_0^{1} \dfrac {x^2-10x}{2x+1}\mathrm{\, d}x=-\dfrac{9}{2}\ln 3-J$.\\
    Tính $J=\displaystyle\int\limits_0^{1} \dfrac {x^2-10x}{2x+1}\mathrm{\, d}x=\displaystyle\int\limits_0^{1} \left(\dfrac {1}{2} x-\dfrac {21}{4} +
    \dfrac {21}{4(2x+1)}\right)\mathrm{\, d}x= \left(\dfrac {x^2}{4} -\dfrac {21}{4} x+\dfrac {21}{8}\ln |2x+1|\right)\Big|_0^{1}
    = -5+\dfrac {21}{8}\ln 3$.\\
    Vậy $I=-\dfrac{9}{2}\ln 3+5-\dfrac {21}{8}\ln 3=5-\dfrac {57}{8}\ln 3$.
    }
    \item $I=\displaystyle\int\limits_0^{\ln 2} \mathrm{e}^x \ln (\mathrm{e}^x+1) \mathrm{\, d}x$.
        \dapso{$I=3\ln 3 -2\ln 2-1$}
    \loigiai{
     Chọn
     $\heva{&u=\ln (\mathrm{e}^x+1) \Rightarrow \mathrm{\, d}u = \dfrac {\mathrm{e}^x}{\mathrm{e}^x+1}\mathrm{\, d}x \\ &\mathrm{\, d}v
     =\mathrm{e}^x \mathrm{\, d}x \Rightarrow v=\mathrm{e}^x.} $\\
    Khi đó
    $I= \mathrm{e}^x \ln (\mathrm{e}^x +1)\Big|_0^{\ln 2} - \displaystyle\int\limits_0^{\ln 2} \dfrac {\mathrm{e}^ {2x}}{\mathrm{e}^x +1}\mathrm{\, d}x
    = 2\ln 3-\ln 2-J.$\\
    Tính $J$. Đặt $t=\mathrm{e}^x\Rightarrow \mathrm{\, d}t=\mathrm{e}^x\mathrm{\, d}x$.\\
    Đổi cận $\heva{&x=0\Rightarrow t=1\\&x=\ln 2\Rightarrow t=2.}$\\
    Khi đó $J=\displaystyle\int\limits_1^{2} \dfrac{t}{t+1}\mathrm{\, d}t=\displaystyle\int\limits_1^{2}\left( 1-\dfrac{1}{t+1}\right)\mathrm{\, d}t
    =\left(t-\ln |t+1|\right)\Big|_1^2=1-\ln 3+\ln 2$.\\
    Vậy $I=3\ln 3-2\ln 2-1$.
    }
    \item $I=\displaystyle\int\limits_0^{1} \dfrac {\ln (x+1)}{(x+2)^2} \mathrm{\, d}x$.
        \dapso{$I=\dfrac{5}{3}\ln 2-\ln 3$}
    \loigiai{
    Chọn $\heva{&u=\ln (x+1) \Rightarrow \mathrm{\, d}u = \dfrac {1}{x+1}\mathrm{\, d}x \\ &\mathrm{\, d}v =(x+2)^{-2} \mathrm{\, d}x
    \Rightarrow v=-\dfrac {1}{x+2}.} $\\
    Khi đó 
    \begin{eqnarray*}I&=& -\dfrac {1}{x+2} \ln (x +1)\Big|_0^{1} + \displaystyle\int\limits_0^{1} \dfrac {1}{(x+1)(x+2)}\mathrm{\, d}x\\
     &=& -\dfrac {1}{3}\ln 2+\displaystyle\int\limits_0^{1}\left(\dfrac{1}{x+1}-\dfrac{1}{x+2}\right)\mathrm{\, d}x\\
     &=& -\dfrac {1}{3}\ln 2+\left(\ln |x+1|-\ln |x+2|\right)\Big|_0^{1}=\dfrac{5}{3}\ln 2-\ln 3.
     \end{eqnarray*}
    }
    \item $I=\displaystyle\int\limits_2^{3} \ln [2+x(x^2-3)] \mathrm{\, d}x$.
        \dapso{$I=-4\ln 2+5\ln 5-3$}
    \loigiai{
    Chọn $\heva{&u=\ln (x^3-3x+2)  \Rightarrow \mathrm{\, d}u = \dfrac {3x^2-3}{x^3-3x+2}\mathrm{\, d}x= \dfrac {3(x+1)}{(x-1)(x+2)}\mathrm{\, d}x\\
     &\mathrm{\, d}v = \mathrm{\, d}x \Rightarrow v=x.}$\\
    Khi đó
    \begin{eqnarray*}I&=& x \ln (x^3-3x+2)\Big|_2^{3} -3 \displaystyle\int\limits_2^{3} \dfrac {x^2+x}{(x-1)(x+2)}\mathrm{\, d}x\\
    &=& 3\ln 20 -2\ln 4 -3  \displaystyle\int\limits_2^{3} \left(1+\dfrac {2}{3}\left(\dfrac {1}{x-1}-\dfrac {1}{x+2}\right)\right)\mathrm{\, d}x\\
    &=&3\ln 20 -2\ln 4 -\left(3x+2\ln \dfrac {|x-1|}{|x+2|}\right)\Big|_2^{3}\\
    &=&3\ln 20 -2\ln 4 -3-2\ln \dfrac{8}{5}
    =-4\ln 2+5\ln 5-3.
    \end{eqnarray*}
    }
    \item $I=\displaystyle\int\limits_0^{1} \dfrac {\ln (4x^2+8x+3)}{(x+1)^3} \mathrm{\, d}x$.
        \dapso{$I=-\dfrac{1}{8}\ln 15+\dfrac{1}{2}\ln 3+\ln \dfrac{25}{16}$}
    \loigiai{
    Chọn $\heva{&u=\ln (4x^2+8x+3)  \Rightarrow \mathrm{\, d}u = \dfrac {8x+8}{4x^2+8x+3}\mathrm{\, d}x \\
    &\mathrm{\, d}v = \dfrac{1}{(x+1)^3}\mathrm{\, d}x \Rightarrow v=-\dfrac{1}{2(x+1)^2}.}$\\
    Khi đó
    \begin{eqnarray*}I&=& -\dfrac{1}{2(x+1)^2}\ln (4x^2+8x+3)\Big|_0^{1} + \displaystyle\int\limits_0^{1} \dfrac {8x+8}{2(4x^2+8x+3)(x+1)^2}\mathrm{\, d}x\\
    &=& -\dfrac{1}{8}\ln 15+\dfrac{1}{2}\ln 3+\displaystyle\int\limits_0^{1} \dfrac{4}{(2x+1)(2x+3)(x+1)}\mathrm{\, d}x\\
    &=& -\dfrac{1}{8}\ln 15+\dfrac{1}{2}\ln 3+4\displaystyle\int\limits_0^{1}\left(\dfrac{1}{2x+1}+\dfrac{1}{2x+3}-\dfrac{1}{x+1}\right)\mathrm{\, d}x\\
    &=&-\dfrac{1}{8}\ln 15+\dfrac{1}{2}\ln 3+\left(2\ln |2x+1|+2\ln |2x+3|-4\ln |x+1|\right)\Big|_0^1\\
    &=&-\dfrac{1}{8}\ln 15+\dfrac{1}{2}\ln 3+\ln \dfrac{25}{16}.
    \end{eqnarray*}
    }
    \item $I=\displaystyle\int\limits_{\tfrac {\pi}{4}}^{\tfrac{\pi}{2}} \dfrac{\log (3\sin x+\cos x)}{\sin^2 x} \mathrm{\, d}x$.
        \dapso{$I=\log\dfrac{128\sqrt{2}}{27}-\dfrac{\pi}{4\ln 10}$}
    \loigiai{
    Với mọi $x\in\left[\dfrac{\pi}{4};\dfrac{\pi}{2}\right]$, ta có
    \[\dfrac{\log (3\sin x+\cos x)}{\sin^2 x}=\dfrac{\log[\sin x(3+\cot x)]}{\sin^2x}=\dfrac{\log\sin x}{\sin^2x}+\dfrac{\log(3+\cot x)}{\sin^2x}.\]
    Tính $I_1=\displaystyle\int\limits_{\tfrac {\pi}{4}}^{\tfrac{\pi}{2}} \dfrac{\log\sin x}{\sin^2 x} \mathrm{\, d}x$:\\
    Chọn $\heva{&u=\log\sin x  \Rightarrow \mathrm{\, d}u = \dfrac {1}{\sin x\cdot\ln 10}\cdot\cos x\mathrm{\, d}x \\
    &\mathrm{\, d}v = \dfrac{1}{\sin^2x}\mathrm{\, d}x \Rightarrow v=-\cot x.}$\\
    Khi đó
    \begin{eqnarray*}I_1&=& -\cot x\log\sin x\Big|_{\tfrac{\pi}{4}}^{\tfrac{\pi}{2}}
     + \displaystyle\int\limits_{\tfrac{\pi}{4}}^{\tfrac{\pi}{2}} \dfrac{\cot^2x}{\ln 10}\mathrm{\, d}x\\
    &=& \log\dfrac{\sqrt{2}}{2}+\dfrac{1}{\ln 10}\displaystyle\int\limits_{\tfrac{\pi}{4}}^{\tfrac{\pi}{2}}\left(\dfrac{1}{\sin^2x}-1\right)\mathrm{\, d}x\\
    &=&  \log\dfrac{\sqrt{2}}{2}+\dfrac{1}{\ln 10}\left(-\cot x-x\right)\Big|_{\tfrac{\pi}{4}}^{\tfrac{\pi}{2}}\\
    &=&\log\dfrac{\sqrt{2}}{2}+\dfrac{1}{\ln 10}\left(-\dfrac{\pi}{4}+1\right).
    \end{eqnarray*}
    Tính $I_2=\displaystyle\int\limits_{\tfrac {\pi}{4}}^{\tfrac{\pi}{2}} \dfrac{\log(3+\cot x)}{\sin^2 x} \mathrm{\, d}x$:\\
    Chọn $\heva{&u=\log(3+\cot x)  \Rightarrow \mathrm{\, d}u = \dfrac {1}{(3+\cot x)\ln 10}\mathrm{\, d}(\cot x) \\
    &\mathrm{\, d}v = \dfrac{1}{\sin^2x}\mathrm{\, d}x \Rightarrow v=-\cot x.}$\\
    Khi đó
    \begin{eqnarray*}I_2&=& -\cot x\log(3+\cot x)\Big|_{\tfrac{\pi}{4}}^{\tfrac{\pi}{2}}
     + \dfrac{1}{\ln 10}\displaystyle\int\limits_{\tfrac{\pi}{4}}^{\tfrac{\pi}{2}} \dfrac{\cot x}{3+\cot x}\mathrm{\, d}(\cot x)\\
    &=& \log4+\dfrac{1}{\ln 10}\displaystyle\int\limits_{\tfrac{\pi}{4}}^{\tfrac{\pi}{2}}\left(1-\dfrac{3}{3+\cot x}\right)\mathrm{\, d}(\cot x)\\
    &=&  \log4+\dfrac{1}{\ln 10}\left(\cot x-3\ln |3+\cot x|\right)\Big|_{\tfrac{\pi}{4}}^{\tfrac{\pi}{2}}\\
    &=&\log4+\dfrac{1}{\ln 10}\left(3\ln \dfrac{4}{3}-1\right).
    \end{eqnarray*}
    Vậy $I=I_1+I_2=\log\dfrac{\sqrt{2}}{2}+\dfrac{1}{\ln 10}\left(-\dfrac{\pi}{4}+1\right)+\log4+\dfrac{1}{\ln 10}\left(3\ln \dfrac{4}{3}-1\right)
    =\log\dfrac{128\sqrt{2}}{27}-\dfrac{\pi}{4\ln 10}$.
    }
    \end{enumerate}
    \end{bt}
    \subsubsection{Ví dụ - Bài tập áp dụng}
    \begin{vd}%[Hồ Sỹ Trường]%[DCHT]%[2D3K2-3]
    Tính $I=\displaystyle\int\limits_{0}^{\tfrac{\pi}{2}} \left(2x-1\right)\cos2x \mathrm{\,d}x$.   \dapso{$I=-1$}
    \loigiai{
        Đặt $\heva{& u=2x-1 \\ & \mathrm{d}v=\cos 2x\mathrm{\,d}x}\Rightarrow\heva{& \mathrm{\,d}u=2\mathrm{\,d}x \\ & v=\dfrac{\sin2x}{2}.}$\\
        Do đó $I=\dfrac{\left(2x-1\right)\sin2x}{2}\Bigg|_0^{\tfrac{\pi}{2}}-\displaystyle\int\limits_{0}^{\tfrac{\pi}{2}} \sin2x \mathrm{\,d}x=\dfrac{\cos2x}{2}\Bigg|_0^{\tfrac{\pi}{2}}=-1$.	
    }	
    \end{vd}
    
    \begin{vd}%[Hồ Sỹ Trường]%[DCHT]%[2D3K2-3]
    Tính	$I=\displaystyle\int\limits_{0}^{\tfrac{\pi}{2}} \mathrm{e}^{2x}\left(1+x\mathrm{e}^{-2x}\cos x\right) \mathrm{\,d}x$. \dapso{$I=\dfrac{\mathrm{e}^{\pi}-3}{2}+\dfrac{\pi}{2}$}
        \loigiai{
            Có $I=\displaystyle\int\limits_{0}^{\tfrac{\pi}{2}} \mathrm{e}^{2x}\left(1+x\mathrm{e}^{-2x}\cos x\right) \mathrm{\,d}x=\int\limits_{0}^{\tfrac{\pi}{2}} \mathrm{e}^{2x} \mathrm{\,d}x+\int\limits_{0}^{\tfrac{\pi}{2}} x\cos x \mathrm{\,d}x=\dfrac{\mathrm{e}^{2x}}{2}\Bigg|_0^{\tfrac{\pi}{2}}+J=\dfrac{\mathrm{e}^{\pi}-1}{2}+J$.\\
            Đặt $\heva{& u=x \\ & \mathrm{\,d}v=\cos x\mathrm{\,d}x}\Rightarrow \heva{& \mathrm{\,d}u=\mathrm{\,d}x \\ & v=\sin x.}$\\
            Do đó $J=x\sin x\Bigg|_0^{\tfrac{\pi}{2}}-\displaystyle\int\limits_{0}^{\tfrac{\pi}{2}} \sin x \mathrm{\,d}x=\dfrac{\pi}{2}+\cos x\Bigg|_0^{\tfrac{\pi}{2}}=\dfrac{\pi}{2}-1$.\\
            Vậy $I=\dfrac{\mathrm{e}^{\pi}-1}{2}+\dfrac{\pi}{2}-1=\dfrac{\mathrm{e}^{\pi}-3}{2}+\dfrac{\pi}{2}$.		
        }
    \end{vd}
    
    \begin{vd}%[Hồ Sỹ Trường]%[DCHT]%[2D3K2-3]
    Tính	$I=\displaystyle\int\limits_{1}^{2} \left(2x^3+\ln x\right)x \mathrm{\,d}x$. \dapso{$I=2\ln 2+\dfrac{233}{20}$}
        \loigiai{
            Có $I=\displaystyle\int\limits_{1}^{2} 2x^4 \mathrm{\,d}x+\int\limits_{1}^{2} x\ln x \mathrm{\,d}x=I_1+I_2$.\\
            + $I_1=\dfrac{2x^5}{5}\Bigg|_1^2=\dfrac{62}{5}$.\\
            + $I_2=\displaystyle\int\limits_{1}^{2} x\ln x \mathrm{\,d}x$.\\
            Đặt $\heva{& u=\ln x \\ & \mathrm{\,d}v=x\mathrm{\,d}x}\Rightarrow\heva{& \mathrm{\,d}u=\dfrac{1}{x}\mathrm{\,d}x \\ & v=\dfrac{x^2}{2}.}$\\
            Suy ra $I_2=\dfrac{x^2\ln x}{2}\Bigg|_1^2-\displaystyle\int\limits_{1}^{2} \dfrac{x}{2} \mathrm{\,d}x=2\ln 2-\dfrac{x^2}{4}\Bigg|_1^2=2\ln 2-\dfrac{3}{4}$.\\
            Vậy $I=2\ln 2+\dfrac{233}{20}$.	
        }
    \end{vd}
    
    \begin{vd}%[Hồ Sỹ Trường]%[DCHT]%[2D3G2-3]
    Tính	$I=\displaystyle\int\limits_{\tfrac{\pi}{4}}^{\tfrac{\pi}{2}} \dfrac{\log_2\left(3\sin x+\cos x\right)}{\sin^2x} \mathrm{\,d}x$.
     \dapso{$I=\dfrac{1}{\ln 2}\left(\ln 2\sqrt{2}+3\ln \dfrac{4}{3}-\dfrac{\pi}{4}\right)$}
        \loigiai{
            Có $I=\dfrac{1}{\ln 2}\displaystyle\int\limits_{\tfrac{\pi}{4}}^{\tfrac{\pi}{2}} \dfrac{\ln \left(3\sin x+\cos x\right)}{\sin^2x} \mathrm{\,d}x$.\\
            Đặt $\heva{& u=\ln \left(3\sin x+\cos x\right) \\ & \mathrm{\,d}v=\dfrac{1}{\sin^2x}\mathrm{\,d}x}\Rightarrow\heva{& \mathrm{\,d}u=\dfrac{3\cos x-\sin x}{3\sin x+\cos x}\mathrm{\,d}x \\ & v=-\cot x.}$\\
            Suy ra $\ln 2\cdot I=-\cot x\cdot\ln \left(3\sin x+\cos x\right)\Bigg|_{\tfrac{\pi}{4}}^{\tfrac{\pi}{2}}+J=\ln 2\sqrt{2}+J$.\\
            Với $J=\displaystyle\int\limits_{\tfrac{\pi}{4}}^{\tfrac{\pi}{2}} \dfrac{3\cos^2x-\sin x\cos x}{3\sin^2x+\sin x\cos x} \mathrm{\,d}x=\int\limits_{\tfrac{\pi}{4}}^{\tfrac{\pi}{2}} \dfrac{3}{3\sin^2x+\sin x\cos x} \mathrm{\,d}x-\int\limits_{\tfrac{\pi}{4}}^{\tfrac{\pi}{2}} \mathrm{\,d}x=\int\limits_{\tfrac{\pi}{4}}^{\tfrac{\pi}{2}} \dfrac{3}{\left(3+\cot x\right)\sin^2x} \mathrm{\,d}x-\dfrac{\pi}{4}$\\
            $=-3\ln \left|3+\cot x\right|\Bigg|_{\tfrac{\pi}{4}}^{\tfrac{\pi}{2}}-\dfrac{\pi}{4}=3\ln \dfrac{4}{3}-\dfrac{\pi}{4}$.\\
            Vậy $I=\dfrac{1}{\ln 2}\left(\ln 2\sqrt{2}+3\ln \dfrac{4}{3}-\dfrac{\pi}{4}\right)$.	
        }
    \end{vd}
    \textbf{Bài tập áp dụng}
    \begin{bt}%[Nguyện Ngô]%[DCHT]%[2D3K2-3]
    Tính các tích phân  sau:
    \begin{enumerate}
    \item $I=\displaystyle\int\limits_0^{\tfrac{\pi}{2}} x\sin x \mathrm{\, d}x$.
        \dapso{$I=1$}
    \loigiai
    {
    Chọn $\heva{&u=x \Rightarrow \mathrm{\, d}u =\mathrm{\, d}x \\ &\mathrm{\, d}v = \sin x \mathrm{\, d}x \Rightarrow v=-\cos x.}$\\
    Khi đó $I=-x\cos x\Big|_0^{\tfrac{\pi}{2}} + \displaystyle\int\limits_0^{\tfrac{\pi}{2}}\cos x \mathrm{\, d}x
    = \sin x \Big|_0^{\tfrac{\pi}{2}} = 1$.
    }
    \item $I=\displaystyle\int\limits_0^{\tfrac{\pi}{4}} 2x\cos x \mathrm{\, d}x$.
        \dapso{$I=\dfrac{\sqrt{2}}{4}\pi+\sqrt{2}-2$}
    \loigiai
    {
    Chọn $\heva{&u=2x \Rightarrow \mathrm{\, d}u =2\mathrm{\, d}x \\ &\mathrm{\, d}v = \cos x \mathrm{\, d}x \Rightarrow v=\sin x.}$\\
    Khi đó $I=2x\sin x\Big|_0^{\tfrac{\pi}{4}} - \displaystyle\int\limits_0^{\tfrac{\pi}{4}}2\sin x \mathrm{\, d}x
    =\dfrac{\sqrt{2}}{4}\pi+2\cos x \Big|_0^{\tfrac{\pi}{4}} = \dfrac{\sqrt{2}}{4}\pi+\sqrt{2}-2$.
    }
    \item $I=\displaystyle\int\limits_0^{\tfrac{\pi}{4}} (x+1)\sin 2x \mathrm{\, d}x$.
        \dapso{$I=\dfrac{3}{4}$}
    \loigiai
    {
    Chọn $\heva{&u=x+1 \Rightarrow \mathrm{\, d}u =\mathrm{\, d}x \\ &\mathrm{\, d}v = \sin 2x \mathrm{\, d}x \Rightarrow v=-\dfrac{1}{2}\cos2x.}$\\
    Khi đó $I=-\dfrac{1}{2}(x+1)\cos2x\Big|_0^{\tfrac{\pi}{4}} + \displaystyle\int\limits_0^{\tfrac{\pi}{4}}\dfrac{1}{2}\cos2x \mathrm{\, d}x
    =\dfrac{1}{2}+\dfrac{1}{4}\sin2x\Big|_0^{\tfrac{\pi}{4}} = \dfrac{3}{4}$.
    }
    \item $I=\displaystyle\int\limits_0^{\tfrac{\pi}{2}} (x-2)\cos x \mathrm{\, d}x$.
        \dapso{$I=\dfrac{\pi}{2}-3$}
    \loigiai
    {
    Chọn $\heva{&u=x-2 \Rightarrow \mathrm{\, d}u =\mathrm{\, d}x \\ &\mathrm{\, d}v = \cos x \mathrm{\, d}x \Rightarrow v=\sin x.}$\\
    Khi đó $I=(x-2)\sin x\Big|_0^{\tfrac{\pi}{2}} - \displaystyle\int\limits_0^{\tfrac{\pi}{2}}\sin x \mathrm{\, d}x
    =\dfrac{\pi}{2}-2+\cos x\Big|_0^{\tfrac{\pi}{2}} = \dfrac{\pi}{2}-3$.
    }
    \item $I=\displaystyle\int\limits_0^{\tfrac{\pi}{2}} (x+1)\cos x \mathrm{\, d}x$.
        \dapso{$I=\dfrac{\pi}{2}$}
    \loigiai
    {
    Chọn $\heva{&u=x-2 \Rightarrow \mathrm{\, d}u =\mathrm{\, d}x \\ &\mathrm{\, d}v = \cos x \mathrm{\, d}x \Rightarrow v=\sin x.}$\\
    Khi đó $I=(x+1)\sin x\Big|_0^{\tfrac{\pi}{2}} - \displaystyle\int\limits_0^{\tfrac{\pi}{2}}\sin x \mathrm{\, d}x
    =\dfrac{\pi}{2}+1+\cos x\Big|_0^{\tfrac{\pi}{2}} = \dfrac{\pi}{2}$.
    }
    \item $I=\displaystyle\int\limits_0^{\tfrac{\pi}{2}} (x-1)\sin x \mathrm{\, d}x$.
        \dapso{$I=0$}
    \loigiai
    {
    Chọn $\heva{&u=x-1 \Rightarrow \mathrm{\, d}u =\mathrm{\, d}x \\ &\mathrm{\, d}v = \sin x \mathrm{\, d}x \Rightarrow v=-\cos x.}$\\
    Khi đó $I=-(x-1)\cos x\Big|_0^{\tfrac{\pi}{2}} + \displaystyle\int\limits_0^{\tfrac{\pi}{2}}\cos x \mathrm{\, d}x
    =-1+\sin x\Big|_0^{\tfrac{\pi}{2}} = 0$.
    }
    \item $I=\displaystyle\int\limits_0^{\tfrac{\pi}{2}} (2x+1)\sin x \mathrm{\, d}x$.
        \dapso{$I=3$}
    \loigiai{
    Chọn $\heva{&u=2x+1 \Rightarrow \mathrm{\, d}u =2\mathrm{\, d}x \\ &\mathrm{\, d}v = \sin x \mathrm{\, d}x \Rightarrow v=-\cos x.}$\\
    Khi đó $I=-(2x+1)\cos x\Big|_0^{\tfrac{\pi}{2}} + \displaystyle\int\limits_0^{\tfrac{\pi}{2}} 2\cos x \mathrm{\, d}x
    = 1+2\sin x \Big|_0^{\tfrac{\pi}{2}} = 3$.
    }
    \item $I=\displaystyle\int\limits_0^{\tfrac{\pi}{2}} x\cos2x \mathrm{\, d}x$.
        \dapso{$I=-\dfrac{1}{2}$}
    \loigiai
    {Chọn $\heva{&u=x \Rightarrow \mathrm{\, d}u =\mathrm{\, d}x \\ &\mathrm{\, d}v = \cos2x \mathrm{\, d}x \Rightarrow v=\dfrac{1}{2}\sin 2x.}$\\
    Khi đó $I=\dfrac{1}{2}x\sin2x\Big|_0^{\tfrac{\pi}{2}} - \displaystyle\int\limits_0^{\tfrac{\pi}{2}} \dfrac{1}{2}\sin 2x \mathrm{\, d}x
    = \dfrac{1}{4}\cos 2x \Big|_0^{\tfrac{\pi}{2}} = -\dfrac{1}{2}$.
    }
    \end{enumerate}
    \end{bt}
    
    \begin{bt}%[Hồ Sỹ Trường]%[DCHT]%[2D3K2-3]
        Tính các tích phân sau:
        \begin{enumerate}
            \item $I=\displaystyle\int\limits_{0}^{\tfrac{\pi}{4}} \left(3-2x\right) \sin2x\mathrm{\,d}x$. \dapso{$I=1$}
            \loigiai{
        Đặt $\heva{& u=3-2x \\ & \mathrm{\,d}v=\sin2x\mathrm{\,d}x}\Rightarrow\heva{& \mathrm{\,d}u=-2\mathrm{\,d}x \\ & v=-\dfrac{\cos2x}{2}.}$\\
        Do đó $I=\left(-\dfrac{\left(3-2x\right)\cos2x}{2}\right)\Bigg|_0^{\tfrac{\pi}{4}}-\displaystyle\int\limits_{0}^{\tfrac{\pi}{4}} \cos2x \mathrm{\,d}x=\dfrac{3}{2}-\dfrac{\sin2x}{2}\Bigg|_0^{\tfrac{\pi}{4}}=1$.
        }
            \item $I=\displaystyle\int\limits_{0}^{\tfrac{\pi}{2}} 3x\cos x \mathrm{\,d}x$. \dapso{$I=\dfrac{3\pi}{2}-3$}
            \loigiai{
        Đặt $\heva{& u=3x \\ & \mathrm{\,d}v=\cos x\mathrm{\,d}x}\Rightarrow \heva{& \mathrm{\,d}u=\mathrm{\,d}x \\ & v=\sin x.}$\\
        Do đó $I=3x\sin x\Bigg|_0^{\tfrac{\pi}{2}}-\displaystyle\int\limits_{0}^{\tfrac{\pi}{2}} 3\sin x \mathrm{\,d}x=\dfrac{3\pi}{2}+3\cos x\Bigg|_0^{\tfrac{\pi}{2}}=\dfrac{3\pi}{2}-3$.	
        }
            \item $I=\displaystyle\int\limits_{0}^{\tfrac{\pi}{2}} x\sin^2x \mathrm{\,d}x$.  \dapso{$I=\dfrac{\pi^2}{16}+\dfrac{1}{4}$}
            \loigiai{
        Có $I=\displaystyle\int\limits_{0}^{\tfrac{\pi}{2}} \dfrac{x-x\cos 2x}{2} \mathrm{\,d}x=\int\limits_{0}^{\tfrac{\pi}{2}} \dfrac{x}{2} \mathrm{\,d}x-\int\limits_{0}^{\tfrac{\pi}{2}} \dfrac{x\cos 2x}{2} \mathrm{\,d}x=\dfrac{1}{2}I_1-\dfrac{1}{2}I_2$.\\
        + $ I_1=\displaystyle\int\limits_{0}^{\tfrac{\pi}{2}} x \mathrm{\,d}x=\dfrac{x^2}{2}\Bigg|_0^{\tfrac{\pi}{2}}=\dfrac{\pi^2}{8}$.\\
        + $I_2=\displaystyle\int\limits_{0}^{\tfrac{\pi}{2}} x\cos2x \mathrm{\,d}x$.\\
        Đặt $\heva{& u=x \\ & \mathrm{\,d}v=\cos2x\mathrm{\,d}x}\Rightarrow\heva{& \mathrm{\,d}u=\mathrm{\,d}x \\ & v=\dfrac{\sin2x}{2}.}$\\
        Do đó $I_2=\dfrac{x\sin2x}{2}\Bigg|_0^{\tfrac{\pi}{2}}-\displaystyle\int\limits_{0}^{\tfrac{\pi}{2}} \dfrac{\sin2x}{2}\mathrm{\,d}x=\dfrac{\cos2x}{4}\Bigg|_0^{\tfrac{\pi}{2}}=-\dfrac{1}{2}$.\\
        Vậy $I=\dfrac{\pi^2}{16}+\dfrac{1}{4}$.
        }
            \item $I=\displaystyle\int\limits_{0}^{\tfrac{\pi}{2}} x\cos^2x \mathrm{\,d}x$. \dapso{$I=\dfrac{\pi^2}{16}-\dfrac{1}{4}$}
            \loigiai{
        Có $I=\displaystyle\int\limits_{0}^{\tfrac{\pi}{2}} x\cos^2x \mathrm{\,d}x=\int\limits_{0}^{\tfrac{\pi}{2}} \dfrac{x+x\cos2x}{2} \mathrm{\,d}x=\int\limits_{0}^{\tfrac{\pi}{2}} \dfrac{x}{2} \mathrm{\,d}x+\int\limits_{0}^{\tfrac{\pi}{2}} \dfrac{x\cos2x}{2} \mathrm{\,d}x=\dfrac{1}{2}I_1+\dfrac{1}{2}I_2$.\\	
        + $I_1=\displaystyle\int\limits_{0}^{\tfrac{\pi}{2}} x \mathrm{\,d}x=\dfrac{x^2}{2}\Bigg|_0^{\tfrac{\pi}{2}}=\dfrac{\pi^2}{8}$.\\
        + $I_2=\displaystyle\int\limits_{0}^{\tfrac{\pi}{2}} x\cos2x \mathrm{\,d}x$.\\
        Đặt $\heva{& u=x \\ & \mathrm{\,d}v=\cos2x\mathrm{\,d}x}\Rightarrow\heva{& \mathrm{\,d}u=\mathrm{\,d}x \\ & v=\dfrac{\sin2x}{2}.}$\\
        Do đó $I_2=\dfrac{x\sin2x}{2}\Bigg|_0^{\tfrac{\pi}{2}}-\displaystyle\int\limits_{0}^{\tfrac{\pi}{2}} \dfrac{\sin2x}{2}\mathrm{\,d}x=\dfrac{\cos2x}{4}\Bigg|_0^{\tfrac{\pi}{2}}=-\dfrac{1}{2}$.\\
        Vậy $I=\dfrac{\pi^2}{16}-\dfrac{1}{4}$.
        }
            \item $I=\displaystyle\int\limits_{0}^{\tfrac{\pi}{3}} \left(x+2\cos^2x\right)x \mathrm{\,d}x$.
     \dapso{$\dfrac{\pi^3}{81}+\dfrac{\pi^2}{18}+\dfrac{\pi\sqrt{3}}{12}-\dfrac{3}{8}$}
            \loigiai{
        Có $I=\displaystyle\int\limits_{0}^{\tfrac{\pi}{3}} \left(x+2\cos^2x\right)x \mathrm{\,d}x=\int\limits_{0}^{\tfrac{\pi}{3}} \left(x^2+x+x\cos2x\right) \mathrm{\,d}x=\int\limits_{0}^{\tfrac{\pi}{3}} \left(x^2+x\right) \mathrm{\,d}x+\int\limits_{0}^{\tfrac{\pi}{3}} x\cos2x \mathrm{\,d}x=I_1+I_2$.\\
        + $I_1=\displaystyle\int\limits_{0}^{\tfrac{\pi}{3}} \left(x^2+x\right) \mathrm{\,d}x=\left(\dfrac{x^3}{3}+\dfrac{x^2}{2}\right)\Bigg|_0^{\tfrac{\pi}{3}}=\dfrac{\pi^3}{81}+\dfrac{\pi^2}{18}$.\\
        + $I_2=\displaystyle\int\limits_{0}^{\tfrac{\pi}{3}} x\cos2x \mathrm{\,d}x$.\\
        Đặt $\heva{& u=x \\ & \mathrm{\,d}v=\cos2x\mathrm{\,d}x}\Rightarrow\heva{& \mathrm{\,d}u=\mathrm{\,d}x \\ & v=\dfrac{\sin2x}{2}.}$\\
        Do đó $I_2=\dfrac{x\sin2x}{2}\Bigg|_0^{\tfrac{\pi}{3}}-\displaystyle\int\limits_{0}^{\tfrac{\pi}{3}} \dfrac{\sin2x}{2}\mathrm{\,d}x=\dfrac{\pi\sqrt{3}}{12}+\dfrac{\cos2x}{4}\Bigg|_0^{\tfrac{\pi}{3}}=\dfrac{\pi\sqrt{3}}{12}-\dfrac{3}{8}$.\\
        Vậy $I=\dfrac{\pi^3}{81}+\dfrac{\pi^2}{18}+\dfrac{\pi\sqrt{3}}{12}-\dfrac{3}{8}$.
        }
            \item $I=\displaystyle\int\limits_{0}^{\tfrac{\pi}{4}} \dfrac{\ln \left(\cos x\right)}{\cos^2x} \mathrm{\,d}x$.   \dapso{$I=-\dfrac{1}{2}\ln 2+1-\dfrac{\pi}{4}$}
            \loigiai{
        Đặt $\heva{& u=\ln \left(\cos x\right) \\ & \mathrm{\,d}v=\dfrac{1}{\cos^2x}\mathrm{\,d}x}\Rightarrow\heva{& \mathrm{\,d}u=-\dfrac{\sin x}{\cos x}\mathrm{\,d}x=-\tan x\mathrm{\,d}x \\ & v=\tan x.}$\\
        Do đó $I=\tan x\ln \left(\cos x\right)\Bigg|_0^{\tfrac{\pi}{4}}+\displaystyle\int\limits_{0}^{\tfrac{\pi}{4}} \tan^2x \mathrm{\,d}x=\ln \dfrac{\sqrt{2}}{2}+\displaystyle\int\limits_{0}^{\tfrac{\pi}{4}} \left(\dfrac{1}{\cos^2x}-1\right) \mathrm{\,d}x=-\dfrac{1}{2}\ln 2+\left(\tan x-x\right)\Bigg|_0^{\tfrac{\pi}{4}}$\\
        $=-\dfrac{1}{2}\ln 2+1-\dfrac{\pi}{4}$.
        }
            \item $I=\displaystyle\int\limits_{0}^{\tfrac{\pi}{2}} \left(x^2+1\right)\sin x \mathrm{\,d}x$. \dapso{$I=\pi-1$}
            \loigiai{
        Đặt $\heva{& u=x^2+1 \\ & \mathrm{\,d}v=\sin x\mathrm{\,d}x}\Rightarrow\heva{& \mathrm{\,d}u=2x\mathrm{\,d}x \\ & v=-\cos x.}$\\
        Do đó $I=-\left(x^2+1\right)\cos x\Bigg|_0^{\tfrac{\pi}{2}}+2\displaystyle\int\limits_{0}^{\tfrac{\pi}{2}} x\cos x \mathrm{\,d}x=1+2J$.\\
        Đặt $\heva{& u=x \\ & \mathrm{\,d}v=\cos x\mathrm{\,d}x}\Rightarrow\heva{& \mathrm{\,d}u=\mathrm{\,d}x \\ & v=\sin x.}$\\
        Do đó $J=x\sin x\Bigg|_0^{\tfrac{\pi}{2}}-\displaystyle\int\limits_{0}^{\tfrac{\pi}{2}} \sin x \mathrm{\,d}x=\dfrac{\pi}{2}+\cos x\Bigg|_0^{\tfrac{\pi}{2}}=\dfrac{\pi}{2}-1$.\\
        Vậy $I=1+2\left(\dfrac{\pi}{2}-1\right)=\pi-1$.
        }
            \item $I=\displaystyle\int\limits_{0}^{\pi} x\left(x-\sin x\right) \mathrm{\,d}x$. \dapso{$I=\dfrac{\pi^3}{3}-\pi$}
            \loigiai{
        Có $I=\displaystyle\int\limits_{0}^{\pi} x\left(x-\sin x\right) \mathrm{\,d}x=\int\limits_{0}^{\pi} x^2 \mathrm{\,d}x-\int\limits_{0}^{\pi} x\sin x \mathrm{\,d}x=\dfrac{x^3}{3}\Bigg|_0^{\pi}-J=\dfrac{\pi^3}{3}-J$.\\
        Đặt $\heva{& u=x \\ & \mathrm{\,d}v=\sin x\mathrm{\,d}x}\Rightarrow\heva{& \mathrm{\,d}u=\mathrm{\,d}x \\ & v=-\cos x.}$\\
        Do đó $J=-x\cos x\Bigg|_0^{\pi}+\displaystyle\int\limits_{0}^{\pi} \cos x \mathrm{\,d}x=\pi+\sin x\Bigg|_0^{\pi}=\pi$.\\
        Vậy $I=\dfrac{\pi^3}{3}-\pi$.	
        }
            \item $I=\displaystyle\int\limits_{0}^{\tfrac{\pi}{2}} \left(x+\cos 3x\right)x \mathrm{\,d}x$. \dapso{$I=\dfrac{\pi^3}{24}-\dfrac{\pi}{6}-\dfrac{1}{9}$}
            \loigiai{
        Có $I=\displaystyle\int\limits_{0}^{\tfrac{\pi}{2}} \left(x+\cos 3x\right)x \mathrm{\,d}x=\int\limits_{0}^{\tfrac{\pi}{2}} x^2 \mathrm{\,d}x+\int\limits_{0}^{\tfrac{\pi}{2}} x\cos 3x \mathrm{\,d}x=\dfrac{x^3}{3}\Bigg|_0^{\tfrac{\pi}{2}}+J=\dfrac{\pi^3}{24}+J$.\\
        Đặt $\heva{& u=x \\ & \mathrm{\,d}v=\cos 3x\mathrm{\,d}x}\Rightarrow\heva{& \mathrm{\,d}u=\mathrm{\,d}x \\ & v=\dfrac{\sin3x}{3}.}$\\
        Do đó $J=\dfrac{x\sin3x}{3}\Bigg|_0^{\tfrac{\pi}{2}}-\dfrac{1}{3}\displaystyle\int\limits_{0}^{\tfrac{\pi}{2}}\sin3x\mathrm{\,d}x=-\dfrac{\pi}{6}+\dfrac{\cos 3x}{9}\Bigg|_0^{\tfrac{\pi}{2}}=-\dfrac{\pi}{6}-\dfrac{1}{9}$.\\
        Vậy $I=\dfrac{\pi^3}{24}-\dfrac{\pi}{6}-\dfrac{1}{9}$.
        }
            \item $I=\displaystyle\int\limits_{0}^{\tfrac{\pi}{4}} x\left(1+\sin2x\right) \mathrm{\,d}x$. \dapso{$I=\dfrac{\pi^2}{32}+\dfrac{1}{4}$}
            \loigiai{
        Có $I=\displaystyle\int\limits_{0}^{\tfrac{\pi}{4}} x\left(1+\sin2x\right) \mathrm{\,d}x=\int\limits_{0}^{\tfrac{\pi}{4}} x \mathrm{\,d}x+\int\limits_{0}^{\tfrac{\pi}{4}} x\sin2x \mathrm{\,d}x=\dfrac{x^2}{2}\Bigg|_0^{\tfrac{\pi}{4}}+\int\limits_{0}^{\tfrac{\pi}{4}} x\sin2x \mathrm{\,d}x=\dfrac{\pi^2}{32}+J$.\\
        Đặt $\heva{& u=x \\ & \mathrm{\,d}v=\sin2x\mathrm{\,d}x}\Rightarrow\heva{& \mathrm{\,d}u=\mathrm{\,d}x \\ & v=-\dfrac{\cos 2x}{2}.}$\\
        Do đó $J=-\dfrac{x\cos2x}{2}\Bigg|_0^{\tfrac{\pi}{4}}+\dfrac{1}{2}\displaystyle\int\limits_{0}^{\tfrac{\pi}{4}} \cos2x \mathrm{\,d}x=\dfrac{\sin2x}{4}\Bigg|_0^{\tfrac{\pi}{4}}=\dfrac{1}{4}$.\\
        Vậy $I=\dfrac{\pi^2}{32}+\dfrac{1}{4}$.	
        }
        \item $I=\displaystyle\int\limits_{0}^{\tfrac{\pi}{2}} \cos x\left(x-2\sin x\right) \mathrm{\,d}x$. \dapso{$I=\dfrac{\pi}{2}-2$}
            \loigiai{
        Có $I=\displaystyle\int\limits_{0}^{\tfrac{\pi}{2}} \cos x\left(x-2\sin x\right) \mathrm{\,d}x=\int\limits_{0}^{\tfrac{\pi}{2}} x\cos x \mathrm{\,d}x-\int\limits_{0}^{\tfrac{\pi}{2}} \sin2x \mathrm{\,d}x=J+\dfrac{\cos2x}{2}\Bigg|_0^{\tfrac{\pi}{2}}=J-1$.\\
        Đặt $\heva{& u=x \\ & \mathrm{\,d}v=\cos x\mathrm{\,d}x}\Rightarrow\heva{& \mathrm{\,d}u=\mathrm{\,d}x \\ & v=\sin x.}$\\
        Do đó $J=x\sin x\Bigg|_0^{\tfrac{\pi}{2}}-\displaystyle\int\limits_{0}^{\tfrac{\pi}{2}} \sin x \mathrm{\,d}x=\dfrac{\pi}{2}+\cos x\Bigg|_0^{\tfrac{\pi}{2}}=\dfrac{\pi}{2}-1$.\\
        Vậy $I=\dfrac{\pi}{2}-2$.	
        }
            \item $I=\displaystyle\int\limits_{0}^{\tfrac{\pi}{2}} \left(x-\sin x\right)^2 \mathrm{\,d}x$. \dapso{$I=\dfrac{\pi^3}{24}+\dfrac{\pi}{4}-2$}
            \loigiai{
        Có $I=\displaystyle\int\limits_{0}^{\tfrac{\pi}{2}} \left(x-\sin x\right)^2 \mathrm{\,d}x=\int\limits_{0}^{\tfrac{\pi}{2}} \left(x^2+\sin^2x\right) \mathrm{\,d}x-2\int\limits_{0}^{\tfrac{\pi}{2}} x\sin x \mathrm{\,d}x=I_1-2I_2$.\\
        + $I_1=\displaystyle\int\limits_{0}^{\tfrac{\pi}{2}} \left(x^2+\sin^2x\right) \mathrm{\,d}x=\int\limits_{0}^{\tfrac{\pi}{2}} x^2 \mathrm{\,d}x+\int\limits_{0}^{\tfrac{\pi}{2}} \dfrac{1-\cos2x}{2} \mathrm{\,d}x=\int\limits_{0}^{\tfrac{\pi}{2}} x^2 \mathrm{\,d}x+\dfrac{1}{2}\int\limits_{0}^{\tfrac{\pi}{2}} \mathrm{\,d}x-\dfrac{1}{2}\int\limits_{0}^{\tfrac{\pi}{2}} \cos2x \mathrm{\,d}x$\\
        $=\dfrac{x^3}{3}\Bigg|_0^{\tfrac{\pi}{2}}+\dfrac{x}{2}\Bigg|_0^{\tfrac{\pi}{2}}-\dfrac{\sin2x}{4}\Bigg|_0^{\tfrac{\pi}{2}}=\dfrac{\pi^3}{24}+\dfrac{\pi}{4}$.\\
        + $I_2=\displaystyle\int\limits_{0}^{\tfrac{\pi}{2}} x\sin x \mathrm{\,d}x$.\\
        Đặt $\heva{& u=x \\ & \mathrm{\,d}v=\sin x\mathrm{\,d}x}\Rightarrow\heva{& \mathrm{\,d}u=\mathrm{\,d}x \\ & v=-\cos x.}$\\
        Do đó $I_2=-x\cos x\Bigg|_0^{\tfrac{\pi}{2}}+\displaystyle\int\limits_{0}^{\tfrac{\pi}{2}} \cos x \mathrm{\,d}x=\sin x\Bigg|_0^{\tfrac{\pi}{2}}=1$.\\
        Vậy $I=\dfrac{\pi^3}{24}+\dfrac{\pi}{4}-2$.
        }
            \item $I=\displaystyle\int\limits_{0}^{\tfrac{\pi^2}{4}} \cos\sqrt{x} \mathrm{\,d}x$. \dapso{$I=\pi-2$}
            \loigiai{
        Đặt $t=\sqrt{x}\Rightarrow t^2=x\Rightarrow 2t\mathrm{\,d}t=\mathrm{\,d}x$.\\
        Đổi cận $\heva{& x=0 &\Rightarrow &t=0 \\ & x=\dfrac{\pi^2}{4}&\Rightarrow &t=\dfrac{\pi}{2}.}$\\
        Vậy $I=\displaystyle\int\limits_{0}^{\tfrac{\pi}{2}} 2t\cos t \mathrm{\,d}t$.\\
        Đặt $\heva{& u=2t \\ & \mathrm{\,d}v=\cos t\mathrm{\,d}t}\Rightarrow \heva{&\mathrm{\,d}u=2\mathrm{\,d}t\\&v=\sin t.}$\\
        Suy ra $I=2t\sin t\Bigg|_0^{\tfrac{\pi}{2}}-\displaystyle\int\limits_{0}^{\tfrac{\pi}{2}} 2\sin t \mathrm{\,d}t=\pi+2\cos t\Bigg|_0^{\tfrac{\pi}{2}}=\pi-2$.	
        }
            \item $I=\displaystyle\int\limits_{0}^{\pi^2} \sin\sqrt{x} \mathrm{\,d}x$. \dapso{$I=2\pi$}
            \loigiai{
        Đặt $t=\sqrt{x}\Rightarrow t^2=x\Rightarrow 2t\mathrm{\,d}t=\mathrm{\,d}x$.\\
        Đổi cận $\heva{& x=0&\Rightarrow &t=0 \\ & x=\pi^2&\Rightarrow&t=\pi.}$\\
        Vậy $I=\displaystyle\int\limits_{0}^{\pi} 2t\sin t \mathrm{\,d}t$.\\
        Đặt $\heva{& u=2t \\ & \mathrm{\,d}v=\sin t\mathrm{\,d}t}\Rightarrow \heva{& \mathrm{\,d}u=2\mathrm{\,d}t \\ & v=-\cos t.}$\\
        Suy ra $I=-2t\cos t\Bigg|_0^{\pi}+\displaystyle\int\limits_{0}^{\pi} 2\cos t \mathrm{\,d}t=2\pi+2\sin t\Bigg|_0^{\pi}=2\pi$.	
        }
            \item $I=\displaystyle\int\limits_{0}^{\tfrac{\pi}{2}} \sin2x\ln \left(1+\cos x\right) \mathrm{\,d}x$. \dapso{$I=\dfrac{1}{2}$}
            \loigiai{
        Có $I=\displaystyle\int\limits_{0}^{\tfrac{\pi}{2}} \sin2x\ln \left(1+\cos x\right) \mathrm{\,d}x=\int\limits_{0}^{\tfrac{\pi}{2}} 2\sin x\cos x\ln \left(1+\cos x\right) \mathrm{\,d}x$.\\
        Đặt $t=1+\cos x\Rightarrow \cos x=t-1\Rightarrow -\sin x\mathrm{\,d}x=\mathrm{\,d}t$.\\
        Đổi cận $\heva{& x=0&\Rightarrow&t=2 \\ & x=\dfrac{\pi}{2}&\Rightarrow&t=1.}$\\
        Vậy $I=-\displaystyle\int\limits_{2}^{1} 2\left(t-1\right) \ln t\mathrm{\,d}t=\int\limits_{1}^{2} \left(2t-2\right) \ln t\mathrm{\,d}t$.\\
        Đặt $\heva{& u=\ln t \\ & \mathrm{\,d}v=\left(2t-2\right)\mathrm{\,d}t}\Rightarrow \heva{& \mathrm{\,d}u=\dfrac{1}{t}\mathrm{\,d}t \\ & v=t^2-2t.}$\\
        Suy ra $I=\left(t^2-2t\right)\ln t\Bigg|_1^2-\displaystyle\int\limits_{1}^{2} \dfrac{t^2-2t}{t} \mathrm{\,d}t=-\int\limits_{1}^{2} \left(t-2\right) \mathrm{\,d}t=-\left(\dfrac{t^2}{2}-2t\right)\Bigg|_1^2=\dfrac{1}{2}$.	
        }
            \item $I=\displaystyle\int\limits_{0}^{\tfrac{\pi}{2}} \sin2x\ln \left(1+\cos^2x\right) \mathrm{\,d}x$. \dapso{$I=2\ln 2-1$}
            \loigiai{
        Đặt $t=1+\cos^2x\Rightarrow \mathrm{\,d}t=-2\cos x\sin x\mathrm{\,d}x=-\sin2x\mathrm{\,d}x$.\\
        Đổi cận $\heva{& x=0&\Rightarrow&t=2 \\ & x=\dfrac{\pi}{2}&\Rightarrow&t=1.}$\\
        Vậy $I=-\displaystyle\int\limits_{2}^{1} \ln t \mathrm{\,d}t=\int\limits_{1}^{2} \ln t \mathrm{\,d}t$.\\
        Đặt $\heva{& u=\ln t \\ & \mathrm{\,d}v=\mathrm{\,d}t}\Rightarrow\heva{& \mathrm{\,d}u=\dfrac{1}{t}\mathrm{\,d}t \\ & v=t.}$\\
        Suy ra $I=t\ln t\Bigg|_1^2-\displaystyle\int\limits_{1}^{2} \mathrm{\,d}t=2\ln 2 - t\Bigg|_1^2=2\ln 2-1$.	
        }
        \end{enumerate}
    \end{bt}
    
    \begin{bt}%[Hồ Sỹ Trường]%[DCHT]%[2D3K2-3]
    Tính các tích phân sau:
        \begin{enumerate}
            \item $I=\displaystyle\int\limits_{0}^{1} \left(1-x\right)\left(2+\mathrm{e}^{2x}\right) \mathrm{\,d}x$. \dapso{$I=\dfrac{\mathrm{e}^2}{4}+\dfrac{1}{4}$}
            \loigiai{
        Có $I=\displaystyle\int\limits_{0}^{1} \left(2-2x\right) \mathrm{\,d}x+\int\limits_{0}^{1} \left(1-x\right)\mathrm{e}^{2x} \mathrm{\,d}x=I_1+I_2$.\\
        + $I_1=\left(2x-x^2\right)\Bigg|_0^1=1$.\\
        + $I_2=\displaystyle\int\limits_{0}^{1} \left(1-x\right)\mathrm{e}^{2x} \mathrm{\,d}x$.\\
        Đặt $\heva{& u=1-x \\ & \mathrm{\,d}v=\mathrm{e}^{2x}\mathrm{\,d}x}\Rightarrow \heva{& \mathrm{\,d}u=-\mathrm{\,d}x \\ & v=\dfrac{\mathrm{e}^{2x}}{2}.}$\\
        Suy ra $I_2=\dfrac{\left(1-x\right)\mathrm{e}^{2x}}{2}\Bigg|_0^1+\dfrac{1}{2}\displaystyle\int\limits_{0}^{1} \mathrm{e}^{2x} \mathrm{\,d}x=-\dfrac{1}{2}+\dfrac{\mathrm{e}^{2x}}{4}\Bigg|_0^1=\dfrac{\mathrm{e}^2}{4}-\dfrac{3}{4}$.\\
        Vậy $I=\dfrac{\mathrm{e}^2}{4}+\dfrac{1}{4}$.		
        }
            \item $I=\displaystyle\int\limits_{0}^{\pi} x\left(x-\sin x\right) \mathrm{\,d}x$. \dapso{$I=\dfrac{\pi^3}{3}-\pi$}
            \loigiai{
        Có $I=\displaystyle\int\limits_{0}^{\pi} x^2 \mathrm{\,d}x-\int\limits_{0}^{\pi} x\sin x \mathrm{\,d}x=I_1-I_2$.\\
        + $I_1=\dfrac{x^3}{3}\Bigg|_0^{\pi}=\dfrac{\pi^3}{3}$.\\
        + $I_2=\displaystyle\int\limits_{0}^{\pi} x\sin x \mathrm{\,d}x$.\\
        Đặt $\heva{& u=x \\ & \mathrm{\,d}v=\sin x\mathrm{\,d}x}\Rightarrow \heva{& \mathrm{\,d}u=\mathrm{\,d}x \\ & v=-\cos x.}$\\
        Suy ra $I_2=-x\cos x\Bigg|_0^\pi+\displaystyle\int\limits_{0}^{\pi} \cos x \mathrm{\,d}x=\pi+\sin x\Bigg|_0^{\pi}=\pi$.\\
        Vậy $I=\dfrac{\pi^3}{3}-\pi$.	
        }
        \item $I=\displaystyle\int\limits_{1}^{2} \dfrac{x^3-2\ln x}{x^2} \mathrm{\,d}x$. \dapso{$I=\ln 2+\dfrac{1}{2}$}
            \loigiai{
        Có $I=\displaystyle\int\limits_{1}^{2} x \mathrm{\,d}x-2\int\limits_{1}^{2} \dfrac{\ln x}{x^2} \mathrm{\,d}x=I_1-2I_2$.\\
        + $I_1=\dfrac{x^2}{2}\Bigg|_1^2=\dfrac{3}{2}$.\\
        + $I_2=\displaystyle\int\limits_{1}^{2} \dfrac{\ln x}{x^2} \mathrm{\,d}x$.\\
        Đặt $\heva{& u=\ln x \\ & \mathrm{\,d}v=\dfrac{1}{x^2}\mathrm{\,d}x}\Rightarrow \heva{& \mathrm{\,d}u=\dfrac{1}{x}\mathrm{\,d}x \\ & v=-\dfrac{1}{x}.}$\\
        Suy ra $I_2=-\dfrac{\ln x}{x}\Bigg|_1^2+\displaystyle\int\limits_{1}^{2} \dfrac{1}{x^2} \mathrm{\,d}x=-\dfrac{\ln 2}{2}-\dfrac{1}{x}\Bigg|_1^2=-\dfrac{\ln 2}{2}+\dfrac{1}{2}$.\\
        Vậy $I=\dfrac{3}{2}+\ln 2-1=\ln 2+\dfrac{1}{2}$.	
        }
            \item $I=\displaystyle \int\limits_{1}^{2} \dfrac{1+x^2\mathrm{e}^x}{x} \mathrm{\,d}x$. \dapso{$I=\mathrm{e}^2+\ln 2$}
            \loigiai{
        Có $I=\displaystyle\int\limits_{1}^{2} \dfrac{1}{x} \mathrm{\,d}x+\int\limits_{1}^{2} x\mathrm{e}^x \mathrm{\,d}x=I_1+I_2$.\\
        + $I_1=\ln x\Bigg|_1^2=\ln 2$.\\
        + $I_2=\displaystyle\int\limits_{1}^{2} x\mathrm{e}^x \mathrm{\,d}x$.\\
        Đặt$\heva{& u=x \\ & \mathrm{\,d}v=\mathrm{e}^x\mathrm{\,d}x}\Rightarrow\heva{& \mathrm{\,d}u=\mathrm{\,d}x \\ & v=\mathrm{e}^x.}$\\
        Suy ra $I_2=x\mathrm{e}^x\Bigg|_1^2-\displaystyle\int\limits_{1}^{2} \mathrm{e}^x \mathrm{\,d}x=2\mathrm{e}^2-\mathrm{e}-\mathrm{e}^x\Bigg|_1^2=\mathrm{e}^2$.\\
        Vậy $I=\mathrm{e}^2+\ln 2$.	
        }
            \item $I=\displaystyle\int\limits_{0}^{1} \dfrac{\mathrm{e}^x+x}{\mathrm{e}^x} \mathrm{\,d}x$. \dapso{$I=2-\dfrac{2}{\mathrm{e}}$}
            \loigiai{
        Có $I=\displaystyle\int\limits_{0}^{1} \mathrm{\,d}x+\int\limits_{0}^{1} x\mathrm{e}^{-x} \mathrm{\,d}x=I_1+I_2$.\\
        + $I_1=x\Bigg|_0^1=1$.\\
        + $I_2=\displaystyle\int\limits_{0}^{1} x\mathrm{e}^{-x} \mathrm{\,d}x$.\\
        Đặt $\heva{& u=x \\ & \mathrm{\,d}v=\mathrm{e}^{-x}\mathrm{\,d}x}\Rightarrow \heva{& \mathrm{\,d}u=\mathrm{\,d}x \\ & v=-\mathrm{e}^{-x}.}$\\
        Suy ra $I_2=-x\mathrm{e}^{-x}\Bigg|_0^1+\displaystyle\int\limits_{0}^{1} \mathrm{e}^{-x} \mathrm{\,d}x=-\dfrac{1}{\mathrm{e}}-\mathrm{e}^{-x}\Bigg|_0^1=1-\dfrac{2}{\mathrm{e}}$.\\
        Vậy $I=2-\dfrac{2}{\mathrm{e}}$.	
        }
            \item $I=\displaystyle\int\limits_{1}^{3} \dfrac{1+\ln \left(x+1\right)}{x^2} \mathrm{\,d}x$. \dapso{$I=\dfrac{2}{3}+\ln 3-\dfrac{2}{3}\ln 2$}
            \loigiai{
        Có $I=\displaystyle\int\limits_{1}^{3} \dfrac{1}{x^2} \mathrm{\,d}x+\int\limits_{1}^{3} \dfrac{\ln \left(x+1\right)}{x^2} \mathrm{\,d}x=I_1+I_2$.\\
        + $I_1=-\dfrac{1}{x}\Bigg|_1^3=\dfrac{2}{3}$.\\
        + $I_2=\displaystyle\int\limits_{1}^{3} \dfrac{\ln \left(x+1\right)}{x^2} \mathrm{\,d}x$.\\
        Đặt $\heva{& u=\ln \left(x+1\right) \\ & \mathrm{\,d}v=\dfrac{1}{x^2}\mathrm{\,d}x}\Rightarrow \heva{& \mathrm{\,d}u=\dfrac{1}{x+1}\mathrm{\,d}x \\ & v=-\dfrac{1}{x}.}$\\
        Suy ra $I_2=-\dfrac{1}{x}\ln \left(x+1\right)\Bigg|_1^3+\displaystyle\int\limits_{1}^{3} \dfrac{1}{x\left(x+1\right)} \mathrm{\,d}x=\dfrac{1}{3}\ln 2+\int\limits_{1}^{3} \left(\dfrac{1}{x}-\dfrac{1}{x+1}\right) \mathrm{\,d}x=\dfrac{1}{3}\ln 2+\ln \left|\dfrac{x}{x+1}\right|\Bigg|_1^3$\\
        $=\ln 3-\dfrac{2}{3}\ln 2$.\\
        Vậy $I=\dfrac{2}{3}+\ln 3-\dfrac{2}{3}\ln 2$.
        }
            \item $I=\displaystyle\int\limits_{0}^{1} x\left(\mathrm{e}^x+\dfrac{2}{x+1}\right) \mathrm{\,d}x$. \dapso{$I=3-2\ln 2$}
            \loigiai{
        Có $I=\displaystyle\int\limits_{0}^{1} x\mathrm{e}^x \mathrm{\,d}x+\int\limits_{0}^{1} \dfrac{2x}{x+1} \mathrm{\,d}x=I_1+I_2$.\\
            + $I_1=\displaystyle\int\limits_{0}^{1} x\mathrm{e}^x \mathrm{\,d}x$.\\
        Đặt$\heva{& u=x \\ & \mathrm{\,d}v=\mathrm{e}^x\mathrm{\,d}x}\Rightarrow\heva{& \mathrm{\,d}u=\mathrm{\,d}x \\ & v=\mathrm{e}^x.}$\\
        Suy ra $I_1=x\mathrm{e}^x\Bigg|_0^1-\displaystyle\int\limits_{0}^{1} \mathrm{e}^x \mathrm{\,d}x=\mathrm{e}-\mathrm{e}^x\Bigg|_0^1=1$.\\
        + $I_2=\displaystyle\int\limits_{0}^{1} \dfrac{2x}{x+1} \mathrm{\,d}x=\int\limits_{0}^{1} \left(2-\dfrac{2}{x+1}\right) \mathrm{\,d}x=\left(2x-2\ln \left|x+1\right|\right)\Bigg|_0^1=2-2\ln 2$.\\
        Vậy $I=3-2\ln 2$.
        }
            \item $I=\displaystyle\int\limits_{0}^{1} \left(\mathrm{e}^x+\sqrt{3x^2+1}\right)x \mathrm{\,d}x$. \dapso{$I=\dfrac{16}{9}$}
            \loigiai{
        Có $I=\displaystyle\int\limits_{0}^{1} x\mathrm{e}^x \mathrm{\,d}x+\int\limits_{0}^{1} x\sqrt{3x^2+1} \mathrm{\,d}x=I_1+I_2$.\\
        + $I_1=\displaystyle\int\limits_{0}^{1} x\mathrm{e}^x \mathrm{\,d}x$.\\
        Đặt$\heva{& u=x \\ & \mathrm{\,d}v=\mathrm{e}^x\mathrm{\,d}x}\Rightarrow\heva{& \mathrm{\,d}u=\mathrm{\,d}x \\ & v=\mathrm{e}^x.}$\\
        Suy ra $I_2=x\mathrm{e}^x\Bigg|_0^1-\displaystyle\int\limits_{0}^{1} \mathrm{e}^x \mathrm{\,d}x=\mathrm{e}-\mathrm{e}^x\Bigg|_0^1=1$.\\
        + $I_2=\displaystyle\int\limits_{0}^{1} x\sqrt{3x^2+1} \mathrm{\,d}x$.\\
        Đặt $t=\sqrt{3x^2+1}\Rightarrow t^2=3x^2+1\Rightarrow t\mathrm{\,d}t=3x\mathrm{\,d}x\Rightarrow x\mathrm{\,d}x=\dfrac{t}{3}\mathrm{\,d}t$.\\
        Đổi cận $\heva{& x=0&\Rightarrow&t=1 \\ & x=1&\Rightarrow&t=2}$.\\
        Suy ra $I_2=\displaystyle\int\limits_{1}^{2} \dfrac{t^2}{3} \mathrm{\,d}t=\dfrac{t^3}{9}\Bigg|_1^2=\dfrac{7}{9}$.\\
        Vậy $I=1+\dfrac{7}{9}=\dfrac{16}{9}$.	
        }
            \item $I=\displaystyle\int\limits_{0}^{\tfrac{\pi}{2}} \left(x+\cos^2x\right)\sin x \mathrm{\,d}x$. \dapso{$I=\dfrac{4}{3}$}
            \loigiai{
        Có $I=\displaystyle\int\limits_{0}^{\tfrac{\pi}{2}} x\sin x \mathrm{\,d}x+\int\limits_{0}^{\tfrac{\pi}{2}} \cos^2x\sin x \mathrm{\,d}x=I_1+I_2$.\\
        + $I_1=\displaystyle\int\limits_{0}^{\tfrac{\pi}{2}} x\sin x \mathrm{\,d}x$.\\
        Đặt $\heva{& u=x \\ & \mathrm{\,d}v=\sin x\mathrm{\,d}x}\Rightarrow\heva{& \mathrm{\,d}u=\mathrm{\,d}x \\ & v=-\cos x.}$\\
        Suy ra $I_2=-x\cos x\Bigg|_0^{\tfrac{\pi}{2}}+\displaystyle\int\limits_{0}^{\tfrac{\pi}{2}} \cos x \mathrm{\,d}x=\sin x\Bigg|_0^{\tfrac{\pi}{2}}=1$.\\
        + $I_2=\displaystyle\int\limits_{0}^{\tfrac{\pi}{2}} \cos^2x\sin x \mathrm{\,d}x$.\\
        Đặt $t=\cos x\Rightarrow \mathrm{\,d}t=-\sin x\mathrm{\,d}x$.\\
        Đổi cận $\heva{& x=0&\Rightarrow&t=1 \\ & x=\dfrac{\pi}{2}&\Rightarrow&t=0.}$\\
        Suy ra $I_2=-\displaystyle\int\limits_{1}^{0} t^2 \mathrm{\,d}t=\int\limits_{0}^{1} t^2 \mathrm{\,d}t=\dfrac{t^3}{3}\Bigg|_0^1=\dfrac{1}{3}$.\\
        Vậy $I=1+\dfrac{1}{3}=\dfrac{4}{3}$.
        }
            \item $I=\displaystyle\int\limits_{1}^{\mathrm{e}} \left(x+\dfrac{1}{x}\right)\ln x \mathrm{\,d}x$. \dapso{$I=\dfrac{\mathrm{e}^2+3}{4}$}
            \loigiai{
        Có $I=\displaystyle\int\limits_{1}^{\mathrm{e}} x\ln x \mathrm{\,d}x+\int\limits_{1}^{\mathrm{e}} \dfrac{\ln x}{x} \mathrm{\,d}x=I_1+I_2$.\\
        + $I_1=\displaystyle\int\limits_{1}^{\mathrm{e}} x\ln x \mathrm{\,d}x$.\\
        Đặt $\heva{& u=\ln x \\ & \mathrm{\,d}v=x\mathrm{\,d}x}\Rightarrow \heva{& \mathrm{\,d}u=\dfrac{1}{x}\mathrm{\,d}x \\ & v=\dfrac{x^2}{2}.}$\\
        Suy ra $I_1=\dfrac{x^2\ln x}{2}\Bigg|_1^{\mathrm{e}}-\displaystyle\int\limits_{1}^{\mathrm{e}} \dfrac{x}{2} \mathrm{\,d}x=\dfrac{\mathrm{e}^2}{2}-\dfrac{x^2}{4}\Bigg|_1^{\mathrm{e}}=\dfrac{\mathrm{e}^2+1}{4}$.\\
        $I_2=\displaystyle\int\limits_{1}^{\mathrm{e}} \dfrac{\ln x}{x} \mathrm{\,d}x$.\\
        Đặt $t=\ln x\Rightarrow \mathrm{\,d}t=\dfrac{1}{x}\mathrm{\,d}x$.\\
        Đổi cận $\heva{& x=1&\Rightarrow&t=0 \\ & x=\mathrm{e}&\Rightarrow&t=1.}$\\
        Suy ra $I_2=\displaystyle\int\limits_{0}^{1} t \mathrm{\,d}t=\dfrac{t^2}{2}\Bigg|_0^1=\dfrac{1}{2}$.\\
        Vậy $I=\dfrac{\mathrm{e}^2+1}{4}+\dfrac{1}{2}=\dfrac{\mathrm{e}^2+3}{4}$.	
        }
            \item $I=\displaystyle\int\limits_{0}^{1} x^3\mathrm{e}^{x^2} \mathrm{\,d}x$. \dapso{$I=\dfrac{1}{2}$}
            \loigiai{
        Đặt $t=x^2\Rightarrow \mathrm{\,d}t=2x\mathrm{\,d}x$.\\
        Đổi cận $\heva{& x=0&\Rightarrow&t=0 \\ & x=1&\Rightarrow&t=1.}$\\
        Suy ra $I=\dfrac{1}{2}\displaystyle\int\limits_{0}^{1} t\mathrm{e}^t \mathrm{\,d}t$.\\
        Đặt $\heva{& u=t \\ & \mathrm{\,d}v=\mathrm{e}^t\mathrm{\,d}t}\Rightarrow\heva{& \mathrm{\,d}u=\mathrm{\,d}t \\ & v=\mathrm{e}^t.}$\\
        Suy ra $I=\dfrac{1}{2}t\mathrm{e}^t\Bigg|_0^1-\dfrac{1}{2}\displaystyle\int\limits_{0}^{1} \mathrm{e}^t \mathrm{\,d}t=\dfrac{\mathrm{e}}{2}-\dfrac{\mathrm{e}^t}{2}\Bigg|_0^1=\dfrac{1}{2}$.	
        }
            \item $I=\displaystyle\int\limits_{0}^{1} x^5\mathrm{e}^{x^3} \mathrm{\,d}x$. \dapso{$I=\dfrac{1}{3}$}
            \loigiai{
        Đặt $t=x^3\Rightarrow \mathrm{\,d}t=3x^2\mathrm{\,d}x$.\\
        Đổi cận $\heva{& x=0&\Rightarrow&t=0 \\ & x=1&\Rightarrow&t=1.}$\\
        Suy ra $I=\dfrac{1}{3}\displaystyle\int\limits_{0}^{1} t\mathrm{e}^t \mathrm{\,d}t$.\\
            Đặt $\heva{& u=t \\ & \mathrm{\,d}v=\mathrm{e}^t\mathrm{\,d}t}\Rightarrow\heva{& \mathrm{\,d}u=\mathrm{\,d}t \\ & v=\mathrm{e}^t.}$\\
        Suy ra $I=\dfrac{1}{3}t\mathrm{e}^t\Bigg|_0^1-\dfrac{1}{3}\displaystyle\int\limits_{0}^{1} \mathrm{e}^t \mathrm{\,d}t=\dfrac{\mathrm{e}}{3}-\dfrac{\mathrm{e}^t}{3}\Bigg|_0^1=\dfrac{1}{3}$.	
        }
            \item $I=\displaystyle\int\limits_{0}^{1} \left(8x^3-2x\right)\mathrm{e}^{x^2} \mathrm{\,d}x$. \dapso{$I=5-\mathrm{e}$}
            \loigiai{
        Có $I=8\displaystyle\int\limits_{0}^{1} x^3\mathrm{e}^{x^2} \mathrm{\,d}x-2\int\limits_{0}^{1} x\mathrm{e}^{x^2} \mathrm{\,d}x=8I_1-2I_2$.\\
        + $I_1=\displaystyle\int\limits_{0}^{1} x^3\mathrm{e}^{x^2} \mathrm{\,d}x$.\\
        Đặt $t=x^2\Rightarrow \mathrm{\,d}t=2x\mathrm{\,d}x$.\\
        Đổi cận $\heva{& x=0&\Rightarrow&t=0 \\ & x=1&\Rightarrow&t=1.}$\\
        Suy ra $I_1=\dfrac{1}{2}\displaystyle\int\limits_{0}^{1} t\mathrm{e}^t \mathrm{\,d}t$.\\
        Đặt $\heva{& u=t \\ & \mathrm{\,d}v=\mathrm{e}^t\mathrm{\,d}t}\Rightarrow\heva{& \mathrm{\,d}u=\mathrm{\,d}t \\ & v=\mathrm{e}^t.}$\\
        Suy ra $I_1=\dfrac{1}{2}t\mathrm{e}^t\Bigg|_0^1-\dfrac{1}{2}\displaystyle\int\limits_{0}^{1} \mathrm{e}^t \mathrm{\,d}t=\dfrac{\mathrm{e}}{2}-\dfrac{\mathrm{e}^t}{2}\Bigg|_0^1=\dfrac{1}{2}$.\\
        + $I_2=\displaystyle\int\limits_{0}^{1} x\mathrm{e}^{x^2} \mathrm{\,d}x$.\\
        Đặt $t=x^2\Rightarrow\mathrm{\,d}t=2x\mathrm{\,d}x$.\\
        Đổi cận $\heva{& x=0&\Rightarrow&t=0 \\ & x=1&\Rightarrow&t=1.}$\\
        Suy ra $I_2=\dfrac{1}{2}\displaystyle\int\limits_{0}^{1} \mathrm{e}^t \mathrm{\,d}t=\dfrac{1}{2}\mathrm{e}^t\Bigg|_0^1=\dfrac{\mathrm{e}-1}{2}$.\\
        Vậy $I=4-\mathrm{e}+1=5-\mathrm{e}$.
        }
            \item $I=\displaystyle\int\limits_{0}^{1} \sqrt{x}\mathrm{e}^{\sqrt{x}} \mathrm{\,d}x$. \dapso{$I=2\mathrm{e}-4$}
            \loigiai{
        Đặt $t=\sqrt{x}\Rightarrow t^2=x\Rightarrow 2t\mathrm{\,d}t=\mathrm{\,d}x$.\\
        Đổi cận $\heva{& x=0&\Rightarrow&t=0 \\ & x=1&\Rightarrow&t=1.}$\\
        Suy ra $I=2\displaystyle\int\limits_{0}^{1} t^2\mathrm{e}^t \mathrm{\,d}t$.\\
        Đặt $\heva{& u=t^2 \\ & \mathrm{\,d}v=\mathrm{e}^t\mathrm{\,d}t}\Rightarrow\heva{& \mathrm{\,d}u=2t\mathrm{\,d}t \\ & v=\mathrm{e}^t.}$\\
        Do đó $I=2t^2\mathrm{e}^t\Bigg|_0^1-4\displaystyle\int\limits_{0}^{1} t\mathrm{e}^t \mathrm{\,d}t=2\mathrm{e}-4J$.\\
        Đặt $\heva{& u=t \\ & \mathrm{\,d}v=\mathrm{e}^t\mathrm{\,d}t}\Rightarrow\heva{& \mathrm{\,d}u=\mathrm{\,d}t \\ & v=\mathrm{e}^t.}$\\
        Suy ra $J=t\mathrm{e}^t\Bigg|_0^1-\displaystyle\int\limits_{0}^{1} \mathrm{e}^t \mathrm{\,d}t=\mathrm{e}-\mathrm{e}^t\Bigg|_0^1=1$.\\
        Vậy $I=2\mathrm{e}-4$.	
        }
            \item $I=\displaystyle\int\limits_{0}^{\tfrac{\pi^3}{27}} \sin\sqrt[3]{x} \mathrm{\,d}x$. \dapso{$I=-\dfrac{\pi^2}{6}+\pi\sqrt{3}-3$}
            \loigiai{
        Đặt $t=\sqrt[3]{x}\Rightarrow t^3=x\Rightarrow 3t^2\mathrm{\,d}t=\mathrm{\,d}x$.\\
        Đổi cận $\heva{& x=0&\Rightarrow&t=0 \\ & x=\dfrac{\pi^3}{27}&\Rightarrow&t=\dfrac{\pi}{3}.}$\\
        Suy ra $I=3\displaystyle\int\limits_{0}^{\tfrac{\pi}{3}} t^2\sin t \mathrm{\,d}t$.\\
        Đặt $\heva{& u=t^2 \\ & \mathrm{\,d}v=\sin t\mathrm{\,d}t}\Rightarrow\heva{& \mathrm{\,d}u=2t\mathrm{\,d}t \\ & v=-\cos t.}$\\
        Suy ra $I=-3t^2\cos t\Bigg|_0^{\tfrac{\pi}{3}}+6\displaystyle\int\limits_{0}^{\tfrac{\pi}{3}} t\cos t \mathrm{\,d}t=-\dfrac{\pi^2}{6}+6J$.\\
        Đặt $\heva{& u=t \\ & \mathrm{\,d}v=\cos t\mathrm{\,d}t}\Rightarrow\heva{& \mathrm{\,d}u=\mathrm{\,d}t \\ & v=\sin t.}$\\
        Suy ra $J=t\sin t\Bigg|_0^{\tfrac{\pi}{3}}-\displaystyle\int\limits_{0}^{\tfrac{\pi}{3}} \sin t \mathrm{\,d}x=\dfrac{\pi\sqrt{3}}{6}+\cos t\Bigg|_0^{\tfrac{\pi}{3}}=\dfrac{\pi\sqrt{3}}{6}-\dfrac{1}{2}=\dfrac{\pi\sqrt{3}-3}{6}$.\\
        Vậy $I=-\dfrac{\pi^2}{6}+\pi\sqrt{3}-3$.	
        }
            \item $I=\displaystyle\int\limits_{1-\tfrac{\pi^2}{4}}^{1} \cos\sqrt{1-x} \mathrm{\,d}x$. \dapso{$I=\pi-2$}
            \loigiai{
        Đặt $t=\sqrt{1-x}\Rightarrow t^2=1-x\Rightarrow 2t\mathrm{\,d}t=-\mathrm{\,d}x$.\\
        Đổi cận $\heva{& x=1-\dfrac{\pi^2}{4}&\Rightarrow&t=\dfrac{\pi}{2} \\ & x=1&\Rightarrow&t=0.}$\\
        Suy ra $I=-2\displaystyle\int\limits_{\tfrac{\pi}{2}}^{0} t\cos t \mathrm{\,d}t=2\int\limits_{0}^{\tfrac{\pi}{2}} t\cos t \mathrm{\,d}t$.\\
        Đặt $\heva{& u=t \\ & \mathrm{\,d}v=\cos t\mathrm{\,d}t}\Rightarrow\heva{& \mathrm{\,d}u=\mathrm{\,d}t \\ & v=\sin t.}$\\
        Suy ra $I=2t\sin t\Bigg|_0^{\tfrac{\pi}{2}}-2\displaystyle\int\limits_{0}^{\tfrac{\pi}{2}} \sin t \mathrm{\,d}t=\pi +2\cos t\Bigg|_0^{\tfrac{\pi}{2}}=\pi-2$.
        }
        \item $I=\displaystyle\int\limits_{\tfrac{\pi}{6}}^{\tfrac{\pi}{2}} \dfrac{\cos x\ln \left(\sin x\right)}{\sin^2x} \mathrm{\,d}x$. \dapso{$I=1-2\ln 2$}
            \loigiai{
        Đặt $t=\sin x\Rightarrow \mathrm{\,d}t=\cos x\mathrm{\,d}x$.\\
        Đổi cận $\heva{& x=\dfrac{\pi}{6}&\Rightarrow&t=\dfrac{1}{2} \\ & x=\dfrac{\pi}{2}&\Rightarrow&t=1.}$\\
        Suy ra $I=\displaystyle\int\limits_{\tfrac{1}{2}}^{1} \dfrac{\ln t}{t^2} \mathrm{\,d}t$.\\
        Đặt $\heva{& u=\ln t \\ & \mathrm{\,d}v=\dfrac{1}{t^2}\mathrm{\,d}t}\Rightarrow\heva{& \mathrm{\,d}u=\dfrac{1}{t}\mathrm{\,d}t \\ & v=-\dfrac{1}{t}.}$\\
        Suy ra $I=-\dfrac{\ln t}{t}\Bigg|_{\tfrac{1}{2}}^{1}+\displaystyle\int\limits_{\tfrac{1}{2}}^{1} \dfrac{1}{t^2} \mathrm{\,d}t=2\ln \dfrac{1}{2}-\dfrac{1}{t}\Bigg|_{\tfrac{1}{2}}^{1}=1-2\ln 2$.	
        }
            \item $I=\displaystyle\int\limits_{\tfrac{\pi}{4}}^{\tfrac{\pi}{3}} \dfrac{\ln \left(\tan x\right)}{\cos^2x} \mathrm{\,d}x$. \dapso{$I=\dfrac{\sqrt{3}\ln 3}{2}-\sqrt{3}+1$}
            \loigiai{
        Đặt $t=\tan x\Rightarrow \mathrm{\,d}t=\dfrac{1}{\cos^2x}\mathrm{\,d}x$.\\
        Đổi cận $\heva{& x=\dfrac{\pi}{4}&\Rightarrow&t=1 \\ & x=\dfrac{\pi}{3}&\Rightarrow&t=\sqrt{3}.}$\\
        Suy ra $I=\displaystyle\int\limits_{1}^{\sqrt{3}} \ln t \mathrm{\,d}t$.\\
        Đặt $\heva{& u=\ln t \\ & \mathrm{\,d}v=\mathrm{\,d}t}\Rightarrow\heva{& \mathrm{\,d}u=\dfrac{1}{t}\mathrm{\,d}t \\ & v=t.}$\\
        Suy ra $I=t\ln t\Bigg|_1^{\sqrt{3}}-\displaystyle\int\limits_{1}^{\sqrt{3}} \mathrm{\,d}t=\sqrt{3}\ln \sqrt{3}-t\Bigg|_1^{\sqrt{3}}=\dfrac{\sqrt{3}\ln 3}{2}-\sqrt{3}+1$.	
        }
        \end{enumerate}
    \end{bt}
    
\subsection{BÀI TẬP TRẮC NGHIỆM}
\Opensolutionfile{ans}[ans/2D3-TN-02]

\begin{ex}%[Đề chính thức THPTQG 2019, Mã đề 101]%[Huỳnh Quy, THPTQG2019]%[2D3Y2-1]
	Biết $\displaystyle \int\limits_{0}^{1}f(x) \mathrm{\,d}x=-2$ và $\displaystyle \int\limits_{0}^{1}g(x)\mathrm{\,d}x=3$, khi đó $\displaystyle \int\limits_{0}^{1}\left[f(x)-g(x)\right]\mathrm{\,d}x$ bằng
	\choice
	{\True $-5$}
	{$5$}
	{$-1$}
	{$1$}
	\loigiai{Ta có $\displaystyle \int\limits_{0}^{1}\left[f(x)-g(x)\right]\mathrm{\,d}x=\int\limits_{0}^{1}f(x) \mathrm{\,d}x-\int\limits_{0}^{1}g(x)\mathrm{\,d}x=-2-3=-5$.}
\end{ex}
\begin{ex}%[Dự án đề thi THPTQG 2019 mã đề 110, Dũng Lê]%[2D3Y2-1]
	Biết $\displaystyle\int\limits_0^1f(x)\mathrm{\,d}x=3$ và $\displaystyle\int\limits_0^1g(x)\mathrm{\,d}x=-4$, khi đó $\displaystyle\int\limits_0^1[f(x)+g(x)]\mathrm{\,d}x$ bằng
	\choice
	{$-7$}
	{$7$}
	{\True $-1$}
	{$1$}
	\loigiai{
		Ta có $\displaystyle\int\limits_0^1[f(x)+g(x)]\mathrm{\,d}x=\displaystyle\int\limits_0^1f(x)\mathrm{\,d}x+\displaystyle\int\limits_0^1g(x)\mathrm{\,d}x=3-4=-1$.
	}
\end{ex}
\begin{ex}%[12-EX-ĐHVinh-L3]%[Nguyễn Trung Kiên]%[2D3Y2-1]
	Cho $f(x)$ và $g(x)$ là các hàm số liên tục trên đoạn $[a;b]$. Mệnh đề nào sau đây đúng?
	\choice
	{$\displaystyle\int\limits_a^b |f(x)+g(x)|\mathrm{\,d}x=\displaystyle\int\limits_a^b |f(x)|\mathrm{\,d}x+\displaystyle\int\limits_a^b |g(x)|\mathrm{\,d}x$}
	{$\displaystyle\int\limits_a^b (f(x)\cdot g(x))\mathrm{\,d}x=\displaystyle\int\limits_a^b f(x)\mathrm{\,d}x\cdot\displaystyle\int\limits_a^b g(x)\mathrm{\,d}x$}
	{\True $\displaystyle\int\limits_a^b \left[f(x)+g(x)\right]\mathrm{\,d}x=\displaystyle\int\limits_a^b f(x)\mathrm{\,d}x+\displaystyle\int\limits_a^b g(x)\mathrm{\,d}x$}
	{$\displaystyle\int\limits_a^b \left|f(x)+g(x)\right|\mathrm{\,d}x=\left|\displaystyle\int\limits_a^b \left[f(x)+ g(x)\right]\mathrm{\,d}x\right|$}
	\loigiai{
		Theo tính chất của tích phân ta có $$\displaystyle\int\limits_a^b \left[f(x)+g(x)\right]\mathrm{\,d}x=\displaystyle\int\limits_a^b f(x)\mathrm{\,d}x+\displaystyle\int\limits_a^b g(x)\mathrm{\,d}x.$$}
\end{ex}
\begin{ex}%[12-EX-ĐHVinh-L3]%[Nguyễn Trung Kiên]%[2D3Y2-1]
	Cho $f(x)$ và $g(x)$ là các hàm số liên tục trên đoạn $[a;b]$ và $h$, $k$ là các hằng số. Mệnh đề nào sau đây đúng?
	\choice
	{\True $\displaystyle\int\limits_a^b \left[hf(x)-kg(x)\right]\mathrm{\,d}x=h\displaystyle\int\limits_a^b f(x)\mathrm{\,d}x-k\displaystyle\int\limits_a^b g(x)\mathrm{\,d}x$}
	{$\displaystyle\int\limits_b^a \left[f(x)-g(x)\right]\mathrm{\,d}x=\displaystyle\int\limits_a^b \left[f(x)-g(x)\right]\mathrm{\,d}x$}
	{$\displaystyle\int\limits_a^b [h+kf(x)]\mathrm{\,d}x=h+k\displaystyle\int\limits_a^b f(x)\mathrm{\,d}x$}
	{$\displaystyle\int\limits_a^b \left[f(x)\cdot g(x)\right]\mathrm{\,d}x=f(x)\displaystyle\int\limits_a^b g(x)\mathrm{\,d}x$}
	\loigiai{
		Theo tính chất của tích phân ta có $$\displaystyle\int\limits_a^b \left[hf(x)-kg(x)\right]\mathrm{\,d}x=h\displaystyle\int\limits_a^b f(x)\mathrm{\,d}x-k\displaystyle\int\limits_a^b g(x)\mathrm{\,d}x.$$}
\end{ex}
\begin{ex}%[Đề thử nghiệm - THPT.QG 2017]%[2D3Y2-1]
	Cho hàm số $f( x)$ có đạo hàm trên đoạn $[ 1;2]$, $f( 1)=1$ và $f( 2)=2$. \\
	Tính $I=\displaystyle\int_{1}^{2}
	f'(x) \mathrm{d}x$
	\choice
	{\True $I=1$}
	{$I=-1$}
	{$I=3$}
	{$I=\dfrac{7}{2}$}
	\loigiai
	{$I=\displaystyle\int_{1}^{2}
		f'(x) \mathrm{d}x =f(x)\bigg|_{1}^{2}=f(2)-f(1)=1$.}
\end{ex}
\begin{ex}%[Đề tham khảo - THPT.QG 2018]%[2D3Y2-1]
	Tích phân $\displaystyle\int\limits_0^2\dfrac{{\rm d}x}{x+3}$ bằng
	\choice
	{$\dfrac{16}{225}$}
	{$\log \dfrac{5}{3}$}
	{\True $\ln \dfrac{5}{3}$}
	{$\dfrac{2}{15}$}
	\loigiai{
		Ta có $\displaystyle\int\limits_0^2\dfrac{\mathrm{d}x}{x+3}=\ln \big|x+3\big|\bigg|_0^2=\ln |2+3|-\ln |0+3|=\ln \dfrac{5}{3}$.}
\end{ex}
\begin{ex}%[2D3B2-1]
		Cho hàm số $f(x)$ liên tục trên đoạn $[0;10]$ thỏa mãn $\displaystyle\int\limits_0^{10} f(x) \mathrm{\,d}x=7$, $\displaystyle\int\limits_2^6 f(x) \mathrm{\,d}x=3$. Tính $\displaystyle\int\limits_0^2 f(x) \mathrm{\,d}x+\displaystyle\int\limits_6^{10} f(x) \mathrm{\,d}x$.
		\choice
		{\True  $4$}
		{$-4$}
		{ $5$}
		{ $7$}
		\loigiai{
			$\displaystyle\int\limits_0^{10} f(x) \mathrm{\,d}x=\displaystyle\int\limits_0^2 f(x) \mathrm{\,d}x+\displaystyle\int\limits_2^6 f(x) \mathrm{\,d}x+\displaystyle\int\limits_6^{10} f(x) \mathrm{\,d}x \Rightarrow \displaystyle\int\limits_0^2 f(x) \mathrm{\,d}x+\displaystyle\int\limits_6^{10} f(x) \mathrm{\,d}x=\displaystyle\int\limits_0^{10} f(x) \mathrm{\,d}x-\displaystyle\int\limits_2^6 f(x) \mathrm{\,d}x=4$.
		} 
	\end{ex}
\begin{ex}%[MH2019, Nguyễn Kiều Nhã Tú]%[2D3B2-1]
 Cho $\displaystyle\int\limits_{0}^{1}f(x)\mathrm{d}x=2$ và $\displaystyle\int\limits_{0}^{1}g(x)\mathrm{d}x=5$, khi đó $\displaystyle\int\limits_{0}^{1}\left[f(x)-2g(x)\right]\mathrm{d}x$ bằng
	\choice
	{ $-3$}
	{$12$}
	{\True $-8$}
	{$1$}
	\loigiai{
Ta có $\displaystyle\int\limits_{0}^{1}\left[f(x)-2g(x)\right]\mathrm{d}x=\displaystyle\int\limits_{0}^{1}f(x)\mathrm{d}x-2\displaystyle\int\limits_{0}^{1}g(x)\mathrm{d}x=2-2\cdot5 =-8$.
	}
\end{ex}
\begin{ex}%[MH2019, Nguyễn Kiều Nhã Tú]%[2D3B2-1]
	 Cho $\displaystyle\int\limits_{-1}^{2}f(x)\mathrm{d}x=2$ và $\displaystyle\int\limits_{-1}^{2}g(x)\mathrm{d}x=-1$. Tính $I=\displaystyle\int\limits_{-1}^{2}\left[x+2f(x)+3g(x)\right]\mathrm{d}x$.
	\choice
	{ $I=\dfrac{11}{2}$}
	{$I=\dfrac{7}{2}$}
	{$I=\dfrac{17}{2}$}
	{\True $I=\dfrac{5}{2}$}
	\loigiai{Ta có $I= \left. \dfrac{x^2}{2} \right|^2_{-1}  +2\displaystyle\int\limits_{-1}^{2}f(x)\mathrm{d}x+3\displaystyle\int\limits_{-1}^{2}g(x)\mathrm{d}x=\dfrac{3}{2}+4-3=\dfrac{5}{2}$.
		
	}
\end{ex}
\begin{ex}%[MH2019, Nguyễn Kiều Nhã Tú]%[2D3B2-1]
Giả sử $\displaystyle\int\limits_{0}^{9}f(x)\mathrm{d}x=37$ và $\displaystyle\int\limits_{9}^{0}g(x)\mathrm{d}x=16$. Khi đó $I=\displaystyle\int\limits_{0}^{9}\left[2f(x)+3g(x)\right]\mathrm{d}x$ bằng
	\choice
	{\True $I=26$}
	{$I=58$}
	{$I=143$}
	{$I=122$}
	\loigiai{Ta có $I=\displaystyle\int\limits_{0}^{9}\left[2f(x)+3g(x)\right]\mathrm{d}x=\displaystyle\int\limits_{0}^{9}2f(x)\mathrm{d}x+3\displaystyle\int\limits_{0}^{9}g(x)\mathrm{d}x=2\displaystyle\int\limits_{0}^{9}f(x)\mathrm{d}x-3\displaystyle\int\limits_{9}^{0}g(x)\mathrm{d}x=26.$
	}
\end{ex}
\begin{ex}%[Đề 101, THPT.QG - 2018]%[Nguyễn Thành Sơn, dự án 12EX11]%[2D3B2-2]
	$\displaystyle\int\limits_1^2 \mathrm{e}^{3x-1} \mathrm{\, d}x$ bằng
	\choice
	{\True $\dfrac{1}{3}(\mathrm{e}^5-\mathrm{e}^2)$}
	{$\dfrac{1}{3}\mathrm{e}^5-\mathrm{e}^2$}
	{$\mathrm{e}^5-\mathrm{e}^2$}
	{$\dfrac{1}{3}(\mathrm{e}^5+\mathrm{e}^2)$}
	\loigiai{
		Ta có $\displaystyle\int\limits_1^2 \mathrm{e}^{3x-1} \mathrm{\, d}x=\dfrac{1}{3}\mathrm{e}^{3x-1}\bigg|_1^2=\dfrac{1}{3}(\mathrm{e}^5-\mathrm{e}^2)$.}
\end{ex}
\begin{ex}%[MH2019, Nguyễn Kiều Nhã Tú]%[2D3B2-1]
Cho $\displaystyle\int\limits_{1}^{3}f(x)\mathrm{d}x=-5$, $\displaystyle\int\limits_{1}^{3}\left[f(x)-2g(x)\right]\mathrm{d}x=9$. Tính $I=\displaystyle\int\limits_{1}^{3}g(x)\mathrm{d}x$.
	\choice
	{ $I=14$}
	{$I=-14$}
	{$I=7$}
	{\True $I=-7$}
	\loigiai{Ta có
	\begin{eqnarray*}
		&& \displaystyle\int\limits_{1}^{3}\left[f(x)-2g(x)\right]\mathrm{d}x=9 \\ & \Leftrightarrow & \displaystyle\int\limits_{1}^{3}f(x)\mathrm{d}x-\displaystyle\int\limits_{1}^{3}2g(x)\mathrm{d}x=9\\ & \Leftrightarrow & \displaystyle\int\limits_{1}^{3}f(x)-2\displaystyle\int\limits_{1}^{3}g(x)\mathrm{d}x=9\\ & \Leftrightarrow & -5-2I=9\\ & \Leftrightarrow & I=-7.
	\end{eqnarray*}
	}
\end{ex}
\begin{ex}%[MH2019, Nguyễn Kiều Nhã Tú]%[2D3B2-1]
	Nếu $\displaystyle\int\limits_{1}^{2}f(x)\mathrm{d}x=2$ thì $I=\displaystyle\int\limits_{1}^{2}\left[3f(x)-2\right]\mathrm{d}x$ bằng bao nhiêu?
	\choice
	{ $I=2$}
	{$I=3$}
	{\True $I=4$}
	{$I=1$}
	\loigiai{
		Ta có $I=\displaystyle\int\limits_{1}^{2}\left[3f(x)-2\right]\mathrm{d}x=3\displaystyle\int\limits_{1}^{2}f(x)\mathrm{d}x-2\displaystyle\int\limits_{1}^{2}\mathrm{d}x=3\cdot 2- \left. {2x} \right|^2_1=6-2=4$.
	}
\end{ex}
\begin{ex}%[MH2019, Phan Thanh Tâm]%[2D3B2-1]
Cho $\displaystyle\int\limits_0^1 \dfrac{x\mathrm{\, d}x}{\left(x + 2\right)^2} = a + b\cdot\ln 2 + c\cdot\ln 3$ với $a,b,c$ là các số hữu tỷ.\\
Giá trị của $3a + b + c$ bằng
\choice
{$-2$}
{\True $-1$}
{$2$}
{$1$}
\loigiai{
Ta có $\displaystyle\int\limits_0^1 \dfrac{x\mathrm{\, d}x}{\left(x + 2\right)^2} =  \int\limits_0^1 \dfrac{(x+2)-2}{\left(x + 2\right)^2}\mathrm{\, d}x =  \int\limits_0^1 \dfrac{\mathrm{\, d}x}{x+2}- \int\limits_0^1 \dfrac{2\mathrm{\, d}x}{\left(x + 2\right)^2} $\\
$ = \left. \ln |x+2| \right|_0^1 + \left. \dfrac{2}{x+2}\right|_0^1 = \ln 3 - \ln 2 + \dfrac{2}{3} - 1 = - \dfrac{1}{3} - \ln 2 + \ln 3$.\\
Suy ra $a = - \dfrac{1}{3},b = - 1,c = 1 \Rightarrow 3a + b + c = - 1$.
}
\end{ex}
\begin{ex}%[Phát triển đề MH2019, Phan Thanh Tâm]%[2D3B2-1]
Biết $I= \displaystyle\int\limits_2^3 \dfrac{x^2-3x+2}{x^2-x+1}\mathrm{\, d}x =a\cdot\ln 7 + b\cdot\ln 3 + c$ với $a$, $b$, $c \in \mathbb{Z}$.\\
Tính $T = a + 2b^2 + 3c^3$.
\choice
{$ T=6 $}
{\True $ T=4 $}
{$ T=5 $}
{$ T=3 $}
\loigiai{
Ta có $ I = \displaystyle\int\limits_2^3 \left(1-\dfrac{2x-1}{x^2-x+1}\right)\mathrm{\, d}x=\left.\left(x-\ln |x^2-x+1|\right)\right|_2^3= - \ln 7 + \ln 3 + 1$.\\
Suy ra $ a=-1,b=1,c=1 $. Vậy $T = a + 2b^2 + 3c^3=4$.
}
\end{ex}

\begin{ex}%[Phát triển đề MH2019, Phan Thanh Tâm]%[2D3B2-1]
Biết $\displaystyle\int\limits_0^2 \dfrac{x^3 + 3x^2 + 2x + 1}{x^2 + 3x + 2}\mathrm{\, d}x = a + \ln \dfrac{b}{c}$ với $b$, $c$ là các số nguyên dương; $\dfrac{b}{c}$ là phân số tối giản. Tính $T = a + b + c$.
\choice
{$ T=7 $}
{$ T=9 $}
{$ T=5 $}
{$ T=11 $}
\loigiai{
Ta  có $\displaystyle\int\limits_0^2 \dfrac{x^3 + 3x^2 + 2x + 1}{x^2 + 3x + 2} \mathrm{\, d}x = \int\limits_0^2 \left( x + \dfrac{1}{x + 1} - \dfrac{1}{x + 2} \right)\mathrm{\, d}x $\\
$= \left. \left( \dfrac{x^2}{2} + \ln \left| \dfrac{x + 1}{x + 2} \right| \right) \right|_0^2 = 2 + \ln \dfrac{3}{2} $.\\
Suy ra $a = c = 2, b = 3$. Vậy $T= a + b + c = 7$.
}
\end{ex}
\begin{ex}%[Phát triển đề MH2019, Phan Thanh Tâm]%[2D3B2-1]
Biết $\displaystyle\int\limits_0^1 {\dfrac{{3x - 1}}{{x^2 + 6x + 9}}\mathrm{\, d}x} = 3\ln \dfrac{a}{b} - \dfrac{5}{6}$, trong đó $a$, $b$ nguyên dương và $\dfrac{a}{b}$ là phân số tối giản. Hãy tính $a\cdot b$.
\choice
{$ a\cdot b=-5 $}
{\True $ a\cdot b=12 $}
{$ a\cdot b=\dfrac{5}{4} $}
{$ a\cdot b=6 $}
\loigiai{
Ta có: $\displaystyle\int\limits_0^1 \dfrac{3x - 1}{x^2 + 6x + 9}\mathrm{\, d}x = \int\limits_0^1 \dfrac{3\left(x + 3\right) - 10}{\left( x + 3 \right)^2}\mathrm{\, d}x = \left.{\left( 3\ln |x + 3| + \dfrac{10}{x + 3}\right)}\right|_0^1$\\
$ = \left(3\ln 4 + \dfrac{5}{2}\right)- \left( 3\ln 3 + \dfrac{10}{3} \right) = 3\ln \dfrac{4}{3} - \dfrac{5}{6}$.\\
Suy ra $ \heva{&a=4\\&b=3} $. Vậy $ a\cdot b=12 $.
}
\end{ex}
\begin{ex}%[Đề thử nghiệm - THPT.QG 2017]%[2D3K2-1]
	Biết $I=\displaystyle\int\limits_{3}^{4}
	{\frac{\mathrm{\,d}x}{x^2+x}=a\ln 2+b\ln 3+c\ln 5,}$ với $a,b,c$ là các số nguyên. Tính $S=a+b+c.$
	\choice
	{$S=6$}
	{\True $S=2$}
	{$S=-2$}
	{$S=0$}
	\loigiai{
		Ta có $ f(x) = \dfrac{1}{x^2+x} = \dfrac{1}{x} - \dfrac{1}{x+1} \Rightarrow \displaystyle\int {f(x)\mathrm{\,d}x} = \ln |x| - \ln |x+1| + C$.\\
		Vậy $ I= (\ln |x| - \ln |x+1|) \big|_3^4 = 4\ln 2 - \ln 3 - \ln 5 $nên $ a=4, b = -1, c= -1 \Rightarrow S = 2.$
	}
\end{ex}
\begin{ex}%[Phát triển đề Lương Thế Vinh Hà Nội, lần 3, 2019; Nguyễn Tấn Linh]%[2D3B2-2]
	Cho $\displaystyle\int\limits_1^2 f(x^2+1)x\mathrm{\,d}x=2$. Tính  $I=\displaystyle\int\limits_2^5 f(x)\mathrm{\,d}x$.
	\choice
	{$I=2$}
	{$I=1$}
	{$I=-1$}
	{\True $I=4$}
	\loigiai
	{
		Đặt $t=x^2+1\Rightarrow\mathrm{\,d}t=2x\mathrm{\,d}x\Rightarrow x\mathrm{\,d}x=\dfrac{1}{2}\mathrm{\,d}t$.\\
		Khi đó $x=2\Rightarrow t=5$ và $x=1\Rightarrow t=2$.\\
		Vậy $\displaystyle\int\limits_1^2 f(x^2+1)x\mathrm{\,d}x=\dfrac{1}{2}\displaystyle\int\limits_2^5 f(t)\mathrm{\,d}t=2 \Leftrightarrow I=4$.
	}
\end{ex}
\begin{ex}%[Đề 101 - THPT.QG 2017]%[2D3B2-2]
	Cho $\displaystyle\int\limits_{0}^{6}f(x)\mathrm{\, d}x=12$. Tính $I=\displaystyle\int\limits_{0}^{2}f(3x)\mathrm{\, d}x$.
	\choice
	{$I=6$}
	{$I=36$}
	{$I=2$}
	{\True $I=4$}
	\loigiai{$I=\displaystyle\int\limits_{0}^{2}f(3x)\;\mathrm{d}x=\dfrac{1}{3}\displaystyle\int\limits_{0}^{2}f(3x)\;\mathrm{d}(3x)=\dfrac{1}{3}\displaystyle\int\limits_{0}^{6}f(u)\;\mathrm{d}u$ (với $u=3x$) \\
		$\Rightarrow I=\dfrac{1}{3}.12=4$.}
\end{ex}
\begin{ex}%[Đề minh họa - THPT.QG 2017]%[2D3B2-2]
	Tính tích phân $I=\displaystyle\int\limits_0^{\pi}\cos^3 x. \sin x \mathrm{\; d}x$.
	\choice
	{$I=-\dfrac{1}{4}\pi^4$}
	{$I=-\pi^4$}
	{\True $I=0$}
	{$I=-\dfrac{1}{4}$}
	\loigiai{
		Đặt $ u=\cos x \Rightarrow \mathrm{\; d}u=-\sin x\mathrm{\; d}x \Rightarrow \sin x\mathrm{\; d}x=-\mathrm{\; d}u$\\
		Đổi cận
		\begin{center}
			\begin{tabular}{c|c|c}
				$x$ & $ 0 $ & $ \pi $ \\ 
				\hline 
				$u$& $ 1 $ & $-1$ \\ 
			\end{tabular}
		\end{center}
		Nên $I=\displaystyle\int\limits_1^{-1}u^3. \left(-\mathrm{\; d}u\right)=\displaystyle\int\limits_{-1}^{1}u^3. \mathrm{\; d}u=\left.{\dfrac{1}{4}u^4}\right|_{-1}^1=0$
	}
\end{ex}
\begin{ex}%[Phát triển đề MH2019, Phan Thanh Tâm]%[2D3B2-1]
	Cho $ \displaystyle\int\limits_0^{\ln 2} \dfrac{\mathrm{\, d}x}{2\mathrm{e}^x+3} = a+\dfrac{b\cdot\ln 7+c\cdot\ln 10}{3} $ với $ a,\ b,\ c\in\mathbb{Z} $.\\
	Tính giá trị $ K=2a+3b+4c $.
	\choice
	{$ K=3 $}
	{$ K=7 $}
	{\True $ K=1 $iu}
	{$ K=-1 $}
	\loigiai{
		Đặt $t = 2\mathrm{e}^x + 3 \Rightarrow 2\mathrm{e}^x = t - 3 \Rightarrow 2\mathrm{e}^x\mathrm{\, d}x = dt$.\\
		Đổi cận $x = 0 \Rightarrow t = 5;x = \ln 2 \Rightarrow t = 7$.\\
		Khi đó $ \displaystyle\int\limits_0^{\ln 2} \dfrac{\mathrm{\, d}x}{2\mathrm{e}^x+3} = \int\limits_0^{\ln 2} \dfrac{2\mathrm{e}^x\mathrm{\, d}x}{2\mathrm{e}^x(2\mathrm{e}^x+3)}=\int\limits_3^7 \dfrac{\mathrm{\, d}t}{t(t-3)}=\dfrac{1}{3}\int\limits_3^7\left(\dfrac{1}{t-3}-\dfrac{1}{t}\right)\mathrm{\, d}t$\\
		$=\left.\dfrac{1}{3}\left[\ln |x-3|-\ln |t|\right]\right |_3^7= = \dfrac{1}{3}\left( \ln 4 - \ln 7 - \ln 2 + \ln 5 \right) = \dfrac{ - \ln 7 + \ln 10}{3}$.\\
		Do đó $\heva{&a = 0\\&b = - 1\\&c = 1}$. Vậy $ K=2a+3b+4c=1$.
	}
\end{ex}
\begin{ex}%[Đề tham khảo - THPT.QG 2017]%[2D3B2-2]
	Tính tích phân $I=\displaystyle\int_1^2 2x\sqrt{x^2-1}\mbox{d}x$ bằng cách đặt $u=x^2-1$, mệnh đề nào dưới đây đúng?
	\choice
	{$I=2\displaystyle\int_0^3 \sqrt{u}\mbox{d}u$}
	{$I=\displaystyle\int_1^2 \sqrt{u}\mbox{d}u$}
	{\True $I=\displaystyle\int_0^3 \sqrt{u}\mbox{d}u$}
	{$I=\dfrac{1}{2}\displaystyle\int_1^2 \sqrt{u}\mbox{d}u$}
	\loigiai{
		Đặt $u=x^2-1 \Rightarrow \mathrm{d}u=2x \mathrm{d}x$. Đổi cận $x=1 \Rightarrow u=0$; $x=2 \Rightarrow u=3$. \\
		Do đó: $I=\displaystyle\int\limits_1^2 2x\sqrt{x^2-1} \mathrm{d}x=\displaystyle\int\limits_0^3 \sqrt{u} \mathrm{d}u$.
	}
\end{ex}

\begin{ex}%[Phát triển đề Lương Thế Vinh, lần 3, 2019; MyNguyen]%[2D3B2-3]
	Biết rằng $\displaystyle\int\limits_{1}^{m} x^2\ln x \mathrm{\, d}x= \dfrac{1}{9}$, với $m$ là số thực. Khẳng định nào dưới đây đúng?
	\choice
	{$m \in (0;1)$}
	{\True $m\in (1;2)$}
	{$m \in (2;3)$}
	{$m \in (-1;0)$}
	\loigiai{ Điều kiện $m>0$.\\
		Đặt $\heva{& u=\ln x \\ & \mathrm{\, d}v = x^2 \mathrm{\, d}x} \Rightarrow \heva{& \mathrm{d}u = \dfrac{1}{x} \mathrm{d}x \\ & v= \dfrac{1}{3}x^3}$. Khi đó
		$$\displaystyle\int\limits_{1}^{m} x^2\ln x \mathrm{\, d}x= \dfrac{1}{3}x^3\ln x\Big|_1^{m} -\dfrac{1}{3}\displaystyle\int\limits_1^{m} x^2 \mathrm{\, d}x = \dfrac{1}{3}m^3\ln m -\dfrac{1}{9}m^3+\dfrac{1}{9}.$$
		Từ đó suy ra $$\dfrac{1}{3}m^3\ln m -\dfrac{1}{9}m^3+\dfrac{1}{9} = \dfrac{1}{9} \Leftrightarrow \hoac{& m=0 \\ & m= \sqrt[3]{\mathrm{e}}}.$$
		So với điều kiện suy ra $m= \sqrt[3]{\mathrm{e}} \in (1;2)$.
	}	
\end{ex}
\begin{ex}%[Đề minh họa - THPT.QG 2017]%[2D3B2-3]
	Tính tích phân $I=\displaystyle\int\limits_1^e x\ln x \mathrm{\; d}x$
	\choice
	{$I=\dfrac{1}{2}$}
	{$I=\dfrac{e^2-2}{2}$}
	{\True $I=\dfrac{e^2+1}{4}$}
	{$I=\dfrac{e^2-1}{4}$}
	\loigiai{
		Đặt $ \heva{&u=\ln x\\& \mathrm{\;d}v=x\mathrm{\; d}x} \Rightarrow \heva{&\mathrm{\; d}u=\dfrac{1}{x}\mathrm{\; d}x\\& v=\dfrac{1}{2}x^2}$, ta có:
		$$ I=\left.{\dfrac{1}{2}x^2\ln x}\right|_1^e - \displaystyle\int\limits_1^e \dfrac{1}{2}x \mathrm{\; d}x=\left.{\dfrac{1}{2}x^2\ln x}\right|_1^e - \left.{\dfrac{1}{4}x^2}\right|_1^e=\dfrac{1}{2}e^2 - \left(\dfrac{1}{4}e^2-\dfrac{1}{4}\right)= \dfrac{e^2+1}{4}.$$
	}
\end{ex}
\begin{ex}%[Đề tham khảo - THPT.QG 2017]%[2D3B2-3]
	Cho hàm số $f(x)$ thỏa mãn $\displaystyle\int\limits_0^1 (x+1)f'(x) \mathrm{d}x=10$ và $2f(1)-f(0)=2$. Tính $\displaystyle\int\limits_0^1 f(x) \mathrm{d}x$.
	\choice
	{$I=-12$}
	{$I=8$}
	{$m=1$}
	{\True $I=-8$}
	\loigiai{
		Đặt $\left\{\begin{aligned}
		&u=x+1 \\
		& \mathrm{d}v=f'(x) \mathrm{d}x
		\end{aligned}\right. \Rightarrow \left\{\begin{aligned}
		& \mathrm{d}u= \mathrm{d}x \\
		&v=f(x)
		\end{aligned}\right. $. Khi đó $I=(x+1)f(x)\bigg|_0^1-\displaystyle\int\limits_0^1 f(x)\mathrm{d}x.$ \\
		Suy ra $10=2f(1)-f(0)-\displaystyle\int\limits_0^1 f(x) \mathrm{d}x \Rightarrow \displaystyle\int\limits_0^1 f(x) \mathrm{d}x=-10+2=-8.$ \\
		Vậy $\displaystyle\int\limits_0^1 f(x) \mathrm{d}x=-8.$}
\end{ex}
\begin{ex}%[Đề chính thức THPTQG 2019, mã 101]%[Trần Bá Huy, dự án 2-CT-2019]%[2D3K2-1]
	Cho hàm số $f(x)$. Biết $f(0)=4$ và $f'(x)=2\cos^2x+1,\ \forall x\in\mathbb{R}$, khi đó $\displaystyle\int\limits_{0}^{\frac{\pi}{4}}f(x)\mathrm{d}x$ bằng
	\choice
	{$\dfrac{\pi^2+4}{16}$}
	{$\dfrac{\pi^2+14\pi}{16}$}
	{\True $\dfrac{\pi^2+16\pi+4}{16}$}
	{$\dfrac{\pi^2+16\pi+16}{16}$}
	\loigiai{
		Ta có $f'(x)=2\cos^2x+1=\cos2x+2,\ \forall x\in\mathbb{R}$. Suy ra
		\[f(x)=\displaystyle\int f'(x)\mathrm{d}x=\displaystyle\int (\cos2x+2)\mathrm{d}x=\dfrac{1}{2}\sin2x+2x+C.\]
		Từ $f(0)=4$, suy ra $C=4$. Vậy
		\[\displaystyle\int\limits_{0}^{\frac{\pi}{4}}f(x)\mathrm{d}x=\displaystyle\int\limits_{0}^{\frac{\pi}{4}}\left[\dfrac{1}{2}\sin 2x+2x+4\right]\mathrm{d}x=\left(-\dfrac{1}{4}\cos2x+x^2+4x\right)\Bigg|_{0}^{\frac{\pi}{4}}=\dfrac{\pi^2}{16}+\pi+\dfrac{1}{4}.\]
	}
\end{ex}

\begin{ex}%[Đề tham khảo - THPT.QG 2018]%[2D3K2-1]
	Biết $I=\displaystyle\int\limits_1^2 \dfrac{\mathrm{\,d}x}{(x+1)\sqrt{x}+x\sqrt{x+1}}=\sqrt{a}-\sqrt{b}-c$ với $a$, $b$, $c$ là các số nguyên dương. Tính $P=a+b+c$.
	\choice
	{$P=24$}
	{$P=12$}
	{$P=18$}
	{\True $P=46$}
	\loigiai{
		Ta có $\sqrt{x+1}-\sqrt{x} e 0$, $\forall x\in [1;2]$ nên \\
		$I=\displaystyle\int\limits_1^2 \dfrac{\mathrm{\,d}x}{(x+1)\sqrt{x}+x\sqrt{x+1}}=\displaystyle\int\limits_1^2 \dfrac{\mathrm{\,d}x}{\sqrt{x(x+1)}\left(\sqrt{x+1}+\sqrt{x}\right)}$ \qquad\textbf{(*)}\\
		\centerline{$=\displaystyle\int\limits_1^2 \dfrac{\left(\sqrt{x+1}-\sqrt{x}\right) \mathrm{d}x}{\sqrt{x(x+1)}\left(x+1-x\right)}=\displaystyle\int\limits_1^2 \dfrac{\left(\sqrt{x+1}-\sqrt{x}\right) \mathrm{d}x}{\sqrt{x(x+1)}}$}
		\centerline{$=\displaystyle\int\limits_1^2 \left(\dfrac{1}{\sqrt{x}}-\dfrac{1}{\sqrt{x+1}}\right) \mathrm{d}x=\left(2\sqrt{x}-2\sqrt{x+1}\right)\bigg|_1^2=\sqrt{32}-\sqrt{12}-2$.}
		Mà $I=\sqrt{a}-\sqrt{b}-c$ nên $\left\{\begin{aligned}
		&a=32 \\
		&b=12 \\
		&c=2
		\end{aligned}\right. $. Suy ra $P=a+b+c=32+12+2=46$.\\
		\textbf{\underline{Lưu ý}:} giải đến bước (*), ta có thể đổi biến với $t=\sqrt{x+1}+\sqrt{x}$.}
\end{ex}
\begin{ex}%[Đề tham khảo - THPT.QG 2017]%[2D3K2-2]
	Cho $\displaystyle\int\limits_0^1 \dfrac{1}{\mathrm{e}^x + 1}\mathrm{d}x = a + b\ln \dfrac{1+\mathrm{e}}{2}$, với $a,b$ là các số hữu tỉ. Tính $S=a^3+b^3$.
	\choice 
	{$S=2$}
	{$S=-2$}
	{\True $S=0$}
	{$S=1$}
	\loigiai{$\displaystyle\int\limits_0^1 \dfrac{\mathrm{d}x}{\mathrm{e}^x+1}=\displaystyle\int\limits_0^1 \dfrac{(\mathrm{e}^x+1)-\mathrm{e}^x}{\mathrm{e}^x+1} \mathrm{d}x=\displaystyle\int\limits_0^1 \mathrm{d}x-\displaystyle\int\limits_0^1 \dfrac{\mathrm{d}(\mathrm{e}^x+1)}{\mathrm{e}^x+1}=x\bigg|_0^1-\ln |\mathrm{e}^x+1|\bigg|_0^1=1-\ln \dfrac{1+e}{2}$. \\
		$\Rightarrow \left\{\begin{aligned}
		&a=1 \\
		&b=-1
		\end{aligned}\right. \Rightarrow S=a^3+b^3=0$. }
\end{ex}

\begin{ex}%[Đề 101, THPT.QG - 2018]%[Nguyễn Thành Sơn, dự án 12EX11]%[2D3K2-2]
	Cho $\displaystyle\int\limits_{16}^{55} \dfrac{\mathrm{\,d}x}{x\sqrt{x+9}}=a\ln 2+b\ln 5+c\ln 11$ với $a,\, b,\, c$ là các số hữu tỉ. Mệnh đề nào dưới đây đúng?
	\choice
	{\True $a-b=-c$}
	{$a+b=c$}
	{$a+b=3c$}
	{$a-b=-3c$}
	\loigiai{
		Đặt $t=\sqrt{x+9} \Rightarrow t^2=x+9 \Rightarrow 2t \mathrm{\,d}t= \mathrm{\,d}x$. \\
		Đổi cận: $x=16 \Rightarrow t=5\;;\; x=55 \Rightarrow t=8$. \\
		$\begin{aligned}\text{Ta có } \displaystyle\int\limits_{16}^{55} \dfrac{\mathrm{d}x}{x\sqrt{x+9}}=\displaystyle\int\limits_5^8 \dfrac{2t \mathrm{\,d}t}{(t^2-9)t}&=2\displaystyle\int\limits_5^8 \dfrac{\mathrm{\,d}t}{t^2-9}=\dfrac{1}{3}\left(\displaystyle\int\limits_5^8 \dfrac{\mathrm{\,d}t}{t-3}-\displaystyle\int\limits_5^8 \dfrac{\mathrm{\,d}t}{t+3}\right)\\
		&=\dfrac{1}{3}\left(\ln \left|x-3\right|-\ln \left|x+3\right|\right)\bigg|_5^8=\dfrac{2}{3}\ln 2+\dfrac{1}{3}\ln 5-\dfrac{1}{3}\ln 11.
		\end{aligned}$. \\
		Vậy $a=\dfrac{2}{3}$, $b=\dfrac{1}{3}$, $c=-\dfrac{1}{3}$. Mệnh đề $a-b=-c$ đúng.}
\end{ex}

\begin{ex}%[12-EX-ĐHVinh-L3]%[Nguyễn Hoàng Thanh]%[2D3G2-2]
	Biết $\displaystyle\int\limits_{\tfrac{\pi}{4}}^{\tfrac{\pi}{3}}\dfrac{\cos ^2x+\sin x\cos x+1}{\cos^2 x+\sin x\cos x}   dx=a\dfrac{\pi}{12}+b\ln 2+c\ln \left(1+\sqrt{3}\right)$, với $a,b,c$ là các số hữu tỉ. Giá trị của $abc$ bằng
	\choice
	{\True$-1$}
	{$1$}
	{ $-2$}
	{$2$}
	%\choice
	%{Phương án sai 1}
	%{Phương án sai 2}
	%{\True Phương án đúng}
	%{Phương án sai 3}
	\loigiai{
		Ta có:
		\begin{align*}
		\int\limits_{\tfrac{\pi}{4}}^{\tfrac{\pi}{3}}\dfrac{\cos^2x+\sin x\cos x+1}{\cos^2 x+\sin x\cos x}   dx
		%&=\int\limits_{\tfrac{\pi}{4}}^{\tfrac{\pi}{3}} \dfrac{\cos^2x+\sin x\cos x+1}{\cos^4x+\sin x\cos^3x}dx\\
		& =\int\limits_{\tfrac{\pi}{4}}^{\tfrac{\pi}{3}} \dfrac{1+\tan x+(1+\tan ^2 x)}{1+\tan x}dx\\
		%&=\int\limits_{\tfrac{\pi}{4}}^{\tfrac{\pi}{3}} \dfrac{\left(1+\tan^2x\right)+\tan x\left(1+\tan^2x\right)+\left(1+\tan^2x\right)^2}{1+\tan x}dx\\
		%&=\int\limits_{\tfrac{\pi}{4}}^{\tfrac{\pi}{3}} \dfrac{1+\tan x+\left(1+\tan^2x\right)}{1+\tan x}\left(1+\tan^2x\right)dx\\
		&=\int\limits_{\tfrac{\pi}{4}}^{\tfrac{\pi}{3}} \left(1+\dfrac{1+\tan^2x}{1+\tan x}\right)dx\\
		&= \dfrac{\pi}{12}+\int\limits_{\tfrac{\pi}{4}}^{\tfrac{\pi}{3}} \dfrac{1+\tan^2x}{1+\tan x}dx
		\end{align*} 
		Đặt $t=1+\tan x$ ta được $dt=\left(1+\tan^2x\right)dx$, đổi cận $x=\dfrac{\pi}{4} \Rightarrow t=2,\ x=\dfrac{\pi}{3} \Rightarrow t=1+\sqrt{3}.$ Ta được,
		\begin{align*}
		\int\limits_{\tfrac{\pi}{4}}^{\tfrac{\pi}{3}} \dfrac{1+\tan^2x}{1+\tan x}dx
		&=\int\limits_2^{1+\sqrt{3}}\dfrac{1}{t}dt=\ln t \bigg|_2^{1+\sqrt{3}}=\ln (1+\sqrt{3})-\ln 2
		\end{align*}
		Từ đây ta suy ra $a\dfrac{\pi}{12}+b\ln 2+c\ln \left(1+\sqrt{3}\right)=\dfrac{\pi}{12}-\ln 2+\ln \left(1+\sqrt{3}\right)$.\\
		Do đó $a=1,b=-1,c=1$ suy ra $abc=-1$.
	}
\end{ex}
\begin{ex}%[Đề tham khảo - THPT.QG 2017]%[2D3K2-2]
	Cho hàm số $f(x)$ liên tục trên $\mathbb{R}$ và thỏa mãn $f(x)+f(-x)=\sqrt{2+2\cos 2x},\forall x\in \mathbb{R}$. Tính $I=\displaystyle\int\limits_{-\tfrac{3\pi}{2}}^{\tfrac{3\pi}{2}} f(x)\mathrm{d}x$.
	\choice
	{$I=-6$}
	{$I=0$}
	{$I=-2$}
	{\True $I=6$}
	\loigiai{
		\textbf{Cách 1.} Tự luận. \\
		Đặt $t=-x \Rightarrow \mathrm{d}t=- \mathrm{d}x$. \\
		Đổi cận $x=-\dfrac{3\pi}{2} \Rightarrow t=\dfrac{3\pi}{2}$; $x=\dfrac{3\pi}{2} \Rightarrow t=-\dfrac{3\pi}{2}$. Suy ra $I=\displaystyle\int\limits_{-\dfrac{3\pi}{2}}^{\tfrac{3\pi}{2}} f(-t) \mathrm{d}t$. \\
		Mặt khác $f(t)+f(-t)=\sqrt{2+2\cos 2t}=\sqrt{4\cos^2t}=2\left|\cos t\right|$ (thay $x=t$). \\
		Ta có $2I=\displaystyle\int\limits_{-\tfrac{3\pi}{2}}^{\tfrac{3\pi}{2}} \left[f(t)+f(-t)\right]  \mathrm{d}t=\displaystyle\int\limits_{-\tfrac{3\pi}{2}}^{\tfrac{3\pi}{2}} 2\left|\cos t\right|  \mathrm{d}t.$ \\
		Suy ra $I=\displaystyle\int\limits_{-\tfrac{3\pi}{2}}^{\tfrac{3\pi}{2}} \left|\cos t\right| \mathrm{d}t.$ \\
		$I=\displaystyle\int\limits_{-\tfrac{3\pi}{2}}^{\tfrac{3\pi}{2}} \left|\cos t\right|  \mathrm{d}t=2\displaystyle\int\limits_0^{\tfrac{3\pi}{2}} \left|\cos t\right|  \mathrm{d}t$. \bigg(Do $\left|\cos t\right|$ là hàm số chẵn trên đoạn $\left[-\dfrac{3\pi}{2};\dfrac{3\pi}{2}\right]$ \bigg) \\
		$=2\displaystyle\int\limits_0^{\tfrac{\pi}{2}} \left|\cos t\right|  \mathrm{d}t+2\displaystyle\int\limits_{\tfrac{\pi}{2}}^{\tfrac{3\pi}{2}} \left|\cos t\right|  \mathrm{d}t=2\displaystyle\int\limits_0^{\tfrac{\pi}{2}} \cos t \mathrm{d}t-2\displaystyle\int\limits_{\tfrac{\pi}{2}}^{\tfrac{3\pi}{2}} \cos t \mathrm{d}t=2\sin t\bigg|_0^{\tfrac{\pi}{2}}-2\sin t\bigg|_{\tfrac{\pi}{2}}^{\tfrac{3\pi}{2}}=6$. \\
		\textbf{Cách 2.} Trắc nghiệm. \\
		Ta có: $f(x)+f(-x)=2\left|\cos x\right| \Leftrightarrow f(x)+f(-x)=\left|\cos x\right|+\left|\cos (-x)\right|$ \\
		nên ta có thể chọn $f(x)=\left|\cos x\right|$. \\
		Suy ra $I=\displaystyle\int\limits_{-\tfrac{3\pi}{2}}^{\tfrac{3\pi}{2}} \left|\cos x\right|  \mathrm{d}x=6$ (bấm máy).}
\end{ex}
\begin{ex}%[Chuyên Vinh Lần 3]%[Phan Anh]%[2D3G2-4]
	Cho hàm số $f(x)$ có đạo hàm liên tục trên $\mathbb{R}$ và thỏa mãn $f(0)=5$ và $f(x)+f(3-x)=x^2-3x+2,\,\forall x \in \mathbb{R}$. Tích phân $\displaystyle\int\limits_0^3 xf'(x)\mathrm{\,d}x$ bằng
	\choice
	{\True $-\dfrac{39}{4}$}
	{$\dfrac{3}{4}$}
	{$\dfrac{3}{2}$}
	{$-\dfrac{15}{4}$}
	\loigiai{
		Thay $x=0$ ta được $f(0)+f(3)=2 \Rightarrow f(3)=2-f(0)=2-5=-3$.\\
		Ta có $\displaystyle\int\limits_0^3 f(x)\mathrm{\,d}x=\displaystyle\int\limits_0^3 f(3-x)\mathrm{\,d}x$.\\
		Từ hệ thức đề ra $\displaystyle\int\limits_0^3 (f(x)+f(3-x))\mathrm{\,d}x=\displaystyle\int\limits_0^3 (x^2-3x+2)\mathrm{\,d}x=\dfrac{3}{2} \Rightarrow \displaystyle\int\limits_0^3f(x)\mathrm{\,d}x=\dfrac{3}{4}$. \\
		Áp dụng công thức tích phân từng phần, ta lại có
		$$\displaystyle\int\limits_0^3 xf'(x)\mathrm{\,d}x=xf(x)\bigg|_0^3-\displaystyle\int\limits_0^3f(x)\mathrm{\,d}x=3f(3)-\dfrac{3}{4}=-\dfrac{39}{4}.$$}
\end{ex}
\begin{ex}%[Chuyên Vinh Lần 3]%[Phan Anh]%[2D3G2-4]
	Cho hàm số $f(x)$ có đạo hàm liên tục trên $\mathbb{R}$ và thỏa mãn $f(0)=1$ và $f(x)+f(a-x)=x^2-ax+2,\,\forall x \in \mathbb{R}$. Có bao nhiêu số nguyên dương $a$ để tích phân $\displaystyle\int\limits_0^a xf'(x)\mathrm{\,d}x$ không vượt quá $\dfrac{16}{3}$?
	\choice
	{$3$}
	{$5$}
	{\True $4$}
	{$6$}
	\loigiai{
		Thay $x=0$ ta được $f(0)+f(a)=2 \Rightarrow f(a)=2-f(0)=2-1=1$ \\
		Ta có: $\displaystyle\int\limits_0^a f(x)\mathrm{\,d}x=\displaystyle\int\limits_0^a f(a-x)\mathrm{\,d}x$ \\
		Từ hệ thức đề ra: $\displaystyle\int\limits_0^a (f(x)+f(a-x))\mathrm{\,d}x=\displaystyle\int\limits_0^a (x^2-ax+2)\mathrm{\,d}x=2a-\dfrac{a^3}{6} \Rightarrow \displaystyle\int\limits_0^a f(x)\mathrm{\,d}x=a-\dfrac{a^3}{12}$. \\
		Áp dụng công thức tích phân từng phần, ta lại có
		$$\displaystyle\int\limits_0^a xf'(x)\mathrm{\,d}x=xf(x)\bigg|_0^a-\displaystyle\int\limits_0^a f(x)\mathrm{\,d}x=af(a)-\left(a-\dfrac{a^3}{12}\right)=\dfrac{a^3}{12}.$$
		Yêu cầu bài toán là $\dfrac{a^3}{12}\le\dfrac{16}{3}\Leftrightarrow a^3\le64\Leftrightarrow a\le4$.\\
		Vậy có $4$ số nguyên dương $a$ thỏa bài toán}
\end{ex}
\begin{ex}%[Đề chính thức THPTQG 2019, Mã đề 101]%[Huỳnh Xuân Tín, THPTQG2019]%[2D3K2-4]
	Cho hàm số $f(x)$ có đạo hàm liên tục trên $\mathbb{R}$. Biết $f(4)=1$ và $\displaystyle\int\limits_0^1xf(4x)\mathrm{\,d}x=1$. Khi đó $\displaystyle\int\limits_0^4x^2f'(x)\mathrm{\,d}x$ bằng
	\choice
	{$\dfrac{31}{2}$}
	{\True $-16$}
	{$8$}
	{$14$}
	\loigiai{Đặt $t=4x\Rightarrow \mathrm{\,d}t=4\mathrm{\,d}x$. Khi đó
		\[1=\displaystyle\int\limits_0^1xf(4x)\mathrm{\,d}x=\displaystyle\int\limits_0^4\dfrac{t\cdot f(t)}{16}\mathrm{\,d}t\Rightarrow \displaystyle\int\limits_0^4xf(x)\mathrm{\,d}x=16.\]
		Xét tích phân $I=\displaystyle\int\limits_0^4x^2f'(x)\mathrm{\,d}x$.
		Ta có $\heva{&u=x^2\\&\mathrm{\,d}u=f'(x)\mathrm{\,d} x}\Rightarrow \heva{&\mathrm{\,d}u=2x\mathrm{\,d}x\\&v=f(x)}$. Khi đó
		\[I=x^2\cdot f(x)\Big|_0^4-\displaystyle\int\limits_0^42x\cdot f(x)\mathrm{\,d}x=4^2\cdot f(4)-2\cdot16=-16.\]	
	}
\end{ex}
\Closesolutionfile{ans}