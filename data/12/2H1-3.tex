\Opensolutionfile{ans}[ans/ansCD2H1-3]
\section{THỂ TÍCH CỦA KHỐI ĐA DIỆN}
\subsection{Kiến thức sách giáo khoa cần cần nắm}
\subsubsection{THỂ TÍCH CỦA KHỐI HỘP CHỮ NHẬT}
\begin{dl}
Thể tích của một khối hình hộp chữ nhật bằng tích số của ba kích thước.
\end{dl}
Như vậy:\\
-Với khối hộp chữ nhật có ba kích thước là $a,b,c$ thì $V=a\cdot b\cdot c$.\\{\color{red} 
	\begin{center}
	\begin{tikzpicture}[scale=0.6]
	\coordinate (A) at (-1,2);
	\coordinate (B) at (4,2);
	\coordinate (C) at (2,0);
	\coordinate (D) at ($(A)+(C)-(B)$);
	\coordinate (E) at (-3,-3);
	\coordinate (H) at ($(A)+(E)-(D)$);
	\coordinate (F) at ($(C)+(E)-(D)$);
	\coordinate (G) at ($(B)+(F)-(C)$);
	\draw[line width=0.4pt,black] (A)--(B) (B)--(C) (A)--(D) (D)--(C) (B)--(G) (D)--(E);
	\draw[line width=0.4pt,black,dashed] (A)--(H) (H)--(G) (H)--(E);
	\draw[line width=0.4pt,black] (E)--(F)node[below,pos=0.5]{$a$};
	\draw[line width=0.4pt,black] (G)--(F)node[below,pos=0.5]{$b$};
	\draw[line width=0.4pt,black] (C)--(F)node[left,pos=0.5]{$c$}; 
	\foreach \diem in {A,B,C,D,E,G,F,H} \fill (\diem)circle(1.5pt);
	\end{tikzpicture}	
\end{center}}
-Khối lập phương có cạnh bằng $a$ thì $V=a^3$. 
\begin{center}
	\begin{tikzpicture}[scale=0.6, font=\footnotesize, line join=round, line cap=round, >=stealth]
	\def\bc{4} % cạnh BC
	\def\ba{2} % cạnh BA
	\def\gocB{35} % góc B của đáy
	\coordinate(B) at (0,0);
	\coordinate(A) at (\gocB:\ba);
	\coordinate(C) at (\bc,0);
	\coordinate(D) at ($(C)-(B)+(A)$);
	\coordinate(A') at ($(A)+(90:\bc)$);
	\coordinate(B') at ($(B)-(A)+(A')$);
	\coordinate(C') at ($(C)-(A)+(A')$);
	\coordinate(D') at ($(D)-(A)+(A')$);
	\draw (B')--(B)--(C)--(D)--(D')--(A')--(B')--(C')--(D') (C)--(C');
	\draw[dashed] (A')--(A)--(D) (A)--(B);
	\draw[line width=0.4pt,black] (C)--(C')node[left,pos=0.5]{$a$};
	\draw[line width=0.4pt,black] (C)--(D)node[below,pos=0.5]{$a$};
	\draw[line width=0.4pt,black] (B)--(C)node[below,pos=0.5]{$a$}; 
	\foreach \diem in {A,B,C,D,A',C',B',D'} \fill (\diem)circle(1.5pt);
	\end{tikzpicture}
\end{center}
\subsubsection{THỂ TÍCH CỦA KHỐI LĂNG TRỤ}
\begin{dl}
Thể tích của một khối lăng trụ bằng tích của diện tích đáy và chiều cao.
\end{dl}
Như vậy: Với khối lăng trụ có diện tích đáy bằng $S$ và chiều cao bằng $h$ ta có: $V=S\cdot h$.
\begin{center}
\begin{tabular}{c c }
 \begin{tikzpicture}[scale=0.8, font=\footnotesize, line join=round, line cap=round, >=stealth]
	\def\ac{4} % cạnh AC
	\def\ab{2} % cạnh AB
	\def\h{4} % chiều cao
	\def\gocA{50} % góc A của đáy
	\coordinate(A) at (0,0);
	\coordinate(C) at (\ac,0);
	\coordinate(B) at (-\gocA:\ab);
	\coordinate(A') at ($(A)+(90:\h)$);
	\coordinate(B') at ($(B)-(A)+(A')$);
	\coordinate(C') at ($(C)-(A)+(A')$);
	\coordinate (M) at ($(C)!1/2!(B)$); 
	\coordinate[label=right:$S$](G) at ($(A)!2/3!(M)$); 
	\draw (A')--(A) (A)--(B) (B)--(C) (C')--(B') (B')--(C') (B)--(B') (A')--(B') (A')--(C');
	\draw[dashed] (A)--(C);
	\draw[line width=0.4pt,black] (C)--(C')node[right,pos=0.5]{$h$};
	\foreach \diem in {A,B,C,A',B',C'} \fill (\diem)circle(1.5pt);
	\end{tikzpicture}
	&
\begin{tikzpicture}[scale=0.8, font=\footnotesize, line join=round, line cap=round, >=stealth]
\def\ac{4} % cạnh AC
\def\ab{2} % cạnh AB
\def\ben{4} % cạnh bên
\def\gocnghieng{65} % góc nghiêng cạnh bên
\def\gocA{50} % góc A của đáy
\coordinate(A) at (0,0);
\coordinate(C) at (\ac,0);
\coordinate(B) at (-\gocA:\ab);
\coordinate(A') at ($(A)+(\gocnghieng:\ben)$);
\coordinate(B') at ($(B)-(A)+(A')$);
\coordinate(C') at ($(C)-(A)+(A')$);
\coordinate(M) at ($(C)!1/2!(B)$); 
\coordinate[label=right:$S$](G) at ($(A)!2/3!(M)$); 
\path ($(A)!(C')!(C)$) coordinate (H);
\draw (A')--(A)--(B)--(C)--(C')--(A')--(B')--(C') (B)--(B') ;
\draw[dashed] (A)--(C);
\draw[line width=0.4pt,black] (H)--(C')node[right,pos=0.5]{$h$};
\foreach \diem in {A,B,C,A',B',C'} \fill (\diem)circle(1.5pt);
\end{tikzpicture}
\end{tabular}
\end{center}
\subsubsection{THỂ TÍCH CỦA KHỐI CHÓP}
\begin{dl}
Thể tích của một khối chóp bằng $\dfrac{1}{3}$ tích của diện tích đáy và chiều cao.
\end{dl}
Như vậy: Với khối chóp có diện tích đáy bằng $S$ và chiều cao bằng $h$ ta có: $V=\dfrac{1}{3}S\cdot h$ 
\begin{center}
	\begin{tikzpicture}[scale=1, font=\footnotesize, line join=round, line cap=round, >=stealth]
	\def\ad{4.5} % cạnh AD
	\def\ab{2} % cạnh AB
	\def\bc{1.5} % chéo AC
	\def\as{4} % cạnh AS
	\def\gocA{50} % góc A của đáy
	\def\gocB{120} % góc B của đáy
	\coordinate[label=left:$A$] (A) at (0,0);
	\coordinate[label=below left:$B$] (B) at (-\gocA:\ab);
	\coordinate[label=below right:$C$] (C) at ($(B)+(180-\gocA-\gocB:\bc)$);
	\coordinate[label=right:$D$] (D) at (\ad,0);
	\coordinate[label=above:$S$] (S) at (75:\as); 
	\coordinate[label=above:$S$] (M) at ($(A)!2/3!(C)$); 
	\coordinate(H) at ($(A)!1/3!(C)$); 
	\draw (A)--(B)--(C)--(D)--(S)--cycle (B)--(S)--(C);
	\draw[dashed] (A)--(D);
	\draw[line width=0.4pt,black,dashed] (S)--(H)node[left,pos=0.5]{$h$}; % Đoạn thẳng AB nhãn là a
	\foreach \diem in {A,B,C,D,S,H}	\fill (\diem)circle(1.5pt);
	\end{tikzpicture}
\end{center}
\subsubsection{TỈ SỐ THỂ TÍCH TỨ DIỆN}
Cho khối tứ diện $SABC$ và $A'$, $B'$, $C'$ là các điểm tùy ý lần lượt thuộc $SA$, $SB$, $SC$ ta có:\\
$\dfrac{V_{SA'B'C'}}{V_{SABC}}=\dfrac{SA'}{SA}\cdot\dfrac{SB'}{SB}\cdot\dfrac{SC'}{SC}$  
\begin{center}
	\begin{tikzpicture}[scale=1, font=\footnotesize, line join=round, line cap=round, >=stealth]
	\coordinate[label=left:$A$] (A) at (-2,0);
	\coordinate[label=right:$C$] (C) at (3,0);
	\coordinate[label=below left:$B$] (B) at (0,-2);
	\coordinate[label=above:$S$] (S) at (0,4);
	\coordinate[label=left:$A'$] (A') at ($(S)!1/2!(A)$);
	\coordinate[label=below right:$B'$] (B') at ($(S)!1/2!(B)$);
	\coordinate[label=right:$C'$] (C') at ($(S)!1/2!(C)$);
	\draw (A)--(B)--(C)--(S)--cycle (S)--(B) (A')--(B') (B')--(C');
	\draw[dashed] (A)--(C) (A')--(C');
	\foreach \diem in {A,B,C,S,A',B',C'}\fill (\diem)circle(1.5pt);
	\end{tikzpicture}
\end{center}
\subsection{Phân loại và phương pháp giải bài tập}
\begin{dang}{Khối lăng trụ đứng có chiều cao hay cạnh đáy.}
\end{dang}
\subsubsection{Các ví dụ}
\begin{vd}%Ví dụ 1%[Nguyễn Diệu Linh]%[2H1B3-3]
	Đáy của lăng trụ đứng tam giác $ABC.A'B'C'$ là tam giác vuông cân tại $A$, có cạnh $BC=a\sqrt{2}$, biết $A'B=3a$. Tính thể tích của khối lăng trụ.
	\loigiai{
	\immini{	Ta có $\triangle ABC$ vuông cân tại $A$ nên $AB=AC=a$.\\
		$ABC.A'B'C'$ là lăng trụ đứng $\Rightarrow AA'\perp AB$.\\
		Xét tam giác vuông $\triangle ABA'$, ta có \\$\Rightarrow AA'^2=A'B^2-AB^2=8a^2$\\$\Rightarrow AA'=2a\sqrt{2}$.\\
		Vậy $V=S\cdot h=a^3\sqrt{2}$.}{
	\begin{tikzpicture}[scale=1, font=\footnotesize, line join=round, line cap=round, >=stealth]
	\def\ac{4} % cạnh AC
	\def\ab{2} % cạnh AB
	\def\h{3} % chiều cao
	\def\gocA{50} % góc A của đáy
	\coordinate[label=left:$A$] (A) at (0,0);
	\coordinate[label=right:$C$] (C) at (\ac,0);
	\coordinate[label=below left:$B$] (B) at (-\gocA:\ab);
	\coordinate[label=left:$A'$] (A') at ($(A)+(90:\h)$);
	\coordinate[label=below left:$B'$] (B') at ($(B)-(A)+(A')$);
	\coordinate[label=right:$C'$] (C') at ($(C)-(A)+(A')$);
	\draw (A')--(A)--(B)--(C)--(C')--(A')--(B')--(C') (B)--(B') (A')--(B);
	\draw[dashed] (A)--(C);
	\foreach \diem in {A,B,C,A',B',C'} \fill (\diem)circle(1.5pt);
	\draw pic[draw,blue,angle radius=3mm] {right angle = B--A--A'};
	\draw pic[draw,blue,angle radius=3mm] {right angle = B--A--C}; % Lệnh vẽ góc vuông BAC
	\end{tikzpicture}}
}
\end{vd}


\begin{vd}%Ví dụ 2%[Nguyễn Diệu Linh]%[2H1B3-3]
	Cho hình lăng trụ tứ giác đều $ABCD.A'B'C'D'$ có cạnh bên bằng $4a$ và đường chéo $5a$. Tính thể tích khối lăng trụ $ABCD.A'B'C'D'$.
	\loigiai{
\immini{	$ABCD.A'B'C'D'$ là lăng trụ đứng nên $BD^2=BD'^2-DD'^2=9a^2$\\$\Rightarrow BD=3a$.\\
	$ABCD$ là hình vuông\\ $\Rightarrow AB=\dfrac{3a}{\sqrt{2}}\Rightarrow S_{ABCD}=\dfrac{9a^2}{4}$.\\
	Vậy $V=S\cdot h=S_{ABCD}\cdot h=9a^3$.}{
	\begin{tikzpicture}[scale=1, font=\footnotesize, line join=round, line cap=round, >=stealth]
	\def\bc{4} % cạnh BC
	\def\ba{2} % cạnh BA
	\def\h{4} % đường cao
	\def\gocB{35} % góc B của đáy
	\coordinate[label=below left:$A$] (A) at (0,0);
	\coordinate[label=above left:$D$] (D) at (\gocB:\ba);
	\coordinate[label=below:$B$] (B) at (\bc,0);
	\coordinate[label=right:$C$] (C) at ($(B)-(A)+(D)$);
	\coordinate[label=above left:$A'$] (A') at ($(A)+(90:\h)$);
	\coordinate[label=above:$B'$] (B') at ($(B)-(A)+(A')$);
	\coordinate[label=above right:$C'$] (C') at ($(C)-(A)+(A')$);
	\coordinate[label=above:$D'$] (D') at ($(D)-(A)+(A')$);
	\draw (B')--(B) (A')--(B')--(C')--(D') (C)--(C') (A')--(A) (A)--(B) (B)--(C) (A')--(D');
	\draw[dashed] (B)--(D) (B)--(D') (C)--(D)--(D') (A)--(D);
	\foreach \diem in {A,B,C,D,A',B',C',D'}	\fill (\diem)circle(1.5pt);
	\draw pic[draw,blue,angle radius=3mm] {right angle = D'--D--B}; % Lệnh vẽ góc vuông BAC
	\end{tikzpicture}}

}
\end{vd}

\begin{vd}%Ví dụ 3%[Nguyễn Diệu Linh]%[2H1B3-3]
	Đáy của lăng trụ đứng tam giác $ABC.A'B'C'$ là tam giác đều cạnh $a=4$ và biết diện tích tam giác $A'BC$ bằng $8$. Tính thể tích khối lăng trụ $ABC.A'B'C'$.
	\loigiai{
\immini{	Gọi $I$ là trung điểm $BC$. Ta có $\triangle ABC$ đều nên $AI=\dfrac{AB\sqrt{3}}{2}=2\sqrt{3}$ và $AI\perp BC$\\$\Rightarrow A'I\perp BC$ (định lý 3 đường vuông góc).\\
	$S_{A'BC}=\dfrac{1}{2}BC\cdot A'I\Rightarrow A'I=\dfrac{2S_{A'BC}}{BC}=4$.\\
	$AA'\perp(ABC)\Rightarrow AA'\perp AI$.\\
	Xét $\triangle A'AI$ vuông tại $A$\\$ \Rightarrow AA'=\sqrt{A'I^2-AI^2}=2$.\\
	Vậy $V_{ABC.A'B'C'}=S_{ABC}\cdot AA'=8\sqrt{3}$.}{
	\begin{tikzpicture}[scale=1, font=\footnotesize, line join=round, line cap=round, >=stealth]
	\def\ac{4} % cạnh AC
	\def\ab{2} % cạnh AB
	\def\h{3} % chiều cao
	\def\gocA{50} % góc A của đáy
	\coordinate[label=left:$A$] (A) at (0,0);
	\coordinate[label=right:$C$] (C) at (\ac,0);
	\coordinate[label=below left:$B$] (B) at (-\gocA:\ab);
	\coordinate[label=left:$A'$] (A') at ($(A)+(90:\h)$);
	\coordinate[label=below left:$B'$] (B') at ($(B)-(A)+(A')$);
	\coordinate[label=right:$C'$] (C') at ($(C)-(A)+(A')$);
	\coordinate[label=below:$I$] (I) at ($(B)!1/2!(C)$); % Định nghĩa điểm M thỏa mãn \vt{AM}=1/2*\vt{AB}
	\draw (A')--(A)--(B)--(C)--(C')--(A')--(B')--(C') (B)--(B') (A')--(B) (A')--(B);
	\draw[dashed] (A)--(C) (A)--(I) (A')--(I);
	\foreach \diem in {A,B,C,A',B',C',I} \fill (\diem)circle(1.5pt);
	\draw pic[draw,blue,angle radius=3mm] {right angle = C--A--A'};
	\draw pic[draw,blue,angle radius=3mm] {right angle = A'--I--C}; % Lệnh vẽ góc vuông BAC
	\draw pic[draw,blue,angle radius=3mm] {right angle = A--I--B};
	\end{tikzpicture}}
}
\end{vd}


\begin{vd}%Ví dụ 4%[Nguyễn Diệu Linh]%[2H1T3-3]
	Một tấm bìa hình vuông có cạnh $44$ cm, người ta cắt bỏ đi ở mỗi góc tấm bìa một hình vuông cạnh $12$ cm rồi gấp lại thành một cái hộp chữ nhật không có nắp. Tính thể tích các hộp này.
	\loigiai{
	\begin{tikzpicture}[scale=0.6]
	\coordinate[label=above:$A$] (A) at (-1,2);
	\coordinate[label=above:$B$] (B) at (4,2);
	\coordinate[label=below right:$C$] (C) at (2,0);
	\coordinate[label=below left:$D$] (D) at ($(A)+(C)-(B)$);
	\coordinate[label=below left:$D'$] (E) at (-3,-3);
	\coordinate[label=left:$A'$] (H) at ($(A)+(E)-(D)$);
	\coordinate[label=below:$C'$] (F) at ($(C)+(E)-(D)$);
	\coordinate[label=right:$B'$] (G) at ($(B)+(F)-(C)$);
	\draw[line width=0.4pt,black] (A)--(B) (B)--(C) (A)--(D) (D)--(C) (B)--(G) (D)--(E) (C)--(F) (E)--(F) (F)--(G);
	\draw[line width=0.4pt,black,dashed] (A)--(H) (H)--(G) (H)--(E);
	\fill[blue] (A)--(B)--(C)--(D)--cycle; % Lưu ý: Miền tô màu phải là miền kín
	\foreach \diem in {A,B,C,D,E,G,F,H} \fill (\diem)circle(1.5pt);
	\end{tikzpicture}
	\begin{tikzpicture}
	\coordinate (X) at (0,0);
	\coordinate (Y) at (4,0);
	\coordinate (Z) at (0,4);
	\coordinate (T) at ($(Y)+(Z)-(X)$);
	\coordinate[label=below:$A'$](M) at ($(X)!1/4!(Y)$);
	\coordinate[label=below:$B'$](N) at ($(X)!3/4!(Y)$);
	\coordinate[label=left:$A'$](E) at ($(X)!1/4!(Z)$);
	\coordinate[label=left:$D'$](F) at ($(X)!3/4!(Z)$);
	\coordinate[label=above:$D'$](G) at ($(Z)!1/4!(T)$);
	\coordinate[label=above:$C'$](H) at ($(Z)!3/4!(T)$);
	\coordinate[label=right:$B'$](I) at ($(Y)!1/4!(T)$);
	\coordinate[label=right:$C'$](K) at ($(Y)!3/4!(T)$);
	\draw[line width=0.4pt,black] (X)--(Y) (Y)--(T) (Z)--(T) (Z)--(X) (M)--(G) (N)--(H) (F)--(K) (E)--(I);
	\path[name path=mg] (M)--(G); 
	\path[name path=fk] (F)--(K); 
	\path[name intersections={of= mg and fk,by=D}];
	\path[name path=ei] (E)--(I); 
	\path[name path=nh] (N)--(H);
	\path[name intersections={of= mg and ei,by=A}];
	\path[name intersections={of= nh and fk,by=C}]; 
	\path[name intersections={of= nh and ei,by=B}];
	\fill[black] (D)node[below right]{$D$} circle (1.5pt) (A)node[above right]{$A$} circle (1.5pt) (B)node[above left]{$B$} circle (1.5pt) (C)node[below left]{$C$} circle (1.5pt);
	\foreach \diem in {X,Y,Z,T,M,N,E,F,G,H,I,K} \fill (\diem)circle(1.5pt);
	\end{tikzpicture}\\
	Theo đề bài, ta có $AA'=BB'=CC'=DD'=12$ cm nên $ABCD$ là hình vuông có $AB=44$ cm$-24 $ cm=$20$ cm và chiều cao hình hộp $h=12$ cm.\\
	Vậy thể tích hình hộp là $V=S_{ABCD}\cdot h=4800$ cm$^3$.}
\end{vd}

\begin{vd}%Ví dụ 5%[Nguyễn Diệu Linh]%[2H1B3-3]
	Cho hình hộp đứng có đáy là hình thoi cạnh $a$ và có góc nhọn bằng $60^{\circ}$. Đường chéo lớn của đáy bằng đường chéo nhỏ của lăng trụ. Tính thể tích của khối hộp.
	\loigiai{
\immini{	Ta có $\triangle ABD$ đều nên: $BD=a$ và $S_{ABCD}=2S_{ABD}=\dfrac{a^2\sqrt{3}}{2}$.\\
	Theo đề bài $BD'=AC=2\dfrac{a\sqrt{3}}{2}=a\sqrt{3}$.\\
	Xét $\triangle DD'B$ vuông tại $D$ \\$\Rightarrow DD'=\sqrt{BD'^2-BD^2}=a\sqrt{2}$.\\
	Vậy $V=S_{ABCD}\cdot DD'=\dfrac{a^3\sqrt{6}}{2}$.}
{	\begin{tikzpicture}[scale=0.6]
	\coordinate[label=above:$D'$](A) at (-1,2);
	\coordinate[label=above:$C'$](B) at (4,2);
	\coordinate[label=below left:$B'$](C) at (2,0);
	\coordinate[label=below left:$A'$](D) at ($(A)+(C)-(B)$);
	\coordinate[label=below:$A$](E) at (-3,-3);
	\coordinate[label=left:$D$](H) at ($(A)+(E)-(D)$);
	\coordinate[label=below:$B$](F) at ($(C)+(E)-(D)$);
	\coordinate[label=right:$C$](G) at ($(B)+(F)-(C)$);
	\draw[line width=0.4pt,black] (A)--(B) (B)--(C) (A)--(D) (D)--(C) (B)--(G) (D)--(E) (F)--(G) (E)--(F) (C)--(F);
	\draw[line width=0.4pt,black,dashed] (A)--(H) (H)--(G) (H)--(E) (A)--(F) (G)--(E) (H)--(F);
	\foreach \diem in {A,B,C,D,E,G,F,H} \fill (\diem)circle(1.5pt);
	\path[name path=ab] (H)--(F); % Đặt tên đoạn thẳng AB là ab
	\path[name path=cd] (E)--(G); % Đặt tên đoạn thẳng CD là cd
	\path[name intersections={of= ab and cd,by=I}]; % Lấy giao điểm của ab và cd đặt tên điểm là M
	\draw pic[draw,blue,angle radius=3mm] {right angle = A--H--F}; 
	\draw pic[draw,blue,angle radius=3mm] {right angle = E--I--F}; 
	%\draw pic[draw,blue,angle radius=3mm,angle eccentricity=1.5,"$60^{\circ}$"] {angle = F--E--H}; % Vẽ góc BAC với tên nhãn \beta. Nếu vẽ góc vuông thì thay angle = thành right angle =	
	\end{tikzpicture}}

}
\end{vd}


\subsubsection{Câu hỏi trắc nghiệm}
\begin{ex}%Câu 1%[Nguyễn Diệu Linh]%[2H1B3-3]
	Cho hình lăng trụ đều $ABC.A'B'C'$ có cạnh đáy bằng $a$, cạnh bên bằng $2a$. Thể tích của khối lăng trụ là
	\choice
	{\True $a^3\dfrac{\sqrt{3}}{2}$ }
	{$a^3\dfrac{\sqrt{3}}{6}$}
	{$a^3$}
	{$\dfrac{a^3}{3}$}
\end{ex}


\begin{ex}%Câu 2%[Nguyễn Diệu Linh]%[2H1B3-3]
	Cho hình lăng trụ tam giác đều có các cạnh đều bằng $a$. Thể tích khối lăng trụ đều là
	\choice
	{$\dfrac{2a^3\sqrt{2}}{3}$}
	{$\dfrac{a^3}{3}$}
	{$\dfrac{2a^3}{3}$}
	{\True $\dfrac{a^3\sqrt{3}}{4}$}
\end{ex}

\begin{ex}%Câu 3%[Nguyễn Diệu Linh]%[2H1Y3-3]
	Nếu ba kích thước của một khối chữ nhật tăng lên $4$ lần thì thể tích của nó tăng lên
	\choice
	{$4$ lần}
	{$16$ lần}
	{\True $64$ lần}
	{$192$ lần}
\end{ex}


\begin{ex}%Câu 4%[Nguyễn Diệu Linh]%[2H1B3-3]
	Cho một khối lập phương biết rằng khi tăng độ dài cạnh của khối lập phương thêm $2$ cm thì thể tích của nó tăng thêm $98$ cm$^3$. Hỏi cạnh của khối lập phương đã cho bằng
	\choice
	{\True $3$ cm}
	{$4$ cm}
	{$5$ cm}
	{$6$ cm}
\end{ex}


\begin{ex}%Câu 5%[Nguyễn Diệu Linh]%[2H1Y3-3]
	Một khối hộp chữ nhật $(H)$ có các kích thước là $a,b,c$. Khối hộp chữ nhật $(H')$ có các kích thước tương ứng lần lượt là $\dfrac{a}{2},\dfrac{2b}{3},\dfrac{3c}{4}$. Khi đó tỉ số thể tích $\dfrac{V_{(H')}}{V_{(H)}}$ là
	\choice
	{$\dfrac{1}{24}$}
	{$\dfrac{1}{12}$}
	{$\dfrac{1}{2}$}
	{\True $\dfrac{1}{4}$}
\end{ex}


\begin{ex}%Câu 6%[Nguyễn Diệu Linh]%[2H1B3-3]
	Cho hình lập phương có độ dài đường chéo bằng $10\sqrt{2}$ cm. Thể tích của khối lập phương là 
	\choice
	{$300$ cm$^3$}
	{$900$ cm$^3$}
	{\True $1000$ cm$^3$}
	{$2700$ cm$^3$}
\end{ex}

\begin{ex}%Câu 7%[Nguyễn Diệu Linh]%[2H1Y3-3]
	Nếu ba kích thước của một khối hộp chữ nhật tăng lên $k$ lần thì thể tích khối hộp tương ứng sẽ
	\choice
	{tăng $k$ lần}
	{tăng $k^2$ lần}
	{\True tăng $k^3$ lần}
	{tăng $3k^3$ lần}
\end{ex}


\begin{ex}%Câu 8%[Nguyễn Diệu Linh]%[2H1B3-3]
	Cho lăng trụ tam giác đều $ABC.A'B'C'$ có cạnh đáy bằng $a$ và cạnh bên bằng $a\sqrt{3}$. Tính thể tích $V$ của khối lăng trụ $ABC.A'B'C'$. 
	\choice
	{$V=\dfrac{a^3}{4}$}
	{$V=\dfrac{3a^3}{8}$}
	{$V=\dfrac{a^3}{8}$}
	{\True $V=\dfrac{3a^3}{4}$}
\end{ex}

\begin{ex}%Câu 9%[Nguyễn Diệu Linh]%[2H1B3-3]
	Cho lăng trụ đứng $ABC.A'B'C'$ có đáy $ABC$ là tam giác vuông, $AB=AC=a,$ cạnh bên $AA'=a\sqrt{2}$. Tính thể tích $V$ của khối lăng trụ $ABC.A'B'C'$. 
	\choice
	{\True $V=\dfrac{a^3\sqrt{2}}{2}$}
	{$V=\dfrac{a^3\sqrt{2}}{6}$}
	{$V=\dfrac{a^3\sqrt{2}}{3}$}
	{$V=a^3\sqrt{2}$}
\end{ex}


\begin{ex}%Câu 10%[Nguyễn Diệu Linh]%[2H1B3-3]
	Tính thể tích $V$ của khối lập phương có các đỉnh là trọng tâm của các mặt của một khối bát diện đều cạnh $a$. 
	\choice
	{\True $V=\dfrac{8a^3}{27}$}
	{$V=\dfrac{a^3}{27}$}
	{$V=\dfrac{16a^3\sqrt{2}}{27}$}
	{$V=\dfrac{2a^3\sqrt{2}}{27}$}
\end{ex}


\begin{ex}%Câu 11%[Nguyễn Diệu Linh]%[2H1B3-3]
	Tính thể tích $V$ của khối lập phương $ABCD.A'B'C'D'$, biết tổng diện tích các mặt của hình lập phương bằng 150. 
	\choice
	{$V=25$}
	{$V=75$}
	{\True $V=125$}
	{$V=100$}
\end{ex}

\begin{ex}%Câu 12%[Nguyễn Diệu Linh]%[2H1B3-3]
	Tính thể tích $V$ của khối lập phương $ABCD.A'B'C'D'$, biết đáy nội tiếp đường tròn có chu vi bằng $4\pi$. 
	\choice
	{$V=\pi^3$}
	{\True $V=8$}
	{$V=16\sqrt{2}$}
	{$V=2\sqrt{2}$}
\end{ex}

\begin{dang}{Lăng trụ đứng có góc giữa đường thẳng và mặt phẳng}
\end{dang}
\subsubsection{Các ví dụ}
\begin{vd}%Ví dụ 1%[Nguyễn Diệu Linh]%[2H1B3-3]
	Cho lăng trụ đứng tam giác $ABC.A'B'C'$ có đáy $ABC$ là tam giác vuông cân tại $B$ với $BA=BC=a$, biết $A'B$ hợp với đáy $ABC$ một góc $60^{\circ}$. Tính thể tích lăng trụ.
	\loigiai{
	\immini{	Ta có $AA'\perp(ABC)\Rightarrow AA'\perp AB$ và $AB$ là hình chiếu của $A'B$ trên đáy $ABC$.\\
		Vậy góc $\left(A'B,(ABC)\right)=\widehat{ABA'}=60^{\circ}$.\\
		Xét trong $\triangle ABA'\text{, ta có}$\\$ AA'=AB\cdot\tan 60^{\circ}=a\sqrt{3}$.\\
		$S_{ABC}=\dfrac{1}{2}BA\cdot BC=\dfrac{a^2}{2}$.\\
		Vậy $V=S_{ABC}\cdot AA'=\dfrac{a^3\sqrt{3}}{2}$.}{
	\begin{tikzpicture}[scale=1, font=\footnotesize, line join=round, line cap=round, >=stealth]
	\def\ac{4} % cạnh AC
	\def\ab{2} % cạnh AB
	\def\h{4} % chiều cao
	\def\gocA{50} % góc A của đáy
	\coordinate[label=left:$A$] (A) at (0,0);
	\coordinate[label=right:$C$] (C) at (\ac,0);
	\coordinate[label=below left:$B$] (B) at (-\gocA:\ab);
	\coordinate[label=left:$A'$] (A') at ($(A)+(90:\h)$);
	\coordinate[label=below left:$B'$] (B') at ($(B)-(A)+(A')$);
	\coordinate[label=right:$C'$] (C') at ($(C)-(A)+(A')$);
	\draw (A')--(A)--(B)--(C)--(C')--(A')--(B')--(C') (B)--(B') (A')--(B);
	\draw[dashed] (A)--(C);
	\foreach \diem in {A,B,C,A',B',C'} \fill (\diem)circle(1.5pt);
	\draw pic[draw,blue,angle radius=3mm] {right angle = A'--A--B};
	\draw pic[draw,blue,angle radius=3mm] {right angle = A--B--C}; 
	\end{tikzpicture}
}	
}
\end{vd}

\begin{vd}%Ví dụ 2%[Nguyễn Diệu Linh]%[2H1B3-3]
	Cho lăng trụ đứng tam giác $ABC.A'B'C'$ có đáy $ABC$ là tam giác vuông tại $A$ với $AC=a$, $\widehat{ACB}=60^{\circ}$, biết $BC'$ hợp với $(AA'C'C)$ một góc bằng $30^{\circ}$. Tính $AC'$ và thể tích lăng trụ.
	\loigiai{
	\immini{	Xét $\triangle ABC$ vuông tại $A$ $\Rightarrow AB=AC\cdot\tan 60^{\circ}=a\sqrt{3}$.\\
		Mặt khác: $AB\perp AC;AB\perp AA'$\\$\Rightarrow AB\perp(AA'C'C)$ nên $AC'$ là hình chiếu của $BC'$ trên $(AA'C'C)$.\\ Vậy góc $\left(BC';(AA'C'C)\right)=\widehat{BC'A}=30^{\circ}$.\\
		Xét $\triangle AC'B$ vuông tại $A$, ta có \\ $ AC'=\dfrac{AB}{\tan{30}^{\circ}}=3a$.\\
		Xét $\triangle AA'C'$ vuông tại $A'$, ta có \\$ AA'=\sqrt{AC'^2-A'C'^2}=2a\sqrt{2}$.\\
		$\triangle ABC$ là nửa tam giác đều nên \break $S_{ABC}=\dfrac{a^2\sqrt{3}}{2}$.\\
		Vậy $V=a^3\sqrt{6}$.}{
	\begin{tikzpicture}[scale=1, font=\footnotesize, line join=round, line cap=round, >=stealth]
	\def\ac{4} % cạnh AC
	\def\ab{2} % cạnh AB
	\def\h{4} % chiều cao
	\def\gocA{50} % góc A của đáy
	\coordinate[label=left:$A$] (A) at (0,0);
	\coordinate[label=right:$C$] (C) at (\ac,0);
	\coordinate[label=below left:$B$] (B) at (-\gocA:\ab);
	\coordinate[label=left:$A'$] (A') at ($(A)+(90:\h)$);
	\coordinate[label=below left:$B'$] (B') at ($(B)-(A)+(A')$);
	\coordinate[label=right:$C'$] (C') at ($(C)-(A)+(A')$);
	\draw (A')--(A)--(B)--(C)--(C')--(A')--(B')--(C') (B)--(B') (B)--(C');
	\draw[dashed] (A)--(C) (A)--(C');
	\foreach \diem in {A,B,C,A',B',C'} \fill (\diem)circle(1.5pt);
	%\draw pic[draw,blue,angle radius=6mm,angle eccentricity=1.5,"$30^\circ$"] {angle = A--C'--B};
	%\draw pic[draw,red,angle radius=6mm,angle eccentricity=1.5,"$60^\circ$"] {angle = A--C--B};
	\begin{scope}
			\clip (A)--(C)--(B);
			\draw (C) circle(.5);
			\end{scope}
	\draw pic[draw,blue,angle radius=3mm] {right angle = C'--A--B};
	\draw pic[draw,blue,angle radius=3mm] {right angle = C--A--B}; 
	\end{tikzpicture}
}

}
\end{vd}


\begin{vd}%Ví dụ 3%[Nguyễn Diệu Linh]%[2H1B3-3]
	Cho lăng trụ đứng $ABCD.A'B'C'D'$ có đáy $ABCD$ là hình vuông cạnh $a$ và đường chéo $BD'$ của lăng trụ hợp với đáy $ABCD$ một góc $30^{\circ}$. Tính thể tích và tổng diện tích các mặt bên của lăng trụ.
	\loigiai{
\immini{	Ta có $ABCD.A'B'C'D'$ là lăng trụ đứng nên ta có: $DD'\perp(ABCD)$\\
	$\Rightarrow DD'\perp BD$ và $BD$ là hình chiếu của $BD'$ trên $ABCD$.\\
	Vậy góc $(BD';(ABCD))=\widehat{DBD'}=30^{\circ}$.\\
	$\triangle BDD'\Rightarrow DD'=BD\cdot\tan 30^{\circ}=\dfrac{a\sqrt{6}}{3}$.\\
	Vậy $V=S_{ABCD}\cdot DD'=\dfrac{a^3\sqrt{6}}{3}$.\\
	$S=4S_{ADD'A'}=\dfrac{4a^2\sqrt{6}}{3}$.}{
	\begin{tikzpicture}[scale=0.7, font=\footnotesize, line join=round, line cap=round, >=stealth]
	\def\bc{4} % cạnh BC
	\def\ba{2} % cạnh BA
	\def\gocB{35} % góc B của đáy
	\coordinate[label=below left:$D$] (B) at (0,0);
	\coordinate[label=above left:$C$] (A) at (\gocB:\ba);
	\coordinate[label=below:$A$] (C) at (\bc,0);
	\coordinate[label=right:$B$] (D) at ($(C)-(B)+(A)$);
	\coordinate[label=above left:$C'$] (A') at ($(A)+(90:\bc)$);
	\coordinate[label=left:$D'$] (B') at ($(B)-(A)+(A')$);
	\coordinate[label=below right:$A'$] (C') at ($(C)-(A)+(A')$);
	\coordinate[label=right:$B'$] (D') at ($(D)-(A)+(A')$);
	\draw (B')--(B)--(C)--(D)--(D')--(A')--(B')--(C')--(D') (C)--(C');
	\draw[dashed] (A')--(A)--(D) (A)--(B) (B')--(D) (B)--(D);
	%\draw pic[draw,blue,angle radius=5mm,angle eccentricity=1.5,"$30^\circ$"] {angle = B'--D--B}; % Vẽ góc BAC với tên nhãn \beta. Nếu vẽ góc vuông thì thay angle = thành right angle =
	\foreach \diem in {A,B,C,D,A',B',C',D'}	\fill (\diem)circle(1.5pt);
	\end{tikzpicture}
}
}
\end{vd}

\begin{vd}%Ví dụ 4%[Nguyễn Diệu Linh]%[2H1B3-3]
	Cho hình hộp đứng $ABCD.A'B'C'D'$ có đáy $ABCD$ là hình thoi cạnh $a$ và $\widehat{BAD}=60^{\circ}$, biết $AB'$ hợp với đáy $(ABCD)$ một góc $30^{\circ}$. Tính thể tích của hình hộp.
	\loigiai{
\immini{
	Ta có $\triangle ABD$ đều cạnh $a$\\$\Rightarrow S_{ABD}=\dfrac{a^2\sqrt{3}}{4}$\\$\Rightarrow S_{ABCD}=2S_{ABD}=\dfrac{a^2\sqrt{3}}{2}$.\\
	$\triangle ABB'$ vuông tại $B$\\$\Rightarrow BB'=AB\cdot\tan 30^{\circ}=a\sqrt{3}$.\\
	Vậy $V=S_{ABCD}\cdot BB'=\dfrac{3a^3}{2}$.}{
	\begin{tikzpicture}[scale=0.8, font=\footnotesize, line join=round, line cap=round, >=stealth]
	\def\bc{4} % cạnh BC
	\def\ba{2} % cạnh BA
	\def\gocB{35} % góc B của đáy
	\coordinate[label=below left:$B$] (B) at (0,0);
	\coordinate[label=above left:$A$] (A) at (\gocB:\ba);
	\coordinate[label=below:$C$] (C) at (\bc,0);
	\coordinate[label=right:$D$] (D) at ($(C)-(B)+(A)$);
	\coordinate[label=above left:$A'$] (A') at ($(A)+(90:\bc)$);
	\coordinate[label=left:$B'$] (B') at ($(B)-(A)+(A')$);
	\coordinate[label=below right:$C'$] (C') at ($(C)-(A)+(A')$);
	\coordinate[label=right:$D'$] (D') at ($(D)-(A)+(A')$);
	\draw (B')--(B)--(C)--(D)--(D')--(A')--(B')--(C')--(D') (C)--(C');
	\draw[dashed] (A')--(A)--(D) (A)--(B) (A')--(B) (A)--(C);
	\foreach \diem in {A,B,C,D,A',B',C',D'}	\fill (\diem)circle(1.5pt);
	%\draw pic[draw,blue,angle radius=5mm,angle eccentricity=1.5,"$60^\circ$"] {angle = C--B--A};
	 \draw pic[draw,blue,angle radius=3mm] {right angle = A'--A--B}; % Lệnh vẽ góc vuông BAC
	\end{tikzpicture}
}}
\end{vd}

\subsubsection{Câu hỏi trắc nghiệm}
\begin{ex}%Câu 1%[Nguyễn Diệu Linh]%[2H1B3-3]
	Cho khối lăng trụ đứng tam giác $ABC.A'B'C'$ có đáy là một tam giác vuông cân tại $A$. Cho $AC=AB=2a$, góc giữa $AC'$ và mặt phẳng $(ABC)$ bằng $30^{\circ}$. Thể tích khối lăng trụ là
	\choice
	{\True $\dfrac{4a^3\sqrt{3}}{3}$}
	{$\dfrac{2a^3\sqrt{3}}{3}$}
	{$\dfrac{4a^2\sqrt{3}}{3}$}
	{$\dfrac{4a\sqrt{3}}{3}$}
\end{ex}

\begin{ex}%Câu 2%[Nguyễn Diệu Linh]%[2H1B3-3]
	Cho hình hộp chữ nhật $ABCD.A'B'C'D'$ có $AB=a;AD=2a$, đường thẳng $A'C$ tạo với đáy một góc $60^{\circ}$. Tính thể tích $V$ của khối hộp chữ nhật $ABCD.A'B'C'D'$. 
	\choice
	{\True $V=2a^3\sqrt{15}$}
	{$V=a^3\sqrt{15}$}
	{$V=2a^3\sqrt{3}$}
	{$V=4a^3\sqrt{3}$}
\end{ex}

\begin{ex}%Câu 3%[Nguyễn Diệu Linh]%[2H1B3-3]
	Cho hình hộp chữ nhật $ABCD.A'B'C'D'$ với $AB=10$ cm, $AD=16$ cm. Biết rằng hợp với đáy một góc $\varphi$ sao cho $\cos\varphi=\dfrac{8}{17}$. Tính thể tích khối hộp. 
	\choice
	{\True $4800$ cm$^3$}
	{$5200$ cm$^3$}
	{$3400$ cm$^3$}
	{$6500$ cm$^3$}
\end{ex}

\begin{ex}%Câu 4%[Nguyễn Diệu Linh]%[2H1B3-3]
	Cho lăng trụ tứ giác đều $ABCD.A'B'C'D'$ có cạnh đáy $a$, góc của đường chéo với đáy là $60^{\circ}$. Tính thể tích khối lăng trụ
	\choice
	{\True $a^3\sqrt{6}$}
	{$a^2\sqrt{6}$}
	{$\dfrac{a^3\sqrt{6}}{3}$}
	{$\dfrac{a^3\sqrt{6}}{2}$}
\end{ex}

\begin{ex}%Câu 5%[Nguyễn Diệu Linh]%[2H1K3-3]
	Cho lăng trụ tam giác đều $ABC.A'B'C'$ có khoảng cách từ điểm $A$ đến mặt phẳng $(A'BC)$ bằng $a$ và đường thẳng $AA'$ hợp với mặt phẳng $(A'BC)$ một góc $30^{\circ}$. Thể tích khối lăng trụ $ABC.A'B'C'$. 
	\choice
	{\True $V=\dfrac{32a^3}{9}$}
	{$V=\dfrac{31a^3}{9}$}
	{$V=\dfrac{30a^3}{9}$}
	{$V=\dfrac{33a^3}{9}$}
\end{ex}

\begin{ex}%Câu 6%[Nguyễn Diệu Linh]%[2H1K3-3]
	Cho lăng trụ đứng $ABCD.A'B'C'D'$ có đáy là hình thoi cạnh bằng $1$, $\widehat{BAD}=120^{\circ}$. Góc giữa $AC'$ và mặt phẳng $(ADD'A')$ bằng $30^{\circ}$. Tính thể tích khối lăng trụ. 
	\choice
	{\True $V=\sqrt{6}$}
	{$V=\dfrac{\sqrt{6}}{6}$}
	{$V=\dfrac{\sqrt{6}}{2}$}
	{$V=\sqrt{3}$}
\end{ex}

\begin{dang}{Lăng trụ đứng có góc giữa hai mặt phẳng}
\end{dang}
\begin{vd}%Ví dụ 1%[Nguyễn Diệu Linh]%[2H1B3-3]
	Cho lăng trụ đứng tam giác $ABC.A'B'C'$ có đáy $ABC$ là tam giác vuông cân tại $B$ với $BA=BC=a$, biết $(A'BC)$ hợp với đáy $(ABC)$ một góc $60^{\circ}$. Tính thể tích lăng trụ.
	\loigiai{
	\immini{	Ta có $AA'\perp(ABC)$ và $BC\perp AB$\\$\Rightarrow BC\perp A'B$ \\ $\Rightarrow \left((A'BC);(ABC)\right)=\widehat{ABA'}=60^{\circ}$.\\
		$\triangle ABA'\Rightarrow AA'=AB\cdot\tan 60^{\circ}=a\sqrt{3}$.\\
		$S_{ABC}=\dfrac{1}{2}BA\cdot BC=\dfrac{a^2}{2}$.\\
		Vậy $V=S_{ABC}\cdot AA'=\dfrac{a^3\sqrt{3}}{2}$.}{
	\begin{tikzpicture}[scale=1, font=\footnotesize, line join=round, line cap=round, >=stealth]
	\def\ac{4} % cạnh AC
	\def\ab{2} % cạnh AB
	\def\h{3} % chiều cao
	\def\gocA{50} % góc A của đáy
	\coordinate[label=left:$A$] (A) at (0,0);
	\coordinate[label=right:$C$] (C) at (\ac,0);
	\coordinate[label=below left:$B$] (B) at (-\gocA:\ab);
	\coordinate[label=left:$A'$] (A') at ($(A)+(90:\h)$);
	\coordinate[label=below left:$B'$] (B') at ($(B)-(A)+(A')$);
	\coordinate[label=right:$C'$] (C') at ($(C)-(A)+(A')$);
	\draw (A')--(A)--(B)--(C)--(C')--(A')--(B')--(C') (B)--(B') (A')--(B);
	\draw[dashed] (A)--(C) (A')--(C);
	\foreach \diem in {A,B,C,A',B',C'} \fill (\diem)circle(1.5pt);
	\draw pic[draw,blue,angle radius=3mm] {right angle = A'--A--B}; 
	\draw pic[draw,blue,angle radius=3mm] {right angle = A'--B--C};
	\draw pic[draw,blue,angle radius=3mm] {right angle = A--B--C}; 
	\draw pic[draw,blue,angle radius=6mm,angle eccentricity=1.5] {angle = A'--B--A}; % Vẽ góc BAC với tên nhãn \beta. Nếu vẽ góc vuông thì thay angle = thành right angle =
	\end{tikzpicture}
}
}
\end{vd}

\begin{vd}%Ví dụ 2%[Nguyễn Diệu Linh]%[2H1K3-3]
	Đáy của lăng trụ đứng tam giác $ABC.A'B'C'$ là tam giác đều. Mặt $(A'BC)$ tạo với đáy một góc $30^{\circ}$ và diện tích tam giác $A'BC$ bằng $8$. Tính thể tích khối lăng trụ.
	\loigiai{
	\immini{	$\triangle ABC$ đều $\Rightarrow AI\perp BC$ mà $AA'\perp(ABC)$ nên $A'I\perp BC$.\\
		Vậy góc $\left((A'BC);(ABC)\right)=\widehat{A'IA}=30^{\circ}$.\\
		Giả sử $BI=x\Rightarrow AI=\dfrac{2x\sqrt{3}}{2}=x\sqrt{3}$. \\
		Xét $\triangle A'AI$ vuông tại $A$\\ $\Rightarrow A'I=\dfrac{AI}{\cos{30}^{\circ}}=2x;$\\$AA'=AI\cdot\tan 30^{\circ}=x\sqrt{3}\cdot\dfrac{\sqrt{3}}{3}=x$.\\
		Vậy $V_{ABC.A'B'C'}=CI\cdot AI\cdot AA'=x^3\sqrt{3}$.\\
		Mà $S_{A'BC}=BI\cdot A'I=x\cdot 2x=8\Rightarrow x=2$.\\
		Do đó $V_{ABC.A'B'C'}=8\sqrt{3}$.}{
	\begin{tikzpicture}[scale=0.9, font=\footnotesize, line join=round, line cap=round, >=stealth]
	\def\ac{4} % cạnh AC
	\def\ab{2} % cạnh AB
	\def\h{4} % chiều cao
	\def\gocA{50} % góc A của đáy
	\coordinate[label=left:$A$] (A) at (0,0);
	\coordinate[label=right:$C$] (C) at (\ac,0);
	\coordinate[label=below left:$B$] (B) at (-\gocA:\ab);
	\coordinate[label=left:$A'$] (A') at ($(A)+(90:\h)$);
	\coordinate[label=below right:$B'$] (B') at ($(B)-(A)+(A')$);
	\coordinate[label=right:$C'$] (C') at ($(C)-(A)+(A')$);
	\coordinate[label=below:$I$] (I) at ($(B)!1/2!(C)$); % Định nghĩa điểm M thỏa mãn \vt{AM}=1/2*\vt{AB}
	\draw (A')--(A)--(B) (C)--(C')--(A')--(B')--(C') (B)--(B') (A')--(B) (C)--(I);
	\draw[dashed] (A)--(C) (A')--(C) (A)--(I) (A')--(I);
	\draw[line width=0.4pt,black] (B)--(I)node[below,pos=0.5]{$x$}; % Đoạn thẳng AB nhãn là a
	\draw pic[draw,blue,angle radius=3mm] {right angle = A'--I--C}; 
	\draw pic[draw,blue,angle radius=3mm] {right angle = A--I--B}; 
	\draw pic[draw,blue,angle radius=6mm,angle eccentricity=1.5] {angle = A'--I--A}; % Vẽ góc BAC với tên nhãn \beta. Nếu vẽ góc vuông thì thay angle = thành right angle =
	\foreach \diem in {A,B,C,A',B',C',I} \fill (\diem)circle(1.5pt);
	\end{tikzpicture}
}
}
\end{vd}

\begin{vd}%Ví dụ 3%[Nguyễn Diệu Linh]%[2H1B3-3]
	Cho lăng trụ tứ giác đều $ABCD.A'B'C'D'$ có cạnh đáy bằng $a$ và mặt phẳng $(BDC')$ hợp với đáy $(ABCD)$ một góc $60^{\circ}$. Tính thể tích lăng trụ.
	\loigiai{
	\immini{Gọi $O$ là tâm $ABCD$. Ta có $ABCD$ là hình vuông nên $OC\perp BD$ và $CC'\perp(ABCD)$ nên $OC'\perp BD$. \\Vậy $\left((BDC');(ABCD)\right)=\widehat{C'OC}=60^{\circ}$.\\
		$S_{ABCD}=a^2$.\\
		$\triangle OCC'$ vuông tại $C$, ta có\\ $CC'=OC\cdot\tan 60^{\circ}=\dfrac{a\sqrt{6}}{2}$.\\
		Vậy $V=S_{ABCD}\cdot CC'=\dfrac{a^3\sqrt{6}}{2}$.}{
	\begin{tikzpicture}[scale=0.7, font=\footnotesize, line join=round, line cap=round, >=stealth]
	\def\bc{4} % cạnh BC
	\def\ba{2} % cạnh BA
	\def\gocB{35} % góc B của đáy
	\coordinate[label=below left:$B$] (B) at (0,0);
	\coordinate[label=above left:$C$] (A) at (\gocB:\ba);
	\coordinate[label=below:$A$] (C) at (\bc,0);
	\coordinate[label=right:$D$] (D) at ($(C)-(B)+(A)$);
	\coordinate[label=above left:$C'$] (A') at ($(A)+(90:\bc)$);
	\coordinate[label=left:$B'$] (B') at ($(B)-(A)+(A')$);
	\coordinate[label=below right:$A'$] (C') at ($(C)-(A)+(A')$);
	\coordinate[label=right:$D'$] (D') at ($(D)-(A)+(A')$);
	\coordinate[label=below:$O$] (O) at ($(A)!1/2!(C)$); 
	\draw (B')--(B)--(C)--(D)--(D')--(A')--(B')--(C')--(D') (C)--(C');
	\draw[dashed] (A')--(A)--(D) (A)--(B) (A')--(B) (A')--(D) (A)--(C) (B)--(D) (A')--(O);
	\draw pic[draw,blue,angle radius=3mm] {right angle = A'--O--D}; 
	\draw pic[draw,blue,angle radius=3mm] {right angle = B--O--C};
	\draw pic[draw,blue,angle radius=6mm,angle eccentricity=1.5] {angle = A'--O--A}; 
	\foreach \diem in {A,B,C,D,A',B',C',D',O}	\fill (\diem)circle(1.5pt);
	\end{tikzpicture}
}
	}
\end{vd}

\begin{vd}%Ví dụ 4%[Nguyễn Diệu Linh]%[2H1K3-3]
	Cho hình hộp chữ nhật $ABCD.A'B'C'D'$ có $AA'=2a$; Mặt phẳng $(A'BC)$ hợp với đáy $(ABCD)$ một góc $60^{\circ}$ và $AC'$ hợp với đáy $(ABCD)$ một góc $30^{\circ}$. Tính thể tích khối hộp chữ nhật.
	\loigiai{
	\immini{	Ta có $AA'\perp (ABCD)$\\$\Rightarrow AC$ là hình chiếu của $A'C$ trên $(ABCD)$.\\
		Vậy $(A'C;(ABCD))=\widehat{A'CA}=30^{\circ}$.\\
		$BC\perp AB\Rightarrow BC\perp A'B$.\\
	 $\Rightarrow \left((A'BC);(ABCD)\right)=\widehat{A'BA}=60^{\circ}$.\\
		$\triangle A'AC$ vuông tại $A$, ta có $$AC=AA'\cdot\cot 30^{\circ}=2a\sqrt{3}.$$
		$\triangle A'AB$ vuông tại $A$, ta có  $$AB=AA'\cdot\cot 60^{\circ}=\dfrac{2a\sqrt{3}}{3}.$$
		$\triangle ABC$ vuông tại $B$, ta có $$ BC=\sqrt{AC^2-AB^2}=\dfrac{4a\sqrt{6}}{3}.$$
		Vậy $V=AB\cdot BC\cdot AA'=\dfrac{16a^3\sqrt{2}}{3}$.}{
	\begin{tikzpicture}[scale=0.7]
	\coordinate[label=above:$A'$](A) at (-1,2);
	\coordinate[label=above:$D'$](B) at (4,2);
	\coordinate[label=below left:$C'$](C) at (2,0);
	\coordinate[label=below left:$B'$](D) at ($(A)+(C)-(B)$);
	\coordinate[label=below left:$B$](E) at (-3,-4);
	\coordinate[label=left:$A$](H) at ($(A)+(E)-(D)$);
	\coordinate[label=below left:$C$](F) at ($(C)+(E)-(D)$);
	\coordinate[label=right:$D$](G) at ($(B)+(F)-(C)$);
	\draw[line width=0.4pt,black] (A)--(B) (B)--(C) (A)--(D) (D)--(C) (B)--(G) (D)--(E) (F)--(E) (F)--(G) (F)--(C);
	\draw[line width=0.4pt,black,dashed] (A)--(H) (H)--(G) (H)--(E) (A)--(F) (A)--(E) (H)--(F);
	\draw pic[draw,blue,angle radius=3mm] {right angle = A--H--F};
	\draw pic[draw,blue,angle radius=3.5mm] {angle = A--F--H}; % Vẽ góc BAC và đánh dấu |. Cần chỉnh hướng xoay kí hiệu đánh dấu hợp lí
	\draw pic[draw,blue,angle radius=6mm,angle eccentricity=1.5] {angle = H--E--A}; % Vẽ góc BAC với tên nhãn \beta. Nếu vẽ góc vuông thì thay angle = thành right angle =
	 
	\foreach \diem in {A,B,C,D,E,G,F,H} \fill (\diem)circle(1.5pt);
	\end{tikzpicture}	
}
}
\end{vd}

\subsubsection{Câu hỏi trắc nghiệm}
\begin{ex}%Câu 1%[Nguyễn Diệu Linh]%[2H1B3-3]
	Cho hình lăng trụ tứ giác đều $ABCD.A'B'C'D'$ cạnh đáy $4\sqrt{3}$ dm. Biết mặt phẳng $(BCD')$ hợp với đáy một góc $60^{\circ}$. Tính thể tích khối lăng trụ. 
	\choice
	{$325$ dm$^3$}
	{$478$ dm$^3$}
	{\True $576$ dm$^3$}
	{$648$ dm$^3$}
\end{ex}

\begin{ex}%Câu 2%[Nguyễn Diệu Linh]%[2H1B3-3]
	Cho lăng trụ đứng $ABC.A'B'C'$ có đáy $ABC$ là tam giác vuông tại $B, AB=a, BC=a\sqrt{2}$, mặt bên $(A'BC)$ hợp với mặt đáy $(ABC)$ một góc $30^{\circ}$. Tính thể tích khối lăng trụ. 
	\choice
	{$\dfrac{a^3\sqrt{3}}{6}$}
	{$\dfrac{a^3\sqrt{6}}{3}$}
	{$\dfrac{a^3\sqrt{3}}{3}$}
	{\True $\dfrac{a^3\sqrt{6}}{6}$}
\end{ex}

\begin{ex}%Câu 3%[Nguyễn Diệu Linh]%[2H1B3-3]
	Cho hình lăng trụ đứng $ABC.A'B'C'$ có đáy $ABC$ là tam giác vuông cân tại $B$ với $BA=BC=a$. Biết rằng mặt phẳng $(A'BC)$ hớp với mặt phẳng $(ABC)$ một góc $60^{\circ}$. Tính thể tích khối lăng trụ $ABC.A'B'C'$. 
	\choice
	{\True $V=\dfrac{a^3\sqrt{3}}{2}$}
	{$V=\dfrac{a^3\sqrt{3}}{4}$}
	{$V=\dfrac{a^3\sqrt{3}}{3}$}
	{$V=\dfrac{a^3\sqrt{3}}{6}$}
\end{ex}

\begin{ex}%Câu 4%[Nguyễn Diệu Linh]%[2H1K3-3]
	Cho hình lăng trụ tam giác đều $ABC.A'B'C'$ có $AB=a,$ góc giữa hai mặt phẳng $A'BC$ và $(ABC)$ bằng $60^{\circ}$. Tính thể tích $V$ của khối lăng trụ đã cho. 
	\choice
	{\True $V=\dfrac{3\sqrt{3}}{8}a^3$}
	{$V=\dfrac{\sqrt{3}}{8}a^3$}
	{$V=\dfrac{3\sqrt{3}}{4}a^3$}
	{$V=\dfrac{\sqrt{3}}{4}a^3$}
\end{ex}

\begin{ex}%Câu 5%[Nguyễn Diệu Linh]%[2H1B3-3]
	Cho hình lăng trụ đứng $ABC.A'B'C'$ có đáy $ABC$ là tam giác vuông cân tại $B$; $AC=2a$. Biết rằng mặt phẳng $(A'BC)$ hợp với mặt phẳng $(ABC)$ một góc $45^{\circ}$. Tính thể tích khối lăng trụ $ABC.A'B'C'$. 
	\choice
	{\True $V=a^3\sqrt{2}$}
	{$V=\dfrac{a^3\sqrt{2}}{2}$}
	{$V=\dfrac{a^3\sqrt{2}}{3}$}
	{$V=\dfrac{a^3\sqrt{2}}{4}$}
\end{ex}

\begin{ex}%Câu 6%[Nguyễn Diệu Linh]%[2H1B3-3]
	Cho hình lăng trụ đứng $ABC.A'B'C'$ có đáy $ABC$ là tam giác vuông cân tại $A$ với $AB=AC=a$ và $\widehat{BAC}=120^{\circ}$. Biết rằng mặt phẳng $(A'BC)$ hợp với mặt phẳng $(ABC)$ một góc $45^{\circ}$. Tính thể tích khối lăng trụ. 
	\choice
	{\True $V=\dfrac{a^3\sqrt{3}}{8}$}
	{$V=\dfrac{a^3\sqrt{3}}{2}$}
	{$V=\dfrac{a^3\sqrt{3}}{4}$}
	{$V=\dfrac{a^3\sqrt{3}}{6}$}
\end{ex}

\begin{ex}%Câu 7%[Nguyễn Diệu Linh]%[2H1K3-3]
	Đáy của lăng trụ đứng tam giác $ABC.A'B'C'$ là tam giác đều. Mặt $(A'BC)$ tạo với đáy một góc $30^{\circ}$ và diện tích tam giác $A'BC$ bằng $8$. Tính thể tích khối lăng trụ
	\choice
	{$16\sqrt{3}$}
	{\True $8\sqrt{3}$}
	{$4\sqrt{3}$}
	{$6\sqrt{3}$}
\end{ex}

\begin{ex}%Câu 8%[Nguyễn Diệu Linh]%[2H1K3-3]
	Cho hình hộp chữ nhật $ABCD.A'B'C'D'$. Mặt phẳng $(A'BC)$ hợp với đáy $(ABCD)$ một góc $60^{\circ}, A'C$ hợp với đáy $(ABCD)$ một góc $30^{\circ}$ và $AA'=a\sqrt{3}$. Tính theo $a$ thể tích khối hộp. 
	\choice
	{\True $V=2a^3\sqrt{6}$}
	{$V=\dfrac{2a^3\sqrt{6}}{3}$}
	{$V=2a^3\sqrt{2}$}
	{$V=a^3$}
\end{ex}

\begin{ex}%Câu 9%[Nguyễn Diệu Linh]%[2H1K3-3]
	Cho hình lăng trụ đứng $ABC.A'B'C'$ có đáy $ABC$ là tam giác vuông tại $B$ và $BB'=AB=h$. Biết rằng mặt phẳng $(B'AC)$ hợp với mặt phẳng chứa đáy $ABC$ một góc $60^{\circ}$. Tính thể tích khối lăng trụ. 
	\choice
	{\True $V=\dfrac{h^3\sqrt{2}}{4}$}
	{$V=\dfrac{h^3\sqrt{2}}{3}$}
	{$V=\dfrac{h^3\sqrt{2}}{6}$}
	{$V=\dfrac{h^3\sqrt{2}}{3}$}
\end{ex}

\begin{dang}{Khối lăng trụ xiên}
\end{dang}
\begin{vd}%[2H1B3-2]%Ví dụ 1.
	Cho lăng trụ xiên tam giác $ABC.A'B'C'$ có đáy $ABC$ là tam giác đều cạnh $a$, biết cạnh bên là $a\sqrt{3}$ và hợp với đáy $ABC$ một góc $60^{\circ}$. Tính thể tích lăng trụ.
	\loigiai{
		\immini{
			Ta có $C'H\perp(ABC)\Rightarrow CH$ là hình chiếu của $CC'$ trên $(ABC)$.\\
			Vậy góc $\left(CC',(ABC)\right)=\widehat{C'CH}=60^{\circ}$.\\
			$\Delta CHC'\Rightarrow C'H=CC'\cdot\sin 60^{\circ}=\dfrac{3a}{2}$.\\
			$S_{ABC}=\dfrac{a^2\sqrt{3}}{4}$. Vậy $V=S_{ABC}\cdot C'H=\dfrac{3a^3\sqrt{3}}{8}$.}
		{
			\begin{tikzpicture}[thick,>=stealth,x=1cm,y=1cm,scale=0.7] 
			\clip(-0.7,-2.0) rectangle (7.5,5.8);
			\coordinate (A) at (0,0);
			\coordinate (B) at (2,-1);
			\coordinate (C) at (5,0);
			\coordinate (A') at ($(A)+(1,5)$);
			\coordinate (B') at ($(B)+(1,5)$);
			\coordinate (C') at ($(C)+(1,5)$);
			\coordinate (H) at (6.5,-0.5);
			\draw (A) circle (1pt) ;
			\draw (B) circle (1pt) ;
			\draw (C) circle (1pt) ;
			\draw (A') circle (1pt) ;
			\draw (B') circle (1pt) ;
			\draw (C') circle (1pt) ;
			\draw (H) circle (1pt) ;
			\tkzDrawSegments [thick,black,smooth](A,B B,C A,A' B,B' C,C' A',B' B',C' C',A' C',H C,H)
			\tkzDrawSegments [thick,black,smooth,dashed](A,C)
			\tkzLabelPoints[below](A,B,C)
			\tkzLabelPoints[above](A',B',C')
			\tkzLabelPoints[right](H)
			\draw pic[draw,angle radius=4mm,angle eccentricity=1.5] {angle = H--C--C'}; 
			
			\end{tikzpicture}
		}
	}
\end{vd}

\begin{vd}%[2H1K3-2]%Ví dụ 2.
	Cho hình lăng trụ xiên tam giác $ABC.A'B'C'$ có đáy $ABC$ là tam giác đều cạnh $a$. Hình chiếu của $A'$ xuống $(ABC)$ là tâm $O$ đường tròn ngoại tiếp tam giác $ABC$ biết $AA'$ hợp với đáy $(ABC)$ một góc $60^{\circ}$.
	\begin{enumEX}[a)]{1}
	\item Chứng minh rằng $BB'C'C$ là hình chữ nhật.
	\item Tính thể tích lăng trụ.
	\end{enumEX}
	\loigiai{
		\immini{
		\begin{enumEX}[a)]{1}
		\item Ta có $A'O\perp(ABC)\Rightarrow OA$ là hình chiếu của $AA'$ trên $ABC$.\\
			Vậy góc $\left(AA',(ABC)\right)=\widehat{OAA'}=60^{\circ}$. Ta có $BB'C'C$ là hình bình hành (vì mặt bên của hình lăng trụ).\\
			$AO\perp BC$ tại trung điểm $H$ của $BC$ nên $BC\perp A'H$\\
			Suy ra $BC\perp(AA'H) \Rightarrow BC\perp AA'$ mà $AA'\parallel BB'$ nên $BC\perp BB'$.
			Vậy $BB'C'C$ là hình chữ nhật.\\
			\item $ABC$ đều nên $AO=\dfrac{2}{3}AH=\dfrac{2}{3}\cdot\dfrac{a\sqrt{3}}{2}=\dfrac{a\sqrt{3}}{3}$.\\
			$\Delta AOA'\Rightarrow A'O=AO\tan 50^{\circ}=a$. Vậy $V=S_{ABC}\cdot A'O=\dfrac{a^3\sqrt{3}}{4}$.
		\end{enumEX} 	
			}{
			\begin{tikzpicture}[thick,>=stealth,x=1cm,y=1cm,scale=0.7] 
			\clip(-0.8,-2.4) rectangle (7.5,6.5);
			\coordinate (A) at (0,0);
			\coordinate (B) at (1,-1.3);
			\coordinate (C) at (4.5,0);
			\tkzCentroid(A,B,C)
			\tkzGetPoint{O}
			\coordinate (A') at ($(O)+(0,5)$);
			\tkzDefPointBy[translation = from A to A'](B)
			\tkzGetPoint{B'}
			\tkzDefPointBy[translation = from A to A'](C)
			\tkzGetPoint{C'}
			\tkzDefMidPoint(B,C)\tkzGetPoint{M}
			\tkzDefMidPoint(B,A)\tkzGetPoint{N}
			\draw (A) circle (1pt) ;
			\draw (B) circle (1pt) ;
			\draw (C) circle (1pt) ;
			\draw (A') circle (1pt) ;
			\draw (B') circle (1pt) ;
			\draw (C') circle (1pt) ;
			\draw (O) circle (1pt) ;
			\tkzDrawSegments [thick,black,smooth](A,B B,C A,A' B,B' C,C' A',B' B',C' C',A')
			\tkzDrawSegments [thick,black,smooth,dashed](A,C A,M C,N A',O A',M)
			\tkzLabelPoints[below](A,B,C,O)
			\tkzLabelPoints[above](A',C')
			\tkzLabelPoints[above right](B')
			\tkzMarkAngles[size=0.4,fill=green,opacity=0.1](O,A,A')
			\tkzLabelAngle[pos=0.85,rotate=30](O,A,A'){\footnotesize $60^\circ$}%
			\tkzMarkRightAngles(M,O,A')
			\end{tikzpicture}
			
		}
	}
\end{vd}

\begin{vd}%[2H1G3-2]%Ví dụ 3.
	Cho hình hộp $ABCD.A'B'C'D'$ có đáy là hình chữ nhật với $AB=\sqrt{3},AD=\sqrt{7}$. Hai mặt bên $(ABB'A')$ và $(ADD'A')$ lần lượt tạo với đáy các góc $45^{\circ}$ và $60^{\circ}$. Tính thể tích khối hộp nếu biết cạnh bên bằng $1$.
	\loigiai{
		% \begin{center}
		% 	\begin{tikzpicture}[thick,>=stealth,x=1cm,y=1cm,scale=1.0] 
		% 	\clip(-0.5,-0.5) rectangle (8.0,6.5);
		% 	\coordinate (A) at (0,0);
		% 	\coordinate (B) at (4,0);
		% 	\coordinate (D) at (2.0,2);
		% 	\coordinate (C) at (6.0,2);
		% 	\coordinate (H) at (1.8,0.8);
		% 	\coordinate (A') at ($(H)+(0,3)$);
		% 	\tkzDefPointBy[translation = from A to A'](B)
		% 	\tkzGetPoint{B'}
		% 	\tkzDefPointBy[translation = from A to A'](C)
		% 	\tkzGetPoint{C'}
		% 	\tkzDefPointBy[translation = from A to A'](D)
		% 	\tkzGetPoint{D'}
		% 	\tkzDefLine[parallel = through H](B,A) \tkzGetPoint{I}
		% 	\tkzDefLine[parallel = through H](D,A) \tkzGetPoint{J}
		% 	\tkzInterLL(A,D)(H,I) \tkzGetPoint{N}
		% 	\tkzInterLL(A,B)(H,J) \tkzGetPoint{M}
		% 	\draw (A) circle (1pt) ;
		% 	\draw (B) circle (1pt) ;
		% 	\draw (C) circle (1pt) ;
		% 	\draw (D) circle (1pt) ;
		% 	\draw (A') circle (1pt) ;
		% 	\draw (B') circle (1pt) ;
		% 	\draw (C') circle (1pt) ;
		% 	\draw (D') circle (1pt) ;
		% 	\draw (H) circle (1pt) ;
		% 	\draw (M) circle (1pt) ;
		% 	\draw (N) circle (1pt) ;
		% 	\tkzDrawSegments [thick,black,smooth](A,B B,C A,A' B,B' C,C' D,D' A',B' B',C' C',D' D',A' M,A')
		% 	\tkzDrawSegments [thick,black,smooth,dashed](D,A D,C D,D' A',H H,M H,N N,A')
		% 	\tkzLabelPoints[below](A,B,C,D)
		% 	\tkzLabelPoints[above](A',C',B',D')
		% 	\tkzLabelPoints[right](H)
		% 	\tkzMarkAngles[size=0.3,fill=green,opacity=0.1](H,N,A')
		% 	\tkzMarkAngles[size=0.4,fill=green,opacity=0.1](H,M,A')
		% 	\tkzLabelAngle[pos=0.90,rotate=30](H,N,A'){\footnotesize $60^\circ$}%
		% 	\tkzLabelAngle[pos=0.85,rotate=30](H,M,A'){\footnotesize $45^\circ$}%
		% 	\tkzMarkRightAngles(M,H,A' N,H,A')
		% 	\end{tikzpicture}
		% \end{center}
		Kẻ $A'H\perp(ABCD),HM\perp AB, HN\perp AD\Rightarrow A'M\perp AB, A'N\perp AD$. \\
		$ \Rightarrow\widehat{A'MH}=45^{\circ},\widehat{A'NH}=60^{\circ} $. Đặt $A'H=x$. Khi đó 
		\[A'N=\dfrac{x}{\sin{60}^{\circ}}=\dfrac{2x}{\sqrt{3}}; AN=\sqrt{AA'^2-A'N^2}=\sqrt{\dfrac{3-4x^2}{3}}=HM.\]
		Mà $HM=x\cdot\cot 45^{\circ}=x$. Nghĩa là $x=\sqrt{\dfrac{3-4x^2}{3}}\Rightarrow x=\sqrt{\dfrac{3}{7}}$.\\
		Vậy $V_{ABCD.A'B'C'D'}=AB\cdot AD\cdot x=\sqrt{3}\cdot\sqrt{7}\cdot\sqrt{\dfrac{3}{7}}=3$.
	}
\end{vd}


\subsubsection{Câu hỏi trắc nghiệm}
\begin{ex}%[2H1Y3-2]%Câu 1.
	Cho lăng trụ $ABC.A'B'C'$ có đáy $ABC$ là tam giác đều cạnh $a$. Hình chiếu vuông góc của $A'$ lên mặt phẳng $(ABC)$ trùng với tâm $O$ của đường tròn ngoại tiếp tam giác $ABC$, biết $A'O=a$. Tính theo $a$ thể tích khối lăng trụ đã cho. 
	\choice
	{$V=\dfrac{a^3\sqrt{3}}{12}$}
	{\True $V=\dfrac{a^3\sqrt{3}}{4}$}
	{$V=\dfrac{a^3}{4}$}
	{$V=\dfrac{a^3}{6}$}
	\loigiai{
		\immini{
			Ta có $O$ là hình chiếu vuông góc của $A'$ lên $(ABC)$ nên $A'O$ là đường cao.\\
			Khi đó
			\[
			V = S_{\Delta ABC} \cdot A'O = \dfrac{a^2\sqrt{3}}{4}\cdot a = \dfrac{a^3\sqrt{3}}{4}.
			\]
		}{
			\begin{tikzpicture}[thick,>=stealth,x=1cm,y=1cm,scale=1.0] 
			\clip(-0.7,-1.8) rectangle (5.0,3.3);
			\coordinate (A) at (0,0);
			\coordinate (B) at (0.5,-1.3);
			\coordinate (C) at (3.0,0);
			\tkzCentroid(A,B,C)
			\tkzGetPoint{O}
			\coordinate (A') at ($(O)+(0,3)$);
			\tkzDefPointBy[translation = from A to A'](B)
			\tkzGetPoint{B'}
			\tkzDefPointBy[translation = from A to A'](C)
			\tkzGetPoint{C'}
			\tkzDefMidPoint(B,C)\tkzGetPoint{M}
			\tkzDefMidPoint(B,A)\tkzGetPoint{N}
			\draw (A) circle (1pt) ;
			\draw (B) circle (1pt) ;
			\draw (C) circle (1pt) ;
			\draw (A') circle (1pt) ;
			\draw (B') circle (1pt) ;
			\draw (C') circle (1pt) ;
			\draw (O) circle (1pt) ;
			\tkzDrawSegments [thick,black,smooth](A,B B,C A,A' B,B' C,C' A',B' B',C' C',A')
			\tkzDrawSegments [thick,black,smooth,dashed](A,C A,M C,N A',O A',M)
			\tkzLabelPoints[below](B,C,O)
			\tkzLabelPoints[left](A)
			\tkzLabelPoints[above](A',C')
			\tkzLabelPoints[above right](B')
			\tkzMarkRightAngles(A,O,A')
			\tkzLabelSegment[left](A',O){$a$}
			\end{tikzpicture}
		}
	}
\end{ex}

\begin{ex}%[2H1B3-2]%Câu 2.
	Cho hình lăng trụ có đáy là tam giác vuông tại $B, AB=a, BC=2a$. Hình chiếu vuông góc của $A'$ trên đáy $ABC$ là trung điểm $H$ của cạnh $AC$, đường thẳng $A'B$ tạo với đáy một góc $45^{\circ}$. Tính thể tích $V$ của khối lăng trụ. 
	\choice
	{$V=\dfrac{a^3\sqrt{5}}{6}$}
	{$V=\dfrac{a^3\sqrt{5}}{3}$}
	{\True $V=\dfrac{a^3\sqrt{5}}{2}$}
	{$V=a^3\sqrt{5}$}
	\loigiai{
		\immini{
			Diện tích mặt đáy 
			\[
			S_{\Delta ABC} = \dfrac{1}{2}AB\cdot BC = \dfrac{1}{2}\cdot a\cdot 2a= a^2.
			\]
			Ta có $(A'B,(ABC)) = \widehat{A'BH} = 45^\circ$.\\
			Mặt khác, $\Delta A'BH$ vuông cân nên\\
			$A'H=BH= \dfrac{AC}{2} =\dfrac{\sqrt{AB^2+BC^2}}{2} = \dfrac{\sqrt{5}a}{2}$.\\
			Thể tích khối lăng trụ là
			\[
			V=S_{\Delta ABC} \cdot A'H =a^2 \cdot \dfrac{\sqrt{5}a}{2} = \dfrac{\sqrt{5}a^3}{2}.
			\]
		}{
			\begin{tikzpicture}[thick,>=stealth,x=1cm,y=1cm,scale=1.0] 
			\clip(-0.5,-2.0) rectangle (5.0,4.0);
			\coordinate (A) at (0,0);
			\coordinate (C) at (1,-1.3);
			\coordinate (B) at (3.5,0);
			\tkzDefMidPoint(A,C)\tkzGetPoint{H}
			\coordinate (A') at ($(H)+(0,3)$);
			\tkzDefPointBy[translation = from A to A'](B)
			\tkzGetPoint{B'}
			\tkzDefPointBy[translation = from A to A'](C)
			\tkzGetPoint{C'}
			\draw (A) circle (1pt) ;
			\draw (B) circle (1pt) ;
			\draw (C) circle (1pt) ;
			\draw (A') circle (1pt) ;
			\draw (B') circle (1pt) ;
			\draw (C') circle (1pt) ;
			\draw (H) circle (1pt) ;
			\tkzDrawSegments [thick,black,smooth](A,C B,C A,A' B,B' C,C' A',B' B',C' C',A' A',H)
			\tkzDrawSegments [thick,black,smooth,dashed](A,B H,B A',B)
			\tkzLabelPoints[below](B,C)
			\tkzLabelPoints[above](A')
			\tkzLabelPoints[below left](C')
			\tkzLabelPoints[above right](B')
			\tkzLabelPoints[left](H,A)
			\tkzMarkAngles[size=0.4,fill=green,opacity=0.3](A',B,H)
			\tkzLabelAngle[pos=0.85,rotate=30](A',B,A){\footnotesize $45^\circ$}%
			\tkzMarkRightAngles(A,H,A')
			\end{tikzpicture}
		}
	}
\end{ex}

\begin{ex}%[2H1B3-2]%Câu 3.
	Một khối lăng trụ tam giác có các cạnh đáy bằng $13$, $14$, $15,$ cạnh bên tạo với mặt phẳng đáy một góc $30^{\circ}$ và có chiều dài bằng $8$. Tính thể tích $V$ của khối lăng trụ đã cho. 
	\choice
	{$V=340$}
	{\True $V=336$}
	{$V=274\sqrt{3}$}
	{$V=124\sqrt{3}$}
	\loigiai{
		\immini{
			Giả sử $ABC.A'B'C'$ là lăng trụ thỏa mãn yêu cầu bài toán.\\
			Ta có $p=\dfrac{13+14+15}{2}=21$\\
			Diện tích mặt đáy
			\[
			S_{\Delta ABC} = \sqrt{p(p-13)(p-14)(p-15)}=84.
			\]
			Gọi $H$ là hình chiếu của $A'$ xuống $(ABC)$. \\
			Ta có $(A'A, (ABC))=\widehat{A'AH}=30^\circ$.\\
			Suy ra $A'H =\sin 30^\circ \cdot A'A = \dfrac{1}{2}\cdot 8 =4$.\\
			Thể tích khối lăng trụ là
			\[
			V=S_{\Delta ABC} \cdot A'H = 84 \cdot 4 = 336.
			\]
		}{
			\begin{tikzpicture}[thick,>=stealth,x=1cm,y=1cm,scale=1.0] 
			\clip(-0.7,-1.8) rectangle (5.0,4.0);
			\coordinate (A) at (0,0);
			\coordinate (B) at (0.5,-1.3);
			\coordinate (C) at (3.0,0);
			\tkzCentroid(A,B,C)
			\tkzGetPoint{H}
			\coordinate (A') at ($(H)+(0,3)$);
			\tkzDefPointBy[translation = from A to A'](B)
			\tkzGetPoint{B'}
			\tkzDefPointBy[translation = from A to A'](C)
			\tkzGetPoint{C'}
			\tkzDefMidPoint(B,C)\tkzGetPoint{M}
			\tkzDefMidPoint(B,A)\tkzGetPoint{N}
			\draw (A) circle (1pt) ;
			\draw (B) circle (1pt) ;
			\draw (C) circle (1pt) ;
			\draw (A') circle (1pt) ;
			\draw (B') circle (1pt) ;
			\draw (C') circle (1pt) ;
			\draw (H) circle (1pt) ;
			\tkzDrawSegments [thick,black,smooth](A,B B,C A,A' B,B' C,C' A',B' B',C' C',A')
			\tkzDrawSegments [thick,black,smooth,dashed](A,C A',H A,H)
			\tkzLabelPoints[below](B,C,H)
			\tkzLabelPoints[left](A)
			\tkzLabelPoints[above](A',C')
			\tkzLabelPoints[above right](B')
			\tkzMarkRightAngles(A,H,A')
			\tkzLabelSegment[left](A',A){$8$}
			\tkzLabelSegment[left](A,B){$13$}
			\tkzLabelSegment[below right](B,C){$14$}
			\tkzLabelSegment[above right](A,C){$15$}
			\tkzMarkAngles[size=0.4,fill=green,opacity=0.3](H,A,A')
			\tkzLabelAngle[pos=0.85,rotate=10](H,A,A'){\footnotesize $30^\circ$}%
			\end{tikzpicture}
		}
	}
\end{ex}

\begin{ex}%[2H1K3-2]%Câu 4.
	Cho hình lăng trụ có đáy $ABCD.A'B'C'D'$ là hình chữ nhật với $AB=a, AD=a\sqrt{3}$ và $A'B=3a$. Hình chiếu vuông góc của điểm $A'$ trên mặt phẳng $(ABCD)$ trùng với tâm $O$ của hình chữ nhật $ABCD$. Tính thể tích $V$ của khối lăng trụ $ABCD.A'B'C'D'$ . 
	\choice
	{\True $V=2a^3\sqrt{6}$}
	{$V=a^3\sqrt{6}$}
	{$V=\dfrac{2}{3}a^3\sqrt{6}$}
	{$V=6a^3\sqrt{2}$}
	\loigiai{
		\immini{
			Diện tích mặt đáy
			\[
			S_{ABCD} = AB\cdot AD = a\cdot a\sqrt{3}=\sqrt{3}a^2.
			\]
			Mặt khác, $BD= \sqrt{AB^2+AD^2}=\sqrt{a^2+3a^2}=2a$.\\
			Xét tam giác vuông $A'OB$ vuông tại $B$ ta có
			\[
			A'O = \sqrt{A'B^2-OB^2} = \sqrt{9a^2 -a^2}=2\sqrt{2}a.
			\]
			Thể tích lăng trụ là
			\[
			V=S_{ABCD} \cdot A'O =\sqrt{3}a^2\cdot 2\sqrt{2}a = 2\sqrt{6}a^3.
			\]
		}{
			\begin{tikzpicture}[thick,>=stealth,x=1cm,y=1cm,scale=1.0] 
			\clip(-0.5,-0.5) rectangle (6.5,5.5);
			\coordinate (A) at (0,0);
			\coordinate (B) at (3,0);
			\coordinate (D) at (1.0,1.3);
			\coordinate (C) at (4.0,1.3);
			\tkzInterLL(A,C)(B,D) \tkzGetPoint{O}
			\coordinate (A') at ($(O)+(0,2.5)$);
			\tkzDefPointBy[translation = from A to A'](B)
			\tkzGetPoint{B'}
			\tkzDefPointBy[translation = from A to A'](C)
			\tkzGetPoint{C'}
			\tkzDefPointBy[translation = from A to A'](D)
			\tkzGetPoint{D'}
			\draw (A) circle (1pt) ;
			\draw (B) circle (1pt) ;
			\draw (C) circle (1pt) ;
			\draw (D) circle (1pt) ;
			\draw (A') circle (1pt) ;
			\draw (B') circle (1pt) ;
			\draw (C') circle (1pt) ;
			\draw (D') circle (1pt) ;
			\draw (O) circle (1pt) ;
			\tkzDrawSegments [thick,black,smooth](A,B B,C A,A' B,B' C,C' A',B' B',C' C',D' D',A' A',B)
			\tkzDrawSegments [thick,black,smooth,dashed](D,A D,C D,D' A',O D,D' A,C B,D)
			\tkzLabelPoints[below](A,B,C,D)
			\tkzLabelPoints[above](A',C',B',D')
			\tkzLabelPoints[below](O)
			\tkzLabelSegment[below](A,B){$a$}
			\tkzLabelSegment[above left](A,D){$a\sqrt{3}$}
			\tkzLabelSegment[right](A',B){$3a$}
			\tkzMarkRightAngles(A',O,A)
			\end{tikzpicture}
		}
	}
\end{ex}

\begin{ex}%[2H1K3-2]%Câu 5.
	Cho hình lăng trụ có đáy $ABC.A'B'C'$ là tam giác đều cạnh $a$, hình chiếu của $C'$ trên $(ABC)$ là trung điểm $I$ của $BC$. Góc giữa $AA'$ và $BC$ là $30^{\circ}$. Thể tích của khối lăng trụ $ABC.A'B'C'$ là
	\choice
	{$\dfrac{a^3}{3}$ }
	{$\dfrac{a^3}{4}$ }
	{$\dfrac{a^3}{6}$ }
	{\True $\dfrac{a^3}{8}$}
	\loigiai{
		\immini{
			Diện tích mặt đáy $S_{\Delta ABC} = \dfrac{a^2\sqrt{3}}{4}$.\\
			Ta có $(AA',BC)=(CC',BC) = \widehat{BCC'}=30^\circ$.\\
			Xét tam giác $C'IC$ vuông tại $I$ ta có
			\[
			\tan \widehat{BCC'} = \dfrac{C'I}{IC} \Rightarrow C'I = \tan \widehat{BCC'} \cdot IC = \tan 30^\circ \cdot \dfrac{a}{2} =\dfrac{\sqrt{3}}{6}a.
			\]
			Thể tích khối lăng trụ là
			\[
			V= S_{\Delta ABC} \cdot C'I =\dfrac{a^2\sqrt{3}}{4} \cdot \dfrac{\sqrt{3}}{6}a = \dfrac{a^3}{8}.
			\]
		}{
			\begin{tikzpicture}[thick,>=stealth,x=1cm,y=1cm,scale=1.0] 
			\clip(-0.7,-1.8) rectangle (5.0,4.0);
			\coordinate (C) at (0,0);
			\coordinate (B) at (1.5,-1.3);
			\coordinate (A) at (3.5,0);
			\tkzDefMidPoint(B,C)\tkzGetPoint{I}
			\coordinate (C') at ($(I)+(0,3)$);
			\tkzDefPointBy[translation = from C to C'](B)
			\tkzGetPoint{B'}
			\tkzDefPointBy[translation = from C to C'](A)
			\tkzGetPoint{A'}
			\draw (A) circle (1pt) ;
			\draw (B) circle (1pt) ;
			\draw (C) circle (1pt) ;
			\draw (A') circle (1pt) ;
			\draw (B') circle (1pt) ;
			\draw (C') circle (1pt) ;
			\draw (I) circle (1pt) ;
			\tkzDrawSegments [thick,black,smooth](A,B B,C A,A' B,B' C,C' A',B' B',C' C',A' C',I)
			\tkzDrawSegments [thick,black,smooth,dashed](A,C)
			\tkzLabelPoints[below](B)
			\tkzLabelPoints[left](C,I)
			\tkzLabelPoints[right](A)
			\tkzLabelPoints[above](A',C',B')
			\tkzMarkRightAngles(C,I,C')
			\tkzMarkAngles[size=0.3,fill=green,opacity=0.3](I,C,C')
			\tkzLabelAngle[pos=0.60,rotate=10](I,C,C'){\footnotesize $30^\circ$}%
			\end{tikzpicture}
		}
	}
\end{ex}

\begin{ex}%[2H1K3-2]%Câu 6.
	Cho lăng trụ tam giác $ABC.A'B'C'$ có đáy $ABC$ là một tam giác đều cạnh $a$ và điểm $A'$ cách đều các điểm $A,B,C$. Cạnh bên $AA'$ tạo với mp đáy một góc $60^{\circ}$. Tính thể tích của lăng trụ. 
	\choice
	{$\dfrac{a^3\sqrt{3}}{12}$}
	{$\dfrac{a^3\sqrt{3}}{6}$}
	{$\dfrac{a^3\sqrt{3}}{3}$}
	{\True $\dfrac{a^3\sqrt{3}}{4}$}
	\loigiai{
		\immini{
			Diện tích mặt đáy $S_{\Delta ABC} = \dfrac{a^2\sqrt{3}}{4}$.\\
			Vì $A'$ cách đều $A, B, C$ nên hình chóp $A'.ABC$ đều do đó hình chiếu $A'$ lên mặt đáy trùng với trọng tâm $G$ của $\Delta ABC$.\\
			Khi đó $(AA',(ABC))=\widehat{A'AG}=60^\circ$.\\
			Xét $\Delta A'AG$ vuông tại $G$ ta có
			\[
			A'G = \tan \widehat{A'AG}\cdot AG = \tan 60^\circ \cdot \dfrac{a\sqrt{3}}{3}=a.
			\]
			Thể tích lăng trụ $ABC.A'B'C'$ là
			\[
			V=S_{\Delta ABC} \cdot A'G = \dfrac{a^2\sqrt{3}}{4}\cdot a = \dfrac{a^3\sqrt{3}}{4}.
			\]
		}{
			\begin{tikzpicture}[thick,>=stealth,x=1cm,y=1cm,scale=1.0] 
			\clip(-0.7,-1.8) rectangle (5.0,4.0);
			\coordinate (A) at (0,0);
			\coordinate (B) at (0.5,-1.3);
			\coordinate (C) at (3.0,0);
			\tkzCentroid(A,B,C)
			\tkzGetPoint{G}
			\coordinate (A') at ($(G)+(0,3)$);
			\tkzDefPointBy[translation = from A to A'](B)
			\tkzGetPoint{B'}
			\tkzDefPointBy[translation = from A to A'](C)
			\tkzGetPoint{C'}
			\tkzDefMidPoint(B,C)\tkzGetPoint{M}
			\tkzDefMidPoint(B,A)\tkzGetPoint{N}
			\draw (A) circle (1pt) ;
			\draw (B) circle (1pt) ;
			\draw (C) circle (1pt) ;
			\draw (A') circle (1pt) ;
			\draw (B') circle (1pt) ;
			\draw (C') circle (1pt) ;
			\draw (G) circle (1pt) ;
			\tkzDrawSegments [thick,black,smooth](A,B B,C A,A' B,B' C,C' A',B' B',C' C',A')
			\tkzDrawSegments [thick,black,smooth,dashed](A,C A',G A,G)
			\tkzLabelPoints[below](B,C,G)
			\tkzLabelPoints[left](A)
			\tkzLabelPoints[above](A',C')
			\tkzLabelPoints[above right](B')
			\tkzMarkRightAngles(A,G,A')
			\tkzMarkAngles[size=0.4,fill=green,opacity=0.3](G,A,A')
			\tkzLabelAngle[pos=0.85,rotate=10](G,A,A'){\footnotesize $60^\circ$}%
			\end{tikzpicture}
		}
	}
\end{ex}

\begin{ex}%[2H1K3-2]%Câu 7.
	Cho hình hộp $ABCD.A'B'C'D'$ có đáy là hình thoi cạnh $a$, góc $\widehat{A}=60^{\circ}$. Chân đường vuông góc hạ từ $B'$ xuống đáy $ABCD$ trùng với giao điểm hai đường chéo của đáy. Cho $BB'=a$. Tính thể tích hình hộp $ABCD.A'B'C'D'$ bằng
	\choice
	{\True $\dfrac{3a^3}{4}$}
	{$\dfrac{a^3\sqrt{3}}{6}$}
	{$\dfrac{a^3}{4}$}
	{$\dfrac{a^3\sqrt{3}}{4}$}
	\loigiai{
		\immini{
			Ta có đáy là hình thoi $ABCD$ và $\widehat{A}=60^\circ$ nên $\Delta BAC$ và $\Delta DBC$ đều.\\
			Diện tích mặt đáy là
			$S_{ABCD}=2S_{\Delta BAC} = 2\cdot \dfrac{a^2\sqrt{3}}{4} = \dfrac{a^2\sqrt{3}}{2}$.\\
			Xét $\Delta BOB'$ vuông taị $O$ ta có
			\[
			B'O = \sqrt{BB'^2-BO^2} = \sqrt{a^2 - \dfrac{a^2}{4}} = \dfrac{\sqrt{3}a}{2}.
			\]
			Thể tích khối hộp là
			\[
			V=S_{ABCD}\cdot B'O =  \dfrac{a^2\sqrt{3}}{2} \cdot  \dfrac{\sqrt{3}a}{2}= \dfrac{3a^3}{4}.
			\]
		}{
			\begin{tikzpicture}[thick,>=stealth,x=1cm,y=1cm,scale=1.0] 
			\clip(-0.5,-0.5) rectangle (6.5,5.5);
			\coordinate (B) at (0,0);
			\coordinate (C) at (3,0);
			\coordinate (A) at (1.0,1.3);
			\coordinate (D) at (4.0,1.3);
			\tkzInterLL(A,C)(B,D) \tkzGetPoint{O}
			\coordinate (B') at ($(O)+(0,2.5)$);
			\tkzDefPointBy[translation = from B to B'](A)
			\tkzGetPoint{A'}
			\tkzDefPointBy[translation = from B to B'](C)
			\tkzGetPoint{C'}
			\tkzDefPointBy[translation = from B to B'](D)
			\tkzGetPoint{D'}
			\draw (A) circle (1pt) ;
			\draw (B) circle (1pt) ;
			\draw (C) circle (1pt) ;
			\draw (D) circle (1pt) ;
			\draw (A') circle (1pt) ;
			\draw (B') circle (1pt) ;
			\draw (C') circle (1pt) ;
			\draw (D') circle (1pt) ;
			\draw (O) circle (1pt) ;
			\tkzDrawSegments [thick,black,smooth](B,C B,B' C,C' A',B' B',C' C',D' D',A' D,C D,D')
			\tkzDrawSegments [thick,black,smooth,dashed](D,A D,C D,D' B',O D,D' A,C B,D A,A' A,B)
			\tkzLabelPoints[below](A,B,C,D)
			\tkzLabelPoints[above](A',C',B',D')
			\tkzLabelPoints[below](O)
			\tkzMarkRightAngles(B',O,A)
			\tkzLabelSegment[left](B,B'){$a$}
			\end{tikzpicture}
		}
	}
\end{ex}

\begin{ex}%[2H1K3-2]%Câu 8.
	Cho $(H)$ lăng trụ xiên $ABC.A'B'C'$ đáy là tam giác đều cạch $a$, hình chiếu vuông góc $A'$ lên đáy trùng với tâm đường tròn ngoại tiếp tam giác $ABC$ và $A'A$ hợp đáy bằng $60^{\circ}$. Thể tích của $(H)$ bằng
	\choice
	{$3\sqrt{6}a^3$}
	{$\dfrac{\sqrt{3}a^3}{6}$}
	{\True $\dfrac{\sqrt{3}a^3}{4}$}
	{$\dfrac{3\sqrt{3}a^3}{4}$}
	\loigiai{
		\immini{
			Diện tích mặt đáy $S_{\Delta ABC} = \dfrac{a^2\sqrt{3}}{4}$.\\
			Do $\Delta ABC$ đều nên trọng tâm $G$ là tâm đường tròn ngoaị tiếp.\\
			Khi đó, $(A'A,(ABC)) = \widehat{A'AG}=60^\circ$.\\
			Xét $\Delta A'AG$ vuông tại $G$ ta có
			\[
			A'G = \tan \widehat{A'AG}\cdot AG = \tan 60^\circ \cdot \dfrac{a\sqrt{3}}{3}=a.
			\]
			Thể tích khối lăng trụ là
			\[
			V=S_{\Delta ABC} \cdot A'G = \dfrac{a^2\sqrt{3}}{4}\cdot a =\dfrac{\sqrt{3}a^3}{4}.
			\]
		}{
			\begin{tikzpicture}[thick,>=stealth,x=1cm,y=1cm,scale=1.0] 
			\clip(-0.7,-1.8) rectangle (5.0,4.0);
			\coordinate (A) at (0,0);
			\coordinate (B) at (0.5,-1.3);
			\coordinate (C) at (3.0,0);
			\tkzCentroid(A,B,C)
			\tkzGetPoint{G}
			\coordinate (A') at ($(G)+(0,3)$);
			\tkzDefPointBy[translation = from A to A'](B)
			\tkzGetPoint{B'}
			\tkzDefPointBy[translation = from A to A'](C)
			\tkzGetPoint{C'}
			\tkzDefMidPoint(B,C)\tkzGetPoint{M}
			\tkzDefMidPoint(B,A)\tkzGetPoint{N}
			\draw (A) circle (1pt) ;
			\draw (B) circle (1pt) ;
			\draw (C) circle (1pt) ;
			\draw (A') circle (1pt) ;
			\draw (B') circle (1pt) ;
			\draw (C') circle (1pt) ;
			\draw (G) circle (1pt) ;
			\tkzDrawSegments [thick,black,smooth](A,B B,C A,A' B,B' C,C' A',B' B',C' C',A')
			\tkzDrawSegments [thick,black,smooth,dashed](A,C A',G A,G)
			\tkzLabelPoints[below](B,C,G)
			\tkzLabelPoints[left](A)
			\tkzLabelPoints[above](A',C')
			\tkzLabelPoints[above right](B')
			\tkzMarkRightAngles(A,G,A')
			\tkzMarkAngles[size=0.4,fill=green,opacity=0.3](G,A,A')
			\tkzLabelAngle[pos=0.85,rotate=10](G,A,A'){\footnotesize $60^\circ$}%
			\end{tikzpicture}
		}
	}
\end{ex}

\begin{ex}%[2H1K3-2]%Câu 9.
	Cho $(H)$ lăng trụ xiên $ABC.A'B'C'$ đáy là tam giác đều cạnh $a$, cạnh bên bằng $a\sqrt{3}$ và hợp đáy bằng $60^{\circ}$. Thể tích của $(H)$ bằng
	\choice
	{$3\sqrt{6}a^3$}
	{$\dfrac{3\sqrt{3}a^3}{6}$}
	{$\dfrac{\sqrt{3}a^3}{2}$}
	{\True $\dfrac{3\sqrt{3}a^3}{8}$}
	\loigiai{
		\immini{
			Diện tích mặt đáy $S_{\Delta ABC} = \dfrac{a^2\sqrt{3}}{4}$.\\
			Gọi $H$ là hình chiếu vuông góc của $A'$ lên $(ABC)$.\\
			Khi đó $(AA',(ABC))=\widehat{A'AH}=60^\circ$.\\
			Xét $\Delta A'AH$ vuông tại $H$, ta có
			\[
			A'H =\sin \widehat{A'AH} \cdot A'A = \sin 60^\circ \cdot a\sqrt{3} = \dfrac{\sqrt{3}}{2}\cdot a\sqrt{3} = \dfrac{3a}{2}.
			\]
			Thể tích khối lăng trụ $ABC.A'B'C'$ là
			\[
			V = S_{ABC}\cdot A'H = \dfrac{a^2\sqrt{3}}{4} \cdot \dfrac{3a}{2} = \dfrac{3\sqrt{3}a^3}{8}.
			\]
		}{
			\begin{tikzpicture}[thick,>=stealth,x=1cm,y=1cm,scale=1.0] 
			\clip(-0.7,-1.8) rectangle (5.0,4.0);
			\coordinate (A) at (0,0);
			\coordinate (B) at (0.5,-1.3);
			\coordinate (C) at (3.0,0);
			\tkzCentroid(A,B,C)
			\tkzGetPoint{H}
			\coordinate (A') at ($(H)+(0,3)$);
			\tkzDefPointBy[translation = from A to A'](B)
			\tkzGetPoint{B'}
			\tkzDefPointBy[translation = from A to A'](C)
			\tkzGetPoint{C'}
			\tkzDefMidPoint(B,C)\tkzGetPoint{M}
			\tkzDefMidPoint(B,A)\tkzGetPoint{N}
			\draw (A) circle (1pt) ;
			\draw (B) circle (1pt) ;
			\draw (C) circle (1pt) ;
			\draw (A') circle (1pt) ;
			\draw (B') circle (1pt) ;
			\draw (C') circle (1pt) ;
			\draw (H) circle (1pt) ;
			\tkzDrawSegments [thick,black,smooth](A,B B,C A,A' B,B' C,C' A',B' B',C' C',A')
			\tkzDrawSegments [thick,black,smooth,dashed](A,C A',H A,H)
			\tkzLabelPoints[below](B,C,H)
			\tkzLabelPoints[left](A)
			\tkzLabelPoints[above](A',C')
			\tkzLabelPoints[above right](B')
			\tkzMarkRightAngles(A,H,A')
			\tkzMarkAngles[size=0.4,fill=green,opacity=0.3](H,A,A')
			\tkzLabelAngle[pos=0.85,rotate=10](H,A,A'){\footnotesize $60^\circ$}%
			\tkzLabelSegment[left](A,A'){$a\sqrt{3}$}
			\end{tikzpicture}
		}
	}
\end{ex}

\begin{ex}%[2H1G3-2]%Câu 10.
	Cho hình lăng trụ $ABC.A'B'C'$ có độ dài cạnh bên bằng $2a$, đáy $ABC$ là tam giác vuông tại $A, AB=a$, $AC=a\sqrt{3}$ và hình chiếu vuông góc của đỉnh $A'$ trên $mp(ABC)$ là trung điểm của cạnh $BC$. Tính theo $a$ thể tích của khối chóp $A'.ABC$ và tính côsin của góc giữa hai đường thẳng $AA'$ và $B'C'$. 
	\choice
	{\True $V=\dfrac{a^3}{2},\cos\varphi=\dfrac{1}{4}$}
	{$V=\dfrac{a^3}{3},\cos\varphi=\dfrac{1}{4}$}
	{$V=\dfrac{a^3}{2},\cos\varphi=\dfrac{1}{2}$}
	{$V=\dfrac{a^3}{3},\cos\varphi=\dfrac{1}{2}$}
	\loigiai{
		\immini{
			Diện tích mặt đáy
			\[
			S_{\Delta ABC} = \dfrac{1}{2}AB\cdot AC = \dfrac{1}{2}a\cdot a\sqrt{3} =\dfrac{a^2\sqrt{3}}{2}.
			\]
			Đồng thời, $BC=\sqrt{AB^2+AC^2}=\sqrt{a^2+3a^2}=2a$.\\ 
			Gọi $H$ là trung điểm $BC$, xét tam giác $A'HA$ vuông tại $H$ ta có
			\[
			A'H = \sqrt{AA'^2-AH^2} = \sqrt{4a^2-a^2} = \sqrt{3}a.
			\] 
			Thể tích khối chóp $A'.ABC$ là
			\[
			V=\dfrac{1}{3}S_{\Delta ABC} \cdot A'H =\dfrac{1}{3} \dfrac{a^2\sqrt{3}}{2} \cdot a\sqrt{3} = \dfrac{a^3}{2}.
			\]
			Ta có $(AA',B'C')=(BB',BC) =\widehat{B'BC}$\\
			Xét $\Delta HA'B$ vuông tại $A'$ ta có:
			\[
			B'H =\sqrt{A'H^2+A'B'^2}=\sqrt{3a^2+a^2}=2a.
			\]
			Khi đó $\cos \widehat{B'BC}=\cos \widehat{B'BH} = \dfrac{4a^2+a^2-4a^2}{2\cdot 2a \cdot a}=\dfrac{1}{4}$.
		}{
			\begin{tikzpicture}[thick,>=stealth,x=1cm,y=1cm,scale=1.0] 
			\clip(-0.7,-1.8) rectangle (5.3,5.0);
			\coordinate (B) at (0,0);
			\coordinate (A) at (1,-1.3);
			\coordinate (C) at (4.0,0);
			\tkzDefMidPoint(B,C)\tkzGetPoint{H}
			\coordinate (A') at ($(H)+(0,3)$);
			\tkzDefPointBy[translation = from A to A'](B)
			\tkzGetPoint{B'}
			\tkzDefPointBy[translation = from A to A'](C)
			\tkzGetPoint{C'}
			\draw (A) circle (1pt) ;
			\draw (B) circle (1pt) ;
			\draw (C) circle (1pt) ;
			\draw (A') circle (1pt) ;
			\draw (B') circle (1pt) ;
			\draw (C') circle (1pt) ;
			\draw (H) circle (1pt) ;
			\tkzDrawSegments [thick,black,smooth](A,B A,C A,A' B,B' C,C' A',B' B',C' C',A')
			\tkzDrawSegments [thick,black,smooth,dashed](B,C A',H A,H)
			\tkzLabelPoints[below](C,H)
			\tkzLabelPoints[left](A,B)
			\tkzLabelPoints[above](A',C')
			\tkzLabelPoints[above right](B')
			\tkzMarkRightAngles(A,H,A' B,A,C)
			%\tkzMarkAngles[size=0.4,fill=green,opacity=0.3](H,A,A')
			%\tkzLabelAngle[pos=0.85,rotate=10](H,A,A'){\footnotesize $60^\circ$}%
			\tkzLabelSegment[left](A,A'){$2a$}
			\end{tikzpicture}
		}
	}
\end{ex}

\begin{dang}{Khối chóp có cạnh bên vuông góc với đáy}
\end{dang}
\begin{vd}%[2H1B3-2]%Ví dụ 1.
	Cho hình chóp $S.ABC$ có $SB=SC=BC=CA=a$. Hai mặt $(ABC)$ và $(ASC)$ cùng vuông góc với $(SBC)$. Tính thể tích hình chóp.
	\loigiai{
		\immini{
			Ta có\\
			$\heva{&(ABC)\perp(SBC)\\&(ASC)\perp(SBC)}\Rightarrow AC\perp(SBC)$.\\
			Do đó $V=\dfrac{1}{3}S_{SBC}\cdot AC=\dfrac{1}{3}\dfrac{a^2\sqrt{3}}{4}\cdot a=\dfrac{a^3\sqrt{3}}{12}$.}
		{
			\begin{tikzpicture}[thick,>=stealth,x=1cm,y=1cm,scale=0.8] 
			\clip(-0.5,-2.0) rectangle (4.5,3.8);
			\coordinate (C) at (0,0);
			\coordinate (S) at (3,-1);
			\coordinate (B) at (4,0);
			\coordinate (A) at (0,3);
			\draw (A) circle (1pt) ;
			\draw (B) circle (1pt) ;
			\draw (C) circle (1pt) ;
			\draw (S) circle (1pt) ;
			\tkzDrawSegments [thick,black,smooth](A,C C,S S,B B,A A,S)
			\tkzDrawSegments [thick,black,smooth,dashed](C,B)
			\tkzLabelPoints[below](C,S,B)
			\tkzLabelPoints[above](A)
			\tkzMarkRightAngles(A,C,S A,C,B)
			\tkzLabelSegment[left](C,A){$a$}
			\tkzLabelSegment[below](C,S){$a$}
			\tkzLabelSegment[above left](C,B){$a$}
			\end{tikzpicture}
		}
	}
\end{vd}

\begin{vd}%[2H1B3-2]%Ví dụ 2.
	Cho hình chóp $S.ABC$ có đáy $ABC$ là tam giác vuông cân tại $B$ với $AC=a$, biết $SA$ vuông góc với đáy $ABC$ và $SB$ hợp với đáy một góc $60^{\circ}$.
	\begin{enumEX}[a)]{1}
	\item Chứng minh các mặt bên là tam giác vuông.
	\item Tính thể tích hình chóp.
	\end{enumEX}
	\loigiai{
		\immini{
		\begin{enumEX}[a)]{1}
		\item $SA\perp(ABC)\Rightarrow SA\perp AB$ và $SA\perp AC$ mà $BC\perp AB\Rightarrow BC\perp SB$. Vậy các mặt bên của hình chóp là tam giác vuông.\\
		\item Ta có $SA\perp(ABC)\Rightarrow AB$ là hình chiếu của $SB$ trên $(ABC)$.\\
			Vậy góc $\left(SB,(ABC)\right)=\widehat{SAB}=60^{\circ}$.\\
			$\Delta ABC$ vuông cân nên $BA=BC=\dfrac{a}{\sqrt{2}}\cdot S_{ABC}=\dfrac{1}{2}BA\cdot BC=\dfrac{a^2}{4}$.\\
			$\Delta SAB\Rightarrow SA=AB\cdot\tan 60^{\circ}=\dfrac{a\sqrt{6}}{2}$.\\
			Vậy $V=\dfrac{1}{3}S_{ABC}\cdot SA=\dfrac{1}{3}\cdot\dfrac{a^2}{4}\cdot\dfrac{a\sqrt{6}}{2}=\dfrac{a^3\sqrt{6}}{24}$.
		\end{enumEX}	
			}
		{
			% \begin{tikzpicture}[thick,>=stealth,x=1cm,y=1cm,scale=1.0] 
			% \clip(-0.5,-2.0) rectangle (4.5,3.7);
			% \coordinate (A) at (0,0);
			% \coordinate (B) at (3,-1);
			% \coordinate (C) at (4,0);
			% \coordinate (S) at (0,3);
			% \draw (A) circle (1pt) ;
			% \draw (B) circle (1pt) ;
			% \draw (C) circle (1pt) ;
			% \draw (S) circle (1pt) ;
			% \tkzDrawSegments [thick,black,smooth](C,S S,B B,A A,S B,C)
			% \tkzDrawSegments [thick,black,smooth,dashed](A,C)
			% \tkzLabelPoints[below](C,B)
			% \tkzLabelPoints[left](A,S)
			% \tkzMarkRightAngles(S,A,C S,A,B A,B,C)
			% \tkzLabelSegment[above left](A,C){$a$}
			% \tkzMarkAngles[size=0.4,fill=green,opacity=0.3](S,B,A)
			% \tkzLabelAngle[pos=0.80,rotate=10](S,B,A){\footnotesize $60^\circ$}%
			% \end{tikzpicture}
		}
	}
\end{vd}

\begin{vd}%[2H1K3-2]%Ví dụ 3.
	Cho hình chóp $S.ABC$ có đáy $ABC$ là tam giác đều cạnh $a$. $SA$ vuông góc với đáy $(ABC)$ và $(SBC)$ hợp với đáy $(ABC)$ một góc $60^{\circ}$. Tính thể tích hình chóp.
	\loigiai{
		\immini{
			Gọi $M$ là trung điểm của $BC$, vì tam giác $ABC$ đều\\
			nên $AM\perp BC\Rightarrow SA\perp BC$.\\
			Vậy góc $\left((SBC),(ABC)\right)=\widehat{SMA}=60^{\circ}$. \\
			Mặt khác, $\Delta SAM\Rightarrow SA=AM\cdot\tan 60^{\circ}=\dfrac{3a}{2}$.\\
			Vậy $V=\dfrac{1}{3}S_{ABC}\cdot SA=\dfrac{a^3\sqrt{3}}{8}$.}
		{
			\begin{tikzpicture}[thick,>=stealth,x=1cm,y=1cm,scale=1.0] 
			\clip(-0.5,-2.0) rectangle (4.5,3.7);
			\coordinate (A) at (0,0);
			\coordinate (B) at (3,-1);
			\coordinate (C) at (4,0);
			\coordinate (S) at (0,3);
			\tkzDefMidPoint(B,C)\tkzGetPoint{M}
			\draw (A) circle (1pt) ;
			\draw (B) circle (1pt) ;
			\draw (C) circle (1pt) ;
			\draw (S) circle (1pt) ;
			\tkzDrawSegments [thick,black,smooth](C,S S,B B,A A,S B,C S,M)
			\tkzDrawSegments [thick,black,smooth,dashed](A,C A,M)
			\tkzLabelPoints[below](C,B)
			\tkzLabelPoints[left](A,S)
			\tkzLabelPoints[below right](M)
			\tkzMarkRightAngles(S,A,C S,A,B A,M,B)
			\tkzLabelSegment[below](A,B){$a$}
			\tkzMarkAngles[size=0.4,fill=green,opacity=0.1](S,M,A)
			\tkzLabelAngle[pos=0.80,rotate=10](S,M,A){\footnotesize $60^\circ$}%
			\end{tikzpicture}
		}
	}
\end{vd}

\begin{vd}%[2H1K3-2]%Ví dụ 4.
	Cho hình chóp $S.ABCD$ có đáy $ABCD$ là hình vuông cạnh $a$ biết $SA$ vuông góc với đáy $ABCD$ và mặt bên $(SCD)$ hợp với đáy một góc $60^{\circ}$.
	\begin{enumEX}[a)]{1}
	\item Tính thể tích hình chóp $S.ABCD$.
	\item Tính khoảng cách từ $A$ đến mặt phẳng $(SCD)$.
	\end{enumEX}
	\loigiai{
		\immini{
		\begin{enumEX}[a)]{1}
	\item Ta có $SA\perp(ABC)$ và $CD\perp AD\Rightarrow CD\perp SD$.\\
			Vậy góc $\left((SCD),(ABCD)\right)=\widehat{SDA}=60^{\circ}$.\\
			$\Delta SAD$ vuông nên $SA=AD\cdot\tan 60^{\circ}=a\sqrt{3}$.\\
			Vậy $V=\dfrac{1}{3}S_{ABCD}\cdot SA=\dfrac{1}{3}a^2\cdot a\sqrt{3}=\dfrac{a^3\sqrt{3}}{3}$.\\
	\item Ta dựng $AH\perp SD$, vì $CD\perp(SAD)$ nên $CD\perp AH\Rightarrow AH\perp(SCD)$. 
			Vậy $AH$ là khoảng cách từ $A$ đến $(SCD)$.\\
			$\Delta SAD\Rightarrow\dfrac{1}{AH^2}=\dfrac{1}{SA^2}+\dfrac{1}{AD^2}=\dfrac{1}{3a^2}+\dfrac{1}{a^2}=\dfrac{4}{3a^2}$. Vậy $AH=\dfrac{a\sqrt{3}}{2}$.
		\end{enumEX}	
			}
		{
			% \begin{tikzpicture}[thick,>=stealth,x=1cm,y=1cm,scale=0.8] 
			% \clip(-1.5,-2.3) rectangle (4.8,4.0);
			% \coordinate (A) at (0,0);
			% \coordinate (B) at (-1,-1.5);
			% \coordinate (D) at (4,0);
			% \coordinate (C) at (3,-1.5);
			% \coordinate (S) at ($(A)+(0,3)$);
			% \draw (A) circle (1pt) ;
			% \draw (B) circle (1pt) ;
			% \draw (C) circle (1pt) ;
			% \draw (D) circle (1pt) ;
			% \draw (S) circle (1pt) ;
			% \tkzDrawSegments [thick,black,smooth](S,C S,B S,D B,C C,D)
			% \tkzDrawSegments [thick,black,smooth,dashed](S,A A,B A,C A,D)
			% \tkzLabelPoints[below](C,B)
			% \tkzLabelPoints[right](D)
			% \tkzLabelPoints[left](A,S)
			% \tkzMarkRightAngles(S,A,B S,A,D B,A,C C,D,A S,D,C)
			% \tkzMarkAngles[size=0.4,fill=green,opacity=0.3](S,D,A)
			% \tkzLabelAngle[pos=0.80,rotate=10](S,D,A){\footnotesize $60^\circ$}%
			% \end{tikzpicture}
		}
	}
\end{vd}


\subsubsection{Câu hỏi trắc nghiệm}
\begin{ex}%[2H1Y3-2]%Câu 1.
	Cho hình chóp tam giác $S.ABC$ có đáy $ABC$ là tam giác vuông tại $A$, $AB=a$, $AC=2a$, cạnh bên $SA$ vuông góc với mặt đáy và $SA=a$. Tính thể tích $V$ của khối chóp $S.ABC$. 
	\choice
	{$V=a^3$}
	{$V=\dfrac{a^3}{2}$}
	{\True $V=\dfrac{a^3}{3}$}
	{$V=\dfrac{a^3}{4}$}
	\loigiai{
	\immini
	{Ta có $V_{SABC}=\dfrac{1}{3}S_{ABC}\cdot SA=\dfrac{1}{3}\cdot\dfrac{1}{2}\cdot AB\cdot AC\cdot SA=\dfrac{1}{6}\cdot a\cdot 2a\cdot a=\dfrac{a^3}{3}$.}
	{\begin{tikzpicture}[scale=0.8, font=\footnotesize, line join=round, line cap=round, >=stealth]
		\def\ac{3.5} 
		\def\ab{1.7} 
		\def\h{3} 
		\def\gocA{50} 
		\coordinate[label=left:$A$] (A) at (0,0);
		\coordinate[label=right:$C$] (C) at (\ac,0);
		\coordinate[label=below left:$B$] (B) at (-\gocA:\ab);
		\coordinate[label=above:$S$] (S) at ($(A)+(90:\h)$);
		\draw (A)--(B)--(C)--(S)--cycle (S)--(B);
		\draw[dashed] (A)--(C);
		\foreach \diem in {A,B,C,S}	\fill (\diem)circle(1.5pt);
		\newcommand{\gocv}[4][black]{\draw[#1] ($(#3)!5pt!(#2)$)--($(#3)!2!($($(#3)!5pt!(#2)$)!.5!($(#3)!5pt!(#4)$)$)$)--($(#3)!5pt!(#4)$);}
		\gocv{S}{A}{C}
	\end{tikzpicture}}
	}
\end{ex}

\begin{ex}%[2H1Y3-2]%Câu 2.
	Cho hình chóp tam giác $S.ABC$ có đáy $ABC$ là tam giác đều cạnh $a$, cạnh bên $SA$ vuông góc với mặt đáy và $SA=a$. Tính thể tích $V$ của khối chóp $S.ABC$. 
	\choice
	{$V=\dfrac{2a^3}{3}$}
	{\True $V=\dfrac{a^3\sqrt{3}}{12}$}
	{$V=\dfrac{a^3\sqrt{3}}{3}$}
	{$V=\dfrac{a^3\sqrt{3}}{4}$}
	\loigiai{
	\immini
	{Ta có $V_{SABC}=\dfrac{1}{3}S_{ABC}\cdot SA=\dfrac{1}{3}\cdot\dfrac{AB^2\sqrt{3}}{4}\cdot SA=\dfrac{1}{3}\cdot \dfrac{a^2\sqrt{3}}{4}\cdot a=\dfrac{a^3\sqrt{3}}{12}$.}
	{\begin{tikzpicture}[scale=0.8, font=\footnotesize, line join=round, line cap=round, >=stealth]
		\def\ac{3.5} 
		\def\ab{1.7} 
		\def\h{3} 
		\def\gocA{50} 
		\coordinate[label=left:$A$] (A) at (0,0);
		\coordinate[label=right:$C$] (C) at (\ac,0);
		\coordinate[label=below left:$B$] (B) at (-\gocA:\ab);
		\coordinate[label=above:$S$] (S) at ($(A)+(90:\h)$);
		\draw (A)--(B)--(C)--(S)--cycle (S)--(B);
		\draw[dashed] (A)--(C);
		\foreach \diem in {A,B,C,S}	\fill (\diem)circle(1.5pt);
		\newcommand{\gocv}[4][black]{\draw[#1] ($(#3)!5pt!(#2)$)--($(#3)!2!($($(#3)!5pt!(#2)$)!.5!($(#3)!5pt!(#4)$)$)$)--($(#3)!5pt!(#4)$);}
		\gocv{S}{A}{C}
	\end{tikzpicture}}
	}
\end{ex}

\begin{ex}%[2H1Y3-2]%Câu 3.
	Cho hình chóp tứ giác $S.ABCD$ có đáy $ABCD$ là hình vuông cạnh $a$, cạnh bên $SA$ vuông góc với mặt đáy và $SA=a\sqrt{2}$. Tính thể tích $V$ của khối chóp $S.ABCD$. 
	\choice
	{$\dfrac{a^3\sqrt{2}}{6}$}
	{$\dfrac{a^3\sqrt{2}}{4}$}
	{$a^3\sqrt{2}$}
	{\True $\dfrac{a^3\sqrt{2}}{3}$}
	\loigiai{
	\immini
	{Ta có $V_{SABCD}=\dfrac{1}{3}S_{ABCD}\cdot SA=\dfrac{1}{3}\cdot AB^2\cdot SA=\dfrac{1}{3}\cdot a^2\cdot a\sqrt{2}=\dfrac{a^3\sqrt{2}}{3}$.}
	{\begin{tikzpicture}[scale=0.7, font=\footnotesize, line join=round, line cap=round, >=stealth]
		\def\bc{4} 
		\def\ba{2} 
		\def\h{4} 
		\def\gocB{30} 
		\coordinate[label=below left:$B$] (B) at (0,0);
		\coordinate[label=above left:$A$] (A) at (\gocB:\ba);
		\coordinate[label=below:$C$] (C) at (\bc,0);
		\coordinate[label=right:$D$] (D) at ($(C)-(B)+(A)$);
		\coordinate[label=above:$S$] (S) at ($(A)+(90:\h)$);
		\draw (B)--(C)--(D)--(S)--cycle (S)--(C);
		\draw[dashed] (A)--(D) (S)--(A)--(B);
		\foreach \diem in {A,B,C,D,S}	\fill (\diem)circle(1.5pt);
		\newcommand{\gocv}[4][black]{\draw[#1] ($(#3)!5pt!(#2)$)--($(#3)!2!($($(#3)!5pt!(#2)$)!.5!($(#3)!5pt!(#4)$)$)$)--($(#3)!5pt!(#4)$);}
		\gocv{S}{A}{D}
	\end{tikzpicture}}
	}
\end{ex}

\begin{ex}%[2H1Y3-2]%Câu 4.
	Cho hình chóp $S.ABCD$ có đáy $ABCD$ là hình vuông cạnh $a$. Biết $SA\perp(ABCD)$ và $SA=a\sqrt{3}$. Thể tích của khối chóp $S.ABCD$ là
	\choice
	{$V=a^3\sqrt{3}$}
	{$V=\dfrac{a^3}{4}$}
	{\True $V=\dfrac{a^3\sqrt{3}}{3}$}
	{$V=\dfrac{a^3\sqrt{3}}{12}$}
	\loigiai{
	\immini
	{Ta có $V_{SABCD}=\dfrac{1}{3}S_{ABCD}\cdot SA=\dfrac{1}{3}\cdot AB^2\cdot SA=\dfrac{1}{3}\cdot a^2\cdot a\sqrt{3}=\dfrac{a^3\sqrt{3}}{3}$.}
	{\begin{tikzpicture}[scale=0.7, font=\footnotesize, line join=round, line cap=round, >=stealth]
		\def\bc{4} 
		\def\ba{2} 
		\def\h{4} 
		\def\gocB{30} 
		\coordinate[label=below left:$B$] (B) at (0,0);
		\coordinate[label=above left:$A$] (A) at (\gocB:\ba);
		\coordinate[label=below:$C$] (C) at (\bc,0);
		\coordinate[label=right:$D$] (D) at ($(C)-(B)+(A)$);
		\coordinate[label=above:$S$] (S) at ($(A)+(90:\h)$);
		\draw (B)--(C)--(D)--(S)--cycle (S)--(C);
		\draw[dashed] (A)--(D) (S)--(A)--(B);
		\foreach \diem in {A,B,C,D,S}	\fill (\diem)circle(1.5pt);
		\newcommand{\gocv}[4][black]{\draw[#1] ($(#3)!5pt!(#2)$)--($(#3)!2!($($(#3)!5pt!(#2)$)!.5!($(#3)!5pt!(#4)$)$)$)--($(#3)!5pt!(#4)$);}
		\gocv{S}{A}{D}
	\end{tikzpicture}}
	}
\end{ex}

\begin{ex}%[2H1Y3-2]%Câu 5.
	Cho hình chóp $S.ABC$ có đáy $ABC$ là tam giác đều cạnh $a$. Biết $SA\perp(ABC)$ và $SA=a\sqrt{3}$. Tính thể tích $V$ của khối chóp $S.ABC$. 
	\choice
	{$\dfrac{3a^3}{4}$}
	{\True $\dfrac{a^3}{4}$}
	{$\dfrac{3a^3}{8}$}
	{$\dfrac{3a^3}{6}$}
	\loigiai{
	\immini
	{Ta có $V_{SABC}=\dfrac{1}{3}S_{ABC}\cdot SA=\dfrac{1}{3}\cdot\dfrac{AB^2\sqrt{3}}{4}\cdot SA=\dfrac{1}{3}\cdot \dfrac{a^2\sqrt{3}}{4}\cdot a\sqrt{3}=\dfrac{a^3}{4}$.}
	{\begin{tikzpicture}[scale=0.8, font=\footnotesize, line join=round, line cap=round, >=stealth]
		\def\ac{3.5} 
		\def\ab{1.7} 
		\def\h{3} 
		\def\gocA{50} 
		\coordinate[label=left:$A$] (A) at (0,0);
		\coordinate[label=right:$C$] (C) at (\ac,0);
		\coordinate[label=below left:$B$] (B) at (-\gocA:\ab);
		\coordinate[label=above:$S$] (S) at ($(A)+(90:\h)$);
		\draw (A)--(B)--(C)--(S)--cycle (S)--(B);
		\draw[dashed] (A)--(C);
		\foreach \diem in {A,B,C,S}	\fill (\diem)circle(1.5pt);
		\newcommand{\gocv}[4][black]{\draw[#1] ($(#3)!5pt!(#2)$)--($(#3)!2!($($(#3)!5pt!(#2)$)!.5!($(#3)!5pt!(#4)$)$)$)--($(#3)!5pt!(#4)$);}
		\gocv{S}{A}{C}
		\end{tikzpicture}}
	}
\end{ex}

\begin{ex}%[2H1Y3-2]%Câu 6.
	Cho hình chóp $S.ABC$ có đáy $ABC$ là tam giác vuông tại $B$ biết $AB=a$; $AC=2a$. $SA\perp(ABC)$ và $SA=a\sqrt{3}$. Tính thể tích khối chóp $S.ABC$. 
	\choice
	{$\dfrac{3a^3}{4}$}
	{$\dfrac{a^3}{4}$}
	{$\dfrac{3a^3}{8}$}
	{\True $\dfrac{a^3}{2}$}
	\loigiai{
	\immini
	{Trong $\triangle ABC$ ta có $BC=\sqrt{AC^2-AB^2}=\sqrt{4a^2-a^2}=a\sqrt{3}$.\\
	Suy ra $V_{SABC}=\dfrac{1}{3}S_{ABC}\cdot SA=\dfrac{1}{3}\cdot\dfrac{1}{2}\cdot AB\cdot BC\cdot SA=\dfrac{1}{6}\cdot a\cdot a\sqrt{3}\cdot a\sqrt{3}=\dfrac{a^3}{2}$.}
	{\begin{tikzpicture}[scale=0.8, font=\footnotesize, line join=round, line cap=round, >=stealth]
		\def\ac{3.5} 
		\def\ab{1.7} 
		\def\h{3} 
		\def\gocA{50} 
		\coordinate[label=left:$A$] (A) at (0,0);
		\coordinate[label=right:$C$] (C) at (\ac,0);
		\coordinate[label=below left:$B$] (B) at (-\gocA:\ab);
		\coordinate[label=above:$S$] (S) at ($(A)+(90:\h)$);
		\draw (A)--(B)--(C)--(S)--cycle (S)--(B);
		\draw[dashed] (A)--(C);
		\foreach \diem in {A,B,C,S}	\fill (\diem)circle(1.5pt);
		\newcommand{\gocv}[4][black]{\draw[#1] ($(#3)!5pt!(#2)$)--($(#3)!2!($($(#3)!5pt!(#2)$)!.5!($(#3)!5pt!(#4)$)$)$)--($(#3)!5pt!(#4)$);}
		\gocv{S}{A}{C}
		\gocv{A}{B}{C}
		\end{tikzpicture}}
	}
\end{ex}

\begin{ex}%[2H1B3-2]%Câu 7.
	Cho hình chóp tam giác $S.ABC$ có đáy $ABC$ là tam giác đều cạnh $2a$, cạnh bên $SA$ vuông góc với mặt đáy và $SB=a\sqrt{5}$. Tính thể tích $V$ của khối chóp $S.ABC$. 
	\choice
	{\True $V=\dfrac{a^3\sqrt{3}}{3}$}
	{$V=a^3\sqrt{3}$}
	{$V=\dfrac{a^3\sqrt{3}}{2}$}
	{$V=\dfrac{a^3\sqrt{3}}{6}$}
	\loigiai{
	\immini
	{Trong $\triangle SAB$ ta có $SA=\sqrt{SB^2-AB^2}=\sqrt{5a^2-4a^2}=a$.\\
	Suy ra $V_{SABC}=\dfrac{1}{3}S_{ABC}\cdot SA=\dfrac{1}{3}\cdot\dfrac{AB^2\sqrt{3}}{4}\cdot SA=\dfrac{1}{3}\cdot \dfrac{4a^2\sqrt{3}}{4}\cdot a=\dfrac{a^3\sqrt{3}}{3}$.}
	{\begin{tikzpicture}[scale=0.8, font=\footnotesize, line join=round, line cap=round, >=stealth]
		\def\ac{3.5} 
		\def\ab{1.7} 
		\def\h{3} 
		\def\gocA{50} 
		\coordinate[label=left:$A$] (A) at (0,0);
		\coordinate[label=right:$C$] (C) at (\ac,0);
		\coordinate[label=below left:$B$] (B) at (-\gocA:\ab);
		\coordinate[label=above:$S$] (S) at ($(A)+(90:\h)$);
		\draw (A)--(B)--(C)--(S)--cycle (S)--(B);
		\draw[dashed] (A)--(C);
		\foreach \diem in {A,B,C,S}	\fill (\diem)circle(1.5pt);
		\newcommand{\gocv}[4][black]{\draw[#1] ($(#3)!5pt!(#2)$)--($(#3)!2!($($(#3)!5pt!(#2)$)!.5!($(#3)!5pt!(#4)$)$)$)--($(#3)!5pt!(#4)$);}
		\gocv{S}{A}{C}
		\end{tikzpicture}}
	}
\end{ex}

\begin{ex}%[2H1B3-2]%Câu 8.
	Cho hình chóp tứ giác $S.ABCD$ có đáy $ABCD$ là hình vuông cạnh $a\sqrt{2}$, cạnh bên $SA$ vuông góc với mặt đáy và $SC=a\sqrt{5}$. Tính thể tích $V$ của khối chóp $S.ABCD$.
	\choice
	{\True $V=\dfrac{2a^3}{3}$}
	{$V=\dfrac{a^3}{3}$}
	{$V=2a^3$}
	{$V=\dfrac{4a^2}{3}$}
	\loigiai{
	\immini
	{Trong hình vuông $ABCD$ ta có $AC=AB\sqrt{2}=2a$.\\
	Trong $\triangle SAC$ ta có $SA=\sqrt{SC^2-AC^2}=\sqrt{5a^2-4a^2}=a$.\\ 
	Suy ra $V_{SABCD}=\dfrac{1}{3}S_{ABCD}\cdot SA=\dfrac{1}{3}\cdot AB^2\cdot SA=\dfrac{1}{3}\cdot (a\sqrt{2})^2\cdot a=\dfrac{2a^3}{3}$.}
	{\begin{tikzpicture}[scale=0.7, font=\footnotesize, line join=round, line cap=round, >=stealth]
		\def\bc{4} 
		\def\ba{2} 
		\def\h{4} 
		\def\gocB{30} 
		\coordinate[label=below left:$B$] (B) at (0,0);
		\coordinate[label=above left:$A$] (A) at (\gocB:\ba);
		\coordinate[label=below:$C$] (C) at (\bc,0);
		\coordinate[label=right:$D$] (D) at ($(C)-(B)+(A)$);
		\coordinate[label=above:$S$] (S) at ($(A)+(90:\h)$);
		\draw (B)--(C)--(D)--(S)--cycle (S)--(C);
		\draw[dashed] (A)--(D) (S)--(A)--(B);
		\foreach \diem in {A,B,C,D,S}	\fill (\diem)circle(1.5pt);
		\newcommand{\gocv}[4][black]{\draw[#1] ($(#3)!5pt!(#2)$)--($(#3)!2!($($(#3)!5pt!(#2)$)!.5!($(#3)!5pt!(#4)$)$)$)--($(#3)!5pt!(#4)$);}
		\gocv{S}{A}{D}
		\end{tikzpicture}}
	}
\end{ex}

\begin{ex}%[2H1B3-2]%Câu 9.
	Cho hình chóp tứ giác $S.ABCD$ có đáy $ABCD$ là hình vuông, cạnh bên $SA$ vuông góc với mặt đáy và $SA=AC=a\sqrt{2}$. Tính thể tích $V$ của khối chóp $S.ABCD$.
	\choice
	{\True $V=\dfrac{a^3\sqrt{2}}{3}$}
	{$V=\dfrac{a^3\sqrt{6}}{9}$}
	{$V=a^3\sqrt{2}$}
	{$V=\dfrac{a^3\sqrt{6}}{3}$}
	\loigiai{
	\immini
	{Trong hình vuông $ABCD$ ta có $AB=\dfrac{AC}{\sqrt{2}}=a$.\\ 
	Suy ra $V_{SABCD}=\dfrac{1}{3}S_{ABCD}\cdot SA=\dfrac{1}{3}\cdot AB^2\cdot SA=\dfrac{1}{3}\cdot a^2\cdot a\sqrt{2}=\dfrac{a^3\sqrt{2}}{3}$.}
	{\begin{tikzpicture}[scale=0.7, font=\footnotesize, line join=round, line cap=round, >=stealth]
		\def\bc{4} 
		\def\ba{2} 
		\def\h{4} 
		\def\gocB{30} 
		\coordinate[label=below left:$B$] (B) at (0,0);
		\coordinate[label=above left:$A$] (A) at (\gocB:\ba);
		\coordinate[label=below:$C$] (C) at (\bc,0);
		\coordinate[label=right:$D$] (D) at ($(C)-(B)+(A)$);
		\coordinate[label=above:$S$] (S) at ($(A)+(90:\h)$);
		\draw (B)--(C)--(D)--(S)--cycle (S)--(C);
		\draw[dashed] (A)--(D) (S)--(A)--(B);
		\foreach \diem in {A,B,C,D,S}	\fill (\diem)circle(1.5pt);
		\newcommand{\gocv}[4][black]{\draw[#1] ($(#3)!5pt!(#2)$)--($(#3)!2!($($(#3)!5pt!(#2)$)!.5!($(#3)!5pt!(#4)$)$)$)--($(#3)!5pt!(#4)$);}
		\gocv{S}{A}{D}
		\end{tikzpicture}}
	}
\end{ex}

\begin{ex}%[2H1B3-2]%Câu 10.
	Cho hình chóp $S.ABCD$ có đáy $ABCD$ là hình vuông cạnh $a$. Biết $SA\perp(ABCD)$ và $SB=a\sqrt{3}$. Tính thể tích khối chóp $S.ABCD$. 
	\choice
	{$V=\dfrac{a^3\sqrt{2}}{2}$}
	{$a^3\sqrt{3}$}
	{\True $V=\dfrac{a^3\sqrt{2}}{3}$}
	{$V=\dfrac{a^3\sqrt{2}}{6}$}
	\loigiai{
	\immini
	{Trong $\triangle SAB$ ta có $SA=\sqrt{SB^2-AB^2}=\sqrt{3a^2-a^2}=a\sqrt{2}$.\\ 
	Suy ra $V_{SABCD}=\dfrac{1}{3}S_{ABCD}\cdot SA=\dfrac{1}{3}\cdot AB^2\cdot SA=\dfrac{1}{3}\cdot a^2\cdot a\sqrt{2}=\dfrac{a^3\sqrt{2}}{3}$.}
	{\begin{tikzpicture}[scale=0.7, font=\footnotesize, line join=round, line cap=round, >=stealth]
		\def\bc{4} 
		\def\ba{2} 
		\def\h{4} 
		\def\gocB{30} 
		\coordinate[label=below left:$B$] (B) at (0,0);
		\coordinate[label=above left:$A$] (A) at (\gocB:\ba);
		\coordinate[label=below:$C$] (C) at (\bc,0);
		\coordinate[label=right:$D$] (D) at ($(C)-(B)+(A)$);
		\coordinate[label=above:$S$] (S) at ($(A)+(90:\h)$);
		\draw (B)--(C)--(D)--(S)--cycle (S)--(C);
		\draw[dashed] (A)--(D) (S)--(A)--(B);
		\foreach \diem in {A,B,C,D,S}	\fill (\diem)circle(1.5pt);
		\newcommand{\gocv}[4][black]{\draw[#1] ($(#3)!5pt!(#2)$)--($(#3)!2!($($(#3)!5pt!(#2)$)!.5!($(#3)!5pt!(#4)$)$)$)--($(#3)!5pt!(#4)$);}
		\gocv{S}{A}{D}
		\end{tikzpicture}}
	}
\end{ex}

\begin{ex}%[2H1B3-2]%Câu 11.
	Cho hình chóp $S.ABC$ có $SA=a$ và vuông góc với đáy $ABC$. Biết rằng tam giác $ABC$ đều và mặt phẳng $(SBC)$ hợp với đáy $(ABC)$ một góc $30^\circ$. Tính thể tích $V$ của khối chóp $S.ABC$. 
	\choice
	{\True $V=\dfrac{a^3\sqrt{3}}{3}$}
	{$V=\dfrac{2a^3}{3}$}
	{$V=\dfrac{a^3\sqrt{3}}{12}$}
	{$V=\dfrac{a^3}{3}$}
	\loigiai{
	\immini
	{Gọi $M$ là trung điểm $BC$. Vì $\triangle ABC$ đều nên $AM \perp BC \Rightarrow BC \perp SM$.\\
	Suy ra $\left((SBC), (ABC)\right)=\left(\widehat{SM, AM}\right)=\widehat{SMA} \Rightarrow \widehat{SMA}=30^\circ$.\\
	Trong $\triangle SAM$ ta có $AM=\dfrac{SA}{\tan \widehat{SMA}}=\dfrac{a}{\tan 30^\circ}=a\sqrt{3}$.\\
	Trong $\triangle ABC$ ta có $AM=\dfrac{AB\sqrt{3}}{2} \Rightarrow AB=\dfrac{2AM}{\sqrt{3}}=2a$.\\ 
	Suy ra $V_{SABC}=\dfrac{1}{3}S_{ABC}\cdot SA=\dfrac{1}{3}\cdot \dfrac{AB^2\sqrt{3}}{4}\cdot SA=\dfrac{1}{3}\cdot \dfrac{4a^2\sqrt{3}}{4}\cdot a=\dfrac{a^3\sqrt{3}}{3}$.}
	{\begin{tikzpicture}[scale=0.7, font=\footnotesize, line join=round, line cap=round, >=stealth]
		\def\ac{4} 
		\def\ab{2} 
		\def\h{4} 
		\def\gocA{50} 
		\coordinate[label=left:$A$] (A) at (0,0);
		\coordinate[label=right:$C$] (C) at (\ac,0);
		\coordinate[label=below left:$B$] (B) at (-\gocA:\ab);
		\coordinate[label=above:$S$] (S) at ($(A)+(90:\h)$);
		\coordinate[label=below:$M$] (M) at ($(B)!0.5!(C)$);
		\draw (A)--(B)--(C)--(S)--cycle (M)--(S)--(B);
		\draw[dashed] (M)--(A)--(C);
		\foreach \diem in {A,B,C,S,M}	\fill (\diem)circle(1.5pt);
		\newcommand{\gocv}[4][black]{\draw[#1] ($(#3)!5pt!(#2)$)--($(#3)!2!($($(#3)!5pt!(#2)$)!.5!($(#3)!5pt!(#4)$)$)$)--($(#3)!5pt!(#4)$);}
		\gocv{S}{A}{C}
		\draw pic[draw,angle radius=1mm,angle eccentricity=1.5] {angle = S--M--A};
		\end{tikzpicture}}
	}
\end{ex}

\begin{ex}%[2H1B3-2]%Câu 12.
	Cho khối chóp $S.ABC$ có $SA$ vuông góc với $(ABC)$, đáy $ABC$ là tam giác vuông cân tại $A$, $BC=2a$, góc giữa $SB$ và $(ABC)$ là $30^\circ$. Tính thể tích khối chóp $S.ABC$. 
	\choice
	{\True $\dfrac{a^3\sqrt{6}}{9}$}
	{$\dfrac{a^3\sqrt{6}}{3}$}
	{$\dfrac{a^3\sqrt{3}}{3}$}
	{$\dfrac{a^3\sqrt{2}}{4}$}
	\loigiai{
	\immini
	{Ta có $\triangle ABC$ vuông cân tại $A$ và $BC=2a$ suy ra $AB=a\sqrt{2}$.\\
	Có $SA \perp (ABC)$ suy ra $AB$ là hình chiếu của $SB$ lên mặt phẳng $ABC$. Suy ra $\left(SB, (ABC)\right)=\left(\widehat{SB, AB}\right)=\widehat{SBA} \Rightarrow \widehat{SBA}=30^\circ$.\\
	Trong $\triangle SAB$ ta có $SA=AB\cdot\tan \widehat{SBA}=a\sqrt{2}\cdot\tan 30^\circ=\dfrac{a\sqrt{2}}{\sqrt{3}}$.\\
	Suy ra $V_{SABC}=\dfrac{1}{3}S_{ABC}\cdot SA=\dfrac{1}{3}\cdot \dfrac{1}{2}\cdot AB\cdot AC\cdot SA=\dfrac{1}{6}\cdot \left(a\sqrt{2}\right)^2\cdot \dfrac{a\sqrt{2}}{\sqrt{3}}=\dfrac{a^3\sqrt{6}}{9}$.}
	{\begin{tikzpicture}[scale=0.7, font=\footnotesize, line join=round, line cap=round, >=stealth]
		\def\ac{4} 
		\def\ab{2} 
		\def\h{4} 
		\def\gocA{50} 
		\coordinate[label=left:$A$] (A) at (0,0);
		\coordinate[label=right:$C$] (C) at (\ac,0);
		\coordinate[label=below left:$B$] (B) at (-\gocA:\ab);
		\coordinate[label=above:$S$] (S) at ($(A)+(90:\h)$);
		\draw (A)--(B)--(C)--(S)--cycle (S)--(B);
		\draw[dashed] (A)--(C);
		\foreach \diem in {A,B,C,S}	\fill (\diem)circle(1.5pt);
		\newcommand{\gocv}[4][black]{\draw[#1] ($(#3)!5pt!(#2)$)--($(#3)!2!($($(#3)!5pt!(#2)$)!.5!($(#3)!5pt!(#4)$)$)$)--($(#3)!5pt!(#4)$);}
		\gocv{S}{A}{C}
		\gocv{B}{A}{C}
		\draw pic[draw,angle radius=2mm,angle eccentricity=1.5] {angle = S--B--A};
		\end{tikzpicture}}
	}
\end{ex}

\begin{ex}%[2H1B3-2]%Câu 13.
	Cho hình chóp tam giác $S.ABC$ có đáy $ABC$ là tam giác vuông tại $B$, cạnh bên $SA$ vuông góc với mặt đáy và $SB$ tạo với mặt đáy một góc $45^\circ$. Biết $AB=a$, $\widehat{ACB}=60^\circ$. Tính thể tích $V$ của khối chóp $S.ABC$. 
	\choice
	{$V=\dfrac{a^3\sqrt{3}}{18}$}
	{$V=\dfrac{a^3}{2\sqrt{3}}$}
	{$V=\dfrac{a^3\sqrt{3}}{9}$}
	{\True $V=\dfrac{a^3\sqrt{3}}{6}$}
	\loigiai{
	\immini
	{Trong $\triangle ABC$ ta có $BC=AB\cdot\cot \widehat{ACB}=a\cdot\cot 60^\circ=\dfrac{a}{\sqrt{3}}$.\\
	Trong $\triangle SAB$ ta có $SA=AB=a$.\\
	Suy ra $V_{SABC}=\dfrac{1}{3}S_{ABC}\cdot SA=\dfrac{1}{3}\cdot \dfrac{1}{2}\cdot AB\cdot AC\cdot SA=\dfrac{1}{6}\cdot a\cdot\dfrac{a}{\sqrt{3}}\cdot a=\dfrac{a^3\sqrt{3}}{6}$.}
	{\begin{tikzpicture}[scale=0.7, font=\footnotesize, line join=round, line cap=round, >=stealth]
		\def\ac{4} 
		\def\ab{2} 
		\def\h{4} 
		\def\gocA{50} 
		\coordinate[label=left:$A$] (A) at (0,0);
		\coordinate[label=right:$C$] (C) at (\ac,0);
		\coordinate[label=below left:$B$] (B) at (-\gocA:\ab);
		\coordinate[label=above:$S$] (S) at ($(A)+(90:\h)$);
		\draw (A)--(B)--(C)--(S)--cycle (S)--(B);
		\draw[dashed] (A)--(C);
		\foreach \diem in {A,B,C,S}	\fill (\diem)circle(1.5pt);
		\newcommand{\gocv}[4][black]{\draw[#1] ($(#3)!5pt!(#2)$)--($(#3)!2!($($(#3)!5pt!(#2)$)!.5!($(#3)!5pt!(#4)$)$)$)--($(#3)!5pt!(#4)$);}
		\gocv{S}{A}{C}
		\gocv{A}{B}{C}
		\draw pic[draw,angle radius=4mm,angle eccentricity=1.5] {angle = S--B--A};
		\draw pic[draw,angle radius=4mm] {angle = A--C--B} pic[draw,angle radius=4.7mm,angle eccentricity=1.5] {angle = A--C--B};
	\end{tikzpicture}}
	}
\end{ex}

\begin{ex}%[2H1B3-2]%Câu 14.
	Cho hình chóp tam giác $S.ABC$ có đáy $ABC$ là tam giác vuông cân tại $B$, cạnh bên $SA$ vuông góc với mặt đáy và $SB=a\sqrt{3}$, $AC=a\sqrt{2}$. Tính thể tích $V$ của khối chóp $S.ABC$. 
	\choice
	{\True $V=\dfrac{\sqrt{2}a^3}{6}$}
	{$V=\dfrac{\sqrt{2}a^3}{2}$}
	{$V=\dfrac{\sqrt{2}a^3}{3}$}
	{$V=\dfrac{a^3}{8}$}
	\loigiai{
	\immini
	{Ta có $\triangle ABC$ vuông cân tại $B$ và $AC=a\sqrt{2}$, suy ra $AB=BC=a$.\\
	Trong $\triangle SAB$ ta có $SA=\sqrt{SB^2-AB^2}=\sqrt{3a^2-a^2}=a\sqrt{2}$.\\
	Suy ra $V_{SABC}=\dfrac{1}{3}S_{ABC}\cdot SA=\dfrac{1}{3}\cdot \dfrac{1}{2}\cdot AB\cdot BC\cdot SA=\dfrac{1}{6}\cdot a^2\cdot a\sqrt{2}=\dfrac{a^3\sqrt{2}}{6}$.}
	{\begin{tikzpicture}[scale=0.7, font=\footnotesize, line join=round, line cap=round, >=stealth]
		\def\ac{4} 
		\def\ab{2} 
		\def\h{4} 
		\def\gocA{50} 
		\coordinate[label=left:$A$] (A) at (0,0);
		\coordinate[label=right:$C$] (C) at (\ac,0);
		\coordinate[label=below left:$B$] (B) at (-\gocA:\ab);
		\coordinate[label=above:$S$] (S) at ($(A)+(90:\h)$);
		\draw (A)--(B)--(C)--(S)--cycle (S)--(B);
		\draw[dashed] (A)--(C);
		\foreach \diem in {A,B,C,S}	\fill (\diem)circle(1.5pt);
		\newcommand{\gocv}[4][black]{\draw[#1] ($(#3)!5pt!(#2)$)--($(#3)!2!($($(#3)!5pt!(#2)$)!.5!($(#3)!5pt!(#4)$)$)$)--($(#3)!5pt!(#4)$);}
		\gocv{S}{A}{C}
		\gocv{A}{B}{C}
		\end{tikzpicture}}
	}
\end{ex}

\begin{ex}%[2H1B3-2]%Câu 15.
	Cho hình chóp $S.ABCD$ có đáy $ABCD$ là hình vuông cạnh $a$. Hai mặt phẳng $(SAC)$ và $(SAB)$ cùng vuông góc với $(ABCD)$. Góc giữa $(SCD)$ và $(ABCD)$ là $60^{\circ}$. Tính thể tích của khối chóp $S.ABCD$. 
	\choice
	{\True $\dfrac{a^3\sqrt{3}}{3}$}
	{$\dfrac{a^3\sqrt{6}}{3}$}
	{$\dfrac{a^3\sqrt{3}}{6}$}
	{$\dfrac{a^3\sqrt{3}}{3}$}
	\loigiai{
		\immini{
			$\heva{&(SAB)\perp (ABCD)\\&(SAC)\perp (ABCD)\\&(SAB)\cap(SAC)=SA}\Rightarrow SA\perp (ABCD)$.\\
			$\heva{&(SCD)\cap (ABCD)=CD\\&AB\subset (ABCD),\ AD\perp CD\\&SD\subset (SCD),\ SD\perp CD}$\\
			$\Rightarrow \widehat{((SCD),(ABCD))}=\widehat{(SD,AD)}=\widehat{SDA}=60^\circ$.\\
			$SA=AB\cdot\tan 60^\circ=a\sqrt{3}$.\\
			$V_{S.ABCD}=\dfrac{1}{3}SA\cdot S_{ABCD}=\dfrac{1}{3}\cdot a\sqrt{3}\cdot a^2=\dfrac{a^3\sqrt{3}}{3}$.
		}{
			\begin{tikzpicture}[line join=round,line cap=round, font=\footnotesize,scale=0.75,>=stealth]
			\def\a{3.5}
			\path
			(0,0) coordinate (A)
			(\a,0) coordinate (B)
			(220:0.6*\a) coordinate (D)		
			(90:\a) coordinate (S)		
			($(B)+(D)$) coordinate (C)		
			;
			\draw 
			(S)--(B)--(C)--(D)--(S)--(C)
			pic [draw,angle radius=2mm] {right angle =B--A--S}
			pic [draw,angle radius=2mm] {right angle =S--A--D}
			;
			\draw[dashed] 
			(S)--(A)--(B)(A)--(D)
			;
			\begin{scope}
			\clip (A)--(D)--(S);
			\draw (D) circle(.5)+(50:1)node{$60^\circ$};
			\end{scope}
			\foreach \x/\g in {S/180,A/-70,B/0,C/-10,D/180}\fill[black] (\x) circle (1pt)+(\g:.3)node{$\x$};
			\end{tikzpicture}
		}
	}
\end{ex}

\begin{ex}%[2H1B3-2]%Câu 16.
	Cho hình chóp $S.ABC$ có đáy là tam giác vuông cân tại $B$; $AB=a$, $SA\perp (ABC)$. Cạnh bên $SB$ hợp với đáy một góc $45^{\circ}$. Thể tích của khối chóp $S.ABC$ tính theo $a$ bằng 
	\choice
	{$\dfrac{a^3}{3}$}
	{$\dfrac{a^3\sqrt{2}}{6}$}
	{$\dfrac{a^3\sqrt{3}}{3}$}
	{\True $\dfrac{a^3}{6}$}
	\loigiai{
		\immini{
			$SA\perp (ABC)$ nên $AB$ là hình chiếu của $SB$ lên $(ABC)$\\
			$\Rightarrow \left(\widehat{SB,(ABC)}\right)=\widehat{(SB,AB)}=\widehat{SBA}=45^\circ$.\\
			$\Rightarrow\triangle SBA$ vuông tại cân $A$, $\Rightarrow SA=AB=a$.\\
			$V_{S.ABC}=\dfrac{1}{3}SA\cdot S_{ABC}=\dfrac{1}{3}a\cdot\dfrac{a^2}{2}=\dfrac{a^3}{6}$.
		}{
			\begin{tikzpicture}[line join=round,line cap=round, font=\footnotesize,scale=0.8,>=stealth]
			\def\a{4}
			\path
			(0,0) coordinate (A)
			(\a,0) coordinate (C)
			(-40:0.7*\a) coordinate (B)		
			(90:\a) coordinate (S)		
			;
			\draw 
			(S)--(A)--(B)--(C)--(S)(B)--(S)
			pic [draw,angle radius=2mm] {right angle =A--B--C}
			pic [draw,angle radius=2mm] {right angle =C--A--S}
			pic [draw,angle radius=2mm] {right angle =B--A--S}
			;
			\draw[dashed] 
			(A)--(C);
			\foreach \x/\g in {S/180,A/180,B/180,C/0}\fill[black] (\x) circle (1pt)+(\g:.3)node{$\x$};
			\end{tikzpicture}
		}
	}
	\loigiai{
		
	}
\end{ex}

\begin{ex}%[2H1B3-2]%Câu 17.
	Cho hình chóp $S.ABCD$ có đáy $ABCD$ là hình vuông tâm $O$, cạnh $2a$. Biết $SA$ vuông góc với mặt phẳng đáy và $SA=a\sqrt{2}$. Thể tích của khối chóp $S.ABO$ là
	\choice
	{$\dfrac{4a^3\sqrt{2}}{3}$}
	{$\dfrac{2a^3\sqrt{2}}{12}$}
	{\True $\dfrac{a^3\sqrt{2}}{3}$}
	{$\dfrac{a^3\sqrt{2}}{12}$}
	\loigiai{
		\immini{
			$V_{S.ABO}=\dfrac{1}{4}V_{S.ABCD}=\dfrac{1}{4}\cdot\dfrac{1}{3}SA\cdot S_{ABCD}=\dfrac{1}{4}\cdot\dfrac{1}{3}a\sqrt{2}\cdot (2a)^2=\dfrac{a^3\sqrt{2}}{3}$.
		}{
			\begin{tikzpicture}[line join=round,line cap=round, font=\footnotesize,scale=0.75,>=stealth]
			\def\a{3.5}
			\path
			(0,0) coordinate (A)
			(\a,0) coordinate (B)
			(220:0.6*\a) coordinate (D)		
			(90:\a) coordinate (S)		
			($(B)+(D)$) coordinate (C)
			($(A)!0.5!(C)$) coordinate (O)		
			;
			\draw 
			(S)--(B)--(C)--(D)--(S)--(C)
			pic [draw,angle radius=2mm] {right angle =B--A--S}
			pic [draw,angle radius=2mm] {right angle =S--A--D}
			;
			\draw[dashed] 
			(S)--(A)--(B)(A)--(D)(A)--(C)(B)--(D)
			;		
			\foreach \x/\g in {S/180,A/-90,B/0,C/-10,D/180,O/-90}\fill[black] (\x) circle (1pt)+(\g:.3)node{$\x$};
			\end{tikzpicture}
		}
	}
\end{ex}

\begin{ex}%[2H1B3-2]%Câu 18.
	Cho khối chóp $S.ABCD$ có đáy $ABCD$ là hình chữ nhật với $AB=a, BD=2a$. Cạnh bên $SA$ vuông góc với mặt phẳng đáy, góc giữa mặt phẳng $(SBD)$ và mặt phẳng $(ABCD)$ bằng $60^{\circ}$. Thể tích $V$ của khối chóp $S.ABCD$ là 
	\choice
	{\True $V=\dfrac{a^3\sqrt{3}}{2}$}
	{$V=\dfrac{a^3\sqrt{3}}{4}$}
	{$V=\dfrac{a^3\sqrt{3}}{6}$}
	{$V=a^3$}
	\loigiai{
		\immini{
			Gọi $H$ là hình chiếu của $A$ lên $BD$.\\
			$\heva{&BD\perp SA\\&BD\perp AH}\Rightarrow BD\perp (SAH)\Rightarrow BD\perp SH$.\\
			$\widehat{((SBD),(ABCD))}=\widehat{(SH,AH)}=\widehat{SHA}=60^\circ$.\\
			$AB=\sqrt{BD^2-AD^2}=\sqrt{(2a)^2-a^2}=a\sqrt{3}$.\\
			$\dfrac{1}{AH^2}=\dfrac{1}{AB^2}+\dfrac{1}{AD^2}=\dfrac{1}{a^2}+\dfrac{1}{(a\sqrt{3})^2}=\dfrac{1}{a^2}+\dfrac{1}{3a^2}=\dfrac{4}{3a^2}$\\
			$\Rightarrow AH=\dfrac{a\sqrt{3}}{2}$.\\
			$SA=AH\cdot\tan 60^\circ=\dfrac{a\sqrt{3}}{2}\cdot\sqrt{3}=\dfrac{3a}{2}$.\\
			$V_{S.ABCD}=\dfrac{1}{3}SA\cdot S_{ABCD}=\dfrac{1}{3}\cdot\dfrac{3a}{2}\cdot a\cdot a\sqrt{3}=\dfrac{a^3\sqrt{3}}{2}$.
		}{
			\begin{tikzpicture}[line join=round,line cap=round, font=\footnotesize,scale=0.8,>=stealth]
			\def\a{3.5}
			\path
			(0,0) coordinate (A)
			(\a,0) coordinate (B)
			(220:0.6*\a) coordinate (D)		
			(90:\a) coordinate (S)		
			($(B)+(D)$) coordinate (C)
			($(A)!0.5!(C)$) coordinate (O)
			($(D)!0.4!(B)$) coordinate (H)		
			;
			\draw 
			(S)--(B)--(C)--(D)--(S)--(C)
			pic [draw,angle radius=2mm] {right angle =B--A--S}
			pic [draw,angle radius=2mm] {right angle =A--H--D}
			;
			\draw[dashed] 
			(S)--(A)--(B)(A)--(D)(A)--(C)(B)--(D)(S)--(H)(A)--(H)
			;		
			\foreach \x/\g in {S/180,A/170,B/0,C/-10,D/180,O/-90,H/-90}\fill[black] (\x) circle (1pt)+(\g:.3)node{$\x$};
			\end{tikzpicture}
		}
	}
\end{ex}

\begin{ex}%[2H1B3-2]%Câu 19.
	Cho hình chóp $S.ABC$ có đáy $ABC$ là tam giác cân tại $A$, $BC=2a\sqrt{3}$, $\widehat{BAC}=120^{\circ}$, cạnh bên $SA$ vuông góc với mặt đáy và $SA=2a$. Tính thể tích $V$ của khối chóp $S.ABC$. 
	\choice
	{\True $V=\dfrac{2a^3\sqrt{3}}{3}$}
	{$V=a^3\sqrt{3}$}
	{$V=\dfrac{a^3\sqrt{3}}{2}$}
	{$V=\dfrac{a^3\sqrt{3}}{6}$}
	\loigiai{
		\immini{
			Gọi $H$ là trung điểm của $BC\Rightarrow AH=BH\cot 60^\circ=a\sqrt{3}\cdot \dfrac{1}{\sqrt{3}}=a$.\\
			$V_{S.ABC}=\dfrac{1}{3}SA\cdot S_{ABC}=\dfrac{1}{3}2a\cdot \dfrac{1}{2}a\cdot 2a\sqrt{3}=\dfrac{2a^3\sqrt{3}}{3}$.
		}{
			\begin{tikzpicture}[line join=round,line cap=round, font=\footnotesize,scale=0.8,>=stealth]
			\def\a{4}
			\path
			(0,0) coordinate (A)
			(\a,0) coordinate (C)
			(-40:0.7*\a) coordinate (B)		
			(90:\a) coordinate (S)
			($(B)!0.5!(C)$) coordinate (H)		
			;
			\draw 
			(S)--(A)--(B)--(C)--(S)(B)--(S)
			pic [draw,angle radius=2mm] {right angle =A--H--B}
			pic [draw,angle radius=2mm] {right angle =C--A--S}
			pic [draw,angle radius=2mm] {right angle =B--A--S}
			;
			\draw[dashed] 
			(H)--(A)--(C);
			\foreach \x/\g in {S/180,A/180,B/180,C/0,H/0}\fill[black] (\x) circle (1pt)+(\g:.3)node{$\x$};
			\end{tikzpicture}
		}
	}
\end{ex}

\begin{ex}%[2H1B3-2]%Câu 20.
	Cho hình chóp $S.ABC$ có đáy $ABC$ là tam giác vuông cân tại $A$, $BC=a\sqrt{2}$, cạnh bên $SA$ vuông góc với mặt phẳng đáy, mặt bên $(SBC)$ tạo với mặt đáy $(ABC)$ một góc bằng $45^{\circ}$. Tính thể tích $V$ của khối chóp $S.ABC$. 
	\choice
	{\True $V=\dfrac{a^3\sqrt{2}}{12}$}
	{$V=\dfrac{a^3\sqrt{2}}{4}$}
	{$V=\dfrac{a^3\sqrt{2}}{6}$}
	{$V=\dfrac{a^3\sqrt{3}}{18}$}
	\loigiai{
		\immini{
			Gọi $H$ là trung điểm của $BC$.\\
			$\heva{&(ABC)\cap (SBC)=BC\\&AH\perp BC,\ SH\perp BC}\Rightarrow \widehat{((SBC),(ABC))}=\widehat{(SH,AH)}=\widehat{SHA}=45^\circ$.\\
			$\triangle SHA$ vuông cân tại $A\Rightarrow SA=AH=\dfrac{BC}{2}=\dfrac{a\sqrt{2}}{2}$.\\
			$V_{S.ABC}=\dfrac{1}{3}SA\cdot S_{ABC}=\dfrac{1}{3}\dfrac{a\sqrt{2}}{2}\cdot\dfrac{1}{2}\cdot\dfrac{a\sqrt{2}}{2}\cdot a\sqrt{2}=\dfrac{a^3\sqrt{2}}{12}$.
		}{
			\begin{tikzpicture}[line join=round,line cap=round, font=\footnotesize,scale=0.8,>=stealth]
			\def\a{4}
			\path
			(0,0) coordinate (A)
			(\a,0) coordinate (C)
			(-40:0.7*\a) coordinate (B)		
			(90:\a) coordinate (S)
			($(B)!0.5!(C)$) coordinate (H)		
			;
			\draw 
			(S)--(A)--(B)--(C)--(S)(B)--(S)--(H)
			pic [draw,angle radius=2mm] {right angle =A--H--B}
			pic [draw,angle radius=2mm] {right angle =C--A--S}
			pic [draw,angle radius=2mm] {right angle =B--A--S}
			;
			\draw[dashed] 
			(H)--(A)--(C);
			\foreach \x/\g in {S/180,A/180,B/180,C/0,H/0}\fill[black] (\x) circle (1pt)+(\g:.3)node{$\x$};
			\end{tikzpicture}
		}
	}
\end{ex}

\begin{ex}%[2H1B3-2]%Câu 21.
	Cho hình chóp $S.ABCD$ có đáy $ABCD$ là hình vuông cạnh $a$, $SA$ vuông góc với đáy $ABCD$. Mặt bên $(SCD)$ hợp với đáy một góc $60^{\circ}$. Tính thể tích $V$ của khối chóp $S.ABCD$. 
	\choice
	{$V=\dfrac{a^3\sqrt{3}}{6}$}
	{\True $V=\dfrac{a^3\sqrt{3}}{3}$}
	{$V=\dfrac{a^3\sqrt{3}}{12}$}
	{$V=\dfrac{a^3\sqrt{2}}{6}$}
	\loigiai{
		\immini{
			$\heva{&(SCD)\cap (ABCD)=CD\\&AB\perp CD,\ SD\perp CD}\Rightarrow \widehat{((SCD),(ABCD))}=\widehat{(SD,AD)}=\widehat{SDA}=60^\circ$.\\
			$SA=AB\cdot\tan 60^\circ=a\sqrt{3}$.\\
			$V_{S.ABCD}=\dfrac{1}{3}SA\cdot S_{ABCD}=\dfrac{1}{3}\cdot a\sqrt{3}\cdot a^2=\dfrac{a^3\sqrt{3}}{3}$.
		}{
			\begin{tikzpicture}[line join=round,line cap=round, font=\footnotesize,scale=0.75,>=stealth]
			\def\a{3.5}
			\path
			(0,0) coordinate (A)
			(\a,0) coordinate (B)
			(220:0.6*\a) coordinate (D)		
			(90:\a) coordinate (S)		
			($(B)+(D)$) coordinate (C)
			($(A)!0.5!(C)$) coordinate (O)		
			;
			\draw 
			(S)--(B)--(C)--(D)--(S)--(C)
			pic [draw,angle radius=2mm] {right angle =B--A--S}
			pic [draw,angle radius=2mm] {right angle =S--A--D}
			;
			\draw[dashed] 
			(O)--(S)--(A)--(B)(A)--(D)(A)--(C)(B)--(D)
			;
			\begin{scope}
			\clip (A)--(D)--(S);
			\draw (D) circle(.5)+(50:1)node{$60^\circ$};
			\end{scope}
			\foreach \x/\g in {S/180,A/-70,B/0,C/-10,D/180,O/-90}\fill[black] (\x) circle (1pt)+(\g:.3)node{$\x$};
			\end{tikzpicture}
		}
	}	
\end{ex}

\begin{ex}%[2H1B3-2]%Câu 22.
	Cho khối chóp $S.ABCD$ có đáy $ABCD$ là hình vuông cạnh $a$, cạnh bên $SA$ vuông góc với mặt phẳng đáy, góc giữa mặt phẳng $(SBD)$ và mặt phẳng đáy bằng $60^{\circ}$. Tính thể tích $V$ của khối chóp $S.ABCD$. 
	\choice
	{\True $V=\dfrac{a^3\sqrt{6}}{6}$}
	{$V=\dfrac{a^3\sqrt{3}}{2}$}
	{$V=\dfrac{a^3\sqrt{3}}{12}$}
	{$V=\dfrac{a^3\sqrt{3}}{7}$}
	\loigiai{
		\immini{
			$\heva{&(SBD)\cap (ABCD)=BD\\&AO\perp BD,\ SO\perp BD}\Rightarrow \widehat{((SBD),(ABCD))}=\widehat{(SO,AO)}=\widehat{SOA}=60^\circ$.\\
			$SA=AO\cdot\tan 60^\circ=\dfrac{a\sqrt{2}}{2}\cdot\sqrt{3}=\dfrac{a\sqrt{6}}{2}$.\\
			$V_{S.ABCD}=\dfrac{1}{3}SA\cdot S_{ABCD}=\dfrac{1}{3}\cdot \dfrac{a\sqrt{6}}{2}\cdot a^2=\dfrac{a^3\sqrt{6}}{6}$.
		}{
			\begin{tikzpicture}[line join=round,line cap=round, font=\footnotesize,scale=0.75,>=stealth]
			\def\a{3.5}
			\path
			(0,0) coordinate (A)
			(\a,0) coordinate (B)
			(220:0.6*\a) coordinate (D)		
			(90:\a) coordinate (S)		
			($(B)+(D)$) coordinate (C)
			($(A)!0.5!(C)$) coordinate (O)		
			;
			\draw 
			(S)--(B)--(C)--(D)--(S)--(C)
			pic [draw,angle radius=2mm] {right angle =B--A--S}
			pic [draw,angle radius=2mm] {right angle =S--A--D}
			;
			\draw[dashed] 
			(O)--(S)--(A)--(B)(A)--(D)(A)--(C)(B)--(D)
			;
			\begin{scope}
			\clip (S)--(O)--(A);
			\draw (O) circle(.5);
			\end{scope}
			\foreach \x/\g in {S/180,A/-70,B/0,C/-10,D/180,O/-90}\fill[black] (\x) circle (1pt)+(\g:.3)node{$\x$};
			\end{tikzpicture}
		}
	}	
\end{ex}

\begin{ex}%[2H1B3-2]%Câu 23.
	Cho khối chóp $S.ABCD$ có đáy $ABCD$ là hình vuông cạnh $a$, cạnh bên $SA$ vuông góc với mặt phẳng đáy. Đường thẳng $SD$ tạo với mặt phẳng $(SAB)$ một góc $30^{\circ}$. Tính thể tích $V$ của khối chóp $S.ABCD$. 
	\choice
	{$V=\dfrac{a^3\sqrt{3}}{2}$}
	{$V=\dfrac{a^3\sqrt{3}}{4}$}
	{$V=\dfrac{a^3\sqrt{3}}{12}$}
	{\True $V=\dfrac{a^3\sqrt{3}}{3}$}
	\loigiai{
		\immini{
			$AD\perp (SAB)\Rightarrow SA$ là hình chiếu của $SD$ lên $(SAB)$\\
			$\Rightarrow \widehat{(SD,(SAB))}=\widehat{(SD,SA)}=\widehat{DSA}=30^\circ$.\\
			$SA=AD\cdot\cot 30^\circ=a\sqrt{3}$.\\
			$V_{S.ABCD}=\dfrac{1}{3}SA\cdot S_{ABCD}=\dfrac{1}{3}\cdot a\sqrt{3}\cdot a^2=\dfrac{a^3\sqrt{3}}{3}$.
		}{
			\begin{tikzpicture}[line join=round,line cap=round, font=\footnotesize,scale=0.75,>=stealth]
			\def\a{3.5}
			\path
			(0,0) coordinate (A)
			(\a,0) coordinate (B)
			(220:0.6*\a) coordinate (D)		
			(90:\a) coordinate (S)		
			($(B)+(D)$) coordinate (C)
			($(A)!0.5!(C)$) coordinate (O)		
			;
			\draw 
			(S)--(B)--(C)--(D)--(S)--(C)
			pic [draw,angle radius=2mm] {right angle =B--A--S}
			pic [draw,angle radius=2mm] {right angle =S--A--D}
			;
			\draw[dashed] 
			(O)--(S)--(A)--(B)(A)--(D)(A)--(C)(B)--(D)
			;
			\begin{scope}
			\clip (S)--(O)--(A);
			\draw (O) circle(.5);
			\end{scope}
			\foreach \x/\g in {S/180,A/-70,B/0,C/-10,D/180,O/-90}\fill[black] (\x) circle (1pt)+(\g:.3)node{$\x$};
			\end{tikzpicture}
		}
	}	
\end{ex}

\begin{ex}%[2H1B3-2]%Câu 24.
	Cho hình chóp $S.ABC$ có đáy $ABC$ là tam giác vuông cân tại $B$, $AB=a$, $SA\perp(ABC)$ góc giữa hai mặt phẳng $(SBC)$ và $(ABC)$ bằng $30^{\circ}$. Tính thể tích $V$ của khối chóp $S.ABC$ 
	\choice
	{\True $V=\dfrac{a^3\sqrt{3}}{18}$}
	{$V=\dfrac{a^3\sqrt{3}}{24}$}
	{$V=\dfrac{a^3\sqrt{3}}{36}$}
	{$V=\dfrac{2a^3\sqrt{3}}{9}$}
	\loigiai{
		\immini{
			$\heva{&(ABC)\cap (SBC)=BC\\&AB\perp BC,\ SB\perp BC}\Rightarrow \widehat{((SBC),(ABC))}=\widehat{(SB,AB)}=\widehat{SBA}=30^\circ$.\\
			$SA=AB\cdot\tan 30^\circ=a\dfrac{1}{\sqrt{3}}=\dfrac{a}{\sqrt{3}}$.\\ 
			$V_{S.ABC}=\dfrac{1}{3}SA\cdot S_{ABC}=\dfrac{1}{3}\cdot\dfrac{a}{\sqrt{3}}\cdot\dfrac{1}{2}\cdot a^2=\dfrac{a^3\sqrt{3}}{18}$.
		}{
			\begin{tikzpicture}[line join=round,line cap=round, font=\footnotesize,scale=0.8,>=stealth]
			\def\a{4}
			\path
			(0,0) coordinate (A)
			(\a,0) coordinate (C)
			(-40:0.7*\a) coordinate (B)		
			(90:\a) coordinate (S)				
			;
			\draw 
			(S)--(A)--(B)--(C)--(S)(B)--(S)
			pic [draw,angle radius=2mm] {right angle =C--B--A}
			pic [draw,angle radius=2mm] {right angle =C--A--S}
			pic [draw,angle radius=2mm] {right angle =B--A--S}
			;
			\draw[dashed] 
			(A)--(C);
			\foreach \x/\g in {S/180,A/180,B/180,C/0}\fill[black] (\x) circle (1pt)+(\g:.3)node{$\x$};
			\end{tikzpicture}
		}
	}
\end{ex}

\begin{ex}%[2H1B3-2]%Câu 25.
	Cho hình chóp $S.ABCD$ có đáy $ABCD$ là hình thang vuông tại $A$ và $B$. $AB=BC=a$, $SA=a$ và vuông góc với mặt phẳng $(ABCD)$. Khoảng cách từ $D$ đến mặt phẳng $(SAC)$ bằng $a\sqrt{2}$. Tính thể tích $V$ của khối chóp $S.ABCD$. 
	\choice
	{$V=\dfrac{a^3\sqrt{3}}{4}$}
	{\True $V=\dfrac{a^3}{2}$}
	{$V=\dfrac{a^3\sqrt{3}}{6}$}
	{$V=\dfrac{a^3}{3}$}
	\loigiai{
		\immini{
			$\triangle ABC$ vuông cân tại $B\Rightarrow AC=AB\cdot \sqrt{2}=a\sqrt{2}$.
			\begin{align*}
			V_{S.ABCD}=&V_{S.ABC}+V_{D.SAC}\\
			=&\ \dfrac{1}{3}SA\cdot S_{ABC}+\dfrac{1}{3}\mathrm {d}(D,(SAC))\cdot S_{SAC}\\
			=&\ \dfrac{1}{3}SA\cdot \dfrac{1}{2}AB\cdot BC+\dfrac{1}{3}\mathrm {d}(D,(SAC))\cdot \dfrac{1}{2}AC\cdot SA\\
			=&\ \dfrac{1}{3}a\cdot \dfrac{1}{2}a^2+\dfrac{1}{3}\cdot a\sqrt{2}\cdot \dfrac{1}{2}\cdot a\cdot a\sqrt{2}\\
			=&\ \dfrac{a^3}{6}+\dfrac{a^3}{3}=\dfrac{a^3}{2}.
			\end{align*}
		}{
			\begin{tikzpicture}[line join=round,line cap=round, font=\footnotesize,scale=0.75,>=stealth]
			\def\a{3.5}
			\path
			(0,0) coordinate (A)
			(\a,0) coordinate (M)
			(220:0.6*\a) coordinate (B)		
			(90:\a) coordinate (S)		
			($(M)+(B)$) coordinate (C)		
			($(A)!2!(M)$) coordinate (D)		
			;
			\draw 
			(C)--(B)--(S)--(C)(S)--(D)--(C)
			pic [draw,angle radius=2mm] {right angle =M--A--S}
			pic [draw,angle radius=2mm] {right angle =S--A--B}
			;
			\draw[dashed] 
			(S)--(A)--(D)(A)--(B)(A)--(C)
			;		
			\foreach \x/\g in {S/180,A/-80,C/-40,B/180,D/0}\fill[black] (\x) circle (1pt)+(\g:.3)node{$\x$};
			\end{tikzpicture}
		}
	}	
\end{ex}


\begin{dang}{Khối chóp có mặt bên vuông góc với đáy}
\end{dang}
\begin{vd}%[2H1B3-2]%Ví dụ 1.
	Cho hình chóp $S.ABCD$ có đáy $ABCD$ là hình vuông cạnh $a$. Mặt bên $SAB$ là tam giác đều và nằm trong mặt phẳng vuông góc với đáy $ABCD$.
	\begin{enumEX}[a)]{1}
	\item Chứng minh rằng chân đường cao khối chóp trùng với trung điểm cạnh $AB$.\\
	\item Tính thể tích khối chóp $S.ABCD$.
	\end{enumEX}
	\loigiai{
		\immini
		{\begin{enumEX}[a)]{1}		
		\item Gọi $H$ là trung điểm $AB$. $\Delta SAB$ đều $\Rightarrow SH\perp AB$.\\
		mà $(SAB)\perp(ABCD)\Rightarrow SH\perp(ABCD)$. Vậy $H$ là chân đường vuông góc.
		\item Ta có tam giác $SAB$ đều nên $SA=\dfrac{a\sqrt{3}}{2}$\\
		$\Rightarrow V=\dfrac{1}{3}S_{ABCD}\cdot SH=\dfrac{a^3\sqrt{3}}{6}$.
		\end{enumEX}
		}
		{\begin{tikzpicture}[scale=1, font=\footnotesize, line join=round, line cap=round, >=stealth]
			\def\bc{3.5} 
			\def\ba{2} 
			\def\h{4} 
			\def\gocB{25} 
			\coordinate[label=below left:$B$] (B) at (0,0);
			\coordinate[label=above right:$A$] (A) at (\gocB:\ba);
			\coordinate[label=below:$C$] (C) at (\bc,0);
			\coordinate[label=right:$D$] (D) at ($(C)-(B)+(A)$);
			\coordinate[label=above left:$H$] (H) at ($(A)!1/2!(B)$); 
			\coordinate[label=above left:$S$] (S) at ($(H)+(90:\h)$);
			\draw (B)--(C)--(D)--(S)--cycle (S)--(C);
			\draw[dashed] (A)--(D) (H)--(S)--(A)--(B);
			\foreach \diem in {A,B,C,D,S,H}	\fill (\diem)circle(1.5pt);
			\newcommand{\gocv}[4][black]{\draw[#1] ($(#3)!5pt!(#2)$)--($(#3)!2!($($(#3)!5pt!(#2)$)!.5!($(#3)!5pt!(#4)$)$)$)--($(#3)!5pt!(#4)$);}
			\gocv{S}{H}{A}
			\end{tikzpicture}}
	}
\end{vd}

\begin{vd}%[2H1B3-2]%Ví dụ 2.
	Cho tứ diện $ABCD$ có $ABC$ là tam giác đều, $BCD$ là tam giác vuông cân tại $D$, $AD=a$, $(ABC)\perp(BCD)$ và $AD$ hợp với $(BCD)$ một góc $60^\circ$. Tính thể tích tứ diện $ABCD$.
	\loigiai{
		\immini
		{Gọi $H$ là trung điểm của $BC$. Ta có $\Delta ABC$ đều nên $AH\perp(BCD)$, mà $(ABC)\perp(BCD)\Rightarrow AH\perp(BCD)$.\\
		Ta có $AH\perp HD\Rightarrow AH=AD\cdot\sin 60^\circ=a\dfrac{\sqrt{3}}{2}$\\
		và $HD=AD\cdot\cos 60^\circ=\dfrac{a}{2}$.\\
		$\triangle ABC$ đều và $AH=a\dfrac{\sqrt{3}}{2} \Rightarrow BC=a$.\\
		Vậy $V=\dfrac{1}{3}S_{BCD}\cdot AH=\dfrac{1}{3}\cdot\dfrac{1}{2}BC\cdot HD\cdot AH=\dfrac{a^3\sqrt{3}}{24}$.}
		{\begin{tikzpicture}[scale=1, font=\footnotesize, line join=round, line cap=round, >=stealth]
			\def\bd{4} 
			\def\bc{3} 
			\def\h{4} 
			\def\gocA{30} 
			\coordinate[label=left:$B$] (B) at (0,0);
			\coordinate[label=right:$D$] (D) at (\bd,0);
			\coordinate[label=below left:$C$] (C) at (-\gocA:\bc);
			\coordinate[label=below left:$H$] (H) at ($(B)!.5!(C)$); 
			\coordinate[label=above left:$A$] (A) at ($(H)+(90:\h)$);
			\draw (B)--(C)--(D)--(A)--cycle (H)--(A)--(C);
			\draw[dashed] (B)--(D)--(H);
			\foreach \diem in {A,B,C,D,H}	\fill (\diem)circle(1.5pt);
			\newcommand{\gocv}[4][black]{\draw[#1] ($(#3)!5pt!(#2)$)--($(#3)!2!($($(#3)!5pt!(#2)$)!.5!($(#3)!5pt!(#4)$)$)$)--($(#3)!5pt!(#4)$);}
			\gocv{A}{H}{D}
			\gocv{B}{D}{C}
			\draw pic[draw,blue,angle radius=3.5mm,angle eccentricity=1.5] {angle = A--D--H};
			\end{tikzpicture}}
	}
\end{vd}

\begin{vd}%[2H1B3-2]%Ví dụ 3.
	Cho hình chóp $S.ABC$ có đáy $ABC$ là tam giác vuông cân tại $B$, có $BC=a$. Mặt bên $SAC$ vuông góc với đáy, các mặt bên còn lại đều tạo với đáy một góc $45^\circ$.\\
	a) Chứng minh rằng chân đường cao khối chóp trùng với trung điểm cạnh $AC$.\\
	b) Tính thể tích khối chóp $S.ABC$.
	\loigiai{
		\immini
		{a) Kẻ $AH\perp BC$ vì mặt phẳng $(SAC)\perp(ABC)\Rightarrow SH\perp(ABC)$.\\
		Gọi $I,J$ là hình chiếu của $H$ trên $AB$ và $BC\Rightarrow SI\perp AB,SJ\perp BC$, theo giả thiết $\widehat{SIH}=\widehat{SJH}=45^\circ$. Ta có $\Delta SHI=\Delta SHJ\Rightarrow HI=HJ$ nên $BH$ là đường phân giác của $\widehat{ABC}$ từ đó suy ra $H$ là trung điểm của $AC$.\\
		b) $HI=HJ=SH=\dfrac{a}{2}\Rightarrow V_{SABC}=\dfrac{1}{3}S_{ABC}\cdot SH=\dfrac{a^3}{12}$.}
		{\begin{tikzpicture}[scale=1, font=\footnotesize, line join=round, line cap=round, >=stealth]
			\def\ab{4}
			\def\ac{2} 
			\def\h{4} 
			\def\gocA{50} 
			\coordinate[label=left:$A$] (A) at (0,0);
			\coordinate[label=right:$B$] (B) at (\ab,0);
			\coordinate[label=below left:$C$] (C) at (-\gocA:\ac);
			\coordinate[label=below left:$H$] (H) at ($(A)!.5!(C)$); 
			\coordinate[label=above right:$I$] (I) at ($(A)!.5!(B)$);
			\coordinate[label=right:$J$] (J) at ($(B)!.5!(C)$);
			\coordinate[label=above left:$S$] (S) at ($(H)+(90:\h)$);
			\draw (A)--(C)--(B)--(S)--cycle (H)--(S)--(C)  (S)--(J);
			\draw[dashed] (A)--(B)--(H)--(J)  (S)--(I)--(H);
			\foreach \diem in {A,B,C,S,H,I,J}	\fill (\diem)circle(1.5pt);
			\newcommand{\gocv}[4][black]{\draw[#1] ($(#3)!5pt!(#2)$)--($(#3)!2!($($(#3)!5pt!(#2)$)!.5!($(#3)!5pt!(#4)$)$)$)--($(#3)!5pt!(#4)$);}
			\gocv{S}{H}{B}
			\draw pic[draw,blue,angle radius=2.5mm,angle eccentricity=1.5] {angle = S--I--H};
			\draw pic[draw,blue,angle radius=3mm,angle eccentricity=1.5] {angle = S--J--H}; 
		\end{tikzpicture}}
	}
\end{vd}

\subsubsection{Câu hỏi trắc nghiệm}
\begin{ex}%[2H1B3-2]%Câu 1.
	Cho hình chóp $S.ABCD$ có đáy $ABCD$ là hình vuông. Mặt bên $SAB$ là tam giác đều cạnh $a$ và nằm trong mặt phẳng vuông góc với $(ABCD)$. Tính thể tích của khối chóp $S.ABCD$. 
	\choice
	{\True $\dfrac{a^3\sqrt{3}}{6}$}
	{$\dfrac{a^3\sqrt{3}}{2}$}
	{$\dfrac{a^3}{3}$}
	{$a^3$}
	\loigiai{
		\immini{
			Gọi $H$ là trung điểm của $AB$, $SH$ là đường cao $\triangle SAB$.\\
			$\heva{&(SAB)\perp (ABCD),\ (SAB)\cap (ABCD)=AB\\&SH\subset (SAB),\ SH\perp AB}\Rightarrow SH\perp (ABCD)$.\\
			$V_{S.ABCD}=\dfrac{1}{3}SH\cdot S_{ABCD}=\dfrac{1}{3}\cdot \dfrac{a\sqrt{3}}{2}\cdot a^2=\dfrac{a^3\sqrt{3}}{6}$.
		}{
			\begin{tikzpicture}[line join=round,line cap=round, font=\footnotesize,scale=0.75,>=stealth]
			\def\a{3.5}
			\path
			(0,0) coordinate (A)
			(\a,0) coordinate (D)
			(220:0.6*\a) coordinate (B)		
			(90:\a) coordinate (S')		
			($(B)+(D)$) coordinate (C)
			($(A)!0.5!(B)$) coordinate (H)
			($(S')+(H)$) coordinate (S)		
			;
			\draw 
			(S)--(D)--(C)--(B)--(S)--(C)
			pic [draw,angle radius=2mm] {right angle =A--H--S}		
			;
			\draw[dashed] 
			(H)--(S)--(A)--(D)(A)--(B)
			;		
			\foreach \x/\g in {S/180,A/-70,D/0,C/-10,B/180,H/-10}\fill[black] (\x) circle (1pt)+(\g:.3)node{$\x$};
			\end{tikzpicture}
		}
	}	
\end{ex}

\begin{ex}%[2H1B3-2]%Câu 2.
	Cho hình chóp $S.ABC$ có đáy $ABC$ là tam giác vuông cân tại $A$, $BC=2a$. Mặt bên $SBC$ là tam giác vuông cân tại $S$ và nằm trong mặt phẳng vuông góc với đáy. Tính thể tích khối chóp $S.ABC$. 
	\choice
	{$V=a^3$}
	{$V=\dfrac{2a^3}{3}$}
	{$V=\dfrac{\sqrt{2}a^3}{3}$}
	{\True $V=\dfrac{a^3}{3}$}
	\loigiai{
		\immini{
			Gọi $H$ là trung điểm của $BC$, $SH$ là đường cao $\triangle SBC\Rightarrow SH=\dfrac{BC}{2}=a$.\\
			$\heva{&(SBC)\perp (ABC),\ (SBC)\cap (ABC)=BC\\&SH\subset (SBC),\ SH\perp BC}\Rightarrow SH\perp (ABC)$.\\
			$V_{S.ABC}=\dfrac{1}{3}SH\cdot S_{ABC}=\dfrac{1}{3}\cdot a\cdot\dfrac{1}{2}\cdot (a\sqrt{2})^2=\dfrac{a^3}{3}$.
		}{
			\begin{tikzpicture}[line join=round,line cap=round, font=\footnotesize,scale=0.8,>=stealth]
			\def\a{4}
			\path
			(0,0) coordinate (A)
			(\a,0) coordinate (C)
			(-40:0.7*\a) coordinate (B)		
			(90:\a) coordinate (S')
			($(C)!0.5!(B)$) coordinate (H)
			($(S')+(H)$) coordinate (S)					
			;
			\draw 
			(S)--(A)--(B)--(C)--(S)(B)--(S)--(H)
			pic [draw,angle radius=2mm] {right angle =C--H--S}		
			;
			\draw[dashed] 
			(A)--(C);
			\foreach \x/\g in {S/180,A/180,B/180,C/0,H/-20}\fill[black] (\x) circle (1pt)+(\g:.3)node{$\x$};
			\end{tikzpicture}
		}
	}
\end{ex}

\begin{ex}%[2H1B3-2]%Câu 3.
	Cho hình chóp $S.ABC$ có $SA=a$, tam giác $ABC$ đều, tam giác $SAB$ vuông cân tại $S$ và nằm trong mặt phẳng vuông góc với mặt phẳng đáy. Thể tích khối chóp $S.ABC$ bằng
	\choice
	{$\dfrac{\sqrt{6}a^3}{4}$}
	{$\dfrac{\sqrt{6}a^3}{24}$}
	{\True $\dfrac{\sqrt{6}a^3}{12}$}
	{$\dfrac{\sqrt{6}a^3}{8}$}
	\loigiai{
		\immini{
			$\triangle SAB$ vuông cân tại $S\Rightarrow AB=SA\sqrt{2}=a\sqrt{2}$.\\
			Gọi $H$ là trung điểm của $AB$, $SH$ là đường cao $\triangle SAB$\\
			suy ra $SH=\dfrac{AB}{2}=\dfrac{a\sqrt{2}}{2}$.\\
			$\heva{&(SAB)\perp (ABC),\ (SAB)\cap (ABC)=AB\\&SH\subset (SAB),\ SH\perp AB}\Rightarrow SH\perp (ABC)$.\\
			$V_{S.ABC}=\dfrac{1}{3}SH\cdot S_{ABC}=\dfrac{1}{3}\cdot \dfrac{a\sqrt{2}}{2}\cdot\dfrac{(a\sqrt{2})^2\sqrt{3}}{4}=\dfrac{a^3\sqrt{6}}{12}$.
		}{
			\begin{tikzpicture}[line join=round,line cap=round, font=\footnotesize,scale=0.8,>=stealth]
			\def\a{4}
			\path
			(0,0) coordinate (A)
			(\a,0) coordinate (C)
			(-40:0.7*\a) coordinate (B)		
			(90:\a) coordinate (S')
			($(A)!0.5!(B)$) coordinate (H)
			($(S')+(H)$) coordinate (S)					
			;
			\draw 
			(S)--(A)--(B)--(C)--(S)(B)--(S)--(H)
			pic [draw,angle radius=2mm] {right angle =S--H--A}		
			;
			\draw[dashed] 
			(A)--(C);
			\foreach \x/\g in {S/180,A/180,B/180,C/0,H/200}\fill[black] (\x) circle (1pt)+(\g:.3)node{$\x$};
			\end{tikzpicture}
		}
	}
\end{ex}

\begin{ex}%[2H1B3-2]%Câu 4.
	Cho hình chóp $S.ABCD$ có đáy $ABCD$ là hình vuông cạnh $2a\sqrt{3}$, mặt bên $SAB$ là tam giác đều và nằm trong mặt phẳng vuông góc với đáy. Thể tích của khối chóp $S.ABCD$ là 
	\choice
	{\True $12a^3$}
	{$14a^3$}
	{$15a^3$}
	{$17a^3$}
	\loigiai{
		\immini{
			Gọi $H$ là trung điểm của $AB$, $SH$ là đường cao $\triangle SAB$.\\
			$\heva{&(SAB)\perp (ABCD),\ (SAB)\cap (ABCD)=AB\\&SH\subset (SAB),\ SH\perp AB}\Rightarrow SH\perp (ABCD)$.\\
			$SH$ là đường cao $\triangle SAB\Rightarrow SH=AB\cdot\dfrac{\sqrt{3}}{2}= 2a\sqrt{3}\cdot \dfrac{\sqrt{3}}{2}=3a$.\\
			$V_{S.ABCD}=\dfrac{1}{3}SH\cdot S_{ABCD}=\dfrac{1}{3}\cdot 3a\cdot (2a\sqrt{3})^2=12a^3$.
		}{
			\begin{tikzpicture}[line join=round,line cap=round, font=\footnotesize,scale=0.75,>=stealth]
			\def\a{3.5}
			\path
			(0,0) coordinate (A)
			(\a,0) coordinate (D)
			(220:0.6*\a) coordinate (B)		
			(90:\a) coordinate (S')		
			($(B)+(D)$) coordinate (C)
			($(A)!0.5!(B)$) coordinate (H)
			($(S')+(H)$) coordinate (S)		
			;
			\draw 
			(S)--(D)--(C)--(B)--(S)--(C)
			pic [draw,angle radius=2mm] {right angle =A--H--S}		
			;
			\draw[dashed] 
			(H)--(S)--(A)--(D)(A)--(B)
			;		
			\foreach \x/\g in {S/180,A/-70,D/0,C/-10,B/180,H/-10}\fill[black] (\x) circle (1pt)+(\g:.3)node{$\x$};
			\end{tikzpicture}
		}
	}	
\end{ex}

\begin{ex}%[2H1B3-2]%Câu 5.
	Cho khối chóp $S.ABCD$ có $ABCD$ là hình vuông cạnh $3a$. Tam giác $SAB$ cân tại $S$ và nằm trong mặt phẳng vuông góc với đáy. Tính thể tích $V$ của khối chóp $S.ABCD$, biết góc giữa $SC$ và $(ABCD)$ bằng $60^{\circ}$. 
	\choice
	{$V=18a^3\sqrt{3}$}
	{\True $V=\dfrac{9a^3\sqrt{15}}{2}$}
	{$V=9a^3\sqrt{3}$}
	{$V=18a^3\sqrt{15}$}
	\loigiai{
		\immini{
			Gọi $H$ là trung điểm của đoạn thẳng $AB$. Suy ra $SH$ là chiều cao của khối chóp. \\
			$(SC,(ABCD))=\widehat{SCH}$.\\
			Tam giác $HBC$ vuông tại $B$: $HC=\sqrt{9a^2+\dfrac{9a^2}{4}}=\dfrac{3a\sqrt{5}}{2}$. \\
			Tam giác $SHC$ vuông tại $H$: $SH=HC \tan 60^\circ=\dfrac{3a\sqrt{15}}{2}$.\\
			Thể tích của khối chóp: $V=\dfrac{1}{3} \cdot 9a^2 \cdot \dfrac{3a\sqrt{15}}{2}=\dfrac{9a^3\sqrt{15}}{2}$.	
		}{
			\begin{tikzpicture}[scale=1, font=\footnotesize, line join=round, line cap=round, >=stealth]
			\def \c{5}	\def \d{4}	
			\def \b{3}	\def \goc{45}
			\def \h{3.5}	
			\path
			(0,0) coordinate (A)
			(\d,0) coordinate (D)
			(\goc:\b) coordinate (B)
			($(D)+(B)-(A)$) coordinate (C)
			($(A)!.5!(B)$) coordinate (H)
			($(H)+(0,\h)$) coordinate (S)
			;
			\draw [dashed] (C)--(H)--(S)--(B)--(C) (A)--(B);
			\draw (S)--(A)--(D)--(S)--(C)--(D);
			\draw  
			($(H)!3mm!(A)$)--($(H)!3mm!(A)+(H)!3mm!(S)-(H)$) --($(H)!3mm!(S)$);
			\foreach \x/\y in {A/180,D/0,C/0,B/180,S/90,H/-100}
			\draw[fill] (\x) circle (1pt) +(\y:.3) node {$\x$};
			\end{tikzpicture}}
	}
\end{ex}

\begin{ex}%[2H1B3-2]%Câu 6.
	Cho hình chóp $S.ABC$ có đáy là tam giác vuông cân tại $B$, $AB=a$. Gọi $I$ là trung điểm $AC$, tam giác $SAC$ cân tại $S$ và nằm trong mặt phẳng vuông góc với đáy. Tính thể tích khối chóp $S.ABC$, biết góc giữa $SB$ và mặt phẳng đáy băng $45^{\circ}$. 
	\choice
	{$\dfrac{a^3\sqrt{3}}{12}$}
	{\True $\dfrac{a^3\sqrt{2}}{12}$}
	{$\dfrac{a^3\sqrt{2}}{4}$}
	{$\dfrac{a^3\sqrt{3}}{4}$}
	\loigiai{
		\immini{
			$SI$ là chiều cao của khối chóp. \\
			$(SB,(ABC))=\widehat{SBI}=45^\circ$.\\
			Tam giác $ABC$ vuông cân tại $B$ nên $BI=\dfrac{BC}{2}=\dfrac{a\sqrt{2}}{2}$.\\ 
			Tam giác $SIB$ vuông tại $I$ và $\widehat{SBI}=45^\circ$ nên $SI=IB=\dfrac{a\sqrt{2}}{2}$.\\
			Thể tích của khối chóp: $V=\dfrac{1}{3} \cdot \dfrac{a^2}{2} \cdot \dfrac{a\sqrt{2}}{2}=\dfrac{a^3\sqrt{2}}{12}$.	
		}{
			\begin{tikzpicture}[scale=1, font=\footnotesize, line join=round, line cap=round, >=stealth]
			\def \c{5}	\def \b{4}	\def \goc{-25}
			\def \h{3.5}	
			\path
			(0,0) coordinate (A)
			(\c,0) coordinate (C)
			(\goc:\b) coordinate (B)
			($(A)!.5!(C)$) coordinate (I)
			($(I)+(0,\h)$) coordinate (S)
			;
			\draw [dashed] (A)--(C) (S)--(I)--(B);
			\draw (S)--(A)--(B)--(S)--(C)--(B);
			\draw  ($(I)!3mm!(A)$)--($(I)!3mm!(A)+(I)!3mm!(S)-(I)$) --($(I)!3mm!(S)$);
			\foreach \x/\y in {A/180,B/-90,C/0,S/90,I/35}
			\draw[fill] (\x) circle (1pt) +(\y:.3) node {$\x$};
			;
			\end{tikzpicture}}
	}
\end{ex}

\begin{ex}%[2H1B3-2]%Câu 7.
	Cho hình chóp $S.ABC$ có đáy là tam giác $ABC$ vuông tại $B$, $AB=a$, $AC=2a$. Hình chiếu vuông góc của $S$ lên $(ABC)$ là trung điểm $M$ của $AC$. Góc giữa $SB$ và đáy bằng $60^{\circ}$. Thể tích $S.ABC$ là bao nhiêu?
	\choice
	{$\dfrac{a^3\sqrt{3}}{2}$}
	{\True $\dfrac{a^3}{2}$}
	{$\dfrac{a^3}{4}$}
	{$\dfrac{a^3\sqrt{2}}{12}$}
	\loigiai{
		\immini{
			Ta có: $SM$ là chiều cao của khối chóp. \\
			$(SB,(ABC))=\widehat{SBM}=60^\circ$.\\
			Tam giác $ABC$ vuông cân tại $B$ có: $BC=\sqrt{4a^2-a^2}=a\sqrt{3}$ và $BM=\dfrac{AC}{2}=a$.\\ 
			Tam giác $SBM$ vuông tại $M$: $SM=BM \tan 60^\circ=a\sqrt{3}$.\\
			Thể tích của khối chóp: $V=\dfrac{1}{3} \cdot \dfrac{a^2\sqrt{3}}{2} \cdot a\sqrt{3}=\dfrac{a^3}{2}$.	
		}{
			\begin{tikzpicture}[scale=1, font=\footnotesize, line join=round, line cap=round, >=stealth]
			\def \c{5}	\def \b{4}	\def \goc{-25}
			\def \h{3.5}	
			\path
			(0,0) coordinate (A)
			(\c,0) coordinate (C)
			(\goc:\b) coordinate (B)
			($(A)!.5!(C)$) coordinate (M)
			($(I)+(0,\h)$) coordinate (S)
			;
			\draw [dashed] (A)--(C) (S)--(I)--(B);
			\draw (S)--(A)--(B)--(S)--(C)--(B);
			\draw  ($(M)!3mm!(A)$)--($(I)!3mm!(A)+(M)!3mm!(S)-(M)$) --($(M)!3mm!(S)$);
			\foreach \x/\y in {A/180,B/-90,C/0,S/90,M/35}
			\draw[fill] (\x) circle (1pt) +(\y:.3) node {$\x$};
			;
			\end{tikzpicture}}
	}
\end{ex}

\begin{ex}%[2H1B3-2]%Câu 8.
	Cho hình chóp $S.ABCD$ có đáy $ABCD$ là hình chữ nhật với $AB=2a$, $AD=a$. Hình chiếu của $S$ lên mặt phẳng $(ABCD)$ là trung điểm $H$ của cạnh $AB$, đường thẳng $SC$ tạo với đáy một góc $45^{\circ}$. Tính thể tích $V$ của khối chóp $S.ABCD$. 
	\choice
	{\True $V=\dfrac{2\sqrt{2}a^3}{3}$}
	{$V=\dfrac{a^3}{3}$}
	{$V=\dfrac{2a^3}{3}$}
	{$V=\dfrac{\sqrt{3}a^3}{2}$}
	\loigiai{
		\immini{
			$SH$ là chiều cao của khối chóp. \\
			$(SC,(ABCD))=\widehat{SCH}=45^\circ$.\\
			Tam giác $HBC$ vuông tại $B$: $HC=\sqrt{a^2+a^2}=a\sqrt{2}$. \\
			Tam giác $SHC$ vuông tại $H$: $SH=HC \tan 45^\circ=a\sqrt{2}$.\\
			Thể tích của khối chóp: $V=\dfrac{1}{3} \cdot 2a^2 \cdot a\sqrt{2}=\dfrac{2a^3\sqrt{2}}{3}$.	
		}{
			\begin{tikzpicture}[scale=1, font=\footnotesize, line join=round, line cap=round, >=stealth]
			\def \c{5}	\def \d{4}	
			\def \b{3}	\def \goc{45}
			\def \h{3.5}	
			\path
			(0,0) coordinate (A)
			(\d,0) coordinate (D)
			(\goc:\b) coordinate (B)
			($(D)+(B)-(A)$) coordinate (C)
			($(A)!.5!(B)$) coordinate (H)
			($(H)+(0,\h)$) coordinate (S)
			;
			\draw [dashed] (C)--(H)--(S)--(B)--(C) (A)--(B);
			\draw (S)--(A)--(D)--(S)--(C)--(D);
			\draw  
			($(H)!3mm!(A)$)--($(H)!3mm!(A)+(H)!3mm!(S)-(H)$) --($(H)!3mm!(S)$);
			\foreach \x/\y in {A/180,D/0,C/0,B/180,S/90,H/-100}
			\draw[fill] (\x) circle (1pt) +(\y:.3) node {$\x$};
			\end{tikzpicture}}
	}
\end{ex}

\begin{ex}%[2H1B3-2]%Câu 9.
	Cho hình chóp $S.ABC$ có tam giác $ABC$ là tam giác đều cạnh $a$. Hình chiếu của $S$ trên mặt phẳng $(ABC)$ là trung điểm của cạnh $AB$, góc tạo bởi cạnh $SC$ và mặt phẳng đáy $(ABC)$ bằng $30^{\circ}$. Tính thể tích của khối chóp $S.ABC$. 
	\choice
	{$\dfrac{a^3\sqrt{3}}{8}$}
	{$\dfrac{a^3\sqrt{2}}{8}$}
	{\True $\dfrac{a^3\sqrt{3}}{24}$}
	{$\dfrac{a^3\sqrt{3}}{2}$}
	\loigiai{
		\immini{
			Ta có: $SM$ là chiều cao của khối chóp. \\
			$(SC,(ABC))=\widehat{SCM}=30^\circ$.\\
			Tam giác $ABC$ đều: $CM=\dfrac{a\sqrt{3}}{2}$ và $BM=\dfrac{AC}{2}=a$.\\ 
			Tam giác $SCM$ vuông tại $M$: $SM=CM \tan 30^\circ=\dfrac{a}{2}$.\\
			Thể tích của khối chóp: $V=\dfrac{1}{3} \cdot \dfrac{a^2\sqrt{3}}{4} \cdot \dfrac{a}{2}=\dfrac{a^3\sqrt{3}}{24}$.	
		}{
			\begin{tikzpicture}[scale=1, font=\footnotesize, line join=round, line cap=round, >=stealth]
			\def \b{5}	\def \c{4}	\def \goc{-25}
			\def \h{3.5}	
			\path
			(0,0) coordinate (A)
			(\b,0) coordinate (B)
			(\goc:\c) coordinate (C)
			($(A)!.5!(B)$) coordinate (H)
			($(M)+(0,\h)$) coordinate (S)
			;
			\draw [dashed] (A)--(B) (S)--(M)--(C);
			\draw (S)--(A)--(C)--(S)--(B)--(C);
			\draw  ($(M)!3mm!(A)$)--($(M)!3mm!(A)+(M)!3mm!(S)-(M)$) --($(M)!3mm!(S)$);
			\foreach \x/\y in {A/180,C/-90,B/0,S/90,M/35}
			\draw[fill] (\x) circle (1pt) +(\y:.3) node {$\x$};
			;
			\end{tikzpicture}}
	}
\end{ex}

\begin{ex}%[2H1B3-2]%Câu 10.
	Cho hình chóp $S.ABCD$ có đáy $ABCD$ là hình chữ nhật, biết $AB=a$, $AD=a\sqrt{3}$. Hình chiếu $S$ lên đáy là trung điểm $H$ của cạnh $AB$, góc tạo bởi $SD$ và đáy là $60^\circ$. Tính thể tích của khối chóp $S.ABCD$. 
	\choice
	{\True $\dfrac{a^3\sqrt{13}}{2}$}
	{$\dfrac{a^3}{2}$}
	{$\dfrac{a^3\sqrt{5}}{5}$}
	{$\dfrac{a^3\sqrt{15}}{5}$}
	\loigiai{
		\immini{
		$SH$ là chiều cao của khối chóp. \\
		$(SD,(ABCD))=\widehat{SDH}=45^\circ$.\\
		Tam giác $HAB$ vuông tại $B$: $HD=\sqrt{3a^2+\dfrac{a^2}{4}}=\dfrac{a\sqrt{13}}{2}$. \\
		Tam giác $SHD$ vuông tại $H$: $SH=HD \tan 60^\circ=\dfrac{a\sqrt{39}}{2}$.\\
		Thể tích của khối chóp: $V=\dfrac{1}{3} \cdot a^2\sqrt{3} \cdot \dfrac{a\sqrt{39}}{2}=\dfrac{a^3\sqrt{13}}{2}$.	
		}{
			\begin{tikzpicture}[scale=1, font=\footnotesize, line join=round, line cap=round, >=stealth]
			\def \c{5}	\def \d{4}	
			\def \b{3}	\def \goc{45}
			\def \h{3.5}	
			\path
			(0,0) coordinate (A)
			(\d,0) coordinate (D)
			(\goc:\b) coordinate (B)
			($(D)+(B)-(A)$) coordinate (C)
			($(A)!.5!(B)$) coordinate (H)
			($(H)+(0,\h)$) coordinate (S)
			;
			\draw [dashed] (D)--(H)--(S)--(B)--(C) (A)--(B);
			\draw (S)--(A)--(D)--(S)--(C)--(D);
			\draw  
			($(H)!3mm!(A)$)--($(H)!3mm!(A)+(H)!3mm!(S)-(H)$) --($(H)!3mm!(S)$);
			\foreach \x/\y in {A/180,D/0,C/0,B/180,S/90,H/-100}
			\draw[fill] (\x) circle (1pt) +(\y:.3) node {$\x$};
			\end{tikzpicture}}
	}
\end{ex}

\begin{ex}%[2H1B3-2]%Câu 11.
	Cho hình chóp $S.ABCD$ có đáy $ABCD$ là hình chữ nhật. Tam giác $SAB$ đều và nằm trong mặt phẳng vuông góc với mặt phẳng đáy $(ABCD)$. Biết $SD=2a\sqrt{3}$ và góc tạo bởi đường thẳng $SC$ và mặt phẳng $(ABCD)$ bằng $30^{\circ}$. Tính thể tích $V$ của khối chóp $S.ABCD$. 
	\choice
	{$V=\dfrac{2\sqrt{3}a^3}{7}$}
	{$V=\dfrac{a^3\sqrt{3}}{13}$}
	{$V=\dfrac{\sqrt{3}a^3}{4}$}
	{\True $V=\dfrac{4\sqrt{6}a^3}{3}$}
	\loigiai{
	\immini{
		Gọi $H$ là trung điểm của $AB$, suy ra $SH$ là chiều cao của khối chóp. \\
		$(SC,(ABCD))=\widehat{SCH}=30^\circ$.\\
		Vì $ABCD$ là hình chữ nhật nên $HD=HC$. Suy ra $\triangle SHD= \triangle SHD$. \\
		Ta có: $SH=HD \sin 30^\circ=a\sqrt{3}$, $HD=SD\cos 30^\circ=3a$.\\
		Vì $\triangle SAB$ đều nên $SH=\dfrac{AB\sqrt{3}}{2} \Rightarrow AB=2a$.\\
		Mặt khác: $AD=\sqrt{9a^2-a^2}=2a\sqrt{2}$.\\
		Thể tích của khối chóp: $V=\dfrac{1}{3} \cdot 2a \cdot 2a\sqrt{2} \cdot a\sqrt{3}=\dfrac{4a^3\sqrt{6}}{3}$.	
		}{
		\begin{tikzpicture}[scale=1, font=\footnotesize, line join=round, line cap=round, >=stealth]
		\def \c{5}	\def \d{4}	
		\def \b{3}	\def \goc{45}
		\def \h{3.5}	
		\path
		(0,0) coordinate (A)
		(\d,0) coordinate (D)
		(\goc:\b) coordinate (B)
		($(D)+(B)-(A)$) coordinate (C)
		($(A)!.5!(B)$) coordinate (H)
		($(H)+(0,\h)$) coordinate (S)
		;
		\draw [dashed] (C)--(H)--(S)--(B)--(C) (A)--(B);
		\draw (S)--(A)--(D)--(S)--(C)--(D);
		\draw  
		($(H)!3mm!(A)$)--($(H)!3mm!(A)+(H)!3mm!(S)-(H)$) --($(H)!3mm!(S)$);
		\foreach \x/\y in {A/180,D/0,C/0,B/180,S/90,H/-100}
		\draw[fill] (\x) circle (1pt) +(\y:.3) node {$\x$};
		\end{tikzpicture}}
	}
\end{ex}

\begin{ex}%[2H1B3-2]%Câu 12.
	Cho hình chóp $S.ABCD$ có đáy $ABCD$ là hình vuông cạnh $a$; hình chiếu của $S$ trên $(ABCD)$ trùng với trung điểm của cạnh $AB$; cạnh bên $SD=\dfrac{3a}{2}$. Tính theo $a$ thể tích của khối chóp $S.ABCD$. 
	\choice
	{$\dfrac{a^3\sqrt{7}}{3}$}
	{\True $\dfrac{a^3}{3}$}
	{$\dfrac{a^3\sqrt{3}}{3}$}
	{$\dfrac{a^3\sqrt{5}}{3}$}
	\loigiai{
	\immini{
		Gọi $H$ là trung điểm của $AB$, suy ra $SH$ là chiều cao của khối chóp. \\
		Ta có: $HD = \sqrt{a^2+\dfrac{a^2}{4}}=\dfrac{a\sqrt{5}}{2}$.\\
		$SH=\sqrt{\dfrac{9a^2}{4}-\dfrac{5a^2}{4}}=a$.\\
		Thể tích của khối chóp: $V=\dfrac{1}{3} \cdot a^2 \cdot a=\dfrac{a^3}{3}$.	
	}{
		\begin{tikzpicture}[scale=1, font=\footnotesize, line join=round, line cap=round, >=stealth]
		\def \c{5}	\def \d{4}	
		\def \b{3}	\def \goc{45}
		\def \h{3.5}	
		\path
		(0,0) coordinate (A)
		(\d,0) coordinate (D)
		(\goc:\b) coordinate (B)
		($(D)+(B)-(A)$) coordinate (C)
		($(A)!.5!(B)$) coordinate (H)
		($(H)+(0,\h)$) coordinate (S)
		;
		\draw [dashed] (D)--(H)--(S)--(B)--(C) (A)--(B);
		\draw (S)--(A)--(D)--(S)--(C)--(D);
		\draw  
		($(H)!3mm!(A)$)--($(H)!3mm!(A)+(H)!3mm!(S)-(H)$) --($(H)!3mm!(S)$);
		\foreach \x/\y in {A/180,D/0,C/0,B/180,S/90,H/-100}
		\draw[fill] (\x) circle (1pt) +(\y:.3) node {$\x$};
		\end{tikzpicture}}
}
\end{ex}

\begin{ex}%[2H1B3-2]%Câu 13.
	Cho hình chóp $S.ABC$ có $SAB$ đều cạnh $a$, tam giác $ABC$ cân tại $C$. Hình chiếu của $S$ lên $(ABC)$ là trung điểm của cạnh $AB$; góc hợp bởi cạnh $SC$ và mặt đáy là $30^{\circ}$. Thể tích khối chóp $S.ABC$ tính theo $a$ là 
	\choice
	{$V=\dfrac{a^3\sqrt{3}}{4}$}
	{$V=\dfrac{a^3\sqrt{2}}{8}$}
	{$V=\dfrac{a^3\sqrt{3}}{2}$}
	{\True $V=\dfrac{a^3\sqrt{3}}{8}$}
	\loigiai{
		\immini{
			Gọi $H$ là trung điểm của $AB$ $\Rightarrow SH$ là chiều cao của khối chóp và $SH=\dfrac{a\sqrt{3}}{2}$. \\
			Ta có: $(SC,(ABC))=\widehat{SCH}=30^\circ$.\\
			Tam giác $SCH$ vuông tại $H$: $CH=\dfrac{SH}{\tan 30^\circ}=\dfrac{a\sqrt{3}}{2} \cdot \sqrt{3}=\dfrac{3a}{2}$.\\
			Thể tích của khối chóp:\\
			$V=\dfrac{1}{3} \cdot \dfrac{1}{2} \cdot AB \cdot CH \cdot SH=\dfrac{1}{6} \cdot a \cdot \dfrac{3a}{2} \cdot \dfrac{a\sqrt{3}}{2}=\dfrac{a^3\sqrt{3}}{8}$.	
		}{
			\begin{tikzpicture}[scale=1, font=\footnotesize, line join=round, line cap=round, >=stealth]
			\def \b{5}	\def \c{2.5}	\def \goc{-60}
			\def \h{3.5}	
			\path
			(0,0) coordinate (A)
			(\b,0) coordinate (B)
			(\goc:\c) coordinate (C)
			($(A)!.5!(B)$) coordinate (H)
			($(H)+(0,\h)$) coordinate (S)
			;
			\draw [dashed] (A)--(B) (S)--(H)--(C);
			\draw (S)--(A)--(C)--(S)--(B)--(C);
			\draw  ($(H)!3mm!(A)$)--($(H)!3mm!(A)+(H)!3mm!(S)-(H)$) --($(H)!3mm!(S)$);
			\foreach \x/\y in {A/180,C/-90,B/0,S/90,H/35}
			\draw[fill] (\x) circle (1pt) +(\y:.3) node {$\x$};
			;
			\end{tikzpicture}}
	}
\end{ex}

\begin{ex}%[2H1B3-2]%Câu 14.
	Cho hình chóp $S.ABCD$ có đáy là hình chữ nhật với $AB=2a, AD=a$. Tam giác $SAB$ cân tại $S$ và nằm trong mặt phẳng vuông góc với đáy, $SC$ tạo với đáy một góc $45^{\circ}$. Thể tích khối chóp $S.ABCD$ là
	\choice
	{$\dfrac{a^3\sqrt{3}}{2}$}
	{\True $\dfrac{2a^3\sqrt{2}}{3}$}
	{$2\sqrt{2}a^3$}
	{$\dfrac{2a^3}{3}$}
	\loigiai{
	\immini{
		Gọi $H$ là trung điểm của $AB$, suy ra $SH$ là chiều cao của khối chóp và $(SC,(ABCD))=\widehat{SCH}=45^\circ$.\\
		Ta có: $HC=\sqrt{a^2+a^2}=a\sqrt{2}$.\\
		$SH=HC \cdot \tan 45^\circ=a\sqrt{2}$.\\
		Thể tích của khối chóp: $V=\dfrac{1}{3} \cdot 2a \cdot a \cdot a\sqrt{2}=\dfrac{2a^3\sqrt{2}}{3}$.	
	}{
		\begin{tikzpicture}[scale=1, font=\footnotesize, line join=round, line cap=round, >=stealth]
		\def \c{5}	\def \d{4}	
		\def \b{3}	\def \goc{45}
		\def \h{3.5}	
		\path
		(0,0) coordinate (A)
		(\d,0) coordinate (D)
		(\goc:\b) coordinate (B)
		($(D)+(B)-(A)$) coordinate (C)
		($(A)!.5!(B)$) coordinate (H)
		($(H)+(0,\h)$) coordinate (S)
		;
		\draw [dashed] (C)--(H)--(S)--(B)--(C) (A)--(B);
		\draw (S)--(A)--(D)--(S)--(C)--(D);
		\draw  
		($(H)!3mm!(A)$)--($(H)!3mm!(A)+(H)!3mm!(S)-(H)$) --($(H)!3mm!(S)$);
		\foreach \x/\y in {A/180,D/0,C/0,B/180,S/90,H/-100}
		\draw[fill] (\x) circle (1pt) +(\y:.3) node {$\x$};
		\end{tikzpicture}}
}
\end{ex}

\begin{ex}%[2H1B3-2]%Câu 15.
	Cho hình chóp $S.ABCD$ có đáy $ABCD$ là hình chữ nhật, $\Delta SAB$ đều cạnh $a$ nằm trong mặt phẳng vuông góc với $(ABCD)$. Biết $(SCD)$ tạo với $(ABCD)$ một góc bằng $30^{\circ}$. Tính thể tích $V$ của khối chóp $S.ABCD$. 
	\choice
	{$V=\dfrac{a^3\sqrt{3}}{8}$}
	{\True $V=\dfrac{a^3\sqrt{3}}{4}$}
	{$V=\dfrac{a^3\sqrt{3}}{2}$}
	{$V=\dfrac{a^3\sqrt{3}}{3}$}
	\loigiai{
		\immini{
			Gọi $H$, $M$ lần lượt là trung điểm của $AB$, $DC$, suy ra $SH$ là chiều cao của khối chóp $\Rightarrow SH=\dfrac{a\sqrt{3}}{2}$.\\
			Ta có: $((SDC),(ABCD))=\widehat{SMH}=30^\circ$.\\
			$AD=HM=\dfrac{SH}{\tan 30^\circ}=\dfrac{3a}{2}$.\\
			Thể tích của khối chóp: $V=\dfrac{1}{3} \cdot a \cdot \dfrac{3a}{2} \cdot \dfrac{a\sqrt{3}}{2}=\dfrac{a^3\sqrt{3}}{4}$.	
		}{
			\begin{tikzpicture}[scale=1, font=\footnotesize, line join=round, line cap=round, >=stealth]
			\def \c{5}	\def \d{4}	
			\def \b{3}	\def \goc{45}
			\def \h{3.5}	
			\path
			(0,0) coordinate (A)
			(\d,0) coordinate (D)
			(\goc:\b) coordinate (B)
			($(D)+(B)-(A)$) coordinate (C)
			($(A)!.5!(B)$) coordinate (H)
			($(H)+(0,\h)$) coordinate (S)
			($(D)!.5!(C)$) coordinate (M)
			;
			\draw [dashed] (M)--(H)--(S)--(B)--(C) (A)--(B);
			\draw (M)--(S)--(A)--(D)--(S)--(C)--(D);
			\draw  
			($(H)!3mm!(A)$)--($(H)!3mm!(A)+(H)!3mm!(S)-(H)$) --($(H)!3mm!(S)$);
			\foreach \x/\y in {A/180,D/0,C/0,B/180,S/90,H/-100,M/0}
			\draw[fill] (\x) circle (1pt) +(\y:.3) node {$\x$};
			\end{tikzpicture}}
	}
\end{ex}

\begin{ex}%[2H1B3-2]%Câu 16.
	Cho hình chóp $S.ABCD$ có đáy là hình chữ nhật với $AB=2a;AD=a$. Tam giác $SAB$ là tam giác cân tại $S$ và nằm trong mặt phẳng vuông góc với mặt đáy. Góc giữa mặt phẳng $(SBC)$ và $(ABCD)$ bằng $45^{\circ}$. Khi đó thể tích khối chóp $S.ABCD$ là 
	\choice
	{$\dfrac{\sqrt{3}}{3}a^3$}
	{$\dfrac{1}{3}a^3$}
	{$2a^3$}
	{\True $\dfrac{2}{3}a^3$}
	\loigiai{
		\immini{
			Gọi $H$ là trung điểm của $AB$, suy ra $SH$ là chiều cao của khối chóp.\\
			Ta có: $((SBC),(ABCD))=\widehat{SBH}=45^\circ$.\\
			$SH=HB \cdot \tan 45^\circ=a$.\\
			Thể tích của khối chóp: $V=\dfrac{1}{3} \cdot 2a \cdot a \cdot a=\dfrac{2a^3}{3}$.	
		}{
			\begin{tikzpicture}[scale=1, font=\footnotesize, line join=round, line cap=round, >=stealth]
			\def \c{5}	\def \d{4}	
			\def \b{3}	\def \goc{45}
			\def \h{3.5}	
			\path
			(0,0) coordinate (A)
			(\d,0) coordinate (D)
			(\goc:\b) coordinate (B)
			($(D)+(B)-(A)$) coordinate (C)
			($(A)!.5!(B)$) coordinate (H)
			($(H)+(0,\h)$) coordinate (S)
			;
			\draw [dashed] (H)--(S)--(B)--(C) (A)--(B);
			\draw (S)--(A)--(D)--(S)--(C)--(D);
			\draw  
			($(H)!3mm!(A)$)--($(H)!3mm!(A)+(H)!3mm!(S)-(H)$) --($(H)!3mm!(S)$);
			\foreach \x/\y in {A/180,D/0,C/0,B/180,S/90,H/-100}
			\draw[fill] (\x) circle (1pt) +(\y:.3) node {$\x$};
			\end{tikzpicture}}
	}
\end{ex}

\begin{ex}%[2H1B3-2]%Câu 17.
	Cho khối chóp $S.ABCD$ có đáy $ABCD$ là hình vuông cạnh $2a$, $\Delta SAD$ cân tại $S$ và nằm trong mặt phẳng vuông góc với đáy. Góc giữa $(SBC)$ và mặt đáy bằng $60^{\circ}$. Tính thể tích $S.ABCD$ bằng 
	\choice
	{$\dfrac{2a^3\sqrt{3}}{3}$}
	{\True $\dfrac{8a^3\sqrt{3}}{3}$}
	{$\dfrac{4a^3\sqrt{3}}{3}$}
	{$2a^3\sqrt{3}$}
	\loigiai{
		\immini{
			Gọi $H$, $M$ lần lượt là trung điểm của $AB$ và $BC$, suy ra $SH$ là chiều cao của khối chóp.\\
			Ta có: $((SBC),(ABCD))=\widehat{SMH}=60^\circ$.\\
			$SH=HM \cdot \tan 60^\circ=2a\sqrt{3}$.\\
			Thể tích của khối chóp: $V=\dfrac{1}{3} \cdot 4a^2 \cdot 2a\sqrt{3}=\dfrac{8a^3\sqrt{3}}{3}$.	
		}{
			\begin{tikzpicture}[scale=1, font=\footnotesize, line join=round, line cap=round, >=stealth]
			\def \c{5}	\def \b{4}	
			\def \d{3}	\def \goc{45}
			\def \h{3.5}	
			\path
			(0,0) coordinate (A)
			(\b,0) coordinate (B)
			(\goc:\d) coordinate (D)
			($(B)+(D)-(A)$) coordinate (C)
			($(A)!.5!(D)$) coordinate (H)
			($(H)+(0,\h)$) coordinate (S)
			($(B)!.5!(C)$) coordinate (M)
			;
			\draw [dashed] (M)--(H)--(S)--(D)--(C) (A)--(D);
			\draw (M)--(S)--(A)--(B)--(S)--(C)--(B);
			\draw  
			($(H)!3mm!(A)$)--($(H)!3mm!(A)+(H)!3mm!(S)-(H)$) --($(H)!3mm!(S)$);
			\foreach \x/\y in {A/180,B/0,C/0,D/180,S/90,H/-100,M/0}
			\draw[fill] (\x) circle (1pt) +(\y:.3) node {$\x$};
			\end{tikzpicture}}
	}
\end{ex}

\begin{ex}%[2H1B3-2]%Câu 18.
	Cho hình chóp $S.ABC$ có đáy $ABC$ là tam giác đều; mặt bên $SAB$ nằm trong mặt phẳng vuông góc với mặt phẳng đáy và tam giác $SAB$ vuông tại $S$, $SA=a\sqrt{3}$, $SB=a$. Tính thể tích khối chóp $S.ABC$. 
	\choice
	{$\dfrac{\sqrt{6}a^3}{6}$}
	{$\dfrac{\sqrt{6}a^3}{3}$}
	{\True $\dfrac{a^3}{2}$}
	{$\dfrac{\sqrt{6}a^3}{2}$}
	\loigiai{
		\immini{
			Gọi $SH$ là đường cao của $\triangle ABC$ ($H$ thuộc $AB$) $\Rightarrow SH$ là chiều cao của khối chóp. \\
			Ta có: $\dfrac{1}{SH^2}=\dfrac{1}{SA^2}+\dfrac{1}{SB^2}=\dfrac{1}{3a^2}+\dfrac{1}{a^2} \Rightarrow SH=\dfrac{a\sqrt{3}}{2}$.\\
			Mặt khác: $AB=\sqrt{3a^2+a^2}=2a$.\\ 
			Thể tích của khối chóp: $V=\dfrac{1}{3} \cdot \dfrac{4a^2\sqrt{3}}{4} \cdot \dfrac{a\sqrt{3}}{2}=\dfrac{a^3}{2}$.	
		}{
			\begin{tikzpicture}[scale=1, font=\footnotesize, line join=round, line cap=round, >=stealth]
			\def \b{5}	\def \c{4}	\def \goc{-25}
			\def \h{3.5}	
			\path
			(0,0) coordinate (A)
			(\b,0) coordinate (B)
			(\goc:\c) coordinate (C)
			($(A)!.4!(B)$) coordinate (H)
			($(H)+(0,\h)$) coordinate (S)
			;
			\draw [dashed] (A)--(B) (S)--(H)--(C);
			\draw (S)--(A)--(C)--(S)--(B)--(C);
			\draw  ($(H)!3mm!(A)$)--($(H)!3mm!(A)+(H)!3mm!(S)-(H)$) --($(H)!3mm!(S)$);
			\foreach \x/\y in {A/180,C/-90,B/0,S/90,H/35}
			\draw[fill] (\x) circle (1pt) +(\y:.3) node {$\x$};
			;
			\end{tikzpicture}}
	}
\end{ex}

\begin{ex}%[2H1B3-2]%Câu 19.
	Cho hình chóp $S.ABC$ có đáy là $ABC$ tam giác vuông cân đỉnh $A,AB=AC=a$. Hình chiếu vuông góc của $S$ lên mặt phẳng $(ABC)$ là trung điểm $H$ của $BC$. Mặt phẳng $(SAB)$ hợp với mặt phẳng đáy một góc bằng $60^{\circ}$. Tính thể tích khối chóp $S.ABC$. 
	\choice
	{$V=\dfrac{a^3\sqrt{2}}{12}$}
	{$V=\dfrac{a^3\sqrt{3}}{4}$}
	{$V=\dfrac{a^3\sqrt{3}}{6}$}
	{\True $V=\dfrac{a^3\sqrt{3}}{12}$}
	\loigiai{
		\immini{
			$SH$ là chiều cao của khối chóp. \\
			Tam giác $ABC$ vuông cân tại $A$ nên $BC=a\sqrt{2}$, $AH=\dfrac{a\sqrt{2}}{2}$.\\
			Gọi $K$ là trung điểm của $AB$ nên $HK \perp AB$. Khi đó: $((SAB),(ABC))=\widehat{SKH}=60^\circ$ và $HK=\dfrac{AB}{2}=\dfrac{a}{2}$.\\
			Ta có: $SH=HK \cdot \tan 60^\circ=\dfrac{a\sqrt{3}}{2}$.\\
			Thể tích của khối chóp: $V=\dfrac{1}{3} \cdot \dfrac{a^2}{2} \cdot \dfrac{a\sqrt{3}}{2}=\dfrac{a^3\sqrt{3}}{12}$.	
		}{
			\begin{tikzpicture}[scale=1, font=\footnotesize, line join=round, line cap=round, >=stealth]
			\def \c{5}	\def \a{4}	\def \goc{-25}
			\def \h{3.5}	
			\path
			(0,0) coordinate (B)
			(\c,0) coordinate (C)
			(\goc:\a) coordinate (A)
			($(B)!.5!(C)$) coordinate (H)
			($(H)+(0,\h)$) coordinate (S)
			($(B)!.5!(A)$) coordinate (K)
			;
			\draw [dashed] (B)--(C) (S)--(H)--(A) (H)--(K);
			\draw (K)--(S)--(B)--(A)--(S)--(C)--(A);
			\draw  ($(H)!3mm!(B)$)--($(H)!3mm!(B)+(H)!3mm!(S)-(H)$) --($(H)!3mm!(S)$);
			\foreach \x/\y in {B/180,A/-90,C/0,S/90,H/35,K/220}
			\draw[fill] (\x) circle (1pt) +(\y:.3) node {$\x$};
			;
			\end{tikzpicture}}
	}
\end{ex}

\begin{ex}%[2H1B3-2]%Câu 20.
	Cho hình chóp $S.ABC$ có đáy $ABC$ là tam giác vuông cân tại $B$, có $BC=a$; Mặt bên $(SAC)$ vuông góc với đáy, các mặt bên còn lại đều tạo với mặt đáy một góc $45^\circ$. Tính thể tích khối chóp $SABC$. 
	\choice
	{\True $\dfrac{a^3}{12}$}
	{$a^3$}
	{$\dfrac{a^3}{6}$}
	{$\dfrac{a^3}{24}$}
	\loigiai{
		\immini{
			Gọi $SH$ là trung điểm của $AC$ $\Rightarrow SH$ là chiều cao của khối chóp. \\
			Ta có: $\triangle ABC$ vuông cân tại $B$ nên $BH=\dfrac{AC}{2}=\dfrac{a\sqrt{2}}{2}$.\\ 
			Vẽ đường cao $HK$ của tam giác vuông $ABH$ suy ra $((SAB),(ABC))=\widehat{SKH}=45^\circ$.\\ 
			Vì $\triangle ABH$ vuông tại $H$ nên $HK=\dfrac{AB}{2}=\dfrac{a}{2}$.\\
			Trong $\triangle SHK$: $SH=HK \cdot \tan 45^\circ=\dfrac{a}{2}$.\\
			Thể tích của khối chóp: $V=\dfrac{1}{3} \cdot \dfrac{a^2}{2} \cdot \dfrac{a}{2}=\dfrac{a^3}{12}$.	
		}{
			\begin{tikzpicture}[scale=1, font=\footnotesize, line join=round, line cap=round, >=stealth]
			\def \c{5}	\def \b{4}	\def \goc{-25}
			\def \h{3.5}	
			\path
			(0,0) coordinate (A)
			(\c,0) coordinate (C)
			(\goc:\b) coordinate (B)
			($(A)!.5!(C)$) coordinate (H)
			($(H)+(0,\h)$) coordinate (S)
			($(A)!.4!(B)$) coordinate (K)
			;
			\draw [dashed] (A)--(C) (S)--(H)--(B) (H)--(K);
			\draw (K)--(S)--(A)--(B)--(S)--(C)--(B);
			\draw  ($(H)!3mm!(A)$)--($(H)!3mm!(A)+(H)!3mm!(S)-(H)$) --($(H)!3mm!(S)$);
			\foreach \x/\y in {A/180,B/-90,C/0,S/90,H/35,K/200}
			\draw[fill] (\x) circle (1pt) +(\y:.3) node {$\x$};
			;
			\end{tikzpicture}}
	}
\end{ex}

\begin{ex}%[2H1B3-2]%Câu 21.
	Cho hình chóp $S.ABCD$ có đáy $ABCD$ là hình chữ nhật, mặt bên $SAD$ là tam giác đều cạnh $2a$ và nằm trong mặt phẳng vuông góc với mặt phẳng $(ABCD)$. Góc giữa mặt phẳng $(SBC)$ và mặt phẳng $(ABCD)$ là $30^{\circ}$. Thể tích của khối chóp $S.ABCD$ là 
	\choice
	{\True $\dfrac{2a^3\sqrt{3}}{3}$}
	{$\dfrac{a^3\sqrt{3}}{2}$}
	{$\dfrac{4a^3\sqrt{3}}{3}$}
	{$2a^3\sqrt{3}$}
	\loigiai{
		\immini{
			Gọi $H$, $M$ lần lượt là trung điểm của $AB$ và $BC$, suy ra $SH$ là chiều cao của khối chóp $\Rightarrow SH=\dfrac{2a\sqrt{3}}{2}=a\sqrt{3}$.\\
			Ta có: $((SBC),(ABCD))=\widehat{SMH}=30^\circ$.\\
			$AB=HM =\dfrac{SH}{\tan 30^\circ}=3a$.\\
			Thể tích của khối chóp: $V=\dfrac{1}{3} \cdot 9a^2 \cdot a\sqrt{3}=\dfrac{2a^3\sqrt{3}}{3}$.	
		}{
			\begin{tikzpicture}[scale=1, font=\footnotesize, line join=round, line cap=round, >=stealth]
			\def \c{5}	\def \b{4}	
			\def \d{3}	\def \goc{45}
			\def \h{3.5}	
			\path
			(0,0) coordinate (A)
			(\b,0) coordinate (B)
			(\goc:\d) coordinate (D)
			($(B)+(D)-(A)$) coordinate (C)
			($(A)!.5!(D)$) coordinate (H)
			($(H)+(0,\h)$) coordinate (S)
			($(B)!.5!(C)$) coordinate (M)
			;
			\draw [dashed] (M)--(H)--(S)--(D)--(C) (A)--(D);
			\draw (M)--(S)--(A)--(B)--(S)--(C)--(B);
			\draw  
			($(H)!3mm!(A)$)--($(H)!3mm!(A)+(H)!3mm!(S)-(H)$) --($(H)!3mm!(S)$);
			\foreach \x/\y in {A/180,B/0,C/0,D/180,S/90,H/-100,M/0}
			\draw[fill] (\x) circle (1pt) +(\y:.3) node {$\x$};
			\end{tikzpicture}}
	}
\end{ex}

\begin{ex}%[2H1B3-2]%Câu 22.
	Cho hình chóp $S.ABC$ có đáy là tam giác vuông tại $A$, $\widehat{ABC}=30^{\circ}$, $SAB$ là tam giác đều cạnh $a$, hình chiếu vuông góc của $S$ lên mặt phẳng $(ABC)$ là trung điểm $AB$. Thể tích khối chóp $S.ABC$ là
	\choice
	{$\dfrac{a^3\sqrt{3}}{9}$}
	{$\dfrac{a^3}{18}$}
	{$\dfrac{a^3\sqrt{3}}{3}$}
	{\True $\dfrac{a^3}{12}$}
	\loigiai{
		\immini{
			Gọi $H$ là trung điểm của $AB$ $\Rightarrow SH$ là chiều cao của khối chóp và $SH=\dfrac{a\sqrt{3}}{2}$. \\
			Tam giác $ABC$ vuông tại $A$ nên $AC=AB \cdot \tan 30^\circ=\dfrac{a\sqrt{3}}{3}$.\\
			Thể tích của khối chóp: $V=\dfrac{1}{3} \cdot \dfrac{1}{2} \cdot a \cdot \dfrac{a\sqrt{3}}{3} \cdot \dfrac{a\sqrt{3}}{2}=\dfrac{a^3}{12}$.	
		}{
			\begin{tikzpicture}[scale=1, font=\footnotesize, line join=round, line cap=round, >=stealth]
			\def \b{5}	\def \c{2.5}	\def \goc{-60}
			\def \h{3.5}	
			\path
			(0,0) coordinate (A)
			(\b,0) coordinate (B)
			(\goc:\c) coordinate (C)
			($(A)!.5!(B)$) coordinate (H)
			($(H)+(0,\h)$) coordinate (S)
			;
			\draw [dashed] (A)--(B) (S)--(H)--(C);
			\draw (S)--(A)--(C)--(S)--(B)--(C);
			\draw  ($(H)!3mm!(A)$)--($(H)!3mm!(A)+(H)!3mm!(S)-(H)$) --($(H)!3mm!(S)$);
			\foreach \x/\y in {A/180,C/-90,B/0,S/90,H/35}
			\draw[fill] (\x) circle (1pt) +(\y:.3) node {$\x$};
			;
			\end{tikzpicture}}
	}
\end{ex}

\begin{ex}%[2H1B3-2]%Câu 23.
	Cho hình chóp $S.ABCD$ có đáy $ABCD$ là hình chữ nhật, $AB=a$, $AD=a\sqrt{3}$, tam giác $SAB$ cân tại $S$ và nằm trong mặt phẳng vuông góc với đáy, khoảng cách giữa $AB$ và $SC$ bằng $\dfrac{3a}{2}$. Tính thể tích $V$ của khối chóp $S.ABCD$. 
	\choice
	{\True $V=a^3\sqrt{3}$}
	{$V=2a^3\sqrt{3}$}
	{$V=\dfrac{2a^3\sqrt{3}}{3}$}
	{$V=3a^3\sqrt{3}$}
	\loigiai{
		\immini{
			Gọi $H$ là trung điểm của $AB$, suy ra $SH$ là chiều cao của khối chóp.\\
			Vì $AB\parallel DC \Rightarrow AB\parallel (SCD)$\\
			$\Rightarrow \mathrm{d}(AB,SC)=\mathrm{d}(AB,(SCD))=\mathrm{d}(H,(SCD))$.\\
			Gọi $M$ là trung điểm của $DC$. Trong tam giác $SHM$ vẽ đường cao $HK \Rightarrow \mathrm{d}(H,(SCD))=HK=\dfrac{3a}{2}$.\\
			Ta có: $\dfrac{1}{SH^2}=\dfrac{1}{HK^2}-\dfrac{1}{HM^2}=\dfrac{4}{9a^2}-\dfrac{1}{3a^2}\Rightarrow SH=3a$.\\
			Thể tích của khối chóp: $V=\dfrac{1}{3} \cdot a \cdot a\sqrt{3} \cdot 3a=a^3\sqrt{3}$.	
		}{
			\begin{tikzpicture}[scale=1, font=\footnotesize, line join=round, line cap=round, >=stealth]
			\def \c{5}	\def \d{4}	
			\def \b{3}	\def \goc{60}
			\def \h{3.5}	
			\path
			(0,0) coordinate (A)
			(\d,0) coordinate (D)
			(\goc:\b) coordinate (B)
			($(D)+(B)-(A)$) coordinate (C)
			($(A)!.5!(B)$) coordinate (H)
			($(H)+(0,\h)$) coordinate (S)
			($(D)!.5!(C)$) coordinate (M)
			($(S)!(H)!(M)$) coordinate (K)
			;
			\draw [dashed] (M)--(H)--(S)--(B)--(C) (A)--(B) (H)--(K);
			\draw (M)--(S)--(A)--(D)--(S)--(C)--(D);
			\draw  
			($(H)!3mm!(A)$)--($(H)!3mm!(A)+(H)!3mm!(S)-(H)$) --($(H)!3mm!(S)$);
			\foreach \x/\y in {A/180,D/0,C/0,B/180,S/90,H/-60,M/0,K/20}
			\draw[fill] (\x) circle (1pt) +(\y:.3) node {$\x$};
			\end{tikzpicture}}
	}
\end{ex}


\begin{ex}%[2H1B3-2]%Câu 24.
	Cho hình chóp $S.ABCD$ có cạnh đáy là hình vuông cạnh $2a$. $SAD$ là tam giác cân tại $S$ và nằm trong mặt phẳng vuông góc với đáy. Góc giữa mặt bên $(SBC)$ và mặt đáy một góc $60^\circ$. Thể tích khối chóp $S.ABC$ là
\choice 
{ \True $\dfrac{8a^3\sqrt{3}}{3}$}
{ $\dfrac{4a^3\sqrt{15}}{5}$}
{ $\dfrac{2a^3\sqrt{15}}{5}$}
{ $6a^3\sqrt{3}$}
\loigiai{
	\immini{
		Gọi $H$, $M$ lần lượt là trung điểm của $AB$ và $BC$, suy ra $SH$ là chiều cao của khối chóp.\\
		Ta có: $((SBC),(ABCD))=\widehat{SMH}=60^\circ$.\\
		$SH =2a \cdot \tan 60^\circ=2a\sqrt{3}$.\\
		Thể tích của khối chóp: $V=\dfrac{1}{3} \cdot 4a^2 \cdot 2a\sqrt{3}=\dfrac{8a^3\sqrt{3}}{3}$.	
	}{
		\begin{tikzpicture}[scale=1, font=\footnotesize, line join=round, line cap=round, >=stealth]
		\def \c{5}	\def \b{4}	
		\def \d{3}	\def \goc{45}
		\def \h{3.5}	
		\path
		(0,0) coordinate (A)
		(\b,0) coordinate (B)
		(\goc:\d) coordinate (D)
		($(B)+(D)-(A)$) coordinate (C)
		($(A)!.5!(D)$) coordinate (H)
		($(H)+(0,\h)$) coordinate (S)
		($(B)!.5!(C)$) coordinate (M)
		;
		\draw [dashed] (M)--(H)--(S)--(D)--(C) (A)--(D);
		\draw (M)--(S)--(A)--(B)--(S)--(C)--(B);
		\draw  
		($(H)!3mm!(A)$)--($(H)!3mm!(A)+(H)!3mm!(S)-(H)$) --($(H)!3mm!(S)$);
		\foreach \x/\y in {A/180,B/0,C/0,D/180,S/90,H/-100,M/0}
		\draw[fill] (\x) circle (1pt) +(\y:.3) node {$\x$};
		\end{tikzpicture}}
}
\end{ex}


\begin{dang}{Khối chóp đều}
\end{dang}
\begin{vd}%[2H1B3-2]%Ví dụ 1.
	Có Cho chóp tam giác đều $S.ABC$ cạnh đáy bằng $a$ và cạnh bên bằng $2a$. Chứng minh rằng chân đường cao kẻ từ $S$ của hình chóp là tâm của tam giác đều $ABC$. Tính thể tích chóp đều $S.ABC$.
	\loigiai{
		\begin{center}
		\begin{tikzpicture}[scale=1, font=\footnotesize, line join=round, line cap=round, >=stealth]
			\def \c{5}	\def \b{4}	\def \goc{-30}
			\def \h{3.5}	
			\path
			(0,0) coordinate (A)
			(\c,0) coordinate (C)
			(\goc:\b) coordinate (B)
			($(B)!.5!(C)$) coordinate (H)
			($(A)!.5!(B)$) coordinate (M)
			(intersection of A--H and C--M) coordinate (O)
			($(O)+(0,\h)$) coordinate (S)
			;
			\draw [dashed] (H)--(A)--(C) (S)--(O) (C)--(M);
			\draw (S)--(A)--(B)--(S)--(C)--(B);
			\draw  ($(O)!3mm!(H)$)--($(O)!3mm!(H)+(O)!3mm!(S)-(O)$) --($(O)!3mm!(S)$);
			\foreach \x/\y in {A/180,B/-90,C/0,S/90,H/0,O/-90}
			\draw[fill] (\x) circle (1pt) +(\y:.3) node {$\x$};
			\path
			($(A)!.5!(C)$) node [above] {$a$}
			($(S)!.5!(C)$) node [right] {$2a$}
			($(A)!.5!(B)$) node [below] {$a$}
			($(B)!.5!(H)$) node [right] {$a$}
			;
		\end{tikzpicture}
		\end{center}
		Dựng $SO\perp(ABC)$. Ta có $SA=SB=SC\Rightarrow OA=OB=OC$.\\
		Vậy $O$ là tâm tam giác đều $ABC$. Ta có tam giác $ABC$ đều nên.\\
		$OA=\dfrac{2}{3}AH=\dfrac{2}{3}\cdot\dfrac{a\sqrt{3}}{2}=\dfrac{a\sqrt{3}}{3}$.\\
		$\Delta SAO\Rightarrow SO^2=SA^2-OA^2=\dfrac{11a^2}{3}\Rightarrow SO=\dfrac{a\sqrt{11}}{\sqrt{3}}$.\\
		Vậy $V=\dfrac{1}{3}S_{ABC}\cdot SO=\dfrac{a^3\sqrt{11}}{12}$.}
\end{vd}

\begin{vd}%[2H1B3-2]%Ví dụ 2.
	Cho khối chóp tứ giác $S.ABCD$ có tất cả các cạnh có độ dài bằng $a$.
	\begin{enumEX}[a)]{1}
	\item  Chứng minh rằng $S.ABCD$ là chóp tứ giác đều.
	\item  Tính thể tích khối chóp $S.ABCD$.
	\end{enumEX}
	\loigiai{
		\begin{center}
		\begin{tikzpicture}[scale=1, font=\footnotesize, line join=round, line cap=round, >=stealth]
			\def \c{5}	\def \b{4}	
			\def \d{3}	\def \goc{45}
			\def \h{3.5}	
			\path
			(0,0) coordinate (A)
			(\b,0) coordinate (B)
			(\goc:\d) coordinate (D)
			($(B)+(D)-(A)$) coordinate (C)
			($(A)!.5!(C)$) coordinate (O)
			($(O)+(0,\h)$) coordinate (S)
			;
			\draw [dashed] (O)--(S)--(D)--(C)--(A)--(D)--(B);
			\draw (S)--(A)--(B)--(S)--(C)--(B);
			\draw  ($(O)!3mm!(B)$)--($(O)!3mm!(B)+(O)!3mm!(S)-(O)$) --($(O)!3mm!(S)$)
			($(O)!3mm!(A)$)--($(O)!3mm!(A)+(O)!3mm!(S)-(O)$) --($(O)!3mm!(S)$);
			\foreach \x/\y in {A/180,B/0,C/0,D/180,S/90,O/-100}
			\draw[fill] (\x) circle (1pt) +(\y:.3) node {$\x$};
			\path
			($(A)!.5!(B)$) node [below] {$a$}
			($(S)!.5!(C)$) node [right] {$a$}
			;
		\end{tikzpicture}
		\end{center}
		Dựng $SO\perp(ABCD)$. Ta có $SA=SB=SC=SD$ nên $OA=OB=OC=OC\Rightarrow ABCD$ là hình thoi có đường tròn ngoại tiếp nên $ABCD$ là hình vuông.\\
		Ta có $SA^2+SB^2=AB^2+BC^2=AC^2$ nên $\Delta ASC$ vuông tại $S\Rightarrow OS=\dfrac{a\sqrt{2}}{2}$\\
		$\Rightarrow V=\dfrac{1}{3}S_{ABCD}\cdot SO=\dfrac{1}{3}a^2\cdot\dfrac{a\sqrt{2}}{2}=\dfrac{a^3\sqrt{2}}{6}$.\\
		Vậy $V=\dfrac{a^3\sqrt{2}}{6}$.}
\end{vd}


\begin{vd}%[2H1B3-2]%Ví dụ 3.
	Cho khối tứ diện đều $ABCD$ cạnh bằng $a$, $M$ là trung điểm $DC$.
	\begin{enumEX}[a)]{1}
	\item Tính thể tích khối tứ diện đều $ABCD$.
	\item Tính khoảng cách từ $M$ đến mặt phẳng $(ABC)$. Suy ra thể tích hình chóp $MABC$.
	\end{enumEX}
	\loigiai{
		\begin{center}
			\begin{tikzpicture}[scale=1, font=\footnotesize, line join=round, line cap=round, >=stealth]
			\def \c{5}	\def \b{4}	\def \goc{-30}
			\def \h{3.5}	
			\path
			(0,0) coordinate (A)
			(\c,0) coordinate (C)
			(\goc:\b) coordinate (B)
			($(B)!.5!(C)$) coordinate (K)
			($(A)!.5!(B)$) coordinate (I)
			(intersection of A--K and C--I) coordinate (O)
			($(O)+(0,\h)$) coordinate (D)
			($(D)!.5!(C)$) coordinate (M)
			($(O)!.5!(C)$) coordinate (H)
			;
			\draw [dashed] (K)--(A)--(C) (D)--(O) (C)--(I) (A)--(M)--(H);
			\draw (D)--(A)--(B)--(D)--(C)--(B);
			\draw  ($(O)!3mm!(K)$)--($(O)!3mm!(K)+(O)!3mm!(D)-(O)$) --($(O)!3mm!(D)$)
			($(H)!3mm!(M)$)--($(H)!3mm!(M)+(H)!3mm!(C)-(H)$) --($(H)!3mm!(C)$);
			\foreach \x/\y in {A/180,B/-90,C/0,D/90,I/200,O/-90,M/0,H/-90}
			\draw[fill] (\x) circle (1pt) +(\y:.3) node {$\x$};
			\path
			($(B)!.5!(C)$) node [right] {$a$}
			($(D)!.5!(A)$) node [left] {$a$}
			;
			\end{tikzpicture}
		\end{center}
		\begin{enumEX}[a)]{1}
		\item Gọi $O$ là tâm của $\Delta ABC\Rightarrow DO\perp(ABC)$.\\
		$S_{ABC}=\dfrac{a^2\sqrt{3}}{4},OC=\dfrac{2}{3}CI=\dfrac{a\sqrt{3}}{3}$.\\
		$\Delta DOC$ vuông có $DO=\sqrt{DC^2-OC^2}=\dfrac{a\sqrt{5}}{3}$ \\
		$ \Rightarrow V=\dfrac{1}{3}\cdot\dfrac{a^2\sqrt{3}}{4}\cdot\dfrac{a\sqrt{6}}{3}=\dfrac{a^3\sqrt{2}}{12} $.
		\item Kẻ $MH\parallel DO$, khoảng cách từ $M$ đến $(ABC)$ là $MH$.\\
		$MH=\dfrac{1}{2}DO=\dfrac{a\sqrt{6}}{6}\Rightarrow V=\dfrac{1}{3}S_{ABC}\cdot MH=\dfrac{1}{3}\cdot\dfrac{a^2\sqrt{3}}{4}\cdot\dfrac{a\sqrt{6}}{6}=\dfrac{a^3\sqrt{2}}{24}$.\\
		Vậy $V=\dfrac{a^3\sqrt{2}}{24}$.
		\end{enumEX}{1}}
\end{vd}

\subsubsection{Câu hỏi trắc nghiệm}
\begin{ex}%[2H1B3-2]%Câu 1.
	Cho hình chóp tam giác đều $S.ABC$ có cạnh đáy bằng $a$ và chiều cao của hình chóp là $a\sqrt{2}$. Tính theo $a$ thể tích khối chóp $S.ABC$ 
	\choice
	{$\dfrac{a^3\sqrt{6}}{4}$}
	{\True $\dfrac{a^3\sqrt{6}}{12}$}
	{$\dfrac{a^3}{6}$}
	{$\dfrac{a^3\sqrt{6}}{6}$}
	\loigiai{
	Thể tích của khối chóp là $V=\dfrac{1}{3} \cdot \dfrac{a^2\sqrt{3}}{4} \cdot a\sqrt{2}=\dfrac{a^3\sqrt{6}}{12}.$	
}
\end{ex}

\begin{ex}%[2H1B3-2]%Câu 2.
	Tính thể tích của chóp tam giác đều có tất cả các cạnh đều bằng $a$. 
	\choice
	{\True $\dfrac{a^3\sqrt{2}}{12}$}
	{$\dfrac{a^3\sqrt{2}}{4}$}
	{$\dfrac{a^3\sqrt{2}}{6}$}
	{$\dfrac{a^3\sqrt{2}}{2}$}
	\loigiai{
	\immini{
	Gọi $M$ là trung điểm của đoạn thẳng $BC$ và $O$ là trọng tâm của tam giác $ABC$.	\\
	Ta có: $AM=\dfrac{a\sqrt{3}}{2}$, $AO=\dfrac{a\sqrt{3}}{3}$.\\
	Tam giác $SAO$ vuông tại $O$: 
	$SO=\sqrt{a^2-\dfrac{3a^2}{9}}=\dfrac{a\sqrt{6}}{3}$.\\
	Thể tích của khối chóp: $V=\dfrac{1}{3}\cdot \dfrac{a^2\sqrt{3}}{4} \cdot \dfrac{a\sqrt{6}}{3}=\dfrac{a^3\sqrt{2}}{12}$.
	}{
		\begin{tikzpicture}[scale=1, font=\footnotesize, line join=round, line cap=round, >=stealth]
		\def \c{5}	\def \b{4}	\def \goc{-30}
		\def \h{3.5}	
		\path
		(0,0) coordinate (A)
		(\c,0) coordinate (C)
		(\goc:\b) coordinate (B)
		($(B)!.5!(C)$) coordinate (M)
		($(A)!.5!(B)$) coordinate (I)
		(intersection of A--K and C--I) coordinate (O)
		($(O)+(0,\h)$) coordinate (S)
				;
		\draw [dashed] (M)--(A)--(C) (S)--(O) (C)--(I) (A);
		\draw (S)--(A)--(B)--(S)--(C)--(B);
		\draw  ($(O)!3mm!(M)$)--($(O)!3mm!(M)+(O)!3mm!(D)-(O)$) --($(O)!3mm!(S)$)
		;
		\foreach \x/\y in {A/180,B/-90,C/0,S/90,M/0,O/-90}
		\draw[fill] (\x) circle (1pt) +(\y:.3) node {$\x$};
		\end{tikzpicture}}
}
\end{ex}

\begin{ex}%[2H1B3-2]%Câu 3.
	Cho hình chóp tam giác đều $S.ABC$ có cạnh đáy bằng $a\sqrt{3}$, cạnh bên bằng $2a$. Tính thể tích $V$ của khối chóp $S.ABC$. 
	\choice
	{$V=\dfrac{a^3\sqrt{3}}{4}$}
	{$V=\dfrac{3a^3\sqrt{3}}{2}$}
	{$V=\dfrac{3a^3\sqrt{3}}{4}$}
	{\True $V=\dfrac{3a^3}{4}$}
	\loigiai{
	\immini{
		Gọi $M$ là trung điểm của đoạn thẳng $BC$ và $O$ là trọng tâm của tam giác $ABC$.	\\
		Ta có: $AM=\dfrac{3a}{2}$, $AO=a$.\\
		Tam giác $SAO$ vuông tại $O$: 
		$SO=\sqrt{4a^2-a^2}=a\sqrt{3}$.\\
		Thể tích của khối chóp: $V=\dfrac{1}{3}\cdot \dfrac{3a^2\sqrt{3}}{4} \cdot a\sqrt{3}=\dfrac{3a^3}{4}$.
	}{
		\begin{tikzpicture}[scale=1, font=\footnotesize, line join=round, line cap=round, >=stealth]
		\def \c{5}	\def \b{4}	\def \goc{-30}
		\def \h{3.5}	
		\path
		(0,0) coordinate (A)
		(\c,0) coordinate (C)
		(\goc:\b) coordinate (B)
		($(B)!.5!(C)$) coordinate (M)
		($(A)!.5!(B)$) coordinate (I)
		(intersection of A--K and C--I) coordinate (O)
		($(O)+(0,\h)$) coordinate (S)
		;
		\draw [dashed] (M)--(A)--(C) (S)--(O) (C)--(I) (A);
		\draw (S)--(A)--(B)--(S)--(C)--(B);
		\draw  ($(O)!3mm!(M)$)--($(O)!3mm!(M)+(O)!3mm!(D)-(O)$) --($(O)!3mm!(S)$)
		;
		\foreach \x/\y in {A/180,B/-90,C/0,S/90,M/0,O/-90}
		\draw[fill] (\x) circle (1pt) +(\y:.3) node {$\x$};
		\end{tikzpicture}}
}
\end{ex}

\begin{ex}%[2H1B3-2]%Câu 4.
	Thể tích khối tứ diện đều có cạnh bằng $2$ là
	\choice
	{\True $\dfrac{2\sqrt{2}}{3}$}
	{$\dfrac{\sqrt{2}}{12}$}
	{$\dfrac{1}{8}$}
	{$2\sqrt{2}$}
	\loigiai{
		\immini{
			Gọi $M$ là trung điểm của đoạn thẳng $BC$ và $O$ là trọng tâm của tam giác $ABC$.	\\
			Ta có: $AM=\sqrt{3}$, $AO=\dfrac{2\sqrt{3}}{3}$.\\
			Tam giác $SAO$ vuông tại $O$: 
			$SO=\sqrt{4-\dfrac{12}{9}}=\dfrac{2\sqrt{6}}{3}$.\\
			Thể tích của khối chóp: $V=\dfrac{1}{3}\cdot \dfrac{4\sqrt{3}}{4} \cdot \dfrac{2\sqrt{6}}{3}=\dfrac{2\sqrt{2}}{3}$.
		}{
			\begin{tikzpicture}[scale=1, font=\footnotesize, line join=round, line cap=round, >=stealth]
			\def \c{5}	\def \b{4}	\def \goc{-30}
			\def \h{3.5}	
			\path
			(0,0) coordinate (A)
			(\c,0) coordinate (C)
			(\goc:\b) coordinate (B)
			($(B)!.5!(C)$) coordinate (M)
			($(A)!.5!(B)$) coordinate (I)
			(intersection of A--K and C--I) coordinate (O)
			($(O)+(0,\h)$) coordinate (S)
			;
			\draw [dashed] (M)--(A)--(C) (S)--(O) (C)--(I) (A);
			\draw (S)--(A)--(B)--(S)--(C)--(B);
			\draw  ($(O)!3mm!(M)$)--($(O)!3mm!(M)+(O)!3mm!(D)-(O)$) --($(O)!3mm!(S)$)
			;
			\foreach \x/\y in {A/180,B/-90,C/0,S/90,M/0,O/-90}
			\draw[fill] (\x) circle (1pt) +(\y:.3) node {$\x$};
			\end{tikzpicture}}
	}
\end{ex}

\begin{ex}%[2H1Y3-2]%Câu 5.
	Kim tự tháp Kê – ốp ở Ai Cập được xây dựng vào khoảng 2500 năm trước Công nguyên. Kim tự tháp này là một khối chóp tứ giác đều có chiều cao $147m$, cạnh đáy dài $230m$. Thể tích của nó là
	\choice
	{\True $2592100$ m$^3$}
	{$2592100$ m$^2$}
	{$7776300$ m$^3$}
	{$3888150$ m$^3$}
	\loigiai{
	Thể tích của khối chóp: $V=\dfrac{1}{3} \cdot 230^2 \cdot 14=72592100$ m$^2$.
}
\end{ex}

\begin{ex}%[2H1B3-2]%Câu 6.
	Cho $(H)$ là khối chóp tứ giác đều có tất cả các cạnh bằng $a$. Thể tích của $(H)$ bằng
	\choice
	{$\dfrac{a^3\sqrt{3}}{4}$}
	{$\dfrac{a^3\sqrt{3}}{2}$}
	{$\dfrac{a^3}{3}$}
	{\True $\dfrac{a^3\sqrt{2}}{6}$}
	\loigiai{
	\immini{
	Gọi $O=AC \cap BD$. Suy ra $SO$ là chiều cao của khối chóp. \\
	Tam giác $SAO$ vuông tại $O$ có: $SO=\sqrt{a^2-\dfrac{2a^2}{4}}=\dfrac{a\sqrt{2}}{2}$. \\
	Thể tích của khối chóp: $V=\dfrac{1}{3} \cdot a^2 \cdot \dfrac{a\sqrt{2}}{2}=\dfrac{a^3\sqrt{2}}{6}$.	
	}{
			\begin{tikzpicture}[scale=1, font=\footnotesize, line join=round, line cap=round, >=stealth]
			\def \c{5}	\def \b{4}	
			\def \d{3}	\def \goc{45}
			\def \h{3.5}	
			\path
			(0,0) coordinate (A)
			(\b,0) coordinate (B)
			(\goc:\d) coordinate (D)
			($(B)+(D)-(A)$) coordinate (C)
			($(A)!.5!(C)$) coordinate (O)
			($(O)+(0,\h)$) coordinate (S)
			;
			\draw [dashed] (O)--(S)--(D)--(C)--(A)--(D)--(B);
			\draw (S)--(A)--(B)--(S)--(C)--(B);
			\draw  ($(O)!3mm!(B)$)--($(O)!3mm!(B)+(O)!3mm!(S)-(O)$) --($(O)!3mm!(S)$)
			($(O)!3mm!(A)$)--($(O)!3mm!(A)+(O)!3mm!(S)-(O)$) --($(O)!3mm!(S)$);
			\foreach \x/\y in {A/180,B/0,C/0,D/180,S/90,O/-100}
			\draw[fill] (\x) circle (1pt) +(\y:.3) node {$\x$};
			\end{tikzpicture}}
	}
\end{ex}

\begin{ex}%[2H1B3-2]%Câu 7.
	Cho hình chóp đều $S.ABCD$ có chiều cao bằng $a\sqrt{2}$ và độ dài cạnh bên bằng $a\sqrt{6}$. Tính thể tích khối chóp $S.ABCD$. 
	\choice
	{\True $\dfrac{8a^3\sqrt{2}}{3}$}
	{$\dfrac{10a^3\sqrt{2}}{3}$}
	{$\dfrac{8a^3\sqrt{3}}{3}$}
	{$\dfrac{10a^3\sqrt{3}}{3}$}
	\loigiai{
		\immini{
			Gọi $O=AC \cap BD$. Suy ra $SO$ là chiều cao của khối chóp. \\
			Tam giác $SAO$ vuông tại $O$ có: $AO=\sqrt{6a^2-2a^2}=2a$. \\
			Ta có $AC=4a$ nên $AB=2a\sqrt{2}$.\\
			Thể tích của khối chóp: $V=\dfrac{1}{3} \cdot 8a^2 \cdot a\sqrt{2}=\dfrac{8a^3\sqrt{2}}{3}$.	
		}{
			\begin{tikzpicture}[scale=1, font=\footnotesize, line join=round, line cap=round, >=stealth]
			\def \c{5}	\def \b{4}	
			\def \d{3}	\def \goc{45}
			\def \h{3.5}	
			\path
			(0,0) coordinate (A)
			(\b,0) coordinate (B)
			(\goc:\d) coordinate (D)
			($(B)+(D)-(A)$) coordinate (C)
			($(A)!.5!(C)$) coordinate (O)
			($(O)+(0,\h)$) coordinate (S)
			;
			\draw [dashed] (O)--(S)--(D)--(C)--(A)--(D)--(B);
			\draw (S)--(A)--(B)--(S)--(C)--(B);
			\draw  ($(O)!3mm!(B)$)--($(O)!3mm!(B)+(O)!3mm!(S)-(O)$) --($(O)!3mm!(S)$)
			($(O)!3mm!(A)$)--($(O)!3mm!(A)+(O)!3mm!(S)-(O)$) --($(O)!3mm!(S)$);
			\foreach \x/\y in {A/180,B/0,C/0,D/180,S/90,O/-100}
			\draw[fill] (\x) circle (1pt) +(\y:.3) node {$\x$};
			\end{tikzpicture}}
	}
\end{ex}

\begin{ex}%[2H1B3-2]%Câu 8.
	Cho hình chóp tứ giác đều $S.ABCD$ có cạnh đáy bằng $2a$, cạnh bên bằng $3a$. Tính thể tích $V$ của khối chóp đã cho
	\choice
	{$V=4\sqrt{7}a^3$}
	{$V=\dfrac{4\sqrt{7}a^3}{9}$}
	{$V=\dfrac{4a^3}{3}$}
	{\True $V=\dfrac{4\sqrt{7}a^3}{3}$}
	\loigiai{
		\immini{
			Gọi $O=AC \cap BD$. Suy ra $SO$ là chiều cao của khối chóp. \\
			Ta có: $AO=\dfrac{2a\sqrt{2}}{2}=a\sqrt{2}$.\\
			Tam giác $SAO$ vuông tại $O$ có: $SO=\sqrt{9a^2-2a}=a\sqrt{7}$. \\
			Thể tích của khối chóp: $V=\dfrac{1}{3} \cdot 4a^2 \cdot a\sqrt{7}=\dfrac{4a^3\sqrt{7}}{3}$.	
		}{
			\begin{tikzpicture}[scale=1, font=\footnotesize, line join=round, line cap=round, >=stealth]
			\def \c{5}	\def \b{4}	
			\def \d{3}	\def \goc{45}
			\def \h{3.5}	
			\path
			(0,0) coordinate (A)
			(\b,0) coordinate (B)
			(\goc:\d) coordinate (D)
			($(B)+(D)-(A)$) coordinate (C)
			($(A)!.5!(C)$) coordinate (O)
			($(O)+(0,\h)$) coordinate (S)
			;
			\draw [dashed] (O)--(S)--(D)--(C)--(A)--(D)--(B);
			\draw (S)--(A)--(B)--(S)--(C)--(B);
			\draw  ($(O)!3mm!(B)$)--($(O)!3mm!(B)+(O)!3mm!(S)-(O)$) --($(O)!3mm!(S)$)
			($(O)!3mm!(A)$)--($(O)!3mm!(A)+(O)!3mm!(S)-(O)$) --($(O)!3mm!(S)$);
			\foreach \x/\y in {A/180,B/0,C/0,D/180,S/90,O/-100}
			\draw[fill] (\x) circle (1pt) +(\y:.3) node {$\x$};
			\end{tikzpicture}}
	}
\end{ex}

\begin{ex}%[2H1B3-2]%Câu 9.
	Cho hình chóp đều $S.ABC$, cạnh đáy bằng $a$. Gọi $M$, $N$ theo thứ tự là trung điểm $SB$, $SC$. Biết $(AMN)\perp (SBC)$. Khi đó $V_{S.ABC}$ là
	\choice
	{$\dfrac{3a^3\sqrt{3}}{2}$}
	{$\dfrac{a^3\sqrt{2}}{15}$}
	{\True $\dfrac{a^3\sqrt{5}}{24}$}
	{$\dfrac{a^3\sqrt{5}}{12}$}
\end{ex}

\begin{ex}%[2H1B3-2]%Câu 10.
	Cho hình chóp đều $S.ABC$ có cạnh đáy bằng $a,$ cạnh bên bằng $a\sqrt{2}$. Thể tích khối chóp $S.ABC$ là
	\choice
	{$\dfrac{a^3\sqrt{3}}{6}$}
	{$\dfrac{a^3\sqrt{3}}{12}$}
	{$\dfrac{a^3\sqrt{5}}{6}$}
	{\True $\dfrac{a^3\sqrt{5}}{12}$}
	\loigiai{
		\immini{
			Gọi $M$ là trung điểm của đoạn thẳng $BC$ và $O$ là trọng tâm của tam giác $ABC$.	\\
			Ta có: $AM=\dfrac{a\sqrt{3}}{2}$, $AO=\dfrac{a\sqrt{3}}{3}$.\\
			Tam giác $SAO$ vuông tại $O$: 
			$SO=\sqrt{2a^2-\dfrac{3a^2}{9}}=\dfrac{a\sqrt{15}}{3}$.\\
			Thể tích của khối chóp: $V=\dfrac{1}{3}\cdot \dfrac{a^2\sqrt{3}}{4} \cdot \dfrac{a\sqrt{15}}{3}=\dfrac{a^3\sqrt{5}}{12}$.
		}{
			\begin{tikzpicture}[scale=1, font=\footnotesize, line join=round, line cap=round, >=stealth]
			\def \c{5}	\def \b{4}	\def \goc{-30}
			\def \h{3.5}	
			\path
			(0,0) coordinate (A)
			(\c,0) coordinate (C)
			(\goc:\b) coordinate (B)
			($(B)!.5!(C)$) coordinate (M)
			($(A)!.5!(B)$) coordinate (I)
			(intersection of A--K and C--I) coordinate (O)
			($(O)+(0,\h)$) coordinate (S)
			;
			\draw [dashed] (M)--(A)--(C) (S)--(O) (C)--(I) (A);
			\draw (S)--(A)--(B)--(S)--(C)--(B);
			\draw  ($(O)!3mm!(M)$)--($(O)!3mm!(M)+(O)!3mm!(D)-(O)$) --($(O)!3mm!(S)$)
			;
			\foreach \x/\y in {A/180,B/-90,C/0,S/90,M/0,O/-90}
			\draw[fill] (\x) circle (1pt) +(\y:.3) node {$\x$};
			\end{tikzpicture}}
	}
\end{ex}

\begin{ex}%[2H1B3-2]%Câu 11.
	Cho hình chóp tứ giác đều $S.ABCD$ có cạnh đáy bằng $2a$, góc giữa mặt bên và mặt đáy bằng $60^\circ$. Tính theo $a$ thể tích khối chóp $S.ABCD$ 
	\choice
	{$\dfrac{2a^3\sqrt{3}}{3}$}
	{$\dfrac{2a^3\sqrt{6}}{3}$}
	{\True $\dfrac{4a^3\sqrt{3}}{3}$}
	{$\dfrac{a^3\sqrt{3}}{3}$}
	\loigiai{
	\immini{
			Gọi $O=AC \cap BD$ và $M$ là trung điểm của đoạn thẳng $BC$. Suy ra $SO$ là chiều cao của khối chóp và $\widehat{((SBC),(ABCD))}=\widehat{SMO}=60^\circ$. \\
			Ta có: $SO=OM \cdot \tan 60^\circ=a\sqrt{3}$.\\
			Thể tích của khối chóp: $V=\dfrac{1}{3} \cdot 4a^2 \cdot a\sqrt{3}=\dfrac{4a^3\sqrt{3}}{3}$.	
		}{
			\begin{tikzpicture}[scale=1, font=\footnotesize, line join=round, line cap=round, >=stealth]
			\def \c{5}	\def \b{4}	
			\def \d{3}	\def \goc{45}
			\def \h{3.5}	
			\path
			(0,0) coordinate (A)
			(\b,0) coordinate (B)
			(\goc:\d) coordinate (D)
			($(B)+(D)-(A)$) coordinate (C)
			($(A)!.5!(C)$) coordinate (O)
			($(O)+(0,\h)$) coordinate (S)
			($(B)!.5!(C)$) coordinate (M)
			;
			\draw [dashed] (M)--(O)--(S)--(D)--(C)--(A)--(D)--(B);
			\draw (M)--(S)--(A)--(B)--(S)--(C)--(B);
			\draw  ($(O)!3mm!(B)$)--($(O)!3mm!(B)+(O)!3mm!(S)-(O)$) --($(O)!3mm!(S)$)
			($(O)!3mm!(A)$)--($(O)!3mm!(A)+(O)!3mm!(S)-(O)$) --($(O)!3mm!(S)$);
			\foreach \x/\y in {A/180,B/0,C/0,D/180,S/90,O/-100,M/0}
			\draw[fill] (\x) circle (1pt) +(\y:.3) node {$\x$};
			\end{tikzpicture}}
	}
\end{ex}

\begin{ex}%[2H1B3-2]%Câu 12.
	Cho hình chóp tứ giác đều $S.ABCD$ có cạnh đáy bằng $3a$. Góc giữa cạnh bên và mặt đáy bằng $30^{\circ}$. Tính theo $a$ thể tích khối chóp $S.ABCD$. 
	\choice
	{$\dfrac{3a^3\sqrt{6}}{4}$}
	{\True $\dfrac{3a^3\sqrt{6}}{2}$}
	{$\dfrac{a^3\sqrt{6}}{2}$}
	{$3a^3\sqrt{6}$}
	\loigiai{
	\immini{
		Gọi $O=AC \cap BD$. Suy ra $SO$ là chiều cao của khối chóp và $\widehat{(SA,(ABCD))}=\widehat{SAO}=30^\circ$. \\
		Ta có: $SO=AO \cdot \tan 30^\circ=\dfrac{3a\sqrt{2}}{2} \cdot \dfrac{\sqrt{3}}{3}=\dfrac{a\sqrt{6}}{2}$.\\
		Thể tích của khối chóp: $V=\dfrac{1}{3} \cdot 9a^2 \cdot \dfrac{a\sqrt{6}}{2}=\dfrac{3a^3\sqrt{6}}{2}$.	
	}{
		\begin{tikzpicture}[scale=1, font=\footnotesize, line join=round, line cap=round, >=stealth]
		\def \c{5}	\def \b{4}	
		\def \d{3}	\def \goc{45}
		\def \h{3.5}	
		\path
		(0,0) coordinate (A)
		(\b,0) coordinate (B)
		(\goc:\d) coordinate (D)
		($(B)+(D)-(A)$) coordinate (C)
		($(A)!.5!(C)$) coordinate (O)
		($(O)+(0,\h)$) coordinate (S)
		;
		\draw [dashed] (O)--(S)--(D)--(C)--(A)--(D)--(B);
		\draw (S)--(A)--(B)--(S)--(C)--(B);
		\draw  ($(O)!3mm!(B)$)--($(O)!3mm!(B)+(O)!3mm!(S)-(O)$) --($(O)!3mm!(S)$)
		($(O)!3mm!(A)$)--($(O)!3mm!(A)+(O)!3mm!(S)-(O)$) --($(O)!3mm!(S)$);
		\foreach \x/\y in {A/180,B/0,C/0,D/180,S/90,O/-100}
		\draw[fill] (\x) circle (1pt) +(\y:.3) node {$\x$};
		\end{tikzpicture}}
}
\end{ex}

\begin{ex}%[2H1B3-2]%Câu 13.
	Cho hình chóp tứ giác đều $S.ABCD$ có cạnh đáy bằng $\dfrac{2a}{3}$. Góc giữa mặt bên và mặt đáy bằng $45^{\circ}$. Tính theo $a$ thể tích khối chóp $S.ABCD$ 
	\choice
	{$\dfrac{4a^3\sqrt{2}}{81}$}
	{$\dfrac{a^3\sqrt{2}}{81}$}
	{$\dfrac{a^3}{81}$}
	{\True $\dfrac{4a^3}{81}$}
	\loigiai{
	\immini{
		Gọi $O=AC \cap BD$ và $M$ là trung điểm của đoạn thẳng $BC$. Suy ra $SO$ là chiều cao của khối chóp và $\widehat{((SBC),(ABCD))}=\widehat{SMO}=45^\circ$. \\
		Ta có: $SO=OM \cdot \tan 45^\circ=\dfrac{a}{3}$.\\
		Thể tích của khối chóp: $V=\dfrac{1}{3} \cdot \dfrac{4a^2}{9} \cdot \dfrac{a}{3}=\dfrac{4a^3}{81}$.	
	}{
		\begin{tikzpicture}[scale=1, font=\footnotesize, line join=round, line cap=round, >=stealth]
		\def \c{5}	\def \b{4}	
		\def \d{3}	\def \goc{45}
		\def \h{3.5}	
		\path
		(0,0) coordinate (A)
		(\b,0) coordinate (B)
		(\goc:\d) coordinate (D)
		($(B)+(D)-(A)$) coordinate (C)
		($(A)!.5!(C)$) coordinate (O)
		($(O)+(0,\h)$) coordinate (S)
		($(B)!.5!(C)$) coordinate (M)
		;
		\draw [dashed] (M)--(O)--(S)--(D)--(C)--(A)--(D)--(B);
		\draw (M)--(S)--(A)--(B)--(S)--(C)--(B);
		\draw  ($(O)!3mm!(B)$)--($(O)!3mm!(B)+(O)!3mm!(S)-(O)$) --($(O)!3mm!(S)$)
		($(O)!3mm!(A)$)--($(O)!3mm!(A)+(O)!3mm!(S)-(O)$) --($(O)!3mm!(S)$);
		\foreach \x/\y in {A/180,B/0,C/0,D/180,S/90,O/-100,M/0}
		\draw[fill] (\x) circle (1pt) +(\y:.3) node {$\x$};
		\end{tikzpicture}}
}
\end{ex}

\begin{ex}%[2H1B3-2]%Câu 14.
	Cho hình chóp tam giác đều $S.ABC$ có cạnh đáy bằng $a$. Góc giữa cạnh bên và mặt đáy là $45^{\circ}$. Thể tích hình chóp $S.ABC$ là 
	\choice
	{$\dfrac{a^3\sqrt{3}}{4}$}
	{$\dfrac{a^3}{4}$}
	{\True $\dfrac{a^3}{12}$}
	{$\dfrac{a^3\sqrt{3}}{12}$}
	\loigiai{
		\immini{
			Gọi $O$ là trọng tâm của tam giác $ABC$, $M$ là trung điểm của đoạn thẳng $BC$. Suy ra $SO$ là chiều cao của khối chóp và $\widehat{(SA,(ABC))}=\widehat{SAO}=45^\circ$. \\
			Ta có: $SO=AO \cdot \tan 45^\circ=\dfrac{2}{3} \cdot \dfrac{a\sqrt{3}}{2}=\dfrac{a\sqrt{3}}{3}$.\\
			Thể tích của khối chóp: $V=\dfrac{1}{3} \cdot \dfrac{a^2\sqrt{3}}{4} \cdot \dfrac{a\sqrt{3}}{3}=\dfrac{a^3}{12}$.	
		}{
			\begin{tikzpicture}[scale=1, font=\footnotesize, line join=round, line cap=round, >=stealth]
			\def \c{5}	\def \b{4}	\def \goc{-30}
			\def \h{3.5}	
			\path
			(0,0) coordinate (A)
			(\c,0) coordinate (C)
			(\goc:\b) coordinate (B)
			($(B)!.5!(C)$) coordinate (M)
			($(A)!.5!(B)$) coordinate (I)
			(intersection of A--K and C--I) coordinate (O)
			($(O)+(0,\h)$) coordinate (S)
			;
			\draw [dashed] (M)--(A)--(C) (S)--(O) (C)--(I) (A);
			\draw (S)--(A)--(B)--(S)--(C)--(B);
			\draw  ($(O)!3mm!(M)$)--($(O)!3mm!(M)+(O)!3mm!(D)-(O)$) --($(O)!3mm!(S)$)
			;
			\foreach \x/\y in {A/180,B/-90,C/0,S/90,M/0,O/-90}
			\draw[fill] (\x) circle (1pt) +(\y:.3) node {$\x$};
			\end{tikzpicture}}
	}
\end{ex}

\begin{ex}%[2H1B3-2]%Câu 15.
	Cho hình chóp tứ giác đều có tất cả các cạnh bằng nhau, đường cao của một mặt bên là $a\sqrt{3}$. Thể tích $V$ của khối chóp đó là 
	\choice
	{$V=\dfrac{2a^3\sqrt{2}}{3}$}
	{\True $V=\dfrac{4a^3\sqrt{2}}{3}$}
	{$V=\dfrac{a^3\sqrt{2}}{6}$}
	{$V=\dfrac{a^3\sqrt{2}}{9}$}
	\loigiai{
	\immini{
		Gọi $O=AC \cap BD$ và $M$ là trung điểm của đoạn thẳng $AB$. Suy ra $SO$ là chiều cao của khối chóp. \\
		Vì tam giác $SBC$ đều và $SM$ là đường cao của tam giác nên $SM=\dfrac{BC\sqrt{3}}{2}=a\sqrt{3} \Rightarrow BC=2a$.\\
		Trong tam giác $SMO$ có: $SO=\sqrt{3a^2-a^2}=a\sqrt{2}$.\\
		Thể tích của khối chóp: $V=\dfrac{1}{3} \cdot 4a^2 \cdot a\sqrt{2}=\dfrac{4a^3\sqrt{2}}{3}$.	
	}{
		\begin{tikzpicture}[scale=1, font=\footnotesize, line join=round, line cap=round, >=stealth]
		\def \c{5}	\def \b{4}	
		\def \d{3}	\def \goc{45}
		\def \h{3.5}	
		\path
		(0,0) coordinate (A)
		(\b,0) coordinate (B)
		(\goc:\d) coordinate (D)
		($(B)+(D)-(A)$) coordinate (C)
		($(A)!.5!(C)$) coordinate (O)
		($(O)+(0,\h)$) coordinate (S)
		($(B)!.5!(C)$) coordinate (M)
		;
		\draw [dashed] (M)--(O)--(S)--(D)--(C)--(A)--(D)--(B);
		\draw (M)--(S)--(A)--(B)--(S)--(C)--(B);
		\draw  ($(O)!3mm!(B)$)--($(O)!3mm!(B)+(O)!3mm!(S)-(O)$) --($(O)!3mm!(S)$)
		($(O)!3mm!(A)$)--($(O)!3mm!(A)+(O)!3mm!(S)-(O)$) --($(O)!3mm!(S)$);
		\foreach \x/\y in {A/180,B/0,C/0,D/180,S/90,O/-100,M/0}
		\draw[fill] (\x) circle (1pt) +(\y:.3) node {$\x$};
		\end{tikzpicture}}
}
\end{ex}

\begin{ex}%[2H1B3-2]%Câu 16.
	Tính thể tích của khối chóp tứ giác đều có cạnh bên bằng 2a, góc giữa cạnh bên và mặt đáy bằng $60^{\circ}$. 
	\choice
	{$2a^3\sqrt{3}$}
	{$2a^3$}
	{\True $\dfrac{2a^3\sqrt{3}}{3}$}
	{$6a^3$}
	\loigiai{
	\immini{
		Gọi $O=AC \cap BD$. Suy ra $SO$ là chiều cao của khối chóp và $\widehat{(SA,(ABCD))}=\widehat{SAO}=60^\circ$. \\
		Ta có: $SO=SA \cdot \sin 60^\circ=2a\cdot \dfrac{\sqrt{3}}{2}=a\sqrt{3}$,\\
		$AO=SA \cdot \cos 60^\circ=2a\cdot \dfrac{1}{2}=a$.\\
		Khi đó: $AB=a\sqrt{2}$.\\
		Thể tích của khối chóp: $V=\dfrac{1}{3} \cdot 2a^2 \cdot a\sqrt{3}=\dfrac{2a^3\sqrt{3}}{3}$.	
	}{
		\begin{tikzpicture}[scale=1, font=\footnotesize, line join=round, line cap=round, >=stealth]
		\def \c{5}	\def \b{4}	
		\def \d{3}	\def \goc{45}
		\def \h{3.5}	
		\path
		(0,0) coordinate (A)
		(\b,0) coordinate (B)
		(\goc:\d) coordinate (D)
		($(B)+(D)-(A)$) coordinate (C)
		($(A)!.5!(C)$) coordinate (O)
		($(O)+(0,\h)$) coordinate (S)
		;
		\draw [dashed] (O)--(S)--(D)--(C)--(A)--(D)--(B);
		\draw (S)--(A)--(B)--(S)--(C)--(B);
		\draw  ($(O)!3mm!(B)$)--($(O)!3mm!(B)+(O)!3mm!(S)-(O)$) --($(O)!3mm!(S)$)
		($(O)!3mm!(A)$)--($(O)!3mm!(A)+(O)!3mm!(S)-(O)$) --($(O)!3mm!(S)$);
		\foreach \x/\y in {A/180,B/0,C/0,D/180,S/90,O/-100}
		\draw[fill] (\x) circle (1pt) +(\y:.3) node {$\x$};
		\end{tikzpicture}}
}
\end{ex}

\begin{ex}%[2H1B3-2]%Câu 17.
	Cho hình chóp tứ giác đều $S.ABCD$ có cạnh đáy bằng 2a, góc giữa cạnh bên và mặt phẳng đáy bằng $45^{\circ}$. Thể tích của khối chóp đó là 
	\choice
	{\True $\dfrac{4\sqrt{2}a^3}{3}$}
	{$\dfrac{8\sqrt{2}a^3}{3}$}
	{$\dfrac{a^3\sqrt{3}}{3}$}
	{$\dfrac{a^3\sqrt{3}}{6}$}
	\loigiai{
		\immini{
			Gọi $O=AC \cap BD$. Suy ra $SO$ là chiều cao của khối chóp và $\widehat{(SA,(ABCD))}=\widehat{SAO}=45^\circ$. \\
			Ta có: $SO=AO \cdot \tan 45^\circ=a\sqrt{2}$.\\
			Thể tích của khối chóp: $V=\dfrac{1}{3} \cdot 4a^2 \cdot a\sqrt{2}=\dfrac{4a^3\sqrt{2}}{3}$.	
		}{
			\begin{tikzpicture}[scale=1, font=\footnotesize, line join=round, line cap=round, >=stealth]
			\def \c{5}	\def \b{4}	
			\def \d{3}	\def \goc{45}
			\def \h{3.5}	
			\path
			(0,0) coordinate (A)
			(\b,0) coordinate (B)
			(\goc:\d) coordinate (D)
			($(B)+(D)-(A)$) coordinate (C)
			($(A)!.5!(C)$) coordinate (O)
			($(O)+(0,\h)$) coordinate (S)
			;
			\draw [dashed] (O)--(S)--(D)--(C)--(A)--(D)--(B);
			\draw (S)--(A)--(B)--(S)--(C)--(B);
			\draw  ($(O)!3mm!(B)$)--($(O)!3mm!(B)+(O)!3mm!(S)-(O)$) --($(O)!3mm!(S)$)
			($(O)!3mm!(A)$)--($(O)!3mm!(A)+(O)!3mm!(S)-(O)$) --($(O)!3mm!(S)$);
			\foreach \x/\y in {A/180,B/0,C/0,D/180,S/90,O/-100}
			\draw[fill] (\x) circle (1pt) +(\y:.3) node {$\x$};
			\end{tikzpicture}}
	}
\end{ex}

\begin{ex}%[2H1B3-2]%Câu 18.
	Một hình chóp tứ giác đều có đáy là hình vuông cạnh $a$, các mặt bên tạo với đáy một góc $\alpha$. Thể tích khối chóp đó là
	\choice
	{$\dfrac{a^3}{2}\sin\alpha$}
	{$\dfrac{a^3}{2}\tan\alpha$}
	{$\dfrac{a^3}{6}\cot\alpha$}
	{\True $\dfrac{a^3}{6}\tan\alpha$}
	\loigiai{
		\immini{
			Gọi $O=AC \cap BD$ và $M$ là trung điểm của đoạn thẳng $BC$. Suy ra $SO$ là chiều cao của khối chóp và $\widehat{((SBC),(ABCD))}=\widehat{SMO}=\alpha$. \\
			Ta có: $SO=OM \cdot \tan \alpha=\dfrac{a}{2} \tan \alpha$.\\
			Thể tích của khối chóp: $V=\dfrac{1}{3} \cdot a^2 \cdot \dfrac{a}{2} \tan \alpha=\dfrac{a^3}{6}\tan \alpha$.	
		}{
			\begin{tikzpicture}[scale=1, font=\footnotesize, line join=round, line cap=round, >=stealth]
			\def \c{5}	\def \b{4}	
			\def \d{3}	\def \goc{45}
			\def \h{3.5}	
			\path
			(0,0) coordinate (A)
			(\b,0) coordinate (B)
			(\goc:\d) coordinate (D)
			($(B)+(D)-(A)$) coordinate (C)
			($(A)!.5!(C)$) coordinate (O)
			($(O)+(0,\h)$) coordinate (S)
			($(B)!.5!(C)$) coordinate (M)
			;
			\draw [dashed] (M)--(O)--(S)--(D)--(C)--(A)--(D)--(B);
			\draw (M)--(S)--(A)--(B)--(S)--(C)--(B);
			\draw  ($(O)!3mm!(B)$)--($(O)!3mm!(B)+(O)!3mm!(S)-(O)$) --($(O)!3mm!(S)$)
			($(O)!3mm!(A)$)--($(O)!3mm!(A)+(O)!3mm!(S)-(O)$) --($(O)!3mm!(S)$);
			\foreach \x/\y in {A/180,B/0,C/0,D/180,S/90,O/-100,M/0}
			\draw[fill] (\x) circle (1pt) +(\y:.3) node {$\x$};
			\end{tikzpicture}}
	}
\end{ex}

\begin{ex}%[2H1B3-2]%Câu 19.
	Cho hình chóp tam giác đều có cạnh đáy bằng $a$ và cạnh bên bằng $b$. Thể tích của khối chóp là
	\choice
	{$\dfrac{a^2}{4}\sqrt{3b^2-a^2}$}
	{\True $\dfrac{a^2}{12}\sqrt{3b^2-a^2}$}
	{$\dfrac{a^2}{6}\sqrt{3b^2-a^2}$}
	{$a^2\sqrt{3b^2-a^2}$}
	\loigiai{
		\immini{
			Gọi $O$ là trọng tâm của tam giác $ABC$, $M$ là trung điểm của đoạn thẳng $BC$. Suy ra $SO$ là chiều cao của khối chóp. \\
			Ta có: $AO=\dfrac{2}{3} \cdot \dfrac{a\sqrt{3}}{2}=\dfrac{a\sqrt{3}}{3}$.\\
			Tam giác $SAO$ vuông tại $O$: $SO=\sqrt{b^2-\dfrac{3a^2}{9}}=\dfrac{\sqrt{3}}{3}\sqrt{3b^2-a^2}$.\\
			Thể tích của khối chóp: $V=\dfrac{1}{3} \cdot \dfrac{a^2\sqrt{3}}{4} \cdot \dfrac{a\sqrt{3}}{3}\sqrt{3b^2-a^2}=\dfrac{a^2}{12}\sqrt{3b^2-a^2}$.	
		}{
			\begin{tikzpicture}[scale=.8, font=\footnotesize, line join=round, line cap=round, >=stealth]
			\def \c{5}	\def \b{4}	\def \goc{-30}
			\def \h{3.5}	
			\path
			(0,0) coordinate (A)
			(\c,0) coordinate (C)
			(\goc:\b) coordinate (B)
			($(B)!.5!(C)$) coordinate (M)
			($(A)!.5!(B)$) coordinate (I)
			(intersection of A--K and C--I) coordinate (O)
			($(O)+(0,\h)$) coordinate (S)
			;
			\draw [dashed] (M)--(A)--(C) (S)--(O) (C)--(I) (A);
			\draw (S)--(A)--(B)--(S)--(C)--(B);
			\draw  ($(O)!3mm!(M)$)--($(O)!3mm!(M)+(O)!3mm!(D)-(O)$) --($(O)!3mm!(S)$)
			;
			\foreach \x/\y in {A/180,B/-90,C/0,S/90,M/0,O/-90}
			\draw[fill] (\x) circle (1pt) +(\y:.3) node {$\x$};
			\end{tikzpicture}}
	}
\end{ex}

\begin{ex}%[2H1B3-2]%Câu 20.
	Cho hình chóp tứ giác đều $S.ABCD$ có cạnh đáy bằng $a$, góc giữa cạnh bên và mặt đáy bằng $\varphi$. Khi đó thể tích khối chóp $S.ABCD$ bằng
	\choice
	{$\dfrac{a^3\sqrt{2}}{2}\tan\varphi$}
	{$\dfrac{a^3}{6}\tan\varphi$}
	{\True $\dfrac{a^3\sqrt{2}}{6}\tan\varphi$}
	{$\dfrac{a^3\sqrt{2}}{6}\cot\varphi$}
	\loigiai{
		\immini{
			Gọi $O=AC \cap BD$. Suy ra $SO$ là chiều cao của khối chóp và $\widehat{(SA,(ABCD))}=\widehat{SAO}=\varphi$. \\
			Ta có: $SO=AO \cdot \tan \varphi=\dfrac{a\sqrt{2}}{2} \cdot \tan \varphi$.\\
			Thể tích của khối chóp: $V=\dfrac{1}{3} \cdot a^2 \cdot \dfrac{a\sqrt{2}}{2} \cdot \tan \varphi=\dfrac{a^3\sqrt{2}}{6} \tan \varphi$.	
		}{
			\begin{tikzpicture}[scale=.8, font=\footnotesize, line join=round, line cap=round, >=stealth]
			\def \c{5}	\def \b{4}	
			\def \d{3}	\def \goc{45}
			\def \h{3.5}	
			\path
			(0,0) coordinate (A)
			(\b,0) coordinate (B)
			(\goc:\d) coordinate (D)
			($(B)+(D)-(A)$) coordinate (C)
			($(A)!.5!(C)$) coordinate (O)
			($(O)+(0,\h)$) coordinate (S)
			;
			\draw [dashed] (O)--(S)--(D)--(C)--(A)--(D)--(B);
			\draw (S)--(A)--(B)--(S)--(C)--(B);
			\draw  ($(O)!3mm!(B)$)--($(O)!3mm!(B)+(O)!3mm!(S)-(O)$) --($(O)!3mm!(S)$)
			($(O)!3mm!(A)$)--($(O)!3mm!(A)+(O)!3mm!(S)-(O)$) --($(O)!3mm!(S)$);
			\foreach \x/\y in {A/180,B/0,C/0,D/180,S/90,O/-100}
			\draw[fill] (\x) circle (1pt) +(\y:.3) node {$\x$};
			\end{tikzpicture}}
	}
\end{ex}

\begin{ex}%[2H1B3-2]%Câu 21.
	Cho hình chóp tam giác đều có cạnh đáy bằng $a$ và cạnh bên tạo với mặt đáy một góc $\varphi$. Tính thể tích của khối chóp đó. 
	\choice
	{\True $\dfrac{a^3\tan\varphi}{12}$}
	{$\dfrac{a^3\tan\varphi}{6}$}
	{$\dfrac{a^3\cot\varphi}{12}$}
	{$\dfrac{a^3\cot\varphi}{6}$}
	\loigiai{
		\immini{
			Gọi $O$ là trọng tâm của tam giác $ABC$, $M$ là trung điểm của đoạn thẳng $BC$. Suy ra $SO$ là chiều cao của khối chóp và $\widehat{(SA,(ABC))}=\widehat{SAO}=\varphi$. \\
			Ta có: $SO=AO \cdot \tan \varphi=\dfrac{2}{3} \cdot \dfrac{a\sqrt{3}}{2}\tan \varphi=\dfrac{a\sqrt{3}}{3}\tan \varphi$.\\
			Thể tích của khối chóp: $V=\dfrac{1}{3} \cdot \dfrac{a^2\sqrt{3}}{4} \cdot \dfrac{a\sqrt{3}}{3}\tan \varphi=\dfrac{a^3}{12}\tan \varphi$.	
		}{
			\begin{tikzpicture}[scale=1, font=\footnotesize, line join=round, line cap=round, >=stealth]
			\def \c{5}	\def \b{4}	\def \goc{-30}
			\def \h{3.5}	
			\path
			(0,0) coordinate (A)
			(\c,0) coordinate (C)
			(\goc:\b) coordinate (B)
			($(B)!.5!(C)$) coordinate (M)
			($(A)!.5!(B)$) coordinate (I)
			(intersection of A--K and C--I) coordinate (O)
			($(O)+(0,\h)$) coordinate (S)
			;
			\draw [dashed] (M)--(A)--(C) (S)--(O) (C)--(I) (A);
			\draw (S)--(A)--(B)--(S)--(C)--(B);
			\draw  ($(O)!3mm!(M)$)--($(O)!3mm!(M)+(O)!3mm!(D)-(O)$) --($(O)!3mm!(S)$)
			;
			\foreach \x/\y in {A/180,B/-90,C/0,S/90,M/0,O/-90}
			\draw[fill] (\x) circle (1pt) +(\y:.3) node {$\x$};
			\end{tikzpicture}}
	}
\end{ex}

\begin{ex}%[2H1B3-2]%Câu 22.
	Cho hình chóp tứ giác đều $S.ABCD$ có cạnh đáy bằng $a$ và cạnh bên tạo với mặt phẳng đáy một góc $60^{\circ}$. Tính thể tích $V$ của khối chóp $S.ABCD$. 
	\choice
	{$V=\dfrac{\sqrt{6}a^3}{2}$}
	{$V=\dfrac{a^3\sqrt{6}}{3}$}
	{$V=\dfrac{\sqrt{3}a^3}{2}$}
	{\True $V=\dfrac{\sqrt{6}a^3}{6}$}
	\loigiai{
		\immini{
			Gọi $O=AC \cap BD$. Suy ra $SO$ là chiều cao của khối chóp và $\widehat{(SA,(ABCD))}=\widehat{SAO}=60^\circ$. \\
			Ta có: $SO=AO \cdot \tan 60^\circ=\dfrac{a\sqrt{6}}{2}$.\\
			Thể tích của khối chóp: $V=\dfrac{1}{3} \cdot a^2 \cdot \dfrac{a\sqrt{6}}{2}=\dfrac{a^3\sqrt{6}}{6}$.	
		}{
			\begin{tikzpicture}[scale=1, font=\footnotesize, line join=round, line cap=round, >=stealth]
			\def \c{5}	\def \b{4}	
			\def \d{3}	\def \goc{45}
			\def \h{3.5}	
			\path
			(0,0) coordinate (A)
			(\b,0) coordinate (B)
			(\goc:\d) coordinate (D)
			($(B)+(D)-(A)$) coordinate (C)
			($(A)!.5!(C)$) coordinate (O)
			($(O)+(0,\h)$) coordinate (S)
			;
			\draw [dashed] (O)--(S)--(D)--(C)--(A)--(D)--(B);
			\draw (S)--(A)--(B)--(S)--(C)--(B);
			\draw  ($(O)!3mm!(B)$)--($(O)!3mm!(B)+(O)!3mm!(S)-(O)$) --($(O)!3mm!(S)$)
			($(O)!3mm!(A)$)--($(O)!3mm!(A)+(O)!3mm!(S)-(O)$) --($(O)!3mm!(S)$);
			\foreach \x/\y in {A/180,B/0,C/0,D/180,S/90,O/-100}
			\draw[fill] (\x) circle (1pt) +(\y:.3) node {$\x$};
			\end{tikzpicture}}
	}
\end{ex}

\begin{ex}%[2H1B3-2]%Câu 23.
	Cho hình chóp tam giác đều có cạnh đáy bằng $a$ và cạnh bên tạo đáy góc $60^{\circ}$. Thể tích của khối chóp đó bằng 
	\choice
	{\True $\dfrac{a^3\sqrt{3}}{12}$}
	{$\dfrac{a^3\sqrt{3}}{6}$}
	{$\dfrac{a^3\sqrt{3}}{36}$}
	{$\dfrac{a^3\sqrt{3}}{18}$}
	\loigiai{
		\immini{
			Gọi $O$ là trọng tâm của tam giác $ABC$, $M$ là trung điểm của đoạn thẳng $BC$. Suy ra $SO$ là chiều cao của khối chóp và $\widehat{(SA,(ABC))}=\widehat{SAO}=60^\circ$. \\
			Ta có: $SO=AO \cdot \tan 60^\circ=\dfrac{2}{3} \cdot \dfrac{a\sqrt{3}}{2}\tan 60^\circ=a$.\\
			Thể tích của khối chóp: $V=\dfrac{1}{3} \cdot \dfrac{a^2\sqrt{3}}{4} \cdot a=\dfrac{a^3\sqrt{3}}{12}$.	
		}{
			\begin{tikzpicture}[scale=1, font=\footnotesize, line join=round, line cap=round, >=stealth]
			\def \c{5}	\def \b{4}	\def \goc{-30}
			\def \h{3.5}	
			\path
			(0,0) coordinate (A)
			(\c,0) coordinate (C)
			(\goc:\b) coordinate (B)
			($(B)!.5!(C)$) coordinate (M)
			($(A)!.5!(B)$) coordinate (I)
			(intersection of A--K and C--I) coordinate (O)
			($(O)+(0,\h)$) coordinate (S)
			;
			\draw [dashed] (M)--(A)--(C) (S)--(O) (C)--(I) (A);
			\draw (S)--(A)--(B)--(S)--(C)--(B);
			\draw  ($(O)!3mm!(M)$)--($(O)!3mm!(M)+(O)!3mm!(D)-(O)$) --($(O)!3mm!(S)$)
			;
			\foreach \x/\y in {A/180,B/-90,C/0,S/90,M/0,O/-90}
			\draw[fill] (\x) circle (1pt) +(\y:.3) node {$\x$};
			\end{tikzpicture}}
	}
\end{ex}

\begin{ex}%[2H1B3-2]%Câu 24.
	Cho hình chóp tứ giác đều $S.ABCD$, đáy $ABCD$ có diện tích $16 cm^2$, diện tích một mặt bên là $8\sqrt{3} cm^2$. Tính thể tích $V$ của khối chóp $S.ABCD$. 
	\choice
	{$V=\dfrac{32\sqrt{2}}{3} cm^3$}
	{$V=\dfrac{32\sqrt{13}}{3} cm^3$}
	{\True $V=\dfrac{32\sqrt{11}}{3} cm^3$}
	{$V=\dfrac{32\sqrt{15}}{3} cm^3$}
	\loigiai{
		\immini{
			Ta có: $S_{ABCD}=a^2=16 \Rightarrow a=4$.\\
			$S_{SBC}=\dfrac{1}{2} \cdot SM \cdot BC =8\sqrt{3} \Rightarrow SM=4\sqrt{3}$.\\
			Gọi $O=AC \cap BD$. Suy ra $SO$ là chiều cao của khối chóp. \\
			Ta có: $SO=\sqrt{48-4}=2\sqrt{11}$.\\
			Thể tích của khối chóp: $V=\dfrac{1}{3} \cdot 16 \cdot 2\sqrt{11}=\dfrac{32\sqrt{11}}{3}$.	
		}{
			\begin{tikzpicture}[scale=1, font=\footnotesize, line join=round, line cap=round, >=stealth]
			\def \c{5}	\def \b{4}	
			\def \d{3}	\def \goc{45}
			\def \h{3.5}	
			\path
			(0,0) coordinate (A)
			(\b,0) coordinate (B)
			(\goc:\d) coordinate (D)
			($(B)+(D)-(A)$) coordinate (C)
			($(A)!.5!(C)$) coordinate (O)
			($(O)+(0,\h)$) coordinate (S)
			($(B)!.5!(C)$) coordinate (M)
			;
			\draw [dashed] (O)--(S)--(D)--(C)--(A)--(D)--(B) (O)--(M);
			\draw (M)--(S)--(A)--(B)--(S)--(C)--(B);
			\draw  ($(O)!3mm!(B)$)--($(O)!3mm!(B)+(O)!3mm!(S)-(O)$) --($(O)!3mm!(S)$)
			($(O)!3mm!(A)$)--($(O)!3mm!(A)+(O)!3mm!(S)-(O)$) --($(O)!3mm!(S)$);
			\foreach \x/\y in {A/180,B/0,C/0,D/180,S/90,O/-100,M/0}
			\draw[fill] (\x) circle (1pt) +(\y:.3) node {$\x$};
			\end{tikzpicture}}
	}
\end{ex}

\begin{ex}%[2H1B3-2]%Câu 25.
	Cho một hình chóp tứ giác đều có góc tạo bởi mặt bên và mặt đáy bằng $60^{\circ}$ và diện tích xung quanh bằng $8a^2$. Tính diện tích $S$ của mặt đáy hình chóp. 
	\choice
	{$4a^2\sqrt{3}$}
	{\True $4a^2$}
	{$2a^2$}
	{$2a^2\sqrt{3}$}
	\loigiai{
		\immini{
			Gọi $O=AC \cap BD$ và $M$ là trung điểm của đoạn thẳng $BC$. Suy ra $SO$ là chiều cao của khối chóp và $\widehat{((SBC),(ABCD))}=\widehat{SMO}=60^\circ$. \\
			Tam giác $SOM$: $SM=\dfrac{OM}{\cos 60^\circ}=2 \cdot OM =BC$.\\
			Ta có: $4 \cdot S_{SBC}=8a^2 \Rightarrow 2 \cdot SM \cdot BC =8a^2 \Rightarrow BC^2=4a^2$.\\			
			Vậy $S_{ABCD}=BC^2=4a^2$.	
		}{
			\begin{tikzpicture}[scale=1, font=\footnotesize, line join=round, line cap=round, >=stealth]
			\def \c{5}	\def \b{4}	
			\def \d{3}	\def \goc{45}
			\def \h{3.5}	
			\path
			(0,0) coordinate (A)
			(\b,0) coordinate (B)
			(\goc:\d) coordinate (D)
			($(B)+(D)-(A)$) coordinate (C)
			($(A)!.5!(C)$) coordinate (O)
			($(O)+(0,\h)$) coordinate (S)
			($(B)!.5!(C)$) coordinate (M)
			;
			\draw [dashed] (M)--(O)--(S)--(D)--(C)--(A)--(D)--(B);
			\draw (M)--(S)--(A)--(B)--(S)--(C)--(B);
			\draw  ($(O)!3mm!(B)$)--($(O)!3mm!(B)+(O)!3mm!(S)-(O)$) --($(O)!3mm!(S)$)
			($(O)!3mm!(A)$)--($(O)!3mm!(A)+(O)!3mm!(S)-(O)$) --($(O)!3mm!(S)$);
			\foreach \x/\y in {A/180,B/0,C/0,D/180,S/90,O/-100,M/0}
			\draw[fill] (\x) circle (1pt) +(\y:.3) node {$\x$};
			\end{tikzpicture}}
	}
\end{ex}

\begin{dang}{Khối chóp và phương pháp tỉ số thể tích}
\end{dang}
\begin{vd}%[2H1B3-3]%Ví dụ 1.
	Cho hình chóp $S.ABC$ có tam giác $ABC$ vuông cân ở $B$, $AC=a\sqrt{2}$. $SA$ vuông góc với đáy $ABC, SA=a$.
	\begin{enumEX}[a)]{1}
	\item Tính thể tích khối chóp $S.ABC$.\\
	\item Gọi $G$ là trọng tâm tam giác $ABC$, mặt phẳng $(\alpha)$ qua $AG$ và song song với $BC$ cắt $SB,SC$ lần lượt tại $M,N$. Tính thể tích khối chóp $S.AMN$.
	\end{enumEX}
	\loigiai{
		\begin{center}
			\begin{tikzpicture}[scale=1, font=\footnotesize, line join=round, line cap=round, >=stealth]
			\def \c{5}	\def \b{4}	\def \gocB{-40}
			\def \h{3.5}	
			\path
			(0,0) coordinate (A)
			(\c,0) coordinate (C)
			(\gocB:\b) coordinate (B)
			(0,\h) coordinate (S)
			($(B)!.5!(C)$) coordinate (H)
			($(S)!2/3!(H)$) coordinate (G)
			($(S)!2/3!(B)$) coordinate (M)
			($(S)!2/3!(C)$) coordinate (N)
			;
			\draw [dashed] (A)--(C) (M)--(A)--(N)--cycle (A)--(G);
			\draw (S)--(A)--(B)--(S)--(C)--(B);
			%\draw  ($(O)!3mm!(K)$)--($(O)!3mm!(K)+(O)!3mm!(D)-(O)$) --($(O)!3mm!(D)$)		;
			\foreach \x/\y in {A/180,B/-90,C/0,S/90,G/0,M/240,N/30}
			\draw[fill] (\x) circle (1pt) +(\y:.3) node {$\x$};
			\end{tikzpicture}
		\end{center}
		\begin{enumEX}[a)]{1}		
		\item Ta có $V_{S.ABC}=\dfrac{1}{3}S_{ABC}\cdot SA$ và $SA=a$. $\Delta ABC$ cân có.\\
		$AC=a\sqrt{2}\Rightarrow AB=a\Rightarrow S_{ABC}=\dfrac{1}{2}a^2$. Vậy $V_{SABC}=\dfrac{1}{3}\cdot\dfrac{1}{2}a^2\cdot a=\dfrac{a^3}{6}$.
		\item Gọi $I$ là trung điểm $BC$. $G$ là trọng tâm nên $\dfrac{SG}{SI}=\dfrac{2}{3}$.
		\end{enumEX}
		$(\alpha)\parallel BC\Rightarrow MN\parallel BC\Rightarrow\dfrac{SM}{SB}=\dfrac{SN}{SC}=\dfrac{SG}{SI}=\dfrac{2}{3}$ \\
		$ \Rightarrow\dfrac{V_{SAMN}}{V_{SABC}}=\dfrac{SM}{SB}\cdot\dfrac{SN}{SC}=\dfrac{4}{9} $. Vậy $V_{SAMN}=\dfrac{4}{9}V_{SABC}=\dfrac{2a^3}{27}$.}
\end{vd}

\begin{vd}%[2H1B3-3]%Ví dụ 2.
	Cho tam giác $ABC$ vuông cân ở $A$ và $AB=a$. Trên đường thẳng qua $C$ và vuông góc với mặt phẳng $(ABC)$ lấy điểm $D$ sao cho $CD=a$. Mặt phẳng qua $C$ vuông góc với $BD$, cắt $BD$ tại $F$ và cắt $AD$ tại $E$.
	\begin{enumEX}[a)]{1}
	\item Tính thể tích khối tứ diện $ABCD$.
	\item Chứng minh $CE\perp(ABD)$.
	\item Tính thể tích khối tứ diện $CDEF$.
	\end{enumEX}
	\loigiai{
		\begin{center}
			\begin{tikzpicture}[scale=1, font=\footnotesize, line join=round, line cap=round, >=stealth]
			\def \a{3}	\def \b{5}	\def \gocA{-45}
			\def \h{3.5}	
			\path
			(0,0) coordinate (C)
			(\b,0) coordinate (B)
			(\gocA:\a) coordinate (A)
			(0,\h) coordinate (D)
			($(B)!(C)!(D)$) coordinate (F)
			($(A)!(C)!(D)$) coordinate (E)
			;
			\draw [dashed] (C)--(B) (C)--(F);
			\draw (D)--(C)--(A)--(D)--(B)--(A) (C)--(E)--(F);
			\newcommand{\gocv}[4][black]{\draw[#1] ($(#3)!6pt!(#2)$)--($(#3)!2!($($(#3)!6pt!(#2)$)!.5!($(#3)!6pt!(#4)$)$)$)--($(#3)!6pt!(#4)$);}
			\gocv{D}{C}{A}; \gocv{D}{C}{B}; \gocv{C}{E}{A}; \gocv{D}{F}{C};
			\foreach \x/\y in {A/-90,B/0,C/180,D/90,E/0,F/30}
			\draw[fill] (\x) circle (1pt) +(\y:.3) node {$\x$};
			\end{tikzpicture}
		\end{center}
		\begin{enumEX}[a)]{1}
		\item $V_{ABCD}=\dfrac{1}{3}S_{ABC}\cdot CD=\dfrac{a^3}{6}$.
		\item Ta có: $AB\perp AC,AB\perp CD\Rightarrow AB\perp(ACD)\Rightarrow AB\perp EC$.\\
		Ta có $DB\perp EC\Rightarrow EC\perp(ABD)$.
		\item Ta có $\dfrac{V_{DCEF}}{V_{DABC}}=\dfrac{DE}{DA}\cdot\dfrac{DF}{DB}$. Mà $DE\cdot DA=DC^2$, chia cho $DA^2$.\\
		Ta có $\dfrac{DE}{DA}=\dfrac{DC^2}{DA^2}=\dfrac{a^2}{2a^2}=\dfrac{1}{2}$. Tương tự $\dfrac{DF}{DB}=\dfrac{DC^2}{DB^2}=\dfrac{a^2}{DC^2+CB^2}=\dfrac{1}{3}$.\\		
		Vậy $\dfrac{V_{DCEF}}{V_{DABC}}=\dfrac{1}{6}\Rightarrow V_{DCEF}=\dfrac{1}{6}V_{ABCD}=\dfrac{a^3}{36}$.
		\end{enumEX}}
\end{vd}

\begin{vd}%[2H1B3-3]%Ví dụ 3.
	Cho khối chóp tứ giác đều $ABCD$. Một mặt phẳng $(\alpha)$ đi qua $A,B$ và trung điểm $M$ của $SC$. Tính tỉ số thể tích của hai phần khối chóp bị phân chia bởi mặt phẳng đó.
	\loigiai{
		\begin{center}
		\begin{tikzpicture}[scale=1, font=\footnotesize, line join=round, line cap=round, >=stealth]
			\def \b{4}		\def \d{2.5}
			\def \goc{45}	\def \h{4.5}
			\path
			(0,0) coordinate (C)
			(\b,0) coordinate (B)
			(\goc:\d) coordinate (D)
			($(B)+(D)-(C)$) coordinate (A)
			($(A)!.5!(C)$) coordinate (O)
			($(O)+(0,\h)$) coordinate (S)
			($(S)!.5!(C)$) coordinate (M)
			($(S)!.5!(D)$) coordinate (N)
			;
			\draw [dashed] (O)--(S)--(D)--(A)--(C)--(D)--(B)--(N)--(A)--(M)--(N);
			\draw (S)--(A)--(B)--(S)--(C)--(B)--(M);
			\newcommand{\gocv}[4][black]{\draw[#1] ($(#3)!6pt!(#2)$)--($(#3)!2!($($(#3)!6pt!(#2)$)!.5!($(#3)!6pt!(#4)$)$)$)--($(#3)!6pt!(#4)$);}
			\gocv{S}{O}{C}	
			\foreach \x/\y in {C/180,B/0,A/0,D/180,S/90,O/-100,M/180,N/45}
			\draw[fill] (\x) circle (1pt) +(\y:.3) node {$\x$};
		\end{tikzpicture}
		\end{center}
		Kẻ $MN\parallel CD(N\in SD)$ thì hình thang $ABMN$ là thiết diện của khối chóp khi cắt bởi mặt phẳng $(ABM)$.\\
		$\dfrac{V_{SBMN}}{V_{SADB}}=\dfrac{SN}{SD}=\dfrac{1}{2}\Rightarrow V_{SANB}=\dfrac{1}{2}V_{SADB}=\dfrac{1}{4}V_{SABCD}$.\\
		$\dfrac{V_{SBMN}}{V_{SBCD}}=\dfrac{SM}{SC}\cdot\dfrac{SN}{SD}=\dfrac{1}{2}\cdot\dfrac{1}{2}=\dfrac{1}{4}\Rightarrow V_{SBMN}=\dfrac{1}{4}V_{SBCD}=\dfrac{1}{8}V_{SABCD}$.\\
		Mà $V_{SABMN}=V_{SANB}+V_{SBMN}=\dfrac{3}{8}V_{SABCD}$. Suy ra $V_{ABMN\cdot ABCD}=\dfrac{5}{8}V_{SABCD}$.\\
		Do đó $\dfrac{V_{SABMN}}{V_{ABMN\cdot ABCD}}=\dfrac{3}{5}$.}
\end{vd}

\begin{vd}%[2H1B3-2]%Ví dụ 4.
	Cho hình chóp tứ giác đều $S.ABCD$ đáy hình vuông cạnh $a$. Cạnh bên tạo với đáy góc $60^{\circ}$. Gọi $M$ là trung điểm $SC$. Mặt phẳng đi qua $AM$ và song song với $BD$, cắt $SB$ tại $E$ và cắt $SD$ tại $F$.
	\begin{enumEX}[a)]{1}
	\item Hãy xác định mặt phẳng $(AEMF)$.
	\item Tính thể tích khối chóp $S.ABCD$.
	\item Tính thể tích khối chóp $S.AEMF$.
	\end{enumEX}
	\loigiai{
		\begin{center}
			\begin{tikzpicture}[scale=1, font=\footnotesize, line join=round, line cap=round, >=stealth]
			\def \c{5}	\def \b{4}	
			\def \d{2.5}	\def \goc{45}
			\def \h{4.5}	
			\path
			(0,0) coordinate (A)
			(\b,0) coordinate (B)
			(\goc:\d) coordinate (D)
			($(B)+(D)-(A)$) coordinate (C)
			($(A)!.5!(C)$) coordinate (O)
			($(O)+(0,\h)$) coordinate (S)
			($(S)!.5!(C)$) coordinate (M)
			($(S)!2/3!(O)$) coordinate (I)
			($(S)!2/3!(B)$) coordinate (E)
			($(S)!2/3!(D)$) coordinate (F)
			;
			\draw [dashed] (O)--(S)--(D)--(C)--(A)--(D)--(B) (A)--(F)--(M)--cycle (E)--(F);
			\draw (S)--(A)--(B)--(S)--(C)--(B) (A)--(E)--(M);
			
			\foreach \x/\y in {A/180,B/0,C/0,D/180,S/90,O/-100,M/0,I/0,E/0,F/180}
			\draw[fill] (\x) circle (1pt) +(\y:.3) node {$\x$};
			\end{tikzpicture}
		\end{center}
		\begin{enumEX}[a)]{1}
		\item Gọi $I=SO\cap AM$. Ta có $(AEMF)\parallel BD\Rightarrow EF\parallel BD$.\\
		\item $V_{S.ABCD}=\dfrac{1}{3}S_{ABCD}\cdot SO$ với $S_{ABCD}=a^2$. $\Delta SOA$ có $SO=AO\cdot\tan 60^{\circ}=\dfrac{a\sqrt{6}}{2}$.\\
		Vậy $V_{S.ABCD}=\dfrac{a^3\sqrt{6}}{6}$.
		\item Phân chia khối chóp tứ giác ta có $V_{S.AEMF}=V_{S.AMF}+V_{S.AME}=2V_{SAMF}$.\\
		$V_{S.ABCD}=2V_{SACD}=2V_{SABC}$. Xét khối chóp $S.AMF$ và $S.ACD$ ta có $\dfrac{SM}{SC}=\dfrac{1}{2}$.\\
		$\Delta SAC$ có trọng tâm $I,EF\parallel BD$ nên:\\
		$\dfrac{SI}{SO}=\dfrac{SF}{SD}=\dfrac{2}{3}\Rightarrow\dfrac{V_{SAMF}}{V_{SACD}}=\dfrac{SM}{SC}\cdot\dfrac{SF}{SD}=\dfrac{1}{3}$\\
		$\Rightarrow V_{SAMF}=\dfrac{1}{3}V_{SACD}=\dfrac{1}{6}V_{SABCD}=\dfrac{a^3\sqrt{6}}{36}$ \\
		$ \Rightarrow V_{S.AEMF}=2\cdot\dfrac{a^3\sqrt{6}}{36}=\dfrac{a^3\sqrt{6}}{18} $.
	\end{enumEX}	}
\end{vd}


\begin{vd}%[2H1B3-2]%Ví dụ 5.
	Cho hình chóp $S.ABCD$ đáy $ABCD$ là hình vuông cạnh $a$. $SA$ vuông góc với đáy, $SA=a\sqrt{2}$. Gọi $B',D'$ là hình chiếu của $A$ lên $SB,SD$. Mặt phẳng $(AB'D')$ cắt $SC$ tại $C'$.
	\begin{enumerate}
		\item Tính thể tích khối chóp $S.ABCD$.
		\item Chứng minh $SC\perp(AB'D')$.
		\item Tính thể tích khối chóp $S.AB'C'D'$.
	\end{enumerate}
	\loigiai{
		\begin{center}
				\begin{tikzpicture}[scale=0.7]
				\tkzDefPoints{0/0/A, -2/-2/B, 5/0/D}
				\coordinate (C) at ($(B)+(D)-(A)$);
				\coordinate (S) at ($(A)+(0,4)$);
				\draw (S)--(B)--(C)--(S)--(D)--(C);
				\draw[dashed] (B)--(A)--(D) (S)--(A);
				\coordinate (O) at ($(A)!0.5!(C)$);
				\coordinate (B') at ($(S)!0.6!(B)$);
				\coordinate (D') at ($(S)!0.6!(D)$);
				\tkzInterLL(S,O)(D',B')\tkzGetPoint{I}
				\tkzInterLL(C,S)(I,A)\tkzGetPoint{C'}
				\draw (B')--(C')--(D');
				\draw[dashed] (C')--(A)--(C) (B)--(D) (B')--(A)--(D')--(B');
				\tkzLabelPoints[below](A,B,C,O)
				\tkzLabelPoints[right](D,D')
				\tkzLabelPoints[above right](S,C')
				\tkzLabelPoints[left](B')
				\tkzMarkRightAngle(S,A,D)
				\tkzMarkRightAngle(S,A,B)
				\tkzMarkRightAngle(A,B',S)
				\tkzMarkRightAngle(A,D',S)
				\tkzMarkRightAngle(A,O,B)
				\end{tikzpicture}
			\end{center}
		\begin{enumerate}
		\item Ta có $V_{S.ABCD}=\dfrac{1}{3}S_{ABCD}\cdot SA=\dfrac{a^3\sqrt{2}}{3}$.
		\item Ta có $BC\perp(SAB)\Rightarrow BC\perp AB'$ và $SB\perp AB'$. Suy ra $AB'\perp(SBC)$.\\
		nên $AB'\perp SC$. Tương tự $AD'\perp SC$. Vậy $SC\perp(AB'D')$.
		\item Tính $V_{S.AB'C'}$ ta có $\dfrac{V_{SAB'C'}}{V_{SABC}}=\dfrac{SB'}{SB}\cdot\dfrac{SC'}{SC}$. Tam giác $SAC$ vuông cân nên $\dfrac{SC'}{SC}=\dfrac{1}{2}$.\\
		Ta có: $\dfrac{SB'}{SB}=\dfrac{SA^2}{SB^2}=\dfrac{2a^2}{SA^2+AB^2}=\dfrac{2a^2}{3a^2}=\dfrac{2}{3}$.\\
		nên $\dfrac{V_{SAB'C'}}{V_{SABC}}=\dfrac{1}{3}\Rightarrow V_{SAB'C'}=\dfrac{1}{3}\dfrac{a^3\sqrt{2}}{3}=\dfrac{a^3\sqrt{2}}{9}$.\\
		Vậy $V_{S.AB'C'D'}=2V_{S.AB'C'}=\dfrac{2a^3\sqrt{2}}{9}$.
	\end{enumerate}}
\end{vd}

\subsubsection{Câu hỏi trắc nghiệm}
\begin{ex}%[2H1B3-3]%Câu 1.
	Cho tứ diện $MNPQ$. Gọi $I$; $J$; $K$ lần lượt là trung điểm của các cạnh $MN$; $MP$; $MQ$. Tỉ số thể tích $\dfrac{V_{MUK}}{V_{MNPQ}}$ là
	\choice
	{$\dfrac{1}{3}$}
	{$\dfrac{1}{4}$}
	{$\dfrac{1}{6}$}
	{\True $\dfrac{1}{8}$}
\end{ex}

\begin{ex}%[2H1B3-3]%Câu 2.
	Cho hình chóp $S.ABC$. Trên 3 cạnh $SA$, $SB$, $SC$ lần lượt lấy 3 điểm $A'$, $B'$, $C'$ sao cho $SA'=\dfrac{1}{2}SA$; $SB'=\dfrac{1}{2}SB$, $SC'=\dfrac{1}{2}SC$. Gọi $V$ và $V'$ lần lượt là thể tích của các khối chóp $S.ABC$ và $S.A'B'C'$. Khi đó tỷ số $\dfrac{V'}{V}$ là
	\choice
	{$\dfrac{1}{8}$}
	{\True $\dfrac{1}{12}$}
	{$\dfrac{1}{6}$}
	{$\dfrac{1}{16}$}
\end{ex}

\begin{ex}%[2H1B3-3]%Câu 3.
	Cho tứ diện $ABCD$, hai điểm $M$ và $N$ lần lượt trên hai cạnh $AB$ và $AD$ sao cho $\dfrac{AM}{MB}=\dfrac{1}{3};\dfrac{AN}{AD}=\dfrac{1}{4}$, khi đó tỉ số $\dfrac{V_{ACMN}}{V_{ABCD}}$ bằng
	\choice
	{$\dfrac{1}{15}$}
	{$\dfrac{1}{9}$}
	{\True $\dfrac{1}{12}$}
	{$\dfrac{1}{16}$}
\end{ex}

\begin{ex}%[2H1B3-3]%Câu 4.
	Cho hình chóp $S.ABC$, gọi $M$, $N$ lần lượt là trung điểm của $SA, SB$. Tính tỉ số $\dfrac{V_{S.ABC}}{V_{S.MNC}}$. 
	\choice
	{\True $4$}
	{$\dfrac{1}{2}$}
	{$2$}
	{$\dfrac{1}{4}$}
\end{ex}

\begin{ex}%[2H1B3-3]%Câu 5.
	Cho khối chop $O.ABC$. Trên ba cạnh $OA,OB,OC$ lần lượt lấy ba điểm sao cho $2OA'=OA, 4OB'=OB, 3OC'=OC$. Tính tỉ số $\dfrac{V_{O.A'B'C'}}{V_{O.ABC}}$. 
	\choice
	{$\dfrac{1}{12}$}
	{\True $\dfrac{1}{24}$}
	{$\dfrac{1}{16}$}
	{$\dfrac{1}{32}$}
\end{ex}

\begin{ex}%[2H1B3-3]%Câu 6.
	Cho tứ diện $ABCD$ có $B'$ là trung điểm $AB$, $C'$ thuộc đoạn $AC$ và thỏa mãn $2AC'=C'C$. Trong các số dưới đây, số nào ghi giá trị tỉ số thể tích giữa khối tứ diện $AB'C'D$ và phần còn lại của khối tứ diện $ABCD$?
	\choice
	{$\dfrac{1}{6}$}
	{\True $\dfrac{1}{5}$}
	{$\dfrac{1}{3}$}
	{$\dfrac{2}{5}$}
\end{ex}

\begin{ex}%[2H1B3-3]%Câu 7.
	Cho khối chóp $S.ACB$. Gọi $G$ là trọng tâm giác $SBC$. Mặt phẳng $(\alpha)$ qua $AG$ và song song với $BC$ cắt $SB, SC$ lần lượt tại $I, J$. Gọi $V_{S.AIJ}, V_{S.ABC}$ lần lượt là thế tích của các khối tứ diện $SAIJ$ và $SABC$. Khi đó khẳng định nào sau đây là đúng?
	\choice
	{$\dfrac{V_{S.AIJ}}{V_{S.ABC}}=1$}
	{$\dfrac{V_{S.AIJ}}{V_{S.ABC}}=\dfrac{2}{3}$}
	{\True $\dfrac{V_{S.AIJ}}{V_{S.ABC}}=\dfrac{4}{9}$}
	{$\dfrac{V_{S.AIJ}}{V_{S.ABC}}=\dfrac{8}{27}$}
\end{ex}

\begin{ex}%[2H1B3-3]%Câu 8.
	Cho khối chóp $S.ABCD$. Gọi $A', B', C', D'$ lần lượt là trung điểm của $SA, SB, SC, SD$. Khi đó tỉ số thế tích của hai khối chóp $S.A'B'C'D'$ và $S.ABCD$ bằng 
	\choice
	{$\dfrac{1}{2}$}
	{$\dfrac{1}{4}$}
	{\True $\dfrac{1}{8}$}
	{$\dfrac{1}{16}$}
\end{ex}

\begin{ex}%[2H1B3-3]%Câu 9.
	Cho khối chóp tứ giác đều $S.ABCD$. Mặt phẳng $(\alpha)$ đi qua $A, B$ và trung điểm $M$ của $SC$. Tỉ số thể tích của hai phần khối chóp bị phân chia bởi mặt phẳng đó là
	\choice
	{$\dfrac{1}{4}$}
	{$\dfrac{3}{8}$}
	{$\dfrac{5}{8}$}
	{\True $\dfrac{3}{5}$}
\end{ex}

\begin{ex}%[2H1B3-3]%Câu 10.
	Cho tứ diện $ABCD$ có thể tích $V$. Gọi $V'$ là thể tích của khối tứ diện có các đỉnh là trọng tâm của các mặt của khối tứ diện $ABCD$. Tính tỉ số $\dfrac{V'}{V}$. 
	\choice
	{$\dfrac{V'}{V}=\dfrac{8}{27}$}
	{$\dfrac{V'}{V}=\dfrac{23}{27}$}
	{\True $\dfrac{V'}{V}=\dfrac{1}{27}$}
	{$\dfrac{V'}{V}=\dfrac{4}{27}$}
\end{ex}

\begin{ex}%[2H1B3-3]%Câu 11.
	Cho tứ diện có thể tích bằng $V$. Gọi $V'$ là thể tích của khối đa diện có các đỉnh là các trung điểm của các cạnh của khối tứ diện đã cho, tính tỉ số $\dfrac{V'}{V}$. 
	\choice
	{\True $\dfrac{V'}{V}=\dfrac{1}{2}$}
	{$\dfrac{V'}{V}=\dfrac{1}{4}$}
	{$\dfrac{V'}{V}=\dfrac{2}{3}$}
	{$\dfrac{V'}{V}=\dfrac{5}{8}$}
\end{ex}

\begin{ex}%[2H1B3-3]%Câu 12.
	Cho hình chóp tam giác $S.ABC$ có $M$ là trung điểm của $SB$, $N$ là điểm trên cạnh $SC$ sao cho $NS=2NC$. Kí hiệu $V_1,V_2$ lần lượt là thể tích của các khối chóp $A.BMNC$ và $S.AMN$. Tính tỉ số $\dfrac{V_1}{V_2}$. 
	\choice
	{$\dfrac{V_1}{V_2}=\dfrac{2}{3}$}
	{$\dfrac{V_1}{V_2}=\dfrac{1}{2}$}
	{\True $\dfrac{V_1}{V_2}=2$}
	{$\dfrac{V_1}{V_2}=3$}
\end{ex}

\begin{ex}%[2H1B3-3]%Câu 13.
	Cho hình chóp $S.ABC$. Gọi $(\alpha)$ là mặt phẳng qua $A$ và song song với $BC$. $(\alpha)$ cắt $SB$, $SC$ lần lượt tại $M,N$. Tính tỉ số $\dfrac{SM}{SB}$ biết $(\alpha)$ chia khối chóp thành 2 phần có thể tích bằng nhau. 
	\choice
	{$\dfrac{1}{2}$}
	{\True $\dfrac{1}{\sqrt{2}}$}
	{$\dfrac{1}{4}$}
	{$\dfrac{1}{2\sqrt{2}}$}
\end{ex}

\begin{ex}%[2H1K3-3]%Câu 14.
	Hình chóp $S.ACB$ có $SA$ vuông góc với mặt phẳng đáy, $SA=a$, $AC=a\sqrt{2}$, $AB=3a$. Gọi $M,N$ là hình chiếu vuông góc của $A$ lên các cạnh $SB$ và $SC$. Đặt $;k=\dfrac{V_{SAMN}}{V_{SABC}}$, khi đó giá trị của $k$ là
	\choice
	{$\dfrac{1}{\sqrt{30}}$}
	{$\dfrac{1}{3}$}
	{\True $\dfrac{1}{30}$}
	{$\dfrac{1}{2}$}
\end{ex}

\begin{ex}%[2H1B3-3]%Câu 15.
	Cho hình chóp $S.ABCD$ có đáy $ABCD$ là hình thoi. Gọi $M;N$ lần lượt là trung điểm của $SB, SC$. Tỷ lệ thể tích của $\dfrac{V_{SABCD}}{V_{SAMND}}$ bằng
	\choice
	{\True $\dfrac{8}{3}$}
	{$\dfrac{1}{4}$}
	{$4$}
	{$\dfrac{3}{8}$}
\end{ex}

\begin{ex}%[2H1B3-3]%Câu 16.
	Cho khối chóp $S.ABC$. Trên 3 cạnh $SA, SB, SC$ lần lượt lấy 3 điểm $A', B', C'$ sao cho $SA'=\dfrac{1}{3}SA; SB'=\dfrac{1}{4}SB; SC'=\dfrac{1}{2}SC$. Gọi $V$ và $V'$ lần lượt là thể tích của các khối chóp $S.ABC$ và $S.A'B'C'$. Khi đó tỷ số $\dfrac{V'}{V}$ là
	\choice
	{$12$}
	{$\dfrac{1}{12}$}
	{$24$}
	{\True $\dfrac{1}{24}$}
\end{ex}

\begin{ex}%[2H1K3-3]%Câu 17.
	Xét khối hình chóp tứ giác đều $S.ABCD$. Mặt phẳng đi qua $A$, trọng tâm $G$ của tam giác $SBC$ và song song với $BC$ chia khối chóp thành hai phần, tính tỉ số thể tích (số lớn chia số bé) của chúng. 
	\choice
	{$\dfrac{5}{3}$}
	{\True $\dfrac{5}{4}$}
	{$\dfrac{3}{2}$}
	{$\dfrac{4}{3}$}
\end{ex}

\begin{ex}%[2H1B3-3]%Câu 18.
	Cho hình chóp $S.ABCD$. Gọi $M,N$ lần lượt là trung điểm của $SA,SB$. Tính tỉ số $\dfrac{V'}{V}$ thể tích của hai khối chóp $S.MNCD$ và khối chóp $S.ABCD$. 
	\choice
	{\True $\dfrac{V'}{V}=\dfrac{3}{8}$}
	{$\dfrac{V'}{V}=\dfrac{1}{4}$}
	{$\dfrac{V'}{V}=\dfrac{1}{2}$}
	{$\dfrac{V'}{V}=\dfrac{5}{8}$}
\end{ex}

\begin{ex}%[2H1K3-3]%Câu 19.
	Cho hình chóp tam giác $S.ABC$ có $M$ là trung điểm của $SB$, $N$ là điểm trên cạnh $SC$ sao cho $NS=2NC$, $P$ là điểm trên cạnh $SA$ sao cho $PA=2PS$. Kí hiệu $V_1,V_2$ lần lượt là thể tích của các khối tứ diện $BMNP$ và $SABC$. Tính tỉ số $\dfrac{V_1}{V_2}$. 
	\choice
	{\True $\dfrac{V_1}{V_2}=\dfrac{1}{9}$}
	{$\dfrac{V_1}{V_2}=\dfrac{3}{4}$}
	{$\dfrac{V_1}{V_2}=\dfrac{2}{3}$}
	{$\dfrac{V_1}{V_2}=\dfrac{1}{3}$}
\end{ex}

\begin{ex}%[2H1K3-3]%Câu 20.
	Cho hình chóp tam giác $S.ABC$ có $M$ là trung điểm $SB$, $N$ là điểm trên $SC$ sao cho $NS=2NC$. Kí hiệu $V_1$, $V_2$ lần lượt là thể tích khối chóp $A.BMNC$ và $S.AMN$. Tính tỉ số $\dfrac{V_1}{V_2}$. 
	\choice
	{$\dfrac{V_1}{V_2}=\dfrac{2}{3}$}
	{$\dfrac{V_1}{V_2}=\dfrac{1}{2}$}
	{\True $\dfrac{V_1}{V_2}=2$}
	{$\dfrac{V_1}{V_2}=3$}
\end{ex}

\begin{ex}%[2H1K3-3]%Câu 21.
	Cho hình chóp tam giác $S.ABC$ có $M$ là trung điểm $SB$, $N$ là điểm trên $SC$ sao cho $NS=2NC$, là điểm trên $SA$ sao cho $PA=2PS$. Kí hiệu $V_1$, $V_2$ lần lượt là thể tích khối chóp $BMNP$ và $S.ABC$. Tính tỉ số $\dfrac{V_1}{V_2}$. 
	\choice
	{\True $\dfrac{V_1}{V_2}=\dfrac{1}{9}$}
	{$\dfrac{V_1}{V_2}=\dfrac{3}{4}$}
	{$\dfrac{V_1}{V_2}=\dfrac{2}{3}$}
	{$\dfrac{V_1}{V_2}=\dfrac{1}{3}$}
\end{ex}

\begin{ex}%[2H1K3-3]%Câu 22.
	Cho hình chóp $S.ABCD$ có đáy $ABCD$ là hình chữ nhật. Mặt phẳng $(\alpha)$ đi qua $A, B$ và trung điểm $M$ của $SC$. Mặt phẳng $(\alpha)$ chia khối chóp đã cho thành hai phần có thể tích lần lượt là $V_1, V_2$ với $V_1<V_2$. Tính tỉ số $\dfrac{V_1}{V_2}$. 
	\choice
	{$\dfrac{V_1}{V_2}=\dfrac{1}{4}$}
	{$\dfrac{V_1}{V_2}=\dfrac{3}{8}$}
	{$\dfrac{V_1}{V_2}=\dfrac{5}{8}$}
	{\True $\dfrac{V_1}{V_2}=\dfrac{3}{5}$}
\end{ex}

\begin{ex}%[2H1K3-3]%Câu 23.
	Cho hình chóp đều $S.ABCD$. Gọi $N$ là trung điểm $SB, M$ là điểm đối xứng với $B$ qua $A$. Mặt phẳng $(MNC)$ chia khối chóp $S.ABCD$ thành hai phần có thể tích lần lượt là $V_1, V_2$ với $V_1<V_2$. Tính tỉ số $\dfrac{V_1}{V_2}$. 
	\choice
	{\True $\dfrac{V_1}{V_2}=\dfrac{5}{7}$}
	{$\dfrac{V_1}{V_2}=\dfrac{5}{11}$}
	{$\dfrac{V_1}{V_2}=\dfrac{5}{9}$}
	{$\dfrac{V_1}{V_2}=\dfrac{5}{13}$}
\end{ex}

\begin{ex}%[2H1K3-3]%Câu 24.
	Cho hình chóp tứ giá đều $S.ABCD$ có cạnh đáy bằng $a$, cạnh bên hợp với đáy một góc $60^{\circ}$. Gọi $M$ là điểm đối xứng của $C$ qua $D$, $N$ là trung điểm $SC$. Mặt phẳng $(BMN)$ chia khối chóp $S.ABCD$ thành hai phần. Tỉ số thể tích giữa hai phần (phần lớn trên phần bé) bằng 
	\choice
	{\True $\dfrac{7}{5}$}
	{$\dfrac{1}{7}$}
	{$\dfrac{7}{3}$}
	{$\dfrac{6}{5}$}
\end{ex}

\begin{ex}%[2H1K3-3]%Câu 25.
	Cho hình chóp $S.ABCD$ có đáy $ABCD$ là hình bình hành. Trên các cạnh $SA,SB,SC$ lần lượt lấy các điểm $A',B',C'$ sao cho $SA=2SA';SB=3SB';SC=4SC'$, mặt phẳng $(A'B'C')$ cắt cạnh $SD$ tại $D'$, gọi $V_1,V_2$ lần lượt là thể tích của hai khối chóp $S.A'B'C'D'$; $S.ABCD$. Khi đó $\dfrac{V_1}{V_2}$ bằng 
	\choice
	{\True $\dfrac{1}{24}$}
	{$\dfrac{1}{26}$}
	{$\dfrac{7}{12}$}
	{$\dfrac{7}{24}$}
\end{ex}

\begin{ex}%[2H1K3-3]%Câu 26.
	Cho tứ diện $S.ABC$, $M$ và $N$ là các điểm thuộc $SA$ và $SB$ sao cho $MA=2SM$, $SN=2NB$, $(\alpha)$ là mặt phẳng qua $MN$ và song song với $SC$. Kí hiệu $(H_1)$ và $(H_2)$ là các khối đa diện có được khi chia khối tứ diện $S.ABC$ bởi mặt phẳng $(\alpha)$, trong đó $(H_1)$ chứa điểm $S$, $(H_2)$ chứa điểm $A$; $V_1$ và $V_2$ lần lượt là thể tích của $(H_1)$ và $(H_2)$. Tính tỉ số $\dfrac{V_1}{V_2}$. 
	\choice
	{\True $\dfrac{4}{5}$}
	{$\dfrac{5}{4}$}
	{$\dfrac{3}{4}$}
	{$\dfrac{4}{3}$}
\end{ex}

\begin{ex}%[2H1K3-3]%Câu 27.
	Cho hình chóp $S.ABCD$ có đáy là hình vuông cạnh $a$, $SA$ vuông góc với mặt phẳng đáy $(ABCD)$ và $SA=a$. Điểm $M$ thuộc cạnh $SA$ sao cho $\dfrac{SM}{SA}=k$. Xác định $k$ sao cho mặt phẳng $(BMC)$ chia khối chóp $S.ABCD$ thành hai phần có thể tích bằng nhau. 
	\choice
	{$k=\dfrac{-1+\sqrt{3}}{2}$}
	{\True $k=\dfrac{-1+\sqrt{5}}{2}$}
	{$k=\dfrac{-1+\sqrt{2}}{2}$}
	{$k=\dfrac{1+\sqrt{5}}{4}$}
\end{ex}

\begin{ex}%[2H1K3-3]%Câu 28.
	Cho hình chóp tam giác $S.ABC$ có $M$ là trung điểm của $SB$, $N$ là điểm trên cạnh $SC$ sao cho $NS=2NC$, $P$ là điểm trên cạnh $SA$ sao cho $PA=2PS$. Kí hiệu $V_1,V_2$ lần lượt là thể tích của các khối tứ diện $BMNP$ và $SABC$. Tính tỉ số $\dfrac{V_1}{V_2}$. 
	\choice
	{\True $\dfrac{V_1}{V_2}=\dfrac{1}{9}$}
	{$\dfrac{V_1}{V_2}=\dfrac{3}{4}$}
	{$\dfrac{V_1}{V_2}=\dfrac{2}{3}$}
	{$\dfrac{V_1}{V_2}=\dfrac{1}{3}$}
\end{ex}

\begin{ex}%[2H1K3-3]%Câu 30.
	Cho hình chóp $S.ABCD$ có đáy $ABCD$ là hình vuông cạnh $a$, hai mặt bên $(SAB),(SAD)$ cùng vuông góc với mặt đáy. Biết góc giữa hai mặt phẳng $(SCD)$ và $(ABCD)$ bằng $45^\circ$. Gọi $V_1,V_2$ lần lượt là thể tích khối chóp $S.AHK$ và $S.ACD$ với $H,K$ lần lượt là trung điểm $SC,SD$. Tính độ dài đường cao $h$ của khối chóp $S.ABCD$ và tỉ số $k=\dfrac{V_1}{V_2}$. 
	\choice
	{\True $h=a;k=\dfrac{1}{4}$}
	{$h=a;k=\dfrac{1}{6}$}
	{$h=2a;k=\dfrac{1}{8}$}
	{$h=2a;k=\dfrac{1}{3}$}
\end{ex}


\begin{dang}{Ôn tập khối chóp và lăng trụ}
\end{dang}
\begin{vd}%[2H1B3-2]%Ví dụ 1.
	Cho hình chóp $SABCD$ có đáy $ABCD$ là hình vuông vuông cạnh $2a$, $SA$ vuông góc với đáy. Góc giữa $SC$ và đáy bằng $60^{^{\circ}}$ và $M$ là trung điểm của $SB$.
	\begin{enumerate}[a)]
		\item Tính thể tích khối chóp $SABCD$.
		\item Tính thể tích khối chóp $MBCD$.
	\end{enumerate}
	\loigiai{
		\begin{center}
			\begin{tikzpicture}[scale=0.7]
			\tkzDefPoints{0/0/A, -2/-2/B, 2.5/-2/C}
			\coordinate (D) at ($(A)+(C)-(B)$);
			\coordinate (S) at ($(A)+(0,3)$);
			\draw (B)--(C)--(D);

			\coordinate (O) at ($(A)!0.5!(C)$);
			\coordinate (M) at ($(S)!0.5!(B)$);
			\coordinate (H) at ($(A)!0.5!(B)$);
			\draw[dashed] (A)--(C) (B)--(D);
			\draw (S)--(B) (S)--(C) (S)--(D);
			\draw[dashed] (S)--(A) (B)--(A)--(D) (M)--(H);
			\tkzMarkRightAngle(S,A,D)
			\tkzMarkRightAngle(S,A,B)
			\tkzMarkRightAngle(A,O,D)
			\tkzMarkRightAngle(A,B,C)
			\draw (S) node[above]{$S$};
			\draw (A) node[left]{$A$};
			\draw (B) node[left]{$B$};
			\draw (C) node[right]{$C$};
			\draw (D) node[right]{$D$};
			\draw (O) node[below]{$O$};
			\tkzLabelPoints[above left](M)
			\tkzLabelPoints[below](H)
			\end{tikzpicture}
		\end{center}
		\begin{enumerate}[a)]
		\item Ta có: $V_{SABCD}=\dfrac{1}{3}S_{ABCD}SA$ và $S_{ABCD}=(2a)^2=4a^2,\Delta SAC$ có.\\
		$SA=AC\cdot\tan C=2a\sqrt{6}\Rightarrow V_{SABCD}=\dfrac{1}{3}4a^2\cdot 2a\sqrt{6}=\dfrac{8a^3\sqrt{6}}{3}$.\\
		\item Kẻ $MH\parallel SA\Rightarrow MH\perp (DBC)$. Ta có $MH=\dfrac{1}{2}SA,S_{BCD}=\dfrac{1}{2}S_{ABCD}\Rightarrow V_{MBCD}=\dfrac{1}{4}V_{S,ABCD}=\dfrac{2a^3\sqrt{6}}{3}$.
	\end{enumerate}}
\end{vd}

\begin{vd}%[2H1K3-2]%Ví dụ 2.
	Cho hình chóp tam giác $S_{ABC}$ có $AB=5a,BC=6a,CA=7a$. Các mặt bên $SAB,SBC,SCA$ tạo với đáy một góc $60^{^{\circ}}$. Tính thể tích khối chóp.
	\loigiai{
		\immini{Hạ $SH\perp (ABC)$, Kẻ $HE\perp B,HF\perp BC,HJ\perp AC$.\\
		suy ra $SE\perp AB,SF\perp BC,SJ\perp AC$.\\
		Ta có $\widehat{SEH}=\widehat{SFH}=\widehat{SJH}=60^{^{\circ}}\Rightarrow\Delta SEH=\Delta SFH=\Delta SJH$.\\
		Nên ta có $HE=HF=HJ=r$ ($r$ là bán kính đường tròn nội tiếp $\Delta ABC$).\\
		Ta có $S_{ABC}=\sqrt{p(p-a)(p-b)(p-c)}$ với $p=\dfrac{a+b+c}{2}=91$.\\
		Nên $S_{ABC}=\sqrt{9\cdot 4\cdot 3\cdot 2}a^2$. \\Mặt khác $S_{ABC}=p\cdot r\Rightarrow r=\dfrac{S}{p}=\dfrac{2\sqrt{6}a}{3}$.\\
		Tam giác $SHE$ vuông có $SH=r\cdot\tan 60^{^{\circ}}=\dfrac{2\sqrt{6}a}{3}\cdot\sqrt{3}=2\sqrt{2}a$.\\
		Vậy $V_{SABC}=\dfrac{1}{3}6\sqrt{6}a^2\cdot 2\sqrt{2}a=8\sqrt{3}a^3$.}
		{\begin{tikzpicture}[scale=1.0]
		\tkzInit[ymin=-3.0,ymax=3.5,xmin=-0.5,xmax=5.5]
		\tkzClip
		
		\tkzDefPoints{0/0/A,5/0/C,1.7/-2.5/B}
		\tkzCentroid(A,B,C) \tkzGetPoint{H}
		\coordinate[label=above:{$S$}] (S) at ($(H)+(0,3.5)$);
		\coordinate (E) at ($(A)!.5!(B)$);
		\coordinate (F) at ($(C)!.5!(B)$);
		\coordinate (J) at ($(A)!.5!(C)$);
		\tkzDrawSegments(A,B B,C A,S S,C S,B S,E)
		\tkzDrawSegments[dashed](A,C S,H H,E H,F H,J)
		
		\tkzMarkAngles[arc=l,size=0.4cm](H,E,S)
		\tkzLabelPoints[below](B,H,E,F)
		\tkzLabelPoints[left](A)
		\tkzLabelPoints[right](C)
		\tkzLabelPoints[above](J)
		
		\foreach \X in {S,A,B,C,H,E,F,J}
		{\draw[fill=black] (\X) circle (1.0pt);}
		
		\end{tikzpicture}}
	}
\end{vd}

\begin{vd}%[2H1K3-2]%Ví dụ 3.
	Cho hình hộp chữ nhật $ABCD.A'B'C'D'$ có $AB=a\sqrt{3},AD=a,AA'=a$, $O$ là giao điểm của $AC$ và $BD$.
	\begin{enumEX}[a)]{1}
	\item Tính thể tích khối hộp chữ nhật, khối chóp $OA'B'C'D'$.
	\item Tính thể tích khối $OBB'C'$.
	\item Tính độ dài đường cao đỉnh $C'$ của tứ diện $OBB'C'$.
	\end{enumEX}
	\loigiai{
		\begin{center}
			\begin{tikzpicture}[scale=0.7]
			\tkzDefPoints{0/0/A, -2/-2/B, 4/-2/C}
			\coordinate (D) at ($(A)+(C)-(B)$);
			\coordinate (A') at ($(A)+(0,5)$);
			\draw[dashed] (B)--(A)--(D);
			\draw (B)--(C)--(D);
			\tkzDefPointsBy[translation= from A to A'](B,C,D){}
			\draw (A')--(B')--(C')--(D')--(A');
			\draw[dashed] (A)--(A');
			\draw (B)--(B') (C)--(C') (D)--(D');
			\draw[dashed] (B)--(A')--(D)--(B) (A)--(C);
			\coordinate (M) at ($(C)!0.5!(B)$);
			\coordinate (O) at ($(A)!0.5!(C)$);
%			\draw (B)--(M);
			\draw[dashed] (M)--(O)--(A') (B')--(O)--(C') (D')--(O);
			\begin{scriptsize}
			\tkzLabelPoints[below](O,M)
			\tkzLabelPoints[left](A,A',B,B')
			\tkzLabelPoints[right](C,C',D,D')
			\end{scriptsize}
			\end{tikzpicture}
		\end{center}
		\begin{enumEX}[a)]{1}
		\item Gọi thể tích khối hộp chữ nhật là $V$. Ta có $V=ABAD\cdot AA'=a\sqrt{3}\cdot a^2=a^3\sqrt{3}$.\\
		$\Delta ABD$ có $DB=\sqrt{AB^2+AD^2}=2a$.\\
		Khối $OA'B'C'D'$ có đáy và đường cao giống khối hộp nên $V_{OAB'CD'}=\dfrac{1}{3}V=\dfrac{a^3\sqrt{3}}{3}$.
		\item $M$ là trung điểm của $BC$ \\
		$ \Rightarrow OM\perp(BB'C')\Rightarrow V_{OBBC'}=\dfrac{1}{3}S_{EB'C'}OM=\dfrac{1}{3}\cdot\dfrac{a^2}{2}\cdot\dfrac{a\sqrt{3}}{2}=\dfrac{a^3\sqrt{3}}{12} $.
		\item Gọi $C'H$ là đường cao đỉnh $C'$ của tứ diện $OBB'C'$. Ta có $C'H=\dfrac{3V_{OBB'C'}}{S_{OBE}}$.\\
		$\Delta ABD$ có $DB=\sqrt{AB^2+AD^2}=2a\Rightarrow S_{OB'B}=\dfrac{1}{2}a^2\Rightarrow C'H=2a\sqrt{3}$.
	\end{enumEX}	
		}
\end{vd}

\begin{vd}%[2H1B3-2]%Ví dụ 4.
	Cho hình lập phương $ABCD.A'B'C'D'$ có cạnh bằng $a$. Tính thể tích khối tứ diện $ACB'D'$.
	\loigiai{
		\immini{Hình lập phương được chia thành khối $ACB'D'$ và bốn khối $CB'D'C'$, $BB'AC$, $D'ACD$, $AB'A'D'$.\\
		Các khối $CB'D'C'$, $BB'AC$, $D'ACD$, $AB'A'D'$ có diện tích đáy và chiều cao bằng nhau nên có cùng thể tích.\\
		Khối $CB'D'C'$ có thể tích $V_1=\dfrac{1}{3}\cdot\dfrac{1}{2}a^2\cdot a=\dfrac{1}{6}a^3$.\\
		Khối lập phương có thể tích $V=a^3$. \\$\Rightarrow V_{ACBD'}=a^3-4\cdot\dfrac{1}{6}a^3=\dfrac{1}{3}a^3$.}{
	\begin{tikzpicture}[scale=0.7]
	\tkzDefPoints{0/0/A, -2/-2/B, 4/-2/C}
	\coordinate (D) at ($(A)+(C)-(B)$);
	\coordinate (A') at ($(A)+(0,5)$);
	\draw[dashed] (B)--(A)--(D);
	\draw (B)--(C)--(D);
	\tkzDefPointsBy[translation= from A to A'](B,C,D){}
	\draw (A')--(B')--(C')--(D')--(A');
	\draw[dashed] (D')--(A)--(A') (A)--(B');
	\draw (B)--(B') (C)--(C') (D)--(D');
	\draw[dashed] (A)--(C);
	\draw (B')--(D')--(C)--(B');
	\begin{scriptsize}
	\tkzLabelPoints[left](A,A',B,B')
	\tkzLabelPoints[right](C,C',D,D')
	\end{scriptsize}
	\end{tikzpicture}}
	}
\end{vd}

\begin{vd}%[2H1K3-2]%Ví dụ 5.
	Cho lăng trụ đứng tam giác $ABC.A'B'C'$ có các cạnh bằng $a$.
	\begin{enumEX}[a)]{1}
	\item Tính thể tích khối tứ diện $A'B'BC$.
	\item Gọi $E$ là trung điểm cạnh $AC$, mặt phẳng $(A'B'E)$ cắt $BC$ tại $F$. Tính thể tích khối $CA'B'FE$.
	\end{enumEX}
	\loigiai{
		\immini{
		\begin{enumEX}[a)]{1}
		\item Khối $A'B'BC$. Gọi $I$ là trung điểm $AB$.\\
		$V_{ABBC}=\dfrac{1}{3}S_{A'B'B}\cdot CI=\dfrac{1}{3}\dfrac{a^2}{2}\cdot\dfrac{a\sqrt{3}}{2}=\dfrac{a^3\sqrt{3}}{12}$.
		\item Khối $CA'B'FE$ phân ra hai khối $CEFA'$ và $CFA'B'$.\\
		Khối $A'CEF$ có đáy $CEF$, đường cao $AA'$ nên $V_{ACBF}=\dfrac{1}{3}S_{CEF}\cdot A'A$.\\
		$S_{CEF}=\dfrac{1}{4}S_{ABC}=\dfrac{a^2\sqrt{3}}{16}\Rightarrow V_{ACEF}=\dfrac{a^3\sqrt{3}}{48}$.\\
		Gọi $J$ là trung điểm $B'C'$.\\ Ta có khối $A'B'CF$ có đáy là $CFB'$, đường cao $JA$.\\
		Nên $V_{ABCF}=\dfrac{1}{3}S_{CFB}\cdot A'J;S_{CFB'}=\dfrac{1}{2}S_{CBB'}=\dfrac{a^2}{4}$ \\
		$ \Rightarrow V_{ABCF}=\dfrac{1}{3}\cdot\dfrac{a^2}{4}\cdot\dfrac{a\sqrt{3}}{2}=\dfrac{a^3\sqrt{3}}{24} $.\\
		Vậy $V_{CABPR}=\dfrac{a^3\sqrt{3}}{16}$.
		\end{enumEX}
		}{
		\begin{tikzpicture}[scale=.7]
	\tkzDefPoints{0/5/A, 2/3/B, 5/5/C, 0/-5/z}
	\coordinate (A') at ($(A)+(z)$);
	\coordinate (B') at ($(B)+(z)$);
	\coordinate (C') at ($(C)+(z)$);
	\coordinate (E) at ($(A)!.5!(C)$);
	\coordinate (F) at ($(B)!.5!(C)$);
	\coordinate (I) at ($(A)!.5!(B)$);
	\tkzLabelPoints [right](C,C')
	\tkzLabelPoints[above](E)
	\tkzLabelPoints[below](B',I)
	\tkzDrawPoints(A,B,C,A',B',C',I,E,F)
	\tkzDrawPolygon(A,B,C)
	\tkzDrawSegments(A',A A',B' B',C' B,B' C,C' A,C C,I E,F B',F)
	\tkzDrawSegments[dashed](A',C' A',E)
	\tkzLabelPoints[below right](F,B)
	\tkzLabelPoints[left](A,A')
	\end{tikzpicture}}
	}
\end{vd}

\subsubsection{Câu hỏi trắc nghiệm}
\begin{ex}%[2H1K3-2]%Câu 1.
	Cho hình chóp $S.ABCD$ có đáy $ABCD$ là hình thoi, $AC=4,BD=2$. Mặt chéo $SBD$ nằm trong mặt phẳng vuông góc với mặt phẳng $(ABCD)$ và $SB=\sqrt{3},SD=1$. Thể tích của khối chóp $S.ABCD$ là
	\choice
	{\True $V=\dfrac{2\sqrt{3}}{3}$}
	{$V=2\sqrt{3}$}
	{$V=\dfrac{8\sqrt{3}}{3}$}
	{$V=\dfrac{4\sqrt{3}}{3}$}
\end{ex}

\begin{ex}%[2H1B3-2]%Câu 2.
	Cho hình hộp $ABCD.A'B'C'D'$ có thể tích $V$. Tính thể tích khối tứ diện $ACB'D'$.
	\choice
	{\True $\dfrac{V}{3}$}
	{$\dfrac{V}{4}$}
	{$\dfrac{V}{6}$}
	{$\dfrac{V}{5}$}
\end{ex}

\begin{ex}%[2H1B3-2]%Câu 3.
	Cho khối chóp $S.ABC$ có $SA=3$; $SB=4$; $SC=5$; $\widehat{ASB}=\widehat{BSC}=\widehat{CSA}=60^{^{\circ}}$. Thể tích khối chóp $S.ABC$ bằng 
	\choice
	{\True $5\sqrt{2}$}
	{$5\sqrt{3}$}
	{$10$}
	{$15$}
\end{ex}

\begin{ex}%[2H1K3-2]%Câu 4.
	Cho khối chóp $S.ABC$ có đường cao $SA=2a$, tam giác $ABC$ vuông ở $C$ có $AB=2a$, góc $\widehat{CAB}=30^{^{\circ}}$. Gọi $H$ là hình chiếu của $A$ trên $SC$. Gọi $B'$ là điểm đối xứng của $B$ qua mặt phẳng $(SAC)$. Tính thể tích khối chóp $H.AB'B$. 
	\choice
	{\True $\dfrac{2a^3\sqrt{3}}{7}$}
	{$\dfrac{2a^3\sqrt{3}}{7}$}
	{$\dfrac{6a^3\sqrt{3}}{7}$}
	{$\dfrac{a^3\sqrt{3}}{7}$}
\end{ex}

\begin{ex}%[2H1K3-2]%Câu 5.
	Cho hình chóp $S.ABCD$ có đáy $ABCD$ là hình thoi cạnh $a$. Tam giác $ABC$ đều. Hình chiếu vuông góc $H$ của đỉnh $S$ trên mặt phẳng $(ABCD)$ trùng với trọng tâm của tam giác $ABC$. Đường thẳng $SD$ hợp với mặt phẳng $(ABCD)$ góc $30^{^{\circ}}$. Tính thể tích khối chóp $SABCD$ theo $a$. 
	\choice
	{$V=\dfrac{a^8\sqrt{3}}{3}$}
	{$V=\dfrac{a^3}{3}$}
	{\True $V=\dfrac{a^3\sqrt{3}}{9}$}
	{$V=\dfrac{2a^3\sqrt{3}}{9}$}
\end{ex}

\begin{ex}%[2H1K3-2]%Câu 6.
	Cho hình chóp tứ giác $S.ABCD$ có đáy $ABCD$ là hình thang vuông tại $A$ và $B$, $BA=BC=1,AD=2$. Cạnh bên $SA$ vuông góc với đáy và $SA=\sqrt{2}$. Gọi $H$ là hình chiếu vuông góc của $A$ lên $SB$. Tính thể tích khối đa diện $SAHCD$ 
	\choice
	{$V=\dfrac{2\sqrt{2}}{3}$}
	{\True $V=\dfrac{4\sqrt{2}}{9}$}
	{$V=\dfrac{4\sqrt{2}}{3}$}
	{$V=\dfrac{2\sqrt{2}}{9}$}
\end{ex}

\begin{ex}%[2H1K3-2]%Câu 7.
	Cho hình hộp $ABCD.A'B'C'D'$ có đáy $ABCD$ là hình vuông cạnh $Q$. Cạnh bên $AA'=a$ hình chiếu vuông góc của $A'$ trên mặt phẳng $(ABCD)$ trùng với trung điểm $H$ của $AB$. Tính theo $a$ thể tích của khối lăng trụ đã cho. 
	\choice
	{$V=\dfrac{a^8\sqrt{3}}{6}$}
	{\True $V=\dfrac{a^3\sqrt{3}}{2}$}
	{$V=a^3$}
	{$V=\dfrac{a^3}{3}$}
\end{ex}

\begin{ex}%[2H1K3-2]%Câu 8.
	Cho hình chóp $S.ABCD$ có đáy $ABCD$ là hình vuông tâm $O$. $SA$ vuông góc với $(ABCD)$, $AB=a$, $SA=a\sqrt{2}$. Gọi $H,K$ lần lượt là hình chiếu vuông góc của $A$ trên $SB$, $SD$. Tính thể tích khối chóp $OHAK$ theo $a$. 
	\choice
	{$V=\dfrac{\sqrt{3}a^8}{27}$}
	{\True $V=\dfrac{\sqrt{2}a^8}{27}$}
	{$V=\dfrac{\sqrt{2}a^8}{13}$}
	{$V=\dfrac{\sqrt{3}a^8}{13}$}
\end{ex}

\begin{ex}%[2H1K3-3]%Câu 9.
	Cho hình chóp $S.ABC$, gọi $G$ là trọng tâm tam giác $SBC$. Mặt phẳng quay quanh $AG$ cắt các cạnh $SB$, $SC$ theo thứ tự tại $M,N$. Gọi $V_1$ là thể tích tứ diện $SAMN$; $V$ là thể tích tứ diện $SABC$. Tìm giá trị lớn nhất và giá trị nhỏ nhất của tỉ số $\dfrac{V_1}{V}$.
	\choice
	{$\min\dfrac{V_1}{V}=\dfrac{4}{7};\max\dfrac{V_1}{V}=\dfrac{1}{3}$}
	{$\min\dfrac{V_1}{V}=\dfrac{4}{9};\max\dfrac{V_1}{V}=\dfrac{1}{3}$}
	{$\min\dfrac{V_1}{V}=\dfrac{4}{7};\max\dfrac{V_1}{V}=\dfrac{1}{2}$}
	{\True $\min\dfrac{V_1}{V}=\dfrac{4}{9};\max\dfrac{V_1}{V}=\dfrac{1}{2}$}
\end{ex}

\begin{ex}%[2H1K3-6]%Câu 10.
	Cho hình chóp $S.ABC$ đáy $ABC$ là tam giác vuông tại $A$ và $AB=SA=SB=SC=a$. Ta có $SA,SB,SC$ cùng tạo với đáy một góc $\alpha$. Xác định $\cos\alpha$ để thể tích hình chóp lớn nhất. 
	\choice
	{$\cos\alpha=\dfrac{5}{2\sqrt{3}}$}
	{\True $\cos\alpha=\dfrac{5}{2\sqrt{2}}$}
	{$\cos\alpha=\dfrac{7}{2\sqrt{2}}$}
	{$\cos\alpha=\dfrac{7}{2\sqrt{3}}$}
\end{ex}

\begin{ex}%[2H1T3-5]%Câu 11.
	Một hộp không nắp được làm từ một tấm bìa các tông. Hộp có đáy là một hình vuông cạnh $x$ cm, đường cao là $h$ cm và có thể tích là $500\text{ cm}^3$. Tìm $x$ sao cho diện tích mảnh bìa các tông là nhỏ nhất.
	\begin{center}
		\begin{tikzpicture}[thick, scale=.6]
		\def\x{4}\def\h{1}
		\draw[thick] ({(-\x/2-\h)},{\x/2}) rectangle ({(\x/2+\h)},{-\x/2});
		\draw[thick,rotate=90] ({(-\x/2-\h)},{\x/2}) rectangle ({(\x/2+\h)},{-\x/2});
		\draw[stealth-stealth,red] ({\x/2+\h+.5},{\x/2})--node[midway,right,black]{$x$ cm} ({\x/2+\h+.5},{-\x/2});
		\draw[stealth-stealth,red] ({\x/2+\h+.5},{\x/2+\h})--node[midway,right,black]{$h$ cm} ({\x/2+\h+.5},{\x/2});
		\draw[stealth-stealth,red] ({\x/2},{\x/2+\h})--node[midway,above,black]{$h$ cm}({\x/2+\h},{\x/2+\h});
		\draw[->,line width=.6mm] ({\x/2+\h+2},0)--++(2,0);
		\begin{scope}[shift={(\x/2+\h+5,-.3*\x)}]
		\draw (0,0)--++(\x,0)--++(45:.5*\x) (\x,0)--++(0,\h) ($ (\x,0)+(45:.5*\x) $)--++(0,\h)
		(0,\h)--++(\x,0)--++(45:.5*\x)--++(-\x,0)--++(-135:.5*\x)--(0,0)--++(0,\h);
		\draw[dashed] (0,0)--++(45:.5*\x)--++(\x,0) (45:.5*\x)--++(0,\h);
		\end{scope}
		\end{tikzpicture}
	\end{center}
	\choice
	{$5\text{ cm}$}
	{\True $10\text{ cm}$}
	{$15\text{ cm}$}
	{$20\text{ cm}$}
\end{ex}

\begin{ex}%[2H1T3-5]%Câu 12.
	Trong một đợt tổ chức cho học sinh tham gia dã ngoại ngoài trời. Để có thể có chỗ nghỉ ngơi trong quá trình tham quan dã ngoại, các bạn học sinh đã dựng trên mặt đất bằng phẳng 1 chiếc lều bằng bạt từ một tấm bạt hình chữ nhật có chiều dài là $12$ m và chiều rộng là $6$ m bằng cách: Gập đôi tấm bạt lại theo đoạn nối trung điểm hai cạnh là chiều rộng của tấm bạt sao cho hai mép chiều dài còn lại của tấm bạt sát đất và cách nhau $x$ m (xem hình vẽ). Tìm $x$ để khoảng không gian phía trong lều là lớn nhất?\\
	\begin{center}
		\begin{tikzpicture}[thick,line join=round,scale=.6]	
		\draw (0,0)coordinate (A)--++(90:4)coordinate (B)node[pos=.6,left]{$6$m}--++(0:8)coordinate (C)node[pos=.5, above]{$12$m}--++(-90:4)coordinate (D)--++(180:8)coordinate (E);
		\draw[dashed] (0,2)coordinate--(8,2)coordinate;
		\end{tikzpicture}
		\begin{tikzpicture}[thick,line join=round,scale=.6]	
		\draw (0,0)coordinate (A)--++(50:2)coordinate (B)node[pos=.6,left]{$3$m}--++(-50:2)coordinate (C)node[pos=.65,right]{$3$m}--++(20:7)--++(130:2)--++(-160:3.5)coordinate (D)node[pos=1,above]{$12$m}--++(-160:3.5)
		($(A)!(B)!(C)$)coordinate (B')--($(B)+(0,1)$)coordinate (E) (A)--(C)node[pos=.4,below]{$x$};
		\draw[fill=red!50] (E)--++(90:.8)--++(-30:.8)--cycle;
		\draw[fill=magenta!50] ($(D)+(20:3.5)$)--++(90:.8)--++(-30:.8)--cycle;
		\draw[dashed](A)--++(20:7)coordinate (A')--++(50:2) (A')--($(C)+(20:7)$) ($(B')+(20:7)$)--($(B)+(20:7)$);
		\draw[fill=red](B)circle (1.5pt) (D)circle (1.5pt);
		\end{tikzpicture}
	\end{center}
	\choice
	{$x=2$}
	{$x=4$}
	{\True $x=3\sqrt{2}$}
	{$x=3\sqrt{3}$}
\end{ex}
\Closesolutionfile{ans}