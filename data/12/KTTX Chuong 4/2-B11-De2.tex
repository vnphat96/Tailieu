\begin{name}
	{NGUYÊN HÀM - TÍCH PHÂN}
	{KT NGUYÊN HÀM}
	{\tentruong}
	{\thoigian}
\end{name}
\setcounter{ex}{0}\setcounter{bt}{0}
\TN
\Opensolutionfile{ans}[ans/ans-2C4B11-De2]
\begin{ex}%[2D4N1-2]
	Họ nguyên hàm của hàm số $f(x)=3x^2+1$ là
	\choice
	{$x^3+C$}
	{$\dfrac{x^3}{3}+x+C$}
	{$6x+C$}
	{\True $x^3+x+C$}
	\loigiai{
	$\displaystyle\int(3x^2+1)\mathrm{d}x=x^3+x+C$.
	}
\end{ex} 
\begin{ex}%[2D4N1-2]
	Hàm số nào sau đây là một nguyên hàm của hàm số $y=12x^5$?
	\choice
	{$y=12x^4$}
	{$y=60x^4$}
	{$y=12x^6+5$}
	{\True $y=2x^6+3$}
	\loigiai{		
		Ta có $\displaystyle\int{12x^5\mathrm{d}\,x}=12\cdot\dfrac{x^6}{6}+C=2x^6+C$.}
\end{ex} 
\begin{ex}%[2D4N1-2]
	Tìm họ nguyên hàm $F(x)$ của hàm số $f(x)=\dfrac{1}{x}$.
	\choice
	{\True  $F(x)=\ln \left| x \right|+C$}
	{$F(x)=\ln x+C$}
	{$F(x)=\ln \left| x \right|$}
	{$F(x)=-\dfrac{1}{x^2}+C$}
	\loigiai{		
		Áp dụng công thức nguyên hàm của hàm số ta có $\displaystyle\int{\frac{1}{x}\mathrm{d}\,x}=\ln \left| x \right|+C$.}
\end{ex} 
\begin{ex}%[2D4N1-3]
	Mệnh đề nào \textbf{sai} trong các mệnh đề sau?
	\choice
	{ $\displaystyle\int\cos x\,\mathrm{d}x=\sin x+C$}
	{\True $\displaystyle\int \sin x \, \mathrm{d}x=\cos x+C$}
	{$\displaystyle\int{\dfrac{1}{\cos^2x}\, \mathrm{d}x=\tan x+C}$}
	{$\displaystyle\int{\dfrac{1}{\sin^2x}\, \mathrm{d}x=-\cot x+C}$}
	\loigiai{
		
		Từ bảng nguyên hàm của các hàm cơ bản suy ra $\displaystyle\int \sin x \, \mathrm{d}x=\cos x+C$ sai}
\end{ex} 
\begin{ex}%[2D4N1-4]
	Tìm nguyên hàm của hàm số $f(x)=7^x$.
	\choice
	{$\displaystyle\int 7^x\mathrm{d}\,x=\dfrac{7^{x+1}}{x+1}+C$}
	{$\displaystyle\int 7^x\mathrm{d}\,x=7^x\ln 7+C$}
	{\True $\displaystyle\int 7^x \mathrm{d}\,x=\dfrac{7^x}{\ln 7}+C$}
	{$\displaystyle\int 7^x\mathrm{d}\,x=7^{x+1}+C$}
	\loigiai{		
		Áp dụng công thức nguyên $\displaystyle\int a^x\mathrm{d}\,x=\dfrac{a^x}{\ln a}+C \Rightarrow \displaystyle  \int 7^x\mathrm{d}\,x=\dfrac{7^x}{\ln 7}+C$.}
\end{ex} 
\begin{ex}%[2D4N1-4]
	Nguyên hàm của hàm số $F(x)=2^x+x$ là
	\choice
	{ $2^x+\dfrac{x^2}{2}+C$}
	{$2^x+x^2+C$}
	{$\dfrac{2^x}{\ln 2}+x^2+C$}
	{\True $\dfrac{2^x}{\ln 2}+\dfrac{x^2}{2}+C$}
	\loigiai{		
		Ta có $\displaystyle\int (2^x+x)\,\mathrm{d}\,x=\dfrac{2^x}{\ln 2}+\dfrac{1}{2} x^2+C$.}
\end{ex} 
\begin{ex}%[2D4N1-4]
	$\displaystyle\int (3^x+4^x)\mathrm{d}\,x$ bằng
	\choice
	{\True  $\dfrac{3^x}{\ln 3}+\dfrac{4^x}{\ln 4}+C$}
	{$\dfrac{3^x}{\ln 4}+\dfrac{4^x}{\ln 3}+C$}
	{$\dfrac{4^x}{\ln 3}-\dfrac{3^x}{\ln 4}+C$}
	{$\dfrac{3^x}{\ln 3}-\dfrac{4^x}{\ln 4}+C$}
	\loigiai{
		Áp dụng công thức $\displaystyle\int a^x\,\mathrm{d}x=\frac{a^x}{\ln a}+C$.\\
		Ta có $\displaystyle\int(3^x+4^x)\mathrm{d}\,x
		=\int 3^x\mathrm{d}\,x+\int 4^x\mathrm{d}\,x=\dfrac{3^x}{\ln 3}+\dfrac{4^x}{\ln 4}+C$.}
\end{ex} 
\begin{ex}%[2D4H1-4]
	Họ nguyên hàm của hàm số $f(x)=\mathrm{e}^x+2x$ là
	\choice
	{ $\dfrac{1}{x+1}\mathrm{e}^x+x^2+C$}
	{$\mathrm{e}^x+2x^2+C$}
	{\True $\mathrm{e}^x+x^2+C$}
	{$\mathrm{e}^x+\dfrac{1}{2} x^2+C$}
	\loigiai{
		Ta có $\displaystyle\int(\mathrm{e}^x+2x)\mathrm{d}\,x=\int\mathrm{e}^x\mathrm{d}\,x+\int 2x\mathrm{d}\,x=\mathrm{e}^x+x^2+C$.}
\end{ex} 
\begin{ex}%[2D4H1-4]
	Trong các mệnh đề sau, mệnh đề nào \textbf{sai}?
	\choice
	{\True  $\displaystyle\int\sin x\mathrm{d}x=\cos x+C$}
	{$\displaystyle\int 2x\mathrm{d}x=x^2+C$}
	{$\displaystyle\int \mathrm{e}^x\mathrm{d}x=\mathrm{e}^x+C$}
	{$\displaystyle\int \dfrac{1}{x}\mathrm{d}x=\ln \left| x \right|+C$}
	\loigiai{
		$\displaystyle\int{\sin x\mathrm{d}x}=-\cos x+C$.}
\end{ex} 
\begin{ex}%[2D4N1-1]
	Khẳng định nào sau đây là \textbf{sai}?
	\choice
	{ Mọi hàm số $f(x)$ liên tục trên đoạn $[a;b]$ đều có nguyên hàm trên đoạn $[a;b]$}
	{\True $\displaystyle\int x^\alpha \mathrm{d}x=\dfrac{x^{\alpha +1}}{\alpha +1}+C$ ($C$ là hằng số, $\alpha $ là hằng số)}
	{$\displaystyle\int \mathrm{e}^x\mathrm{d}x=\mathrm{e}^x+C$ ($C$ là hằng số)}
	{$\displaystyle\int{\dfrac{1}{x}\mathrm{d}x=\ln \left| x \right|+C}$ ($C$ là hằng số) với $x\ne 0$}
	\loigiai{		
		$\displaystyle\int x^{\alpha} \mathrm{d}\,x=\dfrac{x^{\alpha +1}}{\alpha +1}+C$ ($C$ là hằng số, $\alpha $ là hằng số và $\alpha \ne -1$).}
\end{ex} 
\begin{ex}%[2D4N1-4]
	Hàm số nào dưới đây là một nguyên hàm của hàm số $f(x)=\sqrt{x}-1$ trên $(0;+\infty)$?
	\choice
	{$F(x)=\dfrac{1}{2\sqrt{x}}$}
	{$F(x)=\dfrac{1}{2\sqrt{x}}-x$}
	{$F(x)=\dfrac{2}{3}\sqrt[3]{x^2}-x+1$}
	{\True $F(x)=\dfrac{2}{3}\sqrt{x^3}-x+2$}
	\loigiai{		
		Ta có : $\displaystyle\int (\sqrt{x}-1)\mathrm{d}x=\frac{2}{3}\sqrt{x^3}-x+C$.}
\end{ex} 
\begin{ex}%[2D4V2-6]
	Một vật chuyển động với gia tốc $a(t)=\dfrac{3}{t+1}$ (m/s$^2$), trong đó $t$ là khoảng thời gian tính từ thời điểm ban đầu. Vận tốc ban đầu của vật là $6$(m/s). Hỏi vận tốc của vật tại giây thứ $8$ là bao nhiêu?
	\choice
	{\True  $12{,}6$ (m/s)}
	{$12{,}2$ (m/s)}
	{$6{,}6$ (m/s)}
	{$12{,}4$ (m/s)}
	\loigiai{
		Vận tốc của vật tại thời điểm t được tính theo công thức
		$$v(t)=\int a(t)\mathrm{d}t=\displaystyle \int \dfrac{3}{t+1}\mathrm{d} t=3\ln (t+1)+C.$$
		Do vận tốc ban đầu của vật bằng $6$ (m/s) nên ta có:
		$$v(0)=3\ln (0+1)+C=6\Rightarrow C=6\Rightarrow v(t)=3\ln (t+1)+6.$$
		Vận tốc chuyển động của vật tại giây thứ $8$ là:
		$$v(8)=3\ln (8+1)+6=3\ln 9+6\approx 12{,}6 \text{ (m/s)}.$$
}
\end{ex} 

\Closesolutionfile{ans}
% \indapan{6}{ans/ans-2C4B11-De2}
\Opensolutionfile{ans}[ans/ans-2C4B11-De2-ds]
\TNTF
\setcounter{ex}{0}
\begin{ex}%[2D4H1-3]
	Cho hàm số $f(x)=\sin \dfrac{x}{2}$ và hàm số $g(x)=\cos \dfrac{x}{2}$ .
	\choiceTF
	{$F(x)=2\cos \dfrac{x}{2}$  là một nguyên hàm của hàm số $f(x)$}
	{\True $G(x)=2\sin \dfrac{x}{2}+\dfrac{1}{2}$  là một nguyên hàm của hàm số $g(x)$}
	{$\displaystyle\int \left[ f(x)-g(x) \right]^2 \mathrm{d}x=x+\cos x+C$ ($C$ là một hằng số)}
	{\True $\displaystyle\int \dfrac{1}{[2f(x)\cdot g(x)]^2}\mathrm{d}x=-\cot x+C$ ($C$ là một hằng số)}
\loigiai{
	\begin{itemchoice}
	\itemch Vì $F'(x)=-\sin \dfrac{x}{2},\forall x\in R$ nên  $F(x)=2\cos \dfrac{x}{2}$  không là một nguyên hàm của hàm số $F(x)$ trên $\mathbb{R}$.  Sai
	\itemch Vì $G'(x)=\cos \dfrac{x}{2}, \forall x\in \mathbb{R}$ nên  $G(x)=2\sin \dfrac{x}{2}+\dfrac{1}{2}$ là một nguyên hàm của hàm số $g(x)$ trên $R$.  Đúng
	\itemch $\displaystyle\int [f(x)-g(x)]^2\mathrm{d}x=\int \left( \sin\dfrac{x}{2}-\cos\frac{x}{2} \right)^2\mathrm{d}x=\int \left(\sin ^2\frac{x}{2}+2\sin\frac{x}{2}\cos\frac{x}{2}+\cos^2\frac{x}{2}\right)\mathrm{d}x=\int( 1+\sin x )\mathrm{d}x=x-\cos x+C$.  Sai
	\itemch $\displaystyle\int \frac{1}{[2f(x)\cdot g(x)]^2}\mathrm{d}x=\int \frac{1}{(2\sin\frac{x}{2}\cos\frac{x}{2})^2}\mathrm{d}x=\int \frac{1}{\sin^2 x}\mathrm{d}x=-\cot x+C$.  Đúng
	\end{itemchoice}
	}
	\end{ex} 
	\begin{ex}%[2D4H1-4]
		Cho hàm số $f(x)=\dfrac{1}{x}$ và $F(x)=\ln x+C_1$, $G(x)=\ln (-x)+C_2$ ($C_1,C_2$ là các hằng số).
		\choiceTF
{\True Trên $(0;+\infty)$, một nguyên hàm của hàm số $f(x)$ là $H(x)=\ln (x)+e$}
{\True Trên $(-\infty ;0)$, nguyên hàm của hàm số $f(x)$ là $G(x)$}
{\True Trên $(0;+\infty)$, nguyên hàm của hàm số $f(x)$ là $F(x)$}
{\True $\displaystyle\int \left[ f(x)+f^2(x) \right]\mathrm{d}x=\ln (3\left| x \right|)-\dfrac{1}{x}+C$ ($C$ là một hằng số)}
\loigiai{
		\begin{itemchoice}
\itemch Vì $H'(x)=\dfrac{1}{x}=F(x),\forall x\in (0;+\infty)$ nên $H(x)$ là một nguyên hàm của hàm số $F(x)$ trên ($0,+\infty $).  Đúng
\itemch $\displaystyle\int f(x)\mathrm{d}x=\int \frac{1}{x}\mathrm{d}x=\ln \left( \left| x \right| \right)+C_2=\ln (-x)+C_2,\forall x\in (-\infty ;0)$.  Đúng
\itemch $\displaystyle\int f(x)\mathrm{d}x=\int \frac{1}{x}\mathrm{d}x=\ln \left( \left| x \right| \right)+C_1=\ln x+C_1,\forall x\in( 0;+\infty)$.  Đúng
\itemch $\displaystyle\int \left[ f(x)+f^2(x) \right]\mathrm{d}x=\int \left( \frac{1}{x}+\frac{1}{x^2} \right)\mathrm{d}x=\ln (\left| x \right|)-\frac{1}{x}+C_3=\ln \left( \left| x \right| \right)-\frac{1}{x}+\ln 3+C=\ln ( 3\left| x \right|)-\frac{1}{x}+C$.  Đúng 
	\end{itemchoice}}
\end{ex} 
\begin{ex}%[2D4V1-4]
	Cho hàm số $f(x)=\cos x$ và hàm số $g(x)=\sin x$.
	\choiceTF
{\True $F(x)=\sin x+\mathrm{e}$ là một nguyên hàm của hàm số $f(x)$ trên $\mathbb{R}$}
{$G(x)={\mathrm{e}^{-\cos x}}+\ln 3$ là một nguyên hàm của hàm số $\mathrm{e}^{g(x)}$ trên $\mathbb{R}$}
{\True $\displaystyle\int \left[ 5f(x)+6g(x) \right]\mathrm{d}x=5\sin x-6\cos x+C$, ($C$ là một hằng số)}
{\True $\displaystyle\int \left[ 2+\left( \frac{g(x)}{f(x)} \right)^2 \right]\mathrm{d}x=x+\tan x+C$, ($C$ là một hằng số)}
\loigiai{
		\begin{itemchoice}
\itemch Vì $F'(x)=\cos x=f(x),\forall x\in \mathbb{R}$ nên $(x)$ là một nguyên hàm của hàm số $f(x)$ trên $\mathbb{R}$.  Đúng
\itemch Vì $G'(x)=\sin x \mathrm{e}^{-\cos x}\ne \mathrm{e}^{\sin x}, \forall x\in \mathbb{R}$ nên $G(x)$ không là một nguyên hàm của hàm số $\mathrm{e}^{g(x)}$ trên $\mathbb{R}$. Sai
\itemch $\displaystyle\int \left[ 5f(x)+6g(x) \right]\mathrm{d}x=\int \left( 5\cos x+6\sin x \right)\mathrm{d}x=5\sin x-6\cos x+C$.  Đúng
\itemch $\displaystyle\int \left[ 2+\left( \frac{g(x)}{f(x)} \right)^2 \right]\mathrm{d}x=\int \left(2+\frac{\sin^2 x}{\cos^2 x}\right)\mathrm{d}x=\int \left( 1+\frac{\sin^2 x+\cos^2 x}{\cos^2 x} \right)\mathrm{d}x\\
=\int \left( 1+\frac{1}{\cos^2 x} \right)\mathrm{d}x=x+\tan x+C$.  Đúng
	\end{itemchoice} }
\end{ex} 
\begin{ex}%[2D4V1-4]
	Cho hàm số $f(x)=3^{2x}$ và hàm số $g(x)=\tan x$.
	\choiceTF
{$F(x)=\dfrac{3^{2x}\ln 3}{2}$  là một nguyên hàm của hàm số $f(x)$ trên $\mathbb{R}$}
{\True $G(x)=-\ln (3\cos x)$ là một nguyên hàm của hàm số $g(x)$ trên $\left( -\dfrac{\pi}{2};\dfrac{\pi}{2} \right)$}
{\True $\displaystyle\int 3f(x)\mathrm{d}x=\dfrac{3^{2x+1}}{\ln 9}+C$, ($C$ là một hằng số)}
{\True $\displaystyle\int [f(x)+g(x)^2]\mathrm{d}x =\dfrac{9^x}{2\ln 3}-x+\tan x+C$, ($C$ là một hằng số)}
\loigiai{
	\begin{itemchoice}
\itemch Vì $F'(x)=\dfrac{2\cdot 3^{2x}\cdot\ln ^23}{2}=3^{2x}\cdot \ln ^2 3\ne f(x),\forall x\in \mathbb{R}$ nên $f(x)$ không là một nguyên hàm của hàm số $F(x)$ trên $\mathbb{R}$.  Sai
\itemch Vì $G'(x)=-\dfrac{-3\sin x}{3\cos x}=\tan x=g(x),\forall x\in \left( -\dfrac{\pi }{2};\dfrac{\pi }{2} \right)$ nên $G(x)$ là một nguyên hàm của hàm số $g(x)$ trên $\left( -\dfrac{\pi }{2};\dfrac{\pi }{2} \right)$.  Đúng
\itemch $\displaystyle\int 3f(x)\mathrm{d}x=\int 3\cdot 3^{2x}\mathrm{d}x=3\cdot 9^x\mathrm{d}x=3\cdot\dfrac{9^x}{\ln 9}+C=\dfrac{3\cdot3^{2x}}{\ln 9}+C=\dfrac{3^{2x+1}}{\ln 9}+C$.  Đúng
\itemch $\displaystyle\int \left[f(x)+g(x)^2\right]\mathrm{d}x=\int \left( 3^{2x}+\tan^2x\right)\mathrm{d}x=\int \left(9^x-1+1+\tan ^2\right)\mathrm{d}x\\
=\int \left(9^x-1+\dfrac{1}{\cos^2 x}\right)\mathrm{d}x
=\dfrac{9^x}{\ln 9}-x+\tan x+C=\dfrac{9^x}{2\ln 3}-x+\tan x+C$.  Đúng
	\end{itemchoice}
}
\end{ex} 

\Closesolutionfile{ans}
% \indapan{2}{ans/ans-2C4B11-De2-ds}
\Opensolutionfile{ans}[ans/ans-2C4B11-De2-kq]
\TNSA
\setcounter{ex}{0}
\begin{ex}%[2D4V2-4]
	Giả sử hàm số $y=f(x)$ liên tục và thỏa mãn: $f(1)=1$ và $f'(x)\sqrt[3]{x^{-1}}=1$, với mọi $x>0$. Tính $4f(8)$.
	\shortans{$47$}
	\loigiai{
		Ta có $f'(x)=\dfrac{1}{\sqrt[3]{x^{-1}}}=\dfrac{1}{x^{-\tfrac{1}{3}}}=x^{\tfrac{1}{3}}$\\
		$\Rightarrow F(x)=\displaystyle\int f'(x)\mathrm{d}x= \int x^{\tfrac{1}{3}}\mathrm{d}x=\dfrac{3}{4}x^{\frac{4}{3}}+C=\frac{3}{4}\sqrt[3]{x^4}+C$.\\
		$f(1)=1\Rightarrow \dfrac{3}{4}+C=1\Rightarrow C=-\dfrac{1}{4}.\\
		\Rightarrow f(x)=\dfrac{3}{4}\sqrt[3]{x^4}-\dfrac{1}{4}$\\
		$\Rightarrow 4f(8)=47$.}
\end{ex} 
\begin{ex}%[2D4V2-6]
Một ô tô đang chạy với vận tốc $10$(m/s) thì người lái xe đạp phanh. Từ thời điểm đó, ô tô chuyển động chậm dần đều với vận tốc $v(t)=10-2t$ (m/s), trong đó $t$ là khoảng thời gian tính bằng giây kể từ lúc đạp phanh. Tính quãng đường ô tô di chuyển được trong $8$ giây cuối cùng.
\shortans{$55$}
\loigiai{
	Chọn mốc thời gian và gốc tọa độ lúc ô tô bắt đầu đạp phanh. Suy ra $t=0;\,s=0$.\\
	$s(t)=\displaystyle \int v(t)\mathrm{d}t=\int (10-2t)\mathrm{d}t=10t-t^2+C$.\\
	$s(0)=0\Rightarrow C=0 \Rightarrow s(t)=10t-t^2$.\\ 
	Ô tô dừng hẳn khi $v(t)=0\Leftrightarrow 10-2t=0\Leftrightarrow t=5$.\\
	Trong $8$ giây cuối:
	\begin{itemize}
		\item ô tô chuyển động đều với vận tốc $10$(m/s) trong $3$ giây đầu.
		\item ô tô chuyển động chậm dần đều trong $5$ giây cuối.
	\end{itemize}
	Quãng đường ô tô di chuyển là: $s=3\cdot 10+10\cdot 5-5^2=55$ m.}

\end{ex} 
\begin{ex}%[2D4V2-4]
	Gọi $F(x)$ là một nguyên hàm của hàm số $f(x)=3^{2x+1} 2^{1+3x}$, biết $F(0)=\dfrac{8}{\ln 72}$. Tính $F(-2)$. (làm tròn kết quả đến hàng phần trăm).
	\shortans{$0{,}47$}
	\loigiai{
		Ta có:\\
		$F(x)=\displaystyle \int{\left(3^{2x+1}\cdot 2^{1+3x} \right)}\mathrm{d}x=\int\left(3\cdot3^{2x}\cdot2\cdot2^{3x}\right)\mathrm{d}x=\int\left(6\cdot9^x\cdot8^x \right)\mathrm{d}x\\
		=6\int 72^x\mathrm{d}x=6\cdot\dfrac{72^x}{\ln 72}+C$.\\
		Theo giả thiết, $F(0)=\dfrac{8}{\ln 72}\Rightarrow 6\cdot\dfrac{72^0}{\ln 72}+C=\dfrac{8}{\ln 72}\Rightarrow C=\dfrac{2}{\ln 72}$\\
		$\Rightarrow F(x)=6\cdot\dfrac{{{72}^{x}}}{\ln 72}+\dfrac{2}{\ln 72}\Rightarrow F\left( -2 \right)=6\cdot \dfrac{72^{-2}}{\ln 72}+\dfrac{2}{\ln 72}\approx 0{,}47$.
		}
\end{ex} 
\begin{ex}%[2D4V2-6]
Một viên đạn được bắn thẳng đứng lên từ độ cao $1{,}5$ mét so với mặt đất. Giả sử tại thời điểm $t$ giây (coi $t=0$ là thời điểm viên đạn được bắn lên), vận tốc của nó được cho bởi $v(t)=170-9{,}8\,t\,\left( \text{m/s} \right)$. Tìm độ cao lớn nhất của viên đạn (làm tròn kết quả đến hàng đơn vị).
	\shortans{$1476$}
	\loigiai{
		Gọi $h(t)$ là độ cao của viên đạn tại thời điểm $t$ giây sau khi bắn. Ta có:\\
		$h(t)=\displaystyle \int v(t)\mathrm{d}t=\int{(170-9{,}8t)}\mathrm{d}t=170t-4{,}9t^2+C$.\\
		Từ giả thiết suy ra: $h\left( 0 \right)=1,5\Rightarrow C=1{,}5\Rightarrow h(t)=170t-4{,}9t^2+1,5$.\\
		Viên đạn đạt độ cao lớn nhất khi $v(t)=0\Leftrightarrow 170-9,8\,t\,=0\Leftrightarrow t=\dfrac{850}{49}$.\\
		Khi đó, độ cao lớn nhất của viên đạn là:\\
		$h\left(\dfrac{850}{49}\right)=170 \cdot\dfrac{850}{49}-4{,}9\left( \dfrac{850}{49} \right)^2+1{,}5=\dfrac{144647}{98}\approx 1476$ (m).}
\end{ex} 
\begin{ex}%[2D4V2-6]
Một chiếc cốc chứa nước ở $95^\circ$ C được đặt trong phòng có nhiệt độ ${{20}^{0}}C$. Theo định luật làm mát của Newton, nhiệt độ của nước trong cốc sau $t$ phút (xem $t=0$ là thời điểm nước ở $95^\circ$ C là một hàm số $(t)$. Tốc độ giảm nhiệt độ của nước trong cốc tại thời điểm t phút được xác định bởi $T'(t)=\left(-\dfrac{3}{2} \mathrm\mathrm{e}^{-\tfrac{t}{50}}\right)^\circ$ C/phút). Tính nhiệt độ của nước tại thời điểm $t=40$ phút (làm tròn kết quả đến hàng phần chục).
	\shortans{$53{,}7$}
	\loigiai{
		Ta có:\\
$\displaystyle T(t)=\int T'(t)\mathrm{d}t
=\int\left( -\frac{3}{2}\mathrm{e}^{-\tfrac{t}{50}} \right)\mathrm{d}t
=-\frac{3}{2}\int\left({\mathrm{e}^{-\tfrac{1}{50}}} \right)^t\mathrm{d}t\\
=-\frac{3}{2}\cdot\frac{\left(\mathrm{e}^{-\tfrac{1}{50}} \right)^t}{\ln \left(\mathrm{e}^{-\tfrac{1}{50}}\right)}+C
=75\left(\mathrm{e}^{-\frac{1}{50}}\right)^t+C$.\\
		Vì $t=0$ là thời điểm nước ở $95^\circ$ C nên $T(0)=95\Rightarrow 75\left(\mathrm{e}^{-\tfrac{1}{50}} \right)^\circ+C=95\Rightarrow C=20$.\\ 
		Suy ra $T(t)=75\left(\mathrm{e}^{-\frac{1}{50}} \right)^t+20$.\\
		Do đó, nhiệt độ của nước tại thời điểm $t=40$ phút là: \\
		$T(40)=75\left(\mathrm{e}^{-\tfrac{1}{50}} \right)^{40}+20\approx 53{,}7 ^\circ$ C.}
\end{ex} 
\begin{ex}%[2D4V2-6]
	Doanh thu bán hàng của một công ty khi bán một loại sản phẩm là số tiền $R(x)$ (triệu đồng) thu được khi $x$ đơn vị sản phẩm được bán ra. Tốc độ biến động (thay đổi) của doanh thu khi $x$ đơn vị sản phẩm đã được bán là hàm số $M_R(x)=R'(x)$. Một công ty công nghệ cho biết, tốc độ biến đổi của doanh thu khi bán một loại con chip của hãng được cho bởi $M_R(x)=40-0{,}1x$, ở đó $x$ là số lượng chip đã bán. Hỏi doanh thu của công ty khi đã bán 500 con chip bằng bao nhiêu tỉ đồng?
	\shortans{$7{,}5$}
	\loigiai{
		Vì $R'(x)=M_R(x)$ nên doanh thu $R(x)$ là một nguyên hàm của $M_R(x)$.\\
		Ta có: $R(x)=\displaystyle \int M_R(x) \mathrm{d}x=\int{(40-0{,}1x)}\mathrm{d}x=40x-0{,}05 x^2+C$.\\
		Khi $x=0$, tức là chưa bán chip nào thì doanh thu sẽ bằng $0$ (triệu đồng), nên $R\left( 0 \right)=0\Rightarrow C=0$.\\
		Suy ra $R(x)=40x-0{,}05 x^2$.\\
		Do đó, doanh thu của công ty khi đã bán 500 con chip là:\\
		$R(500)=40\cdot 500-0{,}05\cdot 500^2=7500$ (triệu đồng) $=7{,}5$ (tỉ đồng).		
	}
\end{ex}
\Closesolutionfile{ans}
% \indapan{6}{ans/ans-2C4B11-De2-kq}