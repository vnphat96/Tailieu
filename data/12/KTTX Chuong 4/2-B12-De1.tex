\begin{name}
	{NGUYÊN HÀM - TÍCH PHÂN}
	{KT TÍCH PHÂN}
	{\tentruong}
	{\thoigian}
\end{name}
\setcounter{ex}{0}\setcounter{bt}{0}
\Opensolutionfile{ans}[ans/ans-2-B12-De1-NLC]
\TN
\begin{ex}%[2D4N2-1]
	Biết $\displaystyle\displaystyle\int\limits f(x) \mathrm{\,d} x=F(x)+C$. Trong các khẳng định sau, khẳng định nào đúng?
	\choice 
		{$\displaystyle\displaystyle\int\limits\limits_a^b f(x) \mathrm{\,d} x=F(b) \cdot F(a)$}
		{$\displaystyle\displaystyle\int\limits\limits_a^b f(x) \mathrm{\,d}x=F(a)-F(b)$}
		{\True $\displaystyle\displaystyle\int\limits\limits_a^b f(x) \mathrm{\,d}x=F(b)-F(a)$}
		{$\displaystyle\displaystyle\int\limits\limits_a^b f(x) \mathrm{\,d} x=F(b)+F(a)$}
	\loigiai{
		Ta có $\displaystyle\displaystyle\int\limits\limits_a^b f(x) \mathrm{\,d}x=F(b)-F(a)$.
	}
\end{ex}
\begin{ex}%[2D4N2-2]
	Tính tích phân $\displaystyle\displaystyle\int\limits\limits_1^2(2a x+b) \mathrm{\,d} x$.
	\choice 
		{\True $3a+b$}
		{$3a+2b$}
		{$a+2 b$}
		{$a+b$}
	\loigiai{
		Ta có $\displaystyle\displaystyle\int\limits\limits_1^2(2 a x+b) \mathrm{\,d} x=\left(a x^2+b x\right)\Big|_1 ^2=4 a+2 b-(a+b)=3 a+b$.
	}
\end{ex}
\begin{ex}%[2D4H2-1]
	Biết $\displaystyle\int\limits_1^8 f(x) \mathrm{\,d} x=-2$, $\displaystyle\int\limits_1^4 f(x) \mathrm{\,d} x=3$ và $\displaystyle\int\limits_1^4 g(x) \mathrm{\,d} x=7$. Mệnh đề nào sau đây \textbf{sai}?
	\choice 
		{$\displaystyle\int\limits_1^4\left[4 f(x)-2 g(x)\right] \mathrm{d} x=-2$}
		{$\displaystyle\int\limits_4^8 f(x) \mathrm{\,d} x=1$}
		{$\displaystyle\int\limits_1^4\left[f(x)+g(x)\right] \mathrm{d} x=10$}
		{\True $\displaystyle\int\limits_4^8 f(x) \mathrm{\,d} x=-5$}
	\loigiai{
		Ta có
		 $\displaystyle\int\limits_4^8 f(x) \mathrm{\,d} x=\displaystyle\int\limits_1^8 f(x) \mathrm{\,d} x-\displaystyle\int\limits_1^4 f(x) \mathrm{\,d} x=-2-3=-5$.
	}
\end{ex}
\begin{ex}%[2D4N2-4]
	Tích phân $I=\displaystyle\int\limits_0^{2018} 2^x \mathrm{\,d} x$ bằng
	 \choice 
		{$\dfrac{2^{2018}}{\ln 2}$}
		{$2^{2018}$}
		{$2^{2018}-1$}
		{\True $\dfrac{2^{2018}-1}{\ln 2}$}
	\loigiai{
		Ta  có
		$I=\displaystyle\int\limits_0^{2018} 2^x \mathrm{\,d} x=\dfrac{2^x}{\ln 2}\,\bigg|_0 ^{2018}=\dfrac{2^{2018}-1}{\ln 2}$.
	}
\end{ex}
\begin{ex}%[2D4N2-3]
	Tích phân $I=\displaystyle\int\limits_{\tfrac{\pi}{4}}^{\tfrac{\pi}{3}} \dfrac{\mathrm{\,d} x}{\sin ^2 x}$ bằng
	\choice 
		 {$\cot \dfrac{\pi}{3}-\cot \dfrac{\pi}{4}$} 
		 {$\cot \dfrac{\pi}{3}+\cot \dfrac{\pi}{4}$}
		 {\True $-\cot \dfrac{\pi}{3}+\cot \dfrac{\pi}{4}$}
	 	 {$-\cot \dfrac{\pi}{3}-\cot \dfrac{\pi}{4}$}
	\loigiai{
		Ta có $I=\displaystyle\int\limits_{\tfrac{\pi}{4}}^{\tfrac{\pi}{3}} \dfrac{\mathrm{\,d} x}{\sin ^2 x}=-\cot x\,\bigg|_{\tfrac{\pi}{4}} ^{\tfrac{\pi}{3}}=-\cot \dfrac{\pi}{3}+\cot \dfrac{\pi}{4}$.
	}
\end{ex}
\begin{ex}%[2D4N2-2]
	Tính tích phân $I=\displaystyle\int\limits_1^2\left(\dfrac{2}{x}-\dfrac{1}{x^2}\right) \mathrm{d} x$.
	\choice 
		{$I=2 \ln 2$}
		{\True $I=2 \ln 2-\dfrac{1}{2}$}
		{$I=2 \mathrm{e}+\dfrac{1}{2}$}
		{$I=0$}
	\loigiai{
		Ta có $I=\displaystyle\int\limits_1^2\left(\dfrac{2}{x}-\dfrac{1}{x^2}\right) \mathrm{d} x=\left(2 \ln |x|+\dfrac{1}{x}\right)\bigg|_1 ^2=\left(2 \ln 2+\dfrac{1}{2}\right)-(2 \ln 1+1)=2 \ln 2-\dfrac{1}{2}$.
	}
\end{ex}
\begin{ex}%[2D4H2-3]
	Tính tích phân $I=\displaystyle\int\limits_0^{\tfrac{\pi}{4}} \tan ^2 x \mathrm{\,d} x$. 
	\choice 
		{$I=2$}
		{$I=\ln 2$}
		{$I=\dfrac{\pi}{12}$}
		{\True $I=1-\dfrac{\pi}{4}$}
	\loigiai{
		Ta có
		\begin{eqnarray*}
			I=\displaystyle\int\limits_0^{\tfrac{\pi}{4}} \tan ^2 x \mathrm{\,d} x=\displaystyle\int\limits_0^{\tfrac{\pi}{4}}\left( \dfrac{1}{\cos^2x}-1\right)  \mathrm{d} x&=&\left( \tan x-x\right) \bigg|_0^{\tfrac{\pi}{4}}\\
			&=&\left( \tan \dfrac{\pi}{4}-\dfrac{\pi}{4}\right)-\left( \tan 0-0\right)=1-\dfrac{\pi}{4}. 
		\end{eqnarray*} 
		
	}
\end{ex}
\begin{ex}%[2D4H2-2]
	Cho $a$, $b$ là các số thực dương thỏa mãn $\sqrt{a}-\sqrt{b}+1=0$. Tính tích phân $I=\displaystyle\int\limits_a^b \dfrac{\mathrm{\,d} x}{\sqrt{x}}$.
	\choice 
		{$I=-2$}
		{$I=1$}
		{$I=\dfrac{1}{2}$}
		{\True $I=2$}
	\loigiai{
		Ta có\\ $I=\displaystyle\int\limits_a^b \dfrac{\mathrm{\,d} x}{\sqrt{x}}=\displaystyle\int\limits_a^b x^{-\tfrac{1}{2}} \mathrm{\,d} x=2 \sqrt{x}\,\bigg|_a ^b=2\left( \sqrt{b}-\sqrt{a}\right) =2\left( 1-\left(\sqrt{a}-\sqrt{b}+1\right) \right) =2\cdot 1=2 $.
	}
\end{ex}
\begin{ex}%[2D4N2-4]
	Cho $\displaystyle\int\limits_2^5 \dfrac{\mathrm{\,d} x}{x}=\ln a$. Tìm $a$. 
	\choice 
		{$2$}
		{$\dfrac{2}{5}$}
		{\True $\dfrac{5}{2}$}
		{$5$}
	\loigiai{ 
		Ta có $\displaystyle\int\limits_2^5 \dfrac{\mathrm{\,d} x}{x}=\ln a \Leftrightarrow \ln |x|\, \bigg|_2^5=\ln a \Leftrightarrow \ln 5-\ln 2=\ln a \Leftrightarrow \ln \dfrac{5}{2}=\ln a \Leftrightarrow a=\dfrac{5}{2}$.
	}
\end{ex}
\begin{ex}%[2D4H2-2]
	Cho hàm số $f(x)$ liên tục trên $\mathbb{R}$ và $\displaystyle\int\limits_0^2\left( f(x)+2 x\right)  \mathrm{d} x=5$. Tính $\displaystyle\int\limits_0^2 f(x) \mathrm{\,d} x$.
	\choice
		{$-9$}
		{$-1$}
		{$9$}
		{\True $1$}
	\loigiai{
		Ta có $\displaystyle\int\limits_0^2\left( f(x)+2 x\right)  \mathrm{d} x=\displaystyle\int\limits_0^2 f(x) \mathrm{\,d} x+\displaystyle\int\limits_0^2 2 x \mathrm{\,d} x=\displaystyle\int\limits_0^2 f(x) \mathrm{\,d} x+4=5$. Do đó $\displaystyle\int\limits_0^2 f(x) \mathrm{\,d} x=1$.
	}
\end{ex}
\begin{ex}%[2D4H2-1]
	Cho hai tích phân $\displaystyle\int\limits_{-2}^5 f(x) \mathrm{\,d} x=8$ và $\displaystyle\int\limits_5^{-2} g(x) \mathrm{\,d} x=3$. Tính $I=\displaystyle\int\limits_{-2}^5\left[ f(x)-4 g(x)-1\right] \mathrm{d} x$.
	\choice 
		{$I=-11$}
		{\True $I=13$}
		{ $I=27$}
		{ $I=3$}
	\loigiai{
		Ta có\\ $I=\displaystyle\int\limits_{-2}^5\left[ f(x)-4 g(x)-1\right]  \mathrm{d} x=\displaystyle\int\limits_{-2}^5 f(x) \mathrm{\,d} x+4 \displaystyle\int\limits_{5}^{-2} g(x) \mathrm{\,d} x-x\,\bigg|_{-2} ^5=8+4\cdot 3-(5+2)=13$.
	}
\end{ex}
\begin{ex}%[2D4H2-2]
	Cho hàm số $y=f(x)=\heva{&3 x^2 & \text { khi } 0 \leq x \leq 1 \\ &4-x & \text { khi } 1 \leq x \leq 2}$. Tính tích phân $\displaystyle\int\limits_0^2 f(x) \mathrm{\,d} x$.
	\choice 
		{\True $\dfrac{7}{2}$}
		{$1$}
		{$\dfrac{5}{2}$}
		{$\dfrac{3}{2}$}
	\loigiai{ 
		Ta có 
		\begin{eqnarray*}
		\displaystyle\int\limits_0^2 f(x) \mathrm{\,d} x&=&\displaystyle\int\limits_0^1 f(x) \mathrm{\,d} x+\displaystyle\int\limits_1^2 f(x) \mathrm{\,d} x=\displaystyle\int\limits_0^1\left(3 x^2\right) \mathrm{d} x+\displaystyle\int\limits_1^2(4-x) \mathrm{\,d} x\\
		&=& x^3\,\bigg|_0 ^1+\left(4 x-\dfrac{x^2}{2}\right)\bigg|_1 ^2=\left( 1^3-0^3\right) +\left[ \left(4\cdot 2-\dfrac{2^2}{2}\right)-\left( 4\cdot1-\dfrac{1^2}{2}\right) \right] = \dfrac{7}{2}.	
		\end{eqnarray*}
	}
\end{ex}
\Closesolutionfile{ans}
% \indapan{6}{ans/ans-2-B12-De1-NLC}
\TNTF
\Opensolutionfile{ans}[ans/ans-2-B12-De1-DS]
\begin{ex}%[2D4V2-2]
	Cho $f(x)$ và $g(x)$ là các hàm số liên tục bất kì trên đoạn $[a;b]$.
	\choiceTF
		{\True $\displaystyle\int\limits_a^b\left(f(x)-g(x)\right) \mathrm{d} x=\displaystyle\int\limits_a^b f(x) \mathrm{\,d} x-\displaystyle\int\limits_a^b g(x) \mathrm{\,d} x$}
		{$\displaystyle\int\limits_a^a\left[f(x)+g(x)\right] \mathrm{d} x=1$}
		{Nếu $\displaystyle\int\limits_a^b f(x) \mathrm{\,d} x=3$ và $\displaystyle\int\limits_a^b\left[3 f(x)-g(x)\right] \mathrm{d} x=10$ thì $\displaystyle\int\limits_a^b g(x) \mathrm{\,d} x=1$}
		{\True Nếu $f(x)+2 f\left(\dfrac{1}{x}\right)=3 x$ với $x \in\left[\dfrac{1}{2}; 2\right]$. Tính $\displaystyle\int\limits_{\tfrac{1}{2}}^2 \dfrac{f(x)}{x} \mathrm{\,d} x=\dfrac{3}{2}$}
	\loigiai{
		\begin{itemchoice}
			\itemch Đúng. Do tính chất tích phân.
			\itemch Sai. Ta có $\displaystyle\int\limits_a^a\left[f(x)+g(x)\right] \mathrm{d} x=0$.
			\itemch Sai. Ta có 
			\begin{eqnarray*}
				\displaystyle\int\limits_a^b\left[ 3 f(x)-g(x)\right]  \mathrm{d} x=10 &\Leftrightarrow& 3 \displaystyle\int\limits_a^b f(x) \mathrm{\,d} x-\displaystyle\int\limits_a^b g(x) \mathrm{\,d} x=10\\ &\Leftrightarrow& 3\cdot 3-\displaystyle\int\limits_a^b g(x) \mathrm{\,d} x=10 \Leftrightarrow\displaystyle\int\limits_a^b g(x) \mathrm{\,d} x=-1.
			\end{eqnarray*}
			\itemch Đúng. Ta có $f(x)+2 f\left(\dfrac{1}{x}\right)=3 x \Rightarrow f\left(\dfrac{1}{x}\right)+2 f(x)=\dfrac{3}{x}$.\\
			Suy ra $\heva{&f(x)+2 f\left(\dfrac{1}{x}\right)=3x \\& 4 f(x)+2 f\left(\dfrac{1}{x}\right)=\dfrac{6}{x}} \Rightarrow f(x)=\dfrac{2}{x}-x\Rightarrow \dfrac{f(x)}{x}=\dfrac{2}{x^2}-1$.\\
			Do đó $\displaystyle\int\limits_{\tfrac{1}{2}}^2 \dfrac{f(x)}{x} \mathrm{\,d} x=\displaystyle\int\limits_{\tfrac{1}{2}}^2\left(\dfrac{2}{x^2}-1\right) \mathrm{d} x=\dfrac{3}{2}$.
		\end{itemchoice}
	}
\end{ex}
\begin{ex}%[2D4V2-2]
	Cho các số thực $a$, $b$ $(a<b)$. Nếu hàm số $y=f(x)$ có đạo hàm là hàm liên tục trên $\mathbb{R}$ và $\displaystyle\displaystyle\int\limits f(x) \mathrm{\,d} x=F(x)+C$.
	\choiceTF
		{$\displaystyle\int\limits_a^b f(x) \mathrm{\,d} x=F(a)-F(b)$}
		{\True $\displaystyle\int\limits_a^b f'(x) \mathrm{\,d} x=f(b)-f(a)$}
		{Nếu $\displaystyle\int\limits_0^2 f(x) \mathrm{\,d} x=2$ thì $\displaystyle\int\limits_0^2\left[ 3 f(x)-2\right]  \mathrm{d} x=4$} 
		{\True Nếu $f(x)+f(2-x)=x^2-2 x+2,\, \forall x \in \mathbb{R}$ và $f(0)=3$ thì $\displaystyle\int\limits_0^2 f'(x) \mathrm{\,d} x=-4$}
	\loigiai{
		\begin{itemchoice}
			\itemch Sai. Ta có $\displaystyle\int\limits_a^b f(x) \mathrm{\,d} x=F(b)-F(a)$.
			\itemch Đúng. Ta có $\displaystyle\int\limits_a^b f'(x) \mathrm{\,d} x=f(x)\,\bigg|_a ^b=f(b)-f(a)$.
			\itemch Sai. Ta có $J=\displaystyle\int\limits_0^2\left[ 3 f(x)-2\right] \mathrm{d} x=3 \displaystyle\int\limits_0^2 f(x) \mathrm{\,d} x-2 \displaystyle\int\limits_0^2 \mathrm{\,d} x=3\cdot 2-2 x\,\bigg|_0 ^2=6-4=2$.
			\itemch Đúng. Ta có $f(x)+f(2-x)=x^2-2 x+2, \forall x \in \mathbb{R}\quad (1)$.\\
			Thay $x=0$ vào (1) ta được
			$f(0)+f(2)=2 \Rightarrow f(2)=2-f(0)=2-3=-1$.\\
			Từ đó có $ \displaystyle\int\limits_0^2 f'(x) \mathrm{\,d} x=f(2)-f(0)=-1-3=-4.$
		\end{itemchoice}
	}
\end{ex}
\begin{ex}%[2D4V2-2]
	Giả sử $f(x)$ và $g(x)$ là hai hàm số bất kỳ có đạo hàm liên tục trên $\mathbb{R}$ và $a$, $b$, $c$ là các số thực.
	\choiceTF
		{\True $\displaystyle\int\limits_a^b f(x) \mathrm{\,d} x=-\displaystyle\int\limits_b^a f(x) \mathrm{\,d} x$} 
		{Nếu $f(x)=\dfrac{1}{x}$ thì $\displaystyle\int\limits_{-3}^{-2} f(x) \mathrm{\,d} x=\ln x\,\bigg|_{-3} ^{-2}$}
		{\True $\displaystyle\int\limits_a^b f(x) \mathrm{\,d} x+\displaystyle\int\limits_b^c f(x) \mathrm{\,d} x+\displaystyle\int\limits_c^a f(x) \mathrm{\,d} x=0$}
		{Nếu $3 f(x)+x f'(x)=x^{2018}$ với mọi $x \in[0 ; 1]$ thì $\displaystyle\int\limits_0^1 f(x) \mathrm{\,d}x=\dfrac{1}{2020\cdot 2019}$}
	\loigiai{
		\begin{itemchoice}
			\itemch Đúng. Theo tính chất của tích phân.
			\itemch Sai. Ta có $\displaystyle\int\limits_{-3}^{-2} \dfrac{1}{x} \mathrm{\,d} x=\left( \ln |x|\right) \bigg|_{-3} ^{-2}$.
			\itemch Đúng. Ta có \\ $\displaystyle\int\limits_a^b f(x) \mathrm{\,d} x+\displaystyle\int\limits_b^c f(x) \mathrm{\,d} x+\displaystyle\int\limits_c^a f(x) \mathrm{\,d} x=\displaystyle\int\limits_a^c f(x) \mathrm{\,d} x+\displaystyle\int\limits_c^a f(x) \mathrm{\,d} x=\displaystyle\int\limits_a^a f(x) \mathrm{\,d} x=0$.
			\itemch Sai. Nhân hai vế của đẳng thức $3 f(x)+x f'(x)=x^{2018}$ với $x^2$ ta được $$3 x^2 f(x)+x^3 f'(x)=x^{2020} \Rightarrow\left[x^3 f(x)\right]'=x^{2020}\Rightarrow x^3 f(x)=\displaystyle\displaystyle\int\limits x^{2020} \mathrm{\,d} x=\dfrac{x^{2021}}{2021}+C\,(*).$$
			Thay $x=0$ vào hai vế $(*)$ ta được $C=0 \Rightarrow f(x)=\dfrac{x^{2018}}{2021}$.\\
			Vậy $\displaystyle\int\limits_0^1 f(x) \mathrm{\,d} x=\displaystyle\int\limits_0^1 \dfrac{1}{2021} x^{2018} \mathrm{\,d} x=\dfrac{1}{2021} \cdot \dfrac{1}{2019} x^{2019}\,\bigg|_0 ^1=\dfrac{1}{2021 \cdot 2019}$.
		\end{itemchoice}
	}
\end{ex}
\begin{ex}%[2D4V2-2]
	Cho $F(x)$ là nguyên hàm của hàm số $f(x)$.
	\choiceTF
		{\True $\displaystyle\int\limits_1^3 f(x) \mathrm{\,d} x=F(3)-F(1)$}
		{\True Nếu $f(x)=\dfrac{2}{x}+\dfrac{3}{x^2}\,(x \neq 0)$, $F(1)=1$ thì $F(3)=2 \ln 3+3$}
		{Nếu $F(-1)=1$ và $F(2)=4$ thì $\displaystyle\int\limits_{-1}^2\left[ f(x)+2 x\right]  \mathrm{d} x=9$}
		{\True Nếu hàm số $y=f(x)$ có đạo hàm liên tục trên $[0;1]$ thỏa $2 f(x)+3 f(1-x)=\sqrt{1-x^2}$ thì $\displaystyle\int\limits_0^1 f'(x) \mathrm{\,d} x=1$}
	\loigiai{
		\begin{itemchoice}
			\itemch  Đúng. Theo định nghĩa tích phân.
			\itemch Đúng. Ta có $\displaystyle\int\limits_1^3 f(x) \mathrm{\,d} x=F(3)-F(1)$. Suy ra $$F(3)=F(1)+\displaystyle\int\limits_1^3\left(\dfrac{2}{x}+\dfrac{3}{x^2}\right) \mathrm{\,d} x=1+\left(2 \ln x-\dfrac{3}{x}\right)\bigg|_1 ^3=2 \ln 3+3.$$
			\itemch Sai. Ta có $I=\displaystyle\int\limits_{-1}^2\left[ f(x)+2 x\right]  \mathrm{d} x=\left[F(x)+x^2\right]\bigg|_{-1} ^2=F(2)+4-F(-1)-1=6$.
			\itemch Đúng. Ta có 
			$\displaystyle\int\limits_0^1 f'(x) \mathrm{\,d} x=f(x)\,\bigg|_0 ^1=f(1)-f(0)$.\\
			Từ $2 f(x)+3 f(1-x)=\sqrt{1-x^2}\Rightarrow\heva{& 2f(0)+3 f(1)=1 \\ &2 f(1)+3 f(0)=0} \Leftrightarrow\heva{&f(0)=-\dfrac{2}{5} \\& f(1)=\dfrac{3}{5}.}$\\	
			Vậy $I=\displaystyle\int\limits_0^1 f'(x) \mathrm{\,d} x=f(1)-f(0)=\dfrac{3}{5}+\dfrac{2}{5}=1$.
		\end{itemchoice}
	}
\end{ex}
\Closesolutionfile{ans}
% \indapan{2}{ans/ans-2-B12-De1-DS}
\Opensolutionfile{ans}[ans/ans-2-B12-De1-KQ]
\TNSA
\begin{ex}%[2D4H2-6]
	Một xe ô tô đang di chuyển với tốc độ $22$ m/s thì gặp chướng ngại vật. Người lái xe phản ứng $3$ giây sau đó và đạp phanh khẩn cấp, kể từ thời điểm đạp phanh, ô tô chuyển động chậm dần đều với tốc độ $v(t)=36-6 t$ m/s, trong đó $t$ là thời gian tính bằng giây kể từ lúc đạp phanh. Hỏi quãng đường ô tô đi được từ lúc phát hiện chướng ngại vật đến khi ô tô dừng hẳn là bao nhiêu mét?
	\shortans{$174$} 
	\loigiai{
		Quãng đường ô tô đi được từ lúc phát hiện chướng ngại vật đến khi đạp phanh là $66$ m.\\
		Xe ô tô dừng hẳn khi $v(t)=0 \Leftrightarrow 36-6 t=0 \Leftrightarrow t=6$.\\
		Quãng đường ô tô đi được từ lúc đạp phanh đến lúc dừng lại là $\displaystyle\int\limits_0^6(36-6 t) \mathrm{\,d} t=108$ m.\\
		Vậy quãng đường ô tô đi được từ lúc phát hiện chướng ngại vật đến khi ô tô dừng hẳn là $66+108=174$ m.
	}
\end{ex}

\begin{ex}%[2D4V2-2]
	Cho hàm số $y=f(x)$ liên tục trên $\mathbb{R}$. Hàm số $y=f'(x)$ có đồ thị $(C)$ như hình vẽ, $(C)$ cắt trục $Ox$ tại ba điểm phân biệt có hoành độ $a<b<c$.
	\begin{center}
		\begin{tikzpicture}[line join = round, line cap = round,>=stealth,x = 1cm,y = .6cm] 
			%Vẽ hệ trục Oxy 
			\draw[->] (-2.5,0)--(0,0) node[below right]{$O$}--(4.5,0) node[below]{$x$}; 
			\draw (-1.1,.3) node {$a$} (1.1,.3) node {$b$}  (3.1,-.3) node{$c$} (2.9,2) node[rotate=70]{$(C)$};
			\draw[->] (0,-4.5)--(0,4.5) node[right]{$y$}; 
			\draw[samples=200,domain=-1.42:3.42,smooth] plot (\x, {(\x)^3-3*(\x)^2-\x+3}); 
		\end{tikzpicture}
	\end{center}
	Biết rằng diện tích hình phẳng giới hạn bởi $(C)\colon y=f'(x)$ và $O x$ bằng $15$, $f(a)=5$, $f(c)=6$. Tính $f(b)$.
	\shortans{$13$}
	\loigiai{
		Diện tích hình phẳng giới hạn bởi $(C)\colon y=f'(x)$ và $Ox$ bằng $15$, do đó
		$$
		15=\displaystyle\int\limits_a^c\left|f'(x)\right| \mathrm{\,d} x=\displaystyle\int\limits_a^b\left|f'(x)\right| \mathrm{\,d} x+\displaystyle\int\limits_b^c\left|f'(x)\right| \mathrm{\,d} x=2 f(b)-f(a)-f(c) .
		$$
		Suy ra $2 f(b)=15+f(a)+f(c) \Rightarrow f(b)=13$.
	}
\end{ex}

\begin{ex}%[2D4H2-2]
	Biết rằng $\displaystyle\int\limits_0^2 \dfrac{x^2}{x+1} \mathrm{\,d} x=a+\ln b$ với $a, b \in \mathbb{Z}$, $b>0$. Tính $2a+b$.
	\shortans{$3$}
	\loigiai{
		Ta có $\displaystyle\int\limits_0^2 \dfrac{x^2}{x+1} \mathrm{~d} x=\displaystyle\int\limits_0^2\left(x-1+\dfrac{1}{x+1}\right) \mathrm{d} x=\left(\dfrac{x^2}{2}-x+\ln |x+1|\right)\bigg|_0 ^2=\ln 3$.\\
		Suy ra $a=0$, $b=3$. Vậy $2 a+b=3$.
	}
\end{ex}

\begin{ex}%[2D4H2-2] 
	Cho $\displaystyle\int\limits_0^1 \dfrac{\mathrm{\,d} x}{\sqrt{x+2}+\sqrt{x+1}}=a \sqrt{b}-\dfrac{8}{3} \sqrt{a}+\dfrac{2}{3},\left(a, b \in \mathbb{N}^*\right)$. Tính $a+2b$.
	\shortans{$8$}
	\loigiai{
		Ta có
		 \begin{eqnarray*}
		 	\displaystyle\int\limits_0^1 \dfrac{\mathrm{\,d} x}{\sqrt{x+2}+\sqrt{x+1}}&=&\displaystyle\int\limits_0^1\left( \sqrt{x+2}-\sqrt{x+1}\right) \mathrm{d} x\\ &=&\dfrac{2}{3}\left(\sqrt{(x+2)^3}-\sqrt{(x+1)^3}\right)\bigg|_0 ^2 
			=2 \sqrt{3}-\dfrac{8}{3} \sqrt{2}+\dfrac{2}{3}.
	\end{eqnarray*}
	Vậy $a=2$, $b=3$, $a+2b=8$.
	}
\end{ex}

\begin{ex}%[2D4H2-6]
	Tại một nơi không có gió, một chiếc khí cầu đang đứng yên ở độ cao $162$ mét so với mặt đất đã được phi công cài đặt cho nó chế độ chuyển động đi xuống. Biết rằng, khí cầu đã chuyển động theo phương thẳng đứng với vận tốc tuân theo quy luật $v(t)=10 t-t^2$, trong đó $t$ phút là thời gian tính từ lúc bắt đầu chuyển động, $v(t)$ được tính theo đơn vị mét/phút. Tìm vận tốc $v$ của khí cầu khi bắt đầu tiếp đất.
	\shortans{$9$}
	\loigiai{
		Gọi thời điểm khí cầu bắt đầu chuyển động là $t=0$, thời điểm khinh khí cầu bắt đầu tiếp đất là $t_1$.
		Quãng đường khí cầu đi được từ thời điểm $t=0$ đến thời điểm khinh khí cầu bắt đầu tiếp đất  $t_1$ là
		\[
		\displaystyle\int\limits_0^{t_1}\left(10 t-t^2\right) \mathrm{d} t=5 t_1^2-\dfrac{t_1^3}{3}=162 \Leftrightarrow \hoac{&t_1 \approx-4{,}93\\& t_1 \approx 10{,}93 \\& t_1=9.}\]		
		Do $v(t) \geq 0$ nên $0 \leq t_1 \leq 10$, suy ra chọn $t_1=9$.\\
		Vậy khi bắt đầu tiếp đất vận tốc $v$ của khí cầu là $v(9)=10\cdot 9-9^2=9$ mét/phút.
	}
\end{ex}

\begin{ex}%[2D4V2-6] 
	Một ô tô chuyển động nhanh dần đều với vận tốc $v(t)=7 t$ m/s. Đi được $5$ s người lái xe phát hiện chướng ngại vật và phanh gấp, ô tô tiếp tục chuyển động chậm dần đều với gia tốc $a=-35 \mathrm{~m} / \mathrm{s}^2$. Tính quãng đường của ô tô đi được từ lúc bắt đầu chuyển bánh cho đến khi dừng hẳn? (quãng đường tính theo đơn vị m).
	\shortans{$105$}
	\loigiai{
		Quãng đường ô tô đi được trong $5$ s đầu là $s_1=\displaystyle\int\limits_0^5 7 t \mathrm{\,d} t=7 \dfrac{t^2}{2}\,\bigg|_0 ^5=87{,}5$.\\
		Phương trình vận tốc của ô tô khi người lái xe phát hiện chướng ngại vật là $v_2(t)=35-35 t$.\\
		Khi xe dừng lại hẳn thì $v_2(t)=0 \Leftrightarrow 35-35 t=0 \Leftrightarrow t=1$.\\
		Quãng đường ô tô đi được từ khi phanh gấp đến khi dừng lại hẳn là
		\[
		s_2=\displaystyle\int\limits_0^1\left(35-35 t\right) \mathrm{d} t=\left(35 t-35 \dfrac{t^2}{2}\right)\bigg|_0 ^1=17{,}5.\]
		Do đó quãng đường của ô tô đi được từ lúc bắt đầu chuyển bánh cho đến khi dừng hẳn là \[s=s_1+s_2=87{,}5+17{,}5=105.\]
	}
\end{ex}
\Closesolutionfile{ans}
% \indapan{6}{ans/ans-2-B12-De1-KQ}