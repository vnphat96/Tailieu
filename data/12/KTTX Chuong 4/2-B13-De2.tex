\begin{name}
	{NGUYÊN HÀM - TÍCH PHÂN}
	{KT ỨNG DỤNG NGUYÊN HÀM - TÍCH PHÂN}
	{\tentruong}
	{\thoigian}
\end{name}
\setcounter{ex}{0}\setcounter{bt}{0}
\Opensolutionfile{ans}[ans/ans-2-B13-De2-TN]
\TN
\begin{ex}%[Vovanle]%[2D4N3-1]
Diện tích hình phẳng giới hạn bởi đồ thị hàm số $y=\sin x$, trục hoành và hai đường thẳng $x=0$, $x=2\pi$ được xác định bởi công thức
	\choice
	{$S=\displaystyle\displaystyle\int\limits_0^{2\pi}\sin x\mathrm{\,d}x$}
	{$S=\pi\displaystyle\int\limits_0^{2\pi}\sin x\mathrm{\,d}x$}
	{$S=\pi\displaystyle\int\limits_0^{2\pi}\sin^2 x\mathrm{\,d}x$}
	{\True $S=\displaystyle\int\limits_0^{2\pi}\left| \sin x \right|\mathrm{\,d}x$}
	\loigiai{
Diện tích hình phẳng được tính theo công thức 
$$S=\displaystyle\int\limits_0^{2\pi}\left|\sin x\right|\mathrm{\,d}x.$$
}
\end{ex}
\begin{ex}%[Vovanle]%[2D4N3-1]
Diện tích hình phẳng giới hạn bởi parabol $y=x^2-4$, trục hoành và hai đường thẳng $x=0$, $x=3$ bằng
	\choice
	{\True $\dfrac{23}{3}$}
	{$S=3$}
	{$\dfrac{7}{3}$}
	{$\dfrac{16}{3}$}
	\loigiai{
Diện tích hình phẳng là $$S=\displaystyle\int\limits_0^{3}\left|x^2-4\right|\mathrm{\,d}x=\dfrac{23}{3}.$$
}
\end{ex}
\begin{ex}%[Vovanle]%[2D4N3-3]
Thể tích khối tròn xoay do hình phẳng giới hạn bởi các đường thẳng $y=\sqrt{x}$, trục $Ox$ và hai đường thẳng $x=1$ và $x=2$. Khi quay quanh trục hoành được tính theo công thức nào?
	\choice
	{\True $V=\pi\displaystyle\int\limits_1^2 x\mathrm{\,d}x$}
	{$V=\pi \displaystyle\int\limits_1^2 \sqrt{x}\mathrm{\,d}x$}
	{$V=\pi^2\displaystyle\int\limits_1^2 x\mathrm{\,d}x$}
	{$V=\displaystyle\int\limits_1^2 \left|\sqrt{x}\right|\mathrm{\,d}x$}
	\loigiai{
Thể tích khối tròn xoay do hình phẳng được tính theo công thức
$$V=\pi \displaystyle\int\limits_1^2 \left(\sqrt{x}\right)^2\mathrm{\,d}x=\pi \displaystyle\int\limits_1^2 x\mathrm{\,d}x.$$
}
\end{ex}
\begin{ex}%[Vovanle]%[2D4H3-1]
\immini{Hình phẳng $(H)$ được giới hạn bởi đồ thị hàm số bậc ba và trục hoành được chia thành hai phần có diện tích lần lượt là $S_1$ và $S_2$ (như hình vẽ).\\ 
Biết $\displaystyle\int\limits_{-1}^1f(x)\mathrm{\,d}x=\dfrac{8}{3}$ và $\displaystyle\int\limits_1^4f(x)\mathrm{\,d}x=-\dfrac{63}{8}$. Khi đó diện tích $S$ của hình phẳng $(H)$ bằng
}{
\begin{tikzpicture}[line join=round,line cap=round, font=\footnotesize,scale=0.75,>=stealth]
	\draw[-stealth](-1.5,0)--(4.5,0)node[above]{$x$};
	\draw[-stealth](0,-2.5)--(0,2)node[right]{$y$};			
	\fill (0,0) circle(1pt)node[below left]{$O$}(-0.3,0.3)node[above]{$S_1$}(2.5,-1.2)node[above]{$S_2$};
	\draw[smooth,samples=300,domain=-1.5:4.25] plot(\x,{0.3*((\x)^2-1)*(\x-4)})node[above]{$y=f(x)$};
	\fill[pattern=north east lines]plot[domain=-1:4](\x,{0.3*((\x)^2-1)*(\x-4)})--(-1,0);
	\foreach \x/\g in {-1/140,1/60,4/130}\fill[black] (\x,0) circle (1pt)+(\g:.3)node{$\x$};		
	\end{tikzpicture}
}
	\choice
	{$\dfrac{125}{24}$}
	{$\dfrac{8}3$}
	{\True $\dfrac{253}{24}$}
	{$\dfrac{63}{8}$}
	\loigiai{
Ta có 
$$S_1=\displaystyle\int\limits_{-1}^1f(x)\mathrm{\,d}x=\dfrac{8}{3};
\,S_2=-\displaystyle\int\limits_1^4f(x)\mathrm{\,d}x=\dfrac{63}{8}.$$
Suy ra $S=S_1+S_2=\dfrac{8}{3}+\dfrac{63}{8}=\dfrac{253}{24}$.
}
\end{ex}
\begin{ex}%[Vovanle]%[2D4N3-3]
Hình phẳng giới hạn bởi các đường $y=-x^2+9$, $y=0$, $x=-3$, $x=3$ quay quanh trục $Ox$ tạo thành một khối tròn xoay có thể tích $V$. Khẳng định nào sau đây là đúng?
	\choice
	{$V=\displaystyle\int\limits_{-3}^3\left|-x^2+9\right|\mathrm{\,d}x$}
	{$V=\pi \displaystyle\int\limits_{-3}^3\left|-x^2+9\right|\mathrm{\,d}x$}
	{$V=\displaystyle\int\limits_{-3}^3\left(-x^2+9\right)^2\mathrm{\,d}x$}
	{\True $V=\pi\displaystyle\int\limits_{-3}^3\left(-x^2+9\right)^2\mathrm{\,d}x$}
	\loigiai{
Thể tích khối tròn xoay là
$$V=\pi\displaystyle\int\limits_{-3}^3\left(-x^2+9\right)^2\mathrm{\,d}x.$$
}
\end{ex}
\begin{ex}%[Vovanle]%[2D4N3-1]
Diện tích của hình phẳng giới hạn bởi đồ thị hàm số $y=x^2-4$, trục hoành và hai đường thẳng $x=-2$, $x=2$ bằng
	\choice
	{$S=\pi\displaystyle\int\limits_{-2}^2\left(x^2-4\right)\mathrm{\,d}x$}
	{\True $S=\displaystyle\int\limits_{-2}^2\left|x^2-4\right|\mathrm{\,d}x$}
	{$S=\displaystyle\int\limits_{-2}^2\left(x^2-4\right)\mathrm{\,d}x$}
	{$S=\pi\displaystyle\int\limits_{-2}^2\left(x^2-4\right)^2\mathrm{\,d}x$}
	\loigiai{
Diện tích của hình phẳng là
$$S=\displaystyle\int\limits_{-2}^2\left|x^2-4\right|\mathrm{\,d}x.$$
}
\end{ex}

\begin{ex}%[Vovanle]%[2D4H3-1]
Diện tích hình phẳng giới hạn bởi parabol $y=x^2-4x+5$ và đường thẳng $y=x+1$ được tính theo công thức nào sau đây?
	\choice
	{$S=\displaystyle\int\limits_1^4\left(x^2-5x+4\right)\mathrm{\,d}x$}
	{$S=\displaystyle\int\limits_1^4\left(x^2-5x+4\right)^2\mathrm{\,d}x$}
	{$S=\displaystyle\int\limits_1^4\left|x^2-5x+4\right|\mathrm{\,d}x$}
	{\True $S=\displaystyle\int\limits_1^4\left(x^2+5x+4\right)\mathrm{\,d}x$}
	\loigiai{
Phương trình hoành độ giao điểm của parabol $y=x^2-4x+5$ và đường thẳng $y=x+1$ là
$$x^2-4x+5=x+1\Leftrightarrow x^2-5x+4=0\Leftrightarrow\hoac{&x=1\\&x=4.}$$ 
Diện tích hình phẳng giới hạn bởi parabol $y=x^2-4x+5$ và đường thẳng $y=x+1$ là
$$S=\displaystyle\int\limits_1^4\left|x^2-4x+5-\left(x+1\right)\right|\mathrm{\,d}x=\displaystyle\int\limits_1^4\left|x^2-5x+4\right|\mathrm{\,d}x.$$
}
\end{ex}

\begin{ex}%[Vovanle]%[2D4H3-1]
Diện tích hình phẳng giới hạn bởi đồ thị hàm số $y=x^2$ và đường thẳng $y=2x$ là 
	\choice
	{\True $\dfrac{4}{3}$}
	{$\dfrac{5}{3}$}
	{$\dfrac{3}{2}$}
	{$\dfrac{23}{15}$}
	\loigiai{
Xét phương trình $x^2=2x\Leftrightarrow\hoac{&x=0\\&x=2.}$\\ 
Diện tích hình phẳng giới hạn bởi đồ thị hàm số $y=x^2$ và đường thẳng $y=2x$ là  $$S=\displaystyle\int\limits_0^2\left|x^2-x\right|\mathrm{\,d}x=\left| \displaystyle\int\limits_0^2\left(x^2-x\right)\mathrm{\,d}x\right|=\dfrac{4}{3}.$$
}
\end{ex}

\begin{ex}%[Vovanle]%[2D4H3-1]
\immini{Diện tích phần hình phẳng phần gạch sọc trong hình vẽ được tính theo công thức nào dưới đây? 
	\choice
	{$\displaystyle\int\limits_{-2}^3 \left[f(x)-g(x)\right]\mathrm{\,d}x$}
	{$\displaystyle\int\limits_{-2}^{5} \left[f(x)-g(x)\right]\mathrm{\,d}x+\displaystyle\int\limits_{5}^3 \left[ g(x)-f(x)\right]\mathrm{\,d}x$}
	{\True $\displaystyle\int\limits_{-2}^0 \left[f(x)-g(x)\right]\mathrm{\,d}x+\displaystyle\int\limits_0^3 \left[g(x)-f(x)\right]\mathrm{\,d}x$}
	{$\displaystyle\int\limits_{-2}^0 \left[g(x)-f(x)\right]\mathrm{\,d}x+\displaystyle\int\limits_0^3 \left[f(x)-g(x)\right]\mathrm{\,d}x$}
}{
\begin{tikzpicture}[line join=round,line cap=round, font=\footnotesize,scale=0.5,>=stealth]
	\draw[-stealth](-2.5,0)--(4,0)node[below]{$x$};
	\draw[-stealth](0,-3.7)--(0,7)node[right]{$y$};			
	\fill (0,0) circle(1pt)node[below left]{$O$};
	\draw[smooth,samples=300,domain=-2.2:3.5] plot(\x,{(5/6)*(\x+2)*(\x-1)*(\x-3)})node[above]{$y=f(x)$};
	\draw[smooth,samples=300,domain=-2.4:3.25] plot(\x,{(-5/6)*(\x+2)*(\x-3)})node[below right]{$y=g(x)$};
	\fill[pattern=north east lines]plot[domain=-2:3](\x,{(5/6)*(\x+2)*(\x-1)*(\x-3)})--plot[domain=3:-2](\x,{(-5/6)*(\x+2)*(\x-3)});
	\foreach \x/\g in {-2/150}\fill[black] (\x,0) circle (1pt)+(\g:.6)node{$\x$};
	\foreach \x/\g in {1/-120,3/50}\fill[black] (\x,0) circle (1pt)+(\g:.4)node{$\x$};
	\foreach \x/\g in {5/60}\fill[black] (0,\x) circle (1pt)+(\g:.6)node{$\x$};		
	\end{tikzpicture}
}	
	\loigiai{
Diện tích phần hình phẳng là
$$\displaystyle\int\limits_{-2}^0 \left[f(x)-g(x)\right]\mathrm{\,d}x+\displaystyle\int\limits_0^3 \left[g(x)-f(x)\right]\mathrm{\,d}x.$$
}
\end{ex}

\begin{ex}%[Vovanle]%[2D4V3-1]
Diện tích $S$ của hình phẳng giới hạn bởi đồ thị hai hàm số $y=-x^3$ và $y=x^2-2x$ là
	\choice
	{$S=\dfrac{9}{4}$}
	{$S=\dfrac{7}{3}$}
	{\True $S=\dfrac{37}{12}$}
	{$S=\dfrac{4}{3}$}
	\loigiai{
Hoành độ giao điểm của hai đồ thị là nghiệm của phương trình
$$-x^3=x^2-2x\Leftrightarrow x^3+x^2-2x=0\Leftrightarrow \hoac{&x=-2\\&x=0\\&   x=1.}$$
Diện tích hình phẳng cần tìm là 
\allowdisplaybreaks
\begin{eqnarray*}
S&=&\displaystyle\int\limits_{-2}^0 \left|\left(x^3+x^2-2x\right)\right|\mathrm{\,d}x+\displaystyle\int\limits_0^1 \left|\left(x^3+x^2-2x\right) \right|\mathrm{\,d}x\\
&=&\displaystyle\int\limits_{-2}^0\left(x^3+x^2-2x\right)\mathrm{\,d}x-\displaystyle\int\limits_0^1\left(x^3+x^2-2x \right)\mathrm{\,d}x\\
&=&\left.\left(\dfrac{x^4}4+\dfrac{x^3}3-x^2\right)\right|_{-2}^0-\left.\left(\dfrac{x^4}4+\dfrac{x^3}3-x^2\right)\right|_0^1\\
&=&\dfrac{37}{12}.
\end{eqnarray*}
}
\end{ex}

\begin{ex}%[Vovanle]%[2D4H3-3]
Thể tích vật tròn xoay khi quay hình phẳng $(H)$ xác định bởi các đường $y=\dfrac{1}{3}x^3-x^2$, $y=0$, $x=0$ và $x=3$ quanh trục $Ox$ là
	\choice
	{\True $\dfrac{81\pi}{35}$}
	{$\dfrac{81}{35}$}
	{$\dfrac{71\pi}{35}$}
	{$\dfrac{71}{35}$}
	\loigiai{
Phương trình hoành độ giao điểm 
$$\dfrac{1}{3}x^3-x^2=0\Leftrightarrow \hoac{&x=0\\&x=3.}$$
$$V=\pi\displaystyle\int\limits_0^3\left(\dfrac{1}{3}x^3-x^2\right)^2\mathrm{\,d}x=\pi\displaystyle\int\limits_0^3\left(\dfrac{1}{9}x^6-\dfrac{2}{3}x^5+x^4\right)\mathrm{\,d}x=\dfrac{81\pi}{35}.$$
}
\end{ex}

\begin{ex}%[Vovanle]%[2D4H3-3]
Cho $(H)$ là hình phẳng giới hạn bởi các đường $y=\sqrt{x}$, $y=x-2$ và trục hoành. Biết diện tích của $(H)$ bằng $\dfrac{a}{b}$. Tính giá trị biểu thức $T=a+b$.
	\choice
	{$T=11$}
	{\True $T=13$}
	{$T=10$}
	{$T=19$}
	\loigiai{
\immini{Diện tích của $(H)$ bằng 
$$S=\displaystyle\int\limits_0^2\sqrt{x}\mathrm{\,d}x+\displaystyle\int\limits_2^4\left(\sqrt{x}-x+2\right)\mathrm{\,d}x=\dfrac{10}{3}.$$
Vậy $a=10$; $b=3\Rightarrow a+b=13$.
}{
\begin{tikzpicture}[line join=round,line cap=round, font=\footnotesize,scale=1,>=stealth]
	\draw[-stealth](-0.5,0)--(4.6,0)node[below]{$x$};
	\draw[-stealth](0,-0.5)--(0,2.3)node[right]{$y$};			
	\fill (0,0) circle(1pt)node[below left]{$O$}(4,0) circle(1pt)node[below]{$4$}(0,2) circle(1pt)node[left]{$2$};
	\draw[smooth,samples=300,domain=0:4.5] plot(\x,{sqrt (\x)});
	\draw[smooth,samples=300,domain=4.3:1.4] plot(\x,{\x-2})node[below]{$y=x-2$};
	\draw (1,1)node[above,rotate=30]{$y=\sqrt{x}$};
	\draw [dashed] (4,0)|-(0,2);
	\fill[pattern=north east lines]plot[domain=0:4](\x,{sqrt (\x)})--(2,0)--cycle;			
	\end{tikzpicture}
}	
}
\end{ex}
\Closesolutionfile{ans}
% \indapan{6}{ans/ans-2-B13-De2-TN}

\TNTF
\Opensolutionfile{ans}[ans/ans-2-B13-De2-DS]
\setcounter{ex}{0}
\begin{ex}%[Vovanle]%[2D4H3-1]
\immini{Cho đồ thị hàm số $y=\left(\dfrac{1}{2}\right)^x$, $y=x+1$ và hình phẳng được gạch sọc như hình vẽ.
}{
\begin{tikzpicture}[line join=round,line cap=round, font=\footnotesize,scale=0.75,>=stealth]
	\draw[-stealth](-0.5,0)--(4.6,0)node[below]{$x$};
	\draw[-stealth](0,-0.5)--(0,4)node[right]{$y$};			
	\fill (0,0) circle(1pt)node[below left]{$O$}(2,0) circle(1pt)node[below]{$2$}(0,1) circle(1pt)node[left]{$1$};
	\draw[smooth,samples=300,domain=-0.2:2.5] plot(\x,{\x+1})node[above]{$y=x+1$};
	\draw[smooth,samples=300,domain=-0.2:4] plot(\x,{(0.5)^(\x)})node[above]{$y=\left(\dfrac{1}{2}\right)^x$};	
	\draw (2,3)|-(2,0.25);
	\draw [dashed] (2,0.25)--(2,0);
	\fill[pattern=north east lines]plot[domain=0:2](\x,{\x+1})--(2,0.25)--plot[domain=2:0](\x,{(0.5)^(\x)});			
	\end{tikzpicture}
}
	\choiceTF
	{\True Hình phẳng được gạch sọc giới hạn bởi các đường $x=0$; $x=2$; $y=x+1$; $y=\left(\dfrac{1}{2}\right)^x$}
	{\True Gọi $S_1$ là diện hình phẳng giới hạn bởi trục $Ox$, hai đường thẳng $x=0,\,x=2$ và đồ thị hàm số $y=x+1$. Khi đó $S_1=4$}
	{Gọi $S_2$ là diện hình phẳng giới hạn bởi trục $Ox$, hai đường thẳng $x=0,\,x=2$ và đồ thị hàm số $y=\left(\dfrac{1}{2}\right)^x$. Khi đó $S_2=\dfrac{3}{\ln 2}$}
	{Diện tích hình phẳng được giới hạn bởi các đường $x=0$; $x=2$; $y=x+1$; $y=\left( \dfrac{1}{2}\right)^x$ bằng $4-\dfrac{3}{\ln 2}$}
	\loigiai{

	\begin{itemchoice}
	\itemch Đúng. Hình phẳng được gạch sọc giới hạn bởi các đường $x=0$; $x=2$; $y=x+1$; $y=\left(\dfrac{1}{2}\right)^x$.
	\itemch Đúng. Gọi $S_1$ là diện hình phẳng giới hạn bởi trục $Ox$, hai đường thẳng $x=0,\,x=2$ và đồ thị hàm số $y=x+1$. Khi đó $S_1=\displaystyle\int\limits_0^2\left(x+1\right)\mathrm{\,d}x=\left.\left(\dfrac{x^2}2+x\right)\right|_0^2=2+2=4$.
	\itemch Sai. Gọi $S_2$ là diện hình phẳng giới hạn bởi trục $Ox$, hai đường thẳng $x=0,\,x=2$ và đồ thị hàm số $y=\left( \dfrac{1}{2}\right)^x$. Khi đó $S_2=\displaystyle\int\limits_0^2\left(\dfrac{1}{2}\right)^x\mathrm{\,d}x=\left.\dfrac{\left(\tfrac12\right)^x}{\ln \tfrac12}\right|_0^2=\dfrac{\tfrac{1}{4}-1}{-\ln 2}=\dfrac{3}{4\ln 2}$.
	\itemch Sai. Diện tích hình phẳng được giới hạn bởi các đường $x=0$; $x=2$; $y=x+1$; $y=\left(\dfrac{1}{2}\right)^x$.\\
Ta có $x+1>\left(\dfrac{1}{2}\right)^x$ với mọi $x\in\left[0;2\right]$.\\
Do đó $S=S_1-S_2=4-\dfrac{3}{4\ln 2}$.
	\end{itemchoice}
}
\end{ex}

\begin{ex}%[Vovanle]%[2D4H3-1]
Cho đồ thị các hàm số $y=4-x^2$, $y=x^2$.
\begin{center}
\begin{tikzpicture}[line join=round,line cap=round, font=\footnotesize,scale=1,>=stealth]
\draw[-stealth](-2,0)--(3,0)node[below]{$x$};
	\draw[-stealth](0,-0.5)--(0,4.5)node[right]{$y$};	
\fill[pattern=north east lines]plot[domain=-1:1](\x,{(\x)^2})--plot[domain=1:-1](\x,{-(\x)^2+4});
\draw[smooth,samples=300,domain=-2:2] plot(\x,{(\x)^2})node[above]{$y=x^2$};
\draw[smooth,samples=300,domain=-2.2:2.2] plot(\x,{4-(\x)^2})node[below]{$y=4-x^2$};		
	\fill (0,0) circle(1pt)node[below right]{$O$}(-1,0)circle(1pt)+(-0.1,0) node[below]{$-1$}(1,0)circle(1pt)node[below]{$1$};
	\draw (1,1)--(1,3)(-1,1)--(-1,3);
	\draw[dashed] (1,0)--(1,1)(-1,0)--(-1,1);		
	\end{tikzpicture}
\end{center}
	\choiceTF
	{Hình phẳng được gạch sọc, giới hạn bởi các đường $x=-1$; $x=2$; $y=x^2$; $y=4-x^2$}
	{Gọi $S_1$ là diện hình phẳng giới hạn bởi trục $Ox$, hai đường thẳng $x=-1$, $x=1$ và đồ thị hàm số $y=4-x^2$. Khi đó $S_1=\dfrac{22}{3}$}
	{Gọi $S_2$ là diện hình phẳng giới hạn bởi các đường $y=x^2$; $y=4-x^2$. Khi đó $S_2=16\sqrt2$}
	{Diện tích hình phẳng được giới hạn bởi các đường $x=-1$; $x=1$; $y=x^2$; $y=4-x^2$ là $S=\dfrac{20}3$}
	\loigiai{
	\begin{itemchoice}
	\itemch Sai. Hình phẳng được gạch sọc, giới hạn bởi các đường $x=-1$; $x=1$; $y=x^2$; $y=4-x^2$.	
	\itemch Đúng. $S_1=\displaystyle\int\limits_{-1}^1{\left|4-x^2\right|}\mathrm{\,d}x=\displaystyle\int\limits_{-1}^1{\left(4-x^2\right)}\mathrm{\,d}x=\left.\left( 4x-\dfrac{x^3}3 \right) \right|_{-1}^1=\dfrac{22}3$.
	\itemch Sai. Xét phương trình hoành độ giao điểm $$x^2=4-x^2\Leftrightarrow 2x^2=4\Leftrightarrow x^2=2\Leftrightarrow \hoac{&x=2\\&x=-2.}$$
Do đó 
\allowdisplaybreaks
\begin{eqnarray*}
S_2&=&\displaystyle\int\limits_{-\sqrt2}^{\sqrt2}\left|\left(4-x^2\right)-x^2 \right|\mathrm{\,d}x=\displaystyle\int\limits_{-\sqrt2}^{\sqrt2}\left|4-2x^2\right|\mathrm{\,d}x=\displaystyle\int\limits_{-\sqrt2}^{\sqrt2}\left(4-2x^2\right)\mathrm{\,d}x\\
&=&\left.\left(4x-\dfrac{2x^3}{3}\right)\right|_{-\sqrt2}^{\sqrt2}=\dfrac{16\sqrt2}{3}.
\end{eqnarray*}
	\itemch Đúng. Ta có 
\allowdisplaybreaks
\begin{eqnarray*}	
S&=&\displaystyle\int\limits_{-1}^1\left|\left(4-x^2 \right)-x^2\right|\mathrm{\,d}x=\displaystyle\int\limits_{-1}^1\left|4-2x^2\right|\mathrm{\,d}x=\displaystyle\int\limits_{-1}^1\left(4-2x^2\right)\mathrm{\,d}x\\
&=&\left. \left(4x-\dfrac{2x^3}3\right)\right|_{-1}^1=\dfrac{20}{3}.
\end{eqnarray*}
	\end{itemchoice}
}
\end{ex}

\begin{ex}%[Vovanle]%[2D4H3-3]
Cho đồ thị hàm số $y=5x-x^2$, đường thẳng $y=x$ và phần hình phẳng được gạch sọc như hình vẽ
\begin{center}
\begin{tikzpicture}[line join=round,line cap=round, font=\footnotesize,scale=0.6,>=stealth]
\draw[-stealth](-1,0)--(6,0)node[below]{$x$};
	\draw[-stealth](0,-0.5)--(0,6.5)node[right]{$y$};	
\fill[pattern=north east lines]plot[domain=0:4](\x,\x)--plot[domain=4:0](\x,{-(\x)^2+5*\x});
\draw[smooth,samples=300,domain=-0.5:4.7] plot(\x,\x)node[above]{$y=x$};
\draw[smooth,samples=300,domain=-0.1:5.1] plot(\x,{-(\x)^2+5*\x})node[below]{$y=5x-x^2$};		
	\fill (0,0) circle(1pt)node[below right]{$O$}(4,0)circle(1pt) node[below]{$4$}(0,4)circle(1pt)node[left]{$4$};
	\draw[dashed] (4,0)|-(0,4);		
	\end{tikzpicture}
\end{center}
	\choiceTF
	{\True Diện tích phần hình phẳng được gạch sọc trong hình vẽ là $\dfrac{32}{3}$}
	{Diện tích hình phẳng giới hạn bởi đường cong $y=5x-x^2$, trục hoành và hai đường thẳng $x=0$, $x=5$ là $\dfrac{125}{3}$}
	{\True Thể tích khi quay phần hình phẳng giới hạn bởi đồ thị hàm số $y=5x-x^2$ và đường thẳng $y=x$ quanh trục $Ox$ là $\dfrac{384\pi}{5}$}
	{\True Thể tích khi quay phần hình phẳng giới hạn bởi đường thẳng $y=x$, trục $Ox$, hai đường thẳng $x=2$, $x=5$ quanh trục $Ox$ là $39\pi$}
	\loigiai{
	\begin{itemchoice}
	\itemch Đúng. Xét phương trình hoành độ giao điểm 
	$$5x-x^2=x\Leftrightarrow 4x-x^2=0\Leftrightarrow \hoac{&x=0\\&x=4.}$$
Diện tích hình phẳng được gạch sọc giới hạn bởi hai đường là 
$$\displaystyle\int\limits_0^4{\left(5x-x^2-x \right)}\mathrm{\,d}x=\displaystyle\int\limits_0^4{\left(4x-x^2 \right)}\mathrm{\,d}x=\left.\left( 2x^2-\dfrac{x^3}{3}\right)\right|_0^4=\dfrac{32}3.$$
	\itemch Sai. Diện tích hình phẳng giới hạn bởi đường cong $y=5x-x^2$, trục hoành và hai đường thẳng $x=0$, $x=5$ là 
	$$\displaystyle\int\limits_0^{5}{\left(5x-x^2\right)\mathrm{\,d}x}=\left. \left(\dfrac{5x^2}{2}-\dfrac{x^3}{3}\right)\right|_0^{5}=\dfrac{125}{6}.$$
	\itemch Đúng. Phương trình hoành độ giao điểm hai đường là 
	$$5x-x^2=x\Leftrightarrow 4x-x^2=0\Leftrightarrow \hoac{&x=0\\&x=4.}$$
Thể tích khi quay phần hình phẳng giới hạn bởi đồ thị hàm số $y=5x-x^2$ và đường thẳng $y=x$ quanh trục $Ox$ là
\allowdisplaybreaks
\begin{eqnarray*}
V&=&\pi \displaystyle\int\limits_0^4{{{\left( 5x-x^2 \right)}^2}}\mathrm{\,d}x-\pi \displaystyle\int\limits_0^4{x^2}\mathrm{\,d}x=\pi\displaystyle\int\limits_0^4{\left(x^4-10x^3+24x^2 \right)}\mathrm{\,d}x\\
&=&\pi \left. \left( \dfrac{{x^{5}}}{5}-\dfrac{5x^4}2+8x^3 \right) \right|_0^4=\dfrac{384\pi}{5}.
\end{eqnarray*} 
	\itemch Đúng. Thể tích khi quay phần hình phẳng giới hạn bởi đường thẳng $y=x$, trục $Ox$, hai đường thẳng $x=2$, $x=5$ quanh trục $Ox$ là
$$V_1=\pi \displaystyle\int\limits_2^{5}x^2\mathrm{\,d}x=\pi\left.\dfrac{x^3}{3}\right|_2^{5}=39\pi.$$
	\end{itemchoice}
}
\end{ex}

\begin{ex}%[Vovanle]%[2D4H3-3]
Cho hai đồ thị hàm số $y=x^2-2x-1$ và $y=-x^2+3$ và phần hình phẳng được gạch chéo như hình vẽ.
\begin{center}
\begin{tikzpicture}[line join=round,line cap=round, font=\footnotesize,scale=1,>=stealth]
\draw[-stealth](-2,0)--(3,0)node[below]{$x$};
	\draw[-stealth](0,-2.5)--(0,3.5)node[right]{$y$};	
\fill[pattern=north east lines]plot[domain=2:-1](\x,{(\x)^2-2*\x-1})--plot[domain=-1:2](\x,{-(\x)^2+3});
\draw[smooth,samples=300,domain=-1.2:3] plot(\x,{(\x)^2-2*\x-1})node[above]{$y=x^2-2x-1$};
\draw[smooth,samples=300,domain=-1.8:2.2] plot(\x,{-(\x)^2+3})node[below]{$y=-x^2+3$};		
	\fill (0,0) circle(1pt)node[below right]{$O$}(-1,0)circle(1pt)+(0.1,0) node[below]{$-1$}(2,0)circle(1pt)node[above]{$2$};
	\draw[dashed] (2,0)--(2,-1)(-1,0)--(-1,2);		
	\end{tikzpicture}
\end{center}
	\choiceTF
	{\True Biểu thức diện tích phần hình phẳng gạch chéo trong hình vẽ là 
	$\displaystyle\int\limits_{-1}^2 \left(-2x^2+2x+4 \right)\mathrm{\,d}x$}
	{\True Diện tích hình phẳng giới hạn bởi đường cong $y=x^2-2x-1$, trục hoành và hai đường thẳng $x=0$, $x=1$ là $\dfrac{5}{3}$}
	{Thể tích khi quay phần hình phẳng giới hạn bởi đồ thị hàm số $y=-x^2+3$, trục $Ox$, hai đường thẳng $x=1$, $x=2$ quanh trục $Ox$ là $\dfrac{\pi}{5}$}
	{Thể tích khi quay phần hình phẳng giới hạn bởi đồ thị hàm số $y=x^2-2x-1$, trục $Ox$, hai đường thẳng $x=-1$, $x=2$ quanh trục $Ox$ là $\dfrac{33}{5}$}
	\loigiai{
	\begin{itemchoice}
	\itemch Đúng. Dựa vào hình vẽ ta có diện tích hình phẳng được gạch chéo trong hình vẽ được xác định là biểu thức 
	$$\displaystyle\int_{-1}^2\left[\left(-x^2+2\right)-\left(x^2-2x-2\right)\right]\mathrm{\,d}x=\displaystyle\int_{-1}^2\left(-2x^2+2x+4\right)\mathrm{\,d}x.$$ 	
	\itemch Đúng. Diện tích hình phẳng giới hạn bởi đường cong $y=x^2-2x-1$, trục hoành và hai đường thẳng $x=0,x=1$ là 
	$$\displaystyle\int\limits_0^1\left| x^2-2x-1\right|\mathrm{\,d}x=\dfrac{5}{3}.$$	
	\itemch Sai. Thể tích khi quay phần hình phẳng giới hạn bởi đồ thị hàm số $y=-x^2+3$, trục $Ox$, hai đường thẳng $x=1$, $x=2$ quanh trục $Ox$ là 
	$$\pi\displaystyle\int\limits_1^2(-x^2+3)^2\mathrm{\,d}x=\dfrac{6\pi}{5}.$$	
	\itemch Sai. Thể tích khi quay phần hình phẳng giới hạn bởi đồ thị hàm số $y=x^2-2x-1$, trục $Ox$, hai đường thẳng $x=-1$, $x=2$ quanh trục $Ox$ là 
	$$\pi\displaystyle\int\limits_{-1}^2(x^2-2x-1)^2\mathrm{\,d}x=\dfrac{33\pi }{5}.$$
	\end{itemchoice}
}
\end{ex}
\Closesolutionfile{ans}
\indapan3{ans/ans-2-B13-De2-DS}

\Opensolutionfile{ans}[ans/ans-2-B13-De2-KQ]
\TNSA
\setcounter{ex}{0}
\begin{ex}%[Vovanle]%[2D4H3-1]
Tính diện tích hình phẳng giới hạn bởi đồ thị của hàm số $y=x^3-3x$; $y=x$, hai đường thẳng $x=-1$; $x=2$.
\shortans{$5{,}75$}
	\loigiai{
Diện tích hình phẳng cần tìm là $S=\displaystyle\int\limits_{-1}^2{\left| x^3-3x-x \right|\mathrm{\,d}x=\displaystyle\int\limits_{-1}^2{\left| x^3-4x \right|\mathrm{\,d}x.}}$\\
Ta có $$x^3-3x=x\Leftrightarrow x(x^2-4)=0\Leftrightarrow \hoac{&x=0\\ &x=-2\notin[-1;2]\\&x=2.}$$
Phương trình có hai nghiệm thuộc đoạn $\left[-1;2\right]$ là $x=0$; $x=2$.
\allowdisplaybreaks
\begin{eqnarray*}
  S&=&\displaystyle\int\limits_{-1}^2{\left|x^3-4x\right|}\mathrm{\,d}x=\displaystyle\int\limits_{-1}^0{\left|x^3-4x\right|}\mathrm{\,d}x+\displaystyle\int\limits_0^2{\left| x^3-4x\right|}\mathrm{\,d}x\\
  &=&\left|\displaystyle\int\limits_{-1}^0{(x^3-4x)\mathrm{\,d}x}\right|+\left| \displaystyle\int\limits_0^2{(x^3-4x)\mathrm{\,d}x}\right|\\ 
 &=&\left| \left.\left(\dfrac{x^4}4-2x^2 \right)\right|_0^1\right|+\left|\left. \left(\dfrac{x^4}4-2x^2 \right)\right|_0^2 \right|=\dfrac{23}{4}\approx 5{,}75. 
\end{eqnarray*}
}
\end{ex}

\begin{ex}%[Vovanle]%[2D4H3-3]
Cho hình phẳng giới hạn bởi các đường $y=\sqrt{x}-2$, $y=0$ và $x=9$ quay xung quanh trục $Ox$. Tính thể tích khối tròn xoay tạo thành (làm tròn kết quả thể tích đến hàng phần trăm).
\shortans{$5{,}76$}
	\loigiai{
Phương trình hoành độ giao điểm của đồ thị hàm số $y=\sqrt{x}-2$ và trục hoành 
$$\sqrt{x}-2=0\Leftrightarrow \sqrt{x}=2\Leftrightarrow x=4.$$
Thể tích của khối tròn xoay tạo thành là
\allowdisplaybreaks
\begin{eqnarray*}
V&=&\pi \displaystyle\int\limits_4^{9}{{{\left(\sqrt{x}-2\right)}^2}\mathrm{\,d}x}\\
&=&\pi\displaystyle\int\limits_4^{9}{\left(x-4\sqrt{x}+4\right)}\mathrm{\,d}x\\
&=&\pi\left.\left(\dfrac{x^2}2-\dfrac{8x\sqrt{x}}3+4x\right)\right|_4^{9}\\
&=&\pi\left(\dfrac{81}{2}-72+36\right)-\pi\left(\dfrac{16}{2}-\dfrac{64}{3}+16\right)\\
&=&\dfrac{11\pi}{6}\approx 5{,}76.
\end{eqnarray*}
}
\end{ex}

\begin{ex}%[Vovanle]%[2D4V3-1]
\immini{Cho hàm số $y=ax^4+bx^2+c$ có đồ thị $(C)$, biết rằng $(C)$ đi qua điểm $A(-1;0)$, tiếp tuyến $d$ tại $A$ của $(C)$, cắt $(C)$ tại hai điểm có hoành độ lần lượt là $0$ và $2$. Diện tích hình phẳng giới hạn bởi $d$, đồ thị $(C)$ và hai đường thẳng $x=0$; $x=2$ có diện tích bằng $\dfrac{28}{5}$ (phần gạch sọc trong hình vẽ).
 
Tính diện tích hình phẳng giới hạn bởi $(C)$, trục hoành và hai đường thẳng $x=-1$; $x=0$.
}{
\begin{tikzpicture}[line join=round,line cap=round, font=\footnotesize,scale=0.5,>=stealth]
\draw[-stealth](-2,0)--(2.5,0)node[below]{$x$};
	\draw[-stealth](0,-0.7)--(0,7)node[right]{$y$};	
\fill[pattern=north east lines]plot[domain=0:2](\x,{(\x)^4-3*(\x)^2+2})--cycle;
\draw[smooth,samples=300,domain=-2.02:2.02] plot(\x,{(\x)^4-3*(\x)^2+2});
\draw[smooth,samples=300,domain=-1.5:2.3] plot(\x,{2*(\x+1)});		
	\fill (0,0) circle(1pt)node[below right]{$O$}(-1,0)circle(1pt)+(0.1,0) node[below]{$-1$}(2,0)circle(1pt)node[below]{$2$};
	\draw[dashed] (2,0)--(2,6);		
	\end{tikzpicture}
}	
\shortans{$0{,}2$}
	\loigiai{
Ta có $y'=4ax^3+2bx$ $\Rightarrow d\colon y=\left(-4a-2b\right)\left(x+1\right)$.
Phương trình hoành độ giao điểm của $d$ và $(C)$ là $\left(-4a-2b\right)\left(x+1\right)=ax^4+bx^2+c.\hfill(1)$\\
Phương trình $(1)$ phải cho $2$ nghiệm là $x=0$, $x=2$.
$$\Rightarrow\heva{&-4a-2b=c\\&-12a-6b=16a+4b+c}
\Leftrightarrow \heva{&-4a-2b-c=0&(2)\\&28a+10b+c=0&(3).}$$
Mặt khác, diện tích phần gạch sọc là 
\allowdisplaybreaks
\begin{eqnarray*}
&&\dfrac{28}{5}=\displaystyle\int\limits_0^2{\left[\left(-4a-2b \right)\left( x+1 \right)-ax^4-bx^2-c\right]\mathrm{\,d}x}\\
&\Leftrightarrow& \dfrac{28}{5}=4\left(-4a-2b\right)-\dfrac{32}{5}a-\dfrac{8}3b-2c\\
&\Leftrightarrow& \dfrac{112}{5}a+\dfrac{32}3b+2c=-\dfrac{28}{5}\qquad(4)
\end{eqnarray*}
Giải hệ 3 phương trình $(2)$, $(3)$ và $(4)$ ta được $a=1$, $b=-3$, $c=2$.\\
Khi đó, $(C)\colon y=x^4-3x^2+2$, $d\colon y=2\left(x+1\right)$.\\
Diện tích cần tìm là 
$$S=\displaystyle\int\limits_{-1}^0\left[x^4-3x^2+2-2\left(x+1\right)\right]\mathrm{\,d}x=\displaystyle\int\limits_{-1}^0\left(x^4-3x^2-2x\right)\mathrm{\,d}x=\dfrac1{5}=0{,}2.$$
}
\end{ex}

\begin{ex}%[Vovanle]%[2D4V3-2]
\immini{Một khuôn viên dạng nửa hình tròn có đường kính bằng $4\sqrt{5}$ (m). Trên đó người thiết kế hai phần để trồng hoa có dạng của một cánh hoa hình parabol có đỉnh trùng với tâm nửa hình tròn và hai đầu mút của cánh hoa nằm trên nửa đường 
}{
\begin{tikzpicture}[line join=round,line cap=round, font=\footnotesize,scale=0.5,>=stealth]
\path
({-2*sqrt (5)},0) coordinate (A)
({2*sqrt (5)},0) coordinate (B)
(2,4) coordinate (M)
(-2,4) coordinate (N)
(-2,0) coordinate (C)
(2,0) coordinate (D)
($(M)!0.5!(D)$) coordinate (G)node[right]{$4$ m}
;
\fill[pattern=north east lines]plot[domain=-2:2](\x,{sqrt (20-(\x)^2)})--plot[domain=2:-2](\x,{(\x)^2});	
	\draw (A) arc(180:0:{2*sqrt (5)});
	\draw plot[domain=-2:2](\x,{(\x)^2});
	\draw[dashed] (N)--(C)(M)--(D)(M)--(N);
	\path ($(M)!0.5!(N)$) coordinate (H)node[below]{$4$ m};
	\draw (A)--(B);			
	\end{tikzpicture}
}
\noindent tròn (phần gạch sọc), cách nhau một khoảng bằng $4\,\mathrm{m}$, phần còn lại của khuôn viên (phần không gạch sọc) dành để trang trí cỏ nhân tạo. Biết các kích thước cho như hình vẽ và kinh phí cỏ nhân tạo là $100\,000$ đồng/m$^2$. Hỏi cần bao nhiêu tiền để trang trí cỏ trên phần đất đó? (Số tiền được làm tròn đến hàng nghìn).	
\shortans{$1948$}
	\loigiai{
\immini{Đặt hệ trục tọa độ như hình vẽ. Khi đó phương trình nửa đường tròn là
$$y=\sqrt{R^2-x^2}=\sqrt{\left(2\sqrt{5}\right)^2-x^2}=\sqrt{20-x^2}.$$
Phương trình parabol $(P)$ có đỉnh là gốc $O$ sẽ có dạng $y=ax^2$. Mặt khác $(P)$ qua điểm $M(2;4)$. 
}{
\begin{tikzpicture}[line join=round,line cap=round, font=\footnotesize,scale=0.6,>=stealth]
\path
({-2*sqrt (5)},0) coordinate (A)
({-2*sqrt (5)},0) coordinate (B)
(2,4) coordinate (M)node[above right]{$M(2,4)$}
(-2,4) coordinate (N)
(-2,0) coordinate (C)
(2,0) coordinate (D)
;
\fill[pattern=north east lines]plot[domain=-2:2](\x,{sqrt (20-(\x)^2)})--plot[domain=2:-2](\x,{(\x)^2});
	\draw[-stealth](-5,0)--(5,0)node[below]{$x$};
	\draw[-stealth](0,-0.7)--(0,5)node[right]{$y$};	
	\fill (0,0) circle(1pt)node[below left]{$O$}(-2,0)circle(1pt) node[below]{$-2$}(2,0)circle(1pt)node[below]{$2$}(0,4)circle(1pt)node[below left]{$4$};
	\draw (A) arc(180:0:{2*sqrt (5)});
	\draw plot[domain=-2:2](\x,{(\x)^2});
	\draw[dashed] (N)--(C)(M)--(D)(M)--(N);				
	\end{tikzpicture}
}	
\noindent Do đó $4=a\cdot(-2)^2\Rightarrow a=1$.\\
Phần diện tích của hình phẳng giới hạn bởi $(P)$ và nửa đường tròn.(phần gạch sọc).\\
Ta có công thức $S_1=\displaystyle\int\limits_{-2}^2{\left(\sqrt{20-x^2}-x^2 \right)\mathrm{\,d}x}\approx 11{,}94\,\mathrm{m}^2$.\\
Vậy phần diện tích trồng cỏ là $S_{\text{cỏ}}=\dfrac{1}{2}{{S}_{\text{htron}}}-S_1=\dfrac{1}{2}\cdot 20\cdot \pi-11{,}94\approx 19{,}476\,\mathrm{m}^2$.\\
Số tiền cần có là $S_{\text{cỏ}}\times 100000\approx 1947592\text{ (đồng)}\approx 1948$ (nghìn đồng).
}
\end{ex}

\begin{ex}%[Vovanle]%[2D4V3-4]
Một téc nước hình trụ, đang chứa nước được đặt nằm ngang, có chiều dài $3$ m và đường kính đáy $1$ m. Hiện tại mặt nước trong téc cách phía trên đỉnh của téc $0{,}25$ m (xem hình vẽ). 
\begin{center}
\begin{tikzpicture}[line join=round,line cap=round, font=\footnotesize,scale=1,>=stealth]
\def \x{0.5}
\def \y{1.5}
\def \z{6}
\path
(80:{\x} and {\y}) coordinate (A)
(80:{\x} and {\y})+(\z,0) coordinate (B)
(-80:{\x} and {\y}) coordinate (C)
(-80:{\x} and {\y})+(\z,0) coordinate (D)
(10:{\x} and {\y}) coordinate (E)
(200:{\x} and {\y}) coordinate (F)
($(B)!0.5!(D)$) coordinate (K)
($(A)!0.5!(C)$) coordinate (H)
(40:{\x} and {\y}) coordinate (M)
(160:{\x} and {\y}) coordinate (N)
(40:{\x} and {\y})+(\z,0) coordinate (U)
(160:{\x} and {\y})+(\z,0) coordinate (V)
(C)+(0,-0.7) coordinate (I)
(D)+(0,-0.7) coordinate (J)
(B)+(2,0) coordinate (P)
(D)+(2,0) coordinate (Q)
(intersection of B--D and U--V) coordinate (T)
(B)+(0.6,0) coordinate (G)
(T)+(0.6,0) coordinate (R)
($(P)!0.5!(Q)$) coordinate (m)node[right]{$1$ m}
($(I)!0.5!(J)$) coordinate (X)node[above]{$3$ m}
($(G)!0.5!(R)$) coordinate (Z)node[right]{$0{,}25$ m}
;
\fill[blue!20] plot [domain=-90:-200] (0.5*cos \x,{1.5*sin \x})--(M)--(U)--plot [domain=40:-90] (\z+0.5*cos \x,{1.5*sin \x})--cycle;
\draw  (A) arc (80:280:{\x} and {\y});
\draw[dashed] (C) arc (-80:80:{\x} and {\y});
\draw  (B) arc (80:280:{\x} and {\y});
\draw (D) arc (-80:80:{\x} and {\y});
\draw (A)--(B)(C)--(D)(U)--(V)(N)--(V);
\draw[<->](I)--(J);
\draw[<->](P)--(Q);
\draw[<->](G)--(R);
\draw[dashed](C)--(I)(D)--(J)(B)--(P)(C)--(Q)(M)--(N)(M)--(U)(T)--(R);

\end{tikzpicture}
\end{center}
Tính thể tích của nước trong téc (kết quả làm tròn đến hàng phần trăm)?
\shortans{$1{,}9$}
	\loigiai{
\immini{Thế tích phần dầu còn lại sẽ bằng diện tích hình phẳng gạch sọc trong hình nhân với chiều dài của bồn (chiều cao của trụ).
 
Đường tròn có tâm $O(0;0)$, $R=0{,}5$ có phương trình là 
$$x^2+y^2=0{,}25 \Leftrightarrow y=\pm \sqrt{0{,}25-x^2}.$$
 Diện tích hình gạch sọc chính là diện tích hình phẳng giới hạn bởi các đường 
 $$y=\sqrt{0{,}25-x^2};\,y=-\sqrt{0{,}25-x^2};\,x=-0{,}5;\,x=0{,}25.$$
Do đó 
$$V=Sh=3 \displaystyle\int_{-0{,}5}^{0{,}25}\left|\sqrt{0{,}25-x^2}-\left(-\sqrt{0{,}25-x^2}\right)\right|\mathrm{\,d}x \approx 1{,}896\mathrm{\,m}^3 \approx 1{,}9\mathrm{\,m}^3.$$
}{
\begin{tikzpicture}[line join=round,line cap=round, font=\footnotesize,scale=1,>=stealth]
\def \r{1.5}
\fill[pattern=north east lines]plot[domain=60:300](\r*cos \x,\r*sin \x)--cycle;
	\draw[-stealth](-2,0)--(2.5,0)node[below]{$x$};
	\draw[-stealth](0,-2)--(0,2)node[right]{$y$};	
	\fill (0,0) circle(1pt)node[below left]{$O$}(\r,0)circle(1pt)+(-0.1,0.2) node[right]{$0{,}5$}(-\r,0)circle(1pt)+(0.1,0.2) node[left]{$-0{,}5$}(0,-\r)circle(1pt)+(0.1,-0.2) node[left]{$-0{,}5$}(0,\r)circle(1pt)+(0.1,0.2) node[left]{$0{,}5$}(0.5*\r,0)circle(1pt)+(-0.1,-0.2) node[right]{$0{,}25$};
	\draw (60:\r)--(-60:\r);
	\draw (0,0) circle(\r);			
	\end{tikzpicture}
}	
}
\end{ex}

\begin{ex}%[Vovanle]%[2D4V3-1]
\immini{Cho hai đường tròn $\left(O_1;5\right)$ và $\left(O_2;3\right)$ cắt nhau tại hai điểm $A$, $B$ sao cho $AB$ là một đường kính của đường tròn $\left(O_2\right)$. Gọi $(D)$ là hình thẳng được giới hạn bởi hai đường tròn (phần ở ngoài đường tròn lớn, được gạch chéo như hình vẽ). Một vật trang trí có dạng một khối tròn xoay được tạo thành khi quay miền $(D)$ quanh trục $O_1O_2$. Thể tích của khối tròn xoay được tạo thành có $V=\dfrac{a\pi}{b}$ ($\dfrac{a}{b}$ là phân số tối giản) thì $a^2+b^3$ bằng bao nhiêu?
}{
\begin{tikzpicture}[line join=round,line cap=round, font=\footnotesize,scale=1,>=stealth]
\def \r{0.3}
 \path
    (-4*\r,0) coordinate (O_1)
    (0,0) coordinate (O_2)
    (90:3*\r) coordinate (A)
    (-90:3*\r) coordinate (B)    
    ;
    \pgfmathsetmacro\g{atan (3/4)}
    \fill[pattern=north east lines]plot[domain=-\g:\g](5*\r*cos \x-4*\r,5*\r*sin \x)--plot[domain=90:-90](3*\r*cos \x,3*\r*sin \x);	
	\draw (O_1) circle(5*\r);
	\draw (O_2) circle(3*\r);
	\draw (2*\r,0) coordinate (D)node[above]{$(D)$};
	\draw (A)--(B)(-9*\r,0)--(3*\r,0);
	\foreach \x/\g in {O_1/90,O_2/140,A/60,B/-60}\fill[black] (\x) circle (1pt)+(\g:.3)node{$\x$};		
	\end{tikzpicture}
}
\shortans{$1627$}
	\loigiai{
\immini{Chọn hệ tọa độ $Oxy$ với \\
$O_2\equiv O$, $O_2C\equiv Ox$, $O_2A\equiv Oy$.\\
Đoạn $O_1O_2=\sqrt{O_1A^2-O_2A^2}=\sqrt{5^2-3^2}=4$.\\
Suy ra $\left(O_1\right):{{\left( x+4 \right)}^2}+y^2=25$.\\
Kí hiệu $\left(H_1\right)$ là hình phẳng giới hạn bởi các đường $\left(O_1\right)\colon \left(x+4\right)^2+y^2=25$, $Oy\colon x=0$, $x\geq 0$.\\
Kí hiệu $\left(H_2\right)$ là hình phẳng giới hạn bởi các đường $\left(O_2\right)\colon x^2+y^2=9$, $Oy\colon x=0$, $x\geq 0$.
}{
\begin{tikzpicture}[line join=round,line cap=round, font=\footnotesize,scale=1,>=stealth]
\def \r{0.4}
 \path
    (-4*\r,0) coordinate (O_1)
    (0,0) coordinate (O_2)
    (90:3*\r) coordinate (A)
    (-90:3*\r) coordinate (B)    
    ;
    \pgfmathsetmacro\g{atan (3/4)}
    \fill[pattern=north east lines]plot[domain=-\g:\g](5*\r*cos \x-4*\r,5*\r*sin \x)--plot[domain=90:-90](3*\r*cos \x,3*\r*sin \x);
	\draw[-stealth](-9.5*\r,0)--(4*\r,0)node[below]{$x$};
	\draw[-stealth](0,-5.5*\r)--(0,5.5*\r)node[right]{$y$};		
	\draw (O_1) circle(5*\r);
	\draw (O_2) circle(3*\r);
	\draw (2*\r,0) coordinate (D)node[above]{$(D)$};
	\foreach \x/\g in {O_1/90,O_2/140,A/60,B/-60}\fill[black] (\x) circle (1pt)+(\g:.3)node{$\x$};		
	\end{tikzpicture}
}
\noindent Khi đó thể tích $V$ cần tìm chính bằng thể tích $V_2$ của khối tròn xoay thu được khi quay hình $\left(H_2\right)$ xung quanh trục $Ox$ trừ đi thể tích $V_1$ của khối tròn xoay thu được khi quay hình $\left(H_1\right)$ xung quanh trục $Ox$.\\
Ta có $V_2=\dfrac{1}{2}\cdot \dfrac{4}{3}\pi r^3=\dfrac{2}{3}\pi {\cdot 3^3}=18\pi$.\\
Lại có $V_1=\pi\displaystyle\int\limits_0^1y^2\mathrm{\,d}x=\pi\displaystyle\int\limits_0^1\left[25-\left(x+4\right)^2\right]\mathrm{\,d}x=\left.\pi \left[25x-\dfrac{\left(x+4\right)^3}{3}\right]\right|_0^1
=\dfrac{14\pi}{3}$.\\
Do đó $V=V_2-V_1=18\pi-\dfrac{14\pi}{3}=\dfrac{40\pi}{3}$.\\
Vậy $a^2+b^3=1627$.
}
\end{ex}
\Closesolutionfile{ans}
% \indapan{6}{ans/ans-2-B13-De2-KQ}
