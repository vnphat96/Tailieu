\setcounter{chapter}{1}% Đặt lại đếm chương
\chap{Mũ và Logarit}
\section{LŨY THỪA}
\subsection{Kiến thức sách giáo khoa cần cần nắm}
\Opensolutionfile{ans}[ans/ansCD2D2-1]
\subsubsection{LŨY THỪA VỚI SỐ MŨ NGUYÊN DƯƠNG}
\hspace*{5cm} $a^n=a\cdot a\cdots\cdot a,$ ($n$ thừa số).\\
Ở đây $n\in\mathbb{Z}^+, n>1$. Quy ước $a^1=a$.
\subsubsection{LŨY THỪA VỚI SỐ MŨ 0 – LŨY THỪA VỚI SỐ MŨ NGUYÊN ÂM}
\hspace*{5cm} $a^{0}=1(a\neq 0)$; $a^{-n}=\dfrac{1}{a^n}(a\neq 0)$, với $n\in\mathbb{Z}^+$.
\subsubsection{LŨY THỪA VỚI SỐ MŨ HỮU TỈ}
\hspace*{5cm} $a^{\tfrac{m}{n}}=\sqrt[n]{a^m} $ với $m,n \in \mathbb{Z}, n \geq 2.$\\
Lũy thừa số mũ hữu tỷ có tính chất như lũy thừa số mũ nguyên (xem mục 5).
\subsubsection{LŨY THỪA SỐ THỰC}
\hspace*{5cm}  $a^{\alpha}=\lim a^{r_n}$ (với $\alpha$ là số vô tỉ, $r_n$ là số hũy tỉ và $\lim r_n=\alpha$).\\
Lũy thừa số mũ thực có tính chất như lũy thừa số mũ nguyên (xem mục 5).
\subsubsection{TÍNH CHẤT CỦA LŨY THỪA VỚI SỐ MŨ NGUYÊN}
\begin{enumerate}[a)]
	\item Với $a,b \in \mathbb{R};a \neq 0,b \neq 0;m,n \in \mathbb{Z} $, ta có\\
	$a^m\cdot a^n=a^{m+n};\,\dfrac{a^m}{a_n}=a^{m-n};\,\left(a^m\right)^n=a^{mn};\,(ab)^m=a^mb^m;\,\left(\dfrac{a}{b}\right)^m=\dfrac{a^m}{b^m} $.
	\item Nếu $0<a<b \Rightarrow \heva{&a^n<b^n,\forall n>0\\& a^n>b^n,\forall n<0.} $\\
	Nếu $a>1\Rightarrow a^m>a^n$ với $m>n$.\\
	Nếu $0<a<1 \Rightarrow a^m<a^{r} $ với  $m>n$.
\end{enumerate}
\subsubsection{CÔNG THỨC LÃI KÉP}
\textbf{1) Định nghĩa:} Lãi kép là phần lãi của kì sau được tính trên số tiền gốc kì trước cộng với phần lãi của kì trước.\\
\textbf{2) Công thức:} Giả sử số tiền gốc là $A$ lãi suất $r\%$ kì hạn gửi (có thể là tháng, quý hay năm).
\begin{itemize}
	\item Số tiền nhận được cả gốc và lãi sau $n$ kì hạn gửi là $A(1+r)^n$.
	\item Số tiền lãi nhận được sau $n$ kì hạn gửi là $A(1+r)^n-A=A\left[(1+r)^n-1\right]$.
\end{itemize}
\subsection{Phân loại và phương pháp giải bài tập}
\begin{dang}{Các bài toán liên quan đến lí thuyết}
\end{dang}
\begin{ex}%[2D2Y1-1]%Câu 1.
	Cho các số thực $a,b,m,n$ với $(a,b>0)$. Tìm mệnh đề \textbf{sai}. 
	\choice
	{$\sqrt{a^2}=a$}
	{$\left(\dfrac{a}{b}\right)^m=a^mb^{-m}$}
	{\True $(a^m)^n=a^{m+n}$}
	{$(ab)^m=a^mb^m$}
	\loigiai{
		Đáp án $(a^m)^n=a^{m+n}$ sai.
	}
\end{ex}
\begin{ex}%[2D2Y1-1]%Câu 2.
	Cho $x,y$ là hai số thực dương và $m,n$ là hai sô thực tùy ý. Đẳng thức nào sau đây \textbf{sai}?
	\choice
	{$(xy)^m=x^my^m$}
	{$(x^n)^n=x^{nm}$}
	{$x^mx^n=x^{m+n}$}
	{\True $x^my^n=(xy)^{m+n}$}
	\loigiai{
		Đáp án $x^my^n=(xy)^{m+n}$ sai.}
\end{ex}
\begin{ex}%[2D2Y1-1]%Câu 3.
	Cho các số thực $a,b,\alpha (a>b>0,\alpha\neq 1)$. Mệnh đề nào sau đây \textbf{đúng}?
	\choice
	{\True $(ab)^{\alpha}=a^{\alpha}b^{\alpha}$}
	{$(a-b)^{\alpha}=a^{\alpha}-b^{\alpha}$}
	{$\left(\dfrac{a}{b}\right)^{\alpha}=\dfrac{a^{\alpha}}{b^{-\alpha}}$}
	{$(a+b)^{\alpha}=a^{\alpha}+b^{\alpha}$}
	\loigiai{Đáp án đúng $(ab)^{\alpha}=a^{\alpha}b^{\alpha}$.}
\end{ex}
\begin{ex}%[2D2Y1-1]%Câu 4.
	Cho $a, b$ là các số thực dương và $m, n$ là hai số thực tùy ý. Đẳng thức nào sau đây là \textbf{sai}. 
	\choice
	{$(xy)^n=x^n\cdot y^n$}
	{$x^m\cdot x^n=x^{m+n}$}
	{$(x^m)^n=x^{m\cdot n}$}
	{\True $x^m\cdot y^n=(xy)^{m+n}$}
	\loigiai{
		Ta có $(xy)^{m+n}=x^{m+n}\cdot y^{m+n}$.}
\end{ex}
\begin{ex}%[2D2Y1-1]%Câu 5.
	Cho $a>0;b>0;\alpha,\beta\in\mathbb{R}$. Hãy chọn công thức \textbf{đúng} trong các công thức sau: 
	\choice
	{\True $a^{\alpha+\beta}=a^{\alpha}\cdot a^{\beta}$}
	{$\left(\dfrac{a}{b}\right)^{\alpha}=a^{\alpha}-b^{\alpha}$}
	{$(ab)^{\alpha}=a^{\alpha}+b^{\alpha}$}
	{$\left(a^{\alpha}\right)^{\beta}=a^{\alpha+\beta}$}
	\loigiai{Đáp án đúng $a^{\alpha+\beta}=a^{\alpha}\cdot a^{\beta}$. }
\end{ex}

\begin{ex}%[2D2Y1-1]%Câu 6.
	Cho $x,y$ là hai số thực dương và $m,n$ là hai số thực tùy ý. Đẳng thức nào sau đây là \textbf{sai}?
	\choice
	{$(x^n)^m=x^{n\cdot m}$}
	{$x^m\cdot x^n=x^{m+n}$}
	{\True $\dfrac{x^m}{y^n}=\left(\dfrac{x}{y}\right)^{m-n}$}
	{$(xy)^n=x^n\cdot y^n$}
	\loigiai{Đáp án sai là $\dfrac{x^m}{y^n}=\left(\dfrac{x}{y}\right)^{m-n}$.}
\end{ex}
\begin{ex}%[2D2Y1-1]%Câu 7.
	Cho $a, b>0; m, n\in \mathbb{N}*$. Hãy tìm khẳng định \textbf{đúng}?
	\choice
	{$\sqrt[n]{a^m}=a^{\tfrac{m}{n}}$}
	{$a^n\colon b^m=(a\colon b)^{m-n}$}
	{$\sqrt[n]{\sqrt[k]{a}}=\sqrt[n+k]{a}$}
	{$a^n+b^n=(a\cdot b)^n$}
	\loigiai{Đáp án đúng là $\sqrt[n]{a^m}=a^{\tfrac{m}{n}}$.}
\end{ex}
\begin{ex}%[2D2Y1-1]%Câu 8.
	Cho các số thực $a,b,\alpha\left(a>b>0,\alpha\neq 1\right)$. Mệnh đề nào sau đây \textbf{đúng}?
	\choice
	{$(a+b)^{\alpha}=a^{\alpha}+b^{\alpha}$}
	{$\left(\dfrac{a}{b}\right)^{\alpha}=\dfrac{a^{\alpha}}{b^{-\alpha}}$}
	{$(a-b)^{\alpha}=a^{\alpha}-b^{\alpha}$}
	{\True $(ab)^{\alpha}=a^{\alpha}\cdot b^{\alpha}$}
	\loigiai{Đáp án đúng là $(ab)^{\alpha}=a^{\alpha}\cdot b^{\alpha}$.}
\end{ex}
\begin{ex}%[2D2B1-1]%Câu 9.
	Biểu thức $(a+2)^{\pi}$ có nghĩa với: 
	\choice
	{\True $a >-2$}
	{$\forall a\in\mathbb{R}$}
	{$a>0$}
	{$a <-2$}
	\loigiai{
		$(a+2)^{\pi}$ có nghĩa khi $a+2>0\Leftrightarrow a >-2$. Vậy đáp án $a >-2$ đúng.}
\end{ex}
\begin{ex}%[2D2B1-1]%Câu 10.
	Cho $n\in \mathbb{N};n\geq 2$ khẳng định nào sau đây đúng?
	\choice
	{$a^{\tfrac{1}{n}}=\sqrt[n]{a}$, $\forall a\neq 0$}
	{\True $a^{\tfrac{1}{n}}=\sqrt[n]{a}$, $\forall a>0$}
	{$a^{\tfrac{1}{n}}=\sqrt[n]{a}$, $\forall a\geq 0$}
	{$a^{\tfrac{1}{n}}=\sqrt[n]{a}$, $\forall a\in\mathbb{R}$}
	\loigiai{
		Đáp án $a^{\tfrac{1}{n}}=\sqrt[n]{a}$, $\forall a>0$ đúng. Đáp án còn lại sai vì điều kiện của $a$.}
\end{ex}
\begin{ex}%[2D2B1-1]%Câu 11.
	Khẳng định nào sau đây là khẳng định \textbf{sai}?
	\choice
	{\True $\sqrt{ab}=\sqrt{a}\sqrt{b}\,\,\forall a,b$}
	{$\sqrt[2n]{a^{2n}}\geq 0 \,\,\forall a$, $n$ nguyên dương $(n\geq 2)$}
	{$\sqrt[2n]{a^{2n}}=|a|\,\,\forall a$, $n$ nguyên dương $(n\geq 2)$}
	{$\sqrt[4]{a^2}=\sqrt{a}\,\,\forall a\geq 0$}
	\loigiai{
		$\sqrt{ab}=\sqrt{a}\sqrt{b}$ đúng khi $\forall a,b\geq 0$.}
\end{ex}
\begin{dang}{Thực hiện phép tính, thu gọn biểu thức}
\end{dang}
\begin{ex}%[2D2B1-2] %Câu 12.
	Rút gọn biểu thức $P=\dfrac{\sqrt{a}+\sqrt[4]{ab}}{\sqrt[4]{a}+\sqrt[4]{b}}-\dfrac{\sqrt{a}-\sqrt{b}}{\sqrt[4]{a}-\sqrt[4]{b}}$ với $a>0,b>0$ 
	\choice
	{$P=2\sqrt[4]{a}-\sqrt[4]{b}$}
	{\True $P=-\sqrt[4]{b}$}
	{$P=\sqrt[4]{b}$}
	{$P=\sqrt[4]{a}$}
	\loigiai{
		Ta có $P=\dfrac{\sqrt{a}+\sqrt[4]{ab}}{\sqrt[4]{a}+\sqrt[4]{b}}-\dfrac{\sqrt{a}-\sqrt{b}}{\sqrt[4]{a}-\sqrt[4]{b}}=\dfrac{\left(\sqrt[4]{a}\right)^2+\sqrt[4]{ab}}{\sqrt[4]{a}+\sqrt[4]{b}}-\dfrac{\left(\sqrt[4]{a}\right)^2-\left(\sqrt[4]{b}\right)^2}{\sqrt[4]{a}-\sqrt[4]{b}}$.\\
		$=\dfrac{\sqrt[4]{a}\left(\sqrt[4]{a}+\sqrt[4]{b}\right)}{\sqrt[4]{a}+\sqrt[4]{b}}-\dfrac{\left(\sqrt[4]{a}-\sqrt[4]{b}\right)\left(\sqrt[4]{a}+\sqrt[4]{b}\right)}{\sqrt[4]{a}-\sqrt[4]{b}}=\sqrt[4]{a}-\left(\sqrt[4]{a}+\sqrt[4]{b}\right)=-\sqrt[4]{b}$.}
\end{ex}
\begin{ex}%[2D2B1-2]%Câu 13.
	Rút gọn biểu thức $P=x^{\tfrac{1}{3}}\cdot\sqrt[6]{x}$ với $x>0$ 
	\choice
	{$P=x^2$}
	{\True $P=\sqrt{x}$}
	{$P=x^{\tfrac{1}{3}}$}
	{$P=x^{\tfrac{1}{9}}$}
	\loigiai{
		Ta có $P=x^{\tfrac{1}{3}}\cdot\sqrt[6]{x}=x^{\tfrac{1}{3}}\cdot x^{\tfrac{1}{6}}=x^{\tfrac{1}{3}+\tfrac{1}{6}}=x^{\tfrac{1}{2}}$.\\
		Vì $x>0$ nên $x^{\tfrac{1}{2}}=\sqrt{x}$.}
\end{ex}
\begin{ex}%[2D2B1-2]%Câu 14.
	Rút gọn biểu thức $P=\sqrt[3]{x^{5}\sqrt[4]{x}}$ với $x>0$ 
	\choice
	{$P=x^{\tfrac{20}{21}}$}
	{\True $P=x^{\tfrac{21}{12}}$}
	{$P=x^{\tfrac{20}{5}}$}
	{$P=x^{\tfrac{12}{5}}$}
	\loigiai{
		\textbf{Cách 1:} 
		$P=\sqrt[3]{x^{5}\sqrt[4]{x}}=\sqrt[3]{x^{5+\tfrac{1}{4}}}=\left(x^{\tfrac{21}{4}}\right)^{\tfrac{1}{3}}=x^{\tfrac{21}{12}}$.\\
		\textbf{Cách 2:} sử dụng CASIO. Chọn $x>0$ ví dụ như $x=1{,}25$ chẳng hạn.\\
		Nhập $P$ rồi Calc $x=1{,}25$, sau đó lưu vào $A$.		
		Tiếp theo ta tính hiệu, ví dụ như đáp án $P=x^{\tfrac{20}{21}}$ ta cần tính $A-(1,25)^{\tfrac{20}{21}}$. Nếu màn hình máy tính xuất hiện kết quả bằng 0 thì chứng tỏ đáp án này là đúng.}
\end{ex}
\begin{ex}%[2D2B1-2]%Câu 15.
	Rút gọn biểu thức $P=\dfrac{a^{\sqrt{3}+1}\cdot a^{2-\sqrt{6}}}{\left(a^{\sqrt{2}-2}\right)^{\sqrt{2}+2}}$ với $a>0$. 
	\choice
	{$P=a^4$}
	{$P=a$}
	{\True $P=a^5$}
	{$P=a^3$}
	\loigiai{
		Ta có $\heva{&a^{\sqrt{3}+1}\cdot a^{2-\sqrt{3}}=a^{\sqrt{3}+1+(2-\sqrt{3})}=a^3\\&\left(a^{\sqrt{2}-2}\right)^{\sqrt{2}+2}=a^{(\sqrt{2}-2)(\sqrt{2}+2)}=a^{2-4}=a^{-2}} \Rightarrow P=\dfrac{a^3}{a^{-2}}=a^{3-(-2)}=a^5$.}
\end{ex}
\begin{ex}%[2D2B1-2]%Câu 16.
	Rút gọn biểu thức $K=\left(x^{\tfrac{1}{2}}-y^{\tfrac{1}{2}}\right)^2\left(1-2\sqrt{\dfrac{y}{x}}+\dfrac{y}{x}\right)^{-1}$ với $x>0,y>0$. 
	\choice
	{\True $K=x$}
	{$K=2x$}
	{$K=x+1$}
	{$K=x-1$}
	\loigiai{
		Rút gọn $\left(x^{\tfrac{1}{2}}-y^{\tfrac{1}{2}}\right)^2=\left(\sqrt{x}-\sqrt{y}\right)^2$.\\
		Rút gọn $\left(1-2\sqrt{\dfrac{y}{x}}+\dfrac{y}{x}\right)^{-1}=\left[\left(\sqrt{\dfrac{y}{x}}-1\right)^2\right]^{-1}=\left(\dfrac{\sqrt{y}-\sqrt{x}}{\sqrt{x}}\right)^{-2}=\left(\dfrac{\sqrt{x}}{\sqrt{y}-\sqrt{x}}\right)^2$.\\
		Vậy $K=\left(\sqrt{x}-\sqrt{y}\right)^2\left(\dfrac{\sqrt{x}}{\sqrt{y}-\sqrt{x}}\right)^2=x$.}
\end{ex}
\begin{ex}%[2D2B1-2]%Câu 17.
	Cho $P=\left(x^{\tfrac{1}{2}}-y^{\tfrac{1}{2}}\right)^2\left(1-2\sqrt{\dfrac{y}{x}}+\dfrac{y}{x}\right)^{-1}$. Biểu thức rút gọn của $P$ là 
	\choice
	{\True $x$}
	{$x+y$}
	{$x-y$}
	{$2x$}
	\loigiai{
		$P=\left(x^{\tfrac{1}{2}}-y^{\tfrac{1}{2}}\right)^2\left(1-2\sqrt{\dfrac{y}{x}}+\dfrac{y}{x}\right)^{-1}=(\sqrt{x}-\sqrt{y})^2\left[\left(\dfrac{\sqrt{x}-\sqrt{y}}{\sqrt{x}}\right)^2\right]^{-1}=x$.}
\end{ex}
\begin{ex}%[2D2B1-2]%Câu 18.
	Viết biểu thức $P=\sqrt[3]{x\sqrt[4]{x}}$ ($x>0$) dưới dạng luỹ thừa với số mũ hữu tỷ. 
	\choice
	{$P=x^{\tfrac{5}{4}}$}
	{\True $P=x^{\tfrac{5}{12}}$}
	{$P=x^{\tfrac{1}{7}}$}
	{$P=x^{\tfrac{1}{12}}$}
	\loigiai{
		Ta có $P=\left(x\cdot x^{\tfrac{1}{4}}\right)^{\tfrac{1}{3}}=\left(x^{\tfrac{5}{4}}\right)^{\tfrac{1}{3}}=x^{\tfrac{5}{12}}$.}
\end{ex}
\begin{ex}%[2D2B1-2]%Câu 19.
	Cho biểu thức $P=\sqrt[6]{x\sqrt[4]{x^5\cdot\sqrt{x^3}}}$ với $x>0$. Mệnh đề nào dưới đây \textbf{đúng}?
	\choice
	{$P=x^{\tfrac{15}{16}}$}
	{\True $P=x^{\tfrac{7}{16}}$}
	{$P=x^{\tfrac{5}{42}}$}
	{$P=x^{\tfrac{47}{48}}$}
	\loigiai{
		$P=\sqrt[4]{x\cdot \sqrt[3]{x^2\cdot \sqrt{x^3}}}=\sqrt[4]{x\sqrt[3]{x^2\cdot x^{\tfrac{3}{2}}}}=\sqrt[4]{x\sqrt[3]{x^{\tfrac{7}{2}}}}=\sqrt[4]{x\cdot x^{\tfrac{7}{6}}}=\sqrt[4]{x^{\tfrac{13}{6}}}=x^{\tfrac{13}{24}} $.}
\end{ex}
\begin{ex}%[2D2B1-2]%Câu 20.
	Cho biểu thức $P=\sqrt[4]{x\sqrt[3]{x^2\cdot\sqrt{x^3}}}$, với $x>0$. Mệnh đề nào dưới đây \textbf{đúng}?
	\choice
	{$P=x^{\tfrac{2}{3}}$}
	{$P=x^{\tfrac{1}{4}}$}
	{\True $P=x^{\tfrac{13}{24}}$}
	{$P=x^{\tfrac{1}{2}}$}
	\loigiai{
		Ta có $P=\sqrt[4]{x\sqrt[3]{x^2\sqrt{x^3}}}=\sqrt[4]{x}\sqrt[3]{x^2\cdot x^{\tfrac{3}{2}}}=\sqrt[4]{x\sqrt[3]{x^{\tfrac{7}{2}}}}=\sqrt[4]{x\cdot x^{\tfrac{7}{6}}}=\sqrt[4]{x^{\tfrac{13}{6}}}=x^{\tfrac{13}{24}}$.}
\end{ex}
\begin{ex}%[2D2B1-2]%Câu 21.
	Cho biểu thức $P=\sqrt[3]{x\sqrt[4]{x^3\sqrt{x}}}$, với $x>0$ Mệnh đề nào dưới đây \textbf{đúng}?
	\choice
	{$P=x^{\tfrac{1}{2}}$}
	{$P=x^{\tfrac{7}{24}}$}
	{\True $P=x^{\tfrac{15}{24}}$}
	{$P=x^{\tfrac{7}{12}}$}
	\loigiai{
		Ta có $P=\sqrt[3]{x\sqrt[4]{x^3}x^{\tfrac{1}{2}}}=\sqrt[3]{x\sqrt[4]{x^{\tfrac{7}{2}}}}=\sqrt[3]{x\cdot x^{\tfrac{7}{8}}}=\sqrt[3]{x^{\tfrac{15}{8}}}=x^{\tfrac{15}{24}}$.}
\end{ex}
\begin{ex}%[2D2B1-2]%Câu 22.
	Giá trị của $K=\left(\dfrac{1}{81}\right)^{-0\cdot 75}+\left(\dfrac{1}{27}\right)^{-\tfrac{4}{3}}$ bằng 
	\choice
	{$K=180$}
	{\True $K=108$}
	{$K=54$}
	{$K=18$}
	\loigiai{
		Hs dùng MTCT để giải.}
\end{ex}
\begin{ex}%[2D2B1-2]%Câu 23.
	Cho biểu thức $A=\sqrt[5]{a}\cdot\sqrt[4]{b}$, điều kiện xác định của biểu thức $A$ là 
	\choice
	{\True $Q$ tùy ý, $b\geq 0$}
	{$a\neq 0;b\neq 0$}
	{$Q$ tùy ý; $b>0$}
	{$a\geq 0;b\geq 0$}
	\loigiai{
		Căn bậc chẵn xác định khi biểu thức trong căn không âm.\\
		Căn bậc lẻ xác định với mọi biểu thức trong căn.}
\end{ex}
\begin{ex}%[2D2B1-2]%Câu 24.
	Cho biểu thức $P=\sqrt[6]{x\sqrt[4]{x^5\cdot\sqrt{x^3}}}$ với $x>0$. Mệnh đề nào dưới đây \textbf{đúng}?
	\choice
	{$P=x^{\tfrac{15}{16}}$}
	{\True $P=x^{\tfrac{7}{16}}$}
	{$P=x^{\tfrac{5}{42}}$}
	{$P=x^{\tfrac{47}{48}}$}
	\loigiai{
		$P=\sqrt[6]{x\sqrt[4]{x^5\cdot\sqrt{x^3}}}=x^2+\left[\left(\dfrac{1}{2}+5\right)\dfrac{1}{4}+1\right]\dfrac{1}{6}=x^{\tfrac{7}{16}}$.}
\end{ex}
\begin{ex}%[2D2B1-2]%Câu 25.
	Cho biểu thức $P=\sqrt[4]{x^5}$, với $x>0$. Mệnh đề nào dưới đây là mệnh đề \textbf{đúng}?
	\choice
	{$P=x^{\tfrac{4}{5}}$}
	{$P=x^9$}
	{$P=x^{20}$}
	{\True $P=x^{\tfrac{5}{4}}$}
	\loigiai{
		Ta có $P=\sqrt[4]{x^5} =x^{\tfrac{5}{4}}$.}
\end{ex}
\begin{ex}%[2D2B1-2]%Câu 26.
	Cho biểu thức $P=\sqrt[3]{x\sqrt[4]{x^3\sqrt{x}}}$, với $x>0$ Mệnh đề nào dưới đây \textbf{đúng}?
	\choice
	{$P=x^{\tfrac{1}{2}}$}
	{$P=x^{\tfrac{7}{24}}$}
	{\True $P=x^{\tfrac{15}{24}}$}
	{$P=x^{\tfrac{7}{12}}$}
	\loigiai{
		Ta có $P=\sqrt[3]{x\sqrt[4]{x^3}x^{\tfrac{1}{2}}}=\sqrt[3]{x\sqrt[4]{x^{\tfrac{7}{2}}}}=\sqrt[3]{x\cdot x^{\tfrac{7}{8}}}=\sqrt[3]{x^{\tfrac{15}{8}}}=x^{\tfrac{15}{24}}$.}
\end{ex}
\begin{ex}%[2D2B1-2]%Câu 27.
	Giả sử $a$ là số thực dương, khác $1$. Biểu thức $\sqrt{a\sqrt[3]{a}}$ được viết dưới dạng $a^{\alpha}$. Khi đó 
	\choice
	{\True $\alpha=\dfrac{2}{3}$}
	{$\alpha=\dfrac{5}{3}$}
	{$\alpha=\dfrac{1}{6}$}
	{$\alpha=\dfrac{11}{6}$}
	\loigiai{
		$\sqrt{a\sqrt[3]{a}}=\sqrt[1]{a^{1+\frac{1}{3}}}=\left(a^{\tfrac{4}{3}}\right)^{\tfrac{1}{2}}=a^{\tfrac{2}{3}}=a^{\alpha}\Rightarrow\alpha=\dfrac{2}{3}$.}
\end{ex}
\begin{ex}%[2D2B1-2]%Câu 28.
	Cho biểu thức $P=\sqrt[4]{x^2\sqrt[3]{x}}$, $(x>0)$. Mệnh đề nào dưới đây \textbf{đúng}?
	\choice
	{$P=x^{\tfrac{6}{12}}$}
	{$P=x^{\tfrac{8}{12}}$}
	{$P=x^{\tfrac{9}{12}}$}
	{\True $P=x^{\tfrac{7}{12}}$}
	\loigiai{
		$P=\sqrt[4]{x^2\sqrt[3]{x}}=\sqrt[4]{x^2\cdot x^{\tfrac{1}{3}}}=\left(x^{\tfrac{7}{3}}\right)^{\tfrac{1}{4}}=x^{\tfrac{7}{12}}$.}
\end{ex}
\begin{ex}%[2D2B1-2]%Câu 29.
	Cho $a>0$. Đẳng thức nào sau đây \textbf{đúng}?
	\choice
	{\True $\dfrac{\sqrt{a^3}}{\sqrt[3]{a^2}}=a^{\tfrac{5}{6}}$}
	{$\sqrt[7]{a^5}=a^{\tfrac{7}{5}}$}
	{$(a^2)^4=a^6$}
	{$\sqrt{a}\sqrt[3]{a}=\sqrt[4]{a}$}
	\loigiai{
		Xét các đáp án:\\
		$\sqrt{a}\sqrt[3]{a}=a^{\tfrac{1}{2}}\cdot a^{\tfrac{1}{3}}=a^{\tfrac{1}{2}+\dfrac{1}{3}}=a^{\tfrac{5}{6}}$ và $\sqrt[4]{a}=a^{\tfrac{1}{4}}$ nên đây là đáp án sai.\\
		$\dfrac{\sqrt{a^3}}{\sqrt[3]{a^2}}=\dfrac{a^{\tfrac{3}{2}}}{a^{\tfrac{2}{3}}}=a^{\tfrac{3}{2}-\dfrac{2}{3}}=a^{\tfrac{5}{6}}$ nên đáp án này đúng.\\
		$(a^2)^4=a^{2\cdot 4}=a^8\neq a^6$ nên đáp án này sai.\\
		$\sqrt[7]{a^5}=a^{\tfrac{5}{7}}\neq a^{\tfrac{7}{5}}$ nên đáp án này sai.\\ (Chú ý: học sinh khi làm bài sẽ kiểm tra đến đáp án B đúng thì dừng lại).}
\end{ex}
\begin{ex}%[2D2Y1-2]%Câu 30.
	Biến đổi biểu thức $P=\sqrt{x}\sqrt[3]{x}\sqrt[6]{x^5}$ $(x>0)$ thành dạng với số mũ hữu tỉ. 
	\choice
	{$P=x^{\tfrac{7}{3}}$}
	{\True $P=x^{\tfrac{5}{3}}$}
	{$P=x^{\tfrac{5}{2}}$}
	{$P=x^{\tfrac{2}{3}}$}
	\loigiai{
		$P=\sqrt{x}\cdot\sqrt[3]{x}\cdot\sqrt[6]{x^5}=x^{\tfrac{1}{2}+\tfrac{1}{3}+\tfrac{5}{6}}=x^{\tfrac{5}{3}}$.}
\end{ex}
\begin{ex}%[2D2B1-2]%Câu 31.
	Rút gọn $\left(a^{\tfrac{2}{3}}+1\right)\left(a^{\tfrac{4}{9}}+a^{\tfrac{2}{9}}+1\right)\left(a^{\tfrac{2}{9}}-1\right)$ ta được 
	\choice
	{$a^{\tfrac{1}{3}}+1$}
	{$a^{\tfrac{1}{3}}-1$}
	{$a^{\tfrac{4}{3}}+1$}
	{\True $a^{\tfrac{4}{3}}-1$}
	\loigiai{
		\begin{itemize}
			\item Tự luận: \\
			Nhân vào thu gọn, thu được kết quả.\\
			$\left(a^{\tfrac{2}{3}}+1\right)\left(a^{\tfrac{4}{9}}+a^{\tfrac{2}{9}}+1\right)\left(a^{\tfrac{2}{9}}-1\right)=\left(a^{\tfrac{2}{3}}+1\right)\left[\left(a^{\tfrac{2}{9}}\right)^3-1\right]=\left(a^{\tfrac{2}{3}}+1\right)\left(a^{\tfrac{2}{3}}-1\right)=a^{\tfrac{3}{4}}-1$.
			\item Trắc nghiệm: Dùng Casio.\\
			Nhập $\left(A^{\tfrac{2}{3}}+1\right)\left(A^{\tfrac{4}{9}}+A^{\tfrac{2}{9}}+1\right)\left(A^{\tfrac{2}{9}}-1\right)-\left(A^{\tfrac{1}{3}}+1\right)\to CALC\to A=10\to=$.\\
			Nếu kết quả nào bằng $0$ thì đúng.
		\end{itemize}
	}
\end{ex}
\begin{ex}%[2D2Y1-2]%Câu 32.
	Cho $a$ là một số dương, biểu thức $a^{\tfrac{2}{3}}\cdot\sqrt{a}$ viết dưới dạng lũy thừa với số mũ hữu tỉ là 
	\choice
	{$a^{\tfrac{6}{5}}$}
	{$a^{\tfrac{5}{6}}$}
	{$a^{\tfrac{11}{6}}$}
	{\True $a^{\tfrac{7}{6}}$}
	\loigiai{
		$a^{\tfrac{2}{3}} \cdot \sqrt{a}=a^{\tfrac{2}{3}}\cdot a^{\tfrac{1}{2}}=a^{\tfrac{7}{6}}$.}
\end{ex}
\begin{ex}%[2D2Y1-2]%Câu 33.
	Biểu thức $K=\sqrt{2\sqrt[3]{2}}$ viết dưới dạng lũy thừa với số mũ hữu tỉ là 
	\choice
	{$2^{\tfrac{4}{3}}$}
	{$2^{\tfrac{5}{3}}$}
	{$2^{\tfrac{1}{3}}$}
	{\True $2^{\tfrac{2}{3}}$}
	\loigiai{
		$K=\sqrt{2\sqrt[3]{2}}=\sqrt{{2\cdot 2}^{\tfrac{1}{3}}}=\sqrt{2^{\tfrac{4}{3}}}=\left(2^{\tfrac{4}{3}}\right)^{\tfrac{1}{2}}=2^{\tfrac{2}{3}}$.}
\end{ex}
\begin{ex}%[2D2Y1-1]%Câu 34.
	Cho $P=\left(x^{\tfrac{1}{2}}-y^{\tfrac{1}{2}}\right)^2\left(1-2\sqrt{\dfrac{y}{x}}+\dfrac{y}{x}\right)^{-1}$, $x>0$, $y>0$. Biểu thức rút gọn của $P$ là 
	\choice
	{$x-1$}
	{$x+1$}
	{$2x$}
	{\True $x$}
	\loigiai{
		Với $x>0$, $y>0$, ta có\\
		\allowdisplaybreaks
		\begin{eqnarray*}
			P&=&\left(x^{\tfrac{1}{2}}-y^{\tfrac{1}{2}}\right)^2\left(1-2\sqrt{\dfrac{y}{x}}+\dfrac{y}{x}\right)^{-1}	=(\sqrt{x}-\sqrt{y})^2 \left(1-\sqrt{\dfrac{y}{x}}\right)^{-2}\\
			&=&(\sqrt{x}-\sqrt{y})^2\left(\dfrac{\sqrt{x}-\sqrt{y}}{\sqrt{x}}\right)^{-2}=x	
		\end{eqnarray*}
	}
\end{ex}
\begin{ex}%[2D2Y1-2]%Câu 35.
	Biểu thức $\sqrt{x}\cdot\sqrt[3]{x}\cdot\sqrt[6]{x^5}$, $(x>0)$ viết dưới dạng luỹ thừa với số mũ hữu tỉ là 
	\choice
	{$x^{\tfrac{2}{3}}$}
	{$x^{\tfrac{5}{2}}$}
	{$x^{\tfrac{7}{3}}$}
	{\True $x^{\tfrac{5}{3}}$}
	\loigiai{
		$\sqrt{x}\cdot\sqrt[3]{x}\cdot\sqrt[6]{x^5}=x^{\tfrac{1}{2}} \cdot x^{\tfrac{1}{3}} \cdot x^{\tfrac{5}{6}}=x^{\tfrac{10}{6}}=x^{\tfrac{5}{3}}$.}
\end{ex}
\begin{ex}%[2D2B1-2]%Câu 36.
	Biểu diễn biểu thức $P=\sqrt{x\sqrt[3]{x^2\sqrt[4]{x^3}}}$ dưới dạng lũy thừa số mũ hữu tỉ. 
	\choice
	{$P=x^{\tfrac{12}{23}}$}
	{$P=x^{\tfrac{1}{4}}$}
	{$P=x^{\tfrac{23}{12}}$}
	{\True $P=x^{\tfrac{23}{24}}$}
	\loigiai{
		Ta có $P=\sqrt{x\sqrt[3]{x^2\sqrt[4]{x^3}}}=\left[x\left(x^2 \cdot x^{\tfrac{3}{4}}\right)^{\tfrac{1}{3}}\right]^{\tfrac{1}{2}}=x^{\tfrac{23}{24}}$.}
\end{ex}
\begin{ex}%[2D2B1-2]%Câu 37.
	Cho $x>0$. Hãy biểu diễn biểu thức $\sqrt{x\sqrt{x\sqrt{x}}}$ dưới dạng lũy thừa của $x$ với số mũ hữu tỉ?
	\choice
	{$x^{\tfrac{3}{8}}$}
	{\True $x^{\tfrac{7}{8}}$}
	{$x^{\tfrac{1}{8}}$}
	{$x^{\tfrac{5}{8}}$}
	\loigiai{
		Ta có: $\sqrt{x\sqrt{x\sqrt{x}}}=\sqrt{x\sqrt{x^{\tfrac{3}{2}}}}=\sqrt{x^{\tfrac{7}{4}}}=x^{\tfrac{7}{8}}$.}
\end{ex}
\begin{ex}%[2D2Y1-2]%Câu 38.
	Biểu thức $\sqrt{x}\cdot\sqrt[3]{x}\cdot\sqrt[6]{x^5} (x>0)$ viết dưới dạng lũy thừa với số mũ hữu tỉ là 
	\choice
	{$x^{\tfrac{7}{3}}$}
	{$x^{\tfrac{2}{3}}$}
	{\True $x^{\tfrac{5}{3}}$}
	{$x^{\tfrac{5}{2}}$}
	\loigiai{
		$\sqrt{x} \cdot \sqrt[3]{x} \cdot \sqrt[6]{x^5}=x^{\tfrac{1}{2}} \cdot x^{\tfrac{1}{3}} \cdot x^{\tfrac{5}{6}}=x^{\tfrac{1}{2}+\tfrac{1}{3}+\tfrac{5}{6}}=x^{\tfrac{5}{3}}$.}
\end{ex}
\begin{ex}%[2D2B1-2]%Câu 39.
	Cho $x$, $y$ là các số thực dương. Rút gọn biểu thức $P=\left(x^{\tfrac{1}{2}}-y^{\tfrac{1}{2}}\right)^2\left(1-2\sqrt{\dfrac{y}{x}}+\dfrac{y}{x}\right)^{-1}$. 
	\choice
	{\True $P=x$}
	{$P=2x$}
	{$P=x-1$}
	{$P=x+1$}
	\loigiai{
		\allowdisplaybreaks
		\begin{eqnarray*}
			P&=&\left(x^{\tfrac{1}{2}}-y^{\tfrac{1}{2}}\right)^2\left(1-2\sqrt{\dfrac{y}{x}}+\dfrac{y}{x}\right)^{-1}=\left(x^{\tfrac{1}{2}}-y^{\tfrac{1}{2}}\right)^2 \left(1-\sqrt{\dfrac{y}{x}}\right)^{-2}=(\sqrt{x}-\sqrt{y})^2\left(\dfrac{\sqrt{x}-\sqrt{y}}{\sqrt{x}}\right)^{-2}\\
			&=&(\sqrt{x}-\sqrt{y})^2\left(\dfrac{\sqrt{x}}{\sqrt{x}-\sqrt{y}}\right)^2=\left((\sqrt{x}-\sqrt{y})\cdot\dfrac{\sqrt{x}}{\sqrt{x}-\sqrt{y}}\right)^2=x.
		\end{eqnarray*}
	}
\end{ex}
\begin{ex}%[2D2Y1-2]%Câu 40.
	Biến đổi biểu thức $P=\sqrt{x} \cdot \sqrt[3]{x} \cdot \sqrt[6]{x^5}$, $(x>0)$ thành dạng với số mũ hữu tỉ. 
	\choice
	{$P=x^{\tfrac{7}{3}}$}
	{\True $P=x^{\tfrac{5}{3}}$}
	{$P=x^{\tfrac{5}{2}}$}
	{$P=x^{\tfrac{2}{3}}$}
	\loigiai{
		$P=\sqrt{x}\cdot\sqrt[3]{x}\cdot\sqrt[6]{x^5}=x^{\tfrac{1}{2}+\tfrac{1}{3}+\tfrac{5}{6}}=x^{\tfrac{5}{3}}$.}
\end{ex}
\begin{ex}%[2D2Y1-2]%Câu 41.
	Cho $a,b$ là hai số thực dương. Kết quả thu gọn của biểu thức $A=\dfrac{(\sqrt[4]{a^3 \cdot b^2})^4}{\sqrt[3]{\sqrt{a^{12} \cdot b^6}}}$ là 
	\choice
	{$1$}
	{$Q$}
	{\True $ab$}
	{$b$}
	\loigiai{
		$A=\dfrac{(\sqrt[4]{a^3 \cdot b^2})^4}{\sqrt[3]{\sqrt{a^{12} \cdot b^6}}}=\dfrac{a^3 \cdot b^2}{\sqrt[3]{a^6 \cdot b^3}}=\dfrac{a^3 \cdot b^2}{a^2b}$, $\forall a,b>0$.}
\end{ex}
\begin{ex}%[2D2B1-2]%Câu 42.
	Cho $a$ là số thực dương, $a\neq 1$ và $P=\log_{\sqrt[3]{a}}\sqrt{a\sqrt{a\sqrt{a\sqrt{a\sqrt{a}}}}}$. Chọn mệnh đề \bf{đúng}?
	\choice
	{$P=3$}
	{\True $P=\dfrac{93}{32}$}
	{$P=15$}
	{$P=\dfrac{45}{16}$}
	\loigiai{
		Ta có $\sqrt{a\sqrt{a\sqrt{a\sqrt{a\sqrt{a}}}}}=a^{\tfrac{31}{32}}$.\\
		$P=\log_{\sqrt[3]{a}}\sqrt{a\sqrt{a\sqrt{a\sqrt{a\sqrt{a}}}}}=\log_{a^{\tfrac{1}{3}}}a^{\tfrac{31}{32}}=\dfrac{93}{32}$.}
\end{ex}
\begin{ex}%[2D2B1-2]%Câu 43.
	Cho biểu thức $P=\left\{a^{\tfrac{1}{3}}\left[a^{-\tfrac{1}{2}}b^{-\tfrac{1}{3}}\left(a^2b^2\right)^{\tfrac{2}{3}}\right]^{-\tfrac{1}{2}}\right\}^6$ với $a$, $b$ là các số dương. Khẳng định nào sau đây là \bf{đúng}?
	\choice
	{$P=\dfrac{\sqrt{a}}{b^3}$}
	{$P=\dfrac{b^3\sqrt{a}}{a}$}
	{$P=b^3\sqrt{a}$}
	{\True $P=\dfrac{\sqrt{a}}{a \cdot b^3}$}
	\loigiai{
		Ta có
		\allowdisplaybreaks
		\begin{eqnarray*}
			P&=&\left\{a^{\tfrac{1}{3}}\left[a^{-\tfrac{1}{2}} \cdot b^{-\tfrac{1}{3}}\left(a^2 \cdot b^2\right)^{\tfrac{2}{3}}\right]^{-\tfrac{1}{2}}\right\}^6=a^2\left[a^{-\tfrac{1}{2}} \cdot b^{-\tfrac{1}{3}}\left(a^2 \cdot b^2\right)^{\tfrac{2}{3}}\right]^{-3}=a^2\left[a^{\tfrac{3}{2}} \cdot b\left(a^2 \cdot b^2\right)^{-2}\right]\\
			&=&a^{\tfrac{7}{2}} \cdot b\cdot a^{-4} \cdot b^{-4}=a^{-\tfrac{1}{2}} \cdot b^{-3}=\dfrac{1}{\sqrt{a \cdot b^3}}.
		\end{eqnarray*}
	}
\end{ex}
\begin{ex}%[2D2B1-2]%Câu 44.
	Rút gọn biểu thức $\sqrt{x\sqrt{x\sqrt{x\sqrt{x}}}}\colon x^{\tfrac{11}{16}}$, $(x>0)$ ta được
	\choice
	{\True $\sqrt[4]{x}$}
	{$\sqrt[6]{x}$}
	{$\sqrt[5]{x}$}
	{$\sqrt{x}$}
	\loigiai{
		Ta có
		\allowdisplaybreaks
		\begin{eqnarray*}
			\sqrt{x\sqrt{x\sqrt{x\sqrt{x}}}}\colon x^{\tfrac{11}{16}}&=& x^{\tfrac{1}{2}} \cdot x^{\tfrac{1}{4}} \cdot x^{\tfrac{1}{8}}\cdot x^{\tfrac{1}{16}}\colon x^{\tfrac{11}{16}}=x^{\tfrac{1}{2}+\tfrac{1}{4}+\tfrac{1}{8}+\tfrac{1}{16}}\colon x^{\tfrac{11}{16}}\\
			&=&x^{\tfrac{15}{16}}\colon x^{\tfrac{11}{16}}=x^{\tfrac{15}{16}-\tfrac{11}{16}}=x^{\tfrac{1}{4}}=\sqrt[4]{x}.
		\end{eqnarray*}
	}
\end{ex}
\begin{ex}%[2D2Y1-2]%Câu 45.
	Cho biểu thức $P=\dfrac{a^{\sqrt{7}+1}\cdot a^{2-\sqrt{7}}}{\left(a^{\sqrt{2}-2}\right)^{\sqrt{2}+2}}$ với $a>0$. Rút gọn biểu thức $P$ được kết quả
	\choice
	{\True $P=a^5$}
	{$P=a^3$}
	{$P=a$}
	{$P=a^4$}
	\loigiai{
		$P=\dfrac{a^{\sqrt{7}+1}\cdot a^{2-\sqrt{7}}}{\left(a^{\sqrt{2}-2}\right)^{\sqrt{2}+2}}=\dfrac{a^3}{a^{-2}}=a^5$.}
\end{ex}
\begin{ex}%[2D2Y1-1]%Câu 46.
	Viết biểu thức $A=\sqrt[3]{2\sqrt[5]{2\sqrt{2}}}$ dưới dạng lũy thừa của số mũ hữu tỉ ta được
	\choice
	{$A=2^{\tfrac{2}{3}}$}
	{\True $A=2^{\tfrac{13}{30}}$}
	{$A=2^{\tfrac{91}{30}}$}
	{$A=2^{\tfrac{1}{30}}$}
	\loigiai{
		$A=\sqrt[3]{2\sqrt[5]{2\sqrt{2}}}=\sqrt[3]{2\sqrt[5]{2^12^{\tfrac{1}{2}}}}=\sqrt[3]{2\sqrt[5]{2^{\tfrac{3}{2}}}}=\sqrt[3]{2^1{\cdot 2}^{\tfrac{3}{10}}}=\sqrt[3]{2^{\tfrac{13}{10}}}=2^{\tfrac{13}{30}}$.}
\end{ex}
\begin{ex}%[2D2Y1-2]%Câu 47.
	Rút gọn biểu thức $\dfrac{a^{\sqrt{6}+1}a^{2-\sqrt{5}}}{\left(a^{\sqrt{2}-2}\right)^{\sqrt{2}+2}}$ (với $a>0$) được kết quả là
	\choice
	{$a^4$}
	{\True $a^5$}
	{$a^3$}
	{$a$}
	\loigiai{
		$\dfrac{a^{\sqrt{6}+1}\cdot a^{2-\sqrt{5}}}{\left(a^{\sqrt{2}-2}\right)^{\sqrt{2}+2}}=\dfrac{a^3}{a^{-2}}=a^5$.}
\end{ex}
\begin{ex}%[2D2Y1-2]%Câu 48.
	Viết biểu thức $A=\sqrt[3]{2\sqrt[5]{2\sqrt{2}}}$ dưới dạng lũy thừa của số mũ hữu tỉ ta được
	\choice
	{$A=2^{\tfrac{2}{3}}$}
	{\True $A=2^{\tfrac{13}{30}}$}
	{$A=2^{\tfrac{91}{30}}$}
	{$A=2^{\tfrac{1}{30}}$}
	\loigiai{
		$A=\sqrt[3]{2\sqrt[5]{2\sqrt{2}}}=\sqrt[3]{2\sqrt[5]{2^12^{\tfrac{1}{2}}}}=\sqrt[3]{2\sqrt[5]{2^{\tfrac{3}{2}}}}=\sqrt[3]{2^1{\cdot 2}^{\tfrac{3}{10}}}=\sqrt[3]{2^{\tfrac{13}{10}}}=2^{\tfrac{13}{30}}$.}
\end{ex}
\begin{ex}%[2D2Y1-2]%Câu 49.
	Rút gọn biểu thức $Q=b^{\tfrac{5}{3}}\colon\sqrt[3]{b}$ với $b>0$. 
	\choice
	{$Q=b^2$}
	{$Q=b^{-\tfrac{4}{3}}$}
	{\True $Q=b^{\tfrac{4}{3}}$}
	{$Q=b^{\tfrac{5}{9}}$}
	\loigiai{
		Ta có $Q=b^{\tfrac{5}{3}}\colon\sqrt[3]{b}=b^{\tfrac{5}{3}}\colon b^{\tfrac{1}{3}}=b^{\tfrac{4}{3}}$.}
\end{ex}
\begin{ex}%[2D2B1-2]%Câu 50.
	Cho $x$ là số thực dương, viết biểu thức $Q=\sqrt{x\sqrt[3]{x^2}}\cdot\sqrt[6]{x}$ dưới dạng lũy thừa với số mũ hữu tỉ. 
	\choice
	{$Q=x^2$}
	{$Q=x^{\tfrac{2}{3}}$}
	{\True $Q=x$}
	{$Q=x^{\tfrac{5}{36}}$}
	\loigiai{
		Ta có $Q=\sqrt{x\sqrt[3]{x^2}}\cdot\sqrt[6]{x} =x^{\tfrac{1}{2}} \cdot x^{\tfrac{2}{3}\cdot\tfrac{1}{2}} \cdot x^{\tfrac{1}{6}} =x$.}
\end{ex}
\begin{ex}%[2D2B1-2]%Câu 51.
	Cho biểu thức $P=\dfrac{b \cdot \sqrt[3]{a^4}+a \cdot \sqrt[3]{b^4}}{\sqrt[3]{a}+\sqrt[3]{b}} $, với $a>0$, $b>0$. Mệnh đề nào sau đây \bf{đúng}?
	\choice
	{$P=2a \cdot b$}
	{$P=a^{\tfrac{1}{3}} \cdot b^{\tfrac{1}{3}} $}
	{\True $P=ab$}
	{$P=b+a$}
	\loigiai{
		Ta có: $P=\dfrac{b \cdot \sqrt[3]{a^4}+a \cdot \sqrt[3]{b^4}}{\sqrt[3]{a}+\sqrt[3]{b}}=\dfrac{a \cdot b \cdot \sqrt[3]{a}+a \cdot b \cdot \sqrt[3]{b}}{\sqrt[3]{a}+\sqrt[3]{b}}=\dfrac{a \cdot b(\sqrt[3]{a}+\sqrt[3]{b})}{\sqrt[3]{a}+\sqrt[3]{b}}=a \cdot b$.}
\end{ex}
\begin{ex}%[2D2B1-2]%Câu 52.
	Cho biểu thức $P=\dfrac{b \cdot \sqrt[3]{a^4}+a \cdot \sqrt[3]{b^4}}{\sqrt[3]{a}+\sqrt[3]{b}}$, với $a>0$, $b>0$. Mệnh đề nào sau đây \bf{đúng}?
	\choice
	{$P=2a \cdot b$}
	{$P=a^{\tfrac{1}{3}} \cdot b^{\tfrac{1}{3}}$}
	{\True $P=a \cdot b$}
	{$P=b+a$}
	\loigiai{
		Ta có: $P=\dfrac{b \cdot \sqrt[3]{a^4}+a \cdot \sqrt[3]{b^4}}{\sqrt[3]{a}+\sqrt[3]{b}}=\dfrac{a \cdot b \cdot \sqrt[3]{a}+a \cdot b \cdot \sqrt[3]{b}}{\sqrt[3]{a}+\sqrt[3]{b}}=\dfrac{a \cdot b(\sqrt[3]{a}+\sqrt[3]{b})}{\sqrt[3]{a}+\sqrt[3]{b}}=a \cdot b$.}
\end{ex}
\begin{ex}%[2D2Y1-2]%Câu 53.
	Cho $a$ là một số thực dương, biểu thức $a^{\tfrac{2}{3}}\sqrt{a}$ viết dưới dạng lũy thừa với số mũ hữu tỉ là 
	\choice
	{\True $a^{\tfrac{7}{6}}$}
	{$a^{\tfrac{5}{6}}$}
	{$a^{\tfrac{6}{5}}$}
	{$a^{\tfrac{11}{6}}$}
	\loigiai{
		\begin{itemize}
			\item Tự luận: \\
			Ta có: $a^{\tfrac{2}{3}}\sqrt{a}=a^{\tfrac{2}{3}}\cdot a^{\tfrac{1}{2}}=a^{\tfrac{2}{3}+\tfrac{1}{2}}=a^{\tfrac{7}{6}}$.
			\item Trắc nghiệm:\\
			Chọn $a=4$, bấm máy $4^{\tfrac{2}{3}}\sqrt{4}-4^{\tfrac{7}{6}}=0\Rightarrow$ chọn A.
		\end{itemize}
	}
\end{ex}
\begin{ex}%[2D2Y1-1]%Câu 54.
	Cho $f(x)=\sqrt[3]{x}\sqrt[6]{x}$. Khi đó $f(0,09)$ bằng 
	\choice
	{$0,1$}
	{$0,2$}
	{\True $0,3$}
	{$0,4$}
	\loigiai{
		\begin{itemize}
			\item Tự luận: \\
			Ta có: $\sqrt[3]{x}\cdot\sqrt[6]{x}=\sqrt[3]{0,09}\cdot\sqrt[6]{0,09}=\dfrac{3}{10}=0,3$.
			\item Trắc nghiệm:\\
			mode 1; nhập màn hình $\sqrt[3]{x}\cdot\sqrt[6]{x}$ CALC $X= 0,09$ kết quả bằng $0,3$.
		\end{itemize}
	}
\end{ex}
\begin{ex}%[2D2B1-2]%Câu 55.
	Viết biểu thức $A=\sqrt{a\sqrt{a\sqrt{a}}}\colon a^{\tfrac{11}{6}}$, $(a>0)$ dưới dạng lũy thừa của số mũ hữu tỉ. 
	\choice
	{$A=a^{\tfrac{21}{44}}$}
	{$A=a^{\tfrac{-1}{12}}$}
	{$A=a^{\tfrac{23}{24}}$}
	{\True $A=a^{\tfrac{-23}{24}}$}
	\loigiai{
		\begin{itemize}
			\item Tự luận: \\
			Ta có: $A=\sqrt{a\sqrt{a\sqrt{a}}}\colon a^{\tfrac{11}{6}}=a^{\tfrac{1}{2}}a^{\tfrac{1}{4}}a^{\tfrac{1}{8}}\colon a^{\tfrac{11}{6}}=a^{\tfrac{7}{8}}\colon a^{\tfrac{11}{6}}=a^{\tfrac{-23}{24}}$.
			\item Trắc nghiệm:\\
			Thế $a=4$ ta được $A=\left(\sqrt{4\sqrt{4\sqrt{4}}}\colon 4^{\tfrac{11}{6}}\right)-\left(4^{\tfrac{21}{24}}\right)$ thế lần lượt từ phương án A đến phương án D được kết quả bằng 0.
		\end{itemize}
	}
\end{ex}
\begin{ex}%[2D2Y1-2]%Câu 56.
	Biểu thức $\sqrt{x}\cdot\sqrt[3]{x}\cdot\sqrt[6]{x^5}$ (x > 0) viết dưới dạng lũy thừa với số mũ hữu tỉ là 
	\choice
	{$x^{\tfrac{7}{3}}$}
	{$x^{\tfrac{5}{2}}$}
	{$x^{\tfrac{2}{3}}$}
	{\True $x^{\tfrac{5}{3}}$}
	\loigiai{
		\begin{itemize}
			\item Tự luận: \\
			Ta có: $\sqrt{x}\cdot\sqrt[3]{x}\cdot\sqrt[6]{x^5}=x^{\tfrac{1}{2}} \cdot x^{\tfrac{1}{3}}\cdot x^{\tfrac{5}{6}}=x^{\tfrac{5}{3}}$.
			\item Trắc nghiệm:\\
			Thế $a=4$, ta được $\sqrt{4}\cdot\sqrt[3]{4}\cdot\sqrt[6]{4^5}-4^{\tfrac{7}{3}}$ thế lần lượt từ phương án A đến phương án D được kết quả bằng $0$.
		\end{itemize}
	}
\end{ex}
\begin{ex}%[2D2Y1-2]%Câu 57.
	Rút gọn $\dfrac{\left(\sqrt[4]{a^3\cdot b^2}\right)^4}{\sqrt[3]{\sqrt{a^{12}\cdot b^6}}}$, với $a,b$ là các số thực dương ta được
	\choice
	{$a^2 \cdot b$}
	{$a \cdot b^2$}
	{$a^2 \cdot b^2$}
	{\True $a \cdot b$}
	\loigiai{
		\begin{itemize}
			\item Tự luận: \\
			Ta có: $\dfrac{(\sqrt[4]{a^3 \cdot b^2})^4}{\sqrt[3]{\sqrt{a^{12} \cdot b^6}}}=\dfrac{a^3 \cdot b^2}{\sqrt[3]{a^6 \cdot b^3}}=\dfrac{a^3 \cdot b^2}{a^2 \cdot b}=a \cdot b$.
			\item Trắc nghiệm:\\
			Thế $a= 2$, $b = 3$ ta được $\dfrac{\left(\sqrt[4]{2^3\cdot 3^2}\right)^4}{\sqrt[3]{\sqrt{2^{12}\cdot 3^6}}}-2^2\cdot 3$ thế lần lượt từ phương án A đến phương án D được kết quả bằng $0$.
		\end{itemize}
	}
\end{ex}
\begin{ex}%[2D2B1-1]%Câu 58.
	Cho biểu thức $A=(a+1)^{-1}+(b+1)^{-1}$. Nếu $a=(2+\sqrt{3})^{-1}$ và $b=(2-\sqrt{3})^{-1}$ thì giá trị của $A$ là 
	\choice
	{\True $1$}
	{$2$}
	{$3$}
	{$4$}
	\loigiai{
		\begin{itemize}
			\item Tự luận: \\
			Vì $(2+\sqrt{3})^{-1}\cdot (2-\sqrt{3})^{-1}=1$. Nên ta có:\\
			$\begin{aligned}A&=(a+1)^{-1}+(b+1)^{-1}=\dfrac{1}{a+1}+\dfrac{1}{b+1}=\dfrac{a+b+2}{ab+a+b+1}\\&=\dfrac{a+b+2}{a+b+2}=1 \quad (\text{Do } a \cdot b=1). \end{aligned}$
			\item Trắc nghiệm: Thay trực tiếp $a$, $b$ đã cho vào tính.
		\end{itemize}
	}
\end{ex}
\begin{ex}%[2D2Y1-2]%Câu 59.
	Rút gọn biểu thức $P=a^{\sqrt{3}+2}\cdot\left(\dfrac{1}{a}\right)^{\sqrt{3}-1}$ với $a>0$.
	\choice
	{\True $P=a^3$}
	{$P=a^{\sqrt{3}+1}$}
	{$P=a^{2\sqrt{3}+1}$}
	{$P=a$}
	\loigiai{
		\begin{itemize}
			\item Tự luận: \\
			$P=a^{\sqrt{3}+2}\cdot \left(\dfrac{1}{a}\right)^{\sqrt{3}-1}=a^{\sqrt{3}+2}\cdot a^{-(\sqrt{3}-1)}=a^{\sqrt{3}+2-\sqrt{3}+1}=a^3$.
			\item Trắc nghiệm: Sử dụng máy tính Caisio.\\
			Cho $a=2$ nhập vào máy tính biểu thức $P$.
			Nhận thấy $8=2^3$. Vậy$P=a^3$ là đúng. (hoặc có thể lấy kết quả tính được trừ đi đáp án, nếu ra ra $0$ thì đúng).
		\end{itemize}
	}
\end{ex}
\begin{ex}%[2D2Y1-1]%Câu 61.
	Tính $K=\left(\dfrac{1}{16}\right)^{-0,75}+\left(\dfrac{1}{8}\right)^{-\tfrac{4}{3}}$, ta được
	\choice
	{$12$}
	{$18$}
	{\True $24$}
	{$16$}
	\loigiai{
		\begin{itemize}
			\item Tự luận: \\
			$K=\left(\frac{1}{16}\right)^{-0.75}+\left(\frac{1}{8}\right)^{-\frac{4}{3}}=\left(2^{-4}\right)^{-0.75}+\left(2^{-3}\right)^{-\frac{4}{3}}=2^{3}+2^{4}=8+16=24$.
			\item Trắc nghiệm: Sử dụng máy tính Caisio.\\
			Nhập vào máy tính biểu thức $K$ rồi bắm $=$, ta được kết quả $24$.
		\end{itemize}
	}
\end{ex}
\begin{ex}%[2D2B1-2]%Câu 62.
	Cho biểu thức $P=x\sqrt[5]{x\sqrt[3]{x\sqrt{x}}}$, $x>0$. Mệnh đề nào \bf{đúng}?
	\choice
	{$P=x^{\tfrac{2}{3}}$}
	{$P=x^{\tfrac{3}{10}}$}
	{\True $P=x^{\tfrac{13}{10}}$}
	{$P=x^{\tfrac{1}{2}}$}
	\loigiai{
		$P=x \sqrt[5]{x \sqrt[3]{x \sqrt{x}}}=x \sqrt[5]{x \cdot \sqrt[3]{x x^{\tfrac{1}{2}}}}=x \sqrt[5]{x \sqrt[3]{x^{\tfrac{3}{2}}}}=x \sqrt[5]{x \cdot x^{\tfrac{1}{2}}}=x \sqrt[5]{x^{\tfrac{3}{2}}}=x \cdot x^{\tfrac{3}{10}}=x^{\tfrac{13}{10}}$.
	}
\end{ex}
\begin{ex}%[2D2B1-1]%Câu 63.
	Tính giá trị biểu thức $A=(a+1)^{-1}+(b+1)^{-1}$ khi $a=(2+\sqrt{3})^{-1}$, $b=(2-\sqrt{3})^{-1}$. 
	\choice
	{\True $1$}
	{$2$}
	{$3$}
	{$4$}
	\loigiai{
		$a=(2+\sqrt{3})^{-1}=\dfrac{1}{(2+\sqrt{3})}=\dfrac{2-\sqrt{3}}{4-3}=2-\sqrt{3} \Rightarrow a+1=3-\sqrt{3}$. \\
		$b=(2-\sqrt{3})^{-1}=\dfrac{1}{(2-\sqrt{3})}=\dfrac{2+\sqrt{3}}{4-3}=2+\sqrt{3} \Rightarrow b+1=3+\sqrt{3}$.\\
		Suy ra $A=\dfrac{1}{3-\sqrt{3}}+\dfrac{1}{3+\sqrt{3}}=\dfrac{6}{(3-\sqrt{3})(3+\sqrt{3})}=1$.
	}
\end{ex}
\begin{ex}%[2D2B1-2]%Câu 64.
	Rút gọn biểu thức $A=\dfrac{a+3-10a^{-1}}{a^{\tfrac{1}{2}}+5a^{-\tfrac{1}{2}}}-\dfrac{a-9a^{-1}}{a^{\tfrac{1}{2}}-3a^{-\tfrac{1}{2}}}\quad(0<a\not=1)$. 
	\choice
	{$\sqrt{a}$}
	{$-\dfrac{5}{a}$}
	{$a+1$}
	{\True $-\dfrac{5}{\sqrt{a}}$}
	\loigiai{
		$A=\dfrac{a+3-10 a^{-1}}{a^{\tfrac{1}{2}}+5 a^{-\tfrac{1}{2}}}-\dfrac{a-9 a^{-1}}{a^{\tfrac{1}{2}}-3 a^{-\tfrac{1}{2}}}=\dfrac{a^{2}+3 a-10}{\dfrac{a}{\sqrt{a}}(a+5)}-\dfrac{a^{2}-9}{\dfrac{a}{\sqrt{a}}(a-3)}=\dfrac{a-2}{\sqrt{a}}-\dfrac{a+3}{\sqrt{a}}=-\dfrac{5}{\sqrt{a}}$.
	}
\end{ex}
\begin{ex}%[2D2B1-1]%Câu 65.
	Kết quả của phép tính $A=\left(\dfrac{1}{16}\right)^{-0,75}+0,25^{-\tfrac{5}{2}}$ là 
	\choice
	{\True $40$}
	{$\dfrac{5}{32}$}
	{$-24$}
	{$\dfrac{257}{8}$}
	\loigiai{
		$A=\left(\dfrac{1}{16}\right)^{-0,75}+0,25^{-\tfrac{5}{2}}=\left(\dfrac{1}{2^{4}}\right)^{-\tfrac{3}{4}}+\left(\frac{1}{4}\right)^{-\frac{5}{2}}=\left(2^{-4}\right)^{\tfrac{-3}{4}}+\left(2^{-2}\right)^{\tfrac{-5}{2}}=2^{3}+2^{5}=40$.
	}
\end{ex}
\begin{ex}%[2D2Y1-1]%Câu 66.
	Kết quả của phép tính $B=27^{\tfrac{2}{3}}+\left(\dfrac{1}{16}\right)^{-0,25}-25^{0,5}$ là 
	\choice
	{\True $6$}
	{$\dfrac{9}{2}$}
	{$16$}
	{$\dfrac{54}{5}$}
	\loigiai{
		$B=27^{\tfrac{2}{3}}+\left(\dfrac{1}{16}\right)^{-0,25}-25^{0,5}=\left(3^{3}\right)^{\tfrac{2}{3}}+\left(2^{-4}\right)^{-\tfrac{1}{4}}-\left(5^{2}\right)^{\tfrac{1}{2}}=3^{2}+2-5=6$.}
\end{ex}

\begin{ex}%[Danh Trần]%[2D2B1-2]
	Biểu thức $C=\sqrt{x\sqrt{x\sqrt{x\sqrt{x}}}}$ ($x>0$) được viết dưới dạng lũy thừa số mũ hữu tỷ là
	\choice
	{$x^{\tfrac{15}{18}}$}
	{$x^{\tfrac{7}{8}}$}
	{\True $x^{\tfrac{15}{16}}$}
	{$x^{\tfrac{3}{16}}$}
	\loigiai{
		Ta có
		\allowdisplaybreaks
		\begin{eqnarray*}
			C & = & \sqrt{x\sqrt{x\sqrt{x\sqrt{x}}}}=\sqrt{x\sqrt{x\sqrt{x\cdot x^{\tfrac{1}{2}}}}}=\sqrt{x\sqrt{x\sqrt{x^{\tfrac{3}{2}}}}}=\sqrt{x\sqrt{x\cdot x^{\tfrac{3}{4}}}}\\
			& = & \sqrt{x\sqrt{x^{\tfrac{7}{4}}}}=\sqrt{x\cdot x^{\tfrac{7}{8}}}=\sqrt{x^{\tfrac{15}{8}}}=x^{\tfrac{15}{16}}.
		\end{eqnarray*}
	}
\end{ex}
\begin{ex}%[Danh Trần]%[2D2B1-2]
	Cho biểu thức $D=\sqrt[4]{x\cdot\sqrt[3]{x^2\cdot\sqrt{x^3}}}$, với $x>0$. Mệnh đề nào dưới đây đúng?
	\choice
	{$D=x^{\tfrac{1}{2}}$}
	{\True $D=x^{\tfrac{13}{24}}$}
	{$D=x^{\tfrac{1}{4}}$}
	{$D=x^{\tfrac{2}{3}}$}
	\loigiai{
		Ta có $D=\sqrt[4]{x\cdot\sqrt[3]{x^2\cdot\sqrt{x^3}}}=\sqrt[4]{x\cdot\sqrt[3]{x^2\cdot x^{\tfrac{3}{2}}}}=\sqrt[4]{x\cdot\sqrt[3]{x^{\tfrac{7}{2}}}}=\sqrt[4]{x\cdot x^{\tfrac{7}{6}}}=\sqrt[4]{x^{\tfrac{13}{6}}}=x^{\tfrac{13}{24}}$.
	}
\end{ex}
\begin{ex}%[Danh Trần]%[2D2K1-2]
	Rút gọn biểu thức $E=\left[\dfrac{a\sqrt{2}}{\left(1+a^2\right)^{-1}}-\dfrac{2\sqrt{2}}{a^{-1}}\right]\colon\dfrac{1-a^{-2}}{a^{-3}}$ (với $a\neq 0,a\neq\pm 1$) là 
	\choice
	{\True $\sqrt{2}$}
	{$\sqrt{2}a$}
	{$a$}
	{$\dfrac{1}{a}$}
	\loigiai{
		Ta có
		\allowdisplaybreaks
		\begin{eqnarray*}
			E & = & \left[\dfrac{a\sqrt{2}}{\left(1+a^2\right)^{-1}}-\dfrac{2\sqrt{2}}{a^{-1}}\right]\colon\dfrac{1-a^{-2}}{a^{-3}}=\left[\dfrac{a\sqrt{2}}{\dfrac{1}{(1+a^2)}}-\dfrac{2\sqrt{2}}{\dfrac{1}{a}}\right]:\dfrac{1-\dfrac{1}{a^2}}{\dfrac{1}{a^3}}\\
			& = & [a\sqrt{2}(1+a^2)-2\sqrt{2}a]:\dfrac{\dfrac{a^2-1}{a^2}}{\dfrac{a^3-1}{a^3}}=(a\sqrt{2}+a^3\sqrt{2}-2\sqrt{2}a):a(a^2-1)\\
			& = & (a^3\sqrt{2}-a\sqrt{2})\cdot\dfrac{1}{a(a^2-1)}=a\sqrt{2}(a^2-1)\cdot\dfrac{1}{a(a^2-1)}=\sqrt{2}.
		\end{eqnarray*}
	}
\end{ex}
\begin{ex}%[Danh Trần]%[2D2B1-2]
	Rút gọn biểu thức $F=\dfrac{a^{-n}+b^{-n}}{a^{-n}-b^{-n}}-\dfrac{a^{-n}-b^{-n}}{a^{-n}+b^{-n}}$ (với $ab\neq 0,a\neq\pm b$) là 
	\choice
	{$\dfrac{a^nb^n}{b^{2n}-a^{2n}}$}
	{$\dfrac{2a^nb^n}{b^{2n}-a^{2n}}$}
	{$\dfrac{3a^nb^n}{b^{2n}-a^{2n}}$}
	{\True $\dfrac{4a^nb^n}{b^{2n}-a^{2n}}$}
	\loigiai{
		Ta có
		\allowdisplaybreaks
		\begin{eqnarray*}
			F & = & \dfrac{a^{-n}+b^{-n}}{a^{-n}-b^{-n}}-\dfrac{a^{-n}-b^{-n}}{a^{-n}+b^{-n}}=\dfrac{\dfrac{1}{a^n}+\dfrac{1}{b^n}}{\dfrac{1}{a^n}-\dfrac{1}{b^n}}+\dfrac{\dfrac{1}{a^n}-\dfrac{1}{b^n}}{\dfrac{1}{a^n}+\dfrac{1}{b^n}}=\dfrac{\dfrac{b^n+a^n}{a^nb^n}}{\dfrac{b^n-a^n}{a^nb^n}}-\dfrac{\dfrac{b^n-a^n}{a^nb^n}}{\dfrac{b^n+a^n}{a^nb^n}}\\
			& = & \dfrac{b^n+a^n}{b^n-a^n}-\dfrac{b^n-a^n}{b^n+a^n}=\dfrac{(b^n+a^n)^2-(b^n-a^n)^2}{(b^n-a^n)(b^n+a^n)}=\dfrac{4a^nb^n}{b^{2n}-a^{2n}}.
		\end{eqnarray*}
	}
\end{ex}
\begin{ex}%[Danh Trần]%[2D2K1-2]
	Cho $a\geq 0$, $a\neq 1$, $a\neq\dfrac{3}{2}$. Tìm giá trị lớn nhất $P_{\max}$ của biểu thức \[P=\left[\dfrac{4a-9a^{-1}}{2a^{\tfrac{1}{2}}-3a^{\tfrac{-1}{2}}}+\dfrac{a-4+3a^{-1}}{a^{\tfrac{1}{2}}-a^{\tfrac{-1}{2}}}\right]^2-\dfrac{3}{2}a^2.\]
	\choice
	{$P_{\max} =\dfrac{15}{2}$}
	{\True $P_{\max} =\dfrac{27}{2}$}
	{$P_{\max} =15$}
	{$P_{\max} =10$}
	\loigiai{
		Ta có
		\allowdisplaybreaks
		\begin{eqnarray*}
			P & = & \left[\dfrac{4a-9a^{-1}}{2a^{\tfrac{1}{2}}-3a^{\tfrac{-1}{2}}}+\dfrac{a-4+3a^{-1}}{a^{\tfrac{1}{2}}-a^{\tfrac{-1}{2}}}\right]^2-\dfrac{3}{2}a^2=\left[\dfrac{4a^2-9}{a\cdot\dfrac{2a-3}{a^{\tfrac{1}{2}}}}+\dfrac{a^2-4a+3}{a\cdot\dfrac{a-1}{a^{\tfrac{1}{2}}}}\right]^2-\dfrac{3}{2}a^2\\
			& = & \left[\dfrac{(2a-3)(2a+3)}{a^{\tfrac{1}{2}}(2a-3)}+\dfrac{(a-1)(a-3)}{a^{\tfrac{1}{2}}(a-1)}\right]^2-\dfrac{3}{2}a^2=\left[\dfrac{(2a+3)+(a-3)}{a^{\tfrac{1}{2}}}\right]^2-\dfrac{3}{2}a^2\\
			& = & 9a-\dfrac{3}{2}a^2=f(a).
		\end{eqnarray*}
		Ta có $f'(a)=9-3a=0\Leftrightarrow a=3$ nên khảo sát hàm số, ta có $P_{\max}=f(3)=\dfrac{27}{2}$.
	}
\end{ex}
\begin{ex}%[Danh Trần]%[2D2Y1-2]
	Cho $a>0$. Viết biểu thức $P=a^{\tfrac{1}{7}}\cdot\sqrt[7]{a^6}$ dưới dạng lũy thừa với số mũ hữu tỷ. 
	\choice
	{$P=1$}
	{\True $P=a$}
	{$P=a^7$}
	{$P=a^6$}
	\loigiai{
		Ta có $P=a^{\tfrac{1}{7}}\cdot\sqrt[7]{a^6}=a^{\tfrac{1}{7}}\cdot a^{\tfrac{6}{7}}=a$.
	}
\end{ex}
\begin{ex}%[Danh Trần]%[2D2B1-2]
	Cho $x,y>0$, rút gọn $P=\dfrac{x^{\tfrac{7}{6}}\cdot y+x\cdot y^{\tfrac{7}{6}}}{\sqrt[6]{x}+\sqrt[6]{y}}$. 
	\choice
	{$P=x+y$}
	{$P=\sqrt[6]{x}+\sqrt[6]{y}$}
	{\True $P=x\cdot y$}
	{$P=\sqrt[6]{xy}$}
	\loigiai{
		Ta có $P=\dfrac{x^{\tfrac{7}{6}}\cdot y+x\cdot y^{\tfrac{7}{6}}}{\sqrt[6]{x}+\sqrt[6]{y}}=\dfrac{xy\left(x^{\tfrac{1}{6}}+y^{\tfrac{1}{6}}\right)}{\sqrt[6]{x}+\sqrt[6]{y}}=\dfrac{xy(\sqrt[6]{x}+\sqrt[6]{y})}{\sqrt[6]{x}+\sqrt[6]{y}}=xy$.
	}
\end{ex}
\begin{ex}%[Danh Trần]%[2D2B1-2]
	Cho $a>0$, rút gọn $P=\dfrac{\left(a^{\sqrt{5}-2}\right)^{\sqrt{5}+2}}{a^{1-\sqrt{3}}\cdot a^{\sqrt{3}-2}}$ 
	\choice
	{$P=1$}
	{$P=a$}
	{$P=\dfrac{1}{a}$}
	{\True $P=a^2$}
	\loigiai{
		Ta có $P=\dfrac{\left(a^{\sqrt{5}-2}\right)^{\sqrt{5}+2}}{a^{1-\sqrt{3}}\cdot a^{\sqrt{3}-2}}=\dfrac{a^{(\sqrt{5}-2)(\sqrt{5}+2)}}{a^{1-\sqrt{3}+\sqrt{3}-2}}=\dfrac{a^1}{a^{-1}}=a^2$.
	}
\end{ex}
\begin{ex}%[Danh Trần]%[2D2B1-2]%[Đề minh họa của Bộ GD$-$ĐT] 
	Cho biểu thức $P=\sqrt[4]{x\cdot\sqrt[3]{x^2\cdot\sqrt{x^3}}}$, với $x>0$. Mệnh đề nào dưới đây đúng
	\choice
	{$P=x^{\tfrac{1}{2}}$}
	{\True $P=x^{\tfrac{13}{24}}$}
	{$P=x^{\tfrac{1}{4}}$}
	{$P=x^{\tfrac{2}{3}}$}
	\loigiai{
		Ta có $P=\sqrt[4]{x\cdot\sqrt[3]{x^2\cdot\sqrt{x^3}}}=\sqrt[4]{x\cdot\sqrt[3]{x^2\cdot x^{\tfrac{3}{2}}}}=\sqrt[4]{x\cdot\sqrt[3]{x^{\tfrac{7}{2}}}}=\sqrt[4]{x\cdot x^{\tfrac{7}{6}}}=\sqrt[4]{x^{\tfrac{13}{6}}}=x^{\tfrac{13}{24}}$.
	}
\end{ex}
\begin{ex}%[Danh Trần]%[2D2B1-2]
	Cho $a$, $b$ là các số dương. Rút gọn biểu thức $P=\dfrac{\left(\sqrt[4]{a^3\cdot b^2}\right)^4}{\sqrt[3]{\sqrt{a^{12}\cdot b^6}}}$ được kết quả là 
	\choice
	{$ab^2$}
	{$a^2b$}
	{\True $ab$}
	{$a^2b^2$}
	\loigiai{
		Biến đổi $P=\dfrac{\left(\sqrt[4]{a^3\cdot b^2}\right)^4}{\sqrt[3]{\sqrt{a^{12}\cdot b^6}}}=\dfrac{a^3b^2}{a^2b}=ab$.
	}
\end{ex}
\begin{ex}%[Danh Trần]%[2D2B1-2]
	Giá trị của biểu thức $A=(a+1)^{-1}+(b+1)^{-1}$ với $a=2+\sqrt{3}$ và $b=2-\sqrt{3}$ 
	\choice
	{$3$}
	{$2$}
	{\True $1$}
	{$4$}
	\loigiai{
		Ta có $A=(a+1)^{-1}+(b+1)^{-1}=(3+\sqrt{3})^{-1}+(3-\sqrt{3})^{-1}=\dfrac{1}{3+\sqrt{3}}+\dfrac{1}{3-\sqrt{3}}=1$.
	}
\end{ex}
\begin{ex}%[Danh Trần]%[2D2B1-2]
	Cho các số thực dương $a$ và $b$. Kết quả thu gọn của biểu thức $P=\dfrac{a^{\tfrac{1}{3}}\sqrt{b}+b^{\tfrac{1}{3}}\sqrt{a}}{\sqrt[6]{a}+\sqrt[6]{b}}-\sqrt[3]{ab}$ là
	\choice
	{\True $0$}
	{$-1$}
	{$1$}
	{$-2$}
	\loigiai{
		Ta có $P=\dfrac{a^{\frac{1}{3}} \sqrt{b}+b^{\frac{1}{3}} \sqrt{a}}{\sqrt[6]{a}+\sqrt[6]{b}}-\sqrt[3]{a b}=\dfrac{a^{\frac{1}{3}} b^{\frac{1}{2}}+b^{\frac{1}{3}} a^{\frac{1}{2}}}{a^{\frac{1}{6}}+b^{\frac{1}{6}}}-(a b)^{\frac{1}{3}}=\dfrac{a^{\frac{1}{3}} b^{\frac{1}{3}}\left(b^{\frac{1}{b}}+a^{\frac{1}{8}}\right)}{a^{\frac{1}{6}}+b^{\frac{1}{6}}}-(a b)^{\frac{1}{3}}=a^{\frac{1}{3}} b^{\frac{1}{3}}-(a b)^{\frac{1}{3}}=0$.
	}
\end{ex}
\begin{ex}%[Danh Trần]%[2D2B1-2]
	Cho số thực dương $a$. Biểu thức thu gọn của biểu thức $P=\dfrac{a^{\tfrac{4}{3}}\left(a^{-\tfrac{1}{3}}+a^{\tfrac{2}{3}}\right)}{a^{\tfrac{1}{4}}\left(a^{\tfrac{3}{4}}+a^{-\tfrac{1}{4}}\right)}$ là 
	\choice
	{$1$}
	{$a+1$}
	{$2a$}
	{\True $a$}
	\loigiai{
		Ta có $P=\dfrac{a^{\tfrac{4}{3}}\left(a^{-\tfrac{1}{3}}+a^{\tfrac{2}{3}}\right)}{a^{\tfrac{1}{4}}\left(a^{\tfrac{3}{4}}+a^{-\tfrac{1}{4}}\right)}=\dfrac{a+a^2}{a+1}=\dfrac{a(a+1)}{a+1}=a$.
	}
\end{ex}
\begin{ex}%[Danh Trần]%[2D2B1-2] 
	Cho các số thực dương $a$ và $b$. Biểu thức thu gọn của biểu thức \[P=\left(2a^{\tfrac{1}{4}}-3b^{\tfrac{1}{4}}\right)\cdot\left(2a^{\tfrac{1}{4}}+3b^{\tfrac{1}{4}}\right)\cdot\left(4a^{\tfrac{1}{2}}+9b^{\tfrac{1}{2}}\right)\] có dạng là $P=xa+yb$. Tính $x+y?$ 
	\choice
	{$x+y=97$}
	{\True $x+y=-65$}
	{$x-y=56$}
	{$y-x=-97$}
	\loigiai{
		Ta có
		\allowdisplaybreaks
		\begin{eqnarray*}
			P & = & \left(2a^{\tfrac{1}{4}}-3b^{\tfrac{1}{4}}\right)\cdot\left(2a^{\tfrac{1}{4}}+3b^{\tfrac{1}{4}}\right)\cdot\left(4a^{\tfrac{1}{2}}+9b^{\tfrac{1}{2}}\right)=\left[\left(2a^{\tfrac{1}{4}}\right)^2-\left(3b^{\tfrac{1}{4}}\right)^2\right]\cdot\left(4a^{\tfrac{1}{2}}+9b^{\tfrac{1}{2}}\right)\\
			& = & \left(4a^{\tfrac{1}{2}}-9b^{\tfrac{1}{2}}\right)\cdot\left(4a^{\tfrac{1}{2}}+9b^{\tfrac{1}{2}}\right)=\left(4a^{\tfrac{1}{2}}\right)^2-\left(9b^{\tfrac{1}{2}}\right)^2=16a-81b.
		\end{eqnarray*}
		Do đó $x=16$ và $y=-81$ nên $x+y=-65$.
	}
\end{ex}
\begin{ex}%[Danh Trần]%[2D2B1-2]
	Cho các số thực dương phân biệt $a$ và $b$. Biểu thức thu gọn của biểu thức \[P=\dfrac{\sqrt{a}-\sqrt{b}}{\sqrt[4]{a}-\sqrt[4]{b}}-\dfrac{\sqrt{4a}+\sqrt[4]{16ab}}{\sqrt[4]{a}+\sqrt[4]{b}}\] có dạng $P=m\sqrt[4]{a}+n\sqrt[4]{b}$. Khi đó biểu thức liên hệ giữa $m$ và $n$ là 
	\choice
	{$2m-n=3$}
	{\True $m-n=-2$}
	{$m-n=0$}
	{$m+3n=-1$}
	\loigiai{
		Ta có
		\allowdisplaybreaks
		\begin{eqnarray*}
			P & = & \dfrac{\sqrt{a}-\sqrt{b}}{\sqrt[4]{a}-\sqrt[4]{b}}-\dfrac{\sqrt{4a}+\sqrt[4]{16ab}}{\sqrt[4]{a}+\sqrt[4]{b}}=\dfrac{(\sqrt[4]{a})^2-(\sqrt[4]{b})^2}{\sqrt[4]{a}-\sqrt[4]{b}}-\dfrac{2\sqrt[4]{a}\sqrt[4]{a}+2\sqrt[4]{a}\sqrt[4]{b}}{\sqrt[4]{a}+\sqrt[4]{b}}\\
			& = & \dfrac{(\sqrt[4]{a}-\sqrt[4]{b})(\sqrt[4]{a}+\sqrt[4]{b})}{\sqrt[4]{a}-\sqrt[4]{b}}-\dfrac{2\sqrt[4]{a}(\sqrt[4]{a}+\sqrt[4]{b})}{\sqrt[4]{a}+\sqrt[4]{b}}=\sqrt[4]{a}+\sqrt[4]{b}-2\sqrt[4]{a}=\sqrt[4]{b}-\sqrt[4]{a}.
		\end{eqnarray*}
		Vậy $m=-1$ và $n=1$, khi đó $m-n=-2$.
	}
\end{ex}
\begin{ex}%[Danh Trần]%[2D2Y1-2]
	Cho $b$ là số thực dương. Biểu thức $\dfrac{\sqrt[5]{b^2\sqrt{b}}}{\sqrt[3]{b\sqrt{b}}}$ được viết dưới dạng lũy thừa với số mũ hữu tỉ là 
	\choice
	{$b^{2}$}
	{$b^{1}$}
	{$2$}
	{\True $1$}
	\loigiai{
		Ta có $\dfrac{\sqrt[5]{b^2\sqrt{b}}}{\sqrt[3]{b\sqrt{b}}}=\dfrac{\sqrt[5]{b^2b^{\tfrac{1}{2}}}}{\sqrt[3]{bb^{\tfrac{1}{2}}}}=\dfrac{\sqrt[5]{b^{\tfrac{5}{2}}}}{\sqrt[3]{b^{\tfrac{3}{2}}}}=\dfrac{\left(b^{\tfrac{5}{2}}\right)^{\tfrac{1}{5}}}{\left(b^{\tfrac{3}{2}}\right)^{\tfrac{1}{3}}}=\dfrac{b^{\tfrac{1}{2}}}{b^{\tfrac{1}{2}}}=1$.}
\end{ex}
\begin{ex}%[Danh Trần]%[2D2B1-2]
	Cho $x$ là số thực dương. Biểu thức $\sqrt{x\sqrt{x\sqrt{x\sqrt{x\sqrt{x\sqrt{x\sqrt{x\sqrt{x}}}}}}}}$ được viết dưới dạng lũy thừa với số mũ hữu tỉ là 
	\choice
	{$x^{\tfrac{256}{255}}$}
	{\True $x^{\tfrac{255}{256}}$}
	{$x^{\tfrac{127}{128}}$}
	{$x^{\tfrac{128}{127}}$}
	\loigiai{
		Ta có
		\allowdisplaybreaks
		\begin{eqnarray*}
			& & \sqrt{x\sqrt{x\sqrt{x\sqrt{x\sqrt{x\sqrt{x\sqrt{x\sqrt{x}}}}}}}} =\sqrt{x\sqrt{x\sqrt{x\sqrt{x\sqrt{x\sqrt{x\sqrt{x\cdot x^{\tfrac{1}{2}}}}}}}}} =\sqrt{x\sqrt{x\sqrt{x\sqrt{x\sqrt{x\sqrt{x\sqrt{x^{\tfrac{3}{2}}}}}}}}}\\
			& = & \sqrt{x\sqrt{x\sqrt{x\sqrt{x\sqrt{x\sqrt{x\left(x^{\tfrac{3}{2}}\right)^{\tfrac{1}{2}}}}}}}} =\sqrt{x\sqrt{x\sqrt{x\sqrt{x\sqrt{x\sqrt{x^{\tfrac{7}{4}}}}}}}} =\sqrt{x\sqrt{x\sqrt{x\sqrt{x\sqrt{x\cdot x^{\tfrac{7}{8}}}}}}}\\
			& = & \sqrt{x\sqrt{x\sqrt{x\sqrt{x\sqrt{x^{\tfrac{15}{8}}}}}}} =\sqrt{x\sqrt{x\sqrt{x\sqrt{x\cdot x^{\tfrac{15}{16}}}}}} =\sqrt{x\sqrt{x\sqrt{x\sqrt{x^{\tfrac{31}{16}}}}}} =\sqrt{x\sqrt{x\sqrt{xx^{\tfrac{31}{32}}}}}\\
			& = & \sqrt{x\sqrt{x\sqrt{x^{\tfrac{63}{32}}}}}=\sqrt{x\sqrt{x\cdot x^{\tfrac{63}{64}}}} =\sqrt{x\sqrt{x^{\tfrac{127}{64}}}} =\sqrt{x\sqrt{x^{\tfrac{127}{128}}}} =\sqrt{x\cdot x^{\tfrac{255}{128}}} =\sqrt{x^{\tfrac{255}{128}}} =x^{\tfrac{255}{256}}.
		\end{eqnarray*}
		\begin{nx} $\sqrt{x\sqrt{x\sqrt{x\sqrt{x\sqrt{x\sqrt{x\sqrt{x\sqrt{x}}}}}}}}=x^{\tfrac{2^8-1}{2^8}}=x^{\tfrac{255}{256}}$.
		\end{nx}
	}
\end{ex}
\begin{ex}%[Danh Trần]%[2D2Y1-2]
	Cho hai số thực dương $a$ và $b$. Biểu thức $\sqrt[5]{\dfrac{a}{b}\sqrt[3]{\dfrac{b}{a}\sqrt{\dfrac{a}{b}}}}$ được viết dưới dạng lũy thừa với số mũ hữu tỉ là 
	\choice
	{$x^{\tfrac{7}{30}}$}
	{$\left(\dfrac{a}{b}\right)^{\tfrac{31}{30}}$}
	{$\left(\dfrac{a}{b}\right)^{\tfrac{30}{31}}$}
	{\True $\left(\dfrac{a}{b}\right)^{\tfrac{1}{6}}$}
	\loigiai{
		Ta có $\sqrt[5]{\dfrac{a}{b}\sqrt[3]{\dfrac{b}{a}\sqrt{\dfrac{a}{b}}}}=\sqrt[5]{\dfrac{a}{b}\sqrt[3]{\left(\dfrac{a}{b}\right)^{-1}\left(\dfrac{a}{b}\right)^{\tfrac{1}{2}}}}=\sqrt[5]{\dfrac{a}{b}\sqrt[3]{\left(\dfrac{a}{b}\right)^{\tfrac{-1}{2}}}}=\sqrt[5]{\dfrac{a}{b}\left(\dfrac{a}{b}\right)^{\tfrac{-1}{6}}}=\sqrt[5]{\left(\dfrac{a}{b}\right)^{\tfrac{5}{6}}}=\left(\dfrac{a}{b}\right)^{\tfrac{1}{6}}$.}
\end{ex}
\begin{ex}%[Danh Trần]%[2D2B1-2]
	Cho các số thực dương $a$ và $b$. Rút gọn biểu thức $P=\left(a^{\tfrac{1}{3}}-b^{\tfrac{2}{3}}\right)\cdot\left(a^{\tfrac{2}{3}}+a^{\tfrac{1}{3}}\cdot b^{\tfrac{2}{3}}+b^{\tfrac{4}{3}}\right)$ được kết quả là 
	\choice
	{$a-b$}
	{\True $a-b^2$}
	{$b-a$}
	{$a^3-b^3$}
	\loigiai{
		Ta có $P=\left(a^{\tfrac{1}{3}}-b^{\tfrac{2}{3}}\right)\cdot\left(a^{\tfrac{2}{3}}+a^{\tfrac{1}{3}}\cdot b^{\tfrac{2}{3}}+b^{\tfrac{4}{3}}\right)=\left(a^{\tfrac{1}{3}}\right)^3-\left(b^{\tfrac{2}{3}}\right)^3=a-b^2$.}
\end{ex}
\begin{ex}%[Danh Trần]%[2D2B1-2]
	Cho các số thực dương $a$ và $b$. Rút gọn biểu thức $P=\dfrac{\sqrt{a}-\sqrt{b}}{\sqrt[4]{a}-\sqrt[4]{b}}-\dfrac{\sqrt{a}+\sqrt[4]{ab}}{\sqrt[4]{a}+\sqrt[4]{b}}$ được kết quả là 
	\choice
	{$\sqrt[4]{b}$}
	{$\sqrt[4]{a}-\sqrt[4]{b}$}
	{$b-a$}
	{\True $\sqrt[4]{a}$}
	\loigiai{
		Ta có
		\allowdisplaybreaks
		\begin{eqnarray*}
			P & = & \dfrac{\sqrt{a}-\sqrt{b}}{\sqrt[4]{a}-\sqrt[4]{b}}-\dfrac{\sqrt{a}+\sqrt[4]{ab}}{\sqrt[4]{a}+\sqrt[4]{b}}=\dfrac{\left(\sqrt[4]{a}\right)^2-\left(\sqrt[4]{b}\right)^2}{\sqrt[4]{a}-\sqrt[4]{b}}-\dfrac{\sqrt[4]{a}\sqrt[4]{a}+\sqrt[4]{a}\sqrt[4]{b}}{\sqrt[4]{a}+\sqrt[4]{b}}\\
			& = & \dfrac{\left(\sqrt[4]{a}-\sqrt[4]{b}\right)\left(\sqrt[4]{a}+\sqrt[4]{b}\right)}{\sqrt[4]{a}-\sqrt[4]{b}}-\dfrac{\sqrt[4]{a}\left(\sqrt[4]{a}+\sqrt[4]{b}\right)}{\sqrt[4]{a}+\sqrt[4]{b}} =\sqrt[4]{a}+\sqrt[4]{b}-\sqrt[4]{a}=\sqrt[4]{b}.
		\end{eqnarray*}
	}
\end{ex}
\begin{ex}%[Danh Trần]%[2D2B1-2]
	Cho các số thực dương $a$ và $b$. Rút gọn biểu thức $P=\left(\dfrac{a+b}{\sqrt[3]{a}+\sqrt[3]{b}}-\sqrt[3]{ab}\right)\colon\left(\sqrt[3]{a}-\sqrt[3]{b}\right)^2$ được kết quả là 
	\choice
	{$-1$}
	{\True $1$}
	{$2$}
	{$-2$}
	\loigiai{
		Ta có
		\allowdisplaybreaks
		\begin{eqnarray*}
			P & = & \left(\dfrac{a+b}{\sqrt[3]{a}+\sqrt[3]{b}}-\sqrt[3]{ab}\right)\colon\left(\sqrt[3]{a}-\sqrt[3]{b}\right)^2=\left[\dfrac{\left(\sqrt[3]{a}\right)^3+\left(\sqrt[3]{b}\right)^3}{\sqrt[3]{a}+\sqrt[3]{b}}-\sqrt[3]{ab}\right]\colon\left(\sqrt[3]{a}-\sqrt[3]{b}\right)^2\\
			& = & \left\{\dfrac{\left(\sqrt[3]{a}+\sqrt[3]{b}\right)\left[\left(\sqrt[3]{a}\right)^2-\sqrt[3]{a}\sqrt[3]{b}+\left(\sqrt[3]{b}\right)^2\right]}{\sqrt[3]{a}+\sqrt[3]{b}}-\sqrt[3]{ab}\right\}\colon\left(\sqrt[3]{a}-\sqrt[3]{b}\right)^2\\
			& = & \left[\left(\sqrt[3]{a}\right)^2-\sqrt[3]{ab}+\left(\sqrt[3]{b}\right)^2-\sqrt[3]{ab}\right]\colon\left(\sqrt[3]{a}-\sqrt[3]{b}\right)^2 =\left(\sqrt[3]{a}-\sqrt[3]{b}\right)^2\colon\left(\sqrt[3]{a}-\sqrt[3]{b}\right)^2=1.
		\end{eqnarray*}
	}
\end{ex}
\begin{ex}%[Danh Trần]%[2D2B1-2] 
	Cho các số thực dương $a$ và $b$. Biểu thức thu gọn của biểu thức $P=\dfrac{a^{\tfrac{1}{3}}\sqrt{b}+b^{\tfrac{1}{3}}\sqrt{a}}{\sqrt[6]{a}+\sqrt[6]{b}}-\sqrt[3]{ab}$ là
	\choice
	{\True $0$}
	{$-1$}
	{$1$}
	{$-2$}
	\loigiai{
		Có $P=\dfrac{a^{\tfrac{1}{3}}\sqrt{b}+b^{\tfrac{1}{3}}\sqrt{a}}{\sqrt[6]{a}+\sqrt[6]{b}}-\sqrt[3]{ab}=\dfrac{a^{\tfrac{1}{3}}b^{\tfrac{1}{2}}+b^{\tfrac{1}{3}}a^{\tfrac{1}{2}}}{a^{\tfrac{1}{6}}+b^{\tfrac{1}{6}}}-(ab)^{\tfrac{1}{3}}=\dfrac{a^{\tfrac{1}{3}}b^{\tfrac{1}{3}}\left(b^{\tfrac{1}{6}}+a^{\tfrac{1}{6}}\right)}{a^{\tfrac{1}{6}}+b^{\tfrac{1}{6}}}-(ab)^{\tfrac{1}{3}}=a^{\tfrac{1}{3}}b^{\tfrac{1}{3}}-(ab)^{\tfrac{1}{3}}=0$.}
\end{ex}
\begin{ex}%[Danh Trần]%[2D2B1-2] 
	Cho số thực dương $a$. Biểu thức thu gọn của biểu thức $P=\dfrac{a^{\tfrac{4}{3}}\left(a^{-\tfrac{1}{3}}+a^{\tfrac{2}{3}}\right)}{a^{\tfrac{1}{4}}\left(a^{\tfrac{3}{4}}+a^{-\tfrac{1}{4}}\right)}$ là 
	\choice
	{$1$}
	{$a+1$}
	{$2a$}
	{\True $a$}
	\loigiai{
		Ta có $P=\dfrac{a^{\tfrac{4}{3}}\left(a^{-\tfrac{1}{3}}+a^{\tfrac{2}{3}}\right)}{a^{\tfrac{1}{4}}\left(a^{\tfrac{3}{4}}+a^{-\tfrac{1}{4}}\right)}=\dfrac{a+a^2}{a+1}=\dfrac{a(a+1)}{a+1}=a$.}
\end{ex}
\begin{ex}%[Danh Trần]%[2D2B1-2] 
	Cho $a>0,b>0$. Biểu thức thu gọn của biểu thức $P=\left(a^{\tfrac{1}{4}}-b^{\tfrac{1}{4}}\right)\cdot\left(a^{\tfrac{1}{4}}+b^{\tfrac{1}{4}}\right)\cdot\left(a^{\tfrac{1}{2}}+b^{\tfrac{1}{2}}\right)$ là 
	\choice
	{$\sqrt[10]{a}-\sqrt[10]{b}$}
	{$\sqrt{a}-\sqrt{b}$}
	{\True $a-b$}
	{$\sqrt[8]{a}-\sqrt[8]{b}$}
	\loigiai{
		Ta có
		\allowdisplaybreaks
		\begin{eqnarray*}
			P & = & \left(a^{\tfrac{1}{4}}-b^{\tfrac{1}{4}}\right)\cdot\left(a^{\tfrac{1}{4}}+b^{\tfrac{1}{4}}\right)\cdot\left(a^{\tfrac{1}{2}}+b^{\tfrac{1}{2}}\right)=\left[\left(a^{\tfrac{1}{4}}\right)^2-\left(b^{\tfrac{1}{4}}\right)^2\right]\cdot\left(a^{\tfrac{1}{2}}+b^{\tfrac{1}{2}}\right)\\
			& = & \left(a^{\tfrac{1}{2}}-b^{\tfrac{1}{2}}\right)\cdot\left(a^{\tfrac{1}{2}}+b^{\tfrac{1}{2}}\right)		=\left(a^{\tfrac{1}{2}}\right)^2-\left(b^{\tfrac{1}{2}}\right)^2=a-b.
	\end{eqnarray*}}
\end{ex}
\begin{ex}%[Danh Trần]%[2D2B1-2] 
	Cho $a>0$ và $b>0$. Biểu thức thu gọn của biểu thức $P=\left(a^{\tfrac{1}{3}}+b^{\tfrac{1}{3}}\right)\colon\left(2+\sqrt[3]{\dfrac{a}{b}}+\sqrt[3]{\dfrac{b}{a}}\right)$ là 
	\choice
	{$\sqrt[3]{ab}$}
	{\True $\dfrac{\sqrt[3]{ab}}{\sqrt[3]{a}+\sqrt[3]{b}}$}
	{$\dfrac{\sqrt[3]{ab}}{\left(\sqrt[3]{a}+\sqrt[3]{b}\right)^3}$}
	{$\sqrt[3]{ab}\left(\sqrt[3]{a}+\sqrt[3]{b}\right)$}
	\loigiai{
		Ta có
		\allowdisplaybreaks
		\begin{eqnarray*}
			P & = & \left(a^{\tfrac{1}{3}}+b^{\tfrac{1}{3}}\right)\colon\left(2+\sqrt[3]{\dfrac{a}{b}}+\sqrt[3]{\dfrac{b}{a}}\right)=\left(\sqrt[3]{a}+\sqrt[3]{b}\right)\colon\left(2+\dfrac{\sqrt[3]{a}}{\sqrt[3]{b}}+\dfrac{\sqrt[3]{b}}{\sqrt[3]{a}}\right)\\
			& = & \left(\sqrt[3]{a}+\sqrt[3]{b}\right)\colon\left(\dfrac{2\sqrt[3]{a}\sqrt[3]{b}+\sqrt[3]{a}+\sqrt[3]{b}}{\sqrt[3]{a}\sqrt[3]{b}}\right)
			=\left(\sqrt[3]{a}+\sqrt[3]{b}\right)\colon\dfrac{\left(\sqrt[3]{a}+\sqrt[3]{b}\right)^2}{\sqrt[3]{a}\sqrt[3]{b}}\\
			& = & \left(\sqrt[3]{a}+\sqrt[3]{b}\right)\cdot\dfrac{\sqrt[3]{a}\sqrt[3]{b}}{\left(\sqrt[3]{a}+\sqrt[3]{b}\right)^2}=\dfrac{\sqrt[3]{a}\sqrt[3]{b}}{\sqrt[3]{a}+\sqrt[3]{b}}.
		\end{eqnarray*}
	}
\end{ex}
\begin{ex}%[Danh Trần]%[2D2B1-2] 
	Cho $a>0,b>0$ và $a\neq b$. Biểu thức thu gọn của biểu thức $P=\dfrac{\sqrt[3]{a}-\sqrt[3]{b}}{\sqrt[6]{a}-\sqrt[6]{b}}$ là 
	\choice
	{\True $\sqrt[6]{a}+\sqrt[6]{b}$}
	{$\sqrt[6]{a}-\sqrt[6]{b}$}
	{$\sqrt[3]{b}-\sqrt[3]{a}$}
	{$\sqrt[3]{a}+\sqrt[3]{b}$}
	\loigiai{
		Ta có $P=\dfrac{\sqrt[3]{a}-\sqrt[3]{b}}{\sqrt[6]{a}-\sqrt[6]{b}}=\dfrac{\left(\sqrt[6]{a}\right)^2-\left(\sqrt[6]{b}\right)^2}{\sqrt[6]{a}-\sqrt[6]{b}}=\dfrac{\left(\sqrt[6]{a}-\sqrt[6]{b}\right)\left(\sqrt[6]{a}+\sqrt[6]{b}\right)}{\sqrt[6]{a}-\sqrt[6]{b}}=\sqrt[6]{a}+\sqrt[6]{b}$.}
\end{ex}
\begin{dang}{So sánh các lũy thừa}
	
\end{dang}
\begin{ex}%[Danh Trần]%[2D2B1-3] 
	Với giá trị nào của $a$ thì đẳng thức $\sqrt{a\cdot\sqrt[3]{a\cdot\sqrt[4]{a}}}=\sqrt[24]{2^5}\cdot\dfrac{1}{\sqrt{2^{-1}}}$ đúng?
	\choice
	{$a=1$}
	{\True $a=2$}
	{$a=0$}
	{$a=3$}
	\loigiai{
		Ta có $\heva{&\sqrt{a\cdot\sqrt[3]{a\cdot\sqrt[4]{a}}}=\left[a\cdot\left(a\cdot a^{\tfrac{1}{4}}\right)^{\tfrac{1}{3}}\right]^{\tfrac{1}{2}}=a^{\tfrac{17}{24}}\\&\sqrt[24]{2^5}\cdot\dfrac{1}{\sqrt{2^{-1}}}=2^{\tfrac{5}{24}}{\cdot 2}^{\tfrac{1}{2}}=2^{\tfrac{17}{24}}}$ nên $\sqrt{a\cdot\sqrt[3]{a\cdot\sqrt[4]{a}}}=\sqrt[24]{2^5}\cdot\dfrac{1}{\sqrt{2^{-1}}}\Leftrightarrow a=2$.}
\end{ex}
\begin{ex}%[Danh Trần]%[2D2Y1-3]
	So sánh hai số $m$ và $n$ nếu $3,2^m<3,2^n$.
	\choice
	{$m>n$}
	{$m=n$}
	{\True $m<n$}
	{Không so sánh được}
	\loigiai{
		Do $3,2>1$ nên $3,2^m<3,2^n\Leftrightarrow m<n$.}
\end{ex}
\begin{ex}%[Danh Trần]%[2D2Y1-3]
	So sánh hai số $m$ và $n$ nếu $(\sqrt{2})^m<(\sqrt{2})^n$.
	\choice
	{$m>n$}
	{$m=n$}
	{\True $m<n$}
	{Không so sánh được}
	\loigiai{
		Do $\sqrt{2}>1$ nên $(\sqrt{2})^m<(\sqrt{2})^n\Leftrightarrow m<n$.}
\end{ex}
\begin{ex}%[Danh Trần]%[2D2Y1-3]
	So sánh hai số $m$ và $n$ nếu $\left(\dfrac{1}{9}\right)^m>\left(\dfrac{1}{9}\right)^n$.
	\choice
	{Không so sánh được}
	{$m=n$}
	{$m>n$}
	{\True $m<n$}
	\loigiai{
		Do $0<\dfrac{1}{9}<1$ nên $\left(\dfrac{1}{9}\right)^m>\left(\dfrac{1}{9}\right)^n\Leftrightarrow m<n$.}
\end{ex}
\begin{ex}%[Danh Trần]%[2D2Y1-3]
	So sánh hai số $m$ và $n$ nếu $\left(\dfrac{\sqrt{3}}{2}\right)^m>\left(\dfrac{\sqrt{3}}{2}\right)^n$.
	\choice
	{\True $m<n$}
	{$m=n$}
	{$m>n$}
	{Không so sánh được}
	\loigiai{
		Do $0<\dfrac{\sqrt{3}}{2}<1$ nên $\left(\dfrac{\sqrt{3}}{2}\right)^m>\left(\dfrac{\sqrt{3}}{2}\right)^n\Leftrightarrow m<n$.}
\end{ex}
\begin{ex}%[Danh Trần]%[2D2Y1-3]
	So sánh hai số $m$ và $n$ nếu $(\sqrt{5}-1)^m<(\sqrt{5}-1)^n$.
	\choice
	{$m=n$}
	{\True $m<n$}
	{$m>n$}
	{Không so sánh được}
	\loigiai{
		Do $\sqrt{5}-1>1$ nên $(\sqrt{5}-1)^m<(\sqrt{5}-1)^n\Leftrightarrow m<n$.}
\end{ex}
\begin{ex}%[Danh Trần]%[2D2Y1-3]
	So sánh hai số $m$ và $n$ nếu $(\sqrt{2}-1)^m<(\sqrt{2}-1)^n$.
	\choice
	{\True $m>n$}
	{$m=n$}
	{$m<n$}
	{Không so sánh được}
	\loigiai{
		Do $0<\sqrt{2}-1<1$ nên $(\sqrt{2}-1)^m<(\sqrt{2}-1)^n\Leftrightarrow m>n$.}
\end{ex}
\begin{ex}%[Danh Trần]%[2D2Y1-3]
	Kết luận nào đúng về số thực $a$ nếu $(a-1)^{-\tfrac{2}{3}}<(a-1)^{-\tfrac{1}{3}}$?
	\choice
	{\True $a>2$}
	{$a>0$}
	{$a>1$}
	{$1<a<2$}
	\loigiai{
		Do $-\dfrac{2}{3} <-\dfrac{1}{3}$ và số mũ không nguyên nên $(a-1)^{-\tfrac{2}{3}}<(a-1)^{-\tfrac{1}{3}}$ khi $a-1>1\Leftrightarrow a>2$.}
\end{ex}
\begin{ex}%[Danh Trần]%[2D2Y1-3]
	Kết luận nào đúng về số thực $a$ nếu $(2a+1)^{-3}>(2a+1)^{-1}$?
	\choice
	{\True $\hoac{&-\dfrac{1}{2}<a<0\\&a <-1}$}
	{$-\dfrac{1}{2}<a<0$}
	{$\hoac{&0<a<1\\&a <-1}$}
	{$a <-1$}
	\loigiai{
		Do $-3 <-1$ và số mũ nguyên âm nên $(2a+1)^{-3}>(2a+1)^{-1}$ khi $\hoac{&0<2a+1<1\\&2a+1 <-1}\Leftrightarrow\hoac{&-\dfrac{1}{2}<a<0\\&a <-1.}$}
\end{ex}
\begin{ex}%[Danh Trần]%[2D2Y1-3]
	Kết luận nào đúng về số thực $a$ nếu $\left(\dfrac{1}{a}\right)^{-0,2}<a^2$?
	\choice
	{$0<a<1$}
	{$a>0$}
	{\True $a>1$}
	{$a<0$}
	\loigiai{
		$\left(\dfrac{1}{a}\right)^{-0,2}<a^2\Leftrightarrow a^{0,2}<a^2$.\\
		Do $0,2<2$ và có số mũ không nguyên nên $a^{0,2}<a^2$ khi $a>1$.}
\end{ex}
\begin{ex}%[Danh Trần]%[2D2Y1-3]
	Kết luận nào đúng về số thực $a$ nếu $(1-a)^{-\tfrac{1}{3}}>(1-a)^{-\tfrac{1}{2}}$?
	\choice
	{$a<1$}
	{$a>0$}
	{$0<a<1$}
	{\True $a>1$}
	\loigiai{
		Do $-\dfrac{1}{3} >-\dfrac{1}{2}$ và số mũ không nguyên $\Rightarrow(1-a)^{-\tfrac{1}{3}}>(1-a)^{-\tfrac{1}{2}}\Leftrightarrow a>1$.}
\end{ex}
\begin{ex}%[Danh Trần]%[2D2Y1-3]
	Kết luận nào đúng về số thực $a$ nếu $(2-a)^{\tfrac{3}{4}}>(2-a)^2$?
	\choice
	{$a>1$}
	{$0<a<1$}
	{\True $1<a<2$}
	{$a<1$}
	\loigiai{
		Do $\dfrac{3}{4}<2$ và có số mũ không nguyên $\Rightarrow(2-a)^{\tfrac{3}{4}}>(2-a)^2\Leftrightarrow 0<2-a<1\Leftrightarrow-2 <-a <-1\Leftrightarrow 2>a>1$.}
\end{ex}
\begin{ex}%[Danh Trần]%[2D2Y1-3]
	Kết luận nào đúng về số thực $a$ nếu $\left(\dfrac{1}{a}\right)^{\tfrac{1}{2}}>\left(\dfrac{1}{a}\right)^{-\tfrac{1}{2}}$?
	\choice
	{$1<a<2$}
	{$a<1$}
	{$a>1$}
	{\True $0<a<1$}
	\loigiai{
		Do $\dfrac{1}{2} >-\dfrac{1}{2}$ và số mũ không nguyên $\Rightarrow\left(\dfrac{1}{a}\right)^{\tfrac{1}{2}}>\left(\dfrac{1}{a}\right)^{-\tfrac{1}{2}}\Leftrightarrow\dfrac{1}{a}>1\Leftrightarrow 0<a<1$.}
\end{ex}
\begin{ex}%[Danh Trần]%[2D2Y1-3]
	Kết luận nào đúng về số thực $a$ nếu $a^{\sqrt{3}}>a^{\sqrt{7}}$?
	\choice
	{$a<1$}
	{\True $0<a<1$}
	{$a>1$}
	{$1<a<2$}
	\loigiai{
		Do $\sqrt{3}<\sqrt{7}$ và số mũ không nguyên $\Rightarrow a^{\sqrt{3}}>a^{\sqrt{7}}\Leftrightarrow 0<a<1$.}
\end{ex}
\begin{ex}%[Danh Trần]%[2D2Y1-3]
	Kết luận nào đúng về số thực $a$ nếu $a^{-\tfrac{1}{17}}>a^{-\tfrac{1}{8}}$?
	\choice
	{\True $a>1$}
	{$a<1$}
	{$0<a<1$}
	{$1<a<2$}
	\loigiai{
		Do $-\dfrac{1}{17} >-\dfrac{1}{8}$ và số mũ không nguyên nên $a^{-\tfrac{1}{17}}>a^{-\tfrac{1}{8}}$ khi $a>1$.}
\end{ex}
\begin{ex}%[Danh Trần]%[2D2Y1-3]
	Kết luận nào đúng về số thực $a$ nếu $a^{-0,25}>a^{-\sqrt{3}}$?
	\choice
	{$1<a<2$}
	{$a<1$}
	{$0<a<1$}
	{\True $a>1$}
	\loigiai{
		Do $-0,25 >-\sqrt{3}$ và số mũ không nguyên nên $a^{-0,25}>a^{-\sqrt{3}}$ khi $a>1$.}
\end{ex}
\begin{ex}%[Danh Trần]%[2D2Y1-3]
	Cho $\pi^{\alpha}>\pi^{\beta}$. Kết luận nào sau đây đúng?
	\choice
	{$\alpha\cdot\beta=1$}
	{\True $\alpha>\beta$}
	{$\alpha<\beta$}
	{$\alpha+\beta=0$}
	\loigiai{
		Vì $\pi\approx 3,14>0$ nên $\pi^{\alpha}>\pi^{\beta}\Leftrightarrow\alpha>\beta$.}
\end{ex}
\begin{ex}%[Danh Trần]%[2D2B1-3]
	Tìm tất cả các giá trị của $a$ thỏa mãn $\sqrt[4]{a^7}>\sqrt[3]{a^2}$. 
	\choice
	{$a=0$}
	{$a<0$}
	{\True $a>1$}
	{$0<a<1$}
	\loigiai{
		Ta có $\sqrt[15]{a^7}>\sqrt[5]{a^2}\Leftrightarrow a^{\tfrac{7}{15}}>a^{\tfrac{2}{5}}\Leftrightarrow a^{\tfrac{7}{15}}>a^{\tfrac{6}{15}}\Rightarrow a>1$.}
\end{ex}
\begin{ex}%[Danh Trần]%[2D2B1-3]
	Tìm tất cả các giá trị của $a$ thỏa mãn $(a-1)^{-\tfrac{2}{3}}<(a-1)^{-\tfrac{1}{3}}$. 
	\choice
	{$a>2$}
	{$a>1$}
	{\True $1<a<2$}
	{$0<a<1$}
	\loigiai{
		Ta có $-\dfrac{2}{3} <-\dfrac{1}{3}$, kết hợp với $(a-1)^{-\tfrac{2}{3}}<(a-1)^{-\tfrac{1}{3}}$.\\
		Suy ra hàm số đặc trưng $y=(a-1)^x$ đồng biến nên cơ số $a-1>1\Leftrightarrow a>2$.}
\end{ex}
\begin{ex}[THPT chuyên Lê Thánh Tông]%[Danh Trần]%[2D2B1-2]
	Cho biểu thức $P=\sqrt{x\sqrt[3]{x^2\sqrt[3]{x^3}}} (x>0)$. Xác định $k$ sao cho biểu thức $P=x^{\tfrac{23}{24}}$. 
	\choice
	{$k=2$}
	{$k=6$}
	{\True $k=4$}
	{Không tồn tại $k$}
	\loigiai{
		Ta có $P=\sqrt{x\sqrt[3]{x^2\sqrt[3]{x^3}}}=\sqrt{x\sqrt[3]{x^{2+\tfrac{3}{2}}}}=\sqrt{x^{1+\tfrac{2\sqrt{4}-3}{k}}}=x^{\tfrac{5k+3}{6k}}$.\\
		Yêu cầu bài toán xảy ra khi: $\dfrac{5k+3}{6k}=\dfrac{23}{24}\Leftrightarrow k=4$.}
\end{ex}
\begin{ex}[THPT Tiên Lãng]%[Danh Trần]%[2D2B1-3]
	Tìm tập tất cả các giá trị của $a$ để $\sqrt[3]{a^3}>\sqrt{a^2}$.
	\choice
	{$a>0$}
	{\True $0<a<1$}
	{$a>1$}
	{$\dfrac{5}{21}<a<\dfrac{2}{7}$}
	\loigiai{
		Ta có $\sqrt[7]{a^2}=\sqrt[21]{a^6}$ và $\sqrt[2]{a^5}>\sqrt[7]{a^2}\Leftrightarrow\sqrt[3]{a^5}>\sqrt[4]{a^6}$. Mà $5<6$ nên $0<a<1$.}
\end{ex}
\begin{ex}%[Danh Trần]%[2D2Y1-3]
	Trong các khẳng định sau, khẳng định nào \textbf{sai}? 
	\choice
	{Nếu $a>1$ thì $a^x>a^y$ khi và chỉ khi $x>y$}
	{Nếu $a>1$ thì $a^x\leq a^y$ khi và chỉ khi $x\leq y$}
	{\True Nếu $0<a<1$ thì $a^x>a^y$ khi và chỉ khi $x>y$}
	{Nếu $0<a\neq 1$ thì $a^x=a^y$ khi và chỉ khi $x=y$}
	\loigiai{
		Theo tính chất của lũy thừa với số mũ thực, khi $0<a<1$ thì $a^x>a^y$ khi và chỉ khi $x<y$.}
\end{ex}
\begin{ex}%[Danh Trần]%[2D2Y1-3] 
	Khẳng định nào sau đây sai?
	\choice
	{$a^{\circ}=1$, $\forall a$}
	{$a>1\Rightarrow a^2>1$}
	{$2\sqrt{3}<3\sqrt{2}$}
	{\True $\left(\dfrac{1}{4}\right)^{-1}<\left(\dfrac{1}{4}\right)^2$}
	\loigiai{
		Ta có $4=\left(\dfrac{1}{4}\right)^{-1}<\left(\dfrac{1}{4}\right)^2=\dfrac{1}{16}$ là vô lý nên $\left(\dfrac{1}{4}\right)^{-1}<\left(\dfrac{1}{4}\right)^2$ sai.}
\end{ex}
\begin{ex}%[Danh Trần]%[2D2B1-3]
	Nếu $\left(2\sqrt{3}-1\right)^{a+2}<2\sqrt{3}-1$ thì
	\choice
	{\True $a <-1$}
	{$a<1$}
	{$a >-1$}
	{$a\geq-1$}
	\loigiai{
		Do $2\sqrt{3}-1>1$ nên $\left(2\sqrt{3}-1\right)^{a+2}<2\sqrt{3}-1\Leftrightarrow a+2<1\Leftrightarrow a <-1$.}
\end{ex}
\begin{ex}%[Danh Trần]%[2D2B1-3] 
	Nếu $a^{\tfrac{1}{2}}>a^{\tfrac{1}{6}}$ và $b^{\sqrt{2}}>b^{\sqrt{3}}$ thì
	\choice
	{\True $a>1$, $0<b<1$}
	{$a>1$, $b<1$}
	{$0<a<1$, $b<1$}
	{$a<1$, $0<b<1$}
	\loigiai{
		Do $\dfrac{1}{2}>\dfrac{1}{6}$ nên $a^{\tfrac{1}{2}}>a^{\tfrac{1}{6}}\Rightarrow a>1$ và vì $\sqrt{2}<\sqrt{3}$ nên $b^{\sqrt{2}}>b^{\sqrt{3}}\Rightarrow 0<b<1$}
\end{ex}
\begin{ex}%[Danh Trần]%[2D2B1-3] 
	Cho $3^{|\alpha|}<27$. Mệnh đề nào sau đây là đúng?
	\choice
	{$\hoac{&\alpha <-3\\&\alpha>3}$}
	{$\alpha>3$}
	{$\alpha<3$}
	{\True $-3<\alpha<3$}
	\loigiai{
		Ta có $3^{|\alpha|}<27\Leftrightarrow 3^{|\alpha|}<3^3\Leftrightarrow|\alpha|<3\Leftrightarrow-3<\alpha<3$.}
\end{ex}
\begin{ex}%[Danh Trần]%[2D2B1-3] 
	Trong các mệnh đề sau, mệnh đề nào \textbf{sai}?
	\begin{enumerate}[(I):]
		\item $\sqrt[3]{-0,4}>\sqrt[5]{-0,3}$.
		\item $\sqrt[5]{-5}>\sqrt[3]{-3}$.
		\item $\sqrt[3]{-2}>\sqrt[5]{-3}$.
		\item $\sqrt[3]{-5}>\sqrt[3]{-3}$.
	\end{enumerate}
	\choice
	{(I) và (IV)}
	{(I) và (III)}
	{(IV)}
	{\True (III) và (IV)}
	\loigiai{
		Dùng máy tính kiểm tra kết quả.}
\end{ex}
\begin{ex}%[Danh Trần]%[2D2B1-3] 
	Trong các khẳng định sau đây, khẳng định nào sai?
	\choice
	{$(0,01)^{-\sqrt{2}}>(10)^{-\sqrt{2}}$}
	{$(0,01)^{-\sqrt{2}}<(10)^{-\sqrt{2}}$}
	{\True $(0,01)^{-\sqrt{2}}=(10)^{-\sqrt{2}}$}
	{$a^{\circ}=1,\forall a\neq 0$}
	\loigiai{
		Dùng máy tính kiểm tra kết quả.}
\end{ex}
\begin{ex}%[Danh Trần]%[2D2B1-3]  
	Trong các khẳng định sau đây, khẳng định nào đúng?
	\choice
	{$(2-\sqrt{2})^3<(2-\sqrt{2})^4$}
	{\True $\left(\sqrt{11}-\sqrt{2}\right)^6>\left(\sqrt{11}-\sqrt{2}\right)^7$}
	{$(4-\sqrt{2})^3<(4-\sqrt{2})^4$}
	{$\left(\sqrt{3}-\sqrt{2}\right)^4<\left(\sqrt{3}-\sqrt{2}\right)^5$}
	\loigiai{
		Dùng máy tính kiểm tra kết quả.}
\end{ex}
\begin{ex}%[Danh Trần]%[2D2B1-3] 
	Nếu $\left(\sqrt{3}-\sqrt{2}\right)^{2m-2}<\sqrt{3}+\sqrt{2}$ thì
	\choice
	{$m>\dfrac{3}{2}$}
	{$m<\dfrac{1}{2}$}
	{\True $m>\dfrac{1}{2}$}
	{$m\neq\dfrac{3}{2}$}
	\loigiai{
		Ta có $\sqrt{3}+\sqrt{2}=\dfrac{1}{\sqrt{3}-\sqrt{2}}\Rightarrow\left(\sqrt{3}-\sqrt{2}\right)^{2m-2}<\left(\sqrt{3}-\sqrt{2}\right)^{-1}\Leftrightarrow 2m-2 >-1\Leftrightarrow m>\dfrac{1}{2}$.}
\end{ex}
\begin{ex}%[Danh Trần]%[2D2B1-3] 
	Nếu $\left(\sqrt{3}-\sqrt{2}\right)^x>\sqrt{3}+\sqrt{2}$ thì
	\choice
	{\True $\forall x\in\mathbb{R}$}
	{$x<1$}
	{$x >-1$}
	{$x <-1$}
	\loigiai{
		Vì $\left(\sqrt{3}-\sqrt{2}\right)\cdot\left(\sqrt{3}+\sqrt{2}\right)=1\Leftrightarrow\left(\sqrt{3}+\sqrt{2}\right)=\dfrac{1}{\left(\sqrt{3}+\sqrt{2}\right)}$ nên.\\
		$\left(\sqrt{3}-\sqrt{2}\right)^x>\sqrt{3}+\sqrt{2}\Leftrightarrow\left(\sqrt{3}-\sqrt{2}\right)^x>\dfrac{1}{\sqrt{3}-\sqrt{2}}\Leftrightarrow\left(\sqrt{3}-\sqrt{2}\right)^x>\left(\sqrt{3}-\sqrt{2}\right)^{-1}$.\\
		Mặt khác $0<\sqrt{3}-\sqrt{2}<1\Rightarrow x <-1$.}
\end{ex}
\begin{ex}%[Danh Trần]%[2D2K1-3] 
	Với giá trị nào của $a$ thì phương trình $2^{ax^2-4x-2a}=\dfrac{1}{(\sqrt{2})^{-4}}$ có hai nghiệm thực phân biệt. 
	\choice
	{\True $a\neq 0$}
	{$\forall a\in\mathbb{R}$}
	{$a\geq 0$}
	{$a>0$}
	\loigiai{
		Ta có $2^{ax^2-4x-2a}=\dfrac{1}{(\sqrt{2})^{-4}}$ (*) $\Leftrightarrow 2^{ax^2-4x-2a}=2^2\Leftrightarrow ax^2-4x-2a=2\Leftrightarrow ax^2-4x-2(a+1)=0$.\\
		PT (*) có hai nghiệm phân biệt $ax^2-4x-2(a+1)=0\Leftrightarrow\heva{&a\neq 0\\&2a^2+2a+4>o}\Leftrightarrow a\neq 0$.}
\end{ex}
\begin{dang}{Bài toán lãi suất}
	
\end{dang}
\begin{vd}%[Danh Trần]%[2D2B1-1] 
	Bà Hoa gửi $100$ triệu vào tài khoản định kỳ tính lãi kép với lãi suất là $8\%$/năm. Tính số tiền lãi thu được sau $10$ năm.
	\loigiai{
		Áp dụng công thức tính lãi kép, sau 10 năm số tiền cả gốc và lãi bà Hoa thu về là 
		\[A(1+r)^n=100\cdot(1+0,08)^{10}\approx215,89\]
		Suy ra số tiền lãi bà Hoa thu về sau 10 năm là 
		\[A(1+r)^n-A=100(1+0,08)^{10}-10=115,892\text{ triệu.}\]
	}
\end{vd}
\begin{ex}%[Danh Trần]%[2D2B1-1]
	Một người lần đầu gửi vào ngân hàng $100$ triệu đồng với kì hạn 3 tháng, lãi suất $2\%$ một quý. Biết rằng nếu không rút tiền ra khỏi ngân hàng thì cứ sau mỗi quý số tiền lãi sẽ được nhập vào gốc để tính lãi cho quý tiếp theo. Sau đúng $6$ tháng, người đó gửi thêm $100$ triệu đồng với kỳ hạn và lãi suất như trước đó. Tổng số tiền người đó nhận được $1$ năm sau khi gửi tiền (cả vốn lẫn lãi) gần nhất với kết quả nào sau đây?
	\choice
	{$210$ triệu}
	{$220$ triệu}
	{\True $212$ triệu}
	{$216$ triệu}
	\loigiai{
		Số tiền nhận về sau $1$ năm của $100$ triệu gửi trước là $100(1+2\%)^4$ triệu.\\
		Số tiền nhận về sau $6$ tháng của $100$ triệu gửi sau là $100(1+2\%)^2$ triệu.\\
		Vậy tổng số tiền là $100(1+2\%)^4+100(1+2\%)^2\approx 212,283$ triệu.
	}
\end{ex}
\begin{ex}%[Danh Trần]%[2D2B1-1] 
	Bác An đem gửi tổng số tiền $320$ triệu đồng ở hai loại kỳ hạn khác nhau. Bác gửi $140$ triệu đồng theo kỳ hạn ba tháng với lãi suất $2,1\%$ một quý. Số tiền còn lại bác An gửi theo kỳ hạn một tháng với lãi suất $0,73\%$ một tháng. Biết rằng nếu không rút tiền ra khỏi ngân hàng thì cứ sau mỗi kỳ hạn số tiền lãi sẽ được nhập vào gốc để tính lãi cho kỳ hạn tiếp theo. Sau $15$ tháng kể từ ngày gửi bác An đi rút tiền. Tính gần đúng đến hàng đơn vị tổng số tiền lãi thu được của bác An. 
	\choice
	{$36080251$ đồng}
	{$36080254$ đồng}
	{$36080255$ đồng}
	{\True $36080253$ đồng}
	\loigiai{
		Số tiền nhận về sau 15 tháng của 140 triệu gửi trước là $140\cdot (1+2,1\%)^5$ triệu.\\
		Số tiền nhận về sau 15 tháng của 180 triệu gửi sau là $180\cdot (1+0,73\%)^{15}$ triệu.\\
		Suy ra tổng số tiền cả vốn lẫn lãi mà bác An thu được theo đơn vị triệu đồng là
		\[140\cdot (1+2,1\%)^5+180\cdot (1+0,73\%)^{15}\approx 356,080253.\]
		Suy ra số tiền lãi: $356,080253-320=36,080253$ triệu đồng $=36080253$ đồng.}
\end{ex}
\begin{ex}[Đề Minh họa $-$ 2017]%[Danh Trần]%[2D2K1-1] 
	Ông A vay ngắn hạn ngân hàng $100$ triệu đồng, với lãi suất $12$\%/năm. Ông muốn hoàn nợ cho ngân hàng theo cách: Sau đúng một tháng kể từ ngày vay, ông bắt đầu hoàn nợ, hai lần hoàn nợ liên tiếp cách nhau đúng một tháng, số tiền hoàn nợ ở mỗi lần là như nhau và trả hết tiền nợ sau đúng $3$ tháng kể từ ngày vay. Hỏi, theo cách đó, số tiền $m$ mà ông A phải trả cho ngân hàng trong mỗi lần hoàn nợ là bao nhiêu? Biết rằng lãi suất ngân hàng không thay đổi trong thời gian ông A hoàn nợ. 
	\choice
	{$m=\dfrac{100\cdot (1,01)^3}{3}$ (triệu đồng)}
	{\True $m=\dfrac{(1,01)^3}{(1,01)^3-1}$ (triệu đồng)}
	{$m=\dfrac{100\times 1,03}{3}$ (triệu đồng)}
	{$m=\dfrac{120\cdot (1,12)^3}{(1,12)^3-1}$ (triệu đồng)}
	\loigiai{
		Theo đề ta có ông A trả hết tiền sau $3$ tháng, vậy ông A hoàn nợ $3$ lần.\\
		Với lãi suất $12\%$/năm suy ra lãi suất một tháng là $1\%$.
		\begin{itemize}
			\item Hoàn nợ lần $3$, ta có
			\begin{itemize}
				\item Tổng tiền cần trả (gốc và lãi) là $100+0,01+100=100\cdot 1,01$ (triệu đồng).
				\item Số tiền dư là $100\cdot 1,01-m$ (triệu đồng).
			\end{itemize}
			\item Hoàn nợ lần $2$, ta có
			\begin{itemize}
				\item Tổng tiền cần trả (gốc và lãi) tính theo đơn vị triệu đồng là \[(100\cdot1,01-m)\cdot0,01+(100\cdot1,01-m)=(100\cdot1,01-m)\cdot1,01=100\cdot(1,01)^2-1,01\cdot m.\]
				\item Số tiền dư là $100\cdot (1,01)^2-1,01\cdot m-m$ (triệu đồng).
			\end{itemize}
			\item Hoàn nợ lần $3$, ta có
			\begin{itemize}
				\item Tổng tiền cần trả (gốc và lãi) tính theo đơn vị triệu đồng là
				\[[100\cdot(1,01)^2-1,01\cdot m-m]\cdot1,01=100(1,01)^3-m(1,01)^2-1,01m.\]
				\item Số tiền dư tính theo đơn vị triệu đồng là
				\allowdisplaybreaks
				\begin{eqnarray*}
					& & 100(1,01)^3-m(1,01)^2-1,01m-m=0\Leftrightarrow m=\dfrac{100\cdot(1,01)^3}{(1,01)^2+1,01+1}\\
					& \Leftrightarrow & m=\dfrac{100\cdot(1,01)^3(1,01-1)}{[(1,01)^2+1,01+1](1,01-1)}=\dfrac{(1,01)^3}{(1,01)^3-1}.
				\end{eqnarray*}
			\end{itemize}
		\end{itemize}
	}
\end{ex}
\Closesolutionfile{ans}
%\begin{indapan}
%	{10}{ans/ansCD2D2-1}
%\end{indapan}