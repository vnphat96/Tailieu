\chude{TÍNH DIỆN TÍCH GIỚI HẠN BỞI CÁC ĐƯỜNG CONG KHI BIẾT ĐỒ THỊ HÀM SỐ CỦA CÁC ĐƯỜNG CONG}
\Opensolutionfile{ans}[ans/ans-2-C4B3CD1_10-19]

\TN
%%%%-------------Câu 17
\begin{ex}%[2D4N3-1]
\immini{
Gọi $S$  là diện tích hình phẳng giới hạn bởi đồ thị hàm số  $y=f(x)$, trục hoành, đường thẳng $x=a$, $x=b$  (như hình vẽ bên). Hỏi cách tính $S$  nào dưới đây đúng?
}{
\begin{tikzpicture}[scale=.7,>=stealth, font=\footnotesize, line join=round, line cap=round]
	\def\a{-0.25} \def\b{2.5} \def\c{-6.75} \def\d{4.5} % Hệ số
	\def\xmin{-1} \def\xmax{7}
	\def\ymin{-2} \def\ymax{3} 
	\draw[->] (\xmin,0)--(\xmax,0) node [below]{$x$};
	\draw[->] (0,\ymin)--(0,\ymax) node [left]{$y$};
	\node at (0,0) [below left]{$O$};
	\draw[smooth,samples=300] plot[domain=1:6](\x,{\a*(\x)^3+\b*(\x)^2+\c*(\x)+\d});
	\draw[pattern=north east lines] plot[domain=1:6](\x,{\a*(\x)^3+\b*(\x)^2+\c*(\x)+\d});
	\node[below left] at (1,0) {$a$};
	\node[below right] at (3,0) {$c$};
	\node[below] at (6,0) {$b$};
	\node[] at (4,2.5) {$y=f(x)$};
\end{tikzpicture}
}
	\choice
{$S=\displaystyle\int\limits_a^b f(x) \mathrm{\,d}x$}
{$ S= \left|\displaystyle\int\limits_a^c f(x) \mathrm{\,d}x + \displaystyle\int\limits_c^b f(x) \mathrm{\,d}x \right|$}
{\True  $S=-\displaystyle\int\limits_a^c f(x) \mathrm{\,d}x + \displaystyle\int\limits_c^b f(x) \mathrm{\,d}x$}
{$S=\displaystyle\int\limits_a^c f(x) \mathrm{\,d}x + \displaystyle\int\limits_c^b f(x) \mathrm{\,d}x$}
	\loigiai{
Ta có $y=f(x)$ liên tục trên đoạn $\left[a; b\right]$.\\
Dựa vào đồ thị ta có $\left|f(x)\right|=\heva{& -f(x), & a\le x \le c\\& f(x), & c< x \le b.}$\\
Suy ra 
$S= \displaystyle\int\limits_a^b \left|f(x)\right| \mathrm{\,d}x = 
\displaystyle\int\limits_a^c \left|f(x)\right| \mathrm{\,d}x +\displaystyle\int\limits_c^b \left|f(x)\right| \mathrm{\,d}x = -\displaystyle\int\limits_a^c f(x) \mathrm{\,d}x +
 \displaystyle\int\limits_c^b f(x) \mathrm{\,d}x$.

}
\end{ex}	

%%%%%-------------Câu 18
\begin{ex}%[2D4N3-1]
	\immini{
Cho hàm số $y=f(x)$  liên tục trên đoạn  $\left[a; b\right]$. Gọi $D$  là diện tích hình phẳng giới hạn bởi đồ thị  $\left(C\right)\colon y=f(x)$, trục hoành, hai đường thẳng  $x=a$, $x=b$ (như hình vẽ). Giả sử  $S_D$ là diện tích hình phẳng  $D$. Chọn phương án đúng trong các phương án {\bf A}, {\bf B}, {\bf C}, {\bf D} cho dưới đây?
	}{
		\begin{tikzpicture}[yscale=.7,xscale=1,>=stealth, font=\footnotesize, line join=round, line cap=round]
			\def\a{1/3} \def\b{0} \def\c{0} \def\d{0} % Hệ số
			\def\xmin{-3} \def\xmax{3}
			\def\ymin{-3} \def\ymax{3} 
			\draw[->] (\xmin,0)--(\xmax,0) node [below]{$x$};
			\draw[->] (0,\ymin)--(0,\ymax) node [left]{$y$};
			\node at (0,0) [below left]{$O$};
			\draw[smooth,samples=300] plot[domain=-2.1:2.1](\x,{\a*(\x)^3+\b*(\x)^2+\c*(\x)+\d});
			\draw[pattern=north east lines] plot[domain=0:-2](\x,{\a*(\x)^3+\b*(\x)^2+\c*(\x)+\d})--(-2,0)--cycle
			plot[domain=0:2](\x,{\a*(\x)^3+\b*(\x)^2+\c*(\x)+\d})--(2,0)--cycle;
			\node[above] at (-2,0) {$a$};
			\node[below] at (2,0) {$b$};
		\end{tikzpicture}
	}
	\choice
	{$S_D=\displaystyle\int\limits_a^0 f(x) \mathrm{\,d}x +\displaystyle\int\limits_0^b f(x) \mathrm{\,d}x$}
	{\True  $S_D=-\displaystyle\int\limits_a^0 f(x) \mathrm{\,d}x + \displaystyle\int\limits_0^b f(x) \mathrm{\,d}x$}
	{$S_D=\displaystyle\int\limits_a^0 f(x) \mathrm{\,d}x -\displaystyle\int\limits_0^b f(x) \mathrm{\,d}x$}
	{$S_D=-\displaystyle\int\limits_a^0 f(x) \mathrm{\,d}x -\displaystyle\int\limits_0^b f(x) \mathrm{\,d}x$}
	\loigiai{
		Ta có $y=f(x)$ liên tục trên đoạn $\left[a; b\right]$.\\
		Dựa vào đồ thị ta có $\left|f(x)\right|=\heva{& -f(x), & a\le x \le 0\\& f(x), & 0< x \le b.}$\\
		Suy ra $S_D= \displaystyle\int\limits_a^b \left|f(x)\right| \mathrm{\,d}x = 
		\displaystyle\int\limits_a^0 \left|f(x)\right| \mathrm{\,d}x +\displaystyle\int\limits_0^b \left|f(x)\right| \mathrm{\,d}x = -\displaystyle\int\limits_a^0 f(x) \mathrm{\,d}x + \displaystyle\int\limits_0^b f(x) \mathrm{\,d}x$.
		}
\end{ex}	

%%%%%-------------Câu 19
\begin{ex}%[2D4N3-1]
	\immini{
Diện tích của hình phẳng được giới hạn bởi đồ thị hàm số $y=f(x)$, trục hoành và hai đường thẳng  $x=a$,  $x=b$  $(a<b)$ (phần tô đậm trong hình vẽ) tính theo công thức nào dưới đây?
	}{
		\begin{tikzpicture}[yscale=1,xscale=.8,>=stealth, font=\footnotesize, line join=round, line cap=round]
			\def\xmin{-3.5} \def\xmax{3}
			\def\ymin{-1.5} \def\ymax{2} 
			\draw[->] (\xmin,0)--(\xmax,0) node [below]{$x$};
			\draw[->] (0,\ymin)--(0,\ymax) node [left]{$y$};
			\node at (0,0) [below left]{$O$};
			\draw[smooth,samples=300] plot[domain=-3:2](\x,{((\x)+3)^.5-1});
			\draw[pattern=north west lines] plot[domain=-2:-3](\x,{((\x)+3)^.5-1})--(-3,0)--cycle
			plot[domain=-2:2](\x,{((\x)+3)^.5-1})--(2,0)--cycle;
			\node[above] at (-2,0) {$c$};
			\node[above] at (-3,0) {$a$};
			\node[below] at (2,0) {$b$};
			\node[left] at (0,1) {$(C)\colon y = f(x)$};
		\end{tikzpicture}
	}
	\choice
{$S=\displaystyle\int\limits_a^c f(x) \mathrm{\,d}x + 
	\displaystyle\int\limits_c^b f(x) \mathrm{\,d}x$}
{$S=\displaystyle\int\limits_a^b f(x) \mathrm{\,d}x$}
{\True  $S=-\displaystyle\int\limits_a^c f(x) \mathrm{\,d}x + \displaystyle\int\limits_c^b f(x) \mathrm{\,d}x$}
{$ S= \left|\displaystyle\int\limits_a^b f(x) \mathrm{\,d}x\right|$}

	\loigiai{
		Ta có $y=f(x)$ liên tục trên đoạn $\left[a; b\right]$.\\
		Dựa vào đồ thị ta có $\left|f(x)\right|=\heva{& -f(x), & a\le x \le c\\& f(x), & c< x \le b.}$\\
		Suy ra $S= \displaystyle\int\limits_a^b \left|f(x)\right| \mathrm{\,d}x = 
		\displaystyle\int\limits_a^c \left|f(x)\right| \mathrm{\,d}x +\displaystyle\int\limits_c^b \left|f(x)\right| \mathrm{\,d}x = -\displaystyle\int\limits_a^c f(x) \mathrm{\,d}x + \displaystyle\int\limits_c^b f(x) \mathrm{\,d}x$.
		
	}
\end{ex}	

%%%%%-------------Câu 20
\begin{ex}%[2D4H3-1]
Diện tích phần hình phẳng gạch chéo trong hình vẽ bên dưới được tính theo công thức nào dưới đây?
\begin{center}
	 	\begin{tikzpicture}[yscale=.8,xscale=.8,>=stealth, font=\footnotesize, line join=round, line cap=round]
	 	\def\xmin{-2} \def\xmax{3.5}
	 	\def\ymin{-2} \def\ymax{4} 
	 	\draw[->] (\xmin,0)--(\xmax,0) node [below]{$x$};
	 	\draw[->] (0,\ymin)--(0,\ymax) node [left]{$y$};
	 	\node [right] at (3,2){$y=x^2-2x-1$};
	 	\node [right] at (2.2,-2){$y=-x^2+3$};
	 	\clip (-2,-2) rectangle (3,3);
	 	\draw[smooth,samples=300] plot[domain=-1.4:3](\x,{(\x)^2-2*(\x)-1}) ;
	 	\draw[smooth,samples=300] plot[domain=-1.4:3](\x,{-(\x)^2+3});
	 	\draw[pattern=north west lines]plot[domain=-1:2](\x,{-(\x)^2+3})-- plot[domain=-1:2](\x,{(\x)^2-2*(\x)-1});
	 	\node at (0,0) [below right]{$O$};
	 	\draw[dashed] (-1,0) node [below]{$-1$}--(-1,2);
	 	\draw[dashed] (2,0) node [above]{$2$}--(2,-1);
	 	 \end{tikzpicture}
\end{center}

		\choice
	{$\displaystyle\int\limits_{-1}^{2} (-2x +2) \mathrm{\,d}x$}
	{$\displaystyle\int\limits_{-1}^{2} (2x -2) \mathrm{\,d}x$}
	{\True$\displaystyle\int\limits_{-1}^{2} (-2x^2 + 2x + 4) \mathrm{\,d}x$}
	{$\displaystyle\int\limits_{-1}^{2} (2x^2 -2x - 4) \mathrm{\,d}x$}

	\loigiai{
	Ta có 
	$S=\displaystyle\int\limits_{-1}^{2} \left|(-x^2 + 3) - (x^2 - 2x -1) \right| \mathrm{\,d}x = 
	\displaystyle\int\limits_{-1}^{2} \left|-2x^2 + 2x +4\right| \mathrm{\,d}x$.\\
	Vì $-2x^2 + 2x +4 > 0 , \forall x \in (-1; 2) $ nên ta có \\
	$S = \displaystyle\int\limits_{-1}^{2} \left|-2x^2 + 2x +4\right| \mathrm{\,d}x = \displaystyle\int\limits_{-1}^{2} (-2x^2 + 2x + 4) \mathrm{\,d}x$.
		
	}
\end{ex}	

%%%%%-------------Câu 21
\begin{ex}%[2D4N3-1]
Cho hàm số  $y=f(x)$ liên tục trên $\mathbb{R}$. Gọi $S$  là diện tích hình phẳng giới hạn bởi các đường $y=f(x)$, $y=0$, $x= -1$, $x = 5$ (như hình vẽ bên dưới).
\begin{center}
	 	\begin{tikzpicture}[scale=.7,>=stealth, font=\footnotesize, line join=round, line cap=round]
	 	\def\a{1/5} \def\b{-1} \def\c{-1/5} \def\d{1} % Hệ số
	 	\def\xmin{-2} \def\xmax{7}
	 	\def\ymin{-4} \def\ymax{2} 
	 	\draw[->] (\xmin,0)--(\xmax,0) node [below]{$x$};
	 	\draw[->] (0,\ymin)--(0,\ymax) node [left]{$y$};
	 	\node at (0,0) [below left]{$O$};
	 	\draw[smooth,samples=300] plot[domain=-1.7:5.4](\x,{\a*(\x)^3+\b*(\x)^2+\c*(\x)+\d}) node [left]{$y=f(x)$};
	 	\draw[pattern=north east lines] plot[domain=-1:5](\x,{\a*(\x)^3+\b*(\x)^2+\c*(\x)+\d});
	 	\node[above left] at (-1,0) {$-1$};
	 	\node[above right] at (1,0) {$1$};
	 	\node[below right] at (5,0) {$5$};
	 \end{tikzpicture}
\end{center}
Mệnh đề nào sau đây đúng?
	\choice
	{$S=-\displaystyle\int\limits_{-1}^{1} f(x) \mathrm{\,d}x - \displaystyle\int\limits_{1}^{5} f(x) \mathrm{\,d}x$}
	{$S=\displaystyle\int\limits_{-1}^{1} f(x) \mathrm{\,d}x + \displaystyle\int\limits_{1}^{5} f(x) \mathrm{\,d}x$}
	{\True  $S=\displaystyle\int\limits_{-1}^{1} f(x) \mathrm{\,d}x - \displaystyle\int\limits_{1}^{5} f(x) \mathrm{\,d}x$}
	{$S=-\displaystyle\int\limits_{-1}^{1} f(x) \mathrm{\,d}x + \displaystyle\int\limits_{1}^{5} f(x) \mathrm{\,d}x$}
	\loigiai{
		Ta có $y=f(x)$ liên tục trên đoạn $\left[-1; 5\right]$.\\
		Dựa vào đồ thị ta có $\left|f(x)\right|=\heva{& f(x), & -1\le x \le 1\\& -f(x), & 1< x \le 5.}$\\
		Suy ra $S= \displaystyle\int\limits_{1}^{5} \left|f(x)\right| \mathrm{\,d}x = 
		\displaystyle\int\limits_{-1}^{1} \left|f(x)\right| \mathrm{\,d}x +\displaystyle\int\limits_{1}^{5} \left|f(x)\right| \mathrm{\,d}x = \displaystyle\int\limits_{-1}^{1} f(x) \mathrm{\,d}x - \displaystyle\int\limits_{1}^{5} f(x) \mathrm{\,d}x$.
		
	}
\end{ex}	
%%%%%-------------Câu 22
\begin{ex}%[2D4N3-1]
	Cho hàm số  $y=f(x)$ liên tục trên $\mathbb{R}$. Gọi $S$  là diện tích hình phẳng giới hạn bởi các đường $y=f(x)$, $y=0$, $x= -1$, $x = 2$ (như hình vẽ bên dưới).
	\begin{center}
		\begin{tikzpicture}[scale=1,>=stealth, font=\footnotesize, line join=round, line cap=round]
			\def\a{1} \def\b{-2} \def\c{-1} \def\d{2} % Hệ số
			\def\xmin{-2} \def\xmax{3}
			\def\ymin{-2} \def\ymax{3} 
			\draw[->] (\xmin,0)--(\xmax,0) node [below]{$x$};
			\draw[->] (0,\ymin)--(0,\ymax) node [left]{$y$};
			\node at (0,0) [below left]{$O$};
			\draw[smooth,samples=300] plot[domain=-1.2:2.5](\x,{\a*(\x)^3+\b*(\x)^2+\c*(\x)+\d}) node [left]{$y=f(x)$};
			\draw[pattern=north east lines] plot[domain=-1:2](\x,{\a*(\x)^3+\b*(\x)^2+\c*(\x)+\d});
			\node[above left] at (-1,0) {$-1$};
			\node[above right] at (1,0) {$1$};
			\node[below right] at (2,0) {$2$};
			\end{tikzpicture}
	\end{center}
	Mệnh đề nào sau đây đúng?
	\choice
	{$S=\displaystyle\int\limits_{-1}^{1} f(x) \mathrm{\,d}x + \displaystyle\int\limits_{1}^{2} f(x) \mathrm{\,d}x$}
	{$S=-\displaystyle\int\limits_{-1}^{1} f(x) \mathrm{\,d}x - \displaystyle\int\limits_{1}^{2} f(x) \mathrm{\,d}x$}
	{$S=-\displaystyle\int\limits_{-1}^{1} f(x) \mathrm{\,d}x + \displaystyle\int\limits_{1}^{2} f(x) \mathrm{\,d}x$}
	{\True  $S=\displaystyle\int\limits_{-1}^{1} f(x) \mathrm{\,d}x - \displaystyle\int\limits_{1}^{2} f(x) \mathrm{\,d}x$}
	\loigiai{
		Ta có $y=f(x)$ liên tục trên đoạn $\left[-1; 2\right]$.\\
		Dựa vào đồ thị ta có $\left|f(x)\right|=\heva{& f(x), & -1\le x \le 1\\& -f(x), & 1< x \le 2.}$\\
		Suy ra $S= \displaystyle\int\limits_{1}^{2} \left|f(x)\right| \mathrm{\,d}x = 
		\displaystyle\int\limits_{-1}^{1} \left|f(x)\right| \mathrm{\,d}x +\displaystyle\int\limits_{1}^{2} \left|f(x)\right| \mathrm{\,d}x = \displaystyle\int\limits_{-1}^{1} f(x) \mathrm{\,d}x - \displaystyle\int\limits_{1}^{2} f(x) \mathrm{\,d}x$.
		
	}
\end{ex}	
%%%%%-------------Câu 23
\begin{ex}%[2D4N3-1]
	\immini{
Gọi $S$ là diện tích hình phẳng $(H)$ giới hạn bởi các đường $y=f(x)$, trục hoành và hai đường thẳng  $x=-1$, $x=2$. Đặt $a=\displaystyle\int\limits_{-1}^{0} f(x) \mathrm{\,d}x$, $b=\displaystyle\int\limits_{0}^{2} f(x) \mathrm{\,d}x$ (như hình vẽ bên). Mệnh đề nào sau đây đúng?  
\choice
	{\True $S=b-a$}
	{$S=b+a$}
	{$S=-b+a$}
	{$S=-b-a$}
	}{
\begin{tikzpicture}[yscale=.7,xscale=.8,>=stealth, font=\footnotesize, line join=round, line cap=round]
	\def\a{0.37} \def\b{0} \def\c{0.33} \def\d{0} % Hệ số
	\def\xmin{-2} \def\xmax{3}
	\def\ymin{-2.5} \def\ymax{4} 
	\draw[->] (\xmin,0)--(\xmax,0) node [below]{$x$};
	\draw[->] (0,\ymin)--(0,\ymax) node [left]{$y$};
	\node at (0,0) [below right]{$O$};
	\draw[smooth,samples=300] plot[domain=-1.5:2.1](\x,{\a*(\x)^3+\b*(\x)^2+\c*(\x)+\d});
	\draw[pattern=north east lines] plot[domain=0:-1](\x,{\a*(\x)^3+\b*(\x)^2+\c*(\x)+\d})--(-1,0)--cycle
	plot[domain=0:2](\x,{\a*(\x)^3+\b*(\x)^2+\c*(\x)+\d})--(2,0)--cycle;
	\node[above] at (-1,0) {$-1$};
	\node[below] at (2,0) {$2$};
\end{tikzpicture}
}
	\loigiai{
		Ta có $y=f(x)$ liên tục trên đoạn $\left[-1; 2\right]$.\\
		Dựa vào đồ thị ta có $\left|f(x)\right|=\heva{& -f(x), & -1\le x \le 0\\& f(x), & 0< x \le 2.}$\\
		Suy ra $S= \displaystyle\int\limits_{-1}^{2} \left|f(x)\right| \mathrm{\,d}x = 
		\displaystyle\int\limits_{-1}^{0} \left|f(x)\right| \mathrm{\,d}x +\displaystyle\int\limits_{0}^{2} \left|f(x)\right| \mathrm{\,d}x = -\displaystyle\int\limits_{-1}^{0} f(x) \mathrm{\,d}x + \displaystyle\int\limits_{0}^{2} f(x) \mathrm{\,d}x$.\\
		Hay $S=-a + b = b - a$.
		
	}
\end{ex}	
%%%%%-------------Câu 24
\begin{ex}%[2D4N3-1]
	\immini{
		Gọi $S$ là diện tích hình phẳng $(H)$ giới hạn bởi các đường $y=f(x)$, trục hoành và hai đường thẳng  $x=-3$, $x=2$. Đặt $a=\displaystyle\int\limits_{-3}^{1} f(x) \mathrm{\,d}x$, $b=\displaystyle\int\limits_{1}^{2} f(x) \mathrm{\,d}x$ (như hình vẽ bên). Mệnh đề nào sau đây đúng?  
	}{
		\begin{tikzpicture}[yscale=.7,xscale=.7,>=stealth, font=\footnotesize, line join=round, line cap=round]
			\def\a{-0.05} \def\b{-0.08} \def\c{1.07} \def\d{-.94} % Hệ số
			\def\xmin{-4} \def\xmax{3}
			\def\ymin{-4} \def\ymax{1} 
			\draw[->] (\xmin,0)--(\xmax,0) node [below]{$x$};
			\draw[->] (0,\ymin)--(0,\ymax) node [left]{$y$};
			\node at (0,0) [above left]{$O$};
			\draw[smooth,samples=300] plot[domain=-3:2](\x,{\a*(\x)^3+\b*(\x)^2+\c*(\x)+\d});
			\draw[pattern=north west lines] (-3,0)-- (-3,-3.5)-- plot[domain=-3:2](\x,{\a*(\x)^3+\b*(\x)^2+\c*(\x)+\d})-- (2,.5)--(2,0);
			\node[above] at (-3,0) {$-3$};
			\node[below] at (2,0) {$2$};
			\node[below] at (1,0) {$1$};
		\end{tikzpicture}
	}
	\choice
	{$S=a+b$}
	{$S=a-b$}
	{$S=-a-b$}
	{\True $S=b-a$}
	\loigiai{
		Ta có $y=f(x)$ liên tục trên đoạn $\left[-3; 2\right]$.\\
		Dựa vào đồ thị ta có $\left|f(x)\right|=\heva{& -f(x), & -3\le x \le 1\\& f(x), & 1< x \le 2.}$\\
		Suy ra $S= \displaystyle\int\limits_{-3}^{2} \left|f(x)\right| \mathrm{\,d}x = 
		\displaystyle\int\limits_{-3}^{1} \left|f(x)\right| \mathrm{\,d}x +\displaystyle\int\limits_{1}^{2} \left|f(x)\right| \mathrm{\,d}x = -\displaystyle\int\limits_{-3}^{1} f(x) \mathrm{\,d}x + \displaystyle\int\limits_{1}^{2} f(x) \mathrm{\,d}x$.\\
		Hay $S=-a + b = b - a$.
		
	}
\end{ex}	
%%%%%-------------Câu 25
\begin{ex}%[2D4V3-1]
	\immini{
Cho các số $p$, $q$  thỏa mãn các điều kiện $p>0$, $q>1$, $\dfrac{1}{p}+\dfrac{1}{q} = 1$ và các số dương $a, b$. Xét hàm số $y=x^{p-1}$ $(x>0)$ có đồ thị $(C)$. Gọi $S_1$  là diện tích hình phẳng giới hạn bởi  $(C)$, trục hoành, đường thẳng  $x=a$. Gọi $S_2$  là diện tích hình phẳng giới hạn bởi  $(C)$, trục tung, đường thẳng  $y=b$. Gọi $S$ là diện tích hình phẳng giới hạn bởi trục hoành, trục tung và hai đường thẳng  $x=a$,  $y=b$ (như hình vẽ bên). Khi so sánh  $S_1 + S_2$ và  $S$ ta nhận được bất đẳng thức nào trong các bất đẳng thức dưới đây?
	}{
		\begin{tikzpicture}[yscale=.7,xscale=.7,>=stealth, font=\footnotesize, line join=round, line cap=round]
			\def\a{1/8} \def\b{1} \def\c{0} \def\d{0} % Hệ số
			\def\xmin{-1} \def\xmax{4}
			\def\ymin{-1} \def\ymax{6} 
			\draw[->] (\xmin,0)--(\xmax,0) node [below]{$x$};
			\draw[->] (0,\ymin)--(0,\ymax) node [left]{$y$};
			\node at (0,0) [below left]{$O$};
			\draw[smooth,samples=300] plot[domain=0:2.2](\x,{\a*(\x)^3+\b*(\x)^2+\c*(\x)+\d}) node[above right]{$y=x^{p-1}$};
			\fill[pattern=north west lines] plot[domain=0:2](\x,{\a*(\x)^3+\b*(\x)^2+\c*(\x)+\d})--(2,5)--(2,0)--cycle;
			\fill[pattern=north east lines](0,4)-- plot[domain=0:1.802](\x,{\a*(\x)^3+\b*(\x)^2+\c*(\x)+\d})--cycle;
			\draw (-1,4)--(4,4) node[pos=.8,sloped,above]{$y=b$};
			\draw (2,-1)--(2,6)node[pos=.5,sloped,below]{$x=a$};
			\node[circle] at (.8,3){$S_2$};
			\node[circle] at (1.5,.5){$S_1$};
			\node[above left] at (0,4) {$b$};
			\node[below right] at (2,0) {$a$};
			\end{tikzpicture}
	}
	\choice
	{$\dfrac{a^p}{p}+\dfrac{b^q}{q}\le ab$}
	{$\dfrac{a^{p-1}}{p-1}+\dfrac{b^{q-1}}{q-1}\le ab$}
	{$\dfrac{a^{p+1}}{p+1}+\dfrac{b^{q+1}}{q+1}\le ab$}
	{\True $\dfrac{a^p}{p}+\dfrac{b^q}{q}\ge ab$}
	\loigiai{
	\begin{itemize}
		\item Diện tích hình phẳng giới hạn bởi trục hoành, trục tung và hai đường thẳng  $x=a$,  $y=b$ là $S = ab$.
		\item $S_1 = \displaystyle\int\limits_0^a x^{p-1} \mathrm{\,d}x=
		\left.\dfrac{x^p}{p}\right|_0^a = \dfrac{a^p}{p}$.
		\item Ta có $\dfrac{1}{p}+\dfrac{1}{q} = 1 \Leftrightarrow \dfrac{1}{q} = 1 - \dfrac{1}{p} = \dfrac{p - 1}{p} \Leftrightarrow q= \dfrac{p}{p-1}$. Tương tự $p=\dfrac{q}{q-1}$.\\
		Phương trình hoành độ giao điểm $ x^{p-1}=b\Leftrightarrow x= b^{\tfrac{1}{p-1}} \in (0;2)$. Suy ra\\
		$S_2 = \displaystyle\int\limits_0^{b^{\frac{1}{p-1}}} \left(b-x^{p-1}\right)\mathrm{\,d}x=
			\left.\left(bx -\dfrac{x^p}{p}\right)\right|_0^{b^{\frac{1}{p-1}}} $\\
			$= b\cdot b^{\frac{1}{p-1}}-
			\dfrac{\left( b^{\frac{1}{p-1}}\right)^p}{p}= b^{\frac{p}{p-1}}-
			\dfrac{b^{\frac{p}{p-1}}}{\dfrac{q}{q-1}} = b^q - \dfrac{ b^q(q-1)}{q} = \dfrac{b^q}{q}$.
		\item Dựa và hình vẽ đồ thị  ta có $S_1 + S_2 \ge S $.
		Vậy $\dfrac{a^p}{p}+\dfrac{b^q}{q}\ge ab $.
	\end{itemize}
	}
\end{ex}	
%%%%%-------------Câu 26
\begin{ex}%[2D4N3-1]
Diện tích phần hình phẳng được gạch sọc trong hình vẽ sau được tính theo công thức nào dưới đây?

	\begin{center}
		\begin{tikzpicture}[scale=1,>=stealth, font=\footnotesize, line join=round, line cap=round]
			\def\a{0} \def\b{1} \def\c{0} \def\d{-2} % Hệ số
			\def\xmin{-4} \def\xmax{4}
			\def\ymin{-3} \def\ymax{3} 
			\draw[->] (\xmin,0)--(\xmax,0) node [below]{$x$};
			\draw[->] (0,\ymin)--(0,\ymax) node [left]{$y$};
			\node at (0,0) [below left]{$O$};
			\draw[smooth,samples=300] plot[domain=-2:2](\x,{\a*(\x)^3+\b*(\x)^2+\c*(\x)+\d}) node [right]{$y=x^2 -2$};
			\draw[smooth,samples=300] plot[domain=0:3](\x,{-(\x)^.5})node [below]{$y=-\sqrt{|x|}$};
			\draw[smooth,samples=300] plot[domain=-3:0](\x,{-(-\x)^.5});
			\draw[pattern=north east lines] (-1,-1)--  plot[domain=-1:0](\x,{-(-\x)^.5})--(0,0)--plot[domain=0:-1](\x,{\a*(\x)^3+\b*(\x)^2+\c*(\x)+\d}) --cycle;
			\draw[pattern=north east lines] (0,0)--  plot[domain=0:1](\x,{-(\x)^.5})--(1,-1)--plot[domain=1:0](\x,{\a*(\x)^3+\b*(\x)^2+\c*(\x)+\d}) --cycle;
			\foreach \x in {-3,-2,-1,1,2,3} \draw[fill] (\x,0) circle (1pt) node [above] { $\x$};
			\foreach \y in {-2,1,2} \draw[fill] (0,\y) circle (1pt) node [ below left] { $\y$};
			\draw[dashed] (-1,0)--(-1,-1) (1,0)--(1,-1);
					\end{tikzpicture}
	\end{center}
	\choice
	{$\displaystyle\int\limits_{-1}^{1} \left( x^2 -2 + \sqrt{|x|}\right)\mathrm{\,d}x$}
	{$\displaystyle\int\limits_{-1}^{1} \left( x^2 -2 - \sqrt{|x|}\right)\mathrm{\,d}x$}
	{$\displaystyle\int\limits_{-1}^{1} \left( -x^2 + 2 + \sqrt{|x|}\right)\mathrm{\,d}x$}
	{\True $\displaystyle\int\limits_{-1}^{1} \left( -x^2 + 2 - \sqrt{|x|}\right)\mathrm{\,d}x$}
	\loigiai{
		Ta có $ -\sqrt{|x|}\ge x^2 -2$, $\forall x\in [-1; 1]$.\\
		Do đó $-\sqrt{|x|}- (x^2 -2) = -x^2  +2 -\sqrt{|x|} \ge 0, \forall x\in [-1; 1] $.\\
	Diện tích phần hình phẳng được gạch sọc trong hình vẽ là\\
	$\displaystyle\int\limits_{-1}^{1} \left|-\sqrt{|x|}- (x^2 -2)\right|\mathrm{\,d}x = \displaystyle\int\limits_{-1}^{1} \left( -x^2 + 2 - \sqrt{|x|}\right)\mathrm{\,d}x$
	
		
	}
\end{ex}

\Closesolutionfile{ans}
\indapan{6}{ans/ans-2-C4B3CD1_10-19}


\TNTF
\Opensolutionfile{ans}[ans/ans-2-C4B3CD1_10-19-DS]
\begin{ex}%[2D4H3-1]
Cho hàm số  $y=f(x)$ liên tục trên  $\mathbb{R}$. Gọi $S$  là diện tích hình phẳng giới hạn bởi các đường  $y=f(x)$, $y=0$, $x=-1$, $x=4$ (như hình vẽ). Các mệnh đề sau đây đúng hay sai?
		\begin{center}
		\begin{tikzpicture}[scale=1,>=stealth, font=\footnotesize, line join=round, line cap=round]
			\def\a{1/3} \def\b{-4/3} \def\c{-1/3} \def\d{4/3} % Hệ số
			\def\xmin{-2} \def\xmax{5}
			\def\ymin{-3} \def\ymax{2.5} 
			\draw[->] (\xmin,0)--(\xmax,0) node [below]{$x$};
			\draw[->] (0,\ymin)--(0,\ymax) node [left]{$y$};
			\node at (0,0) [below left]{$O$};
			\draw[smooth,samples=300] plot[domain=-1.5:4.2](\x,{\a*(\x)^3+\b*(\x)^2+\c*(\x)+\d}) node [right]{$y=f(x)$};
			\draw[pattern=north east lines] plot[domain=-1:4](\x,{\a*(\x)^3+\b*(\x)^2+\c*(\x)+\d});
		\node[above left] at (-1,0) {$-1$};
		\node[above right] at (1,0) {$1$};
		\node[below right] at (4,0) {$4$};
		\end{tikzpicture}
	\end{center}
	\choiceTF
	{\True $S= \displaystyle\int\limits_{-1}^{1} f(x) \mathrm{\,d}x - \displaystyle\int\limits_{1}^{4} f(x) \mathrm{\,d}x$}
	{\True $S= \displaystyle\int\limits_{-1}^{1} \left|f(x)\right| \mathrm{\,d}x +\displaystyle\int\limits_{1}^{4} \left|f(x)\right| \mathrm{\,d}x$}
	{$S= \left|\displaystyle\int\limits_{-1}^{4} f(x)\mathrm{\,d}x\right|$}
	{$S= \displaystyle\int\limits_{-1}^{1} f(x) \mathrm{\,d}x + \displaystyle\int\limits_{1}^{4} f(x) \mathrm{\,d}x$}
	\loigiai{
		Ta có $y=f(x)$ liên tục trên đoạn $\left[-1; 4\right]$.\\
		Dựa vào đồ thị ta có $\left|f(x)\right|=\heva{& f(x), & -1\le x \le 1\\& -f(x), & 1< x \le 4.}$\\
		Suy ra
		 $S= \displaystyle\int\limits_{-1}^{4} \left|f(x)\right| \mathrm{\,d}x = 
		\displaystyle\int\limits_{-1}^{1} \left|f(x)\right| \mathrm{\,d}x + \displaystyle\int\limits_{1}^{4} \left|f(x)\right| \mathrm{\,d}x = \displaystyle\int\limits_{-1}^{1} f(x) \mathrm{\,d}x - \displaystyle\int\limits_{1}^{4} f(x) \mathrm{\,d}x$.\\
		Do đó suy ra
		\begin{itemchoice}
			\itemch {\bf Đúng.}
			 Vì $S= \displaystyle\int\limits_{-1}^{1} f(x) \mathrm{\,d}x - \displaystyle\int\limits_{1}^{4} f(x) \mathrm{\,d}x$ đúng.
			\itemch {\bf Đúng.}
			Vì $S= \displaystyle\int\limits_{-1}^{1} \left|f(x)\right| \mathrm{\,d}x +\displaystyle\int\limits_{1}^{4} \left|f(x)\right| \mathrm{\,d}x$ đúng.
			\itemch {\bf Sai.} 
			Vì $\left|f(x)\right|=\heva{& f(x), & -1\le x \le 1\\& -f(x), & 1< x \le 4.}$ nên $\left|\displaystyle\int\limits_{-1}^{4} f(x)\mathrm{\,d}x\right| \ne \displaystyle\int\limits_{-1}^{4} \left|f(x)\right| \mathrm{\,d}x$.
			\itemch {\bf Sai.} 
			Vì $S= \displaystyle\int\limits_{-1}^{1} f(x) \mathrm{\,d}x -\displaystyle\int\limits_{1}^{4} f(x) \mathrm{\,d}x$ sai.
		\end{itemchoice}
	}
\end{ex}

\begin{ex}%[2D4H3-1]
Cho hình phẳng được gạch chéo trong hình bên dưới.
	\begin{center}
		\begin{tikzpicture}[yscale=.8,xscale=.8,>=stealth, font=\footnotesize, line join=round, line cap=round]
		\def\xmin{-2} \def\xmax{3.5}
		\def\ymin{-3} \def\ymax{3} 
		\draw[->] (\xmin,0)--(\xmax,0) node [below]{$x$};
		\draw[->] (0,\ymin)--(0,\ymax) node [left]{$y$};
		\node [right] at (3,1){$y=x^2-2x-2$};
		\node [right] at (2.2,-3){$y=-x^2+2$};
		\clip (-2,-3) rectangle (3,3);
		\draw[smooth,samples=300] plot[domain=-1.4:3](\x,{(\x)^2-2*(\x)-2}) ;
		\draw[smooth,samples=300] plot[domain=-1.4:3](\x,{-(\x)^2+2});
		\fill[pattern=north west lines]plot[domain=-1:2](\x,{-(\x)^2+2})-- plot[domain=-1:2](\x,{(\x)^2-2*(\x)-2});
		\node at (0,0) [below right]{$O$};
		\draw[dashed] (-1,0) node [below]{$-1$}--(-1,1);
		\draw[dashed] (2,0) node [above]{$2$}--(2,-2);
	\end{tikzpicture}
	\end{center}
	Các mệnh đề sau đây đúng hay sai?
	\choiceTF
	{\True Hình phẳng được gạch chéo trong hình trên được giới hạn các đồ thị $y=x^2-2x-2$, $y=-x^2+2$ và hai đường thẳng $x=-1$, $x=2$}
	{Diện tích hình phẳng gạch chéo trong hình vẽ là\\
		 $S= \displaystyle\int\limits_{-1}^{2} \left|x^2 -2x -2\right|\mathrm{\,d}x+\displaystyle\int\limits_{-1}^{2} \left|-x^2 + 2\right|\mathrm{\,d}x$}
	{\True Hình phẳng được gạch chéo trong hình trên được giới hạn các đồ thị $y=x^2-2x-2$ và  $y=-x^2+2$}
	{\True Diện tích hình phẳng gạch chéo trong hình vẽ là $S=9$}
	\loigiai{
			\begin{itemchoice}
			\itemch {\bf Đúng.}
			Hình phẳng được gạch chéo trong hình trên được giới hạn các đồ thị $y=x^2-2x-2$, $y=-x^2+2$ và hai đường thẳng $x=-1$, $x=2$.
			\itemch {\bf Sai.}\\
			Vì $\displaystyle\int\limits_{-1}^{2} \left|x^2 -2x -2\right|\mathrm{\,d}x+
			\displaystyle\int\limits_{-1}^{2} \left|-x^2 + 2\right|\mathrm{\,d}x\ge
			\displaystyle\int\limits_{-1}^{2} \left|(x^2 -2x -2)- (-x^2 + 2) \right|\mathrm{\,d}x=S$
			\itemch {\bf Đúng.} 
			Phương trình hoành độ giao điểm\\
			 $ x^2 -2x -2 = -x^2 +2 \Leftrightarrow
			2x^2 -2x - 4 = 0 \Leftrightarrow $ $x= -1$ hoặc $x=2$.\\
			Suy ra $S= \displaystyle\int\limits_{-1}^{2} \left|2x^2 -2x - 4\right|\mathrm{\,d}x$.
			\itemch {\bf Đúng}. 
			Vì $2x^2 -2x - 4<0, \forall x\in (-1;2)$.\\
			$S= \displaystyle\int\limits_{-1}^{2} \left|2x^2 -2x - 4\right|\mathrm{\,d}x=\displaystyle\int\limits_{-1}^{2} (-2x^2 + 2x + 4) \mathrm{\,d}x=\left.\left(\dfrac{2x^3}{3}+x^2+4x\right)\right|_{-1}^{2}= 9$.
		\end{itemchoice}
	}
\end{ex}


\begin{ex}%[2D4H3-1]
	Cho hình phẳng được gạch chéo trong hình bên dưới.
	\begin{center}
		\begin{tikzpicture}[yscale=.8,xscale=.8,>=stealth, font=\footnotesize, line join=round, line cap=round]
			\def\xmin{-3} \def\xmax{3}
			\def\ymin{-1} \def\ymax{5} 
			\node at (0,0) [below left]{$O$};
			\draw[->] (\xmin,0)--(\xmax,0) node [below]{$x$};
			\draw[->] (0,\ymin)--(0,\ymax) node [left]{$y$};
			\node [left] at (-2,4){$y=x^2$};
			\draw[smooth,samples=300] plot[domain=-2.2:2.2](\x,{(\x)^2}) ;
			\draw (1,-1)--(1,5) node[sloped,pos=.6,above]{$x=1$} 
			(2,-1)--(2,5) node[sloped,pos=.6,below]{$x=2$};
			\fill[pattern=north west lines](1,0)--(1,1)-- plot[domain=1:2](\x,{(\x)^2})--(2,4)--(2,0)--cycle;
			\foreach \x in {-1,-2,1,2} \draw[fill] (\x,0) circle (1pt) node [below left] { $\x$};
			\foreach \y in {1,2,3,4} \draw[fill] (0,\y) circle (1pt) node [left] { $\y$};
			\end{tikzpicture}
	\end{center}
	Các mệnh đề sau đây đúng hay sai?
	\choiceTF
{\True Hình phẳng được gạch chéo trong hình trên được giới hạn các đồ thị $y=x^2$, $y=0$ và hai đường thẳng $x=1$, $x=2$}
{\True  Diện tích hình phẳng gạch chéo trong hình vẽ là $S= \displaystyle\int\limits_{1}^{2} x^2 \mathrm{\,d}x$}
{Diện tích hình phẳng gạch chéo trong hình vẽ là $S=\dfrac{4}{3}$}
{Hình phẳng được gạch chéo trong hình trên được giới hạn đồ thị $y=x^2$ và hai đường thẳng $x=1$, $x=2$}
\loigiai{
	\begin{itemchoice}
		\itemch {\bf Đúng}.
		Hình phẳng được gạch chéo trong hình trên được giới hạn các đồ thị $y=x^2$, $y=0$ và hai đường thẳng $x=1$, $x=2$.
		\itemch {\bf Đúng}.
		Vì $S= \displaystyle\int\limits_{1}^{2} \left|x^2\right| \mathrm{\,d}x = \displaystyle\int\limits_{1}^{2} x^2 \mathrm{\,d}x$.
		\itemch {\bf Sai}. 
		Vì $S= \displaystyle\int\limits_{1}^{2} x^2 \mathrm{\,d}x = \left.\dfrac{x^3}{3}\right|_1^2 = \dfrac{8}{3}-\dfrac{1}{3}=\dfrac{7}{3}$.
		\itemch {\bf Sai}. 
		Vì hình phẳng được giới hạn đồ thị $y=x^2$ và hai đường thẳng $x=1$, $x=2$ không xác định được diện tích.
	\end{itemchoice}
	}
\end{ex}



\begin{ex}%[2D4H3-1]
	Cho hình phẳng được gạch chéo trong hình bên dưới.
	\begin{center}
		\begin{tikzpicture}[yscale=.8,xscale=.8,>=stealth, font=\footnotesize, line join=round, line cap=round]
			\def\xmin{-1} \def\xmax{6}
			\def\ymin{-1} \def\ymax{6.5} 
			\draw[->] (\xmin,0)--(\xmax,0) node [below]{$x$};
			\draw[->] (0,\ymin)--(0,\ymax) node [left]{$y$};
			\node at (0,0) [below right]{$O$};
			\draw[smooth,samples=300] plot[domain=-.2:5.2](\x,{-(\x)^2+5*(\x)}) node[left]{$y=5x-x^2$};
			\draw[smooth,samples=300] plot[domain=-1:5.2](\x,{(\x)})node[below right]{$y=x$} ;
			\fill[pattern=north west lines] plot[domain=0:4](\x,{-(\x)^2+5*(\x)});
			\foreach \x/\y in {4/4} \draw[fill] (\x,\y) circle (1pt) node [right] { $(\x,\y)$};
			\end{tikzpicture}
	\end{center}
	Các mệnh đề sau đây đúng hay sai?
	\choiceTF
	{\True Hình phẳng được gạch chéo trong hình trên được giới hạn các đồ thị $y=5x-x^2$, $y=x$ và các đường thẳng $x=0$, $x=4$}
	{Diện tích hình phẳng gạch chéo trong hình vẽ là $S= \displaystyle\int\limits_{0}^{4} \left(x^2 - 4x\right) \mathrm{\,d}x$}
	{\True Diện tích hình phẳng gạch chéo trong hình vẽ là $S= \displaystyle\int\limits_{0}^{4} \left|x^2 - 4x \right| \mathrm{\,d}x $}
	{Diện tích hình phẳng gạch chéo trong hình vẽ $S= \dfrac{56}{3}$}
	\loigiai{
		\begin{itemchoice}
			\itemch {\bf Đúng.}
			Hình phẳng được gạch chéo trong hình trên được giới hạn các đồ thị $y=5x-x^2$, $y=x$ và các đường thẳng $x=0$, $x=4$.
			\itemch {\bf Sai}
			Phương trình hoành độ giao điểm\\
			$ x  = 5x -x^2 \Leftrightarrow
			x^2 -4x = 0 \Leftrightarrow $ $x= 0$ hoặc $x=4$.\\
			Vì $x^2-4x<0, \forall x\in (0;4)$.
			Do đó $S= \displaystyle\int\limits_{0}^{4} \left|x^2 - 4x \right| \mathrm{\,d}x = \displaystyle\int\limits_{0}^{4} \left(- x^2 + 4x \right) \mathrm{\,d}x$.
			\itemch {\bf Đúng}. 
			Vì $S= \displaystyle\int\limits_{0}^{4} \left|x^2 - 4x \right| \mathrm{\,d}x$.
			\itemch {\bf Sai}. 
			Vì $S= \displaystyle\int\limits_{0}^{4} \left|x^2 - 4x \right| \mathrm{\,d}x= \displaystyle\int\limits_{0}^{4} \left(- x^2 + 4x \right) \mathrm{\,d}x= \left.\left(-\dfrac{x^3}{3}+ 2x^2\right)\right|_{0}^{4}=\dfrac{32}{3}$.
		\end{itemchoice}
	}
\end{ex}


\begin{ex}%[2D4H3-1]
	Cho hình phẳng được gạch chéo trong hình bên dưới.
	\begin{center}
		\begin{tikzpicture}[yscale=1,xscale=1,>=stealth, font=\footnotesize, line join=round, line cap=round]
			\def\xmin{-1} \def\xmax{4}
			\def\ymin{-1} \def\ymax{3.5} 
			\draw[->] (\xmin,0)--(\xmax,0) node [below]{$x$};
			\draw[->] (0,\ymin)--(0,\ymax) node [left]{$y$};
			\node at (0,0) [below left]{$O$};
			\draw[smooth,samples=300] plot[domain=0.5:3](\x,{1+1/(\x)});
			\draw[pattern=north east lines](1,0)--(1,2)-- plot[domain=1:2](\x,{1+1/(\x)})--(2,1.5)--(2,0)--cycle;
			\foreach \x in {1,2} \draw[fill] (\x,0) circle (1pt) node [below] { $\x$};
			\foreach \y in {1,2} \draw[fill] (0,\y) circle (1pt) node [left] { $\y$};
			\node[above right] at (1,2) {$y=1+\dfrac{1}{x}$};
		\end{tikzpicture}
	\end{center}
	Các mệnh đề sau đây đúng hay sai?
	\choiceTF
	{Hình phẳng được gạch chéo trong hình trên được giới hạn đồ thị $y=1 + \dfrac{1}{x}$ và các đường thẳng $x=1$, $x=2$}
	{\True  Diện tích hình phẳng gạch chéo trong hình vẽ là $S= \displaystyle\int\limits_{1}^{2} \left(1 + \dfrac{1}{x}\right) \mathrm{\,d}x$}
	{Diện tích hình phẳng gạch chéo trong hình vẽ là $S=2$}
	{\True Diện tích hình phẳng gạch chéo trong hình vẽ là $S= 1+ \displaystyle\int\limits_{1}^{2} \dfrac{1}{x} \mathrm{\,d}x$}
	\loigiai{
			\begin{itemchoice}
			\itemch {\bf Sai.}
			Hình phẳng  giới hạn đồ thị
			$y=1 + \dfrac{1}{x}$ và các đường thẳng $x=1$, $x=2$ không xác định được diện tích.
			\itemch {\bf Đúng.}
			Vì $1+\dfrac{1}{x}>0, \forall x\in (1;2)$ nên $\left|1+\dfrac{1}{x}\right| = 1+\dfrac{1}{x}, \forall x\in (1;2)$.\\
			Do đó $S= \displaystyle\int\limits_{1}^{2} \left|1+\dfrac{1}{x}\right| \mathrm{\,d}x= \displaystyle\int\limits_{1}^{2} \left(1 + \dfrac{1}{x}\right) \mathrm{\,d}x$.
			\itemch {\bf Sai.}
			Vì $S= \displaystyle\int\limits_{1}^{2} \left(1 + \dfrac{1}{x}\right) \mathrm{\,d}x =\left(x+ \ln |x|\right)\Big|_{1}^{2} = 1+ \ln 2$.
			\itemch {\bf Đúng.} 
			Vì $ S=  \displaystyle\int\limits_{1}^{2} \left(1 + \dfrac{1}{x}\right) \mathrm{\,d}x = 1+ \displaystyle\int\limits_{1}^{2} \dfrac{1}{x} \mathrm{\,d}x = 1+ \ln 2$.
		\end{itemchoice}
	}
\end{ex}

