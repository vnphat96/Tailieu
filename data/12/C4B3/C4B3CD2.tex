%\setcounter{chude}{1}
\begin{dang}{	THỂ TÍCH KHỐI TRÒN XOAY}
\end{dang}
% \begin{tomtat}
% 	\subsection{Thế tích của vật thế}
% 	\begin{center}
% 	\begin{tikzpicture}[>=stealth, scale=0.8]
% 		\draw plot[smooth,tension=.65] coordinates{(1,2) (2.5,2.3) (3.5,2.2)};
% 		\draw[dashed] plot[smooth,tension=.65] coordinates{(3.5,2.2) (4,2)};
% 		\draw plot[smooth,tension=.65] coordinates{(4,2) (5,2.2) (5.5,2.1)};
% 		\draw[dashed] plot[smooth,tension=.65] coordinates{(5.5,2.1) (6,2)};
% 		\draw plot[smooth,tension=.65] coordinates{(1,1) (2.3,0.5) (3.5,0.8)};
% 		\draw[dashed] plot[smooth,tension=.65] coordinates{(3.5,0.8) (4,1)};
% 		\draw plot[smooth,tension=.65] coordinates{(4,1) (5,0.7) (5.5,0.8)};
% 		\draw[dashed] plot[smooth,tension=.65] coordinates{(5.5,0.8) (6,1)};
% 		\draw[dashed] (1,1) arc (-90:90:.2 and 0.5);
% 		\draw (1,2) arc (90:270:.2 and 0.5);
% 		\draw[dashed] (4,1) arc (-90:90:.2 and 0.5);
% 		\draw (4,2) arc (90:270:.2 and 0.5);
% 		\draw (6,1) arc (-90:270:.2 and 0.5);
% 		\fill[pattern=north east lines] (4,1) arc (-90:90:.2 and 0.5)--(4,2) arc (90:270:.2 and 0.5)--cycle;
% 		\draw (-.5,0)--(0.5,0) (1,0)--(3.5,0) (4,0)--(5.5,0);
% 		\draw[dashed] (0.5,0)--(1,0) (3.5,0)--(4,0) (5.5,0)--(6,0);
% 		\draw[->] (6,0)--(7,0)node[below]{$x$};
% 		\draw (0.5,-1)--(0.5,3)--(1.5,3.5)--(1.5,2.2) (1.5,.8)--(1.5,-0.5)--(0.5,-1);
% 		\draw[dashed](1.5,2.2)--(1.5,.8);
% 		\draw[dashed] (1,1)--(1,0)node[below]{$a$};
% 		\coordinate (A) at (0.5,3);
% 		\coordinate (B) at (1.5,3.5);
% 		\coordinate (C) at (1.5,2.2);
% 		%\tkzMarkAngle[size=.6](A,B,C);
% 		\draw pic[draw=black, angle eccentricity=1.6, angle radius=0.5cm]{angle=A--B--C};
% 		\draw (1.3,3.2) node {\footnotesize $P$};
% 		\draw (3.5,-1)--(3.5,3)--(4.5,3.5)--(4.5,2) (4.5,1)--(4.5,-0.5)--(3.5,-1);
% 		\draw[dashed](4.5,2)--(4.5,1);
% 		\draw[dashed] (4,1)--(4,0)node[below]{$x$};
% 		\coordinate (D) at (3.5,3);
% 		\coordinate (E) at (4.5,3.5);
% 		\coordinate (F) at (4.5,2);
% 	%	\tkzMarkAngle[size=.6](D,E,F);
% 		\draw pic[draw=black, angle eccentricity=1.6, angle radius=0.5cm]{angle=D--E--F};
% 		\draw (4.3,3.2) node {\footnotesize $R$};
% 		\draw (5.5,-1)--(5.5,3)--(6.5,3.5)--(6.5,-0.5)--(5.5,-1);
% 		\draw[dashed] (6,1)--(6,0)node[below]{$b$};
% 		\coordinate (G) at (5.5,3);
% 		\coordinate (H) at (6.5,3.5);
% 		\coordinate (K) at (6.5,-0.5);
% 	%	\tkzMarkAngle[size=.6](G,H,K);
% 		\draw pic[draw=black, angle eccentricity=1.6, angle radius=0.5cm]{angle=G--H--K};	\draw (6.3,3.2) node {\footnotesize $Q$};
% 		\draw (0,.3) node {$O$};
% 		\fill (0,0) circle(1pt);
% 		\draw[->] (4,1.5)--(4.7,1.7) node[right] {\scriptsize $S(x)$};
% 	\end{tikzpicture}
% 	\end{center}
% 	Trong không gian, cho một vật thể nằm trong khoảng không gian giữa hai mặt phẳng $(P)$ và $(Q)$ cùng vuông góc với trục $O x$ tại các điểm $a$ và $b$. Mặt phẳng vuông góc với trục $O x$ tại điểm $x(a \leq x \leq b)$ cắt vật thể theo mặt cắt có diện tích $S(x)$. Khi đó, nếu $S(x)$ là hàm số liên tục trên $\left[a ; b\right]$ thì thể tích của vật thể được tính bởi công thức
% 	$$
% 	V=\displaystyle\int\limits_a^b S(x) \mathrm{\,d} x.
% 	$$
% \subsection{Thế tích khối tròn xoay}
% 	\begin{center}
% 		\begin{tikzpicture}[line join=round, line cap=round,>=stealth,thick,scale=.4]
% 		\tikzset{label style/.style={font=\normalsize}}
% 		%%Nhập giới hạn đồ thị và hàm số cần vẽ
% 		\def \xmin{-.5}
% 		\def \xmax{11}
% 		\def \ymin{-4}
% 		\def \ymax{4.5}
% 		%\draw[xstep=1 cm, ystep=1 cm,gray,thin] (\xmin,\ymin) grid (\xmax,\ymax);
% 		\def \hamso{0-0.01864083398323228*((\x)-1.0)^(3.0)+0.35126127715584465*((\x)-1.0)^(2.0)-1.5117402856674746*((\x)-1.0)+3.106314636847822}
% 		%%Tự động
% 		\draw[->] (\xmin,0)--(11,0) node[below left] {$x$};
% 		\draw[->] (0,\ymin)--(0,\ymax) node[below left] {$y$};
% 		\draw (0,0) node [below left] {$O$};
% 		\draw[dashed] (2.04,-1.89) arc(-90:90:.5 cm and 1.89 cm);
% 		\draw (2.04,1.89) arc(90:270:.5 cm and 1.89 cm);
		
% 		%\draw[dashed] (5.59,-1.77) arc(-90:90:.5 cm and 1.77 cm);
% 		%	\draw (5.59,1.77) arc(90:270:.5 cm and 1.77 cm);
		
% 		\draw (9,0) ellipse (1 cm and 3.96 cm);
		
% 		\draw[fill=black] (2.04,0) node [below] {$a$} circle (1.2pt);
% 		\draw[fill=black] (9,0) node [below] {$b$} circle (1.2pt);
% 		\draw[fill=black] (5.5,1.8) node [above,rotate=30] {\scriptsize $y=f(x)$};
% 		%%Tự động
% 		\begin{scope}
% 			\clip (\xmin+0.01,\ymin+0.01) rectangle (\xmax-0.01,\ymax-0.01);
% 			\draw[samples=350,domain=1:9.,smooth,variable=\x] plot (\x,{\hamso});
% 			\draw[samples=350,domain=1:9,smooth,variable=\x] plot (\x,{-0+0.01864083398323228*((\x)-1.0)^(3.0)-0.35126127715584465*((\x)-1.0)^(2.0)+1.5117402856674746*((\x)-1.0)-3.106314636847822});		
% 			\draw[pattern=north west lines,opacity=0.5] (2.04,0)--(2.04,1.89)plot[domain=2.04:9] (\x,{0-0.01864083398323228*((\x)-1.0)^(3.0)+0.35126127715584465*((\x)-1.0)^(2.0)-1.5117402856674746*((\x)-1.0)+3.106314636847822})--(9.0,0)--(2.04,0);
% 		\end{scope}
% 	\end{tikzpicture}
% 	\end{center}
% 	Cho hàm số $y=f(x)$ liên tục, không âm trên $\left[a ; b\right]$. Hình phẳng $(H)$ giới hạn bởi đồ thị hàm số $y=f(x)$, trục hoành $O x$ và hai đường thẳng $x=a$ và $x=b$ quay quanh trục $O x$ tạo thành một khối tròn xoay có thể tích bằng
% 	$$
% 	V=\pi \displaystyle\int\limits_a^b\left[f(x)\right]^2 \mathrm{\,d} x
% 	$$
% \end{tomtat}
%\TN
\Opensolutionfile{ans}[ans/ans-2-C4B3CD2-lc]
\begin{ex}%[2D4N3-3]
Viết công thức tính thể tích $V$ của khối tròn xoay được tạo ra khi quay hình thang cong, giới hạn bới đồ thị hàm số $y=f(x)$, trục $O x$ và hai đường thẳng $x=a$, $x=b$, $(a<b)$ xung quanh trục $O x$.
\choice
{$V=\displaystyle\int\limits_a^b \left|f(x)\right| \mathrm{\,d} x$}
{\True $V=\pi \displaystyle\int\limits_a^b f^2(x) \mathrm{\,d} x $}
{$V=\displaystyle\int\limits_a^b f^2(x) \mathrm{\,d} x$}
{$V=\pi \displaystyle\int\limits_a^b f(x) \mathrm{\,d} x$}
\loigiai{
Theo lí thuyết.
}
\end{ex}
\begin{ex}%[2D4N3-4]
	Cắt một vật thể bởi hai mặt phẳng vuông góc với trục $O x$ tại $x=1$ và $x=2$. Một mặt phẳng tùy ý vuông góc với trục $O x$ tại điểm có hoành độ $x$, $(1 \leq x \leq 2)$ cắt vật thể đó có diện tích $S(x)=2024 x$. Tính thể tích của phần vật thể giới hạn bởi hai mặt phẳng trên.
	\choice
	{\True $V=3036$}
	{$V=3036 \pi$}
	{$V=1518$}
	{$V=1518 \pi$}
\loigiai{Thể tích vật thể là $ V=\displaystyle\int\limits_1^2 2024x \mathrm{\,d}x=3036 $.}
\end{ex}

\begin{ex}%[2D4H3-4]
Cắt một vật thể bởi hai mặt phẳng vuông góc với trục $O x$ tại $x=1$ và $x=3$. Một mặt phẳng tùy ý vuông góc với trục $O x$ tại điểm có hoành độ $x$, $(1 \leq x \leq 3)$ cắt vật thể đó theo thiết diện là một hình chữ nhật có độ dài hai cạnh là $3 x$ và $3 x^2-2$. Tính thể tích của phần vật thể giới hạn bởi hai mặt phẳng trên.
\choice
{\True $V=156$}
{$V=156 \pi$}
{$ V=312 $}
{$V=312 \pi$}
\loigiai{Diện tích thiết diện là $S(x)=3 x \cdot \left(3 x^2-2\right)=9 x^3-6 x$.\\
Thể tích vật thể là $V=\displaystyle\int\limits_1^3\left(9 x^3-6 x\right) \mathrm{\,d} x=156$.
}
\end{ex}

\begin{ex}%[2D4N3-3]
Gọi $D$ là hình phẳng giới hạn bởi các đường $y=\mathrm{e}^{3 x}$, $y=0$, $x=0$ và $x=1$. Thể tích của khối tròn xoay tạo thành khi quay $D$ quanh trục $O x$ bằng
\choice
{$\pi \displaystyle\int\limits_0^1 \mathrm{e}^{3 x} \mathrm{\,d} x$}
{$\displaystyle\int\limits_0^1 \mathrm{e}^{6 x} \mathrm{\,d} x$}
{\True $\pi \displaystyle\int\limits_0^1 \mathrm{e}^{6 x} \mathrm{\,d} x$}
{$\displaystyle\int\limits_0^1 \mathrm{e}^{3 x} \mathrm{\,d} x$}
\loigiai{
Thể tích của khối tròn xoay tạo thành khi quay $D$ quanh trục $O x$ là\\
$$\pi \displaystyle\int\limits_0^1\left(\mathrm{e}^{3 x}\right)^2 \mathrm{\,d} x=\pi \displaystyle\int\limits_0^1 \mathrm{e}^{6 x} \mathrm{\,d} x.$$}
\end{ex}

\begin{ex}%[2D4N3-3]
Gọi $D$ là hình phẳng giới hạn bởi các đường $y=\mathrm{e}^{4 x}$, $y=0$, $x=0$ và $x=1$. Thể tích của khối tròn xoay tạo thành khi quay $D$ quanh trục $O x$ bằng
\choice
{$\displaystyle\int\limits_0^1 \mathrm{e}^{4 x} \mathrm{\,d} x$}
{\True $\pi \displaystyle\int\limits_0^1 \mathrm{e}^{8 x} \mathrm{\,d} x$}
{$\pi \displaystyle\int\limits_0^1 \mathrm{e}^{4 x} \mathrm{\,d} x$}
{$\displaystyle\int\limits_0^1 \mathrm{e}^{8 x} \mathrm{\,d} x$}
\loigiai{
Thể tích của khối tròn xoay tạo thành khi quay $D$ quanh trục $O x$ là 
$$V=\pi \displaystyle\int\limits_0^1\left(\mathrm{e}^{4 x}\right)^2 \mathrm{\,d} x=\pi \displaystyle\int\limits_0^1 \mathrm{e}^{8 x} \mathrm{\,d} x.$$}
\end{ex}

\begin{ex}%[2D4N3-3]
Cho hình phẳng $(H)$ giới hạn bởi các đường $y=x^2+3$, $y=0$, $x=0$, $x=2$. Gọi $V$ là thể tích của khối tròn xoay được tạo thành khi quay $(H)$ xung quanh trục $O x$. Mệnh đề nào dưới đây đúng?
\choice
{$V=\displaystyle\int\limits_0^2\left(x^2+3\right) \mathrm{\,d}x$}
{$V=\pi \displaystyle\int\limits_0^2\left(x^2+3\right) \mathrm{\,d}x$}
{$V=\displaystyle\int\limits_0^2\left(x^2+3\right)^2 \mathrm{\,d}x$}
{\True $V=\pi \displaystyle\int\limits_0^2\left(x^2+3\right)^2 \mathrm{\,d}x$}
\loigiai{Thể tích của khối tròn xoay được tạo thành khi quay $(H)$ xung quanh trục $O x$ là\\
$V=\pi \displaystyle\int\limits_0^2\left(x^2+3\right)^2 \mathrm{\,d} x$.
}
\end{ex}

\begin{ex}%[2D4H3-3]
Cho hình phẳng $D$ giới hạn bởi đường cong $y=\mathrm{e}^x$, trục hoành và các đường thẳng $x=0$, $x=1$. Khối tròn xoay tạo thành khi quay $D$ quanh trục hoành có thể tích $V$ bằng bao nhiêu?
\choice
{$V=\dfrac{\pi\left(\mathrm{e}^2+1\right)}{2}$}
{$V=\dfrac{\mathrm{e}^2-1}{2}$}
{$V=\dfrac{\pi \mathrm{e}^2}{3}$}
{\True $V=\dfrac{\pi\left(\mathrm{e}^2-1\right)}{2}$}
\loigiai{
$V=\pi \displaystyle\int\limits_0^1 \mathrm{e}^{2 x} \mathrm{\,d} x=\pi \dfrac{\mathrm{e}^{2x}}{2} \Bigg|_0 ^1=\dfrac{\pi\left(e^2-1\right)}{2}$.}
\end{ex}

\begin{ex}%[2D4H3-3]
Cho hình phẳng $D$ giới hạn bởi đường cong $y=\sqrt{x^2+1}$, trục hoành và các đường thẳng $x=0$, $x=1$. Khối tròn xoay tạo thành khi quay $D$ quanh trục hoành có thể tích $V$ bằng bao nhiêu?
\choice
{$ V=2 $}
{\True $V=\dfrac{4 \pi}{3} $}
{$V=2 \pi$}
{$V=\dfrac{4}{3}$}
\loigiai{Thể tích khối tròn xoay được tính theo công thức
	$$
	V=\pi \displaystyle\int\limits_0^1\left(\sqrt{x^2+1}\right)^2 \mathrm{\,d} x=\pi \displaystyle\int\limits_0^1\left(x^2+1\right) \mathrm{\,d} x=\pi\left(\dfrac{x^3}{3}+x\right)\Bigg|_0 ^1=\dfrac{4 \pi}{3} .
	$$}
\end{ex}

\begin{ex}%[2D4H3-3]
Cho hình phẳng $D$ giới hạn bởi đường cong $y=\sqrt{2+\cos x}$, trục hoành và các đường thẳng $x=0, x=\dfrac{\pi}{2}$. Khối tròn xoay tạo thành khi $D$ quay quanh trục hoành có thể tích $V$ bằng bao nhiêu?
\choice
{\True $V=(\pi+1) \pi$}
{$V=\pi-1$}
{$V=\pi+1$}
{$V=(\pi-1) \pi$}
\loigiai{Ta có 
	$$
	V=\pi \displaystyle\int\limits_0^{\tfrac{\pi}{2}}(\sqrt{2+\cos x})^2 \mathrm{\,d} x=\pi(2 x+\sin x)\Bigg|_0 ^{\tfrac{\pi}{2}}=\pi(\pi+1).
	$$}
\end{ex}

\begin{ex}%[2D4H3-3]
Cho hình phẳng $D$ giới hạn bởi đường cong $y=\sqrt{2+\sin x}$, trục hoành và các đường thẳng $x=0$, $ x=\pi$. Khối tròn xoay tạo thành khi quay $D$ quay quanh trục hoành có thể tích $V$ bằng bao nhiêu?
\choice
{\True $V=2 \pi(\pi+1)$}
{$V=2 \pi$}
{$V=2(\pi+1)$}
{$V=2 \pi^2$}
\loigiai{Ta có $V=\pi \displaystyle\int\limits_0^\pi\left(\sqrt{2+\sin x}\right)^2 \mathrm{\,d} x=\pi \displaystyle\int\limits_0^\pi\left(2+\sin x\right) \mathrm{\,d} x=\pi(2 x-\cos x)\Bigg|_0 ^\pi=2 \pi\left(\pi+1\right)$.}
\end{ex}

\begin{ex}%[2D4H3-3]
Tìm công thức tính thể tích của khối tròn xoay khi cho hình phẳng giới hạn bởi parabol $(P)\colon y=x^2$, đường thẳng $d\colon y=2 x$ và đường thẳng $x=0$, $x=2$ quay xung quanh trục $O x$.
\choice
{$\pi \displaystyle\int\limits_0^2\left(x^2-2 x\right)^2 \mathrm{\,d} x$}
{\True $\pi \displaystyle\int\limits_0^2 4 x^2 \mathrm{\,d} x-\pi \int_0^2 x^4 \mathrm{\,d} x$}
{ $\pi \displaystyle\int\limits_0^2 4 x^2 \mathrm{\,d} x+\pi \int_0^2 x^4 \mathrm{\,d} x$}
{$\pi \displaystyle\int\limits_0^2\left(2 x-x^2\right) \mathrm{\,d} x$}
\loigiai{Với mọi $ x \in \left[0;2\right] $ ta có $ 2x\ge 0 $, $ x^2\ge 0 $ và $ 2x\ge x^2 $ nên $V=\pi \displaystyle\int\limits_0^2 4 x^2 \mathrm{\,d} x-\pi \displaystyle\int\limits_0^2 x^4 \mathrm{\,d} x$.
}
\end{ex}

\begin{ex}%[2D4N3-3]
 Cho hình phẳng $(H)$ giới hạn bởi các đường $y=x^2+3$, $y=0$, $x=0$, $x=2$. Gọi $V$ là thể tích khối tròn xoay được tạo hành khi quay $ (H) $ xung quanh trục $ Ox $. Mệnh đề nào sau đây đúng?
 \choice
 {\True $ V=\pi \displaystyle\int\limits_0^2 \left(x^2+3\right)^2 \mathrm{\,d}x $}
 {$ V=\displaystyle\int\limits_0^2 \left(x^2+3\right)\mathrm{\,d}x $}
 {$ V=\displaystyle\int\limits_0^2 \left(x^2+3\right)^2 \mathrm{\,d}x $}
 {$ V=\pi \displaystyle\int\limits_0^2 \left(x^2+3\right) \mathrm{\,d}x $}
 \loigiai{Thể tích của vật tròn xoay là $ V=\pi \displaystyle\int\limits_0^2 \left(x^2+3\right)^2 \mathrm{\,d}x $.}
\end{ex}
	%Câu 13
\begin{ex}%[2D4N3-3]
	Gọi $V$ là thể tích của khối tròn xoay thu được khi quay hình thang cong, giới hạn bởi đồ thị hàm số $y=\sin x$, trục $Ox$, trục $Oy$ và đường thẳng $x=\dfrac{\pi}{2}$, xung quanh trục $Ox$. Mệnh đề nào dưới đây đúng?
	\choice
	{$V=\displaystyle\int\limits_0^{\tfrac{\pi}{2}}{\sin^2x\mathrm{\,d}x}$}
	{$V=\displaystyle\int\limits_0^{\tfrac{\pi}{2}}{\sin x\mathrm{\,d}x}$}
	{\True $V=\pi\displaystyle\int\limits_0^{\tfrac{\pi}{2}}{\sin^2x\mathrm{\,d}x}$}
	{$V=\pi\displaystyle\int\limits_0^{\tfrac{\pi}{2}}{\sin x\mathrm{\,d}x}$}
	\loigiai{
		Công thức tính $V=\pi\displaystyle\int\limits_a^b{f^2(x)\mathrm{\,d}x}$.
	}
\end{ex}

%Câu 14
\begin{ex}%[2D4H3-3]
	Thể tích khối tròn xoay được sinh ra khi quay hình phẳng giới hạn bởi đồ thị của hàm số $y=x^2-2x$, trục hoành, đường thẳng $x=0$ và $x=1$ quanh trục hoành bằng
	\choice
	{$\dfrac{16\pi}{15}$}
	{$\dfrac{2\pi}{3}$}
	{$\dfrac{4\pi}{3}$}
	{\True $\dfrac{8\pi}{15}$}
	\loigiai{
		Ta có
		\allowdisplaybreaks
		\begin{eqnarray*}
			V&=&\pi\displaystyle\int\limits_0^1\left(x^2-2x\right)^2\mathrm{\,d}x\\
			&=&\pi\displaystyle\int\limits_0^1\left(x^4-4x^3+4x^2\right)\mathrm{\,d}x\\
			&=&\pi \cdot \left(\dfrac{x^5}{5}-x^4+\dfrac{4x^3}{3}\right)\Bigg|_0^1\\
			&=&\pi \cdot \left(\dfrac{1}{5}-1+\dfrac{4}{3}\right)=\dfrac{8\pi}{15}.
		\end{eqnarray*}
	}
\end{ex}

%Câu 15
\begin{ex}%[2D4N3-3]
	Cho miền phẳng $(D)$ giới hạn bởi $y=\sqrt x$, hai đường thẳng $x=1$, $x=2$ và trục hoành. Tính thể tích khối tròn xoay tạo thành khi quay $(D)$ quanh trục hoành.
	\choice
	{$3\pi $}
	{\True $\dfrac{3\pi}{2}$}
	{$\dfrac{2\pi}{3}$}
	{$\dfrac{3}{2}$}
	\loigiai{
		$V=\pi\displaystyle\int\limits_1^2x\mathrm{\,d}x=\dfrac{\pi x^2}{2}\Bigg|_1^2=\dfrac{3\pi}{2}$.
	}
\end{ex}

%Câu 16
\begin{ex}%[2D4N3-3]
	Cho hình phẳng $(H)$ giới hạn bởi các đường $y=2x-x^2$, $y=0$. Quay $(H)$ quanh trục hoành tạo thành khối tròn xoay có thể tích là
	\choice
	{$\displaystyle\int\limits_0^2\left(2x-x^2\right)\mathrm{\,d}x$}
	{\True $\pi\displaystyle\int\limits_0^2\left(2x-x^2\right)^2\mathrm{\,d}x$}
	{$\displaystyle\int\limits_0^2\left(2x-x^2\right)^2\mathrm{\,d}x$}
	{$\pi\displaystyle\int\limits_0^2\left(2x-x^2\right)\mathrm{\,d}x$}
	\loigiai{
		Theo công thức ta chọn $V=\pi\displaystyle\int\limits_0^2\left(2x-x^2\right)^2\mathrm{\,d}x$.
	}
\end{ex}

%Câu 17
\begin{ex}%[2D4H3-3]
	Cho hình phẳng giới hạn bởi các đường $y=\sqrt{x}-2$, $y=0$ và $x=4$, $x=9$ quay xung quanh trục $Ox$. Tính thể tích khối tròn xoay tạo thành.
	\choice
	{$V=\dfrac{7}{6}$}
	{$V=\dfrac{5\pi}{6}$}
	{$V=\dfrac{7\pi}{11}$}
	{\True $V=\dfrac{11\pi}{6}$}
	\loigiai{
		Thể tích của khối tròn xoay tạo thành là
		\allowdisplaybreaks
		\begin{eqnarray*}
			V&=&\pi\displaystyle\int\limits_4^9\left(\sqrt{x}-2\right)^2\mathrm{\,d}x\\ &=&\pi\displaystyle\int\limits_4^9\left(x-4\sqrt{x}+4\right)\mathrm{\,d}x\\
			&=&\pi\cdot \left(\dfrac{x^2}{2}-\dfrac{8x\sqrt{x}}{3}+4x\right)\Bigg|_4^9\\
			&=&\pi\left(\dfrac{81}{2}-72+36\right)-\pi\left(\dfrac{16}{2}-\dfrac{64}{3}+16\right)=\dfrac{11\pi}{6}.
		\end{eqnarray*}
	}
\end{ex}

%Câu 18
\begin{ex}%[2D4H3-3]
	Cho hình phẳng $(H)$ giới hạn bởi các đường thẳng $y=x^2+2$, $y=0$, $x=1$, $x=2$. Gọi $V$ là thể tích của khối tròn xoay được tạo thành khi quay $(H)$ xung quanh trục $Ox$. Mệnh đề nào dưới đây đúng?
	\choice
	{$V=\displaystyle\int\limits_1^2\left(x^2+2\right)\mathrm{\,d}x$}
	{\True $V=\pi\displaystyle\int\limits_1^2\left(x^2+2\right)^2\mathrm{\,d}x$}
	{$V=\displaystyle\int\limits_1^2\left(x^2+2\right)^2\mathrm{\,d}x$}
	{$V=\pi\displaystyle\int\limits_1^2\left(x^2+2\right)\mathrm{\,d}x$}
	\loigiai{
		Ta có $V=\pi\displaystyle\int\limits_1^2\left(x^2+2\right)^2\mathrm{\,d}x$.
	}
\end{ex}
\Closesolutionfile{ans}
% \indapan{6}{ans/ans-2-C4B3CD2-lc}

\Opensolutionfile{ans}[ans/ans-2-C4B3CD2_5-10-KQ]
%\TNSA

%Câu 19
\begin{ex}%[2D4H3-4]
	Cắt một vật thể $(T)$ bởi hai mặt phẳng vuông góc với trục $Ox$ tại $x=0$ và $x=2$. Một mặt phẳng tùy ý vuông góc với trục $Ox$ tại điểm có hoành độ $x$ ($0\le x\le 2$) cắt vật thể đó có theo một thiết diện là một hình vuông có cạnh bằng $\sqrt{x^3}$. Thể tích vật thể $(T)$ là số hữu tỉ có dạng phân số tối giản $\dfrac{a}{b}$. Tính $a+b$.
	\shortans{$135$}
	\loigiai{
		Diện tích thiết diện là $S(x)=\sqrt{x^3}\cdot \sqrt{x^3}=x^6$.\\
		Thể tích của vật thể $(T)$ là $V=\displaystyle\int\limits_0^2S(x)\mathrm{\,d}x=\displaystyle\int\limits_0^2x^6\mathrm{\,d}x=\dfrac{128}{7}$.\\
		Suy ra $a=128$ và $b=7$. Khi đó, $a+b=135$.
	}
\end{ex}

%Câu 20
\begin{ex}%[2D4H3-4]
	Cắt một vật thể bởi hai mặt phẳng vuông góc với trục $Ox$ tại $x=1$; $x=3$. Khi cắt một vật thể bởi mặt phẳng vuông góc với trục $Ox$ tại điểm có hoành độ $x$ ($1\le x\le 3$), mặt cắt là tam giác vuông có một góc $45^\circ$ và độ dài một cạnh góc vuông là $\sqrt{4-\dfrac{1}{2} x^2}$. Thể tích vật thể trên là một số hữu tỉ có dạng phân số tối giản $\dfrac{a}{b}$. Tính $a\cdot b$.
	\shortans{$66$}
	\loigiai{
		Diện tích tam giác vuông cân là $S(x)=\dfrac{1}{2}\sqrt{4-\dfrac{1}{2} x^2}\cdot \sqrt{4-\dfrac{1}{2}x^2}=\dfrac{1}{2}\left(4-\dfrac{1}{2}x^2\right)$.\\
		Vậy thể tích vật thể là \[V=\displaystyle\int\limits_1^3\dfrac{1}{2}\left(4-\dfrac{1}{2}{x^2}\right)\mathrm{\,d}x=\dfrac{11}{6}.\]
		Suy ra $a=11$; $b=6$. Khi đó $a\cdot b=66$.
	}
\end{ex}

%Câu 21
\begin{ex}%[2D4H3-3]
	Tính thể tích khối tròn xoay khi quay hình phẳng $(H)$ xác định bởi các đường $y=\dfrac{1}{3}x^3-x^2$, $y=0$, $x=0$ và $x=3$ quanh trục $Ox$ (kết quả viết dưới dạng số thập phân và làm tròn đến hàng phần trăm).
	\shortans{$7{,}27$}
	\loigiai{
		Thể tích khối tròn xoay sinh ra khi quay hình phẳng $(H)$ quanh trục $Ox$ là
		$$V=\pi\displaystyle\int\limits_0^3\left(\dfrac{1}{3}x^3-x^2\right)^2\mathrm{\,d}x=\pi\displaystyle\int\limits_0^3\left(\dfrac{1}{9}x^6-\dfrac{2}{3}x^5+x^4\right)\mathrm{\,d}x=\dfrac{81\pi}{35} \approx 7{,}27.$$
	}
\end{ex}

%Câu 22
\begin{ex}%[2D4H3-3]
	Tính thể tích của vật thể tạo nên khi quay quanh trục $Ox$ hình phẳng $D$ giới hạn bởi đồ thị $(P)\colon y=2x-x^2$, trục $Ox$ và hai đường thẳng $x=0$, $x=2$ (Kết quả viết dưới dạng số thập phân và làm tròn đến hàng phần trăm).
	\shortans{$3{,}35$}
	\loigiai{
		Ta có
		\allowdisplaybreaks
		\begin{eqnarray*}
			V&=&\pi\displaystyle\int\limits_0^2\left(2x-x^2\right)^2\mathrm{\,d}x\\
			&=&\pi\displaystyle\int\limits_0^2\left(4x^2-4x^3+x^4\right)\mathrm{\,d}x\\
			&=&\pi\left(\dfrac{4}{3}{x^3}-x^4+\dfrac{1}{5}{x^5}\right)\Bigg|_0^2\\
			&=&\dfrac{16}{15}\pi\approx 3{,}35.
		\end{eqnarray*}
	}
\end{ex}

%Câu 23
\begin{ex}%[2D4H3-3]
	Cho hình phẳng giới hạn bởi các đường $y=\tan x$, $y=0$, $x=0$, $x=\dfrac{\pi}{4}$ quay xung quanh trục $Ox$. Tính thể tích vật thể tròn xoay được sinh ra (kết quả viết dưới dạng số thập phân và làm tròn một chữ số thập phân sau dấu phẩy).
	\shortans{$0{,}8$}
	\loigiai{
		Thể tích vật thể tròn xoay được sinh ra là
		\[V=\pi\displaystyle\int\limits_0^{\tfrac{\pi}{4}}{\tan^2x\mathrm{\,d}x}=\pi\displaystyle\int\limits_0^{\tfrac{\pi}{4}}{\left(\dfrac{1}{\cos^2x-1}\right)}\mathrm{\,d}x=\pi\left(\tan x-x\right)\Bigg|_0^{\tfrac{\pi}{4}}=\dfrac{4\pi-\pi^2}{4} \approx 0{,}8.\]
	}
\end{ex}

%Câu 24
\begin{ex}%[2D4H3-3]
	Gọi $V$ là thể tích khối tròn xoay tạo thành do quay xung quanh trục hoành một elip có phương trình $\dfrac{x^2}{25}+\dfrac{y^2}{16}=1$. Tính $V$ (Kết quả làm tròn đến hàng đơn vị).
	\shortans{$335$}
	\loigiai{
		Quay elip đã cho xung quanh trục hoành chính là quay hình phẳng $H$ giới hạn bởi $y=4\sqrt{1-\dfrac{x^2}{25}}$, $y=0$, $x=-5$, $x=5$.\\
		Vậy thể tích khối tròn xoay sinh ra bởi $H$ khi quay xung quanh trục hoành là
		\[V=\pi\displaystyle\int_{-5}^5\left(16-\dfrac{16x^2}{25}\right)\mathrm{\,d}x=\pi\left(16x-\dfrac{16x^3}{75}\right)\Bigg|^5_{-5}=\dfrac{320\pi}{3}\approx 335.\]
	}
\end{ex}

%Câu 25
\begin{ex}%[2D4H3-3]%Câu 13
	\immini{Cho hình phẳng $(H)$ được gạch chéo trong hình bên. Tính thể hình tròn xoay sinh ra bởi $(H)$ khi quay $(H)$ quanh trục $Ox$ (Kết quả viết dưới dạng số thập phân và làm tròn đến hàng phần chục).
	}{
		\begin{tikzpicture}[line join=round, line cap=round,>=stealth,thick,scale=0.7]
			\tikzset{every node/.style={scale=0.8}}
			\draw[->] (-3.1,0)--(3.1,0) node[below left] {$x$};
			\draw[->] (0,-1.1)--(0,5.1) node[below left] {$y$};
			\draw (0,0) node [below left] {$O$};
			\foreach \x/\nx in {1/1,2/2}
			\draw (\x,1pt)--(\x,-1pt) node [below left] {$\nx$};
			\foreach \y/\ny in {1/1,2/2,3/3,4/4}
			\draw (1pt,\y)--(-1pt,\y) node [left] {$\ny$};
			\begin{scope}
				\clip (-3,-1) rectangle (3,5);
				\draw[samples=200,domain=-2:2,smooth,variable=\x] plot (\x,{1*(\x)^2+0*(\x)+0});
				\fill[pattern=north east lines](1,0)--plot[samples=200,domain=1:2,smooth,variable=\x] (\x,{(\x)^2})--(2,0);
				\draw plot[samples=200,domain=-2:2.15,smooth,variable=\x] (\x,{(\x)^2}) node[left=2cm]{$y=x^2$};
				\draw (1,-0.8)--(1,4.3) (2,-0.8)--(2,4.3);
				\fill[black](1,1) circle (2pt);
				\fill[black](2,4) circle (2pt);
			\end{scope}
		\end{tikzpicture}
	}
	\shortans{$19{,}5$}
	\loigiai{
		Ta có $V=\pi\displaystyle\int_1^2{\left(x^2\right)^2\mathrm{\,d}x}=\pi\dfrac{x^5}{5}\Bigg|^2_1=\dfrac{31\pi}{5}\approx 19{,}5$.
	}
\end{ex}

%Câu 26
\begin{ex}%[2D4H3-3]
	\immini{Cho hình phẳng $(D)$ được tô màu trong hình bên. Tính thể hình tròn xoay sinh ra bởi $(D)$ khi quay $(D)$ quanh trục $Ox$ (Kết quả viết dưới dạng số thập phần và làm tròn đến hàng phần trăm).
	}{
		\begin{tikzpicture}[line join=round, line cap=round,>=stealth,thick]
			\tikzset{every node/.style={scale=0.9}}
			\draw[->] (-1.1,0)--(3.1,0) node[below left] {$x$};
			\draw[->] (0,-1.1)--(0,3.1) node[below left] {$y$};
			\draw (0,0) node [below left] {$O$};
			\foreach \x/\nx in {1/1,2/2}
			\draw[thin] (\x,1pt)--(\x,-1pt) node [below] {$\nx$};
			\foreach \y/\ny in {1/1,2/2}
			\draw[thin] (1pt,\y)--(-1pt,\y) node [left] {$\ny$};
			\begin{scope}
				\clip (-1,-1) rectangle (3,3);
				\draw[pattern=north east lines](1,0)--plot[samples=200,domain=1:2,smooth,variable=\x] (\x,{1+1/(\x)})--(2,0);
				\draw plot[samples=200,domain=0.1:2.7,smooth,variable=\x] (\x,{1+1/(\x)});
				\draw (1.7,2.5) node{$y=1+\dfrac{1}{x}$};
				\draw[dashed](1,2)--(0,2);			
			\end{scope}
			\draw (1.5,1) node[circle, fill=white] {$\mathrm{D}$};
		\end{tikzpicture}
	}
	\shortans{$9{,}08$}
	\loigiai{
		Ta có $V=\pi\displaystyle\int_1^2{\left(1+\dfrac{1}{x}\right)^2\mathrm{\,d}x}=\pi\displaystyle\int_1^2{\left(1+\dfrac{2}{x}+\dfrac{1}{x^2}\right)\mathrm{\,d}x}=\pi\left(x+\ln x-\dfrac{1}{x}\right)\Bigg|^2_1 \approx 9{,}08$.
	}
\end{ex}

%Câu 27
\begin{ex}%[2D4H3-3]
	\immini{Cho hình phẳng $(H)$ được tô màu trong hình bên. Tính thể hình tròn xoay sinh ra bởi $(H)$ khi quay $(H)$ quanh trục $Ox$ (Kết quả viết dưới dạng số thập phân và làm tròn đến hàng phần chục)
	}{
		\begin{tikzpicture}[line join=round, line cap=round,>=stealth,thick]
			\tikzset{every node/.style={scale=0.9}}
			\draw[->] (-1.6,0)--(2.1,0) node[below left] {$x$};
			\draw[->] (0,-1.1)--(0,3.1) node[below left] {$y$};
			\draw (0,0) node [below left] {$O$};
			\foreach \x/\nx in {-1/-1,1/1}
			\draw[thin] (\x,1pt)--(\x,-1pt) node [below] {$\nx$};
			\foreach \y/\ny in {1/1}
			\draw[thin] (1pt,\y)--(-1pt,\y) node [left] {$\ny$};
			\begin{scope}
				\clip (-1.5,-1) rectangle (2,3);
				\draw[pattern=north east lines](-1,0)--plot[samples=200,domain=-1:1,smooth,variable=\x] (\x,{e^(\x)})--(1,0);
				\draw[samples=200,domain=-1.5:2,smooth,variable=\x] plot (\x,{e^(\x)});
				\path (0,1)--(1,e) node[pos=0.7, above, sloped]{$y=\mathrm{e}^x$};
			\end{scope}
		\end{tikzpicture}
	}
	\shortans{$11{,}4$}
	\loigiai{
		Ta có $V=\pi\displaystyle\int_{-1}^1{\left(\mathrm{e}^x\right)^2\mathrm{\,d}x}=\pi\displaystyle\int_{-1}^1{\left(\mathrm{e}^{2x}\right)\mathrm{\,d}x}=\dfrac{\pi}{2}\mathrm{e}^{2x}\Bigg|^1_{-1} \approx 11{,}4$.
	}
\end{ex}

%Câu 28
\begin{ex}%[2D4H3-3]
	\immini{Cho hình phẳng $(H)$ được tô màu trong hình bên. Tính thể hình tròn xoay sinh ra bởi $(H)$ khi quay $(H)$ quanh trục $Ox$ (Kết quả viết dưới dạng số thập phân và làm tròn đến hàng phần chục).
	}{
		\begin{tikzpicture}[line join=round, line cap=round,>=stealth,thick]
			\tikzset{every node/.style={scale=0.9}}
			\draw[->] (-1.1,0)--(3.1,0) node[below left] {$x$};
			\draw[->] (0,-1.1)--(0,3.1) node[below left] {$y$};
			\draw (0,0) node [below left] {$O$};
			\foreach \x/\nx in {1/1,2/2}
			\draw[thin] (\x,1pt)--(\x,-1pt) node [below] {$\nx$};
			\draw[thin] (1pt,1)--(-1pt,1) node [below left] {$1$};
			\draw[thin] (1pt,2)--(-1pt,2) node [left] {$2$};
			\begin{scope}
				\clip (-1,-1) rectangle (3,3);
				\draw[pattern=north east lines](0,0)--(0,1)--(2,2)--(2,0);
				\draw[dashed](0,2)--(2,2);
				\fill[black](0,1) circle (1.5pt) node[above left]{$A$};
				\fill[black](2,2) circle (1.5pt) node[right]{$B$};
				\fill[black](2,0) circle (1.5pt) node[above right]{$C$};
			\end{scope}
		\end{tikzpicture}
	}
	\shortans{$14{,}7$}
	\loigiai{
		Gọi đường thẳng $d$ đi qua $A$ và $B$ có phương trình dạng $y=ax+b$.\\
		Ta có hệ phương trình $\heva{&b=1\\&2a+b=2} \Rightarrow \heva{&a=\dfrac{1}{2}\\&b=1.}$\\
		Suy ra $d \colon y=\dfrac{1}{2}x+1$.\\
		Khi đó
		$V=\pi\displaystyle\int_0^1{\left(\dfrac{1}{2}x+1\right)^2\mathrm{\,d}x} \approx 14{,}7$.
	}
\end{ex}

%Câu 29
\begin{ex}%[2D4V3-3]
	\immini{Cho hình phẳng $(H)$ là tam giác cong $OAB$ trong hình vẽ bên. Tính thể hình tròn xoay sinh ra bởi $(H)$ khi quay $(H)$ quanh trục $Ox$ (Kết quả viết dưới dạng số thập phân và làm tròn đến hàng phần trăm).
	}{
		\begin{tikzpicture}[line join=round, line cap=round,>=stealth,thick]
			\tikzset{every node/.style={scale=0.9}}
			\draw[dashed, step=1, gray!50,very thin] (-.5,-0.9) grid (4.5,4.5);
			\draw[->] (-1.6,0)--(5.1,0) node[below] {$x$};
			\draw[->] (0,-1.8)--(0,4.5) node[right] {$y$};
			\draw (0,0) node [below left] {$O$};
			\foreach \x/\nx in {-1/-1,1/1,2/2,3/3,4/4}
			\draw[thin] (\x,1pt)--(\x,-1pt) node [below] {$\nx$};
			\foreach \y/\ny in {-1/-1,1/1,2/2,3/3,4/4}
			\draw[thin] (1pt,\y)--(-1pt,\y) node [left] {$\ny$};
			\begin{scope}
				\clip (-1.5,-1.5) rectangle (4.5,4.5);
				\draw plot[samples=200,domain=-1.2:1.7,smooth,variable=\x] (\x,{(1*(\x)^3});
				\path (1,1)--(2,8) node[pos=0.4,above, sloped]{$y=x^3$};
				\draw plot[samples=200,domain=-0.3:4.3,smooth,variable=\x] (\x,{1*(\x)^2+-4*(\x)+4});
				\fill [pattern=north east lines](0,0)--plot[samples=200,domain=0:1,smooth,variable=\x] (\x,{(\x)^3})--plot[samples=200,domain=1:2,smooth,variable=\x] (\x,{(\x)^2-4*(\x)+4})--(2,0)--cycle;
				\path (3,1)--(4,4) node[pos=0.55,above, sloped]{$y=x^2-4x+4$};
				\fill[black](1,1) circle (1.5pt) node[right]{$A$};
				\fill[black](2,0) circle (1.5pt) node[above]{$B$};
			\end{scope}
		\end{tikzpicture}
	}
	\shortans{$1{,}08$}
	\loigiai{
		Ta có $V=\pi\displaystyle\int_0^1{\left(x^3\right)^2\mathrm{\,d}x}+\pi\displaystyle\int_1^2{\left(x^2-4x+4\right)^2\mathrm{\,d}x} \approx 1{,}08$.
	}
\end{ex}

%Câu 30
\begin{ex}%[2D4V3-3]
	\immini{Gọi $V$ là thể tích khối tròn xoay tạo thành khi quay hình phẳng giới hạn bởi các đường $y=\sqrt{x}$, $y=0$ và $x=4$ quanh trục $Ox$. Đường thẳng $x=a$, $\left(0<a<4\right)$ cắt đồ thị hàm số $y=\sqrt{x}$ tại $M$ (hình vẽ). Gọi $V_1$ là thể tích khối tròn xoay tạo thành khi quay tam giác $OMH$ quanh trục $Ox$. Biết rằng $V=2V_1$. Tìm $a$.
	}{
		\begin{tikzpicture}[line join=round, line cap=round,>=stealth,thick]
			\tikzset{every node/.style={scale=0.9}}
			\draw[->] (-0.6,0)--(5.1,0) node[below left] {$x$};
			\draw[->] (0,-0.4)--(0,2.5) node[below left] {$y$};
			\draw (0,0) node [below left] {$O$};
			\foreach \x/\nx in {3/a,4/4}
			\draw[thin] (\x,1pt)--(\x,-1pt) node [below] {$\nx$};
			\begin{scope}
				\clip (-1.0,-1.0) rectangle (4.5,2.5);
				\draw plot[samples=200,domain=0:4.5,smooth,variable=\x] (\x,{sqrt((\x))});
				\path (0,0)--(4,1) node[pos=0.45,above=0.8cm, sloped]{$y=\sqrt x$};
				\fill[black](3,1.732) circle (1.5pt) node[above right]{$M$} (4,0) node[above right]{$H$};
				\draw (3,2)--(3,-0.1);
				\draw[pattern=north east lines](0,0)--(3,1.732)--(4,0);
			\end{scope}
		\end{tikzpicture}
	}
	\shortans{$3$}
	\loigiai{
		\immini{
			Ta có $V=\pi\displaystyle\int\limits_0^4x\mathrm{\,d}x=\pi\dfrac{x^2}{2}\Bigg|_0^4=8\pi$.\\
			Mà $V=2V_1\Rightarrow{V_1}=4\pi$.\\
			Gọi $K$ là hình chiếu của $M$ trên $Ox$.\\
			Suy ra $OK=a$, $KH=4-a$, $MK=\sqrt a$.\\
			Khi xoay tam giác $OMH$ quanh $Ox$ ta được khối
		}{
			\begin{tikzpicture}[line join=round, line cap=round,>=stealth,thick]
				\tikzset{every node/.style={scale=0.9}}
				\draw[->] (-0.6,0)--(5.1,0) node[below left] {$x$};
				\draw[->] (0,-0.4)--(0,2.5) node[below left] {$y$};
				\draw (0,0) node [below left] {$O$};
				\foreach \x/\nx in {3/a,4/4}
				\draw[thin] (\x,1pt)--(\x,-1pt) node [below] {$\nx$};
				\begin{scope}
					\clip (-1.0,-1.0) rectangle (4.5,2.5);
					\draw plot[samples=200,domain=0:4.5,smooth,variable=\x] (\x,{sqrt((\x))});
					\path (0,0)--(4,1) node[pos=0.45,above=0.8cm, sloped]{$y=\sqrt x$};
					\fill[black](3,1.732) circle (1.5pt) node[above right]{$M$} (4,0) node[above right]{$H$} (3,0)node[below right]{$K$};
					\draw (3,2)--(3,-0.1);
					\draw[pattern=north east lines](0,0)--(3,1.732)--(4,0);
				\end{scope}
			\end{tikzpicture}
		}\hspace{-0.77cm}
		tròn xoay là sự lắp ghép của hai khối nón sinh bởi các tam giác $OMK$, $MHK$, hai khối nón đó có cùng mặt đáy và có tổng chiều cao là $OH=4$ nên thể tích của khối tròn xoay đó là $V_1=\dfrac{1}{3} \cdot \pi \cdot 4 \cdot \left(\sqrt a\right)^2=\dfrac{4\pi a}{3}$, từ đó suy ra $a=3$.
	}
\end{ex}
\Closesolutionfile{ans}
% \indapan{6}{ans/ans-2-C4B3CD2_5-10-KQ}