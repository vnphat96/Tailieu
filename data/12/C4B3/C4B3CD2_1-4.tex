%\setcounter{chude}{1}
\chude{	THỂ TÍCH KHỐI TRÒN XOAY}
%\dang{ỨNG DỤNG HÌNH HỌC CỦA TÍCH PHÂN }
\begin{tomtat}
	\subsection{Thế tích của vật thế}
	\begin{center}
	\begin{tikzpicture}[>=stealth, scale=0.8]
		\draw plot[smooth,tension=.65] coordinates{(1,2) (2.5,2.3) (3.5,2.2)};
		\draw[dashed] plot[smooth,tension=.65] coordinates{(3.5,2.2) (4,2)};
		\draw plot[smooth,tension=.65] coordinates{(4,2) (5,2.2) (5.5,2.1)};
		\draw[dashed] plot[smooth,tension=.65] coordinates{(5.5,2.1) (6,2)};
		\draw plot[smooth,tension=.65] coordinates{(1,1) (2.3,0.5) (3.5,0.8)};
		\draw[dashed] plot[smooth,tension=.65] coordinates{(3.5,0.8) (4,1)};
		\draw plot[smooth,tension=.65] coordinates{(4,1) (5,0.7) (5.5,0.8)};
		\draw[dashed] plot[smooth,tension=.65] coordinates{(5.5,0.8) (6,1)};
		\draw[dashed] (1,1) arc (-90:90:.2 and 0.5);
		\draw (1,2) arc (90:270:.2 and 0.5);
		\draw[dashed] (4,1) arc (-90:90:.2 and 0.5);
		\draw (4,2) arc (90:270:.2 and 0.5);
		\draw (6,1) arc (-90:270:.2 and 0.5);
		\fill[pattern=north east lines] (4,1) arc (-90:90:.2 and 0.5)--(4,2) arc (90:270:.2 and 0.5)--cycle;
		\draw (-.5,0)--(0.5,0) (1,0)--(3.5,0) (4,0)--(5.5,0);
		\draw[dashed] (0.5,0)--(1,0) (3.5,0)--(4,0) (5.5,0)--(6,0);
		\draw[->] (6,0)--(7,0)node[below]{$x$};
		\draw (0.5,-1)--(0.5,3)--(1.5,3.5)--(1.5,2.2) (1.5,.8)--(1.5,-0.5)--(0.5,-1);
		\draw[dashed](1.5,2.2)--(1.5,.8);
		\draw[dashed] (1,1)--(1,0)node[below]{$a$};
		\coordinate (A) at (0.5,3);
		\coordinate (B) at (1.5,3.5);
		\coordinate (C) at (1.5,2.2);
		%\tkzMarkAngle[size=.6](A,B,C);
		\draw pic[draw=black, angle eccentricity=1.6, angle radius=0.5cm]{angle=A--B--C};
		\draw (1.3,3.2) node {\footnotesize $P$};
		\draw (3.5,-1)--(3.5,3)--(4.5,3.5)--(4.5,2) (4.5,1)--(4.5,-0.5)--(3.5,-1);
		\draw[dashed](4.5,2)--(4.5,1);
		\draw[dashed] (4,1)--(4,0)node[below]{$x$};
		\coordinate (D) at (3.5,3);
		\coordinate (E) at (4.5,3.5);
		\coordinate (F) at (4.5,2);
	%	\tkzMarkAngle[size=.6](D,E,F);
		\draw pic[draw=black, angle eccentricity=1.6, angle radius=0.5cm]{angle=D--E--F};
		\draw (4.3,3.2) node {\footnotesize $R$};
		\draw (5.5,-1)--(5.5,3)--(6.5,3.5)--(6.5,-0.5)--(5.5,-1);
		\draw[dashed] (6,1)--(6,0)node[below]{$b$};
		\coordinate (G) at (5.5,3);
		\coordinate (H) at (6.5,3.5);
		\coordinate (K) at (6.5,-0.5);
	%	\tkzMarkAngle[size=.6](G,H,K);
		\draw pic[draw=black, angle eccentricity=1.6, angle radius=0.5cm]{angle=G--H--K};	\draw (6.3,3.2) node {\footnotesize $Q$};
		\draw (0,.3) node {$O$};
		\fill (0,0) circle(1pt);
		\draw[->] (4,1.5)--(4.7,1.7) node[right] {\scriptsize $S(x)$};
	\end{tikzpicture}
	\end{center}
	Trong không gian, cho một vật thể nằm trong khoảng không gian giữa hai mặt phẳng $(P)$ và $(Q)$ cùng vuông góc với trục $O x$ tại các điểm $a$ và $b$. Mặt phẳng vuông góc với trục $O x$ tại điểm $x(a \leq x \leq b)$ cắt vật thể theo mặt cắt có diện tích $S(x)$. Khi đó, nếu $S(x)$ là hàm số liên tục trên $\left[a ; b\right]$ thì thể tích của vật thể được tính bởi công thức
	$$
	V=\displaystyle\int\limits_a^b S(x) \mathrm{\,d} x.
	$$
\subsection{Thế tích khối tròn xoay}
	\begin{center}
		\begin{tikzpicture}[line join=round, line cap=round,>=stealth,thick,scale=.4]
		\tikzset{label style/.style={font=\normalsize}}
		%%Nhập giới hạn đồ thị và hàm số cần vẽ
		\def \xmin{-.5}
		\def \xmax{11}
		\def \ymin{-4}
		\def \ymax{4.5}
		%\draw[xstep=1 cm, ystep=1 cm,gray,thin] (\xmin,\ymin) grid (\xmax,\ymax);
		\def \hamso{0-0.01864083398323228*((\x)-1.0)^(3.0)+0.35126127715584465*((\x)-1.0)^(2.0)-1.5117402856674746*((\x)-1.0)+3.106314636847822}
		%%Tự động
		\draw[->] (\xmin,0)--(11,0) node[below left] {$x$};
		\draw[->] (0,\ymin)--(0,\ymax) node[below left] {$y$};
		\draw (0,0) node [below left] {$O$};
		\draw[dashed] (2.04,-1.89) arc(-90:90:.5 cm and 1.89 cm);
		\draw (2.04,1.89) arc(90:270:.5 cm and 1.89 cm);
		
		%\draw[dashed] (5.59,-1.77) arc(-90:90:.5 cm and 1.77 cm);
		%	\draw (5.59,1.77) arc(90:270:.5 cm and 1.77 cm);
		
		\draw (9,0) ellipse (1 cm and 3.96 cm);
		
		\draw[fill=black] (2.04,0) node [below] {$a$} circle (1.2pt);
		\draw[fill=black] (9,0) node [below] {$b$} circle (1.2pt);
		\draw[fill=black] (5.5,1.8) node [above,rotate=30] {\scriptsize $y=f(x)$};
		%%Tự động
		\begin{scope}
			\clip (\xmin+0.01,\ymin+0.01) rectangle (\xmax-0.01,\ymax-0.01);
			\draw[samples=350,domain=1:9.,smooth,variable=\x] plot (\x,{\hamso});
			\draw[samples=350,domain=1:9,smooth,variable=\x] plot (\x,{-0+0.01864083398323228*((\x)-1.0)^(3.0)-0.35126127715584465*((\x)-1.0)^(2.0)+1.5117402856674746*((\x)-1.0)-3.106314636847822});		
			\draw[pattern=north west lines,opacity=0.5] (2.04,0)--(2.04,1.89)plot[domain=2.04:9] (\x,{0-0.01864083398323228*((\x)-1.0)^(3.0)+0.35126127715584465*((\x)-1.0)^(2.0)-1.5117402856674746*((\x)-1.0)+3.106314636847822})--(9.0,0)--(2.04,0);
		\end{scope}
	\end{tikzpicture}
	\end{center}
	Cho hàm số $y=f(x)$ liên tục, không âm trên $\left[a ; b\right]$. Hình phẳng $(H)$ giới hạn bởi đồ thị hàm số $y=f(x)$, trục hoành $O x$ và hai đường thẳng $x=a$ và $x=b$ quay quanh trục $O x$ tạo thành một khối tròn xoay có thể tích bằng
	$$
	V=\pi \displaystyle\int\limits_a^b\left[f(x)\right]^2 \mathrm{\,d} x
	$$
\end{tomtat}
\TN
\Opensolutionfile{ans}[ans/ans-2-C4B3CD2-lc]
\begin{ex}%[2D4N3-3]
Viết công thức tính thể tích $V$ của khối tròn xoay được tạo ra khi quay hình thang cong, giới hạn bới đồ thị hàm số $y=f(x)$, trục $O x$ và hai đường thẳng $x=a$, $x=b$, $(a<b)$ xung quanh trục $O x$.
\choice
{$V=\displaystyle\int\limits_a^b \left|f(x)\right| \mathrm{\,d} x$}
{\True $V=\pi \displaystyle\int\limits_a^b f^2(x) \mathrm{\,d} x $}
{$V=\displaystyle\int\limits_a^b f^2(x) \mathrm{\,d} x$}
{$V=\pi \displaystyle\int\limits_a^b f(x) \mathrm{\,d} x$}
\loigiai{
Theo lí thuyết.
}
\end{ex}
\begin{ex}%[2D4N3-4]
	Cắt một vật thể bởi hai mặt phẳng vuông góc với trục $O x$ tại $x=1$ và $x=2$. Một mặt phẳng tùy ý vuông góc với trục $O x$ tại điểm có hoành độ $x$, $(1 \leq x \leq 2)$ cắt vật thể đó có diện tích $S(x)=2024 x$. Tính thể tích của phần vật thể giới hạn bởi hai mặt phẳng trên.
	\choice
	{\True $V=3036$}
	{$V=3036 \pi$}
	{$V=1518$}
	{$V=1518 \pi$}
\loigiai{Thể tích vật thể là $ V=\displaystyle\int\limits_1^2 2024x \mathrm{\,d}x=3036 $.}
\end{ex}

\begin{ex}%[2D4H3-4]
Cắt một vật thể bởi hai mặt phẳng vuông góc với trục $O x$ tại $x=1$ và $x=3$. Một mặt phẳng tùy ý vuông góc với trục $O x$ tại điểm có hoành độ $x$, $(1 \leq x \leq 3)$ cắt vật thể đó theo thiết diện là một hình chữ nhật có độ dài hai cạnh là $3 x$ và $3 x^2-2$. Tính thể tích của phần vật thể giới hạn bởi hai mặt phẳng trên.
\choice
{\True $V=156$}
{$V=156 \pi$}
{$ V=312 $}
{$V=312 \pi$}
\loigiai{Diện tích thiết diện là $S(x)=3 x \cdot \left(3 x^2-2\right)=9 x^3-6 x$.\\
Thể tích vật thể là $V=\displaystyle\int\limits_1^3\left(9 x^3-6 x\right) \mathrm{\,d} x=156$.
}
\end{ex}

\begin{ex}%[2D4N3-3]
Gọi $D$ là hình phẳng giới hạn bởi các đường $y=\mathrm{e}^{3 x}$, $y=0$, $x=0$ và $x=1$. Thể tích của khối tròn xoay tạo thành khi quay $D$ quanh trục $O x$ bằng
\choice
{$\pi \displaystyle\int\limits_0^1 \mathrm{e}^{3 x} \mathrm{\,d} x$}
{$\displaystyle\int\limits_0^1 \mathrm{e}^{6 x} \mathrm{\,d} x$}
{\True $\pi \displaystyle\int\limits_0^1 \mathrm{e}^{6 x} \mathrm{\,d} x$}
{$\displaystyle\int\limits_0^1 \mathrm{e}^{3 x} \mathrm{\,d} x$}
\loigiai{
Thể tích của khối tròn xoay tạo thành khi quay $D$ quanh trục $O x$ là\\
$$\pi \displaystyle\int\limits_0^1\left(\mathrm{e}^{3 x}\right)^2 \mathrm{\,d} x=\pi \displaystyle\int\limits_0^1 \mathrm{e}^{6 x} \mathrm{\,d} x.$$}
\end{ex}

\begin{ex}%[2D4N3-3]
Gọi $D$ là hình phẳng giới hạn bởi các đường $y=\mathrm{e}^{4 x}$, $y=0$, $x=0$ và $x=1$. Thể tích của khối tròn xoay tạo thành khi quay $D$ quanh trục $O x$ bằng
\choice
{$\displaystyle\int\limits_0^1 \mathrm{e}^{4 x} \mathrm{\,d} x$}
{\True $\pi \displaystyle\int\limits_0^1 \mathrm{e}^{8 x} \mathrm{\,d} x$}
{$\pi \displaystyle\int\limits_0^1 \mathrm{e}^{4 x} \mathrm{\,d} x$}
{$\displaystyle\int\limits_0^1 \mathrm{e}^{8 x} \mathrm{\,d} x$}
\loigiai{
Thể tích của khối tròn xoay tạo thành khi quay $D$ quanh trục $O x$ là 
$$V=\pi \displaystyle\int\limits_0^1\left(\mathrm{e}^{4 x}\right)^2 \mathrm{\,d} x=\pi \displaystyle\int\limits_0^1 \mathrm{e}^{8 x} \mathrm{\,d} x.$$}
\end{ex}

\begin{ex}%[2D4N3-3]
Cho hình phẳng $(H)$ giới hạn bởi các đường $y=x^2+3$, $y=0$, $x=0$, $x=2$. Gọi $V$ là thể tích của khối tròn xoay được tạo thành khi quay $(H)$ xung quanh trục $O x$. Mệnh đề nào dưới đây đúng?
\choice
{$V=\displaystyle\int\limits_0^2\left(x^2+3\right) \mathrm{\,d}x$}
{$V=\pi \displaystyle\int\limits_0^2\left(x^2+3\right) \mathrm{\,d}x$}
{$V=\displaystyle\int\limits_0^2\left(x^2+3\right)^2 \mathrm{\,d}x$}
{\True $V=\pi \displaystyle\int\limits_0^2\left(x^2+3\right)^2 \mathrm{\,d}x$}
\loigiai{Thể tích của khối tròn xoay được tạo thành khi quay $(H)$ xung quanh trục $O x$ là\\
$V=\pi \displaystyle\int\limits_0^2\left(x^2+3\right)^2 \mathrm{\,d} x$.
}
\end{ex}

\begin{ex}%[2D4H3-3]
Cho hình phẳng $D$ giới hạn bởi đường cong $y=\mathrm{e}^x$, trục hoành và các đường thẳng $x=0$, $x=1$. Khối tròn xoay tạo thành khi quay $D$ quanh trục hoành có thể tích $V$ bằng bao nhiêu?
\choice
{$V=\dfrac{\pi\left(\mathrm{e}^2+1\right)}{2}$}
{$V=\dfrac{\mathrm{e}^2-1}{2}$}
{$V=\dfrac{\pi \mathrm{e}^2}{3}$}
{\True $V=\dfrac{\pi\left(\mathrm{e}^2-1\right)}{2}$}
\loigiai{
$V=\pi \displaystyle\int\limits_0^1 \mathrm{e}^{2 x} \mathrm{\,d} x=\pi \dfrac{\mathrm{e}^{2x}}{2} \Bigg|_0 ^1=\dfrac{\pi\left(e^2-1\right)}{2}$.}
\end{ex}

\begin{ex}%[2D4H3-3]
Cho hình phẳng $D$ giới hạn bởi đường cong $y=\sqrt{x^2+1}$, trục hoành và các đường thẳng $x=0$, $x=1$. Khối tròn xoay tạo thành khi quay $D$ quanh trục hoành có thể tích $V$ bằng bao nhiêu?
\choice
{$ V=2 $}
{\True $V=\dfrac{4 \pi}{3} $}
{$V=2 \pi$}
{$V=\dfrac{4}{3}$}
\loigiai{Thể tích khối tròn xoay được tính theo công thức
	$$
	V=\pi \displaystyle\int\limits_0^1\left(\sqrt{x^2+1}\right)^2 \mathrm{\,d} x=\pi \displaystyle\int\limits_0^1\left(x^2+1\right) \mathrm{\,d} x=\pi\left(\dfrac{x^3}{3}+x\right)\Bigg|_0 ^1=\dfrac{4 \pi}{3} .
	$$}
\end{ex}

\begin{ex}%[2D4H3-3]
Cho hình phẳng $D$ giới hạn bởi đường cong $y=\sqrt{2+\cos x}$, trục hoành và các đường thẳng $x=0, x=\dfrac{\pi}{2}$. Khối tròn xoay tạo thành khi $D$ quay quanh trục hoành có thể tích $V$ bằng bao nhiêu?
\choice
{\True $V=(\pi+1) \pi$}
{$V=\pi-1$}
{$V=\pi+1$}
{$V=(\pi-1) \pi$}
\loigiai{Ta có 
	$$
	V=\pi \displaystyle\int\limits_0^{\tfrac{\pi}{2}}(\sqrt{2+\cos x})^2 \mathrm{\,d} x=\pi(2 x+\sin x)\Bigg|_0 ^{\tfrac{\pi}{2}}=\pi(\pi+1).
	$$}
\end{ex}

\begin{ex}%[2D4H3-3]
Cho hình phẳng $D$ giới hạn bởi đường cong $y=\sqrt{2+\sin x}$, trục hoành và các đường thẳng $x=0$, $ x=\pi$. Khối tròn xoay tạo thành khi quay $D$ quay quanh trục hoành có thể tích $V$ bằng bao nhiêu?
\choice
{\True $V=2 \pi(\pi+1)$}
{$V=2 \pi$}
{$V=2(\pi+1)$}
{$V=2 \pi^2$}
\loigiai{Ta có $V=\pi \displaystyle\int\limits_0^\pi\left(\sqrt{2+\sin x}\right)^2 \mathrm{\,d} x=\pi \displaystyle\int\limits_0^\pi\left(2+\sin x\right) \mathrm{\,d} x=\pi(2 x-\cos x)\Bigg|_0 ^\pi=2 \pi\left(\pi+1\right)$.}
\end{ex}

\begin{ex}%[2D4H3-3]
Tìm công thức tính thể tích của khối tròn xoay khi cho hình phẳng giới hạn bởi parabol $(P)\colon y=x^2$, đường thẳng $d\colon y=2 x$ và đường thẳng $x=0$, $x=2$ quay xung quanh trục $O x$.
\choice
{$\pi \displaystyle\int\limits_0^2\left(x^2-2 x\right)^2 \mathrm{\,d} x$}
{\True $\pi \displaystyle\int\limits_0^2 4 x^2 \mathrm{\,d} x-\pi \int_0^2 x^4 \mathrm{\,d} x$}
{ $\pi \displaystyle\int\limits_0^2 4 x^2 \mathrm{\,d} x+\pi \int_0^2 x^4 \mathrm{\,d} x$}
{$\pi \displaystyle\int\limits_0^2\left(2 x-x^2\right) \mathrm{\,d} x$}
\loigiai{Với mọi $ x \in \left[0;2\right] $ ta có $ 2x\ge 0 $, $ x^2\ge 0 $ và $ 2x\ge x^2 $ nên $V=\pi \displaystyle\int\limits_0^2 4 x^2 \mathrm{\,d} x-\pi \displaystyle\int\limits_0^2 x^4 \mathrm{\,d} x$.
}
\end{ex}

\begin{ex}%[2D4N3-3]
 Cho hình phẳng $(H)$ giới hạn bởi các đường $y=x^2+3$, $y=0$, $x=0$, $x=2$. Gọi $V$ là thể tích khối tròn xoay được tạo hành khi quay $ (H) $ xung quanh trục $ Ox $. Mệnh đề nào sau đây đúng?
 \choice
 {\True $ V=\pi \displaystyle\int\limits_0^2 \left(x^2+3\right)^2 \mathrm{\,d}x $}
 {$ V=\displaystyle\int\limits_0^2 \left(x^2+3\right)\mathrm{\,d}x $}
 {$ V=\displaystyle\int\limits_0^2 \left(x^2+3\right)^2 \mathrm{\,d}x $}
 {$ V=\pi \displaystyle\int\limits_0^2 \left(x^2+3\right) \mathrm{\,d}x $}
 \loigiai{Thể tích của vật tròn xoay là $ V=\pi \displaystyle\int\limits_0^2 \left(x^2+3\right)^2 \mathrm{\,d}x $.}
\end{ex}