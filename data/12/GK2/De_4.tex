\begin{name}
	{\tenchude}
	{TOÁN 12}
	{LỚP TOÁN THẦY PHÁT}
	{Thời gian: 90 phút - Không kể thời gian phát đề}
\end{name}
\TN
\Opensolutionfile{ans}[ans/ansDe4-TN1]
\begin{ex}%[2D4N1-1]
	Cho hàm số $F(x)$ là một nguyên hàm của hàm số $f(x)$ trên $K$. Các mệnh đề sau, mệnh đề nào \textbf{sai}.
	\choice
	{$\displaystyle\int{f(x)\mathrm{\,d}x=}F(x)+C$}
	{$\displaystyle{\left(\displaystyle\int{f(x)\mathrm{\,d}x}\right)'}=f(x)$}
	{\True $\displaystyle{\left(\displaystyle\int{f(x)\mathrm{\,d}x}\right)'}=f'(x)$}
	{$\displaystyle{\left(\displaystyle\int{f(x)\mathrm{\,d}x}\right)'}=F'(x)$}
	\loigiai{
		Ta có $\displaystyle\int{f(x)\mathrm{\,d}x=}F(x)+C\Leftrightarrow F'(x)=f(x)$ nên phương án $\left(\displaystyle\int{f(x)\mathrm{\,d}x}\right)'=f'(x)$ sai.}
\end{ex}

\begin{ex}%[2D4N1-1]%[To 20 - Dot 17 - Chuong 4 - Bai 3 - CD - De 1 - TN]%[Nguyễn Hữu Duy]
	Nếu hàm số $f(x)$ liên tục trên đoạn $[a;b]$ và $c$ là số thực tùy ý thuộc đoạn $[a;b]$, thì tính chất nào sau đây đúng?
	\choice
	{\True $\displaystyle\int_a^b f(x) \mathrm{\,d}x = \displaystyle\int_a^c f(x) \mathrm{\,d}x + \displaystyle\int_c^b f(x)\mathrm{\,d}x$}
	{$\displaystyle\int_a^b f(x)\mathrm{\,d}x = \displaystyle\int_a^c f(x) \mathrm{\,d}x - \displaystyle\int_c^b f(x) \mathrm{\,d}x$}
	{$\displaystyle\int_a^b f(x)\mathrm{\,d}x = \displaystyle\int_a^c f(x)\mathrm{\,d}y + \displaystyle\int_c^b f(x) \mathrm{\,d}z$}
	{$\displaystyle\int_a^b f(x)\mathrm{\,d}x = \displaystyle\int_a^c f(x)\mathrm{\,d}y - \displaystyle\int_c^b f(x)\mathrm{\,d}z$}
	\loigiai{
		Theo định nghĩa tích phân ta có $\displaystyle\int_a^b f(x) \mathrm{\,d}x = \displaystyle\int_a^c f(x) \mathrm{\,d}x + \displaystyle\int_c^b f(x)\mathrm{\,d}x$.
	}
\end{ex}

\begin{ex}%[2D4N1-2]
	Cho hàm số $f(x)=x^2+4$. Mệnh đề nào sau đây đúng?

	\choice
	{$\displaystyle{\displaystyle\int f(x)\mathrm{\,d}x=2 x+C}$}
	{$\displaystyle{\displaystyle\int f(x)\mathrm{\,d}x=x^2+4 x+C}$}
	{\True $\displaystyle{\displaystyle\int f(x)\mathrm{\,d}x=\dfrac{x^3}{3}+4 x+C}$}
	{$\displaystyle{\displaystyle\int f(x)\mathrm{\,d}x=x^3+4 x+C}$}
	\loigiai{
		Ta có $f(x)=x^2+4 $ nên $ \displaystyle\int f(x)\mathrm{\,d}x=\dfrac{x^3}{3}+4 x+C$.}
\end{ex}

\begin{ex}%[2D4N1-3]
	Tìm nguyên hàm của hàm số $ f(x)=\cos 3x$.
	\choice
	{ $\displaystyle\int{\cos 3x\mathrm{d}x=3\sin 3x+C}$}
	{\True $\displaystyle\int{\cos 3x\mathrm{d}x=\dfrac{\sin 3x}{3}+C}$}
	{ $\displaystyle\int{\cos 3x\mathrm{d}x=\sin 3x+C}$}
	{$\displaystyle\int{\cos 3x\mathrm{d}x=-\dfrac{\sin 3x}{3}+C}$}
	\loigiai{
		Ta có:$\displaystyle\int{\cos 3x\mathrm{d}x=\dfrac{\sin 3x}{3}+C}$.
	}
\end{ex}

\begin{ex}%[2D4N1-4]
	Hàm số $F(x)=\mathrm{e}^{2x}$ là một nguyên hàm của hàm số nào dưới đây?
	\choice
	{$f_4(x)=\dfrac{1}{2}\mathrm{e}^{2x}$}
	{$f_1(x)=\mathrm{e}^{2x}$}
	{$f_2(x)=\mathrm{e}^{x^2}$}
	{\True $f_3(x)=2\mathrm{e}^{2x}$}
	\loigiai{
		Ta có $F'(x)=f(x)$ nên $f(x)=\left(\mathrm{e}^{2x}\right)^\prime=2\mathrm{e}^{2x}$.}
\end{ex}

\begin{ex}%[BG-12-4in1, Phạm Đức]%[2D4N1-4]
	Họ nguyên hàm của hàm số $f(x)=2024^x$ là
	\choice
	{\True $\dfrac{2024^x}{\ln 2024}+C$}
	{$2024^x\ln 2024+C$}
	{$2024^x+C$}
	{$2024^x\ln x+C$}
	\loigiai{

	}
\end{ex}

\begin{ex}%[2D4H1-2]
	Tìm nguyên hàm của hàm số $f(x)=(5x+3)^5$.
	\choice
	{$(5x+3)^6+C$}
	{$(5x+3)^4+C$}
	{\True $\dfrac{(5x+3)^6}{30}+C$}
	{$\dfrac{(5x+3)^4}{30}+C$}
	\loigiai{
		$f(x)=(5x+3)^5$ $\displaystyle \Rightarrow \displaystyle\int{f(x)\mathrm{\,d}x=}\displaystyle\int{(5x+ 3)^5\mathrm{\,d}x=}\dfrac{1}{5}\cdot \dfrac{(5x+3)^6}{6}+C=\dfrac{(5x+3)^6}{30}+C$.}
\end{ex}

\begin{ex}%[2D4H1-3]
	Họ nguyên hàm của hàm số $f(x)=\cos 2x$ là
	\choice
	{\True $\dfrac{1}{2} \sin 2x +C$}
	{$-2 \sin 2x +C$}
	{$-\dfrac{1}{2} \sin 2x +C$}
	{$2 \sin 2x +C$}
	\loigiai{
		Ta có  $\displaystyle\int\limits \cos 2x\mathrm{\,d}x=\dfrac{1}{2} \sin 2x+C$.
	}
\end{ex}

\begin{ex}%[2D4N2-1]
	Cho hàm số $f(t)$ liên tục trên $K$ và $a$, $b\in K$, $F(t)$ là một nguyên hàm của $ f(t)$ trên $K$. Chọn khẳng định \textbf{sai} trong các khẳng định sau.
	\choice
	{\True $F(a)-F(b)=\displaystyle\int\limits_a^b f(t)\mathrm{d}t$}
	{$\displaystyle\int\limits_a^bf(t)\mathrm{d}t=F(t)\big|^b_a$}
	{$\displaystyle\int\limits_a^bf(t)\mathrm{d}t=\left(\displaystyle\int f(t)\mathrm{d}t\right)\bigg|^b_a$}
	{$\displaystyle\int\limits_a^bf(x)\mathrm{d}x=\displaystyle\int\limits_a^bf(t)\mathrm{d}t$}
	\loigiai{
		Theo tính chất của tích phân.
		Ta có  $\displaystyle\int\limits_a^b f(t)\mathrm{d}t=F(b)-F(a)$.}
\end{ex}

\begin{ex}%[2D4N2-2]%[Tổ 20 - Đợt 17 - Chương 4 - - CD]%[Phạm Hà Giang]
	Tính tích phân $\displaystyle\int\limits_1^2 \dfrac{1}{x}\mathrm{\,d}x$
	\choice
	{$0$}
	{\True $\ln 2$}
	{$\dfrac{1}{2}$}
	{$\dfrac{-1}{2}$}
	\loigiai
	{
		$\displaystyle\int\limits_1^2 \dfrac{1}{x}\mathrm{\,d}x=\left. \ln \left| x\right| \right| _1^2 =\ln 2$.
	}
\end{ex}

\begin{ex}%[Nguyễn Tuấn, dự án sáng tác đề 12]%[2D4N2-3]
	Giá trị của $\displaystyle\int\limits_0^{\frac{\pi}{2}} \cos x\mathrm{\,d}x$ bằng
	\choice
	{$0$}
	{\True $1$}
	{$\dfrac{\pi}{2}$}
	{$\pi$}
	\loigiai
	{
	Ta có $\displaystyle\int\limits_0^{\frac{\pi}{2}} \cos x\mathrm{\,d}x = \sin x\Big|_0^{\frac{\pi}{2}} = \sin\dfrac{\pi}{2}-\sin 0 = 1$.
	}
\end{ex}

\begin{ex}%[Tổ 20 - Chương 4 - - CD]%[Nguyễn Văn Sang]%[2D4N2-4]
	Biết $I=\displaystyle\int\limits_0^1 3^x \cdot 4^{2 x} \cdot \mathrm{\,d} x=\dfrac{a}{\ln 48}$.  Khi đó $a+1$ bằng
	\choice
	{\True $48$}
	{$46$}
	{$47$}
	{$49$}
	\loigiai{
		Ta có	\[
			I=\displaystyle\int\limits_0^1 3^x \cdot 4^{2 x}\mathrm{\,d}x=\displaystyle\int\limits_0^1 3^x \cdot 16^x d x=\displaystyle\int\limits_0^1 48^x\mathrm{\,d} x=\dfrac{48^x}{\ln 48}\bigg|_0 ^1=\dfrac{47}{\ln 48}.
		\]
		Suy ra $a=47$ và $a+1=48$.
	}
\end{ex}
\Closesolutionfile{ans}

\TNTF
\Opensolutionfile{ans}[ans/ansDe4-TN2]
\begin{ex}%[2025-TLDH- Huỳnh Xuân Tín]%[2D4N1-4]
	Cho $I_1=\displaystyle\int\left(\mathrm e^x+\dfrac{1}{x^2}\right) \mathrm{d}x$ và $I_2=\displaystyle\int\left( \mathrm e^{2x-1}-\dfrac{1}{x^2}\right) \mathrm{d}x$.
	\choiceTF
	{\True  $I_1=\mathrm{e}^x-\dfrac{1}{x}+C$}
	{$I_2=\dfrac{\mathrm{e}^{2x-1}}{2}+\ln |x|+C$ }
	{\True $I_1+I_2=\mathrm{e}^x+\dfrac{{\mathrm{e}^{2x-1}}}{2}+C$ }
	{Gọi $F(x)$ là nguyên hàm của hàm số $f(x)$, với $f(x)=\mathrm{e}^x+\dfrac{1}{x^2}$. Nếu $F(1)=\mathrm{e}$ thì $F(\ln 2)=1-\dfrac{1}{\ln 2}$}
	\loigiai{
		\begin{itemchoice}
			\itemch \textbf{Đúng.}\\
			Vì $I_1=\displaystyle\int(\mathrm{e}^x+\dfrac{1}{x^2})\mathrm{d}x=\mathrm{e}^x-\dfrac{1}{x}+C$.
			\itemch \textbf{Sai.}\\
			Ta có $I_2=\displaystyle\int\left( \mathrm{e}^{2x-1}-\dfrac{1}{x^2}\right) \mathrm{d}x=\dfrac{\mathrm{e}^{2x-1}}{2}+\dfrac{1}{x}+C$.
			\itemch \textbf{Đúng.}\\
			Ta có
			\begin{eqnarray*}
				I_1+I_2	&= & f(x)=g(x)\displaystyle\int( \mathrm{e}^x+\dfrac{1}{x^2} )\mathrm{\,d}x+\displaystyle\int( {e^{2x-1}}-\dfrac{1}{x^2} )\mathrm{\,d}x\\
				&=& \displaystyle\int\left(\mathrm{e}^x+\mathrm{e}^{2x-1} \right) \mathrm{\,d}x\\
				&= &\mathrm{e}^x+\dfrac{\mathrm{e}^{2x-1}}{2}+C.
			\end{eqnarray*}
			\itemch \textbf{Sai.}\\
			Ta có $I_1=\displaystyle\int(\mathrm{e}^x+\dfrac{1}x^2)\mathrm{d}x=\mathrm{e}^x-\dfrac{1}{x}+C$. Vì $F(1)=\mathrm{e}\Rightarrow \mathrm{e}-1+C=\mathrm{e}\Rightarrow C=1$.\\
			$F(x)=\mathrm{e}^x-\dfrac{1}{x}+1\Rightarrow F(\ln 2)=\mathrm{e}^{\ln 2}-\dfrac{1}{\ln 2}+1=2-\dfrac{1}{\ln 2}+1=3-\dfrac{1}{\ln 2}$.
		\end{itemchoice}
	}
\end{ex}

\begin{ex}%[Dự án 2025 - đề cấu trúc mới, Nguyễn Kiều Nhã Tú]%[2D4H2-3]
	Cho hàm số $y=f(x)$. Biết $f'(x)=2\cos^2 x + 3$, $\forall x\in \mathbb{R}$.
	\choiceTF
	{$f'(x)>0$ với $\forall x\in\mathbb{R}$ nên $f(x)>0$,  $\forall x\in \mathbb{R}$}
	{$f'(x)=\displaystyle\int f(x)\mathrm{\,d}x$}
	{\True $f(x)=\dfrac{1}{2}\sin 2x+4x+C$}
	{\True Biết $f(0)=4$. Khi đó $\displaystyle\int\limits_0^{\frac{\pi}{4}}f(x) \mathrm{\,d}x$ bằng $\dfrac{\pi^2+8\pi+2}{8}$}
	\loigiai{
		\begin{itemchoice}
			\itemch \textbf{Sai}. Vì vì đạo hàm không có tính chất này.
			\itemch \textbf{Sai}. Vì $f(x)=\displaystyle\int f'(x)\mathrm{\,d}x$.
			\itemch \textbf{Đúng}. Vì
			\begin{align*}
				f(x) & =\displaystyle\int f'(x)\mathrm{\,d}x=\displaystyle\int\left(2\cos^2 x + 3\right)\mathrm{\,d}x \\
				     & =\displaystyle\int \left(2\cdot\dfrac{1+\cos 2x}{2}+3\right)  \mathrm{\,d}x
				= \displaystyle\int \left(\cos 2x+4\right)\mathrm{\,d}x                                               \\
				     & = \dfrac{1}{2} \sin{2x} + 4x + C.
			\end{align*}
			\itemch \textbf{Đúng}.
			Ta có $f(x)=\dfrac{1}{2}\sin 2x + 4x + C$.
			Do $f(0)=4 \Rightarrow C=4$ \\
			Vậy $f(x)=\dfrac{1}{2}\sin 2x + 4x + 4$ nên \\
			$\displaystyle\int\limits_0^{\frac{\pi}{4}}f(x)\mathrm{\,d}x = \displaystyle\int\limits_0^{\frac{\pi}{4}}\left(\dfrac{1}{2}\sin 2x+ 4x+4\right)\mathrm{\,d}x
				=\left(-\dfrac{1}{4}\cos 2x+2x^2+4x\right)\Big|_0^{\frac{\pi}{4}}
				= \dfrac{\pi^2+8\pi+2}{8}$.
		\end{itemchoice}
	}
\end{ex}
\Closesolutionfile{ans}

\TNSA
\Opensolutionfile{ans}[ans/ansDe4-TN3]
\begin{ex}%[2D4H1-1]%[Đào Trung Kiên]
	Giả sử $F(x)$ là một nguyên hàm của hàm số $f(x)=\mathrm{e}^x$, biết $F(0)=4$. Tìm $F(1)$ (làm tròn kết quả tới phần mười).
	\shortans[]{$5,7$}
	\loigiai{
		Do $F(x)$ là một nguyên hàm của $f(x)=\mathrm{e}^x$ nên $F(x)=\mathrm{e}^x+C$.\\
		Lại có $F(0)=4$ nên $C=3$ hay $F(x)=\mathrm{e}^x+3$ nên $F(1)=\mathrm{e}+3\approx 5{,}7$.
	}
\end{ex}

\begin{ex}%[2D4H2-2]%[Tổ 20 - Đợt 17 - Chương 4 - - CD - Đề 7]%[Lê Thị Thanh Tuyền]
	Có bao nhiêu giá trị nguyên của $a$ để $\displaystyle\int\limits_1^a(2x-3) \mathrm{\,d} x \leq 6$?
	\shortans{$6$}

	\loigiai{
		\begin{itemize}
			\item Ta có: $\displaystyle\int\limits_1^a(2x-3)\mathrm{\,d} x=\left.\left(x^2-3x\right)\right|_1 ^a=a^2-3a+2$.
			\item Khi đó: $\displaystyle\int\limits_1^a(2x-3) \mathrm{\,d} x \leq 6\Leftrightarrow a^2-3a+2\leq 6\Leftrightarrow-1\leq a \leq 4$
			\item	Mà $a$ là số nguyên nên $a \in\{-1; 0; 1; 2; 3; 4\}$.
			\item	Vậy có $6$ giá trị của $a$ thỏa đề bài.
		\end{itemize}


	}
\end{ex}

\begin{ex}%[2D4H3-1]
	Gọi $S$ là hình phẳng giới hạn bởi đồ thị hàm số $(H)\colon y=\dfrac{x-1}{x+1}$ và các trục tọa độ. Tính diện tích hình phẳng $(S)$ (làm tròn đến chữ số thứ hai sau dấu phẩy).
	\shortans{$0{,}39$}
	\loigiai{
		Điều kiện $x\ne -1$.\\
		Hoành độ giao điểm của đồ thị hàm số và trục $Ox$ là nghiệm của phương trình \[\dfrac{x-1}{x+1}=0\Leftrightarrow x=1.\]
		Vậy diện tích hình phẳng cần tìm là
		\allowdisplaybreaks
		\begin{eqnarray*}
			S&=&\displaystyle\int\limits_0^1 \left|\dfrac{x-1}{x+1}\right| \mathrm{\,d}x=\displaystyle\int\limits_0^1 \dfrac{1-x}{x+1} \mathrm{\,d}x\\
			&=&\displaystyle\int\limits_0^1 \left(-1+\dfrac{2}{x+1}\right) \mathrm{\,d}x\\
			&=&\left(-x+2\ln |x+1|\right)\Bigr\rvert_0^1=2\ln 2-1\approx 0{,}39.
		\end{eqnarray*}
	}
\end{ex}

\begin{ex}%[2D4H3-3]
	Tính thể tích của vật thể tròn xoay được tạo thành khi quay hình $(H)$ quanh $Ox$ với $(H)$ được giới hạn bởi đồ thị hàm số $y=\sqrt{4x-x^2}$ và trục hoành. (kết quả làm tròn đến hàng phần mười)
	\shortans{$33{,}5$}
	\loigiai{
	Điều kiện xác định: $4x-x^2\ge 0\Leftrightarrow 0\le x\le 4$.\\
	Phương trình hoành độ giao điểm của đồ thị hàm số $y=\sqrt{4x-x^2}$ và trục hoành là
	\[\sqrt{4x-x^2}=0\Leftrightarrow 4x-x^2=0\Leftrightarrow \hoac{
			&x=0 \\
			&x=4. \\
		}\]
	Thể tích của vật thể tròn xoay khi quay hình $(H)$ quanh $Ox$ là
	\[V=\pi \displaystyle \int\limits_0^4\left(\sqrt{4x-x^2}\right)^2\mathrm{\,d}x=\pi \displaystyle \int\limits_0^4{(4x-x^2)}\mathrm{\,d}x=\dfrac{32}{3}\pi.\]
	Vậy thể tích của vật thể tròn xoay khi quay hình $(H)$ quanh $Ox$ là $\dfrac{32}{3}\pi\approx33{,}5$.
	}
\end{ex}

\Closesolutionfile{ans}

\TL
\begin{ex}%[Mức độ 2]%[BG12, Nguyễn Kiều Nhã Tú]%[2D4H2-2]
	Cho hàm số $f(x)=\heva{&x^2\,\, \text{khi}\,\, 0\le x \le 1\\&2-x\,\, \text{khi} \,\,1< x \le 2}$. Tính $\displaystyle\int_0^2 f(x) \mathrm{\,d}x$.
	\loigiai{
		Ta có $\displaystyle\int_0^2 f(x) \mathrm{\,d}x=\displaystyle\int_0^1 x^2 \mathrm{\,d}x+\displaystyle\int_1^2 (2-x) \mathrm{\,d}x=\dfrac{x^3}{3}\bigg|_0^1+\left( 2x-\dfrac{x^2}{2} \right)\bigg|_1^2=\dfrac{5}{6}$.
	}
\end{ex}

\begin{ex}%[2D4V1-4]
	Cho hàm số $ f(x)$ nhận giá trị dương và thỏa mãn $ f(0)=1$, $\left(f'(x)\right)^3=\mathrm{\mathrm{e}}^ x{\left(f(x)\right)^2}$, $\forall x\in\mathbb{R}$. Tính $ f(3)$ (\textit{kết quả làm tròn đến hàng phần mười}).
	% \shortans{$20{,}1$}
	\loigiai{
	Ta có

	\begin{align*}
		\left(f'(x)\right)^3=\mathrm{e}^x{\left(f(x)\right)^2},\,\forall x\in\mathbb{R}
		 & \Leftrightarrow{f}'(x)=\sqrt[3]{\mathrm{e}^x}\cdot \sqrt[3]{\left(f(x)\right)^2}\Leftrightarrow\dfrac{f'(x)}{\sqrt[3]{\left(f(x)\right)^2}}=\sqrt[3]{\mathrm{e}^x}     \\
		 & \Leftrightarrow\dfrac{f'(x)}{\sqrt[3]{\left(f(x)\right)^2}}=\sqrt[3]{\mathrm{e}^x}\Leftrightarrow{f}'(x)\cdot \left(f(x)\right)^{-\tfrac{2}{3}}=\sqrt[3]{\mathrm{e}^x} \\&\Leftrightarrow 3\left[\left(f(x)\right)^{\tfrac{1}{3}}\right]'=\sqrt[3]{\mathrm{e}^x}\Leftrightarrow{\left[\left(f(x)\right)^{\tfrac{1}{3}}\right]'}=\dfrac{1}{3}\sqrt[3]{\mathrm{e}^x}\\&\Leftrightarrow{\left(f(x)\right)^{\tfrac{1}{3}}}=\dfrac{1}{3}\displaystyle\int{\sqrt[3]{\mathrm{e}^x}}\mathrm{\,d} x \Leftrightarrow{\left(f(x)\right)^{\tfrac{1}{3}}}=e^{\tfrac{x}{3}}+C.
	\end{align*}
	Vì	$f(0)=1$ nên $1=1+C\Rightarrow C=0\Rightarrow{\left(f(x)\right)^{\tfrac{1}{3}}}=e^{\tfrac{x}{3}}\Rightarrow f(x)=\mathrm{e}^x$.\\
	Vậy	$f(3)=e^3\approx 20{,}1$.
	}
\end{ex}

\begin{ex}%[2D4C3-5]
	Cho một mô hình $3-D$ mô phỏng một đường hầm như hình vẽ bên. Biết rằng đường hầm mô hình có chiều dài $5$ (cm); khi cắt hình này bởi mặt phẳng vuông góc với đáy của nó, ta được mặt cắt là một hình parabol có độ dài đáy gấp đôi chiều cao parabol. Chiều cao của mỗi mặt cắt hình parabol cho bởi công thức $ y=3-\dfrac{2}{5}x$ (cm), với $x$ (cm) là khoảng cách tính từ lối vào lớn hơn của đường hầm mô hình. Tính thể tích (theo đơn vị cm$^3$) không gian bên trong đường hầm mô hình (làm tròn kết quả đến hàng đơn vị).
	% \shortans{$29$}
	\begin{center}
	\begin{tikzpicture}[scale=1,declare function={a=0.8;b=0.6;c=0.4;d=0.2;}]
	\tikzset{
	homothety at/.style args={#1 scaled by #2}{shift={($(#1)!#2!(0,0)$)},scale=#2},
	}
	\def\mypath{(-120:2)..controls +(90:0.6) and +(-180:0.6)..(0,3)}
	\def\mydot{(0,3)..controls +(0:0.25) and +(95:0.05)..(60:2)}
	\draw \mypath;
	\draw[dashed] \mydot;
	\path (7,0) coordinate (c1);
	\begin{scope}[homothety at=c1 scaled by a]
	\draw \mypath;
	\draw[dashed] \mydot;
	\end{scope}
	\begin{scope}[homothety at=c1 scaled by b]
	\draw \mypath;
	\draw[dashed] \mydot;
	\end{scope}
	\begin{scope}[homothety at=c1 scaled by c]
	\draw \mypath;
	\draw[dashed] \mydot;
	\end{scope}
	\begin{scope}[homothety at=c1 scaled by d]
	\draw \mypath;
	\draw \mydot;
	\end{scope}
	\path
	(-120:2) coordinate (A)
	(0,3) coordinate (B)
	(60:2) coordinate (C);
	\foreach \x in {A,B,C}{\path ($(c1)!a!(\x)$) coordinate (\x_1);}
	\foreach \x in {A,B,C}{\path ($(c1)!b!(\x)$) coordinate (\x_2);}
	\foreach \x in {A,B,C}{\path ($(c1)!c!(\x)$) coordinate (\x_3);}
	\foreach \x in {A,B,C}{\path ($(c1)!d!(\x)$) coordinate (\x_4);}
	\path ($(A_4)!0.5!(B_4)$) coordinate (D);
	\draw (A)--(A_4) (B)--(B_4) (A_4)--(C_4)
	;
	\draw[dashed] (B)node[above]{$3$}--(0,0)--(D)node[below right]{$5$} (A)--(C) (A_1)--(C_1) (A_2)--(C_2) (A_3)--(C_3)  (C)--(C_4);
	\end{tikzpicture}
	\end{center}
	\loigiai{
	\begin{center}
	\begin{tikzpicture}[scale=1,font=\footnotesize]
	\path (0,0) coordinate (O)
	(2,0) coordinate (A)
	(0,2) coordinate (B)
	;
	\draw[-stealth] (-3.5,0)--(0,0)--(3,0)node[below]{$x$};
	\draw[-stealth] (0,-1.5)--(0,4)node[left]{$y$};
	\draw[smooth,samples=100] plot[domain=-2:2](\x,{(-1/2)*(\x)^2+2});
	\foreach \x in {O,A,B}{\draw[fill=blue!40] (\x) circle (1pt);}
	\foreach \x in {-3,-2,-1,1}{\draw (\x,0.05)--(\x,-0.05);}
	\foreach \x in {-1,1,3}{\draw (-0.05,\x)--(0.05,\x);}
	\node[above left] at (B) {$h$};
	\path (O)--(A)node[below]{$h$};
	\node at (0,0) [below left]{$O$};
	\end{tikzpicture}
	\end{center}
	Xét một mặt cắt hình parabol có chiều cao là $h$ và độ dài đáy $2h$ và chọn hệ trục $Oxy$ như hình vẽ trên.\\
	Parabol $(P)$ có phương trình $(P)\colon y=ax^2+h$, $(a<0)$.\\
	Có $B(h;0)\in(P)\Leftrightarrow 0=ah^2+h\Leftrightarrow a=-\dfrac{1}{h}$ (do $h>0$).\\
	Diện tích $S$ của mặt cắt là \[S=\displaystyle\int\limits_{-h}^h\left(-\dfrac{1}{h}{x^2}+h\right)\mathrm{\,d}x=\dfrac{4h^2}{3}, h=3-\dfrac{2}{5}x.\]
	$\Rightarrow S(x)=\dfrac{4}{3}{\left(3-\dfrac{2}{5}x\right)^2}.$\\
	Suy ra thể tích không gian bên trong của đường hầm mô hình
	\[ V=\displaystyle\int\limits_0^5S(x)\mathrm{\,d}x=\displaystyle\int\limits_0^5\dfrac{4}{3}\left(3-\dfrac{2}{5}x\right)^2\mathrm{\,d}x=\dfrac{260}{9}\approx 29\,\left(\text{cm}^3\right).\]
	}
	\end{ex}

% \Closesolutionfile{ansbook}
% \HetDe
% \label{De4}
% %
% \cleardoublepage
% \setcounter{page}{1}
% \rfoot{Trang \thepage/\pageref{DA4} - Đáp án trắc nghiệm Mã đề 4}
% \begin{center}
% 	\bfseries ĐÁP ÁN TRẮC NGHIỆM MÃ ĐỀ 4
% \end{center}

% \inputansbox{10}{ans/ansDe4-TN1}
% \inputansbox[3]{2}{ans/ansDe4-TN2}
% \inputansbox{3}{ans/ansDe4-TN3}
% \label{DA4}
%
