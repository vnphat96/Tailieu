\Opensolutionfile{ans}[ans/ans-2D4-3]
\section{Phương trình bậc hai trên tập số phức}
\subsection{Dạng toán và bài tập}
\begin{dang}{Căn bậc hai của số phức}
	Căn bậc hai của số phức $z=x+yi$ là một số phức $w=a+bi$ và tìm như sau
	\begin{eqnarray*}
		&&a+bi\Leftrightarrow x+yi=(a+bi)^2\Leftrightarrow (a^2-b^2)+(2ab)\cdot i=x+yi
		\Leftrightarrow \heva{& a^2-b^2=x\\& 2ab=y}
	\end{eqnarray*}
	Giải hệ này ta tìm được $a$, $b$. Từ đó tìm được căn bậc hai của số phức $z$.
\end{dang}

\begin{note}
	Ta có thể làm tương tự đối với các trường hợp căn bậc ba, bậc bốn.
\end{note}
\subsubsection{Ví dụ}

\begin{vd}%[Vương Quyền, dự án 12-EX-1-DCHT]%[2D4B4-1]
	Tìm căn bậc hai của số phức $z=-3+4i$.
	\dapso{$1+2i$ và $-1-2i$}
	\loigiai{
		Gọi $w=a+bi$ là căn bậc hai của số phức $z$ khi đó
		\begin{eqnarray*}
			&&-3+4i=(a+bi)^2\Leftrightarrow (a^2-b^2)+(2ab)\cdot i=-3+4i
			\\&\Leftrightarrow &\heva{& a^2-b^2=-3\\& 2ab=4}
			\Leftrightarrow \heva{& a^2-\dfrac{4}{a^2}+3=0\\& ab=2}
			\Leftrightarrow\heva{& a^4+3a^2-4=0\\& ab=2}
			\\&\Leftrightarrow & \heva{& \hoac{& a^2=1\\& a^2=-4}\\& ab=2}
			\Leftrightarrow\heva{& \hoac{& a=\pm 1\\& a=\pm 2i}\\& ab=2}
			\Leftrightarrow\hoac{& a=\pm 1\Rightarrow b=\pm 2\\& a=\pm 2i\Rightarrow b=\pm i.}
		\end{eqnarray*}
		Vậy có hai căn bậc hai của số phức $z$ là $w_1=1+2i$ và $w_2=-1-2i$. 
	}
\end{vd}

\begin{note}
	Tìm căn bậc, căn bậc bốn của số phức $z=a+bi$ bằng máy tính bỏ túi: Để máy tính ở chế độ ra-đian \fbox{SHIFT} - \fbox{MODE} - \fbox{ $4$} và để chế độ số phức \fbox{SHIFT} - \fbox{MODE} - \fbox{$2$}.
	\begin{itemize}
		\item Tìm căn bậc hai $\sqrt{|a+bi|}\angle\left(\dfrac{\arg(a+bi)}{2}+\dfrac{2\pi X}{2}\right)$ \fbox{CALC} $\heva{& X=0\Rightarrow w_1=\ldots\\& X=1\Rightarrow w_2=\ldots}$
		\item Tìm căn bậc bốn $\sqrt[4]{|a+bi|}\angle\left(\dfrac{\arg(a+bi)}{4}+\dfrac{2\pi X}{4}\right)$ \fbox{CALC} 
		$\heva{& X=0\Rightarrow w_1=\ldots\\& X=1\Rightarrow w_2=\ldots\\& X=2\Rightarrow w_3=\ldots\\& X=3\Rightarrow w_4=\ldots}$\\
		Trong đó $|\quad |$ bằng \fbox{SHIFT} - \fbox{HYP}; $\angle$ bằng \fbox{SHIFT} - \fbox{$(-)$}; $\arg(\quad)$ bằng \fbox{SHIFT} - \fbox{$2$} - \fbox{$1$}.
	\end{itemize}
\end{note}

\subsubsection{Bài tập áp dụng}

\begin{bt}%[Vương Quyền, dự án 12-EX-1-DCHT]%[2D4B4-1]
	Tìm căn bậc hai của số phức sau
	\begin{listEX}[2]
		\item $z=-5+12i$. \dapso{$w=\pm 2\pm 3i$}
		\item $z=8+6i$. \dapso{$w=\pm 3\pm i$}
		\item $z=3-4i$. \dapso{$w=\pm 2\mp i$}
		\item $z=33-56i$.\dapso{$w=\pm 7\mp 4i$}
		\item $z=4+6\sqrt{5}i$.\dapso{$w=\pm 3\pm i\sqrt{5}$}
		\item $z=-1-2\sqrt{6}i$.\dapso{$w=\pm\sqrt{2}\mp i\sqrt{3}$}
	\end{listEX}
	\loigiai{
		\begin{enumerate}
			\item Gọi $w=a+bi$ là căn bậc hai của số phức $z$ khi đó
			\begin{eqnarray*}
				&&  -5+12i=(a+bi)^2\Leftrightarrow (a^2-b^2)+(2ab)\cdot i=-5+12i
				\\&\Leftrightarrow &\heva{& a^2-b^2=-5\\& 2ab=12}
				\Leftrightarrow \heva{& a^2-\dfrac{36}{a^2}+5=0\\& ab=6}
				\Leftrightarrow\heva{& a^4+5a^2-36=0\\& ab=6}
				\\&\Leftrightarrow & \heva{& \hoac{& a^2=4\\& a^2=-9}\\& ab=6}
				\Leftrightarrow\heva{& \hoac{& a=\pm 2\\& a=\pm 3i}\\& ab=6}
				\Leftrightarrow\hoac{& a=\pm 2\Rightarrow b=\pm 3\\& a=\pm 3i\Rightarrow b=\mp 2i.}
			\end{eqnarray*}
			Vậy có hai căn bậc hai của số phức $z$ là $w_1=2+3i$ và $w_2=-2-3i$. 
			\item Gọi $w=a+bi$ là căn bậc hai của số phức $z$ khi đó
			\begin{eqnarray*}
				&&  8+6i=(a+bi)^2\Leftrightarrow (a^2-b^2)+(2ab)\cdot i=8+6i
				\\&\Leftrightarrow &\heva{& a^2-b^2=8\\& 2ab=6}
				\Leftrightarrow \heva{& a^2-\dfrac{9}{a^2}-8=0\\& ab=3}
				\Leftrightarrow\heva{& a^4-8a^2-9=0\\& ab=3}
				\\&\Leftrightarrow & \heva{& \hoac{& a^2=9\\& a^2=-1}\\& ab=3}
				\Leftrightarrow\heva{& \hoac{& a=\pm 3\\& a=\pm i}\\& ab=3}
				\Leftrightarrow\hoac{& a=\pm 2\Rightarrow b=\pm 1\\& a=\pm i\Rightarrow b=\mp 3i.}
			\end{eqnarray*}
			Vậy có hai căn bậc hai của số phức $z$ là $w_1=3+i$ và $w_2=-3-i$. 
			\item Gọi $w=a+bi$ là căn bậc hai của số phức $z$ khi đó
			\begin{eqnarray*}
				&&  3-4i=(a+bi)^2\Leftrightarrow (a^2-b^2)+(2ab)\cdot i=3-4i
				\\&\Leftrightarrow &\heva{& a^2-b^2=3\\& 2ab=-4}
				\Leftrightarrow \heva{& a^2-\dfrac{4}{a^2}-3=0\\& ab=-2}
				\Leftrightarrow\heva{& a^4-3a^2-4=0\\& ab=-2}
				\\&\Leftrightarrow & \heva{& \hoac{& a^2=-1\\& a^2=-4}\\& ab=-2}
				\Leftrightarrow\heva{& \hoac{& a=\pm i\\& a=\pm 2i}\\& ab=-2}
				\Leftrightarrow\hoac{& a=\pm i\Rightarrow b=\pm 2i\\& a=\pm 2i\Rightarrow b=\pm 2i.}
			\end{eqnarray*}
			Vậy có hai căn bậc hai của số phức $z$ là $w_1=-2+i$ và $w_2=2-i$. 
			\item Gọi $w=a+bi$ là căn bậc hai của số phức $z$ khi đó
			\begin{eqnarray*}
				&&  33-56i=(a+bi)^2\Leftrightarrow (a^2-b^2)+(2ab)\cdot i=33-56i
				\\&\Leftrightarrow &\heva{& a^2-b^2=33\\& 2ab=-56}
				\Leftrightarrow \heva{& a^2-\dfrac{784}{a^2}-33=0\\& ab=-28}
				\Leftrightarrow\heva{& a^4-33a^2-784=0\\& ab=-28}
				\\&\Leftrightarrow & \heva{& \hoac{& a^2=-16\\& a^2=49}\\& ab=-28}
				\Leftrightarrow\heva{& \hoac{& a=\pm 4i\\& a=\pm 7}\\& ab=-28}
				\Leftrightarrow\hoac{& a=\pm 4i\Rightarrow b=\pm 7i\\& a=\pm 7\Rightarrow b=\mp 4.}
			\end{eqnarray*}
			Vậy có hai căn bậc hai của số phức $z$ là $w_1=7-4i$ và $w_2=-7+4i$.
			\item Gọi $w=a+bi$ là căn bậc hai của số phức $z$ khi đó
			\begin{eqnarray*}
				&&  4+6\sqrt{5}i=(a+bi)^2\Leftrightarrow (a^2-b^2)+(2ab)\cdot i=4+6\sqrt{5}i
				\\&\Leftrightarrow &\heva{& a^2-b^2=4\\& 2ab=6\sqrt{5}}
				\Leftrightarrow \heva{& a^2-\dfrac{45}{a^2}-4=0\\& ab=3\sqrt{5}}
				\Leftrightarrow\heva{& a^4-4a^2-45=0\\& ab=3\sqrt{5}}
				\\&\Leftrightarrow & \heva{& \hoac{& a^2=9\\& a^2=-5}\\& ab=3\sqrt{5}}
				\Leftrightarrow\heva{& \hoac{& a=\pm 3\\& a=\pm i\sqrt{5}}\\& ab=3\sqrt{5}}
				\Leftrightarrow\hoac{& a=\pm 3\Rightarrow b=\pm \sqrt{5}\\& a=\pm \sqrt{5}i\Rightarrow b=\mp 3i.}
			\end{eqnarray*}
			Vậy có hai căn bậc hai của số phức $z$ là $w_1=3+\sqrt{5}i$ và $w_2=-3-\sqrt{5}i$.  
			\item Gọi $w=a+bi$ là căn bậc hai của số phức $z$ khi đó
			\begin{eqnarray*}
				&&  -1-2\sqrt{6}i=(a+bi)^2\Leftrightarrow (a^2-b^2)+(2ab)\cdot i=-1-2\sqrt{6}i
				\\&\Leftrightarrow &\heva{& a^2-b^2=-1\\& 2ab=-2\sqrt{6}}
				\Leftrightarrow \heva{& a^2-\dfrac{6}{a^2}+1=0\\& ab=-\sqrt{6}}
				\Leftrightarrow\heva{& a^4+a^2-6=0\\& ab=-\sqrt{6}}
				\\&\Leftrightarrow & \heva{& \hoac{& a^2=2\\& a^2=-3}\\& ab=-\sqrt{6}}
				\Leftrightarrow\heva{& \hoac{& a=\pm \sqrt{2}\\& a=\pm i\sqrt{3}}\\& ab=-\sqrt{6}}
				\Leftrightarrow\hoac{& a=\pm \sqrt{2}\Rightarrow b=\mp \sqrt{3}\\& a=\pm i\sqrt{3}\Rightarrow b=\pm i\sqrt{2}.}
			\end{eqnarray*}
			Vậy có hai căn bậc hai của số phức $z$ là $w_1=\sqrt{2}-i\sqrt{3}$ và $w_2=-\sqrt{2}+i\sqrt{3}$. 
		\end{enumerate}
	}
\end{bt}

\begin{dang}{Phương trình bậc hai với hệ số thực}
	Xét phương trình bậc hai $az^2+bz+c=0\quad (1)$ với $a\neq 0$ có biệt thức $\Delta=b^2-4ac$.
	\begin{itemize}
		\item Nếu $\Delta=0$ thì phương trình $(1)$ có nghiệm kép $z_1=z_2=-\dfrac{b}{2a}$.
		\item Nếu $\Delta\neq 0$ và gọi $\delta$ là căn bậc hai của $\Delta$ thì phương trình $(1)$ có hai nghiệm phân biệt là 
		$z_1=\dfrac{-b+\delta}{2a}$ và $z_2=\dfrac{-b-\delta}{2a}$.
	\end{itemize}
\end{dang}
\subsubsection{Ví dụ}

\begin{vd}%[Vương Quyền, dự án 12-EX-1-DCHT]%[2D4B4-1]
	Biết $z_1$, $z_2$ là hai nghiệm phức của phương trình $z^2-2z+4=0$. Tính $|z_1|+|z_2|$.
	\dapso{$z_1=\dfrac{2+i\sqrt{12}}{2}$ và $z_2=\dfrac{2-i\sqrt{12}}{2}$.}
	\loigiai{
		Ta có $\Delta=b^2-4ac=-12$. Căn bậc hai của $\Delta$ là $\pm i\sqrt{12}$.\\
		Suy ra phương trình có hai nghiệm phân biệt $z_1=\dfrac{2+i\sqrt{12}}{2}$ và $z_2=\dfrac{2-i\sqrt{12}}{2}$.
	}
\end{vd}

\subsubsection{Bài tập áp dụng}
\begin{ex}%[Số phức - Tư duy mở]%[Hồng Trường Sơn]%[2D4Y4-1]
    Có bao nhiêu số phức $z$ thỏa mãn phương trình $z^2-4z+3=0$?
    \choice
    {$1$}
    {$3$}
    {\True $2$}
    {$0$}
    \loigiai{
    Ta có $z^2-4z+3=0 \Leftrightarrow \hoac{&z=1\\ &z=3.}$\\
    Vậy phương trình đã cho có $2$ nghiệm phức.
    }
\end{ex} 

\begin{ex}%[Số phức - Tư duy mở]%[Hồng Trường Sơn]%[2D4Y4-1]
    Có bao nhiêu số phức $z$ thuần ảo thỏa mãn phương trình $z^2-3z-6=0$?
    \choice
    {$1$}
    {$2$}
    {$4$}
    {\True $0$}
    \loigiai{
    Ta có $\triangle =15$ nên phương trình có $2$ nghiệm thực $z=\dfrac{3\pm \sqrt{15}}{2}$.\\
    Vậy phương trình không có nghiệm thuần ảo.
    }
\end{ex} 

\begin{ex}%[Số phức - Tư duy mở]%[Hồng Trường Sơn]%[2D4Y4-1]
    Căn bậc hai của số phức $z=4$ là hai số phức
    \choice
    {\True $\pm 2$}
    {$\pm 1$}
    {$\pm 4$}
    {$\pm 16$}
    \loigiai{
    Căn bậc hai của số phức $z=4$ là $\pm 2$.
    }
\end{ex}

\begin{ex}%[Số phức - Tư duy mở]%[Hồng Trường Sơn]%[2D4Y4-1]
    Số phức nào dưới đây có đúng một căn bậc hai?
    \choice
    {$z=1$}
    {$z=i$}
    {$z=1+i$}
    {\True $z=0$}
    \loigiai{
    Số $0$ có đúng một căn bậc hai.
    }
\end{ex}

\begin{ex}%[Số phức - Tư duy mở]%[Hồng Trường Sơn]%[2D4B4-1]
    Căn bậc hai của $u=-16$ là hai số phức
    \choice
    {$\pm 4$}
    {\True $\pm 4i$}
    {$\pm 2$}
    {$\pm 8i$}
    \loigiai{
    Căn bậc hai của $u=-16$ là $\pm 4i$.
    }
\end{ex}

\begin{ex}%[Số phức - Tư duy mở]%[Hồng Trường Sơn]%[2D4B4-1]
    Gọi $z$ là một căn bậc hai của số phức $u=18i$ có phần thực dương. Phần ảo của $z$ là
    \choice
    {$3$}
    {$-3$}
    {$3\sqrt2$}
    {$2\sqrt3$}
    \loigiai{
    Ta có $z^2=u=18i=9\cdot(2i)=9\cdot(1+i)^2=(3+3i)^2 \Rightarrow \hoac{&z=-3-3i\quad \text{(loại)}\\ &z=3+3i \quad \text{(nhận)}.}$\\
    Vậy phần ảo của $z$ là $3$.
    }
\end{ex}

\begin{ex}%[Số phức - Tư duy mở]%[Hồng Trường Sơn]%[2D4B4-1]
    Gọi $z$ là một căn bậc hai của số phức $u=-50i$ có phần ảo dương. Phần thực của $z$ là
    \choice
    {$5$}
    {$5\sqrt2$}
    {\True $-5$}
    {$-5\sqrt2$}
    \loigiai{
    Ta có $z^2=u=-50i=25\cdot(-2i)=25\cdot(1-i)^2=(5-5i)^2 \Rightarrow \hoac{&z=5-5i\quad \text{(loại)}\\ &z=-5+5i \quad \text{(nhận)}.}$\\
    Vậy phần ảo của $z$ là $-5$.
    }
\end{ex}

\begin{ex}%[Số phức - Tư duy mở]%[Hồng Trường Sơn]%[2D4K4-1]
    Căn bậc hai của số phức $z=3-4i$ là hai số phức
    \choice
    {$\pm \left(1+2i\right)$}
    {$\pm \left(1-i\sqrt3\right)$}
    {$\pm \left(\sqrt3-2i\right)$}
    {\True $\pm \left(2-i\right)$}
    \loigiai{
    \textbf{Cách 1:} Tự luận.\\
    Gọi căn bậc hai của số phức $z=3-4i$ là $u=a+bi$ với $a,\,b \in \mathbb{R}$.\\
    Suy ra $u^2=(a+bi)^2=a^2-b^2+2abi=3-4i \Rightarrow \heva{&a^2-b^2=3 &\quad (1)\\ &2ab=-4 &\quad (2).}$\\
    Từ $(2) \Rightarrow b=-\dfrac{2}{a}$ thay vào $(1)$, ta được 
    $$a^2- \left(-\dfrac{2}{a}\right)^2=3 \Leftrightarrow a^4-3a^2-4=0 \Leftrightarrow \hoac{&a^2=-1 \quad \text{(loại)}\\ &a^2=4 \quad \text{(nhận)}.}$$
    Suy ra $\hoac{&a=2 \Rightarrow b=-\dfrac{2}{a}=-1\\ &a=-2 \Rightarrow b=-\dfrac{2}{a}=1} \Rightarrow u=a+bi=\pm \left(2-i\right)$.\\
    Vậy căn bậc hai của $z$ là $\pm \left(2-i\right)$.\\
    \textbf{Cách 2:} Trắc nghiệm.\\
    Dạng lượng giác của $z=|z|\left(\cos \alpha +i\sin \alpha \right) \Rightarrow$ căn bậc hai của $z$ là $\pm \sqrt{|z|}\left(\cos \dfrac{\alpha}{2}+i\sin \dfrac{\alpha}{2} \right)$.\\
    Áp dụng $z=3-4i=5 \left(\cos \alpha +i\sin \alpha\right)$ trong đó $\heva{&\cos \alpha = \dfrac{3}{5} \Rightarrow \alpha \approx 0.927... \Rightarrow \boxed{Sto}-\boxed{A}\\ &\sin \alpha = -\dfrac{4}{5}<0 \Rightarrow \alpha <0.}$\\
    Căn bậc hai của số phức $z=3-4i$ là 
    $$\pm \sqrt5 \left(\cos \dfrac{\alpha}{2} +i\sin \dfrac{\alpha}{2}\right) = \pm \left(\sqrt5 \cos \dfrac{A}{2}+i\sqrt5 \sin \dfrac{A}{2}\right) = \pm \left(2-i\right).$$
    Vậy căn bậc hai của $z$ là $\pm \left(2-i\right)$.
    }
\end{ex}

\begin{ex}%[Số phức - Tư duy mở]%[Hồng Trường Sơn]%[2D4K4-1]
    Căn bậc hai của số phức $z=-24+10i$ là hai số phức
    \choice
    {\True $\pm \left(1+5i\right)$}
    {$\pm \left(-1+5i\right)$}
    {$\pm \left(5-i\right)$}
    {$\pm \left(5+i\right)$}
    \loigiai{
    Gọi căn bậc hai của số phức $z=-24+10i$ là $u=a+bi$ với $a,\,b \in \mathbb{R}$.\\
    Suy ra $u^2=(a+bi)^2=a^2-b^2+2abi=-24+10i \Rightarrow \heva{&a^2-b^2=-24 &\quad (1)\\ &2ab=10 &\quad (2).}$\\
    Từ $(2) \Rightarrow b=\dfrac{5}{a}$ thay vào $(1)$, ta được 
    $$a^2- \left(\dfrac{5}{a}\right)^2=-24 \Leftrightarrow a^4+24a^2-25=0 \Leftrightarrow \hoac{&a^2=-25 \quad \text{(loại)}\\ &a^2=1 \quad \text{(nhận)}.}$$
    Suy ra $\hoac{&a=1 \Rightarrow b=\dfrac{5}{a}=5\\ &a=-1 \Rightarrow b=\dfrac{5}{a}=-5} \Rightarrow u=a+bi=\pm \left(1+5i\right)$.\\
    Vậy căn bậc hai của $z$ là $\pm \left(1+5i\right)$.
    }
\end{ex}

\begin{ex}%[Số phức - Tư duy mở]%[Hồng Trường Sơn]%[2D4Y4-1]
    Có bao nhiêu số phức $z$ thỏa mãn phương trình $z^2+z+2=0$?
    \choice
    {$1$}
    {\True $2$}
    {$0$}
    {$3$}
    \loigiai{
    Ta có phương trình $z^2+z+2=0 \Leftrightarrow \hoac{&z=-\dfrac{1}{2}+\dfrac{7}{2}i\\ &z=-\dfrac{1}{2}-\dfrac{7}{2}i.}$\\
    Vậy phương trình đã cho có $2$ nghiệm phức.
    }
\end{ex}

\begin{ex}%[Số phức - Tư duy mở]%[Hồng Trường Sơn]%[2D4Y4-1]
    Có bao nhiêu số phức thỏa mãn phương trình $z^3-6z-3=0$?
    \choice
    {$2$}
    {\True $3$}
    {Vô số}
    {$1$}
    \loigiai{
    Phương trình bậc $n$ có $n$ nghiệm phức.
    }
\end{ex}

\begin{ex}%[Số phức - Tư duy mở]%[Hồng Trường Sơn]%[2D4Y4-3]
    Có bao nhiêu số phức $z$ có phần thực dương thỏa mãn phương trình $z^4-3z^2-4=0$?
    \choice
    {\True $1$}
    {$2$}
    {$0$}
    {$3$}
    \loigiai{
    Ta có $z^4-3z^2-4=0 \Leftrightarrow \hoac{&z=-1\\ &z=4} \Leftrightarrow \hoac{&z=\pm i\\ &z=\pm 2.}$\\
    Vậy có $1$ số phức $z$ thỏa mãn yêu cầu bài toán.
    }
\end{ex}

\begin{ex}%[Số phức - Tư duy mở]%[Hồng Trường Sơn]%[2D4Y4-3]
    Có bao nhiêu số phức $z$ không thuần thực thỏa mãn phương trình $z^4-2z^2-3=0$?
    \choice
    {\True $2$}
    {$3$}
    {$0$}
    {$4$}
    \loigiai{
    Ta có $z^4-2z^2-3=0 \Leftrightarrow \hoac{&z=-1\\ &z=3} \Leftrightarrow \hoac{&z=\pm i\\ &z=\pm \sqrt3.}$\\
    Vậy có $2$ số phức $z$ thỏa mãn yêu cầu bài toán.
    }
\end{ex}

\begin{ex}%[Số phức - Tư duy mở]%[Hồng Trường Sơn]%[2D4B4-1]
    Cho hai số phức $z_1;z_2$ là nghiệm của phương trình $z^2-2z+3=0$. Gọi $A$ và $B$ là hai điểm biểu diễn số phức $z_1$ và $z_2$. Khoảng cách $AB$ bằng
    \choice
    {$2\sqrt3$}
    {$4\sqrt2$}
    {\True $2\sqrt2$}
    {$4\sqrt5$}
    \loigiai{
    Ta có $z^2-2z+3=0 \Leftrightarrow \hoac{&z_1=1+i\sqrt2 \Rightarrow A\left(1;\sqrt2\right)\\ &z_2=1-i\sqrt2 \Rightarrow B\left(1;-\sqrt2\right).}$\\
    Vậy $AB=2\sqrt2$.
    }
\end{ex}

\begin{ex}%[Số phức - Tư duy mở]%[Hồng Trường Sơn]%[2D4Y4-3]
    Cho bốn điểm $A,\,B,\,C,\,D$ biểu diễn bốn nghiệm của phương trình $z^4-3z-4=0$. Tứ giác $ABCD$ có diện tích bằng
    \choice
    {$4$}
    {$2$}
    {$8$}
    {$6$}
    \loigiai{
    Ta có $z^4-3z^2-4=0 \Leftrightarrow \hoac{&z=-1\\ &z=4} \Leftrightarrow \hoac{&z=\pm i\\ &z=\pm 2.}$\\
    Suy ra $A\left(-2;0\right),\,B\left(0;-1\right),\,C\left(2;0\right),\,D\left(0;1\right) \Rightarrow S_{ABCD} = \dfrac{AC\cdot BD}{2}= \dfrac{4\cdot 2}{2}=4$.
    }
 %%==========Câu 1
\begin{ex}[2D4K3-3]
 Cho ba điểm $A, B, C$ lần lượt biểu diễn ba nghiệm của phương trình phức $z^{3}-1=0$. Tam giác $A B C$ có diện tích bằng
 \choice
 {$\dfrac 34$}
 {\True$\dfrac{3\sqrt 3}4$}
 {$4$}
 {$1$}
 \loigiai{
	 Ta có $z^3-1=0\Leftrightarrow(z-1)\left(z^2+z+1\right)=0\Leftrightarrow\hoac{& z-1=0 \\ & z^2+z+1=0}\Leftrightarrow \hoac{& z=1 \\ & z=\dfrac{-1\pm i\sqrt 3}2.}$\\
	 Tọa độ các điểm biểu diễn ba nghiệm tương ứng là $A(1 ; 0), B\left(-\dfrac{1}{2} ; \dfrac{\sqrt{3}}{2}\right), C\left(-\dfrac{1}{2} ;-\dfrac{\sqrt{3}}{2}\right)$.
	\begin{center}
	\begin{tikzpicture}[>=stealth,line join=round,line cap=round,font=\footnotesize,scale=1]
	\draw[->] (0,-2.5)--(0,2.5)node[right]{$y$};
	\draw[->] (-1,0)--(2,0)node[below]{$x$};
	\pgfmathsetmacro{\a}{1*sqrt(3)}
	\path
	(1,0) coordinate (A)
	(-0.5,\a) coordinate (B)
	(-0.5,-\a) coordinate (C)
	(0,0)node[below right]{$O$}
	(1,0)node[below]{$1$}
	(0,\a)node[right]{$\sqrt{3}$}
	(0,-\a) node[right]{$-\sqrt{3}$}
	(-.5,0)node[below left]{$-\dfrac{1}{2}$}
	;
	\draw[dashed] (0,\a)-|(-0.5,0)|-(0,-\a);
	\foreach \d/\g in{A/90,B/90,C/-90} \fill[black] (\d) circle (1.2pt) node at ($(\d)+(\g:3mm)$){$\d$};
	\foreach \x/\y in{0/\a,0/-\a,-0.5/0,0/0}\fill (\x,\y) circle (1.2pt);
	\draw(A)--(B)--(C)--cycle;
	\end{tikzpicture}
	\end{center}
	 Vẽ hình biểu diễn và nhận thấy tam giác $A B C$ cân tại $A$ và đường cao $A H=\dfrac{3}{2}$, cạnh $B C=\sqrt{3}$.
	 Suy ra diện tích tam giác $A B C$ là $S_{\triangle A B C}=\dfrac{1}{2} B C \cdot A H=\dfrac{1}{2} \cdot \sqrt{3} \cdot \dfrac{3}{2}=\dfrac{3 \sqrt{3}}{4}$.
}
\end{ex}
%%==========Câu 2
\begin{ex}[2D4K4-3]
Cho ba điểm $A, B, C, D$ lần lượt biểu diễn ba nghiệm của phương trình phức $z^4-16=0$. Tứ giác $ABCD$ có diện tích bằng
\choice
{$2\sqrt{2}$}
{$16$}
{\True $8$}
{$4$}
\loigiai{
	 Ta có: $z^{4}-16=0\Leftrightarrow \hoac{&z^2=4\\ & z^2=-4}\Leftrightarrow \hoac{&z=\pm 2\\ & z=-2i}$.
	 Tọa độ các điểm biểu diễn bốn nghiệm là $A(2 ; 0), B(-2 ; 0), C(0 ; 2 i), D(0 ;-2 i)$.
	\begin{center}
\begin{tikzpicture}[>=stealth,line join=round,line cap=round,font=\footnotesize,scale=1]
	\draw[->] (0,-3)--(0,3)node[right]{$y$};
\draw[->] (-3,0)--(3,0)node[below]{$x$};
\path
(2,0) coordinate (A)
(0,2) coordinate (C)
(-2,0) coordinate (B)
(0,-2) coordinate (D)
(0,0) coordinate (O)
;
\draw[dashed] (A)--(C)--(B)--(D)--cycle;
\foreach \d/\g in{A/60,B/120,C/60,D/-60,O/60} \fill[black] (\d) circle (1.2pt) node at ($(\d)+(\g:3mm)$){$\d$};
\end{tikzpicture}	
	\end{center}
	 Dễ thấy tứ giác $A B C D$ là hình vuông cạnh bằng $2 \sqrt{2}$.\\
	Suy ra diện tích $A B C D=(2 \sqrt{2})^{2}=8$.
}
\end{ex}
%%==========Câu 3
\begin{ex}[2D4B4-3]
Cho ba điểm $A, B, C, D$ lần lượt biểu diễn ba nghiệm của phương trình phức $z^4+4 i=0$. Tứ giác $ABCD$ có diện tích bằng
\choice
{$8$}
{$24$}
{\True$4$}
{$1$}
\loigiai{
 Ta có $z^4+4=0\Leftrightarrow z^4=-4=(2i)^2\Leftrightarrow \hoac{&z^2=2i=(1+i)^2\\ & z^2=-2i=(1-i)^2}\Leftrightarrow \hoac{&z=\pm(1+i)\\ & z=\pm(1-i)}$.\\
 Tọa độ các điểm biểu diễn bốn nghiệm là $A(1 ; 1), B(-1 ; 1), C(-1 ;-1), D(1 ;-1)$.
	\begin{center}
	\begin{tikzpicture}[>=stealth,line join=round,line cap=round,font=\footnotesize,scale=1]	
	\draw[->] (0,-2)--(0,2)node[right]{$y$};
	\draw[->] (-2,0)--(2,0)node[below]{$x$};
\path
(1,1) coordinate (A)
(-1,-1) coordinate (C)
(-1,1) coordinate (B)
(1,-1) coordinate (D)
(0,0) coordinate (O)
;
\draw[dashed] (A)--(B)--(C)--(D)--cycle;
\foreach \d/\g in{A/60,B/120,C/-100,D/-60,O/60} \fill[black] (\d) circle (1.2pt) node at ($(\d)+(\g:3mm)$){$\d$};
\end{tikzpicture}
	\end{center}
 Dễ thấy tứ giác $A B C D$ là hình vuông cạnh bằng $2$.\\
Suy ra diện tích $A B C D=(2)^{2}=4$. 
}
\end{ex}
%%==========Câu 4
\begin{ex}[2D4K4-1]
Cho phương trình phức $z^2+a z+4=0$, với $a, b$ là những số thực. Biết phương trình có hai nghiệm phức không thuần thực là $z_1$ và $z_2$. Khi đó giá trị của biểu thức $T=\left|z_1\right|+3\left|z_2\right|$ tương ứng bằng
\choice
{$4$}
{$3$}
{$7$}
{\True$8$}
\loigiai{
Ta nhận thấy phương trình bậc hai hệ số thực nếu có nghiệm phức không thuần thực thì các nghiệm sẽ là liên hợp của nhau.\\
Suy ra $z_2=\overline{z}_1\longrightarrow z_1\cdot z_2=z_1\cdot\overline{z}_1=\left|z_1\right|^2=4\Leftrightarrow\left|z_1\right|=\left|z_2\right|=2$.\\
Suy ra $T=\left|z_{1}\right|+3\left|z_{2}\right|=2+3\cdot 2=8$.
}
\end{ex}
%%==========Câu 5
\begin{ex}[2D4K4-1]
Cho phương trình phức $z^2+4 z+b=0$, với $a, b$ là những số thực. Biết phương trình có hai nghiệm phức không thuần thực là $z_1$ và $z_2.$ Khi đó giá trị của $T=\left|z_1\right|+\left|z_2\right|$ có thể nhận giá trị nào dưới đây?
\choice
{$2$}
{$3$}
{\True$5$}
{$4$}
\loigiai{
	 Điều kiện đề có nghiệm phức không thuần thực: $\Delta^{\prime}=4-b<0 \Leftrightarrow b>4$.\\
	Khi đó phương trình có hai nghiệm phức là liên hợp của nhau
	\[z_2=\overline{z}_1\longrightarrow z_1\cdot z_2=z_1\cdot\overline{z}_1=\left|z_1\right|^2=b\Leftrightarrow\left|z_1\right|=\left|z_2\right|=\sqrt b.\]
	 Suy ra $T=\left|z_{1}\right|+\left|z_{2}\right|=2 \sqrt{b}>2 \cdot \sqrt{4}=4$.
}
\end{ex}
%%==========Câu 6
\begin{ex}[2D4K4-1]
Cho phương trình phức $z^2+2 a z+a^2-2 a=0$, với $a$ là số thực. Biết phương trình có hai nghiệm phức không thuần thực có mô đun bằng $2.$ Gọi $S$ là tập chứa tất cả các giá trị của $a$ thỏa mãn bài toán. Tổng tất cả các phần tử của $S$ bằng
\choice
{\True$1-\sqrt{5}$}
{$-4$}
{$2$}
{$3$}
\loigiai{
	 Vì phương trình phức dạng đa thức hệ số thực nên có hai nghiệm là liên hợp của nhau, có mô đun bằng nhau. Suy ra \[\left|z_{1}\right|=\left|z_{2}\right|=\sqrt{a^{2}-2 a}=2 \Leftrightarrow a^{2}-2 a-4=0 \Leftrightarrow a=1 \pm \sqrt{5}.\eqno (1)\]
	 Điều kiện để phương trình có hai nghiệm phức phân biệt là\[\Delta^{\prime}=a^{2}-a^{2}+2 a=2 a<0.\eqno (2)\]
	 Từ (1) và (2), suy ra giá trị của $a$ thỏa mãn là $a=1-\sqrt{5}$. Suy ra tập $S=\{1-\sqrt{5}\}$.
	 Suy ra tồng tất cả các phần tử của tập S là $1-\sqrt{5}$.
}
\end{ex}
%%==========Câu 7
\begin{ex}[2D4K4-1]
Cho phương trình phức $z^2-m z+n=0$, với $m, n$ là các số thực. Biết phương trình có hai nghiệm phức không thuần thực là $z_1=u+3 i$ và $z_2=2 u+2 i-2$. Giá trị của $\left|z_1\right|$ bằng:
\choice
{$2\sqrt{13}$}
{$\sqrt{10}$}
{$2\sqrt{3}$}
{\True$\dfrac{2\sqrt{13}}{3}$}
\loigiai{
	 Vì phương trình phức dạng đa thức hệ số thực nên có hai nghiệm là liên hợp của nhau, ta có\[z_1=\overline{z}_2\Leftrightarrow u+3i=\overline{2u+2i-2}=2\bar u-2i-2.\]
	 Gọi $u=a+i b$, với $a, b \in \mathbb{R} \Rightarrow a+i b+3 i=2 a-2 i b-2 i-2\Leftrightarrow \heva{& a=2 \\ & b=-\dfrac{5}{3}}\Rightarrow u=2-\dfrac 53i.$
	 $z_1=u+3i=2+\dfrac 43i\Rightarrow\left|z_1\right|=\left|2+\dfrac 43i\right|=\dfrac{2\sqrt{13}}3$.
}
\end{ex}
%%==========Câu 8
\begin{ex}[2D4K4-1]
Cho phương trình phức $z^2+m z+n=0$, với $m, n$ là các số thực. Biết phương trình có hai nghiệm phức không thuần thực là $z_1=u+2 i-1$ và $z_2=i u+3$. Giá trị của biểu thức $|m+i n|$ tương ứng bằng
\choice
{\True$1$}
{$\sqrt{2}$}
{$\sqrt{3}$}
{$2$}
\loigiai{
	 Vì phương trình phức dạng đa thức hệ số thực nên có hai nghiệm là liên hợp của nhau, ta có\[z_1=\overline{z}_2\Leftrightarrow u+2i-1=\overline{i u+3}=-i\bar u+3.\]
	 Gọi $u=a+i b$, với $a, b \in \mathbb{R} \Rightarrow a+i b+2 i-1=-i(a-i b)+3 \Leftrightarrow\heva{& a=1 \\ & b=-3}\Rightarrow u=1-3i$.
	 Suy ra $z_{1}=u+2 i-1=1-3 i+2 i-1=-i \Rightarrow z_{2}=\overline{z_{1}}=i \Rightarrow\heva{& z_1+z_2=0=-111 \\ & z_1\cdot z_2=-i^2=1.}$\\
	 Suy ra $|m+i n|=|0+i\cdot 1|=1$.
}
\end{ex}
%%==========Câu 9
\begin{ex}[2D4K2-3]
Cho phương trình phức $z^3+a z^2+6 z+c=0$, với $a, b, c$ là các số thực. Biết phương trình có một nghiệm phức là $z_1=1+2 i$. Giá trị của biểu thức $(a+3 c)$ tương ứng bằng
\choice
{$6$}
{\True$-10$}
{$10$}
{$-6$}
\loigiai{
Thay $z_{1}=1+2 i$ vào phương trình ban đầu, ta được
\allowdisplaybreaks
\begin{eqnarray*}
	&&(1+2i)^3+a(1+2i)^2+6(1+2i)+c=0\\
	&\Leftrightarrow&-11-2i+a(-3+4i)+6(1+2i)+c=0\\
	&\Leftrightarrow&\heva{& -3a+c-5=0 \\ & 4a+10=0}\Leftrightarrow\heva{& c=-\dfrac 52 \\ & a=-\dfrac 52.}
\end{eqnarray*}
Suy ra $(a+3 c)=-\dfrac{5}{2}+3 \cdot\left(-\dfrac{5}{2}\right)=-10$.
}
\end{ex}
%%==========Câu 10
\begin{ex}[2D4K2-3]
Cho phương trình phức $z^4+a z^3+b z^2+c z+d=0$, với $a, b, c, d$ là các số thực. Biết phương trình có hai nghiệm phức là $z_1=1+2 i; z_2=3-i$. Giá trị của biểu thức $(a+b+c+d)$ bằng
\choice
{$13$}
{$10$}
{\True$19$}
{$-15$}
\loigiai{
Phương trình đa thức phức hệ số thực luôn có cặp nghiệm là liên hợp của nhau. Suy ra các nghiệm còn lại $z_{3}=1-2 i ; z_{4}=3+i$.\\
Áp dụng hệ thức viet ta có\\
$\heva{& z_1+z_2+z_3+z_4=8=-a\Rightarrow a=-8 \\ & z_1z_2z_3z_4=d=(1-2i)(1+2i)(3-i)(3+i)=50\\&z_1z_2+z_1z_3+z_1z_4+z_2z_3+z_2z_4+z_3z_4=z_1z_2+z_3z_4+\left(z_1+z_2\right)\left(z_3+z_4\right)=b=27\\&\left.z_1z_2z_3+z_1z_2z_4+z_1z_3z_4+z_2z_3z_4=z_1z_2\left(z_3\right]+z_4\right)+z_3z_4\left(z_1+z_2\right)=-c=50\Rightarrow c=-50.}$\\
Suy ra $(a+b+c+d)=-8+27-50+50=19$.
}
\end{ex}
%%==========Câu 11
\begin{ex}[2D4K2-3]
Cho phương trình phức $z^4+b z^3+c z^2+d z+e=0$, với $b, c, d, e$ là các số thực. Biết phương trình có ba nghiệm phức không thuần thực là $z_1=u+2 i; z_2=u+3 i-1; z_3=2 u+2+i$. Giá trị của $b$ lớn nhất có thể bằng
\choice
{$10$}
{$16$}
{$19$}
{\True$14$}
\loigiai{
Phương trình đa thức phức hệ số thực luôn có cặp nghiệm là liên hợp của nhau.\\
Nhận thấy, $z_{2}-z_{1}=i-1$ không phải số thuần ảo nên cặp hai số phức $z_{1} \neq \overline{z_{2}}$. Xảy ra hai trường hợp sau:\\
	 \textbf{Trường hợp 1:} $z_1=\overline{z_3}\Leftrightarrow u+2i=\overline{2u+2+i}=2\bar u+2-i$.\\
	Đặt $u=m+i n$, suy ra $(m+n i)+2 i=2 \cdot(m-n i)+2-i \Leftrightarrow\heva{& m=-2 \\ & n=-1}\Rightarrow u=-2-i$.\\
	Suy ra $z_{1}=u+2 i=-2+i ; z_{2}=u+3 i-1=-3+2 i ; z_{3}=-2-i ; z_{4}=\overline{z_{2}}=-3-2 i$.\\
	Suy ra $-b=\left(z_{1}+z_{2}+z_{3}+z_{4}\right)=-10 \Rightarrow b=10$.\\
	 \textbf{Trường hợp 2:} $z_2=\overline{z_3}\Leftrightarrow u+3i-1=\overline{2u+2+i}=2\cdot\bar u+2-i$.\\
	Đặt $u=m+i n$, suy ra $(m+n i)+3 i-1=2 \cdot(m-n i)+2-i \Leftrightarrow\heva{& m=-3 \\ & m=-\dfrac{4}{3}}\Rightarrow u=-3-\dfrac 43i$.\\
	suy ra $z_{1}=u+2 i=-3+\dfrac{2 i}{3} ; z_{2}=u+3 i-1=-4+\dfrac{5 i}{3} ; z_{3}=-4-\dfrac{5 i}{3} ; z_{4}=-3-\dfrac{2 i}{3}$.\\
	suy ra $-b=\left(z_{1}+z_{2}+z_{3}+z_{4}\right)=-14 \Rightarrow b=14$.
Xét cả hai trường hợp thì suy ra giá trị lớn nhất có thể của $b$ là $b=14$.
}
\end{ex}
%%==========Câu 12
\begin{ex}[2D4K3-3]
Cho số phức $z$ không thuần thực sao cho số phức $u=\dfrac{3 z}{z^2+6}$ thuần thực. Giá trị của $|z|$ bằng
\choice
{\True$\sqrt{6}$}
{$2$}
{$\sqrt{3}$}
{$6$}
\loigiai{
	 \textbf{Cách 1:} Giả thiết suy ra $z^{2}-\dfrac{3}{u} z+6=0$.\\
	Đây là phương trình đa thức phức hệ số thực suy ra có hāi nghiệm là liên hợp của nhau:\[z_1=\overline{z}_2\Rightarrow z_1z_2=\overline{z}_2z_2=\left|z_2\right|^2=6\Leftrightarrow\left|z_2\right|=\sqrt 6=|z|.\]
	 \textbf{Cách 2:} Vì $u$ là số thực nên suy ra \allowdisplaybreaks
	\begin{eqnarray*}
	u=\bar u&\Leftrightarrow&\dfrac{3z}{z^2+6}=\overline{\left(\dfrac{3z}{z^2+6}\right)}=\dfrac{3\bar z}{\bar z^2+6}\Leftrightarrow z\left(\bar z^2+6\right)=\bar z\left(z^2+6\right)\\
	&\Leftrightarrow& z\bar z\cdot\bar z+6z=\bar zz\cdot z+6\bar z\Leftrightarrow|z|^2\cdot\bar z+6z=|z|^2\cdot z+6\bar z\\
	&\Leftrightarrow&\left(|z|^2-6\right)(\bar z-z)=0\Leftrightarrow\hoac{&|z|^2=6\\ &\bar z-z=0}\Leftrightarrow\hoac{&|z|=\sqrt 6\\ &\bar z=z(\text{loại}).}
	\end{eqnarray*}
}
\end{ex}
%%==========Câu 13
\begin{ex}[2D4K3-3]
Cho số phức $z$ không thuần thực sao cho số phức $u=\dfrac{2 z}{z^2+9}$ thuần thực. Biểu thức $|u+3 i|$ có thể nhận giá trị nào dưới đây?
\choice
{$3$}
{$2$}
{\True$\dfrac{\sqrt{37}}{2}$}
{$\dfrac{\sqrt{82}}{3}$}
\loigiai{
Giả thiết suy ra $z^{2}-\dfrac{2}{u} z+9=0$.\\
Đây là phương trình đa thức phức hệ số thực. Để có hai nghiệm phức thì điều kiện
\[\Delta'=\left(\dfrac 1 u\right)^2-9<0\Leftrightarrow u^2>\dfrac 19\Rightarrow|u+3i|=\sqrt{u^2+9}>\sqrt{\dfrac 19+9}=\dfrac{\sqrt{82}}3.\]
}
\end{ex}
%%==========Câu 14
\begin{ex}[2D4K3-3]
Cho số phức $z$ không thuần thực sao cho số phức $u=\dfrac{2019 z}{z^2+z+4}$ thuần thực. Giá trị của $|z|$ bằng
\choice
{$1$}
{\True$2$}
{$\sqrt{2019}$}
{$2020$}
\loigiai{
Giả thiết suy ra $z^{2}+\left(1-\dfrac{2019}{u}\right) z+4=0$.\\
Đây là phương trình đa thức phức hệ số thực và có nghiệm phức $z$ không thuần thực. Suy ra hai nghiệm này là liên hợp của nhau và có mô đun bằng nhau. Suy ra
\[z_1\cdot z_2=z_1\cdot\overline{z}_1=\left|z_1\right|^2=4\Leftrightarrow\left|z_1\right|=2.\]
}
\end{ex}
%%==========Câu 15
\begin{ex}[2D4K3-3]
Cho số phức $z$ không thuần thực sao cho số phức $u=\dfrac{2019 z}{z^2+2020 z+a}$ thuần thực. Biết $\dfrac{6|z|}{3+|z|^2}=\sqrt{3}$. Giá trị của số thực $a$ tương ứng bằng:
\choice
{$2$}
{\True$3$}
{$1$}
{$\sqrt{2019}$}
\loigiai{
Giả thiết suy ra $z^{2}+\left(2020-\dfrac{2019}{u}\right) z+a=0$.\\
Đây là phương trình đa thức phức hệ số thực và có nghiệm phức $\mathrm{z}$ không thuần thực. Suy ra hai nghiệm này là liên hợp của nhau và có mô đun bằng nhau. Suy ra
\[z_1\cdot z_2=z_1\cdot\overline{z}_1=\left|z_1\right|^2=a\Leftrightarrow\left|z_1\right|=\sqrt a=|z|.\eqno (1)\]
Lại có: $\dfrac{6|z|}{3+|z|^{2}}=\sqrt{3} \Leftrightarrow|z|^{2}-2 \sqrt{3} .|z|+3=0 \Leftrightarrow|z|=\sqrt{3}$.\hfill (2)\\
Từ $(1)$ và $(2)$, suy ra $|z|=\sqrt{a}=\sqrt{3} \Leftrightarrow a=3$.
}
\end{ex}
\begin{dang}{Phương trình bậc hai và bậc cao trong số phức}
	Xét phương trình bậc hai $az^2 + bz + c = 0$, \,\, (*) với $a \neq 0$ có biệt số $\Delta = b^2 - 4ac$. Khi đó:
	\begin{enumerate}
		\item Nếu $\Delta = 0$ thì phương trình (*) có nghiệm kép $z_1 = z_2 = -\dfrac{b}{2a}$.
		\item Nếu $\Delta \neq 0$ và gọi $\delta$ là một căn bậc hai của $\Delta$ thì phương trình (*) có hai nghiệm phân biệt là $z_1 = \dfrac{-b + \delta}{2a}$ hoặc $z_2 = \dfrac{-b - \delta}{2a}$.
	\end{enumerate}
	\begin{note}
		\begin{enumerate}
			\item Hệ thức Vi-ét vẫn đúng trong trường phức $\mathbb{C}$: $z_1 + z_2 = -\dfrac{b}{a}$ và $z_1 z_2 = \dfrac{c}{a}$.
			\item Căn bậc hai của số phức $z = x + yi$ là một số phức $w$ và tìm như sau:
			\begin{itemize}
				\item[+] Bước 1. Đặt $w =  a + bi$ với $x, y, a, b \in \mathbb{R}$ là một căn bậc hai của số phức $z$.
				\item[+] Bước 2. Biến đổi $w^2 = x + yi = (a + bi)^2 \Leftrightarrow (a^2 - b^2) + 2abi = x + yi \Leftrightarrow \heva{&a^2 - b^2 = x\\&2ab = y} \Rightarrow \heva{&a = \cdots\\&b = \cdots}$
				\item[+] Bước 3. Kết luận các căn bậc hai của số phức $z$ là $w = a + bi$.
			\end{itemize}
		\end{enumerate}
	\end{note}
	Ta có thể làm tương tự đối với trường hợp căn bậc ba, căn bậc bốn. Ngoài cách tìm căn bậc hai của số phức như trên, ta có thể tách ghép đưa về số chính phương dựa vào hằng đẳng thức.
\end{dang}
\subsubsection{Ví dụ}
\begin{vd}%[ĐỀ CƯƠNG GIẢNG DẠY VÀ LUYỆN THI LỚP 12]%[Phan Văn Thành, dự án : 12-EX-DCHT-Lần 1]%[2D4B4-1]
	Tìm căn bậc hai của số phức $z = 16 - 30i$. \dapso{$z = \pm 5 \mp 3i$}
	\loigiai{Đặt $w = a + bi$ là căn bậc hai của số phức $z = 16 - 30i$ với $a, b \in \mathbb{R}$.\\
		Khi đó $(a + bi)^2 = 16 - 30i \Leftrightarrow a^2 - b^2 + 2abi = 16 - 30i \Leftrightarrow \heva{&a^2 - b^2 = 16 &(1)\\&2ab = -30 &(2)}$\\
		Từ $(2) \Rightarrow a = -\dfrac{15}{b}$ thay vào $(1)$, ta có 
		$\dfrac{225}{b^2} - b^2 = 16 \Rightarrow b^4 + 16b^2 - 225 = 0 \Rightarrow \hoac{&b^2 = 9\\&b^2 = -25 \,\,(VN)} \Rightarrow \hoac{&b = -3\\&b = 3.}$\\
		Với $b = -3 \Rightarrow a = 5$. Với $b = 3 \Rightarrow a = -5$.
		Vậy $w = -5 + 3i$, $w = 5 - 3i$.}
\end{vd}

\begin{vd}%[ĐỀ CƯƠNG GIẢNG DẠY VÀ LUYỆN THI LỚP 12]%[Phan Văn Thành, dự án : 12-EX-DCHT-Lần 1]%[2D4B4-1]
	Giải phương trình: $z^2 + 2z + 5 = 0$ trên tập số phức $\mathbb{C}$. \dapso{$z = - 1 \pm 2i$}
	\loigiai{
		Ta có $\Delta' = -4 = 4i^2$ nên $\delta = 2i$ là một căn bậc hai của $\Delta'$.
		Vậy phương trình có hai nghiệm phức phân biệt $x_{1,2} = -1 \pm 2i$.}
\end{vd}
\subsubsection{Bài tập rèn luyện}

\begin{ex}%[Nguyễn Văn Sang - TDM - SP]%[2D4B4-1]
Có bao nhiêu số phức $z$ thỏa mãn phương trình $(z-i)\left(z^2-3i z-2\right)=0$?
	\choice
	{$3$}
	{$0$}
	{$1$}
	{\True$2$}
	\loigiai{
Ta có $(z-i)\left( {z^2 - 3iz - 2} \right) = 0 \Leftrightarrow \hoac{&z-i=0 \\& z^2-3iz-2=0. \qquad(2)}$
\\Phương trình $(2)$ có $\Delta  = \left(-3i\right)^2- 4\cdot1\cdot(-2)=-1=i^2$. 
\\Do đó phương trình (2) có hai nghiệm $z_1 = \dfrac{3i+i}{2} = 2i$, $z_2 =\dfrac{3i-i}{2}=i$. 
\\Vậy có tất cả $2$ số phức thỏa mãn là $i$, $2i$.
	}
\end{ex}
\begin{ex}%[Nguyễn Văn Sang - TDM - SP]%[2D4B4-1]
	Có bao nhiêu số phức $z$ thỏa mãn phương trình $z^2-4i z-4=0$?
	\choice
	{$2$}
	{$0$}
	{Vô số}
	{\True$1$}
	\loigiai{
	Phương trình có $\Delta  = {\left( {-4i} \right)^2} - 4(1)( - 4) =  0 $. 
	\\Do đó phương trình $(2)$ có nghiệm kép $z_1 =z_2= \dfrac{{4i}}{2} = 2i$. 
	\\Vậy có tất cả $1$ số phức thỏa mãn là $2i$.
	}
\end{ex}
\begin{ex}%[Nguyễn Văn Sang - TDM - SP]%[2D4B4-1]
Hai số phức khác nhau $z_1$ và $z_2$ thỏa mãn phương trình $z^2+(4-2i) z-2-16i=0$. Giá trị của biểu thức $P=\left|z_1\right|+\left|z_2\right|$ bằng
	\choice
	{\True$\sqrt{10}+\sqrt{26}$}
	{$2 \sqrt{10}+1$}
	{$2 \sqrt{65}$}
	{$4 \sqrt{5}$}
	\loigiai{
	Ta có $\Delta'=(i-2)^2-(-2-16i)=5+12i=(3+2i)^2$.
\\Suy ra $z_1=(i-2)-(3+2i)=-5-i$; $z_2=(i-2)+(3+2i)=1+3i$.
	\\Suy ra $P=\left|z_1\right|+\left|z_2\right|=|1+3i|+|-5-i|=\sqrt{10}+\sqrt{26}$.
	}
\end{ex}
\begin{ex}%[Nguyễn Văn Sang - TDM - SP]%[2D4B4-1]
Cho phương trình phức $z^3-i=0 .$ Goi $z_1$ và $z_2$ là hai nghiệm phức không thuần ảo của phương trình đã cho. Tổng phần ảo của hai số phức $z_1$ và $z_2$ là
	\choice
	{$-1$}
	{$0$}
	{\True$1$}
	{$2$}
	\loigiai{
		Ta có
		\begin{eqnarray*}
			 z^3+i^3=0 &\Leftrightarrow&(z+i)\left(z^2-i z+i^2\right)=0\\ &\Leftrightarrow&(z+i)\left(z^2-i z-1\right)=0 \\&\Leftrightarrow&\hoac {&z=-i \\&z^2-i z-1=0}\\ &\Leftrightarrow&\hoac {&z=-i \\& z=\dfrac{\pm \sqrt{3}+i}{2}.}
		\end{eqnarray*}
		\noindent
		Suy ra tổng phần ảo của hai nghiệm phức không thuần ảo là $\dfrac{1}{2}+\dfrac{1}{2}=1$.
	}
\end{ex}
\begin{ex}%[Nguyễn Văn Sang - TDM - SP]%[2D4K4-1]
Cho ba điểm $A$, $B$, $C$ lần lượt biểu diễn ba nghiệm của phương trình phức $z^3+8 i=0$. Tam giác $ABC$ có diện tích bằng
	\choice
	{$\sqrt{3}$}
	{\True$3 \sqrt{3}$}
	{$2 \sqrt{3}$}
	{$2$}
	\loigiai{
		Ta có 
		\immini
		{
		\begin{eqnarray*}
			z^3+8 i=0 &\Leftrightarrow& z^3-8 i^3=0 \\
			&\Leftrightarrow&(z-2i)\left(z^2+2i z+(2i)^2\right)=0\\
			&\Leftrightarrow&(z-2i)\left(z^2+2i z-4\right)=0\\ &\Leftrightarrow&\hoac{&z-2i=0 \\& (z+i)^2=3}\\
			&\Leftrightarrow&\hoac{&z=2i \\& z=\pm \sqrt{3}-i.}
		\end{eqnarray*}
		}
		{
		\begin{tikzpicture}[line join = round, line cap = round,>=stealth,font=\footnotesize,scale=1]
		\def\xt{-3} \def\xp{3} \def\yd{-2} \def\yt{3}
		\draw[->] (\xt,0)--(\xp,0) node[below]{$x$};
		\draw[->] (0,\yd)--(0,\yt) node[left]{$y$};
		\fill (0,0) circle (1.5pt) node[above left]{$O$};
		\draw[dashed] (-1.73,0) -- (-1.73,-1);
		\draw[dashed] (1.73,0) -- (1.73,-1);
		\draw (1.73,-1) -- (0,2)--(-1.73,-1)--(1.73,-1);
		\node at (-1.73,0.3) {$-\sqrt{3}$};
		\node at (-1.73,-1.3) {$C$};
		\node at (1.73,-1.3) {$B$};
		\node at (-0.3,-1.3) {$-1$};
		\node at (1.73,0.3) {$\sqrt{3}$};
		\node at (-0.3,2) {$A$};
		\node at (0.3,2) {$2$};
		\end{tikzpicture}
		}\noindent
		Tọa độ các điểm biểu diễn ba nghiệm là $A(0 ; 2), B(\sqrt{3} ;-1), C(-\sqrt{3} ;-1)$.\\
		Ta có tam giác $A B C$ cân tại $A$ và đường cao $A H=3$, cạnh $B C=2\sqrt{3}$. \\Suy ra diện tích tam giác $A B C$ là $S_{\triangle ABC}=\dfrac{1}{2} BC\cdot AH=\dfrac{1}{2} \cdot 2 \sqrt{3} \cdot 3=3\sqrt{3}$.}
\end{ex}
\begin{ex}%[Nguyễn Văn Sang - TDM - SP]%[2D4K4-2]
Cho hai số phức $z_1$ và $z_2$ là hai nghiệm phức phân biệt của phương trình $z^2-az+3i=0$, với $a$ là số phức. Giá trị nhỏ nhất của biểu thức $P=2\left|z_1\right|+3\left|z_2\right|$ bằng
	\choice
	{$3\sqrt{3}$}
	{\True$6 \sqrt{2}$}
	{$4 \sqrt{3}$}
	{$1+\sqrt{3}$}
	\loigiai{
	Áp dụng bất đẳng thức $AM-GM$, ta được $$P=2\left|z_1\right|+3\left|z_2\right| \geq 2 \sqrt{6\left|z_1 \cdot {z}_2\right|}=2 \sqrt{6\cdot|3i|}=2 \sqrt{6.3}=6 \sqrt{2}$$
	Dấu \lq\lq=\rq\rq\, xảy ra khi $\heva{&2\left|z_1\right|=3\left|z_2\right|\\& 2\left|z_1\right|+3\left|z_2\right|=6 \sqrt{2}} \Leftrightarrow  2\left|z_1\right|=3\left|z_2\right|=3 \sqrt{2} \Leftrightarrow\heva{&\left|z_1\right|=\dfrac{3 \sqrt{2}}{2} \\& \left|z_2\right|=\sqrt{2}.}$ \\
	Suy ra giá trị nhỏ nhất của $P$ là $6\sqrt{2}$.
	}
\end{ex}
\begin{ex}%[Nguyễn Văn Sang - TDM - SP]%[2D4K4-2]
Gọi ba số phức $z_1$, $z_2$, $z_3$ là ba nghiệm của phương trình $z^3+b z^2+cz+3+4i=0$ Giá trị nhỏ nhất của biểu thức $P=\left|z_1\right|+\left|z_2\right|+\left|z_3\right|$ bằng
	\choice
	{\True$3 \sqrt{5}$}
	{$3 \sqrt{2}$}
	{$3 $}
	{$3 \sqrt{6}$}
	\loigiai{
		Áp dụng bất đẳng thức $AM-GM$, ta được $$P=\left|z_1\right|+\left|z_2\right|+\left|z_3\right| \geq 3 \cdot \sqrt{\left|z_1\right| \cdot\left|z_2\right| \cdot\left|z_3\right|}=3 \sqrt{\left|z_1z_2 z_3\right|}=3 \sqrt{|-3-4i|}=3 \sqrt{5}.$$
		Dấu \lq\lq=\rq\rq\, xảy ra khi $\left|z_1\right|=\left|z_2\right|=\left|z_3\right|=\sqrt{5}$. \\Suy ra giá trị nhỏ nhất của $P$ là $3 \sqrt{5}$.
	}
\end{ex}
\begin{ex}%[Nguyễn Văn Sang - TDM - SP]%[2D4K4-2]
Cho phương trình phức $z^2-(11-4i) z+5-8 i=0 .$ Gọi $z_1$ và $z_2$ là hai nghiệm của phương trình. Giá trị của biểu thức $T=\left|\left(z_1+3i-1\right)\left(z_2+3i-1\right)\right|$ bằng
	\choice
	{$\sqrt{298}$}
	{$\sqrt{205}$}
	{\True$\sqrt{533}$}
	{$\sqrt{391}$}
	\loigiai{
		Áp dụng hệ thức Vi-ét, ta được $$ P=\left(z_1+3i-1\right)\left(z_2+3i-1\right)=z_1 z_2+(3i-1)\left(z_1+z_2\right)+(3i-1)^2=(5-8 i)+(3i-1)(11-4i)+(3i-1)^2.$$
		Suy ra $P=-2+23i \Rightarrow T=|P|=\sqrt{533}$.
	}
\end{ex}
\begin{ex}%[Nguyễn Văn Sang - TDM - SP]%[2D4K4-2]
Cho phương trình phức $z^2-(3-2i) z+4-i=0 .$ Gọi $z_1$ và $z_2$ là hai nghiệm của phương trình. Phần ảo của số phức $T=\dfrac{1}{z_1-i}+\dfrac{1}{z_2-i}$ bằng
	\choice
	{\True$\dfrac{8}{17}$}
	{$\dfrac{19}{17}$}
	{$-\dfrac{8}{17}$}
	{$-\dfrac{19}{17}$}
	\loigiai{
		Áp dụng hệ thức Vi-ét, ta được $$T=\dfrac{1}{z_1-i}+\dfrac{1}{z_2-i}=\dfrac{z_1+z_2-2i}{\left(z_1-i\right)\left(z_2-i\right)}=\dfrac{z_1+z_2-2i}{z_1 z_2-i\left(z_1+z_2\right)-1}=\dfrac{3-2i-2i}{4-i-i(3-2i)-1}=\dfrac{19}{17}+\dfrac{8}{17} i.$$
		Suy ra phần ảo của $T$ là $\dfrac{8}{17}$.
	}
\end{ex}
\begin{ex}%[Nguyễn Văn Sang - TDM - SP]%[2D4K4-1]
Cho phương trình bậc hai $z^2+(5i-6)z+3-15i=0$. Gọi $z_1$ là nghiệm phức của phương trình và có phần ảo lớn hơn nghiệm kia. Giá trị $\left|z_1\right|$ bằng
	\choice
	{$\sqrt{17}$}
	{\True$\sqrt{13}$}
	{$\sqrt{41}$}
	{$\sqrt{5}$}
	\loigiai{
		Ta có $\Delta=b^2-4 a c=(5i-6)^2-4.1 \cdot(3-15i)=-1 \Rightarrow \sqrt{\Delta}=\pm i$
		\\Suy ra hai nghiệm $\hoac{z=\dfrac{(6-5i)+i}{2}=3-2i \\ z=\dfrac{(6-5i)-i}{2}=3-3i} \Rightarrow z_1=3-2i \Rightarrow\left|z_1\right|=|3-2i|=\sqrt{13}$.
	}
\end{ex}
\begin{ex}%[Nguyễn Văn Sang - TDM - SP]%[2D4K4-1]
Cho phương trình bậc hai $z^2-(7-6i) z+7-19i=0$. Gọi $z_1$ là nghiệm phức của phương trình và có phần thực lớn hơn nghiệm kia. Giá trị $\left|z_1\right|$ bằng
	\choice
	{$\sqrt{13}$}
	{$3 \sqrt{2}$}
	{$\sqrt{26}$}
	{\True$\sqrt{41}$}
	\loigiai{
		Ta có $\Delta=b^2-4 a c=(7-6i)^2-4\cdot1 \cdot(7-19 i)=-15-8 i=16i^2-8 i+1=(1-4i)^2$.\\
		Suy ra $\sqrt{\Delta}=\pm(1-4i)$.\\
		\\Ta được hai nghiệm $\hoac{&z=\dfrac{(7-6i)+(1-4i)}{2}=4-5i \\& z=\dfrac{(7-6i)-(1-4i)}{2}=3-i} \Rightarrow z_1=4-5i.$\\
		Suy ra $\left|z_1\right|=|4-5i|=\sqrt{41}.$
	}
\end{ex}
\begin{ex}%[Nguyễn Văn Sang - TDM - SP]%[2D4K4-2]
Cho hai nghiệm của phương trình bậc hai $z^2+az+b=0$ lần lượt là $z_1=3-2i$ và $z_2=1+4i$. Giá trị của biểu thửc $T=|a+2ib|$ bằng
	\choice
	{$2 \sqrt{37}$}
	{$4 \sqrt{71}$}
	{\True$4 \sqrt{61}$}
	{$12$}
	\loigiai{
		Áp dụng hệ thức Vi-ét, ta được 
		$$\heva{z_1+z_2=-a=3-2i+1+4i=4+2i \\ z_1.z_2=b=(3-2i)(1+4i)=11+10i} \Rightarrow\heva{a=-4-2i \\ b=11+10i.}$$
		Suy ra $T=|a+2i b|=|-4-2i+2i(11+10i)|=4 \sqrt{61}$.
	}
\end{ex}
\begin{ex}%[Nguyễn Văn Sang - TDM - SP]%[2D4K4-2]
Cho phương trình bậc hai $z^2+az+2a-i=0$, với $a$ là số phức, có một nghiệm là $z_1=2-5i$. Giá trị của $|a|$ bằng
	\choice
	{$\dfrac{36}{41}$}
	{$2 \sqrt{41}$}
	{$\sqrt{29}$}
	{\True$\dfrac{21 \sqrt{82}}{41}$}
	\loigiai{
		Ta có $a(z+2)=i-z^2 \Rightarrow a=\dfrac{i-z^2}{z+2}=\dfrac{i-(2-5i)^2}{(2-5i)+2}=-\dfrac{21}{41}+\dfrac{189}{41} i$.\\
		Suy ra $|a|=\dfrac{21 \sqrt{82}}{41}$. 
	}
\end{ex}
\begin{ex}%[Nguyễn Văn Sang - TDM - SP]%[2D4K4-2]
Cho phương trình bậc ba $z^3+az+b=0$ có ba nghiệm phức là $z_1=1-i$; $z_2=2+3i$; $z_3$. Phần thực của số phức $w=a+i b$ bằng
	\choice
	{$24$}
	{\True$13$}
	{$-13$}
	{$-24$}
	\loigiai{
		Áp dụng hệ thức Vi-ét bậc $3$, ta được
		$$\heva{&z_1+z_2+z_3=0=1-i+2+3i+z_3  \\& z_1 z_2+z_2 z_3+z_3 z_1=-11 i \\& z_1z_2z_3=-13-13i}\Rightarrow \heva{& z_3=-3-2i \\ & a=-11 i\\&b=-13-13i.}$$
		$ \Rightarrow a+i b=13-24i$. Suy ra phần thực của số phức ${w}=a+i b$ là 13.
	}
\end{ex}
\begin{ex}%[Nguyễn Văn Sang - TDM - SP]%[2D4K4-2]
Cho phương trình bậc ba $z^3+(5-i) z^2+az+b=0$ có ba nghiệm phức là $z_1=1-i$; $z_2$; $z_2+2 $. Phần ảo của số phức $b$  bằng
	\choice
	{\True$-15$}
	{$15$}
	{$3$}
	{$-3$}
	\loigiai{
		Áp dụng hệ thức Vi-ét bậc $3$, ta được 
		$$\heva{&z_1 + z_2 + z_3=-\left(5-i\right)=1-i+z_2+z_2+2i \Rightarrow z_2=-3\\&z_1z_2 + z_2{z_3} + {z_3}z_1 = a\\&	z_1z_2{z_3} = \left(1-i\right)(-3)(-3+2i)=3-15i.}$$
		\\Suy ra phần ảo của số phức $b$ là $-15$.
	}
\end{ex}
\begin{ex}%[Nguyễn Văn Sang - TDM - SP]%[2D4K4-2]
Cho phương trình bậc bốn $z^{4}+z^3+b z^2+c z+d=0$ có bốn nghiệm phức lần lượt là $z_1=2+i$; $z_2=3-2i$; $z_3$; $z_4=z_3+2i-1$. Phần ảo của số phức $d$ bằng
	\choice
	{$\dfrac{51}{2}$}
	{$74$}
	{\True$-\dfrac{51}{2}$}
	{$-74$}
	\loigiai{
		Áp dụng hệ thức Vi-ét bậc $4$, ta được
		$$\heva{& z_1+z_2+z_3+z_{4}=-1=2+i+3-2i+z_3+z_3+2i-1 \Rightarrow z_3=-\dfrac{5}{2}-\dfrac{1}{2} i \\ & z_1z_2 z_3 z_{4}=d=(2+i)(3-2i)\left(-\dfrac{5}{2}-\dfrac{1}{2} i\right)\left(-\dfrac{5}{2}-\dfrac{1}{2} i+2i-1\right)=74-\dfrac{51}{2} i.}$$
		Suy ra phần ảo của số phức $d$ là $-\dfrac{51}{2}$.
	}
\end{ex}

\begin{dang}{Dạng lượng giác của số phức}
	Cho số phức $ z=a+bi $ ($ a,\ b\in\mathbb{R} $). Đặt $ r=\sqrt{a^{2}+b^{2}} $, $ \cos\varphi=\dfrac{a}{\sqrt{a^{2}+b^{2}}} $, $ \sin\varphi=\dfrac{b}{\sqrt{a^{2}+b^{2}}} $. Khi đó
	\begin{enumerate}
		\item Dạng lượng giác của số phức $ z $ là $ z=r(\cos\varphi+i\sin\varphi) $.
		\item Một acgument của số phức $ z $ là $ \varphi $.
		\item $ \tan\varphi=\dfrac{b}{a} $.
		\item $ z^{n}=r^{n}(\cos n\varphi+i\sin n\varphi) $.
	\end{enumerate}
\end{dang}
\begin{vd}[B-2012]%[Dương BùiĐức, Dự án (12-EX-1-DCHT)]%[2D4B5-3]
	Gọi $ z_{1} $, $ z_{2} $ là hai nghiệm của phương trình $ z^{2}-2\sqrt{3}iz-4=0 $. Viết dạng lượng giác của $ z_{1} $ và $ z_{2} $.\dapso{$ 2\left(\cos\dfrac{2\pi}{3}+i\sin\dfrac{2\pi}{3}\right) $ và $ 2\left(\cos\dfrac{\pi}{3}+i\sin\dfrac{\pi}{3}\right) $}
	\loigiai{
		Ta có $ z^{2}-2\sqrt{3}iz-4=0\Leftrightarrow \hoac{&z_{1}=1+\sqrt{3}i\\ &z_{2}=-1+\sqrt{3}i}\Rightarrow \hoac{&z_{1}=2\left(\cos\dfrac{\pi}{3}+i\sin\dfrac{\pi}{3}\right) \\ &z_{2}=2\left(\cos\dfrac{2\pi}{3}+i\sin\dfrac{2\pi}{3}\right).} $
	}
\end{vd}
\begin{vd}%[Dương BùiĐức, Dự án (12-EX-1-DCHT)]%[2D4K5-3]
	Viết số phức $ z $ dưới dạng lượng giác, biết rằng $ |z-1|=|z-i\sqrt{3}| $ và $ i\overline{z} $ có một acgument bằng $ \dfrac{\pi}{6} $.\\
	\dapso{$ z=\cos\dfrac{\pi}{3}+i\sin\dfrac{\pi}{3} $ và $ z=\dfrac{1}{2}\left(\cos\dfrac{2\pi}{3}+i\sin\dfrac{2\pi}{3}\right) $}
	\loigiai{
		Gọi $ z=a+bi $ (với $ a,\ b\in\mathbb{R} $). Ta có
		\[
		|z-1|=|z-i\sqrt{3}|\Leftrightarrow |(a-1)+bi|=|a+(b-\sqrt{3})i|\Leftrightarrow (a-1)^{2}+b^{2}=a^{2}+(b-\sqrt{3})^{2}\Leftrightarrow 2a-1=\sqrt{3}(2b-\sqrt{3})\Leftrightarrow a=\sqrt{3}b-1.
		\]
		Ta lại có $ i\overline{z}=b+ai\Rightarrow \dfrac{b}{\sqrt{a^{2}+b^{2}}}=\cos\dfrac{\pi}{6}\Rightarrow \dfrac{b}{\sqrt{a^{2}+b^{2}}}=\dfrac{\sqrt{3}}{2}\Rightarrow b^{2}=3a^{2}\Rightarrow b=\pm\sqrt{3}a $.
		\begin{itemize}
			\item Với $ b=\sqrt{3}a $, ta có $ a=3a-1\Rightarrow a=\dfrac{1}{2} $. Do đó $ b=\dfrac{\sqrt{3}}{2}\Rightarrow z=\dfrac{1}{2}+\dfrac{\sqrt{3}}{2}i=\cos\dfrac{\pi}{3}+i\sin\dfrac{\pi}{3} $.
			\item Với $ b=-\sqrt{3}a $, ta có $ a=-3a-1\Rightarrow a=-\dfrac{1}{4} $. Do đó $ b=\dfrac{\sqrt{3}}{4}\Rightarrow z=-\dfrac{1}{4}+\dfrac{\sqrt{3}}{4}i=\dfrac{1}{2}\left(\cos\dfrac{2\pi}{3}+i\sin\dfrac{2\pi}{3}\right) $.
		\end{itemize}
		Vậy $ z=\cos\dfrac{\pi}{3}+i\sin\dfrac{\pi}{3} $ và $ z=\dfrac{1}{2}\left(\cos\dfrac{2\pi}{3}+i\sin\dfrac{2\pi}{3}\right) $.
	}
\end{vd}
\begin{vd}%[Dương BùiĐức, Dự án (12-EX-1-DCHT)]%[2D4K5-3]
	Tìm số phức $ z $, biết rằng $ |1-2z|=|i-2\overline{z}| $ và $ \dfrac{z+3}{z-3} $ có một acgument bằng $ \dfrac{\pi}{4} $.\\
	\dapso{$ z=\dfrac{3\pm 3\sqrt{3}}{2}-\dfrac{3\pm 3\sqrt{3}}{2}i $ và $ z=\dfrac{-3\pm 3\sqrt{3}}{2}+\dfrac{3\mp 3\sqrt{3}}{2}i $}
	\loigiai{
		Gọi $ z=a+bi $ (với $ a,\ b\in\mathbb{R} $). Ta có
		\[
		|1-2z|=|i-2\overline{z}|\Leftrightarrow |(1-2a)-2bi|=|-2a+(2b+1)i|\Leftrightarrow (2a-1)^{2}+4b^{2}=4a^{2}+(2b+1)^{2}\Leftrightarrow 4a-1=-4b-1\Leftrightarrow a=-b.
		\]
		Ta lại có $ \dfrac{z+3}{z-3}=\dfrac{(a+3)+bi}{(a-3)+bi}=\dfrac{[(a+3)+bi][(a-3)-bi]}{(a-3)^{2}+b^{2}}=\dfrac{(a^{2}+b^{2}-9)-6bi}{(a-3)^{2}+b^{2}} $. Suy ra
		\[
		\cos\dfrac{\pi}{4}=\dfrac{2a^{2}-9}{\sqrt{(2a^{2}-9)^{2}+36a^{2}}}\Leftrightarrow (2a^{2}-9)^{2}=36a^{2}\Leftrightarrow 2a^{2}-9=\pm 6a.
		\]
		\begin{itemize}
			\item Với $ 2a^{2}-9=6a\Leftrightarrow 2a^{2}-6a-9=0\Leftrightarrow a=\dfrac{3\pm 3\sqrt{3}}{2} $.
			\begin{itemize}
				\item Nếu $ a=\dfrac{3+3\sqrt{3}}{2} $ thì $ b=-\dfrac{3+3\sqrt{3}}{2}\Rightarrow z=\dfrac{3+3\sqrt{3}}{2}-\dfrac{3+3\sqrt{3}}{2}i $.
				\item Nếu $ a=\dfrac{3-3\sqrt{3}}{2} $ thì $ b=-\dfrac{3-3\sqrt{3}}{2}\Rightarrow z=\dfrac{3-3\sqrt{3}}{2}-\dfrac{3-3\sqrt{3}}{2}i $.
			\end{itemize}
			\item Với $ 2a^{2}-9=-6a\Leftrightarrow 2a^{2}+6a-9=0\Leftrightarrow a=\dfrac{-3\pm 3\sqrt{3}}{2} $.
			\begin{itemize}
				\item Nếu $ a=\dfrac{-3+3\sqrt{3}}{2} $ thì $ b=\dfrac{3-3\sqrt{3}}{2}\Rightarrow z=\dfrac{-3+3\sqrt{3}}{2}+\dfrac{3-3\sqrt{3}}{2}i $.
				\item Nếu $ a=\dfrac{-3-3\sqrt{3}}{2} $ thì $ b=\dfrac{3+3\sqrt{3}}{2}\Rightarrow z=\dfrac{-3-3\sqrt{3}}{2}+\dfrac{3+3\sqrt{3}}{2}i $.
			\end{itemize}
		\end{itemize}
		Vậy có $ 4 $ số phức thỏa mãn là $ z=\dfrac{3\pm 3\sqrt{3}}{2}-\dfrac{3\pm 3\sqrt{3}}{2}i $ và $ z=\dfrac{-3\pm 3\sqrt{3}}{2}+\dfrac{3\mp 3\sqrt{3}}{2}i $.
	}
\end{vd}
\begin{vd}%[Dương BùiĐức, Dự án (12-EX-1-DCHT)]%[2D4K5-3]
	Tìm số phức $ z $, biết rằng $ |z|=|2\overline{z}-\sqrt{3}+i| $ và $ \dfrac{(1+i)z}{1-\sqrt{3}+(1+\sqrt{3})i} $ có một acgument bằng $ -\dfrac{\pi}{6} $.\\
	\dapso{$ z=\sqrt{3}+i $}
	\loigiai{
		Gọi $ z=a+bi $ (với $ a,\ b\in\mathbb{R} $). Ta có
		\[
		|z|=|2\overline{z}-\sqrt{3}+i|\Leftrightarrow |a+bi|=|(2a-\sqrt{3})+(-2b+1)i|\Leftrightarrow a^{2}+b^{2}=(2a-\sqrt{3})^{2}+(2b-1)^{2}\Leftrightarrow 3a^{2}+3b^{2}-4\sqrt{3}a-4b+4=0.\tag{1}
		\]
		Ta lại có
		\begin{eqnarray*}
			\dfrac{(1+i)z}{1-\sqrt{3}+(1+\sqrt{3})i}&=&\dfrac{[(1+i)(a+bi)][1-\sqrt{3}-(1+\sqrt{3})i]}{8}\\
			&=&\dfrac{[(a-b)+(a+b)i][1-\sqrt{3}-(1+\sqrt{3})i]}{8}\\
			&=&\dfrac{(a+\sqrt{3}b)+(-\sqrt{3}a+b)i}{4}.
		\end{eqnarray*}
		Suy ra $ \tan\left( -\dfrac{\pi}{6}\right) =\dfrac{-\sqrt{3}a+b}{a+\sqrt{3}b}\Rightarrow -\dfrac{1}{\sqrt{3}}=\dfrac{-\sqrt{3}a+b}{a+\sqrt{3}b}\Rightarrow a=\sqrt{3}b $.\\
		Thay vào (1) ta được
		\[
		9b^{2}+3b^{2}-12b^{2}-4b+4=0\Leftrightarrow b=1\Rightarrow a=\sqrt{3}.
		\]
		Vậy $ z=\sqrt{3}+i $.
	}
\end{vd}
\begin{vd}%[Dương BùiĐức, Dự án (12-EX-1-DCHT)]%[2D4K5-3]
	Tìm số phức $ z $ thỏa mãn $ |z-1|=|z-3| $ và một acgument của $ z-3 $ bằng một acgument của $ z+3 $ cộng với $ \dfrac{\pi}{2} $.\dapso{$ z=2+i\sqrt{5} $}
	\loigiai{
		Gọi $ z=a+bi $ (với $ a,\ b\in\mathbb{R} $). Ta có
		\[
		|z-1|=|z-3|\Leftrightarrow |(a-1)+bi|=|(a-3)+bi|\Leftrightarrow (a-1)^{2}+b^{2}=(a-3)^{2}+b^{2}\Leftrightarrow 2(2a-4)=0\Leftrightarrow a=2.
		\]
		Lại có $ z-3=(a-3)+bi $ và $ z+3=(a+3)+bi $.\\
		Gọi $ \varphi_{1} $, $ \varphi_{2} $ lần lượt là acgument của $ z-3 $ và $ z+3 $. Theo đề bài ta có
		\[
		\varphi_{1}=\varphi_{2}+\dfrac{\pi}{2}\Leftrightarrow \sin\varphi_{1}=\sin\left(\varphi_{2}+\dfrac{\pi}{2}\right)\Leftrightarrow \sin\varphi_{1}=\cos\varphi_{2}.
		\]
		Do vậy
		\[
		\dfrac{b}{\sqrt{(a-1)^{2}+b^{2}}}=\dfrac{a+3}{\sqrt{(a+3)^{2}+b^{2}}}\Leftrightarrow \dfrac{b}{\sqrt{1+b^{2}}}=\dfrac{5}{\sqrt{25+b^{2}}}\Leftrightarrow \heva{&b>0\\ &b\sqrt{25+b^{2}}=5\sqrt{1+b^{2}}}\Leftrightarrow b=\sqrt{5}.
		\]
		Vậy $ z=2+i\sqrt{5} $.
	}
\end{vd}
\begin{vd}%[Dương BùiĐức, Dự án (12-EX-1-DCHT)]%[2D4K5-3]
	Cho số phức $ z $ thỏa mãn $ |z|+(1+i\sqrt{3})z=3 $. Hãy tìm mô-đun của số phức $ w=z+z^{2}+z^{123} $.\\
	\dapso{$ |w|=2 $}
	\loigiai{
		Gọi $ z=a+bi $ (với $ a,\ b\in\mathbb{R} $). Ta có
		\[
		|z|+(1+i\sqrt{3})z=3\Leftrightarrow \sqrt{a^{2}+b^{2}}+(1+i\sqrt{3})(a+bi)=3\Leftrightarrow (\sqrt{a^{2}+b^{2}}+a-b\sqrt{3})+(a\sqrt{3}+b)i=3.
		\]
		Do đó $ \heva{&\sqrt{a^{2}+b^{2}}+a-b\sqrt{3}=3\\ &b+a\sqrt{3}=0}\Leftrightarrow \heva{&\sqrt{a^{2}+b^{2}}+a-b\sqrt{3}=3\\ &b=-a\sqrt{3}.} $\\
		Suy ra $ 2|a|+4a=3\Leftrightarrow \hoac{&6a=3\ (\text{nếu }a\geq 0)\\ &2a=3\ (\text{nếu }a\leq 0)}\Leftrightarrow \hoac{&a=\dfrac{1}{2}\ (\text{thỏa mãn }a\geq 0)\\ &a=\dfrac{3}{2}\ (\text{không thỏa mãn }a\leq 0).} $\\
		Với $ a=\dfrac{1}{2}$ thì $ b=-\dfrac{\sqrt{3}}{2}\Rightarrow z=\dfrac{1}{2}-\dfrac{\sqrt{3}}{2}i=\cos\left( -\dfrac{\pi}{3}\right)+i\sin\left( -\dfrac{\pi}{3}\right) $. Khi đó 
		\[
		w=\cos\left( -\dfrac{\pi}{3}\right)+i\sin\left( -\dfrac{\pi}{3}\right)+\cos\left( -\dfrac{2\pi}{3}\right)+i\sin\left( -\dfrac{2\pi}{3}\right)+\cos\left( -\dfrac{123\pi}{3}\right)+i\sin\left( -\dfrac{123\pi}{3}\right)=-1-\sqrt{3}i\Rightarrow |w|=2.
		\]
		Vậy $ |w|=2 $.
	}
\end{vd}
\begin{vd}%[Dương BùiĐức, Dự án (12-EX-1-DCHT)]%[2D4K5-3]
	Tìm acgument âm lớn nhất của số phức $ z=(1+i\sqrt{3})^{10} $.\dapso{$ \varphi=-\dfrac{2\pi}{3} $}
	\loigiai{
		Ta có $ 1+i\sqrt{3}=2\left( \dfrac{1}{2}+\dfrac{\sqrt{3}}{2}i\right)=2\left(\cos\dfrac{\pi}{3}+i\sin\dfrac{\pi}{3}\right)\Rightarrow (1+i\sqrt{3})^{10}=2^{10}\left(\cos\dfrac{10\pi}{3}+i\sin\dfrac{10\pi}{3}\right) $.\\
		Ta thấy $\dfrac{10\pi}{3}+k2\pi<0\Rightarrow k<-\dfrac{5}{3} $. Vì $ k\in\mathbb{Z} $ nên $ k=-2,\ -3,\ldots $. Do đó, acgument âm lớn nhất của số phức $ z=(1+i\sqrt{3})^{10} $ là $ -\dfrac{2\pi}{3} $.
	}
\end{vd}
\begin{vd}%[Dương BùiĐức, Dự án (12-EX-1-DCHT)]%[2D4K5-3]
	Tìm số phức $ z $ thỏa mãn $ |z+3+i\sqrt{3}|=\sqrt{3} $ và có acgument dương nhỏ nhất.\dapso{$ z=-3 $}
	\loigiai{		
		\immini{
			Gọi $ z=a+bi $ (với $ a,\ b\in\mathbb{R} $). Ta có
			\[
			|z+3+i\sqrt{3}|=\sqrt{3}\Leftrightarrow (a+3)^{2}+(b+\sqrt{3})^{2}=3.
			\]
			Gọi $ M $ là điểm biểu diễn số phức $ z $, khi đó $ M(a;b) $ nằm trên đường tròn tâm $ I(-3;-\sqrt{3}) $ có bán kính $ R=\sqrt{3} $. Dễ thấy đường tròn tâm $ I $ bán kính $ R=\sqrt{3} $ tiếp xúc với trục hoành tại điểm $ M_{0}(-3;0) $.\\
			Vì acgument của $ z $ là góc tạo bởi $ \overrightarrow{OM} $ và tia $ Ox $ nên điểm biểu diễn số phức có acgument dương nhỏ nhất trùng với điểm $ M_{0} $, hay $ z=-3 $.
		}{
			\begin{tikzpicture}[line join=round, line cap=round,>=stealth]
			\tikzset{label style/.style={font=\footnotesize}}
			\pgfmathsetmacro\h{1}
			\pgfmathsetmacro\r{sqrt(3)}
			\pgfmathsetmacro\xmin{-6*\h}
			\pgfmathsetmacro\xmax{\h}
			\pgfmathsetmacro\ymin{-4*\h}
			\pgfmathsetmacro\ymax{\h}
			\clip(\xmin-0.01,\ymin-0.01)rectangle(\xmax+0.01,\ymax+0.01);
			\draw[->](\xmin,0)--(\xmax,0)node[below left]{$x$};
			\draw[->](0,\ymin)--(0,\ymax)node[below left]{$y$};
			\draw(0,0)node[above left]{\fontsize{12pt}{1pt}\selectfont $ O $};
			\tkzDefPoint(-3,-\r){I}
			\tkzDrawCircle[R](I,\r cm)
			\tkzDrawPoints[fill=black](I)
			\tkzLabelPoints[below](I)
			\node[above]at(-3,0){$-3$};
			\node[right]at(0,{-sqrt(3)}){$-\sqrt{3}$};
			\draw[dashed] (-3,0)--(-3,-\r)--(0,-\r);
			\end{tikzpicture}
		}
	}
\end{vd}
\begin{vd}%[Dương BùiĐức, Dự án (12-EX-1-DCHT)]%[2D4K5-3]
	Tìm các số nguyên dương $ n $ thỏa mãn $ z=\left( \dfrac{3-i\sqrt{3}}{\sqrt{3}-3i}\right)^{n} $ là số thực.\dapso{$ n=6k, 1\leq k\in\mathbb{Z} $}
	\loigiai{
		Ta có $ \dfrac{3-i\sqrt{3}}{\sqrt{3}-3i}=\dfrac{\sqrt{3}-i}{1-i\sqrt{3}}=\dfrac{(\sqrt{3}-i)(1+i\sqrt{3})}{4}=\dfrac{2\sqrt{3}+2i}{4}=\dfrac{\sqrt{3}}{2}+\dfrac{1}{2}i=\cos\dfrac{\pi}{6}+i\sin\dfrac{\pi}{6}$.\\
		Suy ra $ z=\left( \dfrac{3-i\sqrt{3}}{\sqrt{3}-3i}\right)^{n}=\cos\dfrac{n\pi}{6}+i\sin\dfrac{n\pi}{6} $.\\
		Do đó, để $ z $ là số thực thì $ \sin\dfrac{n\pi}{6}=0\Leftrightarrow \dfrac{n\pi}{6}=k\pi\Leftrightarrow n=6k $. Do $ n\in\mathbb{N}^{*} $ nên $ k\geq 1 $ ($ k\in\mathbb{Z} $).
	}
\end{vd}
\begin{vd}%[Dương BùiĐức, Dự án (12-EX-1-DCHT)]%[2D4K5-3]
	Cho số phức $ z_{1}=\left( \dfrac{\sqrt{3}-i}{1-i\sqrt{3}}\right)^{n} $ là số thực và số phức $ z_{2}=\left( \dfrac{5-i}{2-3i}\right)^{n+2} $ là số ảo. Hãy tìm số nguyên dương $ n $ nhỏ nhất.\dapso{$ n=12 $}
	\loigiai{
		Ta có $ \dfrac{\sqrt{3}-i}{1-i\sqrt{3}}=\dfrac{(\sqrt{3}-i)(1+i\sqrt{3})}{4}=\dfrac{2\sqrt{3}+2i}{4}=\dfrac{\sqrt{3}}{2}+\dfrac{1}{2}i=\cos\dfrac{\pi}{6}+i\sin\dfrac{\pi}{6}\Rightarrow z_{1}=\left( \dfrac{\sqrt{3}-i}{1-\sqrt{3}i}\right)^{n}=\cos\dfrac{n\pi}{6}+i\sin\dfrac{n\pi}{6} $.\\
		Do đó, để $ z $ là số thực thì $ \sin\dfrac{n\pi}{6}=0\Leftrightarrow \dfrac{n\pi}{6}=k\pi\Leftrightarrow n=6k $ ($ k\in\mathbb{N}^{*} $).\hfill(1)\\
		Mà $ \dfrac{5-i}{2-3i}=\dfrac{(5-i)(2+3i)}{13}=\dfrac{13+13i}{13}=1+i=\sqrt{2}\left( \cos\dfrac{\pi}{4}+i\sin\dfrac{\pi}{4}\right)\Rightarrow z_{2}=2^{\tfrac{n+2}{2}}\left( \cos\dfrac{(n+2)\pi}{4}+i\sin\dfrac{(n+2)\pi}{4}\right) $.\\
		Để $ z_{2} $ là số thuần ảo thì $ \cos\dfrac{(n+2)\pi}{4}=0\Leftrightarrow \dfrac{(n+2)\pi}{4}=\dfrac{\pi}{2}+k\pi\Leftrightarrow n=4k $ ($ k\in\mathbb{N}^{*} $).\hfill(2)\\
		Từ (1) và (2) suy ra số nguyên dương nhỏ nhất thỏa mãn yêu cầu bài toán là $ n=12 $.
	}
\end{vd}
\begin{vd}%[Dương BùiĐức, Dự án (12-EX-1-DCHT)]%[2D4K5-3]
	Cho số phức $ z $ thỏa mãn $ z^{2}-2z+4=0 $. Tìm số phức $ w=\left(\dfrac{1+\sqrt{3}-z}{2+z}\right)^{7} $.\\
	\dapso{$ w=\dfrac{\sqrt{2}}{2^{4}}\left(\cos\dfrac{\pm\pi}{12}+i\sin\dfrac{\pm\pi}{12}\right) $}
	\loigiai{
		Ta có $ z^{2}-2z+4=0\Leftrightarrow z=1\pm i\sqrt{3} $.
		\begin{itemize}
			\item Với $ z=1+i\sqrt{3} $, ta có
			\[
			z_{1}=\dfrac{1+\sqrt{3}-z}{2+z}=\dfrac{\sqrt{3}-i\sqrt{3}}{3+i\sqrt{3}}=\dfrac{1-i}{\sqrt{3}+i}=\dfrac{(1-i)(\sqrt{3}-i)}{4}=\dfrac{\sqrt{3}-1-i(1+\sqrt{3})}{4}.
			\]
			Gọi $ \varphi $ là acgument của số phức $ z_{1} $, khi đó $ \tan\varphi=\dfrac{1+\sqrt{3}}{1-\sqrt{3}}=\dfrac{\tan\dfrac{\pi}{4}+\tan\dfrac{\pi}{3}}{1-\tan\dfrac{\pi}{4}\tan\dfrac{\pi}{3}}=\tan\left(\dfrac{\pi}{4}+\dfrac{\pi}{3}\right)=\tan\dfrac{7\pi}{12} $. Suy ra $ \varphi=\dfrac{7\pi}{12} $. Từ đó ta có
			\[
			z_{1}=\dfrac{1}{4}\cdot 2\sqrt{2}\left(\cos\dfrac{7\pi}{12}+i\sin\dfrac{7\pi}{12}\right)=\dfrac{\sqrt{2}}{2}\left(\cos\dfrac{7\pi}{12}+i\sin\dfrac{7\pi}{12}\right)\Rightarrow w=\dfrac{\sqrt{2}}{2^{4}}\left(\cos\dfrac{49\pi}{12}+i\sin\dfrac{49\pi}{12}\right)=\dfrac{\sqrt{2}}{2^{4}}\left(\cos\dfrac{\pi}{12}+i\sin\dfrac{\pi}{12}\right)
			\]
			\item Với $ z=1-i\sqrt{3} $, ta có
			\[
			z_{2}=\dfrac{1+\sqrt{3}-z}{2+z}=\dfrac{\sqrt{3}+i\sqrt{3}}{3-i\sqrt{3}}=\dfrac{1+i}{\sqrt{3}-i}=\dfrac{(1+i)(\sqrt{3}+i)}{4}=\dfrac{\sqrt{3}-1+i(1+\sqrt{3})}{4}.
			\]
			Gọi $ \varphi $ là acgument của số phức $ z_{2} $, khi đó $ \tan\varphi=-\dfrac{1+\sqrt{3}}{1-\sqrt{3}}=-\dfrac{\tan\dfrac{\pi}{4}+\tan\dfrac{\pi}{3}}{1-\tan\dfrac{\pi}{4}\tan\dfrac{\pi}{3}}=-\tan\left(\dfrac{\pi}{4}+\dfrac{\pi}{3}\right)=\tan\left( -\dfrac{7\pi}{12}\right) $. Suy ra $ \varphi=-\dfrac{7\pi}{12} $. Từ đó ta có
			\[
			z_{2}=\dfrac{1}{4}\cdot \dfrac{\sqrt{2}}{2}\left(\cos\dfrac{-7\pi}{12}+i\sin\dfrac{-7\pi}{12}\right)\Rightarrow w=\dfrac{\sqrt{2}}{2^{4}}\left(\cos\dfrac{-49\pi}{12}+i\sin\dfrac{-49\pi}{12}\right)=\dfrac{\sqrt{2}}{2^{4}}\left(\cos\dfrac{-\pi}{12}+i\sin\dfrac{-\pi}{12}\right).
			\]
		\end{itemize}
		Vậy có $ 4 $ số phức thỏa mãn là $ w=\dfrac{\sqrt{2}}{2^{4}}\left(\cos\dfrac{\pm\pi}{12}+i\sin\dfrac{\pm\pi}{12}\right) $.
	}
\end{vd}
\begin{vd}%[Dương BùiĐức, Dự án (12-EX-1-DCHT)]%[2D4K5-3]
	Cho số phức $ z $ thỏa mãn $ 2|z|+\sqrt{3}iz=4-z $. Tìm số phức $ w=z^{2012}+\dfrac{1}{z^{2013}} $.\dapso{$ w=-\dfrac{3}{2}-\dfrac{\sqrt{3}}{2}i $}
	\loigiai{
		Gọi $ z=a+bi $ (với $ a,\ b\in\mathbb{R} $). Ta có
		\[
		2|z|+\sqrt{3}iz=4-z\Leftrightarrow 2\sqrt{a^{2}+b^{2}}+i\sqrt{3}(a+bi)=4-(a+bi)\Leftrightarrow (2\sqrt{a^{2}+b^{2}}-b\sqrt{3})+a\sqrt{3}i=(4-a)-bi.
		\]
		Do đó $ \heva{&2\sqrt{a^{2}+b^{2}}-b\sqrt{3}=4-a\\ &a\sqrt{3}=-b}\Leftrightarrow \heva{&2\sqrt{a^{2}+b^{2}}-b\sqrt{3}=4-a\\ &b=-a\sqrt{3}.} $\\
		Suy ra $ |a|+a=1\Leftrightarrow \hoac{&2a=1\ (\text{nếu }a\geq 0)\\ &0=1\ (\text{nếu }a\leq 0)}\Rightarrow a=\dfrac{1}{2}\ (\text{thỏa mãn }a\geq 0) $.\\
		Với $ a=\dfrac{1}{2}$ thì $ b=-\dfrac{\sqrt{3}}{2}\Rightarrow z=\dfrac{1}{2}-\dfrac{\sqrt{3}}{2}i=\cos\dfrac{\pi}{3}-i\sin \dfrac{\pi}{3} $ và $ \dfrac{1}{z}=\dfrac{1}{2}+\dfrac{\sqrt{3}}{2}i=\cos\dfrac{\pi}{3}+i\sin \dfrac{\pi}{3} $. Khi đó 
		\[
		w=\cos\dfrac{2012\pi}{3}-i\sin \dfrac{2012\pi}{3}+\cos\dfrac{2013\pi}{3}+i\sin \dfrac{2013\pi}{3}=\cos\dfrac{2\pi}{3}-i\sin \dfrac{2\pi}{3}+\cos\pi+i\sin \pi=-\dfrac{3}{2}-\dfrac{\sqrt{3}}{2}i.
		\]
		Vậy $ w=-\dfrac{3}{2}-\dfrac{\sqrt{3}}{2}i $.
	}
\end{vd}
\begin{vd}%[Dương BùiĐức, Dự án (12-EX-1-DCHT)]%[2D4B5-3]
	Tìm phần thực và phần ảo của các số phức sau
	\begin{enumerate}
		\item $ z=(\sqrt{3}+i)^{8} $\dapso{$ Re(z)=-\dfrac{1}{2^{9}},\ Im(z)=\dfrac{\sqrt{3}}{2^{9}}$}		
		\item $ z=\dfrac{(1+i)^{10}}{(\sqrt{3}+i)^{9}} $\dapso{$Re(z)=0,\ Im(z)=-2^{14} $}
	\end{enumerate}
	\loigiai{
		\begin{enumerate}	
			\item Ta có $ \sqrt{3}+i=\dfrac{1}{2}\left(\cos\dfrac{\pi}{3}+i\sin\dfrac{\pi}{3}\right)\Rightarrow z=\dfrac{1}{2^{8}}\left(\cos\dfrac{8\pi}{3}+i\sin\dfrac{8\pi}{3}\right) =\dfrac{1}{2^{8}}\left(-\dfrac{1}{2}+\dfrac{\sqrt{3}}{2}i\right)$.\\
			Vậy $ Re(z)=-\dfrac{1}{2^{9}},\ Im(z)=\dfrac{\sqrt{3}}{2^{9}}$.
			\item Ta có 
			\begin{align*}
				&(1+i)^{2}=2i\Rightarrow (1+i)^{10}=(2i)^{5}=32i,\\
				&\sqrt{3}+i=\dfrac{1}{2}\left( \cos\dfrac{\pi}{3}+i\sin\dfrac{\pi}{3}\right) \Rightarrow (\sqrt{3}+i)^{9}=\dfrac{1}{2^{9}}\left( \cos\dfrac{9\pi}{3}+i\sin\dfrac{9\pi}{3}\right)=-\dfrac{1}{2^{9}}.
			\end{align*}
			Do vậy $ z=\dfrac{(1+i)^{10}}{(\sqrt{3}+i)^{9}}=-2^{14}i\Rightarrow Re(z)=0,\ Im(z)=-2^{14} $.
		\end{enumerate}
	}	
\end{vd}
\begin{vd}%[Dương BùiĐức, Dự án (12-EX-1-DCHT)]%[2D4K5-3]
	Tìm phần thực và phần ảo của các số phức sau
	\begin{enumerate}
		\item $ w=z^{2014}+\dfrac{1}{z^{2014}} $ biết $ z+\dfrac{1}{z}=1 $\dapso{$ Re(z)=-1 $, $ Im(z)=0 $}
		\item $ z=\dfrac{(1+i)^{2012}}{(\sqrt{3}+i)^{2011}} $\dapso{$Re(z)=-2^{3016},\ Im(z)=2^{3016}\cdot \sqrt{3}$}
	\end{enumerate}
	\loigiai{
		\begin{enumerate}
			\item Ta có $ z+\dfrac{1}{z}=1\Leftrightarrow z^{2}-z+1=0\Leftrightarrow z=\dfrac{1}{2}\pm \dfrac{\sqrt{3}}{2}i=\cos\dfrac{\pi}{3}\pm i\sin\dfrac{\pi}{3} $.
			\begin{itemize}
				\item Với $ z=\cos\dfrac{\pi}{3}+ i\sin\dfrac{\pi}{3}\Rightarrow w=\cos\dfrac{2014\pi}{3}+i\sin\dfrac{2014\pi}{3}+\cos\dfrac{2014\pi}{3}-i\sin\dfrac{2014\pi}{3}=-1 $.\\
				Do đó $ Re(z)=-1 $, $ Im(z)=0 $.
				\item Với $ z=\cos\dfrac{\pi}{3}- i\sin\dfrac{\pi}{3}\Rightarrow w=\cos\dfrac{2014\pi}{3}-i\sin\dfrac{2014\pi}{3}+\cos\dfrac{2014\pi}{3}+i\sin\dfrac{2014\pi}{3}=-1 $.\\
				Do đó $ Re(z)=-1 $, $ Im(z)=0 $.
			\end{itemize}
			\item Ta có 
			\begin{align*}
				&(1+i)^{2}=2i\Rightarrow (1+i)^{2012}=(2i)^{1006}=-2^{1006},\\
				&\sqrt{3}+i=\dfrac{1}{2}\left( \cos\dfrac{\pi}{3}+i\sin\dfrac{\pi}{3}\right) \Rightarrow (\sqrt{3}+i)^{2011}=\dfrac{1}{2^{2011}}\left( \cos\dfrac{2011\pi}{3}+i\sin\dfrac{2011\pi}{3}\right)=\dfrac{1}{2^{2011}}\left( \dfrac{1}{2}+\dfrac{\sqrt{3}}{2}i\right).
			\end{align*}
			Do vậy $ z=\dfrac{(1+i)^{2012}}{(\sqrt{3}+i)^{2011}}=-2^{3017}\left( \dfrac{1}{2}-\dfrac{\sqrt{3}}{2}i\right)=-2^{3016}+2^{3016}\cdot \sqrt{3}i \Rightarrow Re(z)=-2^{3016},\ Im(z)=2^{3016}\cdot \sqrt{3} $.		
		\end{enumerate}
	}		
\end{vd}

\Closesolutionfile{ans}