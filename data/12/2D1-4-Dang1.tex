\setcounter{section}{3}
\section{Tiệm cận}
\subsection{Kiến thức sách giáo khoa cần cần nắm}
\subsubsection{Đường tiệm cận ngang}
Cho hàm số $y=f(x)$ xác định trên một khoảng vô hạn (là khoảng dạng $(a;+\infty)$, $(-\infty;b)$ hoặc $(-\infty;+\infty)$). Đường thẳng $y=y_0$ là đường tiệm cận ngang (hay tiệm cận ngang) của đồ thị hàm số $y=f(x)$ nếu ít nhất một trong các điều kiện sau được thỏa mãn.
 	$$\lim\limits_{x\to+\infty} f(x)=y_0,\lim\limits_{x\to-\infty} f(x)=y_0$$
\subsubsection{Đường tiệm cận đứng}
Đường thẳng $x=x_0$ được gọi là đường tiệm cận đứng (hay tiệm cận đứng) của đồ thị hàm số $y=f(x)$ nếu ít nhất một trong các điều kiện sau được thỏa mãn.
 $$\lim\limits_{x\to x_0^+} f(x)=+\infty,\lim\limits_{x\to x_0^-} f(x)=-\infty,\lim\limits_{x\to x_0^+} f(x)=-\infty,\lim\limits_{x\to x_0^-} f(x)=+\infty$$
\subsubsection{Dấu hiệu:}
+) Hàm phân thức mà nghiệm của mẫu không là nghiệm của tử có TCĐ.\\
+) Hàm phân thức mà bậc của tử $\leq$ bậc của mẫu có TCN.\\
+) Hàm căn thức dạng: $y=\sqrt{bt1}-\sqrt{bt2},y=\sqrt{bt1}-bt2,y=bt1-\sqrt{bt2}$ thường có TCN. (Dùng liên hợp).\\
% +) Hàm $y=a^x,(0<a\neq 1)$ có TCN $y=0$.\\
% +) Hàm số $y=\log_ax,(0<a\neq 1)$ có TCĐ $x=0$.
\subsubsection{Cách tìm:}
+) TCĐ: Tìm nghiệm của mẫu không là nghiệm của tử.\\
+) TCN: Tính 2 giới hạn: $\lim\limits_{x\to+\infty} y$ hoặc $\lim\limits_{x\to-\infty} y$.
\subsubsection{Nhận xét}
\begin{enumerate}[1.]
\item Nếu $x\to+\infty\Rightarrow x>0\Rightarrow\sqrt{x^2}=|x|=x$.
\item Nếu $x\to-\infty\Rightarrow x<0\Rightarrow\sqrt{x^2}=|x|=-x$.
\item Hàm số $y=\dfrac{ax+b}{cx+d}$ có tiệm cận đứng là $x=-\dfrac{d}{c}$, tiệm cận ngang là $y=\dfrac{a}{c}$.
\item Hàm số $y=\dfrac{ax^2+bx+c}{mx+n}=px+q+\dfrac{k}{mx+n}$ có TCĐ là $x=-\dfrac{n}{m}$, không có tiệm cận ngang.
\item TQ: $\dfrac{a_nx^n+a_{n-1}x^{n-1}+\cdots +a_1x+a_0}{b_mx^m+b_{m-1}x^{m-1}+\cdots +b_1x+b_0}:\hoac{&n\leq m\colon \text{TCĐ/TCN}\\&n>m\colon \text{TCĐ/0TCN}.}$
% \item Hàm số $y=f(x)=\sqrt{ax^2+bx+c}(a>0)$ có tiệm cận xiên là $y=\sqrt{a}\left|x+\dfrac{b}{2a}\right|$.
% \item Hàm số $y=f(x)=mx+n+p\sqrt{ax^2+bx+c}(a>0)$ có tiệm cận xiên là $y=mx+n+p\sqrt{a}\left|x+\dfrac{b}{2a}\right|$.
% \item Hàm số $y=\dfrac{mx+n}{\sqrt{ax^2+bx+c}}$ có tiệm cận ngang, có thể có tiệm cận đứng nếu $ax^2+bx+c=0$ có nghiệm.
\end{enumerate}
\subsection{Phân loại và phương pháp giải bài tập}
\begin{dang}{Lý thuyết về đường tiệm cận}
\end{dang}
\Opensolutionfile{ans}[ans/ans2D1-4-1]
\paragraph{Các ví dụ}
\begin{vd} %[2D1B4-4]
	Cho hàm số $y=f(x)$ có $\lim\limits_{x\to+\infty} f(x)=1$ và $\lim\limits_{x\to-\infty} f(x)=-1$. Khẳng định nào sau đây là \textbf{đúng}?
	\choice
	{\True Đồ thị hàm số đã cho có hai tiệm cận ngang là các đường thẳng $y=1$ và $y=-1$}
	{Đồ thị hàm số đã cho có hai tiệm cận ngang là các đường thẳng $x=1$ và $y=\dfrac{x+1}{4^x}$}
	{Đồ thị hàm số đã cho có đúng một tiệm cận ngang}
	{Đồ thị hàm số đã cho không có tiệm cận ngang}
	\loigiai{
		$\lim\limits_{x\to+\infty} f(x)=1$ nên đồ thị hàm số đã cho có tiệm cận ngang là đường thẳng $y=1$.\\
		$\lim\limits_{x\to-\infty} f(x)=-1$ nên đồ thị hàm số đã cho có tiệm cận ngang là đường thẳng $y=-1$.\\
		Vậy đồ thị hàm số đã cho có hai tiệm cận ngang là các đường thẳng $y=1$ và $y=-1$.}
\end{vd}	
\begin{vd} %[2D1B4-4]
	Cho hàm số $y=f(x)$ liên tục trên $\mathbb{R}$ thỏa mãn $\lim\limits_{x\to-\infty} f(x)=0$, $\lim\limits_{x\to+\infty} f(x)=1$. Tổng số đường tiệm cận đứng và đường tiệm cận ngang của đồ thị hàm số đã cho là
	\choice
	{\True $2$}
	{$1$}
	{$3$}
	{$0$}
	\loigiai{
		Do hàm số $y=f(x)$ liên tục trên $\mathbb{R}$ nên đồ thị hàm số không có đường tiệm cận đứng.\\
		Do $\lim\limits_{x\to-\infty} f(x)=0,\lim\limits_{x\to+\infty} f(x)=1$ nên $y=0$, $y=1$ là các đường tiệm cận ngang.}
\end{vd}	
		
\paragraph{Câu hỏi trắc nghiệm}
\begin{ex}%Câu 1. %[2D1B4-3]
	Cho hàm số $y=f(x)$ có $\lim\limits_{x\to+\infty} f(x)=3$ và $\lim\limits_{x\to-\infty} f(x)=3$. Khẳng định nào sau đây \textbf{đúng}?
	\choice
	{\True Đồ thị hàm số có đúng một tiệm cận ngang}
	{Đồ thị hàm số có hai tiệm cận ngang là các đường thẳng $y=-3$; $y=3$}
	{Đồ thị hàm số không có tiệm cận ngang}
	{Đồ thị hàm số có hai tiệm cận ngang là các đường thẳng $x=-3$; $x=3$}
	\loigiai{
		Ta có $\heva{&\lim\limits_{x\to+\infty} f(x)=3 \\ &\lim\limits_{x\to-\infty} f(x)=3}$ nên đồ thị hàm số có đúng một tiệm cận ngang là $y=3$.}
\end{ex}
\begin{ex}%Câu 2. %[2D1B4-3]
	Cho hàm số $y=f(x)$ có $\lim\limits_{x\to+\infty} f(x)=1$ và $\lim\limits_{x\to-\infty} f(x)=-1$. Khẳng định nào sau đây là \textbf{đúng}?
	\choice
	{Đồ thị hàm số đã cho có hai tiệm cận ngang là các đường thẳng có phương trình $x=1$ và $x=-1$}
	{Đồ thị hàm số đã cho có đúng một tiệm cận ngang}
	{Đồ thị hàm số đã cho không có tiệm cận ngang}
	{\True Đồ thị hàm số đã cho có hai tiệm cận ngang là các đường thẳng có phương trình $y=1$ và $y=-1$}
	\loigiai{
	\begin{enumerate}[+]
	\item $\lim\limits_{x\to+\infty} f(x)=1$ nên $y=1$ là tiệm cận ngang của đồ thị hàm số.
	\item $\lim\limits_{x\to+\infty} f(x)=-1$ nên $y=-1$ là tiệm cận ngang của đồ thị hàm số.
	\end{enumerate}
		}
\end{ex}
\begin{ex}%Câu 3. %[2D1K4-3]
	Cho hàm số $f(x)$ xác định trên tập $\mathscr{D}=[-2018;2018]\setminus\{-2017;2017\}$ thỏa
	$$\lim\limits_{x\to-{2017}^-} f(x)=-\infty, \lim\limits_{x\to-{2017}^+} f(x)=-\infty, \lim\limits_{x\to{2017}^-} f(x)=+\infty, \lim\limits_{x\to{2017}^+} f(x)=+\infty.$$
	Tìm khẳng định đúng?
	\choice
	{Đồ thị hàm số đã cho không có đường tiệm cận đứng}
	{Đồ thị hàm số đã cho có hai tiệm cận đứng là $x=-2018;x=2018$}
	{\True Đồ thị hàm số đã cho có hai tiệm cận đứng là $x=-2017;x=2017$}
	{Đồ thị hàm số đã cho có hai tiệm cận đứng là $x=-2017;x=2017;x=-2018;x=2018$.}
\end{ex}
\begin{ex}%Câu 4. %[2D1B4-3]
	Cho hàm số $y=f(x)$ xác định trên khoảng $(0;+\infty)$ và thỏa mãn $\lim\limits_{x\to+\infty} f(x)=1$. Hãy chọn mệnh đề \textbf{đúng} trong các mệnh đề sau: 
	\choice
	{Đường thẳng $x=1$ là tiệm cận ngang của đồ thị hàm số $y=f(x)$}
	{Đường thẳng $x=1$ là tiệm cận đứng của đồ thị hàm số $y=f(x)$}
	{\True Đường thẳng $y=1$ là tiệm cận ngang của đồ thị hàm số $y=f(x)$}
	{Đường thẳng $y=1$ là tiệm cận đứng của đồ thị hàm số $y=f(x)$}
	\loigiai{
		Dựa vào định nghĩa đường tiệm cận, ta chọn đáp C.}
\end{ex}
\begin{ex}%Câu 5. %[2D1B4-3]
	Cho hàm số $y=f(x)$ có $\lim\limits_{x\to+\infty} f(x)=1$ và $\lim\limits_{x\to-\infty} f(x)=-1$. Khẳng định nào sau đây là khẳng định đúng?
	\choice
	{Đồ thị hàm số đã cho có hai tiệm cận ngang là $x=1$ và $x=-1$}
	{Đồ thị hàm số đã cho có đúng một tiệm cận ngang}
	{Đồ thị hàm số đã cho không có tiệm cận ngang}
	{\True Đồ thị hàm số đã cho có hai đường tiệm cận ngang là $y=1$ và $y=-1$}
	\loigiai{
		Hàm số $y=f(x)$ có $\lim\limits_{x\to+\infty} f(x)=1$ và $\lim\limits_{x\to-\infty} f(x)=-1$ suy ra đồ thị hàm số đã cho có hai đường tiệm cận ngang là $y=1$ và $y=-1$.}
\end{ex}
\begin{ex}%Câu 6. %[2D1B4-3]
	Cho hàm số $y=f(x)$ liên tục trên $\mathbb{R}$ thỏa mãn $\lim\limits_{x\to-\infty} f(x)=0$, $\lim\limits_{x\to+\infty} f(x)=1$. Tổng số đường tiệm cận đứng và tiệm cận ngang của đồ thị hàm số đã cho là
	\choice
	{\True $2$}
	{$1$}
	{$3$}
	{$0$}
	\loigiai{
		$\lim\limits_{x\to-\infty} f(x)=0$, $\lim\limits_{x\to+\infty} f(x)=1\Rightarrow$ ĐTHS có 2 đường TCN là trục Ox và đường thẳng $y=1$.\\
		Vì hàm số liên tục trên $\mathbb{R}$ nên hàm số không có đường TCĐ.\\
		Vậy có tất cả 2 đường tiệm cận.}
\end{ex}
\begin{ex}%Câu 7. %[2D1K4-1]
	Cho hàmsố $y = f(x)$ có đồ thị như hình bên
	\begin{center}
	\begin{tikzpicture}[scale=0.7,font=\footnotesize,line join = round, line cap = round,>=stealth]
	\def\hsf{(-2*(\x)-5)/((\x)+2)}
	\def\hsg{(4*(\x)-3)/(4*(\x)-4)}
	\def\hsh{(-1*(\x)^3+2)/((\x)^2+(\x)-2)}
	\draw[->] (-5,0)--(0,0)node[below left]{$O$}--(6,0)node[below]{$x$};
	\draw[->] (0,-4.5)--(0,5)node[left]{$y$};
	\draw[dashed] (-5,1)--(6,1) (-2,-4.5)--(-2,5) (1,-4.5)--(1,5) (-5,-2)--(6,-2);
	\draw[samples=100,domain=-5:-2.15,smooth] plot (\x, {\hsf});
	\draw[samples=100,domain=1.1:5,smooth] plot (\x, {\hsg});
	\draw[samples=100,domain=-1.5:0.9,smooth] plot (\x, {\hsh});
	\fill (0,-2)node[below left]{$-2$} circle (1.5pt);\fill (0,1) node[above left]{$1$} circle (1.5pt);\fill (-2,0)node[above right]{$-2$} circle (1.5pt);\fill (1,0) node[above left]{$1$} circle (1.5pt);\fill (0,-1) node[right]{$-1$} circle (1.5pt);
	\foreach \i/\j in {-2/0,0/0,1/0,0/-2,0/1} \draw (\i,\j) circle(1pt);
	\end{tikzpicture}
	\end{center}
	Các khẳng định sau:\\
	$(I)\lim\limits_{x\to 1^-} f(x)=-\infty$;\\
	$(II)\lim\limits_{x\to-2^+} f(x)=-\infty$;\\
	$(III)\lim\limits_{x\to+\infty} f(x)=-\infty;(IV)\lim\limits_{x\to-2^-} f(x)=+\infty$.\\
	Số khẳng định \textbf{đúng} là
	\choice
	{$4$}
	{\True $3$}
	{$2$}
	{$1$}
	\loigiai{
		$(I)\lim\limits_{x\to 1^-} f(x)=-\infty$ đúng; $(II)\lim\limits_{x\to-2^+} f(x)=-\infty$ đúng.\\
		$(III)\lim\limits_{x\to+\infty} f(x)=-\infty$ \textbf{sai}; $(IV)\lim\limits_{x\to-2^-} f(x)=+\infty$ đúng.}
\end{ex}
\begin{ex}%Câu 8. %[2D1B4-3]
	Cho hàm số $y=f(x)$ có tập xác định là $\mathscr{D}=(0;+\infty)$ và $\lim\limits_{x\to 0^+} y=-\infty$, $\lim\limits_{x\to+\infty} y=+\infty$. Mệnh đề nào sau đây đúng?
	\choice
	{Đồ thị hàm số $y=f(x)$ không có tiệm cận đứng và có tiệm cận ngang}
	{Đồ thị hàm số $y=f(x)$ có tiệm cận đứng và có tiệm cận ngang}
	{\True Đồ thị hàm số $y=f(x)$ có tiệm cận đứng và không có tiệm cận ngang}
	{Đồ thị hàm số $y=f(x)$ không có tiệm cận đứng và không có tiệm cận ngang}
	\loigiai{
		Do $x=0^+$ là một đầu mút của tập xác định và $\lim\limits_{x\to 0^+} y=-\infty$ nên đường thẳng $x=0$ (hay là trục $Oy$) là tiệm cận đứng của đồ thị hàm số.\\
		Với $\mathscr{D}=(0;+\infty)$, ta kiểm tra được giới hạn của hàm số tại $+\infty$ (không có giới hạn tại $-\infty$). Theo giả thiết, $\lim\limits_{x\to+\infty} y=+\infty$ nên đồ thị hàm số không có tiệm cận ngang.}
\end{ex}
\begin{ex}%Câu 9. %[2D1B4-3]
	Cho hàm số $y=f(x)$ có đồ thị là đường cong $(C)$ và các giới hạn $\lim\limits_{x\to 2^+} f(x)=1$; $\lim\limits_{x\to 2^-} f(x)=1$; $\lim\limits_{x\to-\infty} f(x)=2$; $\lim\limits_{x\to+\infty} f(x)=2$. Hỏi mệnh đề nào sau đây đúng?
	\choice
	{\True Đường thẳng $y=2$ là tiệm cận ngang của $(C)$}
	{Đường thẳng $y=1$ là tiệm cận ngang của $(C)$}
	{Đường thẳng $x=2$ là tiệm cận ngang của $(C)$}
	{Đường thẳng $x=2$ là tiệm cận đứng của $(C)$}
	\loigiai{
		Ta có: $\heva{&\lim\limits_{x\to-\infty} f(x)=2\\&\lim\limits_{x\to+\infty} f(x)=2}\Rightarrow$ đường thẳng $y=2$ là tiệm cận ngang của $(C)$.}
\end{ex}
\begin{ex}%Câu 10. %[2D1K4-1]
	Đồ thị hàm số $y=\dfrac{x-1}{|x|+1}$ có bao nhiêu đường tiệm cận?
	\choice
	{$0$}
	{$1$}
	{\True $2$}
	{$3$}
	\loigiai{
		Tập xác định $\mathscr{D}=\mathbb{R}$.\\
		Ta có $\lim\limits_{x\to+\infty} y=1$ và $\lim\limits_{x\to-\infty} y=-1$ nên đồ thị hàm số có hai đường tiệm cận ngang.}
\end{ex}
\Closesolutionfile{ans}