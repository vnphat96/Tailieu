\begin{ex}%[Câu 1]%[2H5H2-7]
	Trong KG $Oxyz$, cho hai đường thẳng $d_1\colon \dfrac{x-1}{2}=\dfrac{y-2}{-2}=\dfrac{z+1}{-1}$,  $d_2\colon \heva{&x=t \\&y=0 \\&z=-t}$. Mặt phẳng $(P)$ qua $d_1$ tạo với $d_2$ một góc $45^\circ$ và nhận véc-tơ $\vec{n}=(1;b;c)$ làm một véc-tơ pháp tuyến. Xác định tích $b\cdot c$.
	\choice
	{$-4$ hoặc $0$}
	{$4$ hoặc $0$}
	{\True $-4$}
	{$4$}
	\loigiai{
		Ta có véc-tơ chỉ phương của $d_1$, $d_2$ lần lượt là $\overrightarrow{u}_1=(2;-2;-1)$ và $\overrightarrow{u}_2=(1;0;-1)$.\\
		Mặt phẳng $(P)$ qua $d_1$ nên $ \overrightarrow{n}\cdot\overrightarrow{u}_1=0\Leftrightarrow 2-2b-c=0$. \qquad (1)\\
		Ta có \begin{eqnarray*}
			&&\sin (d_2,(P))=\dfrac{\left|\overrightarrow{u}_2\cdot\overrightarrow{n}\right|}{\left|\overrightarrow{u}_2\right|\cdot\left|\overrightarrow{n}\right|}=\sin 45^\circ \\
			&\Leftrightarrow&\dfrac{\left| 1-c\right|}{\sqrt{b^2+c^2+1}\cdot\sqrt{2}}=\dfrac{\sqrt{2}}{2}\\
			&\Leftrightarrow&\left| 1-c\right|=\sqrt{b^2+c^2+1}\\
			&\Leftrightarrow&b^2+2c=0.\qquad (2)
		\end{eqnarray*}
		Từ $(1)$ và $(2)$ suy ra $\heva{&b=2 \\&c=-2}\Rightarrow b\cdot c=-4$.}
\end{ex}
\begin{ex}%[Câu 3]%[2H5H2-7]
	Trong KG $Oxyz$, cho đường thẳng $d\colon \heva{&x=0 \\&y=3-t \\&z=t}$. Gọi $(P)$ là mặt phẳng chứa đường thẳng $d$ và tạo với mặt phẳng $(Oxy)$ một góc $45^\circ$. Điểm nào sau đây thuộc mặt phẳng $(P)$?
	\choice
	{\True $M(3;2;1)$}
	{$N(3;2;-1)$}
	{$P(3;-1;2)$}
	{$M(3;-1;-2)$}
	\loigiai{
		Ta viết PTĐT $d\colon \heva{&x=0 \\&y+z-3=0.}$\\
		Mặt phẳng $(P)$ chứa đường thẳng $d$ nên có dạng $mx+n(y+z-3)=0$, $m^2+n^2\ne 0$ hay $mx+ny+nz-3n=0$ nên $(P)$ có một véc-tơ pháp tuyến là $\overrightarrow{n_P}=(m;n;n)$.\\
		Mặt phẳng $(Oxy)$ có một véc-tơ pháp tuyến là $\overrightarrow{k}=(0;0;1)$.\\
		Ta có 
		\begin{eqnarray*}
			&&\cos ((P);(Oxy))=\left| \cos (\overrightarrow{n_P};\overrightarrow{k})\right|\\
			&\Leftrightarrow&\cos 45^\circ =\dfrac{\left| \overrightarrow{n_P}.\overrightarrow{k}\right|}{\left| \overrightarrow{n_P}\right|.\left| \overrightarrow{k}\right|}\\
			&\Leftrightarrow&\dfrac{1}{\sqrt{2}}=\dfrac{\left| n\right|}{\sqrt{m^2+n^2+n^2}}\\
			&\Leftrightarrow&\sqrt{m^2+2n^2}=\sqrt{2}\left| n\right|\\
			&\Leftrightarrow&m^2=0\Leftrightarrow m=0.
		\end{eqnarray*}
		Chọn $n=1\Rightarrow (P)\colon y+z-3=0$.\\
		Do đó $M(3;2;1)\in (P)$.\\
		\textbf{Bình luận:} Đối với những bài toán viết phương trình mặt phẳng chứa đường thẳng cho trước ta nên sử dụng khái niệm chùm mặt phẳng như sau: Mặt phẳng $(\alpha)$ qua giao tuyến của hai mặt phẳng $(P)\colon a_1x+b_1y+c_1z+d_1=0$ và $(Q)\colon a_2x+b_2y+c_2z+d_2=0$ có phương trình dạng $m(a_1x+b_1y+c_1z+d_1)+n(a_2x+b_2y+c_2z+d_2)=0$, $m^2+n^2\ne 0$.}
\end{ex}
\begin{ex}%[Câu 4]%[2H5H2-7]
	Trong KG $Oxyz$, cho tam giác $ABC$ vuông tại $A$, $\widehat{ABC}=30^\circ$, $BC=3\sqrt{2}$, đường thẳng $BC$ có phương trình $\dfrac{x-4}{1}=\dfrac{y-5}{1}=\dfrac{z+7}{-4}$, đường thẳng $AB$ nằm trong mặt phẳng $(\alpha)\colon x+z-3=0$. Biết đỉnh $C$ có cao độ âm. Tính hoành độ đỉnh $A$.
		\choice
		{$\dfrac{3}{2}$}
		{$3$}
		{\True $\dfrac{9}{2}$}
		{$\dfrac{5}{2}$}
		\loigiai{
			Vì $C\in BC$ nên $C(4+t;5+t;-7-4t)$.\\
			$BC$ có véc tơ chỉ phương $\overrightarrow{u}=(1;1;-4)$. Mặt phẳng $(\alpha)$ có véc-tơ pháp tuyến $\overrightarrow{n}=(1;0;1)$.\\
			Gọi $\varphi $ là góc giữa $BC$ và $(\alpha)$. Ta có $\sin \varphi =\left| \cos (\overrightarrow{u};\overrightarrow{n})\right|=\dfrac{1}{2}\Rightarrow \varphi =30^\circ$. Tức là $A$ là hình chiếu của $C$ lên $(\alpha)$.\\
			Vậy 
			\begin{eqnarray*}
				&&\dfrac{3\sqrt{2}}{2}=CA=\mathrm{d}(C;(\alpha))=\dfrac{\left| 4+t-7-4t-3\right|}{\sqrt{2}}\\
				&\Leftrightarrow&\hoac{&t=-1 \\&t=-3}\\
				&\Leftrightarrow&\hoac{&C(3;4;-3) \\&C(1;2;5).}
			\end{eqnarray*}
			Mà $C$ có cao độ âm, suy ra $C(3;4;-3)$.\\
			Lúc này $AC$ qua $C(3;4;-3)$ và có véc-tơ chỉ phương $\overrightarrow{n}=(1;0;1)$.\\
			Phương trình $AC$ là $\heva{&x=3+t\\&y=4\\&z=-3+t}$. Vì $A\in AC$ nên $A(3+t;4;-3+t)$.\\
			Mặt khác $A$ nằm trong mặt phẳng $(\alpha)\colon x+z-3=0\Rightarrow t=\dfrac{3}{2}$.\\ 
			Do đó, hoành độ đỉnh $A$ là $ x_{A}=\dfrac{9}{2}$.}
	\end{ex}
	\begin{ex}%[Câu 5]%[2H5H2-7]
		Trong KG $Oxyz$, mặt phẳng nào dưới đây đi qua $A(2; 1; - 1)$ tạo với trục $Oz$ một góc $30^\circ $?
		\choice
		{\True $\sqrt{2}(x-2)+(y-1)-(z-2)-3=0$}
		{$(x-2)+\sqrt{2}(y-1)-(z+1)-2=0$}
		{$2(x-2)+(y-1)-(z-2)=0$}
		{$2(x-2)+(y-1)-(z-1)-2=0$}
		\loigiai{
			Gọi phương trình mặt phẳng $(\alpha)$ có dạng $A(x-2)+B(y-1)+C(z+1)=0$, $\overrightarrow{n}=(A;B;C)$ là véc-tơ pháp tuyến.\\
		Ta có $Oz$ có véc-tơ chỉ phương là $\overrightarrow{k}=(0;0;1)$.\\
			Áp dụng công thức 
			\begin{eqnarray*}
				&&\sin ((\alpha),Oz)=\dfrac{\left| \overrightarrow{n}\cdot \overrightarrow{k}\right|}{\overrightarrow{\left| n\right|}\cdot \overrightarrow{\left| k\right|}}=\sin 30^\circ\\
				&\Leftrightarrow&\dfrac{|A\cdot 0+B\cdot 0+C\cdot 1|}{\sqrt{A^2+B^2+C^2}\cdot \sqrt{0^2+0^2+1^2}}=\dfrac{1}{2}\\
				&\Leftrightarrow&\dfrac{|C|}{\sqrt{A^2+B^2+C^2}}=\dfrac{1}{2}\\
				&\Leftrightarrow&3C^2=A^2+B^2.\qquad (1)
			\end{eqnarray*}
			Chọn $A=\sqrt{2}$, $B=1$, $C=-1$ thỏa mãn $(1)$. Khi đó $(\alpha)\colon \sqrt{2}(x-2)+(y-1)-(z+1)=0$ hay $(\alpha)\colon \sqrt{2}(x-2)+(y-1)-(z-2)-3=0$.
			}
	\end{ex}
	\begin{ex}%[Câu 6]%[2H5H2-7]
		Cho mặt phẳng $(\alpha)\colon 3x-2y+2z-5=0$ và điểm $A(1; - 2; 2)$. Có bao nhiêu mặt phẳng đi qua $A$ và tạo với mặt phẳng $(\alpha)$ một góc $45^\circ$.
		\choice
		{\True Vô số}
		{$1$}
		{$2$}
		{$4$}
		\loigiai{
			Gọi $\overrightarrow{n_{\beta}}=(a;b;c)$ là véc-tơ pháp tuyến của mặt phẳng $(\beta)$ cần lập. Ta có
			\begin{eqnarray*}
				&&\cos((\alpha),(\beta))=\left| \cos(\overrightarrow{n_{\alpha}},\overrightarrow{n_{\beta}})\right|\\
				&\Leftrightarrow&\dfrac{\left| \overrightarrow{n_{\alpha}}\cdot \overrightarrow{n_{\beta}}\right|}{\left| \overrightarrow{n_{\alpha}}\right|\cdot \left| \overrightarrow{n_{\beta}}\right|}=\dfrac{\left| 3\cdot a-2\cdot b+2\cdot c\right|}{\sqrt{3^2+(-2)^2+2^2}\cdot \sqrt{a^2+b^2+c^2}}=\dfrac{\sqrt{2}}{2}\\
				&\Leftrightarrow&2(3a-2b+2c)^2=17(a^2+b^2+c^2)\\
				&\Leftrightarrow&2a^2-9b^2-9c^2-24ab-16bc+24ac=0.
			\end{eqnarray*}
			 Phương trình trên có vô số nghiệm. Nên có vô số véc-tơ $\overrightarrow{n_{\beta}}=(a;b;c)$ là véc-tơ pháp tuyến của $(\beta)$.\\
			 Suy ra có vô số mặt phẳng $(\beta)$ thỏa mãn điều kiện bài toán.\\
			}
	\end{ex}
\Closesolutionfile{ans}
% \indapan{7}{ans/C5B4CD6-D1}
\TNSA
\Opensolutionfile{ans}[ans/C5B4CD6-D1-KQ]
\begin{ex}%[Câu 7]%[2H5H2-7]
	Số các mặt phẳng $(\alpha)$ chứa đường thẳng $d\colon\dfrac{x}{1}=\dfrac{y}{-1}=\dfrac{z}{-3}$ và tạo với mặt phẳng $(P)\colon 2x-z+1=0$ góc $45^\circ $ bằng
\shortans{$2$}
\loigiai{
Đường thẳng $d$ đi qua điểm $O(0;0;0)$ có véc-tơ chỉ phương $\overrightarrow{u}=(1;-1;-3)$.\\
Ta có $(\alpha)$ qua $O$ có véc-tơ pháp tuyến $\overrightarrow{n}=(a;b;c)$ có dạng $ax+by+cz=0$.\\
Vì $\overrightarrow{n}\perp \overrightarrow{u}$ nên $\overrightarrow{n}\cdot \overrightarrow{u}=0$. Do đó $ a-b-3c=0$.\\
Mặt phẳng $(P)\colon 2x-z+1=0$ có véc-tơ pháp tuyến $\overrightarrow{k}=(2;0;-1)$.\\
Ta có 
\begin{eqnarray*}
	&&\cos 45^\circ =\dfrac{\left| \overrightarrow{n}\cdot \overrightarrow{k}\right|}{\left|\overrightarrow{n}\right|\cdot\left|\overrightarrow{k}\right|}\\
	&\Leftrightarrow&\dfrac{\left| 2a-c\right|}{\sqrt{5(a^2+b^2+c^2)}}=\dfrac{\sqrt{2}}{2}\\
	&\Leftrightarrow&10(a^2+b^2+c^2)=(4a-2c)^2\\
	&\Leftrightarrow&10(b^2+6bc+9c^2+b^2+c^2)=(4b+12c-2c)^2\\
	&\Leftrightarrow&10(2b^2+6bc+10c^2)=(4b+10c)^2\\
	&\Leftrightarrow&4b^2-20bc=0\\
	&\Leftrightarrow&\hoac{&b=0 \\&b=5c.}
\end{eqnarray*}
Xét 
\begin{itemize}
	\item $b=0\Rightarrow a=3c$ nên $(\alpha)\colon x+3z=0$.
	\item $b=5c$, chọn $c=1\Rightarrow b=5$, $a=8$ nên $(\alpha)\colon 8x+5y+z=0$.
\end{itemize}
}
\end{ex}
\begin{ex}%[Câu 9]%[2H5H2-7]
	Trong KG $Oxyz$, cho điểm $A(3;-1;0)$ và đường thẳng $d\colon \dfrac{x-2}{-1}=\dfrac{y+1}{2}=\dfrac{z-1}{1}$. Phương trình mặt phẳng $(\alpha)$ chứa $d$ sao cho khoảng cách từ $A$ đến $(\alpha)$ lớn nhất có dạng $ax+by+cz=0$. Khi đó $\dfrac{a}{b}$ bằng
\shortans{$1$}
\loigiai{
Gọi $H$ là hình chiếu của $A$ lên $d$.\\
Khi đó $H(2-t;-1+2t;1+t)\Rightarrow \overrightarrow{AH}=(-1-t;2t;1+t)$.\\
Do $AH\perp d$ nên $ -(-1-t)+2\cdot 2t+1+t=0\Leftrightarrow t=-\dfrac{1}{3}$. Khi đó $\overrightarrow{AH}=\left(-\dfrac{2}{3};-\dfrac{2}{3};\dfrac{2}{3}\right)$.\\
Mặt phẳng $(\alpha)$ chứa $d$ sao cho khoảng cách từ $A$ đến $(\alpha)$ lớn nhất khi $AH\perp (\alpha)$.\\
Do đó $(\alpha)$ có véc-tơ pháp tuyến là $\overrightarrow{n}=(1;1;-1)$.\\
Vậy $(\alpha)\colon 1(x-2)+1(y+1)-1(z-1)=0\Leftrightarrow x+y-z=0$.\\ Do đó $a=1$, $b=1$, $c=-1$ và $\dfrac{a}{b}=1$.}
\end{ex}
\begin{ex}%[Câu 10]%[2H5H1-6]
	Trong KG $Oxyz$, cho hai mặt phẳng $(P)\colon x+2y-2z+1=0,$ $(Q)\colon x+my+(m-1)z+2024=0$. Khi hai mặt phẳng $(P)$, $(Q)$ tạo với nhau một góc nhỏ nhất thì giá trị của $m$ bằng bao nhiêu?
\shortans{$0{,}5$}
\loigiai{
Gọi $\varphi $ là góc giữa hai mặt phẳng $(P)$ và $(Q)$.\\
Khi đó
\begin{eqnarray*}
	&&\cos \varphi =\dfrac{\left| 1\cdot 1+2\cdot m-2\cdot (m-1)\right|}{\sqrt{1^2+2^2+(-2)^2}\cdot \sqrt{1^2+m^2+(m-1)^2}}\\
	&\Leftrightarrow&\cos \varphi =\dfrac{3}{3\sqrt{2m^2-2m+2}}=\dfrac{1}{\sqrt{2(m-\dfrac{1}{2})^2+\dfrac{3}{2}}}\\
	&\Leftrightarrow&\cos \varphi \le \dfrac{1}{\sqrt{\dfrac{3}{2}}}.
\end{eqnarray*}
 Góc $\varphi $ nhỏ nhất khi và chỉ khi $\cos \varphi $ lớn nhất $\Leftrightarrow m=\dfrac{1}{2}=0{,}5$.}
\end{ex}
\begin{ex}%[Câu 11]%[2H5H1-6]
	Cho hai điểm $A(1;-1;1);B(2;-2;4)$. Có bao nhiêu mặt phẳng chứa $A$, $B$ và tạo với mặt phẳng $(\alpha)\colon x-2y+z-7=0$ một góc $60^\circ $?
\shortans{$2$}
\loigiai{
Ta có $\overrightarrow{AB}=(1;-1;3)$, $\overrightarrow{n_{\alpha}}=(1;-2;1)$.
Gọi $\overrightarrow{n_{\beta}}=(a;b;c)$ là véc-tơ pháp tuyến của mặt phẳng $(\beta)$ cần lập. Ta có
\begin{eqnarray*}
	&&\cos((\alpha),(\beta))=\left| \cos(\overrightarrow{n_{\alpha}},\overrightarrow{n_{\beta}})\right|=\dfrac{\left| \overrightarrow{n_{\alpha}}\cdot \overrightarrow{n_{\beta}}\right|}{\left| \overrightarrow{n_{\alpha}}\right|\cdot \left| \overrightarrow{n_{\beta}}\right|}\\
	&\Leftrightarrow&\dfrac{\left| 1\cdot a-2\cdot b+1\cdot c\right|}{\sqrt{1^2+(-2)^2+1^2}\cdot \sqrt{a^2+b^2+c^2}}=\dfrac{1}{2}\\
	&\Leftrightarrow&2(a-2b+c)^2=3(a^2+b^2+c^2).\qquad (1)
\end{eqnarray*}
Mặt khác vì mặt phẳng $(\beta)$ chứa $A$, $B$ nên 
$$\overrightarrow{n_{\beta}}\cdot \overrightarrow{AB}=0\Leftrightarrow a-b+3c=0\Leftrightarrow a=b-3c.$$
Thế vào $(1)$ ta được $2b^2-13bc+11c^2=0$ \qquad $(2)$.\\
Phương trình $(2)$ có $2$ nghiệm phân biệt. Suy ra có $2$ véc-tơ $\overrightarrow{n_{\beta}}=(a;b;c)$ thỏa mãn.\\
Suy ra có $2$ mặt phẳng.}
\end{ex}
\begin{ex}%[Câu 12]%[2H5V1-6]
	Trong KG $Oxyz$, cho hai điểm $A(3;0;1)$, $B(6;-2;1)$. Phương trình mặt phẳng $(P)$ đi qua $A$, $B$ và tạo với mặt phẳng $(Oyz)$ một góc $\alpha $ thỏa mãn $\cos \alpha =\dfrac{2}{7}$ có dạng $ax+by+cz+d=0$ với $d\neq 0$. Khi đó $\dfrac{d}{a}$ bằng
\shortans{$-6$}
\loigiai{
Giả sử $(P)$ có véc-tơ pháp tuyến $\overrightarrow{n_1}=(a;b;c)$, $(P)$ có véc-tơ chỉ phương $\overrightarrow{AB}=(3;-2;0)$. \\
Suy ra \[ \overrightarrow{n_1}\perp \overrightarrow{AB}\Rightarrow \overrightarrow{n_1}\cdot \overrightarrow{AB}=0\Leftrightarrow 3a+b(-2)+0\cdot c=0\Rightarrow 3a-2b=0\Rightarrow a=\dfrac{2}{3}b.\qquad (1)\]
$(Oyz)$ có phương trình $x=0$ nên có véc-tơ pháp tuyến $\overrightarrow{n_2}=(1;0;0)$. Mà 
\begin{eqnarray*}
	&&\cos \alpha =\dfrac{2}{7}\\
	&\Leftrightarrow&\dfrac{\left| \overrightarrow{n_1}\cdot \overrightarrow{n_2}\right|}{\left| \overrightarrow{n_1}\right|\cdot \left| \overrightarrow{n_2}\right|}=\dfrac{2}{7}\\
	&\Leftrightarrow&\dfrac{\left| a\cdot 1+b\cdot 0+c\cdot 0\right|}{\sqrt{a^2+b^2+c^2}\cdot \sqrt{1^2+0^2+0^2}}=\dfrac{2}{7}\\
	&\Leftrightarrow&\dfrac{\left| a\right|}{\sqrt{a^2+b^2+c^2}}=\dfrac{2}{7}\\
	&\Leftrightarrow&7\left| a\right|=2\sqrt{a^2+b^2+c^2}\\
	&\Leftrightarrow&45a^2-4b^2-4c^2=0.\qquad (2)
\end{eqnarray*}
Thay $(1)$ vào $(2)$ ta được $4b^2-c^2=0$.\\
Chọn $c=2$ ta có $4b^2-2^2=0\Rightarrow \hoac{&b=1 \\&b=-1}\Rightarrow \hoac{&a=\dfrac{2}{3} \\&a=-\dfrac{2}{3}}$
\begin{itemize}
	\item $a=\dfrac{2}{3}$ thì $\overrightarrow{n}=\left(\dfrac{2}{3};1;2\right)$ hay $\overrightarrow{n}=(2;3;6)$. Do đó $(P)\colon 2x+3y-6z=0$.
	\item $a=-\dfrac{2}{3}$ thì $\overrightarrow{n}=\left(-\dfrac{2}{3};-1;2\right)$ hay $\overrightarrow{n}=(2;3;-6)$. Do đó $(P)\colon 2x+3y+6z-12=0$.
\end{itemize}
Vậy $(P)\colon 2x+3y-6z=0$ hoặc $2x+3y+6z-12=0$.\\
Vì $(P)$ có dạng $ax+by+cz+d=0$, $d\neq 0$ nên $(P)\colon 2x+3y+6z-12=0$ và $a=2$, $d=-12$. Do đó $\dfrac{d}{a}=-6$.}
\end{ex}
\begin{ex}%[Câu 13]%[2H5V1-6]
	Trong KG $Oxyz$, biết mặt phẳng $(P)\colon ax+by+cz+d=0$ với $c<0$ đi qua hai điểm $A(0;1;0)$, $B(1;0;0)$ và tạo với mặt phẳng $(yOz)$ một góc $60^\circ $. Tính giá trị $a+b+c$. (Kết quả lấy đến hàng phần chục)
\shortans{$0{,}6$}
\loigiai{
Ta có $A, B\in (P)$ nên $\heva{&b+d=0 \\&a+d=0.}$ \\
Suy ra $(P)$ có dạng $ax+ay+cz-a=0$ có véc-tơ pháp tuyến là $\overrightarrow{n}=(a;a;c)$.\\
Mặt phẳng $(yOz)$ có véc-tơ pháp tuyến là $\overrightarrow{i}=(1;0;0)$.\\
Ta có 
\begin{eqnarray*}
	&&\cos 60^\circ =\dfrac{\left| \overrightarrow{n}\cdot \overrightarrow{i}\right|}{\left| \overrightarrow{n}\right|\cdot \left| \overrightarrow{i}\right|}\\
	&\Leftrightarrow&\dfrac{1}{2}=\dfrac{\left| a\right|}{\sqrt{2a^2+c^2}\cdot 1}\\
	&\Leftrightarrow&2a^2+c^2=4a^2\Leftrightarrow 2a^2-c^2=0.
\end{eqnarray*}
Chọn $a=1$, ta có $c^2=2\Rightarrow c=-\sqrt{2}$ do $c<0$.\\
Ta có $a+b+c=a+a+c=1+1-\sqrt{2}=2-\sqrt{2}\approx 0{,}6$.}
\end{ex}
\Closesolutionfile{ans}
% \indapan{7}{ans/C5B4CD6-D1-KQ}
\begin{dang}{Khoảng cách}
	\begin{enumerate}
		\item Khoảng cách từ một điểm đến đường thẳng
		\begin{itemize}
			\item Khoảng cách từ điểm $M$ đến một đường thẳng $d$ qua điểm $M_{0}$ có véc-tơ chỉ phương $\overrightarrow{u}_d$ được xác định bởi công thức $\mathrm{d}(M, d)=\dfrac{\left|\left[\overrightarrow{M_0 M}, \vec{u}_d\right]\right|}{\left|\vec{u}_d\right|}$.
			\item Khoảng cách giữa hai đường thẳng song song là khoảng cách từ một điểm thuộc đường thẳng này đến đường thẳng kia.
		\end{itemize}
		\item Khoảng cách giữa hai đường thẳng
		\begin{itemize}
			\item Khoảng cách giữa hai đường thẳng song song là khoảng cách từ một điểm thuộc đường thẳng này đến đường thẳng kia.
			\item Khoảng cách giữa hai đường thẳng chéo nhau: $d$ đi qua điểm $M$ và có véc-tơ chỉ phương $\overrightarrow{u}$ và $d'$ đi qua điểm $M'$ và có véc-tơ chỉ phương $\overrightarrow{u'}$ là $\mathrm{d}\left(d, d'\right)=\dfrac{\left|\left[\vec{u}, \vec{u'}\right] \cdot \overrightarrow{M'M}\right|}{\left|\left[\vec{u}, \vec{u'}\right]\right|}$.
		\end{itemize}
	\end{enumerate}
\end{dang}
\TN
\Opensolutionfile{ans}[ans/C5B4CD6-D2]
\begin{ex}%[Câu 14]%[2H5H2-6]
	Trong KG $Oxyz$, khoảng cách từ điểm $M(2;-4;-1)$ tới đường thẳng $\Delta\colon \heva{&x=t \\&y=2-t \\&z=3+2t}$ bằng
\choice
{$\sqrt{14}$}
{$\sqrt{6}$}
{\True $2\sqrt{14}$}
{$2\sqrt{6}$}
\loigiai{
Đường thẳng $\Delta $ đi qua $N(0;2;3)$, có véc-tơ chỉ phương $\overrightarrow{u}=(1;-1;2)$.\\
Ta có $\overrightarrow{MN}=(-2;6;4); \left[\overrightarrow{MN},\overrightarrow{u}\right]=(16;8;-4)$.\\
Do đó $\mathrm{d}(M,\Delta)=\dfrac{\left| \left[\overrightarrow{MN},\overrightarrow{u}\right]\right|}{\left| \overrightarrow{u}\right|}=\dfrac{\sqrt{336}}{\sqrt{6}}=2\sqrt{14}$. \\}
\end{ex}
\begin{ex}%[Câu 15]%[2H5H2-6]
	Trong KG $Oxyz$, cho đường thẳng $\mathrm{d}\colon\dfrac{x-3}{-2}=\dfrac{y}{-1}=\dfrac{z-1}{1}$ và điểm $A(2;-1;0)$. Khoảng cách từ điểm $A$ đến đường thẳng $d$ bằng
\choice
{$\sqrt{7}$}
{$\dfrac{\sqrt{7}}{2}$}
{\True $\dfrac{\sqrt{21}}{3}$}
{$\dfrac{\sqrt{7}}{3}$}
\loigiai{
Gọi $M(3;0;1)\in d$.\\
Ta có $\overrightarrow{AM}=(1;1;1)$, $\overrightarrow{u_d}=(-2;-1;1)$ nên $ \left[\overrightarrow{AM}, \overrightarrow{u_d}\right]=(2;-3;1)$ và $ \left| \left[\overrightarrow{AM}, \overrightarrow{u_d}\right]\right|=\sqrt{14}$.\\
Vậy khoảng cách từ điểm $A$ đến đường thẳng $d$ bằng $$\mathrm{d}(A,d)=\dfrac{\left| \left[\overrightarrow{AM};\overrightarrow{u_d}\right]\right|}{\left| \overrightarrow{u_d}\right|}=\dfrac{\sqrt{14}}{\sqrt{6}}=\dfrac{\sqrt{21}}{3}.$$}
\end{ex}
\begin{ex}%[Câu 16]%[2H5H2-6]
	Khoảng cách từ điểm $H(1;0;3)$ đến đường thẳng $\mathrm{d}_1\colon \heva{&x=1+t \\&y=2t \\&z=3+t}$, $t\in \mathbb{R}$ và mặt phẳng $(P)\colon z-3=0$ lần lượt là $\mathrm{d}(H,d_1)$ và $\mathrm{d}(H,(P))$. Chọn khẳng định đúng trong các khẳng định sau:
	\choice
	{$\mathrm{d}(H,d_1)>\mathrm{d}(H,(P))$}
	{$\mathrm{d}(H,(P))>\mathrm{d}(H,d_1)$}
	{\True $\mathrm{d}(H,d_1)=6\cdot \mathrm{d}(H,(P))$}
	{$\mathrm{d}(H,(P))=1$}
	\loigiai{
		Vì $H$ thuộc đường thẳng $\mathrm{d}_1$ và $H$ thuộc mặt phẳng $(P)$ nên khoảng cách từ điểm $H$ đến đường thẳng $\mathrm{d}_1$ bằng $0$ và khoảng cách từ điểm $H$ đến mặt phẳng $(P)$ bằng $0$.}
\end{ex}

\begin{ex}%[Câu 17]%[2H5H2-6]
	Tính khoảng cách giữa mặt phẳng $(\alpha)\colon 2x-y-2z-4=0$ và đường thẳng $\mathrm{d}\colon \heva{&x=1+t \\&y=2+4t \\&z=-t}$.
\choice
{$\dfrac{1}{3}$}
{\True $\dfrac{4}{3}$}
{$0$}
{$2$}
\loigiai{Mặt phẳng $(\alpha)$ có véc-tơ pháp tuyến $\overrightarrow{n}=(2;-1;-2)$, đường thẳng $\mathrm{d}$ có véc-tơ chỉ phương $\overrightarrow{u}=(1;4;-1)$.\\
	Ta có $\overrightarrow{n}\cdot \overrightarrow{u}=0$ và $H(1;2;0)\in d$ nhưng $H\notin (\alpha)$ nên đường thẳng $\mathrm{d}$ song song với mặt phẳng $(\alpha)$.\\
Khoảng cách giữa đường thẳng và mặt phẳng song song bằng khoảng cách từ một điểm bất kỳ của đường thẳng đến mặt phẳng.\\
Khi đó $\mathrm{d}(d,(\alpha))=\mathrm{d}(H,(\alpha))=\dfrac{\left| 2\cdot 1-1\cdot 2-2\cdot 0-4\right|}{\sqrt{2^2+(-1)^2+(-2)^2}}=\dfrac{4}{3}$.}
\end{ex}
\begin{ex}%[Câu 18]%[2H5H2-6]
	Trong KG $Oxyz$, cho mặt phẳng $(P)\colon 2x-2y-z+1=0$ và đường thẳng $\Delta\colon\dfrac{x-1}{2}=\dfrac{y+2}{1}=\dfrac{z-1}{2}$. Tính khoảng cách $\mathrm{d}$ giữa $\Delta $ và $(P)$.
\choice
{\True $\mathrm{d}=2$}
{$\mathrm{d}=\dfrac{5}{3}$}
{$\mathrm{d}=\dfrac{2}{3}$}
{$\mathrm{d}=\dfrac{1}{3}$}
\loigiai{
$(P)$ có véc-tơ pháp tuyến $\overrightarrow{n}=(2;-2;-1)$ và đường thẳng $\Delta $ có véc-tơ chỉ phương $\overrightarrow{u}=(2;1;2)$ thỏa mãn $\overrightarrow{n}\cdot \overrightarrow{u}=0$ nên $\Delta \parallel (P)$ hoặc $\Delta \subset (P)$.\\
Lấy $A(1;-2;1)\in \Delta$, ta có $\mathrm{d}(\Delta ,(P))=\mathrm{d}(A, (P))=\dfrac{\left| 2\cdot 1-2\cdot (-2)-1+1\right|}{\sqrt{4+4+1}}=2$.}
\end{ex}
