\begin{ex}%[Dự án 2025_K12_TL TV]%[Phạm Thị Thanh Thủy]%[2H5V1-6]
	Trong không gian $O x y z$, cho mặt phẳng $(P)$ có phương trình $x-2 y+2 z-5=0$. Xét mặt phẳng $(Q): x+(2 m-1) z+7=0$, với $m$ là tham số thực. Tính tổng tất cả giá trị của $m$ để $(P)$ tạo với $(Q)$ góc $\dfrac{\pi}{4}$.
	\shortans{$5$}
	\loigiai{		
		Mặt phẳng $(P),(Q)$ có vectơ pháp tuyến lần lượt là $\overrightarrow{n_p}=(1;-2;2), \overrightarrow{n_Q}=(1;0;2m-1)$.\\		
		Vì $(P)$ tạo với $(Q)$ góc $\dfrac{\pi}{4}$ nên
		 \begin{eqnarray*}
			\cos \dfrac{\pi}{4}=\left|\cos\left(\overrightarrow{n_p}; \overrightarrow{n_Q}\right)\right|&\Leftrightarrow& \dfrac{1}{\sqrt{2}}=\frac{|1+2(2 m-1)|}{3\cdot\sqrt{1+(2m-1)^2}}\\	
			& \Leftrightarrow& 2(4m-1)^2=9\left(4m^2-4m+2\right) \\
			& \Leftrightarrow& 4m^2-20m+16=0 \\
			& \Leftrightarrow& \hoac{&m=1\\
				&m=4.}
		\end{eqnarray*} 
		Do đó tổng các giá trị cần tìm là $4+1=5$.
	}
\end{ex}

\begin{ex}%[Dự án 2025_K12_TL TV]%[Phạm Thị Thanh Thủy]%[2H5V1-6]s
	Biết mặt phẳng $(\alpha):(2m-1)x-3my+2z+3=0$ và $(\beta): mx+(m-1)y+4z-5=0$ vuông góc với nhau. Tính tích tất cả các giá trị tìm được của tham số $m$.
	\shortans{$-8$}
	\loigiai{
		$$
		(\alpha)\perp(\beta)\Leftrightarrow(2 m-1) \cdot m+(-3m) \cdot(m-1)+2\cdot 4=0 \Leftrightarrow-m^2+2m+8=0\Leftrightarrow \hoac{&m=4\\
			&m=-2.}
		$$
		Do đó tích các giá trị cần tìm là $4\cdot (-2)=-8$.
	}
\end{ex}
\Closesolutionfile{ans}
\indapan{6}{ans/ans-goc3-TLN}