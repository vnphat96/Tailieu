\section{PHƯƠNG TRÌNH ĐƯỜNG THẲNG}

\subsection{Phương trình đường thẳng}

\subsubsection{Véc-tơ chỉ phương của đường thẳng}
 Cho đường thẳng $\Delta $ và véc-tơ $\overrightarrow{u}$ khác $\overrightarrow{0}$. Véc-tơ $\overrightarrow{u}$ được gọi là véc-tơ chỉ phương của đường thẳng $\Delta $ nếu giá của $\overrightarrow{u}$ song song hoặc trùng với $\Delta $.
 \begin{center}
	\begin{tikzpicture}
		\draw (-1,1)--(3,4)node[below]{$\Delta$};
		\draw[dashed]  (0,0)--(4,3);
		\draw[>=latex, ->, line width=1.5pt] (1.38,1)--(2.71,2)node[above=0.2,midway]{$\overrightarrow{u}$};
	\end{tikzpicture}
\end{center}
 \textbf{Nhận xét:}
\begin{itemize}
	\item  Một đường thẳng hoàn toàn được xác định khi biết một điểm mà nó đi qua và một véc-tơ chỉ phương của nó.
	\item  Nếu $\overrightarrow{u}$ là một véc-tơ chỉ phương của đường thẳng thì $k\cdot\overrightarrow{u}; (k\ne 0)$ cũng là một véc-tơ chỉ phương của đường thẳng đó.
\end{itemize}

\subsubsection{Phương trình tham số của đường thẳng}

 Trong không gian với hệ trục tọa độ $Oxyz$, phương trình tham số của đường thẳng $\Delta $ đi qua điểm $M(x_{0} ;y_{0} ;z_{0} )$ và nhận $\overrightarrow{u}=(a;b;c)$ (với $a^{2} +b^{2} +c^{2} \ne 0$) làm vectơ chỉ phương có dạng
 $$\heva{&x=x_{0} +at \\& {y=y_{0} +bt} \\ {z=z_{0} +ct}.} $$   với $t\in  \mathbb{R}$ ($t$ được gọi là tham số).
\begin{center}
	\begin{tikzpicture}
		\draw[->] (0,0)--(5,0)node[below]{$y$};
		\draw[->] (0,0)--(0,4)node[left]{$z$};
		\draw[->] (0,0)--(-2,-2)node[left]{$x$};
		\fill (0,0)node[left]{$O$};
		\draw (.15,0)|-(0,.15);\fill (.15,.15) circle(0pt); 
		\draw (1,-1)--(5,3)node[below]{$\Delta$};
		\draw[>=latex, ->, line width=1.5pt] (1.5,0.5)--(2.5,1.5)node[above=0.2,midway]{$\overrightarrow{u}$};
		\coordinate (M) at ($(1,-1)!1/3!(5,3)$); 
		\coordinate (N) at ($(1,-1)!2/3!(5,3)$);
		\draw[>=latex, ->, line width=1.5pt] (M)--(N);
		\draw (M) node[right]{$M$}; 
	\end{tikzpicture}
\end{center}

\subsubsection{Phương trình chính tắc của đường thẳng}

 Trong không gian với hệ trục tọa độ $Oxyz$, cho đường thẳng $\Delta $ đi qua điểm $M(x_{0} ;y_{0} ;z_{0} )$ và có vectơ chỉ phương $\overrightarrow{u}=(a;b;c)$. Nếu $a\cdot b\cdot c\ne 0$ thì hệ phương trình $$\dfrac{x-x_{0} }{a} =\dfrac{y-y_{0} }{b} =\dfrac{z-z_{0} }{c} $$ được gọi là \textbf{phương trình chính tắc} của đường thẳng $\Delta $.


\subsubsection{Lập phương trình đường thẳng đi qua hai điểm cho trước}

 Trong không gian với hệ trục tọa độ $Oxyz$, đường thẳng $\Delta $ đi qua hai điểm $A(x_{A} ;y_{A} ;z_{A} )$, $B(x_{B} ;y_{B} ;z_{B} )$ và nhận $\overrightarrow{AB}=(x_{B} -x_{A} ;y_{B} -y_{A} ;z_{B} -z_{A} )$ làm vectơ chỉ phương có
\begin{itemize}
	\item  Phương trình tham số : $\left\{\begin{array}{l} {x=x_{A} +(x_{B} -x_{A} )t} \\ {y=y_{A} +(y_{B} -y_{A} )t} \\ {z=z_{A} +(z_{B} -z_{A} )t} \end{array}\right. $ với $t\in  \mathbb{R}$.
	\item  Phương trình chính tắc: $\dfrac{x-x_{A} }{x_{B} -x_{A} } =\dfrac{y-y_{A} }{y_{B} -y_{A} } =\dfrac{z-z_{A} }{z_{B} -z_{A} } $ (với $x_{B} \ne x_{A}$, $y_{B} \ne y_{A}$, $z_{B} \ne z_{A} $).
\end{itemize}

 \subsection{Vị trí tương đối giữa hai đường thẳng. Điều kiện để hai đường thẳng vuông góc}
\subsubsection{Vị trí tương đối giữa hai đường thẳng}

 Trong không gian, hai véc-tơ được gọi là cùng phương khi giá của chúng cùng song song với một đường thẳng.\\
 Trong không gian, ba véc-tơ được gọi là đồng phẳng khi giá của chúng cùng song song với một mặt phẳng.\\
 Trong không gian với hệ trục tọa độ $Oxyz$, cho ba véc-tơ $\overrightarrow{a}=(a_{1} ;a_{2} ;a_{3} )$, $\overrightarrow{b}=(b_{1} ;b_{2} ;b_{3} )$, $\overrightarrow{c}=(c_{1} ;c_{2} ;c_{3} )$

\begin{itemize}
	\item  Hai $\overrightarrow{a}$, $\overrightarrow{b}$ cùng phương $\Leftrightarrow \left[\overrightarrow{a}, \overrightarrow{b}\right]=\overrightarrow{0}$.
	\item  Hai $\overrightarrow{a}$, $\overrightarrow{b}$ không cùng phương $\Leftrightarrow \left[\overrightarrow{a}, \overrightarrow{b}\right]\ne \overrightarrow{0}$.
	\item  Ba vectơ $\overrightarrow{a}$, $\overrightarrow{b},\overrightarrow{c}$ đồng phẳng $\Leftrightarrow \left[\overrightarrow{a}, \overrightarrow{b}\right].\, \overrightarrow{c}=0$.
	\item  Ba vectơ $\overrightarrow{a}, \overrightarrow{b}, \overrightarrow{c}$ không đồng phẳng $\Leftrightarrow \left[\overrightarrow{a}, \overrightarrow{b}\right].\, \overrightarrow{c}\ne 0$.
\end{itemize}

 Trong không gian với hệ trục tọa độ $Oxyz$, cho hai đường thẳng $\Delta _{1} ,\Delta _{2} $ lần lượt đi qua các điểm $M_{1} ,M_{2} $ và tương ứng có $\overrightarrow{u}_{1} =(a_{1} ;b_{1} ;c_{1} ),{\rm \; }\overrightarrow{u}_{2} =(a_{2} ;b_{2} ;c_{2} )$ là hai véc-tơ chỉ phương. Khi đó, ta có
 \begin{itemize}
 \item $\Delta_1 \equiv \Delta_2 \Leftrightarrow \heva{&\overrightarrow{u}_1, \overrightarrow{u}_2 \text { cùng phương } \\ &\overrightarrow{u}_1,\overrightarrow{M_1 M_2}  \text { cùng phương }} \Leftrightarrow \heva{&{\left[\overrightarrow{u}_1, \overrightarrow{u}_2\right]=\overrightarrow{0}} \\ &{\left[\overrightarrow{u}_1, \overrightarrow{M_1 M_2}\right]=\overrightarrow{0}}.}$
 \item $\Delta_1 \parallel \Delta_2 \Leftrightarrow \heva{&\overrightarrow{u}_1, \overrightarrow{u}_2  \text { cùng phương } \\ &\overrightarrow{u}_1, \overrightarrow{M_1 M_2}  \text { không cùng phương }} \Leftrightarrow \heva{&{\left[\overrightarrow{u}_1, \overrightarrow{u}_2\right]=\overrightarrow{0}} \\ &{\left[\overrightarrow{u}_1, \overrightarrow{M_1 M_2}\right]\ne \overrightarrow{0}}.}$
 \begin{center}
	\begin{tikzpicture}[scale=0.8]
		\draw (0,0)--(5,2)node[above]{$\Delta_1$};
		\draw (4,0)--(9,2)node[below]{$\Delta_2$};
		\coordinate (M_1) at ($(0,0)!1/5!(5,2)$);
		\coordinate (M_2) at ($(4,0)!1/5!(9,2)$);  
		\draw[>=latex, ->, line width=1pt] (M_1)--($(0,0)!3/5!(5,2)$)node[above=0.1,midway]{$\overrightarrow{u}_1$};
		\draw[>=latex, ->, line width=1pt] (M_2)--($(4,0)!3/5!(9,2)$)node[above=0.1,midway]{$\overrightarrow{u}_2$};
		\draw[line width=1pt] (M_1)--(M_2);
		\foreach \x/\g in {M_1/150,M_2/-30}
\fill[black] (\x) circle (1.5pt)
($(\g:4mm)+(\x)$) node {$\x$};	
	\end{tikzpicture}
\end{center}
 \item $\Delta_1$ cắt $\Delta_2 \Leftrightarrow\heva{&\overrightarrow{u}_1, \overrightarrow{u}_2  \text { không cùng phương } \\ &\overrightarrow{u}_1, \overrightarrow{u}_2, \overrightarrow{M_1 M_2}  \text { đồng phẳng }} \Leftrightarrow \heva{&{\left[\overrightarrow{u}_1, \overrightarrow{u}_2\right] \neq \overrightarrow{0}} \\ &{\left[\overrightarrow{u}_1, \overrightarrow{u}_2\right] \cdot \overrightarrow{M_1 M_2} \neq 0}.}$
  \begin{center}
	\begin{tikzpicture}[scale=0.7]
		\draw (5,-2)--(-2.5,1)node[above]{$\Delta_1$};
		\draw (5,2)--(-2.5,-1)node[below]{$\Delta_2$};
		\coordinate (M_1) at ($(0,0)!3/5!(5,-2)$);
		\coordinate (M_2) at ($(0,0)!4/5!(5,2)$);  
		\draw[>=latex, ->, line width=1pt] (0,0)--($(0,0)!2/5!(5,-2)$)node[below=0.1,midway]{$\overrightarrow{u}_1$};
		\draw[>=latex, ->, line width=1pt] (0,0)--($(0,0)!3/5!(5,2)$)node[above=0.1,midway]{$\overrightarrow{u}_2$};
		\draw[>=latex, ->, line width=1pt] (0,0)--(0,3)node[right]{$\left[ \overrightarrow{u}_1,\overrightarrow{u}_2\right]$};
		\draw[line width=1pt] (M_1)--(M_2);
		\foreach \x/\g in {M_1/-90,M_2/90}
\fill[black] (\x) circle (1.5pt)
($(\g:4mm)+(\x)$) node {$\x$};	
	\end{tikzpicture}
\end{center}
 \item $\Delta_1$ và $\Delta_2$ chéo nhau $\Leftrightarrow\left[\overrightarrow{u}_1, \overrightarrow{u}_2\right] \cdot\overrightarrow{M_1 M_2} \neq 0$.
   \begin{center}
	\begin{tikzpicture}[scale=0.7]
		\draw (3,2)--(-2,0)node[above]{$\Delta_2$};
		\draw (5,-2)--(-1,0) node[below]{$\Delta_1$};
		\coordinate (M_2) at ($(3,2)!1/5!(-2,0)$);
		\coordinate (M_1) at ($(5,-2)!1/5!(-1,0)$);  
		\draw[>=latex, ->, line width=1pt] ($(-1,0)!1/5!(5,-2)$)--($(-1,0)!3/5!(5,-2)$)node[below=0.1,midway]{$\overrightarrow{u}_1$};
		\draw[>=latex, ->, line width=1pt] ($(-2,0)!2/5!(3,2)$)--($(-2,0)!3/5!(3,2)$)node[below=0.1,midway]{$\overrightarrow{u}_2$};
		\draw[>=latex, ->, line width=1pt] (0,1)--(0,4)node[left]{$\left[ \overrightarrow{u}_1,\overrightarrow{u}_2\right]$};
		\draw[line width=1pt] (M_1)--(M_2);
		\foreach \x/\g in {M_1/-90,M_2/90}
\fill[black] (\x) circle (1.5pt)
($(\g:4mm)+(\x)$) node {$\x$};	
	\end{tikzpicture}
\end{center}
 \end{itemize}

\begin{note} Chú ý: Để xét vị trí tương đối giữa hai đường thẳng, ta cũng có thể dựa vào các véc-tơ chỉ phương và phương trình của hai đường thẳng đó.
\end{note}
 Trong không gian với hệ trục tọa độ $Oxyz$, cho hai đường thẳng $\Delta _{1} ,\Delta _{2} $ tương ứng có $\overrightarrow{u}_{1} =(a_{1} ;b_{1} ;c_{1} ),{\rm \; }\overrightarrow{u}_{2} =(a_{2} ;b_{2} ;c_{2} )$ là hai vectơ chỉ phương và có phương trình tham số:
\[\Delta _{1} :\left\{\begin{array}{l} {x=x_{1} +a_{1} t_{1} } \\ {y=y_{1} +b_{1} t_{1} } \\ {z=z_{1} +c_{1} t_{1} } \end{array}\right. {\rm \; }\left(t_{1} \in  \mathbb{R}\right),{\rm \; \; }\Delta _{2} :\left\{\begin{array}{l} {x=x_{2} +a_{2} t_{2} } \\ {y=y_{2} +b_{2} t_{2} } \\ {z=z_{2} +c_{2} t_{2} } \end{array}\right. {\rm \; \; }\left(t_{2} \in  \mathbb{R}\right)\] 
 Xét hệ phương trình hai ẩn $t_{1} ,t_{2} $: $\left\{\begin{array}{l} {x_{1} +a_{1} t_{1} =x_{2} +a_{2} t_{2} } \\ {y_{1} +b_{1} t_{1} =y_{2} +b_{2} t_{2} } \\ {z_{1} +c_{1} t_{1} =z_{2} +c_{2} t_{2} } \end{array}\right. $ \quad$\left(*\right)$.\\
 Khi đó
\begin{itemize}
	\item  $\Delta _{1} \equiv \Delta _{2} \Leftrightarrow $ $\overrightarrow{u}_{1} $ cùng phương với $\overrightarrow{u}_{2} $ và hệ $\left(*\right)$ vô nghiệm.
	\item  $\Delta _{1} \parallel \Delta _{2} \Leftrightarrow $ Hệ $\left(*\right)$ có vô số nghiệm.
	\item  $\Delta _{1} $ cắt $\Delta _{2} \Leftrightarrow $ Hệ $\left(*\right)$ có nghiệm duy nhất.
	\item  $\Delta _{1} $ và $\Delta _{2} $ chéo nhau $\Leftrightarrow \overrightarrow{u}_{1} $ không cùng phương với $\overrightarrow{u}_{2} $ và hệ $\left(*\right)$ vô nghiệm.
\end{itemize}

\subsubsection{Điều kiện để hai đường thẳng vuông góc}

 Trong không gian với hệ trục tọa độ $Oxyz$, cho hai đường thẳng $\Delta _{1} ,\Delta _{2} $ tương ứng có $\overrightarrow{u}_{1} =(a_{1} ;b_{1} ;c_{1} ),{\rm \; }\overrightarrow{u}_{2} =(a_{2} ;b_{2} ;c_{2} )$ là hai vectơ chỉ phương. Khi đó
\[\Delta _{1} \bot \Delta _{2} \Leftrightarrow \overrightarrow{u}_{1} \cdot\overrightarrow{u}_{2} =0\Leftrightarrow a_{1} a_{2} +b_{1} b_{2} +c_{1} c_{2} =0\] 

\subsection{Góc}
\subsubsection{Góc giữa hai đường thẳng}

  Trong không gian với hệ trục tọa độ $Oxyz$, cho hai đường thẳng $\Delta _{1} ,\Delta _{2} $ có hai vectơ chỉ phương lần lượt là: $\overrightarrow{u}_{1} =(a_{1} ;b_{1} ;c_{1} ),{\rm \; }\overrightarrow{u}_{2} =(a_{2} ;b_{2} ;c_{2} )$. Khi đó, ta có
\[\cos \left(\Delta _{1} ,\Delta _{2} \right)=\left|\cos \left(\overrightarrow{u}_{1} ,\overrightarrow{u}_{2} \right)\right|=\dfrac{\left|\overrightarrow{u}_{1} \cdot\overrightarrow{u}_{2} \right|}{\left|\overrightarrow{u}_{1} \right|\cdot\left|\overrightarrow{u}_{2} \right|} =\dfrac{\left|a_{1} a_{2} +b_{1} b_{2} +c_{1} c_{2} \right|}{\sqrt{a_{1}^{2} +b_{1}^{2} +c_{1}^{2} } \cdot\sqrt{a_{2}^{2} +b_{2}^{2} +c_{2}^{2} } }. \] 
   \begin{center}
	\begin{tikzpicture}[scale=1.2]
		\draw (-2,1)--(2,-1)node[above]{$\Delta_2'$};
		\draw (-2,-1)--(2,1) node[below]{$\Delta_1'$};
		\draw (-1,1)--(2,5/2)node[above]{$\Delta_1$};
		\draw (-1,-1)--(2,5/-2)node[above]{$\Delta_2$};
		\coordinate (B) at ($(-2,1)!4/5!(2,-1)$);
		\coordinate (A) at ($(-2,-1)!4/5!(2,1)$); 
		\coordinate (O) at (0,0); 
		\draw[>=latex, ->, line width=1pt] (0,0)--(A)node[above=0.1,midway]{$\overrightarrow{u}_1$};
		\draw[>=latex, ->, line width=1pt] (0,2)--($(0,2)+(A)-(0,0)$) node[above=0.1,midway]{$\overrightarrow{u}_1$};
		\draw[>=latex, ->, line width=1pt] (0,0)--(B)node[below=0.1,midway]{$\overrightarrow{u}_2$};
		\draw[>=latex, ->, line width=1pt] (0,-2)--($(0,-2)+(B)-(0,0)$) node[below=0.1,midway]{$\overrightarrow{u}_2$};
		\draw pic[draw,,angle radius=6mm]{angle=B--O--A};
		\foreach \x/\g in {A/90,B/-90}
\fill[black] (\x) circle (1pt) ($(\g:4mm)+(\x)$) node {$\x$};	
	\end{tikzpicture}
\end{center}

\subsubsection{Góc giữa đường thẳng với mặt phẳng}

  Trong không gian với hệ trục tọa độ $Oxyz$, cho đường thẳng $\Delta $ có vectơ chỉ phương $\overrightarrow{u}=(a;b;c)$ và mặt phẳng $(P)$ có vectơ pháp tuyến $\, \overrightarrow{n}=(A;\, B;\, C)$. Khi đó, ta có
\[\sin \left(\Delta ,(P)\right)=\left|\cos \left(\overrightarrow{u}, \overrightarrow{n}\right)\right|=\dfrac{\left|\overrightarrow{u}\cdot \overrightarrow{n}\right|}{\left|\overrightarrow{u}\right|\cdot \left|\, \overrightarrow{n}\right|} =\dfrac{\left|aA+bB+cC\right|}{\sqrt{a^{2} +b^{2} +c^{2} } \cdot\sqrt{A^{2} +B^{2} +C^{2} } }. \] 
\begin{center}
\begin{tikzpicture}[scale=1]
	\def\d{4}
	\def\r{3}
	\path (0:0) coordinate (B)
			++(0:\d) coordinate (C)
			++(60:\r) coordinate (D)
			($(B)+(D)-(C)$) coordinate (A);
	\coordinate (H') at (9/8*\d,\r/2);
	\coordinate (I) at (3/5*\d,\r/2);
	\coordinate (H) at ($(H')!1!-90:(I)$);
	\coordinate (K) at ($(H)!-1/2!(I)$);
	\coordinate (M) at (intersection of B--C and I--H);
	\coordinate (N) at ($(M)!-1/2!(I)$);
	\coordinate (O) at (\d/2,\r);
	\coordinate (ut) at ($(O)+(H)-(I)$);
	\coordinate (nt) at ($(O)+(H)-(H')$);
	\coordinate (u) at ($(O)!1/2!(ut)$);
	\coordinate (n) at ($(O)!1/2!(nt)$);
	\draw[dashed] (I)--(M) (H)--(H');
	\draw (2/5*\d,\r/2)--(6/5*\d,\r/2)node[above]{$\Delta'$} (M)--(N) (I)--(K)node[right]{$\Delta$};
	\draw[>=latex, ->, line width=1pt] (O)--(u) node[below=0.1]{$\overrightarrow{u}$};
	\draw[>=latex, ->, line width=1pt] (O)--(n) node[below=0.1,left]{$\overrightarrow{n}$};
	\draw (A)--(B)--(C)--(D)--cycle;
	\foreach \x/ \goc in {I/-90,H'/-90,H/-45} 
			\fill (\x) circle (1pt)	
			($(\x)+(\goc:3mm)$) node {$\x$};
	\draw pic[draw,"$P$",angle radius=6mm]{angle=C--B--A};
	\end{tikzpicture}
\end{center}
\subsubsection{Góc giữa hai mặt phẳng}

  Trong không gian với hệ trục tọa độ $Oxyz$, cho hai mặt phẳng $(P_{1} ),(P_{2} )$ có hai vectơ pháp tuyến lần lượt là$\, \overrightarrow{n}_{1} =(A_{1} ;\, B_{1} ;\, C_{1} ),\, {\rm \; }\overrightarrow{n}_{2} =(A_{2} ;\, B_{2} ;\, C_{2} )$. Khi đó, ta có
\[\cos \left((P_{1} ),(P_{2} )\right)=\left|\cos \left(\, \overrightarrow{n}_{1} , \overrightarrow{n}_{2} \right)\right|=\dfrac{\left|\overrightarrow{n}_{1} \cdot \overrightarrow{n}_{2} \right|}{\left|\overrightarrow{n}_{1} \right| \cdot\left|\overrightarrow{n}_{2} \right|} =\dfrac{\left|A_{1} A_{2} +B_{1} B_{2} +C_{1} C_{2} \right|}{\sqrt{A_{1}^{2} +B_{1}^{2} +C_{1}^{2} } \cdot\sqrt{A_{2}^{2} +B_{2}^{2} +C_{2}^{2} } }. \] 
\begin{center}
\begin{tikzpicture}[scale=1]
	\def\d{4}
	\def\r{2}
	\path (0:0) coordinate (B)
			++(-20:\d) coordinate (C)
			++(100:\r) coordinate (D)
			($(B)+(D)-(C)$) coordinate (A);
%	\foreach \x/ \goc in {A/180,B/180,C/0,D/0} 
%			\fill (\x) circle (1pt)	
%			($(\x)+(\goc:3mm)$) node {$\x$};
	\path (0,-1/2*\r) coordinate (B')
			++(15:\d) coordinate (C')
			++(120:\r) coordinate (D')
			($(B')+(D')-(C')$) coordinate (A');
	\coordinate (M) at (intersection of B--C and B'--C');
	\coordinate (N) at (intersection of A--D and A'--D');
	\coordinate (P) at (intersection of A--B and A'--D');
	\coordinate (Q) at (intersection of C--D and B'--C');
	\coordinate (O1) at (\d/7,\r/3);
	\coordinate (u1) at (\d/3,4/3*\r);
	\coordinate (O2) at (3*\d/5,\r/4);
	\coordinate (u2) at (\d/2,3/2*\r);
	\fill[orange!25] (M)--(N)--(A)--(B)--(M);
	\fill[orange!25] (M)--(C)--(Q)--(M);
	\fill[green!25] (A')--(B')--(M)--(B)--(P)--(A');
	\fill[green!25] (M)--(N)--(D')--(C')--(M);
	\draw[>=latex, ->, line width=1pt] (O1)--($(O1)!3/4!(u1)$) node[below=0.1,left]{$\overrightarrow{n_1}$};
	\draw[>=latex, ->, line width=1pt] (O2)--($(O2)!3/4!(u2)$) node[below=0.1,right]{$\overrightarrow{n_2}$};
	\draw (O1)--(u1) (O2)--(u2);
	\draw (A')--(B')--(M)--(B)--(P)--(A')  (M)--(C)--(Q)--(M) (M)--(N)--(A)--(B)--(M) (M)--(N)--(D')--(C')--(M);
%	\foreach \x/ \goc in {A'/180,B'/180,C'/0,D'/0} 
%			\fill (\x) circle (1pt)	
%			($(\x)+(\goc:3mm)$) node {$\x$};
\end{tikzpicture}
\end{center}
\subsection{Một số bài toán}

% \chude{XÁC ĐỊNH CÁC YẾU TỐ CƠ BẢN LIÊN QUAN ĐẾN ĐƯỜNG THẲNG}
 
\begin{dang}{XÁC ĐỊNH VECTƠ CHỈ PHƯƠNG CỦA ĐƯỜNG THẲNG, XÁC ĐỊNH ĐIỂM THUỘC VÀ KHÔNG THUỘC ĐƯỜNG THẲNG}
\end{dang}
 

 \subsubsection{Vectơ chỉ phương của đường thẳng}

\begin{itemize}
	\item  Vectơ chỉ phương $\overrightarrow{u}$ của đường thẳng $\Delta $ là vectơ có giá song song hoặc trùng với đường thẳng $\Delta $.\\
Nếu $\Delta $ có một vectơ chỉ phương là $\overrightarrow{u}$ thì $k.\overrightarrow{u}$ cũng là một vectơ chỉ phương của $\Delta $.
	\item    Nếu có hai vectơ $\overrightarrow{n}_{1} $ và $\overrightarrow{n}_{2} $ cùng vuông góc với $\Delta $ thì $\Delta $ có một vectơ chỉ phương là $\overrightarrow{u}=[\overrightarrow{n}_{1} ,\overrightarrow{n}_{2} ].$
   \item Phương trình đường thẳng \(\Delta\) dạng: \(\left\{\begin{array}{l} x = x_0 + at \\ y = y_0 + bt \\ z = z_0 + ct \end{array}\right. \; (t \in \mathbb{R})\) thì có vectơ chỉ phương là \(\overrightarrow{u} = (a; b; c)\).
    \item Phương trình đường thẳng \(\Delta\) dạng: \(\dfrac{x - x_0}{a} = \dfrac{y - y_0}{b} = \dfrac{z - z_0}{c} \; (a \neq 0, b \neq 0, c \neq 0)\) thì có vectơ chỉ phương là \(\overrightarrow{u} = (a; b; c)\).
\end{itemize}

\begin{note} Chú ý:
\begin{itemize}
	\item  Trục $Ox$ có vectơ chỉ phương là $\overrightarrow{i}=(1;0;0)$.
	\item  Trục $Oy$ có vectơ chỉ phương là $\overrightarrow{j}=(0;1;0)$.
	\item  Trục $Oz$ có vectơ chỉ phương là $\overrightarrow{k}=(0;0;1)$.
\end{itemize}
 \end{note}

\subsubsection{Điểm thuộc và không thuộc đường thẳng}

\begin{itemize}
	\item  Cho điểm $M\left(x_{M} ; y_{M} ; z_{M} \right)$ và đường thẳng $\Delta $ có phương trình $$\dfrac{x-x_{0} }{a} =\dfrac{y-y_{0} }{b} =\dfrac{z-z_{0} }{c} .$$ Khi đó
	\begin{align*}
	M\in \Delta & \Leftrightarrow \dfrac{x_{M} -x_{0} }{a} =\dfrac{x_{M} -y_{0} }{b} =\dfrac{x_{M} -z_{0} }{c} ;\\
	 M\notin \Delta &\Leftrightarrow \hoac{ {\dfrac{x_{M} -x_{0} }{a} \ne \dfrac{x_{M} -y_{0} }{b} } \\ {\dfrac{x_{M} -y_{0} }{b} \ne \dfrac{x_{M} -z_{0} }{c} .} }
	\end{align*}
	\item  Cho điểm $M\left(x_{M} \, ;\, y_{M} \, ;\, z_{M} \right)$ và đường thẳng $\Delta $ có phương trình $$\heva{{x=x_{0} +at} \\ {y=y_{0} +bt} \\ {z=z_{0} +ct.} } $$ 
	Khi đó
\[M\in \Delta \Leftrightarrow t=\dfrac{x_{M} -x_{0} }{a} =\dfrac{x_{M} -y_{0} }{b} =\dfrac{x_{M} -z_{0} }{c} ;      M\notin \Delta \Leftrightarrow \left[\begin{array}{l} {t=\dfrac{x_{M} -x_{0} }{a} \ne \dfrac{x_{M} -y_{0} }{b} } \\ {t=\dfrac{x_{M} -y_{0} }{b} \ne \dfrac{x_{M} -z_{0} }{c} .} \end{array}\right. \]
\end{itemize}
\TN
\Opensolutionfile{ans}[ans/ans-2C5B2CD1-D1]
%%%==============Cau_EX1==============%%%
\begin{ex}%[2H5N2-2]
	Trong không gian $Oxyz$, cho đường thẳng $d$: $\left\{\begin{array}{c} {x=2+t} \\ {y=1-2t} \\ {z=-1+3t} \end{array}\right.$. Vectơ nào dưới đây là một vectơ chỉ phương của $d$?
	\choice
		{$\overrightarrow{u}_1=(2;1;-1)$}
		{$\overrightarrow{u}_2=(1;2;3)$}
		{\True $\overrightarrow{u}_3=(1;-2;3)$}
		{$\overrightarrow{u}_4=(2;1;1)$}
	\loigiai{
		Từ phương trình đường thẳng $d$ ta thấy vectơ $\overrightarrow{u}_3=(1;-2;3)$ là một véctơ chỉ phương của $d$.
		}
\end{ex}
%%%==============HetCau_EX1==============%%%

%%%==============Cau_EX2==============%%%
\begin{ex}%[2H5N2-2]
	Trong không gian $Oxyz$, cho đường thẳng $d:\dfrac{x-3}{2}=\dfrac{y-4}{-5}=\dfrac{z+1}{3}$. Vectơ nào dưới đây là một vectơ chỉ phương của $d$?
	\choice
		{$\overrightarrow{u}_2=\left(2;4;-1\right)$}
		{\True $\overrightarrow{u}_1=\left(2;-5;3\right)$}
		{$\overrightarrow{u}_3=\left(2;5;3\right)$}
		{$\overrightarrow{u}_4=\left(3;4;1\right)$}
	\loigiai{
		Đường thẳng $d:\dfrac{x-3}{2}=\dfrac{y-4}{-5}=\dfrac{z+1}{3}$ có một vectơ chỉ phương là $\overrightarrow{u}_1\left(2;-5;3\right)$.
		}
\end{ex}
%%%==============HetCau_EX2==============%%%

%%%==============Cau_EX3==============%%%
\begin{ex}%[2H5N2-2]
	Trong không gian $Oxyz$, đường thẳng $d:\dfrac{x+3}{1}=\dfrac{y-1}{-1}=\dfrac{z-5}{2}$ có một vectơ chỉ phương là
	\choice
		{$\overrightarrow{u}_1=\left(3;-1; 5\right)$}
		{\True $\overrightarrow{u}_4=\left(-1; 1;-2\right)$}
		{$\overrightarrow{u}_2=\left(-3; 1; 5\right)$}
		{$\overrightarrow{u}_1=\left(1;-1;-2\right)$}
	\loigiai{
		Đường thẳng $\left(P\right)$ có một vectơ chỉ phương là $\overrightarrow{u}_4=\left(1;-1; 2\right)=-1\left(-1; 1;-2\right)\Rightarrow \overrightarrow{u}_4=\left(-1; 1;-2\right)$.
		}
\end{ex}
%%%==============HetCau_EX3==============%%%

%%%==============Cau_EX4==============%%%
\begin{ex}%[2H5N2-2]
	Trong không gian với hệ tọa độ $Oxyz$, cho đường thẳng $d:\dfrac{x}{-1}=\dfrac{y-4}{2}=\dfrac{z-3}{3}$. Hỏi trong các vectơ sau, đâu không phải là vectơ chỉ phương của $d$?
	\choice
		{$\overrightarrow{u}_1=\left(-1;2;3\right)$}
		{$\overrightarrow{u}_2=\left(3;-6;-9\right)$}
		{$\overrightarrow{u}_3=\left(1;-2;-3\right)$}
		{\True $\overrightarrow{u}_4=\left(-2;4;3\right)$}
	\loigiai{
		Ta có một vectơ chỉ phương của $d$ là $\overrightarrow{u}_1=\left(-1;2;3\right)$.\\
		$\overrightarrow{u}_2=-3\overrightarrow{u}_1$, $\overrightarrow{u}_3=-\overrightarrow{u}_1$ $\Rightarrow$ các vectơ $\overrightarrow{u}_2,\overrightarrow{u}_3$ cũng là vectơ chỉ phương của $d$.\\
		Không tồn tại số $k$ để $\overrightarrow{u}_4=k\overrightarrow{.u_1}$ nên $\overrightarrow{u}_4=\left(-2;4;3\right)$ không phải là vectơ chỉ phương của $d$.
		}
\end{ex}
%%%==============HetCau_EX4==============%%%
%Câu 1.
\begin{ex}%[2H5N2-2]
	Trong không gian với hệ tọa độ $Oxyz$, đường thẳng nào sau đây nhận véc-tơ pháp tuyến của mặt phẳng $(P) \colon 2x-y+2z+5=0$ làm một véc-tơ chỉ phương?
	\choice
	{$( Q) \colon x-y+2=0$}
	{$\dfrac{x}{2}=\dfrac{y-1}{1}=\dfrac{z-2}{-1}$}
	{\True $\dfrac{x-1}{-2}=\dfrac{y+1}{-1}=\dfrac{z}{-1}$}
	{$\dfrac{x+2}{2}=\dfrac{y+1}{-1}=\dfrac{z+1}{1}$}
	\loigiai{Xét đường thẳng $\dfrac{x-1}{-2}=\dfrac{y+1}{-1}=\dfrac{z}{-1}$, có một véc-tơ chỉ phương là $\left( -2;-1;-1 \right)=-\left( 2;1;1 \right)$(thỏa đề bài).}
\end{ex}
%Câu 2.
\begin{ex}%[2H5N2-2]
	Trong không gian với hệ tọa độ $Oxyz$, đường thẳng nào sau đây nhận ${\overrightarrow{u}=(-2 ; 4 ; 5)}$ là một véc-tơ chỉ phương?
	\choice
	{$\heva{&x=-2+3 t \\ &y=4-t \\& z=5+4t}$}
	{$\heva{&x=3+2t \\ & y=-1+4t \\ & z=4+5t}$}
	{$\heva{&x=3+2t  \\&y=1+4t  \\&z=4+5t }$}
	{\True $\heva{&x=3+2t  \\&y=-1-4t  \\&z=4-5t}$}
	\loigiai{Xét đường thẳng $\heva{&x=3+2t  \\&y=-1-4t  \\&z=4-5t }$, có một véc-tơ chỉ phương là
		$\overrightarrow{u}=(2;-4;-5)=-(-2;4;5)$ (thỏa đề bài).}
\end{ex}
\begin{ex}%[2H5N2-3]
	Trong không gian với hệ tọa độ $Oxyz$, đường thẳng nào sau đây nhận $\overrightarrow{u}=(-2;4;5)$ là một véc-tơ chỉ phương?
	\choice
	{$\heva{&x=-2+3 t \\ &y=4-t \\& z=5+4t}$}
	{$\heva{&x=3+2t \\ & y=-1+4t \\ & z=4+5t}$}
	{$\heva{&x=3+2t  \\&y=1+4t  \\&z=4+5t }$}
	{\True $\heva{&x=3+2t  \\&y=-1-4t  \\&z=4-5t }$}
	\loigiai{Ta có đường thẳng $\heva{&x=3+2t  \\&y=-1-4t  \\&z=4-5t }$, có một véc-tơ chỉ phương là $\overrightarrow{u}=(2;-4;-5)=-(-2;4;5)$ (thỏa đề bài).}
\end{ex}
\begin{ex}%[2H5N2-2]
	Trong không gian với hệ tọa độ $Oxyz$, cho hai điểm $A(1;1;0)$ và $B( 0;1;2 )$. Véc-tơ nào dưới đây là một véc-tơ chỉ phương của đường thẳng $AB$.
	\choice
	{ $\overrightarrow{d}=(-1;1;2)$}
	{ $\overrightarrow{a}=(-1;0;-2)$}
	{\True $\overrightarrow{b}=(-1;0;2)$}
	{ $\overrightarrow{c}=( 1;2;2)$}
	\loigiai{
		Ta có $\overrightarrow{AB}=( -1;0;2 )$ suy ra đường thẳng ${AB}$ có véc-tơ chỉ phương là $\overrightarrow{b}=(-1;0;2)$.}
\end{ex}
\begin{ex}%[2H5H2-2]
	Trong không gian với hệ tọa độ $Oxyz$, cho điểm $M( 1;2;3 )$. Gọi ${M}_{1}$, ${M}_{2}$ lần lượt là hình chiếu vuông góc của $M$ lên các trục $Ox$, $Oy$. Véc-tơ nào dưới đây là một véc-tơ chỉ phương của đường thẳng $M_1M_2$?
	\choice
	{ \True $\overrightarrow{{{u}_{4}}}=( -1;2;0 )$}
	{ $\overrightarrow{{u}_{1}}=( 0;2;0)$}
	{  $\overrightarrow {u_2}=( 1;2;0 )$}
	{ $\overrightarrow{{{u}_{3}}}=( 1;0;0 )$}
	\loigiai{
		Ta có ${{M}_{1}}$ là hình chiếu của $M$ lên trục $Ox \Rightarrow {{M}_{1}}( 1;0;0)$.\\
		${{M}_{2}}$ là hình chiếu của $M$ lên trục $Oy\Rightarrow {{M}_{2}}( 0;2;0)$.\\
		Khi đó $\overrightarrow{{{M}_{1}}{{M}_{2}}}=( -1;2;0 )$ là một véc-tơ chỉ phương của đường thẳng ${{M}_{1}}{{M}_{2}}$.}
\end{ex}
\begin{ex}%[2H5N2-3]
	Trong không gian $Oxyz$, cho đường thẳng $d \colon\dfrac{x-2}{1}=\dfrac{y-1}{-2}=\dfrac{z+1}{3}$. Điểm nào dưới đây thuộc $d$?
	\choice
	{ $Q( 2;1;1 )$}
	{ $M( 1;2;3 $}
	{ \True $P( 2;1;-1)$}
	{ $N( 1;-2;3) $}
	\loigiai{
		Cho $\heva{& x-2=0 \\ & y-1=0 \\ & z+1=0} \Rightarrow \heva{& x=2 \\ & y=1 \\ & z=-1.}$ Vậy $P( 2;1;-1 ) \in d$.}
\end{ex}
\begin{ex}%[2H5N2-3]
	Trong không gian $Oxyz$, điểm nào dưới đây thuộc đường thẳng $d \colon \dfrac{x+1}{-1}=\dfrac{y-2}{3}=\dfrac{z-1}{3}$?
	\choice
	{\True $P(-1\,2;1)$}
	{ $Q(1;-2;-1)$}
	{ $N(-1;3;2)$}
	{ $M( 1;2;1)$}
	\loigiai{
		Thay tọa độ các điểm vào phương trình đường thẳng ta thấy điểm $P(-1;2;1)$ thỏa $\dfrac{-1+1}{-1}=\dfrac{2-2}{3}=\dfrac{1-1}{3}=0$. Vậy điểm $P( -1;2;1)$ thuộc đường thẳng $d$.}
\end{ex}
\begin{ex}%[2H5N2-3]
	Trong không gian $Oxyz$, cho đường thẳng $d \colon \dfrac{x-4}{2}=\dfrac{z-2}{-5}=\dfrac{z+1}{1}$. Điểm nào sau đây thuộc $d$?
	\choice
	{ \True $N(4;2;-1)$}
	{ $Q(2;5;1)$}
	{ $M(4;2;1)$}
	{ $P(2;-5;1)$}
	\loigiai{
		Ta có điểm $N(4;2;-1)$ thỏa mãn phương trình $d$.}
\end{ex}
\begin{ex}%[2H5N2-3]
	Trong không gian $Oxyz$, điểm nào dưới đây thuộc đường thẳng $d \colon 
	\heva{& x=1-t \\ 
		& y=5+t \\ 
		& z=2+3t.}$
	\choice
	{ \True $N(1;5;2)$}
	{ $Q(-1;1;3)$}
	{ $M(1;1;3)$}
	{ $P(1;2;5)$}
	\loigiai{
		Ta có $N(1;5;2)$ thuộc $d$.}
\end{ex}
\begin{ex}%[2H5N2-3]
	Trong không gian với hệ tọa độ $Oxyz$. Đường thẳng $d \colon \heva{
		& x=t \\ 
		& y=1-t \\ 
		& z=2+t}$ đi qua điểm nào sau sau đây?
	\choice
	{ $K\left( 1;-1;1 \right)$}
	{ $E\left( 1;1;2 \right)$}
	{ $H\left( 1;2;0 \right)$}
	{  \True $F\left( 0;1;2 \right)$}
	\loigiai{
		Thay tọa độ của $K\left( 1;-1;1 \right)$ vào phương trình tham số của $d$ ta được $$\heva{
			& 1=t \\ 
			& -1=1-t \\ 
			& 1=2+t }\Leftrightarrow \heva{
			& t=1 \\ 
			& t=2 \\ 
			& t=-1.}$$ 
		Vậy không tồn tại $t$ hay $K\notin d$.\\
		Tương tự, thay $E\left( 1;1;2 \right)$ vào phương trình tham số của $d$ ta được $$\heva{	& 1=t \\ & 1=1-t \\ & 2=2+t }\Leftrightarrow \heva{
			& t=1 \\ 
			& t=0 \\ 
			& t=0. }$$ 
		Vậy không tồn tại $t$ hay $E\notin d$.\\
		Thay tọa độ của $H\left( 1;2;0 \right)$ vào phương trình tham số của $d$ ta được $$\heva{& 1=t \\ & 2=1-t \\ & 0=2+t }\Leftrightarrow \heva{
			& t=1 \\ 
			& t=-1 \\ 
			& t=-2. }$$
		Vậy không tồn tại $t$ hay $H\notin d$.\\			
		Thay tọa độ của $F\left( 0;1;2 \right)$ vào phương trình tham số của $d$ ta được $$\heva{& 0=t \\ & 1=1-t \\ & 2=2+t}\Leftrightarrow \heva{
			& t=0 \\ 
			& t=0 \\ 
			& t=0 }\Leftrightarrow t=0.$$
		Vậy $F \in d$.
	}
\end{ex}
\begin{ex}%[2H5N2-3]
	Trong không gian $Oxyz$, điểm nào dưới đây thuộc đường thẳng $d \colon \heva{& x=1-t \\ & y=5+t \\ 	& z=2+3t}$ ?
	\choice
	{  $Q\left( -1; 1; 3 \right)$}
	{ $P\left( 1; 2; 5 \right)$}
	{\True $N\left( 1; 5;2 \right)$}
	{ $M\left( 1; 1; 3 \right)$}
	\loigiai{
		Với $t=0\Rightarrow \heva{
			& x=1 \\ 
			& y=5 \\ 
			& z=2 }\Rightarrow N\left( 1;5;2 \right)\in d$.
		
	}
\end{ex}
\Closesolutionfile{ans}
\indapan{10}{ans/ans-2C5B2CD1-D1}
\TNTF
\Opensolutionfile{ans}[ans/ans-2C5B2CD1-D1-DS]
%Câu 3.
\begin{ex}%[2H5N2-2]
	Trong không gian $Oxyz$, cho đường thẳng $d \colon \dfrac{x-2}{3}=\dfrac{y+5}{4}=\dfrac{z-1}{-1}$.  Các mệnh đề sau đây đúng hay sai?
	\choiceTF
	{ Đường thẳng $d$ nhận $\overrightarrow{u}=\left( 3;4;1 \right)$ là một véc-tơ chỉ phương}
	{ \True Đường thẳng $d$ nhận $\overrightarrow{u}=\left( -3;-4;1 \right)$ là một véc-tơ chỉ phương}
	{\True  Đường thẳng $d$ nhận $\overrightarrow{u}=\left( 3;4;-1 \right)$ là một véc-tơ chỉ phương}
	{ \True Đường thẳng $d$ nhận $\overrightarrow{u}=\left( -6;-8;2 \right)$ là một véc-tơ chỉ phương}
	\loigiai{
		Đường thẳng $d \colon \dfrac{x-2}{3}=\dfrac{y+5}{4}=\dfrac{z-1}{-1}$ có một véc-tơ chỉ phương là ${{\overrightarrow{u}}_{d}}=\left( 3;4;-1 \right)$.
		\begin{itemchoice}
			\itemch Sai. Vì  $\overrightarrow{u}\ne {{\overrightarrow{u}}_{d}}$.
			\itemch Đúng. Vì $\overrightarrow{u}=\left( -3;-4;1 \right)=-\left( 3;4;-1 \right)=-{{\overrightarrow{u}}_{d}}$.
			\itemch Đúng.Vì $\overrightarrow{u}=\overrightarrow{u}_{d}$.
			\itemch Đúng.Vì  $\overrightarrow{u}=\left( -6;-8;2 \right)=-2\left( 3;4;-1 \right)=-2{{\overrightarrow{u}}_{d}}$.
			
		\end{itemchoice}
	}
\end{ex}
\begin{ex}%[2H5N2-2]
	Trong không gian $Oxyz$, cho đường thẳng $d\colon \heva{
		& x=3+4t \\ 
		& y=-1-2t \\ 
		& z=-2+3t},( t\in \mathbb{R})$. Các mệnh đề sau đây đúng hay sai?
	\choiceTF
	{ Điểm $M\left( 7;-3;-1 \right)$ thuộc đường thẳng $d$}
	{ \True Điểm $N\left( -1;1;-5 \right)$ thuộc đường thẳng $d$}
	{\True  Đường thẳng $d$ nhận $\overrightarrow{u}=\left( 4;-2;3 \right)$ là một véc-tơ chỉ phương}
	{\True  Đường thẳng $d$ nhận $\overrightarrow{u}=-\left( -4;2;-3 \right)$ là một véc-tơ chỉ phương}
	\loigiai{
		\begin{itemchoice}
			\itemch Sai. Thay $M\left( 7;-3;-1 \right)$ vào đường thẳng $d$, ta có $$\heva{
				& 7=3+4t \\ 
				& -3=-1-2t \\ 
				& -1=-2+3t 
			}\Rightarrow \heva{
				& t=1 \\ 
				& t=1 \\ 
				& t=\frac{1}{3} \
			}\Rightarrow M\left( 7;-3;-1 \right)\notin d.$$
			
			\itemch Đúng. Thay $N\left( -1;1;-5 \right)$ vào đường thẳng $d$, ta có $$\heva{
				& -1=3+4t \\ 
				& 1=-1-2t \\ 
				& -5=-2+3t  
			}\Rightarrow \heva{
				& t=-1 \\ 
				& t=-1 \\ 
				& t=-1 
			}\Rightarrow M\left( 7;-3;-1 \right)\in d.$$
			
			\itemch Đúng. Vì một véc-tơ chỉ phương của đường thẳng $d$ là $\overrightarrow{u}=\left( 4;-2;3 \right)$.
			
			\itemch Đúng.Vì  $\overrightarrow{u}=\left( -4;2;-3 \right)=-\left( 4;-2;3 \right)$.
			
		\end{itemchoice}
	}
\end{ex}
\begin{ex}%[2H5N2-3]
	Trong không gian $Oxyz$, cho đường thẳng $d \colon \dfrac{x-1}{2}=\dfrac{y-2}{-1}=\dfrac{z-3}{2}$. Các mệnh đề sau đây đúng hay sai?
	\choiceTF
	{ Điểm $Q\left( 2;-1;2 \right)$ thuộc đường thẳng $d$}
	{ \True Điểm $P\left( 1;2;3 \right)$ thuộc đường thẳng $d$}
	{ Điểm $M\left( -1;-2;-3 \right)$ thuộc đường thẳng $d$}
	{ Điểm $N\left( -2;1;-2 \right)$ thuộc đường thẳng $d$}
	\loigiai{
		\begin{itemchoice}
			\itemch Sai. Vì tọa độ $Q$ không thỏa phương trình $d$.
			\itemch Đúng. Vì tọa độ $P$ thỏa phương trình $d$.
			\itemch Sai. Vì tọa độ $M$ không thỏa phương trình $d$.
			\itemch Sai. Vì tọa độ $N$ không thỏa phương trình $d$.
		\end{itemchoice}
	}
\end{ex}
\begin{ex}%[2H5N2-3]
	Trong không gian $Oxyz$, cho đường thẳng $d\colon \heva{& x=1+2t \\ 
		& y=3-t \\ 
		& z=1-t}$. Các mệnh đề sau đây đúng hay sai?
	\choiceTF
	{ Điểm $M\left( -3;5;3 \right)$ không thuộc đường thẳng $d$}
	{ \True Điểm $N\left( 1;3;-1 \right)$ không thuộc đường thẳng $d$}
	{ \True Điểm $P\left( 3;5;3 \right)$ không thuộc đường thẳng $d$}
	{ \True Điểm $Q\left( 1;2;-3 \right)$ không thuộc đường thẳng $d$}
	\loigiai{
		\begin{itemchoice}
			\itemch Sai. Vì tọa độ $M$ thỏa phương trình $d$.
			\itemch Đúng. Vì tọa độ $N$ không thỏa phương trình $d$.
			\itemch Đúng. Vì tọa độ $P$ không thỏa phương trình $d$.
			\itemch Đúng. Vì tọa độ $Q$ không thỏa phương trình $d$.
		\end{itemchoice}
	}
\end{ex}
\begin{ex}%[2H5H2-3]
	Trong không gian $Oxyz$, cho ba điểm $A(1;2;0)$,$B(1;1;2)$ và $C(2;3;1)$. Các mệnh đề sau đây đúng hay sai?
	\choiceTF
	{ \True Đường thẳng đi qua $A$ và song song với $BC$ có phương trình là $\dfrac{x-1}{1}=\dfrac{y-2}{2}=\dfrac{z}{-1}$}
	{\True Đường thẳng đi qua hai điểm $B$, $C$ có phương trình là $\dfrac{x-1}{1}=\dfrac{y-1}{2}=\dfrac{z-2}{-1}$}
	{ Điểm $M\left( 2;3;1 \right)$ không thuộc đường thẳng $BC$}
	{\True  Điểm $N\left( 3;5;0 \right)$ không thuộc đường thẳng $BC$}
	\loigiai{
		\begin{itemchoice}
			\itemch Đúng. Gọi $d$ là phương trình đường thẳng qua $A\left( 1;2;0 \right)$ và song song với $BC$.\\
			Ta có $\overrightarrow{BC}=\left( 1;2;-1 \right)\Rightarrow d \colon \dfrac{x-1}{1}=\dfrac{y-2}{2}=\dfrac{z}{-1}$.
			\itemch Đúng. Đường thẳng đi $B$ có vecto chỉ phương $\overrightarrow{BC}=\left( 1;2;-1 \right)$ có phương trình chính tắc là $\dfrac{x-1}{1}=\dfrac{y-1}{2}=\dfrac{z-2}{-1}.$
			\itemch Sai. Vì tọa độ $M$ thỏa phương trình $BC$.
			\itemch  Sai. Vì tọa độ $N$ thỏa phương trình $BC$.
		\end{itemchoice}	
	}
\end{ex}
\begin{ex}%[2H5H2-3]
	Trong không gian $Oxyz$, cho điểm $M(1;2;-1)$ và mặt phẳng $(P)\colon 2x+y-3z+1=0$. Các mệnh đề sau đây đúng hay sai?
	\choiceTF
	{ Đường thẳng đi qua $M$ và vuông góc với $(P)$ có phương trình là $\dfrac{x-1}{2}=\dfrac{y-2}{1}=\dfrac{z+1}{1}$}
	{\True  Đường thẳng đi qua $M$ và vuông góc với $(P)$ có phương trình là $\dfrac{x-1}{2}=\dfrac{y-2}{1}=\dfrac{z+1}{-3}$}
	{\True  Đường thẳng đi qua $M$ và vuông góc với ${(P)}$ có phương trình là $\dfrac{x-1}{-2}=\dfrac{y-2}{-1}=\dfrac{z+1}{3}$}
	{Đường thẳng đi qua $M$ và vuông góc với $(P)$ có phương trình là $\dfrac{x+1}{2}=\dfrac{y+2}{1}=\dfrac{z-1}{-3}$}
	\loigiai{
		\begin{itemchoice}
			\itemch Sai. Gọi $(\Delta)$ là đường thẳng cần tìm. Vì đường thẳng $(\Delta)$ vuông góc với mặt phẳng $(P)$ nên véc-tơ chỉ phương của $(\Delta)$ là $\overrightarrow{u_{\Delta}}=\overrightarrow{n_{P}}=(2;1;-3)$.
			\itemch Đúng. Phương trình chính tắc của đường thẳng $(\Delta)$ đi qua điểm $M(1 ;2;-1)$ và có véc-tơ chỉ phương $\overrightarrow{u_{\Delta}}=(2;1;-3)$
			là $\dfrac{x-1}{2}=\dfrac{y-2}{1}=\dfrac{z+1}{-3}$.
			\itemch Đúng. Vì $\overrightarrow{{u}_{\Delta }}=\overrightarrow{{n}_{P}}=(2;1;-3)=-(-2;-1;3)$.
			\itemch Sai. Vì đường thẳng đi qua $M$ và vuông góc với $(P)$ có phương trình là $\dfrac{x-1}{-2}=\dfrac{y-2}{-1}=\dfrac{z+1}{3}$.
		\end{itemchoice}
	}
\end{ex}

%Câu 4
\Closesolutionfile{ans}
\indapan{3}{ans/ans-2C5B2CD1-D1-DS}

\Opensolutionfile{ans}[ans/ans-2C5B2CD1-D1-KQ]
\TNSA
\begin{ex}%[2H5N2-2]
	Trong không gian với hệ tọa độ $Oxyz$, cho hai điểm $M\left( 1;-2;1 \right)$, $N\left( 0;1;3 \right)$. Một véc-tơ chỉ phương của đường thẳng qua hai điểm $M$, $N$ có dạng $\overrightarrow{u}=(a;b;2)$. Tìm $a+b.$
	\shortans{$2$}
	\loigiai{ 
		Ta có $\overrightarrow{MN}=\left(-1;3;2 \right)$. Véc-tơ chỉ phương của đường thẳng qua hai điểm $M$, $N$ là $\overrightarrow{MN}=\left(-1;3;2\right)$.\\
		Suy ra $a+b=2.$}
\end{ex}
\begin{ex}%[2H5H2-2]
	Trong không gian $Oxyz$, cho ba điểm $B\left( 1;1;1 \right)$, $C\left( 3;4;0 \right)$. Tìm véc-tơchỉ phương của đường thẳng $\Delta $ song song với $BC$ có dạng $(a;b;-1)$. Tìm $a+b$.
	\shortans{$5$}
	\loigiai{ 
		Ta có $\overrightarrow{BC}=\left(2;3;-1 \right)$, đường thẳng $\Delta $ song song với $BC$  nên có véc-tơ chỉ phương cùng phương với $\overrightarrow{BC}$.\\
		Suy ra $a+b=5.$}
\end{ex}

%%23
\begin{ex}%[2H5H2-2]
	Trong không gian $Oxyz$, cho mặt phẳng $(P) \colon x-3y+2z+1=0$.  Một véc-tơ chỉ phương của đường thẳng $\Delta $ vuông góc với mặt phẳng $\left( P \right)$ có dạng $(a;b;2)$. Tìm $a+b$.
	\shortans{$-2$}
	\loigiai{ 
		Đường thẳng $\Delta $ vuông góc với $( P )$ nên có véc-tơ chỉ phương $\overrightarrow{u}=\overrightarrow{n}_{P}=\left( 1;-3;2 \right)$.\\
		Suy ra $a+b=-2.$}
\end{ex}

%%24

\begin{ex}%[2H5V2-2]
	Trong không gian $Oxyz$, cho hai mặt phẳng $( P) \colon 3x-2y-z+2024=0$ và $(Q) \colon x-2y+2025=0$. Một véc-tơ chỉ phương của đường thẳng $\Delta $ song song với hai mặt phẳng $\left( P \right)$ và $\left( Q \right)$ có dạng $(a;1;c)$. Tìm $a+c$.
	\shortans{$-6$}
	\loigiai{ 
		Đường thẳng $\Delta $ song song với hai mặt phẳng $(P)$ và $(Q)$ nên có véc-tơ chỉ phương
		$$\overrightarrow{u}=\left[ \overrightarrow{n}_{P},\overrightarrow{n}_{Q} \right]=\left( -2;1;-4 \right).$$
		Suy ra $a+c=-6.$}
\end{ex}
%%25
\begin{ex}%[2H5N2-2]
	Trong không gian $Oxyz$, cho mặt phẳng $(P) \colon x+3y-2z-2024=0$ và $\overrightarrow{a}=\left( 1;1;0 \right)$. Một véc-tơ chỉ phương của đường thẳng $\Delta $ song song với mặt phẳng $(P)$ và song song véc-tơ $\overrightarrow{a}$ có dạng $(a;1;c)$. Tìm $a+c$.
	\shortans{$0$}
	\loigiai{ 
		Đường thẳng $\Delta $ song song với mặt phẳng $(P)$ và song song véc-tơ $\overrightarrow{a}$ nên có véc-tơ chỉ phương
		$$\overrightarrow{u}=\left[ \overrightarrow{n}_{P},\overrightarrow{a} \right]=( 2;2;-2)=2(1;1;-1).$$
		Suy ra $a+c=0.$}
\end{ex}
\Closesolutionfile{ans}
\indapan{6}{ans/ans-2C5B2CD1-D1-KQ}
\begin{dang}{XÉT VỊ TRÍ TƯƠNG ĐỐI CỦA HAI ĐƯỜNG THẲNG}
	\textbf{Để xét vị trí tương đối đường thẳng ta có hai cách sau:}
	\begin{itemize}
		\item{\textbf{Cách 1}}
		Cho hai đường thẳng $\Delta _1$, $\Delta _2$ lần lượt đi qua các điểm $M_1$, $M_2$ và tương ứng có $\overrightarrow{u}_1=(a_1;b_1;c_1)$, $\overrightarrow{u}_2=(a_2;b_2;c_2)$ là hai véc-tơ chỉ phương. Khi đó, ta có:
		\begin{itemize}
			\item [$\bullet $] $\Delta_1 \equiv \Delta_2 \Leftrightarrow \heva{&\overline{u}_1, \overline{u}_2 & \text { cùng phương } \\&\overline{u}_1, \overline{M}_1 M_2 & \text { cùng phương }} \Leftrightarrow\heva{&\left[\overrightarrow{u}_1, \overrightarrow{u}_2\right]=\overrightarrow{0} \\& \left[\overrightarrow{u}_1, \overrightarrow{M_1 M_2}\right]=\overrightarrow{0}.}$
			\item [$\bullet $] $\Delta_1 \parallel \Delta_2 \Leftrightarrow \heva{&\overline{u}_1, \overline{u}_2 & \text { cùng phương } \\ &\overrightarrow{u}_1, \overline{M_1 M_2} & \text { không cùng phương }} \Leftrightarrow \heva{&{\left[\overrightarrow{u}_1, \overrightarrow{u}_2\right]=\overrightarrow{0}} \\ &\left[\overrightarrow{u}_1, \overline{M_1 M_2}\right] \neq \overrightarrow{0}.}$
			\item [$\bullet $] $\Delta_1$ cắt $\Delta_2 \Leftrightarrow \heva{&\overrightarrow{u}_1, \overrightarrow{u}_2 & \text { không cùng phương } \\ &\overrightarrow{u}_1, \overrightarrow{u}_2, \overline{M_1 M_2} & \text { đồng phẳng }} \Leftrightarrow\heva{& \left[\overrightarrow{u}_1, \overrightarrow{u}_2\right] \neq 0 \\ &\left[\overrightarrow{u}_1, \overrightarrow{u}_2\right] \overline{M_1 M_2} \neq 0.}$
			\item [$\bullet $] $\Delta_1$ và $\Delta_2$ chéo nhau $\Leftrightarrow\left[\overline{u}_1, \overline{u}_2\right] \overline{M_1 M_2} \neq 0$.
		\end{itemize}
		\item{\textbf{Cách 2}}
		Cho hai đường thẳng $\Delta_1, \Delta_2$ tương ứng có $\overrightarrow{u}_1=\left(a_1 ; b_1 ; c_1\right), \overrightarrow{u}_2=\left(a_2 ; b_2 ; c_2\right)$ là hai véc-tơ chỉ phương và có phương trình tham số:
		$$
		\Delta_1 \colon \heva{&
			x=x_1+a_1 t_1 \\
			&y=y_1+b_1 t_1 \\
			&z=z_1+c_1 t_1}\left(t_1 \in \mathbb{R}\right),\quad \Delta_2 \colon \heva{&
			x=x_2+a_2 t_2 \\
			&y=y_2+b_2 t_2 \\
			&z=z_2+c_2 t_2} \quad\left(t_2 \in \mathbb{R}\right).$$
		Khi đó 
		\begin{itemize}
			\item  [$\bullet $] $\Delta_1 \equiv \Delta_2 \Leftrightarrow \overrightarrow{u}_1$ cùng phương với $\overrightarrow{u}_2$ và hệ (*) vô nghiệm.
			\item [$\bullet $] $\Delta_1 \parallel \Delta_2 \Leftrightarrow$ hệ (*) có vô số nghiệm.
			\item [$\bullet $] $\Delta_1$ cắt $\Delta_2 \Leftrightarrow$ hệ (*) có nghiệm duy nhất.
			\item [$\bullet $] $\Delta_1$ và $\Delta_2$ chéo nhau $\Leftrightarrow \overrightarrow{u}_1$ không cùng phương với $\overrightarrow{u}_2$ và hệ (*) vô nghiệm.			
		\end{itemize}
	\end{itemize}
\end{dang}
\Opensolutionfile{ans}[ans/ans-2-C5B2D2]
\TN
%Câu 1.
\begin{ex}%[2H5H2-4]%Câu 1
	Trong không gian với hệ tọa độ $Oxyz$, hai đường thẳng $d \colon \heva{&x=-1+12t\\&y=2+6t\\&z=3+3t}$ và  $d' \colon \heva{&x=7+8t\\&y=6+4t\\&z=5+2t}$ có vị trí tương đối là
	\choice
	{\True trùng nhau}
	{song song}
	{chéo nhau}
	{cắt nhau}
	\loigiai
	{Đường thẳng $d$ có véc-tơ chỉ phương là  $\overrightarrow{u}=(12;6;3)$ và đi qua điểm $M(-1;2;3)$.\\
		Và đường thẳng $d'$ có véc-tơ chỉ phương là  $\overrightarrow{u'}=(8;4;2)$ và đi qua điểm $M'(7;6;5)$.\\
		Từ đó ta có $\overrightarrow{MM'}=(8;4;2)=\overrightarrow{u'}$ nên $d$ trùng với $d'$.}
\end{ex}
%Câu 2.
\begin{ex}%[2H5H2-4]%Câu 2
	Trong không gian với hệ tọa độ $Oxyz$, cho hai đường thẳng $d \colon \dfrac{x-1}{-2}=\dfrac{y+2}{1}=\dfrac{z-4}{3}$ và  $ d' \colon \heva{&x=1+t\\&y=-t\\&z=-2+3t}$ có vị trí tương đối là
	\choice
	{trùng nhau}
	{song song}
	{chéo nhau}
	{\True  cắt nhau}
	\loigiai
	{Ta có 	$d$ có véc-tơ chỉ phương là  $\overrightarrow{u}=(-2;1;3)$ và đi qua điểm $M(1;-2;4)$.\\
		Và $d'$ có véc-tơ chỉ phương là  $\overrightarrow{u'}=(1;-1;3)$ và đi qua điểm $M'(1;0;-2)$.\\
		Từ đó ta có $\overrightarrow{MM'}=(-2;2;-6)$ và $ \left[\overrightarrow{u},\overrightarrow{u'} \right]=(6;9;1)\ne \overrightarrow{0}$. \\
		Ta cũng tính được $ \overrightarrow{MM'} \cdot \left[\overrightarrow{u},\overrightarrow{u'} \right]=0$. Do đó $d$ và $d'$ cắt nhau.}
\end{ex}
\begin{ex}%[2H5H2-4]Câu 1.
	Trong không gian $Oxyz$, cho hai đường thẳng $d\colon \dfrac{x-2}{4}=\dfrac{y}{-6}=\dfrac{z+1}{-8}$ và $d'\colon \dfrac{x-7}{-6}=\dfrac{y-2}{9}=\dfrac{z}{12}$. Trong các mệnh đề sau, mệnh đề nào đúng khi nói về vị trí tương đối của hai đường thẳng trên?
	\choice
	{\True song song}
	{trùng nhau}
	c{héo nhau}
	{cắt nhau}
	\loigiai{
		$d$ có VTCP $\overrightarrow{u}=(4;-6;-8)$ và đi qua $M(2;0;-1)$.\\
		$d'$ có VTCP $\overrightarrow{u'}=(-6;9;12)$ và đi qua $M'(7;2;0)$.\\
		Từ đó ta có $\overrightarrow{MM'}=(5;2;1)$ và $\left[\overrightarrow{u},\overrightarrow{u'}\right]=\overrightarrow{0} $.\\
		Lại có $\left[\overrightarrow{u},\overrightarrow{MM'}\right]=\overrightarrow{0} $.\\
		Suy ra $d$ song song với $d'$.}
\end{ex}
\begin{ex}%[2H5H2-4]Câu 2.	
	Hai đường thẳng $d\colon \heva{& x=-1+12t \\ & y=2+6t\\&z=3+3t}$ và $d'\colon\heva{& x=7+8t \\ & y=6+4t\\&z=5+2t}$ có vị trí tương đối là
	\choice
	{\True trùng nhau}
	{song song}
	{chéo nhau}
	{cắt nhau}
	\loigiai{
		$d$ có VTCP $\overrightarrow{u}=(12;6;3)$ và đi qua $M(-1;2;3)$.\\
		$d'$ có VTCP $\overrightarrow{u'}=(8;4;2)$ và đi qua $M'(7;6;5)$.\\
		Từ đó ta có $\overrightarrow{MM'}=(8;4;2)$ 
		Suy ra $\left[\overrightarrow{u},\overrightarrow{MM'}\right]=\overrightarrow{0} $ và $\left[\overrightarrow{u},\overrightarrow{u'}\right]=\overrightarrow{0} $.\\
		Suy ra $d$ trùng với $d'$.}
\end{ex}
\begin{ex}%[2H5H2-4]Câu 3.	
	Trong không gian $ABCD.A'B'C'D'$, hai đường thẳng $A$ và $B(a;0;0)$ có vị trí tương đối là
	\choice
	{trùng nhau}
	{song song}
	{chéo nhau}
	{\True cắt nhau}
	\loigiai{
		$D(0;a;0)$ có VTCP $A'(0;0;b)$ và đi qua $(a > 0,b > 0)$
		$M$ có VTCP $CC'$ và đi qua $\dfrac{a}{b}$
		Từ đó ta có
		$(A'BD)$
		$\left(MBD\right)$ và $\dfrac{1}{2}$
		Suy ra $ - 1$ cắt $\dfrac{1}{3}$.}
\end{ex}
\Closesolutionfile{ans}
\indapan{10}{ans/ans-2-C5B2D2}
\TNTF
\Opensolutionfile{ans}[ans/ans-2-C5B2D2-DS]
%Câu 3.
\begin{ex}%[2H5H2-4]
	Trong không gian $Oxyz$, cho hai đường thẳng $d \colon \dfrac{x-1}{2}=\dfrac{y-7}{1}=\dfrac{z-3}{4}$ và  $d' \colon \dfrac{x-6}{3}=\dfrac{y+1}{-2}=\dfrac{z+2}{1}$. Các mệnh đề sau đây đúng hay sai?
	\choiceTF
	{Đường thẳng $d$ song song đường thẳng $d'$}
	{Đường thẳng $d$ trùng đường thẳng $d'$}
	{\True  Đường thẳng $d$ cắt đường thẳng $d'$}
	{Đường thẳng $d$ chéo đường thẳng $d'$}
	\loigiai{
		Ta có $d$ có véc-tơ chỉ phương là  $\overrightarrow{u}=(2;1;4)$ và đi qua điểm $M(1;7;3)$.\\
		Và $d'$ có véc-tơ chỉ phương là  $\overrightarrow{u'}=(3;-2;1)$ và đi qua điểm $M'(6;-1;-2)$.\\
		Từ đó ta có $\overrightarrow{MM'}=(5;-8;-5)$ và $ \left[\overrightarrow{u},\overrightarrow{u'} \right]=(9;10;-7)\ne \overrightarrow{0}$. \\
		Ta cũng tính được $ \overrightarrow{MM'} \cdot \left[\overrightarrow{u},\overrightarrow{u'} \right]=0$. Do đó $d$ và $d'$ cắt nhau.}
\end{ex}
%Câu 4
\begin{ex}%[2H5H2-4]
	Trong không gian $Oxyz$, cho hai đường thẳng $d \colon \dfrac{x-1}{-2}=\dfrac{y+2}{1}=\dfrac{z-4}{3}$ và  $ d' \colon \heva{&x=-1+t\\&y=-t\\&z=-2+3t}$. Các mệnh đề sau đây đúng hay sai?
	\choiceTF
	{\True  Tọa độ giao điểm của $d$ và $d'$ là $I(1;-2;4)$}
	{Tọa độ giao điểm của $d$ và $d'$ là $I(1;2;4)$}
	{\True  Đường thẳng $d$ cắt đường thẳng $d'$}
	{Đường thẳng $d$ chéo đường thẳng $d'$}
	\loigiai{
		Thay phương trình $d'$ và phương trình $d$, ta được $$\dfrac{-1+t-1}{-2}=\dfrac{-t+2}{1}=\dfrac{-2+3t-4}{3}\Leftrightarrow t=2.$$
		Suy ra giao điểm của $d$ và $d'$ là $I(1;-2;4)$.}
\end{ex}
%Câu 5
\begin{ex}%[2H5V2-4]
	Trong không gian $Oxyz$, cho bốn đường thẳng $d_1 \colon \dfrac{x-3}{1}=\dfrac{y+1}{-2}=\dfrac{z+1}{1}$, $d_2 \colon \dfrac{x}{1}=\dfrac{y}{-2}=\dfrac{z-1}{1}$, $d_3 \colon \dfrac{x-1}{2}=\dfrac{y+1}{1}=\dfrac{z-1}{1}$ và $d_4 \colon \dfrac{x}{1}=\dfrac{y-1}{-1}=\dfrac{z-1}{1}$. Các mệnh đề sau đây đúng hay sai?
	\choiceTF
	{\True  Hai đường thẳng $d_1$ và $d_2$ song song với nhau}
	{Đường thẳng $d_3$ cắt đường thẳng $d_2$}
	{Đường thẳng $d_4$ không cắt đường thẳng $d_1$}
	{Đường thẳng $d_3$ cắt đường thẳng $d_1$}
	\loigiai{
		Ta có 	$d_1$ có véc-tơ chỉ phương là  $\overrightarrow{u_1}=(1;-2;1)$ và đi qua điểm $M_1(3;-1;-1)$.\\
		Và $d_2$ có véc-tơ chỉ phương là  $\overrightarrow{u_2}=(1;-2;1)$ và đi qua điểm $M_2(0;0;1)$.\\
		Do $\overrightarrow{u_1}=\overrightarrow{u_2}$ và $M_1 \notin d_2$ nên hai đường thẳng $d_1$ và $d_2$ song song với nhau.\\
		Ta có $\overrightarrow{M_1M_2}=(-3;1;2)$ và $ \left[\overrightarrow{M_1M_2},\overrightarrow{u_1} \right]=(5;5;5)=5(1;1;1)$. \\
		Gọi $(\alpha)$  là mặt phẳng chứa $d_1$ và $d_2$, khi đó $(\alpha)$  có một véc-tơ pháp tuyến là $\overrightarrow{n}=(1;1;1)$.\\
		Phương trình mặt phẳng $(\alpha)$ là $x+y+z-1=0$.\\
		Gọi $A =  d_3 \cap(\alpha)  $ thì $A(1;-1;1) $, điểm $A$ không thuộc cả $d_1$ và $d_2$ nên $d_3$ không cắt cả hai đường thẳng $d_1$ và $d_2$.\\
		Gọi $B =  d_4 \cap(\alpha)  $ thì $B(-1;2;0)\notin d_1$ nên $d_4$ không cắt $d_1$.}
\end{ex}
\Closesolutionfile{ans}
\indapan{3}{ans/ans-2-C5B2D2-DS}

\Opensolutionfile{ans}[ans/ans-2-C5B2D2-KQ]
\TNSA
\begin{ex}%[2H5H2-4]
	Trong không gian $Oxyz$, gọi $I(a;b;c)$ là tọa độ giao điểm của hai đường thẳng $\Delta_1 \colon \dfrac{x-1}{2}=\dfrac{y+1}{2}=\dfrac{z}{3}$ và  $\Delta_2 \colon \heva{& x=3-t\\& y=3-2t\\&z=-2+t}$. Tìm $a+b+c$.
	\shortans{$0$}
	\loigiai{
		Giao điểm của $\Delta_1$ và $\Delta_2$ thỏa mãn $$\heva{& x=3-t\\& y=3-2t\\&z=-2+t\\& \dfrac{x-1}{2}=\dfrac{y+1}{2}=\dfrac{z}{3}} \Leftrightarrow \heva{& x=3-t\\& y=3-2t\\&z=-2+t\\& \dfrac{3-t-1}{2}=\dfrac{3-2t+1}{2}=\dfrac{-2+t}{3}}  \Leftrightarrow \heva{&x=1\\&y=-1\\&z=0\\&t=2.} $$
		Suy ra $a+b+c=0.$
	}
\end{ex}
\begin{ex}%[2H5H2-4]
	Trong không gian $Oxyz$, biết hai đường thẳng $d_1 \colon \dfrac{x}{1}=\dfrac{y}{-2}=\dfrac{z-1}{1}$ và  $d_2 \colon \dfrac{x-1}{2}=\dfrac{y+1}{1}=\dfrac{z-1}{1}$ cắt nhau tại $I(a;b;c)$. Tính giá trị $a+b+c$.
	\shortans{$1$}
	\loigiai{
		Giao điểm của $d_1$ và $d_2$ thỏa hệ 
		$$
		\heva{&\dfrac{x}{1}=\dfrac{y}{-2}=\dfrac{z-1}{1}\\
			& \dfrac{x-1}{2}=\dfrac{y+1}{1}=\dfrac{z-1}{1}} \Leftrightarrow \heva{& -2x-y=0\\&x-z=0\\&x-2y=0\\&x-2z=0} \Leftrightarrow \heva{&x=-\dfrac{1}{5}\\&y=\dfrac{2}{5}\\&z=\dfrac{4}{5}.}
		$$
		Vậy $a+b+c=1$.
	}
\end{ex}
\Closesolutionfile{ans}
\indapan{6}{ans/ans-2-C5B2D2-KQ}
\begin{dang}{GÓC GIỮA HAI ĐƯỜNG THẲNG}
	Cho hai đường thẳng có hai vectơ chỉ phương lần lượt là $\overrightarrow {u_1}=(a_1;{b_1};{c_1})$, $\overrightarrow{ u_2}=(a_2;{b_2};{c_2})$. Khi đó, ta có
	$$\cos\left(\Delta_1,\Delta_2\right)=\left|\cos\left(\overrightarrow {u_1}, \overrightarrow{ u_2} \right)\right|=\dfrac{\left|\overrightarrow {u_1}\cdot \overrightarrow{u_2}\right|}{\left|\overrightarrow {u_1}\right|\cdot \left|\overrightarrow {u_2}\right|}=\dfrac{\left|a_1a_2+b_1b_2+c_1c_2 \right| }{\sqrt{a_1^2+b_1^2+c_1^2}\cdot \sqrt{a_2^2+b_2^2+c_2^2}}.$$
	\textbf{Chú ý :}\\
	• $\Delta_1\bot{\Delta_2}\Leftrightarrow{\overrightarrow {u_1}}\cdot \overrightarrow {u_2}=0\Leftrightarrow{a_1}{a_2}+b_1b_2+c_1c_2=0.$\\
	• Hai đường thẳng song song hoặc trùng với nhau thì góc giữa chúng là $0^\circ$.\\
	\textbf{TÍNH GÓC GIỮA ĐƯỜNG THẲNG VỚI MẶT PHẲNG}\\
	Cho đường thẳng $\Delta $ có véc-tơ chỉ phương $\overrightarrow {u}=(a;b;c)$ và mặt phẳng $(P)$ có véc-tơ pháp tuyến $\overrightarrow {n}=(A; B; C)$. Khi đó, ta có
	$$\sin\left(\Delta ,(P)\right)=\left|\cos\left(\overrightarrow {u}, \overrightarrow{n}\right)\right|=\dfrac{\left|\overrightarrow {u} \cdot\overrightarrow {n}\right|}{\left|\overrightarrow {u}\right| \cdot \left| \overrightarrow {n}\right|}=\dfrac{\left|aA+bB+cC\right|}{\sqrt{a^2+b^2+c^2}\cdot \sqrt{A^2+B^2+C^2}}.$$
	\textbf{Chú ý :}\\
	• Đường thẳng song song hoặc trùng với mặt phẳng thì góc giữa chúng là $0^0$.\\
	\textbf{TÍNH GÓC GIỮA HAI MẶT PHẲNG}\\
	Cho hai mặt phẳng $(P_1)$, $(P_2)$ có hai véc-tơ pháp tuyến lần lượt là \break  $\overrightarrow{n_1}=(A_1;B_1;C_1)$, $ \overrightarrow{n_2}=(A_2;B_2;C_2)$. Khi đó, ta có
	$$ \cos\left((P_1),(P_2)\right)=\left|\cos\left( \overrightarrow{n_1} , \overrightarrow{n_2} \right)\right|=\dfrac{\left|\overrightarrow{n_1}\cdot \overrightarrow{n_2}\right|}{\left|\overrightarrow{n_1}\right|\cdot \left|\overrightarrow {n_2}\right|}=\dfrac{\left|A_1A_2+B_1B_2+C_1C_2\right|}{\sqrt{A_1^2+B_1^2+C_1^2}\cdot \sqrt{A_2^2+B_2^2+C_2^2}}.$$
	\textbf{Chú ý :}\\
	• Hai mặt phẳng song song hoặc trùng với nhau thì góc giữa chúng là $0^\circ$.
\end{dang}

\Opensolutionfile{ans}[ans/ans-goc1]
\TN
%Câu 1.
\begin{ex}%[2H5N2-7]
	Gọi $\alpha $ là góc giữa hai đường thẳng $AB$, $CD$. Khẳng định nào sau đây đúng?
	\choice
	{\True $\cos\alpha=\dfrac{\left|\overrightarrow{AB}\cdot\overrightarrow{CD}\right|}{\left|\overrightarrow{AB}\right|\cdot \left|\overrightarrow{CD}\right|}$}
	{$\cos\alpha=\dfrac{\overrightarrow{AB}\cdot\overrightarrow{CD}}{\left|\overrightarrow{AB}\right|\cdot \left|\overrightarrow{CD}\right|}$}
	{$\cos\alpha=\dfrac{\left|\overrightarrow{AB}\cdot\overrightarrow{CD}\right|}{\left|\left[\overrightarrow{AB},\overrightarrow{CD}\right]\right|}$}
	{$\cos\alpha=\dfrac{\left|\left[\overrightarrow{AB}\cdot \overrightarrow{CD}\right]\right|}{\left|\overrightarrow{AB}\right|\cdot \left|\overrightarrow{CD}\right|} $}
	\loigiai{
		Ta có $\cos\alpha=\dfrac{\left|\overrightarrow{AB}\cdot\overrightarrow{CD}\right|}{\left|\overrightarrow{AB}\right|\cdot \left|\overrightarrow{CD}\right|}$.}
	
\end{ex}

\begin{ex}%[2H5H2-5]
	Cho hai đường thẳng $d_1\colon \heva{&
		x=2+t\\&y=-1+t\\&z=3}$ và $d_2\colon \heva{& x=1-t\\&y=2\\& z=-2+t}$. Góc giữa hai đường thẳng $d_1$ và $d_2$ là
	\choice
	{$30^\circ $}
	{$120^\circ $}
	{$150^\circ $}
	{\True $60^\circ $} 
	\loigiai{
		Gọi $\overrightarrow{u_1}$, $\overrightarrow{u_2}$ lần lượt là véc-tơ chỉ phương của đường thẳng $d_1$ và $d_2$.\\
		Ta có $\overrightarrow{u_1}=(1;1;0)$; $\overrightarrow{u_2}=(-1;0;1)$.\\
		Áp dụng công thức ta có $$ \cos\left(d_1,d_2\right)=\left|\cos\left(\overrightarrow{u_1},\overrightarrow{u_2}\right)\right|=\dfrac{\left|\overrightarrow{u_1}\cdot \overrightarrow{u_2}\right|}{\left|\overrightarrow{u_1}\right| \cdot \left|\overrightarrow{u_2}\right|}=\dfrac{\left|-1\right|}{\sqrt{1+1}\cdot\sqrt{1+1}}=\dfrac{1}{2}.$$
		$\Rightarrow\left(d_1,d_2\right)=60^\circ $.
	}
\end{ex}
\begin{ex}%[2H5H2-7]
	Cho đường thẳng $\Delta \colon \dfrac{x}{1}=\dfrac{y}{-2}=\dfrac{z}{1}$ và mặt phẳng $(P) \colon 5x+11y+2z-4=0$. Góc giữa đường thẳng $\Delta $ và mặt phẳng $(P)$ là
	\choice
	{$60^\circ $}
	{$-30^\circ $}
	{\True $30^\circ $}
	{$-60^\circ $}
	\loigiai{
		Gọi $\overrightarrow{u}$, $\overrightarrow{n}$ lần lượt là véc-tơ chỉ phương, pháp tuyến của đường thẳng $\Delta $ và mặt phẳng $(P)$ thì $\overrightarrow{u}=\left(1;-2;1\right)$, $\overrightarrow{ n}=\left(5;11;2\right).$\\
		Áp dụng công thức ta có $$\sin\left(\Delta ,(P)\right)=\left|\cos\left(\overrightarrow{u} ,\overrightarrow{n}\right)\right|=\dfrac{\left|\overrightarrow{u} \cdot \overrightarrow{n}\right|}{\left|\overrightarrow {u}\right|\cdot\left|\overrightarrow {n}\right|}=\dfrac{\left|1\cdot 5-11\cdot 2+1 \cdot2\right|}{\sqrt{5^2+11^2+2^2} \cdot\sqrt{1^2+2^2+1^2}}=\dfrac{1}{2}\\
		\Rightarrow\left(\Delta ,(P)\right)=30^\circ.$$
	}
\end{ex}
\begin{ex}%[2H5H2-7]
	Trong không gian với hệ tọa độ $Oxyz$ cho đường thẳng $d \colon \heva{&x=1-t\\&y=2+2t\\&z=3+t}$ và mặt phẳng $(P) \colon x-y+3=0$. Tính số đo góc giữa đường thẳng $d$ và mặt phẳng (P).
	\choice
	{\True $60^\circ$}
	{$30^\circ$}
	{$120^\circ$}
	{$45^\circ$}
	\loigiai{
		Đường thẳng $d$ có véc tơ chỉ phương là $\overrightarrow{u}=\left(-1;2;1\right).$\\
		Mặt phẳng $(P)$ có véc tơ pháp tuyến là $\overrightarrow{ n}=\left(1;-1;0\right)$.\\
		Gọi $\alpha $ là góc giữa đường thẳng $d$ và mặt phẳng $(P)$. Khi đó ta có
		$$\sin\alpha=\dfrac{\left|\overrightarrow{ u} \cdot \overrightarrow{n}\right|}{\left|\overrightarrow{u}\right| \cdot \left| \overrightarrow{ n}\right|}=\dfrac{\left|-1 \cdot 1+2\cdot (-1)+1 \cdot 0\right|}{\sqrt{\left(-1\right)^2+2^2+1^2}\cdot \sqrt{1^2+\left(-1\right)^2+0^2}}=\dfrac{3}{2\sqrt 3}=\dfrac{\sqrt 3}{2}.$$
		Do đó $\alpha=60^\circ$.}
\end{ex}
\begin{ex}%[2H5H2-7]
	Trong không gian $Oxyz$, cho mặt phẳng $(P) \colon -\sqrt 3 x+y+1=0$. Tính góc tạo bởi $(P)$ với trục $Ox$.
	\choice
	{\True $60^\circ$}
	{$30^\circ$}
	{$120^\circ$}
	{$150^\circ$}
	\loigiai{
		Mặt phẳng $(P)$ có véc-tơ pháp tuyến $\overrightarrow{n}=(-\sqrt 3 ;1;0).$\\
		Trục $Ox$ có véc-tơ chỉ phương $\overrightarrow{ i}=(1;0;0).$\\
		Góc tạo bởi $(P)$ với trục $Ox$ là
		$$ \sin((P),Ox)=\left|\cos((P),Ox) \right|=\dfrac{\left|\overrightarrow{n} \cdot \overrightarrow{i}\right|}{\left|\overrightarrow{n}\right| \cdot \left|\overrightarrow{i} \right|}=\dfrac{\left|-\sqrt{3}  \cdot 1+1 \cdot 0+0 \cdot 0\right|}{\sqrt{3+1}\cdot\sqrt 1}=\dfrac{\sqrt 3}{2}.$$
		Vậy góc tạo bởi $(P)$ với trục $Ox$ bằng $60^\circ.$}
\end{ex}
\begin{ex}%[2H5H2-7]
	Cho mặt phẳng $(P)\colon 3x+4y+5z+2=0$ và đường thẳng $d$ là giao tuyến của hai mặt phẳng $(\alpha)\colon x-2y+1=0$, $(\beta)\colon x-2z-3=0$. Gọi $\varphi $ là góc giữa đường thẳng $d$ và mặt phẳng $(P)$. Khi đó
	\choice
	{\True $60^\circ $}
	{$45^\circ $}
	{$30^\circ $}
	{$90^\circ $}
	\loigiai{
		Đường thẳng d có phương trình: $\heva{ & x  =  2t\\&y= \dfrac{1}{2} + t\\&z = -\dfrac{3}{2}+t}, t  \in  R$.\\
		Suy ra véc-tơ chỉ phương của $d$ là $\overrightarrow{u_d}=(2; 1; 1)$.\\
		Ta có $\sin\left(d,(P)\right)=\left|\cos\left(\overrightarrow{u_d},\overrightarrow n\right)\right|=\dfrac{\left|\overrightarrow{u_d}\cdot \overrightarrow n\right|}{\left|\overrightarrow{u_d}\right|\cdot \left|\overrightarrow n\right|}=\dfrac{\left|2\cdot 3  +  1\cdot 4+1\cdot 5\right|}{\sqrt{2^2  +1^2+1^2}\cdot \sqrt{3^2+4^2+5^2}}=\dfrac{\sqrt 3}{2}$.\\
		$\Rightarrow(d,(P))=60^\circ $.}
\end{ex}

\begin{ex}%[2H5H1-6]
	Cho hai mặt phẳng $(\alpha)\colon 2x-y+2z-1=0$  và  $(\beta)\colon x+2y-2z-3=0$. Cosin góc giữa mặt phẳng $(\alpha)$ và mặt phẳng $(\beta)$ bằng
	\choice
	{\True $\dfrac{4}{9}$}
	{$-\dfrac{4}{9}$}
	{$\dfrac{4}{3\sqrt 3}$}
	{$-\dfrac{4}{3\sqrt 3}$}
	\loigiai{
		Gọi $\overrightarrow{n_\alpha}$, $\overrightarrow{n_\beta}$ lần lượt là vectơ pháp tuyến của mặt phẳng $(\alpha)$ và $(\beta)$ thì $\overrightarrow{n_\alpha}=(2;-1;2)$, $\overrightarrow{n_\beta}=(1;2;-2)$.\\
		Áp dụng công thức:
		$$\cos((\alpha),(\beta))=\left|\cos(\overrightarrow{n_\alpha},\overrightarrow{n_\beta})\right|=\dfrac{\left|\overrightarrow{n_\alpha}\cdot \overrightarrow{n_\beta}\right|}{\left|\overrightarrow{n_\alpha}\right|\cdot \left|\overrightarrow{n_\beta}\right|}=\dfrac{\left|2\cdot 1-1\cdot 2-2\cdot 2\right|}{\sqrt{2^2+(-1)^2+2^2}\cdot \sqrt{(1^2+2^2+(-2)^2}}=\dfrac{4}{9}.$$}
\end{ex}
\begin{ex}%[2H5H1-6]
	Hai mặt phẳng nào dưới đây tạo với nhau một góc $60^\circ $?
	\choice
	{$(P)\colon 2x+11y-5z+3=0$ và $(Q)\colon x+2y-z-2=0$}
	{\True $(P)\colon 2x+11y-5z+3=0$ và $(Q)\colon -x+2y+z-5=0$}
	{$(P)\colon 2x-11y+5z-21=0$ và $(Q)\colon 2x+y+z-2=0$}
	{$(P)\colon 2x-5y+11z-6=0$ và $(Q)\colon -x+2y+z-5=0$}
	\loigiai{
		Áp dụng công thức tính góc giữa hai mặt phẳng.
		$$ \cos\left((P),(Q)\right)=\dfrac{\left|\overrightarrow{n_P} \cdot \overrightarrow{n_Q} \right|}{\left|\overrightarrow{n_P} \right| \cdot \left| \overrightarrow{n_Q}  \right|}=\cos 60^\circ=\dfrac{1}{2}.$$}	
\end{ex}
\begin{ex}%[2H5H1-6]
	Tính tổng các giá trị tham số $m$ để mặt phẳng $(P)\colon \left(m+2\right)x+2my-mz+5=0$ và $(Q)\colon mx+(m-3)y+2z-3=0$ hợp với nhau một góc $\alpha=90^\circ$.
	\choice
	{\True $6$}
	{$4$}
	{$8$}
	{$-4$}
	\loigiai{
		Phhương pháp giải: Xác định các vectơ pháp tuyến của mặt phẳng $(P)$ và $(Q)$. Thay các giá trị vào biểu thức để tìm giá trị đúng. Dùng chức năng CALC trong máy tính bỏ túi để hỗ trợ việc tính toán nhanh nhất.\\
		Mặt phẳng $(P)$, $(Q)$ có véc-tơ pháp tuyến lần lượt là $\overrightarrow{n_P}=\left(m+2;2m;-m\right)$, $ \overrightarrow{n_Q}=\left(m;m-3;2\right)$.\\
		Ta có $(P)\perp (Q)$
		\begin{eqnarray*}
			&\Leftrightarrow&  \overrightarrow{n_P} \cdot \overrightarrow{n_Q}=0\\
			&\Leftrightarrow& \left(m+2\right)m+2m\left(m-3\right)-2m=0\\
			&\Leftrightarrow& 3m^2-6m=0\\
			&\Leftrightarrow & \hoac{&m=0\\ &	m=6.}		
		\end{eqnarray*}
	}
\end{ex}
\Closesolutionfile{ans}
\indapan{10}{ans/ans-goc1}
\TNTF
\Opensolutionfile{ans}[ans/ans-goc2-DS]
%Câu 3.
\begin{ex}%[2H5H1-6]
	Trong không gian $Oxyz$, cho hai mặt phẳng $(P)\colon 2x-y+2z+5=0$ và $(Q)\colon x-y+2=0$. Các mệnh đề sau đây đúng hay sai?
	\choiceTF
	{Góc giữa hai mặt phẳng $(P)$ và $(Q)$ bằng $135^\circ$}
	{\True Góc giữa hai mặt phẳng $(P)$ và $(Q)$ bằng $45^\circ$}
	{Hai mặt phẳng $(P)$ và $(Q)$ song song với nhau}
	{\True Điểm $M\left(0;5;0\right)$ thuộc mặt phẳng $(P)$}
	\loigiai{
		Gọi $\alpha $ là góc giữa hai mặt phẳng $(P)$ và $(Q)$.
		$$\cos\alpha=\dfrac{\left|2 \cdot 1-1 \cdot \left(-1\right)+2 \cdot 0\right|}{\sqrt{2^2+\left(-1\right)^2+2^2} \cdot \sqrt{1^2+\left(-1\right)^2+0^2}}=\dfrac{1}{\sqrt 2}.$$ $\Rightarrow\alpha=45^\circ$.\\
		Thay $M\left(0;5;0\right)$ vào mặt phẳng $(P)$ ta có $2 \cdot 0-5+2 \cdot 0+5=0\Rightarrow M \in(P)$.}
\end{ex}
\begin{ex}%[2H5H1-6]
	Trong không gian $Oxyz$, cho mặt phẳng $(Q)\colon x-y-5=0$, và biết hình chiếu của $O$ lên mặt phẳng $(P)$ là $H\left(2;-1;-2\right)$. Các mệnh đề sau đây đúng hay sai?
	\choiceTF
	{Góc giữa hai mặt phẳng $(P)$ và $(Q)$ bằng $135^\circ$}
	{\True Góc giữa hai mặt phẳng $(P)$ và $(Q)$ bằng $45^\circ $}
	{Góc giữa hai mặt phẳng $(P)$ và $(Q)$ bằng $60^\circ $}
	{Góc giữa hai mặt phẳng $(P)$ và $(Q)$ bằng $120^\circ$}
	\loigiai{
		Mặt phẳng $(Q)$ có một véc-tơ pháp tuyến là $\overrightarrow{n_Q}=\left(1;-1;0\right)$.\\
		Hình chiếu của $O$ lên mặt phẳng $(P)$ là $H\left(2;-1;-2\right)$. \\
		Suy ra mặt phẳng $(P)$ qua $H$ và nhận $\overrightarrow{OH}=\left(2;-1;-2\right)$ làm véc-tơ pháp tuyến.\\
		Gọi $\varphi $ là góc giữa hai mặt phẳng $(P)$ và $(Q)$. Ta có
		$$\cos\varphi=\left|\cos\left( \overrightarrow{OH}, \overrightarrow{n_Q}\right)\right|=\dfrac{\left|2+1+0\right|}{\sqrt{4+1+4} \cdot \sqrt{1+1+0}}=\dfrac{\sqrt 2}{2}\Rightarrow\varphi=45^\circ.$$
	}
\end{ex}
\begin{ex}%[2H5H1-6]
	Trong không gian $Oxyz$, cho ba mặt phẳng $(P)\colon 2x-y+2z+3=0$, $(Q)\colon x-y-z-2=1$, $(R)\colon x+2y+2z-2=0$. Gọi $\alpha_1$, $\alpha_2$, $\alpha_3$ lần lượt là góc giữa hai mặt phẳng $(P)$ và $(Q)$, $(Q)$ và $(R)$, $(R)$ và $(P)$. Các mệnh đề sau đây đúng hay sai?
	\choiceTF
	{\True $\alpha_1>\alpha_3>\alpha_2$}
	{$\alpha_2>\alpha_3>\alpha_1$}
	{$\alpha_3>\alpha_2>\alpha_1$}
	{$\alpha_1>\alpha_2>\alpha_3$}
	\loigiai{
		Áp dụng công thức tính góc giữa hai mặt phẳng. Sử dụng máy tính bỏ túi để tính góc rồi so sánh các giá trị đó với nhau.
	}
\end{ex}

\Closesolutionfile{ans}
\indapan{3}{ans/ans-goc2-DS}

\Opensolutionfile{ans}[ans/ans-goc3-TLN]
\TNSA
\begin{ex}%[2H5V1-6]
	Trong không gian với hệ trục tọa độ $Oxyz$, cho điểm $H(2;1;2)$, $H$ là hình chiếu vuông góc của gốc tọa độ $O$ xuống mặt phẳng $(P)$. Tính số đo góc giữa mặt phẳng $(P)$ và mặt phẳng $(Q)\colon x+y-11=0$.
	\shortans{$45^\circ $}
	\loigiai{
		Mặt phẳng $(P)$ qua $O$ và nhận $\overrightarrow{OH}=\left(2;1;2\right)$ làm véc-tơ pháp tuyến.\\
		Mặt phẳng $(Q)\colon x-y-11=0$ có véc-tơ pháp tuyến $\overrightarrow{n}=(1;1;0)$.\\
		Ta có $$\cos\left(\widehat{(P),(Q)}\right)=\dfrac{\left|\overrightarrow{OH}\cdot \overrightarrow n\right|}{OH \cdot \left|\overrightarrow n\right|}=\dfrac{1}{\sqrt 2}\Rightarrow\widehat{\left((P),(Q)\right)}=45^\circ.$$
	}
\end{ex}
\begin{ex}%[Dự án 2025_K12_TL TV]%[Phạm Thị Thanh Thủy]%[2H5V1-6]
	Trong không gian $O x y z$, cho mặt phẳng $(P)$ có phương trình $x-2 y+2 z-5=0$. Xét mặt phẳng $(Q): x+(2 m-1) z+7=0$, với $m$ là tham số thực. Tính tổng tất cả giá trị của $m$ để $(P)$ tạo với $(Q)$ góc $\dfrac{\pi}{4}$.
	\shortans{$5$}
	\loigiai{		
		Mặt phẳng $(P),(Q)$ có vectơ pháp tuyến lần lượt là $\overrightarrow{n_p}=(1;-2;2), \overrightarrow{n_Q}=(1;0;2m-1)$.\\		
		Vì $(P)$ tạo với $(Q)$ góc $\dfrac{\pi}{4}$ nên
		\begin{eqnarray*}
			\cos \dfrac{\pi}{4}=\left|\cos\left(\overrightarrow{n_p}; \overrightarrow{n_Q}\right)\right|&\Leftrightarrow& \dfrac{1}{\sqrt{2}}=\frac{|1+2(2 m-1)|}{3\cdot\sqrt{1+(2m-1)^2}}\\	
			& \Leftrightarrow& 2(4m-1)^2=9\left(4m^2-4m+2\right) \\
			& \Leftrightarrow& 4m^2-20m+16=0 \\
			& \Leftrightarrow& \hoac{&m=1\\
				&m=4.}
		\end{eqnarray*} 
		Do đó tổng các giá trị cần tìm là $4+1=5$.
	}
\end{ex}

\begin{ex}%[Dự án 2025_K12_TL TV]%[Phạm Thị Thanh Thủy]%[2H5V1-6]s
	Biết mặt phẳng $(\alpha):(2m-1)x-3my+2z+3=0$ và $(\beta): mx+(m-1)y+4z-5=0$ vuông góc với nhau. Tính tích tất cả các giá trị tìm được của tham số $m$.
	\shortans{$-8$}
	\loigiai{
		$$
		(\alpha)\perp(\beta)\Leftrightarrow(2 m-1) \cdot m+(-3m) \cdot(m-1)+2\cdot 4=0 \Leftrightarrow-m^2+2m+8=0\Leftrightarrow \hoac{&m=4\\
			&m=-2.}
		$$
		Do đó tích các giá trị cần tìm là $4\cdot (-2)=-8$.
	}
\end{ex}
\Closesolutionfile{ans}
\indapan{6}{ans/ans-goc3-TLN}