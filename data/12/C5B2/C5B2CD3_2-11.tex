\part{véc-tơ và hệ trục tọa độ trong không gian}
\section{véc-tơ trong không gian}
\Opensolutionfile{ans}[ans/C5B2CD3_2-11]
\TN
%Câu 1
\begin{ex}%[2H5V2-3][Chí Nguyễn]
	Trong không gian $Oxyz$, cho điểm $A(-4;-3;3)$ và mặt phẳng $(P)\colon x+y+z=0$. Đường thẳng đi qua $A$, cắt trục $Oz$ và song song với $(P)$ có phương trình là
	\choice
	{$\dfrac{x-4}{4}=\dfrac{y-3}{3}=\dfrac{z-3}{-7}$}
	{$\dfrac{x+4}{4}=\dfrac{y+3}{3}=\dfrac{z-3}{1}$}
	{$\dfrac{x+4}{-4}=\dfrac{y+3}{3}=\dfrac{z-3}{1}$}
	{\True $\dfrac{x+8}{4}=\dfrac{y+6}{3}=\dfrac{z-10}{-7}$}
	\loigiai{
		Gọi $\Delta $ là đường thẳng cần lập.\\
		Mặt phẳng $(P)$ có một VTPT $\overrightarrow{n}=(1;1;1)$.\\
		Theo đề, ta có $\Delta \cap Oz=B(0;0;c)\Rightarrow \overrightarrow{AB}=(4;3;c-3)$ là một véc-tơ của $\Delta $.\\
		Khi đó $$\overrightarrow{AB}\perp \overrightarrow{n}\Leftrightarrow \overrightarrow{AB}\cdot \overrightarrow{n}=0\Leftrightarrow 4\cdot 1+3\cdot 1+(c-3)\cdot 1=0\Leftrightarrow c-3=-7.$$
		Suy ra $\overrightarrow{AB}=(4;3;-7)$.\\
		Vậy $\Delta \colon \dfrac{x+4}{4}=\dfrac{y+3}{3}=\dfrac{z-3}{-7}$ hay $\Delta \colon \dfrac{x+8}{4}=\dfrac{y+6}{3}=\dfrac{z-10}{-7}$.}
\end{ex}	
%Câu 2
\begin{ex}%[2H5V2-3]
	Trong không gian với hệ tọa độ $Oxyz$, cho mặt phẳng $(P)\colon x+y-z+9=0$, đường thẳng $d\colon \dfrac{x-3}{1}=\dfrac{y-3}{3}=\dfrac{z}{2}$ và điểm $A(1;2;-1)$. Viết phương trình đường thẳng $\Delta$ đi qua điểm $A$ cắt $d$ và song song với mặt phẳng $(P)$.
	\choice
	{\True $\dfrac{x-1}{-1}=\dfrac{y-2}{2}=\dfrac{z+1}{1}$}
	{$\dfrac{x-1}{1}=\dfrac{y-2}{2}=\dfrac{z+1}{-1}$}
	{$\dfrac{x-1}{1}=\dfrac{y-2}{2}=\dfrac{z+1}{1}$}
	{$\dfrac{x-1}{-1}=\dfrac{y-2}{2}=\dfrac{z+1}{-1}$}
	\loigiai{
	$(P)$ có véc-tơ pháp tuyến là $\overrightarrow{n}=(1;1;-1)$.\\
	$d$ có véc-tơ chỉ phương là $\overrightarrow{u}=(1;3;2)$ và $B(3;3;0)\in d$.\\
	$\Delta$ có véc-tơ chỉ phương là $\overrightarrow{u}_{\Delta}=(a;b;c)$ và $A(1;2;-1)\in \Delta$ (trong đó $a^2+b^2+c^2>0$).\\
	$\Rightarrow \overrightarrow{AB}=(2;1;1); d\parallel(P)\Leftrightarrow \overrightarrow{u}_{\Delta}\cdot \overrightarrow{n}=0\Leftrightarrow a+b-c=0\Leftrightarrow c=a+b\Rightarrow \overrightarrow{u}_{\Delta}=(a;b;a+b)$.\\
	Do $d$ cắt $\Delta$ $\Leftrightarrow \left[\overrightarrow{AB},\overrightarrow{u}\right]\cdot \overrightarrow{u}_{\Delta}=0\Leftrightarrow 2a+b=0\Leftrightarrow b=-2a$.\\
	Chọn $a=-1\Rightarrow b=2\Rightarrow c=1\Rightarrow \overrightarrow{u}_{\Delta}=(-1;2;1)\Rightarrow \Delta\colon \dfrac{x-1}{-1}=\dfrac{y-2}{2}=\dfrac{z+1}{1}.$\\
	Vậy $\Delta\colon \dfrac{x-1}{-1}=\dfrac{y-2}{2}=\dfrac{z+1}{1}.$
	}
\end{ex}
%Câu 3
\begin{ex}%[2H5V2-3]
	Trong không gian với hệ toạ độ $Oxyz$, cho điểm $M(1;-3;4)$, đường thẳng $d\colon \dfrac{x+2}{3}=\dfrac{y-5}{-5}=\dfrac{z-2}{-1}$ và mặt phẳng $(P)\colon 2x+z-2=0$. Viết phương trình đường thẳng $\Delta$ qua $M$ vuông góc với $d$ và song song với $(P)$.
	\choice
	{$\Delta\colon\dfrac{x-1}{1}=\dfrac{y+3}{-1}=\dfrac{z-4}{-2}$}
	{$\Delta\colon\dfrac{x-1}{-1}=\dfrac{y+3}{-1}=\dfrac{z-4}{-2}$}
	{\True $\Delta\colon\dfrac{x-1}{1}=\dfrac{y+3}{1}=\dfrac{z-4}{-2}$}
	{$\Delta\colon\dfrac{x-1}{1}=\dfrac{y+3}{-1}=\dfrac{z+4}{2}$}
	\loigiai{
		Ta có $\overrightarrow{u}_d=(3;-5;-1)$ là véc-tơ chỉ phương của $d$.\\
		$\overrightarrow{n}_{(P)}=(2;0;1)$  là véc-tơ pháp tuyến của $(P).$\\
		$\left[\overrightarrow{u}_d,\overrightarrow{n}_{(P)}\right]=(-5;-5;10)=-5(1;1;-2).$\\
		Do $\Delta$ vuông góc với $d$ và song song với $(P)$ nên $\overrightarrow{u}=(1;1;-2)$ là véc-tơ chỉ phương của $\Delta$.\\
		Khi đó, phương trình của $\Delta$ là $\dfrac{x-1}{1}=\dfrac{y+3}{1}=\dfrac{z-4}{-2}$. 
	}
\end{ex}
%Câu 4
\begin{ex}%[2H5V2-3]
	Trong không gian với hệ trục tọa độ $Oxyz$, cho hai mặt phẳng $(\alpha) \colon x-2y+z-1=0$, $(\beta) \colon 2x+y-z=0$ và điểm $A(1;2;-1)$. Đường thẳng $\Delta $ đi qua điểm $A$ và song song với cả hai mặt phẳng $(\alpha)$, $(\beta)$ có phương trình là
	\choice
	{$\dfrac{x-1}{-2}=\dfrac{y-2}{4}=\dfrac{z+1}{-2}$}
	{\True $\dfrac{x-1}{1}=\dfrac{y-2}{3}=\dfrac{z+1}{5}$}
	{$\dfrac{x-1}{1}=\dfrac{y-2}{-2}=\dfrac{z+1}{-1}$}
	{$\dfrac{x}{1}=\dfrac{y+2}{2}=\dfrac{z-3}{1}$}
	\loigiai{
		$(\alpha)$ có véc-tơ pháp tuyến là $\overrightarrow{n}_1=(1;-2;1)$, $(\beta)$ có véc-tơ pháp tuyến là $\overrightarrow{n}_2=(2;1;-1)$.\\
		Đường thẳng $\Delta$ có véc-tơ chỉ phương là $\overrightarrow{u}=\left[ \overrightarrow{n}_1,\overrightarrow{n}_2 \right]=(1;3;5)$.\\
		Phương trình của đường thẳng $\Delta \colon \dfrac{x-1}{1}=\dfrac{y-2}{3}=\dfrac{z+1}{5}$.}
\end{ex}
%Câu 5
\begin{ex}%[2H5V2-3]
	Trong không gian $Oxyz$, cho điểm $A(2;0;-1)$ và mặt phẳng $(P)\colon x+y-1=0$. Đường thẳng đi qua $A$ đồng thời song song với $(P)$ và mặt phẳng $Oxy$ có phương trình là
	\choice
	{$\heva{&x=3+t\\&y=2t\\&z=1-t}$}
	{\True $\heva{&x=2+t\\&y=-t\\&z=-1}$}
	{$\heva{&x=1+2t\\&y=-1\\&z=-t}$}
	{$\heva{&x=3+t\\&y=1+2t\\&z=-t}$}
	\loigiai{
		Ta có $\overrightarrow{n}_{(P)}=(1;1;0), \overrightarrow{n}_{(Oxy)}=(0;0;1).$\\
		Gọi $d$ là đường thẳng đi qua $A$ đồng thời song song với $(P)$ và mặt phẳng $(Oxy)$. Khi đó
		$$\heva{&\overrightarrow{n}_d\perp \overrightarrow{n}_{(P)}\\&\overrightarrow{n}_d\perp \overrightarrow{n}_{(Oxy)}}\Rightarrow \overrightarrow{n}_d=\left[\overrightarrow{n}_{(P)},\overrightarrow{n}_{(Oxy)}\right]=(1;-1;0).$$
		Vậy $d\colon \heva{&x=2+t\\&y=-t\\&z=-1.}$
	}
\end{ex}
%Câu 6
\begin{ex}%[2H5V2-3]
	Trong không gian tọa độ $Oxyz$, viết phương trình chính tắc của đường thẳng đi qua điểm $A(3;-1;5)$ và cùng song song với hai mặt phẳng $(P)\colon x-y+z-4=0$, $(Q)\colon 2x+y+z+4=0$.
	\choice
	{$\dfrac{x-3}{2}=\dfrac{y+1}{1}=\dfrac{z-5}{-3}$}
	{\True $\dfrac{x-3}{2}=\dfrac{y+1}{-1}=\dfrac{z-5}{-3}$}
	{$\dfrac{x+3}{2}=\dfrac{y-1}{1}=\dfrac{z+5}{-3}$}
	{$\dfrac{x+3}{2}=\dfrac{y-1}{-1}=\dfrac{z+5}{-3}$}
	\loigiai{
		Mặt phẳng $(P)$ có một véc-tơ pháp tuyến là $\overrightarrow{n}_P=(1;-1;1)$; mặt phẳng $(Q)$ có một véc-tơ pháp tuyến là $\overrightarrow{n}_Q=(2;1;1)$.\\
		Nhận thấy $A\notin (P), A\notin (Q)$.\\
		Gọi đường thẳng cần lập là $d$ và $\overrightarrow{u}$ là một véc-tơ chỉ phương của nó.\\
		Ta chọn $\overrightarrow{u}=\left[\overrightarrow{n}_P,\overrightarrow{n}_Q\right] =(2;-1;-3).$\\
		Mặt khác, $d$ qua $A(3;-1;5)$ nên có phương trình chính tắc là $\dfrac{x-3}{2}=\dfrac{y+1}{-1}=\dfrac{z-5}{-3}$.
	}
\end{ex}
%Câu 7
\begin{ex}%[2H5V2-3]
	Trong không gian với hệ trục tọa độ $Oxyz$, cho hai mặt phẳng $\left( \alpha \right)\colon x-2y+z-1=0$, $\left( \beta \right)\colon 2x+y-z=0$ và điểm $A(1;2;-1)$. Đường thẳng $\Delta $ đi qua điểm $A$ và song song với cả hai mặt phẳng $\left( \alpha \right),\left( \beta \right)$ có phương trình là
	\choice
	{$\dfrac{x-1}{-2}=\dfrac{y-2}{4}=\dfrac{z+1}{-2}$}
	{\True $\dfrac{x-1}{1}=\dfrac{y-2}{3}=\dfrac{z+1}{5}$}
	{$\dfrac{x-1}{1}=\dfrac{y-2}{-2}=\dfrac{z+1}{-1}$}
	{$\dfrac{x}{1}=\dfrac{y+2}{2}=\dfrac{z-3}{1}$}
	\loigiai{
	$\left( \alpha \right)$ có véc-tơ pháp tuyến là $\overrightarrow{n}_1=(1;-2;1)$, $\left( \beta \right)$ có véc-tơ pháp tuyến là $\overrightarrow{n}_2=(2;1;-1)$.\\
	Đường thẳng $\Delta $ có véc-tơ chỉ phương là $\overrightarrow{u}=\left[ \overrightarrow{n}_1,\overrightarrow{n}_2 \right]=(1;3;5)$.\\
	Phương trình của đường thẳng là $\Delta \colon \dfrac{x-1}{1}=\dfrac{y-2}{3}=\dfrac{z+1}{5}$.}	
\end{ex}
%Câu 8
\begin{ex}%[2H5C2-3]
	Trong không gian $Oxyz,$ cho ba đường thẳng $d_1\colon \dfrac{x-3}{2}=\dfrac{y+1}{1}=\dfrac{z-2}{-2}$; $d_2\colon \dfrac{x+1}{3}=\dfrac{y}{-2}=\dfrac{z+4}{-1}$; $d_3\colon \dfrac{x+3}{4}=\dfrac{y-2}{-1}=\dfrac{z}{6}$. Đường thẳng song song với $d_3$, cắt $d_1$ và $d_2$ có phương trình là
	\choice
	{$\dfrac{x-3}{4}=\dfrac{y+1}{1}=\dfrac{z-2}{6}$}
	{\True $\dfrac{x-3}{-4}=\dfrac{y+1}{1}=\dfrac{z-2}{-6}$}
	{$\dfrac{x+1}{4}=\dfrac{y}{-1}=\dfrac{z-4}{6}$}
	{$\dfrac{x-1}{4}=\dfrac{y}{-1}=\dfrac{z+4}{6}$}
	\loigiai{
		Từ $d_1\colon \dfrac{x-3}{2}=\dfrac{y+1}{1}=\dfrac{z-2}{-2}\Rightarrow d_1\colon \heva{&x=3+2t\\&y=-1+t\\&z=2-2t.}$\\
		Véc-tơ chỉ phương của $d_2$ là $\overrightarrow{u}_2=(3;-2;-1)$.\\
		Véc-tơ chỉ phương của $d_3$ là $\overrightarrow{u}_3=(4;-1;6)=-(-4;1;-6)$.\\
		Gọi $(P)$ là mặt phẳng chứa $d_2$ và song song với $d_3$, suy ra véc-tơ chỉ phương của $(P)$ là
		$\overrightarrow{n}_P=\left[ \overrightarrow{u}_2;\overrightarrow{u}_3\right]=(-13;-22;5)$ và $A(-1;0;-4)\in (P)$.\\
		$\Rightarrow (P)\colon -13(x+1)-22(y-0)+5(z+4)=0\Leftrightarrow (P)\colon 13x+22y-5z-7=0.$\\
		Gọi $B$ là giao điểm của $(P)$ và $d_1$. Đường thẳng đi qua $B$ và song song với $d_3$ chính là đường thẳng cần tìm.\\
		Gọi $B(3+2t;-1+t;2-2t)$. Thay tọa độ $B$ vào $(P)\colon 13(3+2t)+22(-1+t)-5(2-2t)-7=0\Rightarrow t=0\Rightarrow B(3;-1;2).$\\
		Vậy phương trình đường thẳng cần tìm là  $\dfrac{x-3}{-4}=\dfrac{y+1}{1}=\dfrac{z-2}{-6}$.	}
\end{ex}
%Câu 9
\begin{ex}%[2H5V2-3]
	Trong không gian $Oxyz,$ cho ba đường thẳng $d\colon \dfrac{x}{1}=\dfrac{y-1}{2}=\dfrac{z+2}{2}$, mặt phẳng $(P)\colon 2x+y+2z-5=0$ và điểm $A(1;1;-2)$. Phương trình chính tắc của đường thẳng $\Delta$ đi qua điểm $A$ song song với mặt phẳng $(P)$ và vuông góc với $d$ là
	\choice
	{$\Delta\colon\dfrac{x-1}{1}=\dfrac{y-1}{2}=\dfrac{z+2}{-2}$}
	{$\Delta\colon\dfrac{x-1}{2}=\dfrac{y-1}{1}=\dfrac{z+2}{-2}$}
	{\True $\Delta\colon\dfrac{x-1}{2}=\dfrac{y-1}{2}=\dfrac{z+2}{-3}$}
	{$\Delta\colon\dfrac{x-1}{1}=\dfrac{y-1}{2}=\dfrac{z+2}{2}$}
	\loigiai{
		$d$ có véc-tơ chỉ phương là $\overrightarrow{u}=(1;2;2).$\\
		$(P)$ có một véc-tơ pháp tuyến là $\overrightarrow{n}=(2;1;2).$\\
		Đường thẳng $\Delta$ song song với mặt phẳng $(P)$ và vuông góc với $d$.\\
		$\Rightarrow \Delta$ có một véc-tơ chỉ phương là $\overrightarrow{v}=\left[ \overrightarrow{u},\overrightarrow{n}\right]=(2;2;-3)$, và $\Delta$ đi qua điểm $A(1;1;-2)$.\\
		Vậy phương trình của $\Delta$ là $\dfrac{x-1}{2}=\dfrac{y-1}{2}=\dfrac{z+2}{-3}$. 	}
\end{ex}
%Câu 10
\begin{ex}%[2H5V2-2]
	Trong không gian $Oxyz,$ cho mặt phẳng $(P)\colon 2x-y+2z+3=0$ và hai đường thẳng $d_1\colon \dfrac{x}{3}=\dfrac{y-1}{-1}=\dfrac{z+1}{1}, d_2\colon \dfrac{x-2}{1}=\dfrac{y-1}{-2}=\dfrac{z+3}{1}$. Xét các điểm $A, B$ lần lượt di động trên $d_1$ và $d_2$ sao cho $AB$ song song với mặt phẳng $(P)$. Tập hợp trung điểm của đoạn thẳng $AB$ là
	\choice
	{\True Một đường thẳng có véc-tơ chỉ phương $\overrightarrow{u}=(-9;8;-5)$ }
	{Một đường thẳng có véc-tơ chỉ phương $\overrightarrow{u}=(-5;8;-5)$ }
	{Một đường thẳng có véc-tơ chỉ phương $\overrightarrow{u}=(1;-2;-5)$ }
	{Một đường thẳng có véc-tơ chỉ phương $\overrightarrow{u}=(1;5;-2)$ }
	\loigiai{
		$A\in d_1\Rightarrow A(3a;1-a;-1+a)$; $B\in d_2\Rightarrow B(2+b;1-2b;-1+b)$.\\
		$\overrightarrow{AB}=(2+b-3b;-2b+a;b-2-a)$; $n_P=(2;-1;2)$.\\
		Do $AB\parallel (P)$ nên $\overrightarrow{AB}\cdot \overrightarrow{n}_P=0\Leftrightarrow a=\dfrac{2}{3}b$.\\
		Tọa độ trung điểm của đoạn thẳng $AB$ là $I\left(1+\dfrac{2}{3}b;1-\dfrac{8}{6}b;-2+\dfrac{5}{6}b \right).$\\
		Suy ra tập hợp điểm $I$ là một đường thẳng $\heva{&x=1+\dfrac{2}{3}b\\&y=1-\dfrac{8}{6}b\\&z=-2+\dfrac{5}{6}b.}$\\
		Suy ra tập hợp trung điểm của đoạn thẳng $AB$ là một đường thẳng có véc-tơ chỉ phương $\overrightarrow{u}=(-9;8;-5)$.	}
\end{ex}
%Câu 11
\begin{ex}%[2H5V2-3]
	Trong không gian với hệ tọa độ $Oxyz$ cho hai đường thẳng $d\colon \heva{&x=2-t\\&y=1+2t\\&z=4-2t}$ và $d'\colon\dfrac{x-4}{1}=\dfrac{y+1}{-2}=\dfrac{z}{2}$. Phương trình nào dưới đây là phương trình đường thẳng thuộc mặt phẳng chứa $d$ và $d'$ đồng thời cách đều hai đường thẳng đó.
	\choice
	{$\dfrac{x-2}{3}=\dfrac{y-1}{1}=\dfrac{z-4}{-2}$}
	{$\dfrac{x+3}{1}=\dfrac{y+2}{-2}=\dfrac{z+2}{2}$}
	{\True $\dfrac{x-3}{1}=\dfrac{y}{-2}=\dfrac{z-2}{2}$}
	{$\dfrac{x+3}{-1}=\dfrac{y-2}{2}=\dfrac{z+2}{-2}$}
	\loigiai{
		$d$ đi qua $A\left( 2;1;4 \right)$và có véc-tơ chỉ phương $\overrightarrow{u_1}=(-1;2;-2)$.\\
		$d'$ đi qua $B(4;-1;0)$ có véc-tơ chỉ phương $\overrightarrow{u}_2=(1;-2;2)$.\\
		Ta có $\overrightarrow{u_1}=-\overrightarrow{u}_2$ và $\dfrac{2-4}{1}\ne \dfrac{1+1}{-2}\ne \dfrac{4}{2}$ nên $d\parallel d'$.\\
		Đường thẳng $\Delta$ thuộc mặt phẳng chứa $d$ và $d'$ đồng thời cách đều hai đường thẳng đó khi và chỉ khi
		$\heva{&\Delta\parallel d\parallel d'\\&\mathrm{d}\left( \Delta ,d \right)=\mathrm{d}\left( \Delta ,d' \right)}$
		hay $\Delta $ qua trung điểm $I(3;0;2)$ và có một véc-tơ chỉ phương là $\overrightarrow{u}=(1;-2;2)$. Khi đó phương trình của $\Delta$ là $ \dfrac{x-3}{1}=\dfrac{y}{-2}=\dfrac{z-2}{2}$.}
\end{ex}
%Câu 12
\begin{ex}%[2H5V2-3]
	Trong không gian với hệ trục tọa độ $Oxyz$, cho đường thẳng $d$ và mặt phẳng $(P)$ lần lượt có phương trình $\dfrac{x+1}{2}=\dfrac{y}{1}=\dfrac{z-2}{1}$ và $x+y-2z+8=0$, điểm $A(2;-1;3)$. Phương trình đường thẳng $\Delta $ cắt $d$ và $(P)$ lần lượt tại $M$và $N$ sao cho $A$ là trung điểm của đoạn thẳng $MN$ là
	\choice
	{$\dfrac{x+1}{3}=\dfrac{y+5}{4}=\dfrac{z-5}{2}$}
	{$\dfrac{x-2}{6}=\dfrac{y+1}{1}=\dfrac{z-3}{2}$}
	{$\dfrac{x-5}{6}=\dfrac{y-3}{1}=\dfrac{z-5}{2}$}
	{\True $\dfrac{x-5}{3}=\dfrac{y-3}{4}=\dfrac{z-5}{2}$}
	\loigiai{
		Đường thẳng $d$ có phương trình tham số $\heva{&x=-1+2t\\&y=t\\&z=2+t.}$\\
		Điểm $M$ thuộc đường thẳng $d$ nên $M(-1+2t;t;2+t)$.\\
		Điểm $A$ là trung điểm của $MN$ nên 
		$\heva{&{x_N}=2x_A-x_M=5-2t\\&{y_N}=2y_A-y_M=-2-t\\&{z_N}=2z_A-z_M=4-t} \Rightarrow N( 5-2t;-2-t;4-t)$.
		Mặt khác điểm $N\in (P)$ nên $5-2t-2-t-8+2t+8=0\Leftrightarrow t=3$.\\
		Suy ra $M( 5;3;5)$.\\
		Đường thẳng $\Delta $ có véc-tơ chỉ phương $\overrightarrow{AM}=(3;4;2)$ và đi qua điểm $M( 5;3;5)$ nên có phương trình là $\dfrac{x-5}{3}=\dfrac{y-3}{4}=\dfrac{z-5}{2}$.}
\end{ex}
%Câu 13
\begin{ex}%[2H5V2-3]
	Trong không gian với hệ trục tọa độ $Oxyz$, cho điểm $A$ và mặt phẳng $(P)\colon 3x-2y-3z-7=0$, đường thẳng $d\colon \dfrac{x-2}{3}=\dfrac{y+4}{-2}=\dfrac{z-1}{2}$. Phương trình nào sau đây là phương trình đường thẳng $\Delta$ đi qua $A$, song song $(P)$ và cắt đường thẳng $d$?
	\choice
	{\True  $\heva{&x=3+11t\\&y=2-54t\\&z=-4+47t}$}
	{$\heva{&x=3+54t\\&y=2+11t\\&z=-4-47t}$}
	{$\heva{&x=3+47t\\&y=2+54t\\&z=-4+11t}$}
	{$\heva{&x=3-11t\\&y=2-47t\\&z=-4+54t}$}
	\loigiai{
		Ta có $\overrightarrow{n}_{(P)}=(3;-2;-3)$ là véc-tơ pháp tuyến của mặt phẳng $(P)$.\\
		Đường thẳng $d$ đi qua điểm $M(2;-4;1)$ và có véc-tơ chỉ phương $\overrightarrow{u}_d=(3;-2;2)$.\\
		Giả sử $\Delta\cap d =M$ nên $M(2+3t;-4-2t;1+2t)$ khi đó véc-tơ chỉ phương của đường thẳng $\Delta$ là $\overrightarrow{u_\Delta}=\overrightarrow{AM}=(3t-1;-2t-6;2t+5)$.\\
		$\overrightarrow{AM}\perp\overrightarrow{n}_P\Leftrightarrow\overrightarrow{AM}\cdot\overrightarrow{n}_P=0$ nên $3(3t-1)-2(-2t-6)-3(2t+5)=0\Leftrightarrow t=\dfrac{6}{7}$.\\
		Suy ra $\overrightarrow{AM}=(\dfrac{11}{7};-\dfrac{54}{7};\dfrac{47}{7})=\dfrac{1}{7}(11;-54;47)$.\\
		Vậy phương trình đường thẳng $\Delta$ là $\heva{&x=3+11t\\&y=2-54t\\&z=-4+47t.}$
	}
\end{ex}
%Câu 14
\begin{ex}%[2H5V2-3]
		Trong không gian với hệ trục tọa độ $Oxyz$, cho mặt phẳng $(\alpha)\colon x-2z-6=0$, đường thẳng $d\colon \heva{&x=1+t\\&y=3+t\\&z=-1-t}$. Viết phương trình đường thẳng $\Delta$ nằm trong mặt phẳng $(\alpha)$ cắt đồng thời vuông góc với $d$.
		\choice
		{$\dfrac{x-2}{2}=\dfrac{y-4}{1}=\dfrac{z+2}{1}$}
		{\True  $\dfrac{x-2}{2}=\dfrac{y-4}{-1}=\dfrac{z+2}{1}$}
		{$\dfrac{x-2}{2}=\dfrac{y-3}{-1}=\dfrac{z+2}{1}$}
		{$\dfrac{x-2}{2}=\dfrac{y-4}{-1}=\dfrac{z-2}{1}$}
		\loigiai{
			Giao điểm $I$ của $d$ và $\alpha$ là nghiệm của hệ $\heva{&x=1+t\\&y=3+t\\&z=-1-t\\&x-2z-6=0}\Rightarrow I(2;4;-2)$.\\
			Mặt phẳng $(\alpha)$ có một véc-tơ pháp tuyến $\overrightarrow{n}=(1;0;-2)$ đường thẳng $d$ có một véc-tơ chỉ phương $\overrightarrow{u}=(1;1;-1)$.\\
			Khi đó đường thẳng $\Delta$ có một véc-tơ chỉ phương là $\left[\overrightarrow{n},\overrightarrow{u} \right]=(2;-1;1). $\\
			Đường thẳng $\Delta$ qua điểm $I$  và có một véc-tơ chỉ phương  $\left[\overrightarrow{n},\overrightarrow{u} \right]=(2;-1;1)$ nên có phương trình là $\dfrac{x-2}{2}=\dfrac{y-4}{-1}=\dfrac{z+2}{1}$.}
\end{ex}
%Câu 15
\begin{ex}%[2H5V2-3]
	Trong không gian với hệ trục tọa độ $Oxyz$, cho điểm $A(1;-2;3)$ và hai mặt phẳng  $(P)\colon x+y+z+1=0, (Q)\colon x-y+z-2=0$. Phương trình nào dưới đây là phương trình đường thẳng đi qua $A$, song song với $(P)$ và $(Q)$?
	\choice
	{\True  $\heva{&x=1+1t\\&y=-2\\&z=3-t}$}
	{$\heva{&x=-1+1t\\&y=2\\&z=-3-t}$}
	{$\heva{&x=1+2t\\&y=-2\\&z=3+2t}$}
	{$\heva{&x=1\\&y=-2\\&z=3-2t}$}
	\loigiai{
		Mặt phẳng $(P)$ có một véc-tơ pháp tuyến $\overrightarrow{n}_P=(1;1;1)$, mặt phẳng $(Q)$ có một véc-tơ pháp tuyến $\overrightarrow{n}_Q=(1;-1;1)$.\\
		Vì đường thẳng $d$ song song với hai mặt phẳng $(P)$ và $(Q)$ , nên  có véc-tơ chỉ phương là $\left[\overrightarrow{n}_P,\overrightarrow{n}_Q \right]=(2;0;-2)=2(1;0;-1)$.\\
		Vậy phương trình $d$ là $\heva{&x=1+1t\\&y=-2\\&z=3-t.}$
	}
\end{ex}
%Câu 16
\begin{ex}%[2H5C2-3]
	Trong không gian với hệ trục tọa độ $Oxyz$, cho các đường thẳng $d_1\colon \dfrac{x-3}{2}=\dfrac{y+1}{1}=\dfrac{z-2}{2}, 	d_2\colon \heva{&x=-1+3t\\&y=-2t\\&z=-4-t}, d_3\colon \dfrac{x+3}{4}=\dfrac{y-2}{-1}=\dfrac{z}{6}$. Đường thẳng song song với $d_3$ và cắt đồng thời $d_1$ và $d_2$ có phương trình là
	\choice
	{$\dfrac{x+1}{4}=\dfrac{y}{-1}=\dfrac{z-4}{6}$}
	{$\dfrac{x-1}{4}=\dfrac{y}{-1}=\dfrac{z+4}{6}$}
	{$\dfrac{x-3}{4}=\dfrac{y+1}{1}=\dfrac{z-2}{6}$}
	{\True $\dfrac{x-3}{-4}=\dfrac{y+1}{1}=\dfrac{z-2}{-6}$}
	\loigiai{
		Gọi $\Delta$ đường thẳng song song với $d_3$ và cắt $d_1$ và $d_2$.\\
		$\overrightarrow{u}_{\Delta}, \overrightarrow{u}_3$ lần lượt là véc-tơ chỉ phương của $\Delta$ và $d_3$.\\
		Ta có $\Delta \cap d_1=A \Rightarrow A(2x+3;x-1;-2x+2);\Delta \cap d_2=B \Rightarrow B(-1+3y;-2y;-4-y)$.\\
		$\overrightarrow{AB}=(3y-2x-4;-2y-x+1;-y+2x-6)$.\\
		Vì $\Delta\parallel d_3\Rightarrow \overrightarrow{u}_{\Delta}=k \overrightarrow{u}_3\Rightarrow \dfrac{3y-2x-4}{4}=\dfrac{-2y-x+1}{-1}=\dfrac{-y+2x-6}{6}$. \\
		Suy ra $$\heva{&2x-3y+4=-8y-4x+4\\&-12y-6x+6=y-2x+6}\Leftrightarrow \heva{&6x+5y=0\\&-13y+4x=0}\Leftrightarrow x=y=0.$$
		Từ đó suy ra $A(3;-1;2); B(-1;0;-4) \Rightarrow \overrightarrow{AB}=(-4;1;-6)$ là véc-tơ chỉ phương của $\Delta$.
		Vậy phương trình của 	$\Delta$ là $\dfrac{x-3}{-4}=\dfrac{y+1}{1}=\dfrac{z-2}{-6}$.
	}
\end{ex}
%Câu 17
\begin{ex}%[2H5V2-3]
	Trong không gian, cho mặt phẳng $(P)\colon x+y-z-4=0$ và điểm $A(2;-1;3)$. Gọi $\Delta$ là đường thẳng đi qua $A$ và song song với $(P)$, biết $\Delta$ có một véc-tơ chỉ phương là $\overrightarrow{u}=(a;b;c)$, đồng thời $\Delta$ đồng phẳng và không song song với $Oz$. Tính $\dfrac{a}{c}$.
	\choice
	{\True $\dfrac{a}{c}=2$}
	{$\dfrac{a}{c}=-2$}
	{$\dfrac{a}{c}=-\dfrac{1}{2}$}
	{$\dfrac{a}{c}=\dfrac{1}{2}$}
	\loigiai{
		$(P)$ có một véc-tơ pháp tuyến là $\overrightarrow{n}=(1;1;-1)$.\\
		$\Delta$ đi qua điểm $A(2;-1;3)$ và có một véc-tơ chỉ phương là $\overrightarrow{u}=(a;b;c)$.\\
		$Oz$ đi qua điểm $O(0;0;0)$ và có một véc-tơ chỉ phương là $\overrightarrow{k}=(0;0;1)$.\\
		$\Delta$ không song song với $Oz \Leftrightarrow a: b: c \neq 0: 0: 1$.\\
		$\Delta$ đồng phẳng với $Oz \Leftrightarrow$ Ba véc-tơ $\overrightarrow{u};\overrightarrow{k};\overrightarrow{OA}$ đồng phẳng. khi đó ta có
		$$
		\left[\overrightarrow{k}, \overrightarrow{OA}\right]  \overrightarrow{u}=0 \Leftrightarrow a+2b=0 \Leftrightarrow a=-2b .
		$$
		Do $\Delta\parallel(P) \Rightarrow \overrightarrow{u} \perp \overrightarrow{n} \Leftrightarrow \overrightarrow{u} \cdot \overrightarrow{n}=\overrightarrow{0} \Leftrightarrow a+b-c=0 \Rightarrow c=-b$.\\
		Suy ra $\dfrac{a}{c}=2$.
	}
\end{ex}
%Câu 18
\begin{ex}%[2H5V2-3]
	Trong không gian với hệ tọa độ $Oxyz$, viết phương trình tham số của đường thẳng đi qua điểm $M(1;3;-2)$, đồng thời song song với giao tuyến của hai mặt phẳng $(P)\colon x+y-3=0$ và $(Q)\colon 2x-y+z-3=0$.
	\choice
	{$\heva{&x=1+3t\\&y=3-t\\&z=-2+t}$}
	{$\heva{&x=1-3t\\&y=3+t\\&z=-2+t}$}
	{\True  $\heva{&x=1+t\\&y=3-t\\&z=-2-3t}$}
	{$\heva{&x=1+t\\&y=3+t\\&z=-2-3t}$}
	\loigiai{
		Hai mặt phẳng $(P)\colon x+y-3=0$ và $(Q)\colon 2x-y+z-3=0$ có véc-tơ pháp tuyến lần lượt là $\overrightarrow{n}_P=(1;1;0); \overrightarrow{n}_Q=(2;-1;1)$.
		Giao tuyến của hai mặt phẳng $(P)$ và $(Q)$ có véc-tơ chỉ phương là $\overrightarrow{u}=\left[\overrightarrow{n}_P, \overrightarrow{n}_Q\right]=(1;-1;-3)$.\\
		Đường thẳng đi qua điểm $M(1;3;-2)$, đồng thời song song với giao tuyến của hai mặt phẳng $(P)\colon x+y-3=0$ và $(Q)\colon 2x-y+z-3=0$ nhận véc-tơ $\overrightarrow{u}$ làm véc-tơ chỉ phương có phương trình tham số là $\heva{&x=1+t\\&y=3-t\\&z=-2-3t.}$
	}
\end{ex}
%Câu 19
\begin{ex}%[2H5V2-3]	
	Trong không gian với hệ tọa độ $Oxyz$, cho hai đường thẳng $d\colon \heva{&x=2+3t\\& y=-3+t\\&z=4-2t}$ và $d'\colon \dfrac{x-4}{3}=\dfrac{y+1}{1}=\dfrac{z}{-2}$. Phương trình nào dưới đây là phương trình đường thẳng thuộc mặt phẳng chứa $d$ và $d'$, đồng thời cách đều hai đường thẳng đó.
	\choice
	{\True  $\dfrac{x-3}{3}=\dfrac{y+2}{1}=\dfrac{z-2}{-2}$}
	{$\dfrac{x+3}{3}=\dfrac{y+2}{1}=\dfrac{z+2}{-2}$}
	{$\dfrac{x-3}{3}=\dfrac{y-2}{1}=\dfrac{z-2}{-2}$}
	{$\dfrac{x+3}{3}=\dfrac{y-2}{1}=\dfrac{z+2}{-2}$}
	\loigiai{
		Ta thấy hai đường thẳng $d$ và $d'$ có cùng véc-tơ chỉ phương hay $d\parallel d'$.\\
		Vậy đường thẳng cần tìm có véc-tơ chỉ phương là $\overrightarrow{u}=(3;1;-2)$ và đi qua trung điểm $I(3;-2;2)$ của $A B$ với $A(2;-3;4) \in d$ và $B(4;-1;0) \in d'$.\\
		Vậy phương trình đường thẳng cần tìm là $\dfrac{x-3}{3}=\dfrac{y+2}{1}=\dfrac{z-2}{-2}$.
	}
\end{ex}
\Closesolutionfile{ans}
\indapan{6}{ans/C5B2CD3_2-11}

%\TNTF
%\Opensolutionfile{ans}[ans/C5B2CD3_2-11]
%
%\Closesolutionfile{ans}
%\indapan{3}{ans/ans-0-B15-DS}
%
%\Opensolutionfile{ans}[ans/C5B2CD3_2-11]
%\TNSA
%
%\Closesolutionfile{ans}
%\indapan{6}{ans/C5B2CD3_2-11}