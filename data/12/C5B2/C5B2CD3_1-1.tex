\section{Lập phương trình đường thẳng liên quan đến song song và vuông góc}
\TN
\Opensolutionfile{ans}[ans/ans-0-B16-TN]
\setcounter{ex}{0}
\begin{ex}%[Dự án 2025_K12_TL TV]%[Lê Quốc Hiệp]%[2H5V2-3]
	Trong không gian $Oxyz$, cho điểm $A(-4;-3;3)$ và mặt phẳng $(P)\colon x+y+z=0$. Đường thẳng đi qua $A$, cắt trục $Oz$ và song song với $(P)$ có phương trình là
	\choice
	{$\dfrac{x-4}{4}=\dfrac{y-3}{3}=\dfrac{z-3}{-7}$}
	{$\dfrac{x+4}{4}=\dfrac{y+3}{3}=\dfrac{z-3}{1}$}
	{$\dfrac{x+4}{-4}=\dfrac{y+3}{3}=\dfrac{z-3}{1}$}
	{\True $\dfrac{x+8}{4}=\dfrac{y+6}{3}=\dfrac{z-10}{-7}$}
	\loigiai
	{
		Gọi $\Delta$ là đường thẳng cần lập.\\
		Mặt phẳng $(P)$ có một véc-tơ pháp tuyến $\vec{n}=(1;1;1)$.\\
		Theo đề, ta có $\Delta\cap Oz=B(0;0;c) \Rightarrow \overrightarrow{AB}=(4;3;c-3)$ là một véc-tơ chỉ phương của $\Delta$.\\
		Khi đó $\overrightarrow{AB}\perp\vec{n}\Leftrightarrow\overrightarrow{AB}\cdot\vec{n}=0\Leftrightarrow4\cdot1+3\cdot1+(c-3)\cdot1=0\Leftrightarrow c-3=-7$.\\
		Suy ra $\overrightarrow{AB}=(4;3;-7)$.\\
		Vậy $\Delta\colon \dfrac{x+4}{4}=\dfrac{y+3}{3}=\dfrac{z-3}{-7}$ hay $\Delta\colon \dfrac{x+8}{4}=\dfrac{y+6}{3}=\dfrac{z-10}{-7}$.
	}
\end{ex}
\Closesolutionfile{ans}
\indapan{6}{ans/ans-0-B16-TN}
