\chude{ĐƯỜNG THẲNG LIÊN QUAN ĐẾN GÓC VÀ KHOẢNG CÁCH}
\begin{dang}{LẬP PHƯƠNG TRÌNH MẶT PHẲNG LIÊN QUAN ĐẾN GÓC}
\end{dang}
\TN
\Opensolutionfile{ans}[ans/C5B4CD6-D1]
\begin{ex}%Câu 1%[2H5H2-7]
Trong không gian $Oxyz$, cho điểm $A(-2;0;1)$, đường thẳng $d$ qua điểm $A$ và tạo với trục $Oy$ góc $45^\circ$. Phương trình đường thẳng $d$ là
\choice
{\True $\hoac{&\dfrac{x+2}{2}=\dfrac{y}{\sqrt{5}}=\dfrac{z-1}{-1}\\
&\dfrac{x+2}{2}=\dfrac{y}{-\sqrt{5}}=\dfrac{z-1}{-1}}$}
{$\hoac{&\dfrac{x-2}{2}=\dfrac{y}{\sqrt{5}}=\dfrac{z+1}{-1}\\
&\dfrac{x-2}{2}=\dfrac{y}{-\sqrt{5}}=\dfrac{z+1}{-1}}$}
{$\hoac{&\dfrac{x+2}{2}=\dfrac{y}{\sqrt{5}}=\dfrac{z-1}{-1}\\
&\dfrac{x-2}{2}=\dfrac{y}{\sqrt{5}}=\dfrac{z+1}{-1}}$}
{$\hoac{&\dfrac{x+2}{2}=\dfrac{y}{-\sqrt{5}}=\dfrac{z-1}{-1}\\
&\dfrac{x-2}{2}=\dfrac{y}{\sqrt{5}}=\dfrac{z+1}{-1}}$}
\loigiai{
\begin{itemize}
\item Cách $1$: Điểm $M(0;m;0)\in Oy$, $\overrightarrow{j}=(0;1;0)$ là véc-tơ chỉ phương của trục $Oy$.\\ $\overrightarrow{AM}=(2;-m;-1)\Rightarrow\left|\cos\left(\overrightarrow{AM},\overrightarrow{j}\right)\right|=\cos{45^\circ}\Leftrightarrow\dfrac{\left| m\right|}{\sqrt{m^2+5}}=\dfrac{1}{\sqrt{2}}\Leftrightarrow m=\pm\sqrt{5}$ \break nên có 2 đường thẳng
$\dfrac{x+2}{2}=\dfrac{y}{\sqrt{5}}=\dfrac{z-1}{-1}$ và $\dfrac{x+2}{2}=\dfrac{y}{-\sqrt{5}}=\dfrac{z-1}{-1}$.
\item Cách $2$: $\overrightarrow{u_1}=\left(2;\sqrt{5};-1\right)\Rightarrow\left|\cos\left(\overrightarrow{u_1},\overrightarrow{j}\right)\right|=\dfrac{1}{\sqrt{2}}$;\\ $\overrightarrow{u_2}=\left(2;-\sqrt{5};-1\right)\Rightarrow\left|\cos\left(\overrightarrow{u_2},\overrightarrow{j}\right)\right|=\dfrac{1}{\sqrt{2}}$.\\
Đường thẳng $d$ đi qua điểm $A(-2;0;1)$ nên đường thẳng $d$ có phương trình là \\
$$\dfrac{x+2}{2}=\dfrac{y}{\sqrt{5}}=\dfrac{z-1}{-1}\text{ hoặc }
\dfrac{x+2}{2}=\dfrac{y}{-\sqrt{5}}=\dfrac{z-1}{-1}.$$
\end{itemize}
}
\end{ex}

\begin{ex}%Câu 2%[2H5V2-7]
Trong không gian với hệ tọa độ $Oxyz$, cho mặt phẳng $(P)\colon 4x-7y+z+25=0$ và đường thẳng $d_1\colon\dfrac{x+1}{1}=\dfrac{y}{2}=\dfrac{z-1}{-1}$. Gọi $d_1'$ là hình chiếu vuông góc của $d_1$ lên mặt phẳng $(P)$. Đường thẳng $d_2$ nằm trong $(P)$ tạo với $d_1$, $d_1'$ các góc bằng nhau, $d_2$ có véc-tơ chỉ phương $\overrightarrow{u}_2=(a;b;c)$. Tính $\dfrac{a+2b}{c}$.
\choice
{$\dfrac{a+2b}{c}=\dfrac{2}{3}$}
{$\dfrac{a+2b}{c}=0$}
{$\dfrac{a+2b}{c}=\dfrac{1}{3}$}
{\True $\dfrac{a+2b}{c}=1$}
\loigiai{
Véc-tơ chỉ phương của $d_1$ là $\overrightarrow{u}_1=(1;2;-1)$, véc-tơ pháp tuyến của $(P)$ là $\overrightarrow{n}_P=(4;-7;1)$.
\begin{itemize}
\item Cách $1$: Gọi $(Q)=(d_1,d_1')$ khi đó $(Q)$ có véc-tơ pháp tuyến $\overrightarrow{n}_Q=\left[\overrightarrow{n}_P,\overrightarrow{u}_1\right]=(5;5;15)$ .\\
Đường thẳng $d_1'$ có véc-tơ chỉ phương $\overrightarrow{u}'_1=\left[\overrightarrow{n}_P,\overrightarrow{u}_1\right]=(22;11;-11)$ hay một véc-tơ chỉ phương khác $\overrightarrow{u}=(2;1;-1)$ .\\
Vì $\overrightarrow{n}_P\cdot\overrightarrow{u}_2=0\Rightarrow 4a-7b+c=0\Rightarrow c=7b-4a\Rightarrow\overrightarrow{u}_2=(a;b;7b-4a)$.\\
Ta lại có
$$\begin{aligned}
\left(d_1;d_2\right)=\left(d_1';d_2\right)&\Leftrightarrow\left|\cos\left(\overrightarrow{u}_1,\overrightarrow{u_2}\right)\right|=\left|\cos\left(\overrightarrow{u}_1',\overrightarrow{u}_2\right)\right|\\&\Leftrightarrow\left| a+2b+4a-7b\right|=\left| 2a+b+4a-7b\right|\\&\Leftrightarrow\left| 5a-5b\right|=\left| 6a-6b\right|\\&\Leftrightarrow\left| a-b\right|=0\Leftrightarrow a=b.
\end{aligned}$$
Chọn $a=1\Rightarrow b=1$, $c=3\Rightarrow\dfrac{a+2b}{c}=1$.
\item Cách $2$: Gọi $(Q)=\left(d_1,d_1'\right)$, khi đó $(P)\perp(Q)$.\\
Các đường thẳng nằm trong $(P)$ mà vuông góc với $(Q)$ thì vuông góc với tất cả các đường thẳng trong $(Q)$ hay chúng cùng tạo với $d_1,d_1'$ các góc $90^{\circ}$.\\ Do đó, các đường thẳng này thỏa mãn yêu cầu đề bài và có véc-tơ chỉ phương $\overrightarrow{u}=\overrightarrow{n}_Q=(1;1;3)\Rightarrow\dfrac{a+2b}{c}=1$.
\end{itemize}
}
\end{ex}