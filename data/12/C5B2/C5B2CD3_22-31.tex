%%%% Câu 37
\begin{ex}%[2H5V2-3]%[Dự án 2025-K12-TL-TV]%[Thành Đức Trung]
Trong không gian $Oxyz$, cho mặt phẳng $(P)\colon x+2y+z-4=0$ và đường thẳng $d\colon \dfrac{x+1}{2}=\dfrac{y}{1}=\dfrac{z+2}{3}$. Phương trình đường thằng $\Delta$ nằm trong mặt phẳng $(P)$, đồng thời cắt và vuông góc với đường thẳng $d$ là
\choice
{$\dfrac{x-1}{5}=\dfrac{y+1}{-1}=\dfrac{z-2}{2}$}
{$\dfrac{x+1}{5}=\dfrac{y+3}{-1}=\dfrac{z-1}{3}$}
{$\dfrac{x-1}{5}=\dfrac{y-1}{1}=\dfrac{z-1}{-3}$}
{\True $\dfrac{x-1}{5}=\dfrac{y-1}{-1}=\dfrac{z-1}{-3}$}
\loigiai
{
Gọi $d\cap\Delta=M$, mà $\Delta\subset(P)$ nên $M\in (P)$.\\
Vì $M\in d\colon \heva{ & x=-1+2t \\ & y=t \\ & z=-2+3t}$ nên $M(-1+2t;t;-2+3t)$.\\
Mà $M\in (P)$ nên ta có $-1+2t+2t+-2+3t-4=0 \Leftrightarrow t=1$, do đó $M(1;1;1)$.\\
Vì $\heva{ & \Delta\perp d \\ & \Delta\subset (P)} \Rightarrow \heva{ & \overrightarrow{u}_{\Delta}\perp\overrightarrow{u}_{d} \\ & \overrightarrow{u}_{\Delta}\perp\overrightarrow{n}_{P}}$ nên chọn $\overrightarrow{u}_{\Delta}=\left[\overrightarrow{u}_{d},\overrightarrow{n}_{P}\right]=(5;-1;-3)$.\\
Vậy $\Delta\colon \dfrac{x-1}{5}=\dfrac{y-1}{-1}=\dfrac{z-1}{-3}$.
}
\end{ex}

%%%% Câu 38
\begin{ex}%[2H5V2-3]%[Dự án 2025-K12-TL-TV]%[Thành Đức Trung]
Trong không gian $Oxyz$, cho đường thẳng $d\colon \dfrac{x+3}{2}=\dfrac{y+1}{1}=\dfrac{z}{-1}$ và mặt phẳng $(P)\colon x+y-3z-2=0$. Gọi $d'$ là đường thẳng nằm trong mặt phẳng $(P)$, cắt và vuông góc với $d$. Đường thẳng $d'$ có phương trình là
\choice
{$\dfrac{x+1}{-2}=\dfrac{y}{-5}=\dfrac{z+1}{1}$}
{$\dfrac{x+1}{2}=\dfrac{y}{5}=\dfrac{z+1}{1}$}
{\True $\dfrac{x+1}{-2}=\dfrac{y}{5}=\dfrac{z+1}{1}$}
{$\dfrac{x+1}{-2}=\dfrac{y}{5}=\dfrac{z+1}{-1}$}
\loigiai
{
Gọi $d\cap d'=M$, mà $d'\subset(P)$ nên $M\in (P)$.\\
Vì $M\in d\colon \heva{ & x=-3+2t \\ & y=-1+t \\ & z=-t}$ nên $M(-3+2t;-1+t;-t)$.\\
Mà $M\in (P)$ nên ta có $-3+2t-1+t-3(-t)-2=0 \Leftrightarrow t=1$, do đó $M(-1;0;-1)$.\\
Vì $\heva{ & d'\perp d \\ & d'\subset (P)} \Rightarrow \heva{ & \overrightarrow{u}_{d'}\perp\overrightarrow{u}_{d}\\ & \overrightarrow{u}_{d'}\perp \overrightarrow{n}_{P}}$ nên chọn $\overrightarrow{u}_{d'}=\left[\overrightarrow{u}_{d},\overrightarrow{n}_{P}\right]=(-2;5;1)$. \\
Vậy $d'\colon \dfrac{x+1}{-2}=\dfrac{y}{5}=\dfrac{z+1}{1}$.
}
\end{ex}

%%%% Câu 39
\begin{ex}%[2H5V2-3]%[Dự án 2025-K12-TL-TV]%[Thành Đức Trung]
Trong không gian với hệ trục $Oxyz$, đường vuông góc chung của hai đường thẳng $d_1\colon \dfrac{x-2}{2}=\dfrac{y-3}{3}=\dfrac{z+4}{-5}$ và $d_2\colon \dfrac{x+1}{3}=\dfrac{y-4}{-2}=\dfrac{z-4}{-1}$ có phương trình
\choice
{$\dfrac{x-2}{2}=\dfrac{y+2}{3}=\dfrac{z-3}{4}$}
{$\dfrac{x}{2}=\dfrac{y-2}{3}=\dfrac{z-3}{-1}$}
{$\dfrac{x-2}{2}=\dfrac{y+2}{2}=\dfrac{z-3}{2}$}
{\True $\dfrac{x}{1}=\dfrac{y}{1}=\dfrac{z-1}{1}$}
\loigiai
{
Gọi $\Delta$ là đường vuông góc chung của $d_1$ và $d_2$, $\Delta\cap d_1=A$, $\Delta\cap d_2=B$.\\
Ta có $A(2+2a;3+3a;-4-5a)$, $B(-1+3b;4-2b;4-b)$ nên $\overrightarrow{AB}=(3b-2a-3;-2b-3a+1;-b+5a+8)$.\\
Vì $\heva{ & AB\perp d_1 \\ & AB\perp d_2}$ nên $\heva{ & \overrightarrow{AB}\cdot\overrightarrow{u}_{d_1}=0 \\ & \overrightarrow{AB}\cdot\overrightarrow{u}_{d_2}=0.}$ \\
Suy ra $\heva{ & 2(3b-2a-3)+3(-2b-3a+1)-5(-b+5a+8)=0 \\ & 3(3b-2a-3)-2(-2b-3a+1)-(-b+5a+8)=0} \Leftrightarrow \heva{ & a=-1\\ & b=1} \Rightarrow \heva{ & \overrightarrow{AB}=(2;2;2) \\ & A(0;0;1).}$\\ 
Vậy $\Delta\colon \dfrac{x}{1}=\dfrac{y}{1}=\dfrac{z-1}{1}$.
}
\end{ex}

%%%% Câu 40
\begin{ex}%[2H5V2-3]%[Dự án 2025-K12-TL-TV]%[Thành Đức Trung]
Cho hai đường thẳng $(d_1)\colon \heva{ & x=2+t \\ & y=1+t \\ & z=1+t}$ và $(d_2)\colon \dfrac{x}{1}=\dfrac{y-7}{-3}=\dfrac{z}{-1}$. Đường thẳng $(\Delta)$ là đường vuông góc chung của $(d_1)$ và $(d_2)$. Phương trình nào sau đây là phương trình của $(\Delta)$?
\choice
{\True $\dfrac{x-2}{1}=\dfrac{y-1}{1}=\dfrac{z+2}{-2}$}
{$\dfrac{x-2}{2}=\dfrac{y-1}{1}=\dfrac{z-1}{-2}$}
{$\dfrac{x-1}{1}=\dfrac{y-4}{1}=\dfrac{z+1}{-2}$}
{$\dfrac{x-3}{1}=\dfrac{y+2}{-1}=\dfrac{z+3}{-2}$}
\loigiai
{
Gọi $\Delta\cap d_1=A$, $\Delta\cap d_2=B$.\\
Ta có $A(2+a;1+a;1+a)$, $B(b;7-3b;-b)$ nên $\overrightarrow{AB}(b-a-2;-3b-a+6;-b-a-1)$.\\
Vì $\heva{ & AB\perp d_1 \\ & AB\perp d_2} \Leftrightarrow \heva{ & \overrightarrow{AB}\cdot\overrightarrow{u}_{d_1}=0 \\ & \overrightarrow{AB}\cdot\overrightarrow{u}_{d_2}=0.}$ \\
Suy ra $\heva{ & b-a-2-3b-a+6-b-a-1=0 \\ & b-a-2-3(-3b-a+6)-(-b-a-1)=0} \Leftrightarrow \heva{ & a=-1\\ & b=2}\Rightarrow \heva{ & \overrightarrow{AB}=(1;1;-2) \\ & B(2;1;-2).}$\\ 
Vậy $\Delta\colon \dfrac{x-2}{1}=\dfrac{y-1}{1}=\dfrac{z+2}{-2}$.
}
\end{ex}

%%%% Câu 41
\begin{ex}%[2H5V2-3]%[Dự án 2025-K12-TL-TV]%[Thành Đức Trung
Trong không gian với hệ tọa độ $Oxyz$, gọi $(\alpha)$ là mặt phẳng chứa đường thẳng $(d)\colon \dfrac{x-2}{1}=\dfrac{y-3}{1}=\dfrac{z}{2}$ và vuông góc với mặt phẳng $(\beta)\colon x+y-2z+1=0$. Hỏi giao tuyến của $(\alpha)$ và $(\beta)$ đi qua điểm nào?
\choice
{$(0;1;3)$}
{\True $(2;3;3)$}
{$(5;6;8)$}
{$(1;-2;0)$}
\loigiai
{
Ta có $A(2;3;0)\in d$ nên $A\in(\alpha)$.\\
Vì $\heva{ & (\alpha)\perp(\beta) \\ & d\subset(\alpha)} \Rightarrow \heva{ & \overrightarrow{n}_{\alpha}\perp\overrightarrow{n}_{\beta} \\ & \overrightarrow{n}_{\alpha}\perp \overrightarrow{u}_{d}}$ nên chọn $\overrightarrow{u}_{\Delta}=\left[\overrightarrow{u}_{d},\overrightarrow{n}_{\alpha}\right]=(-4;4;0)$.\\
Ta có $(\alpha)\colon -4(x-2)+4(y-3)=0 \Leftrightarrow x-y+1=0$.\\
Gọi $M\in (\alpha)\cap (\beta)$ thì tọa độ $M$ thỏa mãn $\heva{ & x+y-2z+1=0\\ & x-y+1=0}$ nên ta có $(2;3;3)$ thỏa mãn.
}
\end{ex}

%%%% Câu 42
\begin{ex}%[2H5V2-3]%[Dự án 2025-K12-TL-TV]%[Thành Đức Trung]
Trong không gian $Oxyz$ cho điểm $A(1;2;3)$ và đường thẳng $d\colon \dfrac{x-3}{2}=\dfrac{y-1}{1}=\dfrac{z+7}{-2}$. Đường thẳng đi qua $A$, vuông góc với $d$ và cắt trục $Ox$ có phương trình là
\choice
{$\heva{ & x=-1+2t \\ & y=-2t \\ & z=t}$}
{$\heva{ & x=1+t \\ & y=2+2t \\ & z=3+3t}$}
{\True $\heva{ & x=-1+2t \\ & y=2t \\ & z=3t}$}
{$\heva{ & x=1+t \\ & y=2+2t \\ & z=3+2t}$}
\loigiai
{
Gọi đường thẳng cần tìm là $\Delta$, $\Delta \cap Ox=M(a,0,0)$.\\
Ta có $\overrightarrow{AM}=(a-1;-2;-3)$.\\
Vì $\Delta\perp d$ nên $\overrightarrow{AM}\cdot\overrightarrow{u}_{d}=0\Leftrightarrow 2(a-1)-2+6=0\Leftrightarrow a=-1\Rightarrow \overrightarrow{AM}=(-2;-2;-3)$.\\
Vậy $\Delta \heva{ & x=-1+2t \\ & y=2t \\ & z=3t.}$
}
\end{ex}

%%%% Câu 43
\begin{ex}%[2H5V2-3]%[Dự án 2025-K12-TL-TV]%[Thành Đức Trung]
Trong không gian với hệ tọa độ $Oxyz$, cho điểm  $A(1;0;2)$ và đường thẳng $d\colon \dfrac{x-1}{1}=\dfrac{y}{1}=\dfrac{z+1}{2}$. Đường thẳng $\Delta$ đi qua $A$, vuông góc và cắt $d$ có phương trình là
\choice
{\True $\Delta\colon \dfrac{x-2}{1}=\dfrac{y-1}{1}=\dfrac{z-1}{-1}$}
{$\Delta\colon \dfrac{x-1}{1}=\dfrac{y}{1}=\dfrac{z-2}{1}$}
{$\Delta\colon \dfrac{x-2}{2}=\dfrac{y-1}{2}=\dfrac{z-1}{1}$}
{$\Delta\colon \dfrac{x-1}{1}=\dfrac{y}{-3}=\dfrac{z-2}{1}$}
\loigiai
{
Gọi $d\cap\Delta=M(1+t;t;-1+2t)$, $\overrightarrow{AM}=(t;t;2t-3)$.\\
Ta có $\Delta\perp d$ nên $\overrightarrow{AM}\cdot\overrightarrow{u}_{d}=0 \Leftrightarrow t+t+2(2t-3)=0 \Leftrightarrow t=1 \Leftrightarrow \heva{ & \overrightarrow{AM}=(1;1;-1) \\ & M(2;1;1).}$\\
Vậy $\Delta\colon \dfrac{x-2}{1}=\dfrac{y-1}{1}=\dfrac{z-1}{-1}$.
}
\end{ex}

%%%% Câu 44
\begin{ex}%[2H5V2-3]%[Dự án 2025-K12-TL-TV]%[Thành Đức Trung]
Trong không gian với hệ tọa độ $Oxyz$, cho điểm  $M(-1;1;3)$ và hai đường thẳng $\Delta\colon \dfrac{x-1}{3}=\dfrac{y+3}{2}=\dfrac{z-1}{1}$, $\Delta'\colon \dfrac{x+1}{1}=\dfrac{y}{3}=\dfrac{z}{-2}$. Phương trình nào dưới đây là phương trình đường thẳng đi qua $M$, vuông góc với $\Delta$ và $\Delta'$?
\choice
{$\heva{ & x=-1-t \\ & y=1+t \\ & z=1+3t}$}
{$\heva{ & x=-t \\ & y=1+t \\ & z=3+t}$}
{$\heva{ & x=-1-t \\ & y=1-t \\ & z=3+t}$}
{\True $\heva{ & x=-1-t \\ & y=1+t \\ & z=3+t}$}
\loigiai
{
Gọi $d$ là đường thẳng cần tìm.\\
Ta có $\overrightarrow{u}_{\Delta}=(3;2;1)$, $\overrightarrow{u}_{\Delta'}=(1;3;-2)$, $\left[\overrightarrow{u}_{\Delta},\overrightarrow{u}_{\Delta'}\right]=(-7;7;7)$ nên chọn $\overrightarrow{u}_{d}=(-1;1;1)$.\\
Vậy $d\colon \heva{ & x=-1-t \\ & y=1+t \\ & z=3+t.}$
}
\end{ex}

%%%% Câu 45
\begin{ex}%[2H5V2-3]%[Dự án 2025-K12-TL-TV]%[Thành Đức Trung]
Trong không gian với hệ tọa độ $Oxyz$, cho hai đường thẳng $d_1\colon \heva{ & x=1+3t \\ & y=-2+t \\ & z=2}$, $d_2\colon \dfrac{x-1}{2}=\dfrac{y+2}{-1}=\dfrac{z}{2}$ và mặt phẳng $(P)\colon 2x+2y-3z=0$. Phương trình nào dưới đây là phương trình mặt phẳng đi qua giao điểm của $d_1$ và $(P)$, đồng thời vuông góc với $d_2$?
\choice
{$2x-y+2z+13=0$}
{$2x+y+2z-22=0$}
{\True $2x-y+2z-13=0$}
{$2x-y+2z+22=0$}
\loigiai
{
Gọi $d_1\cap (P)=M(1+3t;-2+t;2)$.\\
Vì $M\in (P)$ nên $2(1+3t)+2(-2+t)-6=0\Leftrightarrow t=1 \Leftrightarrow M(4;-1;2)$.\\
Mà $d_2\perp (P)$ nên chọn $\overrightarrow{n}_{P}=(2;-1;2)$.\\
Vậy $(P)\colon 2x-y+2z-13=0$.
}
\end{ex}

%%%% Câu 46
\begin{ex}%[2H5V2-3]%[Dự án 2025-K12-TL-TV]%[Thành Đức Trung]
Trong không gian $Oxyz$, cho hai điểm $A(2;2;1)$, $B\left(-\dfrac{8}{3};\dfrac{4}{3};\dfrac{8}{3}\right)$. Đường thẳng qua tâm đường tròn nội tiếp tam giác $OAB$ và vuông góc với mặt phẳng $OAB$ có phương trình là
\choice
{$\dfrac{x+\dfrac{2}{9}}{1}=\dfrac{y-\dfrac{2}{9}}{-2}=\dfrac{z+\dfrac{5}{9}}{2}$}
{$\dfrac{x+1}{1}=\dfrac{y-8}{-2}=\dfrac{z-4}{2}$}
{$\dfrac{x+\dfrac{1}{3}}{1}=\dfrac{y-\dfrac{5}{3}}{-2}=\dfrac{z-\dfrac{11}{6}}{2}$}
{\True $\dfrac{x+1}{1}=\dfrac{y-3}{-2}=\dfrac{z+1}{2}$}
\loigiai
{
Gọi $d$ là đường thẳng cần tìm.\\
Ta có $\left[\overrightarrow{OA},\overrightarrow{OB}\right]=(4;-8;8)$ nên chọn $\overrightarrow{u}_{d}=(1;-2;2)$.\\
Ta có $OA=3$, $OB=4$, $AB=5$.\\
Gọi $I(x;y;z)$ là tâm đường tròn nội tiếp tam giác $OAB$.\\
Áp dụng hệ thức $OB\cdot\overrightarrow{IA}+OA\cdot\overrightarrow{IB}+AB\cdot\overrightarrow{IA}=\overrightarrow{0}$, ta có
$$4\left(\overrightarrow{OA}-\overrightarrow{OI}\right)+3\left(\overrightarrow{OB}-\overrightarrow{OI}\right)+5\overrightarrow{OI}\Leftrightarrow \overrightarrow{OI}=\dfrac{1}{12}\left(4\overrightarrow{OA}+3\overrightarrow{OB}\right)\Leftrightarrow I(0;1;1).$$
Suy ra $d\colon \heva{ & x=t \\ & y=1-2t \\ & z=1+2t}$, cho $t=-1$ ta có điểm $M(-1;3;-1)\in d$.\\
Vậy $d\colon \dfrac{x+1}{1}=\dfrac{y-3}{-2}=\dfrac{z+1}{2}$.
}
\end{ex}

%%%% Câu 47
\begin{ex}%[2H5V2-3]%[Dự án 2025-K12-TL-TV]%[Thành Đức Trung]
Trong không gian $Oxyz$, cho đường thẳng $d\colon \dfrac{x}{2}=\dfrac{y-3}{1}=\dfrac{z-2}{-3}$ và mặt phẳng $(P)\colon x-y+2z-6=0$. Đường thẳng nằm trong $(P)$ cắt và vuông góc với $d$ có phương trình là
\choice
{$\dfrac{x-2}{1}=\dfrac{y+2}{7}=\dfrac{z+5}{3}$}
{\True $\dfrac{x+2}{1}=\dfrac{y-2}{7}=\dfrac{z-5}{3}$}
{$\dfrac{x-2}{1}=\dfrac{y-4}{7}=\dfrac{z+1}{3}$}
{$\dfrac{x+2}{1}=\dfrac{y+4}{7}=\dfrac{z-1}{3}$}
\loigiai
{
Ta có $\overrightarrow{n}_{(P)}=(1;-1;2)$; $\overrightarrow{u}_d=(2;1;-3)$. \\
Gọi $I=d \cap (P)$. \\
Vì $I \in d \Rightarrow I(2t;3+t;2-3t)$.\\
Mặt khác $I \in (P) \Rightarrow 2t-(3+t)+2(2-3t)-6=0 \Leftrightarrow t=-1 \Rightarrow I(-2;2;5)$.\\
Gọi $\Delta$ là đường thẳng cần tìm.\\
Theo giả thiết $\heva{ & \overrightarrow{u}_{\Delta} \perp \overrightarrow{u}_{d} \\ & \overrightarrow{u}_{\Delta} \perp \overrightarrow{n}_{P}} \Rightarrow \overrightarrow{u}_{\Delta} = \left[ \overrightarrow{n}_{P} , \overrightarrow{u}_{d}\right]=(1;7;3)$.\\
Mà đường thẳng $\Delta$ đi qua điểm $I$.\\
Vậy $\Delta\colon \dfrac{x+2}{1}=\dfrac{y-2}{7}=\dfrac{z-5}{3}$.
}
\end{ex}

%%%% Câu 48
\begin{ex}%[2H5V2-3]%[Dự án 2025-K12-TL-TV]%[Thành Đức Trung]
Trong không gian $Oxyz$, cho hai đường thẳng $d_1\colon \dfrac{x-1}{2}=\dfrac{y+1}{-1}=\dfrac{z}{1}$ và $d_2\colon \heva{ & x=-1+t \\ & y=-1 \\ & z=-t}$ và mặt phẳng $(P)\colon x+y+z-1=0$. Đường thẳng vuông góc với $(P)$ cắt $d_1$ và $d_2$ có phương trình là
\choice
{$\dfrac{x+\dfrac{13}{5}}{1}=\dfrac{y-\dfrac{9}{5}}{1}=\dfrac{z-\dfrac{4}{5}}{1}$}
{\True $\dfrac{x-\dfrac{1}{5}}{1}=\dfrac{y+\dfrac{3}{5}}{1}=\dfrac{z+\dfrac{2}{5}}{1}$}
{$\dfrac{x-\dfrac{7}{5}}{1}=\dfrac{y+1}{1}=\dfrac{z-\dfrac{2}{5}}{1}$}
{$\dfrac{x}{1}=\dfrac{y}{1}=\dfrac{z}{1}$}
\loigiai
{
Giả sử đường thẳng $d$ vuông góc với $(P)$ cắt $d_1$ và $d_2$ tại $M$ và $N$.\\
Ta có $M(1+2a;-1-a;a)$; $N(-1+t;-1;-t)$; $\overrightarrow{NM}=(2a-t+2;-a;a+t)$.\\
Mặt phẳng $(P)$ có véc-tơ pháp tuyến là $\overrightarrow{n}=(1;1;1)$.\\
Vì $MN$ vuông góc với mặt phẳng $(P)$ nên $\overrightarrow{NM}$ cùng phương $\overrightarrow{n}$.\\
Khi đó $\dfrac{2a-t}{1}=\dfrac{-a}{1}=\dfrac{a+t}{1} \Leftrightarrow \heva{ & a=-\dfrac{2}{5} \\ & t=\dfrac{4}{5}} \Rightarrow M\left(\dfrac{1}{5};-\dfrac{3}{5};-\dfrac{2}{5}\right)$.\\
Đường thẳng $d$ qua điểm $M$ nhận $\overrightarrow{n}$ làm véc-tơ chỉ phương.\\
Phương trình $d\colon \dfrac{x-\dfrac{1}{5}}{1}=\dfrac{y+\dfrac{3}{5}}{1}=\dfrac{z+\dfrac{2}{5}}{1}$.
}
\end{ex}

%%%% Câu 49
\begin{ex}%[2H5V2-3]%[Dự án 2025-K12-TL-TV]%[Thành Đức Trung]
Trong không gian $Oxyz$, cho điểm $M(1;0;1)$ và đường thẳng $d\colon \dfrac{x-1}{1}=\dfrac{y-2}{2}=\dfrac{z-3}{3}$. Đường thẳng đi qua $M$, vuông góc với $d$ và cắt $Oz$ có phương trình là
\choice
{\True $\heva{ & x=1-3t \\ & y=0 \\ & z=1+t}$}
{$\heva{ & x=1-3t \\ & y=0 \\ & z=1-t}$}
{$\heva{ & x=1-3t \\ & y=t \\ & z=1+t}$}
{$\heva{ & x=1+3t \\ & y=0 \\ & z=1+t}$}
\loigiai
{
Gọi $\Delta$ là đường thẳng cần tìm và $N=\Delta \cap Oz$.\\
Ta có $N(0;0;c)$. \\
Vì $\Delta$ qua $M$, $N$ và $M\notin Oz$ nên $\overrightarrow{MN}=(-1;0;c-1)$ là véc-tơ chỉ phương của $\Delta$.\\
Ta có $d$ có một véc-tơ chỉ phương $\overrightarrow{u}=(1;2;3)$ và $\Delta \perp d$ nên
\[\overrightarrow{MN}\cdot\overrightarrow{u}=0 \Leftrightarrow -1+3(c-1)=0 \Leftrightarrow c=\dfrac{4}{3} \Rightarrow \overrightarrow{MN}\left(-1;0;\dfrac{1}{3}\right)\]
Chọn $\overrightarrow{v}=(-3;0;1)$ là một véc-tơ chỉ phương của $\Delta$, phương trình tham số của đường thẳng $\Delta$ là $\heva{ & x=1-3t \\ & y=0 \\ & z=1+t.}$
}
\end{ex}

%%%% Câu 50
\begin{ex}%[2H5V2-3]%[Dự án 2025-K12-TL-TV]%[Thành Đức Trung]
Trong không gian với hệ trục tọa độ $Oxyz$, cho mặt phẳng $(P)\colon 2x+y-2z+9=0$ và đường thẳng $d\colon \dfrac{x-1}{-1}=\dfrac{y+3}{2}=\dfrac{z-3}{1}$. Phương trình tham số của đường thẳng $\Delta$ đi qua $A(0;-1;4)$, vuông góc với $d$ và nằm trong $(P)$ là
\choice
{$\Delta \colon \heva{ & x=5t \\ & y=-1+t \\ & z=4+5t}$}
{$\Delta \colon \heva{ & x=2t \\ & y=t \\ & z=4-2t}$}
{\True $\Delta \colon \heva{ & x=t \\ & y=-1 \\ & z=4+t}$}
{$\Delta \colon \heva{ & x=-t \\ & y=-1+2t \\ & z=4+t}$}
\loigiai
{
Ta có $\heva{ & \Delta \perp d \\ & \Delta \subset (P)} \Rightarrow \heva{ & \overrightarrow{u}_{\Delta} \perp \overrightarrow{u}_d \\ & \overrightarrow{u}_{\Delta} \perp \overrightarrow{n}_{(P)}.}$\\
Ta có $\left[\overrightarrow{u}_{d},\overrightarrow{n}_{(P)}\right]=(5;0;5)$. \\
Do đó một véc-tơ chỉ phương của đường thẳng $\Delta$ là $\overrightarrow{u}_{\Delta}=(1;0;1) \Rightarrow \Delta \colon \heva{ & x=t \\ & y=-1 \\ & z=4+t.}$
}
\end{ex}

%%%% Câu 51
\begin{ex}%[2H5V2-3]%[Dự án 2025-K12-TL-TV]%[Thành Đức Trung]
Trong không gian với hệ tọa độ $Oxyz$, cho hai đường thẳng $\Delta_1 \colon \dfrac{x+1}{2}=\dfrac{y+2}{1}=\dfrac{z-1}{1}$ và $\Delta_2 \colon \dfrac{x+2}{-4}=\dfrac{y-1}{1}=\dfrac{z+2}{-1}$. Đường thẳng chứa đoạn vuông góc chung của $\Delta_1$ và $\Delta_2$ đi qua điểm nào sau đây?
\choice
{$M(0;-2;-5)$}
{$N(1;-1;-4)$}
{$P(2;0;1)$}
{\True $Q(3;1;-4)$}
\loigiai
{
Gọi $A(-1+2t;-2+t;1+t)$ và $B(-2-4t';1+t';-2-t')$ là hai điểm lần lượt thuộc $\Delta_1$ và $\Delta_2$.\\
Ta có $\overrightarrow{AB}=(-1-2t-4t';3-t+t';-3-t-t')$; $\Delta_1$ có véc-tơ chỉ phương $\overrightarrow{u}=(2;1;1)$; $\Delta_2$ có véc-tơ chỉ phương $\overrightarrow{u'}=(-4;1;-1)$.\\
Vì $AB$ là đoạn vuông góc chung của $\Delta_1$ và $\Delta_2$ nên 
\[\heva{ & \overrightarrow{AB}\cdot \overrightarrow{u}=0 \\ & \overrightarrow{AB} \cdot \overrightarrow{u'}=0} \Leftrightarrow \heva{ & 2(-1-2t-4t')+(3-t+t')+(-3-t-t')=0 \\ & -4(-1-2t-4t')+(3-t+t')-(-3-t-t')=0} \Leftrightarrow \heva{ & -6t-8t'=2 \\ & 8t+18t'=-10} \Leftrightarrow \heva{ & t=1 \\ & t'=-1.}\]
Suy ra $A(1;-1;2)$ và $\overrightarrow{AB}=(1;1;-3)$.\\
Phương trình đường thẳng chứa đoạn vuông góc chung của $\Delta_1$ và $\Delta_2$ là $
\heva{ & x=1+t_1 \\ & y=-1+t_1 \\ & z=2-3t_1.}$\\
Chỉ có $Q(3;1;-4)$ có tọa độ thỏa mãn phương trình.
}
\end{ex}

%%%% Câu 52
\begin{ex}%[2H5V2-3]%[Dự án 2025-K12-TL-TV]%[Thành Đức Trung]
Trong không gian $Oxyz$ cho hai đường thẳng $\dfrac{x-2}{1}=\dfrac{y-4}{1}=\dfrac{z}{-2}$ và $\dfrac{x-3}{2}=\dfrac{y+1}{-1}=\dfrac{z+2}{-1}$. Gọi $M$ là trung điểm đoạn vuông góc chung của hai đường thẳng trên. Tính đoạn $OM$.
\choice
{$OM=\dfrac{\sqrt{14}}{2}$}
{\True $OM=\sqrt{5}$}
{$OM=2\sqrt{35}$}
{$OM=\sqrt{35}$}
\loigiai
{
Đường thẳng $d \colon \heva{ & x=2+t \\ & y=4+t \\ & z=-2t}$ nhận véc-tơ $\overrightarrow{u}=(1;1;-2)$ làm véc-tơ chỉ phương.\\
Đường thẳng $d' \colon \heva{ & x=3+2m \\ & y=-1-m \\ & z=-2-m}$ nhận véc-tơ $\overrightarrow{v}=(2;-1;-1)$ làm véc-tơ chỉ phương.\\
Gọi $AB$ là đoạn vuông góc chung với $A \in d$ và $B \in d'$.\\
Khi đó $A(2+t;4+t;-2t)$ và $B(3+2m;-1-m;-2-m)$.\\
Suy ra $\overrightarrow{AB}=(2m-t+1;-m-t-5;-m+2t-2)$.\\
Ta có $\heva{ & \overrightarrow{AB} \perp \overrightarrow{u} \\ & \overrightarrow{AB} \perp \overrightarrow{v}} \Leftrightarrow \heva{ & \overrightarrow{AB}\cdot \overrightarrow{u}=0 \\ & \overrightarrow{AB}\cdot\overrightarrow{v}=0} \Leftrightarrow \heva{ & 3m-6t=0 \\ & 6m-3t=-9} \Leftrightarrow \heva{ & m=-2 \\ & t=-1}.$\\
Suy ra $A(1;3;2)$ và $B(-1;1;0)$.\\
Suy ra trung điểm của $AB$ là $M(0;2;1)$.\\
Vậy $OM=\sqrt{5}$.
}
\end{ex}

%%%% Câu 53
\begin{ex}%[2H5V2-3]%[Dự án 2025-K12-TL-TV]%[Thành Đức Trung]
Trong không gian với hệ tọa độ $Oxyz$, gọi $d$ là đường thẳng qua $A(1;0;2)$, cắt và vuông góc với đường thẳng $d_1 \colon \dfrac{x-1}{1}=\dfrac{y}{1}=\dfrac{z-5}{-2}$. Điểm nào dưới đây thuộc $d$?
\choice
{$P(2;-1;1)$}
{\True $Q(0;-1;1)$}
{$N(0;-1;2)$}
{$M(-1;-1;1)$}
\loigiai
{
Đường thẳng $d_1$ có véc-tơ chỉ phương là $\overrightarrow{u}=(1;1;-2)$.\\
Gọi $H$ là giao điểm của đường thẳng $d$ và đường thẳng $d_1$. \\
Vì $H \in d_1 \Rightarrow H(1+t;t;5-2t)$.\\
Ta có $\overrightarrow{AH}=(t;t;3-2t)$.\\
Vì $d \perp d_1$ nên $\overrightarrow{u} \cdot \overrightarrow{AH}=0 \Leftrightarrow t+t-2(3-2t)=0 \Leftrightarrow 6t=6 \Leftrightarrow t=1 \Rightarrow \overrightarrow{AH}=(1;1;1)$.\\
Suy ra $d \colon \heva{ & x=1+t \\ & y=t \\ & z=2+t.}$\\
Vậy $Q(0;-1;1) \in d$.
}
\end{ex}

%%%% Câu 54
\begin{ex}%[2H5V2-3]%[Dự án 2025-K12-TL-TV]%[Thành Đức Trung]
Trong không gian $Oxyz$, cho điểm $A(1;2;-1)$, đường thẳng $d\colon \dfrac{x-1}{2}=\dfrac{y+1}{1}=\dfrac{z-2}{-1}$ và mặt phẳng $(P)\colon x+y+2z+1=0$. Điểm $B$ thuộc mặt phẳng $(P)$ thỏa mãn đường thẳng $AB$ vuông góc và cắt đường thẳng $d$. Tọa độ điểm $B$ là
\choice
{$(6;-7;0)$}
{$(3;-2;-1)$}
{$(-3;8;-3)$}
{\True $(0;3;-2)$}
\loigiai
{
Ta gọi $AB$ cắt $d$ tại điểm $M(1+2m;-1+m;2-m) \in d$.\\
Khi đó $\overrightarrow{AM}=(2m;m-3;3-m)$, mà $AB \perp d$ nên
\[\overrightarrow{AM} \cdot \overrightarrow{u}_d=0 \Rightarrow 2 \cdot 2m+m-3=0 \Rightarrow m=1 \Rightarrow \overrightarrow{AM}=(2;-2;2)\]
Đường thẳng $AB$ đi qua $A$ nhận $\overrightarrow{u}=\dfrac{1}{2}\overrightarrow{AM}=(1;-1;1)$ là véc-tơ chỉ phương, ta có phương trình $AB$ là $\dfrac{x-1}{1}=\dfrac{y-2}{-1}=\dfrac{z+1}{1}$.\\
Gọi $B(1+t;2-t;-1+t) \in AB$.\\
Mà $B \in (P) \Rightarrow 1+t+2-t+2(-1+t)+1=0 \Rightarrow t=-1$.\\
Vậy $B(0;3;-2)$.
}
\end{ex}

%%%% Câu 55
\begin{ex}%[2H5V2-3]%[Dự án 2025-K12-TL-TV]%[Thành Đức Trung]
Trong không gian với hệ tọa độ $Oxyz$, cho $(P) \colon x-2y+z=0$ và đường thẳng $d\colon \dfrac{x-1}{2}=\dfrac{y}{1}=\dfrac{z+2}{-1}$. Đường thẳng $d$ cắt $(P)$ tại điểm $A$. Điểm $M(a;b;c)$ thuộc đường thẳng $d$ và có hoành độ dương sao cho $AM=\sqrt{6}$. Khi đó tổng $S=2016a+b-c$ là
\choice
{\True $2018$}
{$2019$}
{$2017$}
{$2020$}
\loigiai
{
Vì $d \cap (P) = A$ nên tọa độ điểm $A$ là nghiệm của hệ phương trình
\[\heva{ & x-2y+z=0 \\ & \dfrac{x-1}{2}=\dfrac{y}{1}=\dfrac{z+2}{-1}} \Leftrightarrow \heva{ & x-2y+z=0 \\ & x-2y=1 \\ & y+z=-2} \Leftrightarrow \heva{ & x=-1 \\ & y=-1 \\ & z=-1}\Rightarrow A(-1;-1;-1).\]
Vì $M \in d \Rightarrow M(1+2t;t;-2-t)$. \\
Suy ra $AM=\sqrt{ 6t^2+12t+6}$.\\
Mà $AM=\sqrt{6} \Rightarrow \sqrt{ 6t^2+12t+6}=\sqrt{6} \Leftrightarrow \hoac{ & t=0 \\ & t=-2.}$\\
Mà $M$ có hoành độ dương nên $1+2t>0 \Leftrightarrow t>-\dfrac{1}{2}$. \\
Suy ra $t=0 \Rightarrow M(1;0;-2)$.\\
Vậy $S=2018$.
}
\end{ex}

%%%% Câu 56
\begin{ex}%[2H5V2-3]%[Dự án 2025-K12-TL-TV]%[Thành Đức Trung]
Trong không gian $Oxyz$, cho hai đường thẳng $d_1\colon \dfrac{x-1}{1}=\dfrac{y+1}{-1}=\dfrac{z}{2}$; $d_2\colon \dfrac{x}{1}=\dfrac{y-1}{2}=\dfrac{z}{1}$. Đường thẳng $d$ đi qua $A(5;-3;5)$ lần lượt cắt $d_1$ và $d_2$ tại $B$ và $C$. Độ dài $BC$ là
\choice
{\True $\sqrt{19}$}
{$19$}
{$3\sqrt{2}$}
{$2\sqrt{5}$}
\loigiai
{
Ta có $d\cap d_1=B \Rightarrow B(1+t_1;-1-t_1;2t_1)$.\\
Lại có $d\cap d_2=C \Rightarrow C(t_2;1+2t_2;t_2)$.\\
Khi đó $\overrightarrow{AB}=(t_1-4;-t_1+2;2t_1-5)$ và $\overrightarrow{AC}=(t_2-5;2t_2+4;t_2-5)$.\\
Vì $A\notin d_2 \Rightarrow \overrightarrow{AC}\ne\overrightarrow{0}$.\\
Ba điểm $A$, $B$, $C$ cùng thuộc đường thẳng $d \Leftrightarrow \overrightarrow{AB}$ và $\overrightarrow{AC}$ cùng phương.\\
Khi đó $\exists k \in \mathbb{R} \colon \overrightarrow{AB}=k\overrightarrow{AC} \Leftrightarrow \heva{ & t_1-4=k(t_2-5) \\ & -t_1+2=k(2t_2+4) \\ & 2t_1-5=k(t_2-5)} \Leftrightarrow \heva{ & t_1=1 \\ & t_2=-1 \\ & k=\dfrac{1}{2}.}$\\
Do đó $B(2;-2;2)$; $C(-1;-1;-1)$; $\Rightarrow \overrightarrow{BC}=(-3;1;-3)$.\\
Vậy $BC=\sqrt{19}$.
}
\end{ex}