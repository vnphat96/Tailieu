%Câu 1.
\begin{ex}%[2H5N2-2]
Trong KG $Oxyz$, đường thẳng nào sau đây nhận véc-tơ pháp tuyến của mặt phẳng $(P) \colon 2x-y+2z+5=0$ làm một véc-tơ chỉ phương?
\choice
{$( Q) \colon x-y+2=0$}
{$\dfrac{x}{2}=\dfrac{y-1}{1}=\dfrac{z-2}{-1}$}
{\True $\dfrac{x-1}{-2}=\dfrac{y+1}{-1}=\dfrac{z}{-1}$}
{$\dfrac{x+2}{2}=\dfrac{y+1}{-1}=\dfrac{z+1}{1}$}
\loigiai{Xét đường thẳng $\dfrac{x-1}{-2}=\dfrac{y+1}{-1}=\dfrac{z}{-1}$, có một véc-tơ chỉ phương là $\left( -2;-1;-1 \right)=-\left( 2;1;1 \right)$(thỏa đề bài).}
\end{ex}
%Câu 2.
\begin{ex}%[2H5N2-2]
Trong KG $Oxyz$, đường thẳng nào sau đây nhận ${\overrightarrow{u}=(-2 ; 4 ; 5)}$ là một véc-tơ chỉ phương?
\choice
{$\heva{&x=-2+3 t \\ &y=4-t \\& z=5+4t}$}
{$\heva{&x=3+2t \\ & y=-1+4t \\ & z=4+5t}$}
{$\heva{&x=3+2t  \\&y=1+4t  \\&z=4+5t }$}
{\True $\heva{&x=3+2t  \\&y=-1-4t  \\&z=4-5t}$}
\loigiai{Xét đường thẳng $\heva{&x=3+2t  \\&y=-1-4t  \\&z=4-5t }$, có một véc-tơ chỉ phương là
$\overrightarrow{u}=(2;-4;-5)=-(-2;4;5)$ (thỏa đề bài).}
\end{ex}
\begin{ex}%[2H5N2-3]
	Trong KG $Oxyz$, đường thẳng nào sau đây nhận $\overrightarrow{u}=(-2;4;5)$ là một véc-tơ chỉ phương?
	\choice
	{$\heva{&x=-2+3 t \\ &y=4-t \\& z=5+4t}$}
	{$\heva{&x=3+2t \\ & y=-1+4t \\ & z=4+5t}$}
	{$\heva{&x=3+2t  \\&y=1+4t  \\&z=4+5t }$}
	{\True $\heva{&x=3+2t  \\&y=-1-4t  \\&z=4-5t }$}
	\loigiai{Ta có đường thẳng $\heva{&x=3+2t  \\&y=-1-4t  \\&z=4-5t }$, có một véc-tơ chỉ phương là $\overrightarrow{u}=(2;-4;-5)=-(-2;4;5)$ (thỏa đề bài).}
\end{ex}
\begin{ex}%[2H5N2-2]
Trong KG $Oxyz$, cho hai điểm $A(1;1;0)$ và $B( 0;1;2 )$. Véc-tơ nào dưới đây là một véc-tơ chỉ phương của đường thẳng $AB$.
		\choice
		{ $\overrightarrow{d}=(-1;1;2)$}
		{ $\overrightarrow{a}=(-1;0;-2)$}
		{\True $\overrightarrow{b}=(-1;0;2)$}
		{ $\overrightarrow{c}=( 1;2;2)$}
		\loigiai{
			Ta có $\overrightarrow{AB}=( -1;0;2 )$ suy ra đường thẳng ${AB}$ có véc-tơ chỉ phương là $\overrightarrow{b}=(-1;0;2)$.}
\end{ex}
\begin{ex}%[2H5H2-2]
Trong KG $Oxyz$, cho điểm $M( 1;2;3 )$. Gọi ${M}_{1}$, ${M}_{2}$ lần lượt là hình chiếu vuông góc của $M$ lên các trục $Ox$, $Oy$. Véc-tơ nào dưới đây là một véc-tơ chỉ phương của đường thẳng $M_1M_2$?
\choice
{ \True $\overrightarrow{{{u}_{4}}}=( -1;2;0 )$}
{ $\overrightarrow{{u}_{1}}=( 0;2;0)$}
{  $\overrightarrow {u_2}=( 1;2;0 )$}
{ $\overrightarrow{{{u}_{3}}}=( 1;0;0 )$}
	\loigiai{
Ta có ${{M}_{1}}$ là hình chiếu của $M$ lên trục $Ox \Rightarrow {{M}_{1}}( 1;0;0)$.\\
${{M}_{2}}$ là hình chiếu của $M$ lên trục $Oy\Rightarrow {{M}_{2}}( 0;2;0)$.\\
Khi đó $\overrightarrow{{{M}_{1}}{{M}_{2}}}=( -1;2;0 )$ là một véc-tơ chỉ phương của đường thẳng ${{M}_{1}}{{M}_{2}}$.}
\end{ex}
\begin{ex}%[2H5N2-3]
Trong KG $Oxyz$, cho đường thẳng $d \colon\dfrac{x-2}{1}=\dfrac{y-1}{-2}=\dfrac{z+1}{3}$. Điểm nào dưới đây thuộc $d$?
	\choice
{ $Q( 2;1;1 )$}
{ $M( 1;2;3 $}
{ \True $P( 2;1;-1)$}
{ $N( 1;-2;3) $}
	\loigiai{
Cho $\heva{& x-2=0 \\ & y-1=0 \\ & z+1=0} \Rightarrow \heva{& x=2 \\ & y=1 \\ & z=-1.}$ Vậy $P( 2;1;-1 ) \in d$.}
\end{ex}
\begin{ex}%[2H5N2-3]
Trong KG $Oxyz$, điểm nào dưới đây thuộc đường thẳng $d \colon \dfrac{x+1}{-1}=\dfrac{y-2}{3}=\dfrac{z-1}{3}$?
\choice
	{\True $P(-1\,2;1)$}
	{ $Q(1;-2;-1)$}
	{ $N(-1;3;2)$}
	{ $M( 1;2;1)$}
\loigiai{
	Thay tọa độ các điểm vào PTĐT ta thấy điểm $P(-1;2;1)$ thỏa $\dfrac{-1+1}{-1}=\dfrac{2-2}{3}=\dfrac{1-1}{3}=0$. Vậy điểm $P( -1;2;1)$ thuộc đường thẳng $d$.}
\end{ex}
\begin{ex}%[2H5N2-3]
Trong KG $Oxyz$, cho đường thẳng $d \colon \dfrac{x-4}{2}=\dfrac{z-2}{-5}=\dfrac{z+1}{1}$. Điểm nào sau đây thuộc $d$?
	\choice
	{ \True $N(4;2;-1)$}
	{ $Q(2;5;1)$}
	{ $M(4;2;1)$}
	{ $P(2;-5;1)$}
	\loigiai{
	Ta có điểm $N(4;2;-1)$ thỏa mãn phương trình $d$.}
\end{ex}
\begin{ex}%[2H5N2-3]
Trong KG $Oxyz$, điểm nào dưới đây thuộc đường thẳng $d \colon 
\heva{& x=1-t \\ 
	& y=5+t \\ 
	& z=2+3t.}$
\choice
	{ \True $N(1;5;2)$}
	{ $Q(-1;1;3)$}
	{ $M(1;1;3)$}
	{ $P(1;2;5)$}
		\loigiai{
Ta có $N(1;5;2)$ thuộc $d$.}
\end{ex}
\begin{ex}%[2H5N2-3]
Trong KG $Oxyz$. Đường thẳng $d \colon \heva{
		& x=t \\ 
		& y=1-t \\ 
		& z=2+t}$ đi qua điểm nào sau sau đây?
\choice
	{ $K\left( 1;-1;1 \right)$}
	{ $E\left( 1;1;2 \right)$}
	{ $H\left( 1;2;0 \right)$}
	{  \True $F\left( 0;1;2 \right)$}
	\loigiai{
		Thay tọa độ của $K\left( 1;-1;1 \right)$ vào PTTS của $d$ ta được $$\heva{
			& 1=t \\ 
			& -1=1-t \\ 
			& 1=2+t }\Leftrightarrow \heva{
			& t=1 \\ 
			& t=2 \\ 
			& t=-1.}$$ 
Vậy không tồn tại $t$ hay $K\notin d$.\\
Tương tự, thay $E\left( 1;1;2 \right)$ vào PTTS của $d$ ta được $$\heva{	& 1=t \\ & 1=1-t \\ & 2=2+t }\Leftrightarrow \heva{
				& t=1 \\ 
				& t=0 \\ 
				& t=0. }$$ 
Vậy không tồn tại $t$ hay $E\notin d$.\\
Thay tọa độ của $H\left( 1;2;0 \right)$ vào PTTS của $d$ ta được $$\heva{& 1=t \\ & 2=1-t \\ & 0=2+t }\Leftrightarrow \heva{
					& t=1 \\ 
					& t=-1 \\ 
					& t=-2. }$$
Vậy không tồn tại $t$ hay $H\notin d$.\\			
Thay tọa độ của $F\left( 0;1;2 \right)$ vào PTTS của $d$ ta được $$\heva{& 0=t \\ & 1=1-t \\ & 2=2+t}\Leftrightarrow \heva{
						& t=0 \\ 
						& t=0 \\ 
						& t=0 }\Leftrightarrow t=0.$$
Vậy $F \in d$.
}
\end{ex}
\begin{ex}%[2H5N2-3]
	Trong KG $Oxyz$, điểm nào dưới đây thuộc đường thẳng $d \colon \heva{& x=1-t \\ & y=5+t \\ 	& z=2+3t}$ ?
\choice
	{  $Q\left( -1; 1; 3 \right)$}
	{ $P\left( 1; 2; 5 \right)$}
	{\True $N\left( 1; 5;2 \right)$}
	{ $M\left( 1; 1; 3 \right)$}
	\loigiai{
 Với $t=0\Rightarrow \heva{
			& x=1 \\ 
			& y=5 \\ 
			& z=2 }\Rightarrow N\left( 1;5;2 \right)\in d$.
		
	}
\end{ex}
\Closesolutionfile{ans}
% \indapan{10}{ans/ans-2C5B2CD1-D1}
\TNTF
\Opensolutionfile{ans}[ans/ans-2C5B2CD1-D1-DS]
%Câu 3.
\begin{ex}%[2H5N2-2]
	Trong KG $Oxyz$, cho đường thẳng $d \colon \dfrac{x-2}{3}=\dfrac{y+5}{4}=\dfrac{z-1}{-1}$.  Các mệnh đề sau đây đúng hay sai?
\choiceTF
{ Đường thẳng $d$ nhận $\overrightarrow{u}=\left( 3;4;1 \right)$ là một véc-tơ chỉ phương}
{ \True Đường thẳng $d$ nhận $\overrightarrow{u}=\left( -3;-4;1 \right)$ là một véc-tơ chỉ phương}
{\True  Đường thẳng $d$ nhận $\overrightarrow{u}=\left( 3;4;-1 \right)$ là một véc-tơ chỉ phương}
{ \True Đường thẳng $d$ nhận $\overrightarrow{u}=\left( -6;-8;2 \right)$ là một véc-tơ chỉ phương}
\loigiai{
Đường thẳng $d \colon \dfrac{x-2}{3}=\dfrac{y+5}{4}=\dfrac{z-1}{-1}$ có một véc-tơ chỉ phương là ${{\overrightarrow{u}}_{d}}=\left( 3;4;-1 \right)$.
\begin{itemchoice}
	\itemch Sai. Vì  $\overrightarrow{u}\ne {{\overrightarrow{u}}_{d}}$.
	\itemch Đúng. Vì $\overrightarrow{u}=\left( -3;-4;1 \right)=-\left( 3;4;-1 \right)=-{{\overrightarrow{u}}_{d}}$.
	\itemch Đúng.Vì $\overrightarrow{u}=\overrightarrow{u}_{d}$.
	\itemch Đúng.Vì  $\overrightarrow{u}=\left( -6;-8;2 \right)=-2\left( 3;4;-1 \right)=-2{{\overrightarrow{u}}_{d}}$.
	
\end{itemchoice}
}
\end{ex}
\begin{ex}%[2H5N2-2]
	Trong KG $Oxyz$, cho đường thẳng $d\colon \heva{
		& x=3+4t \\ 
		& y=-1-2t \\ 
		& z=-2+3t},( t\in \mathbb{R})$. Các mệnh đề sau đây đúng hay sai?
	\choiceTF
	{ Điểm $M\left( 7;-3;-1 \right)$ thuộc đường thẳng $d$}
	{ \True Điểm $N\left( -1;1;-5 \right)$ thuộc đường thẳng $d$}
	{\True  Đường thẳng $d$ nhận $\overrightarrow{u}=\left( 4;-2;3 \right)$ là một véc-tơ chỉ phương}
	{\True  Đường thẳng $d$ nhận $\overrightarrow{u}=-\left( -4;2;-3 \right)$ là một véc-tơ chỉ phương}
	\loigiai{
\begin{itemchoice}
	\itemch Sai. Thay $M\left( 7;-3;-1 \right)$ vào đường thẳng $d$, ta có $$\heva{
		& 7=3+4t \\ 
		& -3=-1-2t \\ 
		& -1=-2+3t 
	}\Rightarrow \heva{
		& t=1 \\ 
		& t=1 \\ 
		& t=\frac{1}{3} \
	}\Rightarrow M\left( 7;-3;-1 \right)\notin d.$$
	
	\itemch Đúng. Thay $N\left( -1;1;-5 \right)$ vào đường thẳng $d$, ta có $$\heva{
		& -1=3+4t \\ 
		& 1=-1-2t \\ 
		& -5=-2+3t  
	}\Rightarrow \heva{
		& t=-1 \\ 
		& t=-1 \\ 
		& t=-1 
	}\Rightarrow M\left( 7;-3;-1 \right)\in d.$$
	
	\itemch Đúng. Vì một véc-tơ chỉ phương của đường thẳng $d$ là $\overrightarrow{u}=\left( 4;-2;3 \right)$.
	
	\itemch Đúng.Vì  $\overrightarrow{u}=\left( -4;2;-3 \right)=-\left( 4;-2;3 \right)$.
	
\end{itemchoice}
	}
\end{ex}
\begin{ex}%[2H5N2-3]
Trong KG $Oxyz$, cho đường thẳng $d \colon \dfrac{x-1}{2}=\dfrac{y-2}{-1}=\dfrac{z-3}{2}$. Các mệnh đề sau đây đúng hay sai?
	\choiceTF
	{ Điểm $Q\left( 2;-1;2 \right)$ thuộc đường thẳng $d$}
	{ \True Điểm $P\left( 1;2;3 \right)$ thuộc đường thẳng $d$}
	{ Điểm $M\left( -1;-2;-3 \right)$ thuộc đường thẳng $d$}
	{ Điểm $N\left( -2;1;-2 \right)$ thuộc đường thẳng $d$}
		\loigiai{
		\begin{itemchoice}
			\itemch Sai. Vì tọa độ $Q$ không thỏa phương trình $d$.
			\itemch Đúng. Vì tọa độ $P$ thỏa phương trình $d$.
			\itemch Sai. Vì tọa độ $M$ không thỏa phương trình $d$.
			\itemch Sai. Vì tọa độ $N$ không thỏa phương trình $d$.
		\end{itemchoice}
	}
\end{ex}
\begin{ex}%[2H5N2-3]
Trong KG $Oxyz$, cho đường thẳng $d\colon \heva{& x=1+2t \\ 
		& y=3-t \\ 
		& z=1-t}$. Các mệnh đề sau đây đúng hay sai?
	\choiceTF
	{ Điểm $M\left( -3;5;3 \right)$ không thuộc đường thẳng $d$}
	{ \True Điểm $N\left( 1;3;-1 \right)$ không thuộc đường thẳng $d$}
	{ \True Điểm $P\left( 3;5;3 \right)$ không thuộc đường thẳng $d$}
	{ \True Điểm $Q\left( 1;2;-3 \right)$ không thuộc đường thẳng $d$}
	\loigiai{
		\begin{itemchoice}
		\itemch Sai. Vì tọa độ $M$ thỏa phương trình $d$.
		\itemch Đúng. Vì tọa độ $N$ không thỏa phương trình $d$.
		\itemch Đúng. Vì tọa độ $P$ không thỏa phương trình $d$.
		\itemch Đúng. Vì tọa độ $Q$ không thỏa phương trình $d$.
	\end{itemchoice}
	}
\end{ex}
\begin{ex}%[2H5H2-3]
Trong KG $Oxyz$, cho ba điểm $A(1;2;0)$,$B(1;1;2)$ và $C(2;3;1)$. Các mệnh đề sau đây đúng hay sai?
	\choiceTF
	{ \True Đường thẳng đi qua $A$ và song song với $BC$ có phương trình là $\dfrac{x-1}{1}=\dfrac{y-2}{2}=\dfrac{z}{-1}$}
	{\True Đường thẳng đi qua hai điểm $B$, $C$ có phương trình là $\dfrac{x-1}{1}=\dfrac{y-1}{2}=\dfrac{z-2}{-1}$}
	{ Điểm $M\left( 2;3;1 \right)$ không thuộc đường thẳng $BC$}
	{\True  Điểm $N\left( 3;5;0 \right)$ không thuộc đường thẳng $BC$}
	\loigiai{
	\begin{itemchoice}
		\itemch Đúng. Gọi $d$ là PTĐT qua $A\left( 1;2;0 \right)$ và song song với $BC$.\\
		Ta có $\overrightarrow{BC}=\left( 1;2;-1 \right)\Rightarrow d \colon \dfrac{x-1}{1}=\dfrac{y-2}{2}=\dfrac{z}{-1}$.
		\itemch Đúng. Đường thẳng đi $B$ có vecto chỉ phương $\overrightarrow{BC}=\left( 1;2;-1 \right)$ có phương trình chính tắc là $\dfrac{x-1}{1}=\dfrac{y-1}{2}=\dfrac{z-2}{-1}.$
		\itemch Sai. Vì tọa độ $M$ thỏa phương trình $BC$.
		\itemch  Sai. Vì tọa độ $N$ thỏa phương trình $BC$.
	\end{itemchoice}	
	}
\end{ex}
\begin{ex}%[2H5H2-3]
Trong KG $Oxyz$, cho điểm $M(1;2;-1)$ và mặt phẳng $(P)\colon 2x+y-3z+1=0$. Các mệnh đề sau đây đúng hay sai?
\choiceTF
{ Đường thẳng đi qua $M$ và vuông góc với $(P)$ có phương trình là $\dfrac{x-1}{2}=\dfrac{y-2}{1}=\dfrac{z+1}{1}$}
{\True  Đường thẳng đi qua $M$ và vuông góc với $(P)$ có phương trình là $\dfrac{x-1}{2}=\dfrac{y-2}{1}=\dfrac{z+1}{-3}$}
{\True  Đường thẳng đi qua $M$ và vuông góc với ${(P)}$ có phương trình là $\dfrac{x-1}{-2}=\dfrac{y-2}{-1}=\dfrac{z+1}{3}$}
{Đường thẳng đi qua $M$ và vuông góc với $(P)$ có phương trình là $\dfrac{x+1}{2}=\dfrac{y+2}{1}=\dfrac{z-1}{-3}$}
\loigiai{
		\begin{itemchoice}
		\itemch Sai. Gọi $(\Delta)$ là đường thẳng cần tìm. Vì đường thẳng $(\Delta)$ vuông góc với mặt phẳng $(P)$ nên véc-tơ chỉ phương của $(\Delta)$ là $\overrightarrow{u_{\Delta}}=\overrightarrow{n_{P}}=(2;1;-3)$.
		\itemch Đúng. Phương trình chính tắc của đường thẳng $(\Delta)$ đi qua điểm $M(1 ;2;-1)$ và có véc-tơ chỉ phương $\overrightarrow{u_{\Delta}}=(2;1;-3)$
		là $\dfrac{x-1}{2}=\dfrac{y-2}{1}=\dfrac{z+1}{-3}$.
		\itemch Đúng. Vì $\overrightarrow{{u}_{\Delta }}=\overrightarrow{{n}_{P}}=(2;1;-3)=-(-2;-1;3)$.
		\itemch Sai. Vì đường thẳng đi qua $M$ và vuông góc với $(P)$ có phương trình là $\dfrac{x-1}{-2}=\dfrac{y-2}{-1}=\dfrac{z+1}{3}$.
	\end{itemchoice}
}
\end{ex}

%Câu 4
\Closesolutionfile{ans}
% \indapan{3}{ans/ans-2C5B2CD1-D1-DS}

\Opensolutionfile{ans}[ans/ans-2C5B2CD1-D1-KQ]
\TNSA
\begin{ex}%[2H5N2-2]
Trong KG $Oxyz$, cho hai điểm $M\left( 1;-2;1 \right)$, $N\left( 0;1;3 \right)$. Một véc-tơ chỉ phương của đường thẳng qua hai điểm $M$, $N$ có dạng $\overrightarrow{u}=(a;b;2)$. Tìm $a+b.$
\shortans{$2$}
	\loigiai{ 
		Ta có $\overrightarrow{MN}=\left(-1;3;2 \right)$. Véc-tơ chỉ phương của đường thẳng qua hai điểm $M$, $N$ là $\overrightarrow{MN}=\left(-1;3;2\right)$.\\
	Suy ra $a+b=2.$}
\end{ex}
\begin{ex}%[2H5H2-2]
	Trong KG $Oxyz$, cho ba điểm $B\left( 1;1;1 \right)$, $C\left( 3;4;0 \right)$. Tìm véc-tơchỉ phương của đường thẳng $\Delta $ song song với $BC$ có dạng $(a;b;-1)$. Tìm $a+b$.
	\shortans{$5$}
	\loigiai{ 
		Ta có $\overrightarrow{BC}=\left(2;3;-1 \right)$, đường thẳng $\Delta $ song song với $BC$  nên có véc-tơ chỉ phương cùng phương với $\overrightarrow{BC}$.\\
		Suy ra $a+b=5.$}
\end{ex}

%%23
\begin{ex}%[2H5H2-2]
	Trong KG $Oxyz$, cho mặt phẳng $(P) \colon x-3y+2z+1=0$.  Một véc-tơ chỉ phương của đường thẳng $\Delta $ vuông góc với mặt phẳng $\left( P \right)$ có dạng $(a;b;2)$. Tìm $a+b$.
	\shortans{$-2$}
	\loigiai{ 
	Đường thẳng $\Delta $ vuông góc với $( P )$ nên có véc-tơ chỉ phương $\overrightarrow{u}=\overrightarrow{n}_{P}=\left( 1;-3;2 \right)$.\\
	Suy ra $a+b=-2.$}
\end{ex}

%%24

\begin{ex}%[2H5V2-2]
	Trong KG $Oxyz$, cho hai mặt phẳng $( P) \colon 3x-2y-z+2024=0$ và $(Q) \colon x-2y+2025=0$. Một véc-tơ chỉ phương của đường thẳng $\Delta $ song song với hai mặt phẳng $\left( P \right)$ và $\left( Q \right)$ có dạng $(a;1;c)$. Tìm $a+c$.
	\shortans{$-6$}
	\loigiai{ 
	Đường thẳng $\Delta $ song song với hai mặt phẳng $(P)$ và $(Q)$ nên có véc-tơ chỉ phương
		$$\overrightarrow{u}=\left[ \overrightarrow{n}_{P},\overrightarrow{n}_{Q} \right]=\left( -2;1;-4 \right).$$
		Suy ra $a+c=-6.$}
\end{ex}
%%25
\begin{ex}%[2H5N2-2]
	Trong KG $Oxyz$, cho mặt phẳng $(P) \colon x+3y-2z-2024=0$ và $\overrightarrow{a}=\left( 1;1;0 \right)$. Một véc-tơ chỉ phương của đường thẳng $\Delta $ song song với mặt phẳng $(P)$ và song song véc-tơ $\overrightarrow{a}$ có dạng $(a;1;c)$. Tìm $a+c$.
	\shortans{$0$}
	\loigiai{ 
		Đường thẳng $\Delta $ song song với mặt phẳng $(P)$ và song song véc-tơ $\overrightarrow{a}$ nên có véc-tơ chỉ phương
		$$\overrightarrow{u}=\left[ \overrightarrow{n}_{P},\overrightarrow{a} \right]=( 2;2;-2)=2(1;1;-1).$$
		Suy ra $a+c=0.$}
\end{ex}
\Closesolutionfile{ans}
% \indapan{6}{ans/ans-2C5B2CD1-D1-KQ}
\begin{dang}{XÉT VỊ TRÍ TƯƠNG ĐỐI CỦA HAI ĐƯỜNG THẲNG}
	\textbf{Để xét vị trí tương đối đường thẳng ta có hai cách sau:}
	\begin{itemize}
		\item{\textbf{Cách 1}}
		Cho hai đường thẳng $\Delta _1$, $\Delta _2$ lần lượt đi qua các điểm $M_1$, $M_2$ và tương ứng có $\overrightarrow{u}_1=(a_1;b_1;c_1)$, $\overrightarrow{u}_2=(a_2;b_2;c_2)$ là hai véc-tơ chỉ phương. Khi đó, ta có:
		\begin{itemize}
			\item [$\bullet $] $\Delta_1 \equiv \Delta_2 \Leftrightarrow \heva{&\overline{u}_1, \overline{u}_2 & \text { cùng phương } \\&\overline{u}_1, \overline{M}_1 M_2 & \text { cùng phương }} \Leftrightarrow\heva{&\left[\overrightarrow{u}_1, \overrightarrow{u}_2\right]=\overrightarrow{0} \\& \left[\overrightarrow{u}_1, \overrightarrow{M_1 M_2}\right]=\overrightarrow{0}.}$
			\item [$\bullet $] $\Delta_1 \parallel \Delta_2 \Leftrightarrow \heva{&\overline{u}_1, \overline{u}_2 & \text { cùng phương } \\ &\overrightarrow{u}_1, \overline{M_1 M_2} & \text { không cùng phương }} \Leftrightarrow \heva{&{\left[\overrightarrow{u}_1, \overrightarrow{u}_2\right]=\overrightarrow{0}} \\ &\left[\overrightarrow{u}_1, \overline{M_1 M_2}\right] \neq \overrightarrow{0}.}$
			\item [$\bullet $] $\Delta_1$ cắt $\Delta_2 \Leftrightarrow \heva{&\overrightarrow{u}_1, \overrightarrow{u}_2 & \text { không cùng phương } \\ &\overrightarrow{u}_1, \overrightarrow{u}_2, \overline{M_1 M_2} & \text { đồng phẳng }} \Leftrightarrow\heva{& \left[\overrightarrow{u}_1, \overrightarrow{u}_2\right] \neq 0 \\ &\left[\overrightarrow{u}_1, \overrightarrow{u}_2\right] \overline{M_1 M_2} \neq 0.}$
			\item [$\bullet $] $\Delta_1$ và $\Delta_2$ chéo nhau $\Leftrightarrow\left[\overline{u}_1, \overline{u}_2\right] \overline{M_1 M_2} \neq 0$.
		\end{itemize}
		\item{\textbf{Cách 2}}
		Cho hai đường thẳng $\Delta_1, \Delta_2$ tương ứng có $\overrightarrow{u}_1=\left(a_1 ; b_1 ; c_1\right), \overrightarrow{u}_2=\left(a_2 ; b_2 ; c_2\right)$ là hai véc-tơ chỉ phương và có PTTS:
		$$
		\Delta_1 \colon \heva{&
			x=x_1+a_1 t_1 \\
			&y=y_1+b_1 t_1 \\
			&z=z_1+c_1 t_1}\left(t_1 \in \mathbb{R}\right),\quad \Delta_2 \colon \heva{&
			x=x_2+a_2 t_2 \\
			&y=y_2+b_2 t_2 \\
			&z=z_2+c_2 t_2} \quad\left(t_2 \in \mathbb{R}\right).$$
		Khi đó 
		\begin{itemize}
			\item  [$\bullet $] $\Delta_1 \equiv \Delta_2 \Leftrightarrow \overrightarrow{u}_1$ cùng phương với $\overrightarrow{u}_2$ và hệ (*) vô nghiệm.
			\item [$\bullet $] $\Delta_1 \parallel \Delta_2 \Leftrightarrow$ hệ (*) có vô số nghiệm.
			\item [$\bullet $] $\Delta_1$ cắt $\Delta_2 \Leftrightarrow$ hệ (*) có nghiệm duy nhất.
			\item [$\bullet $] $\Delta_1$ và $\Delta_2$ chéo nhau $\Leftrightarrow \overrightarrow{u}_1$ không cùng phương với $\overrightarrow{u}_2$ và hệ (*) vô nghiệm.			
		\end{itemize}
	\end{itemize}
\end{dang}
