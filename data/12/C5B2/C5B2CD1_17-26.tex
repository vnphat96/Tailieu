\Opensolutionfile{ans}[ans/ans-2-C5B2D2]
\TN
%Câu 1.
\begin{ex}%[2H5H2-4]%Câu 1
Trong KG $Oxyz$, hai đường thẳng $d \colon \heva{&x=-1+12t\\&y=2+6t\\&z=3+3t}$ và  $d' \colon \heva{&x=7+8t\\&y=6+4t\\&z=5+2t}$ có vị trí tương đối là
	\choice
	{\True trùng nhau}
	{song song}
	{chéo nhau}
	{cắt nhau}
	\loigiai
	{Đường thẳng $d$ có véc-tơ chỉ phương là  $\overrightarrow{u}=(12;6;3)$ và đi qua điểm $M(-1;2;3)$.\\
	Và đường thẳng $d'$ có véc-tơ chỉ phương là  $\overrightarrow{u'}=(8;4;2)$ và đi qua điểm $M'(7;6;5)$.\\
	Từ đó ta có $\overrightarrow{MM'}=(8;4;2)=\overrightarrow{u'}$ nên $d$ trùng với $d'$.}
	\end{ex}
%Câu 2.
\begin{ex}%[2H5H2-4]%Câu 2
	Trong KG $Oxyz$, cho hai đường thẳng $d \colon \dfrac{x-1}{-2}=\dfrac{y+2}{1}=\dfrac{z-4}{3}$ và  $ d' \colon \heva{&x=1+t\\&y=-t\\&z=-2+3t}$ có vị trí tương đối là
	\choice
	{trùng nhau}
	{song song}
	{chéo nhau}
	{\True  cắt nhau}
	\loigiai
	{Ta có 	$d$ có véc-tơ chỉ phương là  $\overrightarrow{u}=(-2;1;3)$ và đi qua điểm $M(1;-2;4)$.\\
		Và $d'$ có véc-tơ chỉ phương là  $\overrightarrow{u'}=(1;-1;3)$ và đi qua điểm $M'(1;0;-2)$.\\
		Từ đó ta có $\overrightarrow{MM'}=(-2;2;-6)$ và $ \left[\overrightarrow{u},\overrightarrow{u'} \right]=(6;9;1)\ne \overrightarrow{0}$. \\
		Ta cũng tính được $ \overrightarrow{MM'} \cdot \left[\overrightarrow{u},\overrightarrow{u'} \right]=0$. Do đó $d$ và $d'$ cắt nhau.}
\end{ex}
\begin{ex}%[2H5H2-4]Câu 1.
	Trong KG $Oxyz$, cho hai đường thẳng $d\colon \dfrac{x-2}{4}=\dfrac{y}{-6}=\dfrac{z+1}{-8}$ và $d'\colon \dfrac{x-7}{-6}=\dfrac{y-2}{9}=\dfrac{z}{12}$. Trong các mệnh đề sau, mệnh đề nào đúng khi nói về vị trí tương đối của hai đường thẳng trên?
	\choice
	{\True song song}
	{trùng nhau}
	c{héo nhau}
	{cắt nhau}
	\loigiai{
		$d$ có VTCP $\overrightarrow{u}=(4;-6;-8)$ và đi qua $M(2;0;-1)$.\\
		$d'$ có VTCP $\overrightarrow{u'}=(-6;9;12)$ và đi qua $M'(7;2;0)$.\\
		Từ đó ta có $\overrightarrow{MM'}=(5;2;1)$ và $\left[\overrightarrow{u},\overrightarrow{u'}\right]=\overrightarrow{0} $.\\
		Lại có $\left[\overrightarrow{u},\overrightarrow{MM'}\right]=\overrightarrow{0} $.\\
		Suy ra $d$ song song với $d'$.}
\end{ex}
\begin{ex}%[2H5H2-4]Câu 2.	
	Hai đường thẳng $d\colon \heva{& x=-1+12t \\ & y=2+6t\\&z=3+3t}$ và $d'\colon\heva{& x=7+8t \\ & y=6+4t\\&z=5+2t}$ có vị trí tương đối là
	\choice
	{\True trùng nhau}
	{song song}
	{chéo nhau}
	{cắt nhau}
	\loigiai{
		$d$ có VTCP $\overrightarrow{u}=(12;6;3)$ và đi qua $M(-1;2;3)$.\\
		$d'$ có VTCP $\overrightarrow{u'}=(8;4;2)$ và đi qua $M'(7;6;5)$.\\
		Từ đó ta có $\overrightarrow{MM'}=(8;4;2)$ 
		Suy ra $\left[\overrightarrow{u},\overrightarrow{MM'}\right]=\overrightarrow{0} $ và $\left[\overrightarrow{u},\overrightarrow{u'}\right]=\overrightarrow{0} $.\\
		Suy ra $d$ trùng với $d'$.}
\end{ex}
\begin{ex}%[2H5H2-4]Câu 3.	
	Trong không gian $ABCD.A'B'C'D'$, hai đường thẳng $A$ và $B(a;0;0)$ có vị trí tương đối là
	\choice
	{trùng nhau}
	{song song}
	{chéo nhau}
	{\True cắt nhau}
	\loigiai{
		$D(0;a;0)$ có VTCP $A'(0;0;b)$ và đi qua $(a > 0,b > 0)$
		$M$ có VTCP $CC'$ và đi qua $\dfrac{a}{b}$
		Từ đó ta có
		$(A'BD)$
		$\left(MBD\right)$ và $\dfrac{1}{2}$
		Suy ra $ - 1$ cắt $\dfrac{1}{3}$.}
\end{ex}
\Closesolutionfile{ans}
% \indapan{10}{ans/ans-2-C5B2D2}
\TNTF
\Opensolutionfile{ans}[ans/ans-2-C5B2D2-DS]
%Câu 3.
\begin{ex}%[2H5H2-4]
Trong KG $Oxyz$, cho hai đường thẳng $d \colon \dfrac{x-1}{2}=\dfrac{y-7}{1}=\dfrac{z-3}{4}$ và  $d' \colon \dfrac{x-6}{3}=\dfrac{y+1}{-2}=\dfrac{z+2}{1}$. Các mệnh đề sau đây đúng hay sai?
\choiceTF
{Đường thẳng $d$ song song đường thẳng $d'$}
{Đường thẳng $d$ trùng đường thẳng $d'$}
{\True  Đường thẳng $d$ cắt đường thẳng $d'$}
{Đường thẳng $d$ chéo đường thẳng $d'$}
\loigiai{
Ta có $d$ có véc-tơ chỉ phương là  $\overrightarrow{u}=(2;1;4)$ và đi qua điểm $M(1;7;3)$.\\
Và $d'$ có véc-tơ chỉ phương là  $\overrightarrow{u'}=(3;-2;1)$ và đi qua điểm $M'(6;-1;-2)$.\\
Từ đó ta có $\overrightarrow{MM'}=(5;-8;-5)$ và $ \left[\overrightarrow{u},\overrightarrow{u'} \right]=(9;10;-7)\ne \overrightarrow{0}$. \\
Ta cũng tính được $ \overrightarrow{MM'} \cdot \left[\overrightarrow{u},\overrightarrow{u'} \right]=0$. Do đó $d$ và $d'$ cắt nhau.}
\end{ex}
%Câu 4
\begin{ex}%[2H5H2-4]
	Trong KG $Oxyz$, cho hai đường thẳng $d \colon \dfrac{x-1}{-2}=\dfrac{y+2}{1}=\dfrac{z-4}{3}$ và  $ d' \colon \heva{&x=-1+t\\&y=-t\\&z=-2+3t}$. Các mệnh đề sau đây đúng hay sai?
	\choiceTF
	{\True  Tọa độ giao điểm của $d$ và $d'$ là $I(1;-2;4)$}
	{Tọa độ giao điểm của $d$ và $d'$ là $I(1;2;4)$}
	{\True  Đường thẳng $d$ cắt đường thẳng $d'$}
	{Đường thẳng $d$ chéo đường thẳng $d'$}
	\loigiai{
	Thay phương trình $d'$ và phương trình $d$, ta được $$\dfrac{-1+t-1}{-2}=\dfrac{-t+2}{1}=\dfrac{-2+3t-4}{3}\Leftrightarrow t=2.$$
Suy ra giao điểm của $d$ và $d'$ là $I(1;-2;4)$.}
\end{ex}
%Câu 5
\begin{ex}%[2H5V2-4]
	Trong KG $Oxyz$, cho bốn đường thẳng $d_1 \colon \dfrac{x-3}{1}=\dfrac{y+1}{-2}=\dfrac{z+1}{1}$, $d_2 \colon \dfrac{x}{1}=\dfrac{y}{-2}=\dfrac{z-1}{1}$, $d_3 \colon \dfrac{x-1}{2}=\dfrac{y+1}{1}=\dfrac{z-1}{1}$ và $d_4 \colon \dfrac{x}{1}=\dfrac{y-1}{-1}=\dfrac{z-1}{1}$. Các mệnh đề sau đây đúng hay sai?
	\choiceTF
	{\True  Hai đường thẳng $d_1$ và $d_2$ song song với nhau}
	{Đường thẳng $d_3$ cắt đường thẳng $d_2$}
	{Đường thẳng $d_4$ không cắt đường thẳng $d_1$}
	{Đường thẳng $d_3$ cắt đường thẳng $d_1$}
	\loigiai{
		Ta có 	$d_1$ có véc-tơ chỉ phương là  $\overrightarrow{u_1}=(1;-2;1)$ và đi qua điểm $M_1(3;-1;-1)$.\\
		Và $d_2$ có véc-tơ chỉ phương là  $\overrightarrow{u_2}=(1;-2;1)$ và đi qua điểm $M_2(0;0;1)$.\\
		Do $\overrightarrow{u_1}=\overrightarrow{u_2}$ và $M_1 \notin d_2$ nên hai đường thẳng $d_1$ và $d_2$ song song với nhau.\\
		Ta có $\overrightarrow{M_1M_2}=(-3;1;2)$ và $ \left[\overrightarrow{M_1M_2},\overrightarrow{u_1} \right]=(5;5;5)=5(1;1;1)$. \\
	Gọi $(\alpha)$  là mặt phẳng chứa $d_1$ và $d_2$, khi đó $(\alpha)$  có một véc-tơ pháp tuyến là $\overrightarrow{n}=(1;1;1)$.\\
	Phương trình mặt phẳng $(\alpha)$ là $x+y+z-1=0$.\\
	Gọi $A =  d_3 \cap(\alpha)  $ thì $A(1;-1;1) $, điểm $A$ không thuộc cả $d_1$ và $d_2$ nên $d_3$ không cắt cả hai đường thẳng $d_1$ và $d_2$.\\
	Gọi $B =  d_4 \cap(\alpha)  $ thì $B(-1;2;0)\notin d_1$ nên $d_4$ không cắt $d_1$.}
\end{ex}
\Closesolutionfile{ans}
% \indapan{3}{ans/ans-2-C5B2D2-DS}

\Opensolutionfile{ans}[ans/ans-2-C5B2D2-KQ]
\TNSA
\begin{ex}%[2H5H2-4]
Trong KG $Oxyz$, gọi $I(a;b;c)$ là tọa độ giao điểm của hai đường thẳng $\Delta_1 \colon \dfrac{x-1}{2}=\dfrac{y+1}{2}=\dfrac{z}{3}$ và  $\Delta_2 \colon \heva{& x=3-t\\& y=3-2t\\&z=-2+t}$. Tìm $a+b+c$.
	\shortans{$0$}
	\loigiai{
		Giao điểm của $\Delta_1$ và $\Delta_2$ thỏa mãn $$\heva{& x=3-t\\& y=3-2t\\&z=-2+t\\& \dfrac{x-1}{2}=\dfrac{y+1}{2}=\dfrac{z}{3}} \Leftrightarrow \heva{& x=3-t\\& y=3-2t\\&z=-2+t\\& \dfrac{3-t-1}{2}=\dfrac{3-2t+1}{2}=\dfrac{-2+t}{3}}  \Leftrightarrow \heva{&x=1\\&y=-1\\&z=0\\&t=2.} $$
	Suy ra $a+b+c=0.$
	}
\end{ex}
\begin{ex}%[2H5H2-4]
	Trong KG $Oxyz$, biết hai đường thẳng $d_1 \colon \dfrac{x}{1}=\dfrac{y}{-2}=\dfrac{z-1}{1}$ và  $d_2 \colon \dfrac{x-1}{2}=\dfrac{y+1}{1}=\dfrac{z-1}{1}$ cắt nhau tại $I(a;b;c)$. Tính giá trị $a+b+c$.
	\shortans{$1$}
	\loigiai{
		Giao điểm của $d_1$ và $d_2$ thỏa hệ 
		$$
		\heva{&\dfrac{x}{1}=\dfrac{y}{-2}=\dfrac{z-1}{1}\\
		& \dfrac{x-1}{2}=\dfrac{y+1}{1}=\dfrac{z-1}{1}} \Leftrightarrow \heva{& -2x-y=0\\&x-z=0\\&x-2y=0\\&x-2z=0} \Leftrightarrow \heva{&x=-\dfrac{1}{5}\\&y=\dfrac{2}{5}\\&z=\dfrac{4}{5}.}
		$$
	Vậy $a+b+c=1$.
	}
\end{ex}
\Closesolutionfile{ans}
% \indapan{6}{ans/ans-2-C5B2D2-KQ}
\begin{dang}{GÓC GIỮA HAI ĐƯỜNG THẲNG}
	Cho hai đường thẳng có hai vectơ chỉ phương lần lượt là $\overrightarrow {u_1}=(a_1;{b_1};{c_1})$, $\overrightarrow{ u_2}=(a_2;{b_2};{c_2})$. Khi đó, ta có
	$$\cos\left(\Delta_1,\Delta_2\right)=\left|\cos\left(\overrightarrow {u_1}, \overrightarrow{ u_2} \right)\right|=\dfrac{\left|\overrightarrow {u_1}\cdot \overrightarrow{u_2}\right|}{\left|\overrightarrow {u_1}\right|\cdot \left|\overrightarrow {u_2}\right|}=\dfrac{\left|a_1a_2+b_1b_2+c_1c_2 \right| }{\sqrt{a_1^2+b_1^2+c_1^2}\cdot \sqrt{a_2^2+b_2^2+c_2^2}}.$$
	\textbf{Chú ý :}\\
	• $\Delta_1\bot{\Delta_2}\Leftrightarrow{\overrightarrow {u_1}}\cdot \overrightarrow {u_2}=0\Leftrightarrow{a_1}{a_2}+b_1b_2+c_1c_2=0.$\\
	• Hai đường thẳng song song hoặc trùng với nhau thì góc giữa chúng là $0^\circ$.\\
	\textbf{TÍNH GÓC GIỮA ĐƯỜNG THẲNG VỚI MẶT PHẲNG}\\
	Cho đường thẳng $\Delta $ có véc-tơ chỉ phương $\overrightarrow {u}=(a;b;c)$ và mặt phẳng $(P)$ có véc-tơ pháp tuyến $\overrightarrow {n}=(A; B; C)$. Khi đó, ta có
	$$\sin\left(\Delta ,(P)\right)=\left|\cos\left(\overrightarrow {u}, \overrightarrow{n}\right)\right|=\dfrac{\left|\overrightarrow {u} \cdot\overrightarrow {n}\right|}{\left|\overrightarrow {u}\right| \cdot \left| \overrightarrow {n}\right|}=\dfrac{\left|aA+bB+cC\right|}{\sqrt{a^2+b^2+c^2}\cdot \sqrt{A^2+B^2+C^2}}.$$
	\textbf{Chú ý :}\\
	• Đường thẳng song song hoặc trùng với mặt phẳng thì góc giữa chúng là $0^0$.\\
	\textbf{TÍNH GÓC GIỮA HAI MẶT PHẲNG}\\
	Cho hai mặt phẳng $(P_1)$, $(P_2)$ có hai véc-tơ pháp tuyến lần lượt là \break  $\overrightarrow{n_1}=(A_1;B_1;C_1)$, $ \overrightarrow{n_2}=(A_2;B_2;C_2)$. Khi đó, ta có
	$$ \cos\left((P_1),(P_2)\right)=\left|\cos\left( \overrightarrow{n_1} , \overrightarrow{n_2} \right)\right|=\dfrac{\left|\overrightarrow{n_1}\cdot \overrightarrow{n_2}\right|}{\left|\overrightarrow{n_1}\right|\cdot \left|\overrightarrow {n_2}\right|}=\dfrac{\left|A_1A_2+B_1B_2+C_1C_2\right|}{\sqrt{A_1^2+B_1^2+C_1^2}\cdot \sqrt{A_2^2+B_2^2+C_2^2}}.$$
	\textbf{Chú ý :}\\
	• Hai mặt phẳng song song hoặc trùng với nhau thì góc giữa chúng là $0^\circ$.
\end{dang}

\Opensolutionfile{ans}[ans/ans-goc1]
\TN
%Câu 1.
\begin{ex}%[2H5N2-7]
	Gọi $\alpha $ là góc giữa hai đường thẳng $AB$, $CD$. Khẳng định nào sau đây đúng?
	\choice
	{\True $\cos\alpha=\dfrac{\left|\overrightarrow{AB}\cdot\overrightarrow{CD}\right|}{\left|\overrightarrow{AB}\right|\cdot \left|\overrightarrow{CD}\right|}$}
	{$\cos\alpha=\dfrac{\overrightarrow{AB}\cdot\overrightarrow{CD}}{\left|\overrightarrow{AB}\right|\cdot \left|\overrightarrow{CD}\right|}$}
	{$\cos\alpha=\dfrac{\left|\overrightarrow{AB}\cdot\overrightarrow{CD}\right|}{\left|\left[\overrightarrow{AB},\overrightarrow{CD}\right]\right|}$}
	{$\cos\alpha=\dfrac{\left|\left[\overrightarrow{AB}\cdot \overrightarrow{CD}\right]\right|}{\left|\overrightarrow{AB}\right|\cdot \left|\overrightarrow{CD}\right|} $}
	\loigiai{
		Ta có $\cos\alpha=\dfrac{\left|\overrightarrow{AB}\cdot\overrightarrow{CD}\right|}{\left|\overrightarrow{AB}\right|\cdot \left|\overrightarrow{CD}\right|}$.}
	
\end{ex}

\begin{ex}%[2H5H2-5]
	Cho hai đường thẳng $d_1\colon \heva{&
		x=2+t\\&y=-1+t\\&z=3}$ và $d_2\colon \heva{& x=1-t\\&y=2\\& z=-2+t}$. Góc giữa hai đường thẳng $d_1$ và $d_2$ là
	\choice
	{$30^\circ $}
{$120^\circ $}
{$150^\circ $}
{\True $60^\circ $} 
	\loigiai{
Gọi $\overrightarrow{u_1}$, $\overrightarrow{u_2}$ lần lượt là véc-tơ chỉ phương của đường thẳng $d_1$ và $d_2$.\\
Ta có $\overrightarrow{u_1}=(1;1;0)$; $\overrightarrow{u_2}=(-1;0;1)$.\\
Áp dụng công thức ta có $$ \cos\left(d_1,d_2\right)=\left|\cos\left(\overrightarrow{u_1},\overrightarrow{u_2}\right)\right|=\dfrac{\left|\overrightarrow{u_1}\cdot \overrightarrow{u_2}\right|}{\left|\overrightarrow{u_1}\right| \cdot \left|\overrightarrow{u_2}\right|}=\dfrac{\left|-1\right|}{\sqrt{1+1}\cdot\sqrt{1+1}}=\dfrac{1}{2}.$$
$\Rightarrow\left(d_1,d_2\right)=60^\circ $.
}
\end{ex}
\begin{ex}%[2H5H2-7]
	Cho đường thẳng $\Delta \colon \dfrac{x}{1}=\dfrac{y}{-2}=\dfrac{z}{1}$ và mặt phẳng $(P) \colon 5x+11y+2z-4=0$. Góc giữa đường thẳng $\Delta $ và mặt phẳng $(P)$ là
\choice
{$60^\circ $}
{$-30^\circ $}
{\True $30^\circ $}
{$-60^\circ $}
\loigiai{
	Gọi $\overrightarrow{u}$, $\overrightarrow{n}$ lần lượt là véc-tơ chỉ phương, pháp tuyến của đường thẳng $\Delta $ và mặt phẳng $(P)$ thì $\overrightarrow{u}=\left(1;-2;1\right)$, $\overrightarrow{ n}=\left(5;11;2\right).$\\
	Áp dụng công thức ta có $$\sin\left(\Delta ,(P)\right)=\left|\cos\left(\overrightarrow{u} ,\overrightarrow{n}\right)\right|=\dfrac{\left|\overrightarrow{u} \cdot \overrightarrow{n}\right|}{\left|\overrightarrow {u}\right|\cdot\left|\overrightarrow {n}\right|}=\dfrac{\left|1\cdot 5-11\cdot 2+1 \cdot2\right|}{\sqrt{5^2+11^2+2^2} \cdot\sqrt{1^2+2^2+1^2}}=\dfrac{1}{2}\\
	\Rightarrow\left(\Delta ,(P)\right)=30^\circ.$$
}
\end{ex}
\begin{ex}%[2H5H2-7]
Trong KG $Oxyz$ cho đường thẳng $d \colon \heva{&x=1-t\\&y=2+2t\\&z=3+t}$ và mặt phẳng $(P) \colon x-y+3=0$. Tính số đo góc giữa đường thẳng $d$ và mặt phẳng (P).
	\choice
	{\True $60^\circ$}
	{$30^\circ$}
	{$120^\circ$}
	{$45^\circ$}
	\loigiai{
		Đường thẳng $d$ có véc tơ chỉ phương là $\overrightarrow{u}=\left(-1;2;1\right).$\\
		Mặt phẳng $(P)$ có véc tơ pháp tuyến là $\overrightarrow{ n}=\left(1;-1;0\right)$.\\
		Gọi $\alpha $ là góc giữa đường thẳng $d$ và mặt phẳng $(P)$. Khi đó ta có
		$$\sin\alpha=\dfrac{\left|\overrightarrow{ u} \cdot \overrightarrow{n}\right|}{\left|\overrightarrow{u}\right| \cdot \left| \overrightarrow{ n}\right|}=\dfrac{\left|-1 \cdot 1+2\cdot (-1)+1 \cdot 0\right|}{\sqrt{\left(-1\right)^2+2^2+1^2}\cdot \sqrt{1^2+\left(-1\right)^2+0^2}}=\dfrac{3}{2\sqrt 3}=\dfrac{\sqrt 3}{2}.$$
		Do đó $\alpha=60^\circ$.}
	\end{ex}
\begin{ex}%[2H5H2-7]
Trong KG $Oxyz$, cho mặt phẳng $(P) \colon -\sqrt 3 x+y+1=0$. Tính góc tạo bởi $(P)$ với trục $Ox$.
\choice
{\True $60^\circ$}
{$30^\circ$}
{$120^\circ$}
{$150^\circ$}
\loigiai{
Mặt phẳng $(P)$ có véc-tơ pháp tuyến $\overrightarrow{n}=(-\sqrt 3 ;1;0).$\\
Trục $Ox$ có véc-tơ chỉ phương $\overrightarrow{ i}=(1;0;0).$\\
Góc tạo bởi $(P)$ với trục $Ox$ là
	$$ \sin((P),Ox)=\left|\cos((P),Ox) \right|=\dfrac{\left|\overrightarrow{n} \cdot \overrightarrow{i}\right|}{\left|\overrightarrow{n}\right| \cdot \left|\overrightarrow{i} \right|}=\dfrac{\left|-\sqrt{3}  \cdot 1+1 \cdot 0+0 \cdot 0\right|}{\sqrt{3+1}\cdot\sqrt 1}=\dfrac{\sqrt 3}{2}.$$
	Vậy góc tạo bởi $(P)$ với trục $Ox$ bằng $60^\circ.$}
\end{ex}
\begin{ex}%[2H5H2-7]
Cho mặt phẳng $(P)\colon 3x+4y+5z+2=0$ và đường thẳng $d$ là giao tuyến của hai mặt phẳng $(\alpha)\colon x-2y+1=0$, $(\beta)\colon x-2z-3=0$. Gọi $\varphi $ là góc giữa đường thẳng $d$ và mặt phẳng $(P)$. Khi đó
\choice
{\True $60^\circ $}
{$45^\circ $}
{$30^\circ $}
{$90^\circ $}
\loigiai{
	Đường thẳng d có phương trình: $\heva{ & x  =  2t\\&y= \dfrac{1}{2} + t\\&z = -\dfrac{3}{2}+t}, t  \in  R$.\\
	Suy ra véc-tơ chỉ phương của $d$ là $\overrightarrow{u_d}=(2; 1; 1)$.\\
	Ta có $\sin\left(d,(P)\right)=\left|\cos\left(\overrightarrow{u_d},\overrightarrow n\right)\right|=\dfrac{\left|\overrightarrow{u_d}\cdot \overrightarrow n\right|}{\left|\overrightarrow{u_d}\right|\cdot \left|\overrightarrow n\right|}=\dfrac{\left|2\cdot 3  +  1\cdot 4+1\cdot 5\right|}{\sqrt{2^2  +1^2+1^2}\cdot \sqrt{3^2+4^2+5^2}}=\dfrac{\sqrt 3}{2}$.\\
	$\Rightarrow(d,(P))=60^\circ $.}
\end{ex}

\begin{ex}%[2H5H1-6]
Cho hai mặt phẳng $(\alpha)\colon 2x-y+2z-1=0$  và  $(\beta)\colon x+2y-2z-3=0$. Cosin góc giữa mặt phẳng $(\alpha)$ và mặt phẳng $(\beta)$ bằng
\choice
{\True $\dfrac{4}{9}$}
{$-\dfrac{4}{9}$}
{$\dfrac{4}{3\sqrt 3}$}
{$-\dfrac{4}{3\sqrt 3}$}
\loigiai{
Gọi $\overrightarrow{n_\alpha}$, $\overrightarrow{n_\beta}$ lần lượt là vectơ pháp tuyến của mặt phẳng $(\alpha)$ và $(\beta)$ thì $\overrightarrow{n_\alpha}=(2;-1;2)$, $\overrightarrow{n_\beta}=(1;2;-2)$.\\
	Áp dụng công thức:
	$$\cos((\alpha),(\beta))=\left|\cos(\overrightarrow{n_\alpha},\overrightarrow{n_\beta})\right|=\dfrac{\left|\overrightarrow{n_\alpha}\cdot \overrightarrow{n_\beta}\right|}{\left|\overrightarrow{n_\alpha}\right|\cdot \left|\overrightarrow{n_\beta}\right|}=\dfrac{\left|2\cdot 1-1\cdot 2-2\cdot 2\right|}{\sqrt{2^2+(-1)^2+2^2}\cdot \sqrt{(1^2+2^2+(-2)^2}}=\dfrac{4}{9}.$$}
	\end{ex}
\begin{ex}%[2H5H1-6]
Hai mặt phẳng nào dưới đây tạo với nhau một góc $60^\circ $?
\choice
{$(P)\colon 2x+11y-5z+3=0$ và $(Q)\colon x+2y-z-2=0$}
{\True $(P)\colon 2x+11y-5z+3=0$ và $(Q)\colon -x+2y+z-5=0$}
{$(P)\colon 2x-11y+5z-21=0$ và $(Q)\colon 2x+y+z-2=0$}
{$(P)\colon 2x-5y+11z-6=0$ và $(Q)\colon -x+2y+z-5=0$}
\loigiai{
Áp dụng công thức tính góc giữa hai mặt phẳng.
$$ \cos\left((P),(Q)\right)=\dfrac{\left|\overrightarrow{n_P} \cdot \overrightarrow{n_Q} \right|}{\left|\overrightarrow{n_P} \right| \cdot \left| \overrightarrow{n_Q}  \right|}=\cos 60^\circ=\dfrac{1}{2}.$$}	
\end{ex}
\begin{ex}%[2H5H1-6]
Tính tổng các giá trị tham số $m$ để mặt phẳng $(P)\colon \left(m+2\right)x+2my-mz+5=0$ và $(Q)\colon mx+(m-3)y+2z-3=0$ hợp với nhau một góc $\alpha=90^\circ$.
\choice
	{\True $6$}
	{$4$}
	{$8$}
	{$-4$}
\loigiai{
Phhương pháp giải: Xác định các vectơ pháp tuyến của mặt phẳng $(P)$ và $(Q)$. Thay các giá trị vào biểu thức để tìm giá trị đúng. Dùng chức năng CALC trong máy tính bỏ túi để hỗ trợ việc tính toán nhanh nhất.\\
Mặt phẳng $(P)$, $(Q)$ có véc-tơ pháp tuyến lần lượt là $\overrightarrow{n_P}=\left(m+2;2m;-m\right)$, $ \overrightarrow{n_Q}=\left(m;m-3;2\right)$.\\
Ta có $(P)\perp (Q)$
\begin{eqnarray*}
		&\Leftrightarrow&  \overrightarrow{n_P} \cdot \overrightarrow{n_Q}=0\\
		&\Leftrightarrow& \left(m+2\right)m+2m\left(m-3\right)-2m=0\\
		&\Leftrightarrow& 3m^2-6m=0\\
		&\Leftrightarrow & \hoac{&m=0\\ &	m=6.}		
	\end{eqnarray*}
}
\end{ex}
\Closesolutionfile{ans}
% \indapan{10}{ans/ans-goc1}
\TNTF
\Opensolutionfile{ans}[ans/ans-goc2-DS]
%Câu 3.
\begin{ex}%[2H5H1-6]
	Trong KG $Oxyz$, cho hai mặt phẳng $(P)\colon 2x-y+2z+5=0$ và $(Q)\colon x-y+2=0$. Các mệnh đề sau đây đúng hay sai?
	\choiceTF
	{Góc giữa hai mặt phẳng $(P)$ và $(Q)$ bằng $135^\circ$}
	{\True Góc giữa hai mặt phẳng $(P)$ và $(Q)$ bằng $45^\circ$}
	{Hai mặt phẳng $(P)$ và $(Q)$ song song với nhau}
	{\True Điểm $M\left(0;5;0\right)$ thuộc mặt phẳng $(P)$}
	\loigiai{
Gọi $\alpha $ là góc giữa hai mặt phẳng $(P)$ và $(Q)$.
		$$\cos\alpha=\dfrac{\left|2 \cdot 1-1 \cdot \left(-1\right)+2 \cdot 0\right|}{\sqrt{2^2+\left(-1\right)^2+2^2} \cdot \sqrt{1^2+\left(-1\right)^2+0^2}}=\dfrac{1}{\sqrt 2}.$$ $\Rightarrow\alpha=45^\circ$.\\
Thay $M\left(0;5;0\right)$ vào mặt phẳng $(P)$ ta có $2 \cdot 0-5+2 \cdot 0+5=0\Rightarrow M \in(P)$.}
\end{ex}
\begin{ex}%[2H5H1-6]
Trong KG $Oxyz$, cho mặt phẳng $(Q)\colon x-y-5=0$, và biết hình chiếu của $O$ lên mặt phẳng $(P)$ là $H\left(2;-1;-2\right)$. Các mệnh đề sau đây đúng hay sai?
	\choiceTF
{Góc giữa hai mặt phẳng $(P)$ và $(Q)$ bằng $135^\circ$}
{\True Góc giữa hai mặt phẳng $(P)$ và $(Q)$ bằng $45^\circ $}
{Góc giữa hai mặt phẳng $(P)$ và $(Q)$ bằng $60^\circ $}
{Góc giữa hai mặt phẳng $(P)$ và $(Q)$ bằng $120^\circ$}
	\loigiai{
Mặt phẳng $(Q)$ có một véc-tơ pháp tuyến là $\overrightarrow{n_Q}=\left(1;-1;0\right)$.\\
Hình chiếu của $O$ lên mặt phẳng $(P)$ là $H\left(2;-1;-2\right)$. \\
Suy ra mặt phẳng $(P)$ qua $H$ và nhận $\overrightarrow{OH}=\left(2;-1;-2\right)$ làm véc-tơ pháp tuyến.\\
Gọi $\varphi $ là góc giữa hai mặt phẳng $(P)$ và $(Q)$. Ta có
		$$\cos\varphi=\left|\cos\left( \overrightarrow{OH}, \overrightarrow{n_Q}\right)\right|=\dfrac{\left|2+1+0\right|}{\sqrt{4+1+4} \cdot \sqrt{1+1+0}}=\dfrac{\sqrt 2}{2}\Rightarrow\varphi=45^\circ.$$
	}
\end{ex}
\begin{ex}%[2H5H1-6]
	Trong KG $Oxyz$, cho ba mặt phẳng $(P)\colon 2x-y+2z+3=0$, $(Q)\colon x-y-z-2=1$, $(R)\colon x+2y+2z-2=0$. Gọi $\alpha_1$, $\alpha_2$, $\alpha_3$ lần lượt là góc giữa hai mặt phẳng $(P)$ và $(Q)$, $(Q)$ và $(R)$, $(R)$ và $(P)$. Các mệnh đề sau đây đúng hay sai?
	\choiceTF
	{\True $\alpha_1>\alpha_3>\alpha_2$}
	{$\alpha_2>\alpha_3>\alpha_1$}
	{$\alpha_3>\alpha_2>\alpha_1$}
	{$\alpha_1>\alpha_2>\alpha_3$}
	\loigiai{
Áp dụng công thức tính góc giữa hai mặt phẳng. Sử dụng máy tính bỏ túi để tính góc rồi so sánh các giá trị đó với nhau.
}
\end{ex}

\Closesolutionfile{ans}
% \indapan{3}{ans/ans-goc2-DS}

\Opensolutionfile{ans}[ans/ans-goc3-TLN]
\TNSA
\begin{ex}%[2H5V1-6]
	Trong không gian với hệ trục tọa độ $Oxyz$, cho điểm $H(2;1;2)$, $H$ là hình chiếu vuông góc của gốc tọa độ $O$ xuống mặt phẳng $(P)$. Tính số đo góc giữa mặt phẳng $(P)$ và mặt phẳng $(Q)\colon x+y-11=0$.
\shortans{$45^\circ $}
	\loigiai{
Mặt phẳng $(P)$ qua $O$ và nhận $\overrightarrow{OH}=\left(2;1;2\right)$ làm véc-tơ pháp tuyến.\\
Mặt phẳng $(Q)\colon x-y-11=0$ có véc-tơ pháp tuyến $\overrightarrow{n}=(1;1;0)$.\\
Ta có $$\cos\left(\widehat{(P),(Q)}\right)=\dfrac{\left|\overrightarrow{OH}\cdot \overrightarrow n\right|}{OH \cdot \left|\overrightarrow n\right|}=\dfrac{1}{\sqrt 2}\Rightarrow\widehat{\left((P),(Q)\right)}=45^\circ.$$
	}
\end{ex}



