%%%==============EX_12============%%%
\begin{ex}%[2H5V1-6]
	Cho hình chóp $S.ABCD$ có đáy $ABCD$ là hình vuông cạnh $2a$ cạnh bên $SA=a$ và vuông góc với mặt phẳng đáy. Gọi $M$ là trung điểm cạnh $SD$. Tan của góc tạo bởi hai mặt phẳng $\left(AMC\right)$ và $\left(SBC\right)$ bằng
	\choice
	{\True $\dfrac{\sqrt{5}}{5}$}
	{$\dfrac{2\sqrt{5}}{5}$}
	{$\dfrac{\sqrt{3}}{2}$}
	{$\dfrac{2\sqrt{3}}{3}$}
	\loigiai{
		Chọn hệ trục tọa độ sao cho $A\equiv O$ như hình vẽ
		\begin{center}
			\begin{tikzpicture}[line cap=round, line join=round, >=stealth, font=\footnotesize, thick, blue]
				\def\a{1} \def\h{3}
				\path 	
				(0:0) coordinate (A)
				(210:2*\a) coordinate (B)
				($(B)+(210:1cm)$) coordinate (x)
				(0:4*\a) coordinate (D)
				($(D)+(0:1cm)$) coordinate (y)
				($(B)+(D)-(A)$) coordinate (C)
				($(A)+(90:\h)$) coordinate (S)
				($(S)+(90:1cm)$) coordinate (z)
				($(S)!1/2!(D)$) coordinate (M)
				(intersection of A--C and B--D) coordinate (O)
				;
				\foreach \x/\y in {B/x,D/y,S/z}
				\draw[->] (\x)--(\y);
				\draw[dashed] (B)--(A)--(D) (M)--(A)--(S) (A)--(C);
				\draw[] (B)-- (C)--(D) (B)--(S) (M)--(C)--(S) (D)--(S);
				\foreach \x/\y in {A/-90,B/1560,C/-45,D/45,S/45,M/45}
				\fill[black] (\x) circle (1pt) ($(\y:2.75mm)+(\x)$) node {$\x$};	
				\draw pic[draw,angle radius=2.5mm]{right angle=D--A--S}
				pic[draw,angle radius=2.5mm]{right angle=D--A--B}
				pic[draw,angle radius=2.5mm]{right angle=B--A--S};
				\path 
				(B)--(A) node[above,pos=.5]{$2a$}
				(A)--(D) node[above,pos=.475]{$2a$}
				(A)--(S) node[right,pos=.5]{$a$}
				(x) node[above]{$x$} (y) node[above]{$y$} (z) node[left]{$z$};
			\end{tikzpicture}
		\end{center}
		Ta có
		$\begin{aligned}[t]
			& A(0;0;0), B(2a;0;0), D(0;2a;0), C(2a;2a;0), S(0;0;a), M\left(0;a;\dfrac{a}{2}\right).\\
			\Rightarrow\ & \overrightarrow{SB}=(2a;0;-a), \overrightarrow{SC}=(2a;2a;-a), \overrightarrow{MA}=\left(0;-a;-\dfrac{a}{2}\right), \overrightarrow{MC}=\left(2a;a;-\dfrac{a}{2}\right).\\
			& \overrightarrow{n}_1=\left[\overrightarrow{SB},\overrightarrow{SC}\right]=
			\left(
			\begin{vmatrix}
				0&-a\\2a&-a
			\end{vmatrix};
			\begin{vmatrix}
				-a&2a\\-a&2a
			\end{vmatrix};
			\begin{vmatrix}
				2a&0\\2a&2a
			\end{vmatrix}
			\right)
			=2a^2(1;0;2),\\
			& \overrightarrow{n}_2=\left[\overrightarrow{MA},\overrightarrow{MC}\right]=
			\left(
			\begin{vmatrix}
				-a&-\tfrac{a}{2}\\a&-\tfrac{a}{2}
			\end{vmatrix};
			\begin{vmatrix}
				-\tfrac{a}{2}&0\\-\tfrac{a}{2}&2a
			\end{vmatrix};
			\begin{vmatrix}
				0&-a\\2a&a
			\end{vmatrix}
			\right)
			=a^2(1;-1;2).
		\end{aligned}$\\
		Mặt phẳng $(SBC)$ có một véc-tơ pháp tuyến $\overrightarrow{n}_1$, mặt phẳng $(AMC)$ có một véc-tơ pháp tuyến $\overrightarrow{n}_2$.\\
		Gọi $\alpha$ ($0^\circ \le \alpha \le 90^\circ$) là góc tạo bởi hai mặt phẳng $(AMC)$ và $(SBC)$.\\
		Ta có $\cos\alpha=\left|\cos\left(\overrightarrow{n}_1,\overrightarrow{n}_2 \right)\right| =\dfrac{\left|\overrightarrow{n}_1\cdot\overrightarrow{n}_2\right|}{\left|\overrightarrow{n}_1\right|\cdot\left|\overrightarrow{n}_2\right|} =\dfrac{2a^2\cdot a^2\cdot5}{2a^2\sqrt{5}\cdot a^2\sqrt{6}} =\dfrac{5}{\sqrt{30}}$.\\
		Mà $\tan^2\alpha=\dfrac{1}{\cos^2\alpha}-1 =\left(\dfrac{\sqrt{30}}{5}\right)^2-1 =\dfrac{5}{25}$.\\ 
		Suy ra $\tan \alpha=\dfrac{\sqrt{5}}{5}$.\\
	}
\end{ex}

%%%==============EX_13============%%%
\begin{ex}%[2H5V2-7]
	Cho hình chóp $S.ABCD$ đáy là hình thang vuông tại $A$ và $B$, $AB=BC=a$, $AD=2a$. Biết $SA\perp(ABCD)$, $SA=a$. Gọi $M$ và $N$ lần lượt là trung điểm của $SB$ và $CD$. Tính $\sin$ góc giữa đường thẳng $MN$ và mặt phẳng $(SAC)$.
	\choice
	{\True $\dfrac{3\sqrt{5}}{10}$}
	{$\dfrac{2\sqrt{5}}{5}$}
	{$\dfrac{\sqrt{5}}{5}$}
	{$\dfrac{\sqrt{55}}{10}$}
	\loigiai{
		Trong KG $Oxyz$ chọn $A\equiv O(0;0;0), AB\equiv Ox, AD\equiv Oy, AS\equiv Oz$.
		\begin{center}
			\begin{tikzpicture}[line cap=round, line join=round, >=stealth, font=\footnotesize, thick, blue]
				\def\a{1} \def\h{3}
				\path 	
				(0:0) coordinate (A)
				(225:2*\a) coordinate (B)
				($(B)+(225:1cm)$) coordinate (x)
				(0:6*\a) coordinate (D)
				($(D)+(0:1cm)$) coordinate (y)
				($(B)+(D)-(A)$) coordinate (C')
				($(B)!1/2!(C')$) coordinate (C)
				($(A)+(90:\h)$) coordinate (S)
				($(S)+(90:1cm)$) coordinate (z)
				($(S)!1/2!(B)$) coordinate (M)
				($(C)!1/2!(D)$) coordinate (N)
				(intersection of A--C and B--D) coordinate (O)
				;
				\foreach \x/\y in {B/x,D/y,S/z}	\draw[->] (\x)--(\y);
				\draw[dashed,thin] (B)--(A)--(D) (A)--(S) (A)--(C) (M)--(N);
				\draw[] (B)-- (C)--(D) (B)--(S) (C)--(S) (D)--(S);
				\foreach \x/\y in {A/-90,B/1560,C/-45,D/45,S/45,M/135,N/-90}
				\fill[black] (\x) circle (1pt) ($(\y:3.25mm)+(\x)$) node {$\x$};	
				\draw pic[draw,angle radius=2.5mm]{right angle=D--A--S}
				pic[draw,angle radius=2.5mm]{right angle=D--A--B}
				pic[draw,angle radius=2.5mm]{right angle=B--A--S}
				pic[draw,angle radius=2.5mm]{right angle=A--B--C};
				\path (A) node[xshift=-3mm,yshift=3mm]{$O$}
				(B)--(A) node[above,pos=.5]{$a$}
				(A)--(D) node[above,pos=.5]{$2a$}
				(A)--(S) node[right,pos=.5]{$a$}
				(x) node[above]{$x$} (y) node[above]{$y$} (z) node[left]{$z$};
			\end{tikzpicture}
		\end{center}
		Ta có 
		$\begin{aligned}[t]
			&S(0;0;a), B(a;0;0), D(0;2a;0), C(a;a;0), M\left(\dfrac{a}{2};0;\dfrac{a}{2}\right), N\left(\dfrac{a}{2};\dfrac{3a}{2};0\right).\\
			\Rightarrow&\overrightarrow{MN}=\left(0;\dfrac{3a}{2};\dfrac{-a}{2}\right), \overrightarrow{AS}=(0;0;a); \overrightarrow{AC}=(a;a;0).\\
			&\overrightarrow{n}_{(SAC)}=\left[\overrightarrow{AS},\overrightarrow{AC}\right]=
			\left(
			\begin{vmatrix}
				0&a\\a&0
			\end{vmatrix};
			\begin{vmatrix}
				a&0\\a&a
			\end{vmatrix};
			\begin{vmatrix}
				0&0\\a&0
			\end{vmatrix}
			\right)
			=(-a^2;a^2;0)=a^2(-1;1;0).
		\end{aligned}$\\
		Mặt phẳng $(SAC)$ có một véc-tơ pháp tuyến là $\overrightarrow{n}_{(SAC)}$.\\
		Ta có $\sin\left(MN,(SAC)\right) =\dfrac{\overrightarrow{MN}\cdot\overrightarrow{n}_{(SAC)}}{\left|\overrightarrow{MN} \right|\left|\overrightarrow{n}_{(SAC)}\right|} =\dfrac{\dfrac{3a^3}{2}}{\dfrac{a}{2}\cdot\sqrt{10}\cdot a^2\sqrt{2}} =\dfrac{3\sqrt{5}}{10}$.
	}
\end{ex}

%%%==============EX_14============%%%
\begin{ex}%[2H5V2-7]
	Cho hình chóp tứ giác đều $S.ABCD$ có cạnh đáy bằng $a$ tâm $O$. Gọi $M$ và $N$ lần lượt là trung điểm của $SA$ và $BC$. Biết rằng góc giữa $MN$ và $(ABCD)$ bằng $60^\circ$. Côsin của góc giữa đường thẳng $MN$ và mặt phẳng $(SBD)$ bằng
	\choice
	{$\dfrac{\sqrt{5}}{5}$}
	{$\dfrac{\sqrt{41}}{41}$}
	{\True $\dfrac{2\sqrt{5}}{5}$}
	{$\dfrac{2\sqrt{41}}{41}$}
	\loigiai{
		Chọn hệ trục tọa độ $Oxyz$ như hình vẽ.
		\begin{center}
			\begin{tikzpicture}[line cap=round,line join=round, >=stealth, font=\footnotesize, thick, blue]
				\def\a{1}
				\path 	
				(0:0) coordinate (O)
				(200:4*\a) coordinate (A)
				($(A)+(200:1.5cm)$) coordinate (x)
				(-30:1.5*\a) coordinate (B)
				($(B)+(-30:1.5cm)$) coordinate (y)
				(20:4*\a) coordinate (C)
				($(A)+(C)-(B)$) coordinate (D)
				(90:3.75*\a) coordinate (S)
				($(S)+(90:1cm)$) coordinate (z)
				($(S)!1/2!(A)$) coordinate (M)
				($(B)!1/2!(C)$) coordinate (N)
				($(O)!1/2!(A)$) coordinate (H)
				;
				\foreach \x/\y in {A/x,B/y,S/z}	\draw[->] (\x)--(\y);
				\draw[dashed] (B)--(D)--(O) (O)--(S)--(D)  (A)--(D)--(C)--cycle (M)--(N)--(H)--cycle;
				\draw[] (A)--(B)--(C)--(S)--cycle (S)--(B)
				;
				\foreach \x/\y in {O/55,A/150,B/-90,S/45,C/0,D/150,M/150,N/-45, H/-90}
				\fill[black] (\x) circle (1pt) ($(\y:3mm)+(\x)$) node {$\x$};
				\path 
				(A)--(B) node[below,pos=.5]{$a$} (A)--(D) node[left,pos=.5]{$a$}
				(x) node[above]{$x$} (y) node[above]{$y$} (z) node[left]{$z$};
				\draw pic[draw,angle radius=2mm]{right angle=A--O--S}
				pic[draw,angle radius=2mm]{right angle=A--H--M}
				pic[draw,angle radius=5mm,angle eccentricity=1.75,"$60^\circ$"]{ angle=M--N--H}
				;
			\end{tikzpicture}
		\end{center}
		Đặt $SO=m, (m>0)$.\\
		Ta có
		$\begin{aligned}[t]
			& A\left(\dfrac{a\sqrt{2}}{2};0;0\right), S\left(0;0;m\right), N\left(-\dfrac{a\sqrt{2}}{4};\dfrac{a\sqrt{2}}{4};0\right), M\left(\dfrac{a\sqrt{2}}{4};0;\dfrac{m}{2}\right).\\
			\Rightarrow& \overrightarrow{MN}=\left(-\dfrac{a\sqrt{2}}{2};\dfrac{a\sqrt{2}}{4};-\dfrac{m}{2}\right).\\
		\end{aligned}$\\
		Mặt phẳng $(ABCD)$ có véc tơ pháp tuyến $\overrightarrow{k}=(0;0;1)$.\\
		Ta có $\sin\left((MN,(ABCD)\right)=\dfrac{\left|\overrightarrow{MN}\cdot\overrightarrow{k} \right|}{\left|\overrightarrow{MN} \right|\left|\overrightarrow{k} \right|} =\dfrac{\dfrac{m}{2}}{\sqrt{\dfrac{5a^2}{8}+\dfrac{m^2}{4}}} =\dfrac{\sqrt{3}}{2} \Leftrightarrow m^2=\dfrac{15a^2}{8}+\dfrac{3m^2}{4}$.\\
		Suy ra $2m^2=15a^2 \Rightarrow m=\dfrac{a\sqrt{30}}{2}$ \\
		Do đó $\overrightarrow{MN} =\left(-\dfrac{a\sqrt{2}}{2};\dfrac{a\sqrt{2}}{4};-\dfrac{a\sqrt{30}}{4}\right)$.\\
		Mặt phẳng $(SBD)$ có véc tơ pháp tuyến là $\overrightarrow{i}=(1;0;0)$.\\
		Ta lại có $\sin\left(MN,(SBD)\right) =\dfrac{\left|\overrightarrow{MN}\cdot\overrightarrow{i}\right|}{\left|\overrightarrow{MN}\right|\left|\overrightarrow{i}\right|} =\dfrac{\dfrac{a\sqrt{2}}{2}}{\sqrt{\dfrac{a^2}{2}+\dfrac{a^2}{8}+\dfrac{30a^2}{16}}} =\dfrac{\sqrt{5}}{5}$.\\
		Suy ra $\cos\left(MN,(SBD)\right) =\dfrac{2\sqrt{5}}{5}$.
	}
\end{ex}

%%%==============EX_15============%%%
\begin{ex}%[2H5V2-7]
	Cho hình chóp $S.ABCD$ có đáy hình vuông. Cho tam giác $SAB$ vuông tại $S$ và góc $SBA$ bằng $30^\circ$. Mặt phẳng $(SAB)$ vuông góc mặt phẳng đáy. Gọi $M$, $N$ là trung điểm $AB$, $BC$. Tìm cô-sin góc tạo bởi hai đường thẳng $\left(SM, DN\right)$.
	\choice
	{$\dfrac{2}{\sqrt{5}}$}
	{\True $\dfrac{1}{\sqrt{5}}$}
	{$\dfrac{1}{\sqrt{3}}$}
	{$\dfrac{\sqrt{2}}{\sqrt{3}}$}
	\loigiai{
		Trong $(SAB)$ kẻ $SH\perp AB$ tại $H$.\\ 
		Ta có $\left\{\begin{aligned}
			& (SAB)\perp (ABCD) \\ 
			& (SAB)\cap (ABCD)=AB\\ 
			& SH\subset (SAB),\ SH\perp AB
		\end{aligned} \right.
		\Rightarrow SH\perp (ABCD)$.\\
		Kẻ tia $Az\parallel SH$ và chọn hệ trục tọa độ $Axyz$ như hình vẽ sau đây.
		\begin{center}
			\begin{tikzpicture}[line cap=round, line join=round, >=stealth, font=\footnotesize, thick, blue]
				\def\a{1} \def\h{4}
				\path 	
				(0:0) coordinate (A)
				(210:4*\a) coordinate (B)
				($(B)+(210:1cm)$) coordinate (y)
				(0:5*\a) coordinate (D)
				($(D)+(0:1cm)$) coordinate (x)
				($(B)+(D)-(A)$) coordinate (C)
				($(A)+(90:\h)$) coordinate (A')
				($(A')+(90:.5cm)$) coordinate (z)
				($(A)!1/2!(B)$) coordinate (M)
				($(C)!1/2!(B)$) coordinate (N)
				($(A)!1/4!(B)$) coordinate (H)
				($(H)+(90:\h)$) coordinate (S)
				(intersection of A--A' and S--D) coordinate (A'')
				;
				\foreach \x/\y in {B/y,D/x,A'/z} \draw[->] (\x)--(\y);
				\draw[dashed] (B)--(A)--(D) (H)--(S)--(M) (N)--(D) (S)--(A)--(A'');
				\draw[] (B)-- (C)--(D)--(S)--cycle (S)--(C) (A')--(A'');
				\foreach \x/\y in {A/-90,B/150,C/-45,D/60,M/-90,N/-90, H/-90,S/90}
				\fill[black] (\x) circle (1pt) ($(\y:2.5mm)+(\x)$) node {$\x$};	
				\draw 
				pic[draw,angle radius=5mm,angle eccentricity=1.75,"$30^\circ$"]{ angle=A--B--S}
				pic[draw,angle radius=3mm,angle eccentricity=1.75,"$60^\circ$"]{ angle=S--A--B}
				pic[draw,angle radius=2.5mm,angle eccentricity=1.75]{ angle=S--A--B}
				pic[draw,angle radius=2mm]{right angle=S--H--B}
				pic[draw,angle radius=2mm]{right angle=D--A--A'};
				\path 
				(A)--(D) node[above,pos=.5]{$a$}
				(x) node[above]{$x$} (y) node[above]{$y$} (z) node[left]{$z$};
			\end{tikzpicture}
		\end{center}
		Trong tam giác $SAB$ vuông tại $S$, $SB=AB\cdot\cos \widehat{SBA}=a\cdot\cos30^\circ =\dfrac{a\sqrt{3}}{2}$.\\
		Trong tam giác $SBH$ vuông tại $H$, $BH=SB\cdot\cos\widehat{SBH}=\dfrac{3a}{4}$ và $SH=BH\cdot\sin\widehat{SBA}=\dfrac{a\sqrt{3}}{4}$.\\
		$AH=AB-BH=a-\dfrac{3a}{4}=\dfrac{a}{4}$ $\Rightarrow H\left(0;\dfrac{a}{4};0\right) \Rightarrow S\left(0;\dfrac{a}{4};\dfrac{a\sqrt{3}}{4}\right)$.\\
		Có các điểm $M\left(0;\dfrac{a}{2};0\right)$, $D\left(a;0;0\right)$, $N\left(\dfrac{a}{2};a;0\right)$.\\
		Ta có $\overrightarrow{SM}=\left(0;\dfrac{a}{4};-\dfrac{a\sqrt{3}}{4}\right)$, $\overrightarrow{DN}=\left(-\dfrac{a}{2};a;0\right)$.\\
		Suy ra $\cos\left(SM,DN\right) =\dfrac{\left|\overrightarrow{SM}\cdot\overrightarrow{DN}\right|}{SM\cdot DN} =\dfrac{\dfrac{a^2}{4}}{\dfrac{a}{2}\cdot\dfrac{a\sqrt{5}}{2}}=\dfrac{1}{\sqrt{5}}$.
	}
\end{ex}

%%%==============EX_16============%%%
\begin{ex}%[2H5V2-7]
	Cho hình chóp $S.ABCD$ có đáy $ABCD$ là hình vuông cạnh $a$ cạnh bên $SA$ vuông góc với mặt phẳng đáy, $SA=a\sqrt{2}$. Gọi $M$, $N$ lần lượt là hình chiếu vuông góc của điểm $A$ trên các cạnh $SB$, $SD$. Góc giữa mặt phẳng $(AMN)$ và đường thẳng $SB$ bằng
	\choice
	{$45^\circ$}
	{$90^\circ$}
	{$120^\circ$}
	{\True $60^\circ$}
	\loigiai{
		\begin{center}
			\begin{tikzpicture}[line cap=round, line join=round, >=stealth, font=\footnotesize, thick, blue]
				\def\a{1} \def\h{3}
				\path 	
				(0:0) coordinate (A)
				(210:2*\a) coordinate (B)
				($(B)+(210:1cm)$) coordinate (x)
				(0:3*\a) coordinate (D)
				($(D)+(0:1cm)$) coordinate (y)
				($(B)+(D)-(A)$) coordinate (C)
				($(A)+(90:\h)$) coordinate (S)
				($(S)+(90:1cm)$) coordinate (z)
				($(B)!1/3!(S)$) coordinate (M)
				($(D)!1/3!(S)$) coordinate (N)
				(intersection of A--C and B--D) coordinate (O)
				;
				\foreach \x/\y in {B/x,D/y,S/z} \draw[->] (\x)--(\y);
				\draw[dashed,thin] (B)--(A)--(D)--cycle (A)--(S) (A)--(M)--(N)--cycle;
				\draw[] (B)-- (C)--(D) (B)--(S) (C)--(S) (D)--(S);
				\foreach \x/\y in {A/-90,B/1560,C/-45,D/45,S/45,M/135,N/45}
				\fill[black] (\x) circle (1pt) ($(\y:2.75mm)+(\x)$) node {$\x$};	
				\draw pic[draw,angle radius=2.5mm]{right angle=D--A--S}
				pic[draw,angle radius=2.5mm]{right angle=D--A--B}
				pic[draw,angle radius=2.5mm]{right angle=B--A--S};
				\path 
				(B)--(A) node[above,pos=.5]{$a$}
				(A)--(D) node[above,pos=.475]{$a$}
				(A)--(S) node[above,pos=.4,sloped]{$a\sqrt{2}$}
				(x) node[above]{$x$} (y) node[above]{$y$} (z) node[left]{$z$};
			\end{tikzpicture}
		\end{center}
		Ta có $BC\perp(SAB)
		\Rightarrow BC\perp AM
		\Rightarrow AM\perp (SBC)
		\Rightarrow AM\perp SC$. \\
		Tương tự ta cũng có $AN\perp SC
		\Rightarrow \left(AMN\right)\perp SC$.\\ 
		Gọi $\varphi$ là góc giữa đường thẳng $SB$ và $(AMN)$.\\
		Chọn $a=1$ (đơn vị độ dài) và hệ trục tọa độ $Oxyz$ sao cho $O\equiv A(0;0;0)$, $B(1;0;0)$, $D(0;1;0)$, $S(0;0;\sqrt{2})$, $C(1;1;0)$.\\
		Có các véc-tơ $\overrightarrow{SC}=(1;1;-\sqrt{2})$, $\overrightarrow{SB}=(1;0;-\sqrt{2})$.\\
		Do $(AMN)\perp SC$ nên mặt phẳng $(AMN)$ có một véc-tơ pháp tuyến là $\overrightarrow{SC}$. \\
		Cho nên $\sin\varphi=\left|\cos\left(\overrightarrow{SC},\overrightarrow{SB}\right)\right| =\dfrac{\left|1\cdot1+1\cdot0+(-\sqrt{2})\cdot(-\sqrt{2})\right|}{2\cdot\sqrt{3}}=\dfrac{\sqrt{3}}{2}
		\Rightarrow \varphi=60^\circ$.\\
		Vậy góc giữa mặt phẳng $(AMN)$ và đường thẳng $SB$ bằng $60^\circ$.
	}
\end{ex}

%%%==============EX_17============%%%
\begin{ex}%[2H5V2-7]
	Cho hình chóp $S.ABCD$ có đáy $ABCD$ là hình chữ nhật, $AB=a$, $BC=a\sqrt{3}$, $SA=a$ và $SA$ vuông góc với đáy $ABCD$. Tính $\sin \alpha$ với $\alpha$ là góc tạo bởi giữa đường thẳng $BD$ và mặt phẳng $(SBC)$.
	\choice
	{$\sin \alpha=\dfrac{\sqrt{7}}{8}$}
	{$\sin \alpha=\dfrac{\sqrt{3}}{2}$}
	{\True $\sin \alpha=\dfrac{\sqrt{2}}{4}$}
	{$\sin \alpha=\dfrac{\sqrt{3}}{5}$}
	\loigiai{
		Đặt hệ trục tọa độ $Oxyz$ như hình vẽ.
		\begin{center}
			\begin{tikzpicture}[line cap=round, line join=round, >=stealth, font=\footnotesize, thick, blue]
				\def\a{1} \def\h{3}
				\path 	
				(0:0) coordinate (A)
				(210:2*\a) coordinate (B)
				($(B)+(210:1cm)$) coordinate (x)
				(0:4*\a) coordinate (D)
				($(D)+(0:1cm)$) coordinate (y)
				($(B)+(D)-(A)$) coordinate (C)
				($(A)+(90:\h)$) coordinate (S)
				($(S)+(90:1cm)$) coordinate (z)
				($(B)!1/3!(S)$) coordinate (M)
				($(D)!1/3!(S)$) coordinate (N)
				(intersection of A--C and B--D) coordinate (O)
				;
				\foreach \x/\y in {B/x,D/y,S/z} \draw[->] (\x)--(\y);
				\draw[dashed] (B)--(A)--(D)--cycle (A)--(S) ;
				\draw[] (B)-- (C)--(D) (B)--(S) (C)--(S) (D)--(S);
				\foreach \x/\y in {A/-90,B/1560,C/-45,D/45,S/45}
				\fill[black] (\x) circle (1pt) ($(\y:3mm)+(\x)$) node {$\x$};	
				\draw pic[draw,angle radius=2.5mm]{right angle=D--A--S}
				pic[draw,angle radius=2.5mm]{right angle=D--A--B}
				pic[draw,angle radius=2.5mm]{right angle=B--A--S};
				\path 
				(B)--(A) node[above,pos=.5]{$a$}
				(B)--(C) node[below,pos=.5]{$a\sqrt{3}$}
				(A)--(S) node[right,pos=.5]{$a$}
				(x) node[above]{$x$} (y) node[above]{$y$} (z) node[left]{$z$};
			\end{tikzpicture}
		\end{center}
		Khi đó, ta có $A(0;0;0)$, $B(a;0;0)$, $D\left(0;a\sqrt{3};0\right)$, $S(0;0;a)$.\\
		Nên đường thẳng $BD$ có một véc-tơ chỉ phương là $\overrightarrow{u}=\left(-1;\sqrt{3};0\right)$.\\
		Ta có 
		$\begin{aligned}[t]
			\overrightarrow{BD}&=\left(-a;a\sqrt{3};0\right)=a\left(-1;\sqrt{3};0\right),\\
			\overrightarrow{SB}&=\left(a;0;-a\right),\\ \overrightarrow{BC}&=\left(0;a\sqrt{3};0\right),\\
			\Rightarrow \left[\overrightarrow{SB},\overrightarrow{BC}\right] &=\left(a^2\sqrt{3};0;a^2\sqrt{3}\right) =a^2\sqrt{3}\left(1;0;1\right).
		\end{aligned}$\\
		Như vậy, mặt phẳng $(SBC)$ có một véc-tơ pháp tuyến là $\overrightarrow{n}=(1;0;1)$.\\
		Do đó, $\alpha$ là góc tạo bởi giữa đường thẳng $BD$ và mặt phẳng $(SBC)$\\ thì
		$\sin \alpha=\dfrac{\left|\overrightarrow{u}\cdot\overrightarrow{n}\right|}{\left|\overrightarrow{u} \right|\cdot\left|\overrightarrow{n} \right|} =\dfrac{\left|(-1)\cdot1+\sqrt{3}\cdot0+0\cdot1\right|}{\sqrt{(-1)^2+\sqrt{3}^2+0^2}\cdot\sqrt{1^2+0^2+1^2}}=\dfrac{\sqrt{2}}{4}$.
	}
\end{ex}

%%%==============EX_18============%%%
\begin{ex}%[2H5V1-6]
	Cho hình lăng trụ tam giác đều $ABC.A'B'C'$ có $AB=2\sqrt{3}$ và $AA'=2$. Gọi $M$, $N$, $P$ lần lượt là trung điểm các cạnh $A'B'$, $A'C'$ và $BC$ (tham khảo hình vẽ bên). Cô-sin của góc tạo bởi hai mặt phẳng $(AB'C')$ và $(MNP)$ bằng
	\begin{center}
		\begin{tikzpicture}[line cap=round, line join=round, >=stealth, font=\footnotesize, thick, blue]
			\def\a{1} \def\h{4.5}
			\path 	
			(0:0) coordinate (A)
			(180:6*\a) coordinate (B)
			(135:3*\a) coordinate (C)
			($(A)+(90:\h)$) coordinate (A')
			($(B)+(90:\h)$) coordinate (B')
			($(C)+(90:\h)$) coordinate (C')
			($(B')!1/2!(A')$) coordinate (M)
			($(C')!1/2!(A')$) coordinate (N)
			($(B)!1/2!(C)$) coordinate (P)
			;
			\draw[dashed] (A)--(C)--(B) (C)--(C')--(A) (M)--(P)--(N) (P)--(A);
			\draw[]	(A)--(B)--(B')--(A')--cycle (B')--(C')--(A') (M)--(N);
			\foreach \x/\y in {A/0,B/180,C/30,A'/0,B'/180,C'/90, M/135,N/45,P/-90}
			\fill[black] (\x) circle (1pt) ($(\y:3mm)+(\x)$) node {$\x$};	
		\end{tikzpicture}
	\end{center}
	\choice
	{$\dfrac{17\sqrt{13}}{65}$}
	{$\dfrac{18\sqrt{13}}{65}$}
	{$\dfrac{6\sqrt{13}}{65}$}
	{\True $\dfrac{\sqrt{13}}{65}$}
	\loigiai{
		Gắn hệ trục tọa độ $Oxyz$ như hình vẽ.
		\begin{center}
			\begin{tikzpicture}[line cap=round, line join=round, >=stealth, font=\footnotesize, thick, blue]
				\def\a{1} \def\h{4.5}
				\path 	
				(0:0) coordinate (A)
				(180:6*\a) coordinate (B)
				(135:3*\a) coordinate (C)
				($(A)+(90:\h)$) coordinate (A')
				($(B)+(90:\h)$) coordinate (B')
				($(C)+(90:\h)$) coordinate (C')
				($(B')!1/2!(A')$) coordinate (M)
				($(C')!1/2!(A')$) coordinate (N)
				($(B)!1/2!(C)$) coordinate (P)
				($(B')!1/2!(C')$) coordinate (P')
				($(A)!-1cm!(P)$) coordinate (x)
				($(B)!-1cm!(C)$) coordinate (y)
				($(P')!-1cm!(P)$) coordinate (z)
				;
				\foreach \x/\y in {A/x,B/y,P'/z} \draw[->] (\x)--(\y);
				\draw[dashed] (A)--(C)--(B) (C)--(C')--(A) (M)--(P)--(N) (P')--(P)--(A);
				\draw[]	(A)--(B)--(B')--(A')--cycle (B')--(C')--(A') (M)--(N) (A)--(B');
				\foreach \x/\y in {A/30,B/150,C/30,A'/0,B'/180,C'/90, M/135,N/45,P/150}
				\fill[black] (\x) circle (1pt) ($(\y:3mm)+(\x)$) node {$\x$};	
				\path 
				(B)--(A) node[below,pos=.5]{$2\sqrt{3}$}
				(A)--(A') node[right,pos=.5]{$2$}
				(P) node[below]{$O$}
				(x) node[above]{$x$} (y) node[above]{$y$} (z) node[left]{$z$};
				\fill[cyan,opacity=.5] (A)--(B')--(C');
				\fill[green,opacity=.5] (M)--(N)--(P);
			\end{tikzpicture}
		\end{center}
		Ta có
		$\begin{aligned}[t]
			& P(0;0;0), A(3;0;0), B(0;\sqrt{3};0), C(0;-\sqrt{3};0), A'(3;0;2), B'(0;\sqrt{3};2), C'(0;-\sqrt{3};2),\\
			& M\left(\dfrac{3}{2};\dfrac{\sqrt{3}}{2};2\right), N\left(\dfrac{3}{2};-\dfrac{\sqrt{3}}{2};2\right).\\
			\Rightarrow\ & \overrightarrow{AB'}=(-3;\sqrt{3};2), \overrightarrow{AC'}=(-3;-\sqrt{3};2), \overrightarrow{PM}=\left(\dfrac{3}{2};\dfrac{\sqrt{3}}{2};2\right), \overrightarrow{PN}=\left(\dfrac{3}{2};-\dfrac{\sqrt{3}}{2};2\right).\\
			& \overrightarrow{n}_1 =\left[\overrightarrow{AB'},\overrightarrow{AC'}\right] =2\sqrt{3}(2;0;3), \overrightarrow{n}_2=\left[\overrightarrow{PM},\overrightarrow{PN}\right]=\dfrac{\sqrt{3}}{2}(4;0;-3)
		\end{aligned}$\\
		Ta có véc-tơ pháp tuyến của $(AB'C')$ là $\overrightarrow{n}_1$ và véc-tơ pháp tuyến của $(MNP)$ là $\overrightarrow{n}_2$.\\
		Gọi $\varphi$ là góc giữa hai mặt phẳng $(AB'C')$ và $(MNP)$.\\
		Suy ra $\cos\varphi=\left|\cos\left(\overrightarrow{n}_1,\overrightarrow{n}_2\right)\right| =\dfrac{\left|8-9\right|}{\sqrt{13}\sqrt{25}}=\dfrac{\sqrt{13}}{65}$.
	}
\end{ex}

%%%==============EX_19============%%%
\begin{ex}%[2H5V1-6]
	Cho hình lăng trụ đứng $ABC.A'B'C'$ có $AB=AC=a$, góc $\widehat{BAC}=120^\circ$, $AA'=a$. Gọi $M$, $N$ lần lượt là trung điểm của $B'C'$ và $CC'$. Số đo góc giữa mặt phẳng $(AMN)$ và mặt phẳng $(ABC)$ bằng
	\choice
	{$60^\circ$}
	{$30^\circ$}
	{$\arcsin \dfrac{\sqrt{3}}{4}$}
	{\True $\arccos \dfrac{\sqrt{3}}{4}$}
	\loigiai{
		Gọi $H$ là trung điểm $BC$, $BC=a\sqrt{3}$, $AH=\dfrac{a}{2}$.\\
		\begin{center}
			\begin{tikzpicture}[thick,blue,>=stealth, line cap=round, line join=round, font=\footnotesize]
				\def\a{1} \def\h{4.5}
				\path 	
				(0:0) coordinate (A)
				(-60:3*\a) coordinate (B)
				(0:6*\a) coordinate (C)
				($(A)+(90:\h)$) coordinate (A')
				($(B)+(90:\h)$) coordinate (B')
				($(C)+(90:\h)$) coordinate (C')
				($(B')!1/2!(C')$) coordinate (M)
				($(C)!1/2!(C')$) coordinate (N)
				($(B)!1/2!(C)$) coordinate (H)
				($(A)!-1.5cm!(H)$) coordinate (x)
				($(B)!-1.5cm!(H)$) coordinate (y)
				($(M)!-2cm!(H)$) coordinate (z)
				;
				\foreach \x/\y in {A/x,B/y,M/z} \draw[->] (\x)--(\y);
				\draw[dashed] (M)--(A)--(H) (N)--(A)--(C);
				\draw[] (C)--(C') (B)--(B') (A)--(A') (A)--(B)--(C) (N)--(M)--(H) (A)--(B') (A')--(B')--(C')--cycle;
				\foreach \x/\y in {A/240,B/-90,C/0,A'/180,B'/75,C'/0,M/150,N/0,H/-90}
				\fill[black] (\x) circle (1pt) ($(\y:3mm)+(\x)$) node {$\x$};
				\path 
				(A)--(B) node[below,pos=.5]{$a$}
				(A)--(A') node[left,pos=.5]{$a$}
				(x) node[above]{$x$} (y) node[above]{$y$} (z) node[left]{$z$};
				\draw pic[draw,angle radius=2.5mm]{right angle=M--H--C}
				pic[draw,angle radius=2.5mm]{right angle=M--H--A}
				pic[draw,angle radius=2.5mm]{right angle=B--H--A};
				\fill[cyan,opacity=.5] (A)--(B)--(C);
				\fill[green,opacity=.5] (M)--(N)--(A);
			\end{tikzpicture}
		\end{center}
		Chọn hệ trục tọa độ theo hình vẽ.\\
		Ta có
		$\begin{aligned}[t]
			& H(0;0;0), A\left(\dfrac{a}{2};0;0\right), B\left(0;\dfrac{a\sqrt{3}}{2};0\right), C\left(0;-\dfrac{a\sqrt{3}}{2};0\right), M(0;0;a), N\left(0;-\dfrac{a\sqrt{3}}{2};\dfrac{a}{2}\right). \\
			\Rightarrow\ & \overrightarrow{AM}=\left(-\dfrac{a}{2};0;a\right), \overrightarrow{AN}=\left(0;-\dfrac{a\sqrt{3}}{2};\dfrac{a}{2}\right).\\
			& \overrightarrow{n}=\left[\overrightarrow{AM},\overrightarrow{AN}\right] =\dfrac{a^2}{4}(2\sqrt{3};-1;\sqrt{3}).
		\end{aligned}$\\
		Gọi $\varphi$ là góc giữa mặt phẳng $(AMN)$ và mặt phẳng $(ABC)$.\\
		Mặt phẳng $(AMN)$ có một véc-tơ pháp tuyến là $\overrightarrow{n}$. \\
		Mặt phẳng $(ABC)$ có một véc-tơ pháp tuyến $\overrightarrow{HM}=(0;0;1)$.\\
		Từ đó $\cos\varphi=\dfrac{\left|\overrightarrow{n}\cdot\overrightarrow{HM}\right|}{\left|{\overrightarrow{n}}\right|\cdot \left|\overrightarrow{HM}\right|} =\dfrac{\sqrt{3}}{4\cdot1}=\dfrac{\sqrt{3}}{4}$.
	}
\end{ex}

%%%==============EX_20============%%%
\begin{ex}%[2H5V1-6]
	Cho hình chóp $S.ABCD$ có đáy $ABCD$ là hình vuông cạnh $a$ cạnh bên $SA=2a$ và vuông góc với mặt phẳng đáy. Gọi $M$ là trung điểm cạnh $SD$. Tan của góc tạo bởi hai mặt phẳng $(AMC)$ và $(SBC)$ bằng
	\choice
	{$\dfrac{\sqrt{5}}{5}$}
	{$\dfrac{\sqrt{3}}{2}$}
	{\True $\dfrac{2\sqrt{5}}{5}$}
	{$\dfrac{2\sqrt{3}}{3}$}
	\loigiai{
		Chọn hệ trục toạ độ theo hình vẽ.
		\begin{center}
			\begin{tikzpicture}[line cap=round, line join=round, >=stealth, font=\footnotesize, thick, blue]
				\def\a{1} \def\h{3}
				\path 	
				(0:0) coordinate (A)
				(210:2*\a) coordinate (B)
				($(B)+(210:1cm)$) coordinate (x)
				(0:3*\a) coordinate (D)
				($(D)+(0:1cm)$) coordinate (y)
				($(B)+(D)-(A)$) coordinate (C)
				($(A)+(90:\h)$) coordinate (S)
				($(S)+(90:1cm)$) coordinate (z)
				($(D)!1/2!(S)$) coordinate (M)
				($(D)!1/3!(S)$) coordinate (N)
				(intersection of A--C and B--D) coordinate (O)
				;
				\foreach \x/\y in {B/x,D/y,S/z} \draw[->] (\x)--(\y);
				\draw[dashed,thin] (B)--(A)--(D)--cycle (A)--(S) (C)--(A)--(M);
				\draw[] (B)-- (C)--(D) (B)--(S) (M)--(C)--(S) (D)--(S);
				\foreach \x/\y in {A/-90,B/1560,C/-45,D/45,S/45,M/45}
				\fill[black] (\x) circle (1pt) ($(\y:3mm)+(\x)$) node {$\x$};	
				\draw pic[draw,angle radius=2.5mm]{right angle=D--A--S}
				pic[draw,angle radius=2.5mm]{right angle=D--A--B}
				pic[draw,angle radius=2.5mm]{right angle=B--A--S};
				\path 
				(B)--(A) node[above,pos=.5]{$a$}
				(B)--(C) node[below,pos=.5]{$a$}
				(A)--(S) node[right,pos=.4]{$2a$}
				(x) node[above]{$x$} (y) node[above]{$y$} (z) node[left]{$z$};
			\end{tikzpicture}
		\end{center}
		Ta có $A(0;0;0)$, $B(a;0;0)$, $C(a;a;0)$, $D(0;a;0)$, $S(0;0;2a)$.\\
		Ta có $M$ là trung điểm $SD
		\Rightarrow M\left(0;\dfrac{a}{2};a\right)$.\\
		$\overrightarrow{AM}=\left(0;\dfrac{a}{2};a\right)$, $\overrightarrow{AC}=(a;a;0)$.\\
		$\left[\overrightarrow{AM},\overrightarrow{AC}\right]=\dfrac{a^2}{2}\left(-2;1;-1\right)
		\Rightarrow (AMC)$ có một véc-tơ pháp tuyến $\overrightarrow{n}=(-2;2;-1)$.\\
		$\overrightarrow{SB}=(a;0;-2a)$, $\overrightarrow{SC}=(a;a;-2a)$.\\
		$\left[\overrightarrow{SB},\overrightarrow{SC}\right]=a^2(2;0;1)
		\Rightarrow (SBC)$ có một véc-tơ pháp tuyến $\overrightarrow{k}=(2;0;1)$.\\
		Gọi $\alpha$ là góc giữa hai mặt phẳng $(AMC)$ và $(SBC)$.\\
		Ta có $\cos\alpha =\dfrac{\left|\overrightarrow{n}\cdot\overrightarrow{k}\right|}{\left|{\overrightarrow{n}}\right|\cdot\left|{\overrightarrow{k}}\right|} 
		=\dfrac{5}{3\cdot\sqrt{5}}
		=\dfrac{\sqrt{5}}{3}$.\\
		Do $\tan\alpha >0$ nên $\tan\alpha =\sqrt{\dfrac{1}{\cos^2\alpha}-1}=\dfrac{2\sqrt{5}}{5}$.
	}
\end{ex}

%%%==============EX_21============%%%
\begin{ex}%[2H5V1-6]
	\immini{
	Cho hình chóp $S.ABCD$ có đáy $ABCD$ là hình vuông cạnh $a$ mặt bên $SAB$ là tam giác đều và nằm trong mặt phẳng vuông góc với mặt phẳng $\left(ABCD\right)$. Gọi $G$ là trọng tâm của tam giác $SAB$ và $M,N$ lần lượt là trung điểm của $SC,SD$ (tham khảo hình vẽ bên). Tính cô-sin của góc giữa hai mặt phẳng $\left(GMN\right)$ và $\left(ABCD\right)$.
	\choice
	{$\dfrac{2\sqrt{39}}{39}$}
	{$\dfrac{\sqrt{3}}{6}$}
	{\True $\dfrac{2\sqrt{39}}{13}$}
	{$\dfrac{\sqrt{13}}{13}$}
	}{
		\begin{tikzpicture}[line cap=round, line join=round, >=stealth, x=5mm,y=5mm, thick, blue, font=\scriptsize]
			\def\a{1} \def\h{6}
			\path 	
			(0:0) coordinate (A)
			(220:3*\a) coordinate (B)
			(0:5*\a) coordinate (D)
			($(B)+(D)-(A)$) coordinate (C)
			($(A)!1/2!(B)$) coordinate (H)
			($(H)+(90:\h)$) coordinate (S)
			($(S)!1/2!(C)$) coordinate (M)
			($(S)!1/2!(D)$) coordinate (N)
			($(S)!2/3!(H)$) coordinate (G)
			;
			\draw[dashed] (B)--(A)--(D) (S)--(A) (M)--(G)--(N);
			\draw[] (B)-- (C)--(D)--(S)--cycle (S)--(C) (M)--(N);
			\foreach \x/\y in {A/-90,B/-90,C/-90,D/0,M/-110,N/30, S/90,G/-90}
			\fill[black] (\x) circle (1pt) ($(\y:2.5mm)+(\x)$) node {$\x$};
		\end{tikzpicture}
	}
	\loigiai{
		Chọn hệ trục tọa độ $Oxyz$ như hình vẽ.
		\begin{center}
			\begin{tikzpicture}[line cap=round, line join=round, >=stealth, font=\footnotesize, thick, blue]
				\def\a{1} \def\h{6}
				\path 	
				(0:0) coordinate (A)
				(220:3*\a) coordinate (B)
				($(B)+(220:1cm)$) coordinate (x)
				(0:5*\a) coordinate (D)
				($(B)+(D)-(A)$) coordinate (C)
				($(A)!1/2!(B)$) coordinate (H)
				($(C)!1/2!(D)$) coordinate (H')
				($(H')+(0:2cm)$) coordinate (y)
				($(H)+(90:\h)$) coordinate (S)
				($(S)+(90:1cm)$) coordinate (z)
				($(S)!1/2!(C)$) coordinate (M)
				($(S)!1/2!(D)$) coordinate (N)
				($(S)!2/3!(H)$) coordinate (G)	;
				\foreach \x/\y in {B/x,H'/y,S/z} \draw[->] (\x)--(\y);
				\draw[dashed] (B)--(A)--(D) (H')--(H)--(S)--(A) (M)--(G)--(N);
				\draw[] (B)-- (C)--(D)--(S)--cycle (S)--(C) (M)--(N);
				\foreach \x/\y in {A/-90,B/-90,C/-90,D/0,M/-110,N/30, H/150,S/180,G/180}
				\fill[black] (\x) circle (1pt) ($(\y:3mm)+(\x)$) node {$\x$};	
				\draw 
				pic[draw,angle radius=5mm,angle eccentricity=1.75,"$60^\circ$"]{ angle=A--B--S}
				pic[draw,angle radius=3mm,angle eccentricity=1.75,"$60^\circ$"]{ angle=S--A--B}
				pic[draw,angle radius=2mm]{right angle=S--H--A}
				pic[draw,angle radius=2mm]{right angle=A--H--H'};
				\path 
				(H) node[below]{$O$}
				(B)--(C) node[below,pos=.5]{$a$}
				(x) node[above]{$x$} (y) node[above]{$y$} (z) node[left]{$z$};
			\end{tikzpicture}
		\end{center}
		Khi đó
		$S\left(0;0;\dfrac{a\sqrt{3}}{2} \right)$, $A\left(-\dfrac{a}{2};0;0 \right)$, $B\left(\dfrac{a}{2};0;0 \right)$, $C\left(\dfrac{a}{2};a;0 \right)$, $D\left(-\dfrac{a}{2};a;0 \right)$.\\
		Suy ra $G\left(0;0;\dfrac{a\sqrt{3}}{6}\right)$, $M\left(\dfrac{a}{4};\dfrac{a}{2};\dfrac{a\sqrt{3}}{4}\right)$, $N\left(-\dfrac{a}{4};\dfrac{a}{2};\dfrac{a\sqrt{3}}{4}\right)$.\\
		Ta có mặt phẳng $(ABCD)$ có vectơ pháp tuyến là $\overrightarrow{k}=(0;0;1)$.\\
		Mặt phẳng $(GMN)$ có cặp véc-tơ chỉ phương
		$\heva{&\overrightarrow{GM}=\dfrac{a}{12}\left(3;6;\sqrt{3}\right),\\&\overrightarrow{GN}=\dfrac{a}{12}\left(-3;6;\sqrt{3}\right).}$\\
		Suy ra véc-tơ pháp tuyến $\overrightarrow{n}=\left[\overrightarrow{GM};\overrightarrow{GN}\right] 
		=\dfrac{a^2}{144}
		\left(
		\begin{vmatrix}
			6&\sqrt{3}\\6&\sqrt{3}
		\end{vmatrix};
		\begin{vmatrix}
			\sqrt{3}&3\\\sqrt{3}&-3
		\end{vmatrix};
		\begin{vmatrix}
			3&6\\-3&6
		\end{vmatrix}
		\right)
		=\dfrac{a^2}{24}\left(0;-\sqrt{3};6\right)$.\\
		Gọi $\alpha$ là góc giữa hai mặt phẳng $(GMN)$ và $(ABCD)$.\\
		Ta có
		$\cos\alpha=\dfrac{\left|\overrightarrow{n}\cdot\overrightarrow{k}\right|}{\left|{\overrightarrow{n}}\right|\cdot\left|{\overrightarrow{k}}\right|}=\dfrac{6}{\sqrt{39}}=\dfrac{2\sqrt{39}}{13}$.
	}
\end{ex}

%%%==============EX_22============%%%
\begin{ex}%[2H5V1-6]
	Cho hình lăng trụ đứng $ABC.A'B'C'$ có đáy $ABC$ là tam giác cân với $AB=AC=a$ và góc $\widehat{BAC}=120^\circ$ và cạnh bên $BB'=a$. Gọi $I$ là trung điểm của $CC'$. Tính cô-sin góc giữa hai mặt phẳng $(ABC)$ và $(AB'I)$.
	\choice
	{$\dfrac{\sqrt{3}}{10}$}
	{\True $\dfrac{\sqrt{30}}{10}$}
	{$\dfrac{\sqrt{30}}{30}$}
	{$\dfrac{\sqrt{10}}{30}$}
	\loigiai{
		Gọi $O$ là trung điểm của $BC$.\\ 
		Gắn hệ trục tọa độ như hình vẽ.
		\begin{center}
			\begin{tikzpicture}[line cap=round, line join=round, >=stealth, font=\footnotesize, thick, blue]
				\def\a{1} \def\h{4.5}
				\path 	
				(0:0) coordinate (A)
				(180:4*\a) coordinate (B)
				(45:2*\a) coordinate (C)
				($(A)+(90:\h)$) coordinate (A')
				($(B)+(90:\h)$) coordinate (B')
				($(C)+(90:\h)$) coordinate (C')
				($(C)!1/2!(C')$) coordinate (I)
				($(B)!1/2!(C)$) coordinate (O)
				($(B')!1/2!(C')$) coordinate (O')
				($(A)!-1cm!(O)$) coordinate (x)
				($(B)!-1cm!(C)$) coordinate (y)
				($(O')!-1cm!(O)$) coordinate (z)
				;
				\fill[cyan,opacity=.5] (A)--(B)--(C);
				\fill[green,opacity=.5] (A)--(B')--(I);
				\foreach \x/\y in {A/x,B/y,O'/z} \draw[->] (\x)--(\y);
				\draw[dashed] (C)--(B) (A)--(O)--(O') (B')--(I);
				\draw[]	(A)--(B)--(B')--(A')--cycle (B')--(C')--(A') (A)--(C)--(C') (B')--(A)--(I);
				\foreach \x/\y in {A/-90,B/-90,C/0,A'/0,B'/180,C'/0,O/150,I/0}
				\fill[black] (\x) circle (1pt) ($(\y:3mm)+(\x)$) node {$\x$};	
				\path 
				(B)--(A) node[below,pos=.5]{$a$}
				(A)--(C) node[below,pos=.5]{$a$}
				(B)--(B') node[left,pos=.5]{$a$}
				(x) node[above]{$x$} (y) node[above]{$y$} (z) node[left]{$z$};
				\draw 
				pic[draw,angle radius=4mm]{ angle=C--A--B}
				pic[draw,angle radius=2mm]{right angle=O'--O--C}
				pic[draw,angle radius=2mm]{right angle=A--O--B};
			\end{tikzpicture}
		\end{center}
		Ta có $OB=AB\sin60^\circ=\dfrac{a\sqrt{3}}{2}$ ; $OA=AB\cos60^\circ=\dfrac{a}{2}$.\\
		Suy ra $A\left(\dfrac{a}{2};0;0\right)$, $B\left(0;\dfrac{a\sqrt{3}}{2};0\right)$, $C\left(0;-\dfrac{a\sqrt{3}}{2};0\right)$, $I\left(0;-\dfrac{\sqrt{3}}{2};\dfrac{a}{2}\right)$, ${B}'\left(0;\dfrac{a\sqrt{3}}{2};a\right)$.\\
		Mặt phẳng $(ABC)$ có cặp véc-tơ chỉ phương
		$\left\{\begin{aligned}
			&\overrightarrow{AB}=\left(-\dfrac{a}{2};\dfrac{a\sqrt{3}}{2};\dfrac{a}{2}\right),\\
			&\overrightarrow{AC}=\left(-\dfrac{a}{2};-\dfrac{a\sqrt{3}}{2};\dfrac{a}{2}\right).
		\end{aligned}\right.$\\
		Suy ra véc-tơ pháp tuyến
		$\begin{aligned}[t]
			\overrightarrow{n}_1=\left[\overrightarrow{AB},\overrightarrow{AC}\right]
			&=\left(
			\begin{vmatrix}
				\dfrac{a\sqrt{3}}{2}&0\\-\dfrac{a\sqrt{3}}{2}&0
			\end{vmatrix};
			\begin{vmatrix}
				0&-\dfrac{a}{2}\\0&-\dfrac{a}{2}
			\end{vmatrix};
			\begin{vmatrix}
				-\dfrac{a}{2}&\dfrac{a\sqrt{3}}{2}\\-\dfrac{a}{2}&-\dfrac{a\sqrt{3}}{2}
			\end{vmatrix}
			\right)\\
			&=\left(0;0;\dfrac{a^2\sqrt{3}}{2}\right).
		\end{aligned}$\\
		Mặt phẳng $AB'I$ có cặp véc-tơ chỉ phương
		$\heva{&\overrightarrow{AB'}=\left(-\dfrac{a}{2};\dfrac{a\sqrt{3}}{2};a\right),\\ &\overrightarrow{AI}=\left(-\dfrac{a}{2};-\dfrac{a\sqrt{3}}{2};\dfrac{a}{2}\right).}$\\
		Suy ra véc-tơ pháp tuyến 
		$\begin{aligned}[t]
			\overrightarrow{n}_2=\left[\overrightarrow{A{B}'},\overrightarrow{AI}\right]
			&=\left(
			\begin{vmatrix}
				\dfrac{a\sqrt{3}}{2}&a\\-\dfrac{a\sqrt{3}}{2}&\dfrac{a}{2}
			\end{vmatrix};
			\begin{vmatrix}
				a&-\dfrac{a}{2}\\\dfrac{a}{2}&-\dfrac{a}{2}
			\end{vmatrix};
			\begin{vmatrix}
				-\dfrac{a}{2}&\dfrac{a\sqrt{3}}{2}\\-\dfrac{a}{2}&-\dfrac{a\sqrt{3}}{2}
			\end{vmatrix}
			\right)\\
			&=\left(\dfrac{3a^2\sqrt{3}}{4};-\dfrac{a^2}{4};\dfrac{a^2\sqrt{3}}{2}\right).
		\end{aligned}$\\
		Gọi $\alpha$ là góc giữa hai mặt phẳng $(ABC)$ và $(AB'I)$.\\ 
		Ta có $\cos\alpha=\dfrac{\left|\overrightarrow{n}_1\cdot\overrightarrow{n}_2\right|}{\left|\overrightarrow{n}_1\right|\cdot\left|\overrightarrow{n}_2\right|} =\dfrac{\dfrac{3}{4}}{\dfrac{\sqrt{3}}{2}\cdot\dfrac{\sqrt{10}}{2}} =\sqrt{\dfrac{3}{10}}=\dfrac{\sqrt{30}}{10}$.
	}
\end{ex}
\Closesolutionfile{ans}
\inputansbox{10}{ans/ans-LC-3-C5B2CD4_12-21}