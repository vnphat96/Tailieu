\Opensolutionfile{ans}[ans/C5B2CD3_12-21]
\TN
\begin{ex}%[2H5H2-3][Doan Huy-GV62]
	Trong KG $Oxyz$, cho mặt phẳng $\left(P\right) \colon 2x-y+z-10=0$, điểm $A\left(1 ; 3 ; 2\right)$ và đường thẳng $d \colon \heva{x& =-2+2t \\ y&=1+t \\ z&=1-t}$. Tìm PTĐT $\Delta $ cắt $\left(P\right)$ và $d$ lần lượt tại hai điểm $M$ và $N$ sao cho $A$ là trung điểm của đoạn $MN$. 
	\choice 
	{\True $\dfrac{x+6}{7} =\dfrac{y+1}{4} =\dfrac{z-3}{-1} $}
	{$\dfrac{x-6}{7} =\dfrac{y-1}{4} =\dfrac{z+3}{-1} $}
	{$\dfrac{x-6}{7} =\dfrac{y-1}{-4} =\dfrac{z+3}{-1} $}
	{$\dfrac{x+6}{7} =\dfrac{y+1}{-4} =\dfrac{z-3}{-1} $} 
	\loigiai{
		Theo giả thiết  $N\in d\Rightarrow N\left(2t-2; t+1 ; 1-t\right)$.\\
		Mà $A$ là trung điểm $MN\Rightarrow M\left(4-2t ; 5-t ; 3+t\right)$.\\
		Mặt khác, $M\in \left(P\right)\Leftrightarrow 2\left(4-2t\right)-\left(5-t\right)+\left(3+t\right)-10=0\Leftrightarrow t=-2$.\\
		$\Rightarrow N\left(-6 ; -1 ; 3\right)\Rightarrow \overrightarrow{NA}=\left(7 ; 4 ; -1\right)$.\\
		Đường thẳng $\Delta $ đi qua $N\left(-6 ; -1 ; 3\right)$ và có một véc-tơ chỉ phương là $\overrightarrow{u}=\overrightarrow{NA}=\left(7 ; 4 ; -1\right)$ nên có phương trình chính tắc là $\dfrac{x+6}{7} =\dfrac{y+1}{4} =\dfrac{z-3}{-1} $.} 
\end{ex} 
\begin{ex}%[2H5H2-2]
	Trong KG $Oxyz$, cho đường thẳng $d \colon \dfrac{x+1}{2} =\dfrac{y}{1} =\dfrac{z-2}{1} $, mặt phẳng $\left(P\right) \colon x+y-2z+5=0$ và $A\left(1 ; -1 ; 2\right)$. Đường thẳng $\Delta $ cắt $d$ và $\left(P\right)$ lần lượt tại $M$ và $N$ sao cho $A$ là trung điểm của đoạn thẳng $MN$. Một véc-tơ chỉ phương của $\Delta $ là 
	\choice 
	{$\vec{u}=\left(4 ; 5 ; -13\right)$}
	{\True $\vec{u}=\left(2 ; 3 ; 2\right)$}
	{$\vec{u}=\left(1 ; -1 ; 2\right)$}
	{$\vec{u}=\left(-3 ; 5 ; 1\right)$} 
	\loigiai{
		\immini{
				Ta có $d \colon \dfrac{x+1}{2} =\dfrac{y}{1} =\dfrac{z-2}{1} \Rightarrow \heva{x&=-1+2t \\ y&=t \\ z&=2+t.}$\\
				Do đó $M\in d \Rightarrow M\left(-1+2t ; t ; 2+t\right)$.\\
				Vì $A\left(1 ; -1 ; 2\right)$ là trung điểm $MN$.\\
				Suy ra $N\left(3-2t ; -2-t ; 2-t\right)$.\\
				Mặt khác $N \in \left(P\right) \Rightarrow 3-2t-2-t-2 \left( 2-t \right)+5=0 \Leftrightarrow t = 2$.\\
				$\Rightarrow M \left( 3 ; 2 ; 4 \right) \Rightarrow \overrightarrow{AM}=\left(2 ; 3 ; 2\right)$ là một vec-tơ chỉ phương của $\Delta$.
		}{
		\begin{tikzpicture}[scale=0.7, font=\footnotesize,line join=round, line cap=round, >=stealth]
			\coordinate (B) at (3,2);
			\coordinate (C) at (10,2);
			\coordinate (E) at (0,0);
			\coordinate (D) at ($(C)+(E)-(B)$);
			\coordinate (F) at (6,-2);
			\coordinate (G) at (3,5);
			\node  at  (G) {$\Delta$}; %% không hiểu lỗi gì chỗ này mà không sửa đc. Nhờ Thầy chỉ giúp.
			\coordinate (M) at (3.9,3);
			\coordinate (N) at (4.75,1);
			\coordinate (I) at (0,2);
			\coordinate (J) at (10,4.5);
			\coordinate (A) at ($(N)!0.5!(M)$);
			\coordinate (Q) at (5.18,0);
			\draw(B)--(C)--(D)--(E)--cycle;
			\draw(N)--(G);
			\node [left]  at (J){d};
			\node [right] at (E){P};
			\draw (1.3,0) to [out=90,in=-50] (1,0.65);
			\draw(I)--(J);
			\draw[dashed](N)--(Q);
			\draw(Q)--(F);
			\foreach \i/\g in {A/90,M/70,N/0}{\draw[fill=black](\i) circle (1.5pt) ($(\i)+(\g:5mm)$) node[scale=1]{$\i$};}
		\end{tikzpicture}
		}
	} 
\end{ex} 
\begin{ex}%[2H5H2-8]
	Trong không gian với hệ trục tọa độ $Oxyz$ cho hai đường thẳng $d_{1} \colon \heva{x&=4+t \\ y&=-4-t \\ z&=6+2t} $; $d_{2}  \colon \dfrac{x-5}{2} =\dfrac{y-11}{4} =\dfrac{z-5}{2}$. Đường thẳng $d$ đi qua $A\left(5;-3;5\right)$ cắt $d_{1}$ ; $d_{2} $ lần lượt ở $B$, $C$. Tính tỉ sô $\dfrac{AB}{AC}$. 
	\choice 
	{$2$}
	{$3$}
	{\True $\dfrac{1}{2} $}
	{$\dfrac{1}{3} $} 
	\loigiai{
		$B \in d_{1} \Rightarrow B\left(4+t;-4-t;6+2t\right)$. PTTS của $d_{2} \colon \heva {x&=5+2s \\ y&=11+4s  \\ z&=5+2s} $.\\
		$C\in d_{2} \Rightarrow C\left(5+2s;11+4s;5+2s\right)$.\\
		Khi đó  $\overrightarrow{AB}=(1-t;-1-t;2t+1);\overrightarrow{AC}=(2s;4s+14;2s)$.\\
		Do $A$, $B$, $C$ thẳng hàng $\Leftrightarrow \overrightarrow{AB}$, $\overrightarrow{AC}$ cùng phương.\\
		$\Leftrightarrow \exists k\in \mathbb{R} \colon \overrightarrow{AB}=k \cdot \overrightarrow{AC}$ $\Leftrightarrow \heva{t-1&=2ks \\ -t-1&=4ks+14k \\ 2t+1&=2ks} \Leftrightarrow\heva{t&=-2\\ s&=-3 \\ k&=\dfrac{1}{2} }$. Do đó $\overrightarrow{AB}=\dfrac{1}{2} \overrightarrow{AC}$.\\
		Suy ra $\dfrac{AB}{AC} =\dfrac{1}{2}$.} 
\end{ex} 
\newpage
\begin{dang}{LẬP PTĐT LIÊN QUAN ĐẾN VUÔNG GÓC.}
\end{dang}
\begin{ex}%[2H5H2-3]
	Trong KG $Oxyz$, cho điểm $M\left(1 ; 0 ; 1\right)$ và đường thẳng $d \colon \dfrac{x-1}{1} =\dfrac{y-2}{2} =\dfrac{z-3}{3}$. Đường thẳng đi qua $M$, vuông góc với $d$ và cắt $Oz$ có phương trình là 
	\choice 
	{\True $\heva{x&=1-3t \\ y&=0 \\ z&=1+t} $}
	{$\heva{x&=1-3t \\ y&=0 \\ z&=1-t} $}
	{$\heva{x&=1-3t \\ y&=t \\ z&=1+t} $}
	{$\heva{x&=1+3t \\ y&=0 \\ z&=1+t} $} 
	\loigiai{
		Đường thẳng $d$ có một véc-tơ chỉ phương là $\overrightarrow{u}=\left(1 ; 2 ; 3\right)$.\\
		Gọi $\Delta $ là đường thẳng đi qua $M$, vuông góc với $d$ và cắt $Oz$.\\
		Gọi $N\left(0 ; 0 ; t\right)=\Delta \cap Oz$ $\Rightarrow \overrightarrow{MN}=\left(-1 ; 0 ; t-1\right)$.\\
		$\Delta \perp d \Leftrightarrow \overrightarrow{MN}\cdot \overrightarrow{u}=0$ $\Leftrightarrow t=\dfrac{4}{3} $$\Rightarrow \overrightarrow{MN}=\left(-1 ; 0 ; \dfrac{1}{3} \right)$. \\
		Khi đó $\overrightarrow{MN}$ cùng phương với $\overrightarrow{u_{1} }=\left(-3 ; 0 ; 1\right)$.\\
		Đường thẳng $\Delta $ đi qua điểm $M\left(1;0;1\right)$ và có một véc-tơ chỉ phương $\left(-3;0;1\right)$ nên có phương trình là $\heva{x&=1-3t \\ y&=0 \\ z&=1+t.}$} 
\end{ex} 
\begin{ex}%[2H5H2-4]
	Trong KG $Oxyz$, cho điểm $A\left(2; 1; 3\right)$ và đường thẳng $d \colon \dfrac{x+1}{1} =\dfrac{y-1}{-2} =\dfrac{z-2}{2}$. Đường thẳng đi qua $A$, vuông góc với $d$ và cắt trục $Oy$ có phương trình là
	\choice 
	{\True $\heva{x&=2t \\ y&=-3+4t \\ z&=3t} $}
	{$\heva{x&=2+2t \\ y&=1+t \\ z&=3+3t} $}
	{$\heva{x&=2+2t \\ y&=1+3t \\ z&=3+2t} $}
	{$\heva{x&=2t \\ y&=-3+3t \\ z&=2t} $} 
	\loigiai{
		Gọi đường thẳng cần tìm là $\Delta $.\\
		$d \colon \dfrac{x+1}{1} =\dfrac{y-1}{-2} =\dfrac{z-2}{2} $ có véc-tơ chỉ phương $\vec{u}=\left(1; -2; 2\right)$.\\ 
		Gọi $M\left(0; m; 0\right)\in Oy$, ta có $\overrightarrow{AM}=\left(-2; m-1; -3\right)$. \\
		Do $\Delta \perp d$ $\Leftrightarrow \overrightarrow{AM} \cdot \vec{u}=0$ $\Leftrightarrow -2-2\left(m-1\right)-6=0$$\Leftrightarrow m=-3$. Ta có $\Delta $ có véc-tơ chỉ phương $\overrightarrow{AM}=\left(-2; -4; -3\right)$ nên có phương trình $\heva{x&=2t \\ y&=-3+4t \\ z&=3t.}$} 
\end{ex} 
\begin{ex}%[2H5H2-4]
	Trong không gian với hệ toạ độ $Oxyz$  cho điểm $A\left(1;0;2\right)$ và đường thẳng $d$  có phương trình $\colon$ $\dfrac{x-1}{1} =\dfrac{y}{1} =\dfrac{z+1}{2}$. Viết PTĐT $\Delta$ đi qua $A$, vuông góc và cắt $d$. 
	\choice 
	{$\dfrac{x-1}{2} =\dfrac{y}{2} =\dfrac{z-2}{1} $}
	{$\dfrac{x-1}{1} =\dfrac{y}{-3} =\dfrac{z-2}{1} $}
	{$\dfrac{x-1}{1} =\dfrac{y}{1} =\dfrac{z-2}{1} $}
	{\True $\dfrac{x-1}{1} =\dfrac{y}{1} =\dfrac{z-2}{-1} $} 
	\loigiai{
		\textbf{Cách 1}\\
		Đường thẳng $d  \colon \dfrac{x-1}{1} =\dfrac{y}{1} =\dfrac{z+1}{2}$ có vec-tơ chỉ phương $\vec{u}=\left(1;1;2\right)$. \\
		Gọi $\left(P\right)$ là mặt phẳng qua điểm $A$ và vuông góc với đường thẳng  $d$, nên nhận vec-tơ chỉ phương của $d$ là vec-tơ pháp tuyến $\left(P\right) \colon 1\left(x-1\right)+y+2\left(z-2\right)=0\Leftrightarrow x+y+2z-5=0$. \\
		Gọi $B$ là giao điểm của mặt phẳng $\left(P\right)$ đường thẳng $d \Rightarrow B\left(1+t;t;-1+2t\right)$. \\
		Vì $B\in \left(P\right)\Leftrightarrow \left(1+t\right)+t+2\left(-1+2t\right)-5=0\Leftrightarrow t=1\Rightarrow B\left(2;1;1\right)$. \\
		Ta có đường thẳng $\Delta $ đi qua $A$ và nhận vec-tơ $\overrightarrow{AB}=\left(1;1;-1\right)$ là vec-tơ chỉ phương có dạng $\Delta  \colon \dfrac{x-1}{1} =\dfrac{y}{1} =\dfrac{z-2}{-1}$.\\
		\textbf{Cách 2}\\
		Gọi $d\cap \Delta =B \Rightarrow B\left(1+t;t;-1+2t\right)$. \\
		$\overrightarrow{AB}=\left(t;t;-3+2t\right)$, đường thẳng $d$ có véc-tơ chỉ phương là $\overrightarrow{u_{d} }=\left(1;1;2\right)$. \\
		Vì $d\perp \Delta $ nên $\overrightarrow{AB}\perp \overrightarrow{u_{d} }\Leftrightarrow \overrightarrow{AB}\cdot\overrightarrow{u_{d} }=0\Leftrightarrow t+t+2\left(-3+2t\right)=0\Leftrightarrow t=1$.\\ 
		Suy ra $\overrightarrow{AB}=\left(1;1;-1\right)$. Ta có đường thẳng $\Delta $ đi qua $A\left(1;0;2\right)$ và nhận véc-tơ $\overrightarrow{AB}=\left(1;1;-1\right)$ là véc-tơ chỉ phương có dạng $\Delta  \colon \dfrac{x-1}{1} =\dfrac{y}{1} =\dfrac{z-2}{-1} $.} 
\end{ex} 
\begin{ex}%[2H5H2-5]
	Trong không gian Oxyz, cho đường thẳng $d \colon \dfrac{x+1}{2} =\dfrac{y}{-1} =\dfrac{z+2}{2} $ và mặt phẳng $(P) \colon x+y-z+1=0$. Đường thẳng nằm trong mặt phẳng $(P)$ đồng thời cắt và vuông góc với $d$ có phương trình là
	\choice 
	{$\heva{x&=-1+t \\ y&=-4t \\ z&=-3t} $}
	{$\heva{x&=3+t \\ y&=-2+4t \\ z&=2+t} $}
	{\True $\heva{x&=3+t \\ y&=-2-4t \\ z&=2-3t} $}
	{$\heva{x&=3+2t \\ y&=-2+6t \\ z&=2+t} $} 
	\loigiai{
		$d \colon \heva{x&=-1+2t \\ y&=-t \\ z&=-2+2t} $. \\
		Gọi $\Delta $ là đường thẳng nằm trong $(P)$ vuông góc với $d$ $\overrightarrow{u_{\Delta } }=\left[\overrightarrow{u_{d}};\overrightarrow{n_{P} }\right]=(-1;4;3)$. \\
		Gọi $A$ là giao điểm của $d$ và $(P)$. Tọa độ $A$ là nghiệm của phương trình \\  $(-1+2t)+(-t)-(-2+2t)+1=0$ $\Leftrightarrow t=2\Rightarrow A(3;-2;2)$. \\
		Phương trình $\Delta $ qua $A(3;-2;2)$ có véc-tơ chỉ phương $\overrightarrow{u_{\Delta } }=(-1;4;3)$ có dạng $\heva{x&=3+t \\ y&=-2-4t \\ z&=2-3t.} $} 
\end{ex} 
\begin{ex}%[2H5H2-4]
	Trong KG $Oxyz$, cho hai đường thẳng $d_{1}  \colon \dfrac{x-3}{-1} =\dfrac{y-3}{-2} =\dfrac{z+2}{1} $; $d_{2}  \colon \dfrac{x-5}{-3} =\dfrac{y+1}{2} =\dfrac{z-2}{1} $ và mặt phẳng $\left(P\right) \colon x+2y+3z-5=0$. Đường thẳng vuông góc với $\left(P\right)$, cắt $d_{1} $ và $d_{2} $ có phương trình là 
	\choice 
	{$\dfrac{x-1}{3} =\dfrac{y+1}{2} =\dfrac{z}{1} $}
	{$\dfrac{x-2}{1} =\dfrac{y-3}{2} =\dfrac{z-1}{3} $}
	{$\dfrac{x-3}{1} =\dfrac{y-3}{2} =\dfrac{z+2}{3} $}
	{\True $\dfrac{x-1}{1} =\dfrac{y+1}{2} =\dfrac{z}{3} $} 
	\loigiai{
		Phương trình $d_{1}  \colon \heva{x&=3-t_{1}  \\ y&=3-2t_{1}  \\ z&=-2+t_{1} } $ và $d_{2}  \colon \heva{x&=5-3t_{2}  \\ y&=-1+2t_{2}  \\ z&=2+t_{2}.} $\\
		Gọi đường thẳng cần tìm là $\Delta $. \\
		Giả sử đường thẳng $\Delta $ cắt đường thẳng $d_{1} $ và $d_{2} $ lần lượt tại $A$, $B$. \\
		Gọi $A\left(3-t_{1} ;3-2t_{1} ;-2+t_{1} \right)$, $B\left(5-3t_{2} ;-1+2t_{2} ;2+t_{2} \right)$.\\ $\overrightarrow{AB}=\left(2-3t_{2} +t_{1} ;-4+2t_{2} +2t_{1} ;4+t_{2} -t_{1} \right).$ \\
		véc-tơ pháp tuyến của $\left(P\right)$ là $\vec{n}=\left(1;2;3\right)$. \\
		Do $\overrightarrow{AB}$ và $\vec{n}$ cùng phương nên $\dfrac{2-3t_{2} +t_{1} }{1} =\dfrac{-4+2t_{2} +2t_{1} }{2} =\dfrac{4+t_{2} -t_{1} }{3} $\\
		$\Leftrightarrow \heva{\dfrac{2-3t_{2} +t_{1} }{1} =\dfrac{-4+2t_{2} +2t_{1} }{2}  \\ \dfrac{-4+2t_{2} +2t_{1} }{2} =\dfrac{4+t_{2} -t_{1} }{3} } $$\Leftrightarrow \heva{t_{1} =2. \\ t_{2} =1} $ Do đó $A\left(1;-1;0\right)$, $B\left(2;-1;3\right)$.\\ 
		PTĐT $\Delta $ đi qua $A\left(1;-1;0\right)$ và có véc-tơ chỉ phương $\vec{n}=\left(1;2;3\right)$ là $\dfrac{x-1}{1} =\dfrac{y+1}{2} =\dfrac{z}{3} .$} 
\end{ex} 
\begin{ex}%[2H5H2-4]
	Trong KG $Oxyz$ cho đường thẳng $\Delta  \colon \dfrac{x}{1} =\dfrac{y+1}{2} =\dfrac{z-1}{1} $ và mặt phẳng $\left(P\right) \colon x-2y-z+3=0$. Đường thẳng nằm trong $\left(P\right)$ đồng thời cắt và vuông góc với $\Delta $ có phương trình là
	\choice 
	{$\heva{x&=1+2t \\ y&=1-t \\ z&=2} $}
	{$\heva{x&=-3 \\ y&=-t \\ z&=2t} $}
	{$\heva{x&=1+t \\ y&=1-2t \\ z&=2+3t} $}
	{\True $\heva{x&=1 \\ y&=1-t \\ z&=2+2t} $} 
	\loigiai{
		Ta có $\Delta  \colon \dfrac{x}{1} =\dfrac{y+1}{2} =\dfrac{z-1}{1} $$\Rightarrow \Delta  \colon \heva{x&=t \\ y&=-1+2t \\ z&=1+t.} $ \\
		Gọi $M=\Delta \cap \left(P\right)$ $\Rightarrow M\in \Delta \Rightarrow M\left(t;2t-1;t+1\right)$ $M\in \left(P\right)\Rightarrow t-2\left(2t-1\right)-\left(t+1\right)+3=0 \Leftrightarrow 4-4t=0\Leftrightarrow t=1\Rightarrow M\left(1;1;2\right)$. \\
		Véc-tơ pháp tuyến của mặt phẳng $\left(P\right)$ là $\overrightarrow{n}=\left(1;-2;-1\right)$. \\
		Véc-tơ chỉ phương của đường thẳng $\Delta $ là $\overrightarrow{u}=\left(1;2;1\right)$.\\ 
		Đường thẳng $d$ nằm trong mặt phẳng $\left(P\right)$ đồng thời cắt và vuông góc với $\Delta $. \\
		$\Rightarrow $ đường thẳng $d$ nhận $\dfrac{1}{2} \left[\overrightarrow{n},\overrightarrow{u}\right]=\left(0;-1;2\right)$ làm véc-tơ chỉ phương và $M\left(1;1;2\right)\in d$.\\ 
		$\Rightarrow $ PTĐT $d \colon \heva{x&=1 \\ y&=1-t \\ z&=2+2t.}$} 
\end{ex} 
\begin{ex}%[2H5H2-4]
	Trong KG $Oxyz$ cho $A\left(1; -1; 3\right)$ và hai đường thẳng $d_{1}  \colon \dfrac{x-4}{1} =\dfrac{y+2}{4} =\dfrac{z-1}{-2} ,$ $d_{2}  \colon \dfrac{x-2}{1} =\dfrac{y+1}{-1} =\dfrac{z-1}{1}$. PTĐT qua $A$, vuông góc với $d_{1} $ và cắt $d_{2} $ là 
	\choice 
	{$\dfrac{x-1}{2} =\dfrac{y+1}{1} =\dfrac{z-3}{3} $}
	{$\dfrac{x-1}{4} =\dfrac{y+1}{1} =\dfrac{z-3}{4} $}
	{$\dfrac{x-1}{-1} =\dfrac{y+1}{2} =\dfrac{z-3}{3} $}
	{\True $\dfrac{x-1}{2} =\dfrac{y+1}{-1} =\dfrac{z-3}{-1} $} 
	\loigiai{
		Gọi $d$ là đường thẳng qua $A$ và $d$ cắt $d_{2} $ tại $K$. Khi đó $K\left(2+t; -1-t; 1+t\right)$. \\
		Ta có $\overrightarrow{AK}=\left(1+t; -t; t-2\right)$. Đường $AK\perp d_{1} $$\Leftrightarrow \overrightarrow{AK}\cdot\overrightarrow{u_{1} }=0$, với $\vec{u}_{1} =\left(1; 4; -2\right)$ là một véc-tơ chỉ phương của $d_{1} $. \\
		Do đó $1+t-4t-2t+4=0\Leftrightarrow t=1$, suy ra $\overrightarrow{AK}=\left(2; -1; -1\right)$. \\
		Vậy PTĐT $d \colon \dfrac{x-1}{2} =\dfrac{y+1}{-1} =\dfrac{z-3}{-1}.$} 
\end{ex} 
\begin{ex}%[2H5H2-8]
	Trong KG $Oxyz$ cho điểm $A\left(1 ;-1 ;3\right)$ và hai đường thẳng $d_{1}  \colon \dfrac{x-3}{3} =\dfrac{y+2}{3} =\dfrac{z-1}{-1} $. PTĐT $d$ đi qua $A$, vuông góc với đường thẳng $d_{1} $ và cắt thẳng $d_{2}$. 
	\choice 
	{$\dfrac{x-1}{5} =\dfrac{y+1}{-4} =\dfrac{z-3}{2} $}
	{$\dfrac{x-1}{3} =\dfrac{y+1}{-2} =\dfrac{z-3}{3} $}
	{\True $\dfrac{x-1}{6} =\dfrac{y+1}{-5} =\dfrac{z-3}{3} $}
	{$\dfrac{x-1}{2} =\dfrac{y+1}{-1} =\dfrac{z-3}{3} $} 
	\loigiai{
		Gọi $M\left(2+t ; -1-t ; 1+t\right)=d\cap d_{2} $ với $t \in \mathbb{R}$. \\
		Ta có $\overrightarrow{AM}=\left(1+t ; -t\_ ; -2+t\right)$ và $\overrightarrow{u_{1} }=\left(3 ; 3 ; -1\right)$ là véc-tơ chỉ phương của $d_{1} $. \\
		Mặt khác $\overrightarrow{AM}\cdot\overrightarrow{u_{1} }=0$ nên $3\cdot(1+t)+3 \cdot(-t)-1\cdot\left(-2+t\right)=0\Leftrightarrow t=5$. \\
		$\Rightarrow \overrightarrow{AM}=(6;-5;3)$  là một véc-tơ  chỉ phương của $d$. \\
		Vậy PTĐT có dạng $d \colon  \dfrac{x-1}{6} =\dfrac{y+1}{-5} =\dfrac{z-3}{3}.$} 
\end{ex} 
\begin{ex}%[2H5V2-4]
	Trong không gian $Oxyz,$ cho điểm $M\left(1;-1;2\right)$ và hai đường thẳng $d \colon \heva{x&=t \\ y&=-1-4t \\ z&=6+6t} ,$ $d' \colon \; \dfrac{x}{2} =\dfrac{y-1}{1} =\dfrac{z+2}{-5}$. Phương trình nào dưới đây là PTĐT đi qua $M,$ vuông góc với $d$ và $d'$? 
	\choice 
	{$\dfrac{x-1}{17} =\dfrac{y+1}{14} =\dfrac{z-2}{9} $}
	{$\dfrac{x-1}{14} =\dfrac{y+1}{17} =\dfrac{z+2}{9} $}
	{$\dfrac{x-1}{17} =\dfrac{y+1}{9} =\dfrac{z-2}{14} $}
	{\True $\dfrac{x-1}{14} =\dfrac{y+1}{17} =\dfrac{z-2}{9} $} 
	\loigiai{
		Đường thẳng $d$ có một véc-tơ chỉ phương $\overrightarrow{u}=\left(1;-4;6\right)$. \\
		Đường thẳng $d'$ có một véc-tơ chỉ phương $\overrightarrow{u'}=\left(2;1;-5\right)$. \\
		Gọi $\Delta $ là đường thẳng qua $M,$ vuông góc với $d$ và $d'$ nên có một véc-tơ chỉ phương là  $\overrightarrow{u}_{\Delta } =\left[\overrightarrow{u},\overrightarrow{u'}\right]=\left(14;17;9\right).$ \\
		Vậy PTĐT $\Delta  \colon  \dfrac{x-1}{14} =\dfrac{y+1}{17} =\dfrac{z-2}{9}.$} 
\end{ex} 
\begin{ex}%[2H5H2-5]
	Trong KG $Oxyz$, cho mặt phẳng $\left(P\right) \colon 3x+y+z=0$ và đường thẳng $d \colon \dfrac{x-1}{1} =\dfrac{y}{-2} =\dfrac{z+3}{2} $. Gọi $\Delta $ là đường thẳng nằm trong $\left(P\right)$, cắt và vuông góc với $d$. Phương trình nào sau đây là PTTS của $\Delta $? 
	\choice 
	{$\heva{x&=-2+4t \\ y&=3-5t \\ z&=3-7t} $}
	{\True $\heva{x&=-3+4t \\ y&=5-5t \\ z&=4-7t} $}
	{$\heva{x&=1+4t \\ y&=1-5t \\ z&=-4-7t} $}
	{$\heva{x&=-3+4t \\ y&=7-5t \\ z&=2-7t} $}
	\loigiai{
		Do $\Delta $ nằm trong nằm trong $(P)$ và vuông góc với $d$ nên $\Delta$ có véc-tơ chỉ phương là $$\overrightarrow{u_{\Delta } }=\left[\overrightarrow{n_{\left(P\right)} },\overrightarrow{u_{d} }\right]=\left(4;-5;-7\right).$$
		Gọi $A=\Delta \cap d$ thì $A=\left(P\right)\cap d\Rightarrow A\left(1;0;-3\right)$. \\
		Vậy PTTS của $\Delta $ là $\heva{x&=1+4t \\ y&=0-5t \\ z&=-3-7t} $ hay $\heva{x&=-3+4t \\ y&=5-5t \\ z&=4-7t.} $} 
\end{ex} 
\begin{ex}%[2H5V2-4]
	Trong KG $Oxyz$, cho điểm $A(1;-1;3)$ và hai đường thẳng $d_{1}  \colon \dfrac{x-4}{1} = \dfrac{y+2}{4} = \dfrac{z-1}{-2}$ , $d_{2}  \colon  \dfrac{x-2}{1} = \dfrac{y+1}{-1} = \dfrac{z-1}{1} $. Viết PTĐT $d$ đi qua $A$, vuông góc với đường thẳng $d_{1} $ và cắt đường thẳng $d_2$.
	\choice 
	{\True $\dfrac{x-1}{2}=\dfrac{y+1}{-1}=\dfrac{z-3}{-1}$}
	{$\dfrac{x-1}{6}=\dfrac{y+1}{1}=\dfrac{z-3}{5}$}
	{$\dfrac{x-1}{6}=\dfrac{y+1}{-4}=\dfrac{z-3}{-1}$}
	{$\dfrac{x-1}{2}=\dfrac{y+1}{1}=\dfrac{z-3}{3}$} 
	\loigiai{
		Ta có  $\overrightarrow{u_{d_{1}}}=(1;4;-2)$.\\
		$d_2 \colon \dfrac{x-2}{1}=\dfrac{y+1}{-1}=\dfrac{z-1}{1}$ nên PTTS của $d_2 \colon \heva{x&=2+t \\ y&=-1-t (t \in \mathbb{R}) \\ z&=1+t}.$ \\
		Gọi đường thẳng $d$ cắt đường thẳng $d_2$ tại $M\left(2+t;-1-t;1+t\right)$. \\
		Ta có $ \overrightarrow{AM}=(1+t;-t;t-2)$.\\
		Đường thẳng $d$ đi qua $A$, $M$ nên véc-tơ chỉ phương $\overrightarrow{u_d}=(1+t;-t;t-2)$.\\
		Theo đề bài $d$ vuông góc $d_1$ $\Leftrightarrow \overrightarrow{u_d} \perp \overrightarrow{u_{d_{1}}} \Leftrightarrow \overrightarrow{u_d}\cdot \overrightarrow{u_{d_{1}}} =0 \Leftrightarrow 1\cdot(1+t)+4\cdot(-t)-2\cdot(t-2)=0 \Leftrightarrow t=1$.\\
	$\Rightarrow \overrightarrow{u_d}=(2;-1;-1)$.\\
PTĐT $d$ đi qua $A(1;-1;3)$ và có $\overrightarrow{u_d}=(2;-1;-1)$ có dạng $$\dfrac{x-1}{2}=\dfrac{y+1}{-1}=\dfrac{z-3}{-1}.$$} 
\end{ex} 
\begin{ex}%[2H5V2-5]
	Trong KG $Oxyz$, cho mặt phẳng $(P) \colon x+2y+3z-7=0$ và hai đường thẳng $d_1 \colon \dfrac{x+3}{2}=\dfrac{y+2}{-1}=\dfrac{z+2}{-4}$; $d_2 \colon \dfrac{x+1}{3}=\dfrac{y+1}{2} =\dfrac{z-2}{3}$. Đường thẳng vuông góc mặt phẳng $(P)$ và cắt cả hai đường thẳng $d_1$; $d_2$ có phương trình là 
	\choice 
	{$\dfrac{x+7}{1}=\dfrac{y}{2}=\dfrac{z-6}{3}$}
	{\True $\dfrac{x+5}{1}=\dfrac{y+1}{2}=\dfrac{z-2}{3}$}
	{$\dfrac{x+4}{1}=\dfrac{y+3}{2}=\dfrac{z+1}{3}$}
	{$\dfrac{x+3}{1}=\dfrac{y+2}{2}=\dfrac{z+2}{3}$} 
	\loigiai{
		Gọi $\Delta$ là đường thẳng cần tìm.\\
	$\Delta \cap d_1 =M$ nên $M(-3+2t;-2-t;-2-4t)$.\\
	 $\Delta \cap d_2 =N$ nên $N(-1+3u;-1=2u;2+3u)$.\\
	 $\overrightarrow{MN}=(2+3u-2t;1+2u+t;4+3u+4t)$.\\
	 Ta có $\overrightarrow{MN}$ cùng phương với $\overrightarrow{n_{(p)}}$ nên ta có\\
	 $\dfrac{2+3u-2t}{1}=\dfrac{1+2u+t}{2}=\dfrac{4+3u+4t}{3}$. Giải hệ phương trìng tìm được $\heva{u&=-2 \\ t&=-1.}$\\
	 Khi đó toạ độ điểm $M(-5;-1;2)$ và véc-tơ chỉ phương $\overrightarrow{MN}=(-2;-4;-6)=-2(1;2;3)$.\\
	 PTTS $\Delta $ là $\dfrac{x+5}{1}=\dfrac{y+1}{2}=\dfrac{z-2}{3}.$} 
\end{ex} 
\begin{ex}%[2H5V2-4]
	Trong không gian với hệ trục tọa độ $Oxyz$, cho đường thẳng $\left(\Delta \right)$ đi qua điểm $M\left(0 ; 1 ; 1\right)$, vuông góc với đường thẳng $\left(d_{1} \right) \colon \heva{x&=t \\ y&=1-t \\ z&=-1} \left(t\in \mathbb{R}\right)$ và cắt đường thẳng $\left(d_{2} \right) \colon  \dfrac{x}{2} =\dfrac{y-1}{1} =\dfrac{z}{1}$. Phương trình của $\left(\Delta \right)$ là? 
	\choice 
	{$\heva{x&=0 \\ y&=t \\ z&=1+t} $}
	{\True $\heva{x&=0 \\ y&=1 \\ z&=1+t} $}
	{$\heva{x&=0 \\ y&=1+t \\ z&=1} $}
	{$\heva{x&=0 \\ y&=0 \\ z&=1+t} $} 
	\loigiai{
		Gọi $A\left(2t'; 1+t' ; t'\right)\in \left(d_{2} \right)$ là giao điểm giữa đường thẳng $\left(\Delta \right)$ và đường thẳng $\left(d_{2} \right)$.\\
		 Ta có vecto chỉ phương $\overrightarrow{u_{d_{1} } }=\left(1 ; -1; 0\right)$, $\overrightarrow{MA}=\left(2t'; t' ; t'-1\right)$. \\
		 Theo đề bài $\overrightarrow{u_{d_{1} } }\cdot\overrightarrow{MA}=0\Leftrightarrow 2t'-t'=0\Leftrightarrow t'=0$. \\
		 Suy ra $A\left(0 ; 1 ; 0\right)$. \\
		 Khi đó vecto chỉ phương của đường thẳng $\left(\Delta \right)$ là $\overrightarrow{u_{\Delta } }=\overrightarrow{AM}=\left(0 ; 0 ;1\right)$. \\
		 PTĐT $\left(\Delta \right)$ qua $M\left(0 ; 1 ; 1\right)$ có vecto chỉ phương $\overrightarrow{u_{\Delta } }=\left(0 ; 0 ;1\right)$ có dạng $\heva{x&=0 \\ y&=1 \\ z&=1+t.}$} 
\end{ex} 
\begin{ex}%[2H5V2-4]
	Trong không gian với hệ toạ độ $Oxyz$ cho điểm $A\left(1;0;2\right)$ và đường thẳng $d$ có phương trình $\dfrac{x-1}{1} =\dfrac{y}{1} =\dfrac{z+1}{2} $. Viết PTĐT $\Delta $ đi qua $A$, vuông góc và cắt $d$. 
	\choice 
	{$\dfrac{x-1}{1} =\dfrac{y}{1} =\dfrac{z-2}{1} $}
	{\True $\dfrac{x-1}{1} =\dfrac{y}{1} =\dfrac{z-2}{-1} $}
	{$\dfrac{x-1}{2} =\dfrac{y}{2} =\dfrac{z-2}{1} $}
	{$\dfrac{x-1}{1} =\dfrac{y}{-3} =\dfrac{z-2}{1} $}
	\loigiai{
		Đường thẳng $d \colon \dfrac{x-1}{1} =\dfrac{y}{1} =\dfrac{z+1}{2} $ có vec-tơ chỉ phương $\vec{u}=\left(1;1;2\right)$. \\
		Gọi $\left(P\right)$ là mặt phẳng qua điểm $A$ và vuông góc với đường thẳng $d$, nên nhận vec-tơ chỉ phương của  $d$  là vec-tơ pháp tuyến $\left(P\right) \colon 1\left(x-1\right)+y+2\left(z-2\right)=0\Leftrightarrow x+y+2z-5=0$. \\
		Gọi $B$ là giao điểm của mặt phẳng $\left(P\right)$ và đường thẳng $d\Rightarrow B\left(1+t;t;-1+2t\right)$. \\
		Vì $B\in \left(P\right)\Leftrightarrow \left(1+t\right)+t+2\left(-1+2t\right)-5=0\Leftrightarrow t=1\Rightarrow B\left(2;1;1\right)$. \\
		Ta có đường thẳng $\Delta $ đi qua $A$ và nhận vec-tơ $\overrightarrow{AB}=\left(1;1;-1\right)$ là vec-tơ chỉ phương có dạng $\Delta  \colon \dfrac{x-1}{1} =\dfrac{y}{1} =\dfrac{z-2}{-1}.$}
\end{ex} 
\begin{ex}%[2H5V2-5]
	Trong KG $Oxyz$, cho mặt phẳng $\left(P\right) \colon  x+2y+z-4=0$ và đường thẳng $d \colon \dfrac{x+1}{2}=\dfrac{y}{1}=\dfrac{z+2}{3}$. Phương trình đường thằng $\Delta $ nằm trong mặt phẳng $(P)$, đồng thời cắt và vuông góc với đường thẳng $d$ là
	\choice 
	{$\dfrac{x-1}{5}=\dfrac{y+1}{-1}=\dfrac{z-2}{2}$}
	{$\dfrac{x+1}{5}=\dfrac{y+3}{-1}=\dfrac{z-1}{3}$}
	{$\dfrac{x-1}{5}=\dfrac{y11}{1}=\dfrac{z-1}{-3}$}
	{\True $\dfrac{x-1}{5}=\dfrac{y-1}{-1}=\dfrac{z-1}{-3}$}
	\loigiai{
		Gọi $M=d \cap \Delta$ $\Rightarrow M \in d \colon \dfrac{x+1}{2}=\dfrac{y}{1}=\dfrac{z+2}{3} \Rightarrow M(2t-1;t;3t-2)$.\\
		$M \in \Delta \subset (P) \Rightarrow M \in (P) \colon x+2y+z-4=0 \Rightarrow 2t-12t+3t-2-4=0 \Rightarrow t=1 \Rightarrow M(1;1;1)$.\\
		Vì $\Delta \perp d$ và $ \Delta \subset (P) \Rightarrow \Delta$ có véc-tơ chỉ phương $\vec{u}=\left[\vec{u};\overrightarrow{u}_{d}\right]=(5;-1;-3)$.\\
		Vậy phương trình $\Delta$ là $\Delta \colon \dfrac{x-1}{5}=\dfrac{y-1}{-1}=\dfrac{z-1}{-3}.$
	} 
\end{ex} 
\Closesolutionfile{ans}
\indapan{6}{ans/C5B2CD3_12-21}
