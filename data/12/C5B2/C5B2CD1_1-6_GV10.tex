\section{PHƯƠNG TRÌNH ĐƯỜNG THẲNG}

\subsection{Phương trình đường thẳng}

\subsubsection{Véc-tơ chỉ phương của đường thẳng}
 Cho đường thẳng $\Delta $ và véc-tơ $\overrightarrow{u}$ khác $\overrightarrow{0}$. Véc-tơ $\overrightarrow{u}$ được gọi là véc-tơ chỉ phương của đường thẳng $\Delta $ nếu giá của $\overrightarrow{u}$ song song hoặc trùng với $\Delta $.
 \begin{center}
	\begin{tikzpicture}
		\draw (-1,1)--(3,4)node[below]{$\Delta$};
		\draw[dashed]  (0,0)--(4,3);
		\draw[>=latex, ->, line width=1.5pt] (1.38,1)--(2.71,2)node[above=0.2,midway]{$\overrightarrow{u}$};
	\end{tikzpicture}
\end{center}
 \textbf{Nhận xét:}
\begin{itemize}
	\item  Một đường thẳng hoàn toàn được xác định khi biết một điểm mà nó đi qua và một véc-tơ chỉ phương của nó.
	\item  Nếu $\overrightarrow{u}$ là một véc-tơ chỉ phương của đường thẳng thì $k\cdot\overrightarrow{u}; (k\ne 0)$ cũng là một véc-tơ chỉ phương của đường thẳng đó.
\end{itemize}

\subsubsection{Phương trình tham số của đường thẳng}

 Trong không gian với hệ trục tọa độ $Oxyz$, phương trình tham số của đường thẳng $\Delta $ đi qua điểm $M(x_{0} ;y_{0} ;z_{0} )$ và nhận $\overrightarrow{u}=(a;b;c)$ (với $a^{2} +b^{2} +c^{2} \ne 0$) làm vectơ chỉ phương có dạng
 $$\heva{&x=x_{0} +at \\& {y=y_{0} +bt} \\ {z=z_{0} +ct}.} $$   với $t\in  \mathbb{R}$ ($t$ được gọi là tham số).
\begin{center}
	\begin{tikzpicture}
		\draw[->] (0,0)--(5,0)node[below]{$y$};
		\draw[->] (0,0)--(0,4)node[left]{$z$};
		\draw[->] (0,0)--(-2,-2)node[left]{$x$};
		\fill (0,0)node[left]{$O$};
		\draw (.15,0)|-(0,.15);\fill (.15,.15) circle(0pt); 
		\draw (1,-1)--(5,3)node[below]{$\Delta$};
		\draw[>=latex, ->, line width=1.5pt] (1.5,0.5)--(2.5,1.5)node[above=0.2,midway]{$\overrightarrow{u}$};
		\coordinate (M) at ($(1,-1)!1/3!(5,3)$); 
		\coordinate (N) at ($(1,-1)!2/3!(5,3)$);
		\draw[>=latex, ->, line width=1.5pt] (M)--(N);
		\draw (M) node[right]{$M$}; 
	\end{tikzpicture}
\end{center}

\subsubsection{Phương trình chính tắc của đường thẳng}

 Trong không gian với hệ trục tọa độ $Oxyz$, cho đường thẳng $\Delta $ đi qua điểm $M(x_{0} ;y_{0} ;z_{0} )$ và có vectơ chỉ phương $\overrightarrow{u}=(a;b;c)$. Nếu $a\cdot b\cdot c\ne 0$ thì hệ phương trình $$\dfrac{x-x_{0} }{a} =\dfrac{y-y_{0} }{b} =\dfrac{z-z_{0} }{c} $$ được gọi là \textbf{phương trình chính tắc} của đường thẳng $\Delta $.


\subsubsection{Lập phương trình đường thẳng đi qua hai điểm cho trước}

 Trong không gian với hệ trục tọa độ $Oxyz$, đường thẳng $\Delta $ đi qua hai điểm $A(x_{A} ;y_{A} ;z_{A} )$, $B(x_{B} ;y_{B} ;z_{B} )$ và nhận $\overrightarrow{AB}=(x_{B} -x_{A} ;y_{B} -y_{A} ;z_{B} -z_{A} )$ làm vectơ chỉ phương có
\begin{itemize}
	\item  Phương trình tham số : $\left\{\begin{array}{l} {x=x_{A} +(x_{B} -x_{A} )t} \\ {y=y_{A} +(y_{B} -y_{A} )t} \\ {z=z_{A} +(z_{B} -z_{A} )t} \end{array}\right. $ với $t\in  \mathbb{R}$.
	\item  Phương trình chính tắc: $\dfrac{x-x_{A} }{x_{B} -x_{A} } =\dfrac{y-y_{A} }{y_{B} -y_{A} } =\dfrac{z-z_{A} }{z_{B} -z_{A} } $ (với $x_{B} \ne x_{A}$, $y_{B} \ne y_{A}$, $z_{B} \ne z_{A} $).
\end{itemize}

 \subsection{Vị trí tương đối giữa hai đường thẳng. Điều kiện để hai đường thẳng vuông góc}
\subsubsection{Vị trí tương đối giữa hai đường thẳng}

 Trong không gian, hai véc-tơ được gọi là cùng phương khi giá của chúng cùng song song với một đường thẳng.\\
 Trong không gian, ba véc-tơ được gọi là đồng phẳng khi giá của chúng cùng song song với một mặt phẳng.\\
 Trong không gian với hệ trục tọa độ $Oxyz$, cho ba véc-tơ $\overrightarrow{a}=(a_{1} ;a_{2} ;a_{3} )$, $\overrightarrow{b}=(b_{1} ;b_{2} ;b_{3} )$, $\overrightarrow{c}=(c_{1} ;c_{2} ;c_{3} )$

\begin{itemize}
	\item  Hai $\overrightarrow{a}$, $\overrightarrow{b}$ cùng phương $\Leftrightarrow \left[\overrightarrow{a}, \overrightarrow{b}\right]=\overrightarrow{0}$.
	\item  Hai $\overrightarrow{a}$, $\overrightarrow{b}$ không cùng phương $\Leftrightarrow \left[\overrightarrow{a}, \overrightarrow{b}\right]\ne \overrightarrow{0}$.
	\item  Ba vectơ $\overrightarrow{a}$, $\overrightarrow{b},\overrightarrow{c}$ đồng phẳng $\Leftrightarrow \left[\overrightarrow{a}, \overrightarrow{b}\right].\, \overrightarrow{c}=0$.
	\item  Ba vectơ $\overrightarrow{a}, \overrightarrow{b}, \overrightarrow{c}$ không đồng phẳng $\Leftrightarrow \left[\overrightarrow{a}, \overrightarrow{b}\right].\, \overrightarrow{c}\ne 0$.
\end{itemize}

 Trong không gian với hệ trục tọa độ $Oxyz$, cho hai đường thẳng $\Delta _{1} ,\Delta _{2} $ lần lượt đi qua các điểm $M_{1} ,M_{2} $ và tương ứng có $\overrightarrow{u}_{1} =(a_{1} ;b_{1} ;c_{1} ),{\rm \; }\overrightarrow{u}_{2} =(a_{2} ;b_{2} ;c_{2} )$ là hai véc-tơ chỉ phương. Khi đó, ta có
 \begin{itemize}
 \item $\Delta_1 \equiv \Delta_2 \Leftrightarrow \heva{&\overrightarrow{u}_1, \overrightarrow{u}_2 \text { cùng phương } \\ &\overrightarrow{u}_1,\overrightarrow{M_1 M_2}  \text { cùng phương }} \Leftrightarrow \heva{&{\left[\overrightarrow{u}_1, \overrightarrow{u}_2\right]=\overrightarrow{0}} \\ &{\left[\overrightarrow{u}_1, \overrightarrow{M_1 M_2}\right]=\overrightarrow{0}}.}$
 \item $\Delta_1 \parallel \Delta_2 \Leftrightarrow \heva{&\overrightarrow{u}_1, \overrightarrow{u}_2  \text { cùng phương } \\ &\overrightarrow{u}_1, \overrightarrow{M_1 M_2}  \text { không cùng phương }} \Leftrightarrow \heva{&{\left[\overrightarrow{u}_1, \overrightarrow{u}_2\right]=\overrightarrow{0}} \\ &{\left[\overrightarrow{u}_1, \overrightarrow{M_1 M_2}\right]\ne \overrightarrow{0}}.}$
 \begin{center}
	\begin{tikzpicture}[scale=0.8]
		\draw (0,0)--(5,2)node[above]{$\Delta_1$};
		\draw (4,0)--(9,2)node[below]{$\Delta_2$};
		\coordinate (M_1) at ($(0,0)!1/5!(5,2)$);
		\coordinate (M_2) at ($(4,0)!1/5!(9,2)$);  
		\draw[>=latex, ->, line width=1pt] (M_1)--($(0,0)!3/5!(5,2)$)node[above=0.1,midway]{$\overrightarrow{u}_1$};
		\draw[>=latex, ->, line width=1pt] (M_2)--($(4,0)!3/5!(9,2)$)node[above=0.1,midway]{$\overrightarrow{u}_2$};
		\draw[line width=1pt] (M_1)--(M_2);
		\foreach \x/\g in {M_1/150,M_2/-30}
\fill[black] (\x) circle (1.5pt)
($(\g:4mm)+(\x)$) node {$\x$};	
	\end{tikzpicture}
\end{center}
 \item $\Delta_1$ cắt $\Delta_2 \Leftrightarrow\heva{&\overrightarrow{u}_1, \overrightarrow{u}_2  \text { không cùng phương } \\ &\overrightarrow{u}_1, \overrightarrow{u}_2, \overrightarrow{M_1 M_2}  \text { đồng phẳng }} \Leftrightarrow \heva{&{\left[\overrightarrow{u}_1, \overrightarrow{u}_2\right] \neq \overrightarrow{0}} \\ &{\left[\overrightarrow{u}_1, \overrightarrow{u}_2\right] \cdot \overrightarrow{M_1 M_2} \neq 0}.}$
  \begin{center}
	\begin{tikzpicture}[scale=0.7]
		\draw (5,-2)--(-2.5,1)node[above]{$\Delta_1$};
		\draw (5,2)--(-2.5,-1)node[below]{$\Delta_2$};
		\coordinate (M_1) at ($(0,0)!3/5!(5,-2)$);
		\coordinate (M_2) at ($(0,0)!4/5!(5,2)$);  
		\draw[>=latex, ->, line width=1pt] (0,0)--($(0,0)!2/5!(5,-2)$)node[below=0.1,midway]{$\overrightarrow{u}_1$};
		\draw[>=latex, ->, line width=1pt] (0,0)--($(0,0)!3/5!(5,2)$)node[above=0.1,midway]{$\overrightarrow{u}_2$};
		\draw[>=latex, ->, line width=1pt] (0,0)--(0,3)node[right]{$\left[ \overrightarrow{u}_1,\overrightarrow{u}_2\right]$};
		\draw[line width=1pt] (M_1)--(M_2);
		\foreach \x/\g in {M_1/-90,M_2/90}
\fill[black] (\x) circle (1.5pt)
($(\g:4mm)+(\x)$) node {$\x$};	
	\end{tikzpicture}
\end{center}
 \item $\Delta_1$ và $\Delta_2$ chéo nhau $\Leftrightarrow\left[\overrightarrow{u}_1, \overrightarrow{u}_2\right] \cdot\overrightarrow{M_1 M_2} \neq 0$.
   \begin{center}
	\begin{tikzpicture}[scale=0.7]
		\draw (3,2)--(-2,0)node[above]{$\Delta_2$};
		\draw (5,-2)--(-1,0) node[below]{$\Delta_1$};
		\coordinate (M_2) at ($(3,2)!1/5!(-2,0)$);
		\coordinate (M_1) at ($(5,-2)!1/5!(-1,0)$);  
		\draw[>=latex, ->, line width=1pt] ($(-1,0)!1/5!(5,-2)$)--($(-1,0)!3/5!(5,-2)$)node[below=0.1,midway]{$\overrightarrow{u}_1$};
		\draw[>=latex, ->, line width=1pt] ($(-2,0)!2/5!(3,2)$)--($(-2,0)!3/5!(3,2)$)node[below=0.1,midway]{$\overrightarrow{u}_2$};
		\draw[>=latex, ->, line width=1pt] (0,1)--(0,4)node[left]{$\left[ \overrightarrow{u}_1,\overrightarrow{u}_2\right]$};
		\draw[line width=1pt] (M_1)--(M_2);
		\foreach \x/\g in {M_1/-90,M_2/90}
\fill[black] (\x) circle (1.5pt)
($(\g:4mm)+(\x)$) node {$\x$};	
	\end{tikzpicture}
\end{center}
 \end{itemize}

\begin{note} Chú ý: Để xét vị trí tương đối giữa hai đường thẳng, ta cũng có thể dựa vào các véc-tơ chỉ phương và phương trình của hai đường thẳng đó.
\end{note}
 Trong không gian với hệ trục tọa độ $Oxyz$, cho hai đường thẳng $\Delta _{1} ,\Delta _{2} $ tương ứng có $\overrightarrow{u}_{1} =(a_{1} ;b_{1} ;c_{1} ),{\rm \; }\overrightarrow{u}_{2} =(a_{2} ;b_{2} ;c_{2} )$ là hai vectơ chỉ phương và có phương trình tham số:
\[\Delta _{1} :\left\{\begin{array}{l} {x=x_{1} +a_{1} t_{1} } \\ {y=y_{1} +b_{1} t_{1} } \\ {z=z_{1} +c_{1} t_{1} } \end{array}\right. {\rm \; }\left(t_{1} \in  \mathbb{R}\right),{\rm \; \; }\Delta _{2} :\left\{\begin{array}{l} {x=x_{2} +a_{2} t_{2} } \\ {y=y_{2} +b_{2} t_{2} } \\ {z=z_{2} +c_{2} t_{2} } \end{array}\right. {\rm \; \; }\left(t_{2} \in  \mathbb{R}\right)\] 
 Xét hệ phương trình hai ẩn $t_{1} ,t_{2} $: $\left\{\begin{array}{l} {x_{1} +a_{1} t_{1} =x_{2} +a_{2} t_{2} } \\ {y_{1} +b_{1} t_{1} =y_{2} +b_{2} t_{2} } \\ {z_{1} +c_{1} t_{1} =z_{2} +c_{2} t_{2} } \end{array}\right. $ \quad$\left(*\right)$.\\
 Khi đó
\begin{itemize}
	\item  $\Delta _{1} \equiv \Delta _{2} \Leftrightarrow $ $\overrightarrow{u}_{1} $ cùng phương với $\overrightarrow{u}_{2} $ và hệ $\left(*\right)$ vô nghiệm.
	\item  $\Delta _{1} \parallel \Delta _{2} \Leftrightarrow $ Hệ $\left(*\right)$ có vô số nghiệm.
	\item  $\Delta _{1} $ cắt $\Delta _{2} \Leftrightarrow $ Hệ $\left(*\right)$ có nghiệm duy nhất.
	\item  $\Delta _{1} $ và $\Delta _{2} $ chéo nhau $\Leftrightarrow \overrightarrow{u}_{1} $ không cùng phương với $\overrightarrow{u}_{2} $ và hệ $\left(*\right)$ vô nghiệm.
\end{itemize}

\subsubsection{Điều kiện để hai đường thẳng vuông góc}

 Trong không gian với hệ trục tọa độ $Oxyz$, cho hai đường thẳng $\Delta _{1} ,\Delta _{2} $ tương ứng có $\overrightarrow{u}_{1} =(a_{1} ;b_{1} ;c_{1} ),{\rm \; }\overrightarrow{u}_{2} =(a_{2} ;b_{2} ;c_{2} )$ là hai vectơ chỉ phương. Khi đó
\[\Delta _{1} \bot \Delta _{2} \Leftrightarrow \overrightarrow{u}_{1} \cdot\overrightarrow{u}_{2} =0\Leftrightarrow a_{1} a_{2} +b_{1} b_{2} +c_{1} c_{2} =0\] 

\subsection{Góc}
\subsubsection{Góc giữa hai đường thẳng}

  Trong không gian với hệ trục tọa độ $Oxyz$, cho hai đường thẳng $\Delta _{1} ,\Delta _{2} $ có hai vectơ chỉ phương lần lượt là: $\overrightarrow{u}_{1} =(a_{1} ;b_{1} ;c_{1} ),{\rm \; }\overrightarrow{u}_{2} =(a_{2} ;b_{2} ;c_{2} )$. Khi đó, ta có
\[\cos \left(\Delta _{1} ,\Delta _{2} \right)=\left|\cos \left(\overrightarrow{u}_{1} ,\overrightarrow{u}_{2} \right)\right|=\dfrac{\left|\overrightarrow{u}_{1} \cdot\overrightarrow{u}_{2} \right|}{\left|\overrightarrow{u}_{1} \right|\cdot\left|\overrightarrow{u}_{2} \right|} =\dfrac{\left|a_{1} a_{2} +b_{1} b_{2} +c_{1} c_{2} \right|}{\sqrt{a_{1}^{2} +b_{1}^{2} +c_{1}^{2} } \cdot\sqrt{a_{2}^{2} +b_{2}^{2} +c_{2}^{2} } }. \] 
   \begin{center}
	\begin{tikzpicture}[scale=1.2]
		\draw (-2,1)--(2,-1)node[above]{$\Delta_2'$};
		\draw (-2,-1)--(2,1) node[below]{$\Delta_1'$};
		\draw (-1,1)--(2,5/2)node[above]{$\Delta_1$};
		\draw (-1,-1)--(2,5/-2)node[above]{$\Delta_2$};
		\coordinate (B) at ($(-2,1)!4/5!(2,-1)$);
		\coordinate (A) at ($(-2,-1)!4/5!(2,1)$); 
		\coordinate (O) at (0,0); 
		\draw[>=latex, ->, line width=1pt] (0,0)--(A)node[above=0.1,midway]{$\overrightarrow{u}_1$};
		\draw[>=latex, ->, line width=1pt] (0,2)--($(0,2)+(A)-(0,0)$) node[above=0.1,midway]{$\overrightarrow{u}_1$};
		\draw[>=latex, ->, line width=1pt] (0,0)--(B)node[below=0.1,midway]{$\overrightarrow{u}_2$};
		\draw[>=latex, ->, line width=1pt] (0,-2)--($(0,-2)+(B)-(0,0)$) node[below=0.1,midway]{$\overrightarrow{u}_2$};
		\draw pic[draw,,angle radius=6mm]{angle=B--O--A};
		\foreach \x/\g in {A/90,B/-90}
\fill[black] (\x) circle (1pt) ($(\g:4mm)+(\x)$) node {$\x$};	
	\end{tikzpicture}
\end{center}

\subsubsection{Góc giữa đường thẳng với mặt phẳng}

  Trong không gian với hệ trục tọa độ $Oxyz$, cho đường thẳng $\Delta $ có vectơ chỉ phương $\overrightarrow{u}=(a;b;c)$ và mặt phẳng $(P)$ có vectơ pháp tuyến $\, \overrightarrow{n}=(A;\, B;\, C)$. Khi đó, ta có
\[\sin \left(\Delta ,(P)\right)=\left|\cos \left(\overrightarrow{u}, \overrightarrow{n}\right)\right|=\dfrac{\left|\overrightarrow{u}\cdot \overrightarrow{n}\right|}{\left|\overrightarrow{u}\right|\cdot \left|\, \overrightarrow{n}\right|} =\dfrac{\left|aA+bB+cC\right|}{\sqrt{a^{2} +b^{2} +c^{2} } \cdot\sqrt{A^{2} +B^{2} +C^{2} } }. \] 
\begin{center}
\begin{tikzpicture}[scale=1]
	\def\d{4}
	\def\r{3}
	\path (0:0) coordinate (B)
			++(0:\d) coordinate (C)
			++(60:\r) coordinate (D)
			($(B)+(D)-(C)$) coordinate (A);
	\coordinate (H') at (9/8*\d,\r/2);
	\coordinate (I) at (3/5*\d,\r/2);
	\coordinate (H) at ($(H')!1!-90:(I)$);
	\coordinate (K) at ($(H)!-1/2!(I)$);
	\coordinate (M) at (intersection of B--C and I--H);
	\coordinate (N) at ($(M)!-1/2!(I)$);
	\coordinate (O) at (\d/2,\r);
	\coordinate (ut) at ($(O)+(H)-(I)$);
	\coordinate (nt) at ($(O)+(H)-(H')$);
	\coordinate (u) at ($(O)!1/2!(ut)$);
	\coordinate (n) at ($(O)!1/2!(nt)$);
	\draw[dashed] (I)--(M) (H)--(H');
	\draw (2/5*\d,\r/2)--(6/5*\d,\r/2)node[above]{$\Delta'$} (M)--(N) (I)--(K)node[right]{$\Delta$};
	\draw[>=latex, ->, line width=1pt] (O)--(u) node[below=0.1]{$\overrightarrow{u}$};
	\draw[>=latex, ->, line width=1pt] (O)--(n) node[below=0.1,left]{$\overrightarrow{n}$};
	\draw (A)--(B)--(C)--(D)--cycle;
	\foreach \x/ \goc in {I/-90,H'/-90,H/-45} 
			\fill (\x) circle (1pt)	
			($(\x)+(\goc:3mm)$) node {$\x$};
	\draw pic[draw,"$P$",angle radius=6mm]{angle=C--B--A};
	\end{tikzpicture}
\end{center}
\subsubsection{Góc giữa hai mặt phẳng}

  Trong không gian với hệ trục tọa độ $Oxyz$, cho hai mặt phẳng $(P_{1} ),(P_{2} )$ có hai vectơ pháp tuyến lần lượt là$\, \overrightarrow{n}_{1} =(A_{1} ;\, B_{1} ;\, C_{1} ),\, {\rm \; }\overrightarrow{n}_{2} =(A_{2} ;\, B_{2} ;\, C_{2} )$. Khi đó, ta có
\[\cos \left((P_{1} ),(P_{2} )\right)=\left|\cos \left(\, \overrightarrow{n}_{1} , \overrightarrow{n}_{2} \right)\right|=\dfrac{\left|\overrightarrow{n}_{1} \cdot \overrightarrow{n}_{2} \right|}{\left|\overrightarrow{n}_{1} \right| \cdot\left|\overrightarrow{n}_{2} \right|} =\dfrac{\left|A_{1} A_{2} +B_{1} B_{2} +C_{1} C_{2} \right|}{\sqrt{A_{1}^{2} +B_{1}^{2} +C_{1}^{2} } \cdot\sqrt{A_{2}^{2} +B_{2}^{2} +C_{2}^{2} } }. \] 
\begin{center}
\begin{tikzpicture}[scale=1]
	\def\d{4}
	\def\r{2}
	\path (0:0) coordinate (B)
			++(-20:\d) coordinate (C)
			++(100:\r) coordinate (D)
			($(B)+(D)-(C)$) coordinate (A);
%	\foreach \x/ \goc in {A/180,B/180,C/0,D/0} 
%			\fill (\x) circle (1pt)	
%			($(\x)+(\goc:3mm)$) node {$\x$};
	\path (0,-1/2*\r) coordinate (B')
			++(15:\d) coordinate (C')
			++(120:\r) coordinate (D')
			($(B')+(D')-(C')$) coordinate (A');
	\coordinate (M) at (intersection of B--C and B'--C');
	\coordinate (N) at (intersection of A--D and A'--D');
	\coordinate (P) at (intersection of A--B and A'--D');
	\coordinate (Q) at (intersection of C--D and B'--C');
	\coordinate (O1) at (\d/7,\r/3);
	\coordinate (u1) at (\d/3,4/3*\r);
	\coordinate (O2) at (3*\d/5,\r/4);
	\coordinate (u2) at (\d/2,3/2*\r);
	\fill[orange!25] (M)--(N)--(A)--(B)--(M);
	\fill[orange!25] (M)--(C)--(Q)--(M);
	\fill[green!25] (A')--(B')--(M)--(B)--(P)--(A');
	\fill[green!25] (M)--(N)--(D')--(C')--(M);
	\draw[>=latex, ->, line width=1pt] (O1)--($(O1)!3/4!(u1)$) node[below=0.1,left]{$\overrightarrow{n_1}$};
	\draw[>=latex, ->, line width=1pt] (O2)--($(O2)!3/4!(u2)$) node[below=0.1,right]{$\overrightarrow{n_2}$};
	\draw (O1)--(u1) (O2)--(u2);
	\draw (A')--(B')--(M)--(B)--(P)--(A')  (M)--(C)--(Q)--(M) (M)--(N)--(A)--(B)--(M) (M)--(N)--(D')--(C')--(M);
%	\foreach \x/ \goc in {A'/180,B'/180,C'/0,D'/0} 
%			\fill (\x) circle (1pt)	
%			($(\x)+(\goc:3mm)$) node {$\x$};
\end{tikzpicture}
\end{center}
\subsection{Một số bài toán}

\chude{XÁC ĐỊNH CÁC YẾU TỐ CƠ BẢN LIÊN QUAN ĐẾN ĐƯỜNG THẲNG}
 
\begin{dang}{XÁC ĐỊNH VECTƠ CHỈ PHƯƠNG CỦA ĐƯỜNG THẲNG, XÁC ĐỊNH ĐIỂM THUỘC VÀ KHÔNG THUỘC ĐƯỜNG THẲNG}
\end{dang}
 

 \subsubsection{Vectơ chỉ phương của đường thẳng}

\begin{itemize}
	\item  Vectơ chỉ phương $\overrightarrow{u}$ của đường thẳng $\Delta $ là vectơ có giá song song hoặc trùng với đường thẳng $\Delta $.\\
Nếu $\Delta $ có một vectơ chỉ phương là $\overrightarrow{u}$ thì $k.\overrightarrow{u}$ cũng là một vectơ chỉ phương của $\Delta $.
	\item    Nếu có hai vectơ $\overrightarrow{n}_{1} $ và $\overrightarrow{n}_{2} $ cùng vuông góc với $\Delta $ thì $\Delta $ có một vectơ chỉ phương là $\overrightarrow{u}=[\overrightarrow{n}_{1} ,\overrightarrow{n}_{2} ].$
   \item Phương trình đường thẳng \(\Delta\) dạng: \(\left\{\begin{array}{l} x = x_0 + at \\ y = y_0 + bt \\ z = z_0 + ct \end{array}\right. \; (t \in \mathbb{R})\) thì có vectơ chỉ phương là \(\overrightarrow{u} = (a; b; c)\).
    \item Phương trình đường thẳng \(\Delta\) dạng: \(\dfrac{x - x_0}{a} = \dfrac{y - y_0}{b} = \dfrac{z - z_0}{c} \; (a \neq 0, b \neq 0, c \neq 0)\) thì có vectơ chỉ phương là \(\overrightarrow{u} = (a; b; c)\).
\end{itemize}

\begin{note} Chú ý:
\begin{itemize}
	\item  Trục $Ox$ có vectơ chỉ phương là $\overrightarrow{i}=(1;0;0)$.
	\item  Trục $Oy$ có vectơ chỉ phương là $\overrightarrow{j}=(0;1;0)$.
	\item  Trục $Oz$ có vectơ chỉ phương là $\overrightarrow{k}=(0;0;1)$.
\end{itemize}
 \end{note}

\subsubsection{Điểm thuộc và không thuộc đường thẳng}

\begin{itemize}
	\item  Cho điểm $M\left(x_{M} ; y_{M} ; z_{M} \right)$ và đường thẳng $\Delta $ có phương trình $$\dfrac{x-x_{0} }{a} =\dfrac{y-y_{0} }{b} =\dfrac{z-z_{0} }{c} .$$ Khi đó
	\begin{align*}
	M\in \Delta & \Leftrightarrow \dfrac{x_{M} -x_{0} }{a} =\dfrac{x_{M} -y_{0} }{b} =\dfrac{x_{M} -z_{0} }{c} ;\\
	 M\notin \Delta &\Leftrightarrow \hoac{ {\dfrac{x_{M} -x_{0} }{a} \ne \dfrac{x_{M} -y_{0} }{b} } \\ {\dfrac{x_{M} -y_{0} }{b} \ne \dfrac{x_{M} -z_{0} }{c} .} }
	\end{align*}
	\item  Cho điểm $M\left(x_{M} \, ;\, y_{M} \, ;\, z_{M} \right)$ và đường thẳng $\Delta $ có phương trình $$\heva{{x=x_{0} +at} \\ {y=y_{0} +bt} \\ {z=z_{0} +ct.} } $$ 
	Khi đó
\[M\in \Delta \Leftrightarrow t=\dfrac{x_{M} -x_{0} }{a} =\dfrac{x_{M} -y_{0} }{b} =\dfrac{x_{M} -z_{0} }{c} ;      M\notin \Delta \Leftrightarrow \left[\begin{array}{l} {t=\dfrac{x_{M} -x_{0} }{a} \ne \dfrac{x_{M} -y_{0} }{b} } \\ {t=\dfrac{x_{M} -y_{0} }{b} \ne \dfrac{x_{M} -z_{0} }{c} .} \end{array}\right. \]
\end{itemize}
\TN
\Opensolutionfile{ans}[ans/ans-2C5B2CD1-D1]
%%%==============Cau_EX1==============%%%
\begin{ex}%[2H5N2-2]
	Trong không gian $Oxyz$, cho đường thẳng $d$: $\left\{\begin{array}{c} {x=2+t} \\ {y=1-2t} \\ {z=-1+3t} \end{array}\right.$. Vectơ nào dưới đây là một vectơ chỉ phương của $d$?
	\choice
		{$\overrightarrow{u}_1=(2;1;-1)$}
		{$\overrightarrow{u}_2=(1;2;3)$}
		{\True $\overrightarrow{u}_3=(1;-2;3)$}
		{$\overrightarrow{u}_4=(2;1;1)$}
	\loigiai{
		Từ phương trình đường thẳng $d$ ta thấy vectơ $\overrightarrow{u}_3=(1;-2;3)$ là một véctơ chỉ phương của $d$.
		}
\end{ex}
%%%==============HetCau_EX1==============%%%

%%%==============Cau_EX2==============%%%
\begin{ex}%[2H5N2-2]
	Trong không gian $Oxyz$, cho đường thẳng $d:\dfrac{x-3}{2}=\dfrac{y-4}{-5}=\dfrac{z+1}{3}$. Vectơ nào dưới đây là một vectơ chỉ phương của $d$?
	\choice
		{$\overrightarrow{u}_2=\left(2;4;-1\right)$}
		{\True $\overrightarrow{u}_1=\left(2;-5;3\right)$}
		{$\overrightarrow{u}_3=\left(2;5;3\right)$}
		{$\overrightarrow{u}_4=\left(3;4;1\right)$}
	\loigiai{
		Đường thẳng $d:\dfrac{x-3}{2}=\dfrac{y-4}{-5}=\dfrac{z+1}{3}$ có một vectơ chỉ phương là $\overrightarrow{u}_1\left(2;-5;3\right)$.
		}
\end{ex}
%%%==============HetCau_EX2==============%%%

%%%==============Cau_EX3==============%%%
\begin{ex}%[2H5N2-2]
	Trong không gian $Oxyz$, đường thẳng $d:\dfrac{x+3}{1}=\dfrac{y-1}{-1}=\dfrac{z-5}{2}$ có một vectơ chỉ phương là
	\choice
		{$\overrightarrow{u}_1=\left(3;-1; 5\right)$}
		{\True $\overrightarrow{u}_4=\left(-1; 1;-2\right)$}
		{$\overrightarrow{u}_2=\left(-3; 1; 5\right)$}
		{$\overrightarrow{u}_1=\left(1;-1;-2\right)$}
	\loigiai{
		Đường thẳng $\left(P\right)$ có một vectơ chỉ phương là $\overrightarrow{u}_4=\left(1;-1; 2\right)=-1\left(-1; 1;-2\right)\Rightarrow \overrightarrow{u}_4=\left(-1; 1;-2\right)$.
		}
\end{ex}
%%%==============HetCau_EX3==============%%%

%%%==============Cau_EX4==============%%%
\begin{ex}%[2H5N2-2]
	Trong không gian với hệ tọa độ $Oxyz$, cho đường thẳng $d:\dfrac{x}{-1}=\dfrac{y-4}{2}=\dfrac{z-3}{3}$. Hỏi trong các vectơ sau, đâu không phải là vectơ chỉ phương của $d$?
	\choice
		{$\overrightarrow{u}_1=\left(-1;2;3\right)$}
		{$\overrightarrow{u}_2=\left(3;-6;-9\right)$}
		{$\overrightarrow{u}_3=\left(1;-2;-3\right)$}
		{\True $\overrightarrow{u}_4=\left(-2;4;3\right)$}
	\loigiai{
		Ta có một vectơ chỉ phương của $d$ là $\overrightarrow{u}_1=\left(-1;2;3\right)$.\\
		$\overrightarrow{u}_2=-3\overrightarrow{u}_1$, $\overrightarrow{u}_3=-\overrightarrow{u}_1$ $\Rightarrow$ các vectơ $\overrightarrow{u}_2,\overrightarrow{u}_3$ cũng là vectơ chỉ phương của $d$.\\
		Không tồn tại số $k$ để $\overrightarrow{u}_4=k\overrightarrow{.u_1}$ nên $\overrightarrow{u}_4=\left(-2;4;3\right)$ không phải là vectơ chỉ phương của $d$.
		}
\end{ex}
%%%==============HetCau_EX4==============%%%
 

 

