\chude{VỊ TRÍ TƯƠNG ĐỐI CỦA ĐƯỜNG THẲNG VỚI MẶT PHẲNG}
\begin{tomtat}
{\bf Vị trí tương đối giữa đường thẳng $d$ và mặt phẳng $(P)$}\\
Cho đường thẳng $d:\heva{&x=x_0+a_1t\\&y=y_0+a_2t\\&z=z_0+a_3t}$ và mặt phẳng $(\alpha):Ax+By+Cz+D=0$.\\
Xét hệ phương trình: $\heva{&x=x_0+a_1t& (1) \\& y=y_0+a_2t & (2) \\& z=z_0+a_3t & (3) \\ &Ax+By+Cz+D=0 & (4)} $ $(*)$
\begin{center}
	\begin{tikzpicture}[line cap=round,line join=round,>=stealth,scale=0.8]
	\path
	(0,0) coordinate (A)
	(6,0) coordinate (B)
	(5,-2.5) coordinate (C)
	(-1,-2.5) coordinate (D)
	
	(4.5,2) coordinate (E)
	(4.5,-1) coordinate (F)
	(5,-1) coordinate (G)
	
	(2.5,1.5) coordinate (E1)
	(2.5,-1.25) coordinate (F1)
	(2,-1.25) coordinate (G1)	
	
	(-0.8,-2.2)node[right]{$P$}
	(2.2,1.3)node[left]{$\overrightarrow{u_d}$}
	(3.1,1.8)node[right]{$d$}
	(4.6,1.8)node[right]{$\overrightarrow{n_P}$}
	;
	\draw (A)--(B)--(C)--(D)--(A)
	(3,-3.5)--(3,-2.5) (3,-1)--(3,2)
	;
	\draw[dashed] (3,-2.5)--(3,-1);
	\draw[->] (2.5,-1.25)--(2.5,1.5);
	\draw[->] (4.5,-1)--(4.5,2);
	\pic[draw,angle radius=8mm,angle eccentricity=3.5] {angle = C--D--A};
	\pic[draw,angle radius=2mm,angle eccentricity=1.5] {right angle = E--F--G};
	\pic[draw,angle radius=2mm,angle eccentricity=1.5] {right angle = G1--F1--E1};
	\end{tikzpicture}
\end{center}
\begin{center}
	\begin{tikzpicture}[line cap=round,line join=round,>=stealth,scale=0.8]
	\path
	(0,0) coordinate (A)
	(6,0) coordinate (B)
	(5,-2.5) coordinate (C)
	(-1,-2.5) coordinate (D)
	
	(3,2) coordinate (E)
	(3,-1) coordinate (F)
	(4,-1) coordinate (G)
	
	(3,2) coordinate (E1)
	(3,1.4) coordinate (F1)
	(4,1.4) coordinate (G1)
	
	(-0.8,-2.2)node[right]{$P$}
	(5,1.4)node[above]{$\overrightarrow{u_d}$}
	(6,1.15)node[above]{$d$}
	(3.1,2.2)node[right]{$\overrightarrow{n_P}$}
	;
	\draw (A)--(B)--(C)--(D)--(A)
	(-1,1.15)--(6,1.15)
	;
	\draw[->] (1.5,1.4)--(5,1.4);
	\draw[->] (3,-1)--(3,2.5);
	\pic[draw,angle radius=8mm,angle eccentricity=3.5] {angle = C--D--A};
	\pic[draw,angle radius=2mm,angle eccentricity=1.5] {right angle = E--F--G};
	\pic[draw,angle radius=2mm,angle eccentricity=1.5] {right angle = E1--F1--G1};
	\end{tikzpicture}
\end{center}
$\bullet$ Nếu $(*)$ có nghiệm duy nhất $\Leftrightarrow d$ cắt $(\alpha)$.\\
$\bullet$ Nếu $(*)$ có vô nghiệm $\Leftrightarrow d\parallel (\alpha)$.\\
$\bullet$ Nếu $(*)$ vô số nghiệm $\Leftrightarrow d\subset (\alpha)$.	
\end{tomtat}

\begin{ex}%[2H5H2-5]
Trong KG $Oxyz$, cho đường thẳng $\Delta:\dfrac{x-2}{-3}=\dfrac{y}{1}=\dfrac{z+1}{2}$. Gọi $M$ là giao điểm của $\Delta$ với mặt phẳng $(P): x+2y-3z+2=0$. Tọa độ điểm $M$ là
\choice
{$M(2;0;-1)$}
{$M(5;-1;-3)$}
{$M(1;0;1)$}
{\True $M(-1;1;1)$}
\loigiai{
Tọa độ của điểm $M$ là nghiệm của hệ $$\heva{&\frac{x-2}{-3}=\frac{y}{1}\\&\frac{y}{1}=\frac{z+1}{2}\\& x+2y-3z+2=0}\Leftrightarrow\heva{&x+3y=2 \\& 2y-z=1 \\& x+2y-3z=-2}\Leftrightarrow\heva{&x=-1\\&y=1\\& z=1.}$$
Vậy $M(-1; 1; 1)$.}
\end{ex}

\begin{ex}%[2H5H2-5]
Trong KG $Oxyz$, giao điểm của mặt phẳng $(P):3x+5y-z-2=0$ và đường thẳng $\Delta:\dfrac{x-12}{4}=\dfrac{y-9}{3}=\dfrac{z-1}{1}$ là điểm $M\left(x_0;y_0;z_0\right)$. Giá trị tổng $x_0+y_0+z_0$ bằng
\choice
{$1$}
{$2$}
{$5$}
{\True $-2$}
\loigiai{
Ta có $M\in\Delta\Rightarrow M(12+4t;9+3t;1+t)$.\\
$M\in (P)\Leftrightarrow 3(12+4t)+5(9+3t)-(1+t)-2=0\Leftrightarrow t=-3$.\\
$M(0;0;-2)\Rightarrow x_0+y_0+z_0=-2$.}
\end{ex}

\begin{ex}%[2H5H2-5]
Trong KG $Oxyz$, cho $3$ điểm $A(1;0;0)$, $B(0;2;0)$, $C(0;0;3)$ và $d\colon\heva{&x=-t\\&y=2+t\\&z=3+t}$. Gọi $M(a;b;c)$ là tọa độ giao điểm của đường thẳng $d$ và mặt phẳng $(ABC)$. Tổng $S=a+b+c$ là
\choice
{$-7$}
{\True $11$}
{$5$}
{$6$}
\loigiai{
Mặt phẳng $(ABC)$ qua các điểm $A(1;0;0)$, $B(0;2;0)$, $C(0;0;3)$ nằm trên các trục $Ox$, $Oy$, $Oz$ có phương trình là $\dfrac{x}{1}+\dfrac{y}{2}+\dfrac{z}{3}=1$.\\
Điểm $M(a;b;c)$ là tọa độ giao điểm của của $d$ và mặt phẳng.\\
Suy ra $\dfrac{-t}{1}+\dfrac{2+t}{2}+\dfrac{3+t}{3}=1\Leftrightarrow t=6$ suy ra $\heva{&a=-6\\& b=8\\&c=9.}$\\
Vậy $S=-6+8+9=11$.}
\end{ex}

\begin{ex}%[2H5H2-5]
Trong KG $Oxyz$, cho đường thẳng $d: \dfrac{x+1}{1}=\dfrac{y}{-3}=\dfrac{z-5}{-1}$ và mặt phẳng $(P): 3x-3y+2z+6=0$. Mệnh đề nào dưới đây đúng?
\choice
{\True $d$ cắt và không vuông góc với $(P)$}
{$d$ vuông góc với $(P)$}
{$d$ song song với $(P)$}
{$d$ nằm trong $(P)$}
\loigiai{
Đường thẳng $d$ có véc-tơ chỉ phương $\overrightarrow{u}=(1;-3;-1)$.\\
Mặt phẳng $(P)$ có véc-tơ pháp tuyến  $\overrightarrow{n}=(3;-3;2)$.\\
Ta có $\overrightarrow{u}\cdot\overrightarrow{n}=3+9-2=10\neq 0$ nên loại trường hợp $d\parallel (P)$ và $d\subset (P)$.\\
Lại có $\overrightarrow{u}$ và $\overrightarrow{n}$ không cùng phương nên loại trường hợp $d\perp (P)$.\\
Vậy $d$ cắt và không vuông góc với $(P)$.}
\end{ex}

\begin{ex}%[2H5H2-5]
Trong KG $Oxyz$, cho mặt phẳng $(P): 3x+5y-z-2=0$ và đường thẳng $d: \dfrac{x-12}{4}=\dfrac{y-9}{3}=\dfrac{z-1}{1}$. Trong các mệnh đề sau, mệnh đề nào đúng?
\choice
{$d\subset (Q)$}
{$d\parallel (Q)$}
{\True $d$ cắt $(Q)$}
{$d\perp (Q)$}
\loigiai{
  $(P): 3x+5y-z-2=0$ có véc-tơ pháp tuyến $\overrightarrow{n}=(3;5;-1)$.  \\
  $d: \dfrac{x-12}{4}=\dfrac{y-9}{3}=\dfrac{z-1}{1}$ có véc-tơ chỉ phương $\overrightarrow{u}=(4;3;1)$.  \\
  $\overrightarrow{n}\cdot\overrightarrow{u}=26\neq 0$ nên $d$ không song song với $(P)$ và $d\not\subset (P)$.\\
  $\left[\overrightarrow{n},\overrightarrow{u}\right]\neq 0$ suy ra $d$ không vuông góc $(P)$.\\
  Vậy $d$ cắt $(P)$.}
\end{ex}

\begin{ex}%[2H5H2-5]
Trong KG $Oxyz$, cho mặt phẳng $(P): 3x-3y+2z-5=0$ và đường thẳng $d:\heva{&x=-1+2t\\&y=3+4t\\&z=3t}$. Trong các mệnh đề sau, mệnh đề nào đúng?
\choice
{\True $d\parallel (P)$}
{$d\subset (P)$}
{$d$ cắt $(P)$}
{$d\perp (P)$}
\loigiai{
$(P): 3x-3y+2z-5=0$ có véc-tơ pháp tuyến là $\overrightarrow{n}=(3;-3;2)$.\\
$d$ có véc-tơ chỉ phương là $\overrightarrow{u}=(2;4;3)$.\\
Ta có $\heva{&\overrightarrow{n}\cdot\overrightarrow{u}=0\\&A(-1;3;3)\in d\\&A\not\in (P)}\Leftrightarrow d\parallel (P)$.}
\end{ex}

\begin{ex}%[2H5H2-5]
Trong KG $Oxyz$, cho mặt phẳng $(P):x+y+z-4=0$ và đường thẳng $d:\heva{&x=1+t\\&y=1+2t\\&z=2-3t}$. Số giao điểm của đường thẳng $d$ và mặt phẳng $(P)$ là
\choice
{\True Vô số}
{$1$}
{Không có}
{$2$}
\loigiai{
$(P)$ có véc-tơ pháp tuyến $\overrightarrow{n}=(1;1;1)$.\\
$d$ có véc-tơ chỉ phương là $\overrightarrow{u}=(1;2;-3)$.\\
Ta có $\heva{&\overrightarrow{n}\cdot\overrightarrow{u}=0\\&A(1;1;2)\in d\\&A\in (P)}\Leftrightarrow d\subset (P)$.\\
Vậy $d$ và $(P)$ có vô số giao điểm.}
\end{ex}

\begin{ex}%[2H5H2-5]
Trong KG $Oxyz$, tọa độ giao điểm $M$ của đường thẳng $d:\dfrac{x-12}{4}=\dfrac{y-9}{3}=\dfrac{z-1}{1}$ và mặt phẳng $(P):3x+5y-z-2=0$ là
\choice
{$M(0;2;3)$}
{\True $M(0;0;-2)$}
{$M(0;0;2)$}
{$M(0;-2;-3)$}
\loigiai{
Giải hệ $\heva{&x=12+4t\\&y=9+3t\\&z=1+t\\&3x+5y-z-2=0}\Leftrightarrow\heva{&x=0\\&y=0\\&z=-2\\&t=-3.}$\\
Vậy $M(0;0;-2)$.}
\end{ex}

\begin{ex}%[2H5H2-5]
Giao điểm của mặt phẳng $(P):x+y-z-2=0$ và đường thẳng $d:\heva{&x=2+t\\& y=-t\\& z=3+3t}$ là
\choice
{\True $(1;1;0)$}
{$(0;2;4)$}
{$(0;4;2)$}
{$(2;0;3)$}
\loigiai{
Gọi $A(x;y;z)$ là giao điểm của đường thẳng $d$ và mặt phẳng $(P)$.\\
Ta có $2+t-t-(3+3t)-2=0\Leftrightarrow -3t-3=0\Leftrightarrow t=-1$.\\
$\Rightarrow\heva{&x=1\\& y=1\\&z=0}\Rightarrow A(1;1;0)$.}
\end{ex}

\begin{ex}%[2H5H2-5]
Trong không gian$ Oxyz$, cho đường thẳng $d:\heva{&x=1+2t\\&y=3-t\\&z=1-t}, t\in\mathbb{R}$ và mặt phẳng $(P):x+2y-3z+2=0$. Tìm tọa độ của điểm $A$ là giao điểm của đường thẳng $d$ và mặt phẳng $(P)$.
\choice
{$A(3;5;3)$}
{$A(1;3;1)$}
{\True $A(-3;5;3)$}
{$A(1;2;-3)$}
\loigiai{
Vì $A$ là giao điểm của đường thẳng $d$ và mặt phẳng $(P)$ nên
\begin{itemize}
	\item $A\in d\Rightarrow A(1+2t;3-t;1-t)$.
	\item $A\in (P)\Rightarrow (1+2t)+2(3-t)-3(1-t)+2=0\Rightarrow t=-2$.
\end{itemize}
Vậy tọa độ điểm$ A(-3;5;3)$.}
\end{ex}

\begin{ex}%[2H5H2-5]
Trong KG $Oxyz$, giao điểm của mặt phẳng $(P):3x+5y-z-2=0$ và đường thẳng $\Delta:\dfrac{x-12}{4}=\dfrac{y-9}{3}=\dfrac{z-1}{1}$ là điểm $M\left(x_0;y_0;z_0\right)$. Giá trị tổng $x_0+y_0+z_0$ bằng
\choice
{$1$}
{$2$}
{$5$}
{\True $-2$}
\loigiai{
$M\in\Delta\Rightarrow M(12+4t;9+3t;1+t)$.\\
$M\in (P)\Leftrightarrow 3(12+4t)+5(9+3t)-(1+t)-2=0\Leftrightarrow t=-3$.\\
$M(0;0;-2)\Rightarrow x_0+y_0+z_0=-2$.}
\end{ex}

\begin{ex}%[2H5H2-5]
Trong KG $Oxyz$, cho đường thẳng $d:\heva{&x=4-2t\\&y=-3+t\\&z=1-t}$, giao điểm của $d$ với mặt phẳng $(Oxy)$ có tọa độ là
\choice
{$(4;-3;0)$}
{\True $(2;-2;0)$}
{$(0;-1;-1)$}
{$(-2;0;-2)$}
\loigiai{
Mặt phẳng $(Oxy)$ có phương trình $z=0$.\\
Gọi $M(4-2m;-3+m;1-m)$ là giao điểm của $d$ với mặt phẳng $(Oxy)$ thì ta có
$$1-m=0\Leftrightarrow m=1.$$
Vậy $M(2;-2;0)$.}
\end{ex}

\begin{ex}%[2H5H2-5]
Trong không gian với hệ toạ độ $Oxyz$, cho $3$ điểm $A(1;0;0)$, $B(0;2;0)$, $C(0;0;3)$ và đường thẳng $d:\heva{&x=-t\\&y=2+t\\&z=3+t}$. Gọi $M(a;b;c)$ là toạ độ giao điểm của đường thẳng $d$ với mặt phẳng $(ABC)$. Tính tổng $S=a+b-c$.
\choice
{$6$}
{$5$}
{\True $-7$}
{$11$}
\loigiai{
Phương trình mặt phẳng $(ABC)$ có dạng $\dfrac{x}{1}+\dfrac{y}{2}+\dfrac{z}{3}=1\Leftrightarrow 6x+3y+2z-6=0$.\\
Điểm $M\in d\Rightarrow M(-t;2+t;3+t)$. Lại vì $M=d\cap (ABC)$ nên ta có
$$6(-t)+3(2+t)+2(3+t)-6=0\Leftrightarrow -t=-6\Leftrightarrow t=6\Rightarrow M(-6;8;9).$$
Vậy ta có $S=a+b-c=-6+8-9=-7$.}
\end{ex}

\begin{ex}%[2H5H2-5]
Trong không gian với hệ trục tọa độ $Oxyz$, hình chiếu vuông góc của điểm $M(-4;5;2)$ lên mặt phẳng $(P):y+1=0$ là điểm có tọa độ
\choice
{\True $(-4;-1;2)$}
{$(-4;1;2)$}
{$(0;-1;0)$}
{$(0;1;0)$}
\loigiai{
Gọi $H$ là hình chiếu vuông góc của $M$ lên $(P)\Rightarrow MH:\heva{&x=-4\\&y=5+t\\&z=2.}$\\
$H\in MH\Rightarrow H(-4;5+t;2)$.\\
$H\in (P)\Leftrightarrow 5+t+1=0\Leftrightarrow t=-6\Rightarrow H(-4;-1;2)$.}
\end{ex}

\begin{ex}%[2H5H2-5]
Trong không gian với hệ trục tọa độ $Oxyz$, cho đường thẳng $d:\dfrac{x-12}{4}=\dfrac{y-9}{3}=\dfrac{z-1}{1}$ và mặt phẳng $(P):3x+5y-z-2=0$. Tìm tọa độ giao điểm của $d$ và $(P)$.\\
\choice
{$(1;0;1)$}
{\True $(0;0;-2)$}
{$(1;1;6)$}
{$(12;9;1)$}
\loigiai{
Ta có $d:\dfrac{x-12}{4}=\dfrac{y-9}{3}=\dfrac{z-1}{1}\Rightarrow d:\heva{&x=12+4t\\&y=9+3t\\&z=1+t}, t\in\mathbb{R}$.\\
Thay $x=12+4t$, $y=9+3t$, $z=1+t$ vào $(P):3x+5y-z-2=0$, ta được\\
$$3(12+4t)+5(9+3t)-(1+t)-2=0\Leftrightarrow t=-3.$$
Với $t=-3\Rightarrow x=0$, $y=0$, $z=-2$.\\
Vậy tọa độ giao điểm của $d$ và $(P)$ là $(0;0;-2)$.}
\end{ex}

\begin{ex}%[2H5V2-5]
Trong KG $Oxyz$, cho đường thẳng $\Delta:\dfrac{x}{-2}=\dfrac{y-2}{1}=\dfrac{z+1}{3}$ và mặt phẳng $(P):11x+my+nz-16=0$. Biết $\Delta\subset (P)$, tính giá trị của $T=m+n$.
\choice
{$T=2$}
{$T=-2$}
{\True $T=14$}
{$T=-14$}
\loigiai{
	\begin{itemize}
		\item Cách $1$: Lấy $\heva{&A(0;2;-1)\in\Delta \\& B(-2;3;2)\in \Delta.}$\\
		Mà $\Delta\subset (P)\Rightarrow \heva{&A\in (P)\\& B\in (P)}\Leftrightarrow \heva{&2m-n-16=0\\&11\cdot (-2)+3m+2n-16=0} \Leftrightarrow \heva{&m=10\\&n=4.}$\\
		$\Rightarrow T=m+n=14$.
		\item Cách $2$: Đường thẳng $\Delta$ đi qua $A(0;2;-1)$ có véc-tơ chỉ phương $\overrightarrow{u}=(-2;1;3)$.\\
		Mặt phẳng $(P)$ có véc-tơ pháp tuyến $\overrightarrow{n}=(11;m;n)$.\\
		$\Delta\subset (P)\Rightarrow \heva{&A\in (P) \\&\overrightarrow{n}\cdot\overrightarrow{u}=0}\Leftrightarrow \heva{&2m-n-16=0\\&-22+m+3n=0}\Leftrightarrow \heva{&m=10\\&n=4.}$\\
		$\Rightarrow T=m+n=14$.
	\end{itemize}
}
\end{ex}

\begin{ex}%[2H5V2-5]
Trong không gian tọa độ $Oxyz$, cho đường thẳng $d:\dfrac{x-1}{1}=\dfrac{y-2}{3}=\dfrac{z-9}{-1}$ và mặt phẳng $(\alpha)$ có phương trình $m^2x-my-2z+19=0$ với $m$ là tham số. Tập hợp các giá trị $m$ thỏa mãn $d \parallel (\alpha)$ là
\choice
{$\left\{1\right\}$}
{$\emptyset$}
{$\left\{1;2\right\}$}
{\True $\left\{2\right\}$}
\loigiai{
Đường thẳng $d$ có vectơ chỉ phương là $\overrightarrow{u}=(1;3;-1)$.\\
Mặt phẳng $(\alpha)$ có vectơ pháp tuyến là $\overrightarrow{n}=\left(m^2;-m;-2\right)$.\\
Để $d\parallel (\alpha)$ thì $\heva{&\overrightarrow{u}\cdot\overrightarrow{n}=0\\& M(1;2;9)\notin (\alpha)}\Leftrightarrow \heva{&m^2-3m+2=0\\&m^2-2m-18+19\neq 0}\Leftrightarrow \heva{&\hoac{&m=1\\&m=2}\\& m\ne 1} \Leftrightarrow m=2$.}
\end{ex}

\begin{ex}%[2H5V2-5]
Trong không gian với hệ trục toạ độ $Oxyz$, tìm tất cả các giá trị của tham số $m$ để đường thẳng $d:\dfrac{x-1}{1}=\dfrac{y+1}{-1}=\dfrac{z-2}{1}$ song song với mặt phẳng $(P):2x+y-m^2z+m=0$
\choice
{$m=1$}
{$m\in\emptyset$}
{$m\in\left\{-1;1\right\}$}
{\True $m=-1$}
\loigiai{
Một véc-tơ chỉ phương của $d$ là $\overrightarrow{u}=(1;-1;1)$; $A(1;-1;2)\in d$.\\
Một véc-tơ pháp tuyến của $(P)$ là $\overrightarrow{n}=\left(2;1;-m^2\right)$.
\begin{eqnarray*}
	&d\parallel (P)&\Leftrightarrow\heva{&\overrightarrow{u}\perp \overrightarrow{n}\\& A\notin (P)}\Leftrightarrow \heva{&1\cdot 2-1\cdot 1-1\cdot m^2=0\\& 2\cdot 1-1-2m^2+m\neq 0}\\
	&&\Leftrightarrow \heva{&1-m^2=0\\&1-2m^2+m\neq 0}\Leftrightarrow \heva{&m=\pm 1\\&1-2m^2+m\neq 0} \Leftrightarrow m=-1.
\end{eqnarray*}
}
\end{ex}

\begin{ex}%[2H5V2-5]
Trong KG $Oxyz$, cho mặt phẳng $(P):x-2y+3z+-4=0$ và đường thẳng $d: \dfrac{x-m}{1}=\dfrac{y+2m}{3}=\dfrac{z}{2}$. Với giá trị nào của $m$ thì giao điểm của đường thẳng $d$ và mặt phẳng $(P)$ thuộc mặt phẳng $(Oyz)$.
\choice
{$m=\dfrac{4}{5}$}
{$m=-1$}
{\True $m=1$}
{$m=\dfrac{12}{17}$}
\loigiai{
	Ta có 
$d\cap (P)=A\in (Oyz)\Rightarrow A\left(0;\dfrac{3}{2} a-2;a\right)$.\\
$A\in d\Rightarrow 0-m=\dfrac{\frac{3}{2}a-2+2m}{3} =\dfrac{a}{2}\Rightarrow \heva{&a=-2m\\&\frac{3}{2}a-2+2m=-3m} \Rightarrow\heva{&a=-2\\& m=1.}$}
\end{ex}

\begin{ex}%[2H5V2-5]
Trong KG $Oxyz$, cho mặt phẳng $(P):2x+my-3z+m-2=0$ và đường thẳng $d:\heva{&x=2+4t\\&y=1-t\\&z=1+3t}$. Với giá trị nào của $m$ thì $d$ cắt $(P)$
\choice
{$m\neq\dfrac{1}{2}$}
{$m=-1$}
{$m=\dfrac{1}{2}$}
{\True $m\neq -1$}
\loigiai{
$(P)$ có véc-tơ pháp tuyến $\overrightarrow{n}=(2;m;-3)$.\\
$d$ có véc-tơ chỉ phương là $\overrightarrow{u}=(4;-1;3)$.\\
Ta có $d$ cắt $(P)\Leftrightarrow\overrightarrow{n}\cdot\overrightarrow{u}\neq 0\Leftrightarrow 2\cdot 4+m\cdot (-1)+(-3)\cdot (-3)\neq 0\Leftrightarrow m\neq -1$.
}
\end{ex}

\begin{ex}%[2H5V2-5]
Trong không gian $(P)$, cho đường thẳng $d:\heva{&x=2-t\\&y=-3+t\\&z=1+t}$ và mặt phẳng $(P): m^2x-2my+(6-3m)z-5=0$. Tìm $m$ để $d\parallel (P)$.
\choice
{\True $\hoac{&m=1\\&m=-6}$}
{$\hoac{&m=-1\\&m=6}$}
{$\hoac{&m=-1\\&m=-6}$}
{$\emptyset$}
\loigiai{
Ta có $d$ đi qua $M(2;-3;1)$và có véc-tơ chỉ phương $\overrightarrow{u}=(-1;1;1)$.\\
Và $(P)$ có véc-tơ pháp tuyến $\overrightarrow{n}=(m^2;-2m;6-3m)$.\\
Để $d$ song song với $(P)$ thì
\begin{eqnarray*}
	&\heva{&\overrightarrow{u}\perp\overrightarrow{n}\\&M\not\in (P)}\Leftrightarrow\heva{&\overrightarrow{u}\cdot\overrightarrow{n}=0\\&M\not\in (P)}&\Leftrightarrow\heva{& -1\cdot m^2+1\cdot (-2m)+1\cdot (6-3m)=0\\&2m^2-2\cdot (-3)m+6-3m-5\neq 0}\\
	&&\Leftrightarrow\heva{&-m^2-5m+6=0\\&2m^2+3m+1\neq 0}\Leftrightarrow\hoac{&m=1\\&m=-6.}
\end{eqnarray*}
}
\end{ex}

\begin{ex}%[2H5V2-5]
Gọi $m,n$ là hai giá trị thực thỏa mãn giao tuyến của hai mặt phẳng $\left(P_{m}\right):mx+2y+nz+1=0$ và $\left(Q_{m}\right):x-my+nz+2=0$ vuông góc với mặt phẳng $\left(\alpha\right):4x-y-6z+3=0$.
\choice
{$m+n=0$}
{$m+n=2$}
{$m+n=1$}
{\True $m+n=3$}
\loigiai{
Ta có
$\left(P_{m}\right):mx+2y+nz+1=0$ có véc-tơ pháp tuyến $\overrightarrow{n}_P=(m;2;n)$.\\
$\left(Q_{m}\right):x-my+nz+2=0$ có véc-tơ pháp tuyến $\overrightarrow{n}_Q=(1;-m;n)$.\\
$\left(\alpha\right):4x-y-6z+3=0$ có véc-tơ pháp tuyến $\overrightarrow{n}_{\alpha}=(4;-1;-6)$.\\
Do giao tuyến của $\left(P_{m}\right)$ và $\left(Q_{n}\right)$ vuông góc với $\left(\alpha\right)$ nên
$$\heva{&\left(P_{m}\right)\perp\left(\alpha\right)\\& \left(Q_{n}\right)\perp\left(\alpha\right)}\Rightarrow \heva{&\overrightarrow{n}_P\perp\overrightarrow{n}_{\alpha}\\& \overrightarrow{n}_Q\perp\overrightarrow{n}_{\alpha}}\Rightarrow \heva{&4m-2-6n=0\\&4+m-6n=0}\Rightarrow \heva{&4m-6n=2 \\&m-6n=-4}\Rightarrow\heva{&m=2\\&n=1.}$$
Vậy $m+n=3$.}
\end{ex}
