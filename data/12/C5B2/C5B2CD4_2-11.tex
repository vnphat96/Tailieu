%%%% Câu 2
\begin{ex}%[2H5V2-7]%[Dự án 2025-K12-TL-TV]%[Thành Đức Trung]
\immini{
Cho hình chóp $S.ABCD$ có đáy $ABCD$ là hình vuông tâm $I$ có độ dài đường chéo bằng $a\sqrt{2}$ và $SA$ vuông góc với mặt phẳng $\left(ABCD\right)$. Gọi $\alpha$ là góc giữa hai mặt phẳng $\left(SBD\right)$ và $\left(ABCD\right)$. Chọn hệ trục tọa độ $Oxyz$ như hình vẽ. Nếu $\tan\alpha = \sqrt{2}$ thì góc giữa hai mặt phẳng $\left(SAC\right)$ và $\left(SBC\right)$ bằng
\choice
{$30^\circ$}
{\True $60^\circ$}
{$45^\circ$}
{$90^\circ$}
}{
\begin{tikzpicture}[scale=0.7, font=\footnotesize, line join=round, line cap=round, >=stealth]
\tikzset{label style/.style={font=\footnotesize}}
\def\h{4} \def\r{5} \def\x{2.2} \def\y{1.5}
\coordinate[label={below}:$B$] (B) at (-3,-3);
\coordinate[label={above,xshift=2mm}:{$A\equiv O$}] (A) at ($(B)+(\x,\y)$);
\coordinate[label={above right}:$S$] (S) at ($(A)+(0,\h)$);
\coordinate[label={below right}:$D$] (D) at ($(A)+(\r,0)$);
\coordinate[label={below right}:$C$] (C) at ($(B)+(\r,0)$);
\coordinate[label={above}:{$x$}] (x) at ($(A)!1.4!(B)$);
\coordinate[label={below}:{$y$}] (y) at ($(A)!1.3!(D)$);
\coordinate[label={right}:{$z$}] (z) at ($(A)!1.3!(S)$);
\coordinate[label={below}:{$I$}] (I) at ($(A)!.5!(C)$);

\draw (B)--(C)--(D)--(S)--(B) (S)--(C);
\draw[dashed] (B)--(A)--(D)--(B) (S)--(A)--(C);
\draw[->] (B)--(x);
\draw[->] (S)--(z);
\draw[->] (D)--(y);

\foreach \x in{A, B, C, D, S, I}\fill[black](\x)circle(2pt);
\end{tikzpicture}
}
\loigiai
{
Ta có $\left(\left(SBD\right);\left(ABCD\right)\right) = \left(SI,AI\right) = \widehat{SIA}$. \\
Do đó $\tan\alpha = \tan\widehat{SIA} = \dfrac{SA}{AI} \Rightarrow SA=a$.\\
Với hệ trục tọa độ như hình vẽ thì ta có $A\left(0;0;0\right)$, $B(a;0;0)$, $C\left(a;a;0\right)$, $S\left(0;0;a\right)$. \\
Suy ra $\overrightarrow{SA} = \left(0;0;-a\right)$, $\overrightarrow{SC} = \left(a;a;-a\right)$, $\overrightarrow{SB} = \left(a;0;-a\right)$.\\
Mặt phẳng $\left(SAC\right)$ có véc-tơ pháp tuyến $\overrightarrow{n}_1 = \left(-1;1;0\right)$. \\
Mặt phẳng $\left(SBC\right)$ có véc-tơ pháp tuyến $\overrightarrow{n}_2 = \left(1;0;1\right)$.\\
Suy ra
\[\cos\left(\left(SAC\right),\left(SBC\right)\right) = \dfrac{\left|\overrightarrow{n}_1\cdot\overrightarrow{n}_2\right|}{ \left|\overrightarrow{n}_1\right|\cdot\left|\overrightarrow{n}_2\right| } = \dfrac{1}{\sqrt{2}\cdot\sqrt{2}} = \dfrac{1}{2}.\]
Vậy $\left(\left(SAC\right),\left(SBC\right)\right)=60^\circ$.
}
\end{ex}

%%%% Câu 3
\begin{ex}%[2H5V2-7]%[Dự án 2025-K12-TL-TV]%[Thành Đức Trung]
Cho hình chóp $S.ABCD$ có đáy $ABCD$ là hình vuông cạnh $a$, cạnh bên $SA = 2a$ và vuông góc với mặt phẳng đáy. Gọi $M$ là trung điểm cạnh $SD$. Tính $\tan$ của góc tạo bởi hai mặt phẳng $\left(AMC\right)$ và $\left(SBC\right)$.
\choice
{$\dfrac{\sqrt{3}}{2}$}
{$\dfrac{2\sqrt{3}}{2}$}
{$\dfrac{\sqrt{5}}{5}$}
{\True $\dfrac{2\sqrt{5}}{5}$}
\loigiai
{
\begin{center}
\begin{tikzpicture}[scale=0.7, font=\footnotesize, line join=round, line cap=round, >=stealth]
\tikzset{label style/.style={font=\footnotesize}}
\def\h{4} \def\r{5} \def\x{2.2} \def\y{1.5}
\coordinate[label={below}:$B$] (B) at (-3,-3);
\coordinate[label={above left}:{$A$}] (A) at ($(B)+(\x,\y)$);
\coordinate[label={above right}:$S$] (S) at ($(A)+(0,\h)$);
\coordinate[label={above right}:$D$] (D) at ($(A)+(\r,0)$);
\coordinate[label={below right}:$C$] (C) at ($(B)+(\r,0)$);
\coordinate[label={above}:{$x$}] (x) at ($(A)!1.4!(B)$);
\coordinate[label={below}:{$y$}] (y) at ($(A)!1.3!(D)$);
\coordinate[label={right}:{$z$}] (z) at ($(A)!1.3!(S)$);
\coordinate[label={above right}:{$M$}] (M) at ($(S)!.5!(D)$);

\draw (B)--(C)--(D)--(S)--(B) (S)--(C)--(M);
\draw[dashed] (B)--(A)--(D)--(B) (S)--(A)--(C) (A)--(M);
\draw[->] (B)--(x);
\draw[->] (S)--(z);
\draw[->] (D)--(y);

\foreach \x in{A, B, C, D, S, M}\fill[black](\x)circle(2pt);
\end{tikzpicture}
\end{center}
Gắn trục tọa độ như hình vẽ. Không mất tính tổng quát, ta đặt $a=1$.\\
Ta có $A\left(0;0;0\right)$, $B\left(1;0;0\right)$, $D\left(0;1;0\right)$, $C\left(1;1;0\right)$, $S\left(0;0;2\right)$.\\
Do $M$ là trung điểm của $SD$ nên $M\left(0;\dfrac{1}{2};1\right)$.\\
Khi đó 
\begin{itemize}
\item $\overrightarrow{BC} = \left(0;1;0\right)$, $\overrightarrow{SB} = \left(1;0;-2\right)$ $\Rightarrow \left[\overrightarrow{BC},\overrightarrow{SB}\right] = \left(2;0;1\right)$. \\
Một véc-tơ pháp tuyến của $\left(SBC\right)$ là $\overrightarrow{n}_1 = \left(2;0;1\right)$.
\item $\overrightarrow{MA} = \left(0;\dfrac{1}{2};1\right)$, $\overrightarrow{AC} = \left(1;1;0\right)$ $\Rightarrow\left[\overrightarrow{MA},\overrightarrow{AC}\right] = \left(-1;1;-\dfrac{1}{2}\right)$. \\
Một véc-tơ pháp tuyến của $\left(AMC\right)$ là $\overrightarrow{n}_2 = \left(2;-2;1\right)$.
\end{itemize}
Suy ra 
\[\cos\left(\left(SBC\right),\left(AMC\right)\right) = \dfrac{\sqrt{5}}{3}\Rightarrow \tan \left(\left(SBC\right),\left(AMC\right)\right) = \dfrac{2\sqrt{5}}{5}.\]
}
\end{ex}

%%%% Câu 4
\begin{ex}%[2H5V2-7]%[Dự án 2025-K12-TL-TV]%[Thành Đức Trung]
Cho hình chóp $S.ABCD$ có $ABCD$ là hình vuông cạnh $a$, $SA\perp\left(ABCD\right)$ và $SA=a$. Gọi $E$ và $F$ lần lượt là trung điểm của $SB$ và $SD$. Tính cô-sin của góc hợp bởi hai mặt phẳng $\left(AEF\right)$ và $\left(ABC\right)$.
\choice
{$\dfrac{1}{2}$}
{\True $\dfrac{\sqrt{3}}{3}$}
{$\sqrt{3}$}
{$\dfrac{\sqrt{3}}{2}$}
\loigiai
{
\begin{center}
\begin{tikzpicture}[scale=0.7, font=\footnotesize, line join=round, line cap=round, >=stealth]
\tikzset{label style/.style={font=\footnotesize}}
\def\h{4} \def\r{5} \def\x{2.2} \def\y{1.5}
\coordinate[label={below}:$B$] (B) at (-3,-3);
\coordinate[label={below}:{$A$}] (A) at ($(B)+(\x,\y)$);
\coordinate[label={above right}:$S$] (S) at ($(A)+(0,\h)$);
\coordinate[label={above right}:$D$] (D) at ($(A)+(\r,0)$);
\coordinate[label={below right}:$C$] (C) at ($(B)+(\r,0)$);
\coordinate[label={above}:{$x$}] (x) at ($(A)!1.4!(B)$);
\coordinate[label={below}:{$y$}] (y) at ($(A)!1.3!(D)$);
\coordinate[label={right}:{$z$}] (z) at ($(A)!1.3!(S)$);
\coordinate[label={above right}:{$F$}] (F) at ($(S)!.5!(D)$);
\coordinate[label={above left}:{$E$}] (E) at ($(S)!.5!(B)$);
\draw (B)--(C)--(D)--(S)--(B) (S)--(C);
\draw[dashed] (B)--(A)--(D)--(B) (S)--(A)--(C) (A)--(E)--(F)--(A);
\draw[->] (B)--(x);
\draw[->] (S)--(z);
\draw[->] (D)--(y);

\foreach \x in{A, B, C, D, S, E, F}\fill[black](\x)circle(2pt);
\end{tikzpicture}
\end{center}
Gắn trục tọa độ như hình vẽ. Không mất tính tổng quát, ta đặt $a=1$.\\
Ta có $A\left(0;0;0\right)$, $B\left(1;0;0\right)$, $D\left(0;1;0\right)$, $S\left(0;0;1\right)$. Khi đó $E\left(\dfrac{1}{2};0;\dfrac{1}{2}\right)$ và $F\left(0;\dfrac{1}{2};\dfrac{1}{2}\right)$.\\
Ta có $\overrightarrow{A E}=\left(\dfrac{1}{2} ; 0 ; \dfrac{1}{2}\right), \overrightarrow{A F}=\left(0 ; \dfrac{1}{2} ; \dfrac{1}{2}\right)$.\\
Một vec-tơ pháp tuyến của $(A E F)$ là $\overrightarrow{n_1}=\left[\overrightarrow{A B}, \overrightarrow{A F}\right]=\left(\dfrac{-1}{4} ; \dfrac{-1}{4} ; \dfrac{1}{4}\right) =-\dfrac{1}{4}(1 ; 1 ;-1)$.\\
Một vec-tơ pháp tuyến của $(A B C D)$ là  $\overrightarrow{n_2}=\overrightarrow{A S}=(0 ; 0 ; 1)$.\\
Vậy cô-sin góc giữa 2 mặt phẳng $(A E F)$ và $(A B C D)$ là
\[
\cos ((A E F),(A B C D))=\dfrac{\left|\overrightarrow{n}_1 \cdot \overrightarrow{n}_2\right|}{\left|\overrightarrow{n}_1\right| \cdot\left|\overrightarrow{n}_2\right|}=\dfrac{1}{\sqrt{3}}=\dfrac{\sqrt{3}}{3}.
\]
}
\end{ex}

%%%% Câu 5
\begin{ex}%[2H5V2-7]%[Dự án 2025-K12-TL-TV]%[Thành Đức Trung]
Cho hình chóp $O.ABC$ có ba cạnh $OA$, $OB$, $OC$ đôi một vuông góc và $OA = OB = OC = a$. Gọi $M$ là trung điểm cạnh $AB$. Góc tạo bởi hai véc-tơ $\overrightarrow{BC}$ và $\overrightarrow{OM}$ bằng
\choice
{$135^\circ$}
{$150^\circ$}
{\True $120^\circ$}
{$60^\circ$}
\loigiai
{
\begin{center}
\begin{tikzpicture}[scale=0.7, font=\footnotesize, line join=round, line cap=round, >=stealth]
\tikzset{label style/.style={font=\footnotesize}}
\def\h{4} \def\r{5} \def\x{2.2} \def\y{1.5}
\coordinate[label={below}:$A$] (A) at (-3,-3);
\coordinate[label={below}:{$O$}] (O) at ($(A)+(\x,\y)$);
\coordinate[label={above right}:$C$] (C) at ($(O)+(0,\h)$);
\coordinate[label={above right}:$B$] (B) at ($(O)+(\r,0)$);

\coordinate[label={above}:{$x$}] (x) at ($(O)!1.4!(B)$);
\coordinate[label={below}:{$y$}] (y) at ($(O)!1.4!(A)$);
\coordinate[label={right}:{$z$}] (z) at ($(O)!1.4!(C)$);
\coordinate[label={below}:{$M$}] (M) at ($(A)!.5!(B)$);

\draw (A)--(B)--(C)--(A);
\draw[dashed] (C)--(O)--(A) (B)--(O)--(M);
\draw[->] (B)--(x);
\draw[->] (C)--(z);
\draw[->] (A)--(y);

\foreach \x in{A, B, C, O, M}\fill[black](\x)circle(2pt);
\end{tikzpicture}
\end{center}
Gắn trục tọa độ như hình vẽ. \\
Ta có $O(0 ; 0 ; 0)$, $A(0 ; a ; 0)$, $B(a ; 0 ; 0)$, $C(0 ; 0 ; a)$, $M\left(\dfrac{a}{2} ; \dfrac{a}{2} ; 0\right)$. \\
Khi đó ta có $\overrightarrow{B C}=(-a ; 0 ; a)$, $\overrightarrow{O M}=\left(\dfrac{a}{2} ; \dfrac{a}{2} ; 0\right)$. Suy ra
\[
\cos \left(\overrightarrow{B C} ; \overrightarrow{O M}\right)=\dfrac{\overrightarrow{B C} \cdot \overrightarrow{O M}}{B C \cdot O M}=\dfrac{-\frac{a^2}{2}}{a\sqrt{2} \cdot \frac{a \sqrt{2}}{2}}=-\dfrac{1}{2} \Rightarrow\left(\overrightarrow{B C} ; \overrightarrow{O M}\right)=120^{\circ}.
\]
}
\end{ex}

%%%% Câu 6
\begin{ex}%[2H5V2-7]%[Dự án 2025-K12-TL-TV]%[Thành Đức Trung]
Cho hình chóp tứ giác đều $S.ABCD$ có $AB = a$, $SA = a\sqrt{2}$. Gọi $G$ là trọng tâm tam giác $SCD$. Góc giữa đường thẳng $BG$ với đường thẳng $SA$ bằng
\choice
{$\arccos\dfrac{\sqrt{3}}{5}$}
{\True $\arccos\dfrac{\sqrt{5}}{5}$}
{$\arccos\dfrac{\sqrt{5}}{3}$}
{$\arccos\dfrac{\sqrt{15}}{5}$}
\loigiai
{
\begin{center}
\begin{tikzpicture}[scale=0.85, font=\footnotesize, line join=round, line cap=round, >=stealth]
\tikzset{label style/.style={font=\footnotesize}}
\def\h{5} \def\r{5} \def\x{2.2} \def\y{1.5}
\coordinate[label={below}:$B$] (B) at (-3,-3);
\coordinate[label={below}:{$A$}] (A) at ($(B)+(\x,\y)$);
\coordinate[label={above right}:$D$] (D) at ($(A)+(\r,0)$);
\coordinate[label={below right}:$C$] (C) at ($(B)+(\r,0)$);
\coordinate[label={above right}:{$O$}] (O) at ($(A)!.5!(C)$);
\coordinate[label={above}:$S$] (S) at ($(O)+(0,\h)$);
\coordinate[label={above}:{$x$}] (x) at ($(A)!1.4!(B)$);
\coordinate[label={below}:{$y$}] (y) at ($(A)!1.3!(D)$);
\coordinate[label={right}:{$z$}] (z) at ($(A)+(0,\h)$);

\coordinate (M) at ($(C)!.5!(D)$);
\coordinate[label={above right}:{$G$}] (G) at ($(S)!.666!(M)$);
\draw (B)--(C)--(D)--(S)--(B) (M)--(S)--(C);
\draw[dashed] (B)--(A)--(D)--(B)--(G) (O)--(S)--(A)--(C);
\draw[->] (B)--(x);
\draw[->] (A)--(z);
\draw[->] (D)--(y);

\foreach \x in{A, B, C, D, S, O, G}\fill[black](\x)circle(1pt);
\end{tikzpicture}
\end{center}
Gọi $O$ là giao điểm của $AC$ và $BD$. Trong $\triangle SAO$ vuông tại $O$ ta có $SO = \sqrt{SA^2 - OA^2} = \dfrac{a\sqrt{6}}{2}$.\\
Gắn trục tọa độ như hình vẽ. \\
Ta có $
A(0 ; 0 ; 0)$, $B(a ; 0 ; 0)$, $C(a ; a ; 0)$, $D(0 ; a ; 0)$, $O\left(\dfrac{a}{2} ; \dfrac{a}{2} ; 0\right)$, $S\left(\dfrac{a}{2} ; \dfrac{a}{2} ; \dfrac{a \sqrt{6}}{2}\right)$.\\
Vì $G$ là trọng tâm tam giác $S C D$ nên $G\left(\dfrac{a}{2} ; \dfrac{5 a}{6} ; \dfrac{a \sqrt{6}}{6}\right)$.\\
Ta có $\overrightarrow{A S}=\left(\dfrac{a}{2} ; \dfrac{a}{2} ; \dfrac{a \sqrt{6}}{2}\right)=\dfrac{a}{2}(1 ; 1 ; \sqrt{6})$, $ \overrightarrow{B G}=\left(\dfrac{-a}{2} ; \dfrac{5 a}{6} ; \dfrac{a \sqrt{6}}{6}\right)=\dfrac{a}{6}(-3 ; 5 ; \sqrt{6})$.\\
Góc giữa đường thẳng $B G$ với đường thẳng $S A$ bằng
\[
\cos (B G ; S A)=\dfrac{|\overrightarrow{B G} \cdot \overrightarrow{A S}|}{B G \cdot A S}=\dfrac{|-3+5+6|}{\sqrt{40} \cdot \sqrt{8}}=\dfrac{\sqrt{5}}{5} .\]
}
\end{ex}

%%%% Câu 7
\begin{ex}%[2H5C4-1]%[Dự án 2025-K12-TL-TV]%[Thành Đức Trung]
Cho hình hộp đứng $ABCD. A'B'C'D'$ có đáy là hình thoi, tam giác $ABD$ đều. Gọi $M, N$ lần lượt là trung điểm của $BC$ và $C'D'$, biết rằng $MN \perp B'D$. Gọi $\alpha$ là góc tạo bởi đường thẳng $MN$ và mặt đáy $(ABCD)$, khi đó $\cos \alpha$ bằng
\choice
{\True $\cos \alpha=\dfrac{1}{\sqrt{3}}$}
{$\cos \alpha=\dfrac{\sqrt{3}}{2}$}
{$\cos \alpha=\dfrac{1}{\sqrt{10}}$}
{$\cos \alpha=\dfrac{1}{2}$}
\loigiai
{\immini{Chọn $AB=2 \Rightarrow BD=2; AC=2\sqrt{3}$, đặt
$AA'=h$.\\
Chọn hệ trục tọa độ $Oxyz$ như hình vẽ ta có\\ $D(1 ; 0 ; 0), B(-1 ; 0 ; 0), C(0 ; \sqrt{3} ; 0)$,\\ $D'(1 ; 0 ; h)$, $C'(0 ; \sqrt{3} ; h), B'(-1 ; 0 ; h)$.\\
Suy ra
$M\left(-\dfrac{1}{2} ; \dfrac{\sqrt{3}}{2} ; 0\right), N\left(\dfrac{1}{2} ; \dfrac{\sqrt{3}}{2} ; h\right)$,\\ $\overrightarrow{M N}=(1 ; 0 ; h), \overrightarrow{B'D}=(2 ; 0 ;-h)$.}
{\begin{tikzpicture}[line join=round,line cap=round,>=stealth,font=\footnotesize,scale=.7]
\path
(0,0)coordinate(A)
(-130:2)coordinate(B)
(3,0)coordinate(D)
($(A)+(90:2.5)$)coordinate(A')
($(D)-(A)+(B)$)coordinate(C)
($(D)+(90:2.5)$)coordinate(D')
($(B)+(90:2.5)$)coordinate(B')
($(D')-(A')+(B')$)coordinate(C')
(intersection of A--C and B--D)coordinate(O)
(intersection of A'--C' and B'--D')coordinate(O')
;
\coordinate (M) at ($(C)!0.5!(B)$);
\coordinate (N) at ($(C')!0.5!(D')$);
\draw[dashed](B)--(A)--(D)(A)--(A')(A)--(C)(B)--(D)(O)--(O') (M)--(N);
\draw(B)--(C)--(D)--(D')--(A')--(B')--cycle(B')--(C')--(D') (C')--(C) (B')--(B)(A')--(C')(B')--(D');
\draw[->](C)--($(A)!1.5!(C)$)node[right]{$y$};
\draw[->](D)--($(O)!1.5!(D)$)node[above]{$x$};
\draw[->](O')--($(O)!1.5!(O')$)node[right]{$z$};
\foreach \p/\g in {A/left,B/left,C/right,D/above right,A'/left,B'/left,C'/right,D'/right,O/below,O'/right,M/below,N/right} 
\fill (\p)circle(1pt)node[\g]{\footnotesize$\p$};
\end{tikzpicture}}\noindent
Do $MN \perp B'D \Rightarrow \overrightarrow{MN} \cdot \overrightarrow{B'D}=0 \Leftrightarrow 2-h^2=0 \Rightarrow h=\sqrt{2} \Rightarrow \overrightarrow{MN}=(1 ; 0 ; \sqrt{2})$.\\ Ta có
$MN \parallel \vec{u}=\overrightarrow{MN}=(1 ; 0 ; \sqrt{2})$, mặt phẳng $(ABCD) \perp \vec{n}=\vec{j}=(0 ; 0 ; 1)$.\\
Do $\alpha$ là góc tạo bởi đường thẳng $MN$ và mặt đáy $(ABCD)$ nên ta có
\[
\sin \alpha=|\cos (\vec{u} ; \vec{n})|=\dfrac{|\vec{u} \cdot \vec{n}|}{|\vec{u}| \cdot |\vec{n}|}=\dfrac{\sqrt{2}}{\sqrt{3}} \Rightarrow \cos \alpha=\sqrt{1-\sin ^2 \alpha}=\dfrac{1}{\sqrt{3}}.\]
}
\end{ex}

%%%% Câu 8
\begin{ex}%[2H5C4-1]%[Dự án 2025-K12-TL-TV]%[Thành Đức Trung]
Cho hình chóp $S.ABCD$ có đáy $ABCD$ là hình vuông cạnh $a$, mặt bên $(SAB)$ là tam giác đều và vuông góc với $(ABCD)$. Tính $\cos \varphi$ với $\varphi$ là góc tạp bởi $(SAC)$ và $(SCD)$.
\choice
{$\dfrac{\sqrt{3}}{7}$}
{$\dfrac{\sqrt{6}}{7}$}
{\True $\dfrac{5}{7}$}
{$\dfrac{\sqrt{2}}{7}$}
\loigiai
{Gọi $O, M$ lần lượt là trung điểm của $AB, CD$.\\ Vì mặt bên $(SAB)$ là tam giác đều và vuông góc với $(ABCD)$ nên $SO \perp(ABCD)$.
\immini{Xét hệ trục $Oxyz$ có $O(0 ; 0 ; 0), M(1 ; 0 ; 0), A\left(0 ; \dfrac{1}{2} ; 0\right)$,\\ $S\left(0 ; 0 ; \dfrac{\sqrt{3}}{2}\right), C\left(1 ; \dfrac{-1}{2} ; 0\right), D\left(1 ; \dfrac{1}{2} ; 0\right)$.\\
Suy ra $\overrightarrow{SA}=\left(0 ; \dfrac{1}{2} ; \dfrac{-\sqrt{3}}{2}\right), \overrightarrow{AC}(1 ;-1 ; 0)$\\ và $\overrightarrow{SC}=\left(1 ; \dfrac{-1}{2} ; \dfrac{-\sqrt{3}}{2}\right), \overrightarrow{CD}=(0 ; 1 ; 0)$.}{		\begin{tikzpicture}[>=stealth,line join=round,line cap=round,font=\footnotesize,scale=1]
\path
(0,0)coordinate(A)
(-150:1.5)coordinate(B)
(2,0)coordinate(D)
($(B)+(D)-(A)$)coordinate(C)
($(A)!1/2!(B)$)coordinate(O)
($(O)+(90:3)$)coordinate(S)
($(C)!1/2!(D)$)coordinate(N)
;
\coordinate (M) at ($(C)!0.5!(D)$);
\draw[dashed](B)--(A)--(D)(S)--(A)(S)--(O)(O)--(N);
\draw(S)--(B)--(C)--(D)--cycle(S)--(C);
\draw[->](N)--($(O)!1.5!(N)$)coordinate(Y)node[above]{$x$};	
\draw[dashed,->](A)--($(B)!1.6!(A)$)coordinate(X)node[above]{$y$};	
\draw[->](S)--($(O)!1.2!(S)$)coordinate(Z)node[right]{$z$};	
\foreach \p/\g in {S/left,C/below,D/above,A/left,B/left,O/left,M/below}\fill (\p)circle(1pt)node[\g]{$\p$};	
\end{tikzpicture}}\noindent
Mặt phẳng $(SAC)$ có véc tơ pháp tuyến $\overrightarrow{n_1}=[\overrightarrow{S A}, \overrightarrow{AC}]=\left(\dfrac{-\sqrt{3}}{2} ; \dfrac{-\sqrt{3}}{2} ; \dfrac{-1}{2}\right)$.\\
Mặt phẳng $(SAD)$ có véc tơ pháp tuyến $\overrightarrow{n_1}=[\overrightarrow{SC}, \overrightarrow{CD}]=\left(\dfrac{\sqrt{3}}{2} ; 0 ; 1\right)$.\\
Vậy $\cos \varphi=\dfrac{\left|\overrightarrow{n_1} \cdot \overrightarrow{n_2}\right|}{\left|\overrightarrow{n_1}\right| \cdot\left|\overrightarrow{n_2}\right|}=\dfrac{5}{7}$.
}
\end{ex}

%%%% Câu 9
\begin{ex}%[2H2C2-4]%[Dự án 2025-K12-TL-TV]%[Thành Đức Trung]
Cho hình lập phương $ABCD. A'B'C'D'$ có cạnh $a$. Góc giữa hai mặt phẳng $\left(A'B'CD\right)$ và $\left(AC C'A'\right)$ bằng
\choice
{\True $60^{\circ}$}
{$30^{\circ}$}
{$45^{\circ}$}
{$75^{\circ}$}
\loigiai
{\immini{Chọn hệ trục tọa độ $Oxyz$ sao cho $O \equiv A'$, $Ox \equiv A'D'$, $Oy \equiv A'B'$, $Oz \equiv A'A$.\\
Ta có $A'(0 ; 0 ; 0), D'(a ; 0 ; 0), B'(0 ; a ; 0), C'(a ; a ; 0)$\\ và 
$A(0 ; 0 ; a), D(a ; 0 ; a), B(0 ; a ; a), C(a ; a ; a)$.\\ Suy ra 
$\overrightarrow{A'B'}=(0 ; a ; 0), \overrightarrow{A'D}=(a ; 0 ; a), \overrightarrow{A'A}=(0 ; 0 ; a)$\\ và $ \overrightarrow{A'C'}=(a ; a ; 0)$. Ta có
$\left[\overrightarrow{A'B'}, \overrightarrow{A'D}\right]=\left(a^2 ; 0 ;-a^2\right)$.}{		\begin{tikzpicture}[line join=round,line cap=round,>=stealth,font=\footnotesize,scale=1]
\path
(0,0)coordinate(A')
(-130:1)coordinate(B')
(2,0)coordinate(D')
($(A')+(90:2)$)coordinate(A)
($(D')-(A')+(B')$)coordinate(C')
($(D')+(90:2)$)coordinate(D)
($(B')+(90:2)$)coordinate(B)
($(D)-(A)+(B)$)coordinate(C)
;
\draw[dashed](B')--(A')--(D')(A)--(A');
\draw(B')--(C')--(D')--(D)--(A)--(B)--cycle(B)--(C)--(D) (C')--(C) (B')--(B);
\draw[->](B')--($(A')!1.7!(B')$)node[left]{$y$};
\draw[->](D')--($(A')!1.5!(D')$)node[above]{$x$};
\draw[->](A)--($(A')!1.5!(A)$)node[right]{$z$};
\foreach \p/\g in {A/left,B/left,C/right,D/above right,A'/left,B'/left,C'/below,D'/below} 
\fill (\p)circle(1pt)node[\g]{$\p$};
\end{tikzpicture}}\noindent
Chọn $\overrightarrow{n_1}=(1 ; 0 ;-1)$ là vectơ pháp tuyến của mặt phẳng $\left(A'B'CD\right)$.\\
Suy ra $
\left[\overrightarrow{A'A}, \overrightarrow{A'C}\right]=\left(-a^2 ; a^2 ; 0\right)
$.\\
Chọn $\overrightarrow{n_2}=(-1 ; 1 ; 0)$ là vectơ pháp tuyến của mặt phẳng $\left(ACC'A'\right)$.\\
Góc giữa hai mặt phẳng $\left(A'B'CD\right)$ và $\left(ACC'A'\right)$ là
$$
\cos \alpha=\left|\cos \left(\overrightarrow{n_1}, \overrightarrow{n_2}\right)\right|=\dfrac{|-1|}{\sqrt{2} \cdot \sqrt{2}}=\dfrac{1}{2} \Rightarrow \alpha=60^{\circ}.
$$
}
\end{ex}

%%%% Câu 10
\begin{ex}%[2H5V4-5]%[Dự án 2025-K12-TL-TV]%[Thành Đức Trung]
Cho hình chóp tứ giác đêu $S.ABCD$ có đáy $ABCD$ là hình vuông cạnh $a$, tâm $O$. Gọi $M$ và $N$ lần lượt là trung điểm của hai cạnh $SA$ và $BC$, biết $MN=\dfrac{a \sqrt{6}}{2}$. Khi đó giá trị $\sin$ của góc giữa đường thẳng $MN$ và mặt phẳng $(SBD)$ bằng
\choice
{$\dfrac{\sqrt{2}}{5}$}
{\True $\dfrac{\sqrt{3}}{3}$}
{$\dfrac{\sqrt{5}}{5}$}
{$\sqrt{3}$}
\loigiai
{\immini{Gọi $I$ hình chiếu của $M$ lên $(ABCD)$, suy ra $I$ là trung điểm của $AO$ suy ra
$CI=\dfrac{3}{4} AC=\dfrac{3a\sqrt{2}}{4}$.\\
Xét $\triangle CNI$ có $CN=\dfrac{a}{2}, \widehat{NCI}=45^{\circ}$.\\
Áp dụng định lý cosin ta có\\
$NI=\sqrt{CN^2+CI^2-2CN \cdot CI \cdot \cos 45^{\circ}}=\dfrac{a\sqrt{10}}{4}$.\\
Xét $\triangle MIN$ vuông tại $I$ ta có\\ $MI=\sqrt{MN^2-NI^2}=\dfrac{a\sqrt{14}}{4}$.\\
Mà $MI \parallel SO, MI=\dfrac{1}{2}SO \Rightarrow SO=\dfrac{a\sqrt{14}}{2}$.}
{\begin{tikzpicture}[>=stealth,line join=round,line cap=round,font=\footnotesize,scale=.8]
\path
(0,0)coordinate(C)
(-155:2.4)coordinate(D)
(3.6,0)coordinate(B)
($(B)+(D)-(C)$)coordinate(A)
(intersection of A--C and B--D)coordinate(O)
($(O)+(90:3.5)$)coordinate(S)
($(C)!1/2!(D)$)coordinate(M)
($(A)!1/2!(B)$)coordinate(N)
($(C)!1/2!(B)$)coordinate(P)
($(A)!1/2!(D)$)coordinate(Q)
;
\coordinate (I) at ($(A)!0.5!(O)$);
\coordinate (N) at ($(C)!0.5!(B)$);
\coordinate (M) at ($(A)!0.5!(S)$);
\draw[dashed](B)--(C)--(S)(C)--(D)(S)--(O)(A)--(C)(B)--(D) (I)--(M)--(N)--(I);
\draw(S)--(D)--(A)--(B)--cycle (S)--(A);
\draw[->](B)--($(O)!1.2!(B)$)node[above]{$y$};	
\draw[dashed,->](C)--($(O)!2!(C)$)node[right]{$x$};	
\draw[->](S)--($(O)!1.2!(S)$)node[right]{$z$};	
\foreach \p/\g in {S/left,C/left,D/left,A/right,B/above,O/left,N/right,I/left,M/right}
\fill (\p)circle(1pt)node[\g]{$\p$};	
\end{tikzpicture}}\noindent
Chọn hệ trục tọa độ $Oxyz$ như hình vẽ
ta có\\ $O(0 ; 0 ; 0), B\left(0 ; \dfrac{\sqrt{2}}{2} ; 0\right), D\left(0 ;-\dfrac{\sqrt{2}}{2} ; 0\right), C\left(\dfrac{\sqrt{2}}{2} ; 0 ; 0\right)$,\\$ N\left(\dfrac{\sqrt{2}}{4} ; \dfrac{\sqrt{2}}{4} ; 0\right),
A\left(-\dfrac{\sqrt{2}}{2} ; 0 ; 0\right), S\left(0 ; 0 ; \dfrac{\sqrt{14}}{4}\right), M\left(-\dfrac{\sqrt{2}}{4} ; 0 ; \frac{\sqrt{14}}{4}\right)$.\\
Khi đó $\overrightarrow{MN}=\left(\dfrac{\sqrt{2}}{2} ; \dfrac{\sqrt{2}}{4} ;-\dfrac{\sqrt{14}}{4}\right), \overrightarrow{SB}=\left(0 ; \dfrac{\sqrt{2}}{2} ;-\dfrac{\sqrt{14}}{2}\right), \overrightarrow{S D}=\left(0 ;-\dfrac{\sqrt{2}}{2} ;-\dfrac{\sqrt{14}}{2}\right)$.\\
Vectơ pháp tuyến mặt phẳng $(SBD) \vec{n}=\overrightarrow{SB} \wedge \overrightarrow{SD}=(-\sqrt{7} ; 0 ; 0)$.\\
Suy ra $\sin (MN,(SBD))=\dfrac{|\overrightarrow{MN} \cdot \vec{n}|}{|\overrightarrow{MN}| \cdot|\vec{n}|}=\dfrac{\left|-\sqrt{7} \cdot \frac{\sqrt{2}}{2}\right|}{\sqrt{7} \cdot \frac{\sqrt{6}}{2}}=\dfrac{\sqrt{3}}{3}$.
}
\end{ex}

%%%% Câu 11
\begin{ex}%[2H5V1-6]%[Dự án 2025-K12-TL-TV]%[Thành Đức Trung]
Cho hình lăng trụ $ABC. A'B'C'$ có $A'. ABC$ là tứ diện đều cạnh $a$. Gọi $M, N$ lần lượt là trung điểm của $AA'$ và $BB'$. Tính $\tan$ của góc giữa hai mặt phẳng $(ABC)$ và $(CMN)$.
\choice
{$\dfrac{\sqrt{2}}{5}$}
{$\dfrac{3\sqrt{2}}{4}$}
{\True $\dfrac{2\sqrt{2}}{5}$}
{$\dfrac{4\sqrt{2}}{13}$}
\loigiai
{Gọi $O, H$ lần lượt là trung điểm của $AB$ và trọng tâm tam giác $ABC$.\\Vì $A'. ABC$ là tứ diện đều cạnh $a$ nên $A'H\perp (ABC)$.\immini{
Qua $O$ kẻ tia $Oz\parallel A'H$ và
 chọn hệ trục tọa độ sao cho $O(0 ; 0 ; 0),
A\left(\dfrac{1}{2} ; 0 ; 0\right), B\left(-\dfrac{1}{2} ; 0 ; 0\right), C\left(0 ; \dfrac{\sqrt{3}}{2} ; 0\right)$,\\ $H\left(0 ; \dfrac{\sqrt{3}}{6} ; 0\right), A'H=\dfrac{a \sqrt{6}}{3} \Rightarrow A'\left(0 ; \dfrac{\sqrt{3}}{6} ; \dfrac{\sqrt{6}}{3}\right)
$\\
và $\overrightarrow{AB}=\overrightarrow{A'B'} \Rightarrow B'\left(-1 ; \dfrac{\sqrt{3}}{6} ; \dfrac{\sqrt{6}}{3}\right)$.\\ Dễ thấy $(ABC)$ có véc-tơ pháp tuyến $\overrightarrow{n_1}=(0 ; 0 ; 1)$.}{\begin{tikzpicture}[scale=1, font=\footnotesize, line join=round, line cap=round, >=stealth]
\def\ac{5} % cạnh AC
\def\ab{1.75} % cạnh AB
\def\ben{4} % cạnh bên
\def\gocnghieng{45} % góc nghiêng cạnh bên
\def\gocA{40} % góc A của đáy
\coordinate (A) at (0,0);
\coordinate (C) at (\ac,0);
\coordinate (B) at (-\gocA:\ab);
\coordinate (M') at ($(B)!0.5!(C)$);
\coordinate (O) at ($(B)!0.5!(A)$);
\coordinate (H) at ($(A)!2/3!(M')$);
\coordinate (A') at ($(H)+(0,4)$);
\coordinate (M) at ($(A)!0.5!(A')$);
\coordinate (S) at ($(O)+(0,5)$);
\coordinate (B') at ($(B)-(A)+(A')$);
\coordinate (N) at ($(B)!0.5!(B')$);
\coordinate (C') at ($(C)-(A)+(A')$);
\draw (B)--(A')--(A)--(B)--(C)--(C')--(A')--(B')--(C') (B)--(B') (C)--(N)--(M);
\draw[dashed] (A)--(C)--(A')--(H) (A)--(M)--(N) (O)--(C)--(M);
\foreach \x/\g in{A'/90,B'/90, C'/90,A/-90, B/-90, C/-90,H/-100,M/120,O/-90,N/-90}	\fill[black](\x) circle (1pt)($(\x)+(\g:3.0mm)$) node{\small $\x$};
\draw[-stealth](O)--(S)node[right]{$z$};
\draw[-stealth](A)--($(O)!1.7!(A)$)node[above]{$x$};
\draw[-stealth](C)--($(O)!1.3!(C)$)node[above]{$y$};
\foreach \x/\o/\y/\r in {S/O/B/3,S/O/C/3,C/O/B/3} \draw ($(\o)!\r mm!(\x)$)--($($(\o)!\r mm!(\x)$)+($(\o)!\r mm!(\y)$)-(\o)$)--($(\o)!\r mm!(\y)$);
\end{tikzpicture}}\noindent
Gọi $M$ là trung điểm $AA' \Rightarrow M\left(\dfrac{1}{4} ; \dfrac{\sqrt{3}}{12} ; \dfrac{\sqrt{6}}{6}\right), N$ là trung điểm $BB' \Rightarrow N\left(\dfrac{-3}{4} ; \dfrac{\sqrt{3}}{12} ; \dfrac{\sqrt{6}}{6}\right)$.\\
Ta có $\overrightarrow{MN}=(-1 ; 0 ; 0), \overrightarrow{CM}=\left(\dfrac{1}{4} ; \dfrac{-5 \sqrt{3}}{12} ; \dfrac{\sqrt{6}}{6}\right)$.\\
Mặt phẳng $(CMN)$ có véc-tơ pháp tuyến $\overrightarrow{n_2}=\left(0 ; \dfrac{\sqrt{6}}{6} ; \dfrac{5 \sqrt{3}}{12}\right)=\dfrac{\sqrt{3}}{12}(0 ; 2 \sqrt{2} ; 5)$
\[\cos \varphi=\dfrac{5}{\sqrt{33}} \Rightarrow \tan \varphi=\sqrt{\frac{1}{\cos ^2 \varphi}-1}=\dfrac{2 \sqrt{2}}{5}\]
}
\end{ex}