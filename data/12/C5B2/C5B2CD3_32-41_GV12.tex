\TN
%Câu 57
\begin{ex}%[GVSB: Xuan Vy Pham]%[2H5V2-3]
Trong KG $Oxyz$, cho điểm $M(3;3;-2)$ và hai đường thẳng $d_1 \colon \dfrac{x-1}{1}=\dfrac{y-2}{3}=\dfrac{z}{1}$, $d_2 \colon \dfrac{x+1}{-1}=\dfrac{y-1}{2}=\dfrac{z-2}{4}$. Đường thẳng $d$ đi qua $M$ cắt $d_1$, $d_2$ lần lượt tại $A$ và $B$. Độ dài đoạn thẳng $AB$ bằng
\choice
{\True $3$}
{$\sqrt{6}$}
{$4$}
{$2$}
\loigiai{PTTS của đường thẳng $d_1$ là $d_1 \colon \heva{&x=1+t_1 \\&y=2+3t_1\\&z=t_1}, t_1 \in \mathbb{R}.$\\
Vì $A \in d_1$ nên $A(1+t_1;2+3t_1;t_1)$.\\
PTTS của đường thẳng $d_2$ là $d_2 \colon \heva{&x=-1-t_2 \\&y=1+2t_2\\&z=2+4t_2}, t_2 \in \mathbb{R}.$\\
Vì $B \in d_2$ nên $B(-1-t_2;1+2t_2;2+4t_2)$.\\
Ta có $\overrightarrow{MA}=(-2+t_1;-1+3t_1;2+t_1)$, $\overrightarrow{MB}=(-4-t_2;-2+2t_2;4+4t_2)$.\\
Vì $M$, $A$, $B$ thẳng hàng $\overrightarrow{MA}$, $\overrightarrow{MB}$ là hai véc-tơ cùng phương.\\ Do đó ta được
\begin{align*}
	\overrightarrow{MA}=k\overrightarrow{MB} &\Leftrightarrow (-2+t_1;-1+3t_1;2+t_1)=k(-4-t_2;-2+2t_2;4+4t_2) \\
	&\Leftrightarrow \heva{&-2+t_1=k(-4-t_2)\\ &-1+3t_1=k(-2+2t_2)\\ &2+t_1=k(4+4t_2)}\\
	&\Leftrightarrow \heva{&kt_2+4k+t_1=2\\&-2kt_2+2k+3t_1=1\\&-4kt_2-4k+t_1=-2}\\
	&\Leftrightarrow \heva{&kt_2=0\\&k=\dfrac{1}{2}\\&t_1=0}\\
	&\Leftrightarrow \heva{&k=\dfrac{1}{2}\\&t_1=0\\&t_2=0.}
\end{align*}\\
Vậy $A(1;2;0)$, $B(-1;1;2)$. Do đó $AB=\sqrt{(-1-1)^2+(1-2)^2+(2-0)^2}=3$.
}
\end{ex}
%Câu 58
\begin{ex}%[GVSB: Xuan Vy Pham]%[2H5V2-3]
Cho ba điểm $A(1;1;1)$, $B(0;0;2)$, $C(2;3;-2)$ và đường thẳng $\Delta \colon \heva{&x=2+t\\&y=1-t\\&z=t.}$ Biết điểm $M(a;b;c)$ với $a>0$ thuộc mặt phẳng $(ABC)$ sao cho $AM \perp \Delta$ và $AM=\sqrt{14}$. Tính giá trị của biểu thức $T=a+b+c$.
\choice
{$T=-1$}
{$T=5$}
{\True $T=7$}
{$T=-6$}
\loigiai{Ta có $\overrightarrow{AB}=(-1;-1;1)$, $\overrightarrow{AC}=(1;2;-3)$, $\left[\overrightarrow{AB};\overrightarrow{AC}\right]=(1;-2;-1)$. \\
Khi đó mặt phẳng $(ABC)$ có một véc-tơ pháp tuyến là $\overrightarrow{n}_{(ABC)}=(1;-2;-1)$.\\
Gọi $(Q)$ là mặt phẳng đi qua $A$ và vuông góc với đường thẳng $\Delta$. Khi đó một véc-tơ pháp tuyến của mặt phẳng $(Q)$ là $\overrightarrow{n}_Q=\overrightarrow{u}_\Delta=(1;-1;1)$.\\
Vì $AM \perp \Delta$ nên $AM \subset (Q)$.\\
Do đó $M \in (Q)$.\\
Hơn nữa theo giả thiết $M \in (ABC)$ nên $M$ thuộc vào giao tuyến của $(Q)$ và $(ABC)$.\\
Gọi $d$ là giao tuyến của $(Q)$ và $(ABC)$. \\
Khi đó một véc-tơ chỉ phương của đường thẳng $d$ là $ \overrightarrow{u}_d=\left[\overrightarrow{n}_Q;\overrightarrow{n}_{(ABC)}\right]
=(3;2;-1)$.\\
Vì $A$ là điểm chung của hai mặt phẳng $(ABC)$ và $(Q)$ nên $A \in d$.\\
Vậy PTTS của đường thẳng $d$ có dạng $d \colon \heva{&x=1+3t_1\\&y=1+2t_1\\&z=1-t_1.}$\\
Mà $M \in d$ nên $M(1+3t_1;1+2t_1;1-t_1)$.\\
Ta có
\begin{align*}
\overrightarrow{AM}=\sqrt{14} &\Leftrightarrow AM^2=14 \\&\Leftrightarrow (1+3t_1-1)^2+(1+2t_1-1)^2+(1-t_1-1)^2=14 \\&\Leftrightarrow 14t_1^2=14\\ &\Leftrightarrow \hoac{&t_1=1\\&t_1=-1.}
\end{align*}\\
Với $t_1=1$ ta được điểm $M(4;3;0)$ (thỏa mãn vì $x_M=4>0$).\\
Với $t_1=-1$ ta được điểm $M(-2;-1;2)$ (không thỏa mãn vì $x_M=-2<0$).\\
Vậy $M(4;3;0)$ là điểm cần tìm nên $T=a+b+c=4+3+0=7$.
}	
\end{ex}
%Câu 59
\begin{ex}%[GVSB: Xuan Vy Pham]%[2H5V2-3]
	Trong KG $Oxyz$, cho điểm $A(1;2;-1)$ và đường thẳng $d \colon \dfrac{x-1}{2}=\dfrac{y+1}{1}=\dfrac{z-2}{-1}$ và mặt phẳng $(P) \colon x+y+2z+1=0$. Điểm $B$ thuộc mặt phẳng $(P)$ thỏa mãn đường thẳng $AB$ vuông góc và cắt đường thẳng $d$. Tọa độ điểm $B$ là
	\choice
	{$(3;-2;-1)$}
	{$(-3;8;-3)$}
	{\True $(0;3;-2)$}
	{$(6;-7;0)$}
	\loigiai{Gọi $C$ là giao điểm của đường thẳng $AB$ và đường thẳng $d$. \\
	PTTS đường thẳng $d$ có dạng $d \colon \heva{&x=1+2t
	\\&y=-1+t\\&z=2-t.}$\\
	Do $C \in d$ nên $C(1+2t,-1+t,2-t)$.\\
	Ta có $\overrightarrow{AC}=(2t;-3+t;3-t)$ và một véc-tơ chỉ phương của đường thẳng $d$ là $\overrightarrow{u_d}=(2;1;-1)$.\\
	Vì $AB \perp d$ nên $AC \perp d$. \\
	Do đó $\overrightarrow{AC} \cdot \overrightarrow{u_d}=0 \Leftrightarrow 4t-3+t-1(3-t)=0 \Leftrightarrow t=1.$\\
	Vậy $C(3;0;1)$ và $\overrightarrow{AC}=(2;-2;2)$. \\
	Do đó đường thẳng $AC$ có một véc-tơ chỉ phương có tọa độ là $(1;-1;1)$.\\
	PTTS của đường thẳng $AC$ có dạng $AC \colon \heva{&x=1+t_1\\&y=2-t_1\\&z=-1+t_1.}$\\
	Vì $B \in AC$ nên $B(1+t_1;2-t_1;-1+t_1)$.\\
	Mà $B \in (P)$ nên $1+t_1+2-t_1+2(-1+t_1)+1=0 \Leftrightarrow 2t_1+2=0 \Leftrightarrow t_1=-1.$\\
	Vậy $B(0;3;-2)$ là điểm cần tìm.
}
\end{ex} 
%Câu 60
\begin{ex}%[GVSB: Xuan Vy Pham]%[2H5V2-3]
	Trong KG $Oxyz$, cho đường thẳng $d_1 \colon \dfrac{x-1}{1}=\dfrac{y-2}{-2}=\dfrac{z-3}{1}$ và điểm $A(1;0;-1)$. Gọi $d_2$ là đường thẳng đi qua điểm $A$ và có véc-tơ chỉ phương $\overrightarrow{v}=(a;1;2)$. Giá trị của $a$ sao cho đường thẳng $d_1$ cắt đường thẳng $d_2$ là
	\choice
	{$a=-1$}
	{$a=2$}
	{\True $a=0$}
	{$a=1$}
\loigiai{PTTS của đường thẳng $d_1$ là $d_1 \colon \heva{&x=1+t\\&y=2-2t\\&y=3+t.}$\\
PTTS của đường thẳng $d_2$ đi qua $A(1;0;-1)$ và có véc-tơ chỉ phương $\overrightarrow{v}=(a;1;2)$ có dạng $d_2 \colon \heva{&x=1+at'\\&y=t'\\&z=-1+2t'.}$\\
$d_1$ cắt $d_2$ khi vì chỉ hệ phương trình $\heva{&1+t=1+at'\\&2-2t=t'\\&3+t=-1+2t'} (*)$ có duy nhất một nghiệm.\\
Ta có $(*) \Leftrightarrow \heva{&t-at'=0\\&-2t-t'=-2\\&t-2t'=-4} \Leftrightarrow \heva{&t-at'=0\\&t=0\\&t'=2} \Leftrightarrow  \heva{&0-a \cdot 2=0\\&t=0\\&t'=2} \Leftrightarrow \heva{&a=0\\&t=0\\&t'=2.}$\\
Vậy $a=0$ là giá trị cần tìm.
}
\end{ex}
\section{PTĐT LIÊN QUAN ĐIỂM ĐỐI XỨNG VÀ HÌNH CHIẾU}
\subsection{Tìm hình chiếu $H$ của điểm $M$ lên mặt phẳng $(P) \colon ax+by+cz+d=0$.}
Viết PTĐT $MH$ qua $M(x_0;y_0;z_0)$ và vuông góc với $(P)$. Khi dó
\begin{align*}
	H=MH \cap (P) \; \text{thỏa} \; \heva{&x=x_0+at\\&y=y_0+bt\\&z=z_0+ct\\&ax+by+cz+d=0} \Rightarrow t \Rightarrow \heva{&x=?\\&y=?\\&z=?} \Rightarrow H.
\end{align*}
\begin{center}
	\begin{tikzpicture}
	\def\a{4}
	\def\b{2.5}
	\def\h{3.4}
	\path (0:0) coordinate (B)
	++(0:\a) coordinate (C)
	++(-150:\b) coordinate (D)
	($(B)+(D)-(C)$) coordinate (A)
	($(A)!0.28!(D)$) coordinate (E)
	($(A)!1/2!(C)$) coordinate (H)
	($(C)!1/2!(D)$) coordinate (L);
	\draw[thick] (B)--(A)--(D)--(C)--(B);
	\draw[thick] (H)--++(90:2)coordinate (M);
	\coordinate (M') at ($(H)!-1!(M)$);
	\draw[thick,dashed] (H)--($(H)!1/3!(M')$);
	\draw[thick] ($(H)!1/3!(M')$)--(M');
	\draw[thick](E) arc (0:33:1) node [pos=0.35,left]{$P$};
	\foreach \x/\g in {H/180,M/180,M'/180}
	\fill[black] (\x) circle (1pt) ($(\g:4mm)+(\x)$) node {$\x$};	
	\newcommand{\gv}[4][black]{\draw[thick] ($(#3)!8pt!(#2)$)--($(#3)!2!($($(#3)!8pt!(#2)$)!.5!($(#3)!8pt!(#4)$)$)$)--($(#3)!8pt!(#4)$);}
	\gv{M}{H}{L}
\end{tikzpicture}
\end{center}
Lưu ý: Để tìm điểm đối xứng $M'$ của điểm $M$ qua $(P)$ $\Rightarrow H$ là trung điểm $MM'$.
\subsection{Tìm hình chiếu $H$ của điểm $M$ lên đường thẳng $d$.}
Viết phương trình mặt phẳng $(P) \colon ax+by+cz+d=0$ qua $M$ và vuông góc với $d$, khi đó:
\begin{align*}
	H=d \cap (P) \; \text{thỏa} \; \heva{&x=x_0+a_1t\\&y=y_0+a_2t\\&z=z_0+a_3t\\&ax+by+cz+d=0} \Rightarrow t \Rightarrow \heva{&x=?\\&y=?\\&z=?} \Rightarrow H.
\end{align*}
\begin{center}
	\begin{tikzpicture}
		\def\a{4}
		\def\b{2.5}
		\def\h{3.4}
		\path (0:0) coordinate (B)
		++(90:\a) coordinate (C)
		++(150:\b) coordinate (D)
		($(B)+(D)-(C)$) coordinate (A)
		($(A)!0.28!(B)$) coordinate (E)
		($(A)!1/2!(C)$) coordinate (H)
		($(C)!1/2!(D)$) coordinate (L);
		\draw[thick] (B)--(A)--(D)--(C)--(B);
		\draw[thick] (H)--++(90:1.2)coordinate (M);
		\coordinate (M') at ($(H)!-1!(M)$);
		\draw[thick] (H)--(M');
		\draw[thick] (H)--++(180:2.2)coordinate (L);
		\coordinate (L') at ($(H)!-1!(L)$);
		\draw[dashed,thick] (H)--($(H)!1/2!(L')$);
		\draw[thick] ($(H)!1/2!(L')$)--(L') node[above]{$d$};
		\draw[thick](E) arc (0:60:1.2) node [pos=0.4,left]{$P$};
		\foreach \x/\g in {H/45,M/100,M'/-30}
		\fill[black] (\x) circle (1pt) ($(\g:4mm)+(\x)$) node {$\x$};	
		\newcommand{\gv}[4][black]{\draw[thick] ($(#3)!8pt!(#2)$)--($(#3)!2!($($(#3)!8pt!(#2)$)!.5!($(#3)!8pt!(#4)$)$)$)--($(#3)!8pt!(#4)$);}
		\gv{M}{H}{L}
	\end{tikzpicture}
\end{center}
Lưu ý: Để tìm điểm đối xứng $M'$ của điểm $M$ qua $(P)$ $\Rightarrow H$ là trung điểm $MM'$.
%Câu 61
\begin{ex}%[GVSB: Xuan Vy Pham]%[2H5H2-6]
	Trong KG $Oxyz$, khoảng cách từ điểm $M(2;-4;-1)$ tới đường thẳng $\Delta \colon \heva{&x=t\\&y=2-t\\&z=3+2t}$ bằng
	\choice
	{$\sqrt{14}$}
	{$\sqrt{6}$}
	{\True $2\sqrt{14}$}
	{$2\sqrt{6}$}
\loigiai{Đường thẳng $\Delta$ đi qua điểm $N(0;2;3)$, có véc-tơ chỉ phương là $\overrightarrow{u}=(1;-1;2)$.\\
Ta có $\overrightarrow{MN}=(-2;6;4)$, $\left[\overrightarrow{MN};\overrightarrow{u}\right]=(16;8;-4)$.\\
Do đó $\mathrm{d}(M,\Delta)=\dfrac{\left|\left[\overrightarrow{MN};\overrightarrow{u}\right] \right|}{\left| \overrightarrow{u}\right|}=\dfrac{\sqrt{336}}{\sqrt{6}}=2\sqrt{14}$.}
\end{ex}
%Câu 62
\begin{ex}%[GVSB: Xuan Vy Pham]%[2H5V2-6]
	Trong KG $Oxyz$, tọa độ hình chiếu vuông góc của $M(1;0;1)$ lên đường thẳng $(\Delta) \colon \dfrac{x}{1}=\dfrac{y}{2}=\dfrac{z}{3}$ là
	\choice
	{$\left(2;4;6\right)$}
	{$\left(1;\dfrac{1}{2};\dfrac{1}{3}\right)$}
	{$\left(0;0;0\right)$}
	{\True $\left(\dfrac{2}{7};\dfrac{4}{7};\dfrac{6}{7}\right)$}
\loigiai{PTTS của đường thẳng $\Delta$ có dạng $(\Delta) \colon \heva{&x=t\\&y=2t\\&z=3t} \; (t \in \mathbb{R})$.\\
Gọi $N(t;2t;3t) \in \Delta$ là hình chiếu vuông góc của điểm $M$ lên $\Delta$, khi đó ta có
\begin{align*}
	&\overrightarrow{MN} \cdot \overrightarrow{u}=0 \\
	&\Leftrightarrow (t-1) \cdot 1+ (2t-0) \cdot 2 + (3t-1) \cdot 3 =0 \\
	&\Leftrightarrow 14t-4=0\\
	& \Leftrightarrow t = \dfrac{2}{7} \Rightarrow N \left(\dfrac{2}{7};\dfrac{4}{7};\dfrac{6}{7}\right).
\end{align*}
	}
\end{ex}
%Câu 63
\begin{ex}%[GVSB: Xuan Vy Pham]%[2H5V2-6]
	Trong KG $Oxyz$, cho điểm $M(-4;0;0)$ và đường thẳng $\Delta \colon \heva{&x=1-t\\&y=-2+3t\\&z=-2t}.$ Gọi $H(a;b;c)$ là hình chiếu của $M$ lên $\Delta$. Tính $a+b+c$.
	\choice
	{$5$}
	{\True $-1$}
	{$-3$}
	{$7$}
\loigiai{
Gọi $H$ là hình chiếu của $M$ lên $\Delta$ nên tọa độ của $H$ có dạng $H(1-t;-2+3t;-2t)$ và $\overrightarrow{MH} \perp \overrightarrow{u}_\Delta$.\\
 Do đó
\begin{align*}
	\overrightarrow{MH} \cdot \overrightarrow{u_\Delta}=0 &\Leftrightarrow (1-t) \cdot (-1) + (-2+3t)\cdot 3 + (-2t) \cdot -2 =0\\ &\Leftrightarrow 14t-11=0\\ &\Leftrightarrow t=\dfrac{11}{14} \\ &\Rightarrow H \left(\dfrac{3}{14};\dfrac{5}{14};\dfrac{-22}{14}\right) \Rightarrow a+b+c=-1.
\end{align*}}
\end{ex}
%Câu 64
\begin{ex}%[GVSB: Xuan Vy Pham]%[2H5V2-6]
	Trong KG $Oxyz$, tọa độ hình chiếu vuông góc của $A(3;2;-1)$ lên mặt phẳng $(\alpha) \colon x+y+z=0$ là
	\choice
	{$\left(-2;1;1\right)$}
	{\True $\left(\dfrac{5}{3};\dfrac{2}{3};-\dfrac{7}{3}\right)$}
	{$\left(1;1;-2\right)$}
	{$\left(\dfrac{1}{2};\dfrac{1}{4};\dfrac{1}{4}\right)$}
\loigiai{Gọi $H$ là hình chiếu vuông góc của $A(3;2;-1)$ lên mặt phẳng $(\alpha) \colon x+y+z=0$. \\ Khi đó $AH$ nhận $\overrightarrow{n}=(1;1;1)$ làm véc-tơ chỉ phương và đi qua điểm $A(3;2;-1)$ nên PTTS của $AH$ có dạng $(AH) \colon \heva{&x=3+t\\&y=2+t\\&z=-1+t} \; (t \in \mathbb{R})$.\\
Vì $H \in AH$ nên tọa độ của $H$ có dạng $H(3+t;2+t;-1+t)$.\\
Vì $H \in (\alpha)$ nên $3+t+2+t-1+t=0 \Leftrightarrow t =-\dfrac{4}{3} \Rightarrow H \left(\dfrac{5}{3};\dfrac{2}{3};-\dfrac{7}{3}\right)$.} 
\end{ex}
%Câu 65
\begin{ex}%[GVSB: Xuan Vy Pham]%[2H5V2-6]
	Trong không gian với hệ trục tọa độ $Oxyz$, hình chiếu của điểm $M(-1;0;3)$ theo phương của véc-tơ $\overrightarrow{v}=(1;-2;1)$ trên mặt phẳng $(P) \colon x-y+z+2=0$ có tọa độ là
	\choice
	{\True $\left(2;-2;-2\right)$}
	{$\left(-1;0;1\right)$}
	{\True $\left(-2;2;2\right)$}
	{$\left(1;0;-1\right)$}
\loigiai{
	\begin{center}
		\begin{tikzpicture}
			\def\a{4}
			\def\b{2.5}
			\def\h{3.4}
			\path (0:0) coordinate (B)
			++(0:\a) coordinate (C)
			++(-150:\b) coordinate (D)
			($(B)+(D)-(C)$) coordinate (A)
			($(A)!0.28!(D)$) coordinate (E)
			($(A)!2/3!(C)$) coordinate (M')
			++(150:2) coordinate (M)
			++(-8:1.8) coordinate (F)
			++(150:1.5) coordinate (G);
			\draw[thick] (M)--(M');
			\draw[thick] (M')--($(M)!-0.8!(M')$) node [above right] {$d$};
			\draw[dashed,thick] (M')--($(M)!1.4!(M')$);
			\draw[thick] ($(M)!1.4!(M')$)--($(M)!2!(M')$);
			\draw[thick] (B)--(A)--(D)--(C)--(B);
			\draw[thick](E) arc (0:33:1) node [pos=0.47,left]{$P$};
			\draw[thick,->] (F)--(G) node [pos=0.5,above right] {$\overrightarrow{v}$};
			\foreach \x/\g in {M'/240,M/90}
			\fill[black] (\x) circle (1pt) ($(\g:4mm)+(\x)$) node {$\x$};	
		\end{tikzpicture}
	\end{center}
	Đường thẳng $d$ đi qua điểm $M(-1;0;3)$ và có véc-tơ chỉ phương là $\overrightarrow{v}=(1;-2;1)$ có PTTS là $\heva{&x=-1+t\\&y=-2t\\&z=3+t.}$\\
Gọi $M'$ là hình chiếu của điểm $M$ theo phương véc-tơ $\overrightarrow{v}=(1;-2;1)$ trên mặt phẳng $(P) \colon x-y+z+2=0$.\\
Khi đó $M'=d \cap (P)$ nên tọa độ của $M'$ là nghiệm của hệ phương trình
$$\heva{&x=-1+t \\ &y=-2t\\ &z=3+t\\ &x-y+z+2=0} \Leftrightarrow \heva{&x=-1+t \\ &y=-2t\\ &z=3+t\\ &-1+t+2t+3+t+2=0} \Leftrightarrow \heva{&x=-1+t \\ &y=-2t\\ &z=3+t\\ &t=-1} \Rightarrow M'(-2;2;2).$$}
	\end{ex}
%Câu 66
\begin{ex}%[GVSB: Xuan Vy Pham]%[2H5V2-6]
	Trong không gian với hệ trục tọa độ $Oxyz$, cho mặt phẳng $(P) \colon 6x-2y+z-35=0$ và điểm $A(-1;3;6)$. Gọi $A'$ là điểm đối xứng với $A$ qua $(P)$. Tính $OA'$
	\choice
	{$OA'=5\sqrt{3}$}
	{$OA'=\sqrt{46}$}
	{\True $OA'=\sqrt{186}$}
	{$OA'=3\sqrt{26}$}
\loigiai{Vì $A'$ đối xứng với $A$ qua $(P)$ nên $AA'$ vuông góc với $(P)$.\\
Do đó PTTS của đường thẳng $AA'$ có dạng $AA' \colon \heva{&x=-1+6t\\&y=3-2t\\&z=6+t}.$\\
Gọi $H$ là giao điểm của $AA'$ và mặt phẳng $(P)$. \\
Vì $H$ nằm trên $AA'$ nên $H(-1+6t;3-2t;6+t)$.\\
Do $H \in (P)$ nên $6(-1+6t)-2(3-2t)+1(6+t)-35=0 \Leftrightarrow 41t-41=0 \Leftrightarrow t=1 \Rightarrow H(5;1;7)$.\\
Vì $A'$ đối xứng với $A$ qua $(P)$ nên $H$ là trung điểm của $AA'$.\\
Do đó $$\heva{&x_H=\dfrac{x_A+ x_{A'}}{2}\\&y_H=\dfrac{y_A+y_{A'}}{2}\\&z_H=\dfrac{z_A+z_{A'}}{2}} \Leftrightarrow 
\heva{&x_{A'}=2x_H-x_A\\&y_{A'}=2y_H-y_A\\&z_{A'}=2z_H-z_A}\Leftrightarrow  \heva{&x_{A'}=11\\&y_{A'}=-1\\&z_{A'}=8.}$$
Vậy $A'(11;-1;8)$ nên $OA'=\sqrt{11^2+(-1)^2+8^2}=\sqrt{186}$.
}
\end{ex}
%Câu 67
\begin{ex}%[GVSB: Xuan Vy Pham]%[2H5V2-5]
	Trong không gian với hệ trục tọa độ $Oxyz$, cho đường thẳng \\$d \colon \dfrac{x-1}{1}=\dfrac{y-1}{2}=\dfrac{z-2}{-1}$ và mặt phẳng $(P) \colon 2x+y+2z-1=0$. Gọi $d'$ là hình chiếu của đường thẳng $d$ lên mặt phẳng $(P)$, véc-tơ chỉ phương của đường thẳng $d'$ là
	\choice
	{$\overrightarrow{u_1}=(5;-6;-13)$}
	{$\overrightarrow{u_2}=(5;-4;-3)$}
	{$\overrightarrow{u_1}=(5;16;13)$}
	{\True $\overrightarrow{u_1}=(5;16;-13)$}
\loigiai{Đường thẳng $d$ đi qua điểm $A(1;1;2)$ và có một véc-tơ chỉ phương $\overrightarrow{u}_d=(1;2;-1)$.\\
Mặt phẳng $(P)$ có một véc-tơ pháp tuyến $\overrightarrow{n}_{(P)}=(2;1;2)$.\\
Gọi $\overrightarrow{u}_{d'}$ là một véc-tơ chỉ phương của đường thẳng $d'$.\\
Gọi $(Q)$ là mặt phẳng chứa đường thẳng $d'$ và vuông góc với mặt phẳng $(P)$. \\
Khi đó $(Q)$ đi qua điểm $A(1;1;2)$ và có một véc-tơ pháp tuyến là $\overrightarrow{n}_{(Q)}=\left[\overrightarrow{u}_d;\overrightarrow{n}_{(P)}\right]=(5;-4;-3)$.\\
$d'$ là hình chiếu của đường thẳng $d$ trên mặt phẳng $(P)$ $\Leftrightarrow d'=(P)\cap (Q)$ nên $\heva{&\overrightarrow{u}_{d'} \perp ;\overrightarrow{n}_{(P)}\\&\overrightarrow{u}_{d'} \perp \overrightarrow{n}_{(Q)}}.$\\
Vậy véc-tơ chỉ phương của đường thẳng $d'$ là $u_{d'}=\left[\overrightarrow{n}_{(P)};\overrightarrow{n}_{(Q)}\right]=(5;16;-13).$
}
\end{ex}
%Câu 68
\begin{ex}%[GVSB: Xuan Vy Pham]%[2H5V2-5]
	Trong KG $Oxyz$, cho mặt phẳng $(\alpha) \colon 2x+y+z-3=0$ và đường thẳng $d \colon \dfrac{x+4}{3}=\dfrac{y-3}{-6}=\dfrac{z-2}{-1}$. Viết PTĐT $d'$ đối xứng với đường thẳng $d$ qua mặt phẳng $(\alpha)$.
	\choice
	{$\dfrac{x}{11}=\dfrac{y+5}{-17}=\dfrac{z-4}{-2}$}
	{$\dfrac{x}{11}=\dfrac{y-5}{-17}=\dfrac{z+4}{-2}$}
	{\True$\dfrac{x}{11}=\dfrac{y-5}{-17}=\dfrac{z-4}{-2}$}
	{$\dfrac{x}{11}=\dfrac{y-5}{-17}=\dfrac{z-4}{2}$}
\loigiai{Mặt phẳng $(\alpha) \colon 2x+y+z-3=0$ có véc-tơ pháp tuyến là $\overrightarrow{n}=(2;1;1)$.\\
Gọi $I$ là giao điểm của $d$ và $(\alpha)$.\\
Vì điểm $I \in d$ nên $I(-4+3t;3-6t;2-t)$.\\
Vì điểm $I \in (P)$ nên $2(-4+3t)+(3-6t)+(2-t)-3=0 \Leftrightarrow t=-6 \Rightarrow I(-22;39;8)$.\\
Lấy $A(-4;3;2) \in d$. Gọi $\Delta$ là đường thẳng đi qua $A$ và vuông góc $(\alpha)$. \\
Khi đó PTTS của đường thẳng $\Delta$ là $\Delta \colon \heva{&x=-4+2t\\&y=3+t\\&z=2+t.}$\\
Gọi $H$ là hình chiếu của $A$ lên $(\alpha)$. Vì $H \in \Delta$ nên $H(-4+2t;3+t;2+t)$.\\
Vì $H \in (\alpha)$ nên $2(-4+2t)+(3+t)+(2+t)-3=0 \Leftrightarrow t=1 \Rightarrow H(-2;4;3)$.\\
Vì $A'$ đối xứng với $A$ qua $(\alpha)$ nên $H$ là trung điểm $AA'$. Do đó
$$\heva{&x_H=\dfrac{x_A+ x_{A'}}{2}\\&y_H=\dfrac{y_A+y_{A'}}{2}\\&z_H=\dfrac{z_A+z_{A'}}{2}} \Leftrightarrow 
\heva{&x_{A'}=2x_H-x_A\\&y_{A'}=2y_H-y_A\\&z_{A'}=2z_H-z_A}\Leftrightarrow  \heva{&x_{A'}=0\\&y_{A'}=5\\&z_{A'}=4.} \Rightarrow H(0;5;4).$$
Đường thẳng $d'$ đối xứng với đường thẳng $d$ qua mặt phẳng $(\alpha)$ $\Rightarrow d'$ đi qua điểm $I$, $A'$ và có véc-tơ chỉ phương là $\overrightarrow{A'I}=(22;-34;-4)=2(11;-17;-2)$ có phương trình là $\dfrac{x}{11}=\dfrac{y-5}{-17}=\dfrac{z-4}{-2}$.
}
\end{ex}
%Câu 69
\begin{ex}%[GVSB: Xuan Vy Pham]%[2H5V2-5]
	Trong KG $Oxyz$, cho đường thẳng $d \colon \dfrac{x-1}{2}=\dfrac{y+5}{-1}=\dfrac{z-3}{4}$. Phương trình nào dưới đây là phương trình hình chiếu vuông góc của $d$ trên mặt phẳng $x+3=0$?
	\choice
	{$\heva{&x=-3\\&y=-5+2t\\&z=3-t}$}
	{\True $\heva{&x=-3\\&y=-6-t\\&z=7+4t}$}
	{$\heva{&x=-3\\&y=-5-t\\&z=-3+4t}$}
	{$\heva{&x=-3\\&y=-5+t\\&z=3+4t}$}
\loigiai{Đường thẳng $d$ đi qua điểm $M_0(1;-5;3)$ và có véc-tơ chỉ phương là $\overrightarrow{u}_d=(2;-1;4)$.\\
Gọi $(Q)$ là mặt phẳng chứa $d$ và vuông góc với $(P) \colon x+3=0$.\\
Suy ra mặt phẳng $(Q)$ đi qua điểm $M_0(1;-5;3)$ và có véc-tơ pháp tuyến là $\left[\overrightarrow{n}_P;\overrightarrow{u}_d\right]=(0;4;1)$ nên $(Q) \colon 4y+z+17=0$.\\
Gọi $d'$ là hình chiếu vuông góc của đường thẳng $d$ trên mặt phẳng $(P)$. Khi đó $d'$ là giao tuyến của $(Q)$ và $(P)$. Do đó $\overrightarrow{u_{d'}}=\left[\overrightarrow{n}_{(Q)};\overrightarrow{n}_{(P)}\right]=(0;-1;4)$.\\
Gọi $M$ là một điểm nằm trên $d'$. Khi đó tọa độ của $M$ là nghiệm của hệ phươg trình
$\heva{&4y+z+17=0\\&x+3=0.}$\\ 
Chọn $z=7$, khi đó ta được $\heva{&4y+7+17=0\\&x+3=0} \Leftrightarrow \heva{&x=-3\\&y=-6\\&z=7.}$\\
Vậy $M(-3;-6;7)$ là một điểm nằm trên $d'$.\\
Vậy PTTS của $d'$ là $d' \colon \heva{&x=-3\\&y=-6-t\\&z=7+4t} \; (t \in \mathbb{R}).$
}
\end{ex}
%Câu 70
\begin{ex}%[GVSB: Xuan Vy Pham]%[2H5V2-5]
	Trong KG $Oxyz$, cho mặt phẳng $(P) \colon x+y+z-3=0$ và đường thẳng $d \colon \dfrac{x}{1}=\dfrac{y+1}{2}=\dfrac{z-2}{-1}$. Hình chiếu vuông góc của $d$ trên $(P)$ có phương trình là
	\choice
	{\True $\dfrac{x-1}{1}=\dfrac{y-1}{4}=\dfrac{z-1}{-5}$}
	{$\dfrac{x-1}{1}=\dfrac{y-4}{1}=\dfrac{z+5}{1}$}
	{$\dfrac{x+1}{-1}=\dfrac{y+1}{-4}=\dfrac{z+1}{5}$}
	{$\dfrac{x-1}{3}=\dfrac{y-1}{-2}=\dfrac{z-1}{-1}$}
\loigiai{Gọi $(Q)$ là mặt phẳng chứa $d$ và vuông góc với $(P) \colon x+y+z-3=0$.\\
	Suy ra mặt phẳng $(Q)$ đi qua điểm $M_0(0;-1;2)$ và có véc-tơ pháp tuyến là $\left[\overrightarrow{n}_P;\overrightarrow{u}_d\right]=(-3;2;1)=-(3;-2;-1)$ nên $(Q) \colon 3x-2y-z=0$.\\
	Gọi $d'$ là hình chiếu vuông góc của đường thẳng $d$ trên mặt phẳng $(P)$. Khi đó $d'$ là giao tuyến của $(Q)$ và $(P)$. Do đó $\overrightarrow{u}_{d'}=\left[\overrightarrow{n}_{(Q)};\overrightarrow{n}_{(P)}\right]=(1;4;-5)$.\\
	Gọi $M$ là một điểm nằm trên $d'$. Khi đó tọa độ của $M$ là nghiệm của hệ phươg trình
	$\heva{&3x-2y-z=0\\&x+y+z-3=0.}$\\ 
	Chọn $x=1$, khi đó ta được $\heva{&3-2y-z=0\\&1+y+z-3=0} \Leftrightarrow \heva{&2y+z=3\\&y+z=2} \heva{&x=1\\&y=1\\&z=1.}$\\
	Vậy $M(1;1;1)$ là một điểm nằm trên $d'$.\\
	Vậy phương trình chính tắc của $d'$ là $d' \colon \dfrac{x-1}{1}=\dfrac{y-1}{4}=\dfrac{z-1}{-5}$.}
\end{ex}
%Câu 71
\begin{ex}%[GVSB: Xuan Vy Pham]%[2H5V2-5]
	Trong không gian với hệ trục tọa độ $Oxyz$, cho mặt phẳng $(P) \colon x+y-z-1=0$ và đường thẳng $d \colon \dfrac{x+2}{2}=\dfrac{y-4}{-2}=\dfrac{z+1}{1}$. Viết PTĐT $d'$ là hình chiếu vuông góc của $d$ trên $(P)$.
	\choice
	{$d'\colon \dfrac{x+2}{7}=\dfrac{y}{-5}=\dfrac{z+1}{2}$}
	{\True $d' \colon \dfrac{x-2}{7}=\dfrac{y}{-5}=\dfrac{z-1}{2}$}
	{ $d'\colon \dfrac{x+2}{7}=\dfrac{y}{5}=\dfrac{z+1}{2}$}
	{$d' \colon \dfrac{x-2}{7}=\dfrac{y}{5}=\dfrac{z-1}{2}$}
\loigiai{Gọi $(Q)$ là mặt phẳng chứa $d$ và vuông góc với $(P) \colon x+y-z-1=0$.\\
	Suy ra mặt phẳng $(Q)$ đi qua điểm $M_0(-2;4;-1)$ và có véc-tơ pháp tuyến là \\ $\left[\overrightarrow{n}_P;\overrightarrow{u}_d \right]=(1;3;4)$ nên $(Q) \colon x+3y+4z+D=0$.\\
	Mà $M_0(-2;4;-1) \in (Q)$ nên $-2+3 \cdot 4 +4 \cdot (-1) +D =0 \Leftrightarrow D=-6$. \\
	Vậy $(Q) \colon x+3y+4z-6=0$.\\
	Gọi $d'$ là hình chiếu vuông góc của đường thẳng $d$ trên mặt phẳng $(P)$. Khi đó $d'$ là giao tuyến của $(Q)$ và $(P)$. Do đó $\overrightarrow{u}_{d'}=\left[\overrightarrow{n}_{(Q)};\overrightarrow{n}_{(P)}\right]=(7;-5;2)$.\\
	Gọi $M$ là một điểm nằm trên $d'$. Khi đó tọa độ của $M$ là nghiệm của hệ phươg trình
	$\heva{&x+y-z-1=0\\&x+3y+4z-6=0.}$\\ 
	Chọn $x=2$, khi đó ta được $\heva{&2+y-z-1=0\\&2+3y+4z-6=0} \Leftrightarrow \heva{&y-z=-1\\&3y+4z=4} \heva{&x=2\\&y=0\\&z=1.}$\\
	Vậy $M(2;0;1)$ là một điểm nằm trên $d'$.\\
	Vậy phương trình chính tắc của $d'$ là $d' \colon \dfrac{x-2}{7}=\dfrac{y}{-5}=\dfrac{z-1}{2}$.}
\end{ex}