\begin{dang}{Lập PTĐT khi biết điểm và VTCP}

\end{dang}
\TN
\Opensolutionfile{ans}[ans/ans2-C5B1D2-TN]
%%%==============EX_1============%%% 
\begin{ex}%[Dự án 2025_K12_TL TV]%[Phạm Thị Thanh Thủy]%[2H5N2-3]
	Trong KG $Oxyz$, cho đường thẳng $d$ đi qua điểm $M(2; 2; 1)$ và có một véc-tơ chỉ phương $\overrightarrow{u}=(5; 2;-3)$. Phương trình của $d$ là
	\choice
	{$\heva{&x=2+5t \\
			&y=2+2t \\
			&z=-1-3t
		}$}
	{$\heva{&x=2+5t \\
			&y=2+2t \\
			&z=1+3t
		}$}
	{\True $\heva{&x=2+5t \\
			&y=2+2t \\
			&z=1-3t
		}$}
	{$\heva{&x=5+2t \\
			&y=2+2t \\
			&z=-3+t
		}$}
	\loigiai{
		Đường thẳng $d$ đi qua điểm $M(2; 2; 1)$ và có một véc-tơ chỉ phương $\overrightarrow{u}=(5; 2;-3)$, phương trình của $d$ là $\heva{&x=2+5t \\
			&y=2+2t \\
			&z=1-3t.
		}$
	}
\end{ex}
%%%==============EX_2============%%%
\begin{ex}%[Dự án 2025_K12_TL TV]%[Phạm Thị Thanh Thủy]%[2H5H2-3]
	Trong KG $Oxyz$, cho hai điểm $M(1; 0; 1)$ và $N(3; 2;-1)$. Đường thẳng $MN$ có PTTS là
	\choice
	{$\heva{&x=1+2t \\
			&y=2t \\
			&z=1+t}$}
	{$\heva{&x=1+t \\
			&y=t \\
			&z=1+t}$}
	{$\heva{&x=1-t \\
			&y=t \\
			&z=1+t}$}
	{\True $\heva{&x=1+t \\
			&y=t \\
			&z=1-t}$}
	\loigiai{
	Đường thẳng $MN$ nhận $\overrightarrow{MN}=(2; 2;-2)$ hoặc $\overrightarrow{u}=(1; 1;-1)$ là véc-tơ chỉ phương.\\
	Thay tọa độ điểm $M(1; 0; 1)$ vào phương trình $\heva{&x=1+t \\
		&y=t \\
		&z=1-t}$ ta thấy thỏa mãn.
	}
\end{ex}
%%%==============EX_3============%%%
	\begin{ex}%[Dự án 2025_K12_TL TV]%[Phạm Thị Thanh Thủy]%[2H5H2-3]
		Trong không gian tọa độ $Oxyz$, phương trình nào dưới đây là PTCT của đường thẳng $d\colon\heva{&x=1+2t\\&y=3t \\&z=-2+t}$ $(t\in\mathbb{R})$?
		\choice
		{$\dfrac{x+1}{2}=\dfrac{y}{3}=\dfrac{z-2}{1}$}
		{$\dfrac{x-1}{1}=\dfrac{y}{3}=\dfrac{z+2}{-2}$}
		{$\dfrac{x+1}{2}=\dfrac{y}{3}=\dfrac{z-2}{-2}$}
		{\True $\dfrac{x-1}{2}=\dfrac{y}{3}=\dfrac{z+2}{1}$}
		\loigiai{
		Do đường thẳng $d$ đi qua điểm $M(1; 0;-2)$ và có véc-tơ chỉ phương $\overrightarrow{u}=(2; 3; 1)$ nên có PTCT là $\dfrac{x-1}{2}=\dfrac{y}{3}=\dfrac{z+2}{1}$.
		}
\end{ex}
%%%==============EX_4============%%%
\begin{ex}%[Dự án 2025_K12_TL TV]%[Phạm Thị Thanh Thủy]%[2H5N2-3]
	Trong KG $Oxyz$, đường thẳng $Oy$ có PTTS là
	\choice
	{$\heva{&x=t \\
		&y=t \\
		&z=t}$ $(t \in \mathbb{R})$} 
	{\True $\heva{&x=0\\
		&y=2+t \\
		&z=0}$ $(t \in \mathbb{R})$}
	{$\heva{&x=0\\
		&y=0 \\
		&z=t}$ $(t \in \mathbb{R})$}
	{$\heva{&x=t \\
		&y=0 \\
		&z=0}$ $(t \in \mathbb{R})$}
	\loigiai{
	Đường thẳng $Oy$ đi qua điểm $A(0;2;0)$ và nhận véc-tơ đơn vị $\overrightarrow{j}=(0;1;0)$ làm véc-tơ chỉ phương nên có PTTS là $\heva{&x=0\\&y=2+t \\&z=0}$ $(t \in \mathbb{R})$.
	}
\end{ex}
%%%==============EX_5============%%%
\begin{ex}%[Dự án 2025_K12_TL TV]%[Phạm Thị Thanh Thủy]%[2H5N2-3]
	Trong không gian với hệ trục tọa độ $Oxyz$, PTTS trục $Oz$ là
	\choice
	{$z=0$}
	{$\heva{&x=0\\
			&y=t \\
			&z=0}$}
	{$\heva{&x=t \\
			&y=0\\
			&z=0}$}
	{\True $\heva{&x=0\\
			&y=0\\
			&z=t
			}$}
	\loigiai{Trục $Oz$ đi qua gốc tọa độ $O(0; 0; 0)$ và nhận véc-tơ đơn vị $\overrightarrow{k}=(0; 0; 1)$ làm véc-tơ chỉ phương nên có PTTS $\heva{&x=0\\
			&y=0\\
			&z=t.}$
	}
\end{ex}
%%%==============EX_6============%%%
\begin{ex}%[Dự án 2025_K12_TL TV]%[Phạm Thị Thanh Thủy]%[2H5N2-3]
	Trong KG $Oxyz$, trục $Ox$ có PTTS
	\choice
	{$x=0$}
	{$y+z=0$}
	{$\heva{&x=0\\
			&y=0\\
			&z=t}$}
	{\True $\heva{&x=t \\
			&y=0 \\
			&z=0}$}
	\loigiai{Trục $Ox$ đi qua $O(0; 0; 0)$ và có véctơ chỉ phương $\overrightarrow{i}=(1; 0; 0)$ nên có PTTS là
		$\heva{&x=0+1\cdot t \\
		&y=0+0t \\
		&z=0+0t} \Leftrightarrow\heva{&x=t \\
		&y=0\\
		&z=0.}$
	}
\end{ex}
%%%==============EX_7============%%%
\begin{ex}%[Dự án 2025_K12_TL TV]%[Phạm Thị Thanh Thủy]%[2H5H2-3]
	Trong KG $Oxyz$, cho đường thẳng $d\colon \dfrac{x-1}{-1}=\dfrac{y+1}{2}=\dfrac{z-2}{-1}$. Đường thẳng đi qua điểm $M(2; 1;-1)$ và song song với đường thẳng $d$ có phương trình là
	\choice
	{$\dfrac{x+2}{-1}=\dfrac{y+1}{2}=\dfrac{z-1}{-1}$}
	{\True $\dfrac{x}{1}=\dfrac{y-5}{-2}=\dfrac{z+3}{1}$}
	{$\dfrac{x+1}{2}=\dfrac{y-2}{1}=\dfrac{z+1}{-1}$}
	{$\dfrac{x-2}{1}=\dfrac{y-1}{-1}=\dfrac{z+1}{2}$}
	\loigiai{
	Vì đường thẳng song song với đường thẳng $d$ nên nó có véc-tơ  chỉ phương là $\overrightarrow{u}=(-1; 2;-1)$ hoặc $\overrightarrow{u}=(1;-2; 1)$.\\
		Lại có điểm $M(2; 1;-1)$ thuộc đường thẳng $\dfrac{x}{1}=\dfrac{y-5}{-2}=\dfrac{z+3}{1}$.\\
		Vậy phương trình của đường thẳng là $\dfrac{x}{1}=\dfrac{y-5}{-2}=\dfrac{z+3}{1}$.
	}
\end{ex}
%%%==============EX_8============%%%
\begin{ex}%[Dự án 2025_K12_TL TV]%[Phạm Thị Thanh Thủy]%[2H5H2-3]
	Trong KG $Oxyz$, cho điểm $M(2;-2; 1)$ và mặt phẳng $(P)\colon 2x-3y-z+1=0$. Đường thẳng đi qua $M$ và vuông góc với $(P)$ có phương trình là
	\choice
	{$\heva{	&x=2+2t \\
			&y=2-3t \\
			&z=1-t}$}
	{\True $\heva{	&x=2+2t \\
			&y=-2-3t \\
			&z=1-t}$}
	{$\heva{	&x=2+2t \\
			&y=-2+3t \\
			&z=1+t}$}
	{$\heva{	&x=2+2t \\
			&y=-3-2t \\
			&z=-1+t	}$}
	\loigiai{
		Gọi $d$ là đường thẳng đi qua $M$ và vuông góc với $(P)$.\\
		Do $d$ vuông góc với $(P)$ nên $d$ có một véc-tơ  chỉ phương là $\overrightarrow{u}=(2;-3;-1)$.\\
		Vậy phương trình của đường thẳng $d$ là $\heva{&x=2+2t\\
			&y=-2-3t\\
			&z=1-t.}$}
\end{ex}
%%%==============EX_9============%%%
\begin{ex}%[Dự án 2025_K12_TL TV]%[Phạm Thị Thanh Thủy]%[2H5H2-3]
	Trong KG $Oxyz$, đường thẳng đi qua điểm $A(1; 1; 1)$ và vuông góc với mặt phẳng tọa độ $(Oxy)$ có PTTS là
	\choice
	{$\heva{	&x=1+t \\
			&y=1\\
			&z=1}$}
	{\True $\heva{	&x=1\\
			&y=1\\
			&z=1+t
			&}$}
	{$\heva{	&x=1+t \\
			&y=1\\
			&z=1}$}
	{$\heva{	&x=1+t \\
			&y=1+t \\
			&z=1}$}
	\loigiai
		{Đường thẳng $d$ vuông góc với mặt phẳng tọa độ $(Oxy)$ nên nhận $\overrightarrow{k}=(0; 0; 1)$ làm véc-tơ  chỉ phương.\\
		Mặt khác $d$ đi qua $A(1; 1; 1)$ nên đường thẳng $d$ có phương trình là $\heva{	&x=1\\
			&y=1\\
			&z=1+t.}$
		}
\end{ex}
%%%==============EX_10============%%%
\begin{ex}%[Dự án 2025_K12_TL TV]%[Lê Quốc Hiệp]%[2H5H2-3]
	Trong KG $Oxyz$, cho điểm $M(3;2 ;-1)$ và mặt phẳng $(P)\colon x+z-2=0$. Đường thẳng đi qua $M$ và vuông góc với $(P)$ có phương trình là
	\choice
	{\True $\heva{&x=3+t\\&y=2\\&z=-1+t}$}
	{$\heva{&x=3+t\\&y=2+t\\&z=-1}$}
	{$\heva{&x=3+t\\&y=2t\\&z=1-t}$}
	{$\heva{&x=3+t\\&y=1+2t\\&z=-t}$}
	\loigiai{
		Ta có mặt phẳng $(P)\colon x+z-2=0$, mặt phẳng $(P)$ có véc tơ pháp tuyến là $\overrightarrow{n}_{(P)}=(1;0;1)$.\\
		Gọi đường thẳng cần tìm là $\Delta$.\\
		Vì đường thẳng $\Delta$ vuông góc với $(P)$ nên véc tơ pháp tuyến của mặt phẳng $(P)$ là véc tơ chỉ phương của đường thẳng $\Delta$ $\Rightarrow \overrightarrow{u}_{\Delta}=\overrightarrow{n}_{(P)}=(1;0;1)$.\\
		PTĐT $\Delta$ đi qua $M(3;2;-1)$ và có véc tơ chỉ phương $\overrightarrow{u}_{\Delta}=(1;0;1)$ là
		\[\heva{&x=3+t\\&y=2\\&z=-1+t.}\]
	}
\end{ex}
\begin{ex}%[Dự án 2025_K12_TL TV]%[Lê Quốc Hiệp]%[2H5H2-3]
	Trong KG $Oxyz$, cho ba điểm $A(1;2;-1)$, $B(3;0;1)$ và $C(2;2;-2)$. Đường thẳng đi qua $A$ và vuông góc với mặt phẳng $(ABC)$ có phương trình là
	\choice
	{$\dfrac{x-1}{1}=\dfrac{y-2}{-2}=\dfrac{z+1}{3}$}
	{$\dfrac{x+1}{1}=\dfrac{y+2}{2}=\dfrac{z-1}{1}$}
	{$\dfrac{x-1}{1}=\dfrac{y-2}{2}=\dfrac{z-1}{-1}$}
	{\True $\dfrac{x-1}{1}=\dfrac{y-2}{2}=\dfrac{z+1}{1}$}
	\loigiai
	{Ta có $\overrightarrow{AB}=(2;-2;2)$, $\overrightarrow{AC}=(1;0;-1)$.\\
		Mặt phẳng $(ABC)$ có một véctơ pháp tuyến là $\vec{n}=\left[\overrightarrow{AB},\overrightarrow{AC}\right]=(2;4;2)$.\\
		Đường thẳng vuông góc với mặt phẳng $(ABC)$ có một véctơ chỉ phương là $\vec{u}=(1;2;1)$.\\
		Đường thẳng đi qua $A$ và vuông góc với mặt phẳng $(ABC)$ có phương trình là
		\[\dfrac{x-1}{1}=\dfrac{y-2}{2}=\dfrac{z+1}{1}.\]
	}
\end{ex}
\begin{ex}%[Dự án 2025_K12_TL TV]%[Lê Quốc Hiệp]%[2H5H2-3]
	Trong KG $Oxyz$ cho $A(0;0;2)$, $B(2;1;0)$, $C(1;2;-1)$ và $D(2;0;-2)$. Đường thẳng đi qua $A$ và vuông góc với $(BCD)$ có phương trình là
	\choice
	{$\heva{&x=3\\&y=2\\&z=-1+2t}$}
	{\True $\heva{&x=3+3t\\&y=2+2t\\&z=1-t}$}
	{$\heva{&x=3t\\&y=2t\\&z=2+t}$}
	{$\heva{&x=3+3t\\&y=-2+2t\\&z=1-t}$}
	\loigiai
	{
		Gọi $d$ là đường thẳng đi qua $A$ và vuông góc với $(BCD)$.\\
		Ta có $\overrightarrow{BC}=(-1;1;-1)$, $\overrightarrow{BD}=(0;-1;-2)$.\\
		Mặt phẳng $(BCD)$ có vec tơ pháp tuyến là $\vec{n}_{(BCD)}=\left[\overrightarrow{BD},\overrightarrow{BC}\right]=(3;2;-1)$.\\
		Gọi $\vec{u}_{d}$ là vec tơ chỉ phương của đường thẳng $d$.\\
		Vì $d\perp(BCD)$ nên $\overrightarrow{u}_{d}=\vec{n}_{(BCD)}=(3;2;-1)$.\\
		PTĐT $d\colon\dfrac{x}{3}=\dfrac{y}{2}=\dfrac{z-2}{-1}$.\\
		Do $M(3;2;1)$ thuộc $d$ nên $d\colon\heva{&x=3+3t\\&y=2+2t\\&z=1-t.}$
	}
\end{ex}
\begin{ex}%[Dự án 2025_K12_TL TV]%[Lê Quốc Hiệp]%[2H5H2-3]
	Đường thẳng $\Delta$ là giao tuyến của $2$ mặt phẳng $x+z-5=0$ và $x-2y-z+3=0$ thì có phương trình là
	\choice
	{$\dfrac{x+2}{1}=\dfrac{y+1}{3}=\dfrac{z}{-1}$}
	{$\dfrac{x+2}{1}=\dfrac{y+1}{2}=\dfrac{z}{-1}$}
	{\True $\dfrac{x-2}{1}=\dfrac{y-1}{1}=\dfrac{z-3}{-1}$}
	{$\dfrac{x-2}{1}=\dfrac{y-1}{2}=\dfrac{z-3}{-1}$}
	\loigiai
	{
		Ta có $(P)\colon x+z-5=0$ có một vectơ pháp tuyến $\overrightarrow{n_{1}}=(1;0;1)$ và $(Q)\colon x-2y-z+3=0$ có một vectơ pháp tuyến $\overrightarrow{n_{2}}=(1;-2;-1)$.\\
		$\vec{u}=\left[\overrightarrow{n}_{1},\overrightarrow{n}_{2}\right]=(2;2;-2)=2(1;1;1)$.
		Suy ra $\Delta$ có một vectơ chỉ phương là $\vec{u}_{1}=(1;1;1)$.
		Do đó đường thẳng $\Delta$ là giao tuyến của $2$ mặt phẳng $x+z-5=0$ và $x-2y-z+3=0$ thì có phương trình là 
		$$\dfrac{x-2}{1}=\dfrac{y-1}{1}=\dfrac{z-3}{-1}.$$
		Do $M(3;2;1)$ thuộc $d$ nên $d\colon\heva{&x=3+3t\\&y=2+2t\\&z=1-t.}$
	}
\end{ex}
\Closesolutionfile{ans}
\indapan{6}{ans/ans2-C5B1D2-TN}
\TNTF
\Opensolutionfile{ans}[ans/ans2-C5B1D2-DS]
\setcounter{ex}{13}% Reset lại số đếm câu hỏi
\begin{ex}%[Dự án 2025_K12_TL TV]%[Lê Quốc Hiệp]%[2H5H2-3]
	Trong KG $Oxyz$, cho đường thẳng $d$ đi qua điểm $M(3;-1;4)$ và có một vectơ chỉ phương $\vec{u}=(-2;4;5)$. 
	\choiceTF
	{PTTS của đường thẳng $d$ là $\heva{&x=-2+3t\\&y=4-t\\&z=5+4t}$}
	{PTTS của đường thẳng $d$ là $\heva{&x=3+2t\\&y=-1+4t\\&z=4+5t}$}
	{PTTS của đường thẳng $d$ là $\heva{&x=3-2t\\&y=1+4t\\&z=4+5t}$}
	{\True PTTS của đường thẳng $d$ là $\heva{&x=3-2t\\&y=-1+4t\\&z=4+5t}$}
	\loigiai{
		\begin{itemchoice}
			\itemch \textbf{Sai.}\\
			Đường thẳng $d\colon\heva{&x=-2+3t\\&y=4-t\\&z=5+4t}$ có véc-tơ chỉ phương $\overrightarrow{u}=(3;-1;4)$.
			\itemch \textbf{Sai.}\\
			Đường thẳng $\heva{&x=3+2t\\&y=-1+4t\\&z=4+5t}$ có véc-tơ chỉ phương $\overrightarrow{u}=(2;4;5)$.
			\itemch \textbf{Sai.}\\
			Đường thẳng $\heva{&x=3-2t\\&y=1+4t\\&z=4+5t}$ không đi qua $M$.
			\itemch \textbf{Đúng.}\\
			Đường thẳng $d$ đi qua điểm $M(3;-1;4)$ và có một vectơ chỉ phương $\vec{u}=(-2;4;5)$. Phương trình của $d$ là $\heva{&x=3-2t\\&y=-1+4 t \\ &z=4+5t.}$
		\end{itemchoice}
	}
\end{ex}
\begin{ex}%[Dự án 2025_K12_TL TV]%[Lê Quốc Hiệp]%[2H5H2-3]
	Trong KG $Oxyz$, cho hai điểm $M(1;-2;1)$, $N(0;1;3)$. 
	\choiceTF
	{PTĐT qua hai điểm $M$, $N$ là $\dfrac{x+1}{-1}=\dfrac{y-2}{3}=\dfrac{z+1}{2}$}
	{PTĐT qua hai điểm $M$, $N$ là $\dfrac{x+1}{1}=\dfrac{y-3}{-2}=\dfrac{z-2}{1}$}
	{\True PTĐT qua hai điểm $M$, $N$ là $\dfrac{x}{-1}=\dfrac{y-1}{3}=\dfrac{z-3}{2}$}
	{\True PTĐT qua hai điểm $M$, $N$ là $\dfrac{x}{1}=\dfrac{y-1}{-3}=\dfrac{z-3}{-2}$}
	\loigiai{
		Ta có $\overrightarrow{MN}=(-1;3;2)$, $\overrightarrow{MN}=(1;-3;-2)$.
		\begin{itemchoice}
			\itemch \textbf{Sai.}\\
			Đường thẳng $d\colon\dfrac{x+1}{-1}=\dfrac{y-2}{3}=\dfrac{z+1}{2}$ không qua $M$.
			\itemch \textbf{Sai.}\\
			Đường thẳng $d\colon\dfrac{x+1}{1}=\dfrac{y-3}{-2}=\dfrac{z-2}{1}$ không qua $M$.
			\itemch \textbf{Đúng.}\\
			Đường thẳng $MN$ qua $N$ nhận $\overrightarrow{MN}=(-1;3;2)$ làm vectơ chỉ phương có phương trình là $\dfrac{x}{-1}=\dfrac{y-1}{3}=\dfrac{z-3}{2}$.
			\itemch \textbf{Đúng.}\\
			Đường thẳng $MN$ qua $N$ nhận $\overrightarrow{NM}=(1;-3;-2)$ làm vectơ chỉ phương có phương trình là
			$\dfrac{x}{1}=\dfrac{y-1}{-3}=\dfrac{z-3}{-2}$.
		\end{itemchoice}
	}
\end{ex}
\begin{ex}%[Dự án 2025_K12_TL TV]%[Lê Quốc Hiệp]%[2H5H2-3]
	Trong KG $Oxyz$, đường thẳng có PTTS là $(d)\colon\heva{&x=1+2t\\&y=2-t\\&z=-3+t}$. 
	\choiceTF
	{\True PTCT của đường thẳng $d$ là $\dfrac{x-1}{2}=\dfrac{y-2}{-1}=\dfrac{z+3}{1}$}
	{PTCT của đường thẳng $d$ là $\dfrac{x-1}{2}=\dfrac{y-2}{-1}=\dfrac{z-3}{1}$
	}
	{PTCT của đường thẳng $d$ là $\dfrac{x-1}{2}=\dfrac{y-2}{1}=\dfrac{z+3}{1}$}
	{\True PTCT của đường thẳng $d$ là $\dfrac{1-x}{-2}=\dfrac{2-y}{1}=\dfrac{-z-3}{-1}$}
	\loigiai{
		Đường thẳng $d$ đi qua điểm $M(1;2;-3)$ có véc tơ chỉ phương $\vec{u}=(2;-1;1)$.
		\begin{itemchoice}
			\itemch \textbf{Đúng.}\\
			Đường thẳng $d$ đi qua điểm $M(1;2;-3)$ nhận véc tơ $\vec{u}=(2;-1;1)$ làm véc-tơ chỉ phương nên có phương trình dạng chính tắc là $\dfrac{x-1}{2}=\dfrac{y-2}{-1}=\dfrac{z+3}{1}$.
			\itemch \textbf{Sai.}\\
			Đường thẳng $d\colon\dfrac{x-1}{2}=\dfrac{y-2}{-1}=\dfrac{z-3}{1}$ không qua $M$.
			\itemch \textbf{Sai.}\\
			Đường thẳng $\dfrac{x-1}{2}=\dfrac{y-2}{1}=\dfrac{z+3}{1}$ có véc tơ chỉ phương $\vec{u}=(2;1;1)$.
			\itemch \textbf{Đúng.}\\
			Ta có $\dfrac{x-1}{2}=\dfrac{y-2}{-1}=\dfrac{z+3}{1} \Leftrightarrow \dfrac{1-x}{-2}=\dfrac{2-y}{1}=\dfrac{-z-3}{-1}$.
		\end{itemchoice}
	}
\end{ex}
\begin{ex}%[Dự án 2025_K12_TL TV]%[Lê Quốc Hiệp]%[2H5H2-3]
	Trong KG $Oxyz$, cho điểm $A(1;2;3)$ và đường thẳng $d\colon\dfrac{x+4}{-2}=\dfrac{y+3}{-3}=\dfrac{z-3}{1}$. Khi đó
	\choiceTF
	{\True Đường thẳng $\Delta$ đi qua điểm $A$ và song song với đường thẳng $d$ có phương trình là $\heva{&x=1-2t \\ &y=2-3t \\ &z=3+t}$}
	{\True Đường thẳng $\Delta$ đi qua điểm $A$ và song song với đường thẳng $d$ có phương trình là $\heva{&x=1+2t\\&y=2+3t\\&z=3-t}$}
	{\True Đường thẳng $\Delta$ đi qua điểm $A$ và song song với đường thẳng $d$ có phương trình là $\dfrac{x-1}{2}=\dfrac{y-2}{3}=\dfrac{z-3}{-1}$}
	{Đường thẳng $\Delta$ đi qua điểm $A$ và song song với đường thẳng $d$ có phương trình là $\dfrac{x+1}{2}=\dfrac{y+2}{3}=\dfrac{z+3}{-1}$}
	\loigiai{
		Ta có đường thẳng $\Delta$ song song với đường thẳng $d$ nên có vectơ chỉ phương là\\ $\vec{u}_{\Delta}=\vec{u}_{d}=(-2;-3;1)=-(2;3;-1)$.
		\begin{itemchoice}
			\itemch \textbf{Đúng.}\\
			Đường thẳng $d$ đi qua điểm $A(1;2;3)$ nhận véc tơ chỉ phương $\vec{u}=(-2;-3;1)$ có phương trình là $\heva{&x=1-2t \\ &y=2-3t \\ &z=3+t}$.
			\itemch \textbf{Đúng.}\\
			Đường thẳng $d$ đi qua điểm $A(1;2;3)$ nhận véc tơ chỉ phương $\vec{u}=(2;3;-1)$ có phương trình là $\heva{&x=1+2t \\ &y=2+3t \\ &z=3-t}$.
			\itemch \textbf{Đúng.}\\
			Đường thẳng $d$ đi qua điểm $A(1;2;3)$ nhận véc tơ chỉ phương $\vec{u}=(2;3;-1)$ có phương trình là $\dfrac{x-1}{2}=\dfrac{y-2}{3}=\dfrac{z-3}{-1}$.
			\itemch \textbf{Sai.}\\
			Đường thẳng $\dfrac{x+1}{2}=\dfrac{y+2}{3}=\dfrac{z+3}{-1}$ không qua $A(1;2;3)$.
		\end{itemchoice}
	}
\end{ex}
\begin{ex}%[Dự án 2025_K12_TL TV]%[Lê Quốc Hiệp]%[2H5H2-3]
	Trong KG $Oxyz$, cho ba điểm $A(2;-2;3)$, $B(1;3;4)$ và $C(3;-1;5)$. 
	\choiceTF
	{\True Đường thẳng đi qua $A$ và song song với $BC$ có phương trình là $\heva{&x=2-2t\\&y=-2+4t\\&z=3-t}$}
	{Đường thẳng đi qua $A$ và song song với $BC$ có phương trình là\\ $\dfrac{x+2}{2}=\dfrac{y-2}{-4}=\dfrac{z+3}{1}$}
	{Đường thẳng đi qua $A$ và song song với $BC$ có phương trình là\\ $\dfrac{x-2}{4}=\dfrac{y+2}{2}=\dfrac{z-3}{9}$}
	{\True Đường thẳng đi qua $A$ và song song với $BC$ có phương trình là\\ $\dfrac{x-2}{2}=\dfrac{y+2}{-4}=\dfrac{z-3}{1}$}
	\loigiai{
		Véctơ chỉ phương của đường thẳng cần tìm là  $\overrightarrow{BC}=(2;-4;1)=-(-2;4;-1)$.
		\begin{itemchoice}
			\itemch \textbf{Đúng.}\\
			Đường thẳng đi qua $A(2;-2;3)$ và song song với $BC$ nhận véc-tơ chỉ phương\\ $\vec{u}=(-2;4;-1)$ có phương trình là $\heva{&x=2-2t\\&y=-2+4t\\&z=3-t.}$
			\itemch \textbf{Sai.}\\
			Đường thẳng $d$ $\colon\dfrac{x+2}{2}=\dfrac{y-2}{-4}=\dfrac{z+3}{1}$ không đi qua $A(2;-2;3)$.
			\itemch \textbf{Sai.}\\
			Đường thẳng $d\colon\dfrac{x-2}{4}=\dfrac{y+2}{2}=\dfrac{z-3}{9}$ có véc-tơ chỉ phương $\vec{u}=(4;2;9)$.
			\itemch \textbf{Đúng.}\\
			Đường thẳng $\dfrac{x+1}{2}=\dfrac{y+2}{3}=\dfrac{z+3}{-1}$ không qua $A(1;2;3)$.
		\end{itemchoice}
	}
\end{ex}
\Closesolutionfile{ans}
\indapan{3}{ans/ans2-C5B1D2-DS}