\Opensolutionfile{ans}[ans/ans-0-B15]
%\begin{note}
%	\begin{enumerate}
%		\item Trục $O x$ có vectơ chỉ phương là $\overrightarrow{i}=(1 ; 0 ; 0)$.
%		\item Trục $O y$ có vectơ chỉ phương là $\overrightarrow{j}=(0 ; 1 ; 0)$.
%		\item Trục $O z$ có vectơ chỉ phương là $\overrightarrow{k}=(0 ; 0 ; 1)$.
%		\item Vectơ chỉ phương $\overrightarrow{u}$ của đường thẳng $\Delta$ là vectơ có giá song song hoặc trùng với đường thẳng $\Delta$. Nếu $\Delta$ có một vectơ chỉ phương là $\overrightarrow{u}$ thì $k . \overrightarrow{u}$ cũng là một vectơ chỉ phương của $\Delta$.
%		\item Nếu có hai vectơ $\overrightarrow{n}_{1}$ và $\overrightarrow{n}_{2}$ cùng vuông góc với $\Delta$ thì $\Delta$ có một vectơ chỉ phương là $\overrightarrow{u}=\left[\overrightarrow{n}_{1}, \overrightarrow{n}_{2}\right]$.	
%	\end{enumerate}		
%\end{note}
%
%\begin{dang}{Lập PTĐT $\Delta$ dạng tham số và dạng chính tắc (nếu có)} 
%	\textbf{Phương pháp:} Đường thẳng $\Delta$ qua $M(x_0;y_0;z_0)$ và có vectơ chỉ phương $\overrightarrow{u}_\Delta=(a; b; c)$.\\
%	PTTS của $\Delta:\heva{&x=x_0+at \\&y=y_0+bt \\	&z=z_0+ct}$ $(t\in\mathbb{R})$.\\
%	Phương trình chính tắc của $\Delta: \dfrac{x-x_0}{a}=\dfrac{y-y_0}{b}=\dfrac{z-z_0}{c}$ $(a_1 a_2 a_3 \neq 0)$.
%\end{dang}
%
%\begin{dang}
%	{Lập PTTS và chính tắc (nếu có) của đường thẳng $\Delta$ đi qua hai điểm $A$ và $B$.}
%	\textbf{Phương pháp:} Đường thẳng $\Delta$ đi qua điểm $A$ hay $B$ và có vectơ chỉ phương $\overrightarrow{u}_{\Delta}=\overrightarrow{AB}$. Bài toán quay về dạng 1.
%\end{dang}
%
%\begin{dang}
%	{Viết PTĐT $\Delta$ dạng tham số và chính tắc (nếu có), biết $\Delta$ đi qua điểm $M$ và song song với đường thẳng $d$.}
%	\textbf{Phương pháp:} Đường thẳng $\Delta$ qua $M$ và có vectơ chỉ phương $\overrightarrow{u}_\Delta=\overrightarrow{d}$. Bài toán quay về dạng 1.
%\end{dang}
% 
%\begin{dang}
%	{Viết PTĐT $\Delta$ qua $M$ và song song với hai mặt phẳng $(P),(Q)$.}
%	\textbf{Phương pháp:} Đường thẳng $\Delta$ qua $M$ và có vectơ chỉ phương $\overrightarrow{u}_{\Delta}=\left[\overrightarrow{n}_{P}, \overrightarrow{n}_{Q}\right]$. Bài toán quay về dạng 1.
%\end{dang}
%
%\begin{dang}
%	{Viết PTĐT $\Delta$ dạng tham số và chính tắc (nếu có), biết $\Delta$ đi qua điểm $M$ và vuông góc với mặt phẳng $(P): Ax+By+Cz+D=0$.}
%	\textbf{Phương pháp:} Đường thẳng $\Delta$ qua $M$ và có vectơ chỉ phương $\overrightarrow{u}_\Delta=\overrightarrow{n}_{(P)}=(A;B;C)$. Bài toán quay về dạng 1.
%\end{dang} 

\TN
Mỗi câu hỏi thí sinh chỉ chọn một phương án.
%%%==============EX_1============%%% 
\begin{ex}%[Dự án 2025_K12_TL TV]%[Phạm Thị Thanh Thủy]%[2H5N2-3]
	Trong KG $Oxyz$, cho đường thẳng $d$ đi qua điểm $M(2; 2; 1)$ và có một véc-tơ chỉ phương $\overrightarrow{u}=(5; 2;-3)$. Phương trình của $d$ là
	\choice
	{$\heva{&x=2+5t \\
			&y=2+2t \\
			&z=-1-3t
		}$}
	{$\heva{&x=2+5t \\
			&y=2+2t \\
			&z=1+3t
		}$}
	{\True $\heva{&x=2+5t \\
			&y=2+2t \\
			&z=1-3t
		}$}
	{$\heva{&x=5+2t \\
			&y=2+2t \\
			&z=-3+t
		}$}
	\loigiai{
		Đường thẳng $d$ đi qua điểm $M(2; 2; 1)$ và có một véc-tơ chỉ phương $\overrightarrow{u}=(5; 2;-3)$, phương trình của $d$ là $\heva{&x=2+5t \\
			&y=2+2t \\
			&z=1-3t.
		}$
	}
\end{ex}
%%%==============EX_2============%%%
\begin{ex}%[Dự án 2025_K12_TL TV]%[Phạm Thị Thanh Thủy]%[2H5H2-3]
	Trong KG $Oxyz$, cho hai điểm $M(1; 0; 1)$ và $N(3; 2;-1)$. Đường thẳng $MN$ có PTTS là
	\choice
	{$\heva{&x=1+2t \\
			&y=2t \\
			&z=1+t}$}
	{$\heva{&x=1+t \\
			&y=t \\
			&z=1+t}$}
	{$\heva{&x=1-t \\
			&y=t \\
			&z=1+t}$}
	{\True $\heva{&x=1+t \\
			&y=t \\
			&z=1-t}$}
	\loigiai{
	Đường thẳng $MN$ nhận $\overrightarrow{MN}=(2; 2;-2)$ hoặc $\overrightarrow{u}=(1; 1;-1)$ là véc-tơ chỉ phương.\\
	Thay tọa độ điểm $M(1; 0; 1)$ vào phương trình $\heva{&x=1+t \\
		&y=t \\
		&z=1-t}$ ta thấy thỏa mãn.
	}
\end{ex}
%%%==============EX_3============%%%
	\begin{ex}%[Dự án 2025_K12_TL TV]%[Phạm Thị Thanh Thủy]%[2H5H2-3]
		Trong không gian tọa độ $Oxyz$, phương trình nào dưới đây là phương trình chính tắc của đường thẳng $d\colon\heva{&x=1+2t\\&y=3t \\&z=-2+t}$ $(t\in\mathbb{R})$?
		\choice
		{$\dfrac{x+1}{2}=\dfrac{y}{3}=\dfrac{z-2}{1}$}
		{$\dfrac{x-1}{1}=\dfrac{y}{3}=\dfrac{z+2}{-2}$}
		{$\dfrac{x+1}{2}=\dfrac{y}{3}=\dfrac{z-2}{-2}$}
		{\True $\dfrac{x-1}{2}=\dfrac{y}{3}=\dfrac{z+2}{1}$}
		\loigiai{
		Do đường thẳng $d$ đi qua điểm $M(1; 0;-2)$ và có véc-tơ chỉ phương $\overrightarrow{u}=(2; 3; 1)$ nên có phương trình chính tắc là $\dfrac{x-1}{2}=\dfrac{y}{3}=\dfrac{z+2}{1}$.
		}
\end{ex}
%%%==============EX_4============%%%
\begin{ex}%[Dự án 2025_K12_TL TV]%[Phạm Thị Thanh Thủy]%[2H5N2-3]
	Trong KG $Oxyz$, đường thẳng $Oy$ có PTTS là
	\choice
	{$\heva{&x=t \\
		&y=t \\
		&z=t}$ $(t \in \mathbb{R})$} 
	{\True $\heva{&x=0\\
		&y=2+t \\
		&z=0}$ $(t \in \mathbb{R})$}
	{$\heva{&x=0\\
		&y=0 \\
		&z=t}$ $(t \in \mathbb{R})$}
	{$\heva{&x=t \\
		&y=0 \\
		&z=0}$ $(t \in \mathbb{R})$}
	\loigiai{
	Đường thẳng $Oy$ đi qua điểm $A(0;2;0)$ và nhận véc-tơ đơn vị $\overrightarrow{j}=(0;1;0)$ làm véc-tơ chỉ phương nên có PTTS là $\heva{&x=0\\&y=2+t \\&z=0}$ $(t \in \mathbb{R})$.
	}
\end{ex}
%%%==============EX_5============%%%
\begin{ex}%[Dự án 2025_K12_TL TV]%[Phạm Thị Thanh Thủy]%[2H5N2-3]
	Trong không gian với hệ trục tọa độ $Oxyz$, PTTS trục $Oz$ là
	\choice
	{$z=0$}
	{$\heva{&x=0\\
			&y=t \\
			&z=0&}$}
	{$\heva{&x=t \\
			&y=0\\
			&z=0&}$}
	{\True $\heva{&x=0\\
			&y=0\\
			&z=t
			&}$}
	\loigiai{Trục $Oz$ đi qua gốc tọa độ $O(0; 0; 0)$ và nhận véc-tơ đơn vị $\overrightarrow{k}=(0; 0; 1)$ làm véc-tơ chỉ phương nên có PTTS $\heva{&x=0\\
			&y=0\\
			&z=t.}$
	}
\end{ex}
%%%==============EX_6============%%%
\begin{ex}%[Dự án 2025_K12_TL TV]%[Phạm Thị Thanh Thủy]%[2H5N2-3]
	Trong KG $Oxyz$, trục $Ox$ có PTTS
	\choice
	{$x=0$}
	{$y+z=0$}
	{$\heva{&x=0\\
			&y=0\\
			&z=t} \quad$}
	{\True $\heva{&x=t \\
			&y=0 \\
			&z=0&}$}
	\loigiai{Trục $Ox$ đi qua $O(0; 0; 0)$ và có véctơ chỉ phương $\overrightarrow{i}=(1; 0; 0)$ nên có PTTS là
		$\heva{&x=0+1\cdot t \\
		&y=0+0t \\
		&z=0+0t} \Leftrightarrow\heva{&x=t \\
		&y=0\\
		&z=0.}$
	}
\end{ex}
%%%==============EX_7============%%%
\begin{ex}%[Dự án 2025_K12_TL TV]%[Phạm Thị Thanh Thủy]%[2H5H2-3]
	Trong KG $Oxyz$, cho đường thẳng $d\colon \dfrac{x-1}{-1}=\dfrac{y+1}{2}=\dfrac{z-2}{-1}$. Đường thẳng đi qua điểm $M(2; 1;-1)$ và song song với đường thẳng $d$ có phương trình là
	\choice
	{$\dfrac{x+2}{-1}=\dfrac{y+1}{2}=\dfrac{z-1}{-1}$}
	{\True $\dfrac{x}{1}=\dfrac{y-5}{-2}=\dfrac{z+3}{1}$}
	{$\dfrac{x+1}{2}=\dfrac{y-2}{1}=\dfrac{z+1}{-1}$}
	{$\dfrac{x-2}{1}=\dfrac{y-1}{-1}=\dfrac{z+1}{2}$}
	\loigiai{
	Vì đường thẳng song song với đường thẳng $d$ nên nó có véc-tơ  chỉ phương là $\overrightarrow{u}=(-1; 2;-1)$ hoặc $\overrightarrow{u}=(1;-2; 1)$.\\
		Lại có điểm $M(2; 1;-1)$ thuộc đường thẳng $\dfrac{x}{1}=\dfrac{y-5}{-2}=\dfrac{z+3}{1}$.\\
		Vậy phương trình của đường thẳng là $\dfrac{x}{1}=\dfrac{y-5}{-2}=\dfrac{z+3}{1}$.
	}
\end{ex}
%%%==============EX_8============%%%
\begin{ex}%[Dự án 2025_K12_TL TV]%[Phạm Thị Thanh Thủy]%[2H5H2-3]
	Trong KG $Oxyz$, cho điểm $M(2;-2; 1)$ và mặt phẳng $(P)\colon 2x-3y-z+1=0$. Đường thẳng đi qua $M$ và vuông góc với $(P)$ có phương trình là
	\choice
	{$\heva{	&x=2+2t \\
			&y=2-3t \\
			&z=1-t
			&}$}
	{\True $\heva{	&x=2+2t \\
			&y=-2-3t \\
			&z=1-t
			&}$}
	{$\heva{	&x=2+2t \\
			&y=-2+3t \\
			&z=1+t
			&}$}
	{$\heva{	&x=2+2t \\
			&y=-3-2t \\
			&z=-1+t
			&}$}
	\loigiai{
		Gọi $d$ là đường thẳng đi qua $M$ và vuông góc với $(P)$.\\
		Do $d$ vuông góc với $(P)$ nên $d$ có một véc-tơ  chỉ phương là $\overrightarrow{u}=(2;-3;-1)$.\\
		Vậy phương trình của đường thẳng $d$ là $\heva{&x=2+2t\\
			&y=-2-3t\\
			&z=1-t.}$}
\end{ex}
%%%==============EX_9============%%%
\begin{ex}%[Dự án 2025_K12_TL TV]%[Phạm Thị Thanh Thủy]%[2H5H2-3]
	Trong KG $Oxyz$, đường thẳng đi qua điểm $A(1; 1; 1)$ và vuông góc với mặt phẳng tọa độ $(Oxy)$ có PTTS là
	\choice
	{$\heva{	&x=1+t \\
			&y=1\\
			&z=1	&}$}
	{\True $\heva{	&x=1\\
			&y=1\\
			&z=1+t
			&}$}
	{$\heva{	&x=1+t \\
			&y=1\\
			&z=1	&}$}
	{$\heva{	&x=1+t \\
			&y=1+t \\
			&z=1	&}$}
	\loigiai
		{Đường thẳng $d$ vuông góc với mặt phẳng tọa độ $(Oxy)$ nên nhận $\overrightarrow{k}=(0; 0; 1)$ làm véc-tơ  chỉ phương.\\
		Mặt khác $d$ đi qua $A(1; 1; 1)$ nên đường thẳng $d$ có phương trình là $\heva{	&x=1\\
			&y=1\\
			&z=1+t.}$
		}
\end{ex}
%%%==============EX_10============%%%
\begin{ex}%[Dự án 2025_K12_TL TV]%[Phạm Thị Thanh Thủy]%[2H5H2-3]
	Trong KG $Oxyz$, cho điểm $M(3; 2;-1)$ và mặt phẳng $(P)\colon x+z-2=0$. Đường thẳng đi qua $M$ và vuông góc với $(P)$ có phương trình là
	\choice
	{\True $\heva{	&x=3+t \\
			&y=2\\
			&z=-1+t
			&}$}
	{$\heva{	&x=3+t \\
			&y=2+t \\
			&z=-1	&}$}
	{$\heva{	&x=3+t \\
			&y=2t \\
			&z=1-t
			&}$}
	{$\heva{	&x=3+t \\
			&y=1+2t \\
			&z=-t
			&}$}
	\loigiai{
		Mặt phẳng $(P)$ có véc-tơ pháp tuyến là $\overrightarrow{n_P}=(1; 0; 1)$.\\		
		Gọi đường thẳng cần tìm là $\Delta$. Vì đường thẳng $\Delta$ vuông góc với $(P)$ nên véc-tơ pháp tuyến của mặt phẳng $(P)$ là véc-tơ chỉ phương của đường thẳng $\Delta$.\\		
		$\Rightarrow \overrightarrow{u_{\Delta}}=\overrightarrow{n_{(P)}}=(1; 0; 1)$.\\		
		Vậy PTĐT $\Delta$ đi qua $M(3; 2;-1)$ và có véc-tơ chỉ phương $\overrightarrow{u_{\Delta}}=(1; 0; 1)$ là		
		$\heva{	&x=3+t \\
			&y=2\\
			&z=-1+t.}$
	}
\end{ex}
%%%==============EX_11============%%%
\begin{ex}%[Dự án 2025_K12_TL TV]%[Phạm Thị Thanh Thủy]%[2H5H2-3]
	Trong KG $Oxyz$, cho ba điểm $A(1; 2;-1), B(3; 0; 1)$ và $C(2; 2;-2)$. Đường thẳng đi qua $A$ và vuông góc với mặt phẳng $(ABC)$ có phương trình là
	\choice
	{\True $\dfrac{x-1}{1}=\dfrac{y-2}{-2}=\dfrac{z+1}{3}$}
	{$\dfrac{x+1}{1}=\dfrac{y+2}{2}=\dfrac{z-1}{1}$}
	{$\dfrac{x-1}{1}=\dfrac{y-2}{2}=\dfrac{z-1}{-1}$}
	{$\dfrac{x-1}{1}=\dfrac{y-2}{2}=\dfrac{z+1}{1}$}
	\loigiai{Ta có $\overrightarrow{AB}=(2;-2; 2)$, $\overrightarrow{AC}=(1; 0;-1)$.\\
		Mặt phẳng $(ABC)$ có một véc-tơ pháp tuyến là $\overrightarrow{n}=\left[\overrightarrow{AB}, \overrightarrow{AC}\right]=(2; 4; 2)$.\\
		Đường thẳng vuông góc với mặt phẳng $(ABC)$ có một véc-tơ chỉ phương là $\overrightarrow{u}=(1; 2; 1)$.\\		
		Đường thẳng đi qua $A$ và vuông góc với mặt phẳng $(ABC)$ có phương trình là $$\dfrac{x-1}{1}=\dfrac{y-2}{2}=\dfrac{z+1}{1}.$$
	}
\end{ex}
%%%==============EX_12============%%%
\begin{ex}%[Dự án 2025_K12_TL TV]%[Phạm Thị Thanh Thủy]%[2H5H2-3]
	Trong KG $Oxyz$ cho $A(0; 0; 2)$, $B(2; 1; 0)$, $C(1; 2;-1)$ và $D(2; 0;-2)$. Đường thẳng đi qua $A$ và vuông góc với $(BCD)$ có phương trình là
	\choice
	{$\heva{&x=3\\
			&y=2\\
			&z=-1+2t
			&}$}
	{\True $\heva{&x=3+3t \\
			&y=2+2t \\
			&z=1-t
			&}$}
	{$\heva{&x=3t \\
			&y=2t \\
			&z=2+t
			&}$}
	{$\heva{&x=3+3t \\
			&y=-2+2t \\
			&z=1-t
			&}$}
	\loigiai
		{Gọi $d$ là đường thẳng đi qua $A$ và vuông góc với $(BCD)$.\\
		Ta có $\overrightarrow{BC}=(-1; 1;-1); \overrightarrow{BD}=(0;-1;-2)$.\\
		Mặt phẳng $(BCD)$ có véc-tơ pháp tuyến là $\overrightarrow{n}_{(BCD)}=\left[\overrightarrow{BD}, \overrightarrow{BC}\right]=(3; 2;-1)$.\\
		Gọi $\overrightarrow{u}_d$ là véc-tơ chỉ phương của đường thẳng $d$.\\
		Vì $d \perp(BCD)$ nên $\overrightarrow{u_d}=\overrightarrow{n}_{(BCD)}=(3; 2;-1)$.\\
		Đường thẳng qua điểm $A(0; 0; 2)$.
	}
\end{ex}
%%%==============EX_13============%%%
\begin{ex}%[Dự án 2025_K12_TL TV]%[Phạm Thị Thanh Thủy]%[2H5H2-3]
	Đường thẳng $\Delta$ là giao tuyến của hai mặt phẳng là $x+z-5=0$ và $x-2y-z+3=0$ thì có phương trình là
	\choice
	{$\dfrac{x+2}{1}=\dfrac{y+1}{3}=\dfrac{z}{-1}$}
	{$\dfrac{x+2}{1}=\dfrac{y+1}{2}=\dfrac{z}{-1}$}
	{\True $\dfrac{x-2}{1}=\dfrac{y-1}{1}=\dfrac{z-3}{-1}$}
	{$\dfrac{x-2}{1}=\dfrac{y-1}{2}=\dfrac{z-3}{-1}$}
	\loigiai{$(P)\colon x+z-5=0$ có một vtpt $\overrightarrow{n_1}=(1; 0; 1)$.\\		
		$(Q)\colon x-2y-z+3=0$ có một vtpt $\overrightarrow{n_2}=(1;-2;-1)$.\\		
		Gọi $\Delta$ là giao tuyến của hai mặt phẳng thì $\Delta$ có một véc-tơ  chỉ phương $\overrightarrow{u}=\left[\overrightarrow{n_1}, \overrightarrow{n_2}\right]=(2; 2;-2)$.
	}
\end{ex}

\TNTF
Trong mỗi ý a), b), c), d) ở mỗi câu, thí sinh chọn đúng hoặc sai.
%%%==============EX_1============%%%
\begin{ex}%[Dự án 2025_K12_TL TV]%[Phạm Thị Thanh Thủy]%[2H5N2-3]
	Trong KG $Oxyz$, cho đường thẳng $d$ đi qua điểm $M(3;-1; 4)$ và có một véc-tơ  chỉ phương $\overrightarrow{u}=(-2; 4; 5)$. Các mệnh đề sau đây đúng hay sai?
	\choiceTF
	{PTTS của đường thẳng $d$ là $\heva{&x=-2+3t \\
			&y=4-t \\
			&z=5+4t
			&}$}
	{PTTS của đường thẳng $d$ là $\heva{&x=3+2t \\
			&y=-1+4t \\
			&z=4+5t
			&}$}
	{PTTS của đường thẳng $d$ là $\heva{&x=3-2t \\
			&y=1+4t \\
			&z=4+5t
			&}$}
	{\True PTTS của đường thẳng $d$ là $\heva{&x=3-2t \\
			&y=-1+4t \\
			&z=4+5t
			&}$}
	\loigiai{
		Đường thẳng $d$ đi qua điểm $M(3;-1; 4)$ và có một véc-tơ  chỉ phương $\overrightarrow{u}=(-2; 4; 5)$. Phương trình của $d$ là $\heva{&x=3-2t \\
				&y=-1+4t \\
				&z=4+5t.}$
		}
\end{ex}
%%%==============EX_2============%%%
\begin{ex}%[Dự án 2025_K12_TL TV]%[Phạm Thị Thanh Thủy]%[2H5H2-3]
	Trong KG $Oxyz$, cho hai điểm $M(1;-2; 1), N(0; 1; 3)$. Các mệnh đề sau đây đúng hay sai?
	\choiceTF
	{PTĐT qua hai điểm $M, N$ là $\dfrac{x+1}{-1}=\dfrac{y-2}{3}=\dfrac{z+1}{2}$}
	{PTĐT qua hai điểm $M, N$ là $\dfrac{x+1}{1}=\dfrac{y-3}{-2}=\dfrac{z-2}{1}$}
	{\True PTĐT qua hai điểm $M, N$ là $\dfrac{x}{-1}=\dfrac{y-1}{3}=\dfrac{z-3}{2}$}
	{\True PTĐT qua hai điểm $M, N$ là $\dfrac{x}{1}=\dfrac{y-1}{-3}=\dfrac{z-3}{-2}$}
	\loigiai{
		Ta có $\overrightarrow{MN}=(-1; 3; 2)=-1(1;-3;-2)$.\\
		Đường thẳng $MN$ qua $N$ nhận $\overrightarrow{MN}=(-1; 3; 2)$ làm véc-tơ  chỉ phương có phương trình $$\dfrac{x}{-1}=\dfrac{y-1}{3}=\dfrac{z-3}{2}.$$
		Đường thẳng $MN$ qua $N$ nhận $\overrightarrow{MN}=(1;-3;-2)$ làm véc-tơ  chỉ phương có phương trình $$\dfrac{x}{1}=\dfrac{y-1}{-3}=\dfrac{z-3}{-2}.$$
	}
\end{ex}
%%%==============EX_3============%%%
\begin{ex}%[Dự án 2025_K12_TL TV]%[Phạm Thị Thanh Thủy]%[2H5H2-3]
	Trong không gian $\mathrm{Ox} y z$, đường thẳng có PTTS là $(d)\colon \heva{	&x=1+2t \\
		&y=2-t \\
		&z=-3+t
		&}$. Các mệnh đề sau đây đúng hay sai?
	\choiceTF
	{\True Phương trình chính tắc của đường thẳng $d$ là $\dfrac{x-1}{2}=\dfrac{y-2}{-1}=\dfrac{z+3}{1}$}
	{Phương trình chính tắc của đường thẳng $d$ là $\dfrac{x-1}{2}=\dfrac{y-2}{-1}=\dfrac{z-3}{1}$}
	{Phương trình chính tắc của đường thẳng $d$ là $\dfrac{x-1}{2}=\dfrac{y-2}{1}=\dfrac{z+3}{1}$}
	{\True Phương trình chính tắc của đường thẳng $d$ là $\dfrac{1-x}{-2}=\dfrac{2-y}{1}=\dfrac{-z-3}{-1}$}
	\loigiai{
		Đường thẳng $d$ đi qua điểm $M(1; 2;-3)$ nhận véc-tơ $\overrightarrow{u}=(2;-1; 1)$ nên có phương trình dạng chính tắc là $\dfrac{x-1}{2}=\dfrac{y-2}{-1}=\dfrac{z+3}{1}$.\\
		Ta có $\dfrac{x-1}{2}=\dfrac{y-2}{-1}=\dfrac{z+3}{1} \Leftrightarrow \dfrac{1-x}{-2}=\dfrac{2-y}{1}=\dfrac{-z-3}{-1}$.
	}
\end{ex}
%%%==============EX_4============%%%
\begin{ex}%[Dự án 2025_K12_TL TV]%[Phạm Thị Thanh Thủy]%[2H5H2-3]
	Trong KG $Oxyz$, cho điểm $A(1; 2; 3)$ và đường thẳng $d: \dfrac{x+4}{-2}=\dfrac{y+3}{-3}=\dfrac{z-3}{1}$. Các mệnh đề sau đây đúng hay sai?
	\choiceTF
	{\True Đường thẳng $\Delta$ đi qua điểm $A$ và song song với đường thẳng $d$ có phương trình là: $\heva{	&x=1-2t \\
			&y=2-3t \\
			&z=3+t
			&}$}
	{\True Đường thẳng $\Delta$ đi qua điểm $A$ và song song với đường thẳng $d$ có phương trình là: $\heva{	&x=1+2t \\
			&y=2+3t \\
			&z=3-t
			&}$}
	{\True Đường thẳng $\Delta$ đi qua điểm $A$ và song song với đường thẳng $d$ có phương trình là: $\dfrac{x-1}{2}=\dfrac{y-2}{3}=\dfrac{z-3}{-1}$}
	{Đường thẳng $\Delta$ đi qua điểm $A$ và song song với đường thẳng $d$ có phương trình là: $\dfrac{x+1}{2}=\dfrac{y+2}{3}=\dfrac{z+3}{-1}$}
	\loigiai{		
	Ta có đường thẳng $\Delta$ song song với đường thẳng $d$ nên có véc-tơ chỉ phương $\overrightarrow{u}_{\Delta}=\overrightarrow{u}_d=(-2;-3; 1)=-(2; 3;-1)$.\\
	Do vậy đường thẳng đi qua $A$ và song song với $d$ có phương trình là\\
	$\heva {&x=1-2t \\
			&y=2-3t \\
			&z=3+t}$ hoặc $\heva{	&x=1+2t \\
			&y=2+3t \\
			&z=3-t}$ hoặc $\dfrac{x-1}{2}=\dfrac{y-2}{3}=\dfrac{z-3}{-1}$.
		
	}
\end{ex}
\Closesolutionfile{ans}

