\chude{Lập phương trình đường thẳng liên quan đến song song và vuông góc}
\begin{ex}%[2H5V2-3]
	Trong không gian với hệ tọa độ $Oxyz$, cho mặt phẳng $(P)\colon x+y-z-1=0$ và đường thẳng $d\colon \dfrac{x+2}{2}=\dfrac{y-4}{-2}=\dfrac{z+1}{1}$. Viết phương trình đường thẳng $d'$ là hình chiếu vuông góc của $d$ trên $(P)$.
	\choice
		{$d'\colon \dfrac{x+2}{7}=\dfrac{y}{-5}=\dfrac{z+1}{2}$}
		{\True $d'\colon \dfrac{x-2}{7}=\dfrac{y}{-5}=\dfrac{z-1}{2}$}
		{$d'\colon \dfrac{x+2}{7}=\dfrac{y}{5}=\dfrac{z+1}{2}$}
		{$d'\colon \dfrac{x-2}{7}=\dfrac{y}{5}=\dfrac{z-1}{2}$}
	\loigiai{
		$(\Delta)\colon \dfrac{x-0}{1}=\dfrac{y-2}{1}=\dfrac{z-0}{-1} \Leftrightarrow(\Delta)\colon \heva{&x=c \\& y=2+c\\& z=-c}$, $c\in \mathbb{R}$.\\
		Gọi $M'=(c;2+c;-c)$ là giao điểm của $\Delta$ với mặt phẳng $(P)$. Lúc đó $$c+(2+c)-(-c)-1=0 \Leftrightarrow c=-\dfrac{1}{3}.$$
		Suy ra $M'\left(-\dfrac{1}{3};\dfrac{5}{3}; \dfrac{1}{3}\right)$, $\overrightarrow{MM'}=\left(-\dfrac{7}{3}; \dfrac{5}{3};-\dfrac{2}{3}\right)$. \\
		Đường thẳng $d'$ là hình chiếu vuông góc của $d$ trên mặt phẳng $(P)$ nên $d'$ chính là đường thẳng $MM'$.\\
		Vậy $d'$ đi qua $M(2;0;1)$ và nhận vectơ $\vec{u}=-3\overrightarrow{MM'}=(7 ;-5;2)$ làm vectơ chỉ phương nên phương trình của $d'$ là $d'\colon \dfrac{x-2}{7}=\dfrac{y}{-5}=\dfrac{z-1}{2}$.
	}
\end{ex}
	
\begin{ex}%[2H5V2-3]
	Trong không gian với hệ tọa độ $Oxyz$, cho mặt phẳng $(\alpha)\colon x+y-z+6=0$ và đường thẳng $d\colon\dfrac{x-1}{2}=\dfrac{y+4}{3}=\dfrac{z}{5}$. Hình chiếu vuông góc của $d$ trên $(\alpha)$ có phương trình là
	\choice 										      
	    {$\dfrac{x+1}{2}=\dfrac{y+4}{3}=\dfrac{z-1}{5}$}
		{\True $\dfrac{x}{2}=\dfrac{y+5}{3}=\dfrac{z-1}{5}$}
		{$\dfrac{x+5}{2}=\dfrac{y}{3}=\dfrac{z-1}{5}$}
		{$\dfrac{x}{2}=\dfrac{y-5}{3}=\dfrac{z-1}{5}$}
	\loigiai{ 
		Mặt phẳng $(\alpha)\colon x+y-z+1=0$ có vectơ pháp tuyến $\vec{n}=(1;1;-1)$. \\
		Đường thẳng $d\colon \dfrac{x-1}{2}=\dfrac{y+4}{3}=\dfrac{z}{5}$ có vectơ chỉ phương $\vec{u}=(2;3;5)$.\\
		Vì $\vec{n} \cdot \vec{u}=1 \cdot 2+1.3+(-1) \cdot 5=0$ nên $d\parallel (\alpha)$.\\
		Gọi $d'$ là hình chiếu vuông góc của $d$ trên $(\alpha)$. Lúc đó $d'\parallel d$.\\
		Lấy $A(1;-4;0) \in d$. Gọi $\Delta$ là đường thẳng đi qua $A$ và vuông góc với $(\alpha)$. Suy ra phương trình đường thẳng $\Delta$ là $\heva{&x=1+t \\ &y=-4+t \\ &z=-t.}$\\
		Gọi $A'$ là hình chiếu của $A$ lên $(\alpha)$ thì $A'=\Delta \cap(\alpha) \Rightarrow A'(0;-5;1)$.\\
		Đường thẳng $d'$ là đường thẳng đi qua $A'(0;-5;1)$, có vectơ chỉ phương $\vec{u}=(2;3;5)$ có phương trình là $\dfrac{x}{2}=\dfrac{y+5}{3}=\dfrac{z-1}{5}$.
	}
\end{ex}

	
\begin{ex}%[2H5V2-3]
	Trong không gian với hệ tọa độ $Oxyz$, cho mặt phẳng $(P)\colon x+y+z-3=0$ và đường thẳng $d\colon \dfrac{x}{1}=\dfrac{y+1}{2}=\dfrac{z-2}{-1}$. Hình chiếu của $d$ trên $(P)$ có phương trình là đường thẳng $d'$. Trong các điểm sau điểm nào thuộc đường thẳng $d'$?
	\choice 
		{\True $M(2;5;-4)$}
		{$P(1;3;-1)$}
		{$N(1;-1;3)$}
		{$Q(2;7;-6)$}
	\loigiai{
		Gọi  $A=d \cap (P)$. Vì $A \in d\colon \heva{&	x=t\\&y=-1+2t\\&z=2-t} \Rightarrow A(t;-1+2t;2-t)$.\\
		Mặt khác $A \in (P) \Rightarrow t-1+2 t+2-t-3=0 \Leftrightarrow t=1$. Vậy $A(1; 1;1)$.\\
		Lấy $B(0;-1;2)\in d$. Gọi $\Delta$ là đường thẳng qua $B$ và vuông góc $(P)$ thì $\Delta\colon\heva{&x=t'\\&y=-1+t'\\&z=2+t'.}$\\
		Gọi $C$ là hình chiếu của $B$ lên $(P)$. Ta có $C \in \Delta \Rightarrow C\left(t';-1+t';2+t'\right)$.\\
		Mặt khác $C \in(P) \Rightarrow t'-1+t'+2+t'-3=0 \Leftrightarrow t'=\dfrac{2}{3}$.\\
		Vậy 
		$C\left(\dfrac{2}{3};\dfrac{-1}{3};\dfrac{8}{3}\right)$.\\
		Lúc này $d'$ qua $A(1;1;1)$ và có một vectơ chỉ phương là $\overrightarrow{AC}=\left(-\dfrac{1}{3};-\dfrac{4}{3};\dfrac{5}{3}\right)$. Hay $d'$ nhận $\vec{u}=(1;4;-5)$ làm một vectơ chỉ phương.\\
		Suy ra $d'\colon\heva{&x=1+s \\ &y=1+4 s \\ &z=1-5 s.}$ \\
		Vậy điểm thuộc đường thẳng $d'$ là $M(2;5;-4)$.
	}
\end{ex}
	
\begin{ex}%[2H5V2-3]
	Trong không gian với hệ tọa độ $Oxyz$, cho đường thẳng $d\colon \dfrac{x-1}{2}=\dfrac{y-2}{1}=\dfrac{z+1}{3}$ và mặt phẳng $(P)\colon x+y+z-3=0$. Đường thẳng $d'$ là hình chiếu của $d$ theo phương $Ox$ lên $(P)$, $d'$ nhận $\vec{u}=(a;b;2019)$ làm một vectơ chỉ phương. Xác định tổng $a+b$.
	\choice 
		{$2019$}
		{\True $-2019$}
		{$2018$}
		{$-2020$}
	\loigiai{
		 \begin{center}
		 	\begin{tikzpicture}[scale=2]
			\def\a{3}
			\def\b{1}
			\def\g{30}
			\def\h{2}
			\path
			(0:0) coordinate (A)--++(\g:\b) coordinate (B)--++(0:\a) coordinate (C)--++(\g-180:\b) coordinate (D)--++(\g+143:2.1) coordinate (E)--++(0:.3) coordinate (F)--++(0:1.7) coordinate (G)--++(0:.2) coordinate (H)--++(150:2) coordinate (M)--++(150:.3) coordinate (N)--++(-20:2)coordinate (O);
			\draw  pic [draw, angle radius = 10 mm,"$P$"] {angle = D--A--B}; 		
			 \coordinate (I) at ($(M)!1.2!(F)$);
			\coordinate (J) at ($(F)!1.3!(M)$);
			\coordinate (K) at ($(N)!1.1!(G)$);
			\coordinate[label = right:$x$] (P) at ($(J)+(O)-(F)$); 
			\coordinate (Q) at ($(P)!1.1!(O)$);	
			\draw
			(A)--(B)--(C)--(D)--cycle (Q)--(P)  (H)--(E) node [left]{$d'$} (N) node[left]{$d$}--(G)node[below]{$A$} (F) node[below right]{$H$}--(J);
			\draw[dashed](I)--(F)(K)--(G);
			\fill (F) circle (1pt) (G) circle (1pt) (M) node[above right]{$M$} circle (1pt) (O) node[right]{$O$} circle (1pt);
			\end{tikzpicture}
		 \end{center}
		Mặt phẳng $(P)$ có vectơ pháp tuyến là $\vec{n}_{(P)}=(1;1;1)$, đường thẳng $d$ có vectơ chỉ phương là $\vec{u}_d=(2;1;3)$, đường thẳng chứa trục  $Ox$ có vectơ chỉ phương $\vec{i}=(1;0;0)$.\\
		Gọi $(Q)$ là mặt phẳng chứa đường thẳng $d$ và song song (hoặc chứa) trục $Ox$. Khi đó $(Q)$ có vectơ pháp tuyến $\vec{n}_{(Q)}=\left[\vec{u}_d, \vec{i}\right]=(0;3;-1)$.\\
		Đường thẳng $d'$ chính là giao tuyến của $(P)$ và $(Q)$. Từ đó có vectơ chỉ phương của $d'$ là $\vec{u}_1=\left[\vec{n}_{(P)}, \vec{n}_{(Q)}\right]=(-4;1;3)$.\\
		Suy ra $\vec{u}=(-2692;673;2019)$ cũng là vectơ chỉ phương của $d'$.\\ Ta có $a+b=-2692+673=-2019$.
	}
\end{ex}

\begin{ex}%[2H5V2-3]
	Trong không gian với hệ tọa độ $Oxyz$, cho hai đường thẳng $d\colon \heva{&x=-2\\& y=t\\&z=2+2t}$, $(t \in \mathbb{R})$, $\Delta\colon\dfrac{x-3}{1}=\dfrac{y-1}{-1}=\dfrac{z-4}{1}$ và mặt phẳng $(P)\colon x+y-z+2=0$. Gọi $d'$ và $\Delta'$ lần lượt là hình chiếu của $d$ và $\Delta$ lên mặt phẳng $(P)$. Gọi $M(a;b;c)$ là giao điểm của hai đường thẳng $d'$ và $\Delta'$. Biểu thức $a+b\cdot c$ bằng
	\choice 
		{$4$}
		{\True $5$}
		{$3$}
		{$6$}
	\loigiai{
		Do $d'$ là hình chiếu của $d$ lên mặt phẳng $(P)$ nên $d'$ là giao tuyến của mặt phẳng $(P)$ và mặt phẳng $(\alpha)$ chứa $d$ và vuông góc với mặt phẳng $(P)$. Suy ra một vectơ pháp tuyến của mặt phẳng $(\alpha)$ là $\vec{n}_{(\alpha)}=\left[\vec{u_d}, \vec{n}_{P}\right]=(-3;2;-1)$.\\
		Mặt phẳng $(\alpha)$ đi qua $A(-2;0;2)$ và có một vectơ pháp tuyến $\vec{n}_{(\alpha)}=(-3;2;-1)$ có phương trình là $$3x-2y+z+4=0.$$
		Do $\Delta'$ là hình chiếu của $\Delta$ lên mặt phẳng $(P)$ khi đó $\Delta'$ là giao tuyến của mặt phẳng $(P)$ và mặt phẳng $(\beta)$ chứa $\Delta$ và vuông góc vởi mặt phẳng $(P)$. Suy ra  một vectơ pháp tuyến của mặt phẳng $(\beta)$ là $\vec{n}_{(\beta)}=\left[\vec{u}_{\Delta}, \vec{n}_P\right]=(0;-2;-2)$.\\
		Mặt phẳng $(\beta)$ đi qua $B(3;1;4)$ và có một vectơ pháp tuyến
		$\vec{n}_{(\beta)}=(0;-2;-2)$ có phương trình là 
		$$y+z-5=0.$$
		Tọa độ điểm $M$ là nghiệm của hệ phương trình $\heva{&x+y-z+2=0 \\&3x-2y+z+4=0 \\&y+z-5=0} \Leftrightarrow\heva{&x=-1 \\ &y=2 \\&z=3.}$\\
		Vậy $M(-1;2;3) \Rightarrow a+b \cdot c=-1+2 \cdot 3=5$.
	}
\end{ex}

\begin{ex}%[2H5V2-3]
	Trong không gian với hệ tọa độ $Oxyz$, cho điểm  $A(1;1;1)$ và đường thẳng $d\colon\heva{&x=1+t \\& y=1+t \\&z=t}$. Tìm tọa độ điểm $H$ là hình chiếu của $A$ lên đường thẳng $\Delta$.
	\choice 
		{\True $H\left(\dfrac{4}{3};\dfrac{4}{3};\dfrac{1}{3}\right)$}
		{$H(1;1;1)$}
		{$H(0;0;-1)$}
		{$H(1;1;0)$}
	\loigiai{ 
		Đường thẳng $d$ có vectơ chỉ phương là $\overrightarrow{u}=(1;1;1)$.\\
		Do $H \in d \Rightarrow H\left(1+t;1+t;t\right)\Rightarrow\overrightarrow{AH}=(t;t;t-1)$.\\
		Do $H$ là hình chiếu của điểm $A$ lên đường thẳng $d$ nên suy ra $$\overrightarrow{AH} \perp \vec{u} \Leftrightarrow \overrightarrow{AH} \cdot \vec{u}=0 \Leftrightarrow t+t+t-1=0 \Leftrightarrow t=\dfrac{1}{3} \Rightarrow H\left(\dfrac{4}{3};\dfrac{4}{3};1\right).$$
	}
\end{ex}

\begin{ex}%[2H5V2-3]
	Trong không gian với hệ tọa độ $Oxyz$, cho điểm $A(1;1;1)$ và đường thẳng $(d)\colon \heva{&x=6-4t\\&y=-2-t \\&z=-1+2t}$. Tìm tọa độ hình chiếu $A'$ của $A$ trên $(d)$.
	\choice 
		{$A'(2;3;1)$}
		{$A'(-2;3;1)$}
		{\True $A'(2;-3;1)$}
		{$A'(2;-3;-1)$}
	\loigiai{
		Ta có $A'\in(d)$ nên gọi $A'(6-4 t;-2-t;-1+2 t)$, suy ra $\overrightarrow{AA'}=(5-4t;-3-t;-2+2t)$.\\
		Đường thẳng $(d)$ có vectơ chỉ phương $\vec{u}=(-4;-1;2)$.\\
		Vì $AA' \perp (d) \Leftrightarrow \overrightarrow{AA'} \cdot \vec{u}=0 \Leftrightarrow(5-4t) \cdot(-4)+(-3-t) \cdot(-1)+(-2+2 t) \cdot 2=0 \Leftrightarrow t=1$. Vậy $A'(2 ;-3 ; 1)$.
	}
\end{ex}

\begin{ex}%[2H5V2-3]
	Trong không gian với hệ tọa độ $Oxyz$, cho đường thẳng $d\colon \dfrac{x+1}{1}=\dfrac{y+3}{2}=\dfrac{z+2}{2}$ và điểm $A(3;2;0)$. Điểm đối xứng của điểm $A$ qua đường thẳng $d$ có tọa độ là
	\choice 
		{\True $(-1;0;4)$}
		{$(7;1;-1)$}
		{$(2;1;-2)$}
		{$(0;2;-5)$}
	\loigiai{
		Gọi $(P)$ là mặt phẳng đi qua $A$ và vuông góc với đường thẳng $d$. Phương trình của mặt phẳng $(P)$ là $1(x-3)+2(y-2)+2(z-0)=0$ $\Leftrightarrow x+2 y+2 z-7=0$.\\
		Gọi $H$ là hình chiếu của $A$ lên đường thẳng $d$, khi đó $H=d \cap (P)$.\\
		Vì $H \in d$ nên $ H(-1+t ;-3+2 t ;-2+2 t)$.\\
		Mặt khác $H \in(P) $ nên $-1+t-6+4 t-4+4 t-7=0 \Rightarrow t=2$.\\ Vậy $H(1;1;2)$.\\
		Gọi $A'$ là điểm đối xứng với $A$ qua đường thẳng $d$, khi đó $H$ là trung điểm của $AA'$.\\
		Suy ra $A'(-1;0;4)$.
	}
\end{ex}

\begin{ex}%[2H5V2-3]
	Trong không gian với hệ tọa độ $Oxyz$, xác định tọa độ điểm $M'$ là hình chiếu vuông góc của điểm $M(2;3;1)$ lên mặt phẳng $(\alpha)\colon x-2y+z=0$.
	\choice
		{$M'\left(2;\dfrac{5}{2};3\right)$}
		{$M'(1;3;5)$}
		{\True  $M'\left(\dfrac{5}{2};2;\dfrac{3}{2}\right)$}
		{$M'(3;1;2)$}
	\loigiai{
		Gọi $\Delta$ là đường thẳng qua $M$ và vuông góc với $(\alpha)$.\\ Phương trình tham số của $\Delta$ là $\heva{&x=2+t \\ &y=3-2 t \\& z=1+t}$. Ta có $M'=\Delta \cap(\alpha)$.\\
		Xét phương trình $2+t-2(3-2t)+1+t=0 \Leftrightarrow t=\dfrac{1}{2}$.\\
		Vậy $M'\left(\dfrac{5}{2};2;\dfrac{3}{2}\right)$.
	}
\end{ex}

\begin{ex}%[2H5V1-3]
		Trong không gian với hệ trục tọa độ $Oxyz$, điểm $M'$ đối xứng với điểm $M(1;2;4)$ qua mặt phẳng $(\alpha)\colon 2x+y+2z-3=0$ có tọa độ là
	\choice
		{\True $(-3;0;0)$}
		{$(-1;1;2)$}
		{$(-1 ;-2 ;-4)$}
		{$(2;1;2)$}
	\loigiai{
		Mặt phẳng $(\alpha)$ có vectơ pháp tuyến là $\vec{n}=(2;1;2)$.\\
		Vì $MM'$ vuông góc với mặt phẳng $(\alpha)$ nên đường thẳng $MM'$ nhận $\vec{n}=(2;1;2)$ làm vectơ chỉ phương.\\
		Lúc đó  đường thẳng $MM'$ có 	phương trình là $\heva{&x=1+2t \\& y=2+t \\ &z=4+2t.}$\\
		Gọi $H$ là giao điểm của đường thẳng $MM'$ và mặt phẳng $(\alpha)$.\\
		Lúc đó vì $H \in MM'$ nên  $H(1+2t;2+t;4+2t)$.\\ Mặt khác $H \in(\alpha)$ nên $2(1+2t)+2+t+2(4+2 t)-3=0 \Leftrightarrow 9t+9=0 \Leftrightarrow t=-1$.\\
		Vậy $H(-1;1;2)$.\\
		$M'$ đối xứng với điểm $M$ qua mặt phẳng $(\alpha)$ nên $H$ là trung điểm của $MM'$.\\
		Suy ra $M'(-3;0;0)$.
	}
\end{ex}
	
\begin{ex}%[2H5V2-3]
		Trong không gian với hệ trục tọa độ $Oxyz$, cho mặt phẳng $(P)\colon x+y+z-3=0$ và đường thẳng $d\colon \dfrac{x}{1}=\dfrac{y+1}{2}=\dfrac{z-2}{-1}$. Đường thẳng $d'$ đối xứng với $d$ qua mặt phẳng $(P)$ có phương trình là
	\choice 
		{\True $\dfrac{x-1}{1}=\dfrac{y-1}{-2}=\dfrac{z-1}{7}$}
		{$\dfrac{x-1}{1}=\dfrac{y-1}{2}=\dfrac{z-1}{7}$}
		{$\dfrac{x+1}{1}=\dfrac{y+1}{2}=\dfrac{z+1}{7}$}
		{$\dfrac{x+1}{1}=\dfrac{y+1}{-2}=\dfrac{z+1}{7}$}
	\loigiai{
		Ta có $d$ không vuông góc với $(P)$. Phương trình tham số của đường thẳng $d\colon\heva{&x=t \\&y=-1+2t\\&z=2-t.}$\\
		Tọa độ giao điểm $I$ của $d$ và mặt phẳng $(P)$ là nghiệm của hệ phương trình $$\heva{&x=t \\&y=-1+2 t \\&z=2-t \\ &x+y+z-3=0} \Rightarrow\heva{&x=1 \\& y=1 \\ &z=1} \Rightarrow I(1;1;1).$$
		Lấy điểm $M(0;-1;2) \in d$.\\ Đường thẳng $\Delta$ qua $M$ và vuông góc với $(P)$ có phương trình
		$\heva{&x=t \\&y=-1+t \\&z=2+t.}$\\
		Ta có $\Delta \cap(P)=H \Rightarrow H\left(\dfrac{2}{3};-\dfrac{1}{3}; \dfrac{8}{3}\right)$.\\ Vì $M'$ đối xứng với $M$ qua $(P)$ nên $H$ là trung điểm của $MM'$. Suy ra $M'\left(\dfrac{4}{3};\dfrac{1}{3};\dfrac{10}{3}\right)$.\\
		Đường thẳng $d'$ đối xứng với $d$ qua mặt phẳng $(P)$ suy ra $d'$ đi qua $I(1;1;1)$ và $M'\left(\dfrac{4}{3};\dfrac{1}{3};\dfrac{10}{3}\right)$ có vectơ chỉ phương $\vec{IM'} =\left(\dfrac{1}{3};-\dfrac{2}{3};\dfrac{7}{3}\right)=\dfrac{1}{3}(1;-2;7)$.\\
		Phương trình $d'$ là $\dfrac{x-1}{1}=\dfrac{y-1}{-2}=\dfrac{z-1}{7}$.
	}
\end{ex}

\begin{ex}%[2H5V2-3]
		Trong không gian với hệ trục tọa độ $Oxyz$, cho mặt phẳng $(P)\colon x+y+z-3=0$ và đường thẳng $d\colon \dfrac{x}{1}=\dfrac{y+1}{2}=\dfrac{z-2}{-1}$. Hình chiếu vuông góc của $d$ trên $(P)$ có phương trình là
	\choice 			
		{$\dfrac{x+1}{-1}=\dfrac{y+1}{-4}=\dfrac{z+1}{5}$}
		{$\dfrac{x-1}{3}=\dfrac{y-1}{-2}=\dfrac{z-1}{-1}$}
		{\True $\dfrac{x-1}{1}=\dfrac{y-1}{4}=\dfrac{z-1}{-5}$}
		{$\dfrac{x-1}{1}=\dfrac{y+4}{1}=\dfrac{z+5}{1}$}
	\loigiai{
		\begin{itemize}
			\item \textbf{Cách 1:} \\
			Đường thẳng $d$ đi qua điểm $M(0;-1;2)$ và có một vectơ chỉ phương là $\vec{u}_{d}~=~(1;2;-1)$.\\
			Gọi $(Q)$ là mặt phẳng chứa $d$ và vuông góc với $(P)$. Lúc đó $(Q)$ đi qua điểm $M(0;-1;2)$ và có một vectơ pháp tuyến là $\vec{n}_{Q}=\left[\vec{u}_d, \overrightarrow{n}_P\right]=(3;-2;-1)$.\\
			Suy ra $(Q)$ có phương trình là $3x-2y-z=0$.\\
			Gọi $\Delta$ là hình chiếu vuông góc của $d$ trên $(P)$, khi đó tập hợp các điểm thuộc $\Delta$ là nghiệm của hệ phương trình 
			$$\heva{&3x-2y-z=0 \\& x+y+z-3=0.}\quad (I)$$
			Trong hệ $(I)$ cho $z=1$, ta được $x=1$, $y=1$. Vậy điểm $A(1;1;1)$ thuộc $\Delta$.\\
			Suy ra $\Delta$ là đường thẳng đi qua điểm $A(1;1;1)$ và có một vectơ chỉ phương $\vec{u}_\Delta~=~\left[\vec{n}_P, \vec{n}_Q\right]~=~(1;4;-5)$.\\
			Vậy $\Delta$ có phương trình chính tắc là $\dfrac{x-1}{1}=\dfrac{y-1}{4}=\dfrac{z-1}{-5}$.
			\item \textbf{Cách 2:} \\
			Gọi $A=d \cap (P)$. Vì $A \in d$ nên $A(t;-1+2t;2-t)$.
			\\ Vì $A \in(P)$ nên $t+(-1+2t)+(2-t)-3=0 \Rightarrow 2 t-2=0 \Rightarrow t=1$. Vậy $A(1;1;1)$.\\ Lấy điểm $M(0;-1;2) \in d$. Gọi $\Delta$ là đường thẳng đi qua $M$ và vuông góc với $(P)$. Khi đó $\Delta$ có phương trình tham số là $\heva{&x=t \\ &y=-1+t \\ &z=2+t.}$\\
			Gọi $B=\Delta \cap(P)$. Lúc đó $B \in \Delta \Rightarrow B(t;-1+t;2+t)$.\\ Vì $B \in (P) \Rightarrow t+(-1+t)+(2+t)-3=0 \Rightarrow 3 t-2=0 \Rightarrow t=\dfrac{2}{3}\Rightarrow B\left(\dfrac{2}{3};-\dfrac{1}{3};\dfrac{8}{3}\right)$.\\ Phương trình hình chiếu vuông góc của $d$ trên mặt phẳng $(P)$ là đường thẳng $AB$ đi qua điểm $A(1;1;1)$ và có một vectơ chỉ phương là $$\vec{u}=-3 \overrightarrow{AB}=-3 \cdot\left(-\dfrac{1}{3};-\dfrac{4}{3};\dfrac{5}{3}\right)=(1;4;-5).$$ 
			Vậy $\Delta$  có phương trình chính tắc là $\dfrac{x-1}{1}=\dfrac{y-1}{4}=\dfrac{z-1}{-5}$.
		\end{itemize}
	}
\end{ex}

\begin{ex}%[2H5V2-3]
	Trong không gian với hệ trục tọa độ $Oxyz$, cho đường thẳng $\Delta$ có phương trình là $ \dfrac{x}{1}=\dfrac{y-1}{2}=\dfrac{z+2}{3}$. Biết điểm $M(a;b;c)$ thuộc $\Delta$ và $M$ có tung độ âm và cách mặt phẳng $(Oyz)$ một khoảng bằng $2$. Xác định giá trị $T=a+b+c$.
	\choice
		{$T=-1$}
		{$T=11$}
		{\True $T=-13$}
		{$T=1$}
	\loigiai{
		$M \in \Delta \Rightarrow M(t; 1+2t;-2+3t)$. Theo giả thiết thì $d\left(M ;(Oyz)\right)=|t|=2 \Leftrightarrow\hoac{&t=2 \\ &t=-2.}$ \\
		Với $t=2$, tung độ $M$ là $1+2t=5>0$ (không thỏa mãn giả thiết).\\
		Với $t=-2$,  tung độ $M$ là $1+2t=-3<0$ (thỏa mãn giả thiết). Lúc đó ta có $M(-2 ;-3 ;-8)$.\\
		Vây $a=-2$, $b=-3$, $c=-8$. Suy ra $T=a+b+c=-13$.
	}
\end{ex}

\begin{ex}%[2H5V2-3]
	Trong không gian với hệ tọa độ $Oxyz$, cho hai điểm $A(1;-1;2)$, $B(-1;2;3)$ và đường thẳng $d\colon \dfrac{x-1}{1}=\dfrac{y-2}{1}=\dfrac{z-1}{2}$. Tìm điểm $M(a;b;c)$ thuộc $d$ sao cho $MA^2+MB^2=28$, biết $c<0$.
	\choice 
		{\True $M\left(\dfrac{1}{6};\dfrac{7}{6};-\dfrac{2}{3}\right)$}
		{$M\left(-\dfrac{1}{6};-\dfrac{7}{6};-\dfrac{2}{3}\right)$}
		{$M(-1;0;-3)$}
		{$M(2;3;3)$}
	\loigiai{
		Ta có $M \in d$ nên $\exists t \in \mathbb{R}$ sao cho $ M(1+t;2+t;1+2t)$.\\
		Do $c=1+2t<0$ suy ra $t<-\dfrac{1}{2}$.\\
		Ta có
		\begin{eqnarray*}
			MA^2+MB^2=28 			& \Leftrightarrow &(-t)^2+(-3-t)^2+(1-2 t)^2+(-2-t)^2+(-t)^2+(2-2 t)^2=28 \\
			& \Leftrightarrow & 12 t^2-2 t-10=0 \Leftrightarrow\hoac{& t=1\text{ (loại)}\\ &
				t=-\dfrac{5}{6}\text{ (thỏa mãn)}.}
		\end{eqnarray*}
		Với $t=-\dfrac{5}{6}$, ta có $M\left(\dfrac{1}{6};\dfrac{7}{6};-\dfrac{2}{3}\right)$.
	}
\end{ex}






	