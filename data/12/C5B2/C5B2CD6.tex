% \chude{ĐƯỜNG THẲNG LIÊN QUAN ĐẾN GÓC VÀ KHOẢNG CÁCH}
\begin{dang}{LẬP PHƯƠNG TRÌNH MẶT PHẲNG LIÊN QUAN ĐẾN GÓC}
\end{dang}
\TN
\Opensolutionfile{ans}[ans/C5B4CD6-D1]
\begin{ex}%Câu 1%[2H5H2-7]
Trong không gian $Oxyz$, cho điểm $A(-2;0;1)$, đường thẳng $d$ qua điểm $A$ và tạo với trục $Oy$ góc $45^\circ$. Phương trình đường thẳng $d$ là
\choice
{\True $\hoac{&\dfrac{x+2}{2}=\dfrac{y}{\sqrt{5}}=\dfrac{z-1}{-1}\\
&\dfrac{x+2}{2}=\dfrac{y}{-\sqrt{5}}=\dfrac{z-1}{-1}}$}
{$\hoac{&\dfrac{x-2}{2}=\dfrac{y}{\sqrt{5}}=\dfrac{z+1}{-1}\\
&\dfrac{x-2}{2}=\dfrac{y}{-\sqrt{5}}=\dfrac{z+1}{-1}}$}
{$\hoac{&\dfrac{x+2}{2}=\dfrac{y}{\sqrt{5}}=\dfrac{z-1}{-1}\\
&\dfrac{x-2}{2}=\dfrac{y}{\sqrt{5}}=\dfrac{z+1}{-1}}$}
{$\hoac{&\dfrac{x+2}{2}=\dfrac{y}{-\sqrt{5}}=\dfrac{z-1}{-1}\\
&\dfrac{x-2}{2}=\dfrac{y}{\sqrt{5}}=\dfrac{z+1}{-1}}$}
\loigiai{
\begin{itemize}
\item Cách $1$: Điểm $M(0;m;0)\in Oy$, $\overrightarrow{j}=(0;1;0)$ là véc-tơ chỉ phương của trục $Oy$.\\ $\overrightarrow{AM}=(2;-m;-1)\Rightarrow\left|\cos\left(\overrightarrow{AM},\overrightarrow{j}\right)\right|=\cos{45^\circ}\Leftrightarrow\dfrac{\left| m\right|}{\sqrt{m^2+5}}=\dfrac{1}{\sqrt{2}}\Leftrightarrow m=\pm\sqrt{5}$ \break nên có 2 đường thẳng
$\dfrac{x+2}{2}=\dfrac{y}{\sqrt{5}}=\dfrac{z-1}{-1}$ và $\dfrac{x+2}{2}=\dfrac{y}{-\sqrt{5}}=\dfrac{z-1}{-1}$.
\item Cách $2$: $\overrightarrow{u_1}=\left(2;\sqrt{5};-1\right)\Rightarrow\left|\cos\left(\overrightarrow{u_1},\overrightarrow{j}\right)\right|=\dfrac{1}{\sqrt{2}}$;\\ $\overrightarrow{u_2}=\left(2;-\sqrt{5};-1\right)\Rightarrow\left|\cos\left(\overrightarrow{u_2},\overrightarrow{j}\right)\right|=\dfrac{1}{\sqrt{2}}$.\\
Đường thẳng $d$ đi qua điểm $A(-2;0;1)$ nên đường thẳng $d$ có phương trình là \\
$$\dfrac{x+2}{2}=\dfrac{y}{\sqrt{5}}=\dfrac{z-1}{-1}\text{ hoặc }
\dfrac{x+2}{2}=\dfrac{y}{-\sqrt{5}}=\dfrac{z-1}{-1}.$$
\end{itemize}
}
\end{ex}

\begin{ex}%Câu 2%[2H5V2-7]
Trong không gian với hệ tọa độ $Oxyz$, cho mặt phẳng $(P)\colon 4x-7y+z+25=0$ và đường thẳng $d_1\colon\dfrac{x+1}{1}=\dfrac{y}{2}=\dfrac{z-1}{-1}$. Gọi $d_1'$ là hình chiếu vuông góc của $d_1$ lên mặt phẳng $(P)$. Đường thẳng $d_2$ nằm trong $(P)$ tạo với $d_1$, $d_1'$ các góc bằng nhau, $d_2$ có véc-tơ chỉ phương $\overrightarrow{u}_2=(a;b;c)$. Tính $\dfrac{a+2b}{c}$.
\choice
{$\dfrac{a+2b}{c}=\dfrac{2}{3}$}
{$\dfrac{a+2b}{c}=0$}
{$\dfrac{a+2b}{c}=\dfrac{1}{3}$}
{\True $\dfrac{a+2b}{c}=1$}
\loigiai{
Véc-tơ chỉ phương của $d_1$ là $\overrightarrow{u}_1=(1;2;-1)$, véc-tơ pháp tuyến của $(P)$ là $\overrightarrow{n}_P=(4;-7;1)$.
\begin{itemize}
\item Cách $1$: Gọi $(Q)=(d_1,d_1')$ khi đó $(Q)$ có véc-tơ pháp tuyến $\overrightarrow{n}_Q=\left[\overrightarrow{n}_P,\overrightarrow{u}_1\right]=(5;5;15)$ .\\
Đường thẳng $d_1'$ có véc-tơ chỉ phương $\overrightarrow{u}'_1=\left[\overrightarrow{n}_P,\overrightarrow{u}_1\right]=(22;11;-11)$ hay một véc-tơ chỉ phương khác $\overrightarrow{u}=(2;1;-1)$ .\\
Vì $\overrightarrow{n}_P\cdot\overrightarrow{u}_2=0\Rightarrow 4a-7b+c=0\Rightarrow c=7b-4a\Rightarrow\overrightarrow{u}_2=(a;b;7b-4a)$.\\
Ta lại có
$$\begin{aligned}
\left(d_1;d_2\right)=\left(d_1';d_2\right)&\Leftrightarrow\left|\cos\left(\overrightarrow{u}_1,\overrightarrow{u_2}\right)\right|=\left|\cos\left(\overrightarrow{u}_1',\overrightarrow{u}_2\right)\right|\\&\Leftrightarrow\left| a+2b+4a-7b\right|=\left| 2a+b+4a-7b\right|\\&\Leftrightarrow\left| 5a-5b\right|=\left| 6a-6b\right|\\&\Leftrightarrow\left| a-b\right|=0\Leftrightarrow a=b.
\end{aligned}$$
Chọn $a=1\Rightarrow b=1$, $c=3\Rightarrow\dfrac{a+2b}{c}=1$.
\item Cách $2$: Gọi $(Q)=\left(d_1,d_1'\right)$, khi đó $(P)\perp(Q)$.\\
Các đường thẳng nằm trong $(P)$ mà vuông góc với $(Q)$ thì vuông góc với tất cả các đường thẳng trong $(Q)$ hay chúng cùng tạo với $d_1,d_1'$ các góc $90^{\circ}$.\\ Do đó, các đường thẳng này thỏa mãn yêu cầu đề bài và có véc-tơ chỉ phương $\overrightarrow{u}=\overrightarrow{n}_Q=(1;1;3)\Rightarrow\dfrac{a+2b}{c}=1$.
\end{itemize}
}
\end{ex}
\begin{ex}%[Câu 1]%[2H5H2-7]
	Trong không gian với hệ tọa độ $Oxyz$, cho hai đường thẳng $d_1\colon \dfrac{x-1}{2}=\dfrac{y-2}{-2}=\dfrac{z+1}{-1}$,  $d_2\colon \heva{&x=t \\&y=0 \\&z=-t}$. Mặt phẳng $(P)$ qua $d_1$ tạo với $d_2$ một góc $45^\circ$ và nhận véc-tơ $\vec{n}=(1;b;c)$ làm một véc-tơ pháp tuyến. Xác định tích $b\cdot c$.
	\choice
	{$-4$ hoặc $0$}
	{$4$ hoặc $0$}
	{\True $-4$}
	{$4$}
	\loigiai{
		Ta có véc-tơ chỉ phương của $d_1$, $d_2$ lần lượt là $\overrightarrow{u}_1=(2;-2;-1)$ và $\overrightarrow{u}_2=(1;0;-1)$.\\
		Mặt phẳng $(P)$ qua $d_1$ nên $ \overrightarrow{n}\cdot\overrightarrow{u}_1=0\Leftrightarrow 2-2b-c=0$. \qquad (1)\\
		Ta có \begin{eqnarray*}
			&&\sin (d_2,(P))=\dfrac{\left|\overrightarrow{u}_2\cdot\overrightarrow{n}\right|}{\left|\overrightarrow{u}_2\right|\cdot\left|\overrightarrow{n}\right|}=\sin 45^\circ \\
			&\Leftrightarrow&\dfrac{\left| 1-c\right|}{\sqrt{b^2+c^2+1}\cdot\sqrt{2}}=\dfrac{\sqrt{2}}{2}\\
			&\Leftrightarrow&\left| 1-c\right|=\sqrt{b^2+c^2+1}\\
			&\Leftrightarrow&b^2+2c=0.\qquad (2)
		\end{eqnarray*}
		Từ $(1)$ và $(2)$ suy ra $\heva{&b=2 \\&c=-2}\Rightarrow b\cdot c=-4$.}
\end{ex}
\begin{ex}%[Câu 3]%[2H5H2-7]
	Trong không gian với hệ tọa độ $Oxyz$, cho đường thẳng $d\colon \heva{&x=0 \\&y=3-t \\&z=t}$. Gọi $(P)$ là mặt phẳng chứa đường thẳng $d$ và tạo với mặt phẳng $(Oxy)$ một góc $45^\circ$. Điểm nào sau đây thuộc mặt phẳng $(P)$?
	\choice
	{\True $M(3;2;1)$}
	{$N(3;2;-1)$}
	{$P(3;-1;2)$}
	{$M(3;-1;-2)$}
	\loigiai{
		Ta viết phương trình đường thẳng $d\colon \heva{&x=0 \\&y+z-3=0.}$\\
		Mặt phẳng $(P)$ chứa đường thẳng $d$ nên có dạng $mx+n(y+z-3)=0$, $m^2+n^2\ne 0$ hay $mx+ny+nz-3n=0$ nên $(P)$ có một véc-tơ pháp tuyến là $\overrightarrow{n_P}=(m;n;n)$.\\
		Mặt phẳng $(Oxy)$ có một véc-tơ pháp tuyến là $\overrightarrow{k}=(0;0;1)$.\\
		Ta có 
		\begin{eqnarray*}
			&&\cos ((P);(Oxy))=\left| \cos (\overrightarrow{n_P};\overrightarrow{k})\right|\\
			&\Leftrightarrow&\cos 45^\circ =\dfrac{\left| \overrightarrow{n_P}.\overrightarrow{k}\right|}{\left| \overrightarrow{n_P}\right|.\left| \overrightarrow{k}\right|}\\
			&\Leftrightarrow&\dfrac{1}{\sqrt{2}}=\dfrac{\left| n\right|}{\sqrt{m^2+n^2+n^2}}\\
			&\Leftrightarrow&\sqrt{m^2+2n^2}=\sqrt{2}\left| n\right|\\
			&\Leftrightarrow&m^2=0\Leftrightarrow m=0.
		\end{eqnarray*}
		Chọn $n=1\Rightarrow (P)\colon y+z-3=0$.\\
		Do đó $M(3;2;1)\in (P)$.\\
		\textbf{Bình luận:} Đối với những bài toán viết phương trình mặt phẳng chứa đường thẳng cho trước ta nên sử dụng khái niệm chùm mặt phẳng như sau: Mặt phẳng $(\alpha)$ qua giao tuyến của hai mặt phẳng $(P)\colon a_1x+b_1y+c_1z+d_1=0$ và $(Q)\colon a_2x+b_2y+c_2z+d_2=0$ có phương trình dạng $m(a_1x+b_1y+c_1z+d_1)+n(a_2x+b_2y+c_2z+d_2)=0$, $m^2+n^2\ne 0$.}
\end{ex}
\begin{ex}%[Câu 4]%[2H5H2-7]
	Trong không gian $Oxyz$, cho tam giác $ABC$ vuông tại $A$, $\widehat{ABC}=30^\circ$, $BC=3\sqrt{2}$, đường thẳng $BC$ có phương trình $\dfrac{x-4}{1}=\dfrac{y-5}{1}=\dfrac{z+7}{-4}$, đường thẳng $AB$ nằm trong mặt phẳng $(\alpha)\colon x+z-3=0$. Biết đỉnh $C$ có cao độ âm. Tính hoành độ đỉnh $A$.
	\choice
	{$\dfrac{3}{2}$}
	{$3$}
	{\True $\dfrac{9}{2}$}
	{$\dfrac{5}{2}$}
	\loigiai{
		Vì $C\in BC$ nên $C(4+t;5+t;-7-4t)$.\\
		$BC$ có véc tơ chỉ phương $\overrightarrow{u}=(1;1;-4)$. Mặt phẳng $(\alpha)$ có véc-tơ pháp tuyến $\overrightarrow{n}=(1;0;1)$.\\
		Gọi $\varphi $ là góc giữa $BC$ và $(\alpha)$. Ta có $\sin \varphi =\left| \cos (\overrightarrow{u};\overrightarrow{n})\right|=\dfrac{1}{2}\Rightarrow \varphi =30^\circ$. Tức là $A$ là hình chiếu của $C$ lên $(\alpha)$.\\
		Vậy 
		\begin{eqnarray*}
			&&\dfrac{3\sqrt{2}}{2}=CA=\mathrm{d}(C;(\alpha))=\dfrac{\left| 4+t-7-4t-3\right|}{\sqrt{2}}\\
			&\Leftrightarrow&\hoac{&t=-1 \\&t=-3}\\
			&\Leftrightarrow&\hoac{&C(3;4;-3) \\&C(1;2;5).}
		\end{eqnarray*}
		Mà $C$ có cao độ âm, suy ra $C(3;4;-3)$.\\
		Lúc này $AC$ qua $C(3;4;-3)$ và có véc-tơ chỉ phương $\overrightarrow{n}=(1;0;1)$.\\
		Phương trình $AC$ là $\heva{&x=3+t\\&y=4\\&z=-3+t}$. Vì $A\in AC$ nên $A(3+t;4;-3+t)$.\\
		Mặt khác $A$ nằm trong mặt phẳng $(\alpha)\colon x+z-3=0\Rightarrow t=\dfrac{3}{2}$.\\ 
		Do đó, hoành độ đỉnh $A$ là $ x_{A}=\dfrac{9}{2}$.}
\end{ex}
\begin{ex}%[Câu 5]%[2H5H2-7]
	Trong không gian với hệ tọa độ $Oxyz$, mặt phẳng nào dưới đây đi qua $A(2; 1; - 1)$ tạo với trục $Oz$ một góc $30^\circ $?
	\choice
	{\True $\sqrt{2}(x-2)+(y-1)-(z-2)-3=0$}
	{$(x-2)+\sqrt{2}(y-1)-(z+1)-2=0$}
	{$2(x-2)+(y-1)-(z-2)=0$}
	{$2(x-2)+(y-1)-(z-1)-2=0$}
	\loigiai{
		Gọi phương trình mặt phẳng $(\alpha)$ có dạng $A(x-2)+B(y-1)+C(z+1)=0$, $\overrightarrow{n}=(A;B;C)$ là véc-tơ pháp tuyến.\\
		Ta có $Oz$ có véc-tơ chỉ phương là $\overrightarrow{k}=(0;0;1)$.\\
		Áp dụng công thức 
		\begin{eqnarray*}
			&&\sin ((\alpha),Oz)=\dfrac{\left| \overrightarrow{n}\cdot \overrightarrow{k}\right|}{\overrightarrow{\left| n\right|}\cdot \overrightarrow{\left| k\right|}}=\sin 30^\circ\\
			&\Leftrightarrow&\dfrac{|A\cdot 0+B\cdot 0+C\cdot 1|}{\sqrt{A^2+B^2+C^2}\cdot \sqrt{0^2+0^2+1^2}}=\dfrac{1}{2}\\
			&\Leftrightarrow&\dfrac{|C|}{\sqrt{A^2+B^2+C^2}}=\dfrac{1}{2}\\
			&\Leftrightarrow&3C^2=A^2+B^2.\qquad (1)
		\end{eqnarray*}
		Chọn $A=\sqrt{2}$, $B=1$, $C=-1$ thỏa mãn $(1)$. Khi đó $(\alpha)\colon \sqrt{2}(x-2)+(y-1)-(z+1)=0$ hay $(\alpha)\colon \sqrt{2}(x-2)+(y-1)-(z-2)-3=0$.
	}
\end{ex}
\begin{ex}%[Câu 6]%[2H5H2-7]
	Cho mặt phẳng $(\alpha)\colon 3x-2y+2z-5=0$ và điểm $A(1; - 2; 2)$. Có bao nhiêu mặt phẳng đi qua $A$ và tạo với mặt phẳng $(\alpha)$ một góc $45^\circ$.
	\choice
	{\True Vô số}
	{$1$}
	{$2$}
	{$4$}
	\loigiai{
		Gọi $\overrightarrow{n_{\beta}}=(a;b;c)$ là véc-tơ pháp tuyến của mặt phẳng $(\beta)$ cần lập. Ta có
		\begin{eqnarray*}
			&&\cos((\alpha),(\beta))=\left| \cos(\overrightarrow{n_{\alpha}},\overrightarrow{n_{\beta}})\right|\\
			&\Leftrightarrow&\dfrac{\left| \overrightarrow{n_{\alpha}}\cdot \overrightarrow{n_{\beta}}\right|}{\left| \overrightarrow{n_{\alpha}}\right|\cdot \left| \overrightarrow{n_{\beta}}\right|}=\dfrac{\left| 3\cdot a-2\cdot b+2\cdot c\right|}{\sqrt{3^2+(-2)^2+2^2}\cdot \sqrt{a^2+b^2+c^2}}=\dfrac{\sqrt{2}}{2}\\
			&\Leftrightarrow&2(3a-2b+2c)^2=17(a^2+b^2+c^2)\\
			&\Leftrightarrow&2a^2-9b^2-9c^2-24ab-16bc+24ac=0.
		\end{eqnarray*}
		Phương trình trên có vô số nghiệm. Nên có vô số véc-tơ $\overrightarrow{n_{\beta}}=(a;b;c)$ là véc-tơ pháp tuyến của $(\beta)$.\\
		Suy ra có vô số mặt phẳng $(\beta)$ thỏa mãn điều kiện bài toán.\\
	}
\end{ex}
\Closesolutionfile{ans}
\indapan{7}{ans/C5B4CD6-D1}
\TNSA
\Opensolutionfile{ans}[ans/C5B4CD6-D1-KQ]
\begin{ex}%[Câu 7]%[2H5H2-7]
	Số các mặt phẳng $(\alpha)$ chứa đường thẳng $d\colon\dfrac{x}{1}=\dfrac{y}{-1}=\dfrac{z}{-3}$ và tạo với mặt phẳng $(P)\colon 2x-z+1=0$ góc $45^\circ $ bằng
	\shortans{$2$}
	\loigiai{
		Đường thẳng $d$ đi qua điểm $O(0;0;0)$ có véc-tơ chỉ phương $\overrightarrow{u}=(1;-1;-3)$.\\
		Ta có $(\alpha)$ qua $O$ có véc-tơ pháp tuyến $\overrightarrow{n}=(a;b;c)$ có dạng $ax+by+cz=0$.\\
		Vì $\overrightarrow{n}\perp \overrightarrow{u}$ nên $\overrightarrow{n}\cdot \overrightarrow{u}=0$. Do đó $ a-b-3c=0$.\\
		Mặt phẳng $(P)\colon 2x-z+1=0$ có véc-tơ pháp tuyến $\overrightarrow{k}=(2;0;-1)$.\\
		Ta có 
		\begin{eqnarray*}
			&&\cos 45^\circ =\dfrac{\left| \overrightarrow{n}\cdot \overrightarrow{k}\right|}{\left|\overrightarrow{n}\right|\cdot\left|\overrightarrow{k}\right|}\\
			&\Leftrightarrow&\dfrac{\left| 2a-c\right|}{\sqrt{5(a^2+b^2+c^2)}}=\dfrac{\sqrt{2}}{2}\\
			&\Leftrightarrow&10(a^2+b^2+c^2)=(4a-2c)^2\\
			&\Leftrightarrow&10(b^2+6bc+9c^2+b^2+c^2)=(4b+12c-2c)^2\\
			&\Leftrightarrow&10(2b^2+6bc+10c^2)=(4b+10c)^2\\
			&\Leftrightarrow&4b^2-20bc=0\\
			&\Leftrightarrow&\hoac{&b=0 \\&b=5c.}
		\end{eqnarray*}
		Xét 
		\begin{itemize}
			\item $b=0\Rightarrow a=3c$ nên $(\alpha)\colon x+3z=0$.
			\item $b=5c$, chọn $c=1\Rightarrow b=5$, $a=8$ nên $(\alpha)\colon 8x+5y+z=0$.
		\end{itemize}
	}
\end{ex}
\begin{ex}%[Câu 9]%[2H5H2-7]
	Trong không gian với hệ tọa độ $Oxyz$, cho điểm $A(3;-1;0)$ và đường thẳng $d\colon \dfrac{x-2}{-1}=\dfrac{y+1}{2}=\dfrac{z-1}{1}$. Phương trình mặt phẳng $(\alpha)$ chứa $d$ sao cho khoảng cách từ $A$ đến $(\alpha)$ lớn nhất có dạng $ax+by+cz=0$. Khi đó $\dfrac{a}{b}$ bằng
	\shortans{$1$}
	\loigiai{
		Gọi $H$ là hình chiếu của $A$ lên $d$.\\
		Khi đó $H(2-t;-1+2t;1+t)\Rightarrow \overrightarrow{AH}=(-1-t;2t;1+t)$.\\
		Do $AH\perp d$ nên $ -(-1-t)+2\cdot 2t+1+t=0\Leftrightarrow t=-\dfrac{1}{3}$. Khi đó $\overrightarrow{AH}=\left(-\dfrac{2}{3};-\dfrac{2}{3};\dfrac{2}{3}\right)$.\\
		Mặt phẳng $(\alpha)$ chứa $d$ sao cho khoảng cách từ $A$ đến $(\alpha)$ lớn nhất khi $AH\perp (\alpha)$.\\
		Do đó $(\alpha)$ có véc-tơ pháp tuyến là $\overrightarrow{n}=(1;1;-1)$.\\
		Vậy $(\alpha)\colon 1(x-2)+1(y+1)-1(z-1)=0\Leftrightarrow x+y-z=0$.\\ Do đó $a=1$, $b=1$, $c=-1$ và $\dfrac{a}{b}=1$.}
\end{ex}
\begin{ex}%[Câu 10]%[2H5H1-6]
	Trong không gian với hệ tọa độ $Oxyz$, cho hai mặt phẳng $(P)\colon x+2y-2z+1=0,$ $(Q)\colon x+my+(m-1)z+2024=0$. Khi hai mặt phẳng $(P)$, $(Q)$ tạo với nhau một góc nhỏ nhất thì giá trị của $m$ bằng bao nhiêu?
	\shortans{$0{,}5$}
	\loigiai{
		Gọi $\varphi $ là góc giữa hai mặt phẳng $(P)$ và $(Q)$.\\
		Khi đó
		\begin{eqnarray*}
			&&\cos \varphi =\dfrac{\left| 1\cdot 1+2\cdot m-2\cdot (m-1)\right|}{\sqrt{1^2+2^2+(-2)^2}\cdot \sqrt{1^2+m^2+(m-1)^2}}\\
			&\Leftrightarrow&\cos \varphi =\dfrac{3}{3\sqrt{2m^2-2m+2}}=\dfrac{1}{\sqrt{2(m-\dfrac{1}{2})^2+\dfrac{3}{2}}}\\
			&\Leftrightarrow&\cos \varphi \le \dfrac{1}{\sqrt{\dfrac{3}{2}}}.
		\end{eqnarray*}
		Góc $\varphi $ nhỏ nhất khi và chỉ khi $\cos \varphi $ lớn nhất $\Leftrightarrow m=\dfrac{1}{2}=0{,}5$.}
\end{ex}
\begin{ex}%[Câu 11]%[2H5H1-6]
	Cho hai điểm $A(1;-1;1);B(2;-2;4)$. Có bao nhiêu mặt phẳng chứa $A$, $B$ và tạo với mặt phẳng $(\alpha)\colon x-2y+z-7=0$ một góc $60^\circ $?
	\shortans{$2$}
	\loigiai{
		Ta có $\overrightarrow{AB}=(1;-1;3)$, $\overrightarrow{n_{\alpha}}=(1;-2;1)$.
		Gọi $\overrightarrow{n_{\beta}}=(a;b;c)$ là véc-tơ pháp tuyến của mặt phẳng $(\beta)$ cần lập. Ta có
		\begin{eqnarray*}
			&&\cos((\alpha),(\beta))=\left| \cos(\overrightarrow{n_{\alpha}},\overrightarrow{n_{\beta}})\right|=\dfrac{\left| \overrightarrow{n_{\alpha}}\cdot \overrightarrow{n_{\beta}}\right|}{\left| \overrightarrow{n_{\alpha}}\right|\cdot \left| \overrightarrow{n_{\beta}}\right|}\\
			&\Leftrightarrow&\dfrac{\left| 1\cdot a-2\cdot b+1\cdot c\right|}{\sqrt{1^2+(-2)^2+1^2}\cdot \sqrt{a^2+b^2+c^2}}=\dfrac{1}{2}\\
			&\Leftrightarrow&2(a-2b+c)^2=3(a^2+b^2+c^2).\qquad (1)
		\end{eqnarray*}
		Mặt khác vì mặt phẳng $(\beta)$ chứa $A$, $B$ nên 
		$$\overrightarrow{n_{\beta}}\cdot \overrightarrow{AB}=0\Leftrightarrow a-b+3c=0\Leftrightarrow a=b-3c.$$
		Thế vào $(1)$ ta được $2b^2-13bc+11c^2=0$ \qquad $(2)$.\\
		Phương trình $(2)$ có $2$ nghiệm phân biệt. Suy ra có $2$ véc-tơ $\overrightarrow{n_{\beta}}=(a;b;c)$ thỏa mãn.\\
		Suy ra có $2$ mặt phẳng.}
\end{ex}
\begin{ex}%[Câu 12]%[2H5V1-6]
	Trong không gian $Oxyz$, cho hai điểm $A(3;0;1)$, $B(6;-2;1)$. Phương trình mặt phẳng $(P)$ đi qua $A$, $B$ và tạo với mặt phẳng $(Oyz)$ một góc $\alpha $ thỏa mãn $\cos \alpha =\dfrac{2}{7}$ có dạng $ax+by+cz+d=0$ với $d\neq 0$. Khi đó $\dfrac{d}{a}$ bằng
	\shortans{$-6$}
	\loigiai{
		Giả sử $(P)$ có véc-tơ pháp tuyến $\overrightarrow{n_1}=(a;b;c)$, $(P)$ có véc-tơ chỉ phương $\overrightarrow{AB}=(3;-2;0)$. \\
		Suy ra \[ \overrightarrow{n_1}\perp \overrightarrow{AB}\Rightarrow \overrightarrow{n_1}\cdot \overrightarrow{AB}=0\Leftrightarrow 3a+b(-2)+0\cdot c=0\Rightarrow 3a-2b=0\Rightarrow a=\dfrac{2}{3}b.\qquad (1)\]
		$(Oyz)$ có phương trình $x=0$ nên có véc-tơ pháp tuyến $\overrightarrow{n_2}=(1;0;0)$. Mà 
		\begin{eqnarray*}
			&&\cos \alpha =\dfrac{2}{7}\\
			&\Leftrightarrow&\dfrac{\left| \overrightarrow{n_1}\cdot \overrightarrow{n_2}\right|}{\left| \overrightarrow{n_1}\right|\cdot \left| \overrightarrow{n_2}\right|}=\dfrac{2}{7}\\
			&\Leftrightarrow&\dfrac{\left| a\cdot 1+b\cdot 0+c\cdot 0\right|}{\sqrt{a^2+b^2+c^2}\cdot \sqrt{1^2+0^2+0^2}}=\dfrac{2}{7}\\
			&\Leftrightarrow&\dfrac{\left| a\right|}{\sqrt{a^2+b^2+c^2}}=\dfrac{2}{7}\\
			&\Leftrightarrow&7\left| a\right|=2\sqrt{a^2+b^2+c^2}\\
			&\Leftrightarrow&45a^2-4b^2-4c^2=0.\qquad (2)
		\end{eqnarray*}
		Thay $(1)$ vào $(2)$ ta được $4b^2-c^2=0$.\\
		Chọn $c=2$ ta có $4b^2-2^2=0\Rightarrow \hoac{&b=1 \\&b=-1}\Rightarrow \hoac{&a=\dfrac{2}{3} \\&a=-\dfrac{2}{3}}$
		\begin{itemize}
			\item $a=\dfrac{2}{3}$ thì $\overrightarrow{n}=\left(\dfrac{2}{3};1;2\right)$ hay $\overrightarrow{n}=(2;3;6)$. Do đó $(P)\colon 2x+3y-6z=0$.
			\item $a=-\dfrac{2}{3}$ thì $\overrightarrow{n}=\left(-\dfrac{2}{3};-1;2\right)$ hay $\overrightarrow{n}=(2;3;-6)$. Do đó $(P)\colon 2x+3y+6z-12=0$.
		\end{itemize}
		Vậy $(P)\colon 2x+3y-6z=0$ hoặc $2x+3y+6z-12=0$.\\
		Vì $(P)$ có dạng $ax+by+cz+d=0$, $d\neq 0$ nên $(P)\colon 2x+3y+6z-12=0$ và $a=2$, $d=-12$. Do đó $\dfrac{d}{a}=-6$.}
\end{ex}
\begin{ex}%[Câu 13]%[2H5V1-6]
	Trong không gian với hệ tọa độ $Oxyz$, biết mặt phẳng $(P)\colon ax+by+cz+d=0$ với $c<0$ đi qua hai điểm $A(0;1;0)$, $B(1;0;0)$ và tạo với mặt phẳng $(yOz)$ một góc $60^\circ $. Tính giá trị $a+b+c$. (Kết quả lấy đến hàng phần chục)
	\shortans{$0{,}6$}
	\loigiai{
		Ta có $A, B\in (P)$ nên $\heva{&b+d=0 \\&a+d=0.}$ \\
		Suy ra $(P)$ có dạng $ax+ay+cz-a=0$ có véc-tơ pháp tuyến là $\overrightarrow{n}=(a;a;c)$.\\
		Mặt phẳng $(yOz)$ có véc-tơ pháp tuyến là $\overrightarrow{i}=(1;0;0)$.\\
		Ta có 
		\begin{eqnarray*}
			&&\cos 60^\circ =\dfrac{\left| \overrightarrow{n}\cdot \overrightarrow{i}\right|}{\left| \overrightarrow{n}\right|\cdot \left| \overrightarrow{i}\right|}\\
			&\Leftrightarrow&\dfrac{1}{2}=\dfrac{\left| a\right|}{\sqrt{2a^2+c^2}\cdot 1}\\
			&\Leftrightarrow&2a^2+c^2=4a^2\Leftrightarrow 2a^2-c^2=0.
		\end{eqnarray*}
		Chọn $a=1$, ta có $c^2=2\Rightarrow c=-\sqrt{2}$ do $c<0$.\\
		Ta có $a+b+c=a+a+c=1+1-\sqrt{2}=2-\sqrt{2}\approx 0{,}6$.}
\end{ex}
\Closesolutionfile{ans}
\indapan{7}{ans/C5B4CD6-D1-KQ}
\begin{dang}{Khoảng cách}
	\begin{enumerate}
		\item Khoảng cách từ một điểm đến đường thẳng
		\begin{itemize}
			\item Khoảng cách từ điểm $M$ đến một đường thẳng $d$ qua điểm $M_{0}$ có véc-tơ chỉ phương $\overrightarrow{u}_d$ được xác định bởi công thức $\mathrm{d}(M, d)=\dfrac{\left|\left[\overrightarrow{M_0 M}, \vec{u}_d\right]\right|}{\left|\vec{u}_d\right|}$.
			\item Khoảng cách giữa hai đường thẳng song song là khoảng cách từ một điểm thuộc đường thẳng này đến đường thẳng kia.
		\end{itemize}
		\item Khoảng cách giữa hai đường thẳng
		\begin{itemize}
			\item Khoảng cách giữa hai đường thẳng song song là khoảng cách từ một điểm thuộc đường thẳng này đến đường thẳng kia.
			\item Khoảng cách giữa hai đường thẳng chéo nhau: $d$ đi qua điểm $M$ và có véc-tơ chỉ phương $\overrightarrow{u}$ và $d'$ đi qua điểm $M'$ và có véc-tơ chỉ phương $\overrightarrow{u'}$ là $\mathrm{d}\left(d, d'\right)=\dfrac{\left|\left[\vec{u}, \vec{u'}\right] \cdot \overrightarrow{M'M}\right|}{\left|\left[\vec{u}, \vec{u'}\right]\right|}$.
		\end{itemize}
	\end{enumerate}
\end{dang}
\TN
\Opensolutionfile{ans}[ans/C5B4CD6-D2]
\begin{ex}%[Câu 14]%[2H5H2-6]
	Trong không gian $Oxyz$, khoảng cách từ điểm $M(2;-4;-1)$ tới đường thẳng $\Delta\colon \heva{&x=t \\&y=2-t \\&z=3+2t}$ bằng
	\choice
	{$\sqrt{14}$}
	{$\sqrt{6}$}
	{\True $2\sqrt{14}$}
	{$2\sqrt{6}$}
	\loigiai{
		Đường thẳng $\Delta $ đi qua $N(0;2;3)$, có véc-tơ chỉ phương $\overrightarrow{u}=(1;-1;2)$.\\
		Ta có $\overrightarrow{MN}=(-2;6;4); \left[\overrightarrow{MN},\overrightarrow{u}\right]=(16;8;-4)$.\\
		Do đó $\mathrm{d}(M,\Delta)=\dfrac{\left| \left[\overrightarrow{MN},\overrightarrow{u}\right]\right|}{\left| \overrightarrow{u}\right|}=\dfrac{\sqrt{336}}{\sqrt{6}}=2\sqrt{14}$. \\}
\end{ex}
\begin{ex}%[Câu 15]%[2H5H2-6]
	Trong không gian với hệ tọa độ $Oxyz$, cho đường thẳng $\mathrm{d}\colon\dfrac{x-3}{-2}=\dfrac{y}{-1}=\dfrac{z-1}{1}$ và điểm $A(2;-1;0)$. Khoảng cách từ điểm $A$ đến đường thẳng $d$ bằng
	\choice
	{$\sqrt{7}$}
	{$\dfrac{\sqrt{7}}{2}$}
	{\True $\dfrac{\sqrt{21}}{3}$}
	{$\dfrac{\sqrt{7}}{3}$}
	\loigiai{
		Gọi $M(3;0;1)\in d$.\\
		Ta có $\overrightarrow{AM}=(1;1;1)$, $\overrightarrow{u_d}=(-2;-1;1)$ nên $ \left[\overrightarrow{AM}, \overrightarrow{u_d}\right]=(2;-3;1)$ và $ \left| \left[\overrightarrow{AM}, \overrightarrow{u_d}\right]\right|=\sqrt{14}$.\\
		Vậy khoảng cách từ điểm $A$ đến đường thẳng $d$ bằng $$\mathrm{d}(A,d)=\dfrac{\left| \left[\overrightarrow{AM};\overrightarrow{u_d}\right]\right|}{\left| \overrightarrow{u_d}\right|}=\dfrac{\sqrt{14}}{\sqrt{6}}=\dfrac{\sqrt{21}}{3}.$$}
\end{ex}
\begin{ex}%[Câu 16]%[2H5H2-6]
	Khoảng cách từ điểm $H(1;0;3)$ đến đường thẳng $\mathrm{d}_1\colon \heva{&x=1+t \\&y=2t \\&z=3+t}$, $t\in \mathbb{R}$ và mặt phẳng $(P)\colon z-3=0$ lần lượt là $\mathrm{d}(H,d_1)$ và $\mathrm{d}(H,(P))$. Chọn khẳng định đúng trong các khẳng định sau:
	\choice
	{$\mathrm{d}(H,d_1)>\mathrm{d}(H,(P))$}
	{$\mathrm{d}(H,(P))>\mathrm{d}(H,d_1)$}
	{\True $\mathrm{d}(H,d_1)=6\cdot \mathrm{d}(H,(P))$}
	{$\mathrm{d}(H,(P))=1$}
	\loigiai{
		Vì $H$ thuộc đường thẳng $\mathrm{d}_1$ và $H$ thuộc mặt phẳng $(P)$ nên khoảng cách từ điểm $H$ đến đường thẳng $\mathrm{d}_1$ bằng $0$ và khoảng cách từ điểm $H$ đến mặt phẳng $(P)$ bằng $0$.}
\end{ex}

\begin{ex}%[Câu 17]%[2H5H2-6]
	Tính khoảng cách giữa mặt phẳng $(\alpha)\colon 2x-y-2z-4=0$ và đường thẳng $\mathrm{d}\colon \heva{&x=1+t \\&y=2+4t \\&z=-t}$.
	\choice
	{$\dfrac{1}{3}$}
	{\True $\dfrac{4}{3}$}
	{$0$}
	{$2$}
	\loigiai{Mặt phẳng $(\alpha)$ có véc-tơ pháp tuyến $\overrightarrow{n}=(2;-1;-2)$, đường thẳng $\mathrm{d}$ có véc-tơ chỉ phương $\overrightarrow{u}=(1;4;-1)$.\\
		Ta có $\overrightarrow{n}\cdot \overrightarrow{u}=0$ và $H(1;2;0)\in d$ nhưng $H\notin (\alpha)$ nên đường thẳng $\mathrm{d}$ song song với mặt phẳng $(\alpha)$.\\
		Khoảng cách giữa đường thẳng và mặt phẳng song song bằng khoảng cách từ một điểm bất kỳ của đường thẳng đến mặt phẳng.\\
		Khi đó $\mathrm{d}(d,(\alpha))=\mathrm{d}(H,(\alpha))=\dfrac{\left| 2\cdot 1-1\cdot 2-2\cdot 0-4\right|}{\sqrt{2^2+(-1)^2+(-2)^2}}=\dfrac{4}{3}$.}
\end{ex}
\begin{ex}%[Câu 18]%[2H5H2-6]
	Trong không gian với hệ tọa độ $Oxyz$, cho mặt phẳng $(P)\colon 2x-2y-z+1=0$ và đường thẳng $\Delta\colon\dfrac{x-1}{2}=\dfrac{y+2}{1}=\dfrac{z-1}{2}$. Tính khoảng cách $\mathrm{d}$ giữa $\Delta $ và $(P)$.
	\choice
	{\True $\mathrm{d}=2$}
	{$\mathrm{d}=\dfrac{5}{3}$}
	{$\mathrm{d}=\dfrac{2}{3}$}
	{$\mathrm{d}=\dfrac{1}{3}$}
	\loigiai{
		$(P)$ có véc-tơ pháp tuyến $\overrightarrow{n}=(2;-2;-1)$ và đường thẳng $\Delta $ có véc-tơ chỉ phương $\overrightarrow{u}=(2;1;2)$ thỏa mãn $\overrightarrow{n}\cdot \overrightarrow{u}=0$ nên $\Delta \parallel (P)$ hoặc $\Delta \subset (P)$.\\
		Lấy $A(1;-2;1)\in \Delta$, ta có $\mathrm{d}(\Delta ,(P))=\mathrm{d}(A, (P))=\dfrac{\left| 2\cdot 1-2\cdot (-2)-1+1\right|}{\sqrt{4+4+1}}=2$.}
\end{ex}
%%==========Câu 21
\begin{ex}%[Câu 2]%[2H5V2-6]
	Trong không gian $Oxyz$, khoảng cách giữa đường thẳng $d\colon \dfrac{x-1}{1}=\dfrac{y}{1}=\dfrac{z}{-2}$ và mặt phẳng $\left(P\right):x+y+z+2=0$ bằng
	\choice
	{$2\sqrt{3}$}
	{$\dfrac{\sqrt{3}}{3}$}
	{$\dfrac{2\sqrt{3}}{3}$}
	{\True $\sqrt{3}$}
	\loigiai{
		Đường thẳng $d$ qua $M\left(1;0;0\right)$ và có véctơ chỉ phương $\overrightarrow{a}=\left(1;1;-2\right)$.\\
		Mặt phẳng $\left(P\right)$ có véctơ pháp tuyến $\overrightarrow{n}=\left(1;1;1\right)$.\\
		Ta có $\heva{&\overrightarrow{a}\cdot\overrightarrow{n}=1\cdot 1+1\cdot 1-2\cdot 1=0 \\&M\notin \left(P\right)}\Rightarrow d\parallel \left(P\right)$.\\
		Do đó $\mathrm{d}\left(d,(P)\right)=\mathrm{d}\left(M,(P)\right)=\dfrac{\left| 1+0+0+2\right|}{\sqrt{1^2+1^2+1^2}}=\sqrt{3}$.}
\end{ex}
%%==========Câu 22
\begin{ex}%[Câu 3]%[2H5V2-6]
	Trong không gian $Oxyz$, khoảng cách giữa đường thẳng $d\colon \dfrac{x-1}{2}=\dfrac{y-3}{2}=\dfrac{z-2}{1}$ và mặt phẳng $(P)\colon x-2y+2z+4=0$ bằng
	\choice
	{\True $1$}
	{$0$}
	{$3$}
	{$2$}
	\loigiai{
		Đường thẳng $d$ qua $M\left(1;3;2\right)$ và có véctơ chỉ phương $\overrightarrow{a}=\left(2;2;1\right)$.\\
		Mặt phẳng $\left(P\right)$ có véctơ pháp tuyến $\overrightarrow{n}=\left(1;-2;2\right)$.\\
		Ta có $\heva{&\overrightarrow{a}\cdot\overrightarrow{n}=2\cdot 1+2\cdot (-2)+1\cdot 2=0 \\&M\notin \left(P\right)}\Rightarrow d\parallel \left(P\right)$.\\
		Do đó $\mathrm{d}(d;(P))=\mathrm{d}(M;(P))=\dfrac{\left| 1-6+4+4\right|}{\sqrt{1^2+(-2)^2+2^2}}=1$.}
\end{ex}
%%==========Câu 23
\begin{ex}%[Câu 4]%[2H5V2-6]
	Trong không gian $Oxyz$, cho điểm $A(3;-2;4)$ và đường thẳng $d\colon \dfrac{x-5}{2}=\dfrac{y-1}{3}=\dfrac{z-2}{-2}$. Điểm $M$ thuộc đường thẳng $d$ sao cho $M$ cách $A$ một khoảng bằng $\sqrt{17}$. Tọa độ điểm $M$ là
	\choice
	{$(5;1;2)$ và $(6;9;2)$}
	{$(5;1;2)$ và $(-1;-8;-4)$}
	{$(5;-1;2)$ và $(1;-5;6)$}
	{\True $(5;1;2)$ và $(1;-5;6)$}
	\loigiai{
		Gọi $M(5+2t;1+3t;2-2t)\in d$. Ta có $\overrightarrow{AM}=(2+2t;3+3t;-2-2t)$.\\
		Với $AM=\sqrt{17}\Leftrightarrow 17(1+t)^2=17\Leftrightarrow \hoac{&t=0\Rightarrow M(5;1;2) \\&t=-2\Rightarrow M(1;-5;6).}$
	}
\end{ex}
%%==========Câu 24
\begin{ex}%[Câu 5]%[2H5V2-6]
	Trong không gian $Oxyz$, cho hai đường thẳng $d_1\colon \dfrac{x-1}{2}=\dfrac{y}{1}=\dfrac{z}{3}$ và $d_2\colon\heva{&x=1+t \\&y=2+t \\&z=m}$. Gọi $S$ là tập tất cả các số $m$ sao cho $d_1$ và $d_2$ chéo nhau và khoảng cách giữa chúng bằng $\dfrac{5}{\sqrt{19}}$. Tính tổng các phần tử của $S$.
	\choice
	{$-11$}
	{$12$}
	{\True $-12$}
	{$11$}
	\loigiai{
		Đường thẳng $d_1$ đi qua điểm $M(1;0;0)$, có véctơ chỉ phương $\vec{u}_1=(2;1;3)$.\\
		Đường thẳng $d_2$ đi qua điểm $N(1;2;m)$, có véctơ chỉ phương $\vec{u}_2=(1;1;0)$.\\
		Ta có $\left[\vec{u}_1,\vec{u}_2\right]=(-3;3;1)$ và $\overrightarrow{MN}=(0;2;m)$.\\
		Hai đường thẳng $d_1$ và $d_2$ chéo nhau khi và chỉ khi $\left[\vec{u}_1,\vec{u}_2\right]\cdot\overrightarrow{MN}\ne 0\Leftrightarrow m\ne -6$.\\
		Mặt khác $\begin{aligned}[t]
			\mathrm{d}\left(d_1,d_2\right)=\dfrac{5}{\sqrt{19}}&\Leftrightarrow \dfrac{\left| \left[\vec{u}_1,\vec{u}_2\right]\cdot\overrightarrow{MN}\right|}{\left| \left[\vec{u}_1,\vec{u}_2\right]\right|}=\dfrac{5}{\sqrt{19}}\\&\Leftrightarrow \dfrac{\left| m+6\right|}{\sqrt{19}}=\dfrac{5}{\sqrt{19}}\Leftrightarrow \hoac{&m=-1 \\&m=-11.}\end{aligned}$\\
		Khi đó tổng các phần tử của $m$ là $-12$.}
\end{ex}
%%==========Câu 25
\begin{ex}%[Câu 6]%[2H5V2-6]
	Trong không gian $Oxyz$, tính khoảng cách giữa hai đường thẳng $d_1\colon\dfrac{x}{1}=\dfrac{y-3}{2}=\dfrac{z-2}{1}$ và $d_2\colon\dfrac{x-3}{1}=\dfrac{y+1}{-2}=\dfrac{z-2}{1}$.
	\choice
	{$\dfrac{\sqrt{2}}{3}$}
	{$\dfrac{12}{5}$}
	{\True $\dfrac{3\sqrt{2}}{2}$}
	{$3$}
	\loigiai{
		Đường thẳng $d_1$ qua $M(0;3;2)$ và có véctơ chỉ phương $\vec{u}=(1;2;1)$.\\
		Đường thẳng $d_2$ qua $N(3;-1;2)$ và có véctơ chỉ phương $\vec{v}=(1;-2;1)$.\\
		Ta có $\left[\vec{u},\vec{v}\right]=(4;0;-4)$ và $\overrightarrow{MN}=(3;-4;0)$.\\
		Khi đó $\mathrm{d}\left(d_1,d_2\right)=\dfrac{\left| \left[\vec{u},\vec{v}\right]\cdot \overrightarrow{MN}\right|}{\left| \left[\vec{u},\vec{v}\right]\right|}=\dfrac{12}{4\sqrt{2}}=\dfrac{3\sqrt{2}}{2}$.}
\end{ex}
%%==========Câu 26
\begin{ex}%[Câu 7]%[2H5V2-6]
	Trong không gian $Oxyz$, cho hai đường thẳng $d\colon \heva{&x=1+t \\&y=-3-t \\&z=2+2t}$ và $d'\colon\dfrac{x}{3}=\dfrac{y-3}{-1}=\dfrac{z-1}{1}$. Khi đó khoảng cách giữa $d$ và $d'$ bằng
	\choice
	{$\dfrac{13\sqrt{30}}{30}$}
	{$\dfrac{\sqrt{30}}{3}$}
	{\True $\dfrac{9\sqrt{30}}{10}$}
	{$0$}
	\loigiai{
		Đường thẳng $d$ qua $A\left(1;-3;2\right)$ và có véctơ chỉ phương $\overrightarrow{u}=(1;-1;2)$.\\
		Đường thẳng $d'$ qua $B\left(0;3;1\right)$ và có véctơ chỉ phương $\overrightarrow{u'}=(3;-1;1)$.\\
		Khi đó $\mathrm{d}\left(d,d'\right)=\dfrac{\left| \left[\overrightarrow{u},\overrightarrow{u'}\right]\cdot\overrightarrow{AB}\right|}{\left| \left[\overrightarrow{u},\overrightarrow{u'}\right]\right|}=\dfrac{27}{\sqrt{30}}=\dfrac{9\sqrt{30}}{10}$.}
\end{ex}
%%==========Câu 27
\begin{ex}%[Câu 8]%[2H5V2-6]
	Trong không gian $Oxyz$, cho hai đường thẳng $d_1\colon \dfrac{x-1}{2}=\dfrac{y+2}{-1}=\dfrac{z}{1}$ và $d_2\colon \heva{&x=1+4t \\&y=-1-2t \\&z=2+2t}$. Khoảng cách giữa hai đường thẳng đã cho bằng
	\choice
	{$\dfrac{\sqrt{87}}{6}$}
	{\True $\dfrac{\sqrt{174}}{6}$}
	{$\dfrac{\sqrt{174}}{3}$}
	{$\dfrac{\sqrt{87}}{3}$}
	\loigiai{
		Đường thẳng $d_1$ đi qua điểm $M(1;-2;0)$ và có véctơ chỉ phương  $\overrightarrow{u_1}=(2;-1;1)$.\\
		Đường thẳng $d_2$ đi qua điểm $N(1;-1;2)$ và có véctơ chỉ phương  $\overrightarrow{u_2}=(4;-2;2)$.\\
		Ta có $\heva{&\overrightarrow{u_2}=2\cdot\overrightarrow{u_1}\\&M\left(1;-2;0\right)\notin d_2}\Rightarrow d_1\parallel d_2$.\\ 
		Ta có $\overrightarrow{MN}=(0;1;2)\Rightarrow\left [\overrightarrow{MN},\overrightarrow{u_2}\right ]=(6;8;-4)$.\\
		Suy ra $\mathrm{d}\left (d_1,d_2\right )=\mathrm{d}\left(M;d_2\right)=\dfrac{\left| \left [\overrightarrow{MN}, \overrightarrow{u_2}\right ]\right|}{\left| \overrightarrow{u_2}\right|}=\dfrac{\sqrt{6^2+8^2+(-4)^2}}{\sqrt{4^2+(-2)^2+2^2}}=\dfrac{\sqrt{174}}{6}$.\\
	}
\end{ex}
%%==========Câu 28
\begin{ex}%[Câu 9]%[2H5V2-6]
	Trong không gian $Oxyz$, tính khoảng cách từ giao điểm của hai đường thẳng $d_1$ và $d_2$ tới mặt phẳng $(P)$. Với $d_1\colon \dfrac{x+1}{2}=\dfrac{y}{3}=\dfrac{z-1}{3}$; $d_2\colon \dfrac{-x+1}{2}=\dfrac{y}{1}=\dfrac{z-1}{1}$ và $(P)\colon 2x+4y-4z-3=0$.
	\choice
	{\True $\dfrac{4}{3}$}
	{$\dfrac{7}{6}$}
	{$\dfrac{13}{6}$}
	{$\dfrac{5}{3}$}
	\loigiai{
		Phương trình tham số của hai đường thẳng $d_1,d_2$ là
		$d_1\colon \heva{&x=-1+2t \\&y=3t \\&z=1+3t};d_2\colon\heva{&x=1-2t' \\&y=t' \\&z=1+t'}$.\\
		Xét hệ phương trình: $\heva{&-1+2t=1-2t' \\&3t=t' \\&1+3t=1+t'}\Leftrightarrow \heva{&2t+2t'=2 \\&3t-t'=0 \\&3t-t'=0}\Leftrightarrow \heva{&t=\dfrac{1}{4} \\&t'=\dfrac{3}{4}.}$\\
		Suy ra giao điểm của $d_1,d_2$ là $A\left(-\dfrac{1}{2};\dfrac{3}{4};\dfrac{7}{4}\right)$.\\
		Khoảng cách từ $A$ đến mặt phẳng $\left(P\right)$ là
		$$\mathrm{d}\left(A;(P)\right)=\dfrac{\left| 2\cdot \left(-\dfrac{1}{2}\right)+4\cdot\left(\dfrac{3}{4}\right)-4\cdot\left(\dfrac{7}{4}\right)-3\right|}{\sqrt{2^2+4^2+\left(-4\right)^2}}=\dfrac{4}{3}.$$}
\end{ex}
%%==========Câu 29
\begin{ex}%[Câu 10]%[2H5V2-6]
	Trong không gian $Oxyz$, cho mặt phẳng $(P)\colon 2x-y+2z-3=0$ và đường thẳng $\Delta\colon\dfrac{x-1}{2}=\dfrac{y+1}{2}=\dfrac{x-1}{-1}$. Khoảng cách giữa đường thẳng $\Delta$ và mặt phẳng $(P)$ bằng
	\choice
	{\True $\dfrac{2}{3}$}
	{$\dfrac{8}{3}$}
	{$\dfrac{2}{9}$}
	{$1$}
	\loigiai{
		Mặt phẳng $(P)$ có véctơ pháp tuyến là $\overrightarrow{n}=(2;-1;2)$.\\
		Đường thẳng $\Delta$ đi qua điểm $M=(1;-1;1)$ và có véctơ chỉ phương là $\overrightarrow{u}=(2;2;-1)$.\\
		Ta có $\heva{&\overrightarrow{n}\cdot\overrightarrow{u}=2\cdot2+(-1)\cdot 2+2\cdot (-1)=0 \\&M\notin \left(P\right)}\Rightarrow \Delta \parallel (P)$.\\
		Khi đó $\mathrm{d}\left(\Delta,(P)\right)=\mathrm{d}\left(M,(P)\right)=\dfrac{\left| 2+1+2-3\right|}{\sqrt{2^2+2^2+(-1)^2}}=\dfrac{2}{3}$.\\
	}
\end{ex}
%%==========Câu 30
\begin{ex}%[Câu 11]%[2H5C2-6]
	Trong không gian $Oxyz$, cho đường thẳng $d\colon \dfrac{x-3}{2}=\dfrac{y+2}{1}=\dfrac{z+1}{-1}$, mặt phẳng $(P)\colon x+y+z+2=0$. Gọi $M$ là giao điểm của $d$ và $(P)$, $\Delta$ là đường thẳng nằm trong mặt phẳng $(P)$ vuông góc với $d$ và cách $M$ một khoảng bằng $\sqrt{42}$. Phương trình đường thẳng $\Delta$ là
	\choice
	{$\dfrac{x-5}{2}=\dfrac{y+2}{-3}=\dfrac{z+4}{1}$}
	{$\dfrac{x-1}{-2}=\dfrac{y+1}{-3}=\dfrac{z+1}{1}$}
	{$\dfrac{x-3}{2}=\dfrac{y+4}{-3}=\dfrac{z+5}{1}$}
	{\True $\dfrac{x+3}{2}=\dfrac{y+4}{-3}=\dfrac{z-5}{1}$}
	\loigiai{
		Ta có $M = d \cap (P)$.\\
		Suy ra $M \in d \Rightarrow M (3 + 2 t; - 2 + t; - 1 - t)$ và $M \in (P) \Rightarrow t = - 1 \Rightarrow M (1; - 3; 0)$.\\
		Mặt phẳng $(P)$ có véctơ pháp tuyến là $\vec{n}_P=(1;1;1)$.\\
		Đường thẳng $d$ có véctơ chỉ phương $\vec{a}_d=(2;1;-1)$.\\
		Đường thẳng $\Delta$ có véctơ chỉ phương $\vec{a}_{\Delta}=\left[\vec{a}_d,\vec{n}_P\right]=(2;-3;1)$.\\
		Gọi $N (x; y; z)$ là hình chiếu vuông góc của $M$ trên $\Delta$, khi đó $\vec{MN} = (x - 1; y + 3; z)$.\\
		Ta có $\heva{&\vec{M N} \perp \vec{a _{\Delta}} \\&N \in (P) \\&M N = \sqrt{42}} \Leftrightarrow \heva{&2 x - 3 y + z - 11 = 0 \\&x + y + z + 2 = 0 \\&(x - 1) ^2 + (y + 3) ^2 + z ^2 = 42.}$\\
		Giải hệ ta tìm được $N (5; - 2; - 5)$ hoặc $N (- 3; - 4; 5)$.\\
		Với $N (5; - 2; - 5)$, ta có $\Delta: \dfrac{x - 5}{2} = \dfrac{y + 2}{- 3} = \dfrac{z + 5}{1}$.\\
		Với $N (- 3; - 4; 5)$, ta có $\Delta: \dfrac{x + 3}{2} = \dfrac{y + 4}{- 3} = \dfrac{z - 5}{1}$.}
\end{ex}
%%==========Câu 31
\begin{ex}%[Câu 12]%[2H5C2-6]
	Trong không gian $Oxyz$, cho 4 điểm $A(2;0;0)$, $B(0;3;0)$, $C(0;0;6)$ và $D(1;1;1)$. Gọi $\Delta $ là đường thẳng qua $D$ và thỏa mãn tổng khoảng cách từ các điểm $A,B,C$ đến $\Delta $ là lớn nhất. Khi đó $\Delta $ đi qua điểm nào dưới đây?
	\choice
	{$\left(4;3;7\right)$}
	{$\left(-1;-2;1\right)$}
	{\True $\left(7;5;3\right)$}
	{$\left(3;4;3\right)$}
	\loigiai{
		Phương trình mặt phẳng $(ABC)\colon\dfrac{x}{2}+\dfrac{y}{3}+\dfrac{z}{6}=1\Leftrightarrow 3x+2y+z-6=0$.\\
		Dễ thấy $D\in \left(ABC\right)$.\\
		Ta có $P=\mathrm{d}\left(A,\Delta\right)+\mathrm{d}\left(B,\Delta\right)+\mathrm{d}\left(C,\Delta\right)\le AD+BD+CD$.\\
		Vậy $P$ lớn nhất khi và chỉ khi các hình chiếu vuông góc của các điểm $A,B,C$ trên $\Delta $ trùng $D$ hay $\Delta \perp \left(ABC\right)$ tại $D$.\\
		Phương trình đường thẳng $\Delta $ là $\heva{&x=1+3t \\&y=1+2t \\&z=1+t}$, ta thấy $\Delta $ đi qua điểm có tọa độ $(7;5;3)$.}
\end{ex}
%%==========Câu 32
\begin{ex}%[Câu 13]%[2H5C2-6]
	Trong không gian $Oxyz$, gọi $d$ là đường thẳng đi qua $O$ thuộc mặt phẳng $(Oyz)$ và cách điểm $M(1;-2;1)$ một khoảng nhỏ nhất. Côsin của góc giữa $d$ và trục tung bằng
	\choice
	{$\dfrac{2}{5}$}
	{$\dfrac{1}{5}$}
	{$\dfrac{1}{\sqrt{5}}$}
	{\True$\dfrac{2}{\sqrt{5}}$}
	\loigiai{
		\immini{Gọi $H,K$ lần lượt là hình chiếu của $M$ trên mặt phẳng $(Oyz)$ và trên đường thẳng $d$.\\
			Ta có $d\left(M,d\right)=MK\ge MH=1$ với $H\left(0;-2;1\right)$.\\
			Suy ra $\mathrm{d}\left(M,d\right)_{min}\Leftrightarrow K\equiv H$.\\
			Khi đó $d$ có một véctơ chỉ phương là $\overrightarrow{OH}=(0;-2;1)$.
		}{\begin{tikzpicture}[>=stealth,line join=round,line cap=round,font=\footnotesize,scale=1]
				\path (0,0)coordinate(A) (6,0)coordinate(B) (8,2)coordinate(C) (2,2)coordinate(D);
				\coordinate[label=left:$H$] (H)at(3,1);
				\coordinate[label=above:$M$] (M)at(3,4);
				\coordinate[label=above:$K$] (K)at(6,1);
				\coordinate[label=below:$O$] (O)at(4,0.5);
				\coordinate (I) at ($(O)!1.3!(K)$);
				\coordinate (J) at ($(K)!1.3!(O)$);
				\path pic[angle radius=13mm,draw=blue,"$Oyz$",angle eccentricity=.65] {angle = B--A--D};
				\draw (A)--(B)--(C)--(D)--cycle (M)--(H)--(K)--cycle (I)--(J);
				\foreach  \diem in {M,H,K,O}\fill (\diem)circle(1.5pt);
		\end{tikzpicture}}
		Vậy $\cos \left(d,Oy\right)=\dfrac{\left| \overrightarrow{OH}\cdot\vec{j}\right|}{\left| \overrightarrow{OH}\right|\cdot \left|\vec{j}\right|}=\dfrac{2}{\sqrt{5}}$.}
\end{ex}
%%==========Câu 33
\begin{ex}%[Câu 14]%[2H5C2-6]
	Trong không gian $Oxyz$, cho điểm $A(2;1;1)$, mặt phẳng $(P)\colon x-z-1=0$ và đường thẳng $d\colon \heva{&x=1-t \\&y=2 \\&z=-2+t}$. Gọi $d_1;d_2$ là các đường thẳng đi qua $A$, nằm trong $(P)$ và đều có khoảng cách đến đường thẳng $d$ bằng $\sqrt{6}$. Côsin của góc giữa $d_1$ và $d_2$ bằng
	\choice
	{\True $\dfrac{1}{3}$}
	{$\dfrac{2}{3}$}
	{$\dfrac{\sqrt{3}}{3}$}
	{$\dfrac{\sqrt{2}}{3}$}
	\loigiai{
		\begin{center}
			\begin{tikzpicture}[>=stealth,line join=round,line cap=round,font=\footnotesize,scale=1]
				\path (0,0)coordinate(O) (6,0)coordinate(B) (9,2.5)coordinate(C) (3,2.5)coordinate(D);
				\coordinate[label=below:$A$] (A)at(3,1);
				\coordinate[label=right:$M$] (M)at(6,1);
				\coordinate[label=left:$H$] (H)at(5,2);
				\coordinate[label=right:$K$] (K)at(5.5,.2);
				\coordinate [label=below right:$d$](N)at(6,4);
				\coordinate (E) at ($(A)!1.3!(H)$);
				\coordinate[label=left:$d_1$] (F) at ($(H)!1.3!(A)$);
				\coordinate (I) at ($(A)!1.05!(K)$);
				\coordinate[label=left:$d_2$] (J) at ($(K)!1.3!(A)$);
				\path pic[angle radius=13mm,draw=blue,"$P$",angle eccentricity=.65] {angle = B--O--D};
				\path pic[angle radius=3mm,draw] {right angle = M--H--A};
				\path pic[angle radius=3mm,draw] {right angle = M--K--A};
				\draw (O)--(B)--(C)--(D)--cycle (E)--(F) (I)--(J) (M)--(N) (H)--(M)--(K) (A)--(M);
				\foreach  \diem in {A,M,H,K}\fill (\diem)circle(1.5pt);
			\end{tikzpicture}
		\end{center}
		\begin{itemize}
			\item Ta có $\overrightarrow{n}_P=(1;0;-1),\overrightarrow{u}_{d}=(-1;0;1)\Rightarrow d\perp \left(P\right)$ và $d\cap (P)=M(0;2;-1)$.\\
			Suy ra $\overrightarrow{MA}=\left(2;-1;2\right)\Rightarrow MA=3$.
			\item Gọi $H; K$ lần lượt là hình chiếu vuông góc của $M$ lên $d_1$ và $d_2$, ta có:\\
			$\heva{&\mathrm{d}\left(d_1;d\right)=\mathrm{d}\left(M;d_1\right)=MH\\&
				\mathrm{d}\left(d_2;d\right)=\mathrm{d}\left(M;d_2\right)=MK}\Rightarrow MH=MK=\sqrt{6}$.\\
			$\Rightarrow \sin \widehat{MAK}=\sin \widehat{MAH}=\dfrac{HM}{AM}=\dfrac{\sqrt{6}}{3}$.\\
			$\Rightarrow \cos \left(d_1;d_2\right)=\left| \cos \left(2.\widehat{MAH}\right)\right|=\left| 1-2\sin^2\widehat{MAH}\right|=\left| 1-\dfrac{4}{3}\right|=\dfrac{1}{3}$.
		\end{itemize}
	}
\end{ex}
%%==========Câu 34
\begin{ex}%[Câu 15]%[2H5C2-6]
	Trong không gian $Oxyz$, cho đường thẳng $d\colon\dfrac{x-3}{1}=\dfrac{y-3}{3}=\dfrac{z}{2}$, mặt phẳng $(P)\colon x+y-z+3=0$ và điểm $A(1;2;-1)$. Đường thẳng $\Delta$ đi qua $A$, cắt $d$ và song song với mặt phẳng $\left(P\right)$. Tính khoảng cách từ gốc tọa độ $O$ đến $\Delta$.
	\choice
	{$\sqrt{3}$}
	{$\dfrac{16}{3}$}
	{$\dfrac{2\sqrt{3}}{3}$}
	{\True $\dfrac{4\sqrt{3}}{3}$}
	\loigiai{
		Gọi $M=\Delta\cap d\Rightarrow M(t+3;3t+3;2t)\,\left(t\in \mathbb{R}\right)\Rightarrow \overrightarrow{AM}=(t+2;3t+1;2t+1)$.\\
		Gọi $\overrightarrow{n}=(1;1;-1)$ là vectơ pháp tuyến của mặt phẳng $(P)$.\\
		Ta có $\begin{aligned}[t]
			\Delta \parallel (P)\Rightarrow \overrightarrow{AM}\perp \overrightarrow{n}&\Leftrightarrow \overrightarrow{AM}\cdot\overrightarrow{n}=0\\
			&\Leftrightarrow t+2+3t+1-2t-1=0\\
			&\Leftrightarrow t=-1\Rightarrow \overrightarrow{AM}=(1;-2;-1).\end{aligned}$\\
		Khi đó $\mathrm{d}\left(O;\Delta\right)=\dfrac{\left| \left[\overrightarrow{AM},\overrightarrow{OA}\right]\right|}{\left| \overrightarrow{AM}\right|}=\dfrac{4\sqrt{3}}{3}$.}
\end{ex}
%%==========Câu 35
\begin{ex}%[Câu 16]%[2H5C2-6]
	Trong không gian $Oxyz$, đường thẳng $d\colon \heva{&x=t \\&y=-1+2t \\&z=2-t}, t\in \mathbb{R}$ cắt mặt phẳng $(P)\colon x+y+z-3=0$ tại điểm $I$. Gọi $\Delta $ là đường thẳng nằm trong mặt phẳng $(P)$ sao cho $\Delta \perp d$ và khoảng cách từ điểm $I$ đến đường thẳng $\Delta $ bằng $\sqrt{42}$. Tìm tọa độ hình chiếu $M(a;b;c)$ (với $a+b>c$) của điểm $I$ trên đường thẳng $\Delta $.
	\choice
	{\True $M\left(2;5;-4\right)$}
	{$M\left(6;-3;0\right)$}
	{$M\left(5;2;-4\right)$}
	{$M\left(-3;6;0\right)$}
	\loigiai{
		\begin{center}
			\begin{tikzpicture}[>=stealth,line join=round,line cap=round,font=\footnotesize,scale=1]
				\path (0,0)coordinate(A) (6,0)coordinate(B) (8,2)coordinate(C) (2,2)coordinate(D);
				\coordinate[label=left:$I$] (I)at(3,1);
				\coordinate[label=below:$M$] (M)at(5,1);
				\coordinate[label=below left:$d$] (N)at(4,4);
				\coordinate (P)at(2.67,0);
				\coordinate (Q)at(6,1.5);
				\coordinate (R) at ($(I)!1.5!(P)$);
				\coordinate[label=left:$\Delta$] (T) at ($(Q)!2!(M)$);
				\draw (A)--(B)--(C)--(D)--cycle (I)--(M) (N)--(I) (P)--(R) (Q)--(T);
				\draw[dashed] (I)--(P);
				\path pic[angle radius=5mm,draw=blue] {right angle = Q--M--I};
				\path pic[angle radius=9mm,draw=blue,"$P$",angle eccentricity=.75] {angle = B--A--D};
				\foreach  \diem in {I,M}\fill (\diem)circle(1.5pt);
			\end{tikzpicture}
		\end{center}
		\begin{itemize}
			\item \textbf{Cách 1}.\\
			$(P)$ có véctơ pháp tuyến $\overrightarrow{n}=(1;1;1)$ và $d$ có véctơ chỉ phương $\overrightarrow{u}=\left(1;2;-1\right)$.\\
			$I=d\cap (P)\Rightarrow I(1;1;1)$.\\
			Vì $\Delta \subset (P)$ và $\Delta \perp d\Rightarrow \Delta $ có véctơ chỉ phương $\overrightarrow{u}_{\Delta}=\left[\overrightarrow{n},\overrightarrow{u}\right]=(-3;2;1)$.\\
			$M$ là hình chiếu của $I$ trên $\Delta $ nên $M$ thuộc mặt phẳng $(Q)$ đi qua $I$ và vuông góc với $\Delta $.\\
			Mặt phẳng $(Q)$ nhận $\overrightarrow{u}_{\Delta}=(-3;2;1)$ làm véctơ pháp tuyến nên ta có phương trình của $(Q)\colon -3(x-1)+2(y-1)+1(z-1)=0\Leftrightarrow 3x-2y-z=0$.\\
			Gọi $d_1=(P)\cap (Q)\Rightarrow d_1$ có véctơ chỉ phương $\overrightarrow{v}=\left[\overrightarrow{u}_{\Delta},\overrightarrow{n}\right]=(1;4;-5)$ và $d_1$ đi qua $I$, phương trình của $d_1\colon \heva{&x=1+t \\&y=1+4t \\&z=1-5t.}$\\
			Mặt khác $M\in \Delta \Rightarrow M\in (P)\Rightarrow M\in d_1$.\\
			Giả sử $M(1+t;1+4t;1-5t)\Rightarrow \overrightarrow{IM}=(t;4t;-5t)$.\\
			Ta có $IM=\sqrt{42}\Leftrightarrow \sqrt{t^2+16t^2+25t^2}=\sqrt{42}\Leftrightarrow t=\pm 1$.\\
			+) Với $t=1\Rightarrow M(2;5;-4)$.\\
			+) Với $t=-1\Rightarrow M(0;-3;6)$.\\
			Vì $M(a;b;c)$ (với $a+b>c$) nên $M(2;5;-4)$.\\
			\item \textbf{Cách 2}.\\ 
			Vì $M(a;b;c)$ là hình chiếu vuông góc của $I$ lên $\Delta $. Khi đó ta có\\
			$\begin{aligned}[t]
				\heva{&M\in \left(P\right) \\&\overrightarrow{IM}\perp{\overrightarrow{u}_{\Delta}} \\&IM=\sqrt{42}}
				&\Leftrightarrow \heva{&a+b+c-3=0 \\&-3\left(a-1\right)+2\left(b-1\right)+\left(c-1\right)=0 \\&\left(a-1\right)^2+\left(b-1\right)^2+\left(c-1\right)^2=42}\\
				&\Leftrightarrow \heva{&a+b+c-3=0 \\&-3a+2b+c=0 \\&\left(a-1\right)^2+\left(b-1\right)^2+\left(c-1\right)^2=42}\\
				&\Leftrightarrow \heva{&4a-b=3 \\&a+b+c-3=0 \\&\left(a-1\right)^2+\left(b-1\right)^2+\left(c-1\right)^2=42}\\
				&\Leftrightarrow \heva{&b=4a-3 \\&c=-5a+6 \\&\left(a-1\right)^2+\left(b-1\right)^2+\left(c-1\right)^2=42}\\
				&\Leftrightarrow \heva{&a=0 \\&b=-3 \\&c=6} \,\text{hoặc}\,\heva{&a=2 \\&b=5 \\&c=-4.}\end{aligned}$\\
			Vì $M\left(a;b;c\right)$ (với $a+b>c$) nên $M\left(2;5;-4\right)$.
		\end{itemize}
	}
\end{ex}
%%==========Câu 36
\begin{ex}%[Câu 17]%[2H5C2-6]
	Trong không gian $Oxyz$, cho hai điểm $A(3;3;1)$, $B(0;2;1)$ và mặt phẳng $(P)\colon x+y+z-7=0$. Đường thẳng $d$ nằm trong $(P)$ sao cho mọi điểm của $d$ cách đều hai điểm $A,B$ có phương trình là
	\choice
	{$\heva{&x=2t \\&y=7-3t \\&z=t}$}
	{$\heva{&x=t \\&y=7+3t \\&z=2t}$}
	{\True $\heva{&x=t \\&y=7-3t \\&z=2t}$}
	{$\heva{&x=-t \\&y=7-3t \\&z=4t}$}
	\loigiai{
		\begin{itemize}
			\item Các điểm cách đều hai điểm $A,B$ thì nằm trên mặt phẳng $(\alpha)$ là mặt phẳng trung trực của đoạn $AB$.
			\item Gọi $I$ là trung điểm của $AB\Rightarrow I\left(\dfrac{3}{2};\dfrac{5}{2};1\right)$.
			\item Phương trình mặt phẳng $\left(\alpha\right)$ là $3x+y-7=0$.
		\end{itemize}
		Do đó đường thẳng $d$ là giao tuyến của $2$ mặt phẳng $(P)$ và $(\alpha)$.\\
		Phương trình đường thẳng $d$ đi qua điểm $M(0;7;0)=(P)\cap (\alpha)$ và nhận $\overrightarrow{u}=\left[\overrightarrow{n}_{(\alpha)},\overrightarrow{n}_{(P)}\right]=(1;-3;2)$ làm một véctơ chỉ phương là $\heva{&x=t \\&y=7-3t \\&z=2t.}$}
\end{ex}
%%==========Câu 37
\begin{ex}%[Câu 18]%[2H5C2-6]
	Trong không gian $Oxyz,$ cho hai đường thẳng $d_1\colon \dfrac{x-1}{1}=\dfrac{y+2}{1}=\dfrac{z-1}{2}$ và $d_2\colon \dfrac{x-1}{2}=\dfrac{y-1}{1}=\dfrac{z+2}{1}$. Mặt phẳng $(P)\colon x+ay+bz+c=0\,(c>0)$ song song với $d_1$, $d_2$ và khoảng cách từ $d_1$ đến $(P)$ bằng hai lần khoảng cách từ $d_2$ đến $(P)$. Giá trị của $a+b+c$ bằng
	\choice
	{\True $14$}
	{$6$}
	{$-4$}
	{$-6$}
	\loigiai{
		Gọi $\vec{u}_1=(1;1;2)$, $\vec{u}_2=(2;1;1)$ lần lượt là một véctơ chỉ phương của $d_1$, $d_2$.\\
		Gọi $\vec{n}_1=\left[\vec{u}_1,\vec{u}_2\right]=(-1;3;-1)$. Ta có $\vec{n}_2=\left(1;-3;1\right)$ cùng phương $\vec{n}_1$.\\
		$\vec{n}=(1;a;b)$ là một véctơ chỉ phương của $(P)$.\\
		Do $(P)$ song song với $d_1$, $d_2$ nên có véctơ pháp tuyến là $\vec{n}=\left(1;-3;1\right)$.\\
		Suy ra phương trình mặt phẳng $\left(P\right)$ có dạng: $x-3y+z+c=0$.\\
		Lấy $M_1\left(1;-2;1\right)\in d_1$, $M_2\left(1;1;-2\right)\in d_2$.\\
		Ta có $\begin{aligned}[t]
			\mathrm{d}\left(d_1;(P)\right)=2\mathrm{d}\left(d_2;(P)\right)
			&\Leftrightarrow \mathrm{d}\left(M_1;(P)\right)=2\mathrm{d}\left(M_2;(P)\right)\\
			&\Leftrightarrow \dfrac{\left| 1-3\left(-2\right)+1+c\right|}{\sqrt{11}}=2\dfrac{\left| 1-3-2+c\right|}{\sqrt{11}}\\
			&\Leftrightarrow \left| 8+c\right|=2\left| -4+c\right|\\
			&\Leftrightarrow \hoac{&8+c=2\left(-4+c\right) \\&8+c=2\left(4-c\right)}
			\Leftrightarrow \hoac{&c=16\,\left(\text{nhận}\right) \\&c=0.\,\left(\text{loại}\right)}\end{aligned}$\\
		Nên $(P)\colon x-3y+z+16=0$, suy ra $a=-3$, $b=1$, $c=16$.\\
		Vậy $a+b+c=14$.}
\end{ex}
\Closesolutionfile{ans}
\indapan{10}{ans/C5B4CD6-D2}