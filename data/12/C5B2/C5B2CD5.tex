% \chude{Lập phương trình mặt phẳng liên quan đến đường thẳng}
\begin{dang}{Viết PTMP biết vị trí tương đối với đường thẳng}
	\immini{\tickv Viết phương trình mặt phẳng $(P)$ qua $M$ và vuông góc với đường thẳng $d$ (hoặc vuông góc với đường thẳng $AB$)\\
		\textbf{Phương pháp}:
		$(P)\colon\heva{&\text{Qua } M\left(x_0;y_0;z_0\right) \\&
		\text{Vectơ pháp tuyến } \vec{n}_{(P)}=\vec{u}_d=\overrightarrow{AB}.}$}
		{\begin{tikzpicture}
			\def\a{4}
			\def\b{2}
			\def\g{35}
			\def\h{3}
			\path
			(0:0) coordinate (E)--++(\g:\b) coordinate (F)--++(0:\a) coordinate (C)--++(\g-180:\b) coordinate (D)--++(120:.8)coordinate (G)--++(90:.7)coordinate (A)--++(90:1.3)coordinate (B)--++(90:.5)coordinate (H)--++(-90:3.2)coordinate (K)--++(160:1.6)coordinate (M);
			\foreach \x/\g in {A/180,B/180,M/180}
			\fill[black](\x) circle (1pt)
			($(\x)+(\g:3mm)$) node{$\x$};
			\draw (H) node[right]{$d$};
			\draw[gray] ($(G)!.4!(A)$) coordinate (X) ($(G)!.15!(C)$) coordinate (Y) (X)--($(X)+(Y)-(G)$)-- (Y);
			\draw  pic [draw, angle radius = 9 mm, "$P$"] {angle = D--E--F}; 
			\draw (G)--(B) (H)--(A) (E)--(F)--(C)--	(D)--cycle;
			\draw[dashed] (G)--(K);
			\draw[-stealth] (A)--(B);
		\end{tikzpicture}}
		\immini{\tickv Viết phương trình mặt phẳng qua $M$ và chứa đường thẳng $d$ với $M \notin d$.\\
		\textbf{Phương pháp}:
	\begin{itemize}
		\item Chọn điểm $A \in d$ và một vectơ chỉ phương $\overrightarrow{u_d}$. Tính $\left[\overrightarrow{A M}, \overrightarrow{u_d}\right]$.
		\item Phương trình mặt phẳng $(P)\colon\heva{& \text{Đi qua } M \\ & \text{có vectơ pháp tuyến } \vec{n}=\left[\overrightarrow{AM}, \overrightarrow{u_d}\right].} $
	\end{itemize}}
	{\begin{tikzpicture}
		\def\a{4}
		\def\b{2}
		\def\g{35}
		\def\h{3}
		\draw
		(0:0) coordinate (E)--++(\g:\b) coordinate (F)--++(0:\a) coordinate (C)--++(\g-180:\b) coordinate (D)--cycle ($(E)!.7!(F)$)--++(-10:1) coordinate (A)--++(-10:1.25)coordinate (B)--++(-10:1);
		\draw[-stealth]
		(A)--++(90:2)coordinate (N) node[below left]{$\vec{n}$}; \draw[-stealth](A)--++(10:1.4) coordinate (M);
		\draw[gray] ($(A)!.2!(B)$) coordinate (X) ($(A)!.15!(N)$) coordinate (Y) (X)--($(X)+(Y)-(A)$)-- (Y);
		\draw  pic [draw, angle radius = 9 mm, "$P$"] {angle = D--E--F}; 
		\foreach \x/\g in {A/-90,M/-10}
		\fill[black](\x) circle (1pt)
		($(\x)+(\g:3mm)$) node{$\x$};
		\draw
		($(B)+(-150:4mm)$) node{$\vec{u}_d$};
		\draw[-stealth] (A)--(B);
	\end{tikzpicture}}
\end{dang}
\Opensolutionfile{ans}[ans/ans-C5B2CD5]
\TN
\begin{ex}%[2H5H2-5]
	Trong KG $Oxyz$, cho đường thẳng $d\colon \dfrac{x-1}{1}=\dfrac{y-2}{-2}=\dfrac{z+2}{1}$. Mặt phẳng nào sau đây vuông góc với đường thẳng $d$?
	\choice
		{$(T)\colon x+y+2 z+1=0$}
		{\True $(P)\colon x-2 y+z+1=0$}
		{$(Q)\colon x-2 y-z+1=0$}
		{$(R)\colon x+y+z+1=0$}
	\loigiai{
		Đường thẳng vuông góc với mặt phẳng nếu vectơ chỉ phương của đường thẳng cùng phương với vectơ pháp tuyến của mặt phẳng.\\
		Đường thẳng $d$ có một vectơ chỉ phương là $\vec{u}=(1;-2;1)$.\\
		Mặt phẳng $(T)$ có một vectơ pháp tuyến là $\vec{n}_{(T)}=(1;1;2)$.\\
		Do $\dfrac{1}{1} \neq \dfrac{-2}{1} \neq \dfrac{1}{2}$ nên $\vec{u}$ không cùng phương với $\vec{n}_{(T)}$. Do đó $d$ không vuông góc với $(T)$.\\
		Mặt phẳng $(P)$ có một vectơ pháp tuyến là $\vec{n}_{(P)}=(1;-2;1)$.\\
		Do $\dfrac{1}{1}=\dfrac{-2}{-2}=\dfrac{1}{1}$ nên $\vec{u}$ cùng phương với $\overrightarrow{n}_{(P)}$. Do đó $d$ vuông góc với $(P)$.\\
		Mặt phẳng $(Q)$ có một vectơ pháp tuyến là $\vec{n}_{(Q)}=(1;-2;-1)$.\\
		Do $\dfrac{1}{1}=\dfrac{-2}{-2} \neq \dfrac{1}{-1}$ nên $\vec{u}$ không cùng phương với $\vec{n}_{(Q)}$. Do đó $d$ không vuông góc với $(Q)$.\\
		Mặt phẳng $(R)$ có một vectơ pháp tuyến là $\vec{n}_{(R)}=(1;1;1)$.\\ Do $\dfrac{1}{1} \neq \dfrac{-2}{1} \neq \dfrac{1}{1}$ nên $\vec{u}$ không cùng phương với $\vec{n}_{(R)}$. Do đó $d$ không vuông góc với $(R)$.
	}
\end{ex}

\begin{ex}%[2H5H2-5]
	Trong KG $Oxyz$, phương trình mặt phẳng đi qua gốc tọa độ và vuông góc với đường thẳng $d\colon \dfrac{x}{1}=\dfrac{y}{1}=\dfrac{z}{1}$ là
	\choice 
		{$x+y+z+1=0$}
		{$x-y-z=1$}
		{$x+y+z=1$}
		{\True $x+y+z=0$}
	\loigiai{
		Mặt phẳng $(P)$ vuông góc với đường thẳng $(d)\colon \dfrac{x}{1}=\dfrac{y}{1}=\dfrac{z}{1}$ nên nhận vectơ chỉ phương $\overrightarrow{u}_d=(1;1;1)$ làm vectơ pháp tuyến.\\
		Suy ra phương trình mặt phẳng $(P)$ có dạng $x+y+z+D=0$.\\
		Mặt khác $(P)$ đi qua gốc tọa độ nên $D=0$.\\
		Vậy phương trình $(P)$ là $x+y+z=0$.
	}
\end{ex}

\begin{ex}%[2H5H2-5]
	Trong không gian với hệ trục $Oxyz$, cho điểm $A(0;0;3)$ và đường thẳng $d$ có phương trình $ \heva{&x=1+2 t \\ &y=1-t \\&z=t}$. Phương trình mặt phẳng đi qua điểm $A$ và vuông góc với đường thẳng $d$ là
	\choice
		{\True $2x-y+z-3=0$}
		{$2x-y+2 z-6=0$}
		{$2x-y+z+3=0$}
		{$2x-y-z+3=0$}
	\loigiai{
		Mặt phẳng cần tìm đi qua điểm $A(0;0;3)$ và vuông góc với đường thẳng $d$ nên nhận vectơ chỉ phương của đường thẳng $d$ là $\vec{u}=(2;-1;1)$ làm vectơ pháp tuyến. Do đó phương trình mặt phẳng cần tìm là $2 x-y+z-3=0$.
	}
\end{ex}

\begin{ex}%[2H5H2-5]
	Trong KG $Oxyz$, cho đường thẳng $\Delta$ có phương trình $\dfrac{x-10}{5}=\dfrac{y-2}{1}=\dfrac{z+2}{1}$. Xét mặt phẳng $(P)\colon 10x+2y+m z+11=0$, với $m$ là tham số thực. Tìm tất cả các giá trị của $m$ để mặt phẳng $(P)$ vuông góc với đường thẳng $\Delta$.
	\choice 
		{\True $m=2$}
		{$m=-52$}
		{$m=52$}
		{$m=-2$}
	\loigiai{
		Đường thẳng $\Delta\colon  \dfrac{x-10}{5}=\dfrac{y-2}{1}=\dfrac{z+2}{1}$ có vectơ chỉ phương $\vec{u}=(5;1;1)$.\\
		Mặt phẳng $(P)\colon 10x+2y+mz+11=0$ có vectơ pháp tuyến $\vec{n}=(10;2;m)$.\\
		Để mặt phẳng $(P)$ vuông góc với đường thẳng $\Delta$ thì $\vec{u}$ phải cùng phương với $\vec{n}$, tức là cần $$\dfrac{10}{2}=\dfrac{2}{1}=\dfrac{m}{1} \Leftrightarrow m=2.$$
	}
\end{ex}

\begin{ex}%[2H5H2-5]
	Trong KG $Oxyz$, cho đường thẳng $\Delta\colon \dfrac{x+1}{-1}=\dfrac{y-2}{2}=\dfrac{z}{-3}$ và mặt phẳng $(P)\colon x-y+z-3=0$. Phương trình mặt phẳng $(\alpha)$ đi qua $O$, song song với $\Delta$ và vuông góc với mặt phẳng $(P)$ là
	\choice 
		{\True $x+2 y+z=0$}
		{$x-2 y+z=0$}
		{$x+2 y+z-4=0$}
		{$x-2 y+z+4=0$}
	\loigiai{
		$\Delta$ có vectơ chỉ phương là $\vec{u}=(-1;2;-3)$ và $(P)$ có vectơ pháp tuyến là $\vec{n}=(1;-1;1)$.\\
		Mặt phẳng $(\alpha)$ qua $O$ và nhận vectơ pháp tuyến là  $\overrightarrow{n'}=-[\vec{u}, \vec{n}]=(1;2;1)$.\\
		Suy ra $(\alpha)\colon x+2y+z=0$.
	}
\end{ex}
\begin{ex}%Câu 1%[2H5H2-7]
	Trong KG $Oxyz$, cho đường thẳng $d_1$ có véc-tơ chỉ phương $\overrightarrow{u}=(1;0;-2)$ và đi qua điểm $M(1;-3;2)$, $d_2\colon\dfrac{x+3}{1}=\dfrac{y-1}{-2}=\dfrac{z+4}{3}$. Phương trình mặt phẳng $(P)$ cách đều hai đường thẳng $d_1$ và $d_2$ có dạng $ax+by+cz+11=0$. Giá trị $ a+2b+3c$ bằng
	\choice
	{$-42$}
	{$-32$}
	{$ 11$}
	{\True $20$}
	\loigiai
	{
		Đường thẳng $d_2$ có véc-tơ chỉ phương $\overrightarrow{v}=(1;-2;3)$ và đi qua điểm $ N(-3;1;-4)$.\\
		Ta có $\left[\overrightarrow{v},\overrightarrow{u}\right]=(4;5;2)\ne\overrightarrow{0}$; $\overrightarrow{MN}=\left(-4;4;-6\right)$; $\left[\overrightarrow{v},\overrightarrow{u}\right]\cdot \overrightarrow{MN}=-16+20-12=-8\ne 0$\\
		$\Rightarrow $ $d_1$ và $d_2$ chéo nhau.\\
		Mặt phẳng $(P)$ cách đều hai đường thẳng $d_1$ và $d_2$ nên $(P)$ nhận $\left[\overrightarrow{v},\overrightarrow{u}\right]=(4;5;2)$ làm một véc-tơ pháp tuyến và đi qua trung điểm $ I(-1;-1;-1)$ của đoạn $ MN$.\\
		Do đó $(P)\colon 4\left(x+1\right)+5\left(y+1\right)+2\left(z+1\right)=0\Leftrightarrow 4x+5y+2z+11=0$.\\
		Suy ra $ a=4$, $b=5$, $c=2$ $\Rightarrow a+2b+3c=20$.}
\end{ex}

\begin{ex}%Câu 2%[2H5H2-7]
	Trong không gian $ Oxyz$, mặt phẳng chứa hai đường thẳng cắt nhau\break $\dfrac{x-1}{-2}=\dfrac{y+2}{1}=\dfrac{z-4}{3}$ và $\dfrac{x+1}{1}=\dfrac{y}{-1}=\dfrac{z+2}{3}$ có phương trình là
	\choice
	{$-2x-y+9z-36=0$}
	{$ 2x-y-z=0$}
	{\True $ 6x+9y+z+8=0$}
	{$ 6x+9y+z-8=0$}
	\loigiai
	{
		Đường thẳng $d_1\colon\dfrac{x-1}{-2}=\dfrac{y+2}{1}=\dfrac{z-4}{3}$ đi qua điểm $ M(1;-2;4)$, có một véc-tơ chỉ phương là $\overrightarrow{u}_1=(-2;1;3)$.\\
		Đường thẳng $d_2\colon\dfrac{x+1}{1}=\dfrac{y}{-1}=\dfrac{z+2}{3}$ có một véc-tơ chỉ phương là $\overrightarrow{u}_2=(1;-1;3)$.\\
		Mặt phẳng $(P)$ chứa hai đường thẳng cắt nhau $d_1$, $d_2$ suy ra $(P)$ qua điểm $ M(1;-2;4)$, có một véc-tơ pháp tuyến là $\overrightarrow{n}=\left[\overrightarrow{u}_1,\overrightarrow{u}_2\right]=(6;9;1)$.\\ Phương trình mặt phẳng $(P)$ là 
		$$(P)\colon 6(x-1)+9(y+2)+(z-4)=0\Leftrightarrow 6x+9y+z+8=0.$$}
\end{ex}

\begin{ex}%Câu 3%[2H5H1-3]
	Trong không gian tọa độ $ Oxyz$, cho điểm $ A(0;1;0),$ mặt phẳng \break $(Q)\colon x+y-4z-6=0$ và đường thẳng $ d\colon\heva{
		& x=3\\ 
		& y=3+t\\ 
		& z=5-t\\ 
	}$. Phương trình mặt phẳng $(P)$ qua $ A$, song song với $ d$ và vuông góc với $(Q)$ là 
	\choice
	{\True $ 3x+y+z-1=0$}
	{$ 3x-y-z+1=0$}
	{$ x+3y+z-3=0$}
	{$ x+y+z-1=0$}
	\loigiai
	{
		Mặt phẳng $(Q)$ có véc-tơ pháp tuyến $\overrightarrow{n}_Q=(1;1;-4)$.\\
		Đường thẳng $ d$ có véc-tơ chỉ phương $\overrightarrow{u}_d=(0;1;-1)$.\\
		Gọi véc-tơ pháp tuyến của mặt phẳng $(P)$ là $\overrightarrow{n}_P$.\\
		Ta có $\overrightarrow{n}_P\perp\overrightarrow{n}_Q$ và $\overrightarrow{n}_P\perp\overrightarrow{u}_d$ nên chọn $\overrightarrow{n}_P=\left[\overrightarrow{n}_Q,\overrightarrow{u}_d\right]=(3;1;1)$.\\
		$(P)$ đi qua điểm $ A(0;1;0)$, nhận véc-tơ pháp tuyến $\overrightarrow{n}_P=(3;1;1)$ có phương trình là $$ 3x+y+z-1=0.$$}
\end{ex}

\begin{ex}%Câu 4%[2H5H1-3]
	Trong KG $Oxyz$, cho hai đường thẳng $d_1\colon\dfrac{x-2}{2}=\dfrac{y-6}{-2}=\dfrac{z+2}{1}$ và $d_2\colon\dfrac{x-4}{1}=\dfrac{y+1}{3}=\dfrac{z+2}{-2}$ chéo nhau. Phương trình mặt phẳng $(P)$ chứa $d_1$ và $(P)$ song song với đường thẳng $d_2$ là
	\choice
	{\True $(P)\colon x+5y+8z-16=0$}
	{$(P)\colon x+5y+8z+16=0$}
	{$(P)\colon x+4y+6z-12=0$}
	{$(P)\colon 2x+y-6=0$}
	\loigiai
	{
		Đường thẳng $d_1$ đi qua $ A\left(2;6;-2\right)$ và có một véc-tơ chỉ phương $\overrightarrow{u}_1=\left(2;-2;1\right)$.\\
		Đường thẳng $d_2$ có một véc-tơ chỉ phương $\overrightarrow{u}_2=(1;3;-2)$.\\
		Gọi $\overrightarrow{n}$ là một véc-tơ pháp tuyến của mặt phẳng $(P)$. Do mặt phẳng $(P)$ chứa $d_1$ và $(P)$ song song với đường thẳng $d_2$ nên $\overrightarrow{n}_P=\left[\overrightarrow{u}_1,\overrightarrow{u}_2\right]=(1;5;8)$.\\
		Vậy phương trình mặt phẳng $(P)$ đi qua $ A(2;6;-2)$ nhận véc-tơ pháp tuyến $\overrightarrow{n}_P=(1;5;8)$ là $ x+5y+8z-16=0$.}
\end{ex}

\begin{ex}%Câu 5%[2H5V1-2]
	Trong không gian với hệ trục tọa độ $Oxyz$ , cho hai điểm $ A(1;1;0)$, $ B(0;-1;2)$. Biết rằng có hai mặt phẳng cùng đi qua hai điểm $ A$, $ O$ và cùng cách $ B$ một khoảng bằng $\sqrt{3}$. Véc-tơ nào trong các véc-tơ dưới đây là một véc-tơ pháp tuyến của một trong hai mặt phẳng đó?
	\choice
	{$\overrightarrow{n}=(1;-1;-1)$}
	{$\overrightarrow{n}=(1;-1;-3)$}
	{\True $\overrightarrow{n}=(1;-1;5)$}
	{$\overrightarrow{n}=(1;-1;-5)$}
	\loigiai
	{
		PTĐT qua hai điểm $ A$, $ O$ có dạng $\heva{
			& x=t\\ 
			& y=t\\ 
			& z=0\\ 
		}\Leftrightarrow\heva{
			& x-y=0\\ 
			& z=0.
		}$\\
		Gọi $(P)$ là mặt phẳng cùng đi qua hai điểm $ A$, $ O$ nên $(P)\colon  m\left(x-y\right)+nz=0$, $m^2+n^2>0$. Khi đó véc-tơ pháp tuyến của $(P)$ có dạng $\overrightarrow{n}=(m;-m;n)$.\\
		Ta có $ \mathrm{d}\left(B,(P)\right)=\sqrt{3}\Leftrightarrow\dfrac{\left| m+2n\right|}{\sqrt{m^2+m^2+n^2}}=\sqrt{3}$ $\Leftrightarrow 2m^2-4mn-n^2=0\Leftrightarrow\hoac{
			&\dfrac{m}{n}=1\\ 
			&\dfrac{m}{n}=\dfrac{1}{5}.
		}$\\
		Vậy một véc-tơ pháp tuyến của một trong hai mặt phẳng đó là $$\overrightarrow{n}_P=\left(\dfrac{1}{5}n; \dfrac{-1}{5}n; n\right)=\dfrac{n}{5}\left(1; -1; 5\right).$$
		Do đó $\overrightarrow{n}=(1;-1;-5)$ cũng là một véc-tơ pháp tuyến của một trong hai mặt phẳng đó.}
\end{ex}

\begin{ex}%Câu 6%[2H5H1-3]
	Trong không gian tọa độ $ Oxyz$, cho điểm $A(1;0;0)$ và đường thẳng \break $ d\colon\dfrac{x-1}{2}=\dfrac{y+2}{1}=\dfrac{z-1}{2}$. Phương trình mặt phẳng chứa điểm $A$ và đường thẳng $d$ là
	\choice
	{$(P)\colon 5x+2y+4z-5=0$}
	{$(P)\colon 2x+1y+2z-1=0$}
	{\True $(P)\colon 5x-2y-4z-5=0$}
	{$(P)\colon 2x+1y+2z-2=0$}
	\loigiai
	{
		Véc-tơ chỉ phương của $ d$ là $\overrightarrow{a}=(2; 1; 2)$ và $ B(1; -2; 1)\in d$.\\
		Khi đó $\overrightarrow{AB}=(0; -2; 1)$.\\
		Do đó véc-tơ pháp tuyến của mặt phẳng là $\overrightarrow{n}=\left[\overrightarrow{AB},\overrightarrow{a}\right]=(5,-2;-4)$.\\
		Từ đó suy ra phương trình mặt phẳng cần tìm là $$ 5\cdot(x-1)-2\cdot(y-0)-4\cdot(z-0)=0\Rightarrow 5x-2y-4z-5=0.$$}
\end{ex}

\begin{ex}%Câu 7%[2H5V1-3]
	Trong KG $Oxyz$ , cho hai đường thẳng $d_1\colon\dfrac{x-2}{-1}=\dfrac{y}{1}=\dfrac{z}{1}$ và $d_2\colon\dfrac{x}{-2}=\dfrac{y-1}{1}=\dfrac{z-2}{1}$. Phương trình mặt phẳng $(P)$ song song và cách đều hai đường thẳng $d_1$, ${d_2}$ là
	\choice
	{\True $ 2y-2z+1=0$}
	{$ 2y-2z-1=0$}
	{$2x-2z+1=0$}
	{$2x-2z-1=0$}
	\loigiai
	{
		Ta có đường thẳng $d_1$ đi qua điểm $ A(2;0;0)$ có véc-tơ chỉ phương là $\overrightarrow{u}_1=(-1;1;1)$ và đường thẳng $d_2$ đi qua điểm $ A(0;1;2)$ có véc-tơ chỉ phương là $\overrightarrow{u}_1=(-2; 1; 1)$.\\
		Mặt phẳng $(P)$ song song $d_1$, ${d_2}$ nên $(P)$ có véc-tơ pháp tuyến là $\overrightarrow{n}=\left[\overrightarrow{u}_1,\overrightarrow{u}_2\right]=(0;-1;1)$.\\
		Do đó mặt phẳng $(P)$ có dạng $ y-z+m=0$.\\
		Mặt khác $(P)$ cách đều hai đường thẳng $d_1$, ${d_2}$ nên\\
		$$ \mathrm{d}\left(d_1,(P)\right)=\mathrm{d}\left(d_2,(P)\right)\Leftrightarrow \mathrm{d}\left(A,(P)\right)=d\left(B;(P)\right)\Leftrightarrow\left| m\right|=\left| m-1\right|\Leftrightarrow m=\dfrac{1}{2}.$$
		Vậy $(P)\colon y-z+\dfrac{1}{2}=0\Leftrightarrow 2y-2z+1=0$.}
\end{ex}
\Closesolutionfile{ans}
% \indapan{10}{ans/ans-C5B2CD5}
\TNSA
\Opensolutionfile{ans}[ans/ans-C5B2CD5-KQ]
\begin{ex}%Câu 8%[2H5H1-3]
	Trong KG $Oxyz$, cho điểm $M(2; -2; 3)$ và đường thẳng \break $d\colon \dfrac{x-1}{3}=\dfrac{y+2}{2}=\dfrac{z-3}{-1}$. Phương trình mặt phẳng đi qua điểm $M$ và vuông góc với đường thẳng $d$ có dạng $3x+by+cz+d=0$. Tính $b^2+cd$.
	\shortans{$3$}
	\loigiai{
		Gọi $(P)$ là mặt phẳng đi qua $M$ và vuông góc với đường thẳng $d$.\\
		Ta có $\overrightarrow{n}_p=\overrightarrow{u}_d=(3; 2;-1)$ là một véc-tơ pháp tuyến của mặt phẳng $(P)$.\\
		Phương trình mặt phẳng $(P)$ là $$3(x-2)+2(y+2)-1(z-3)=0 \Leftrightarrow 3 x+2 y-z+1=0.$$
		Vậy $b^2+cd=2^2+(-1)\cdot 1=3$.}
\end{ex}

\begin{ex}%Câu 9%[2H5H1-3]
	Trong KG $Oxyz$, phương trình mặt phẳng đi qua điểm $A(0; 1; 0)$ và chứa đường thẳng $\Delta\colon \dfrac{x-2}{1}=\dfrac{y-1}{-1}=\dfrac{z-3}{1}$ có dạng $3x+ay+bz-c$. Tính $a+b+c$.
	\shortans{$0$}
	\loigiai{
		Ta lấy điểm $M(2; 1; 3) \in(\Delta) \Rightarrow\heva{&\overrightarrow{A M}=(2; 0; 3) \\& \text { véc-tơ chỉ phương } \overrightarrow{u}_{\Delta}=(1; -1; 1).}$\\
		Suy ra $\overrightarrow{n}=\left[\overrightarrow{A M}, \overrightarrow{u}_{\Delta}\right]=(3; 1;-2)$.\\
		Mặt phẳng cần tìm qua $A(0; 1; 0)$ và nhận $\overrightarrow{n}=(3; 1;-2)$ làm véc-tơ pháp tuyến có phương trình là  $$3\cdot(x-0)+1 \cdot(y-1)-2 \cdot(z-0)=0 \Leftrightarrow 3 x+y-2 z-1=0.$$
		Suy ra $a=1$, $b=-2$, $c=1$. Vậy $a+b+c=0$.}
\end{ex}

\begin{ex}%Câu 10%[2H5H1-3]
	Trong không gian với hệ trục tọa độ $Oxyz$, cho điểm $A(-1; 3; 2)$ và đường thẳng $d$ có phương trình $\heva{&x=1-4 t \\ &y=t \\& z=2+t}$. Phương trình mặt phẳng $(P)$ chứa điểm $A$ và vuông góc đường thẳng $d$ có dạng $ax+by+10z+c=0$. Tính $c$.
	\shortans{$-23$}
	\loigiai{
		Đường thẳng $d$ đi qua điểm $M(1; 0; 2)$ và có véc-tơ chỉ phương $\overrightarrow{u}=(-4; 1; 1)$.\\
		Ta có $\overrightarrow{A M}=(2; -3; 0)$, $[\overrightarrow{u },\overrightarrow{AM}]=(3; 2; 10)$.\\
		Mặt phẳng $(P)$ chứa điểm $A$ và đường thẳng $d$ có véc-tơ pháp tuyến $[\overrightarrow{u}, \overrightarrow{AM}]=(3; 2; 10)$.\\
		Do đó phương trình mặt phẳng $(P)$ là $$3(x+1)+2(y-3)+10(z-2)=0 \Leftrightarrow 3 x+2 y+10 z-23=0.$$\\
		Vậy $c=-23$.}
\end{ex}

\begin{ex}%Câu 11%[2H5V1-3]
	Trong KG $Oxyz$, phương trình mặt phẳng $(P)$ song song và cách đều hai đường thẳng $d_1\colon\dfrac{x-2}{-1}=\dfrac{y}{1}=\dfrac{z}{1}$ và $d_2\colon\dfrac{x}{2}=\dfrac{y-1}{-1}=\dfrac{z-2}{-1}$ có dạng $ax+by+cz+1=0$. Tính $a^2+b^2+c^2$.
	\shortans{$8$}
	\loigiai{
		Ta có $d_1$ đi qua điểm $ A(2;0;0)$ và có véc-tơ chỉ phương $\overrightarrow{u}_1=(-1;1;1)$,
		$d_2$ đi qua điểm $ B(0;1;2)$ và có véc-tơ chỉ phương $\overrightarrow{u}_2=(2;-1;-1)$.\\
		Vì $(P)$ song song với hai đường thẳng $d_1$ và $d_2$ nên véc-tơ pháp tuyến \break của $(P)$ là $\overrightarrow{n}=[\overrightarrow{u}_1,\overrightarrow{u}_2]=(0;1;-1)$.\\
		Vì $(P)$ cách đều $d_1$ và $d_2$ nên $(P)$ đi qua trung điểm $ M\left(0;\dfrac{1}{2};1\right)$ của $ AB$\break  nên  $(P)\colon2y-2z+1=0$.\\
		Suy ra $a=0$, $b=2$, $c=-2$. Vậy $a^2+b^2+c^2=8$.}
\end{ex}

\begin{ex}%Câu 12%[2H5H1-3]
	Trong KG $Oxyz$, phương trình mặt phẳng chứa hai đường thẳng $d\colon\heva{
		& x=t+2\\ 
		& y=3t-1\\ 
		& z=2t+1\\ 
	}$ và $\Delta\colon\heva{
		& x=m+3\\ 
		& y=3m-2\\ 
		& z=2m+1\\ 
	}$ có dạng $x+ay+bz+c=0$. Tính $P=a+2b+3c$.
	\shortans{$0$}
	\loigiai{
		Ta có $d\parallel\Delta $.\\
		Chọn $A(2;-1;1)\in d$, $B(3;-2;1)\in \Delta$ suy ra $\overrightarrow{AB}=(1;-1;0)$.\\
		Ta có $\left[\overrightarrow{AB},\overrightarrow{u}_{d}\right]=(-2;-2;4)$.\\
		Phương trình mặt phẳng chứa hai đường thẳng $d$ và $\Delta$ qua $A(2;-1;1)$ và có véc-tơ pháp tuyến $\overrightarrow{n}=-\dfrac{1}{2}\left[\overrightarrow{AB},\overrightarrow{u}_{d}\right]=(1;1;-2)$ là
		$$1\cdot(x-2)+1\cdot(y+1)-2\cdot(z-1)=0\Leftrightarrow x+y-2z+1=0.$$
		Vậy  $P=a+2b+3c=1-2\cdot 2+3\cdot 1=0$.}
\end{ex}
%
\begin{ex}%Câu 13%[2H5V1-3]
	Trong KG $Oxyz$, cho hai đường thẳng cắt nhau $$d\colon\dfrac{x-1}{-2}=\dfrac{y+2}{1}=\dfrac{z-4}{3} \text{ và } d'\colon \heva{&x=-1+t\\&y=-t\\&z=-2+3t.}$$ 
	Phương trình mặt phẳng $(P)$ chứa $d$ và $d'$ có dạng $ax+by+cz+8=0$. Tính $T=a-b+3c$.\\
	\shortans{$0$}
	\loigiai{
		Ta có $d$ có véc-tơ chỉ phương $\overrightarrow{u}=(-2;1;3)$ và đi qua $M(1;-2;4)$,\\
		$d'$ có véc-tơ chỉ phương $\overrightarrow{u}'=(1;-1;3)$ và đi qua $M'(-1;0;-2)$.\\
		Từ đó $\overrightarrow{MM'}=(-2;2;-6)$,
		$[\overrightarrow{u},\overrightarrow{u'}]=(6;9;1)\ne\overrightarrow{0}$ và $[\overrightarrow{u},\overrightarrow{u}']\cdot \overrightarrow{MM'}=0$.\\
		Suy ra $d$ cắt $d'$.\\
		Mặt phẳng $(P)$ chứa $d$ và $d'$ đi qua giao điểm của $d$ và $d'$ có véc-tơ pháp tuyến $\overrightarrow{n}=[\overrightarrow{u},\overrightarrow{u'}]$\\
		Gọi $I=d\cap d'$, giả sử $I(-1+t;-t;-2+3t)\in d'$ mà $I\in d$ do đó
		$$\begin{aligned}
			&\dfrac{-1+t-1}{-2}=\dfrac{-t+2}{1}=\dfrac{-2+3t-4}{3}\\ 
			\Leftrightarrow &\dfrac{-2+t}{-2}=\dfrac{-t+2}{1}=\dfrac{-6+3t}{3}\\ 
			\Leftrightarrow & t=2.
		\end{aligned}$$
		Vậy $I(1;-2;4)$.\\
		Khi đó ta có $(P)$ đi qua $I(1;-2;4)$ và có véc-tơ pháp tuyến $\overrightarrow{n}=[\overrightarrow{u},\overrightarrow{u'}]=(6;9;1)$\\
		Phương trình mặt phẳng $(P)$ là
		$$6\cdot(x-1)+9\cdot(y+2)+(z-4)=0\Leftrightarrow 6x+9y+z+8=0.$$ 
		Suy ra $\heva{
			& a=6\\ 
			& b=9\\ 
			& c=1}\Rightarrow T=a-b+3c=6- 9+3\cdot 1=0.$}
\end{ex}

\begin{ex}%Câu 14%[2H5V1-3]
	Trong không gian với hệ tọa độ $ Oxyz$, cho hai điểm $ A(3;1;7)$, $B(5;5;1)$ và mặt phẳng $(P)\colon 2x-y-z+4=0$. Điểm $ M$ thuộc $(P)$ sao cho $ MA=MB=\sqrt{35}.$ Biết $ M$ có hoành độ nguyên, tính $ OM$.(Kết quả làm tròn đến hàng phần mười).
	\shortans{$2{,}8$}
	\loigiai{
		Ta có  $\overrightarrow{AB}=(2;4;-6)=2(1;2;-3)$.\\
		Gọi $ I(4;3;4)$ là trung điểm của $ AB$\\
		Phương trình mặt phẳng trung trực $(Q)$ của $ AB$ là $$(x-4)+2(y-3)-3(z-4)=0 \Leftrightarrow x+2y-3z+2=0.$$
		Gọi $ d=(P)\cap(Q)$. Đường thẳng $ d$ có $ 1$ véc-tơ chỉ phương là $\overrightarrow{u}=\left[\overrightarrow{n_{(P)}},\overrightarrow{n_{(Q)}}\right]=(1;1;1)$ và đi qua điểm $ N(-2;0;0)$, có phương trình là $ d\colon\heva{
			& x=-2+t\\ 
			& y=t\\ 
			& z=t.
		}$\\
		Gọi $ M\in(P)\colon MA=MB$. Khi đó $ M\in d$ và $ M(-2+t;t;t)$.\\
		Theo giả thiết, ta có  
		$$\begin{aligned}
			MA=\sqrt{35}&\Leftrightarrow\sqrt{\left(t-5\right)^2+\left(t-1\right)^2+\left(t-7\right)^2}=\sqrt{35}\\&\Leftrightarrow 3t^2-26t+40=0 \\&\Leftrightarrow\hoac{
				& t=\dfrac{20}{3}\\ 
				& t=2. 	}
		\end{aligned} $$
		Vì $M$ có hoành độ nguyên nên $t=2$ suy ra $M=(0; 2; 2)$.\\
		Vậy $ OM=2\sqrt{2}\approx 2{,}8$.}
\end{ex}

\begin{ex}%Câu 15%[2H5V2-3]
	Trong KG $Oxyz$, cho ba đường thẳng $d\colon\dfrac{x}{1}=\dfrac{y}{1}=\dfrac{z+1}{-2}$,\break  $\Delta_1\colon\dfrac{x-3}{2}=\dfrac{y}{1}=\dfrac{z-1}{1}$, $\Delta_2\colon\dfrac{x-1}{1}=\dfrac{y-2}{2}=\dfrac{z}{1}$. Đường thẳng $\Delta $ vuông góc với $d$ đồng thời cắt $\Delta_1$, $\Delta_2$ tương ứng tại $H$, $K$ sao cho độ dài $HK$ nhỏ nhất. Biết rằng $\Delta $ có một véc-tơ chỉ phương $\overrightarrow{u}=(h; k; 1)$. Tính giá trị $h-k$.
	\shortans{$0$}
	\loigiai{
		Vì $H\in{\Delta_1}\Leftrightarrow H(3+2t; t; 1+t)$, 
		$K\in{\Delta_2}\Leftrightarrow K(1+m; 2+2m; m)$.\\
		Ta có $\overrightarrow{HK}=(m-2t-2; 2m-t+2; m-t-1)$.\\
		Đường thẳng $d$ có một véc-tơ chỉ phương là $\overrightarrow{u}_d=(1;1;-2)$.\\
		$\Delta\perp d\Leftrightarrow $ $\overrightarrow{u}_d\cdot \overrightarrow{HK}=0$ $\Leftrightarrow m-t+2=0\Leftrightarrow m=t-2\Rightarrow\overrightarrow{HK}=(-t-4; t-2; -3)$.\\
		Ta có $HK^2=(-t-4)^2+(t-2)^2+(-3)^2=2(t+1)^2+27\ge 27$, $\forall t\in\mathbb{R}$.\\
		Suy ra $\min HK=\sqrt{27}$, đạt được khi $t=-1$.\\
		Khi đó ta có $\overrightarrow{HK}=(-3;-3;-3)$, suy ra $\overrightarrow{u}=(1;1;1)\Rightarrow h=k=1\Rightarrow h-k=0$.}
\end{ex}

\begin{ex}%Câu 16%[2H5V2-6]
	Trong không gian $ Oxyz$, cho hai điểm $ A(3; 1; 2)$, $ B(-3;-1;0)$ và mặt phẳng $(P)\colon x+y+3z-14=0$. Điểm $ M$ thuộc mặt phẳng $(P)$ sao cho $\Delta MAB$ vuông tại $M$. Tính khoảng cách từ điểm $M$ đến mặt phẳng $(Oxy)$.\\
	\shortans{$4$}
	\loigiai{
		Gọi $ M(x;y;z)$ là điểm cần tìm.\\
		Suy ra $\overrightarrow{AM}=(x-3;y-1;z-2)$, $\overrightarrow{BM}=(x+3; y+1; z)$.\\
		Vì $\Delta MAB$ vuông tại $ M$ nên $\overrightarrow{AM}\cdot \overrightarrow{BM}=0$. Suy ra $$\begin{aligned}
			& (x-3)(x+3)+(y-1)(y+1)+z(z-2)=0\\\Leftrightarrow &x^2-9+y^2-1+z^2-2z=0\\\Leftrightarrow& x^2+y^2+(z-1)^2=11.
		\end{aligned}$$
		Do đó $ M$ thuộc mặt cầu $(S)$ có tâm $ I(0; 0; 1)$ và bán kính $ R=\sqrt{11}$.\\
		Nhận xét thấy $ \mathrm{d}(I,(P))=\dfrac{\left| 0+0+3\cdot 1-14\right|}{\sqrt{1^2+1^2+3^3}}=\sqrt{11}=R$.\\
		$\Rightarrow(P)$ tiếp xúc với $(S)$ tại $ M$\\
		$\Rightarrow M$ là hình chiếu vuông góc của $ I$ trên $(P)$\\
		$\Rightarrow\heva{
			& M\in(P)\\ 
			&\overrightarrow{IM} \text{ cùng phương với }\overrightarrow{n_{(P)}}\\ 
		}\Rightarrow\heva{
			& x+y+3z=14\\ 
			&\dfrac{x}{1}=\dfrac{y}{1}=\dfrac{z-1}{3}\\ 
		}\Rightarrow\heva{
			& x=1\\ 
			& y=1\\ 
			& z=4\\ 
		}\Rightarrow M(1; 1;4).$\\
		Vậy $ d\left(M,(Oxy)\right)=\left| 4\right|=4$.}
\end{ex}

\begin{ex}%Câu 17%[2H5V2-6]
	
	Trong không gian với hệ tọa độ $ Oxyz$, cho đường thẳng $ d\colon\dfrac{x-5}{2}=\dfrac{y+7}{2}=\dfrac{z-12}{-1}$ và mặt phẳng $(\alpha)\colon x+2y-3z-3=0$. Gọi $ M$ là giao điểm của $ d$ và $(\alpha)$, $ A$ thuộc $ d$ sao cho $ AM=\sqrt{14}$. Tính khoảng cách từ $ A$ đến mặt phẳng $(\alpha)$.\\
	\shortans{$3$}
	\loigiai{
		\begin{center}
			\begin{tikzpicture}[line cap=round,line join=round, >=stealth,scale=1]
				\def \a{-1} \def \b{3} \def \c{6} \def \h{4}
				\path (0,0)coordinate(A')
				++(120:\b)coordinate(D)
				++(0:\c)coordinate(C)
				($(C)+(A')-(D)$)coordinate(B)
				($(A')+(60:1)$) coordinate (I)
				($(I)+(0:4.5)$) coordinate (J)
				($(I)!0.2!(J)$) coordinate (H)
				($(I)!0.7!(J)$) coordinate (M)
				($(H)+(90:\b)$) coordinate (A)
				($(M)!1.5!(A)$) coordinate (K)
				($(M)!1.4!(A)$) coordinate (K') node [right] {$d$}
				;
				\draw  (B)--(A')--(D)--(C)--cycle (I)--(J) (A)--(H) (M)--(K) 
				;
				\draw pic[angle radius=2mm,draw=black] {right angle = A--H--M};
				\foreach \x/\g in {A/30,H/-90,M/-90}\fill[draw] (\x) circle (1pt)+(\g:3mm) node[black]{$\x$};
			\end{tikzpicture}
		\end{center}
		Đường thẳng $ d\colon\dfrac{x-5}{2}=\dfrac{y+7}{2}=\dfrac{z-12}{-1}$ có một véc-tơ chỉ phương là $\overrightarrow{u}=(2; 2; -1)$.\\
		Mặt phẳng $(\alpha)\colon x+2y-3z-3=0$ có một véc-tơ pháp tuyến là $\overrightarrow{n}=(1; 2; -3)$.\\
		Ta có $\sin(d, (\alpha))=\dfrac{\left|\overrightarrow{u}_d\cdot\overrightarrow{n}_{\alpha}\right|}{\left|\overrightarrow{u}_d\right|\cdot \left|\overrightarrow{n}_{\alpha}\right|}=\dfrac{3\sqrt{14}}{14}$ .\\
		Gọi $H$ là hình chiếu vuông góc của $ A$ lên mặt phẳng $(\alpha)$.\\
		Khi đó tam giác $\Delta MAH$ vuông tại $H$ nên $\sin\left(d,(\alpha)\right)=\sin\widehat{AMH}=\dfrac{AH}{AM}$.\\
		$\Rightarrow AH=AM\cdot \sin\left(d,(\alpha)\right)=3$.\\
		Vậy khoảng cách từ $A$ đến mặt phẳng $(\alpha)$ bằng $3$.}
\end{ex}
\Closesolutionfile{ans}
% \indapan{6}{ans/ans-C5B2CD5-KQ}