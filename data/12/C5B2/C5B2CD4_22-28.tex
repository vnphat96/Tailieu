\Opensolutionfile{ans}[ans/ans-C5B2CD4-KQ]
\TNSA
\begin{ex}%[2H5V1-6] 
	Cho hình lập phương $ABCD.A'B'C'D'$ có tâm $O$. Gọi $I$ là tâm cùa hình vuông $A'B'C'D'$ và điểm $M$ thuộc đoạn $OI$ sao cho $MO=2MI$ (tham khảo hình vẽ).
	\begin{center}
		\begin{tikzpicture}
			\def\a{3}
			\def\b{1}
			\def\g{35}
			\def\h{3}
			\path (0:0) coordinate (A)--++(\g:\b) coordinate (B)--++(0:\a) coordinate (C)--++(\g-180:\b) coordinate (D)
			\foreach \x in {A,B,C,D}{($(\x)-(90:\h)$) coordinate (\x')};
			\coordinate (O) at ($(A')!.5!(C)$);
			\coordinate (I) at ($(A')!.5!(C')$);
			\coordinate (M) at ($(O)!.66!(I)$);
			\foreach \x/\g in {A/120,B/150,C/30,D/130,A'/140,B'/150,C'/-45,D'/-30,O/90,M/45,I/180} 
			\fill[black](\x) circle (1pt) ($(\x)+(\g:3mm)$) node{$\x$};
			\draw[dashed] (A')--(B')--(C') (B)--(B') (A)--(M)--(B) (C')--(M)--(D') (O)--(I);
			\draw (D')--(A')--(A)--(B)--(C)--(C')--(D')--(D)--(A) (D)--(C);
		\end{tikzpicture}
	\end{center} 
	Tính sin của góc tạo bởi hai mặt phẳng $\left(MC'D'\right)$ và $(MAB)$ (kết quả viết ở dạng thập phân làm tròn đến hàng phần trăm).
	\shortans{$0{,}65$}
	\loigiai{
		\begin{center}
			\begin{tikzpicture}
				\def\a{3}
				\def\b{1}
				\def\g{35}
				\def\h{3}
				\path (0:0) coordinate (A)--++(\g:\b) coordinate (B)--++(0:\a) coordinate (C)--++(\g-180:\b) coordinate (D)
				\foreach \x in {A,B,C,D}{ ($(\x)-(90:\h)$) coordinate (\x')};
				\coordinate (O) at ($(A')!.5!(C)$);
				\coordinate (I) at ($(A')!.5!(C')$);
				\coordinate (M) at ($(O)!.66!(I)$);
				\foreach \x/\g in {A/120,B/150,C/30,D/130,A'/140,B'/150,C'/-45,D'/-30,O/90,M/45,I/180}
				\fill[black](\x) circle (1pt)
				($(\x)+(\g:3mm)$) node{$\x$};
				\draw[dashed] (A')--(B')--(C') (B)--(B') (A)--(M)--(B) (C')--(M)--(D') (O)--(I);
				\draw (D')--(A')--(A)--(B)--(C)--(C')--(D')--(D)--(A) (D)--(C) (D)--(C')
				;
				\foreach \diem/\anh/\ts in {A'/x/1.8,C'/y/1.4,B/z/1.4} \coordinate[label = below right:$\anh$] (\anh) at ($(B')!\ts!(\diem)$);
				\draw[-stealth] (A')--(x); \draw[-stealth](C')--(y); \draw[-stealth](B)--(z);
			\end{tikzpicture}
		\end{center}
		Gắn hệ trục tọa độ như hình vẽ, cạnh hình lập phương là $6$, ta được tọa độ các điểm như sau  $C'(0;6;0)$, $D'(6;6;0)$, $A(6;0;6)$, $B(0; 0;6)$,  $O\left( 3;3;3\right)$, $I\left(3;3;0\right)$ và  $M\left(3;3;1\right)$.\\
		Lúc đó $\overrightarrow{MC'}=\left(-3;3;-1\right)$, $\overrightarrow{MD'}=\left(3;3;-1\right)$, $\overrightarrow{MA}=\left(3;-3;5\right)$ và $\overrightarrow{MB}=\left(-3;-3;5\right)$.\\
		Ta có $\left[ \overrightarrow{MC'},\overrightarrow{MD'}\right]=-6\left(0;1;3\right)$. Suy ra mặt phẳng $(MC'D')$ có  một vectơ pháp tuyến  là
		$\vec{n}_{\left(MC'D'\right)}=(0;1;3)$.\\
		Lại có $\left[ \overrightarrow{MA},\overrightarrow{MB}\right]=-6\left(0;5;3\right)$. Suy ra mặt phẳng $(MAB)$ có một vectơ pháp tuyến  là
		$\vec{n}_{(MAB)}=(0;5;3)$. \\
		Suy ra $\cos \widehat{\left((MAB),(MC'D')\right)}=\dfrac{|5\cdot 1+3\cdot 3|}{\sqrt{5^2+3^2} \cdot \sqrt{1^2+3^2}}=\dfrac{7 \sqrt{85}}{85}$.\\
		Từ đó có $\sin \widehat{\left((MAB),(MC'D')\right)}=\sqrt{1-\left(\dfrac{7 \sqrt{85}}{85}\right)^2}=\dfrac{6 \sqrt{85}}{85}$.
	}
\end{ex}

\begin{ex}%[2H5V2-7] 
	Cho hình lăng trụ $ABC.A'B'C'$ có đáy $ABC$ là tam giác vuông tại $A$, $AB=a$, $AC=a\sqrt{3}$. Hình chiếu vuông góc của $A'$ lên mặt phẳng $(ABC)$ là trung điểm $H$ của $BC$, $A'H=a\sqrt{5}$. Gọi $\varphi$ là góc giữa hai đường thẳng $A'B$ và $B'C$. Tính $\cos \varphi$. Kết quả viết ở dạng thập phân làm tròn đến hàng phần trăm.
	\shortans{ $0{,}51$}
	\loigiai{
		\begin{center}
			\begin{tikzpicture}
				\def\a{5}
				\def\b{3}
				\def\g{-160}
				\def\h{4.5}
				\path (0:0) coordinate (A)--++(0:\a) coordinate (C)--++(\g:\b) coordinate (B)--++(180+\g:\b/2) coordinate (H)--++(90:\h) coordinate (A')--++(0:\a) coordinate (C')--++(\g:\b) coordinate (B');
				\draw[dashed]  (A')--(B) (A)--(C) (A')--(H)--(A)  (A')--($(A)+(90:\h)$) coordinate (D);
				\draw (A')--(B')--(C')-- (A')--(A)--(B)-- (C)--(C')  (B)--(B')--(C);
				\draw[gray] ($(A)!.15!(B)$) coordinate (E) ($(A)!.15!(C)$) coordinate (F) (E)--($(E)+(F)-(A)$)-- (F);
				\draw[-stealth] (A)--++(90:5) node[right]{$z$};
				\draw[-stealth] (B)--($(B)!-.3!(A)$) node[right]{$x$};
				\draw[-stealth] (C)--($(C)!-.3!(A)$) node[right]{$y$};
				\foreach \x/\g in {A/180,B/-120,C/-90,A'/120,B'/-45,C'/30,H/-40,D/180}	\fill[black](\x) circle (1pt) ($(\x)+(\g:3mm)$) node{$\x$};
			\end{tikzpicture}
		\end{center}
		Ta chọn hệ trục tọa độ $Oxyz$ với $O \equiv A$ như hình vẽ, chọn $a=1$ đơn vị, khi đó ta có tọa độ điểm $B(1;0;0)$, $C(0;\sqrt{3};0)$, suy ra trung điểm của $BC$ là $H\left(\dfrac{1}{2};\dfrac{\sqrt{3}}{2};0\right)$.\\
		Vì $H$ là hình chiếu của $A'$ nên suy ra tọa độ của $A'\left(\dfrac{1}{2};\dfrac{\sqrt{3}}{2};\sqrt{5}\right)$.\\ Ta tìm tọa độ $B'$.\\
		Gọi tọa độ $B'(x;y;z)$ khi đó ta có $\overrightarrow{A'B'}=\overrightarrow{OB}$ nên tọa độ $B'\left(\dfrac{3}{2};\dfrac{\sqrt{3}}{2};\sqrt{5}\right)$.\\
		Ta cũng có $\overrightarrow{B'C}=\left(-\dfrac{3}{2};\dfrac{\sqrt{3}}{2};-\sqrt{5}\right)$ và $\vec{A'B}=\left(\dfrac{1}{2};-\dfrac{\sqrt{3}}{2} ;-\sqrt{5}\right)$.\\
		Từ đó ta có $\cos \varphi=\dfrac{\left|\overrightarrow{A'B} \cdot \overrightarrow{B'C}\right|}{\left|\overrightarrow{A' B}\right| \cdot \left| \overrightarrow{B'C}\right|}=\dfrac{7}{2 \cdot \sqrt{6} \cdot \sqrt{8}}=\dfrac{7 \sqrt{3}}{24}$.
	}
\end{ex}

\begin{ex}%[2H5V1-6] 
	Cho hình hộp chữ nhật $ABCD.A'B'C'D'$, có $AB=a$, $AD=a\sqrt{2}$, góc giữa $A'C$ và mặt phẳng $(ABCD)$ bằng $30^{\circ}$. Gọi $H$ là hình chiếu vuông góc của $A$ trên $A'B$ và $K$ là hình chiếu vuông góc của $A$ trên $A'D$. Góc giữa hai mặt phẳng $(AHK)$ và $\left(ABB'A'\right)$ bằng bao nhiêu độ?
	\shortans{$45$}
	\loigiai{
		\begin{center}
			\begin{tikzpicture}
				\def\a{5}
				\def\b{2}
				\def\g{40}
				\def\h{3}
				\path (0:0) coordinate (D)--++(\g:\b) coordinate (A)--++(0:\a) coordinate (B)--++(\g-180:\b) coordinate (C)
				\foreach \x in {A,B,C,D}{($(\x)-(90:\h)$) coordinate (\x')};			
				\draw[dashed] (A)--(A')--(B') (A')--(D') (B)--(A')--(D) (A)--($(D)!.3!(A')$) coordinate (K)--($(B)!.7!(A')$)coordinate (H)--(A);
				\draw[dashed] (C')--(A')--(C) ($(D')!.3!(A')$) coordinate (E)--(K)--($(A)!.3!(A')$) coordinate (F) (K)--(H);
				\draw (A)--(C) (D')--(D)--(A)--(B)--(C)--(C')--(B')--(B) (D')--(C') (D)--(C);
				\foreach \goc/\t/\tt/\ts/\tss in {H/B/A/.08/.12,K/D/A/.3/.15,F/K/A/.15/.3,E/D'/K/.3/.15}
				\draw[gray] ($(\goc)!\ts!(\t)$) coordinate (X) ($(\goc)!\tss!(\tt)$) coordinate (Y) (X)--($(X)+(Y)-(\goc)$)-- (Y);
				\foreach \diem/\anh/\ts in {D'/x/1.5,B'/y/1.4,A/z/1.4} \coordinate[label = below right:$\anh$] (\anh) at ($(A')!\ts!(\diem)$);
				\draw[-stealth] (D')--(x); \draw[-stealth](B')--(y); \draw[-stealth](A)--(z);
				\foreach \x/\g in {A/140,B/70,C/-30,D/130,A'/-60,B'/-60,C'/-45,D'/-60,K/-130,H/-90,E/-20,F/0}
				\fill[black](\x) circle (1pt)
				($(\x)+(\g:3mm)$) node{$\x$};
			\end{tikzpicture}
		\end{center}
		Do $ABCD.A'B'C'D'$ là hình hộp chữ nhật nên $A'C'$ là hình chiếu vuông góc của $A'C$ trên $(ABCD)$. Suy ra $$\left(A'C,(ABCD)\right)=\left(A'C,A'C'\right)=\widehat{CA'C'}=30^{\circ}.$$
		Ta có $AC=\sqrt{AB^2+AD^2}=a\sqrt{3}$ và $\tan \widehat{CA'C'}=\dfrac{CC'}{A'C'} \Rightarrow CC'=a$.\\
		Kết hợp với giả thiết ta được $ABB'A'$ là hình vuông và có $H$ là tâm.\\
		Gọi $E$, $F$ lần lượt là hình chiếu vuông góc của $K$ trên $A'D'$ và $A'A$. Ta có $$\dfrac{1}{AK^2}=\dfrac{1}{A'A^2}+\dfrac{1}{AD^2} \Rightarrow AK=\dfrac{a \sqrt{6}}{3}, A'K=\sqrt{A'A^2-AK^2}=\dfrac{a}{\sqrt{3}}$$
		và 
		$$
		\dfrac{1}{K F^2}=\dfrac{1}{K A^2}+\dfrac{1}{A' K^2} \Rightarrow K F=\dfrac{a \sqrt{2}}{3}, K E=\sqrt{A' K^2-K F^2} \Rightarrow K E=\dfrac{a}{3}.
		$$
		Ta chọn hệ trục tọa độ $Oxyz$ thỏa mãn $O \equiv A'$ còn $D'$, $B'$, $A$ theo thứ tự thuộc các tia $Ox$, $Oy$, $Oz$.\\
		Khi đó ta có tọa độ các điểm lần lượt là $A(0;0;a)$, $B'(0;a;0)$, $H\left(0;\dfrac{a}{2};\dfrac{a}{2}\right)$, $K\left(\dfrac{a\sqrt{2}}{3};0;\dfrac{a}{3}\right)$, $E\left(\dfrac{a\sqrt{2}}{3};0;0\right)$, $F\left(0;0;\dfrac{a \sqrt{2}}{3}\right)$.\\
		Mặt phẳng $\left(ABB'A'\right)$ là mặt phẳng $( Oyz)$ nên có vectơ  pháp tuyến là $\vec{n}_1=(1;0;0)$.\\
		Ta có $\left[ \overrightarrow{AK},\overrightarrow{AH}\right] =\dfrac{a^2}{6} \vec{n}_2$, với $\vec{n}_2(2 ; \sqrt{2};\sqrt{2})$.\\
		Mặt phẳng $(AKH)$ có vectơ  pháp tuyến là $\vec{n}_2=(2 ;\sqrt{2};\sqrt{2})$.\\
		Gọi $\alpha$ là góc giữa hai mặt phẳng $(AHK)$ và $\left(ABB'A'\right)$. Ta có 
		$$\cos \alpha=\left|\cos \left(\vec{n}_1, \vec{n}_2\right)\right|= \dfrac{\left| 1\cdot 2+0\cdot \sqrt{2}+0\cdot \sqrt{2}\right| }{\sqrt{1^2+0^2+0^2}\cdot \sqrt{2^2+\sqrt{2}^2+\sqrt{2}^2}}=\dfrac{1}{\sqrt{2}} \Rightarrow \alpha=45^{\circ}.$$
	}
\end{ex}

\begin{ex}%[2H5V1-6] 
	Cho hình lăng trụ đứng $ABC.A'B'C'$ có $AB=AC=a$, $BAC=120^{\circ}$. Gọi $M$, $N$ lần lượt là trung điểm của $B'C'$ và $CC'$. Biết thể tích khối lăng trụ $ABC.A'B'C'$ bằng $\dfrac{\sqrt{3} a^3}{4}$. Gọi $\alpha$ là góc giữa mặt phẳng $(AMN)$ và mặt phẳng $(ABC)$, tính $\cos \alpha$. Kết quả viết ở dạng thập phân làm tròn đến hàng phần trăm.
	\shortans{$0{,}43$} 
	\loigiai{ 
		\begin{center}
			\begin{tikzpicture}
				\def\a{6}
				\def\b{4}
				\def\g{20}
				\def\h{5}
				\path
				(0:0) coordinate (A)--++(\g:\b) coordinate (B)--++(170:\a) coordinate (C)
				\foreach \x in {A,B,C}{($(\x)-(90:\h)$) coordinate (\x')};
				\draw[dashed] (B')--(C')  ($(B')!.5!(C')$) coordinate (M)--($(C)!.5!(C')$) coordinate (N) (A)--(M)--(A') (M)--($(B)!.5!(C)$)coordinate (P);
				\draw (A)--(A')--(C')--(C)--(A)--	(B)--(B')--(A') (B)--(C) (N)--(A);
				\foreach \diem/\anh/\ts in {A'/x/1.5,B'/y/1.4,P/z/1.2} \coordinate[label = below right:$\anh$] (\anh) at ($(M)!\ts!(\diem)$);
				\draw[-stealth] (A')--(x); \draw[-stealth](B')--(y); \draw[-stealth](P)--(z);
				\foreach \x/\g in {A/60,B/70,C/90,A'/190,B'/50,C'/180,M/40,N/180}
				\fill[black](\x) circle (1pt)
				($(\x)+(\g:3mm)$) node{$\x$};
			\end{tikzpicture}
		\end{center}
		Lấy $H$ là trung điểm của $BC$.\\ Ta có 
		$V_{ABC.A'BC'}=CC' \cdot S_{\triangle ABC}=\dfrac{\sqrt{3} a^3}{4} \Rightarrow CC'=a$ vì $S_{\triangle ABC}=\dfrac{\sqrt{3} a^2}{4}$.\\
		Chọn hệ trục tọa độ $Oxyz$ như hình vẽ. Ta có $M \equiv O$, $M(0;0;0)$, $A'\left(\dfrac{a}{2};0;0\right)$, $B'\left(0;\dfrac{\sqrt{3}a}{2};0\right)$, $C'\left(0;-\dfrac{\sqrt{3} a}{2};0\right)$, $A\left(\dfrac{a}{2};0;a\right)$, $N\left(0;-\dfrac{\sqrt{3} a}{2};\dfrac{a}{2}\right)$.\\
		Ta có $(ABC) \perp Oz$ nên $(ABC)$ có một vectơ pháp tuyến là $\vec{k}=(0;0;1)$.\\
		Lại có $\vec{MA}=\left(\dfrac{a}{2};0;a\right)$,  $\overrightarrow{MN}=\left(0;-\dfrac{\sqrt{3}a}{2} ;\dfrac{a}{2}\right)$.\\
		Gọi $\overrightarrow{v_1}=\dfrac{2}{a}\overrightarrow{MA}\Rightarrow \overrightarrow{v_1}=(1;0;2)$, $\vec{v}_2=\dfrac{2}{a} \overrightarrow{MN} \Rightarrow \vec{v}_2=(0;-\sqrt{3};1)$. Khi đó mặt phẳng $(AMN)$ song song hoặc chứa giá của hai vectơ không cùng phương là $\overrightarrow{v_1}$ và $\overrightarrow{v_2}$ nên có một vectơ pháp tuyến là $\vec{n}=\left[\overrightarrow{v_1}, \overrightarrow{v_2}\right]=(2 \sqrt{3};-1 ;-\sqrt{3})$.\\
		Vậy $\cos \alpha=\left| \cos (\vec{k}, \vec{n})\right| =\dfrac{\left| \vec{k}\cdot  \vec{n}\right| }{\left| \vec{k}\right| \left| \vec{n}\right| }=\dfrac{\sqrt{3}}{4}$.
	}
\end{ex}

\begin{ex}%[2H5V1-6]  
	Cho hình chóp $S.ABC$ có đáy $ABC$ là tam giác vuông cân tại $B$, $AC=2a$, tam giác $SAB$ và tam giác $SCB$ lần lượt vuông tại $A$, $C$. Khoảng cách từ $S$ đến mặt phẳng $(ABC)$ bằng $2a$. Tính côsin của góc giữa hai mặt phẳng $(SAB)$ và $(SCB)$. Kết quả viết ở dạng thập phân làm tròn đến hàng phần trăm.
	\shortans{ $0{,}33$}
	\loigiai{
		\begin{center}
			\begin{tikzpicture}[scale=.8]
				\def\a{5}
				\def\b{4}
				\def\g{-40}
				\def\h{2}
				\path
				(0:0) coordinate (B)--++(\g:\b) coordinate (A)--++(30:\a) coordinate (C)--++(130:5)coordinate (S)
				(0,0)--++(90:5) coordinate (D);
				\draw[dashed] (B)--(C);
				\draw (S)--(B)--(A)--(C)--(S)--(A);
				\draw[gray] ($(B)!.12!(A)$) coordinate (E) ($(B)!.08!(C)$) coordinate (F) (E)--($(E)+(F)-(B)$)-- (F);
				\foreach \diem/\anh/\ts in {A/x/1.4,C/y/1.2} \coordinate[label = below right:$\anh$] (\anh) at ($(B)!\ts!(\diem)$);
				\draw[-stealth] (A)--(x); \draw[-stealth](C)--(y); \draw[-stealth](B)--(D)node[left]{$z$};
				\foreach \x/\g in {A/-110,B/-130,C/65,S/90}
				\fill[black](\x) circle (1pt)
				($(\x)+(\g:3mm)$) node{$\x$};
			\end{tikzpicture}
		\end{center}
		Chọn hệ trục tọa độ sao cho $B(0;0;0)$, $A(a\sqrt{2};0;0)$, $C(0;a\sqrt{2};0)$, $S(x;y;z)$.\\
		Ta có phương trình mặt phẳng $(ABC)$ là $z=0$, $\overrightarrow{AS}=(x-a\sqrt{2};y;z)$, $\vec{CS}=(x;y-a\sqrt{2};z)$.\\
		Do $\overrightarrow{AS} \cdot\overrightarrow{AB}=0 \Rightarrow(x-a \sqrt{2}) a \sqrt{2}=0 \Rightarrow x=a \sqrt{2}$.\\
		Mặt khác $\mathrm{d}(S,(ABC))=2a \Rightarrow z=2a(z>0)$.\\
		Lại có $\overrightarrow{CS}\cdot \overrightarrow{CB}=0 \Rightarrow(y-a\sqrt{2})a\sqrt{2}=0 \Rightarrow y=a \sqrt{2}$. \\
		Vậy $S\left( a\sqrt{2};a \sqrt{2};2 a\right)$.\\
		Ta có $\overrightarrow{AS}=\left(0;a\sqrt{2};2a\right)$, $\overrightarrow{CS}=\left(a\sqrt{2};0;2a\right)$, $\overrightarrow{BS}=\left(a\sqrt{2};a\sqrt{2};2a\right)$. Lúc đớ\\ $\left[\overrightarrow{AS},\overrightarrow{BS}\right] =\left( 0;2a^2\sqrt{2};-a^2\sqrt{a}\right) =2a^2\left( 0;\sqrt{2};1\right)$,\\ $\left[\overrightarrow{CS},\overrightarrow{BS}\right]=\left( -2a^2\sqrt{2};0;2a^2\right)=2a^2\left( -\sqrt{2};0;1\right)$.\\
		Vậy $(SBC)$ có một vectơ  pháp tuyến là $\vec{n}=\left( -\sqrt{2};0;1\right)$ và $(SAB)$ có một vectơ  pháp tuyến $\vec{m}=\left( 0;\sqrt{2};-1\right)$. Suy ra $$\cos \varphi=\dfrac{\left| \vec{n}\cdot \vec{m}\right|} {\left| \vec{n}\right| \cdot\left| \vec{m}\right|  }=\dfrac{\left| -\sqrt{2}\cdot 0+0\cdot \sqrt{2}+1\cdot (-1)\right| }{\sqrt{\left( -\sqrt{2}\right)^2+0^2+1^2}\cdot \sqrt{0^2+\left( \sqrt{2}\right)^2+\left(-1\right)^2}}=\dfrac{1}{\sqrt{3} \cdot \sqrt{3}}=\dfrac{1}{3}.$$
	}
\end{ex}

\begin{ex}%[2H5V1-6] 
	Cho hình lăng trụ đứng $ABC.A'B'C'$ có đáy là tam giác cân đỉnh $A$. Biết $BC=a \sqrt{3}$ và $\widehat{ABC}=30^{\circ}$, cạnh bên $AA'=a$. Gọi $M$ là điểm thỏa mãn $2 \vec{CM}=3 \vec{CC'}$. Gọi $\alpha$ là góc tạo bởi hai mặt phẳng $(ABC)$ và $\left(AB'M\right)$, khi đó tính $\sin \alpha$. Kết quả viết ở dạng thập phân làm tròn đến hàng phần trăm.
	\shortans{ $0{,}93$} 
	\loigiai{ 
		Gọi $O$ là trung điểm $BC$. Lúc đó 
		$$BO=AB \cdot \cos 30^{\circ} \Leftrightarrow AB=\dfrac{BO}{\cos 30^{\circ}}=\dfrac{a \sqrt{3}}{2 \cdot \dfrac{\sqrt{3}}{2}}=a=AC$$
		và 
		$$AO=AB \cdot \sin 30^{\circ}=\dfrac{a}{2}.$$
		Theo đề bài ta có 
		$$2 \overrightarrow{CM}=3\vec{CC'} \Leftrightarrow \vec{CM}=\dfrac{3}{2} \vec{CC'} \Leftrightarrow \overrightarrow{CC'}+\overrightarrow{C'M}=\dfrac{3}{2} \overrightarrow{CC'} \Leftrightarrow \overrightarrow{C'M}=\dfrac{1}{2} \overrightarrow{CC'} \Rightarrow C'M=\dfrac{a}{2}.$$
		\begin{center}
			\begin{tikzpicture}
				\def\a{3}
				\def\b{1}
				\def\g{-140}
				\def\h{3}
				\path
				(0:0) coordinate (A)--++(0:\a) coordinate (C)--++(\g:\b) coordinate (B) 
				\foreach \x in {A,B,C}{					($(\x)+(90:\h)$) coordinate (\x')};
				\draw[dashed] (C)--(A)--($(B)!.5!(C)$) coordinate (O);
				\draw (A)--(B)--(C)--(C')--(B')--(A')--(A) (B)--(B') (C')--($(C)!1.6!(C')$) coordinate (M) (O)--($(B')!.5!(C')$) coordinate (O');
				\path[name path=c1] (A') --(C');
				\path[name path=c2] (A)--(M);
				\path[name intersections={of=c1 and c2}] (intersection-1) coordinate (N);
				\draw (M)--(N)--(B')--cycle (N)--(A') ;
				\foreach \t/\tt/\ts/\tss in {O'/A/.06/.06,O'/B/.06/.25,A/B/.06/.25}
				\draw[gray] ($(O)!\ts!(\t)$) coordinate (X) ($(O)!\tss!(\tt)$) coordinate (Y) (X)--($(X)+(Y)-(O)$)--(Y);
				\draw[dashed] (A)--(N) (N)--(C');
				\foreach \diem/\anh/\ts in {B/x/3,A/y/1.4,O'/z/2} \coordinate[label = below right:$\anh$] (\anh) at ($(O)!\ts!(\diem)$);
				\draw[-stealth] (B)--(x); \draw[-stealth](A)--(y); \draw[-stealth](O')--(z);
				\foreach \x/\g in {A/-90,B/-90,C/0,A'/180,B'/-140,C'/0,M/0,N/130,O/-90,O'/-40}
				\fill[black](\x) circle (1pt)
				($(\x)+(\g:3mm)$) node{$\x$};
			\end{tikzpicture}
		\end{center}
		 Coi $a=1$. Gắn hệ trục tọa độ $Oxyz$ như hình vẽ với $O(0;0;0)$, $A\left(0;\dfrac{1}{2};0\right)$, $B\left(\dfrac{\sqrt{3}}{2};0;0\right)$, $C \left(-\dfrac{\sqrt{3}}{2};0;0\right)$, $B'\left(\dfrac{\sqrt{3}}{2};0;1\right)$, $C' \left(-\dfrac{\sqrt{3}}{2};0;\dfrac{3}{2}\right)$.\\
		 Khi đó $(ABC) \equiv(Oxy)\colon z=0 \Rightarrow(ABC)$ có một vectơ pháp tuyến là $\vec{k}=(0;0;1)$.\\
		Ta có $\overrightarrow{AB'}=\left(\dfrac{\sqrt{3}}{2} ;-\dfrac{1}{2};1\right)$, $\overrightarrow{AM}=\left(-\dfrac{\sqrt{3}}{2};-\dfrac{1}{2};\dfrac{3}{2}\right)$ suy ra $$\overrightarrow{n}_{\left(AB'M\right)}=4\left[\overrightarrow{AB'},\overrightarrow{AM}\right]=\left(1;5\sqrt{3};2\sqrt{3}\right).$$
		 Gọi $\alpha$ là góc giữa hai mặt phẳng $(ABC)$ và $(AB'M)$.\\
		 Ta có $$\cos \alpha=\dfrac{\left|\vec{k}\cdot \vec{n}_{\left(AB'M\right)}\right|}{|\vec{k}| \cdot\left|\overrightarrow{n}_{\left(AB'M\right)}\right|}=\dfrac{|2 \sqrt{3}|}{1\cdot 2 \sqrt{22}}=\sqrt{\dfrac{3}{22}}.$$
		 Suy ra $\sin \alpha=\sqrt{1-\cos ^2 \alpha}=\sqrt{\dfrac{19}{22}}=\dfrac{\sqrt{418}}{22}$.
	}
\end{ex}

\begin{ex}%[2H5V1-6] 
	Cho khối tứ diện $ABCD$ có $BC=3$, $CD=4$, $\widehat{ABC}=\widehat{ADC}=\widehat{BCD}=90^{\circ}$. Góc giữa đường thẳng $AD$ và $BC$ bằng $60^{\circ}$. Tính côsin góc giữa hai mặt phẳng $(ABC)$ và $(ACD)$. Kết quả viết ở dạng thập phân làm tròn đến hàng phần trăm.
	\shortans{ $0{,}3$}
	\loigiai{
		\begin{center}
			\begin{tikzpicture}[scale=.8]
				\path (0:0) coordinate (A)--++(-90:5) coordinate (O)--++(0:4) coordinate (D)--++(-140:3) coordinate (C)--++(180:4)coordinate (B);
				\draw (A)--(B)--(C)--(D)--(A)--(C);
				\draw[dashed] (A)--(O)--(D) (O)--(B);
				\draw[gray] ($(O)!.08!(A)$) coordinate (X) ($(O)!.09!(D)$) coordinate (Y) (X)--($(X)+(Y)-(O)$)-- (Y);
				\foreach \diem/\anh/\ts in {B/x/1.5,D/y/1.4,A/z/1.3} \coordinate[label = below right:$\anh$] (\anh) at ($(O)!\ts!(\diem)$);
				\draw[-stealth] (B)--(x); \draw[-stealth](D)--(y); \draw[-stealth](A)--(z);
				\foreach \x/\g in {A/140,B/-70,C/-30,O/140,D/60}
				\fill[black](\x) circle (1pt)
				($(\x)+(\g:3mm)$) node{$\x$};
			\end{tikzpicture}
		\end{center}
		Dựng $AO \perp(BCD)$ khi đó $O$ là đỉnh thứ tư của hình chữ nhật $BCDO$.\\
		Góc giữa đường thẳng $AD$ và $BC$ là góc giữa đường thẳng $AD$ và $OD$ và bằng $\widehat{ADO}=60^{\circ}$.\\
		Xét tam giác $ADO$ vuông tại $O$ ta có $\tan 60^{\circ}=\dfrac{OA}{OD} \Rightarrow OA=3\sqrt{3}$.\\
		Gắn hệ tọa độ $Oxyz$ vào hình chóp như hình vẽ.\\
		Ta có $O(0;0;0)$, $B(4;0;0)$, $D(0;3;0)$, $C(4;3;0)$, $A(0;0;3\sqrt{3})$.\\
		Suy ra $\overrightarrow{AB}=(4;0;-3\sqrt{3})$, $\overrightarrow{BC}=(0;3;0)$, $\overrightarrow{AD}=(0;3;-3\sqrt{3})$, $\overrightarrow{CD}=(-4;0;0)$.\\
		Mặt phẳng $(ABC)$ nhận vectơ $\vec{n_1}=\left[ \overrightarrow{AB},\overrightarrow{BC}\right] =(9\sqrt{3};0;12)$ làm vectơ pháp tuyến.\\
		Mặt phẳng $(ADC)$ nhận vectơ $\vec{n_2}=\left[ \overrightarrow{AD}, \overrightarrow{CD}\right] =\left(0;12\sqrt{3};12\right)$ làm vectơ pháp tuyến.\\
		Nên $\cos\left((ABC);(ADC)\right) =\dfrac{\left|\overrightarrow{n_1} \cdot \overrightarrow{n_2}\right|}{\left|\overrightarrow{n_1}\right| \cdot\left| \vec{n}_2 \right| }=\dfrac{144}{72\sqrt{43}}=\dfrac{2 \sqrt{43}}{43}$.
	}
\end{ex}
\Closesolutionfile{ans}
\inputansbox{10}{ans/ans-C5B2CD4-KQ}
