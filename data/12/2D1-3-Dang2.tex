\Opensolutionfile{ans}[ans/ans2D1-3-2]
%\setcounter{dang}{1}
\begin{dang}{Tìm GTLN – GTNN bằng phương pháp đổi biến}
\underline{\textit{Cơ sở lý thuyết:}} Cho hàm số $y=f[u(x)]$ xác định trên $K$ và có GTLN, GTLNN là $M$, $m$. Đặt $t=u(x)$. Vì $x \in K \Rightarrow t \in K_t$. Khi đó GTLN, GTNN của hàm số $f(t)$ trên $K_t$ cũng tương ứng bằng $M$, $m$.\\
\textit{Tóm tắt:} Việc đổi biến không làm thay đổi tập giá trị của hàm số.
\end{dang}
\paragraph{Các ví dụ}
\begin{vd}%[Dự án-TLDH1-Quảng Đại Mưa]%[2D1Y3-1]%Ví dụ 1: 
	Tìm giá trị lớn nhất của hàm số $y=x^6+4\left(1-x^2\right)^3$ trên đoạn $[-1;1]$.
	\loigiai{
		Giải theo tự luận:\\
		Đặt $t=x^2,\left(0\leq t\leq 1\right)$, hàm số đã cho trở thành $y=t^3+4(1-t)^3$
		$y'=3t^2-12(1-t)^2;y'=0\Leftrightarrow\hoac{&t=\dfrac{2}{3}\\&t=2(loai).}$ \\
		Ta có: $y(0)=4;y(1)=1;y\left(\dfrac{2}{3}\right)=\dfrac{4}{9}$. Vậy $\max\limits_{[-1;1]} y=4$\\
		Giải theo Casio:\\
		Nhập biểu thức $f(X)=4+4\left(1-X^{2}\right)^{3}$\\
		Nhập các giá trị: Start?$-1$ End $1$ Step?$0.1$\\
		Kết quả.} 
\end{vd}
\begin{vd}%[Dự án-TLDH1-Quảng Đại Mưa]%[2D1K3-1]%Ví dụ 2: 
	Tìm giá trị lớn nhất, giá trị nhỏ nhất của hàm số $y=\sqrt{3+2x-x^2}+(x-1)^2-5$.
	\loigiai{
		Giải theo tự luận:\\
		Đk: $3+2x-x^2\geq 0\Leftrightarrow-1\leq x\leq 3$.\\
		Đặt $t=\sqrt{3+2x-x^2}=\sqrt{4-(x-1)^2}\Rightarrow 0\leq t\leq 2$.\\
		Hàm số đã cho trở thành: $y=-t^2+t-1$.\\
		$y'=-2t+1;y'=0\Leftrightarrow t=\dfrac{1}{2}$.\\
		Ta có: $y(0)=-1;y(2)=-3;y\left(\dfrac{1}{2}\right)=\dfrac{-3}{4}$. Vậy $\max\limits_{[-1;3]} y=-\dfrac{3}{4};\min\limits_{[-1;3]}=-3$.\\
		Giải theo Casio:\\
		Đk: $3+2x-x^2\geq 0\Leftrightarrow-1\leq x\leq 3$.\\
		Nhập biểu thức $f(x)$ vào máy.\\
		Lần 1: ấn “=” sau đó nhập giá trị $start=-1;end=3;step=0\cdot 2$.\\
		Lần 2: ấn “=” sau đó nhập giá trị $start=2\cdot 6;end=3;step=0\cdot 02$.}
\end{vd}
\begin{vd}%[Dự án-TLDH1-Quảng Đại Mưa]%[2D1K3-1]%Ví dụ 3: 
	Giá trị nhỏ nhất của hàm số $y=\cos^22x-\sin x\cdot\cos x+4$ bằng
	\loigiai{
		Giải theo tự luận:\\
		Ta có: $y=\left(1-\sin^22x\right)-\dfrac{1}{2}\sin 2x+4$.\\
		Đặt $t=\sin 2x,\left(-1\leq t\leq 1\right)$, hàm số đã cho trở thành $y=-t^2-\dfrac{1}{2}t+5$.\\
		$y'=-2t-\dfrac{1}{2};y'=0\Leftrightarrow t=\dfrac{-1}{4}$.\\
		$y(-1)=\dfrac{9}{2};y(1)=\dfrac{7}{2};y\left(-\dfrac{1}{4}\right)=\dfrac{81}{6}$ \\
		$ \Rightarrow\min\limits_{\mathbb{R}}=\dfrac{7}{2} $ tại $x=\dfrac{\pi}{4}+k\pi,k\in\mathbb{Z}$.}
\end{vd}
\begin{vd}%[Dự án-TLDH1-Quảng Đại Mưa]%[2D1K3-1]%Ví dụ 4: 
	Gọi $M$, $m$ theo thứ tự là giá trị lớn nhất và giá trị nhỏ nhất của hàm số $y=\sqrt{2-x}+2\sqrt{2+x}+4\sqrt{4-x^2}+3x+1$. Tính $P=M+m$.
	\loigiai{
		Tập xác định $\mathscr{D}=[-2;2]$.\\
		Đặt $t=\sqrt{2-x}+2\sqrt{2+x}\Rightarrow t'=-\dfrac{1}{2\sqrt{2-x}}+\dfrac{1}{\sqrt{2+x}}\xrightarrow{{t'=0}}x=\dfrac{6}{5}$.\\
		Ta có $\heva{&t(-2)=2\\&t(2)=4\\&t\left(\dfrac{6}{5}\right)=2\sqrt{5}}\Rightarrow\min\limits_{[-2;2]} t\leq t\leq\max\limits_{[-2;2]} t\Leftrightarrow 2\leq t\leq 2\sqrt{5}\Rightarrow t\in\left[2;2\sqrt{5}\right]$.\\
		Lại có $t^2=3x+10+4\sqrt{4-x^2}\Rightarrow 3x+\sqrt{4-x^2}=t^2-10$.\\
		Từ đó, $y=t+t^2-9\Rightarrow y'=2t+1>0,\forall t\in\left[2;2\sqrt{5}\right]$.\\
		suy ra $\heva{&M=y(2\sqrt{5})=11+2\sqrt{5}\\&m=y(2)=-3}\Rightarrow P=8+2\sqrt{5.}$}
\end{vd}
\begin{vd}%[Dự án-TLDH1-Quảng Đại Mưa]%[2D1G3-1]%Ví dụ 5: 
	Tìm GTLN của hàm số $y=\sin\dfrac{2x}{x^2+1}+\cos\dfrac{4x}{x^2+1}+1$.
	\loigiai{
		Đặt $t=\sin\dfrac{2x}{x^2+1}\in[-1;1]$ suy ra $\cos\dfrac{4x}{x^2+1}=\cos2\left(\dfrac{2x}{x^2+1}\right)=1-\sin^2\dfrac{2x}{x^2+1}$.\\
		Khi đó $y=\sin t+\cos2t+1=2+\sin t-2\sin^2t=\dfrac{17}{8}-2\left(t-\dfrac{1}{4}\right)^2\leq\dfrac{17}{8}\Rightarrow y_{\max} =\dfrac{17}{8}$.}
\end{vd}
\paragraph{Câu hỏi trắc nghiệm}
\begin{ex}%[Dự án-TLDH1-Quảng Đại Mưa]%[2D1K3-2]%Câu 1.
	Giá trị nhỏ nhất của hàm số $y=2\cos^3x-\dfrac{9}{2}\cos^2x+3\cos x+\dfrac{1}{2}$ là 
	\choice
	{$1$}
	{$-24$}
	{$-12$}
	{\True $-9$}
\end{ex}
\begin{ex}%[Dự án-TLDH1-Quảng Đại Mưa]%[2D1K3-2]%Câu 2.
	Giá trị lớn nhất của hàm số $y=2\sin^2x+2\sin x-1$ bằng 
	\choice
	{$3$ tại $x=\dfrac{\pi}{4}+k2\pi,k\in\mathbb{Z}$}
	{$2$ tại $x=\dfrac{\pi}{4}+k2\pi,k\in\mathbb{Z}$}
	{\True $3$ tại $x=\dfrac{\pi}{2}+k2\pi,k\in\mathbb{Z}$}
	{$2$ tại $x=\dfrac{\pi}{2}+k2\pi,k\in\mathbb{Z}$}
%<MyLT>
\end{ex}
\begin{ex}%[Dự án-TLDH1-Quảng Đại Mưa]%[2D1K3-1]%Câu 3.
	Giá trị lớn nhất và giá trị nhỏ nhất của hàm số $y=\sin^3x-3\sin x+1$ trên đoạn $[0;\pi]$ là 
	\choice
	{$\max\limits_{[0;\pi]} y=3;\min\limits_{[0;\pi]}=-1$}
	{$\max\limits_{[0;\pi]} y=3;\min\limits_{[0;\pi]}=1$}
	{\True $\max\limits_{[0;\pi]} y=1;\min\limits_{[0;\pi]}=-1$}
	{$\max\limits_{[0;\pi]} y=1;\min\limits_{[0;\pi]}=-3$}
\end{ex}
\begin{ex}%[Dự án-TLDH1-Quảng Đại Mưa]%[2D1K3-2]%Câu 4.
	Tìm giá trị nhỏ nhất của hàm số $y=\dfrac{2\cos x-1}{\cos x+2}$. 
	\choice
	{$\dfrac{1}{3}$}
	{$1$}
	{\True $-3$}
	{$-1$}
\end{ex}
\begin{ex}%[Dự án-TLDH1-Quảng Đại Mưa]%[2D1K3-2]%Câu 5.
	Giá trị nhỏ nhất của hàm số $y=\dfrac{\sin x+1}{\sin^2x+\sin x+1}$ là 
	\choice
	{$-\dfrac{1}{3}$}
	{\True $0$}
	{$\dfrac{2}{3}$}
	{$\dfrac{3}{2}$}
\end{ex}
\begin{ex}%[Dự án-TLDH1-Quảng Đại Mưa]%[2D1K3-1]%Câu 6.
	Giá trị lớn nhất của hàm số $y=2x+1+\dfrac{1}{2x+1}$ trên đoạn $[1;2]$ là 
	\choice
	{\True $\dfrac{26}{5}$}
	{$\dfrac{10}{3}$}
	{$\dfrac{14}{3}$}
	{$\dfrac{24}{3}$}
\end{ex}
\begin{ex}%[Dự án-TLDH1-Quảng Đại Mưa]%[2D1K3-2]%Câu 7.
	Giá trị lớn nhất của hàm số $y=3\sin x-4\sin^3x$ trên khoảng $\left(-\dfrac{\pi}{2};\dfrac{\pi}{2}\right)$ là 
	\choice
	{$7$}
	{$3$}
	{\True $1$}
	{$-1$}
\end{ex}
\begin{ex}%[Dự án-TLDH1-Quảng Đại Mưa]%[2D1B3-1]%Câu 8.
	Giá trị lớn nhất của hàm số $y=2\sin x-\dfrac{4}{3}\sin^3x$ trên đoạn $[0;\pi]$ là 
	\choice
	{$\dfrac{2}{3}$}
	{$\dfrac{\sqrt{2}}{3}$}
	{\True $\dfrac{2\sqrt{2}}{3}$}
	{$\dfrac{\sqrt{3}}{3}$}
\end{ex}
\begin{ex}%[Dự án-TLDH1-Quảng Đại Mưa]%[2D1B3-1]%Câu 9.
	Giá trị nhỏ nhất của hàm số $y=\cos^4x-6\cos^2x+5$ là 
	\choice
	{$5$}
	{$-5$}
	{\True $0$}
	{$1$}
\end{ex}
\begin{ex}%[Dự án-TLDH1-Quảng Đại Mưa]%[2D1B3-1]%Câu 10.
	Giá trị lớn nhất và giá trị nhỏ nhất của hàm số $y=2\sin^2x-\cos x+1$ là 
	\choice
	{\True $\max\limits_{\mathbb{R}} y=\dfrac{25}{8};\min\limits_{\mathbb{R}}=0$}
	{$\max\limits_{\mathbb{R}} y=4;\min\limits_{\mathbb{R}}=2$}
	{$\max\limits_{\mathbb{R}} y=\dfrac{25}{8};\min\limits_{\mathbb{R}}=2$}
	{$\max\limits_{\mathbb{R}} y=0;\min\limits_{\mathbb{R}}=-2$}
\end{ex}
\begin{ex}%[Dự án-TLDH1-Quảng Đại Mưa]%[2D1B3-1]%Câu 11.
	Tìm giá trị nhỏ nhất của hàm số $y=\sin^3x-\cos2x+\sin x+2$ trên đoạn $\left(-\dfrac{\pi}{2};\dfrac{\pi}{2}\right)$. 
	\choice
	{\True $\dfrac{23}{27}$}
	{$\dfrac{1}{27}$}
	{$5$}
	{$1$}
\end{ex}
\begin{ex}%[Dự án-TLDH1-Quảng Đại Mưa]%[2D1B3-1]%Câu 12.
	Tìm giá trị nhỏ nhất của hàm số $y=\sin^6x+\cos^6x$. 
	\choice
	{\True $\min\limits_{\mathbb{R}} y=\dfrac{1}{4}$}
	{$\min\limits_{\mathbb{R}} y=\dfrac{1}{2}$}
	{$\min\limits_{\mathbb{R}} y=\dfrac{3}{4}$}
	{$\min\limits_{\mathbb{R}} y=1$}
\end{ex}
\begin{ex}%[Dự án-TLDH1-Quảng Đại Mưa]%[2D1B3-2]%Câu 13.
	Giá trị lớn nhất của hàm số $y=\sin^4x+\cos^2x+2$ là 
	\choice
	{$-3$}
	{\True $3$}
	{$-\dfrac{11}{4}$}
	{$\dfrac{11}{4}$}
\end{ex}
\begin{ex}%[Dự án-TLDH1-Quảng Đại Mưa]%[2D1B3-2]%Câu 14.
	Giá trị nhỏ nhất của hàm số $y=2\sin^8x+\cos^42x$ là 
	\choice
	{$1$}
	{$3$}
	{$-\dfrac{1}{27}$}
	{\True $\dfrac{1}{27}$}
\end{ex}
\begin{ex}%[Dự án-TLDH1-Quảng Đại Mưa]%[2D1B3-2]%Câu 15.
	Giá trị lớn nhất của hàm số $y=\cos2x+3\sin^2x+2\sin x$ là 
	\choice
	{\True $4$}
	{$6$}
	{$5$}
	{$2$}
\end{ex}
\begin{ex}%[Dự án-TLDH1-Quảng Đại Mưa]%[2D1K3-2]%Câu 16.
	Tìm giá trị nhỏ nhất của hàm số $y=\dfrac{3-2\sin^2x}{2\cos^2x-3}$. 
	\choice
	{$-4$}
	{\True $-3$}
	{$0$}
	{$-1$}
\end{ex}
\begin{ex}%[Dự án-TLDH1-Quảng Đại Mưa]%[2D1K3-1]%Câu 17.
	Tìm giá trị lớn nhất của hàm số $y=x+1+\sqrt{9+6x-3x^2}$. 
	\choice
	{\True $4$}
	{$3$}
	{$0$}
	{Số khác}
\end{ex}

\begin{ex}%[Dự án-TLDH1-Quảng Đại Mưa]%[2D1K3-2]%Câu 19.
	Hàm số $y=x^3+\dfrac{1}{x^3}+\left(x^2+\dfrac{1}{x^2}\right)-2\left(x+\dfrac{1}{x}\right)$ với $x>0$ đạt GTNN bằng 
	\choice
	{\True $0$}
	{$-4$}
	{$2$}
	{$-1$}
\end{ex}
\begin{ex}%[Dự án-TLDH1-Quảng Đại Mưa]%[2D1K3-2]%Câu 20.
	Giá trị nhỏ nhất của hàm số $y=\dfrac{\sin^4x-2\cos^2x+4}{2\sin^2x+\cos^2x}$ là 
	\choice
	{$-2$}
	{\True $2$}
	{$\dfrac{5}{2}$}
	{$-\dfrac{5}{2}$}
\end{ex}
\begin{ex}%[Dự án-TLDH1-Quảng Đại Mưa]%[2D1B3-1]%Câu 21.
	Tìm giá trị lớn nhất của hàm số $y=\sin^6x+\cos^6x+2\cos 4x+\sin 2x-5$. 
	\choice
	{$\max\limits_{\mathbb{R}} y=-\dfrac{37}{19}$}
	{$\max\limits_{\mathbb{R}} y=-\dfrac{31}{4}$}
	{\True $\max\limits_{[-1;1]} y=-\dfrac{37}{19}$}
	{$\max\limits_{[-1;1]} y=-\dfrac{31}{4}$}
	\loigiai{
		Ta có: $\sin^6x+\cos^6x=1-\dfrac{3}{4}\sin^22x;\cos4x=1-2\sin^22x$ \\
		$ \Rightarrow y=-\dfrac{19}{4}\sin^22x+\sin 2x-2 $
		Đặt $t=\sin 2x;t\in[-1;1]$
		Hàm số đã cho trở thành $y=-\dfrac{19}{4}t^2+t-2$
		Tìm GTLN, GTNN của hàm số trên đoạn $[-1;1]$ ta được $\min\limits_{\mathbb{R}} y=-\dfrac{31}{4};\max\limits_{\mathbb{R}} y=-\dfrac{37}{19}$.}
\end{ex}
\begin{ex}%[Dự án-TLDH1-Quảng Đại Mưa]%[2D1K3-1]%Câu 22.
	Tìm giá trị lớn nhất và giá trị nhỏ nhất của hàm số $y=\dfrac{x^3+x^2+x}{\left(x^2+1\right)^2}$. 
	\choice
	{$\max\limits_{\mathbb{R}} y=\dfrac{3}{4};\min\limits_{\mathbb{R}}=\dfrac{1}{4}$}
	{$\max\limits_{\mathbb{R}} y=\dfrac{1}{2};\min\limits_{\mathbb{R}}=-\dfrac{1}{2}$}
	{$\max\limits_{\mathbb{R}} y=\dfrac{3}{4};\min\limits_{\mathbb{R}}=-\dfrac{1}{2}$}
	{\True $\max\limits_{\mathbb{R}} y=\dfrac{3}{4};\min\limits_{\mathbb{R}} y=-\dfrac{1}{4}$}
	\loigiai{
		Ta có: $y=\dfrac{\left(x^2+1\right)x+x}{\left(x^2+1\right)^2}=\dfrac{x}{x^2+1}+\left(\dfrac{x}{x^2+1}\right)^2$.\\
		Đặt $t=\dfrac{x}{x^2+1}\Rightarrow t\in\left[\dfrac{-1}{2};\dfrac{1}{2}\right]$;(dùng bảng biến thiên để tìm miền giá trị của t).\\
		Hàm số đã cho trở thành $y=t^2+t$.\\
		Tìm GTLN, GTNN của hàm số trên đoạn $\left[\dfrac{-1}{2};\dfrac{1}{2}\right]$ ta được $\min\limits_{\mathbb{R}} y=-\dfrac{1}{4};\max\limits_{\mathbb{R}} y=\dfrac{3}{4}$.}
\end{ex}
\begin{ex}%[Dự án-TLDH1-Quảng Đại Mưa]%[2D1K3-1]%Câu 23.
	Tìm GTLN, GTNN của hàm số $y=\sqrt{x+4}+\sqrt{4-x}-4\sqrt{(x+4)\cdot (4-x)}+5$. 
	\choice
	{$\max\limits_{[-4;4]} y=4;\min\limits_{[-4;4]}=2\sqrt{2}$}
	{\True $\max\limits_{[-4;4]} y=5+2\sqrt{2};\min\limits_{[-4;4]}=-7$}
	{$\max\limits_{[-4;4]} y=4;\min\limits_{[-4;4]}=-2\sqrt{2}$}
	{$\max\limits_{[-4;4]} y=4;\min\limits_{[-4;4]}=-7$}
	\loigiai{
		Điều kiện: $-4\leq x\leq 4$.\\
		Đặt $t=\sqrt{x+4}+\sqrt{4-x}\Rightarrow t^2=8+2\sqrt{(x+4)(4-x)}\Rightarrow\sqrt{(x+4)(4-x)}=\dfrac{t^2-8}{2}$.\\
		Hàm số đã cho trở thành $y=t-4\left(\dfrac{t^2-8}{2}\right)+5=-2t^2+t+21$.\\
		Tìm điều kiện của t: tìm max,min của hàm số: $g(x)=\sqrt{x+4}+\sqrt{4-x},x\in[-4;4]$ \\
		$ \Rightarrow t\in\left[2\sqrt{2};4\right] $.\\
		Tìm GTLN, GTNN của hàm số trên đoạn $\left[2\sqrt{2};4\right]$ ta được $\min\limits_{\mathbb{R}} y=-7;\max\limits_{\mathbb{R}} y=5+2\sqrt{2}$.}
\end{ex}
\begin{ex}%[Dự án-TLDH1-Quảng Đại Mưa]%[2D1K3-1]%Câu 24.
	Giá trị nhỏ nhất của hàm số $y=x(x+2)(x+4)(x+6)+5$ với $x\geq-4$ là 
	\choice
	{$-9$}
	{\True $-11$}
	{$-3$}
	{$-12$}
	\loigiai{
		Ta có: $y=\left(x^2+6x\right)\left(x^2+6x+8\right)+5$.\\
		Đặt $t=x^2+6x\Rightarrow t\in[-9;+\infty)$.\\
		Hàm số đã cho trở thành $y=t^2+8t+5$.\\
		Tìm GTLN, GTNN của hàm số trên nửa khoảng $[-9;+\infty)$ ta được $\min\limits_{[-4;+\infty)} y=-11$.}
\end{ex}
\begin{ex}%[Dự án-TLDH1-Quảng Đại Mưa]%[2D1K3-1]%Câu 25.
	Hàm số $y=4\sqrt{x^2-2x+3}+2x-x^2$ đạt GTLN tại hai giá trị x mà tích của chúng là 
	\choice
	{$2$}
	{\True $7$}
	{$0$}
	{$-1$}
	\loigiai{
		Ta có: $y=4\sqrt{(x-1)^2+2}-(x-1)^2+1$.\\
		Đặt $t=\sqrt{(x-1)^2+2};t>0\Rightarrow(x-1)^2=t^2-2$.\\
		Hàm số đã cho trở thành $y=-t^2+4t+3$.\\
		Tìm GTLN, GTNN của hàm số trên khoảng $(0;+\infty)$ ta được $\max\limits_{(0;+\infty)} y=7$ tại $t=2\Leftrightarrow\hoac{&x_1=1-\sqrt{2}\\&x_2=1+\sqrt{2}}\Rightarrow x_1x_2=-1$.}
\end{ex}
\Closesolutionfile{ans}
