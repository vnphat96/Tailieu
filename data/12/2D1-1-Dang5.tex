\begin{dang}{Câu hỏi lý thuyết}.
\end{dang}
\Opensolutionfile{ans}[ans/ans2D1-1-5]
\subsubsection{Câu hỏi trắc nghiệm}
\begin{ex}%[2D1Y1-5]%[Đỗ Đường Hiếu - ĐCHT THPT]%Câu 1.
	Cho hàm số $y=f(x)$ đơn điệu trên $(a;b)$. Mệnh đề nào dưới đây đúng?
	\choice
	{$f'(x)\geq 0,\forall x\in(a;b)$}
	{$f'(x)>0,\forall x\in(a;b)$}
	{\True $f'(x)$ không đổi dấu trên khoảng $(a;b)$}
	{$f'(x)\neq 0,\forall x\in(a;b)$}
	\loigiai{
		Ta có  ``Hàm số $y=f(x)$ đơn điệu trên $(a;b)$ thì $f'(x)$ không đổi dấu trên khoảng $(a;b)$''.
	}
\end{ex}

\begin{ex}%[2D1B1-5]%[Đỗ Đường Hiếu - ĐCHT THPT]%Câu 2.
	Cho hàm số $y=f(x)$ xác định trên khoảng $K$. Điều kiện đủ để hàm số đồng biến trên $K$ là 
	\choice
	{$f'(x)>0,\,\forall x\in K$}
	{$f'(x)>0$ tại hữu hạn điểm thuôc $K$}
	{$f'(x)\leq 0,\,\forall x\in K$}
	{\True $f'(x)\geq 0,\,\forall x\in K$}
	\loigiai{
		Cho hàm số $y=f(x)$ xác định trên khoảng $K$. Điều kiện đủ để hàm số đồng biến trên $K$ là $f'(x)\geq 0,\,\forall x\in K$.
	}
\end{ex}

\begin{ex}%[2D1B1-5]%[Đỗ Đường Hiếu - ĐCHT THPT]%Câu 4.
	Hàm số $y=f(x)$ có đạo hàm trên khoảng $(a;b)$. Mệnh đề nào sau đây \textbf{sai}?
	\choice
	{Nếu $f'(x)=0$ với mọi $x$ thuộc $(a;b)$ thì hàm số $y=f(x)$ không đổi trên khoảng $(a; b)$}
	{\True Nếu $f'(x)\geq 0$ với mọi $x$ thuộc $(a;b)$ thì hàm số $y=f(x)$ đồng biến trên khoảng $(a; b)$}
	{Nếu hàm số $y=f(x)$ không đổi trên khoảng $(a; b)$ thì $f'(x)=0$ với mọi $x$ thuộc $(a;b)$}
	{Nếu hàm số $y=f(x)$ đồng biến trên khoảng $(a; b)$ thì $f'(x)\geq 0$ với mọi $x$ thuộc $(a;b)$}
	\loigiai{
		Theo nội dung định lý mở rộng: Nếu $f'(x)\geq 0$ với mọi $x$ thuộc $(a;b)$ và $f'(x)=0$ tại một số hữu hạn điểm thì hàm số $y=f(x)$ đồng biến trên khoảng $(a; b)$. Bởi vậy mệnh đề sai là mệnh đề ``Nếu $f'(x)\geq 0$ với mọi $x$ thuộc $(a;b)$ thì hàm số $y=f(x)$ đồng biến trên khoảng $(a; b)$''.
	}
\end{ex}

\begin{ex}%[2D1B1-5]%[Đỗ Đường Hiếu - ĐCHT THPT]%Câu 5.
	Cho hàm số $y=f(x)$ có đạo hàm trên khoảng $K$. Khẳng định nào trong các khẳng định sau là đúng? 
	\choice
	{Nếu $f'(x_0)<0$ với mọi $x$ thuộc khoảng $K$ thì hàm số đồng biến trên $K$}
	{Nếu $f'(x_0)\leq 0$ với mọi $x$ thuộc khoảng $K$ thì hàm số đồng biến trên $K$}
	{\True Nếu $f'(x_0)>0$ với mọi $x$ thuộc khoảng $K$ thì hàm số đồng biến trên $K$}
	{Nếu $f'(x_0)\geq 0$ với mọi $x$ thuộc khoảng $K$ thì hàm số đồng biến trên $K$}
	\loigiai{
		Khẳng định đúng là khẳng định: ``Nếu $f'(x_0)>0$ với mọi $x$ thuộc khoảng $K$ thì hàm số đồng biến trên $K$''.
	}
\end{ex}

\begin{ex}%[2D1B1-5]%[Đỗ Đường Hiếu - ĐCHT THPT]%Câu 6.
	Cho hàm số $y=f(x)$ có đạo hàm trên $(a;b)$. Phát biểu nào sau đây là đúng?
	\choice
	{Hàm số $y=f(x)$ không đổi khi và chỉ khi $f'(x)<0,\forall x\in(a;b)$}
	{\True Hàm số $y=f(x)$ đồng biến khi và chỉ khi $f'(x)\geq 0,\forall x\in(a;b)$ và $f'(x)=0$ tại hữu hạn giá trị $x\in(a;b)$}
	{Hàm số $y=f(x)$ nghịch biến khi và chỉ khi $f'(x)\leq 0,\forall x\in(a;b)$}
	{Hàm số $y=f(x)$ đồng biến khi và chỉ khi $f'(x)\geq 0,\forall x\in(a;b)$}
	\loigiai{
	Phát biểu đúng là: ``Hàm số $y=f(x)$ đồng biến khi và chỉ khi $f'(x)\geq 0,\forall x\in(a;b)$ và $f'(x)=0$ tại hữu hạn giá trị $x\in(a;b)$''
}
\end{ex}

\begin{ex}%[2D1B1-5]%[Đỗ Đường Hiếu - ĐCHT THPT]%Câu 7.
	Cho hàm số $y=f(x)$ xác định trên khoảng K. Điều kiện đủ để hàm số $y=f(x)$ đồng biến trên K là 
	\choice
	{\True $f'(x)>0$ với mọi $x\in K$}
	{$f'(x)>0$ tại hữu hạn điểm thuộc khoảng K}
	{$f'(x)\leq 0$ với mọi $x\in K$}
	{$f'(x)\geq 0$ với mọi $x\in K$}
	\loigiai{
		Mệnh đề: ``Nếu A thì B''. Khi đó: A là điều kiện đủ để có B.\\
		Áp dụng vào mệnh đề: ``Nếu $f'(x)>0$ với mọi $x\in K$ với mọi $x\in K$ thì hàm số đồng biến trên $K$.'', ta có Điều kiện đủ để hàm số $y=f(x)$ đồng biến trên $K$ là $f'(x)>0$ với mọi $x\in K$.}
\end{ex}

\begin{ex}%[2D1B1-5]%[Đỗ Đường Hiếu - ĐCHT THPT]%Câu 8.
	Phát biểu nào sau đây là đúng?
	\choice
	{Hàm số $y=f(x)$ nghịch biến trên $(a;b)$ khi và chỉ khi $f'(x)\leq 0,\forall x\in(a;b)$}
	{Nếu $f'(x)\leq 0,\forall x\in(a;b)$ thì hàm số $y=f(x)$ nghịch biến trên $(a;b)$}
	{Hàm số $y=f(x)$ nghịch biến trên $(a;b)$ khi và chỉ khi $f'(x)<0,\forall x\in(a;b)$}
	{\True Nếu $f'(x)<0,\forall x\in(a;b)$ thì hàm số $y=f(x)$ nghịch biến trên $(a;b)$}
	\loigiai{
	Phát biểu đúng là: ``Nếu $f'(x)<0,\forall x\in(a;b)$ thì hàm số $y=f(x)$ nghịch biến trên $(a;b)$.''}
\end{ex}

\begin{ex}%[2D1B1-5]%[Đỗ Đường Hiếu - ĐCHT THPT]%Câu 9.
	Cho hàm số $y=f(x)$ đơn điệu và có đạo hàm trên khoảng $(a;b)$. Khẳng định nào sau đây là đúng?
	\choice
	{$f'(x)\geq 0,\forall x\in(a;b)$}
	{$f'(x)\leq 0,\forall x\in(a;b)$}
	{$f'(x)\neq 0,\forall x\in(a;b)$}
	{\True $f'(x)$ không đổi dấu trên khoảng $(a;b)$}
	\loigiai{
	Hàm số $y=f(x)$ đơn điệu và có đạo hàm trên khoảng $(a;b)$ thì $f'(x)$ không đổi dấu trên khoảng $(a;b)$.}
\end{ex}

\begin{ex}%[2D1B1-5]%[Đỗ Đường Hiếu - ĐCHT THPT]%Câu 10.
	Hàm số $y=f(x)$ có đạo hàm trên khoảng $(a;b)$. Mệnh đề nào sau đây là \textbf{sai}?
	\choice
	{Nếu $f'(x)=0$ với mọi $x$ thuộc $(a;b)$ thì hàm số $y=f(x)$ không đổi trên khoảng $(a;b)$}
	{\True Nếu $f'(x)\geq 0$ với mọi $x$ thuộc $(a;b)$ thì hàm số $y=f(x)$ đồng biến trên khoảng $(a;b)$}
	{Nếu hàm số $y=f(x)$ không đổi trên khoảng $(a;b)$ thì $f'(x)=0$ với mọi $x$ thuộc $(a;b)$}
	{Nếu hàm số $y=f(x)$ đồng biến trên khoảng $(a;b)$ thì $f'(x)\geq 0$ với mọi $x$ thuộc $(a;b)$}
	\loigiai{
		Mệnh đề ``Nếu $f'(x)\geq 0$ với mọi $x$ thuộc $(a;b)$ thì hàm số $y=f(x)$ đồng biến trên khoảng $(a;b)$'' sai do thiếu điều kiện dấu bằng trong $f'(x)\geq 0$ xảy ra tại hữu hạn điểm, vì nếu $f'(x)=0$ với mọi $x$ thuộc $(a;b)$ thì hàm số $y=f(x)$ là hàm hằng không phải là hàm đồng biến trên khoảng $(a;b)$.}
\end{ex}
\Closesolutionfile{ans}
%ans/ans2D1-1-5