\Opensolutionfile{ans}[ans/ans2D1-4-5]
\begin{dang}{Tổng hợp tiệm cận với diện tích, góc, khoảng cách}.
\end{dang}
\paragraph{Các ví dụ}
\begin{vd}%[2D1B4-3]%[Thầy Hải Toán]%Ví dụ 1.
	Khoảng cách từ gốc tọa độ đến giao điểm của hai đường tiệm cận của đồ thị hàm số $y=\dfrac{2x+1}{x+1}$ bằng
	\choice
	{\True $\sqrt{5}$}
	{$5$}
	{$\sqrt{3}$}
	{$\sqrt{2}$}
	\loigiai{
		Ta có $\lim\limits_{x\to+\infty} y=\lim\limits_{x\to-\infty} y=2$ nên đồ thị hàm số có đường tiệm cận ngang là $y=2$.\\
		Vì $\lim\limits_{x\to-1^+} y=-\infty$; $\lim\limits_{x\to-1^-} y=+\infty$ nên đồ thị hàm số có đường tiệm cận đứng là $x=-1$.\\
		Giao điểm của hai đường tiệm cận là $I(-1; 2)$. Vậy $OI=\sqrt{5}$.}
\end{vd}
\begin{vd}%[2D1B4-3]%[Thầy Hải Toán]%Ví dụ 2.
	Gọi $(H)$ là đồ thị hàm số $y=\dfrac{2x+3}{x+1}$. Điểm $M(x_0;y_0)$ thuộc (H) có tổng khoảng cách đến hai đường tiệm cận là nhỏ nhất, với $x_0<0$ khi đó $x_0+y_0$ bằng
	\choice
	{$-2$}
	{\True $-1$}
	{$0$}
	{$3$}
	\loigiai{
		Tập xác định. $\mathbb{R}\setminus\{-1\}$
		Dễ có tiệm cận đứng $d_1\colon x=-1$ và tiệm cận ngang $d_2\colon y=2$.\\
		Ta có $\mathrm{d}(M, d_1)+\mathrm{d}(M, d_2)=|x_0+1|+\left|\dfrac{2x_0+3}{x_0+1}-1\right|=|x_0+1|+\left|\dfrac{1}{x_0+1}\right|\geq 2$.\\
		Đẳng thức xảy ra khi và chỉ khi $|x_0+1|=\left|\dfrac{1}{x_0+1}\right|\Leftrightarrow x_0=0\vee x_0=-2$.\\
		Vì $x_0<0$ nên $x_0=-2\Rightarrow y_0=1\Rightarrow x_0+y_0=-1$.}
\end{vd}
\begin{vd}%[2D1K4-2]%[Thầy Hải Toán]%Ví dụ 3.
	Cho hàm số $y=\dfrac{4x-3}{x-3}$ có đồ thị $(C)$. Biết đồ thị $(C)$ có hai điểm phân biệt $M$, $N$ và tổng khoảng cách từ $M$ hoặc $N$ tới hai tiệm cận là nhỏ nhất. Khi đó $MN$ có giá trị bằng 
	\choice
	{$MN=4\sqrt{2}$}
	{$MN=6$}
	{$MN=4\sqrt{3}$}
	{\True $MN=6\sqrt{2}$}
	\loigiai{
		- Giả sử $M=\left(m;\dfrac{4m-3}{m-3}\right)\in(C)$, với $m\neq 3$.\\
		- Tiệm cận đứng là $x=3$, riệm cận ngang là $y=4$.\\
		Do đó tổng khoảng cách từ $M$ đến hai tiệm cận là\\
		$\mathrm{d}=|m-3|+\left|\dfrac{4m-3}{m-3}-4\right|=|m-3|+\dfrac{9}{|m-3|}\geq 2\cdot\sqrt{|m-3|\cdot\dfrac{9}{|m-3|}}=6$.\\
		Dấu \lq\lq =\rq\rq\, xảy ra khi và chỉ khi $|m-3|=\dfrac{9}{|m-3|}\Leftrightarrow(m-3)^2=9\Leftrightarrow\hoac{&m-3=3\\&m-3=-3}\Leftrightarrow\hoac{&m=6\\&m=0}$ \\
		$ \Rightarrow\hoac{&M=(6;7)\\&M=(0;1)} $. Một cách tương tự ta có các điểm $\hoac{&N=(6;7)\\&N=(0;1).}$ \\
		Do $M$, $N$ phân biệt nên $MN=6\sqrt{2}$.}
\end{vd}
\paragraph{Câu hỏi trắc nghiệm}
\begin{ex}%[2D1B4-3]%[Thầy Hải Toán]%Câu 1.
	Gọi $I$ là giao điểm của hai đường tiệm cận của đồ thị hàm số $y=\dfrac{2x-3}{x+1}$. Khi đó, điểm $I$ nằm trên đường thẳng có phương trình: 
	\choice
	{$x+y+4=0$}
	{\True $2x-y+4=0$}
	{$x-y+4=0$}
	{$2x-y+2=0$}
	\loigiai{
		Đồ thị hàm số đã cho có đường tiệm cận đứng là $x=-1$, tiệm cận ngang là $y=2$, do đó $I(-1;2)$, thay vào các phương trình thì $I$ thuộc đường thẳng $2x-y+4=0$.}
\end{ex}
\begin{ex}%[2D1K4-2]%[Thầy Hải Toán]%Câu 2.
	Đường tiệm cận đứng và đường tiệm cận ngang của đồ thị hàm số $y=\dfrac{mx+1}{2m+1-x}$ cùng với hai trục tọa độ tạo thành một hình chữ nhật có diện tích bằng $3$. Tìm $m$. 
	\choice
	{$m=1$; $m=\dfrac{3}{2}$}
	{$m=-1$; $m=-\dfrac{3}{2}$}
	{\True $m=1$; $m=-\dfrac{3}{2}$}
	{$m=-1$; $m=3$}
	\loigiai{
		Ta có $\lim\limits_{x\to+\infty}\dfrac{mx+1}{2m+1-x}=-m$; $\lim\limits_{x\to(2m+1)^+}\dfrac{mx+1}{2m+1-x} =\lim\limits_{x\to(2m+1)^+}\dfrac{m(2m+1)+1}{2m+1-x} =\lim\limits_{x\to(2m+1)^+}\dfrac{2m^2+m+1}{2m+1-x}$
		$\lim\limits_{x\to(2m+1)^+}\left(2m^2+m+1\right)=2m^2+m+1>0$; $\lim\limits_{x\to(2m+1)^+}(2m+1-x)=0$ và $2m+1-x<0\forall x>2m+1$ \\
		$ \Rightarrow\lim\limits_{x\to(2m+1)^+}\dfrac{mx+1}{2m+1-x}=-\infty $.\\
		Vậy đồ thị hàm số có hai đường tiệm cận $x=2m+1$ và $y=-m$.\\
		Hai đường tiệm cận tạo với hai trục tọa độ một hình chữ nhật có diện tích bằng $3$ suy ra $|2m+1|\cdot|m|=3\Leftrightarrow\hoac{&2m^2+m=3\\&2m^2+m=-3(PTVN)}\Leftrightarrow 2m^2+m-3=0\Leftrightarrow\hoac{&m=1\\&m=\dfrac{-3}{2}}$.}
\end{ex}
% \begin{ex}%[2D1K4-3]%[Thầy Hải Toán]%Câu 3.
% 	Cho hàm số $y=\dfrac{x-1}{2x-3}$. Gọi $I$ là giao điểm của hai tiệm cận của đồ thị hàm số. Khoảng cách từ $I$ đến tiếp tuyến của đồ thị hàm số đã cho đạt giá trị lớn nhất bằng
% 	\choice
% 	{\True $d=\dfrac{1}{\sqrt{2}}$}
% 	{$d=1$}
% 	{$d=\sqrt{2}$}
% 	{$d=\sqrt{5}$}
% 	\loigiai{
% 		Tọa độ giao điểm $I=\left(\dfrac{3}{2};\dfrac{1}{2}\right)$.\\
% 		Gọi tọa độ tiếp điểm là $\left(x_0;\dfrac{x_0-1}{2x_0+3}\right)$. Khi đó phương trình tiếp tuyến $\Delta$ với đồ thị hàm số tại điểm $\left(x_0;\dfrac{x_0-1}{2x_0+3}\right)$ là\\
% 		$y=-\dfrac{1}{(2x_0-3)^2}(x-x_0)+\dfrac{x_0-1}{2x_0-3}\Leftrightarrow x+(2x_0-3)^2y-2x_0^2+4x_0-3=0$.\\
% 		Khi đó: $\mathrm{d}(I,\Delta)=\dfrac{\left|\dfrac{3}{2}+\dfrac{1}{2}(2x_0-3)^2-2x_0^2+4x_0-3\right|}{\sqrt{1+(2x_0-3)^4}}=\dfrac{|-2x_0+3|}{\sqrt{1+(2x_0-3)^4}}\leq\dfrac{|2x_0-3|}{\sqrt{2(2x_0-3)^2}}=\dfrac{1}{\sqrt{2}}$.\\
% 		(Theo bất đẳng thức Cô si).\\
% 		Dấu xảy ra khi và chỉ khi $(2x_0-3)^2=1\Leftrightarrow\hoac{&2x_0-3=1\\&2x_0-3=-1}\Leftrightarrow\hoac{&x_0=2\\&x_0=1.}$ \\
% 		Vậy $\max\mathrm{d}(I,\Delta)=\dfrac{1}{\sqrt{2}}$.}
% \end{ex}
\begin{ex}%[2D1K4-3]%[Thầy Hải Toán]%Câu 4.
	Cho hàm số $y=\dfrac{2x+1}{x-m}$ có đồ thị là $(C_m)$. Tìm tổng tất cả các giá trị $m$ nguyên dương sao cho diện tích hình hình chữ nhật tạo bởi các trục tọa độ và hai đường tiệm cận của đồ thị $(C_m)$ không vượt quá $2018$ (đvdt). 
	\choice
	{\True $509545$}
	{$1009$}
	{$2018!$}
	{$2018$}
	\loigiai{
		Do $m\in\mathbb{Z}^+$ nên $(C_m)$ luôn có hai đường tiệm cận là $x=m$ và $y=2$.\\
		Khi đó diện tích hình thang cần tìm là $S=2m$.\\
		Có $S\leq 2018\Leftrightarrow 2m\leq 2018\Leftrightarrow m\leq 1009\Leftrightarrow m\in\left\{1;2;3;\ldots;1009\right\}$.\\
		$1+2+3+\cdots +1009=\dfrac{1009\cdot 1010}{2}=509545$.}
\end{ex}
\begin{ex}%[2D1K4-3]%[Thầy Hải Toán]%Câu 5.
	Cho hàm số $y=\dfrac{4x-3}{x-3}$ có đồ thị $(C)$. Biết đồ thị $(C)$ có hai điểm $M,N$ và tổng khoảng cách từ $M$ hoặc $N$ đến hai đường tiệm cận là nhỏ nhất. Khi đó $MN$ có giá trị bằng
	\choice
	{$MN=4\sqrt{2}$}
	{$MN=6$}
	{$MN=4\sqrt{3}$}
	{\True $MN=6\sqrt{2}$}
	\loigiai{
		$M\in(C)\Rightarrow M\left(m;\dfrac{4m-3}{m-3}\right)$, $m\neq 3$.\\
		Tiệm cận đứng $\Delta_1\colon x-3=0\Rightarrow\mathrm{d}\left(M,{\Delta}_1\right)=|m-3|$.\\
		Tiệm cận ngang $\Delta_2\colon y-4=0\Rightarrow\mathrm{d}\left(M,{\Delta}_2\right)=\left|\dfrac{4m-3}{m-3}-4\right|=\dfrac{9}{|m-3|}$ \\
		$ \Rightarrow\mathrm{d}\left(M,{\Delta}_1\right)+\mathrm{d}\left(M,{\Delta}_2\right) =|m-3|+\dfrac{9}{|m-3|}\geq 6 $ \\
		$ \Rightarrow\left(\mathrm{d}\left(M,{\Delta}_1\right)+\mathrm{d}\left(M,{\Delta}_2\right)\right)_{\min} =6 $ đạt được khi $|m-3|=\dfrac{9}{|m-3|}$ \\
		$ \Leftrightarrow(m-3)^2=9\Leftrightarrow m^2-6m=0\Leftrightarrow\hoac{&m=0\\&m=6.} $ \\
		Với $m=0$ ta có $M(0;1)$.\\
		Với $m=6$ ta có $N(6;7)$ \\
		$ \Rightarrow MN=6\sqrt{2} $.}
\end{ex}
\begin{ex}%[2D1K4-3]%[Thầy Hải Toán]%Câu 6.
	Cho hàm số $y=\dfrac{2x-3}{x-1}\quad(C)$. Gọi $M$ là điểm thuộc $(C)$ và $d$ là tổng khoảng cách từ $M$ đến hai tiệm cận của $(C)$. Giá trị nhỏ nhất của $d$ là
	\choice
	{\True $2$}
	{$\dfrac{3}{2}$}
	{$1$}
	{$6$}
	\loigiai{
		Ta có: $y=\dfrac{2x-3}{x-1}=2-\dfrac{1}{x-1}$. Gọi $M\left(x_o; 2-\dfrac{1}{x_o-1}\right)$, $x_o\neq 1$ là điểm thuộc $(C)$.\\
		Tiệm cận đứng $x=1$ và tiệm cận ngang $y=2$.\\
		Khoảng cách $M$ đến hai tiệm cận là\\
		$d=|x_o-1|+\dfrac{1}{|x_o-1|}\geq 2$ và $d=2$ khi $|x_o-1|=\dfrac{1}{|x_o-1|}\Leftrightarrow\hoac{&x_o=0\\&x_o=2}$.}
\end{ex}
\begin{ex}%[2D1K4-3]%[Thầy Hải Toán]%Câu 7.
	Giả sử đường thẳng $d\colon x=a(a>0)$ cắt đồ thị hàm số $y=\dfrac{2x+1}{x-1}$ tại một điểm duy nhất, biết khoảng cách từ điểm đó đến tiệm cận đứng của đồ thị hàm số bằng 1, ký hiệu $(x_0;y_0)$ là tọa độ của điểm đó. Tìm $y_0$. 
	\choice
	{$y_0=-1$}
	{\True $y_0=5$}
	{$y_0=1$}
	{$y_0=2$}
	\loigiai{
		Gọi $M\left(a;\dfrac{2a+1}{a-1}\right)=d\cap(C)$. Khoảng cách từ $M$ đến tiệm cận đứng $\Delta\colon x=1$ là $\mathrm{d}(M,\Delta)=|a-1|=1\Rightarrow\hoac{&a=0(l)\\&a=2}$ suy ra $M(2;5)$ nên $y_0=5$.}
\end{ex}
% \begin{ex}%[2D1G4-3]%[Thầy Hải Toán]%Câu 8.
% 	Cho hàm số $y=\dfrac{2x+2}{x-1}$ có đồ thị $(C)$. Một tiếp tuyến bất kỳ với $(C)$ cắt đường tiệm cận đứng và đường tiệm cận ngang của $(C)$ lần lượt tại $A,B$. Gọi $I(1;2)$. Giá trị lớn nhất của bán kính đường tròn nội tiếp $\Delta IAB$ là
% 	\choice
% 	{$8-4\sqrt{2}$}
% 	{\True $4-2\sqrt{2}$}
% 	{$8-3\sqrt{2}$}
% 	{$7-3\sqrt{2}$}
% 	\loigiai{
% 		Ta có $y’=-\dfrac{4}{(x-1)^2}$.\\
% 		Gọi $M(x_0;y_0)$ là 1 điểm bất kì thuộc (C). Ta có $x_0\neq 1$.\\
% 		PTTT $\Delta$ tại $M$ có dạng: $y=\dfrac{-4}{(x_0-1)^2}(x-x_0)+\dfrac{2x_0+2}{x_0-1}$.\\
% 		Ta có tiệm cận đứng $d_1\colon x=1$ và tiệm cận ngang $d_2\colon y=2$.\\
% 		Khi đó $\Delta\cap d_1=A\left(1;2+\dfrac{8}{x_0-1}\right)$ và $\Delta\cap d_2=B(2x_0-1;2)$ \\
% 		$ \Rightarrow IA=\dfrac{8}{|x_0-1|} $, $IB=2|x_0-1|$, $AB=2\sqrt{(x_0-1)^2+\dfrac{16}{(x_0-1)^2}}$.\\
% 		Do $\Delta IAB$ vuông tại $I$ nên: $S_{\triangle IAB}=\dfrac{1}{2}IA\cdot IB=8$.\\
% 		Mà $S_{\triangle IAB}=p\cdot r\Rightarrow 8=\dfrac{IA+IB+AB}{2}\cdot r=\left[\dfrac{4}{|x_0-1|}+|x_0-1|+\sqrt{(x_0-1)^2+\dfrac{16}{(x_0-1)^2}}\right]\cdot r\geq\left(4+2\sqrt{2}\right)r$ \\
% 		$ \Rightarrow r\leq\dfrac{8}{4+2\sqrt{2}}=4-2\sqrt{2} $.\\
% 		$x_0=3\Rightarrow r=4-2\sqrt{2}\Rightarrow 4-2\sqrt{2}$ là GTLN của r.}
% \end{ex}
\begin{ex}%[2D1K4-2]%[Thầy Hải Toán]%Câu 9.
	Tìm $m$ để tiệm cận ngang của đồ thị hàm số $y=\dfrac{(m-1)x+2}{3x+4}$ cắt đường thẳng $2x-3y+5=0$ tại điểm có hoành độ bằng $2$. 
	\choice
	{$m=2$}
	{$m=1$}
	{\True $m=10$}
	{$m=7$}
	\loigiai{
		Ta có: $\lim\limits_{x\to\pm\infty}\dfrac{(m-1)x+2}{3x+4}=\dfrac{m-1}{3}$ suy ra tiệm cận ngang của đồ thị là $y=\dfrac{m-1}{3}$.\\
		Với $M(2;y)\in d\colon 2x-3y+5=0\Rightarrow M(2; 3)$.\\
		Theo đề: $M(2; 3)\in TCN\colon y=\dfrac{m-1}{3}\Rightarrow\dfrac{m-1}{3}=3\Leftrightarrow m=10$.}
\end{ex}
\begin{ex}%[2D1K4-3]%[Thầy Hải Toán]%Câu 10.
	Cho hàm số $y=\dfrac{2x-3}{x-2}$ có đồ thị $(C)$. Gọi $M$ là điểm thuộc đồ thị $(C)$ và $d$ là tổng khoảng cách từ $M$ tới hai tiệm cận của $(C)$. Giá trị nhỏ nhất của $d$ có thể đạt được là
	\choice
	{$5$}
	{\True $2$}
	{$6$}
	{$10$}
	\loigiai{
		Ta $M$ là điểm thuộc đồ thị $(C) y=\dfrac{2x-3}{x-2}$ nên $M\left(a;\dfrac{2a-3}{a-2}\right)$.\\
		Phương trình của 2 đường tiệm cận của đồ thị $(C) x=2;y=2$.\\
		Ta lại có $\mathrm{d}(M;x=2)=|a-2|;\mathrm{d}(M;y=2)=\left|\dfrac{2a-3}{a-2}-2\right|=\dfrac{1}{|a-2|}$ \\
		$ \Rightarrow d=\mathrm{d}(M;x=2)+\mathrm{d}(M;y=2)=|a-2|+\dfrac{1}{|a-2|}\geq 2 $.\\
		Dấu \lq\lq =\rq\rq\, xảy ra khi $\Rightarrow|a-2|=\left|\dfrac{1}{a-2}\right|\Leftrightarrow(a-2)^2=1\Leftrightarrow a=3(do a>2)$.\\
		Vậy khoảng cách nhỏ nhất của $d$ có thể đạt được là 2. Chọn đáp án B.}
\end{ex}
\begin{ex}%[2D1K4-3]%[Thầy Hải Toán]%Câu 11.
	Cho hàm số $y=\dfrac{4x-3}{x-3}$ có đồ thị $(C)$. Biết đồ thị $(C)$ có hai điểm $M,N$ và tổng khoảng cách từ $M$ hoặc $N$ đến hai đường tiệm cận là nhỏ nhất. Khi đó $MN$ có giá trị bằng
	\choice
	{$MN=4\sqrt{2}$}
	{$MN=6$}
	{$MN=4\sqrt{3}$}
	{\True $MN=6\sqrt{2}$}
	\loigiai{
		$M\in(C)\Rightarrow M\left(m;\dfrac{4m-3}{m-3}\right)$, $m\neq 3$.\\
		Tiệm cận đứng $\Delta_1\colon x-3=0\Rightarrow\mathrm{d}\left(M,{\Delta}_1\right)=|m-3|$.\\
		Tiệm cận ngang $\Delta_2\colon y-4=0\Rightarrow\mathrm{d}\left(M,{\Delta}_2\right)=\left|\dfrac{4m-3}{m-3}-4\right|=\dfrac{9}{|m-3|}$ \\
		$ \Rightarrow\mathrm{d}\left(M,{\Delta}_1\right)+\mathrm{d}\left(M,{\Delta}_2\right) =|m-3|+\dfrac{9}{|m-3|}\geq 6 $ \\
		$ \Rightarrow\left(\mathrm{d}\left(M,{\Delta}_1\right)+\mathrm{d}\left(M,{\Delta}_2\right)\right)_{\min} =6 $ đạt được khi $|m-3|=\dfrac{9}{|m-3|}$ \\
		$ \Leftrightarrow(m-3)^2=9\Leftrightarrow m^2-6m=0\Leftrightarrow\hoac{&m=0\\&m=6.} $ \\
		Với $m=0$ ta có $M(0;1)$.\\
		Với $m=6$ ta có $N(6;7)$ \\
		$ \Rightarrow MN=6\sqrt{2} $.}
\end{ex}
\begin{ex}%[2D1K4-3]%[Thầy Hải Toán]%Câu 12.
	Cho hàm số $y=\dfrac{2x-3}{x-1}\quad(C)$. Gọi $M$ là điểm thuộc $(C)$ và $d$ là tổng khoảng cách từ $M$ đến hai tiệm cận của $(C)$. Giá trị nhỏ nhất của $d$ là
	\choice
	{\True $2$}
	{$\dfrac{3}{2}$}
	{$1$}
	{$6$}
	\loigiai{
		Ta có: $y=\dfrac{2x-3}{x-1}=2-\dfrac{1}{x-1}$. Gọi $M\left(x_o; 2-\dfrac{1}{x_o-1}\right)$, $x_o\neq 1$ là điểm thuộc $(C)$.\\
		Tiệm cận đứng $x=1$ và tiệm cận ngang $y=2$.\\
		Khoảng cách $M$ đến hai tiệm cận là\\
		$d=|x_o-1|+\dfrac{1}{|x_o-1|}\geq 2$ và $d=2$ khi $|x_o-1|=\dfrac{1}{|x_o-1|}\Leftrightarrow\hoac{&x_o=0\\&x_o=2}$.}
\end{ex}
\begin{ex}%[2D1K4-3]%[Thầy Hải Toán]%Câu 13.
	Cho hàm số $y=\dfrac{2x-3}{x-2}$ có đồ thị $(C)$. Gọi $M$ là điểm thuộc đồ thị $(C)$ và $d$ là tổng khoảng cách từ $M$ tới hai tiệm cận của $(C)$. Giá trị nhỏ nhất của $d$ có thể đạt được là
	\choice
	{$5$}
	{\True $2$}
	{$6$}
	{$10$}
	\loigiai{
		Ta $M$ là điểm thuộc đồ thị $(C) y=\dfrac{2x-3}{x-2}$ nên $M\left(a;\dfrac{2a-3}{a-2}\right)$.\\
		Phương trình của 2 đường tiệm cận của đồ thị $(C) x=2;y=2$.\\
		Ta lại có $\mathrm{d}(M;x=2)=|a-2|;\mathrm{d}(M;y=2)=\left|\dfrac{2a-3}{a-2}-2\right|=\dfrac{1}{|a-2|}$ \\
		$ \Rightarrow d=\mathrm{d}(M;x=2)+\mathrm{d}(M;y=2)=|a-2|+\dfrac{1}{|a-2|}\geq 2 $.\\
		Dấu “=” xảy ra khi $\Rightarrow|a-2|=\left|\dfrac{1}{a-2}\right|\Leftrightarrow(a-2)^2=1\Leftrightarrow a=3(do a>2)$.\\
		Vậy khoảng cách nhỏ nhất của $d$ có thể đạt được là 2. Chọn đáp án B.}
\end{ex}
\begin{ex}%[2D1K4-3]%[Thầy Hải Toán]%Câu 14.
	Cho đồ thị $(C)$ hàm số $y=\dfrac{2x+2}{x-1}$. Tọa độ điểm $M$ nằm trên $(C)$ sao cho tổng khoảng cách từ $M$ đến hai tiệm cận của $(C)$ nhỏ nhất là
	\choice
	{\True $M(-1;0)$ hoặc $M(3;4)$}
	{$M(-1;0)$ hoặc $M(0;-2)$}
	{$M(2;6)$ hoặc $M(3;4)$}
	{$M(0;-2)$ hoặc $M(2;6)$}
	\loigiai{
		Ta có tiệm cận đứng: $x=1$, tiệm cận ngang $y=2$.\\
		Gọi $M(x_0;y_0)\in(C)$ với $x_0\neq 1$ thì $y_0=\dfrac{2x_0+2}{x_0-1}=2+\dfrac{4}{x_0-1}$.\\
		Gọi $A$, $B$ lần lượt là hình chiếu của $M$ trên tiệm cận đứng và tiệm cận ngang.\\
		Ta có $MA=|x_0-1|$, $MB=|y_0-2|=\left|\dfrac{4}{x_0-1}\right|$.\\
		Áp dụng bất đẳng thức AM-GM ta có: $MA+MB\geq 2\sqrt{MA\cdot MB}$ \\
		$ \Rightarrow MA+MB\geq 2\sqrt{|x_0-1|\cdot\left|\dfrac{4}{x_0-1}\right|}=4 $.\\
		Do đó $MA+MB$ nhỏ nhất bằng $4$ khi và chỉ khi $|x_0-1|=\dfrac{4}{|x_0-1|}$ \\
		$ \Leftrightarrow|x_0-1|^2=4\Leftrightarrow\hoac{&x_0=3\Rightarrow y_0=4\\&x_0=-1\Rightarrow y_0=0.}$ \\
		Vậy có hai điểm cần tìm là $M(-1;0)$ hoặc $M(3;4)$.}
\end{ex}
\begin{ex}%[2D1K4-3]%[Thầy Hải Toán]%Câu 15.
	Cho đường cong $(C)\colon y=\dfrac{2x+3}{x-1}$ và $M$ là một điểm nằm trên $(C)$. Giả sử $d_1$, $d_2$ tương ứng là các khoảng cách từ $M$ đến hai tiệm cận của $(C)$, khi đó $d_1\cdot d_2$ bằng
	\choice
	{$3$}
	{$4$}
	{\True $5$}
	{$6$}
	\loigiai{
		Ta có $\lim\limits_{x\to 1^+} y=+\infty\Rightarrow x=1$ là tiệm cận đứng; $\lim\limits_{x\to+\infty} y=2\Rightarrow y=2$ là tiệm cận ngang.\\
		$M\in(C)\Rightarrow M\left(a; 2+\dfrac{5}{a-1}\right)$ với $a\neq 1$.\\
		Khoảng cách từ $M$ đến tiệm cận đứng $d_1=\dfrac{|a-1|}{\sqrt{1}}=|a-1|$,\\
		Khoảng cách từ $M$ đến tiệm ngang $d_2=\dfrac{\left|2+\dfrac{5}{a-1}-2\right|}{\sqrt{0^2+1^2}}=\left|\dfrac{5}{a-1}\right|$.\\
		Xét $d_1\cdot d_2=|a-1|\cdot\left|\dfrac{5}{a-1}\right|=\left|(a-1)\cdot\dfrac{5}{a-1}\right|=5$.}
\end{ex}	
\Closesolutionfile{ans}