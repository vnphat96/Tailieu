\begin{dang}
    {TÍNH GIÁ TRỊ LƯỢNG GIÁC}
\end{dang}
% \Opensolutionfile{ans}[ans/ans10D1-TN-3]
% \setcounter{ex}{0}
\begin{ex}%[0D6B3-2]
Rút gọn biểu thức $M=\cos ^415^{\circ}-\sin ^415^{\circ}.$
\choice
{$M=1$}
{\True $M=\dfrac{{\sqrt{3}}}{2}$}
{$M=\dfrac{1}{4}$}
{$M=0$}
\loigiai{ Ta có $M=\cos ^415^{\circ}-\sin ^415^{\circ}=\left({{\cos}^2{15}^{\circ}}\right)^2-\left({{\sin}^2{15}^{\circ}}\right)^2$
$=\left({{\cos}^2{15}^{\circ}-{\sin}^2{15}^{\circ}}\right)\left({{\cos}^2{15}^{\circ}+{\sin}^2{15}^{\circ}}\right)$
$=\cos ^215^{\circ}-\sin ^215^{\circ}=\cos \left({{2.15}^{\circ}}\right)=\cos 30^{\circ}=\dfrac{{\sqrt{3}}}{2}.$
  } 
\end{ex}

\begin{ex}%[0D6B3-2]
Tính giá trị của biểu thức $M=\cos ^415^{\circ}-\sin ^415^{\circ}+\cos ^215^{\circ}-\sin ^215^{\circ}.$
\choice
{\True $M=\sqrt{3}$}
{$M=\dfrac{1}{2}$}
{$M=\dfrac{1}{4}$}
{$M=0$}
\loigiai{ Áp dụng công thức nhân đôi $\cos ^2a-\sin ^2a=\cos 2a$.\\
Ta có
\begin{eqnarray*}
 M & = & \left({{\cos}^4{15}^{\circ}-{\sin}^4{15}^{\circ}}\right)+\left({{\cos}^2{15}^{\circ}-{\sin}^2{15}^{\circ}}\right)\\
& =& \left({{\cos}^2{15}^{\circ}-{\sin}^2{15}^{\circ}}\right)\left({{\cos}^2{15}^{\circ}+{\sin}^2{15}^{\circ}}\right)+\left({{\cos}^2{15}^{\circ}-{\sin}^2{15}^{\circ}}\right)\\
& = & \left({{\cos}^2{15}^{\circ}-{\sin}^2{15}^{\circ}}\right)+\left({{\cos}^2{15}^{\circ}-{\sin}^2{15}^{\circ}}\right)\\
& = &\cos 30^{\circ}+\cos 30^{\circ}=\sqrt{3}.
\end{eqnarray*}

  } 
\end{ex}

\begin{ex}%[0D6B3-2]
Tính giá trị của biểu thức $M=\cos ^615^{\circ}-\sin ^615^{\circ}.$
\choice
{$M=1$}
{$M=\dfrac{1}{2}$}
{$M=\dfrac{1}{4}$}
{\True $M=\dfrac{{15\sqrt{3}}}{{32}}$}
\loigiai{ Ta có 
\begin{eqnarray*}
\cos ^6\alpha-\sin ^6\alpha & = & \left({\cos}^2\alpha-{\sin}^2\alpha\right)\left({{\cos}^4\alpha+{\cos}^2\alpha\cdot {\sin}^2\alpha+{\sin}^4\alpha}\right) \\
&& =\cos 2\alpha\cdot \left[{{\left({{\cos}^2\alpha+{\sin}^2\alpha}\right)}^2-{\cos}^2\alpha \cdot {\sin}^2\alpha}\right] \\
&&=\cos 2\alpha \cdot \left({1-\dfrac{1}{4}{\sin}^22\alpha}\right). 
\end{eqnarray*}
Vậy $M=\cos 30^{\circ}\cdot \left(1-\dfrac{1}{4}{\sin}^2{30}^{\circ}\right)=\dfrac{\sqrt{3}}{2}\cdot \left(1-\dfrac{1}{4}\cdot\dfrac{1}{4}\right)=\dfrac{15\sqrt{3}}{32}.$
  } 
\end{ex}

\begin{ex}%[0D6B3-1]
Giá trị của biểu thức $\cos \dfrac{\pi}{{30}}\cos \dfrac{\pi}{5}+\sin \dfrac{\pi}{{30}}\sin \dfrac{\pi}{5}$ là
\choice
{\True $\dfrac{{\sqrt{3}}}{2}$}
{$-\dfrac{{\sqrt{3}}}{2}$}
{$\dfrac{{\sqrt{3}}}{4}$}
{$\dfrac{1}{2}$}
\loigiai{ Ta có $\cos \dfrac{\pi}{{30}}\cos \dfrac{\pi}{5}+\sin \dfrac{\pi}{{30}}\sin \dfrac{\pi}{5}=\cos \left({\dfrac{\pi}{{30}}-\dfrac{\pi}{5}}\right)=\cos \left({-\dfrac{\pi}{6}}\right)=\dfrac{{\sqrt{3}}}{2}.$  } 
\end{ex}

\begin{ex}%[0D6B3-1]
Giá trị của biểu thức $P=\dfrac{{\sin \dfrac{{5\pi}}{{18}}\cos \dfrac{\pi}{9}-\sin \dfrac{\pi}{9}\cos \dfrac{{5\pi}}{{18}}}}{{\cos \dfrac{\pi}{4}\cos \dfrac{\pi}{{12}}-\sin \dfrac{\pi}{4}\sin \dfrac{\pi}{{12}}}}$ là
\choice
{\True $1$}
{$\dfrac{1}{2}$}
{$\dfrac{{\sqrt{2}}}{2}$}
{$\dfrac{{\sqrt{3}}}{2}$}
\loigiai{ Áp dụng công thức $\heva{& \sin a\cdot \cos b-\cos a\cdot \sin b=\sin \left({a-b}\right) \\
& \cos a\cdot\cos b-\sin a\cdot\sin b=\cos \left({a+b}.\right)}$

Khi đó $\sin \dfrac{{5\pi}}{{18}}\cos \dfrac{\pi}{9}-\sin \dfrac{\pi}{9}\cos \dfrac{{5\pi}}{{18}}=\sin \left({\dfrac{{5\pi}}{{18}}-\dfrac{\pi}{9}}\right)=\sin \dfrac{\pi}{6}=\dfrac{1}{2}.$

Và $\cos \dfrac{\pi}{4}\cos \dfrac{\pi}{{12}}-\sin \dfrac{\pi}{4}\sin \dfrac{\pi}{{12}}=\cos \left({\dfrac{\pi}{4}+\dfrac{\pi}{{12}}}\right)=\cos \dfrac{\pi}{3}=\dfrac{1}{2}.$ Vậy $P=\dfrac{1}{2}:\dfrac{1}{2}=1.$
  } 
\end{ex}

\begin{ex}%[0D6B3-1]
Giá trị đúng của biểu thức $\dfrac{{\tan {225}^{\circ}-\cot {81}^{\circ}\cdot \cot{69}^{\circ}}}{{\cot {261}^{\circ}+\tan {201}^{\circ}}}$ bằng
\choice
{$\dfrac{1}{{\sqrt{3}}}$}
{$-\dfrac{1}{{\sqrt{3}}}$}
{\True $\sqrt{3}$}
{$-\sqrt{3}$}
\loigiai{  Ta có $\dfrac{{\tan {225}^{\circ}-\cot {81}^{\circ}\cdot \cot{69}^{\circ}}}{{\cot {261}^{\circ}+\tan {201}^{\circ}}}=\dfrac{{\tan \left({{180}^{\circ}+{45}^{\circ}}\right)-\tan 9^{\circ}\cdot \cot{69}^{\circ}}}{{\cot \left({{180}^{\circ}+{81}^{\circ}}\right)+\tan \left({{180}^{\circ}+{21}^{\circ}}\right)}} =\dfrac{{1-\tan 9^{\circ}\cdot \tan {21}^{\circ}}}{{\tan 9^{\circ}+\tan {21}^{\circ}}} $ 

$=\dfrac{1}{{\tan \left({9^{\circ}+{21}^{\circ}}\right)}}=\dfrac{1}{{\tan {30}^{\circ}}}=\sqrt{3}.$
 } 
\end{ex}

\begin{ex}%[0D6K3-4]
Giá trị của biểu thức $M=\sin \dfrac{\pi}{{24}}\sin \dfrac{{5\pi}}{{24}}\sin \dfrac{{7\pi}}{{24}}\sin \dfrac{{11\pi}}{{24}}$ bằng
\choice
{$\dfrac{1}{2}$}
{$\dfrac{1}{4}$}
{$\dfrac{1}{8}$}
{\True $\dfrac{1}{{16}}$}
\loigiai{  
	Ta có $\sin \dfrac{{7\pi}}{{24}}=\cos \dfrac{{5\pi}}{{24}}$ và $\sin \dfrac{{11\pi}}{{24}}=\cos \dfrac{\pi}{{24}}$. \\
Do đó $M=\sin \dfrac{\pi}{{24}}\sin \dfrac{{5\pi}}{{24}}\cos \dfrac{{5\pi}}{{24}}\cos \dfrac{\pi}{{24}}=\dfrac{1}{4}\cdot \left({2\sin \dfrac{\pi}{{24}}\cdot \cos \dfrac{\pi}{{24}}}\right)\cdot \left({2\sin \dfrac{{5\pi}}{{24}}\cdot\cos \dfrac{{5\pi}}{{24}}}\right)$\\
$=\dfrac{1}{4}\cdot \sin \dfrac{\pi}{{12}}\cdot\sin \dfrac{{5\pi}}{{12}}=\dfrac{1}{4}\cdot \dfrac{1}{2}\left({\cos \dfrac{{6\pi}}{{12}}+\cos \dfrac{\pi}{3}}\right)=\dfrac{1}{8}\cdot \left({0+\dfrac{1}{2}}\right)=\dfrac{1}{{16}}.$
 } 
\end{ex}

\begin{ex}%[0D6K3-4]
Giá trị của biểu thức $M=\sin \dfrac{\pi}{48} \cos\dfrac{\pi}{48} \cos \dfrac{ \pi}{24} \cos \dfrac{\pi}{12} \cos \dfrac{\pi}{6}$ là
\choice
{$\dfrac{1}{32}$}
{$\dfrac{\sqrt{3}}{8}$}
{$\dfrac{\sqrt{3}}{16}$}
{\True $\dfrac{\sqrt{3}}{32}$}
\loigiai{  Áp dụng công thức $\sin 2a=2\sin a\cdot \cos a,$ ta có\\
$A=\sin \dfrac{\pi}{{48}}\cdot \cos \dfrac{\pi}{{48}}\cdot \cos \dfrac{\pi}{{24}}\cdot\cos\dfrac{\pi}{{12}}\cdot\cos\dfrac{\pi}{6}=\dfrac{1}{2}\cdot\sin \dfrac{\pi}{{24}}\cdot\cos \dfrac{\pi}{{24}}\cdot \cos\dfrac{\pi}{{12}}\cdot \cos\dfrac{\pi}{6}$\\
$=\dfrac{1}{4}\cdot \sin \dfrac{\pi}{{12}}\cdot\cos \dfrac{\pi}{{12}}\cdot\cos \dfrac{\pi}{6}=\dfrac{1}{8}\cdot\sin \dfrac{\pi}{6}\cdot\cos \dfrac{\pi}{6}=\dfrac{1}{{16}}\cdot\sin \dfrac{\pi}{3}=\dfrac{{\sqrt{3}}}{{32}}.$
 } 
\end{ex}

\begin{ex}%[0D6K3-4]
Tính giá trị của biểu thức $M=\cos 10^{\circ}\cos 20^{\circ}\cos 40^{\circ}\cos 80^{\circ}.$
\choice
{$M=\dfrac{1}{{16}}\cos 10^{\circ}$}
{$M=\dfrac{1}{2}\cos 10^{\circ}$}
{$M=\dfrac{1}{4}\cos 10^{\circ}$}
{\True $M=\dfrac{1}{8}\cos 10^{\circ}$}
\loigiai{  Vì $\sin 10^{\circ}\ne 0$ nên suy ra \\
$M=\dfrac{{16\sin {10}^{\circ}\cos {10}^{\circ}\cos {20}^{\circ}\cos {40}^{\circ}\cos {80}^{\circ}}}{{16\sin {10}^{\circ}}}=\dfrac{{8\sin {20}^{\circ}\cos {20}^{\circ}\cos {40}^{\circ}\cos {80}^{\circ}}}{{16\sin {10}^{\circ}}}$\\
$\Rightarrow M=\dfrac{{4\sin {40}^{\circ}\cos {40}^{\circ}\cos {80}^{\circ}}}{{16\sin {10}^{\circ}}}=\dfrac{{2\sin {80}^{\circ}\cos {80}^{\circ}}}{{16\sin {10}^{\circ}}}=\dfrac{{\sin {160}^{\circ}}}{{16\sin {10}^{\circ}}}$.\\
$\Rightarrow M=\dfrac{{\sin {20}^{\circ}}}{{16\sin {10}^{\circ}}}=\dfrac{{2\sin {10}^{\circ}\cos {10}^{\circ}}}{{16\sin {10}^{\circ}}}=\dfrac{1}{8}\cos 10^{\circ}$.
 } 
\end{ex}

\begin{ex}%[0D6K3-4]
Tính giá trị của biểu thức $M=\cos \dfrac{{2\pi}}{7}+\cos \dfrac{{4\pi}}{7}+\cos \dfrac{{6\pi}}{7}.$
\choice
{$M=0$}
{\True $M=-\dfrac{1}{2}$}
{$M=1$}
{$M=2$}
\loigiai{  Áp dụng công thức $\sin a-\sin b=2\cdot\cos \dfrac{{a+b}}{2}\cdot \sin \dfrac{{a-b}}{2}.$ \\
Ta có $2\sin \dfrac{\pi}{7}\cdot M=2\cdot\cos \dfrac{{2\pi}}{7}\cdot \sin \dfrac{\pi}{7}+2\cdot\cos \dfrac{{4\pi}}{7}\cdot\sin \dfrac{\pi}{7}+2\cdot\cos \dfrac{{6\pi}}{7}\cdot\sin \dfrac{\pi}{7}$\\
$=\sin \dfrac{{3\pi}}{7}-\sin \dfrac{\pi}{7}+\sin \dfrac{{5\pi}}{7}-\sin \dfrac{{3\pi}}{7}+\sin \dfrac{{7\pi}}{7}-\sin \dfrac{{5\pi}}{7}$ $=-\sin \dfrac{\pi}{7}+\sin \pi =-\sin \dfrac{\pi}{7}.$ \\
Vậy giá trị biểu thức $M=-\dfrac{1}{2}$.
 } \end{ex}

\begin{dang}
    {TÍNH ĐÚNG SAI}
\end{dang}

\begin{ex}%[0D6Y3-8]
Công thức nào sau đây sai?
\choice
{$\cos \left({a-b}\right)=\sin a\sin b+\cos a\cos b$}
{\True $\cos \left({a+b}\right)=\sin a\sin b-\cos a\cos b$}
{$\sin \left({a-b}\right)=\sin a\cos b-\cos a\sin b$}
{$\sin \left({a+b}\right)=\sin a\cos b+\cos a\sin b$}
\loigiai{Ta có $\cos \left({a+b}\right)=\cos a\cos b-\sin a\sin b$. } 
\end{ex}

\begin{ex}%[0D6B3-8]
Khẳng định nào sau đây đúng?
\choice
{$\sin \left({2018a}\right)=2018\sin a\cdot\cos a$}
{$\sin \left({2018a}\right)=2018\sin \left({1009a}\right)\cdot\cos \left({1009a}\right)$}
{$\sin \left({2018a}\right)=2\sin a\cos a$}
{\True $\sin \left({2018a}\right)=2\sin \left({1009a}\right)\cdot\cos \left({1009a}\right)$}
\loigiai{  Áp dụng công thức $\sin 2\alpha =2\sin \alpha\cdot\cos \alpha $ ta được
$\sin \left({2018a}\right)=2\sin \left({1009a}\right)\cdot\cos \left({1009a}\right)$.
 } 
\end{ex}

\begin{ex}%[0D6B3-8]
Khẳng định nào sai trong các khẳng định sau?
\choice
{$\cos 6a=\cos ^23a-\sin ^23a$}
{$\cos 6a=1-2\sin ^23a$}
{\True $\cos 6a=1-6\sin ^2a$}
{$\cos 6a=2\cos ^23a-1$}
\loigiai{  Áp dụng công thức $\cos 2\alpha =\cos ^2\alpha-\sin ^2\alpha =2\cos ^2\alpha-1=1-2\sin ^2\alpha $, ta được\\
$\cos 6a=\cos ^23a-\sin ^23a=2\cos ^23a-1=1-2\sin ^23a$.
 }
 \end{ex}

\begin{ex}%[0D6B3-8]
Khẳng định nào sai trong các khẳng định sau?
\choice
{$\sin ^2x=\dfrac{{1-\cos 2x}}{2}$}
{$\cos ^2x=\dfrac{{1+\cos 2x}}{2}$}
{$\sin x=2\sin \dfrac{x}{2}\cos \dfrac{x}{2}$}
{\True $\cos 3x=\cos ^3x-\sin ^3x$}
\loigiai{ Ta có $\cos 3x=4\cos ^3x-3\cos x$ } 
\end{ex}

\begin{ex}%[0D6B3-8]
Khẳng định nào đúng trong các khẳng định sau?
\choice
{$\sin a+\cos a=\sqrt{2}\sin \left({a-\dfrac{\pi}{4}}\right)$}
{\True $\sin a+\cos a=\sqrt{2}\sin \left({a+\dfrac{\pi}{4}}\right)$}
{$\sin a+\cos a=-\sqrt{2}\sin \left({a-\dfrac{\pi}{4}}\right)$}
{$\sin a+\cos a=-\sqrt{2}\sin \left({a+\dfrac{\pi}{4}}\right)$}
\loigiai{  } 
\end{ex}

\begin{ex}%[0D6B3-8]
Có bao nhiêu đẳng thức dưới đây là đồng nhất thức?\\
1) $\cos x-\sin x=\sqrt{2}\sin \left({x+\dfrac{\pi}{4}}\right).$\\
2) $\cos x-\sin x=\sqrt{2}\cos \left({x+\dfrac{\pi}{4}}\right).$\\
3) $\cos x-\sin x=\sqrt{2}\sin \left({x-\dfrac{\pi}{4}}\right).$\\
4) $\cos x-\sin x=\sqrt{2}\sin \left({\dfrac{\pi}{4}-x}\right).$
\choice
{$1$}
{\True $2$}
{$3$}
{$4$}
\loigiai{ Ta có $\cos x-\sin x=\sqrt{2}\cos \left({x+\dfrac{\pi}{4}}\right)=\sqrt{2}\cos \left[{\dfrac{\pi}{2}-\left({\dfrac{\pi}{4}-x}\right)}\right]=\sqrt{2}\sin \left({\dfrac{\pi}{4}-x}\right)$.

  } 
\end{ex}

\begin{ex}%[0D6B3-8]
Công thức nào sau đây đúng?
\choice
{$\cos 3a=3\cos a-4\cos ^3a$}
{\True $\cos 3a=4\cos ^3a-3\cos a$}
{$\cos 3a=3\cos ^3a-4\cos a$}
{$\cos 3a=4\cos a-3\cos ^3a$}
\loigiai{  } 
\end{ex}

\begin{ex}%[0D6B3-8]
Công thức nào sau đây đúng?
\choice
{\True $\sin 3a=3\sin a-4\sin ^3a$}
{$\sin 3a=4\sin ^3a-3\sin a$}
{$\sin 3a=3\sin ^3a-4\sin a$}
{$\sin 3a=4\sin a-3\sin ^3a$}
\loigiai{  } 
\end{ex}

\begin{ex}%[0D6K3-8]
Nếu $\cos \left({a+b}\right)=0$ thì khẳng định nào sau đây đúng?
\choice
{$\left|{\sin \left({a+2b}\right)}\right|=\left|{\sin a}\right|$}
{$\left|{\sin \left({a+2b}\right)}\right|=\left|{\sin b}\right|$}
{$\left|{\sin \left({a+2b}\right)}\right|=\left|{\cos a}\right|$}
{\True $\left|{\sin \left({a+2b}\right)}\right|=\left|{\cos b}\right|$}
\loigiai{  Ta có $\cos \left({a+b}\right)=0\Leftrightarrow a+b=\dfrac{\pi}{2}+k\pi \Rightarrow =-b+\dfrac{\pi}{2}+k\pi $.\\
$\Rightarrow\left|{\sin \left({a+2b}\right)}\right|=\left|{\sin \left({-b+2b+\dfrac{\pi}{2}+k\pi}\right)}\right|=\left|{\cos \left({b+k\pi}\right)}\right|=\left|{\cos b}\right|$.
 } 
\end{ex}

\begin{ex}%[0D6K3-8]
Nếu $\sin \left({a+b}\right)=0$ thì khẳng định nào sau đây đúng?
\choice
{$\left|{\cos \left({a+2b}\right)}\right|=\left|{\sin a}\right|$}
{$\left|{\cos \left({a+2b}\right)}\right|=\left|{\sin b}\right|$}
{$\left|{\cos \left({a+2b}\right)}\right|=\left|{\cos a}\right|$}
{\True $\left|{\cos \left({a+2b}\right)}\right|=\left|{\cos b}\right|$}
\loigiai{ Ta có $\sin \left({a+b}\right)=0\Leftrightarrow a+b=k\pi \Rightarrow a =-b+k\pi $
$\Rightarrow\left|{\cos \left({a+2b}\right)}\right|=\left|{\cos \left({-b+2b+k\pi}\right)}\right|=\left|{\cos \left({b+k\pi}\right)}\right|=\left|{\cos b}\right|$.
  } 
\end{ex}

\begin{dang}
    { VẬN DỤNG CÔNG THỨC LƯỢNG GIÁC}
\end{dang}

\begin{ex}%[0D6B3-1] 
Rút gọn $M=\sin \left({x-y}\right)\cos y+\cos \left({x-y}\right)\sin y.$
\choice
{$M=\cos x$}
{\True $M=\sin x$}
{$M=\sin x\cos 2y$}
{$M=\cos x\cos 2y$}
\loigiai{  Áp dụng công thức $\sin\left({a+b}\right)=\sin a\cos b+\sin b\cos a$, ta được\\
$M=\sin \left({x-y}\right)\cos y+\cos \left({x-y}\right)\sin y=\sin \left[{\left({x-y}\right)+y}\right]=\sin x.$
 } 
\end{ex}

\begin{ex}%[0D6B3-1] 
Rút gọn $M=\cos \left({a+b}\right)\cos \left({a-b}\right)-\sin \left({a+b}\right)\sin \left({a-b}\right).$
\choice
{$M=1-2\cos ^2a$}
{\True $M=1-2\sin ^2a$}
{$M=\cos 4a$}
{$M=\sin 4a$}
\loigiai{  Áp dụng công thức $\cos x\cos y-\sin x\sin y=\cos \left({x+y}\right)$, ta được\\
$M=\cos \left({a+b}\right)\cos \left({a-b}\right)-\sin \left({a+b}\right)\sin \left({a-b}\right)=\cos \left({a+b+a-b}\right)=\cos 2a=1-2\sin ^2a.$

 } 
\end{ex}

\begin{ex}%[0D6B3-1] 
Rút gọn $M=\cos \left({a+b}\right)\cos \left({a-b}\right)+\sin \left({a+b}\right) \sin\left({a-b}\right).$
\choice
{\True $M=1-2\sin ^2b$}
{$M=1+2\sin ^2b$}
{$M=\cos 4b$}
{$M=\sin 4b$}
\loigiai{  Áp dụng công thức $\cos x\cos y+\sin x\sin y=\cos \left({x-y}\right)$, ta được\\
$M=\cos \left({a+b}\right)\cos \left({a-b}\right)+\sin \left({a+b}\right)\sin \left({a-b}\right)$\\
$=\cos \left({a+b-(a-b)}\right)=\cos 2b=1-2\sin ^2b.$
 } 
\end{ex}

\begin{ex}%[0D6K3-1] 
Giá trị nào sau đây của $x$ thỏa mãn $\sin 2x\cdot \sin 3x=\cos 2x\cdot\cos 3x$?
\choice
{\True $18^\circ$}
{$30^\circ$}
{$36^\circ$}
{$45^\circ$}
\loigiai{ Áp dụng công thức $\cos a\cdot\cos b-\sin a\cdot\sin b=\cos \left({a+b}\right)$, ta được\\
$\sin 2x\cdot\sin 3x=\cos 2x\cdot\cos 3x\Leftrightarrow \cos 2x\cdot\cos 3x-\sin 2x\cdot\sin 3x=0$\\
$\Leftrightarrow \cos 5x=0\Leftrightarrow 5x=\dfrac{\pi}{2}+k\pi \Leftrightarrow x=\dfrac{\pi}{{10}}+k\dfrac{\pi}{5}.$
  } 
\end{ex}

\begin{ex}%[0D6B3-1] 
Đẳng thức nào sau đây đúng?
\choice
{$\cot a+\cot b=\dfrac{{\sin \left({b-a}\right)}}{{\sin a\cdot \sin b}}$}
{\True $\cos ^2a=\dfrac{1}{2}\left({1+\cos 2a}\right)$}
{$\sin \left({a+b}\right)=\dfrac{1}{2}\sin 2\left({a+b}\right)$}
{$\tan \left({a+b}\right)=\dfrac{{\sin \left({a+b}\right)}}{{\cos a\cdot\cos b}}$}
\loigiai{ Xét các đáp án sau
\begin{itemize}
    \item  Ta có $\cot a+\cot b=\dfrac{{\cos a}}{{\sin a}}+\dfrac{{\cos b}}{{\sin b}}=\dfrac{{\cos a\cdot\sin b+\sin a\cdot\cos b}}{{\sin a\cdot\sin b}}=\dfrac{{\sin \left({a+b}\right)}}{{\sin a\cdot\sin b}}$.
    \item Ta có $\cos 2a=2\cos ^2a-1\Leftrightarrow \cos ^2a=\dfrac{1}{2}\left({1+\cos 2a}\right)$.
\end{itemize}  

  } 
\end{ex}

\begin{ex}%[0D6B3-8] 
Chọn công thức đúng trong các công thức sau:
\choice
{$\sin a\cdot \sin b=-\dfrac{1}{2}\left[{\cos \left({a+b}\right)-\cos \left({a-b}\right)}\right]$}
{\True $\sin a-\sin b=2\sin \dfrac{{a+b}}{2}\cdot \cos\dfrac{{a-b}}{2}$}
{$\tan 2a=\dfrac{{2\tan a}}{{1-\tan a}}$}
{$\cos 2a=\sin ^2a-\cos ^2a$}
\loigiai{ 
  } 
\end{ex}

\begin{ex}%[0D62A3-3]
Rút gọn $M=\cos \left({x+\dfrac{\pi}{4}}\right)-\cos \left({x-\dfrac{\pi}{4}}\right).$
\choice
{$M=\sqrt{2}\sin x$}
{\True $M=-\sqrt{2}\sin x$}
{$M=\sqrt{2}\cos x$}
{$M=-\sqrt{2}\cos x$}
\loigiai{ Áp dụng công thức $\cos a-\cos b=-2\sin \dfrac{{a+b}}{2}\cdot \sin \dfrac{{a-b}}{2}$, ta được\\
$M=\cos \left({x+\dfrac{\pi}{4}}\right)-\cos \left({x-\dfrac{\pi}{4}}\right)=-2\sin \left({\dfrac{{x+\dfrac{\pi}{4}+x-\dfrac{\pi}{4}}}{2}}\right)\cdot \sin \left({\dfrac{{x+\dfrac{\pi}{4}-x+\dfrac{\pi}{4}}}{2}}\right)$\\
$=-2\sin x\cdot \sin \dfrac{\pi}{4}=-\sqrt{2}\sin x.$
  } 
\end{ex}

\begin{ex}%[0D6B3-4]
Tam giác $ABC$ có $\cos A=\dfrac{4}{5}$ và $\cos B=\dfrac{5}{{13}}$. Khi đó $\cos C$ bằng
\choice
{$\dfrac{{56}}{{65}}$}
{$-\dfrac{{56}}{{65}}$}
{\True $\dfrac{{16}}{{65}}$}
{$\dfrac{{33}}{{65}}$}
\loigiai{ Ta có $\heva{& \cos A=\dfrac{4}{5} \\
& \cos B=\dfrac{5}{13}}
\Rightarrow 
\heva{& \sin A=\dfrac{3}{5} \\
& \sin B=\dfrac{12}{13}.}$\\
Mà $A+B+C=180^\circ$, do đó
$\cos C=\cos \left[ 180^\circ - \left( A+B \right) \right] =-\cos \left(A+B\right)$\\
$=-\left(\cos A \cdot \cos B-\sin A \cdot \sin B\right) =-\left(\dfrac{4}{5} \cdot \dfrac{5}{13}-\dfrac{3}{5} \cdot \dfrac{12}{13}\right) =\dfrac{16}{65}.$

 } 
\end{ex}

\begin{ex}%[0D6B3-4]
Cho $A,B,C$ là ba góc nhọn thỏa mãn $\tan A=\dfrac{1}{2},\tan B=\dfrac{1}{5},\tan C=\dfrac{1}{8}$. Tổng $A+B+C$ bằng
\choice
{$\dfrac{\pi}{6}$}
{$\dfrac{\pi}{5}$}
{\True $\dfrac{\pi}{4}$}
{$\dfrac{\pi}{3}$}
\loigiai{  Ta có $\tan \left({A+B}\right)=\dfrac{{\tan A+\tan B}}{{1-\tan A.\tan B}}=\dfrac{{\dfrac{1}{2}+\dfrac{1}{5}}}{{1-\dfrac{1}{2}\cdot\dfrac{1}{5}}}=\dfrac{7}{9}$\\
$\Rightarrow  \tan \left({A+B+C}\right)=\dfrac{{\tan \left({A+B}\right)+\tan C}}{{1-\tan \left({A+B}\right)\cdot \tan C}}=\dfrac{{\dfrac{7}{9}+\dfrac{1}{8}}}{{1-\dfrac{7}{9}\cdot\dfrac{1}{8}}}=1\Rightarrow  A+B+C=\dfrac{\pi}{4}$.
 } 
\end{ex}

\begin{ex}%[0D6K3-4]
Cho $A,B,C$ là các góc của tam giác $ABC$. Khi đó $P=\sin A+\sin B+\sin C$ tương đương với
\choice
{\True $P=4\cos \dfrac{A}{2}\cos \dfrac{B}{2}\cos \dfrac{C}{2}$}
{$P=4\sin \dfrac{A}{2}\sin \dfrac{B}{2}\sin \dfrac{C}{2}$}
{$P=2\cos \dfrac{A}{2}\cos \dfrac{B}{2}\cos \dfrac{C}{2}$}
{$P=2\cos \dfrac{A}{2}\cos \dfrac{B}{2}\cos \dfrac{C}{2}$}
\loigiai{   Do $\heva{& \dfrac{{A+B}}{2}=\dfrac{\pi}{2}-\dfrac{C}{2} \\
& \dfrac{C}{2}=\dfrac{\pi}{2}-\dfrac{{A+B}}{2}}\Rightarrow  \heva{& \sin \dfrac{{A+B}}{2}=\cos \dfrac{C}{2} \\
& \sin \dfrac{C}{2}=\cos \dfrac{{A+B}}{2}.}$\\
Áp dụng, ta được\\
$P=\left({\sin A+\sin B}\right)+\sin C=2\sin \dfrac{{A+B}}{2}\cos \dfrac{{A-B}}{2}+2\sin \dfrac{C}{2}\cos \dfrac{C}{2}$\\
$=2\cos \dfrac{C}{2}\cos \dfrac{{A-B}}{2}+2\cos \dfrac{{A+B}}{2}\cos \dfrac{C}{2}$\\
$=2\cos \dfrac{C}{2}\left({\cos \dfrac{{A-B}}{2}+\cos \dfrac{{A+B}}{2}}\right)=4\cos \dfrac{C}{2}\cos \dfrac{A}{2}\cos \dfrac{B}{2}.$ } 
\end{ex}

\begin{ex}%[0D6K3-4]
Cho $A,B,C$ là các góc của tam giác $ABC$. Khi đó $P=\sin 2A+\sin 2B+\sin 2C$ tương đương với:
\choice
{$P=4\cos A\cdot\cos B\cdot\cos C$}
{\True $P=4\sin A\cdot \sin B\cdot \sin C$}
{$P=-4\cos A\cdot\cos B\cdot\cos C$}
{$P=-4\sin A\cdot \sin B\cdot \sin C$}
\loigiai{Do $A+B=\pi-C\Rightarrow\sin \left({A+B}\right)=\sin C.$
Áp dụng, ta được \\
$P=\left({\sin 2A+\sin 2B}\right)+\sin 2C=2\sin \left({A+B}\right)\cdot\cos \left({A-B}\right)+2\sin C\cdot\cos C$ \\
$=2\sin C\cdot\cos \left({A-B}\right)+2\sin C\cdot\cos C=2\sin C\left[{\cos \left({A-B}\right)+\cos C}\right]$\\
$=4\sin C\cdot\cos \dfrac{A-B+C}{2}\cdot\cos \dfrac{A-B-C}{2}$\\
$=4\sin C\cdot \cos \dfrac{ \left( A+B+C \right)-2B }{2} \cdot \cos \dfrac{ \left( -A-B-C \right) +2A }{2}$
$=4\sin C\cdot \cos \left({\dfrac{\pi}{2}-B}\right)\cdot\cos \left(-\dfrac{\pi}{2}+A\right)\\
=4\sin C\cdot\sin B\cdot\sin A = 4\sin A\cdot\sin B\cdot \sin C.$
 } 
\end{ex}

\begin{ex}%[0D6K3-4]
Cho $A,B,C$ là các góc của tam giác $ABC$ (không phải tam giác vuông). Khi đó $P=\tan A+\tan B+\tan C$ tương đương với 
\choice
{$P=\tan \dfrac{A}{2}\cdot \tan \dfrac{B}{2}\cdot \tan \dfrac{C}{2}$}
{$P=-\tan \dfrac{A}{2}\cdot\tan \dfrac{B}{2}\cdot\tan \dfrac{C}{2}$}
{$P=-\tan \cdot\tan B\cdot\tan C$}
{\True $P=\tan A\cdot\tan B\cdot \tan C$}
\loigiai{ Ta có $P=\tan A+\tan B+\tan C=\left({\tan A+\tan B}\right)+\tan C=\dfrac{{\sin \left({A+B}\right)}}{{\cos A.\cos B}}+\dfrac{{\sin C}}{{\cos C}}$.\\
Mà $A+B=\pi-C\Rightarrow\heva{& \sin \left({A+B}\right)=\sin C \\
&-\cos \left({A+B}\right)=\cos C.}$\\
Khi đó, ta được\\
$P=\dfrac{{\sin C}}{{\cos A\cdot\cos B}}+\dfrac{{\sin C}}{{\cos C}}=\sin C\left({\dfrac{{\cos C+\cos A\cdot\cos B}}{{\cos A\cdot\cos B\cdot\cos C}}}\right)=\sin C\cdot \left({\dfrac{{-\cos \left({A+B}\right)+\cos A\cdot\cos B}}{{\cos A\cdot\cos B\cdot\cos C}}}\right)$\\
$=\sin C\cdot\dfrac{{-\cos A\cdot\cos B+\sin A\cdot\sin B+\cos A\cdot\cos B}}{{\cos A\cdot\cos B\cdot\cos C}}=\dfrac{{\sin A\cdot\sin B\cdot\sin C}}{{\cos A\cdot\cos B\cdot\cos C}}=\tan A\cdot\tan B\cdot\tan C.$

  }
 \end{ex}

\begin{ex}%[0D6K3-4]
Cho $A,B,C$ là các góc của tam giác $ABC$. 
Khi đó $P=\tan \dfrac{A}{2}\cdot\tan \dfrac{B}{2}+\tan \dfrac{B}{2}\cdot\tan \dfrac{C}{2}+\tan \dfrac{C}{2}\cdot\tan \dfrac{A}{2}$ tương đương với
\choice
{\True $P=1$}
{$P=-1$}
{$P=\left({\tan \dfrac{A}{2}\cdot\tan \dfrac{B}{2}\cdot\tan \dfrac{C}{2}}\right)^2$}
{Đáp án khác}
\loigiai{ Do $A+B+C=\pi \Rightarrow \dfrac{{C+B}}{2}=\dfrac{\pi}{2}-\dfrac{A}{2}$
$\Rightarrow \tan \left({\dfrac{{C+B}}{2}}\right)=\tan \left({\dfrac{\pi}{2}-\dfrac{A}{2}}\right).\\
\Rightarrow\dfrac{{\tan \dfrac{C}{2}+\tan \dfrac{B}{2}}}{{1-\tan \dfrac{C}{2}\tan \dfrac{B}{2}}}=\cot \dfrac{A}{2}=\dfrac{1}{{\tan \dfrac{A}{2}}}$\\
$\Rightarrow \tan\dfrac{A}{2}\left({\tan \dfrac{C}{2}+\tan \dfrac{B}{2}}\right)+\tan \dfrac{C}{2}\cdot \tan\dfrac{B}{2}=1$\\
$\Rightarrow\tan \dfrac{A}{2}\cdot \tan \dfrac{B}{2}+\tan \dfrac{B}{2}\cdot\tan \dfrac{C}{2}+\tan \dfrac{C}{2}\cdot\tan \dfrac{A}{2}=1$.
  } 
\end{ex}

\begin{ex}%[0D6K3-6]
Trong $\Delta ABC$, nếu $\dfrac{{\sin B}}{{\sin C}}=2\cos A$ thì $\Delta ABC$ là tam giác có tính chất nào sau đây?
\choice
{\True Cân tại $B$}
{Cân tại $A$}
{Cân tại $C$}
{Vuông tại $B$}
\loigiai{ Ta có $\dfrac{{\sin B}}{{\sin C}}=2\cos A\Rightarrow \sin B=2\sin C\cdot\cos A=\sin \left({C+A}\right)+\sin \left({C-A}\right)$.\\
Mặt khác $A+B+C=\pi \Rightarrow B=\pi-\left( {A+C} \right)\Rightarrow \sin B=\sin \left({A+C}\right)$. \\
Do đó, ta được
$\sin \left({C-A}\right)=0\Rightarrow A=C$.
 } 
\end{ex}

\begin{ex}%[0D6K3-6]
Trong $\Delta ABC$, nếu $\dfrac{{\tan A}}{{\tan C}}=\dfrac{{{\sin}^2A}}{{{\sin}^2C}}$ thì $\Delta ABC$ là tam giác gì?
\choice
{Tam giác vuông}
{Tam giác cân}
{Tam giác đều}
{\True Tam giác vuông hoặc cân}
\loigiai{ Ta có $\dfrac{{\tan A}}{{\tan C}}=\dfrac{{{\sin}^2A}}{{{\sin}^2C}} \Leftrightarrow \dfrac{{\sin A\cos C}}{{\cos A\sin C}}=\dfrac{{{\sin}^2A}}{{{\sin}^2C}}\Leftrightarrow \sin 2C=\sin 2A$ 
$\Rightarrow \hoac{ &{2C=2A} \\ &{2C=\pi-2A} }\Rightarrow \hoac{ &{C=A} \\ &{A+C=\dfrac{\pi}{2}}.}$
  } 
\end{ex}

\begin{dang}
   { TÍNH BIỂU THỨC LƯỢNG GIÁC}
\end{dang}

\begin{ex}%[0D6B3-4] 
Cho góc $\alpha $ thỏa mãn $\dfrac{\pi}{2}<\alpha <\pi $ và $\sin \alpha =\dfrac{4}{5}$. Tính $P=\sin 2\left({\alpha+\pi}\right).$
\choice
{\True $P=-\dfrac{{24}}{{25}}$}
{$P=\dfrac{{24}}{{25}}$}
{$P=-\dfrac{{12}}{{25}}$}
{$P=\dfrac{{12}}{{25}}$}
\loigiai{  Ta có $P=\sin 2\left({\alpha+\pi}\right)=\sin \left({2\alpha+2\pi}\right)=\sin 2\alpha =2\sin \alpha \cos \alpha $.\\
Từ hệ thức $\sin ^2\alpha+\cos ^2\alpha =1$, suy ra $\cos \alpha =\pm \sqrt{{1-{\sin}^2\alpha}}=\pm \dfrac{3}{5}$.\\
Do $\dfrac{\pi}{2}<\alpha <\pi $ nên ta chọn $\cos \alpha =-\dfrac{3}{5}$.\\
Thay $\sin \alpha =\dfrac{4}{5}$ và $\cos \alpha =-\dfrac{3}{5}$ vào $P$, ta được $P=2\cdot\dfrac{4}{5}\cdot\left({-\dfrac{3}{5}}\right)=-\dfrac{{24}}{{25}}$.
 } 
\end{ex}

\begin{ex}%[0D6B3-4] 
Cho góc $\alpha $ thỏa mãn $0<\alpha <\dfrac{\pi}{2}$ và $\sin \alpha =\dfrac{2}{3}$. Tính $P=\dfrac{{1+\sin 2\alpha+\cos 2\alpha}}{{\sin \alpha+\cos \alpha}}$.
\choice
{$P=-\dfrac{{2\sqrt{5}}}{3}$}
{$P=\dfrac{3}{2}$}
{$P=-\dfrac{3}{2}$}
{\True $P=\dfrac{{2\sqrt{5}}}{3}$}
\loigiai{  Ta có $P=\dfrac{{2\sin \alpha \cos \alpha+2{\cos}^2\alpha}}{{\sin \alpha+\cos \alpha}}=\dfrac{{2\cos \alpha \left({\sin \alpha+\cos \alpha}\right)}}{{\sin \alpha+\cos \alpha}}=2\cos \alpha $.\\
Từ hệ thức $\sin ^2\alpha+\cos ^2\alpha =1$, suy ra $\cos \alpha =\pm \sqrt{{1-{\sin}^2\alpha}}=\pm \dfrac{{\sqrt{5}}}{3}$.\\
Do $0<\alpha <\dfrac{\pi}{2}$ nên ta chọn $\cos \alpha =\dfrac{{\sqrt{5}}}{3}\Rightarrow P=\dfrac{{2\sqrt{5}}}{3}.$
 } 
\end{ex}

\begin{ex}%[0D6B3-4] 
Biết $\sin \left({\pi-\alpha}\right)=-\dfrac{3}{5}$ và $\pi <\alpha <\dfrac{{3\pi}}{2}$. Tính $P=\sin \left({\alpha+\dfrac{\pi}{6}}\right).$
\choice
{$P=-\dfrac{3}{5}$}
{$P=\dfrac{3}{5}$}
{\True $P=\dfrac{{-4-3\sqrt{3}}}{{10}}$}
{$P=\dfrac{{4-3\sqrt{3}}}{{10}}$}
\loigiai{  Ta có $-\dfrac{3}{5}=\sin \left({\pi-\alpha}\right)=\sin \alpha $.
Từ hệ thức $\sin ^2\alpha+\cos ^2\alpha =1$, suy ra $\cos \alpha =\pm \sqrt{{1-{\sin}^2\alpha}}=\pm \dfrac{4}{5}$.
Do $\pi <\alpha <\dfrac{{3\pi}}{2}$ nên ta chọn $\cos \alpha =-\dfrac{4}{5}$.\\
Suy ra $P=\sin \left({\alpha+\dfrac{\pi}{6}}\right)=\dfrac{{\sqrt{3}}}{2}\sin \alpha+\dfrac{1}{2}\cos \alpha =\dfrac{{\sqrt{3}}}{2}\left({-\dfrac{3}{5}}\right)+\dfrac{1}{2}\left({-\dfrac{4}{5}}\right)=\dfrac{{-4-3\sqrt{3}}}{{10}}$.
 } 
\end{ex}

\begin{ex}%[0D6B3-4] 
Cho góc $\alpha $ thỏa mãn $\sin \alpha =\dfrac{3}{5}.$ Tính $P=\sin \left({\alpha+\dfrac{\pi}{6}}\right)\sin \left({\alpha-\dfrac{\pi}{6}}\right).$
\choice
{\True $P=\dfrac{{11}}{{100}}$}
{$P=-\dfrac{{11}}{{100}}$}
{$P=\dfrac{7}{{25}}$}
{$P=\dfrac{{10}}{{11}}$}
\loigiai{ Áp dụng công thức $\sin a.\sin b=\dfrac{1}{2}\left[{\cos \left({a-b}\right)-\cos \left({a+b}\right)}\right]$, ta được\\
$P=\sin \left({\alpha+\dfrac{\pi}{6}}\right)\sin \left({\alpha-\dfrac{\pi}{6}}\right)=\dfrac{1}{2}\left({\cos \dfrac{\pi}{3}-\cos 2\alpha}\right).$\\
Ta có $\cos 2\alpha =1-2\sin ^2\alpha =1-2\cdot \left({\dfrac{3}{5}}\right)^2=\dfrac{7}{{25}}.$
Thay vào $P$, ta được $P=\dfrac{1}{2}\left({\dfrac{1}{2}-\dfrac{7}{{25}}}\right)=\dfrac{{11}}{{100}}.$
  } 
\end{ex}

\begin{ex}%[0D6B3-4] 
Cho góc $\alpha $ thỏa mãn $\sin \alpha =\dfrac{4}{5}.$ Tính $P=\cos 4\alpha.$
\choice
{$P=\dfrac{{527}}{{625}}$}
{\True $P=-\dfrac{{527}}{{625}}$}
{$P=\dfrac{{524}}{{625}}$}
{$P=-\dfrac{{524}}{{625}}$}
\loigiai{  Ta có $\cos 2\alpha =1-2\sin^2\alpha =1-2.\left({\dfrac{4}{5}}\right)^2=-\dfrac{7}{{25}}.$\\
Suy ra $P=\cos 4\alpha =2\cos ^22\alpha-1=2.\dfrac{{49}}{{625}}-1=-\dfrac{{527}}{{625}}.$
 } 
\end{ex}

\begin{ex}%[0D6B3-4] 
Cho góc $\alpha $ thỏa mãn $\sin 2\alpha =-\dfrac{4}{5}$ và $\dfrac{{3\pi}}{4}<\alpha <\pi $. Tính $P=\sin \alpha-\cos \alpha $.
\choice
{\True $P=\dfrac{3}{{\sqrt{5}}}$}
{$P=-\dfrac{3}{{\sqrt{5}}}$}
{$P=\dfrac{{\sqrt{5}}}{3}$}
{$P=-\dfrac{{\sqrt{5}}}{3}$}
\loigiai{  Vì $\dfrac{{3\pi}}{4}<\alpha <\pi $ suy ra $\heva{& \sin \alpha >0 \\
& \cos \alpha <0}$ nên $\sin \alpha-\cos \alpha >0$.\\
Ta có $\left({\sin \alpha-\cos \alpha}\right)^2=1-\sin 2\alpha =1+\dfrac{4}{5}=\dfrac{9}{5}$.\\
 Suy ra $\sin \alpha-\cos \alpha =\pm \dfrac{3}{{\sqrt{5}}}$.
Do $\sin \alpha-\cos \alpha >0$ nên $\sin \alpha-\cos \alpha =\dfrac{3}{{\sqrt{5}}}$.\\
 Vậy $P=\dfrac{3}{{\sqrt{5}}}.$
 } 
\end{ex}

\begin{ex}%[0D6B3-4] 
Cho góc $\alpha $ thỏa mãn $\sin 2\alpha =\dfrac{2}{3}$. Tính $P=\sin ^4\alpha+\cos ^4\alpha $.
\choice
{$P=1$}
{$P=\dfrac{{17}}{{81}}$}
{\True $P=\dfrac{7}{9}$}
{$P=\dfrac{9}{7}$}
\loigiai{  Áp dụng $a^4+b^4=\left({a^2+b^2}\right)^2-2a^2b^2$.\\
Ta có $P=\sin ^4\alpha+\cos ^4\alpha =\left({{\sin}^2\alpha+{\cos}^2\alpha}\right)^2-2\sin ^2\alpha.\cos ^2\alpha =1-\dfrac{1}{2}\sin ^22\alpha =\dfrac{7}{9}$. 

 } 
\end{ex}

\begin{ex}%[0D6B3-4] 
Cho góc $\alpha $ thỏa mãn $\cos \alpha =\dfrac{5}{{13}}$ và $\dfrac{{3\pi}}{2}<\alpha <2\pi $. Tính $P=\tan 2\alpha $.
\choice
{$P=-\dfrac{{120}}{{119}}$}
{$P=-\dfrac{{119}}{{120}}$}
{\True $P=\dfrac{{120}}{{119}}$}
{$P=\dfrac{{119}}{{120}}$}
\loigiai{ Ta có $P=\tan 2\alpha =\dfrac{{\sin 2\alpha}}{{\cos 2\alpha}}=\dfrac{{2\sin \alpha\cdot \cos \alpha}}{{2{\cos}^2\alpha-1}}$.\\
Từ hệ thức $\sin ^2\alpha+\cos ^2\alpha =1$, suy ra $\sin \alpha =\pm \sqrt{{1-{\cos}^2\alpha}}=\pm \dfrac{{12}}{{13}}$.\\
Do $\dfrac{{3\pi}}{2}<\alpha <2\pi $ nên ta chọn $\sin \alpha =-\dfrac{{12}}{{13}}$.\\
Thay $\sin \alpha =-\dfrac{{12}}{{13}}$ và $\cos \alpha =\dfrac{5}{{13}}$ vào $P$, ta được $P=\dfrac{{120}}{{119}}$.
  } 
\end{ex}

\begin{ex}%[0D6B3-4] 
Cho góc $\alpha $ thỏa mãn $\cos 2\alpha =-\dfrac{2}{3}$. Tính $P=\left({1+3{\sin}^2\alpha}\right)\left({1-4{\cos}^2\alpha}\right)$.
\choice
{$P=12$}
{$P=\dfrac{{21}}{2}$}
{$P=6$}
{\True $P=21$}
\loigiai{  Ta có $P=\left({1+3\cdot\dfrac{{1-\cos 2\alpha}}{2}}\right)\left({1-4\cdot\dfrac{{1+\cos 2\alpha}}{2}}\right)=\left({\dfrac{5}{2}-\dfrac{3}{2}\cos 2\alpha}\right)\left({-1-2\cos 2\alpha}\right)$.\\
Thay $\cos 2\alpha =-\dfrac{2}{3}$ vào $P$, ta được $P=\left({\dfrac{5}{2}+1}\right)\left({-1+\dfrac{4}{3}}\right)=\dfrac{7}{6}$.
 } 
\end{ex}

\begin{ex}%[0D6B3-4] 
Cho góc $\alpha $ thỏa mãn $\cos \alpha =\dfrac{3}{4}$ và $\dfrac{{3\pi}}{2}<\alpha <2\pi $. Tính $P=\cos \left({\dfrac{\pi}{3}-\alpha}\right).$
\choice
{$P=\dfrac{{3+\sqrt{{21}}}}{8}$}
{\True $P=\dfrac{{3-\sqrt{{21}}}}{8}$}
{$P=\dfrac{{3\sqrt{3}+\sqrt{7}}}{8}$}
{$P=\dfrac{{3\sqrt{3}-\sqrt{7}}}{8}$}
\loigiai{ Ta có $P=\cos \left({\dfrac{\pi}{3}-\alpha}\right)=\cos \dfrac{\pi}{3}\cos \alpha+\sin \dfrac{\pi}{3}\sin \alpha =\dfrac{1}{2}\cos \alpha+\dfrac{{\sqrt{3}}}{2}\sin \alpha $.\\
Từ hệ thức $\sin ^2\alpha+\cos ^2\alpha =1$, suy ra $\sin \alpha =\pm \sqrt{{1-{\cos}^2\alpha}}=\pm \dfrac{{\sqrt{7}}}{4}$.\\
Do $\dfrac{{3\pi}}{2}<\alpha <2\pi $ nên ta chọn $\sin \alpha =-\dfrac{{\sqrt{7}}}{4}$.\\
Thay $\sin \alpha =-\dfrac{{\sqrt{7}}}{4}$ và $\cos \alpha =\dfrac{3}{4}$ vào $P$, ta được $P=\dfrac{1}{2}.\dfrac{3}{4}+\dfrac{{\sqrt{3}}}{2}.\left({-\dfrac{{\sqrt{7}}}{4}}\right)=\dfrac{{3-\sqrt{{21}}}}{8}$.

  } 
\end{ex}

\begin{ex}%[0D6B3-4] 
Cho góc $\alpha $ thỏa mãn $\cos \alpha =-\dfrac{4}{5}$ và $\pi <\alpha <\dfrac{{3\pi}}{2}$. Tính $P=\tan \left({\alpha-\dfrac{\pi}{4}}\right)$.
\choice
{\True $P=-\dfrac{1}{7}$}
{$P=\dfrac{1}{7}$}
{$P=-7$}
{$P=7$}
\loigiai{ Ta có $P=\tan \left({\alpha-\dfrac{\pi}{4}}\right)=\dfrac{{\tan \alpha-1}}{{1+\tan \alpha}}$.\\
Từ hệ thức $\sin ^2\alpha+\cos ^2\alpha =1$, suy ra $\sin \alpha =\pm \sqrt{{1-{\cos}^2\alpha}}=\pm \dfrac{3}{5}$.\\
Do $\pi <\alpha <\dfrac{{3\pi}}{2}$ nên ta chọn $\sin \alpha =-\dfrac{3}{5}$. Suy ra $\tan \alpha =\dfrac{{\sin \alpha}}{{\cos \alpha}}=\dfrac{3}{4}$.\\
Thay $\tan \alpha =\dfrac{3}{4}$ vào $P$, ta được $P=-\dfrac{1}{7}$.
  } 
\end{ex}

\begin{ex}%[0D6B3-4] 
Cho góc $\alpha $ thỏa mãn $\cos 2\alpha =-\dfrac{4}{5}$ và $\dfrac{\pi}{4}<\alpha <\dfrac{\pi}{2}$. Tính $P=\cos \left({2\alpha-\dfrac{\pi}{4}}\right)$.
\choice
{$P=\dfrac{{\sqrt{2}}}{{10}}$}
{\True $P=-\dfrac{{\sqrt{2}}}{{10}}$}
{$P=-\dfrac{1}{5}$}
{$P=\dfrac{1}{5}$}
\loigiai{ Ta có $P=\cos \left({2\alpha-\dfrac{\pi}{4}}\right)=\dfrac{{\sqrt{2}}}{2}\left({\cos 2\alpha+\sin 2\alpha}\right)$.\\
Từ hệ thức $\sin ^22\alpha+\cos ^22\alpha =1$, suy ra $\sin 2\alpha =\pm \sqrt{{1-{\cos}^22\alpha}}=\pm \dfrac{3}{5}$.\\
Do $\dfrac{\pi}{4}<\alpha <\dfrac{\pi}{2}\Leftrightarrow \dfrac{\pi}{2}<2\alpha <\pi $ nên ta chọn $\sin 2\alpha =\dfrac{3}{5}$.\\
Thay $\sin 2\alpha =\dfrac{3}{5}$ và $\cos 2\alpha =-\dfrac{4}{5}$ vào $P$, ta được $P=-\dfrac{{\sqrt{2}}}{{10}}$.
  } 
\end{ex}

\begin{ex}%[0D6B3-4] 
Cho góc $\alpha $ thỏa mãn $\cos \alpha =-\dfrac{4}{5}$ và $\pi <\alpha <\dfrac{{3\pi}}{2}$. Tính $P=\sin \dfrac{\alpha}{2}.\cos \dfrac{{3\alpha}}{2}$.
\choice
{$P=-\dfrac{{39}}{{50}}$}
{$P=\dfrac{{49}}{{50}}$}
{$P=-\dfrac{{49}}{{50}}$}
{\True $P=\dfrac{{39}}{{50}}$}
\loigiai{  Ta có $P=\sin \dfrac{\alpha}{2}.\cos \dfrac{{3\alpha}}{2}=\dfrac{1}{2}\left({\sin 2\alpha-\sin \alpha}\right)=\dfrac{1}{2}\sin \alpha \left({2\cos \alpha-1}\right)$.\\
Từ hệ thức $\sin ^2\alpha+\cos ^2\alpha =1$, suy ra $\sin \alpha =\pm \sqrt{{1-{\cos}^2\alpha}}=\pm \dfrac{3}{5}$.\\
Do $\pi <\alpha <\dfrac{{3\pi}}{2}$ nên ta chọn $\sin \alpha =-\dfrac{3}{5}$.\\
Thay $\sin \alpha =-\dfrac{3}{5}$ và $\cos \alpha =-\dfrac{4}{5}$ vào $P$, ta được $P=\dfrac{{39}}{{50}}.$
 } 
\end{ex}

\begin{ex}%[0D6B3-4] 
Cho góc $\alpha $ thỏa mãn $\cot \left({\dfrac{{5\pi}}{2}-\alpha}\right)=2$. Tính $P=\tan \left({\alpha+\dfrac{\pi}{4}}\right)$.
\choice
{$P=\dfrac{1}{2}$}
{$P=-\dfrac{1}{2}$}
{\True $P=3$}
{$P=4$}
\loigiai{  Ta có $P=\tan \left({\alpha+\dfrac{\pi}{4}}\right)=\dfrac{{\tan \alpha+\tan \dfrac{\pi}{4}}}{{1-\tan \alpha.\tan \dfrac{\pi}{4}}}=\dfrac{{\tan \alpha+1}}{{1-\tan \alpha}}$.\\
Từ giả thiết $\cot \left({\dfrac{{5\pi}}{2}-\alpha}\right)=2\Leftrightarrow \cot \left({2\pi+\dfrac{\pi}{2}-\alpha}\right)=2\Leftrightarrow \cot \left({\dfrac{\pi}{2}-\alpha}\right)=2\Leftrightarrow \tan \alpha =2$.\\
Thay $\tan \alpha =2$ vào $P$, ta được $P=-3.$
 } 
\end{ex}

\begin{ex}%[0D6B3-4] 
Cho góc $\alpha $ thỏa mãn $\cot \alpha =15.$ Tính $P=\sin 2\alpha.$
\choice
{$P=\dfrac{{11}}{{113}}$}
{$P=\dfrac{{13}}{{113}}$}
{\True $P=\dfrac{{15}}{{113}}$}
{$P=\dfrac{{17}}{{113}}$}
\loigiai{ Ta có $\cot \alpha =15\Leftrightarrow \dfrac{{\cos \alpha}}{{\sin \alpha}}=15\Leftrightarrow \cos \alpha =15\sin \alpha.$\\
Suy ra $P=\sin 2\alpha =2\sin \alpha\cdot\cos \alpha =30\sin ^2\alpha =\dfrac{{30}}{{\dfrac{1}{{{\sin}^2\alpha}}}}=\dfrac{{30}}{{1+{\cot}^2\alpha}}=\dfrac{{30}}{{1+{15}^2}}=\dfrac{{15}}{{113}}.$

  } 
\end{ex}

\begin{ex}%[0D6B3-4] 
Cho góc $\alpha $ thỏa mãn $\cot \alpha =-3\sqrt{2}$ và $\dfrac{\pi}{2}<\alpha <\pi.$ Tính $P=\tan \dfrac{\alpha}{2}+\cot \dfrac{\alpha}{2}.$
\choice
{$P=2\sqrt{{19}}$}
{$P=-2\sqrt{{19}}$}
{\True $P=\sqrt{{19}}$}
{$P=-\sqrt{{19}}$}
\loigiai{ Ta có $P=\tan \dfrac{\alpha}{2}+\cot \dfrac{\alpha}{2}=\dfrac{{\sin \dfrac{\alpha}{2}}}{{\cos \dfrac{\alpha}{2}}}+\dfrac{{\cos \dfrac{\alpha}{2}}}{{\sin \dfrac{\alpha}{2}}}=\dfrac{{{\sin}^2\dfrac{\alpha}{2}+{\cos}^2\dfrac{\alpha}{2}}}{{\sin \dfrac{\alpha}{2}\cos \dfrac{\alpha}{2}}}=\dfrac{2}{{\sin \alpha}}.$\\
Từ hệ thức $1+\cot ^2\alpha =\dfrac{1}{{{\sin}^2\alpha}}\Rightarrow\sin \alpha =\pm \dfrac{1}{{\sqrt{{19}}}}$.\\
Do $\dfrac{\pi}{2}<\alpha <\pi \Rightarrow \sin \alpha >0$ nên ta chọn $\sin \alpha =\dfrac{1}{{\sqrt{{19}}}}\Rightarrow P=2\sqrt{{19}}.$
  } 
\end{ex}

\begin{ex}%[0D6B3-4] 
Cho góc $\alpha $ thỏa mãn $\tan \alpha =-\dfrac{4}{3}$ và $\alpha \in \left({\dfrac{{3\pi}}{2};2\pi}\right]$. Tính $P=\sin \dfrac{\alpha}{2}+\cos \dfrac{\alpha}{2}$.
\choice
{$P=\sqrt{5}$}
{$P=-\sqrt{5}$}
{\True $P=-\dfrac{{\sqrt{5}}}{5}$}
{$P=\dfrac{{\sqrt{5}}}{5}$}
\loigiai{ Ta có $P^2=1+\sin \alpha.$ Với $\alpha \in \left({\dfrac{{3\pi}}{2};2\pi}\right]\Rightarrow \dfrac{\alpha}{2}\in \left({\dfrac{{3\pi}}{4};\pi}\right]$.
Khi đó $\heva{& 0\leqslant \sin \dfrac{\alpha}{2}<\dfrac{{\sqrt{2}}}{2} \\
&-1\leqslant \cos \dfrac{\alpha}{2}<-\dfrac{{\sqrt{2}}}{2}}$, suy ra $P=\sin \dfrac{\alpha}{2}+\cos \dfrac{\alpha}{2}<0$.\\
Từ hệ thức $\sin ^2\alpha+\cos ^2\alpha =1$, suy ra $\sin ^2\alpha =1-\cos ^2\alpha =1-\dfrac{1}{{1+{\tan}^2\alpha}}=\dfrac{{16}}{{25}}$.
Vì $\alpha \in \left({\dfrac{{3\pi}}{2};2\pi}\right]$ nên ta chọn $\sin \alpha =-\dfrac{4}{5}$.\\
Thay $\sin \alpha =-\dfrac{4}{5}$ vào $P^2$, ta được $P^2=\dfrac{1}{5}$. Suy ra $P=-\dfrac{{\sqrt{5}}}{5}$.
  } 
\end{ex}

\begin{ex}%[0D6B3-4] 
Cho góc $\alpha $ thỏa mãn $\tan \alpha =-2$. Tính $P=\dfrac{{\sin 2\alpha}}{{\cos 4\alpha+1}}$.
\choice
{$P=\dfrac{{10}}{9}$}
{$P=\dfrac{9}{{10}}$}
{\True $P=-\dfrac{{10}}{9}$}
{$P=-\dfrac{9}{{10}}$}
\loigiai{  Ta có $P=\dfrac{{\sin 2\alpha}}{{\cos 4\alpha+1}}=\dfrac{{\sin 2\alpha}}{{2{\cos}^22\alpha}}$.\\
Nhắc lại công thức: Nếu đặt $t=\tan \alpha $ thì $\sin 2\alpha =\dfrac{{2t}}{{1+t^2}}$ và $\cos 2\alpha =\dfrac{{1-t^2}}{{1+t^2}}$.\\
Do đó $\sin 2\alpha =\dfrac{{2\tan \alpha}}{{1+{\tan}^2\alpha}}=-\dfrac{4}{5}$, $\cos 2\alpha =\dfrac{{1-{\tan}^2\alpha}}{{1+{\tan}^2\alpha}}=-\dfrac{3}{5}$.\\
Thay $\sin 2\alpha =-\dfrac{4}{5}$ và $\cos 2\alpha =-\dfrac{3}{5}$ vào $P$, ta được $P=-\dfrac{{10}}{9}$.
 } 
\end{ex}

\begin{ex}%[0D6B3-4] 
Cho góc $\alpha $ thỏa mãn $\tan \alpha+\cot \alpha <0$ và $\sin \alpha =\dfrac{1}{5}$. Tính $P=\sin 2\alpha $.
\choice
{$P=\dfrac{{4\sqrt{6}}}{{25}}$}
{\True $P=-\dfrac{{4\sqrt{6}}}{{25}}$}
{$P=\dfrac{{2\sqrt{6}}}{{25}}$}
{$P=-\dfrac{{2\sqrt{6}}}{{25}}$}
\loigiai{  Ta có $A=\sin 2\alpha =2\sin \alpha \cos \alpha $.
Từ hệ thức $\cot ^2\alpha+1=\dfrac{1}{{{\sin}^2\alpha}}=25\Leftrightarrow \cot ^2\alpha =24\Rightarrow \cot \alpha =\pm 2\sqrt{6}$.
Vì $\tan \alpha $, $\cot \alpha $ cùng dấu và $\tan \alpha+\cot \alpha <0$ nên $\tan \alpha <0, \cot \alpha <0$.
Do đó ta chọn $\cot \alpha =-2\sqrt{6}$. Suy ra $\cos \alpha =\cot \alpha.\sin \alpha =-\dfrac{{2\sqrt{6}}}{5}$.\\
Thay $\sin \alpha =\dfrac{1}{5}$ và $\cos \alpha =-\dfrac{{2\sqrt{6}}}{5}$ vào $P$, ta được 
$P=2\cdot\dfrac{1}{5}\cdot \left({-\dfrac{{2\sqrt{6}}}{5}}\right)=-\dfrac{{4\sqrt{6}}}{{25}}.$
 } 
\end{ex}

\begin{ex}%[0D6B3-4] 
Cho góc $\alpha $ thỏa mãn $\dfrac{\pi}{2}<\alpha <\pi $ và $\sin \alpha+2\cos \alpha =-1$. Tính $P=\sin 2\alpha $.
\choice
{$P=\dfrac{{24}}{{25}}$}
{$P=\dfrac{{2\sqrt{6}}}{5}$}
{\True $P=-\dfrac{{24}}{{25}}$}
{$P=-\dfrac{{2\sqrt{6}}}{5}$}
\loigiai{  Với $\dfrac{\pi}{2}<\alpha <\pi $ suy ra $\heva{& \sin \alpha >0 \\
& \cos \alpha <0}$.\\
Ta có $\heva{& \sin \alpha+2\cos \alpha =-1 \\
& \sin ^2\alpha+\cos ^2\alpha =1}\Rightarrow \left({-1-2\cos \alpha}\right)^2+\cos ^2\alpha =1$\\
$\Leftrightarrow 5\cos ^2\alpha+4\cos \alpha =0\Leftrightarrow \hoac{& \cos \alpha =0 \left(\text{loại}\right) \\
& \cos \alpha =-\dfrac{4}{5}.}$\\
Từ hệ thức $\sin ^2\alpha+\cos ^2\alpha =1$, suy ra $\sin \alpha =\dfrac{3}{5}$ (do $\sin \alpha >0$).\\
Vậy $P=\sin 2\alpha =2\sin \alpha.\cos \alpha =2\cdot\dfrac{3}{5}\cdot \left({-\dfrac{4}{5}}\right)=-\dfrac{{24}}{{25}}$.
 } 
\end{ex}

\begin{ex}%[0D6B3-4] 
Biết $\sin a=\dfrac{5}{{13}};\cos b=\dfrac{3}{5};\dfrac{\pi}{2}<a<\pi;0<b<\dfrac{\pi}{2}.$ Hãy tính $\sin \left({a+b}\right).$
\choice
{$\dfrac{{56}}{{65}}$}
{$\dfrac{{63}}{{65}}$}
{\True $-\dfrac{{33}}{{65}}$}
{$0$}
\loigiai{ Ta có $\cos ^2a=1-\sin ^2a=1-\left({\dfrac{5}{{13}}}\right)^2=\dfrac{{144}}{{169}}$ mà $a\in \left({\dfrac{\pi}{2};\pi}\right)\Rightarrow \cos a=-\dfrac{{12}}{{13}}.$\\
Tương tự, ta có $\sin ^2b=1-\cos ^2b=1-\left({\dfrac{3}{5}}\right)^2=\dfrac{{16}}{{25}}$ mà $b\in \left({0;\dfrac{\pi}{2}}\right)\Rightarrow \sin b=\dfrac{4}{5}.$\\
Khi đó $\sin \left({a+b}\right)=\sin a\cdot\cos b+\sin b.\cos a=\dfrac{5}{{13}}\cdot \dfrac{3}{5}-\dfrac{{12}}{{13}}\cdot\dfrac{4}{5}=-\dfrac{{33}}{{65}}.$
  } 
\end{ex}

\begin{ex}%[0D6B3-4] 
Nếu biết rằng $\sin \alpha =\dfrac{5}{{13}}\left({\dfrac{\pi}{2}<\alpha <\pi}\right),\cos \beta =\dfrac{3}{5}\left({0<\beta <\dfrac{\pi}{2}}\right)$ thì giá trị đúng của biểu thức $\cos \left({\alpha-\beta}\right)$ là
\choice
{$\dfrac{{16}}{{65}}$}
{\True $-\dfrac{{16}}{{65}}$}
{$\dfrac{{18}}{{65}}$}
{$-\dfrac{{18}}{{65}}$}
\loigiai{ Ta có $\sin \alpha =\dfrac{5}{{13}}$ với $\dfrac{\pi}{2}<\alpha <\pi $ suy ra $\cos \alpha =-\sqrt{{1-\dfrac{{25}}{{169}}}}=-\dfrac{{12}}{{13}}.$\\
Tương tự, có $\cos \beta =\dfrac{3}{5}$ với $0<\beta <\dfrac{\pi}{2}$ suy ra $\sin \beta =\sqrt{{1-\dfrac{9}{{25}}}}=\dfrac{4}{5}.$\\
Vậy $\cos \left({\alpha-\beta}\right)=\cos \alpha.\cos \beta+\sin \alpha.\sin \beta =-\dfrac{{12}}{{13}}\cdot\dfrac{3}{5}+\dfrac{5}{{13}}\cdot\dfrac{4}{5}=-\dfrac{{16}}{{65}}.$
  } 
\end{ex}

\begin{ex}%[0D6B3-4] 
Cho hai góc nhọn $a;b$ và biết rằng $\cos a=\dfrac{1}{3};\cos b=\dfrac{1}{4}.$ Tính giá trị của biểu thức $P=\cos \left({a+b}\right)\cdot\cos \left({a-b}\right).$
\choice
{$-\dfrac{{113}}{{144}}$}
{$-\dfrac{{115}}{{144}}$}
{$-\dfrac{{117}}{{144}}$}
{\True $-\dfrac{{119}}{{144}}$}
\loigiai{  Ta có $P=\cos \left({a+b}\right)\cdot\cos \left({a-b}\right)=\left({\cos a\cdot\cos b+\sin a\cdot\sin b}\right)\left({\cos a\cdot\cos b-\sin a\cdot\sin b}\right)$\\
$=\left({\cos a\cdot\cos b}\right)^2-\left({\sin a\cdot\sin b}\right)^2=\cos ^2a\cdot\cos ^2b-\left({1-{\cos}^2a}\right)\cdot\left({1-{\cos}^2b}\right).$
$=\dfrac{1}{9}\cdot\dfrac{1}{{16}}-\left({1-\dfrac{1}{9}}\right)\cdot\left({1-\dfrac{1}{{16}}}\right)=-\dfrac{{119}}{{144}}.$
 }
 \end{ex}

\begin{ex}%[0D6B3-4] 
Nếu $a,b$ là hai góc nhọn và $\sin a=\dfrac{1}{3};\sin b=\dfrac{1}{2}$ thì $\cos 2\left({a+b}\right)$ có giá trị bằng
\choice
{$\dfrac{{7-2\sqrt{6}}}{{18}}$}
{$\dfrac{{7+2\sqrt{6}}}{{18}}$}
{$\dfrac{{7+4\sqrt{6}}}{{18}}$}
{\True $\dfrac{{7-4\sqrt{6}}}{{18}}$}
\loigiai{ Vì $a,b\in \left({0;\dfrac{\pi}{2}}\right)$ nên suy ra $\heva{& \cos a=\sqrt{{1-{\sin}^2a}}=\sqrt{{1-{\left({\dfrac{1}{3}}\right)}^2}}=\dfrac{{2\sqrt{2}}}{3} \\
& \cos b=\sqrt{{1-{\sin}^2b}}=\sqrt{{1-{\left({\dfrac{1}{2}}\right)}^2}}=\dfrac{{\sqrt{3}}}{2}.}$ \\
Khi đó $\cos \left({a+b}\right)=\cos a\cdot\cos b-\sin a\cdot\sin b=\dfrac{{2\sqrt{2}}}{3}\cdot\dfrac{{\sqrt{3}}}{2}-\dfrac{1}{3}\cdot\dfrac{1}{2}=\dfrac{{-1+2\sqrt{6}}}{6}.$\\
Vậy $\cos 2\left({a+b}\right)=2\cos ^2\left({a+b}\right)-1=2\cdot\left({\dfrac{{-1+2\sqrt{6}}}{6}}\right)^2-1=\dfrac{{7-4\sqrt{6}}}{{18}}.$
  }
 \end{ex}

\begin{ex}%[0D6B3-4] 
Cho $0<\alpha, \beta <\dfrac{\pi}{2}$ và thỏa mãn $\tan \alpha =\dfrac{1}{7}$, $\tan \beta =\dfrac{3}{4}$. Góc $\alpha+\beta $ có giá trị bằng
\choice
{$\dfrac{\pi}{3}$}
{\True $\dfrac{\pi}{4}$}
{$\dfrac{\pi}{6}$}
{$\dfrac{\pi}{2}$}
\loigiai{  Ta có $\tan \left({\alpha+\beta}\right)=\dfrac{{\tan \alpha+\tan \beta}}{{1-\tan \alpha\cdot\tan \beta}}=\dfrac{{\dfrac{1}{7}+\dfrac{3}{4}}}{{1-\dfrac{1}{7}\cdot\dfrac{3}{4}}}=1$ suy ra $a+b=\dfrac{\pi}{4}.$ } 
\end{ex}

\begin{ex}%[0D6B3-4] 
Cho $x, y$ là các góc nhọn và dương thỏa mãn $\cot x=\dfrac{3}{4},\cot y=\dfrac{1}{7}.$ Tổng $x+y$ bằng
\choice
{$\dfrac{\pi}{4}$}
{\True $\dfrac{{3\pi}}{4}$}
{$\dfrac{\pi}{3}$}
{$\pi$}
\loigiai{  Ta có $\cot \left({x+y}\right)=\dfrac{{\cot x\cdot\cot y-1}}{{\cot x+\cot y}}=\dfrac{{\dfrac{3}{4}\cdot\dfrac{1}{7}-1}}{{\dfrac{3}{4}+\dfrac{1}{7}}}=-1.$\\
Mặt khác $0<x,y<\dfrac{\pi}{2}$ suy ra $0<x+y<\pi.$ Do đó $x+y=\dfrac{{3\pi}}{4}.$
 } 
\end{ex}

\begin{ex}%[0D6B3-4] 
Nếu $\alpha,\beta,\gamma $ là ba góc nhọn thỏa mãn $\tan \left({\alpha+\beta}\right)\cdot\sin \gamma =\cos \gamma $ thì
\choice
{$\alpha+\beta+\gamma =\dfrac{\pi}{4}$}
{$\alpha+\beta+\gamma =\dfrac{\pi}{3}$}
{\True $\alpha+\beta+\gamma =\dfrac{\pi}{2}$}
{$\alpha+\beta+\gamma =\dfrac{{3\pi}}{4}$}
\loigiai{  Ta có $\tan \left({\alpha+\beta}\right)\cdot\sin \gamma =\cos \gamma \Rightarrow \sin \left({\alpha+\beta}\right)\cdot\sin \gamma =\cos \left({\alpha+\beta}\right)\cdot\cos \gamma.$\\
$\Rightarrow \cos \left({\alpha+\beta}\right)\cdot\cos \gamma-\sin \left({\alpha+\beta}\right)\cdot\sin \gamma =0\Rightarrow \cos \left({\alpha+\beta+\gamma}\right)=0.$\\
Vậy tổng ba góc $\alpha+\beta+\gamma =\dfrac{\pi}{2}$ (vì $\alpha,\beta,\gamma $ là ba góc nhọn).
 }
 \end{ex}

\begin{ex}%[0D6B3-4] 
Biết rằng $\tan a=\dfrac{1}{2}\left({0<a<{90}^{\circ}}\right)$ và $\tan b=-\dfrac{1}{3}\left({{90}^{\circ}<b<{180}^{\circ}}\right)$ thì biểu thức $\cos \left({2a-b}\right)$ có giá trị bằng
\choice
{\True $-\dfrac{{\sqrt{{10}}}}{{10}}$}
{$\dfrac{{\sqrt{{10}}}}{{10}}$}
{$-\dfrac{{\sqrt{5}}}{5}$}
{$\dfrac{{\sqrt{5}}}{5}$}
\loigiai{  Ta có $\cos 2a=\dfrac{{1-{\tan}^2a}}{{1+{\tan}^2a}}=\dfrac{{1-{\left({\dfrac{1}{2}}\right)}^2}}{{1+{\left({\dfrac{1}{2}}\right)}^2}}=\dfrac{3}{5}$ suy ra $\sin 2a=\sqrt{{1-{\cos}^22a}}=\dfrac{4}{5}.$\\
Lại có $1+\tan ^2b=\dfrac{1}{{{\cos}^2b}}\Rightarrow \cos b=-\dfrac{1}{{\sqrt{{1+{\tan}^2b}}}}=-\dfrac{3}{{\sqrt{{10}}}}$ vì $90^{\circ}<b<180^{\circ}$\\
Mặt khác $\sin b=\tan b\cdot\cos b=\left({-\dfrac{1}{3}}\right)\cdot\left({-\dfrac{3}{{\sqrt{{10}}}}}\right)=\dfrac{1}{{\sqrt{{10}}}}.$\\
Khi đó $\cos \left({2a-b}\right)=\cos 2a\cdot\cos b+\sin 2a\cdot\sin b=\dfrac{3}{5}\cdot\left({-\dfrac{3}{{\sqrt{{10}}}}}\right)+\dfrac{4}{5}\cdot\dfrac{1}{{\sqrt{{10}}}}=-\dfrac{1}{{\sqrt{{10}}}}.$
 } 
\end{ex}

\begin{ex}%[0D6B3-4] 
Nếu $\sin a-\cos a=\dfrac{1}{5}\left({{135}^{\circ}<a<{180}^{\circ}}\right)$ thì giá trị của biểu thức $\tan 2a$ bằng
\choice
{$-\dfrac{{20}}{7}$}
{$\dfrac{{20}}{7}$}
{\True $\dfrac{{24}}{7}$}
{$-\dfrac{{24}}{7}$}
\loigiai{  Ta có $\sin a-\cos a=\dfrac{1}{5}\Rightarrow \left({\sin a-\cos a}\right)^2=\dfrac{1}{{25}}\Leftrightarrow 1-\sin 2a=\dfrac{1}{{25}}\Leftrightarrow \sin 2a=\dfrac{{24}}{{25}}.$
Khi đó $\cos 2a=\sqrt{{1-{\sin}^22a}}=\sqrt{{1-{\left({\dfrac{{24}}{{25}}}\right)}^2}}=\dfrac{7}{{25}}$ vì $270^{\circ}<2a<360^{\circ}.$ \\
Vậy giá trị của biểu thức $\tan 2a=\dfrac{{\sin 2a}}{{\cos 2a}}=\dfrac{{24}}{7}.$
 } 
\end{ex}

\begin{ex}%[0D6B3-4] 
Nếu $\tan \left({a+b}\right)=7,\tan \left({a-b}\right)=4$ thì giá trị đúng của $\tan 2a$ là
\choice
{\True $-\dfrac{{11}}{{27}}$}
{$\dfrac{{11}}{{27}}$}
{$-\dfrac{{13}}{{27}}$}
{$\dfrac{{13}}{{27}}$}
\loigiai{  Ta có $\tan 2a=\tan \left[{\left({a+b}\right)+\left({a-b}\right)}\right]=\dfrac{{\tan \left({a+b}\right)+\tan \left({a-b}\right)}}{{1+\tan \left({a+b}\right)\cdot\tan \left({a-b}\right)}}=\dfrac{{7+4}}{{1-7\cdot4}}=-\dfrac{{11}}{{27}}.$
 }
 \end{ex}

\begin{ex}%[0D6B3-4] 
Nếu $\sin \alpha\cdot\cos \left({\alpha+\beta}\right)=\sin \beta $ với $\alpha+\beta \ne \dfrac{\pi}{2}+k\pi,\alpha \ne \dfrac{\pi}{2}+l\pi,\left({k,l\in \mathbb{Z}}\right)$ thì
\choice
{$\tan \left({\alpha+\beta}\right)=2\cot \alpha$}
{$\tan \left({\alpha+\beta}\right)=2\cot \beta$}
{$\tan \left({\alpha+\beta}\right)=2\tan \beta$}
{\True $\tan \left({\alpha+\beta}\right)=2\tan \alpha$}
\loigiai{  Ta có $\sin \alpha\cdot\cos \left({\alpha+\beta}\right)=\sin \beta =\sin \left[{\left({\alpha+\beta}\right)-\alpha}\right]$
$\Leftrightarrow \sin \alpha\cdot\cos \left({\alpha+\beta}\right)=\sin \left({\alpha+\beta}\right)\cdot\cos\alpha-cos\left({\alpha+\beta}\right)\cdot\sin \alpha$\\
$\Leftrightarrow 2\sin \alpha.\cos \left({\alpha+\beta}\right)=\sin \left({\alpha+\beta}\right)\cdot\cos \alpha \Rightarrow \dfrac{{\sin \left({\alpha+\beta}\right)}}{{\cos \left({\alpha+\beta}\right)}}=2\cdot\dfrac{{\sin \alpha}}{{\cos \alpha}}=2\tan \alpha.$
 } 
\end{ex}

\begin{ex}%[0D6B3-4] 
Nếu $\alpha+\beta+\gamma =\dfrac{\pi}{2}$ và $\cot \alpha+\cot \gamma =2\cot \beta $ thì $\cot \alpha\cdot\cot \gamma $ bằng
\choice
{$\sqrt{3}$}
{$-\sqrt{3}$}
{\True $3$}
{$-3$}
\loigiai{ Từ giả thiết, ta có $\alpha+\beta+\gamma =\dfrac{\pi}{2}\Rightarrow \beta =\dfrac{\pi}{2}-\left({\alpha+\gamma}\right).$\\
Suy ra $\cot \alpha+\cot \gamma =2\cot \beta =2\cot \left[{\dfrac{\pi}{2}-\left({\alpha+\gamma}\right)}\right]=2.\tan \left({\alpha+\gamma}\right)=2\dfrac{{\tan \alpha+\tan \gamma}}{{1-\tan \alpha\cdot\tan \gamma}}$\\
Mặt khác $\dfrac{{\tan \alpha+\tan \gamma}}{{1-\tan \alpha.\tan \gamma}}=\dfrac{{\dfrac{1}{{\cot \alpha}}+\dfrac{1}{{\cot \gamma}}}}{{1-\dfrac{1}{{\cot \alpha}}\cdot\dfrac{1}{{\cot \gamma}}}}=\dfrac{{\cot \alpha+\cot \gamma}}{{\cot \alpha\cdot\cot \gamma-1}}$ nên suy ra
$$\cot \alpha+\cot \gamma =2\cdot\dfrac{{\cot \alpha+\cot \gamma}}{{\cot \alpha\cdot\cot \gamma-1}}\Leftrightarrow \cot \alpha\cdot\cot \gamma-1=2\Leftrightarrow \cot \alpha\cdot\cot \gamma =3.$$
  } 
\end{ex}

\begin{ex}%[0D6K3-4] 
Nếu $\tan \alpha $ và $\tan \beta $ là hai nghiệm của phương trình $x^2+px+q=0 \left({q\ne 1}\right)$ thì $\tan \left({\alpha+\beta}\right)$ bằng
\choice
{\True $\dfrac{p}{{q-1}}$}
{$-\dfrac{p}{{q-1}}$}
{$\dfrac{{2p}}{{1-q}}$}
{$-\dfrac{{2p}}{{1-q}}$}
\loigiai{  Vì $\tan \alpha,\tan \beta $ là hai nghiệm của phương trình $x^2+px+q=0$ nên theo định lí Viet, ta có$\heva{& \tan \alpha+\tan \beta =-p \\
& \tan \alpha.\tan \beta =q}.$ Khi đó $\tan \left({\alpha+\beta}\right)=\dfrac{{\tan \alpha+\tan \beta}}{{1-\tan \alpha \tan \beta}}=\dfrac{p}{{q-1}}.$
 } 
\end{ex}

\begin{ex}%[0D6K3-4] 
Nếu $\tan \alpha $; $\tan \beta $ là hai nghiệm của phương trình $x^2-px+q=0 \left({p\cdot q\ne 0}\right)$. Và $\cot \alpha $; $\cot \beta $ là hai nghiệm của phương trình $x^2-rx+s=0$ thì tích $P=rs$ bằng
\choice
{$pq$}
{\True$\dfrac{p}{{q^2}}$}
{$\dfrac{1}{{pq}}$}
{$\dfrac{q}{{p^2}}$}
\loigiai{  Theo định lí Viet, ta có $\heva{& \tan \alpha+\tan \beta =p \\
& \tan \alpha\cdot\tan \beta =q}$ và $\heva{& \cot \alpha+\cot \beta =r \\
& \cot \alpha\cdot \cot \beta =s}.$\\
Khi đó $P=r\cdot s=\left({\cot \alpha+\cot \beta}\right)\cdot\cot \alpha\cdot\cot \beta =\left({\dfrac{1}{{\tan \alpha}}+\dfrac{1}{{\tan \beta}}}\right)\cdot\dfrac{1}{{\tan \alpha}}\cdot\dfrac{1}{{\tan \beta}}$
$=\dfrac{{\tan \alpha+\tan \beta}}{{{\left({\tan \alpha\cdot\tan \beta}\right)}^2}}=\dfrac{p}{{q^2}}.$ Vậy $P=r\cdot s=\dfrac{p}{{q^2}}.$
 } 
\end{ex}

\begin{ex}%[0D6G3-4] 
Nếu $\tan \alpha $ và $\tan \beta $ là hai nghiệm của phương trình $x^2-px+q=0 \left({q\ne 0}\right)$ thì giá trị biểu thức $P=\cos ^2\left({\alpha+\beta}\right)+p\sin \left({\alpha+\beta}\right)\cdot \cos \left({\alpha+\beta}\right)+q\sin ^2\left({\alpha+\beta}\right)$ bằng: 
\choice
{$p$}
{$q$}
{\True $1$}
{$\dfrac{p}{q}$}
\loigiai{  Vì $\tan \alpha,\tan \beta $ là hai nghiệm của phương trình $x^2-px+q=0$ nên theo định lí Viet, ta có\\ $\heva{& \tan \alpha+\tan \beta =p \\
& \tan \alpha\cdot \tan \beta =q}\Rightarrow \tan \left({\alpha+\beta}\right)=\dfrac{{\tan \alpha+\tan \beta}}{{1-\tan \alpha\cdot\tan \beta}}=\dfrac{p}{{1-q}}.$\\
Khi đó $P=\cos ^2\left({\alpha+\beta}\right)\cdot \left[{1+p\cdot \tan \left({\alpha+\beta}\right)+q\cdot{\tan}^2\left({\alpha+\beta}\right)}\right].$\\
$=\dfrac{{1+p.\tan \left({\alpha+\beta}\right)+q\cdot{\tan}^2\left({\alpha+\beta}\right)}}{{1+{\tan}^2\left({\alpha+\beta}\right)}}=\dfrac{{1+p\cdot\dfrac{p}{{1-q}}+q\cdot{\left({\dfrac{p}{{1-q}}}\right)}^2}}{{1+{\left({\dfrac{p}{{1-q}}}\right)}^2}}$\\
$=\dfrac{{{\left({1-q}\right)}^2+p^2\left({1-q}\right)+q\cdot p^2}}{{{\left({1-q}\right)}^2+p^2}}=\dfrac{{{\left({1-q}\right)}^2+p^2-p^2\cdot q+q\cdot p^2}}{{{\left({1-q}\right)}^2+p^2}}=1.$
 } 
\end{ex}

\begin{dang}
    { RÚT GỌN BIỂU THỨC}
\end{dang}

\begin{ex}%[0D6B3-4] 
Rút gọn biểu thức $M=\tan x-\tan y$.
\choice
{$M=\tan \left({x-y}\right)$}
{$M=\dfrac{{\sin \left({x+y}\right)}}{{\cos x\cdot \cos y}}$}
{\True $M=\dfrac{{\sin \left({x-y}\right)}}{{\cos x\cdot \cos y}}$}
{$M=\dfrac{{\tan x-\tan y}}{{1+\tan x\cdot\tan y}}$}
\loigiai{  Ta có$M=\tan x-\tan y=\dfrac{{\sin x}}{{\cos x}}-\dfrac{{\sin y}}{{\cos y}}=\dfrac{{\sin x\cos y-\cos x\sin y}}{{\cos x\cos y}}=\dfrac{{\sin \left({x-y}\right)}}{{\cos x\cos y}}.$

 } 
\end{ex}

\begin{ex}%[0D6B3-4] 
Rút gọn biểu thức $M=\cos ^2\left({\dfrac{\pi}{4}+\alpha}\right)-\cos ^2\left({\dfrac{\pi}{4}-\alpha}\right).$
\choice
{$M=\sin 2\alpha$}
{$M=\cos 2\alpha$}
{$M=-\cos 2\alpha$}
{\True $M=-\sin 2\alpha$}
\loigiai{  Vì hai góc $\left({\dfrac{\pi}{4}-\alpha}\right)$ và $\left({\dfrac{\pi}{4}+\alpha}\right)$ phụ nhau nên $\cos \left({\dfrac{\pi}{4}-\alpha}\right)=\sin \left({\dfrac{\pi}{4}+\alpha}\right).$\\
Suy ra $M=\cos ^2\left({\dfrac{\pi}{4}+\alpha}\right)-\cos ^2\left({\dfrac{\pi}{4}-\alpha}\right)=\cos ^2\left({\dfrac{\pi}{4}+\alpha}\right)-\sin ^2\left({\dfrac{\pi}{4}+\alpha}\right)$
$=\cos \left({\dfrac{\pi}{2}+2\alpha}\right)=-\sin 2\alpha.$
 } 
\end{ex}

\begin{ex}%[0D6B3-4] 
Chọn đẳng thức đúng
\choice
{\True $\cos ^2\left({\dfrac{\pi}{4}+\dfrac{a}{2}}\right)=\dfrac{{1-\sin a}}{2}$}
{$\cos ^2\left({\dfrac{\pi}{4}+\dfrac{a}{2}}\right)=\dfrac{{1+\sin a}}{2}$}
{$\cos ^2\left({\dfrac{\pi}{4}+\dfrac{a}{2}}\right)=\dfrac{{1-\cos a}}{2}$}
{$\cos ^2\left({\dfrac{\pi}{4}+\dfrac{a}{2}}\right)=\dfrac{{1+\cos a}}{2}$}
\loigiai{  $\cos ^2\left({\dfrac{\pi}{4}+\dfrac{a}{2}}\right)=\dfrac{{1+\cos \left({\dfrac{\pi}{2}+a}\right)}}{2}=\dfrac{{1+\sin \left({-a}\right)}}{2}=\dfrac{{1-\sin a}}{2}$. }

 \end{ex}

\begin{ex}%[0D6B3-4] 
Gọi $M=\dfrac{{\sin \left({y-x}\right)}}{{\sin x\cdot\sin y}}$ thì
\choice
{$M=\tan x-\tan y$}
{\True $M=\cot x-\cot y$}
{$M=\cot y-\cot x$}
{$M=\dfrac{1}{{\sin x}}-\dfrac{1}{{\sin y}}$}
\loigiai{  Ta có
$M=\dfrac{{\sin y\cdot\cos x-\cos y\cdot\sin x}}{{\sin x\cdot\sin y}}=\dfrac{{\sin y\cdot\cos x}}{{\sin x\cdot\sin y}}-\dfrac{{\cos y\cdot\sin x}}{{\sin x\cdot\sin y}}=\dfrac{{\cos x}}{{\sin x}}-\dfrac{{\cos y}}{{\sin y}}=\cot x-\cot y$.

 } 
\end{ex}

\begin{ex}%[0D6B3-4] 
Gọi $M=\cos x+\cos 2x+\cos 3x$ thì
\choice
{$M=2\cos 2x\left({\cos x+1}\right)$}
{$M=4\cos 2x\cdot\left({\dfrac{1}{2}+\cos x}\right)$}
{$M=\cos 2x\left({2\cos x-1}\right)$}
{\True $M=\cos 2x\left({2\cos x+1}\right)$}
\loigiai{ Ta có $M=\cos x+\cos 2x+\cos 3x=\left({\cos x+\cos 3x}\right)+\cos 2x$
$=2\cos 2x\cdot\cos x+\cos 2x=\cos 2x\left({2\cos x+1}\right)$.

  } 
\end{ex}

\begin{ex}%[0D6B3-4] 
Rút gọn biểu thức $M=\dfrac{{\sin 3x-\sin x}}{{2{\cos}^2x-1}}$. 
\choice
{$\tan 2x$}
{$\sin x$}
{$2\tan x$}
{\True $2\sin x$}
\loigiai{  Ta có $\dfrac{{\sin 3x-\sin x}}{{2{\cos}^2x-1}}=\dfrac{{2\cos 2x\sin x}}{{\cos 2x}}=2\sin x$. } 
\end{ex}

\begin{ex}%[0D6B3-4] 
Rút gọn biểu thức $A=\dfrac{{1+\cos x+\cos 2x+\cos 3x}}{{2{\cos}^2x+\cos x-1}}$.
\choice
{$\cos x$}
{$2\cos x-1$}
{\True $2\cos x$}
{$\cos x-1$}
\loigiai{ Ta có\\
$A=\dfrac{{\left({1+\cos 2x}\right)+\left({\cos x+\cos 3x}\right)}}{{\left({2{\cos}^2x-1}\right)+\cos x}}=\dfrac{{2{\cos}^2x+2\cos 2x\cos x}}{{\cos x+\cos 2x}}$
$=\dfrac{{2\cos x\left({\cos x+\cos 2x}\right)}}{{\cos x+\cos 2x}}=2\cos x.$
  } 
\end{ex}

\begin{ex}%[0D6B3-4] 
Rút gọn biểu thức $A=\dfrac{{\tan \alpha-\cot \alpha}}{{\tan \alpha+\cot \alpha}}+\cos 2\alpha $. 
\choice
{\True $0$}
{$2\cos ^2x$}
{$2$}
{$\cos 2x$}
\loigiai{ Ta có
$\dfrac{{\dfrac{{\sin \alpha}}{{\cos \alpha}}-\dfrac{{\cos \alpha}}{{\sin \alpha}}}}{{\dfrac{{\sin \alpha}}{{\cos \alpha}}+\dfrac{{\cos \alpha}}{{\sin \alpha}}}}=\dfrac{{\dfrac{{{\sin}^2\alpha-{\cos}^2\alpha}}{{\sin \alpha.\cos \alpha}}}}{{\dfrac{{{\sin}^2\alpha+{\cos}^2\alpha}}{{\sin \alpha\cdot\cos \alpha}}}}=\dfrac{{{\sin}^2\alpha-{\cos}^2\alpha}}{{{\sin}^2\alpha+{\cos}^2\alpha}}=\sin ^2\alpha-\cos ^2\alpha =-\cos 2\alpha $.\\
Do đó $A=-\cos 2\alpha+\cos 2\alpha =0.$
  }
 \end{ex}

\begin{ex}%[0D6B3-4] 
Rút gọn biểu thức $A=\dfrac{{1+\sin 4\alpha-cos4\alpha}}{{1+\sin 4\alpha+cos4\alpha}}$.
\choice
{$\sin 2\alpha $}
{$\cos 2\alpha $}
{\True $\tan 2\alpha $}
{$\cot 2\alpha $}
\loigiai{  Ta có
$A=\dfrac{{\left({1-\cos 4\alpha}\right)+\sin 4\alpha}}{{\left({1+\cos 4\alpha}\right)+\sin 4\alpha}}=\dfrac{{2{\sin}^22\alpha+2\sin 2\alpha \cos2\alpha}}{{2{\cos}^22\alpha+2\sin 2\alpha \cos2\alpha}}=\dfrac{{2\sin 2\alpha (\sin 2\alpha+\cos2\alpha)}}{{2\cos 2\alpha (\sin 2\alpha+\cos2\alpha)}}=\tan 2\alpha $.

 } 
\end{ex}

\begin{ex}%[0D6B3-4] 
Biểu thức $A=\dfrac{{3-4\cos 2\alpha+\cos4\alpha}}{{3+4\cos 2\alpha+\cos4\alpha}}$ có kết quả rút gọn bằng
\choice
{$-\tan ^4\alpha$}
{\True $\tan ^4\alpha$}
{$-\cot ^4\alpha$}
{$\cot ^4\alpha$}
\loigiai{  Ta có $\cos 2\alpha =1-2\sin ^2\alpha;\cos 4\alpha =2\cos ^22\alpha-1=2\left({1-2{\sin}^2\alpha}\right)^2-1$. \\
Do đó
$A=\dfrac{{3-4\left({1-2{\sin}^2\alpha}\right)+2{\left({1-2\sin^2\alpha}\right)}^2-1}}{{3+4\left({2{\cos}^2\alpha-1}\right)+2{\left({2\cos^2\alpha-1}\right)}^2-1}}=\dfrac{{8{\sin}^2a-8{\sin}^2\alpha+8{\sin}^4\alpha}}{{8{\cos}^2a-8{\cos}^2\alpha+8{\cos}^4\alpha}}=\tan ^4\alpha $.

 } 
\end{ex}

\begin{ex}%[0D6B3-4] 
Khi $\alpha =\dfrac{\pi}{6}$ thì biểu thức $A=\dfrac{{\sin^22\alpha+4\sin^4\alpha-4{\sin}^2\alpha.{\cos}^2\alpha}}{{4-{\sin}^22\alpha-4{\sin}^2\alpha}}$ có giá trị bằng
\choice
{$\dfrac{1}{3}$}
{$\dfrac{1}{6}$}
{\True $\dfrac{1}{9}$}
{$\dfrac{1}{{12}}$}
\loigiai{  Ta có 
$A=\dfrac{{\sin^22\alpha+4\sin^4\alpha-4{\sin}^2\alpha\cdot{\cos}^2\alpha}}{{4-{\sin}^22\alpha-4{\sin}^2\alpha}}=\dfrac{{4\sin^4\alpha}}{{4(1-{\sin}^2\alpha)-4{\sin}^2\alpha\cdot{\cos}^2\alpha}}$\\$=\dfrac{{\sin^4\alpha}}{{{\cos}^2\alpha(1-{\sin}^2\alpha)}}=\dfrac{{\sin^4\alpha}}{{{\cos}^4\alpha}}=\tan^4a.$\\
Do đó giá trị của biểu thức $A$ tại $\alpha =\dfrac{\pi}{6}$ là $\tan ^4\left({\dfrac{\pi}{6}}\right)=\left({\dfrac{1}{{\sqrt{3}}}}\right)^4=\dfrac{1}{9}$.
 }
 \end{ex}

\begin{ex}%[0D6B3-4] 
Rút gọn biểu thức $A=\dfrac{{\sin 2\alpha+\sin \alpha}}{{1+\cos 2\alpha+\cos \alpha}}$.
\choice
{\True $\tan \alpha$}
{$2\tan \alpha$}
{$\tan 2\alpha+\tan \alpha$}
{$\tan 2\alpha$}
\loigiai{  Ta có $A=\dfrac{{\sin 2\alpha+\sin \alpha}}{{1+\cos 2\alpha+ \cos\alpha}}=\dfrac{{\sin \alpha \left({2\cos\alpha+1}\right)}}{{2 \cos^2\alpha+ \cos\alpha}} =\dfrac{{\sin \alpha \left({2\cos\alpha+1}\right)}}{{\cos\alpha \left({2\cos\alpha+1}\right)}}=\tan \alpha $.
 } 
\end{ex}

\begin{ex}%[0D6B3-4] 
Rút gọn biểu thức $A=\dfrac{{1-\sin a-\cos 2a}}{{\sin 2a-\cos a}}$.
\choice
{$1$}
{\True $\tan \alpha$}
{$\dfrac{5}{2}$}
{$2\tan \alpha$}
\loigiai{ Ta có $A=\dfrac{{1-\sin a+2{\sin}^2a-1}}{{2\sin a.\cos a-\cos a}}=\dfrac{{\sin a\left({2\sin a-1}\right)}}{{\cos a\left({2\sin a-1}\right)}}=\dfrac{{\sin a}}{{\cos a}}=\tan a.$  } 
\end{ex}

\begin{ex}%[0D6B3-4] 
Rút gọn biểu thức $A=\dfrac{{\sin x+\sin \dfrac{x}{2}}}{{1+\cos x+\cos \dfrac{x}{2}}}$ được kết quả là
\choice
{\True $\tan \dfrac{x}{2}$}
{$\cot x$}
{$\tan ^2\left({\dfrac{\pi}{4}-x}\right)$}
{$\sin x$}
\loigiai{  Ta có $\sin x=\sin \left({2.\dfrac{x}{2}} \right)=2\sin \dfrac{x}{2}\cos \dfrac{x}{2},$\\$1+\cos x=1+\cos \left({2\cdot\dfrac{x}{2}}\right)=2\cos^2\dfrac{x}{2}$.\\
Do đó $A=\dfrac{{2\sin \dfrac{x}{2}\cos \dfrac{x}{2}+\sin \dfrac{x}{2}}}{{2{\cos}^2\dfrac{x}{2}+\cos \dfrac{x}{2}}}=\dfrac{{\sin \dfrac{x}{2}\left({2\cos \dfrac{x}{2}+1}\right)}}{{\cos \dfrac{x}{2}\left({2\cos \dfrac{x}{2}+1}\right)}}=\tan \dfrac{x}{2}$.
 } 
\end{ex}

\begin{ex}%[0D6B3-4] 
Rút gọn biểu thức $A=\sin \alpha\cdot\cos ^5\alpha-\sin ^5\alpha.\cdot\cos \alpha $. 
\choice
{$\dfrac{1}{2}\sin 2\alpha$}
{$-\dfrac{1}{2}\sin 4\alpha$}
{$\dfrac{3}{4}\sin 4\alpha$}
{\True $\dfrac{1}{4}\sin 4\alpha.$ }
\loigiai{  Ta có
	\begin{eqnarray*}
	A & = & \sin \alpha\cdot\cos ^5\alpha-\sin ^5\alpha\cdot\cos \alpha \\
	& = & \sin \alpha\cdot\cos \alpha \left({{\cos}^4\alpha-{\sin}^4\alpha}\right)\\
	& = & \dfrac{1}{2}\sin 2\alpha \left({{\cos}^2\alpha-{\sin}^2\alpha}\right)\left({{\cos}^2\alpha+{\sin}^2\alpha}\right)\\
	& = & \dfrac{1}{2}\sin 2\alpha \left({{\cos}^2\alpha-{\sin}^2\alpha}\right) \\
	& = & \dfrac{1}{2}\sin 2\alpha \cos 2\alpha \\
	& = & \dfrac{1}{4}\sin 4\alpha.
	\end{eqnarray*}


 } 
\end{ex}

\begin{dang}
    { TÌM GIÁ TRỊ LỚN NHẤT-GIÁ TRỊ NHỎ NHẤT}
\end{dang}

\begin{ex}%[0D6B3-5]
Tìm giá trị lớn nhất $M$ và nhỏ nhất $m$ của biểu thức $P=3\sin x-2.$
\choice
{\True $M=1, m=-5$}
{$M=3, m=1$}
{$M=2, m=-2$}
{$M=0, m=-2$}
\loigiai{ 
Ta có $-1 \le \sin x \le 1 \Rightarrow -3 \le 3 \sin x \le 3 \Rightarrow -5 \le 3 \sin x -2 \le 1 \Rightarrow -5 \le P \le 1 \Rightarrow \heva{& M=1\\ & m =-5.}$
  } 
\end{ex}

\begin{ex}%[0D6B3-5]
Cho biểu thức $P=-2\sin \left({x+\dfrac{\pi}{3}}\right)+2$. Mệnh đề nào sau đây là đúng?
\choice
{$P\geqslant-4, \forall x\in \mathbb{R}$}
{$P\geqslant 4, \forall x\in \mathbb{R}$}
{\True $P\geqslant 0, \forall x\in \mathbb{R}$}
{$P\geqslant 2, \forall x\in \mathbb{R}$}
\loigiai{ Ta có $-1\leqslant \sin \left({x+\dfrac{\pi}{3}}\right)\leqslant 1 \Rightarrow 2\geqslant-2\sin \left({x+\dfrac{\pi}{3}}\right)\geqslant-2$
$\Rightarrow 4\geqslant-2\sin \left({x+\dfrac{\pi}{3}}\right)+2\geqslant 0$.\\
Vậy $ 4 \geqslant P\geqslant 0$.
  } 
\end{ex}

\begin{ex}%[0D6K3-5]
Biểu thức $P=\sin \left({x+\dfrac{\pi}{3}}\right)-\sin x$ có tất cả bao nhiêu giá trị nguyên?
\choice
{$1$}
{$2$}
{\True $3$}
{$4$}
\loigiai{ Áp dụng công thức $\sin a-\sin b=2\cos \dfrac{{a+b}}{2}\sin \dfrac{{a-b}}{2}$.\\
Ta có
$\sin \left({x+\dfrac{\pi}{3}}\right)-\sin x=2\cos \left({x+\dfrac{\pi}{6}}\right)\sin \dfrac{\pi}{6}=\cos \left({x+\dfrac{\pi}{6}}\right).$\\
Ta có $-1\leqslant \cos \left({x+\dfrac{\pi}{6}}\right)\leqslant 1 \Rightarrow -1 \leqslant P\leqslant 1\xrightarrow{{P\in \mathbb{Z}}}P\in \left\{{-1;0;1}\right\}.$
  } 
\end{ex}

\begin{ex}%[0D6B3-5]
Tìm giá trị lớn nhất $M$ và nhỏ nhất $m$ của biểu thức $P=\sin ^2x+2\cos ^2x.$
\choice
{$M=3, m=0$}
{$M=2, m=0$}
{\True $M=2, m=1$}
{$M=3, m=1$}
\loigiai{ Ta có $P = \sin^2 x + 2 \cos^2 x = \left( \sin^2 x + \cos^2 x \right) + \cos^2 x = 1 + \cos^2 x$. \\
Do $-1 \le \cos x \le 1 \Rightarrow 0 \le \cos^2 x \le 1 \Rightarrow 1 \le 1 + \cos^2 x \le 2 \Rightarrow \heva{& M = 2\\ & m=1.}$
 } 
\end{ex}

\begin{ex}%[0D6K3-5]
Gọi $M, m$ lần lượt là giá trị lớn nhất và giá trị nhỏ nhất của biểu thức $P=8\sin ^2x+3\cos 2x$. Tính $T=2M-m^2.$
\choice
{\True $T=1$}
{$T=2$}
{$T=112$}
{$T=130$}
\loigiai{  Ta có $P = 8 \sin^2 x + 3 \cos 2x = 8 \sin^2 x + 3 \left( 1 - 2 \sin^2 x \right) = 2 \sin^2 x + 3.$\\
Mà $-1 \le \sin x \le 1 \Rightarrow 0 \le \sin^2 x \le 1 \Rightarrow 3 \le 2 \sin^2 x + 3 \le 5 \Rightarrow 3 \le P \le 5 \Rightarrow \heva{& M= 5\\& m=3.}$\\
$ \Rightarrow T = 2M - m^2 = 1.$
  } 
\end{ex}

\begin{ex}%[0D6K3-5]
Cho biểu thức $P=\cos ^4x+\sin ^4x$. Mệnh đề nào sau đây là đúng?
\choice
{$P\leqslant 2, \forall x\in \mathbb{R}$}
{\True $P\leqslant 1, \forall x\in \mathbb{R}$}
{$P\leqslant \sqrt{2}, \forall x\in \mathbb{R}$}
{$P\leqslant \dfrac{{\sqrt{2}}}{2}, \forall x\in \mathbb{R}$}
\loigiai{  Ta có 
	\begin{eqnarray*}
		P & = & \cos^4 x + \sin^4 x \\
		& = & \left( \sin^2 x + \cos^2 x \right)^2 - 2\sin^2 x \cos^2 x \\
		& = & 1 - \dfrac{1}{2} \sin^2 2x \\
		& = & 1 - \dfrac{1}{2} \cdot \dfrac{1 -\cos 4x}{2} \\
		& = & \dfrac{3}{4} + \dfrac{1}{4} \sin 4x. 
	\end{eqnarray*}

  } 
\end{ex}

\begin{ex}%[0D6K3-5]
Tìm giá trị lớn nhất $M$ và nhỏ nhất $m$ của biểu thức $P=\sin ^4x-\cos ^4x.$
\choice
{$M=2, m=-2$}
{$M=\sqrt{2}, m=-\sqrt{2}$}
{\True $M=1, m=-1$}
{$M=1, m=\dfrac{1}{2}$}
\loigiai{ Ta có $P = \sin^4 x - \cos^4 x = \left( \sin^2 x + \cos^2 x \right)^2 \left( \sin^2 x - \cos^2 x \right) = -\cos 2x$.\\
Mà $ -1 \le \cos 2x \le 1 \Rightarrow -1 \ge - \cos 2x \ge 1 \Rightarrow -1 \le P \le 1 \Rightarrow \heva{& M = 1\\& m=-1.}$
  } 
\end{ex}

\begin{ex}%[0D6K3-5]
Tìm giá trị lớn nhất $M$ và nhỏ nhất $m$ của biểu thức $P=\sin ^6x+\cos ^6x.$
\choice
{$M=2, m=0$}
{$M=1, m=\dfrac{1}{2}$}
{\True $M=1, m=\dfrac{1}{4}$}
{$M=\dfrac{1}{4}, m=0$}
\loigiai{  Ta có $P=\sin ^6x+\cos ^6x=\left({{\sin}^2x+{\cos}^2x}\right)^3 - 3\sin ^2x\cos ^2x\left({{\sin}^2x+{\cos}^2x}\right) = 1 -\dfrac{3}{4} \sin^2 2x$.
 } 
\end{ex}

\begin{ex}%[0D6B3-5]
Tìm giá trị lớn nhất $M$ và nhỏ nhất $m$ của biểu thức $P=1-2\left|{\cos 3x}\right|.$
\choice
{$M=3, m=-1$}
{\True $M=1, m=-1$}
{$M=2, m=-2$}
{\True $M=0, m=-2$}
\loigiai{ } 
\end{ex}

\begin{ex}%[0D6G3-5]
Tìm giá trị lớn nhất $M$ của biểu thức $P=4\sin ^2x+\sqrt{2}\sin \left({2x+\dfrac{\pi}{4}}\right).$
\choice
{$M=\sqrt{2}$}
{$M=\sqrt{2}-1$}
{$M=\sqrt{2}+1$}
{\True $M=\sqrt{2}+2$}
\loigiai{ Ta có $P = 4 \dfrac{1 - \cos 2x}{2} + \sin 2x + \cos 2x = \sin 2x - \cos 2x + 2 = \sqrt{2} \sin \left( 2x - \dfrac{\pi}{4} \right)  + 2 \le \sqrt{2} + 2$.
  } 
\end{ex}
% \Closesolutionfile{ans}    
% \begin{center}
% \bf ĐÁP ÁN
% \end{center}
% \begin{multicols}{10}
% \input{ans/ans10D1-TN-3}
% \end{multicols}
%        