
\begin{dang}
   { XÁC ĐỊNH DẤU CỦA CÁC GIÁ TRỊ LƯỢNG GIÁC}
\end{dang}
% \Opensolutionfile{ans}[ans/ans10D1-TN-2]
% \setcounter{ex}{0}
\begin{ex}%[0D6B2-1]
Cho $\alpha $ thuộc góc phần tư thứ nhất của đường tròn lượng giác. Hãy chọn kết quả đúng trong các kết quả sau đây.
\choice
{\True $\sin \alpha >0$}
{$\cos \alpha <0$}
{$\tan \alpha <0$}
{$\cot \alpha <0$}
\loigiai{
Do $\alpha $ thuộc góc phần tư thứ nhất $\Rightarrow \heva{& \sin \alpha >0 \\
& \cos \alpha >0 \\
& \tan \alpha >0 \\
& \cot \alpha >0.} $
} 
\end{ex}
\begin{ex}%[0D6B2-1]
Cho $\alpha $ thuộc góc phần tư thứ hai của đường tròn lượng giác. Hãy chọn kết quả đúng trong các kết quả sau đây.
\choice
{$\sin \alpha >0; \cos\alpha >0$}
{$\sin \alpha <0; \cos\alpha <0$}
{\True $\sin \alpha >0; \cos\alpha <0$}
{$\sin \alpha <0; \cos\alpha >0$}
\loigiai{$\alpha $ thuộc góc phần tư thứ hai $\Rightarrow \heva{& \sin \alpha >0 \\
& \cos \alpha <0.} $
 } 
\end{ex}
\begin{ex}%[0D6B2-1]
Cho $\alpha $ thuộc góc phần tư thứ ba của đường tròn lượng giác. Khẳng định nào sau đây là {\bf SAI}?
\choice
{\True $\sin \alpha >0$}
{$\cos \alpha <0$}
{$\tan \alpha >0$}
{$\cot \alpha >0$}
\loigiai{$\alpha $ thuộc góc phần tư thứ hai $\Rightarrow \heva{& \sin \alpha <0 \\
& \cos \alpha <0 \\
& \tan \alpha >0 \\
& \cot \alpha >0.}$
 } 
\end{ex}

\begin{ex}%[0D6B2-1]
Cho $\alpha $ thuộc góc phần tư thứ tư của đường tròn lượng giác. Khẳng định nào sau đây là đúng?
\choice
{$\sin \alpha >0$}
{\True $\cos \alpha >0$}
{$\tan \alpha >0$}
{$\cot \alpha >0$}
\loigiai{$\alpha $ thuộc góc phần tư thứ tư $\Rightarrow \heva{& \sin \alpha <0 \\
& \cos \alpha >0 \\
& \tan \alpha <0 \\
& \cot \alpha <0.}$
 } 
\end{ex}

\begin{ex}%[0D6B2-1]
Điểm cuối của góc lượng giác $\alpha $ ở góc phần tư thứ mấy nếu $\sin \alpha, \cos \alpha $ cùng dấu? 
\choice
{Thứ $II$}
{Thứ $ IV$}
{Thứ $ II$ hoặc $ IV$}
{\True Thứ $ I$ hoặc $ III$}
\loigiai{
} 
\end{ex}

\begin{ex}%[0D6B2-1]
Điểm cuối của góc lượng giác $\alpha $ ở góc phần tư thứ mấy nếu $\sin \alpha, tan\alpha $ trái dấu? 
\choice
{Thứ $ I$}
{Thứ $ II$ hoặc $ IV$}
{\True Thứ $ II$ hoặc $ III$}
{Thứ $ I$ hoặc $ IV$}
\loigiai{
 } 
\end{ex}

\begin{ex}%[0D6B2-1]
Điểm cuối của góc lượng giác $\alpha $ ở góc phần tư thứ mấy nếu $\cos \alpha =\sqrt{{1-{\sin}^2\alpha}}.$
\choice
{Thứ $ II$}
{Thứ $ I$ hoặc $ II$}
{Thứ $ II$ hoặc $ III$}
{\True Thứ $ I$ hoặc $ IV$}
\loigiai{Ta có $\cos \alpha =\sqrt{{1-{\sin}^2\alpha}}\Leftrightarrow \cos \alpha =\sqrt{{{\cos}^2\alpha}}\Leftrightarrow \cos \alpha =\left|{\cos \alpha}\right| \Rightarrow\cos \alpha \geqslant 0\Rightarrow$ điểm cuối của góc lượng giác $\alpha $ ở góc phần tư thứ $ I$ hoặc $ IV\text.$
 } 
\end{ex}

\begin{ex}%[0D6B2-1]
Điểm cuối của góc lượng giác $\alpha $ ở góc phần tư thứ mấy nếu $\sqrt{{{\sin}^2}}\alpha =\sin \alpha.$
\choice
{Thứ $ III$}
{Thứ $ I$ hoặc $ III$}
{\True Thứ $ I$ hoặc $ II$}
{Thứ $ III$ hoặc $ IV$}
\loigiai{ Ta có $\sqrt{{{\sin}^2\alpha}}\Leftrightarrow \sin \alpha \Leftrightarrow \left|{\sin \alpha}\right|=\sin \alpha.$
Đẳng thức $\left|{\sin \alpha}\right|=\sin \alpha \Rightarrow\sin \alpha \geqslant 0\Rightarrow$ điểm cuối của góc lượng giác $\alpha $ là góc phần tư thứ $ I$ hoặc $ II.$
}
 \end{ex}

\begin{ex}%[0D6B2-1]
Cho $2\pi <\alpha <\dfrac{{5\pi}}{2}.$ Khẳng định nào sau đây đúng?
\choice
{\True $\tan \alpha >0; \cot \alpha >0$}
{$\tan \alpha <0; \cot \alpha <0$}
{$\tan \alpha >0; \cot \alpha <0$}
{$\tan \alpha <0; \cot \alpha >0$}
\loigiai{Ta có $2\pi <\alpha <\dfrac{{5\pi}}{2}\Rightarrow$ điểm cuối cùng $\alpha-\pi $ thuộc góc phần tư thứ $I$
$\Rightarrow \heva{& \tan \alpha >0 \\
& \cot \alpha >0.}$
} \end{ex}

\begin{ex}%[0D6B2-1]
Cho $0<\alpha <\dfrac{\pi}{2}.$ Khẳng định nào sau đây đúng?
\choice
{$\sin \left({\alpha-\pi}\right)\geqslant 0$}
{$\sin \left({\alpha-\pi}\right)\leqslant 0$}
{$\sin \left({\alpha-\pi}\right)<0$}
{\True $\sin \left({\alpha-\pi}\right)<0$}
\loigiai{Ta có $0<\alpha <\dfrac{\pi}{2}\Rightarrow -\pi <\alpha-\pi <-\dfrac{\pi}{2}\Rightarrow$ điểm cuối cung $\alpha-\pi $ thuộc góc phần tư thứ $III\Rightarrow\sin \left({\alpha-\pi}\right)<0.$ } 
\end{ex}

\begin{ex}%[0D6B2-1]
Cho $0<\alpha <\dfrac{\pi}{2}.$ Khẳng định nào sau đây đúng?
\choice
{$\cot \left({\alpha+\dfrac{\pi}{2}}\right)>0$}
{$\cot \left({\alpha+\dfrac{\pi}{2}}\right)\geqslant 0$}
{$\tan \left({\alpha+\pi}\right)<0$}
{\True $\tan \left({\alpha+\pi}\right)>0$}
\loigiai{ Ta có $\heva{& 0<\alpha <\dfrac{\pi}{2}\Rightarrow \dfrac{\pi}{2}<\alpha+\dfrac{\pi}{2}<\pi \Rightarrow\cot \left({\alpha+\dfrac{\pi}{2}}\right)<0 \\
& 0<\alpha <\dfrac{\pi}{2}\Rightarrow \pi <\alpha+\pi <\dfrac{{3\pi}}{2}\Rightarrow\tan \left({\alpha+\pi}\right)>0}.$
} 
\end{ex}

\begin{ex}%[0D6B2-1]
Cho $\dfrac{\pi}{2}<\alpha <\pi.$ Giá trị lượng giác nào sau đây luôn dương?
\choice
{$\sin \left({\pi+\alpha}\right)$}
{\True $\cot \left({\dfrac{\pi}{2}-\alpha}\right)$}
{$\cos \left({-\alpha}\right)$}
{$\tan \left({\pi+\alpha}\right)$}
\loigiai{Ta có 
$\sin \left({\pi+\alpha}\right)=-\sin \alpha;\cot \left({\dfrac{\pi}{2}-\alpha}\right)=\sin \alpha;\cos \left({-\alpha}\right)=\cos \alpha;\tan \left({\pi+\alpha}\right)=\tan \alpha.$\\
Do $\dfrac{\pi}{2}<\alpha <\pi \Rightarrow \heva{& \sin \alpha >0 \\
& \cos \alpha <0 \\
& \tan \alpha <0.}$
 } 
\end{ex}

\begin{ex}%[0D6B2-1]
Cho $\pi <\alpha <\dfrac{{3\pi}}{2}.$ Khẳng định nào sau đây đúng?
\choice
{$\tan \left({\dfrac{{3\pi}}{2}-\alpha}\right)<0$}
{\True $\tan \left({\dfrac{{3\pi}}{2}-\alpha}\right)>0$}
{$\tan \left({\dfrac{{3\pi}}{2}-\alpha}\right)\leqslant 0$}
{$\tan \left({\dfrac{{3\pi}}{2}-\alpha}\right)\geqslant 0$}
\loigiai{ Ta có $\pi <\alpha <\dfrac{{3\pi}}{2}\Rightarrow 0<\dfrac{{3\pi}}{2}-\alpha <\dfrac{\pi}{2}\Rightarrow\heva{& \sin \left({\dfrac{{3\pi}}{2}-\alpha}\right)>0 \\
& \cos \left({\dfrac{{3\pi}}{2}-\alpha}\right)>0}\Rightarrow\tan \left({\dfrac{{3\pi}}{2}-\alpha}\right)>0.$

} \end{ex}

\begin{ex}%[0D6B2-1]
Cho $\dfrac{\pi}{2}<\alpha <\pi $. Xác định dấu của biểu thức $M=\cos \left({-\dfrac{\pi}{2}+\alpha}\right) \cdot\tan \left({\pi-\alpha}\right).$
\choice
{$M\geqslant 0$}
{\True $M>0$}
{$M\leqslant 0$}
{$M<0$}
\loigiai{Ta có $\heva{& \dfrac{\pi}{2}<\alpha <\pi \Rightarrow 0<-\dfrac{\pi}{2}+\alpha <\dfrac{\pi}{2}\Rightarrow\cos \left({-\dfrac{\pi}{2}+\alpha}\right)>0 \\
& \dfrac{\pi}{2}<\alpha <\pi \Rightarrow 0<\pi-\alpha <\dfrac{\pi}{2}\Rightarrow\tan \left({\pi-\alpha}\right)>0}$
$\Rightarrow M>0.$
 } 
\end{ex}

\begin{ex}%[0D6B2-1]
Cho $\pi <\alpha <\dfrac{{3\pi}}{2}$. Xác định dấu của biểu thức $M=\sin \left({\dfrac{\pi}{2}-\alpha}\right)\cdot \cot \left({\pi+\alpha}\right).$
\choice
{$M\geqslant 0$}
{$M>0$}
{$M\leqslant 0$}
{\True $M<0$}
\loigiai{Ta có $\heva{& \pi <\alpha <\dfrac{{3\pi}}{2}\Rightarrow-\dfrac{{3\pi}}{2}<-\alpha <-\pi \Rightarrow-\pi <\dfrac{\pi}{2}-\alpha <-\dfrac{\pi}{2}\Rightarrow\sin \left({\dfrac{\pi}{2}-\alpha}\right)<0 \\
& \pi <\alpha <\dfrac{{3\pi}}{2}\Rightarrow 2\pi <\pi+\alpha <\dfrac{{5\pi}}{2}\Rightarrow \cot \left({\pi+\alpha}\right)>0}$
$\Rightarrow M<0$.
 } 
\end{ex}

\begin{dang}
    {TÍNH GIÁ TRỊ LƯỢNG GIÁC}
\end{dang}

\begin{ex}%[0D6B2-2]
Tính giá trị của $\sin \dfrac{{47\pi}}{6}.$
\choice
{$\sin \dfrac{{47\pi}}{6}=\dfrac{{\sqrt{3}}}{2}$}
{$\sin \dfrac{{47\pi}}{6}=\dfrac{1}{2}$}
{$\sin \dfrac{{47\pi}}{6}=\dfrac{{\sqrt{2}}}{2}$}
{\True $\sin \dfrac{{47\pi}}{6}=-\dfrac{1}{2}$}
\loigiai{Ta có $\sin \dfrac{{47\pi}}{6}=\sin \left({8\pi-\dfrac{\pi}{6}}\right)=\sin \left({-\dfrac{\pi}{6}}\right)=-\sin \dfrac{\pi}{6}=-\dfrac{1}{2}.$ } 
\end{ex}

\begin{ex}%[0D6B2-2]
Tính giá trị của $\cot \dfrac{{89\pi}}{6}.$
\choice
{$\cot \dfrac{{89\pi}}{6}=\sqrt{3}$}
{\True $\cot \dfrac{{89\pi}}{6}=-\sqrt{3}$}
{$\cot \dfrac{{89\pi}}{6}=\dfrac{{\sqrt{3}}}{3}$}
{$\cot \dfrac{{89\pi}}{6}=-\dfrac{{\sqrt{3}}}{3}$}
\loigiai{ Ta có $\cot \dfrac{{89\pi}}{6}=\cot \left({\dfrac{{5\pi}}{6}+14\pi}\right)=\cot \dfrac{{5\pi}}{6}=-\sqrt{3}.$

%{\bf Cách khác.} Máy tính.
%Bấm lên màn hình $\dfrac{1}{{\tan \left({\dfrac{{89\pi}}{6}}\right)}}$ và bấm \key{=}. Màn hình hiện ra kết quả.
} 
\end{ex}

\begin{ex}%[0D6B2-2]
Tính giá trị của $\cos \left[{\dfrac{\pi}{4}+\left({2k+1}\right)\pi}\right].$
\choice
{$\cos \left[{\dfrac{\pi}{4}+\left({2k+1}\right)\pi}\right]=-\dfrac{{\sqrt{3}}}{2}$}
{\True $\cos \left[{\dfrac{\pi}{4}+\left({2k+1}\right)\pi}\right]=-\dfrac{{\sqrt{2}}}{2}$}
{$\cos \left[{\dfrac{\pi}{4}+\left({2k+1}\right)\pi}\right]=-\dfrac{1}{2}$}
{$\cos \left[{\dfrac{\pi}{4}+\left({2k+1}\right)\pi}\right]=\dfrac{{\sqrt{3}}}{2}$}
\loigiai{Ta có $\cos \left[{\dfrac{\pi}{4}+\left({2k+1}\right)\pi}\right]=\cos \left({\dfrac{{5\pi}}{4}+2k\pi}\right)=\cos \dfrac{{5\pi}}{4}$
$=\cos \left({\pi+\dfrac{\pi}{4}}\right)=-\cos \dfrac{\pi}{4}=-\dfrac{{\sqrt{2}}}{2}.$
 } 
\end{ex}

\begin{ex}%[0D6B2-2]
Tính giá trị của $\cos \left[{\dfrac{\pi}{3}+\left({2k+1}\right)\pi}\right].$
\choice
{$\cos \left[{\dfrac{\pi}{3}+\left({2k+1}\right)\pi}\right]=-\dfrac{{\sqrt{3}}}{2}$}
{$\cos \left[{\dfrac{\pi}{3}+\left({2k+1}\right)\pi}\right]=\dfrac{1}{2}$}
{\True $\cos \left[{\dfrac{\pi}{3}+\left({2k+1}\right)\pi}\right]=-\dfrac{1}{2}$}
{$\cos \left[{\dfrac{\pi}{3}+\left({2k+1}\right)\pi}\right]=\dfrac{{\sqrt{3}}}{2}$}
\loigiai{ Ta có $\cos \left[{\dfrac{\pi}{3}+\left({2k+1}\right)\pi}\right]=\cos \left({\dfrac{\pi}{3}+\pi+k2\pi}\right)=\cos \left({\dfrac{\pi}{3}+\pi}\right)=-\cos \dfrac{\pi}{3}=-\dfrac{1}{2}.$

} 
\end{ex}

\begin{ex}%[0D6B2-3]
Tính giá trị biểu thức $P=\dfrac{{\left({\cot {44}^{\circ}+\tan {226}^{\circ}}\right)\cos {406}^{\circ}}}{{\cos {316}^{\circ}}}-\cot 72^{\circ}\cot 18^{\circ}.$ 
\choice
{$P=1$}
{\True $P=1$}
{$P=-\dfrac{1}{2}$}
{$P=\dfrac{1}{2}$}
\loigiai{Sử dụng mối quan hệ của các cung có liên quan đặc biệt, ta có\\
$P=\dfrac{{\left({\cot {44}^{\circ}+\tan {46}^{\circ}}\right)\cos {46}^{\circ}}}{{\cos {44}^{\circ}}}-1=\dfrac{{2\tan {46}^{\circ}\cos {46}^{\circ}}}{{\sin {46}^{\circ}}}-1=2-1=1.$
 } 
\end{ex}

\begin{ex}%[0D6B2-3]
Tính giá trị biểu thức $P=\sin \left({-\dfrac{{14\pi}}{3}}\right)+\dfrac{1}{{{\sin}^2\dfrac{{29\pi}}{4}}}-\tan ^2\dfrac{{3\pi}}{4}.$
\choice
{$P=1+\dfrac{{\sqrt{3}}}{2}$}
{\True $P=1-\dfrac{{\sqrt{3}}}{2}$}
{$P=2+\dfrac{{\sqrt{3}}}{2}$}
{$P=3-\dfrac{{\sqrt{3}}}{2}$}
\loigiai{Ta có $P=\sin \left({-4\pi-\dfrac{{2\pi}}{3}}\right)+\dfrac{1}{{{\sin}^2\left({6\pi+\pi+\dfrac{\pi}{4}}\right)}}-\tan ^2\left({\pi-\dfrac{\pi}{4}}\right)$
$=\sin \left({-\dfrac{{2\pi}}{3}}\right)+\dfrac{1}{{{\sin}^2\left({\pi+\dfrac{\pi}{4}}\right)}}-\tan ^2\left({-\dfrac{\pi}{4}}\right)=-\dfrac{{\sqrt{3}}}{2}+\dfrac{1}{{{\left({-\dfrac{{\sqrt{2}}}{2}}\right)}^2}}-\left({-1}\right)^2=1-\dfrac{{\sqrt{3}}}{2}.$
 } 
\end{ex}

\begin{ex}%[0D6B2-3]
Tính giá trị biểu thức $P=\cos ^2\dfrac{\pi}{8}+\cos ^2\dfrac{{3\pi}}{8}+\cos ^2\dfrac{{5\pi}}{8}+\cos ^2\dfrac{{7\pi}}{8}.$
\choice
{$P=-1$}
{$P=0$}
{$P=1$}
{\True $P=2$}
\loigiai{Ta có $\tan \left({2017\pi+\alpha}\right)$
$\Rightarrow P=2\left({{\cos}^2\dfrac{\pi}{8}+{\cos}^2\dfrac{{3\pi}}{8}}\right)$.\\
Vì $\dfrac{\pi}{8}+\dfrac{{3\pi}}{8}=\dfrac{\pi}{2}\Rightarrow \cos \dfrac{\pi}{8}=\sin \dfrac{{3\pi}}{8}\Rightarrow \cos ^2\dfrac{\pi}{8}=\sin ^2\dfrac{{3\pi}}{8}.$\\
Do đó $ P=2 \left({{\sin}^2\dfrac{{3\pi}}{8}+{\cos}^2\dfrac{{3\pi}}{8}}\right)=2\cdot1=2.$
 } 
\end{ex}

\begin{ex}%[0D6B2-3]
Tính giá trị biểu thức $P=\sin ^210^{\circ}+\sin ^220^{\circ}+\sin ^230^{\circ}+ \ldots +\sin ^280^{\circ}.$
\choice
{$P=0$}
{$P=2$}
{\True $P=4$}
{$P=8$}
\loigiai{Do $10^{\circ}+80^{\circ}=20^{\circ}+70^{\circ}=30^{\circ}+60^{\circ}=40^{\circ}+50^{\circ}=90^{\circ}$ nên các cung lượng giác tương ứng đôi một phụ nhau. Áp dụng công thức $\sin \left({{90}^{\circ}-x}\right)= \cos x$, ta được\\
$P=\left({{\sin}^2{10}^{\circ}+{\cos}^2{10}^{\circ}}\right)+\left({{\sin}^2{20}^{\circ}+{\cos}^2{20}^{\circ}}\right)$\\$+\left({{\sin}^2{30}^{\circ}+{\cos}^2{30}^{\circ}}\right)+\left({{\sin}^2{40}^{\circ}+{\cos}^2{40}^{\circ}}\right)$
$=1+1+1+1=4.$
 } 
\end{ex}

\begin{ex}%[0D6B2-3]
Tính giá trị biểu thức $P=\tan 10^\circ\cdot\tan 20^\circ\cdot\tan 30^\circ \ldots \tan 80^\circ.$
\choice
{$P=0$}
{\True $P=1$}
{$P=4$}
{$P=8$}
\loigiai{ Áp dụng công thức $\tan x\cdot\tan \left({90^\circ-x}\right)=\tan x\cdot \cot x=1.$
Do đó $P=1.$
} 
\end{ex}

\begin{ex}%[0D6B2-3]
Tính giá trị biểu thức $P=\tan 1^{\circ}\tan 2^{\circ}\tan 3^{\circ} \ldots \tan 89^{\circ}.$ 
\choice
{$P=0$}
{\True $P=1$}
{$P=2$}
{$P=3$}
\loigiai{Áp dụng công thức $\tan x\cdot\tan \left({90^\circ-x}\right)=\tan x\cdot \cot x=1.$
Do đó $P=1.$
 } \end{ex}

\begin{dang}
    { TÍNH ĐÚNG SAI}
\end{dang}

\begin{ex}%[0D6Y2-7]
Với góc $\alpha $ bất kì. Khẳng định nào sau đây đúng?
\choice
{$\sin \alpha+\cos \alpha =1$}
{\True $\sin ^2\alpha+\cos ^2\alpha =1$}
{$\sin ^3\alpha+\cos ^3\alpha =1$}
{$\sin ^4\alpha+\cos ^4\alpha =1$}
\loigiai{
 } 
\end{ex}

\begin{ex}%[0D6B2-7]
Với góc $\alpha $ bất kì. Khẳng định nào sau đây đúng?
\choice
{$\sin 2\alpha ^2+\cos ^22\alpha =1$}
{$\sin \left({\alpha ^2}\right)+\cos \left({\alpha ^2}\right)=1$}
{\True $\sin ^2\alpha+\cos ^2\left({180^\circ-\alpha}\right)=1$}
{$\sin ^2\alpha-\cos ^2\left({180^\circ-\alpha}\right)=1$}
\loigiai{ Ta có $\cos \left({180^\circ-\alpha}\right)=-\cos \alpha \Rightarrow \cos ^2\left({180^\circ-\alpha}\right)=\cos ^2\alpha.$\\
Do đó $\sin ^2\alpha+\cos ^2\left({180^\circ-\alpha}\right)=\sin ^2\alpha+\cos ^2\alpha =1.$
} 
\end{ex}

\begin{ex}%[0D6B2-7]
Mệnh đề nào sau đây là {\bf SAI}?
\choice
{$-1\leqslant \sin \alpha \leqslant 1;-1\leqslant \cos \alpha \leqslant 1$}
{$\tan \alpha =\dfrac{{\sin \alpha}}{{\cos \alpha}} \left({\cos \alpha \ne 0}\right)$}
{$\cot \alpha =\dfrac{{\cos \alpha}}{{\sin \alpha}} \left({\sin \alpha \ne 0}\right)$}
{\True $\sin ^2\left({2018\alpha}\right)+\cos ^2\left({2018\alpha}\right)=2018$}
\loigiai{Vì $\sin ^2\left({2018\alpha}\right)+\cos ^2\left({2018\alpha}\right)=1.$} 
\end{ex}

\begin{ex}%[0D6B2-7]
Mệnh đề nào sau đây là {\bf SAI}?
\choice
{$1+\tan ^2\alpha =\dfrac{1}{{{\sin}^2\alpha}}$}
{$1+\cot ^2\alpha =\dfrac{1}{{{\cos}^2\alpha}}$}
{\True $\tan \alpha+\cot \alpha =2$}
{$\tan \alpha.\cot \alpha =1$}
\loigiai{} 
\end{ex}

\begin{ex}%[0D6B2-7]
Để $\tan x$ có nghĩa khi
\choice
{$x=\pm \dfrac{\pi}{2}$}
{$x=0$}
{\True $x\ne \dfrac{\pi}{2}+k\pi$}
{$x\ne k\pi$}
\loigiai{} 
\end{ex}

\begin{ex}%[0D6B2-7]
Điều kiện trong đẳng thức $\tan \alpha \cdot \cot \alpha =1$ là
\choice
{\True $\alpha \ne k\dfrac{\pi}{2}, k\in \mathbb{Z}$}
{$\alpha \ne \dfrac{\pi}{2}+k\pi, k\in \mathbb{Z}$}
{$\alpha \ne k\pi, k\in \mathbb{Z}$}
{$\alpha \ne \dfrac{\pi}{2}+k2\pi, k\in \mathbb{Z}$}
\loigiai{Ta có $\tan \alpha.\cot \alpha =1\Leftrightarrow \dfrac{{\sin \alpha}}{{\cos \alpha}}.\dfrac{{\cos \alpha}}{{\sin \alpha}}=1$.\\
Đẳng thức xác định khi $\heva{& \cos \alpha \ne 0 \\
& \sin \alpha \ne 0}\Leftrightarrow \heva{& \alpha \ne \dfrac{\pi}{2}+k\pi \\
& \alpha \ne k\pi}\Leftrightarrow \alpha \ne k\dfrac{\pi}{2},\left({k\in \mathbb{Z}}\right).$ } 
\end{ex}

\begin{ex}%[0D6B2-7]
Điều kiện để biểu thức $P=\tan \left({\alpha+\dfrac{\pi}{3}}\right)+\cot \left({\alpha-\dfrac{\pi}{6}}\right)$ xác định là
\choice
{$\alpha \ne \dfrac{\pi}{6}+k2\pi,k\in \mathbb{Z}$}
{$\alpha \ne \dfrac{{2\pi}}{3}+k\pi,k\in \mathbb{Z}$}
{\True $\alpha \ne \dfrac{\pi}{6}+k\pi,k\in \mathbb{Z}$}
{$\alpha \ne-\dfrac{\pi}{3}+k2\pi,k\in \mathbb{Z}$}
\loigiai{ Biểu thức xác định khi $\heva{& \alpha+\dfrac{\pi}{3}\ne \dfrac{\pi}{2}+k\pi \\
& \alpha-\dfrac{\pi}{6}\ne k\pi}\Leftrightarrow \alpha \ne \dfrac{\pi}{6}+k\pi \left({k\in \mathbb{Z}}\right).$
 } 
\end{ex}

\begin{ex}%[0D6B2-7]
Mệnh đề nào sau đây đúng?
\choice
{$\sin 60^{\circ}<\sin 150^{\circ}$}
{$\cos 30^{\circ}<\cos 60^{\circ}$}
{\True $\tan 45^{\circ}<\tan 60^{\circ}$}
{$\cot 60^{\circ}>\cot 240^{\circ}$}
\loigiai{Dùng MTCT kiểm tra từng đáp án. 
} \end{ex}

\begin{ex}%[0D6B2-7]
Mệnh đề nào sau đây đúng?
\choice
{$\tan 45^\circ>\tan 46^\circ$}
{\True $\cos 142^\circ>\cos 143^\circ$}
{$\sin 90^\circ13'<\sin 90^\circ14'$}
{$\cot 128^\circ>\cot 126^\circ$}
\loigiai{Trong khoảng giá trị từ $90^\circ$ đến $180^\circ$, khi giá trị góc tăng thì giá trị cos của góc tương ứng giảm } 
\end{ex}

\begin{dang}
    { CÁC CUNG LIÊN QUAN ĐẶC BIỆT}
\end{dang}

\begin{ex}%[0D6Y2-3]
Chọn mệnh đề đúng trong các mệnh đề sau:
\choice
{\True $\cos \left({\dfrac{\pi}{2}-\alpha}\right)=\sin \alpha$}
{$\sin \left({\pi+\alpha}\right)=\sin \alpha$}
{$\cos \left({\dfrac{\pi}{2}+\alpha}\right)=\sin \alpha$}
{$\tan \left({\pi+2\alpha}\right)=\cot \left({2\alpha}\right)$}
\loigiai{
} 
\end{ex}

\begin{ex}%[0D6B2-3]
Với mọi số thực $\alpha $, ta có $\sin \left({\dfrac{{9\pi}}{2}+\alpha}\right)$ bằng
\choice
{$-\sin \alpha$}
{\True $\cos \alpha$}
{$\sin \alpha$}
{$-\cos \alpha$}
\loigiai{Ta có $\sin \left({\dfrac{{9\pi}}{2}+\alpha}\right)=\sin \left({4\pi+\dfrac{\pi}{2}+\alpha}\right)=\sin \left({\dfrac{\pi}{2}+\alpha}\right)=\cos \alpha.$  } 
\end{ex}

\begin{ex}%[0D6B2-3]
Cho $\cos \alpha =\dfrac{1}{3}$. Khi đó $\sin \left({\alpha-\dfrac{{3\pi}}{2}}\right)$ bằng
\choice
{$-\dfrac{2}{3}$}
{$-\dfrac{1}{3}$}
{\True $\dfrac{1}{3}$}
{$\dfrac{2}{3}$}
\loigiai{Ta có $\sin \left({\alpha-\dfrac{{3\pi}}{2}}\right)=\sin \left({\alpha+\dfrac{\pi}{2}-2\pi}\right)=\sin \left({\alpha+\dfrac{\pi}{2}}\right)=\cos \alpha =\dfrac{1}{3}.$} 
\end{ex}

\begin{ex}%[0D6B2-3]
Với mọi $\alpha \in \mathbb{R}$ thì $\tan \left({2017\pi+\alpha}\right)$ bằng 
\choice
{$-\tan \alpha$}
{$\cot \alpha$}
{\True $\tan \alpha$}
{$-\cot \alpha$}
\loigiai{  Ta có $\tan \left({2017\pi+\alpha}\right)=\tan \alpha.$} 
\end{ex}

\begin{ex}%[0D6B2-3]
Đơn giản biểu thức $A=\cos \left({\alpha-\dfrac{\pi}{2}}\right)+\sin (\alpha-\pi)$, ta được
\choice
{$A=\cos \alpha+\sin \alpha$}
{$A=2\sin \alpha$}
{$A=\sin \alpha \cos \alpha$}
{\True $A=0$}
\loigiai{Ta có $A=\cos \left({\alpha-\dfrac{\pi}{2}}\right)+\sin \left({\alpha-\pi}\right)=\cos \left({\dfrac{\pi}{2}-\alpha}\right)-\sin \left({\pi-\alpha}\right)=\sin \alpha-\sin \alpha =0.$
}
 \end{ex}

\begin{ex}%[0D6B2-3]
Rút gọn biểu thức $S=\cos \left({\dfrac{\pi}{2}-x}\right)\sin \left({\pi-x}\right)-\sin \left({\dfrac{\pi}{2}-x}\right)\cos \left({\pi-x}\right)$ ta được
\choice
{$S=0$} 
{$S=\sin ^2x-\cos ^2x$}
{$S=2\sin x\cos x$}
{\True $S=1$}
\loigiai{Ta có $S=\cos \left({\dfrac{\pi}{2}-x}\right)\cdot\sin \left({\pi-x}\right)-\sin \left({\dfrac{\pi}{2}-x}\right)\cdot\cos \left({\pi-x}\right)$
$=\sin x\cdot\sin x-\cos x\cdot\left({-\cos x}\right)=\sin ^2x+\cos ^2x=1.$
 } 
\end{ex}

\begin{ex}%[0D6B2-3]
Cho $P=\sin \left({\pi+\alpha}\right)\cdot\cos \left({\pi-\alpha}\right)$ và $Q=\sin \left({\dfrac{\pi}{2}-\alpha}\right)\cdot\cos \left({\dfrac{\pi}{2}+\alpha}\right).$ Mệnh đề nào dưới đây là đúng? 
\choice
{\True $P+Q=0$}
{$P+Q=-1$}
{$P+Q=1$}
{$P+Q=2$}
\loigiai{Ta có $P=\sin \left({\pi+\alpha}\right)\cdot\cos \left({\pi-\alpha}\right)=-\sin \alpha\cdot\left({-\cos \alpha}\right)=\sin \alpha\cdot\cos\alpha.$ \\
Và $Q=\sin \left({\dfrac{\pi}{2}-\alpha}\right)\cdot\cos \left({\dfrac{\pi}{2}+\alpha}\right)=\cos \alpha\cdot \left({-\sin \alpha}\right)=-\sin \alpha\cdot\cos \alpha.$ \\
Khi đó $P+Q=\sin \alpha\cdot\cos \alpha-\sin \alpha\cdot\cos \alpha =0.$} \end{ex}

\begin{ex}%[0D6K2-3]
Biểu thức lượng giác $\left[{\sin \left({\dfrac{\pi}{2}-x}\right)+\sin \left({10\pi+x}\right)}\right]^2+\left[{\cos \left({\dfrac{{3\pi}}{2}-x}\right)+\cos \left({8\pi-x}\right)}\right]^2$ có giá trị bằng?
\choice
{$1$}
{\True $2$}
{$\dfrac{1}{2}$}
{$\dfrac{3}{4}$}
\loigiai{Ta có $\sin \left({\dfrac{\pi}{2}-x}\right)=\cos x;\,\sin \left({10\pi+x}\right)=\sin x.$\\
Và $\cos \left({\dfrac{{3\pi}}{2}-x}\right)=\cos \left({2\pi-\dfrac{\pi}{2}-x}\right)=\cos \left({\dfrac{\pi}{2}+x}\right)=-\sin x;\,\cos \left({8\pi-x}\right)=\cos x.$\\
Khi đó $\left[{\sin \left({\dfrac{\pi}{2}-x}\right)+\sin \left({10\pi+x}\right)}\right]^2+\left[{\cos \left({\dfrac{{3\pi}}{2}-x}\right)+\cos \left({8\pi-x}\right)}\right]^2$\\
$=\left({\cos x+\sin x}\right)^2+\left({\cos x-\sin x}\right)^2$
$=\cos ^2x+2\cdot\sin x\cdot\cos x+\sin ^2x+\cos ^2x-2\cdot \sin x \cdot \cos x+\sin ^2x=2.$
 } 
\end{ex}

\begin{ex}%[0D6K2-3]
Giá trị biểu thức $P=\left[{\tan \dfrac{{17\pi}}{4}+\tan \left({\dfrac{{7\pi}}{2}-x}\right)}\right]^2+\left[{\cot \dfrac{{13\pi}}{4}+\cot \left({7\pi-x}\right)}\right]^2$ bằng
\choice
{$\dfrac{1}{{{\sin}^2x}}$}
{$\dfrac{1}{{{\cos}^2x}}$}
{\True $\dfrac{2}{{{\sin}^2x}}$}
{$\dfrac{2}{{{\cos}^2x}}$}
\loigiai{Ta có $\tan \dfrac{{17\pi}}{4}=\tan \left({\dfrac{\pi}{4}+4\pi}\right)=\tan \dfrac{\pi}{4}=1$ và $\tan \left({\dfrac{{7\pi}}{2}-x}\right)=\cot x.$ \\
Và $\cot \dfrac{{13\pi}}{4}=\cot \left({\dfrac{\pi}{4}+3\pi}\right)=\cot \dfrac{\pi}{4}=1;\cot \left({7\pi-x}\right)=-\cot x.$\\
Suy ra $P=\left({1+\cot x}\right)^2+\left({1-\cot x}\right)^2=2+2\cot ^2x=\dfrac{2}{{{\sin}^2x}}.$} 
\end{ex}

\begin{ex}%[0D6B2-3]
Biết rằng $\sin \left({x-\dfrac{\pi}{2}}\right)+\sin \dfrac{{13\pi}}{2}=\sin \left({x+\dfrac{\pi}{2}}\right)$ thì giá trị đúng của $\cos x$ là
\choice
{$1$}
{$-1$}
{\True $\dfrac{1}{2}$}
{$-\dfrac{1}{2}$}
\loigiai{ Ta có $\sin \left({x-\dfrac{\pi}{2}}\right)=-\sin \left({\dfrac{\pi}{2}-x}\right)=-\cos x$ và $\sin \left({x+\dfrac{\pi}{2}}\right)=\cos x.$ \\
Kết hợp với giá trị $\sin \dfrac{{13\pi}}{2}=\sin \left({\dfrac{\pi}{2}+6\pi}\right)=\sin \dfrac{\pi}{2}=1.$ \\
Suy ra $\sin \left({x-\dfrac{\pi}{2}}\right)+\sin \dfrac{{13\pi}}{2}=\sin \left({x+\dfrac{\pi}{2}}\right)\Leftrightarrow-\cos x+1=\cos x\Leftrightarrow \cos x=\dfrac{1}{2}.$
} 
\end{ex}

\begin{ex}%[0D6K2-3]
Nếu $\cot 1{,}25\cdot\tan \left({4\pi+1{,}25}\right)-\sin \left({x+\dfrac{\pi}{2}}\right)\cdot \cos \left({6\pi-x}\right)=0$ thì $\tan x$ bằng
\choice
{$1$}
{$-1$}
{\True $0$}
{Một giá trị khác}
\loigiai{  Ta có $\tan \left({4\pi+1{,}25}\right)=\tan 1{,}25$ suy ra $\cot 1{,}25\cdot \tan 1{,}25=1$. \\
Và $\sin \left({x+\dfrac{\pi}{2}}\right)=\cos x;\cos \left({6\pi-x}\right)=\cos \left({x-6\pi}\right)=\cos x.$\\
Khi đó $\cot 1{,}25\cdot\tan \left({4\pi+1{,}25}\right)-\sin \left({x+\dfrac{\pi}{2}}\right)\cdot \cos\left({6\pi-x}\right)=1-\cos ^2x=0\Leftrightarrow \sin x=0.$\\
Mặt khác $\tan x=\dfrac{{\sin x}}{{\cos x}}\Rightarrow\tan x=0.$
} 
\end{ex}

\begin{ex}%[0D6B2-3]
Biết $A,B,C$ là các góc của tam giác $ABC$, mệnh đề nào sau đây đúng:
\choice
{$\sin \left({A+C}\right)=-\sin B$}
{\True $\cos \left({A+C}\right)=-\cos B$}
{$\tan \left({A+C}\right)=\tan B$}
{$\cot \left({A+C}\right)=\cot B$}
\loigiai{Vì $A,B,C$ là ba góc của một tam giác suy ra $A+C=\pi-B.$\\
Khi đó $\sin \left({A+C}\right)=\sin \left({\pi-B}\right)=\sin B;\cos \left({A+C}\right)=\cos \left({\pi-B}\right)=-\cos B.$\\
$\tan \left({A+C}\right)=\tan \left({\pi-B}\right)=-\tan B;\cot \left({A+C}\right)=\cot \left({\pi-B}\right)=-\cot B.$
} 
\end{ex}

\begin{ex}%[0D6B2-3]
Biết $A,B,C$ là các góc của tam giác $ABC,$ khi đó
\choice
{$\sin C=-\sin \left({A+B}\right)$}
{\True $\cos C=\cos \left({A+B}\right)$}
{$\tan C=\tan \left({A+B}\right)$}
{$\cot C=-\cot \left({A+B}\right)$}
\loigiai{ Vì $A,B,C$ là các góc của tam giác $ABC$ nên $C=180^{\circ}-\left({A+B}\right).$\\
Do đó $C$ và $A+B$ là 2 góc bù nhau $\Rightarrow \sin C=\sin \left({A+B}\right);\cos C=-\cos \left({A+B}\right).$\\
Và $\tan C=-\tan \left({A+B}\right);\cot C=\cot \left({A+B}\right).$
} 
\end{ex}

\begin{ex}%[0D6B2-3]
Cho tam giác $ABC$. Khẳng định nào sau đây là {\bf SAI}? 
\choice
{$\sin \dfrac{{A+C}}{2}=\cos \dfrac{B}{2}$}
{$\cos \dfrac{{A+C}}{2}=\sin \dfrac{B}{2}$}
{$\sin \left({A+B}\right)=\sin C$}
{\True $\cos \left({A+B}\right)=\cos C$}
\loigiai{Ta có $A+B+C=\pi \Leftrightarrow A+B=\pi-C$ \\
Do đó $\cos \left({A+B}\right)=\cos \left({\pi-C}\right)=-\cos C.$
 } 
\end{ex}

\begin{ex}%[0D6B2-3]
$A,$$B,$$C$ là ba góc của một tam giác. Hãy tìm hệ thức {\bf SAI}:
\choice
{$\sin A=-\sin \left({2A+B+C}\right)$}
{$\sin A=-\cos \dfrac{{3A+B+C}}{2}$}
{$\cos C=\sin \dfrac{{A+B+3C}}{2}$}
{\True $\sin C=\sin \left({A+B+2C}\right)$}
\loigiai{$A,B,C$ là ba góc của một tam giác $\Rightarrow A+B+C=180^{\circ}\Leftrightarrow A+B=180^{\circ}-C.$\\
Ta có $\sin \left({A+B+2C}\right)=\sin \left({{180}^{\circ}-C+2C}\right)=\sin \left({{180}^{\circ}+C}\right)=-\sin C.$
 } 
\end{ex}

\begin{dang}
    {TÍNH BIỂU THỨC LƯỢNG GIÁC}
\end{dang}

\begin{ex}%[0D6B2-2]
Cho góc $\alpha $ thỏa mãn $\sin \alpha =\dfrac{{12}}{{13}}$ và $\dfrac{\pi}{2}<\alpha <\pi $. Tính $\cos \alpha.$
\choice
{$\cos \alpha =\dfrac{1}{{13}}$}
{$\cos \alpha =\dfrac{5}{{13}}$}
{$\cos \alpha =-\dfrac{5}{{13}}$}
{\True $\cos \alpha =-\dfrac{1}{{13}}$}
\loigiai{Ta có $\heva{& \cos \alpha =\pm \sqrt{{1-{\sin}^2\alpha}}=\pm \dfrac{5}{{13}} \\
& \dfrac{\pi}{2}<\alpha <\pi}\Rightarrow\cos \alpha =-\dfrac{5}{{13}}.$
 } 
\end{ex}

\begin{ex}%[0D6B2-2]
Cho góc $\alpha $ thỏa mãn $\cos \alpha =-\dfrac{{\sqrt{5}}}{3}$ và $\pi <\alpha <\dfrac{{3\pi}}{2}$. Tính $\tan \alpha.$
\choice
{$\tan \alpha =-\dfrac{3}{{\sqrt{5}}}$}
{\True $\tan \alpha =\dfrac{2}{{\sqrt{5}}}$}
{$\tan \alpha =-\dfrac{4}{{\sqrt{5}}}$}
{$\tan \alpha =-\dfrac{2}{{\sqrt{5}}}$}
\loigiai{ Ta có $\heva{& \sin \alpha =\pm \sqrt{{1-{\cos}^2\alpha}}=\pm \dfrac{2}{3} \\
& \pi <\alpha <\dfrac{{3\pi}}{2}}\Rightarrow\alpha =-\dfrac{2}{3}\Rightarrow\tan \alpha =\dfrac{{\sin\alpha}}{{\cos \alpha}}=\dfrac{2}{{\sqrt{5}}}.$

} 
\end{ex}

\begin{ex}%[0D6B2-2]
Cho góc $\alpha $ thỏa mãn $\tan \alpha =-\dfrac{4}{3}$ và $\dfrac{{2017\pi}}{2}<\alpha <\dfrac{{2019\pi}}{2}$. Tính $\sin \alpha.$
\choice
{$\sin \alpha =-\dfrac{3}{5}$}
{$\sin \alpha =\dfrac{3}{5}$}
{$\sin \alpha =-\dfrac{4}{5}$}
{\True $\sin \alpha =\dfrac{4}{5}$}
\loigiai{Ta có $\heva{& 1+\tan ^2\alpha =\dfrac{1}{\cos^2\alpha} \\
& \dfrac{2017\pi}{2}<\alpha <\dfrac{2019\pi}{2}}\overset{\longleftrightarrow}{}\heva{& 1+\left(-\dfrac{4}{3}\right)^2=\dfrac{1}{\cos^2\alpha} \\
& \dfrac{\pi}{2}+504 \cdot 2\pi <\alpha <\dfrac{3\pi}{2}+504 \cdot 2\pi}$ 

$\Rightarrow\cos \alpha =-\dfrac{3}{5}$. Mà $\tan \alpha =\dfrac{{\sin \alpha}}{{\cos \alpha}}=-\dfrac{4}{3}=\dfrac{\sin \alpha}{-\dfrac{3}{5}}\Rightarrow\sin \alpha =\dfrac{4}{5}$.
 } 
\end{ex}

\begin{ex}%[0D6B2-2]
Cho góc $\alpha $ thỏa mãn $\cos \alpha =-\dfrac{{12}}{{13}}$ và $\dfrac{\pi}{2}<\alpha <\pi.$ Tính $\tan \alpha.$
\choice
{$\tan \alpha =-\dfrac{{12}}{5}$}
{$\tan \alpha =\dfrac{5}{{12}}$}
{\True $\tan \alpha =-\dfrac{5}{{12}}$}
{$\tan \alpha =\dfrac{{12}}{5}$}
\loigiai{Ta có $\heva{& \sin \alpha =\pm \sqrt{{1-{\cos}^2\alpha}}=\pm \dfrac{5}{13} \\
& \dfrac{\pi}{2}<\alpha <\pi}\Rightarrow\alpha =\dfrac{5}{13}\Rightarrow\tan \alpha =\dfrac{\sin \alpha}{\cos \alpha}=-\dfrac{5}{12}.$
 } 
\end{ex}

\begin{ex}%[0D6K2-2]
Cho góc $\alpha $ thỏa mãn $\tan \alpha =2$ và $180^{\circ}<\alpha <270^{\circ}.$ Tính $P=\cos \alpha+\sin \alpha.$
\choice
{\True $P=-\dfrac{{3\sqrt{5}}}{5}$}
{$P=1-\sqrt{5}$}
{$P=\dfrac{3\sqrt{5}}{2}$}
{$P=\dfrac{\sqrt{5}-1}{2}$}
\loigiai{ Ta có $\heva{& \cos ^2\alpha =\dfrac{1}{{1+{\tan}^2\alpha}}=\dfrac{1}{5}\Rightarrow \cos \alpha =\pm \dfrac{1}{{\sqrt{5}}} \\
& 180^{\circ}<\alpha <270^{\circ}}\Rightarrow\cos \alpha =-\dfrac{1}{{\sqrt{5}}}$\\
$\Rightarrow\sin \alpha =\tan \alpha.\cos \alpha =-\dfrac{2}{{\sqrt{5}}}$.\\
 Do đó, $\sin \alpha+\cos \alpha =-\dfrac{3}{{\sqrt{5}}}=-\dfrac{{3\sqrt{5}}}{5}.$
} 
\end{ex}

\begin{ex}%[0D6B2-2]
Cho góc $\alpha $ thỏa $\sin \alpha =\dfrac{3}{5}$ và $90^{\circ}<\alpha <180^{\circ}.$ Khẳng định nào sau đây đúng?
\choice
{$\cot \alpha =-\dfrac{4}{5}$}
{$ \cos\alpha =\dfrac{4}{5}$}
{$\tan \alpha =\dfrac{5}{4}$}
{\True $ \cos\alpha =-\dfrac{4}{5}$}
\loigiai{  Ta có $\heva{& \cos \alpha =\pm \sqrt{{1-{\sin}^2\alpha}}=\pm \dfrac{4}{5} \\
& 90^\circ<\alpha <180^\circ}\Rightarrow\cos \alpha =-\dfrac{4}{5}.$
} 
\end{ex}

\begin{ex}%[0D6B2-2]
Cho góc $\alpha $ thỏa $ \cot\alpha =\dfrac{3}{4}$ và $0^{\circ}<\alpha <90^{\circ}.$ Khẳng định nào sau đây đúng?
\choice
{$ cos\alpha =-\dfrac{4}{5}$}
{$ cos\alpha =\dfrac{4}{5}$}
{\True $\sin \alpha =\dfrac{4}{5}$}
{$ sin\alpha =-\dfrac{4}{5}$}
\loigiai{Ta có $\heva{& \dfrac{1}{{{\sin}^2\alpha}}=1+\cot ^2\alpha =1+\left({\dfrac{3}{4}}\right)^2=\dfrac{{25}}{{16}} \\
& 0^\circ<\alpha <90^\circ}\Rightarrow\sin \alpha =\dfrac{4}{5}.$
} 
\end{ex}

\begin{ex}%[0D6B2-2]
Cho góc $\alpha $ thỏa mãn $\sin \alpha =\dfrac{3}{5}$ và $\dfrac{\pi}{2}<\alpha <\pi $. Tính $P=\dfrac{{\tan \alpha}}{{1+{\tan}^2\alpha}}.$
\choice
{$P=-3$}
{$P=\dfrac{3}{7}$}
{$P=\dfrac{{12}}{{25}}$}
{\True $P=-\dfrac{{12}}{{25}}$}
\loigiai{Ta có $\heva{& \cos \alpha =\pm \sqrt{{1-{\sin}^2\alpha}}=\pm \dfrac{4}{5} \\
& \dfrac{\pi}{2}<\alpha <\pi}\Rightarrow\cos \alpha =-\dfrac{4}{5}\Rightarrow\tan \alpha =-\dfrac{3}{4}$.\\
Thay $\tan \alpha =-\dfrac{3}{4}$ vào $P$, ta được $P=-\dfrac{{12}}{{25}}$.
 } 
\end{ex}

\begin{ex}%[0D6B2-2]
Cho góc $\alpha $ thỏa $\sin \alpha =\dfrac{1}{3}$ và $90^{\circ}<\alpha <180^{\circ}$. Tính $P=\dfrac{{2\tan \alpha+3\cot \alpha+1}}{{\tan \alpha+\cot \alpha}}.$
\choice
{$P=\dfrac{{19+2\sqrt{2}}}{9}$}
{$P=\dfrac{{19-2\sqrt{2}}}{9}$}
{\True $P=\dfrac{{26-2\sqrt{2}}}{9}$}
{$P=\dfrac{{26+2\sqrt{2}}}{9}$}
\loigiai{Ta có $\heva{& \cos \alpha =\pm \sqrt{{1-{\sin}^2\alpha}}=\pm \dfrac{{2\sqrt{2}}}{3} \\
& 90^{\circ}<\alpha <180^{\circ}}\Rightarrow\cos \alpha =-\dfrac{{2\sqrt{2}}}{3}\Rightarrow\heva{& \tan \alpha =-\dfrac{{\sqrt{2}}}{4} \\
& \cot \alpha =-2\sqrt{2}.}$\\
Thay $\heva{& \tan \alpha =-\dfrac{{\sqrt{2}}}{4} \\
& \cot \alpha =-2\sqrt{2}}$ vào $P$, ta được $P=\dfrac{{26-2\sqrt{2}}}{9}$.
 } 
\end{ex}

\begin{ex}%[0D6B2-2]
Cho góc $\alpha $ thỏa mãn $\sin \left({\pi+\alpha}\right)=-\dfrac{1}{3}$ và $\dfrac{\pi}{2}<\alpha <\pi $. Tính $P=\tan \left({\dfrac{{7\pi}}{2}-\alpha}\right)$.
\choice
{$P=2\sqrt{2}$}
{\True $P=-2\sqrt{2}$}
{$P=\dfrac{{\sqrt{2}}}{4}$}
{$P=-\dfrac{{\sqrt{2}}}{4}$}
\loigiai{ Ta có $P=\tan \left({\dfrac{{7\pi}}{2}-\alpha}\right)=\tan \left({3\pi+\dfrac{\pi}{2}-\alpha}\right)=\tan \left({\dfrac{\pi}{2}-\alpha}\right)=\cot \alpha =\dfrac{{\cos \alpha}}{{\sin \alpha}}$.\\
Theo giả thiết: $\sin \left({\pi+\alpha}\right)=-\dfrac{1}{3}\Leftrightarrow-\sin \alpha =-\dfrac{1}{3}\Leftrightarrow \sin \alpha =\dfrac{1}{3}$.\\
Ta có $\heva{& \cos \alpha =\pm \sqrt{{1-{\sin}^2\alpha}}=\pm \dfrac{{2\sqrt{2}}}{3} \\
& \dfrac{\pi}{2}<\alpha <\pi}\Rightarrow\cos \alpha =-\dfrac{2\sqrt{2}}{3}\Rightarrow P=-2\sqrt{2}.$ 
} 
\end{ex}

\begin{ex}%[0D6B2-2]
Cho góc $\alpha $ thỏa mãn $\cos \alpha =\dfrac{3}{5}$ và $-\dfrac{\pi}{2}<\alpha <0$. Tính $P=\sqrt{{5+3\tan a}}+\sqrt{{6-4\cot a}}.$
\choice
{\True $P=4$}
{$P=-4$}
{$P=6$}
{$P=-6$}
\loigiai{Ta có $\heva{& \sin \alpha =\pm \sqrt{{1-{\cos}^2\alpha}}=\pm \dfrac{4}{5} \\
&-\dfrac{\pi}{2}<\alpha <0}\Rightarrow\sin \alpha =-\dfrac{4}{5}\Rightarrow\heva{& \tan \alpha =-\dfrac{4}{3} \\
& \cot \alpha =-\dfrac{3}{4}}$.\\
Thay $\heva{& \tan \alpha =-\dfrac{4}{3} \\
& \cot \alpha =-\dfrac{3}{4}}$ vào $P$, ta được $P=4$.
} 
\end{ex}

\begin{ex}%[0D6B2-2]
Cho góc $\alpha $ thỏa mãn $\cos \alpha =\dfrac{3}{5}$ và $\dfrac{\pi}{4}<\alpha <\dfrac{\pi}{2}$. Tính $P=\sqrt{{{\tan}^2\alpha-2\tan \alpha+1}}$.
\choice
{$P=-\dfrac{1}{3}$}
{\True $P=\dfrac{1}{3}$}
{$P=\dfrac{7}{3}$}
{$P=-\dfrac{7}{3}$}
\loigiai{ Ta có $P=\sqrt{{{\left({\tan \alpha-1}\right)}^2}}=\left|{\tan \alpha-1}\right|$.
Vì $\dfrac{\pi}{4}<\alpha <\dfrac{\pi}{2}\Rightarrow\tan \alpha >1\Rightarrow P=\tan \alpha-1.$\\
Theo giả thiết $\heva{& \sin \alpha =\pm \sqrt{{1-{\cos}^2\alpha}}=\pm \dfrac{4}{5} \\
& \dfrac{\pi}{4}<\alpha <\dfrac{\pi}{2}}
\Rightarrow\sin \alpha =\dfrac{4}{5}
\Rightarrow\tan \alpha =\dfrac{4}{3}
\Rightarrow P=\dfrac{1}{3}.$ 
} 
\end{ex}

\begin{ex}%[0D6B2-2]
Cho góc $\alpha $ thỏa mãn $\dfrac{\pi}{2}<\alpha <2\pi $ và $\tan \left({\alpha+\dfrac{\pi}{4}}\right)=1$. Tính $P=\cos \left({\alpha-\dfrac{\pi}{6}}\right)+\sin \alpha $.
\choice
{$P=\dfrac{{\sqrt{3}}}{2}$}
{$P=\dfrac{{\sqrt{6}+3\sqrt{2}}}{4}$}
{\True $P=-\dfrac{{\sqrt{3}}}{2}$}
{$P=\dfrac{{\sqrt{6}-3\sqrt{2}}}{4}$}
\loigiai{Ta có $\heva{& \dfrac{\pi}{2}<\alpha <2\pi \Leftrightarrow \dfrac{{3\pi}}{4}<\alpha+\dfrac{\pi}{4}<\dfrac{{9\pi}}{4} \\
& \tan \left({\alpha+\dfrac{\pi}{4}}\right)=1}\Rightarrow \alpha+\dfrac{\pi}{4}=\dfrac{{5\pi}}{4}\Rightarrow \alpha =\pi.$\\
Thay $\alpha =\pi $ vào $P$, ta được $P=-\dfrac{{\sqrt{3}}}{2}$.
 } 
\end{ex}

\begin{ex}%[0D6B2-2]
Cho góc $\alpha $ thỏa mãn $\dfrac{\pi}{2}<\alpha <2\pi $ và $\cot \left({\alpha+\dfrac{\pi}{3}}\right)=-\sqrt{3}$. Tính giá trị của biểu thức $P=\sin \left({\alpha+\dfrac{\pi}{6}}\right)+\cos \alpha $.
\choice
{$P=\dfrac{{\sqrt{3}}}{2}$}
{$P=1$}
{$P=-1$}
{$P=-\dfrac{{\sqrt{3}}}{2}$}
\loigiai{Ta có $\heva{& \dfrac{\pi}{2}<\alpha <2\pi \Leftrightarrow \dfrac{{5\pi}}{6}<\alpha+\dfrac{\pi}{3}<\dfrac{{7\pi}}{3} \\
& \cot \left({\alpha+\dfrac{\pi}{3}}\right)=-\sqrt{3}}\Rightarrow \alpha+\dfrac{\pi}{3}=\dfrac{{11\pi}}{6}\Rightarrow \alpha =\dfrac{{3\pi}}{2}.$
\\
Thay $\alpha =\dfrac{{3\pi}}{2}$ vào $P$, ta được $P=-\dfrac{{\sqrt{3}}}{2}$.
 } 
\end{ex}

\begin{ex}%[0D6K2-2]
Cho góc $\alpha $ thỏa mãn $\tan \alpha =-\dfrac{4}{3}$ và $\dfrac{\pi}{2}<\alpha <\pi $. Tính $P=\dfrac{{{\sin}^2\alpha-\cos \alpha}}{{\sin \alpha-{\cos}^2\alpha}}.$ 
\choice
{$P=\dfrac{{30}}{{11}}$}
{\True $P=\dfrac{{31}}{{11}}$}
{$P=\dfrac{{32}}{{11}}$}
{$P=\dfrac{{34}}{{11}}$}
\loigiai{ Ta có $\heva{& \cos ^2\alpha =\dfrac{1}{{1+{\tan}^2\alpha}}=\dfrac{9}{{25}}\Rightarrow \cos \alpha =\pm \dfrac{3}{5} \\
& \dfrac{\pi}{2}<\alpha <\pi}\Rightarrow  \cos \alpha =-\dfrac{3}{5}$
$\Rightarrow  \sin \alpha =\tan \alpha\cdot\cos \alpha =\dfrac{4}{5}$. \\
Thay $\sin \alpha =\dfrac{4}{5}$ và $\cos \alpha =-\dfrac{3}{5}$ vào $P$, ta được $P=\dfrac{{31}}{{11}}.$
} 
\end{ex}

\begin{ex}%[0D6K2-2]
Cho góc $\alpha $ thỏa mãn $\tan \alpha =2.$ Tính $P=\dfrac{{3\sin \alpha-2\cos \alpha}}{{5\cos \alpha+7\sin \alpha}}.$ 
\choice
{$P=-\dfrac{4}{9}$}
{$P=\dfrac{4}{9}$}
{$P=-\dfrac{4}{{19}}$}
{\True $P=\dfrac{4}{{19}}$}
\loigiai{Chia cả tử và mẫu của $P$ cho $\cos \alpha $ ta được $P=\dfrac{{3\tan \alpha-2}}{{5+7\tan \alpha}}=\dfrac{{3\cdot 2-2}}{{5+7\cdot 2}}=\dfrac{4}{{19}}.$

 } 
\end{ex}

\begin{ex}%[0D6K2-2]
Cho góc $\alpha $ thỏa mãn $\cot\alpha =\dfrac{1}{3}.$ Tính $P=\dfrac{{3\sin \alpha+4\cos \alpha}}{{2\sin \alpha-5\cos \alpha}}.$ 
\choice
{$P=-\dfrac{{15}}{{13}}$}
{$P=\dfrac{{15}}{{13}}$}
{$P=-13$}
{\True $P=13$}
\loigiai{ Chia cả tử và mẫu của $P$ cho $\sin \alpha $ ta được $P=\dfrac{{3+4\cot \alpha}}{{2-5\cot \alpha}}=\dfrac{{3+4\cdot \dfrac{1}{3}}}{{2-5\cdot \dfrac{1}{3}}}=13$. 
} 
\end{ex}

\begin{ex}%[0D6K2-2]
Cho góc $\alpha $ thỏa mãn $\tan\alpha =2.$ Tính $P=\dfrac{{2{\sin}^2\alpha+3\sin \alpha \cdot\cos \alpha+4{\cos}^2\alpha}}{{5{\sin}^2\alpha+6{\cos}^2\alpha}}.$ 
\choice
{\True $P=\dfrac{9}{{13}}$}
{$P=\dfrac{9}{{65}}$}
{$P=-\dfrac{9}{{65}}$}
{$P=\dfrac{{24}}{{29}}$}
\loigiai{Chia cả tử và mẫu của $P$ cho $\cos ^2\alpha $ ta được 
$P=\dfrac{{2{\tan}^2\alpha+3\tan \alpha+4}}{{5{\tan}^2\alpha+6}}=\dfrac{{{2\cdot2}^2+3\cdot 2+4}}{{{5\cdot 2}^2+6}}=\dfrac{9}{{13}}.$
 } 
\end{ex}

\begin{ex}%[0D6K2-2]
Cho góc $\alpha $ thỏa mãn $\tan\alpha =\dfrac{1}{2}.$ Tính $P=\dfrac{{2{\sin}^2\alpha+3\sin \alpha\cdot \cos \alpha-4{\cos}^2\alpha}}{{5{\cos}^2\alpha-{\sin}^2\alpha}}.$ 
\choice
{$P=-\dfrac{8}{{13}}$}
{$P=\dfrac{2}{{19}}$}
{$P=-\dfrac{2}{{19}}$}
{\True $P=-\dfrac{8}{{19}}$}
\loigiai{Chia cả tử và mẫu của $P$ cho $\cos ^2\alpha $ ta được
$$P=\dfrac{{2{\tan}^2\alpha+3\tan \alpha-4}}{{5-{\tan}^2\alpha}}=\dfrac{{2\cdot{\left({\dfrac{1}{2}}\right)}^2+3\cdot \dfrac{1}{2}-4}}{{5-{\left({\dfrac{1}{2}}\right)}^2}}=-\dfrac{8}{{19}}.$$
 }
 \end{ex}

\begin{ex}%[0D6K2-2]
Cho góc $\alpha $ thỏa mãn $\tan\alpha =5.$ Tính $P=\sin ^4\alpha-\cos ^4\alpha.$ 
\choice
{$P=\dfrac{9}{{13}}$}
{$P=\dfrac{{10}}{{13}}$}
{$P=\dfrac{{11}}{{13}}$}
{\True $P=\dfrac{{12}}{{13}}$}
\loigiai{Ta có $P=\left({{\sin}^2\alpha-{\cos}^2\alpha}\right)\cdot \left({{\sin}^2\alpha+{\cos}^2\alpha}\right)=\sin ^2\alpha-\cos ^2\alpha\cdot \left(*\right)$ \\
Chia hai vế của $\left(*\right)$ cho $\cos ^2\alpha $ ta được $\dfrac{P}{{{\cos}^2\alpha}}=\dfrac{{{\sin}^2\alpha}}{{{\cos}^2\alpha}}-1$
$$\Leftrightarrow P\left({1+\tan^2\alpha}\right)=\tan ^2\alpha-1\Leftrightarrow P=\dfrac{{{\tan}^2\alpha-1}}{{1+\tan^2\alpha}}=\dfrac{{5^2-1}}{{1+5^2}}=\dfrac{{12}}{{13}}.$$
 } 
\end{ex}

\begin{ex}%[0D6B2-2]
Cho góc $\alpha $ thỏa mãn $\sin \alpha+\cos \alpha =\dfrac{5}{4}.$ Tính $P=\sin \alpha\cdot\cos \alpha.$
\choice
{$P=\dfrac{9}{{16}}$}
{\True $P=\dfrac{9}{{32}}$}
{$P=\dfrac{9}{8}$}
{$P=\dfrac{1}{8}$}
\loigiai{ Từ giả thiết, ta có $\left({\sin \alpha+\cos \alpha}\right)^2=\dfrac{{25}}{{16}}\Leftrightarrow 1+2\sin \alpha\cdot \cos \alpha =\dfrac{{25}}{{16}}$
$\Rightarrow  P=\sin \alpha\cdot\cos \alpha =\dfrac{9}{{32}}.$
} 
\end{ex}

\begin{ex}%[0D6K2-2]
Cho góc $\alpha $ thỏa mãn $\sin \alpha \cos \alpha =\dfrac{{12}}{{25}}$ và $\sin \alpha+\cos \alpha >0.$ Tính $P=\sin ^3\alpha+\cos ^3\alpha.$
\choice
{\True $P=\dfrac{{91}}{{125}}$}
{$P=\dfrac{{49}}{{25}}$}
{$P=\dfrac{7}{5}$}
{$P=\dfrac{1}{9}$}
\loigiai{Áp dụng $a^3+b^3=\left({a+b}\right)^3-3ab\left({a+b}\right)$, ta có \\
$P=\sin ^3\alpha+\cos ^3\alpha =\left({\sin \alpha+\cos \alpha}\right)^3-3\sin \alpha \cos \alpha \left({\sin \alpha+\cos \alpha}\right).$\\
Ta có $\left({\sin \alpha+\cos \alpha}\right)^2=\sin ^2\alpha+2\sin \alpha \cos \alpha+\cos ^2\alpha =1+\dfrac{{24}}{{25}}=\dfrac{{49}}{{25}}$.\\
Vì $\sin \alpha+\cos \alpha >0$ nên ta chọn $\sin \alpha+\cos \alpha =\dfrac{7}{5}$.\\
Thay $\heva{& \sin \alpha+\cos \alpha =\dfrac{7}{5} \\
& \sin \alpha \cos \alpha =\dfrac{{12}}{{25}}}$ vào $P$, ta được $P=\left({\dfrac{7}{5}}\right)^3-3\cdot \dfrac{{12}}{{25}}\cdot \dfrac{7}{5}=\dfrac{{91}}{{125}}.$
 } 
\end{ex}

\begin{ex}%[0D6K2-2]
Cho góc $\alpha $ thỏa mãn $0<\alpha <\dfrac{\pi}{4}$ và $\sin \alpha+\cos \alpha =\dfrac{{\sqrt{5}}}{2}$. Tính $P=\sin \alpha-\cos \alpha.$
\choice
{$P=\dfrac{{\sqrt{3}}}{2}$}
{$P=\dfrac{1}{2}$}
{$P=-\dfrac{1}{2}$}
{\True $P=-\dfrac{{\sqrt{3}}}{2}$}
\loigiai{ Ta có $\left({\sin \alpha-\cos \alpha}\right)^2+\left({\sin \alpha+\cos \alpha}\right)^2=2\left({{\sin}^2\alpha+{\cos}^2\alpha}\right)=2$.\\
Suy ra $\left({\sin \alpha-\cos \alpha}\right)^2=2-\left({\sin \alpha+\cos \alpha}\right)^2=2-\dfrac{5}{4}=\dfrac{3}{4}$.\\
Do $0<\alpha <\dfrac{\pi}{4}$ suy ra $\sin \alpha <\cos \alpha $ nên $\sin \alpha-\cos \alpha <0$. Vậy $P=-\dfrac{{\sqrt{3}}}{2}.$
} 
\end{ex}

\begin{ex}%[0D6K2-2]
Cho góc $\alpha $ thỏa mãn $\sin \alpha+\cos \alpha =m.$. Tính $P=\left|{\sin \alpha-\cos \alpha}\right|.$
\choice
{$P=2-m$}
{$P=2-m^2$}
{$P=m^2-2$}
{\True $P=\sqrt{{2-m^2}}$}
\loigiai{Ta có $\left({\sin \alpha-\cos \alpha}\right)^2+\left({\sin \alpha+\cos \alpha}\right)^2=2\left({{\sin}^2\alpha+{\cos}^2\alpha}\right)=2$.\\
Suy ra $\left({\sin \alpha-\cos \alpha}\right)^2=2-\left({\sin \alpha+\cos \alpha}\right)^2=2-m^2$
$\Rightarrow   P=\left|{\sin \alpha-\cos \alpha}\right|=\sqrt{{2-m^2}}.$
 }
 \end{ex}

\begin{ex}%[0D6K2-2]
Cho góc $\alpha $ thỏa mãn $\tan \alpha+\cot \alpha =2.$ Tính $P=\tan ^2\alpha+\cot ^2\alpha.$
\choice
{$P=1$}
{\True $P=2$}
{$P=3$}
{$P=4$}
\loigiai{ Ta có $P=\tan ^2\alpha+\cot ^2\alpha =\left({\tan \alpha+\cot \alpha}\right)^2-2\tan \alpha\cdot\cot \alpha =2^2-2\cdot 1=2.$
} 
\end{ex}

\begin{ex}%[0D6K2-2]
Cho góc $\alpha $ thỏa mãn $\tan \alpha+\cot \alpha =5.$ Tính $P=\tan ^3\alpha+\cot ^3\alpha.$
\choice
{$P=100$}
{\True $P=110$}
{$P=112$}
{$P=115$}
\loigiai{ Ta có $P=\tan ^3\alpha+\cot ^3\alpha =\left({\tan \alpha+\cot \alpha}\right)^3-3\tan \alpha \cot \alpha \left({\tan \alpha+\cot \alpha}\right)$
$=5^3-3\cdot 5=110$.
} 
\end{ex}

\begin{ex}%[0D6K2-2]
Cho góc $\alpha $ thỏa mãn $\sin \alpha+\cos \alpha =\dfrac{{\sqrt{2}}}{2}.$ Tính $P=\tan ^2\alpha+\cot ^2\alpha.$
\choice
{$P=12$}
{\True $P=14$}
{$P=16$}
{$P=18$}
\loigiai{Ta có $\sin \alpha+\cos \alpha =\dfrac{{\sqrt{2}}}{2}\to \left({\sin \alpha+\cos \alpha}\right)^2=\dfrac{1}{2}\Leftrightarrow \sin \alpha \cos \alpha =-\dfrac{1}{4}.$\\
Khi đó $P=\dfrac{{{\sin}^2\alpha}}{{{\cos}^2\alpha}}+\dfrac{{{\cos}^2\alpha}}{{{\sin}^2\alpha}}=\dfrac{{{\sin}^4\alpha+{\cos}^4\alpha}}{{{\sin}^2\alpha\cdot {\cos}^2\alpha}}$\\
$=\dfrac{{{\left({{\sin}^2\alpha+{\cos}^2\alpha}\right)}^2-2{\sin}^2\alpha.{\cos}^2\alpha}}{{{\sin}^2\alpha\cdot {\cos}^2\alpha}}=\dfrac{{1-2{\left({\sin \alpha \cos \alpha}\right)}^2}}{{{\left({\sin \alpha \cos \alpha}\right)}^2}}=14.$
 } 
\end{ex}

\begin{ex}%[0D6K2-2]
Cho góc $\alpha $ thỏa mãn $\dfrac{\pi}{2}<\alpha <\pi $ và $\tan \alpha-\cot \alpha =1$. Tính $P=\tan \alpha+\cot \alpha.$
\choice
{$P=1$}
{$P=-1$}
{\True $P=-\sqrt{5}$}
{$P=\sqrt{5}$}
\loigiai{ Ta có 
$\tan \alpha-\cot \alpha =1\Leftrightarrow \tan \alpha-\dfrac{1}{{\tan \alpha}}=1$ \\
$\Leftrightarrow \tan ^2\alpha-\tan \alpha-1=0\Leftrightarrow \tan \alpha =\dfrac{{1\pm \sqrt{5}}}{2}.$\\
Do $\dfrac{\pi}{2}<\alpha <\pi $ suy ra $\tan \alpha <0$ nên $\tan \alpha =\dfrac{{1-\sqrt{5}}}{2}\Rightarrow  \cot \alpha =\dfrac{1}{{\tan \alpha}}=\dfrac{2}{{1-\sqrt{5}}}.$\\
Thay $\tan \alpha =\dfrac{{1-\sqrt{5}}}{2}$ và $\cot \alpha =\dfrac{2}{{1-\sqrt{5}}}$ vào $P$, ta được $P=\dfrac{{1-\sqrt{5}}}{2}+\dfrac{2}{{1-\sqrt{5}}}=-\sqrt{5}.$

}
\end{ex}

\begin{ex}%[0D6K2-2]
Cho góc $\alpha $ thỏa mãn $3\cos \alpha+2\sin \alpha =2$ và $\sin \alpha <0$. Tính $\sin \alpha.$
\choice
{\True $\sin \alpha =-\dfrac{5}{{13}}$}
{$\sin \alpha =-\dfrac{7}{{13}}$}
{$\sin \alpha =-\dfrac{9}{{13}}$}
{$\sin \alpha =-\dfrac{{12}}{{13}}$}
\loigiai{ Ta có $3\cos \alpha+2\sin \alpha =2\Leftrightarrow \left({3\cos \alpha+2\sin \alpha}\right)^2=4$\\
$\Leftrightarrow 9\cos ^2\alpha+12\cos \alpha \cdot \sin \alpha+4\sin ^2\alpha =4\Leftrightarrow 5\cos ^2\alpha+12\cos \alpha \cdot \sin \alpha =0$\\$\Leftrightarrow\cos\alpha\left({5\cos\alpha+12\sin\alpha}\right)=0\Leftrightarrow\hoac{&\cos\alpha=0\\&5\cos\alpha+12\sin\alpha=0.}$
\begin{itemize}
\item $\cos\alpha =0$
$\Rightarrow \sin \alpha =1$: loại (vì $\sin \alpha <0$).
\item $5 cos\alpha+12\sin \alpha =0$, ta có hệ phương trình $\heva{& 5\cos \alpha+12\sin \alpha =0 \\
& 3\cos \alpha+2\sin \alpha =2}\Leftrightarrow \heva{& \sin \alpha =-\dfrac{5}{{13}} \\
& \cos \alpha =\dfrac{{12}}{{13}}.}$
\end{itemize}

} 
\end{ex}

\begin{ex}%[0D6K2-2]
Cho góc $\alpha $ thỏa mãn $\pi <\alpha <\dfrac{{3\pi}}{2}$ và $\sin \alpha-2\cos \alpha =1$. Tính $P=2\tan \alpha-\cot \alpha.$
\choice
{$P=\dfrac{1}{2}$}
{$P=\dfrac{1}{4}$}
{\True $P=\dfrac{1}{6}$}
{$P=\dfrac{1}{8}$}
\loigiai{ Với $\pi <\alpha <\dfrac{{3\pi}}{2}$ suy ra $\heva{& \sin \alpha <0 \\
& \cos \alpha <0}$.\\
Ta có $\heva{& \sin \alpha-2\cos \alpha =1 \\
& \sin ^2\alpha+\cos ^2\alpha =1}\Rightarrow \left({1+2\cos \alpha}\right)^2+\cos ^2\alpha =1$
$\Leftrightarrow 5\cos ^2\alpha+4\cos \alpha =0\Leftrightarrow \hoac{& \cos \alpha =0 \left(\text{loại}\right) \\
& \cos \alpha =-\dfrac{4}{5}.}$\\
Từ hệ thức $\sin ^2\alpha+\cos ^2\alpha =1$, suy ra $\sin \alpha =-\dfrac{3}{5}$ (do $\sin \alpha <0$)\\
$\Rightarrow  \tan \alpha =\dfrac{{\sin \alpha}}{{\cos \alpha}}=\dfrac{3}{4}$ và $\cot \alpha =\dfrac{{\cos \alpha}}{{\sin \alpha}}=\dfrac{4}{3}.$\\
Thay $\tan \alpha =\dfrac{3}{4}$ và $\cot \alpha =\dfrac{4}{3}$ vào $P$, ta được $P=\dfrac{1}{6}.$
} \end{ex}

\begin{dang}
    {RÚT GỌN BIỂU THỨC}
\end{dang}

\begin{ex}%[0D6Y2-5]
Rút gọn biểu thức $M=\left({\sin x+\cos x}\right)^2+\left({\sin x-\cos x}\right)^2.$ 
\choice
{$M=1$}
{\True $M=2$}
{$M=4$}
{$M=4\sin x \cdot \cos x$}
\loigiai{ Ta có $\heva{& \left({\sin x+\cos x}\right)^2=\sin ^2x+\cos ^2x+2\sin x\cdot\cos x=1+2\sin x\cdot\cos x \\
& \left({\sin x-\cos x}\right)^2=\sin ^2x+\cos ^2x-2\sin x\cdot \cos x=1-2\sin x\cdot \cos x}$.\\
Suy ra $M=2.$
} 
\end{ex}

\begin{ex}%[0D6B2-5]
Mệnh đề nào sau đây là đúng?
\choice
{$\sin ^4x+\cos ^4x=\dfrac{1}{4}+\dfrac{3}{4}\cos 4x$}
{$\sin ^4x+\cos ^4x=\dfrac{5}{8}+\dfrac{3}{8}\cos 4x$}
{\True $\sin ^4x+\cos ^4x=\dfrac{3}{4}+\dfrac{1}{4}\cos 4x$}
{$\sin ^4x+\cos ^4x=\dfrac{1}{2}+\dfrac{1}{2}\cos 4x$}
\loigiai{ Ta có $\sin ^4x+\cos ^4x=\left({{\sin}^2x}\right)^2+2\cdot\sin ^2x\cdot\cos ^2x+\left({{\cos}^2x}\right)^2-2\cdot\sin ^2x\cdot\cos ^2x$\\
$=\left({{\sin}^2x+{\cos}^2x}\right)^2-\dfrac{1}{2}\left({2\cdot\sin x\cdot\cos x}\right)^2=1-\dfrac{1}{2}\sin ^22x=1-\dfrac{1}{2}\cdot\dfrac{{1-\cos 4x}}{2}=\dfrac{3}{4}+\dfrac{1}{4}\cos 4x.$

} 
\end{ex}

\begin{ex}%[0D6B2-5]
Mệnh đề nào sau đây là đúng?
\choice
{\True $\sin ^4x-\cos ^4x=1-2\cos ^2x$}
{$\sin ^4x-\cos ^4x=1-2\sin ^2x\cos ^2x$}
{$\sin ^4x-\cos ^4x=1-2\sin ^2x$}
{$\sin ^4x-\cos ^4x=2\cos ^2x-1$}
\loigiai{ Ta có $\sin ^4x-\cos ^4x=\left({{\sin}^2x}\right)^2-\left({{\cos}^2x}\right)^2=\left({{\sin}^2x-{\cos}^2x}\right)\left({{\sin}^2x+{\cos}^2x}\right)$\\
$=\sin ^2x-\cos ^2x=\left({1-{\cos}^2x}\right)-\cos ^2x=1-2\cos ^2x.$
} 
\end{ex}

\begin{ex}%[0D6B2-5]
Rút gọn biểu thức $M=\sin ^6x+\cos ^6x.$
\choice
{$M=1+3\sin ^2x\cos ^2x$}
{$M=1-3\sin ^2x$}
{$M=1-\dfrac{3}{2}\sin ^22x$}
{\True $M=1-\dfrac{3}{4}\sin ^22x$}
\loigiai{Ta có $M=\sin ^6x+\cos ^6x=\left({{\sin}^2x}\right)^3+\left({{\cos}^2x}\right)^3$
$=\left({{\sin}^2x+{\cos}^2x}\right)^3-3\sin ^2x\cos ^2x\left({{\sin}^2x+{\cos}^2x}\right)=1-3\sin ^2x\cos ^2x=1-\dfrac{3}{4}\sin ^22x.$

 }
 \end{ex}

\begin{ex}%[0D6K2-5]
Rút gọn biểu thức $M=2\left({{\sin}^4x+\cos^4x+ \cos^2x{\sin}^2x}\right)^2-\left({{\sin}^8x+\cos^8x}\right).$
\choice
{\True $M=1$}
{$M=-1$}
{$M=2$}
{$M=-2$}
\loigiai{Ta có 
$\sin ^4x+ \cos^4x+ \cos^2x\sin ^2x=\left({{\sin}^2x+ \cos^2x}\right)^2- \cos^2x\sin ^2x=1- \cos^2x\sin ^2x.$\\
Suy ra $M=2\left({1-{\sin}^2x{\cos}^2x}\right)^2-\left({{\sin}^8x+{\cos}^8x}\right)$
$=2\left({1-2{\sin}^2x \cos^2x+{\sin}^4x \cos^4x}\right)-\left({{\sin}^8x+ \cos^8x}\right)$\\$=2-4\sin^2x \cos^2x+2\sin^4x \cos^4x-\left({{\sin}^8x+ \cos^8x}\right)$\\$=2-4\sin^2x \cos^2x-\left({{\sin}^4x- \cos^4x}\right)^2=2-4\sin^2x\cdot \cos^2x-\left({{\sin}^2x-\cos^2x}\right)^2$\\$=2-2\sin^2x\cdot \cos^2x-\sin^4x- \cos^4x$
$=2-\left({{\sin}^2x+ \cos^2x}\right)^2=2-1=1.$
 } 
\end{ex}

\begin{ex}%[0D6B2-5]
Rút gọn biểu thức $M=\tan ^2x-\sin ^2x.$
\choice
{$M=\tan ^2x$}
{$M=\sin ^2x$}
{\True $M=\tan ^2x\cdot \sin ^2x$}
{$M=1$}
\loigiai{Ta có $M=\tan ^2x-\sin ^2x=\dfrac{{{\sin}^2x}}{{{\cos}^2x}}-\sin ^2x=\sin ^2x\left({\dfrac{1}{{{\cos}^2x}}-1}\right)=\sin ^2x\cdot \tan ^2x.$

 } 
\end{ex}

\begin{ex}%[0D6B2-5]
Rút gọn biểu thức $M=\cot ^2x-\cos ^2x.$
\choice
{$M=\cot ^2x$}
{$M=\cos ^2x$}
{$M=1$}
{\True $M=\cot ^2x \cdot \cos ^2x$}
\loigiai{Ta có $M=\cot ^2x-\cos ^2x=\dfrac{{{\cos}^2x}}{{{\sin}^2x}}-\cos ^2x=\cos ^2x\left({\dfrac{1}{{{\sin}^2x}}-1}\right)=\cos ^2x\cdot\cot ^2x.$

 } 
\end{ex}

\begin{ex}%[0D6B2-5]
Rút gọn biểu thức $M=\left({1-{\sin}^2x}\right)\cot ^2x+\left({1-{\cot}^2x}\right).$
\choice
{\True $M=\sin ^2x$}
{$M=\cos ^2x$}
{$M=\sin ^2x$}
{$M=\cos ^2x$}
\loigiai{Ta biến đổi $M=\left({{\cot}^2x-{\cos}^2x}\right)+\left({1-{\cot}^2x}\right)=1-\cos ^2x=\sin ^2x.$ } 
\end{ex}

\begin{ex}%[0D6B2-5]
Rút gọn biểu thức $M=\sin ^2\alpha \tan ^2\alpha+4\sin ^2\alpha-\tan ^2\alpha+3\cos ^2\alpha.$
\choice
{$M=1+\sin ^2\alpha$}
{$M=\sin \alpha$}
{$M=2\sin \alpha$}
{\True $M=3$}
\loigiai{ Ta có $M=\tan ^2\alpha \left({{\sin}^2\alpha-1}\right)+4\sin ^2\alpha+3\cos ^2\alpha $\\
$=\tan ^2\alpha \left({-{\cos}^2\alpha}\right)+4\sin ^2\alpha+3\cos ^2\alpha $\\
$=-\sin ^2\alpha+4\sin ^2\alpha+3\cos ^2\alpha =3\left({{\sin}^2\alpha+{\cos}^2\alpha}\right)=3.$
} 
\end{ex}

\begin{ex}%[0D6K2-5]
Rút gọn biểu thức $M=\left({{\sin}^4x+{\cos}^4x-1}\right)\left({{\tan}^2x+{\cot}^2x+2}\right).$
\choice
{$M=-4$}
{$M=-2$}
{$M=2$}
{\True $M=4$}
\loigiai{ Ta có $M=\left({1-2{\sin}^2x\cdot{\cos}^2x-1}\right)\left({\dfrac{{{\sin}^2x}}{{{\cos}^2x}}+\dfrac{{{\cos}^2x}}{{{\sin}^2x}}+2}\right)$\\
$=\left({-2{\sin}^2x\cdot{\cos}^2x}\right)\left({\dfrac{{{\sin}^4x+{\cos}^4x+2{\sin}^2x\cdot {\cos}^2x}}{{{\sin}^2x{\cos}^2x}}}\right)=\left({-2}\right)\cdot \left({{\sin}^2x+{\cos}^2x}\right)^2=-2.$

} 
\end{ex}

\begin{ex}%[0D6B2-5]
Đơn giản biểu thức $P=\sqrt{{{\sin}^4\alpha+{\sin}^2\alpha {\cos}^2\alpha}}.$
\choice
{\True $P=\left|{\sin \alpha}\right|$}
{$P=\sin \alpha$}
{$P=\cos \alpha$}
{$P=\left|{\cos \alpha}\right|$}
\loigiai{Ta có $P=\sqrt{{{\sin}^4\alpha+{\sin}^2\alpha {\cos}^2\alpha}}=\sqrt{{{\sin}^2\alpha \left({{\sin}^2\alpha+{\cos}^2\alpha}\right)}}=\sqrt{{{\sin}^2\alpha}}=\left|{\sin \alpha}\right|.$ 
 } 
\end{ex}

\begin{ex}%[0D6B2-5]
Đơn giản biểu thức $P=\dfrac{{1+{\sin}^2\alpha}}{{1-{\sin}^2\alpha}}.$
\choice
{\True $P=1+2\tan ^2\alpha$}
{$P=1-2\tan ^2\alpha$}
{$P=-1+2\tan ^2\alpha$}
{$P=-1-2\tan ^2\alpha$}
\loigiai{Ta có $P=\dfrac{{1+{\sin}^2\alpha}}{{1-{\sin}^2\alpha}}=\dfrac{{1+{\sin}^2\alpha}}{{\cos^2\alpha}}=\dfrac{1}{{\cos^2\alpha}}+\tan ^2\alpha =1+2\tan ^2\alpha.$ } \end{ex}

\begin{ex}%[0D6B2-5]
Đơn giản biểu thức $P=\dfrac{{1-\cos \alpha}}{{{\sin}^2\alpha}}-\dfrac{1}{{1+\cos \alpha}}.$
\choice
{$P=-\dfrac{{2\cos \alpha}}{{{\sin}^2\alpha}}$}
{$P=\dfrac{2}{{{\sin}^2\alpha}}$}
{$P=\dfrac{2}{{1+\cos \alpha}}$}
{\True $P=0$}
\loigiai{Ta có $P=\dfrac{{1-\cos \alpha}}{{{\sin}^2\alpha}}-\dfrac{1}{{1+\cos \alpha}}=\dfrac{{1-\cos \alpha}}{{1-{\cos}^2\alpha}}-\dfrac{1}{{1+\cos \alpha}}$\\
$=\dfrac{{1-\cos \alpha}}{{\left({1-\cos \alpha}\right)\left({1+\cos \alpha}\right)}}-\dfrac{1}{{1+\cos \alpha}}=\dfrac{1}{{1+\cos \alpha}}-\dfrac{1}{{1+\cos \alpha}}=0.$
 } 
\end{ex}

\begin{ex}%[0D6B2-5]
Đơn giản biểu thức $P=\dfrac{{1-{\sin}^2\alpha {\cos}^2\alpha}}{{{\cos}^2\alpha}}-\cos ^2\alpha.$
\choice
{\True $P=\tan ^2\alpha$}
{$P=1$}
{$P=-\cos ^2\alpha$}
{$P=\cot ^2\alpha$}
\loigiai{ Ta có $P=\dfrac{{1-{\sin}^2\alpha {\cos}^2\alpha-{\cos}^4\alpha}}{{{\cos}^2\alpha}}=\dfrac{{1-{\cos}^2\alpha \left({{\sin}^2\alpha+{\cos}^2\alpha}\right)}}{{{\cos}^2\alpha}}$\\
$=\dfrac{{1-{\cos}^2\alpha}}{{{\cos}^2\alpha}}=\dfrac{{{\sin}^2\alpha}}{{{\cos}^2\alpha}}=\tan ^2\alpha.$
} 
\end{ex}

\begin{ex}%[0D6B2-5]
Đơn giản biểu thức $P=\dfrac{{2{\cos}^2x-1}}{{\sin x+\cos x}}.$
\choice
{$P=\cos x+\sin x$}
{\True $P=\cos x-\sin x$}
{$P=\cos 2x-\sin 2x$}
{$P=\cos 2x+\sin 2x$}
\loigiai{Ta có $P=\dfrac{{2{\cos}^2x-\left({{\sin}^2x+{\cos}^2x}\right)}}{{\sin x+\cos x}}=\dfrac{{{\cos}^2x-{\sin}^2x}}{{\sin x+\cos x}}=\cos x-\sin x.$ } 
\end{ex}

\begin{ex}%[0D6B2-5]
Đơn giản biểu thức $P=\dfrac{{{\left({\sin \alpha+\cos \alpha}\right)}^2-1}}{{\cot \alpha-\sin \alpha \cos\alpha}}.$
\choice
{\True $P=2\tan ^2\alpha$}
{$P=\dfrac{{\sin \alpha}}{{{\cos}^3\alpha}}$}
{$P=2\cot ^2\alpha$}
{$P=\dfrac{2}{{{\cos}^2\alpha}}$}
\loigiai{ Ta có $P=\dfrac{{{\left({\sin \alpha+\cos \alpha}\right)}^2-1}}{{\cot \alpha-\sin \alpha \cos \alpha}}=\dfrac{{{\sin}^2\alpha+2\sin \alpha\cdot \cos \alpha+{\cos}^2\alpha-1}}{{\cos \alpha\cdot \left({\dfrac{1}{{\sin \alpha}}-\sin \alpha}\right)}}.$\\
$=\dfrac{{1+2\sin \alpha \cdot \cos \alpha-1}}{{\cos \alpha\cdot \dfrac{{1-{\sin}^2\alpha}}{{\sin \alpha}}}}=\dfrac{{2\sin \alpha \cdot \cos \alpha}}{{\dfrac{{{\cos}^3\alpha}}{{\sin \alpha}}}}=\dfrac{{2{\sin}^2\alpha}}{{{\cos}^2\alpha}}=2\tan ^2\alpha.$
} 
\end{ex}

\begin{ex}%[0D6B2-5]
Đơn giản biểu thức $P=\left({\dfrac{{\sin \alpha+\tan \alpha}}{{\cos \alpha+1}}}\right)^2+1.$
\choice
{$P=2$}
{$P=1+\tan \alpha$}
{\True $P=\dfrac{1}{{{\cos}^2\alpha}}$}
{$P=\dfrac{1}{{{\sin}^2\alpha}}$}
\loigiai{Ta có $\dfrac{{\sin \alpha+\tan \alpha}}{{\cos \alpha+1}}=\dfrac{{\sin \alpha \left({1+\dfrac{1}{{\cos \alpha}}}\right)}}{{\cos \alpha+1}}=\dfrac{{\sin \alpha \left({\dfrac{{\cos \alpha+1}}{{\cos \alpha}}}\right)}}{{\cos \alpha+1}}=\dfrac{{\sin \alpha}}{{\cos \alpha}}=\tan \alpha.$\\
Suy ra $P=\tan ^2\alpha+1=\dfrac{1}{{{\cos}^2\alpha}}.$
 } 
\end{ex}

\begin{ex}%[0D6K2-5]
Đơn giản biểu thức $P=\tan \alpha \left({\dfrac{{1+{\cos}^2\alpha}}{{\sin \alpha}}-\sin \alpha}\right).$
\choice
{$P=2$}
{\True $P=2\cos \alpha$}
{$P=2\tan \alpha$}
{$P=2\sin \alpha$}
\loigiai{Ta có $P=\tan \alpha \left({\dfrac{{1+{\cos}^2\alpha}}{{\sin \alpha}}-\sin \alpha}\right)=\dfrac{{\sin \alpha}}{{\cos \alpha}}\left({\dfrac{1}{{\sin \alpha}}+\dfrac{{{\cos}^2\alpha}}{{\sin \alpha}}-\sin \alpha}\right)$\\
$=\dfrac{1}{{\cos \alpha}}+\cos \alpha-\dfrac{{{\sin}^2\alpha}}{{\cos \alpha}}=\dfrac{{1+{\cos}^2\alpha-{\sin}^2\alpha}}{{\cos \alpha}}=\dfrac{{\left({1-{\sin}^2\alpha}\right)+{\cos}^2\alpha}}{{\cos \alpha}}=\dfrac{{2{\cos}^2\alpha}}{{\cos \alpha}}=2\cos \alpha.$

 } 
\end{ex}

\begin{ex}%[0D6B2-5]
Đơn giản biểu thức $P=\dfrac{{{\cot}^2x-\cos^2x}}{{{\cot}^2x}}+\dfrac{{\sin x \cos x}}{{\cot x}}.$
\choice
{\True $P=1$}
{$P=-1$}
{$P=\dfrac{1}{2}$}
{$P=-\dfrac{1}{2}$}
\loigiai{ Ta có $\dfrac{{{\cot}^2x- \cos^2x}}{{{\cot}^2x}}=1-\dfrac{{{\cos}^2x}}{{{\cot}^2x}}=1-\cos ^2x\cdot \dfrac{{{\sin}^2x}}{{{\cos}^2x}}=1-\sin ^2x.$\\
Và $\dfrac{{\sin x\cdot \cos x}}{{\cot x}}=\sin x\cdot \cos x\cdot \dfrac{{\sin x}}{{\cos x}}=\sin ^2x$.

} 
\end{ex}

\begin{ex}%[0D6K2-5]
Hệ thức nào sau đây là {\bf SAI}? 
\choice
{$\dfrac{{{\sin}^2\alpha+1}}{{2\left({1-{\sin}^2\alpha}\right)}}+\dfrac{{1+\cos^2\alpha}}{{2\left({1-\cos^2\alpha}\right)}}+1=\left({\tan \alpha+\cot \alpha}\right)^2$}
{$\dfrac{{1-4{\sin}^2x\cdot \cos^2x}}{{4{\sin}^2x\cdot \cos^2x}}=\dfrac{{1+{\tan}^4x-2{\tan}^2x}}{{4{\tan}^2x}}$}
{\True $\dfrac{{\sin x+\tan x}}{{\tan x}}=1+\sin x+\cot x$}
{$\tan x+\dfrac{{\cos x}}{{1+\sin x}}=\dfrac{1}{{\cos x}}$}
\loigiai{ Ta có $\dfrac{{\sin x+\tan x}}{{\tan x}}=\dfrac{{\sin x}}{{\tan x}}+1=\sin x\cdot \dfrac{{\cos x}}{{\sin x}}+1=1+\cos x\ne 1+\sin x+\cot x.$} 
\end{ex}
% \Closesolutionfile{ans}      
% \begin{center}
% \bf ĐÁP ÁN
% \end{center}
% \begin{multicols}{10}
% \input{ans/ans10D1-TN-2}
% \end{multicols}