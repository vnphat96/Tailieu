\section*{Đề kiểm tra Chương 5}
\subsection*{Đề số 2}
\setcounter{ex}{0}\setcounter{bt}{0}
\Opensolutionfile{ans}[ans/ans-KT-502]
\noindent\textbf{I. PHẦN TRẮC NGHIỆM}
\begin{ex}%[Dự án BG10-2022 - Đề KT]%[Đỗ Viết Lân]%[0D1Y5-1]
	Cho số $a$ là số gần đúng của số $\overline{a}$. Mệnh đề nào sau đây là mệnh đề đúng?
	\choice
	{$a>\overline{a}$}               
	{$a<\overline{a}$}
	{\True $|\overline{a}-a|>0$}
	{$-a<\overline{a}<a$}
	\loigiai{
		Do $a$ là số gần đúng của số $\overline{a}$ nên $a>\overline{a}$ hoặc $a<\overline{a}$.\\
		Suy ra $|\overline{a}-a|>0$.
	}
\end{ex}

\begin{ex}%[Dự án BG10-2022 - Đề KT]%[Đỗ Viết Lân]%[0D1B5-2]
	Nếu lấy $3{,}14$ làm giá trị gần đúng cho số $\pi$ thì sai số tuyệt đối không vượt quá  
	\choice{\True $ 0{,}01$}
	{$0{,}02$}
	{$0{,}03$}
	{$0{,}04$}
	\loigiai{
		Ta có $\pi = 3{,}141592654...$.\\
		Do $3{,}14 < \pi = 3{,}141592654... < 3{,}15$ nên ta có $\Delta = \left|\pi - 3{,}14 \right| < |3{,}15 - 3{,}14| = 0{,}01$.
	}
\end{ex}

\begin{ex}%[Dự án BG10-2022 - Đề KT]%[Đỗ Viết Lân]%[0D1B5-2]
	Cho giá trị gần đúng của $\dfrac{8}{17}$ là $0{,}47$ thì sai số tuyệt đối không vượt quá 
	\choice{\True $0{,}01$}
	{$0{,}02$}
	{$0{,}03$}
	{$0{,}04$}
	\loigiai{
		Ta có $\dfrac{8}{17} = 0{,}4705882...$. Do $0{,}47 < \dfrac{8}{17} = 0{,}4705882... < 0{,}48$ nên 
		\begin{equation*}
			\Delta = \left|\dfrac{8}{17} - 0{,}47 \right| < |0{,}48 - 0{,}47| = 0{,}01. 
		\end{equation*}
	}
\end{ex}

\begin{ex}%[Dự án BG10-2022 - Đề KT]%[Đỗ Viết Lân]%[0D1K5-2]
	Các nhà toán học cổ đại Trung Quốc đã dùng phân số $\dfrac{22}{7}$ để xấp xỉ số $\pi$. Hãy đánh giá sai số tuyệt đối $\Delta$ của giá trị gần đúng này, biết $3{,}1415<\pi<3{,}1416$.
	\choice{$\Delta<0{,}0012$}
	{\True$\Delta<0{,}0014$}
	{$\Delta<0{,}0013$}
	{$\Delta<0{,}0011$}
	\loigiai{Ta có $\dfrac{22}{7}<3{,}1429$ và $-\pi<-3{,}1415$ nên $\Delta=\left|\pi-\dfrac{22}{7}\right|=\dfrac{22}{7}-\pi<3{,}1429-3{,}1415=0{,}0014$.}
\end{ex}

\begin{ex}%[Dự án BG10-2022 - Đề KT]%[Đỗ Viết Lân]%[0D1K5-2]
	Cho $\overline{a} = \dfrac{1}{1+x}$ $(0 < x < 1)$. Giả sử ta lấy $a = 1- x$ làm giá trị gần đúng của $\overline{a}$. Khi đó, sai số tương đối của $a$ theo $x$ bằng
	\choice{\True $\dfrac{x^2}{1- x^2}$}
	{$\dfrac{x}{1-x}$}
	{$\dfrac{x^2}{1-x}$}
	{$\dfrac{x}{1-x^2}$}
	\loigiai{Ta có $\Delta_{\overline{a}} = \left|\dfrac{1}{1+x} - (1-x)\right| = \dfrac{x^2}{1+x}$.
		Vậy sai số tương đối là $\delta_a = \dfrac{\Delta_{\overline{a}}}{|1-x|} = \dfrac{x^2}{1- x^2}.$
	}
\end{ex}

\begin{ex}%[Dự án BG10-2022 - Đề KT]%[Đỗ Viết Lân]%[0D1K5-2]
	Hình chữ nhật có các cạnh là $x = 2 \mathrm{\, m} \pm 1 \mathrm{\, cm}$ và $y = 5 \mathrm{\, m} \pm 2 \mathrm{\, cm}$. Diện tích của hình chữ nhật và sai số tương đối của giá trị đó là
	\choice
	{$10 \mathrm{\, m}^2$ và $\delta \leq 0{,}91 \%$}
	{\True $10 \mathrm{\, m}^2$ và $\delta \leq 0{,}9 \%$}
	{$10 \mathrm{\, m}^2$ và $\delta \leq 0{,}92 \%$}
	{$10 \mathrm{\, m}^2$ và $\delta \leq 0{,}93 \%$}
	\loigiai{Ta có $1{,}99 \mathrm{\, m} \leq x \leq 2{,}01 \mathrm{\, m}$ và $4{,}98 \mathrm{\, m} \leq y \leq 5{,} 02 \mathrm{\, m}$.
		\noindent Khi đó $S = x \cdot y \Rightarrow 9{,}9102\mathrm{\, m}^2 \leq S \leq 10{,}0902 \mathrm{\, m}^2 \Rightarrow S = 10 \mathrm{\, m}^2 \pm 0{,}0902 \mathrm{\, m}^2$.
		\noindent Sai số tương đối là $\delta_S \leq \dfrac{0{,}0902}{10} \approx 0{,}9 \%$.
	}
\end{ex}

%dạng 2

\begin{ex}%[Dự án BG10-2022 - Đề KT]%[Đỗ Viết Lân]%[0D1Y5-2]
Chiều cao của một ngọn đồi là $\overline{h}=347{,}13\ \mathrm{m}\pm 0{,}2\ \mathrm{m}$. Độ chính xác của phép đo trên là 
\choice
{$d=347{,}33$ m}
{\True $d=0{,}2$ m}
{$d=347{,}13$ m}
{$d=346{,}93$}
\loigiai
{Độ chính xác của phép đo trên là $d=0{,}2$ m.
}
\end{ex}

\begin{ex}%[Dự án BG10-2022 - Đề KT]%[Đỗ Viết Lân]%[0D1Y5-2]
	Bạn Hương Giang vừa thi đậu vào lớp $ 10 $ năm học $ 2020 - 2021 $, ba mẹ của bạn thưởng cho bạn một chiếc laptop. Khi mang về bạn phát hiện ngoài bao bì có ghi trọng lượng $ 1{,}5456\pm 0{,}001 $ kg. Giá trị quy tròn trọng lượng của chiếc laptop đó là
	\choice
	{$1{,}545$ kg}
	{$1{,}54 $ kg}
	{$ 1{,}546$ kg}
	{\True $ 1{,}55 $ kg}
	\loigiai{Chữ số hàng quy tròn là chữ số $ 4 $.\\
	Chữ số đứng sau hàng quy tròn là chữ số $ 5 $. Vậy trọng lượng lượng của chiếc laptop sau khi quy tròn là $ 1{,}55 $ kg.}
\end{ex}

%dạng 3

\begin{ex}%[Dự án BG10-2022 - Đề KT]%[Đỗ Viết Lân]%[0D1Y5-1]
	Kết quả làm tròn số $b=500\sqrt{7}$ đến chữ số thập phân thứ hai là
	\choice
	{\True $b\approx 132,88$}
	{$b\approx 1322,87$}
	{$b\approx 1322,8 $}
	{$b\approx 1322,9 $}
	\loigiai{
		Có $b=500\sqrt{7}\approx 1322{,}875656$.\\
		Do làm tròn đến chữ số thập phân thứ hai nên ta có $b\approx 1322{,}88$.
	}
\end{ex}

\begin{ex}%[Dự án BG10-2022 - Đề KT]%[Đỗ Viết Lân]%[0D1Y5-1]
	Kết quả làm tròn của số $c=76324753{,}3695$ đến  hàng nghìn là
	\choice
	{$c\approx 76324000$}
	{\True $c\approx 76325000$}
	{$c\approx 76324753{,}369$}
	{$c\approx 76324753{,}37$}
	\loigiai{
		Làm tròn đến hàng nghìn của $c=76324753{,}3695$ ta được $c\approx 76325000$.
	}
\end{ex}

\begin{ex}%[Dự án BG10-2022 - Đề KT]%[Đỗ Viết Lân]%[0D1Y5-1]
	Viết số quy tròn của số gần đúng $a=505360{,}996$ biết $\overline{a}=505360{,}996\pm 100$.
	\choice
	{$a\approx 505$}
	{$a\approx 5054$}
	{$a\approx 505400$}
	{\True $a\approx 505000$}
	\loigiai{
		Do $\overline{a}=505360{,}996\pm 100$ nên ta làm tròn đến hàng nghìn.\\
		Suy ra $a\approx 505000$.
	}
\end{ex}

\begin{ex}%[Dự án BG10-2022 - Đề KT]%[Đỗ Viết Lân]%[0D1Y5-1]
	Viết số quy tròn số gần đúng $b=3257{,}6254$ với độ chính xác $d=0{,}01$.
	\choice
	{$b\approx 3257{,}63$}
	{$b\approx 3257{,}62$}
	{\True $b\approx 3257{,}6$}
	{$b\approx 3257{,}7$}
	\loigiai{
		Do $d=0,01$ nên ta làm tròn đến hàng phần đơn vị. Do đó $b\approx 3257{,}6$.
	}
\end{ex}

%xu the trung tam

\begin{ex}%[Dự án BG10-2022 - Đề KT]%[Đỗ Viết Lân]%[0D5B3-1]
	Tuổi đời của $16$ công nhân trong xưởng sản xuất được thống kê trong bảng sau
	\begin{center}
		\begin{tabular}{|c|c|c|c|c|c|c|c|}
		\hline 
		Tuổi & $25$ & $26$ & $27$ & $29$ & $30$ & $33$ & Cộng \\
		\hline
		Số người & $2$ & $3$ & $4$ & $3$ & $3$ & $1$ & $16$ \\ 
		\hline
		\end{tabular}
	\end{center}
Tìm số trung bình $\overline{x}$ của mẫu số liệu trên
	\choice
	{$28$}
	{$27{,}75$}
	{\True $27{,}875$}
	{$27$}
	\loigiai
	{
		Mẫu số liệu có số trung bình là $\overline{x}=\dfrac{25\cdot 2+26\cdot 3+27\cdot 4+29\cdot 3+30\cdot 3+33\cdot 1}{16}=27{,}875$.
	}
\end{ex}

\begin{ex}%[Dự án BG10-2022 - Đề KT]%[Đỗ Viết Lân]%[0-HK2-2021, Đề minh họa Bộ Giáo dục, 2020-2021]%[Kiều Ngân]%[0D5B3-1]
	Điểm kiểm tra môn Toán cuối năm của một nhóm gồm $9$ học sinh lớp $6$ lần lượt là
	$$1;1;3;6;7;8;8;9;10.$$
	Điểm trung bình của cả nhóm gần nhất với số nào dưới đây?
	\choice
	{$7{,}$}
	{$7$}
	{$6{,}5$}
	{\True $5{,}9$}
	\loigiai{
		Điểm trung bình môn Toán của cả nhóm là
		$$\overline{x}=\dfrac{1+1+3+6+7+8+8+9+10}{9}=\dfrac{53}{9}\approx 5{,}9.$$
	}
\end{ex}

\begin{ex}%[Dự án BG10-2022 - Đề KT]%[Đỗ Viết Lân]%[0D5B3-1]
Ba nhóm học sinh gồm $10$ người, $15$ người, $25$ người. Khối lượng trung bình của mỗi nhóm lần lượt là $50$ kg, $38$ kg, $42$ kg. Khối lượng trung bình của cả ba nhóm học sinh là
\choice
{$39$ kg}
{\True $42{,}4$ kg}
{$41{,}4$ kg}
{$43{,}4$ kg}
\loigiai{Khối lượng trung bình của cả ba nhóm học sinh là
\[\overline{m}=\dfrac{10\cdot 50+15\cdot 38+25\cdot 42}{10+15+25 }=42{,}4.\]
}
\end{ex}

\begin{ex}%[Dự án BG10-2022 - Đề KT]%[Đỗ Viết Lân]%[0D5Y3-2]
Cho bảng số liệu điểm bài kiểm tra môn toán của $20$ học sinh
\begin{center}
	\begin{tabular}{|c|c|c|c|c|c|c|c|c|}
		\hline 
		Điểm	&4	&5	&6	&7	&8	&9 &10 &Cộng\\
		\hline 
		Số học sinh	&1	&2	&3	&4	&5	&4 &1 &20 \\
		\hline 
	\end{tabular}
\end{center}
Tìm số trung vị của bảng số liệu trên
\choice
{$8$}
{\True $7{,}5$}
{$7{,}3$}
{$7$}
\loigiai{
	Số trung vị của bảng trên là $M_e=\dfrac{7+8}{2}=7{,}5$.
}
\end{ex}

\begin{ex}%[Dự án BG10-2022 - Đề KT]%[Đỗ Viết Lân]%[0-HK2-2021, THPT Thông Huề - Cao Bằng, 2020-2021]%[Lê Minh Thiện Anh]%[0D5Y3-2]
Tiền lương tháng của $7$ công nhân trong một công ty lần lượt là
\begin{center}
$\quad 6{,}5 \quad 8{,}5 \quad 9{,}0 \quad 7{,}2 \quad 5{,}5 \quad 7{,}0 \quad 6{,}0 \quad$
\textit{(đơn vị triệu đồng)}
\end{center}
Số trung vị của dãy số liệu thống kê trên bằng
	\choice
	{\True $7{,}0$ triệu đồng}
	{$7{,}2$ triệu đồng}
	{$8{,}5$ triệu đồng}
	{$9{,}0$ triệu đồng}
	\loigiai{
		Số liệu thống kê được sắp xếp lại như sau: $\quad 5{,}5 \quad 6{,}0 \quad 6{,}5 \quad 7{,}0 \quad 7{,}2 \quad 8{,}5 \quad 9{,}0$.\\
		Số trung vị của dãy số liệu thống kê trên là $M_e=7{,}0$.		
	}
\end{ex}

\begin{ex}%[Dự án BG10-2022 - Đề KT]%[Đỗ Viết Lân]%[0D5Y3-2]
Cho mẫu số liệu thống kê $\left\lbrace 52,49,54,50,51,49,55,53,54\right\rbrace$. Số trung vị của mẫu số liệu trên là bao nhiêu?	
	\choice
	{\True $52$}
	{$53$}
	{$54$}
	{$51$}
	\loigiai{
Ta sắp xếp các giá trị như sau $49,49,50,51,52,53,54,54,55$. Vậy số trung vị là $52$.		
	}
\end{ex}

\begin{ex}%[Dự án BG10-2022 - Đề KT]%[Đỗ Viết Lân]%[0-HK2-2021, THPT Thông Huề - Cao Bằng, 2020-2021]%[Lê Minh Thiện Anh]%[0D5Y3-3]
Kết quả điểm kiểm tra môn Toán của $40$ học sinh lớp $10 \mathrm{A}$ được trình bày ở bảng sau
\begin{center}
\begin{tabular}{|c|c|c|c|c|c|c|c|c|}
	\hline Điểm & 4 & 5 & 6 & 7 & 8 & 9 & 10 & Cộng \\
	\hline Số học sinh & 3 & 8 & 7 & 9 & 8 & 3 & 2 & 40 \\
	\hline
\end{tabular}
\end{center}
Giá trị mốt của bảng phân bố tần số trên bằng
	\choice
	{$10$}
	{\True $7$}
	{$8$}
	{$9$}
	\loigiai{
		Giá trị mốt của bảng phân bố tần số trên là $M_0=7$.	
	}
\end{ex}

\begin{ex}%[Dự án BG10-2022 - Đề KT]%[Đỗ Viết Lân]%[0-HK2-2021,THPT Phạm Văn Đồng - Quảng Ngãi, 2021-2022]%[Huỳnh Đức Vũ]%[0D5Y3-3]
Khi điều tra “Năng suất lúa hè thu năm $2021$ của $11$ huyện tại tỉnh Quảng Ngãi, người ta thu thập được bảng phân bố tần số sau
\begin{center}
	\begin{tabular}{|c|c|}
		\hline
		Năng suất lúa (tạ/ha) & Số huyện \\
		\hline
		$15$ & $2$ \\
		\hline
		$20$ & $3$ \\
		\hline
		$25$ & $4$ \\
		\hline
		$30$ & $2$ \\
		\hline
		Cộng & $11$ \\
		\hline
	\end{tabular}
\end{center}
Mốt $M_O$ của bảng phân bố tần số trên là
\choice
{$M_O=20$}
{\True $M_O=25$}
{$M_O=4$}
{$M_O=30$}
\loigiai{Mốt $M_O$ của bảng phân bố tần số trên là $25$.}
\end{ex}

\begin{ex}%[Dự án BG10-2022 - Đề KT]%[Đỗ Viết Lân]%[0-HK2-2021, THPT Lê Hồng Phong - Thái Nguyên, 2020-2021]%[Hoàng Anh]%[0D5Y3-3]
	Số áo bán được trong một quý ở của hàng bán áo sơ mi nam được thống kê như sau 
	\begin{center}
		\begin{tabular}{|c|c|c|c|c|c|c|c|}
			\hline Cỡ áo & 36 & 37 & 38 & 39 & 40 & 41 & 42 \\
			\hline Tần số (Số áo bán được) & 13 & 45 & 126 & 125 & 110 & 40 & 12 \\
			\hline
		\end{tabular}
	\end{center}
	Giá trị mốt của bảng phân bố tần số trên bằng
	\choice
	{$12$}
	{$42$}
	{\True $38$}
	{$126$}
	\loigiai{Mốt của mẫu số liệu là giá trị có tần số lớn nhất.\\
		Trong trường hợp này, $38$ là giá trị có tần số lớn nhất nên là mốt.}
\end{ex}

\begin{ex}%[Dự án BG10-2022 - Đề KT]%[Đỗ Viết Lân]%[0D5Y3-5]
Có bao nhiêu giá trị của mẫu số liệu nằm giữa $Q_1$ và và số trung vị?
\choice
{\True $25\%$}
{$50\%$}
{$75\%$}
{$100\%$}
\loigiai{
Theo lý thuyết có $25\%$ giá trị của mẫu số liệu nằm giữa $Q_1$ và số trung vị.
}
\end{ex}

\begin{ex}%[Dự án BG10-2022 - Đề KT]%[Đỗ Viết Lân]%[0D5Y3-5]
	Cho mẫu số liệu $5 ; 13 ; 5 ; 7 ; 10 ; 2 ; 3$. Tứ phân vị thứ nhất, thứ hai, thứ ba lần lượt là
	\choice
	{\True $3;5;10$}
	{$5;3;10$}
	{$10;3;5$}
	{$10;5;3$}
	\loigiai{
		Sắp xếp lại mẫu số liệu theo thứ tự không giảm, ta được: $2 ; 3 ; 5 ; 5 ; 7 ; 10 ; 13$.
		\begin{itemize}
			\item Vì cỡ mẫu là $n=7$, là số lẻ, nên giá trị tứ phân vị thứ hai là $Q_{2}=5$.
			\item Tứ phân vị thứ nhất là trung vị của mẫu: $2 ; 3 ; 5$. Do đó $Q_{1}=3$.
			\item Tứ phân vị thứ ba là trung vị của mẫu: $7 ; 10 ; 13$. Do đó $Q_{3}=10$.
		\end{itemize}
	}	
	
\end{ex}
\begin{ex}%[Dự án BG10-2022 - Đề KT]%[Đỗ Viết Lân]%[0D5Y3-5]
	Cho mẫu số liệu  $2 ; 3 ; 10 ; 13 ; 5 ; 15 ; 5 ; 7$. Tứ phân vị thứ nhất, thứ hai, thứ ba lần lượt là
	\choice
	{ $11,5; \,6; \,4$}
	{\True $4; \,6; \,11,5$}
	{$6; \,4; \,11,5$}
	{$6; \,11,5;\, 4$}
	\loigiai{
		Sắp xếp lại mẫu số liệu theo thứ tự không giảm, ta được: $2 ; 3 ; 5 ; 5 ; 7 ; 10 ; 13$.
		\begin{itemize}
			\item Vì cỡ mẫu là $n=8$, là số chẵn, nên giá trị tứ phân vị thứ hai là
			$$
			Q_{2}=\dfrac{1}{2}(5+7)=6.
			$$
			\item Tứ phân vị thứ nhất là trung vị của mẫu: $2 ; 3 ; 5 ; 5$. Do đó $Q_{1}=4$.
			\item Tứ phân vị thứ ba là trung vị của mẫu: $7 ; 10 ; 13 ; 15$. Do đó $Q_{3}=11,5$.
		\end{itemize}
	}	
	
\end{ex}

\begin{ex}%[Dự án BG10-2022 - Đề KT]%[Đỗ Viết Lân]%[0D5Y3-6]
	Hai chữ số cuối giải đặc biệt Xổ số miền Bắc trong $9$ ngày được ghi lại như sau:
	\begin{longtable}{p{1cm}p{1cm}p{1cm}p{1cm}p{1cm}p{1cm}p{1cm}p{1cm}p{1cm}}
		16 & 11 & 25 & 28 & 45 & 42 & 24 & 33 & 11
	\end{longtable}
	Hãy tìm khoảng biến thiên của mẫu số liệu trên.
	\choice
	{$18$}
	{\True $34$}
	{$56$}
	{$27$}
	\loigiai{
		Khoảng biến thiên của mẫu số liệu là $R=45-11=34$.
	}
\end{ex}

\begin{ex}%[Dự án BG10-2022 - Đề KT]%[Đỗ Viết Lân]%[0D5Y3-6]
	Mẫu số liệu nào dưới đây có khoảng biến thiên là $53$?
	\choice
	{$18$, $57$, $11$, $26$}
	{\True $44$, $2$, $55$, $46$, $27$}
	{$21$, $3$, $55$, $89$}
	{$4$, $16$, $23$, $20$}
	\loigiai{
		Khoảng biến thiên của các mẫu số liệu lần lượt là
		\begin{itemize}
			\item $R_1=57-11=46$.
			\item $R_2=55-2=53$.
			\item $R_3=89-3=86$.
			\item $R_4=23-4=19$.
		\end{itemize}
	}
\end{ex}

\begin{ex}%[Dự án BG10-2022 - Đề KT]%[Đỗ Viết Lân]%[0D5Y3-6]
	Trong một tuần, nhiệt độ cao nhất trong ngày (đơn vị $^\circ$C) tại hai thành phố Hà Nội và TP Hồ Chí Minh được cho như sau:
	\begin{longtable}{p{3.5cm}p{1cm}p{1cm}p{1cm}p{1cm}p{1cm}p{1cm}p{1cm}p{1cm}}
		Hà Nội: & 28 & 27 & 30 & 29 & 27 & 24 & 25\\
		TP Hồ Chí Minh: & 31 & 33 & 32 & 33 & 29 & 32 & 34\\
	\end{longtable}
	Dựa vào khoảng biến thiên của hai mẫu số liệu, hãy chỉ ra mẫu số liệu nào có độ phân tán lớn hơn.
	\choice
	{\True Mẫu số liệu ``Hà Nội'' có độ phân tán lớn hơn mẫu số liệu ``TP Hồ Chí Minh''}
	{Mẫu số liệu ``TP Hồ Chí Minh'' có độ phân tán lớn hơn mẫu số liệu ``Hà Nội''}
	{Hai mẫu số liệu có độ phân tán bằng nhau}
	{Tất cả đều sai}
	\loigiai{
		Mẫu số liệu ``Hà Nội'' có giá trị lớn nhất và giá trị nhỏ nhất lần lượt là $ 30 $ và $ 24 $.\\
		Do đó khoảng biến thiên của mẫu số liệu đã cho là $ R_1=30-24=6 $.\\
		Mẫu số liệu ``TP Hồ Chí Minh'' có giá trị lớn nhất và giá trị nhỏ nhất lần lượt là $ 33 $ và $ 29 $.\\
		Do đó khoảng biến thiên của mẫu số liệu đã cho là $ R_2=33-29=4 $.\\
		Do $R_1>R_2$ nên mẫu số liệu ``Hà Nội'' có độ phân tán lớn hơn mẫu số liệu ``TP Hồ Chí Minh''.	
	}
\end{ex}

\begin{ex}%[Dự án BG10-2022 - Đề KT]%[Đỗ Viết Lân]%[0D5Y3-7]
	Cho mẫu số liệu $21 ; 35 ; 17 ; 43 ; 8 ;59 ;72 ; 119$. Tìm khoảng tứ phân vị của mẫu số liệu.
	\choice
	{ \True $46{,}5$}
	{ $39{,}5$}
	{$26{,}5$}
	{$22{,}5$}
	\loigiai{
		Sắp xếp lại mẫu số liệu theo thứ tự không giảm, ta được: $8 ; 17 ; 21 ; 35 ; 43 ; 59 ; 72 ; 119$.
		\begin{itemize}
			\item Vì cỡ mẫu là $n=8$, là số chẵn, nên giá trị tứ phân vị thứ hai là
			$$
			Q_{2}=\dfrac{1}{2}(35+43)=39.
			$$
			\item Tứ phân vị thứ nhất là trung vị của mẫu: $8 ; 17 ; 21 ; 35$. Do đó $Q_{1}=19$.
			\item Tứ phân vị thứ ba là trung vị của mẫu:  $43 ; 59 ; 72 ; 119$. Do đó $Q_{3}=65{,}5$.
		\end{itemize}
		Khoảng tứ phân vị là $\Delta_Q = Q_3 - Q_1 = 46{,}5$
	}	
\end{ex}

\begin{ex}%[Dự án BG10-2022 - Đề KT]%[Đỗ Viết Lân]%[0D5B4-1]
	Cho dãy số liệu thống kê: $ 1 $, $ 2 $, $ 3 $, $ 4 $, $ 5 $, $ 6 $, $ 7 $, $ 8 $. Phương sai của dãy số liệu này xấp xỉ bằng
	\choice
	{$2{, }3$}
	{\True $5{, }25$}
	{$3{, }3$}
	{$5{, }3$}
	\loigiai{
		Số trung bình của dãy số liệu trên là $ \overline x =\dfrac{1+2+3+4+5+6+7+8}{8}=4{, }5 $.\\
		Phương sai của dãy số liệu là
		\[ s^2=\dfrac{(4{, }5-1)^2+(4{, }5-2)^2+(4{, }5-3)^2+(4{, }5-4)^2+(4{, }5-5)^2+(4{, }5-6)^2+(4{, }5-7)^2+(4{, }5-8)^2}{8}=5{, }25. \]
	}
\end{ex}

\begin{ex}%[Dự án BG10-2022 - Đề KT]%[Đỗ Viết Lân]%[0D5B4-1]%[]
	Số ôtô đi qua một cây cầu trong một tuần đếm được như sau: $83$; $74$; $71$; $79$; $83$; $69$; $92$. Phương sai và độ lệch chuẩn lần lượt là
	\choice
	{$78{,}71 - 8{,}87$}
	{$52{,}99 - 7{,}28$}
	{$61{,}82 - 7{,}86$}
	{\True $55{,}63 - 7{,}46$}
	\loigiai{	
		Ta có
		$$\overline{x}=\dfrac{1}{7}(69+71+74+79+83\cdot2+92)\approx 78{,}7.$$
		Phương sai
		\begin{eqnarray*}
			s^{2}&=& \dfrac{1}{7}\left[(69-78{,}7)^{2}+ (71-78{,}7)^{2}+(74-78{,}7)^{2}+(79-78{,}7)^{2}+2\cdot(83-78{,}7)^{2}+(92-78{,}7)^{2}\right]\\
			&\approx & 55{,}63.
		\end{eqnarray*}
		Độ lệch chuẩn $s=\sqrt{s^2}\approx 7{,}46$.
		
	}
\end{ex}

\begin{ex}%[Dự án BG10-2022 - Đề KT]%[Đỗ Viết Lân]%[0D5B4-1]
	Cho dãy số liệu thống kê: $1,4,7,8$. Độ lệch chuẩn của dãy số liệu thống kê này (\textit{làm tròn đến một chữ số thập phân}) là
	\choice
	{$5,1$}
	{$7,5$}
	{$2,8$}
	{\True $2,7$}
	\loigiai{
		Ta có $\overline{x} =\dfrac{1+4+7+8}{4}=5$ và $s^2=\dfrac{(1-5)^2+(4-5)^2+(7-5)^2+(8-5)^2}{4} =\dfrac{29}{4}$.\\
		Suy ra độ lện chuẩn là $s=\dfrac{\sqrt{29}}{4}\approx 2,7$.
	}
\end{ex}

\begin{ex}%[Dự án BG10-2022 - Đề KT]%[Đỗ Viết Lân]%[0D5Y4-1]
	Sản lượng lúa (đơn vị là tạ) của 40 thửa ruộng thí nghiệm có cùng diện tích được trình bày trong bảng tần số sau đây.
	\begin{center}
		\begin{tabular}{|>{\centering\arraybackslash}p{4cm}|>{\centering\arraybackslash}p{1.5cm}|>{\centering\arraybackslash}p{1.5cm}|>{\centering\arraybackslash}p{1.5cm}|>{\centering\arraybackslash}p{1.5cm}|>{\centering\arraybackslash}p{1.5cm}|>{\centering\arraybackslash}p{2cm}|}
			\hline  {\bf Sản lượng}&$20$  &$21$  &$22$  &$23$  &$24$  &  \\ 
			\hline  {\bf Số thửa ruộng}&$5$  &$8$  &$11$  &$10$  &$6$  & $N=40$  \\ 
			\hline 
		\end{tabular} 
	\end{center}
	Tính độ lệch chuẩn.
	\choice
	{$s\approx 1,34$ (tạ)}
	{\True $s\approx 1,24$ (tạ)}
	{$s\approx 1,54$ (tạ)}
	{$s\approx 1,64$ (tạ)}
	\loigiai{
		Giá trị trung bình $\overline{x}=\dfrac{20\cdot 5+21\cdot 8+22\cdot 11+23\cdot 10+24\cdot 6}{40}=22{,}1$.\\
		Phương sai \\
		$S^2=\dfrac{5\cdot (22{,}1-20)^2+8\cdot (21{,}1-21)^2+11\cdot (2{,}1-22)^2+10\cdot (22{,}1-23)^2+6\cdot (22{,}1-24)^2}{40}=1{,}54$.\\
		Độ lệch chuẩn $s=\sqrt{1{,}54}\approx 1{,}24$.
	}
\end{ex}

\begin{ex}%[Dự án BG10-2022 - Đề KT]%[Đỗ Viết Lân]%[0D5B3-5]
	Một mẫu số liệu thống kê có các tứ phân vị lần lượt là $Q_1=22$, $Q_2= 27$, $Q_3= 32$. Giá trị nào sau đây là giá trị bất thường của mẫu số liệu?
	\choice
	{$30$}
	{$8$}
	{\True $6$}
	{$46$} 
	\loigiai{
		Ta có $\Delta_{Q} = Q_3 - Q_1 = 32-22=10$. Do đó 	$\left[Q_1-1{,}5 \cdot  \Delta_{Q}; Q_3+1{,}5 \cdot  \Delta_{Q} \right] =\left[7;47\right]$.\\
		Do $6 \notin \left[7;47\right]$ nên là một giá trị bất thường của mẫu số liệu.
	}
\end{ex}


\begin{ex}%[Dự án BG10-2022 - Đề KT]%[Đỗ Viết Lân]%[0D5B3-5]
	Hãy tìm các giá trị bất thường của mẫu số liệu thống kê sau
	\begin{center}	
		\begin{tabular}{cccccccc}
			7 & 19 & 6 & 12 & 5 & 17 & 6 & 13  
		\end{tabular} 
	\end{center}
	\choice
	{$5$; $6$}
	{$5$; $6$; $19$}
	{\True Không có số liệu bất thường}
	{$5$; $19$} 
	\loigiai{
		Sắp xếp các số liệu trong mẫu theo thứ tự không giảm ta có
		\begin{center}
			\begin{tabular}{cccccccc}
				5 & 6 & 6 & 7 & 12 & 13 & 17 & 19  
			\end{tabular} 
		\end{center} 
		Từ bảng số liệu ta tìm được số trung vị $Q_2=\dfrac{7+12}{2}=9{,}5$, tứ phân vị dưới $Q_1= 6$, tứ phân vị trên $Q_3= 15$ và khoảng tứ phân vị $\Delta_{Q} = 15 - 6 = 9$.\\
		Ta có $\left[Q_1-1{,}5 \cdot  \Delta_{Q}; Q_3+1{,}5 \cdot  \Delta_{Q} \right] =\left[-7{,}5;28{,}5\right]$.\\
		Từ đó ta có mẫu số liệu trên không có số liệu bất thường.			
	}
\end{ex}

\begin{ex}%[Dự án BG10-2022 - Đề KT]%[Đỗ Viết Lân]%[0D5B3-5]
	Hãy tìm các giá trị bất thường của mẫu số liệu thống kê sau
	\begin{center}	
		\begin{tabular}{cccccccccc}
			$20$ & $52$ & $86$ & $80$ & $44$ & $49$ & $57$ & $41$& $44$ & $55$ 
		\end{tabular} 
	\end{center}
	\choice
	{$80$; $86$}
	{$41$; $80$; $86$}
	{\True $80$; $20$; $86$}
	{$86$} 
	\loigiai{
		Sắp xếp các số liệu trong mẫu theo thứ tự không giảm ta có
		\begin{center}	
			\begin{tabular}{cccccccccc}
				$20$ & $41$ & $44$ & $44$ & $49$ & $52$ & $55$ & $57$& $80$ & $86$ 
			\end{tabular} 
		\end{center}
		Từ bảng số liệu ta tìm được số trung vị $Q_2=\dfrac{49+52}{2}=50{,}5$, 
		tứ 
		phân vị dưới $Q_1= 44$, tứ phân vị trên $Q_3= 57$ và khoảng tứ phân vị 
		$\Delta_{Q} = 57 - 44 = 13$.\\
		Ta có $\left[Q_1-1{,}5 \cdot  \Delta_{Q}; Q_3+1{,}5 \cdot  \Delta_{Q} 
		\right] =\left[24{,}5;76{,}5\right]$.\\
		Từ đó ta có $80$; $20$ và $86$ là các số liệu bất thường.			
	}
\end{ex}




\noindent\textbf{II. PHẦN TỰ LUẬN}

\begin{bt}%[Dự án BG10-2022 - Đề KT]%[Đỗ Viết Lân]%[0D1Y5-2]
Nhà sản xuất công bố chiều dài và chiều rộng của một tấm thép hình chữ nhật lần lượt là $100\pm 0{,}5$ cm và $70 \pm 0{,}5$ cm. Hãy tính diện tích của tấm thép.
\loigiai{
Gọi $\overline{a}$ và $\overline{b}$ lần lượt là chiều dài và chiều rộng thực của tấm thép.\\
Ta có $99,5 \leq \overline{a} \leq 100,5$ và $69,5\leq \overline{b} \leq 70,5$.\\
Suy ra $6915,25 \leq \overline{a} \cdot \overline{b} \leq 7085,25$.\\
Do đó $-84,75 \leq \overline{a} \cdot \overline{b} -7000 \leq 85,25$.\\
Vậy diện tích tấm thép là $7000 \pm 85,25$ cm$^2$.
}
\end{bt}

\begin{bt}%[Dự án BG10-2022 - Đề KT]%[Đỗ Viết Lân]%[0D5Y3-5]
	Chiều cao (đơn vị: xăng-ti-mét) của các bạn tổ $I$ ở lớp $10A$ lần lượt là:
	\[165\quad 155\quad 171\quad 167\quad 159\quad 175\quad 165\quad 160\quad 158\]
	Đối với mẫu số liệu trên, hãy tìm:
	\begin{listEX}[4]
		\item Số trung bình cộng.
		\item Trung vị.
		\item Mốt.
		\item Tứ phân vị.
	\end{listEX}
	
	\loigiai{
		Mẫu số liệu trên được sắp xếp theo thứ tự tăng dần như sau:
		\[155\quad 158\quad 159\quad 160\quad 165\quad 165\quad 167\quad 171\quad 175\]
		\begin{listEX}[1]
			\item Số trung bình cộng: 
			\[\overline{x}=\dfrac{155+158+159+160+165+165+167+171+175}{9} \approx 163{,}9 \text{ (cm)}.\]\vspace{-1cm}
			\item Ta có $N=9$ là số lẻ. Số liệu thứ $\dfrac{9+1}{2}=5$. Vậy số trung vị là $M_e=165$.
			\item Ta thấy giá trị $165$ có tần số $2$ lớn nhất, do đó mốt của mẫu số liệu trên là: $M_O=165$.
			\item Trung vị của dãy $155\quad 158\quad 159\quad 160$ là: $Q_1=\dfrac{158+159}{2}=158{,}5$.\\
			Trung vị của dãy $165\quad 167\quad 171\quad 175$ là: $Q_3=\dfrac{167+171}{2}=169$.\\
			Vậy $Q_1=158{,}5$ (cm), $Q_2=165$ (cm), $Q_3=169$.
		\end{listEX}
	}
\end{bt}

\begin{bt}%[Dự án BG10-2022 - Đề KT]%[Đỗ Viết Lân]%[0D5B4-1]
	Cho mẫu số liệu gồm $15$ số dương không hoàn toàn giống nhau. Các số đo độ phân tán (khoảng biến thiên, khoảng tứ phân vị, độ lệch chuẩn) sẽ thay đổi như thế nào nếu
	\begin{enumerate}
		\item Nhân mỗi giá trị của mẫu số liệu với $3$.
		\item Cộng mỗi giá trị của mẫu số liệu với $3$.
	\end{enumerate}
	\loigiai{
		Giả sử $15$ số liệu được sắp xếp theo thứ tự không giảm là $x_1$; $x_2$; $\ldots$; $x_{15}$.
		\begin{enumerate}
			\item Nhân mỗi giá trị của mẫu số liệu với $3$. Ta có
			\begin{itemize}
				\item Khoảng biến thiên $R=3x_{15}-3x_1=3\left( x_{15}-x_1\right)$.
				\item Ta có $Q_2=3x_8$; $Q_1=3x_4$; $Q_3=3x_{12}$. Khoảng tứ phân vị: $\Delta_Q=Q_3-Q_1=3x_{12}-3x_8=3\left(x_{12}-x_8\right)$.
				\item Độ lệch chuẩn $s=\sqrt{\dfrac{1}{15} \sum\limits_{i=1}^{15} \left(3x_i-3\overline{x}\right)^2}=3\sqrt{\dfrac{1}{15} \sum\limits_{i=1}^{15} \left(x_i-\overline{x}\right)^2}$.
			\end{itemize}
			Vậy khi nhân mỗi giá trị của mẫu số liệu với $3$ thì các số đo độ phân tán (khoảng biến thiên, khoảng tứ phân vị, độ lệch chuẩn) sẽ tăng lên $3$ lần.
			\item Cộng mỗi giá trị của mẫu số liệu với $3$.
			\begin{itemize}
				\item Khoảng biến thiên $R=x_{15}+3-\left(x_1+3\right) =x_{15}-x_1$.
				\item Ta có $Q_2=x_8+3$; $Q_1=x_4+3$; $Q_3=x_{12}+3$.\\ Khoảng tứ phân vị: $\Delta_Q=Q_3-Q_1=x_{12}+3-\left(x_8+3\right)=x_{12}-x_8$.
				\item Độ lệch chuẩn $s=\sqrt{\dfrac{1}{15} \sum\limits_{i=1}^{15} \left(x_i+3-\left(\overline{x}+3\right)\right)^2}=\sqrt{\dfrac{1}{15} \sum\limits_{i=1}^{15} \left(x_i-\overline{x}\right)^2}$.
			\end{itemize}
		\end{enumerate}
		Vậy khi cộng mỗi giá trị của mẫu số liệu với $3$ thì các số đo độ phân tán (khoảng biến thiên, khoảng tứ phân vị, độ lệch chuẩn) sẽ không thay đổi.
	}
\end{bt}

\begin{bt}%[Dự án BG10-2022 - Đề KT]%[Đỗ Viết Lân]%[0D5B3-5]
	Kết quả kiểm tra môn Toán của lớp $10$A có $21$ học sinh, thể hiện ở bảng dưới đây
	\begin{center}
		\begin{tabular}{|c|c|c|c|c|c|c|c|c|c|c|c|c|c|c|c|c|c|c|c|c|}
			\hline $10$ & $6$ & $7$ & $7$ & $1$ & $7$ & $6$ & $9$ & $9$ & $10$ & $8$ & $8$ & $7$ & $8$ & $6$ & $7$ & $5$ & $6$ & $7$ & $8$ & $9$\\
			\hline
		\end{tabular}
	\end{center}
	Hãy tìm các số liệu bất thường trong mẫu số liệu trên.
	\loigiai{
		Sắp xếp các số liệu trong mẫu theo thứ tự không giảm ta có
		\begin{center}
			\begin{tabular}{|c|c|c|c|c|c|c|c|c|c|c|c|c|c|c|c|c|c|c|c|c|}
				\hline $1$ & $5$ & $6$ & $6$ & $6$ & $6$ & $7$ & $7$ & $7$ & $7$ & $7$ & $7$  & $8$ & $8$ & $8$ & $8$ & $9$ & $9$& $9$&  $10$& $10$ \\
				\hline
			\end{tabular}
		\end{center}
		Từ bảng số liệu ta tìm được số trung vị $Q_2=7$, tứ phân vị dưới $Q_1= 6$, tứ phân vị trên $Q_3= 8{,}5$ và khoảng tứ phân vị $\Delta_{Q} = 8{,}5 - 6 = 2{,}5$.\\
		Ta có $\left[Q_1-1{,}5 \cdot  \Delta_{Q}; Q_3+1{,}5 \cdot  \Delta_{Q} \right] =\left[2{,}5;12{,}25\right]$.\\
		Từ đó ta có $1$ là số liệu bất thường trong mẫu số liệu.		
	}
\end{bt}
\Closesolutionfile{ans}
\Closesolutionfile{ansbook}
% \indapan{10}{ans/ans-KT-502}
