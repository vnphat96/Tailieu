\section*{ÔN TẬP CHƯƠNG II}
\setcounter{ex}{0}\setcounter{bt}{0}
\Opensolutionfile{ans}[ans/ansOC2-CD-1]

\begin{ex}%[0D2Y1-2]
    Cặp số nào sau đây là nghiệm của bất phương trình $x+2y+1>0$.
    \choice
    {\True $(-1; 1)$}
    {$(1; -1)$}
    {$(-1; -1)$}
    {$(-2; 0)$}
    \loigiai{
        Thay lần lượt các cặp số vào bất phương trình thì ta thấy khi thay cặp số $(-1;1)$ vào bất phương trình ta được $-1+2 \cdot 1+1>0$ (luôn đúng).
    }
\end{ex}
% Câu 2
\begin{ex}%[0D2Y1-2]
    Cặp số $(x;y)$ nào sau đây \textbf{không} là nghiệm của bất phương trình $2x-y-4 \geq 0$?
    \choice
    {$(3;-1)$}
    {$(1;-2)$}
    {\True $(-1;-4)$}
    {$(2;-1)$}
    \loigiai{
        Thay các cặp số $(x;y)$ ở các phương án vào bất phương trình $2x-y-4 \geq 0$ ta thấy cặp $(-1;-4)$ không thỏa mãn bất phương trình.\\
        Do đó cặp $(-1;-4)$ \textbf{không} là nghiệm của bất phương trình đã cho.
    }
\end{ex}
% Câu 3
\begin{ex}%[0D2Y1-1]
    Bất phương trình nào sau đây là bất phương trình bậc nhất đối với hai ẩn $x$ và $y$?
    \choice
    {$x+2y+z-1>0$}
    {$x^2+y-1>0$}
    {$x+y+z+2t>0$}
    {\True $x+2y-3>0$}
    \loigiai{
        Bất phương trình $x+2y-3>0$ là bất phương trình bậc nhất đối với hai ẩn $x$ và $y$.
    }
\end{ex}
%Câu 4
\begin{ex}%[0D2Y1-2]
    Cặp số nào sau đây là nghiệm của bất phương trình $3x-y \leq 5$.
    \choice
    {$(2;-1)$}
    {\True $(2; 1)$}
    {$(1; -3)$}
    {$(2; 0)$}
    \loigiai{
        Thay lần lượt các cặp số vào bất phương trình thì ta thấy khi thay cặp số $(2; 1)$ vào bất phương trình ta được $3 \cdot 2-1 \leq 5$ (luôn đúng).
    }
\end{ex}
%Câu 5
\begin{ex}%[0D2Y1-1]
    Bất phương trình nào sau đây là bất phương trình bậc nhất đối với hai ẩn $x$ và $y$?
    \choice
    {$x+2y^2+3>0$}
    {\True $2x+3y \leq 5$}
    {$x+z+2t<2$}
    {$x+2y-3=0$}
    \loigiai{
        Bất phương trình $2x+3y \leq 5$ là bất phương trình bậc nhất đối với hai ẩn $x$ và $y$.
    }
\end{ex}
%Câu 6
\begin{ex}%[0D2Y1-3]
    Miền nghiệm của bất phương trình $x-2y+5>0$ là nửa mặt phẳng chứa điểm nào dưới đây?
    \choice
    {\True $(2;2)$}
    {$(1;3)$}
    {$(-2;2)$}
    {$(-2;4)$}
    \loigiai{
        Ta có $2-2 \cdot 2+5>0$ nên điểm $(2;2)$ thuộc miền nghiệm của bất phương trình $x-2y+5>0$.
    }
\end{ex}
%Câu 7
\begin{ex}%[0D2Y1-3]
    Miền nghiệm của bất phương trình $-x+2+2(y-2)<2(1-x)$ là nửa mặt phẳng không chứa điểm nào dưới đây?
    \choice
    {$(0;0)$}
    {$(1;1)$}
    {\True $(4;2)$}
    {$(1;-1)$}
    \loigiai{
        Ta có $-x+2+2(y-2)<2(1-x)$ $\Leftrightarrow x+2y<4$.\\
        Dễ thấy tại điểm $(4;2)$, ta có $4+2 \cdot 2=8>4$ nên điểm $(4;2)$ không thuộc miền nghiệm của bất phương trình $-x+2+2(y-2)<2(1-x)$.
    }
\end{ex}
%Câu 8
\begin{ex}%[0D2Y1-3]
    Miền nghiệm của bất phương trình $3(x-1)+4(y-2)<5x-3$ là nửa mặt phẳng chứa điểm nào dưới đây?
    \choice
    {\True $(0;0)$}
    {$(-4;2)$}
    {$(-2;2)$}
    {$(-5;3)$}
    \loigiai{
        Ta có $3(x-1)+4(y-2)<5x-3 \Leftrightarrow 2x-4y+8>0 \Leftrightarrow x-2y+4>0$.\\
        Dễ thấy tại điểm $(0;0)$, ta có $0-2 \cdot 0+4=4>0$ nên điểm $(0;0)$ thuộc miền nghiệm của bất phương trình $3(x-1)+4(y-2)<5x-3$.
    }
\end{ex}
%Câu 9
\begin{ex}%[0D2B1-3]
    \immini[thm]{
    Miền nghiệm của bất phương trình nào sau đây được biểu diễn bởi nửa mặt phẳng không bị gạch trong hình vẽ sau (tính cả biên)?
    \choice
    {\True $3x+2y \geq 6$}
    {$3x+2y>6$}
    {$2x+3y<6$}
    {$3x-2y \geq 6$}}
    {\begin{tikzpicture}[>=stealth,line join=round,line cap=round,font=\footnotesize,scale=0.6]
            \tikzset{
                declare function={x0=3;y0=4;m=-3/2;n=3;},
                declare function={f(\x)=m*(\x)+n;}
            }
            %   \tikzset{declare function = {func(\x) = \x^2;}}
            
            \clip (-x0,-y0+2.5)rectangle (x0,y0);
            \draw(-x0-1,-y0-1)rectangle(x0+1,y0+1);
            \draw[fill=black](0,0)circle(1.2pt)node[below left]{$O$};
            \draw [smooth,samples=100,domain=-x0:x0]plot(\x,{f(\x)});
            \fill[pattern=north east lines](x0,{f(x0)})--(-x0,{f(-x0)})--(-x0,-y0)--(x0,-y0)--(x0,y0)--cycle;
            \draw[thick,->] (-x0,0)--(x0,0)node[above left]{$x$};
            \draw[thick,->] (0,-y0)--(0,y0)node[below right]{$y$};
            \draw[fill=black]
            (0,3)node[right]{$3$}circle[radius=1.3pt](2,0)node[above right=-1mm]{$2$}circle[radius=1.3pt]
            ;
        \end{tikzpicture}}
    \loigiai{
        Đường thẳng $d \colon y=ax+b$ đi qua $A(0;3);B(2;0)$, khi đó
        \[\heva{&3=a \cdot 0+b\\&0=2a+b} \Leftrightarrow \heva{&a=-\dfrac{3}{2}\\&b=3} \Rightarrow y=-\dfrac{3}{2}x+3 \Rightarrow 3x+2y=6.\]
        Xét điểm $O(0,0)$ ta thấy $3 \cdot 0+2 \cdot 0<6$.\\
        Mà miền nghiệm trên hình biểu diễn là miền không chứa $O$ (kể cả đường thẳng $d$) suy ra miền nghiệm trên hình biểu diễn cho bất phương trình $3x+2y \geq 6$.
    }
\end{ex}
%Câu 10
\begin{ex}%%[0D2Y1-3]
    \immini[thm]{Miền nghiệm của bất phương trình nào sau đây được biểu diễn bởi nửa mặt phẳng không bị gạch trong hình vẽ sau (tính cả biên)?
    \choice
    {\True $3x+2y \geq -6$}
    {$2x-3y>-6$}
    {$3x-2y<-6$}
    {$3x+2y \geq 6$}}
    {\begin{tikzpicture}[>=stealth,line join=round,line cap=round,font=\footnotesize,scale=0.6]
            \tikzset{
                declare function={x0=3;y0=4;m=-3/2;n=-3;},
                declare function={f(\x)=m*(\x)+n;}
            }
            \clip (-x0-2,-y0)rectangle (x0-1,y0-2.5);
            \draw(-x0-3,-y0-1)rectangle(x0,y0-1);
            \draw[fill=black](0,0)circle(1.2pt)node[below left]{$O$};
            \draw [smooth,samples=100,domain=-x0:x0]plot(\x,{f(\x)});
            \fill[pattern=north east lines](x0-1,{f(x0-1)})--(-x0-2,{f(-x0-2)})--(-x0-2,-y0)--(x0-1,-y0)--(x0-1,y0-2.5)--cycle;
            \draw[thick,->] (-x0-2,0)--(x0-1,0)node[above left]{$x$};
            \draw[thick,->] (0,-y0)--(0,y0-2.5)node[below right]{$y$};
            \draw[fill=black]
            (0,-3)node[right]{$-3$}circle[radius=1.3pt](-2,0)node[above right=-1mm]{$-2$}circle[radius=1.3pt]
            ;
        \end{tikzpicture}}
    \loigiai{
        Đường thẳng $d \colon y=ax+b$ đi qua $A(0;-3);B(-2;0)$, khi đó
    \[  \heva{&-3=a \cdot 0+b\\&0=-2a+b} \Leftrightarrow \heva{&a=-\dfrac{3}{2}\\&b=-3} \Rightarrow y=-\dfrac{3}{2}x-3 \Rightarrow 3x+2y=-6.\]
        Xét điểm $O(0,0)$, ta được $3 \cdot 0+2 \cdot 0>-6$.\\
        Mà miền nghiệm trên hình biểu diễn là miền chứa $O$ (kể cả đường thẳng $d$) suy ra miền nghiệm trên hình biểu diễn cho bất phương trình $3x+2y \geq -6$.
    }
\end{ex}
%Câu 11
\begin{ex}%[0D2Y2-3]
    Trong các điểm sau, điểm nào thuộc miền nghiệm của hệ bất phương trình $\heva{&x+3y-2 \geq 0\\&2x+y+1 \leq 0}$?
    \choice
    {$M(0;1)$}
    {\True $N(-1;1)$}
    {$P(1;3)$}
    {$Q(-1;0)$}
    \loigiai{
        Thay tọa độ điểm $N(-1;1)$ vào hệ ta được $\heva{&-1+3\cdot 1-2 \geq 0\\&2\cdot (-1)+1+1 \leq 0}\Leftrightarrow \heva{&0\geq 0\\&0\leq 0}$ thỏa mãn.
    }
\end{ex}
%Câu 12
\begin{ex} %[0D2B2-3]
    Trong các điểm sau, điểm nào không thuộc miền nghiệm của hệ bất phương trình $\heva{&x+y-2 \leq 0\\&2x-3y+2>0}$?
    \choice
    {$O(0;0)$}
    {$M(1;1)$}
    {\True $N(-1;1)$}
    {$P(-1;-1)$}
    \loigiai{
        Thay tọa độ điểm $N(-1;1)$ vào hệ ta được $\heva{&-1+1-2\le 0\\&2\cdot (-1)-3\cdot 1+2>0}\Leftrightarrow \heva{&-2\le 0\\&-3>0}$ không thỏa mãn.
    }
\end{ex}
%Câu 13
\begin{ex}%[0D2B2-3]
    Điểm $M(0;-3)$ thuộc miền nghiệm của hệ bất phương trình nào sau đây?
    \choice
    {\True $\heva{&2x-y \leq 3\\&2x+5y \leq 12x+8}$}
    {$\heva{&2x-y>3\\&2x+5y \leq 12x+8}$}
    {$\heva{&2x-y>-3\\&2x+5y \geq 12x+8}$}
    {$\heva{&2x-y \leq -3\\&2x+5y \geq 12x+8}$}
    \loigiai{
        Thay tọa độ điểm $M(0;-3)$ vào các bất phương trình trong từng hệ cho trong phương án ở trên ta thấy chỉ có hệ $\heva{&2x-y \leq 3\\&2x+5y \leq 12x+8}$ thỏa mãn.
    }
\end{ex}
%Câu 14
\begin{ex}%[0D2B2-3]
    \immini{Phần không gạch chéo trong hình vẽ dưới đây (không chứa biên), biểu diễn miền nghiệm của hệ bất phương trình nào?
    \choice
    {$\heva{&x-y \geq 0\\&2x-y \leq 1}$}
    {\True $\heva{&x-y>0\\&2x-y>1}$}
    {$\heva{&x-y<0\\&2x-y>1}$}
    {$\heva{&x-y<0\\&2x-y<1}$}}
    {\begin{tikzpicture}[>=stealth,line join=round,line cap=round,font=\footnotesize,scale=0.6]
        \tikzset{
            declare function={x0=3;y0=4;m=2;n=-1;a=1;b=0;},
            declare function={f(\x)=m*(\x)+n; g(\x)=a*(\x)+b;}
        }
        %   \tikzset{declare function = {func(\x) = \x^2;}}
        
        \clip (-x0,-y0)rectangle (x0,y0);
        \draw[fill=black](0,0)circle(1.2pt)node[above left]{$O$};
        \draw [smooth,samples=100,domain=-x0:x0]plot(\x,{f(\x)});
        \fill[pattern=north west lines,opacity=.4](x0,{f(x0)})--(-x0,y0)--(-x0,-y0)--(-x0,{f(-x0)})--cycle;
        \draw [smooth,samples=100,domain=-x0:x0]plot(\x,{g(\x)});
        \fill[pattern=north west lines,opacity=.4](x0,{g(x0)})--(x0,y0)--(-x0,y0)--(-x0,{g(-x0)})--cycle;
        \draw[thick,->] (-x0,0)--(x0,0)node[above left]{$x$};
        \draw[thick,->] (0,-y0)--(0,y0)node[below right]{$y$};
        \draw[fill=black]
        (0,-1)node[right]{$-1$}circle[radius=1.3pt](1,0)node[above right=-1mm]{$1$}circle[radius=1.3pt](0,1)node[left]{$1$};
        ;
        \draw[dashed](0,1)-|(1,0);
    \end{tikzpicture}}
    \loigiai{
        Dễ thấy tọa độ điểm $M(1;0)$ thuộc miền nghiệm của hệ trên hình vẽ.\\
        Thay vào các phương án thì nó thỏa mãn hệ $\heva{&x-y>0\\&2x-y>1.}$
    }
\end{ex}
%Câu 15
\begin{ex}%[0D2B2-3]
    Miền nghiệm của hệ bất phương trình $\heva{&x+y-1>0\\&y<2\\&-x+2y>3}$ là phần không gạch chéo và không tính biên của hình vẽ nào trong các hình vẽ sau?
    \begin{center}
        \begin{tikzpicture}[>=stealth,line join=round,line cap=round,font=\footnotesize,scale=0.5]
            \tikzset{
                declare function={x0=3.2;y0=4;m=-1;n=1;a=1/2;b=3/2;},
                declare function={f(\x)=m*(\x)+n; g(\x)=a*(\x)+b;h1(\x)=2;h(\x)=-2;}
            }
            %   \tikzset{declare function = {func(\x) = \x^2;}}
            \node at (0,-y0+1.5) {Hình $1$};
            \clip (-x0,-y0+2)rectangle (x0,y0);
            \draw[fill=black](0,0)circle(1.2pt)node[above left]{$O$};
            \draw [smooth,samples=100,domain=-x0:x0]plot(\x,{f(\x)});
            \fill[pattern=north west lines,opacity=.4](-x0,{f(-x0)})--(x0,y0)--(x0,-y0)--(x0,{f(x0)})--cycle;
            \draw [smooth,samples=100,domain=-x0:x0]plot(\x,{g(\x)});
            \fill[pattern=north west lines,opacity=.4](-x0,{g(-x0)})--(-x0,-y0)--(x0,-y0)--(x0,{g(x0)})--cycle;
            \draw [smooth,samples=100,domain=-x0:x0]plot(\x,{h1(\x)});
            \fill[pattern=north west lines,opacity=.4](-x0,{h1(-x0)})--(-x0,y0)--(x0,y0)--(x0,{h1(x0)});
            \draw[thick,->] (-x0,0)--(x0,0)node[above left]{$x$};
            \draw[thick,->] (0,-y0)--(0,y0)node[below right]{$y$};
            \foreach \p/\goc in {-3/-90,1/90}{
                \draw[fill=black](\p,0)circle(1.3pt)+(\goc:2.5mm)node{$\p$};
            }
            \foreach \p/\goc in {2/60,1/0}{
                \draw[fill=black](0,\p)circle(1.3pt)+(\goc:2.8mm)node{$\p$};
            }
            
        \end{tikzpicture}\qquad
        \begin{tikzpicture}[>=stealth,line join=round,line cap=round,font=\footnotesize,scale=0.5]
            \tikzset{
                declare function={x0=3.2;y0=4;m=-1;n=1;a=1/2;b=3/2;},
                declare function={f(\x)=m*(\x)+n; g(\x)=a*(\x)+b;h1(\x)=2;h(\x)=-2;}
            }   
            \node at (0,-y0+1.5) {Hình $2$};
            \clip (-x0,-y0+2)rectangle (x0,y0);
            \draw[fill=black](0,0)circle(1.2pt)node[above left]{$O$};
            \draw [smooth,samples=100,domain=-x0:x0]plot(\x,{f(\x)});
            %%%%%%%%%%%%%%%%%%%%%%%%%%%%%%%%%%%%%%%%%%%%%%%%%%%%%%%%%%%%%%%
            \fill[pattern=north west lines,opacity=.4](-x0,{f(-x0)})--(-x0,-y0)--(x0,{f(x0)})--cycle;
            %%%%%%%%%%%%%%%%%%%%%%%%%%%%%%%%%%%%%%%%%%%%%%%%%%%%%%%%%%%%%%%%%%%%%%%%
            \draw [smooth,samples=100,domain=-x0:x0]plot(\x,{g(\x)});
            %%%%%%%%%%%%%%%%%%%%%%%%%%%%%%%%%%%%%%%%%%%%%%%%%%%%%%%%%%%%%%%%%%%%%%
            \fill[pattern=north west lines,opacity=.4](-x0,{g(-x0)})--(-x0,-y0)--(x0,-y0)--(x0,{g(x0)})--cycle;
            %%%%%%%%%%%%%%%%%%%%%%%%%%%%%%%%%%%%%%%%%%%%%%%%%%%%%%%%%%%%%%%%%%%%%%%%
            \draw [smooth,samples=100,domain=-x0:x0]plot(\x,{h1(\x)});
            %%%%%%%%%%%%%%%%%%%%%%%%%%%%%%%%%%%%%%%%%%%%%%%%%%%%%%%%%%%%%%%%%%%%%%%%
            \fill[pattern=north west lines,opacity=.4](-x0,{h1(-x0)})--(-x0,-y0)--(x0,-y0)--(x0,{h1(x0)});
            
            %%%%%%%%%%%%%%%%%%%%%%%%%%%%%%%%%%%%%%%%%%%%%%%%%%%%%%%%%%%%%%%%%%%%%
            \draw[thick,->] (-x0,0)--(x0,0)node[above left]{$x$};
            \draw[thick,->] (0,-y0)--(0,y0)node[below right]{$y$};
            \foreach \p/\goc in {-3/-90,1/90}{
                \draw[fill=black](\p,0)circle(1.3pt)+(\goc:2.5mm)node{$\p$};
            }
            \foreach \p/\goc in {2/60,1/0}{
                \draw[fill=black](0,\p)circle(1.3pt)+(\goc:2.8mm)node{$\p$};
            }
        \end{tikzpicture}\qquad     
        \begin{tikzpicture}[>=stealth,line join=round,line cap=round,font=\footnotesize,scale=0.5]
            \tikzset{
                declare function={x0=3.2;y0=4;m=-1;n=1;a=1/2;b=3/2;},
                declare function={f(\x)=m*(\x)+n; g(\x)=a*(\x)+b;h1(\x)=2;h(\x)=-2;}
            }   
            \node at (0,-y0+1.5) {Hình $3$};
            \clip (-x0,-y0+2)rectangle (x0,y0);
            \draw[fill=black](0,0)circle(1.2pt)node[above left]{$O$};
            \draw [smooth,samples=100,domain=-x0:x0]plot(\x,{f(\x)});
            %%%%%%%%%%%%%%%%%%%%%%%%%%%%%%%%%%%%%%%%%%%%%%%%%%%%%%%%%%%%%%%
            \fill[pattern=north west lines,opacity=.4](-x0,{f(-x0)})--(-x0,-y0)--(x0,{f(x0)})--cycle;
            %%%%%%%%%%%%%%%%%%%%%%%%%%%%%%%%%%%%%%%%%%%%%%%%%%%%%%%%%%%%%%%%%%%%%%%%
            \draw [smooth,samples=100,domain=-x0:x0]plot(\x,{g(\x)});
            %%%%%%%%%%%%%%%%%%%%%%%%%%%%%%%%%%%%%%%%%%%%%%%%%%%%%%%%%%%%%%%%%%%%%%
            \fill[pattern=north west lines,opacity=.4](-x0,{g(-x0)})--(-x0,y0)--(x0,y0)--(x0,{g(x0)})--cycle;
            %%%%%%%%%%%%%%%%%%%%%%%%%%%%%%%%%%%%%%%%%%%%%%%%%%%%%%%%%%%%%%%%%%%%%%%%
            \draw [smooth,samples=100,domain=-x0:x0]plot(\x,{h1(\x)});
            %%%%%%%%%%%%%%%%%%%%%%%%%%%%%%%%%%%%%%%%%%%%%%%%%%%%%%%%%%%%%%%%%%%%%%%%
            \fill[pattern=north west lines,opacity=.4](-x0,{h1(-x0)})--(-x0,y0)--(x0,y0)--(x0,{h1(x0)});
            
            %%%%%%%%%%%%%%%%%%%%%%%%%%%%%%%%%%%%%%%%%%%%%%%%%%%%%%%%%%%%%%%%%%%%%
            \draw[thick,->] (-x0,0)--(x0,0)node[above left]{$x$};
            \draw[thick,->] (0,-y0)--(0,y0)node[below right]{$y$};
            \foreach \p/\goc in {-3/-90,1/90}{
                \draw[fill=black](\p,0)circle(1.3pt)+(\goc:2.5mm)node{$\p$};
            }
            \foreach \p/\goc in {2/60,1/0}{
                \draw[fill=black](0,\p)circle(1.3pt)+(\goc:2.8mm)node{$\p$};
            }
        \end{tikzpicture}\qquad 
        \begin{tikzpicture}[>=stealth,line join=round,line cap=round,font=\footnotesize,scale=0.5]
            \tikzset{
                declare function={x0=3.2;y0=4;m=-1;n=1;a=1/2;b=3/2;},
                declare function={f(\x)=m*(\x)+n; g(\x)=a*(\x)+b;h1(\x)=2;h(\x)=-2;}
            }   
            \node at (0,-y0+1.5) {Hình $4$};
            \clip (-x0,-y0+2)rectangle (x0,y0);
            \draw[fill=black](0,0)circle(1.2pt)node[above left]{$O$};
            \draw [smooth,samples=100,domain=-x0:x0]plot(\x,{f(\x)});
            %%%%%%%%%%%%%%%%%%%%%%%%%%%%%%%%%%%%%%%%%%%%%%%%%%%%%%%%%%%%%%%
            \fill[pattern=north west lines,opacity=.4](-x0,{f(-x0)})--(-x0,-y0)--(x0,{f(x0)})--cycle;
            %%%%%%%%%%%%%%%%%%%%%%%%%%%%%%%%%%%%%%%%%%%%%%%%%%%%%%%%%%%%%%%%%%%%%%%%
            \draw [smooth,samples=100,domain=-x0:x0]plot(\x,{g(\x)});
            %%%%%%%%%%%%%%%%%%%%%%%%%%%%%%%%%%%%%%%%%%%%%%%%%%%%%%%%%%%%%%%%%%%%%%
            \fill[pattern=north west lines,opacity=.4](-x0,{g(-x0)})--(-x0,-y0)--(x0,-y0)--(x0,{g(x0)})--cycle;
            %%%%%%%%%%%%%%%%%%%%%%%%%%%%%%%%%%%%%%%%%%%%%%%%%%%%%%%%%%%%%%%%%%%%%%%%
            \draw [smooth,samples=100,domain=-x0:x0]plot(\x,{h1(\x)});
            %%%%%%%%%%%%%%%%%%%%%%%%%%%%%%%%%%%%%%%%%%%%%%%%%%%%%%%%%%%%%%%%%%%%%%%%
            \fill[pattern=north west lines,opacity=.4](-x0,{h1(-x0)})--(-x0,y0)--(x0,y0)--(x0,{h1(x0)});
            
            %%%%%%%%%%%%%%%%%%%%%%%%%%%%%%%%%%%%%%%%%%%%%%%%%%%%%%%%%%%%%%%%%%%%%
            \draw[thick,->] (-x0,0)--(x0,0)node[above left]{$x$};
            \draw[thick,->] (0,-y0)--(0,y0)node[below right]{$y$};
            \foreach \p/\goc in {-3/-90,1/90}{
                \draw[fill=black](\p,0)circle(1.3pt)+(\goc:2.5mm)node{$\p$};
            }
            \foreach \p/\goc in {2/60,1/0}{
                \draw[fill=black](0,\p)circle(1.3pt)+(\goc:2.8mm)node{$\p$};
            }
        \end{tikzpicture}
    \end{center}
    
    \choice
    {Hình $1$}
    {Hình $2$}
    {Hình $3$}
    {\True Hình $4$}
    \loigiai{
        Thay tọa độ điểm $O(0;0)$ vào các bất phương trình trong hệ đề cho ta thấy miền nghiệm của hệ là miền tam giác không tính biên ứng với hình $4$.
    }
\end{ex}
%Câu 16
\begin{ex}%[0D2T1-1]
    Sau trận cuối cùng của một mùa giải bóng đá nữ trường THPT A, huấn luyện viên trưởng đội lớp $10$B dẫn cả đội vào cửa hàng Pizza – Trà sữa. Mỗi cái bánh Pizza có giá $50$ nghìn đồng, một ly trà sữa có giá $20$ nghìn đồng. Huấn luyện viên trưởng không muốn trả quá $500$ nghìn đồng. Bất phương trình nào sau đây mô tả tốt cho tình huống trên (với $x$ là số bánh Pizza và $y$ là số ly trà sữa)?
    \choice
    {$5x+2y>50$}
    {\True $5x+2y \leq 50$}
    {$5x+2y \geq 50$}
    {$x+y \leq \dfrac{50}{7}$}
    \loigiai{
        Với $x$ là số bánh Pizza và $y$ là số ly trà sữa thì
        \begin{enumerate}[\faCheck]
            \item Giá tiền của $x$ cái bánh Pizza là $50x$ nghìn đồng.
            \item Giá tiền của $y$ ly trà sữa là $20y$ nghìn đồng.  
        \end{enumerate}
        Tổng số tiền cần thanh toán là $50x+20y$ nghìn đồng.\\
        Vì huấn luyện viên trưởng không muốn trả quá $500$ nghìn đồng nên ta có bất phương trình \[50x+20y \leq 500 \Leftrightarrow 5x+2y \leq 50.\]
    }
\end{ex}
%Câu 17
\begin{ex}%[0D2T1-1]
    Bác sĩ Minh Trang có một phòng khám thú y tư nhân. Mỗi ngày phòng khám làm việc không quá $7$ tiếng. Mỗi ca khám bệnh thông thường tốn khoảng thời gian là $20$ phút, mỗi ca phẫu thuật cần khoảng thời gian là $40$ phút. Bất phương trình nào sau đây mô tả tốt cho tình huống trên (trong đó $v$ là số ca khám và $s$ là số ca phẫu thuật mỗi ngày).
    \choice
    {$40s+20v \geq 420$}
    {\True $40s+20v \leq 420$}
    {$40s+20v \geq 7$}
    {$40s+20v \leq 7$}
    \loigiai{
        Với $s$ là số ca phẫu thuật và $v$ là số ca khám mỗi ngày thì
        \begin{enumerate}[\faCheck]
            \item Thời gian thực hiện phẫu thuật trong một ngày là $40s$ phút.
            \item Thời gian khám bệnh trong một ngày là $20v$ phút.
        \end{enumerate}
        Tổng số thời gian làm việc trong ngày là $40s+20v$ phút.\\
        Vì mỗi ngày phòng khám làm việc không quá $7$ tiếng nên ta có bất phương trình $40s+20v \leq 7 \cdot 60$.
    }
\end{ex}
%Câu 18
\begin{ex}%[0D2K1-2]
    Tìm tất cả các giá trị của tham số $m$ để $(x;y)=(-3;2)$ là một nghiệm của bất phương trình $5x-my>1$.
    \choice
    {$m>-8$}
    {\True $m<-8$}
    {$m>8$}
    {$m<8$}
    \loigiai{
        Để $(x;y)=(-3;2)$ là một nghiệm của bất phương trình $5x-my>1$ thì $5 \cdot (-3)-m \cdot 2>1 \Leftrightarrow m<-8$.
    }
\end{ex}
%Câu 19
\begin{ex}%[0D2K1-2]
    Tìm tất cả các giá trị của tham số $m$ để $(x;y)=(m;-1)$ là một nghiệm của bất phương trình $x-y<3$.
    \choice
    {$m>-2$}
    {\True $m<2$}
    {$m<-2$}
    {$m>2$}
    \loigiai{
        Để $(x;y)=(m;-1)$ là một nghiệm của bất phương trình $x-y<3$ thì $m+1<3 \Leftrightarrow m<2$.
    }
\end{ex}
%Câu 20
\begin{ex} %[0D2T1-2]
    Khẩu phần ăn trong một ngày của một gia đình nọ cần ít nhất $900$g chất protit. Biết rằng mỗi kilôgam thị bò chứa $80\%$ protit, mỗi kilôgam thịt heo chứa $60\%$. Một phương án hợp lí mà gia đình này có thể chọn để đáp ứng nhu cầu chất protit mỗi ngày là
    \choice
    {\True $500$ g thịt bò và $900$ g thịt heo}
    {$500$ g thịt bò và $500$ g thịt heo}
    {$700$ g thịt bò và $500$ g thịt heo}
    {$550$ g thịt bò và $750$ g thịt heo}
    \loigiai{
        Vì $80\% \cdot 500+60\% \cdot 900 \geq 900$ là mệnh đề đúng.
    }
\end{ex}
%Câu 21
\begin{ex}%[0D2K1-3]
    Miền nghiệm của bất phương trình $x+2y-4>0$ là phần không gạch chéo trong hình nào sau đây?
    \begin{center}
        \begin{tikzpicture}[>=stealth,line join=round,line cap=round,font=\footnotesize,scale=0.7]
            \tikzset{
                declare function={x2=6;y2=4;x1=-1;y1=-2;m=-1/2;n=2;a=1;b=0;},
                declare function={f(\x)=m*(\x)+n; g(\x)=a*(\x)+b;}
            }
            \node at (2,y1-.5){Hình $1$};
            \clip (x1,y1)rectangle (x2,y2);
            \draw[fill=black](0,0)circle(1.2pt)node[above left]{$O$};
            %%%%%%%%%%%%%%%%%%%%%%%%%%%%%%%%%%%%%%%%%%%%%%%%%%%%%%%%%%%%%%%
            \draw [smooth,samples=100,domain=x1:x2]plot(\x,{f(\x)});
            %%%%%%%%%%%%%%%%%%%%%%%%%%%%%%%%%%%%%%%%%%%%%%%%%%%%%%%%%%%%%%%
            \fill[pattern=north west lines,opacity=.4](x1,{f(x1)})--(x1,y2)--(x2,y2)--(x2,{f(x2)});
            
            \draw[thick,->] (x1,0)--(x2,0)node[above left]{$x$};
            \draw[thick,->] (0,y1)--(0,y2)node[below right]{$y$};
            
            \foreach \p/\goc in {4/-90}{
                \draw[fill=black](\p,0)circle(1.3pt)+(\goc:2.5mm)node{$\p$};
            }
            \foreach \p/\goc in {2/10}{
                \draw[fill=black](0,\p)circle(1.3pt)+(\goc:2.8mm)node{$\p$};
            }
            
        \end{tikzpicture}\hspace{1cm}
        \begin{tikzpicture}[>=stealth,line join=round,line cap=round,font=\footnotesize,scale=0.7]
            \tikzset{
                declare function={x2=6;y2=4;x1=-1;y1=-2;m=-1/2;n=2;a=1;b=0;},
                declare function={f(\x)=m*(\x)+n; g(\x)=a*(\x)+b;}
            }
            \node at (2,y1-.5){Hình $2$};
            \clip (x1,y1)rectangle (x2,y2);
            \draw[fill=black](0,0)circle(1.2pt)node[above left]{$O$};
            %%%%%%%%%%%%%%%%%%%%%%%%%%%%%%%%%%%%%%%%%%%%%%%%%%%%%%%%%%%%%%%
            \draw [smooth,samples=100,domain=x1:x2]plot(\x,{f(\x)});
            %%%%%%%%%%%%%%%%%%%%%%%%%%%%%%%%%%%%%%%%%%%%%%%%%%%%%%%%%%%%%%%
            \fill[pattern=north west lines,opacity=.4](x1,{f(x1)})--(x1,y1)--(x2,y1)--(x2,{f(x2)});
            
            \draw[thick,->] (x1,0)--(x2,0)node[above left]{$x$};
            \draw[thick,->] (0,y1)--(0,y2)node[below right]{$y$};
            
            \foreach \p/\goc in {4/-90}{
                \draw[fill=black](\p,0)circle(1.3pt)+(\goc:2.5mm)node{$\p$};
            }
            \foreach \p/\goc in {2/10}{
                \draw[fill=black](0,\p)circle(1.3pt)+(\goc:2.8mm)node{$\p$};
            }
            
        \end{tikzpicture}\\
        
        \begin{tikzpicture}[>=stealth,line join=round,line cap=round,font=\footnotesize,scale=0.7]
            \tikzset{
                declare function={x2=1;y2=4;x1=-6;y1=-2;m=1/2;n=2;a=1;b=0;},
                declare function={f(\x)=m*(\x)+n; g(\x)=a*(\x)+b;}
            }
            \node at (-2,y1-.5){Hình $3$};
            \clip (x1,y1)rectangle (x2,y2);
            \draw[fill=black](0,0)circle(1.2pt)node[above left]{$O$};
            %%%%%%%%%%%%%%%%%%%%%%%%%%%%%%%%%%%%%%%%%%%%%%%%%%%%%%%%%%%%%%%
            \draw [smooth,samples=100,domain=x1:x2]plot(\x,{f(\x)});
            %%%%%%%%%%%%%%%%%%%%%%%%%%%%%%%%%%%%%%%%%%%%%%%%%%%%%%%%%%%%%%%
            \fill[pattern=north west lines,opacity=.4](x1,{f(x1)})--(x1,y2)--(x2,y2)--(x2,{f(x2)});
            
            \draw[thick,->] (x1,0)--(x2,0)node[above left]{$x$};
            \draw[thick,->] (0,y1)--(0,y2)node[below right]{$y$};
            
            \foreach \p/\goc in {-4/-90}{
                \draw[fill=black](\p,0)circle(1.3pt)+(\goc:2.5mm)node{$\p$};
            }
            \foreach \p/\goc in {2/170}{
                \draw[fill=black](0,\p)circle(1.3pt)+(\goc:2.8mm)node{$\p$};
            }
            
        \end{tikzpicture}\hspace{1cm}
        \begin{tikzpicture}[>=stealth,line join=round,line cap=round,font=\footnotesize,scale=0.7]
            \tikzset{
                declare function={x2=1;y2=4;x1=-6;y1=-2;m=1/2;n=2;a=1;b=0;},
                declare function={f(\x)=m*(\x)+n; g(\x)=a*(\x)+b;}
            }
            \node at (-2,y1-.5){Hình $4$};
            \clip (x1,y1)rectangle (x2,y2);
            \draw[fill=black](0,0)circle(1.2pt)node[above left]{$O$};
            %%%%%%%%%%%%%%%%%%%%%%%%%%%%%%%%%%%%%%%%%%%%%%%%%%%%%%%%%%%%%%%
            \draw [smooth,samples=100,domain=x1:x2]plot(\x,{f(\x)});
            %%%%%%%%%%%%%%%%%%%%%%%%%%%%%%%%%%%%%%%%%%%%%%%%%%%%%%%%%%%%%%%
            \fill[pattern=north west lines,opacity=.4](x1,{f(x1)})--(x1,y1)--(x2,y1)--(x2,{f(x2)});
            
            \draw[thick,->] (x1,0)--(x2,0)node[above left]{$x$};
            \draw[thick,->] (0,y1)--(0,y2)node[below right]{$y$};
            
            \foreach \p/\goc in {-4/-90}{
                \draw[fill=black](\p,0)circle(1.3pt)+(\goc:2.5mm)node{$\p$};
            }
            \foreach \p/\goc in {2/170}{
                \draw[fill=black](0,\p)circle(1.3pt)+(\goc:2.8mm)node{$\p$};
            }
            
        \end{tikzpicture}
    \end{center}
    
    \choice
    {Hình $1$}
    {\True Hình $2$}
    { Hình $3$}
    { Hình $4$}
    \loigiai{
        Đường thẳng $x+2y-4=0$ có đồ thị ở hình $2$ và hình $4$.\\
        Điểm $O(0;0)$ không thỏa bất phương trình $x+2y-4>0$ nên miền nghiệm của bất phương trình $x+2y-4>0$ không chứa điểm $O(0;0)$. \\
        Do đó hình $2$ biểu diễn miền nghiệm của bất phương trình đã cho.
    }
\end{ex}
%Câu 22
\begin{ex}
    Miền nghiệm của bất phương trình $x+y-2>0$ là phần không gạch chéo trong hình nào sau đây?
    \begin{center}
        \begin{tikzpicture}[>=stealth,line join=round,line cap=round,font=\footnotesize,scale=0.7]
            \tikzset{
                declare function={x2=4;y2=4;x1=-3;y1=-2;m=-1;n=2;a=1;b=0;},
                declare function={f(\x)=m*(\x)+n; g(\x)=a*(\x)+b;}
            }
            \node at (1,y1-.5){Hình $1$};
            \clip (x1,y1)rectangle (x2,y2);
            \draw[fill=black](0,0)circle(1.2pt)node[above left]{$O$};
            %%%%%%%%%%%%%%%%%%%%%%%%%%%%%%%%%%%%%%%%%%%%%%%%%%%%%%%%%%%%%%%
            \draw [smooth,samples=100,domain=x1:x2]plot(\x,{f(\x)});
            %%%%%%%%%%%%%%%%%%%%%%%%%%%%%%%%%%%%%%%%%%%%%%%%%%%%%%%%%%%%%%%
            \fill[pattern=north west lines,opacity=.4](x1,{f(x1)})--(x1,y2)--(x2,y2)--(x2,{f(x2)});
            
            \draw[thick,->] (x1,0)--(x2,0)node[above left]{$x$};
            \draw[thick,->] (0,y1)--(0,y2)node[below right]{$y$};
            
            \foreach \p/\goc in {2/-90}{
                \draw[fill=black](\p,0)circle(1.3pt)+(\goc:2.5mm)node{$\p$};
            }
            \foreach \p/\goc in {2/10}{
                \draw[fill=black](0,\p)circle(1.3pt)+(\goc:2.8mm)node{$\p$};
            }
            
        \end{tikzpicture}\hspace{1cm}
        \begin{tikzpicture}[>=stealth,line join=round,line cap=round,font=\footnotesize,scale=0.7]
            \tikzset{
                declare function={x2=4;y2=4;x1=-3;y1=-2;m=-1;n=2;a=1;b=0;},
                declare function={f(\x)=m*(\x)+n; g(\x)=a*(\x)+b;}
            }
            \node at (1,y1-.5){Hình $2$};
            \clip (x1,y1)rectangle (x2,y2);
            \draw[fill=black](0,0)circle(1.2pt)node[above left]{$O$};
            %%%%%%%%%%%%%%%%%%%%%%%%%%%%%%%%%%%%%%%%%%%%%%%%%%%%%%%%%%%%%%%
            \draw [smooth,samples=100,domain=x1:x2]plot(\x,{f(\x)});
            %%%%%%%%%%%%%%%%%%%%%%%%%%%%%%%%%%%%%%%%%%%%%%%%%%%%%%%%%%%%%%%
            \fill[pattern=north west lines,opacity=.4](x1,{f(x1)})--(x1,y1)--(x2,y1)--(x2,{f(x2)});
            
            \draw[thick,->] (x1,0)--(x2,0)node[above left]{$x$};
            \draw[thick,->] (0,y1)--(0,y2)node[below right]{$y$};
            
            \foreach \p/\goc in {2/-90}{
                \draw[fill=black](\p,0)circle(1.3pt)+(\goc:2.5mm)node{$\p$};
            }
            \foreach \p/\goc in {2/10}{
                \draw[fill=black](0,\p)circle(1.3pt)+(\goc:2.8mm)node{$\p$};
            }
            
        \end{tikzpicture}
        
        \begin{tikzpicture}[>=stealth,line join=round,line cap=round,font=\footnotesize,scale=0.7]
            \tikzset{
                declare function={x2=3;y2=4;x1=-4;y1=-2;m=1;n=2;a=1;b=0;},
                declare function={f(\x)=m*(\x)+n; g(\x)=a*(\x)+b;}
            }
            \node at (0,y1-.5){Hình $3$};
            \clip (x1,y1)rectangle (x2,y2);
            \draw[fill=black](0,0)circle(1.2pt)node[above left]{$O$};
            %%%%%%%%%%%%%%%%%%%%%%%%%%%%%%%%%%%%%%%%%%%%%%%%%%%%%%%%%%%%%%%
            \draw [smooth,samples=100,domain=x1:x2]plot(\x,{f(\x)});
            %%%%%%%%%%%%%%%%%%%%%%%%%%%%%%%%%%%%%%%%%%%%%%%%%%%%%%%%%%%%%%%
            \fill[pattern=north west lines,opacity=.4](x1,{f(x1)})--(x1,y2)--(x2,y2)--(x2,{f(x2)});
            
            \draw[thick,->] (x1,0)--(x2,0)node[above left]{$x$};
            \draw[thick,->] (0,y1)--(0,y2)node[below right]{$y$};
            
            \foreach \p/\goc in {-2/-90}{
                \draw[fill=black](\p,0)circle(1.3pt)+(\goc:2.5mm)node{$\p$};
            }
            \foreach \p/\goc in {2/170}{
                \draw[fill=black](0,\p)circle(1.3pt)+(\goc:2.8mm)node{$\p$};
            }
            
        \end{tikzpicture}\hspace{1cm}
        \begin{tikzpicture}[>=stealth,line join=round,line cap=round,font=\footnotesize,scale=0.7]
            \tikzset{
                declare function={x2=3;y2=4;x1=-4;y1=-2;m=1;n=2;a=1;b=0;},
                declare function={f(\x)=m*(\x)+n; g(\x)=a*(\x)+b;}
            }
            \node at (0,y1-.5){Hình $4$};
            \clip (x1,y1)rectangle (x2,y2);
            \draw[fill=black](0,0)circle(1.2pt)node[above left]{$O$};
            %%%%%%%%%%%%%%%%%%%%%%%%%%%%%%%%%%%%%%%%%%%%%%%%%%%%%%%%%%%%%%%
            \draw [smooth,samples=100,domain=x1:x2]plot(\x,{f(\x)});
            %%%%%%%%%%%%%%%%%%%%%%%%%%%%%%%%%%%%%%%%%%%%%%%%%%%%%%%%%%%%%%%
            \fill[pattern=north west lines,opacity=.4](x1,{f(x1)})--(x1,y1)--(x2,y1)--(x2,{f(x2)});
            
            \draw[thick,->] (x1,0)--(x2,0)node[above left]{$x$};
            \draw[thick,->] (0,y1)--(0,y2)node[below right]{$y$};
            
            \foreach \p/\goc in {-2/-90}{
                \draw[fill=black](\p,0)circle(1.3pt)+(\goc:2.5mm)node{$\p$};
            }
            \foreach \p/\goc in {2/170}{
                \draw[fill=black](0,\p)circle(1.3pt)+(\goc:2.8mm)node{$\p$};
            }
            
        \end{tikzpicture}   
    \end{center}
    \choice
    {Hình $1$}
    {Hình $2$}
    {\True Hình $3$}
    {Hình $4$}
    \loigiai{
        Đường thẳng $x+y-2=0$ có đồ thị ở hình $3$ và hình $4$.\\
        Điểm $O(0;0)$ không thỏa bất phương trình $x+y-2>0$ nên miền nghiệm của bất phương trình $x+y-2>0$ không chứa điểm $O(0;0)$.\\ 
        Vì vậy miền nghiệm của bất phương trình là hình $3$.
    }
\end{ex}
%Câu 23
\begin{ex}%[0D2K1-3]
    \immini[thm]{Phần không gạch chéo trong hình sau biểu diễn miền nghiệm của bất phương trình nào?
    \choice
    {$x-2y+6>0$}
    {$x-y+6>y-3$}
    {$x-2y-6>0$}
    {\True $2x+y>3(x+2)-y$}
    }
    {\begin{tikzpicture}[>=stealth,line join=round,line cap=round,font=\footnotesize,scale=0.6]
            \tikzset{
                declare function={x2=3;y2=4;x1=-7;y1=-2;m=1/2;n=3;a=1;b=0;},
                declare function={f(\x)=m*(\x)+n; g(\x)=a*(\x)+b;h1(\x)=2;h2(\x)=-2;}
            }
            %   \tikzset{declare function = {func(\x) = \x^2;}}
            
            \clip (x1,y1)rectangle (x2,y2);
            \draw[fill=black](0,0)circle(1.2pt)node[above left]{$O$};
            %%%%%%%%%%%%%%%%%%%%%%%%%%%%%%%%
            \draw [smooth,samples=100,domain=x1:x2]plot(\x,{f(\x)});
            %%%%%%%%%%%%%%%%%%%%%%%%%%%%%%%%
            \fill[pattern=north west lines,opacity=.4](x1,{f(x1)})--(x1,y1)--(x2,y1)--(x2,{f(x2)});
            
            %%%%%%%%%%%%%%%%%%%%%%%%%%%%%%%%
            \draw[thick,->] (x1,0)--(x2,0)node[above left]{$x$};
            \draw[thick,->] (0,y1)--(0,y2)node[below right]{$y$};
            
            \foreach \p/\goc in {-6/-90}{
                \draw[fill=black](\p,0)circle(1.3pt)+(\goc:2.5mm)node{$\p$};
            }
            \foreach \p/\goc in {3/-30}{
                \draw[fill=black](0,\p)circle(1.3pt)+(\goc:2.8mm)node{$\p$};
            }
        \end{tikzpicture}   }
    \loigiai{
        Đường thẳng trong hình có phương trình là $x-2y+6=0$.\\
        Điểm $O(0;0)$ thỏa mãn bất phương trình $x-2y+6>0$ nên phần không gạch trong hình sẽ biểu diễn miền nghiệm của bất phương trình $x-2y+6<0$.\\
        Ta thấy $2x+y>3(x+2)-y \Leftrightarrow 2x+y>3x+6-y \Leftrightarrow x-2y+6<0$.
    }
\end{ex}
%Câu 24
\begin{ex}%[0D2B1-2]
    Miền nghiệm của bất phương trình $2x-3y+5 \geq 0$ là phần gạch chéo trong hình vẽ nào dưới đây?
    \begin{center}
        \begin{tikzpicture}[>=stealth,line join=round,line cap=round,font=\footnotesize,scale=0.8]
            \tikzset{
                declare function={x2=3;y2=4;x1=-4;y1=-2;m=2/3;n=5/3;a=1;b=0;},
                declare function={f(\x)=m*(\x)+n; g(\x)=a*(\x)+b;h1(\x)=2;h2(\x)=-2;}
            }
            %   \tikzset{declare function = {func(\x) = \x^2;}}
            \node at (0,y1-.5){Hình $1$};
            \clip (x1,y1)rectangle (x2,y2);
            \draw[fill=black](0,0)circle(1.2pt)node[above left]{$O$};
            %%%%%%%%%%%%%%%%%%%%%%%%%%%%%%%%
            \draw [smooth,samples=100,domain=x1:x2]plot(\x,{f(\x)});
            %%%%%%%%%%%%%%%%%%%%%%%%%%%%%%%%
            \fill[pattern=north west lines,opacity=.4](x1,{f(x1)})--(x1,y2)--(x2,y2)--(x2,{f(x2)});
            %%%%%%%%%%%%%%%%%%%%%%%%%%%%%%%%
            
            %%%%%%%%%%%%%%%%%%%%%%%%%%%%%%%%
            \draw[thick,->] (x1,0)--(x2,0)node[above left]{$x$};
            \draw[thick,->] (0,y1)--(0,y2)node[below right]{$y$};
            \draw[fill=black](-5/2,0)circle(1.3pt)+(-90:3.5mm)node{$-\tfrac{5}{2}$};
            \draw[fill=black](0,5/3)circle(1.3pt)+(160:2.8mm)node{$\tfrac{5}{3}$};              
        \end{tikzpicture}\hspace{1cm}
        \begin{tikzpicture}[>=stealth,line join=round,line cap=round,font=\footnotesize,scale=0.8]
            \tikzset{
                declare function={x2=3;y2=4;x1=-4;y1=-2;m=2/3;n=5/3;a=1;b=0;},
                declare function={f(\x)=m*(\x)+n; g(\x)=a*(\x)+b;h1(\x)=2;h2(\x)=-2;}
            }
            %   \tikzset{declare function = {func(\x) = \x^2;}}
            \node at (0,y1-.5){Hình $2$};
            \clip (x1,y1)rectangle (x2,y2);
            \draw[fill=black](0,0)circle(1.2pt)node[above left]{$O$};
            %%%%%%%%%%%%%%%%%%%%%%%%%%%%%%%%
            \draw [smooth,samples=100,domain=x1:x2]plot(\x,{f(\x)});
            %%%%%%%%%%%%%%%%%%%%%%%%%%%%%%%%
            \fill[pattern=north west lines,opacity=.4](x1,{f(x1)})--(x1,y1)--(x2,y1)--(x2,{f(x2)});
            %%%%%%%%%%%%%%%%%%%%%%%%%%%%%%%%
            
            %%%%%%%%%%%%%%%%%%%%%%%%%%%%%%%%
            \draw[thick,->] (x1,0)--(x2,0)node[above left]{$x$};
            \draw[thick,->] (0,y1)--(0,y2)node[below right]{$y$};
            \draw[fill=black](-5/2,0)circle(1.3pt)+(-90:3.5mm)node{$-\tfrac{5}{2}$};
            \draw[fill=black](0,5/3)circle(1.3pt)+(150:2.8mm)node{$\tfrac{5}{3}$};              
        \end{tikzpicture}\\
        
        \begin{tikzpicture}[>=stealth,line join=round,line cap=round,font=\footnotesize,scale=0.8]
            \tikzset{
                declare function={x2=4.5;y2=2;x1=-3;y1=-3;m=2/3;n=-5/3;a=1;b=0;},
                declare function={f(\x)=m*(\x)+n; g(\x)=a*(\x)+b;h1(\x)=2;h2(\x)=-2;}
            }
            %   \tikzset{declare function = {func(\x) = \x^2;}}
            \node at (0,y1-.5){Hình $3$};
            \clip (x1,y1)rectangle (x2,y2);
            \draw[fill=black](0,0)circle(1.2pt)node[above left]{$O$};
            %%%%%%%%%%%%%%%%%%%%%%%%%%%%%%%%
            \draw [smooth,samples=100,domain=x1:x2]plot(\x,{f(\x)});
            %%%%%%%%%%%%%%%%%%%%%%%%%%%%%%%%
            \fill[pattern=north west lines,opacity=.4](x1,{f(x1)})--(x1,y2)--(x2,y2)--(x2,{f(x2)});
            %%%%%%%%%%%%%%%%%%%%%%%%%%%%%%%%
            
            %%%%%%%%%%%%%%%%%%%%%%%%%%%%%%%%
            \draw[thick,->] (x1,0)--(x2,0)node[above left]{$x$};
            \draw[thick,->] (0,y1)--(0,y2)node[below right]{$y$};
            \draw[fill=black](5/2,0)circle(1.3pt)+(90:3.5mm)node{$\tfrac{5}{2}$};
            \draw[fill=black](0,-5/3)circle(1.3pt)+(150:3.8mm)node{$-\tfrac{5}{3}$};            
        \end{tikzpicture}\hspace{1cm}
        \begin{tikzpicture}[>=stealth,line join=round,line cap=round,font=\footnotesize,scale=0.8]
            \tikzset{
                declare function={x2=4.5;y2=2;x1=-3;y1=-3;m=2/3;n=-5/3;a=1;b=0;},
                declare function={f(\x)=m*(\x)+n; g(\x)=a*(\x)+b;h1(\x)=2;h2(\x)=-2;}
            }
            %   \tikzset{declare function = {func(\x) = \x^2;}}
            \node at (0,y1-.5){Hình $4$};
            \clip (x1,y1)rectangle (x2,y2);
            \draw[fill=black](0,0)circle(1.2pt)node[above left]{$O$};
            %%%%%%%%%%%%%%%%%%%%%%%%%%%%%%%%
            \draw [smooth,samples=100,domain=x1:x2]plot(\x,{f(\x)});
            %%%%%%%%%%%%%%%%%%%%%%%%%%%%%%%%
            \fill[pattern=north west lines,opacity=.4](x1,{f(x1)})--(x1,y1)--(x2,y1)--(x2,{f(x2)});
            %%%%%%%%%%%%%%%%%%%%%%%%%%%%%%%%
            
            %%%%%%%%%%%%%%%%%%%%%%%%%%%%%%%%
            \draw[thick,->] (x1,0)--(x2,0)node[above left]{$x$};
            \draw[thick,->] (0,y1)--(0,y2)node[below right]{$y$};
            \draw[fill=black](5/2,0)circle(1.3pt)+(90:3.5mm)node{$\tfrac{5}{2}$};
            \draw[fill=black](0,-5/3)circle(1.3pt)+(150:3.8mm)node{$-\tfrac{5}{3}$};        
        \end{tikzpicture}
    \end{center}
    \choice
    {\True Hình $1$ }
    {Hình $2$ }
    {Hình $3$ }
    {Hình $4$ }
    \loigiai{
        Đường thẳng $2x-3y+5=0$ có đồ thị ở hình $1$ và hình $2$.\\
        Điểm $O(0;0)$ thỏa mãn bất phương trình $2x-3y+5 \geq 0$ nên phần gạch chéo biểu diễn miền nghiệm bất phương trình $2x-3y+5 \geq 0$ phải chứa điểm $O(0;0)$.\\
        Do đó miền nghiệm của bất phương trình là hình $1$.
        
    }
\end{ex}
%Câu 25
\begin{ex}%[0D2B1-2]
    Miền nghiệm của bất phương trình $5(x+2)-9>2x-y$ không chứa điểm nào trong các điểm sau?
    \choice
    {\True $(-2;-1)$}
    {$(2;1)$}
    {$(2;3)$}
    {$(0;0)$}
    \loigiai{
    Lần lượt thay tọa độ các điểm, ta thấy điểm $(-2;-1)$ không thỏa mãn nên miền nghiệm của bất phương trình $5(x+2)-9>2x-y$ không chứa điểm $(-2;-1)$.
    }
\end{ex}
%Câu 26
\begin{ex}%[0D2B2-3]
    Trong các điểm sau, điểm nào thuộc miền nghiệm của hệ bất phương trình $\heva{&x-5y-3 \geq 0\\&3x-y+1 \leq 0}$?
    \choice
    {$M(0;-1)$}
    {\True $N(-1;-1)$}
    {$P(1;-3)$}
    {$Q(-1;0)$}
    \loigiai{
        Ta thay lần lượt tọa độ các điểm trong các phương án vào hệ bất phương trình $\heva{&x-5y-3 \geq 0\\&3x-y+1 \leq 0.}$\\
        Ta thấy $N(-1;-1) \Rightarrow$ $\heva{&-1-5 \cdot (-1)-3 \geq 0\\&3 \cdot (-1)+1+1 \leq 0}$ đúng.
    }
\end{ex}
%Câu 27
\begin{ex}%[0D2B2-3]
    Cho hệ bất phương trình $\heva{&4x-3y \geq 5\\&3-4x+3y>0}$ có tập nghiệm $S$. Khẳng định nào sau đây là khẳng định đúng?
    \choice
    {$(1;-2) \in S$}
    {$(2;-1) \in S$}
    {$(-1;-3) \in S$}
    {\True $S=\varnothing$}
    \loigiai{
        Ta có $\heva{&4x-3y \geq 5\\&3-4x+3y>0} \Leftrightarrow \heva{&4x-3y \geq 5\\&-4x+3y>-3} \Leftrightarrow \heva{&4x-3y \geq 5\\&4x-3y<3}$ vô lý, suy ra hệ vô nghiệm.
    }
\end{ex}
%Câu 28
\begin{ex}%[0D2B2-3]
    Cho hệ bất phương trình $\heva{&4x-5y<2\quad(1)\\&2x-\dfrac{5}{2}y<3\quad(2)}$. Gọi $S_1$ là tập nghiệm của bất phương trình $(1)$, $S_2$ là tập nghiệm của bất phương trình $(2)$ và $S$ là tập nghiệm của hệ bất phương trình trên. Khẳng định nào sau đây là khẳng định đúng?
    \choice
    {\True $S_1 \subset S_2$}
    {$S_2 \subset S_1$}
    {$S_2=S$}
    {$S_1 \ne S$}
    \loigiai{
        $\heva{&4x-5y<2\\&2x-\dfrac{5}{2}y<3} \Leftrightarrow \heva{&4x-5y<2\quad(1)\\&4x-5y<6\quad(2).}$\\
        Vẽ hai đường thẳng  \[(d_1) \colon 4x-5y=2;\,\,
        (d_2) \colon 4x-5y=6.\]
        \begin{center}
            \begin{tikzpicture}[>=stealth,line join=round,line cap=round,font=\footnotesize,scale=1]
                \tikzset{
                    declare function={x2=3;y2=2;x1=-3;y1=-3;m=4/5;n=-2/5;a=4/5;b=-6/5;},
                    declare function={f(\x)=m*(\x)+n; g(\x)=a*(\x)+b;h1(\x)=2;h2(\x)=-2;}
                }
                %   \tikzset{declare function = {func(\x) = \x^2;}}
                
                \clip (x1,y1)rectangle (x2,y2);
                \draw[fill=black](0,0)circle(1.2pt)node[above left]{$O$};
                %%%%%%%%%%%%%%%%%%%%%%%%%%%%%%%%
                \draw [smooth,samples=100,domain=x1:x2]plot(\x,{f(\x)})node[shift={(-.4,-.1)}]{$d_1$};
                %%%%%%%%%%%%%%%%%%%%%%%%%%%%%%%%
                \fill[pattern=north west lines,opacity=.4](x1,{f(x1)})--(x1,y1)--(x2,y1)--(x2,{f(x2)})--cycle;
                %%%%%%%%%%%%%%%%%%%%%%%%%%%%%%%%
                \draw [smooth,samples=100,domain=x1:x2]plot(\x,{g(\x)})node[shift={(-.2,.1)}]{$d_2$};
                %%%%%%%%%%%%%%%%%%%%%%%%%%%%%%%%
                \fill[pattern=north west lines,opacity=.4];
                
                %%%%%%%%%%%%%%%%%%%%%%%%%%%%%%%%
                \draw[thick,->] (x1,0)--(x2,0)node[above left]{$x$};
                \draw[thick,->] (0,y1)--(0,y2)node[below right]{$y$};
                \draw[fill=black]
                (1/2,0)circle(1.3pt)+(90:3.5mm)node{$\tfrac{1}{2}$}                 (3/2,0)circle(1.3pt)+(90:3.5mm)node{$\tfrac{3}{2}$}
                (0,-2/5)circle(1.3pt)+(-10:3.5mm)node{$-\tfrac{3}{2}$}
                (0,-6/5)circle(1.3pt)+(-10:3.5mm)node{$-\tfrac{6}{5}$}
                ;       
            \end{tikzpicture}
        \end{center}
        Ta thấy $(0;0)$ là nghiệm của cả hai bất phương trình $(1)$ và $(2)$, nghĩa là gốc tọa độ thuộc cả hai miền nghiệm của hai bất phương trình đó.\\
        Sau khi gạch bỏ các miền không thích hợp, miền không bị gạch là miền nghiệm của hệ bất phương trình.\\
        Ta suy ra $S_1 \subset S_2$.
    }
\end{ex}
%Câu 29
\begin{ex}%[0D2K2-3]
    \immini[thm]{Miền nghiệm của hệ bất phương trình $\heva{&y-2x \leq 2\\&2y-x \geq 4\\&x+y \leq 5}$ là miền tam giác $ABC$ (như hình vẽ).
    Tìm giá trị lớn nhất của biểu thức $F=3x+y$, với $(x;y)$ là nghiệm của hệ bất phương trình trên.
    \choice
    {$2$}
    {\True $9$}
    {$7$}
    {$10$}}
    {\begin{tikzpicture}[>=stealth,line join=round,line cap=round,font=\footnotesize,scale=0.6]
            \tikzset{
                declare function={x2=3;y2=6;x1=-3;y1=-1;m=2;n=2;a=1/2;b=2;p=-1;q=5;},
                declare function={f(\x)=m*(\x)+n; g(\x)=a*(\x)+b;h(\x)=p*(\x)+q;}
            }
            \clip (x1,y1)rectangle (x2,y2);
            \draw[fill=black](0,0)circle(1.2pt)node[above left]{$O$};
            %%%%%%%%%%%%%%%%%%%%%%%%%%%%%%%%
            \draw [smooth,samples=100,domain=x1:x2]plot(\x,{f(\x)});
            %%%%%%%%%%%%%%%%%%%%%%%%%%%%%%%%
            \fill[pattern=north west lines,opacity=.4](x2,{f(x2)})--(x1,y2)--(x1,y1)--(x1,{f(x1)});
            %%%%%%%%%%%%%%%%%%%%%%%%%%%%%%%%
            \draw [smooth,samples=100,domain=x1:x2]plot(\x,{g(\x)});
            %%%%%%%%%%%%%%%%%%%%%%%%%%%%%%%%
            \fill[pattern=north west lines,opacity=.4](x1,{g(x1)})--(x1,y1)--(x2,y1)--(x2,{g(x2)});
            %%%%%%%%%%%%%%%%%%%%%%%%%%%%%%%%
            \draw [smooth,samples=100,domain=x1:x2]plot(\x,{h(\x)});
            %%%%%%%%%%%%%%%%%%%%%%%%%%%%%%%%
            \fill[pattern=north west lines,opacity=.4](x1,{h(x1)})--(x2,y2)--(x2,{h(x2)})--cycle;   
            %%%%%%%%%%%%%%%%%%%%%%%%%%%%%%%%
            \draw[thick,->] (x1,0)--(x2,0)node[above left]{$x$};
            \draw[thick,->] (0,y1)--(0,y2)node[below right]{$y$};
            \draw[fill=black]
            (1,4)circle(1.3pt)+(90:3.5mm)node{$B$}                  
            (2,3)circle(1.3pt)+(90:3.5mm)node{$C$}
            (0,2)circle(1.3pt)+(-10:3.5mm)node{$A$}
            ;       
            \foreach \p/\goc in {-2/-90,2/-90,1/-90}{
                \draw[fill=black](\p,0)circle(1.3pt)+(\goc:2.5mm)node{$\p$};
            }
            \foreach \p/\goc in {2/170,5/180}{
                \draw[fill=black](0,\p)circle(1.3pt)+(\goc:2.8mm)node{$\p$};
            }
            
        \end{tikzpicture}}
    \loigiai{
        Miền nghiệm của hệ bất phương trình đã cho là miền tam giác $ABC$ với $A(0; 2)$, $B(1; 4)$, $C(2; 3)$.\\
        Người ta chứng minh được biểu thức $F=3x+y$ có giá trị lớn nhất tại một trong các đỉnh của tam giác $ABC$.\\
        Tại $A(0; 2)$ thì $F=2$.\\
        Tại $B(1; 4)$ thì $F=7$\\
        Tại $C(2; 3)$ thì $F=9$.\\
        Vậy giá trị lớn nhất của $F=9$ khi $x=2$, $y=3$.
    }
\end{ex}
% Câu 30
\begin{ex}%[0D2K2-3]
    \immini[thm]{Miền nghiệm của hệ bất phương trình $\heva{&2x+y \geq 14\\&2x+5y \geq 30\\&0 \leq x \leq 10\\&0 \leq y \leq 9}$ là miền tứ giác $ABCD$ (như hình vẽ). Tìm giá trị nhỏ nhất của biểu thức $F=x+6y$, với $(x;y)$ là nghiệm của hệ bất phương trình trên.
    \choice
    {$29$}
    {$64$}
    {\True $22$}
    {$20$}}
    {\begin{tikzpicture}[>=stealth,line join=round,line cap=round,font=\footnotesize,scale=0.5]
            \tikzset{
                declare function={x2=11;y2=10;x1=-1.5;y1=-1;m=-2;n=14;a=-2/5;b=6;p=-1;q=5;k=9;},
                declare function={f(\x)=m*(\x)+n; g(\x)=a*(\x)+b;h(\x)=p*(\x)+q;f1(\x)=k;}
            }
            \clip (x1,y1)rectangle (x2,y2);
            \draw[fill=black](0,0)circle(1.2pt)node[above left]{$O$};
            %%%%%%%%%%%%%%%%%%%%%%%%%%%%%%%%
            \draw [smooth,samples=100,domain=x1:x2]plot(\x,{f(\x)});
            %%%%%%%%%%%%%%%%%%%%%%%%%%%%%%%%
            \fill[pattern=north west lines,opacity=.4](x1,{f(x1)})--(x1,y2)--(x1,y1)--(x2,{f(x2)});
            %%%%%%%%%%%%%%%%%%%%%%%%%%%%%%%%
            \draw [smooth,samples=100,domain=x1:x2]plot(\x,{g(\x)});
            %%%%%%%%%%%%%%%%%%%%%%%%%%%%%%%%
            \fill[pattern=north west lines,opacity=.4](x1,{g(x1)})--(x1,y1)--(x2,y1)--(x2,{g(x2)});
            %%%%%%%%%%%%%%%%%%%%%%%%%%%%%%%%
            \draw [smooth,samples=100,domain=x1:x2]plot(\x,{f1(\x)});
            %%%%%%%%%%%%%%%%%%%%%%%%%%%%%%%%
            \fill[pattern=north west lines,opacity=.4](x1,{f1(x1)})--(x1,y2)--(x2,y2)--(x2,{f1(x2)});   
            %%%%%%%%%%%%%%%%%%%%%%%%%%%%%%%%
            \draw (10,y1)--(10,y2);
            %%%%%%%%%%%%%%%%%%%%%%%%%%%%%%%%
            \fill[pattern=north west lines,opacity=.4](10,y2)--(x2,y2)--(x2,y1)--(10,y1)--cycle;    
            %%%%%%%%%%%%%%%%%%%%%%%%%%%%%%%%
            \draw[thick,->] (x1,0)--(x2,0)node[above left]{$x$};
            \draw[thick,->] (0,y1)--(0,y2)node[below right]{$y$};
            \draw[fill=black]
            (5,4)circle(1.3pt)+(-120:3.5mm)node{$A$}                    
            (10,2)circle(1.3pt)+(40:3.5mm)node{$B$}
            (10,9)circle(1.3pt)+(45:3.5mm)node{$C$}
            (5/2,9)circle(1.3pt)+(70:3.5mm)node{$D$}
            ;       
            %       \foreach \p/\goc in {-2/-90,2/-90,1/-90}{
                %           \draw[fill=black](\p,0)circle(1.3pt)+(\goc:2.5mm)node{$\p$};
                %       }
            %       \foreach \p/\goc in {2/170,5/180}{
                %           \draw[fill=black](0,\p)circle(1.3pt)+(\goc:2.8mm)node{$\p$};
                %       }
            
        \end{tikzpicture}}
    \loigiai{
        Miền nghiệm của hệ bất phương trình đã cho là miền tứ giác $ABCD$ với $A(5;4),B(10;2)$, $C(10;9),D\left( \dfrac{5}{2};9 \right)$.\\
        Người ta chứng minh được biểu thức $F=x+6y$ có giá trị nhỏ nhất tại một trong các đỉnh của tứ giác $ABCD$.\\
        Tại $A(5;4)$ thì $F=29$.\\
        Tại $B(10;2)$ thì $F=22$.\\
        Tại $C(10;9)$ thì $F=64$.\\
        Tại $D\left( \dfrac{5}{2};9 \right)$ thì $F=\dfrac{113}{2}$.\\
        Vậy giá trị nhỏ nhất của $F=22$ khi $x=10$, $y=2$.
    }
\end{ex}
% Câu 31
\begin{ex}%[0D2K2-3]
    Điểm $M(1;0)$ không thuộc miền nghiệm của hệ bất phương trình nào sau đây?
    \choice
    {$\heva{&-x+3y<0\\&2x+y+4>0}$}
    {$\heva{&x+3y \geq 0\\&2x+y-4<0}$}
    {\True $\heva{&x+3y-6<0\\&2x-y-4>0}$}
    {$\heva{&x+3y-6<0\\&2x+y+4 \geq 0}$}
    \loigiai{
        Thay tọa độ điểm $M(1;0)$ vào hệ ở $\heva{&x+3y-6<0\\&2x-y-4>0}$ ta được $\heva{&1-6<0\\&2-4>0}$ không thỏa.\\
        Vậy điểm $M(1;0)$ không thuộc miền nghiệm của hệ $\heva{&x+3y-6<0\\&2x-y-4>0.}$
    }
\end{ex}
% Câu 32.
\begin{ex}%[0D2K2-3]
    Miền nghiệm của hệ bất phương trình $\heva{&-x+3y<0\\&2x+y+4>0\\&x>0}$ là miền chứa điểm nào trong các điểm sau?
    \choice
    {\True $M(2;-2)$}
    {$N(2;2)$}
    {$P(-2;2)$}
    {$Q(0;2)$}
    \loigiai{
        Với điểm $M(2;-2)$: $\heva{&-2+3 \cdot (-2)<0\\&2 \cdot 2+2+4>0\\&2>0}$ thỏa mãn;\\
        Với điểm $N(2;2)$: bất phương trình thứ nhất của hệ không thỏa;\\
        Với điểm $P(-2;2)$: bất phương trình thứ nhất của hệ không thỏa;\\
        Với điểm $Q(0;2)$: không thỏa $x>0$.\\
        Vậy miền nghiệm của hệ chứa điểm $M(2;-2)$.
    }
\end{ex}
% Câu 33
\begin{ex}%[0D2K2-3]
    Điểm nào trong các điểm sau thuộc miền nghiệm của hệ bất phương trình $\heva{&3x-4y+12 \geq 0\\&x+y-5>0\\&x+1 \geq 0}$?
    \choice
    {\True $M(4;3)$}
    {$N(4;1)$}
    {$P(2;2)$}
    {$Q(2;6)$}
    \loigiai{
        Lần lượt thay tọa độ điểm các điểm ở các phương án vào hệ $\heva{&3x-4y+12 \geq 0\\&x+y-5>0\\&x+1 \geq 0}$, ta được:\\
        Với điểm $M(4;3)$: $\heva{&3 \cdot 4-4 \cdot 3+12 \geq 0\\&4+3-5>0\\&4+1 \geq 0}$ thỏa mãn;\\
        Với điểm $N(4;1)$: bất phương trình thứ hai của hệ không thỏa;\\
        Với điểm $P(2;2)$: bất phương trình thứ hai của hệ không thỏa;\\
        Với điểm $Q(2;6)$: bất phương trình thứ nhất của hệ không thỏa.\\
        Vậy điểm $M(4;3)$ thuộc miền nghiệm của hệ.
    }
\end{ex}
%Câu 34
\begin{ex}%[0D2B2-3] 
    \immini[thm]{Miền không bị gạch chéo (kể cả hai đường thẳng $d, \Delta$) như hình bên dưới là miền nghiệm của hệ bất phương trình nào sau đây?
    \choice
    {$\heva{&x+y-1<0\\&x-y+2>0}$}
    {$\heva{&x+y-1 \geq 0\\&x-y+2 \leq 0}$}
    {\True $\heva{&x+y-1 \leq 0\\&x-y+2 \geq 0}$}
    {$\heva{&x+y-1 \leq 0\\&x-y+2>0}$}}
    { \begin{tikzpicture}[>=stealth,line join=round,line cap=round,font=\footnotesize,scale=0.6]
            \tikzset{
                declare function={x2=3.5;y2=4;x1=-4.8;y1=-1;m=-1;n=1;a=1;b=2;p=-1;q=5;k=9;},
                declare function={f(\x)=m*(\x)+n; g(\x)=a*(\x)+b;h(\x)=p*(\x)+q;f1(\x)=k;}
            }
            \clip (x1,y1)rectangle (x2,y2);
            \draw[fill=black](0,0)circle(1.2pt)node[above left]{$O$};
            %%%%%%%%%%%%%%%%%%%%%%%%%%%%%%%%
            \draw [smooth,samples=100,domain=x1:x2,thick,red]plot(\x,{f(\x)});
            %%%%%%%%%%%%%%%%%%%%%%%%%%%%%%%%
            \fill[pattern=north west lines,opacity=.4](x1,{f(x1)})--(x2,y2)--(x2,y1)--(x2,{f(x2)});
            %%%%%%%%%%%%%%%%%%%%%%%%%%%%%%%%
            \draw [smooth,samples=100,domain=x1:x2,thick,green]plot(\x,{g(\x)});
            %%%%%%%%%%%%%%%%%%%%%%%%%%%%%%%%
            \fill[pattern=north west lines,opacity=.4](x2,{g(x2)})--(x1,y2)--(x1,y1)--(x1,{g(x1)});
            %%%%%%%%%%%%%%%%%%%%%%%%%%%%%%%%
            \draw[thick,->] (x1,0)--(x2,0)node[above left]{$x$};
            \draw[thick,->] (0,y1)--(0,y2)node[below right]{$y$};
            
            \foreach \p/\goc in {-2/90,1/-90}{
                \draw[fill=black](\p,0)circle(1.3pt)+(\goc:2.5mm)node{$\p$};
            }
            \foreach \p/\goc in {2/180,1/180}{
                \draw[fill=black](0,\p)circle(1.3pt)+(\goc:2.8mm)node{$\p$};
            }
            \node at (-2.3,4)[shift={(-.8,-1)},scale=.8]{$d\colon x+y-1=0$};
            \node at (2.8,3)[shift={(-.8,-.5)},scale=.8]{$\Delta\colon x-y+2=0$};       
        \end{tikzpicture}}
    \loigiai{
        
        Nhận thấy miền không bị gạch chéo có chứa gốc tọa độ $O(0;0)$ thỏa mãn hệ $\heva{&x+y-1 \leq 0\\&x-y+2 \geq 0.}$
    }
\end{ex}
% Câu 35
\begin{ex}%[0D2K2-3]
    \immini[thm]{Trên miền tứ giác $OABC$, phần không bị gạch sọc như hình vẽ bên dưới. Giá trị lớn nhất của biểu thức $F=2x+3y+2022$ bằng
    \choice
    {$2022$}
    {$2038$}
    {\True $2040$}
    {$4044$}}
    {\begin{tikzpicture}[>=stealth,line join=round,line cap=round,font=\footnotesize,scale=0.5]
            \tikzset{
                declare function={x2=11;y2=10;x1=-6.5;y1=-2;m=-1;n=8;a=-2/3;b=6;p=-1;q=5;k=9;},
                declare function={f(\x)=m*(\x)+n; g(\x)=a*(\x)+b;h(\x)=p*(\x)+q;f1(\x)=k;}
            }
            
            \clip (x1,y1)rectangle (x2,y2);
            \draw[fill=black](0,0)circle(1.2pt)node[above left]{$O$};
            %%%%%%%%%%%%%%%%%%%%%%%%%%%%%%%%
            \draw [smooth,samples=100,domain=-2:x2]plot(\x,{f(\x)});
            %%%%%%%%%%%%%%%%%%%%%%%%%%%%%%%%
            \fill[pattern=north west lines,opacity=.2](x1,{f(x1)})--(x1,y2)--(x2,y2)--(x2,y1)--(x2,{f(x2)});
            %%%%%%%%%%%%%%%%%%%%%%%%%%%%%%%%
            \draw [smooth,samples=100,domain=x1:x2]plot(\x,{g(\x)});
            %%%%%%%%%%%%%%%%%%%%%%%%%%%%%%%%
            \fill[pattern=north west lines,opacity=.2](x1,{g(x1)})--(x1,y2)--(x2,y2)--(x2,y1)--(x2,{g(x2)});
            \fill[pattern=north west lines,opacity=.2](0,y2)--(x1,y2)--(x1,y1)--(0,y1)--cycle;
            \fill[pattern=north west lines,opacity=.2](x1,0)--(x1,y1)--(x2,y1)--(x2,0)--cycle;
            %%%%%%%%%%%%%%%%%%%%%%%%%%%%%%%%
            \draw[thick,->] (x1,0)--(x2,0)node[above left]{$x$};
            \draw[thick,->] (0,y1)--(0,y2)node[below right]{$y$};
            \draw[fill=black]
            (8,0)circle(1.3pt)+(100:3.5mm)node{$A$}                 
            (6,2)circle(1.3pt)+(40:3.5mm)node{$B$}
            (0,6)circle(1.3pt)+(45:3.5mm)node{$C$}
            ;
            \draw[dashed](0,2)-|(6,0);
            \node at (-2,10)[shift={(-.5,-.15)}]{$x+y=8$};
            \node at (-2,8)[shift={(-1,-.5)}]{$2x+3y=18$};
            \foreach \p/\goc in {6/-90,8/-90,9/90}{
                \draw[fill=black](\p,0)circle(1.3pt)+(\goc:2.5mm)node{$\p$};
            }
            \foreach \p/\goc in {8/10,6/190,2/180}{
                \draw[fill=black](0,\p)circle(1.3pt)+(\goc:2.8mm)node{$\p$};
            }
            
        \end{tikzpicture}}
    \loigiai{
        Tứ giác $OABC$ có tọa độ các đỉnh là $O(0;0)$, $A(8;0)$, $B(6;2)$, $C(0;6)$.\\
        Đặt $F(x,y)=2x+3y+2022=F$.\\
        Tại đỉnh $O(0;0)$, ta có $F(0,0)=2022$;\\
        Tại đỉnh $A(8;0)$, ta có $F(8,0)=2038$;\\
        Tại đỉnh $B(6;2)$, ta có $F(6,2)=2040$;\\
        Tại đỉnh $C(0;6)$, ta có $F(0,6)=2040$.\\
        Vậy biểu thức $F=2x+3y+2022$ đạt giá trị lớn nhất bằng $2040$ đạt được tại $B(6;2)$ hoặc tại $C(0;6)$.
    }
\end{ex}
% Câu 36
\begin{ex}%[0D2K1-1]
    Tìm số nghiệm nguyên dương $(x;y)$ của bất phương trình $\dfrac{x}{3}+\dfrac{y}{4} \leq 1$.
    \choice
    {\True $3$}
    {$4$}
    {$5$}
    {$6$}
    \loigiai{
        Do $x>0$ và $\dfrac{x}{3}+\dfrac{y}{4} \leq 1$ nên ta có $\dfrac{y}{4}<1 \Leftrightarrow y<4$.\\
        Mà $y$ là số nguyên dương nên $y \in \big\{1;2;3\big\}$.
    \begin{enumerate}[--]
        \item Với $y=1$ ta có $0<\dfrac{x}{3} \leq \dfrac{3}{4} \Leftrightarrow 0<x \leq \dfrac{9}{4} \Leftrightarrow x \in \big\{1;2 \big\}$.
        \item Với $y=2$ ta có $0<\dfrac{x}{3} \leq \dfrac{1}{2} \Leftrightarrow 0<x \leq \dfrac{3}{2} \Leftrightarrow x \in \big\{1 \big\}$.
        \item Với $y=3$ ta có $0<\dfrac{x}{3} \leq \dfrac{1}{4} \Leftrightarrow 0<x \leq \dfrac{3}{4} \Leftrightarrow x \in \varnothing$.   
    \end{enumerate}
        Vậy bất phương trình có các nghiệm nguyên dương là $(1;1),(2;1),(1;2)$.
    }
\end{ex}
% Câu 37
\begin{ex}%[0D2K1-1]
    Tìm số các giá trị nguyên của tham số $m \in [-2022;2022]$ sao cho $\heva{&x=-1\\&y=2}$ là nghiệm của bất phương trình $mx+(m-1)y>2$.
    \choice
    {$2022$}
    {$2000$}
    {\True $2018$}
    {$2016$}
    \loigiai{
        Ta có $\heva{&x=-1\\&y=2}$ là nghiệm của bất phương trình $mx+(m-1)y>2$ 
        nên \[-m+2(m-1)>2 \Leftrightarrow m>4.\]
        Mà $m \in [-2022;2022] \Leftrightarrow -2022 \leq m \leq 2022$ nên $4<m \leq 2022$.\\
        Do $m \in \mathbb{Z}$ nên $m \in \big\{5;6;7;\ldots;2022 \big\}$\\
        Số các giá trị nguyên của tham số $m$ thỏa mãn đề là $2022-5+1=2018$ (số).\\
    }
\end{ex}

% Câu 38
\begin{ex}%[0D2K1-1]
    Cho $2$ bất phương trình sau
    \allowdisplaybreaks
    \begin{eqnarray*}
        &&2x+3y<5\qquad(1)\\
        &&x+\dfrac{3}{2}y<5.\qquad(2)
    \end{eqnarray*}
    Gọi $S_1$ là tập nghiệm của bất phương trình $(1)$, $S_2$ là tập nghiệm của bất phương trình $(2)$. Khẳng định nào sau đây là đúng?
    \choice
    {$S_2=S_1$}
    {$S_2 \subset S_1$}
    {$S_2 \cap S_1=\varnothing$}
    {\True $S_1 \subset S_2$}
    \loigiai{
    \begin{enumerate}[--]
        \item Trước hết, ta vẽ hai đường thẳng $(d_1) \colon 2x+3y=5$ và $(d_2) \colon x+\dfrac{3}{2}y=5$.\\
        Đường thẳng $(d_1) \colon 2x+3y=5$ cắt trục hoành tại điểm $A(\dfrac{5}{2};0)$ và cắt trục tung tại điểm $B(0;\dfrac{5}{3})$\\
        Đường thẳng: $(d_2) \colon x+\dfrac{3}{2}y=5$ cắt trục hoành tại điểm $C(5;0)$ và cắt trục tung tại điểm $D(0;\dfrac{10}{3})$.
        \begin{center}
                \begin{tikzpicture}[>=stealth,line join=round,line cap=round,font=\footnotesize,scale=0.7]
                \tikzset{
                    declare function={x2=6;y2=5;x1=-4.5;y1=-2;m=-2/3;n=5/3;a=-2/3;b=10/3;p=-1;q=5;k=9;},
                    declare function={f(\x)=m*(\x)+n; g(\x)=a*(\x)+b;h(\x)=p*(\x)+q;f1(\x)=k;}
                }
                
                \clip (x1,y1)rectangle (x2,y2);
                \draw[fill=black](0,0)circle(1.2pt)node[above left]{$O$};
                %%%%%%%%%%%%%%%%%%%%%%%%%%%%%%%%
                \draw [smooth,samples=100,domain=x1:x2]plot(\x,{f(\x)});
                %%%%%%%%%%%%%%%%%%%%%%%%%%%%%%%%
                \fill[pattern=north west lines,opacity=.2](x1,{f(x1)})--(x1,y2)--(x2,y2)--(x2,y1)--(x2,{f(x2)});
                %%%%%%%%%%%%%%%%%%%%%%%%%%%%%%%%
                \draw [smooth,samples=100,domain=x1:x2]plot(\x,{g(\x)});
                %%%%%%%%%%%%%%%%%%%%%%%%%%%%%%%%
                \fill[pattern=north west lines,opacity=.2](x1,{g(x1)})--(x1,y2)--(x2,y2)--(x2,y1)--(x2,{g(x2)});
                
                %%%%%%%%%%%%%%%%%%%%%%%%%%%%%%%%
                \draw[thick,->] (x1,0)--(x2,0)node[above left]{$x$};
                \draw[thick,->] (0,y1)--(0,y2)node[below right]{$y$};
                \draw[fill=black]
                (5,0)circle(1.3pt)+(100:3.5mm)node[scale=.8,below]{$5$} 
                (5/2,0)circle(1.3pt)+(100:3.5mm)node[below=2mm,scale=.8]{$\dfrac{5}{2}$}                                    
                (0,10/3)circle(1.3pt)+(40:3.5mm)node[scale=.8]{$\dfrac{10}{3}$}
                (0,5/3)circle(1.3pt)+(45:3.5mm)node[scale=.8]{$\dfrac{5}{3}$}
                ;
                \node at (-1.5,5)[shift={(-.5,-.5)}]{$(d_2)$};
                \node at (-2,4.5)[shift={(-.5,-1)}]{$(d_1)$};
                
            \end{tikzpicture}
        \end{center}
        \item Ta thấy $(0;0)$ là nghiệm của cả hai bất phương trình. Điều đó có nghĩa gốc tọa độ thuộc cả hai miền nghiệm của hai bất phương trình. Sau khi gạch bỏ các miền không thích hợp, miền không bị gạch là miền nghiệm của $2$ bất phương trình đã cho.
        \item Nhìn vào hình vẽ ta thấy, tập nghiệm của bất phương trình $(1)$ nằm trong tập nghiệm của bất phương trình $(2)$.      
    \end{enumerate}
    }
\end{ex}
% Câu 39
\begin{ex}%[0D2T1-1]
    Ông Minh trồng hai loại hoa gồm hoa hồng và hoa ly để bán vào dịp Tết Nguyên Đán. Hoa hồng có giá $80000$ đồng/chậu và hoa ly có giá 120000 đồng/chậu. Ông Minh tính toán rằng, để không phải bù lỗ thì số tiền bán hoa thu được phải đạt tối thiểu $30$ triệu đồng. Gọi $x$ và $y$ lần lượt là số chậu hoa hồng và hoa ly bán được. Hỏi $x$ và $y$ thỏa mãn điều kiện nào dưới đây thì ông Minh không phải bù lỗ.
    \choice
    {$2x+3y>750$}
    {\True $2x+3y \geq 750$}
    {$3x+2y \geq 750$}
    {$3x+2y \leq 750$}
    \loigiai{
        \begin{enumerate}[--]
        \item Ta có $x$ và $y$ lần lượt là số chậu hoa hồng và hoa ly bán được. Điều kiện $x,y \in \mathbb{N}$.
        Khi đó số tiền bán hoa thu được là $80000x+120000y$ (đồng).\\
        \item Để \textbf{không} phải bù lỗ thì số tiền bán hoa thu được phải đạt tối thiểu 30 triệu đồng nên
        \[80000x+120000y \geq 30000000
        \Leftrightarrow 2x+3y \geq 750\]    
        \end{enumerate}
    }
\end{ex}
% Câu 40
\begin{ex}%[0D2T1-1]
    Một cửa hàng dự định kinh doanh hai loại áo loại $I$ và loại $II$ với số vốn nhập hàng nhỏ hơn $70$ triệu đồng. Giá mua vào một chiếc áo loại $I$ và loại $II$ lần lượt là $70$ nghìn đồng, $140$ nghìn đồng. Hỏi cửa hàng có thể nhập tối đa bao nhiêu áo loại $I$? Biết rằng số lượng áo loại $II$ nhập nhiều hơn áo loại $I$?
    \choice
    {\True $333$}
    {$334$}
    {$335$}
    {$332$}
    \loigiai{
    \begin{enumerate}[--]
        \item Gọi $x$ và $y$ lần lượt là số lượng chiếc áo loại $I$ và loại $II$ được nhập. Điều kiện $x,y \in \mathbb{N}$.\\
        Khi đó số tiền cần để nhập hàng là $70000x+140000y$ (đồng).
        \item Do số vốn nhập hàng nhỏ hơn $70$ triệu đồng nên
        \[70000x+140000y<70000000\Leftrightarrow x+2y<1000.\qquad (1)\]
        \item Mặt khác số lượng áo loại $II$ nhập nhiều hơn áo loại $I$ nên $y>x$.\\
        Suy ra $x+2y>3x$ $(2)$.\\
        Từ $(1)$ và $(2)$ ta được $3x<1000 \Leftrightarrow x<\dfrac{1000}{3} \approx 333,3$.\\
        Vậy cửa hàng có thể nhập tối đa $333$ áo loại $I$.      
    \end{enumerate}
    }
\end{ex}
% Câu 41
\begin{ex}%[0D2T2-1]
    Một hộ nông dân cần không quá $180$ ngày công để trồng đậu và trồng cà trên diện tích $8$ ha. Nếu trồng đậu thì cần $20$ ngày công và thu lợi nhuận $3$ triệu đồng trên diện tích mỗi ha, nếu trồng cà thì cần $30$ ngày công và thu lợi nhuận $4$ triệu đồng trên diện tích mỗi ha. Lợi nhuận cao nhất mà hộ nông dân thu được khi trồng đậu và trồng cà trên mảnh đất đó là
    \choice
    {$32$ triệu đồng}
    {$24$ triệu đồng}
    {\True $26$ triệu đồng}
    {$36$ triệu đồng}
    \loigiai{
        Gọi $x,y$ lần lượt là số ha trồng đậu và trồng cà của hộ nông dân (Điều kiện $x,y \geq 0$).\\
        Số ngày công trồng đậu và cà của hộ nông dân là $20x+30y$.\\
        Vì có tổng diện tích là $8$ ha trồng đậu và cà nên ta có bất phương trình $x+y \leq 8$.\\
        Vì tổng số ngày công không vượt quá $180$ nên ta có bất phương trình $20x+30y \leq 180$ hay $2x+3y \leq 18$.\\
        Khi đó ta có hệ bất phương trình $\heva{&x \geq 0\\&y \geq 0\\&x+y \leq 8\\&2x+3y \leq 18}$. Hệ bất phương trình có miền nghiệm là tứ giác $OBCD$ với $O(0;0),B(0;6),C(6;2)$ và $D(8;0)$ (như hình vẽ bên dưới).
        \begin{center}
                \begin{tikzpicture}[>=stealth,line join=round,line cap=round,font=\footnotesize,scale=0.7]
                \tikzset{
                    declare function={x2=10;y2=9;x1=-4.5;y1=-2;m=-1;n=8;a=-2/3;b=6;p=-1;q=5;k=9;},
                    declare function={f(\x)=m*(\x)+n; g(\x)=a*(\x)+b;h(\x)=p*(\x)+q;f1(\x)=k;}
                }
                
                \clip (x1,y1)rectangle (x2,y2);
                \draw[fill=black](0,0)circle(1.2pt)node[above left]{$O$};
                %%%%%%%%%%%%%%%%%%%%%%%%%%%%%%%%
                \draw [smooth,samples=100,domain=x1:x2]plot(\x,{f(\x)});
                %%%%%%%%%%%%%%%%%%%%%%%%%%%%%%%%
                \fill[pattern=north west lines,opacity=.2](x1,{f(x1)})--(x1,y2)--(x2,y2)--(x2,y1)--(x2,{f(x2)});
                %%%%%%%%%%%%%%%%%%%%%%%%%%%%%%%%
                \draw [smooth,samples=100,domain=x1:x2]plot(\x,{g(\x)});
                %%%%%%%%%%%%%%%%%%%%%%%%%%%%%%%%
                \fill[pattern=north west lines,opacity=.2](x1,{g(x1)})--(x1,y2)--(x2,y2)--(x2,y1)--(x2,{g(x2)});
                
                %%%%%%%%%%%%%%%%%%%%%%%%%%%%%%%%
                \fill[pattern=north west lines,opacity=.2](0,y2)--(x1,y2)--(x1,y1)--(0,y1)--cycle;
                \fill[pattern=north west lines,opacity=.2](x1,0)--(x1,y1)--(x2,y1)--(x2,0)--cycle;
                %%%%%%%%%%%%%%%%%%%%%%%%%%%%%%%%
                \draw[thick,->] (x1,0)--(x2,0)node[above left]{$x$};
                \draw[thick,->] (0,y1)--(0,y2)node[below right]{$y$};
                \draw[fill=black]
                (9,0)circle(1.3pt)+(-90:2.5mm)node[scale=.8]{$9$}   
                (8,0)circle(1.3pt)+(-90:3.5mm)node[scale=.8]{$8$}                           (0,8)circle(1.3pt)+(10:3.5mm)node[scale=.8]{$8$}        
                (0,6)circle(1.3pt)+(40:3.5mm)node[scale=.8]{$B$}node[scale=.8, below left=2.5mm]{$6$}
                (6,2)circle(1.3pt)+(45:3.5mm)node[scale=.8]{$C$}
                ;
                \node at (-1.5,8.5)[shift={(.25,-0.1)}]{$(d_2)$};
                \node at (-2,6.5)[shift={(-.5,-0.1)}]{$(d_1)$};
            \end{tikzpicture}
        \end{center}
        Lợi nhuận thu được khi trồng $x$ ha đậu và $y$ ha cà là $F(x;y)=3x+4y$.\\
        Ta thấy $F(0;0)=0$, $F(0;6)=24$, $F(6;2)=26$ và $F(8;0)=24$ nên lợi nhuận nhiều nhất là $26$ triệu đồng khi trồng $6$ ha đậu và $2$ ha cà.
    }
\end{ex}
% Câu 42
\begin{ex}%[0D2T1-1]
    Một câu lạc bộ CKTU của trường Chuyên Kon Tum có $5$ thành viên và mỗi người chỉ làm việc tối đa trong $5$ giờ để dự định làm tối thiểu $220$ tấm thiệp gửi lời chúc mừng đến các em học sinh lớp $10$ đầu năm học mới. Cần $5$ phút để một người làm một tấm thiệp loại A với chi phí $2$ $000$ đồng và cần $9$ phút để một người làm một tấm thiệp loại B với chi phí $1$ $500$ đồng. Hỏi Câu lạc bộ làm bao nhiêu tấm thiệp loại A và bao nhiêu tấm thiệp loại B để tốn chi phí thấp nhất?
    \choice
    {$100$ tấm thiệp loại A, $120$ tấm thiệp loại B}
    {\True $120$ tấm thiệp loại A, $100$ tấm thiệp loại B}
    {$220$ tấm thiệp loại A, $0$ tấm thiệp loại B}
    {$0$ tấm thiệp loại A, $220$ tấm thiệp loại B}
    \loigiai{
        Gọi $x,y$ lần lượt là số tấm thiệp loại A và số tấm thiệp loại B mà câu lạc bộ đã làm (Điều kiện $x,y \geq 0$).\\
        Thời gian làm xong số thiệp là $5x+9y$ phút.\\
        Vì phải làm tối thiểu $220$ tấm thiệp nên ta có bất phương trình $x+y \geq 220$.\\
        Vì tổng thời gian của các em học sinh không vượt quá 1500 phút nên ta có bất phương trình $5x+9y \leq 1500$.\\
        Khi đó ta có hệ bất phương trình $\heva{&x \geq 0\\&y \geq 0\\&x+y \geq 220\\&5x+9y \leq 1500}$. Hệ bất phương trình có miền nghiệm là tam giác $ABC$ với $A(120;100),B(300;0)$ và $C(220;0)$ (như hình vẽ bên dưới).
        \begin{center}
                \begin{tikzpicture}[>=stealth,line join=round,line cap=round,font=\footnotesize,scale=1.3]
                    \tikzset{
                        declare function={x2=4;y2=3;x1=-.5;y1=-.5;m=-1;n=2.2;a=-5/9;b=5/3;p=-1;q=5;k=9;},
                        declare function={f(\x)=m*(\x)+n; g(\x)=a*(\x)+b;h(\x)=p*(\x)+q;f1(\x)=k;}
                    }
                    
                    \clip (x1,y1)rectangle (x2,y2);
                    \draw[fill=black](0,0)circle(1.2pt)node[above left]{$O$};
                    %%%%%%%%%%%%%%%%%%%%%%%%%%%%%%%%
                    \draw [smooth,samples=100,domain=x1:x2]plot(\x,{f(\x)});
                    %%%%%%%%%%%%%%%%%%%%%%%%%%%%%%%%
                    \fill[pattern=north west lines,opacity=.2](x1,{f(x1)})--(x1,y2)--(x2,y2)--(x2,y1)--(x2,{f(x2)});
                    %%%%%%%%%%%%%%%%%%%%%%%%%%%%%%%%
                    \draw [smooth,samples=100,domain=x1:x2]plot(\x,{g(\x)});
                    %%%%%%%%%%%%%%%%%%%%%%%%%%%%%%%%
                    \fill[pattern=north west lines,opacity=.2](x1,{g(x1)})--(x1,y2)--(x2,y2)--(x2,y1)--(x2,{g(x2)});
                    
                    %%%%%%%%%%%%%%%%%%%%%%%%%%%%%%%%
                    \fill[pattern=north west lines,opacity=.2](0,y2)--(x1,y2)--(x1,y1)--(0,y1)--cycle;
                    \fill[pattern=north west lines,opacity=.2](x1,0)--(x1,y1)--(x2,y1)--(x2,0)--cycle;
                    %%%%%%%%%%%%%%%%%%%%%%%%%%%%%%%%
                    \draw[thick,->] (x1,0)--(x2,0)node[above left]{$x$};
                    \draw[thick,->] (0,y1)--(0,y2)node[below right]{$y$};
                    \draw[fill=black]
                    (3,0)circle(.8pt)+(-90:1.5mm)node[scale=.8]{$300$}node[scale=.8,above=2mm]{$B$}     
                    (2,0)circle(.8pt)+(-90:1.5mm)node[scale=.8]{$200$}
                    (2.2,0)circle(.8pt)+(-90:1.5mm)node[scale=.8,above=2mm]{$C$}        (0,.5)circle(.8pt)+(180:1.5mm)node[scale=.8]{$50$}      
                    (0,1)circle(.8pt)+(180:1.5mm)node[scale=.8]{$100$}
                    (0,1.5)circle(.8pt)+(180:1.5mm)node[scale=.8]{$150$}
                    (0,2)circle(.8pt)+(180:1.5mm)node[scale=.8]{$200$}
                    (1.2,1)circle(1.3pt)+(45:1.5mm)node[scale=.8]{$A$}
                    ;
                    \node at (-1.5,8.5)[shift={(.25,-0.1)}]{$(d_2)$};
                    \node at (-2,6.5)[shift={(-.5,-0.1)}]{$(d_1)$};
                    
                    
                \end{tikzpicture}
        \end{center}
        Chi phí dùng để làm $x$ tấm thiệp loại A và $y$ tấm thiệp loại B là $F(x;y)=2000x+1500y$ đồng.\\
        Ta thấy $F(220;0)=440000$, $F(300;0)=600000$ và $F(120;100)=390000$ nên chi phí thấp nhất để là các tấm thiệp là 390000 đồng khi làm $120$ tấm thiệp loại A và $100$ tấm thiệp loại B.
    }
\end{ex}
% Câu 43
\begin{ex}%[0D2T2-1]
    Có ba nhóm máy A, B,C dùng để sản xuất ra hai loại sản phẩm I và II. Để sản xuất một đơn vị sản phẩm mỗi loại phải lần lượt dùng các máy thuộc các nhóm khác nhau. Số máy trong một nhóm và số máy của từng nhóm cần thiết để sản xuất ra một đơn vị sản phẩm thuộc mỗi loại được cho trong bảng sau
    \begin{center}
        \begin{tikzpicture}[>=stealth,line join=round,line cap=round,scale=.7]
            \draw (0.25,-.5)node{Nhóm} (4,-.25)node{Số máy trong } (10,0.1)node{Số máy trong từng nhóm để} (4,-.75)node{mỗi nhóm} (10,-0.4)node{sản xuất ra một đơn vị sản phẩm}
            (8,-1.25)node{Loại $I$} (12,-1.25)node{Loại $II$} 
            (0.25,-2)node{$A$} (0.25,-3)node{$B$} (0.25,-4)node{$C$}
            (4,-2)node{$10$} (4,-3)node{$4$} (4,-4)node{$12$}
            (8,-2)node{$2$} (8,-3)node{$0$} (8,-4)node{$2$}
            (12,-2)node{$2$} (12,-3)node{$2$} (12,-4)node{$4$};
            \draw (-1.5,.5)--(14,0.5)--(14,-4.5)--(-1.5,-4.5)--cycle (2,0.5)--(2,-4.5) (6,.5)--(6,-4.5) (10,-.75)--(10,-4.5) (-1.5,-1.5)--(14,-1.5) (6,-.75)--(14,-.75) (-1.5,-2.5)--(14,-2.5) (-1.5,-3.5)--(14,-3.5);
        \end{tikzpicture}
    \end{center}
    Một đơn vị sản phẩm loại I lãi ba triệu đồng, một đơn vị sản phẩm loại II lãi năm triệu đồng. Lãi suất cao nhất mà đơn vị thu được là
    \choice
    {$10$ triệu đồng}
    {$15$ triệu đồng}
    {$16$ triệu đồng}
    {\True $17$ triệu đồng}
    \loigiai{
        Gọi $x,y$ lần lượt là số sản phẩm loại I và số sản phẩm lại II được sản xuất (Điều kiện $x,y \geq 0$).\\
        Số máy loại A cần để sản xuất không vượt quá $10$ nên $2x+2y \leq 10$ hay $x+y \leq 5$.\\
        Số máy loại B cần để sản xuất không vượt quá $4$ nên $0x+2y \leq 4$ hay $y \leq 2$.\\
        Số máy loại C cần để sản xuất không vượt quá $12$ nên $2x+4y \leq 12$ hay $x+2y \leq 6$.\\
        Vì số máy của mỗi nhóm được cho chi tiết trong bảng nên ta có hệ bất phương trình
        $\heva{&x \geq 0\\&y \geq 0\\&x+y \leq 5\\&y \leq 2\\&x+2y \leq 6}$.\\
        Hệ bất phương trình có miền nghiệm là ngũ giác $OBCDA$ với $O(0;0),B(0;2)$ $,C(2;2),D(4;1)$ và $A(5;0)$ (như hình vẽ bên dưới).
        \begin{center}
            \begin{tikzpicture}[line join=round, line cap=round,>=stealth,thick]
                \tikzset{label style/.style={font=\footnotesize}}
                \begin{scope}
                    \clip (-1,-1) rectangle (8,6);
                    \fill[pattern=north east lines,opacity=.4] (-2,7)--(9,7)--(9,-4)--cycle;
                    \fill[pattern=north east lines,opacity=.4] (-1,2)--(-1,6)--(8,6)--(8,2)--cycle;
                    \fill[pattern=north east lines,opacity=.4] (-7,6.5)--(9,6.5)--(9,-1.5)--cycle;
                    \fill[pattern=north east lines,opacity=.4] (0,-1)--(-1,-1)--(-1,6)--(0,6)--cycle;
                    \fill[pattern=north east lines,opacity=.4] (-1,0)--(-1,-1)--(8,-1)--(8,0)--cycle;
                    \draw (-1,6)--(6,-1) node [pos=0.45, above, sloped] {$2x+2y=10$};
                    \draw (-1,2)--(8,2) node [pos=0.7, above, sloped] {$2y=4$};
                    \draw (-6,6)--(8,-1) node [pos=0.45, above, sloped] {$2x+4y=12$};
                \end{scope}
                \draw[->] (-1,0)--(8,0) node[below]{$x$};
                \draw[->] (0,-1)--(0,6) node[left]{$y$};
                \draw (0,0) node[below left]{$O$} (0,2)node[below left]{$2$} (0,2)node[below right]{$C$} (2,2)node[below]{$B$} (4,1)node[below]{$A$} (5,0)node[above]{$D$} (5,0)node[below]{$5$};
            \end{tikzpicture}
        \end{center}
        Lợi nhuận thu được khi sản xuất $x$ sản phẩm loại I và $y$ sản phẩm loại II là $F(x;y)=3x+5y$.\\
        Ta thấy $F(0;0)=0$, $F(0;2)=10$, $F(2;2)=16$, $F(4;1)=17$ và $F(5;0)=15$ nên lợi nhuận thu được nhiều nhất là $17$ triệu đồng khi sản xuất $4$ sản phẩm loại I và $1$ sản phẩm loại II.
    }
\end{ex}
% Câu 44
\begin{ex}%[0D2K2-3]
    Gọi $(S)$ là tập hợp các điểm trong mặt phẳng tọa độ có tọa độ thỏa mãn hệ $\heva{&y-2x \leq 2\\&2y-x \geq 4\\&x+y \leq 5}$. Trong $(S)$ điểm có tọa độ $(x,y)$ làm cho biểu thức $F(x;y)=y-x$ đạt giá trị nhỏ nhất là
    \choice
    {\True $(2; 3)$}
    {$(1; 4)$}
    {$(2; 0)$}
    {$(4; 1)$}
    \loigiai{
        Miền nghiệm của hệ $\heva{&y-2x \leq 2\\&2y-x \geq 4\\&x+y \leq 5}$ là miền trong của tam giác $ABC$ kể cả biên (như hình vẽ)
        
        \begin{center}
            \begin{tikzpicture}[>=stealth,line join=round,line cap=round,font=\footnotesize,scale=0.7]
                \tikzset{
                    declare function={x2=3;y2=6;x1=-3;y1=-1;m=2;n=2;a=1/2;b=2;p=-1;q=5;},
                    declare function={f(\x)=m*(\x)+n; g(\x)=a*(\x)+b;h(\x)=p*(\x)+q;}
                }
                \clip (x1,y1)rectangle (x2,y2);
                \draw[fill=black](0,0)circle(1.2pt)node[above left]{$O$};
                %%%%%%%%%%%%%%%%%%%%%%%%%%%%%%%%
                \draw [smooth,samples=100,domain=x1:x2]plot(\x,{f(\x)});
                %%%%%%%%%%%%%%%%%%%%%%%%%%%%%%%%
                \fill[pattern=north west lines,opacity=.4](x2,{f(x2)})--(x1,y2)--(x1,y1)--(x1,{f(x1)});
                %%%%%%%%%%%%%%%%%%%%%%%%%%%%%%%%
                \draw [smooth,samples=100,domain=x1:x2]plot(\x,{g(\x)});
                %%%%%%%%%%%%%%%%%%%%%%%%%%%%%%%%
                \fill[pattern=north west lines,opacity=.4](x1,{g(x1)})--(x1,y1)--(x2,y1)--(x2,{g(x2)});
                %%%%%%%%%%%%%%%%%%%%%%%%%%%%%%%%
                \draw [smooth,samples=100,domain=x1:x2]plot(\x,{h(\x)});
                %%%%%%%%%%%%%%%%%%%%%%%%%%%%%%%%
                \fill[pattern=north west lines,opacity=.4](x1,{h(x1)})--(x2,y2)--(x2,{h(x2)})--cycle;   
                %%%%%%%%%%%%%%%%%%%%%%%%%%%%%%%%
                \draw[thick,->] (x1,0)--(x2,0)node[above left]{$x$};
                \draw[thick,->] (0,y1)--(0,y2)node[below right]{$y$};
                \draw[fill=black]
                (1,4)circle(1.3pt)+(90:3.5mm)node{$B$}                  
                (2,3)circle(1.3pt)+(90:3.5mm)node{$C$}
                (0,2)circle(1.3pt)+(-10:3.5mm)node{$A$}
                ;       
                \foreach \p/\goc in {-2/-90,2/-90,1/-90}{
                    \draw[fill=black](\p,0)circle(1.3pt)+(\goc:2.5mm)node{$\p$};
                }
                \foreach \p/\goc in {2/170,5/180}{
                    \draw[fill=black](0,\p)circle(1.3pt)+(\goc:2.8mm)node{$\p$};
                }
                
            \end{tikzpicture}
        \end{center}
        Ta có $A(0; 2)$, $B(1; 4)$, $C(2; 3)$.\\
        Ta thấy rằng $F(x;y)=y-x$ đạt giá trị nhỏ nhất tại một đỉnh của tam giác $ABC$.\\
        Tại $A(0; 2)$ thì $F=2$.\\
        Tại $B(1; 4)$ thì $F=3$.\\
        Tại $C(2; 3)$ thì $F=1$.\\
        Vậy $\min F=1$ khi $x=2$ và $y=3$.
    }
\end{ex}

% Câu 45
\begin{ex}%[0D2K2-2]
    Các số $x$ và $y$ thỏa mãn hệ bất phương trình $\heva{&0 \leq y \leq 4\\&x \geq 0\\&x-y-1 \leq 0\\&x+2y-10 \leq 0}$. Giá trị lớn nhất của biết thức $F(x;y)=x+2y$ là
    \choice
    {$6$}
    {$8$}
    {\True $10$}
    {$12$}
    \loigiai{
        Vẽ đường thẳng $d_1 \colon x-y-1=0$, đường thẳng $d_1$ qua hai điểm $(0;-1)$ và $(1;0)$.\\
        Vẽ đường thẳng $d_2 \colon x+2y-10=0$, đường thẳng $d_2$ qua hai điểm $(0;5)$ và $(2;4)$.\\
        Vẽ đường thẳng $d_3 \colon y=4$.
        \begin{center}
                \begin{tikzpicture}[>=stealth,line join=round,line cap=round,font=\footnotesize,scale=.7]
                \tikzset{
                    declare function={x2=12;y2=6.5;x1=-.5;y1=-.5;m=1;n=-1;a=-1/2;b=5;p=-1;q=5;k=4;},
                    declare function={f(\x)=m*(\x)+n; g(\x)=a*(\x)+b;h(\x)=p*(\x)+q;f1(\x)=k;}
                }
                
                \clip (x1,y1)rectangle (x2,y2);
                \draw[fill=black](0,0)circle(1.2pt)node[above left]{$O$};
                %%%%%%%%%%%%%%%%%%%%%%%%%%%%%%%%
                \draw [smooth,samples=100,domain=x1:x2]plot(\x,{f(\x)});
                %%%%%%%%%%%%%%%%%%%%%%%%%%%%%%%%
                \fill[pattern=north west lines,opacity=.2](x1,{f(x1)})--(x2,y1)--(x2,{f(x2)});
                %%%%%%%%%%%%%%%%%%%%%%%%%%%%%%%%
                \draw [smooth,samples=100,domain=x1:x2]plot(\x,{g(\x)});
                %%%%%%%%%%%%%%%%%%%%%%%%%%%%%%%%
                \fill[pattern=north west lines,opacity=.2](x1,{g(x1)})--(x1,y2)--(x2,y2)--(x2,y1)--(x2,{g(x2)});
                %%%%%%%%%%%%%%%%%%%%%%%%%%%%%%%%
                \draw [smooth,samples=100,domain=x1:x2]plot(\x,{f1(\x)});
                %%%%%%%%%%%%%%%%%%%%%%%%%%%%%%%%
                \fill[pattern=north west lines,opacity=.2](x1,{f1(x1)})--(x1,y2)--(x2,y2)--(x2,y1)--(x2,{f1(x2)});
                %%%%%%%%%%%%%%%%%%%%%%%%%%%%%%%%
                \fill[pattern=north west lines,opacity=.2](0,y2)--(x1,y2)--(x1,y1)--(0,y1)--cycle;
                \fill[pattern=north west lines,opacity=.2](x1,0)--(x1,y1)--(x2,y1)--(x2,0)--cycle;
                %%%%%%%%%%%%%%%%%%%%%%%%%%%%%%%%
                \draw[thick,->] (x1,0)--(x2,0)node[above left]{$x$};
                \draw[thick,->] (0,y1)--(0,y2)node[below right]{$y$};
                \draw[fill=black]
                (0,4)circle(1.3pt)+(60:3.5mm)node[scale=.8]{$C$}
                (2,4)circle(1.3pt)+(60:3.5mm)node[scale=.8]{$B$}
                (4,3)circle(1.3pt)+(90:3.5mm)node[scale=.8]{$A$}
                (1,0)circle(1.3pt)+(-60:3.5mm)node[scale=.8]{$D$}       
                ;
                \node at (.5,5)[shift={(.55,-0.1)}]{$(d_2)$};
                \node at (6.4,5.5)[shift={(-.5,-0.1)}]{$(d_1)$};
            \end{tikzpicture}
        \end{center}
        Miền nghiệm của hệ bất phương trình là miền ngũ giác $ABCOE$ với $A(4;3),B(2;4),C(0;4),E(1;0)$.\\
        Ta có $F(4;3)=10$, $F(2;4)=10$, $F(0;4)=8$, $F(1;0)=1$, $F(0;0)=0$.\\
        Vậy giá trị lớn nhất của biết thức $F(x;y)=x+2y$ bằng $10$.
    }
\end{ex}
% Câu 46
\begin{ex}%[0D2K2-3]
    Các số $x$ và $y$ thỏa mãn hệ bất phương trình $\heva{&2x+y \leq 2\\&x-y \leq 2\\&5x+y \geq -4}$. Giá trị nhỏ nhất của biểu thức $F(x;y)=x+y$ là
    \choice
    {$\min F(x;y)=\dfrac{2}{3}$}
    {$\min F(x;y)=0$}
    {\True $\min F(x;y)=\dfrac{-8}{3}$}
    {$\min F(x;y)=4$}
    \loigiai{       
        \immini{
        Miền nghiệm của hệ bất phương trình $\heva{&2x+y \leq 2\\&x-y \leq 2\\&5x+y \geq -4}$ là miền tam giác $ABC$ với $A(-2;6)$, $C\left( \dfrac{4}{3};-\dfrac{2}{3} \right)$, $B\left( \dfrac{-1}{3};\dfrac{-7}{3} \right)$.\\      
        Biểu thức $F(x;y)=x+y$ có giá trị nhỏ nhất và giá trị ấy đạt được tại một trong các đỉnh của tam giác $ABC$\\
        $\begin{aligned}
            & F(-2;6)=4\\
            & F\left( \dfrac{4}{3};-\dfrac{2}{3} \right)=\dfrac{2}{3}\\
            & F\left( \dfrac{-1}{3};\dfrac{-7}{3} \right)=\dfrac{-8}{3}
        \end{aligned}$\\
        Vậy $\min F(x;y)=\dfrac{-8}{3}$.    
        }{
            \begin{tikzpicture}[>=stealth,line join=round,line cap=round,font=\footnotesize,scale=.7]
                    \tikzset{
                        declare function={x2=3;y2=6.5;x1=-3.5;y1=-4.5;m=-2;n=2;a=1;b=-2;p=-5;q=-4;k=4;},
                        declare function={f(\x)=m*(\x)+n; g(\x)=a*(\x)+b;h(\x)=p*(\x)+q;f1(\x)=k;}
                    }
                    
                    \clip (x1,y1)rectangle (x2,y2);
                    \draw[fill=black](0,0)circle(1.2pt)node[above left]{$O$};
                    %%%%%%%%%%%%%%%%%%%%%%%%%%%%%%%%
                    \draw [smooth,samples=100,domain=x1:x2]plot(\x,{f(\x)});
                    %%%%%%%%%%%%%%%%%%%%%%%%%%%%%%%%
                    \fill[pattern=north west lines,opacity=.2](x1,{f(x1)})--(x2,y2)--(x2,y1)--(x2,{f(x2)});
                    %%%%%%%%%%%%%%%%%%%%%%%%%%%%%%%%
                    \draw [smooth,samples=100,domain=x1:x2]plot(\x,{g(\x)});
                    %%%%%%%%%%%%%%%%%%%%%%%%%%%%%%%%
                    \fill[pattern=north west lines,opacity=.2](x2,{g(x2)})--(x2,y1)--(x1,{g(x1)});
                    %%%%%%%%%%%%%%%%%%%%%%%%%%%%%%%%
                    \draw [smooth,samples=100,domain=x1:x2]plot(\x,{h(\x)});
                    %%%%%%%%%%%%%%%%%%%%%%%%%%%%%%%%
                    \fill[pattern=north west lines,opacity=.2](x1,{h(x1)})--(x1,y2)--(x1,y1)--(x2,{h(x2)});
                    %%%%%%%%%%%%%%%%%%%%%%%%%%%%%%%%
                    %       \fill[pattern=north west lines,opacity=.2](0,y2)--(x1,y2)--(x1,y1)--(0,y1)--cycle;
                    %       \fill[pattern=north west lines,opacity=.2](x1,0)--(x1,y1)--(x2,y1)--(x2,0)--cycle;
                    %%%%%%%%%%%%%%%%%%%%%%%%%%%%%%%%
                    \draw[thick,->] (x1,0)--(x2,0)node[above left]{$x$};
                    \draw[thick,->] (0,y1)--(0,y2)node[below right]{$y$};
                    \draw[fill=black]
                    (4/3,-2/3)circle(1.3pt)+(-10:3.5mm)node[scale=.8]{$C$}
                    (-1/3,-7/3)circle(1.3pt)+(180:3.5mm)node[scale=.8]{$B$}
                    (-2,6)circle(1.3pt)+(190:3.5mm)node[scale=.8]{$A$}
                    ;
                    %       \node at (.5,5)[shift={(.55,-0.1)}]{$(d_2)$};
                    %       \node at (6.4,5.5)[shift={(-.5,-0.1)}]{$(d_1)$};
                \end{tikzpicture}   
        }
    }
\end{ex}
%Câu 47
\begin{ex}%[0D2K2-3]
    Các số $x$ và $y$ thỏa mãn hệ bất phương trình $\heva{&3x-y \geq -1\\&2x+y \leq 6\\&x+3y \geq 3}$ $(*)$. Giá trị lớn nhất và nhỏ nhất của biểu thức $f(x;y)=2x-3y+1$ là
    \choice
    {\True $\min f(x;y)=-9$ và $\max f(x;y)=7$}
    {$\min f(x;y)=-2$ và $\max f(x;y)=7$}
    {$\min f(x;y)=-9$ và $\max f(x;y)=-2$}
    {$\min f(x;y)=-9$ và $\max f(x;y)=-7$}
    \loigiai{
    \begin{enumerate}[--]
        \item Trước hết ta biểu diễn miền nghiệm của hệ $(*)$:
        \begin{enumerate}[+]
            \item Vẽ các đường thẳng $d_1 \colon 3x-y=-1$; $d_2 \colon 2x+y=6$; $d_3 \colon x+3y=3$.
            \item Điểm $M(1;1)$ có tọa độ thỏa mãn tất cả các bất phương trình trong hệ nên ta gạch chéo các nửa mặt phẳng bờ $d_1;d_2;d_3$ không chứa điểm $M$. Miền không bị gạch chéo là hình tam giác $ABC$, tính cả ba cạnh $AB,BC,CA$ trong hình vẽ là miền nghiệm của hệ bất phương trình đã cho.
            \begin{center}
                    \begin{tikzpicture}[>=stealth,line join=round,line cap=round,font=\footnotesize,xscale=.7]
                    \tikzset{
                        declare function={x2=5;y2=6.5;x1=-2.5;y1=-1.5;m=3;n=1;a=-2;b=6;p=-1/3;q=1;k=4;},
                        declare function={f(\x)=m*(\x)+n; g(\x)=a*(\x)+b;h(\x)=p*(\x)+q;f1(\x)=k;}
                    }
                    
                    \clip (x1,y1)rectangle (x2,y2);
                    \draw[fill=black](0,0)circle(1.2pt)node[below right]{$O$};
                    %%%%%%%%%%%%%%%%%%%%%%%%%%%%%%%%
                    \draw [smooth,samples=100,domain=x1:x2]plot(\x,{f(\x)});
                    %%%%%%%%%%%%%%%%%%%%%%%%%%%%%%%%
                    \fill[pattern=north west lines,opacity=.2](x2,{f(x2)})--(x1,y2)--(x1,y1)--(x1,{f(x1)});
                    %%%%%%%%%%%%%%%%%%%%%%%%%%%%%%%%
                    \draw [smooth,samples=100,domain=x1:x2]plot(\x,{g(\x)});
                    %%%%%%%%%%%%%%%%%%%%%%%%%%%%%%%%
                    \fill[pattern=north west lines,opacity=.2](x1,{g(x1)})--(x2,y2)--(x2,y1)--(x2,{g(x2)});
                    %%%%%%%%%%%%%%%%%%%%%%%%%%%%%%%%
                    \draw [smooth,samples=100,domain=x1:x2]plot(\x,{h(\x)});
                    %%%%%%%%%%%%%%%%%%%%%%%%%%%%%%%%
                    \fill[pattern=north west lines,opacity=.2](x1,{h(x1)})--(x1,y1)--(x2,y1)--(x2,{h(x2)});
                    %%%%%%%%%%%%%%%%%%%%%%%%%%%%%%%%
                    %       \fill[pattern=north west lines,opacity=.2](0,y2)--(x1,y2)--(x1,y1)--(0,y1)--cycle;
                    %       \fill[pattern=north west lines,opacity=.2](x1,0)--(x1,y1)--(x2,y1)--(x2,0)--cycle;
                    %%%%%%%%%%%%%%%%%%%%%%%%%%%%%%%%
                    \draw[thick,->] (x1,0)--(x2,0)node[above left]{$x$};
                    \draw[thick,->] (0,y1)--(0,y2)node[below right]{$y$};
                    \draw[fill=black]
                    (3,0)circle(1.3pt)+(-110:3.5mm)node[scale=.8]{$C$}
                    (1,4)circle(1.3pt)+(180:3.5mm)node[scale=.8]{$B$}
                    (0,1)circle(1.3pt)+(130:3.5mm)node[scale=.8]{$A$}node[shift={(-.01,-.28)},scale=.9]{$1$}
                    (1,1)circle(1.3pt)+(10:3.5mm)node[scale=.8]{$M$}
                    ;
                    \draw[dashed] (0,1)-|(1,0)node[below,scale=.8]{$1$};
                    \node at (1,5)[shift={(.55,-0.1)}]{$(d_1)$};
                    \node at (-1.4,1.5)[shift={(-.5,-0.1)}]{$(d_3)$};
                    \node at (3.7,-1)[shift={(-.5,-0.1)}]{$(d_2)$};
                \end{tikzpicture}
            \end{center}        
        \end{enumerate}
        \item Tìm tọa độ các điểm $A,B,C$:
        \begin{enumerate}[+]
        \item $\big\{A\big\}=d_1 \cap d_3$ nên tọa độ của nó là nghiệm của hệ $\heva{&3x-y=-1\\&x+3y=3} \Leftrightarrow \heva{&x=0\\&y=1}$. Vậy $A(0;1)$.\\
        \item $\big\{B\big\}=d_1 \cap d_2$ nên tọa độ của nó là nghiệm của hệ $\heva{&3x-y=-1\\&2x+y=6} \Leftrightarrow \heva{&x=1\\&y=4}$. Vậy $B(1;4)$.\\
        \item $\big\{C\big\}=d_2 \cap d_3$ nên tọa độ của nó là nghiệm của hệ $\heva{&2x+y=6\\&x+3y=3} \Leftrightarrow \heva{&x=3\\&y=0}$. Vậy $C(3;0)$.\\      
        \end{enumerate}
        \item Tính giá trị của $f(x;y)=2x-3y+1$ tại tất cả các đỉnh của tam giác $ABC$:
        \begin{center}
            \begin{tabular}{|>{\centering\arraybackslash}p{3cm}|>{\centering\arraybackslash}p{3cm}|>{\centering\arraybackslash}p{3cm}|>{\centering\arraybackslash}p{3cm}|}
                \hline
                $(x;y)$&$A(0;1)$  &$B(1;4)$  & $(3;4)$ \\
                \hline
                $f(x;y)=2x-3$&$-2$  &$-9$  &$7$  \\
                \hline
            \end{tabular}
        \end{center}    
    \end{enumerate}
        Suy ra $\min f(x;y)=f(1;4)=-9$ và $\max f(x;y)=f(3;0)=7$.
    }
\end{ex}
% Câu 48
\begin{ex}%[0D2T2-1]
    Lượng calo từ tinh bột khuyến nghị hàng ngày cho một người bình thường khoảng $480$ đến $1200$ calo. Để nạp đủ chất thì người ta cần nạp cả hai loại tinh bột hấp thu nhanh và tinh bột hấp thu chậm vào cơ thể. Biết rằng trong $100$ g gạo (chứa tinh bột hấp thu nhanh) có khoảng $150$ calo và $100$ g yến mạch (chứa tinh bột hấp thu chậm) có khoảng $50$ calo. Hôm nay bạn An đã ăn ít nhất là $200$ g gạo. Hỏi bạn ấy cần ăn nhiều nhất bao nhiêu gam yến mạch để có thể nạp vào cơ thể lượng calo tối thiểu cần thiết.
    \choice
    {$800$ gam}
    {$200$ gam}
    {$320$ gam}
    {\True $360$ gam}
    \loigiai{       
        \immini{
            Gọi $x,y$ lần lượt là số gam gạo và yến mạch bạn An cần nạp vào cơ thể để có lượng calo tối thiểu cần thiết.\\
            Ta có hệ bất phương trình $\heva{&1,5x+0,5y \leq 1200\\&1,5x+0,5y \geq 480\\&x \geq 200\\&y>0}(I)$.\\
            Bài toán đưa về tìm $x,y$ là nghiệm của hệ bất phương trình (I) sao cho $T=1,5x+0,5y$ có giá trị nhỏ nhất.\\
            Trước hết, ta xác định miền nghiệm của hệ bất phương trình (I).\\
            Miền nghiệm của hệ bất phương trình (I) là miền tứ giác $ABCD$ (trừ cạnh $BC$) với $A(200;1800),B(800;0),C(320;0),D(200;360)$.\\        
            Tính giá trị biểu thức $T=1,5x+0,5y$ tại cặp số $(x;y)$ là tọa độ các đỉnh $A,D$ của tứ giác $ABCD$ rồi so sánh các giá trị đó.\\
            Tại $A(200;1800) \colon T=1,5.200+0,5.1800=1200$;\\
            Tại $D(200;360) \colon T=1,5.200+0,5.360=480$;\\
            Ta được $T=1,5x+0,5y$ đạt được giá trị nhỏ nhất khi tại $D$ nên số lượng yến mạch nhiều nhất An cần ăn là $360$ gam để có thể nạp vào cơ thể lượng calo tối thiểu cần thiết.    
        }{
            \begin{tikzpicture}[>=stealth,line join=round,line cap=round,font=\footnotesize,xscale=.34,yscale=.3]
                \tikzset{
                    declare function={x2=12;y2=30;x1=-6;y1=-6.5;m=-3;n=24;a=-3;b=9.6;p=-1;q=5;k=.2;},
                    declare function={f(\x)=m*(\x)+n; g(\x)=a*(\x)+b;h(\x)=p*(\x)+q;f1(\x)=k;}
                }
                
                \clip (x1,y1)rectangle (x2,y2);
                \draw[fill=black](0,0)circle(1.2pt)node[above left]{$O$};
                %%%%%%%%%%%%%%%%%%%%%%%%%%%%%%%%
                \draw [smooth,samples=100,domain=x1:x2]plot(\x,{f(\x)});
                %%%%%%%%%%%%%%%%%%%%%%%%%%%%%%%%
                \fill[gray,opacity=.2](x1,{f(x1)})--(x1,y2)--(x2,y2)--(x2,y1)--(x2,{f(x2)});
                %%%%%%%%%%%%%%%%%%%%%%%%%%%%%%%%
                \draw [smooth,samples=100,domain=x1:x2]plot(\x,{g(\x)});
                %%%%%%%%%%%%%%%%%%%%%%%%%%%%%%%%
                \fill[gray,opacity=.2](x1,{g(x1)})--(x1,y1)--(x2,{g(x2)});  
                %%%%%%%%%%%%%%%%%%%%%%%%%%%%%%%
                \draw (2,y1)--(2,y2);
                %%%%%%%%%%%%%%%%%%%%%%%%%%%%%%%%
                \fill[gray,opacity=.2](2,y2)--(x1,y2)--(x1,y1)--(2,y1)--cycle;
                \fill[gray,opacity=.2](0,y2)--(x1,y2)--(x1,y1)--(0,y1)--cycle;
                \fill[gray,opacity=.2](x1,0)--(x1,y1)--(x2,y1)--(x2,0)--cycle;
                %%%%%%%%%%%%%%%%%%%%%%%%%%%%%%%%
                \draw[thick,->] (x1,0)--(x2,0)node[above left]{$x$};
                \draw[thick,->] (0,y1)--(0,y2)node[below right]{$y$};
                \draw[fill=black]
                (8,0)circle(2.5pt)node[scale=.9,below left]{$800$}node[scale=.9,above right]{$B$}
                (3.2,0)circle(2.5pt)node[scale=.9,below right]{$320$}node[scale=.9,above right]{$C$}
                (2,18)circle(2.5pt)node[scale=.9,above right]{$A$}
                (2,3.6)circle(2.5pt)node[scale=.9,above left]{$D$}
                (0,18)circle(2.5pt)node[scale=.9, left]{$1800$}
                (0,3.6)circle(2.5pt)node[scale=.9,left]{$360$}
                (2,0)circle(2.5pt)node[scale=.9,below left]{$200$}
                ;
                
            \end{tikzpicture}
        }
    }
\end{ex}
% Câu 49
\begin{ex}%[0D2T2-1]
    Kinh Đô là một thương hiệu bánh nổi tiếng ở Việt Nam. Trong dịp tết trung thu An muốn đặt mua hai loại bánh để làm quà biếu cho bạn bè. Theo báo giá trên website thì bánh nướng một trứng thập cẩm Jambon là $50\,000$ VNĐ/$1$ cái còn bánh nướng một trứng đậu xanh là $40\,000$ VNĐ/$1$ cái. An dự định chi không quá $2\,300\,000$ VNĐ để mua bánh với mong muốn mua được ít nhất $10$ cái bánh nướng một trứng thập cẩm Jambon và không quá $15$ bánh nướng một trứng đậu xanh. Hỏi An phải mua bao nhiêu cái bánh nướng một trứng thập cẩm Jambon và bao nhiêu cái bánh nướng một trứng đậu xanh để số bánh mua được là nhiều nhất.
    \choice
    {\True $34$ và $15$}
    {$38$ và $12$}
    {$33$ và $16$}
    {$30$ và $20$}
    \loigiai{
        Gọi $x,y$ lần lượt là số lượng bánh nướng một trứng thập cẩm Jambon và bánh nướng một trứng đậu xanh mà An có thể mua. Theo giả thuyết, Ta có $x \in \mathbb{N},y \in \mathbb{N},x \geq 10,0 \leq y \leq 15$.\\
        Tổng số bánh là $T=x+y$.\\
        Số tiền An cần chi là $50000x+40000y$ (VNĐ).\\
        Do An dự định chi không quá $2\,300\,000$ VNĐ nên $50000x+40000y \leq 2300000$ hay $5x+4y \leq 230$.\\
        Ta có hệ bất phương trình: $\heva{&5x+4y \leq 230\\&x \geq 10\\&0 \leq y \leq 15}(I)$\\
        Bài toán đưa về tìm $x,y$ là nghiệm của hệ bất phương trình (I) sao cho $T=x+y$ có giá trị lớn nhất.\\
        Trước hết, ta xác định miền nghiệm của hệ bất phương trình (I). Miền nghiệm của hệ bất phương trình (I) là miền tứ giác $ABCD$ với $A(34;15),B(46;0),C(10;0),D(10;15)$.
        \begin{center}
                \begin{tikzpicture}[>=stealth,line join=round,line cap=round,font=\footnotesize,scale=1]
                \tikzset{
                    declare function={x2=6;y2=3.5;x1=-.5;y1=-.5;m=-5/4;n=23/4;a=-1/2;b=5;p=-1;q=5;k=1.5;},
                    declare function={f(\x)=m*(\x)+n; g(\x)=a*(\x)+b;h(\x)=p*(\x)+q;f1(\x)=k;}
                }
                
                \clip (x1,y1)rectangle (x2,y2);
                \draw[fill=black](0,0)circle(1.2pt)node[above left]{$O$};
                %%%%%%%%%%%%%%%%%%%%%%%%%%%%%%%%
                \draw [smooth,samples=100,domain=x1:x2]plot(\x,{f(\x)});
                %%%%%%%%%%%%%%%%%%%%%%%%%%%%%%%%
                \fill[gray,opacity=.2](x1,{f(x1)})--(x2,y2)--(x2,y1)--(x2,{f(x2)});
                %%%%%%%%%%%%%%%%%%%%%%%%%%%%%%%%
                \draw [smooth,samples=100,domain=x1:x2]plot(\x,{f1(\x)});
                %%%%%%%%%%%%%%%%%%%%%%%%%%%%%%%%
                \fill[gray,opacity=.2](x1,{f1(x1)})--(x1,y2)--(x2,y2)--(x2,y1)--(x2,{f1(x2)});
                \draw (1,y1)--(1,y2);
                %%%%%%%%%%%%%%%%%%%%%%%%%%%%%%%%
                \fill[gray,opacity=.2](1,y2)--(x1,y2)--(x1,y1)--(1,y1)--cycle;
                \fill[gray,opacity=.2](0,y2)--(x1,y2)--(x1,y1)--(0,y1)--cycle;
                \fill[gray,opacity=.2](x1,0)--(x1,y1)--(x2,y1)--(x2,0)--cycle;
                %%%%%%%%%%%%%%%%%%%%%%%%%%%%%%%%
                \draw[thick,->] (x1,0)--(x2,0)node[above left]{$x$};
                \draw[thick,->] (0,y1)--(0,y2)node[below right]{$y$};
                \draw[fill=black]
                (1,0)circle(1.1pt)+(60:3.5mm)node[scale=.8]{$C$}
                (4.6,0)circle(1.1pt)+(60:3.5mm)node[scale=.8]{$B$}
                (3.4,1.5)circle(1pt)+(90:3.5mm)node[scale=.8]{$A$}
                (1,1.5)circle(1pt)+(-60:3.5mm)node[scale=.8]{$D$}
                (1,0)circle(1pt)node[scale=.8,below left]{$10$} 
                (4.6,0)circle(1pt)node[scale=.8,below left]{$46$}
                (0,1.5)circle(1pt)node[scale=.8,above left]{$15$}
                (3.4,0)circle(1pt)node[scale=.8,below]{$34$}
                ;
                \node at (.5,5)[shift={(.55,-0.1)}]{$(d_2)$};
                \node at (6.4,5.5)[shift={(-.5,-0.1)}]{$(d_1)$};
            \end{tikzpicture}
        \end{center}
        Tính giá trị biểu thức $T=x+y$ tại các cặp số $(x;y)$ là tọa độ các đỉnh của tứ giác $ABCD$ rồi so sánh các giá trị đó.\\
        Tại $A(34;15) \colon T=34+15=49$;\\
        Tại $B(46;0) \colon T=46+0=46$;\\
        Tại $C(10;0) \colon T=10+0=10$;\\
        Tại $C(10;15) \colon T=10+15=25$;\\
        Ta được $T=x+y$ đạt được giá trị lớn nhất khi $x=34,y=15$ ứng với tọa độ đỉnh $A$.\\
        Vậy để mua được nhiều bánh nhất thì cần mua số bánh nướng một trứng thập cẩm Jambon và bánh nướng một trứng đậu xanh lần lượt là $34$ và $15$ cái.
    }
\end{ex}
% Câu 50
\begin{ex}%[0D2T2-1]
    Một nhà máy sản xuất giày thể thao dùng hai loại nguyên liệu vải, cao su để sản xuất ra hai loại giày chạy bộ và giày tập luyện đa năng. Để sản xuất một đôi giày phải dùng một số gam nguyên liệu khác nhau. Tổng số kilôgam nguyên liệu mỗi loại mà nhà sản xuất đó có trong một ngày và số gam từng loại nguyên liệu cần thiết để sản xuất ra một đôi giày mỗi loại được cho trong bảng sau
    \begin{center}
        \begin{tikzpicture}[>=stealth,line join=round,line cap=round,scale=.8]
            \draw (0.25,-.5)node{Loại nguyên liệu} (4,-.25)node[scale=.9]{Số kilôgam nguyên liệu  } (10,0.1)node[scale=.9]{Số gam từng loại nguyên liệu} (4,-.75)node[scale=.9]{có trong một ngày} (10,-0.4)node[scale=.9]{cần để sản xuất một đôi giày}
            (8,-1.25)node[scale=.9]{Giày chạy bộ} (12,-1.25)node[scale=.9]{Giày tập luyện đa năng} 
            (0.25,-2)node{Vải} (0.25,-3)node{Cao su} 
            (4,-2)node{$12$} (4,-3)node{$15$}
            (8,-2)node{$200$} (8,-3)node{$150$} 
            (12,-2)node{$150$} (12,-3)node{$300$};
            \draw (-1.5,.5)--(14,0.5)--(14,-3.5)--(-1.5,-3.5)--cycle 
            (2,0.5)--(2,-3.5)    (6,.5)--(6,-3.5)   (10,-.75)--(10,-3.5) (-1.5,-1.5)--(14,-1.5)     (6,-.75)--(14,-.75)     (-1.5,-2.5)--(14,-2.5)   (-1.5,-3.5)--(14,-3.5)  ;
        \end{tikzpicture}
    \end{center}
    Biết một đôi giày chạy bộ được bán với giá $2$ triệu đồng và một đôi giày tập luyện đa năng được bán với giá $2,5$ triệu đồng. Hỏi với số giày sản xuất được trong một ngày thì số tiền bán được nhiều nhất là bao nhiêu?
    \choice
    {\True $152$ triệu đồng}
    {$160$ triệu đồng}
    {$125$ triệu đồng}
    {$120$ triệu đồng}
    \loigiai{
        Gọi $x,y$ lần lượt là số đôi giày chạy bộ và giày tập luyện đa năng $(x,y \in \mathbb{N})$.\\
        Ta có hệ bất phương trình: $\heva{&2x+1,5y \leq 120\\&1,5x+3y \leq 150\\&x \geq 0\\&y \geq 0.}$\\
        Biểu diễn miền nghiệm của hệ bất phương trình trên hệ trục tọa độ $Oxy$ ta được như hình.\\
        Miền nghiệm là miền tứ giác $ABOC$ với các đỉnh $A(36;32),B(0;50),C(60;0),O(0;0)$.\\
        \begin{center}
                \begin{tikzpicture}[>=stealth,line join=round,line cap=round,font=\footnotesize,scale=.7]
                \tikzset{
                    declare function={x2=11;y2=9.5;x1=-1.5;y1=-.5;m=-4/3;n=8;a=-1/2;b=5;p=-1;q=5;k=1.5;},
                    declare function={f(\x)=m*(\x)+n; g(\x)=a*(\x)+b;h(\x)=p*(\x)+q;f1(\x)=k;}
                }
                
                \clip (x1,y1)rectangle (x2,y2);
                \draw[fill=black](0,0)circle(1.2pt)node[above left]{$O$};
                %%%%%%%%%%%%%%%%%%%%%%%%%%%%%%%%
                \draw [smooth,samples=100,domain=x1:x2]plot(\x,{f(\x)});
                %%%%%%%%%%%%%%%%%%%%%%%%%%%%%%%%
                \fill[gray,opacity=.2](x1,{f(x1)})--(x1,y2)--(x2,y2)--(x2,{f(x2)});
                \draw [smooth,samples=100,domain=x1:x2]plot(\x,{g(\x)});
                %%%%%%%%%%%%%%%%%%%%%%%%%%%%%%%%
                \fill[gray,opacity=.2](x1,{g(x1)})--(x1,y2)--(x2,y2)--(x2,{g(x2)});
                %%%%%%%%%%%%%%%%%%%%%%%%%%%%%%%%
                %%%%%%%%%%%%%%%%%%%%%%%%%%%%%%%%
                \fill[gray,opacity=.2](0,y2)--(x1,y2)--(x1,y1)--(0,y1)--cycle;
                \fill[gray,opacity=.2](x1,0)--(x1,y1)--(x2,y1)--(x2,0)--cycle;
                %%%%%%%%%%%%%%%%%%%%%%%%%%%%%%%%
                \draw[thick,->] (x1,0)--(x2,0)node[above left]{$x$};
                \draw[thick,->] (0,y1)--(0,y2)node[below right]{$y$};
                \draw[fill=black]
                (6,0)circle(1.1pt)+(60:3.5mm)node[scale=.8]{$C$}
                (0,5)circle(1.1pt)+(60:3.5mm)node[scale=.8]{$B$}
                (3.6,3.2)circle(1pt)+(90:3.5mm)node[scale=.8]{$A$}
                (0,3.2)circle(1pt)+(180:3.5mm)node[scale=.8]{$32$}
                (6,0)circle(1pt)node[scale=.8,below left]{$60$}
                (0,5)circle(1pt)node[scale=.8,below left]{$50$}
                (3.6,0)circle(1pt)node[scale=.8,below]{$36$}
                ;
                
            \end{tikzpicture}
        \end{center}
        Gọi $F$ là số tiền lãi (triệu đồng) thu được, ta có $F=2x+2,5y$.\\
        Tính giá trị của $F$ tại các đỉnh của tứ giác $ABOC$\\
        Tại $O(0;0) \colon F=2 \cdot 0+2,5 \cdot 0=0$;\\
        Tại $A(36;32) \colon F=2 \cdot 36+2,5 \cdot 32=152$;\\
        Tại $B(0;50) \colon F=2 \cdot 0+2,5 \cdot 50=125$;\\
        Tại $C(60;0) \colon F=2 \cdot 60+2,5 \cdot 0=120$;\\
        $F$ đạt được giá trị lớn nhất bằng $152$ tại $A(36;32)$.
    }
\end{ex}

\Closesolutionfile{ans}