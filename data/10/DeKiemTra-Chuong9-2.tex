\section*{Đề kiểm tra Chương 9}
\subsection*{Đề số 2}
\setcounter{ex}{0}\setcounter{bt}{0}
\Opensolutionfile{ans}[ans/ans-KT-902]
\noindent\textbf{I. PHẦN TRẮC NGHIỆM}
\Opensolutionfile{ansbook}[ans/ansbook-0D9-KT-TN-2]
\Opensolutionfile{ans}[ans/ans-0D9-KT-TN-2]
\begin{ex}%[Nguyễn Kiều Nhã Tú- BG Toán 10]%[1D2Y5-2]
	Một tổ có $7$ nam và $3$ nữ. Chọn ngẫu nhiên $2$ người. Tính xác suất sao cho $2$ người được chọn đều là nữ.
	\choice
	{$\dfrac{8}{15}$}
	{\True $\dfrac{1}{15}$}
	{$\dfrac{2}{15}$}
	{$\dfrac{7}{15}$}
	\loigiai{
		Số phần tử của không gian mẫu
		$n(\Omega)=\mathrm{C}_{10}^2=45$.\\
		Gọi $A$ là biến cố : \lq\lq  2 người được chọn là nữ \rq\rq, suy ra $n(A)=\mathrm{C}_3^2=3$.\\
		Vậy $\mathrm{P}(A)=\dfrac{3}{45}=\dfrac{1}{15}$.
	}
\end{ex}


\begin{ex}%[Nguyễn Kiều Nhã Tú- BG Toán 10]%[1D2Y5-4]
	Xác suất bắn trúng mục tiêu của một vận động viên khi bắn một viên đạn là $0{,}6$. Người đó bắn hai viên đạn một cách độc lập. Xác suất để một viên trúng mục tiêu và một viên trượt mục tiêu là
	\choice
	{$0{,}4$}
	{$0{,}6$}
	{$0{,}24$}
	{\True $0{,}48$}
	\loigiai{
		Có thể lần $1$ bắn trúng hoặc lần $2$ bắn trúng, suy ra để bắn trúng có $2$ cách.\\
		Xác suất để $1$ viên trúng mục tiêu là $0{,}6$.\\
		Xác suất để $1$ viên trượt mục tiêu là $1-0{,}6=0{,}4$.\\
		Theo quy tắc nhân xác suất có $\mathrm{P}(A)=2\cdot 0{,}6\cdot 0{,}4=0{,}48$.
	}
\end{ex}


\begin{ex}%[Nguyễn Kiều Nhã Tú- BG Toán 10]%[1D2Y5-2]
	Một hộp chứa 5 bi xanh và 10 bi đỏ có cùng kích thước và khối lượng. Lấy ngẫu nhiên 3 bi. Xác suất để được đúng một bi xanh là
	\choice
	{\True $\dfrac{45}{91}$}
	{$\dfrac{3}{4}$}
	{$\dfrac{200}{273}$}
	{$\dfrac{2}{3}$}
	\loigiai{
		Số phần tử của không gian mẫu là $|\Omega|=\mathrm{C}_{15}^3$.\\
		Gọi A là biến cố \lq\lq  để được đúng một bi xanh\rq\rq.\\
		Ta có $|{\Omega}_A|=\mathrm{C}_5^1\cdot\mathrm{C}_{10}^2$.\\
		Vậy xác suất biến cố $A$ là $\mathrm{P}(A)=\dfrac{45}{91}$.
	}
\end{ex}


\begin{ex}%[Nguyễn Kiều Nhã Tú- BG Toán 10]%[1D2Y5-2]
	Từ một hộp chứa ba quả cầu trắng và hai quả cầu đen lấy ngẫu nhiên hai quả. Xác suất để lấy được cả hai quả trắng là
	\choice
	{\True $\dfrac{9}{30}$}
	{$\dfrac{6}{30}$}
	{$\dfrac{12}{30}$}
	{$\dfrac{10}{30}$}
	\loigiai{
		Số phần tử của không gian mẫu là $n(\Omega)=\mathrm{C}_5^2=10$.\\
		Gọi $A$ là biến cố \lq\lq  Lấy được hai quả cầu màu trắng\rq\rq.\\
		Ta có $n(A)=\mathrm{C}_3^2=3$, suy ra $\mathrm{P}(A)=\dfrac{3}{10}=\dfrac{9}{30}$.
	}
\end{ex}


\begin{ex}%[Nguyễn Kiều Nhã Tú- BG Toán 10]%[1D2Y5-2]
	Một bình chứa $2$ bi xanh và $3$ bi đỏ có cùng kích thước và khối lượng. Lấy ngẫu nhiên $3$ bi. Xác suất để được ít nhất một bi xanh là
	\choice
	{$\dfrac{1}{10}$}
	{\True $\dfrac{9}{10}$}
	{$\dfrac{1}{5}$}
	{$\dfrac{4}{5}$}
	\loigiai{
		Số phần tử của không gian mẫu là $|\Omega|=\mathrm{C}_5^3$.\\
		Gọi A là biến cố \lq\lq  lấy để được ít nhất một bi xanh \rq\rq.\\
		Số phần tử của không gian thuận lợi là $|{\Omega}_A|=\mathrm{C}_5^3-\mathrm{C}_3^3$.\\
		Xác suất biến cố $A$ là $\mathrm{P}(A)=\dfrac{9}{10}$.
	}
\end{ex}


\begin{ex}%[Nguyễn Kiều Nhã Tú- BG Toán 10]%[1D2Y5-2]
	Một hộp chứa $6$ bi xanh, $7$ bi đỏ. Nếu chọn ngẫu nhiên $2$ bi từ hộp này thì xác suất để được $2$ bi cùng màu là
	\choice
	{\True $0{,}46$}
	{$0{,}51$}
	{$0{,}55$}
	{$0{,}64$}
	\loigiai{
		Số phần tử của không gian mẫu là $n(\Omega)=\mathrm{C}_{13}^2$.\\
		Gọi $ A $ là biến cố : \lq\lq  hai viên bi được chọn cùng màu \rq\rq.\\
		Ta có $n(A)=\mathrm{C}_6^2+\mathrm{C}_7^2$.\\
		Vậy xác suất biến cố $A$ là $\mathrm{P}(A)=\dfrac{n(\Omega)}{n(A)}=\dfrac{6}{13}=0{,}46$.
	}
\end{ex}


\begin{ex}%[Nguyễn Kiều Nhã Tú- BG Toán 10]%[1D2Y5-2]
	Một tổ học sinh gồm có $6$ nam và $4$ nữ. Chọn ngẫu nhiên $3$ em học sinh. Xác suất để $3$ học sinh được chọn có ít nhất $1$ nữ là
	\choice
	{$\dfrac{1}{30}$}
	{$\dfrac{1}{2}$}
	{$\dfrac{1}{6}$}
	{\True $\dfrac{5}{6}$}
	\loigiai{
		Số phần tử của không gian mẫu là $n(\Omega)=\mathrm{C}_{10}^3$.\\
		Gọi $ A $ là biến cố : \lq\lq  hai viên bi được chọn cùng màu \rq\rq.\\
		Ta có $ n(A)=\mathrm{C}_{10}^3-\mathrm{C}_6^3 $
		Xác suất $3$ em được chọn có ít nhất $1$ nữ là $\dfrac{\mathrm{C}_{10}^3-\mathrm{C}_6^3}{\mathrm{C}_{10}^3}=\dfrac{5}{6}$.
	}
\end{ex}


\begin{ex}%[Nguyễn Kiều Nhã Tú- BG Toán 10]%[1D2Y5-2]
	Một chứa $6$ bi đỏ, $7$ bi xanh. Nếu chọn ngẫu nhiên $5$ bi từ hộp này thì xác suất đúng đến phần trăm để có đúng $2$ bi đỏ là
	\choice
	{\True $0{,}41$}
	{$0{,}14$}
	{$0{,}34$}
	{$0{,}28$}
	\loigiai{
		Số phần tử của không gian mẫu là $n(\Omega)=\mathrm{C}_{13}^5$.\\
		Gọi biến cố $A$: \lq\lq  $5$ bi được chọn có đúng $2$ bi đỏ''.\rq\rq
		Ta có $n(A)=\mathrm{C}_6^2\cdot\mathrm{C}_7^3=525$.\\
		Vậy xác suất biến cố $A$ là $\mathrm{P}(A)=\dfrac{n(A)}{n(\Omega)}=\dfrac{175}{429}=0{,}41$.
	}
\end{ex}


\begin{ex}%[Nguyễn Kiều Nhã Tú- BG Toán 10]%[1D2Y5-4]
	Xếp ngẫu nhiên ba bạn An, Bình, Cường đứng trên thành một hàng dọc. Số phần tử không gian mẫu là
	\choice
	{\True $6$}
	{$4$}
	{$2$}
	{$3$}	
	\loigiai{
		Số phần tử của không gian mẫu: $n(\Omega)=3!=6$.
	}
\end{ex}


\begin{ex}%[Nguyễn Kiều Nhã Tú- BG Toán 10]%[1D2Y5-2]
	Một hộp chứa ba quả cầu trắng và hai quả cầu đen. Lấy ngẫu nhiên đồng thời hai quả cầu. Xác suất để lấy được cả hai quả trắng là
	\choice
	{\True $\dfrac{3}{10}$}
	{$\dfrac{5}{10}$}
	{$\dfrac{2}{10}$}
	{$\dfrac{4}{10}$}
	\loigiai{
		Gọi A là biến cố ``lấy được cả hai quả cầu trắng''.\\
		Số phần tử của không gian mẫu là $\mathrm{C}_5^2=10$.\\
		Ta có $n(A)=\mathrm{C}_3^2=3$.\\
		Vậy $\mathrm{P}(A)=\dfrac{n(A)}{\left|\Omega\right|}=\dfrac{3}{10}$.
	}
\end{ex}


\begin{ex}%[Nguyễn Kiều Nhã Tú- BG Toán 10]%[1D2Y5-2]
	Một lô hàng gồm $1000$ sản phẩm, trong đó có $50$ phế phẩm. Lấy ngẫu nhiên từ lô hàng đó $1$ sản phẩm. Xác suất để lấy được sản phẩm tốt là
	\choice
	{\True $0{,}95$}
	{$0{,}96$}
	{$0{,}94$}
	{$0{,}97$}
	\loigiai{
		Gọi $A$ là biến cố ``lấy được 1 sản phẩm tốt''.\\
		Ta có $|\Omega|=\mathrm{C}_{1000}^1=1000$.\\
		Suy ra $n(A)=\mathrm{C}_{950}^1=950$.\\
		Vậy $\mathrm{P}(A)=\dfrac{n(A)}{|\Omega|}=\dfrac{950}{1000}=0{,}95$.
	}
\end{ex}


\begin{ex}%[Nguyễn Kiều Nhã Tú- BG Toán 10]%[1D2Y5-4]
	Xét phép thử $T$: \lq\lq  Gieo một con súc sắc\rq\rq\;  và biến cố $B$: \lq\lq  Số chấm trên mặt xuất hiện là một số lẻ\rq\rq. Tập hợp nào dưới đây mô tả biến cố $B$?
	\choice
	{\True $\{1;3;5\}$}
	{$\{2;3;4\}$}
	{$\{2;4;6\}$}
	{$\{2;3;6\}$}	
	\loigiai{
		Các phần tử của biến cố $B$ là $\Omega=\{1;3;5\}$.
	}
\end{ex}

\begin{ex}%[Nguyễn Kiều Nhã Tú- BG Toán 10]%[1D2B5-2]
	Cho đa giác đều $12$ đỉnh. Chọn ngẫu nhiên $3$ đỉnh trong $12$ đỉnh của đa giác. Xác suất để $3$ đỉnh được chọn tạo thành tam giác đều là
	\choice
	{$\mathrm{P}=\dfrac{1}{14}$}
	{$\mathrm{P}=\dfrac{1}{220}$}
	{\True $\mathrm{P}=\dfrac{1}{55}$}
	{$\mathrm{P}=\dfrac{1}{4}$}
	\loigiai{
		Số phần tử không gian mẫu có $n(\Omega)=\mathrm{C}_{12}^3=220$.\\
		Gọi $A$ là biến cố: ``$3$ đỉnh được chọn tạo thành tam giác đều''.\\
		(Chia $12$ đỉnh thành $3$ phần. Mỗi phần gồm $4$ đỉnh liên tiếp nhau. Mỗi đỉnh của tam giác đều ứng với một phần ở trên. Chỉ cần chọn 1 đỉnh thì 2 đỉnh còn lại xác định là duy nhất).\\
		Suy ra $n(A)=\mathrm{C}_4^1=4$.\\
		Khi đó $\mathrm{P}(A)=\dfrac{n(A)}{n(\Omega)}=\dfrac{4}{220}=\dfrac{1}{55}$.
	}
\end{ex}

\begin{ex}%[Nguyễn Kiều Nhã Tú- BG Toán 10]%[1D2B5-2]
	Gieo một con súc sắc cân đối và đồng chất hai lần. Xác suất để ít nhất một lần xuất hiện mặt sáu chấm là
	\choice
	{$\dfrac{6}{36}$}
	{$\dfrac{12}{36}$}
	{\True $\dfrac{11}{36}$}
	{$\dfrac{8}{36}$}
	\loigiai{
		Số phần tử của không gian mẫu là $n(\Omega)=6\cdot 6=36$. \\
		Gọi $A$ là biến cố \lq\lq  ít nhất một lần xuất hiện mặt sáu chấm\rq\rq.\\
		Khi đó $\overline{A}$: \lq\lq  không có lần nào xuất hiện mặt sáu chấm\rq\rq.\\
		Ta có $n(\overline{A})=5\cdot 5=25$. Vậy $\mathrm{P}(A)=1-\mathrm{P}(\overline{A})=1-\dfrac{25}{36}=\dfrac{11}{36}$.
	}
\end{ex}

\begin{ex}%[Nguyễn Kiều Nhã Tú- BG Toán 10]%[1D2B5-2]
	Trên giá sách có $4$ quyến sách Toán, $3$ quyến sách Lý, $2$ quyến sách Hóa. Lấy ngẫu nhiên $3$ quyển sách. Tính xác suất để $3$ quyển lấy thuộc $3$ môn khác nhau.
	\choice
	{$\dfrac{5}{42}$}
	{$\dfrac{1}{21}$}
	{$\dfrac{37}{42}$}
	{\True $\dfrac{2}{7}$}
	\loigiai{
		Ta có $n(\Omega)=\mathrm{C}_9^3=84$.\\
		Gọi $A$ là biế cố: ``$3$ quyển lấy được thuộc 3 môn khác nhau''.\\
		Suy ra $n(A)=4\cdot 3\cdot 2=24$.\\
		Vậy $\mathrm{P}(A)=\dfrac{24}{84}=\dfrac{2}{7}$.
	}
\end{ex}

\begin{ex}%[Nguyễn Kiều Nhã Tú- BG Toán 10]%[1D2B5-2]
	Gieo ngẫu nhiên hai con súc sắc cân đối và đồng chất. Xác suất để sau hai lần gieo kết quả như nhau là
	\choice
	{\True $\dfrac{1}{6}$}
	{$\dfrac{5}{36}$}
	{$1$}
	{$\dfrac{1}{2}$}
	\loigiai{
		Số phần tử của không gian mẫu là  $n(\Omega)=6\cdot 6=36$.\\
		Biến cố xuất hiện hai lần như nhau: $A=\left\{(1;1);(2;2);(3;3);(4;4);(5;5);(6;6)\right\} \Rightarrow n(A)=6$.\\
		
		Suy ra $\mathrm{P}(A)=\dfrac{n(A)}{n(\Omega)}=\dfrac{6}{36}=\dfrac{1}{6}$.
	}
\end{ex}

\begin{ex}%[Nguyễn Kiều Nhã Tú- BG Toán 10]%[1D2B5-4]
	Một đề thi có $20$ câu hỏi trắc nghiệm khách quan, mỗi câu hỏi có $4$ phương án lựa chọn, trong đó chỉ có một phương án đúng. Khi thi, một học sinh đã chọn ngẫu nhiên một phương án trả lời với mỗi câu của đề thi đó. Xác suất để học sinh đó trả lời không đúng cả $20$ câu là
	\choice
	{\True $\left(\dfrac{3}{4}\right)^{20}$}
	{$\dfrac{1}{20}$}
	{$\dfrac{3}{4}$}
	{$\dfrac{1}{4}$}
	\loigiai{
		Ta có $|\Omega|=4^{20}$.\\
		Gọi $A$ là biến cố ``học sinh đó trả lời không đúng cả $20$ câu''.\\
		Suy ra $n(A)=3^{20}$.\\
		Vậy $\mathrm{P}(A)=\dfrac{n(A)}{|\Omega|}=\dfrac{3^{20}}{4^{20}}=\left(\dfrac{3}{4}\right)^{20}$.
	}
\end{ex}

\begin{ex}%[Nguyễn Kiều Nhã Tú- BG Toán 10]%[1D2B5-2]
	Rút ra một lá bài từ bộ bài $52$ lá. Xác suất để được lá bích là
	\choice
	{\True $\dfrac{1}{4}$}
	{$\dfrac{1}{13}$}
	{$\dfrac{12}{13}$}
	{$\dfrac{3}{4}$}
	\loigiai{
		Số phần tử không gian mẫu là  $n(\Omega)=52$.\\
		Số phần tử của biến cố xuất hiện lá bích là $n(A)=13$.\\
		Suy ra $\mathrm{P}(A)=\dfrac{n(A)}{n(\Omega)}=\dfrac{13}{52}=\dfrac{1}{4}$.
	}
\end{ex}

\begin{ex}%[1D2K5-2]
	Gieo một con súc sắc cân đối và đồng chất có sáu mặt;  các mặt $1, 2, 3, 4$ được sơn đỏ, mặt $5, 6$ sơn xanh. Gọi A là biến cố được số lẻ, B là biến cố được nút đỏ (mặt sơn màu đỏ). Xác suất của $A \cap B$ là
	\choice
	{$\dfrac{3}{4}$}
	{\True $\dfrac{1}{3}$}
	{$\dfrac{2}{3}$}
	{$\dfrac{1}{4}$}
	\loigiai{
		Số phần tử của không gian mẫu là $|\Omega|=6$.\\
		Số phần tử của biến cố $ A\cap B $ là $\left|{\Omega}_{A\cap B}\right|=2$.\\
		Xác suất biến cố $\mathrm{P}(A\cap B)=\dfrac{1}{3}$.
	}
\end{ex}


\begin{ex}%[Nguyễn Kiều Nhã Tú- BG Toán 10]%[1D2K5-2]
	Một con xúc sắc cân đối và đồng chất được gieo ba lần. Gọi $P$ là xác suất để tổng số chấm xuất hiện ở hai lần gieo đầu bằng số chấm xuất hiện ở lần gieo thứ ba. Khi đó $P$ bằng
	\choice
	{$\dfrac{10}{216}$}
	{\True $\dfrac{15}{216}$}
	{$\dfrac{16}{216}$}
	{$\dfrac{12}{216}$}
	\loigiai{
		Số phần tử của không gian mẫu $n(\Omega)=6\cdot 6\cdot 6=216$. \\
		Gọi $A$ là biến cố: \lq\lq  tổng số chấm xuất hiện ở hai lần gieo đầu bằng số chấm xuất hiện ở lần gieo thứ ba\rq\rq. \\
		Ta chỉ cần chọn $1$ bộ $2$ số chấm ứng với hai lần gieo đầu sao cho tổng của chúng thuộc tập $\{1;2;3;4;5;6\}$ và số chấm lần gieo thứ ba sẽ là tổng hai lần gieo đầu.\\
		Liệt kê ra ta có
		$$\{(1;1);(1;2);(1;3);(1;4);(1;5);(2;1);(2;2);(2;3);(2;4);(3;1);(3;2);(3;3);(4;1);(4;2);(5;1)\}.$$
		Suy ra $n(A)=15$. Do đó $\mathrm{P}(A)=\dfrac{15}{216}$.
	}
\end{ex}

\begin{ex}%[Nguyễn Kiều Nhã Tú- BG Toán 10]%[1D2Y5-4]
	Xếp ngẫu nhiên năm bạn An, Bình, Cường, Đức và Minh đứng  thành một hàng dọc. Số phần tử không gian mẫu là
	\choice
	{\True $120$}
	{$24$}
	{$6$}
	{$5$}	
	\loigiai{
		Số phần tử của không gian mẫu: $n(\Omega)=5!=120$.
	}
\end{ex}
\begin{ex}%[Nguyễn Kiều Nhã Tú- BG Toán 10]%[1D2Y5-4]
	Xét phép thử $T$: \lq\lq  Gieo một con súc sắc\rq\rq\;  và biến cố $Y$: \lq\lq  Số chấm trên mặt xuất hiện là một số lẻ\rq\rq. Tập hợp nào dưới đây mô tả biến cố $Y$?
	\choice
	{\True $\{1;3;5\}$}
	{$\{2;3;4\}$}
	{$\{2;4;6\}$}
	{$\{2;3;6\}$}	
	\loigiai{
		Các phần tử của biến cố $Y$ là $\{1;3;5\}$.
	}
\end{ex}
\begin{ex}%[Nguyễn Kiều Nhã Tú- BG Toán 10]%[1D2Y5-4]
	Xét phép thử $T$: \lq\lq  Gieo một con súc sắc\rq\rq\
	và  biến cố $X$: \lq\lq  Số chấm xuất hiện trên mặt là số nguyên tố\rq\rq. Tập hợp nào dưới đây mô tả biến cố $X$?
	\choice
	{\True $\{2;3;5\}$}
	{$\{2;3;6\}$}
	{$\{1;3;5\}$}
	{$\{2;3;4\}$}	
	\loigiai{
		Các phần tử của biến cố $X$ là $\{2;3;5\}$.
	}
\end{ex}

\begin{ex}%[Nguyễn Kiều Nhã Tú- BG Toán 10]%[1D2Y4-1]
	Xét phép thử: \lq\lq  Gieo hai đồng xu phân biệt\rq\rq. Nếu kí hiệu $S$ để chỉ đồng xu xuất hiện mặt \lq\lq  sấp\rq\rq \,, kí hiệu $N$
	để chỉ đồng xu xuất hiện mặt \lq\lq  ngửa\rq\rq \, thì không gian mẫu của phép thử trên là
	\choice
	{$\Omega=\{SS;NN\}$}
	{ $\Omega=\{SN;NS\}$}
	{$\Omega=\{SS;SN;NN\}$}
	{\True $\Omega=\{SS;SN;NS;NN\}$}	
	\loigiai{
		$$\Omega=\{SS;SN;NS;NN\}.$$
	}
\end{ex}
\begin{ex}%[Nguyễn Kiều Nhã Tú- BG Toán 10]%[1D2Y4-1]
	Xét phép thử $T$: \lq\lq  Gieo một con súc sắc\rq\rq\; có không gian mẫu là $\Omega=\{1 ; 2 ; 3 ; 4 ; 5 ; 6\}$. Xét biến cố $A$: \lq\lq  Số chấm trên mặt xuất hiện là số lẻ \rq\rq.	Số phần tử thuận lợi của biến cố $A$ là
	\choice
	{$6$}
	{ $94$}
	{$5$}
	{\True $3$}	
	\loigiai{
		Các kết quả được gọi là \textit{kết quả thuận lợi cho} $A$ được mô tả bởi: $\Omega_{A}=\{1;3;5\}$ là một tập con của $\Omega \Rightarrow$ Số phần tử thuận lợi của biến cố $A$ là $n(A)=3$.
	}
\end{ex}
\begin{ex}%[Nguyễn Kiều Nhã Tú- BG Toán 10]%[1D2B5-2]
	Gieo ngẫu nhiên một con súc sắc cân đối và đồng chất. Tính xác suất của biến cố $A$: \lq\lq  Xuất hiện mặt có số chấm lẻ\rq\rq.
	\choice
	{$\dfrac{1}{6}$}
	{ $\dfrac{1}{4}$}
	{$\dfrac{1}{3}$}
	{\True $\dfrac{1}{2}$}	
	\loigiai{
		Các phần tử không gian mẫu là $\Omega=\{1; 2; 3; 4; 5; 6\}\Rightarrow n(\Omega)=6$.\\
		Các phần tử của biến cố $A$ là $\{1; 3; 5\} \Rightarrow n\left( A  \right) =3$.\\
		Do đó xác suất cần tìm của biến cố $A$ là
		$$P(A)=\dfrac{n(A)}{n(\Omega)}=\dfrac{3}{6}=\dfrac{1}{2}.$$
	}
\end{ex}
\begin{ex}%[Nguyễn Kiều Nhã Tú- BG Toán 10]%[1D2Y5-4]
	Gieo ngẫu nhiên một con súc sắc cân đối và đồng chất. Tính xác suất của biến cố $B$: \lq\lq  Xuất hiện mặt có số chấm chia hết cho $4$\rq\rq.
	\choice
	{\True$\dfrac{1}{6}$}
	{ $\dfrac{1}{4}$}
	{$\dfrac{1}{3}$}
	{ $\dfrac{1}{2}$}	
	\loigiai{
		Các phần tử không gian mẫu là $\Omega=\{1; 2; 3; 4; 5; 6\}\Rightarrow n(\Omega)=6$.\\
		Các phần tử của biến cố $B$ là $\{4\} \Rightarrow n\left( B \right) =1$.\\
		Do đó xác suất cần tìm của biến cố $B$ là
		$$P(B)=\dfrac{n(B)}{n(\Omega)}=\dfrac{1}{6}.$$
	}
\end{ex}
\begin{ex}%[Nguyễn Kiều Nhã Tú- BG Toán 10]%[1D2Y5-4]
	Gieo ngẫu nhiên một con súc sắc cân đối và đồng chất. Tính xác suất của biến cố $C$:  \lq\lq  Mặt xuất hiện có số chấm lớn hơn $3$\rq\rq.
	\choice
	{$\dfrac{1}{6}$}
	{ $\dfrac{2}{3}$}
	{$\dfrac{1}{3}$}
	{\True  $\dfrac{1}{2}$}	
	\loigiai{
		Các phần tử không gian mẫu là $\Omega=\{1; 2; 3; 4; 5; 6\}\Rightarrow n(\Omega)=6$.\\
		Các phần tử của biến cố $C$ là $\{4; 5; 6\} \Rightarrow n\left( \Omega \right) =3$.\\
		Do đó xác suất cần tìm của biến cố $A$ là
		$$P(C)=\dfrac{n(C)}{n(\Omega)}=\dfrac{3}{6}=\dfrac{1}{2}.$$
	}
\end{ex}


\begin{ex}%[Nguyễn Kiều Nhã Tú- BG Toán 10]%[1D2B5-2]
	Cho hai đường thẳng song song $a$ và $b$. Trên đường thẳng $a$ lấy $6$ điểm phân biệt, trên đường thẳng $b$ lấy $5$ điểm phân biệt. Chọn ngẫu nhiên ba điểm trong các điểm đã cho trên hai đường thẳng $a$ và $b$. Tính xác suất để ba điểm được chọn tạo thành một tam giác.
	
	\choice
	{\True $\dfrac{9}{11}$}
	{ $\dfrac{3}{11}$}
	{$\dfrac{5}{22}$}
	{$\dfrac{2}{5}$}	
	\loigiai{
		Chọn ngẫu nhiên ba điểm trong 11 điểm đã cho trên hai đường thẳng $a$ và $b$ có $\mathrm{C}_{11}^3=165$ cách.\\
		Vậy $n\left( \Omega\right) =165$.\\
		Gọi $A$ là biến cố ``ba điểm được chọn tạo thành một tam giác''.\\
		Số kết quả thuận lợi của biến cố $A$ là $n\left( A\right)=6\cdot \mathrm{C}_5^2+5\cdot \mathrm{C}_6^2=135$.\\
		Xác suất của biến cố $A$ là $P\left( A\right)= \dfrac{135}{165}=\dfrac{9}{11}$.
	}
\end{ex}
\begin{ex}%[Nguyễn Kiều Nhã Tú- BG Toán 10]%[1D2B5-2]
	Gọi $E$ là tập hợp tất cả các số tự nhiên gồm ba chữ số khác nhau lập từ các chữ số  $1$, $2$, $3$, $4$, $7$. Tập $E$ có bao nhiêu phần tử ? Chọn ngẫu nhiên một phần tử từ $E$,  tính xác suất để phần tử được chọn chia hết cho $3$.
		\choice
	{ $\dfrac{9}{11}$}
	{ $\dfrac{3}{11}$}
	{$\dfrac{5}{22}$}
	{\True $\dfrac{2}{5}$}	
	\loigiai{
		Gọi $\overline{abc}$ số tự nhiên gồm ba chữ số khác nhau lập từ  $1$, $2$, $3$, $4$, $7$.\\
		Khi đó mỗi số $\overline{abc}$   là một chỉnh hợp chập $3$ của $5$.\\ Vậy số phần tử của tập $E$ là $\mathrm{A}_5^3=60$.\\ 
		Chọn ngẫu nhiên một số từ $60$ số trong tập $E$. Ta có $n\left( \Omega\right)=60$.\\
		Gọi $A$ là biến cố ``số được chọn chia hết cho 3 khi $a+b+c$ chia hết cho $3$''.\\
		Ta có các bộ số sau thỏa yêu cầu bài toán là $\left(1;2;3 \right) $, $\left(1;4;7 \right) $, $\left(2;3;4 \right) $, $\left(2;3;7 \right) $.\\
		Số kết quả thuận lợi của biến cố $A$ là $n\left( A\right)=4\cdot 3!=24$.\\
		Xác suất của biến cố $A$ là $P\left( A\right)= \dfrac{24}{60}=\dfrac{2}{5}$.	
		}
\end{ex}
\begin{ex}%[Nguyễn Kiều Nhã Tú- BG Toán 10]%[1D2B5-2]
	Gọi $X$ là tập hợp các số gồm hai chữ số khác nhau được lấy từ: $0$; $1$; $2$; $3$; $4$; $5$; $6$. Lấy ngẫu nhiên
	$2$ phần tử của $X$. Tính xác suất để $2$ số lấy được đều là số chẵn.
		\choice
	{$\dfrac{1}{6}$}
	{ $\dfrac{2}{3}$}
	{\True $\dfrac{1}{3}$}
	{ $\dfrac{1}{2}$}	
	\loigiai{Có thể lập được $6\cdot 6=36$ số có $2$ chữ số khác nhau.\\
		Ta có $\mathrm{n(X)}=36$.\\
		Số phần tử của không gian mẫu là $\mathrm{n(\Omega)}=\mathrm{C}_{36}^2=630$.\\
		Gọi số có $2$ chữ số là $\overline{ab}$.
		\begin{itemize}
			\item $b=0 \Rightarrow a$ có $6$ cách chọn.
			\item $b\ne 0 \Rightarrow b$ có $3$ cách chọn, $a$ có $5$ cách chọn.
		\end{itemize}
		Suy ra có $6+3\cdot 5=21$ số chẵn.\\
		Số cách lấy $2$ phần tử của $X$ đều là số chẵn là $\mathrm{C}_{21}^2=210$.\\
		Xác suất để $2$ số lấy được đều là số chẵn là $\mathrm{P}=\dfrac{210}{630}=\dfrac{1}{3}$
	}
	
\end{ex}
\begin{ex}%[Nguyễn Kiều Nhã Tú- BG Toán 10]%[1D2B5-2]
	Có $5$ học sinh nam và $7$ học sinh nữ tập trung ngẫu nhiên theo một hàng dọc. Tính xác suất để người đứng ở đầu hàng và cuối hàng đều là học sinh nữ.
	\choice
	{\True $\dfrac{7}{22}$}
	{ $\dfrac{7}{11}$}
	{$\dfrac{4}{11}$}
	{$\dfrac{3}{11}$}	
	\loigiai{
		Ta có $n(\Omega)=12!$.\\
		Gọi A là biến cố : \lq\lq  người đứng ở đầu hàng và cuối hàng đều là học sinh nữ\rq\rq.\\
		Chọn hai bạn nữ đứng đầu và cuối hàng có $\mathrm{A}^2_7$ cách.\\
		Hoán vị $10$ bạn còn lại ở giữa hàng có $10!$ cách.\\
		Suy ra $n(A)=\mathrm{A}^2_7\cdot 10!$.\\
		Vậy $P(A)=\dfrac{\mathrm{A}^2_7\cdot 10!}{12!}=\dfrac{7}{22}$.	
	}
\end{ex}
\begin{ex}%[Nguyễn Kiều Nhã Tú- BG Toán 10]	%[1D2B5-2]
Cho 14 tấm thẻ đánh số từ 1 đến 14. Chọn ngẫu nhiên 3 thẻ. Tính xác suất để tích 3 số ghi trên 3 tấm thẻ này chia hết cho 3.
			\choice
	{\True  $\dfrac{61}{91}$}
	{ $\dfrac{60}{91}$}
	{$\dfrac{62}{91}$}
	{ $\dfrac{31}{91}$}	
	\loigiai{$n(\Omega)=\mathrm{C}^3_{14}$.\\
	Gọi A là	biến cố  \lq\lq  3 thẻ lấy được có tích chia hết cho 3 \rq\rq.\\
		Biến cố $\overline{\mathrm{A}}$ : \lq\lq  3 thẻ lấy được có tích không chia hết cho 3 \rq\rq.\\
		Khi đó $n(\overline{\mathrm{A}})=\mathrm{C}^3_{10}$.\\
		Vậy $P(A)=1-P(\overline{\mathrm{A}})=1-\dfrac{\mathrm{C}^3_{10}}{\mathrm{C}^3_{14}}=\dfrac{61}{91}$.}
\end{ex}

\begin{ex}%[Nguyễn Kiều Nhã Tú- BG Toán 10]%[1D2B5-2]
	Từ một hộp chứa 16 thẻ đánh số từ 1 đến 16, chọn ngẫu nhiên 4 thẻ. Tính xác suất để 4 thẻ được chọn đều là số chẵn.
			\choice
	{$\dfrac{1}{16}$}
	{ $\dfrac{2}{13}$}
	{\True $\dfrac{1}{26}$}
	{ $\dfrac{1}{8}$}	
	\loigiai{$n(\Omega)=\mathrm{C}^4_{16}$.\\
		Gọi A là biến cố   \lq\lq  4 thẻ lấy được đều là số chẵn \rq\rq.\\
		Khi đó $P(A)=\mathrm{C}^4_8$.\\
		Vậy $P(A)=\dfrac{\mathrm{C}^4_8}{\mathrm{C}^4_{16}}=\dfrac{1}{26}$.}
\end{ex}
\begin{ex}%[Nguyễn Kiều Nhã Tú- BG Toán 10]%[1D2B5-2] 
	Một chi đoàn có $15$ giáo viên, trong đó có $7$ nam và $8$ nữ. Chọn ra $4$ người trong chi đoàn đó để lập một đội thanh niên tình nguyện. Tính xác suất sao cho trong $4$ người được chọn có ít nhất một nữ.
			\choice
	{$\dfrac{35}{39}$}
	{ $\dfrac{2}{39}$}
	{\True $\dfrac{38}{39}$}
	{ $\dfrac{1}{39}$}	
	\loigiai{
		Chọn ngẫu nhiên $4$ người trong chi đoàn đó để lập thành một đội thanh niên tình nguyện nên số phần tử của không gian mẫu là $n(\Omega) = \mathrm{C}_{15}^4=1365$.\\
		Gọi $A$ là biến cố: \lq\lq  $4$ người được chọn có ít nhất một nữ\rq\rq\ thì $\overline{A}$ là biến cố: \lq\lq  $4$ người được chọn không có người nữ nào\rq\rq.\\
		Ta có $n(\overline{A})=\mathrm{C}_{7}^4=35$ nên $\mathrm{P}(\overline{A})=\dfrac{n(\overline{A})}{n(\Omega)}=\dfrac{35}{1365}=\dfrac{1}{39}$.\\
		Suy ra xác suất để $4$ người được chọn có ít nhất một nữ là $\mathrm{P}(A)=1-\mathrm{P}(\overline{A})=\dfrac{38}{39}$.}
\end{ex}


\noindent\textbf{II. PHẦN TỰ LUẬN}
\begin{bt}%[Nguyễn Kiều Nhã Tú- BG Toán 10]%[1D2B5-2]
	Một tàu điện gồm 3 toa tiến vào một sân ga, ở đó đang có 12 hành khách chờ lên tàu. Giả sử hành
	khách lên tàu một cách ngẫu nhiên và độc lập với nhau, mỗi toa còn ít nhất 12 chỗ trống. Tìm xác suất xảy ra các tình huống sau
	\begin{itemize}
		\item[a)] Tất cả cùng lên toa thứ ba.
		\item[b)] Tất cả cùng lên một toa. 
		\item[c)] Toa thứ nhất có 4 người, toa thứ hai có 5 người và còn lại toa ba. 
	\end{itemize}
	\loigiai{
		Ở đây bài toán không quan tâm đến chỗ ngồi mà chỉ quan tâm đến toa. Phép thử ở đây là: Mỗi người chọn cho mình một toa, mỗi người có quyền chọn 1 trong 3 toa để lên nên có 3 cách chọn. Theo quy tắc nhân, suy ra số phần tử không gian mẫu là $n(\Omega)=3^{12}$.\\
		\begin{itemize}
			\item[a)] Gọi A là biến cố: “tất cả cùng lên toa thứ ba”.\\
			Mỗi người chỉ có một cách chọn là lên toa thứ ba.\\ Số trường hợp thuận lợi cho biến cố A là
			$n(A)=1^{12}\Rightarrow P(A)=\dfrac{1}{3^{12}}$.			
			\item[b)] Gọi B là biến cố: “tất cả cùng lên một toa”.\\	
			Người đầu tiên chọn 1 toa trong 3 toa để lên tàu có 3 cách.\\
			11 người lên sau chỉ có 1 cách chọn là lên toa mà người đầu tiên đã chọn.\\
			Số phần tử của biến cố $B$ là $n(B)=3.1^{11}=3\Rightarrow p(B)=\dfrac{3}{3^{12}}=\dfrac{1}{3^{11}}$.
			\item[c)] Gọi C là biến cố: “toa thứ nhất có 4 người, toa thứ hai có 5 người và còn lại toa ba”.
			\begin{itemize}
				\item Chọn 4 người trong 12 người lên toa thứ nhất có $\mathrm{C}^4_{12}$ cách.
				\item Chọn 5 người trong 8 người lên toa thứ hai có $\mathrm{C}^5_{8}$ cách.
				\item 	3 người còn lại bắt buộc lên toa thứ ba có 1 cách.
			\end{itemize}
			
			Suy ra số phần tử của biến cố C là $n(C)=\mathrm{C}^4_{12}\cdot \mathrm{C}^5_{8}\cdot 1=27720\Rightarrow \mathrm{P}(C)=\dfrac{27720}{3^{12}}=\dfrac{3080}{59049}$.
		\end{itemize}
	}
\end{bt}

\begin{bt}%[Nguyễn Kiều Nhã Tú- BG Toán 10] %[1D2B5-2] 
	Một ngân hàng đề thi gồm $ 15 $ câu hỏi. Mỗi đề thi gồm $ 4 $ câu được lấy ngẫu nhiên từ ngân hàng đề thi. Bạn Thủy đã học thuộc $ 8 $ câu trong ngân hàng đề thi. Tính xác suất để bạn Thủy rút ngẫu nhiên được một đề thi có ít nhất $ 2 $ câu đã học thuộc.
	\loigiai{
		Lấy ngẫu nhiên từ ngân hàng đề thi $ 4 $ câu hỏi để lập một đề thi cho ta một tổ hợp chập bốn của mười lăm phần tử. Do đó không gian mẫu là $ n(\Omega) = \mathrm{C}_{15}^{4} = 1365 $.\\
		Gọi $ A $ là biến cố: \lq\lq  bạn Thủy rút ngẫu nhiên được một đề thi có ít nhất $ 2 $ câu đã học thuộc\rq\rq.
		\begin{itemize}
			\item Bạn Thủy rút ngẫu nhiên được một đề thi có $ 2 $ câu đã học thuộc có $ \mathrm{C}_{8}^{2}\cdot \mathrm{C}_{7}^{2}=588 $ trường hợp.
			\item Bạn Thủy rút ngẫu nhiên được một đề thi có $ 3 $ câu đã học thuộc có $ \mathrm{C}_{8}^{3}\cdot \mathrm{C}_{7}^{1}=392 $ trường hợp.
			\item Bạn Thủy rút ngẫu nhiên được một đề thi có $ 4 $ câu đã học thuộc có $ \mathrm{C}_{8}^{4}=70 $ trường hợp.
		\end{itemize}
		Do đó bạn Thủy rút ngẫu nhiên được một đề thi có ít nhất $ 2 $ câu đã học thuộc có $ 588+392+70=1050 $ trường hợp.\\
		Xác suất cần tìm $ \mathrm{P}(A) = \dfrac{n(A)}{n(\Omega)} = \dfrac{1050}{1365} = \dfrac{10}{13} $.
	}
\end{bt}


\begin{bt}%[Nguyễn Kiều Nhã Tú- BG Toán 10]%[1D2B5-2] 
	Một hộp chứa $4$ viên bi trắng, $5$ viên bi đỏ và $6$ viên bi xanh. Lấy ngẫu nhiên từ hộp ra $4$ bi. Tính xác suất để $4$ bi được chọn có đủ $3$ màu và số bi đỏ nhiều nhất.

\loigiai{
	Chọn $4$ viên bi trong tổng 15 viên bi có, suy ra số phần tử của không gian mẫu là $n\left(\Omega\right)=\mathrm{C}_{15}^4=1365$ cách.\\
	Gọi $A$ là biến cố \lq\lq  $4$ bi được chọn đủ $3$ màu và số bi đỏ nhiều nhất\rq\rq.\\
	Vì $4$ bi được chọn đủ $3$ màu và số bi đỏ nhiều nhất nên suy ra trong các bi được lấy ra sẽ có $2$ bi đỏ, $1$ bi trắng, $1$ bi xanh.\\
	Chọn $2$ bi đỏ có $\mathrm{C}_5^{2}=10$ cách.\\
	Chọn $1$ bi trắng có $\mathrm{C}_4^{1}=4$ cách.\\
	Chọn $1$ bi xanh có $\mathrm{C}_6^{1}=6$ cách.\\
	Suy ra $n\left(A\right)=240$ cách.\\
	Suy ra $P\left(A\right)=\dfrac{n\left(A\right)}{n\left(\Omega\right)}=\dfrac{240}{1365}=\dfrac{16}{91}$.
}

\end{bt}
\Closesolutionfile{ans}
\Closesolutionfile{ansbook}
\indapan{10}{ans/ans-KT-902}