\subsection{BÀI TẬP TRẮC NGHIỆM}
\Opensolutionfile{ans}[ans/ansTL-0H3-1]
\setcounter{ex}{0}
\begin{ex}%[0H2B1-2]
	Giá trị $\cos 45^\circ+\sin 45^\circ$ bằng bao nhiêu?
	\choice
	{$1$}
	{\True $\sqrt{2}$}
	{$\sqrt{3}$}
	{$0$}
	\loigiai
	{Bằng cách tra bảng giá trị lượng giác của các góc đặc biệt hay dùng MTCT ta được $\heva{& \cos 45^\circ=\dfrac{\sqrt{2}}{2} \\ 
			& \sin 45^\circ=\dfrac{\sqrt{2}}{2} \\ 
		}$\\
		$\Rightarrow \cos 45^\circ+\sin 45^\circ=\sqrt{2}$.}
\end{ex}
\begin{ex}%[0H2B1-2]
	Giá trị của $\tan 30^\circ+\cot 30^\circ$ bằng bao nhiêu?
	\choice
	{\True $\dfrac{4}{\sqrt{3}}$}
	{$\dfrac{1+\sqrt{3}}{3}$}
	{$\dfrac{2}{\sqrt{3}}$}
	{$2$}
	\loigiai
	{Bằng cách tra bảng giá trị lượng giác của các góc đặc biệt hay dùng MTCT ta được $\heva{& \tan 30^\circ=\dfrac{1}{\sqrt{3}} \\ 
			& \cot 30^\circ=\sqrt{3} \\ 
		}$\\
		$\Rightarrow\tan 30^\circ+\cot 30^\circ=\dfrac{4}{\sqrt{3}}$.}
\end{ex}
\begin{ex}%[0H2B1-2]
	Trong các đẳng thức sau đây đẳng thức nào là \textbf{đúng}?
	\choice
	{$\sin {150^\circ}=-\dfrac{\sqrt{3}}{2}$}
	{$\cos {150^\circ}=\dfrac{\sqrt{3}}{2}$}
	{\True $\tan {150^\circ}=-\dfrac{1}{\sqrt{3}}$}
	{$\cot {150^\circ}=\sqrt{3}$}
	\loigiai
	{Bằng cách tra bảng giá trị lượng giác của các góc đặc biệt hay dùng MTCT ta được\\
		$\tan {150^\circ}=-\dfrac{1}{\sqrt{3}}$.}
\end{ex}
\begin{ex}%[0H2B1-2]
	Tính giá trị biểu thức $P=\cos 30^\circ\cos 60^\circ-\sin 30^\circ\sin 60^\circ$.
	\choice
	{$P=\sqrt{3}$}
	{$P=\dfrac{\sqrt{3}}{2}$}
	{$P=1$}
	{\True $P=0$}
	\loigiai
	{Vì $30^\circ$ và $60^\circ$ là hai góc phụ nhau nên $\heva{& \sin 30^\circ=\cos 60^\circ \\ 
			& \sin 60^\circ=\cos 30^\circ \\ 
		}$.\\
		$\Rightarrow P=\cos 30^\circ\cos 60^\circ-\sin 30^\circ\sin 60^\circ=\cos 30^\circ\cos 60^\circ-\cos 60^\circ\cos 30^\circ=0$.}
\end{ex}

\begin{ex}%[0H2B1-2]
	Trong các khẳng định sau đây, khẳng định nào \textbf{sai?}
	\choice
	{$\cos 45^\circ=\sin 45^\circ$}
	{$\cos 45^\circ=\sin {135^\circ}$}
	{$\cos 30^\circ=\sin 120^\circ$}
	{\True $\sin 60^\circ=\cos 120^\circ$}
	\loigiai
	{Bằng cách tra bảng giá trị lượng giác của các góc đặc biệt hay dùng MTCT ta được $\heva{& \cos 120^\circ=-\dfrac{1}{2} \\ 
			& \sin 60^\circ=\dfrac{\sqrt{3}}{2} \\ 
		}$.}
\end{ex}
\begin{ex}%[0H2B1-2]
	Tam giác $ABC$ vuông ở $A$ có góc $\widehat{B}=30^\circ$ Khẳng định nào sau đây là \textbf{sai}?
	\choice
	{\True $\cos B=\dfrac{1}{\sqrt{3}}$}
	{$\sin C=\dfrac{\sqrt{3}}{2}$}
	{$\cos C=\dfrac{1}{2}$}
	{$\sin B=\dfrac{1}{2}$}
	\loigiai
	{Từ giả thiết suy ra $\widehat{C}=60^\circ$\\
		Bằng cách tra bảng giá trị lượng giác của các góc đặc biệt hay dùng MTCT ta được\\
		$\cos B=\cos 30^\circ=\dfrac{\sqrt{3}}{2}$.}
\end{ex}

\begin{ex}%[0H2B1-2]
	Trong các đẳng thức sau, đẳng thức nào \textbf{đúng}?
	\choice
	{$\sin (180^\circ -\alpha )=-\cos \alpha $}
	{$\sin (180^\circ -\alpha )=-\sin \alpha $}
	{\True $\sin (180^\circ -\alpha )=\sin \alpha $}
	{$\sin (180^\circ -\alpha )=\cos \alpha $}
	\loigiai
	{Hai góc bù nhau $\alpha $ và $(180^\circ -\alpha )$ thì cho có giá trị của sin bằng nhau.}
\end{ex}

\begin{ex}%[0H2B1-2]
	Cho hai góc $\alpha $ và $\beta $ với $\alpha +\beta =180^\circ $. Tính giá trị của biểu thức \break $P=\cos \alpha \cos \beta -\sin \beta \sin \alpha $.
	\choice
	{$P=0$}
	{$P=1$}
	{\True $P=-1$}
	{$P=2$}
	\loigiai
	{Hai góc $\alpha $ và $\beta $ bù nhau nên $\sin \alpha =\sin \beta $; $\cos \alpha =-\cos \beta $.\\
		Do đó $P=\cos \alpha \cos \beta -\sin \beta \sin \alpha =-\cos^2\alpha -\sin ^2\alpha =-(\sin ^2\alpha +\cos^2\alpha )=-1$.}
\end{ex}
\begin{ex}%[0H2K1-2]
	Cho tam giác $ABC$. Tính $P=\sin A\cdot \cos (B+C )+\cos A\cdot \sin (B+C )$.
	\choice
	{\True $P=0$}
	{$P=1$}
	{$P=-1$}
	{$P=2$}
	\loigiai
	{Giả sử $\widehat{A}=\alpha$; $\widehat{B}+\widehat{C}=\beta $.\\ Biểu thức trở thành $P=\sin \alpha \cos \beta +\cos \alpha \sin \beta $.\\
		Trong tam giác $ABC$, có\\ $\widehat{A}+\widehat{B}+\widehat{C}=180^\circ \Rightarrow \alpha +\beta =180^\circ $.\\
		Do hai góc $\alpha $ và $\beta $ bù nhau nên $\sin \alpha =\sin \beta $; $\cos \alpha =-\cos \beta $.\\
		Do đó, $P=\sin \alpha \cos \beta +\cos \alpha \sin \beta =-\sin \alpha \cos \alpha +\cos \alpha \sin \alpha =0$.}
\end{ex}
\begin{ex}%[0H2K1-2]
	Cho tam giác $ABC$. Tính $P=\cos A\cdot\cos (B+C )-\sin A\cdot\sin (B+C )$.
	\choice
	{$P=0$}
	{$P=1$}
	{\True $P=-1$}
	{$P=2$}
	\loigiai
	{Giả sử $\widehat{A}=\alpha$; $\widehat{B}+\widehat{C}=\beta $.\\ Biểu thức trở thành $P=\cos \alpha \cos \beta -\sin \alpha \sin \beta $.\\
		Trong tam giác $ABC$ có\\ $\widehat{A}+\widehat{B}+\widehat{C}=180^\circ \Rightarrow \alpha +\beta =180^\circ $.\\
		Do hai góc $\alpha $ và $\beta $ bù nhau nên $\sin \alpha =\sin \beta $; $\cos \alpha =-\cos \beta $.\\
		Do đó $P=\cos \alpha \cos \beta -\sin \alpha \sin \beta =-\cos^2\alpha -\sin ^2\alpha =-(\sin ^2\alpha +\cos^2\alpha )=-1$.}
\end{ex}
\begin{ex}%[0H2B1-2]
	Cho hai góc nhọn $\alpha $ và $\beta $ phụ nhau. Hệ thức nào sau đây là \textbf{sai?}
	\choice
	{\True $\sin \alpha =-\cos \beta $}
	{$\cos \alpha =\sin \beta $}
	{$\tan \alpha =\cot \beta $}
	{$\cot \alpha =\tan \beta $}
	\loigiai
	{Hai góc nhọn $\alpha $ và $\beta $ phụ nhau thì $\sin \alpha =\cos \beta$;  $\cos\alpha =\sin\beta$; $\tan\alpha =\cot\beta$; $\cot \alpha =\tan\beta $.}
\end{ex}

\begin{ex}%[0H2B1-2]
	Cho hai góc $\alpha $ và $\beta $ với $\alpha +\beta =90^\circ $. Tính giá trị của biểu thức \break $P=\sin \alpha \cos \beta +\sin \beta \cos \alpha $.
	\choice
	{$P=0$}
	{\True $P=1$}
	{$P=-1$}
	{$P=2$}
	\loigiai
	{Hai góc $\alpha $ và $\beta $ phụ nhau nên $\sin \alpha =\cos \beta$; $\cos \alpha =\sin \beta $.\\
		Do đó, $P=\sin \alpha \cos \beta +\sin \beta \cos \alpha =\sin ^2\alpha +\cos^2\alpha =1$.}
\end{ex}

\begin{ex}%[0H2B1-1]
	Cho $\alpha $ là góc tù. Khẳng định nào sau đây là \textbf{đúng}?
	\choice
	{$\sin \alpha <0$}
	{$\cos \alpha >0$}
	{\True $\tan \alpha <0$}
	{$\cot \alpha >0$} 
	\loigiai{}
\end{ex}

\begin{ex}%[0H2K1-2]
	Cho biết $\sin \dfrac{\alpha }{3}=\dfrac{3}{5}$. Giá trị của $P=3\sin ^2\dfrac{\alpha }{3}+5\cos^2\dfrac{\alpha }{3}$ bằng bao nhiêu?
	\choice
	{$P=\dfrac{105}{25}$}
	{\True $P=\dfrac{107}{25}$}
	{$P=\dfrac{109}{25}$}
	{$P=\dfrac{111}{25}$}
	\loigiai
	{Ta có biểu thức $\sin ^2\dfrac{\alpha }{3}+\cos^2\dfrac{\alpha }{3}=1\Leftrightarrow \cos^2\dfrac{\alpha }{3}=1-\sin ^2\dfrac{\alpha }{3}=\dfrac{16}{25}$.\\
		Do đó ta có $P=3\sin ^2\dfrac{\alpha }{3}+5\cos^2\dfrac{\alpha }{3}=3\cdot {(\dfrac{3}{5} )}^2+5\cdot \dfrac{16}{25}=\dfrac{107}{25}$.}
\end{ex}
\begin{ex}%[0H2B1-2]
	Cho biết $\tan \alpha =-3$. Giá trị của $P=\dfrac{6\sin \alpha -7\cos \alpha }{6\cos \alpha +7\sin \alpha }$ bằng bao nhiêu?
	\choice
	{$P=\dfrac{4}{3}$}
	{\True $P=\dfrac{5}{3}$}
	{$P=-\dfrac{4}{3}$}
	{$P=-\dfrac{5}{3}$}
	\loigiai
	{Ta có $P=\dfrac{6\sin \alpha -7\cos \alpha }{6\cos \alpha +7\sin \alpha }=\dfrac{6\dfrac{\sin \alpha }{\cos \alpha }-7}{6+7\dfrac{\sin \alpha }{\cos \alpha }}=\dfrac{6\tan \alpha -7}{6+7\tan \alpha }=\dfrac{5}{3}$.}
\end{ex}

\begin{ex}%[0H2B1-2]
	Cho biết $\sin \alpha +\cos \alpha =a$. Tính giá trị của $\sin \alpha \cos \alpha $.
	\choice
	{$\sin \alpha \cos \alpha =a^2$}
	{$\sin \alpha \cos \alpha =2a$}
	{\True $\sin \alpha \cos \alpha =\dfrac{a^2-1}{2}$}
	{$\sin \alpha \cos \alpha =\dfrac{a^2-11}{2}$}
	\loigiai
	{Ta có $\sin \alpha +\cos \alpha =a\Rightarrow {{(\sin \alpha +\cos \alpha )}^2}=a^2$\\
		$\Leftrightarrow 1+2\sin \alpha \cos \alpha =a^2\Leftrightarrow \sin \alpha \cos \alpha =\dfrac{a^2-1}{2}$.}
\end{ex}
\begin{ex}%[0H2K1-2]
	Cho biết $\cos \alpha +\sin \alpha =\dfrac{1}{3}$. Giá trị của $P=\sqrt{\tan^2\alpha +\cot^2\alpha }$ bằng bao nhiêu?
	\choice
	{$P=\dfrac{5}{4}$}
	{\True $P=\dfrac{7}{4}$}
	{$P=\dfrac{9}{4}$}
	{$P=\dfrac{11}{4}$}
	\loigiai
	{Ta có $\cos \alpha +\sin \alpha =\dfrac{1}{3}\Rightarrow (\cos \alpha +\sin \alpha )^2=\dfrac{1}{9}$\\
		$\Leftrightarrow 1+2\sin \alpha \cos \alpha =\dfrac{1}{9}\Leftrightarrow \sin \alpha \cos \alpha =-\dfrac{4}{9}$.\\
		Ta có \begin{eqnarray*}
			P&=&\sqrt{\tan^2\alpha +\cot^2\alpha }=\sqrt{(\tan \alpha +\cot \alpha )^2-2\tan \alpha \cot \alpha }=\sqrt{\left(\dfrac{\sin \alpha }{\cos \alpha }+\dfrac{\cos \alpha }{\sin \alpha }\right)^2-2}\\
			&=&\sqrt{\left(\dfrac{\sin ^2\alpha +\cos^2\alpha }{\sin \alpha \cos \alpha } \right)^2-2}=\sqrt{\left(\dfrac{1}{\sin \alpha \cos \alpha } \right)^2-2}=\sqrt{\left(-\dfrac{9}{4} \right)^2-2}=\dfrac{7}{4}.
	\end{eqnarray*}}
\end{ex}
\begin{ex}%[0H2K1-2]
	Cho biết $\sin \alpha -\cos \alpha =\dfrac{1}{\sqrt{5}}$. Giá trị của $P=\sqrt{{\sin ^4}\alpha +{\cos ^4}\alpha }$ bằng bao nhiêu?
	\choice
	{$P=\dfrac{\sqrt{15}}{5}$}
	{\True $P=\dfrac{\sqrt{17}}{5}$}
	{$P=\dfrac{\sqrt{19}}{5}$}
	{$P=\dfrac{\sqrt{21}}{5}$}
	\loigiai
	{Ta có $\sin \alpha -\cos \alpha =\dfrac{1}{\sqrt{5}}\Rightarrow {{(\sin \alpha -\cos \alpha )}^2}=\dfrac{1}{5}$\\
		$\Leftrightarrow 1-2\sin \alpha \cos \alpha =\dfrac{1}{5}\Leftrightarrow \sin \alpha \cos \alpha =\dfrac{2}{5}$.\\
		Ta có \begin{eqnarray*}
			P&=&\sqrt{{\sin ^4}\alpha +{\cos ^4}\alpha }=\sqrt{(\sin ^2\alpha +\cos^2\alpha )^2-2\sin ^2\alpha \cos^2\alpha }\\
			&=&\sqrt{1-2(\sin \alpha \cos\alpha )^2}=\dfrac{\sqrt{17}}{5}.
	\end{eqnarray*}}
\end{ex}
\begin{ex}
	Trên mặt phẳng toạ độ $Oxy$, lấy điểm $M$ thuộc nửa đường tròn đơn vị sao cho $\widehat{xOM}=135^{\circ}$. Tích hoành độ và tung độ của điểm $M$ bằng
	\choice
	{$\dfrac{1}{2 \sqrt{2}}$}
	{$\dfrac{1}{2}$}
	{\True $-\dfrac{1}{2}$ }
	{$-\dfrac{1}{2 \sqrt{2}}$ }
\end{ex}
\begin{ex}
	Trên mặt phẳng toạ độ $Oxy$, lấy điểm $M$ thuộc nửa đường tròn đơn vị sao cho $\widehat{xOM}=150^{\circ}$. Gọi $N$ là điểm đối xứng với $M$ qua trục tung. Giá trị của $\tan \widehat{xON}$ bằng
	\choice
	{\True $\dfrac{1}{\sqrt{3}}$ }
	{$-\dfrac{1}{\sqrt{3}}$}
	{$\sqrt{3}$}
	{$-\sqrt{3}$}
\end{ex}

\begin{ex}
	\immini[thm]{Trên mặt phẳng toạ độ $Oxy$ lấy điểm $M$ thuộc nửa đường tròn đơn vị, sao cho $\cos \widehat{xOM}=-\dfrac{3}{5}$.
		Diện tích của tam giác $AOM$ bằng
		\choice
		{$\dfrac{4}{5}$}
		{\True $\dfrac{2}{5}$}
		{$\dfrac{3}{5}$}
		{ $\dfrac{3}{10}$}}{
		\begin{tikzpicture}[smooth,font=\footnotesize,scale=1.7,>=latex]
			\path
			(1,0) coordinate (A)
			(0,0) coordinate (O)
			($(O)+(127:1cm)$) coordinate (M);
			\draw[->] (-1.5,0)--(-1,0)node[below]{\tiny$-1$}--(1,0)node[below]{\tiny$1$}--(1.5,0)node[below]{$x$};
			\draw[->] (0,-0.5)--(0,1.2)node[left]{\tiny$1$}--(0,1.5)node[right]{$y$};
			\draw 
			(1,0) arc (0:180:1)
			(A)--(M)--(O);
			\foreach \x/\g in {A/49,M/100,O/-130} \draw [fill=black] (\x) circle (.02) + (\g:.2) node{$\x$};
			\fill[pattern=dots] (M)--(A)--(O);
	\end{tikzpicture}}
	\loigiai{
	}
\end{ex}

\begin{ex}
	\immini[thm]{
		Trên mặt phẳng toạ độ $Oxy$ lấy điểm $M$ thuộc nửa đường tròn đơn vị, sao cho $\widehat{xOM}=150^{\circ}$. Lấy $N$ đối xứng với $M$ qua trục tung. Diện tích của tam giác $MAN$ bằng
		\choice
		{\True $\dfrac{\sqrt{3}}{4}$}
		{$\dfrac{\sqrt{3}}{2}$}
		{$\sqrt{3}$}
		{$2 \sqrt{3}$}}{
		\begin{tikzpicture}[smooth,font=\footnotesize,scale=1.7,>=latex]
			\path
			(1,0) coordinate (A)
			(0,0) coordinate (O)
			($(O)+(150:1cm)$) coordinate (M)
			($(O)+(30:1cm)$) coordinate (N);
			\draw[->] (-1.5,0)--(-1,0)node[below]{\tiny$-1$}--(1,0)node[below]{\tiny$1$}--(1.5,0)node[below]{$x$};
			\draw[->] (0,-0.5)--(0,1.2)node[left]{\tiny$1$}--(0,1.5)node[right]{$y$};
			\draw 
			(1,0) arc (0:180:1)
			(A)--(M)--(N)--cycle
			(O)--(M);
			\foreach \x/\g in {A/49,M/100,O/-130,N/50} \draw [fill=black] (\x) circle (.02) + (\g:.2) node{$\x$};
			\fill[pattern=dots] (M)--(A)--(N);
	\end{tikzpicture}}
	\loigiai{
	}
\end{ex}
\Closesolutionfile{ans}