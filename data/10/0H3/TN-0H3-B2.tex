\subsection{BÀI TẬP TRẮC NGHIỆM}
\Opensolutionfile{ans}[ans/ansTL-0H3-2]
\setcounter{ex}{0}

\begin{ex}%[0H2Y3-1]
	Tam giác $ ABC$ có $ \widehat{B}=60^\circ$, $\widehat{C}=45^\circ $ và $ AB=5$. Tính độ dài cạnh $ AC$.
	\choice
	{$ AC=5\sqrt{2}$}
	{$ AC=5\sqrt{3}$}
	{$ AC=10$}
	{\True $ AC=\dfrac{5\sqrt{6}}{2}$}
	\loigiai
	{Theo định lí hàm sin, ta có $ \dfrac{AB}{\sin C}=\dfrac{AC}{\sin B}\Leftrightarrow \dfrac{5}{\sin 45^\circ}=\dfrac{AC}{\sin 60^\circ}\Rightarrow AC=\dfrac{5\sqrt{6}}{2}$.}
\end{ex}

\begin{ex}%[0H2Y3-1]
	Tam giác $ ABC$ vuông tại $ A$ và có $ AB=AC=a$. Tính độ dài đường trung tuyến $ BM$ của tam giác đã cho.
	\choice
	{\True $ BM=\dfrac{\sqrt{5}}{2}a$}
	{$ BM=1{,}5a$}
	{$ BM=\sqrt{2}a$}
	{$ BM=\sqrt{3}a$}
	\loigiai{
		\immini{
			$M$ là trung điểm của $ AC\Rightarrow AM=\dfrac{AC}{2}=\dfrac{a}{2}$.\\
			Xét tam giác $BAM$ vuông tại $ A$, ta có\\
			$BM=\sqrt{AB^2+AM^2}=\sqrt{a^2+\dfrac{a^2}{4}}=\dfrac{a\sqrt{5}}{2}$.}
		{
			\begin{tikzpicture}[scale=1,font=\footnotesize,line join=round, line cap=round,>=stealth]
			\tkzDefPoints{0/0/A,0/3/B,4/0/C}
			\tkzDefMidPoint(A,C) \tkzGetPoint{M}
			\tkzDrawPoints[fill=black](A,B,C,M)
			\tkzDrawPolygon(A,B,C)
			\tkzDrawSegments(B,M)
			\tkzLabelPoints[above](B)
			\tkzLabelPoints[below](M,C,A)
			\tkzMarkRightAngles[size=0.2](C,A,B)
			\end{tikzpicture}
		}
	}
\end{ex}


\begin{ex}%[0H2Y3-1]
	Tam giác $ ABC$ có $ BC=10$ và $ \widehat{A}=30^\circ$. Tính bán kính $ R$ của đường tròn ngoại tiếp tam giác $ ABC$.
	\choice
	{$ R=\dfrac{10}{\sqrt{3}}$}
	{\True $ R=10$}
	{$ R=10\sqrt{3}$}
	{$ R=5$}
	\loigiai
	{Áp dụng định lí sin, ta có $ \dfrac{BC}{\sin A}=2R\Rightarrow R=\dfrac{BC}{2}\cdot \sin  A=\dfrac{10}{2}\cdot \sin 30^\circ=10$.}
\end{ex}


\begin{ex}%[0H2Y3-1]
	Tam giác $ ABC$ có $ AC=4$, $\widehat{BAC}=30^\circ$, $\widehat{ACB}=75^\circ $. Tính diện tích tam giác $ ABC$.
	\choice
	{\True $ S_{\Delta ABC}=4$}
	{$ S_{\Delta ABC}=8\sqrt{3}$}
	{$ S_{\Delta ABC}=4\sqrt{3}$}
	{$ S_{\Delta ABC}=8$}
	\loigiai
	{Ta có $ \widehat{ABC}=180^\circ-(\widehat{BAC}+\widehat{ACB} )=75^\circ =\widehat{ACB}$.\\
		Suy ra tam giác $ ABC$ cân tại $ A$ nên $ AB=AC=4$.\\
		Diện tích tam giác $ ABC$ là $ S_{\Delta ABC}=\dfrac{1}{2}AB\cdot AC\sin \widehat{BAC}=4$.}
\end{ex}


\begin{ex}%[0H2Y3-1]
	Tam giác $ ABC$ có $ AB=3$, $AC=6$, $\widehat{BAC}=60^\circ $. Tính diện tích tam giác $ ABC$.
	\choice
	{$ S_{\Delta ABC}=9$}
	{$ S_{\Delta ABC}=9\sqrt{3}$}
	{$ S_{\Delta ABC}=\dfrac{9}{2}$}
	{\True $ S_{\Delta ABC}=\dfrac{9\sqrt{3}}{2}$}
	\loigiai
	{Ta có $ S_{\Delta ABC}=\dfrac{1}{2}\cdot AB\cdot AC\cdot \sin{A}=\dfrac{1}{2}\cdot 3\cdot 6\cdot \sin 60^\circ=\dfrac{9\sqrt{3}}{2}$.}
\end{ex}


\begin{ex}%[0H2Y3-1]
	Tam giác $ ABC$ có $ AB=5$, $BC=7$, $CA=8$. Số đo góc $ \widehat{A}$ bằng
	\choice
	{$ 90^\circ $}
	{$ 45^\circ $}
	{\True $ 60^\circ $}
	{$ 30^\circ $}
	\loigiai
	{Theo định lí hàm cô-sin, ta có $ \cos{A}=\dfrac{AB^2+AC^2-BC^2}{2AB\cdot AC}=\dfrac{5^2+8^2-7^2}{2\cdot 5\cdot 8}=\dfrac{1}{2}$.\\
		Do đó, $ \widehat{A}=60^\circ $.}
\end{ex}


\begin{ex}%[0H2Y3-1]
	Tam giác $ ABC$ có $ AB=9$ cm, $ AC=12$ cm và $ BC=15$ cm. Tính độ dài đường trung tuyến $ AM$ của tam giác đã cho.
	\choice
	{\True $ AM=\dfrac{15}{2}$ cm}
	{$ AM=9$ cm}
	{$ AM=10$ cm}
	{$ AM=\dfrac{13}{2}$ cm}
	\loigiai{
		\immini{
			Áp dụng hệ thức đường trung tuyến. Ta có\\
			$\begin{aligned}
			m_a^2&=\dfrac{b^2+c^2}{2}-\dfrac{a^2}{4}\\
			&=\dfrac{12^2+9^2}{2}-\dfrac{15^2}{4}\\
			&=\dfrac{225}{4}.\\
			\Rightarrow m_a&=\dfrac{15}{2}.
			\end{aligned}$ \\
		}
		{
			\begin{tikzpicture}[scale=1,font=\footnotesize,line join=round, line cap=round,>=stealth]
			\tkzDefPoints{0/0/B,1/3/A,6/0/C}
			\tkzDefMidPoint(C,B) \tkzGetPoint{M}
			\tkzDrawPoints[fill=black](A,B,C,M)
			\tkzDrawPolygon(A,B,C)
			\tkzDrawSegments(A,M)
			\tkzLabelPoints[above](A)
			\tkzLabelPoints[below](B,C,M)
			\end{tikzpicture}}
	}
\end{ex}


\begin{ex}%[0H2Y3-1]
	Tam giác $ ABC$ có $ AB=\sqrt{2}$, $AC=\sqrt{3}$ và $ \widehat{C}=45^\circ $. Tính độ dài cạnh $ BC$.
	\choice
	{$ BC=\sqrt{5}$}
	{\True $ BC=\dfrac{\sqrt{6}+\sqrt{2}}{2}$}
	{$ BC=\sqrt{6}$}
	{$ BC=\dfrac{\sqrt{6}-\sqrt{2}}{2}$}
	\loigiai
	{Theo định lí hàm cô-sin, ta có\\
		$ AB^2=AC^2+BC^2-2\cdot AC\cdot BC\cdot \cos \widehat{C}\Rightarrow {(\sqrt{2} )}^2={(\sqrt{3} )}^2+BC^2-2\cdot \sqrt{3}\cdot BC\cdot \cos 45^\circ $ \\
		$ \Rightarrow BC=\dfrac{\sqrt{6}+\sqrt{2}}{2}$.}
\end{ex}


\begin{ex}%[0H2Y3-1]
	Tam giác $ ABC$ có $ AB=6$ cm, $AC=8$ cm và $ BC=10$ cm. Độ dài đường trung tuyến xuất phát từ đỉnh $ A$ của tam giác bằng
	\choice
	{$ 4$ cm}
	{$ \sqrt{3}$ cm}
	{$ 7$ cm}
	{\True $ 5$ cm}
	\loigiai{
		\immini{
			Áp dụng công thức đường trung tuyến ta có\\
			$\begin{aligned}
			m_a^2&=\dfrac{b^2+c^2}{2}-\dfrac{a^2}{4}\\
			&=\dfrac{8^2+6^2}{2}-\dfrac{10^2}{4}\\
			&=25\\
			\Rightarrow m_a&=5.
			\end{aligned}$.
		}
		{
			\begin{tikzpicture}[scale=1,font=\footnotesize,line join=round, line cap=round,>=stealth]
			\tkzDefPoints{0/0/B,1/3/A,6/0/C}
			\tkzDefMidPoint(C,B) \tkzGetPoint{M}
			\tkzDrawPoints[fill=black](A,B,C,M)
			\tkzDrawPolygon(A,B,C)
			\tkzDrawSegments(A,M)
			\tkzLabelPoints[above](A)
			\tkzLabelPoints[below](B,C,M)
			\end{tikzpicture}
		}
	}
\end{ex}


\begin{ex}%[0H2Y3-1]
	Tam giác $ ABC$ có $ AB=2$, $AC=1$ và $ \widehat{A}=60^\circ $. Tính độ dài cạnh $ BC$.
	\choice
	{$ BC=\sqrt{2}$}
	{\True $ BC=\sqrt{3}$}
	{$ BC=1$}
	{$ BC=2$}
	\loigiai
	{Theo định lí hàm cô-sin, ta có\\
		$ BC^2=AB^2+AC^2-2AB\cdot AC\cdot \cos{A}=2^2+1^2-2\cdot 2\cdot 1\cdot \cos 60^\circ =3\\
		\Rightarrow BC=\sqrt{3}$.}
\end{ex}


\begin{ex}%[0H2B3-1]
	Tam giác $ ABC$ vuông tại $ A$ có $ AB=AC=30$ cm. Hai đường trung tuyến $ BF$ và $ CE$ cắt nhau tại $ G$. Diện tích tam giác $ GFC$ bằng
	\choice
	{$ \text{50 c}{{\text{m}}^{\text{2}}}$}
	{\True $ \text{75 c}{{\text{m}}^{\text{2}}}$}
	{$ \text{50}\sqrt{2}$ cm$^2$}
	{$ \text{15}\sqrt{105}$ cm$^2$}
	\loigiai{
		\immini{
			Vì $ F$ là trung điểm của $ AC$ $ \Rightarrow $ $ FC=\dfrac{1}{2}AC=15$ cm. \\
			Do $ G$ là trọng tâm tam giác $ ABC$ nên $ \dfrac{\mathrm{d}(B;(AC ) )}{\mathrm{d}(G;(AC ) )}=\dfrac{BF}{GF}=3.\\
			\Rightarrow \mathrm{d}(G;(AC ) )=\dfrac{1}{3}\mathrm{d}(B;(AC ) )=\dfrac{AB}{3}=10$ cm. \\
			Vậy diện tích tam giác $ GFC$ là\\
			$ S_{\Delta GFC}=\dfrac{1}{2}\cdot \mathrm{d}(G;(AC))\cdot FC=\dfrac{1}{2}\cdot 10\cdot 15=75$ cm$^2$.
		}
		{
			\begin{tikzpicture}[scale=1, font=\footnotesize, line join = round, line cap = round,>=stealth]
			\tkzDefPoints{0/0/B,4/0/C,2/2/A}
			\tkzDefMidPoint(A,C) \tkzGetPoint{F}
			\tkzDefMidPoint(A,B) \tkzGetPoint{E}
			\tkzInterLL(F,B)(C,E)    \tkzGetPoint{G}
			\tkzDrawPoints[fill=black](A,B,C,E,F,G)
			\tkzDrawPolygon(A,B,C)
			\tkzDrawSegments(F,B C,E)
			\tkzLabelPoints[above](A)
			\tkzLabelPoints[below](B,C,G)
			\tkzLabelPoints[left](E)
			\tkzLabelPoints[right](F)
			\tkzMarkRightAngles[size=0.2](C,A,B)
			\end{tikzpicture}
		}
	}
\end{ex}


\begin{ex}%[0H2B3-1]
	Tam giác $ ABC$ cân tại $ C$, có $ AB=9$ cm và $ AC=\dfrac{15}{2}$ cm. Gọi $ D$ là điểm đối xứng của $ B$ qua $ C$. Tính độ dài cạnh $ AD$
	\choice
	{$ AD=12\sqrt{2}$ cm}
	{$ AD=6$ cm}
	{\True $ AD=12$ cm}
	{$ AD=9$ cm}
	\loigiai{
		\immini{
			Ta có $ D$ là điểm đối xứng của $ B$ qua $ C$ suy ra $ C$ là trung điểm của $ BD$. \\
			$ \Rightarrow  AC$ là trung tuyến của tam giác $DAB$. \\
			Suy ra $ BD=2BC=2AC=15$. \\
			Theo hệ thức trung tuyến ta có\\
			$\begin{aligned}
			&AC^2=\dfrac{AB^2+AD^2}{2}-\dfrac{BD^2}{4}\\ \Rightarrow& AD^2=2AC^2+\dfrac{BD^2}{2}-AB^2\\
			\Rightarrow& AD^2= 2\cdot \left(\dfrac{15}{2} \right)^2+\dfrac{15^2}{2}-9^2=144.\\
			\Rightarrow &AD=12.
			\end{aligned}$
		}
		{\begin{tikzpicture}[scale=1,font=\footnotesize,line join=round, line cap=round,>=stealth]
			\tkzDefPoints{0/0/A,0/4/D,3/0/B}
			\tkzDefMidPoint(D,B) \tkzGetPoint{C}
			\tkzDrawPoints[fill=black](A,B,C,D)
			\tkzDrawPolygon(A,B,D)
			\tkzDrawSegments(A,C)
			\tkzLabelPoints[above](D)
			\tkzLabelPoints[below](B,A)
			\tkzLabelPoints[right](C)
			\tkzMarkRightAngles[size=0.2](D,A,B)
			\end{tikzpicture}}
	}
\end{ex}


\begin{ex}%[0H2B3-1]
	Tam giác $ ABC$ có $ AB=3$, $AC=6$, $\widehat{BAC}=60^\circ $. Tính độ dài đường cao $ h_a$ của tam giác.
	\choice
	{$ h_a=3\sqrt{3}$}
	{$ h_a=\sqrt{3}$}
	{$ h_a=\dfrac{3}{2}$}
	{\True $ h_a=3$}
	\loigiai
	{Áp dụng định lý hàm số cô-sin, ta có
		$ BC^2=AB^2+AC^2-2AB\cdot AC\cos A=27\Rightarrow BC=3\sqrt{3}$.\\
		Ta có $ S_{\Delta ABC}=\dfrac{1}{2}\cdot AB\cdot AC\cdot \sin{A}=\dfrac{1}{2}\cdot 3\cdot 6\cdot \sin 60^\circ=\dfrac{9\sqrt{3}}{2}$.\\
		Lại có $ S_{\Delta ABC}=\dfrac{1}{2}\cdot BC\cdot h_a\Rightarrow h_a=\dfrac{2S}{BC}=3$.}
\end{ex}


\begin{ex}%[0H2B3-1]
	Tam giác $ ABC$ có $ BC=21$ cm, $CA=17$ cm, $AB=10$ cm. Tính bán kính $ R$ của đường tròn ngoại tiếp tam giác $ ABC$.
	\choice
	{\True $ R=\dfrac{85}{8}$ cm}
	{$ R=\dfrac{7}{2}$ cm}
	{$ R=\dfrac{85}{2}$ cm}
	{$ R=\dfrac{7}{4}$ cm}
	\loigiai
	{Đặt $ p=\dfrac{AB+BC+CA}{2}=24$. Áp dụng công thức Hê – rông, ta có\\
		$ S_{\Delta ABC}=\sqrt{p(p-AB )(p-BC )(p-CA )}=\sqrt{24\cdot (24-21 )\cdot (24-17 )\cdot (24-10 )}=84$ cm$^2$. \\
		Vậy bán kính cần tìm là $ R=\dfrac{AB\cdot BC\cdot CA}{4\cdot S_{\Delta ABC}}=\dfrac{21\cdot 17\cdot 10}{4\cdot 84}=\dfrac{85}{8}$ cm.}
\end{ex}


\begin{ex}%[0H2B3-1]
	Tam giác $ ABC$ vuông tại $ A$, đường cao $ AH=32$ cm. Hai cạnh $ AB$ và $ AC$ tỉ lệ với $ 3$ và $ 4$. Cạnh nhỏ nhất của tam giác này có độ dài bằng bao nhiêu?
	\choice
	{\True $ 40$ cm}
	{$ 38$ cm}
	{$ 45$ cm}
	{$ 42$ cm}
	\loigiai
	{Do tam giác $ ABC$ vuông tại $ A$, có tỉ lệ 2 cạnh góc vuông $ AB:AC$ là $ 3:4$ nên $ AB$ là cạnh nhỏ nhất trong tam giác.\\
		Ta có $ \dfrac{AB}{AC}=\dfrac{3}{4}\Rightarrow AC=\dfrac{4}{3}AB$.\\
		Xét $ \Delta ABC$ có  đường cao $ AH$ . Ta có\\
		$$\dfrac{1}{AH^2}=\dfrac{1}{AB^2}+\dfrac{1}{AC^2}=\dfrac{1}{AB^2}+\dfrac{1}{\left(\dfrac{4}{3}AB^2\right)}\Rightarrow \dfrac{1}{32^2}=\dfrac{1}{AB^2}+\dfrac{9}{16AB^2}\Rightarrow AB=40.$$}
\end{ex}



\begin{ex}%[0H2B3-1]
	Tam giác $ ABC$ có $ AB=8$ cm, $ AC=18$ cm và có diện tích bằng $ 64$ cm$^2 $. Tính $ \sin A$.
	\choice
	{$ \sin A=\dfrac{3}{8}$}
	{$ \sin A=\dfrac{\sqrt{3}}{2}$}
	{\True $ \sin A=\dfrac{8}{9}$}
	{$ \sin A=\dfrac{4}{5}$}
	\loigiai
	{Ta có $ S_{\Delta ABC}=\dfrac{1}{2}\cdot AB\cdot AC\cdot \sin \widehat{BAC}\Leftrightarrow 64=\dfrac{1}{2}\cdot 8\cdot 18\cdot \sin A\Leftrightarrow \sin A=\dfrac{8}{9}$.}
\end{ex}


\begin{ex}%[0H2B3-1]
	Tam giác $ ABC$ có $ AB=3$, $BC=8$. Gọi $ M$ là trung điểm của $ BC$. Biết $ \cos \widehat{AMB}=\dfrac{5\sqrt{13}}{26}$ và $ AM>3$. Tính độ dài cạnh $ AC$.
	\choice
	{\True $ AC=7$}
	{$ AC=\sqrt{7}$}
	{$ AC=13$}
	{$ AC=\sqrt{13}$}
	\loigiai{
		\immini{
			Ta có $ M$ là trung điểm của $ BC \Rightarrow BM=\dfrac{BC}{2}=4$ \\
			Trong tam giác $ ABM$ ta có \\
			$\begin{aligned}
			&\cos \widehat{AMB}=\dfrac{AM^2+BM^2-AB^2}{2AM\cdot BM}\\
			\Leftrightarrow& AM^2-2AM\cdot BM\cdot \cos \widehat{AMB}+BM^2-AB^2=0 \\
			\Leftrightarrow& AM^2-\dfrac{20\sqrt{13}}{13}AM+7=0\\
			\Leftrightarrow& \hoac{& AM=\sqrt{13}>3\text{ (thỏa mãn)} \\
				& AM=\dfrac{7\sqrt{13}}{13}<3\text{ (loại)}}\\
			\Rightarrow &AM=\sqrt{13}.
			\end{aligned}$ \\
		}
		{
			\begin{tikzpicture}[scale=1,font=\footnotesize,line join=round, line cap=round,>=stealth]
			\tkzDefPoints{0/0/B,1/3/A,6/0/C}
			\tkzDefMidPoint(C,B) \tkzGetPoint{M}
			\tkzDrawPoints[fill=black](A,B,C,M)
			\tkzDrawPolygon(A,B,C)
			\tkzDrawSegments(A,M)
			\tkzLabelPoints[above](A)
			\tkzLabelPoints[below](B,C,M)
			\tkzMarkAngles[size=0.5cm,arc=l](A,M,B)
			\end{tikzpicture}}
		Ta có $ \widehat{AMB}$ và $ \widehat{AMC}$ là hai góc kề bù nên $\cos \widehat{AMC}=-\cos \widehat{AMB}=-\dfrac{5\sqrt{13}}{26}$. \\
		Trong tam giác $AMC$ ta có\\
		$ \begin{aligned}
		AC^2&=AM^2+CM^2-2AM\cdot CM\cdot \cos \widehat{AMC}
		=13+16-2\cdot \sqrt{13}\cdot 4\cdot \left(-\dfrac{5\sqrt{13}}{26} \right)=49.\\
		\Rightarrow AC&=7.
		\end{aligned}$
	}
\end{ex}


\begin{ex}%[0H2B3-1]
	Hình bình hành $ ABCD$ có $ AB=a$, $BC=a\sqrt{2}$ và $ \widehat{BAD}= 45^\circ$. Khi đó hình bình hành có diện tích bằng
	\choice
	{\True $ a^2$}
	{$ a^2\sqrt{3}$}
	{$ 2a^2$}
	{$ a^2\sqrt{2}$}
	\loigiai
	{Diện tích tam giác $ ABD$ là $ S_{\Delta ABD}=\dfrac{1}{2}\cdot AB\cdot AD\cdot \sin \widehat{BAD}=\dfrac{1}{2}\cdot a\cdot a\sqrt{2}\cdot \sin 45^\circ=\dfrac{a^2}{2}$. \\
		Vậy diện tích hình bình hành $ ABCD$ là $ S_{ABDD}=2\cdot S_{\Delta ABD}=2\cdot \dfrac{a^2}{2}=a^2$.}
\end{ex}


\begin{ex}%[0H2B3-1]
	Tam giác $ ABC$ có $ AB=5$, $AC=8$ và $ \widehat{BAC}=60^\circ$. Tính bán kính $ r$ của đường tròn nội tiếp tam giác đã cho.
	\choice
	{$ r=2$}
	{\True $ r=\sqrt{3}$}
	{$ r=1$}
	{$ r=2\sqrt{3}$}
	\loigiai
	{Áp dụng định lý hàm số cô-sin, ta có
		$ BC^2=AB^2+AC^2-2AB\cdot AC\cos A=49\Rightarrow BC=7$.\\
		Diện tích $ S=\dfrac{1}{2}AB\cdot AC\cdot \sin A=\dfrac{1}{2}\cdot 5\cdot 8\cdot \dfrac{\sqrt{3}}{2}=10\sqrt{3}$.\\
		Lại có $ S=p\cdot r\Rightarrow r=\dfrac{S}{p}=\dfrac{2S}{AB+BC+CA}=\sqrt{3}$.}
\end{ex}

\begin{ex}%[0H2T3-4]
	Từ vị trí $A$ người ta quan sát một cây cao (Hình vẽ). Biết $AH=4$ m, $HB=20$ m, $\widehat{BAC}=45^{\circ}$. Chiều cao của cây gần nhất với giá trị nào sau đây?
	\begin{center}
		\usetikzlibrary{decorations.pathmorphing,shapes}
		\tikzset{
			treetop/.style = {
				decoration={random steps, segment length=0.4mm},
				decorate
			},
			trunk/.style = {
				decoration={random steps, segment length=2mm, amplitude=0.2mm},
				decorate
			}
		}
		\tikzset{
			man/.pic={%
				\fill [rounded corners=1.5] (0,0.4) -- (0,0.4 -- (0.4,0.5 -- (0.4,0.4) --
				(0.325,0.4) -- (0.325,0.7) -- (0.3,0.7) -- (0.3,0) -- (0.225,0) --
				(0.225,0.4) -- (0.175,0.4) -- (0.175,0) -- (0.1,0) -- (0.1,0.7) --
				(0.075,0.7) -- (0.075,0.4) -- cycle;
				\fill (0.2,0.9) circle (0.1);
				\coordinate (-head) at (0.2,1);
				\coordinate (-foot) at (0.2,0);
			}
		}
		\begin{tikzpicture}
			\path 
			(-5,-2.09) coordinate (A)
			(0,1.5) coordinate (C)
			(0.1,-3) coordinate (T)
			(-5,-3) coordinate (H)
			(0,-3) coordinate (B)		
			;
			%\pic[red] at (-6.3,-3) (myman) {man};
			\foreach \w/\f in {0.3/30,0.2/50,0.1/70} {
				\fill [brown!\f!black, trunk] (0,0) ++(-\w/2,0) rectangle +(\w,-3);
			}
			\foreach \n/\f in {1.4/40,1.2/50,1/60,0.8/70,0.6/80,0.4/90} {
				\fill [green!\f!black, treetop] ellipse (\n/1.5 and \n);
			}
			\draw (H)--(T) (A)--(H) (A)--(C) (A)--(B);
			\pic[draw,"$45^{\circ}$", angle eccentricity=1.4,angle radius=0.8cm]{angle=B--A--C};
			\pic[draw,"$ $", angle eccentricity=1.4,angle radius=0.7cm]{angle=B--A--C};
			\draw pic[draw, angle radius=2mm]{right angle=B--H--A};
			\path (H)--(B) node[below,midway,sloped]{$20$ m};
			\foreach \x/\g in {A/180,B/-90,C/90,H/-90} \fill[black] (\x)+(\g:.3) node {$\x$};
		\end{tikzpicture}
	\end{center}
	\choice
	{$14$ m}
	{$15$ m}
	{\True $17$ m}
	{$16$ m}
	\loigiai{\immini{Ta có $AB= \sqrt{AH^2 + BH^2} = \sqrt{4^2+20^2} = 4 \sqrt{26}$.\\
			$\tan \widehat{HAB} = \dfrac{HB}{HA} = \dfrac{20}{4} = 5 \Rightarrow \widehat{HAB} \approx 78{,}69^{\circ}$.\\
			Do $AH \parallel BC$ nên $ \widehat{ABC} = \widehat{HAB} \approx 78{,}69^{\circ}$.\\
			$\widehat{ACB} = 180^{\circ} - 45^{\circ} - \widehat{ABC} \approx 56{,}31^{\circ}$.\\
			Áp dụng định lí hàm số $\sin$ trong tam giác $ABC$ ta có
			$$ \dfrac{BC}{\sin 45^{\circ}} = \dfrac{AB}{\sin 56{,}31^{\circ}} = \dfrac{4 \sqrt{26}}{\sin 56{,}31^{\circ}} \Rightarrow BC \approx  17{,}33.$$	}
		{\begin{tikzpicture}[scale=0.8, font=\footnotesize,line join = round, line cap = round, >=stealth]
				\tkzDefPoints{-5/-2.09/A, 0/1.5/C, 0.1/-3/T, -5/-3/H,0/-3/B}
				\tkzDrawSegments(H,B A,H A,C A,B B,C)
				\tkzMarkAngles[size=0.6,arc=ll](B,A,C)
				\tkzMarkRightAngles(B,H,A)
				\tkzLabelAngles[pos=1.1](B,A,C){$45^{\circ}$}
				\tkzLabelPoints[above](C)
				\tkzLabelPoints[above](A)
				\tkzLabelPoints[below](B)
				\tkzLabelPoints[below](H)
		\end{tikzpicture}}
	}
\end{ex}


\begin{ex}%[0H2B3-1]
	Tam giác $ ABC$ có $ AB=\dfrac{\sqrt{6}-\sqrt{2}}{2}$, $BC=\sqrt{3}$, $CA=\sqrt{2}$. Gọi $ D$ là chân đường phân giác trong góc $ \widehat{A}$. Khi đó góc $ \widehat{ADB}$ bằng
	\choice
	{$ 90^\circ $}
	{$ 45^\circ $}
	{$ 60^\circ $}
	{\True $ 75^\circ $}
	\loigiai{
		\immini
		{
			Theo định lí hàm cô-sin, ta có\\
			$ \begin{aligned}
			& \cos \widehat{BAC}=\dfrac{AB^2+AC^2-BC^2}{2\cdot AB\cdot AC}=-\dfrac{1}{2}. \\
			\Rightarrow& \widehat{BAC}=120^\circ \Rightarrow \widehat{BAD}=60^\circ. \\
			\end{aligned}$ \\
			$ \begin{aligned}
			& \cos \widehat{ABC}=\dfrac{AB^2+BC^2-AC^2}{2\cdot AB\cdot BC}=\dfrac{\sqrt{2}}{2}.\\
			\Rightarrow &\widehat{ABC}=45^\circ.
			\end{aligned}$ \\
			Trong $ \Delta ABD$ có $ \widehat{BAD}=60^\circ$, $\widehat{ABD}=45^\circ. \\
			\Rightarrow \widehat{ADB}=75^\circ $.
		}
		{
			\begin{tikzpicture}[scale=1, font=\footnotesize, line join = round, line cap = round,>=stealth]
			\tkzDefPoints{-2/0/B,0/3/A,5/0/C}
			\clip (-2.2,-0.5) rectangle (5.2,3.5);
			\tkzInCenter(A,B,C)    \tkzGetPoint{I}
			\tkzInterLL(A,I)(C,B)    \tkzGetPoint{D}
			\tkzDrawPoints[fill=black](A,B,C,D)
			\tkzDrawPolygon(A,B,C)
			\tkzDrawSegments(D,A)
			\tkzLabelPoints[above](A)
			\tkzLabelPoints[below](D,B,C)
			\tkzMarkAngles[size=0.5cm,arc=l,mark=||](B,A,D D,A,C)
			\end{tikzpicture}
		}
	}
\end{ex}


\begin{ex}%[0H2B3-1]
	Tam giác $ ABC$ có đoạn thẳng nối trung điểm của $ AB$ và $ BC$ bằng $ 3$, cạnh $ AB=9$ và $ \widehat{ACB}=60^\circ $. Tính độ dài cạnh cạnh $ BC$.
	\choice
	{\True $ BC=3+3\sqrt{6}$}
	{$ BC=3\sqrt{7}$}
	{$ BC=\dfrac{3+3\sqrt{33}}{2}$}
	{$ BC=3\sqrt{6}-3$}
	\loigiai{
		\immini{
			Gọi $ M$, $N$ lần lượt là trung điểm của $ AB$, $BC$.\\
			$ \Rightarrow MN$ là đường trung bình của $ \Delta ABC$
			$ \Rightarrow MN=\dfrac{1}{2}AC$.\\
			Mà $ MN=3$, suy ra $ AC=6$.\\
			Theo định lí hàm cô-sin, ta có\\
			$ \begin{aligned}
			&AB^2=AC^2+BC^2-2\cdot AC\cdot BC\cdot \cos \widehat{ACB} \\
			\Leftrightarrow & 9^2=6^2+BC^2-2\cdot 6\cdot BC\cdot \cos 60^\circ \\
			\Rightarrow& BC=3+3\sqrt{6}.
			\end{aligned}$}{
			\begin{tikzpicture}[scale=0.7,font=\footnotesize,line join=round, line cap=round,>=stealth]
			\tkzDefPoints{0/0/B,4/3/A,6/0/C}
			\tkzCircumCenter(A,B,C)
			\tkzDefMidPoint(A,B) \tkzGetPoint{M}
			\tkzDefMidPoint(C,B) \tkzGetPoint{N}
			\tkzDrawPoints[fill=black](A,B,C,M,N)
			\tkzDrawPolygon(A,B,C)
			\tkzDrawSegments(M,N)
			\tkzLabelPoints[above](A,M)
			\tkzLabelPoints[below](B,C,N)
			\tkzMarkAngles[size=0.5cm,arc=l](A,C,B)
			\tkzLabelAngles[pos=0.8](A,C,B){$60^\circ$}
			\end{tikzpicture}}
	}
\end{ex}


\begin{ex}%[0H2B3-4]
	\immini[thm]
	{Trên nóc một tòa nhà có một cột ăng-ten cao $ 5$ m. Từ vị trí quan sát $ A$ cao $ 7$ m so với mặt đất, có thể nhìn thấy đỉnh $ B$ và chân $ C$ của cột ăng-ten dưới góc $ 50^\circ$ và $ 40^\circ$ so với phương nằm ngang. Chiều cao của tòa nhà gần nhất với giá trị nào sau đây?
		\choicew{0.5\textwidth}
		\choice
		{ $ 29$ m}
		{ $ 24$ m}
		{ \True $ 19$ m}
		{$ 12$ m}
	}
	{
		\begin{tikzpicture}[scale=.5, font=\footnotesize, line join=round, >=stealth]
		\coordinate (O) at (0,0);
		\def\nhathap{2} %so tang nha thap
		\def\nhacao{5} %so tang nha cao
		\def\kc{10} %khoang cach 2 nha
		\def\dai{1} %chieu dai 1 khoi
		\def\cao{2} %chieu cao 1 khoi
		\coordinate (H) at (\kc,0);
		\newcommand{\xaynha}[2]{
			\draw[very thick] (#1,#2)rectangle(#1+\dai,#2+\cao);
			\draw[very thick,fill=black] (#1,#2)rectangle(#1+\dai,#2+0.25*\cao);
			\draw[very thick] (#1+0.2*\dai,#2+0.25*\cao)rectangle(#1+0.8*\dai,#2+0.9*\cao);
		}
		%Xay nha thap
		\foreach \j in {1,...,\nhathap}{
			\foreach \i in {0,...,4}{
				\xaynha{\i*1.1}{\j*\cao}
			}
		}
		\draw[line width=0.2cm] (-0.1,\cao*\nhathap+\cao)--(5*\dai*1.1,\cao*\nhathap+\cao);
		\foreach \j in {1,...,\nhacao}{
			\foreach \i in {0,...,4}{
				\xaynha{\i*1.1+\kc}{\j*\cao}
			}
		}
		\draw[line width=0.2cm] (-0.1+\kc,\cao*\nhacao+\cao)--(5*\dai*1.1+\kc,\cao*\nhacao+\cao);
		\coordinate (A) at (5*\dai*1.1,\cao*\nhathap+\cao+0.1);
		\draw (A) node[above left]{$A$};
		\coordinate (C) at (\kc,\cao*\nhacao+\cao);
		\draw (C) node[above right]{$C$};
		\coordinate (B) at ($(C)+(0,2)$);
		\draw (B) node[above]{$B$};
		\coordinate (H) at (\kc,\cao);
		\draw (H) node[below]{$H$};
		\coordinate (D) at (\kc,\cao*\nhathap+\cao+0.1);
		\draw (D) node[above left]{$D$};
		\draw[very thick] (-0.5,\cao)--(\kc+5*\dai*1.2,\cao);
		\draw [dashed](A)--(D); \draw (B)--(A)--(C);\draw[line width=0.07cm] (C)--(B);
		\draw ($(B)+(135:0.5cm)$)--($(B)+(-45:0.5cm)$);
		\tkzMarkAngle[size=2 cm](D,A,B) %[mark=s,arc=ll,size=2 cm,mkcolor=red]
		\tkzLabelAngle[pos=2.7](D,A,B){$50^{\circ}$}
		\tkzMarkAngle[size=.5 cm,arc=l](D,A,C) %[mark=s,arc=ll,size=2 cm,mkcolor=red]
		\tkzLabelAngle[pos=1.4](D,A,B){$40^{\circ}$}
		\end{tikzpicture}
	}
	\loigiai
	{Từ hình vẽ, suy ra $ \widehat{BAC}=10^\circ$ và
		$ \widehat{ABD}=180^\circ-(\widehat{BAD}+\widehat{ADB} )=180^\circ-(50^\circ+90^\circ )=40^\circ$.\\
		Áp dụng định lí sin trong tam giác $ ABC$, ta có\\
		$ \dfrac{BC}{\sin \widehat{BAC}}=\dfrac{AC}{\sin \widehat{ABC}}\Rightarrow AC=\dfrac{BC\cdot \sin \widehat{ABC}}{\sin \widehat{BAC}}\text{=}\dfrac{5\cdot \sin 40^\circ}{\sin 10^\circ}\simeq  18{,}5$ m.\\
		Trong tam giác vuông $ ADC$, ta có $ \sin \widehat{CAD}=\dfrac{CD}{AC}\Rightarrow CD=AC\cdot \sin \widehat{CAD}=11{,}9$ m. \\
		Vậy $ CH=CD+DH=11{,}9+7=18{,}9$ m.}
\end{ex}


\begin{ex}%[0H2B3-1]
	Tam giác $ ABC$ có $ AB=4$, $BC=6$, $AC=2\sqrt{7}$. Điểm $ M$ thuộc đoạn $ BC$ sao cho $ MC=2MB$. Tính độ dài cạnh $ AM$.
	\choice
	{$ AM=4\sqrt{2}$}
	{$ AM=3\sqrt{2}$}
	{\True $ AM=2\sqrt{3}$}
	{$ AM=3$}
	\loigiai{
		\immini
		{
			Theo định lí hàm cô-sin, ta có\\
			$ \cos B=\dfrac{AB^2+BC^2-AC^2}{2\cdot AB\cdot BC}=\dfrac{4^2+6^2-(2\sqrt{7} )^2}{2\cdot 4\cdot 6}=\dfrac{1}{2}$.\\
			Do $ MC=2MB\Rightarrow BM=\dfrac{1}{3}BC=2$.\\
			Theo định lí hàm cô-sin, ta có\\
			$ \begin{aligned}
			AM^2&=AB^2+BM^2-2\cdot AB\cdot BM\cdot \cos B \\
			& =4^2+2^2-2\cdot 4\cdot 2\cdot \dfrac{1}{2}=12.
			\end{aligned}\\
			\Rightarrow AM=2\sqrt{3}$.
		}
		{
			\begin{tikzpicture}[scale=0.9, font=\footnotesize, line join = round, line cap = round,>=stealth]
			\tkzDefPoints{-2/0/B,0/3/A,5/0/C}
			\coordinate (M) at ($(B)!0. 33!(C)$);
			\tkzDrawPoints[fill=black](A,B,C,M)
			\tkzDrawPolygon(A,B,C)
			\tkzDrawSegments(M,A)
			\tkzLabelPoints[above](A)
			\tkzLabelPoints[below](M,B,C)
			\end{tikzpicture}
	}}
\end{ex}


\begin{ex}%[0H2B3-1]
	Cho hình thoi $ ABCD$ cạnh bằng $ 1$ cm và có $ \widehat{BAD}=60^\circ $. Tính độ dài cạnh $ AC$.
	\choice
	{$ AC=2$}
	{\True $ AC=\sqrt{3}$}
	{$ AC=2\sqrt{3}$}
	{$ AC=\sqrt{2}$}
	\loigiai{
		\immini{
			Do $ ABCD$ là hình thoi, có $ \widehat{BAD}=60^\circ \Rightarrow \widehat{ABC}=120^\circ $.\\
			Theo định lí hàm cô-sin, ta có\\
			$ \begin{aligned}
			AC^2&=AB^2+BC^2-2\cdot AB\cdot BC\cdot \cos \widehat{ABC} \\
			& =1^2+1^2-2\cdot 1\cdot 1\cdot \cos 120^\circ\\
			& =3.\\
			\Rightarrow AC=\sqrt{3}\cdot  \\
			\end{aligned}$}
		{
			\begin{tikzpicture}[scale=1, font=\footnotesize, line join = round, line cap = round,>=stealth]
			\tkzDefPoints{-2/0/A,0/3/B}
			\tkzDefPointBy[rotation = center A angle -60](B)    \tkzGetPoint{D}
			\coordinate (C) at ($(B)+(D)-(A)$);
			\tkzDrawPoints[fill=black](A,B,C,D)
			\tkzDrawPolygon(A,B,C,D)
			\tkzDrawSegments(C,A D,B)
			\tkzLabelPoints[above](B,C)
			\tkzLabelPoints[below](A,D)
			\tkzMarkAngles[size=0.5cm,arc=l](D,A,B)
			\end{tikzpicture}}
	}
\end{ex}


\begin{ex}%[0H2B3-2]
	Cho hình bình hành $ ABCD$ có $ AB=a$, $BC=b$, $BD=m$ và $ AC=n$. Trong các biểu thức sau, biểu thức nào đúng?
	\choice
	{$ m^2+m^2=3(a^2+b^2 )$}
	{$ 3(m^2+m^2 )=a^2+b^2$}
	{\True $ m^2+m^2=2(a^2+b^2 )$}
	{ $ 2(m^2+m^2 )=a^2+b^2$}
	\loigiai{
		\immini
		{
			Gọi $ O$ là giao điểm của $ AC$ và $ BD$ ta có
			$ BO=\dfrac{1}{2}BD=\dfrac{m}{2}$. \\
			Do $ BO$ là trung tuyến của tam giác $ABC$ nên \\
			$\begin{aligned}
			&BO^2=\dfrac{BA^2+BC^2}{2}-\dfrac{AC^2}{4}\\
			\Leftrightarrow& \dfrac{m^2}{4}=\dfrac{a^2+b^2}{2}-\dfrac{n^2}{4}\\
			\Leftrightarrow& m^2+n^2=2(a^2+b^2 ).
			\end{aligned} $
		}
		{
			\begin{tikzpicture}[scale=0.7, font=\footnotesize, line join = round, line cap = round]
			\def \d{6} \def \r{-2}
			\tkzDefPoints{0/0/A,\r/-3/B,\d/0/D}
			\coordinate (C) at ($(B)+(D)-(A)$);
			\tkzDefMidPoint(A,C) \tkzGetPoint{O}
			\tkzDrawPoints[fill=black](A,B,C,D,O)
			\tkzDrawPolygon(B,C,D,A)
			\tkzDrawSegments(A,C B,D)
			\tkzLabelPoints[below](B,C,O)
			\tkzLabelPoints[above](A,D)
			\end{tikzpicture}
		}
	}
\end{ex}


\begin{ex}%[0H2B3-3]
	Tam giác $ ABC$ có ba đường trung tuyến $ m_a$, $m_b$, $m_c$ thỏa mãn $ 5m_a^2=m_b^2+m_{c}^2$. Khi đó tam giác này là tam giác gì?
	\choice
	{Tam giác đều}
	{Tam giác vuông cân}
	{ \True Tam giác vuông}
	{Tam giác cân}
	\loigiai{
		Ta có
		\begin{eqnarray*}
			&&5m_a^2=m_b^2+m_{c}^2\\
			&\Leftrightarrow& 5\left(\dfrac{b^2+c^2}{2}-\dfrac{a^2}{4} \right)=\dfrac{a^2+c^2}{2}-\dfrac{b^2}{4}+\dfrac{a^2+b^2}{2}-\dfrac{c^2}{4}\\
			& \Leftrightarrow& 10b^2+10c^2-5a^2=2a^2+2c^2-b^2+2a^2+2b^2-c^2\\
			& \Leftrightarrow& b^2+c^2=a^2.
		\end{eqnarray*}
		Suy ra tam giác $ABC$ vuông.}
\end{ex}


\begin{ex}%[0H2K3-1]
	Cho tam giác $ ABC$ có $ AB=3\sqrt{3}$, $BC=6\sqrt{3}$ và $ CA=9$. Gọi $ D$ là trung điểm $ BC$. Tính bán kính $ R$ của đường tròn ngoại tiếp tam giác $ ABD$.
	\choice
	{$ R=\dfrac{9}{2}$}
	{\True $ R=3$}
	{$ R=\dfrac{9}{6}$}
	{$ R=3\sqrt{3}$}
	\loigiai{
		Vì $ D$ là trung điểm của $ BC$ $ \Rightarrow $ $ AD^2=\dfrac{AB^2+AC^2}{2}-\dfrac{BC^2}{4}=27$ $ \Rightarrow $ $ AD=3\sqrt{3}$. \\
		Tam giác $ ABD$ có $ AB=BD=DA=3\sqrt{3}$ $ \Rightarrow $ tam giác $ ABD$ đều.\\
		Suy ra bán kính đường tròn ngoại tiếp là $ R=\dfrac{\sqrt{3}}{3}AB=\dfrac{\sqrt{3}}{3}\cdot 3\sqrt{3}=3$.}
\end{ex}


\begin{ex}%[0H2K3-1]
	Cho góc $ \widehat{xOy}=30^\circ $. Gọi $ A$ và $ B$ là hai điểm di động lần lượt trên $ Ox$ và $ Oy$ sao cho $ AB=1$. Độ dài lớn nhất của đoạn $ OB$ bằng
	\choice
	{$ \dfrac{3}{2}$}
	{\True $ 2$}
	{$ 2\sqrt{2}$}
	{$ \sqrt{3}$}
	\loigiai{
		\immini
		{
			Theo định lí hàm sin, ta có\\
			$\begin{aligned}
			OB&=\dfrac{AB}{\sin \widehat{AOB}}\cdot \sin \widehat{OAB}\\
			&=\dfrac{1}{\sin 30^\circ}\cdot \sin \widehat{OAB}\\
			&=2\sin \widehat{OAB}.\\
			&\ge 2
			\end{aligned}$  \\
			Suy ra  $ \max OB=2$  khi
			$ \sin \widehat{OAB}=1 \Leftrightarrow \widehat{OAB}=90^\circ $.
		}
		{
			\begin{tikzpicture}[scale=1, font=\footnotesize, line join = round, line cap = round,>=stealth]
			\tkzDefPoints{0/0/O,5/0/x,4/3/y,2/0/A,2.5/1.88/B}
			\tkzDrawPoints[fill=black](A,B,O)
			\tkzDrawSegments(O,x O,y A,B)
			\tkzLabelPoints[above](B,y)
			\tkzLabelPoints[below](O,A,x)
			\tkzMarkAngles[size=1cm,arc=l](A,O,B)
			\tkzLabelAngles[pos=0.75,rotate=30](A,O,B){$30^\circ$}
			\end{tikzpicture}
		}
	}
\end{ex}


\begin{ex}%[0H2K3-1]
	Tam giác $ ABC$ có $ AB=c$, $BC=a$, $CA=b$. Các cạnh $ a$, $b$, $c$ liên hệ với nhau bởi đẳng thức $ a^2+b^2=5c^2$. Góc giữa hai trung tuyến $ AM$ và $ BN$ bằng
	\choice
	{$ 30^\circ$}
	{\True $ 90^\circ$}
	{$ 60^\circ$}
	{$ 45^\circ$}
	\loigiai
	{Gọi $ G$ là trọng tâm tam giác $ABC$. Ta có\\
		$AM^2=\dfrac{AC^2+AB^2}{2}-\dfrac{BC^2}{4}=\dfrac{b^2+c^2}{2}-\dfrac{a^2}{4}$ $ \Rightarrow AG^2=\dfrac{4}{9}AM^2=\dfrac{2(b^2+c^2 )}{9}-\dfrac{a^2}{9}$. \\
		$BN^2=\dfrac{BA^2+BC^2}{2}-\dfrac{AC^2}{4}=\dfrac{c^2+a^2}{2}-\dfrac{b^2}{4}$ $ \Rightarrow GN^2=\dfrac{1}{9}BN^2=\dfrac{c^2+a^2}{18}-\dfrac{b^2}{36}$.\\
		Trong tam giác $AGN$ ta có
		\begin{eqnarray*}
			\cos \widehat{AGN}&=&\dfrac{AG^2+GN^2-AN^2}{2\cdot AG\cdot GN}
			=\dfrac{\dfrac{2(b^2+c^2 )}{9}-\dfrac{a^2}{9}+\dfrac{c^2+a^2}{18}-\dfrac{b^2}{36}-\dfrac{b^2}{4}}{2\cdot \sqrt{\dfrac{2(b^2+c^2 )}{9}-\dfrac{a^2}{9}}\cdot \sqrt{\dfrac{c^2+a^2}{18}-\dfrac{b^2}{36}}} \\
			&=&\dfrac{\dfrac{2(b^2+c^2 )}{9}-\dfrac{a^2}{9}+\dfrac{c^2+a^2}{18}-\dfrac{b^2}{36}-\dfrac{b^2}{4}}{2\cdot \sqrt{\dfrac{2(b^2+c^2 )}{9}-\dfrac{a^2}{9}}\cdot \sqrt{\dfrac{c^2+a^2}{18}-\dfrac{b^2}{36}}}  =\dfrac{10c^2-2(a^2+b^2 )}{36\cdot 2\cdot \sqrt{\dfrac{2(b^2+c^2 )}{9}-\dfrac{a^2}{9}}\cdot \sqrt{\dfrac{c^2+a^2}{18}-\dfrac{b^2}{36}}}=0.
		\end{eqnarray*}
		$ \Rightarrow \widehat{AGN}=90^\circ$.}
\end{ex}

\centerline{\textbf{---HẾT---}}
\Closesolutionfile{ans}