\begin{name}
	{\tenchude}
	{ĐỀ ÔN TẬP CHƯƠNG I}
	{LỚP TOÁN THẦY PHÁT}
	{\thoigian}
\end{name}
\TN
\setcounter{ex}{0}
\Opensolutionfile{ans}[ans/ans-TN-C1-OT-De1]

\begin{ex}%[0D1N1-1]
	Câu nào sau đây là một mệnh đề?
	\choice
	{\True Số 2 là số nguyên tố}
	{$x+y>2$}
	{Bạn nào học giỏi Toán nhất lớp 10A?}
	{Hôm nay trời đẹp quá!}
	\loigiai{
		Câu \lq\lq  Số 2 là số nguyên tố\rq\rq\ là một mệnh đề.
	}
\end{ex}

\begin{ex}%[0D1H1-2]
	Cho mệnh đề chứa biến $P(x)\colon$ \lq\lq  $x^{2}-3x>0$\rq\rq\ với $x$ là số thực. Mệnh đề nào sau đây đúng?
	\choice
	{$P(3)$}
	{\True $P(4)$}
	{$P(1)$}
	{$P(2)$}
	\loigiai
	{
		Ta có $P(4)=4^2-3\cdot4=4>0$ nên $P(4)$ là mệnh đề đúng.
	}
\end{ex}

\begin{ex}%[0D1N1-3]
	Mệnh đề nào sau đây là phủ định của mệnh đề mệnh đề: \lq\lq  $\pi$ là một số hữu tỉ\rq\rq?
	\choice{\lq\lq  $\pi$ không phải là một số thực\rq\rq}
	{\True\lq\lq  $\pi$ không phải là một số hữu tỉ\rq\rq}
	{\lq\lq  $\pi$ không phải là một số vô tỉ\rq\rq}
	{\lq\lq  $\pi$ là một số nguyên\rq\rq}
	\loigiai{
		Mệnh đề phủ định: \lq\lq  $\pi$ không phải là một số hữu tỉ\rq\rq.
	}
\end{ex}
\begin{ex}%[0D1H1-4]
	Mệnh đề kéo theo $P \Rightarrow Q$ sai khi
	\choice
	{\True $P$ đúng và $Q$ sai}
	{$P$ sai và $Q$ sai}
	{$P$ sai và $Q$ đúng}
	{$P$ đúng và $Q$ đúng}
	\loigiai{
		Mệnh đề kéo theo $P \Rightarrow Q$ sai khi $P$ đúng và $Q$ sai.
	}
\end{ex}

\begin{ex}%[0D1H1-2]
	Trong các mệnh đề sau, mệnh đề nào \textbf{sai}?
	\choice
	{\True \lq\lq  $\forall x \in \mathbb{N}: x<2 x $ \rq\rq}
	{\lq\lq  $\forall x \in \mathbb{R}: x^2 \geq 0 $\rq\rq}
	{\lq\lq  $\exists x \in \mathbb{R}: x^2-3 x+2=0$ \rq\rq}
	{\lq\lq  $\exists x \in \mathbb{N}: x^2=x$\rq\rq}
	\loigiai{
		Mệnh đề \lq\lq  $\forall x \in \mathbb{N}: x<2 x $ \rq\rq\ là mệnh đề sai. Ví dụ $x=0$ thì $x=2x$.
	}
\end{ex}

\begin{ex}%[0D1B2-1]
	Hãy liệt kê các phần tử của tập hợp $X=\left\{x \in \mathbb{Z} | 2 x^{2}-5 x+2=0\right\}$.
	\choice
	{$X=\{0\}$}
	{$X=\left\{\dfrac{1}{2}\right\}$}
	{\True $X=\{2\}$}
	{$X=\left\{2 ; \dfrac{1}{2}\right\}$}
	\loigiai{
		Giải phương trình:	$2 x^{2}-5 x+2=0$ ta được hai nghiệm $x_1 =2 ; x_2 = \dfrac{1}{2} $.\\
		Do $x \in \mathbb{Z}$ nên $X=\{2\}$.
	}
\end{ex}

\begin{ex}%[0D1N3-1]
	Cho hai tập hợp $A=\{0 ; 2 ; 3 ; 6 ; 7 ; 8\}$ và $B=\{-1 ; 2 ; 5\}$. Hợp của hai tập hợp $A$ và $B$ là
	\choice
	{$A \cup B=\{-4 ;-3\}$}
	{\True $A \cup B=\{-1 ; 0 ; 2 ; 3 ; 5 ; 6 ; 7 ; 8\}$}
	{$A \cup B=\{0 ; 2 ; 5\}$}
	{$A \cup B=\{-1 ; 0 ; 2 ; 3 ; 5\}$}
	\loigiai
	{
		Ta có $A \cup B=\{-1 ; 0 ; 2 ; 3 ; 5 ; 6 ; 7 ; 8\}$.
	}
\end{ex}

\begin{ex}%[0D1N3-1]
	Cho tập hợp $A=\{0;1;2;3;4\}$ và $B=\{-2;-1;0;1\}$. Giao của hai tập hợp $A$ và $B$ là
	\choice
	{\True $A\cap B=\{0;1\}$}
	{$A\cap B=\{0;1;2;3;4\}$}
	{$A\cap B=\{-2;-1;0;1\}$}
	{$A\cap B=\{-2;-1;0;1;2;3;4\}$}
	\loigiai
	{
		Ta có $A\cap B=\{0;1\}$.
	}
\end{ex}

\begin{ex}%[0D1H3-2]
	\immini
	{Cho $A$, $B$, $C$ là ba tập hợp bất kì khác rỗng, được biểu diễn bằng biểu đồ Ven như hình bên. Phần gạch sọc trong hình vẽ biểu diễn tập hợp nào sau đây?
		\choice
		{$(A \cup B) \setminus C$}
		{\True $(A \cap B) \setminus C$}
		{$(A \cap B) \cap C$}
		{$(A \cap B) \cup C$}}
	{
		\begin{tikzpicture}[line cap=round,line join=round,font=\footnotesize,>=stealth,scale=1]
			\coordinate (A) at (180:1);
			\coordinate (B) at (0:1);
			\coordinate (C) at (-90:1);
			\begin{scope}
				\clip (B)  ellipse (1.5cm and 0.75cm);
				\fill[pattern=north west lines] (A)  ellipse (1.5cm and 0.75cm);
				\fill[white] (C) circle[radius=1];
			\end{scope}
			\foreach \coord in {A,B}
				{
					\draw (\coord)  ellipse (1.5cm and 0.75cm);
					\node at (\coord) {$\coord$};
				}
			\draw (C) circle[radius=1];
			\node at (C) {$C$};
		\end{tikzpicture}}
	\loigiai
	{
		Giả sử phần tử $x$ thuộc phần gạch sọc suy ra $\heva{&x\in A\cap B\\&x\notin C}\Rightarrow x\in(A \cap B) \setminus C$.
	}
\end{ex}
\begin{ex}%[0D1H3-3]
	Cho hai tập hợp $A=[-2 ; 4]$, $B=(0 ;+\infty)$. Hãy xác định tập $A \cup B$.
	\choice
	{$[-2 ; 4]$}
	{$[-2 ; 0)$}
	{$(-2 ;+\infty)$}
	{\True $[-2 ;+\infty)$}
	\loigiai
	{
	Ta có  $A \cup B=[-2 ;+\infty)$.
	}
\end{ex}
\begin{ex}%[0D1H3-4]
	Cho hai tập hợp  $A=\left( -1;+\infty  \right),B=\left( -\infty ;3 \right]$. Tìm $A\setminus B$.
		\choice
		{\True  $A\setminus B=\left( 3;+\infty  \right)$}
		{$A\setminus B=\left( -1;3 \right)$}
		{$A\setminus B=\left[ 3;+\infty  \right)$}
	{$A\setminus B=\left( -\infty ;1 \right]$}
	\loigiai
	{$A\setminus B=\left( 3;+\infty  \right)$.
	}
\end{ex}

\begin{ex}%[0D1H3-4]
	Cho tập hợp $A=(-\infty;-1) \cup[2; 3)$. Tìm $C_\mathbb{R} A$.
	\choice{$C_\mathbb{R} A=(-1; 2) \cup(3;+\infty)$}
	{$C_\mathbb{R} A=[-1; 3)$}
	{$C_\mathbb{R} A=[-1;+\infty)$}
	{\True$C_\mathbb{R} A=[-1; 2) \cup[3;+\infty)$}
	\loigiai{
	Ta có $C_\mathbb{R} A=[-1; 2) \cup[3;+\infty)$.
	}
\end{ex}

\Closesolutionfile{ans}

% \indapan{12}{ans/ans-TN-C1-OT-De1}

\TNTF
\setcounter{ex}{0}
\Opensolutionfile{ans}[ans/ans-TF-C1-OT-De1]

\begin{ex}%[0D1N1-4]
	Cho hai mệnh đề $P$: \lq\lq  Tứ giác $ABCD$ là hình vuông\rq\rq\ và mệnh đề $Q$: \lq\lq  Tứ giác $ABCD$ là hình chữ nhật có hai đường chéo vuông góc với nhau\rq\rq. Các câu sau là đúng hay sai?
	\choiceTF
	{\True Mệnh đề đảo của mệnh đề $P \Rightarrow Q$ là mệnh đề \lq\lq  Nếu $ABCD$ là hình chữ nhật có hai đường chéo vuông góc với nhau thì tứ giác $ABCD$ là hình vuông\rq\rq}
	{\True Mệnh đề $P \Rightarrow Q$ là mệnh đề đúng}
	{Mệnh đề $Q \Rightarrow P$ là mệnh đề sai}
	{\True $P$ là điều kiện cần và đủ để có $Q$}
	\loigiai{
		Mệnh đề $P \Rightarrow Q$: \lq\lq  Nếu $ABCD$ là hình vuông thì tứ giác $ABCD$ là hình chữ nhật có hai đường chéo vuông góc với nhau \rq\rq là mệnh đề đúng.\\
		Mệnh đề đảo của mệnh đề \lq\lq  $P \Rightarrow Q$ là mệnh đề \lq\lq  Nếu $ABCD$ là hình chữ nhật có hai đường chéo vuông góc với nhau thì tứ giác $ABCD$ là hình vuông\rq\rq cũng là mệnh đề đúng.\\
		Vậy $P$ và $Q$ là hai mệnh đề tương đương nên $P$ là điều kiện cần và đủ để có $Q$.
	}
\end{ex}

\begin{ex}%[0D1N2-2]
	Các câu sau là đúng hay sai?
	\choiceTF
	{\True Với hai tập $A=\left\{-\sqrt{3};\sqrt{3}\right\}$ và $B=\left\{x\in\mathbb{R}\Big|x^2-3=0\right\}$, ta có $A=B$}
	{\True Với $C$ là tập hợp các tam giác đều và $D$ là tập hợp các tam giác cân, ta có $C\subset D$}
	{Với hai tập $E=\left\{x\in\mathbb{N}\Big|x \text{ là ước của } 12\right\}$ và $F=\left\{x\in\mathbb{R}\Big|x \text{ là bội của } 24\right\}$, ta có $F\subset E$}
	{Tất cả các tập con của tập $\{g;h\}$ là $\{g\}$, $\{h\}$, $\{g;h\}$}
	\loigiai
	{
		\begin{itemchoice}
			\itemch Ta có $x^2-3=0 \Leftrightarrow \hoac{ & x=-\sqrt{3} \\ & x=\sqrt{3}} \Rightarrow B=\left\{-\sqrt{3};\sqrt{3}\right\}$. \\
			Vậy $A=B$.
			\itemch Vì tam giác đều là tam giác cân nên $C\subset D$.
			\itemch Với $x=24$, ta có $x\in F$ nhưng $x\notin E$ nên $F\not\subset E$.
			\itemch Tất cả các tập con của tập $\{g;h\}$ là $\varnothing$, $\{g\}$, $\{h\}$, $\{g;h\}$.
		\end{itemchoice}
	}
\end{ex}
\begin{ex}%[0D1N3-2]
	Cho các tập hợp $A=\{0;1;2;3;4;5;6\}$,  $B=\{-3;-1;1;2;3\}$ và  $C=\{x\in \mathbb{N}|x\text{ là ước của }6\}$.
	\choiceTF
	{$B\setminus C=\{-3;-1;1\}$}
	{$C\setminus B=\{2;3\}$}
	{\True $C_A B=\{0;4;5;6\}$}
	{\True $B\setminus A=\{-3;-1\}$}
	\loigiai{Ta có $A=\{0;1;2;3;4;5;6\}$,  $B=\{-3;-1;1;2;3\}$, $C=\{1;2;3;6\}$.
		\begin{itemchoice}
			\itemch $B\setminus C=\{-3;-1\}$.
			\itemch $C\setminus B=\{6\}$.
			\itemch $C_A B=\{0;4;5;6\}$.
			\itemch $B\setminus A=\{-3;-1\}$.
		\end{itemchoice}
	}
\end{ex}

\begin{ex}%[0D1N3-4]
	Cho hai tập hợp $A=(-1;+\infty)$,  $B=(-\infty;-1]$.
	\choiceTF
	{\True $A\setminus B=(-1;+\infty)$}
	{\True $B\setminus A=(-\infty;-1]$}
	{ $C_\mathbb{R}A=(-\infty;-1)$}
	{\True $C_\mathbb{R}B=(-1;+\infty)$}
	\loigiai{
		\begin{itemchoice}
			\itemch $A\setminus B=(-1;+\infty)$.
			\itemch   $B\setminus A=(-\infty;-1]$.
			\itemch $C_\mathbb{R}A=(-\infty;-1]$.
			\itemch $C_\mathbb{R}B=(-1;+\infty)$.
		\end{itemchoice}
	}
\end{ex}

\Closesolutionfile{ans}

% \indapan{4}{ans/ans-TF-C1-OT-De1}
\TNSA
\setcounter{ex}{0}
\Opensolutionfile{ans}[ans/ans-SA-C1-OT-De1]

\begin{ex}%[0D1H2-2]
	Cho tập hợp $A=\{0;1;2\}$. Số tập con của tập hợp $A$ bằng $m$. Làm tròn $\sqrt{m}$ đến hàng phần trăm.
	\shortans{2,83}
	\loigiai{
		Ta có tập con của tập hợp $\{0;1;2\}$ là $\{1;2\}$, $\{0\}$, $\{1\}$, $\{0;1\}$, $\{0;2\}$, $\{2\}$, $\varnothing$. \\
		Vậy có $8$ tập con của $A$. Suy ra $\sqrt{m} \approx 2,83$
	}
\end{ex}
\begin{ex}%[0D1H3-3]
	Biết rằng  $[-2;7]\cap(3;10)=(a;b]$. Tính $a^b$.
	\shortans{2187}
	\loigiai{
		Ta có $[-2;7]\cap(3;10)=(3;7]$. Vậy $a^b=3^7=2187$.
	}
\end{ex}

\begin{ex}%[0D1V3-5]
	Lớp 10A có $45$ học sinh, trong đó có $15$ bạn biết bơi lội, $20$ bạn biết chơi bóng rổ, $10$ bạn vừa biết bơi lội vừa biết chơi bóng rổ. Hỏi có bao nhiêu học sinh của lớp 10A biết ít nhất một môn thể thao là bơi lội hoặc chơi bóng rổ?
	\shortans{25}
	\loigiai
	{
		Gọi $A$ là tập hợp học sinh biết bơi lội, $B$ là tập hợp học sinh biết chơi bóng rổ.\\
		Số học sinh biết ít nhất một môn thể thao là bơi lội hoặc chơi bóng rổ là
		\[n(A\cup B)=n(A)+n(B)-n(A\cap B)=15+20-10=25\text{ (học sinh)}. \]
	}
\end{ex}

\begin{ex}
	Bạn Meo thống kê số ngày có mưa, có sương mù ở bản mình trong tháng $3$ vào một thời điểm nhất định và được kết quả như sau: $14$ ngày có mưa, $15$ ngày có sương mù, trong đó $10$ ngày có cả mưa và sương mù. Số ngày không có mưa và không có sương mù trong tháng $3$ là $\overline{ab}$. Tính $b^{10a}$.
	\shortans{1024}
	\loigiai{
	Gọi $A$, $B$ lần lượt là tập hợp các ngày có mưa, có sương mù. Khi đó, ${A \cap B}$ là tập hợp các ngày có cả mưa và sương mù, ${A \cup B}$ là tập hợp các ngày hoặc có mưa hoặc có sương mù.\\
	Ta có: $n(A)=14$, $n(B)=15$, $n(A \cap B)=10$.\\
	Số ngày hoặc có mưa hoặc có sương mù là:
	$${n(A \cup B)=n(A)+n(B)-n(A \cap B)=14+15-10=19}.$$
	Tháng 3 có $31$ ngày nên số ngày không có mưa và không có sương mù trong tháng 3 đó là: $31-19=12$ (ngày).\\
	Vậy $b^{10a}=2^{10}=1024$.
		}
\end{ex}

\begin{ex}%[0D1V3-3]
	Cho hai tập hợp $A=(1;5)$, $B=(m;m+1)$. Biết rằng $m\in (a;b)$ khi và chỉ khi $A\cap B \ne \varnothing $. Tính $b^{a+b}$.
	\shortans{3125}
	\loigiai{
		Điều kiện tồn tại tập hợp $B$ là $m<m+1$, luôn đúng $\forall m \in \mathbb{R}$.\\
		Xét ngược yêu cầu bài toán, ta có
		$$A\cap B=\varnothing \Leftrightarrow \hoac{& m+1\le 1 \\& m\ge 5} \Leftrightarrow \hoac{& m\le 0\\& m\ge 5 .}$$
		Vậy $A\cap B$ là một khoảng khi $0<m<5$. Nên $b^{a+b}=5^5=3125$.
	}
\end{ex}

\begin{ex}%[GK1, SoBacNinh, 2023-2024]%[0D1C3-3]
	Cho hai tập hợp khác rỗng $A=\left[-2m+3;6\right]$ và $B=\left[2;3m+1\right)$, với $m\in\mathbb{R}$. Tất cả các giá trị của $m$ để $A\cap B=\varnothing$ là tập $\left(\dfrac{a}{b};\dfrac{c}{d}\right] $. Số $\overline{abcd}$ bằng
	\shortans{1325}
	\loigiai{
		Ta có $A\cap B\Leftrightarrow\heva{&-2m+3< 6\\&2<3m+1\\&3m+1\leq -2m+3}\Leftrightarrow\dfrac{1}{3}<m\leq \dfrac{2}{5}$.
	}
\end{ex}
\Closesolutionfile{ans}

% \indapan{6}{ans/ans-SA-C1-OT-De1}