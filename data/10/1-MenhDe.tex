\setlistsEX{column-sep=-25pt,after-skip=-10pt,after-item-skip=0ex}
\section{Mệnh đề}
\subsection{Tóm tắt lý thuyết}
\begin{tomtat}
\subsubsection{Mệnh đề}
\begin{boxdn}{}
	\textit{Mệnh đề toán học} (gọi tắt là \textit{mệnh đề}) là một khẳng định về một sự kiện toán học \textbf{hoặc đúng hoặc sai}, \textbf{không thể vừa đúng vừa sai}.	
	\begin{itemize}
		\item Mệnh đề thường được kí hiệu bằng các chữ cái in hoa. Ví dụ: Q: \lq\lq  6 chia hết cho 3\rq\rq.
	\end{itemize}
\end{boxdn}

\begin{note}
	\begin{itemize}
		\item Các câu hỏi, câu cảm thán, câu mệnh lệnh không phải là mệnh đề.
		\item Một câu chưa xác định được đúng hay sai nhưng chắc chắn nó chỉ đúng hoặc sai (không thể vừa đúng vừa sai) cũng là một mệnh đề. Ví dụ: \lq\lq  $2^{2023^2+2023+1}+1$ là số nguyên tố\rq\rq\ là một mệnh đề.
		\item Trong thực tế, có những mệnh đề mà tính đúng sai của nó luôn gắn với một thời gian và địa điểm cụ thể: đúng ở thời gian hoặc địa điểm này nhưng sai ở thời gian hoặc địa điểm khác. Nhưng ở bất kì thời gian, địa điểm nào cũng luôn có giá trị chân lí hoặc đúng hoặc sai. Ví dụ: Số 1 là số tự nhiên nhỏ nhất. (Trong một số chương trình, tập số tự nhiên không bao gồm số 0. Tìm hiểu thêm ở topic: \lq\lq  Natural Number\rq\rq\ trên Wikipedia) 
	\end{itemize}
\end{note}
\subsubsection{Mệnh đề chứa biến}
\begin{boxdn}{}
	Những khẳng định mà tính đúng, sai của chúng phụ thuộc vào giá trị của biến gọi là \textit{mệnh đề chứa biến}.
\end{boxdn}
Ví dụ: Cho $P(x): x>x^2$ với $x$ là số thực. Ta chưa khẳng định được tính đúng sai của câu này, do đó nó chưa phải là mệnh đề.\\
Tuy nhiên, khi thay $x$ bởi những giá trị cụ thể thì ta được một mệnh đề, chẳng hạn, $P(2)$ là mệnh đề sai, $P\left(\dfrac{1}{2}\right)$ là mệnh đề đúng.

\subsubsection{Mệnh đề phủ định}

\begin{boxdn}{}
	Cho mệnh đề $P$. Mệnh đề \lq\lq  Không phải $P$\rq\rq\ được gọi là mệnh đề phủ định của $P$ và kí hiệu là $\overline{P}$.
	\begin{itemize}
		\item Mệnh đề $P$ và mệnh đề phủ định $\overline{P}$ là hai khẳng định trái ngược nhau. Nếu $P$ đúng thì $\overline{P}$ sai, nếu $P$ sai thì $\overline{P}$ đúng.
		\item Mệnh đề phủ định của $P$ có thể diễn đạt theo nhiều cách khác nhau. Chẳng hạn, xét mệnh đề $P$: \lq\lq  $2$ là số chẵn\rq\rq. Khi đó, mệnh đề phủ định của $P$ có thể phát biểu là $\overline{P}$: \lq\lq  $2$ không phải là số chẵn\rq\rq\ hoặc \lq\lq  $2$ là số lẻ\rq\rq.
	\end{itemize} 
\end{boxdn}

\subsubsection{Mệnh đề kéo theo và mệnh đề đảo}

\begin{boxdn}{}
	Cho hai mệnh đề $P$ và $Q$. Mệnh đề \lq\lq  Nếu $P$ thì $Q$\rq\rq\ được gọi là mệnh đề kéo theo.
	\begin{itemize}
		\item Kí hiệu là $P\Rightarrow Q.$
		\item Mệnh đề kéo theo chỉ sai khi $P$ đúng $Q$ sai.
		\item $P\Rightarrow Q$ còn được phát biểu là \lq\lq  $P$ kéo theo $Q$\rq\rq, \lq\lq  $P$ suy ra $Q$\rq\rq\ hay \lq\lq  Vì $P$ nên $Q$\rq\rq.
	\end{itemize}
\end{boxdn}

\begin{note}
	Trong toán học, định lí là một mệnh đề đúng, thường có dạng $P\Rightarrow Q$.
	Khi đó ta nói 
	\begin{itemize}
		\item $P$ là giả thiết, $Q$ là kết luận của định lí.
		\item $P$ là $\underline{\textit{điều kiện đủ}}$ để có $Q$, còn $Q$ là $\underline{\textit{điều kiện cần}}$ để có $P$.
	\end{itemize}
\end{note}

% \begin{note}
% 	Trong logic toán học, khi xét giá trị chân lí của mệnh đề $P\Rightarrow Q$ người ta không quan tâm đến mối quan hệ về nội dung của hai mệnh đề $P$, $Q$. Không phân biệt trường hợp $P$ có phải là nguyên nhân để có $Q$ hay không mà chỉ quan tâm đến tính đúng, sai của chúng.
	
% 	Ví dụ: \lq\lq  Nếu mặt trời quay quanh trái đất thì Việt Nam nằm ở châu Âu\rq\rq\ là một mệnh đề đúng. Vì ở đây hai mệnh đề $P$: \lq\lq  Mặt trời quay xung quanh trái đất\rq\rq\ và $Q$: \lq\lq  Việt Nam nằm ở châu Âu\rq\rq\ đều là mệnh đề sai.
% 	(Tìm hiểu thêm ở topic \lq\lq  Mệnh đề toán học\rq\rq trên Wikipedia)
% \end{note}

\begin{boxdn}{}
	Cho mệnh đề kéo theo $P\Rightarrow Q$. Mệnh đề $Q\Rightarrow P$ được gọi là mệnh đề đảo của mệnh đề $P\Rightarrow Q$.
\end{boxdn}

\begin{note}
	Mệnh đề đảo của một mệnh đề đúng không nhất thiết là một mệnh đề đúng.
\end{note}

\subsubsection{Mệnh đề tương đương}

\begin{boxdn}{}
	Cho hai mệnh đề $P$ và $Q$. Mệnh đề có dạng \lq\lq  $P$ nếu và chỉ nếu $Q$\rq\rq\ được gọi là mệnh đề tương đương.
	\begin{itemize}
		\item Kí hiệu là $P \Leftrightarrow Q$.
		\item Mệnh đề $P \Leftrightarrow Q$ đúng khi cả hai mệnh đề $P\Rightarrow Q$ và $Q \Rightarrow P$ cùng đúng hoặc cùng sai. \\
		(Hay $P \Leftrightarrow Q$ đúng khi cả hai mệnh đề $P$ và $Q$ cùng đúng hoặc cùng sai).
		\item $P\Leftrightarrow Q$ còn được phát biểu là \lq\lq  $P$ khi và chỉ khi $Q$\rq\rq, \lq\lq  $P$ tương đương với $Q$\rq\rq, hay \lq\lq  $P$ là điều kiện cần và đủ để có $Q$\rq\rq.
	\end{itemize}
\end{boxdn}

% \begin{note}
% 	Trong logic học, hai mệnh đề $P$, $Q$ tương đương với nhau hoàn toàn không có nghĩa là nội dung của chúng như nhau, mà nó chỉ nói lên rằng chúng có cùng giá trị chân lí (cùng đúng hoặc cùng sai).\\
% 	Ví dụ: \lq\lq  Hình vuông có một góc tù khi và chỉ khi 100 là số nguyên tố\rq\rq\ là một mệnh đề đúng.
% \end{note}

\subsubsection{Mệnh đề có chứa kí hiệu $\forall$ và $\exists$}

\begin{itemize}
	\item Kí hiệu $\forall$ (với mọi): \lq\lq $ \forall x \in X, P(x)$\rq\rq\ hoặc \lq\lq $ \forall x \in X : P(x)$\rq\rq.
	\item Kí hiệu $\exists$ (tồn tại): \lq\lq $ \exists x \in X, P(x)$\rq\rq\ hoặc \lq\lq $ \exists x \in X : P(x)$\rq\rq.
\end{itemize}

\begin{note}\hfil
	\begin{itemize}
		\item Phủ định của mệnh đề \lq\lq $ \forall x \in X, P(x)$\rq\rq\ là mệnh đề \lq\lq $ \exists x\in X, \overline{P(x)}$\rq\rq.
		\item Phủ định của mệnh đề \lq\lq $ \exists x\in X, P(x)$\rq\rq\ là mệnh đề  \lq\lq $ \forall x\in X, \overline{P(x)}$\rq\rq.
	\end{itemize}
\end{note}

\end{tomtat}


\subsection{Các dạng toán}

\begin{dang}{Xác định mệnh đề và xét tính đúng - sai của mệnh đề}
\end{dang}

\subsubsection{Ví dụ minh hoạ}

\begin{vd}%[Thành Đức Trung]%[0D1Y1-1]
	Phát biểu nào sau đây là một mệnh đề toán học?
	\begin{enumerate}
		\item Hà Nội là Thủ đô của Việt Nam.
		\item Số $\pi$ là một số hữu tỉ.
		\item $x=1$ có phải là nghiệm của phương trình $x^2-1=0$ không?
		\item Phương trình $3x^2-5x+2=0$ có nghiệm nguyên.
		\item $5<7-3$.
		\item Đây là cách xử lí khôn ngoan!
	\end{enumerate}
	\loigiai
	{
		\begin{enumerate}
			\item Phát biểu \lq\lq  Hà Nội là Thủ đô của Việt Nam\rq\rq\ là mệnh đề nhưng không phải là mệnh đề toán học.
			\item Phát biểu \lq\lq  Số $\pi$ là một số hữu tỉ\rq\rq\ là một mệnh đề toán học.
			\item Phát biểu \lq\lq  $x=1$ có phải là nghiệm của phương trình $x^2-1=0$ không?\rq\rq\ là một câu hỏi nên không phải là một mệnh đề toán học.
			\item Phát biểu \lq\lq  Phương trình $3x^2-5x+2=0$ có nghiệm nguyên\rq\rq\ là một mệnh đề toán học.
			\item Phát biểu \lq\lq  $5<7-3$\rq\rq\ là một mệnh đề toán học.
			\item Phát biểu \lq\lq  Đây là cách xử lí khôn ngoan!\rq\rq\ là một câu cảm thán nên không phải là một mệnh đề toán học.
		\end{enumerate}
	}
\end{vd}

\begin{vd}%[Thành Đức Trung]%[0D1Y1-2]
	Trong các mệnh đề toán học sau đây, mệnh đề nào là một khẳng định đúng? Mệnh đề nào là một khẳng định sai?
	\begin{enumerate}
		\item $P\colon$\lq\lq  Tổng hai góc đối của một tứ giác nội tiếp bằng $180^{\circ}$\rq\rq.
		\item $Q\colon$\lq\lq  $7$ là số chính phương\rq\rq.
		\item $R\colon$\lq\lq  $1$ là số nguyên tố\rq\rq.
	\end{enumerate}
	\loigiai
	{
		Mệnh đề $P$ là mệnh đề đúng. \\
		Mệnh đề $Q$ và $R$ là mệnh đề sai. \\
	}
\end{vd}

% \begin{vd}%[Thành Đức Trung]%[0D1Y1-2]
% 	Thay dấu \lq\lq ?\rq\rq\ bằng dấu \lq\lq  x\rq\rq\ vào ô thích hợp trong bảng sau
% 	\begin{center}
% 		\begin{tabular}{|>{\centering\arraybackslash}m{5.5cm}|>{\centering\arraybackslash}m{2cm}|>{\centering\arraybackslash}m{2cm}|>{\centering\arraybackslash}m{2cm}|}
% 			\hline
% 			Câu & Không phải MĐ & MĐ đúng & MĐ sai \\
% 			\hline
% 		\end{tabular}
% 		\begin{tabular}{|>{\raggedright\arraybackslash}m{5.5cm}|>{\centering\arraybackslash}m{2cm}|>{\centering\arraybackslash}m{2cm}|>{\centering\arraybackslash}m{2cm}|}
% 			$13$ là số nguyên tố. & ? & ? & ? \\
% 			\hline
% 			Tổng độ dài hai cạnh bất kì của một tam giác nhỏ hơn độ dài cạnh còn lại. & ? & ? & ? \\
% 			\hline
% 			Bạn đã làm bài tập chưa? & ? & ? & ? \\
% 			\hline
% 			Thời tiết hôm nay thật đẹp! & ? & ? & ? \\
% 			\hline
% 			$9>2$. & ? & ? & ? \\
% 			\hline
% 			$27$ chia hết cho $5$. & ? & ? & ? \\
% 			\hline
% 			$2+3=6$. & ? & ? & ? \\
% 			\hline
% 			$36$ là số chính phương. & ? & ? & ? \\
% 			\hline
% 			Chó có khôn hơn lợn không? & ? & ? & ? \\
% 			\hline
% 		\end{tabular}
% 	\end{center}
% 	\loigiai
% 	{
% 		\begin{center}
% 			\begin{tabular}{|>{\centering\arraybackslash}m{5.5cm}|>{\centering\arraybackslash}m{2cm}|>{\centering\arraybackslash}m{2cm}|>{\centering\arraybackslash}m{2cm}|}
% 				\hline
% 				Câu & Không phải mệnh đề & Mệnh đề đúng & Mệnh đề sai \\
% 				\hline
% 			\end{tabular}
% 			\begin{tabular}{|>{\raggedright\arraybackslash}m{5.5cm}|>{\centering\arraybackslash}m{2cm}|>{\centering\arraybackslash}m{2cm}|>{\centering\arraybackslash}m{2cm}|}
% 				$13$ là số nguyên tố. &  & x &  \\
% 				\hline
% 				Tổng độ dài hai cạnh bất kì của một tam giác nhỏ hơn độ dài cạnh còn lại. &  &  & x \\
% 				\hline
% 				Bạn đã làm bài tập chưa? & x &  &  \\
% 				\hline
% 				Thời tiết hôm nay thật đẹp! & x &  &  \\
% 				\hline
% 				$9>2$. &  & x &  \\
% 				\hline
% 				$27$ chia hết cho $5$. &  &  & x \\
% 				\hline
% 				$2+3=6$. &  &  & x \\
% 				\hline
% 				$36$ là số chính phương. &  & x &  \\
% 				\hline
% 				Chó có khôn hơn lợn không? & x &  &  \\
% 				\hline
% 			\end{tabular}
% 		\end{center}
% 	}
% \end{vd}

\subsubsection{Bài tập tự luận}
\begin{bt}%[Thành Đức Trung]%[0D1Y1-1]
	Trong các phát biểu sau, phát biểu nào là mệnh đề toán học?
	\begin{enumerate}
		\item Tích hai số thực trái dấu là một số thực âm.
		\item Mọi số tự nhiên đều là số dương.
		\item Có sự sống ngoài Trái Đất.
		\item Ngày $1$ tháng $5$ là ngày Quốc tế Lao động.
	\end{enumerate}
	\loigiai
	{
		\begin{itemize}
			\item Phát biểu \lq\lq  Tích hai số thực trái dấu là một số thực âm\rq\rq\ là mệnh đề toán học.
			\item Phát biểu \lq\lq  Mọi số tự nhiên đều là số dương\rq\rq\ là mệnh đề toán học.
			\item Phát biểu \lq\lq  Có sự sống ngoài Trái Đất\rq\rq\ là mệnh đề nhưng không là mệnh đề toán học.
			\item Phát biểu \lq\lq  Ngày $1$ tháng $5$ là ngày Quốc tế Lao động\rq\rq\ là mệnh đề nhưng không là mệnh đề toán học.
		\end{itemize}
	}
\end{bt}
\begin{bt}%[Thành Đức Trung]%[0D1Y1-2]
	Xét tính đúng sai của mỗi mệnh đề sau
	\begin{listEX}[2]
		\item $\pi<\dfrac{10}{3}$.
		\item Phương trình $3x+7=0$ có nghiệm.
		\item Tồn tại số cộng với chính nó bằng $0$.
		\item $2022$ là hợp số.
	\end{listEX}
	\loigiai
	{
		\begin{enumerate}
			\item Mệnh đề \lq\lq  $\pi<\dfrac{10}{3}$\rq\rq\ là mệnh đề đúng.
			\item Mệnh đề \lq\lq  Phương trình $3x+7=0$ có nghiệm\rq\rq\ là mệnh đề đúng vì $3x+7=0 \Leftrightarrow x=-\dfrac{7}{3}$.
			\item Mệnh đề \lq\lq  Tồn tại số cộng với chính nó bằng $0$\rq\rq\ là mệnh đề đúng vì $0+0=0$.
			\item Mệnh đề \lq\lq  $2022$ là hợp số\rq\rq\ là mệnh đề đúng vì $2022$ có ít nhất $3$ ước là $1$; $2$ và $2022$.
		\end{enumerate}
	}
\end{bt}

\begin{bt}%[Thành Đức Trung]%[0D1Y1-2]
	Xét tính đúng sai của mỗi mệnh đề sau
	\begin{listEX}[2]
		\item $1993$ chia hết cho $3$.
		\item $\sqrt{12}$ là một số hữu tỉ.
		\item $9$ là một số chính phương.
		\item $|-1997|\leqslant0$.
	\end{listEX}
	\loigiai
	{
		\begin{enumerate}
			\item Mệnh đề \lq\lq  $1993$ chia hết cho $3$\rq\rq\ là mệnh đề sai vì $1993$ chia $3$ dư $1$.
			\item Mệnh đề \lq\lq  $\sqrt{12}$ là một số hữu tỉ\rq\rq\ là mệnh đề sai vì $\sqrt{12}$ là một số vô tỉ.
			\item Mệnh đề \lq\lq  $9$ là một số chính phương\rq\rq\ là mệnh đề đúng vì $\sqrt{9}=3$.
			\item Mệnh đề \lq\lq  $|-1997|\leqslant0$\rq\rq\ là mệnh đề sai vì $|-1997|=1997>0$.
		\end{enumerate}
	}
\end{bt}

\begin{bt}%[Thành Đức Trung]%[0D1Y1-2]
	Xét tính đúng sai của mỗi mệnh đề sau
	\begin{listEX}[3]
		\item $\sqrt{3}+\sqrt{2}=\dfrac{1}{\sqrt{3}-\sqrt{2}}$.
		\item $\left(\sqrt{2}-\sqrt{18}\right)^2\geqslant8$.
		\item $\left(\sqrt{3}+\sqrt{12}\right)^2$ là một số hữu tỉ.
		\item! $x=2$ là một nghiệm của phương trình $\dfrac{x^2-4}{x-2}=0$.
	\end{listEX}
	\loigiai
	{
		\begin{enumerate}
			\item Mệnh đề \lq\lq  $\sqrt{3}+\sqrt{2}=\dfrac{1}{\sqrt{3}-\sqrt{2}}$\rq\rq\ là mệnh đề đúng.
			\item Mệnh đề \lq\lq  $\left(\sqrt{2}-\sqrt{18}\right)^2\geqslant8$\rq\rq\ là mệnh đề đúng vì $\left(\sqrt{2}-\sqrt{18}\right)^2=8$.
			\item Mệnh đề \lq\lq  $\left(\sqrt{3}+\sqrt{12}\right)^2$ là một số hữu tỉ\rq\rq\ là mệnh đề đúng vì $\left(\sqrt{3}+\sqrt{12}\right)^2=27$.
			\item Mệnh đề \lq\lq  $x=2$ là một nghiệm của phương trình $\dfrac{x^2-4}{x-2}=0$\rq\rq\ là mệnh đề sai vì $x=2$ vi phạm điều kiện xác định của phương trình.
		\end{enumerate}
	}
\end{bt}

\begin{bt}%[Thành Đức Trung]%[0D1Y1-2]
	Thay dấu \lq\lq ?\rq\rq\ bằng dấu \lq\lq  x\rq\rq\ vào ô thích hợp trong bảng sau
	\begin{center}
		\begin{tabular}{|>{\centering\arraybackslash}m{5.5cm}|>{\centering\arraybackslash}m{2cm}|>{\centering\arraybackslash}m{2cm}|>{\centering\arraybackslash}m{2cm}|}
			\hline
			Câu & Không phải mệnh đề & Mệnh đề đúng & Mệnh đề sai \\
			\hline
		\end{tabular}
		\begin{tabular}{|>{\raggedright\arraybackslash}m{5.5cm}|>{\centering\arraybackslash}m{2cm}|>{\centering\arraybackslash}m{2cm}|>{\centering\arraybackslash}m{2cm}|}
			Hãy đi nhanh lên! & ? & ? & ? \\
			\hline
			$5+7+4=15$. & ? & ? & ? \\
			\hline
			Phương trình $x^2-3x+2=0$ có nghiệm. & ? & ? & ? \\
			\hline
			$2^{10}-1$ chia hết cho $11$. & ? & ? & ? \\
			\hline
			Có vô số số nguyên tố. & ? & ? & ? \\
			\hline
			Bây giờ là mấy giờ? & ? & ? & ? \\
			\hline
			$\sqrt{5}$ là số vô tỉ. & ? & ? & ? \\
			\hline
		\end{tabular}
	\end{center}
	\loigiai
	{
		\begin{center}
			\begin{tabular}{|>{\centering\arraybackslash}m{5.5cm}|>{\centering\arraybackslash}m{2cm}|>{\centering\arraybackslash}m{2cm}|>{\centering\arraybackslash}m{2cm}|}
				\hline
				Câu & Không phải mệnh đề & Mệnh đề đúng & Mệnh đề sai \\
				\hline
			\end{tabular}
			\begin{tabular}{|>{\raggedright\arraybackslash}m{5.5cm}|>{\centering\arraybackslash}m{2cm}|>{\centering\arraybackslash}m{2cm}|>{\centering\arraybackslash}m{2cm}|}
				Hãy đi nhanh lên! & x &  &  \\
				\hline
				$5+7+4=15$. &  &  & x \\
				\hline
				Phương trình $x^2-3x+2=0$ có nghiệm. &  & x &  \\
				\hline
				$2^{10}-1$ chia hết cho $11$. &  & x &  \\
				\hline
				Có vô số số nguyên tố. &  & x &  \\
				\hline
				Bây giờ là mấy giờ? & x &  &  \\
				\hline
				$\sqrt{5}$ là số vô tỉ. &  & x &  \\
				\hline
			\end{tabular}
		\end{center}
	}
\end{bt}

\begin{dang}{Mệnh đề phủ định, mệnh đề đảo, mệnh đề kéo theo, tương đương}
\end{dang}

\subsubsection{Ví dụ minh hoạ}

\begin{vd}
	Phát biểu mệnh đề phủ định của các mệnh đề sau và cho biết tính đúng sai của mệnh đề phủ định đó.
	\begin{enumerate}
		\item $P\colon$\lq\lq  $\sqrt{5}$ là số hữu tỉ\rq\rq.
		\item $Q\colon $\lq\lq  Tổng ba góc trong một tam giác bằng $180^\circ$\rq\rq.
		\item $R\colon$\lq\lq  $25$ là một số chính phương\rq\rq.
		\item $T\colon $\lq\lq  Hình vuông không phải là hình bình hành\rq\rq.
	\end{enumerate}
	\loigiai{
		\begin{enumerate}
			\item Mệnh đề phủ định của mệnh đề $P$ là $\overline{P}\colon$\lq\lq  $\sqrt{5}$ không phải là số hữu tỉ\rq\rq.\\
			Đây là một mệnh đề đúng vì $\sqrt{5}$ không thể biểu diễn dưới dạng $\dfrac{a}{b}$ với $a$, $b\in \mathbb{Z}$.
			\item Mệnh đề phủ định của mệnh đề $Q$ là $\overline{Q}\colon $\lq\lq  Tổng ba góc trong tam giác không bằng $180^\circ$.\\
			Đây là một mệnh đề sai.
			\item Mệnh đề phủ định của mệnh đề $R$ là $\overrightarrow{R}\colon $\lq\lq  $25$ không phải là một số chính phương\rq\rq.\\
			Đây là một mệnh đề sai.
			\item Mệnh đề phủ định của mệnh đề $T$ là $\overline{T}\colon$\lq\lq  Hình vuông là hình bình hành\rq\rq.\\
			Đây là một mệnh đề đúng.
		\end{enumerate}
	}
\end{vd}

\begin{vd}
	Cho tam giác $ABC$. Xét hai mệnh đề $P\colon $\lq\lq  tam giác $ABC$ vuông\rq\rq\text{} và $Q\colon $\lq\lq  $AB^2+AC^2=BC^2$\rq\rq. Phát biểu và cho biết mệnh đề sau đúng hay sai.
	\begin{enumEX}{2}
		\item $P\Rightarrow Q$.
		\item $Q\Rightarrow P$.
	\end{enumEX}
	\loigiai{
		\begin{enumerate}
			\item Mệnh đề $P\Rightarrow Q$ là \lq\lq  Nếu tam giác $ABC$ vuông thì $AB^2+AC^2=BC^2$.\\
			Mệnh đề $P\Rightarrow Q$ sai vì chưa chắc tam giác $ABC$ đã vuông tại $A$.
			\item Mệnh đề $Q\Rightarrow P$ là \lq\lq  Nếu tam giác $ABC$ có $AB^2+AC^2=BC^2$ thì tam giác vuông\rq\rq.\\
			Mệnh đề $Q\Rightarrow P$ đúng (theo định lí Py-ta-go).
		\end{enumerate}
	}
\end{vd}

\begin{vd}
	Cho $\triangle ABC$ có hai đường trung tuyến $BM$, $CN$. Lập mệnh đề $P\Rightarrow Q$ và mệnh đề đảo của nó, rồi xét tính đúng sai của chúng khi
	\begin{enumerate}
		\item $P\colon$\lq\lq  Góc $A$ tù\rq\rq\text{} và $Q\colon $\lq\lq  Cạnh $BC$ lớn nhất\rq\rq.
		\item $P\colon$\lq\lq  $BM=CN$\rq\rq\text{} và $Q\colon $\lq\lq  tam giác $ABC$ cân\rq\rq.
	\end{enumerate}
	\loigiai{
		\begin{enumerate}
			\item $P\colon$\lq\lq  Góc $A$ tù\rq\rq\text{} và $Q\colon $\lq\lq  Cạnh $BC$ lớn nhất\rq\rq.
			\begin{itemize}
				\item Mệnh đề $P\Rightarrow Q$ là \lq\lq  Nếu góc $A$ tù thì cạnh $BC$ lớn nhất\rq\rq. Đây là mệnh đề đúng.
				\item Mệnh đề $Q\Rightarrow P$ là \lq\lq  Nếu cạnh $BC$ lớn nhất thì $A$ là góc tù\rq\rq. Đây là mệnh đề sai ($A$ vẫn có thể là góc nhọn hoặc góc vuông).
			\end{itemize}
			\item $P\colon$\lq\lq  $BM=CN$\rq\rq\text{} và $Q\colon $\lq\lq  tam giác $ABC$ cân\rq\rq.
			\begin{itemize}
				\item Mệnh đề $P\Rightarrow
				Q$ là \lq\lq  Nếu $BM=CN$ thì tam giác $ABC$ cân\rq\rq. Đây là một mệnh đề đúng.
				\item Mệnh đề $Q\Rightarrow P$ là \lq\lq  Nếu tam giác $ABC$ cân thì $BM=CN$\rq\rq. Đây là một mệnh đề sai vì chưa chắc tam giác $ABC$ đã cân tại $A$.
			\end{itemize}
		\end{enumerate}
	}
\end{vd}

\begin{vd}
	Cho định lí \lq\lq  Nếu $MA\perp MB$ thì $M$ thuộc đường tròn đường kính $AB$\rq\rq. Hãy xác định giả thiết của định lí, kết luận của định lí và dùng thuật ngữ \lq\lq  điều kiện cần\rq\rq, \lq\lq  điều kiện đủ\rq\rq\text{} để phát biểu lại định lí.
	\loigiai{
		Giả thiết của định lí là $MA\perp MB$.\\
		Kết luật của định lí là $M$ thuộc đường  tròn đường kính $AB$.
		\begin{itemize}
			\item Điều kiện cần để $MA\perp MB$ là $M$  thuộc đường tròn đường kính $AB$.
			\item Điều kiện đủ để $M$ thuộc đường tròn đường kính $AB$ là $MA\perp MB$.
		\end{itemize}
	}
\end{vd}

\begin{vd}
	Phát biểu mệnh đề $P\Leftrightarrow Q$ và cho biết tính đúng sai của nó.
	\begin{enumerate}
		\item $P\colon $\lq\lq  Tứ giác $ABCD$ là hình vuông\rq\rq \text{ và }$Q\colon$ \lq\lq  Tứ giác $ABCD$ là hình thoi có $AC=BD$\rq\rq.
		\item $P\colon $\lq\lq  Điểm $M$ nằm trên phân giác của góc $xOy$\rq\rq \text{} và $Q\colon $\lq\lq  Điểm $M$ cách đều hai cạnh $Ox$, $Oy$\rq\rq.
		\item $P\colon$\lq\lq  Tam giác $ABC$ đều\rq\rq \text{} và $Q\colon$\lq\lq  Tam giác $ABC$ có ba đường cao bằng nhau\rq\rq.
	\end{enumerate}
	\loigiai{
		\begin{enumerate}
			\item Mệnh đề tương đương $P\Leftrightarrow Q$ là \lq\lq  Tứ giác $ABCD$ là hình vuông khi và chỉ khi tứ giác $ABCD$ là hình thoi có $AC=BD$\rq\rq.\\
			Mệnh đề $P\Leftrightarrow Q$ đúng vì mệnh đề $P\Rightarrow Q$ và mệnh đề $Q\Rightarrow P$ là hai mệnh đề đúng.
			\item Mệnh đề tương đương $P\Leftrightarrow Q$ là \lq\lq  Điểm $M$ nằm trên phân giác của góc $xOy$ khi và chỉ khi điểm $M$ cách đều hai cạnh $Ox$, $Oy$\rq\rq.\\
			Mệnh đề $P\Leftrightarrow Q$ đúng vì mệnh đề $P\Rightarrow Q$ và $Q\Rightarrow P$ là hai mệnh đề đúng.
			\item Mệnh đề tương đương $P\Leftrightarrow Q$ là \lq\lq  Tam giác $ABC$ đều khi và chỉ khi ba đường cao bằng nhau\rq\rq.\\
			Mệnh đề $P\Leftrightarrow Q$ đúng vì hai mệnh đề $P\Rightarrow Q$ và $Q\Rightarrow P$ là hai mệnh đề đúng. 
		\end{enumerate}
	}
\end{vd}

\subsubsection{Bài tập tự luận}

\begin{bt}
	Phát biểu mệnh đề phủ định của các mệnh đề sau
	\begin{enumerate}
		\item $A\colon$\lq\lq  $2022$ chia hết cho $7$\rq\rq.
		\item $B\colon$\lq\lq  Tích của ba số tự nhiên liên tiếp chia hết cho $6$\rq\rq.
		\item $C\colon $\lq\lq  Phương trình $x^2+x+1=0$ vô nghiệm\rq\rq.
	\end{enumerate}
	\loigiai{
		\begin{enumerate}
			\item Mệnh đề phủ định của mệnh đề $A$ là $\overline{A}\colon$\lq\lq  $2022$ không chia hết cho $7$\rq\rq.
			\item Mệnh đề phủ định của mệnh đề $B$ là $\overline{B}\colon$\lq\lq  Tích của ba số tự nhiên liên tiếp không chia hết cho $6$\rq\rq.
			\item Mệnh đề phủ định của mệnh đề $C$ là $\overline{C}\colon$\lq\lq  Phương trình $x^2-x+1=0$ có nghiệm\rq\rq.
		\end{enumerate}
	}
\end{bt}

\begin{bt}
	Hãy lập mệnh đề phủ định của các mệnh đề sau đây và cho biết các mệnh đề phủ định đó đúng hay sai?
	\begin{enumerate}
		\item $A\colon$\lq\lq  $735$ là số nguyên tố\rq\rq.
		\item $B\colon$\lq\lq  Phương trình $x^2+9x-2011=0$ vô nghiệm\rq\rq.
		\item $C\colon$\lq\lq  Đường tròn có một tâm đối xứng\rq\rq.
		\item $D\colon$\lq\lq  Hai đường thẳng song song không có điểm chung\rq\rq.
	\end{enumerate}
	\loigiai{
		\begin{enumerate}
			\item Phủ định của mệnh đề $A$ là $\overline{A}\colon$\lq\lq  Số $735$ không phải là số nguyên tố\rq\rq. Đây là mệnh đề đúng vì $735\,\vdots\,5$.
			\item Phủ định của mệnh đề $B$ là $\overline{B}\colon$\lq\lq  Phương trình $x^2+9x-2022=0$ có nghiệm\rq\rq. Đây là mệnh đề đúng vì $a=1$ và $c=-2022$ trái dấu.
			\item Phủ định của mệnh đề $C$ là $\overline{C}\colon$\lq\lq  Không phải đường tròn có một tâm đối xứng\rq\rq. Đây là một mệnh đề sai.
			\item Phủ định của mệnh đề $D$ là $\overline{D}\colon$\lq\lq  Hai đường thẳng song song có điểm chung\rq\rq. Đây là mệnh đề sai.
		\end{enumerate}
	}
\end{bt}

\begin{bt}
	Phát biểu mệnh đề đảo của mệnh đề sau và xét tính đúng sai của mệnh đề đảo.
	\begin{enumerate}
		\item Nếu một số chia hết cho $6$ thì số đó chia hết cho $3$.
		\item Nếu một số là số tự nhiên lẻ thì nó là số nguyên tố.
		\item Nếu $\dfrac{AB}{MN}=\dfrac{AC}{MP}$ thì $\triangle ABC\backsim \triangle MNP$.
	\end{enumerate}
	\loigiai{
		\begin{enumerate}
			\item Nếu một số chia hết cho $3$ thì số đó chia hết cho $6$. Đây là mệnh đề sai.
			\item Nếu một số là số nguyên tố thì nó là số lẻ. Đây là mệnh đề sai vì $2$ là số nguyên tố chẵn.
			\item Nếu $\triangle ABC\backsim \triangle MNP$ thì $\dfrac{AB}{MN}=\dfrac{AC}{MP}$. Đây là mệnh đề đúng.
		\end{enumerate}
	}
\end{bt}

\begin{bt}
	Phát biểu mệnh đề đảo của mệnh đề sau và cho biết tính đúng sai của mệnh đề đảo.
	\begin{enumerate}
		\item Nếu hai tam giác bằng nhau thì chúng có diện tích bằng nhau.
		\item Nếu tứ giác $ABCD$ là hình bình hành thì nó có hai cạnh đối song song và bằng nhau.
	\end{enumerate}
	\loigiai{
		\begin{enumerate}
			\item Nếu hai tam giác có diện tích bằng nhau thì nó bằng nhau.\\
			Đây là một mệnh đề sai.
			\item Nếu tứ giác $ABCD$ có hai cạnh đối song song và bằng nhau thì nó là hình bình hành.\\
			Đây là mệnh đề đúng. 
		\end{enumerate}
	}
\end{bt}

\begin{bt}
	Hãy xác định giả thiết, kết luận đồng thời dùng thuật ngữ \lq\lq  điều kiện đủ\rq\rq,\text{} để phát biểu các định lí sau
	\begin{enumerate}
		\item Nếu $a$ và $b$ là hai số hữu tỉ thì tổng $a+b$ cũng là số hữu tỉ.
		\item Nếu một số tự nhiên $n$ có tổng các chữ số chia hết cho $9$ thì nó chia hết cho $9$.
	\end{enumerate}
	\loigiai{
		\begin{enumerate}
			\item Giả thiết của định lí là \lq\lq  $a$ và $b$ là hai số hữu tỉ\rq\rq.\\
			Kết luận của định lí là \lq\lq  tổng $a+b$ là số hữu tỉ\rq\rq.\\
			Phát biểu định lí dưới dạng điều kiện đủ \lq\lq  Điều kiện đủ để tổng $a+b$ là số hữu tỉ là cả hai số $a$ và $b$ đều là số hữu tỉ\rq\rq.
			\item Giả thiết của định lí là \lq\lq  Một số tự nhiên $n$ có tổng các chữ số chia hết cho $9$\rq\rq.\\
			Kết luận của định lí là \lq\lq  $n$ chia hết cho $9$\rq\rq.\\
			Phát biểu định lí dưới dạng điều kiện đủ \lq\lq  Điều kiện đủ để $n$ chia hết cho $9$ là tổng các chữ số của $n$ chia hết cho $9$\rq\rq.
		\end{enumerate}
	}
\end{bt}

\begin{bt}
	Cho định lí \lq\lq  Cho số tự nhiên $n$, nếu $n^5$ chia hết cho $5$ thì $n$ chia hết cho $5$\rq\rq. Định lí này được viết dưới dạng $P\Rightarrow Q$.
	\begin{enumerate}
		\item Hãy xác định các mệnh đề $P$ và $Q$.
		\item Phát biểu định lí trên bằng cách dùng thuật ngữ \lq\lq  điều kiện cần\rq\rq.
		\item Phát biểu định lí trên bằng cách dùng thuật ngữ \lq\lq  điều kiện đủ\rq\rq.
		Hãy phát biểu định lí đảo (nếu có) của định lí trên rồi dùng các thuật ngữ \lq\lq  điều kiện cần và điều kiện đủ\rq\rq\text{} phát biểu gộp cả hai định lí thuận và đảo.
	\end{enumerate}
	\loigiai{
		\begin{enumerate}
			\item $P\colon $\lq\lq  $n$ là số tự nhiên và $n^5$ chia hết cho $5$\rq\rq, $Q\colon $\lq\lq  $n$ chia hết cho $5$\rq\rq.
			\item Với $n$ là số tự nhiên, $n$ chia hết cho $5$ là điều kiện cần để $n^5$ chia hết cho $5$.
			\item Với $n$ là số tự nhiên, $n^5$ chia hết cho $5$ là điều kiện đủ để $n$ chia hết cho $5$.
			\item 
			\begin{itemize}
				\item Định lí đảo \lq\lq  Cho số tự nhiên $n$, nếu $n$ chia hết cho $5$ thì $n^5$ chia hết cho $5$\rq\rq.\\
				\item Phát biểu gộp cả hai định lí \lq\lq  Điều kiện cần và đủ để $n$ chia hết cho $5$ là $n^5$ chia hết cho $5$\rq\rq.
			\end{itemize} 
		\end{enumerate}
	}
\end{bt}

\begin{bt}
	Cho tam giác ABC với trung tuyến $AM$. Xét hai mệnh đề\\
	$P\colon $\lq\lq  Tam giác $ABC$ vuông tại $A$\rq\rq.
	$Q\colon $\lq\lq  Trung tuyến $AM$ bằng một nửa cạnh $BC$\rq\rq
	\begin{enumerate}
		\item Hãy phát biểu mệnh đề $P\Rightarrow Q$. Mệnh đề này đúng hay sai?
		\item Hãy phát biểu mệnh đề $Q\Rightarrow P$. Mệnh đề này đúng hay sai?
		\item Phát biểu mệnh đề $P\Leftrightarrow Q$ và cho biết mệnh đề đó đúng hay sai?
	\end{enumerate}
	\loigiai{
		\begin{enumerate}
			\item Mệnh đề $P\Rightarrow Q$ là \lq\lq  Nếu tam giác $ABC$ vuông tại $A$ thì trung tuyến $AM$ bằng một nửa cạnh $BC$\rq\rq.\\
			Đây là mệnh đề đúng.
			\item Mệnh đề $Q\Rightarrow P$ là \lq\lq  Nếu trung tuyến $AM$ bằng một nửa cạnh $BC$ thì tam giác $ABC$ vuông tại $A$\rq\rq.\\
			Đây là mệnh đề đúng.
			\item Mệnh đề $P\Leftrightarrow Q$ là \lq\lq  Tam giác $ABC$ vuông tại $A$ khi và chỉ khi trung tuyến $AM$ bằng một nửa cạnh $BC$\rq\rq.\\
			Mệnh đề tương đương $P\Leftrightarrow Q$ đúng vì $P\Rightarrow Q$ và $Q\Rightarrow P$ là hai mệnh đề đúng.
		\end{enumerate}
	}
\end{bt}

\begin{bt}
	Phát biểu mệnh đề $P\Rightarrow Q$ và phát biểu mệnh đề đảo, xét tính đúng sai của nó.
	\begin{enumerate}
		\item $P\colon $\lq\lq  Tứ giác $ABCD$ là hình chữ nhật\rq\rq \text{} và $Q\colon $\lq\lq  Tứ giác $ABCD$ có $AC$ và $BD$ cắt nhau tại trung điểm của mỗi đường\rq\rq.
		\item $P\colon$\lq\lq  Hình thang $ABCD$ nội tiếp một đường tròn \rq\rq \text{} và $Q\colon$\lq\lq  Hình thang $ABCD$ cân\rq\rq.
	\end{enumerate}
	\loigiai{
		\begin{enumerate}
			\item Mệnh đề đảo của mệnh đề $P\Rightarrow Q$ là $Q\Rightarrow P\colon $\lq\lq  Nếu tứ giác $ABCD$ có $AC$ và $BD$ cắt nhau tại trung điểm của mỗi đường thì nó là hình chữ nhật\rq\rq.\\
			Đây là một mệnh đề sai vì tứ giác có hai đường chéo cắt nhau tại trung điểm của mỗi đường thì nó chỉ là hình bình hành, chưa đủ điều kiện để là hình chữ nhật.
			\item Mệnh đề đảo của mệnh đề $P\Rightarrow Q$ là $Q\Rightarrow P\colon $\lq\lq  Nếu $ABCD$ là hình thang cân thì $ABCD$ nội tiếp một đường tròn\rq\rq.\\
			Đây là một mệnh đề đúng vì hình thang cân có tổng hai góc đối bằng $180^\circ$.
		\end{enumerate}
	}
\end{bt}

\begin{bt}
	Hãy phát biểu mệnh đề $P\Leftrightarrow Q$ và cho biết mệnh đề đó đúng hay sai nếu biết
	\begin{enumerate}
		\item $P\colon $\lq\lq  $a$ và $b$ cùng chia hết cho $c$\rq\rq\text{} và $Q\colon $\lq\lq  $a+b$ chia hết cho $c$\rq\rq.
		\item $P\colon $\lq\lq  $a$ chia hết cho $3$\rq\rq\text{} và $Q\colon $\lq\lq  $a$ chia hết cho $9$\rq\rq.
		\item $P\colon $\lq\lq  $ABCD$ là hình chữ nhật\rq\rq\text{} và $Q\colon $\lq\lq  Tứ giác $ABCD$ có ba góc vuông\rq\rq.
	\end{enumerate}
	\loigiai{
		\begin{enumerate}
			\item Mệnh đề $P\Leftrightarrow Q\colon $\lq\lq  $a$ và $b$ cùng chia hết cho $c$ nếu và chỉ nếu $a+b$ chia hết cho $c$\rq\rq.\\
			Đây là mệnh đề sai vì mệnh đề $P\Rightarrow Q$ đúng nhưng mệnh đề $Q\Rightarrow P$ là sai.
			\item Mệnh đề $P\Leftrightarrow Q\colon $\lq\lq  $a$ chia hết cho $3$ nếu và chỉ nếu $a$ chia hết cho $9$\rq\rq.\\
			Đây là mệnh đề sai vì mệnh đề $P\Rightarrow Q$ là mệnh đề đúng còn mệnh đề $Q\Rightarrow P$ là mệnh đề sai.
			\item Mệnh đề $P\Leftrightarrow Q\colon $\lq\lq  $ABCD$ là hình chữ nhật khi và chỉ khi nó có ba góc vuông\rq\rq.\\
			Đây là một mệnh đề đúng vì mệnh đề $P\Rightarrow Q$ và $Q\Rightarrow P$ là hai mệnh đề đúng.
		\end{enumerate}
	}
\end{bt}

\begin{dang}{Mệnh đề chứa biến- mệnh đề chứa kí hiệu $\forall$ và $\exists$}
	% Kí hiệu $\forall$ đọc là \lq \lq với mọi\rq \rq.\\
	% Kí hiệu $\exists$ đọc là \lq \lq có một\rq \rq \,(tồn tại một) hay \lq \lq có ít nhất một\rq \rq\,(tồn tại ít nhất một).\\
	% Mối quan hệ giữa $\exists$ và $\forall$.\\
	% Cho mệnh đề \lq \lq $P(x),\, x \in X$\rq \rq.\\
	% Phủ định của mệnh đề \lq \lq $ \forall x \in X,\;P(x)$\rq \rq \;là mệnh đề \lq \lq $\exists x \in X,\;\overline{P(x)}$\rq \rq.\\
	% Phủ định của mệnh đề \lq \lq $ \exists x \in X,\;P(x)$\rq \rq \;là mệnh đề \lq \lq $ \forall x \in X,\;\overline{P(x)}$\rq \rq.
\end{dang}

\subsubsection{Ví dụ minh hoạ}

\begin{vd}%[Nguyễn Cường- BG Toán 10]%[0D1Y1-1]
	Xét câu \lq \lq $n$ là số chẵn\rq \rq. (với $n$ là số nguyên) \\
	Ta chưa khẳng định được tính đúng sai của câu này. Tuy nhiên, với mỗi giá trị của $n$ thuộc tập số nguyên, câu này cho ta một mệnh đề.
	Chẳng hạn,
	\begin{itemize}
		\item Với $n=1$ ta được mệnh đề \lq \lq $1$ là số chẵn\rq \rq\, (đây là mệnh đề sai).
		\item Với $n=2$ ta được mệnh đề \lq \lq $2$ là số chẵn\rq \rq\, (đây là mệnh đề đúng).
	\end{itemize}
	Ta nói rằng câu \lq \lq $n$ là số chẵn\rq \rq\, là một mệnh đề chứa biến.	
\end{vd}
%%==========Ví dụ 2
\begin{vd}%[Nguyễn Cường- BG Toán 10]%[0D1Y1-2]
	Xét câu \lq\lq $x>1$\rq\rq. Hãy tìm hai giá trị thực của $x$, ta nhận được một mệnh đề đúng và một mệnh đề sai.
	\loigiai{\begin{enumerate}
			\item Cho $x=2$ ta được mệnh đề đúng.
			\item Cho $x=0$ ta được mệnh đề sai.
		\end{enumerate}
	}
\end{vd}
%%==========Ví dụ 3
\begin{vd}%[Nguyễn Cường- BG Toán 10]%[0D1Y1-1]
	Trong các câu sau, câu nào là mệnh đề chứa biến?
	\begin{enumerate}
		\item $18$ chia hết cho $9$;
		\item $3n$ chia hết cho $9$.
	\end{enumerate}
	\loigiai{
		\begin{enumerate}
			\item Câu \lq\lq $18$ chia hết cho $9$\rq\rq\,là mệnh đề nhưng không phải là mệnh đề chứa biến.
			\item Câu \lq\lq $3n$ chia hết cho $9$\rq\rq\,là mệnh đề chứa biến, kí hiệu là $P(n)\colon$\lq\lq $3n$ chia hết cho $9$\rq\rq.
		\end{enumerate}
	}
\end{vd}
%%==========Ví dụ 4
\begin{vd}%[Nguyễn Cường- BG Toán 10]%[0D1Y1-5]
	Cho mệnh đề $P\colon$\lq\lq  $\forall x \in \mathbb{N}: x-2>0$\rq\rq. Tìm mệnh đề phủ định của mệnh đề $P$. Xét tính đúng sai của mệnh đề $\overline{P}$.
	\loigiai{
		Ta có $\overline{P}\colon$\lq\lq  $\exists x \in \mathbb{N}: x-2\leq 0$\rq\rq.\\
		Đây là mệnh đề đúng, vì với $x=0$ thì $x-2=-2<0$.
	}
\end{vd}
%%==========Ví dụ 5
\begin{vd}%[Nguyễn Cường- BG Toán 10]%[0D1Y1-5]
	Viết mệnh đề phủ định của mệnh đề sau và xác định tính đúng sai của nó.\break
	$P\colon$ \lq\lq $\exists x\in\mathbb{R}, x^2+1=0$\rq\rq.
	\loigiai{
		Mệnh đề $P$ có thể phát biểu là \lq\lq  Tồn tại một số thực mà bình phương của nó cộng với $1$ bằng $0$\rq\rq.\\
		Phủ định của mệnh đề $P$ là \lq\lq  Không tồn tại một số thực mà bình phương của nó cộng với $1$ bằng $0$\rq\rq.\\
		Tức là \lq\lq  Mọi số thực mà bình phương của nó cộng với $1$ khác $0$\rq\rq.\\
		Ta có thể viết mệnh đề phủ định của $P$ là $\overline{P}\colon$\lq\lq $\forall x\in\mathbb{R}, x^2+1\ne 0$\rq\rq. Mệnh đề phủ định này đúng.
	}
\end{vd}

\subsubsection{Bài tập tự luận}

%%==========Bài 1
% \begin{bt}%[Nguyễn Cường- BG Toán 10]%[0D1Y1-2]
% 	Cho câu \lq\lq $x>5$\rq\rq. Hãy tìm hai giá trị thực của $x$ để từ câu đã cho, ta nhận được một mệnh đề đúng và một mệnh đề sai.
% 	\loigiai{
% 		\begin{enumerate}
% 			\item Cho $x=7$ ta được mệnh đề đúng.
% 			\item Cho $x=5$ ta được mệnh đề sai.
% 		\end{enumerate}
% 	}
% \end{bt}
%%==========Bài 2
\begin{bt}%[Nguyễn Cường- BG Toán 10]%[0D1Y1-5]
	Sử dụng kí hiệu \lq\lq $\forall$\rq\rq \,để viết mỗi mệnh đề sau và xét xem mệnh đề đó là đúng hay sai, giải thích vì sao.
	\begin{enumerate}
		\item $P\colon$\lq\lq  Với mọi số thực $x, x^2+1>0$\rq\rq.
		\item $Q\colon$\lq\lq  Với mọi số tự nhiên $n, n^2+n$ chia hết cho $6$\rq\rq.
	\end{enumerate}
	\loigiai{
		\begin{enumerate}
			\item $P\colon$\lq\lq  Với mọi số thực $x, x^2+1>0$\rq\rq.\\
			Mệnh đề được viết là $P \colon \lq\lq \forall x \in \mathbb{R}, x^2+1>0$\rq\rq.\\
			Xét một số thực $x$ tùy ý, ta phải chứng tỏ rằng $x^2+1>0$.\\
			Thật vậy, ta có $x^2+1 \geq 1>0$.\\
			Vậy mệnh đề $P$ là mệnh đề đúng.
			\item $Q\colon$\lq\lq  Với mọi số tự nhiên $n, n^2+n$ chia hết cho $6$\rq\rq.\\
			Mệnh đề được viết là $Q\colon\lq\lq  \forall n \in \mathbb{N},\left(n^2+n\right) \,\vdots\, 6$\rq\rq.\\
			Để chứng minh mệnh đề $Q$ là sai, ta cần chỉ ra một giá trị cụ thể của $n$ để nhận được mệnh đề sai.\\
			Thật vậy, chọn $n=1$, ta thấy $n^2+n=2$ không chia hết cho $6$.\\
			Vậy mệnh đề $Q$ là mệnh đề sai.
		\end{enumerate}
	}
\end{bt}
%%==========Bài 3
\begin{bt}%[Nguyễn Cường- BG Toán 10]%[0D1Y1-5]
	Sử dụng kí hiệu \lq\lq $\exists$\rq\rq\, để viết mỗi mệnh đề sau và xét xem mệnh đề đó là đúng hay sai, giải thích vì sao.
	\begin{enumerate}
		\item $M\colon$\lq\lq  Tồn tại số thực $x$ sao cho $x^3=-8$\rq\rq.
		\item $N\colon$\lq\lq  Tồn tại số nguyên $x$ sao cho $2x+1=0$\rq\rq.
	\end{enumerate}
	\loigiai{
		\begin{enumerate}
			\item $M\colon$\lq\lq  Tồn tại số thực $x$ sao cho $x^3=-8$\rq\rq.\\
			Mệnh đề được viết là $M\colon\lq\lq \exists x \in \mathbb{R}, x^3=-8$\rq\rq.
			Để chứng tỏ mệnh đề $M$ là đúng, ta cần chỉ ra một giá trị cụ thể của $x$ để nhận được mệnh đề đúng.\\
			Thật vậy, chọn $x=-2$, ta thấy $(-2)^3=-8$.\\
			Vậy mệnh đề $M$ là mệnh đề đúng.\\
			Mệnh đề $N\colon\lq\lq \exists x \in \mathbb{Z}, 2x+1=0$\rq\rq.
			\item $N\colon$\lq\lq  Tồn tại số nguyên $x$ sao cho $2x+1=0$\rq\rq.\\
			Để chứng minh mệnh đề $N$ là sai, ta phải chứng tỏ rằng với số nguyên $x$ tùy ý thì $2x+1 \neq 0$.\\
			Thật vậy, xét một số nguyên $x$ tùy ý, ta có $2x+1 \neq 0$.\\
			Vì thế mệnh đề $N$ là mệnh đề sai.
		\end{enumerate}
	}
\end{bt}
%%==========Bài 4
\begin{bt}%[Nguyễn Cường- BG Toán 10]%[0D1B1-5]
	Bạn An nói \lq\lq  Mọi số thực đều có bình phương là một số không âm\rq\rq.
	Bạn Bình phủ định lại câu nói của bạn An \lq\lq  Có một số thực mà bình phương của nó là một số âm\rq\rq.
	\begin{enumerate}
		\item Sử dụng kí hiệu \lq\lq $\forall$\rq\rq\,để viết mệnh đề của bạn An.
		\item Sử dụng kí hiệu \lq\lq $\exists$\rq\rq\,để viết mệnh đề của bạn Bình.
	\end{enumerate}
	\loigiai{
		\begin{enumerate}
			\item \lq\lq $\forall x\in\mathbb{R}, x^2\ge 0$\rq\rq.
			\item \lq\lq $\exists x\in\mathbb{R}, x^2< 0$\rq\rq.
		\end{enumerate}	
	}
\end{bt}
%%==========Bài 5
\begin{bt}%[Nguyễn Cường- BG Toán 10]%[0D1B1-5]
	Lập mệnh đề phủ định của mỗi mệnh đề sau
	\begin{enumerate}
		\item $\forall x \in \mathbb{R},|x| \geq x$.
		\item $\exists x \in \mathbb{R}, x^2+1=0$.
	\end{enumerate}
	\loigiai{
		\begin{enumerate}
			\item Phủ định của mệnh đề \lq\lq $\forall x \in \mathbb{R},|x| \geq x$\rq\rq\,là mệnh đề \lq\lq $\exists x \in \mathbb{R},|x|<x$\rq\rq.
			\item Phủ định của mệnh đề \lq\lq $\exists x \in \mathbb{R}, x^2+1=0$\rq\rq\,là mệnh đề \lq\lq $\forall x \in \mathbb{R}, x^2+1 \neq 0$\rq\rq.
		\end{enumerate}
	}
\end{bt}
%%==========Bài 6
% \begin{bt}%[Nguyễn Cường- BG Toán 10]%[0D1B1-5]
% 	Phát biểu mệnh đề phủ định của mỗi mệnh đề sau
% 	\begin{enumerate}
% 		\item Tồn tại số nguyên chia hết cho $3$.
% 		\item Mọi số thập phân đều viết được dưới dạng phân số.
% 	\end{enumerate}
% 	\loigiai{
% 		\begin{enumerate}
% 			\item Mọi số nguyên đều không chia hết cho $3$.
% 			\item Tồn tại số thập phân không viết được dưới dạng phân số.
% 		\end{enumerate}	
% 	}
% \end{bt}
%%==========Bài 7
% \begin{bt}%[Nguyễn Cường- BG Toán 10]%[0D1B1-5]
% 	Phát biểu các mệnh đề sau
% 	\begin{enumerate}
% 		\item $\forall x \in \mathbb{R}, x^2 \geq 0$.
% 		\item $\exists x \in \mathbb{R}, \dfrac{1}{x}>x$.
% 	\end{enumerate}
% 	\loigiai{
% 		\begin{enumerate}
% 			\item Mọi số thực đều không âm.
% 			\item Tồn tại số thực sao cho nghịch đảo của số đó lớn hơn chính số đó.
% 		\end{enumerate}	
% 	}
% \end{bt}
%%==========Bài 8
\begin{bt}%[Nguyễn Cường- BG Toán 10]%[0D1B1-5]
	Lập mệnh đề phủ định của mỗi mệnh đề sau và xét tính đúng sai của mỗi mệnh đề phủ định đó
	\begin{enumerate}
		\item $\forall x \in \mathbb{R}, x^2 \neq 2x-2$.
		\item $\forall x \in \mathbb{R}, x^2 \leq 2x-1$.
		\item $\exists x \in \mathbb{R}, x+\dfrac{1}{x} \geq 2$.
		\item $\exists x \in \mathbb{R}, x^2-x+1<0$.
	\end{enumerate}
	\loigiai{
		\begin{enumerate}
			\item $\exists x \in \mathbb{R}, x^2=2x-2$.\\
			Mệnh đề này sai vì phương trình $x^2-2x+2=0$ vô nghiệm trên tập số thực.
			\item $\exists x \in \mathbb{R}, x^2 > 2x-1$.\\
			Mệnh đề này đúng vì với $x=2$ thì $2^2>2\cdot 2-1$.
			\item $\forall x \in \mathbb{R}, x+\dfrac{1}{x}<2$.\\
			Mệnh đề này sai vì với $x=1$ thì $1+\dfrac{1}{1}=2$.
			\item $\forall x \in \mathbb{R}, x^2-x+1\ge 0$.\\
			Mệnh đề này đúng vì $x^2-x+1=\left(x-\dfrac{1}{2}\right)^2+\dfrac{3}{4}> 0$ với mọi $x\in\mathbb{R}$.
		\end{enumerate}	
	}
\end{bt}
%%==========Bài 9
\begin{bt}%[Nguyễn Cường- BG Toán 10]%[0D1B1-5]
	Trong tiết học môn Toán, Nam phát biểu: \lq\lq  Mọi số thực đều có bình phương khác $1$\rq\rq. Mai phát biểu: \lq\lq  Có một số thực mà bình phương của nó bằng $1$\rq\rq.
	\begin{enumerate}
		\item Hãy cho biết bạn nào phát biểu đúng.
		\item Dùng kí hiệu $\forall$, $\exists$ để viết lại các phát biểu của Nam và Mai dưới dạng mệnh đề.
	\end{enumerate}
	\loigiai{
		\begin{enumerate}
			\item Bạn Mai phát biểu là đúng vì có số $1$ bình phương lên bằng $1$.
			\item Nam phát biểu \lq\lq $\forall x\in \mathbb{R}, x^2\ne 1$\rq\rq.\\
			Mai phát biểu \lq\lq $\exists x\in \mathbb{R}, x^2=1$\rq\rq.\\
		\end{enumerate}	
	}
\end{bt}
%%==========Bài 10
\begin{bt}%[Nguyễn Cường- BG Toán 10]%[0D1B1-5]
	Phát biểu bằng lời mệnh đề sau và cho biết mệnh đề đó đúng hay sai.
	$$
	\forall x \in \mathbb{R}, x^2+1 \leq 0
	$$
	\loigiai{
		Mọi số thực bình phương lên và cộng cho một luôn không dương.\\
		Đây là một mệnh đề sai vì $0^2+1=1>0$.	
	}
\end{bt}

\subsection{BÀI TẬP TRẮC NGHIỆM ÔN TẬP CUỐI BÀI}

% \Opensolutionfile{ansbook}[ans/ansbook-0D1-1-TN]
\Opensolutionfile{ans}[ans/ans-0D1-1-TN]

\begin{ex}%[Lương Như Quỳnh]%[0D1Y1-1]
	Phát biểu nào dưới đây là mệnh đề?
	\choice
	{\True $2+3=9$}
	{Phong cảnh đẹp quá!}
	{$5-x=7$}
	{Bây giờ là mấy giờ?}
	\loigiai{
		\lq\lq $2+3=9$\rq\rq\ là mệnh đề sai.\\
		\lq\lq  Phong cảnh đẹp quá!\rq\rq\ không là mệnh đề vì đây là câu cảm thán.\\
		\lq\lq $5-x=7$\rq\rq\ là mệnh đề chứa biến.\\
		\lq\lq  Bây giờ là mấy giờ?\rq\rq\ không là mệnh đề vì đây là câu nghi vấn.
	}
\end{ex}
\begin{ex}%[Lương Như Quỳnh]%[0D1B1-1]
	Các câu sau đây, câu nào {\bf không} là mệnh đề?
	\choice
	{Phương trình $ x^2-x+1=0$ vô nghiệm}
	{\True $x+y>1$}
	{$12$ không là số nguyên tố}
	{Hai phương trình $ x^2-4x+3=0$ và $ 2x^2-\sqrt{x+3}=0$ có nghiệm chung}
	\loigiai{
		\lq\lq  Phương trình $ x^2-x+1=0$ vô nghiệm\rq\rq\ là mệnh đề sai.\\
		\lq\lq  $12$ không là số nguyên tố\rq\rq\ là mệnh đề đúng.\\
		\lq\lq  Hai phương trình $ x^2-4x+3=0$ và $ 2x^2-\sqrt{x+3}=0$ có nghiệm chung\rq\rq\ là mệnh đề đúng.\\
		\lq\lq  $x+y>1$\rq\rq\ là mệnh đề chứa biến.}
\end{ex}
\begin{ex}%[Lương Như Quỳnh]%[0D1B1-4]
	Trong các câu sau, câu nào là mệnh đề \textbf{đúng}?
	\choice
	{Nếu $a\ge b$ thì $a^2\ge b^2$}
	{\True Nếu $a$ chia hết cho $9$ thì $a$ chia hết cho $3$}
	{Nếu bạn tự tin thì bạn thành công}
	{Nếu một tam giác có một góc bằng $60^\circ $ thì tam giác đó đều}
	\loigiai
	{
		\begin{itemize}
			\item Mệnh đề \lq\lq  Nếu $a\ge b$ thì $a^2\ge b^2$\rq\rq\ là một mệnh đề sai vì $b\le a < 0$ thì $a^2\le b^2$ .
			\item Mệnh đề \lq\lq  Nếu $a$ chia hết cho $9$ thì $a$ chia hết cho $3$\rq\rq\ là mệnh đề đúng.\\
			Vì $a$ $\vdots$ $9\Rightarrow \heva{&a=9n, n\in \mathbb{Z}\\&9\hspace{0.15cm}\vdots\hspace{0.15cm} 3}\Rightarrow a$ $\vdots$  $3$.
			\item \lq\lq  Nếu bạn tự tin thì bạn thành công\rq\rq\ chưa là mệnh đề vì chưa khẳng định được tính đúng, sai.
			\item Mệnh đề \lq\lq  Nếu một tam giác có một góc bằng $60^\circ $ thì tam giác đó đều\rq\rq\ là mệnh đề sai vì chưa đủ điều kiện để khẳng định một tam giác là đều.
		\end{itemize}
	}
\end{ex}
\begin{ex}%[Lương Như Quỳnh]%[0D1Y1-2]
	Mệnh đề nào sau đây là \textbf{sai}?
	\choice
	{Phương trình $ x^2+bx+c=0$ có nghiệm $\Leftrightarrow b^2-4c\geqslant 0$}
	{\True $\heva{
			&a>b\\
			&b>c} \Leftrightarrow a>c$}
	{$\Delta ABC$ vuông tại $A\Leftrightarrow \widehat{B}+\widehat{C}=90^\circ$}
	{ $ n^2$ chẵn $\Leftrightarrow n$ chẵn}
	\loigiai{
		Xét mệnh đề $\heva{
			&a>b\\
			&b>c\\
		} \Leftrightarrow a>c$, ta có
		\begin{itemize}
			\item $\heva{
				&a>b\\
				&b>c\\
			} \Rightarrow a>c$ đúng.
			\item $ a>c\Rightarrow \heva{
				&a>b\\
				&b>c.\\
			} $ sai. Chẳng hạn $ a=5$; $c=3$; $b=1$ thì $5>3\Rightarrow \heva{
				&5>1\\
				&1>3} $ vô lý.
		\end{itemize} 
	}
\end{ex}
\begin{ex}%[Lương Như Quỳnh]%[0D1Y1-5]
	Trong các mệnh đề sau, mệnh đề nào \textbf{sai}?
	\choice
	{$\exists x\in \mathbb{R},\,x^2-3x+2=0$}
	{$\forall x\in \mathbb{R},\,x^2+1>0$}
	{\True $\exists x\in \mathbb{R},\,x^2<0$}
	{$\forall x\in \mathbb{R},\,|x+1|\ge 0$}
	\loigiai{
		Mệnh đề \lq\lq $\exists x\in \mathbb{R},x^2<0$\rq\rq\, sai, vì $ x^2\ge 0,\,\forall x\in \mathbb{R}$.
	}
\end{ex}

\begin{ex}%[Lương Như Quỳnh]%[0D1B1-4]
	Trong các mệnh đề sau, mệnh đề nào có mệnh đề đảo \textbf{đúng}?
	\choice
	{Nếu số nguyên $n$ có chữ số tận cùng là $5$ thì số nguyên $n$ chia hết cho $5$}
	{\True Nếu tứ giác $ABCD$ có hai đường chéo cắt nhau tại trung điểm mỗi đường thì tứ giác $ABCD$ là hình bình hành}
	{Nếu tứ giác $ABCD$ là hình chữ nhật thì tứ giác $ABCD$ có hai đường chéo bằng nhau}
	{Nếu tứ giác $ABCD$ là hình thoi thì tứ giác $ABCD$ có hai đường chéo vuông góc với nhau}
	\loigiai
	{
		\begin{itemize}
			\item Mệnh đề đảo của mệnh đề \lq\lq  Nếu số nguyên $n$ có chữ số tận cùng là $5$ thì số nguyên $n$ chia hết cho $5$\rq\rq\, là \lq\lq  Nếu số nguyên $n$ chia hết cho $5$ thì số nguyên $n$ có chữ số tận cùng là $5$ \rq\rq. Mệnh đề này sai vì số nguyên $n$ cũng có thể có chữ số tận cùng là $0$.
			\item Mệnh đề đảo của mệnh đề \lq\lq  Nếu tứ giác $ABCD$ có hai đường chéo cắt nhau tại trung điểm mỗi đường thì tứ giác $ABCD$ là hình bình hành\rq\rq\, là \lq\lq  Nếu tứ giác $ABCD$ là hình bình hành thì tứ giác $ABCD$ có hai đường chéo cắt nhau tại trung điểm mỗi đường\rq\rq. Mệnh đề này đúng.
			\item Mệnh đề đảo của mệnh đề \lq\lq  Nếu tứ giác $ABCD$ là hình chữ nhật thì tứ giác $ABCD$ có hai đường chéo bằng nhau\rq\rq\, là \lq\lq  Nếu tứ giác $ABCD$ có hai đường chéo bằng nhau thì tứ giác $ABCD$ là hình chữ nhất\rq\rq. Mệnh đề này sai vì hình thang cân cũng có hai đường chéo bằng nhau, nhưng không là hình chữ nhật.
			\item Mệnh đề đảo của mệnh đề \lq\lq  Nếu tứ giác $ABCD$ là hình thoi thì tứ giác $ABCD$ có hai đường chéo vuông góc\rq\rq\, là \lq\lq  Nếu tứ giác $ABCD$ có hai đường chéo vuông góc thì tứ giác $ABCD$ là hình thoi\rq\rq. Mệnh đề này sai.
		\end{itemize}
	}
\end{ex}
\begin{ex}%[Lương Như Quỳnh]%[0D1B1-3]
	Trong các mệnh đề sau, mệnh đề nào có mệnh đề đảo là \textbf{sai}?
	\choice
	{Nếu tam giác $ABC$ cân thì tam giác có hai cạnh bằng nhau}
	{Nếu $ a$ chia hết cho $6$ thì $ a$ chia hết cho $2$ và $3$}
	{\True Nếu $ABCD$ là hình bình hành thì $AB$ song song với $CD$}
	{Nếu tứ giác có hai đường chéo vuông góc thì tứ giác đó là hình thoi}
	\loigiai{
		Mệnh đề đảo của mệnh đề \lq\lq  Nếu $ABCD$ là hình bình hành thì $AB$ song song với $CD$\rq\rq\, là \lq\lq  Nếu tứ giác $ABCD$ có $AB$ song song với $CD$ thì $ABCD$ là hình bình hành \rq\rq. Mệnh đề này sai vì tứ giác $ABCD$ có thể là hình thang có hai đáy là $AB$ và $CD$. }
\end{ex}
\begin{ex}%[Lương Như Quỳnh]%[0D1B1-5]
	Cho mệnh đề $P(x)\colon$ \lq\lq $\forall x\in \mathbb{R},\ x^2+x+1>0$\rq\rq. Mệnh đề phủ định của mệnh đề $P(x)$ là
	\choice
	{\lq\lq $\forall x\in \mathbb{R},\ x^2+x+1<0$\rq\rq}
	{\lq\lq $\forall x\in \mathbb{R},\ x^2+x+1\leqslant 0$\rq\rq}
	{\True \lq\lq $\exists x\in \mathbb{R},\ x^2+x+1\leqslant 0$\rq\rq}
	{\lq\lq $x\in \mathbb{R},\ x^2+x+1>0$\rq\rq}
	\loigiai{
		Phủ định của mệnh đề $P(x)$ là $\overline{P(x)}\colon$ \lq\lq $\exists x\in \mathbb{R},\ x^2+x+1\leqslant 0$\rq\rq.}
\end{ex}
\begin{ex}%[Lương Như Quỳnh]%[0D1Y1-3]
	Cho mệnh đề $P\colon$ \lq\lq $\exists x\in \mathbb{R},\, x<\dfrac{1}{x}$\rq\rq. Xác định mệnh đề phủ định của mệnh đề $P$.
	\choice
	{$\overline{P}\colon$ \lq\lq  $\exists x\in \mathbb{R},\, x\ge \dfrac{1}{x}$\rq\rq}
	{$\overline{P}\colon$ \lq\lq  $\forall x\in \mathbb{R},\, x> \dfrac{1}{x}$\rq\rq}
	{\True 	$\overline{P}\colon$ \lq\lq  $\forall x\in \mathbb{R},\, x\ge \dfrac{1}{x}$\rq\rq}
	{$\overline{P}\colon$ \lq\lq  $\exists x\in \mathbb{R},\, x> \dfrac{1}{x}$\rq\rq}
	\loigiai{
		Phủ định của mệnh đề $P\colon$ \lq\lq $\exists x\in \mathbb{R},\, x<\dfrac{1}{x}$\rq\rq\, là mệnh đề $\overline{P}\colon$ \lq\lq  $\forall x\in \mathbb{R},\, x\ge \dfrac{1}{x}$\rq\rq.}
\end{ex}
\begin{ex}%[Lương Như Quỳnh]%[0D1B1-2]
	Cách phát biểu nào sau đây \textbf{không} thể dùng để phát biểu mệnh đề $A \Rightarrow B$?
	\choice
	{Nếu $A$ thì $B$}
	{$A$ kéo theo $B$}
	{$A$ là điều kiện đủ để có $B$}
	{\True $A$ là điều kiện cần để có $B$}
	\loigiai{
		$A$ là điều kiện cần để có $B$ dùng để phát biểu mệnh đề $B \Rightarrow A$.
	}
\end{ex}
\begin{ex}%[Lương Như Quỳnh]%[0D1B1-5]
	Trong các mệnh đề sau đây, mệnh đề nào đúng?
	\choice
	{\True Với mọi số thực $x$, nếu $x <-2$ thì $x^2> 4$}
	{Với mọi số thực $x$, nếu $x^2< 4$ thì $x <-2$}
	{Với mọi số thực $x$, nếu $x <-2$ thì $x^2< 4$}
	{Với mọi số thực $x$, nếu $x^2> 4$ thì $x >-2$}
	\loigiai{
		Mệnh đề \lq\lq  Với mọi số thực $x$, nếu $x^2< 4$ thì $x <-2$\rq\rq\ sai. Chẳng hạn $x=1\Rightarrow{x^2}=1 < 4$ nhưng $1 >-2$.\\
		Mệnh đề \lq\lq  Với mọi số thực $x$, nếu $x <-2$ thì $x^2< 4$\rq\rq\  sai. Chẳng hạn $x=-3 <-2$ nhưng $x^2=9 > 4$.\\
		Mệnh đề \lq\lq  Với mọi số thực $x$, nếu $x^2> 4$ thì $x >-2$\rq\rq\ sai. Chẳng hạn $x=-3\Rightarrow{x^2}=9 > 4$ nhưng $-3 <-2$.}
\end{ex}
\begin{ex}%[Lương Như Quỳnh]%[0D1B1-2]
	Biết $A$ là mệnh đề sai và $B$ là mệnh đề đúng. Mệnh đề nào sau đây đúng?
	\choice
	{$B\Rightarrow A$}
	{$B\Leftrightarrow A$}
	{$\overline{A}\Leftrightarrow \overline{B}$}
	{\True $B\Rightarrow \overline{A}$}
	\loigiai{
		Ta có $\overline{A}$ và $B$ đúng nên $B\Rightarrow \overline{A}$ là mệnh đề đúng.
	}
\end{ex}
\begin{ex}%[Lương Như Quỳnh]%[0D1B1-4]
	Cho $P\Leftrightarrow Q$ là mệnh đề đúng. Khẳng định nào sau đây là \textbf{sai}?
	\choice
	{$\overline{P}\Leftrightarrow Q$ sai}
	{$\overline{P}\Leftrightarrow\overline{Q}$ đúng}
	{$\overline{Q}\Leftrightarrow P$ sai}
	{\True $\overline{P}\Leftrightarrow \overline{Q}$ sai}
	\loigiai
	{
		Ta có $P\Leftrightarrow Q$ đúng nên $P\Rightarrow Q$ đúng và $Q\Rightarrow P$ đúng.\\
		Do đó $\overline P\Rightarrow\overline Q $ đúng và $\overline Q\Rightarrow\overline P $ đúng.\\
		Vậy $\overline P\Leftrightarrow\overline Q $ đúng.
	}
\end{ex}
\begin{ex}%[Lương Như Quỳnh]%[0D1B1-2]
	Cho $A$, $B$, $C$ là ba mệnh đề đúng. Mệnh đề nào sau đây là đúng?
	\choice
	{$A\Rightarrow (B\Rightarrow \overline{C})$}
	{$C\Rightarrow \overline{A}$}
	{$B\Rightarrow (\overline{A\Rightarrow C})$}
	{\True $C\Rightarrow (A\Rightarrow B)$}
	\loigiai{
		Ta có $A$, $B$, $C$ là ba mệnh đề đúng nên 
		\begin{itemize}
			\item $ B\Rightarrow \overline{C} $ sai và $A\Rightarrow (B\Rightarrow \overline{C})$ sai.
			\item $ \overline{A} $ sai và $C\Rightarrow \overline{A}$ sai.
			\item $ \overline{A}\Rightarrow C $ đúng và $B\Rightarrow (\overline{A\Rightarrow C})$ sai.
			\item $ A\Rightarrow B $ đúng và $C\Rightarrow (A\Rightarrow B)$ đúng.
	\end{itemize}}
\end{ex}
\begin{ex}%[Lương Như Quỳnh]%[0D1B1-2]
	Trong các mệnh đề nào sau đây mệnh đề nào \textbf{sai}?
	\choice
	{Hai tam giác bằng nhau khi và chỉ khi chúng đồng dạng và có một góc bằng nhau}
	{Một tứ giác là hình chữ nhật khi và chỉ khi chúng có $3$ góc vuông}
	{\True Một tam giác là vuông khi và chỉ khi nó có một góc bằng tổng hai góc còn lại}
	{Một tam giác là đều khi và chỉ khi chúng có hai đường trung tuyến bằng nhau và có một góc bằng $60^{\circ}$}
	\loigiai{
		Mệnh đề \lq\lq  Một tam giác là vuông khi và chỉ khi nó có một góc bằng tổng hai góc còn lại\rq\rq sai. Chẳng hạn tam giác có $A=60^\circ$, $B=70^\circ$, $C=50^\circ$ nhưng tam giác $ABC$ không là tam giác vuông.}
\end{ex}
\begin{ex}%[Lương Như Quỳnh]%[0D1B1-2]
	Trong các mệnh đề sau, mệnh đề nào là mệnh đề đúng?
	\choice
	{Tổng của hai số tự nhiên là một số chẵn khi và chỉ khi cả hai số đều là số chẵn}
	{Tích của hai số tự nhiên là một số chẵn khi và chỉ khi cả hai số đều là số chẵn}
	{Tổng của hai số tự nhiên là một số lẻ khi và chỉ khi cả hai số đều là số lẻ}
	{\True Tích của hai số tự nhiên là một số lẻ khi và chỉ khi cả hai số đều là số lẻ}
	\loigiai{
		Mệnh đề \lq\lq  Tổng của hai số tự nhiên là một số chẵn khi và chỉ khi cả hai số đều là số chẵn\rq\rq\, sai. Ví dụ: $3+5=8$ là số chẵn nhưng $3$ và $5$ là hai số lẻ.\\
		Mệnh đề \lq\lq  Tích của hai số tự nhiên là một số chẵn khi và chỉ khi cả hai số đều là số chẵn\rq\rq\, sai. Ví dụ: $2\cdot 3=6$ là số chẵn nhưng $3$ là số lẻ.\\
		Mệnh đề \lq\lq  Tổng của hai số tự nhiên là một số lẻ khi và chỉ khi cả hai số đều là số lẻ\rq\rq\, sai. Ví dụ: $1+3=4$ là số chẵn nhưng $1$, $3$ là hai số lẻ.}
\end{ex}
\begin{ex}%[Lương Như Quỳnh]%[0D1Y1-2]
	Cho mệnh đề chứa biến $P(x)\colon$ \lq\lq $x>x^3$\rq\rq. Trong các khẳng định sau, khẳng định nào đúng?
	\choice
	{\True $P(1)$ là mệnh đề sai}
	{$P(1)$ là mệnh đề đúng}
	{$P(1)$ là mệnh đề vừa đúng vừa sai}
	{$P(1)$ không phải là mệnh đề}
	\loigiai{
		Mệnh đề $P(1)\colon\lq\lq  1>1^3$\rq\rq\, sai.}
\end{ex}
\begin{ex}%[Lương Như Quỳnh]%[0D1Y1-2]
	Xét mệnh đề chứa biến $P(x)\colon\lq\lq  x\in\mathbb{R},\ x^2-2x\geqslant 0$\rq\rq. Tìm một giá trị của biến để được mệnh đề đúng.
	\choice
	{$ x=\dfrac{1}{4}$}
	{\True $ x=3$}
	{$ x=1$}
	{$ x=0{,}5$}
	\loigiai{
		\begin{itemize}
			\item Với $x=\dfrac14$ ta có $P\left(\dfrac14\right)\colon\lq\lq \left(\dfrac14\right)^2-2\cdot\dfrac14\geqslant0$\rq\rq\ là mệnh đề sai.
			\item Với $x=3$ ta có $P\left(3\right)\colon\lq\lq 3^2-2\cdot3\geqslant0$\rq\rq\ là mệnh đề đúng.
			\item Với $x=1$ ta có $P\left(1\right)\colon\lq\lq 1^2-2\cdot1\geqslant0$\rq\rq\ là mệnh đề sai.
			\item Với $x=0{,}5$ ta có $P\left(0{,}5\right)\colon\lq\lq 0{,}5^2-2\cdot0{,}5\geqslant0$\rq\rq\ là mệnh đề sai.
		\end{itemize}
	}
\end{ex}

\begin{ex}%[Lương Như Quỳnh]%[0D1B1-5]
	Mệnh đề nào dưới đây {\bf sai}?
	\choice
	{$x\left(1-2x\right)\le\dfrac{1}{8},\, \forall x$}
	{\True $x^2+2+\dfrac{1}{x^2+2}>\dfrac{5}{2},\, \forall x$}
	{$\dfrac{x^2-x+1}{x^2+x+1}\ge\dfrac{1}{3},\, \forall x$}
	{$\dfrac{x}{x^2+1}\le\dfrac{1}{2},\, \forall x$}
	\loigiai{
		Ta có
		\begin{itemize}
			\item $x\left(1-2x\right)\le\dfrac{1}{8}\Leftrightarrow 2\left(x-\dfrac{1}{4}\right)^2\ge 0$ (đúng).
			\item $\dfrac{x^2-x+1}{x^2+x+1}\ge \dfrac{1}{3}\Leftrightarrow \dfrac{3(x^2-x+1)-(x^2+x+1)}{x^2+x+1}\ge 0\Leftrightarrow \dfrac{2(x-1)^2}{x^2+x+1}\ge 0$ (đúng).
			\item $\dfrac{x}{x^2+1}\le \dfrac{1}{2}\Leftrightarrow (x-1)^2\ge 0$ (đúng).
			\item Với $x=0$ dễ thấy $0^2+2+\dfrac{1}{0^2+2}>\dfrac{5}{2}$ sai.
		\end{itemize}
	}
\end{ex}

\begin{ex}%[Lương Như Quỳnh]%[0D1K1-5]
	Mệnh đề nào sau đây {\bf sai}?
	\choice
	{$\forall x \in \mathbb{R},\,3x^2-4x+4>0$}
	{\True $\exists x \in \mathbb{R},\,(x-1)^2+(x+1)^2=0$}
	{$\exists x \in \mathbb{Q},\,x<\dfrac{1}{x}$}
	{$\exists n \in \mathbb{N},\,(1+2+3+ \cdots +n)\ \vdots\ 11$}
	\loigiai{
		\begin{itemize}
			\item Mệnh đề \lq\lq  $\forall x \in \mathbb{R},\,3x^2-4x+4>0$\rq\rq\, đúng vì $3x^2-4x+4=2x^2+(x-2)^2>0,\ \forall x\in\mathbb{R}$.
			\item Mệnh đề \lq\lq  $\exists x \in \mathbb{Q},\,x<\dfrac{1}{x}$\rq\rq\, đúng vì với $x=\dfrac{1}{2}$ thì $x<\dfrac{1}{x}$.
			\item Mệnh đề \lq\lq  $\exists n \in \mathbb{N},\,(1+2+3+ \cdots +n)\ \vdots\ 11$\rq\rq\, đúng vì với $n=10$ thì $1+2+\cdots +10=5\cdot 11\ \vdots\ 11$.
			\item Mệnh đề \lq\lq  $\exists x \in \mathbb{R},\,(x-1)^2+(x+1)^2=0$\rq\rq\, sai vì $(x-1)^2+(x+1)^2>0,\ \forall x\in\mathbb{R}$.
		\end{itemize}
	}
\end{ex}

\Closesolutionfile{ans}
% \Closesolutionfile{ansbook}
% \indapan{10}{ans/ans-0D1-1-TN}