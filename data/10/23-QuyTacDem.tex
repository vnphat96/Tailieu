\def\tenchude{QUY TẮC ĐẾM}
\setcounter{section}{0}
\setcounter{dang}{0}
\section{Quy tắc đếm}
\subsection{Tóm tắt lí thuyết}
\subsubsection{Quy tắc cộng}
Giả sử một công việc nào đó có thể thực hiện theo một trong hai phương án khác nhau
\begin{itemize}
	\item Phương án một có $n_1$ cách thực hiện,
	\item Phương án hai có $n_2$ cách thực hiện.
\end{itemize}
Khi đó, số cách thực hiện công việc sẽ là \fbox{$\mathbf{n_1+n_2}$} cách.
\subsubsection{Quy tắc nhân}
Giả sử một công việc nào đó phải hoàn thành qua hai công đoạn liên tiếp nhau
\begin{itemize}
	\item Công đoạn một có $m_1$ cách thực hiện,
	\item Với mỗi cách thực hiện công đoạn một, có $m_2$ cách thực hiện công đoạn hai.
\end{itemize}
Khi đó, số cách thực hiện công việc là \fbox{$\mathbf{m_1 \cdot m_2}$} cách.

\subsection{Các dạng toán}
\begin{dang}{Bài toán sử dụng quy tắc cộng}
	- Ta áp dụng quy tắc cộng cho một công việc có nhiều phương án khi các phương án đó phải rời nhau, không phụ thuộc vào nhau (độc lập với nhau).	\\
	- Nếu $ A $ và $ B $ là các tập hợp hữu hạn không giao nhau, thì $ n(A\cup B) =n(A)+n(B) $.
\end{dang}
\viduminhhoa
% \begin{vd}%[BG Toán 10, Nguyễn Tài Tuệ]%[1D2Y1-1]
% Trên giá sách có $ 8 $ cuốn truyện ngắn, $ 7 $ cuốn tiểu thuyết và $ 5 $ tập thơ (tất cả đều khác nhau). Vẽ sơ đồ hình cây minh hoạ và cho biết bạn Phong có bao nhiêu cách chọn một cuốn để đọc vào ngày cuối tuần.
% \loigiai{
% \immini{
% Để chọn một cuốn để đọc bạn Phong có thể thực hiện theo một trong ba phương án sau
% \begin{itemize}
% 	\item Chọn một truyện ngắn có $ 8 $ cách.
% 	\item Chọn một tiểu thuyết có $ 7 $ cách.
% 	\item Chọn một tập thơ có $ 5 $ cách.
% \end{itemize}
% Theo quy tắc cộng ta có $ 8+7+5=20 $ cách.}{\begin{tikzpicture}[scale=1,line cap=round,line join=round,font=\footnotesize,>=stealth]
% 	\path
% 	(0,0) coordinate (A)
% 	(2,1) coordinate (B)
% 	(2,0) coordinate (C)
% 	(2,-1) coordinate (D)
% 	;
% 	\draw (A)--(B)node[right]{\text{Chọn truyện ngắn có } $8$ \text{cách.}}
% 	(A)--(C) node[right]{\text{Chọn tiểu thuyết có } $7$ \text{cách.}}
% 	(A)--(D) node[right]{\text{Chọn tập thơ} $5$ \text{cách.}};
% \end{tikzpicture}}
% }
% \end{vd}
\begin{vd}%[BG Toán 10, Nguyễn Tài Tuệ]%[1D2Y1-1]
	Giả sử từ tỉnh $C$ đến tỉnh $D$ có thể đi bằng các phương tiện: ô tô, tàu hỏa hoặc máy bay. Mỗi ngày có $6$ chuyến ô tô, $4$ chuyến tàu hỏa và $2$ chuyến máy bay. Số cách lựa chọn chuyến đi từ tỉnh $C$ đến tỉnh $D$ là
	% \dotfill
	\loigiai{
		Để đi từ $C$ đến $D$ có $3$ phương án lựa chọn:
		\begin{itemize}
			\item Đi bằng ô tô có $6$ cách chọn.
			\item Đi bằng tàu hỏa có $4$ cách chọn.
			\item Đi bằng máy bay có $2$ cách chọn.
		\end{itemize}
		Theo quy tắc cộng, có $6+4+2=12$ cách chọn.
	}
\end{vd}

\begin{vd}%[BG Toán 10, Nguyễn Tài Tuệ]%[1D2Y1-1]
	Giả sử bạn muốn mua một áo sơ mi cỡ $39$ hoặc cỡ $40.$ Áo cỡ $39$ có $5$ màu khác nhau, áo cỡ $40$ có $4$ màu khác nhau. Hỏi có bao nhiêu sự lựa chọn (về màu áo và cỡ áo)?
	\loigiai{
		\begin{itemize}
			\item Nếu chọn cỡ áo $39$ thì sẽ có $5$ cách.
			\item Nếu chọn cỡ áo $40$ thì sẽ có $4$ cách.
		\end{itemize}
		Theo quy tắc cộng, ta có $5+4=9$ cách chọn mua áo.}
\end{vd}

\begin{vd}%[BG Toán 10, Nguyễn Tài Tuệ]%[1D2Y1-1]
	Một hộp có $12$ viên bi trắng, $10$ viên bi xanh và $8$ viên bi đỏ. Một em bé muốn chọn $1$ viên bi để chơi. Hỏi có bao nhiêu cách chọn? \dapso{$30$ cách}
	\loigiai{Để chọn $1$ viên bi để chơi có các phương án
		\begin{enumerate}
			\item Chọn $1$ viên bi trắng có $12$ cách.
			\item Chọn $1$ viên bi xanh có $10$ cách.
			\item Chọn $1$ viên bi đỏ có $8$ cách.
		\end{enumerate}
		Theo quy tắc cộng, số cách để chọn $1$ viên bi để chơi là $12+10+8=30$ cách.
	}
\end{vd}

% \baitaptl

% \begin{bt}%[BG Toán 10, Nguyễn Tài Tuệ]%[1D2Y1-1]
% 	Một hộp có $10$ viên bi trắng, $8$ viên bi xanh và $9$ viên bi đỏ. Một em bé muốn chọn $1$ viên bi để chơi thì có số cách chọn là
% 	 \dapso{27}
% 	\loigiai{
% 		Để chọn $1$ viên bi để chơi có các phương án
% 		\begin{itemize}
% 			\item Chọn $1$ viên bi trắng có $10$ cách.
% 			\item Chọn $1$ viên bi xanh có $8$ cách.
% 			\item Chọn $1$ viên bi đỏ có $9$ cách.
% 		\end{itemize}
% 		Theo quy tắc cộng, số cách để chọn $1$ viên bi để chơi là $10+8+9=27$ cách.
% 	}
% \end{bt}

% \begin{bt}%[BG Toán 10, Nguyễn Tài Tuệ]%[1D2Y1-1]
% 	Một học sinh thi cuối kỳ có thể chọn một trong ba loại đề: đề dễ có $48$ câu hỏi, đề trung bình có $40$ câu hỏi và đề khó có $32$ câu hỏi. Hỏi có bao nhiêu cách chọn một câu hỏi từ các đề thi trên?
% 	\dapso{$120$}
% 	\loigiai
% 	{
% 		Số cách chọn $1$ câu hỏi từ đề dễ là $48$ cách.\\
% 		Số cách chọn $1$ câu hỏi từ đề trung bình là $40$ cách.\\
% 		Số cách chọn $1$ câu hỏi từ đề khó là $32$ cách.\\
% 		Vậy số cách chọn $1$ câu hỏi là $48+40+32=120$ cách.
% 	}
% \end{bt}

% \begin{bt}%[BG Toán 10, Nguyễn Tài Tuệ]%[1D2Y1-1]
% 	Có $8$ quyển sách Toán, $7$ quyển sách Lí, $5$ quyển sách Hóa. Một học sinh chọn $1$ quyển trong bất kỳ $3$ loại trên. Hỏi có bao nhiêu cách chọn? \dapso{$20$ cách}
% 	\loigiai{Để chọn $1$ quyển sách trong $3$ loại sách, ta có các phương án
% 		\begin{enumerate}
% 			\item Chọn $1$ quyển sách Toán có $8$ cách.
% 			\item Chọn $1$ quyển sách Lí có $7$ cách.
% 			\item Chọn $1$ quyển sách Hóa có $5$ cách.
% 		\end{enumerate}
% 		Theo quy tắc cộng, số cách để chọn $1$ viên bi để chơi là $8+7+5=20$ cách.
% 	}
% \end{bt}

% \begin{bt}%[BG Toán 10, Nguyễn Tài Tuệ]%[1D2Y1-1]
% 	Một nhà hàng có $3$ loại rượu, $4$ loại bia và $6$ loại nước ngọt. Thực khách cần chọn đúng một loại thức uống. Hỏi có mấy cách chọn?  \dapso{13}
% 	\loigiai{
% 		Chọn rượu có $3$ cách, chọn bia có $4$ cách, chọn nước ngọt có $6$ cách.\\
% 		Vậy có $3+4+6=13$ cách chọn.
% 	}
% \end{bt}

% \begin{bt}%[BG Toán 10, Nguyễn Tài Tuệ]%[1D2Y1-1]
% 	Một lớp có $40$ học sinh, đăng ký chơi ít nhất một trong hai môn thể thao là bóng đá và cầu lông. Có $30$ em đăng ký môn bóng đá, $25$ em đăng ký môn cầu lông. Hỏi có bao nhiêu em đăng ký cả hai môn thể thao?
% 	\dapso{$15$ học sinh}
% 	\loigiai{
% 		Số em học sinh đăng ký cả hai môn thể thao là $30+25-40=15$ học sinh.
% 	}
% \end{bt}

% \begin{bt}%[BG Toán 10, Nguyễn Tài Tuệ]%[1D2Y1-1]
% 	Trong một trường THPT A, khối $11$ mỗi học sinh tham gia một trong hai câu lạc bộ Toán và Tin học. Có $160$ em tham gia câu lạc bộ Toán, $140$ em tham gia câu lạc bộ Tin học, $50$ em tham gia cả hai câu lạc bộ. Hỏi khối $11$ có bao nhiêu học sinh?
% 	\dapso{$250$ học sinh}
% 	\loigiai{
% 		Số học sinh khối $11$ là $160+140-50=250$ học sinh.
% 	}
% \end{bt}

% \begin{bt}%[BG Toán 10, Nguyễn Tài Tuệ]%[1D2Y1-1]
% 	Trong một cuộc thi tìm hiểu về đất nước Việt Nam, ban tổ chức công bố danh sách các đề tài bao gồm: $8$ đề tài về lịch sử, $7$ đề tài về thiên nhiên, $10$ đề tài về con người và $6$ đề tài về văn hóa. Hỏi mỗi thí sinh có bao nhiêu cách chọn đề tài? \dapso{31}
% 	\loigiai{
% 		Mỗi thí sinh có các $4$ phương án chọn đề tài:
% 		\begin{itemize}
% 			\item Chọn đề tài về lịch sử có $8$ cách chọn.
% 			\item Chọn đề tài về thiên nhiên có $7$ cách chọn.
% 			\item Chọn đề tài về con người có $10$ cách chọn.
% 			\item Chọn đề tài về văn hóa có $6$ cách chọn.
% 		\end{itemize}
% 		Theo quy tắc cộng, có $8+7+10+6=31$ cách chọn đề tài.
% 	}
% \end{bt}

% \begin{bt}%[BG Toán 10, Nguyễn Tài Tuệ]%[1D2Y1-1]
% 	Lớp $11A$ có $30$ học sinh và lớp $11B$ có $32$ học sinh, có bao nhiêu cách chọn $1$ học sinh từ $2$ lớp trên để tham gia đội công tác xã hội?  \dapso{62}
% 	\loigiai{
% 		\begin{itemize}
% 			\item Chọn học sinh lớp $11A$ có $30$ cách chọn.
% 			\item Chọn học sinh lớp $11B$ có $32$ cách chọn.
% 		\end{itemize}
% 		Vậy có $30+32=62$ cách chọn.
% 	}
% \end{bt}

% \begin{bt}%[BG Toán 10, Nguyễn Tài Tuệ]%[1D2Y1-1]
% 	Trong một trường THPT, khối $11$ có $280$ học sinh nam và $325$ học sinh nữ. Nhà trường cần chọn một học sinh ở khối $11$ đi dự dạ hội của học sinh thành phố. Hỏi nhà trường có bao nhiêu cách chọn?  \dapso{605}
% 	\loigiai{
% 		\begin{itemize}
% 			\item Nếu chọn một học sinh nam có $280$ cách.
% 			\item Nếu chọn một học sinh nữ có $325$ cách.
% 		\end{itemize}
% 		Theo quy tắc cộng, ta có $280+325=605$ cách chọn.}
% \end{bt}

% \begin{bt}%[1D2Y1-1]
% 	Một bó hoa gồm có $5$ bông hồng trắng, $6$ bông hồng đỏ và $7$ bông hồng vàng. Hỏi có mấy cách chọn lấy một bông hoa?  \dapso{18}
% 	\loigiai{
% 		\begin{itemize}
% 			\item Chọn bông hồng trắng có $5$ cách chọn.
% 			\item Chọn bông hồng đỏ có $6$ cách chọn.
% 			\item Chọn bông hồng vàng có $7$ cách chọn.
% 		\end{itemize}
% 		Vậy có $5+6+7=18$ cách chọn.
% 	}
% \end{bt}

% \begin{bt}%[BG Toán 10, Nguyễn Tài Tuệ]%[1D2Y1-1]
% 	Giả sử từ tỉnh $A$ đến tỉnh $B$ có thể đi bằng các phương tiện: ô tô, tàu hỏa hoặc máy bay. Mỗi ngày có $10$ chuyến ô tô, $5$ chuyến tàu hỏa và $3$ chuyến máy bay. Hỏi có bao nhiêu cách lựa chọn chuyến đi từ tỉnh $A$ đến tỉnh $B$? \dapso{18}
% 	\loigiai{
% 		Để đi từ $A$ đến $B$ có $3$ phương án lựa chọn:
% 		\begin{itemize}
% 			\item Đi bằng ô tô có $10$ cách chọn.
% 			\item Đi bằng tàu hỏa có $5$ cách chọn.
% 			\item Đi bằng máy bay có $3$ cách chọn.
% 		\end{itemize}
% 		Theo quy tắc cộng, có $10+5+3=18$ cách chọn.
% 	}
% \end{bt}

\subsubsection{Bài tập trắc nghiệm}
\Opensolutionfile{ansbook}[ans/ansbook-0D8-1-1]
\Opensolutionfile{ans}[ans/ans-0D8-1-1]
\setcounter{ex}{0}

% \begin{ex}%[BG Toán 10, Nguyễn Tài Tuệ]%[1D2Y1-1]
% 	Có $10$ cuốn sách Toán khác nhau, $11$ cuốn sách Văn khác nhau và $7$ cuốn sách Anh văn khác nhau. Một học sinh được chọn $1$ quyển sách trong các quyển sách trên. Hỏi có bao nhiêu cách lựa chọn?
% 	\choice
% 	{$26$}
% 	{$20$}
% 	{\True $28$}
% 	{$32$}
% 	\loigiai{
% 		Theo quy tắc cộng, ta có $10+11+7=28$ (cách).}
% \end{ex}

% \begin{ex}%[BG Toán 10, Nguyễn Tài Tuệ]%[1D2Y1-1]
% 	Một nhà hàng có $3$ loại rượu, $4$ loại bia và $5$ loại nước uống. Một thực khách muốn lựa chọn một loại đồ uống thì có bao nhiêu cách chọn?
% 	\choice
% 	{$7$}
% 	{$15$}
% 	{\True $12$}
% 	{$60$}
% 	\loigiai{
% 		\begin{itemize}
% 			\item Nếu thực khách chọn rượu làm đồ uống thì có $3$ cách chọn.
% 			\item Nếu thực khách chọn bia làm đồ uống thì có $4$ cách chọn.
% 			\item Nếu thực khách chọn $5$ loại nước uống còn lại làm đồ uống thì có $5$ cách chọn.
% 		\end{itemize}
% 		Như vậy thực khách có tất cả  $3+4+5=12$ cách chọn.}
% \end{ex}

% \begin{ex}%[BG Toán 10, Nguyễn Tài Tuệ]%[1D2Y1-1]
% 	Một tổ có $5$ học sinh nữ và $6$ học sinh nam. Có bao nhiêu cách chọn một học sinh của tổ đó đi trực nhật?
% 	\choice
% 	{$10$}
% 	{$20$}
% 	{\True $11$}
% 	{$30$}
% 	\loigiai{
% 		Số cách chọn một học sinh của tổ là $5+6=11$.
% 	}
% \end{ex}

% \begin{ex}%[BG Toán 10, Nguyễn Tài Tuệ]%[1D2Y1-1]
% 	Từ một nhóm học sinh gồm $ 7 $ nam và $ 9 $ nữ, có bao nhiêu cách chọn ra một học sinh?
% 	\choice
% 	{\True $ 16 $}
% 	{$ 7 $}
% 	{$ 9 $}
% 	{$ 63 $}
% 	\loigiai{
% 		Áp dụng quy tắc cộng ta có số cách chọn một học sinh là $ 7+9=16 $ cách.
% 	}
% \end{ex}

% \begin{ex}%[BG Toán 10, Nguyễn Tài Tuệ]%[1D2Y1-1]
% 	Lớp $11A$ có $26$ học sinh nam và $19$ học sinh nữ. Có bao nhiêu cách chọn ra một học sinh lớp $11A$ để làm lớp trưởng?
% 	\choice
% 	{$26$}
% 	{$19$}
% 	{\True$45$}
% 	{$494$}
% 	\loigiai{
% 		Số cách chọn một học sinh làm lớp trưởng từ $45$ học sinh của lớp $11A$ là $45$ cách chọn.
% 	}
% \end{ex}

\begin{ex}%[BG Toán 10, Nguyễn Tài Tuệ]%[1D2Y1-1]
	Một lớp có $39$ bạn nam và $10$ bạn nữ. Hỏi có bao nhiêu cách chọn một bạn phụ trách quỹ lớp?
	\choice
	{$390$}
	{$10$}
	{\True $49$}
	{$39$}
	\loigiai{
		Tổng cộng lớp có $49$ bạn nên sẽ có $49$ cách chọn một bạn phụ trách quỹ lớp.
	}
\end{ex}


\begin{ex}%[BG Toán 10, Nguyễn Tài Tuệ]%[1D2Y1-1]
	Trên giá sách có $5$ quyển sách Tiếng Anh khác nhau, $6$ quyển sách Toán khác nhau và $8$ quyển sách Tiếng Việt khác nhau. Số cách chọn $1$ quyển sách là
	\choice
	{$240$}
	{\True $19$}
	{$6$}
	{$8$}
	\loigiai{
		Có $5$ cách chọn một quyển sách Tiếng Anh, $6$ cách chọn một quyển sách Toán và $8$ cách chọn một quyển sách Tiếng Việt. Vậy có $5+6+8=19$ cách chọn một quyển sách.
	}
\end{ex}

\begin{ex}%[BG Toán 10, Nguyễn Tài Tuệ]%[1D2Y1-1]
	Một trường THPT được cử một học sinh đi dự trại hè toàn quốc. Nhà trường quyết định chọn một học sinh tiên tiến lớp $11A$ hoặc lớp $12B$. Hỏi nhà trường có bao nhiêu cách chọn, nếu biết rằng lớp $11A$ có $31$ học sinh tiên tiến và lớp $12B$ có $22$ học sinh tiên tiến?
	\choice
	{$682$}
	{$31$}
	{$9$}
	{\True $53$}
	\loigiai{
		\begin{itemize}
			\item Nếu chọn một học sinh lớp $11A$ có $31$ cách.
			\item Nếu chọn một học sinh lớp $12B$ có $22$ cách.
		\end{itemize}
		Theo quy tắc cộng, ta có $31+22=53$ cách chọn.
	}
\end{ex}

\begin{ex}%[BG Toán 10, Nguyễn Tài Tuệ]%[1D2Y1-1]
	Một lớp có $25$ học sinh nam và $20$ học sinh nữ. Hỏi có bao nhiêu cách chọn $1$ học sinh?
	\choice
	{\True $45$}
	{$20$}
	{$500$}
	{$25$}
	\loigiai{
		Có $25+20=45$ cách chọn $1$ học sinh.}
\end{ex}

\begin{ex}%[BG Toán 10, Nguyễn Tài Tuệ]%[1D2Y1-1]
	Trên giá sách có $10$ quyển sách Toán khác nhau, $11$ quyển sách Văn khác nhau và $7$ quyển sách Tiếng Anh khác nhau. Hỏi có bao nhiêu cách chọn một quyển sách trong các quyển sách nói trên?
	\choice
	{$32$}
	{$26$}
	{$20$}
	{\True $28$}
	\loigiai
	{Có $10$ cách để chọn $1$ quyển sách Toán, $11$ cách để chọn $1$ quyển sách Văn và $7$ cách để chọn $1$ quyển sách Tiếng Anh nên theo quy tắc cộng có $28$ cách chọn một quyển sách trong các quyển sách nói trên.}
\end{ex}

\begin{ex}%[BG Toán 10, Nguyễn Tài Tuệ]%[1D2Y1-1]
	Một người vào cửa hàng ăn nhưng chỉ đủ tiền mua $1$ món ăn. Thực đơn gồm $5$ món cơm, $6$ món mì và $3$ món cháo. Hỏi người đó có bao nhiêu cách chọn món?
	\choice
	{$5$}
	{$3$}
	{\True $14$}
	{$6$}
	\loigiai{
		Có $5$ cách chọn cơm, $6$ cách chọn mì và $3$ cách chọn cháo.\\
		Vậy có tất cả $5+6+3=14$ cách chọn món.
	}
\end{ex}

\begin{ex}%[BG Toán 10, Nguyễn Tài Tuệ]%[1D2B1-1]
	Có $8$ quyển sách khác nhau và $6$ quyển vở khác nhau. Số cách chọn một trong các quyển đó là
	\choice
	{$8$}
	{\True $14$}
	{$6$}
	{$48$}
	\loigiai{
	Để chọn được $1$ quyển sách hoặc vở, ta có hai phương án\\
	{\bf Phương án 1.} Chọn được quyển sách có $8$ cách.\\
	{\bf Phương án 2.} Chọn được quyển vở có $6$ cách.\\
	Do đó theo quy tắc cộng có $8 + 6 = 13$ cách.
	}
\end{ex}

\begin{ex}%[BG Toán 10, Nguyễn Tài Tuệ]%[1D2B1-1]
	Một lớp học có $7$ học sinh giỏi Toán, $5$ học sinh giỏi Văn, $4$ học sinh giỏi Anh. Hỏi có bao nhiêu cách chọn ra một học sinh giỏi bất kì?
	\choice
	{$7 $}
	{\True$16 $}
	{$12 $}
	{$140 $}
	\loigiai{
	Để chọn được $1$ học sinh giỏi, ta có ba phương án\\
	{\bf Phương án 1.} Chọn học sinh giỏi Toán có $7$ cách.\\
	{\bf Phương án 2.} Chọn học sinh giỏi Văn  có $5$ cách.\\
	{\bf Phương án 3.} Chọn học sinh giỏi Anh  có $4$ cách.\\
	Do đó theo quy tắc cộng có $7 + 5+4 = 16$ cách.
	}
\end{ex}


\begin{ex}%[BG Toán 10, Nguyễn Tài Tuệ]%[1D2B1-1]
	Giả sử bố bạn An muốn mua một chiếc xe hiệu Vision hoặc SH. Biết rằng xe máy hiệu Vision có 5 màu  khác nhau, xe máy hiệu SH có 9 màu khác nhau. Hỏi bố bạn An có bao nhiêu sự lựa chọn?
	\choice
	{$9$}
	{\True $14$}
	{$5$}
	{$45$}
	\loigiai{
		Có $5$ loại xe Vision và $9$ loại xe SH nên theo quy tắc cộng sẽ có $14$ cách chọn mua một chiếc xe.}
\end{ex}

\begin{ex}%[BG Toán 10, Nguyễn Tài Tuệ]%[1D2B1-1]
	Một cô gái có $2$ cái mũ màu trắng, $3$ cái mũ màu xanh và $5$ cái mũ màu vàng, tất cả các cái mũ đều khác kiểu. Hỏi cô gái này có bao nhiêu cách chọn một cái mũ để đội đi dạo?
	\choice
	{$5$}
	{\True $10$}
	{$30$}
	{$6$}
	\loigiai{
		Theo quy tắc cộng ta có $2+3+5=10$ cách chọn một cái mũ.
	}
\end{ex}

\begin{ex}%[BG Toán 10, Nguyễn Tài Tuệ]%[1D2B1-1]
	Một bạn muốn đi từ tỉnh $A$ tới tỉnh $B$ trong một ngày nhất định. Biết rằng trong ngày hôm đó từ tỉnh $A$ đến tỉnh $B$ có $14$ chuyến ô tô, $5$ chuyến tàu. Hỏi bạn đó có bao nhiêu sự lựa chọn để đi từ $A$ đến $B$?
	\choice
	{$70$}
	{\True $19$}
	{$14$}
	{$5$}
	\loigiai{
		Để đi từ $A$ đến $B$ có thể chọn đi ô tô hoặc đi tàu nên theo quy tắc cộng ta có $19$ cách chọn.}
\end{ex}

\begin{ex}%[BG Toán 10, Nguyễn Tài Tuệ]%[1D2Y1-1] 
	Trong một hộp chứa sáu quả cầu trắng được đánh số từ $1$ đến $6$ và ba quả cầu đen được đánh số từ $7$ đến $9$. Có bao nhiêu cách chọn một trong các quả cầu ấy?
	\choice
	{$1$}
	{\True $9$}
	{$6$}
	{$3$}
	\loigiai{
		Mỗi quả cầu được đánh một số khác nhau, nên mỗi lần lấy ra một quả cầu bất kì là một lần.\\
		Số quả cầu là $6+3=9$.\\
		Tương ứng với $9$ cách.}
\end{ex}


\begin{ex}%[BG Toán 10, Nguyễn Tài Tuệ]%[1D2Y1-1]
	Trong một trường THPT, khối $11$ có $280$ học sinh nam và $325$ học sinh nữ. Nhà trường chọn một học sinh ở khối $11$ đi dự dạ hội của học sinh thành phố. Hỏi nhà trường có bao nhiêu cách chọn?
	\choice
	{\True $605$}
	{$280$}
	{$325$}
	{$45$}
	\loigiai{
		\begin{itemize}
			\item Chọn một học sinh nam có $280$ cách.
			\item Chọn một học sinh nữ có $325$ cách.
		\end{itemize}
		Vậy có $280+325=605$ cách chọn.
	}
\end{ex}

\begin{ex}%[BG Toán 10, Nguyễn Tài Tuệ]%[1D2Y1-1]
	Giả sử một công việc có thể được tiến hành theo hai phương án $A$ và $B$. Phương án $A$ có thể thực hiện bằng $n$ cách, phương án $B$ có thể thực hiện bằng $m$ cách không trùng với cách nào của phương án $A$. Khi đó
	\choice
	{Công việc có thể được thực hiện bằng $m\cdot n$ cách}
	{\True Công việc có thể được thực hiện bằng $m+n$ cách}
	{Công việc có thể được thực hiện bằng $\dfrac{1}{2}(m+n)$ cách}
	{Công việc có thể được thực hiện bằng $\dfrac{1}{2}\cdot m\cdot n$ cách}
	\loigiai{
		Theo quy tắc cộng có $m+n$ cách.
	}
\end{ex}


\begin{ex}%[BG Toán 10, Nguyễn Tài Tuệ]%[1D2Y1-1]  
	Từ một bó hoa hồng gồm $3$ bông hồng trắng, $5$ bông hồng đỏ và $6$ bông hồng vàng, có bao nhiêu cách chọn ra một bông hồng?
	\choice
	{$11$}
	{$90$}
	{\True $14$}
	{$8$}
	\loigiai{
		Ta có
		\begin{itemize}
			\item Chọn một bông hồng trắng có $3$ cách.
			\item Chọn một bông hồng đỏ có $5$ cách.
			\item Chọn một bông hồng vàng có $6$ cách.
		\end{itemize}
		Theo quy tắc cộng, ta có $3 + 5 + 6 = 14$ cách chọn một bông hồng.
	}
\end{ex}

\Closesolutionfile{ans}
\Closesolutionfile{ansbook}
% \begin{indapan}{10}{ans/ans-0D8-1-1}\end{indapan}
%============
\begin{dang}{Bài toán sử dụng quy tắc nhân}
	Ta sử dụng quy tắc nhân để giải các bài toán đếm trong đó việc thực hiện một công việc được chia thành nhiều giai đoạn, ứng với mỗi cách thực hiện giai đoạn trước sẽ có số cách thực hiện giai đoạn sau cố định.
\end{dang}
%==============================================
\viduminhhoa
\begin{vd}%[BG Toán 10, Nguyễn Tiến]%[1D2Y1-2]
	Bạn An có $4$ áo sơ-mi khác màu và $3$ quần dài khác nhau. Hỏi bạn An có bao nhiêu cách chọn ra một bộ đồ gồm $1$ áo sơ-mi và $1$ quần dài?
	\loigiai{
		Mỗi cách chọn một áo sơ-mi sẽ có tương ứng $3$ cách chọn quần dài.\\
		Do đó, bạn An có $4$ cách chọn áo sơ-mi và $3$ cách chọn quần dài.\\
		Áp dụng quy tắc nhân ta có $4\cdot 3=12$ (cách chọn).
	}
\end{vd}

\begin{vd}%[BG Toán 10, Nguyễn Tiến]%[1D2Y1-2]
	Một trường phổ thông có $12$ học sinh chuyên tin và $18$ học sinh chuyên toán. Thành lập một đoàn gồm hai người dự hội nghị sao cho có một học sinh chuyên tin và một học sinh chuyên toán. Hỏi có bao nhiêu cách lập một đoàn như trên?
	\loigiai{
		Để có một đoàn đi dự hội nghị phải có đồng thời một học sinh chuyên tin và một học sinh chuyên toán.\\
		Mỗi cách chọn một học sinh chuyên tin trong số $12$ học sinh chuyên tin sẽ có $18$ cách chọn một học sinh chuyên toán trong $18$ học sinh chuyên toán.\\
		Theo quy tắc nhân ta có $12\cdot 18=216$ (cách).
	}
\end{vd}

\begin{vd}%[BG Toán 10, Nguyễn Tiến]%[1D2Y1-2]
	Từ Quảng Trị đến Quảng Ngãi có $4$ con đường và có $6$ con đường từ Quảng Ngãi đến TPHCM. Hỏi có bao nhiêu con đường khác nhau để đi từ Quảng Trị đến TPHCM qua Quảng Ngãi?
	\loigiai{
		\begin{itemize}
			\item Số cách chọn đường đi từ Quảng Trị đến Quảng Ngãi là $4$.
			\item Số cách chọn đường đi từ Quảng Ngãi đến TPHCM là $6$.
		\end{itemize}
		Vậy có $4\cdot 6=24$ (cách chọn).
	}
\end{vd}

\begin{vd}%[BG Toán 10, Nguyễn Tiến]%[1D2B1-2]
	Cho tập hợp $A=\{1,2,3,4,5\}$. Có bao nhiêu số tự nhiên gồm ba chữ số đôi một khác nhau được tạo từ các chữ số trong tập $A$?
	\loigiai{
		Gọi số tự nhiên có ba chữ số cần tìm là $\overline{abc}$, trong đó
		\begin{itemize}
			\item $a$ có $5$ cách chọn.
			\item $b$ có $4$ cách chọn.
			\item $c$ có $3$ cách chọn.
		\end{itemize}
		Vậy có $5\cdot 4\cdot 3=60$ (số).
	}
\end{vd}
%==============================================
\baitaptl
\begin{bt}%[BG Toán 10, Nguyễn Tiến]%[1D2Y1-2]
	Cho tập hợp $A=\{0,1,2,3,4,5\}$. Có bao nhiêu số tự nhiên gồm năm chữ số đôi một khác nhau được tạo từ các chữ số trong tập $A$?
	\dapso{
		$600$ số.
	}
	\loigiai{
		Gọi số cần tìm là $\overline{abcde}$.\\
		Do các chữ số đôi một khác nhau nên có $5$ cách chọn $a\neq 0$; $5$ cách chọn $b$; $4$ cách chọn $c$; $3$ cách chọn $d$ và $2$ cách chọn $e$.\\
		Vậy có $5\cdot 5\cdot 4\cdot 3\cdot 2=600$ (số).
	}
\end{bt}

\begin{bt}%[BG Toán 10, Nguyễn Tiến]%[1D2B1-2]
	Cho tập hợp $A=\{1,2,3,4,5,7,8\}$. Từ $A$ có thể lập được bao nhiêu số tự nhiên gồm năm chữ số đôi một khác nhau và chia hết cho $5$?
	\dapso{
		$360$ số.
	}
	\loigiai{
		Gọi số cần tìm là $\overline{abcde}$.\\
		Do chia hết cho $5$ nên có $1$ cách chọn $e=5$.\\
		Đồng thời, do các chữ số đôi một khác nhau nên có $6$ cách chọn $a$; $5$ cách chọn $b$; $4$ cách chọn $c$ và $3$ cách chọn $d$.\\
		Vậy có $1\cdot 6\cdot 5\cdot 4\cdot 3=360$ (số).
	}
\end{bt}

\begin{bt}%[BG Toán 10, Nguyễn Tiến]%[1D2B1-2]
	Có bao nhiêu biển đăng kí xe ô tô nếu mỗi biển số chứa một dãy ba chữ cái (trong bảng $26$ chữ cái tiếng Anh), tiếp sau là bốn chữ số?
	\dapso{
		$175 760 000$ số.
	}
	\loigiai{
		Giả sử mỗi biển số xe có dạng $a_1 a_2 a_3 b_1 b_2 b_3 b_4$, trong đó $a_i$ ($i=\overline{1,3}$) là các chữ cái và $b_j$ ($j=\overline{1,4}$) là các số.\\
		Do các chữ cái có thể giống nhau nên có $26$ cách chọn $a_1$, $26$ cách chọn $a_2$, $26$ cách chọn $a_3$.\\
		Đồng thời, do các số có thể giống nhau nên có $10$ cách chọn $b_1$, $10$ cách chọn $b_2$, $10$ cách chọn $b_3$ và $10$ cách chọn $b_4$.\\
		Vậy có $26^3\cdot 10^4=175 760 000$ số.
	}
\end{bt}

\begin{bt}%[BG Toán 10, Nguyễn Tiến]%[1D2B1-2]
	Có bao nhiêu số tự nhiên có ba chữ số bắt đầu bằng chữ số lẻ và các chữ số đôi một khác nhau?
	\dapso{
		$360$ số.
	}
	\loigiai{
		Gọi số cần tìm là $\overline{abc}$.\\
		Do bắt đầu bằng chữ số lẻ nên có $5$ cách chọn $a$.\\
		Đồng thời, do các chữ số đôi một khác nhau nên có $9$ cách chọn $b$ và $8$ cách chọn $c$.\\
		Vậy có $5\cdot 9\cdot 8=360$ (số).
	}
\end{bt}

\begin{bt}%[BG Toán 10, Nguyễn Tiến]%[1D2B1-2]
	Từ các số $1;2; \ldots ;9$ có thể lập được bao nhiêu số tự nhiên có bốn chữ số đôi một khác nhau, bắt đầu bằng chữ số lẻ và kết thúc bằng chữ số chẵn?
	\dapso{
		$840$ số.
	}
	\loigiai{
		Gọi số cần tìm là $\overline{abcd}$.\\
		Do kết thúc bằng chữ số chẵn nên có $4$ cách chọn $d$.\\
		Do bắt đầu bằng chữ số lẻ nên có $5$ cách chọn $a$.\\
		Đồng thời, do các chữ số đôi một khác nhau nên có $7$ cách chọn $b$ và $6$ cách chọn $c$.\\
		Vậy có $4\cdot 5\cdot 7\cdot 6=840$ (số).
	}
\end{bt}

\begin{bt}%[BG Toán 10, Nguyễn Tiến]%[1D2B1-2]
	Từ các số $0;4;5;7;8;9$ có thể lập được bao nhiêu số tự nhiên có bốn chữ số khác nhau và lớn hơn $5000$?
	\dapso{
		$240$ số.
	}
	\loigiai{
		Gọi số cần tìm là $\overline{abcd}$.\\
		Do số cần tìm lớn hơn $5000$ nên có $4$ cách chọn $a\in \{5;7;8;9\}$.\\
		Đồng thời, do các chữ số khác nhau nên có $5$ cách chọn $b$; $4$ cách chọn $c$ và $3$ cách chọn $d$.\\
		Vậy có $4\cdot 5\cdot 4\cdot 3=240$ (số).
	}
\end{bt}

\begin{bt}%[BG Toán 10, Nguyễn Tiến]%[1D2B1-2]
	Có bao nhiêu số tự nhiên có năm chữ số khác nhau được viết từ các số $1;2;3;4;5$, trong đó ba chữ số đầu là ba chữ số lẻ và hai chữ số cuối là hai chữ số chẵn?
	\dapso{
		$12$ số.
	}
	\loigiai{
		Gọi số cần tìm là $\overline{abcde}$.\\
		Do số cần tìm có ba chữ số đầu là ba chữ số lẻ nên có $3$ cách chọn $a$, $2$ cách chọn $b$, $1$ cách chọn $c$.\\
		Đồng thời, do số cần tìm có hai chữ số cuối là hai chữ số chẵn nên có $2$ cách chọn $d$ và $1$ cách chọn $e$.\\
		Vậy có $3\cdot 2\cdot 1\cdot 2\cdot 1=12$ (số).
	}
\end{bt}

\subsubsection{Bài tập trắc nghiệm}
\Opensolutionfile{ansbook}[ans/ansbook-0D8-1-2]
\Opensolutionfile{ans}[ans/ans-0D8-1-2]
\setcounter{ex}{0}
\begin{ex}%[BG Toán 10, Nguyễn Tiến]%[1D2Y1-2]
	Một công việc được hoàn thành bởi hai hành động liên tiếp. Nếu có $ m $ cách thực hiện hành động thứ nhất và ứng với mỗi cách đó có $ n $ cách thực hiện hành động thứ hai. Hỏi có bao nhiêu cách thực hiện công việc?
	\choice
	{$ m+n $}
	{$ m-n $}
	{$ \dfrac{m}{n}$}
	{\True $ m\cdot n $}
	\loigiai{
		Áp dụng qui tắc nhân.
	}
\end{ex}

\begin{ex}%[BG Toán 10, Nguyễn Tiến]%[1D2Y1-2]
	Anh $A$ có $7$ cái áo màu sắc khác nhau và $6$ cái quần có kiểu khác nhau. Anh $A$ có thể chọn nhiều nhất bao nhiêu bộ quần áo?
	\choice
	{$7$}
	{$13$}
	{$6$}
	{\True $42$}
	\loigiai{
		Ứng với mỗi cái áo anh $A$ chọn được $6$ kiểu quần.\\
		Vậy anh $A$ có thể chọn nhiều nhất $6\cdot7=42$ bộ quần áo.
	}
\end{ex}

\begin{ex}%[BG Toán 10, Nguyễn Tiến]%[1D2Y1-2]
	Để đi từ thị trấn $A$ đến thị trấn $C$ phải qua thị trấn $B$. Biết từ $A$ đến $B$ có 4 con đường, từ $B$ đến $C$ có 3 con đường. Khi đó số cách đi từ $A$ đến $C$ mà phải qua $B$ là:
	\choice
	{$6$}
	{$7$}
	{$15$}
	{\True{$12$}}
	\loigiai{
		Từ $A$ đến $B$ có $3$ cách đi.\\
		Từ $B$ đến $C$ có $4$ cách đi.\\
		Theo quy tắc nhân, từ $A$ đến $C$ phải qua $B$ có $3\cdot4=12$ cách.
	}
\end{ex}

\begin{ex}%[BG Toán 10, Nguyễn Tiến]%[1D2Y1-2]
	An muốn mua một cây bút mực và một cây bút chì. Các cây bút mực có $8$ màu khác nhau, các cây bút chì cũng có $8$ màu khác nhau. Vậy An có bao nhiêu cách chọn?
	\choice
	{\True $64$}
	{$16$}
	{$32$}
	{$20$}
	\loigiai{
		Số cách chọn mua một cây bút mực là $8$ cách.\\
		Số cách chọn mua một cây bút chì là $8$ cách.\\
		Nên theo quy tắc nhân thì An có $8 \cdot 8= 64$ cách.
	}
\end{ex}

\begin{ex}%[BG Toán 10, Nguyễn Tiến]%[1D2Y1-2]
	Lớp $12$A có $20$ bạn nữ, lớp $12$B có $16$ bạn nam. Có bao nhiêu cách chọn $1$ bạn nữ lớp $12$A và $1$ bạn nam lớp $12$B để dẫn chương trình hoạt động ngoại khóa?
	\choice
	{\True $320$}
	{$630$}
	{$36$}
	{$1220$}
	\loigiai{
		Để chọn $1$ bạn nữ của lớp $12$A ta có $20$ cách.\\
		Để chọn $1$ bạn nam của lớp $12$B ta có $16$ cách.\\
		Vậy theo quy tắc nhân ta có $20\times 16 = 320$.
	}
\end{ex}

\begin{ex}%[BG Toán 10, Nguyễn Tiến]%[1D2Y1-2]
	Một hộp có $3$ viên bi đỏ và $4$ viên bi xanh. Số cách lấy ra hai viên bi, trong đó có $1$ viên bi đỏ và $1$ viên bi xanh bằng
	\choice
	{$7$}
	{$81$}
	{$64$}
	{\True $12$}
	\loigiai{
		Số cách lấy ra hai viên bi, trong đó có $1$ viên bi đỏ và $1$ viên bi xanh bằng $3\cdot 4=12$ (cách).
	}
\end{ex}

\begin{ex}%[BG Toán 10, Nguyễn Tiến]%[1D2Y1-2]
	Có hai kiểu mặt đồng hồ đeo tay (vuông, tròn) và có ba kiểu dây (kim loại, da, nhựa). Hỏi có bao nhiêu cách chọn một chiếc đồng hồ có một mặt và một dây?
	\choice
	{$8$}
	{$7$}
	{$5$}
	{\True $6$}
	\loigiai{
		Theo quy tắc nhân, số cách chọn ra một chiếc đồng hồ là $2\cdot 3=6$.
	}
\end{ex}

\begin{ex}%[BG Toán 10, Nguyễn Tiến]%[1D2Y1-2]
	Số các số tự nhiên gồm $3$ chữ số được tạo thành từ $4$ chữ số $0;1;2;3$ là
	\choice
	{$56$}
	{$96$}
	{$52$}
	{\True $48$}
	\loigiai{
		Có $3$ cách chọn chữ số hàng trăm, $4$ cách chọn chữ số hàng chục, $4$ cách chọn chữ số hàng đơn vị, nên số các số thoả mãn là $3\cdot 4\cdot 4=48$.
	}
\end{ex}

\begin{ex}%[BG Toán 10, Nguyễn Tiến]%[1D2Y1-2]
	Liên quan đến chuyên ngành bạn Linh muốn học ở bậc đại học, có $4$ trường đại học, mỗi trường có $1$ khoa và ở mỗi khoa đó có $3$ ngành học về chuyên ngành bạn Linh muốn học. Hỏi bạn
	Linh có bao nhiêu lựa chọn?
	\choice
	{$64$}
	{\True $12$}
	{$81$}
	{$7$}
	\loigiai{
		Số cách chọn trường: $4$ cách.\\
		Số cách chọn khoa trong trường: $1$ cách.\\
		Số cách chọn ngành trong khoa: $3$ cách.\\
		Theo quy tắc nhân ta có $4\cdot 1\cdot 3=12$ cách.
	}
\end{ex}

\begin{ex}%[BG Toán 10, Nguyễn Tiến]%[1D2B1-2]
	Cho các chữ số $2,\,3,\,4,\,5,\,6,\,7.$ Khi đó có bao nhiêu số tự nhiên có bốn chữ số được thành lập từ các chữ số đã cho?
	\choice
	{$1296$}
	{\True $360$}
	{$24$}
	{$720$}
	\loigiai{
		Từ $6$ chữ số tự nhiên đã cho, ta có $6$ cách chọn chữ số hàng đơn vị, $6$ cách chọn chữ số hàng chục, $6$ cách chọn chữ số hàng trăm, $6$ cách chọn chữ số hàng nghìn.\\
		Theo quy tắc nhân suy ra số các số thỏa mãn yêu cầu bài toán là $6^4=1296$ cách.
	}
\end{ex}

\begin{ex}%[BG Toán 10, Nguyễn Tiến]%[1D2B1-2]
	Đề kiểm tra học kì $1$ môn Toán khối $11$ ở một trường THPT gồm $2$ phần tự luận và trắc nghiệm, trong đó phần tự luận có $13$ đề, phần trắc nghiệm có $10$ đề. Mỗi học sinh phải làm bài thi gồm một đề tự luận và một đề trắc nghiệm. Hỏi trường THPT đó có bao nhiêu cách chọn đề thi?
	\choice
	{\True $130$}
	{ $23$}
	{$253$}
	{$506$}
	\loigiai{
		Số cách chọn một đề tự luận và một đề trắc nghiệm lần lượt là $13$, $10$.\\
		Vậy số cách chọn đề thi là $13 \cdot 10 = 130$.
	}
\end{ex}

\begin{ex}%[BG Toán 10, Nguyễn Tiến]%[1D2B1-2]
	Cho $6$ chữ số $2, 3, 4, 5, 6, 7.$ Có bao nhiêu số tự nhiên chẵn có $3$ chữ số lập từ $6$ chữ số đó.
	\choice
	{$256$}
	{\True $108$}
	{$36$}
	{$18$}
	\loigiai{
		Gọi $\overline{a_1a_2a_3}$ là số tự nhiên cần lập.\\
		Ta có $a_3$ có $3$ cách chọn, $a_2, a_1$ có $6$ cách.\\
		Vậy có $3\cdot 6\cdot 6 = 108$.
	}
\end{ex}

\begin{ex}%[BG Toán 10, Nguyễn Tiến]%[1D2B1-2]
	Trong mặt phẳng, cho một đa giác lồi có $20$ cạnh. Số đường chéo của đa giác là
	\choice
	{$340$}
	{$380$}
	{$190$}
	{\True $170$}
	\loigiai{
		Từ mỗi đỉnh của đa giác ta kẻ được $17$ đường chéo.\\
		Từ $20$ đỉnh kẻ được $17\cdot20=340$ đường chéo.\\
		Tuy nhiên, theo cách vẽ ở trên thì mỗi đường chéo của đa giác được kẻ $2$ lần.\\
		Vậy số đường chéo của đa giác là $\dfrac{340}{2}=170$.
	}
\end{ex}

\begin{ex}%[BG Toán 10, Nguyễn Tiến]%[1D2B1-2]
	Một đa giác đều có số đường chéo gấp đôi số cạnh. Hỏi đa giác đó có bao nhiêu cạnh?
	\choice
	{$6$}
	{\True $7$}
	{$5$}
	{$8$}
	\loigiai{
		Gọi số đỉnh của đa giác là $n$. Mà số cạnh bằng số đỉnh nên số cạnh của đa giác là $n$.\\
		Cứ mỗi đỉnh nối với $(n-3)$ đỉnh còn lại tạo thành $(n-3)$ đường chéo nên số đường chéo của đa giác là $\dfrac{n(n-3)}{2}$ (do mỗi đường chéo được tính hai lần).
		Vì số đường chéo gấp đôi số cạnh nên
		$$\dfrac{n(n-3)}{2}=2n\Leftrightarrow n-3=4\Leftrightarrow n=7.$$
		Vậy đa giác có $7$ cạnh.
	}
\end{ex}

\begin{ex}%[BG Toán 10, Nguyễn Tiến]%[1D2B1-2]
	Có bao nhiêu số tự nhiên có ba chữ số khác nhau?
	\choice
	{$1000$}
	{$729$}
	{\True $648$}
	{$720$}
	\loigiai{
		Gọi số cần lập là $\overline{abc}$ với $a\ne 0$.\\
		Chọn $a$ có 9 cách.\\
		Chọn $b$ có 9 cách.\\
		Chọn $c$ có 8 cách.\\
		Vậy có $9\cdot 9\cdot 8=648$ số tự nhiên có ba chữ số khác nhau.
	}
\end{ex}

\begin{ex}%[BG Toán 10, Nguyễn Tiến]%[1D2B1-2]
	Có bao nhiêu số tự nhiên có hai chữ số mà tất cả các chữ số đều là chữ số lẻ?
	\choice
	{$ 10 $}
	{\True $ 25 $}
	{$ 45 $}
	{$ 50 $}
	\loigiai{
		Tập hợp các chữ số lẻ là $ \{1, 3, 5, 7, 9\} $.\\
		Số tự nhiên có hai chữ số có dạng $ \overline{ab} $.\\
		Vì tất cả các chữ số đều là chữ số lẻ nên $ a, b\in \{1, 3, 5, 7, 9\}$.
		\begin{itemize}
			\item Vị trí $ a $ có $ 5 $ cách chọn.
			\item Vị trí $ b $ có $ 5 $ cách chọn.
		\end{itemize}
		Vậy có tất cả $ 5\times 5=25 $ số tự nhiên có hai chữ số mà tất cả các chữ số đều lẻ.
	}
\end{ex}

\begin{ex}%[BG Toán 10, Nguyễn Tiến]%[1D2B1-2]
	Cho tập $A=\left\{0;1;2;3;4;5;6\right\}$. Từ tập $A$ có thể lập được bao nhiêu số tự nhiên có $5$ chữ số và chia hết cho $2$?
	\choice
	{\True $8232$}
	{$1230$}
	{$1260$}
	{$2880$}
	\loigiai{
		Gọi $\overline{abcde}\left(a\ne 0\right)$ là số cần lập. Vì không có yêu cầu các chữ số phải khác nhau nên ta có: \\
		Chọn $a$ có $6$ cách. \\
		Chọn $e$ có $4$ cách. \\
		Chọn các chữ số $b,c,d$ thì có $7$ cách chọn mỗi chữ số. \\
		Vậy có $7 \cdot 7 \cdot 7 \cdot 6 \cdot 4=8232$ (số).
	}
\end{ex}

\begin{ex}%[BG Toán 10, Nguyễn Tiến]%[1D2B1-2]
	Một phòng có $12$ người. Cần lập một tổ đi công tác $3$ người, một người làm tổ trưởng, một người làm tổ phó và một người là thành viên. Hỏi có bao nhiêu cách lập?
	\choice
	{$220$}
	{$1728$}
	{$1230$}
	{\True $1320$}
	\loigiai{
		\begin{itemize}
			\item Có $12$ cách chọn một người làm tổ trưởng.
			\item Có $11$ cách chọn một người làm tổ phó.
			\item Có $10$ cách chọn một người làm thành viên.
		\end{itemize}
		Suy ra, số cách lập một tổ đi công tác $3$ người bằng $12 \cdot 11 \cdot 10 = 1320$.
	}
\end{ex}

\begin{ex}%[BG Toán 10, Nguyễn Tiến]%[1D2B1-2]
	Giả sử có $8$ vận động viên tham gia chạy thi. Nếu không kể trường hợp có hai vận động viên về đích cùng lúc thì có bao nhiêu kết quả có thể xảy ra đối với các vị trí nhất, nhì, ba?
	\choice
	{$56$}
	{$120$}
	{\True $336$}
	{$24$}
	\loigiai{
		Vị trí thứ nhất có $8$ khả năng, vị trí thứ nhì có $7$ khả năng và vị trí thứ ba có $6$ khả năng.\\
		Vậy có $8\times 7\times 6=336$.
	}
\end{ex}

\begin{ex}%[BG Toán 10, Nguyễn Tiến]%[1D2K1-2]
	Cho đa giác đều $16$ đỉnh. Hỏi có bao nhiêu tam giác vuông có ba đỉnh là ba đỉnh của đa giác đều đó?
	\choice
	{$560$}
	{\True $112$}
	{$121$}
	{$128$}
	\loigiai{
		Chọn $2$ đỉnh đối diện trong $16$ đỉnh ta được $8$ cạnh là đường kính của đường tròn ngoại tiếp đa giác đều.\\
		Khi đó, ta chọn $1$ trong $14$ đỉnh còn lại ta sẽ được một tam giác vuông tại đỉnh vừa chọn.\\
		Vậy có tất cả $8\times 14=112$ tam giác vuông tạo thành.
	}
\end{ex}

\begin{ex}%[BG Toán 10, Nguyễn Tiến]%[1D2K1-2]
	Từ các chữ số $0$, $1$, $2$, $3$, $5$, $8$ có thể lập được bao nhiêu số tự nhiên lẻ có bốn chữ số đôi một khác nhau và phải có mặt chữ số $3$.
	\choice
	{\True $108$ số}
	{$228$ số}
	{$36$ số}
	{$144$ số}
	\loigiai{
		Gọi $\overline{a_1a_2a_3a_4}$ là số lẻ có $4$ chữ số khác nhau, với $a_1,a_2,a_3,a_4\in \left\{0;1;2;3;5;8\right\}\Rightarrow a_4$ có $3$ cách chọn, $a_1$ có $4$ cách chọn, $a_2$ có $4$ cách chọn và $a_3$ có $3$ cách chọn.\\
		Khi đó, có $4\cdot 4\cdot 3=144$ số thỏa mãn yêu cầu trên.\\
		Gọi $\overline{b_1b_2b_3b_4}$ là số lẻ có $4$ chữ số khác nhau, với $b_1,b_2,b_3,b_4\in \left\{0;1;2;5;8\right\}$ $\Rightarrow b_4$ có $2$ cách chọn, $b_1$ có $3$ cách chọn, $b_2$ có $3$ cách chọn và $b_3$ có $2$ cách chọn.\\
		Do đó, có $2\cdot 3\cdot 3\cdot2=36$ số thỏa mãn yêu cầu trên.\\
		Vậy có tất cả $144-36=108$ số thỏa mãn yêu cầu bài toán.
	}
\end{ex}

\begin{ex}%[BG Toán 10, Nguyễn Tiến]%[1D2K1-2]
	Gieo một con súc sắc $6$ mặt cân đối $3$ lần, có bao nhiêu kết quả có thể xảy ra thỏa mãn điều kiện ``Tổng số chấm xuất hiện trong $3$ lần là số chẵn''?
	\choice
	{$162$}
	{$54$}
	{\True $108$}
	{$27$}
	\loigiai{
		Dù kết quả hai lần gieo đầu tiên như thế nào thì lần thứ ba cũng có $3$ khả năng xảy ra để phù hợp với điều kiện ``Tổng số chấm xuất hiện trong $3$ lần là số chẵn''.\\
		Do đó, số kết quả thỏa mãn điều kiện trên là $6 \times 6 \times 3 = 108$.
	}
\end{ex}

\begin{ex}%[BG Toán 10, Nguyễn Tiến]%[1D2G1-2]
	Cho $ 5 $ chữ số $ 1, 2, 3, 4, 6 $. Lập các số tự nhiên có $ 3 $ chữ số đôi một khác nhau từ $ 5 $ chữ số đã cho. Tính tổng của tất cả các số lập được.
	\choice
	{$ 12321 $}
	{\True $ 21312 $}
	{$ 12312 $}
	{$ 21321 $}
	\loigiai{
		Xét tập $ X = \{ 1, 2, 3, 4, 6\} $.\\
		Số các số tự nhiên có ba chữ số đôi một khác nhau lấy từ tập $ X $ là $ 5 \times 4 \times 3 = 60  $.\\
		Do vai trò các chữ số là như nhau, nên số lần xuất hiện của mỗi chữ số trong tập $ X $ tại mỗi hàng trăm, hàng chục, hàng đơn vị là $ \dfrac{60}{5} = 12$.\\
		Tống các số lập được $ S = (1 + 2 + 3 + 4 + 6) \times 12 \times 111 = 21312 $.
	}
\end{ex}

\Closesolutionfile{ans}
\Closesolutionfile{ansbook}
% \indapan{10}{ANS/ans-0D8-1-2}
\begin{dang}{Kết hợp quy tắc cộng và quy tắc nhân}
	Hầu hết các bài toán đếm trong thực tế sẽ phức tạp và cần áp dụng cả hai quy tắc cộng và quy tắc nhân để giải bài toán.
\end{dang}
\viduminhhoa
\begin{vd}%[1D2B1-3]
	Cho tập hợp $A=\left\{0;1;2;3;4;5;6;7\right\}$. Có bao nhiêu số tự nhiên gồm bốn chữ số được lấy từ $A$ sao cho các chữ số
	\begin{enumerate}
		\item Khác nhau từng đôi một.
		\item Khác nhau từng đôi một và nó là số lẻ.
		\item Khác nhau từng đôi một và nó là số chẵn.
		\item Khác nhau đôi một và chia hết cho $5$.
	\end{enumerate}
	\loigiai{
		\begin{enumerate}
			\item Gọi số có bốn chữ số cần lập là $\overline{abcd}$ với $a\ne b\ne c\ne d$.
			      \begin{itemize}
				      \item Chọn chữ số $a$ có $7$ cách do $a\ne 0$.
				      \item Chọn chữ số $b$ có $7$ cách do $b \ne a$.
				      \item Chọn chữ số $c$ có $6$ cách do $c \ne b$ và $c\ne a$.
				      \item Chọn chữ số $d$ có $5$ cách do $d \ne c$; $d\ne b$ và $d\ne a$.
			      \end{itemize}
			      Vậy theo quy tắc nhân có $7\cdot7\cdot 6\cdot 5=1470$ số.
			\item  Gọi số có bốn chữ số cần lập là $\overline{abcd}$ với $a\ne b\ne c\ne d$.
			      \begin{itemize}
				      \item Chọn chữ số $d$ có $4$ cách do $d\in\{1;3;5;7\}$.
				      \item Chọn chữ số $a$ có $6$ cách do $a \ne d$ và $a\ne 0$.
				      \item Chọn chữ số $b$ có $6$ cách do $b \ne d$ và $b\ne a$.
				      \item Chọn chữ số $c$ có $5$ cách do $c \ne d$; $c\ne a$ và $c\ne b$.
			      \end{itemize}
			      Vậy theo quy tắc nhân có $4\cdot6\cdot 6\cdot 5=720$ số.
			\item Gọi số có bốn chữ số cần lập là $\overline{abcd}$ với $a\ne b\ne c\ne d$.
			      \begin{enumerate}
				      \item \textbf{Trường hợp 1.} Chữ số $d=0$.
				            \begin{itemize}
					            \item Chọn chữ số $a$ có $7$ cách do $a\ne 0$.
					            \item Chọn chữ số $b$ có $6$ cách do $b\ne a$ và $b\ne 0$.
					            \item Chọn chữ số $c$ có $5$ cách do $c\ne a$; $c\ne b$ và $c\ne 0$.
				            \end{itemize}
				            Theo quy tắc nhân có $7\cdot 6\cdot 5=210$ số.
				      \item \textbf{Trường hợp 2.} Chữ số $d\in \{2;4;6\}$ nên có $3$ cách chọn.
				            \begin{itemize}
					            \item Chọn chữ số $a$ có $6$ cách do $a\ne 0$ và $a \ne d$.
					            \item Chọn chữ số $b$ có $6$ cách do $b\ne a$ và $b\ne d$.
					            \item Chọn chữ số $c$ có $5$ cách do $c\ne a$; $c\ne b$ và $c\ne d$.
				            \end{itemize}
				            Theo quy tắc nhân có $3\cdot 6\cdot 6\cdot 5=540$ số.
			      \end{enumerate}
			      Vậy theo quy tắc cộng có $210+540=750$ số.
			\item Gọi số có bốn chữ số cần lập là $\overline{abcd}$ với $a\ne b\ne c\ne d$.
			      \begin{enumerate}
				      \item \textbf{Trường hợp 1.} Chữ số $d=0$.
				            \begin{itemize}
					            \item Chọn chữ số $a$ có $7$ cách do $a\ne 0$.
					            \item Chọn chữ số $b$ có $6$ cách do $b\ne a$ và $b\ne 0$.
					            \item Chọn chữ số $c$ có $5$ cách do $c\ne a$; $c\ne b$ và $c\ne 0$.
				            \end{itemize}
				            Theo quy tắc nhân có $7\cdot 6\cdot 5=210$ số.
				      \item \textbf{Trường hợp 2.} Chữ số $d=5$.
				            \begin{itemize}
					            \item Chọn chữ số $a$ có $6$ cách do $a\ne 0$ và $a \ne 5$.
					            \item Chọn chữ số $b$ có $6$ cách do $b\ne a$ và $b\ne 5$.
					            \item Chọn chữ số $c$ có $5$ cách do $c\ne a$; $c\ne b$ và $c\ne 5$.
				            \end{itemize}
				            Theo quy tắc nhân có $6\cdot 6\cdot 5=180$ số.
			      \end{enumerate}
			      Vậy theo quy tắc cộng có $210+180=390$ số.
		\end{enumerate}
	}
\end{vd}
\begin{vd}%[1D2B1-3]
	Cho tập hợp $X=\left\{0;2;3;4;5;6;8\right\}$. Có bao nhiêu số tự nhiên gồm ba chữ số được lấy từ $X$ sao cho các chữ số
	\begin{enumerate}
		\item Khác nhau từng đôi một.
		\item Khác nhau từng đôi một và nó là số lẻ.
		\item Khác nhau từng đôi một và chia hết cho $2$.
		\item Khác nhau đôi một và chia hết cho $5$.
	\end{enumerate}
	\loigiai{
		\begin{enumerate}
			\item Gọi số có ba chữ số cần lập là $\overline{abc}$ với $a\ne b\ne c$.
			      \begin{itemize}
				      \item Chọn chữ số $a$ có $6$ cách do $a\ne 0$.
				      \item Chọn chữ số $b$ có $6$ cách do $b \ne a$.
				      \item Chọn chữ số $c$ có $5$ cách do $c \ne b$ và $c\ne a$.
			      \end{itemize}
			      Vậy theo quy tắc nhân có $6\cdot 6\cdot 5=180$ số.
			\item  Gọi số có ba chữ số cần lập là $\overline{abc}$ với $a\ne b\ne c$.
			      \begin{itemize}
				      \item Chọn chữ số $c$ có $2$ cách do $d\in\{3;5\}$.
				      \item Chọn chữ số $a$ có $5$ cách do $a \ne c$ và $a\ne 0$.
				      \item Chọn chữ số $b$ có $5$ cách do $b \ne c$ và $b\ne a$.
			      \end{itemize}
			      Vậy theo quy tắc nhân có $2\cdot5\cdot 5=50$ số.
			\item Gọi số có ba chữ số cần lập là $\overline{abc}$ với $a\ne b\ne c$.
			      \begin{enumerate}
				      \item \textbf{Trường hợp 1.} Chữ số $c=0$.
				            \begin{itemize}
					            \item Chọn chữ số $a$ có $6$ cách do $a\ne 0$.
					            \item Chọn chữ số $b$ có $5$ cách do $b\ne a$ và $b\ne 0$.
				            \end{itemize}
				            Theo quy tắc nhân có $6\cdot 5=30$ số.
				      \item \textbf{Trường hợp 2.} Chữ số $c\in \{2;4;6;8\}$ nên có $4$ cách chọn.
				            \begin{itemize}
					            \item Chọn chữ số $a$ có $5$ cách do $a\ne 0$ và $a \ne c$.
					            \item Chọn chữ số $b$ có $5$ cách do $b\ne a$ và $b\ne c$.
				            \end{itemize}
				            Theo quy tắc nhân có $4\cdot 5\cdot 5=100$ số.
			      \end{enumerate}
			      Vậy theo quy tắc cộng có $30+100=130$ số.
			\item Gọi số có ba chữ số cần lập là $\overline{abc}$ với $a\ne b\ne c$.
			      \begin{enumerate}
				      \item \textbf{Trường hợp 1.} Chữ số $c=0$.
				            \begin{itemize}
					            \item Chọn chữ số $a$ có $6$ cách do $a\ne 0$.
					            \item Chọn chữ số $b$ có $5$ cách do $b\ne a$ và $b\ne 0$.
				            \end{itemize}
				            Theo quy tắc nhân có $6\cdot 5=30$ số.
				      \item \textbf{Trường hợp 2.} Chữ số $d=5$.
				            \begin{itemize}
					            \item Chọn chữ số $a$ có $5$ cách do $a\ne 0$ và $a \ne 5$.
					            \item Chọn chữ số $b$ có $5$ cách do $b\ne a$ và $b\ne 5$.
				            \end{itemize}
				            Theo quy tắc nhân có $5\cdot 5=25$ số.
			      \end{enumerate}
			      Vậy theo quy tắc cộng có $25+30=55$ số.
		\end{enumerate}
	}
\end{vd}
\begin{vd}%[Huỳnh Đức Vũ-Bài giảng Toán 10-đợt 2]%[1D2K1-3]
	Có bao nhiêu số tự nhiên chẵn gồm bốn chữ số đôi một khác nhau được lập từ tập $E=\{0; 1; 2; 3; 4; 5; 6;7;8\}$?
	\loigiai{
		Mỗi cách lập ra số tự nhiên $\overline{abcd}$ là số chẵn, gồm $4$ chữ số đôi một khác nhau  từ tập $E$ được thực hiện theo một trong các phương án sau:
		\begin{itemize}
			\item Phương án $1$: $d=0$.
			      \begin{itemize}
				      \item Công đoạn $1$: Chọn $a\in E\setminus \{0\}$. Có $8$ cách.
				      \item Công đoạn $2$: Chọn $b\in E\setminus \{a;0\}$. Có $7$ cách.
				      \item Công đoạn $3$: Chọn $c\in E\setminus\{a;b;0\}$. Có $6$ cách.
			      \end{itemize}
			      Theo quy tắc nhân số cách chọn trong phương án này là $8\cdot 7\cdot 6=336$.\qquad (1)
			\item Phương án $2$: $d\in \{2;4;6;8\}$.
			      \begin{itemize}
				      \item Công đoạn $1$: Chọn $d\in \{2;4;6;8\}$. Có $4$ cách.
				      \item Công đoạn $2$: Chọn $a\in E\setminus\{d;0\}$. Có $7$ cách.
				      \item Công đoạn $3$: Chọn $b\in E\setminus\{a;d\}$. Có $7$ cách.
				      \item Công đoạn $4$: Chọn $c\in E\setminus\{a;d;b\}$. Có $6$ cách.
			      \end{itemize}
			      Theo quy tắc nhân số cách chọn trong phương án này là $4\cdot 7\cdot 7\cdot 6=1176$.\qquad (2)\\
			      Từ $(1)$ và $(2)$ theo quy tắc cộng, ta có số các số tự  nhiên thỏa đề bài là $336+1176=1512$.
		\end{itemize}  }
\end{vd}
\begin{vd}%[Huỳnh Đức Vũ-Bài giảng Toán 10-đợt 2]%[1D2K1-3]
	Từ thành phố $A$ đến thành phố $B$ có $3$ con đường, từ thành phố  $B$ đến thành phố $D$ có $4$ con đường, từ thành phố $A$ đến thành phố $C$ có $5$ con đường, từ thành phố $C$ đến thành phố $D$ có $6$ con đường, các con đường này đôi một khác nhau. Có bao nhiêu cách chọn đường đi $A$ đến $D$ rồi trở về $A$ mà không có con đường nào được đi lặp trở lại, biết rằng không có con đường nào  đi trực tiếp $B$ đến $C$ và đi trực tiếp từ  $A$ đến $D$.
	\begin{center}
		\begin{tikzpicture}[scale=1,font=\footnotesize,line cap=round,line join=round,>=stealth]
			\def\a{5}
			\path (0:0) coordinate(A)
			(0:\a) coordinate(D) (0:{0*\a}) coordinate(G) ($(A)!.5!(D)$) coordinate(I) ($(G)!.5!(D)$) coordinate(J) (I)+(90:1.2) coordinate(B) (I)+(-90:1.2) coordinate(C) (J)+(90:1.2) coordinate(E) (J)+(-90:1.2) coordinate(F);
			\draw (A)--(B)--(D)--(C)--cycle;
			\foreach \d/\g in {A/180, B/90, C/-90,D/0}
			\fill (\d) circle(1pt) node[shift={(\g:.6)}]{$\d$};
			\path (A)--(B) node[midway, above]{$3$} (B)--(D) node[midway, above]{$4$}  (A)--(C) node[midway, below]{$5$} (C)--(D) node[midway, below]{$6$};
			\draw (A) circle(7pt); \fill[blue](A) circle(7pt);
			\draw (B) circle(7pt); \fill[violet](B) circle(7pt);
			\draw (C) circle(7pt); \fill[cyan](C) circle(7pt);
			\draw (D) circle(7pt); \fill[blue](D) circle(7pt);
		\end{tikzpicture}
	\end{center}
	\loigiai
	{Mỗi cách chọn đường đi từ $A$ đến $D$ rồi trở về $A$ mà không có con đường nào được đi lặp trở lại được thực hiện theo một trong các phương án sau
		\begin{itemize}
			\item Phương án $1$:
			      Đi theo hướng $A\longrightarrow B\longrightarrow D\longrightarrow B \longrightarrow A$.
			      \begin{itemize}
				      \item Công đoạn $1$: Chọn đường đi từ $A$ đến $B$. Có $3$ cách.
				      \item Công đoạn $2$: Chọn đường đi từ $B$ đến $D$. Có $4$ cách.
				      \item Công đoạn $3$: Chọn đường đi từ $D$ trở về $B$ mà không đi lại con đường đã đi qua. Có $3$ cách.
				      \item Công đoạn $4$: Chọn đường đi từ $B$ trở về $A$ mà không đi lại con đường đã đi qua. Có $2$ cách.\\
				            Theo quy tắc nhân, số cách chọn trong phương án này là $3\cdot 4 \cdot 3 \cdot 2= 72$. \qquad (1)
			      \end{itemize}
			\item Phương án $2$:
			      Đi theo hướng $A\longrightarrow B\longrightarrow D\longrightarrow C\longrightarrow A$.
			      \begin{itemize}
				      \item Công đoạn $1$: Chọn đường đi từ $A$ đến $B$. Có $3$ cách.
				      \item Công đoạn $2$: Chọn đường đi từ $B$ đến $D$. Có $4$ cách.
				      \item Công đoạn $3$: Chọn đường đi từ $D$ đến $C$. Có $6$ cách.
				      \item Công đoạn $4$: Chọn đường đi từ $C$ đến $A$. Có $5$ cách.\\
				            Theo quy tắc nhân, số cách chọn trong phương án này là $3\cdot 4 \cdot 6 \cdot 5= 360$. \qquad (2)
			      \end{itemize}
			\item Phương án $3$:
			      Đi theo hướng $A\longrightarrow C\longrightarrow D\longrightarrow B\longrightarrow A$.
			      \begin{itemize}
				      \item Công đoạn $1$: Chọn đường đi từ $A$ đến $C$. Có $5$ cách.
				      \item Công đoạn $2$: Chọn đường đi từ $C$ đến $D$. Có $6$ cách.
				      \item Công đoạn $3$: Chọn đường đi từ $D$ đến $B$. Có $4$ cách.
				      \item Công đoạn $4$: Chọn đường đi từ $B$ đến $A$. Có $3$ cách.\\
				            Theo quy tắc nhân, số cách chọn trong phương án này là $5\cdot 6 \cdot 4 \cdot 3=360$. \qquad (3)
			      \end{itemize}
			\item Phương án $4$:
			      Đi theo hướng $A\longrightarrow C\longrightarrow D\longrightarrow C \longrightarrow A$.
			      \begin{itemize}
				      \item Công đoạn $1$: Chọn đường đi từ $A$ đến $C$. Có $5$ cách.
				      \item Công đoạn $2$: Chọn đường đi từ $C$ đến $D$. Có $6$ cách.
				      \item Công đoạn $3$: Chọn đường đi từ $D$ trở về $cC$ mà không đi lại con đường đã đi qua. Có $5$ cách.
				      \item Công đoạn $4$: Chọn đường đi từ $C$ trở về $A$ mà không đi lại con đường đã đi qua. Có $4$ cách.
			      \end{itemize}
			      Theo quy tắc nhân, số cách chọn trong phương án này là $5\cdot 6 \cdot 5 \cdot 4= 600$. \qquad (4)
		\end{itemize}
		Từ $(1)$, $(2)$, $(3)$ và $(4)$ theo quy tắc cộng, ta có số cách chọn đường đi thỏa yêu cầu đề bài là $$72+360+360+600=1392.$$
	}
\end{vd}
\begin{vd}%[Huỳnh Đức Vũ-Bài giảng Toán 10-đợt 2]%[1D2G1-3]
	Có bao nhiêu cách chọn một vé Xổ số kiến thiết có $5$ chữ số mà số ghi trên vé không có chữ số $0$ hoặc không có chữ số $9$?
	\loigiai{
	Gọi $A$ là tập hợp các vé Xổ số mà số ghi trên vé không có chữ số $0$,
	$B$ là tập hợp các vé Xổ số có $5$ chữ số mà số ghi trên vé không có chữ số $9$
	thì $A \cup B$ là tập hợp các vé số có $5$ chữ số mà số ghi trên vé không có chữ số $0$ hoặc không có chữ số $9$ và $A \cap B$ là tập hợp các vé số có $5$ chữ số mà số ghi trên vé không có chữ số $0$ và không có chữ số $9$.\\
	Vì $A \cap B \neq \varnothing$ nên $n(A \cup B)=n(A)+n(B)-n(A \cap B)$.
	\begin{itemize}
		\item Tìm $n(A)$. \\
		      Số ghi trên vé là một dãy gồm $5$ chữ số $abcde$. Vì số ghi trên vé không có chữ số $0$ nên ở mỗi vị trí có $9$ cách chọn. Suy ra $n(A)=9\cdot 9\cdot 9 \cdot 9 \cdot 9 \cdot 9=9^{5}$.
		\item Tìm $n(B)$.\\
		      Vì số dãy số ghi trên vé không có chữ số 9 và $a$ có thể bằng $0$ nên mỗi vị trí $a, b, c, d, e )$ có có $9$ cách chọn.
		      Do đó, $n(B)=9\cdot 9\cdot 9 \cdot 9 \cdot 9 \cdot 9=9^{5}$.
		\item Tìm $n(A \cap B)$.\\
		      Mỗi cách chọn ra dãy số gồm $5$ chữ số $abcde$ sao cho trong đó không có chữ số $0$ và chữ số $9$ được thực hiện qua $5$ công đoạn, mỗi công đoạn có $8$ cách chọn trong tập $\{1;2;3;4;5;6;7;8\}$. Suy ra $n(A \cap B)=8\cdot 8 \cdot 8 \cdot 8 \cdot 8=8^{5}$.
	\end{itemize}
	Vậy số vé Xổ số thỏa đề bài là $n(A \cup B)=2\cdot 9^{5}-8^{5}=85330$.}
\end{vd}
\begin{vd}%[Huỳnh Đức Vũ-Bài giảng Toán 10-đợt 2]%[1D2G1-3]%ví dụ 4
	Từ tập $E=\{1,2,3,4,5,6,7,8,9\}$ có thể lập được bao nhiêu số tự nhiên chẵn gồm $3$ chữ số đôi một khác nhau và không lớn hơn $789$?
	\loigiai{
		Mỗi cách lập ra số tự nhiên $\overline{abc}$ là số chẵn gồm $3$ chữ số đôi một khác nhau  từ $E$ thỏa $\overline{abcd}\le 789$ được thực hiện theo một trong các phương án sau
		\begin{itemize}
			\item Phương án $1$: $\overline{abc}=\overline{7bc}$ với $b<9$.
			      \begin{itemize}
				      \item Công đoạn $1$: Chọn $c\in\{2 ; 4 ; 6 ; 8\}$. Có $4$ cách.
				      \item Công đoạn $2$: Chọn $b \in E \backslash\{9,7 ; c\}$. Có $6$ cách.
			      \end{itemize}
			      Theo quy tắc nhân, số cách chọn trong phương án này là
			      $4 \cdot 6=24$.\qquad (1)
			\item Phương án $2$: $\overline{a b c}$ với $a<7, c=8$
			      \begin{itemize}
				      \item Công đoạn $1$: Chọn $a\in\{1 ; 2 ; 3 ; 4 ; 5 ; 6\}$. Có $6$ cách.
				      \item Công đoạn $2$: Chọn $b \in E \backslash\{8, a\}$. Có $7$ cách.\\
				            Theo quy tắc nhân, số cách chọn trong phương án này là
				            $6\cdot 7=42$. \qquad (2)
			      \end{itemize}
			\item Phương án $3$: $\overline{a b c}$ với $a<7, c\neq 8$.
			      \begin{itemize}
				      \item Công đoạn $1$: Chọn $c \in\{2 ; 4 ; 6\}$. Có $3$ cách
				      \item Công đoạn $2$: Chọn $a\in E \backslash\{7,8,9, c\}$. Có $5$ cách.
				      \item Công đoạn $3$: Chọn $b \in E \backslash\{a, c\}$. Có $7$ cách.\\
				            Theo quy tắc nhân, số cách chọn trong phương án này là
				            $3\cdot 5 \cdot 7=105$. \qquad (3)
			      \end{itemize}
			      Từ $(1), (2)$, và $(3)$ theo quy tắc cộng, ta có số các số thỏa đề là $24+42+105=171$.
		\end{itemize}}
\end{vd}
\baitaptl
\begin{bt}%[1D2B1-3]
	Từ các chữ số $0$, $1$, $2$, $3$, $4$, $5$, $6$, $7$ có thể lập được bao nhiêu số có bốn chữ số khác nhau trong đó phải có chữ số $2$?
	\loigiai{
		Gọi $n=\overline{a_1a_2a_3a_4}$ là số cần tìm.
		\begin{itemize}
			\item Nếu $a_1=2$ thì $a_2$ có $7$ cách chọn, $a_3$ có $6$ cách chọn, $a_4$ có $5$ cách chọn.\\
			      Suy ra có $5\cdot 6\cdot 7=210$ số.
			\item Nếu $a_1\neq 2$ và $a_2=2$ thì $a_1$ có $6$ cách chọn (vì $a_1\neq 0$), $a_3$ có $6$ cách chọn, $a_4$ có $5$ cách chọn.\\
			      Suy ra có $5\cdot 6\cdot 6=180$ số.\\
			      Tương tự đối với các trường hợp $a_3$, $a_4$ bằng $2$ đều giống trường hợp $a_2=2$.
		\end{itemize}
		Suy ra số các số cần tìm là $210+180\cdot 3=750$ số.
	}
\end{bt}



\begin{bt}%[1D2B1-3]
	Cho các  số $1$, $2$, $3$, $4$, $5$.
	\begin{enumerate}
		\item Hãy tìm tất cả các số có ba chữ số khác nhau nằm trong khoảng $(300; 500)$.
		\item Hãy tìm tất cả các số có ba chữ số  nằm trong khoảng $(300; 500)$ (các chữ số không cần khác nhau).
	\end{enumerate}
	\loigiai
	{Số có ba chữ số có dạng $n=\overline{a_1a_2a_3}$.
		\begin{enumerate}
			\item Ta có $300<n<500$ nên $a_1$ chỉ có thể là $3$ hoặc $4$.
			      \begin{itemize}
				      \item Nếu $a_1=3$ thì $n=\overline{3a_2a_3}$. Khi đó,\\
				            + $a_2$ có $4$ cách chọn.\\
				            + $a_3$ có $3$ cách chọn.\\
				            Do đó, có $4\times 3=12$ số.
				      \item Nếu $a_1=4$ thì $n=\overline{4a_2a_3}$. Khi đó,\\
				            + $a_2$ có $4$ cách chọn.\\
				            + $a_3$ có $3$ cách chọn.\\
				            Do đó, có $4\times 3=12$ số.
			      \end{itemize}
			      Vậy có tất cả $12+12=24$ số.
			\item Ta có $300<n<500$ nên $a_1\in\{3, 4\}$. Kết hợp với các chữ số không cần khác nhau thì
			      \begin{itemize}
				      \item $a_1$ có $2$ cách chọn.
				      \item $a_2$ có $5$ cách chọn.
				      \item $a_3$ có $5$ cách chọn.
			      \end{itemize}
			      Vậy có tất cả $2\times 5\times 5=50$ số.
		\end{enumerate}
	}
\end{bt}

\begin{bt}%[1D2B1-3]
	Từ các chữ số $0$, $4$, $5$, $7$, $9$.
	\begin{enumerate}
		\item Có thể lập được bao nhiêu số có bốn chữ số khác nhau.
		\item Có thể lập được bao nhiêu số có bốn chữ số khác nhau và lớn hơn $5000$?
		\item Có thể lập được bao nhiêu số có bốn chữ số chia hết cho $5$?
	\end{enumerate}
	\loigiai{
		\begin{enumerate}
			\item Gọi số cần tìm là $n=\overline{a_1a_2a_3a_4}$.
			      \begin{itemize}
				      \item $a_1$ có $4$ cách chọn (vì $a_1\neq 0$).
				      \item $a_2$ có $4$ cách chọn.
				      \item $a_3$ có $3$ cách chọn.
				      \item $a_4$ có $2$ cách chọn.
			      \end{itemize}
			      Suy ra có $2\cdot 3\cdot 4\cdot 4=96$ số.
			\item Số lớn hơn $5000$ thì chữ số hàng nghìn $a_1\geq 5$.
			      \begin{itemize}
				      \item Nếu $a_1=5$ thì $n=\overline{5a_2a_3a_4}$.\\
				            Khi đó $a_2$ có $4$ cách chọn, $a_3$ có $3$ cách chọn, $a_4$ có $2$ cách chọn.\\
				            Suy ra có $2\cdot 3\cdot 4=24$ số.
				      \item Nếu $a_1=7$ hoặc $a_1=9$ thì cũng giống trường hợp $a_1=5$
			      \end{itemize}
			      Suy ra có tất cả $24\cdot 3=72$ số lớn hơn $5000$.
			\item Số chia hết cho $5$ phải có chữ số tận cùng là $0$ hoặc $5$ nên $a_4$ có $2$ cách chọn.
			      \begin{itemize}
				      \item Nếu $a_4=0$ thì $n=\overline{a_1a_2a_30}$.\\
				            Khi đó $a_1$ có $4$ cách chọn, $a_2$ có $3$ cách chọn, $a_3$ có $2$ cách chọn.\\
				            Suy ra có $2\cdot 3\cdot 4=24$ số.
				      \item Nếu $a_4=5$ thì $n=\overline{a_1a_2a_35}$.\\
				            Khi đó $a_1$ có ba cách chọn (vì $a_1\neq 0$), $a_2$ có $3$ cách chọn, $a_3$ có hai cách chọn.\\
				            Suy ra có $2\cdot 3\cdot 3=18$ số.
			      \end{itemize}
			      Vậy có tất cả $24+18=42$ số.
		\end{enumerate}
	}
\end{bt}
\begin{bt}%[Huỳnh Đức Vũ-Bài giảng Toán 10-đợt 2]%[1D2K1-3]
	Một lớp học có $3$ tổ. Tổ I gồm có $3$ học sinh nam và $7$ học sinh nữ; tổ II gồm có $5$ học sinh nam và $5$ học sinh nữ; tổ III gồm có $6$ học sinh nam và $4$ học sinh nữ. Cô giáo chủ nhiệm cần chọn ra một học sinh nam và một học sinh nữ để tham gia hoạt động tình nguyện. Hỏi cô giáo có bao nhiêu cách chọn, nếu cô muốn chọn hai em học sinh ở hai tổ khác nhau?
	\loigiai{
		Mỗi cách chọn ra một học sinh nam và học sinh nữ thỏa yêu cầu đề bài được thực hiện theo một trong các phương án sau:
		\begin{itemize}
			\item Phương án $1$:
			      Chọn nam tổ $I$ và nữ ở hai tổ còn lại.
			      \begin{itemize}
				      \item Công đoạn $1$: Chọn $1$ học sinh nam trong tổ $I$. Có $3$ cách.
				      \item Công đoạn $2$: Chọn $1$ học sinh nữ từ hai tổ còn lại. Có $9$ cách.\\
				            Theo quy tắc nhân, số cách trong phương án này là $3\times 9=27$ cách. \qquad(1)
			      \end{itemize}
			\item Phương án $2$: Chọn nam tổ $II$ và nữ ở hai tổ còn lại.\\
			      Tương tự phương án $1$, ta có số cách trong phương án này là $5\times 11=55$ cách. \qquad (2)
			\item Phương án $3$: Chọn nam tổ $III$ và nữ ở hai tổ còn lại. Có $6\times 12=72$ cách. \qquad (3)
		\end{itemize}
		Từ $(1)$, $(2)$ và $(3)$, theo quy tắc cộng, ta có tổng số cách chọn là $27+55+72=154$ cách.
	}
\end{bt}
\begin{bt}%[Huỳnh Đức Vũ-Bài giảng Toán 10-đợt 2]%[1D2K1-3]
	Từ tập $E=\{0;1;2;3;4;5;6\}$ lập được bao nhiêu số tự nhiên gồm $4$ chữ số khác nhau và số tự nhiên này lớn hơn $3452$?
	\loigiai{
		Mỗi cách lập ra số tự nhiên $\overline{abcd}$ gồm $4$ chữ số phân khác nhau từ tập $E$ thỏa $\overline{abcd}> 3452$ được thực hiện theo một trong các phương án sau:
		\begin{itemize}
			\item Phương án $1$:  $\overline{abcd}=\overline{345d}$ với $d>2$.\\
			      Vì $d$ có duy nhất một cách chọn là $d=6$ nên phương án này có $1$ số thỏa mãn.\qquad (1)
			\item Phương án $2$: $\overline{abcd}=\overline{34cd}$ với $c>5$.
			      \begin{itemize}
				      \item Công đoạn $1$: Chọn $c\in E$, $c>5$. Có $1$ cách.
				      \item Công đoạn $2$: Chọn $d \in E \backslash\{3;4;c\}$. Có $4$ cách.\\
				            Theo quy tắc nhân, số cách chọn trong phương án này là
				            $1\cdot 4=4$. \qquad (2)
			      \end{itemize}
			\item Phương án $3$: $\overline{abcd}=\overline{3bcd}$ với $b>4$.
			      \begin{itemize}
				      \item Công đoạn $1$: Chọn $b \in\{5 ;6\}$. Có $2$ cách
				      \item Công đoạn $2$: Chọn $c\in E \backslash\{3;b\}$. Có $5$ cách.
				      \item Công đoạn $3$: Chọn $d \in E \backslash\{3; b, c\}$. Có $4$ cách.\\
				            Theo quy tắc nhân, số cách chọn trong phương án này là
				            $2\cdot 5 \cdot 4=40 $. \qquad (3)
			      \end{itemize}
			\item Phương án $4$: $\overline{abcd}$ với $a>3$.
			      \begin{itemize}
				      \item Công đoạn $1$: Chọn $a \in\{4;5 ;6\}$. Có $3$ cách
				      \item Công đoạn $2$: Chọn $b\in E \backslash\{a\}$. Có $6$ cách.
				      \item Công đoạn $3$: Chọn $c \in E \backslash\{a; b\}$. Có $5$ cách.
				      \item Công đoạn $4$: Chọn $d \in E \backslash\{a; b;c\}$. Có $4$ cách.\\
				            Theo quy tắc nhân, số cách chọn trong phương án này là
				            $3\cdot 6 \cdot 5\cdot 4=360$. \qquad (4)
			      \end{itemize}
			      Từ $(1), (2)$, $(3)$ và $(4)$ theo quy tắc cộng, ta có số các số thỏa đề là $1+4+40+360=405$.
		\end{itemize}}
\end{bt}
\begin{bt}%[Huỳnh Đức Vũ-Bài giảng Toán 10-đợt 2]%[1D2G1-3]
	Từ tập $E=\{0 ; 1 ; 2 ; 3 ; 4 ; 5 ; 6\}$ lập được bao nhiêu số tự nhiên gồm $4$ chữ số đôi một khác nhau  chia hết cho $3$?
	\loigiai{
		Các tập con gồm $4$ phần tử của $E$ mà có tổng các chữ số chia hết cho $3$ là \\
		$\{0 ; 1 ; 2 ; 3\},\{0 ; 1 ; 2 ; 6\},\{0 ; 1 ; 3 ; 5\},\{0 ; 1 ; 5 ; 6\},
			\{0 ; 2 ; 3 ; 4\},\{0 ; 2 ; 4 ; 6\},\{0 ; 3 ; 4 ; 5\},\{0 ; 4 ; 5 ; 6\}, \\
			\{1 ; 2 ; 3 ; 6\},\{1 ; 2 ; 4 ; 5\},\{1 ; 3 ; 4 ; 5\},\{2 ; 3 ; 4 ; 6\},\{3 ; 4 ; 5 ; 6\}$.\\
		Mỗi cách lập ra số tự nhiên $\overline{abcd}$ gồm $4$ chữ số đôi một khác nhau  chia hết cho $3$ được thực hiện theo một trong các phương án sau
		\begin{itemize}
			\item Phương án $1$: Số $\overline{abcd}$ được tạo thành từ một tập con có chữ số $0$.
			      \begin{itemize}
				      \item Công đoạn $1$: Chọn $a\neq 0$. Có $3$ cách.
				      \item  Công đoạn $2$: Chọn $b$, $c$ phân biệt từ $3$ số còn lại. Có $3\cdot 2=6$ cách.\\
				            Theo quy tắc nhân, số các số abcd được tạo thành từ một tập con có chữ số $0$ là $3\cdot 6=18$.
			      \end{itemize}
			      Vì có $8$ tập con chứa số $0$ nên trong phương án này có $8\cdot 18=144$ số. \qquad(1)
			\item Phương án $2$: Số $\overline{abcd}$ được tạo thành từ một tập con không có chữ số $0$.
			      \begin{itemize}
				      \item Công đoan $1$: Chọn $a$. Có $4$ cách.
				      \item Công đoạn $2$: Chọn $b$, $c$ phân biệt từ $3$ số còn lại. Có $3\cdot 2=6$ cách.\\
				            Theo quy tắc nhân, số các số $\overline{abcd}$ được tạo thành từ một tập con không có chữ số $0$ là $4\cdot 6=24$.
			      \end{itemize}
			      Vì có $5$ tập con không chứa số $0$ nên trong phương án này có $5\cdot 24=120$. \qquad (2)
		\end{itemize}
		Từ $(1)$ và $(2)$ theo quy tắc cộng ta có số các số thỏa đề là $144+120=264$.}
\end{bt}
\begin{bt}%[Huỳnh Đức Vũ-Bài giảng Toán 10-đợt 2]%[1D2G1-3]
	Có bao nhiêu cách chọn một vé số có $5$ chữ số mà số ghi trên vé có chữ số $5$ và có số chẵn?
	\loigiai{
		Gọi $x$ là số các vé số gồm $5$ chữ số, còn $y$ là số vé số gồm $5$ chữ số sao cho trong đó không có chữ số $5$ hoặc không có chữ số chẵn thì $x-y$ là số các vé số gồm $5$ chữ số trong đó có có số $5$ và có chữ số chẵn.
		\begin{itemize}
			\item Tìm $x$.\\
			      Mỗi số ghi trên vé số là một dãy số có $5$ chữ số $abcde$, mỗi chữ số có thể bằng $0$ và các chữ số có thể giống nhau nên theo quy tắc nhân, ta có $x=10\cdot 10\cdot 10\cdot 10 \cdot10=10^{5}$.
			\item Tìm $y$.\\
			      Gọi $C$ là tập hợp các vé số có $5$ chữ số mà số ghi trên vé không có chữ số $5$,
			      $D$ là tập hợp các vé số có $5$ chữ số mà số ghi trên vé không có chữ số chẵn
			      thì $C\cup D$ là tập hợp các vé số có $5$ chữ số mà số ghi trên vé không có chữ số $5$ hoặc không có
			      chữ số chẵn và  $C\cap D$ là tập hợp các vé số có $5$ chữ số mà số ghi trên vé không có chữ số $5$ và không có chữ số chẵn (tức là các số ghi trên vé chỉ gồm các số trong tập $\{1;3;7;9\}$).
			      \begin{itemize}
				      \item Áp dụng quy tắc nhân, ta tìm được
				            $$n(C)=9\cdot9\cdot 9 \cdot 9\cdot9=9, n(D)=5\cdot 5 \cdot 5 \cdot 5 \cdot 5=5^{5},
					            n(C \cap D)=4\cdot 4 \cdot 4 \cdot 4 \cdot 4=4^{5}.$$
				      \item Ta có $y=n(C \cup D)=n(C)+n(D)-n(C\cap D)=9^{5}+5^{5}-4^{5}=61150$.
			      \end{itemize}
		\end{itemize}
		Vậy số các vé số thỏa đề bài là $x-y=100000-38850$.}
\end{bt}


\begin{bt}%[BG Toán 10, Nguyễn Tiến]%[1D2B1-2]
	Cho tập $A=\{0;1;2; \ldots ;8;9\}$. Từ $A$ có thể lập được bao nhiêu số tự nhiên gồm bảy chữ số đôi một khác nhau và chia hết cho $2$?
	\dapso{
		$275 520$ số.
	}
	\loigiai{
		Gọi số cần tìm là $\overline{abcdefg}$.
		\begin{itemize}
			\item \textbf{TH1:} $g=0$
			      Do các chữ số đôi một khác nhau nên có $9$ cách chọn $a$; $8$ cách chọn $b$; $7$ cách chọn $c$; $6$ cách chọn $d$; $5$ cách chọn $e$ và $4$ cách chọn $f$.\\
			      Nên có $1\cdot 9\cdot 8\cdot 7\cdot 6\cdot 5\cdot 4=60480$ (số).
			\item \textbf{TH2:} $g\in \{2;4;6;8\}$
			      Do các chữ số đôi một khác nhau nên có $4$ cách chọn $g$; $8$ cách chọn $a\neq 0$; $8$ cách chọn $b$; $7$ cách chọn $c$; $6$ cách chọn $d$; $5$ cách chọn $e$ và $4$ cách chọn $f$.\\
			      Nên có $4\cdot 8\cdot 8\cdot 7\cdot 6\cdot 5\cdot 4=215040$ (số).
		\end{itemize}
		Vậy có $60480+215040=275520$ (số).
	}
\end{bt}

\begin{bt}%[BG Toán 10, Nguyễn Tiến]%[1D2B1-2]
	Có bao nhiêu số tự nhiên trong đó các chữ số khác nhau và nhỏ hơn $10000$ được tạo thành từ năm chữ số $0$, $1$, $2$, $3$, $4$?
	\dapso{
		$165$ số.
	}
	\loigiai{
		Các số cần tìm được bắt đầu từ các chữ số $1$, $2$, $3$, $4$ và có bốn, ba, hai, một chữ số.
		\begin{itemize}
			\item Số cần tìm có bốn chữ số là $\overline{abcd}$.\\
			      Do các chữ số khác nhau nên có $4$ cách chọn $a\neq 0$; $4$ cách chọn $b$; $3$ cách chọn $c$ và $2$ cách chọn $d$.\\
			      Nên có $4\cdot 4\cdot 3\cdot 2=96$ (số).
			\item Số cần tìm có ba chữ số là $\overline{abc}$.\\
			      Do các chữ số khác nhau nên có $4$ cách chọn $a\neq 0$; $4$ cách chọn $b$ và $3$ cách chọn $c$.\\
			      Nên có $4\cdot 4\cdot 3=48$ (số).
			\item Số cần tìm có hai chữ số là $\overline{ab}$.\\
			      Do các chữ số khác nhau nên có $4\neq 0$ cách chọn $a$ và $4$ cách chọn $b$.\\
			      Nên có $4\cdot 4=16$ (số).
			\item Số cần tìm có một chữ số: $5$ (số).
		\end{itemize}
		Vậy có $96+48+16+5=165$ (số).
	}
\end{bt}

\begin{bt}%[BG Toán 10, Nguyễn Tiến]%[1D2K1-2]
	Từ các số $0;1;2;3;4;5$ có thể lập được bao nhiêu số tự nhiên có năm chữ số khác nhau và không bắt đầu bằng $123$?
	\dapso{
		$594$ số.
	}
	\loigiai{
		\begin{itemize}
			\item Gọi số tự nhiên có năm chữ số khác nhau có dạng $\overline{abcde}$.\\
			      Ta có $5$ cách chọn $a\neq 0$; $5$ cách chọn $b$; $4$ cách chọn $c$; $3$ cách chọn $d$ và $2$ cách chọn $e$.\\
			      Nên có $5\cdot 5\cdot 4\cdot 3\cdot 2=600$ (số).
			\item Gọi số tự nhiên có năm chữ số khác nhau và bắt đầu bằng $123$ có dạng $\overline{123 b_1 b_2}$.\\
			      Ta có $3$ cách chọn $b_1$ và $2$ cách chọn $b_2$.\\
			      Nên có $3\cdot 2=6$ (số).
		\end{itemize}
		Vậy có $600-6=594$ số tự nhiên có năm chữ số khác nhau và không bắt đầu bằng $123$.
	}
\end{bt}

\begin{bt}%[BG Toán 10, Nguyễn Tiến]%[1D2K1-2]
	Cho tập hợp $A=\{0;1;2;3;4;5\}$.
	\begin{enumerate}
		\item Có bao nhiêu số tự nhiên gồm năm chữ số đôi một khác nhau, chia hết cho $5$ và chữ số $2$ luôn có mặt đúng một lần?
		\item Có bao nhiêu số tự nhiên gồm ba chữ số đôi một khác nhau và chia hết cho $3$?
		\item Tính tổng các số tự nhiên có năm chữ số đôi một khác nhau mà các số này không có chữ số $0$?
	\end{enumerate}
	\dapso{
		$174$ số; $40$ số; $3 999 960$.
	}
	\loigiai{
		\begin{enumerate}
			\item Gọi số cần tìm là $\overline{abcde}$.
			      \begin{itemize}
				      \item \textbf{Trường hợp 1:} $e=0$.
				            \begin{itemize}
					            \item Ta có $1$ cách chọn $e$.
					            \item Chữ số $2$ có $4$ vị trí đặt là $a$ hoặc $b$ hoặc $c$ hoặc $d$.
					            \item Ba chữ số còn lại có $4\cdot 3\cdot 2=24$ (cách).
				            \end{itemize}
				            Nên có $1\cdot 4\cdot 24=96$ (số).
				      \item \textbf{Trường hợp 2:} $e=5$, $a=2$.
				            Ta có $1$ cách chọn $e$, $1$ cách chọn $a$.\\
				            Do các chữ số đôi một khác nhau nên có $4$ cách chọn $b$, $3$ cách chọn $c$ và $2$ cách chọn $d$.\\
				            Nên có $1\cdot 1\cdot 4\cdot 3\cdot 2=24$ (số).
				      \item \textbf{Trường hợp 3:} $e=5$, $a\neq 2$.
				            \begin{itemize}
					            \item Ta có $1$ cách chọn $e$, $3$ cách chọn $a\neq 0$.
					            \item Chữ số $2$ có $3$ vị trí đặt là $b$ hoặc $c$ hoặc $d$.
					            \item Hai chữ số còn lại có $3\cdot 2=6$ (cách).
				            \end{itemize}
				            Nên có $1\cdot 3\cdot 3\cdot 6=54$ (số).
			      \end{itemize}
			      Vậy có $96+24+54=174$ (số).
			\item Gọi số cần tìm là $\overline{abc}$.\\
			      Xét các tập con gồm $3$ phần tử của tập hợp $A$, ta thấy các tập hợp sau có tổng các phần tử là số chia hết cho $3$ là
			      \allowdisplaybreaks
			      \begin{eqnarray*}
				      & & A_1=\{0;1;2\}, \, A_2=\{0;1;5\}, \, A_3=\{0;2;4\}, \, A_4=\{0;4;5\},\\
				      & & A_5=\{1;2;3\}, \, A_6=\{1;3;5\}, \, A_7=\{2;3;4\}, \, A_8=\{3;4;5\}.
			      \end{eqnarray*}
			      \begin{itemize}
				      \item Khi $a,b,c,\in A_1, A_2, A_3, A_4$: mỗi trường hợp có $2$ cách chọn $a\neq 0$, $2$ cách chọn $b$ và $1$ cách chọn $c$.
				            Nên có $4\cdot (2\cdot 2\cdot 1)=16$ (số).
				      \item Khi $a,b,c,\in A_5, A_6, A_7, A_8$: mỗi trường hợp có $3$ cách chọn $a$, $2$ cách chọn $b$ và $1$ cách chọn $c$.
				            Nên có $4\cdot (3\cdot 2\cdot 1)=24$ (số).
			      \end{itemize}
			      Vậy có $16+24=40$ (số).
			\item Gọi số cần tìm là $\overline{abcde}$.\\
			      Do các chữ số đôi một khác nhau mà các số này không có chữ số $0$ nên có $5$ cách chọn $a$, $4$ cách chọn $b$, $3$ cách chọn $c$, $2$ cách chọn $d$ và $1$ cách chọn $e$.\\
			      Nên có $5\cdot 4\cdot 3\cdot 2\cdot 1=120$ số thỏa mãn yêu cầu bài toán.\\
			      Gọi $S$ là tổng của $120$ số tự nhiên có $5$ chữ số khác nhau vừa tìm được.\\
			      Mỗi chữ số $1$, $2$, $3$, $4$, $5$ đều xuất hiện ở $a$, $b$, $c$, $d$, $e$ là $24$ lần.\\
			      Mà $1+2+3+4+5=15$ nên
			      $$S=24\cdot \left(15\cdot 10^4+15\cdot 10^3+15\cdot 10^2+15\cdot 10+15\right)=3999960.$$
		\end{enumerate}
	}
\end{bt}
\subsubsection{Bài tập trắc nghiệm}
\Opensolutionfile{ansbook}[ans/ansbook-0D8-1-3]
\Opensolutionfile{ans}[ans/ans-0D8-1-3]
\setcounter{ex}{0}
\begin{ex}%[1D2B1-2]
	Có bao nhiêu số tự nhiên có $ 6  $ chữ số khác nhau?
	\choice
	{\True $ 136080 $}
	{$ 136800 $}
	{$ 1360800 $}
	{$ 138060 $}
	\loigiai{
		Số số tự nhiên có $ 6 $ chữ số khác nhau là $ 9 \times 9 \times 8\times 7 \times 6 \times  5 = 136080 $.
	}
\end{ex}
\begin{ex}%[1D2B1-2]
	Bạn Anh muốn qua nhà bạn Bình để rủ Bình đến nhà bạn Châu chơi. Từ nhà Anh đến nhà Bình có $3$ con đường. Từ nhà Bình đến nhà Châu có $5$ con đường. Hỏi bạn Anh có bao nhiêu cách chọn đường đi từ nhà mình đến nhà bạn Châu?
	\choice
	{$6$}
	{\True $15$}
	{$4$}
	{$8$}
	\loigiai{
		Có $3$ cách chọn một đường đi từ nhà Anh đến nhà Bình và có $5$ cách chọn một đường đi từ nhà Bình đến nhà Châu. Do đó có $3 \cdot 5 = 15$ cách để chọn một đường đi từ nhà Anh đến nhà Châu.
	}
\end{ex}

\begin{ex}%[1D2Y1-3]
	Bạn Mai có ba cái áo màu khác nhau và hai quần kiểu khác nhau. Hỏi Mai có bao nhiêu cách chọn một bộ quần áo?
	\choice
	{$10$}
	{$ 20 $}
	{\True $ 6 $}
	{$ 5 $}
	\loigiai{Chọn một cái áo trong ba cái áo màu khác nhau, số cách chọn là $ 3 $.\\
		Chọn một cái quần trong hai quần kiểu khác nhau, số cách chọn là $ 2 $.\\
		Theo quy tắc nhân, số cách chọn một bộ quần áo là $ 3\cdot 2 =6$.}
\end{ex}
\begin{ex}%[1D2B1-3]
	Từ các chữ số $1$, $2$, $3$, $4$, $5$ có thể lập được bao nhiêu số tự nhiên bé hơn $60$?
	\choice
	{\True $30$}
	{$17$}
	{$25$}
	{$42$}
	\loigiai{
		\begin{itemize}
			\item Số cần tìm có $1$ chữ số $\Rightarrow$ có $5$ số thỏa mãn yêu cầu.
			\item Số cần tìm có $2$ chữ số $\Rightarrow$ có $5\cdot 5=25$ số thỏa mãn yêu cầu.
		\end{itemize}
		Vậy có $5+25=30$ (số thỏa mãn yêu cầu).
	}
\end{ex}
\begin{ex}%[1D2K1-3]
	Từ các số của tập hợp $\{0;1;2;3;4;5\}$ lập được bao nhiêu số tự nhiên chẵn có ít nhất $5$ chữ số và các chữ số đôi một phân biệt?
	\choice
	{\True $624$}
	{$522$}
	{$312$}
	{$405$}
	\loigiai
	{Theo đề bài ta cần tìm số các số tự nhiên chẵn có $6$ chữ số và $5$ chữ số đôi một phân biệt từ tập hợp đã cho.
		\begin{enumerate}
			\item Số tự nhiên có $6$ chữ số có dạng $n=\overline{abcdef}$.
			      \begin{itemize}
				      \item Nếu $f=0$ thì mỗi cách chọn chữ số cho các vị trí $a$, $b$, $c$, $d$, $e$ là một hoán vị của $5$ phần tử $1$, $2$, $3$, $4$, $5$. Do đó có $5!$ số.
				      \item Nếu $f\in\{2;4\}$ thì $a\neq 0$ nên $a$ có $4$ cách chọn và  mỗi cách chọn chữ số cho các vị trí $b$, $c$, $d$, $e$ là một hoán vị của $4$ phần tử còn lại. Do đó có $2\times 4\times 4!$ số.\\
				            Vậy có tất cả $5!+2\times 4\times 4!=312$ số chẵn có $6$ chữ số đôi một phân biệt.
			      \end{itemize}
			\item Số tự nhiên có $5$ chữ số có dạng $n=\overline{abcde}$.
			      \begin{itemize}
				      \item Nếu $e=0$ thì mỗi cách chọn chữ số cho các vị trí $a$, $b$, $c$, $d$ là một chỉnh hợp chập $4$ của $5$ phần tử $1$, $2$, $3$, $4$, $5$. Do đó có $\mathrm{A}_5^4$ số.
				      \item Nếu $e\in\{2;4\}$ thì $a\neq 0$ nên $a$ có $4$ cách chọn và  mỗi cách chọn chữ số cho các vị trí $b$, $c$, $d$ là một chỉnh hợp chập $3$ của $4$ phần tử còn lại. Do đó có $2\times 4\times\mathrm{A}_4^3$ số.\\
				            Như thế có tất cả $\mathrm{A}_5^4+2\times 4\times\mathrm{A}_4^3=312$ số chẵn có $5$ chữ số đôi một phân biệt.
			      \end{itemize}
		\end{enumerate}
		Vậy có tất cả $312+312=624$ số tự nhiên chẵn có ít nhất $5$ chữ số và các chữ số đôi một phân biệt.
	}
\end{ex}
\begin{ex}%[1D2B1-3]
	Cho tập $A = \{0; 1; 2; 3; 4; 5; 6 \}$, từ tập $A$ có thể lập được bao nhiêu số tự nhiên có $5$ chữ số khác nhau và chia hết cho $2$?
	\choice
	{$1230$}
	{$2880$}
	{\True $1260$}
	{$8232$}
	\loigiai{
		Gọi các số thỏa mãn bài toán có dạng $\overline{a_1a_2a_3a_4a_5}$, với $a_1,a_2,a_3,a_4,a_5 \in A$. \\
		\textbf{Trường hợp 1}: $a_5 = 0$.
		\begin{itemize}
			\item Vị trí $a_1$ có $6$ cách chọn từ tập $A \setminus \{ 0 \} $.
			\item Vị trí $a_2$ có $5$ cách chọn từ tập $A \setminus \{0; a_1\}$.
			\item Vị trí $a_3$ có $4$ cách chọn từ tập $A \setminus \{0; a_1; a_2\}$.
			\item Vị trí $a_4$ có $3$ cách chọn từ tập $A \setminus \{0; a_1; a_2; a_3\}$.
		\end{itemize}
		Theo quy tắc nhân, số các số thỏa mãn bài toán trong trường hợp này là $6 \cdot 5 \cdot 4 \cdot 3 = 360$ số.\\
		\textbf{Trường hợp 2}: $a_5 \neq 0$.
		\begin{itemize}
			\item Vì số cần tìm chia hết cho $2$ nên $a_5$ có $3$ cách chọn từ tập $\{ 2; 4; 6 \}$.
			\item 	Vị trí $a_1$ có $5$ cách chọn từ tập $A \setminus \{ 0; a_5 \} $.
			\item 	Vị trí $a_2$ có $5$ cách chọn từ tập $A \setminus \{a_5; a_1\}$.
			\item 	Vị trí $a_3$ có $4$ cách chọn từ tập $A \setminus \{a_5; a_1; a_2\}$.
			\item 	Vị trí $a_4$ có $3$ cách chọn từ tập $A \setminus \{a_5; a_1; a_2; a_3\}$.
		\end{itemize}
		Theo quy tắc nhân, số các số thỏa mãn bài toán trong trường hợp này là $3 \cdot 5 \cdot 5 \cdot 4 \cdot 3 = 900$ số.\\
		Theo quy tắc cộng, số các số thỏa mãn bài toán là $360 + 900 = 1260$ số.
	}
\end{ex}

\begin{ex}%[1D2B1-3]
	Cho các chữ số $0$, $1$, $2$, $3$, $4$, $5$. Từ các chữ số đã cho lập được bao nhiêu số tự nhiên chẵn có $4$ chữ số và các chữ số đôi một bất kỳ khác nhau?
	\choice
	{$160$}
	{\True $156$}
	{$752$}
	{$240$}
	\loigiai{
		Gọi các số thỏa mãn bài toán có dạng $\overline{a_1a_2a_3a_4}$, với $a_1,a_2,a_3,a_4 \in A = \{0; 1; 2; 3; 4; 5 \}$.

		\textbf{Trường hợp 1}: $a_4 = 0$. \\
		Vị trí $a_1$ có $5$ cách chọn từ tập $A \setminus \{ 0 \} $. \\
		Vị trí $a_2$ có $4$ cách chọn từ tập $A \setminus \{0; a_1\}$. \\
		Vị trí $a_3$ có $3$ cách chọn từ tập $A \setminus \{0; a_1; a_2\}$. \\
		Theo quy tắc nhân, số các số thỏa mãn bài toán trong trường hợp này là $5 \cdot 4 \cdot 3 = 60$ số.

		\textbf{Trường hợp 2}: $a_4 \neq 0$. \\
		Vì số cần tìm là số chẵn nên $a_4$ có $2$ cách chọn từ tập $\{ 2; 4 \}$. \\
		Vị trí $a_1$ có $4$ cách chọn từ tập $A \setminus \{ 0; a_4 \} $. \\
		Vị trí $a_2$ có $4$ cách chọn từ tập $A \setminus \{a_4; a_1\}$. \\
		Vị trí $a_3$ có $3$ cách chọn từ tập $A \setminus \{a_4; a_1; a_2\}$. \\
		Theo quy tắc nhân, số các số thỏa mãn bài toán trong trường hợp này là $2 \cdot 4 \cdot 4 \cdot 3 = 96$ số.

		Theo quy tắc cộng, số các số thỏa mãn bài toán là $60 + 96 = 156$ số.
	}
\end{ex}

\begin{ex}%[1D2K1-3]
	Từ các chữ số $0$, $1$, $2$, $3$, $5$, $8$ có thể lập được bao nhiêu số tự nhiên lẻ có bốn chữ số đôi một khác nhau và phải có mặt chữ số $ 3 $?
	\choice
	{\True $108$}
	{$228$}
	{$36$}
	{$144$}
	\loigiai{
		Gọi các số thỏa mãn bài toán có dạng $\overline{a_1a_2a_3a_4}$, với $a_1,a_2,a_3,a_4 \in A = \{0; 1; 2; 3; 5; 8 \}$.

		\textbf{Trường hợp 1}: $a_4 = 3$. \\
		Vị trí $a_1$ có $4$ cách chọn từ tập $A \setminus \{ 0; 3 \} $. \\
		Vị trí $a_2$ có $4$ cách chọn từ tập $A \setminus \{ a_1; 3 \} $. \\
		Vị trí $a_3$ có $3$ cách chọn từ tập $A \setminus \{ a_1; a_2; 3 \} $. \\
		Theo quy tắc nhân, số các số thỏa mãn bài toán trong trường hợp này là $4 \cdot 4 \cdot 3 = 48$ số.

		\textbf{Trường hợp 2}: $a_1 = 3$. \\
		Vì số cần tìm là số lẻ nên $a_4$ có $2$ cách chọn từ tập $\{ 1; 5 \}$. \\
		Vị trí $a_2$ có $4$ cách chọn từ tập $A \setminus \{ 3; a_4 \} $. \\
		Vị trí $a_3$ có $3$ cách chọn từ tập $A \setminus \{ 3; a_4; a_2 \} $. \\
		Theo quy tắc nhân, số các số thỏa mãn bài toán trong trường hợp này là $2 \cdot 4 \cdot 3 = 24$ số.

		\textbf{Trường hợp 3}: $a_1 \neq 3$ và $a_4 \neq 3$. \\
		Vì số cần tìm là số lẻ nên $a_4$ có $2$ cách chọn từ tập $\{ 1; 5 \}$. \\
		Vị trí $a_1$ có $3$ cách chọn từ tập $A \setminus \{ 0; 3; a_4 \} $. \\
		Chọn 1 vị trí để đặt số $3$, có $2$ cách (vị trí $a_2$, $a_3$). \\
		Vị trí cuối cùng có $3$ cách chọn từ tập $A \setminus \{a_4; a_1; 3 \}$. \\
		Theo quy tắc nhân, số các số thỏa mãn bài toán trong trường hợp này là $2 \cdot 3 \cdot 2 \cdot 3 = 36$ số.

		Theo quy tắc cộng, số các số thỏa mãn bài toán là $48 + 24 + 36 = 108$ số.
	}
\end{ex}

\begin{ex}%[1D2K1-3]
	Từ các chữ số $1$, $2$, $3$, $4$, $5$, $6$ có thể lập được bao nhiêu số tự nhiên chẵn có sáu chữ số và thỏa mãn điều kiện: sáu chữ số của mỗi số là khác nhau và chữ số hàng nghìn lớn hơn $2$?
	\choice
	{$720$}
	{$360$}
	{$288$}
	{\True $240$}
	\loigiai{
		Gọi các số thỏa mãn bài toán có dạng $\overline{a_1a_2a_3a_4a_5a_6}$, với $a_1, a_2, a_3, a_4, a_5, a_6 \in A = \{1; 2; 3; 4; 5; 6 \}$.
		\\
		Vì số cần tìm có hàng nghìn lớn hơn $2$ nên $a_3 \geq 3$.

		\textbf{Trường hợp 1}: $a_3$ là số lẻ. \\
		Vị trí $a_3$ có $2$ cách chọn từ tập $\{ 3; 5\} $. \\
		Vì số cần tìm là số chẵn nên $a_6$ có $3$ cách chọn từ tập $\{ 2; 4; 6 \}$. \\
		Vị trí $a_1$ có $4$ cách chọn từ tập $A \setminus \{ a_3; a_6 \} $. \\
		Vị trí $a_2$ có $3$ cách chọn từ tập $A \setminus \{ a_3; a_6; a_1 \} $. \\
		Vị trí $a_4$ có $2$ cách chọn từ tập $A \setminus \{ a_3; a_6; a_1; a_2 \} $. \\
		Vị trí $a_5$ có $1$ cách chọn từ tập $A \setminus \{ a_3; a_6; a_1; a_2; a_3 \} $. \\
		Theo quy tắc nhân, số các số thỏa mãn bài toán trong trường hợp này là $2 \cdot 3 \cdot 4 \cdot 3 \cdot 2 \cdot 1 = 144$ số.

		\textbf{Trường hợp 2}: $a_3$ là số chẵn. \\
		Vị trí $a_3$ có $2$ cách chọn từ tập $\{ 4; 6\} $. \\
		Vì số cần tìm là số chẵn nên $a_6$ có $2$ cách chọn từ tập $\{2; 4; 6\} \setminus \{ a_3 \}$. \\
		Vị trí $a_1$ có $4$ cách chọn từ tập $A \setminus \{ a_3; a_6 \} $. \\
		Vị trí $a_2$ có $3$ cách chọn từ tập $A \setminus \{ a_3; a_6; a_1 \} $. \\
		Vị trí $a_4$ có $2$ cách chọn từ tập $A \setminus \{ a_3; a_6; a_1; a_2 \} $. \\
		Vị trí $a_5$ có $1$ cách chọn từ tập $A \setminus \{ a_3; a_6; a_1; a_2; a_3 \} $. \\
		Theo quy tắc nhân, số các số thỏa mãn bài toán trong trường hợp này là $2 \cdot 2 \cdot 4 \cdot 3 \cdot 2 \cdot 1 = 96$ số.

		Theo quy tắc cộng, số các số thỏa mãn bài toán là $144 + 96 = 240$ số.
	}
\end{ex}

\begin{ex}%[1D2B1-3]
	Xét mạng đường nối các tỉnh $ A$, $B$, $C$, $D$, $E$, $F$, $G $, trong đó số viết trên một cạnh cho biết số con đường nối hai tỉnh nằm ở hai đầu mút của cạnh. Số cách đi từ tỉnh $ A $ đến tỉnh $ G $ là
	\choice
	{$23$}
	{$252$}
	{\True $2880$}
	{$522$}
	\loigiai{
		\begin{itemize}
			\item Đi từ $ A $ đến $ D $.
			      \begin{itemize}
				      \item Đi có qua $ B $ có $ 2\times 3=6 $ cách.
				      \item Đi có qua $ C $ có $ 3\times 4=12 $ cách.
			      \end{itemize}
			      Theo quy tắc cộng có $ 6+12=18 $ cách đi từ $ A $ đến $ D $.
			\item Đi từ $ D $ đến $ G $.
			      \begin{itemize}
				      \item Đi có qua $ E $ có $ 2\times 5=10 $ cách.
				      \item Đi có qua $ F $ có $ 2\times 2=4 $ cách.
			      \end{itemize}
			      Theo quy tắc cộng có $ 10+4=14 $ cách đi từ $ D $ đến $ G $.
		\end{itemize}
		Theo quy tắc nhân có $ 18\times 14=252 $ cách đi từ $ A $ đến $ G $.
	}
\end{ex}
\begin{ex}%[1D2K1-3]
	Từ các chữ số $0,1,2,3,4,5,6$ có thể lập được bao nhiêu số tự nhiên chẵn có $3$ chữ số đôi một khác nhau?
	\choice
	{$168$}
	{$210$}
	{$84$}
	{\True $105$}
	\loigiai{
		Gọi số tự nhiên cần tìm là $ n = \overline{abc} $ với $ a\ne 0 $.
		\begin{enumerate}
			\item Trường hợp 1. Xét $ n = \overline{ab0} $.
			      \begin{itemize}
				      \item $ a $ có $ 6 $ cách chọn vì $ a\ne 0 $.
				      \item $ b $ có $ 5 $ cách chọn vì $ b\ne 0,b\ne a $.
			      \end{itemize}
			      Theo quy tắc nhân ta có $ 6\times 5 = 30 $ số cần tìm.
			\item Trường hợp 2. Xét $ n = \overline{abc} $ với $ c\in \{2;4;6\} $.
			      \begin{itemize}
				      \item $ c $ có $ 3 $ cách chọn.
				      \item $ a $ có $ 5 $ cách chọn vì $ a\ne 0,a\ne c $.
				      \item $ b $ có $ 5 $ cách chọn vì $ b\ne a,b\ne c $.
			      \end{itemize}
			      Theo quy tắc nhân ta có $ 3\times 5\times 5 = 75 $ số cần tìm.
		\end{enumerate}
		Theo quy tắc cộng ta có $ 30+75 = 105 $ số cần tìm.
	}
\end{ex}
\begin{ex}%[1D2B1-3]
	Một hộp đựng $9$ thẻ được đánh số từ $1$ đến $9$. Có bao nhiêu cách chọn hai thẻ sao cho tích hai số trên hai thẻ là số chẵn?
	\choice
	{$32$}
	{$36$}
	{\True $26$}
	{$72$}
	\loigiai{
		Trong $ 9 $ thẻ có $ 4 $ số chẵn và $ 5 $ số lẻ. Ta có các trường hợp sau:
		\begin{itemize}
			\item Cả 2 thẻ đều là số chẵn thì có $ \dfrac{4\times3}{2} = 6 $ cách.
			\item $ 1 $ thẻ là số chẵn, 1 thẻ là số lẻ thì có $ 4\times 5 = 20 $ cách.
		\end{itemize}
		Theo quy tắc cộng ta có $ 6 + 20 = 26 $ cách.
	}
\end{ex}
\begin{ex}%[1D2K2-2]
	Từ tập hợp $A=\{0;1;2;3;4;5;6\}$ lập được bao nhiêu số tự nhiên chia hết cho $5$, gồm năm chữ số khác nhau sao cho trong đó luôn có mặt các chữ số $1$, $2$, $3$ và chúng đứng cạnh nhau ?
	\choice
	{$46$}
	{\True $66$}
	{$52$}
	{$44$}
	\loigiai{
		\begin{itemize}
			\item Trường hợp 1: Số cần tìm có dạng $\overline{123de}$.\\
			      + Chọn $e\in \{0;5\}$ có $2$ cách chọn.\\
			      + Chọn $d\in \{0;4;6;5\}\setminus \{e\}$ có $3$ cách chọn.\\
			      + Có $2\cdot 3\cdot 6=36$ số cần tìm.
			\item Trường hợp 2: Số cần tìm có dạng $\overline{a123e}$.\\
			      + Chọn $e=0$, $a=5$, trường hợp này có $1\cdot 6\cdot 1=6$ số.\\
			      + Chọn $e\in \{0;5\}$, $a\in \{6;4\}$, trường hợp này có $2\cdot 6\cdot 2=24$ số.\\
			      Vậy trường hợp này có $6+24=30$ số.
		\end{itemize}
		Số các số cần tìm là $36+30=66$ số.}
\end{ex}

\begin{ex}%[1D2B1-3]
	Cho tập hợp $A=\{0;1;2;3;4;5\}$. Có thể lập bao nhiêu số tự nhiên có $3$ chữ số khác nhau và chia hết cho $5$?
	\choice
	{$42$}
	{$40$}
	{$38$}
	{\True $36$}
	\loigiai{
		Số tự nhiên $x$ có dạng $\overline{abc}$ với $a,\, b,\, c\in A$ và đôi một phân biệt.\\
		Vì số tạo ra chia hết cho $5$ nên $c\in \{0;5\}$.\\
		Với $c=0,\, b$ có $5$ cách chọn, $a$ có $4$ cách chọn nên $5 \times 4=20$ số cần tìm.\\
		Với $c=5$, số số $\overline{ab}$ thỏa mãn tiếp theo là $5 \times 4-4=16$.\\
		Vậy có tất cả $20+16=36$ số.}
\end{ex}
\begin{ex}%[1D2K1-3]
	Có bao nhiêu số tự nhiên chẵn gồm hai chữ số khác nhau được lập từ các chữ số $0, 1, 2, 3, 4, 5$?
	\choice
	{$5$}
	{$15$}
	{\True $13$}
	{$22$}
	\loigiai{
		Số tự nhiên thỏa mãn có dạng $\overline{ab}$. Vì cần số chẵn nên $b\in \{0;2;4\}$.\\
		Với $b=0 \Rightarrow a\in \{1;2;3;4;5\} \Rightarrow $ $5$ số.\\
		Với $b\ne 0 \Rightarrow b$ có $2$ cách chọn là $2, 4$; $a$ có $4$ cách chọn.\\
		Khi đó số các số cần tìm là $2 \times 4=8$ số.\\
		Vậy có tất cả $8 + 5 = 13$ số.}
\end{ex}
\begin{ex}%[1D2B1-3]
	Từ các chữ số $1,\, 2,\, 3,\, 4,\, 5,\, 6$ có thể lập được bao nhiêu chữ số tự nhiên bé hơn $100$?
	\choice
	{$36$}
	{$62$}
	{$54$}
	{\True $42$}
	\loigiai{
		Các số bé hơn $100$ chính là các số có một chữ số và hai chữ số được hình thành từ tập $A=\{1,\, 2,\, 3,\, 4,\, 5,\, 6\}$. Từ tập $A$ có thể lập được $6$ số có một chữ số.\\
		Gọi số có hai chữ số có dạng $\overline{ab}$ với $a,\, b\in A$. \\
		Trong đó:\\
		$\bullet $ $a$ được chọn từ tập $A$ (có $6$ phần tử) nên có $6$ cách chọn\\
		$\bullet $ $b$ được chọn từ tập $A$ (có $6$ phần tử) nên có $6$ cách chọn.\\
		Như vậy, ta có $6\times 6=36$ số có hai chữ số.\\
		Vậy, từ $A$ có thể lập được $36+6=42$ số tự nhiên bé hơn $100$.}
\end{ex}
\begin{ex}%[1D2B1-3]
	Từ các chữ số $0,\, 1,\, 2,\, 3,\, 4,\, 5$ có thể lập được bao nhiêu số chẵn gồm $4$ chữ số khác nhau?
	\choice
	{\True $156$}
	{$144$}
	{$96$}
	{$134$}
	\loigiai{
		Đặt $A=\{0,\, 1,\, 2,\, 3,\, 4,\, 5\}$. Gọi số cần tìm có dạng $\overline{abcd}$ với $a,\, b,\, c,\, d\in A$. \\
		Vì $\overline{abcd}$ là số chẵn $ \Rightarrow d=\{0,\, 2,\, 4\}$. \\
		TH1. Nếu $d=0$,\, số cần tìm là $\overline{abc0}$. Khi đó:\\
		$\bullet $ $a$ được chọn từ tập $A\setminus\{0\}$ nên có $5$ cách chọn\\
		$\bullet $ $b$ được chọn từ tập $A\setminus\{0,\, a\}$ nên có $4$ cách chọn.\\
		$\bullet $ $c$ được chọn từ tập $A\setminus\{0,\, a,\, b\}$ nên có $3$ cách chọn.\\
		Như vậy, ta có $5\times 4\times 3=60$ số có dạng $\overline{abc0}$. \\
		TH2. Nếu $d=\{2,\, 4\} \Rightarrow d\colon $ có $2$ cách chọn.\\
		Khi đó $ a $ có $4$ cách chọn (khác $0$ và $d$), $b $ có $4$ cách chọn và $c $ có $3$ cách chọn.\\
		Như vậy, ta có $2\times 4\times 4\times 3=96$ số cần tìm như trên.\\
		Vậy có tất cả $60+96=156$ số cần tìm.}
\end{ex}

\begin{ex}%[1D2K1-3]
	Cho tập $A=\left\{0;1;2;3;4;5;6\right\}$. Từ tập $A$ có thể lập được bao nhiêu số tự nhiên gồm $5$ chữ số và chia hết cho $5$.
	\choice
	{\True $600$}
	{$432$}
	{$679$}
	{$523$}
	\loigiai{
		Gọi $x=\overline{abcde}$ là số cần lập, $e\in\left\{0;5\right\}, a\ne 0$.
		\begin{itemize}
			\item $e=0\Rightarrow e$ có $1$ cách chọn, cách chọn $a,b,c,d$ tương ứng là $6,5,4,3$.\\
			      Trường hợp này có $6\times5\times4\times3=360$ số.
			\item $e=5\Rightarrow e$ có $1$ cách chọn, cách chọn $a,b,c,d$ tương ứng là $5,5,4,3$.\\
			      Trường hợp này có $5\times5\times4\times3=300$ số.
		\end{itemize}
		Vậy có $660$ số thỏa mãn yêu cầu bài toán.}
\end{ex}

\begin{ex}%[1D2B1-3]
	Từ thành phố $A$ đến thành phố $B$ có  $3$ con đường, từ thành phố $A$ đến thành phố $C$ có $2$ con đường, từ thành phố $B$ đến thành phố $D$ có $2$ con đường, từ thành phố $C$ đến thành phố $D$ có $3$ con đường, không có con đường nào nối từ thành phố $C$ đến thành phố $B$. Hỏi có bao nhiêu con đường đi từ thành phố $A$ đến thành phố $D$.
	\choice
	{$6$}
	{\True $12$}
	{$18$}
	{$36$}
	\loigiai{
		\immini{
			Số cách đi từ $A$ đến $D$ bằng cách đi từ $A$ đến $B$ rồi đến $D$ là $3\times2=6$.\\
			Số cách đi từ $A$ đến $D$ bằng cách đi từ $A$ đến $C$ rồi đến $D$ là $2\times3=6$.\\
			Nên có: $6+6=12$ cách.
		}{
			\begin{tikzpicture}[>=stealth,scale=1, line join = round, line cap = round]
				\tikzset{label style/.style={font=\footnotesize}}
				\tkzDefPoints{0/0/A,5/0/D,3/1/B,2/-1/C}
				\tkzDrawSegments(A,B B,D C,D A,C)
				\tkzDrawPoints[fill=black](A,B,C,D)
				\tkzLabelPoints[left](A)
				\tkzLabelPoints[right](D)
				\tkzLabelPoints[below](C)
				\tkzLabelPoints[above](B)
				\node at ($(A)!1/2!(B)$)[above]{$3$};
				\node at ($(A)!1/2!(C)$)[above]{$2$};
				\node at ($(D)!1/2!(C)$)[above]{$3$};
				\node at ($(B)!1/2!(D)$)[above]{$2$};
			\end{tikzpicture}}}
\end{ex}
\begin{ex}%[1D2B1-3]
	Số $1746360$ có bao nhiêu ước số nguyên?
	\choice
	{$120$}
	{$240$}
	{$60$}
	{\True $480$}
	\loigiai{
		Ta có $1746360=2^3\cdot 3^4\cdot 5\cdot 7^2\cdot 11$.\\
		Mỗi ước nguyên dương của $1746360$ có dạng $2^a\cdot 3^b\cdot 5^c\cdot 7^d\cdot 11^e$ với
		$a\in \{0;1;2;3\}$, $b\in \{0;1;2;3;4\}$, $c\in \{0;1\}$, $d\in \{0;1;2\}$ và $e\in \{0;1\}$.\\
		Suy ra có $4\cdot 5\cdot 2\cdot 3\cdot 2=240$ ước nguyên dương của $1746360$.\\
		Vậy số $1746360$ có $480$ ước số nguyên.
	}
\end{ex}

\begin{ex}%[1D2B1-3]
	Từ các chữ số $0$, $2$, $3$, $5$, $7$, $8$, $9$ lập được bao nhiêu số tự nhiên có $4$ chữ số khác nhau và luôn chứa một bộ phận là \lq\lq $35$\rq\rq?
	\choice
	{$60$}
	{$70$}
	{\True $52$}
	{$56$}
	\loigiai{
		\begin{enumerate}
			\item[TH 1.] Số có dạng $\overline{35ab}$.
			      \begin{enumerate}
				      \item $a$ có $5$ cách chọn.
				      \item $b$ có $4$ cách chọn.
			      \end{enumerate}
			      Theo quy tắc nhân thì ta có $5\cdot4=20$ số.
			\item[TH 2.] Số có dạng $\overline{a35b}$ hoặc $\overline{ab35}$.
			      \begin{enumerate}
				      \item $a$ có $4$ cách chọn.
				      \item $b$ có $4$ cách chọn.
			      \end{enumerate}
			      Theo quy tắc nhân thì ta có $4\cdot4\cdot 2=32$ số.
		\end{enumerate}
		Theo quy tắc cộng ta có $20+32=52$ số.
	}
\end{ex}
\begin{ex}%[Huỳnh Đức Vũ-Bài giảng Toán 10-đợt 2]%[1D2B1-3]
	Bình $A$ chứa $3$ quả cầu xanh, $4$ quả cầu đỏ và $5$ quả cầu trắng. Bình $B$ chứa $4$ quả cầu xanh, $3$ quả cầu đỏ và $6$ quả cầu trắng. Bình $C$ chứa $5$ quả cầu xanh, $5$ quả cầu đỏ và $2$ quả cầu trắng. Từ mỗi bình lấy ra một quả cầu. Có bao nhiêu cách lấy để cuối cùng được $3$ quả có màu giống nhau?
	\choice
	{\True $180$}
	{$60$}
	{$150$}
	{$120$}
	\loigiai{
		Mỗi cách lấy ra từ mỗi bình $1$ quả cầu sao cho $3$ quả cầu lấy ra có cùng màu được thực hiện theo một trong các phương án sau
		\begin{itemize}
			\item Phương án $1$: Ba quả cầu lấy ra cùng màu xanh, có $3\times 4\times 5=60$ cách lấy.
			\item Phương án $2$: Ba quả cầu lấy ra cùng màu đỏ, có $4\times 3\times 5=60$ cách lấy.
			\item Phương án $3$: Ba quả cầu lấy ra cùng màu trắng, có $5\times 6\times 2=60$ cách lấy.
		\end{itemize}
		Vậy có tất cả $60+60+60=180$ cách lấy quả cầu thoả mãn yêu cầu bài toán.}
\end{ex}
\begin{ex}%[Huỳnh Đức Vũ-Bài giảng Toán 10-đợt 2]%[1D2B1-3]
	Có bao nhiêu số tự nhiên có $4$ chữ số được viết từ các chữ số $1$, $2$, $3$, $4$, $5$, $6$, $7$, $8$, $9$ sao cho số đó chia hết cho $15$?
	\choice
	{$132$}
	{$432$}
	{$234$}
	{\True $243$}
	\loigiai{
		Gọi số cần tìm là $N=\overline{a_1a_2a_3a_4}.$\\
		Do $N$ chia hết cho $15$ nên $N$ phải chia hết cho $3$ và $5$, nên $a_4$ phải bằng $5$ và $a_1+a_2+a_3+a_4$ phải chia hết cho $3$.\\
		Do vai trò các chữ số $a_1$, $a_2$, $a_3$ là như nhau, mỗi  chữ số $a_1$ và $a_2$ có $9$ cách chọn nên ta xét các trường
		hợp
		\begin{itemize}
			\item Nếu $a_1+a_2+a_4=3k$ thì $a_3\in\{3;6;9\}$ có $3$ cách chọn.
			\item Nếu $a_1+a_2+a_4=3k+1$ thì $a_3\in\{2;5;8\}$ có $3$ cách chọn.
			\item Nếu $a_1+a_2+a_4=3k+2$ thì $a_3\in\{1;4;7\}$ có $3$ cách chọn.
		\end{itemize}
		Vậy trong phương án  thì $a_3$ có $3$ cách chọn.\\
		Vậy có tất cả $1\times 9^2\times 3=243$ số thỏa mãn.
	}
\end{ex}
\begin{ex}%[Huỳnh Đức Vũ-Bài giảng Toán 10-đợt 2]%[1D2K1-3]
	Có bao nhiêu số tự nhiên chẵn gồm ba chữ số khác nhau?
	\choice
	{\True $328$}
	{$500$}
	{$360$}
	{$405$}
	\loigiai{Đặt $E=\{0;1;2;3;4;5;6;7;8;9\}$.\\ Mỗi cách lập ra số tự nhiên $\overline{abc}$ là số chẵn, gồm $3$ chữ số phân biệt từ tập $E$ được thực hiện theo một trong các phương án sau:
		\begin{itemize}
			\item Phương án 1: $c=0$.
			      \begin{itemize}
				      \item Công đoạn $1$: Chọn $a\in E\setminus \{0\}$. Có $9$ cách.
				      \item Công đoạn $2$: Chọn $b\in E\setminus \{a;0\}$. Có $8$ cách.
			      \end{itemize}
			      Theo quy tắc nhân số cách chọn trong phương án này là $9\cdot 8=72$.\qquad (1)
			\item Phương án 2: $c\in \{2;4;6;8\}$.
			      \begin{itemize}
				      \item Công đoạn $1$: Chọn $c\in \{2;4;6;8\}$. Có $4$ cách.
				      \item Công đoạn $2$: Chọn $a\in E\setminus\{c;0\}$. Có $8$ cách.
				      \item Công đoạn $3$: Chọn $b\in E\{a,c\}$. Có $8$ cách.
			      \end{itemize}
			      Theo quy tắc nhân số cách chọn trong phương án này là $4\cdot 8\cdot 8=256$.\qquad (2)\\
			      Từ $(1)$ và $(2)$ theo quy tắc cộng, ta có số các số tự  nhiên thỏa đề bài là $72+256=328$.
		\end{itemize}
	}
\end{ex}
\begin{ex}%[Huỳnh Đức Vũ-Bài giảng Toán 10-đợt 2]%[1D2K1-3]
	Từ các chữ số $0$, $1$, $2$, $3$, $4$, $5$ có thể lập được tất cả bao nhiêu số tự nhiên có $3$ chữ số phân biệt và chia hết cho $3$?
	\choice
	{$34$}
	{$30$}
	{$48$}
	{\True $40$}
	\loigiai{
		Giả sử số tự nhiên cần lập có dạng $\overline{abc}.$
		\\Để số lập được chia hết cho $3$ thì $a+b+c$ phải chia hết cho $3$.
		\\ Khi đó $a,b,c$ thuộc các tập hợp sau đây
		$$\{ 0;1;2 \},\,\{ 0;1;5 \},\,\{ 0;2;4 \},\,\{0;4;5 \},\,\{ 1;2;3 \},\,\{ 1;3;5 \},\,\{ 2;3;4 \},\,\{ 3;4;5 \}.$$
		Suy ra có $4\cdot 2\cdot 2 \cdot 1+4\cdot 3\cdot2\cdot 1=40$ số chia hết cho $3.$
		\\Vậy ta có $40$ số thoả yêu cầu bài toán.
	}
\end{ex}
\begin{ex}%[Huỳnh Đức Vũ-Bài giảng Toán 10-đợt 2]%[1D2K1-3]
	Từ các chữ số $1$, $3$, $5$, $7$, $9$ có thể lập được bao nhiêu số tự nhiên bé hơn $500$?
	\choice
	{$120$}
	{\True $80$}
	{$60$}
	{$45$}
	\loigiai{
		Mỗi các lập số tự nhiên bé hơn $500$ từ các chữ số đã cho được thực hiện theo một trong các phương án sau:
		\begin{itemize}
			\item Phương án $1$: Số có một chữ số: Có $5$ cách lập.
			\item Phương án $2$: Số có $2$ chữ số có $5\cdot 5=25$ cách.
			\item Phương án $3$: Số có $3$ chữ số chữ số. Gọi số cần tìm là $\overline{abc}$ khi đó chữ số $a$ nhỏ hơn bằng $4$ và các chữ số $b$, $c$ được chọn tùy ý.\\
			      $a\in\{1;3\}$: có $2$ cách chọn.\\
			      $b$ có $5$ cách chọn, $c$  có $5$ cách chọn.\\
			      Vậy có $2\cdot 5\cdot 5=50$ cách.
		\end{itemize}
		Theo quy tắc cộng ta có số các số thỏa đề bài là $5+25+50=80$.
	}
\end{ex}
\begin{ex}%[Huỳnh Đức Vũ-Bài giảng Toán 10-đợt 2]%[1D2K1-3]
	Có bao nhiêu số tự nhiên có sáu chữ số khác nhau từng đôi một, trong đó chữ số $5$ đứng liền giữa hai chữ số $1$ và $4$?
	\choice
	{$249$}
	{\True $1500$}
	{$3204$}
	{$2942$}
	\loigiai{
		Mỗi cách lập ra số $\overline{abcdef}$ thỏa đề bài được thực hiện theo một trong các phương án sau
		\begin{itemize}
			\item Phương án $1$: Số cần tìm có dạng $\overline{154def}$.
			      \begin{itemize}
				      \item Công đoạn $1$: Chọn $d$. Có $7$ cách chọn.
				      \item Công đoạn $2$: Chọn $e$. Có $6$ cách chọn.
				      \item Công đoạn $3$: Chọn $f$. Có $5$ cách chọn. \\
				            Theo quy tắc nhân, phương án này có $7 \cdot 6 \cdot 5 = 210$ cách chọn.
			      \end{itemize}
			\item Phương án $2$: Số cần tìm có dạng $\overline{a154ef}$.
			      \begin{itemize}
				      \item Công đoạn $1$: Chọn $a$. Có $6$ cách chọn.
				      \item Công đoạn $2$: Chọn $e$. Có $6$ cách chọn.
				      \item Công đoạn $3$: Chọn $f$. Có $5$ cách chọn. \\
				            Theo quy tắc nhân, phương án này có $6 \cdot 6 \cdot 5 = 180$ cách chọn.
			      \end{itemize}
			\item Phương án $3$: Số cần tìm có dạng $\overline{ab154f}$.\\
			      Tương tự phương án $2$, có $180$ cách chọn.
			\item Phương án $4$: Số cần tìm có dạng $\overline{abc154}$.\\
			      Tương tự phương án $2$, có $180$ cách chọn.
		\end{itemize}
		Do vị trí số $1$ và $4$ có vai trò như nhau nên tất cả có $\left(210 + 3 \cdot 180 \right) \cdot 2 = 1500$ cách chọn.
	}
\end{ex}
\begin{ex}%[Huỳnh Đức Vũ-Bài giảng Toán 10-đợt 2]%[1D2K1-3]
	Từ các chữ số $1$, $3$, $5$, $7$, $9$ có thể lập được bao nhiêu số tự nhiên gồm $3$ chữ số đôi một khác nhau và nhỏ hơn $379$?
	\choice
	{$30$}
	{$60$}
	{$12$}
	{\True $20$}
	\loigiai{
		Mỗi cách lập ra số  $\overline {abc}<379 $ được thực hiện theo một trong các phương án sau
		\begin{itemize}
			\item Phương án $1$. $\overline {abc}$ với $a<3$.
			      \begin{itemize}
				      \item Công đoạn $1$: Chọn $a<3$. Có $1$ cách chọn.
				      \item Công đoạn $2$: Chọn $b$. Có $4$ cách chọn.
				      \item Công đoạn $3$: Chọn $c$. Có $3$ cách chọn.\\
				            Theo quy tắc nhân, phương án này có $1\cdot4\cdot3=12$ số. \qquad (1)
			      \end{itemize}
			\item Phương án $2$. $\overline {abc}=\overline {3bc}$ với $b<7$.
			      \begin{itemize}
				      \item Công đoạn $1$. Chọn $b$. Có $2$ cách chọn.
				      \item Công đoạn $2$. Chọn $c$. Có $3$ cách chọn.\\
				            Theo quy tắc nhân, phương án này có
				            $1\cdot2\cdot3=6$ số. \qquad (2)
			      \end{itemize}
			\item Phương án $3$: $\overline {abc}=\overline {37c}$ với $c<9$.\\
			      Vì $c\in\{1;5\}$ nên có $2$ cách chọn $c$. Phương án này có $2$ số thỏa mãn. \qquad (3)
		\end{itemize}
		Từ $(1)$, $(2)$ và $(3)$, theo quy tắc cộng, ta có số các số thỏa đề bài là $12+6+2=20$ số.
	}
\end{ex}
\begin{ex}%[Huỳnh Đức Vũ-Bài giảng Toán 10-đợt 2]%[1D2K1-3]
	Xếp $6$ người $A$, $B$, $C$, $D$, $E$, $F$ vào một ghế dài. Hỏi có bao nhiêu cách sắp xếp sao cho $A$ và $F$ không ngồi cạnh nhau?
	\choice
	{$260$}
	{\True $480$}
	{$460$}
	{$240$}
	\loigiai{
		Để xếp $6$ người $A$, $B$, $C$, $D$, $E$, $F$ vào một ghế dài sao cho $A$ và $F$ không ngồi cạnh nhau, ta có $2$ phương án  sau:
		\begin{itemize}
			\item Phương án  $1$: $A$ ở hai đầu ghế, có $2$ cách chọn vị trí cho $A$. Tiếp đến chọn vị trí cho $F$, có $4$ cách chọn. Xếp $4$ người còn lại có $4!$ cách xếp.\\
			      Suy ra có $2\cdot 4\cdot 4!=192$ cách.
			\item Phương án  $2$: $A$ không ngồi ở hai đầu ghế, có $4$ cách chọn vị trí cho $A$. Tiếp đến chọn vị trí cho $F$, có $3$ cách chọn. Xếp $4$ người còn lại có $4!$ cách xếp.\\
			      Suy ra có $4\cdot 3\cdot 4!=288$ cách.
		\end{itemize}
		Vậy có $192+288=480$ cách sắp xếp thỏa mãn bài toán.\\
		\textbf{Cách khác:}\\
		Xếp $6$ người vào ghế ta có $6!=720$ cách.\\
		Ta xếp $A$ và $F$ ngồi cạnh nhau như sau
		\begin{itemize}
			\item Xem hai người $A$ và $F$ là nhóm $X$. Xếp nhóm $X$ và $4$ người $B$, $C$, $D$ vào ghế, có $5!$ cách.
			\item $A$ và $F$ có thể đổi chỗ cho nhau, nên có $2$ cách đổi chỗ cho $A$ và $F$.
			\item Khi đó có $5!\cdot 2=240$ cách xếp hai người $A$ và $F$ ngồi cạnh nhau.
		\end{itemize}
		Vậy có $720-240=480$ cách xếp sao cho $A$ và $F$ không ngồi cạnh nhau.
	}
\end{ex}
\begin{ex}%[Huỳnh Đức Vũ-Bài giảng Toán 10-đợt 2]%[1D2K1-3]
	Từ các chữ số $0; 1; 2; 3; 4; 5; 6$ có thể lập được bao nhiêu số tự nhiên có 5 chữ số khác nhau và chia hết cho $15$?
	\choice
	{ $200$}
	{ $240$}
	{\True $222$}
	{ $120$}
	\loigiai{
		Gọi số cần tìm có dạng $\overline{abcde}$, thỏa mãn các chữ số đều khác nhau.\\
		Để chia hết cho $15$ thì phải chia hết cho $3$ và $5$. Do đó tận cùng phải là $0$ hoặc $5$.
		\begin{description}
			\item[\textbf{Phương án  1.}]  $e=0$, khi đó $a+b+c+d$ phải chia hết cho $3$. Suy ra ta có các cặp gồm\\
			      $\{1, 2, 3, 6\}$; $\{1, 2, 4, 5\}$; $\{1, 3, 5, 6\}$;$\{2, 3, 4, 6\}$; $\{3, 4, 5, 6\}$.\\ Phương án  này có $5\times 4!=120$ cách.
			\item[\textbf{Phương án  2.}]  $e=5$, khi đó $a+b+c+d$ phải chia cho $3$ dư 1. Suy ra ta có các cặp gồm\\
			      $\{0, 1, 2, 4\}$; $\{0, 1, 3, 6\}$; $\{0, 3, 4, 6\}$; $\{1, 2, 3, 4\}$, $\{1, 2, 4, 6\}$.\\
			      Suy ra có $3\times 3\times 3! +2\times 4!=102$.
		\end{description}
		Vậy có tất cả là $120+102=222$ cách.
	}
\end{ex}
\begin{ex}%[Huỳnh Đức Vũ-Bài giảng Toán 10-đợt 2]%[1D2K1-3]
	Từ các chữ số $0$, $1$, $2$ có thể thành lập được bao nhiêu số tự nhiên gồm $9$ chữ số và là bội số của $ 3 $ đồng thời bé hơn $2\cdot10^8$?
	\choice
	{$4374$}
	{\True $2187$}
	{$6561$}
	{$3645$}
	\loigiai{
		Gọi số thỏa mãn bài có dạng $ A=\overline{ a_1 a_2 a_3 a_4 a_5 a_6a_7a_8a_9} $ trong đó $ a_i\in \{0;1;2\} $ và các $ a_i $ không đồng thời bằng $ 0. $\\
		Vì $ A<2\cdot10^8 $ nên $ a_1=1\Rightarrow a_1 $ có $ 1 $ cách chọn.\\
		Các chữ số từ $ a_2 $ đến $ a_8 $ đều có $ 3 $ cách chọn.\\
		Khi đó $ a_1+a_2+\ldots+a_8 $ có thể chia hết cho $ 3 $ hoặc chia cho $ 3 $ dư $ 1 $ hoặc chia cho $ 3 $ dư $ 2. \\$
		+ Nếu $ a_1+a_2+\ldots+a_8 $ chia hết cho $ 3 $ thì $ a_9=0. $\\
		+ Nếu $ a_1+a_2+\ldots+a_8 $ chia cho $ 3 $ dư $ 1 $ thì $ a_9=2. $\\
		+ Nếu $ a_1+a_2+\ldots+a_8 $ chia cho $ 3 $ dư $ 2 $ thì $ a_9=1. $\\
		$ \Rightarrow $ chữ số $ a_9 $ có đúng $ 1 $ cách chọn.\\
		Vậy có $ 1.3^7.1=2187 $ số cần tìm.
	}
\end{ex}
\begin{ex}%[Huỳnh Đức Vũ-Bài giảng Toán 10-đợt 2]%[1D2K1-3]
	Từ các chữ số $1$, $2$, $3$, $4$, $5$, $6$ có thể lập được bao nhiêu số tự nhiên chẵn có sáu chữ số và thỏa mãn điều kiện: sáu chữ số của mỗi số là khác nhau và chữ số hàng nghìn lớn hơn $2$?
	\choice
	{\True $240$}
	{$720$}
	{$360$}
	{$288$}
	\loigiai{
		Gọi các số thỏa mãn bài toán có dạng $\overline{a_1a_2a_3a_4a_5a_6}$, với $a_1, a_2, a_3, a_4, a_5, a_6 \in A = \{1; 2; 3; 4; 5; 6 \}$.
		\\
		Vì số cần tìm có hàng nghìn lớn hơn $2$ nên $a_3 \geq 3$.
		Mỗi cách lập ra số thỏa đề bài được thực hiện theo một trong các phương án sau:
		\begin{itemize}
			\item \textbf{Phương án  1}: $a_3$ là số lẻ. \\
			      Vị trí $a_3$ có $2$ cách chọn từ tập $\{ 3; 5\} $. \\
			      Vì số cần tìm là số chẵn nên $a_6$ có $3$ cách chọn từ tập $\{ 2; 4; 6 \}$. \\
			      Vị trí $a_1$ có $4$ cách chọn từ tập $A \setminus \{ a_3; a_6 \} $. \\
			      Vị trí $a_2$ có $3$ cách chọn từ tập $A \setminus \{ a_3; a_6; a_1 \} $. \\
			      Vị trí $a_4$ có $2$ cách chọn từ tập $A \setminus \{ a_3; a_6; a_1; a_2 \} $. \\
			      Vị trí $a_5$ có $1$ cách chọn từ tập $A \setminus \{ a_3; a_6; a_1; a_2; a_3 \} $. \\
			      Theo quy tắc nhân, số các trong phương án  này là $2 \cdot 3 \cdot 4 \cdot 3 \cdot 2 \cdot 1 = 144$ số. \qquad (1)
			\item \textbf{Phương án  2}: $a_3$ là số chẵn. \\
			      Vị trí $a_3$ có $2$ cách chọn từ tập $\{ 4; 6\} $. \\
			      Vì số cần tìm là số chẵn nên $a_6$ có $2$ cách chọn từ tập $\{2; 4; 6\} \setminus \{ a_3 \}$. \\
			      Vị trí $a_1$ có $4$ cách chọn từ tập $A \setminus \{ a_3; a_6 \} $. \\
			      Vị trí $a_2$ có $3$ cách chọn từ tập $A \setminus \{ a_3; a_6; a_1 \} $. \\
			      Vị trí $a_4$ có $2$ cách chọn từ tập $A \setminus \{ a_3; a_6; a_1; a_2 \} $. \\
			      Vị trí $a_5$ có $1$ cách chọn từ tập $A \setminus \{ a_3; a_6; a_1; a_2; a_3 \} $. \\
			      Theo quy tắc nhân, số các số trong phương án  này là $2 \cdot 2 \cdot 4 \cdot 3 \cdot 2 \cdot 1 = 96$ số.\qquad (2)
		\end{itemize}
		Từ $(1)$ và $(2)$ theo quy tắc cộng, số các số thỏa mãn bài toán là $144 + 96 = 240$ số.
	}
\end{ex}
\begin{ex}%[Huỳnh Đức Vũ-Bài giảng Toán 10-đợt 2]%[1D2K1-3]
	Có bao nhiêu số tự nhiên có ba chữ số dạng $ \overline{abc} $ với $ a,\ b,\ c\in\{0;1;\ldots;6\}$, sao cho \break $a<b<c$?
	\choice
	{$ 120 $}
	{\True $ 20 $}
	{$ 40 $}
	{$ 30 $}
	\loigiai{
		Vì $ a\neq 0 $ nên $ a\geq 1 $. Do $ a<b<c $ và $ c\leq 6 $ nên $ a=\{1;\, 2;\,3;\, 4\}$.
		\begin{itemize}
			\item Phương án $1$. Với $ a=1 $:
			      \begin{itemize}
				      \item Xét $ b=2\Rightarrow c\geq 3 $, do đó có $ 4 $ số thỏa mãn.
				      \item Xét $ b=3\Rightarrow c\geq 4 $, do đó có $ 3 $ số thỏa mãn.
				      \item Xét $ b=4\Rightarrow c\geq 5 $, do đó có $ 2 $ số thỏa mãn.
				      \item Xét $ b=5\Rightarrow c\geq 6 $, do đó có $ 1 $ số thỏa mãn.
			      \end{itemize}
			\item Phương án $2$. Với $ a=2 $:
			      \begin{itemize}
				      \item Xét $ b=3\Rightarrow c\geq 4 $, do đó có $ 3 $ số thỏa mãn.
				      \item Xét $ b=4\Rightarrow c\geq 5 $, do đó có $ 2 $ số thỏa mãn.
				      \item Xét $ b=5\Rightarrow c\geq 6 $, do đó có $ 1 $ số thỏa mãn.
			      \end{itemize}
			\item Phương án $3$. Với $ a=3 $:
			      \begin{itemize}
				      \item Xét $ b=4\Rightarrow c\geq 5 $, do đó có $ 2 $ số thỏa mãn.
				      \item Xét $ b=5\Rightarrow c\geq 6 $, do đó có $ 1 $ số thỏa mãn.
			      \end{itemize}
			\item Phương án $4$. Với $ a=4\Rightarrow b=5$ và $ c= 6 $, do đó có $ 1 $ số thỏa mãn.
		\end{itemize}
		Vậy có tất cả $ (4+3+2+1)+(3+2+1)+(2+1)+1=20 $ số.
	}
\end{ex}
\begin{ex}%[Huỳnh Đức Vũ-Bài giảng Toán 10-đợt 2]%[1D2K1-3]
	Một túi có $14$ viên bi gồm $5$ viên màu trắng được đánh số từ $1$ đến $5$; $4$ viên màu đỏ được đánh số từ $1$ đến $4$; $3$ viên màu xanh được đánh số từ $1$ đến $3$ và $2$ viên màu vàng được đánh số từ $1$ đến $2$. Có bao nhiêu cách chọn $3$ viên bi từng đôi khác số?
	\choice
	{$184$}
	{$120$}
	{$243$}
	{\True $190$}
	\loigiai{
		Số viên bi được đánh số $1$, $2$, $3$, $4$, $5$ lần lượt là $4$, $4$, $3$, $2$, $1$.\\
		Vì ba viên bi từng đôi khác số nên khi chọn, ta có thể có những phương án  sau: $$(1,2,3); (1,2,4); (1,2,5); (1,3,4); (1,3,5); (1,4,5); (2,3,4); (2,3,5); (2,4,5); (3,4,5).$$
		\begin{itemize}
			\item Phương án  $(1,2,3)$: Vì số viên bi được đánh số $1,2,3$ lần lượt là $4$, $4$, $3$ nên số cách chọn ba viên bi trong phương án  này là $48$ cách.
			\item Tương tự, những phương án còn lại lần lượt có số cách chọn là  $48$, $32$, $16$, $24$, $12$, $8$, $24$, $12$, $8$, $6$.
		\end{itemize}
		Vậy có tổng cộng $48+32+16+24+12+8+24+12+8+6=190$ cách.
	}
\end{ex}
\begin{ex}%[Huỳnh Đức Vũ-Bài giảng Toán 10-đợt 2]%[1D2G1-3]
	Một hộp đựng $26$ tấm thẻ được đánh số từ $1$ đến $26$. Bạn Hải rút ngẫu nhiên cùng một lúc ba tấm thẻ. Hỏi có bao nhiêu cách rút sao cho bất kỳ hai trong ba tấm thẻ lấy ra đó có hai số tương ứng ghi trên hai tấm thẻ luôn hơn kém nhau ít nhất $2$ đơn vị?
	\choice
	{$1350$}
	{$1768$}
	{\True $2024$}
	{$1771$} %[1D2K1-3]
	\loigiai{
		Số cách rút ra ba thẻ, sao cho trong ba thẻ đó luôn có ít nhất hai thẻ mà số ghi trên hai thẻ đó là hai số tự nhiên liên tiếp, ta có các phương án .
		\begin{itemize}
			\item Rút hai thẻ liên tiếp có cặp số là $1$; $2$, thì thẻ thứ $3$ ta có $24$ cách rút.
			\item Rút hai thẻ liên tiếp có cặp số là $2$; $3$, thì thẻ thứ $3$ không thể là thẻ có số $1$, suy ra có $23$ cách rút thẻ thứ $3$.
			\item Rút hai thẻ liên tiếp có cặp số là $3$; $4$, thì thẻ thứ $3$ không thể là thẻ có số $2$, suy ra có $23$ cách rút thẻ thứ $3$.
			\item Rút hai thẻ liên tiếp có cặp số là $24$; $25$, thì thẻ thứ $3$ không thể là thẻ có số $23$, suy ra có $23$ cách rút thẻ thứ $3$.
			\item Rút hai thẻ liên tiếp có cặp số là $25$; $26$, thì thẻ thứ $3$ không thể là thẻ có số $24$, suy ra có $23$ cách rút thẻ thứ $3$.
		\end{itemize}
		Từ đó suy ra, có $24+23\times 24=576 $ cách rút ra ba thẻ sao cho trong ba thẻ luôn có ít nhất hai thẻ mà số ghi trên hai thẻ đó là hai số tự nhiên liên tiếp.\\
		Vậy số cách rút ra ba thẻ mà trong hai thẻ bất kỳ lấy ra có hai số tương ứng luôn hơn kém nhau ít nhất hai đơn vị là $n(\Omega)-576=2024$.
	}
\end{ex}
\begin{ex}%[Huỳnh Đức Vũ-Bài giảng Toán 10-đợt 2]%[1D2K1-3]
	Xếp $6$ người $A$, $B$, $C$, $D$, $E$, $F$ vào một ghế dài. Hỏi có bao nhiêu cách sắp xếp sao cho $A$ và $F$ không ngồi cạnh nhau?
	\choice
	{$460$}
	{\True $480$}
	{$260$}
	{$240$}
	\loigiai{
		Mỗi cách sắp xếp $6$ người $A$, $B$, $C$, $D$, $E$, $F$ vào một ghế dài sao cho $A$ và $F$ không ngồi cạnh nhau được thực hiện theo một trong các phương án sau
		\begin{itemize}
			\item Phương án $1$: $A$ ở hai đầu ghế, có $2$ cách chọn vị trí cho $A$. Tiếp đến chọn vị trí cho $F$, có $4$ cách chọn. Xếp $4$ người còn lại có $4!$ cách xếp.\\
			      Suy ra có $2\cdot 4\cdot 4!=192$ cách.
			\item Phương án $2$: $A$ không ngồi ở hai đầu ghế, có $4$ cách chọn vị trí cho $A$. Tiếp đến chọn vị trí cho $F$, có $3$ cách chọn. Xếp $4$ người còn lại có $4!$ cách xếp.\\
			      Suy ra có $4\cdot 3\cdot 4!=288$ cách.
		\end{itemize}
		Vậy có $192+288=480$ cách sắp xếp thỏa mãn bài toán.\\
		\textbf{Cách khác:}\\
		Xếp $6$ người vào ghế ta có $6!=720$ cách.\\
		Ta xếp $A$ và $F$ ngồi cạnh nhau như sau
		\begin{itemize}
			\item Xem hai người $A$ và $F$ là nhóm $X$. Xếp nhóm $X$ và $4$ người $B$, $C$, $D$ vào ghế, có $5!$ cách.
			\item $A$ và $F$ có thể đổi chỗ cho nhau, nên có $2$ cách đổi chỗ cho $A$ và $F$.
			\item Khi đó có $5!\cdot 2=240$ cách xếp hai người $A$ và $F$ ngồi cạnh nhau.
		\end{itemize}
		Vậy có $720-240=480$ cách xếp sao cho $A$ và $F$ không ngồi cạnh nhau.
	}
\end{ex}
\begin{ex}%[Huỳnh Đức Vũ-Bài giảng Toán 10-đợt 2]%[1D2K1-3]
	Từ các chữ số $0$, $1$, $2$, $3$, $5$, $8$ có thể lập được bao nhiêu số tự nhiên lẻ có bốn chữ số đôi một khác nhau và phải có chữ số $3$?
	\choice
	{\True $108$}
	{$144$}
	{$228$}
	{$36$}
	\loigiai{
		Gọi các số thỏa mãn bài toán có dạng $\overline{a_1a_2a_3a_4}$, với $a_1,a_2,a_3,a_4 \in A = \{0; 1; 2; 3; 5; 8 \}$.
		\begin{itemize}
			\item Phương án $1$: Xét $a_4 = 3$. \\
			      Vị trí $a_1$ có $4$ cách chọn từ tập $A \setminus \{ 0; 3 \} $. \\
			      Vị trí $a_2$ có $4$ cách chọn từ tập $A \setminus \{ a_1; 3 \} $. \\
			      Vị trí $a_3$ có $3$ cách chọn từ tập $A \setminus \{ a_1; a_2; 3 \} $. \\
			      Theo quy tắc nhân, số các số thỏa mãn bài toán trong trường hợp này là $4 \cdot 4 \cdot 3 = 48$ số.
			\item Phương án $2$: Xét $a_1 = 3$. \\
			      Vì số cần tìm là số lẻ nên $a_4$ có $2$ cách chọn từ tập $\{ 1; 5 \}$. \\
			      Vị trí $a_2$ có $4$ cách chọn từ tập $A \setminus \{ 3; a_4 \} $. \\
			      Vị trí $a_3$ có $3$ cách chọn từ tập $A \setminus \{ 3; a_4; a_2 \} $. \\
			      Theo quy tắc nhân, số các số thỏa mãn bài toán trong trường hợp này là $2 \cdot 4 \cdot 3 = 24$ số.
			\item Phương án $3$: Xét $a_1 \neq 3$ và $a_4 \neq 3$. \\
			      Vì số cần tìm là số lẻ nên $a_4$ có $2$ cách chọn từ tập $\{ 1; 5 \}$. \\
			      Vị trí $a_1$ có $3$ cách chọn từ tập $A \setminus \{ 0; 3; a_4 \} $. \\
			      Chọn 1 vị trí để đặt số $3$, có $2$ cách (vị trí $a_2$, $a_3$). \\
			      Vị trí cuối cùng có $3$ cách chọn từ tập $A \setminus \{a_4; a_1; 3 \}$. \\
			      Theo quy tắc nhân, số các số thỏa mãn bài toán trong trường hợp này là $2 \cdot 3 \cdot 2 \cdot 3 = 36$ số.
		\end{itemize}
		Theo quy tắc cộng, số các số thỏa mãn bài toán là $48 + 24 + 36 = 108$ số.
	}
\end{ex}
\begin{ex}%[Huỳnh Đức Vũ-Bài giảng Toán 10-đợt 2]%[1D2K1-3]
	Từ tập $E=\{0,1,2,3,4,5,6,7\}$ lập được bao nhiêu số tự  nhiên  gồm ba chữ số phân biệt trong đó luôn có chữ số $2$?
	\choice
	{\True $114$}
	{$144$}
	{$58$}
	{$228$}
	\loigiai{
		Mỗi cách lập ra số tự nhiên $\overline{abc}$ gồm $3$ chữ số phân biệt từ $E$ sao cho trong đó luôn có chữ số $2$ được thực hiện theo một trong các phương án sau:
		\begin{itemize}
			\item Phương án $1$: Xét $\overline{abc}=\overline{ab2}$.
			      \begin{itemize}
				      \item Công đoạn $1$: Chọn $a \in E \setminus \{0 ; 2\}$. Có $6$ cách.
				      \item Công đoạn $2$. Chọn $ b \in \mathrm{E} \setminus\{2 ; a\}$. Có 6 cách.\\
				            Theo quy tác nhân, số cách chọn trong phương án này là $6\cdot 6=36$ cách. \qquad (1)
			      \end{itemize}
			\item Phương án $2$: Xét $\overline{a b c}=\overline{a 2 c}$.
			      \begin{itemize}
				      \item Công đoạn $1$: Chọn $a \in E \setminus\{0 ; 2\}$. Có $6$ cách.
				      \item Công đoạn $2$: Chọn $c \in E \setminus\{2 ; a\}$. Có $6$ cách.\\
				            Theo quy tác nhân, số cách chọn trong phương án này là $6\cdot 6=36$ cách \qquad (2)
			      \end{itemize}
			\item Phương án $3$: Xét $\overline{a b c}=\overline{2 b c}$.
			      \begin{itemize}
				      \item Công đoạn $1$: Chọn $b\in E \setminus\{2\}$. Có $7$ cách.
				      \item Công đoạn $2$: Chọn $c \in E \setminus\{2 ; a\}$. Có $6$ cách.\\
				            Theo quy tác nhân, số cách chọn trong phương án này là $7\cdot 6=42$ cách.  \qquad (3)
			      \end{itemize}
			      Từ $(1)$, $(2)$ và $(3)$ theo quy tắc cộng, ta có số các số thỏa đề bài là $36+36+42=114$.
		\end{itemize}}
\end{ex}
\begin{ex}%[Huỳnh Đức Vũ-Bài giảng Toán 10-đợt 2]%[1D2G1-3]
	Cho tập hợp $A=\{0; 1; 2; 3; 4; 5; 6; 7\}$. Có bao nhiêu số tự nhiên chẵn có $6$ chữ số đôi một khác nhau được lập thành từ các chữ số của tập $A$, đồng thời có đúng $3$ chữ số lẻ và $3$ chữ số lẻ đó đứng cạnh nhau?
	\choice
	{$48$}
	{$4464$}
	{$240$}
	{\True $1440$}
	\loigiai{
		Giả sử số cần tìm gồm $3$ chữ số lẻ $l_1$, $l_2$, $l_3$ và $3$ chữ số chẵn $c_1$, $c_2$, $c_3$. Ta thấy rằng các chữ số $l_1$, $l_2$, $l_3$ được chọn ngẫu nhiên đôi một khác nhau trong tập hợp con của $A$ gồm các chữ số lẻ $\{1;3;5;7\}$ và các chữ số $c_1$, $c_2$, $c_3$ được chọn ngẫu nhiên đôi một khác nhau trong tập hợp con của $A$ gồm các chữ số chẵn $\{0;2;4;6\}$.\\
		Vì tập $A$ có chữ số $0$ nên mỗi cách laapj ra số thỏa đề bài được thực hiện theo một trong các phương án sau:
		\begin{enumerate}
			\item \textbf{Phương án $1$}: Số tự nhiên lập thành có dạng $\overline{l_1l_2l_3c_1c_2c_3}$.\\
			      Theo thứ tự từ trái qua phải, $l_1$ có $4$ cách chọn, $l_2$ có $3$ cách chọn và $l_3$ có $2$ cách chọn.\\
			      Tương tự, $c_1$ có $4$ cách chọn, $c_2$ có $3$ cách chọn và $c_3$ có $2$ cách chọn.\\
			      Phương án  này, ta có $4 \times 3 \times 2 \times 4 \times 3 \times 2=576$ số thỏa mãn đề.
			\item \textbf{Phương án $2$}: Số tự nhiên lập thành có dạng $\overline{c_1l_1l_2l_3c_2c_3}$ hoặc $\overline{c_1c_2l_1l_2l_3c_3}$.\\
			      Ở cả hai dạng này, theo thứ tự từ trái qua phải, vì $c_1 \ne 0$ nên $c_1$ có $3$ cách chọn, $c_2$ có $3$ cách chọn và $c_3$ có $2$ cách chọn. Chữ số lẻ $l_1$ có $4$ cách chọn, $l_2$ có $3$ cách chọn và $l_3$ có $2$ cách chọn.\\
			      Phương án  này, ta có $2 \times (3 \times 3 \times 2 \times 4 \times 3 \times 2 )= 864$ số thỏa mãn đề.
		\end{enumerate}
		Vậy, ta có tổng cộng $576+864=1440$ số thỏa đề.
	}
\end{ex}
\begin{ex}%[Huỳnh Đức Vũ-Bài giảng Toán 10-đợt 2]%[1D2G1-3]
	Cho $10$ chữ số $0$, $1$, $2$, $3$, $4$, $5$, $6$, $7$, $8$, $9$. Có thể tạo ra được bao nhiêu số tự nhiên gồm $5$ chữ số khác nhau, trong đó có mặt đủ $3$ chữ số $2$, $3$ và $4$?
	\choice
	{$25056$}
	{\True $2376$}
	{$27216$}
	{$25592$}
	\loigiai{
		Mỗi cách lập ra số $\overline{abcde}$ thỏa mãn đề bài được thực hiện theo một trong các phương án sau:
		\begin{enumerate}
			\item \textbf{Phương án $1$}: Xét $a\notin \{2;3;4\}$.
			      \begin{itemize}
				      \item Có $6$ cách chọn $a$ (trừ các số $\{0;2;3;4\}$).
				      \item Có $4\cdot 3 \cdot 2$ cách chọn vị trí cho các số $2$, $3$ và $4$.
				      \item Có $6$ cách chọn một số vào vị trí còn lại (trừ $a$ và các số $\{2;3;4\}$).
			      \end{itemize}
			\item \textbf{Phương án $2$}: Xét $a\in\{2,3,4\}$.
			      \begin{itemize}
				      \item Có $3$ cách chọn $a$.
				      \item Có $4\cdot 3$ cách chọn vị trí cho $2$ trong $3$ số $2$, $3$ và $4$.
				      \item Có $7\cdot 6$ cách chọn hai số vào hai vị trí còn lại.
			      \end{itemize}
		\end{enumerate}
		Vậy có $6\cdot 4 \cdot 3 \cdot 2 \cdot 6+3 \cdot 4 \cdot 3\cdot 7\cdot 6=2376$ số thỏa mãn bài toán.
	}
\end{ex}
\begin{ex}%[Huỳnh Đức Vũ-Bài giảng Toán 10-đợt 2]%[1D2G1-3]
	Trong mặt phẳng, cho hai đường thẳng phân biệt $a$ và $b$ song song với nhau. Trên đường thẳng $a$ lấy $5$ điểm phân biệt $A$, $B$, $C$, $D$, $E$ và trên đường thẳng $b$ lấy $5$ điểm phân biệt $G$, $H$, $I$, $J$, $K$ sao cho $AB=BC=CD=DE=GH=HI=IJ=JK=20$ cm. Có bao nhiêu hình bình hành có $4$ đỉnh là $4$ điểm trong $10$ điểm nói trên?
	\choice
	{\True $30$}
	{$210$}
	{$16$}
	{$100$}
	\loigiai{
		Đặt $x=20$ cm. Mỗi cách lập ra hình bình hành thỏa đề bài được thực thiện theo một trong các phương án sau
		\begin{itemize}
			\item \textbf{Phương án $1$}: Hình bình hành có một cặp cạnh độ dài $x$.\\
			      Có $4$ cách chọn một cạnh trên đường thẳng $a$.\\
			      Có $4$ cách chọn một cạnh trên đường thẳng $b$.\\
			      Theo quy tắc nhân, ta có $4\times4=16$ cách chọn hình bình hành có một cặp cạnh độ dài $x$. \quad(1)
			\item \textbf{Phương án $2$}: Hình bình hành có một cặp cạnh độ dài $2x$.\\
			      Có $3$ cách chọn một cạnh trên đường thẳng $a$.\\
			      Có $3$ cách chọn một cạnh trên đường thẳng $b$.\\
			      Theo quy tắc nhân, ta có $3\times3=9$ cách chọn hình bình hành có một cặp cạnh độ dài $2x$.  \quad(2)
			\item \textbf{Phương án $3$}: Hình bình hành có một cặp cạnh độ dài $3x$.\\
			      Có $2$ cách chọn một cạnh trên đường thẳng $a$.\\
			      Có $2$ cách chọn một cạnh trên đường thẳng $b$.\\
			      Theo quy tắc nhân, ta có $2\times2=4$ cách chọn hình bình hành có một cặp cạnh độ dài $3x$.  \quad(3)
			\item \textbf{Phương án $4$}: Hình bình hành có một cặp cạnh độ dài $4x$.\\
			      Có $1$ cách chọn một cạnh trên đường thẳng $a$.\\
			      Có $1$ cách chọn một cạnh trên đường thẳng $b$.\\
			      Do đó, phương án này có $1\times1=1$ cách chọn hình bình hành có một cặp cạnh độ dài $4x$.  \quad(4)
		\end{itemize}
		Từ $(1)$, $(2)$, $(3)$ và $(4)$, theo quy tắc cộng, ta có số cách tạo ra hình bình hành từ các điểm đã cho là  $16+9+4+1=30$ cách.}
\end{ex}
\Closesolutionfile{ans}
\Closesolutionfile{ansbook}
% \begin{indapan}{10}{ans/ans-0D8-1-3}\end{indapan}