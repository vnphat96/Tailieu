\newpage\setcounter{dang}{0}
\section{Nhị thức Newton}
\subsection{Tóm tắt lý thuyết}
\subsubsection{Nhị thức Newton}
Cho $a$, $b$ là các số thực. Ta có 
$$(a+b)^4=\mathrm{C}_4^0a^4+\mathrm{C}_4^1a^{3}b+\mathrm{C}_4^2a^{2}b^2 +\mathrm{C}_4^{3}ab^{3}+\mathrm{C}_4^4b^4 = a^4+4a^3b+6a^2b^2+4ab^3+b^4.$$
$$(a+b)^5=\mathrm{C}_5^0a^5+\mathrm{C}_5^1a^{4}b+\mathrm{C}_5^2a^{3}b^2 +\mathrm{C}_5^{3}a^2b^{3}+\mathrm{C}_5^4ab^4+\mathrm{C}_5^5b^5 = a^5+5a^4+10a^3b^2+10a^2b^3+5ab^4+b^5.$$
	\subsection{Các dạng toán}	
\begin{dang}{Khai triển một nhị thức Newton}
	Cho $a$, $b$ là các số thực. Ta có 
	\begin{eqnarray*}
		&&(a+b)^4=a^4+4a^3b+6a^2b^2+4ab^3+b^4.\\&&(a+b)^5=a^5+5a^4+10a^3b^2+10a^2b^3+5ab^4+b^5.
	\end{eqnarray*}
\end{dang}
\viduminhhoa
\begin{vd}%[Lê Nguyễn Viết Tường,BG10-2022]%%[0D8N3-2]
	Khai triển $(x+1)^4$.
	\loigiai
	{
		Ta có
		\begin{eqnarray*}
			(x+1)^4&=&x^4+4\cdot x^{3}\cdot 1+6\cdot x^{2}\cdot 1^2+4\cdot x\cdot 1^3+ 1^4\\&=&x^4+4x^3+6x^2+4x+1.
		\end{eqnarray*}
	}
\end{vd}

\begin{vd}%[Lê Nguyễn Viết Tường,BG10-2022]%%[0D8N3-2]
	Khai triển $(x-1)^4$.
	\loigiai
	{
		Ta có
		\begin{eqnarray*}
			(x+1)^4&=&x^4-4\cdot x^{3}\cdot 1+6\cdot x^{2}\cdot 1^2-4\cdot x\cdot 1^3+ 1^4\\&=&x^4-4x^3+6x^2-4x+1.
		\end{eqnarray*}
	}
\end{vd}

\begin{vd}%[Lê Nguyễn Viết Tường,BG10-2022]%%[0D8H3-2]
	Khai triển các biểu thức sau
	\begin{listEX}[2]
		\item $(x-2y)^4$;
		\item $(3x-y)^5$.
	\end{listEX}
	\loigiai
	{
		\begin{listEX}[1]
			\item Ta có
			\begin{eqnarray*}
				(x-2y)^4&=&x^4-4\cdot x^{3}\cdot (2y)+6\cdot x^{2}\cdot (2y)^{2}-4\cdot x\cdot (2y)^{3}+ (2y)^4\\&=&x^4-8x^3y+24x^2y^2-32xy^3+16y^4.
			\end{eqnarray*}
			\item Ta có
			\begin{eqnarray*}
				(3x-y)^5&=&(3x)^5-5\cdot (3x)^{4} y+10\cdot (3x)^{3} y^2-10\cdot (3x)^{2} y^3+5\cdot (3x)y^4- y^5\\&=&243x^5-405x^4y+270x^3y^2-90x^2y^3+15xy^4-y^5.
			\end{eqnarray*}
		\end{listEX}
	}
\end{vd}

\baitaptl
\begin{bt}%[Lê Nguyễn Viết Tường,BG10-2022]%%[0D8H3-2]
	Khai triển các biểu thức sau
	\begin{listEX}[3]
		\item $(2+x)^4$;
		\item $(2-3y)^5$;
		\item $(3x-2y)^4$
	\end{listEX}
	\loigiai
	{
		\begin{listEX}[1]
			\item \begin{eqnarray*}
				(2+x)^4&=& 2^4+4\cdot 2^3x+6\cdot 2^2x^2+4\cdot 2x^3+ x^4\\&=&16+32x+24x^2+8x^3+x^4.
			\end{eqnarray*}
			\item \begin{eqnarray*}
				(2-3y)^5&=& 2^5-5\cdot 2^4\cdot (3y)+10\cdot 2^3\cdot (3y)^2-10\cdot 2^2\cdot (3y)^3+5\cdot 2\cdot (3y)^4-\mathrm{C}_5^5\cdot (3y)^5\\&=&32-240y+720y^2-1080y^3+810y^4-243y^5.
			\end{eqnarray*}
			\item \begin{eqnarray*}
				(3x-2y)^4&=&(3x)^4+4\cdot (3x)^3\cdot (2y)+6\cdot (3x)^2\cdot (2y)^2+4\cdot (3x)\cdot (2y)^3+(2y)^4\\&=&81x^4+216x^3y+216x^2y^2+96xy^3+16y^4.
			\end{eqnarray*}
		\end{listEX}
	}
\end{bt}

\begin{bt}%[Lê Nguyễn Viết Tường,BG10-2022]%%[0D8H3-2]
	Khai triển các biểu thức sau
	\begin{listEX}[2]
		\item $\left (x+\dfrac{1}{2} \right )^4$;
		\item $\left (x-\dfrac{1}{3} \right )^4$.
	\end{listEX}
	\loigiai
	{
		\begin{listEX}[1]
			\item \begin{eqnarray*}
				\left (x+\dfrac{1}{2} \right )^4&=&x^4+4\cdot x^3\cdot \dfrac{1}{2}+6\cdot x^2\cdot \left (\dfrac{1}{2} \right )^2+4\cdot x\cdot \left (\dfrac{1}{2} \right )^3+ \left (\dfrac{1}{2} \right )^4\\&=&x^4+2x^3+\dfrac{3}{2}x^2+\dfrac{1}{2}x+\dfrac{1}{16}.
			\end{eqnarray*}
			\item \begin{eqnarray*}
				\left (x-\dfrac{1}{3} \right )^4&=&x^4-4\cdot x^3\cdot\dfrac{1}{3}+6\cdot x^2\cdot \left (\dfrac{1}{3} \right )^2-4\cdot x\cdot \left (\dfrac{1}{3} \right )^3 \left (\dfrac{1}{3} \right )^4\\&=&x^4-\dfrac{4}{3}x^3+\dfrac{2}{3}x^2-\dfrac{4}{27}x+\dfrac{1}{81}.
			\end{eqnarray*}
		\end{listEX}
	}
\end{bt}

\begin{bt}%[Lê Nguyễn Viết Tường,BG10-2022]%%[0D8H3-2]
	Khai triển đa thức $(x+5)^4+(x-5)^4$.
	\loigiai
	{
		\begin{eqnarray*}
			(x+5)^4+(x-5)^4&=&x^4+20x^3+150x^2+500x+625+x^4-20x^3+150x^2-500x+625\\&=&2\left (x^4+150x^2+625 \right ).
		\end{eqnarray*}
	}
\end{bt}

\begin{bt}%[Lê Nguyễn Viết Tường,BG10-2022]%%[0D8C3-2]
	Số dân của một tỉnh ở thời điểm hiện tại là khoảng $800$ nghìn người. Giả sử rằng tỉ lệ tăng dân số hằng năm của tỉnh đó là $r\%$.
	\begin{enumerate}
		\item Viết công thức tính số dân của tỉnh đó sau $1$ năm, sau $2$ năm. Từ đó suy ra công thức tính số dân của tỉnh đó sau $5$ năm nữa là $P=800\left (1+\dfrac{r}{100} \right )^5$ (nghìn người).
		\item Với $r=1{,}5\%$, dùng hai số hạng đầu trong khai triển của $\left (1+0{,}015 \right )^5$, hãy ước tính số dân của tỉnh đó sau $5$ năm nữa (theo đơn vị nghìn người).
	\end{enumerate}
	\loigiai
	{
		\begin{enumerate}
			\item Số dân của tỉnh đó sau $1$ năm là
			$$P_1=800+800\cdot r\%=800\left (1+\dfrac{r}{100} \right )^1\text{ (nghìn người)}.$$
			Số dân của tỉnh đó sau $2$ năm là $$P_2=P_1+P_1\cdot r\%=800\left (1+\dfrac{r}{100} \right )^1+800\left (1+\dfrac{r}{100} \right )^1\cdot\dfrac{r}{100}=800\left (1+\dfrac{r}{100} \right )^2\text{ (nghìn người)}.$$
			Do đó công thức tính số dân của tỉnh đó sau $5$ năm nữa là $P=800\left (1+\dfrac{r}{100} \right )^5$ (nghìn người).
			\item Với $r=1{,}5\%$, ta có khai triển
			\begin{eqnarray*}
				&&(1+0{,}015)^5\\&=&1^5+5\cdot 1^4\cdot 0{,}015+10\cdot 1^3\cdot (0{,}015)^2+10\cdot 1^2\cdot (0{,}015)^3+5\cdot 1\cdot (0{,}015)^4+(0{,}015)^5\\&\approx &1^5+5\cdot 1^4\cdot 0{,}015=1{,}075.
			\end{eqnarray*}
			Số dân của tỉnh đó sau $5$ năm nữa là $$P=800\cdot (1+0{,}015)^5\approx 800\cdot 1{,}075=860\text{ (nghìn người)}.$$
			Vậy số dân của tỉnh đó sau $5$ năm nữa xấp xỉ khoảng $860$ nghìn người.
		\end{enumerate}
	}
\end{bt}
\begin{dang}{Tìm hệ số số hạng trong khai triển nhị thức Newton}	
	Để tìm số hạng hay hệ số của số hạng trong khai triển nhị thức Newton ta có thể làm theo các cách sau
	\begin{itemize}
		\item Cách 1: Sử dụng tam giác Pascal để khai triển toàn bộ nhị thức rồi tìm số hạng thích hợp. Thường sử dụng cách này với đa thức bậc nhỏ hơn hoặc bằng $5$.
		\item Cách 2: Sử dụng số hạng tổng quát (được giới thiệu ở Chuyên đề học tập Toán 10).\\
		Số hạng tổng quát trong khai triển của $(a+b)^n$ là $\mathrm{C}_n^{k}a^{n-k}b^k$ hay $\mathrm{C}_n^{n-k}a^kb^{n-k}$.\\
		Nếu trong khai triển có chứa $x$, chẳng hạn $(ax+b)^n$ thì ta có số hạng chứa $x^k$ là $\mathrm{C}_n^{n-k}a^kb^{n-k}x^k$. Do đó hệ số của $x^k$ trong khai triển của $(ax+b)^n$ là $\mathrm{C}_n^{n-k}a^kb^{n-k}$.
	\end{itemize}
	Khi tìm hệ số lớn nhất trong khai triển $(a+b)^n$ ta sử dụng nhận xét sau. \\
	Dãy hệ số $\mathrm{C}_n^0$; $\mathrm{C}_n^1$; $\mathrm{C}_n^2$;\ldots ; $\mathrm{C}_n^{n-1}$; $\mathrm{C}_n^n$ trong khai triển $(a+b)^n$ có hai tính chất sau
	\begin{itemize}
		\item Các cặp hệ số tính từ hai đầu trở vào (tương tứng) thì bằng nhau.
		$$\mathrm{C}_n^k=\mathrm{C}_n^{n-k},\forall k\in\mathbb{N}, k\le n, n\in\mathbb{N}^{*}.$$
		\item Dãy hệ số tăng dần đến ``giữa'' rồi giảm dần
		\begin{eqnarray*}
			&& \mathrm{C}_n^0<\mathrm{C}_n^1<\mathrm{C}_n^2<\ldots\\
			&& \ldots >\mathrm{C}_n^{n-2}>\mathrm{C}_n^{n-1}>\mathrm{C}_n^{n}.
		\end{eqnarray*} 
	\end{itemize}
\end{dang}
%=================BẮT ĐẦU VÍ DỤ=================================

\begin{vd}%[An Le - Dự án BG 10]%[SBT Chân Trời]%[0D8H3-3]
%
Khai triển biểu thức $(a+b x)^4$, viết các số hạng theo thứ tự bậc của $x$ tăng dần, nhận được biểu thức gồm hai số hạng đầu tiên là $16-96 x$. Hãy tìm số hạng chứa $x^2$.
\loigiai{
Áp dụng công thức nhị thức Newton, ta có
$$
\begin{aligned}
(a+b x)^4 &=a^4+4 a^3 b x+6 a^2(b x)^2+4 a(b x)^3+(b x)^4 \\
&=a^4+4 a^3 b x+6 a^2 b^2 x^2+4 a b^3 x^3+b^4 x^4.
\end{aligned}
$$
Theo giả thiết, ta có $\heva{& a^4=16 \\ & 4a^3b=-96}\Leftrightarrow\heva{& a=2 \\ & b=-3}$ hoặc $\heva{& a=-2 \\ & b=3}\Rightarrow a^2b^2=4\cdot 9=36$.\\
Vậy số hạng thứ $3$ là $6 a^2 b^2 x^2=6\cdot 36\cdot x^2=216x^2$.
}
\end{vd}

\begin{vd}%[An Le - Dự án BG 10]%[SBT Chân Trời]%[0D8H3-3]
%
Tìm hệ số của $x^4$ trong khai triển biếu thức $(2 x+1)(x-1)^5$.
\loigiai{
Áp dụng công thức nhị thức Newton, ta có
$$(x-1)^5=x^5-5 x^4+10 x^3-10 x^2+5 x-1.\qquad (*)$$
Khi nhân biểu thức $2 x+1$ với biểu thức bên phải của $(*)$, ta được hệ số của $x^4$ bằng $$2\cdot 10+1\cdot (-5)=15.$$
Vậy hệ số của $x^4$ trong khai triển biểu thức $(2 x+1)(x-1)^5$ bằng $15$.\\
Nhận xét: Nếu tìm tất cả các số hạng của khai triển, ta được
$$
\begin{aligned}
(2 x+1)(x-1)^5 &=(2 x+1)\left(x^5-5 x^4+10 x^3-10 x^2+5 x-1\right) \\
&=2 x^6-9 x^5+15 x^4-10 x^3+3 x-1.
\end{aligned}
$$
Từ đó, cũng tìm được hệ số của $x^4$ bằng $15$.
}
\end{vd}

\begin{vd}%[An Le - Dự án BG 10]%[SCD Kết nối]%[0D8H3-4]
%
Tìm hệ số của $x^7$ trong khai triển thành đa thức của $(2-3x)^{10}$.
\loigiai{
Số hạng tổng quát trong khai triển của $(2-3x)^{10}$ là $$\mathrm{C}_{10}^{k}2^{10-k}(-3x)^{k}=\mathrm{C}_{10}^{k}2^{10-k}(-3)^{k}x^k.$$
Số hạng chứa $x^7$ nên $k=7$.\\
Hệ số cần tìm là $$\mathrm{C}_{10}^{k}2^{10-k}(-3)^{k}=\mathrm{C}_{10}^{7}2^{3}(-3)^{7}=-2099520.$$
}
\end{vd}

\begin{vd}%[An Le - Dự án BG 10]%[SCD Chân Trời]%[0D8V3-4]
%
Cho $a$ là một số thực dương. Biết rằng trong khai triển của $(3x+a)^8$, hệ số của $x^4$ là $70$. Tìm giá trị của $a$.
\loigiai{
Số hạng tổng quát trong khai triển $(3x+a)^8$ là $\mathrm{C}_8^k(3x)^{8-k}a^k=\mathrm{C}_8^k3^{8-k}a^kx^{8-k}$.\\
Số hạng chứa $x^4$ thì $8-k=4\Leftrightarrow k=4$.\\
Hệ số của $x^4$ là $\mathrm{C}_8^k3^{8-k}a^k=\mathrm{C}_8^43^4a^4=5670a^4$.\\
Theo giả thiết ta có $5670a^4=70\Leftrightarrow a=\dfrac{1}{3}$.
}
\end{vd}

\begin{vd}%[An Le - Dự án BG 10]%[SCD Cánh diều]%[0D8V3-6]
%
Tìm hệ số lớn nhất trong khai triển của
\begin{enumerate}
\item $(a+b)^6$;
\item $(a+b)^7$.
\end{enumerate}
\loigiai{
\begin{enumerate}
\item Ta có $\mathrm{C}_6^0<\mathrm{C}_6^1<\mathrm{C}_6^2<\mathrm{C}_6^3$ và $\mathrm{C}_6^3>\mathrm{C}_6^4>\mathrm{C}_6^5>\mathrm{C}_6^6$.\\
Vậy hệ số lớn nhất trong khai triển của $(a+b)^6$ là $\mathrm{C}_6^3=20$.
\item Ta có $\mathrm{C}_7^0<\mathrm{C}_7^1<\mathrm{C}_7^2<\mathrm{C}_7^3=\mathrm{C}_7^4$ và $\mathrm{C}_7^4>\mathrm{C}_7^5>\mathrm{C}_7^6>\mathrm{C}_7^7$.\\
Vậy hệ số lớn nhất trong khai triển của $(a+b)^7$ là $\mathrm{C}_7^3=\mathrm{C}_7^4=35$.
\end{enumerate}
}
\end{vd}

\baitaptl

\begin{bt}%[An Le - Dự án BG 10]%[SGK Chân Trời]%[0D8H3-3]
%
Tìm hệ số của $x^3$ trong khai triển $(3x-2)^5$.
\loigiai
{
Áp dụng công thức nhị thức Newton ta có
\begin{eqnarray*}
(3x-2)^5 & = & (3x)^5+5(3x)^4\cdot (-2)+10(3x)^3\cdot (-2)^2+10(3x)^2\cdot (-2)^3+5\cdot 3x\cdot (-2)^4+(-2)^5\\
& = & 243x^5-810x^4+1080x^3-720x^2+240x-32.
\end{eqnarray*}
Vậy hệ số của $x^3$ là $1080$.
}
\end{bt}

\begin{bt}%[An Le - Dự án BG 10]%[SBT Kết nối]%[0D8H3-3]
%
Trong khai triển của $(5x-2)^5$, số mũ của $x$ được sắp xếp theo lũy thừa tăng dần, hãy tìm hạng tử thứ hai tính từ trái sang phải.
\loigiai
{
Áp dụng công thức nhị thức Newton ta có
\begin{eqnarray*}
(5x-2)^5 & = & (5x)^5+5(5x)^4\cdot (-2)+10(5x)^3\cdot (-2)^2+10(5x)^2\cdot (-2)^3+5\cdot 5x\cdot (-2)^4+(-2)^5\\
& = & 3125x^5-6250x^4+5000x^3-2000x^2+400x-32\\
& = & -32+400x-2000x^2+5000x^3-6250x^4+3125x^5.
\end{eqnarray*}
Vậy hạng tử thứ hai là $400x$.
}
\end{bt}

\begin{bt}%[An Le - Dự án BG 10]%[SBT Kết nối]%[0D8V3-3]
%
Xác định hạng tử không chứa $x$ trong khai triển của $\left(x+\dfrac{2}{x}\right)^4$.
\loigiai
{
Áp dụng công thức nhị thức Newton ta có
\begin{eqnarray*}
\left(x+\dfrac{2}{x}\right)^4
& = & x^4+4x^3\cdot\dfrac{2}{x}+6x^2\cdot\left(\dfrac{2}{x}\right)^2+4x\cdot\left(\dfrac{2}{x}\right)^3+\left(\dfrac{2}{x}\right)^4\\
& = & x^4+8x^2+24+\dfrac{32}{x^2}+\dfrac{16}{x^4}.
\end{eqnarray*}
Vậy hạng tử không chứa $x$ là $24$.
}
\end{bt}

\begin{bt}%[An Le - Dự án BG 10]%[SBT Chân Trời]%[0D8V3-3]
%
Tìm giá trị tham số $a$ để trong khai triển $(a+x)(1+x)^4$ có một số hạng là $22 x^2$.
\loigiai
{
Áp dụng công thức nhị thức Newton ta có
$$(1+x)^4 = 1+4x+6x^2+4x^3+x^4.$$
Do đó hệ số của $x^2$ trong khai triển $(a+x)(1+x)^4$ là $a\cdot 6+1\cdot 4=6a+4$.\\
Theo bài ra ta có $6a+4=22\Leftrightarrow a=3$.
}
\end{bt}

\begin{bt}%[An Le - Dự án BG 10]%[SBT Chân Trời]%[0D8V3-3]
%
Cho số thực $a\neq 0$, biết rằng trong khai triển $(a x-1)^5$, hệ số của $x^4$ gấp bốn lần hệ số của $x^2$. Hãy tìm giá trị của tham số $a$.
\loigiai
{
Áp dụng công thức nhị thức Newton ta có
\begin{eqnarray*}
(ax-1)^5 & = & (ax)^5-5(ax)^4+10(ax)^3-10(ax)^2+5\cdot ax-1\\
& = & a^5x^5-5a^4x^4+10a^3x^3-10a^2x^2+5ax-1.
\end{eqnarray*}
Vì hệ số của $x^4$ gấp bốn lần hệ số của $x^2$ nên
$$-5a^4=4\cdot (-10a^2)\Leftrightarrow a^2=8\Leftrightarrow a=\pm 2\sqrt{2}.$$
}
\end{bt}

\begin{bt}%[An Le - Dự án BG 10]%[SBT Chân Trời]%[0D8V3-3]
%
Biết rằng trong khai triển của $\left(a x+\dfrac{1}{x}\right)^4$, số hạng không chứa $x$ là $24$. Hãy tìm giá trị của tham số $a$
\loigiai
{
Áp dụng công thức nhị thức Newton ta có
\begin{eqnarray*}
\left(ax+\dfrac{1}{x}\right)^4
& = & (ax)^4+4(ax)^3\cdot\dfrac{1}{x}+6(ax)^2\cdot\left(\dfrac{1}{x}\right)^2+4(ax)\cdot\left(\dfrac{1}{x}\right)^3+\left(\dfrac{1}{x}\right)^4\\
& = & a^4x^4+4a^3x^2+6a^2+\dfrac{4a}{x^2}+\dfrac{1}{x^4}.
\end{eqnarray*}
Vì số hạng không chứa $x$ là $24$ nên $6a^2=24\Leftrightarrow a=\pm 2$.
}
\end{bt}

\begin{bt}%[An Le - Dự án BG 10]%[SCD Cánh Diều]%[0D8H3-4]
%
Xác định hệ số của
\begin{enumerate}
\item $x^{10}$ trong khai triển của $(x+4)^{20}$;
\item $x^{12}$ trong khai triển của $(3+2x)^{30}$;
\item $x^{15}$ trong khai triển của $\left(\dfrac{2x}{3}-\dfrac{1}{7}\right)^{31}$;
\end{enumerate}
\loigiai
{
\begin{enumerate}
\item Số hạng tổng quát trong khai triển của $(x+4)^{20}$ là $$\mathrm{C}_{20}^{20-k}x^k\cdot 4^{20-k}=\mathrm{C}_{20}^{20-k}4^{20-k}\cdot x^k.$$
Do đó hệ số của $x^{10}$ là $\mathrm{C}_{20}^{10}4^{10}$.
\item Số hạng tổng quát trong khai triển của $(3+2x)^{30}$ là $$\mathrm{C}_{30}^k\cdot 3^{30-k}\cdot (2x)^k=\mathrm{C}_{30}^k\cdot 3^{30-k}\cdot 2^kx^k.$$
Do đó hệ số của $x^{12}$ là $\mathrm{C}_{30}^{12}\cdot 3^{18}\cdot 2^{12}$.
\item Số hạng tổng quát trong khai triển của $\left(\dfrac{2x}{3}-\dfrac{1}{7}\right)^{31}$ là $$\mathrm{C}_{31}^{31-k}\left(\dfrac{2x}{3}\right)^k\left(-\dfrac{1}{7}\right)^{31-k}=\mathrm{C}_{31}^{31-k}\left(\dfrac{2}{3}\right)^k\left(-\dfrac{1}{7}\right)^{31-k}\cdot x^k.$$
Do đó hệ số của $x^{12}$ là $\mathrm{C}_{31}^{19}\left(\dfrac{2}{3}\right)^{12}\left(-\dfrac{1}{7}\right)^{19}=-\dfrac{\mathrm{C}_{31}^{19}\cdot 2^{12}}{3^{12}\cdot 7^{19}}$.
\end{enumerate}
}
\end{bt}

\begin{bt}%[An Le - Dự án BG 10]%[SCD Kết nối]%[0D8V3-4]
%
Tìm hệ số của $x^5$ trong khai triển thành đa thức của biểu thức
$$x(1-2x)^5+x^2(1+3x)^{10}.$$
\loigiai
{
\begin{itemize}
\item Số hạng tổng quát của khai triển $(1-2x)^5$ là $\mathrm{C}_5^k(-2x)^k=\mathrm{C}_5^k(-2)^k\cdot x^k$.\\
Do đó số hạng tổng quát của khai triển $x(1-2x)^5$ là $\mathrm{C}_5^k(-2)^k\cdot x^{k+1}$.\\
Vì số hạng chứa $x^5$ nên $k+1=5\Leftrightarrow k=4$.
\item Số hạng tổng quát của khai triển $(1+3x)^{10}$ là $\mathrm{C}_{10}^i(3x)^i=\mathrm{C}_{10}^i\cdot 3^i\cdot x^i$.\\
Do đó số hạng tổng quát của khai triển $x^2(1+3x)^{10}$ là $\mathrm{C}_{10}^i\cdot 3^i\cdot x^{i+2}$.\\
Vì số hạng chứa $x^5$ nên $i+2=5\Leftrightarrow i=3$.
\end{itemize}
Vậy hệ số của $x^5$ trong khai triển thành đa thức của biểu thức
$x(1-2x)^5+x^2(1+3x)^{10}$ là
$$\mathrm{C}_5^4(-2)^4+\mathrm{C}_{10}^3\cdot 3^3=3320.$$
}
\end{bt}

\begin{bt}%[An Le - Dự án BG 10]%[SCD Chân Trời]%[0D8V3-4]
%
Biết rằng $a$ là một số thực khác $0$ và trong khai triển của $(ax+1)^6$, hệ số của $x^4$ gấp ba lần hệ số của $x^2$. Tìm giá trị của $a$.
\loigiai
{
Số hạng tổng quát trong khai triển của $(ax+1)^6$ là $\mathrm{C}_6^{6-k}(ax)^k=\mathrm{C}_6^{6-k}a^k\cdot x^k$.\\
Do đó hệ số của $x^k$ là $\mathrm{C}_6^{6-k}a^k$.\\
Vì hệ số của $x^4$ gấp ba lần hệ số của $x^2$ và $a\neq 0$ nên
$$\mathrm{C}_6^2a^4=3\mathrm{C}_6^4a^2\Leftrightarrow a=\pm \sqrt{3}.$$
}
\end{bt}

\begin{bt}%[An Le - Dự án BG 10]%[SCD Cánh Diều]%[0D8V3-4]
%
Tìm hệ số lớn nhất trong khai triển của
\begin{enumerate}
\item $(a+b)^8$;
\item $(a+b)^9$.
\end{enumerate}
\loigiai
{
\begin{enumerate}
\item Ta có $\mathrm{C}_8^0<\mathrm{C}_8^1<\mathrm{C}_8^2<\mathrm{C}_8^3<\mathrm{C}_8^4$ và $\mathrm{C}_8^4>\mathrm{C}_8^5>\mathrm{C}_8^6>\mathrm{C}_8^7>\mathrm{C}_8^8$.\\
Vậy hệ số lớn nhất trong khai triển của $(a+b)^8$ là $\mathrm{C}_8^4=70$.
\item Ta có $\mathrm{C}_9^0<\mathrm{C}_9^1<\mathrm{C}_9^2<\mathrm{C}_9^3<\mathrm{C}_9^4=\mathrm{C}_9^5$ và $\mathrm{C}_9^5>\mathrm{C}_9^6>\mathrm{C}_9^7>\mathrm{C}_9^8>\mathrm{C}_9^9$.\\
Vậy hệ số lớn nhất trong khai triển của $(a+b)^9$ là $\mathrm{C}_9^4=\mathrm{C}_9^5=126$.
\end{enumerate}
}
\end{bt}

\begin{bt}%[An Le - Dự án BG 10]%[SCD Cánh Diều]%%[0D8C3-6]
%
Biết rằng $(2+x)^{100}=a_0+a_1x+a_2x^2+\ldots+a_{100}x^{100}$. Với giá trị nào của $k$ ($0\le k\le 100$) thì $a_k$ lớn nhất.
\loigiai
{
Số hạng tổng quát trong khai triển $(2+x)^{100}$ là $\mathrm{C}_{100}^k\cdot 2^{100-k}\cdot x^k$.\\
Do đó $a_k=\mathrm{C}_{100}^k\cdot 2^{100-k}$.
\begin{itemize}
\item Xét $a_k\le a_{k+1}$
\begin{eqnarray*}
& \Leftrightarrow & \mathrm{C}_{100}^k\cdot 2^{100-k}\le \mathrm{C}_{100}^{k+1}\cdot 2^{99-k}\\
& \Leftrightarrow & \dfrac{100!}{(100-k)!k!}\cdot 2^{100-k}\le \dfrac{100!}{(99-k)!(k+1)!}\cdot 2^{99-k}\\
& \Leftrightarrow & \dfrac{2}{100-k}\le \dfrac{1}{k+1}\Leftrightarrow k\le \dfrac{98}{3}.
\end{eqnarray*}
Các giá trị $k$ thỏa mãn là $\{0;1;2;\ldots;32\}$.\\
Do đó $a_0<a_1<a_2<\ldots<a_{33}$.
\item Xét $a_k> a_{k+1}$, ta được $k> \dfrac{98}{3}$.
Các giá trị $k$ thỏa mãn là $\{33;34;35;\ldots;99\}$.\\
Do đó $a_{33}>a_{34}>\ldots>a_{100}$.\\
Vậy với $k=33$ thì $a_k$ lớn nhất.

\end{itemize}
}
\end{bt}
\begin{dang}{Chứng minh, tính giá trị của biểu thức tổ hợp có sử dụng khai triển nhị thức Newton.}
	\begin{itemize}
		\item \textbf{\textit{Phương pháp:}} Sử dụng khai triển nhị thức Newton tổng quát $\left(a+b\right)^n=
		\displaystyle\sum\limits_{k=0}^{n}{\mathrm{C}_n^ka^{n-k}b^k}$, sau đó thay thế các giá trị $a$ và $b$ thích hợp.
		\item \textbf{\textit{Một số hệ thức thường gặp:}}
		\begin{enumerate}
			\item $\mathrm{C}_n^k=\mathrm{C}_n^{n-k}$.
			\item $\mathrm{C}_{n-1}^{k-1}+\mathrm{C}_{n-1}^{k}=\mathrm{C}_n^k$.
			\item $\mathrm{C}_n^0+\mathrm{C}_n^1+\ldots +\mathrm{C}_n^k+\ldots +\mathrm{C}_n^n=2^n$. 
			\item $\mathrm{C}_n^0-\mathrm{C}_n^1+\ldots +(-1)^k\mathrm{C}_n^k+\ldots +(-1)^n\mathrm{C}_n^n=0$.
			\item $\mathrm{C}_{2n}^0+\mathrm{C}_{2n}^2+\mathrm{C}_{2n}^4+..+\mathrm{C}_{2n}^{2n}=2^{2n-1}$.
			\item $\mathrm{C}_{2n}^1+\mathrm{C}_{2n}^3+\mathrm{C}_{2n}^5+\ldots .+\mathrm{C}_{2n}^{2n-1}=2^{2n-1}$.
		\end{enumerate}
	\end{itemize}
\end{dang}

\begin{vd}%[Đỗ Vũ Minh Thắng,BG10-2022]%%[0D8V3-5]
Với $n$ là số nguyên dương, chứng minh rằng $1+4\mathrm{C}_n^1+4^2\mathrm{C}_n^2+\ldots+4^n\mathrm{C}_n^n=5^n$.
\loigiai{
Áp dụng khai triển nhị thức ta có\\
$VT=1+4\mathrm{C}_n^1+4^2\mathrm{C}_n^2+\ldots +4^n\mathrm{C}_n^n=\mathrm{C}_n^0+4\mathrm{C}_n^1+4^2\mathrm{C}_n^2+\ldots +4^n\mathrm{C}_n^n=(1+4)^n=5^n=VP$.
}
\end{vd}
\begin{vd}%[Đỗ Vũ Minh Thắng,BG10-2022]%%[0D8C3-5]
Với $n$ là số nguyên dương, chứng minh rằng
\begin{center}
$4^n\mathrm{C}_n^0-4^{n-1}\mathrm{C}_n^1+4^{n-2}\mathrm{C}_n^2+\ldots +(-1)^n\mathrm{C}_n^n=\mathrm{C}_n^0+2\mathrm{C}_n^1+2^2\mathrm{C}_n^2+\ldots +2^n\mathrm{C}_n^n$.
\end{center}
\loigiai{
Trong khai triển nhị thức Newton dạng tổng quát\\
$\circ$ Chọn $a=4,b=-1\Rightarrow $ $4^n\mathrm{C}_n^0-4^{n-1}\mathrm{C}_n^1+4^{n-2}\mathrm{C}_n^2+\ldots +(-1)^n\mathrm{C}_n^n=3^n$.\\
$\circ$ Chọn $a=1,b=2\Rightarrow $ $\mathrm{C}_n^0+2\mathrm{C}_n^1+2^2\mathrm{C}_n^2+\ldots +2^n\mathrm{C}_n^n=3^n$.\\
Từ đó ta có điều phải chứng minh.
}
\end{vd}

\begin{vd}%[Đỗ Vũ Minh Thắng,BG10-2022]%%[0D8V3-5]
Tính tổng $S=2^{18}\mathrm{C}_{18}^0-2^{17}\mathrm{C}_{18}^1+2^{16}\mathrm{C}_{18}^2-\ldots +\mathrm{C}_{18}^{18}$.
\loigiai{
Các số hạng của tổng đều có dạng: $\mathrm{C}_{18}^k 2^{18-k}(-1)^k$.\\ Do đó:
$S=2^{18}\mathrm{C}_{18}^0-2^{17}\mathrm{C}_{18}^1+2^{16}\mathrm{C}_{18}^2-\ldots +\mathrm{C}_{18}^{18}=\displaystyle\sum\limits_{k=0}^{18}{\mathrm{C}_{18}^k 2^{18-k}(-1)^k}=(2-1)^{18}=1$.
}
\end{vd}

\begin{vd}%[Đỗ Vũ Minh Thắng,BG10-2022]%%[0D8C3-5]
Tính tổng $S=\mathrm{C}_{10}^02^{11}3^1+\mathrm{C}_{10}^12^{10}3^2+\mathrm{C}_{10}^{2}2^93^3+\ldots +\mathrm{C}_{10}^{9}2^23^{10}+\mathrm{C}_{10}^{10}2^13^{11}$.
\loigiai{
$S=6\left(\mathrm{C}_{10}^02^{10}+\mathrm{C}_{10}^12^{9}3+\mathrm{C}_{10}^{2}2^83^2+\ldots +\mathrm{C}_{10}^{9}23^{9}+\mathrm{C}_{10}^{10}3^{10}\right)=6(2+3)^{10}=6\cdot 5^{10}$.
}
\end{vd}
\baitaptl
\begin{bt}%[Đỗ Vũ Minh Thắng,BG10-2022]%%[0D8V3-5]
Chứng minh
\begin{enumEX}{1}
\item $\mathrm{C}_{2n}^0 + \mathrm{C}_{2n}^1 + \mathrm{C}_{2n}^2 + \mathrm{C}_{2n}^3 + \cdots + \mathrm{C}_{2n}^{2n-1} + \mathrm{C}_{2n}^{2n}=4^n$.
\item $\mathrm{C}_n^0\cdot3^n - \mathrm{C}_n^1\cdot3^{n-1} + \cdots + (-1)^n\mathrm{C}_n^n=\mathrm{C}_n^0 + \mathrm{C}_n^1 + \cdots +  \mathrm{C}_n^n$.
\end{enumEX}
\loigiai{
\begin{enumEX}{1}
\item Xét nhị thức $$(x+1)^{2n} = \mathrm{C}_{2n}^0 x^{2n} + \mathrm{C}_{2n}^1 x^{2n-1} + \mathrm{C}_{2n}^2 x^{2n-2} + \mathrm{C}_{2n}^3 x^{2n-3} + \cdots + \mathrm{C}_{2n}^{2n-1} x + \mathrm{C}_{2n}^{2n}.$$
Thay $x=1$ ta được $$\mathrm{C}_{2n}^0 + \mathrm{C}_{2n}^1 + \mathrm{C}_{2n}^2 + \mathrm{C}_{2n}^3 + \cdots + \mathrm{C}_{2n}^{2n-1} + \mathrm{C}_{2n}^{2n} = 2^{2n}.$$
Vậy $\mathrm{C}_{2n}^0 + \mathrm{C}_{2n}^1 + \mathrm{C}_{2n}^2 + \mathrm{C}_{2n}^3 + \cdots + \mathrm{C}_{2n}^{2n-1} + \mathrm{C}_{2n}^{2n} = 4^n$.
\item Xét nhị thức $(x-1)^n = \mathrm{C}_n^0 x^n - \mathrm{C}_n^1 x^{n-1} + \cdots + (-1)^n\mathrm{C}_n^n$.\\
Thay $x=3$ ta được $$\mathrm{C}_n^0\cdot3^n - \mathrm{C}_n^1\cdot3^{n-1} + \cdots + (-1)^n\mathrm{C}_n^n = 2^n.$$
Lại có $\mathrm{C}_n^0 + \mathrm{C}_n^1 + \cdots +  \mathrm{C}_n^n = 2^n$.\\
Vậy $\mathrm{C}_n^0\cdot3^n - \mathrm{C}_n^1\cdot3^{n-1} + \cdots + (-1)^n\mathrm{C}_n^n=\mathrm{C}_n^0 + \mathrm{C}_n^1 + \cdots +  \mathrm{C}_n^n$.
\end{enumEX}
}
\end{bt}
\begin{bt}%[Đỗ Vũ Minh Thắng,BG10-2022]%%[0D8V3-5]
Tính các tổng sau
\begin{enumEX}{1}
\item $S=\mathrm{C}_5^0 + \mathrm{C}_5^1 + \mathrm{C}_5^2 + \cdots + \mathrm{C}_5^5$.
\item $S=2\mathrm{C}_{2010}^1 + 2^3\mathrm{C}_{2010}^3 + 2^5\mathrm{C}_{2010}^5 + \cdots + 2^{2009}\mathrm{C}_{2010}^{2009}$.
\end{enumEX}
\loigiai{
\begin{enumEX}{1}
\item Ta có $S=\mathrm{C}_5^0 + \mathrm{C}_5^1 + \mathrm{C}_5^2 + \cdots + \mathrm{C}_5^5 = 2^5 =32$.
\item Xét nhị thức $$(1+x)^{2010} = \mathrm{C}_{2010}^0 + \mathrm{C}_{2010}^1 x + \mathrm{C}_{2010}^2 x^2 + \mathrm{C}_{2010}^3 x^3 + \cdots + \mathrm{C}_{2010}^{2009} x^{2009} + \mathrm{C}_{2010}^{2010} x^{2010}.$$
Thay $x=2$ ta được $$\mathrm{C}_{2010}^0 + 2\mathrm{C}_{2010}^1 + 2^2\mathrm{C}_{2010}^2 + 2^3\mathrm{C}_{2010}^3 + \cdots + 2^{2009}\mathrm{C}_{2010}^{2009} + 2^{2010}\mathrm{C}_{2010}^{2010} = 3^{2010}. \qquad \qquad (1)$$
Thay $x=-2$ ta được $$\mathrm{C}_{2010}^0 - 2\mathrm{C}_{2010}^1 + 2^2\mathrm{C}_{2010}^2 - 2^3\mathrm{C}_{2010}^3 + \cdots - 2^{2009}\mathrm{C}_{2010}^{2009} + 2^{2010}\mathrm{C}_{2010}^{2010} = 1. \quad \quad \qquad \qquad (2)$$
Trừ hai vế $(1)$ và $(2)$ suy ra $$2\left(2\mathrm{C}_{2010}^1 + 2^3\mathrm{C}_{2010}^3 + 2^5\mathrm{C}_{2010}^5 + \cdots + 2^{2009}\mathrm{C}_{2010}^{2009}\right) = 3^{2010}-1.$$
Vậy $S=\dfrac{3^{2010}-1}{2}$.
\end{enumEX}
}
\end{bt}
\begin{bt}%[Đỗ Vũ Minh Thắng,BG10-2022]%%[0D8V3-5]
Tính tổng $S=\mathrm{C}_{15}^8+\mathrm{C}_{15}^9+\mathrm{C}_{15}^{10}+\ldots +\mathrm{C}_{15}^{15}$.
\loigiai{
Sử dụng tính chất $\mathrm{C}_n^k=\mathrm{C}_n^{n-k}$, ta có\\
$$2S=\mathrm{C}_{15}^0+\mathrm{C}_{15}^1+\mathrm{C}_{15}^2+\ldots +\mathrm{C}_{15}^{8}+\ldots +\mathrm{C}_{15}^{15}=2^{15}\Rightarrow S=2^{14}$$
}
\end{bt}
\begin{bt}%[Lê Nguyễn Viết Tường,BG10-2022]%%[0D8V3-5]
Tính tổng
\begin{enumerate}
\item $S=\mathrm{C}_5^0+2\mathrm{C}_5^1+2^2\mathrm{C}_5^2+\cdots +2^5\mathrm{C}_5^5$.
\item $S=4^0\mathrm{C}_8^0+4^1\mathrm{C}_8^1+4^2\mathrm{C}_8^2+\cdots +4^8\mathrm{C}_8^8$.
\end{enumerate}
\loigiai
{
Xét khai triển $(a+b)^n=\displaystyle\sum\limits_{k=0}^{n}\mathrm{C}_n^ka^{n-k}b^k$.\\
Trong khai triển trên, số mũ của $a$ giảm dần từ $n$ đến $0$ và số mũ của $b$ tăng dần từ $0$ đến $n$.
\begin{enumerate}
\item Thay $a=1,b=2,n=5$ vào khai triển $(a+b)^n$ ta được $S=(1+2)^5=3^5$.
\item Thay $a=1,b=4,n=8$ vào khai triển $(a+b)^n$ ta được $S=(1+4)^8=5^8$.
\end{enumerate}
}
\end{bt}
\begin{bt}%[Đỗ Vũ Minh Thắng,BG10-2022]%%[0D8V3-5]
Với $n$ là số nguyên dương, chứng minh
$\mathrm{C}_n^1+2\mathrm{C}_n^2+3\mathrm{C}_n^3+\ldots +n\mathrm{C}_n^n=n\cdot 2^{n-1}$.
\loigiai{
Đặt $S=\mathrm{C}_n^1+2\mathrm{C}_n^2+3\mathrm{C}_n^3+\ldots +n\mathrm{C}_n^n\ (*)$.\\
Áp dụng hệ thức $\mathrm{C}_n^k=\mathrm{C}_n^{n-k}$. Ta có
$$\begin{aligned}
\mathrm{C}_n^1&=&\mathrm{C}_n^{n-1}\\
2\mathrm{C}_n^2&=&2\mathrm{C}_n^{n-2}\\
\ldots &\ldots &\\
(n-1)\mathrm{C}_n^{n-1}&=&(n-1)\mathrm{C}_n^{1}\\
n\mathrm{C}_n^n&=&n\mathrm{C}_n^0
\end{aligned}$$
Cộng vế với vế ta được $S=\mathrm{C}_n^{n-1}+2\mathrm{C}_n^{n-2}+\ldots +(n-1)\mathrm{C}_n^1+n\mathrm{C}_n^0\ (**)$.\\
Từ $(*), (**)$ ta có $2S=n\left(\mathrm{C}_n^0+\mathrm{C}_n^1+\ldots +\mathrm{C}_n^{n-1}+\mathrm{C}_n^n\right)=n\cdot 2^n$.\\
Từ đó suy ra điều phải chứng minh.
}
\end{bt}
\begin{bt}%[Đỗ Vũ Minh Thắng,BG10-2022]%%[0D8C3-5]
Chứng minh rằng $\mathrm{C}_{2022}^0+2^2\mathrm{C}_{2022}^2+\ldots +2^{2022}\mathrm{C}_{2022}^{2022}=\dfrac{3^{2022}+1}{2}$
\loigiai{
$(1+x)^{2022}=\displaystyle\sum\limits_{k=0}^{2022}{\mathrm{C}_{2022}^kx^k}$.\\
$(1-x)^{2022}=\displaystyle\sum\limits_{k=0}^{2022}{\mathrm{C}_{2022}^k(-x)^k}$.\\
Cộng vế với vế ta được $(1+x)^{2022}+(1-x)^{2022}=2\left(\mathrm{C}_{2022}^0+\mathrm{C}_{2022}^2x^2+\ldots +\mathrm{C}_{2022}^{2022}x^{2022}\right)$.\\
Chọn $x=2$ ta có $\mathrm{C}_{2022}^0+2^2\mathrm{C}_{2022}^2+\ldots +2^{2022}\mathrm{C}_{2022}^{2022}=\dfrac{3^{2022}+1}{2}$.
}
\end{bt}
\begin{bt}%[Đỗ Vũ Minh Thắng,BG10-2022]%%[0D8C3-5]
Với $p,a,b$ là các số nguyên dương và $p\le a,b$. Chứng minh rằng
\begin{center}
$\displaystyle\mathrm{C}_a^p+\mathrm{C}_a^{p-1}\mathrm{C}_b^1+\mathrm{C}_a^{p-2}\mathrm{C}_b^2+\ldots +\mathrm{C}_a^{p-q}\mathrm{C}_b^q+\ldots +\mathrm{C}_b^p=\mathrm{C}_{a+b}^p$.
\end{center}
\loigiai{
Xét hai khai triển
$$(1+x)^a=\mathrm{C}_a^0+\mathrm{C}_a^1x+\mathrm{C}_a^2x^2\ldots +\mathrm{C}_a^ax^a.$$
$$(1+x)^b=\mathrm{C}_b^0+\mathrm{C}_b^1x+\mathrm{C}_b^2x^2\ldots +\mathrm{C}_b^bx^b.$$
Suy ra $ (1+x)^{a+b}=M+\left(\mathrm{C}_a^p+\mathrm{C}_a^{p-1}\mathrm{C}_b^1+\ldots +\mathrm{C}_a^{p-q}\mathrm{C}_b^q+\ldots +\mathrm{C}_b^p\right)x^p\ (*)$.\\
Với $M$ là một đa thức không chứa $x^p$.\\
Mặt khác: $(1+x)^{a+b}=\mathrm{C}_{a+b}^0+\mathrm{C}_{a+b}^1x+\mathrm{C}_{a+b}^2x^2+\ldots +\mathrm{C}_{a+b}^{p}x^p+\ldots +\mathrm{C}_{a+b}^{a+b}x^{a+b}\ (**)$.\\
Đồng nhất hệ số ở $(*), (**)$.
Ta có điều phải chứng minh.
}
\end{bt}
\begin{bt}%[Đỗ Vũ Minh Thắng,BG10-2022]%%[0D8C3-5]
Với $n$ là số nguyên dương, chứng minh rằng: $\left(\mathrm{C}_n^0\right)^2+\left(\mathrm{C}_n^1\right)^2+\ldots +\left(\mathrm{C}_n^n\right)^2=\left(\mathrm{C}_{2n}^n\right)^2$.
\loigiai{
Áp dụng kết quả $\displaystyle\mathrm{C}_a^p+\mathrm{C}_a^{p-1}\mathrm{C}_b^1+\mathrm{C}_a^{p-2}\mathrm{C}_b^2+\ldots +\mathrm{C}_a^{p-q}\mathrm{C}_b^q+\ldots +\mathrm{C}_b^p=\mathrm{C}_{a+b}^p$ với $p=a=b=n$. Ta có điều phải chứng minh.
}
\end{bt}

\begin{bt}%[Đỗ Vũ Minh Thắng,BG10-2022]%%[0D8H3-5]
Tính tổng: $S=3^{2019}-\mathrm{C}_{2019}^13^{2018}\cdot 4+\mathrm{C}_{2019}^23^{2017}\cdot 4^2-\ldots +\mathrm{C}_{2019}^{2018}3\cdot 4^{2018}-4^{2019}$
\loigiai{
$S=3^{2019}-\mathrm{C}_{2019}^13^{2018}\cdot 4+\mathrm{C}_{2019}^23^{2017}\cdot 4^2-\ldots +\mathrm{C}_{2019}^{2018}3\cdot 4^{2018}-4^{2019}=(3-4)^{2019}=-1$.
}
\end{bt}

\begin{bt}%[Đỗ Vũ Minh Thắng,BG10-2022]%%[0D8H3-5]
Tính tổng $S=\mathrm{C}_{2004}^0+2^2\mathrm{C}_{2004}^1+\ldots +2^{2005}\mathrm{C}_{2004}^{2004}$.
\loigiai{ Ta có\\
$\begin{aligned}
S&=\mathrm{C}_{2004}^0+2^2\mathrm{C}_{2004}^1+\ldots +2^{2005}\mathrm{C}_{2004}^{2004}\\&=2\left(\mathrm{C}_{2004}^0+2\mathrm{C}_{2004}^1+\ldots +2^{2004}\mathrm{C}_{2004}^{2004}\right)-1=2.3^{2004}-1
\end{aligned}
$
}
\end{bt}
\begin{bt}%[Đỗ Vũ Minh Thắng,BG10-2022]%%[0D8H3-5]
Tính tổng $S=\mathrm{C}_{2018}^0+3^2\mathrm{C}_{2018}^2+3^4\mathrm{C}_{2018}^4+\ldots +3^{2018}\mathrm{C}_{2018}^{2018}$.
\loigiai{
Vẫn sử dụng nhị thức với $a=1; b=3$ và $n=2018$.\\
Các số hạng của tổng đều có dạng: $\mathrm{C}_{2018}^k 1^{2018-k}3^k$ với $k$ chẵn. Do đó ta triệt tiêu số hạng \lq\lq  lẻ\rq\rq\ bằng cách bổ sung nhị thức với $a=-1;b=3$ và $n=2018$. Khi đó:\\
$\circ$ $S_1=\mathrm{C}_{2018}^0+3^1\mathrm{C}_{2018}^1+3^2\mathrm{C}_{2018}^2+\ldots +3^{2018}\mathrm{C}_{2018}^{2018}=4^{2018}$.\\
$\circ$ $S_1=\mathrm{C}_{2018}^0-3^1\mathrm{C}_{2018}^1+3^2\mathrm{C}_{2018}^2-\ldots +3^{2018}\mathrm{C}_{2018}^{2018}=2^{2018}$.\\
$S=\dfrac{S_1+S_2}{2}=\dfrac{4^{2018}+2^{2018}}{2}$.\\
\textbf{Tổng quát: } Với $a\ne 0$.
\begin{itemize}
\item $S=\mathrm{C}_{2n}^0+a^2\mathrm{C}_{2n}^2+a^4\mathrm{C}_{2n}^4+\ldots +a^{2n}\mathrm{C}_{2n}^{2n}=\dfrac{(a+1)^{2n}+(a-1)^{2n}}{2}$.
\item $S=a\mathrm{C}_{2n}^1+a^3\mathrm{C}_{2n}^3+a^5\mathrm{C}_{2n}^5+\ldots +a^{2n-1}\mathrm{C}_{2n}^{2n-1}=\dfrac{(a+1)^{2n}-(a-1)^{2n}}{2}$.
\end{itemize}
}
\end{bt}



\begin{bt}%[Lê Nguyễn Viết Tường,BG10-2022]%%[0D8V3-5]
Chứng minh
\begin{enumerate}
\item $\mathrm{C}_{2n}^0 - \mathrm{C}_{2n}^1 + \mathrm{C}_{2n}^2 - \mathrm{C}_{2n}^3 + \cdots - \mathrm{C}_{2n}^{2n-1} + \mathrm{C}_{2n}^{2n}=0$.
\item $3^{16}\mathrm{C}_{16}^0 - 3^{15}\mathrm{C}_{16}^1 + 3^{14}\mathrm{C}_{16}^2 -  \cdots 3\mathrm{C}_{16}^{15} + \mathrm{C}_{16}^{16}=2^{16}$.
\item $\mathrm{C}_{2n}^0 + \mathrm{C}_{2n}^2\cdot3^2 + \mathrm{C}_{2n}^4\cdot3^4 + \cdots + \mathrm{C}_{2n}^{2n}\cdot3^{2n}=2^{2n-1}\cdot(2^{2n}+1)$.
\end{enumerate}
\loigiai
{
\begin{enumerate}
\item Ta có $$
VP=0^{2n}=(1-1)^{2n}=\mathrm{C}_{2n}^0 - \mathrm{C}_{2n}^1 + \mathrm{C}_{2n}^2 - \mathrm{C}_{2n}^3 + \cdots - \mathrm{C}_{2n}^{2n-1} + \mathrm{C}_{2n}^{2n}=VT.$$
\item Ta có
$$VP=(3-1)^{16}=3^{16}\mathrm{C}_{16}^0 - 3^{15}\mathrm{C}_{16}^1 + 3^{14}\mathrm{C}_{16}^2 -  \cdots 3\mathrm{C}_{16}^{15} + \mathrm{C}_{16}^{16}=VT.$$
\item Ta có
\begin{eqnarray*}
&&4^{2n}=(3+1)^{2n}=\mathrm{C}_{2n}^0+3^1\mathrm{C}_{2n}^1+3^2\mathrm{C}_{2n}^2+\cdots +3^{2n-1}\mathrm{C}_{2n}^{2n-1}+3^{2n}\mathrm{C}_{2n}^{2n}.\\
&&2^{2n}=(3-1)^{2n}=\mathrm{C}_{2n}^0-3^1\mathrm{C}_{2n}^1+3^2\mathrm{C}_{2n}^2-\cdots -3^{2n-1}\mathrm{C}_{2n}^{2n-1}+3^{2n}\mathrm{C}_{2n}^{2n}.\\
&\Rightarrow &4^{2n}+2^{2n}=2\left ( \mathrm{C}_{2n}^0 + \mathrm{C}_{2n}^2\cdot3^2 + \mathrm{C}_{2n}^4\cdot3^4 + \cdots + \mathrm{C}_{2n}^{2n}\cdot3^{2n}\right )\\&\Rightarrow &VT=\dfrac{4^{2n}+2^{2n}}{2}=2^{2n-1}\cdot\left (2^{2n}+1 \right )=VP.
\end{eqnarray*}
\end{enumerate}
}
\end{bt}

\begin{bt}%[Lê Nguyễn Viết Tường,BG10-2022]%%[0D8V3-5]
Tìm số nguyên dương $n$ thỏa mãn
\begin{enumerate}
\item $\mathrm{C}_{2n}^0+\mathrm{C}_{2n}^2+\mathrm{C}_{2n}^4+\cdots +\mathrm{C}_{2n}^{2n}=512$.
\item $\mathrm{C}_{2n+1}^1+\mathrm{C}_{2n+1}^3+\mathrm{C}_{2n+1}^5+\cdots +\mathrm{C}_{2n+1}^{2n+1}=1024$.
\end{enumerate}
\loigiai
{
\begin{enumerate}
\item Ta có
\begin{eqnarray*}
&&2^{2n}=(1+1)^{2n}=\mathrm{C}_{2n}^0+\mathrm{C}_{2n}^1+\mathrm{C}_{2n}^2+\cdots +\mathrm{C}_{2n}^{2n-1}+\mathrm{C}_{2n}^{2n}.\\
&&0^{2n}=(1-1)^{2n}=\mathrm{C}_{2n}^0-\mathrm{C}_{2n}^1+\mathrm{C}_{2n}^2-\cdots -\mathrm{C}_{2n}^{2n-1}+\mathrm{C}_{2n}^{2n}.\\
&\Rightarrow &2^{2n}+0^{2n}=2\left ( \mathrm{C}_{2n}^0+\mathrm{C}_{2n}^2+\mathrm{C}_{2n}^4+\cdots +\mathrm{C}_{2n}^{2n}\right )\\&\Rightarrow &2^{2n}=2\cdot 512\Rightarrow 2^{2n}=2^{10}\Rightarrow 2n=10\Rightarrow n=5.
\end{eqnarray*}
Vậy $n=5$ thỏa yêu cầu bài toán.
\item Ta có
\begin{eqnarray*}
&&2^{2n+1}=(1+1)^{2n+1}=\mathrm{C}_{2n+1}^0+\mathrm{C}_{2n+1}^1+\mathrm{C}_{2n+1}^2+\cdots +\mathrm{C}_{2n+1}^{2n}+\mathrm{C}_{2n+1}^{2n+1}.\\
&&0^{2n+1}=(1-1)^{2n+1}=\mathrm{C}_{2n+1}^0-\mathrm{C}_{2n+1}^1+\mathrm{C}_{2n+1}^2-\cdots +\mathrm{C}_{2n+1}^{2n}-\mathrm{C}_{2n+1}^{2n+1}.\\
&\Rightarrow &2^{2n+1}-0^{2n+1}=2\cdot\left ( \mathrm{C}_{2n+1}^1+\mathrm{C}_{2n+1}^3+\mathrm{C}_{2n+1}^5+\cdots +\mathrm{C}_{2n+1}^{2n+1}\right )\\&\Rightarrow &2^{2n+1}=2\cdot 1024\Rightarrow 2^{2n+1}=2^{11}\Rightarrow 2n+1=11\Rightarrow n=5.
\end{eqnarray*}
Vậy $n=5$ thỏa yêu cầu bài toán.
\end{enumerate}
}
\end{bt}






	\subsection{Bài tập trắc nghiệm}	
	\Opensolutionfile{ansbook}[ans/ansbook-Nhom11-25-TN]
	\Opensolutionfile{ans}[ans/ans-Nhom11-25-TN]
	\begin{ex}%[Lê Nguyễn Viết Tường,BG10-2022]%%[0D8V3-4]
Cho biết $2\mathrm{C}_n^2-3\mathrm{A}_n^1=5(n+2)$ hỏi khai triển $(2x-1)^{n+1}$ có bao nhiêu số hạng?
\choice{$11$}{\True $12$}{$10$}{$9$}
\loigiai{
Điều kiện: $n\in\mathbb{N}, n\ge 2$.\\
Phương trình tương đương
$$n(n-1)-3n=5(n+2)\Leftrightarrow n=10.$$
Khi đó $(2n-1)^{11}$ có tất cả $12$ số hạng.
}
\end{ex}
	
	\begin{ex}%[Lê Nguyễn Viết Tường,BG10-2022]%%[0D8H3-4]
Số hạng tổng quát trong khai triển biểu thức $\left(x-\dfrac{2}{x^2}\right)^{15}$, $x\neq 0$ là
\choice
{\True $(-2)^k\,\mathrm{C}_{15}^{k}\,x^{15-3k}$}
{$2^k\,\mathrm{C}_{15}^{k}\,x^{15-3k}$}
{$2^k\,\mathrm{C}_{15}^{k}\,x^{15-2k}$}
{$(-2)^k\,\mathrm{C}_{15}^{k}\,x^{15-2k}$}
\loigiai{
Ta có $\left(x-\dfrac{2}{x^2}\right)^{15}=
\displaystyle\sum_{k=0}^{15}\mathrm{C}_{15}^{k}\,x^{15-k}\left(-\dfrac{2}{x^2}\right)^k
=\sum_{k=0}^{15}\mathrm{C}_{15}^{k}(-2)^k\,x^{15-3k}$.\\
Do đó, số hạng tổng quát trong khai triển trên là $\mathrm{C}_{15}^{k}(-2)^k\,x^{15-3k}$.
}
\end{ex}
	
	\begin{ex}%[Lê Nguyễn Viết Tường,BG10-2022]%%[0D8H3-3]
Khai triển nhị thức $(x-2)^4$ ta được biểu thức nào sau đây?
\choice
{$-x^4+8x^3-24x^2+32x-16$}
{$x^4+8x^3+24x^2+32x+16$}
{\True $x^4-8x^3+24x^2-32x+16$}
{$x^4+8x^3-24x^2+32x-16$}
\loigiai{Ta có $\displaystyle (x-2)^4 =\sum \limits_{k=0}^4 \mathrm{C}_4^k \cdot x^{4-k} \cdot (-2)^k =x^4-8x^3+24x^2-32x+16.$}
\end{ex}
	
	\begin{ex}%[Lê Nguyễn Viết Tường,BG10-2022]%%[0D8V3-3]
Biểu diễn $\left (3+\sqrt{2} \right )^5-\left (3-\sqrt{2} \right )^5$ dưới dạng $a+b\sqrt{2}$ với $a,b\in\mathbb{Z}$. Giá trị của biểu thức $M=a+b$ là
\choice
{$1177$}
{\True  $1178$}
{$1179$}
{$1180$}
\loigiai
{
Ta có
\begin{eqnarray*}
\left (3+\sqrt{2} \right )^5&=&3^5+5\cdot 3^4\cdot\sqrt{2}+10\cdot 3^3\cdot\left (\sqrt{2} \right )^2+10\cdot 3^2\cdot\left (\sqrt{2} \right )^3+5\cdot 3\cdot\left (\sqrt{2} \right )^4+\left (\sqrt{2} \right )^5\\&=&243+405\sqrt{2}+540+180\sqrt{2}+60+4\sqrt{2}\\
&=&843+589\sqrt{2}.
\end{eqnarray*}
\begin{eqnarray*}
\left (3-\sqrt{2} \right )^5&=&3^5-5\cdot 3^4\cdot\sqrt{2}+10\cdot 3^3\cdot\left (\sqrt{2} \right )^2-10\cdot 3^2\cdot\left (\sqrt{2} \right )^3+5\cdot 3\cdot\left (\sqrt{2} \right )^4-\left (\sqrt{2} \right )^5\\&=&243-405\sqrt{2}+540-180\sqrt{2}+60-4\sqrt{2}\\
&=&843-589\sqrt{2}.
\end{eqnarray*}
$\Rightarrow \left (3+\sqrt{2} \right )^5-\left (3-\sqrt{2} \right )^5=1178\sqrt{2}$.\\
Vậy $M=0+1178=1178$.
}
\end{ex}
\begin{ex}%[Nguyễn Tất Thu, BG10-2022]%%[0D8H3-3]
Hệ số của $x^4$ trong khai triển nhị thức $(3 x-4)^5$ là
\choice
{$1620 $}
{$60$}
{$-60$}
{\True $-1620$}
\loigiai{
Ta có
$$\begin{aligned}
(3x-4)^5=\mathrm{C}_5^0(3x)^5&+\mathrm{C}_5^1(3x)^4\cdot (-4)+\mathrm{C}_5^2(3x)^3\cdot (-4)^2\\&+\mathrm{C}_5^3(3x)^2\cdot (-4)^3+\mathrm{C}_5^4(3x)\cdot (-4)^4+\mathrm{C}_5^5(-4)^5.
\end{aligned}$$
Suy ra hệ số của $x^4$ là
$$\mathrm{C}_5^1\cdot 3^4\cdot (-4)=-1620.$$
}
\end{ex}
\begin{ex}%[Nguyễn Tất Thu, BG10-2022]%%[0D8H3-3]
Hệ số của $x^2$ trong khai triển $(1-2x)^4$ là
\choice
{\True $24$}
{$-24$}
{$48$}
{$-48$}
\loigiai{
Ta có
$$(1-2x)^4=\mathrm{C}_4^0+\mathrm{C}_4^1\cdot (-2x)+\mathrm{C}_4^2\cdot (-2x)^2+\mathrm{C}_4^3\cdot (-2x)^3+\mathrm{C}_4^4\cdot (-2x)^4.$$
Hệ số của $x^2$ là $$\mathrm{C}_4^2\cdot (-2)^2=24.$$
}
\end{ex}
\begin{ex}%[Nguyễn Tất Thu, BG10-2022]%%[0D8H3-3]
Hệ số của $x^3$ trong khai triển $(3+2x)^5$ bằng
\choice
{$1080$}
{\True $720$}
{$50$}
{$100$}
\loigiai{
Ta có
$$\begin{aligned}
(3+2x)^5=\mathrm{C}_5^03^5&+\mathrm{C}_5^1\cdot 3^4\cdot (2x)+\mathrm{C}_5^2\cdot 3^3\cdot (2x)^2\\&+\mathrm{C}_5^3\cdot 3^2\cdot (2x)^3+\mathrm{C}_5^4\cdot 3^1\cdot (2x)^4+\mathrm{C}_5^5\cdot (2x)^5.
\end{aligned}$$
Hệ số $x^3$ là
$$\mathrm{C}_5^3\cdot 3^2\cdot 2^3=720.$$
}
\end{ex}
\begin{ex}%[Nguyễn Tất Thu, BG10-2022]%%[0D8H3-3]
Hệ số của $a^3b^2$ trong khai triển $(a+2b)^5$ bằng
\choice
{$5$}
{\True $10$}
{$4$}
{$6$}
\loigiai{
Ta có $$\begin{aligned}[t]
(a+2b)^5&=\mathrm{C}_5^0a^5+\mathrm{C}_5^1a^4b+\mathrm{C}_5^2a^3b^2+\mathrm{C}_5^3a^2b^3+\mathrm{C}_5^4ab^4+\mathrm{C}_5^5b^5\\
&= a^5+5a^4b+10a^3b^2+10a^2b^3+5ab^4+b^5.
\end{aligned}$$
}
\end{ex}
\begin{ex}%[Nguyễn Tất Thu, BG10-2022]%%[0D8H3-3]
Số hạng không chứa $x$ trong khai triển $\left(x+\dfrac{2}{x}\right)^4,\ x\neq 0$ bằng
\choice
{$0$}
{$12$}
{\True $24$}
{$6$}
\loigiai{
Ta có
$$\left(x+\dfrac{2}{x}\right)^4=\mathrm{C}_4^0x^4+\mathrm{C}_4^1\cdot x^3\cdot \left(\dfrac{2}{x}\right)+\mathrm{C}_4^2\cdot x^2\cdot \left(\dfrac{2}{x}\right)^2+\mathrm{C}_4^3\cdot x\cdot \left(\dfrac{2}{x}\right)^3+\mathrm{C}_4^4\cdot \left(\dfrac{2}{x}\right)^4. $$
Số hạng không chứa $x$ là
$$\mathrm{C}_4^2\cdot x^2\cdot \left(\dfrac{2}{x}\right)^2=\mathrm{C}_4^2\cdot 2^2=24.$$
}
\end{ex}
\begin{ex}%[Nguyễn Tất Thu, BG10-2022]%%[0D8H3-3]
Số hạng không chứa $x$ trong khai triển $\left(x^3-\dfrac{1}{x^2}\right)^5,\ x\neq 0$ bằng
\choice
{$0$}
{$10$}
{\True $-10$}
{$6$}
\loigiai{
Ta có
$$\begin{aligned} \left(x^3-\dfrac{1}{x^2}\right)^5&=\mathrm{C}_5^0(x^3)^5+\mathrm{C}_5^1(x^3)^4\cdot\left(-\dfrac{1}{x^2}\right) +\mathrm{C}_5^2(x^3)^3\cdot\left(-\dfrac{1}{x^2}\right)^2\\&+\mathrm{C}_5^3(x^3)^2\cdot\left(-\dfrac{1}{x^2}\right)^3 +\mathrm{C}_5^4(x^3)^1\cdot\left(-\dfrac{1}{x^2}\right)^4 +\mathrm{C}_5^5\cdot\left(-\dfrac{1}{x^2}\right)^5 . \end{aligned}$$
Số hạng không chứa $x$ bằng
$$\mathrm{C}_5^3(x^3)^2\cdot\left(-\dfrac{1}{x^2}\right)^3=-10.$$
}
\end{ex}
\begin{ex}%[Nguyễn Tất Thu, BG10-2022]%%[0D8H3-3]
Biết rằng $(1-\sqrt{2})^4=a+b\sqrt{2}$ với $a,\ b$ là các số nguyên. Giá trị của $b$ bằng
\choice
{$-11$}
{$11$}
{$12$}
{\True $-12$}
\loigiai{
Ta có
$$(1-\sqrt{2})^4=\mathrm{C}_4^01^4+\mathrm{C}_4^1\cdot 1^3\cdot (-\sqrt{2})+\mathrm{C}_4^2\cdot 1^2\cdot (-\sqrt{2})^2+\mathrm{C}_4^3\cdot 1^1\cdot (-\sqrt{2})^3+\mathrm{C}_4^4\cdot (-\sqrt{2})^4.$$
Suy ra
$$b=-\mathrm{C}_4^1-\mathrm{C}_4^3\cdot 2=-12$$
}
\end{ex}
\begin{ex}%[Nguyễn Tất Thu, BG10-2022]%%[0D8V3-2]
Biết rằng $\left(1+\sqrt{3}\right)^5-2\left(1-\sqrt{3}\right)^4=a+b\sqrt{3}$ với $a,\ b$ là các số nguyên. Tính $T=a-b$
\choice
{$T=96$}
{\True $T=-56$}
{$T=56$}
{$T=-96$}
\loigiai{
Ta có
$$\begin{aligned} \left(1+\sqrt{3}\right)^5&=\mathrm{C}_5^0+\mathrm{C}_5^1\sqrt{3}+\mathrm{C}_5^2\left(\sqrt{3}\right)^2+\mathrm{C}_5^3\left(\sqrt{3}\right)^3+\mathrm{C}_5^4\left(\sqrt{3}\right)^4+\mathrm{C}_5^5\left(\sqrt{3}\right)^5\\&=76+44\sqrt{3}. \end{aligned}$$
và
$$\begin{aligned}
\left(1-\sqrt{3}\right)^4&=\mathrm{C}_4^0+\mathrm{C}_4^1\left(-\sqrt{3}\right)+\mathrm{C}_4^2\left(-\sqrt{3}\right)^2+\mathrm{C}_4^3\left(-\sqrt{3}\right)^3+\mathrm{C}_4^4\left(-\sqrt{3}\right)^4\\&=28-16\sqrt{3}
\end{aligned}$$
Suy ra
$$\left(1+\sqrt{3}\right)^5-2\left(1-\sqrt{3}\right)^4=20+76\sqrt{3}.$$
Do đó $a-b=-56.$
}
\end{ex}
\begin{ex}%[Nguyễn Tất Thu, BG10-2022]%%[0D8V3-2]
Xét khai triển $(a+bx)^5=a_0+a_1x+\cdots+a_5x^5$.
Biết $a_3=40$ và $a_4=10$. Tính $T=a\cdot b$
\choice
{\True $T=2$}
{$T=1$}
{$T=\dfrac{1}{2}$}
{$T=\dfrac{1}{4}$}
\loigiai{
Ta có
$$(a+bx)^5=\mathrm{C}_5^0a^5+\mathrm{C}_5^1a^4bx+\mathrm{C}_5^2a^3b^2x^2+\mathrm{C}_5^3a^2b^3x^3+\mathrm{C}_5^4ab^4x^4+\mathrm{C}_5^5b^5x^5.$$
Suy ra $a_3=\mathrm{C}_5^3a^2b^3=10a^2b^3=40$ và $a_4=\mathrm{C}_5^4ab^4=5ab^4=10$. Do đó
$$\dfrac{a_3}{a_4}=\dfrac{2a}{b}=4\Rightarrow a=2b\Rightarrow b=1,\ a=2\Rightarrow T=2.$$
}
\end{ex}
\begin{ex}%[Nguyễn Tất Thu, BG10-2022]%%[0D8V3-3]
Xét khai triển
$f(x)=(2+x)^5-3(1+2x)^4=a_0+a_1x+\cdots+a_5x^5$.
Tính $a_4$
\choice
{$a_4=71$}
{$a_4=74$}
{$a_4=21$}
{\True $a_4=26$}
\loigiai{
Ta có
$$\begin{aligned} (2+x)^5&=\mathrm{C}_5^02^5+\mathrm{C}_5^12^4x+\mathrm{C}_5^22^3x^2+\mathrm{C}_5^32^2x^3+\mathrm{C}_5^42^1x^4+\mathrm{C}_5^5x^5\\
(1+2x)^4&=\mathrm{C}_4^0+\mathrm{C}_4^1\cdot 2x+\mathrm{C}_4^2(2x)^2+\mathrm{C}_4^3(2x)^3+\mathrm{C}_4^4(2x)^4.  \end{aligned} $$
Suy ra
$$a_4=\mathrm{C}_5^42^1+\mathrm{C}_4^42^4=26.$$
}
\end{ex}
%\textbf{Phần chuyên đề}
\begin{ex}%[Nguyễn Tất Thu, BG10-2022]%%[0D8H3-4]
Hệ số của $ x^6 $ trong khai triển $ \left( \dfrac{1}{x} + x^3 \right)^{10} $ bằng
\choice
{\True $ 210 $}
{$ 252 $}
{$ 165 $}
{$ 792 $}
\loigiai{ Ta có
$$ \left( \dfrac{1}{x} + x^3 \right)^{10} =\sum\limits_{k=0}^{10}\mathrm{C}_{10}^k(x^{-1})^k(x^3)^{10-k}.$$
Để có hạng tử $ x^6 $ thì $ -k+3(10-k)=6\Leftrightarrow k=6 $.\\
Vậy hệ số của $ x^6 $ là $ \mathrm{C}^6_{10}=210 $.
}
\end{ex}
\begin{ex}%[Nguyễn Tất Thu, BG10-2022]%%[0D8H3-4]
Trong khai triển $\left(\dfrac{1}{x^3}+x^5\right)^{12}$ với $x\neq 0$. Số hạng chứa $x^4$ là
\choice{$792$}
{$924$}
{\True $792x^4$}
{$924x^4$}
\loigiai{
Số hạng tổng quát của khai triển là $$\mathrm{C}_{12}^{k}\left(\dfrac{1}{x^3}\right)^{12-k}\cdot\left(x^5\right)^k=\mathrm{C}_{12}^{k}\cdot x^{8k-36}.$$
Xét số hạng chứa $x^4$ thì $$8k-36=4\Leftrightarrow k=5.$$ Vậy số hạng chứa $x^4$ là $\mathrm{C}_{12}^{5}x^4=792x^4$.
}
\end{ex}
\begin{ex}%[Nguyễn Tất Thu, BG10-2022]%%[0D8H3-4]
Tìm số hạng chứa $x^7$ trong khai triển nhị thức $\left(2x^2-\dfrac{1}{x}\right)^8$.
\choice
{$-1792$}
{\True $-1792x^7$}
{$1792$}
{$1792x^7$}
\loigiai{
Ta có  $$\left(2x^2-\dfrac{1}{x}\right)^8=\displaystyle\sum_{k=0}^8\mathrm{C}^k_8(2x^2)^{8-k}\left(-\dfrac{1}{x}\right)^k
= \displaystyle\sum_{k=0}^8\mathrm{C}^k_8 2^{8-k}(-1)^kx^{16-3k}.$$
Số hạng chứa $x^7$ ứng với $$16-3k=7\Leftrightarrow k=3.$$
Khi đó số hạng chứa $x^7$ là
$$\mathrm{C}^3_82^{8-3}(-1)^3x^7=-1792x^7.$$

}
\end{ex}
\begin{ex}%[Nguyễn Tất Thu, BG10-2022]%%[0D8H3-4]
Trong khai triển $(1+3x)^{20}$ với số mũ tăng dần, hệ số của số hạng đứng chính giữa là
\choice
{$3^{11}\mathrm{C}_{20}^{11}$}
{$3^{12}\mathrm{C}_{20}^{12}$}
{\True $3^{10}\mathrm{C}_{20}^{10}$}
{ $3^{9}\mathrm{C}_{20}^9$}
\loigiai{
Ta có $$(1+3x)^{20} = \displaystyle \sum \limits_{k=0}^{20} \mathrm{C}_{20}^{k} 1^{20-k} \left( 3x \right) ^k = \displaystyle \sum \limits_{k=0}^{20} \mathrm{C}_{20}^k \cdot 3^k \cdot x^k.$$
Vế phải của khai triển có $21$ số hạng. Do đó số hạng đứng chính giữa sẽ là số hạng thứ $11$, tương ứng với $k=10$.\\
Vậy, hệ số cần tìm là $3^{10}\mathrm{C}_{20}^{10}$.
}
\end{ex}
\begin{ex}%[Nguyễn Tất Thu, BG10-2022]%%[0D8H3-4]
Số hạng không chứa $x$ trong khai triển $\left(x - \dfrac{1}{x^2}\right)^{45}$ là
\choice
{$\mathrm{C}^{15}_{45}$}
{$\mathrm{C}^{30}_{45}$}
{$- \mathrm{C}^{5}_{45}$}
{\True $ - \mathrm{C}^{15}_{45}$}
\loigiai{Điều kiện $x\neq 0$.\\
Khi đó ta có $$\left(x - \dfrac{1}{x^2}\right)^{45} = \displaystyle\sum\limits_{k = 0}^{45}\left(- 1\right)^{k}\mathrm{C}^{k}_{45}x^{45 - k}\left(\dfrac{1}{x^2}\right)^{k} = \displaystyle\sum\limits_{k = 0}^{k = 45}\left(- 1\right)^{k}\mathrm{C}^{k}_{45}x^{45 - 3k}.$$
Suy ra số hạng thứ $k + 1$ là $$\left(- 1\right)^{k}\mathrm{C}^{k}_{45}x^{45 - 3k}$$ để thỏa mãn bài toán $$45 - 3k = 0\Leftrightarrow k = 15.$$
Khi đó số hạng không chứa $x$ là $\left(- 1\right)^{15}\mathrm{C}^{15}_{45} = - \mathrm{C}^{15}_{45}$.
}
\end{ex}
\begin{ex}%[Nguyễn Tất Thu, BG10-2022]%%[0D8H3-4]
Tìm	hệ số của $x^5$ trong khai triển $\left(2x-\dfrac{3}{x^2}\right)^{11}$.
\choice{$-253440$}{$55$}{$28160$}{\True$253440$}
\loigiai{
Số hạng tổng quát có dạng
$$T_{k+1}=\mathrm{C}_{11}^k(2x)^{11-k}\cdot \left(-\dfrac{3}{x^2}\right)^k
=\mathrm{C}_{11}^k\cdot 2^{11-k}\cdot (-3)^k\cdot \dfrac{x^{11-k}}{(x^2)^k}
=\mathrm{C}_{11}^k\cdot 2^{11-k}\cdot(-3)^k\cdot x^{11-3k}.$$
Số hạng này chứa $x^5$  khi $11-3k=5 \Leftrightarrow k=2$.\\
Vậy hệ số của $x^5$ là $T_3=\mathrm{C}_{11}^2\cdot2^9\cdot(-3)^2=253440$.
}
\end{ex}
\begin{ex}%[Nguyễn Tất Thu, BG10-2022]%%[0D8H3-4]
Tìm số hạng không chứa $x$ trong khai triển $\left(\dfrac{1}{x}-x^2\right)^{12}$
\choice
{\True $495$}
{$-495$}
{$924$}
{$-924$}
\loigiai{
Số hạng tổng quát của khai triển là $$(-1)^k\mathrm{C}^k_{12}x^{3k-12}.$$
Số hạng không chứa $x$ trong khai triển ứng với $k=4$ và bằng $$(-1)^4\mathrm{C}^4_{12}=495.$$
}
\end{ex}
\begin{ex}%[Nguyễn Tất Thu, BG10-2022]%%[0D8H3-4]
Hệ số của $x^5$ trong khai triển nhị thức $x(2x-1)^6+(3x-1)^8$ bằng
\choice
{\True $-13368$}
{$13368$}
{$-13848$}
{$13848$}
\loigiai{
$\begin{aligned}\text{Ta có } x(2x-1)^6+(3x-1)^8 &=x\displaystyle\sum_{k=0}^6 \mathrm{C}_6^k\cdot (2x)^k\cdot (-1)^{6-k}+\displaystyle\sum_{l=0}^8 \mathrm{C}_8^l\cdot (3x)^l\cdot (-1)^{8-l}\\
&=x\displaystyle\sum_{k=0}^6 \mathrm{C}_6^k\cdot (2x)^k\cdot (-1)^{6-k}+\displaystyle\sum_{l=0}^8 \mathrm{C}_8^l\cdot (3x)^l\cdot (-1)^{8-l}
\end{aligned}$ \\
Suy ra hệ số của $x^5$ trong khai triển nhị thức là: $\mathrm{C}_6^4\cdot 2^4\cdot (-1)^{6-4}+\mathrm{C}_8^5\cdot 3^5\cdot (-1)^{6-5}=-13368$.}
\end{ex}

\begin{ex}%[Nguyễn Tất Thu, BG10-2022]%%[0D8H3-4]
Biết rằng hệ số $x^{n-2}$ trong khai triển $\left(x-\dfrac{1}{4}\right)^n$ bằng $31$. Tìm $n$.
\choice
{$30 $}
{\True $ 32$}
{$ 31$}
{ $33 $}
\loigiai{
Số hạng thứ $k+1$ của khai triển nhị thức $(a+b)^n$ là $T_{k+1}=\mathrm{C}_n^k a^k b^{n-k}$.\\
Do đó ta có $T_{n-2}=\mathrm{C}_n^{n-2} x^{n-2} \left( \dfrac{-1}{4}\right)^2$.\\
Suy ra $\mathrm{C}_n^{n-2}  \left( \dfrac{-1}{4}\right)^2=31 \Leftrightarrow n^2-n-31\cdot 32=0 \Leftrightarrow n=32$.
}
\end{ex}

\begin{ex}%[Nguyễn Tất Thu, BG10-2022]%%[0D8V3-4]
Với $ n$ là số nguyên dương thỏa mãn $ \mathrm{A}_{n}^2-2\mathrm{C}_{n+2}^2+82=0$, số hạng không chứa $ x$ trong khai triển của biểu thức $ \left(x^3-\dfrac{3}{x}\right)^n$ bằng
\choice
{$ -15504$}
{$ 15504$}
{\True $ -15504\cdot3^{15}$}
{$ 15504\cdot 3^{15}$}
\loigiai{
Điều kiện $ \heva{
&n\ge 2\\
&n\in N.\\
}$\\
Ta có
\begin{eqnarray*}
&&\mathrm{A}_n^2-2\mathrm{C}_{n+2}^2+82=0\\
&\Leftrightarrow &\dfrac{n!}{\left(n-2\right)!}-2\dfrac{\left(n+2\right)!}{n!\cdot 2!}+82=0\\
&\Leftrightarrow & n\left(n-1\right)-\left(n+2\right)\left(n+1\right)+82=0\\
&\Leftrightarrow & n^2-n-n^2-3n-2+82=0\\
&\Leftrightarrow &-4n+80=0\\
&\Leftrightarrow &n=20.
\end{eqnarray*}
Số hạng tổng quát trong khai triển $ \left(x^3-\dfrac{3}{x}\right)^{20}$ là
$$ \mathrm{T}_{k+1}=\mathrm{C}_{20}^k\left(x^3\right)^{20-k}\left(\dfrac{-3}{x}\right)^k=\mathrm{C}_{20}^k\left(-3\right)^k x^{60-4k}.$$
Ta có $ 60-4k=0\Leftrightarrow k=15$.\\
Vậy số hạng không chứa $ x$ là $ \mathrm{C}_{20}^{15}\left(-3\right)^{15}=-15504\cdot 3^{15}$.
}
\end{ex}
\begin{ex}%[Đỗ Vũ Minh Thắng,BG10-2022]%%[0D8V3-5]
Tính tổng $S = \mathrm{C}_{20}^0+\mathrm{C}_{20}^1+\mathrm{C}_{20}^2+\ldots +\mathrm{C}_{20}^{20}.$
\choice{$S = 0 $}
{$S=1 $}
{$S =2$}
{\True $S = 2^{20} $}
\loigiai{
\[ 2^{20}=(1+1)^{20}=\mathrm{C}_{20}^0+\mathrm{C}_{20}^1+\mathrm{C}_{20}^2+\ldots +\mathrm{C}_{20}^{20}. \]
}
\end{ex}

\begin{ex}%[Đỗ Vũ Minh Thắng,BG10-2022]%%[0D8V3-5]
Tính tổng $S = \mathrm{C}_{20}^0-\mathrm{C}_{20}^1+\mathrm{C}_{20}^2-\ldots +\mathrm{C}_{20}^{20}.$
\choice{\True $S = 0 $}
{$S=1 $}
{$S =-2$}
{$S = (-2)^{20} $}
\loigiai{
\[ 0=(1-1)^{20}=\mathrm{C}_{20}^0-\mathrm{C}_{20}^1+\mathrm{C}_{20}^2-\ldots +\mathrm{C}_{20}^{20}. \]
}
\end{ex}

\begin{ex}%[Đỗ Vũ Minh Thắng,BG10-2022]%%[0D8V3-5]
Tính tổng $S = \mathrm{C}_{20}^0+2\mathrm{C}_{20}^1+2^2\mathrm{C}_{20}^2+\ldots +2^{20}\mathrm{C}_{20}^{20}.$
\choice{$ S=2^{21}$}
{$S=3^{21} $}
{\True $S=3^{20} $}
{$ S=2^{20}$}
\loigiai{
\[ 3^{20}=(1+2)^{20}=\mathrm{C}_{20}^0+2\mathrm{C}_{20}^1+2^2\mathrm{C}_{20}^2+\ldots +2^{20}\mathrm{C}_{20}^{20}. \]
}
\end{ex}

\begin{ex}%[Đỗ Vũ Minh Thắng,BG10-2022]%%[0D8V3-5]
Tính tổng $S = \mathrm{C}_{21}^0-2\mathrm{C}_{21}^1+2^2\mathrm{C}_{21}^2-\ldots -2^{21}\mathrm{C}_{21}^{21}.$
\choice{\True $S= -1 $}
{$S=1 $}
{$S = (-3)^{21} $}
{$S = 3^{21} $}
\loigiai{
\[ -1=(1-2)^{21}=\mathrm{C}_{21}^0-2\mathrm{C}_{21}^1+2^2\mathrm{C}_{21}^2-\ldots -2^{21}\mathrm{C}_{21}^{21}. \]
}
\end{ex}
\begin{ex}%[Đỗ Vũ Minh Thắng,BG10-2022]%%[0D8V3-5]
Tính tổng $S = \mathrm{C}_{21}^0-\dfrac{1}{2}\mathrm{C}_{21}^1+\dfrac{1}{2^2}\mathrm{C}_{21}^2-\ldots -\dfrac{1}{2^{21}}\mathrm{C}_{21}^{21}.$
\choice{$ S =\left(- \dfrac{1}{2}\right)^{21}$}
{$S = \dfrac{1}{2} $}
{\True $S = \dfrac{1}{2^{21}} $}
{$S = -\dfrac{1}{2} $}
\loigiai{
\[ \left(\dfrac{1}{2}\right)^{21}=\left (1-\dfrac{1}{2}\right )^{21}=\mathrm{C}_{21}^0-\dfrac{1}{2}\mathrm{C}_{21}^1+\dfrac{1}{2^2}\mathrm{C}_{21}^2-\ldots -\dfrac{1}{2^{21}}\mathrm{C}_{21}^{21}. \]
}
\end{ex}


\begin{ex}%[Đỗ Vũ Minh Thắng,BG10-2022]%%[0D8V3-5]
Tính tổng $S = 3^{20}\mathrm{C}_{20}^0+3^{19}\mathrm{C}_{20}^1+3^{18}\mathrm{C}_{20}^2+\ldots +3\mathrm{C}_{20}^{20}+\mathrm{C}_{20}^{20}.$
\choice{$S=2^{20} $}
{$ S=3^{20}$}
{\True $S =4^{20} $}
{$S=-4^{20} $}
\loigiai{
\[ 4^{20}=(1+3)^{20}=3^{20}\mathrm{C}_{20}^0+3^{19}\mathrm{C}_{20}^1+3^{18}\mathrm{C}_{20}^2+\ldots +3\mathrm{C}_{20}^{20}+\mathrm{C}_{20}^{20}. \]
}
\end{ex}

\begin{ex}%[Đỗ Vũ Minh Thắng,BG10-2022]%%[0D8V3-5]
Tính tổng $S = 3^{20}\mathrm{C}_{20}^0-3^{19}\mathrm{C}_{20}^1+3^{18}\mathrm{C}_{20}^2-\ldots -3\mathrm{C}_{19}^{20}+\mathrm{C}_{20}^{20}.$
\choice{\True $S=2^{20} $}
{$ S=3^{20}$}
{$S =4^{20} $}
{$S=-4^{20} $}
\loigiai{
\[ 2^{20}=(3-1)^{20}=3^{20}\mathrm{C}_{20}^0-3^{19}\mathrm{C}_{20}^1+3^{18}\mathrm{C}_{20}^2-\ldots -3\mathrm{C}_{19}^{20}+\mathrm{C}_{20}^{20}. \]
}
\end{ex}
\begin{ex}%[Đỗ Vũ Minh Thắng,BG10-2022]%%[0D8V3-5]
Tính tổng $S = 3^{20}\mathrm{C}_{20}^0+3^{19}\cdot 2\mathrm{C}_{20}^1+3^{18}\cdot 2^2\mathrm{C}_{20}^2+\ldots +3\cdot 2^{19}\mathrm{C}_{19}^{20}+2^{20}\mathrm{C}_{20}^{20}.$
\choice{$S=1 $}
{$S= 6^{20}$}
{\True $S = 5^{20} $}
{$S=-1 $}
\loigiai{
\[ 5^{20}=(3+2)^{20}=3^{20}\mathrm{C}_{20}^0+3^{19}\cdot 2\mathrm{C}_{20}^1+3^{18}\cdot 2^2\mathrm{C}_{20}^2+\ldots +3\cdot 2^{19}\mathrm{C}_{19}^{20}+2^{20}\mathrm{C}_{20}^{20}. \]
}
\end{ex}

\begin{ex}%[Đỗ Vũ Minh Thắng,BG10-2022]%%[0D8V3-5]
Tính tổng $S = 3^{20}\mathrm{C}_{20}^0-3^{19}\cdot 2\mathrm{C}_{20}^1+3^{18}\cdot 2^2\mathrm{C}_{20}^2-\ldots -3\cdot 2^{19}\mathrm{C}_{19}^{20}+2^{20}\mathrm{C}_{20}^{20}.$
\choice{\True $S=1 $}
{$S= 6^{20}$}
{$S = 5^{20} $}
{$S=-1 $}
\loigiai{
\[ 1=(3-2)^{20}=3^{20}\mathrm{C}_{20}^0-3^{19}\cdot 2\mathrm{C}_{20}^1+3^{18}\cdot 2^2\mathrm{C}_{20}^2-\ldots -3\cdot 2^{19}\mathrm{C}_{19}^{20}+2^{20}\mathrm{C}_{20}^{20}. \]
}
\end{ex}
\begin{ex}%[Đỗ Vũ Minh Thắng,BG10-2022]%%[0D8H3-5]
Công thức thu gọn của $S = \mathrm{C}_{n}^0-\mathrm{C}_{n}^1x+\mathrm{C}_{n}^2x^2-\ldots +(-1)^n\mathrm{C}_{n}^{n}x^{n}$ là
\choice{$S=(x-1)^n $}
{\True $S=(1-x)^n $}
{$S=(x+1)^n $}
{$S=2^n $}
\loigiai{
Theo khai triển nhị thức Newton, ta có
\[ (1-x)^n=\mathrm{C}_{n}^0-\mathrm{C}_{n}^1x+\mathrm{C}_{n}^2x^2-\ldots +(-1)^n\mathrm{C}_{n}^{n}x^{n}. \]
}
\end{ex}

\begin{ex}%[Lê Nguyễn Viết Tường,BG10-2022]%%[0D8V3-5]
Tổng $\mathrm{C}_{2018}^{2}+\mathrm{C}_{2018}^{3}+\mathrm{C}_{2018}^{4}+\cdots+\mathrm{C}_{2018}^{2018}$ bằng
\choice
{$2^{2018}$}
{$2^{2018}-1$}
{\True $2^{2018}-2019$}
{$2^{2018}-2018$}
\loigiai{
Ta có $\mathrm{C}_{2018}^{0}+\mathrm{C}_{2018}^{1}+\mathrm{C}_{2018}^{2}+\cdots+\mathrm{C}_{2018}^{2018}=2^{2018}$. Do đó
$$ \mathrm{C}_{2018}^{2}+\mathrm{C}_{2018}^{3}+\mathrm{C}_{2018}^{4}+\cdots+\mathrm{C}_{2018}^{2018}=2^{2018}-\left(\mathrm{C}_{2018}^{0}++\mathrm{C}_{2018}^{1}\right)=2^{2018}-(1+2018)=2^{2018}-2019.$$
}
\end{ex}

\begin{ex}%[Lê Nguyễn Viết Tường,BG10-2022]%%[0D8V3-5]
Tổng $\mathrm{C}_{2019}^0 + \mathrm{C}_{2019}^1 + \mathrm{C}_{2019}^2 + \mathrm{C}_{2019}^3 + \ldots + \mathrm{C}_{2019}^{2018} + \mathrm{C}_{2019}^{2019}$ bằng
\choice
{\True $2^{2019}$}
{$2^{2019}+1$}
{$4^{2019}-1$}
{$2^{2019}-1$}
\loigiai
{
$(1+x)^{2019}=\mathrm{C}_{2019}^0+\mathrm{C}_{2019}^1x+\mathrm{C}_{2019}^2x^2+\cdots+\mathrm{C}_{2019}^{2019}x^{2019}$.\\
Chọn $x=1$, ta được $\mathrm{C}_{2019}^0 + \mathrm{C}_{2019}^1 + \mathrm{C}_{2019}^2 + \mathrm{C}_{2019}^3 + \ldots + \mathrm{C}_{2019}^{2018} + \mathrm{C}_{2019}^{2019}=2^{2019}$.
}
\end{ex}

\begin{ex}%[Lê Nguyễn Viết Tường,BG10-2022]%%[0D8V3-5]
Giải phương trình $ \mathrm{C}_{1}^{n}+3\cdot\mathrm{C}_{2}^{n}+7\cdot\mathrm{C}_{3}^{n}+ \cdots + \left(2^n-1\right)\cdot
\mathrm{C}_{n}^{n}=3^{2n}-2^n-6480 $ trên tập $ \mathbb{N^{*}} $.
\choice
{$ n=3 $}
{\True $ n=4 $}
{$ n=5 $}
{$ n=6 $}
\loigiai{
Xét khai triển $ \left(1+x\right)^n=	\mathrm{C}_{0}^{n}+x\mathrm{C}_{1}^{n}+x^2\mathrm{C}_{2}^{n}+ \cdots + x^n\mathrm{C}_{n}^{n} $.\\
Thay $ x=2 $ ta có $ 3^n=\mathrm{C}_{0}^{n}+2\mathrm{C}_{1}^{n}+2^2\mathrm{C}_{2}^{n}+ \cdots + 2^n\mathrm{C}_{n}^{n}\quad (1) $.\\
Thay $ x=1 $ ta có $ 2^n=\mathrm{C}_{0}^{n}+\mathrm{C}_{1}^{n}+\mathrm{C}_{2}^{n}+ \cdots + \mathrm{C}_{n}^{n}\quad (2) $.\\
Trừ vế theo vế của $ (1) $ cho $ (2) $ thì $ \mathrm{C}_{1}^{n}+3\mathrm{C}_{2}^{n}+7\mathrm{C}_{3}^{n}+ \cdots + \left(2^n-1\right)\mathrm{C}_{n}^{n} = 3^n - 2^n $.\\
Khi đó $ 3^n - 2^n = 3^{2n}-2^n-6480 \Leftrightarrow 3^n=81 \Leftrightarrow n=4 $.
}
\end{ex}

\begin{ex}%[Lê Nguyễn Viết Tường,BG10-2022]%%[0D8C3-5]
Tính tổng $S=\dfrac{1}{2!2017!}+\dfrac{1}{4!2015!}+\dfrac{1}{6!2013!}+\ldots+\dfrac{1}{2016!3!}+\dfrac{1}{2018!}$.
\choice
{$S=\dfrac{2^{2018}-1}{2019}$}
{$S=\dfrac{2^{2018}-1}{2018!}$}
{$S=\dfrac{2^{2018}}{2018}$}
{\True $S=\dfrac{2^{2018}-1}{2019!}$}
\loigiai{
Ta có
\begin{eqnarray*}
2019!S &=& \dfrac{2019!}{2!2017!}+\dfrac{2019!}{4!2015!}+\dfrac{2019!}{6!2013!}+\ldots+\dfrac{2019!}{2016!3!}+\dfrac{2019!}{2018!}\\
&=& \mathrm{C}_{2019}^2+\mathrm{C}_{2019}^4+\mathrm{C}_{2019}^6+\ldots+\mathrm{C}_{2019}^{2016}+\mathrm{C}_{2019}^{2018}
\end{eqnarray*}
Ta có $(1+x)^{2019}=\mathrm{C}_{2019}^0+\mathrm{C}_{2019}^1 x+\mathrm{C}_{2019}^2 x^2+\ldots+\mathrm{C}_{2019}^{2019} x^{2019}$\\
và $(1-x)^{2019}=\mathrm{C}_{2019}^0-\mathrm{C}_{2019}^1 x+\mathrm{C}_{2019}^2 x^2-\ldots-\mathrm{C}_{2019}^{2019} x^{2019}$\\
nên suy ra $(1+x)^{2019}+(1-x)^{2019}=2\left(\mathrm{C}_{2019}^0+\mathrm{C}_{2019}^2 x^2+\mathrm{C}_{2019}^4 x^4+\ldots+\mathrm{C}_{2019}^{2018} x^{2018}\right)$.\\
Thay $x=1$ ta được $\mathrm{C}_{2019}^0+\mathrm{C}_{2019}^2+\mathrm{C}_{2019}^4+\mathrm{C}_{2019}^6+\ldots+\mathrm{C}_{2019}^{2016}+\mathrm{C}_{2019}^{2018}=2^{2018}$.\\
Từ đó ta có $2019!S=2^{2018}-1$ hay $S=\dfrac{2^{2018}-1}{2019!}$.
}
\end{ex}

\begin{ex}%[Lê Nguyễn Viết Tường,BG10-2022]%%[0D8V3-5]
Tính giá trị biểu thức $S=\mathrm{C}_{2017}^{1}+\mathrm{C}_{2017}^{2}+\mathrm{C}_{2017}^{3}+\cdots+\mathrm{C}_{2017}^{2016}$.
\choice
{$S=2^{2016}-1$}
{$S=2^{2017}$}
{\True $S=2^{2017}-2$}
{$S=2^{2017}-1$}
\loigiai{
Xét khai triển $(a+b)^n=\mathrm{C}_{n}^{0}a^nb^0+\mathrm{C}_{n}^{1}a^{n-1}b^1+\mathrm{C}_{n}^{2}a^{n-2}b^2
+\cdots+\mathrm{C}_{n}^{n}a^{0}b^{n}$.\\
Cho $a=b=1$ và $n=2017$ vào khai triển trên, ta được
\begin{align*}
&2^{2017} = \mathrm{C}_{2017}^{0}+\mathrm{C}_{2017}^{1}+\mathrm{C}_{2017}^{2}
+\cdots+\mathrm{C}_{2017}^{2017}\\
\Rightarrow\;\,&\mathrm{C}_{2017}^{1}+\mathrm{C}_{2017}^{2}+\mathrm{C}_{2017}^{3}
+\cdots+\mathrm{C}_{2017}^{2016}=2^{2017}-\mathrm{C}_{2017}^{0}-\mathrm{C}_{2017}^{2017}\\
\text{hay}\;\,& S=2^{2017}-2.
\end{align*}\vspace*{-12pt}
}
\end{ex}

\begin{ex}%[Lê Nguyễn Viết Tường,BG10-2022]%%[0D8C3-5]
Cho khai triển $\left(1+x+x^{2}\right)^{n}=a_{0}+a_{1}x+a_{2}x^{2}+\cdots+a_{2 n}x^{2 n}$, với $n \geq 2$ và $a_{0}, a_{1}, a_{2}, \ldots, a_{2 n}$ là các hệ số. Biết rằng $a_{3}=210$, khi đó tổng $S=a_{0}+a_{1}+a_{2}+\cdots+a_{2n}$ bằng
\choice
{$S=3^{13}$}
{\True $S=3^{10}$}
{$S=3^{12}$}
{$S=3^{11}$}
\loigiai{
Ta có $\left(1+x+x^{2}\right)^{n}=\sum\limits_{k=0}^{n} \mathrm{C}_n^k \left(1+x\right)^k\left(x^2\right)^{n-k}=\sum\limits_{k=0}^{n} \mathrm{C}_n^k \left(\sum\limits_{i=0}^{k}\mathrm{C}_k^i x^i\right) x^{2n-2k}=\sum\limits_{k=0}^{n}\sum\limits_{i=0}^{k} \mathrm{C}_n^k\mathrm{C}_k^ix^{2n-2k+i}$.\\
Xét $2n-2k+i=3$ với $\heva{&0\le i\le k\le n\\ &i,k,n\in\mathbb{N}.}$\\
Vì $n\ge k$ nên $2n-2k\ge 0 \Rightarrow i\le 3$.
\begin{itemize}
\item $i=0 \Rightarrow 2n-2k=3$ (vô nghiệm).
\item $i=1 \Rightarrow 2n-2k=2 \Leftrightarrow k=n-1$.
\item $i=2 \Rightarrow 2n-2k=1$ (vô nghiệm).
\item $i=3 \Rightarrow k=n$.
\end{itemize}
Do đó hệ số của $x^3$ là $\mathrm{C}_n^{n-1}\mathrm{C}_{n-1}^1+\mathrm{C}_n^n\mathrm{C}_n^3=\dfrac{1}{6}n^3+\dfrac{1}{2}n^2-\dfrac{2}{3}n$.\\
Theo giả thiết $\dfrac{1}{6}n^3+\dfrac{1}{2}n^2-\dfrac{2}{3}n=210 \Leftrightarrow n=10$.\\
Khi đó $S=\sum\limits_{k=0}^{10}\sum\limits_{i=0}^{k} \mathrm{C}_{10}^k\mathrm{C}_k^i=\sum\limits_{k=0}^{10}\left(\sum\limits_{i=0}^{k}\mathrm{C}_k^i\right)\mathrm{C}_{10}^k=\sum\limits_{k=0}^{10}2^k\mathrm{C}_{10}^k=3^{10}$.
}
\end{ex}

\begin{ex}%[Lê Nguyễn Viết Tường,BG10-2022]%%[0D8C3-5]
Tính tổng $S=\mathrm{C}_{2018}^0 + \dfrac{1}{2}\mathrm{C}_{2018}^1 + \dfrac{1}{3}\mathrm{C}_{2018}^2 + \cdots + \dfrac{1}{2018}\mathrm{C}_{2018}^{2017} + \dfrac{1}{2019}\mathrm{C}_{2018}^{2018}$.
\choice
{$S=\dfrac{2^{2018} + 1}{2019}$}
{$S=\dfrac{2^{2018}-1}{2019} + 1$}
{\True $S=\dfrac{2^{2019}-1}{2019}$}
{$S=\dfrac{2^{2018}-1}{2019}-1$}
\loigiai
{
Xét khai triển
$$(1+x)^{2019}=\mathrm{C}_{2019}^0+\mathrm{C}_{2019}^1x+\mathrm{C}_{2019}^2x^2+\mathrm{C}_{2019}^3x^3+\cdots+\mathrm{C}_{2019}^{2019}x^{2019}.$$
Ta có $\mathrm{C}_{n+1}^{k+1}=\dfrac{n+1}{k+1}\mathrm{C}_n^k$. Suy ra\\
\allowdisplaybreaks
\begin{eqnarray*}
(1+x)^{2019}&=&\mathrm{C}_{2019}^0+\dfrac{2019}{1}\mathrm{C}_{2018}^0x+\dfrac{2019}{2}\mathrm{C}_{2018}^1x^2+\cdots+\dfrac{2019}{2019}\mathrm{C}_{2018}^{2018}x^{2019} \\
(1+x)^{2019}-1&=&2019\left(\mathrm{C}_{2018}^0x+\dfrac{1}{2}\mathrm{C}_{2018}^1x^2+\dfrac{1}{3}\mathrm{C}_{2018}^2x^3+\cdots+\dfrac{1}{2019}\mathrm{C}_{2018}^{2018}x^{2019}\right).
\end{eqnarray*}
Chọn $x=1$, ta có
\[\dfrac{2^{2019}-1}{2019}=\mathrm{C}_{2018}^0 + \dfrac{1}{2}\mathrm{C}_{2018}^1 + \dfrac{1}{3}\mathrm{C}_{2018}^2 + \cdots + \dfrac{1}{2018}\mathrm{C}_{2018}^{2017} + \dfrac{1}{2019}\mathrm{C}_{2018}^{2018}.\]
Vậy $S=\dfrac{2^{2019}-1}{2019}$.
}
\end{ex}

\begin{ex}%[Lê Nguyễn Viết Tường,BG10-2022]%%[0D8C3-5]
Tính giá trị của biểu thức
$$P=\left(\dfrac{1}{\mathrm{C}^1_{2017}}+\dfrac{1}{\mathrm{C}^2_{2017}}+\cdots+\dfrac{1}{\mathrm{C}^{2017}_{2017}}\right)
:\left(\dfrac{1}{\mathrm{C}^0_{2016}}+\dfrac{1}{\mathrm{C}^1_{2016}}+\cdots+\dfrac{1}{\mathrm{C}^{2016}_{2016}}\right).$$
\choice
{$P=\dfrac{1008}{2017}$}
{$P=\dfrac{2016}{2017}$}
{\True $P=\dfrac{1009}{2017}$}
{$P=\dfrac{2018}{2017}$}
\loigiai{
Với mọi $k\in[0; 2016]$, $k\in\mathbb{N}$, ta có
\begin{align*}
\dfrac{1}{\mathrm{C}^{k+1}_{2017}}+\dfrac{1}{\mathrm{C}^{2017-k}_{2017}}&=\dfrac{(k+1)!(2016-k)!}{2017!}+\dfrac{k!(2017-k)!}{2017!}\\
&=\dfrac{k!(2016-k)!(k+1+2017-k)}{2016!2017}\\
&=\dfrac{2018}{2017}\cdot\dfrac{1}{\mathrm{C}^k_{2016}}\\
&=\dfrac{1009}{2017}\left(\dfrac{1}{\mathrm{C}^k_{2016}}+\dfrac{1}{\mathrm{C}^{2016-k}_{2016}}\right)\quad
\left(\text{vì}\quad\dfrac{1}{\mathrm{C}^k_{2016}}=\dfrac{1}{\mathrm{C}^{2016-k}_{2016}}\right).
\end{align*}
Đặt $A=\dfrac{1}{\mathrm{C}^1_{2017}}+\dfrac{1}{\mathrm{C}^2_{2017}}+\cdots+\dfrac{1}{\mathrm{C}^{2017}_{2017}}$,
$B=\dfrac{1}{\mathrm{C}^0_{2016}}+\dfrac{1}{\mathrm{C}^1_{2016}}+\cdots+\dfrac{1}{\mathrm{C}^{2016}_{2016}}$.\\
Khi đó
\begin{align*}
2A&=\dfrac{1}{\mathrm{C}^1_{2017}}+\dfrac{1}{\mathrm{C}^2_{2017}}+\cdots+\dfrac{1}{\mathrm{C}^{2017}_{2017}}\\
&+\dfrac{1}{\mathrm{C}^{2017}_{2017}}+\dfrac{1}{\mathrm{C}^{2016}_{2017}}+\cdots+\dfrac{1}{\mathrm{C}^1_{2017}}\\
&=\dfrac{1009}{2017}\left(\dfrac{1}{\mathrm{C}^0_{2016}}+\dfrac{1}{\mathrm{C}^{2016}_{2016}}\right)
+\dfrac{1009}{2017}\left(\dfrac{1}{\mathrm{C}^1_{2016}}+\dfrac{1}{\mathrm{C}^{2015}_{2016}}\right)
+\cdots+\dfrac{1009}{2017}\left(\dfrac{1}{\mathrm{C}^{2016}_{2016}}+\dfrac{1}{\mathrm{C}^0_{2016}}\right)\\
&=2B.
\end{align*}
Suy ra $P=\dfrac{2A}{2B}=\dfrac{1009}{2017}$.
}
\end{ex}
	\Closesolutionfile{ans}
	\Closesolutionfile{ansbook}
	\indapan{10}{ans/ans-Nhom11-25-TN}
