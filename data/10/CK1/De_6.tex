\begin{name}
	{\tenchude}
	{\tendethi}
	{\tentruong}
	{\thoigian}
\end{name}
\setcounter{ex}{0}\setcounter{bt}{0}
\TN
\Opensolutionfile{ans}[ans/ansDe6-TN1]
\begin{ex}%[Huỳnh Thanh Tiến]%[41-60 Ky Tong]%[0D1N1-1]
	Cho các phát biểu sau:
	\begin{enumerate}
		\begin{multicols}{2}
			\item  Hãy đi nhanh lên!
			\item $4+5+6=15$.
			\item  Năm 2000 là năm nhuận.
			\item  Trái đất hình lập phương.
			\item Cần Thơ là thành phố trực thuộc trung ương.
		\end{multicols}
	\end{enumerate}
	Hỏi có bao nhiêu câu là mệnh đề?
	\choice
	{\True $4$}
	{$2$}
	{$5$}
	{$3$}
	\loigiai{
		Mệnh đề là một câu khẳng định đúng hoặc câu khẳng định sai.\\
		Phát biểu: Hãy đi nhanh lên! không phải là câu khẳng định nên phát biểu trên không là mệnh đề. }
\end{ex}

\begin{ex}%[0-GHK1-2122, THPT Trương Vĩnh Ký - Bến Tre, 2021 - 2022]%[Nguyễn Kiều Nhã Tú]%[0D1H1-3]
	Cho mệnh đề $A \colon$ \lq \lq $2021$ không là số tự nhiên \rq \rq. Tìm mệnh đề phủ định của mệnh đề $A$?
	\choice
	{$2021$ là số vô tỉ}
	{\True $2021$ là số tự nhiên}
	{$2021$ không là số tự nhiên}
	{$2021$ không là số nguyên}
	\loigiai{

	}
\end{ex}

\begin{ex}%[GHK1, THPT C Bình Lục - Hà Nam, 2021]%[Thành Đức Trung, 0-EX-GHK1]%[0D1N2-1]
	Tổng $\overrightarrow{MN} + \overrightarrow{PQ} + \overrightarrow{RN} + \overrightarrow{NP} + \overrightarrow{QR}$ bằng
	\choice
	{$\overrightarrow{MR}$}
	{$\overrightarrow{MQ}$}
	{\True $\overrightarrow{MN}$}
	{$\overrightarrow{MP}$}
	\loigiai{
		Ta có $\overrightarrow{MN} + \overrightarrow{PQ} + \overrightarrow{RN} + \overrightarrow{NP} + \overrightarrow{QR} = \overrightarrow{MN} + \overrightarrow{NP} + \overrightarrow{PQ} + \overrightarrow{QR} + \overrightarrow{RN} = \overrightarrow{MN}$.
	}
\end{ex}

\begin{ex}%[Dự án Giảng 10-11 Nhóm Toán & LaTex, Mui Doan]%[0D1H3-3]
	Cho hai tập $A=[-2;1]$ và $B=(0;+\infty)$. Xác định tập hợp $A\cup B$.
	\choice
	{$[1;+\infty)$}
	{$[-2;0)$}
	{\True $[-2;+\infty)$}
	{$(0;1]$}
	\loigiai{
	Ta có $A\cup B=[-2;+\infty)$.
	}
\end{ex}

\begin{ex}%[Dự án Giảng 10-11 Nhóm Toán & LaTex, Lê Minh Thiện Anh]%[0D1H3-5]
	Một lớp học có $ 50 $ học sinh trong đó có $ 30 $ em biết chơi bóng chuyền, $25$ em biết chơi bóng đá, $ 10 $ em biết chơi cả bóng đá và bóng chuyền. Hỏi có bao nhiêu em không biết chơi môn nào trong hai môn ở trên?
	\choice
	{$15$}
	{\True $5$}
	{$10$}
	{$20$}
	\loigiai{
		Gọi tập $A$ là tập hợp các học sinh biết chơi bóng chuyền.\\
		Tập $B$ là tập hợp các học sinh biết chơi bóng đá.\\
		Khi đó số học sinh biết chơi ít nhất một trong hai môn bóng chuyền hoặc bóng đá là
		\[n(A\cup B)=n(A)+n(B)-n(A\cap B)=30+25-10=45.\]
		Vậy số học sinh không biết chơi môn nào là $50-45=5$.
	}
\end{ex}

\begin{ex}%[Đề GK1 trường THPT Chuyên Hùng Vương]%[An Le, 10EX-HK1-2223]%[0D2N1-1]
	Miền nghiệm của bất phương trình $3(x-1)+5(y-3)<2 x+7$ là nửa mặt phẳng chứa điểm nào sau đây?
	\choice
	{$(0; 5)$}
	{$(3; 6)$}
	{$(25; 0)$}
	{\True $(0; 0)$}
	\loigiai{
		Bất phương trình tương đương với $x+5y-25<0$, có điểm $(0;0)$ thỏa mãn bất phương trình. Do đó miền nghiệm của bất phương trình là nửa mặt phẳng chứa điểm $(0; 0)$.
	}
\end{ex}

\begin{ex}%[0D2H1-2]%[Đề chuẩn hóa số 2]%[BCTuan]
	Điểm $A(1;-3)$ là điểm thuộc miền nghiệm của bất phương trình?
	\choice
	{$2x+5y+4\geq0$}
	{\True $-3x+2y-4<0$}
	{$x+3y>0$}
	{$3x-y\leq0$}
	\loigiai{
		Thay toạ độ điểm $A(1;-3)$ vào các bất phương trình ta thấy chỉ có bất phương trình $-3x+2y-4<0$ thoả mãn.
	}
\end{ex}

\begin{ex}%[0D2N2-1]%[Đề cương Nguyễn Thượng Hiền]%[Mui Doan,DA4-ĐC-NTH-T10]
	Cặp số $(x; y)$ nào sau đây {\bf{không}} là nghiệm của hệ bất phương trình $\heva{&3x+4y-1> 0\\&x+2y-3\leq 0}$?
	\choice
	{$(3; 0)$}
	{$(-1; 2)$}
	{$(2; 0)$}
	{\True $(0; 0)$}
	\loigiai{
		Ta có
		\begin{itemize}
			\item $\heva{&3\cdot 3+4\cdot 0-1> 0\\&3+2\cdot 0-3\leq 0}$ đúng, suy ra $(3;0)$ là nghiệm của hệ bất phương trình \[\heva{&3x+4y-1> 0\\&x+2y-3\leq 0.}\]
			\item $\heva{&3\cdot (-1)+4\cdot 2-1> 0\\&-1+2\cdot 2-3\leq 0}$ đúng, suy ra $(-1;2)$ là nghiệm của hệ bất phương trình \[\heva{&3x+4y-1> 0\\&x+2y-3\leq 0.}\]
			\item $\heva{&3\cdot 2+4\cdot 0-1> 0\\&3+2\cdot 0-3\leq 0}$ đúng, suy ra $(2;0)$ là nghiệm của hệ bất phương trình \[\heva{&3x+4y-1> 0\\&x+2y-3\leq 0.}\]
			\item $\heva{&3\cdot 0+4\cdot 0-1> 0\\&0+2\cdot 0-3\leq 0}$ sai, suy ra $(0;0)$ không là nghiệm của hệ bất phương trình \[\heva{&3x+4y-1> 0\\&x+2y-3\leq 0.}\]
		\end{itemize}
	}
\end{ex}

\begin{ex}%[0D2H2-2]%[Đề chuẩn hóa số 2]%[BCTuan]
	Cho hệ bất phương trình $\heva{&2 x-3 y+1<0 \\& x+4 y-2 \leq 0}$. Trong các điểm sau, điểm nào thuộc miền nghiệm của hệ bất phương trình?
	\choice
	{$N(-1;1)$}
	{$P(1;3)$}
	{\True $Q(-1;0)$}
	{$M(0;1)$}
	\loigiai{
		Thay lần lượt toạ độ các điểm vào hệ bất phương trình ta thấy chỉ có $Q(-1;0)$ thoả mãn hệ bất phương trình.
	}
\end{ex}

\begin{ex}%[24-25-Bai-Giang-K10-K11, Tran Quoc]%[0H4N2-1]
	Cho tam giác $ABC$ có $\widehat{A}=45^\circ$, $b=5$, $a=5\sqrt{2}$. Tính số đo góc $\widehat{B}$.
	\choice
	{$\widehat{B}=90^\circ$}
	{$\widehat{B}=60^\circ$}
	{\True $\widehat{B}=30^\circ$}
	{$\widehat{B}=120^\circ$}
	\loigiai{
		Áp dụng định lí sin, ta có
		\[\dfrac{a}{\sin A}=\dfrac{b}{\sin B} \Rightarrow \sin B=\dfrac{b \sin A}{a}=\dfrac{5\sin{45^\circ}}{5\sqrt{2}}=\dfrac{1}{2}.\]
		Vậy $\widehat{B}=30^\circ$.}
\end{ex}

\begin{ex}%[0H4N2-2]
	Cho $\triangle ABC$ có các cạnh $BC=a$, $AC=b$, $AB=c$. Diện tích của $\triangle ABC$ là
	\choice
	{$S_{\triangle ABC}=\dfrac{1}{2}ac\sin C$}
	{$S_{\triangle ABC}=\dfrac{1}{2}bc\sin B$}
	{\True $S_{\triangle ABC}=\dfrac{1}{2}ac\sin B$}
	{$S_{\triangle ABC}=\dfrac{1}{2}bc\sin C$}
	\loigiai{
		Ta có $S_{\triangle ABC}=\dfrac{1}{2}ac\sin B$.
	}
\end{ex}

\begin{ex}%[BG12-3in1, Nguyễn Xuân Bảo-Phạm Ngọc Trung]%[0H4H2-1]
	Cho tam giác đều $ MNP $ có có cạnh bằng $10$. Gọi $I$ là trung điểm $NP$. Tính $PI$.
	\choice
	{$ 5 $}
	{$ 5\sqrt{2}$}
	{\True $ 5\sqrt{3} $}
	{$  5\sqrt{5}$}
	\loigiai{


		\begin{center}
			\begin{tikzpicture}[scale=1, font=\footnotesize, line join=round, line cap=round, >=stealth]
				\coordinate[label=left:$N$] (N) at (0,0);
				\coordinate[label=right:$P$] (P) at (5,0);
				\coordinate[shift=(60:5),label=above:$M$] (M) at (N);
				\coordinate[label=left:$I$] (I) at ($(M)!0.5!(N)$);
				\draw (M)--(N)--(P)--cycle (P)--(I);
				\foreach \diem in {M,N,P,I}\fill (\diem)circle(1.5pt);
			\end{tikzpicture}
		\end{center}
		Ta có  $PI^2=MI^2+MP^2-2\cdot MI \cdot MP \cdot \cos M=5^2+10^2-2 \cdot 5 \cdot 10 \cdot \cos 60^{\circ}=75 \Rightarrow PI=5\sqrt{3}$.

	}

\end{ex}

\begin{ex}%[Mức độ 2]%[Nguyễn Trung Kiên, dự án giảng 10-11, CTGDPT2018]%[0H4H1-2]
	Cho góc $\sin\alpha=\dfrac{2}{3}$ với $0^\circ <\alpha <90^\circ$. Khẳng định nào sau đây là đúng?
	\choice
	{$\sin\left(180^\circ -\alpha\right) = -\dfrac{2}{3}$}
	{\True $\cos\left(90^\circ -\alpha\right)=\dfrac{2}{3}$}
	{$\sin\left(180^\circ -\alpha\right) = \dfrac{3}{2}$}
	{$\cos\left(90^\circ -\alpha\right)=-\dfrac{2}{3}$}
	\loigiai{
		Ta có $\sin\left(180^\circ-\alpha\right)=\sin\alpha =\dfrac{2}{3}$, $\cos\left(90^\circ -\alpha\right) =\sin\alpha =\dfrac{2}{3}$.
	}
\end{ex}

\begin{ex}%[De-chuan-hoa-so-8]%[Duong Xuan Loi]%[0H5N1-3]
	Cho tam giác đều $ABC$. Mệnh đề nào sau đây \textbf{sai}?
	\choice
	{$\left|\overrightarrow{AB}\right|=\left|\overrightarrow{BC}\right|$}
	{$\overrightarrow{AC}$ không cùng phương $\overrightarrow{BC}$}
	{\True $\overrightarrow{AB}=\overrightarrow{BC}$}
	{$\overrightarrow{AC}\neq \overrightarrow{BC}$}
	\loigiai{
		Mệnh đề sai là $\overrightarrow{AB}=\overrightarrow{BC}$.
	}
\end{ex}

\begin{ex}%[Đề HK1, Lê Quý Đôn, HCM, 23 - 24]%[Trần Hoà, 10-11EX-HK1-2324]%[0H5N2-2]
	Cho hình bình hành $ABCD$. Hệ thức nào sau đây là \textbf{sai}?
	\choice
	{$\overrightarrow{AB}+\overrightarrow{AD}=\overrightarrow{AC}$}
	{$\overrightarrow{AB}+\overrightarrow{BC}=\overrightarrow{AC}$}
	{$\overrightarrow{AB}+\overrightarrow{CD}=\overrightarrow{AD}+\overrightarrow{CB}$}
	{\True $\overrightarrow{AB}=\overrightarrow{CD}$}
	\loigiai
	{\immini{Do $ABCD$ là hình bình hành nên $\overrightarrow{AB}=\overrightarrow{CD}$ là hệ thức sai.}{\begin{tikzpicture}[scale=1,font=\footnotesize,line join = round, line cap = round, >= stealth]
				%\draw[opacity=0.3] (0,0) grid (5,5);
				\def\a{3} \def\b{2} \def\g{70}
				\coordinate (A) at (0,0);
				\coordinate (B) at ($(\a,0)$);
				\coordinate (D) at ($(A)+(\g:\b)$);
				\coordinate (C) at ($(B)+(D)-(A)$);
				\draw (A)--(B)--(C)--(D)--cycle;
				\foreach \p/\g in {A/-90,C/90,B/-90,D/90} \draw[fill] (\p) circle(.5pt)node [shift={(\g:.3)}] {$\p$};
			\end{tikzpicture}}}
\end{ex}

\begin{ex}%[0H5H2-2]%[Nguyễn Thắng,DA3-DC-NTH-T10]
	Cho ba điểm phân biệt $A$, $B$, $C$. Trong các khẳng định sau, khẳng định nào sai?
	\choice
	{$\overrightarrow{AB}+\overrightarrow{BC}=\overrightarrow{AC}$}
	{$\overrightarrow{AC}+\overrightarrow{CB}=\overrightarrow{AB}$}
	{$\overrightarrow{CA}+\overrightarrow{BC}=\overrightarrow{BA}$}
	{\True $\overrightarrow{CB}+\overrightarrow{AC}=\overrightarrow{BA}$}
	\loigiai{
		Ta có $\overrightarrow{CB}+\overrightarrow{AC}=\overrightarrow{AB}$.
	}
\end{ex}

\begin{ex}%[0H5H3-2]
	Cho ba điểm phân biệt $A$, $B$, $C$. Nếu $\vec{AB}=-3\vec{AC}$ thì đẳng thức nào dưới đây \textbf{đúng}?
	\choice
	{$\vec{BC}=-4\vec{AC}$}
	{$\vec{BC}=-2\vec{AC}$}
	{$\vec{BC}=2\vec{AC}$}
	{\True $\vec{BC}=4\vec{AC}$}
	\loigiai{
		Ta có $\vec{AB}=-3\vec{AC}$ nên $\vec{AC}+\vec{CB}=-3\vec{AC}$ $\Leftrightarrow$ $4\vec{AC}=\vec{BC}$.
	}
\end{ex}

\begin{ex}%[0H5N4-1]%[CD-Lớp 10-Ôn tập cuối học kì 1-Đề 5]%[Hoàng Ngọc Lâm]
	Cho tam giác $A B C$ đều cạnh $a$. Tích vô hướng $\overrightarrow{A B}\cdot \overrightarrow{A C}$ có giá trị là
	\choice
	{\True $\overrightarrow{A B}\cdot \overrightarrow{A C}=\dfrac{a^2}{2}$}
	{$\overrightarrow{A B}\cdot \overrightarrow{A C}=-\dfrac{a^2}{2}$}
	{$\overrightarrow{A B}\cdot \overrightarrow{A C}=\dfrac{\sqrt{3}}{2}a^2$}
	{$\overrightarrow{A B}\cdot \overrightarrow{A C}=-\dfrac{\sqrt{3}}{2}a^2$}
	\loigiai{
		$\overrightarrow{A B} \cdot \overrightarrow{A C}=A B \cdot A C \cdot \cos A= a\cdot a \cdot \cos 60^{\circ}=\dfrac{a^2}{2}$.
	}
\end{ex}

\begin{ex}%[Mức 2]%[Dự án Giảng 10-11 Nhóm Toán & LaTex, Lê Minh Thiện Anh]%[0H5H4-2]
	Cho hình vuông $ABCD$ có cạnh $a$. Khẳng định nào sau đây là đúng?
	\choice
	{$(\overrightarrow{AB},\overrightarrow{BD})=45^\circ$}
	{\True $(\overrightarrow{AC},\overrightarrow{BC})=45^\circ$ và $\overrightarrow{AC}\cdot\overrightarrow{BC}=a^2$}
	{$\overrightarrow{AC}\cdot\overrightarrow{BD}=a^2\sqrt{2}$}
	{$\overrightarrow{BA}\cdot\overrightarrow{BD}=-a^2$}
	\loigiai{
		\immini{Khẳng định đúng là $(\overrightarrow{AC},\overrightarrow{BC})=45^\circ$ và $\overrightarrow{AC}\cdot\overrightarrow{BC}=a^2$.\\
			Có $(\overrightarrow{AC},\overrightarrow{BC})=(\overrightarrow{AC},\overrightarrow{AD})=45^\circ$.\\
			$\overrightarrow{AC}\cdot\overrightarrow{BC}=AC\cdot BC \cdot \cos 45^\circ=a\cdot a\sqrt{2}\cdot \dfrac{\sqrt{2}}{2}=a^2$.}
		{\begin{tikzpicture}[scale=1, font=\footnotesize,>=stealth]%<DTools>
				%Định nghĩa điểm.
				\coordinate (A) at (0,3);
				\coordinate (B) at (0,0);
				\coordinate (C) at (3,0);
				\coordinate (D) at (3,3);
				%Vẽ tứ giác ABCD.
				\draw (A)--(B)--(C)--(D)--cycle;
				%Hiển thị các điểm.
				\foreach \x/\y in {A/90,B/180,C/0,D/90}{\fill (\x) circle(1pt) ($(\x)+(\y:0.3cm)$) node{$\x$};}
			\end{tikzpicture}}
	}
\end{ex}

\begin{ex}%[0D8N1-1]
	Lớp bạn An dự định tham gia thi văn nghệ do Đoàn trường triển khai nhân dịp kỷ niệm 26/3. Có $4$ bạn đăng ký tiết mục đơn ca, $2$ nhóm đăng ký tiết mục nhảy hiện đại và $2$ nhóm đăng ký tiết mục hát múa kết hợp. Hỏi lớp bạn An có bao nhiêu cách chọn một tiết mục để dự thi?
	\choice
	{$16$}
	{$256$}
	{\True $8$}
	{$12$}
	\loigiai
	{Lớp bạn An có $4+2+2=8$ cách lựa chọn $1$ tiết mục tham gia dự thi.}
\end{ex}

\begin{ex}%[0D8H1-3]
	Bình $A$ chứa $3$ quả cầu xanh, $4$ quả cầu đỏ và $5$ quả cầu trắng. Bình $B$ chứa $4$ quả cầu xanh, $3$ quả cầu đỏ và $6$ quả cầu trắng. Bình $C$ chứa $5$ quả cầu xanh, $5$ quả cầu đỏ và $2$ quả cầu trắng. Từ mỗi bình lấy một quả cầu. Có bao nhiêu cách lấy để cuối cùng được $3$ quả có màu giống nhau.
	\choice
	{\True $180$}
	{$150$}
	{$120$}
	{$60$}
	\loigiai{
		\begin{itemize}
			\item Số cách lấy được ba quả cầu màu xanh là $3\cdot 4\cdot 5=60$.
			\item Số cách lấy được ba quả cầu màu đỏ là $4\cdot 3\cdot 5=60$.
			\item Số cách lấy được ba quả cầu màu trắng là $5\cdot 6\cdot 2=60$.
		\end{itemize}
		Vậy số cách lấy được ba quả cầu cùng màu là $60+60+60=180$.
	}
\end{ex}

\begin{ex}%[0D8H1-2]
	An muốn qua nhà Bình để cùng Bình đến chơi nhà Cường. Từ nhà An đến nhà Bình có $4$ con đường đi, từ nhà Bình đến nhà Cường có $6$ con đường đi (tham khảo hình vẽ minh họa bên dưới). Hỏi An có bao nhiêu cách chọn đường đi đến nhà Cường cùng Bình?
	\begin{center}
		\begin{tikzpicture}[every node/.style={circle, fill=red!30},scale=1]
			\tikzset{decoration={markings,mark=between positions 0.5 and 0.5 step 1 with {\draw
								(0,0)--++(0:.1);}}}
			\path (0,0) node(A){An} (3,0) node(B){Bình} (6.5,0) node(C){Cường};
			\draw[postaction={decorate},out=150,in=30] (B) to (A);
			\draw[postaction={decorate},out=170,in=10] (B) to (A);
			\draw[postaction={decorate},out=-170,in=-10] (B) to (A);
			\draw[postaction={decorate},out=-150,in=-30] (B) to (A);
			\draw[postaction={decorate},out=30,in=150] (B) to (C);
			\draw[postaction={decorate},out=10,in=170] (B) to (C);
			\draw[postaction={decorate},out=-10,in=-170] (B) to (C);
			\draw[postaction={decorate},out=50,in=130] (B) to (C);
			\draw[postaction={decorate},out=-50,in=-130] (B) to (C);
			\draw[postaction={decorate},out=-150,in=-30] (C) to (B);
		\end{tikzpicture}
	\end{center}
	\choice
	{$10$}
	{$16$}
	{\True $24$}
	{$36$}
	\loigiai{
		Từ nhà An đến nhà Bình có $4$ cách.\\
		Từ nhà Bình đến nhà Cường có $6$ cách.\\
		Vậy có $4\cdot6=24$ cách di chuyển từ nhà An qua nhà Bình đến nhà Cường.}
\end{ex}

\begin{ex}%[0D8N2-1]%[Dự án đề kiểm tra Toán 10 HK2 NH23-24- Nguyễn Cường]%[Sở GD-ĐT Bắc Ninh]
	Khẳng định nào sau đây là đúng?
	\choice
	{$\mathrm{A}_8^3=24$}
	{$\mathrm{A}_8^3=512$}
	{\True $\mathrm{A}_8^3=336$}
	{$\mathrm{A}_8^3=56$}
	\loigiai{
		Ta có $\mathrm{A}_8^3=336$.
	}
\end{ex}

\begin{ex}%[0D8H2-3]%[De-chuan-hoa-so-1]%[Viết Tường]
	Từ các chữ số $1$, $5$, $6$, $7$ có thể lập được bao nhiêu chữ số tự nhiên có $4$ chữ số khác nhau?
	\choice
	{$256$}
	{$16$}
	{$20$}
	{\True $24$}
	\loigiai{
		Từ các chữ số $1$, $5$, $6$, $7$ có thể lập được số chữ số tự nhiên có $4$ chữ số khác nhau là $4!=24$.
	}
\end{ex}

\begin{ex}%[BG - 10 New - 4in1,Phú Thạch]%[0D8N3-2]
	Tính tổng các hệ số là số lẻ trong khai triển $(x+1)^5$.
	\choice
	{$2$}
	{\True $12$}
	{$10$}
	{$15$}
	\loigiai{
		Ta có $(x+1)^5=x^5+5x^4+10x^3+10x^2+5x+1$.\\
		Suy ra tổng các hệ số là số lẻ là $1+5+5+1=12$.
	}
\end{ex}

\begin{ex}%[BG - 10 New - 4in1,Phú Thạch]%[0D8H3-3]
	Hệ số của $x^4$ trong khai triển $(1-x)^5$ là
	\choice
	{\True $5$}
	{$-5$}
	{$10$}
	{$-10$}
	\loigiai{
		Theo công thức nhị thức Newton ta có
		\begin{eqnarray*}
			(1-x)^5 & =&1+5 \cdot(-x)+10 \cdot(-x)^2+10 \cdot(-x)^3+5 \cdot(-x)^4+1 \cdot(-x)^5 \\
			& =&1-5 x+10 x^2-10 x^3+5 x^4-x^5.
		\end{eqnarray*}
	}
\end{ex}

\begin{ex}%[De-chuan-hoa-so-14]%[Lê Đạt]%[0H9N1-3]
	Trong mặt phẳng tọa độ $Oxy$, cho $\vec{u}=-2\vec{i}+3\vec{j}$. Tìm tọa độ véc-tơ $\vec{u}$.
	\choice
	{$\vec{u}=(-2\vec{i};3\vec{j})$}
	{$\vec{u}=(2;-3)$}
	{\True $\vec{u}=(-2;3)$}
	{$\vec{u}=(3;-2)$}
	\loigiai{
		Tọa độ véc-tơ $\vec{u}$ là $(-2;3)$.
	}
\end{ex}

\begin{ex}%[0H9H1-3]%[KNTT - Lớp 10 - Ôn tập cuối học kì 1 - Đề 6]%[Nguyễn Hoài Nam]
	Trong mặt phẳng tọa độ $Oxy$, cho hai điểm $A(2;2)$, $B(5;-2)$. Tìm điểm $M$ thuộc trục hoành sao cho $\widehat{AMB} = 90^\circ$?
	\choice
	{$M(0;1)$}
	{\True $M(6;0)$}
	{$M(1;6)$}
	{$M(0;6)$}
	\loigiai{
		Ta có $M \in Ox$ nên $M(m;0)$ và $\overrightarrow{AM}=(m-2;-2)$, $\overrightarrow{BM}=(m-5;2)$.\\
		Vì $\widehat{AMB} = 90^\circ$ suy ra $\overrightarrow{AM}\cdot \overrightarrow{BM}=0$.\\
		Suy ra $(m-2)(m-5)+(-2)\cdot 2=0 \Leftrightarrow m^2-7m+6=0 \Leftrightarrow \hoac{&m=1\\&m=6.}$\\
		Vậy $M(1;0)$ hoặc $M(6;0)$.
	}
\end{ex}
\begin{ex}%[0H9N1-3]  %[Dự án đề kiểm tra Toán 10 HKII NH23-24- Thầy Hải Toán]%[THPT Đống Đa Hà Nội]
	Trong mặt phẳng $Oxy$, cho ba điểm $A(-2;3)$, $B(1;0)$, $C(3;-1)$ không thẳng hàng. Tứ giác $ABCD$ là hình bình hành khi điểm $D$ có tọa độ nào sau đây?
	\choice
	{\True $(0;2)$}
	{$(2;0)$}
	{$(0;-2)$}
	{$(6;-4)$}
	\loigiai{
		Gọi $D\left(x_D;y_D\right)$ là đỉnh của hình bình hành $ABCD$.\\
		$\vv{BA}=\left(-3;3\right)$, $\vv{CD}=\left(x_D-3;y_D+1\right)$
		Suy ra $\vv{BA}=\vv{CD}\Rightarrow\heva{& x_D-3=-3 \\ & y_D+1=3}\Leftrightarrow\heva{& x_D=0 \\ & y_D=2.}$\\
		Vậy toạ độ điểm $D(0;2)$.
	}
\end{ex}
\begin{ex}
	Cho $\overrightarrow{a} = (a_1; a_2)$, $\overrightarrow{b} = (b_1; b_2)$. Khi đó tích vô hướng của hai vectơ $\overrightarrow{a}$ và $\overrightarrow{b}$ được tính theo công thức
	\choice
	{$\overrightarrow{a} \cdot \overrightarrow{b} = a_1b_2 + a_2b_1$}
	{\True $\overrightarrow{a} \cdot \overrightarrow{b} = a_1b_1 + a_2b_2$.}
	{$\overrightarrow{a} \cdot \overrightarrow{b} = (a_1b_1; a_2b_2)$}
	{$\overrightarrow{a} \cdot \overrightarrow{b} = (a_1+b_1; a_2+b_2)$}
\end{ex}

\begin{ex}%[Dự án VNMT-9, Đoàn Mạnh Hùng]%[0H9N2-1]
	Trong hệ trục toạ độ $Oxy$, vectơ $\overrightarrow{a}=(1;-2)$ và $\overrightarrow{b}=(-1;-3)$. Tính góc giữa hai vectơ $\overrightarrow{a}$ và $\overrightarrow{b}$.
	\choice
	{$30^\circ $}
	{\True $45^\circ $}
	{$60^\circ $}
	{$90^\circ $}
	\loigiai{
		Ta có $\cos \left(\overrightarrow{a};\overrightarrow{b} \right)=\dfrac{\overrightarrow{a}\cdot \overrightarrow{b}}{\left|\overrightarrow{a}\right|\cdot \left|\overrightarrow{b}\right|}=\dfrac{1\cdot (-1)+(-2)\cdot (-3)}{\sqrt{1^2+(-1)^2}\cdot \sqrt{(-3)^2+(-1)^2}}=\dfrac{\sqrt{2}}{2}\Rightarrow \left(\overrightarrow{a};\overrightarrow{b}\right)=45^\circ $.
	}
\end{ex}

\begin{ex}%[0H9H2-2]
	Trong mặt phẳng tọa độ $Oxy$, cho $\overrightarrow{u}=(1;-2),\overrightarrow{v}=(-2;1)$. Khẳng định nào sau đây \textbf{sai}?
	\choice
	{$\overrightarrow{u} \cdot \overrightarrow{v}=-4$}
	{$\left|\overrightarrow{u}\right|=\sqrt{5}$}
	{\True $\overrightarrow{u} \perp \overrightarrow{v}$}
	{$\left|\overrightarrow{u}\right|=\left|\overrightarrow{v}\right|$}
	\loigiai{
		Với $\overrightarrow{u}=(1;-2)$, $\overrightarrow{v}=(-2;1)$ ta có $\overrightarrow{u} \cdot \overrightarrow{v}=1\cdot (-2)+(-2)\cdot 1=-4 \neq 0$ nên $\overrightarrow{u} \perp \overrightarrow{v}$ sai.
	}
\end{ex}
\begin{ex}%[0H5H3-5]
	Cho tam giác $ABC$ có trung tuyến $BM$ và trọng tâm $G$. Khi đó $\overrightarrow{BG} =$
	\choice
	{$\overrightarrow{BA} + \overrightarrow{BC}$}
	{$\dfrac{1}{2}\left(\overrightarrow{BA} + \overrightarrow{BC}\right)$}
	{$\dfrac{1}{3} \overrightarrow{BA} + \overrightarrow{BC}$}
	{\True $\dfrac{1}{3}\left(\overrightarrow{BA} + \overrightarrow{BC}\right)$}
	\loigiai{
		\immini{
			Ta có $\overrightarrow{BG} = \dfrac{2}{3} \overrightarrow{BM} = \dfrac{2}{3} \cdot \dfrac{1}{2}\left(\overrightarrow{BA} + \overrightarrow{BC}\right) = \dfrac{1}{3}\left(\overrightarrow{BA} + \overrightarrow{BC}\right)$.
		}{
			\begin{tikzpicture}[>=stealth,line join=round,line cap=round,font=\footnotesize,scale=1]
				\tikzset{
				pics/tamgiacbg/.style n args={3}{
				code={
				\tikzset{
					% Khai báo độ dài cạnh va 2 goc
					declare function={a=4;goc1=70;goc2=-40;}
				}
				% Vẽ tam giác
				\path (0,0)coordinate (#1)--+(0:a)coordinate (#2)
				($(#1)!{sin(goc1)*.2}!{goc1}:(#2)$)coordinate (x)
				($(#2)!{sin(goc2)*.2}!{goc2}:(#1)$)coordinate (y)
				(intersection of #1--x and #2--y)coordinate (#3)
				;
				\foreach \pointo/\pointt in {#1/#3,#1/#2,#2/#3}{
						\draw[fill=black](\pointo)--(\pointt);
					}
				}
				}
				}
				\path
				(0,0)pic{tamgiacbg={B}{C}{A}}
				($(C)!.5!(A)$)coordinate (M)
				($(B)!2/3!(M)$)coordinate (G)
				;
				\foreach \pointo/\pointt in {B/M}{
						\draw[fill=black](\pointo)--(\pointt);
					}
				\foreach \point/\goc in {A/90,B/190,C/-10,M/60,G/120}{
						\draw[fill=black](\point)circle(.8pt)+(\goc:2mm)node[scale=.8]{$\point$};
					}
			\end{tikzpicture}
		}
	}
\end{ex}

\begin{ex}
	Cho hình chữ nhật $ABCD$ có $AB=a$ và $AD=a\sqrt{2}$. Gọi $K$ là trung điểm của cạnh $AD$. Tính $\vec{BK} \cdot \vec{AC}$.
	\choice
	{$\vec{BK} \cdot \vec{AC}=0$}
	{$\vec{BK} \cdot \vec{AC}=-a^2\sqrt{2}$}
	{$\vec{BK} \cdot \vec{AC}=a^2\sqrt{2}$}
	{$\vec{BK} \cdot \vec{AC}=2a^2$}
	\loigiai{.\\
		Ta có $AC=BD=\sqrt{AB^2+AD^2}=\sqrt{2a^2+a^2}=a\sqrt{3}$.\\
		Ta có $\heva{& \vec{BK}=\vec{BA}+\vec{AK}=\vec{BA}+\dfrac{1}{2}\vec{AD} \\& \vec{AC}=\vec{AB}+\vec{AD}}$\\
		$\xrightarrow{{}}\vec{BK} \cdot \vec{AC}=\left(\vec{BA}+\dfrac{1}{2}\vec{AD}\right)\left(\vec{AB}+\vec{AD}\right)$\\

	}
\end{ex}
\begin{ex}%[De-chuan-hoa-so-13]%[Huỳnh Quy]%[0H9H2-5]
	Trong mặt phẳng tọa độ $Oxy$, cho hai điểm $A(1;5)$ và $B(8;4)$ Tìm tọa độ điểm $C$ thuộc trục tung sao cho tam giác $ABC$ vuông tại $A$.
	\choice
	{$(3;0)$}
	{$(-1;0)$}
	{\True $\left(0;- 2 \right)$}
	{$(0;4)$}
	\loigiai{
		Ta có $\overrightarrow{AB} = (7 ; -1 )$.\\
		Vì $C$ thuộc $Oy$ nên $C(0;c)$, khi đó $\overrightarrow{AC} = (-1; c-5)$.\\
		Tam giác $ABC$ vuông tại $A$ khi $ \overrightarrow{AB} \cdot \overrightarrow{AC} = 0 \Leftrightarrow -7 + 5 - c = 0 \Leftrightarrow c = -2$. Vậy $C(0;-2)$.
	}
\end{ex}

\begin{ex}%[Đề kiểm tra HKI THPT Trần Phú, TPHCM]%[Nguyễn Vương Hiển]%[0H4H2-2]
	Cho tam giác $ABC$ có $AB=5$, $BC=8$, $\widehat{ABC}=60^{\circ}$. Tính  bán kính đường tròn ngoại tiếp tam giác $ABC$.
	\loigiai{
		Áp dụng định lý cô-sin trong $\triangle ABC$, ta có
		\[AC=\sqrt{AB^2+BC^2-2\cdot AB\cdot BC\cdot\cos B}=\sqrt{5^2+8^2-2\cdot5\cdot8\cdot\cos60^\circ}=7.\]
		Áp dụng hệ quả định lý sin, ta có $R=\dfrac{AC}{2\sin B}=\dfrac{7}{2\cdot\sin60^\circ}=\dfrac{7\sqrt{3}}{3}$.
	}
\end{ex}
\begin{ex}
	Có bao nhiêu lớn hơn $100$ trong khai triển nhị thức Newton $(3x+2y)^5$
\end{ex}

\begin{ex}%[24-25 giảng K10 - K11, Phan Quốc Trí]%[0D8V2-4]
	Một nhóm học sinh có $6$ bạn nam và $5$ bạn nữ. Có bao nhiêu cách chọn ra $5$ bạn học sinh sao cho có đủ cả nam và nữ?
	% \shortans[oly]{455}
	\loigiai{
	\textbf{Cách 1:}\\
	Các học sinh chọn ra có cả nam và nữ nên ta có các trường hợp
	\begin{center}
		\begin{tabular}{|c|c|c|}
			\hline
			\text{Số học sinh nam} & \text{Số học sinh nữ} & \text{Số cách chọn}                    \\ \hline
			$1$                    & $4$                   & $\mathrm{C}^1_6 \times \mathrm{C}^4_5$ \\ \hline
			$2$                    & $3$                   & $\mathrm{C}^2_6 \times \mathrm{C}^3_5$ \\ \hline
			$3$                    & $2$                   & $\mathrm{C}^3_6 \times \mathrm{C}^2_5$ \\ \hline
			$4$                    & $1$                   & $\mathrm{C}^4_6 \times \mathrm{C}^1_5$ \\ \hline
		\end{tabular}
	\end{center}
	Vậy có tất cả $\mathrm{C}^1_6 \times \mathrm{C}^4_5 + \mathrm{C}^2_6 \times \mathrm{C}^3_5 + \mathrm{C}^3_6 \times \mathrm{C}^2_5 + \mathrm{C}^4_6 \times \mathrm{C}^1_5 = 455$ cách chọn thoả mãn.\\
	\textbf{Cách 2: Dùng phần bù.}\\
	Số cách chọn $5$ học sinh tuỳ ý từ $11$ học sinh là $\mathrm{C}^5_{11}$ cách.\\
	Số cách chọn $5$ học sinh nam là $\mathrm{C}^5_6$ cách.\\
	Số cách chọn $5$ học sinh nữ là $\mathrm{C}^5_5$ cách.\\
	Vậy có $\mathrm{C}^5_{11} - \mathrm{C}^5_6 - \mathrm{C}^5_5 = 455$ cách chọn $5$ học sinh có đủ cả nam và nữ.
	}
\end{ex}

\begin{ex}
	Cho $vec{a}$, $\vec{b}$ thoả $|\vec{a}|=3$, $|\vec{b}|=2$ và hai vectơ $2\vec{a}+3\vec{b}$ với $\vec{a}-2\vec{b}$ vuông góc nhau. Tính độ dài vectơ $(\vec{a}-\vec{b})$
\end{ex}

\begin{ex}%[0H9C2-6]%[CTST-Lop 10-On-tap-cuoi-hoc-ki-1-De-6]%[Trần Hưng]
	Trong mặt phẳng tọa độ $Oxy$, cho điểm $M(3;1)$. Giả sử $A(a;0)$ và $B(0;b)$ là hai điểm sao cho tam giác $MAB$ vuông tại $M$ và có diện tích nhỏ nhất. Tính giá trị của biểu thức $T=a^{2}+b^{2}$.
	\loigiai{
	Ta có $\overrightarrow{MA}=(a-3;-1)$, $\overrightarrow{MB}=(-3;b-1)$.\\
	Tam giác $MAB$ vuông tại $M$ khi và chỉ khi
	\[\overrightarrow{MA}\cdot \overrightarrow{MB}=0\Leftrightarrow -3(a-3)-(b-1)=0\Leftrightarrow b=10-3a.\]
	Với $a\geq 0$, $b\geq 0$, suy ra $0\leq a\leq \dfrac{10}{3}$.\\
	Ta có
	\[S_{MAB}=\dfrac{1}{2}MA\cdot MB=\dfrac{1}{2}\sqrt{(a-3)^{2}+1}\cdot \sqrt{9+(b-1)^{2}}=\dfrac{3}{2}(a^{2}-6a+10)=\dfrac{3}{2}(a-3)^{2}+\dfrac{3}{2}\geq \dfrac{3}{2}.\]
	Do đó, $\min S_{MAB}=\dfrac{3}{2}$ đạt được khi $a=3$, khi đó $b=1$.\\
	Vậy $T=a^{2}+b^{2}=10$.
	}
\end{ex}

\Closesolutionfile{ans}
