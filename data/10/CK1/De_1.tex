\begin{name}
	{\tenchude}
	{\tendethi}
	{\tentruong}
	{\thoigian}
\end{name}
\setcounter{ex}{0}\setcounter{bt}{0}
\TN
\Opensolutionfile{ans}[ans/ansDe1-TN1]
\begin{ex}%[0-HK1-161718, THPT Nguyễn Trãi - Đà Nẵng, 2016-2017]%[Lê Văn Toàn]%[0D1N1-1]
	Câu nào sau đây \textbf{không} phải là mệnh đề?
	\choice
	{$x^2+1\ge 0$}
	{$3-2=1$}
	{$\pi >3$}
	{\True Mấy giờ rồi?}
	\loigiai{
		\begin{itemize}
			\item $x^2+1\ge 0$ là khẳng định luôn đúng với mọi $x$ nên cũng được xem là mệnh đề. \\(Chú ý: các mệnh đề chứa biến nhưng tính đúng sai không cố định thì không là mệnh đề.)
			\item $3-2=1$ là mệnh đề đúng.
			\item $\pi >3$ là mệnh đề đúng.
			\item Mấy giờ rồi? là câu hỏi nên không là một mệnh đề.
		\end{itemize}
	}
\end{ex}

\begin{ex}%[0D1H1-3]
	Lập mệnh đề phủ định của mệnh đề: \lq\lq Số $6$ chia hết cho $2$ và $3$\rq\rq.
	\choice
	{\lq\lq Số $6$ chia hết cho $2$ hoặc $3$\rq\rq }
	{\lq\lq Số $6$ không chia hết cho $2$ và $3$\rq\rq }
	{\True \lq\lq Số $6$ không chia hết cho $2$ hoặc $3$\rq\rq }
	{\lq\lq Số $6$ không chia hết cho $2$ và chia hết cho $3$\rq\rq }
	\loigiai{
		Phủ định của mệnh đề \lq\lq Số $6$ chia hết cho $2$ và $3$\rq\rq~ là mệnh đề: \lq\lq Số $6$ không chia hết cho $2$ hoặc $3$\rq\rq.}
\end{ex}

\begin{ex}%[0-HK2-2021, THPT Tam Dương 2 - Vĩnh Phúc, 2020-2021]%[Đỗ Văn Dự]%[0D1N2-1]
	Cho hai tập hợp $X=\left\{ 1;3;4;6;9\right\}$ và $Y=\left\{ -1;0;6;7;9\right\}$. Tập hợp $X\cup Y$ có bao nhiêu phần tử?
	\choice
	{ $10$}
	{ $9$}
	{\True $8$}
	{ $7$}
	\loigiai{
		Ta có $X\cup Y=\left\{ 1; 3; 4; 6; 9; -1; 0; 7 \right\}$.\\
		Vậy tập hợp $X\cup Y$có $8$ phần tử.}
\end{ex}

\begin{ex}%[Dự án Giảng 10-11 Nhóm Toán & LaTex, Mui Doan]%[0D1H3-3]
	Cho các tập hợp $A=(-5;3)$ và $B=[-2;7)$. Tìm $A\cup B$.
	\choice
	{$[-2;3)$}
	{$(-5;-2)$}
	{\True $(-5;7)$}
	{$[3;7)$}
	\loigiai
	{
		Ta có $A\cup B=(-5;7)$.
	}
\end{ex}

\begin{ex}%[Dự án Giảng 10-11 Nhóm Toán & LaTex, Lê Minh Thiện Anh]%[0D1H3-5]
	Lớp 10A có $40$ học sinh, trong đó có $20$ học sinh thích môn Ngữ văn, $18$ học sinh thích môn Toán, $4$ học sinh thích cả hai môn Ngữ văn và Toán. Hỏi có bao nhiêu học sinh không thích môn nào trong hai môn Ngữ văn và Toán?
	\choice
	{$5$}
	{\True $6$}
	{$7$}
	{$8$}
	\loigiai{
		Ta có $20+18-4=34$ học sinh hoặc thích môn Toán hoặc thích môn Ngữ văn.\\
		Do đó có $40-34=6$ học sinh không thích môn nào trong hai môn Toán và Ngữ văn.
	}
\end{ex}

\begin{ex}%[Đề GK1 trường THPT Chuyên Hùng Vương]%[An Le, 10EX-HK1-2223]%[0D2N1-1]
	Cho bất phương trình $2 x+3 y-6 \leq 0$\ \ $(1)$. Chọn khẳng định đúng trong các khẳng định sau
	\choice
	{Bất phương trình $(1)$ chỉ có một nghiệm duy nhất}
	{\True Bất phương trình $(1)$ có vô số nghiệm}
	{Bất phương trình $(1)$ vô nghiệm}
	{Bất phương trình $(1)$ có tập nghiệm là $\mathbb{R}$}
	\loigiai{
		Bất phương trình $(1)$ có vô số nghiệm, chẳng hạn cặp số $(3t;-2t+2)$ với mọi $t\in\mathbb{R}$ là các nghiệm (thỏa mãn dấu bằng).
	}
\end{ex}

\begin{ex}%[0D2H1-2]%[Đề chuẩn hóa số 2]%[BCTuan]
	Phần không bị gạch kể cả bờ trong hình vẽ là miền nghiệm của bất phương trình nào sau đây?
	\begin{center}
		\begin{tikzpicture}[font=\footnotesize, line join=round, line cap=round, >=stealth,thick]
			\tikzset{every node/.style={scale=0.7}}
			\begin{scope}
				\clip (-2,-3) rectangle (2,2);
				\fill[pattern=north east lines] (-3,-4)--(4,-4)--(4,3)--cycle;
				\draw (3,2)--(-2,-3) node [pos=0.45, above, sloped] {$x-y-1=0$};
			\end{scope}
			\draw[->] (-2,0)--(2,0) node[below]{$x$};
			\draw[->] (0,-3)--(0,2) node[left]{$y$};
			\draw (0,0) node[below left]{$O$};
			\foreach \x in {1}
			\draw[thin] (\x,1pt)--(\x,-1pt) node [below] {$\x$};
			\foreach \y in {-1}
			\draw[thin] (1pt,\y)--(-1pt,\y) node [left] {$\y$};
		\end{tikzpicture}
	\end{center}
	\choice
	{\True $x-y\leq 1$}
	{$x+y\leq 1$}
	{$x+y>1$}
	{$x-y<1$}
	\loigiai{
		Vì miền nghiệm kể luôn bờ nên loại 	$x+y>1$ và loại $x-y<1$.\\
		Lấy điểm $O(0;0)$ không thuộc phần bị gạch chéo thế vào $x-y\leq 1$ ta được $0-0\leq 1$ là bất đẳng thức đúng nên phần không bị gạch kể cả bờ trong hình vẽ là miền nghiệm của bất phương trình \[x-y\leq 1.\]
	}
\end{ex}

\begin{ex}%[0D2N2-1]
	Trong các cặp số $(x;y)$ sau, cặp nào là nghiệm của hệ bất phương trình $\heva{&2x>y-1\\&x+2y\leq 3.}$
	\choice
	{$(1;2)$}
	{\True $(1;0)$}
	{$(1;4)$}
	{$(1;3)$}
	\loigiai{
		\begin{itemize}
			\item Thế $x=1$, $y=2$ vào bất phương trình ta được $\heva{&2\cdot 1>2-1\\ &1+2\cdot 2\leq 3}$ không thỏa.\\
			      Vậy $(1;2)$ không là nghiệm của hệ bất phương trình.
			\item Thế $x=1$, $y=0$ vào bất phương trình ta được $\heva{&2\cdot 1>0-1\\ &1+2\cdot 0\leq 3}$  thỏa.\\
			      Vậy $(1;0)$ là nghiệm của hệ bất phương trình.
			\item Thế $x=1$, $y=3$ vào bất phương trình ta được $\heva{&2\cdot 1>3-1\\ &1+2\cdot 3\leq 3}$ không thỏa.\\
			      Vậy $(1;3)$ không là nghiệm của hệ bất phương trình.
			\item Thế $x=1$, $y=4$ vào bất phương trình ta được $\heva{&2\cdot 1>4-1\\ &1+2\cdot 4\leq 3}$ không thỏa.\\
			      Vậy $(1;4)$ không là nghiệm của hệ bất phương trình.
		\end{itemize}
	}
\end{ex}

\begin{ex}%[0-HK1-KN-9-ChuongMyB-HaNoi-2324]%[VN-MT-6,Nguyễn Tuấn]%[0D2H2-2]
	Miền nghiệm của hệ bất phương trình $\heva{&x-y\ge 2021 \\& x+y \le 2022}$ {\bf không} chứa điểm nào sau đây?
	\choice
	{$(1001;-1021)$}
	{\True $(-2021;2022)$}
	{$(2021;-2022)$}
	{$(2021; 0)$}
	\loigiai{
		Vì $-2021-2022=-4043<2021$ nên miền nghiệm của hệ bất phương trình $\heva{&x-y\ge 2021 \\& x+y \le 2022}$ {\bf không} chứa điểm $(-2021;2022)$.
	}
\end{ex}

\begin{ex}%[0H4N2-1]
	Cho tam giác $ABC$, khẳng định nào sau đây đúng?
	\choice
	{$AB^2=AC^2+BC^2-2AC\cdot AB\cos C$}
	{$AB^2=AC^2-BC^2+2AC\cdot BC\cos C$}
	{\True $AB^2=AC^2+BC^2-2AC\cdot BC\cos C$}
	{$AB^2=AC^2+BC^2-2AC\cdot BC+\cos C$}
	\loigiai{
		Áp dụng định lí cosin trong $\triangle ABC$ ta có
		\[AB^2=AC^2+BC^2-2AC\cdot BC\cos C\] }
\end{ex}

\begin{ex}%[De-chuan-hoa-so-16]%[Mui Doan]%[0H4N2-2]
	Cho tam giác $ABC$ với $BC=a,AC=b,AB=c$ và $p=\dfrac{a+b+c}{2}$. Diện tích $S$ của $\triangle AB C$ được tính bằng công thức nào?
	\choice
	{\True $S=\sqrt{p(p-a)(p-b)(p-c)}$}
	{$S=p(p-a)(p-b)(p-c)$}
	{$S=\sqrt{(p-a)(p-b)(p-c)}$}
	{$S=\dfrac{1}{2}\sqrt{p(p-a)(p-b)(p-c)}$}
	\loigiai{
		Diện tích $S$ của $\triangle AB C$ được tính bằng công thức $S=\sqrt{p(p-a)(p-b)(p-c)}$.
	}
\end{ex}

\begin{ex}%[BG12-3in1, Nguyễn Xuân Bảo-Phạm Ngọc Trung]%[0H4H2-1]
	Trong tam giác $ABC$, có $AB=5$, $AC=4$, $\widehat{A}=60^{\circ}$. Tính $BC$.
	\choice
	{\True $\sqrt{21}$}
	{$2\sqrt{5}$}
	{$3$}
	{$5$}
	\loigiai{
		Ta có  $BC^2=AB^2+AC^2-2\cdot AB \cdot AC \cdot \cos A=5^2+4^2-2 \cdot 5 \cdot 4 \cdot \cos 60^{\circ}=21 \Rightarrow BC=\sqrt{21}$.
	}
\end{ex}

\begin{ex}%[De-chuan-hoa-so-6]%[Đình Nguyên]%[0H4H1-2]
	Trong mặt phẳng tọa độ $Oxy$, lấy điểm $M$ thuộc nửa đường tròn đơn vị sao cho $\widehat{xOM}=45^\circ$. Tổng hoành độ và tung độ của điểm $M$ bằng
	\choice
	{\True $\sqrt{2}$}
	{$\dfrac{\sqrt{2}}{2}$}
	{$2$}
	{$2 \sqrt{2}$}
	\loigiai{
		Ta có hoành độ điểm $M$ là $x_M=\cos 45^\circ=\dfrac{\sqrt{2}}{2}$; tung độ điểm $M$ là $y_M=\sin 45^\circ=\dfrac{\sqrt{2}}{2}$.\\
		Vậy $x_M+y_M=\dfrac{\sqrt{2}}{2}+\dfrac{\sqrt{2}}{2}=\sqrt{2}$.
	}

\end{ex}

\begin{ex}%[0H5N1-3]
	Hai vectơ được gọi là bằng nhau nếu
	\choice
	{\True Chúng có cùng hướng và cùng độ dài}
	{Chúng có hướng ngược nhau và cùng độ dài}
	{Chúng có cùng độ dài}
	{Chúng có cùng phương và cùng độ dài}
	\loigiai{
		Theo định nghĩa hai vectơ bằng nhau ta có \lq\lq Hai vectơ được gọi là bằng nhau nếu chúng có cùng hướng và cùng độ dài\rq\rq.
	}
\end{ex}

\begin{ex}%[0H5N2-2]%[Nguyễn Thắng,DA3-DC-NTH-T10]
	Cho ba vectơ $\overrightarrow{a}$, $\overrightarrow{b}$ và $\overrightarrow{c}$ khác vectơ-không. Trong các khẳng định sau, khẳng định nào sai?
	\choice
	{$\overrightarrow{a}+\overrightarrow{b}=\overrightarrow{b}+\overrightarrow{a}$}
	{$\left (\overrightarrow{a}+\overrightarrow{b}\right )+\overrightarrow{c}=\overrightarrow{a}+\left (\overrightarrow{b}+\overrightarrow{c}\right )$}
	{$\overrightarrow{a}+\overrightarrow{0}=\overrightarrow{a}$}
	{\True  $\overrightarrow{0}+\overrightarrow{a}=\overrightarrow{0}$}
	\loigiai{
		Ta có $\overrightarrow{a}+\overrightarrow{0}=\overrightarrow{a}$ nên câu sai là $\overrightarrow{0}+\overrightarrow{a}=\overrightarrow{0}$.
	}
\end{ex}

\begin{ex}%[0H5H2-2]%[CTST - Lớp 10 - Ôn tập cuối học kì 1 - Đề 5]%[Thầy Tú TLH]
	Cho đoạn thẳng $AB$, gọi $M$ là trung điểm của $AB$. Đẳng thức vectơ nào sau đây đúng?
	\choice
	{$\overrightarrow{AB}=2\overrightarrow{MA}$}
	{$\overrightarrow{AM}=\overrightarrow{BM}$}
	{\True $\overrightarrow{AB}=2\overrightarrow{AM}$}
	{$\overrightarrow{AB}=2\overrightarrow{BM}$}
	\loigiai{
		Ta có $\overrightarrow{AB}=2\overrightarrow{AM}$.
	}
\end{ex}

\begin{ex}%[0H5H3-2]%[KNTT - Lớp 10 - Ôn tập cuối học kì 1 - Đề 2]%[TanNguyen]
	Cho tam giác $A B C$ có trọng tâm $G$ và $M$ là trung điểm của cạnh $B C$. Mệnh đề nào sau đây là \textbf{sai}?
	\choice
	{$\overrightarrow{A G}=\dfrac{1}{3} \overrightarrow{A B}+\dfrac{1}{3} \overrightarrow{A C}$}
	{$\overrightarrow{A B}+\overrightarrow{A C}=2 \overrightarrow{A M}$}
	{$\overrightarrow{A G}=\dfrac{2}{3} \overrightarrow{A M}$}
	{\True $\overrightarrow{A G}=\dfrac{1}{3} \overrightarrow{A B}+\dfrac{2}{3} \overrightarrow{A C}$}
	\loigiai
	{
		Vì $M$ là trung điểm của cạnh $B C$ nên $\overrightarrow{A B}+\overrightarrow{A C}=2 \overrightarrow{A M}$.\\
		Vì $G$ là trọng tâm của tam giác $A B C$ nên $\overrightarrow{A G}=\dfrac{2}{3} \overrightarrow{A M}=\dfrac{2}{3} \cdot \dfrac{1}{2}(\overrightarrow{A B}+\overrightarrow{A C})=\dfrac{1}{3}(\overrightarrow{A B}+\overrightarrow{A C})$.
	}
\end{ex}

\begin{ex}%[Dự án EX-10-11-Chuẩn hóa]%[Hoàng Thanh Phương]%[0H5N4-1]
	Cho tam giác $ABC$ đều có cạnh bằng $a$. Tính tích vô hướng $\overrightarrow{AB} \cdot \overrightarrow{AC}$.
	\choice
	{$\dfrac{a^2\sqrt{3}}{2}$}
	{$a^2$}
	{\True $\dfrac{a^2}{2}$}
	{$-\dfrac{a^2}{2}$}
	\loigiai{
		Ta có \[\overrightarrow{AB}\cdot \overrightarrow{AC} = AB\cdot AC\cdot \cos\widehat{BAC} = a\cdot a\cdot \cos  60^\circ = \dfrac{a^2}{2}.\]
	}
\end{ex}

\begin{ex}%[Mức 2]%[Dự án Giảng 10-11 Nhóm Toán & LaTex, Lê Minh Thiện Anh]%[0H5H4-2]
	Cho hình vuông ABCD, tính $ \cos\left(\overrightarrow{AB},\overrightarrow{CA}\right)$.
	\choice
	{$\dfrac{1}{2}$}
	{$-\dfrac{1}{2}$}
	{$\dfrac{\sqrt{2}}{2}$}
	{\True $-\dfrac{\sqrt{2}}{2}$}
	\loigiai{
		\immini{Vì $\left(\overrightarrow{AB},\overrightarrow{CA}\right)=180^\circ -\left(\overrightarrow{AB},\overrightarrow{AC}\right)=135^\circ $ nên $ \cos\left(\overrightarrow{AB},\overrightarrow{CA}\right)=-\dfrac{\sqrt{2}}{2}$.}
		{\begin{tikzpicture}[scale=1, font=\footnotesize,>=stealth]%<DTools>
				%Định nghĩa điểm.
				\coordinate (A) at (0,3);
				\coordinate (B) at (0,0);
				\coordinate (C) at (3,0);
				\coordinate (D) at (3,3);
				%Vẽ tứ giác ABCD.
				\draw (A)--(B)--(C)--(D)--cycle;
				%Hiển thị các điểm.
				\foreach \x/\y in {A/90,B/180,C/0,D/90}{\fill (\x) circle(1pt) ($(\x)+(\y:0.3cm)$) node{$\x$};}
			\end{tikzpicture}}
	}
\end{ex}

\begin{ex}%[0D8N1-1]%[Dự án đề kiểm tra Toán khối 10 HKI NH23-24-Dot 15- Hồ Văn Trung]%[THPT Lê Qúy Đôn- TPHCM]
	Có bao nhiêu cách chọn một học sinh từ một nhóm gồm 8 học sinh nam và 9 học sinh nữ?
	\choice
	{$72$}
	{$8$}
	{\True $17$}
	{$9$}
	\loigiai{Tổng số học sinh là $8+9=17$.\\
		Số cách chọn một học sinh là $17$ cách.}
\end{ex}

\begin{ex}%[0D8H1-3]
	Từ các chữ số $1$, $2$, $3$, $4$, $5$ có thể lập được bao nhiêu số tự nhiên bé hơn $60$?
	\choice
	{$42$}
	{\True $30$}
	{$25$}
	{$17$}
	\loigiai{
		\begin{itemize}
			\item Số cần tìm có $1$ chữ số $\Rightarrow$ có $5$ số thỏa mãn yêu cầu.
			\item Số cần tìm có $2$ chữ số $\Rightarrow$ có $5\cdot 5=25$ số thỏa mãn yêu cầu.
		\end{itemize}
		Vậy có $5+25=30$ (số thỏa mãn yêu cầu).
	}
\end{ex}

\begin{ex}%[BG - 10 New - 3in1, Nguyễn Văn Cường (Cường NV)]%[0D8H1-2]
	Trong một tuần bạn $A$ dự định mỗi ngày đi thăm một người bạn trong $12$ người bạn của mình. Hỏi bạn $A$ có thể lập được bao nhiêu kế hoạch đi thăm bạn của mình (thăm một bạn không quá một lần)?
	\choice
	{\True $3991680$}
	{$12!$}
	{$35831808$}
	{$7!$}
	\loigiai{
		Một tuần có bảy ngày và mỗi ngày thăm một bạn.
		\begin{itemize}
			\item Có $12$ cách chọn bạn vào ngày thứ nhất.
			\item Có $11$ cách chọn bạn vào ngày thứ hai.
			\item Có $10$ cách chọn bạn vào ngày thứ ba.
			\item Có $9$ cách chọn bạn vào ngày thứ tư.
			\item Có $8$ cách chọn bạn vào ngày thứ năm.
			\item Có $7$ cách chọn bạn vào ngày thứ sáu.
			\item Có $6$ cách chọn bạn vào ngày thứ bảy.
		\end{itemize}
		Vậy theo quy tắc nhân có $12\times 11\times 10\times 9\times 8\times 7\times 6=3991680$ cách chọn.
	}
\end{ex}

\begin{ex}%[Đề chuẩn hóa Toán 11]%[Nguyễn Trung Kiên]%[0D8N2-1]
	Tính số chỉnh hợp chập $4$ của $7$ phần tử.
	\choice
	{$35$}
	{\True  $840$}
	{$336$}
	{$56$}
	\loigiai{
		Số chỉnh hợp chập $4$ của $7$ phần tử là $\mathrm{A}_7^4=\dfrac{7!}{(7-4)!}=840$.
	}
\end{ex}

\begin{ex}%[0D8H2-3]
	Có bao nhiêu số tự nhiên có bảy chữ số khác nhau từng đôi một, trong đó chữ số $ 2$ đứng liền giữa hai chữ số $ 1$ và $3$ ?
	\choice
	{$3204$ số}
	{$249$ số}
	{$2942$ số}
	{\True $7440$ số}
	\loigiai{
		Gọi $\overline{a_1a_2a_3a_4a_5a_6a_7}$ là số cần lập.
		\begin{enumerate}[T/h 1.]
			\item $\{a_1;a_2;a_3\}=\{1;2;3\}$.\\
			      Vì chữ số $ 2$ đứng liền giữa hai chữ số $ 1$ và $3$ nên có $2$ cách sắp xếp thứ tự $1$, $2$, $3$ là $123$ và $321$.\\
			      Sắp xếp $4$ trong $7$ chữ số $0$, $4$,..., $9$ vào $a_4a_5a_6a_7$ có $\mathrm{A}_7^4=840$ cách.\\
			      Theo quy tắc nhân có $2 \cdot 840 = 1680$ số thoả trường hợp này.
			\item $\{a_1;a_2;a_3\} \ne \{1;2;3\}$.\\
			      Chọn $3$ vị trí liên tiếp từ $a_2$ đến $a_7$ có $4$ cách.\\
			      Đặt $123$ hoặc $321$ vào $3$ vị trí vừa chọn có $2$ cách.\\
			      Chọn $a_1$ khác $0$, $1$, $2$, $3$ có $6$ cách.\\
			      Sắp xếp $3$ trong $6$ chữ số còn lại vào $3$ vị trí còn lại có $\mathrm{A}_6^3=120$ cách.\\
			      Theo quy tắc nhân có $4 \cdot 2 \cdot 6 \cdot 120 = 5760$ số thoả trường hợp này.
		\end{enumerate}
		Vậy theo quy tắc cộng có $1680+5760=7440$ số thoả yêu cầu bài toán.
	}
\end{ex}

\begin{ex}%[BG - 10 New - 4in1,Phú Thạch]%[0D8N3-2]
	Tìm hệ số của $y^4$ trong khai triển nhị thức $(x+3y)^4$.
	\choice
	{$-81$}
	{\True $81$}
	{$27$}
	{$-27$}
	\loigiai{
		Ta có $(x+3y)^4=x^4+12x^3y+54x^2y^2+108xy^3+81 y^4$.\\
		Suy ra hệ số của $y^4$ trong khai triển là $81$.
	}

\end{ex}

\begin{ex}%[0D8H3-3]%[HKII-NGUYỄN THÁI BÌNH 2324]%[TheHung Nguyen]
	Tìm hệ số của số hạng chứa $x^3$ trong khai triển $(x-2)(2x+1)^4$.
	\choice
	{\True $-40$}
	{$-24$}
	{$24$}
	{$40$}
	\loigiai{
		\textbf{Cách 1:}
		Ta có $(x-2)(2x+1)^4=(x-2)(16x^4+32x^3+24x^2+8x+1)$.\\
		Số hạng chứa $x^3$ trong khai triển trên là $x\cdot 24x^2-2\cdot 32x^3=-40x^3$.\\
		\textbf{Cách 2:} \textit{(Áp dụng sách chuyên đề)}
		\begin{eqnarray*}
			(x-2)(2x+1)^4&= &x(2x+1)^4-2(2x+1)^4\\
			&= & x\sum\limits_{k=0}^4 \mathrm{C}_4^k(2x)^k-2\sum\limits_{k=0}^4 \mathrm{C}_4^k(2x)^k\\
			&= & \sum\limits_{k=0}^4 \mathrm{C}_4^k2^k x^{k+1}-2\sum\limits_{k=0}^4 \mathrm{C}_4^k2^k x^k
		\end{eqnarray*}
		Hệ số của số hạng chứa $x^3$ trong khai triển $(x-2)(2x+1)^4$ là
		\[\mathrm{C}_4^22^2-2\cdot \mathrm{C}_4^32^3=-40.\]
	}
\end{ex}

\begin{ex}
	Trọng tâm $G$ của tam giác $ABC$ có tọa độ là
	\choice
	{\True $\left(\dfrac{x_A+x_B+x_C}{3};\dfrac{y_A+y_B+y_C}{3}\right)$}
	{$\left(\dfrac{x_A+x_B+x_C}{2};\dfrac{y_A+y_B+y_C}{2}\right)$}
	{$\left(\dfrac{y_A+y_B+y_C}{3};\dfrac{x_A+x_B+x_C}{3}\right)$}
	{$\left(\dfrac{x_Ax_Bx_C}{3};\dfrac{y_Ay_By_C}{3}\right)$}
\end{ex}

\begin{ex}%[De-chuan-hoa-so-16]%[Mui Doan]%[0H9N1-3]
	Trong mặt phẳng tọa độ $Oxy$, cho $M\left(x_1;y_1\right)$ và $N\left(x_2;y_2\right)$. Tọa độ trung điểm $I$ của đoạn thẳng $MN$ là
	\choice
	{$I\left(\dfrac{x_1+y_1}{2};\dfrac{x_2+y_2}{2}\right)$}
	{\True $I\left(\dfrac{x_1+x_2}{2};\dfrac{y_1+y_2}{2}\right)$}
	{$I\left(\dfrac{x_1+x_2}{3};\dfrac{y_1+y_2}{3}\right)$}
	{$I\left(\dfrac{x_1-x_2}{2};\dfrac{y_1-y_2}{2}\right)$}
	\loigiai{
		Tọa độ trung điểm $I$ của đoạn thẳng $MN$ là $I\left(\dfrac{x_1+x_2}{2};\dfrac{y_1+y_2}{2}\right)$.
	}
\end{ex}

\begin{ex}%[0-HK1-KN-3-DoanThiDiem-HaNoi-2324]%[VN-MT-6,VM032]%[0H9H1-3]
	Trong mặt phẳng tọa độ $Oxy$, cho điểm $M (-3;1)$; $N (0;-1)$. Tọa độ của vectơ $\overrightarrow{MN}$ là
	\choice
	{\True $\overrightarrow{MN}=( 3;-2 )$}
	{$\overrightarrow{MN}=( -2;0 )$}
	{$\overrightarrow{MN}=( 3;0 )$}
	{$\overrightarrow{MN}=( -3;2)$}
	\loigiai{Toạ độ của vectơ $\overrightarrow{MN}$ là $\big(0-(-3);(-1)-1\big)$, hay là $\overrightarrow{MN}=(3;-2)$.}
\end{ex}

\begin{ex}%[HK1, THPT Lê Quảng Chí, 2023]%[TVN-001, 10-11EX-HK1-2324]%[0H9H1-2]
	Trong hệ tọa độ $Oxy$, cho ba điểm $A(2;1)$, $B(0;-3)$, $C(3;1)$. Tìm tọa độ điểm $D$ để $ABCD$ là hình bình hành.
	\choice
	{\True $(5;5)$}
	{$(5;-2)$}
	{$(5;-4)$}
	{$(-1;-4)$}
	\loigiai{
		Tọa độ điểm $D$ cần tìm là
		\[\heva{&x_D=x_A+x_C-x_B=2+3-0=5\\&y_D=y_A+y_C-y_B=1+1-(-3)=5.}\]}
\end{ex}

\begin{ex}%[0H9N2-1]%[Dự án GHKII - K10-11 - PNL - NH23-24]%[Thanh Thủy]
	Trong mặt phẳng với hệ tọa độ $Oxy$, cho hai vectơ $\overrightarrow{a}=(-1; 2)$ và $\overrightarrow{b}=(-3; 2)$. Kết quả của $\overrightarrow{a} \cdot \overrightarrow{b}$ bằng
	\choice
	{$(3; 4)$}
	{$-16$}
	{\True $7$}
	{$(-2;-6)$}
	\loigiai
	{
		Ta có $\overrightarrow{a} \cdot \overrightarrow{b}=(-1)\cdot (-3)+2\cdot 2=7$.
	}
\end{ex}

\begin{ex}%[0-TK-HK1-CT-6-2425]%[VN-MT-9, Phan Hoàng Anh]%[0H9H2-2]
	Trong mặt phẳng tọa độ $Oxy$, tính số đo của góc giữa hai vectơ $\overrightarrow{a}=(-2;-1)$ và \break$\overrightarrow{b}=(3;-1)$.
	\choice
	{\True $135^{\circ}$}
	{$45^{\circ}$}
	{$90^{\circ}$}
	{$60^{\circ}$}
	\loigiai{Ta có $\cos\left(\overrightarrow{a},\overrightarrow{b}\right)=\dfrac{\overrightarrow{a}\cdot\overrightarrow{b}}{\left|\overrightarrow{a}\right|\cdot\left|\overrightarrow{b}\right|}=\dfrac{-2\cdot3+(-1)\cdot(-1)}{\sqrt{(-2)^2+(-1)^2}\cdot\sqrt{3^2+(-1)^2}}=\dfrac{-6+1}{\sqrt{5}\cdot\sqrt{10}}=-\dfrac{\sqrt{2}}{2}$.\\
	Vậy $\left(\overrightarrow{a},\overrightarrow{b}\right)=135^{\circ}$.}
\end{ex}
\begin{ex}%[24-25-Bai-Giang-K10-K11, Tran Quoc]%[0H5H3-5]
	Cho hình chữ nhật $ABCD$. Gọi $I$, $K$ lần lượt là trung điểm của $BC$ và $CD$. Khẳng định nào dưới đây là đúng?
	\choice
	{\True $\overrightarrow{AI}+\overrightarrow{AK}=\dfrac{3}{2}\overrightarrow{AC}$}
	{$\overrightarrow{AI}+\overrightarrow{AK}=\overrightarrow{IK}$}
	{$\overrightarrow{AI}+\overrightarrow{AK}=\overrightarrow{AB}+\overrightarrow{AD}$}
	{$\overrightarrow{AI}+\overrightarrow{AK}=2\overrightarrow{AC}$}
	\loigiai{
		\immini
		{Vì $I$, $K$ lần lượt là trung điểm của $BC$ và $CD$ nên ta có
			\begin{align*}
				\overrightarrow{AI}+\overrightarrow{AK} & =\left(\dfrac{1}{2}\overrightarrow{AB}+\dfrac{1}{2}\overrightarrow{AC}\right)+\left(\dfrac{1}{2}\overrightarrow{AC}+\dfrac{1}{2}\overrightarrow{AD}\right) \\
				                                        & =\overrightarrow{AC}+\dfrac{1}{2}\left(\overrightarrow{AB}+\overrightarrow{AD}\right)                                                                      \\
				                                        & =\overrightarrow{AC}+\dfrac{1}{2}\overrightarrow{AC}                                                                                                       \\
				                                        & =\dfrac{3}{2}\overrightarrow{AC}.
			\end{align*}
		}
		{\begin{tikzpicture}[font=\footnotesize,line join=round, line cap=round, >=stealth,scale=0.8]
				\foreach \x/\y/\pos in {0/0/A, 4/0/B, 0/3/D} \path ($(\x,\y)$) coordinate (\pos);
				\path ($(B)+(D)-(A)$) coordinate (C) ($(B)!1/2!(C)$) coordinate (I) ($(D)!1/2!(C)$) coordinate (K);
				\draw (A)--(B)--(C)--(D)--cycle (I)--(A)--(K)--cycle (A)--(C);
				\foreach \x/\pos in {A/-150,B/-30,C/30,D/150, I/0, K/90} \fill (\x) circle(1pt) node[{shift=(\pos:0.25)}]{$\x$};
			\end{tikzpicture}}
	}
\end{ex}

\begin{ex}
	Cho hình chữ nhật $ABCD$ có $AB=3a$, $AD=4a$. Tính $P=\vec{AC} \cdot \left(\vec{DC}+\vec{DB}+\vec{DA}\right)$.
	\choice
	{\True $P=-14a^2$}
	{$P=-11a^2$}
	{$P=10a^2$}
	{$P=-7a^2$}
	\loigiai{
		\begin{align*}
			P & =\vec{AC} \cdot \left(\vec{DC}+\vec{DB}+\vec{DA}\right)                \\
			  & = \left(\vec{DC}-\vec{DA}\right)\cdot 2 \left(\vec{DA}+\vec{DC}\right) \\
			  & = 2 \left(DC^2-DA^2\right) (\text{ vì } \vec{DA}\cdot \vec{DC}=0)      \\
			  & = 2 \left(9a^2-16a^2\right) = -14a^2.
		\end{align*}
	}
\end{ex}
\begin{ex}%[0H9V1-4]
	Trong mặt phẳng tọa độ $Oxy$, cho $A(-1;2)$, $B(2;3)$. Tọa độ điểm $C$ nằm trên trục tung sao cho $A$, $B$, $C$ thẳng hàng là
	\choice
	{$C\left(0;-\dfrac{1}{3}\right)$}
	{$C\left(0;\dfrac{4}{3}\right)$}
	{\True $C\left(0;\dfrac{7}{3}\right)$}
	{$C(3;0)$}
	\loigiai{
		Ta có $C \in Oy \Leftrightarrow C(0;c)$. Suy ra $\heva{&\overrightarrow{AC}=(1;c-2)\\&\overrightarrow{AB}=(3;1).}$\\
		$A$, $B$, $C$ thẳng hàng khi và chỉ khi $\overrightarrow{AC}$ và $\overrightarrow{AB}$ cùng phương $\Leftrightarrow \dfrac{1}{3}=\dfrac{c-2}{1} \Leftrightarrow c=\dfrac{7}{3}$.\\
		Vậy $C\left(0;\dfrac{7}{3}\right)$.
	}
\end{ex}
\TL
\begin{ex}%[0-HK1-CT-6-HungVuong-HCM-2324]%[VN-MT-6, Nguyễn Hữu Đức]%[0H4H2-2]
	Cho tam giác $ABC$ có $AB=6$, $BC=9\sqrt{2}$, $CA=10$. Tính diện tích tam giác $ABC$ (Làm tròn kết quả đến hàng phần chục).
	\loigiai{
		Nửa chu vi: $p=\dfrac{AB+BC+CA}{2}=8+\dfrac{9\sqrt{2}}{2}$.\\
		Diện tích tam giác bằng
		\begin{align*}
			S & = \sqrt{p(p-AB)(p-BC)(p-AC)}                                                                                                                           \\
			  & = \sqrt{\left(8+\dfrac{9\sqrt{2}}{2}\right)\left(2+\dfrac{9\sqrt{2}}{2}\right)\left(8-\dfrac{9\sqrt{2}}{2}\right)\left(-2+\dfrac{9\sqrt{2}}{2}\right)} \\
			  & = \sqrt{\left(64-\dfrac{81}{2}\right)\left(\dfrac{81}{2}-4\right)} = \dfrac{3431}{2} \approx 29,3.
		\end{align*}
	}
\end{ex}

\begin{ex}
	Khai triển nhị thức Newton $(3x-4)^5$.
	\loigiai{
		\begin{align*}
			  & (3x-4)^5                                                                                                                           \\
			= & \ (3x)^5 +5 \cdot (3x)^4 \cdot (-4) +10 \cdot (3x)^3 \cdot (-4)^2 +10 \cdot (3x)^2 \cdot (-4)^3 +5 \cdot (3x) \cdot (-4)^4 +(-4)^5 \\
			= & \ 243x^5 - 1620x^4 + 4320x^3 - 5760x^2 + 3840x - 1024.
		\end{align*}
	}
\end{ex}

\begin{ex}%[0D8C2-3]
	Từ các số $0$, $1$, $2$, $3$, $4$, $5$, $6$, $7$, $8$ có thể lập được bao nhiêu số tự nhiên có $5$ chữ số khác nhau sao cho luôn có mặt ba chữ số $0$, $1$, $2$ và ba chữ số này luôn phải đứng cạnh nhau?
	% \shortans[oly]{$480$}
	\loigiai{Số tự nhiên có $5$ chữ số khác nhau cần lập có dạng $\overline{abcde}$.\\
		Chọn $1$ trong $3$ vị trí kề nhau trong các vị trí $abc,bcd,cde$ có $3$ cách.\\
		Xếp $3$ chữ số $0,1,2$ vào $3$ vị trí kề nhau có $3!$ cách. \\
		Chọn $2$ trong $6$ số $\{3,4,5,6,7,8\}$ xếp vào $2$ vị trí còn lại có $\mathrm{A}_6^2$ cách.\\
		Vậy có $3 \cdot 3! \cdot \mathrm{A}_6^2 =540$ số.\\
		Trong các số trên sẽ có các số có dạng $\overline{0bcde}$.\\
		Xếp $2$ số $1,2$ vào các vị trí $b,c$ có $2!$ cách.\\
		Chọn $2$ trong $6$ số $\{3,4,5,6,7,8\}$ xếp vào $2$ vị trí còn lại có $\mathrm{A}_6^2$ cách.\\
		Vậy có $2! \cdot \mathrm{A}_6^2 =60$ số có dạng $\overline{0bcde}$ trong $540$ số đã lập ở trên.\\
		Do đó số các số tự nhiên có $5$ chữ số sao cho luôn có mặt ba chữ số $0,1,2$ và ba chữ số này luôn phải đứng cạnh nhau là $540-60=480$ số.
	}
\end{ex}
\begin{ex}%[HK1, SoBacNinh, 2023-2024]%[0H5C3-8]
	Cho tam giác $ABC$ vuông tại $A$, có $AB=3$, $BC=5$. Gọi $E$ là trung điểm $AB$. Tìm tập hợp điểm $M$ thoả mãn
	\[\left|\overrightarrow{MA}+\overrightarrow{MB}+2\overrightarrow{MC}\right|=8.\]
	\loigiai{
		\immini{
			Gọi $I$ là điểm thoã mãn $\overrightarrow{IA}+\overrightarrow{IB}+2\overrightarrow{IC}=\overrightarrow{0}$. \\
			Khi đó ta có $\overrightarrow{IA}+\overrightarrow{IB}+2\overrightarrow{IC}=\overrightarrow{0}\Leftrightarrow 2\overrightarrow{IE}+2\overrightarrow{IC}=\overrightarrow{0}\Leftrightarrow\overrightarrow{IE}+\overrightarrow{IC}=\overrightarrow{0}$ .\\
			Suy ra $I$ là trung điểm của $EC$.
			\begin{align*}
				  & \left|\overrightarrow{MA}+\overrightarrow{MB}+2\overrightarrow{MC}\right|                                                                          \\
				= & \left|\overrightarrow{MI}+\overrightarrow{IA}+\overrightarrow{MI}+\overrightarrow{IB}+2\left(\overrightarrow{MI}+\overrightarrow{IC}\right)\right| \\
				= & \left| 4\overrightarrow{MI}+\overrightarrow{IA}+\overrightarrow{IB}+2\overrightarrow{IC}\right|                                                    \\
				= & \left| 4\overrightarrow{MI}+\overrightarrow{0}\right|=\left| 4\overrightarrow{MI}\right|=4MI.
			\end{align*}
			Suy ra $MI=2$. Do vậy tập hợp điểm $M$ thoả yêu cầu bài toán là đường tròn tâm $I$ bán kính bằng $2$.

		}{
			\begin{tikzpicture}[scale=1, font=\footnotesize, line join=round, line cap=round, >=stealth]
				\path
				(0,0) coordinate (A)
				(0,3) coordinate (B)
				(4,0) coordinate (C)
				($(A)!1/2!(B)$) coordinate (E)
				($(C)!1/2!(E)$) coordinate (I)
				;
				\draw (A)--(B)--(C)--(A) (C)--(E);
				\draw pic[draw, angle radius=2mm]{right angle=B--A--C};
				\foreach \x/\g in {A/180,B/180,C/-90,E/180,I/-100}
				\fill[black](\x) circle (1.2pt) ($(\x)+(\g:3mm)$) node{$\x$};
			\end{tikzpicture}
		}
	}
\end{ex}

\begin{ex}%[HK1, THPT Lê Quảng Chí, 2023]%[TVN-001, 10-11EX-HK1-2324]%[0H9C2-6]
	Trong mặt phẳng $Oxy$, cho ba điểm $A(1;-4)$, $B(4;5)$, $C(0;-7)$. Điểm $M$ di chuyển trên trục $Ox$. Đặt $Q=2\left|\overrightarrow{MA}+2\overrightarrow{MB}\right|+3\left|\overrightarrow{MB}+\overrightarrow{MC}\right|$. Tìm giá trị nhỏ nhất của $Q$.
	\loigiai{
		\immini{Tọa độ trung điểm $J$ của đoạn $BC$ là $J(2;-1)$.
			Gọi $I$ là điểm xác định sao cho $\overrightarrow{IA}+2\overrightarrow{IB}=\overrightarrow{0}$. Ta có $I\left(3;2\right)$.
			Khi đó
			\begin{eqnarray*}
				Q&=&2|\overrightarrow{MI}+\overrightarrow{IA}+2\left(\overrightarrow{MI}+\overrightarrow{IB}\right)|+3|\overrightarrow{MJ}+\overrightarrow{JB}+\overrightarrow{MJ}+\overrightarrow{JC}|\\
				&=&6|\overrightarrow{MJ}|+6|\overrightarrow{MI}|=6\left(MI+MJ\right).
			\end{eqnarray*}
			Từ hình vẽ, ta thấy, điểm $I$, $J$ khác phía so với trục $Ox$. Do đó
			\[MI+MJ\geq IJ\quad \text{và } MI+MJ=IJ\Leftrightarrow M=IJ\cap Ox.\]
			Ta có $IJ=\sqrt{10}$ hay $Q\geq 6\sqrt{10}$. Gọi $M(m;0)$, do $\overrightarrow{MI}$, $\overrightarrow{MJ}$ cùng phương nên $m=\dfrac{7}{3}$.
		}{\begin{tikzpicture}[scale=0.6]
				%		 \pgfmathsetmacro{\r}{dfrac{7}{3}}
				\draw[->] (-1,0)--(5,0)node[below]{$x$};
				\draw[->] (0,-8)--(0,6)node[left]{$y$};
				\fill (0,0)node[below left]{ $O$};
				\foreach \x/\y in{1/-4,4/5,0/-7,2/-1,3/2,2.3333/0}\draw[fill=black](\x,\y)circle(1.25pt);
				\foreach \x/\y/\goc/\z in{1/-4/135/A,4/5/-70/B,0/-7/-40/C,2/-1/-45/J,3/2/30/I,2.333333/0/-40/M}
				\path(\x,\y)node[shift={(\goc:7pt)}]{\scriptsize$\z$};
				\draw (4,5)--(0,-7);
			\end{tikzpicture}
		}
		\noindent	Vậy giá trị nhỏ nhất của biểu thức $Q$ bằng $6\sqrt{10}$ khi $M\left(\dfrac{7}{3};0\right)$.\\
		\textbf{Nhận xét:} \\
		+ Có thể tìm tọa độ điểm $M=AB\cap Ox$, với $(AB)\colon y=3x-7$.\\
		+ Có thể áp dụng bất đẳng thức $|\overrightarrow {a}|+|\overrightarrow{b}|\geq |\overrightarrow{a}-\overrightarrow{b}|$.
	}
\end{ex}
\Closesolutionfile{ans}
