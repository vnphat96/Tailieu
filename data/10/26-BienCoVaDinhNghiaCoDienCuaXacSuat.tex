\setcounter{section}{4}
\setcounter{dang}{0}
\section*{BIẾN CỐ VÀ ĐỊNH NGHĨA CỔ ĐIỂN CỦA XÁC SUẤT}
\subsection{Tóm tắt lý thuyết}
\subsubsection{Biến cố}
\begin{itemize}
	\item {\bf{Phép thử ngẫu nhiên}} (gọi tắt là phép thử) là một thí nghiệm hay một hành động mà kết quả của nó không thể biết được trước khi phép thử được thực hiện.
	\item {\bf{Không gian mẫu}} của phép thử là tập hợp tất cả  các kết quả có thể xảy ra khi thực hiện phép thử. Không gian mẫu của phép thử được kí hiệu là $ \Omega $.
	\item {\bf{Kết quả thuận lợi}} Cho một biến cố $ \mathrm{E} $ liên quan tới phép thử $ \mathrm{T} $ là kết quả của phép thử $ \mathrm{T} $ làm cho biến cố  đó xảy ra.
\end{itemize}
\begin{note}
	Ta chỉ xét các phép thử mà không gian mẫu gồm hữu hạn kết quả.
\end{note} 
Mỗi biến cố là một tập con của không gian mẫu $ \Omega $. \\
Tập con này  là tập tất cả các kết quả thuận lợi cho biến cố đó.\\
\text{Nhận xét}: Biến cố chắc chắn là tập $ \Omega $, biến cố không thể là tập $ \varnothing $.	\\
{\bf{Biến cố đối}} của biến cố $ \mathrm{E} $ là biến cố \lq\lq  $ \mathrm{E}$ không xảy ra\rq\rq.\\
Biến cố đối của $ \mathrm{E} $ được kí hiệu là $ \mathrm{\overline{E}} $.
\subsubsection{Định nghĩa cổ điển của xác suất}
\begin{itemize}
	\item Các kết quả của phép thử $ \mathrm{T} $ gọi là đồng khả năng nếu chúng có khả năng xuất hiện như nhau.
	\item Giả sử các kết quả có thể của phép thử $ \mathrm{T} $  là đồng khả năng. Khi đó xác suất của biến cố $ \mathrm{E} $ bằng tỉ số giữa kết quả thuận lợi của $ \mathrm{E} $ và kết quả có thể.
\end{itemize}
\begin{dn}
		Cho phép thử $ \mathrm{T} $ có không gian mẫu là $ \Omega $. Giả thiết rằng các kết quả có thể của $ \mathrm{T} $ là đồng khả năng. Khi đó nếu $ \mathrm{E} $ là một biến cố liên quan đến phép thử $ \mathrm{T} $ thì {\bf{Xác suất}} của $ \mathrm{E} $ được cho bởi công thức $$\mathrm{P}(\mathrm{E})=\dfrac{n(\mathrm{E})}{n(\Omega)}.$$
	Trong đó $ n(\Omega) $ và $ n(\mathrm{E}) $ tương ứng là số phần tử của tập $ \Omega $ và tập $ \mathrm{E} $.
\end{dn}
\text{nhận xét}: 
\begin{itemize}
	\item Với mỗi biến cố $ \mathrm{E} $, ta có $ 0\leq \mathrm{P}(\mathrm{E})\leq 1 $.
	\item Với biến cố chắc chắn (là tập $ \Omega $), ta có $ \mathrm{P}(\Omega)=1 $.
	\item Với biến cố không thể (là tập $ \varnothing $), ta có $ \mathrm{P}(\varnothing)=0 $.
\end{itemize}
\begin{note}
	Trong những phép thử đơn giản, ta đếm số phần tử của tập $ \Omega $ và số phần tử của biến cố $ \mathrm{E} $ bằng cách liệt kê ra tất cả các phần tử của hai tập hợp này.
\end{note}
% \subsubsection{Nguyên lý xác suất bé}
% Nếu một biến cố có xác suất rất bé thì trong một phép thử biến cố đó sẽ không xảy ra.
\subsection{Các dạng toán}
\begin{dang}{Xác định phép thử, mô tả không gian mẫu}
	
\end{dang}
\viduminhhoa
\begin{vd}%[1D2B4-1]
	Một tổ trong lớp $ 10A $ có ba học sinh nữ là Hương, Hồng, Dung và bốn học sinh nam là Sơn, Tùng, Hoàng, Tiến. Giáo viên chọn ngẫu nhiên một học sinh trong tổ đó để kiểm tra vở bài tập. Phép thử ngẫu nhiên là gì? Mô tả không gian mẫu.
	\loigiai{
	Phép thử ngẫu nhiên là chọn ngẫu nhiên một học sinh trong tổ để kiểm tra vở  bài tập.\\
	Không gian mẫu là tập hợp tất cả các học sinh trong tổ.\\
	$ \Omega=\{\text{Hương}; \text{Hồng}; \text{Dung}; \text{Sơn}; \text{Tùng}; \text{Hoàng}; \text{Tiến}\} $.
		}
\end{vd}
\begin{vd}%[1D2B4-1]
	Khi tham gia một trò chơi bốc thăm trúng thưởng, mỗi người chơi chọn một bộ $ 6 $ số đôi một khác nhau từ $ 45 $ số: $ 1 $; $ 2 $; \ldots ; 45, chẳng hạn bạn An chọn $ \{5; 13; 20; 31; 32; 35\} $.\\
	Sau đó, người quản trò  bốc ngẫu nhiên $ 6 $ quả bóng (không hoàn lại) từ một thùng kín đựng $ 45 $ quả bóng như nhau ghi các số $ 1 $; $ 2 $; $\ldots$ ; $ 45 $. Bộ $ 6 $ số ghi trên $ 6 $ quả bóng đó được gọi là bộ số trúng thưởng.\\
	Nếu bộ số người chơi trùng với bộ số trúng thưởng thì người chơi trúng giải độc đắc; nếu trùng với $ 5 $ số của bộ số trúng thưởng thì người chơi trúng giải nhất.
	\begin{enumerate}
		\item Phép thử là gì? Mô tả không gian mẫu $ \Omega $.
		\item Gọi $ \mathrm{F} $ là biến cố: \lq\lq  Bạn An trúng giải độc đắc\rq\rq. Hỏi $ \mathrm{F} $ là tập con nào của không gian mẫu?
		 \item Gọi $ \mathrm{G} $ là biến cố: \lq\lq  Bạn An trúng giải nhất\rq\rq. Hãy chỉ ra ba phần tử của tập $ \mathrm{G} $. Từ đó, hãy mô tả tập hợp $ \mathrm{G} $ bằng cách chỉ ra tính chất đặc trưng cho các phần tử của $ \mathrm{G} $.
	\end{enumerate}
\loigiai{
	\begin{enumerate}
		\item Phép thử là chọn ngẫu nhiên $ 6 $ số trong $ 45 $ số: $ 1 $ ; $ 2 $; \ldots ; $ 45 $. Không gian mẫu $ \Omega $ là tập hợp tất cả các tập con có $ 6 $ phần tử của tập $ \{1; 2; \ldots ;44; 45\} $.
		\item $ \mathrm{F}=\{5; 13; 20; 31; 32; 35\} $.
		\item Ba phần tử thuộc $ \mathrm{G} $ chẳng hạn là  $$ \{6; 13; 20; 31; 32; 35\};  \{5; 7; 20; 31; 32; 35\};  \{5; 13; 8; 31; 32; 35\}.$$
		$ \mathrm{G} $  là tập hợp tất cả các tập con gồm sáu phần tử của tập $ \{1; 2; \ldots ;44; 45\} $ có tính chất 
		\begin{itemize}
			\item Năm phần tử của nó thuộc tập $ \{5; 13; 20; 31; 32; 35\} $;
			\item Một phần tử còn lại không thuộc tập $ \{5; 13; 20; 31; 32; 35\} $.
		\end{itemize}
	\end{enumerate}
	}
\end{vd}

\begin{vd}%[1D2B4-1]
	Phần thưởng trong một chương trình khuyến mãi của một siêu thị là tivi, bàn ghế, tủ lạnh, máy tính, bếp từ bộ bát đĩa. Ông Dũng tham gia chương trình được chọn ngẫu nhiên một mặt hàng.
	\begin{enumerate}
		\item Mô tả không gian mẫu,
		\item Gọi $ \mathrm{D} $ là biến cố \lq\lq  Ông Dũng được chọn mặt hàng là đồ điện\rq\rq. Hỏi $ \mathrm{D} $ là tập con nào của không gian mẫu?
	\end{enumerate}
	\loigiai{
	\begin{enumerate}
		\item $ \Omega=\{\text{ti vi}; \text{bàn ghế}; \text{tủ lạnh}; \text{máy tính}; \text{bếp từ}; \text{bộ bát đĩa}\} $.
		\item $\mathrm{D}=\{\text{ti vi};  \text{tủ lạnh}; \text{máy tính}; \text{bếp từ}\} $.
	\end{enumerate}
	}
\end{vd}

\begin{vd}%[1D2B4-1]
	Gieo một con xúc xắc $ 6 $ mặt và quan sát số chấm xuất hiện trên con xúc xắc.
	\begin{enumerate}
		\item Mô tả không gian mẫu.
		\item Gọi $ \mathrm{M} $ là biến cố \lq\lq  Số chấm xuất hiện trên con xúc xắc là một số chẵn\rq\rq. Nội dung biến cố đối $\mathrm{\overline{M}}$ của $ \mathrm{M} $ là gì?
		\item Biến cố $ \mathrm{M} $ và $\mathrm{\overline{M}}$ là tập con nào của không gian mẫu?
	\end{enumerate}
\loigiai{
	\begin{enumerate}
		\item Không gian mẫu $ \Omega=\{1; 2; 3; 4; 5; 6\} $.
		\item Biến cố đối $\mathrm{\overline{M}}$ của $ \mathrm{M} $ là biến cố \lq\lq  Số chấm xuất hiện trên con xúc xắc là một số lẻ\rq\rq.
		\item Ta có $ \mathrm{M}=\{2; 4; 6\}\subset \Omega $; $\mathrm{\overline{M}}=\{1; 3; 5\}\subset \Omega$.
	\end{enumerate}
	}
\end{vd}

\begin{vd}%[1D2B4-1]
	Gieo một con xúc xắc. Gọi $ \mathrm{K} $ là biến cố \lq\lq  Số chấm xuất hiện trên con xúc xắc là một số nguyên tố\rq\rq.
	\begin{enumerate}
		\item Biến cố \lq\lq  Số chấm xuất hiện trên con xúc xắc là một hợp số\rq\rq\, có là biến cố $\mathrm{\overline{K}}$ không?
		\item Biến cố $ \mathrm{K} $ và $\mathrm{\overline{K}}$ là tập con nào của không gian mẫu?
	\end{enumerate}
	\loigiai{
		\begin{enumerate}
			\item Biến cố \lq\lq  Số chấm xuất hiện trên con xúc xắc là một hợp số\rq\rq\, không phải là biến cố $\mathrm{\overline{K}}$. \\ $\mathrm{\overline{K}}$ là biến cố \lq\lq  Số chấm xuất hiện trên con xúc xắc là $ 1 $ hoặc là một hợp số\rq\rq.
			\item Ta có $ \mathrm{K}=\{2; 3; 5\}\subset \Omega $; $\mathrm{\overline{K}}=\{1; 4; 6\}\subset \Omega$.
		\end{enumerate}
	}
\end{vd}
%%CTST
\begin{vd}%[1D2B4-1]
	Một đồng xu có hai mặt, trên một mặt có ghi giá trị của đồng xu, thường gọi là mật sấp, mặt kia là mặt ngửa. Hãy xác định không gian mẫu của mỗi phép thử ngẫu nhiên sau
	\begin{enumerate}
		\item Tung đồng xu một lần.
		\item Tung đồng xu hai lần.
	\end{enumerate}
\loigiai{
	\begin{enumerate}
		\item Khi tung đồng xu một lần, ta có không gian mẫu là $ \Omega=\{\mathrm{S}; \mathrm{N}\} $, trong đó kí hiệu $ \mathrm{S} $ để chỉ đồng xu xuất hiện mặt sấp và $ \mathrm{N} $ để chỉ đồng xu xuất hiện mặt ngửa.
		\item Khi tung đồng xu hai lần, ta có không gian mẫu là $ \Omega=\{\mathrm{S}\mathrm{S}; \mathrm{S} \mathrm{N}; \mathrm{N}\mathrm{S}; \mathrm{N}\mathrm{N}\} $.
	\end{enumerate}
	}
\end{vd}

\begin{vd}%[Tex hóa SGK-CT,T8/22, Mui Doan]%[1D2B4-1]
	Trong hộp có bốn quả bóng được đánh số từ $ 1 $ đến $ 4 $. Hãy xác định không gian mẫu của phép thử sau
	\begin{enumerate}
		\item Lấy ngẫu nhiên một quả bóng;
		\item Lấy ngẫu nhiên cùng một lúc hai quả bóng;
		\item Lấy ngẫu nhiên lần lượt hai quả bóng.
	\end{enumerate}
	\loigiai{
		\begin{enumerate}
			\item Không gian mẫu là $ \Omega=\{1; 2; 3; 4\} $.
			\item Do mỗi lần ta lấy hai quả bóng mà không tính đến thứ tự nên không gian mẫu sẽ gồm các tập con gồm hai phần tử của tập hợp $\{1; 2; 3; 4\} $, tức là $$\Omega=\left\{\{1; 2\}; \{1; 3\}; \{1; 4\}; \{2; 3\}; \{2; 4\}; \{3; 4\}\right\}.$$
			\item Do hai quả bóng được lấy lần lượt nên ta cần phải tính đến thứ tự lấy bóng. Nếu lần đầu lấy được bóng số $ 3 $, lần sau lấy được bóng số $ 1 $ thì ta sẽ kí hiệu kết quả của phép thử là cặp $ (3; 1) $. Khi đó không gian mẫu của phép thử là $$\Omega=\left\{(1; 2); (2; 1); (1; 3); (3; 1); (1; 4); (4; 1); (2; 3); (3; 2); (2; 4); (4; 2); (3; 4); (4; 3)\right\}.$$
		\end{enumerate}
	}
\end{vd}

\begin{vd}%[Tex hóa SGK-CT,T8/22, Mui Doan]%[1D2B4-1]
	Trong hộp có bốn quả bóng được đánh số từ $ 1 $ đến $ 4 $. Lấy ngẫu nhiên một quả bóng từ hộp, xem số, sau đó trả lại hộp, trộn đều rồi lấy lại ngẫu nhiên một quả bóng từ hộp đó. Hãy xác định không gian mẫu của phép thử hai lần lấy bóng này.
	\loigiai{
		Không gian mẫu của phép thử là 
		\begin{align*}
		\Omega=\{&(1;1); (1; 2);  (1; 3);  (1; 4); (2; 1); (2; 2); (2; 3); (2; 4);\\
		 &(3; 1); (3; 2); (3; 3); (3;4); (4;1); (4;2); (4;3); (4;4)\}.	
		\end{align*}
		Ta cũng có thể viết không gian mẫu dưới dạng $$\Omega=\left\{(i; j)\,|\,i,j=1, 2, 3, 4\right\}.$$
	}
\end{vd}

\begin{vd}%[Tex hóa SGK-CT,T8/22, Mui Doan]%[1D2B4-1]
	Xét phép thử gieo hai con xúc xắc.
	\begin{enumerate}
		\item Hãy xác định không gian mẫu của phép thử.
		\item Viết tập hợp mô tả biến cố \lq\lq  Tổng số chấm xuất hiện trên hai con xúc xắc bằng $ 4 $\rq\rq. Có bao nhiêu kết quả thuận lợi cho biến cố đó?
	\end{enumerate}
	\loigiai{
		\begin{enumerate}
			\item Kết quả của phép thử là một cặp số $ (i; j) $, trong đó $ i $ và $ j $ lần lượt là số chấm xuất hiện trên con xúc xắc thứ nhất và thứ hai.\\
			Không gian mẫu của phép thử là
			\begin{align*}
				\Omega=\{&(1; 1); (1; 2); (1; 3); (1; 4); (1; 5); (1; 6);\\
				&(2; 1); (2; 2); (2; 3); (2; 4); (2; 5); (2; 6);\\
				& (3; 1); (3; 2); (3; 3); (3; 4); (3; 5); (3; 6);\\
				& (4; 1); (4; 2); (4; 3); (4; 4); (4; 5); (4; 6);\\
				& (5; 1); (5; 2); (5; 3); (5; 4); (5; 5); (5; 6);\\
				& (6; 1); (6; 2); (6; 3); (6; 4); (6; 5); (6; 6)\}.
			\end{align*}
			Ta cũng có thể viết không gian mẫu dưới dạng $ \Omega=\{(i;j)\,|\, i, j=1, 2, \ldots, 6\} $.
		\item Gọi $ \mathrm{A} $ là biến cố \lq\lq  Tổng số chấm xuất hiện bằng $ 4 $\rq\rq. Tập hợp mô tả biến cố $ \mathrm{A} $ là $$\mathrm{A}=\{(1; 3); (2; 2); (3; 1)\}.$$
		Như vậy ta có ba kết quả thuận lợi cho biến cố $ \mathrm{A} $.
	\end{enumerate}
	}
\end{vd}

\begin{vd}%[Tex hóa SGK-CT,T8/22, Mui Doan]%[1D2B4-1]
	Trong phép thử gieo hai con xúc xắc, gọi $ \mathrm{B} $ là biến cố \lq\lq  Xuất hiện hai mặt có cùng số chấm\rq\rq\, và $ \mathrm{C} $ là biến cố \lq\lq  Số chấm xuất hiện ở con xúc xắc thứ nhất gấp $ 2 $ lần số chấm xuất hiện ở con xúc xắc thứ hai\rq\rq.
	\begin{enumerate}
		\item Hãy xác định biến cố $ \mathrm{B} $ và $ \mathrm{C} $ bằng cách liệt kê các phần tử.
		\item Có bao nhiêu kết quả thuận lợi cho $ \mathrm{B} $ và bao nhiêu kết quả thuận lợi cho $ \mathrm{C} $?
	\end{enumerate} 
	\loigiai{
		\begin{enumerate}
			\item $\mathrm{B}=\{(1; 1); (2; 2); (3; 3); (4;4); (5;5); (6;6)\}$; 
			$\mathrm{C}=\{(2; 1); (4; 2); (6; 3)\}$.
			\item Có $ 6 $ kết quả thuận lợi cho $ \mathrm{B} $ và $ 3 $ kết quả thuận lợi cho $ \mathrm{C} $. 
		\end{enumerate}
	}
\end{vd}

\begin{vd}%[Tex hóa SGK-CT,T8/22, Mui Doan]%[1D2B4-1]
	Một nhóm có $ 5 $ bạn nam và $ 4 $ bạn nữ. Chọn ngẫu nhiên cùng một lúc ra $ 3 $ bạn đi làm công tác tình nguyện.
	\begin{enumerate}
		\item Hãy xác định số phần tử của không gian mẫu.
		\item Hãy xác định số các kết quả thuận lợi cho biến cố \lq\lq  Trong $ 3 $ bạn được chọn có đúng $ 2 $ bạn nữ\rq\rq. 
	\end{enumerate}
	\loigiai{
		\begin{enumerate}
			\item Do ta chọn ra $ 3 $ bạn khác nhau từ $ 9 $ bạn trong nhóm và không tính đến thứ tự nên số phần tử của không gian mẫu là $ \mathrm{C}_9^3=84 $.
			\item Ta có $ \mathrm{C}_4^2 $ cách chọn ra $ 2 $ bạn nữ từ $ 4 $ bạn nữ. Ứng với mỗi cách chọn $ 2 $ bạn nữ có $ \mathrm{C}_5^1 $ cách chọn ra $ 1 $	 bạn nam từ $ 5 $ bạn nam.\\
			Theo quy tắc nhân ta có tất cả  $ \mathrm{C}_4^2\cdot \mathrm{C}_5^1 $ cách chọn ra $ 2 $ bạn nữ và $ 1 $ bạn nam từ nhóm bạn.\\
			Do đó số các kết quả thuận lợi cho biến cố \lq\lq  Trong $ 3 $ bạn được chọn có đúng $ 2 $ bạn nữ\rq\rq\, là $ \mathrm{C}_4^2\cdot \mathrm{C}_5^1 =30$.
		\end{enumerate}
	}
\end{vd}

\begin{vd}%[Tex hóa SGK-CT,T8/22, Mui Doan]%[1D2B4-1]
	Một nhóm có $ 5 $ bạn nam và $ 4 $ bạn nữ. Chọn ngẫu nhiên cùng một lúc ra $ 3 $ bạn đi làm công tác tình nguyện. Hãy xác định số các kết quả thuận lợi cho biến cố
	\begin{enumerate}
		\item \lq\lq  Trong $ 3 $ bạn được chọn có đúng một bạn nữ\rq\rq;
		\item \lq\lq  Trong $ 3 $ bạn được chọn không có bạn nam nào\rq\rq.
	\end{enumerate}
	\loigiai{
		\begin{enumerate}
			\item Ta có $ \mathrm{C}_4^1 $ cách chọn ra $ 1 $ bạn nữ từ $ 4 $ bạn nữ. Ứng với mỗi cách chọn $ 1 $ bạn nữ có $ \mathrm{C}_5^2 $ cách chọn ra $ 2 $	 bạn nam từ $ 5 $ bạn nam.\\
			Theo quy tắc nhân ta có tất cả  $ \mathrm{C}_4^1\cdot \mathrm{C}_5^2 $ cách chọn ra $ 1 $ bạn nữ và $ 2 $ bạn nam từ nhóm bạn.\\
			Do đó số các kết quả thuận lợi cho biến cố \lq\lq  Trong $ 3 $ bạn được chọn có đúng $ 1 $ bạn nữ\rq\rq\, là $ \mathrm{C}_4^1\cdot \mathrm{C}_5^2 =40$.
			\item Ta có $ \mathrm{C}_4^3$ cách chọn ra $ 3 $ bạn không có bạn nam nào.\\
			Do đó số các kết quả thuận lợi cho biến cố \lq\lq  Trong $ 3 $ bạn được chọn không có bạn nam nào\rq\rq\, là $ \mathrm{C}_4^3=4$.
		\end{enumerate}
	}
\end{vd}
\baitaptl
\begin{bt}%[Dự án Bài giảng Toán 10, đợt 2]%[Đào-V- Thủy]%[1D2B4-1]
	Gieo một con xúc xắc liên tiếp hai lần
	\begin{enumerate}
		\item Mô tả không gian mẫu.
		\item Gọi $A$ là biến cố: \lq\lq  Tổng số chấm xuất hiện lớn hơn hay bằng $8$\rq\rq. Biến cố $A$ và $\overline{A}$ là các tập con nào của không gian mẫu.
	\end{enumerate}
	\loigiai
	{
		\begin{enumerate}
			\item $\Omega= \{(a, b) \text{ với } 1\le a \le 6 \text{ và } 1\le b\le 6\}$, trong đó $a$, $b$ tương ứng là số chấm xuất hiện ở lần gieo thứ nhất và thứ hai.
			\item $A= \{(2,6); (3,5), (3,6); (4,4); (4,5); (4,6); (5,3); (5,4); (5,5); (5,6); (6,2); (6,3); (6,4); (6,5); (6,6)\}$.\\
			$\overline{A}= \{(1,1); (1,2); (1,3); (1,4); (1,5); (1,6); (2,1); (2,2); (2,3); (2,4); (2,5); (3,1); (3,2); (3,3); (3,4);\\
			\hspace*{1cm} (4,1); (4,2); (4,3); (5,1); (5,2); (6,1)\}$.
		\end{enumerate}
	}
\end{bt}

\begin{bt}%[Dự án Bài giảng Toán 10, đợt 2]%[Đào-V- Thủy]%[1D2B4-1]
	Gieo một con xúc xắc đồng thời rút ngẫu nhiên một thẻ từ một hộp chứa $4$ thẻ $A$, $B$, $C$, $D$.
	\begin{enumerate}
		\item Mô tả không gian mẫu.
		\item Xét các biến cố sau:
		\begin{description}
			\item $E$: \lq\lq  Con xúc xắc xuất hiện mặt $6$\rq\rq.
			\item $F$: \lq\lq  Rút được thẻ $A$ hoặc con xúc xắc xuất hiện mặt $5$\rq\rq.
		\end{description}		
		Các biến cố $E$, $\overline{E}$, $F$ và $\overline{F}$ là các tập con nào của không gian mẫu?
	\end{enumerate}
	\loigiai
	{
		\begin{enumerate}
			\item $\Omega= \{(1,A); (2,A); (3,A); (4,A); (5,A); (6,A);\\
			\hspace*{1cm} (1,B); (2,B); (3,B); (4,B); (5,B); (6,B);\\
			\hspace*{1cm} (1,C); (2,C); (3,C); (4,C); (5,C); (6,C);\\
			\hspace*{1cm} (1,D); (2,D); (3,D); (4,D); (5,D); (6,D)\}$.
			\item $E=\{(6,A); (6,B); (6,C); (6,D)\}$.\\
			$\overline{E}= \{(1,A); (2,A); (3,A); (4,A); (5,A);\\
			\hspace*{1cm} (1,B); (2,B); (3,B); (4,B); (5,B);\\
			\hspace*{1cm} (1,C); (2,C); (3,C); (4,C); (5,C);\\
			\hspace*{1cm} (1,D); (2,D); (3,D); (4,D); (5,D)\}$.\\
			$F=\{(5,A); (5,B); (5,C); (5,D); (1,A); (2,A); (3,A); (4,A); (6,A)\}$.\\
			$\overline{F}= \{(1,B); (2,B); (3,B); (4,B); (6,B);\\
			\hspace*{1cm} (1,C); (2,C); (3,C); (4,C); (6,C);\\
			\hspace*{1cm} (1,D); (2,D); (3,D); (4,D); (6,D)\}$.
		\end{enumerate}
	}
\end{bt}

\begin{bt}%[Dự án Bài giảng Toán 10, đợt 2]%[Đào-V- Thủy]%[1D2B4-1]
	Tung một đồng xu ba lần liên tiếp. Phát biểu mỗi biến cố sau dưới dạng mệnh đề:
	\begin{enumerate}
		\item $A=\{SSS; NSS; SNS; NNS\}$.
		\item $B=\{SSN, SNS, NSS\}$.
	\end{enumerate}
	\loigiai
	{
		\begin{enumerate}
			\item $A$: \lq\lq  Lần gieo thứ ba xuất hiện mặt sấp\rq\rq.
			\item $B$: \lq\lq  Mặt sấp xuất hiện hai lần\rq\rq.
		\end{enumerate}
	}
\end{bt}

\begin{bt}%[Dự án Bài giảng Toán 10, đợt 2]%[Đào-V- Thủy]%[1D2B4-1]
	Hai túi $I$ và $II$ chứa các tấm thẻ được đánh số. Túi $I \colon \{1; 2; 3; 4\}$, túi $II \colon \{1; 2; 3; 4; 5\}$. Rút ngẫu nhiên từ mỗi túi $I$ và $II$ một tấm thẻ.
	\begin{enumerate}
		\item Mô tả không gian mẫu.
		\item Xét các biến cố sau:
		\begin{description}
			\item $A$: \lq\lq  Hai số trên hai tấm thẻ bằng nhau\rq\rq.
			\item $B$: \lq\lq  Hai số trên hai tấm thẻ chênh nhau $2$\rq\rq.
			\item $C$: \lq\lq  Hai số trên hai tấm thẻ chênh nhau lớn hơn hay bằng $2$\rq\rq.
		\end{description}
		Các biến cố $A$, $\overline{A}$, $B$, $\overline{B}$, $C$, $\overline{C}$ là các tập con nào của không gian mẫu?
	\end{enumerate}
	\loigiai
	{
		\begin{enumerate}
			\item $\Omega= \{(a,b) \text{ với } 1\le a\le 4 \text{ và } 1\le b\le 5\}$, trong đó $a$, $b$ lần lượt là số ghi trên thẻ được rút từ túi $I$ và túi $II$.
			\item $A=\{(1,1); (2,2); (3,3); (4,4)\}$.\\
			$\overline{A}= \{(1,2); (1,3); (1,4); (1,5); (2,1); (2,3); (2,4); (2,5);\\
			\hspace*{1cm} (3,1); (3,2); (3,4); (3,5); (4,1); (4,2); (4,3); (4,5)\}$.\\
			$B= \{(1,3); (3,1); (2,4); (4,2); (3,5); (5,3)\}$.\\
			$\overline{B}=\{(1,1); (1,2); (1,4); (1,5); (2,1); (2,2); (2,3); (2,5);\\
			\hspace*{1cm} (3,2); (3,3); (3,4); (3,5); (4,1); (4,3); (4,4); (4,5)\}$.\\
			$C=\{(1,3); (1,4); (1,5); (2,4); (2,5); (3,1); (3,5); (4,1); (4,2)\}$.\\
			$\overline{C}=\{(1,1); (1,2); (2,1); (2,2); (2,3); (3,2); (3,3); (3,4); (4,3); (4,4); (4,5)\}$.
		\end{enumerate}
	}
\end{bt}

% \begin{bt}%[Dự án Bài giảng Toán 10, đợt 2]%[Đào-V- Thủy]%[1D2B4-1]
% 	Trường mới của bạn Dũng có $3$ câu lạc bộ ngoại ngữ là câu lạc bộ tiếng Anh, câu lạc bộ tiếng Bồ Đào Nha và câu lạc bộ tiếng Campuchia.
% 	\begin{enumerate}
% 		\item Dũng chọn ngẫu nhiên $1$ câu lạc bộ ngoại ngữ để tìm hiểu thông tin. Hãy mô tả không gian mẫu của phép thử trên.
% 		\item Dũng thử chọn ngẫu nhiên $1$ câu lạc bộ ngoại ngữ để tham gia trong học kì $1$ và $1$ câu lạc bộ ngoại ngữ khác để tham gia trong học kì $2$. Hãy mô tả không gian mẫu của phép thử nêu trên.
% 	\end{enumerate}
% 	\loigiai
% 	{
% 		Kí hiệu $A$, $B$, $C$ lần lượt là kết quả Dũng chọn câu lạc bộ tiếng Anh, câu lạc bộ tiếng Bồ Đào Nha và câu lạc bộ tiếng Campuchia. Khi đó
% 		\begin{enumerate}
% 			\item $\Omega = \{A; B; C\}$.
% 			\item $\Omega= \{AB; AC; BA; BC; CA; CB\}$.
% 		\end{enumerate}
% 	}
% \end{bt}

\begin{bt}%[Dự án Bài giảng Toán 10, đợt 2]%[Đào-V- Thủy]%[1D2B4-1]
	Một bình chứa $10$ quả bóng được đánh số từ $1$ đến $10$. Tùng và Cúc mỗi người lấy ra ngẫu nhiên $1$ quả bóng từ bình.
	\begin{enumerate}
		\item Mô tả không gian mẫu của phép thử.
		\item Có bao nhiêu kết quả thuận lợi cho biến cố \lq\lq  Tổng hai số ghi trên hai quả bóng lấy ra bằng $10$\rq\rq.
		\item Có bao nhiêu kết quả thuận lợi cho biến cố \lq\lq  Tích hai số ghi trên hai quả bóng lấy ra chia hết cho $3$\rq\rq.
	\end{enumerate}
	\loigiai
	{
		\begin{enumerate}
			\item Không gian mẫu $\Omega= \{(i, j) \text{ với } 1\le i\le 10, 1\le j\le 10, i\neq j\}$, trong đó $(i, j)$ kí hiệu kết quả Tùng chọn được quả bóng ghi số $i$, Cúc chọn được quả bóng ghi số $j$.
			\item Gọi $A$ là biến cố \lq\lq  Tổng hai số ghi trên hai quả bóng lấy ra bằng $10$\rq\rq. Ta có
			\[
			A=\{(1,9); (2,8); (3,7); (4,6); (6,4); (7,3); (8,2); (9,1)\}.
			\]
			Số kết quả thuận lợi cho biến cố $A$ là $n(A)=8$.
			\item Gọi $B$ là biến cố \lq\lq  Tích hai số ghi trên hai quả bóng lấy ra chia hết cho $3$\rq\rq.\\
			Ta thấy, có $7$ số không chia hết cho $3$ là $1$, $2$, $4$, $5$, $7$, $8$, $10$.\\
			Do đó số kết quả thuận lợi cho biến cố \lq\lq  Tích hai số ghi trên hai quả bóng lấy ra không chia hết cho $3$\rq\rq\ là $7\cdot 6=42$.\\
			Tổng số kết quả có thể xảy ra là $10\cdot 9=90$.\\
			Vậy số kết quả thuận lợi cho biến cố $B$ là $n(B)= 90-42=48$.
		\end{enumerate}
	}
\end{bt}

\begin{bt}%[Dự án Bài giảng Toán 10, đợt 2]%[Đào-V- Thủy]%[1D2B4-1]
	Có $3$ khách hàng nam và $4$ khách hàng nữ cùng đến một quầy giao dịch. Quầy giao dịch sẽ chọn ngẫu nhiên lần lượt từng khách hàng để phục vụ. Tính số các kết quả thuận lợi cho biến cố:
	\begin{enumerate}
		\item \lq\lq  Các khách hàng nam và nữ được phục vụ xen kẽ nhau\rq\rq.
		\item \lq\lq  Người được phục vụ đầu tiên là khách hàng nữ\rq\rq.
		\item \lq\lq  Người được phục vụ cuối cùng là khách hàng nam\rq\rq.
	\end{enumerate}
	\loigiai
	{
		Kí hiệu thứ tự phục vụ các khách hàng được đánh số theo thứ tự từ $1$ đến $7$.
		\begin{enumerate}
			\item Gọi $A$ là biến cố \lq\lq  Các khách hàng nam và nữ được phục vụ xen kẽ nhau\rq\rq. Khi đó
			\begin{description}
				\item Các khách hàng nữ cần được phục vụ tại các vị trí $1$, $3$, $5$, $7$, có $4!$ cách.
				\item Các khách hàng nam cần đươc phục vụ tại các vị trí $2$, $4$, $6$, có $3!$ cách.
			\end{description}
			Số kết quả thuận lợi cho biến cố $A$ là $n(A)= 4! \cdot 3!= 144$.
			\item Gọi $B$ là biến cố \lq\lq  Người được phục vụ đầu tiên là khách hàng nữ\rq\rq.
			\begin{description}
				\item Chọn người phục vụ đầu tiên là khách hàng nữ, có $4$ cách.
				\item Phục vụ $6$ người khách còn lại có $6!$ cách.
			\end{description}		
			Số kết quả thuận lợi cho biến cố $B$ là $n(B)= 4\cdot 6!= 2880$.
			\item Gọi $B$ là biến cố \lq\lq  Người được phục vụ đầu tiên là khách hàng nam\rq\rq.
			\begin{description}
				\item Chọn người phục vụ đầu tiên là khách hàng nam, có $3$ cách.
				\item Phục vụ $6$ người khách còn lại có $6!$ cách.
			\end{description}		
			Số kết quả thuận lợi cho biến cố $C$ là $n(C)= 3\cdot 6!= 2160$.
		\end{enumerate}
	}
\end{bt}
% \begin{dang}{Tính xác suất của biến cố bằng định nghĩa cổ điển}
	
% \end{dang}
% \viduminhhoa
% \begin{vd}%[1D2B4-1]
% 	Một hộp chứa $ 12 $ tấm thẻ được đánh số  $ 1 $; $ 2 $; $ 3 $; $ 4 $; $ 5 $; $ 6 $; $ 7 $; $ 8 $; $ 9 $; $ 10 $; $ 11 $; $ 12 $. Rút ngẫu nhiên từ hộp đó một tấm thẻ.
% 	\begin{enumerate}
% 		\item Mô tả không gian mẫu $ \Omega $. Các kết quả có thể có đồng khả năng không?
% 		\item Xét biến cố $ \mathrm{E} $: \lq\lq  Rút được thẻ ghi số nguyên tố\rq\rq. Biến cố $ \mathrm{E} $ là tập con   nào của không gian mẫu?
% 		\item Phép thử có bao nhiêu kết quả có thể? Biến cố $ \mathrm{E} $ có bao nhiêu kết quả thuận lợi? Từ đó, hãy tính xác suất của biến cố  $ \mathrm{E} $.
% 	\end{enumerate}
% \loigiai{
% 	\begin{enumerate}
% 		\item $ \Omega=\{1; 2;\ldots ; 12\} $. Vì rút ngẫu nhiên nên các kết quả có thể là đồng khả năng.
% 		\item $ \mathrm{E}=\{2; 3; 5; 7; 11\} $.
% 		\item Phép thử có $ 12 $ kết quả có thể. Biến cố $ \mathrm{E} $ có $ 5 $ kết quả thuận lợi. Từ đó, $ \mathrm{P}(\mathrm{E})=\dfrac{n(\mathrm{E})}{n(\Omega)}=\dfrac{5}{12} $.
% 	\end{enumerate}
% 	}
% \end{vd}

% \begin{vd}%[1D2B4-1]
% 	Gieo một đồng xu cân đối liên tiếp ba lần. Gọi $ \mathrm{E} $ là biến cố: \lq\lq  Có hai lần xuất hiện mặt sấp và một lần xuất hiện mặt ngửa\rq\rq. Tính xác suất của biến cố $ \mathrm{E} $.
% 	\loigiai{
% 	Kí hiệu $ \mathrm{S} $ và $ \mathrm{N} $ tương ứng là đồng xu ra mặt sấp và đồng xu ra mặt ngửa.\\
% 	Không gian mẫu $ \Omega=\{ \mathrm{S} \mathrm{S} \mathrm{N};  \mathrm{S} \mathrm{N} \mathrm{S};  \mathrm{S} \mathrm{N} \mathrm{N};  \mathrm{S} \mathrm{S} \mathrm{S};  \mathrm{N} \mathrm{S} \mathrm{N}; \mathrm{N} \mathrm{N} \mathrm{S};  \mathrm{N} \mathrm{N} \mathrm{N};  \mathrm{N} \mathrm{S} \mathrm{S}\} $.\\
% 	$\mathrm{E}=\{ \mathrm{S} \mathrm{S} \mathrm{N};  \mathrm{S} \mathrm{N} \mathrm{S};  \mathrm{N} \mathrm{S} \mathrm{S}$.\\
% 	Ta có $ n(\Omega)=8 $; $ n(\mathrm{E})=3 $. Do đồng xu cân đối nên các kết quả  có thể là đồng khả năng.\\
% 	Vậy $ \mathrm{P}(\mathrm{E})=\dfrac{n(\mathrm{E})}{n(\Omega)}=\dfrac{3}{8} $.
% 	}
% \end{vd}

% \begin{vd}%[1D2B4-1]
% 	Hai túi $ \mathrm{I} $ và $ \mathrm{II} $ chứa các tấm thẻ được đánh số.  Túi $ \mathrm{I}: \{1; 2; 3; 4; 5\} $, túi $ \mathrm{II}: \{1; 2; 3; 4\} $. Rút ngẫu nhiên một tấm thẻ từ mỗi túi $ \mathrm{I} $ và $ \mathrm{II} $. Tính xác suất để tổng hai số trên hai tấp thẻ lớn hơn $ 6 $.
% 	\loigiai{
% 	Mô tả không gian mẫu $ \Omega $ bằng cách lập bảng như sau
% 	\begin{center}
% 	\begin{tabular}{|l|c|c|c|c|}%
% 		\hline
% 		\diaghead(-4,1){\hskip4.2cm}%^^A
% 		{Túi  I}{Túi  II}&
% 		\thead{$ 1 $}&\thead{$ 2 $}&\thead{$ 3 $}&\thead{$ 4 $}\\
% 		\hline
% 		\makecell[b]{$ 1 $} & $ (1,1) $ & $ (1,2) $&$ (1,3) $&$ (1,4) $\\
% 		\hline
% 		\makecell[b]{$ 2 $} & $ (2,1) $ & $ (2,2) $&$ (2,3) $&$ (2,4) $\\
% 		\hline
% 		\makecell[b]{$ 3 $} & $ (3,1) $ & $ (3,2) $&$ (3,3) $&$ (3,4) $\\
% 		\hline
% 		\makecell[b]{$ 4 $} & $ (4,1) $ & $ (4,2) $&$ (4,3) $&$ (4,4) $\\
% 		\hline
% 		\makecell[b]{$ 5 $} & $ (5,1) $ & $ (5,2) $&$ (5,3) $&$ (5,4) $\\
% 		\hline
% 	\end{tabular}%
% 	\end{center}
% 	Mỗi ô là một kết quả có thể. Có $ 20 $ ô, vậy $ n(\Omega)=20 $.\\
% 	Biến cố $ \mathrm{E} $: \lq\lq  Tổng hai số trên hai tấp thẻ lớn hơn $ 6 $\rq\rq\, xảy ra khi tổng là một trong ba trường hợp 
% 	\begin{itemize}
% 		\item Tổng bằng $ 7 $ gồm các kết quả: $ (3;4); (4;3); (5;2) $. 
% 		\item Tổng bằng $ 8 $ gồm các kết quả: $ (4;4); (5;3)$. 
% 		\item Tổng bằng $ 9 $ có một kết quả: $ (5;4)$. 
% 	\end{itemize}
% 	Vậy biến cố $ \mathrm{E}=\{ (3;4); (4;3); (5;2); (4;4); (5;3); (5;4)\} $. \\Từ đó $ n( \mathrm{E})=6 $ và $\mathrm{P}(\mathrm{E})=\dfrac{n(\mathrm{E})}{n(\Omega)}=\dfrac{6}{20}=\dfrac{3}{10} =0{,}3$.
% 	}
% \end{vd}

% \begin{vd}%[1D2B4-1]
% 	Gieo đồng thời hai con xúc xắc cân đối. Tính xác suất để tổng số chấm xuất hiện trên hai con xúc xắc bằng $ 4 $ hoặc bằng $ 6 $.
% 	\loigiai{
% 	Ta mô tả không gian mẫu $ \Omega $ bằng cách vẽ bảng như sau
% 	\begin{center}
% 	\begin{tabular}{|l|c|c|c|c|c|c|}
% 		\hline
% 		\diaghead(-4,1){\hskip4.2cm}%^^A
% 		{Xúc xắc II}{Xúc xắc  I}&
% 		\thead{$ 1 $}&\thead{$ 2 $}&\thead{$ 3 $}&\thead{$ 4 $}&\thead{$ 5 $}&\thead{$ 6 $}\\
% 		\hline
% 		\makecell[b]{$ 1 $} & $ (1,1) $ & $ (1,2) $&$ (1,3) $&$ (1,4) $&$ (1,5) $&$ (1,6) $\\
% 		\hline
% 		\makecell[b]{$ 2 $} & $ (2,1) $ & $ (2,2) $&$ (2,3) $&$ (2,4) $&$ (2,5) $&$ (2,6) $\\
% 		\hline
% 		\makecell[b]{$ 3 $} & $ (3,1) $ & $ (3,2) $&$ (3,3) $&$ (3,4) $&$ (3,5) $&$ (3,6) $\\
% 		\hline
% 		\makecell[b]{$ 4 $} & $ (4,1) $ & $ (4,2) $&$ (4,3) $&$ (4,4) $&$ (4,5) $&$ (4,6) $\\
% 		\hline
% 		\makecell[b]{$ 5 $} & $ (5,1) $ & $ (5,2) $&$ (5,3) $&$ (5,4) $&$ (5,5) $&$ (5,6) $\\
% 		\hline
% 		\makecell[b]{$ 6 $} & $ (6,1) $ & $ (6,2) $&$ (6,3) $&$ (6,4) $&$ (6,5) $&$ (6,6) $\\
% 		\hline
% 	\end{tabular}
% 	\end{center}
% 	Mỗi ô là một kết quả có thể. Ta có $ n(\Omega)=36 $.\\
% 	Biến cố $ \mathrm{E}$: \lq\lq  Tổng số chấm xuất hiện trên hai con xúc xắc bằng $ 4 $ hoặc bằng $ 6 $\rq\rq\, xảy ra khi xuất hiện các ô có tổng bằng $ 4 $: $ (1;3); (2;2); (3;1)$ hoặc các ô có tổng bằng $ 6 $: $ (1;5); (2;4); (3;3); (4;2); (5;1) $.\\
% 	Vậy $ \mathrm{E}=\{(1;3); (2;2); (3;1); (1;5); (2;4); (3;3); (4;2); (5;1)\}$. Từ đó $ n(\mathrm{E})=8 $.\\
% 	Vậy $\mathrm{P}(\mathrm{E})=\dfrac{n(\mathrm{E})}{n(\Omega)}=\dfrac{8}{36}=\dfrac{2}{9}$.
% 	}
% \end{vd}


% %%%%sách CTST%%%%
% \begin{vd}%[1D2B4-1]
% 	Hộp thứ nhất đựng $4$ tấm thẻ  cùng loại được đánh số từ $1$ đến $4$. Hộp thứ hai đựng $6$ tấm thẻ cùng loại được đánh số từ $1$ đến $6$. Lấy ra ngẫu nhiên từ mỗi hộp một tấm thẻ.
% 	\begin{enumerate}
% 		\item Hãy xác định không gian mẫu của phép thử.
% 		\item Gọi $A$ là biến cố \lq\lq  Hai thẻ lấy ra có cùng số\rq\rq. Hãy liệt kê các kết quả thuận lợi cho $A$ và tính xác suất của biến cố $A$.
% 		\item Gọi $B$ là biến cố \lq\lq  Tổng hai số trên hai thẻ lấy ra lớn hơn 8\rq\rq. Hãy liệt kê các kết quả thuận lợi cho $B$ và tính xác suất của biến cố $B$.
% 	\end{enumerate}
% 	\loigiai{
% 		\begin{enumerate}
% 			\item Kết quả của mỗi lần thử là một cặp $(i;j)$ với $i\in\{1;2;3;4\}$ là số trên thẻ được lấy ra từ hộp thứ nhất và $j\in\{1;2;3;4;5;6\}$ là số thẻ được lấy ra từ hộp thứ hai. Không gian mẫu của phép thử là
% 			\begin{align*}
% 				\Omega=\{&(1;1);(1;2);(1;3);(1;4);(1;5);(1;6);\\
% 				&(2;1);(2;2);(2;3);(2;4);(2;5);(2;6);\\
% 				&(3;1);(3;2);(3;3);(3;4);(3;5);(3;6);\\
% 				&(4;1);(4;2);(4;3);(4;4);(4;5);(4;6)\}.
% 			\end{align*}
% 			\item Không gian mẫu gồm $24$ kết quả, tức là $n(\Omega)=24$.\\
% 			Biến cố $A=\{(1;1);(2;2);(3;3);(4;4)\}$.\\
% 			Số các kết quả thuận lợi cho $A$ là $n(A)=4$. Do đó, xác suất của biến cố $A$ là
% 			$$\mathrm P(A)=\dfrac{4}{24}=\dfrac{1}{6}.$$
% 			\item Biến cố $B=\{(3;6);(4;5);(4;6);\}$.\\
% 			Số các kết quả thuận lợi cho $B$ là $n(B)=3$. Do đó, xác suất của biến cố $B$ là $$\mathrm P(B)=\dfrac{3}{24}=\dfrac{1}{8}.$$
% 		\end{enumerate}
% 	}
% \end{vd}

% \begin{vd}%[1D2B4-1]
% 	Gieo hai con xúc xắc cân đối và đồng chất. Tính xác suất của các biến cố 
% 	\begin{enumerate}
% 		\item \lq\lq  Hai mặt xuất hiện có cùng số chấm \rq\rq;
% 		\item \lq\lq  Tổng số chấm trên hai mặt bằng $9$\rq\rq.
% 	\end{enumerate}
% 	\loigiai{
% 		\begin{enumerate}
% 			\item Không gian mẫu là $\Omega=\{(i;j)\,|\,1\leq i,j\leq 6\}$. Do đó $n(\Omega)=6\cdot 6=36$.\\
% 			Biến cố $A=\{(1;1);(2;2);(3;3);(4;4);(5;5);(6;6)\}$.\\
% 			Số các kết quả thuận lợi cho $A$ là $n(A)=6$. Do đó, xác suất của biến cố $A$ là
% 			$$\mathrm P(A)=\dfrac{6}{36}=\dfrac{1}{6}.$$
% 			\item Biến cố $B=\{(6;3);(5;4);(4;5);(3;6)\}$.\\
% 			Số các kết quả thuận lợi cho $B$ là $n(B)=4$. Do đó, xác suất của biến cố $B$ là $$\mathrm P(B)=\dfrac{4}{36}=\dfrac{1}{9}.$$
% 		\end{enumerate}
% 	}
% \end{vd}

% \begin{vd}%[1D2B4-1]
% 	Trong hộp có $5$ viên bi xanh và $7$ viên bi trắng có kích thước và khối lượng như nhau. Ta lấy hai viên bi bằng hai cách như sau:
% 	\begin{itemize}
% 		\item Cách thứ nhất: Lấy ngẫu nhiên một viên bi, xem màu rồi trả lại hộp. Sau đó lại lấy một viên bi một cách ngẫu nhiên.
% 		\item Cách thứ hai: Lấy cùng lúc hai viên bi từ hộp. 
% 	\end{itemize}
% 	Gọi $A$ là biến cố \lq\lq  Cả hai bi lấy ra đều là màu trắng\rq\rq. Với cách lấy bi nào thì biến cố $A$ có khả năng xảy ra cao hơn.
% 	\loigiai{
% 		Theo cách lấy bi thứ nhất, áp dụng quy tắc nhân ta có số phần tử của không gian mẫu là $n(\Omega)=12\cdot 12=144$.\\
% 		Số khả năng thuận lợi cho $A$ là $7\cdot 7=49$.\\
% 		Do đó xác suất của biến cố $A$ theo cách lấy thứ nhất là $\dfrac{49}{144}$.\\
% 		Theo cách lấy bi thứ hai, số phần tử của không gian mẫu là $n(\Omega)=\mathrm C_{12}^2=66$.\\
% 		Số khả năng thuận lợi cho $A$ là $n(A)=\mathrm C_7^2=21$.\\
% 		Do đó xác suất của biến cố $A$ theo cách lấy thứ hai là $\dfrac{21}{66}=\dfrac{7}{22}$.\\
% 		Ta có $\dfrac{49}{144}>\dfrac{7}{22}$ nên với cách lấy bi thứ nhất thì biến cố $A$ có khả năng xảy ra cao hơn.
% 	}
% \end{vd}

% \begin{vd}%[1D2B4-1]
% 	Lấy ra ngẫu nhiên đồng thời $2$ viên bi từ một hộp có chứa $5$ bi xanh và $5$ bi đỏ có cùng kích thước và trọng lượng. Biến cố lấy được $2$ viên bi cùng màu hay $2$ viên bi khác màu có khả năng xảy ra cao hơn?
% 	\loigiai{
% 		Số phần tử của không gian mẫu là $n(\Omega)=\mathrm C_{10}^2=45$.\\
% 		Gọi $A$ là biến cố lấy được $2$ viên bi cùng màu, $B$ là biến cố lấy được $2$ viên bi khác màu.\\
% 		Ta có $n(A)=\mathrm C_5^2+\mathrm C_5^2=20$. Suy ra $\mathrm P(A)=\dfrac{20}{45}=\dfrac{4}{9}$.\\
% 		Ta có $n(B)=5\cdot 5=25$. Suy ra $\mathrm P(B)=\dfrac{25}{45}=\dfrac{5}{9}$.\\
% 		Vì $P(B)>P(A)$ nên biến cố lấy được hai bi khác màu có khả năng xảy ra cao hơn biến cố lấy được hai bi cùng màu.
% 	}
% \end{vd}
% \baitaptl
% \begin{bt}%[Dự án Bài giảng Toán 10, đợt 2]%[Đào-V- Thủy]%[1D2B5-2]
% 	Gieo một con xúc xắc $4$ mặt cân đối và đồng chất ba lần. Tính xác suất của các biến cố:
% 	\begin{enumerate}
% 		\item \lq\lq  Tổng các số xuất hiện ở đỉnh phía trên của con xúc xắc trong ba lần gieo lớn hơn $2$\rq\rq.
% 		\item \lq\lq  Có đúng một lần số xuất hiện ở đỉnh phía trên của con xúc xắc là $2$\rq\rq.
% 	\end{enumerate}
% 	\loigiai
% 	{
% 		\begin{enumerate}
% 			\item Biến cố \lq\lq  Tổng các số xuất hiện ở đỉnh phía trên của con xúc xắc trong ba lần gieo lớn hơn $2$\rq\rq\ là biến cố chắc chắn nên có xác suất bằng $1$.
% 			\item Không gian mẫu $\Omega= \{(a,b,c) \text{ với } 1\le a,b,c \le 4\}$, trong đó $a$, $b$, $c$ lần lượt là số chấm xuất hiện trong các lần gieo 1, lần gieo 2, lần gieo 3.\\
% 			Số các kết quả có thể xảy ra là $n(\Omega)= 4^3=64$.\\
% 			Gọi $B$ là biến cố \lq\lq  Có đúng một lần số xuất hiện ở đỉnh phía trên của con xúc xắc là $2$\rq\rq.\\
% 			Số $2$ xuất hiện trong $1$ lần gieo, có $3$ cách.\\
% 			Hai lần gieo còn lại, số xuất hiện thuộc tập $\{1; 3; 4\}$, có $3^2$ cách.\\
% 			Theo quy tắc nhân, ta có $n(B)= 3\cdot 3^2=27$.\\
% 			Vậy xác suất cần tìm là $\mathrm{P}(B)= \dfrac{27}{64}$.
% 		\end{enumerate}
% 	}
% \end{bt}

% \begin{bt}%[Dự án Bài giảng Toán 10, đợt 2]%[Đào-V- Thủy]%[1D2B5-2]
% 	Gieo một xúc xắc hai lần liên tiếp. Tính xác suất của mỗi biến cố sau:
% 	\begin{enumerate}
% 		\item \lq\lq  Tổng số chấm xuất hiện trong hai lần gieo không bé hơn $10$\rq\rq.
% 		\item \lq\lq  Mặt $6$ chấm xuất hiện ít nhất một lần\rq\rq.
% 	\end{enumerate}
% 	\loigiai
% 	{
% 		Không gian mẫu $\Omega= \{(a, b), \, 1\le a, b\le 6\}$, trong đó $a$, $b$ tương ứng là số chấm xuất hiện ở lần gieo thứ nhất và thứ hai $\Rightarrow n(\Omega)= 6\cdot 6=36$.
% 		\begin{enumerate}
% 			\item Gọi $A$ là biến cố \lq\lq  Tổng số chấm xuất hiện trong hai lần gieo không bé hơn $10$\lq\lq .\\
% 			Ta có $A=\{(4,6); (5,5), (5,6); (6,4); (6,5); (6,6)\}$. Do đó $n(A)= 6$.\\
% 			Vậy xác suất cần tìm là $\mathrm{P}(A)= \dfrac{6}{36}= \dfrac{1}{6}$.
% 			\item Gọi $B$ là biến cố \lq\lq  Mặt $6$ chấm xuất hiện ít nhất một lần\lq\lq .\\
% 			Ta có $B=\{(6,1); (6,2); (6,3); (6,4); (6,5); (6,6); (1,6); (2,6); (3,6); (4,6); (5,6)\}$.\\
% 			Suy ra $n(B)=11$.\\
% 			Vậy xác suất cần tìm là $\mathrm{P}(B)= \dfrac{11}{36}$.
% 		\end{enumerate}
% 	}
% \end{bt}

% \begin{bt}%[Dự án Bài giảng Toán 10, đợt 2]%[Đào-V- Thủy]%[1D2B5-2]
% 	Gieo một đồng xu và một con xúc xắc đồng thời. Tính xác suất của biến cố $A$: \lq\lq  Đồng xu xuất hiện mặt sấp hoặc con xúc xắc xuất hiện mặt $5$ chấm\rq\rq.
% 	\loigiai
% 	{
% 		Không gian mẫu $\Omega= \{(1,S); (2,S); (3,S); (4,S); (5,S); (6,S); (1,N); (2,N); (3,N); (4,N); (5,N); (6,N)\}$. Suy ra $n(\Omega)= 12$.\\
% 		Ta có $A=\{(1,S); (2,S); (3,S); (4,S); (5,S); (6,S); (5,N)\}$ $\Rightarrow n(A)=7$.\\
% 		Vậy xác suất cần tìm là $\mathrm{P}(A)= \dfrac{7}{12}$.
% 	}
% \end{bt}

% \begin{bt}%[Dự án Bài giảng Toán 10, đợt 2]%[Đào-V- Thủy]%[1D2B5-2]
% 	Có hai hộp $I$ và $II$. Hộp thứ nhất chứa $12$ tấm thẻ vàng đánh số từ $1$ đến $12$. Hộp thứ hai chứa $6$ tấm thẻ đỏ đánh số từ $1$ đến $6$. Rút ngẫu nhiên từ mỗi hộp một tấm thẻ. Tính xác suất của các biến cố.
% 	\begin{enumerate}
% 		\item $A$: \lq\lq  Cả hai tấm thẻ đều mang số $5$\rq\rq.
% 		\item $B$: \lq\lq  Tổng hai số trên hai tấm thẻ bằng $6$\rq\rq.
% 	\end{enumerate}
% 	\loigiai
% 	{
% 		Không gian mẫu $\Omega= \{(a,b) \text{ với } 1\le a\le 12 \text{ và } 1\le b\le 6\}$, trong đó $a$, $b$ lần lượt là số ghi trên các thẻ được rút ra từ hộp $I$ và hộp $II$.\\
% 		Ta có $n(\Omega)= 12\cdot 6=72$.
% 		\begin{enumerate}
% 			\item $A= \{(5,5)\} \Rightarrow n(A)=1$.\\
% 			Vậy xác suất cần tìm là $\mathrm{P}(A)= \dfrac{1}{72}$.
% 			\item $B= \{(1,5), (2,4), (3,3), (4,2), (5,1)\} \Rightarrow n(B)=5$.\\
% 			Vậy xác suất cần tìm là $\mathrm{P}(B)= \dfrac{5}{72}$.
% 		\end{enumerate}
% 	}
% \end{bt}

% \begin{bt}%[Dự án Bài giảng Toán 10, đợt 2]%[Đào-V- Thủy]%[1D2B5-2]
% 	Có ba chiếc hộp. Hộp thứ nhất chứa $5$ tấm thẻ đánh số từ $1$ đến $5$. Hộp thứ hai chứa $6$ tấm thẻ đánh số từ $1$ đến $6$. Hộp thứ ba chứa $7$ tấm thẻ đánh số từ $1$ đến $7$. Từ mỗi hộp rút ngẫu nhiên một tấm thẻ. Tính xác suất để tổng ba số ghi trên ba tấm thẻ bằng $15$.
% 	\loigiai
% 	{
% 		Không gian mẫu $\Omega= \{(a,b,c) \text{ với } 1\le a\le 5,\, 1 \le b\le 6,\, 1\le c\le 7\}$. Suy ra $n(\Omega)= 5\cdot 6\cdot 7 = 210$.\\
% 		Gọi $A$ là biến cố \lq\lq  Tổng ba số ghi trên ba tấm thẻ bằng $15$\rq\rq.\\
% 		Ta có $A=\{(2,6,7); (3,6,6); (3,5,7); (4,6,5); (4,5,6); (4,4,7); (5,3,7); (5,4,6); (5,5,5); (5,6,4)\}$.\\
% 		Do đó $n(A)=10$.\\
% 		Vậy xác suất cần tìm là $\mathrm{P}(A)= \dfrac{10}{210}= \dfrac{1}{21}$.
% 	}
% \end{bt}


% \begin{dang}{Vận dụng}
% 	Giả sử biến cố $ \mathrm{A} $ có xác suất $\mathrm{P}(\mathrm{A})$. Khi thự hiện phép thử $ n $ lần ($ n\geq 30 $) thì số lần xuất hiện biến cố $  \mathrm{A} $ sẽ xấp xỉ bằng $ n\cdot \mathrm{P}(\mathrm{A}) $ (nói chung khi $ n $ càng lớn thì sai số tương đối  càng bé).
% \end{dang}

% \begin{vd}%[1D2B4-1]
% 	Giả thiết rằng xác suất sinh con trai là $ 0{,}512 $ và xác suất sinh con gái là $ 0{,}488 $. Vận dụng ý nghĩa thực tế của xác suất, hãy ước tính trong số trẻ mới sinh với $ 10~000 $ bé gái thì có bao nhiêu bé trai.
% 	\loigiai{
% 	Vận dụng ý nghĩa thực tế của xác suất, ta có $ n\cdot 0{,}488\approx 10~000 $.\\
% 	Vậy $ n\approx \dfrac{10~000}{0{,}488}=20~491{,}80328\ldots $\\
% 	Vậy có khoảng $ 20~492 $ trẻ mới sinh. Từ đó với $ 10~000 $ bé gái thì có khoảng $$20~492-10~000=10~492\,\text{bé trai}.$$
% 	}
% \end{vd}
% \subsection{Câu hỏi trắc nghiệm}

% \Opensolutionfile{ansbook}[ans/ansbook-0D26-TN]
% \Opensolutionfile{ans}[ans/ans-0D26-TN]
% \begin{dang}{Mô tả biến cố}
	
% \end{dang}
% % \begin{center}
% % 	%	\baitaptn
% % \end{center}

% %Bài 1
% \begin{ex}%[Mui Doan]%[1D2Y4-1]
% 	Một hộp có bốn loại bi: bi xanh, bi đỏ, bi trắng và bi vàng. Lấy ngẫu nhiên ra một viên bi. Gọi $ \mathrm{E} $	là biến cố: \lq\lq  Lấy được viên bi đỏ\rq\rq. Biến cố đối của $ \mathrm{E} $  là biến cố 
% 	\choice
% 	{Lấy được viên bi xanh}
% 	{Lấy được viên bi vàng hoặc bi trắng}
% 	{Lấy được viên bi trắng}
% 	{\True Lấy được viên bi vàng hoặc bi trắng hoặc bi xanh}
% 	\loigiai{
% 		Biến cố đối của $ \mathrm{E} $  là biến cố \lq\lq 	Lấy được viên bi vàng hoặc bi trắng hoặc bi xanh\rq\rq.
% 	}
% \end{ex}
% %Bài 2
% \begin{ex}%[Mui Doan]%[1D2B5-2]
% 	Rút ngẫu nhiên ra một thẻ từ một hộp có $ 30 $ tấm thẻ được đánh số từ $ 1 $ đến $ 30 $. Xác suất để số  trên tấm thẻ được rút ra chia hết cho $ 5 $ là
% 	\choice
% 	{$ \dfrac{1}{30} $}
% 	{\True $ \dfrac{1}{5} $}
% 	{$ \dfrac{1}{3} $}
% 	{$ \dfrac{2}{5} $}
% 	\loigiai{
% 		Ta có $ n(\Omega)=30 $.\\
% 		Số chia hết cho $ 5 $ từ $ 1 $ đến $ 30 $ thuộc tập $ \mathrm{A}=\{5;10;15;20;25;30\}\Rightarrow n(\mathrm{A})=6 $.\\
% 		Vậy $ \mathrm{P}(\mathrm{A})=\dfrac{n(\mathrm{A})}{n(\Omega)}=\dfrac{1}{5} $.
% 	}
% \end{ex}
% %Bài 3
% \begin{ex}%[Mui Doan]%[1D2B5-2]
% 	Gieo hai con xúc xắc cân đối. Xác suất để tổng số chấm xuất hiện trên hai con xúc xắc không lớn hơn $ 4 $ là 
% 	\choice
% 	{$ \dfrac{1}{7} $}
% 	{\True $ \dfrac{1}{6} $}
% 	{$ \dfrac{1}{8} $}
% 	{$ \dfrac{2}{9} $}
% 	\loigiai{
% 		Ta có $ n(\Omega)=6\cdot 6=36 $.\\	
% 		Gọi $ \mathrm{A} $	là biến cố: \lq\lq  tổng số chấm xuất hiện trên hai con xúc xắc không lớn hơn $ 4 $\rq\rq.\\
% 		Ta có $  \mathrm{A}=\{(1;1); (1;2); (1;3); (2;1); (2;2); (3;1)\}\Rightarrow n(\mathrm{A})=6$.\\
% 		Vậy $ \mathrm{P}(\mathrm{A})=\dfrac{n(\mathrm{A})}{n(\Omega)}=\dfrac{1}{6} $.
% 	}
% \end{ex}
% %Bài 4
% \begin{ex}%[Mui Doan]%[1D2B5-2]
% 	Một tổ trong lớp $ 10\mathrm{T} $	có $ 4 $ bạn nữ và $ 3 $ bạn nam. Giáo viên chọn ngẫu nhiên hai bạn trong tổ đó tham gia đội làm báo của lớp. Xác suất để hai bạn được chọn có một bạn nam và một bạn nữ là 
% 	\choice
% 	{\True $ \dfrac{4}{7} $}
% 	{$ \dfrac{2}{7} $}
% 	{$ \dfrac{1}{6} $}
% 	{$ \dfrac{2}{21} $}
% 	\loigiai{
% 		Ta có $ n(\Omega)=\mathrm{C}_7^2=21 $.\\	
% 		Gọi $ \mathrm{A} $	là biến cố: \lq\lq  hai bạn được chọn có một bạn nam và một bạn nữ\rq\rq.\\
% 		Ta có $ n(\mathrm{A})=\mathrm{C}_4^1\cdot \mathrm{C}_3^1=12$.\\
% 		Vậy $ \mathrm{P}(\mathrm{A})=\dfrac{n(\mathrm{A})}{n(\Omega)}=\dfrac{4}{7} $.	
% 	}
% \end{ex}

% % \begin{dang}{Bài tập trắc nghiệm sách BT-CTST}
	
% % \end{dang}
% % \setcounter{ex}{0}
% %Bài 1
% \begin{ex}%[Mui Doan]%[1D2B5-2]
% 	Một hộp có $ 4 $	 viên bi xanh, $ 5 $ viên bi đỏ có kích thước và khối lượng như nhau. Lấy ra ngẫu nhiên đồng thời $ 2 $ viên bi. Xác suất của biến cố \lq\lq  $ 2 $ viên bi lấy ra đều là bi xanh\rq\rq \,là
% 	\choice
% 	{$ \dfrac{1}{2} $}
% 	{$ \dfrac{1}{3} $}
% 	{$ \dfrac{1}{5} $}
% 	{\True $ \dfrac{1}{6} $}
% 	\loigiai{
% 		Ta có $ n(\Omega)=\mathrm{C}_9^2=36 $.\\	
% 		Gọi $ \mathrm{A} $	là biến cố: \lq\lq  $ 2 $ viên bi lấy ra đều là bi xanh\rq\rq.\\
% 		Ta có $ n(\mathrm{A})=\mathrm{C}_4^2=6$.\\
% 		Vậy $ \mathrm{P}(\mathrm{A})=\dfrac{n(\mathrm{A})}{n(\Omega)}=\dfrac{1}{6} $.		
% 	}
% \end{ex}
% %Bài 2
% \begin{ex}%[Mui Doan]%[1D2B5-2]
% 	Gieo hai con xúc xắc cân đối và đồng chất. Xác suất để tích số chấm xuất hiện bằng $ 7 $ là
% 	\choice
% 	{\True $ 0 $}
% 	{$ \dfrac{1}{36} $}
% 	{$ \dfrac{1}{7} $}
% 	{$ \dfrac{1}{6} $}
% 	\loigiai{
% 		Gọi $ \mathrm{A} $	là biến cố: \lq\lq  tích số chấm xuất hiện bằng $ 7 $ \rq\rq.\\
% 		Nếu tích số chấm xuất hiện bằng $ 7 $ thì phải có một số bằng $ 7 $ mà không có mặt nào của con xúc xắc ghi số  $ 7 $ nên $ \mathrm{A}$ là biến cố không thể.\\
% 		Vậy $ \mathrm{P}(\mathrm{A})=0 $.
% 	}
% \end{ex}
% %Bài 3
% \begin{ex}%[Mui Doan]%[1D2B5-2]
% 	Tung $ 3 $ đồng xu cân đối và đồng chất. Xác suất để có ít nhất một đồng xu xuất hiện mặt sấp là
% 	\choice
% 	{$ \dfrac{1}{2} $}
% 	{\True $ \dfrac{7}{8} $}
% 	{$ \dfrac{1}{3} $}
% 	{$ \dfrac{1}{4} $}
% 	\loigiai{
% 		Ta có không gian mẫu là $ \Omega=\{SNS; SNN; SSN; SSS; NSN; NSS; NNS; NNN \} \Rightarrow n(\Omega)=8$.\\
% 		Gọi $ \mathrm{A} $	là biến cố: \lq\lq  có ít nhất một đồng xu xuất hiện mặt sấp\rq\rq.\\
% 		Ta có 	$\mathrm{A} =\{SNS; SNN; SSN; SSS; NSN; NSS; NNS \} \Rightarrow n( \mathrm{A} )=7$.\\
% 		Vậy $ \mathrm{P}(\mathrm{A})=\dfrac{n(\mathrm{A})}{n(\Omega)}=\dfrac{7}{8} $.		
% 	}
% \end{ex}
% %Bài 4
% \begin{ex}%[Mui Doan]%[1D2B5-2]
% 	Một hộp chứa $ 2 $	loại bi xanh và đỏ. Lấy ra ngẫu nhiên từ hộp  $ 1 $ viên bi. Biết xác suất lấy được bi đỏ là $ 0{,}3 $. Xác suất lấy được bi xanh là
% 	\choice
% 	{$ 0{,}3 $}
% 	{$ 0{,}5 $}
% 	{\True $ 0{,}7 $}
% 	{$ 0{,}09 $}
% 	\loigiai{
% 		Vì xác suất lấy được bi đỏ là $ 0{,}3 $ nên xác suất lấy được bi xanh là $ 1-0{,}3=0{,}7 $.
% 	}
% \end{ex}
% %Bài 5
% \begin{ex}%[Mui Doan]%[1D2B5-2]
% 	Gieo một con xúc xắc bốn mặt cân đối và đồng chất ba lần. Xác suất xảy ra biến cố \lq\lq  Có ít nhất một lần xuất hiện đỉnh ghi số $ 4 $\rq\rq \,là
% 	\choice
% 	{$ \dfrac{1}{4} $}
% 	{$ \dfrac{27}{64} $}
% 	{\True $ \dfrac{37}{64} $}
% 	{$ \dfrac{3}{4} $}
% 	\loigiai{
% 		Ta có $ n(\Omega)=4\cdot 4\cdot 4=64 $.\\	
% 		Gọi $ \mathrm{A} $	là biến cố: \lq\lq  Có ít nhất một lần xuất hiện đỉnh ghi số $ 4 $\rq\rq.\\
% 		Suy ra $ n\left(\overline{\mathrm{A}}\right)=3\cdot 3\cdot 3=27 \Rightarrow n(\mathrm{A})=64-27=37$.\\
% 		Vậy $ \mathrm{P}(\mathrm{A})=\dfrac{n(\mathrm{A})}{n(\Omega)}=\dfrac{37}{64}$.				
% 	}
% \end{ex}
% %Bài 6
% \begin{ex}%[Mui Doan]%[1D2B5-2]
% 	Chọn ngẫu nhiên ra $ 2 $	 người từ $ 35 $ người trong lớp của Hùng. Xác suất xảy ra biến cố \lq\lq  Hùng được chọn\rq\rq \,là
% 	\choice
% 	{\True $ \dfrac{2}{35} $}
% 	{$ \dfrac{1}{34} $}
% 	{$ \dfrac{1}{35} $}
% 	{$ \dfrac{1}{17} $}
% 	\loigiai{
% 		Ta có $ n(\Omega)=\mathrm{C}_{35}^2=595 $.\\	
% 		Gọi $ \mathrm{A} $	là biến cố: \lq\lq  Hùng được chọn\rq\rq.\\
% 		Ta có $ n(\mathrm{A})=\mathrm{C}_{34}^1=34$.\\
% 		Vậy $ \mathrm{P}(\mathrm{A})=\dfrac{n(\mathrm{A})}{n(\Omega)}=\dfrac{34}{595} =\dfrac{2}{35}$.			
% 	}
% \end{ex}
% %Bài 7
% \begin{ex}%[Mui Doan]%[1D2B5-2]
% 	Xếp $ 4 $	quyển sách toán và $ 2 $ quyển sách văn thành một hàng ngang trên giá sách một cách ngẫu nhiên. Xác suất xảy ra biến cố \lq\lq  $ 2 $ quyển sách văn không được xếp cạnh nhau\rq\rq \,là
% 	\choice
% 	{$ \dfrac{1}{3} $}
% 	{\True $ \dfrac{2}{3} $}
% 	{$ \dfrac{1}{2} $}
% 	{$ \dfrac{1}{5} $}
% 	\loigiai{
% 		Ta có $ n(\Omega)=6!=720 $.\\	
% 		Gọi $ \mathrm{A} $	là biến cố: \lq\lq  $ 2 $ quyển sách văn không được xếp cạnh nhau\rq\rq.\\
% 		Trước hết ta xếp $ 4 $ sách toán có $ 4! $ cách xếp.\\
% 		Có $ 5 $ vị trí trống ta xếp $ 2 $ quyển sách văn vào có $ \mathrm{A}_5^2 $ cách.\\
% 		Suy ra  $ n(\mathrm{A})=4!\cdot \mathrm{A}_5^2=480$ cách.\\
% 		Vậy $ \mathrm{P}(\mathrm{A})=\dfrac{n(\mathrm{A})}{n(\Omega)}=\dfrac{480}{720} =\dfrac{2}{3}$.			
% 	}
% \end{ex}
% %Bài 8
% \begin{ex}%[Mui Doan]%[1D2B5-2]
% 	Cô giáo chia tổ của Lan và Phương thành hai nhóm, mỗi nhóm gồm $ 4 $	 người để làm việc nhóm một cách ngẫu nhiên. Xác suất của biến cố Lan và Phương thuộc cùng một nhóm là
% 	\choice
% 	{$ \dfrac{1}{2} $}
% 	{$ \dfrac{1}{3} $}
% 	{$ \dfrac{4}{7} $}
% 	{\True $ \dfrac{3}{7} $}
% 	\loigiai{
% 		Ta có $ n(\Omega)=\mathrm{C}_8^4\cdot \mathrm{C}_4^4 =70 $.\\	
% 		Gọi $ \mathrm{A} $	là biến cố: \lq\lq  Lan và Phương thuộc cùng một nhóm\rq\rq.
% 		\begin{itemize}
% 			\item Xếp Lan và Phương vào hai nhóm có $ 2 $ cách.
% 			\item Chọn $ 3 $ người vào nhóm Lan có $ \mathrm{C}_6^3 $ cách.
% 			\item Chọn $ 3 $ người còn lại vào nhóm Phương có $ \mathrm{C}_3^3 $ cách.
% 		\end{itemize}
% 		Suy ra  $ n\left(\overline{\mathrm{A}}\right)=2\cdot \mathrm{C}_6^3 \cdot \mathrm{C}_3^3=40\Rightarrow  n(\mathrm{A})=70-40=30$ cách.\\
% 		Vậy $ \mathrm{P}(\mathrm{A})=\dfrac{n(\mathrm{A})}{n(\Omega)}=\dfrac{30}{70} =\dfrac{3}{7}$.				
% 	}
% \end{ex}
% \begin{dang}{Bài tập trắc nghiệm bổ sung}
	
% \end{dang}
% \setcounter{ex}{0}
% %Bài 1
% \begin{ex}%[Mui Doan]%[1D2B5-2]
% 	Chi đoàn lớp 12 Toán có $30$ đoàn viên trong đó có $12$ đoàn viên nam và $18$ đoàn viên nữ. Tính xác suất để khi chọn $3$ đoàn viên thì có ít nhất một đoàn viên nữ.
% 	\choice
% 	{$\dfrac{44}{203}$}
% 	{\True $\dfrac{192}{203}$}
% 	{$\dfrac{204}{1015}$}
% 	{$\dfrac{11}{203}$}
% 	\loigiai{Số cách chọn $3$ đoàn viên tùy ý là $\mathrm{C}_{30}^3$.\\
% 		Số cách chọn $3$ đoàn viên toàn là nam là $\mathrm{C}_{12}^3$.\\
% 		Số cách chọn $3$ đoàn viên sao cho có ít nhất một đoàn viên nữ là $\mathrm{C}_{30}^3-\mathrm{C}_{12}^3$.\\
% 		Vậy xác suất để chọn được $3$ đoàn viên trong đó có ít nhất một đoàn viên nữ là
% 		$$\mathrm{P}=\dfrac{\mathrm{C}_{30}^3-\mathrm{C}_{12}^3}{\mathrm{C}_{30}^3}=\dfrac{192}{203}.$$}
% \end{ex}
% %%Bài 2
% \begin{ex}%[Mui Doan]%[1D2B5-2]
% 	Để kiểm tra sản phẩm của một công ty sữa, người ta gửi đến bộ phận kiểm nghiệm $5$ hộp sữa cam, $4$ hộp sữa nho và $3$ hộp sữa dâu. Bộ phận kiểm nghiệm chọn ngẫu nhiên $3$ hộp sữa để phân tích mẫu. Xác suất để $3$ hộp sữa được chọn đủ cả $3$ loại là
% 	\choice
% 	{\True $\dfrac{3}{11}$}
% 	{$\dfrac{1}{5}$}
% 	{$\dfrac{3}{7}$}
% 	{$\dfrac{1}{6}$}
% 	\loigiai{
% 		Số khả năng xảy ra khi chọn $3$ hộp sữa để kiểm nghiệm là $\mathrm{C}^3_{12}=220$.\\
% 		Số khả năng thuận lợi chọn được $3$ hộp sữa đủ cả $3$ màu là $\mathrm{C}_5^1\cdot\mathrm{C}_4^1\cdot\mathrm{C}_3^1=5\cdot 4\cdot 3=60 $.	\\
% 		Vậy xác suất để chọn đủ cả $3$ loại sữa là
% 		$$\mathrm{P}=\dfrac{60}{220}=\dfrac{3}{11}.$$
% 	}
% \end{ex}
% %%Bài 3
% \begin{ex}%[Mui Doan]%[1D2B5-2]
% 	Một hộp chứa $9$ cái thẻ được đánh số $1,2,3,4,5,6,7,8,9$. Lấy ngẫu nhiên hai thẻ. Tính xác suất sao cho tổng các số trên hai thẻ là số chẵn.
% 	\choice
% 	{$\dfrac{5}{9}$}
% 	{\True $\dfrac{4}{9}$}
% 	{$\dfrac{1}{9}$}
% 	{$\dfrac{5}{3}$}
% 	\loigiai{
% 		Không gian mẫu $\left|\Omega\right|=\mathrm{C}_9^2$.\\
% 		Gọi $A$ là biến cố : \lq\lq  Tổng các số trên hai thẻ là số chẵn\rq\rq.
% 		Vì trong $9$ số nguyên dương đầu tiên có $5$ số lẻ và $4$ số chẵn nên xét $2$ trường hợp sau:
% 		\begin{itemize}
% 			\item \textbf{Trường hợp 1.} Cả hai thẻ đều chẵn. Có $\mathrm{C}_4^2$ cách chọn.
% 			\item \textbf{Trường hợp 2.} Cả hai thẻ đều lẻ. Có $\mathrm{C}_5^2$ cách chọn.
% 		\end{itemize}
% 		Suy ra $\left| A\right|=\mathrm{C}_5^2+\mathrm{C}_4^2$.\\
% 		Vậy $\mathrm{P}(A)=\dfrac{\left| A\right|}{\left|\Omega\right|}=\dfrac{\mathrm{C}_5^2+\mathrm{C}_4^2}{\mathrm{C}_9^2}=\dfrac{4}{9}$.
% 	}
% \end{ex}
% %%Bài 4

% %%Bài 5
% \begin{ex}%[Mui Doan]%[1D2K5-2]
% 	Một bài kiểm tra kiến thức về an toàn giao thông có $10$ câu hỏi trắc nghiệm, mỗi câu hỏi trắc nghiệm có bốn phương án lựa chọn và chỉ có duy nhất một lựa chọn đúng. Với mỗi câu hỏi, lựa chọn đúng được $1$ điểm, lựa chọn sai được $0$ điểm. Một thí sinh làm bài bằng cách chọn ngẫu nhiên một lựa chọn cho tất cả $10$ câu hỏi của bài kiểm tra. Tính xác suất để thí sinh được $5$ điểm.
% 	\choice
% 	{\True $\dfrac{\mathrm{C}_{10}^{5}\cdot 3^{5}}{4^{10}}$}
% 	{$\dfrac{\mathrm{C}_{10}^{5}}{4^{10}}$}
% 	{$\dfrac{1}{2}$}
% 	{$\dfrac{1}{\mathrm{C}_{10}^{5}}$}
% 	\loigiai{
% 		Mỗi câu hỏi có $4$ phương án nên không gian mẫu có số phần tử là $n(\Omega)=4^{10}$.\\
% 		Gọi $A$ là biến cố ``thí sinh chọn ngẫu nhiên được $5$ điểm''.\\
% 		Chọn ngẫu nhiên $5$ câu đúng trong $10$ câu có $\mathrm{C}_{10}^{5}$ cách chọn.\\
% 		$5$ câu còn lại chọn sai, mỗi câu có $3$ phương án sai nên có $3^5$ cách chọn.\\
% 		Suy ra $n(A)=\mathrm{C}_{10}^{5}\cdot 3^{5}$.\\
% 		Vậy $\mathrm{P}(A)=\dfrac{\mathrm{C}_{10}^{5}\cdot 3^{5}}{4^{10}}$.
% 	}
% \end{ex}
% %%Bài 6
% \begin{ex}%[Mui Doan]%[1D2K5-2]
% 	Ba bạn Chuyên, Quang, Trung mỗi bạn viết ngẫu nhiên lên bảng một số tự nhiên thuộc đoạn $ [1;17] $. Xác suất để ba số được viết ra có tổng chia hết cho $ 3 $ bằng
% 	\choice
% 	{$ \dfrac{1079}{4913} $}
% 	{$ \dfrac{23}{68} $}
% 	{\True $ \dfrac{1637}{4913} $}
% 	{$ \dfrac{1728}{4913} $}
% 	\loigiai{
% 		Số phần tử của không gian mẫu $ n\left(\Omega \right) =17\cdot 17\cdot 17=4913 $.\\
% 		Nhận thấy trong $ 17 $ số viết ra có
% 		\begin{itemize}
% 			\item $ 5 $ số chia hết cho $ 3 $.
% 			\item $ 6 $ số chia cho $ 3 $ dư $ 1 $.
% 			\item $ 6 $ số chia cho $ 3 $ dư $ 2 $.
% 		\end{itemize}
% 		Tổng ba số mà ba bạn viết ra chia hết cho $ 3 $ có thể xảy ra các trường hợp sau
% 		\begin{enumerate}
% 			\item Cả ba bạn đều viết số chia hết cho $ 3 $, trường hợp này có $ 5\cdot 5\cdot 5 =125$ cách.
% 			\item Cả ba bạn đều viết số chia cho $ 3 $ dư $ 1 $, trường hợp này có $ 6\cdot 6\cdot 6 =216$ cách.
% 			\item Cả ba bạn đều viết số chia cho $ 3 $ dư $ 2 $, trường hợp này có $ 6\cdot 6\cdot 6 =216$ cách.
% 			\item Một bạn viết số chia hết cho $ 3 $, một bạn viết số chia $3$ dư $ 1 $, một bạn viết số chia $ 3 $ dư 2. Khi đó ta có $ 3!\cdot 5\cdot 6\cdot 6 =1080$ cách.
% 		\end{enumerate}
% 		Do đó xác suất cần tìm là $ \mathrm{P}=\dfrac{125+216+216+1080}{4913}= \dfrac{1637}{4913}$.
% 	}
% \end{ex}
% %%Bài 7
% \begin{ex}%[Mui Doan]%[1D2K5-2]
% 	Cho đa giác đều $32$ cạnh. Gọi $S$ là tập hợp các tứ giác tạo thành có $4$ đỉnh lấy từ các đỉnh của đa giác đều. Chọn ngẫu nhiên một phần tử của $S$. Xác suất để chọn được một hình chữ nhật là
% 	\choice
% 	{$\dfrac{1}{341}$}
% 	{\True $\dfrac{3}{899}$}
% 	{$\dfrac{1}{385}$}
% 	{$\dfrac{1}{261}$}
% 	\loigiai{
% 		Số cách chọn ra $4$ đỉnh bất kỳ từ $32$ đỉnh là $|\Omega|=\mathrm{C}_{32}^4$.\\
% 		Đa giác đều $32$ cạnh thì có $16$ đường chéo là đường kính của đường tròn ngoại tiếp đa giác trên. Khi đó để tạo thành một hình chữ nhật từ $32$ đỉnh trên thì ta cần chọn $2$ đường chéo bất kỳ từ $16$ đường chéo đó, và có số cách chọn là $\mathrm{C}_{16}^2$.\\
% 		Vậy xác suất để chọn được một hình chữ nhật là $\mathrm{P}=\dfrac{\mathrm{C}_{16}^2}{\mathrm{C}_{32}^4}=\dfrac{120}{35960}=\dfrac{3}{899}$.
% 	}
% \end{ex}
% %%Bài 8
% \begin{ex}%[Mui Doan]%[1D2K5-2]
% 	Một nhóm học sinh gồm có $3$ học sinh lớp $A$, $4$ học sinh lớp $B$ và $5$ học sinh lớp $C$. Chọn ngẫu nhiên $2$ học sinh tham gia câu lạc bộ Toán học. Xác suất sao cho $2$ học sinh được chọn nếu có học sinh lớp $B$ thì không có học sinh lớp $C$ là
% 	\choice
% 	{$\dfrac{17}{22}$}
% 	{$\dfrac{9}{11}$}
% 	{$\dfrac{2}{3}$}
% 	{\True $\dfrac{23}{33}$}
% 	\loigiai{
% 		Không gian mẫu là $n(\Omega) = \mathrm{C}^2_{12} = 66$.\\
% 		Gọi $A$ là biến cố \lq\lq  $2$ học sinh được chọn nếu có học sinh lớp $B$ thì không có học sinh lớp $C$\rq\rq.\\
% 		Khi đó $\overline{A}$ là biến cố \lq\lq  $2$ học sinh được chọn là $1$ học sinh lớp $B$ và $1$ học sinh lớp $C$\rq\rq.\\
% 		Ta có $n\left( \overline{A} \right) = 4\times 5 = 20\Rightarrow n(A)=66-20=46$ nên $\mathrm{P}(A) = \dfrac{n(A)}{n(\Omega)} =  \dfrac{46}{66} = \dfrac{23}{33}$.
% 	}
% \end{ex}
% %Bài 9
% \begin{ex}%[Mui Doan]%[1D2K5-2]
% 	An và Bình cùng chơi một trò chơi, mỗi lượt chơi một bạn đặt úp năm tấm thẻ, trong đó có hai thẻ ghi số $2$, hai thẻ ghi số $3$ và một thẻ ghi số $4$, bạn còn lại chọn ngẫu nhiên ba thẻ trong năm tấm thẻ đó. Người chọn thẻ thắng lượt chơi nếu tổng các số trên ba tấm thẻ được chọn bằng $8$, ngược lại người kia sẽ thắng. Xác suất để An thắng lượt chơi khi An là người chọn thẻ bằng
% 	\choice
% 	{$\dfrac{1}{5}$}
% 	{$\dfrac{1}{10}$}
% 	{$\dfrac{3}{20}$}
% 	{\True $\dfrac{3}{10}$}
% 	\loigiai{
% 		Số cách chọn $3$ thẻ trong $5$ tấm thẻ là $n(\Omega)=\mathrm{C}_{5}^{3}=10$.\\
% 		Gọi $A$ là biến cố để An thắng lượt chơi.\\
% 		Số các trường hợp xảy ra cho $A$ là
% 		\begin{itemize}
% 			\item $2$ thẻ số $2$ và một thẻ số $4$ có $1$ cách.
% 			\item $2$ thẻ số $3$ và $1$ thẻ số $2$ có $2$ cách.
% 		\end{itemize}
% 		Suy ra số các trường hợp xảy ra cho A là $n(A)=3$.\\
% 		Vậy $\mathrm{P}(A)=\dfrac{n(A)}{n(\Omega)}=\dfrac{3}{10}$.
% 	}
% \end{ex}
% %Bài 10
% \begin{ex}%[Mui Doan]%[1D2K5-2]
% 	Vì yêu Toán nên khi đặt mật khẩu cho tài khoản facebook của mình, bạn Toàn đã dùng dãy các chứ cái \lq\lq  TOANYEUTOAN\rq\rq\,\, rồi thay đổi ngẫu nhiên các vị trí các chữ cái này để tạo ra mật khẩu. Tính xác suất để mật khẩu đó là một dãy chữ cái mà các chữ cái nếu xuất hiện $1$ lần thì không đứng cạnh nhau, đồng thời các chữ $T,\,N$ giống nhau thì đứng cạnh nhau.
% 	\choice
% 	{$\dfrac{1}{264}$}
% 	{$\dfrac{1}{1584}$}
% 	{$\dfrac{1}{54}$}
% 	{\True $\dfrac{1}{66}$}
% 	\loigiai{
% 		Số phần tử không gian mẫu là $n\left(\Omega\right)=\dfrac{11!}{2!2!2!2!}=\dfrac{11!}{16}$.\\
% 		Số cách sắp xếp các chữ cái TT, NN, O, O, A, A là $\dfrac{6!}{2!2!}$ cách. \\
% 		Do $6$ phần tử trên khi xếp tạo ra $7$ vị trí nên xếp $3$ chứ cái Y, E, U có $\mathbb{A}_7^3$ cách.\\
% 		Gọi $A$ là biến cố thỏa mãn yêu cầu đề bài, ta có $\mathrm{P}(A)=\dfrac{n(A)}{n(\Omega)}=\dfrac{\mathrm{A}_7^3\cdot\dfrac{6!}{2!2!}}{\dfrac{11!}{6}}=\dfrac{1}{66}$.\\
% 		Vậy xác suất cần tính bằng $\dfrac{1}{66}$.
% 	}
% \end{ex}
% %Bài 11
% \begin{ex}%[Mui Doan]%[1D2K5-2]
% 	Cho các chữ số $1,\,2,\,3,\,4,\,5$, gọi $S$ là tập hợp các số tự nhiên có $5$ chữ số trong đó có chữ số $3$ có mặt $3$ lần, các chữ số còn lại có mặt đúng một lần. Chọn ngẫu nhiên trong tập $S$ một số, tính xác suất để chọn được số chia hết cho $3$.
% 	\choice
% 	{$\dfrac{2}{5}$}
% 	{$\dfrac{1}{4}$}
% 	{$\dfrac{1}{3}$}
% 	{\True $\dfrac{2}{3}$}
% 	\loigiai{
% 		Gọi số cần tìm là $\overline{abcde}$.
% 		\begin{itemize}
% 			\item Sắp xếp $3$ chữ số $3$ vào $3$ trong $5$ vị trí, có $\mathrm{C}_5^3=10$ cách.
% 			\item Còn lại hai vị trí, chọn $2$ trong $4$ số còn lại và xếp, có $\mathrm{A}_4^2=12$ cách.
% 		\end{itemize}
% 		Suy ra có $10\cdot 12=120$ số có thể lập được $\Rightarrow n\left(\Omega\right)=120$.\\
% 		Các cặp số còn lại chỉ có thể là $(1;5)$, $(1;2)$, $(4;5)$, $(4;2)$.\\
% 		Do vậy có tất cả $4\cdot \mathrm{C}_5^3\cdot 2!=80$ số.\\
% 		Vậy xác suất cần tính là $\mathrm{P}=\dfrac{80}{120}=\dfrac{2}{3}$.
% 	}
% \end{ex}
% \Closesolutionfile{ans}
% \Closesolutionfile{ansbook}
% \indapan{10}{ans/ans-0D26-TN}
