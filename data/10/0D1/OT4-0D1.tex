\Opensolutionfile{ans}[ans/ansTL-CD7]
\setcounter{ex}{0}
\subsubsection{Đề số 4}

\begin{ex}%[0D1B1]
	Trong các phát biểu sau, phát biểu nào là mệnh đề đúng?
	\choice
	{ $\pi $ là một số hữu tỉ}
	{ Bạn có chăm học không?}
	{ Con thì thấp hơn cha}
	{ \True Tổng hai cạnh của một tam giác luôn lớn hơn cạnh thứ ba}
	\loigiai{
	}
\end{ex}

\begin{ex}%[0D1B5]
	Kết quả làm tròn số $b=500\sqrt{7}$ đến  chữ số thập phân thứ hai là
	\choice
	{$b\approx 1322,9 $}
	{$b\approx 1322,87$}
	{\True $b\approx 1322,88$}
	{$b\approx 1322,8 $}
	\loigiai{
	}
\end{ex}

\begin{ex}%[0D1B3]
	Cho hai tập hợp $A=\left\{a; b; 1; 2\right\}$ và $B=\left\{a; b; c; 1; 3\right\}.$ Tìm tập hợp $A\cap B.$
	\choice
	{$A\cap B=\left\{a,b,2\right\}$}
	{$A\cap B=\left\{2,3,c\right\}$}
	{\True$A\cap B=\left\{a,b,1\right\}$}
	{$A\cap B=\left\{a,b,3\right\}$}
	\loigiai{
	}
\end{ex}

\begin{ex}%[0D1Y1]
	Cho mệnh đề chứa biến $P(x)$: $x+2>x^2$. Mệnh đề nào sau đây là đúng?
	\choice
	{\True $P(1)$}
	{$P(3)$}
	{$P(-1)$}
	{$P(-3)$}
	\loigiai{
	}
\end{ex}

\begin{ex}%[0D1Y3-1]
	Cho hai tập hợp $A=(-3;3)$, $B=[-1;5]$. Tìm $A \cup B$?
	\choice
	{$[-1;3)$}
	{$(-3;-1]$}
	{$(3;5]$}
	{\True $(-3;5]$}
	\loigiai{
		Ta có $A \cup B=(-3;5]$.}
\end{ex}

\begin{ex}%[0D1Y4-1]
	Cho tập $A=\left\{x \in \mathbb{R}|1<x \leq 2\right\}$. Khẳng định nào sau đây đúng?
	\choice
	{$A=[1;2)$}
	{$A=[1;2]$}
	{\True $A=(1;2]$}
	{$A=(1;2)$}
	\loigiai{
		Ta có $A=\left\{x \in \mathbb{R}|1<x \leq 2\right\}=(1;2]$.
	}
\end{ex}


\begin{ex}%[0D1Y1-3]
	Cho mệnh đề $P\colon$ \lq\lq  $9$ là số chia hết cho $3$\rq\rq. Mệnh đề phủ định của mệnh đề $P$ là
	\choice
	{$\overline{P}\colon$\lq\lq  $9$ là bội của $3$\rq\rq}
	{\True $\overline{P}\colon$ \lq\lq  $9$ là số không chia hết cho $3$\rq\rq}
	{$\overline{P}\colon$\lq\lq  $9$ là ước của $3$\rq\rq}
	{$\overline{P}\colon$\lq\lq  $9$ là số lớn hơn $3$\rq\rq}
	\loigiai{
		Mệnh đề $P\colon$\lq\lq  $9$ là số chia hết cho $3$\rq\rq      có mệnh đề phủ định là $\overline{P}\colon$ \lq\lq  $9$ là số không chia hết cho $3$\rq\rq.}
\end{ex}

\begin{ex}%[0D1Y2-2]
	Cho tập $A=\{a;b;5\}$. Số tập con của tập $A$ là
	\choice
	{$5$}
	{$4$}
	{$7$}
	{\True $8$}
	\loigiai{
		Tập con của $A$ là $\varnothing, \{a\}, \{b\},\{5\}, \{a;b\}, \{a;5\}, \{b;5\}, \{a;b;5\} $. Vậy số tập con của $A$ là $8$.\\
		Có thể dùng công thức $2^3=8$ tập con.
	}
\end{ex}

\begin{ex}%[0D1B2-1]
	Cho tập hợp $B=\left\{n \in \mathbb{N}^*\bigg|3<n^2<100\right\}$. Số phần tử của $B$ là
	\choice
	{$7$}
	{$6$}
	{\True $8$}
	{$5$}
	\loigiai{
		Ta có $3<n^2<100 \Leftrightarrow 2 \leq n \leq 9$ (do $n \in \mathbb{N}^*$ ). Vậy tập hợp $B$ có $8$ phần tử.}
\end{ex}

\begin{ex}%[0D1B1-2]
	Cho mệnh đề $\forall x \in \mathbb{R} \colon x^2-2+a>0$ với $a$ là số thực cho trước. Tìm $a$ để mệnh đề đúng
	\choice
	{$a \leq 2$}
	{\True $a>2$}
	{$a=2$}
	{$a<2$}
	\loigiai{
		Điều kiện $x^2-2+a>0$ với mọi $x \in \mathbb{R} \Leftrightarrow x^2>2-a$ với mọi $x \in \mathbb{R} \Leftrightarrow 2-a<0 \Leftrightarrow a>2$.}
\end{ex}

\begin{ex}%[0D1B2]
	Viết tập hợp $A=\lbrace x \in \mathbb{Z}|x^2<17 \rbrace $ theo cách liệt kê các phần tử, ta được tập hợp nào sau đây?
	\choice
	{\True $ \lbrace -4;-3;-2;-1;0;1;2;3;4 \rbrace $}
	{$ \lbrace 1;2;3;4 \rbrace $}
	{$\lbrace 0;1;2;3;4  \rbrace $}
	{$ \lbrace -4; -3;-2;-1 \rbrace $}
	\loigiai{
	}
\end{ex}

\begin{ex}%[0D1B4-1]
	Cho tập hợp $M=[-3;6]$ và $N=(-\infty;-2)\cup (3;+\infty)$. Khi đó $M\cap N$ là
	\choice
	{\True $[-3;-2)\cup (3;6]$}
	{$(-\infty;-2)\cup [3;+\infty)$}
	{$(-\infty;-2)\cup [3;6]$}
	{$(-3;-2)\cup (3;6)$}
	\loigiai{
		$M\cap N=[-3;-2)\cup (3;6]$.
	}
\end{ex}

\begin{ex}%[0D1B3]
	Cho hai tập hợp $A = \left\{0, 2, 4, 6, 8\right\}$ và $B = \left\{0, 2, 4\right\}$. Tìm tập hợp $C_{A}B$.
	\choice
	{\True $\left\{6, 8\right\}$}
	{$\left\{2, 4\right\}$}
	{$\left\{0, 2, 4, 6\right\}$}
	{$\left\{0, 2, 4, 8\right\}$}
	\loigiai{
	}
\end{ex}

\begin{ex}%[0D1B4-1]
	Cho các tập hợp $M=[-3;6]$ và $N=(-\infty;-2) \cup (3;+\infty)$. Khi đó $M \cap N$ là
	\choice
	{\True $[-3;-2) \cup (3;6]$}
	{$(-3;-2) \cup (3;6)$}
	{$(-\infty;-2) \cup [3;+\infty)$}
	{$(-\infty;-2) \cup [3;6]$}
	\loigiai{
		Ta có $M \cap N=[-3;-2)\cup (3;6]$.}
\end{ex}

\begin{ex}%[0D1B3-2]
	Phần bù của $[-2;1)$ trong $\mathbb{R}$ là
	\choice
	{\True $(-\infty;-2) \cup [1;+\infty)$}
	{$(-\infty;-2)$}
	{$(2;+\infty)$}
	{$(-\infty;1]$}
	\loigiai{
		Ta có $\mathbb{R} \setminus [-2;1)=(-\infty;-2) \cup [1;+\infty)$.}
\end{ex}

\begin{ex}%[0D1B3-1]
	Cho hai tập hợp $A=(-3;3)$ và $B=(0;+\infty)$. Tìm  $A \cup B$.
	\choice
	{$ A \cup B = [-3;+\infty) $}
	{$ A \cup B = (0;3) $}
	{$ A \cup B = [-3;0) $}
	{\True $ A \cup B = (-3;+\infty) $}
	\loigiai{
		\immini{Tập hợp $A=(-3;3)$ có biểu diễn là}{
			\begin{tikzpicture}[scale=1, font=\footnotesize, line join = round, line cap = round, >=stealth]
			\draw[thick,->] (-2,0)node[below=6pt]{$-\infty$} -- (0,0) node[scale=1.5]{\bf ( } node[below=6pt]{$-3$} -- (3,0) node[scale=1.5]{\bf )} node[below=6pt]{$3$} -- (5,0)node[below=6pt]{$+\infty$};
			\foreach \i in {1,...,10}
			\draw ($(0,0)-(.2*\i,0)$) node[scale=.6]{/};
			\draw ($(0,0)$) node[scale=.6]{/};
			\foreach \i in {1,...,9}
			\draw ($(3,0)+(.2*\i,0)$) node[scale=.6]{/};
			\draw ($(3,0)$) node[scale=.6]{/};
			\end{tikzpicture}}
		\immini{Tập hợp $B=(0;+\infty)$ có biểu diễn là}{
			\begin{tikzpicture}[scale=1, font=\footnotesize, line join = round, line cap = round, >=stealth]
			\draw[thick,->] (-2,0)node[below=6pt]{$-\infty$}--(1,0) node[scale=1.5]{\bf (} node[below=6pt]{$0$}-- (5,0)node[below=6pt]{$+\infty$};
			\foreach \i in {1,...,15}
			\draw ($(1,0)-(.2*\i,0)$) node[scale=.6,rotate=60]{/};
			\draw ($(1,0)$) node[scale=.6,rotate=60]{/};
			%\foreach \i in {1,...,5}
			%\draw ($(4,0)+(.2*\i,0)$) node[scale=.6,rotate=60]{/};
			\end{tikzpicture}}
		\noindent Do đó $A \cup B=(-3;+\infty)$.
	}
\end{ex}

\begin{ex}%[0D1B4]
	Hãy xác định tập hợp $\left[-2;2\right]\setminus\left[1;2\right].$
	\choice
	{$\left(-2;1\right]$}
	{$\left(-2;1\right)$}
	{$\left[-2;1\right]$}
	{\True$\left[-2;1\right)$}
	\loigiai{
	}
\end{ex}

\begin{ex}%[0D1B2]
	Cho tập hợp $X=\left\{x\in \mathbb{R}\big|x^2+x+1=0\right\} $. Hãy chọn mệnh đề \textbf{đúng} trong các mệnh đề dưới đây.
	\choice
	{$X=\left\{-\dfrac{1}{2}+\dfrac{\sqrt{3}}{2}i \right\}$}
	{$X=\lbrace 0\rbrace$}
	{\True $X=\varnothing$}
	{$X=\lbrace \varnothing\rbrace$}
	\loigiai{
		Phương trình $x^2+x+1=0$ vô nghiệm trên tập số thực nên $X=\varnothing$.
	}
\end{ex}

\begin{ex}%[0D1B1]
	Xét $x \in \mathbb{R}$. Trong các mệnh đề sau, mệnh đề nào đúng?
	\choice
	{$``x\in [-4;1)\Leftrightarrow -4 < x <1 " $}
	{\True $``x\in [-4;1)\Leftrightarrow -4 \le x <1 "$}
	{$``x\in [-4;1)\Leftrightarrow -4 < x \le 1  "$}
	{$``x\in [-4;1)\Leftrightarrow -4 \le x \le 1 " $}
	\loigiai{
	}
\end{ex}

\begin{ex}%[0D1K2]
	Trong các mệnh đề sau, mệnh đề nào \textbf{sai}?
	\choice
	{$\left\{1; 3\right\}\subset [1; 3]$}
	{\True $[2; 5]=\left\{2; 3; 4; 5\right\}$}
	{$\varnothing \subset \mathbb{Q}$}
	{$\mathbb{N}\subset [0;+\infty)$}
	\loigiai{
	}
\end{ex}

\begin{ex}%[0D1K3]
	Biểu diễn trên trục số của tập hợp $ \left( 0;2\right) \cup \left[ {-1;1}\right)$ là hình nào?
	\def\dotEX{}
	\choice
	{\begin{tikzpicture}[xscale=0.5,thick,>=stealth']
		\draw[->](-5,0)->(6,0);
		\IntervalLR{-5}{-1}
		\IntervalGRF{}{}{(}{-1}
		\IntervalLR{2}{6}
		\IntervalGRF{)}{2}{}{}
		\end{tikzpicture}}
	{\True \begin{tikzpicture}[xscale=0.5,thick,>=stealth']
		\draw[->](-5,0)->(6,0);
		\IntervalLR{-5}{-1}
		\IntervalGRF{}{}{[}{-1}
		\IntervalLR{2}{6}
		\IntervalGRF{)}{2}{}{}
		\end{tikzpicture}}
	{\begin{tikzpicture}[xscale=0.5,thick,>=stealth']
		\draw[->](-5,0)->(6,0);
		\IntervalLR{-5}{-1}
		\IntervalGRF{}{}{[}{-1}
		\IntervalLR{2}{6}
		\IntervalGRF{]}{2}{}{}
		\end{tikzpicture}}
	{\begin{tikzpicture}[xscale=0.5,thick,>=stealth']
		\draw[->](-5,0)->(6,0);
		\IntervalLR{-5}{-1}
		\IntervalGRF{}{}{(}{-1}
		\IntervalLR{2}{6}
		\IntervalGRF{]}{2}{}{}
		\end{tikzpicture}}
	\loigiai{
		Ta có $ \left( 0;2\right) \cup \left[ {-1;1}\right)=[-1;2).$
	}
\end{ex}

\begin{ex}%[0D1K4]
	Cho các số thực $a$, $b$, $c$, $d$ và $a<b<c<d.$ Khẳng định nào sau đây đúng?
	\choice
	{$\left(a;c\right)\cap \left[b;d\right)=\left[b;c\right]$}
	{\True$\left(a;c\right)\cap \left(b;d\right)=\left(b;c\right)$}
	{$\left(a;c\right)\cup \left(b;d\right)=\left(b;d\right)$}
	{$\left(a;c\right)\cap \left(b;d\right)=\left[b;c\right)$}
	\loigiai{
	}
\end{ex}

\begin{ex}%[0D1K4]
	Cho hai tập hợp $M=\left[-4;7\right]$ và $N=\left(-\infty;-2\right)\cup\left(3;+\infty\right)$. Hãy xác định tập hợp $M\cap N.$
	\choice
	{$M\cap N=(-\infty;-2) \cup [3;+\infty)$}
	{$M\cap N=(-\infty;2] \cup (3;+\infty)$}
	{$M\cap N=[-4;2)\cup (3;7)$}
	{\True$M\cap N=[-4;-2)\cup(3;7]$}
	\loigiai{
	}
\end{ex}

\begin{ex}%[0D1B3]
	\immini[thm]{Cho các tập hợp $A,B,C$ được minh họa bằng biểu đồ Ven như hình bên. Phần tô màu xám trong hình là biểu diễn của tập hợp nào sau đây?
		\choice
		{$A\cap B\cap C$}
		{$\left(A\backslash C\right)\cup \left(A\backslash B\right)$}
		{$\left(A\cup B\right)\backslash C$}
		{\True $\left(A\cap B\right)\backslash C$}
	}{
		\begin{venndiagram3sets}[tikzoptions={scale=0.7,thick}]
			\fillACapBNotC
	\end{venndiagram3sets}}
\loigiai{
}
\end{ex}

\begin{ex}%[0D1G3]
	Cho các tập hợp $A=\left\{a,b\right\}$, $B=\left\{1,2,3\right\}$, $C=\left\{b,c\right\}$ và $D=\left\{2,3,4\right\}$, trong đó $a, b, c\in\mathbb{R}$ và $a$, $b$, $c\notin B,D$ là các phần tử khác nhau từng đôi một. Tìm tập hợp $E=\left(A\cup B\right)\cap\left(C\cup D\right)$.
	\choice
	{$E=\left\{a,b,c\right\}$}
	{$E=\left\{a,2,3\right\}$}
	{\True $E=\left\{b,2,3\right\}$}
	{$E=\left\{2,3,4\right\}$}
	\loigiai{
		Ta có $\heva{& A\cup B=\left\{a,b,1,2,3\right\}\\ & C\cup D=\left\{b,c,2,3,4\right\}}\Rightarrow E=\left(A\cup B\right)\cap\left(C\cup D\right)=\left\{b,2,3\right\}$
	}
\end{ex}
\centerline{\textbf{---HẾT---}}
\Closesolutionfile{ans}
