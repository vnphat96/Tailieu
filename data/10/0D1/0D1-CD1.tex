\section{MỆNH ĐỀ}

\subsection{TÓM TẮT LÝ THUYẾT}
\subsubsection{Mệnh đề, mệnh đề chứa biến}
\begin{itemize}
	\item [\iconMT] \indam{Mệnh đề:} Mệnh đề là một \indamm{ câu khẳng định} đúng hoặc sai. Một mệnh đề không thể vừa đúng hoặc vừa sai. Những mệnh đề liên quan đến toán học được gọi là mệnh đề toán học.
	\item [\iconMT] \indam{Mệnh đề chứa biến:} Mệnh đề chứa biến là một câu khẳng định chứa biến nhận giá trị trong một tập $X$ nào đó và với mỗi giá trị của biến thuộc $X$ ta được một mệnh đề.
	\begin{vidu} Xét $P\left( n \right):$ ``$n$ chia hết cho $5$'' với $n$ là số tự nhiên. Khẳng định này còn phụ thuộc vào biến $n$. Với $n=2$ ta được $P(2)$ là mệnh đề \textbf{sai}. Với $n=10$ ta được $P(10)$ là mệnh đề đúng.
	\end{vidu}
	\begin{vidu}$P\left( x;y \right)$ : ``$2x+y=5$'', với $x,y$ là số thực. Khẳng định này còn phụ thuộc hai biến $x,y$. Với $x=1$, $y=2$ ta được mệnh đề \textbf{sai}. Với $x=1$, $y=3$ ta được mệnh đề đúng.
	\end{vidu}
\end{itemize}
\subsubsection{Mệnh đề phủ định}
\begin{itemize}
	\item [\iconMT] Cho mệnh đề $P$. Mệnh đề \lq\lq không phải $P$ \rq\rq gọi là mệnh đề phủ định của $P$.
	\item [\iconMT] Chú ý: 
		\begin{boxkn}
	\begin{itemize}
		\item  Mệnh đề phủ định của $P$, kí hiệu là $\overline{P}$.
		\item  Nếu $P$ đúng thì $\overline{P}$ sai, nếu $P$ sai thì $\overline{P}$ đúng.
	\end{itemize}
	\end{boxkn}
\end{itemize}
\subsubsection{Mệnh đề kéo theo và mệnh đề đảo}
	Cho hai mệnh đề $P$ và $Q$.
		\begin{itemize}
			\item[\iconMT] \indam{Mệnh đề kéo theo:}
				\begin{boxkn}
			\begin{itemize}
				\item  Mệnh đề "Nếu $P$ thì $Q$" gọi là mệnh đề kéo theo, kí hiệu  $P\Rightarrow Q$.
				\item  Mệnh đề này chỉ sai khi $P$ đúng và $Q$ sai.
			\end{itemize}
					\end{boxkn}
				\begin{khung4}{Lưu ý}
					Xét định lý dạng $P\Rightarrow Q$. Khi đó, ta có thể phát biểu định lý này theo một trong 2 cách sau:
					\begin{itemize}
						\item[\ding{172}] $P$ là điều kiện đủ để có $Q$.
						\item[\ding{173}] $Q$ là điều kiện cần để có $P$.
					\end{itemize}
				\end{khung4}
			\item[\iconMT] \indam{Mệnh đề đảo:} Cho mệnh đề $P\Rightarrow Q$. Khi đó, $Q\Rightarrow P$ gọi là mệnh đề đảo của $P\Rightarrow Q$.
			\end{itemize}
\subsubsection{Mệnh đề tương đương}
\begin{itemize}
	\item [\iconMT] Cho hai mệnh đề $P$ và $Q$. Mệnh đề ``$P$ nếu và chỉ nếu $Q$''  gọi là hai mệnh đề tương đương.
	\item [\iconMT] Chú ý:
	\begin{boxkn}
	\begin{itemize}
		\item  Mệnh đề ``$P$ nếu và chỉ nếu $Q$'' được kí hiệu là $P\Leftrightarrow Q$.
		\item 	Mệnh đề $P\Leftrightarrow Q$ đúng khi cả $P\Rightarrow Q$ và $Q\Rightarrow P$ cùng đúng.
	\end{itemize}
	\end{boxkn}
\end{itemize}
			\begin{khung4}{Lưu ý}
				Xét định lý dạng $P \Leftrightarrow Q$. Khi đó, ta có thể phát biểu định lý này theo một trong 2 cách sau:
				\begin{itemize}
					\item[\ding{172}] $P$ là điều cần và đủ để có $Q$.
					\item[\ding{173}] $P$ khi và chỉ khi $Q$.
				\end{itemize}
			\end{khung4}
\subsubsection{Mệnh đề có chứa kí hiệu $\forall, \exists$}
	\begin{itemize}
		\item[\iconMT] Mệnh đề chứa kí hiệu với mọi: $\forall x \in X,\, P(x) $.
	\begin{boxkn}
		\begin{itemize}
			\item Mệnh đề này đúng khi tất cả các giá trị của $x \in X$ đều làm cho phát biểu $P(x)$ đúng.
			\item Nếu ta tìm được ít nhất một giá trị $x \in X$ làm cho $P(x)$ sai thì mệnh đề này \textbf{sai}.
		\end{itemize}
	\end{boxkn}
		\begin{vidu}
			Mệnh đề "Bình phương mọi số thực đều không âm" được viết là
			$\forall x \in \mathbb{R},\, x^2 \ge 0.$
		\end{vidu}
		\item[\iconMT] Mệnh đề chứa kí hiệu tồn tại: $\exists x \in X,\, P(x) $.
	\begin{boxkn}
		\begin{itemize}
			\item Mệnh đề này đúng khi ta tìm được ít nhất một giá trị của $x \in X$ làm cho phát biểu $P(x)$ đúng.
			\item Nếu tất cả giá trị của $x \in X$ đều làm cho $P(x)$ sai thì mệnh đề này \textbf{sai}.
		\end{itemize}
	\end{boxkn}
		\begin{vidu}
			Mệnh đề "Có một số tự nhiên mà bình phương của nó bằng 3" được viết là
			$\exists x \in \mathbb{N},\, x^2=3.$
		\end{vidu}
		\item[\iconMT] Phủ định của Mệnh đề chứa kí hiệu $\forall $, $\exists $.
	\begin{boxkn}
		\begin{itemize}
			\item Phủ định của mệnh đề ``$\forall x\in X,P\left( x \right)"$ là mệnh đề ``$\exists x\in X,\overline{P(x)}"$.
			\item Phủ định của mệnh đề ``$\exists x\in X,P\left( x \right)"$ là mệnh đề ``$\forall x\in X,\overline{P(x)}"$.
		\end{itemize}
	\end{boxkn}
	\end{itemize}

	
\subsection{RÈN LUYỆN KĨ NĂNG GIẢI TOÁN}
\begin{dang}{Mệnh đề, phủ định của mệnh đề}
	\begin{enumerate}[\iconMT]
		\item \indam{Mệnh đề:}
		\begin{listEX}[1]
			\item [\ding{172}] Khẳng định đúng là mệnh đề đúng, khẳng định sai là mệnh đề sai.
			\item [\ding{173}] Câu không phải là câu khẳng định hoặc câu khẳng định mà không có tính đúng-sai đều không phải là mệnh đề.
		\end{listEX}
		\item \indam{Mệnh đề phủ định:} Cho mệnh đề $P$. 
		\begin{listEX}[1]
			\item [\ding{172}] Mệnh đề phủ định của $P$, kí hiệu là $\overline{P}$.
			\item [\ding{173}] Nếu $P$ đúng thì $\overline{P}$ \textbf{sai}; $P$ \textbf{sai} thì $\overline{P}$ đúng.
		\end{listEX} 
	\end{enumerate}	
\end{dang}

\begin{vd}
	Các câu sau đây, câu nào là mệnh đề, câu nào không phải là mệnh đề? Nếu là mệnh đề hay cho biết mệnh đề đó đúng hay \textbf{sai}?
	\begin{listEX}[2]
		\item Không được đi lối này!
		\item Bây giờ là mấy giờ?
		\item $7$ không là số nguyên tố.
		\item $\sqrt{5}$ là số vô tỉ.
	\end{listEX}
	\loigiai{
		\begin{listEX}[2]
			\item Không phải là mệnh đề. Đây là câu cảm thán.
			\item Không phải mệnh đề. Đây là câu hỏi.
			\item Mệnh đề sai.
			\item Mệnh đề đúng
	\end{listEX}}
\end{vd}

\begin{vd}
	Xét tính đúng sai của các mệnh đề sau và lập mệnh đề phủ định của chúng.\begin{tasks}(1)
		\task $5$ là số nguyên tố.
		\task Phương trình $2x^2-3x+1=0$ có nghiệm nguyên.
		\task Tổng ba góc trong một tam giác bằng $180^\circ$.
		\task Ấn Độ có dân số lớn nhất thế giới.
	\end{tasks}
\end{vd}

\begin{dang}{Mệnh đề kéo theo, mệnh đề đảo, mệnh đề tương đương}
\end{dang}

\begin{vd}
	Cho tam giác $ABC$. Xét hai mệnh đề $P\colon$ \lq\lq Tam giác $ABC$ vuông\rq\rq \, và $Q\colon$\lq\lq Tam giác $ABC$ có $AB^2+AC^2=BC^2$ \rq\rq. Phát biểu các mệnh đề sau và cho biết mệnh đề sau đúng hay \textbf{sai}?
	\begin{listEX}[2]
		\item $P\Rightarrow Q$.
		\item $Q\Rightarrow P$.
	\end{listEX}
	\loigiai{
		\begin{listEX}[2]
			\item Mệnh đề $P\Rightarrow Q$ là ``Nếu tam giác $ABC$ vuông thì $AB^2+AC^2=BC^2$''.\\
			Mênh đề $P\Rightarrow Q$ sai vì chưa chắc tam giác đã vuông tại $A$.
			\item Mệnh đề $Q\Rightarrow P$ là ``Nếu tam giác $ABC$ có $AB^2+AC^2=BC^2$ thì tam giác vuông''.\\
			Mệnh đề $Q\Rightarrow P$ đúng (theo định lí Pitago).
	\end{listEX}}
\end{vd}

\begin{vd}
	Xét hai câu sau:
	\begin{itemize}
		\item [] $P \colon $ \lq\lq Phương trình bậc hai $ax^2+bx+c=0$ có hai nghiệm thực phân biệt\rq\rq.
		\item [] $Q \colon $ \lq\lq Phương trình bậc hai $ax^2+bx+c=0$ có biệt thức $\Delta = b^2-4ac>0$\rq\rq.
	\end{itemize}
\begin{tasks}(2)
	\task Phát biểu mệnh đề $P \Rightarrow Q$.
	\task Phát biểu mệnh đề $Q \Rightarrow P$.
\end{tasks}

\end{vd}
\begin{vd}
	Cho tam giác $ABC$ với trung tuyến $AM$. Xét hai mệnh đề	$P\colon $\lq\lq Tam giác $ABC$ vuông tại $A$\rq\rq \, và
	$Q\colon $\lq\lq Trung tuyến $AM$ bằng một nửa cạnh $BC$\rq\rq.
	\begin{enumerate}
		\item Hãy phát biểu mệnh đề $P\Rightarrow Q$. Mệnh đề này đúng hay sai?
		\item Hãy phát biểu mệnh đề $Q\Rightarrow P$. Mệnh đề này đúng hay sai?
		\item Phát biểu mệnh đề $P\Leftrightarrow Q$ và cho biết mệnh đề đó đúng hay sai?
	\end{enumerate}
	\loigiai{
		\begin{enumerate}
			\item Mệnh đề $P\Rightarrow Q$ là \lq\lq Nếu tam giác $ABC$ vuông tại $A$ thì trung tuyến $AM$ bằng một nửa cạnh $BC$\rq\rq.\\
			Đây là mệnh đề đúng.
			\item Mệnh đề $Q\Rightarrow P$ là \lq\lq Nếu trung tuyến $AM$ bằng một nửa cạnh $BC$ thì tam giác $ABC$ vuông tại $A$\rq\rq.\\
			Đây là mệnh đề đúng.
			\item Mệnh đề $P\Leftrightarrow Q$ là \lq\lq Tam giác $ABC$ vuông tại $A$ khi và chỉ khi trung tuyến $AM$ bằng một nửa cạnh $BC$\rq\rq.\\
			Mệnh đề tương đương $P\Leftrightarrow Q$ đúng vì $P\Rightarrow Q$ và $Q\Rightarrow P$ là hai mệnh đề đúng.
		\end{enumerate}
	}
\end{vd}

\begin{vd}
	Cho định lí \lq\lq Cho số tự nhiên $n$, nếu $n^5$ chia hết cho $5$ thì $n$ chia hết cho $5$\rq\rq. Định lí này được viết dưới dạng $P\Rightarrow Q$.
	\begin{enumerate}
		\item Hãy xác định các mệnh đề $P$ và $Q$.
		\item Phát biểu định lí trên bằng cách dùng thuật ngữ \lq\lq điều kiện cần\rq\rq.
		\item Phát biểu định lí trên bằng cách dùng thuật ngữ \lq\lq điều kiện đủ\rq\rq.
		Hãy phát biểu định lí đảo (nếu có) của định lí trên rồi dùng các thuật ngữ \lq\lq điều kiện cần và điều kiện đủ\rq\rq\text{} phát biểu gộp cả hai định lí thuận và đảo.
	\end{enumerate}
	\loigiai{
		\begin{enumerate}
			\item $P\colon $\lq\lq $n$ là số tự nhiên và $n^5$ chia hết cho $5$\rq\rq, $Q\colon $\lq\lq $n$ chia hết cho $5$\rq\rq.
			\item Với $n$ là số tự nhiên, $n$ chia hết cho $5$ là điều kiện cần để $n^5$ chia hết cho $5$.
			\item Với $n$ là số tự nhiên, $n^5$ chia hết cho $5$ là điều kiện đủ để $n$ chia hết cho $5$.
			\item 
			\begin{itemize}
				\item Định lí đảo \lq\lq Cho số tự nhiên $n$, nếu $n$ chia hết cho $5$ thì $n^5$ chia hết cho $5$\rq\rq.\\
				\item Phát biểu gộp cả hai định lí \lq\lq Điều kiện cần và đủ để $n$ chia hết cho $5$ là $n^5$ chia hết cho $5$\rq\rq.
			\end{itemize} 
		\end{enumerate}
	}
\end{vd}

\begin{vd}
	Cho hai mệnh đề $P:$ ``Tứ giác $ABCD$ là hình thoi'' và $Q:$ ``Tứ giác $ABCD$ là hình bình hành có hai đường chéo vuông góc với nhau''. Phát biểu định lý $P\Leftrightarrow Q$ bằng hai cách.
\end{vd}

\begin{vd}
	\begin{listEX}[1]
		\item []
		\item [a)] Phát biểu điều kiện cần và đủ để số tự nhiên $n$ chia hết cho $2$.
		\item [b)] Phát biểu điều kiện cần và đủ để số tự nhiên $n$ chia hết cho $5$.
		\item [c)] Phát biểu điều kiện cần và đủ để số tự nhiên $n$ chia hết cho $3$.
	\end{listEX}
\end{vd}
\begin{dang}{Mệnh đề chứa kí hiệu với mọi, tồn tại}
	\begin{enumerate}[\iconMT]
		\item \indam{Tính đúng sai:}
		\begin{listEX}[1]
			\item [\ding{172}]  Mệnh đề chứa kí hiệu với mọi: $\forall x \in X,\, P(x) $.
				\begin{itemize}
					\item Mệnh đề này đúng khi tất cả các giá trị của $x \in X$ đều làm cho phát biểu $P(x)$ đúng.
					\item Nếu ta tìm được ít nhất một giá trị $x \in X$ làm cho $P(x)$ sai thì mệnh đề này \textbf{sai}.
				\end{itemize}
			\item [\ding{173}] Mệnh đề chứa kí hiệu tồn tại: $\exists x \in X,\, P(x) $.
					\begin{itemize}
					\item Mệnh đề này đúng khi ta tìm được ít nhất một giá trị của $x \in X$ làm cho phát biểu $P(x)$ đúng.
					\item Nếu tất cả giá trị của $x \in X$ đều làm cho $P(x)$ sai thì mệnh đề này \textbf{sai}.
				\end{itemize}
		\end{listEX}
		\item \indam{Phủ định của mệnh đề có dấu $\forall,\exists $:}
		\begin{listEX}[1]
			\item [\ding{172}] $\forall x\in X,P(x)$ thành $\exists x\in X,\overline{P(x)}$.
			\item [\ding{173}] $\exists x\in X,P(x)$ thành $\forall x\in X,\overline{P(x)}$.
		\end{listEX}
	\indam{Chú ý:} \indamm{Khi lấy phủ định, ta chú ý các vấn đề đối lập sau:}
	\begin{listEX}[1]
		\item [\ding{172}] Quan hệ $=$ thành quan hệ $\ne $, và ngượclại.
		\item [\ding{173}] Quan hệ $>$ thành quan hệ $\le $, và ngược lại.
		\item [\ding{174}] Quan hệ $\ge $ thành quan hệ $<$, và ngược lại.
		\item [\ding{175}] Liên kết "và" thành liên kết "hoặc", và ngược lại.
	\end{listEX}
	\end{enumerate}
\end{dang}

\begin{vd}%[0D1Y1-5]
	Sử dụng kí hiệu \lq\lq$\forall$\rq\rq \,để viết mỗi mệnh đề sau và xét xem mệnh đề đó là đúng hay sai? Giải thích vì sao?
	\begin{enumerate}
		\item $P\colon$\lq\lq Với mọi số thực $x, x^2+1>0$\rq\rq.
		\item $Q\colon$\lq\lq Với mọi số tự nhiên $n, n^2+n$ chia hết cho $6$\rq\rq.
	\end{enumerate}
	\loigiai{
		\begin{enumerate}
			\item $P\colon$\lq\lq Với mọi số thực $x, x^2+1>0$\rq\rq.\\
			Mệnh đề được viết là $P \colon \lq\lq\forall x \in \mathbb{R}, x^2+1>0$\rq\rq.\\
			Xét một số thực $x$ tùy ý, ta phải chứng tỏ rằng $x^2+1>0$.\\
			Thật vậy, ta có $x^2+1 \geq 1>0$.\\
			Vậy mệnh đề $P$ là mệnh đề đúng.
			\item $Q\colon$\lq\lq Với mọi số tự nhiên $n, n^2+n$ chia hết cho $6$\rq\rq.\\
			Mệnh đề được viết là $Q\colon\lq\lq \forall n \in \mathbb{N},\left(n^2+n\right) \,\vdots\, 6$\rq\rq.\\
			Để chứng minh mệnh đề $Q$ là sai, ta cần chỉ ra một giá trị cụ thể của $n$ để nhận được mệnh đề sai.\\
			Thật vậy, chọn $n=1$, ta thấy $n^2+n=2$ không chia hết cho $6$.\\
			Vậy mệnh đề $Q$ là mệnh đề sai.
		\end{enumerate}
	}
\end{vd}

\begin{vd}%[0D1Y1-5]
	Sử dụng kí hiệu \lq\lq$\exists$\rq\rq\, để viết mỗi mệnh đề sau và xét xem mệnh đề đó là đúng hay sai? Giải thích vì sao?
	\begin{enumerate}
		\item $M\colon$\lq\lq Có ít nhất một số thực $x$ sao cho $x^3=-8$\rq\rq.
		\item $N\colon$\lq\lq Tồn tại số nguyên $x$ sao cho $2x+1=0$\rq\rq.
	\end{enumerate}
	\loigiai{
		\begin{enumerate}
			\item $M\colon$\lq\lq Có ít nhất một  số thực $x$ sao cho $x^3=-8$\rq\rq.\\
			Mệnh đề được viết là $M\colon\lq\lq\exists x \in \mathbb{R}, x^3=-8$\rq\rq.
			Để chứng tỏ mệnh đề $M$ là đúng, ta cần chỉ ra một giá trị cụ thể của $x$ để nhận được mệnh đề đúng.\\
			Thật vậy, chọn $x=-2$, ta thấy $(-2)^3=-8$.\\
			Vậy mệnh đề $M$ là mệnh đề đúng.\\
			Mệnh đề $N\colon\lq\lq\exists x \in \mathbb{Z}, 2x+1=0$\rq\rq.
			\item $N\colon$\lq\lq Tồn tại số nguyên $x$ sao cho $2x+1=0$\rq\rq.\\
			Để chứng minh mệnh đề $N$ là sai, ta phải chứng tỏ rằng với số nguyên $x$ tùy ý thì $2x+1 \neq 0$.\\
			Thật vậy, xét một số nguyên $x$ tùy ý, ta có $2x+1 \neq 0$.\\
			Vì thế mệnh đề $N$ là mệnh đề sai.
		\end{enumerate}
	}
\end{vd}

\begin{vd}
	Nêu mệnh đề phủ định của các mệnh đề sau và cho biết tính đúng sai của mệnh đề phủ định đó.
	\begin{listEX}[2]
		\item $A:``\exists n\in \mathbb{N},n^2+3$ chia hết cho $4$''.
		\item $B:``\exists x\in \mathbb{N}$, $x$ chia hết cho $x+1$''.
	\end{listEX}
	\loigiai{
		\begin{listEX}
			\item $\overline{A}:``\forall x\in \mathbb{N},n^2+3$ không chia hết cho $4$''. Mệnh đề này sai.
			\item $\overline{B}:``\forall x\in \mathbb{N},x$ không chia hết $x+1$''. Mệnh đề này sai.
		\end{listEX}
	}
\end{vd}

\begin{vd}
	Xét tính đúng sai của mệnh đề sau và nêu mệnh đề phủ định của nó.
	\begin{tasks}(2)
		\task $\exists x\in \mathbb{Z},x^2=3$.
		\task $\forall n\in {\mathbb{N}}^{*}:2^n+3$ là một số nguyên tố.
		\task $\forall x\in \mathbb{R},x^2+4x+5>0$.
		\task $\forall x\in \mathbb{R},x^4-x^2+2x+2\ge 0$.
	\end{tasks}
	\loigiai{
		\begin{listEX}
			\item Ta có $x^2=3\Leftrightarrow x=\pm \sqrt{3}$. Vì $\pm \sqrt{3}\notin \mathbb{Z}$ nên mệnh đề đã cho sai.\\
			Mệnh đề phủ định là $\overline{P(x)}:``\forall x\in \mathbb{Z},x^2\ne 3"$.
			\item Với $n= 5$ thì $2^n+3=2^5+3=35$, số này chia hết cho $5$ (không nguyên tố). Do đó mệnh đề đã cho sai.\\
			Mệnh đề phủ định là $\overline{P(n)}:``\exists n\in {\mathbb{N}}^{*}:2^n+3$ không là một số nguyên tố''.
			\item Mệnh đề đúng vì $x^2+4x+5={\left(x+2\right)}^2+1>0, \forall x\in \mathbb{R}$.\\
			Mệnh đề phủ định là $\overline{P(x)}:``\exists x\in \mathbb{R},x^2+4x+5\le 0"$.
			\item Do $x^4-x^2+2x+2={\left(x^2-1\right)}^2+{\left(x+1\right)}^2\ge 0, \forall x\in \mathbb{R}$ nên mệnh đề đã cho đúng.\\
			Mệnh đề phủ định là $\overline{P(x)}:``\exists x\in \mathbb{R},x^4-x^2+2x+2<0"$.
		\end{listEX}
	}
\end{vd}

\subsection{BÀI TẬP TỰ LUYỆN}
\begin{baitap}%[0D1Y1-2]
	Trong các mệnh đề toán học sau đây, mệnh đề nào là một khẳng định đúng? Mệnh đề nào là một khẳng định sai?
	\begin{itemize}
		\item [a)]$P\colon$\lq\lq Tổng hai góc đối của một tứ giác nội tiếp bằng $180^{\circ}$\rq\rq.
		\item [b)]$Q\colon$\lq\lq $7$ là số chính phương\rq\rq.
		\item [c)]$R\colon$\lq\lq $1$ là số nguyên tố\rq\rq.
	\end{itemize}
	\loigiai
	{
		Mệnh đề $P$ là mệnh đề đúng. \\
		Mệnh đề $Q$ và $R$ là mệnh đề sai. \\
	}
\end{baitap}

\begin{baitap}%[0D1Y1-2]
	Xét tính đúng sai của mỗi mệnh đề sau
	\begin{tasks}(2)
		\task $\pi<\dfrac{10}{3}$.
		\task Phương trình $3x+7=0$ có nghiệm.
		\task Tồn tại một số cộng với chính nó bằng $0$.
		\task $2022$ là hợp số.
	\end{tasks}

	\loigiai
	{
		\begin{enumerate}[a)]
			\item Mệnh đề \lq\lq $\pi<\dfrac{10}{3}$\rq\rq\ là mệnh đề đúng.
			\item Mệnh đề \lq\lq Phương trình $3x+7=0$ có nghiệm\rq\rq\ là mệnh đề đúng vì $3x+7=0 \Leftrightarrow x=-\dfrac{7}{3}$.
			\item Mệnh đề \lq\lq Tồn tại một số cộng với chính nó bằng $0$\rq\rq\ là mệnh đề đúng vì $0+0=0$.
			\item Mệnh đề \lq\lq $2022$ là hợp số\rq\rq\ là mệnh đề đúng vì $2022$ có ít nhất $3$ ước là $1$; $2$ và $2022$.
		\end{enumerate}
	}
\end{baitap}

\begin{baitap}%[0D1Y1-2]
	Xét tính đúng sai của mỗi mệnh đề sau
	\begin{tasks}(2)
		\task $1993$ chia hết cho $3$.
		\task $\sqrt{12}$ là một số hữu tỉ.
		\task $9$ là một số chính phương.
		\task $|-1997|\leqslant0$.
	\end{tasks}
	\loigiai
	{
		\begin{enumerate}[a)]
			\item Mệnh đề \lq\lq $1993$ chia hết cho $3$\rq\rq\ là mệnh đề sai vì $1993$ chia $3$ dư $1$.
			\item Mệnh đề \lq\lq $\sqrt{12}$ là một số hữu tỉ\rq\rq\ là mệnh đề sai vì $\sqrt{12}$ là một số vô tỉ.
			\item Mệnh đề \lq\lq $9$ là một số chính phương\rq\rq\ là mệnh đề đúng vì $\sqrt{9}=3$.
			\item Mệnh đề \lq\lq $|-1997|\leqslant0$\rq\rq\ là mệnh đề sai vì $|-1997|=1997>0$.
		\end{enumerate}
	}
\end{baitap}

\begin{baitap}%[0D1K1-4]
	Dùng thuật ngữ ``điều kiện cần'' để phát biểu các định lí sau.
	\begin{enumerate}
		\item Nếu $MA\perp MB$ thì $M$ thuộc đường tròn đường kính $AB$.
		\item $a\ne 0$ hoặc $b\ne 0$ là điều kiện đủ để $a^2+b^2>0$.
	\end{enumerate}
	\loigiai{
		\begin{enumerate}
			\item Điều kiện cần để $MA\perp MB$ là $M$ thuộc đường tròn đường kính $AB$.\\
			Hoặc: $M$ thuộc đường tròn đường kính $AB$ là điều kiện cần để $MA\perp ~MB$.
			\item $a^2+b^2>0$ là điều kiện cần để $a\ne 0$ hoặc $b\ne 0$.
		\end{enumerate}
	}
\end{baitap}
\begin{baitap}%[0D1K1-4]
	Sử dụng thuật ngữ ``điều kiện đủ'' để phát biểu các định lí sau.
	\begin{enumerate}
		\item Nếu $a$ và $b$ là hai số hữu tỉ thì tổng $a+b$ là số hữu tỉ.
		\item Nếu hai tam giác bằng nhau thì chúng có diện tích bằng nhau.
		\item Nếu một số tự nhiên có chữ số tận cùng là chữ số 5 thì nó chia hết cho $5$.
	\end{enumerate}
	\loigiai{
		\begin{enumerate}
			\item Điều kiện đủ để tổng $a+b$ là số hữu tỉ là cả hai số $a$ và $b$ đều là số hữu tỉ.\\
			Hoặc: $a$ và $b$ là hai số hữu tỉ là điều kiện đủ để tổng $a+b$ là số hữu tỉ.
			\item Điều kiện đủ để hai tam giác có diện tích bằng nhau là chúng bằng nhau.\\
			Hoặc: hai tam giác bằng nhau là điều kiện đủ để chúng có diện tích bằng nhau.
			\item Điều kiện đủ để một số chia hết cho 5 là số đó tận cùng bằng 5.\\
			Hoặc: một số tự nhiên có chữ số tận cùng là chữ số 5 là điều kiện đủ để nó chia hết cho 5.
			
		\end{enumerate}
	}
\end{baitap}

\begin{baitap}%[0D1B1-5]
	Nêu mệnh đề phủ định của các mệnh đề sau và cho biết tính đúng sai của mệnh đề phủ định đó.
	\begin{enumerate}
		\item $A\colon$ ``$\forall x\in \mathbb{R},x^3-x^2+1>0$''.
		\item $B\colon$ ``Tồn tại số thực $a$ sao cho $a+1+\dfrac{1}{a+1}\leqslant 2$''.
	\end{enumerate}
	\loigiai{
		\begin{enumerate}
			\item $\overline{A}\colon $ ``$\exists x\in \mathbb{R},x^3-x^2+1\leqslant 0$''. Mệnh đề này đúng vì chẳng hạn $x=-1$, ta có $(-1)^3-(-1)^2+1=-1<0$.
			\item $\overline{B}\colon $ ``Với mọi số thực $a$ thì $a+1+\dfrac{1}{a+1}>2$''. Mệnh đề này sai chẳng hạn khi $a=-2$.
		\end{enumerate}
	}
\end{baitap}
\begin{baitap}%[Nguyễn Cường- BG Toán 10]%[0D1B1-5]
	Lập mệnh đề phủ định của mỗi mệnh đề sau
	\begin{tasks}(2)
		\task $\forall x \in \mathbb{R},|x| \geq x$.
		\task $\exists x \in \mathbb{R}, x^2+1=0$.
	\end{tasks}
	\loigiai{
		\begin{enumerate}[a)]
			\item Phủ định của mệnh đề \lq\lq$\forall x \in \mathbb{R},|x| \geq x$\rq\rq\,là mệnh đề \lq\lq$\exists x \in \mathbb{R},|x|<x$\rq\rq.
			\item Phủ định của mệnh đề \lq\lq$\exists x \in \mathbb{R}, x^2+1=0$\rq\rq\,là mệnh đề \lq\lq$\forall x \in \mathbb{R}, x^2+1 \neq 0$\rq\rq.
		\end{enumerate}
	}
\end{baitap}

\begin{baitap}%[Nguyễn Cường- BG Toán 10]%[0D1B1-5]
	Lập mệnh đề phủ định của mỗi mệnh đề sau và xét tính đúng sai của mỗi mệnh đề phủ định đó
	\begin{tasks}(2)
		\task $\forall x \in \mathbb{R}, x^2 \neq 2x-2$.
		\task $\forall x \in \mathbb{R}, x^2 \leq 2x-1$.
		\task $\exists x \in \mathbb{R}, x+\dfrac{1}{x} \geq 2$.
		\task $\exists x \in \mathbb{R}, x^2-x+1<0$.
	\end{tasks}
	\loigiai{
		\begin{enumerate}[a)]
			\item $\exists x \in \mathbb{R}, x^2=2x-2$.\\
			Mệnh đề này sai vì phương trình $x^2-2x+2=0$ vô nghiệm trên tập số thực.
			\item $\exists x \in \mathbb{R}, x^2 > 2x-1$.\\
			Mệnh đề này đúng vì với $x=2$ thì $2^2>2\cdot 2-1$.
			\item $\forall x \in \mathbb{R}, x+\dfrac{1}{x}<2$.\\
			Mệnh đề này sai vì với $x=1$ thì $1+\dfrac{1}{1}=2$.
			\item $\forall x \in \mathbb{R}, x^2-x+1\ge 0$.\\
			Mệnh đề này đúng vì $x^2-x+1=\left(x-\dfrac{1}{2}\right)^2+\dfrac{3}{4}> 0$ với mọi $x\in\mathbb{R}$.
		\end{enumerate}	
	}
\end{baitap}
