\subsection{BÀI TẬP TRẮC NGHIỆM}
\Opensolutionfile{ans}[ans/ansTL-CD1]

\begin{ex}
	Trong các câu sau, câu nào là mệnh đề ?
	\choice
	{Các bạn hãy làm bài đi!}
	{Các bạn có chăm học không ?}
	{An học lớp mấy ?}
	{\True Việt Nam là một nước thuộc Châu Á}
	\loigiai{
		Chú ý:
		\begin{itemize}
			\item [$\bullet$] Mệnh đề là một câu khẳng định hoặc đúng hoặc sai.
			\item [$\bullet$] Mệnh đề không thể là các câu hỏi, câu cảm thán.
		\end{itemize}	
	Suy ra, khẳng định "Việt Nam là một nước thuộc Châu Á" là một mệnh đề và là mệnh đề đúng.
	}
\end{ex}

\begin{ex}
	Trong các câu sau, câu nào là mệnh đề ?
	\choice
	{\True 15 là số nguyên tố}
	{$a+b=c$}
	{$x^2+x=0$}
	{$2n+1$ chia hết cho $3$}
	\loigiai{
		Chú ý:
		\begin{itemize}
			\item [$\bullet$] Mệnh đề là một câu khẳng định hoặc đúng hoặc sai.
			\item [$\bullet$] Mệnh đề không thể là các câu hỏi, câu cảm thán.
		\end{itemize}
	Suy ra, khẳng định "15 là số nguyên tố" là một mệnh đề và là mệnh đề sai.
	}
\end{ex}

\begin{ex}
	Trong các câu sau, câu nào \textbf{không }phải là mệnh đề?
	\choice
	{$5+2=8$}
	{$2>0$}
	{$4-\sqrt{17}>0$}
	{\True $5+x=2$}
	\loigiai{
		Chú ý:
		\begin{itemize}
			\item [$\bullet$] Mệnh đề là một câu khẳng định hoặc đúng hoặc sai.
			\item [$\bullet$] Mệnh đề không thể là các câu hỏi, câu cảm thán.
		\end{itemize}
		Suy ra, "$5+x=2$' không là một mệnh đề vì ta chưa biết tính đúng sai của nó.
	}
\end{ex}

\begin{ex}
	Câu nào sau đây là một mệnh đề?
	\choice
	{Số 150 có phải là số chẵn không?}
	{\True Số 30 là số chẵn}
	{$2x-1$ là số lẻ}
	{$x^3+1=0$}
	\loigiai{
		Chú ý:
	\begin{itemize}
		\item [$\bullet$] Mệnh đề là một câu khẳng định hoặc đúng hoặc sai.
		\item [$\bullet$] Mệnh đề không thể là các câu hỏi, câu cảm thán.
	\end{itemize}
	Suy ra, "Số 30 là số chẵn" là một mệnh đề và là mệnh đề đúng.	
	}
\end{ex}

\begin{ex}
	Định lý có dạng $A\Rightarrow B$ được hiểu như thế nào?
	\choice
	{A khi và chỉ khi B}
	{B suy ra A}
	{A là điều kiện cần để có B}
	{\True A là điều kiện đủ để có B}
	\loigiai{
	Với định lý dạng $A\Rightarrow B$ thì
	\begin{itemize}
		\item [$\bullet$] A là điều kiện đủ để có B;
		\item [$\bullet$] B là điều kiện cần để có A.
	\end{itemize}
	}
\end{ex}

\begin{ex}
	Phủ định của mệnh đề "$5+4=10$" là mệnh đề nào sau đây ?
	\choice
	{$5+4<10$}
	{$5+4>10$}
	{$5+4\le 10$}
	{\True $5+4\ne 10$}
	\loigiai{
	Phủ định của "$5+4=10$" là "$5+4 \ne 10$".
	}
\end{ex}

\begin{ex}
	Phủ định của mệnh đề “$5+\pi >10$” là mệnh đề nào sau đây ?
	\choice
	{$5+\pi <10$}
	{$5+\pi >10$}
	{\True $5+\pi \le 10$}
	{$5+\pi \ne 10$}
	\loigiai{
		Phủ định của “$5+\pi >10$” là “$5+\pi \le 10$”.
	}
\end{ex}

\begin{ex}
	Phủ định của mệnh đề “$14$ là số nguyên tố” là mệnh đề nào sau đây?
	\choice
	{\True $14$ không phải là số nguyên tố}
	{$14$ chia hết cho $2$}
	{$14$ không phải là hợp số}
	{$14$ chia hết cho $7$}
	\loigiai{
	Phủ định của “$14$ là số nguyên tố” là “$14$ không là số nguyên tố”.
	}
\end{ex}

\begin{ex}
	Phủ định của mệnh đề “Dơi là một loài chim” là mệnh đề nào sau đây?
	\choice
	{Dơi là một loài có cánh}
	{Chim cùng loài với dơi}
	{Dơi là một loài ăn trái cây}
	{\True Dơi không phải là loài chim}
	\loigiai{
		Phủ định của “Dơi là một loài chim” là “Dơi không phải là một loài chim”.
	}
\end{ex}

\begin{ex}
	Trong các mệnh đề sau, mệnh đề nào là mệnh đề \textbf{sai}?
	\choice
	{$20$ chia hết cho $5$}
	{\True $5$ chia hết cho $20$}
	{$20$ là bội số của $5$}
	{$5$ là ước số của $20$}
	\loigiai{
	}
\end{ex}

\begin{ex}
	Cho mệnh đề chứa biến $P\left(x\right):x^2-3x+2=0$, với $x\in \mathbb{R}$. Tìm mệnh đề đúng trong các mệnh đề sau đây
	\choice
	{$P\left(0\right)$}
	{\True $P\left(1\right)$}
	{$P\left(-1\right)$}
	{$P\left(-2\right)$}
	\loigiai{
		Phát biểu đang là mệnh đề chứa biến. Ta kiểm tra biến đó với lần lượt các giá trị đã cho ở các phương án.\\
		Nhận xét với $x=1$ thì $1^2 - 3 \cdot 1 + 2=0$ đúng, nên $P(1)$ đúng.
	}
\end{ex}

\begin{ex}
	Với giá trị nào của $n\in \mathbb{N}$, mệnh đề chứa biến $P\left(n\right)$: "$n$ chia hết cho $12$" là đúng?
	\choice
	{\True $n=48$}
	{$n=4$}
	{$n=3$}
	{$n=88$}
	\loigiai{
		Phát biểu đang là mệnh đề chứa biến. Ta kiểm tra biến đó với lần lượt các giá trị đã cho ở các phương án.\\
		Nhận xét với $n=48$ thì $48$ chia hết cho $12$ đúng, nên mệnh đề đúng với $n=48$.
	}
\end{ex}

\begin{ex}
	Cho mệnh đề chứa biến $P\left(x\right)$: "$\sqrt{x}>x$", với $x\in \mathbb{R}$. Tìm mệnh đề đúng.
	\choice
	{$P\left(0\right)$}
	{$P\left(1\right)$}
	{\True $P\left(\dfrac{1}{2}\right)$}
	{$P\left(2\right)$}
	\loigiai{
	Phát biểu đang là mệnh đề chứa biến. Ta kiểm tra biến đó với lần lượt các giá trị đã cho ở các phương án.\\
	Nhận xét với $x=\dfrac{1}{2}$ thì $\sqrt{\dfrac{1}{2}}>\dfrac{1}{2}$ đúng, nên $P\left(\dfrac{1}{2}\right)$ đúng.
	}
\end{ex}

\begin{ex}
	Xét mệnh đề chứa biến $P\left(x\right)$ "$x^2-3x=0$", với $x\in \mathbb{R}$. Với giá trị nào của $x$ thì $P\left(x\right)$ là mệnh đề đúng?
	\choice
	{\True $x=0$}
	{$x=2$}
	{$x=-1$}
	{$x=-3$}
	\loigiai{
		Phát biểu đang là mệnh đề chứa biến. Ta kiểm tra biến đó với lần lượt các giá trị đã cho ở các phương án.\\
		Nhận xét với $x=0$ thì $0^2-3 \cdot 0 = 0$ đúng, nên mệnh đề đúng với $x=0$.
	}
\end{ex}

\begin{ex}
	Trong các mệnh đề sau, mệnh đề nào là mệnh đề đúng?
	\choice
	{Nếu “$33$ là hợp số” thì “$15$ chia hết cho $25$”}
	{Nếu “$7$ là số nguyên tố” thì “$8$ là bội số của $3$”}
	{\True Nếu “$20$ là hợp số” thì “$24$ chia hết cho $6$ ”}
	{Nếu “$3+9=12$” thì “$4>7$”}
	\loigiai{
		Mệnh đề $P \Rightarrow Q$ chỉ sai khi $P$ đúng và $Q$ sai.\\
		Xét mệnh đề "Nếu $20$ là hợp số thì $24$ chia hết cho $6$", có
		\begin{itemize}
			\item [$\bullet$] $P \colon$ "$20$ là hợp số" đúng.
			\item [$\bullet$] $Q \colon$ "$24$ chia hết cho $6$" đúng.
		\end{itemize}
		nên $P \Rightarrow Q$ đúng.	
	}
\end{ex}

\begin{ex}
	Trong các phát biểu sau phát biểu nào là mệnh đề đúng?
	\choice
	{$\pi $ là số hữu tỉ}
	{\True Tổng hai cạnh của một tam giác lớn hơn cạnh còn lại}
	{Bạn có chăm học không ?}
	{Số 12 không chia hết cho 3}
	\loigiai{
	}
\end{ex}

\begin{ex}
	Trong các mệnh đề sau, mệnh đề nào có mệnh đề đảo \textbf{sai}?
	\choice
	{“Tứ giác là hình bình hành thì có hai cặp cạnh đối song song và bằng nhau”}
	{“Tam giác đều thì có ba góc có số đo bằng $60^{\circ}$ ”}
	{\True “Hai tam giác bằng nhau thì có diện tích bằng nhau”}
	{“Một tứ giác có $4$ góc vuông thì tứ giác đó là hình chữ nhật”}
	\loigiai{
	}
\end{ex}

\begin{ex}
	Mệnh đề "$\exists x\in \mathbb{R}:\text{}{x^2}=3$" khẳng định rằng
	\choice
	{Bình phương của mỗi số thực bằng 3}
	{\True Có ít nhất một số thực mà bình phương của nó bằng 3}
	{Chỉ có một số thực bình phương bằng 3}
	{Nếu $x$ là số thực thì $x^2=3$}
	\loigiai{
	}
\end{ex}

\begin{ex}
	Kí hiệu $X$ là tập hợp các cầu thủ $x$ trong đội bóng rổ, $P\left(x\right)$ là mệnh đề chứa biến  $x$ cao trên 180 cm. Mệnh đề "$\forall x\in X,\text{}P(x)$" khẳng định rằng 
	\choice
	{\True Mọi cầu thủ trong đội tuyển bóng rổ đều cao trên 180cm}
	{Trong số các cầu thủ của đội tuyển bóng rổ có một cầu thủ cao trên 180cm}
	{Bất cứ ai cao trên 180cm đề là cầu thủ của đội tuyển bóng rổ}
	{Có một số người cao trên 180cm là cầu thủ của đội tuyển bóng rổ}
	\loigiai{
	}
\end{ex}

\begin{ex}
	Mệnh đề “Mọi động vật đều di chuyển” có mệnh đề phủ định là
	\choice
	{Mọi động vật đều không di chuyển}
	{Mọi động vật đều đứng yên}
	{Có ít nhất một động vật di chuyển}
	{\True Có ít nhất một động vật không di chuyển}
	\loigiai{
	}
\end{ex}

\begin{ex}
	Phủ định của mệnh đề “Có ít nhất một số vô tỷ là số thập phân vô hạn tuần hoàn” là mệnh đề nào sau đây?
	\choice
	{Mọi số vô tỷ đều là số thập phân vô hạn tuần hoàn}
	{Có ít nhất một số vô tỷ là số thập phân vô hạn không tuần hoàn}
	{\True Mọi số vô tỷ đều không phải là số thập phân vô hạn tuần hoàn}
	{Mọi số vô tỷ đều là số thập phân tuần hoàn}
	\loigiai{
	}
\end{ex}

\begin{ex}
	Tìm mệnh đề phủ định của mệnh đề $P$: “$\forall x\in \mathbb{N},\,x^2+x-1>0"$.
	\choice
	{$\overline{P}$: “$\exists x\in \mathbb{N},\,x^2+x-1>0"$}
	{$\overline{P}$: “$\forall x\in \mathbb{N},\,x^2+x-1>0"$}
	{\True $\overline{P}$: “$\exists x\in \mathbb{N},\,x^2+x-1\le 0"$}
	{$\overline{P}$: “$\forall x\in \mathbb{N},\,x^2+x-1\le 0"$}
	\loigiai{
	}
\end{ex}

\begin{ex}
	Xét mệnh đề $P$: "$\exists x\in \mathbb{R}:2x-3<0$". Mệnh đề phủ định của mệnh đề $P$ là
	\choice
	{“$\forall x\in \mathbb{R}:2x-3\le 0$”}
	{“$\exists x\in \mathbb{R}:2x-3>0$”}
	{\True “$\forall x\in \mathbb{R}:2x-3\ge 0$”}
	{“$\forall x\in \mathbb{R}:2x-3\le 0$”}
	\loigiai{
	}
\end{ex}

\begin{ex}
	Cho mệnh đề $\forall x\in \mathbb{R}:x^2+x>0$. Phủ định của mệnh đề này là
	\choice
	{$\forall x\in \mathbb{R},\text{}x^2+x\le 0$}
	{$\exists x\in \mathbb{R},\text{}{x^2}+x=0$}
	{$\exists x\in \mathbb{R}\text{,}x^2+x<0$}
	{\True $\exists x\in \mathbb{R},\text{}{x^2}+x\le 0$}
	\loigiai{
	}
\end{ex}

\begin{ex}
	Tìm mệnh đề \textbf{sai.}
	\choice
	{$\forall x\in \mathbb{R},\text{}x^2+2x+3>0$}
	{\True $\forall x\in \mathbb{R},\text{}{x^2}\ge x$}
	{$\exists x\in \mathbb{R},\text{}x^2+5x+6=0$}
	{$\exists x\in \mathbb{R},\text{}x<\dfrac{1}{x}$}
	\loigiai{
		Chú ý:
		\begin{itemize}
			\item [$\bullet$] Mệnh đề $\forall x\in X,\text{} P(x)$ đúng khi tất cả các giá trị của $x \in X$ đều làm cho $P(x)$ đúng.
			\item [$\bullet$] Mệnh đề $\forall x\in X,\text{} P(x)$ sai khi tồn tại ít nhất một giá trị $x_0 \in X$ mà $P(x_0)$ sai.
		\end{itemize}
	 Xét mệnh đề $\forall x\in \mathbb{R},\text{}{x^2}\ge x$. Thử với $x=0.5$ ta thấy $0.5^2 \ge 0.5$ sai, nên $\forall x\in \mathbb{R},\text{}{x^2}\ge x$ sai.
	}
\end{ex}

\begin{ex}
	Tìm mệnh đề đúng.
	\choice
	{$\exists x\in \mathbb{R},\text{}x^2+3=0$}
	{$\exists x\in \mathbb{R},\text{}{x^4}+3x^2+2=0$}
	{\True $\forall x\in \mathbb{N},\text{}{\left(2x+1\right)^2}-1$ chia hết cho 4}
	{$\forall x\in \mathbb{Z},\text{}x^5>x^2$}
	\loigiai{
		Xét $(2x+1)^2-1=4x^2+4x=4(x^2+x)$ chia hết cho 4, $\forall x \in \mathbb{N}$. Suy ra 
	"$\forall x\in \mathbb{N},\text{}{\left(2x+1\right)^2}-1$ chia hết cho 4" là mệnh đề đúng.
	}
\end{ex}

\begin{ex}
	Mệnh đề nào sau đây \textbf{sai}?
	\choice
	{$\forall n\in \mathbb{N},\text{}n\le 2n$}
	{\True $\forall x\in \mathbb{R},\text{}x^2>0$}
	{$\exists n\in \mathbb{N},\text{}{n^2}=n$}
	{$\exists x\in \mathbb{R},\text{}x>x^2$}
	\loigiai{
			Kiểm tra với $x=0$ ta thấy $0^2>0$ là sai. Vậy mệnh đề $\forall x\in \mathbb{R},\text{}x^2>0$ là mệnh đề sai.
	}
\end{ex}

\begin{ex}
	Cho các mệnh đề
	\begin{listEX}[2]
		\item [\ding{172}] $X$: “$\forall x\in \mathbb{R},\,x^2-2x+3>0$”
		\item [\ding{173}]  $Y$: “$\exists x\in \mathbb{R},\,x^2-4=0$”
		\item [\ding{174}] $P$: “$\exists x\in \mathbb{R},\,x^2+2=0$”
		\item [\ding{175}] $Q$: “$\forall x\in \mathbb{R},\,x>0$”
	\end{listEX}
	Các mệnh đề đúng là
	\choice
	{X, P}
	{Y, Q}
	{\True X, Y}
	{P, Q}
	\loigiai{
	Xét từng phương án
	\begin{itemize}
		\item [$\bullet$] $x^2-2x+3=(x-1)^2+2>0$ luôn đúng. Suy ra $X$ đúng.
		\item [$\bullet$] $x^2-4=0 \Leftrightarrow x = \pm 2 \in \mathbb{R}$. Suy ra $Y$ đúng.
		\item [$\bullet$] $x^2+2=0$ vô nghiệm trên $\mathbb{R}$ nên $P$ sai.
		\item [$\bullet$] $x=0$ thì $0>0$ là sai nên $Q$ sai.
	\end{itemize}
Vậy chỉ có $X$, $Y$ là mệnh đề đúng.
	}
\end{ex}

\begin{ex}
	Trong các mệnh đề sau mệnh đề nào đúng ?
	\choice
	{$\exists n\in \mathbb{N}$, $n^3-n$ không chia hết cho $3$}
	{$\forall x\in \mathbb{R}$, $x<3\Rightarrow x^2<9$}
	{$\exists m\in \mathbb{Z}$, $m^2+m+1$ là một số chẵn}
	{\True $\forall x\in \mathbb{Z}$, $\dfrac{2x^3-6x^2+x-3}{2{x^2}+1}\in \mathbb{Z}$}
	\loigiai{
		Xét từng phương án
		\begin{itemize}
			\item [$\bullet$] $n^3-n=(n-1)n(n+1)$ luôn chia hết cho 3 nên mệnh đề "$\exists n\in \mathbb{N}$, $n^3-n$ không chia hết cho $3$" là mệnh đề sai.
			\item [$\bullet$] Với $x=-5<3$ nhưng $x^2=(-5)^2=25>9$ nên  mệnh đề "$\forall x\in \mathbb{R}$, $x<3\Rightarrow x^2<9$" là mệnh đề sai.
			\item [$\bullet$] $m^2+m+1=m(m+1)+1$. Do $m(m+1)$ là tích hai số liên tiếp nên luôn là số chẵn, suy ra $m(m+1)+1$ là số lẻ. Suy ra mệnh đề "$\exists m\in \mathbb{Z}$, $m^2+m+1$ là một số chẵn" là mệnh đề sai.
			\item [$\bullet$] Ta có $\dfrac{2x^3-6x^2+x-3}{2{x^2}+1}=x^2-4x-3$ (chia đa thức). Với $x \in \mathbb{Z}$ thì $x^2-4x-3$ luôn là số nguyên nên mệnh đề "$\forall x\in \mathbb{Z}$, $\dfrac{2x^3-6x^2+x-3}{2{x^2}+1}\in \mathbb{Z}$" là mệnh đề đúng.
		\end{itemize}
	}
\end{ex}


\centerline{\textbf{---HẾT---}}
\Closesolutionfile{ans}