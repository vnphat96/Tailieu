\section*{ÔN TẬP KIỂM TRA CUỐI KÌ 2 - ĐỀ 01}
\setcounter{ex}{0}\setcounter{bt}{0}
\noindent{\bf\fontfamily{qag}\selectfont\color{violet}A. PHẦN TRẮC NGHIỆM}
\Opensolutionfile{ans}[ans/ansBTTeXCK2]
%Câu 1
\begin{ex}
 Một người vào cửa hàng ăn, người đó chọn thực đơn gồm $1$ món ăn trong 5 món, $1$ loại quả tráng miệng trong 5 loại quả tráng miệng và một loại nước uống trong 3 loại nước uống. Có bao nhiêu cách chọn thực đơn?
\choice
{$25$}
{\True $75$}
{$100$}
{$15$}
\loigiai{
Chọn $1$ món ăn trong $5$ món: có $5$ cách.\\
Chọn $1$ loại quả tráng miệng trong $5$ loại quả tráng miệng: có $5$ cách.\\
Chọn $1$ nước uống trong $3$ loại nước uống: có $5$ cách.\\
Theo qui tắc nhân, có $5 \cdot 5 \cdot 3=75$ cách
}
\end{ex}
%Câu 2
\begin{ex}
 Từ các chữ số $2$, $3$, $4$, $5$ có thể lập được bao nhiêu số tự nhiên có $4$ chữ số khác nhau?
\choice
{$256$}
{$120$}
{\True $24$}
{$16$}
\loigiai{
Mỗi số cần lập là một hoán vị của $4$ phần tử $\Rightarrow $ có $4!=24$ số cần lập
}
\end{ex}
%Câu 3
\begin{ex}
 Số tập hợp con có $3$ phần tử của một tập hợp có $7$ phần tử là
\choice
{\True $C_7^3$}
{$A_7^3$}
{$3!$}
{$7!$}
\loigiai{
Số tập hợp con có $3$ phần tử của một tập hợp có $7$ phần tử là $C_7^3$.}
\end{ex}
%Câu 4
\begin{ex}
 Số cách chọn một ban chấp hành gồm một trưởng ban, một phó ban, một thư kí và một thủ quỹ được chọn từ $16$ thành viên là
\choice
{$4$}
{$\dfrac{16!}{4}$ }
{$\dfrac{16!}{12!4!}$ }
{\True $\dfrac{16!}{12!}$ }
\loigiai{
Mỗi cách chọn một ban chấp hành là một chỉnh hợp chập 4 của $16$ nên có $A_{16}^4=\dfrac{16!}{12!}$ cách.}
\end{ex}
%Câu 5
\begin{ex}
 Trong khai triển nhị thức $(a+2)^{n+6}(n\in \mathbb{N})$ có tất cả $17$ số hạng. Giá trị của $n$ là 
\choice
{$17$}
{$11$}
{\True $10$}
{$12$}
\loigiai{
Ta có: $n+6+1=17\Leftrightarrow n=10$
}
\end{ex}
%Câu 6
\begin{ex}
 Kết quả đo chiều dài của một cây cầu được ghi là $152m\pm 0{,}2m$. Tìm sai số tương đối của phép đo chiều dài cây cầu.
\choice
{\True $\delta _a<0{,}1316\%$}
{$\delta _a<1{,}316\%$}
{$\delta _a=0{,}1316\%$}
{$\delta _a>0{,}1316\%$}
\loigiai{
Sai số tương đối $\delta _a\le \dfrac{0{,}2}{152}=0{,}001315789\approx 0{,}1316\%$
}
\end{ex}
%Câu 7
\begin{ex}
 Thời gian chạy 50m của 20 học sinh được ghi lại trong bảng dưới đây:
\begin{center}
\begin{tabular}{|c|c|c|c|c|c|}
\hline
Thời gian (giây) & $8{,}3$ & $8{,}4$ & $8{,}5$ & $8{,}7$& $8{,}8$ \\
\hline
Tần số & $2$ & $3$ & $9$ & $5$ & $1$\\
\hline
\end{tabular}
\end{center}
Số trung bình cộng thời gian chạy của học sinh là:
\choice
{$8{,}54$}
{$4$}
{$8{,}50$}
{\True $8{,}53$}
\loigiai{
$\overline{x}=\dfrac{\left(8{,}3 \cdot 2+8{,}4 \cdot 3+8{,}5 \cdot 9+8{,}7 \cdot 5+8{,}8 \cdot 1\right)}{20}=8{,}53$
}
\end{ex}
%Câu 8
\begin{ex}
 Điểm kiểm tra của 24 học sinh được ghi lại trong bảng sau:
\begin{center}
\begin{tabular}{|c|c|c|c|c|c|}
\hline
$7$ & $2$ & $3$ & $5$ & $8$ & $2$\\ \hline $8$ & $5$ & $8$ & $4$ & $9$ & $6$\\ \hline $6$ & $1$ & $9$ & $3$ & $6$ & $7$\\ \hline $3$ & $6$ & $6$ & $7$ & $2$ & $9$\\ \hline
\end{tabular}
\end{center}
Tìm mốt của điểm điều tra
\choice
{2}
{7}
{\True 6}
{9}
\loigiai{
Ta có bảng phân bố tần số:
\begin{center}
\begin{tabular}{|c|c|c|c|c|c|c|c|c|c|c|}
\hline Điểm & $1$ & $2$ & $3$ & $4$ & $5$ & $6$ & $7$ & $8$ & $9$ & \\ \hline Tần số & $1$ & $3$ & $3$ & $1$ & $2$ & $5$ & $3$ & $3$ & $3$ & $N=24$\\ \hline
\end{tabular}
\end{center}
Ta thấy điểm 6 có tần số lớn nhất nên $M_0=6$.
}
\end{ex}
%Câu 9
\begin{ex}
 Cho bảng phân bố tần số khối lượng 30 quả trứng gà của một rổ trứng gà:
Khối lượng (g)	Tần số
25	3
30	5
35	10
40	6
45	4
50	2
Cộng	30
Số trung vị là 
\choice
{37{,}5}
{40}
{\True 35}
{75}
\loigiai{
Ta thấy N chẵn nên số trung vị là:$M_e=\dfrac{35+35}{2}=35$
}
\end{ex}
%Câu 10
\begin{ex}
Gieo một đồng tiền cân đối và đồng chất bốn lần. Xác suất để cả bốn lần xuất hiện mặt sấp là?
\choice
{$\dfrac{4}{16}$}
{$\dfrac{2}{16}$}
{\True $\dfrac{1}{16}$}
{$\dfrac{6}{16}$}
\loigiai{
Số phần tử của không gian mẫu là $n\left(\Omega\right)=2 \cdot 2 \cdot 2 \cdot 2=16$.\\
Gọi $A$ là biến cố $''$Cả bốn lần gieo xuất hiện mặt sấp$''n(A)=1$.\\
Vậy xác suất cần tính $P(A)=\dfrac{1}{16}$
}
\end{ex}
%Câu 11
\begin{ex}
Một tổ có 7 học sinh nữ, 5 học sinh nam. Chọn ngẫu nhiên 2 bạn đi trực nhật. Xác suất để 2 bạn được chọn đều là nữ là
\choice
{$\dfrac{5}{33}$}
{$\dfrac{10}{33}$}
{$\dfrac{5}{33}$}
{\True $\dfrac{7}{22}$}
\loigiai{ Chọn $2$ trong $12$ bạn làm trực nhật, số phần tử của không gian mẫu là: $n\left(\Omega\right)=C_{12}^2=66$\\
Gọi $A$ là biến cố ‘$2$ bạn được chọn đều là nữ’. Ta có: $n(A)=C_7^2=21$.\\
Suy ra xác suất để 2 bạn được chọn đều là nữ là: $P(A)=\dfrac{n(A)}{n\left(\Omega\right)}=\dfrac{7}{22}$
}
\end{ex}
%Câu 12
\begin{ex}
Gieo một con súc sắc cân đối đồng chất. Tính xác suất để mặt có số chấm là chẵn xuất hiện?
\choice
{$0{,}2$}
{$0{,}3$}
{$0{,}4$}
{\True $0{,}5$}
\loigiai{ Số phần tử của không gian mẫu: $n\left(\Omega\right)=C_6^1$.\\
Mặt có sô chấm là chẵn gồm: 2; 4; 6 chấm. Khi đó xác suất để mặt có số chấm là số chẵn xuất hiện là: $P=\dfrac{3}{6}=\dfrac{1}{2}$
}
\end{ex}
%Câu 13
\begin{ex}
Trong mặt phẳng tọa độ $Oxy$, cho 2 điểm $A(-1;5)$; $B(2;4)$. $M(x;y)$ là điểm thỏa mãn hệ thức $3\vec{MA}+\vec{MB}=\vec{0}$. Hiệu $y-x$ bằng
\choice
{$\dfrac{18}{4}$}
{$-\dfrac{18}{4}$}
{$-5$}
{\True $5$}
\loigiai{ Ta có: $\vec{MA}=\left(-1-x;5-y\right)$ và $\vec{MB}=\left(2-x;4-y\right)$. Khi đó:\\
$3\vec{MA}+\vec{MB}=\vec{0}\Leftrightarrow \heva{& -3-3x+2-x=0 \\& 15-3y+4-y=0}\Leftrightarrow \heva{& -4x=1 \\& -4y+19=0}\Leftrightarrow \heva{& x=-\dfrac{1}{4} \\& y=\dfrac{19}{4}}$. \\
Vậy $y-x=5$.
}
\end{ex}
%Câu 14
\begin{ex}
Trong mặt phẳng tọa độ $Oxy$, phương trình tham số của đường thẳng đi qua 2 điểm $A(1;2)$, $B(0;-2)$ là
\choice
{$\heva{& x=1-t \\& y=2+4t}$}
{$\heva{& x=-4t \\& y=-2-t}$}
{$\heva{& x=-1 \\& y=-4-2t}$}
{\True $\heva{& x=1+t \\& y=2+4t}$ }
\loigiai{ Ta có $\vec{AB}=(-1;-4)$, \\
Đường thẳng$AB$ đi qua điểm $A(1;2)$ và nhận $\vec{u}=(1;4)$ làm vtcp nên có phương trình tham số là:\\
$\heva{& x=1+t \\& y=2+4t}\left(t\in \mathbb{R}\right)$
}
\end{ex}
%Câu 15
\begin{ex}
Cho đường thẳng $(d)\colon \heva{& x=-1+2t \\& y=3-4t} \left(t\in \mathbb{R}\right)$. Vectơ nào sau đây là vectơ chỉ phương của đường thẳng $d$?
\choice
{$\vec{u}=(-1;3)$}
{$\vec{u}=(-4;2)$}
{$\vec{u}=(-2;-4)$}
{\True $\vec{u}=(1;-2)$}
\loigiai{ Đường thẳng $(d)\colon \heva{& x=-1+2t \\& y=3-4t} \left(t\in \mathbb{R}\right)$ có 1 vectơ chỉ phương là $\vec{v}=(2;-4)$.\\
$\vec{u}=(1;-2)\Rightarrow \vec{u}=\dfrac{1}{2}\vec{v}\Rightarrow \vec{u}\left(1;-2\right)$ là vectơ chỉ phương $d$
}
\end{ex}
%Câu 16
\begin{ex}
 Xét vị trí tương đối của hai đường thẳng $d_1\colon x-2y+1=0$ và $d_2\colon 3x+5y-10=0$.

\choice
{\True cắt nhau}
{trùng nhau}
{song song}
{không xác định được}
\loigiai{
$\heva{& d_1\colon x-2y+1=0 \\& d_2\colon 3x+5y-10=0}\to \dfrac{1}{3}\ne \dfrac{-2}{5}$ . Suy ra $d_1;d_2$ cắt nhau
}
\end{ex}
%Câu 17
\begin{ex}
 Khoảng cách từ điểm $A(-1;2)$ đến đường thẳng $\left(\Delta\right)\colon 3x-4y+6=0$ là
\choice
{\True $1$}
{$2$}
{$3$}
{$5$}
\loigiai{
Ta có: $d\left(A,\Delta\right)=\dfrac{\left| 3 \cdot (-1)-4 \cdot 2+6 \right|}{\sqrt{4^2+(-3)^2}}=1$
}
\end{ex}
%Câu 18
\begin{ex}
 Tìm tâm và bán kính của đường tròn có phương trình: ${{(x-5)}^2}+{{(y+3)}^2}=36$.
\choice
{$I(-5;-3),R=6$}
{$I(5;-3),R=36$}
{$I(-5;3),R=6$}
{\True $I(5;-3),R=6$}
\loigiai{
Đường tròn đã cho có tâm $I(5;-3),R=6$
}
\end{ex}
%Câu 19
\begin{ex}
 Đường tròn $(C)$ có phương trình $x^2+y^2+2x+4y-4=0$. Đường tròn $(C)$ có tâm và bán kính là
\choice
{$I(-1;-2);R=1$}
{$I(1;2);R=3$}
{\True $I(-1;-2);R=3$}
{$I(1;2);R=9$}
\loigiai{
Đường tròn đã cho có tâm $I(-1;-2)$ và bán kính $R=\sqrt{a^2+b^2-c}=\sqrt{1+4-(-4)}=3$
}
\end{ex}
%Câu 20
\begin{ex}
Trong các phương trình sau, phương trình nào là phương trình chính tắc của một elip?
\choice
{\True $\dfrac{x^2}{36}+\dfrac{y^2}{16}=1$ }
{$\dfrac{x^2}{36}-\dfrac{y^2}{9}=1$}
{$\dfrac{x^2}{24}+\dfrac{y^2}{64}=1$}
{$\dfrac{x^2}{100}-\dfrac{y^2}{16}=-1$}
\loigiai{ \\
Phương trình elip có dạng: $\dfrac{x^2}{a^2}+\dfrac{y^2}{b^2}=1$.
}
\end{ex}
%Câu 21
\begin{ex}
 Từ các chữ số $0;1;2;3;4;5;6;7;9$ có thể lập được bao nhiêu số tự nhiên gồm 3 chữ số khác nhau, bắt đầu bằng 2 hoặc 4, và chia hết cho 5?
\choice
{32}
{14}
{\True 28}
{16}
\loigiai{ $A=\left\{ 0;1;2;3;4;5;6;7;9 \right\}$ có 9 số.\\
Gọi số tự nhiên cần tìm có dạng $\overline{abc}$.\\
Công đoạn 1: Chọn $a\in \left\{ 2;4 \right\}$ có 2 cách.\\
Công đoạn 2: Chọn $c\in \left\{ 0;5 \right\}$ có 2 cách.\\
Công đoạn 3: Chọn $b\in A\setminus \left\{ a;c \right\}$ có 7 cách.\\
Theo quy tắc nhân ta có $2 \cdot 2 \cdot 7=28$ (STN)
}
\end{ex}
%Câu 22
\begin{ex}
 Cần xếp 12 bạn, trong đó có An và Bình thành một hàng dọc để chuẩn bị cho 1 tiết mục múa. Có bao nhiêu cách xếp khác nhau để An và Bình đứng cạnh nhau?
\choice
{7.257.600 cách}
{958.003.200 cách}
{479.001.600 cách}
{\True 79.833.600 cách}
\loigiai{ Công đoạn 1: Nhóm An và Bình thành nhóm X có $2!$ cách.\\
Công đoạn 2: Xếp X và 10 bạn còn lại có $11!$ cách.\\
Theo quy tắc nhân ta có $2! \cdot 11!=79.833.600$ cách
}
\end{ex}
%Câu 23
\begin{ex}
 Lớp 10T-Math có 40 học sinh gồm 25 nam và 15 nữ. Thầy Trà cần chọn ra một ban cán sự lớp có 8 bạn gồm một lớp trưởng và một lớp phó học tập là nam; một lớp phó kỉ luật và một lớp phó văn thể mỹ là nữ; hai tổ trưởng tổ 1, 3 là nam và hai tổ trưởng tổ 2, 4 là nữ. Trong kết quả của phép tính lựa chọn trên có bao nhiêu số 0?
\choice
{Có một số 0}
{\True Có ba số 0}
{Có hai số 0}
{Không có số 0 nào}
\loigiai{ Công đoạn 1: Chọn 4 bạn nam từ 25 nam và sắp xếp vào 4 vai trò tương ứng có $A_{25}^4$ cách.\\
Công đoạn 2: Chọn 4 bạn nữ từ 15 nữ và sắp xếp vào 4 vai trò tương ứng có $A_{15}^4$ cách.\\
Theo quy tắc nhân ta có $A_{25}^4 \cdot A_{15}^4=9 \cdot 945 \cdot 936 \cdot 000$ cách. Trong kết quả có ba số 0
}
\end{ex}
%Câu 24
\begin{ex}
 Cho $9$ điểm sao cho không có 3 điểm nào thẳng hàng. Số tam giác với $3$ đỉnh là $3$ điểm trong $9$ điểm đã cho bằng 
\choice
{$ 70$}
{$6$}
{\True $84$}
{$504$}
\loigiai{ Mỗi cách chọn $3$ điểm trong $9$ điểm đã cho để tạo được một tam giác là một tổ hợp chập $3$ của $9$ phần tử, do đó có $C_9^3 = 84$ tam giác
}
\end{ex}
%Câu 25
\begin{ex}
 Một bồn cây có dạng hình tròn với bán kính là $1{,}2$ m. Hai bạn Minh và Anh cùng muốn tính diện tích $S$ của bồn hoa đó. Bạn Minh lấy một giá trị gần đúng của $\pi $ là $3{,}14$ và được kết quả là $S_1$. Bạn Anh lấy một giá trị gần đúng của $\pi $ là $3{,}14159$ và được kết quả là $S_2$. Bạn nào cho kết quả chính xác hơn? 
\choice
{Bạn Minh}
{Cả hai bạn đều sai}
{Cả hai bạn đều đúng}
{\True Bạn Anh}
\loigiai{ Ta có $S_1 = 3{,}14 \cdot \left(1 \cdot 2\right)^2 = 4{,}5216 \left(m^2\right)$\\
$S_2 = 3{,}14159 \cdot \left(1 \cdot 2\right)^2 = 4{,}5238896 \left(m^2\right)$\\
Ta thấy: $3{,}14< 3{,}14159 < \pi $ nên $3{,}14 \cdot \left(1{,}2\right)^2 < 3{,}14159 \cdot \left(1{,}2\right)^2< \pi \cdot \left(1{,}2\right)^2$ tức là $S_1< S_2 < S$.\\
Suy ra $\Delta_{S_2}= \left| S - S_2 \right| < \left| S - S_1 \right| = \Delta_{S_1}$.\\
Vậy bạn Anh cho kết quả chính xác hơn
}
\end{ex}
%Câu 26
\begin{ex}
 Mẫu số liệu thống kê cân nặng (đơn vị: tấn) của 10 con voi châu Á trưởng thành là:
\begin{center}
$3{,}5$ $4{,}9$ $3{,}7$ $4{,}6$ $4{,}6$ $5{,}0$ $3{,}2$ $3{,}6$ $3{,}7$ $4{,}5$
\end{center}
 Khoảng tứ phân vị của mẫu số liệu bằng 
\choice
{\True $\Delta Q = 1$}
{$\Delta Q = 0{,}1$}
{$\Delta Q = 0{,}5$}
{$\Delta Q = 1{,}3$}
\loigiai{ Sắp xếp mẫu số liệu theo thứ tự không giảm, ta được 
\begin{center}
$3{,}2$ $3{,}5$ $3{,}6$ $3{,}7$ $3{,}7$ $4{,}5$ $4{,}6$ $4{,}6$ $4{,}9$ $5{,}0$
\end{center}
Do đó $Q_1 = 3{,}6$, (tấn); $Q_2 = \dfrac{3{,}7 + 4{,}5}{2} = 4{,}1$(tấn); $Q_3 = 4{,}6$, (tấn)\\
Vậy khoảng tứ phân vị của mẫu số liệu là $\Delta Q = Q_3 - Q_1 = 4{,}6 -3{,}6= 1$(tấn)
}
\end{ex}
%Câu 27
\begin{ex}
Từ một lớp gồm 16 học sinh nam và 18 học sinh nữ. Chọn ngẫu nhiên 5 học sinh tham gia đội Thanh niên xung kích. Tính xác suất chọn được 2 học sinh nam và 3 học sinh nữ.
\choice
{\True $\dfrac{120}{341}$}
{$\dfrac{105}{341}$}
{$\dfrac{91}{5797}$}
{$\dfrac{21}{682}$}
\loigiai{ Số phần tử của không gian mẫu $n\left(\Omega\right)=C_{34}^5$\\
Gọi $A$ là biến cố: "Chọn được 2 học sinh nam và 3 học sinh nữ".\\
Chọn 2 học sinh nam trong số 16 học sinh nam thì có $C_{16}^2$ cách chọn.\\
Chọn 3 học sinh nữ trong số 18 học sinh nữ thì có $C_{18}^3$ cách chọn.\\
Áp dụng quy tắc nhân, sẽ có $C_{16}^2 \cdot C_{18}^3$ cách chọn 2 học sinh nam và 3 học sinh nữ.\\
Vậy xác suất cần tìm $P(A)=\dfrac{n(A)}{n\left(\Omega\right)}=\dfrac{C_{16}^2 \cdot C_{18}^3}{C_{34}^5}=\dfrac{120}{341}$
}
\end{ex}
%Câu 28
\begin{ex}
Xếp ngẫu nhiên 12 người, gồm 9 nam và 3 nữ thành một hàng ngang. Tính xác suất sao cho không có học sinh nữ đứng cạnh nhau
\choice
{$\dfrac{10}{21}$}
{\True $\dfrac{6}{11}$ }
{$\dfrac{1}{4}$}
{$\dfrac{1}{12}$}
\loigiai{ Số phần tử của không gian mẫu $n\left(\Omega\right)=12!$\\
Gọi $A$ là biến cố: "Không có học sinh nữ đứng cạnh nhau".\\
Xếp ngẫu nhiên 9 học sinh nam thành 1 hàng ngang có $9!$ cách xếp.\\
Chọn 3 khe hở trong 10 khe hở để xếp 3 học sinh nữa có $A_{10}^3$ cách.\\
Vậy xác suất cần tìm $P(A)=\dfrac{n(A)}{n\left(\Omega\right)}=\dfrac{9! \cdot A_{10}^3}{12!}=\dfrac{6}{11}$
}
\end{ex}
%Câu 29
\begin{ex}
Gọi S là tập hợp các số tự nhiên có ba chữ số đôi một khác nhau được lập thành từ các chữ số $0{,}1,2{,}3,4{,}5,6$. Chọn ngẫu nhiên một số từ S, tính xác suất để số được chọn là một số chia hết cho $5$.
\choice
{\True $\dfrac{11}{36}$}
{$\dfrac{1}{12}$}
{$\dfrac{10}{21}$}
{$\dfrac{1}{4}$}
\loigiai{ Số phần tử của không gian mẫu: $n\left(\Omega\right)=A_7^3-A_6^3=180$.\\
Gọi $A$ là biến cố: "Số chọn được là một số chia hết cho $5$".\\
Số chia hết cho $5$ được lập từ các chữ số trên có dạng $\overline{ab5}$ hoặc $\overline{ab0}$\\
TH1: Số có dạng $\overline{ab0}$. Chọn $2$ số $a,b$ từ các chữ số $1{,}2,3{,}4,5{,}6$ là một chỉnh hợp chập $2$ của $6$ phần tử.\\
TH2: Số có dạng $\overline{ab5}$. Khi đó $a$ có 5 cách chọn, $b$ có 5 cách chọn\\
Số cách chọn là $n(A)=A_6^2+5 \cdot 5=55$.\\
Vậy xác suất cần tìm là: $P(A)=\dfrac{n(A)}{n\left(\Omega\right)}=\dfrac{55}{180}=\dfrac{11}{36}$
}
\end{ex}
%Câu 30
\begin{ex}
Trong mặt phẳng tọa độ $Oxy$, cho các điểm $A(2;1)$, $B(1;3)$. Khi đó tọa độ véc tơ $\vec{BA}$ là
\choice
{$(-1;2)$}
{$(1;2)$}
{$(2;-1)$}
{$(1;-2)$}
\loigiai{
$\vec{AB}=(-1;2)\Rightarrow \vec{BA}=(1;-2)$ 
}
\end{ex}
%Câu 31
\begin{ex}
Góc giữa hai đường thẳng $\Delta _1\colon y+5=0$ và $\Delta _2\colon \heva{& x=-1+\sqrt{3} t \\& y=9+t} \left(t\in \mathbb{R}\right)$ bằng
\choice
{$0^\circ $}
{\True $30^\circ $}
{$60^\circ $}
{$90^\circ (1;-2)$}
\loigiai{ Ta có $\vec{n}_{{{\Delta _1}}}=(0;1),\vec{n}_{{{\Delta _2}}}=\left(-1;\sqrt{3}\right)$. Khi đó $\cos \left(\Delta _1,\Delta _2\right)=\dfrac{\left| \sqrt{3} \right|}{\sqrt{1} \cdot \sqrt{4}}=\dfrac{\sqrt{3}}{2}\Rightarrow \left(\Delta _1,\Delta _2\right)=30^\circ $
}
\end{ex}
%Câu 32
\begin{ex}
Phương trình tham số của đường thẳng $d$ đi qua $A(0;3)$ và song song với trục $Ox$ là
\choice
{$\heva{& x=3 \\& y=t}$}
{$\heva{& x=t \\& y=3+t}$}
{$\heva{& x=3+t \\& y=t}$}
{$\heva{& x=t \\& y=3}$}
\loigiai{ Đường thẳng $d \parallel Ox;$ trục hoành nhận vecto đơn vị $\vec{i}(1;0)$ là vecto chỉ phương nên ta loại B, C, A. \\
Mặt khác $A(0;3)\in d$ nên chọn D
}
\end{ex}
%Câu 33
\begin{ex}
 Trên hệ trục tọa độ $Oxy$, cho đường tròn $(C)\colon x^2+y^2-10x+2y+1=0$. Bán kính của đường tròn đã cho là
\choice
{\True $5$ }
{$\sqrt{26}$ }
{$\sqrt{24}$ }
{$25$}
\loigiai{ Ta có: $a=5,b=-1$ và $c=1$. Khi đó bán kính của đường tròn đã cho là:$R=\sqrt{a^2+b^2-c}=\sqrt{5^2+{{(-1)}^2}-1}=5$
}
\end{ex}
%Câu 34
\begin{ex}
Trên hệ trục tọa độ $Oxy$, có bao nhiêu giá trị nguyên của $m\in [-10;10]$ để phương trình $x^2+y^2-2(m+1)x+4y+7m+5=0$ là phương trình đường tròn?
\choice
{$16$}
{$11$}
{\True $15$}
{$12$}
\loigiai{ Ta có $a=m+1$, $b=-2$ và $c=7m+5$.\\
Để phương trình đã cho là phương trình đường tròn thì $a^2+b^2-c>0\Leftrightarrow {{(m+1)}^2}+{{(-2)}^2}-(7m+5)>0$ $\Leftrightarrow m^2-5m>0$ $\Leftrightarrow \hoac{& m<0 \\& m>5}$.\\
Vì $m\in [-10;10]\Rightarrow $ có $15$ giá trị $m$ thỏa mãn yêu cầu bài toán
}
\end{ex}
%Câu 35
\begin{ex}
Cho đường elip có phương trình chính tắc sau: $(E)\colon \dfrac{x^2}{25}+\dfrac{y^2}{9}=1$. Giao điểm của đường elip với trục hoành là
\choice
{\True $A(5;0);B(-5;0)$}
{$M(0;5),N(0;-5)$}
{$P(0;3),Q(0;-3)$}
{$C(3;0),D(-3;0)$}
\end{ex}
\noindent{\bf\fontfamily{qag}\selectfont\color{violet}B. PHẦN TỰ LUẬN}
%Câu 36
\begin{ex}
Tìm phương trình chính tắc của Elip có tiêu cự bằng $4$ và đi qua điểm $A(0;6)$.
\loigiai{ Phương trình chính tắc của elip có dạng $\dfrac{x^2}{a^2}+\dfrac{y^2}{b^2}=1\left(a,b>0\right)$.\\
Theo giả thiết:$2c=4\Leftrightarrow c=2$. Vì $A(0;6)\in (E)$ nên ta có phương trình: $\dfrac{0^2}{a^2}+\dfrac{6^2}{b^2}=1 \Leftrightarrow b=6$.\\
Khi đó: $a^2=b^2+c^2\Leftrightarrow a^2=6^2+2^2\Leftrightarrow a^2=40\Leftrightarrow a=\sqrt{40}$.\\
Vậy phương trình chính tắc của Elip là: $\dfrac{x^2}{40}+\dfrac{y^2}{36}=1$.\\
}
\end{ex}
%Câu 37
\begin{ex}
 Cho đường tròn $(C) \colon x^2+y^2+4x-2y-4=0$. Lập phương trình tiếp tuyến tại điểm có tung độ bằng $1$ thuộc đường tròn $(C)$
\loigiai{
Đường tròn $(C)$ có tâm $I(-2;1)$.\\
Thay $y=1$ vào phương trình đường tròn $(C)$ ta được:\\
$x^2+1^2+4x-2 \cdot 1-4=0\Leftrightarrow x^2+4x-5=0\Leftrightarrow \hoac{& x=1 \\& x=-5}$.\\
Với $x=1$; $y=1$ ta được $A(1;1)$. \\
Tiếp tuyến tại $A(1;1)$ nhận vectơ $\vec{IA}=(3;0)$ làm VTPT có phương trình:\\
$3(x-1)+0(y-1)=0\Leftrightarrow x-1=0$.\\
Với $x=-5$; $y=1$ ta được $B(-5;1)$. \\
Tiếp tuyến tại $B(-5;1)$ nhận vectơ $\vec{IB}=(-3;0)$ làm VTPT có phương trình:\\
$-3(x+5)+0(y-1)=0\Leftrightarrow x+5=0$
}
\end{ex}
%Câu 38
\begin{ex}
Từ $20$ câu hỏi trắc nghiệm gồm $9$ câu dễ, $7$ câu trung bình và $4$ câu khó. Người ta chọn ra $7$ câu đề làm đề kiểm tra sao cho phải có đủ $3$ loại dễ, trung bình, khó. Hỏi có bao nhiêu đề kiểm tra?
\loigiai{
Chọn $7$ câu bất kỳ trong $20 $ câu hỏi có $C_{20}^7$ cách.\\
Xét bài toán đối: Chọn ra $7$ câu sao cho không đủ $3$ loại.\\
Ký hiệu: Dễ ($D$), Trung bình ($T$), Khó ($K$)\\
TH1: Chọn $7$ câu trong đó không có câu khó (chỉ có $D$ hoặc chỉ có $T$ hoặc chỉ có $D\And T$), nghĩa là chọn 7 câu trong 16 câu có $C_{16}^7$ cách.\\
TH2: Chọn $7$ câu trong đó không có câu trung bình (chỉ có $D$ hoặc chỉ có $K$ hoặc chỉ có$D\And K$), nghĩa là có $C_{13}^7 – C_9^7$ cách (vì chỉ có $D$ đã được tính ở TH1).\\
TH3: Chọn $7$ câu trong đó không có câu dễ (chỉ có $K$ hoặc chỉ có $T$ hoặc chỉ có $K\And T$), nghĩa là có $C_{11}^7 – C_7^7$ cách (vì chỉ có $T$ đã được tính ở TH1).\\
Vậy có $C_{20}^7 – \left(C_{16}^7+C_{13}^7-C_9^7 + C_{11}^7 – C_7^7\right) = 64071$ đề kiểm tra được tạo ra thỏa yêu cầu bài toán
}
\end{ex}
%Câu 39
\begin{ex}
Trong mặt phẳng với hệ tọa độ $Oxy$, cho đường tròn $(C)$: ${{(x-1)}^2}+{{(y-1)}^2}=25$ và điểm 
 $M(-1;2)$. Lập phương trình đường thẳng $d$ qua $M$ cắt $(C)$ tại 2 điểm phân biệt $A$, $B$ sao cho 
 độ dài dây cung $AB$ nhỏ nhất
\loigiai{
Đường tròn $(C)$ có tâm $I(1;1)$ bán kính $R=5$. Ta có: $IM=\sqrt{5}\Rightarrow IM<R$ nên điểm $M$ nằm trong đường tròn $(C)$, kẻ $IH\bot d\Rightarrow IH\le IM$ và $HA=HB=\dfrac{AB}{2}$.\\
Ta có $AH^2=IA^2-IH^2=25-IH^2$, $AB$ nhỏ nhất khi và chỉ khi $AH$ nhỏ nhất $\Leftrightarrow $ $IH$ lớn nhất $\Leftrightarrow IH=IM\Leftrightarrow H\equiv M$. Khi đó đường thẳng $d$ đi qua $M$ và vuông góc với $IM$ nên đường thẳng $d$ có một véctơ pháp tuyến là $\vec{IM}=(-2;1)$. Vậy phương trình đường thẳng $d$ là: $-2(x+1)+1(y-2)=0\Leftrightarrow -2x+y-4=0$.\\
$d\left(I;\Delta\right)=2\Leftrightarrow \dfrac{\left| a-2b+a-2b \right|}{\sqrt{a^2+b^2}}=2$\\
$\Leftrightarrow |a-2b|=\sqrt{a^2+b^2}$\\
$\Leftrightarrow a^2+4b^2-4ab=a^2+b^2$\\
$\Leftrightarrow -4ab+3b^2=0\Leftrightarrow b\left(-4a+3b\right)=0\Leftrightarrow \hoac{& b=0 \\& 3b=4a}$\\
* Nếu $b=0$, do $a^2+b^2>0$, ta chọn $a=1\Rightarrow \Delta \colon x+1=0$.\\
* Nếu $3b=4a$, do $a^2+b^2>0$, ta chọn $a=3;b=4\Rightarrow \Delta \colon 3x+4y-5=0$.\\
Vậy có hai phương trình đường thẳng $\Delta $ thỏa mãn yêu cầu bài toán là $\Delta \colon x+1=0$ và $\Delta \colon 3x+4y-5=0$
}
\end{ex}

\Closesolutionfile{ans}