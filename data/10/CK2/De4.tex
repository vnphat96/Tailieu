\section*{ÔN TẬP KIỂM TRA CUỐI KÌ 2 - ĐỀ 04}
\setcounter{ex}{0}\setcounter{bt}{0}
\noindent{\bf\fontfamily{qag}\selectfont\color{violet}A. PHẦN TRẮC NGHIỆM}
\Opensolutionfile{ans}[ans/ansBTTeXCK24]
%Câu 1
\begin{ex}
Một công việc được hoàn thành bởi một trong hai hành động. Nếu hành động thứ nhất có $m$ cách thực hiện, hành động thứ hai có $n$ cách thực hiện (các cách thực hiện của hai hành động là khác nhau đôi một) thì công việc đó có bao nhiêu cách hoàn thành?
\choice
{$m-n$}
{$\dfrac{m}{n}$}
{\True $m+n$}
{$m \cdot n$}
\end{ex}
%Câu 2
\begin{ex}
Một học sinh muốn mua một phần quà tặng mẹ nhân ngày 8/3 bao gồm $1$ bông hoa hồng và $1$ cái thiệp. Biết rằng cửa hàng có $8$ bông hoa hồng với các màu khác nhau và $10$ cái thiệp với họa tiết khác nhau dành tặng mẹ, hỏi bạn học sinh đó có bao nhiêu sự lựa chọn cho phần quà?
\choice
{\True $80$}
{$1$}
{$18$}
{$2$}
\end{ex}
%Câu 3
\begin{ex}
Số cách xếp $6$ nam sinh và $9$ nữ sinh vào một dãy ghế hàng ngang có $15$ chỗ ngồi là
\choice
{$9! \cdot 6$}
{\True $15!$}
{$9! \cdot 6!$}
{$9!+6!$}
\end{ex}
%Câu 4
\begin{ex}
Có bao nhiêu cách xếp 4 bạn Toán, Vận, Dụng, Cao thành một hàng dọc sao cho Bạn Toán luôn đứng đầu hàng?
\choice
{$24$}
{\True $6$}
{$10$}
{$4$}
\end{ex}
%Câu 5
\begin{ex}
Lớp $10A$ có $45$ bạn học sinh. Có bao nhiêu cách để cô giáo chủ nhiệm thành lập một ban cán bộ lớp gồm $1$ bạn lớp trưởng, $1$ bạn bí thư, $1$ bạn lớp phó học tập và $1$ bạn lớp phó lao động?
\choice
{$148995$}
{\True $3575880$}
{$893970$}
{$595980$}
\end{ex}
%Câu 6
\begin{ex}
Giải bóng đá V-LEAGUE 2022 có tất cả $13$ đội bóng tham gia, các đội bóng thi đấu vòng tròn $2$ lượt (tức là hai đội A và B bất kỳ thi đấu với nhau hai trận, một trận trên sân đội A, trận còn lại trên sân đội B). Hỏi giải đấu có bao nhiêu trận đấu?
\choice
{\True $156$}
{$78$}
{$169$}
{$13$}
\end{ex}
%Câu 7
\begin{ex}
Một hộp đựng $5$ viên bi màu xanh,$7$ viên bi màu vàng. Có bao nhiêu cách lấy ra $6$ viên bi bất kỳ?
\choice
{$665280$}
{\True $924$}
{$7$}
{$12$}
\end{ex}
%Câu 8
\begin{ex}
Trong hộp có 6 bi trắng và 7 bi đen. Số cách chọn ra 3 bi có đúng 1 bi đen là
\choice
{\True $105$}
{$286$}
{$210$}
{$126$}
\end{ex}
%Câu 9
\begin{ex}
Đa thức $P(x)=x^5-5x^4y+10x^3y^2-10x^2y^3+5xy^4-y^5$ là khai triển của nhị thức nào dưới đây?
\choice
{\True $(x-y)^5$}
{${{(x+y)}^5}$}
{${{(2x-y)}^5}$}
{${{(x-2y)}^5}$}
\end{ex}
%Câu 10
\begin{ex}
Sử dụng máy tính bỏ túi, hãy viết giá trị gần đúng của $\sqrt{3}$ chính xác đến hàng phần nghìn
\choice
{1{,}73205}
{\True 1{,}732}
{1{,}733}
{1{,}731}
\end{ex}
%Câu 11
\begin{ex}
Hãy viết số quy tròn của số gần đúng $a=17658$ biết $\bar{a}=17658\pm 16$.
\choice
{\True 17700}
{17800}
{17500}
{17600}
\end{ex}
%Câu 12
\begin{ex}
Số nào sau đây chia đôi mẫu số liệu, không bị ảnh hưởng bởi giá trị bất thường của mẫu số liệu?
\choice
{Số trung bình}
{\True Trung vị}
{Mốt}
{Một trong ba số của tứ phân vị}
\end{ex}
%Câu 13
\begin{ex}
Theo kết quả thống kê điểm thi học kỳ 1 môn toán khối 10 của một trường THPT, người ta tính được phương sai của bảng thống kê đó là $S^2=0{,}573$. Độ lệch chuẩn của bảng thống kê đó gần nhất với số nào sau đây.
\choice
{$0{,}812$}
{\True $0{,}757$}
{$0{,}936$}
{$0{,}657$}
\end{ex}
%Câu 14
\begin{ex}
Mẫu số liệu nào dưới đây có khoảng biến thiên là 35?
\choice
{35, 57, 11, 22}
{\True 47, 15, 12, 32}
{55, 3, 26, 89}
{4, 17, 23, 20}
\end{ex}
%Câu 15
\begin{ex}
Nếu tứ phân vị của mẫu số liệu theo thứ tự là ${m, n, p}$ thì khoảng tứ phân vị là:
\choice
{\True ${p-m}$}
{${n-m}$}
{${p-n}$}
{${n-p}$}
\end{ex}
%Câu 16
\begin{ex}
Từ một hộp chứa 5 quả cầu xanh và 9 quả cầu trắng. Chọn ngẫu nhiên 3 quả cầu từ hộp cầu đó. Tính xác suất để chọn được 3 quả cầu trắng
\choice
{\True $\dfrac{3}{13}$}
{$\dfrac{5}{182}$}
{$\dfrac{1}{3}$}
{$\dfrac{1}{5}$}
\end{ex}
%Câu 17
\begin{ex}
Một hộp đựng 9 thẻ được đánh số $1{,}2,3{,}4 \cdot \cdot \cdot ,9$. Rút ngẫu nhiên đồng thời 2 thẻ và nhân hai số ghi trên hai thẻ lại với nhau. Tính xác suất để tích nhận được là số chẵn.
\choice
{$\dfrac{5}{18}$}
{\True $\dfrac{13}{18}$}
{$\dfrac{8}{9}$}
{$\dfrac{1}{6}$}
\end{ex}
%Câu 18
\begin{ex}
Có hai cái hộp, mỗi hộp chứa $5$ cái thẻ được đánh số từ $1$ đến $5$. Rút ngẫu nhiên từ mỗi hộp một tấm thẻ. Xác suất để $2$ thẻ rút ra đều ghi số lẻ là
\choice
{$\dfrac{1}{3}$}
{\True $\dfrac{9}{25}$}
{$\dfrac{3}{10}$}
{$\dfrac{3}{5}$}
\end{ex}
%Câu 19
\begin{ex}
Cho tập $A=\left\{ 1{,}2,3{,}4,5{,}6,7{,}8 \right\}$. Gọi $S$ là tập các số có 4 chữ số khác nhau lập từ $A$. Lấy ngẫu nhiên một số từ tập $S$. Tính xác suất để số lấy được bắt đầu bằng chữ số 2?
\choice
{$\dfrac{1}{4}$}
{\True $\dfrac{1}{8}$}
{$\dfrac{1}{5}$}
{$\dfrac{2}{7}$}
\end{ex}
%Câu 20
\begin{ex}
Rút ngẫu nhiên đồng thời $2$ thẻ từ một hộp có $20$ tấm thẻ được đánh số từ $1$ đến $20$. Xác suất để tổng hai số trên hai tấm thẻ được rút ra bằng 10 là
\choice
{$\dfrac{4}{95}$}
{\True $\dfrac{2}{95}$}
{$\dfrac{5}{190}$}
{$\dfrac{9}{190}$}
\end{ex}
%Câu 21
\begin{ex}
Cho các số $0{,}1,2{,}3,4{,}5,6{,}7$. Có bao nhiêu số gồm $7$ chữ số mà số tìm được chia hết cho $6$ đồng thời các chữ số là khác nhau.
\choice
{\True $6960$}
{$6961$}
{$6959$}
{$6958$}
\end{ex}
%Câu 22
\begin{ex}
Trong mặt phẳng tọa độ Oxy, cho điểm $A(1;-3)$, $B(0;-2)$. Tọa độ điểm $D$ sao cho $\vec{AD}=-3\vec{AB}$ là:
\choice
{$D(2;4)$}
{$D(-3;2)$}
{\True $D(4;-6)$}
{$D(-6;8)$}
\end{ex}
%Câu 23
\begin{ex}
Trong mặt phẳng tọa độ Oxy, cho điểm $A(1;2)$, $B(3;-2)$. Điểm M trên trục Oy sao cho ba điểm $A$, $B$, $M$ thẳng hàng thì tọa độ điểm M là
\choice
{$M(-4;0)$}
{\True $M(0;4)$}
{$M(0;-4)$}
{$M(4;0)$}
\end{ex}
%Câu 24
\begin{ex}
Trong mặt phẳng $Oxy$, đường thẳng $(d)\colon ax+by+c=0,\left(a^2+b^2\ne 0\right)$. Vectơ nào sau đây là một vectơ pháp tuyến của đường thẳng $(d)$?
\choice
{$\vec{n}=(a;-b)$}
{$\vec{n}=(b;a)$}
{$\vec{n}=(b;-a)$}
{\True $\vec{n}=(a;b)$}
\end{ex}
%Câu 25
\begin{ex}
Vectơ chỉ phương của đường thẳng $d$: $\heva{& x=1-4t \\& y=-2+3t}$ là:
\choice
{\True $\vec{u}=(-4;3)$}
{$\vec{u}=(4;3)$}
{$\vec{u}=(3;4)$}
{$\vec{u}=(1;-2)$}
\end{ex}
%Câu 26
\begin{ex}
Trong mặt phẳng tọa độ $Oxy$, cho tam giác $ABC$ với $A(3;1)$, $B(1;2)$ và $C(3;5)$. Phương trình tổng quát của đường cao kẻ từ $A$ của tam giác $ABC$ là:
\choice
{\True $2x+3y-9=0$}
{$3x-2y-11=0$}
{$2x+3y+9=0$}
{$4x+7y-19=0$}
\end{ex}
%Câu 27
\begin{ex}
Trong mặt phẳng với hệ tọa độ $Oxy$, giả sử điểm $A\left(a;b\right)$ thuộc đường thẳng $d\colon \,x-y-3=0$ và cách $\Delta \colon \,2x-y+1=0$ một khoảng bằng $\sqrt{5}$. Tính $P=a \cdot b$ biết $a>0$. 
\choice
{$4$}
{\True $-2$}
{$2$}
{$-4$}
\end{ex}
%Câu 28
\begin{ex}
Khoảng cách từ điểm $M(3;-4)$ đến đường thẳng $\Delta \colon 3x-4y-1=0$ bằng:
\choice
{\True $\dfrac{24}{5}$}
{$\dfrac{12}{5}$}
{$\dfrac{8}{5}$}
{$\dfrac{12}{5}$}
\end{ex}
%Câu 29
\begin{ex}
Tìm côsin góc giữa $2$ đường thẳng $\Delta _1$: $10x+5y-1=0$ và $\Delta _2$:$\heva{& x=2+t \\& y=1-t}$.
\choice
{$\dfrac{\sqrt{10}}{10}$}
{\True $\dfrac{3\sqrt{10}}{10}$}
{$\dfrac{3}{5}$}
{$\dfrac{3}{10}$}
\end{ex}
%Câu 30
\begin{ex}
Trong các phương trình sau, phương trình nào là phương trình đường tròn?
\choice
{$x^2+y^2+2x-4y+9=0$}
{$\,x^2+y^2-6x+4y+13=0$}
{\True $2x^2+2y^2-6x-4y-1=0$}
{$2x^2+y^2+2x-3y+9=0$}
\end{ex}
%Câu 31
\begin{ex}
Xác định tâm và bán kính của đường tròn $(C)\colon x^2+y^2+2x-6y-15=0$
\choice
{$I(1;3);R=5$}
{$I(1;-3);R=5$}
{\True $I(-1;3);R=5$}
{$I(-1;3);R=-5$}
\end{ex}
%Câu 32
\begin{ex}
Tìm tất cả các giá trị của tham số $m$ để phương trình $x^2+y^2-4mx+2(m-1)y+6m^2-5m+3=0$ là phương trình của một đường tròn trong mặt phẳng $(Oxy)$ 
\choice
{\True $1<m<2$}
{$\hoac{& m<1 \\& m>2}$}
{$\hoac{& m<-2 \\& m>-1}$}
{$-2<m<-1$}
\end{ex}
%Câu 33
\begin{ex}
Trong mặt phẳng tọa độ $Oxy$, cho hai điểm $I(-1;2);A(1;-1)$. Phương trình đường tròn tâm $I$ và đi qua điểm $A$ là:
\choice
{${{(x+1)}^2}+{{(y-1)}^2}=13$}
{${{(x-1)}^2}+{{(y+2)}^2}=5$}
{\True ${{(x+1)}^2}+{{(y-2)}^2}=13$}
{${{(x-1)}^2}+{{(y+2)}^2}=20$}
\end{ex}
%Câu 34
\begin{ex}
Đường elip $\dfrac{x^2}{16}+\dfrac{y^2}{4}=1$ cắt trục hoành tại điểm có hoành độ bằng
\choice
{\True $8$}
{$16$}
{$4$}
{$2$}
\end{ex}
%Câu 35
\begin{ex}
Phương trình chính tắc của đường elip đi qua điểm $(5;0)$ và có tiêu cự bằng $2\sqrt{5}$ là
\choice
{$\dfrac{x^2}{25}+\dfrac{y^2}{5}=1$}
{\True $\dfrac{x^2}{25}+\dfrac{y^2}{20}=1$}
{$\dfrac{x^2}{25}-\dfrac{y^2}{5}=1$}
{$\dfrac{x^2}{25}-\dfrac{y^2}{20}=1$}
\end{ex}

\noindent{\bf\fontfamily{qag}\selectfont\color{violet}B. PHẦN TỰ LUẬN}
%Câu 36
\begin{ex}
Viết phương trình chính tắc của đường elip đi qua điểm $(5;0)$ và có tiêu cự bằng $2\sqrt{5}$
\end{ex}
%Câu 37
\begin{ex}
Viết phương trình tiếp tuyến đường tròn $(C)\colon x^2+y^2+4x-5y+6=0$ tại giao điểm với trục $Oy$
\end{ex}
%Câu 38
\begin{ex}
Trong một giải thi đấu cờ vua gồm có cả nam và nữ vận động viên tham gia, mỗi vận động viên phải chơi hai ván cờ với từng vận động viên còn lại. Biết rằng có hai vận động viên nữ tham gia giải và số ván cờ vận động viên nam chơi với nhau hơn số ván cờ họ chơi với vận động viên nữ là $66$. Hỏi có bao nhiêu vận động viên tham dự giải và số ván cờ tất cả các vận động viên đã chơi là bao nhiêu?
\end{ex}
%Câu 39
\begin{ex}
Trong mặt phẳng tọa độ $Oxy$, cho tam giác $ABC$ có $A(2;3)$. Các điểm $I(-2;0),K(0;1)$ lần lượt là tâm đường tròn ngoại tiếp và tâm đường tròn nội tiếp của tam giác $ABC$. Viết phương trình tổng quát của đường thẳng $BC$
\end{ex}


\Closesolutionfile{ans}