\section*{ÔN TẬP KIỂM TRA CUỐI KÌ 2 - ĐỀ 05}
\setcounter{ex}{0}\setcounter{bt}{0}
\noindent{\bf\fontfamily{qag}\selectfont\color{violet}A. PHẦN TRẮC NGHIỆM}
\Opensolutionfile{ans}[ans/ansBTTeXCK25]
%Câu 1
\begin{ex}
Một cửa hàng có $10$ bó hoa ly, $14$ bó hoa huệ, $6$ bó hoa lan. Một bạn muốn mua một bó hoa tại cửa hàng này. Hỏi bạn đó có bao nhiêu sự lựa chọn?
\choice
{$140$}
{$30$}
{$24$}
{$840$}
\end{ex}
%Câu 2
\begin{ex}
Có $10$ cặp vợ chồng đi dự tiệc. Tổng số cách chọn một người đàn ông và một người đàn bà trong bữa tiệc phát biểu ý kiến sao cho hai người đó không là vợ chồng?
\choice
{$100$}
{$91$}
{$10$}
{$90$}
\end{ex}
%Câu 3
\begin{ex}
Một tổ có $10$ học sinh. Hỏi có bao nhiêu cách chọn ra $2$ học sinh từ tổ đó để giữ hai chức vụ tổ trưởng và tổ phó.
\choice
{$A_{10}^2$}
{$C_{10}^2$}
{$A_{10}^8$}
{$10^2$}
\end{ex}
%Câu 4
\begin{ex}
Tính số chỉnh hợp chập $3$ của $5$ phần tử.
\choice
{$10$}
{$720$}
{$60$}
{$6$}
\end{ex}
%Câu 5
\begin{ex}
Trong kì thi TN THPT Quốc gia năm $2022$ tại một điểm thi có $5$ sinh viên tình nguyện được phân công trực hướng dẫn thí sinh ở $5$ vị trí khác nhau. Yêu cầu mỗi vị trí có đúng $1$ sinh viên. Hỏi có bao nhiêu cách phân công vị trí trực cho 5 sinh viên đó?
\choice
{$120$}
{$25$}
{$10$}
{$24$}
\end{ex}
%Câu 6
\begin{ex}
Có thể tạo thành bao nhiêu véctơ khác vectơ - không từ hai mươi điểm phân biệt trên mặt phẳng?
\choice
{$20!$}
{$C_{20}^2$}
{$20$}
{$A_{20}^2$}
\end{ex}
%Câu 7
\begin{ex}
Số cách chọn hai học sinh từ mười học sinh để tham gia thi văn nghệ là
\choice
{$45$}
{$90$}
{$20$}
{$100$}
\end{ex}
%Câu 8
\begin{ex}
Một nhóm có năm học sinh nam và sáu học sinh nữ. Hỏi có bao nhiêu cách chọn bốn học sinh trong đó có ít nhất một học sinh nữ đi tham gia giải chạy việt dã?
\choice
{$410$}
{$5040$}
{$205$}
{$504$}
\end{ex}
%Câu 9
\begin{ex}
Số hạng tự do trong khai triển ${{(2x-1)}^8}$ là
\choice
{$1$}
{$-1$}
{$2^8$}
{$2$}
\end{ex}
%Câu 10
\begin{ex}
Viết số quy tròn của số 410237 đến hàng trăm.
\choice
{410200}
{410000}
{410300}
{410240}
\end{ex}
%Câu 11
\begin{ex}
Chiều dài của một cây cầu là $l=1547{,}25m\pm 0{,}01m$. Hãy cho biết số quy tròn của $l$.
\choice
{$1547{,}3m$}
{$1547{,}2m$}
{$1547m$}
{$1548m$}
\end{ex}
%Câu 12
\begin{ex}
Thống kê điểm kiểm tra một tiết môn Toán của một nhóm gồm $12$ học sinh lớp $10\text{D}$ ta được $7;5;7;7;7;7;5;8;9;6;10;10$. Tìm mốt của mẫu số liệu.
\choice
{$M_0=7$}
{$M_0=8$}
{$M_0=5$}
{$M_0=9$}
\end{ex}
%Câu 13
\begin{ex}
Hãy tìm khoảng biến thiên của mẫu số liệu thống kê sau: 22 24 33 17 11 4 18 87 72 30
\choice
{11}
{33}
{87}
{83}
\end{ex}
%Câu 14
\begin{ex}
Hãy tính khoảng tứ phân vị của mẫu số liệu: $15;20;1;2;4;3;7;5$.
\choice
{$8{,}5$}
{$4{,}5$}
{$13{,}5$}
{Đáp án khác}
\end{ex}
%Câu 15
\begin{ex}
Nhiệt độ cao nhất của Hà Nội trong $7$ ngày liên tiếp trong tháng ba được ghi lại là: $25;26;28;31;33;33;27$ (Độ C). Độ lệch chuẩn của mẫu số liệu thuộc khoảng nào
\choice
{$\left(3;4\right)$}
{$\left(1;3\right)$}
{$\left[6;11\right]$}
{$\left(0;\dfrac{3}{4}\right)$}
\end{ex}
%Câu 16
\begin{ex}
Chọn ngẫu nhiên hai số khác nhau từ $15$ số nguyên dương đầu tiên. Xác suất để chọn được hai số có tổng là một số lẻ là:
\choice
{$\dfrac{1}{7}$}
{$\dfrac{8}{15}$}
{$\dfrac{4}{15}$}
{$\dfrac{1}{14}$}
\end{ex}
%Câu 17
\begin{ex}
Từ một hộp chứa $16$ quả cầu gồm $7$ quả màu đỏ và $9$ quả màu xanh, lấy ngẫu nhiên đồng thời hai quả. Xác suất để lấy được hai quả có màu khác nhau bằng
\choice
{$\dfrac{7}{40}$}
{$\dfrac{21}{40}$}
{$\dfrac{3}{10}$}
{$\dfrac{2}{15}$}
\end{ex}
%Câu 18
\begin{ex}
Một tổ có 7 nam và 3 nữ. Chọn ngẫu nhiên đồng thời 2 người. Xác suất để 2 người được chọn có ít nhất một nữ bằng
\choice
{$\dfrac{8}{15}$}
{$\dfrac{7}{15}$}
{$\dfrac{1}{15}$}
{$\dfrac{2}{15}$}
\end{ex}
%Câu 19
\begin{ex}
Rút ra một lá bài từ bộ bài $52$ lá. Xác suất để được lá bích là
\choice
{$\dfrac{1}{13}$}
{$\dfrac{12}{13}$}
{$\dfrac{3}{4}$}
{$\dfrac{1}{4}$}
\end{ex}
%Câu 20
\begin{ex}
Gieo ngẫu nhiên hai con súc sắc cân đối và đồng chất. Xác suất để sau hai lần gieo kết quả như nhau là
\choice
{$\dfrac{5}{36}$}
{$\dfrac{1}{6}$}
{$\dfrac{1}{2}$}
{$1$}
\end{ex}
%Câu 21
\begin{ex}
Một đội gồm $5$ nam và $8$ nữ. Lập một nhóm gồm 4 người hát tốp ca, xác suất để trong 4 người được chọn có ít nhất $3$ nữ là
\choice
{$\dfrac{73}{143}$}
{$\dfrac{56}{143}$}
{$\dfrac{70}{143}$}
{$\dfrac{87}{143}$}
\end{ex}
%Câu 22
\begin{ex}
Trong hệ trục tọa độ $Oxy$,cho $\vec{u}=3\vec{i}+4\vec{j}$. Tọa độ của $\vec{u}$ là
\choice
{$(3;-4)$}
{$(4;3)$}
{$(3;4)$}
{$(-4;3)$}
\end{ex}
%Câu 23
\begin{ex}
Trong hệ trục tọa độ $Oxy$, tính khoảng cách giữa hai điểm $M(-3;1)$ và $N(1;2)$.
\choice
{$2\sqrt{17}$}
{$17\sqrt{2}$}
{$\sqrt{17}$}
{$17$}
\end{ex}
%Câu 24
\begin{ex}
Trong hệ trục tọa độ $Oxy$, cho đường thẳng $d\colon 4x-2y-1=0$. Vectơ nào sau đây là vectơ pháp tuyến của đường thẳng d
\choice
{$\vec{n}=(4;-2)$}
{$\vec{n}=(4;2)$}
{$\vec{n}=(4;-1)$}
{$\vec{n}=(-2;-1)$}
\end{ex}
%Câu 25
\begin{ex}
Cho đường thẳng $(d)$ có phương trình $\heva{& x=1-t \\& y=3+2t} $. Khi đó, đương thẳng $(d)$ có 1 véc tơ pháp tuyến là:
\choice
{$\vec{n}=(-1;2)$}
{$\vec{n}=(1;2)$}
{$\vec{n}=(2;1)$}
{$\vec{n}=(2;-1)$}
\end{ex}
%Câu 26
\begin{ex}
Cho $\triangle ABC$ có $A(2;-1); B(4;5); C(-3;2)$Viết phương trình tổng quát của đường cao$AH$.
\choice
{$7x + 3y-11 = 0$}
{$3x + 7y + 1 = 0$}
{$7x + 3y +11 = 0$}
{$-7x +\text{ 3}y + 11 = 0$}
\end{ex}
%Câu 27
\begin{ex}
Đường Thẳng $\triangle \colon ax+by-3=0(a,b\in \mathbb{N})$ đi qua điểm $N(1;1)$ và cách điểm $M(2;3)$ một khoảng bằng $\sqrt{5}$. Khi đó $a-2b$ bằng
\choice
{5}
{2}
{4}
{0}
\end{ex}
%Câu 28
\begin{ex}
Trong mặt phẳng tọa độ $Oxy$, khoảng cách từ điểm $M(15;1)$ đến đường thẳng $\triangle \colon x-3y-2=0$ là
\choice
{$d\left(M,\triangle\right)=\sqrt{2}$}
{$d\left(M,\triangle\right)=\sqrt{10}$}
{$d\left(M,\triangle\right)=\dfrac{\sqrt{2}}{2}$}
{$d\left(M,\triangle\right)=\dfrac{\sqrt{10}}{10}$}
\end{ex}
%Câu 29
\begin{ex}
Trong mặt phẳng tọa độ $Oxy$, tìm góc giữa $2$ đường thẳng $\triangle _1$: $2x-y+6=0$ và
$\triangle _2$: $-x+3y+5=0$.
\choice
{${{45}^\circ}$}
{$0^\circ$}
{${{60}^\circ}$}
{${{90}^\circ}$}
\end{ex}
%Câu 30
\begin{ex}
Trong mặt phẳng tọa độ $Oxy$, tìm tọa độ tâm $I$ và bán kính $R$ của đường tròn $(C)\colon {{(x-1)}^2}+{{(y+5)}^2}=9$.
\choice
{$I(-1;5),R=3$}
{$I(-1;5),R=9$}
{$I\left(1;-5\right),R=9$}
{$I\left(1;-5\right),R=3$}
\end{ex}
%Câu 31
\begin{ex}
Đường tròn $x^2+y^2-5y=0$ có bán kính bằng bao nhiêu?
\choice
{$\sqrt{5}$}
{$25$}
{$\dfrac{5}{2}$}
{$\dfrac{25}{2}$}
\end{ex}
%Câu 32
\begin{ex}
Đường tròn tâm $I(a;b)$ và bán kính $R$ có dạng:
\choice
{${{(x+a)}^2}+{{(y+b)}^2}=R^2$}
{${{(x-a)}^2}+{{(y-b)}^2}=R^2$}
{${{(x-a)}^2}+{{(y+b)}^2}=R^2$}
{${{(x+a)}^2}+{{(y-b)}^2}=R^2$}
\end{ex}
%Câu 33
\begin{ex}
Tìm tọa độ tâm đường tròn đi qua $3$ điểm$A(0;5),B(3;4),C(-4;3)$.
\choice
{$(-6;-2)$}
{$(-1;-1)$}
{$(3;1)$}
{$(0;0)$}
\end{ex}
%Câu 34
\begin{ex}
Trong mặt phẳng $Oxy$, phương trình nào sau đây là phương trình chính tắc của một elip?
\choice
{$\dfrac{x^2}{2}-\dfrac{y^2}{3}=1$}
{$\dfrac{x^2}{9}-\dfrac{y^2}{8}=1$}
{$\dfrac{x}{9}+\dfrac{y}{8}=1$}
{$\dfrac{x^2}{9}+\dfrac{y^2}{1}=1$}
\end{ex}
%Câu 35
\begin{ex}
Viết phương trình chính tắc của Parabol biết đường chuẩn có phương trình $x+1=0$.
\choice
{$y^2=2x$}
{$y^2=4x$}
{$y=4x^2$}
{$y^2=8x$}
\end{ex}

\noindent{\bf\fontfamily{qag}\selectfont\color{violet}B. PHẦN TỰ LUẬN}
%Câu 36
\begin{ex}
Viết phương trình chính tắc của đường elip đi qua điểm $(5;0)$ và có tiêu cự bằng $2\sqrt{5}$
\end{ex}
%Câu 37
\begin{ex}
Viết phương trình tiếp tuyến đường tròn $(C)\colon x^2+y^2+4x-5y+6=0$ tại giao điểm với trục $Oy$
\end{ex}
%Câu 38
\begin{ex}
Lập được bao nhiêu số tự nhiên có 5 chữ số từ các số 1, 2, 3, 4 biết rằng chữ số 1 có mặt đúng hai lần, các chữ số còn lại mỗi số có mặt đúng một lần
\end{ex}
%Câu 39
\begin{ex}
Trong mặt phẳng Oxy, cho tam giác $ABC$ có đỉnh $A(4;6)$, trực tâm $H(4;2)$. Đường thẳng chứa cạnh $BC$ có phương trình $y=1$. Biết cạnh $BC$ có trung điểm $M(2;1)$ và đỉnh $B$ có hoành độ dương. Viết phương trình tổng quát của đường thẳng chứa cạnh $AB$
\end{ex}


\Closesolutionfile{ans}