\section*{ÔN TẬP KIỂM TRA CUỐI KÌ 2 - ĐỀ 06}
\setcounter{ex}{0}\setcounter{bt}{0}
\noindent{\bf\fontfamily{qag}\selectfont\color{violet}A. PHẦN TRẮC NGHIỆM}
\Opensolutionfile{ans}[ans/ansBTTeXCK26]
%Câu 1
\begin{ex}
Có bao nhiêu số hạng trong khai triển nhị thức $(x+3)^4$?
\choice
{7}
{6}
{5}
{4}
\end{ex}
%Câu 2
\begin{ex}
Khai triển biểu thức ${{(a+b)}^5}$ thành tổng các đơn thức, ta được kết quả là
\choice
{${{(a+b)}^5}=a^5+b^5$}
{${{(a+b)}^5}=a^5+a^4b+a^3b^2+a^2b^3+ab^4+b^5$}
{${{(a+b)}^5}=a^5+5a^4b+10a^3b^2+10a^2b^3+5ab^4+b^5$}
{${{(a+b)}^5}=a^5b^5+5a^4b^4+10a^3b^3+20a^2b^2+5ab^{}+1$}
\end{ex}
%Câu 3
\begin{ex}
Khai triển biểu thức ${{(a+bx)}^4}$, viết các số hạng theo thứ tự bậc của $x$ tăng dần, nhận được biểu thức gồm hai số hạng đầu tiên là $16-96x$. Tính $S=a^2+b^2$
\choice
{$S=2$}
{$S=12$}
{$S=9$}
{$S=13$}
\end{ex}
%Câu 4
\begin{ex}
Biến cố chắc chắn xảy ra của phép thử có xác suất bằng bao nhiêu?
\choice
{$1$}
{$0$}
{$0{,}5$}
{$0{,}99$}
\end{ex}
%Câu 5
\begin{ex}
Trong thực đơn của một nhà hàng có $14$ món ăn mặn, $10$ món ăn nhẹ và $7$ thứ nước uống. Chọn ngẫu nhiên $1$ món ăn mặn, $1$ món ăn nhẹ và $1$ đồ uống. Tính số phần tử của không gian mẫu, biết rằng trong tất cả các món ăn và đồ uống của nhà hàng đó không có món nào kị món nào.
\choice
{$980$}
{$21$}
{$140$}
{$3$}
\end{ex}
%Câu 6
\begin{ex}
Gieo một con súc sắc cân đối đồng chất hai lần. Tính xác suất để biến cố có tổng hai mặt bằng 8.
\choice
{$\dfrac{1}{6}$}
{$\dfrac{5}{36}$}
{$\dfrac{1}{9}$}
{$\dfrac{1}{2}$}
\end{ex}
%Câu 7
\begin{ex}
Một tổ có 7 nam và 3 nữ. Chọn ngẫu nhiên 2 người. Tính xác suất sao cho 2 người được chọn đều là nữ.
\choice
{$\dfrac{1}{15}$}
{$\dfrac{2}{15}$}
{$\dfrac{7}{15}$}
{$\dfrac{8}{15}$}
\end{ex}
%Câu 8
\begin{ex}
Một hộp đựng 9 chiếc thẻ được đánh số từ 1 đến 9. Rút ngẫu nhiên hai thẻ và nhân hai số ghi trên hai thẻ với nhau. Xác suất để tích hai số ghi trên hai thẻ là số lẻ là:
\choice
{${\dfrac{1}{9}}$}
{${\dfrac{5}{18}}$}
{${\dfrac{3}{18}}$}
{${\dfrac{7}{18}}$}
\end{ex}
%Câu 9
\begin{ex}
Trong mặt phẳng tọa độ $Oxy$, cho $\vec{a}=(1;2);\vec{b}=(-2;3)$. Chọn khẳng định đúng trong các khẳng định sau đây.
\choice
{$\vec{a}-\vec{b}=(3;-1)$}
{$\vec{a}+\vec{b}=(3;-1)$}
{$2\vec{a}-\vec{b}=(0;7)$}
{$\vec{a}+2\vec{b}=(0;7)$}
\end{ex}
%Câu 10
\begin{ex}
Trong mặt phẳng tọa độ $Oxy$, cho vectơ $\vec{OM}=\vec{i}-2\vec{j}$. Khi đó tọa độ của điểm $M$ là
\choice
{$M(1;-2)$}
{$M(1;2)$}
{$M(0;-2)$}
{$M(-1;2)$}
\end{ex}
%Câu 11
\begin{ex}
Trong mặt phẳng với hệ trục $Oxy$ cho hai điểm $B(3;-2),C(-1;1)$. Biết $D(x;y)$ là điểm thỏa mãn $3\vec{BD}-2\vec{BC}=\vec{0}$. Tính $M=3x+2y$.
\choice
{$2$}
{$-2$}
{$-1$}
{$1$}
\end{ex}
%Câu 12
\begin{ex}
Viết phương trình tổng quát của đường thẳng $(d)$, biết $(d)$ đi qua $A\left(2{,}1\right)$ và có vecto chỉ phương $\vec{u}=(3;-2)$.
\choice
{$2x+3y-7=0$}
{$3x-2y-4=0$}
{$-3x+2y+7=0$}
{$3x+2y-4=0$}
\end{ex}
%Câu 13
\begin{ex}
Cho PTĐT $\triangle \colon \heva{& x=3-t \\& y=2+3t}$. Vectơ chỉ phương của đường thẳng $\triangle $ có tọa độ là
\choice
{$(1;3)$}
{$(-2;6)$}
{$\left(3\,;2\right)$}
{$(4;6)$}
\end{ex}
%Câu 14
\begin{ex}
Trong mặt phẳng tọa độ $Oxy$, khoảng cách từ điểm $A(0;-2)$ đến đường thẳng $\triangle \colon 3x-4y+1=0$ là
\choice
{$\dfrac{9}{5}$}
{$\dfrac{7}{5}$}
{$\dfrac{3}{5}$}
{$\dfrac{12}{5}$}
\end{ex}
%Câu 15
\begin{ex}
Trong mặt phẳng toạ độ $Oxy$, xác định vị trí tương đối của hai đường thẳng sau: $\left(d_1\right)\colon 2x-3y+1=0$ và $\left(d_2\right)\colon -4x+6y-1=0$
\choice
{Song song}
{Trùng nhau}
{Vuông góc}
{Cắt nhau nhưng không vuông góc}
\end{ex}
%Câu 16
\begin{ex}
Phương trình tổng quát của đường thẳng $d$ đi qua $A(1;2)$ và vuông góc với đường thẳng $\triangle \colon 2x-y+4=0$ là
\choice
{$-x+2y-5=0$}
{$x+2y-3=0$}
{$x+2y=0$}
{$x+2y-5=0$}
\end{ex}
%Câu 17
\begin{ex}
Đường thẳng đi qua hai điểm $A(1;1)$ và $B(2;2)$ có PTTS là:
\choice
{$\heva{& x=2+t \\& y=2+2t}\cdot $}
{$\heva{& x=1+t \\& y=1+2t}\cdot $}
{$\heva{& x=2+2t \\& y=1+t}\cdot $}
{$\heva{& x=t \\& y=t}\cdot $}
\end{ex}
%Câu 18
\begin{ex}
Trong mặt phẳng với hệ trục tọa độ $Oxy$, đường tròn có tâm $M(2;1)$ và đi qua điểm $N(1;2)$ có phương trình là
\choice
{$(x-1)^2+(y-2)^2=4$}
{${{(x-2)}^2}+{{(y-1)}^2}=2$}
{${{(x-1)}^2}+{{(y-2)}^2}=2$}
{${{(x+2)}^2}+{{(y+1)}^2}=2$}
\end{ex}
%Câu 19
\begin{ex}
Trong mặt phẳng với hệ tọa độ $Oxy$, tọa độ tâm $I$ và bán kính $R$ của đường tròn $(C)\colon x^2+y^2-6x+2y+6=0$ là
\choice
{$I(-3;1),\ R=2$}
{$I(3;-1),\ R=4$}
{$I(-3;1),\ R=4$}
{$I(3;-1),\ R=2$}
\end{ex}
%Câu 20
\begin{ex}
Cho đường tròn $(C)\colon \ {{(x+1)}^2}+{{(y-3)}^2}=5$. Phương trình tiếp tuyến của $(C)$ song song với đường thẳng $\triangle \colon x+2y-15=0$.
\choice
{$x+2y-1=0$}
{$x+2y+10=0$}
{$x+2y=0$ hoặc $x+2y-10=0$}
{$x+2y-2=0$}
\end{ex}
%Câu 21
\begin{ex}
Viết phương trình đường tròn đi qua ba điểm $A\left(-2\,;-1\right),B(3;-2),C(-1;4)$.
\choice
{$x^2+y^2+2x-2y+11=0$}
{$x^2+y^2+2x+2y+11=0$}
{$x^2+y^2-2x+2y-11=0$}
{$x^2+y^2-2x-2y-11=0$}
\end{ex}
%Câu 22
\begin{ex}
Trong các phương trình sau, phương trình nào là phương trình chính tắc của đường elip?
\choice
{$\dfrac{x^2}{3^2}+\dfrac{y^2}{3^2}=1$}
{$\dfrac{x^2}{4^2}+\dfrac{y^2}{3^2}=1$}
{$\dfrac{x^2}{3^2}+\dfrac{y^2}{4^2}=1$}
{$\dfrac{x^2}{4^2}+\dfrac{y^2}{3^2}=-1$}
\end{ex}
%Câu 23
\begin{ex}
An có $5$ quyển truyện tranh và 8 quyển truyện ngắn (các quyển sách khác nhau từng đôi một). An đồng ý cho Bình mượn một quyển để đọc. Hỏi Bình có bao nhiêu cách lựa chọn sách để mượn?
\choice
{$25$}
{$1$}
{$3$}
{$13$}
\end{ex}
%Câu 24
\begin{ex}
Bạn Nam muốn mua một cây bút mực và một cây bút chì. Các cây bút mực có $7$ màu khác nhau, các cây bút chì có $4$ màu khác nhau. Như vậy bạn Nam có bao nhiêu cách chọn?
\choice
{$11$}
{$55$}
{$28$}
{$2$}
\end{ex}
%Câu 25
\begin{ex}
Từ tập $A=\left\{ 1;2;3;4;5;6 \right\}$ có thể lập được bao nhiêu số tự nhiên nhỏ hơn 100?
\choice
{$30$}
{$42$}
{$36$}
{$45$}
\end{ex}
%Câu 26
\begin{ex}
Từ các chữ số $1\,,2\,,3\,,4\,,5$ có thể lập được bao nhiêu số tự nhiên có 5 chữ số khác nhau đôi một.
\choice
{$25$}
{$120$}
{$32$}
{$60$}
\end{ex}
%Câu 27
\begin{ex}
Với $k$ và $n$ là hai số nguyên dương tùy ý thỏa mãn $k\le n$, công thức tính số các chỉnh hợp chập $k$ của $n$ phần tử là
\choice
{$\dfrac{n!}{(n-k)!}$}
{$\dfrac{n!}{(n-k)!k!}$}
{$\dfrac{k!}{(n-k)!n!}$}
{$\dfrac{k!}{(n-k)!}$}
\end{ex}
%Câu 28
\begin{ex}
Cho đa giác lồi $n$ đỉnh $\left(n>3\right)$. Số tam giác có $3$ đỉnh là $3$ đỉnh của đa giác đã cho là
\choice
{$A_n^3$}
{$\dfrac{C_n^3}{3!}$}
{$C_n^3$}
{$n!$}
\end{ex}
%Câu 29
\begin{ex}
Số trung bình của mẫu số liệu $x_1,x_2,\ldots,x_n$ là
\choice
{$\overline{x}=\dfrac{x_1+x_2+\ldots+x_n}{n}$}
{$\overline{x}=x_1+x_2+\ldots+x_n$}
{$\overline{x}=\dfrac{x_1+x_2+\ldots+x_n}{n^2}$}
{$\overline{x}=\dfrac{x_1^2+x_2^2+\ldots+x_n^2}{n^2}$}
\end{ex}
%Câu 30
\begin{ex}
Tứ phân vị của mẫu số liệu: 43, 46, 46, 40, 50, 48, 47, 42 là
\choice
{$Q_1=42{,}5;Q_2=46;Q_3=47{,}5$}
{$Q_1=44;Q_2=46;Q_3=48$}
{$Q_1=44;Q_2=46;Q_3=49$}
{$Q_1=45{,}5;Q_2=46;Q_3=47$}
\end{ex}
%Câu 31
\begin{ex}
Khoảng biến biên của mẫu số liệu: 3{,}5; 4{,}2; 5{,}0; 6{,}7; 6{,}8; 7{,}8; 8{,}5 bằng
\choice
{5{,}0}
{6{,}7}
{3{,}6}
{6{,}8}
\end{ex}
%Câu 32
\begin{ex}
Tìm khoảng tứ phân vị của mẫu số liệu sau: 5{,}25; 5{,}26; 8{,}52; 18{,}55; 30{,}02.
\choice
{19{,}03}
{8{,}52}
{5{,}25}
{13{,}52}
\end{ex}
%Câu 33
\begin{ex}
Một nhóm học sinh gồm $4$ học sinh nam và $5$ học sinh nữ. Hỏi có bao nhiêu cách sắp xếp $9$ học sinh trên thành một hàng ngang sao cho không có $2$ học sinh nam nào đứng cạnh nhau?
\choice
{$362880$}
{$1800$}
{$43200$}
{$2880$}
\end{ex}
%Câu 34
\begin{ex}
Trong kho đèn trang trí đang còn $5$ bóng đèn loại I, $7$ bóng đèn loại II, các bóng đèn đều khác nhau về màu sắc và hình dáng. Lấy ra ngẫu nhiên $5$ bóng đèn. Tính xác suất để 5 bóng đèn lấy ra có số bóng đèn loại I nhiều hơn số bóng đèn loại II?
\choice
{$\dfrac{41}{132}$}
{$\dfrac{35}{792}$}
{$\dfrac{11}{36}$}
{$\dfrac{245}{792}$}
\end{ex}
%Câu 35
\begin{ex}
Trong một hộp kín có 100 thẻ giống nhau được đánh số từ 1 đến 100. Bốc ngẫu nhiên 3 thẻ. Tính xác suất để 3 thẻ bốc được sao cho có ít nhất 2 thẻ mang số chia hết cho 3?
\choice
{$\dfrac{928}{3675}$}
{$\dfrac{124}{3675}$}
{$\dfrac{2747}{3675}$}
{$\dfrac{11}{136}$}
\end{ex}

\noindent{\bf\fontfamily{qag}\selectfont\color{violet}B. PHẦN TỰ LUẬN}
%Câu 36
\begin{ex}
Viết phương trình chính tắc của đường elip đi qua điểm $(0;4)$ và có tiêu điểm bằng $F_2(\dfrac92;0)$
\end{ex}
%Câu 37
\begin{ex}
Viết phương trình tiếp tuyến đường tròn $(C)\colon x^2+y^2-4x-7y+6=0$ tại điểm có hoành độ bằng $4$.
\end{ex}
%Câu 36
\begin{ex}
Một lớp học có 30 em học sinh trong đó có 5 cặp yêu nhau. Hỏi có bao nhiêu cách chọn 5 em học sinh trong lớp sao cho không có cặp đôi nào?
\end{ex}
%Câu 37
\begin{ex}
Trong mặt phẳng với hệ tọa độ $Oxy$, hãy xác định tọa độ đỉnh $C$ của $\triangle ABC$ biết rằng hình chiếu của $C$ lên đường thẳng $AB$ là điểm $H(-1;-1)$, đường phân giác trong của góc $A$ có phương trình $x-y+2=0$ và đường cao kẻ từ $B$ có phương trình $4x+3y-1=0$.
\loigiai{
PTĐT $d$ qua $H(-1;-1)$ và vuông góc với $\Delta \colon x-y+2=0$ có dạng
$1(x+1)+1(y+1)=0\Leftrightarrow x+y+2=0$ 
Gọi $I$ là giao điểm của $d$ và $\triangle $, nên tọa độ của $I$ là nghiệm của hệ phương trình:
$$\heva{
 & x+y+2=0 \\
 & x-y+2=0} \Rightarrow I(-2;0) $$
Gọi $K$ là điểm đối xứng của $H$ qua $\triangle $ thì $K(-3;1)$.
$AC$ qua $K$ và vuông góc với đường cao: $4x+3y-1=0$.
Phương trình $AC\colon 3(x+3)-4(y-1)=0\Leftrightarrow 3x-4y+13=0$.
Tọa độ điểm $A$ là nghiệm của hệ phương trình:
$$\heva{
 &3x-4y+13=0 \\
 &x-y+2\quad =0 }\Rightarrow A(5;7)$$
$CH$ qua $H$ và có véc-tơ pháp tuyến $\vec{HA}=2\vec{n}$ với $\vec{n}=(3;4)$.
Phương trình $CH\colon 3(x+1)+4(y+1)=0$.
Tọa độ $C$ là nghiệm của hệ phương trình:
$$\heva{
 &3x+4y+7=0 \\
 &3x-4y+13=0
}\Rightarrow C\left(-\dfrac{10}{3};\dfrac{3}{4}\right) $$

}
\end{ex}



\Closesolutionfile{ans}