\begin{name}
	{\tenchude}
	{TOÁN 10}
	{LỚP TOÁN THẦY PHÁT}
	{Thời gian: 90 phút - Không kể thời gian phát đề}
\end{name}
\Opensolutionfile{ansbook}[ans/ansbookDe2]
\TN
\Opensolutionfile{ans}[ans/ansDe2-TN1]
\begin{ex}%[0D3H1-5]
	Cho hàm số $ f(x) $ xác định trên $ \mathbb{R} $. Biết rằng hàm số $ y = g(x) = f(x) - f(-x) $ đồng biến trên khoảng $(2;5)$. Hỏi hàm số $ g(x) $ luôn đồng biến trên khoảng nào dưới đây?
	\choice
	{$(1;4)$}
	{$(-\infty;-5)$}
	{$\mathbb{R}$}
	{\True $(-4;-3)$}
	\loigiai{
		Ta có $g(-x)=f(-x)-f(x)=-(f(x)-f(-x))=-g(x)$.\\
		Suy ra $g(x)$ là hàm số lẻ.\\
		Khi đó đồ thị hàm số đối xứng qua gốc toạ  độ $O$.\\
		Vì hàm số đồng biến trên $(2;5)$ nên hàm số cũng đồng biến $(-5;-2)$.
	}
\end{ex}

\begin{ex}%[Tex hoá đề CK2-Form 2025-Đợt 2- Trần Hữu Khải]%[0D3N2-1]
	Hàm số nào dưới đây là hàm số bậc hai?
	\choice
	{$y=x^4-x+5$}
	{$y=\dfrac{1}{x^2}$}
	{\True $y=-2x^2+7$}
	{$y=4\left(\dfrac{1}{x}\right)+\dfrac{1}{x}-6$}
	\loigiai{
		Từ định nghĩa hàm số bậc hai ta có hàm số bậc hai là $y=-2x^2+7$.
	}
\end{ex}

\begin{ex}%[0-TK-HK1-CT-3-2425]%[VN-MT-9, Chương Ngô Toàn Phúc]%[0D3H2-2]
	Cho hàm số $y=-x^2+6x-1$. Hàm số đồng biến trên khoảng nào dưới đây?
	\choice
	{\True $(-\infty;3)$}
	{$(3;+\infty)$}
	{$(-\infty;6)$}
	{$(6;+\infty)$}
	\loigiai{Ta có $a=-1<0$, và hoành độ đỉnh $x=\dfrac{-b}{2a}=\dfrac{-6}{2\cdot (-1)}=3$.\\
		Suy ra hàm số đồng biến trên khoảng $(-\infty;3)$.}
\end{ex}

\begin{ex}%[De-chuan-hoa-so-13]%[Huỳnh Quy]%[0D6H1-1]
	Làm tròn số $8316{,}2$ đến hàng chục. Khi đó sai số tuyệt đối của số quy tròn là
	\choice
	{$3{,}6$}
	{$6{,}2$}
	{$3{,}16$}
	{\True $3{,}8$}
	\loigiai{
	Số quy tròn của số $8316{,}2$ đến hàng chục là $8320$.\\
	Do đó sai số tuyệt đối của số quy tròn là $|8320-8316{,}2|=3{,}8$.
	}
\end{ex}

\begin{ex}%[H]%[BG10-3in1, Phạm Ngọc Trung]%[0D6H4-2]
	Cho mẫu số liệu sau
	\begin{center}
		\begin{tabular}{ccccccccc}
			$10$ & $13$ & $15$ & $2$ & $10$ & $19$ & $2$ & $5$ & $7$
		\end{tabular}
	\end{center}
	Giá trị các tứ phân vị của mẫu số liệu sau lần lượt là
	\choice
	{$Q_1=3$, $Q_2=5$, $Q_3=14$}
	{$Q_1=4$, $Q_2=10$, $Q_3=14$}
	{$Q_1=3$, $Q_2=10$, $Q_3=15$}
	{\True $Q_1=3$, $Q_2=10$, $Q_3=14$}
	\loigiai{
		Mẫu số liệu trên được sắp xếp theo thứ tự tăng dần
		\begin{center}
			\begin{tabular}{ccccccccc}
				$2$ & $2$ & $4$ & $7$ & $10$ & $10$ & $13$ & $15$ & $19$
			\end{tabular}
		\end{center}
		Trung vị của mẫu số liệu trên là $10$.\\
		Trung vị của dãy $2$, $2$, $4$, $7$ là $\dfrac{2+4}{2}=3$.\\
		Trung vị của dãy $10$, $13$, $15$, $19$ là $\dfrac{13+15}{2}=14$.\\
		Vậy $Q_1=3$, $Q_2=10$, $Q_3=14$.
	}
\end{ex}

\begin{ex}%[0D0N1-2]
	Gieo con súc sắc hai lần. Biến cố $A$ là biến cố để sau hai lần gieo có ít nhất một mặt $6$ chấm xuất hiện là
	\choice
	{$A=\{(1;6),(2;6),(3;6),(4;6),(5;6)\}$}
	{$A=\{(1;6),(2;6),(3;6),(4;6),(5;6),(6;6)\}$}
	{\True $A=\{(1;6),(2;6),(3;6),(4;6),(5;6),(6;6),(6;1),(6;2),(6;3),(6;4),(6;5)\}$}
	{$A=\{(1;6),(2;6),(3;6),(4;6),(5;6),(6;1),(6;2),(6;3),(6;4),(6;5)\}$}
	\loigiai{ Gieo con súc sắc hai lần.\\
		Biến cố $A$ là biến cố để sau hai lần gieo có ít nhất một mặt $6$ chấm xuất hiện là  \[A=\{(1;6),(2;6),(3;6),(4;6),(5;6),(6;6),(6;1),(6;2),(6;3),(6;4),(6;5)\}.\]
	}
\end{ex}

\begin{ex}%[0D0N1-3]
	Gieo một đồng xu liên tiếp hai lần. Số phần tử của biến cố để mặt ngửa xuất hiện đúng $1$ lần là
	\choice
	{\True $2$}
	{$4$}
	{$5$}
	{$6$}
	\loigiai{
		Liệt kê ta có: $A=\{NS;SN\}$.
	}
\end{ex}

\begin{ex}%[0H9N3-1]%[tex hóa đề ck2-form 2025-đợt 2-Hồ Đức Bân]
	Trong mặt phẳng tọa độ $Oxy$, cho đường thẳng $d\colon x-2y+3=0$. Một vectơ pháp tuyến của đường thẳng $d$ là
	\choice
	{\True $\overrightarrow{n}=(1;-2)$}
	{$\overrightarrow{n}=(2;1)$}
	{$\overrightarrow{n}=(-2;3)$}
	{$\overrightarrow{n}=(1;3)$}
	\loigiai{
		Một vectơ pháp tuyến của đường thẳng $d$ là $\overrightarrow{n}=(1;-2)$.
	}
\end{ex}

\begin{ex}
	Phương trình nào sau đây là phương trình tham số của đường thẳng $d$ đi qua điểm $A(1;2)$ và có vectơ chỉ phương $\overrightarrow{u}=(3;4)$ là
	\choice
	{$\heva{&x=1+3t\\&y=2+4t}$}
	{$\heva{&x=1-3t\\&y=2-4t}$}
	{\True $\heva{&x=1+3t\\&y=2-4t}$}
	{$\heva{&x=1-3t\\&y=2+4t}$}
	\loigiai{
		Phương trình tham số của đường thẳng $d$ đi qua điểm $A(1;2)$ và có vectơ chỉ phương $\overrightarrow{u}=(3;4)$ là
		\[\heva{&x=1+3t\\&y=2-4t}.\]
	}
\end{ex}

\begin{ex}%[0H9N3-3]
	Xác định vị trí tương đối của đường thẳng
	$d \colon x-2y+1=0$ và đường thẳng $d' \colon \heva{& x=t\\ & y=3-2t}$.
	\choice
	{Song song}
	{Trùng nhau}
	{\True Vuông góc nhau}
	{Cắt nhau}
	\loigiai{
		Ta có $\vec{n_d} =\vec{u_{d'}}$ nên $d$ và $d'$ vuông góc.
	}
\end{ex}


\begin{ex}%[0H9N5-1]
	Trong mặt phẳng $O x y$, tọa độ các tiêu điểm của elip có phương trình chính tắc $\dfrac{x^2}{25}+\dfrac{y^2}{16}=1$ là
	\choice
	{\True $F_1(-3 ; 0)$; $F_2(3 ; 0)$}
	{$F_1(0 ;-4)$; $F_2(0 ; 4)$}
	{$F_1(-4 ; 0)$; $F_2(4 ; 0)$}
	{$F_1(0 ;-3)$; $F_2(0 ; 3)$}
	\loigiai{ Ta có $\heva{&a^2=25\\&b^2=16}\Rightarrow c^2=a^2-b^2=25-16=9.$\\
		Tọa độ các tiêu điểm của elip là $F_1(-3 ; 0)$; $F_2(3 ; 0)$.
	}
\end{ex}

\begin{ex}%[0H9N5-5]%[Duan15-CKII-HN23-24-Trần Xuân Hòa]%[Chuyen Hùng Vương - Phú Thọ]
	Phương trình nào sau đây là phương trình chính tắc của hypebol?
	\choice
	{\True $\dfrac{x^2}{4}-\dfrac{y^2}{9}=1$}
	{$\dfrac{x^2}{9}+\dfrac{y^2}{4}=1$}
	{$\dfrac{x^2}{4}-\dfrac{y^2}{9}=-1$}
	{$\dfrac{x^2}{9}+\dfrac{y^2}{9}=1$}
	\loigiai{
		Phương trình chính tắc của hypebol là $\dfrac{x^2}{4}-\dfrac{y^2}{9}=1$.
	}
\end{ex}
\Closesolutionfile{ans}

\TNTF
\Opensolutionfile{ans}[ans/ansDe2-TN2]
\begin{ex}%[0D6H4-2]
	Mẫu số liệu thống kê chiều cao (đơn vị: mét) của $10$ cây bạch đàn là:
	\begin{center}
		\begin{tabular}{cccccccccc}
			6,0 & 6,5 & 7,4 & 8,6 & 8,0 & 7,6 & 7,2 & 9,1 & 8,5 & 7,2
		\end{tabular}
	\end{center}
	Các mệnh đề sau \textbf{đúng} hay \textbf{sai}?
	\choiceTF
	{\True Khoảng biến thiên của mẫu số liệu là $3,1$}
	{\True Khoảng tứ phân vị của mẫu số liệu $1,3$}
	{Quy tròn phương sai của mẫu số liệu với độ chính xác $d=0,005$ là $0,85$}
	{\True Sai số tương đối của số quy tròn của phương sai trên xấp xỉ $0,6\%$}
	\loigiai{
		\begin{itemchoice}
			\itemch Chiều cao lớn nhất, nhỏ nhất của cây bạch đàn tương ứng là $9,1; 6,0$. Do đó khoảng biến thiên của mẫu số liệu là $R=9,1-6,0=3,1$.
			\itemch Sử dụng máy tính bỏ túi ta được $Q_1=7,2$, $Q_3=8,5$.\\
			Khoảng tứ phân vị của mẫu số liệu là $\Delta Q=Q_3-Q_1=8,5-7,2=1,3$.
			\itemch Phương sai của mẫu số liệu là $s^2=0,8349$. Quy tròn phương sai với độ chính xác $d=0,005$ là $0,83$.
			\itemch Sai số tương đối của số quy tròn của phương sai trên là
			\[\delta=\dfrac{0,8349-0,83}{0,83}\approx 0,006=0,6\%.\]
		\end{itemchoice}
	}
\end{ex}

\begin{ex}%[0H9H3-2]
	Cho đường thẳng $d$ có phương trình $-2x+y-1=0$.
	\choiceTF
	{Một vectơ chỉ phương của đường thẳng $d$ là $\overrightarrow u_d =\left(-2;1\right)$}
	{\True Điểm $M(1;3)$ thuộc đường thẳng $d$}
	{Đường tròn $(C)$ tâm $I(1;2)$, bán kính $R=\sqrt{5}$ có phương trình là $x^2+y^2-2x-4y+5=0$}
	{\True Đường thẳng $d$ cắt đường tròn $(C)$ tại hai điểm}
	\loigiai{
		\begin{itemchoice}
			\itemch
			Một vectơ chỉ phương của đường thẳng $d$ là $\overrightarrow u_d =\left(1;2\right)$.
			\itemch
			Thay $M(1;3)$ vào phương trình đường thẳng $d$ ta có $-2\cdot 1 +3-1=0$ (luôn đúng).
			\itemch
			Phương trình đường tròn $(C)$ có dạng $(x-1)^2+(y-2)^2=5 \Leftrightarrow x^2+y^2-2x-4y=0$.
			\itemch
			Ta có $\mathrm{d}(I,d)=\dfrac{|-2\cdot 1+2-1|}{\sqrt{(-2)^2+1^2}}=\dfrac{1}{\sqrt{5}}<R$.
			Do đó đường thẳng $d$ cắt đường tròn $(C)$ tại hai điểm.
		\end{itemchoice}
	}
\end{ex}
\Closesolutionfile{ans}

\TNSA
\Opensolutionfile{ans}[ans/ansDe2-TN3]
\begin{ex}%[0-HK1-CD-1-SoBacNinh-2324]%[VN-MT-6, Tống Văn Ký]%[0D3H2-3]
	Cho hàm số $y=ax^2+bx+2$ có đồ thị là parabol $(P)$. Tính giá trị của biểu thức $S=a+b$, biết $(P)$ có đỉnh là điểm $I(2;6)$.
	\shortans[]{$3$} %Chuẩn hóa ko quá 4 ký tự
	\loigiai{
		Parabol $y=ax^2+bx+2$ có đỉnh là $I(2;6)$, nên ta có hệ phương trình
		\[\heva{ & -\dfrac{b}{2a}=2 \\ & y(2)=6} \Leftrightarrow \heva{ & 4a+b=0 \\ & 4a+2b+2=6} \Leftrightarrow \heva{& a=-1 \\ & b=4.}
		\]
		Vậy hàm số là $y=-x^2+4x+2$. Do đó $S=a+b=3$.}
\end{ex}

\begin{ex}%[10-11-12EX-HK1-2425]%[Lê Quốc Hiệp]%[0D7V2-7]
	Lợi nhuận $T(x)$ của một cơ sở sản xuất khi bán hết $x$ sản phẩm trang trí trong một tuần theo công thức $T(x)=-x^2+40x-309$ với đơn vị tính bằng triệu đồng. Nếu muốn lợi nhuận không dưới $10$ triệu đồng thì số sản phẩm ít nhất của cơ sở sản xuất trong một tuần là bao nhiêu? Biết rằng số sản phẩm được làm ra đều bán hết trong tuần đó.
	\shortans[oly]{$11$}
	\loigiai
	{
		Điều kiện để lợi nhuận không dưới $10$ triệu đồng là
		\begin{eqnarray*}
			& & T(x) \geq 10\\
			& \Leftrightarrow & -x^2+40x-309 \geq 10\\
			& \Leftrightarrow & -x^2+40x-319 \geq 0\\
			& \Leftrightarrow & 11 \leq x \leq 29.
		\end{eqnarray*}
		Vậy số sản phẩm ít nhất để đảm bảo lợi nhuận không dưới $10$ triệu đồng là $11$ sản phẩm.
	}
\end{ex}

\begin{ex}%[0D6H3-2]
	Một xạ thủ bắn 30 viên đạn vào bia kết quả được ghi lại trong bảng phân bố tần số sau:
	\begin{center}
		\begin{tabular}{|c|c|c|c|c|c|}
			\hline
			Lớp    & 6 & 7 & 8 & 9 & 10 \\
			\hline
			Tần số & 4 & 3 & 8 & 9 & 6  \\
			\hline
		\end{tabular}
	\end{center}
	Điểm số trung bình của xạ thủ trên bằng bao nhiêu? (kết quả làm tròn đến hàng phần trăm).
	\shortans{$8,33$}
	\loigiai{
		Điểm số trung bình của xạ thủ là:
		$\overline{x}=\dfrac{6\cdot 4+7\cdot 3+8\cdot 8+9\cdot 9+10\cdot 6}{30}\approx 8,33$.
	}
\end{ex}

\begin{ex}%[0H9V4-2]
	Trong mặt phẳng tọa độ $Oxy$, xét phương trình $x^2+y^2-2mx+2\left(m+1\right)y+5=0$ ($m$ là số thực). Có bao nhiêu giá trị nguyên của $m$ để phương trình đã cho là phương trình đường tròn có bán kính không vượt quá $2\sqrt{2}$.
	\par\shortans[oly]{$2$}
	\loigiai{
		Ta có $x^2+y^2-2mx+2\left(m+1\right)y+5=0$ \hfill (1).\\
		Phương trình $(1)$ là phương trình đường tròn khi và chỉ khi
		\begin{eqnarray*}
			&&m^2+\left(m+1\right)^2-5> 0\\
			&\Leftrightarrow& m^2+m-2> 0\\
			&\Leftrightarrow& \hoac{&{m > 1} \\&{m <-2}} \left(*\right).
		\end{eqnarray*}
		Khi đó đường tròn có bán kính $R=\sqrt{m^2+\left(m+1\right)^2-5}=\sqrt{2m^2+2m-4}$.\\
		Ta có $R\le 2\sqrt{2} \Leftrightarrow \sqrt{2m^2+2m-4} \le 2\sqrt{2} \Leftrightarrow m^2+m-6\le 0\Leftrightarrow-3\le m\le 2$.\\
		Kết hợp điều kiện $\left(*\right)$ ta được $m\in \left[-3;-2\right)\cup \left(1;2\right]$.\\
		Do $m\in \mathbb{Z}$ nên $m\in \left\{-3;2\right\}$. Vậy có $2$ giá trị nguyên $m$ thỏa mãn bài toán.
	}
\end{ex}

\Closesolutionfile{ans}

\TL
\begin{ex}%[0D3H1-2]
	Tìm tập xác định của hàm số $y=\dfrac{\sqrt{2x+5}}{x^2-1}+\sqrt{4-x}$.
	\loigiai{
		Hàm số xác định khi và chỉ khi $\heva{&2x+5\geq 0\\ &x^2-1\ne0\\ &4-x\geq0}\Leftrightarrow\heva{&x\ne-1\\ &x\ne1\\ &-\dfrac{5}{2}\leq x\leq4.}$\\
		Vậy tập xác định của hàm số đã cho là $\mathscr{D}=\left[-\dfrac{5}{2};4\right]\setminus\left\{-1;1\right\}$.
	}
\end{ex}


\begin{ex}%[0D7C3-6]
	Hằng ngày bạn Hùng đều đón bạn Minh đi học tại một vị trí trên lề đường thẳng đến trường. Minh đứng tại vị trí $A$ cách lề đường một khoảng $50m$ để chờ Hùng. Khi nhìn thấy Hùng đạp xe đến địa điểm $B$, cách mình một đoạn $200m$ thì Minh bắt đầu đi bộ ra lề đường để bắt kịp xe. Vận tốc đi bộ của Minh là $5km/h$, vận tốc xe đạp của Hùng là $15km/h$. Hãy xác định vị trí $C$ trên lề đường cách $B$ bao nhiêu $ km$ (H.6.22) để hai bạn gặp nhau mà không bạn nào phải chờ người kia (làm tròn kết quả đến hàng phần trăm).
	\begin{center}
		\begin{tikzpicture}[line join=round, line cap=round, thick]
			\path
			(0,0) coordinate (B)
			(7,0) coordinate (H)
			(7,-2) coordinate (A)
			(4.5,0) coordinate (C);
			\draw (B)--(A)--(H) (A)--(C) (-1,0)--(8,0);
			\draw (B)node[above]{Hùng} (A)node[below left]{Minh};
			\draw[angle radius=\linewidth] (B)--(H)--(A);
			\foreach \i/\g in {B/-120, C/90, H/90, A/-45}{\draw[fill=black](\i) circle (1pt) ($(\i)+(\g:3mm)$) node[scale=1]{$\i$};}
			\draw (1,0.2)node{$\rightarrow$};
			\draw (6.8,-1.6)node[rotate=141.3]{$\rightarrow$};
			\draw (3.5,-1)node[below left]{$200~m$} (7,-1)node[right]{$50~m$};
		\end{tikzpicture}
	\end{center}
	% \shortans{$0,17$}
	\loigiai{
		Vận tốc của bạn Minh: $ v_1=5~(km/h)$.\\
		Vận tốc của bạn Hùng: $ v_2=15~(km/h)$.\\
		Áp dụng định lý Pithago vào tam giác vuông $AHB$: $BH=\sqrt{(0,2)^2-(0,05)^2}=\dfrac{\sqrt{15}}{20}~(km)$\\
		Gọi $BC=x~(km)$, $x>0$.\\
		Suy ra $CH=\dfrac{\sqrt{15}}{20}-x$, $x\le \dfrac{\sqrt {15}}{20}$.\\
		Ta cần xác định vị trí điểm $C$ để Minh và Hùng gặp nhau mà không bạn nào phải chờ người kia.\\
		Nghĩa là ta cần tìm $x$ để thời gian hai bạn di chuyển đến $C$ là bằng nhau.\\
		Thời gian Hùng đi từ $B$ đến $C$ là $ t_2=\dfrac{S_{BC}}{v_2}=\dfrac{x}{15}~(h)$.\\
		Quãng đường $AC$ Minh đã đi là $AC=\sqrt{CH^2+AH^2}=\sqrt {\left(\dfrac{\sqrt{15}}{20}-x \right)^2+(0,05)^2}$. \\
		Thời gian Minh đã đi từ $A$ đến $C$ là $ t_1=\dfrac{S_{AC}}{v_1}=\dfrac{\sqrt{\left(\dfrac{\sqrt{15}}{20}-x \right)^2+(0,05)^2}}{5}~(h)$.\\
		Theo yêu cầu bài toán $\dfrac{\sqrt{\left(\dfrac{\sqrt{15}}{20}-x \right)^2+(0,05)^2}}{5}=\dfrac{x}{15}$.\\
		Bình phương 2 vế $\dfrac{\left(\dfrac{\sqrt{15}}{20}-x \right)^2+(0,05)^2}{25}=\dfrac{x^2}{225}$.\\
		\begin{align}\\
			 & \Leftrightarrow 9\left( \dfrac {3}{80}-\dfrac {\sqrt {15}}{10}x+x^2 \right)+\dfrac {9}{400}=x^2                   \\
			 & \Leftrightarrow 8x^2-\dfrac {9\sqrt {15}}{10}x+\dfrac {9}{25}=0                                                   \\
			 & \Leftrightarrow \hoac{                                                                          & x\approx 0{,}27 \\ & x\approx 0,17 }
		\end{align} \\
		Vì $0<x\le \dfrac{\sqrt{15}}{20}\approx 0,19$ nên $ x\approx 0,17$ thỏa mãn.\\
		Vậy hai bạn Minh và Hùng di chuyển đến vị trí $C$ cách điểm $B$ một đoạn $ x\approx 0,17~(km)$.\\
		Đáp án: $0,17$.
	}
\end{ex}

\begin{ex}%[0D0C2-6]%[Dự án đề kiểm tra Toán 10 HKII NH23-24- Trần Văn Hùng]%[THPT Nam Lý- Hà Nam]
	Cho đa giác đều có $15$ đỉnh. Gọi $M$ là tập tất cả các tam giác có ba đỉnh là ba đỉnh của đa giác đã cho. Chọn ngẫu nhiên một tam giác thuộc tập $M$. Tính xác suất để chọn được một tam giác cân nhưng không phải là tam giác đều.
	\loigiai{
	Số tam giác có ba đỉnh là ba đỉnh của đa giác đều $15$ cạnh là $\mathrm{C}_{15}^3=455$.\\
	Số phần tử không gian mẫu là $n(\Omega)=\mathrm{C}_{455}^1=455$.\\
	Gọi $A$ là biến cố \lq\lq Chọn được một tam giác cân nhưng không phải là tam giác đều\rq\rq\rq.
	\begin{itemize}
	\item Gọi $O$ là tâm đường tròn ngoại tiếp đa giác đều.
	\item Xét một đỉnh $A$ bất kì của đa giác đều: có $7$ cặp đỉnh của đa giác đối xứng với nhau qua $OA$, hay có $7$ tam giác cân tại đỉnh $A$.\\
	Như vậy, với mỗi đỉnh của đa giác có $7$ tam giác nhận nó làm đỉnh tam giác
	cân.
	\item Số tam giác đều có ba đỉnh là ba đỉnh của đa giác là $\dfrac{15}{3}=5$ tam giác.
	\item Tuy nhiên, trong các tam giác cân đã xác định ở trên có cả tam giác đều, do mọi tam giác đều thì đều cân tại ba đỉnh nên các tam giác đều được đếm ba
	lần.
	\item Suy ra số tam giác cân nhưng không phải tam giác đều có ba đỉnh là ba
	đỉnh của đa giác đã cho là $7\cdot 15-3\cdot 5=90$.
	\end{itemize}
	Xác suất của biến cố $A$ là $\mathrm{P}(A)=\dfrac{n(A)}{n(\Omega)}=\dfrac{90}{455}=\dfrac{18}{91}$.
	}
	\end{ex}
\Closesolutionfile{ansbook}

