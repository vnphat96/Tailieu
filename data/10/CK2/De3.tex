\section*{ÔN TẬP KIỂM TRA CUỐI KÌ 2 - ĐỀ 03}
\setcounter{ex}{0}\setcounter{bt}{0}
\noindent{\bf\fontfamily{qag}\selectfont\color{violet}A. PHẦN TRẮC NGHIỆM}
\Opensolutionfile{ans}[ans/ansBTTeXCK23]
%Câu 1
\begin{ex}
	Khi sử dụng máy tính bỏ túi với 10 chữ số thập phân ta được: $\sqrt{3}=1{,}732050808$. Giá trị gần đúng của $\sqrt{3}$ chính xác đến hàng phần trăm là
	\choice
	{$1{,}7$}
	{$1{,}732$}
	{$2$}
	{$1{,}73$}
\end{ex}
%Câu 3
\begin{ex} 
	Điểm kiểm tra môn Toán của một nhóm gồm $10$ học sinh như sau: 7 8 9 5 6 9 8 9 10 6.
	Điểm trung bình môn Toán của $10$ học sinh đó là
	\choice
	{$7{,}5$}
	{$7{,}7$}
	{$7{,}6$}
	{$7{,}8$}
\end{ex}
%Câu 4
\begin{ex}
	Trên mặt phẳng tọa độ $Oxy$, cho $\vec{a}(1;-4),\vec{b}(-3;2)$. Vecto $2\vec{a}-3\vec{b}$ có tọa độ là
	\choice
	{$(-7;-2)$}
	{$(11;-2)$}
	{$(-2;-6)$}
	{$(11;-14)$}
\end{ex}
%Câu 5
\begin{ex}
	Trong hệ trục tọa độ $Oxy$, cho hai điểm $A(2;-1)$ và $B(-1;5)$. Tọa độ của vectơ $\vec{AB}$ là:
	\choice
	{$(1;4)$}
	{$(3;-6)$}
	{$(-3;6)$}
	{$(1;6)$}
\end{ex}
%Câu 6
\begin{ex}
	Cho đường thẳng $d\colon 5x-3y-6=0$. Đường thẳng $d$ vuông góc đường thẳng nào sau đây ?
	\choice
	{$d_1\colon 3x+5y=0$}
	{$d_2\colon 10x-6y-12=0$}
	{$d_3\colon -3x+5y-6=0$}
	{$d_4\colon 5x-3y=0$}
\end{ex}
%Câu 7
\begin{ex}
	Khoảng cách từ điểm $M(3;-1)$ đến đường thẳng $\Delta \colon \heva{& x=-2+t \\& y=1+2t}$ nằm trong khoảng nào sau đây?
	\choice
	{$(1;3)$}
	{$(3;5)$}
	{$(7;9)$}
	{$(5;7)$}
\end{ex}
%Câu 8
\begin{ex}
	Trong mặt phẳng với hệ trục $Oxy$, cho đường tròn $(C)\colon (x-2)^2+(y+4)^2=16$. Đường tròn $(C)$ có toạ độ tâm $I$ và bán kính $R$ bằng
	\choice
	{$I(2;-4);R=4$}
	{$I(2;-4);R=16$}
	{$I(-2;4);R=4$}
	{$I(-2;4);R=16$}
\end{ex}
%Câu 9
\begin{ex}
Trong mặt phẳng với hệ toạ độ $Oxy$, phương trình đường tròn có tâm $I(3;1)$ và đi qua điểm $M(2;-1)$ là
\choice
{${{(x+3)}^2}+{{(y+1)}^2}=\sqrt{5}$}
{${{(x-3)}^2}+{{(y-1)}^2}=\sqrt{5}$}
{${{(x-3)}^2}+{{(y-1)}^2}=5$}
{${{(x+3)}^2}+{{(y+1)}^2}=5$}
\end{ex}
%Câu 10
\begin{ex}
Cho hai điểm $F_1,F_2$ cố định có khoảng cách $F_1F_2=2c$ $\left(c>0\right)$. Đường elip là tập hợp các điểm $M$ trong mặt phẳng sao cho
	\choice
	{$MF_1+MF_2=2a,\left(a<c\right)$}
	{$MF_1-MF_2=2a,\left(a>c\right)$}
	{$MF_1+MF_2=2a,\left(a>c\right)$}
	{$\left| MF_1-MF_2 \right|=2a,\left(a<c\right)$}
\end{ex}

%Câu 12
\begin{ex}
	Lớp 10A có $20$ nam và $25$ nữ. Giáo viên chủ nhiệm cần chọn ra một học sinh làm lớp trưởng. Hỏi giáo viên đó có bao nhiêu cách chọn ?
	\choice
	{$45$}
	{$20$}
	{$25$}
	{$500$}
\end{ex}
%Câu 13
\begin{ex}
	Trên giá sách có $5$ quyển sách Toán và $3$ quyển sách Vật lý (các quyển sách đều khác nhau). Bạn Hoa muốn chọn $1$ quyển sách Toán và $1$ quyển sách Vật lý để đọc. Hỏi Hoa có bao nhiêu cách chọn?
	\choice
	{$1$}
	{$2$}
	{$8$}
	{$15$}
\end{ex}
%Câu 14
\begin{ex}
	Trong một cuộc thi điền kinh gồm $8$ vận động viên chạy trên $8$ đường chạy. Số cách xếp các vận động viên đó vào các đường chạy là 
	\choice
	{$1$}
	{$8!$}
	{$8$}
	{$64$}
\end{ex}
%Câu 15
\begin{ex}
	Một đoàn công tác gồm $10$ người. Số cách chọn $2$ người trong đoàn để phân công làm trưởng đoàn và phó đoàn là 
	\choice
	{$45$}
	{$2$}
	{$90$}
	{$20$}
\end{ex}
%Câu 16
\begin{ex}
	Trong khai triển $(2022x-2023)^5$ có bao nhiêu số hạng?
	\choice
	{$5$}
	{$4$}
	{$6$}
	{$3$}
\end{ex}
%Câu 17
\begin{ex}
Hệ số của số hạng chứa $x^3$ trong khai triển ${{(x+3)}^5}$ là
\choice
{$5$}
{$90$}
{$30$}
{$10$}
\end{ex}
%Câu 18
\begin{ex}
Từ bộ bài tú lơ khơ có 52 quân bài, rút ngẫu nhiên ra 4 quân bài. Tìm xác suất của biến cố A: 
\lq\lq Rút ra được tứ quý J.\rq\rq
\choice
{$\dfrac{2}{270725}$}
{$\dfrac{1}{270725}$}
{$\dfrac{3}{270725}$}
{$\dfrac{4}{270725}$}
\end{ex}
%Câu 19
\begin{ex}
Một tổ học sinh có $7$ nam và $3$ nữ. Chọn ngẫu nhiên $2$ người. Tính xác suất sao cho $2$ người được chọn đều là nữ.
\choice
{$\dfrac{1}{15}$}
{$\dfrac{7}{15}$}
{$\dfrac{8}{15}$}
{$\dfrac{1}{5}$}
\end{ex}
%Câu 20
\begin{ex}
Chọn ngẫu nhiên một số trong 18 số nguyên dương đầu tiên. Xác suất để chọn được số lẻ
bằng
\choice
{$\dfrac{7}{8}$}
{$\dfrac{8}{15}$}
{$\dfrac{7}{15}$}
{$\dfrac{1}{2}$}
\end{ex}
%Câu 21
\begin{ex}
Khi tính diện tích hình tròn bán kính $R=3(cm)$. Nếu lấy $\pi =3{,}14$ thì độ chính xác là bao nhiêu
\choice
{$d=0{,}09$}
{$d=0{,}009$}
{$d=0{,}01$}
{$d=0{,}1$}
\end{ex}
%Câu 22
\begin{ex}
Tìm tứ phân vị của mẫu số liệu sau: 12 3 8 15 23 18 29 36.
\choice
{$Q_1=10;Q_2=16{,}5;Q_3=26$}
{$Q_1=16{,}5;Q_2=10;Q_3=26$}
{$Q_1=10;Q_2=12{,}5;Q_3=16$}
{$Q_1=16{,}5;Q_2=26;Q_3=10$}
\end{ex}
%Câu 23
\begin{ex}
	Điểm thi cuối học kì II tám môn Toán, Văn, Anh, Sinh, Sử, Địa, Lý, Hóa của một học sinh lần lượt là 8 7{,}5 8{,}5 7 9 8 6{,}5 9{,}5. Điểm trung bình tám môn thi của học sinh là 
	\choice
	{$7{,}8$}
	{$8{,}0$}
	{$7{,}5$}
	{$8{,}2$}
\end{ex}
%Câu 24
\begin{ex}
	Trong mặt phẳng tọa độ $Oxy$, cho $\vec{a}=(-2;3)$, $\vec{b}=(3;-1)$, $\vec{c}=(-5;2)$. Tọa độ của vectơ $\vec{v}=2\vec{a}-3\vec{b}+\vec{c}$ là
	\choice
	{$\vec{v}=\left(-\,18;11\right)$}
	{$\vec{v}=(-11;18)$}
	{$\vec{v}=(-18;-11)$}
	{$\vec{v}=(11;18)$}
\end{ex}
%Câu 25
\begin{ex}
	Tìm các giá trị của tham số $m$ để hai đường thẳng $\Delta _1$: $\heva{& x=-3+mt \\& y=5+3t}$ và $\Delta _2$: $x+(m+2)y-2023=0$ vuông góc.
	\choice
	{$m=-1$ hoặc $m=3$}
	{$m=1$ hoặc $m=3$}
	{$m=1$ hoặc $m=-3$}
	{$m=-1$ hoặc $m=-3$}
\end{ex}
%Câu 26
\begin{ex}
	Viết phương trình đường tròn có tâm $I(2;-3)$ và tiếp xúc với đường thẳng $\Delta \colon 3x+4y+16=0$.
	\choice
	{$(x+2)^2+{{(y-3)}^2}=4$}
	{${{(x+2)}^2}+{{(y+3)}^2}=4$}
	{${{(x-2)}^2}+{{(y-3)}^2}=4$}
	{${{(x-2)}^2}+{{(y+3)}^2}=4$}
\end{ex}
%Câu 28
\begin{ex}
Từ các số $1{,}2,3$ có thể lập được bao nhiêu số tự nhiên khác nhau và mỗi số có các chữ số khác nhau.
\choice
{$15$}
{$20$}
{$72$}
{$36$}
\end{ex}
%Câu 29
\begin{ex}
Biển số xe máy của tỉnh $A$ (nếu không kể mã số tỉnh) có $6$ kí tự, trong đó kí tự ở vị trí đầu tiên là một chữ cái (trong bảng $26$ chữ cái tiếng Anh), kí tự ở vị trí thứ hai là một chữ số thuộc tập $\left\{ 1;2;3; \cdot \cdot \cdot ;9 \right\}$, mỗi kí tự ở bốn vị trí tiếp theo là một chữ số thuộc tập $\left\{ 0;1;2;3; \cdot \cdot \cdot ;9 \right\}$. Hỏi nếu chỉ dùng một mã số tỉnh thì tỉnh $A$ có thể làm được nhiều nhất bao nhiêu biển số xe máy khác nhau.
\choice
{$2340000$}
{$234000$}
{$75$}
{$2600000$}
\end{ex}
%Câu 30
\begin{ex}
Một đội văn nghệ có 20 người gồm 10 nam và 10 nữ, có bao nhiêu cách chọn ra một nhóm 5 người sao cho có ít nhất 2 nam và có ít nhất 1 nữ.
\choice
{12900 (cách)}
{450 (cách)}
{633600 (cách)}
{15494 (cách)}
\end{ex}
%Câu 31
\begin{ex}
Số đường chéo của đa giác 20 cạnh là 
\choice
{$190$}
{$180$}
{$200$}
{$170$}
\end{ex}
%Câu 32
\begin{ex}
Hệ số $x^2$ trong khai triển ${{(x+1)}^2}+{{(x+1)}^3}+{{(x+1)}^4}+{{(x+1)}^5}$ là 
\choice
{$20$}
{$10$}
{$12$}
{$18$}
\end{ex}
%Câu 33
\begin{ex}
Tại một kì SeaGames, môn bóng đá nam có 10 đội bóng tham gia (trong đó có Việt Nam và Thái Lan). Ban tổ chức bốc thăm ngẫu nhiên để chia 10 đội bóng nói trên thành 2 bảng A và B, mỗi bảng 5 đội. Xác suất đội Việt Nam và đội Thái Lan ở cùng một bảng là:
\choice
{$\dfrac{2}{9}$}
{$\dfrac{3}{5}$}
{$\dfrac{4}{9}$}
{$\dfrac{2}{5}$}
\end{ex}
%Câu 34
\begin{ex}
Từ các số $1;2;4;7;9;10$ lấy ngẫu nhiên một số. Xác suất để lấy được một số nguyên tố là:
\choice
{$\dfrac{1}{2}$}
{$\dfrac{1}{3}$}
{$\dfrac{1}{4}$}
{$\dfrac{1}{6}$}
\end{ex}
%Câu 35
\begin{ex}
Một bình đựng $4$ quả cầu xanh và $6$ quả cầu trắng. Chọn ngẫu nhiên $4$ quả cầu. Xác suất để được $2$ quả cầu xanh và $2$ quả cầu trắng là
\choice
{$\dfrac{1}{20}$}
{$\dfrac{3}{7}$}
{$\dfrac{1}{7}$}
{$\dfrac{4}{7}$}
\end{ex}
%Câu 36
\begin{ex}
Kết quả đo chiều dài một cây cầu có độ chính xác là $0{,}75m$ với dụng cụ đo đảm bảo sai số tương đối không vượt quá $0{,}15\%$. Tính độ dài gần đúng của cây cầu.
\choice
{$500{,}1m$}
{$499{,}9m$}
{$500m$}
{$501m$}
\end{ex}
%Câu 38
\begin{ex}
Mẫu số liệu sau cho biết số ghế trống tại một rạp chiếu phim trong $9$ ngày như sau: 7 8 22 20 15 18 19 13 11. Khoảng tứ phân vị cho mẫu số liệu là
\choice
{$\Delta Q=11$}
{$\Delta Q=9$}
{$\Delta Q=10$}
{$\Delta Q=9{,}8$}
\end{ex}
\begin{ex}
	Lập phương trình đường tròn đi qua hai điểm $A(3;0),B(0;2)$ và có tâm thuộc đường thẳng $d\colon x+y=0$.
	\choice
	{$\left(x-\dfrac{1}{2}\right)^2+{{\left(y+\dfrac{1}{2}\right)}^2}=\dfrac{13}{2}$}
	{${{\left(x+\dfrac{1}{2}\right)}^2}+{{\left(y+\dfrac{1}{2}\right)}^2}=\dfrac{13}{2}$}
	{${{\left(x-\dfrac{1}{2}\right)}^2}+{{\left(y-\dfrac{1}{2}\right)}^2}=\dfrac{13}{2}$}
	{${{\left(x+\dfrac{1}{2}\right)}^2}+{{\left(y-\dfrac{1}{2}\right)}^2}=\dfrac{13}{2}$}
\end{ex}



\noindent{\bf\fontfamily{qag}\selectfont\color{violet}B. PHẦN TỰ LUẬN}
%Câu 36
\begin{ex}
Lập phương trình chính tắc elip $(E)$ có tiêu điểm $F_1(-4;0)$ và đi qua điểm $A(7;0)$.
\end{ex}
%Câu 37
\begin{ex}
Viết phương trình tiếp tuyến đường tròn $(C)\colon x^2+y^2+4x-5y+3=0$ tại giao điểm với trục $Ox$.
\end{ex}
%Câu 44
\begin{ex}
Một nhóm học sinh gồm 15 học sinh lớp 10A, 10 học sinh lớp 10B và 5 học sinh lớp 10C. Tìm số cách chọn ra 15 học sinh sao cho có ít nhất 5 học sinh lớp 10A và đúng 2 học sinh lớp 10C.
\end{ex}
%Câu 47
\begin{ex}
Trong mặt phẳng với hệ tọa độ $Oxy$ cho đường tròn $(C)\colon x^2+y^2-4x-2y-1=0$ và đường thẳng $d\colon x+y+1=0$. Tìm những điểm $M$ thuộc đường thẳng $d$ sao cho từ điểm $M$ kẻ được đến $(C)$ hai tiếp tuyến hợp với nhau một góc ${{90}^\circ}$.
\end{ex}

\Closesolutionfile{ans}