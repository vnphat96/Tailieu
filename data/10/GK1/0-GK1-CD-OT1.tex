\begin{name}
	{\tenchude}
	{TOÁN 10}
	{LỚP TOÁN THẦY PHÁT}
	{Thời gian: 90 phút - Không kể thời gian phát đề}
\end{name}
%Câu 1
\begin{ex}
	Trong các câu sau, câu nào là mệnh đề?
	\choice
	{$1+x=2$}
	{$x<3$}
	{Số $5$ là số nguyên tố phải không?}
	{\True Phú Thọ là tỉnh thuộc miền Bắc Việt Nam}
	\loigiai{
		Các câu ở phương án A và B là mệnh đề chứa biến, câu ở phương án C không là mệnh đề.\\
		Câu “Phú Thọ là tỉnh thuộc miền Bắc Việt Nam” là một khẳng định đúng nên là mệnh đề
	}
\end{ex}
%Câu 2
\begin{ex}
	Phủ định của mệnh đề “$1+2=3$” là mệnh đề
	\choice
	{$1+2>3$}
	{$1+2<3$}
	{\True $1+2\ne 3$}
	{$1+2\le 3$}
	\loigiai{
		Phủ định của mệnh đề “$1+2=3$” là mệnh đề “$1+2\ne 3$”
	}
\end{ex}
%Câu 3
\begin{ex}
	Cho $x$ là một phần tử của tập hợp $X$. Mệnh đề nào sau đây là đúng?
	\choice
	{$\{x\}\in X$}
	{\True $x\in X$}
	{$x\subset X$}
	{$X\in x$}
	\loigiai{
		Nếu $x$ là một phần tử của tập hợp $X$ thì $x\in X$
	}
\end{ex}
%Câu 4
\begin{ex}
	Cho hai tập hợp $A=\left\{ x\in \mathbb{Z}|\,x^2-x-6=0 \right\}$ và $B=\left\{ 3;m \right\}$. Với giá trị nào của tham số $m$ thì $A\,=\,B$?
	\choice
	{$m=3$}
	{\True $m=-2$}
	{$m=3$ hoặc $m=-2$}
    {$m=2$}
	\loigiai{
		Ta có $x^2-x-6=0\Leftrightarrow \hoac{& x=-2\in \mathbb{Z} \\& x=3\,\in \mathbb{Z}}\Rightarrow A=\left\{ -2;3 \right\}$.\\
		Suy ra $A\,=\,B\Leftrightarrow m=-2$
	}
\end{ex}
%Câu 5
\begin{ex}
	Cho hai tập hợp $A=\left(-3;4\right]$ và $B=\left[2;6\right]$. Trong các mệnh đề sau, mệnh đề nào sai?
	\choice
	{$A\cap B=\left[2;4\right]$}
    {$A\cup B=\left(-3;6\right]$}
	{\True $A\setminus B=\left(-3;2\right]$}
	{$B\setminus A=\left(4;6\right]$}
	\loigiai{
		Ta có: $A\cap B=\left[2;4\right]$; $A\cup B=\left(-3;6\right]$; $A\setminus B=\left(-3;2\right)$ và $B\setminus A=\left(4;6\right]$.
	}
\end{ex}
%Câu 6
\begin{ex}
	Cho tập hợp $A=\left\{ x\in \mathbb{Z}|\,4x^2-3x-7=0 \right\}$, $B=\left\{ x\in \mathbb{Q}|\,x^2-7=0 \right\}$, $C=\left\{ x\in \mathbb{N}|\,x^2+6x+5=0 \right\}$ và $D=\left\{ x\in \mathbb{R}|\,x^2-3x+7=0 \right\}$. Trong các tập hợp trên có bao nhiêu tập rỗng?
	\choice
	{$1$}
	{\True $3$}
	{$2$}
	{$4$}
	\loigiai{
		Ta có $4x^2-3x-7=0\Leftrightarrow \hoac{& x=-1\in \mathbb{Z} \\& x=\dfrac{7}{4}\notin \mathbb{Z}}\Rightarrow A=\{-1\}$.\\
		+) $x^2-7=0\Leftrightarrow x=\pm \sqrt{7}\notin \mathbb{Q}\Rightarrow B=\varnothing $.\\
		+) $x^2+6x+5=0\Leftrightarrow \hoac{& x=-1\notin \mathbb{N} \\& x=-5\notin \mathbb{N}}\Rightarrow C=\varnothing $.\\
		+) Phương trình $x^2-3x+7=0$ có $\triangle =-19<0$ suy ra phương trình vô nghiệm hay $D=\varnothing $.\\
		Vậy trong 4 tập hợp trên có 3 tập rỗng
	}
\end{ex}
%Câu 7
\begin{ex}
	Bất phương trình nào sau đây là bất phương phương trình bậc nhất hai ẩn?
	\choice
	{$x^2+y^2\le 0$}
	{$\dfrac{1}{2}x^2+3y+5<0$}
	{$2x+3y^2\ge 5$}
	{\True $2x+3y<5$}
	\loigiai{

	}
\end{ex}
%Câu 8
\begin{ex}
	Cặp số $(x;y)=(3;-1)$ là nghiệm của bất phương trình bậc nhất hai ẩn nào sau đây?
	\choice
	{$x^2+y^2\le 50$}
	{\True $x-3y>0$}
	{$\dfrac{1}{4}x^2-y\le 0$}
	{$5x-2y\le -4$}
	\loigiai{
		Chọn B
	}
\end{ex}
%Câu 9
\begin{ex}
	Hệ bất phương trình nào sau đây là hệ bất phương phương trình bậc nhất hai ẩn?
	\choice
	{$\heva{& -2x+5y<4 \\& x^2+3y>6}$}
	{$\heva{& -2x+5y<4 \\& x^2+3y^2>6}$}
	{$\heva{& -2x^2+5y<4 \\& x^2+3y>6}$}
	{\True $\heva{& -2x+5y<4 \\& x+3y>6}$}
	\loigiai{
		Chọn D
	}
\end{ex}
%Câu 10
\begin{ex}
	Điểm $M(0;-3)$ thuộc miền nghiệm của hệ bất phương trình:
	\choice
	{\True $\heva{& 2x-y\le 3 \\& 2x+5y\le 12x+8}$}
	{$\heva{& 2x-y>3 \\& 2x+5y\le 12x+8}$}
	{$\heva{& 2x-y\le 3 \\& 2x+5y\ge 12x+8}$}
	{$\heva{& 2x-y\ge 3 \\& 2x+5y\ge 12x+8}$}
	\loigiai{
		Thế tọa độ điểm $M(0;-3)$ vào các hệ bất phương trình, ta thấy chỉ có hệ bất phương trình ở phương án A thỏa mãn.\\
		Vậy điểm $M(0;-3)$ thuộc miền nghiệm của hệ bất phương trình $\heva{& 2x-y\le 3 \\& 2x+5y\le 12x+8}$
	}
\end{ex}
%Câu 11
\begin{ex}%[Dự Án 6 -Đề GHK1 - Khối 10]%[Huu Hien Maths]%[0C2Y2-1]%Câu 22
	\immini
	{
		Cho hình vẽ bên dưới, miền nghiệm được biểu diễn bởi phần không bị tô màu (không có đường thẳng) là miền nghiệm của bất phương trình nào sau đây?
		\choice
		{\True $x+y>2$}
		{$x+y \geq 2$}
		{$x+y \leq 2$}
		{$x+y<2$}
	}
	{
		\begin{tikzpicture}[scale=.65, font=\footnotesize, line join=round, line cap=round, >=stealth]
			\def\a{-1}\def\b{2}
			%\draw[color=gray,dash pattern=on 1pt off 1pt,xstep=1.0cm,ystep=1.0cm] (-5.2,-5.2) grid (5.2,5.2);
			\draw[->] (-2,0) -- (5,0) node[below] {$x$};
			\draw[->] (0,-2) -- (0,4) node[left] {$y$};
			\draw[thick,samples=150,smooth,domain=-2:4] plot(\x,{\a*\x+(\b)});
			\draw[pattern = north west lines,draw=none] (-2,4)--(-2,-2)--(4,-2)--cycle;
			\fill (0,0) circle (1pt) node[below left]{$O$};
			\fill (2,0) circle (1pt) node[above]{$2$};
			\fill (0,2) circle (1pt) node[above right]{$2$};
		\end{tikzpicture}
	}
	\loigiai{
		Ta thấy đường thẳng đi qua hai điểm $A(0;2)$ và $B(2;0)$ nên đường thẳng có phương trình $\Delta \colon x+y-2=0$.\\
		Lấy điểm $O(0;0) \notin \Delta$, không thuộc miền nghiệm của bất phương trình và ta có $0+0<2$ nên hình trên biểu diễn miền nghiệm của bất phương trình $x+y>2$.
	}
\end{ex}
%Câu 12
\begin{ex}
	Cho hệ bất phương trình $\heva{& x+3y-2\ge 0 \\& 2x+y+1>0}$. Trong các điểm sau, điểm nào thuộc miền nghiệm của hệ bất phương trình?
	\choice
	{$N(-1;1)$}
	{$Q(-1;0)$}
	{$P(1;-3)$}
	{\True $M(0;1)$}
	\loigiai{
		Thay tọa độ các điểm $N$, $Q,P$, $M$ vào hệ bất phương trình, chỉ có tọa độ điểm $M$ thỏa mãn hệ bất phương trình đã cho
	}
\end{ex}
%Câu 13
\begin{ex}
	Cho góc $\alpha $ thỏa mãn $0^\circ \le \alpha \le 180^\circ $. Khẳng định nào sau đây đúng?
	\choice
	{\True $\sin \left(180^\circ -\alpha\right)=\sin \alpha $}
	{$\cos \left(180^\circ -\alpha\right)=\cos \alpha $}
	{$\tan \left(180^\circ -\alpha\right)=\tan \alpha $}
	{$\cot \left(180^\circ -\alpha\right)=\cot \alpha $}
	\loigiai{
		Ta có tính chất $\sin \left(180^\circ -\alpha\right)=\sin \alpha $; $\cos \left(180^\circ -\alpha\right)=-\cos \alpha $; $\tan \left(180^\circ -\alpha\right)=-\tan \alpha $;$\cot \left(180^\circ -\alpha\right)=-\cot \alpha $. Do đó A là khẳng định đúng
	}
\end{ex}
%Câu 14
\begin{ex}
	Không dùng máy tính, tính giá trị của biểu thức $A=\cos 10^\circ +\cos 20^\circ + \cdot \cdot \cdot +\cos 170^\circ +\cos 180^\circ $.
	\choice
	{$A=0$}
	{$A=1$}
	{\True $A=-1$}
	{$A=\dfrac{3}{2}$}
	\loigiai{
		Ta có\\
		$A=\left(\cos 10^\circ +\cos 170^\circ\right)+\left(\cos 20^\circ +\cos 160^\circ\right)+ \cdot \cdot \cdot +\left(\cos 80^\circ +\cos 100^\circ\right)+\cos 90^\circ +\cos 180^\circ $\\
		$=\left(\cos 10^\circ -\cos 10^\circ\right)+\left(\cos 20^\circ -\cos 20^\circ\right)+ \cdot \cdot \cdot +\left(\cos 80^\circ -\cos 80^\circ\right)+0+(-1)=-1$
	}
\end{ex}
%Câu 15
\begin{ex}
	Cho góc $\alpha $ thỏa mãn $\tan \alpha =4$. Tính giá trị của biểu thức $A=\dfrac{\sin \alpha +\cos \alpha }{\sin \alpha -3\cos \alpha }$
	\choice
	{$A=1$}
	{$A=\dfrac{1}{2}$}
	{$A=\dfrac{1}{5}$}
	{\True $A=5$}
	\loigiai{
		Ta có $\cos \alpha \ne 0$ nên $A=\dfrac{\sin \alpha +\cos \alpha }{\sin \alpha -3\cos \alpha }=\dfrac{\dfrac{\sin \alpha }{\cos \alpha }+1}{\dfrac{\sin \alpha }{\cos \alpha }-3}=\dfrac{\tan \alpha +1}{\tan \alpha -3}=\dfrac{4+1}{4-3}=5$
	}
\end{ex}
%Câu 16
\begin{ex}
	Cho tam giác $ABC$ có $AB=4$, $AC=5$ và $\cos A=\dfrac{3}{5}$. Độ dài cạnh $BC$ bằng
	\choice
	{\True $\sqrt{17}$}
	{$17$}
	{$3\sqrt{2}$}
	{$18$}
	\loigiai{
		Áp dụng định lí côsin trong tam giác $ABC$ ta có:\\
		$BC^2=AB^2+AC^2-2 \cdot AB \cdot AC \cdot \cos A=4^2+5^2-2 \cdot 4 \cdot 5 \cdot \dfrac{3}{5}=17$.\\
		Suy ra: $BC=\sqrt{17}$
	}
\end{ex}
%Câu 17
\begin{ex}
	Cho tam giác nhọn $ABC$ có $\widehat{A}=30^\circ $ và $BC=4$. Bán kính $R$ đường tròn ngoại tiếp tam giác $ABC$ bằng
	\choice
	{$R=2$}
	{$R=3$}
	{\True $R=4$}
	{$R=5$}
	\loigiai{
		Áp dụng định lí hàm sin trong tam giác $ABC$ có:\\
		$\dfrac{BC}{\sin A}=2R\Rightarrow R=\dfrac{BC}{2\sin A}=\dfrac{4}{2\sin 30^\circ }=4$
	}
\end{ex}
%Câu 18
\begin{ex}
	Cho tam giác $ABC$ có $AB=8$, $AC=9$ và $\widehat{A}=60^\circ $. Diện tích tam giác $ABC$ bằng
	\choice
	{\True $18\sqrt{3}$}
	{$18$}
	{$36\sqrt{3}$}
	{$36$}
	\loigiai{
		Ta có: $S=\dfrac{1}{2}AB \cdot AC \cdot \sin A=\dfrac{1}{2} \cdot 8 \cdot 9 \cdot \sin 60^\circ =18\sqrt{3}$
	}
\end{ex}
%Câu 19
\begin{ex}
	Cho tam giác $ABC$ có $AB=5$, $C=30^\circ $. Tính bán kính đường tròn ngoại tiếp của tam giác $ABC$.
	\choice
	{$\dfrac{5\sqrt{3}}{3}$}
	{\True $5$}
	{$10$}
	{$20$}
	\loigiai{
		Ta có $\dfrac{c}{\sin C}=2R\Rightarrow R=\dfrac{c}{2 \cdot \sin C}=\dfrac{5}{2\sin 30^\circ }=5$
	}
\end{ex}
%Câu 20
\begin{ex}
	Cho tam giác $ABC$. Chọn khẳng định sai:
	\choice
	{$S=\dfrac{1}{2}a \cdot h_a$}
	{$S=\dfrac{1}{2}a \cdot c \cdot \sin B$}
	{\True $S=\dfrac{abc}{R}$}
	{$S=p \cdot r$}
	\loigiai{
	}
\end{ex}
%Câu 21
\begin{ex}
	Cho tam giác $ABC$ có $a=6,b=4,C=30^\circ $. Tính độ dài đường cao vẽ từ đỉnh $B$ của tam giác $ABC$.
	\choice
	{$8$}
	{$48$}
	{$\dfrac{3}{2}$}
	{\True $3$}
	\loigiai{
		+) $S=\dfrac{1}{2}a \cdot b \cdot \sin C=\dfrac{1}{2} \cdot 6 \cdot 4 \cdot \sin 30^\circ =6$.\\
		+) $S=\dfrac{1}{2}b \cdot h_b\Rightarrow h_b=\dfrac{2S}{b}=\dfrac{2 \cdot 6}{4}=3$
	}
\end{ex}
%Câu 22
\begin{ex}
	Trong các mệnh đề dưới đây, mệnh đề nào là mệnh đề đúng?
	\choice
	{$\forall x\in \mathbb{R},x^2>0$}
	{$\exists x\in \mathbb{N},x^2+4x+3=0$}
	{$\exists x\in \mathbb{R},x^2+2x+4<0$}
	{\True $\exists x\in \mathbb{R},x^2\le 0$}
	\loigiai{
		+) Mệnh đề ở phương án A sai khi $x=0$.
		+) $x^2+4x+3=0\Leftrightarrow \hoac{& x=-1 \\& x=-3}$ và $-1\notin \mathbb{N}$; $-3\notin \mathbb{N}$ nên mệnh đề ở phương án B sai.
		+) $x^2+2x+4=(x+1)^2+3>0,\forall x\in \mathbb{R}$ nên mệnh đề ở phương án C sai.
		+) Mệnh đề ở phương án D đúng khi $x=0$.
	}
\end{ex}
%Câu 23
\begin{ex}
	Cho tập hợp $A=\left\{x\in \mathbb{R} \mid |x+1|\le 3 \right\}$. Chọn khẳng định đúng.
	\choice
	{$A\cap \mathbb{N}=\left\{ 1;2 \right\}$}
	{\True $A\cap \mathbb{N}^*=\left\{ 1;2 \right\}$}
	{$A\cap \mathbb{Z}=\left\{ -3;-2;-1;0;1;2 \right\}$}
	{$A\cap \mathbb{Z}=\left\{ 0;1;2 \right\}$}
	\loigiai{
	Ta có: $|x+1|\le 3\Leftrightarrow -3\le x+1\le 3\Leftrightarrow -4\le x\le 2$. Do đó: $A=[-4;2]$.
	Nên:
	+) $A\cap {{\mathbb{N}}^*}=\left\{ 1;2 \right\}$.
	+) $A\cap \mathbb{N}=\left\{ 0;1;2 \right\}$.
	+) $A\cap \mathbb{Z}=\left\{ -4;-3;-2;-1;0;1;2 \right\}$.
	}
\end{ex}
%Câu 24
\begin{ex}
	Cho các tập hợp $A=\left(-\infty ;3\right)$ và $B=[0;10]$. Số phần tử là số nguyên của tập $B\setminus A$ là?
	\choice
	{$6$}
	{$7$}
	{\True $8$}
	{vô số}
	\loigiai{
	Ta có: $B\setminus A=[3;10]$.\\
	Suy ra các phần tử là số nguyên của tập $B\setminus A$ là $\left\{ 3;4;5;6;7;8;9;10 \right\}$.
	}
\end{ex}
%Câu 25
\begin{ex}
	Cho các tập hợp $A=(-7;2)\cup \left[6;+\infty\right)$ và $B=[-5;8]$. Khẳng định nào sau đây là sai?
	\choice
	{$A\cap B=[-5;2)\cup [6;8]$}
	{$A\cup B=\left(-7;+\infty\right)$}
	{$A\setminus B=(-7;-5)\cup \left(8;+\infty\right)$}
	{\True $B\setminus A=[2;6]$}
	\loigiai{
	Ta có $B\setminus A=[2;6)$.\\
	Suy ra $B\setminus A=[2;6]$ sai.\\
	Chọn đáp án D
	}
\end{ex}
%Câu 26
\begin{ex}
	Với giá trị nào của $m$, cặp số $(2;-1)$ là một nghiệm của bất phương trình $2x-(m-2)y\ge 3$?
	\choice
	{$m\ge -1$}
	{$m\le 3$}
	{$m\le 1$}
	{\True $m\ge 1$}
	\loigiai{
		Cặp số $(2;-1)$ là một nghiệm của bất phương trình $2x-(m-2)y\ge 3$ khi và chỉ khi:\\
		$2 \cdot 2-(m-2) \cdot (-1)\ge 3\Leftrightarrow 4+m-2\ge 3\Leftrightarrow m\ge 1$
	}
\end{ex}
%Câu 27
\begin{ex}%[Dự Án 6 -Đề GHK1 - Khối 10]%[Huu Hien Maths]%[0C2B1-2]%Câu 23
	\immini
	{
		Miền không bị gạch (không tính đường thẳng) được cho bởi hình sau, là miền nghiệm của bất phương trình nào?
		\choice
		{$2x+y-6>0$}
		{\True $2x+y-6<0$}
		{$x+2y-6<0$}
		{$x+2y-6>0$}
	}
	{
		\begin{tikzpicture}[scale=.65,yscale=0.65, font=\footnotesize, line join=round, line cap=round, >=stealth]
			\def\a{-2}\def\b{6}
			%\draw[color=gray,dash pattern=on 1pt off 1pt,xstep=1.0cm,ystep=1.0cm] (-5.2,-5.2) grid (5.2,5.2);
			\draw[->] (-1,0) -- (3.5,0) node[below] {$x$};
			\draw[->] (0,-1) -- (0,8) node[left] {$y$};
			\draw[thick,samples=150,smooth,domain=-1:3.5] plot(\x,{\a*\x+(\b)});
			\draw[pattern = north west lines,draw=none] (-1,8)--(3.5,-1)--++(90:9)--cycle;
			\fill (0,0) circle (1pt) node[below left]{$O$};
			\fill (3,0) circle (1pt) node[below]{$3$};
			\fill (0,6) circle (1pt) node[left]{$6$};
		\end{tikzpicture}
	}
	\loigiai{
		Từ đồ thị ta thấy:\\
		Đường thẳng $d$ đi qua $2$ điểm $A(3;0)$, $B(0;6)$ suy ra phương trình là $2x+y-6=0$.\\
		Điểm $O(0;0)$ thuộc miền nghiệm của bất phương trình.\\
		Nên đáp án là $2x+y-6<0$.
	}
\end{ex}
%Câu 28
\begin{ex}
	Điểm $M(1;2)$ thuộc miền nghiệm của hệ bất phương trình nào sau đây?
	\choice
	{$\heva{& 2x+3y-6\le 0 \\& x-y+2\ge 0 \\& y\ge 0}$}
	{$\heva{& x+2y+1>0 \\& x-y+1<0 \\& x<0}$}
	{\True $\heva{& 2x+y-6\le 0 \\& x-y+2\ge 0 \\& x\ge 0}$}
	{$\heva{& 4x+y-10\le 0 \\& x-y+2\ge 0 \\& y-3>0}$}
	\loigiai{
		+) Ta thay $\heva{& x=1 \\& y=2}$ vào các phương trình trong hệ ở phương án A ta được: $\heva{& 2(1)+3(2)-6\le 0 \\& 1-2+2\ge 0 \\& 2\ge 0}\Leftrightarrow \heva{& 2\le 0 \\& 1\ge 0 \\& 2\ge 0}(KTM)$.\\
		Tương tự: $\heva{& 1+2(2)+1>0 \\& 1-2+1<0 \\& 1<0}\Leftrightarrow \heva{& 6>0 \\& 0<0 \\& 1<0}(KTM)$, ta loại phương án B. \\
		$\heva{& 2(1)+(2)-6\le 0 \\& 1-2+2\ge 0 \\& 1\ge 0}\Leftrightarrow \heva{& -2\le 0 \\& 1\ge 0 \\& 1\ge 0}(TM)$.\\
		Vậy phương án C thỏa mãn
	}
\end{ex}
%Câu 29
\begin{ex}%[Dự Án 6 -Đề GHK1 - Khối 10]%[Huu Hien Maths]%[0C2Y2-1]%Câu 24
	Miền nghiệm của hệ bất phương trình $\heva{&3x-2y+6 \geq 0\\&2x+y-10 \geq 0\\&y-1>0}$ là miền chứa điểm nào trong các điểm sau?
	\choice
	{$M(1;-3)$}
	{\True $N(4;3)$}
	{$P(-1;5)$}
	{$Q(-2;-3)$}
	\loigiai{
		Thay $x=1$, $y=-3$ vào từng hệ bất phương trình đã cho, ta được:\\
		$\heva{&15 \geq 0\\&-11 \geq 0\\&-4>0}$ (không thoả mãn).\\
		Thay $x=4$, $y=3$ vào từng hệ bất phương trình đã cho, ta được:\\
		$\heva{&12 \geq 0\\&1 \geq 0\\&2>0}$ (thoả mãn).\\
		Thay $x=-1$, $y=5$ vào từng hệ bất phương trình đã cho, ta được:\\
		$\heva{&-7 \geq 0\\&-7 \geq 0\\&4>0}$ (không thoả mãn).\\
		PThay $x=-2$, $y=-3$ vào từng hệ bất phương trình đã cho, ta được:\\
		$\heva{&6 \geq 0\\&-17 \geq 0\\&-4>0}$ (không thoả mãn).\\
		Vậy miền nghiệm của hệ bất phương trình $\heva{&3x-2y+6 \geq 0\\&2x+y-10 \geq 0\\&y-1>0}$ là miền chứa điểm $N(4;3)$.
	}
\end{ex}
%Câu 30
\begin{ex}
	Cho biết $\cos \alpha =-\dfrac{2}{3}$. Giá trị của $P=\dfrac{\cot \alpha +3\tan \alpha }{2\cot \alpha +\tan \alpha }$ bằng
	\choice
	{$P=-\dfrac{19}{13}$}
	{\True $P=\dfrac{19}{13}$}
	{$P=\dfrac{25}{13}$}
	{$P=-\dfrac{25}{13}$}
	\loigiai{
	Ta có ${{\sin }^2}\alpha +{{\cos }^2}\alpha =1\Leftrightarrow {{\sin }^2}\alpha =1-{{\cos }^2}\alpha =\dfrac{5}{9}$.\\
	Khi đó ta có\\
	$P=\dfrac{\cot \alpha +3\tan \alpha }{2\cot \alpha +\tan \alpha }=\dfrac{\dfrac{\cos \alpha }{\sin \alpha }+3\dfrac{\sin \alpha }{\cos \alpha }}{2\dfrac{\cos \alpha }{\sin \alpha }+\dfrac{\sin \alpha }{\cos \alpha }}=\dfrac{{{\cos }^2}\alpha +3{{\sin }^2}\alpha }{2{{\cos }^2}\alpha +{{\sin }^2}\alpha }=\dfrac{{{\left(-\dfrac{2}{3}\right)}^2}+3\cdot \dfrac{5}{9}}{2\cdot {{\left(-\dfrac{2}{3}\right)}^2}+\dfrac{5}{9}}=\dfrac{19}{13}$
	}
\end{ex}
%Câu 31
\begin{ex}
	Cho biết $\tan \alpha =-3$. Giá trị của $P=\dfrac{6\sin \alpha -7\cos \alpha }{6\cos \alpha +7\sin \alpha }$ bằng
	\choice
	{$P=\dfrac{4}{3}$}
	{\True $P=\dfrac{5}{3}$}
	{$P=-\dfrac{4}{3}$}
	{$P=-\dfrac{5}{3}$}
	\loigiai{
		Điều kiện: $\cos \alpha \ne 0$.\\
		Ta có\\
		$P=\dfrac{6\sin \alpha -7\cos \alpha }{6\cos \alpha +7\sin \alpha }=\dfrac{6\dfrac{\sin \alpha }{\cos \alpha }-7}{6+7\dfrac{\sin \alpha }{\cos \alpha }}P=\dfrac{6 \sin \alpha-7 \cos \alpha}{6 \cos \alpha+7 \sin \alpha}=\dfrac{6 \dfrac{\sin \alpha}{\cos \alpha}-7}{6+7 \dfrac{\sin \alpha}{\cos \alpha}}=\dfrac{6 \tan \alpha-7}{6+7 \tan \alpha}=\dfrac{5}{3}$
	}
\end{ex}
%Câu 32
\begin{ex}
	Cho tam giác $ABC$ có cạnh $AB=a$; $AC=a\sqrt{3}$; $BC=a\sqrt{7}$. Tính góc $\widehat{BAC}$
	\choice
	{${{30}^\circ}$}
	{\True ${{150}^\circ}$}
	{${{60}^\circ}$}
	{${{120}^\circ}$}
	\loigiai{
	Từ định lý cosin trong tam giác $ABC$ ta có:\\
	$\cos \widehat{BAC}=\dfrac{AB^2+AC^2-BC^2}{2AB \cdot AC}=\dfrac{a^2+3a^2-7a^2}{2a \cdot a\sqrt{3}}=-\dfrac{\sqrt{3}}{2}\Rightarrow \widehat{BAC}={{150}^\circ}$
	}
\end{ex}
%Câu 33
\begin{ex}
	Cho tam giác $ABC$ có cạnh $AB=2\,\text{cm}$; $\widehat{ABC}={{60}^\circ}$; $\widehat{BAC}={{75}^\circ}$. Diện tích tam giác $ABC$ gần nhất với giá trị nào sau đây?
	\choice
	{\True $2{,}37\,cm^2$}
	{$0{,}63\,cm^2$}
	{$2{,}45\,cm^2$}
	{$1{,}58\,cm^2$}
	\loigiai{
	Ta có $\widehat{ACB}={{180}^\circ}-\widehat{ABC}-\widehat{BAC}={{45}^\circ}$\\
	Áp dụng định lý sin ta có: $\dfrac{AB}{\sin {{45}^\circ}}=\dfrac{AC}{\sin {{60}^\circ}}\Rightarrow AC=\dfrac{AB\sin {{60}^\circ}}{\sin {{45}^\circ}}=\sqrt{6}\,\text{cm}$.\\
	Diện tích tam giác $ABC$ là: $S=\dfrac{1}{2}AB \cdot AC \cdot \sin {{75}^\circ}=\dfrac{1}{2} \cdot 2 \cdot \sqrt{6} \cdot \sin {{75}^\circ}\approx 2{,}37\,cm^2$
	}
\end{ex}
%Câu 34
\begin{ex}
	Cho tam giác $ABC$ có $AB=3$, $BC=5$ và độ dài đường trung tuyến $BM=\sqrt{13}$. Tính độ dài $AC$.
	\choice
	{$2$}
	{$\sqrt{10}$}
	{\True $4$}
	{$\sqrt{13}$}
	\loigiai{
		+) Xét $\triangle ABC$, theo công thức tính độ dài đường trung tuyến, ta có:\\
		$BM^2=\dfrac{BA^2+BC^2}{2}-\dfrac{AC^2}{4}\Leftrightarrow {{\left(\sqrt{13}\right)}^2}=\dfrac{3^2+5^2}{2}-\dfrac{AC^2}{4}$ $\Leftrightarrow AC=4$}
\end{ex}
%Câu 35
\begin{ex}
	\immini{Từ vị trí $A$ cách mặt đất $1\text{m}$, một bạn nhỏ quan sát một cây đèn đường.
	Biết $HC=6\text{m}$, $\widehat{BAC}={{44}^\circ}$. Chiều cao của cây đèn đường gần nhất với giá trị nào sau đây?
	\choice
	{$\text{4m}$}
	{\True $5\,\text{m}$}
	{$\text{4}\text{,5m}$}
	{$6{,}5\,\text{m}$}}
	{
	\begin{tikzpicture}[line join=round,scale=.5]
		\draw[fill=black!80]
		(-.5,0) arc (180:360:{.5} and {.2})
		--(.3,.1)--(.4,.2)
		to[out=120,in=-60] (.2,1.5)
		to[out=120,in=-120](.25,1.7)
		to[out=95,in=-90] (.05,8)
		--(-.05,8)
		to[out=-90,in=85] (-.25,1.7)
		to[out=-60,in=60] (-.2,1.5)
		to[out=-120,in=60](-.4,.2)
		--(-.3,.1)--cycle
		;
		\fill[black]
		(0,8) circle (.2)
		(0,9.2)ellipse ({.25} and {.05})
		(1.5,8.3)ellipse ({.25} and {.05})
		(-1.5,8.3)ellipse ({.25} and {.05})
		;
		\draw[thick]
		(.5,8.3) arc (90:240:.3) to[out=-30,in=-90] (1.5,8.3)
		(-.5,8.3) arc (90:-60:.3) to[out=-150,in=-90] (-1.5,8.3)
		(.5,8.3) arc (90:240:.3) arc (-120:0:{.6} and {.4}) arc (0:180:.1)
		(-.5,8.3) arc (90:-60:.3) arc (-60:-180:{.6} and {.4}) arc (180:0:.1)
		(.5,8.3) arc (90:360:.3) arc (0:180:.15)
		(-.5,8.3) arc (90:-180:.3) arc (180:0:.15)
		(0,8)--(0,10)coordinate(B)--(8,1.5)coordinate(A)--(0,0)coordinate(C)
		(A)--(8,0)coordinate(H)--(C)
		;
		\draw[thin] pic[draw, angle radius = 12pt, "$44^\circ$", angle eccentricity = 1.5]{ angle = B--A--C};
		\draw[thin] pic[draw, angle radius = 6pt]{right angle = A--H--C};

		\fill[blue!30,opacity=.5]
		(1.5,8.7) ellipse ({.5} and {.4})
		(-1.5,8.7) ellipse ({.5} and {.4})
		(0,9.6) ellipse ({.5} and {.4})
		;
		\draw
		(-.3,.1) arc (180:360:{.3} and {.1})
		(-.4,.2) arc (180:360:{.4} and {.1})
		(-.25,1.7) arc (180:360:{.25} and {.1})
		;
		\draw[stealth-stealth] (-1,0)--(-1,10)node[pos=.5,left]{$h$};
		\draw[stealth-stealth] (0,-.5)--(8,-.5)node[pos=.5,below]{$6$ m};
		\draw[stealth-stealth] (8.5,1.5)--(8.5,0)node[pos=.5,right]{$1$ m};
		\foreach \x/\g in {A/45,B/90,C/-135,H/-45}
		\fill (\x) circle (1.5pt)
		+(\g:5mm) node {$\x$};

	\end{tikzpicture}
	}
	\loigiai{
	Trong tam giác $AHB$ vuông tại $H$ ta có:\\
	$AB=\sqrt{AH^2+HB^2}=\sqrt{37}$, $\tan \widehat{ABH}=\dfrac{AH}{HB}=\dfrac{1}{6}\Rightarrow \widehat{ABH}\approx {{9{,}5}^\circ}$.\\
	Suy ra $\widehat{ABC}\approx {{90}^\circ}-{{9{,}5}^\circ}\approx {{80{,}5}^\circ}\Rightarrow \widehat{ACB}={{180}^\circ}-\widehat{BAC}-\widehat{ABC}\approx {{55{,}5}^\circ}$.\\
	Áp dụng định lý sin trong tam giác $ABC$ ta có: $\dfrac{AB}{\sin C}=\dfrac{BC}{\sin A}\Rightarrow BC=\dfrac{AB \cdot \sin A}{\sin C}\approx 5{,}1$.\\
	Vậy cột đèn đường có chiều cao xấp xỉ $5{,}1\text{m}$.
	}
\end{ex}
%Câu 36
\begin{ex}
    \begin{enumerate}[a)]
        \item Xác định điều kiện của $a,b$ để $A\cap B=\varnothing $ với $A=\left[a-1;a+2\right]$ và $B=\left(b;b+4\right]$.
        \item Xác định điều kiện của $a$ để $E\subset \left(C\cup D\right)$ với $C=[-1;4];D=\mathbb{R}\setminus (-3;3)$ và $E=\left[a-2;a\right]$.
    \end{enumerate}
	
		\loigiai{
		a) $A\cap B=\varnothing $ với $A=\left[a-1;a+2\right]$, $B=\left(b;b+4\right]$.\\
	$A\cap B=\varnothing \Leftrightarrow B\subset \mathbb{R}\setminus A$\\
	Từ đó, $A\cap B=\varnothing \Leftrightarrow \hoac{& b\ge a+2 \\& b+4<a-1}\Leftrightarrow \hoac{& a-b\le -2 \\& a-b>5}\cdot $\\
	Vậy với $\hoac{& a-b\le -2 \\& a-b>5}$ thì $A\cap B=\varnothing $.\\
	b) $E\subset \left(C\cup D\right)$ với $C=[-1;4];D=\mathbb{R}\setminus (-3;3)$, và $E=\left[a-2;a\right]$.\\
	Ta có $C\cup D=\left(-\infty ;-3\right]\cup \left[-1;+\infty\right)$.\\
	$E\subset \left(C\cup D\right)\Leftrightarrow \hoac{& a\le -3 \\& a-2\ge -1}\Leftrightarrow \hoac{& a\le -3 \\& a\ge 1}$.\\
	Vậy với $\hoac{& a\le -3 \\& a\ge 1}$ thì $E\subset \left(C\cup D\right)$
	}
\end{ex}
%Câu 37
\begin{ex}
	Cho góc $\alpha $ thỏa mãn $\sin \alpha =\dfrac{3}{4}$. Tính giá trị của biểu thức: $P=2{{\cos }^2}\alpha -3{{\tan }^2}\alpha $
	\loigiai{
	Ta có:\\
	$P=2{{\cos }^2}\alpha -3{{\tan }^2}\alpha =2\left(1-{{\sin }^2}\alpha\right)-\dfrac{3{{\sin }^2}\alpha }{{{\cos }^2}\alpha }$\\
	$=2-2{{\sin }^2}\alpha -\dfrac{3{{\sin }^2}\alpha }{1-{{\sin }^2}\alpha }=2-2{{\left(\dfrac{3}{4}\right)}^2}-\dfrac{3{{\left(\dfrac{3}{4}\right)}^2}}{1-{{\left(\dfrac{3}{4}\right)}^2}}=-\dfrac{167}{56}$
	}
\end{ex}
%Câu 38
\begin{ex}
Bác Ngọc thực hiện chế độ ăn kiêng với nhu cầu tối thiểu hàng ngày qua thức uống là 300 calo, 36 đơn vị vitamin A và 90 đơn vị vitamin C. Một cốc đồ uống ăn kiêng thứ nhất giá 20 nghìn đồng có dung tích 200ml cung cấp 60 calo, 12 đơn vị vitamin A và 10 đơn vị vitamin C. Một cốc đồ uống ăn kiêng thứ hai giá 25 nghìn đồng có dung tích 200ml cung cấp 60 calo, 6 đơn vị vitamin A và 30 đơn vị vitamin C. Biết rằng bác Ngọc không thể uống quá 2 lít thức uống mỗi ngày. Hãy cho biết bác Ngọc cần uống mỗi loại thức uống bao nhiêu cốc để tiết kiệm chi phí nhất mà vẫn đảm bảo nhu cầu tối thiểu trên.
\loigiai{
	Gọi số cốc đồ uống ăn kiêng thứ nhất và thứ hai bác Ngọc cần uống mỗi ngày lần lượt là $x$ và $y\left(x,y\,\in \mathbb{N}\right)$.\\
	Khi đó, lượng calo nhận được là $60x+60y$, lượng vitamin A nhận được là $12x+6y$ đơn vị, lượng vitamin C nhận được là $10x+30y$ đơn vị. Tổng dung tích thức uống nhận được là $200x+200y$ ml. Số tiền cần để mua thức uống là $T=20x+25y$.\\
	Căn cứ nhu cầu tối thiểu, ta có hệ bất phương trình:\\
	$\heva{& x\ge 0 \\& y\ge 0 \\& 60x+60y\ge 300 \\& 12x+6y\ge 36 \\& 10x+30y\ge 90 \\& 200x+200y\le 2000}$\\
	Bài toán trở thành tìm $(x;y)$ thỏa mãn sao cho $T=20x+25y$ đạt giá trị nhỏ nhất.\\
	Biểu diễn miền nghiệm của hệ, ta được miền nghiệm là miền không bị gạch, kể cả đường biên trong hình vẽ sau:
    \begin{center}
        
	\begin{tikzpicture}[line join = round, line cap = round, >=stealth, thick, scale = 1]
		\draw[->] (-.5,0)--(0,0) node[below left]{$O$}--(11,0) node[above right]{$x$};
		\foreach \x in {1,...,10} \draw[thin] (\x,2pt)--(\x,-2pt) node[below]{$\x$};
		\draw[->] (0,-.5)--(0,11) node[above right]{$y$};
		\foreach \y in {1,...,10} \draw[thin] (2pt,\y)--(-2pt,\y) node[left]{$\y$};
		\draw[thin,dashed] (-.3,-.3) grid (10.8,10.8);
		\draw
		plot[domain = -.3:5.3, samples = 200, variable=\x]({\x},{5-\x})
		plot[domain = -.3:3.2, samples = 200, variable=\x]({\x},{6-2*\x})
		plot[domain = -.3:9.3, samples = 200, variable=\x]({\x},{3-\x/3})
		plot[domain = -.3:10.3, samples = 200, variable=\x]({\x},{10-\x})
		;
		\fill[pattern=north east lines]
		(9,0)node[below right]{$A$}
		--(10,0)node[above right]{$B$}
		--(0,10)node[above right]{$C$}
		--(0,6)node[below left]{$D$}
		--(1,4)node[below left]{$E$}
		--(3,2)node[below left]{$F$};
	\end{tikzpicture}
    \end{center}
	Dễ thấy $A(9;0),B(10;0),C(0;10),D(0;6),E(1;4),F(3;2)$. Ta có:\\
	\begin{center}
	\begin{tabular}{|c|c|c|c|c|c|}
	\hline
	$A(9;0)$ & $B(10;0)$ & $C(0;10)$ & $D(0;6)$ & $E(1;4)$ & $F(3;2)$\\
	\hline
	$T=180$ & $T=200$ & $T=250$ & $T=150$ & $T=120$ & $T=110$\\
	\hline
    \end{tabular}
	\end{center}
	Như vậy, bác Ngọc nên uống 3 cốc thức uống loại 1, 2 cốc thức uống loại 2
}
\end{ex}
%Câu 39
\begin{ex}
	Cho tam giác $ABC$ có trọng tâm $G$ và hai trung tuyến $AM$, $BN$. Biết $AM=15$, $BN=12$ và tam giác $CMN$ có diện tích bằng $15\sqrt{3}$. Tính độ dài đoạn thẳng $MN$.
	\loigiai{
	Do $G$ là trọng tâm tam giác $ABC$ nên $GM=5,GN=4$.\\
	Ta có: ${{S}_{MGN}}=\dfrac{1}{3}{{S}_{AMN}}=\dfrac{1}{3} \cdot {{S}_{MNC}}=5\sqrt{3}$.\\
	Lại có: ${{S}_{MGN}}=\dfrac{1}{2} \cdot GM \cdot GN \cdot \sin \widehat{MGN}$\\
	$\Rightarrow \sin \widehat{MGN}=\dfrac{2{{S}_{MGN}}}{GM \cdot GN}=\dfrac{2 \cdot 5\sqrt{3}}{5 \cdot 4}=\dfrac{\sqrt{3}}{2}\Rightarrow \hoac{& \widehat{MGN}={{60}^\circ} \\& \widehat{MGN}={{120}^\circ}}$.\\
	* Trường hợp 1: $\widehat{MGN}={{60}^\circ}$\\
	Áp dụng định lí cosin cho tam giác $GMN$, ta có:\\
	$MN^2=GM^2+GN^2-2 \cdot GM \cdot GN \cdot \cos \widehat{MGN}=25+16-2 \cdot 5 \cdot 4 \cdot \dfrac{1}{2}=21\Leftrightarrow MN=\sqrt{21}$.\\
	* Trường hợp 2: $\widehat{MGN}={{120}^\circ}$\\
	Áp dụng định lí cosin cho tam giác $GMN$, ta có:\\
	$MN^2=GM^2+GN^2-2 \cdot GM \cdot GN \cdot \cos \widehat{MGN}=25+16+2 \cdot 5 \cdot 4 \cdot \dfrac{1}{2}=61\Leftrightarrow MN=\sqrt{61}$.\\
	Vậy $MN=\sqrt{21}$ hoặc $MN=\sqrt{61}$.
	}
\end{ex}
