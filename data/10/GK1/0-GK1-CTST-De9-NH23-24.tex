\begin{name}
	{\tenchude}
	{TOÁN 10}
	{LỚP TOÁN THẦY PHÁT}
	{Thời gian: 90 phút - Không kể thời gian phát đề}
\end{name}
\noindent{\bf\fontfamily{qag}\selectfont\color{violet}A. PHẦN TRẮC NGHIỆM}
\Opensolutionfile{ans}[ans/ans-0-GK1-CTST-De9-NH23-24]
%%==========Câu 1
\begin{ex}%[0T1Y1-1]
	Trong các câu sau, có bao nhiêu câu là mệnh đề toán học?
		\begin{enumerate}
		\item Thời tiết hôm nay đẹp quá!
		\item $\sqrt{2}$ là số vô tỉ.
		\item $3^2+4^2=7^2$.
		\item Vàng là kim loại đẹp nhất trên thế giới
	\end{enumerate}
	\choice
	{\True $2$}
	{$3$}
	{$1$}
	{$4$}
	\loigiai{
\begin{enumerate}
	\item Thời tiết hôm nay đẹp quá! Đây không là mệnh đề.
	\item $\sqrt{2}$ là số vô tỉ. Đây là mệnh đề toán học.
	\item $3^2+4^2=7^2$. Đây là mệnh đề toán học.
	\item Vàng là kim loại đẹp nhất trên thế giới. Đây không là mệnh đề.
\end{enumerate}
	}
\end{ex}
%%==========Câu 2
\begin{ex}%[0T1Y1-2]
	Trong các mệnh đề sau, mệnh đề nào đúng?
	\choice
	{$\sqrt{3}+\sqrt{5}=\sqrt{3+5}$}
	{\True Số $2$ là số nguyên tố chẵn duy nhất}
	{Tam giác $ABC$ vuông thì $AB<BC$}
	{$\pi=3{,}14$}
	\loigiai{
	\lq\lq Số $2$ là số nguyên tố chẵn duy nhất\rq\rq \, là mệnh đề đúng.
	}
\end{ex}
%%==========Câu 3
\begin{ex}%[0T1Y1-2]
	Trong các phát biểu sau, phát biểu nào là mệnh đề \textbf{sai}?
	\choice
	{Số $3$ là số nguyên tố}
	{\True $\pi$ là một số hữu tỉ}
	{Bạn khỏe không?}
	{$11$ là số tự nhiên lẻ}
	\loigiai{
		\lq\lq $\pi$ là một số hữu tỉ\rq\rq \, là mệnh đề sai.
	}
\end{ex}
%%==========Câu 4
\begin{ex}%[0T1Y1-1]
	Trong các phát biểu sau, phát biểu nào là mệnh đề chứa biến?
	\choice
	{$\pi <4$}
	{$16$ là số chính phương}
	{\True $2x+3>0$}
	{$-3\in\mathbb{Q}$}
	\loigiai{
		\lq\lq $2x+3>0$\rq\rq \, là mệnh đề chứa biến.
	}
\end{ex}
%%==========Câu 5
\begin{ex}%[0T1B1-5]
	Mệnh đề phủ định của mệnh đề $A\colon$\lq\lq $\forall x\in\mathbb{R}| x\ne \dfrac{1}{x}$\rq\rq\, là
	\choice
	{$\overline{A}\colon$\lq\lq $\forall x\in\mathbb{R}| x= \dfrac{1}{x}$\rq\rq}
	{\True $\overline{A}\colon$\lq\lq $\exists x\in\mathbb{R}| x= \dfrac{1}{x}$\rq\rq}
	{$\overline{A}\colon$\lq\lq $\forall x\in\mathbb{R}| x\le \dfrac{1}{x}$\rq\rq}
	{$\overline{A}\colon$\lq\lq $\exists x\in\mathbb{R}| x> \dfrac{1}{x}$\rq\rq}
	\loigiai{
		Mệnh đề phủ định của mệnh đề $A\colon$\lq\lq $\forall x\in\mathbb{R}| x\ne \dfrac{1}{x}$\rq\rq\, là $\overline{A}\colon$\lq\lq $\exists x\in\mathbb{R}| x= \dfrac{1}{x}$\rq\rq.
	}
\end{ex}
%%==========Câu 6
\begin{ex}%[0T1B1-5]
	Trong các mệnh đề sau, mệnh đề phủ định của mệnh đề nào là đúng?
	\choice
	{$A\colon$\lq\lq $\forall n\in\mathbb{N}| n^2\ge 0$\rq\rq}
	{\True $B\colon$\lq\lq $\exists x\in\mathbb{Q}| x^2=5$\rq\rq}
	{$C\colon$\lq\lq $\exists x\in\mathbb{R}| x^3<x^2$\rq\rq}
	{$D\colon$\lq\lq $\forall x\in\mathbb{R}| x^2+1>0$\rq\rq}
	\loigiai{
		Mệnh đề phủ định của mệnh đề $B\colon$\lq\lq $\exists x\in\mathbb{Q}| x^2=5$\rq\rq\, là $\overline{B}\colon$\lq\lq $\forall x\in\mathbb{Q}| x^2\ne 5$\rq\rq \,là mệnh đề đúng.
	}
\end{ex}
%%==========Câu 7
\begin{ex}%[0T1Y3-1]
	Cho hai tập hợp $A=\{1;3;5\}$ và $B=\{1;2;3;4;5\}$. Tìm $A\cup B$.
	\choice
	{$A\cup B=\{3;5\}$}
	{\True $A\cup B=\{1;2;3;4;5\}$}
	{$A\cup B=\{2;4\}$}
	{$A\cup B=\{1;3;5\}$}
	\loigiai{
	Ta có $A\cup B=\{1;2;3;4;5\}$.
	}
\end{ex}
%%==========Câu 8
\begin{ex}%[0T1B3-1]
	Cho hai tập hợp $A=(-\infty;2]$ và $B=(-6;+\infty)$. Tìm $A\cap B$.
	\choice
	{$A\cap B=\{-6;2\}$}
	{\True $A\cap B=(-6;2]$}
	{$A\cap B=(-6;2)$}
	{$A\cap B=(-\infty;+\infty)$}
	\loigiai{
		Ta có $A\cap B=(-6;2]$.
	}
\end{ex}
%%==========Câu 9
\begin{ex}%[0T1B3-1]
	Cho hai tập hợp $A=(-\infty;8)$ và $B=[-5;10]$. Khẳng định nào sau đây đúng?
	\choice
	{\True $A\setminus B=(-\infty;-5)$}
	{$A\setminus B=(-\infty;-5]$}
	{$A\setminus B=(-\infty;10]$}
	{$A\setminus B=[8;10]$}
	\loigiai{
		Ta có $A\setminus B=(-\infty;-5)$.
	}
\end{ex}
%%==========Câu 10
\begin{ex}%[0T1Y2-1]
	Cho hai tập hợp $A=\left\{x\in\mathbb{N}| |x|<3\right\}$. Tập hợp $A$ chứa bao nhiêu phần tử?
	\choice
	{$5$}
	{$7$}
	{\True $3$}
	{$2$}
	\loigiai{
		Ta có $|x|<3\Leftrightarrow -3<x<3$.\\
		Vì $x\in\mathbb{N}$ nên $A=\{0;1;2\}$. Vậy, tập hợp $A$ chứa $3$ phần tử.
	}
\end{ex}
%%==========Câu 11
\begin{ex}%[0T1Y2-2]
	Hình nào sau đây minh họa tập $B$ là con của tập $A$?
	\choice
{
	\begin{tikzpicture}[thick,rotate=-80,scale=.5]
		\def\ElipA{(0,0) ellipse (2cm and 3cm)}
		\def\ElipB{(1,3) ellipse (2cm and 3cm)}
		\begin{scope}
			\clip \ElipA;
			\fill[pattern = north east lines] \ElipB;
		\end{scope}
		\draw\ElipA node[left=0.7cm] {$A$};
		\draw\ElipB node[right=0.7cm] {$B$};
	\end{tikzpicture}
}
{
	\begin{tikzpicture}[thick,rotate=-80,scale=.5]
		\def\ElipA{(0,0) ellipse (2cm and 3cm)}
		\def\ElipB{(1,3) ellipse (2cm and 3cm)}
		\begin{scope}
			\clip \ElipA;
			\fill[pattern = north east lines] \ElipB;
		\end{scope}
		\draw\ElipA node[left=0.7cm] {$B$};
		\draw\ElipB node[right=0.7cm] {$A$};
	\end{tikzpicture}
}
{\True 
	\begin{tikzpicture}[thick,rotate=-80,scale=.5]
		\def\ElipA{(0,0) ellipse (1cm and 1.5cm)}
		\def\ElipB{(0,0) ellipse (2cm and 3cm)}
		\begin{scope}
			\clip \ElipA;
			\fill[pattern = north east lines] \ElipB;
		\end{scope}
		\draw\ElipA node{$B$};
		\draw\ElipB node[right=0.7cm] {$A$};
	\end{tikzpicture}	
}
{
\begin{tikzpicture}[thick,rotate=-80,scale=.5]
	\def\ElipA{(0,0) ellipse (1cm and 1.5cm)}
	\def\ElipB{(0,0) ellipse (2cm and 3cm)}
	\begin{scope}
		\clip \ElipA;
		\fill[pattern = north east lines] \ElipB;
	\end{scope}
	\draw\ElipA node{$A$};
	\draw\ElipB node[right=0.7cm] {$B$};
\end{tikzpicture}	
}
	\loigiai{
		
	}
\end{ex}
%%==========Câu 12
\begin{ex}%[0T1K3-1]
	Cho hai tập hợp $A=[-2;3)$ và $B=[m;m+5)$. Tìm tất cả các giá trị của tham số $m$ để $A\cap B\ne\varnothing$
	\choice
	{$-7<m\le-2$}
	{$-2<m\le 3$}
	{$-2\le m<3$}
	{\True $-7<m<3$}
	\loigiai{
		Ta giải mệnh đề phủ định, tức là tìm $m$ để $A\cap B=\varnothing$. Ta có hai trường hợp
		\begin{itemize}
			\item $m\ge 3$.
			\item $m+5\le -2\Leftrightarrow m\le -7$.
		\end{itemize}
	Do đó, để $A\cap B\ne\varnothing$ thì $-7<m<3$.
	}
\end{ex}
%%==========Câu 13
\begin{ex}%[0T2Y1-1]
	Bất phương trình nào sau đây là bất phương trình bậc nhất hai ẩn?
	\choice
	{$x(x-y)\ge 0$}
	{$2x-3y^2+1\le 0$}
	{$2x-xy+1>0$}
	{\True $2x-3y+1<0$}
	\loigiai{
		Bất phương trình bậc nhất hai ẩn là $2x-3y+1<0$.
	}
\end{ex}
%%==========Câu 14
\begin{ex}%[0T2Y1-2]
	Cặp số $(x;y)$ nào sau đây là nghiệm của bất phương trình $2x+y-1\ge 0$?
	\choice
	{$(0;-1)$}
	{\True $(0;2)$}
	{$(1;-2)$}
	{$(-2;1)$}
	\loigiai{
	Ta có $2\cdot 0+2-1=1\ge 0$ nên $(0;2)$ là nghiệm của bất phương trình $2x+y-1\ge 0$.
	}
\end{ex}
%%==========Câu 15
\begin{ex}%[0T2Y1-2]
	Cho bất phương trình bậc nhất hai ẩn $3x-4y+7\le 0$. Cặp số nào dưới đây không thuộc miền nghiệm của bất phương trình đã cho?
	\choice
	{$(-1;1)$}
	{\True $(1;1)$}
	{$(-2;1)$}
	{$(1;3)$}
	\loigiai{
		Ta có $3\cdot 1-4\cdot 1+7=6> 0$ nên $(1;1)$ không thuộc miền nghiệm của bất phương trình $3x-4y+7\le 0$.
	}
\end{ex}
%%==========Câu 16
\begin{ex}%[0T2B1-2]
	\immini{
Cho miền nghiệm (phần không gạch chéo) của bất phương trình bậc nhất hai ẩn như hình vẽ. Bất phương trình nào sau đây nhận miền nghiệm trên làm tập nghiệm?
	\choice
	{$3x+2y>8$}
	{$3x+2y<8$}
	{$2x+3y>8$}
	{\True $2x+3y<8$}
}
{
\begin{tikzpicture}[
	line join=round, line cap=round, >=stealth, thick, font=\footnotesize,
	declare function={xmin=-1; xmax=5; ymin=-1; ymax=4;},scale=0.7
	]
	\clip (xmin,ymin) rectangle (xmax,ymax);
	%BPT ax+by+c<0, a>0, b>0
	\begin{scope}[declare function={a=2; b=3; c=-8; m=(-b/a)*ymin-(c/a); n=(-b/a)*ymax-(c/a);}]
		\draw[thick] (n,ymax)--(m,ymin) node[below, pos=.5, sloped]{$ (d_1) $};
		\fill[pattern=north east lines, opacity=.5] (n,ymax)--(m,ymin)--(xmax,ymin)--(xmax,ymax)--cycle;
	\end{scope}
	%Vẽ hệ trục
	\draw[->] (xmin,0)--(0,0) node[below right]{$O$}--(xmax,0) node[above left]{$x$};
	\draw[->] (0,ymin)--(0,ymax) node[below right]{$y$};
	%Vẽ các điểm trên trục Ox
	\foreach \x/\g in {1/-90,2/-90,3/-90,4/-90}
	\draw[thin] (\x,2pt)--(\x,-2pt) + (\g:3mm) node{$\x$};
	%Vẽ các điểm trên trục Oy
	\foreach \y/\g in {1/180,2/180,3/180}
	\draw[thin] (2pt,\y)--(-2pt,\y) + (\g:3mm) node{$\y$};
\end{tikzpicture}

}
	\loigiai{
	Đường thẳng đi qua điểm $(4;0)$ nên loại hai bất phương trình $3x+2y<8$ và $3x+2y>8$.\\
	Vì điểm $O(0;0)$ thuộc miền nghiệm nhưng không là nghiệm của bất phương trình $2x+3y>8$.\\
	Vậy bất phương trình thỏa mãn yêu cầu bài toán là $2x+3y<8$.
	}
\end{ex}
%%==========Câu 17
\begin{ex}%[0T2Y2-1]
	Hệ bất phương trình nào sau đây là hệ bất phương trình bậc nhất hai ẩn?
	\choice
	{$\heva{&2x-y>0\\&x^2-1\le 0}$}
	{\True $\heva{&2x-y-1>0\\&x-1\le 0}$}
	{$\heva{&2x-y>y^2\\&x-1\le 0}$}
	{$\heva{&x^2-y^2>0\\&x-1\le 0}$}
	\loigiai{
		Hệ bất phương trình $\heva{&2x-y-1>0\\&x-1\le 0}$ là hệ bất phương trình bậc nhất hai ẩn.
	}
\end{ex}
%%==========Câu 18
\begin{ex}%[0T2Y2-1]
	Trong mặt phẳng $Oxy$, điểm nào sau đây thuộc miền nghiệm của hệ bất phương trình $\heva{&x-y+1>0\\&x+y-1<0}$?
	\choice
	{\True $M(1;-1)$}
	{$N(1;2)$}
	{$P(-1;2)$}
	{$Q(1;1)$}
	\loigiai{
		Tọa độ $M(1;-1)$ thế vào hệ $\heva{&x-y+1>0\\&x+y-1<0}$ thỏa mãn nên $M(1;-1)$ thuộc miền nghiệm của hệ bất phương trình này.
	}
\end{ex}
%%==========Câu 19
\begin{ex}%[0T2Y2-1]
	Trong các cặp số sau, cặp nào không là nghiệm của hệ bất phương trình $\heva{&x+y-2\le 0\\&2x-3y+2>0}$?
	\choice
	{$(1;1)$}
	{$(0;0)$}
	{\True $(-1;1)$}
	{$(-1;-1)$}
	\loigiai{
	Ta thay cặp số $(-1;1)$ vào hệ ta thấy không thỏa mãn.
	}
\end{ex}
%%==========Câu 20
\begin{ex}%[0T2B2-3]
	\immini{
		Miền trong tam giác $ABC$ kể cả ba cạnh sau đây là miền nghiệm của hệ bất phương trình nào trong bốn hệ bất phương trình dưới đây
		\choice
		{$\heva{&y\ge 0\\&x-2y\ge 2\\&2x+y\le -2}$}
		{$\heva{&x>0\\&x-2y\le -2\\&2x+y\le 2}$}
		{\True $\heva{&x\ge 0\\&x-2y\le 2\\&2x+y\le 2}$}
		{$\heva{&x\ge 0\\&x-2y\ge 2\\&2x+y\le 2}$}
	}
	{
		\begin{tikzpicture}[
			line join=round, line cap=round, >=stealth, thick, font=\small,
			declare function={xmin=-0.5; xmax=2.5; ymin=-2; ymax=3;}
			]
			\clip (xmin,ymin) rectangle (xmax,ymax);
			%BPT ax+c>=0, a>0
			\begin{scope}[declare function={a=1; c=0;m=-c/a;}]
				\draw[thick] (m,ymin)--(m,ymax);
				\fill[pattern=north east lines, opacity=.5] (xmin,ymin) rectangle (m,ymax);
			\end{scope}
			%BPT ax+by+c<=0, a>0, b<0
			\begin{scope}[declare function={a=1; b=-2; c=-2; m=(-b/a)*ymin-(c/a); n=(-b/a)*ymax-(c/a);}]
				\draw[thick] (m,ymin)--(n,ymax);
				\fill[pattern=north east lines, opacity=.5] (m,ymin)--(n,ymax)--(xmax,ymax)--(xmax,ymin)--cycle;
			\end{scope}
			%BPT ax+by+c<=0, a>0, b>0
			\begin{scope}[declare function={a=2; b=1; c=-2; m=(-b/a)*ymin-(c/a); n=(-b/a)*ymax-(c/a);}]
				\draw[thick] (n,ymax)--(m,ymin);
				\fill[pattern=north east lines, opacity=.5] (n,ymax)--(m,ymin)--(xmax,ymin)--(xmax,ymax)--cycle;
			\end{scope}
			%Vẽ hệ trục
			\draw[->] (xmin,0)--(0,0) node[below right]{$O$}--(xmax,0) node[above left]{$x$};
			\draw[->] (0,ymin)--(0,ymax) node[below right]{$y$};
			%Vẽ các điểm trên trục Ox
			\foreach \x/\g in {1/-90,2/-90,3/-90}
			\draw[thin] (\x,2pt)--(\x,-2pt) + (\g:3mm) node{$\x$};
			%Vẽ các điểm trên trục Oy
			\foreach \y/\g in {-2/180,-1/180,1/180,2/180}
			\draw[thin] (2pt,\y)--(-2pt,\y) + (\g:3mm) node{$\y$};
			\draw (0,2)node[right]{$A$} (1.5,-0.5)node[below]{$B$} (0,-1)node[right]{$C$};
		\end{tikzpicture}
	}
	\loigiai{
		Cạnh $AC$ có phương trình $x=0$ và cạnh $AC$ nằm trong miền nghiệm nên $x\ge 0$ là một bất phương trình của hệ.\\
		Cạnh $AB$ qua hai điểm $\left(1;0\right)$ và $(0;2)$ nên có phương trình $\dfrac{x}{1}+\dfrac{y}{2}=1\Leftrightarrow 2x+y=2$.\\
		Vậy hệ bất phương trình cần tìm là $\heva{&x\ge 0\\&x-2y\le 2\\&2x+y\le 2.}$
	}
\end{ex}
%%==========Câu 21
\begin{ex}%[0T2K2-3]
	Cho hệ bất phương trình $\heva{&x+y\ge 2\\&2x-3y\ge -1\\&6x+y\le 22}$. Tìm giá trị lớn nhất của biểu thức $F(x;y)=5x-y$, với $(x;y)$ nằm trong miền nghiệm của hệ bất phương trình đã cho.
		\choice
		{$-2$}
		{$11$}
		{\True $22$}
		{$33$}
	\loigiai{
		\immini
		{
		Biểu diễn miền nghiệm của hệ bất phương trình đã cho trên hệ trục tọa độ $Oxy$ , ta được miền
		tam giác $MNP$ như hình vẽ bên.\\
		Tọa độ các đỉnh của tam giác $MNP$ là $M(0;-2)$, $N(4;-2)$ và $P(3;4)$.\\
		Với $F(x;y)=5x-y$, ta có $F(0;2)=-2$, $F(4;-2)=22$ và $F(3;4)=11$.\\
		Vậy $F$ đạt giá trị lớn nhất bằng $22$ tại $N(4;-2)$.
	}
{
\begin{tikzpicture}[
line join = round, line cap = round >=stealth, thick, font=\footnotesize,
	declare function={xmin=-0.5; xmax=5; ymin=-2.5; ymax=5;},scale=0.7
	]
	\clip (xmin,ymin) rectangle (xmax,ymax);
	%BPT ax+by+c>=0, a>0, b>0
	\begin{scope}[declare function={a=1; b=1; c=-2; m=(-b/a)*ymin-(c/a); n=(-b/a)*ymax-(c/a);}]
		\draw[blue] (n,ymax)--(m,ymin) node[above, pos=.5, sloped]{$ (d_1) $};
		\fill[pattern=north east lines, opacity=.5] (n,ymax)--(m,ymin)--(xmin,ymin)--(xmin,ymax)--cycle;
	\end{scope}
	%BPT ax+by+c>=0, a>0, b<0
	\begin{scope}[declare function={a=2; b=-3; c=6; m=(-b/a)*ymin-(c/a); n=(-b/a)*ymax-(c/a);}]
		\draw[blue] (m,ymin)--(n,ymax) node[below, pos=.5, sloped]{$ (d_2) $};
		\fill[pattern=north east lines, opacity=.5] (m,ymin)--(n,ymax)--(xmin,ymax)--(xmin,ymin)--cycle;
	\end{scope}
	%BPT ax+by+c<=0, a>0, b>0
	\begin{scope}[declare function={a=6; b=1; c=-22; m=(-b/a)*ymin-(c/a); n=(-b/a)*ymax-(c/a);}]
		\draw[blue] (n,ymax)--(m,ymin) node[below, pos=.5, sloped]{$ (d_3) $};
		\fill[pattern=north east lines, opacity=.5] (n,ymax)--(m,ymin)--(xmax,ymin)--(xmax,ymax)--cycle;
	\end{scope}
	%Vẽ hệ trục
	\draw[->] (xmin,0)--(0,0) node[below right]{$O$}--(xmax,0) node[above left]{$x$};
	\draw[->] (0,ymin)--(0,ymax) node[below right]{$y$};
	%Vẽ các điểm trên trục Ox
	\foreach \x/\g in {0/-90,1/-90,2/-90,3/-90,4/-90}
	\draw[thin] (\x,2pt)--(\x,-2pt) + (\g:3mm) node{$\x$};
	%Vẽ các điểm trên trục Oy
	\foreach \y/\g in {-2/180,-1/180,1/180,2/180,3/180,4/180}
	\draw[thin] (2pt,\y)--(-2pt,\y) + (\g:3mm) node{$\y$};
\end{tikzpicture}
}
	}
\end{ex}
%%==========Câu 22
\begin{ex}%[0T2T2-3]
	Ông An dự định trồng lúa và khoai lang trên một mảnh đất có diện tích $10$ ha. Nếu 	trồng $1$ ha lúa thì cần $10$ ngày công và thu được $20$ triệu đồng. Nếu trồng $1$ ha khoai lang thì cần $30$ ngày công và thu được $30$ triệu đồng. Biết rằng, Ông An chỉ có thể sử dụng không quá $180$ ngày cho công việc trồng lúa và khoai lang. Số tiền nhiều nhất Ông An thu được từ trồng hai loại cây nói trên là bao nhiêu?
	\choice
	{$180$ triệu đồng}
	{$200$ triệu đồng}
	{\True $240$ triệu đồng}
	{$260$ triệu đồng}
	\loigiai{
		\immini
		{
			Gọi $x$, $y$ lần lượt là số ha đất trồng lúa và khoai lang.\\
			Ta có các điều kiện như sau
			\begin{itemize}
				\item Số ha $x\ge 0$, $y\ge 0$.
				\item Diện tích canh tác không vượt quá $10$ ha nên $x+y\le 10$.
				\item Số ngày công sử dụng không vượt quá $180$ ngày nên $10x+30y\le 180$.
			\end{itemize}
		Từ đó, ta có hệ bất phương trình mô tả các điều kiện ràng buộc là $\heva{&x\ge 0; y\ge 0\\&x+y\le 10\\&10x+30y\le 180.}$
		}
		{
			\begin{tikzpicture}[
			line join = round, line cap = round >=stealth, thick, font=\footnotesize,
				declare function={xmin=-0.5; xmax=11; ymin=-0.5; ymax=11;},scale=0.7
				]
				\clip (xmin,ymin) rectangle (xmax,ymax);
				%BPT ax+c>=0, a>0
				\begin{scope}[declare function={a=1; c=0;m=-c/a;}]
					\draw[blue] (m,ymin)--(m,ymax) node[below, pos=.95, sloped]{$ (d_1) $};
					\fill[pattern=north east lines, opacity=.5] (xmin,ymin) rectangle (m,ymax);
				\end{scope}
				%BPT by+c>=0, b>0
				\begin{scope}[declare function={b=1; c=0;m=-c/b;}]
					\draw[blue] (xmin,m)--(xmax,m) node[above, pos=.95]{$ (d_2) $};
					\fill[pattern=north east lines, opacity=.5] (xmin,ymin) rectangle (xmax,m);
				\end{scope}
				%BPT ax+by+c<=0, a>0, b>0
				\begin{scope}[declare function={a=1; b=1; c=-10; m=(-b/a)*ymin-(c/a); n=(-b/a)*ymax-(c/a);}]
					\draw[blue] (n,ymax)--(m,ymin) node[below, pos=.5, sloped]{$ (d_3) $};
					\fill[pattern=north east lines, opacity=.5] (n,ymax)--(m,ymin)--(xmax,ymin)--(xmax,ymax)--cycle;
				\end{scope}
				%BPT ax+by+c<=0, a>0, b>0
				\begin{scope}[declare function={a=1; b=3; c=-18; m=(-b/a)*ymin-(c/a); n=(-b/a)*ymax-(c/a);}]
					\draw[blue] (n,ymax)--(m,ymin) node[below, pos=.5, sloped]{$ (d_4) $};
					\fill[pattern=north east lines, opacity=.5] (n,ymax)--(m,ymin)--(xmax,ymin)--(xmax,ymax)--cycle;
				\end{scope}
				%Vẽ hệ trục
				\draw[->] (xmin,0)--(0,0) node[below right]{$O$}--(xmax,0) node[above left]{$x$};
				\draw[->] (0,ymin)--(0,ymax) node[below right]{$y$};
				%Vẽ các điểm trên trục Ox
				\foreach \x/\g in {0/-90,1/-90,2/-90,3/-90,4/-90,5/-90,6/-90,7/-90,8/-90,9/-90,10/-90,11/-90}
				\draw[thin] (\x,2pt)--(\x,-2pt) + (\g:3mm) node{$\x$};
				%Vẽ các điểm trên trục Oy
				\foreach \y/\g in {0/180,1/180,2/180,3/180,4/180,5/180,6/180,7/180,8/180,9/180,10/180,11/180}
				\draw[thin] (2pt,\y)--(-2pt,\y) + (\g:3mm) node{$\y$};
				\path 
				(0,6)coordinate(B) (6,4)coordinate(C) (10,0)coordinate(D)
				;
				\foreach \x/\g in {B/-45,C/-90,D/150}\fill[black] (\x) circle (1pt)+(\g:.7)node{$\x$};
			\end{tikzpicture}
		}
	\noindent
	Bài toán trở thành: Tìm giá trị lớn nhất của biểu thức $F(x;y)=20x+30y$ với $(x;y)$ nằm	trong miền nghiệm của hệ bất phương trình trên.\\
	Biểu diễn miền nghiệm của hệ bất phương trình  trên hệ trục tọa độ $Oxy$ , ta được miền tứ giác $OBCD$ như hình vẽ bên.\\
	Tọa độ các đỉnh của tứ giác là $O(0;0)$, $B(0;6)$, $C(6;4)$, $D(10;0)$.\\
	Với $F(x;y)=20x+30y$, ta có $F(0;0)=0$, $F(0;6)=180$, $F(6;4)=240$ và $F(10;0)=200$.\\
	Suy ra $F$ đạt giá trị lớn nhất là $240$ tại $C(6;4)$.\\
	Vậy số tiền nhiều nhất Ông An thu được từ trồng hai loại cây nói trên là $240$ triệu đồng.
	}
\end{ex}
%%==========Câu 23
\begin{ex}%[0T4B1-2]
	Cho $\sin\alpha+\cos\alpha=\dfrac{\sqrt{2}}{4}$. Giá trị của $\sin\alpha\cdot\cos\alpha$ bằng bao nhiêu?
	\choice
	{$\sin\alpha\cdot\cos\alpha=\dfrac{1}{8}$}
	{$\sin\alpha\cdot\cos\alpha=\dfrac{\sqrt{2}}{2}$}
	{\True $\sin\alpha\cdot\cos\alpha=\dfrac{-7}{16}$}
	{$\sin\alpha\cdot\cos\alpha=\dfrac{7}{16}$}
	\loigiai{
	Ta có 
	\allowdisplaybreaks
	\begin{eqnarray*}
		\sin \alpha+\cos\alpha=\dfrac{\sqrt{2}}{4}&\Leftrightarrow&\left(\sin \alpha+\cos\alpha\right)^2=\dfrac{1}{8}\\
		&\Leftrightarrow&1+2\sin \alpha\cdot\cos\alpha=\dfrac{1}{8}\\
		&\Leftrightarrow&\sin\alpha\cdot\cos\alpha=\dfrac{\tfrac{1}{8}-1}{2}=-\dfrac{7}{16}.
	\end{eqnarray*}
	}
\end{ex}
%%==========Câu 24
\begin{ex}%[0T4B1-2]
	Cho hai góc nhọn $\alpha$ và $\beta$ phụ nhau, hệ thức nào dưới đây là \textbf{sai}?
	\choice
	{$\sin\beta=\cos\alpha$}
	{\True $\cos\alpha=-\sin\beta$}
	{$\cot\alpha=\tan\beta$}
	{$\tan\beta=\dfrac{1}{\tan\alpha}$}
	\loigiai{
		Ta có $\cos\alpha=\cos\left(90^\circ-\beta\right)=\sin\beta\ne-\sin\beta$.
	}
\end{ex}
%%==========Câu 25
\begin{ex}%[0T4B1-2]
	Biết $\tan\alpha=-3$, giá trị $M=\dfrac{6\sin\alpha-7\cos\alpha}{6\cos\alpha+7\sin\alpha}$ bằng
	\choice
	{$M=\dfrac{4}{3}$}
	{\True $M=\dfrac{5}{3}$}
	{$M=-\dfrac{4}{3}$}
	{$M=-\dfrac{5}{3}$}
	\loigiai{
		Ta có $M=\dfrac{6\sin\alpha-7\cos\alpha}{6\cos\alpha+7\sin\alpha}=\dfrac{6\tan\alpha-7}{7\tan\alpha+6}=\dfrac{5}{3}$.
	}
\end{ex}
%%==========Câu 26
\begin{ex}%[0T4B1-2]
	Biết $\sin\alpha=\dfrac{1}{3}$, giá trị $P=3\sin^2\alpha+4\cos^2\alpha-2$ bằng
	\choice
	{$P=\dfrac{9}{25}$}
	{$P=\dfrac{9}{17}$}
	{\True $P=\dfrac{17}{9}$}
	{$P=\dfrac{25}{9}$}
	\loigiai{
		Ta có \allowdisplaybreaks
		\begin{eqnarray*}
			P&=&3\sin^2\alpha+4\cos^2\alpha-2\\
			&=&3\sin^2\alpha+4(1-\sin^2\alpha)-2\\
			&=&-\sin^2\alpha+2\\
			&=&-\dfrac{1}{9}+2=\dfrac{17}{9}.
		\end{eqnarray*}
	}
\end{ex}
%%==========Câu 27
\begin{ex}%[0T4B1-2]
	Biết $\cos\alpha=\dfrac{1}{3}$, giá trị $P=2\cos\alpha-\sin^2\alpha$ bằng
	\choice
	{$P=\dfrac{16}{5}$}
	{$P=-\dfrac{4}{9}$}
	{$P=\dfrac{14}{9}$}
	{\True $P=\dfrac{-2}{9}$}
	\loigiai{
		Ta có \allowdisplaybreaks
		\begin{eqnarray*}
			P&=&2\cos\alpha-\sin^2\alpha\\
			&=&2\cos\alpha-(1-\cos^2\alpha)\\
			&=&\cos^2\alpha+2\cos\alpha-1\\
			&=&\dfrac{1}{9}+\dfrac{2}{3}-1=\dfrac{-2}{9}.
		\end{eqnarray*}
	}
\end{ex}
%%==========Câu 28
\begin{ex}%[0T4Y2-2]
	Cho tam giác $ABC$ với $BC=a$, $AC=b$ và $AB=c$. Công thức nào sau đây đúng?
	\choice
	{$a^2=b^2+c^2+2bc\cos A$}
	{$a^2=b^2+c^2+2bc\cos B$}
	{\True $a^2=b^2+c^2-2bc\cos A$}
	{$a^2=b^2+c^2-2bc\cos B$}
	\loigiai{
		Ta có $a^2=b^2+c^2-2bc\cos A$.
	}
\end{ex}
%%==========Câu 29
\begin{ex}%[0T4B2-2]
	Cho tam giác $ABC$ với $BC=a$, $AC=b$ và $AB=c$ và $\widehat{A}=120^\circ$. Khẳng định nào sau đây đúng?
	\choice
	{\True $a^2=b^2+c^2+bc$}
	{$a^2=b^2+c^2+3bc$}
	{$a^2=b^2+c^2-bc$}
	{$a^2=b^2+c^2-3bc$}
	\loigiai{
		Ta có $a^2=b^2+c^2-2bc\cos A=b^2+c^2-2bc\cos 120^\circ=b^2+c^2+bc$.
	}
\end{ex}
%%==========Câu 30
\begin{ex}%[0T4B3-1]
	Cho tam giác $ABC$ có tổng hai góc $B$ và $C$ bằng $135^\circ$ và độ dài $BC=a$. Tính bán kính đường tròn ngoại tiếp tam giác $ABC$.
	\choice
	{$\dfrac{a\sqrt{3}}{2}$}
	{$a\sqrt{3}$}
	{\True $\dfrac{a\sqrt{2}}{2}$}
	{$a\sqrt{2}$}
	\loigiai{
		Ta có $\widehat{A}=180^\circ-135^\circ=45^\circ$.\\
		Do đó, $2R=\dfrac{a}{\sin A}\Rightarrow R=\dfrac{a}{2\sin 45^\circ}=\dfrac{a\sqrt{2}}{2}$.
	}
\end{ex}
%%==========Câu 31
\begin{ex}%[0T4B3-1]
	Cho tam giác $ABC$ có $a=8$, $b=3$, $C=120^\circ$. Khi đó diện tích tam giác $ABC$ bằng
	\choice
	{\True $6\sqrt{3}$}
	{$12\sqrt{3}$}
	{$24$}
	{$12$}
	\loigiai{
		Ta có $S=\dfrac{1}{2}ab\sin C=\dfrac{1}{2}\cdot 8\cdot 3\sin 120^\circ=6\sqrt{3}$.
	}
\end{ex}
%%==========Câu 32
\begin{ex}%[0T4B3-1]
	Cho tam giác $ABC$ có $AB=6$, $AC=8$ và $BC=10$. Tính $R$ bán kính đường tròn ngoại tiếp tam giác đó.
	\choice
	{\True $5$}
	{$8$}
	{$20$}
	{$\dfrac{1}{5}$}
	\loigiai{
		Ta có $AB^2+AC^2=BC^2$ nên $\triangle ABC$ vuông tại $A$. Suy ra $R=\dfrac{BC}{2}=5$.\\
		Khi đó, $\dfrac{BC}{\sin A}=2R\Rightarrow R=\dfrac{BC}{2\sin A}=\dfrac{a}{2\sin 45^\circ}=\dfrac{a\sqrt{2}}{2}$.
	}
\end{ex}
%%==========Câu 33
\begin{ex}%[0T4B3-1]
	Cho tam giác $ABC$ thỏa mãn hệ thức $b+c=2a$. Trong các mệnh đề sau, mệnh đề nào đúng?
	\choice
	{$\cos B+\cos C=2\cos A$}
	{$\sin B+\sin C=\dfrac{1}{2}\sin A$}
	{\True $\sin B+\sin C=2\sin A$}
	{$\sin B+\cos C=2\sin A$}
	\loigiai{
		Ta có 
		\allowdisplaybreaks
		\begin{eqnarray*}
			\dfrac{a}{\sin A}=\dfrac{b}{\sin B}=\dfrac{c}{\sin C}&\Leftrightarrow&\dfrac{\dfrac{b+c}{2}}{\sin A}=\dfrac{b}{\sin B}=\dfrac{c}{\sin C}\\
			&\Leftrightarrow&\dfrac{b+c}{2\sin A}=\dfrac{b+c}{\sin B+\sin C}\\
			&\Leftrightarrow&\sin B+\sin C=2\sin A.
		\end{eqnarray*}
	}
\end{ex}
%%==========Câu 34
\begin{ex}%[0T4B3-1]
	Cho tam giác $ABC$ thỏa mãn hệ thức $b+c=2a$. Trong các mệnh đề sau, mệnh đề nào đúng?
	\choice
	{$\dfrac{4}{h_a}=\dfrac{1}{h_b}+\dfrac{1}{h_c}$}
	{$\dfrac{1}{h_a^2}=\dfrac{1}{h_b^2}+\dfrac{1}{h_c^2}$}
	{\True $\dfrac{2}{h_a}=\dfrac{1}{h_b}+\dfrac{1}{h_c}$}
	{$\dfrac{4}{h_a^2}=\dfrac{1}{h_b^2}+\dfrac{1}{h_c^2}$}
	\loigiai{
		Ta có $b+c=2a\Leftrightarrow \dfrac{2S}{h_b}+\dfrac{2S}{h_c}=2\cdot \dfrac{2S}{h_a}\Leftrightarrow \dfrac{2}{h_a}=\dfrac{1}{h_b}+\dfrac{1}{h_c}$.
	}
\end{ex}
%%==========Câu 35
\begin{ex}%[0T4T3-1]
	Hai chiếc tàu thủy $P$ và $Q$ trên biển cách nhau $100$m va thẳng hàng với chân $A$ của tháp hải đăng $AB$ trên bờ biển. Từ $P$ và $Q$ người ta nhìn chiều cao $AB$ của tháp dưới các góc $\widehat{BPA}=15^\circ$ và $\widehat{BQA}=55^\circ$. Tính chiều cao của tháp (kết quả làm tròn đến hàng đơn vị).
	\choice
	{$30$}
	{$32$}
	{$34$}
	{\True $33$}
	\loigiai{
	\immini{
	Ta có $\widehat{PBQ}=55^\circ-15^\circ=40^\circ$.\\
	Áp dụng định lý $\sin$ cho tam giác $PBQ$ ta có $\dfrac{BQ}{\sin 15^\circ}=\dfrac{100}{\sin 40^\circ}\Leftrightarrow BQ=\dfrac{100\cdot\sin 15^\circ}{\sin 40^\circ}$.\\
	Khi đó, chiều cao của tháp là $AB=\sin 55^\circ\cdot BQ\approx 33$m.	
}
{
\begin{tikzpicture}[scale=1, font=\footnotesize, line join=round, line cap=round, >=stealth]
	\path
	(0,0) coordinate (P)++(0:5)coordinate(Q)++(0:1.15)coordinate(A)++(90:1.65)coordinate(B)
	;
	\draw (P)--(Q)node[midway,above]{$100$ m}--(A)--(B)--cycle
	(B)--(Q)
	;
	\foreach \x/\g in {A/-90,B/90,P/-90,Q/-90}\fill[black] (\x) circle (1pt)+(\g:.3)node{$\x$};
	\draw pic[draw, angle radius = 12pt, "$15^\circ$", angle eccentricity = 1.5]{ angle = A--P--B};
		\draw pic[draw, angle radius = 12pt, "$55^\circ$", angle eccentricity = 1.5]{ angle = A--Q--B};
\end{tikzpicture}
}
	}
\end{ex}
\noindent{\bf\fontfamily{qag}\selectfont\color{violet}B. PHẦN TỰ LUẬN}
\setcounter{bt}{35}

%%==========Bài 1
\begin{bt}%[Pj17-0-CK2-NH22-23--TeamTeXHoa--VoVanTu]%[0T2B1-2]
	Biểu diễn miền nghiệm của bất phương trình $2x-4y>8$ trên mặt phẳng tọa độ $Oxy$.
	\loigiai{
		\immini
	{
		Vẽ đường thẳng $\Delta\colon 2x-4y-8=0$ đi qua hai điểm $A(4;0)$ và $B(0;-2)$.\\
		Xét gốc tọa độ $O(0;0)$, ta thấy $O\notin\Delta$ và $2\cdot 0-4\cdot 0-8<0$.\\
		Do đó, miền nghiệm của bất phương trình là nửa mặt phẳng không kể bờ $\Delta$, không chứa gốc tọa độ (miền không gạch chéo).
	}
{
\begin{tikzpicture}[
line join = round, line cap = round >=stealth, thick, font=\footnotesize,
	declare function={xmin=-2; xmax=5; ymin=-3; ymax=2;}
	]
	\clip (xmin,ymin) rectangle (xmax,ymax);
	%BPT ax+by+c>0, a>0, b<0
	\begin{scope}[declare function={a=2; b=-4; c=-8; m=(-b/a)*ymin-(c/a); n=(-b/a)*ymax-(c/a);}]
		\draw[blue] (m,ymin)--(n,ymax) node[below, pos=.5, sloped]{$ (d_1) $};
		\fill[pattern=north east lines, opacity=.5] (m,ymin)--(n,ymax)--(xmin,ymax)--(xmin,ymin)--cycle;
	\end{scope}
	%Vẽ hệ trục
	\draw[->] (xmin,0)--(0,0) node[below right]{$O$}--(xmax,0) node[above left]{$x$};
	\draw[->] (0,ymin)--(0,ymax) node[below right]{$y$};
	%Vẽ các điểm trên trục Ox
	\foreach \x/\g in {-1/-90,1/-90,2/-90,3/-90,4/-90}
	\draw[thin] (\x,2pt)--(\x,-2pt) + (\g:3mm) node{$\x$};
	%Vẽ các điểm trên trục Oy
	\foreach \y/\g in {-2/180,-1/180,1/180}
	\draw[thin] (2pt,\y)--(-2pt,\y) + (\g:3mm) node{$\y$};
\end{tikzpicture}
}
	}
\end{bt}
\begin{bt}%[0H2K1-2]
	Cho biết $\sin \alpha -\cos \alpha =\dfrac{1}{\sqrt{5}}$. Giá trị của $P=\sqrt{{\sin ^4}\alpha +{\cos ^4}\alpha }$ bằng bao nhiêu?
	   \loigiai
	   {Ta có $\sin \alpha -\cos \alpha =\dfrac{1}{\sqrt{5}}\Rightarrow {{(\sin \alpha -\cos \alpha )}^2}=\dfrac{1}{5}$\\
		   $\Leftrightarrow 1-2\sin \alpha \cos \alpha =\dfrac{1}{5}\Leftrightarrow \sin \alpha \cos \alpha =\dfrac{2}{5}$.\\
		   Ta có \begin{eqnarray*}
			   P&=&\sqrt{{\sin ^4}\alpha +{\cos ^4}\alpha }=\sqrt{(\sin ^2\alpha +\cos^2\alpha )^2-2\sin ^2\alpha \cos^2\alpha }\\
			   &=&\sqrt{1-2(\sin \alpha \cos\alpha )^2}=\dfrac{\sqrt{17}}{5}.
	   \end{eqnarray*}}
   \end{bt}
   \begin{bt}%[0-De03-GK1-NH22-23--TeamTeXHoa--Thầy Hải Toán]%[0D1T3-3]
	Lớp $10A$ có tất cả $40$ học sinh trong đó có $13$ học sinh chỉ thích đá bóng, $18$ học sinh chỉ thích chơi cầu lông và số học sinh còn lại thích chơi cả hai môn thể thao nói trên. Hỏi:	
	\begin{enumerate}[a)]
		\item Có bao nhiêu học sinh thích chơi cả hai môn cầu lông và bóng đá?
		\item Có bao nhiêu học sinh thích bóng đá?
		\item Có bao nhiêu học sinh thích cầu lông?
	\end{enumerate}
	\loigiai{
		\begin{enumerate}[a)]
			\item Số học sinh thích chơi cả hai môn cầu lông và bóng đá: $40-\left(18+13\right)=9$ (học sinh)
			\item Số học sinh thích bóng đá: $13+9=22$ (học sinh) 
			\item Số học sinh thích cầu lông: $18+9=27$ (học sinh)
		\end{enumerate}
	}
\end{bt}
%%==========Bài 3
\begin{bt}%[Pj17-0-CK2-NH22-23--TeamTeXHoa--VoVanTu]%[0T2T2-3]
	Có ba nhóm máy A, B,C dùng để sản xuất ra hai loại sản phẩm I và II. Để sản xuất một đơn vị sản phẩm mỗi loại phải lần lượt dùng các máy thuộc các nhóm khác nhau. Số máy trong một nhóm và số máy của từng nhóm cần thiết để sản xuất ra một đơn vị sản phẩm thuộc mỗi loại được cho trong bảng sau
	\begin{center}
		\begin{tikzpicture}[>=stealth,line join=round,line cap=round,scale=1]
			\draw (0.5,-.5)node{Nhóm} (4,-.25)node{Số máy} (10,0.1)node{Số máy trong từng nhóm để sản xuất} (4,-.75)node{trong mỗi nhóm} (10,-0.4)node{ra một đơn vị sản phẩm}
			(8,-1.25)node{Loại I} (12,-1.25)node{Loại II} 
			(0.5,-2)node{$A$} (0.5,-3)node{$B$} (0.5,-4)node{$C$}
			(4,-2)node{$10$} (4,-3)node{$4$} (4,-4)node{$12$}
			(8,-2)node{$2$} (8,-3)node{$0$} (8,-4)node{$2$}
			(12,-2)node{$2$} (12,-3)node{$2$} (12,-4)node{$4$};
			\draw (-1,.5)--(14,0.5)--(14,-4.5)--(-1,-4.5)--cycle (2,0.5)--(2,-4.5) (6,.5)--(6,-4.5) (10,-.75)--(10,-4.5) (-1,-1.5)--(14,-1.5) (6,-.75)--(14,-.75) (-1,-2.5)--(14,-2.5) (-1,-3.5)--(14,-3.5);
		\end{tikzpicture}
	\end{center}
Một đơn vị sản phẩm I lãi ba nghìn đồng, một đơn vị sản phẩm loại II lãi năm nghìn đồng. Tìm số sản phẩm mỗi loại để sản xuất đạt lãi cao nhất.
	\loigiai{
		\immini
		{
			Gọi $x$; $y$ lần lượt là số sản phẩm loại I và loại II cần sản xuất. Điều kiện $x,y\ge 0$.\\
			Số máy nhóm A cần dùng là $2x+2y$.\\
			SỐ máy nhóm B cần dùng là $2y$.\\
			Số máy nhóm C cần dùng là $2x+4y$.\\
			Ta có hệ bất phương trình $$\heva{&x\ge 0\\&y\ge 0\\&2x+2y\le 10\\&2y\le 4\\&x+2y\le 6}\Leftrightarrow\heva{&x\ge 0\\&0\le x\le 2\\&x+y\le 5\\&x+2y\le 6.}$$
		}
		{
			\begin{tikzpicture}[
			line join = round, line cap = round >=stealth, thick, font=\footnotesize,
				declare function={xmin=-1; xmax=6.5; ymin=-1; ymax=6.5;}
				]
				\clip (xmin,ymin) rectangle (xmax,ymax);
				%BPT by+c<=0, b>0
				\begin{scope}[declare function={b=1; c=-2;m=-c/b;}]
					\draw[blue] (xmin,m)--(xmax,m) node[below, pos=.95]{$ (d_1) $};
					\fill[pattern=north east lines, opacity=.5] (xmin,m) rectangle (xmax,ymax);
				\end{scope}
				%BPT ax+by+c<=0, a>0, b>0
				\begin{scope}[declare function={a=1; b=1; c=-5; m=(-b/a)*ymin-(c/a); n=(-b/a)*ymax-(c/a);}]
					\draw[blue] (n,ymax)--(m,ymin) node[below, pos=.5, sloped]{$ (d_2) $};
					\fill[pattern=north east lines, opacity=.5] (n,ymax)--(m,ymin)--(xmax,ymin)--(xmax,ymax)--cycle;
				\end{scope}
				%BPT ax+by+c<=0, a>0, b>0
				\begin{scope}[declare function={a=1; b=2; c=-6; m=(-b/a)*ymin-(c/a); n=(-b/a)*ymax-(c/a);}]
					\draw[blue] (n,ymax)--(m,ymin) node[below, pos=.5, sloped]{$ (d_3) $};
					\fill[pattern=north east lines, opacity=.5] (n,ymax)--(m,ymin)--(xmax,ymin)--(xmax,ymax)--cycle;
				\end{scope}
				%BPT ax+c>=0, a>0
				\begin{scope}[declare function={a=1; c=0;m=-c/a;}]
					\draw[blue] (m,ymin)--(m,ymax) ;
					\fill[pattern=north east lines, opacity=.5] (xmin,ymin) rectangle (m,ymax);
				\end{scope}
				%BPT by+c>=0, b>0
				\begin{scope}[declare function={b=1; c=0;m=-c/b;}]
					\draw[blue] (xmin,m)--(xmax,m) ;
					\fill[pattern=north east lines, opacity=.5] (xmin,ymin) rectangle (xmax,m);
				\end{scope}
				%Vẽ hệ trục
				\draw[->] (xmin,0)--(0,0) node[below right]{$O$}--(xmax,0) node[above left]{$x$};
				\draw[->] (0,ymin)--(0,ymax) node[below right]{$y$};
				%Vẽ các điểm trên trục Ox
				\foreach \x/\g in {-1/-90,1/-90,2/-90,3/-90,4/-90,5/-90,6/-90}
				\draw[thin] (\x,2pt)--(\x,-2pt) + (\g:3mm) node{$\x$};
				%Vẽ các điểm trên trục Oy
				\foreach \y/\g in {1/180,2/180,3/180,4/180,5/180,6/180}
				\draw[thin] (2pt,\y)--(-2pt,\y) + (\g:3mm) node{$\y$};
				\path 
				(0,2)coordinate(A) (2,2)coordinate(B) (4,1)coordinate(C) (5,0)coordinate(D)
				;
				\foreach \x/\g in {A/-45,B/-90,C/-90,D/150}\fill[black] (\x) circle (1pt)+(\g:.3)node{$\x$};
			\end{tikzpicture}
		}
	\noindent
	Vẽ các đường thẳng $d_1\colon y=2$, $d_2\colon x+y=5$, $d_3\colon x+2y=6$.\\
	Ta có miền nghiệm của hệ bất phương trình là ngũ giác $ABCDE$ với $A(0;2)$, $B(2;2)$, $C(4;1)$, $D(5;0)$ và $E\equiv O(0;0)$.\\
	Lãi suất thu được là $f(x;y)=3x+5y$ (nghìn đồng).
	\begin{center}
	\begin{tabular}{|c|c|c|c|c|c|}
		\hline
	$M(x;y)$	& $A$ & $B$ & $C$ & $D$ & $E$ \\
		\hline
	$f(x;y)$	& $10$ & $16$ & $17$ & $15$ & $0$ \\
		\hline
	\end{tabular}
	\end{center}
Do đó, $f(x;y)$ đạt giá trị lớn nhất tại $C(4;1)$.\\
Vậy sản xuất $4$ sản phẩm lại I và $1$ sản phẩm loại II sẽ cho lãi cao nhất.
	}
\end{bt}
%%==========Bài 4
\begin{bt}%[Pj17-0-CK2-NH22-23--TeamTeXHoa--VoVanTu]%[0T4T3-1]
	Hai chiếc tàu thủy cùng xuất phát từ vị trí $A$, đi thẳng theo hai hướng tạo với nhau một góc $60^\circ$. Tàu thứ nhất chạy với tốc độ $20$ km/h, tàu thứ hai chạy với tốc độ $30$ km/h. Hỏi sau $3$ giờ hai tàu cách nhau bao nhiêu km?
	\loigiai{
		\immini
		{
			Ta có quảng đường tàu thứ nhất đi được là $s_1=v_1t=20\cdot 3=60$ (km).\\
			Quảng đường tàu thứ hai đi được là $s_2=v_2t=30\cdot 3=90$ (km).\\
			Áp dụng định lý $\cos$ vào tam giác $ABC$ với $B$, $C$ lần lượt là vị trí của tàu thứ nhất và tàu thứ hai sau ba giờ khởi hành. Tức là $AB=60$ km và $AC=90$ km.
		}
		{
			\begin{tikzpicture}[scale=1, font=\footnotesize, line join=round, line cap=round, >=stealth]
				\path
				(0,0) coordinate (A)++(0:3)coordinate(C)
				(A)++(60:2)coordinate(B)
				;
				\draw (A)--(B)--(C)--cycle;
				\foreach \x/\g in {A/180,B/90,C/0}\fill[black] (\x) circle (1pt)+(\g:.3)node{$\x$};
				\draw pic[draw, angle radius = 12pt, "$60^\circ$", angle eccentricity = 1.5]{ angle = C--A--B};
			\end{tikzpicture}
		}
	\noindent
		$$BC^2=AB^2+AC^2-2AB\cdot AC\cdot\cos 60^\circ=60^2+90^2-2\cdot 60\cdot 90\cos 60^\circ=6300.$$
	Vây, khoảng cách của hai tàu sau $3$ giờ chạy là $BC=\sqrt{6300}=30\sqrt{7}$ (km).
	}
\end{bt}
\Closesolutionfile{ans}
% \inputans{10}{ans/ans-0-GK1-CTST-De9-NH23-24}

