\section*{ÔN TẬP KIỂM TRA GIỮA KÌ 1 - ĐỀ 03}
\setcounter{ex}{0}\setcounter{bt}{0}
\noindent{\bf\fontfamily{qag}\selectfont\color{violet}A. PHẦN TRẮC NGHIỆM}
\Opensolutionfile{ans}[ans/ans-0-GK1-CanhDieu-De3-NH23-24]
\begin{ex}%[Dự Án 6 -Đề GHK1 - Khối 10]%[Lại Thị Hảo]%[0C1Y1-1]
Trong các phát biểu sau đây, phát biểu nào là một mệnh đề toán học?
\choice
{Hình chữ nhật là hình bình hành phải không?}
{\True Số $1$ là số nguyên tố}
{Tam giác cân có một góc $60^{\circ}$ có là tam giác đều không?}
{Học, học nữa, học mãi}
\loigiai{
\lq\lq Số $1$ là số nguyên tố\rq\rq\, là một mệnh đề toán học.
}
\end{ex}
\begin{ex}%[Dự Án 6 -Đề GHK1 - Khối 10]%[Lại Thị Hảo]%[0C1Y1-3]
Trong các phát biểu sau đây, phát biểu nào không phải là mệnh đề?
\choice
{\True Túi của bạn xinh quá!}
{$n(n+1)(n+2)$ chia hết cho $6$ với mọi số tự nhiên $n$}
{$x^2-2 x+1=0$ có nghiệm là $x=2$}
{$\pi$ là số vô tỉ}
\loigiai{
Câu \lq\lq Túi của bạn xinh quá!\rq\rq\, không phải là mệnh đề.
}
\end{ex}
\begin{ex}%[Dự Án 6 -Đề GHK1 - Khối 10]%[Lại Thị Hảo]%[0C1Y1-3]
Cho mệnh đề $A$: \lq\lq Buôn Mê Thuột là một thành phố thuộc tỉnh Đăk Lăk\rq\rq. Mệnh đề phủ định của mệnh đề $A$ là
\choice
{\True $\overline{A}$: \lq\lq Buôn Mê Thuột không là một thành phố thuộc tỉnh Đăk Lăk\rq\rq}
{$\overline{A}$: \lq\lq Buôn Mê Thuột là một thành phố của nước Việt Nam\rq\rq}
{$\overline{A}$: \lq\lq Buôn Mê Thuột không là một thành phố của nước Việt Nam\rq\rq}
{$\overline{A}$: \lq\lq Buôn Mê Thuột là một thành phố thuộc Tây Nguyên\rq\rq}
\loigiai{
	Phủ định của mệnh đề $A$ là $\overline{A}$: \lq\lq Buôn Mê Thuột không là một thành phố thuộc tỉnh Đăk Lăk\rq\rq.}
\end{ex}
\begin{ex}%[Dự Án 6 -Đề GHK1 - Khối 10]%[Lại Thị Hảo]%[0C1Y1-2]
 Mệnh đề toán học nào sau đây là một mệnh đề đúng?
\choice
{Nếu $a\, \vdots \, 3$ thì $a \, \vdots \, 9$}
{Nếu $a+b$ chia hết cho $5$ thì $a$ và $b$ cùng chia hết cho $5$}
{\True Nếu $1=2$ thì $2022=2023$}
{Nếu một số tự nhiên chia hết cho $5$ thì số đó có chữ số tận cùng là $0$}
\loigiai{
Mệnh đề $P \Rightarrow Q$ chỉ sai khi $P$ đúng, $Q$ sai.\\
Ta thấy \lq\lq$1=2$\rq\rq\, là mệnh đề sai nên \lq\lq Nếu $1=2$ thì $2022=2023$\rq\rq\, là mệnh đề đúng.
}
\end{ex}
\begin{ex}%[Dự Án 6 -Đề GHK1 - Khối 10]%[Lại Thị Hảo]%[0C1B2-1] 
Hãy liệt kê các phần tử của tập hợp $A=\{n \in \mathbb{N} \mid n=2 k+1, k \in \mathbb{Z}, 0 \leq k \leq 4\}$.
\choice
{$A=\{0 ; 1 ; 2 ; 3 ; 4\}$}
{$A=\{3 ; 5 ; 7\}$}
{\True $A=\{1 ; 3 ; 5 ; 7 ; 9\}$}
{$A=\{1 ; 3 ; 5 ; 6 ; 9\}$}
\loigiai{
Với $k=0 \Rightarrow n=2 \cdot 0+1=1 \in \mathbb{N}$ (nhận).\\
Với $k=1 \Rightarrow n=2\cdot 1+1=3 \in \mathbb{N}$ (nhận).\\
Với $k=2 \Rightarrow n=2\cdot 2+1=5 \in \mathbb{N}$ (nhận).\\
Với $k=3 \Rightarrow n=2\cdot 3+1=7 \in \mathbb{N}$ (nhận).\\
Với $k=4 \Rightarrow n=2\cdot 4+1=9 \in \mathbb{N}$ (nhận).\\
Do đó $A=\{1 ; 3 ; 5 ; 7 ; 9\}$.
}
\end{ex}
\begin{ex}%[Dự Án 6 -Đề GHK1 - Khối 10]%[Lại Thị Hảo]%[0C1Y2-1] 
Tìm số phần tử của tập hợp $S=\left\{x \in \mathbb{Z} \mid\left(3 x^2-4 x+1\right)\left(x^2-2\right)=0\right\}$.
\choice
{\True $1$}
{$2$}
{$3$}
{$4$}
\loigiai{
Ta có $\left(3 x^2-4 x+1\right)\left(x^2-2\right)=0 \Leftrightarrow \hoac{&x=1 \\ &x=\dfrac{1}{3} \\ &x= \pm \sqrt{2}.}$\\
 Mà $x \in \mathbb{Z}$ nên $S=\{1\}$. Do đó số phần tử của $S$ bằng $1$.
}
\end{ex}
\begin{ex}%[Dự Án 6 -Đề GHK1 - Khối 10]%[Lại Thị Hảo]%[0C2Y1-1]
Bất phương trình nào sau đây không phải là bất phương trình bậc nhất hai ẩn?
\choice
{$x-5 y-1 \geq 0$}
{$2 x-3 y+5<0$}
{$\dfrac{x}{2}-\dfrac{y}{3}+10<0$}
{\True $x+3 y^2-2 x+1 \leq 0$}
\loigiai{
Ta thấy $x+3 y^2-2 x+1 \leq 0$ không là bất phương trình bậc nhất hai ẩn vì chứa $y^2$.
}
\end{ex}
\begin{ex}%[Dự Án 6 -Đề GHK1 - Khối 10]%[Lại Thị Hảo]%[0C2Y1-2]
Miền nghiệm của bất phương trình $5(x+1)+2(y+1)>x+y+9$ là nửa mặt phẳng chứa điểm
\choice
{$(0 ; 0)$}
{$(-1 ; 1)$}
{$(1 ;-2)$}
{\True $(2 ; 1)$}
\loigiai{
Ta có $5(x+1)+2(y+1)>x+y+9 \Leftrightarrow 4 x+y>2$.\\
Vì $4 \cdot 2+1>2$ là mệnh đề đúng nên miền nghiệm của bất phương trình trên chứa điểm có tọa độ $(2;1)$.
}
\end{ex}
\begin{ex}%[Dự Án 6 -Đề GHK1 - Khối 10]%[Lại Thị Hảo]%[0C2Y2-1]
Hệ bất phương trình nào sau đây không là hệ bất phương trình bậc nhất hai ẩn?
\choice
{$\heva{&x-y>0 \\
	&x-3 y+3<0 \\
	&x+y-5>0}$}
{$\heva{&2 x-1 \leq 0 \\ &-3 x+5 \leq 0}$}
{\True $\heva{&3-y<0 \\ &2 x^2-3 y+1>0}$}
{$\heva{&x-2 y=0 \\ &x+3 y=-2}$}
\loigiai{
Ta thấy hệ $\heva{&3-y<0 \\ &2 x^2-3 y+1>0}$ chứa $x^2$ nên không là hệ bất phương trình bậc nhất hai ẩn.
}
\end{ex}
\begin{ex}%[Dự Án 6 -Đề GHK1 - Khối 10]%[Lại Thị Hảo]%[0C4Y1-2] 
 Giá trị của biểu thức $\sin \left(90^{\circ}\right)+\cos \left(60^{\circ}\right)$ là
\choice
{$-\dfrac{1}{2}$}
{$1$}
{$\dfrac{1}{2}$}
{\True $\dfrac{3}{2}$}
\loigiai{
Ta có $\sin \left(90^{\circ}\right) + \cos\left(60^{\circ}\right) =1+\dfrac{1}{2}=\dfrac{3}{2}$. 
}
\end{ex}
\begin{ex}%[Dự Án 6 -Đề GHK1 - Khối 10]%[Lại Thị Hảo]%[0C4Y2-1]
Cho $\triangle ABC$ có $AB=13$, $AC=8$, $BC=7$. Tính góc $ACB$.
\choice
{$30^{\circ}$}
{\True $90^{\circ}$}
{$60^{\circ}$}
{$120^{\circ}$}
\loigiai{
Ta có $\cos \widehat{ACB}=\dfrac{AC^2+BC^2-AB^2}{2 A C \cdot BC}=\dfrac{8^2+7^2-13^2}{2 \cdot 8\cdot 7}=-\dfrac{1}{2}$.\\
Suy ra $\widehat{ACB}=120^{\circ}$.
}
\end{ex}
\begin{ex}%[Dự Án 6 -Đề GHK1 - Khối 10]%[Lại Thị Hảo]%[0C4Y2-1]
Cho $\triangle A B C$ có $AB=6$, $BC=8$, $ABC=60^{\circ}$. Tính độ dài cạnh $AC$.
\choice
{$\sqrt{13}$}
{\True $2 \sqrt{13}$}
{$\sqrt{6}$}
{$2 \sqrt{6}$}
\loigiai{Áp dụng định lý côsin, ta có
	$$AC^2=AB^2+BC^2-2AB \cdot BC \cdot \cos \widehat{ABC}=6^2+8^2-2 \cdot 6 \cdot 8 \cdot \cos 60^{\circ}=52.$$
	Suy ra $A C=\sqrt{52}=2 \sqrt{13}$.}
\end{ex}
\begin{ex}%[Phan Anh]%[0D2B1-2]
	Tìm tập xác định $\mathscr{D}$ của hàm số $y=\dfrac{2x+1}{x^3-3x+2}$.
	\choice
	{$\mathscr{D}=\mathbb{R}\setminus\left\{1;2\right\}$}
	{\True $\mathscr{D}=\mathbb{R}\setminus\left\{-2;1\right\}$}
	{$\mathscr{D}=\mathbb{R}\setminus\left\{-2\right\}$}
	{$\mathscr{D}=\mathbb{R}$}
	\loigiai{
		Hàm số xác định khi $\begin{aligned}[t]
		&x^3-3x+2\ne 0\Leftrightarrow (x-1)(x^2+x-2)\ne 0\\
		\Leftrightarrow&\,\heva{
			& x-1\ne 0 \\ 
			& x^2+x-2\ne 0}\Leftrightarrow \heva{
			& x\ne 1 \\ 
			& \heva{
				& x\ne 1 \\ 
				& x\ne-2}}\Leftrightarrow \heva{
			& x\ne 1 \\ 
			& x\ne-2.}
		\end{aligned}$\\
		Vậy tập xác định của hàm số là $\mathscr{D}=\mathbb{R}\setminus\left\{-2;1\right\}$.}
\end{ex}
\begin{ex}%[Phan Anh]%[0D2B1-3]
	Xét sự biến thiên của hàm số $f(x)=\dfrac{3}{x}$ trên khoảng $(0;+\infty)$. Khẳng định nào sau đây đúng?
	\choice
	{Hàm số đồng biến trên khoảng $\left(0;+\infty \right)$}
	{\True Hàm số nghịch biến trên khoảng $\left(0;+\infty \right)$}
	{Hàm số vừa đồng biến, vừa nghịch biến trên khoảng $\left(0;+\infty \right)$}
	{Hàm số không đồng biến, cũng không nghịch biến trên khoảng $\left(0;+\infty \right)$}
	\loigiai{
		Ta có $f\left(x_1\right)-f\left(x_2\right)=\dfrac{3}{x_1}-\dfrac{3}{x_2}=\dfrac{3\left(x_2-x_1\right)}{x_1x_2}=-\dfrac{3\left(x_1-x_2\right)}{x_1x_2}.$\\
		Với mọi $x_1, x_2\in \left(0;+\infty \right)$ và $x_1<x_2$. Ta có $\heva{
			& x_1>0 \\ 
			& x_2>0 \\}\Rightarrow x_1\cdot x_2>0$.\\
		Suy ra $\dfrac{f\left(x_1\right)-f\left(x_2\right)}{x_1-x_2}=-\dfrac{3}{x_1x_2}<0\Rightarrow f(x)$ nghịch biến trên $\left(0;+\infty \right)$.}
\end{ex}
\begin{ex}%[Phan Anh]%[0D2K1-3]
	\immini{Cho hàm số $y=f(x)$ có tập xác định là $\left[-3;3\right]$ và đồ thị của nó được biểu diễn bởi hình bên. Khẳng định nào sau đây là đúng?
		\choice
		{\True Hàm số đồng biến trên khoảng $\left(-3;-1\right)$ và $\left(1;3\right)$}
		{Hàm số đồng biến trên khoảng $\left(-3;-1\right)$và $\left(1;4\right)$}
		{Hàm số đồng biến trên khoảng $\left(-3;3\right)$}
		{Hàm số nghịch biến trên khoảng $\left(-1;0\right)$}}
	{\begin{tikzpicture}[>=stealth,scale=0.7]
		\draw[->](-4,0)--(4,0)node[above]{$x$};
		\draw[->](0,-2)--(0,5)node[right]{$y$};
		\draw (-3,-1)--(-1,1)--(0,1)node[above left]{$1$}--(3,4);
		\draw[dashed](-3,0)node[above]{$-3$}--(-3,-1)--(0,-1)node[right]{$-1$};
		\draw[dashed](-1,0)node[below]{$-1$}--(-1,1);
		\draw[dashed](3,0)node[below]{$3$}--(3,4)--(0,4)node[left]{$4$};
		\fill (-3,0)circle(1.2pt) (-3,-1)circle(1.2pt) (0,-1)circle(1.2pt) (-1,0)circle(1.2pt) (-1,1)circle(1.2pt) (0,1)circle(1.2pt) (3,0)circle(1.2pt) (3,4)circle(1.2pt) (0,4)circle(1.2pt) (0,0)node[above right]{$O$}circle(1.2pt);
		\end{tikzpicture}}	
	\loigiai{
		Trên khoảng $\left(-3;-1\right)$ và $\left(1;3\right)$ đồ thị hàm số đi lên từ trái sang phải\\
		$\Rightarrow $ Hàm số đồng biến trên khoảng $\left(-3;-1\right)$ và $\left(1;3\right).$}
\end{ex}
\begin{ex}%[Phan Anh]%[0D2K1-2]
	Tìm tất cả các giá trị thực của tham số $m$ để hàm số $y=\dfrac{x+2m+2}{x-m}$ xác định trên $\left(-1;0\right)$.
	\choice
	{$\hoac{
			& m>0 \\ 
			& m<-1}$}
	{$m\le-1$}
	{\True $\hoac{
			& m\ge 0 \\ 
			& m\le-1}$}
	{$m\ge 0$}
	\loigiai{
		Hàm số xác định khi $x-m\ne 0\Leftrightarrow x\ne m$.
		Tập xác định của hàm số là $\mathscr{D}=\mathbb{R}\setminus\left\{m\right\}$.\\
		Hàm số xác định trên $\left(-1;0\right)$ khi và chỉ khi $m\notin \left(-1;0\right)\Leftrightarrow \hoac{
			& m\ge 0 \\ 
			& m\le-1.}$}
\end{ex}
\begin{ex}%[Dự Án 6 -Đề GHK1 - Khối 10]%[Lại Thị Hảo]%[0C1B2-2] 
Cho tập hợp $B=\{1 ; 2 ; 3 ; 2023\}$. Có bao nhiêu tập hợp $A$ thỏa mãn $\{1\} \subset A \subset B$?
\choice
{$4$}
{\True $8$}
{$32$}
{$16$}
\loigiai{
Có $8$ tập hợp $A$ thỏa mãn $\{1\} \subset A \subset B$, gồm:
$$\{1\};\quad \{1 ; 2\}; \quad\{1 ; 3\}; \quad \{1 ; 2023\};\quad \{1 ; 2 ; 3\};\quad \{1 ; 2 ; 2023\};\quad \{1 ; 3 ; 2023\};\quad \{1 ; 2 ; 3 ; 2023\}.
$$
}
\end{ex}
\begin{ex}%[Dự Án 6 -Đề GHK1 - Khối 10]%[Lại Thị Hảo]%[0C1K2-5] 
 Một quyển sách có 150 trang được đánh số từ 1 đến 150. Tính số các chữ số dùng để đánh số trang của quyển sách này.
\choice
{$343$}
{\True $342$}
{$344$}
{$340$}
\loigiai{
Ta thấy
\begin{itemize}
	\item Từ trang $1$ đến trang $9$ có $9$ trang gồm một chữ số.
	\item Từ trang $10$ đến trang $99$ có $90$ trang có hai chữ số.
	\item Từ trang $100$ đến trang $150$ có $51$ trang có ba chữ số.
\end{itemize}
Do đó số các chữ số dùng để đánh số trang của quyển sách này là
$$9+90 \cdot 2+51 \cdot 3=342 \, \text{(số)}.$$
}
\end{ex}
\begin{ex}%[Dự Án 6 -Đề GHK1 - Khối 10]%[Lại Thị Hảo]%[0C2G2-2]
Quảng cáo sản phẩm trên truyền hình là một hoạt động quan trọng trong kinh doanh của các doanh nghiệp. Theo Thông báo số 10/2019, giá quảng cáo trên VTV1 là $30$ triệu đồng cho $15$ giây/$1$ lần quảng cáo vào khoảng $20$h$30$; là $6$ triệu đồng cho $15$ giây/$1$ lần quảng cáo vào khung giờ $16$h$00$-$17$h$00$.\\
Một công ty dự định chi không quá $600$ triệu đồng để quảng cáo trên VTV1 sau: ít nhất $10$ lần quảng cáo vào khoảng $20$h$30$ và không quá $40$ lần quảng cáo vào khung giờ $16$h$00$-$17$h$00$. Gọi $x$, $y$ lần lượt là số lần phát quảng cáo vào khoảng $20$h$30$ và vào khung giờ $16$h$00$-$17$h$00$. Tìm $x$ và $y$ sao cho tổng số lần xuất hiện quảng cáo của công ty là nhiều nhất.
	\choice
	{\True $x=12$; $y=40$}
	{$x=20$; $y=40$}
	{$x=10$; $y=40$}
	{$x=20$; $y=30$}
	\loigiai{
Gọi $x$, $y$ lần lượt là số lần phát quảng cáo vào khoảng $20$h$30$ và vào khung giờ $16$h$00$-$17$h$00$. Theo giả thiết, ta có: $x \in \mathbb{N}$, $y \in \mathbb{N}$, $x \geq 10$, $0 \leq y \leq 40$.\\
Tổng số lần phát quảng cáo là $T=x+y$.\\
Số tiền công ty cần chỉ là $30 x+6 y$ (triệu đồng).\\
Do công ty dự định chi không quá $600$ triệu đồng nên $30 x+6 y \leq 600$ hay $5 x+y \leq 100$.\\
Ta có hệ bất phương trình:
$$
\left\{\begin{array}{l}
	5 x+y \leq 100 \\
	x \geq 10 \\
	0 \leq y \leq 40
\end{array}\right.
$$
Bài toán đưa về tìm $x$, $y$ là nghiệm của hệ bất phương trình (1) sao cho $T=x+y$ có giá trị lớn nhất. Trước hết, ta xác định miền nghiệm của hệ bất phương trình (1).
\begin{center}
	\begin{tikzpicture}[>=stealth,line join=round,line cap=round,font=\footnotesize,scale=1.2,every node/.style={scale=0.9},declare function={
	a=5;b=1;c=-100;
	f(\x)=-c/b-a/b*(\x);
	}]
	\begin{scope}
	\clip (-3,-2) rectangle (5,6);
	\draw[dotted] (-3,-2) rectangle (5,6);
	\fill[pattern=north west lines,pattern color=red] (-0.4,12)--(2.4,-2)--(5,-2)--(5,12)--cycle;
	\fill[pattern=vertical lines,pattern color=gray] (-4,-3) rectangle (7,0);
	\fill[pattern=vertical lines,pattern color=gray] (-4,4) rectangle (7,7); 
	\fill[pattern=horizontal lines,pattern color=blue] (-3,-2) rectangle (1,7);
	\draw[line width=1pt] (-0.4,12)--(2.4,-2) (1,-2)--(1,6) (-3,4)--(5,4);
	\end{scope}
	\draw[->] (-3,0)--(5,0)node[below right]{$x$};
	\draw[->] (0,-2)--(0,6)node[above left]{$y$};
	\fill (0,0)circle(1.5pt)node[above left]{$O$}
	(1,0) circle(1.5pt)node[above left]{$10$} 
	(2,0) circle(1.5pt) node[above right]{$20$}
	(0,4)circle(1.5pt)node[below left]{$40$}
	(1,0) circle(1.5pt)node[above right]{$D$} 
	(2,0) circle(1.5pt) node[above left]{$A$}
	(1.2,4) circle(1.5pt)node[below right]{$B$} (1,4) circle(1.5pt)node[below left]{$C$} 
	;
	\end{tikzpicture}
\end{center}
Miền nghiệm của hệ bất phương trình (1) là miền tứ giác $ABCD$ với $A(20 ; 0)$, $B(12 ; 40), C(10 ; 40), D(10 ; 0)$.\\
Biểu thức $T=x+y$ đạt được giá trị lớn nhất tại một trong các đỉnh của tứ giác $ABCD$.\\
Ta có $\heva{&T(20;0)=20\\ &T(12;40)=52\\&T(10;40)=40\\&T(10;0)=0.}$\\
 Ta thấy $T$ đạt giá trị lớn nhất khi $x=12$, $y=40$ ứng với tọa độ đỉnh $B$.
	}
\end{ex}
\begin{ex}%[Dự Án 6 -Đề GHK1 - Khối 10]%[Lại Thị Hảo]%[0C4B2-1] 
	Hai chiếc tàu thuỷ cùng xuất phát từ vị trí $A$, đi thẳng theo hai hướng tạo với nhau một góc $60^{\circ}$. Tàu thứ nhất chạy với tốc độ $35$ km/h, tàu thứ hai chạy với tốc độ $40$ km/h. Hỏi sau $2$ giờ hai tàu cách nhau bao nhiêu km, bỏ qua vận tốc dòng nước?
	\choice
	{$20$}
	{$5 \sqrt{57}$}
	{\True $10 \sqrt{57}$}
	{$15$}
	\loigiai{
\immini{	
Sau $2$h quãng đường tàu thứ nhất chạy được là $S_{1}=35 \cdot 2=70$ km.\\
Sau $2$h quãng đường tàu thứ hai chạy được là $S_{2}=40 \cdot 2=80$ km.\\
Vậy, sau $2$h hai tàu cách nhau là
 $$S=\sqrt{S_{1}^{2}+S_{2}^{2}-2 S_{1} \cdot S_{2} \cdot \cos 60^{\circ}}=10 \sqrt{57}.$$
	}{
	\begin{tikzpicture}[>=stealth,line join=round,line cap=round,font=\footnotesize,scale=.8]
	\draw	(0,0) coordinate (A)node[below left]{$A$}
	(4,0) coordinate (B)node[below right]{$B$}
	(60:5) coordinate (C)node[above left]{$C$}
	;
	\draw[->] (A)--(B)node[pos=.5,sloped,below]{$35$ km/h} ;
	\draw[->] (A)--(C) node[pos=.5,sloped,above]{$40$ km/h};
	\draw pic["$60^\circ$", angle radius=8mm]{angle=B--A--C};
\end{tikzpicture}}
}
\end{ex}
\begin{ex}%[Dự Án 6 -Đề GHK1 - Khối 10]%[Lại Thị Hảo]%[0C1B2-6]
Cho hai tập hợp $A=\{x \in \mathbb{R} \, \mid \, x-1>0\}$ và $B=\{x \in \mathbb{R} \, \mid \, x-2\,022 \leq 0\}$. Khi đó $A \cup B$ là
\choice
{$(1 ; 2\,022]$}
{$(1 ;+\infty)$}
{\True $\mathbb{R}$}
{$[2\,022 ;+\infty)$}
\loigiai{
Ta có $A=\{x \in \mathbb{R}\, \mid \, x-1>0\}=(1 ;+\infty)$ và $B=\{x \in \mathbb{R} \, \mid \, x-2\,022 \leq 0\}=(-\infty ; 2\,022]$.\\
Vậy $A \cup B=\mathbb{R}$.
}
\end{ex}
\begin{ex}%[Dự Án 6 -Đề GHK1 - Khối 10]%[Lại Thị Hảo]%[0C1B2-7]
Cho hai tập hợp $A=\left\{x \in \mathbb{N} \mid(x-2)\left(x^{2}+3 x-4\right)=0\right\}$ và $B=\{x \in \mathbb{Z}\, \mid \,| x-3 \mid \leq 1\}$. Tập $A \setminus B$ có tất cả bao nhiêu phần tử?
\choice
{$2$}
{$3$}
{\True $1$}
{$4$}
\loigiai{
Ta có $A=\left\{x \in \mathbb{N}\, \mid \,(x-2)\left(x^{2}+3 x-4\right)=0\right\}=\{1 ; 2\}$ và $B=\{x \in \mathbb{Z}\, \mid \, |x-3 \mid \leq 1\}=\{2 ; 3 ; 4\}$.\\
Suy ra $A \setminus B=\{1\}$. Vậy $A \setminus B$ có $1$ phần tử.
}
\end{ex}
\begin{ex}%[Dự Án 6 -Đề GHK1 - Khối 10]%[Lại Thị Hảo]%[0C1K2-5]
Trường THPT Xuân Vân có hai câu lạc bộ thể thao là câu lạc bộ bóng đá và câu lạc bộ bóng chuyền. Để duy trì và phát triển các hoạt động trong năm học mới, hai câu lạc bộ này thông báo tuyển thành viên. Lớp 10A1 có $39$ học sinh đăng ký tham gia các câu lạc bộ thể thao của trường, trong đó có $25$ học sinh đăng ký tham gia câu lạc bộ bóng đá, có $22$ học sinh đăng ký tham gia câu lạc bộ bóng chuyền. Hỏi có bao nhiêu bạn học sinh lớp 10A1 vừa tham gia cả hai câu lạc bộ bóng đá và bóng chuyền?
\choice
{$39$}
{$47$}
{$3$}
{\True $8$}
\loigiai{
Gọi $A$ là tập hợp các bạn học sinh lớp 10A1 tham gia câu lạc bộ bóng đá.\\
$B$ là tập hợp các bạn học sinh lớp 10A1 tham gia câu lạc bộ bóng chuyền.\\
Ta có sơ đồ Ven:
\begin{center}
\begin{tikzpicture}
	\def\r{2}\def\d{1.5}
	\def\firstC{(210:\d) circle(\r)}
	\def\secondC{(-30:\d) circle(\r)}
	\draw \firstC \secondC ;
	\path
	(150:\d) node{$A$} 
	(35:\d) node{$B$} 
	(90:0.2) node{$C$} 
	(90:0.2)+(0,-.7) node{?}
	(210:1.5*\d) node{Bóng đá}
	(210:1.5*\d)+(0,-.7) node{($25$ HS)}
	(-30:1.5*\d) node{Bóng chuyền}
	(-30:1.5*\d)+(0,-.7) node{($22$ HS)};
	\draw (-5,-3) rectangle (5,3)
	(0,2) node{$39$ HS lớp 10A1};
\end{tikzpicture}
\end{center}
Theo sơ đồ Ven, số học sinh chỉ tham gia câu lạc bộ bóng chuyền là $39-25 =14$ (học sinh).\\
Vậy số học sinh tham gia cả hai câu lạc bộ là $22-14=8$ (học sinh).
}
\end{ex}
\begin{ex}%[Dự Án 6 -Đề GHK1 - Khối 10]%[Lại Thị Hảo]%[0C2Y1-2]
Điểm $A(5 ;-3)$ là điểm thuộc miền nghiệm của bất phương trình nào sau đây?
\choice
{\True $5 x-2 y+1 \geq 0$}
{$-3 x+y+2>0$}
{$2 x-3 y \leq 0$}
{$x-2 y<0$}
\loigiai{
 Thay toạ độ điểm $A$ vào bất phương trình $5 x-2 y+1 \geq 0$, ta có $5 \cdot 5-2 \cdot (-3)+1=32>0$ nên điểm $A$ thuộc miền nghiệm của bất phương trình $5 x-2 y+1 \geq 0$.
}
\end{ex}
\begin{ex}%[Dự Án 6 -Đề GHK1 - Khối 10]%[Lại Thị Hảo]%[0C2B2-3]
 Điểm nào sau đây thuộc miền nghiệm của hệ bất phương trình $\heva{&2(x+1)-3 y \geq 4 x+5(y-1) \\ &5 x+3 y<3(1+2 y)}$?
\choice
{$(2 ;-3)$}
{$(0 ;-1)$}
{\True $(-3 ; 1)$}
{$(9 ; 6)$}
\loigiai{
Ta có $\heva{&2(x+1)-3 y \geq 4 x+5(y-1) \\ &5 x+3 y<3(1+2 y)} \Leftrightarrow \heva{&-2 x-8 y \geq-7 \\ &5 x-3 y<3}$ (I).\\
Thay toạ độ các điểm $(2 ;-3)$, $(0 ;-1)$, $(-3 ; 1)$, $(9 ; 6)$ vào hệ bất phương trình (I), ta thấy chỉ có điểm $(-3 ; 1)$ thỏa hệ nên điểm $(-3 ; 1)$ thuộc miền nghiệm của hệ bất phương trình đã cho.
}
\end{ex}
\begin{ex}%[Dự Án 6 -Đề GHK1 - Khối 10]%[Lại Thị Hảo]%[0C2B2-1]
Cho tam giác $MNP$ có $MN=9$, $P=30^{\circ}$, $N=105^{\circ}$. Độ dài cạnh $PN$ bằng
\choice
{$\dfrac{9 \sqrt{6}-9 \sqrt{2}}{2}$}
{$\dfrac{9 \sqrt{2}}{2}$}
{\True $9 \sqrt{2}$}
{$\dfrac{9 \sqrt{6}+9 \sqrt{2}}{2}$}
\loigiai{
Ta có $\widehat{M}=180^{\circ}-30^{\circ}-105^{\circ}=45^{\circ}$.\\
Theo định lí sin trong tam giác $MNP$, ta có
$$\dfrac{M N}{\sin P}=\dfrac{P N}{\sin M} \Rightarrow P N=\dfrac{M N \cdot \sin M}{\sin P}=\dfrac{9 \cdot \sin 45^{\circ}}{\sin 30^{\circ}}=9 \sqrt{2}.$$
}
\end{ex}
\begin{ex}%[Dự Án 6 -Đề GHK1 - Khối 10]%[Lại Thị Hảo]%[0C2B2-1]
Cho tam giác $ABC$ có $AB=2a$; $AC=4a$ và tổng hai góc $B$ và $C$ bằng $60^{\circ}$. Tính diện tích tam giác $ABC$.
\choice
{$S=8 a^{2}$}
{\True $S=2 a^{2} \sqrt{3}$}
{$S=a^{2} \sqrt{3}$}
{$S=4 a^{2}$}
\loigiai{
	Ta có $\widehat{A}=180^{\circ}-(B+C)=120^{\circ}$.\\
	Diện tích của tam giác $ABC$ là
	$$S_{A B C}=\dfrac{1}{2} A B \cdot A C \cdot \sin \widehat{BAC}=\dfrac{1}{2} \cdot 2 a \cdot 4 a \cdot \sin 120^{\circ}=2 a^{2} \sqrt{3}\, (\text{đvdt}).$$
}
\end{ex}

\begin{ex}%[Phan Anh]%[0D2B1-3]
	Cho hàm số $f(x)=4-3x$. Khẳng định nào sau đây đúng?
	\choice
	{Hàm số đồng biến trên $\left(-\infty;\dfrac{4}{3}\right)$}
	{\True Hàm số nghịch biến trên $\left(\dfrac{4}{3};+\infty \right)$}
	{Hàm số đồng biến trên $\mathbb{R}$}
	{Hàm số đồng biến trên $\left(\dfrac{3}{4};+\infty \right)$}
	\loigiai{
		TXĐ: $\mathscr{D}=\mathbb{R}$. \\Với mọi $x_1,x_2\in \mathbb{R}$ và $x_1<x_2$, ta có
		$f\left(x_1\right)-f\left(x_2\right)=\left(4-3x_1\right)-\left(4-3x_2\right)=-3\left(x_1-x_2\right)>0.$\\
		Suy ra $f\left(x_1\right)>f\left(x_2\right)$. Do đó, hàm số nghịch biến trên $\mathbb{R}$.\\
		Mà $\left(\dfrac{4}{3};+\infty \right)\subset \mathbb{R}$ nên hàm số cũng nghịch biến trên $\left(\dfrac{4}{3};+\infty \right)$.}
\end{ex}\begin{ex}%[Phan Anh]%[0D2K1-2]
	Tìm tất cả các giá trị thực của tham số $m$ để hàm số $y=\dfrac{2x+1}{\sqrt{x^2-6x+m-2}}$ xác định trên $\mathbb{R}$.
	\choice
	{$m\ge 11$}
	{\True $m>11$}
	{$m<11$}
	{$m\le 11$}
	\loigiai{
		Hàm số xác định khi $x^2-6x+m-2>0\Leftrightarrow {\left(x-3\right)}^2+m-11>0$.\\
		Hàm số xác định với $\forall x\in \mathbb{R}\Leftrightarrow (x-3)^2+m-11>0$ đúng với mọi $x\in \mathbb{R}$
		$\Leftrightarrow m-11>0\Leftrightarrow m>11$.}
\end{ex}
\begin{ex}%[Phan Anh]%[0D2B1-1]
	Cho hàm số $f(x)=\left\{\begin{array}{*{35}{l}}
	\dfrac{2}{x-1} &, x\in(-\infty;0) \\
	\sqrt{x+1} &, x\in[0;2] \\
	x^2-1 &, x\in(2;5]
	\end{array}\right.$. Tính giá trị của $f(4)$.
	\choice
	{$f(4)=\dfrac{2}{3}$}
	{\True $f(4)=15$}
	{$f(4)=\sqrt{5}$}
	{Không tính được}
	\loigiai{Do $4\in(2;5]$ nên $f(4)=4^2-1=15$.}
\end{ex}
\begin{ex}%[Dự Án 6 -Đề GHK1 - Khối 10]%[Lại Thị Hảo]%[0C1K2-6]
Cho các tập hợp khác rỗng $A=[0 ; 2\,022]$, $B=(2\,022 a ; 2023 a+1]$, $a>-1$. Tìm $a \in \mathbb{R}$ để $A \cap B=\varnothing$.
\choice
{$a \geq-4$}
{$a<-1$}
{\True $\hoac{&a \geq 1 \\ &-1<a<\dfrac{-1}{2023}.}$}
{$a \geq-1$}
\loigiai{
Ta có $A \cap B = \varnothing \Leftrightarrow \heva{& \hoac{&2\,022 a \geq 202 \\ &2023 a+1<0} \\ &a>-1} \Leftrightarrow\ \heva{&\hoac{&a \geq 1 \\ &a<\dfrac{-1}{2023}} \\ &a>-1} \Leftrightarrow \hoac{&a \geq 1 \\ &-1<a<\dfrac{-1}{2023}.}$
}
\end{ex}
\begin{ex}%[Dự Án 6 -Đề GHK1 - Khối 10]%[Lại Thị Hảo]%[0C2B2-2]
Bạn Minh đạt được danh hiệu Học sinh giỏi nên được mẹ thưởng cho $600$ nghìn đồng để mua kem. Minh đến siêu thị dự định mua hai hãng kem Merino và TH. Giá của mỗi chiếc kem Merino là $12$ nghìn đồng một chiếc, giá của một chiếc kem TH là $15$ nghìn đồng. Do tủ lạnh đã chứa nhiều đồ nên không gian ngăn bảo quản chỉ có thể chứa tối đa $30$ chiếc kem. Gọi $x$, $y$ lần lượt là số kem loại Merino và TH mà Minh có thể mua. Hãy lập hệ bất phương trình biểu thị các điều kiện ràng buộc của bài toán theo $x$, $y$.
\choice
{$\heva{&x >0 \\ &y > 0 \\ &x+y < 30 \\ &4 x+5 y < 200}$}
{\True $\heva{&x \geq 0 \\ &y \geq 0 \\ &x+y \leq 30 \\ &4 x+5 y \leq 200}$}
{$\heva{&x >0 \\ &y > 0 \\ &x+y > 30 \\ &4 x+5 y > 200}$}
{$\heva{&x \geq 0 \\ &y \geq 0 \\ &x+y \geq 30 \\ &4 x+5 y \geq 200}$}
\loigiai{
Gọi $x$, $y$ lần lượt là số kem loại Merino và TH mà Minh có thể mua $(x, y \geq 0; x, y \in \mathbb{N})$.\\
Do số kem có thể chứa trong tủ lạnh là 30 chiếc nên ta có bất phương trình $x+y \leq 30$.\\
Số tiền mua kem không vượt quá $600$ nghìn nên ta có bất phương trình
$12 x+15 y \leq 600 \Leftrightarrow 4 x+5 y \leq 200.$
Vậy ta có hệ sau $\heva{&x \geq 0 \\ &y \geq 0 \\ &x+y \leq 30 \\ &4 x+5 y \leq 200.}$
}
\end{ex}
\begin{ex}%[Phan Anh]%[0D2B1-2]
	Tìm tập xác định $\mathscr{D}$ của hàm số $y=\dfrac{3x-1}{2x-2}$.
	\choice
	{$\mathscr{D}=\mathbb{R}$}
	{$\mathscr{D}=(1;+\infty)$}
	{\True $\mathscr{D}=\mathbb{R}\setminus\{1\}$}
	{$\mathscr{D}=[1;+\infty)$}
	\loigiai{
		Hàm số xác định khi $2x-2\ne0\Leftrightarrow x\ne1$.\\
		Vậy tập xác định của hàm số là $\mathscr{D}=\mathbb{R}\setminus\{1\}$.}
\end{ex}
\begin{ex}%[Phan Anh]%[0D2B1-2]
	Tìm tập xác định $\mathscr{D}$ của hàm số $y=\dfrac{x^2+1}{x^2+3x-4}$.
	\choice
	{$\mathscr{D}=\{1;-4\}$}
	{\True $\mathscr{D}=\mathbb{R}\setminus\{1;-4\}$}
	{$\mathscr{D}=\mathbb{R}\setminus\{1;4\}$}
	{$\mathscr{D}=\mathbb{R}$}
	\loigiai{
		Hàm số xác định khi $x^2+3x-4\ne 0\Leftrightarrow \heva{
			& x\ne 1 \\ 
			& x\ne-4}.$\\
		Vậy tập xác định của hàm số là $\mathscr{D}=\mathbb{R}\setminus\{1;-4\}$.}
\end{ex}
\begin{ex}%[Phan Anh]%[0D2B1-1]
	Điểm nào sau đây thuộc đồ thị hàm số $y=\dfrac{1}{x-1}$?
	\choice
	{\True $M_1(2;1)$}
	{$M_2(1;1)$}
	{$M_3(2;0)$}
	{$M_4(0;-2)$}
	\loigiai{
		Xét điểm $M_1$, thay $x=2$ và $y=1$
		vào hàm số $y=\dfrac{1}{x-1}$ ta được $1=\dfrac{1}{2-1}$ ta thấy đúng nên nhận $M_1$.}
\end{ex}
\Closesolutionfile{ans}
\noindent{\bf\fontfamily{qag}\selectfont\color{violet}B. PHẦN TỰ LUẬN}
\begin{bt} 
Tìm tập xác định của hàm số $y=\dfrac{\sqrt{3-2x}}{3x^2-5x+2}$.
\end{bt}
\begin{bt}
	Tam giác $ABC$ có $b= 6$, $c= 8$ và $\widehat{A}=60^\circ$. Tính độ dài cạnh $BC$, trung tuyến $AM$.
	\loigiai{ }
\end{bt}
\begin{bt}%[Dự Án 6 -Đề GHK1 - Khối 10]%[Lại Thị Hảo]%[0C4K1-3] 
Với mọi góc $\alpha$ sao cho $\cos \alpha \neq 0$. Ta luôn có $\dfrac{1}{\cos ^{2} \alpha}=1+\tan ^{2} \alpha$ và $\tan \alpha=\dfrac{\sin \alpha}{\cos \alpha}$.
Hãy tính giá trị biểu thức $B=\dfrac{\sin \alpha-\cos \alpha}{\sin ^{3} \alpha+3 \cos ^{3} \alpha+2 \sin ^{5} \alpha}$, biết tan $\alpha=2$.
\loigiai{Ta có
	\allowdisplaybreaks
\begin{eqnarray*}
	B & =&\dfrac{\sin \alpha-\cos \alpha}{\sin ^{3} \alpha+3 \cos ^{3} \alpha+2 \sin ^{5} \alpha}\\
	&=&\dfrac{\cos \alpha(\tan \alpha-1)}{\cos ^{3} \alpha\left(\tan ^{3} \alpha+3+2 \tan ^{3} \alpha \cdot \dfrac{\tan ^{2} \alpha}{1+\tan ^{2} \alpha}\right)} \\
	& =&\dfrac{\left(1+\tan ^{2} \alpha\right)(\tan \alpha-1)}{\tan ^{3} \alpha+3+2 \tan ^{3} \alpha \cdot \dfrac{\tan ^{2} \alpha}{1+\tan ^{2} \alpha}}\\
	&=&\dfrac{25}{119}.
\end{eqnarray*}
Vậy $B=\dfrac{25}{119}$.
}
\end{bt}
\begin{bt}%[BG10-2022]%[Toanvo]%[0D4G4-4]
	Người ta định dùng hai loại nguyên liệu để chiết xuất ít nhất $120$ kg hóa chất $A$ và $9$ kg hóa chất $B$. Từ mỗi tấn nguyên liệu loại I giá $4$ triệu đồng có thể chiết xuất được $20$ kg chất $A$ và $0{,}6$ kg chất $B$. Từ mỗi tấn nguyên liệu loại II giá $3$ triệu đồng có thể chiết xuất được $10$ kg chất $A$ và $1{,}5$ kg chất $B$. Hỏi phải dùng bao nhiêu tấn nguyên liệu mỗi loại để chi phí mua nguyên liệu là ít nhất. Biết rằng cơ sở cung cấp nguyên liệu chỉ có thể cung cấp không quá $10$ tấn nguyên liệu loại I và không quá $9$ tấn nguyên liệu loại II. 
	\loigiai{
		Gọi số tấn nguyên liệu loại I cần sử dụng là $x$ (tấn); số tấn nguyên liệu loại II cần sử dụng là  $y$ (tấn).\\
		Điều kiện: $0 \leq x \leq 10$, $0 \leq y \leq 9$.\\
		Khi đó số kg chất $A$ thu được là $20x+10y$, số kg chất $B$ thu được là $0{,}6x+1{,}5y$.\\
		Ta có hệ bất phương trình $$\heva{&0\leq x\leq 10\\&0\leq y\leq 9\\&20x+10y\geq120\\&0{,}6x+1{,}5y\geq9}\Leftrightarrow \heva{&0\leq x \leq10\\&0\leq y \leq 9\\&2x+y\geq 12\\&2x+5y\geq30.}$$
		Vẽ các đường thẳng $(d_1) \colon x=10$, $(d_2) \colon y=9$, $(d_3) \colon 2x+y=12$, $(d_4) \colon 2x+5y=30$. \\
		Ta có miền nghiệm của hệ bất phương trình là phần tô màu như hình vẽ:
		\begin{center}
			\begin{tikzpicture}[line join = round, line cap = round, >=stealth, font=\footnotesize, scale=.5]
				\tikzset{label style/.style={font=\footnotesize}}
				\def \xmin{-8}
				\def \xmax{17}
				\def \ymin{-5}
				\def \ymax{15}
				\tkzDefPoints{1.5/9/A, 10/9/B, 10/2/C, 3.75/4.5/D}
				\draw[gray!20](\xmin,\ymin) grid (\xmax,\ymax);
				\draw [->](\xmin,0)--(0,0)node[below left]{$O$}--(\xmax,0)node[above]{$x$};
				\draw [->](0,\ymin)--(0,\ymax)node[right]{$y$};
				\foreach \x in {-5,5,15} \draw[shift={(\x,0)}] (0pt,2pt)--(0pt,-2pt) node[below]{$\x$};
				\draw[shift={(10,0)}] (0pt,2pt)--(0pt,-2pt) node[below left]{$10$};
				\foreach \y in {5,10} \draw [shift={(0,\y)}] (2pt,0pt)--(-2pt,0pt) node[left]{$\y$};
				\tkzDrawLine[add=.8 and .8](C,B)
				\tkzDrawLine[add=.8 and .5](A,B)
				\tkzDrawLine[add=2 and 1.2](D,A)
				\tkzDrawLines[add=1 and 1.5](C,D)
				\tkzDrawPoints[fill=black](A,B,C,D)
				\tkzLabelPoints[above right](A,B,C)
				\tkzLabelPoints[below](D)
				\tkzFillPolygon[color=cyan,fill opacity=.5](A,B,C,D)
				\tkzLabelLine[pos=1.7,right](C,B){$(d_1)$}
				\tkzLabelLine[pos=1.3,above right](A,B){$(d_2)$}
				\tkzLabelLine[pos=2.2,below left](D,A){$(d_3)$}
				\tkzLabelLine[pos=2.4,below left](C,D){$(d_4)$}
			\end{tikzpicture}
		\end{center}
		Ta có 
		{\allowdisplaybreaks
			\begin{eqnarray*}
				&&(d_2) \cap (d_3)=A\left(\dfrac{3}{2};9\right),\ (d_2) \cap (d_1)=B(10;9),\\
				&&(d_1) \cap (d_4)=C(10;2),\ (d_4) \cap (d_3)=D\left(\dfrac{15}{4};\dfrac{9}{2}\right).
			\end{eqnarray*}
		}
		Chi phí mua nguyên liệu cần bỏ ra là $f(x,y)=4x+3y$ (triệu đồng).
		\begin{center}
			\renewcommand\arraystretch{1.6}
			\renewcommand{\tabcolsep}{6mm}
			\begin{tabular}{|c|c|c|c|c|}
				\hline 
				$M(x;y)$& $A$ & $B$ & $C$ & $D$ \\ 
				\hline 
				$f(x,y)=4x+3y$& $3$ & $67$ & $46$ & $28{,}5$ \\
				\hline 
			\end{tabular} 
		\end{center}
		Do đó $f(x,y)$ đạt giá trị nhỏ nhất tại $D\left(\dfrac{15}{4};\dfrac{9}{2}\right)$.\\
		Vậy để chi phí nguyên liệu là ít nhất ta cần sử dụng $\dfrac{15}{4}=3{,}75$ tấn nguyên liệu loại I và $\dfrac{9}{2}=4{,}5$ tấn nguyên liệu loại II.
	}
\end{bt}
\begin{bt}%[0H2T3-4]
		Từ vị trí $A$ người ta quan sát một cây cao (Hình vẽ). Biết $AH=4$ m, $HB=20$ m, $\widehat{BAC}=45^{\circ}$. Chiều cao của cây gần nhất với giá trị nào sau đây?
		\begin{center}
			\usetikzlibrary{decorations.pathmorphing,shapes}
			\tikzset{
				treetop/.style = {
					decoration={random steps, segment length=0.4mm},
					decorate
				},
				trunk/.style = {
					decoration={random steps, segment length=2mm, amplitude=0.2mm},
					decorate
				}
			}
			\tikzset{
				man/.pic={%
					\fill [rounded corners=1.5] (0,0.4) -- (0,0.4 -- (0.4,0.5 -- (0.4,0.4) --
					(0.325,0.4) -- (0.325,0.7) -- (0.3,0.7) -- (0.3,0) -- (0.225,0) --
					(0.225,0.4) -- (0.175,0.4) -- (0.175,0) -- (0.1,0) -- (0.1,0.7) --
					(0.075,0.7) -- (0.075,0.4) -- cycle;
					\fill (0.2,0.9) circle (0.1);
					\coordinate (-head) at (0.2,1);
					\coordinate (-foot) at (0.2,0);
				}
			}
			\begin{tikzpicture}
				\path 
				(-5,-2.09) coordinate (A)
				(0,1.5) coordinate (C)
				(0.1,-3) coordinate (T)
				(-5,-3) coordinate (H)
				(0,-3) coordinate (B)		
				;
				%\pic[red] at (-6.3,-3) (myman) {man};
				\foreach \w/\f in {0.3/30,0.2/50,0.1/70} {
					\fill [brown!\f!black, trunk] (0,0) ++(-\w/2,0) rectangle +(\w,-3);
				}
				\foreach \n/\f in {1.4/40,1.2/50,1/60,0.8/70,0.6/80,0.4/90} {
					\fill [green!\f!black, treetop] ellipse (\n/1.5 and \n);
				}
				\draw (H)--(T) (A)--(H) (A)--(C) (A)--(B);
				\pic[draw,"$45^{\circ}$", angle eccentricity=1.4,angle radius=0.8cm]{angle=B--A--C};
				\pic[draw,"$ $", angle eccentricity=1.4,angle radius=0.7cm]{angle=B--A--C};
				\draw pic[draw, angle radius=2mm]{right angle=B--H--A};
				\path (H)--(B) node[below,midway,sloped]{$20$ m};
				\foreach \x/\g in {A/180,B/-90,C/90,H/-90} \fill[black] (\x)+(\g:.3) node {$\x$};
			\end{tikzpicture}
		\end{center}
		\loigiai{\immini{Ta có $AB= \sqrt{AH^2 + BH^2} = \sqrt{4^2+20^2} = 4 \sqrt{26}$.\\
				$\tan \widehat{HAB} = \dfrac{HB}{HA} = \dfrac{20}{4} = 5 \Rightarrow \widehat{HAB} \approx 78{,}69^{\circ}$.\\
				Do $AH \parallel BC$ nên $ \widehat{ABC} = \widehat{HAB} \approx 78{,}69^{\circ}$.\\
				$\widehat{ACB} = 180^{\circ} - 45^{\circ} - \widehat{ABC} \approx 56{,}31^{\circ}$.\\
				Áp dụng định lí hàm số $\sin$ trong tam giác $ABC$ ta có
				$$ \dfrac{BC}{\sin 45^{\circ}} = \dfrac{AB}{\sin 56{,}31^{\circ}} = \dfrac{4 \sqrt{26}}{\sin 56{,}31^{\circ}} \Rightarrow BC \approx  17{,}33.$$	}
			{\begin{tikzpicture}[scale=0.8, font=\footnotesize,line join = round, line cap = round, >=stealth]
					\tkzDefPoints{-5/-2.09/A, 0/1.5/C, 0.1/-3/T, -5/-3/H,0/-3/B}
					\tkzDrawSegments(H,B A,H A,C A,B B,C)
					\tkzMarkAngles[size=0.6,arc=ll](B,A,C)
					\tkzMarkRightAngles(B,H,A)
					\tkzLabelAngles[pos=1.1](B,A,C){$45^{\circ}$}
					\tkzLabelPoints[above](C)
					\tkzLabelPoints[above](A)
					\tkzLabelPoints[below](B)
					\tkzLabelPoints[below](H)
			\end{tikzpicture}}
		}
	\end{bt}
\begin{bt}%[0H2G3-1]
		Một miếng giấy hình tam giác $ABC$ diện tích $S$ có $I$ là trung điểm $BC$ và $O$ là trung điểm của $AI$. Cắt miếng giấy theo một đường thẳng qua $O$, đường thẳng này đi qua $M$, $N$ lần lượt trên các cạnh $AB$, $AC$. Khi đó diện tích miếng giấy chứa điểm $A$ có diện tích thuộc đoạn $\left[mS; nS\right]$. Tính $T = \dfrac{1}{m} + \dfrac{1}{n}$.
		\loigiai
		{\immini
			{Ta có $\dfrac{S_{\triangle AMN}}{S_{\triangle ABC}} = \dfrac{AM}{AB}\cdot\dfrac{AN}{AC}$\\
				Dễ thấy $S_{\triangle ABI} = S_{\triangle ACI} = \dfrac{1}{2}\cdot S_{\triangle ABC}$.\\
				Mặt khác   
				\begin{eqnarray*}
					&{  }&\dfrac{S_{\triangle AMO}}{S_{\triangle ABI}} = \dfrac{AO}{AI}\cdot\dfrac{AM}{AB}\\
					&=& \dfrac{1}{2}\cdot\dfrac{AM}{AB}\Rightarrow \dfrac{2\cdot S_{\triangle AMO}}{S_{\triangle ABC}} = \dfrac{1}{2}\cdot\dfrac{AM}{AB}\quad (1)
				\end{eqnarray*}	
			}
			{\begin{tikzpicture}[scale=1, font=\footnotesize, line join=round, line cap=round, >=stealth]
					\clip(-1,-1) rectangle (6.5,4.5);
					\tkzDefPoints{0/0/B,1/4/A, 6/0/C}
					\tkzDefMidPoint(B,C) \tkzGetPoint{I}
					\tkzDefMidPoint(A,I) \tkzGetPoint{O}
					\tkzDefBarycentricPoint(A=2,B=3)
					\tkzGetPoint{M}
					\tkzInterLL(M,O)(A,C)\tkzGetPoint{N}
					\tkzInterLL(B,O)(A,C)\tkzGetPoint{N'}
					\tkzInterLL(C,O)(A,B)\tkzGetPoint{M'}
					\tkzDefLine[parallel = through I](B,N') \tkzGetPoint{c}
					\tkzInterLL(I,c)(A,C)\tkzGetPoint{I'}
					\tkzLabelPoints[above](A)
					\tkzLabelPoints[left](M,M')
					\tkzLabelPoints[above right](N,N',I')
					\tkzLabelPoints[below left](I)
					\tkzLabelPoints[below](B,C)
					\tkzDrawPoints[fill=black](A,B,C,M,N,O,I,M',N',I')
					\tkzDrawSegments(A,B B,C C,A A,I M,N)
					\tkzDrawSegments [dashed](B,N' C,M' I,I')
					\draw ($(O)+(-0.1,-0.35)$) node {$O$};
				\end{tikzpicture}
			}\noindent
			Tương tự  $\dfrac{2\cdot S_{\triangle ANO}}{S_{\triangle ABC}} = \dfrac{1}{2}\cdot\dfrac{AN}{AC}\quad (2)$.	
			Từ $(1)$ và $(2)$ suy ra 
			$$\dfrac{2\cdot S_{\triangle AMN}}{S_{\triangle ABC}} = \dfrac{1}{2}\cdot\left(\dfrac{AM}{AB} + \dfrac{AN}{AC}\right)\Leftrightarrow \dfrac{S_{\triangle AMN}}{S_{\triangle ABC}} = \dfrac{1}{4}\cdot\left(\dfrac{AM}{AB} + \dfrac{AN}{AC}\right)$$
			Theo bất đẳng thức Côsi suy ra 
			$$\dfrac{AM}{AB} + \dfrac{AN}{AC}\geq 2\sqrt{\dfrac{AM}{AB}\cdot \dfrac{AN}{AC}}\Leftrightarrow \left(\dfrac{AM}{AB} + \dfrac{AN}{AC}\right)^2\geq 4\cdot \dfrac{AM}{AB}\cdot \dfrac{AN}{AC}$$
			Đặt $t = \dfrac{S_{\triangle AMN}}{S_{\triangle ABC}}$ điều kiện $t > 0$. Khi đó ta có  $16t^2\geq 4t\Leftrightarrow t\geq\dfrac{1}{4}$ suy ra $S_{\triangle AMN}\geq \dfrac{1}{4}\cdot S_{ABC}$.\\
			Khi $M\equiv B$ suy ra $N\equiv N'$ khi đó $S_{\triangle AMN} =  S_{\triangle ABN'}$.\\
			Mà $S_{\triangle ABN'} = S_{\triangle ABO}+ S_{\triangle AON'}$.\\
			Dễ thấy $S_{\triangle ABO} = \dfrac{1}{2}\cdot S_{\triangle ABI} = \dfrac{1}{4}\cdot S_{\triangle ABC}$.\\
			Mặt khác từ $I$ kẻ $II'\parallel BN'$, khi đó $AN' = N'I' = I'C$ nên
			$$\dfrac{S_{\triangle AON'}}{S_{\triangle AIC}} = \dfrac{AO}{AI}\cdot\dfrac{AN'}{AC} = \dfrac{1}{2}\cdot\dfrac{1}{3} = \dfrac{1}{6}\Rrightarrow S_{\triangle AON'} = \dfrac{1}{6}\cdot S_{\triangle AIC}$$
			Do đó $S_{\triangle AON'} = \dfrac{1}{12}\cdot S_{\triangle ABC}$ nên $S_{\triangle ABN'} = \dfrac{1}{4}\cdot S_{\triangle ABC} + \dfrac{1}{12}\cdot S_{\triangle ABC} = \dfrac{1}{3}\cdot S_{\triangle ABC}$.\\
			Khi $N\equiv C$ suy ra $M\equiv M'$ khi đó $S_{\triangle AMN} =  S_{\triangle ACM'}$.\\
			Chứng minh tương tự, ta có  $S_{\triangle ACM'} = \dfrac{1}{3}\cdot S_{\triangle ABC}$.\\
			Do đó khi $MN$ đi thay đổi qua $O$ suy ra 
			$$\dfrac{1}{4}\cdot S_{\triangle ABC}\leq  S_{\triangle AMN}\leq  \dfrac{1}{3}\cdot S_{\triangle ABC}\Leftrightarrow \dfrac{1}{4}\cdot S\leq S_{\triangle AMN} \leq  \dfrac{1}{3}\cdot S$$
			Do đó $m = \dfrac{1}{4}$ và $n = \dfrac{1}{3}$ nên $T = \dfrac{1}{m} + \dfrac{1}{n} = 4 + 3 = 7$.
		}
	\end{bt}
\inputans{10}{ans/ans-0-GK1-CanhDieu-De3-NH23-24}
