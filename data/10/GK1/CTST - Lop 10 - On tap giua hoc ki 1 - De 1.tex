\begin{name}
	{ÔN TẬP KIỂM TRA GIỮA HỌC KÌ 1}
	{TOÁN 10}
	{LỚP TOÁN THẦY PHÁT}
	{\thoigian}
\end{name}

\caulc

\Opensolutionfile{ans}[ans-ABCD]
%Câu 1.
\begin{ex}%%[0D1N1-1]%[CTST - Lớp 10 - Ôn tập giữa học kì 1 - Đề 1]%[Trần Đức Thắng]
	Trong các câu sau, câu nào không phải là mệnh đề Toán học?
	\choice
	{$1$ là nghiệm đa thức $x^2+1$}
	{\True Số $3$ có là ước của số $10$ hay không?}
	{$7$ là số nguyên tố}
	{$\sqrt{2}$ là số vô tỉ}
	\loigiai{
	Câu không phải mệnh đề là \lq\lq Buồn ngủ quá\rq\rq.
}
\end{ex}
%Câu 2.
\begin{ex}%%[0D1N1-2]%[CTST - Lớp 10 - Ôn tập giữa học kì 1 - Đề 1]%[Trần Đức Thắng]
	Mệnh đề nào sau đây \textbf{đúng}?
	\choice
	{\True Một số vừa chia hết cho $2$ vừa chia hết cho $3$ thì nó chia hết cho $6$}
	{$\forall x\in\mathbb{R}:\sqrt{x^2}=x$}
	{Phương trình $x^2-2=0$ có nghiệm hữu tỉ}
	{Hình thoi có hai đường chéo bằng nhau}
	\loigiai{
	Vì $2$ và $3$ là hai số nguyên tố cùng nhau nên \lq\lq Một số vừa chia hết cho $2$ vừa chia hết cho $3$ thì nó chia hết cho $6$\rq\rq.
}
\end{ex}
%Câu 3.
\begin{ex}%%[0D1H1-2]%[CTST - Lớp 10 - Ôn tập giữa học kì 1 - Đề 1]%[Trần Đức Thắng]
	Trong các mệnh đề sau, mệnh đề nào đúng?
	\choice
	{$\forall n\in\mathbb{N}, n > 1$}
	{$\exists n\in\mathbb{N}, n(n+1)$ là một số lẻ}
	{\True $\exists n\in\mathbb{Q}, n^2=n$}
	{$\forall n\in\mathbb{N}, n^2> n$}
	\loigiai{
	\begin{itemize}
		\item $\forall n\in\mathbb{N}, n > 1$:Với $n=0$, thì mệnh đề sai.
		\item $\exists n\in\mathbb{N}, n(n+1)$ là một số lẻ: Với $n\in\mathbb{N}$ thì $n(n+1)$ là một số chẵn. 
		\item $\exists n\in\mathbb{Q}, n^2=n$: Với $n=0$, thì mệnh đề đúng.
		\item $\forall n\in\mathbb{N}, n^2> n$: Với $n=0$, thì mệnh đề sai.
	\end{itemize}
}
\end{ex}
%Câu 4.
\begin{ex}%%[0D1N3-3]%[CTST - Lớp 10 - Ôn tập giữa học kì 1 - Đề 1]%[Trần Đức Thắng]
	Cho $A=(-2;\,4], B=(1;\,8)$. Khi đó $A\cap B$ bằng
\choice
{$(4;\,8)$}
{$(-2;\,1]$}
{$(-2;\,8)$}
{\True $(1;\,4]$}
\loigiai{
	Ta có $A\cap B=(1;\,4]$.
}
\end{ex}
%Câu 5.
\begin{ex}%%[0D1H3-2]%[CTST - Lớp 10 - Ôn tập giữa học kì 1 - Đề 1]%[Trần Đức Thắng]
	Cho hai tập hợp $M=\{x\in\mathbb{R}|-1\leq x\leq 4\}$ và $C_\mathbb{R}N=(-\infty;\,0)$. Tập hợp $M\setminus N$ bằng
	\choice
	{$[-1;\,4]$}
	{\True $[-1;\,0)$}
	{$[0;\,4]$}
	{$(4;\,+\infty)$}
	\loigiai{
	Ta có $M=[-1;\,4]$; $N=[0;\,+\infty)$. Suy ra $M\setminus N=[-1;\,0)$.
	}
\end{ex}
%Câu 6.
\begin{ex}%%[0D2N1-2]%[CTST - Lớp 10 - Ôn tập giữa học kì 1 - Đề 1]%[Trần Đức Thắng]
	Miền nghiệm của bất phương trình $3x+2(y+3)>4(x+1)-y+3$ là nửa mặt phẳng chứa điểm nào sau đây?
	\choice
	{$(3;\,0)$}
	{$(3;\,1)$}
	{\True $(1;\,2)$}
	{$(0;\,0)$}
	\loigiai{
	Ta có $3x+2(y+3)>4(x+1)-y+3\Leftrightarrow -x+3y-1>0$.\\
	Vì $-1+3\cdot 2-1>0$ là mệnh đề đúng nên miền nghiệm của bất phương trình trên chứa điểm có tọa độ $(1;\,2)$.
}
\end{ex}
%Câu 7.
\begin{ex}%%[0D2H1-2]%[CTST - Lớp 10 - Ôn tập giữa học kì 1 - Đề 1]%[Trần Đức Thắng]
	Miền nghiệm của bất phương trình $3x-2y>-6$ là miền không bị gạch (không tính đường biên) trong hình nào sau đây?
	\choice
	{\begin{tikzpicture}[line join=round, line cap=round,>=stealth,thick,scale=.5]
			\tikzset{every node/.style={scale=0.9}}
			\begin{scope}
				\clip (-1,-2) rectangle (4,4);
				\fill[pattern=north west lines] (-4,9)--(-4,-4.5)--(5,-4.5)--cycle;
				\draw (-0.67,4)--(3.33,-2);
			\end{scope}
			\draw[->] (-1,0)--(4,0) node[below]{$x$};
			\draw[->] (0,-2)--(0,4) node[left]{$y$};
			\draw (0,0) node[below left]{$O$};
			\foreach \x in {2}
			\draw[thin] (\x,1pt)--(\x,-1pt) node [above right] {$\x$};
			\foreach \y in {3}
			\draw[thin] (1pt,\y)--(-1pt,\y) node [above right] {$\y$};
	\end{tikzpicture}}
	{\begin{tikzpicture}[line join=round, line cap=round,>=stealth,thick,scale=.5]
			\tikzset{every node/.style={scale=0.9}}
			\begin{scope}
				\clip (-3,-2) rectangle (2,4);
				\fill[pattern=north west lines] (-4,-3)--(5,-3)--(5,10.5)--cycle;
				\draw (0.67,4)--(-3.33,-2);
			\end{scope}
			\draw[->] (-3,0)--(2,0) node[below]{$x$};
			\draw[->] (0,-2)--(0,4) node[left]{$y$};
			\draw (0,0) node[below left]{$O$};
			\foreach \x in {-2}
			\draw[thin] (\x,1pt)--(\x,-1pt) node [above left] {$\x$};
			\foreach \y in {3}
			\draw[thin] (1pt,\y)--(-1pt,\y) node [left] {$\y$};
	\end{tikzpicture}}
	{\True \begin{tikzpicture}[line join=round, line cap=round,>=stealth,thick,scale=.5]
			\tikzset{every node/.style={scale=0.9}}
			\begin{scope}
				\clip (-3,-2) rectangle (2,4);
				\fill[pattern=north west lines] (-4,-3)--(-4,10.5)--(5,10.5)--cycle;
				\draw (0.67,4)--(-3.33,-2) ;
			\end{scope}
			\draw[->] (-3,0)--(2,0) node[below]{$x$};
			\draw[->] (0,-2)--(0,4) node[left]{$y$};
			\draw (0,0) node[below right]{$O$};
			\foreach \x in {-2}
			\draw[thin] (\x,1pt)--(\x,-1pt) node [below right] {$\x$};
			\foreach \y in {3}
			\draw[thin] (1pt,\y)--(-1pt,\y) node [below right] {$\y$};
	\end{tikzpicture}}
	{\begin{tikzpicture}[line join=round, line cap=round,>=stealth,thick,scale=.5]
			\tikzset{every node/.style={scale=0.9}}
			\begin{scope}
				\clip (-3,-4) rectangle (2,2);
				\fill[pattern=north west lines] (-5,4.5)--(-5,-7.5)--(3,-7.5)--cycle;
				\draw (-3.33,2)--(0.67,-4);
			\end{scope}
			\draw[->] (-3,0)--(2,0) node[below]{$x$};
			\draw[->] (0,-4)--(0,2) node[left]{$y$};
			\draw (0,0) node[below left]{$O$};
			\foreach \x in {-2}
			\draw[thin] (\x,1pt)--(\x,-1pt) node [above right] {$\x$};
			\foreach \y in {-3}
			\draw[thin] (1pt,\y)--(-1pt,\y) node [above right] {$\y$};
	\end{tikzpicture}}
	\loigiai
	{Trước hết, ta vẽ đường thẳng $(d)\colon 3x-2y=-6$.\\
		Ta thấy $(0;\,0)$ là một nghiệm của bất phương trình đã cho.\\
		Vậy miền nghiệm cần tìm là nửa mặt phẳng bờ $(d)$ chứa điểm $(0;\,0)$.
	}
\end{ex}
%Câu 8.
\begin{ex}%%[0D2N2-2]%[CTST - Lớp 10 - Ôn tập giữa học kì 1 - Đề 1]%[Trần Đức Thắng]
	Trong các cặp số sau, cặp nào \textbf{không} là nghiệm của hệ bất phương trình $\heva{&x+y-2\leq 0\\& 2x-3y+2>0}$?
	\choice
	{$(0;\,0)$}
	{$(1;\,1)$}
	{\True $(-1;\,1)$}
	{$(-1;\,-1)$}
	\loigiai{
	Ta thay cặp số $(-1;\,1)$ vào hệ ta thấy không thỏa mãn.
	}
\end{ex}
%Câu 9.
\begin{ex}%%[0H4N1-2]%[CTST - Lớp 10 - Ôn tập giữa học kì 1 - Đề 1]%[Trần Đức Thắng]
Giá trị của $\cos 60^{\circ}+\sin 30^{\circ}$ bằng bao nhiêu?
\choice
{$\dfrac{\sqrt{3}}{2}$}
{$\sqrt{3}$}
{$\dfrac{\sqrt{3}}{3}$}
{\True $1$}
\loigiai{
	Sử dụng máy tính cầm tay.
}
\end{ex}
%Câu 10.
\begin{ex}%%[0H4N1-1]%[CTST - Lớp 10 - Ôn tập giữa học kì 1 - Đề 1]%[Trần Đức Thắng]
Trong các đẳng thức sau, đẳng thức nào đúng?
\choice
{$\sin(180^{\circ}-a)=-\cos a$}
{$\sin(180^{\circ}-a)=-\sin a$}
{\True $\sin(180^{\circ}-a)=\sin a$}
{$\sin(180^{\circ}-a)=\cos a$}
\loigiai{
Ta có $\sin(180^{\circ}-a)=\sin a$.
}
\end{ex}
%Câu 11.
\begin{ex}%%[0H4N2-1]%[CTST - Lớp 10 - Ôn tập giữa học kì 1 - Đề 1]%[Trần Đức Thắng]
Gọi $R$ là bán kính đường tròn ngoại tiếp tam giác $ABC$. Đẳng thức nào dưới đây \textbf{sai}?
\choice
{$\dfrac{a}{\sin A}=2R$}
{$\sin A=\dfrac{a}{2R}$}
{\True $b \sin B=2R$}
{$\dfrac{a}{\sin A}=\dfrac{b}{\sin B}=\dfrac{c}{\sin C}=2R$}
\loigiai{
	Từ định lý hàm số $\sin $ trong tam giác $ABC$, ta có $\dfrac{a}{\sin A}=\dfrac{b}{\sin B}=\dfrac{c}{\sin C}=2R$.\\ Suy ra đẳng thức $b \sin B=2R$ là sai.
}
\end{ex}
%Câu 12.
\begin{ex}%%[0H4H1-3]%[CTST - Lớp 10 - Ôn tập giữa học kì 1 - Đề 1]%[Trần Đức Thắng]
	Cho tam giác $ABC$ có $\cos(A-B)-\cos(A+B)=1+\cos C$. Chọn mệnh đề \textbf{đúng} trong các mệnh đề sau
	\choice
	{Tam giác $ABC$ cân tại $A$}
	{\True Tam giác $ABC$ cân tại $C$}
	{Tam giác $ABC$ vuông tại $C$}
	{Tam giác $ABC$ cân tại $B$}
	\loigiai{
	\begin{eqnarray*}
		&&\cos(A-B)-\cos(A+B)=1+\cos C\\
		&\Leftrightarrow &\cos(A-B)-\cos\left(180^{\circ}-C\right)=1+\cos C\\
		&\Leftrightarrow &\cos(A-B)+\cos C=1+\cos C\\
		&\Leftrightarrow &\cos(A-B)=1\\
		&\Leftrightarrow &A-B=0\\
		&\Leftrightarrow &A=B.
	\end{eqnarray*}
	Vậy tam giác $ABC$ cân tại $C$.
}
\end{ex}
\Closesolutionfile{ans}

% \indapan{6}{ans-ABCD}

\cauds

\Opensolutionfile{ans}[ans-DS]

\begin{ex}%%[0D2V2-2]%[CTST - Lớp 10 - Ôn tập giữa học kì 1 - Đề 1]%[Trần Đức Thắng]
	Cho hệ bất phương trình
	$\heva{&3x+2y\geq 9\\&x-2y\leq 3\\& x+y\leq 6\\& x\geq 1}(I)$.
	\choiceTF
	{\True Hệ $(I)$ là hệ bất phương trình bậc nhất hai ẩn}
	{\True $(3;\,2)$ là một nghiệm của hệ bất phương trình}
	{Miền nghiệm của bất phương trình $(I)$ là tam giác}
	{Diện tích miền nghiệm của hệ $(I)$ bằng $7$}
	\loigiai{
		\begin{itemchoice}
			\itemch Đúng.\\
			Dễ thấy hệ $(I)$ là hệ bất phương trình bậc nhất.
			\itemch Đúng.\\
			Thay $(3;\,2)$ vào hệ, ta được tất cả các bất phương trình đều thỏa mãn.
			\itemch Sai.\\
			Mô tả vùng nghiệm của bất phương trình ta được miền nghiệm là tứ giác $ABCD$ như hình vẽ, kể cả các điểm thuộc cạnh của tứ giác đó.
			\begin{center}
				\begin{tikzpicture}[line join=round, line cap=round,>=stealth,thick]
					\path 
					(3,0) coordinate (A)
					(1,3) coordinate (B)
					(1,5) coordinate (C)
					(5,1) coordinate (D)
					(3,3) coordinate (M);
					\foreach \i/\g in {A/90,B/45,C/45,D/90,M/90}
					\fill[black] (\i) circle(1pt)+(\g:4mm)node[scale=1]{$\i$};
					\tikzset{every node/.style={scale=0.9}}
					\begin{scope}
						\clip (-3,-3) rectangle (7,7);
						\fill[pattern=north west lines] (-4,10.5)--(-4,-7.5)--(8,-7.5)--cycle;
						\fill[pattern=north west lines] (-4,-3.5)--(18,-3.5)--(18,7.5)--cycle;
						\fill[pattern=north east lines] (-4,10)--(10,10)--(10,-4)--cycle;
						\fill[pattern=north west lines] (1,-3)--(-3,-3)--(-3,7)--(1,7)--cycle;
						\draw (-1.67,7)--(5,-3);
						\draw (17,7)--(-3,-3);
						\draw (-1,7)--(9,-3);
						\draw (1,-3)--(1,7);
					\end{scope}
					\draw[->] (-3,0)--(7,0) node[below left]{$x$};
					\draw[->] (0,-3)--(0,7) node[below left]{$y$};
					\draw (0,0) node[below left]{$O$};
					\foreach \x in {1,3,5,6}
					\draw (\x,1pt)--(\x,-1pt) node [below left] {$\x$};
					\foreach \y in {1,3,5,6}
					\draw (1pt,\y)--(-1pt,\y) node [ left] {$\y$};
				\end{tikzpicture}
			\end{center}
			\itemch Sai.\\
			Ta tính diện tích vùng nghiệm $S_{ABCD}=S_{\triangle ABM}+S_{\triangle ADM}+S_{\triangle CDM}=3+3+2=8$.
		\end{itemchoice}
	}
\end{ex}

\begin{ex}%%[0H4V3-1]%[CTST - Lớp 10 - Ôn tập giữa học kì 1 - Đề 1]%[Trần Đức Thắng]
	Cho tam giác $ABC$ biết cạnh $a=137{,}5$ cm, $\widehat{B}=83^{\circ}$, $\widehat{C}=57^{\circ}$.
	\choiceTF
	{\True $A=40^{\circ}$}
	{$\dfrac{a}{\sin A}=\dfrac{b}{\sin B}=\dfrac{c}{\sin C}=R$}
	{\True $R\approx 106{,}96$ cm}
	{$b\approx 179{,}4$ cm}
	\loigiai{
		\begin{itemchoice}
			\itemch Đúng.\\
			$A=180^{\circ}-(\widehat{B}+\widehat{C})=180^{\circ}-(83^{\circ}+57^{\circ})=40^{\circ}$.
			\itemch Sai.\\
			Theo định lí $\sin$: $\dfrac{a}{\sin A}=\dfrac{b}{\sin B}=\dfrac{c}{\sin C}=2R$.
			\itemch Đúng.\\
			Suy ra $R=\dfrac{137{,}5}{2\sin A}=\dfrac{137{,}5}{2\sin 40^{\circ}}\approx 106{,}96$ cm.
			\itemch Sai.\\
			$b=\dfrac{a\cdot \sin B}{\sin A}=\dfrac{137{,}5\cdot \sin 83^{\circ}}{\sin 40^{\circ}}\approx 212{,}32$ cm.
		\end{itemchoice}
	}
\end{ex}
\Closesolutionfile{ans}

% \indapan{3}{ans-DS}

\caukq

\Opensolutionfile{ans}[ans-KQ]
%Câu 1.
\begin{ex}%%[0D1V3-1]%[CTST - Lớp 10 - Ôn tập giữa học kì 1 - Đề 1]%[Trần Đức Thắng]
	Cho các tập hợp $A=\{x\in\mathbb{Q}:(x-1)(3x-2)(x+\sqrt{2})=0\}$ và $B=\{x\in\mathbb{R}:(2+x)(3x-m^2+4)=0\}$. 
	Tích các giá trị của tham số $m$ để $n(A\cup B)=3$ bằng bao nhiêu?
\shortans{42}
\loigiai{
	Xét các phương trình
	\begin{itemize}
		\item $(x-1)(3x-2)(x+\sqrt{2})=0\Leftrightarrow \heva{&x-1=0\\&3x-2=0\\&x+\sqrt{2}=0}\Leftrightarrow \heva{&x=1\in\mathbb{Q}\\&x=\dfrac{2}{3}\in\mathbb{Q}\\&x=-\sqrt{2}\notin\mathbb{Q}}\Rightarrow A=\left\{1;\dfrac{2}{3}\right\}$.
		\item $(2+x)(3x-m^2+4)=0\Leftrightarrow \heva{&x+2=0\\&3x-m^2+4=0} \Leftrightarrow \heva{&x=-2\\&x=\dfrac{m^2-4}{3}}\Rightarrow B=\left\{\dfrac{m^2-4}{3};\,-2\right\}$.
	\end{itemize}
	Ta thấy \\
	$m^2\geq 0,\forall m\in\mathbb{R}\Rightarrow\dfrac{m^2-4}{3}\geq-\dfrac{4}{3},\forall m\in\mathbb{R}\Rightarrow-2\ne \dfrac{m^2-4}{3}\Rightarrow B=\left\{\dfrac{m^2-4}{3};\,-2\right\}$.\\
	Khi đó\\ 
	$n(A\cup B)=3\Leftrightarrow \hoac{&\dfrac{m^2-4}{3}=1\\&\dfrac{m^2-4}{3}=\dfrac{2}{3}}
	\Leftrightarrow \hoac{&m=\pm \sqrt{7}\\&m=\pm \sqrt{6}.}$\\
	Suy ra tích các giá trị của tham số $m$ là $-\sqrt{7}\cdot \sqrt{7}\cdot (-\sqrt{6})\cdot \sqrt{6}=42$.
}
\end{ex}

\begin{ex}%%[0D2V2-3]%[CTST - Lớp 10 - Ôn tập giữa học kì 1 - Đề 1]%[Trần Đức Thắng]
	 Một hộ nông dân dự định trồng nha đam và măng tây trên diện tích $10$ ha. Nếu trồng nha đam thì cần $10$ công và thu được $4$ triệu đồng trên diện tích mỗi ha. Nếu trồng măng tây thì cần $30$ công và thu được $6$ triệu đồng trên diện tích mỗi ha. Hỏi số tiền người nông dân thu được nhiều nhất là bao nhiêu, biết rằng tổng số công không vượt quá $150$ công. 
	\shortans{45}
	\loigiai{
	 Gọi $x,\, y$ lần lượt là diện tích (ha) trồng nha đam và măng tây $(x\geq 0,\, y\geq 0)$.\\
	Theo đề bài, ta có hệ phương trình sau $\heva{&x\geq 0\\& y\geq 0\\& x+y\leq 10\\&10x+30y\leq 150}\Leftrightarrow \heva{&x\geq 0\\&y\geq 0\\&x+y\leq 10\\&x+3y\leq 15.}$\\
	Số tiền người nông dân thu được là $\mathrm{F}(x,\, y)=4x+6y$ (triệu đồng).\\
	Ta cần tìm giá trị lớn nhất của hàm $\mathrm{F}(x,\, y)=4x+6y$ với $x,\, y$ thỏa mãn các điều kiện trong đề bài.\\
	\textbf{Bước $1$}. Biểu diễn miền nghiệm và xác định miền nghiệm của hệ bất phương trình trên.
	\begin{center}
		\begin{tikzpicture}[line join=round, line cap=round,>=stealth,thick,scale=.7]
		\path 
		(0,4.8) coordinate (A)
		(7.5,2.5) coordinate (B)
		(9,0) coordinate (C);
		\foreach \i/\g in {A/-45,B/-135,C/90}
		\fill[black] (\i) circle(1pt)+(\g:4mm)node[scale=1]{$\i$};
			\tikzset{every node/.style={scale=0.9}}
			\begin{scope}
				\clip (-2,-2) rectangle (15,11);
				\fill[pattern=north west lines] (0,-2)--(-2,-2)--(-2,11)--(0,11)--cycle;
				\fill[pattern=north west lines] (-2,0)--(-2,-2)--(15,-2)--(15,0)--cycle;
				\fill[pattern=north west lines] (-3,13)--(16,13)--(16,-6)--cycle;
				\fill[pattern=north west lines] (-19,11.33)--(22,11.33)--(22,-2.33)--cycle;
				\draw (-1,11)--(12,-2);
				\draw (-18,11)--(21,-2);
			\end{scope}
			\draw[->] (-2,0)--(15,0) node[below left]{$x$};
			\draw[->] (0,-2)--(0,11) node[below right]{$y$};
			\draw (0,0) node[below left]{$O$};
			\foreach \x in {2,4,6,8,10,12,14}
			\draw (\x,1pt)--(\x,-1pt) node [below left] {$\x$};
			\foreach \y in {2,4,6,8,10}
			\draw (1pt,\y)--(-1pt,\y) node [left] {$\y$};
		\end{tikzpicture}
	\end{center}
	Miền nghiệm là tứ giác $OABC$ với tọa độ các đỉnh $O(0;\,0)$,  $A(0;\, 5)$, $B(7{,}5;\,2{,}5)$, $C(10;\, 0)$.\\
	\textbf{Bước $2$}. Tính giá trị của biểu thức $F$ tại các đỉnh của ngũ giác này là
	\[\mathrm{F}(0;\,0)= 0,\, \mathrm{F}(0;\,5)=30,\, \mathrm{F}(7{,}5;\, 2{,}5)=30+15=45,\, \mathrm{F}(10;\,0)=40.\]
	\textbf{Bước $3$}. So sánh các giá trị thu được của $\mathrm{F}$ ở Bước $2$, ta được giá trị lớn nhất cần tìm là
	\[\mathrm{F}(7{,}5;\,2{,}5)=30+15=45.\]
	Vậy số tiền bác nông dân thu được nhiều nhất là $45$ triệu. 
	}
\end{ex}
%Câu 4.
\begin{ex}%%[0H4V1-2]%[CTST - Lớp 10 - Ôn tập giữa học kì 1 - Đề 1]%[Trần Đức Thắng]
	Cho $\tan a=2$, tính giá trị của biểu thức $A=\dfrac{1+\cos a}{\sin a}\left[1-\dfrac{(1-\cos a)^2}{\sin^2a}\right]$.
	\shortans{1}
	\loigiai{
	Ta có
	\begin{eqnarray*}
		A&=&\dfrac{1+\cos a}{\sin a}\left[1-\dfrac{(1-\cos a)^2}{\sin^2a}\right]\\
		&=&\dfrac{1+\cos a}{\sin a}\left[1-\dfrac{(1-\cos a)^2}{1-\cos^2a}\right]\\
		&=&\dfrac{1+\cos a}{\sin a}\left[1-\dfrac{(1-\cos a)^2}{(1-\cos a)(1+\cos a)}\right]\\
		&=&\dfrac{1+\cos a}{\sin a}\left[1-\dfrac{1-\cos a}{1+\cos a}\right]\\
		&=&\dfrac{1+\cos a}{\sin a}\left[\dfrac{1+\cos a-(1-\cos a)}{1+\cos a}\right]\\
		&=&\dfrac{1+\cos a}{\sin a}\cdot \dfrac{2\cos a}{1+\cos a}\\
		&=&2\cot a\\
		&=&\dfrac{2}{\tan a}\\
		&=&1.
	\end{eqnarray*}
	Vậy $A=1$.
	}
\end{ex}

\begin{ex}%%[0H4C3-2]%[CTST - Lớp 10 - Ôn tập giữa học kì 1 - Đề 1]%[Trần Đức Thắng]
	Để đo chiều cao toà tháp người ta dùng dụng cụ đo góc có chiều cao 1,2 m đặt tại hai vị trí trên mặt đất cách nhau một khoảng $AB=30$ m. Tại vị trí $A$ và $B$ góc đo thu được so với phương ngang lần lượt là $\alpha=65^\circ;\,\beta=50^\circ$ (hình minh hoạ). Chiều cao $h$ của toà tháp là bao nhiêu? (kết quả làm tròn đến hàng phần trăm).
	\begin{center}
		\begin{tikzpicture}[line join=round, line cap=round,scale=1,transform shape,font=\small]
			\clip (-5,-4) rectangle (6,4);
			\definecolor{darkgray}{rgb}{0.66, 0.66, 0.66}
			\definecolor{arsenic}{rgb}{0.23, 0.27, 0.29}
			\tikzset{landmark81/.pic={
					\def\N{ 
						(.5,-3)--(.5,-.5)--(.35,-.5)--(.35,1.3)--(.2,1.3)--(.2,2.8)--(.1,2.8)--(.1,3.7)--(-.1,3.7)--(-.1,2.4)--(-.2,2.4)--(-.2,1)--(-.4,1)--(-.4,-3)--cycle
						;
					}
					\fill[darkgray] \N;
					\draw[pattern=horizontal lines, pattern color=gray]\N;
					\draw[pattern={vertical lines}, pattern color=gray]\N;
					
					\fill[arsenic!90] (.5,-2.2)--(-.4,-2.2)--(-.4,-2.1)--(.5,-2.1)--cycle
					(.5,-.5)--(-.4,-.5)--(-.4,-.65)--(.5,-.65)--cycle
					(.35,.8)--(-.4,.8)--(-.4,.9)--(.35,.9)--cycle
					(.2,2.2)--(-.2,2.2)--(-.2,2.1)--(.2,2.1)--cycle
					(.2,2.5)--(.2,2.8)--(.1,2.8)--(.1,3.7)--(-.1,3.7)--(-.1,2.5)--cycle
					;
					
					\draw (.4,-3)--(.4,-2.4)--(.5,-2.4)
					(-.3,-3)--(-.3,-2.5)--(-.4,-2.5)
					(.2,-3)--(.2,1.3) (0,-3)--(0,2.5) (-.1,-3)--(-.1,2.5) (-.2,-3)--(-.2,0)--(-.1,0);
			}}
			\path (-3,0)pic[scale=1]{landmark81};
			
			\path 	(3,-3) coordinate (B)
			(-2,-3) coordinate (H)
			($(H)!1/3!(B)$) coordinate (A)
			($(A)+(0,1)$) coordinate (C)
			(-2,3.7) coordinate (N)
			(-3,3.7) coordinate (M)
			($(N)!(C)!(H)$) coordinate (C')
			($(B)+(C)-(A)$) coordinate (D)
			;
			\node at (1.5,-3) [below]{$30$ m};
			\node at (3.5,-2.3) [below]{$1{,}2$ m};
			\node at (-4.2,0) [below]{$h$};
			\draw (N)--(H)--(B) (A)--(C) (B)--(D) ;
			\draw[<->]($(N)-(2,0)$)--($(H)-(2,0)$);
			\draw[<->]($(A)-(0,.5)$)--($(B)-(0,.5)$);
			\draw[dashed] (N)--(C)--(C') (N)--(D)--(C);
			\foreach \d/\g in {A/-90,B/-45,H/-90,N/90,C/45,D/45,M/90}{
				\draw[fill=black](\d) circle (.5pt) +(\g:.2)node{$\d$};}
			\draw pic["$\alpha$", draw=black, angle eccentricity=1.5, angle radius=.5cm]{angle=N--C--C'}; 
			\draw pic["$\beta$", draw=black, double, angle eccentricity=1.5, angle radius=.5cm]{angle=N--D--C'}; 
		\end{tikzpicture}
	\end{center}
	\shortans{81,7}
	\loigiai{
	Đặt các điểm như hình vẽ.
	Ta có\\ $\alpha=65^{\circ}\Rightarrow\widehat{DCN}=180^{\circ}-65^{\circ}=115^{\circ}$. Do đó $\widehat{CND}=180^\circ-115^\circ-50^\circ=15^\circ$.\\
	Áp dụng định lí sin trong tam giác $CDN$, ta được \[\dfrac{CD}{\sin N}=\dfrac{CN}{\sin D}.\]
	Mà $CD=AB=30$ m nên $\dfrac{30}{\sin15^\circ}=\dfrac{CN}{\sin50^\circ}\Leftrightarrow CN=\dfrac{30\cdot \sin50^\circ}{\sin15^\circ}=88{,}8$ m.\\
	Xét tam giác $NHC$ vuông tại $H$, ta có
	\[NH=CN\cdot \sin\alpha=88{,}8\cdot \sin 65^\circ=80{,}5\, \text{m}.\]
	Vậy chiều cao của toà tháp là $h=80{,}5+1{,}2=81{,}7$ m.
	}
\end{ex}
\Closesolutionfile{ans}
\TL
% \indapan{6}{ans-KQ}
\begin{ex}%%[0D1V3-5]%[CTST - Lớp 10 - Ôn tập giữa học kì 1 - Đề 1]%[Trần Đức Thắng]
	 Lớp 10A có $45$ học sinh trong đó có $25$ em học sinh học giỏi môn Toán, $23$ em học sinh học giỏi môn Văn, $20$ em học sinh học giỏi môn Tiếng Anh. Đồng thời có $11$ em học sinh học giỏi cả môn Toán và môn Văn, $8$ em học sinh học sinh giỏi cả môn Văn và môn Tiếng Anh, $9$ em học sinh học giỏi cả môn Toán và môn Tiếng Anh, biết rằng mỗi học sinh trong lớp học giỏi ít nhất một trong ba môn Toán, Văn, Tiếng Anh. Hỏi lớp 10A có bao nhiêu bạn học giỏi cả ba môn Toán, Văn, Tiếng Anh? 
	% \shortans{5}
	\loigiai{
	Ta có
	\begin{center}
\begin{tikzpicture}[line join=round, line cap=round,>=stealth]
	\tikzset{label style/.style={font=\footnotesize}}
	\draw[name path=elip1,rotate=35] (0,1) ellipse (2 cm and 1.5 cm);
	\draw[color=black] (-0.5,2) node[right] {$\text{Văn}$};
	\draw[name path=elip2] (-2,0) ellipse (2 cm and 1.5 cm);
	\draw[color=black] (-3.9,0) node[right] {$\text{Toán}$};
	\draw[name path=elip1,rotate=-60] (1,-0.5) ellipse (2 cm and 1.5 cm);
	\draw[color=black] (0,-2) node[right] {$\text{Anh}$};
\end{tikzpicture}
	\end{center}
	 Gọi $x\, (x\in\mathbb{N})$ là số học sinh giỏi cả ba môn Toán, Văn, Anh.\\
	Số học sinh chỉ giỏi Toán và Văn là $11-x$ (học sinh).\\
	Số học sinh chỉ giỏi Toán và Anh là $9-x$ (học sinh).\\
	Số học sinh chỉ giỏi Văn và Anh là: $8-x$ (học sinh).\\
	Số học sinh chỉ giỏi Toán là $25-(11-x)-(9-x)-x=5+x$ (học sinh).\\
	Số học sinh chỉ giỏi Văn là $23-(11-x)-(8-x)-x=4+x$(học sinh).\\
	Số học sinh chỉ giỏi Anh là $20-(9-x)-(8-x)-x=3+x$ (học sinh).\\
	Lớp có $45$ học sinh nên ta có
	\[x+(11-x)+(9-x)+(8-x)+5+x+4+x+3+x=45\Leftrightarrow x+40=45\Leftrightarrow x=5.\]
	Vậy có $5$ học sinh giỏi cả ba môn Toán, Văn và Anh. 
	}
\end{ex}

\begin{ex}%[Dự án 21 - TLDH - TeamTeXHoa - Hồng Trường Sơn]%[0H4H1-2]
	Tính giá trị của $A=\tan 5^\circ \cdot\tan 10^\circ \cdot\tan 15^\circ \ldots\tan 80^\circ \cdot\tan 85^\circ$.
	% \shortans[]{$1$}
	\loigiai{
		Ta có
		$$A \left( \tan 5^\circ \cdot\tan 85^\circ \right)\cdot\left( \tan 10^\circ \cdot\tan 80^\circ \right)\ldots\left( \tan 40^\circ \tan 50^\circ \right)\cdot\tan 45^\circ =1.$$
	}
\end{ex}
\begin{ex}%[0-HK1-CTST-4-2324]%[VN-MT-9, Phúc Hậu]%[0D2V2-3]
Một cơ sở chiết xuất ít nhất $140$ kg chất $X$ và ít nhất $18$ kg chất $Y$ từ hai loại nguyên liệu loại I và loại II. Với mỗi tấn nguyên liệu loại I, người ta chiết xuất được $20$ kg chất $X$ và $1{,}2$ kg chất $Y$. Với mỗi tấn nguyên liệu loại II, người ta chiết xuất được $10$ kg chất $X$ và $3$ kg chất $Y$. Giá mỗi tấn nguyên liệu loại I là $12$ triệu đồng và loại II là $8$ triệu đồng. Hỏi người ta phải dùng ít nhất bao nhiêu triệu đồng để mua nguyên liệu mà vẫn đạt mục tiêu đề ra. Biết rằng cơ sở nhập nguyên liệu tối đa $9$ tấn nguyên liệu loại I và tối đa $8$ tấn nguyên liệu loại II. (Làm tròn kết quả đến hàng đơn vị).
% \par
% \shortans{92}
\loigiai{
Gọi $x$, $y$ lần lượt là số tấn nguyên liệu loại I và loại II cần dùng.\\
Điều kiện $0\leq x\le9$; $0\leq y\le8$.\\
Theo giả thiết, ta có bất phương trình $0{,}02x+0{,}01y\geq0{,}14$ hay $2x+y\geq14$.\\
Và $0{,}0012x+0{,}003y\geq0{,}018$ hay $2x+5y\geq30$.\\
Từ đó ta có hệ bất phương trình
$(*)\colon \heva{&0\leq x\le9&\\&0\leq y\le8\\&2x+y\geq14\\&2x+5y\geq30.}$\\
Tìm $x$, $y$ thỏa mãn hệ bất phương trình $(*)$ sao cho biểu thức $F(x,y)=12x+8y$ nhỏ nhất.\\
Miền nghiệm của hệ là miền trong tứ giác $ABDC$ (hình vẽ) với $A(3;8)$, $B(5;4)$, $D\left(9;\dfrac{12}{5}\right)$, $C(9;8)$.
\begin{center}
\begin{tikzpicture}[scale=1,>=stealth, font=\footnotesize, line join=round, line cap=round,x=0.5cm, y=0.5cm]
\draw[->] (-1.5,0)--(11,0) node[below right]{$x$};
\draw[->] (0,-1.2)--(0,10) node[right]{$y$};
\clip (-1.45,-1.19) rectangle (10.95,9.95);
\filldraw[pattern = north east lines]
(-1.5,-1.2)--(-1.5,10)--(11,10)--(11,-1.2)--cycle;
\filldraw[white] (3,8)--(9,8)--(9,12/5)--(5,4)--cycle;
\draw[samples=100, domain=2.0:7.5] plot(\x,{-2*(\x)+14});
\draw[samples=100, domain=-1:10.9] plot(\x,{(-2/5)*(\x)+6});
\draw (-1,8)--(11,8) (9,-1)--(9,10);
\fill (3,8)node[above]{$A$}circle(1pt)
(5,4)node[below left]{$B$}circle(1pt)
(9,8)node[above right]{$C$}circle(1pt)
(9,12/5)node[below left]{$D$}circle(1pt)
(0,0)node[below left]{$O$}circle(1pt);
\foreach \x/\y/\z/\t in {3/0/3/-90,5/0/5/-90, 7/0/7/-120, 9/0/9/-120, 0/2.4/\frac{12}{5}/180, 0/4/4/180, 0/6/6/-150, 0/8/8/150} \fill(\x,\y)circle(1pt) ($(\x,\y)+(\t:3mm)$)node{$\z$};
\draw[dashed] (3,0)--(3,8) (5,0)--(5,4)--(0,4) (0,2.4)--(9,2.4)
;
\end{tikzpicture}
\end{center}
Tại đỉnh $A$, ta có $F=100$.\\
Tại đỉnh $B$, ta có $F=92$.\\
Tại đỉnh $D$, ta có $F=127{,}2$.\\
Tại đỉnh $C$, ta có $F=172$.\\
Vậy cơ sở chi phí thấp nhất $92$ triệu đồng.
}
\end{ex}
\begin{ex}%%[0H4V3-1]%[CTST - Lớp 10 - Ôn tập giữa học kì 1 - Đề 1]%[Trần Đức Thắng]
	Cho tam giác $ABC$ có hai trung tuyến $BM$ và $CN$ hợp với nhau một góc $120^\circ$. Biết $BM=12,\, CN=15$. Tính chu vi của tam giác $ABC$ (kết quả là tròn đến hàng đơn vị).
	\begin{center}
		\begin{tikzpicture}[scale=0.7, font=\footnotesize,line join=round, line cap=round, >=stealth]
			\path 
			(2,4) coordinate (A)
			(0,0) coordinate (B)
			(6,0) coordinate (C)
			($(A)!1/2!(B)$) coordinate (N)
			($(A)!1/2!(C)$) coordinate (M)
			($(B)!2/3!(M)$) coordinate (G);
			\draw (A)--(B)--(C)--cycle (B)--(M) (C)--(N);
			\pic["$120^\circ$",draw,angle eccentricity=1.8,angle radius=2mm]{angle=B--G--C};
			\foreach \i/\g in {A/90,B/-90,C/-90,M/60,N/120,G/90}
			\fill[black] (\i) circle(1pt)+(\g:4mm)node[scale=1]{$\i$};
		\end{tikzpicture}
	\end{center}
	% \shortans{47}
	\loigiai{
	Gọi $BM\cap CN=\{G\}\Rightarrow G$ là trọng tâm của tam giác $ABC$.\\
	$\Rightarrow GB=\dfrac{2}{3}BM=8,\, GC=\dfrac{2}{3}CN=10\Rightarrow GM=4,\, GN=5$.\\
	Áp dụng định lý côsin trong tam giác $GBC$, ta có
	\[BC^2=GB^2+GC^2-2GB\cdot GC\cdot \cos120^\circ=244\Rightarrow BC=2\sqrt{61}.\]
	Do đó $\widehat{BGN}=180^\circ-\widehat{BGC}=60^\circ,\, \widehat{MGC}=180^\circ-\widehat{BGC}=60^\circ$.\\
	Áp dụng định lý côsin, ta được
	\[BN^2=GB^2+GN^2-2GB\cdot GN\cdot \cos60^\circ=49\Rightarrow BN=7\Rightarrow AB=2BN=14.\]
	\[MC^2=GM^2+GC^2-2GM\cdot GC\cdot \cos 60^\circ=76\Rightarrow MC=2\sqrt{19}\Rightarrow AC=2MC=4\sqrt{19}.\]
	Vậy chu vi của tam giác $ABC$ là \[AB+BC+CA=14+2\sqrt{61}+4\sqrt{19}=47{,}06\approx 47.\]
	}
\end{ex}