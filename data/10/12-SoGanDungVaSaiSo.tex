\section{Số gần đúng và sai số}
\subsection{Tóm tắt lý thuyết}
\subsubsection{Số gần đúng}
	\begin{boxdn}
	Trong nhiều trường hợp, ta không biết hoặc khó biết số đúng (kí hiệu là $\overline{a}$) mà chỉ tìm được giá trị khác xấp xỉ nó. Giá trị này được gọi là \textbf{số gần đúng}, kí hiệu là $\mathbf{a}$.
	\end{boxdn}	
\subsubsection{Sai số tuyệt đối và sai số tương đối}
\begin{boxdn}
	\textbf{sai số tuyệt đối} của số gần đúng $a$ là giá trị $|a-\overline{a}|$ phản ánh mức độ sai lệch giữa số đúng $\overline{a}$ và số gần đúng $a$, kí hiệu là $\mathbf{\Delta_a}$, tức là \fbox{$\Delta_a=|a-\overline{a}|$}
\end{boxdn}
\begin{boxdn}
	Sai số tương đối của số gần đúng $a$ là tỉ số giữa sai số tuyệt đối và $|\overline{a}|$, kí hiệu là $\delta_a$, tức là \fbox{$\delta_a=\dfrac{\Delta_a}{|a|}$}.
\end{boxdn}
Trên thực tế, nhiều khi ta không biết $\overline{a}$ nên cũng không biết $\Delta_a$. Tuy nhiên, ta có thể đánh giá được $\Delta_a$ không vượt quá một số dương $d$ nào đó.
\begin{boxdn}
	Nếu $\Delta_{a} \leq d$ thì $a-d \le \overline{a} \le a+d$, khi đó ta viết $mathbf{\overline{a}=a \pm d}$ và hiểu là số đúng $\overline{a}$ nằm trong đoạn $[a-d; a+d]$. Do $d$ càng nhỏ thì $a$ càng gần $\overline{a}$ nên $d$ được gọi là \textbf{độ chính xác của số gần đúng} $a$.
\end{boxdn}
\begin{nx}
Nếu $\overline{a}=a \pm d$ thì $\Delta_a \leq d$, do đó $\delta_a \leq \dfrac{d}{|a|}$. Nếu $\dfrac{d}{|a|}$ càng nhỏ thì chất lượng của phép đo hay tính toán càng cao. Người ta thường viết sai số tương đối dưới dạng phần trăm.	
\end{nx}
\subsubsection{Quy tròn số gần đúng}
\begin{boxdn}
	Số thu được sau khi thực hiện làm tròn số được gọi là \textbf{số quy tròn}. Số quy tròn là một \textit{số gần đúng} của số ban đầu.
\end{boxdn}
\textbf{Quy tắc quy tròn số}
\begin{enumerate}
	\item Đối với chữ số hàng làm tròn:\\
	- Giữ nguyên nếu chữ số ngay bên phải nó nhỏ hơn $5$;\\
	- Tăng $1$ đơn vị nếu chữ số ngay bên phải nó lớn hơn hoặc bằng $5$.
	\item Đối với chữ số sau hàng làm tròn:\\
	- Bỏ đî nếu ở phần thập phân;\\
	- Thay bởi các chữ số $0$ nếu ở phần số nguyên.
\end{enumerate}
\begin{nx}
- Khi thay số đúng bởi số quy tròn đến một hàng nào đó thì sai số tuyệt đối của số quy tròn không vượt quá nửa đơn vị của hàng làm tròn.\\
- Cho số gần đúng $a$ với độ chính xác $d$. Khi được yêu cầu làm tròn số $a$ mà không nói rõ làm tròn đến hàng nào thì ta làm tròn số $a$ đến hàng thấp nhất mà $d$ nhỏ hơn $1$ đơn vị của hàng đó.
\end{nx}
\subsection{Các dạng toán}
\begin{dang}{Xác định số gần đúng của một số với độ chính xác cho trước, đánh giá độ chính xác}
\end{dang}
\viduminhhoa
% \begin{vd}%[Duong Xuan Loi]%[0D1Y5-2]
% Đỉnh Everest được mệnh danh là "nóc nhà của thế giới", bởi đây là đỉnh núi cao nhất trên Trái Đất so với mực nước biển. Có rất nhiều con số khác nhau đã từng được công bố về chiều cao của đỉnh Everest
% $$
% 8848 \text { m; } 8848{,}13 \text { m; } 8844{,}43 \text { m; } 8850 \text { m;... }
% $$
% Các con số này đều là số gần đúng chiều cao của đỉnh Everest. 
% \end{vd}
% \begin{vd}%[Duong Xuan Loi]%[0D1Y5-2]
% 	Điền dấu “X” vào ô tương ứng
% 	\begin{center}
% 		\begin{tabular}{|c|c|c|}
% 			\hline
% 			Thông tin&  Số đúng&  Số gần đúng\\
% 			\hline
% 			Bán kính đường Xích Đạo của Trái Đất là $6~378$ km&  &  \\
% 			\hline
% 			Khoảng cách từ Mặt Trăng đến Trái Đất là $384~400$ km&  &  \\
% 			\hline
% 			$1$ m=$100$ cm&&\\
% 			\hline
% 		\end{tabular}
% 	\end{center}
% 	\loigiai{
% 		\begin{center}
% 			\begin{tabular}{|c|c|c|}
% 				\hline
% 				Thông tin&  Số đúng&  Số gần đúng\\
% 				\hline
% 				Bán kính đường Xích Đạo của Trái Đất là $6~378$ km&  & X \\
% 				\hline
% 				Khoảng cách từ Mặt Trăng đến Trái Đất là $384~400$ km&  & X \\
% 				\hline
% 				$1$ m=$100$ cm	&	X	&\\
% 				\hline
% 			\end{tabular}
% 		\end{center}
% 	}
% \end{vd}
% \begin{vd}%[Duong Xuan Loi]%[0D1Y5-2]
% 	Gọi $d$ là độ dài đường chéo của hình vuông cạnh bằng $1$. Trong hai số $\sqrt{2}$ và $1{,}41$ số nào là số đúng, số nào là số gần đúng của $d$?
% 	\loigiai{
% 		Hình vuông có cạnh bằng $1$ có độ dài của đường chéo là $d=\sqrt{1^2+1^2}=\sqrt{2}$.\\
% 		Vậy $\sqrt{2}$ là số đúng; $1{,}41$ là số gần đúng của $d$.
% 	}
% \end{vd}	
\begin{vd}
	Giả sử khối lượng đúng của một hộp kẹo là $0{,}85$ kg. Bình và An cân hộp kẹo này và ghi nhận kết quả lần lượt là $0{,}8$ kg và $1$ kg.
	\begin{enumerate}
		\item Tìm sai số tuyệt đối của kết quả cân của mỗi bạn.
		\item Kết quả cân của bạn nào chính xác hơn? Vì sao?
	\end{enumerate}
	\loigiai{
		\begin{enumerate}
			\item Sai số tuyệt đối của kết quả cân của bạn Bình là $0{,}85-0{,}8=0{,}05$ kg.\\
			Sai số tuyệt đối của kết quả cân của bạn An là $1-0{,}85=0{,}15$ kg.
			\item Vì sai số tuyệt đối của bạn Bình nhỏ hơn sai số của bạn An nên kết quả cân của bạn Bình chính xác hơn.
		\end{enumerate}
	}
\end{vd}
\begin{vd}
	Người ta dùng một đồng hồ bấm giờ với độ chia nhỏ nhất là $0{,}1$ giây và đo được thời gian hoàn thành phần thi bơi của một vận động viên là $27{,}2$ giây.
	\begin{enumerate}
		\item Tìm độ chính xác $d$ của phép đo.
		\item Nếu thời gian đúng là $a$ giây thì hãy tìm khoảng giá trị mà $a$ có thể nhận được.
	\end{enumerate}
	\loigiai{
		\begin{enumerate}
			\item Vì độ chia nhỏ nhất của đồng hồ là $0{,}1$ giây nên độ chính xác của phép đo là $d=\dfrac{0{,}1}{2}=0{,}05$ giây.
			\item Ta có $27{,}2-0{,}05\leq a<27{,}2+0{,}05\Rightarrow 27{,}15\leq a < 27{,}25$.
		\end{enumerate}
	}
\end{vd}

\baitaptl
	\begin{bt}%[Mui Doan]%[0D1Y5-2]
	Một bao gạo ghi thông tin khối lượng là $ 5\pm 0{,}2 $ kg.
	\begin{enumerate}
		\item Xác định khối lượng đúng, khối lượng gần đúng và độ chính xác của bao gạo.
		\item Khối lượng thực của bao gạo nằm trong đoạn nào?
	\end{enumerate}
	\loigiai{
		\begin{enumerate}
			\item Không xác định được khối lượng đúng của bao gạo nhưng ta xem bao gạo có khối lượng là $ 5 $ kg. Do đó $ 5 $ là khối lượng gần đúng của bao gạo. Độ chính xác là $ d=0{,}2 $ kg.
			\item Khối lượng thực của bao gạo nằm trong đoạn $ [5-0{,}2;5+0{,}2]=[4{,}8;5{,}2] $.
		\end{enumerate}
		
	}
\end{bt}
%Bài 2
% \begin{bt}%[Mui Doan]%[0D1Y5-2]
% 	Một phép đo đường kính nhân tế bào cho kết quả là $ 5\pm 0{,}3\, \mu$m. Đường kính thực của nhân tế bào thuộc đoạn nào?
% 	\loigiai{
% 		Đường kính thực của nhân tế bào thuộc đoạn 	$ [5-0{,}3;5+0{,}3]=[4{,}7;5{,}3] $.
% 	}
% \end{bt}
%Bài 3
% \begin{bt}%[Mui Doan]%[0D1Y5-2]
% 	Chiều dài một cái cầu là $ \ell=1745{,}25 $m $ \pm 0{,}01 $m.
% 	\begin{enumerate}
% 		\item Xác định chiều dài đúng, chiều dài gần đúng và độ chính xác của của cái cầu.
% 		\item Chiều dài thực của cái cầu nằm trong đoạn nào?
% 	\end{enumerate}
% 	\loigiai{
% 		\begin{enumerate}
% 			\item Không xác định được chiều dài đúng của của cái cầu nhưng ta xem cái cầu có chiều dài  là $ 1745{,}25 $ m. Do đó $ 1745{,}25 $ là chiều dài gần đúng của cái cầu. Độ chính xác là $ d=0{,}01 $ m.
% 			\item Chiều dài thực của của cái cầu nằm trong đoạn $ [1745{,}25-0{,}01;1745{,}25+0{,}01]=[1745{,}24;1745{,}26] $.
% 		\end{enumerate}
% 	}
% \end{bt}
\begin{bt}%[Mui Doan]%[0D1Y5-2]
	Biết $ \sqrt{7}=2{,}6457513\ldots $
	\begin{enumerate}
		\item Làm tròn kết quả đến phần mười và ước lượng sai số tuyệt đối.
		\item Làm tròn kết quả đến phần nghìn và ước lượng sai số tuyệt đối.
	\end{enumerate}
	\loigiai{
		\begin{enumerate}
			\item Ta có $ \sqrt{7}\approx 2{,}6 $.\\
			Sai số tuyệt đối là $ \Delta_{\sqrt{7}}=\vert \sqrt{7}-2{,}6\vert < \vert 2{,}6457513-2{,}6\vert=0{,}0457513$.
			\item Ta có $ \sqrt{7}\approx 2{,}646 $.\\
			Sai số tuyệt đối là $ \Delta_{\sqrt{7}}=\vert \sqrt{7}-2{,}646\vert < \vert 2{,}6457513-2{,}646\vert=0{,}0002487$.
		\end{enumerate}	
	}
\end{bt}

\begin{dang}{Xác định sai số tương đối của số gần đúng}
\end{dang}
\viduminhhoa
% \begin{vd}%[Duong Xuan Loi]%[0D1B5-1]
% 	$a=3{,}14$ là số gần đúng của $\overline{a}=\pi$.\\
% 	Ta có $\Delta_{a}=|\pi-3{,}14|<|3{,}15-3{,}14|=0{,}01$.\\
% 	Ta nói $a=3{,}14$ là giá trị gần đúng của $\pi$ với độ chính xác $d=0{,}01$.
% \end{vd}
% \begin{vd}%[Duong Xuan Loi]%[0D1B5-1]
% 	Một bồn hoa có dạng hình tròn với bán kính là $0{,}8$ m. Hai bạn Ngân và Ánh cùng muốn tính diện tích $S$ của bồn hoa đó. Bạn Ngân lấy một giá trị gần đúng của $\pi$ là $3{,}1$ và được kết quả là $S_1$. Bạn Ánh lấy một giá trị gần đúng của $\pi$ là $3{,}14$ và được kết quả là $S_2$. So sánh sai số tuyệt đối $\Delta_{S_1}$ của số gần đúng $S_1$ và sai số tuyệt đối $\Delta_{S_2}$ của số gần đúng $S_2$. Bạn nào cho kết quả chính xác hơn?
% 	\loigiai{
% 		Ta có $S_1=3{,}1\cdot (0{,}8)^2=1{,}984\left(\mathrm{m}^2\right)$; 
% 		$S_2=3{,}14\cdot (0,8)^2=2{,}0096\left(\mathrm{m}^2\right)$.\\
% 		Ta thấy $3{,}1<3{,}14<\pi $ nên $3{,}1\cdot (0{,}8)^2<3{,}14\cdot (0{,}8)^2<\pi \cdot(0{,}8)^2$ tức là $S_1<S_2<S$.\\
% 		Suy ra $\Delta _{S_2}=\left|S-S_2\right|<\left|S-S_1\right|=\Delta _{S_1}$.\\
% 		Vậy bạn Ánh cho kết quả chính xác hơn.
% 	}
% \end{vd}
\begin{vd}%[Duong Xuan Loi]%[0D1B5-1]
	Một tờ giấy A$4$ có dạng hình chữ nhật với chiều dài, chiều rộng lần lượt là $29{,}7$ cm và $21$ cm. Tính độ dài đường chéo của tờ giấy A$4$ đó và xác định độ chính xác của kết quả tìm được.
	\loigiai{
		Gọi $x$ là độ dài đường chéo của tờ giấy A$4$ đã cho.\\
		Theo định lí Pythagore, ta có
		$x=\sqrt{29{,}7^2+21^2}=\sqrt{882{,}09+441}=\sqrt{1323{,}09}=36{,}3743 \ldots$\\
		Nếu lấy giá trị gần đúng của $x$ là $36,37$ ta có $36{,}37<x<36{,}375$.\\
		Suy ra $|x-36{,}37|<36{,}375-36{,}37=0{,}005$.\\
		Vậy độ dài đường chéo của tờ giấy A$4$ đã cho là $x \approx 36{,}37$ và độ chính xác của kết quả tìm được là $0{,}005$ hay nói cách khác $x=36{,}37 \pm 0,005$.
	}
\end{vd}

\begin{vd}%[Mui Doan]%[0D1B5-1]
	Cho số gần đúng $ a =6547$ với độ chính xác $ d=100 $.\\
	Hãy viết số quy tròn của số $ a $ và ước lượng sai số tương đối của số quy tròn đó.
	\loigiai{
		Số quy tròn là $  7000  $.
		Sai số tuyệt đối là $ \Delta_{6547}=\vert 6547-7000\vert =453$.\\
		Sai số tương đối là $ \delta_{6547}=\dfrac{\Delta_{6547}}{7000}=0{,}935285714$ .			
	}
\end{vd}

\baitaptl
\setcounter{bt}{0}
\begin{bt}%[Mui Doan]%[0D1B5-1]
	Ở Babylon, một tấm đất sét có niên đại khoảng $ 1900-1600 $ trước Công nguyên đã ghi lại một phát biểu hình học, trong đó ám chỉ ước lượng số $ \pi $ bằng $ \dfrac{25}{8} =3{,}1250$. Hãy ước lượng sai số tuyệt đối và sai số tương đối của giá trị gần đúng này, biết $ 3{,}141<\pi<3{,}142 $.
	\loigiai{
		Sai số tuyệt đối là $ \Delta_{\pi}=\left\vert \dfrac{25}{8}-\pi\right \vert <\vert 3{,}1250-3{,}141\vert =0{,}016$.	\\
		Sai số tương đối là $ \delta_{\pi}=\dfrac{\Delta_{\pi}}{3{,}1250}= 0{,}00512$.			
	}
\end{bt}

\begin{bt}%[Mui Doan]%[0D1B5-1]
	Cho số gần đúng $ a =23748023$ với độ chính xác $ d=101 $.\\
	Hãy viết số quy tròn của số $ a $ và ước lượng sai số tương đối của số quy tròn đó.
	\loigiai{
		Số quy tròn là $  23748000  $.
		Sai số tuyệt đối là $ \Delta_{23748023}=\vert 23748000-23748023\vert =23$.\\
		Sai số tương đối là $ \delta_{23748023}=\dfrac{\Delta_{23748023}}{23748000}=0{,}0000009685026107$ .			
	}
\end{bt}

\begin{bt}%[Mui Doan]%[0D1B5-1]
	Cho biết $\sqrt{3}=1{,}7320508\ldots$. Hãy quy tròn $ \sqrt{3} $ đến hàng phần trăm và ước lượng sai số tương đối.
	
	\loigiai{
		Số quy tròn là $  1{,}73  $.
		Sai số tuyệt đối là $ \Delta_{\sqrt{3}}=\vert \sqrt{3}-1{,}73\vert < \vert 1{,}7320508-1{,}73\vert=0{,}0020508$.\\
		Sai số tương đối là $ \delta_{\sqrt{3}}=\dfrac{\Delta_{\sqrt{3}}}{1{,}73}=0{,}0011854335526$ .		
	}
\end{bt}

% \begin{bt}%[Mui Doan]%[0D1B5-1]
% 	Cho $ \overline{a}=\dfrac{1}{1+x}, (0<x<1)$. Giả sử ta lấy $ a=1-x $ làm giá trị gần đúng của $ \overline{a} $. Hãy tính sai số tương đối của $ a $ theo $ x $.
% 	\loigiai{
% 		Ta có $ \Delta_{\overline{a}}=\dfrac{1}{1+x}-(1-x)=\dfrac{x^2}{1+x} $.\\
% 		Sai số tương đối là $ \delta_a=\dfrac{\Delta_{\overline{a}}}{1-x}=\dfrac{x^2}{1-x^2} $.	
% 	}
% \end{bt}

\begin{dang}{Xác định số quy tròn của số gần đúng với độ chính xác cho trước}
\end{dang}
\viduminhhoa
\begin{vd}%[Duong Xuan Loi]%[0D1B5-1]
	Quy tròn số $3{,}141$ đến hàng phần trăm rồi tính sai số tuyệt đối của số quy tròn.
	\loigiai{
		Khi quy tròn số $3{,}141$ đến hàng phần trăm ta được số $3{,}14$ và sai số tuyệt đối của số quy tròn là $|3{,}141-3{,}14|=0{,}001<0{,}005$. Do vậy $3{,}14$ là số gần đúng của $3{,}141$ với độ chính xác $0{,}005$.
	}
\end{vd}
\begin{vd}%[Duong Xuan Loi]%[0D1B5-1]
	\begin{enumerate}
		\item Làm tròn số $2395{,}3$ đến hàng chục, số $18{,}693$ đến hàng phần trăm và số đúng $d \in[5{,}5; 6{,}5)$ đến hàng đơn vị. Đánh giá sai số tuyệt đối của phép làm tròn số đúng $d$.
		\item Cho số gần đúng $a=2{,}53$ với độ chính xác $d=0{,}01$. Số đúng $\overline{a}$ thuộc đoạn nào? Nếu làm tròn số a thì nên làm tròn đến hàng nào? Vì sao?	
	\end{enumerate}
	\loigiai{
		\begin{enumerate}
			\item Số quy tròn của số $2395{,}3$ đến hàng chục là $2~400$; số quy tròn của số $18{,}693$ đến hàng phần trăm là $18{,}69$. Mọi số đúng $d \in[5{,}5; 6{,}5)$ khi làm tròn đến hàng đơn vị đều thu được số quy tròn là $6$ và sai số tuyệt đối $|d-6|\le 0{,}5$.
			\item Số đúng $\bar{a}$ thuộc đoạn $[2{,}53-0{,}01; 2{,}53+0{,}01]$ hay $[2{,}52; 2{,}54]$. Khi làm tròn số gần đúng $a$ ta nên làm tròn đến hàng phần chục do chữ số hàng phần trăm của $a$ là chữ số không chắc chắn đúng.
		\end{enumerate}
	}
\end{vd}
\begin{vd}%[Duong Xuan Loi]%[0D1B5-1]
	Cho số gần đúng $a=581~268$ với độ chính xác $d=200$. Hãy viết số quy tròn của số $a$.
	\loigiai{
		Vì độ chính xác đến hàng trăm $d=200$ nên ta làm tròn $a$ đến hàng nghìn theo quy tắc làm tròn ở trên. Số quy tròn của $a$ là $581~000$.
	}
\end{vd}
\begin{vd}%[Duong Xuan Loi]%[0D1B5-1]
	Viết số quy tròn của mỗi số sau vối độ chính xác $d$.
	\begin{listEX}[3]
		\item $2~841~331$ với $d=400$;
		\item $4{,}1463$ với $d=0{,}01$;
		\item $1{,}4142135$ với $d=0{,}001$.
	\end{listEX}
	\loigiai{
		\begin{listEX}[1]
			\item Vì độ chính xác $d=400$ thoả mãn $100<400<500$ nên ta quy tròn số $2~841~331$ đến hàng nghìn theo quy tắc ở trên.\\
			Vậy số quy tròn của số $2~841~331$ với độ chính xác $d=400$ là $2~841~000$.
			\item Vì độ chính xác $d=0{,}01$ thoả mãn $0{,}01<0{,}05$ nên ta quy tròn số $4{,}1463$ đến hàng phần mười theo quy tắc ở trên.\\
			Vậy số quy tròn của số $4{,}1463$ với độ chính xác $d=0{,}01$ là $4{,}1$.
			\item Vì độ chính xác $d=0{,}001$ thoả mãn $0{,}001<0{,}005$ nên ta quy tròn số $1{,}4142135$ đến hàng phần trăm theo quy tắc ở trên.\\
			Vậy số quy tròn của số $1{,}4142135 \ldots$ với độ chính xác $d=0{,}001$ là $1{,}41$.
		\end{listEX}
	}
\end{vd}
\baitaptl
	\setcounter{bt}{0}
% % Bai 1
% \begin{bt}%[Mui Doan]%[0D1Y5-2]
% 	Làm tròn các số sau đến chữ số hàng chục
% 	\begin{listEX}[5]
% 		\item $ 199 $.
% 		\item $ 999 $.
% 		\item $ 9999 $.
% 		\item $ 2683 $.
% 		\item $ 1099 $.
% 		\item $ 12345 $.
% 		\item $ 123456 $.
% 		\item $ 43781 $.
% 		\item $ 454995 $.
% 		\item $ 14350 $.
% 		\item $ 99999 $.
% 		\item $ 987698 $.
% 		\item $ 3400065 $.
% 		\item $ 1000587 $.
% 		\item $ 987654 $.
% 		\item $ 28051989 $.
% 		\item $ 2602283 $.
% 		\item $ 123{,}45 $.
% 		\item $ 12345{,}67 $.
% 		\item $ 98765{,}432 $.
% 	\end{listEX}
	
% 	\loigiai{
% 		\begin{listEX}[3]
% 			\item $ 199\approx 200 $.
% 			\item $ 999 \approx1000$.
% 			\item $ 9999 \approx 10~000$.
% 			\item $ 2683 \approx 2680$.
% 			\item $ 1099 \approx 1100$.
% 			\item $ 12~345 \approx 12~350$.
% 			\item $ 123~456 \approx 123~460$.
% 			\item $ 43~781 \approx 43~780$.
% 			\item $ 454~995 \approx 455~000$.
% 			\item $ 14~350 \approx 14~350$.
% 			\item $ 99~999 \approx 100~000$.
% 			\item $ 987~698 \approx 987~700$.
% 			\item $ 3~400~065 \approx 3~400~070$.
% 			\item $ 1~000~587 \approx 1~000~590$.
% 			\item $ 987~654 \approx 987~650$.
% 			\item $ 28~051~989 \approx 28~051~990$.
% 			\item $ 2~602~283 \approx 2~602~280$.
% 			\item $ 123{,}45 \approx 120$.
% 			\item $ 12345{,}67 \approx 12~350$.
% 			\item $ 98765{,}432 \approx 98~770$.
% 		\end{listEX}	
% 	}
% \end{bt}
% % Bài 2
% \begin{bt}%[Mui Doan]%[0D1Y5-2]
% 	Làm tròn các số sau đến chữ số hàng trăm
% 	\begin{listEX}[5]
% 		\item $ 199 $.
% 		\item $ 999 $.
% 		\item $ 9999 $.
% 		\item $ 1099 $
% 		\item $ 2683 $.
% 		\item $ 12345 $.
% 		\item $ 43781 $.
% 		\item $ 14350 $.
% 		\item $ 1234567 $.
% 		\item $ 454995 $.
% 		\item $ 99999 $.
% 		\item $ 987698 $.
% 		\item $ 3400065 $.
% 		\item $ 987654 $.
% 		\item $ 260283 $.
% 		\item $ 23456{,}7 $.
% 		\item $ 12345{,}678 $.
% 		\item $ 8765{,}432 $.
% 		\item $ 9999{,}99 $.
% 	\end{listEX}
% 	\loigiai{
% 		\begin{listEX}[3]
% 			\item $ 199 \approx 200$.
% 			\item $ 999 \approx 1000$.
% 			\item $ 9999 \approx 10~000$.
% 			\item $ 1099 \approx 1100$
% 			\item $ 2683 \approx 2700$.
% 			\item $ 12~345 \approx 12~300$.
% 			\item $ 43~781 \approx 43~800$.
% 			\item $ 14~350 \approx 14~400$.
% 			\item $ 1~234~567 \approx 1~234~600$.
% 			\item $ 454~995 \approx 455~000$.
% 			\item $ 99~999 \approx 100~000$.
% 			\item $ 987~698 \approx 987~700$.
% 			\item $ 3~400~065 \approx 3~400~100$.
% 			\item $ 987~654 \approx 987~700$.
% 			\item $ 260~283 \approx 260~300$.
% 			\item $ 23456{,}7 \approx 23~500$.
% 			\item $ 12345{,}678 \approx 12~300$.
% 			\item $ 8765{,}432 \approx 8800$.
% 			\item $ 9999{,}99 \approx 10~000$.
% 		\end{listEX}	
% 	}
% \end{bt}
% \begin{bt}%[Mui Doan]%[0D1Y5-2]
% 	Làm tròn các số sau đến chữ số hàng nghìn
% 	\begin{listEX}[5]
% 		\item $ 12~345 $.
% 		\item $ 43~781 $.
% 		\item $ 28~634 $.
% 		\item $ 21~999 $.
% 		\item $ 22~999 $.
% 		\item $ 9999 $.
% 		\item $ 12~099 $.
% 		\item $ 454~995 $.
% 		\item $ 14~350 $.
% 		\item $ 99~999 $.
% 		\item $ 987~698 $.
% 		\item $ 3~400~065 $.
% 		\item $ 1~000~587 $.
% 		\item $ 987~654 $.
% 		\item $ 260~283 $.
% 		\item $ 23456{,}7 $.
% 		\item $ 1~234~567 $
% 		\item $ 12345{,}678 $.
% 		\item $ 8765{,}432 $.
% 		\item $ 9999{,}99 $.
% 	\end{listEX}
% 	\loigiai{
% 		\begin{listEX}[3]
% 			\item $ 12~345 \approx 12~000$.
% 			\item $ 43~781 \approx 44~000$.
% 			\item $ 28~634 \approx 29~000$.
% 			\item $ 21~999 \approx 22~000$.
% 			\item $ 22~999 \approx 23~000$.
% 			\item $ 9999 \approx 10~000$.
% 			\item $ 12~099 \approx 12~000$.
% 			\item $ 454~995 \approx 455~000$.
% 			\item $ 14~350 \approx 14~000$.
% 			\item $ 99~999 \approx 100~000$.
% 			\item $ 987~698 \approx 988~000$.
% 			\item $ 3~400~065 \approx 3~400~000$.
% 			\item $ 1~000~587 \approx 1~001~000$.
% 			\item $ 987~654 \approx 988~000$.
% 			\item $ 260~283 \approx 260~000$.
% 			\item $ 23456{,}7 \approx 23~000$.
% 			\item $ 1~234~567 \approx 1~235~000$
% 			\item $ 12345{,}678 \approx 12~000$.
% 			\item $ 8765{,}432 \approx 9000$.
% 			\item $ 9999{,}99 \approx 10~000$.
% 		\end{listEX}	
% 	}
% \end{bt}
% \begin{bt}%[Mui Doan]%[0D1Y5-2]
% 	Làm tròn các số sau đến hàng phần mười
% 	\begin{listEX}[5]
% 		\item $ 10{,}00905 $.
% 		\item $ 60{,}991 $.
% 		\item $ 999{,}994 $.
% 		\item $ 10{,}0456 $.
% 		\item $ 23{,}0009 $.
% 		\item $ 99{,}999 $.
% 		\item $ 90{,}0909 $.
% 		\item $ 9876{,}1 $.
% 		\item $ 1234{,}56 $.
% 		\item $ 98765{,}43 $.
% 	\end{listEX}	
% 	\loigiai{
% 		\begin{listEX}[3]
% 			\item $ 10{,}00905 \approx 10$.
% 			\item $ 60{,}991 \approx 61$.
% 			\item $ 999{,}994 \approx 1000$.
% 			\item $ 10{,}0456 \approx 10$.
% 			\item $ 23{,}0009 \approx 23$.
% 			\item $ 99{,}999 \approx 100$.
% 			\item $ 90{,}0909 \approx 90{,}1$.
% 			\item $ 9876{,}1 \approx 9876{,}1$.
% 			\item $ 1234{,}56 \approx 1234{,}6$.
% 			\item $ 98765{,}43 \approx 98765{,}4$.
% 		\end{listEX}		
% 	}
% \end{bt}
% \begin{bt}%[Mui Doan]%[0D1Y5-2]
% 	Làm tròn các số sau đến hàng phần trăm
% 	\begin{listEX}[5]
% 		\item $ 3{,}0468 $.
% 		\item $ 12{,}3475 $.
% 		\item $ 0{,}31069 $.
% 		\item $ 12{,}516 $.
% 		\item $ 0{,}999 $.
% 		\item $ 7{,}923 $.
% 		\item $ 17{,}418 $.
% 		\item $ 79{,}1364 $.
% 		\item $ 50{,}401 $.
% 		\item $ 0{,}155 $.
% 		\item $ 60{,}996 $.
% 		\item $ 12{,}349 $.
% 		\item $ 2{,}9999 $.
% 		\item $ 123{,}456 $.
% 		\item $ 98{,}7654 $.
% 	\end{listEX}	
	
% 	\loigiai{
% 		\begin{listEX}[3]
% 			\item $ 3{,}0468 \approx 3{,}05$.
% 			\item $ 12{,}3475 \approx 12{,}35$.
% 			\item $ 0{,}31069 \approx 0{,}31$.
% 			\item $ 12{,}516 \approx 12{,}52$.
% 			\item $ 0{,}999 \approx 1$.
% 			\item $ 7{,}923 \approx 7{,}92$.
% 			\item $ 17{,}418 \approx 17{,}42$.
% 			\item $ 79{,}1364 \approx 79{,}14$.
% 			\item $ 50{,}401 \approx 50{,}40$.
% 			\item $ 0{,}155 \approx 0{,}16$.
% 			\item $ 60{,}996 \approx 61$.
% 			\item $ 12{,}349 \approx 12{,}35$.
% 			\item $ 2{,}9999 \approx 3$.
% 			\item $ 123{,}456 \approx 123{,}46$.
% 			\item $ 98{,}7654 \approx 98{,}77$.
% 		\end{listEX}	
		
% 	}
% \end{bt}
% Bài 6
\begin{bt}%[Mui Doan]%[0D1B5-1]
	Viết số quy tròn của mỗi số sau với độ chính xác $ d $
	\begin{enumerate}
		\item $1~234~567$ với $ d=400 $.
		\item $ 8{,}7654 $ với $ d=0{,} 01$.
		\item $ 28{,}4156 $ với $ d=0{,}001$.
		\item $ 1{,}7320508\ldots $ với $ d=0{,}0001$.
	\end{enumerate}	
	\loigiai{
		\begin{enumerate}
			\item Vì độ chính xác $ d=400 $ thỏa mãn $ 100<400<500 $ nên ta quy tròn số  $1~234~567$ đến hàng nghìn.\\
			Vậy số quy tròn của số $1~234~567$ với độ chính xác $ d=400 $ là $1~235~000$.
			\item Vì độ chính xác $ d=0{,}01 $ thỏa mãn $ 0{,}01<0{,}05 $ nên ta quy tròn số  $8~7654$ đến hàng phần mười.\\
			Vậy số quy tròn của số $8~7654$ với độ chính xác $ d=0{,}01 $ là $8{,}8$.
			\item Vì độ chính xác $ d=0{,}001 $ thỏa mãn $ 0{,}001<0{,}005 $ nên ta quy tròn số  $28{,}4156$ đến hàng phần trăm.\\
			Vậy số quy tròn của số $28{,}4156$ với độ chính xác $ d=0{,}001 $ là $28{,}42$.
			\item Vì độ chính xác $ d=0{,}0001 $ thỏa mãn $ 0{,}0001<0{,}0005 $ nên ta quy tròn số  $1{,}7320508\ldots$ đến hàng phần nghìn.\\
			Vậy số quy tròn của số $1{,}7320508\ldots$ với độ chính xác $ d=0{,}0001 $ là $1{,}732$.
		\end{enumerate}
	}	
\end{bt}
\begin{bt}%[Mui Doan]%[0D1B5-1]
	Hãy viết số quy tròn của 
	\begin{enumerate}
		\item $ a $ biết $ \overline{a}=1~951~890\pm 200 $.
		\item $ b $ biết $ \overline{b}=1{,}236\pm 0{,}002 $.
		\item $ c $ biết $ \overline{c}=3{,}1463\pm 0{,}002 $.
	\end{enumerate}
	\loigiai{
		\begin{enumerate}
			\item $ a $ biết $ \overline{a}=1~951~890\pm 200 $.\\
			Vì $ d=200 $ thỏa mãn $ 100<200<500 $ nên ta quy tròn $ 1~951~890 $ đến hàng nghìn.\\
			Vậy $ a\approx 1~952~000 $.
			\item $ b $ biết $ \overline{b}=1{,}236\pm 0{,}002 $.
			Vì độ chính xác $ d=0{,}002 $ thỏa mãn $ 0{,}002<0{,}005 $ nên ta quy tròn số  $1{,}236$ đến hàng phần trăm.\\
			Vậy $ b\approx 1{,}24 $.
			\item $ c $ biết $ \overline{c}=3{,}1463\pm 0{,}002 $.
			Vì độ chính xác $ d=0{,}002 $ thỏa mãn $ 0{,}002<0{,}005 $ nên ta quy tròn số  $3{,}1463$ đến hàng phần trăm.\\
			Vậy $ c\approx 3{,}15 $.
		\end{enumerate}
	}
\end{bt}
\begin{bt}%[Mui Doan]%[0D1B5-1]
	Chiều dài một cái cầu là $ \ell =1745{,}25 $ m $ \pm 0{,}01 $ m. Hãy viết số quy tròn của số gần đúng $ 1745{,}25 $.
	\loigiai{
		Vì độ chính xác $ d=0{,}01 $ thỏa mãn $ 0{,}01<0{,}05 $ nên ta quy tròn số  $1745{,}25$ đến hàng phần mười.\\
		Vậy số quy tròn của số $1745{,}25$ với độ chính xác $ d=0{,}01 $ là $1745{,}3$.
	}
\end{bt}
\begin{dang}{Sử dụng máy tính cầm tay để tính toán với số gần đúng}	
\end{dang}
\viduminhhoa
\begin{vd}%[Duong Xuan Loi]%[0D1B5-1]
	Sử dụng máy tính cầm tay, tính $3^{7} \cdot \sqrt{14}$ (trong kết quả lấy bốn chữ số ở phần thập phân).
	\loigiai{
		Ấn liên tiếp \key{3}\key{D}\key{7}\key{\$}\key{O}\key{s}\key{1}\key{4}\key{=}\\
		Tiếp tục ấn lần lượt \shiftk\menuk\key{3}\key{1} thì màn hình hiện ra Fix $0 \sim 9$?\\
		Ấn tiếp \key{4}\key{=} để lấy bốn chữ số thập phân. Kết quả hiện ra màn hình là $8183.0047$.
	}
\end{vd}
% \begin{vd}%[Duong Xuan Loi]%[0D1B5-1]
% 	Dùng máy tính cầm tay, tính kết quả của phép tính $\sqrt[3]{15}\,\colon 5-2$ (trong kết quả lấy hai chữ số ở phần thập phân).
% 	\loigiai{
% 		Ấn liên tiếp \shiftk\sqrtk\key{1}\key{5}\key{\$}\divk\key{5}\subk\key{2}\key{=}\\
% 		Tiếp tục ấn lần lượt \shiftk\menuk\key{3}\key{1} thì màn hình hiện ra Fix $0 \sim 9$?\\
% 		Ấn tiếp \key{2}\key{=} để lấy hai chữ số thập phân. Kết quả hiện ra màn hình là $-1.51$.
% 	}
% \end{vd}
\begin{vd}%[Duong Xuan Loi]%[0D1B5-1]
	Gọi $P$ là chu vi của đường tròn bán kính $1$cm. Hãy tìm giá trị gần đúng của $P$ (trong kết quả lấy hai chữ số ở phần thập phân).
	\loigiai{
		Ấn liên tiếp \key{2}\mulk\shiftk \mul10expk \key{=}\\
		Tiếp tục ấn lần lượt \shiftk\menuk\key{3}\key{1} thì màn hình hiện ra Fix $0 \sim 9$?\\
		Ấn tiếp \key{2}\key{=}\convertk để lấy hai chữ số thập phân. Kết quả hiện ra màn hình là $6.28$.
	}
\end{vd}
\baitaptl
	\setcounter{bt}{0}
\begin{bt}%[Mui Doan]%[0D1B5-1]
	Sử dụng máy tính bỏ túi tính gần đúng các số sau (kết quả lấy $ 4 $ chữ số thập phân).
	\begin{listEX}[3]
		\item $ 3^7\cdot \sqrt{14} $.
		\item $ \sqrt[3]{15}\cdot 12^4 $.
		\item $ \sqrt[3]{15}\cdot 14^4 $.
	\end{listEX}
	\loigiai{
		\begin{enumerate}
			\item Dùng máy tính cầm tay ta có $ 3^7\cdot \sqrt{14}\approx 8183{,}004705 $.\\
			Vậy số gần đúng của $ 3^7\cdot \sqrt{14} $ với $ 4 $ chữ số thập phân là $ 8183{,}0047 $.
			\item Dùng máy tính cầm tay ta có $ \sqrt[3]{15}\cdot 12^4 \approx 51139{,}37357 $.\\
			Vậy số gần đúng của $ \sqrt[3]{15}\cdot 12^4 $ với $ 4 $ chữ số thập phân là $ 51139{,}3736 $.
			\item Dùng máy tính cầm tay ta có $ \sqrt[3]{15}\cdot 14^4 \approx 94742{,}00305$.\\
			Vậy số gần đúng của $ \sqrt[3]{15}\cdot 14^4 $ với $ 4 $ chữ số thập phân là $ 94742{,}0031 $.
		\end{enumerate}
	}
\end{bt}

% \begin{bt}%[Mui Doan]%[0D1B5-1]
% 	Thực hiện các phép tính sau trên máy tính cầm tay (trong kết lấy $ 4 $ chữ số ở phần thập phân)
% 	\begin{listEX}[3]
% 		\item $4^6\cdot \sqrt{0{,}1}$.
% 		\item $ \sqrt[8]{2{,}1^{18}+1}- \sqrt{2{,}1^{12}+1}$.
% 		\item $ \dfrac{1{,}5^3}{\sqrt[3]{6{,}8}} $.
% 	\end{listEX}
% 	\loigiai{
% 		\begin{enumerate}
% 			\item Dùng máy tính cầm tay ta có $4^6\cdot \sqrt{0{,}1} \approx 1295{,}26893 $.\\
% 			Vậy số gần đúng của $ 4^6\cdot \sqrt{0{,}1} $ với $ 4 $ chữ số thập phân là $ 1295{,}2689 $.
% 			\item Dùng máy tính cầm tay ta có $\sqrt[8]{2{,}1^{18}+1}- \sqrt{2{,}1^{12}+1} \approx -80{,}46318563 $.\\
% 			Vậy số gần đúng của $\sqrt[8]{2{,}1^{18}+1}- \sqrt{2{,}1^{12}+1} $ với $ 4 $ chữ số thập phân là $ -80{,}4632 $.
% 			\item Dùng máy tính cầm tay ta có $\dfrac{1{,}5^3}{\sqrt[3]{6{,}8}}  \approx 1{,}781438386$.\\
% 			Vậy số gần đúng của $ \dfrac{1{,}5^3}{\sqrt[3]{6{,}8}} $ với $ 4 $ chữ số thập phân là $1{,}7814 $.
% 		\end{enumerate}	
% 	}
% \end{bt}
\subsection{Câu hỏi trắc nghiệm}

\Opensolutionfile{ansbook}[ans/ansbook-0D5-1-TN]
\Opensolutionfile{ans}[ans/ans-0D5-1-TN]
\begin{ex}%[Du an Bai giang Toan 10-2022]%[Dao-V- Thuy]%[0D1Y5-1]
	Cho $a$ là số gần đúng của số đúng $\overline{a}$. Khi đó $\Delta_a = |\overline{a} - a|$ được gọi là 
	\choice{số quy tròn của $\overline{a}$}
	{sai số tương đối của số gần đúng $a$}
	{\True sai số tuyệt đối của số gần đúng $a$}
	{số quy tròn của $a$}
	\loigiai{$\Delta_a = |\overline{a} - a|$ được gọi là sai số tuyệt đối của số gần đúng $a$.}
\end{ex}

\begin{ex}%[Du an Bai giang Toan 10-2022]%[Dao-V- Thuy]%[0D1Y5-1]
	Cho số $a$ là số gần đúng của số $\overline{a}$. Mệnh đề nào sau đây là mệnh đề đúng?
	\choice
	{$a>\overline{a}$}               
	{$a<\overline{a}$}
	{\True $|\overline{a}-a|>0$}
	{$-a<\overline{a}<a$}
	\loigiai{
		Do $a$ là số gần đúng của số $\overline{a}$ nên $a>\overline{a}$ hoặc $a<\overline{a}$.\\
		Suy ra $|\overline{a}-a|>0$.
	}
\end{ex}

\begin{ex}%[Du an Bai giang Toan 10-2022]%[Dao-V- Thuy]%[0D1Y5-1]
	Cho số $a$ là số gần đúng của $\overline{a}$ với độ chính xác $d$. Mệnh đề nào sau đây là mệnh đề đúng?
	\choice
	{$\overline{a}=a+d$}
	{$\overline{a}=a-d$}
	{$\overline{a}=a$}	
	{\True $\overline{a}=a\pm d$}
	\loigiai{
		Nếu $a$ là số gần đúng của $\overline{a}$ với độ chính xác $d$ thì $\overline{a}=a\pm d$.
	}
\end{ex}

\begin{ex}%[Du an Bai giang Toan 10-2022]%[Dao-V- Thuy]%[0D1Y5-1]
	Kết quả làm tròn số $b=500\sqrt{7}$ đến chữ số thập phân thứ hai là
	\choice
	{\True $b\approx 132,88$}
	{$b\approx 1322,87$}
	{$b\approx 1322,8 $}
	{$b\approx 1322,9 $}
	\loigiai{
		Có $b=500\sqrt{7}\approx 1322{,}875656$.\\
		Do làm tròn đến chữ số thập phân thứ hai nên ta có $b\approx 1322{,}88$.
	}
\end{ex}

\begin{ex}%[Du an Bai giang Toan 10-2022]%[Dao-V- Thuy]%[0D1Y5-1]
	Kết quả làm tròn của số $c=76324753{,}3695$ đến  hàng nghìn là
	\choice
	{$c\approx 76324000$}
	{\True $c\approx 76325000$}
	{$c\approx 76324753{,}369$}
	{$c\approx 76324753{,}37$}
	\loigiai{
		Làm tròn đến hàng nghìn của $c=76324753{,}3695$ ta được $c\approx 76325000$.
	}
\end{ex}

\begin{ex}%[Du an Bai giang Toan 10-2022]%[Dao-V- Thuy]%[0D1Y5-1]
	Viết số quy tròn của số gần đúng $a=505360{,}996$ biết $\overline{a}=505360{,}996\pm 100$.
	\choice
	{$a\approx 505$}
	{$a\approx 5054$}
	{$a\approx 505400$}
	{\True $a\approx 505000$}
	\loigiai{
		Do $\overline{a}=505360{,}996\pm 100$ nên ta làm tròn đến hàng nghìn.\\
		Suy ra $a\approx 505000$.
	}
\end{ex}

% \begin{ex}%[Du an Bai giang Toan 10-2022]%[Dao-V- Thuy]%[0D1Y5-1]
% 	Viết số quy tròn số gần đúng $b=3257{,}6254$ với độ chính xác $d=0{,}01$.
% 	\choice
% 	{$b\approx 3257{,}63$}
% 	{$b\approx 3257{,}62$}
% 	{\True $b\approx 3257{,}6$}
% 	{$b\approx 3257{,}7$}
% 	\loigiai{
% 		Do $d=0,01$ nên ta làm tròn đến hàng phần đơn vị. Do đó $b\approx 3257{,}6$.
% 	}
% \end{ex}

% \begin{ex}%[Du an Bai giang Toan 10-2022]%[Dao-V- Thuy]%[0D1Y5-1]
% 	Cho giá trị gần đúng của số $\pi$ là $x=3{,}141592653589$ với độ chính xác $10^{-10}$. Hãy viết số quy tròn của $x$.
% 	\choice
% 	{\True $x\approx 3{,}141592654$}
% 	{$x\approx 3{,}1415926535$}
% 	{$x\approx 3{,}1415926536$}
% 	{$x\approx 3{,}141592653$}
% 	\loigiai{
% 		Do độ chính xác là $10^{-10}$ nên ta làm tròn đến chữ số thứ $9$ sau dấu $,$.
% 	}
% \end{ex}

% \begin{ex}%[Du an Bai giang Toan 10-2022]%[Dao-V- Thuy]%[0D1Y5-1]
% 	Cho  $ \overline{a} = 1{,}7059\pm 0{,}001$, kết quả làm tròn số $ a = 1{,}7059 $ là
% 	\choice
% 	{\True $ 1{,}71$}
% 	{ $ 1{,}706 $}
% 	{ $ 1{,}7 $}
% 	{ $ 1{,}705 $}
% 	\loigiai{
% 		Do $d=0{,}001$ nên ta làm tròn đến hàng phần trăm, do đó $a\approx1{,}71$.
% 	}
% \end{ex}

% \begin{ex}%[Du an Bai giang Toan 10-2022]%[Dao-V- Thuy]%[0D1Y5-1]
% 	Cho $ \overline{a} = 123564\pm 100$. Kết quả làm tròn số $ x = 123564 $ là
% 	\choice
% 	{$12360 $}
% 	{ $ 123000 $}
% 	{ $ 123570 $}
% 	{\True  $ 124000 $}
% 	\loigiai{
% 		Do $d=100$ nên ta làm tròn đến chữ số hàng nghìn.
% 	}
% \end{ex}

% \begin{ex}%[Du an Bai giang Toan 10-2022]%[Dao-V- Thuy]%[0D1B5-1]
% 	Số gần đúng $a = 173{,}4592$ có sai đố tuyệt đối không vượt quá $0{,}01$. Số quy tròn của $a$ là 
% 	\choice{$173{,}45$}
% 	{$173{,}46$}
% 	{\True $173{,}5$}
% 	{$173$}
% 	\loigiai{Số quy tròn của $a$ là $173{,}5$.}
% \end{ex}

% \begin{dang}
% 	{Xác định sai số của số gần đúng}
% \end{dang}

% \begin{ex}%[Du an Bai giang Toan 10-2022]%[Dao-V- Thuy]%[0D1B5-2]
% 	Trong các số dưới đây, giá trị gần đúng của $\sqrt{30}-5$ với sai số tuyệt đối bé nhất là
% 	\choice
% 	{$0{,}476$}
% 	{\True $0{,}477$}
% 	{$0{,}478$}
% 	{$0{,}479$}
% 	\loigiai{Dùng máy tính cầm tay ta tính được giá trị gần đúng của $\sqrt{30}-5$ là $0{,}477225$ nên giá trị gần đúng của nó với sai số tuyệt đối bé nhất trong bốn đáp án trên là $0{,}477$. }
% \end{ex}

% \begin{ex}%[Du an Bai giang Toan 10-2022]%[Dao-V- Thuy]%[0D1B5-2]
% 	Nếu lấy $3{,}14$ làm giá trị gần đúng cho số $\pi$ thì sai số tuyệt đối không vượt quá  
% 	\choice{\True $ 0{,}01$}
% 	{$0{,}02$}
% 	{$0{,}03$}
% 	{$0{,}04$}
% 	\loigiai{
% 		Ta có $\pi = 3{,}141592654...$.\\
% 		Do $3{,}14 < \pi = 3{,}141592654... < 3{,}15$ nên ta có $\Delta = \left|\pi - 3{,}14 \right| < |3{,}15 - 3{,}14| = 0{,}01$.
% 	}
	
% \end{ex}
% \begin{ex}%[Du an Bai giang Toan 10-2022]%[Dao-V- Thuy]%[0D1B5-2]
% 	Nếu lấy $3,1416$ làm giá trị gần đúng cho $\pi$ thì sai số tuyệt đối không vượt quá 
% 	\choice{$0, 0002$}
% 	{$0, 0003$}
% 	{\True $0, 0001$}
% 	{$0, 0004$}
% 	\loigiai{Ta có $\pi = 3, 141592654...$. Do $3, 1415 < \pi = 3, 141592654... < 3, 1416$ nên ta có 
% 		\begin{equation*}
% 			\Delta = |3,1416 - \pi| < |3,1416 - 3, 1415| = 0, 0001.
% 		\end{equation*}
% 	}
% \end{ex}

% \begin{ex}%[Du an Bai giang Toan 10-2022]%[Dao-V- Thuy]%[0D1B5-2]
% 	Cho giá trị gần đúng của $\dfrac{8}{17}$ là $0{,}47$ thì sai số tuyệt đối không vượt quá 
% 	\choice{\True $0{,}01$}
% 	{$0{,}02$}
% 	{$0{,}03$}
% 	{$0{,}04$}
% 	\loigiai{
% 		Ta có $\dfrac{8}{17} = 0{,}4705882...$. Do $0{,}47 < \dfrac{8}{17} = 0{,}4705882... < 0{,}48$ nên 
% 		\begin{equation*}
% 			\Delta = \left|\dfrac{8}{17} - 0{,}47 \right| < |0{,}48 - 0{,}47| = 0{,}01. 
% 		\end{equation*}
% 	}
% \end{ex}

\begin{ex}%[Du an Bai giang Toan 10-2022]%[Dao-V- Thuy]%[0D1B5-2]
	Cho giá trị gần đúng của $\dfrac{3}{7}$ là $0{,}429$ thì sai số tuyệt đối không vượt quá 
	\choice{$0{,}002$}
	{\True $0{,}001$}
	{$ 0{,}003$}
	{$0{,}004$}
	\loigiai{
		Ta có $\dfrac{3}{7} = 0{,}428571...$.\\
		Do $0{,}428 < \dfrac{3}{7} = 0{,}428571... < 0{,}429$ nên $\Delta = \left|0, 429 - \dfrac{3}{7}\right| < |0{,}429 - 0{,}428| = 0{,}001$.  
	}
\end{ex}

\begin{ex}%[Du an Bai giang Toan 10-2022]%[Dao-V- Thuy]%[0D1K5-2]
	Một vật có thể tích $V = 180{,}37 \text{\ cm}^3 \pm 0,05 \text{\ cm}^3$. Nếu lấy $180{,}37 \text{\ cm}^3$ làm giá trị gần đúng cho $V$ thì sai số tương đối của giá trị gần đúng đó không vượt quá 
	\choice{\True $0{,}03 \%$}
	{$0{,}01 \%$}
	{$0{,}02 \%$}
	{$0{,}001 \%$}
	\loigiai{Ta có $\delta = \dfrac{\Delta}{|V|} \leq \dfrac{d}{|V|} = \dfrac{0,05}{180{,}37} \approx 0{,}03 \%$. 
	}
\end{ex}

\begin{ex}%[Du an Bai giang Toan 10-2022]%[Dao-V- Thuy]%[0D1K5-2]
	Số $\overline{a}$ được cho bởi giá trị gần đúng $a = 5{,}7824$ với sai số tương đối không vượt quá $0{,}05\%$. Khi đó, sai số tuyệt đối của $a$ không vượt quá
	\choice{\True $0{,}0028912$}
	{$0{,}0027912$}
	{$0{,}0026912$}
	{$0{,}0025912$}
	\loigiai{Ta có $\delta_a = \dfrac{\Delta_a}{|a|} \leq 0{,}0005 \Rightarrow \Delta_a \leq 0{,}0005 \cdot 5{,}7824 = 0{,}0028912$.
	}
\end{ex}

\begin{ex}%[Du an Bai giang Toan 10-2022]%[Dao-V- Thuy]%[0D1K5-2]
	Cho $\overline{a} = \dfrac{1}{1+x}$ $(0 < x < 1)$. Giả sử ta lấy $a = 1- x$ làm giá trị gần đúng của $\overline{a}$. Khi đó, sai số tương đối của $a$ theo $x$ bằng
	\choice{\True $\dfrac{x^2}{1- x^2}$}
	{$\dfrac{x}{1-x}$}
	{$\dfrac{x^2}{1-x}$}
	{$\dfrac{x}{1-x^2}$}
	\loigiai{Ta có $\Delta_{\overline{a}} = \left|\dfrac{1}{1+x} - (1-x)\right| = \dfrac{x^2}{1+x}$.
		Vậy sai số tương đối là $\delta_a = \dfrac{\Delta_{\overline{a}}}{|1-x|} = \dfrac{x^2}{1- x^2}.$
	}
\end{ex}

\begin{ex}%[Du an Bai giang Toan 10-2022]%[Dao-V- Thuy]%[0D1K5-2]
	Các nhà toán học cổ đại Trung Quốc đã dùng phân số $\dfrac{22}{7}$ để xấp xỉ số $\pi$. Hãy đánh giá sai số tuyệt đối $\Delta$ của giá trị gần đúng này, biết $3\text{,}1415<\pi<3\text{,}1416$.
	\choice{$\Delta<0{,}0012$}
	{\True$\Delta<0{,}0014$}
	{$\Delta<0{,}0013$}
	{$\Delta<0{,}0011$}
	\loigiai{Ta có $\dfrac{22}{7}<3{,}1429$ và $-\pi<-3{,}1415$ nên $\Delta=\left|\pi-\dfrac{22}{7}\right|=\dfrac{22}{7}-\pi<3{,}1429-3{,}1415=0{,}0014$.}
\end{ex}

\begin{ex}%[Du an Bai giang Toan 10-2022]%[Dao-V- Thuy]%[0D1K5-2]
	Hình chữ nhật có các cạnh là $x = 2 \mathrm{\, m} \pm 1 \mathrm{\, cm}$ và $y = 5 \mathrm{\, m} \pm 2 \mathrm{\, cm}$. Diện tích của hình chữ nhật và sai số tương đối của giá trị đó là
	\choice
	{$10 \mathrm{\, m}^2$ và $\delta \leq 0{,}91 \%$}
	{\True $10 \mathrm{\, m}^2$ và $\delta \leq 0{,}9 \%$}
	{$10 \mathrm{\, m}^2$ và $\delta \leq 0{,}92 \%$}
	{$10 \mathrm{\, m}^2$ và $\delta \leq 0{,}93 \%$}
	\loigiai{Ta có $1{,}99 \mathrm{\, m} \leq x \leq 2{,}01 \mathrm{\, m}$ và $4{,}98 \mathrm{\, m} \leq y \leq 5{,} 02 \mathrm{\, m}$.
		\noindent Khi đó $S = x \cdot y \Rightarrow 9{,}9102\mathrm{\, m}^2 \leq S \leq 10{,}0902 \mathrm{\, m}^2 \Rightarrow S = 10 \mathrm{\, m}^2 \pm 0{,}0902 \mathrm{\, m}^2$.
		\noindent Sai số tương đối là $\delta_S \leq \dfrac{0{,}0902}{10} \approx 0{,}9 \%$.
	}
\end{ex}
\Closesolutionfile{ans}
\Closesolutionfile{ansbook}
% \indapan{10}{ans/ans-0D5-1-TN}
