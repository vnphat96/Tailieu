\section{MẠNG XÃ HỘI: LỢI VÀ HẠI}
\subsection{Tóm tắt lý thuyết}
Những vấn đề bạn quan tâm là
\begin{enumerate}
	\item Lợi ích, bất lợi lớn nhất khi dùng mạng xã hội là gì?
	\item Thời gian sử dụng mạng xã hội của các bạn trong lớp như thế nào?
	\item Các bạn nam và bạn nữ có thời gian sử dụng mạng xã hội khác nhau nhiều không?
\end{enumerate}
\subsubsection{Phiếu khảo sát thu thập dữ liệu}
Các bạn trong nhóm đã lập một phiếu khảo sát để thu thập dữ liệu như sau:
%\begin{tomtat}
\begin{center}
	KHẢO SÁT VỀ SỬ DỤNG MẠNG XÃ HỘI
\end{center}
\begin{enumerate}
	\item Giới tính của bạn
	\begin{enumEX}[$\square$]{2}
		\item  Nữ
		\item  Nam
	\end{enumEX}
	\item Lợi ích lớn nhất mà mạng xã hội mang lại là (chọn một phương án):
	\begin{enumEX}[$\square$]{2}
		\item Kết nối bạn bè
		\item Giải trí
		\item Thu thập thông tin
		\item Tìm hiểu thế giới xung quanh
	\end{enumEX}
	\item Điều bất lợi lớn nhất khi sử dụng mạng xã hội là (chọn một phương án)
	\begin{enumerate}
		\item[$\square$]  Có nguy cơ tiếp xúc với những bài viết, hình ảnh, video, ý kiến tiêu cực, không thích hợp
		\item[$\square$] Thông tin cá nhân bị đánh cắp
		\item[$\square$] Có thể bị bắt nạt trên internet
		\item[$\square$] Mất thời gian sử dụng internet
	\end{enumerate}
\item Thời gian (ước lượng số phút) bạn sử dụng mạng xã hội trong một ngày:
\dotfill
\end{enumerate}	
%\end{tomtat}

\subsubsection{Thu thập dữ liệu theo phiếu khảo sát}
\begin{center}
	\begin{tabular}{|c|c|c|c|c|}
	\hline
	STT & Giới tính & Thời gian sử dụng mạng xã hội & Lợi ích & Bất lợi\\
	\hline
	1& Nam &60 &3 &2 \\
	\hline
	&  &  &  & \\
	\hline
	&  &  &  & \\
	\hline
	&  &  &  & \\
	\hline
\end{tabular}
\end{center}
\subsubsection{Lợi ích của mạng xã hội}
Để biết các bạn nam học sinh tham gia khảo sát đánh giá thế nào về lợi ích và bất lợi của mạng xã hội, hãy thực hiện các yếu cầu sau đây:
\begin{enumerate}
	\item Lập bảng tần số của dữ liệu ý kiến về lợi ích/bất lợi của mạng xã hội theo mẫu sau:
\begin{center}
		\begin{tabular}{|c|c|c|c|c|}
		\hline
		Ý kiến & Kết nối với bạn bè & Giải trí &Thu thập thông tin & Tìm hiểu thế giới xung quanh\\
		\hline
		Số học sinh & & & &\\
		\hline
	\end{tabular}
\end{center}
	\item Từ đó rút ra nhận xét từ bảng số liệu thu được.
\end{enumerate}

\subsubsection{Thời gian sử dụng mạng xã hội}
Hãy tính một số đo thống kê mô tả được liệt kê trong Bảng T.2 của mẫu số liệu về thời gian sử dụng mạng xã hội:
\begin{center}
		\begin{tabular}{|c|c|c|c|c|c|c|}
		\hline
		Giá trị nhỏ nhất & $Q_1$ &Số trung bình & Trung vị &$Q_3$ & Mốt & Giá trị lớn nhất \\
		\hline
		& & & & & &\\
		\hline
	\end{tabular}
\end{center}

\subsubsection{Thời gian sử dụng mạng xã hội của học sinh nam và học sinh nữ}
\begin{enumerate}
	\item Hãy tính số trung bình, trung vị, tứ phân vị của thời gian sử dụng mạng xã hội trên hai nhóm học sinh nữ và học sinh nam để so sánh thời gian sử dụng mạng xã hội của hai nhóm
\begin{center}
		\begin{tabular}{|c|c|c|c|c|}
		\hline
		& Số trung bình & $Q_1$ & Trung vị ($Q_2$) & $Q_3$\\
		\hline
		Nữ & & & &\\
		\hline
		Nam & & & &\\
		\hline
	\end{tabular}
\end{center}
	\item Hãy tính một vài số đo độ phân tán để sô sánh sự biến động của thời gian sử dụng mạng xã hội trên hai nhóm học sinh
\begin{center}
		\begin{tabular}{|c|c|c|c|}
		\hline
		& Khoảng biến thiên & Khoảng tứ phân vị & Độ lệch chuẩn\\
		\hline
		Nữ & & &\\
		\hline
		Nam & & &\\
		\hline
	\end{tabular}
\end{center}
\end{enumerate}
	\subsection{Các dạng toán}
	\begin{dang}{Mạng xã hội cụ thể: Facebook; Youtube; Zalo; Tick Tock;....}
Hãy thu thập và xử lí số liệu liên quan đến một mạng xã hội cụ thể: Facebook; Youtube; Zalo; Tick Tock;....
	\end{dang}
\viduminhhoa	
\begin{vd}%[Dương Phước Sang]%[2D1B1]
Lập phiếu khảo sát và thu thập dữ liệu  về mạng xã hội Facebook để xử lí số liệu về
\begin{enumerate}
	\item Lợi ích của mạng xã hội Facebook.
	\item Thời gian sử dụng mạng xã hội Facebook.
	\item Thời gian sử dụng mạng xã hội Facebook của nam và nữ học sinh.
\end{enumerate}
\loigiai{
 \begin{center}
 		PHIẾU KHẢO SÁT VỀ SỬ DỤNG MẠNG XÃ HỘI FACEBOOK
 \end{center}
	\begin{enumerate}
		\item Giới tính của bạn
		\begin{enumEX}[$\square$]{2}
			\item   Nữ
			\item   Nam
		\end{enumEX}
		\item Lợi ích lớn nhất mà mạng xã hội Facebook mang lại là (chọn một phương án):
		\begin{enumEX}[$\square$]{2}
			\item Kết nối bạn bè
			\item Giải trí
			\item Thu thập thông tin
			\item Tìm hiểu thế giới xung quanh
		\end{enumEX}
		\item Điều bất lợi lớn nhất khi sử dụng mạng xã hội Facebook là (chọn một phương án)
		\begin{enumEX}{1}
			\item[$\square$] Có nguy cơ tiếp xúc với những bài viết, hình ảnh, video, ý kiến tiêu cực, không thích hợp
			\item[$\square$] Thông tin cá nhân bị đánh cắp
			\item[$\square$] Có thể bị bắt nạt trên mạng xã hội Facebook
			\item[$\square$] Mất thời gian sử dụng mạng xã hội Facebook
		\end{enumEX}
		\item Thời gian (ước lượng số phút) bạn sử dụng mạng xã hội mạng xã hội Zalo trong một ngày:
		....\dotfill....\\
	\end{enumerate}	
}
\end{vd}

\begin{vd}%[Dương Phước Sang]%[2D1B1]
	Lập phiếu khảo sát và thu thập dữ liệu  về mạng xã hội Zalo để xử lí số liệu về
	\begin{enumerate}
		\item Lợi ích của mạng xã hội Zalo.
		\item Thời gian sử dụng mạng xã hội Zalo.
		\item Thời gian sử dụng mạng xã hội Zalo của nam và nữ học sinh.
\end{enumerate}
\loigiai{
	\begin{center}
		PHIẾU KHẢO SÁT VỀ SỬ DỤNG MẠNG XÃ HỘI ZALO
	\end{center}
	\begin{enumerate}
		\item Giới tính của bạn
		\begin{enumEX}[$\square$]{2}
			\item  Nữ
			\item  Nam
		\end{enumEX}
		\item Lợi ích lớn nhất mà mạng xã hội Zalo mang lại là (chọn một phương án):
		\begin{enumEX}[$\square$]{2}
			\item Kết nối bạn bè
			\item Giải trí
			\item Thu thập thông tin
			\item Tìm hiểu thế giới xung quanh
		\end{enumEX}
		\item Điều bất lợi lớn nhất khi sử dụng mạng xã hội Zalo là (chọn một phương án)
		\begin{enumEX}{1}
			\item[$\square$] Có nguy cơ tiếp xúc với những bài viết, hình ảnh, video, ý kiến tiêu cực, không thích hợp
			\item[$\square$] Thông tin cá nhân bị đánh cắp
			\item[$\square$] Có thể bị bắt nạt trên mạng xã hội Zalo
			\item[$\square$] Mất thời gian sử dụng mạng xã hội Zalo
		\end{enumEX}
		\item Thời gian (ước lượng số phút) bạn sử dụng mạng xã hội mạng xã hội Zalo trong một ngày:
		....\dotfill....\\
	\end{enumerate}	
}
\end{vd}
		\baitaptl

\begin{bt}%[Dương Phước Sang]%[2D1B1]
	Lập phiếu khảo sát và thu thập dữ liệu  về mạng xã hội TickTock để xử lí số liệu về
	\begin{enumerate}
		\item Lợi ích của mạng xã hội TickTock.
		\item Thời gian sử dụng mạng xã hội TickTock.
		\item Thời gian sử dụng mạng xã hội TickTock của nam và nữ học sinh.
	\end{enumerate}
\loigiai{
	\begin{center}
		PHIẾU KHẢO SÁT VỀ SỬ DỤNG MẠNG XÃ HỘI TICKTOCK
	\end{center}
	\begin{enumerate}
		\item Giới tính của bạn
		\begin{enumEX}[$\square$]{2}
			\item  Nữ
			\item   Nam
		\end{enumEX}
		\item Lợi ích lớn nhất mà mạng xã hội TickTock mang lại là (chọn một phương án):
		\begin{enumEX}[$\square$]{2}
			\item Kết nối bạn bè
			\item Giải trí
			\item Thu thập thông tin
			\item Tìm hiểu thế giới xung quanh
		\end{enumEX}
		\item Điều bất lợi lớn nhất khi sử dụng mạng xã hội TickTock là (chọn một phương án)
		\begin{enumEX}[$\square$]{1}
			\item Có nguy cơ tiếp xúc với những bài viết, hình ảnh, video, ý kiến tiêu cực, không thích hợp
			\item Thông tin cá nhân bị đánh cắp
			\item Có thể bị bắt nạt trên mạng xã hội TickTock
			\item Mất thời gian sử dụng mạng xã hội TickTock
		\end{enumEX}
		\item Thời gian (ước lượng số phút) bạn sử dụng mạng xã hội mạng xã hội TickTock trong một ngày:
		....\dotfill....\\
	\end{enumerate}	
}
\end{bt}


