\begin{name}
	{\tenchude}
	{CHƯƠNG I - TOÁN 10}
	{LỚP TOÁN THẦY PHÁT}
	{Thời gian: 20 phút - Không kể thời gian phát đề}
\end{name}

\TN
%Câu 1
\begin{ex}
Trong các mệnh đề sau, mệnh đề nào đúng?
\choice
{$\exists n\in \mathbb{N},n^2+n+3$ là số lẻ}
{$\forall x\in \mathbb{Z},x^2+3=0$}
{$\forall n\in \mathbb{N},n^2+n+4$ chia hết cho 3}
{$\exists x\in \mathbb{N},2x^2+5x+2=0$}
\end{ex}
%Câu 2
\begin{ex}
Cho tam giác $ABC$. Trong các phát biểu sau, phát biểu nào là đúng về định lí “Nếu tam giác $ABC$ là tam giác vuông tại $A$ thì $AB^2+AC^2=BC^2$.”?
\choice
{$AB^2+AC^2=BC^2$ là điều kiện cần để tam giác $ABC$ là tam giác vuông}
{Tam giác $ABC$ là tam giác vuông tại $A$ là điều kiện đủ để $AB^2+AC^2= BC^2$}
{$AB^2+AC^2 \le BC^2$ là điều kiện đủ để tam giác $ABC$ là tam giác vuông}
{Tam giác $ABC$ là tam giác vuông tại $A$ là điều kiện cần để $AB^2+AC^2=BC^2$}
\end{ex}
%Câu 3
\begin{ex}
Mệnh đề phủ định của mệnh đề \lq\lq $\forall x\in \mathbb{Z},x^2+1\in \mathbb{N}$\rq\rq.
\choice
{$\forall x\in \mathbb{Z},x^2+1\notin \mathbb{N}$}
{$\exists x\in \mathbb{Z},x^2+1\notin \mathbb{N}$}
{$\forall x\notin \mathbb{Z},x^2-1\in \mathbb{Q}$}
{$\exists x\in \mathbb{N},x^2+1\notin \mathbb{Z}$}
\end{ex}
%Câu 4
\begin{ex}
Cho mệnh đề chứa biến $P(x)$: \lq\lq$x^2-x+1$ không là số nguyên tố\rq\rq. Trong các mệnh đề sau, mệnh đề nào là mệnh đề đúng?
\choice
{$P(-1)$}
{$P(2)$}
{$P(5)$}
{$P(3)$}
\end{ex}
%Câu 5
\begin{ex}
Trong các câu sau, câu nào là mệnh đề Toán học?
\choice
{$x^2+1\ge 2$}
{$2x+y=3$}
{Số 1 có phải là số nguyên không?}
{Hình tròn có một tâm đối xứng}
\end{ex}
%Câu 6
\begin{ex}
Trong các mệnh đề sau, mệnh đề nào đúng? 
\choice
{$\left[-2024;2025\right)\subset \left[-2024;2025\right]$}
{$\left[-2024;2025\right)\subset \left(-2024;2025\right]$}
{$\left(-2024;2025\right]\subset \left[-2024;2025\right)$}
{$\left[-2024;2025\right]\subset \left(-2024;2025\right)$}
\end{ex}
\TL
\begin{ex}
Cho $A=\{x \in \mathbb{R}\ \mid\ x\le 2 \}$, $B=[1;4)$ và $C=\{x \in \mathbb{R}\ \mid\ x^2<1\}$. Viết các tập hợp $A$, $C$ về dạng khoảng, đoạn, nửa khoảng. Thực hiện các phép toán sau và biểu diễn kết quả lên trục số $A\cap B$, $A\cup B$, $A\setminus C$.
\end{ex}
\begin{ex}
Lớp 10A có 45 học sinh. Biết trong lớp có 20 học sinh biết chơi guitar, 14 học sinh biết chơi violin, 23 học sinh biết chơi piano; trong đó có
\begin{itemize}
    \item 6 học sinh biết chơi guitar và violin,
    \item 7 học sinh biết chơi violin và piano,
    \item 6 học sinh biết chơi guitar và piano,
    \item 1 học sinh biết chơi cả ba loại đoàn guitar, piano, violin.
\end{itemize}
Hỏi lớp 10A có bao nhiêu bạn không biết chơi cả ba nhạc cụ trên?
\end{ex}
