\begin{name}
	{\tenchude}
	{CHƯƠNG I - TOÁN 10}
	{LỚP TOÁN THẦY PHÁT}
	{ }
\end{name}
%Câu 1
\begin{ex}
	Cho định lí: “Trong mặt phẳng, nếu hai đường thẳng phân biệt cùng vuông góc với đường thẳng thứ ba thì hai đường thẳng đó song song với nhau”. Phát biểu nào dưới đây sai?
	\choice
	{Điều kiện đủ để hai đường thẳng phân biệt cùng vuông góc với đường thẳng thứ ba là hai đường thẳng đó song song với nhau}
	{Hai đường thẳng phân biệt cùng vuông góc với đường thẳng thứ ba là điều kiện đủ để hai đường thẳng đó song song với nhau}
	{Hai đường thẳng phân biệt cùng vuông góc với đường thẳng thứ ba kéo theo nó song song với nhau}
	{Hai đường thẳng song song với nhau là điều kiện cần để hai đường thẳng đó cùng vuông góc với đường thẳng thứ ba}
\end{ex}
%Câu 2
\begin{ex}
	Cho các phát biểu sau:
	\begin{enumerate}
		\item Mọi số tự nhiên $n$ đều chia hết cho 2.
		\item $x^2- |x|\ge 0$.
		\item $2025^{2024}-1$ là số nguyên tố.
		\item Phương trình $x^2-2x+1=0$ vô nghiệm.
		\item $x$ là số nguyên dương.
	\end{enumerate}
	Số câu phát biểu không phải mệnh đề?
	\choice
	{$3$}{$2$}{$1$}{$4$}
\end{ex}
%Câu 3
\begin{ex}
	Khẳng định nào sau đây sai?
	\choice
	{$(-1;1)\subset [-1;2)$}
	{$\varnothing \in \{\sqrt{2}\}$}
	{$\{0\}\subset [-1;2)$}
	{$-2\in \left(-\infty ;-1\right)$}
\end{ex}
%Câu 4
\begin{ex}
	Cho mệnh đề chứa biến $P\left(a,b\right)\colon''{a^{2024}}+{b^{2022}}-1\ge 0''$ với $a$, $b$ là các số thực. Tìm khẳng định sai.
	\choice
	{$P\left(1{,}1\right)$ đúng}
	{$P\left(0,-1\right)$ sai}
	{$P\left(1,-1\right)$ đúng}
	{$P\left(-1{,}0\right)$ đúng}
\end{ex}
\begin{ex}
	Gọi $A$ là tập hợp các tứ giác là hình bình hành, $B$ là tập hợp các tứ giác là hình thoi, $C$ là tập hợp các tứ giác là hình chữ nhật; kí hiệu phần tử $x$ là một hình chữ nhật $ABCD$. Khẳng định nào sau đây là đúng?
	\choice
	{$A\subset B\subset C$}
	{$x\notin A$ và $B\not\subset C$}
	{$x\in B$ và $B\subset C$}
	{$x\in B$ và $B\subset A$}
\end{ex}
%Câu 2
\begin{ex}
	Cho định lí "Nếu hai tam giác bằng nhau thì diện tích của chúng bằng nhau". Phát biểu lại định lí dưới dạng nào sau đây là đúng?
	\choice
	{Hai tam giác có diện tích bằng nhau là điều kiện cần và đủ để chúng có diện tích bằng nhau}
	{Hai tam giác bằng nhau là điều kiện cần để diện tích chúng bằng nhau}
	{Hai tam giác bằng nhau là điều kiện đủ để diện tích chúng bằng nhau}
	{Hai tam giác có diện tích bằng nhau là điều kiện đủ đê chúng bằng nhau	}
\end{ex}
%Câu 3
\begin{ex}
	Tìm mệnh đề sai.
	\choice
	{$\exists n\in \mathbb{N}\colon n^2+1$ chia hết cho $3$}
	{$\forall n\in \mathbb{N}\colon n(n+1)(n+2)$ chia hết cho $6$}
	{$\exists x\in \mathbb{R}\colon x^2\le 0$}
	{$\forall n\in \mathbb{N}\colon n^2+1$ không chia hết cho $4$}
\end{ex}
%Câu 4
\begin{ex}
	Lập mệnh đề phủ định của mệnh đề $''\forall x\in \mathbb{R}\colon 2x^2-5x+2024>0''$.
	\choice
	{$\forall x\in \mathbb{R}\colon 2x^2-5x+2024\le 0$}
	{$\exists x\in \mathbb{R}\colon 2x^2-5x+2024\le 0$}
	{$\exists x\in \mathbb{R}\colon 2x^2-5x+2024<0$}
	{$\forall x\in \mathbb{R}\colon 2x^2-5x+2024\ne 0$
	}
\end{ex}
%Câu 5
\begin{ex}
	Cho mệnh đề chứa biến $P\left(x,y,z\right)\colon$ \lq\lq$x^2+y^2+z^2\le 5 \text{ và }x+y+z=2$\rq\rq. Mệnh đề nào sau đây là đúng?
	\choice
	{$P\left(1{,}1,1\right)$}
	{$P\left(-1{,}2,1\right)$}
	{$P\left(0,-1{,}1\right)$}
	{$P\left(1{,}1,0\right)$
	}
\end{ex}
%Câu 6
\begin{ex}
	Trong các câu sau, có bao nhiêu câu là mệnh đề?
	\begin{enumerate}
		\item $n^3-3n\ge 2025$.
		\item Phương trình $-4x^2+3\sqrt{x}-2=0$ có nghiệm.
		\item Với mọi số thực $x$, $x^2+1\le 3$.
		\item Có bao nhiêu số tự nhiên $n$ nhỏ $50$ là số nguyên tố?
		\item Tổng $1+2+\cdots+10$ là số may mắn.
	\end{enumerate}
	\choice
	{$2$}
	{$3$}
	{$4$}
	{$1$}
\end{ex}
%Câu 5
\begin{ex}
	Mệnh đề nào sau đây đúng?
	\choice
	{$\exists n\in \mathbb{N}^*,1+2+3+ \cdots+n$ chia hết cho $11$}
	{$\exists x\in R,\dfrac{2x}{x^2+1}>1$}
	{$\exists x\in \mathbb{R},x^2+1<1$}
	{$\exists n\in N,\left(n^2+1\right)$ chia hết cho $4$	}
\end{ex}
%Câu 6
\begin{ex}
	Mệnh đề phủ định của mệnh đề “$\forall x\in R,\sqrt{x}$ là số hữu tỷ”
	\choice
	{“$\forall x\in R,\sqrt{x}$ không phải là số hữu tỷ”}
	{“$\exists x\in R,\sqrt{x}$ là số hữu tỷ”}
	{“$\forall x\in R,\sqrt{x}$ là số vô tỷ”}
	{“$\exists x\in R,\sqrt{x}$ là số vô tỷ”}
\end{ex}
%Câu 7
\begin{ex}
	Cho hai tập hợp $A=\left(-\infty ;1\right)\cup \left(2;+\infty\right)$ và $B=\left\{ x\in \mathbb{R}\mid 0\le x<3 \right\}$. Tìm $A \cap B$, $A\cup B$ và $B\setminus A$ và biểu diễn chúng lên trục số.
\end{ex}
%Câu 7
\begin{ex}
	Cho $A=\left\{ x\in R|-2<x<3 \right\}$, $B=\left\{ x\in R|x\le 1 \right\}$, $C=\left\{ x\in R|-1\le x\le 5 \right\}$. Tìm $A \cap B$, $A\cup B$ và $C\setminus A$ và biểu diễn chúng lên trục số.
\end{ex}
%Câu 8
\begin{ex}
	Lớp 10A có 45 học sinh trong đó có 25 em giỏi Toán, 23 em giỏi Lý, 20 em giỏi Hoá, 11 em giỏi cả hai môn Toán và Lý, 8 em giỏi cả môn Lý và Hoá, 9 em giỏi cả Toán và Hoá. Hỏi Lớp 10A có bao nhiêu học sinh giỏi cả ba môn Toán, Lý và Hoá?
\end{ex}
%Câu 8
\begin{ex}
	Lớp 10C có 43 học sinh chuẩn bị cho hội diễn văn nghệ chào mừng ngày nhà giáo Việt Nam 20/11. Trong danh sách đăng kí tham gia tiết mục nhảy Flashmob và tiết mục hát, có 35 học sinh tham gia tiết mục nhảy Flashmob, 10 học sinh tham gia cả hai tiết mục. Hỏi có bao nhiêu học sinh trong lớp tham gia tiết mục hát? Biết rằng lớp 10C có các bạn Việt, Nam, Bá, Đạo bận đi thi Olympic nên không tham gia tiết mục nào?
\end{ex}
