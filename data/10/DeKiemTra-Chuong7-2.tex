\section*{Đề kiểm tra Chương 7}
\subsection*{Đề số 2}
\setcounter{ex}{0}\setcounter{bt}{0}
\Opensolutionfile{ans}[ans/ans-KT-702]
\noindent\textbf{I. PHẦN TRẮC NGHIỆM}
	\begin{ex}%[0H3B1-2]%[Yến Trần BG10]
	Véc-tơ nào dưới đây là một véc-tơ chỉ phương của đường thẳng song song với trục $Ox$?
	\choice
	{\True $\overrightarrow{u_{1}}=(1;0)$}
	{$\overrightarrow{u_{2}}=(0;-1)$}
	{$\overrightarrow{u_{3}}=(-1;1)$}
	{$\overrightarrow{u_{4}}=(1;1)$}
	\loigiai{
		Trục $ Ox \colon y=0$ có VTCP $\overrightarrow{i}=(1;0)$ nên một đường thẳng song song với $Ox$ cũng có  VTCP $\overrightarrow{i}=(1;0)$.
	}
\end{ex}
	\begin{ex}%[0H3B1-2]%[Yến Trần BG10]
	Đường thẳng $d$ có một véc-tơ chỉ phương là $\overrightarrow{u}=(3;-4)$. Đường thẳng $\Delta$ song song với $d$ có một véc-tơ pháp tuyến là
	\choice
	{\True $\overrightarrow{n_{1}}=(4;3)$}
	{$\overrightarrow{n_{2}}=(-4;3)$}
	{$\overrightarrow{n_{3}}=(3;4)$}
	{$\overrightarrow{n_{4}}=(3;-4)$}
	\loigiai{
		Đường thẳng $\Delta$ song song với $d$ có một véc-tơ pháp tuyến là $\overrightarrow{n_{1}}=(4;3)$.
	}
\end{ex}	
	\begin{ex}%[0H3Y3-1]%[Yến Trần BG10]
		Trong mặt phẳng, cho hai điểm cố định $F_1, F_2$ và một độ dài không đổi $2>F_1F_2$. Elip là tập hợp những điểm $M$ thỏa mãn hệ thức nào sau đây
		\choice
		{$MF_1-MF_2>2a$}
		{$MF_1+MF_2>2a$}
		{\True $MF_1+MF_2=2a$}
		{$MF_1-MF_2=2a$}
		\loigiai{
			Theo định nghĩa Elip, ta có đáp án đúng là $MF_1+MF_2=2a$.
		}
	\end{ex}
\begin{ex}%[0H3B1-2]%[Yến Trần BG10]
	Véc-tơ nào dưới đây là một véc-tơ chỉ phương của đường thẳng $\Delta \colon \heva{&x=5-\dfrac{1}{2}t\\&y=-3+3t}$
	\choice
	{\True $\overrightarrow{u_{1}}=(-1;6)$}
	{$\overrightarrow{u_{2}}=\left(\dfrac{1}{2};3\right)$}
	{$\overrightarrow{u_{3}}=(-5;3)$}
	{$\overrightarrow{u_{4}}=(5;-3)$}
	\loigiai{ Ta có
		$\Delta \colon \heva{&x=5-\dfrac{1}{2}t\\&y=-3+3t}$ suy ra VTCP $\overrightarrow{u}=\left(-\dfrac{1}{2};3\right)=\dfrac{1}{2}(-1;6)$.\\
		Chọn $\overrightarrow{u}=(-1;6)$.
	}
\end{ex}
\begin{ex}%[0H3B1-2]%[Yến Trần BG10]
	Trong mặt phẳng với hệ tọa độ $Oxy$ cho tam giác $ABC$ có  $A(1;4)$, $B(3;2)$ và $C(7;3)$. Viết PTTS của đường trung tuyến $CM$ của tam giác
	\choice
	{$\heva{&x=7\\&y=3+5t}$}
	{$\heva{&x=3-5t\\&y=-7}$}
	{\True $\heva{&x=7+t\\&y=3}$}
	{$\heva{&x=2\\&y=3-t}$}
	\loigiai{ Ta có $\heva{&A(1;4)\\&B(3;2)} \Rightarrow M(2;3) \Rightarrow \overrightarrow{MC}=(5;0)=5(1;0)$.\\
		Suy ra $CM \colon \heva{&x=7+t\\&y=3}$.
	}
\end{ex}
\begin{ex}%[0H3Y2-1]%[Yến Trần BG10]
	Phương trình nào sau đây là phương trình đường tròn?
	\choice
	{$x^2 - y^2 + 2x - 4y - 6 = 0$}
	{$x^2 + 2y^2 + 4x - 2y + 6 = 0$}
	{\True $x^2 + y^2 + 2x - 4y - 6 = 0$}
	{$x^2 + y^2 + 2x - 4y + 6 = 0$}
	\loigiai
	{
		Phương trình $x^2 + y^2 + 2x - 4y - 6 = 0$ là phương trình đường tròn, với tâm $I(-1;2)$ và bán kính $R = \sqrt{11}$.
	}
\end{ex}
\begin{ex}%[0H3B2-2]%[Yến Trần BG10]
	Đường tròn tiếp xúc với hai đường thẳng $d \colon x - 2y = 0$ và $d'\colon x-2y - 10 = 0$ có bán kính là
	\choice
	{\True $\sqrt{5}$}
	{$5$}
	{$10$}
	{$2\sqrt{5}$}
	\loigiai
	{
		Khoảng cách giữa hai đường thẳng $(d)$ và $\left(d'\right)$ là \[\mathrm{d} = \dfrac{|0-(-10)|}{\sqrt{1^2 + (-2)^2}} = 2\sqrt{5}.\]
		Suy ra bán kính đường tròn tiếp xúc với cả $(d)$ và $\left(d'\right)$ là $R = \dfrac{1}{2}\mathrm{d} = \sqrt{5}$.
	}
\end{ex}
	\begin{ex}%[0H3Y2-1]%[Yến Trần BG10]
	Trong mặt phẳng tọa độ $Oxy$, cho đường tròn $(C)$ có phương trình $(x-2)^2+(y+1)^2=25$. Tọa độ tâm $I$ và độ dài bán kính $R$ là
	\choice
	{$I(-2; -1), R=5$}
	{$I(2; -1), R=\sqrt{5}$}
	{$I(2; 1), R=\sqrt 5$}
	{\True $I(2; -1), R=5$}
	\loigiai{
		Đường tròn có tâm $I(2;-1)$ và bán kính $R=\sqrt{25}=5$.
	}
\end{ex}
\begin{ex}%[0H3B1-2]%[Yến Trần BG10]
	Với giá trị nào của $m$ thì ba đường thẳng $d_{1} \colon 2x+y-1=0, d_{2} \colon x+2y+1=0$ và $d_{3} \colon mx-y-7=0$ đồng quy?
	\choice
	{$m=-6$}
	{\True $m=6$}
	{$m=-5$}
	{$m=5$}
	\loigiai{ Gọi $A(x;y)$ là giao điểm của $d_{1}$ và $d_{2}$ nên tọa độ của nó là nghiệm của hệ 
		\begin{center}
			$\heva{&2x+y-1=0\\&x+2y+1=0} \Leftrightarrow \heva{&x=1\\&y=-1} \Rightarrow A(1;-1$).
		\end{center}
		Vì $A \in d_{3} \Rightarrow m+1-7=0 \Leftrightarrow m=6$.
	}
\end{ex}
\begin{ex}%[0H3B1-2]%[Yến Trần BG10]
	Cho đường thẳng $d \colon 3x+5y+2018=0$. Tìm mệnh đề sai trong các mệnh đề sau.
	\choice
	{$d$ có véc-tơ pháp tuyến $\overrightarrow{n}=(3;5)$}
	{$d$ có véc-tơ chỉ phương $\overrightarrow{u}=(5;-3)$}
	{$d$ có hệ số góc $k=\dfrac{5}{3}$}
	{\True $d$ song song với đường thẳng $\Delta \colon3x+5y=0$}
	\loigiai{ Ta có $d \colon 3x+5y+2018=0$ $\Rightarrow \heva{&\overrightarrow{n_{d}}=(3;5)\\&\overrightarrow{u_{d}}=(5;-3)\\&k_{d}=-\dfrac{3}{5}.}$\\
		Mặt khác ta có $d \colon 3x+5y+2018=0 \Rightarrow d \parallel \Delta 3x+5y=0$. 
	}
\end{ex}
\begin{ex}%[0H3B2-3]%[Yến Trần BG10]
	Phương trình tiếp tuyến tại điểm $M(3;4)$ với đường tròn $(C) \colon x^2+y^2-2x-4y-3=0$ là
	\choice
	{$x+y-3=0$}
	{\True $x+y-7=0$}
	{$x-y-7=0$}
	{$x+y+7=0$}
	\loigiai{
		Đường tròn $(C)$ có tâm $I(1;2)$. $\vec{IM} = (2;2)$.\\
		Tiếp tuyến $\Delta$ tại $M$ với đường tròn $(C)$ có một véc-tơ pháp tuyến là $\vec{n} = (1;1)$. \\
		Phương trình tiếp tuyến $\Delta$ là $1(x-3) + 1(y-4) = 0 \Leftrightarrow x + y - 7 = 0$.
	}
\end{ex}
\begin{ex}%[0H3Y1-5]%[Yến Trần BG10]
	Trong mặt phẳng $Oxy$, khoảng cách từ $M(3;-4)$ đến đường thẳng $\Delta \colon 3x-4y-1=0$ là
	\choice
	{$\dfrac{8}{5}$}
	{\True $\dfrac{24}{5}$}
	{$\dfrac{12}{5}$}
	{$-\dfrac{24}{5}$}
	\loigiai{
		Ta có $\mathrm{d}(M,\Delta)=\dfrac{|3\cdot 3-4\cdot (-4)-1|}{\sqrt{3^2+(-4)^2}}=\dfrac{24}{5}$.	
	}
\end{ex}
\begin{ex}%[0H3B1-2]%[Yến Trần BG10]
	Cho đường thẳng $d_{1} \colon x+2y-2=0$ và $d_{2} \colon x-y=0$. Tính cosin của góc tạo bởi giữa hai đường thẳng đã cho
	\choice
	{\True $\dfrac{\sqrt{10}}{10}$}
	{$\dfrac{\sqrt{2}}{3}$}
	{$\dfrac{\sqrt{3}}{3}$}
	{$\sqrt{3}$}
	\loigiai{Ta có $\heva{&d_{1} \colon x+2y-2=0 \Rightarrow \overrightarrow{n_{1}}=(1;2)\\& d_{2} \colon x-y=0 \Rightarrow \overrightarrow{n_{2}}=(1;-1).}$\\
		Gọi góc tạo bởi giữa hai đường thẳng trên là $\alpha$.
		Suy ra $\cos \alpha=\dfrac{|1-2|}{\sqrt{1+4}\cdot \sqrt{1+1}}=\dfrac{1}{\sqrt{10}}$.
		
	}
\end{ex}

\begin{ex}%[0H3B1-2]%[Yến Trần BG10]
	Elip $E \colon \dfrac{x^{2}}{25}+\dfrac{y^{2}}{9}=1$ có độ dài trục lớn bằng 
	\choice
	{$5$}
	{\True $10$}
	{$25$}
	{$50$}
	\loigiai{Gọi phương trình của elip là $\dfrac{x^{2}}{a^{2}}+\dfrac{y^{2}}{b^{2}}=1$, có độ dài trục lớn $A_{1}A_{2}=2a$. Xét 
		\begin{center}
			$E \colon \dfrac{x^{2}}{25}+\dfrac{y^{2}}{9}=1\Rightarrow \heva{&a^{2}=25\\&b^{2}=9} \Rightarrow \heva{&a=5\\&b=3} \Rightarrow A_{1}A_{2}=2\cdot 5=10$.
		\end{center}
		
	}
\end{ex}
\begin{ex}%[0H3B1-2]%[Yến Trần BG10]
	Đường thẳng song song với đường thẳng $y=3x+5$ và đi qua điểm $A(1;11)$ có phương trình là
	\choice
	{\True $y=3x+8$}
	{$y=x+10$}
	{$y=3x+11$}
	{$y=-3x+14$}
	\loigiai{
		Đường thẳng cần tìm song song với đường thẳng $y=3x+5$ nên có dạng $y=3x+b$, với $b\ne 5$. \\
		Thay $A(1;11)$ vào đường thẳng $y=3x+b$ ta được $b=8$ (thỏa mãn).\\ 
		Vậy PTĐT là $y=3x+8$.	
	}
\end{ex}
\begin{ex}%[0H3B2-2]%[Yến Trần BG10]
	Đường tròn đường kính $AB$ với $A(3;-1)$, $B(1;-5)$ có phương trình là
	\choice
	{\True $(x+2)^2+(y-3)^2=5$}
	{$(x+1)^2+(y+2)^2=17$}
	{$(x-2)^2+(y+3)^2=\sqrt{5}$}
	{$(x-2)^2+(y+3)^2=5$}
	\loigiai{
		Gọi $I$ là trung điểm $AB \Rightarrow I(2;-3)$. $\vec{IA} = (1;2)$.\\
		Đường tròn $(C)$ đường kính có tâm $I(2;-3)$ và bán kính $R = IA = \sqrt{5}$. \\
		Vậy phương trình đường tròn $(C) \colon (x-2)^2 + (y+3)^2 = 5$.
	}
\end{ex}

\begin{ex}%[0H3B1-2]%[Yến Trần BG10]
	Đường thẳng nào là đường chuẩn của parabol $y^{2}=\dfrac{3}{2}x$.
	\choice
	{$x=-\dfrac{3}{4}$}
	{$x=\dfrac{3}{4}$}
	{$x=\dfrac{3}{2}$}
	{\True $x=-\dfrac{3}{8}$}
	\loigiai{Phương trình chính tắc của parabol $(P) \colon y^{2}=2px$.\\
		Suy ra $p=\dfrac{3}{4}$ nên phương trình đường chuẩn là $x=-\dfrac{p}{2}=-\dfrac{3}{8}$.
	}
\end{ex}
	\begin{ex}%[0H3Y2-1]%[Yến Trần BG10]
		Trong mặt phẳng $Oxy$, phương trình nào sau đây là phương trình đường tròn?
		\choice
		{$x^2 + y^2 = -1$}
		{$x^2 - y^2 = 1$}
		{\True $x^2 + y^2 = 1$}
		{$x^2 - y^2 = -1$}
		\loigiai{
			Đường tròn có tâm $I(a;b)$ và bán kính $R$ có phương trình $(x - a)^2 + (y - b)^2 = R^2$. Như vậy, $x^2 + y^2 = 1$ là một phương trình đường tròn với tâm $O(0;0)$ và bán kính $R = 1$.
		}
	\end{ex}
	
\begin{ex}%[0H3B3-2]%[Yến Trần BG10]
	Trong mặt phẳng $Oxy$, phương trình chính tắc của elip có độ dài trục lớn bằng$10$ và độ dài trục nhỏ bằng $8$ là
	\choice
	{$\dfrac{x^2}{100}+\dfrac{y^2}{64}=1$}
	{\True $\dfrac{x^2}{25}+\dfrac{y^2}{16}=1$}
	{$\dfrac{x^2}{10}+\dfrac{y^2}{8}=1$}
	{$\dfrac{x^2}{5}+\dfrac{y^2}{4}=1$}
	\loigiai{
		Theo giả thiết, ta có độ dài trục lớn bằng$10$ và độ dài trục nhỏ bằng $8$ có thể có phương trình là 
		\[\dfrac{x^2}{5^2}+\dfrac{y^2}{4^2}=1 \Leftrightarrow \dfrac{x^2}{25}+\dfrac{y^2}{16}=1.\]}
\end{ex}
	\begin{ex}%[0H3Y1-5]%[Yến Trần BG10]
		Trong mặt phẳng $Oxy$, cho điểm $M(1;2)$ và đường thẳng $\Delta\colon x + y + 1 = 0$. Khoảng cách từ $M$ đến đường thẳng $\Delta$ bằng
		\choice
		{$\dfrac{3}{\sqrt{2}}$}
		{$\dfrac{4}{\sqrt{3}}$}
		{$4\sqrt{2}$}
		{\True $2\sqrt{2}$}
		\loigiai{
			Khoảng cách từ $M$ đến đường thẳng $\Delta$ bằng
			$$ \textrm{d}(M,\Delta) = \dfrac{|1 + 2 + 1|}{\sqrt{1^2 + 1^2}} = 2\sqrt{2}. $$
		}
	\end{ex}
\begin{ex}%[0H3Y3-1]%[Yến Trần BG10]
	Trong mặt phẳng $Oxy$, cho $(E) \colon \dfrac{x^2}{a^2}+\dfrac{y^2}{b^2}=1$, $(a>b>0)$. Độ dài trục nhỏ của $(E)$ cho bằng
	\choice
	{$2a$}
	{$a^2$}
	{\True $2b$}
	{$b^2$}
	\loigiai{
		Vì $a>b>0$ nên độ dài trục nhỏ bằng $2b$.
	}
\end{ex}

	\begin{ex}%[0H3B1-2]%[Yến Trần BG10]
		Định nghĩa nào sau đây là định nghĩa đường parabol?
		\choice
		{\True Cho điểm $F$ cố định và một đường thẳng $\Delta$ cố định không đi qua $F$. Parabol $(P)$ là tập hợp các điểm $M$ sao cho khoảng cách từ $M$ đến $F$ bằng khoảng cách từ $M$ đến $\Delta$}
		{Cho $F_{1}, F_{2}$ cố định với $F_{1}F_{2}=2c, (c>0)$. Parabol $(P)$ là tập hợp điểm $M$ sao cho $|MF_{1}-MF_{2}| =2a$ với $a$ là một số không đổi và $a<c$}
		{Cho $F_{1}, F_{2}$ cố định với $F_{1}F_{2}=2c, (c>0)$ và một độ dài $2a$ không đổi $(a>c)$. Parabol $(P)$ là tập hợp điểm $M$ sao cho $M \in (P) \Leftrightarrow MF_{1}+MF_{2} =2a$}
		{Cả ba định nghĩa trên đều không đúng định nghĩa của parabol}
		\loigiai{Cho điểm $F$ cố định và một đường thẳng $\Delta$ cố định không đi qua $F$. Parabol $(P)$ là tập hợp các điểm $M$ sao cho khoảng cách từ $M$ đến $F$ bằng khoảng cách từ $M$ đến $\Delta$.
		}
	\end{ex}

	\begin{ex}%[0H3Y1-1]%[Yến Trần BG10]
		Trong mặt phẳng $Oxy$, hệ số góc của đường thẳng $d\colon 2x + 3y - 5 = 0$ là
		\choice
		{$\dfrac{2}{3}$}
		{$\dfrac{3}{2}$}
		{$-\dfrac{3}{2}$}
		{\True $-\dfrac{2}{3}$}
		\loigiai{
			Ta có $2x + 3y - 5 = 0 \Leftrightarrow y = -\dfrac{2}{3}x + \dfrac{5}{3}$. Vậy hệ số góc của đường thẳng $d$ là $-\dfrac{2}{3}$.
		}
	\end{ex}
	
	\begin{ex}%[0H3Y2-1]%[Yến Trần BG10] 
		Trong mặt phẳng $Oxy$,  toạ độ tâm $I$ và bán kính $R$ của đường tròn $(C)\colon x^2 + y^2 - 2x + 4y - 1 = 0$ là
		\choice
		{$I(-1;2), R = \sqrt{6}$}
		{$I(1;-2), R = 6$}
		{\True $I(1;-2), R = \sqrt{6}$}
		{$I(1;2), R = \sqrt{6}$}
		\loigiai{
			Đường tròn $(C)\colon x^2 + y^2 - 2x + 4y - 1 = 0$ có tâm $I(1;-2)$ và bán kính $R = \sqrt{1^2 + (-2)^2 - (- 1)} = \sqrt{6}$.
		}
	\end{ex}
	

	\begin{ex}%[0H3Y1-1]%[Yến Trần BG10]
		Cho đường thẳng $(d)\colon x-3 y+9=0$. Véc-tơ nào sau đây là véc-tơ pháp tuyến của $(d)$? 
		\choice
		{$\vec{n}=(3 ;  -1)$}
		{$\vec{n}=(-3 ;  -1)$}
		{\True $\vec{n}=(1 ;  -3)$}
		{$\vec{n}=(-1 ;  -3)$}
		\loigiai{Đường thẳng $(d)\colon x-3 y+9=0$ có véc-tơ pháp tuyến   là $\vec{n}=(1 ;  -3)$. }
	\end{ex}
	
	\begin{ex}%[0H3B1-2]%[Yến Trần BG10]
		PTTS của đường thẳng đi qua điểm $A(1 ;  -2)$ và vuông góc với đường thẳng $\Delta\colon  4 x-3 y+5=0$ là
		\choice
		{$\heva{&x=1-4 t \\& y=-2-3t}$}
		{\True $\heva{&x=1+4 t \\ &y=-2-3t}$}
		{$\heva{&x=1+4 t \\& y=2-3t}$}
		{$\heva{&x=1+4 t \\ &y=-2+3t}$}
		\loigiai{Đường thẳng vuông góc với đường thẳng $\Delta\colon  4 x-3 y+5=0$ có véc-tơ pháp tuyến là $\vec{a}=\vec{n}_{\Delta}=(4;  -3)$.\\
			PTTS của đường thẳng đi qua điểm $A(1 ;  -2)$ và có véc-tơ pháp tuyến là $\vec{a}=(4;  -3)$ là $\heva{&x=1+4t\\&y=-2-3t.}$
		}
	\end{ex}
	
	\begin{ex}%[0H3B2-1]%[Yến Trần BG10]
		Phương trình nào dưới đây là phương trình của đường tròn?
		\choice
		{$(x+5)^2-(y+3)^2=16$}
		{\True $(x+1)^2+y^2=2$}
		{$x^2+2 y^2-2 x+4 y-1=0$}
		{$x^2+y^2-4 x+2 y+30=0$}
		\loigiai{Đường tròn có tâm $I(a; b)$ và bán kính $R$ là 
			$(x-a)^2+(y-b)^2=R^2$.\\
			Hoặc $x^2+y^2-2ax-2by+c=0$ với điều kiện $a^2+b^2-c>0$.\\
			Từ đó ta loại phương án $(x+5)^2-(y+3)^2=16$ và $x^2+2 y^2-2 x+4 y-1=0$.\\
			Phương án $x^2+y^2-4 x+2 y+30=0$ có $a^2+b^2-c=2^2+(-1)^2-30<0$ nên loại.\\
			Vậy chỉ có phương án $(x+1)^2+y^2=2$ là đúng.
		}
	\end{ex}
	
	\begin{ex}%[0H3B2-1]%[Yến Trần BG10]
		Cho đường tròn $(C)\colon  x^2+y^2-4x+12y-9=0$. Tọa độ tâm $I$ và bán kính của đường tròn $(C)$ có tọa độ là
		\choice
		{\True $I(2 ;  -6) ;   R=7$}
		{$I(-2 ;   6) ;   R=7$}
		{$I(2 ;  -6) ;   R=49$}
		{$I(-2 ;   6) ;   R=49$}
		\loigiai{ Đường tròn $(C)\colon  x^2+y^2-4x+12y-9=0$ có  tâm $I(2;-6)$ và bán kính $R=7$.}
	\end{ex}
	
	\begin{ex}%[0H3Y1-1]%[Yến Trần BG10]
		Trong mặt phẳng tọa độ $Oxy$, cho đường thẳng $d\colon \heva{&x=2+3t\\&y=1-4t}$, một véc-tơ chỉ phương của đường thẳng $d$ có tọa độ là
		\choice
		{$(3;4)$}
		{$(4;3)$}
		{$(2;1)$}
		{\True $(3;-4)$}
		\loigiai{
			Đường thẳng $d\colon \heva{&x=2+3t\\&y=1-4t}$ có một véc-tơ chỉ phương là $\overrightarrow{a}=(3;-4)$.
		}
	\end{ex}
	\begin{ex}%[0H3Y1-1]%[Yến Trần BG10]
		Trong mặt phẳng tọa độ $Oxy$, cho đường thẳng $d\colon 2x-3y+1=0$, một véc-tơ pháp tuyến của $d$ có tọa độ là
		\choice
		{$(-3;2)$}
		{$(2;3)$}
		{$(3;2)$}
		{\True $(2;-3)$}
		\loigiai{
			Đường thẳng $d\colon 2x-3y+1=0$ có một véc-tơ pháp tuyến là $\overrightarrow{n}=(2;-3)$.
		}
	\end{ex}
	\begin{ex}%[0H3Y2-1]%[Yến Trần BG10]
		Trong mặt phẳng tọa độ $Oxy$, cho đường tròn có phương trình $x^2+y^2-4x+2y+3=0$, bán kính của đường tròn bằng
		\choice
		{$7$}
		{$\sqrt{7}$}
		{\True $\sqrt{2}$}
		{$2$}
		\loigiai{
			Đường tròn với phương trình $x^2+y^2-4x+2y+3=0$ có $a=\dfrac{-4}{-2}=2$, $b=\dfrac{2}{-2}=-1$, $c=3$.\\
			Vậy đường tròn có tâm $I(2;-1)$ và bán kính $R=\sqrt{a^2+b^2-c}=\sqrt{2^2+(-1)^2-3}=\sqrt{2}$.
		}
	\end{ex}
	\begin{ex}%[0H3Y1-5]%[Yến Trần BG10]
		Trong mặt phẳng tọa độ $Oxy$, cho điểm $A(-1;4)$ và đường thẳng $d$ có phương trình $3x+4y-5=0$, khoảng cách từ điểm $A$ đến đường thẳng $d$ bằng
		\choice
		{$\dfrac{2}{5}$}
		{\True $\dfrac{8}{5}$}
		{$\dfrac{4}{5}$}
		{$\dfrac{8}{25}$}
		\loigiai{
			Ta có $\mathrm{d} (A;d)=\dfrac{|3\cdot (-1)+4\cdot (4)-5|}{\sqrt{3^2+4^2}}=\dfrac{8}{5}$.
		}
	\end{ex}
	
	\begin{ex}%[0H3B2-3]%[Yến Trần BG10]
		Trong mặt phẳng tọa độ $Oxy$, cho đường tròn có phương trình $(C)\colon x^2+y^2-2x-4y-3=0$, phương trình tiếp tuyến của $(C)$ tại $M(3;4)$ là
		\choice
		{\True $x+y-7=0$}
		{$x+y+7=0$}
		{$x-y-7=0$}
		{$x+y-3=0$}
		\loigiai{
			Đường tròn $(C)\colon x^2+y^2-2x-4y-3=0$ có tâm $I(1;2)$.\\
			Tiếp tuyến $(\Delta)$ của $(C)$ tại $M$ nhận véc-tơ $\overrightarrow{IM}=(2;2)$ làm một véc-tơ pháp tuyến.\\
			Phương trình tiếp tuyến $(\Delta)$ là $2(x-3)+2(y-4)=0\Leftrightarrow x+y-7=0$.
		}
	\end{ex}
	\begin{ex}%[0H3B1-3]%[Yến Trần BG10]
		Trong mặt phẳng tọa độ $Oxy$, cho hai đường thẳng $\Delta\colon \heva{&x=1+3t\\&y=-1-2t}$, $d\colon 6x-4y-2=0$. Chọn phát biểu đúng.
		\choice
		{$\Delta$ cắt $d$ nhưng không vuông góc với $d$}
		{\True $\Delta$ vuông góc với $d$}
		{$\Delta\parallel d$}
		{$\Delta\equiv d$}
		\loigiai{
			Đường thẳng $\Delta$ có một véc-tơ chỉ phương là $\overrightarrow{a}=(3;-2)$.\\
			Đường thẳng $d$ có một véc-tơ chỉ phương là $\overrightarrow{u}=(4;6)$.\\
			Ta thấy $\overrightarrow{a}\cdot \overrightarrow{u}=0\Rightarrow \overrightarrow{a}\perp \overrightarrow{u}$.\\
			Do đó đường thẳng $\Delta$ vuông góc với $d$.
		}
	
	\end{ex}
\begin{ex}%[0H3B1-4]%[Yến Trần BG10]
	Góc giữa hai đường thẳng $d\colon x-3y+2=0$ và $\Delta \colon 3x+y-1=0$ là
	\choice
	{$30^\circ$}
	{$120^\circ$}
	{$60^\circ$}
	{\True $90^\circ$}
	\loigiai{
		Ta có $\vec{n}_d=(1;-3)$, $\vec{n}_\Delta=(3;1) \Rightarrow \vec{n}_d\cdot \vec{n}_\Delta=0\Rightarrow (d,\Delta)=90^\circ$.	
	}
\end{ex}


\noindent\textbf{II. PHẦN TỰ LUẬN}
\begin{bt}%[0H3K1-6]%[Yến Trần BG10]
	Cho đường thẳng $\Delta\colon\heva{&x=3-2t \\ &y=-1+t}$ và đường thẳng $d\colon 3x-4y-8=0$. Điểm $M(a;b)$ thuộc $\Delta$ sao cho khoảng cách từ $A$ đến $d$ bằng $3$. Biết $a$ dương, tính $T=a-b$.
	\loigiai{
		Vì $M$ thuộc $\Delta$ và có $a>0$ nên $M(3-2t;-1+t)$ với điều kiện $3-2t>0\Rightarrow t<\dfrac{3}{2}$.\\
		Theo đề ta có
		\begin{eqnarray*}
			\mathrm{d}(M,d)=3&\Leftrightarrow&\dfrac{\left|3(3-2t) -4(-1+t)-8\right| }{\sqrt{3^2+(-4)^2}}=3\\
			&\Leftrightarrow&\dfrac{\left| 5-10t\right| }{5}=3\\
			&\Leftrightarrow&\left|5-10t \right| =15\\
			&\Leftrightarrow&\hoac{&5-10t=15\\&5-10t=-15}\\
			&\Leftrightarrow&\hoac{&t=-1\text{ (nhận)}\\&t=2\text { (loại).}}
		\end{eqnarray*}
		Với $t=-1$ suy ra $M(5;-2)$ nên $a=5$, $b=-2$.\\
		Vậy $T=a-b=5-(-2)=7$.
	}
\end{bt}
\begin{bt}%[0H3K1-2]%[Yến Trần BG10]
	Trong mặt phẳng tọa độ $Oxy$, cho đường tròn $(C) \colon (x-2)^2 + (y-1)^2=4$ và đường thẳng $d \colon 2x+y+m=0$. Viết PTĐT $\Delta$ đi qua tâm đường tròn $(C)$ và vuông góc với đường thẳng $d$.
	\loigiai{
		Đường tròn $(C)$ có tâm $I(2;1)$.\\
		Đường thẳng $d$ có một véc-tơ chỉ phương là $\vec{u}=(-1;2)$.\\
		Vì $\Delta$ vuông góc với $d$ nên $\Delta$ nhận $\vec{u}=(-1;2)$ làm véc-tơ pháp tuyến. \\
		Mà $\Delta$ đi qua điểm $I(2;1)$ nên $\Delta$ có phương trình tổng quát là $-1(x-2) + 2(y-1)=0 \Leftrightarrow x-2y=0$.
	}
\end{bt}


	\begin{bt}%[0H3K2-4]%[Yến Trần BG10]
		Đường tròn $x^2+y^2-2x+4y-4=0$ cắt đường thẳng $\Delta\colon x+y-2=0$ theo một dây cung có độ dài bằng bao nhiêu?
		
		\loigiai{
			\immini{
				Đường tròn $x^2+y^2-2x+4y-4=0$ có tâm $I(1;-2)$ và\\
				bán kính $R=\sqrt{1^2+(-2)^2-(-4)}=3$.
				Giả sử đường tròn cắt đường thẳng $\Delta$ theo dây cung $AB$. Gọi $H$ là trung điểm đoạn $AB$. Khi đó $IH\perp AB$, do đó $$IH=\mathrm{d}_{(I;\Delta)}=\dfrac{|1+(-2)-2|}{\sqrt{1^2+1^2}}=\dfrac{3}{\sqrt{2}}.$$
				Theo định lý Py-ta-go, ta có
				$$HA=\sqrt{IA^2-IH^2}=\sqrt{3^2-\left( \dfrac{3}{\sqrt{2}}\right)^2 }=\dfrac{3\sqrt{2}}{2}.$$
				Do đó $AB=2IH=3\sqrt{2}$.
			}
			{\begin{tikzpicture}[scale=1, font=\footnotesize, line join=round, line cap=round, >=stealth]
					\tkzDefPoints{0/0/I,0/-1/C,1/-1/D}
					\draw (I) circle [radius=2];
					\tkzInterLC[R](C,D)(I,2cm)
					\tkzGetPoints{A}{B}
					\tkzDrawLines[add=.3 and 0.1](A,B)
					\tkzLabelLine[pos=-.2,above](A,B){$\Delta$}
					\coordinate (H) at ($(A)!(I)!(B)$);
					\tkzDrawPoints[fill=black](I,A,B,H)
					\tkzDrawSegments(I,H I,A)
					\tkzLabelSegment[pos=.5,above](I,A){$R$}
					\tkzMarkRightAngles[size=0.2](I,H,A)
					\tkzLabelPoints[above](I)
					\tkzLabelPoints[below left](B)
					\tkzLabelPoints[below right](A)
					\tkzLabelPoints[below](H)
			\end{tikzpicture}}
		}
	\end{bt}
	\begin{bt}%[0H3G1-6]%[Yến Trần BG10]
		Trong mặt phẳng tọa độ $Oxy$, cho điểm ${M}(1 ;   4)$. Viết PTĐT đi qua ${M}$ cắt tia $Ox,\, Oy$ lần lượt tại ${A}, \, {B}$ sao cho tam giác ${AOB}$ có diện tích nhỏ nhất.
		\loigiai
		{
			Vì $A \in Ox;  B\in Oy$ nên đặt $A(x_A;  0)$ và $B(0; y_B)$ với $x_A>0$, $y_B>0$.\\
			Đường thẳng qua $A$ và $B$ là $\dfrac{x}{x_A}+\dfrac{y}{y_B}=1$.\\
			Do $M\in AB$ nên $\dfrac{1}{x_A}+\dfrac{4}{y_B}=1\Rightarrow y_B=\dfrac{4x_A}{x_A-1}$.\\
			Do $x_A>0$ và $y_B>0$ nên $x_A-1>0 \Leftrightarrow x_A>1$. \\
			Diện tích tam giác $OAB$ là $S=\dfrac{1}{2}\cdot|x_A|\cdot|y_B|=\dfrac{1}{2}\cdot\dfrac{4x_A^2}{x_A-1}=\dfrac{2x_A^2}{x_A-1}$.\\
			Ta có $S=\dfrac{2x_A^2}{x_A-1}=2x_A+2+\dfrac{2}{x_A-1}=2(x_A-1)+\dfrac{2}{x_A-1}+4\geq2\cdot\sqrt{2\cdot 2}+4=8$. \\
			Đẳng thức xẩy ra khi $2\left(x_A-1\right)=\dfrac{2}{x_A-1} \Leftrightarrow \left(x_A-1\right)^2=1 \Leftrightarrow x_A=2$ (vì $x_A>1$). \\
			Với $x_A=2$ thì $y_B=8$. \\
			Khi đó $S_{\min}=8$ khi $x_A=2$ và $ y_B=8$. \\
			Vậy đường thẳng qua $M$ cắt tia $Ox,\, Oy$ lần lượt tại ${A}, \, {B}$ sao cho tam giác ${AOB}$ có diện tích nhỏ nhất có phương trình là $\dfrac{x}{2}+\dfrac{y}{8}=1$.
		}
	\end{bt}
\Closesolutionfile{ans}
\Closesolutionfile{ansbook}
\indapan{10}{ans/ans-KT-702}