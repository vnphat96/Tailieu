% \setcounter{section}{20}
\section{PHƯƠNG TRÌNH ĐƯỜNG TRÒN}
\subsection{Tóm tắt lí thuyết}
%\Opensolutionfile{ans}[ans/0H7-21]
\setcounter{ex}{0}
\setcounter{bt}{0}
\begin{itemize}
	\item Phương trình của đường tròn $(C)$ có tâm $I(a;b)$, bán kính $R$ là
	$$(x-a)^2+(y-b)^2=R^2.$$
	\item Với các hằng số $a$, $b$, $c$ thoả mãn $a^2+b^2-c>0$, phương trình
	$$x^2+y^2-2ax-2by+c=0$$
	là phương trình của một đường tròn có tâm $I(a;b)$ và có bán kính $R=\sqrt{a^2+b^2-c}$.
	\item Cho đường tròn $(C)$ có tâm $I(a;b)$, bán kính $R$. Phương trình tiếp tuyến $\Delta$ của $(C)$ tại $M_0\left(x_0; y_0\right)$ là $\left(a-x_0\right) \cdot \left(x-x_{0}\right)+\left(b-y_{0}\right) \cdot\left(y-y_{0}\right)=0$.
\end{itemize}
\subsection{Các dạng toán}
\begin{dang}{Xác định tâm và bán kính đường tròn}
	\begin{itemize}
		\item Nếu phương trình đường tròn có dạng $(C)\colon \left(x-a\right)^2+\left(y-b\right)^2=R^2$ thì $(C)$ có tâm là $I(a;b)$ và bán kính bằng $R$.
		\item Nếu phương trình đường tròn có dạng $(C)\colon x^2+y^2-2ax-2by+c=0$ thì tâm $I$ được xác định 
		$\heva{&-2a=\cdots\\&-2b=\cdots} \Leftrightarrow \heva{&a=\cdots\\&b=\cdots }\Rightarrow I(a;b)$ và bán kính $R=\sqrt{a^2+b^2-c}$.
	\end{itemize}
	\begin{note}
		\begin{itemize}
			\item Phương trình $x^2+y^2-2ax-2by+c=0$ là phương trình đường tròn khi và chỉ khi $a^2+b^2-c>0$.
			\item Điều kiện đường thẳng $\Delta$ tiếp xúc với đường tròn $(I,R)$ là $\mathrm{d}\left(I,\Delta\right)=R$.
		\end{itemize}
	\end{note}
\end{dang}
\subsubsection{Ví dụ}
\begin{vd}%[0H3Y2-1]
	Trong các phương trình sau, phương trình nào là phương trình đường tròn. Tìm tâm và bán kính của đường tròn đó.
	\begin{enumEX}{2}
		\item $x^2+y^2-2x-2y-2=0$.
		\item $x^2+y^2+2x-8y+1=0$.
		\item $16x^2+16y^2+16x-8y=11$.
		\item $7x^2+7y^2-4x+6y-1=0$.
	\end{enumEX}
	\loigiai{
		\begin{listEX}[1]
			\item Ta có
			$$x^2+y^2-2x-2y-2=0\Leftrightarrow \left(x-1\right)^2+\left(y-1\right)^2=\left(-1\right)^2+\left(-1\right)^2+2\Leftrightarrow \left(x-1\right)^2+\left(y-1\right)^2=4.$$ 
			Suy ra tâm $I(1;1)$ và bán kính $R=2$. 
			\item Ta có
			$$x^2+y^2+2x-8y+1=0\Leftrightarrow (x+1)^2+\left(y-4\right)^2=1^2+4^2-1\Leftrightarrow (x+1)^2+\left(y-4\right)^2=16.$$ 
			Suy ra tâm $I\left(-1;4\right)$ và bán kính $R=4$.
			\item Ta có
			$$16x^2+16y^2+16x-8y=11\Leftrightarrow x^2+y^2+x-\dfrac{1}{2}y=\dfrac{11}{16}\Leftrightarrow \left(x+\dfrac{1}{2}\right)^2+\left(y-\dfrac{1}{4}\right)^2=1.$$ 
			Suy ra tâm $I\left(\dfrac{-1}{2};\dfrac{1}{4}\right)$ và bán kính $R=1$. 			
			\item Ta có
			$$7x^2+7y^2-4x+6y-1=0\Leftrightarrow x^2+y^2-\dfrac{4}{7}x+\dfrac{6}{7}y-\dfrac{1}{7}=0\Leftrightarrow \left(x-\dfrac{2}{7}\right)^2+\left(y+\dfrac{3}{7}\right)^2=\dfrac{20}{49}.$$ 
			Suy ra tâm $I\left(\dfrac{2}{7};-\dfrac{3}{7}\right)$ và bán kính $R=\dfrac{2\sqrt{5}}{7}$.
		\end{listEX}
		
	}
\end{vd}

\begin{vd}%[0H3Y2-1]
	Tìm tọa độ tâm $I$ và bán kính $R$ của đường tròn $(x-2)^2+(y+3)^2=5$.
	
	\loigiai{Đường tròn $(x-2)^2+(y+3)^2=5$ có tâm $I(2;-3)$ và bán kính $R=\sqrt{5}$.
	}
\end{vd}

\begin{vd}%[0H3Y2-1]
	Trong mặt phẳng tọa độ $Oxy$, cho đường tròn $(C) \colon x^2+y^2+4x-2y-7=0$. Tìm tọa độ tâm $I$ và bán kính của đường tròn $(C)$.
	\loigiai{
		Đường tròn $(C)$ có tâm $I(-2;1)$ và có bán kính $R=\sqrt{(-2)^2+1^2+7}=2\sqrt{3}$.}
\end{vd}

\begin{vd}%[0H3Y2-1]
	Trong mặt phẳng với hệ tọa độ $Oxy$, cho đường tròn $(C) \colon x^2+y^2-2x+10y+1=0$. Trong các điểm $M(-1;3),N(4;-1),P(2;1),Q(3;-2)$, điểm nào thuộc $(C)$?
	\loigiai{
		Thay tọa độ các điểm vào phương trình của $(C)$ thì chỉ có điểm $N$ thỏa mãn phương trình đường tròn. Vậy điểm $N \in (C)$.}
\end{vd}

\subsubsection{Bài tập tự luận}
\begin{bt}%[0H3Y2-1]
	 Tìm tâm và bán kính của đường tròn $(C)$ trong các trường hợp sau:
	\begin{enumEX}{2}
		\item $(x-2)^2+(y-8)^2=49$;
		\item $(x+3)^2+(y-4)^2=23$.
	\end{enumEX}
	\loigiai{
		\begin{listEX}[1]
			\item Từ phương trình đường tròn ta có tâm $I(2;8)$, bán kính $R=7$.
			\item Từ phương trình đường tròn ta có tâm  $I(-3;4)$, bán kính $R=\sqrt{23}$.
		\end{listEX}
	}
\end{bt}

\begin{bt}%[0H3Y2-1]
	Phương trình nào dưới đây là phương trình của một đường tròn? Khi đó hãy tìm tâm và bán kính của nó.
	\begin{enumEX}{2}
		\item $x^2+2y^2-4x-2y+1=0$;
		\item $x^2+y^2-4 x+3y+2xy=0$;
		\item  $x^2+y^2-8x-6y+26=0$;
		\item  $x^2+y^2+6x-4y+13=0$;
		\item  $x^2+y^2-4x+2y+1=0$.
	\end{enumEX}
	\loigiai{
		\begin{listEX}[1]
			\item Phương trình đã cho không là phương trình của đường tròn (hệ số của $x^2$ và $y^2$ không bằng nhau).
			\item Phương trình đã cho không là phương trình của đường tròn (trong phương trình của đường tròn không có thành phần tích $x \cdot y$).
			\item Phương trình đã cho có các hệ số $a=4$, $b=3$, $c=26$, suy ra $a^2+b^2-c=3^2+4^2-26=-1<0$, do đó nó không là phương trình của đường tròn.
			\item Phương trình đã cho có các hệ số $a=-3$, $b=2$, $c=13$, suy ra $a^2+b^2-c=(-3)^2+2^2-13=0$, do đó nó không là phương trình của đường tròn.
			\item Phương trình đã cho có các hệ số $a=2$, $b=-1$, $c=1$ thoả mãn \break $a^2+b^2-c=2^2+(-1)^2-1=4>0$, nên là phương trình của đường tròn có tâm $I(2 ;-1)$ và có bán kính $R=\sqrt{4}=2$.
		\end{listEX}
	}
\end{bt}
\begin{bt}%[0H3K2-1]
	Tìm $m$ để các phương trình sau là phương trình đường tròn.
	\begin{enumerate}
		\item $x^2+y^2+4mx-2my+2m+3=0$.
		\dapso{$\left(-\infty;-\dfrac{3}{5}\right)\cup \left(1;+\infty\right)$}
		\item $x^2+y^2-2\left(m-3\right)x+4my-m^2+5m+4=0$.
		\dapso{$\left(-\infty;\dfrac{5}{6}
			\right)\cup \left(1;+\infty\right)$}
	\end{enumerate}
	\loigiai{
		\begin{enumerate}
			\item Phương trình $x^2+y^2+4mx-2my+2m+3=0$ là phương trình đường tròn khi và chỉ khi
			$$4m^2+m^2-(2m+3)>0\Leftrightarrow 5m^2-2m-3>0\Leftrightarrow \hoac{&m<-\dfrac{3}{5}\\&m>1.}$$
			Vậy tập hợp các giá trị $m$ cần tìm là $\left(-\infty;-\dfrac{3}{5}\right)\cup \left(1;+\infty\right)$.
			
			\item Phương trình $x^2+y^2-2\left(m-3\right)x+4my-m^2+5m+4=0$ là phương trình đường tròn khi và chỉ khi
			$$\left(m-3\right)^2+4m^2+m^2-5m-4>0\Leftrightarrow 6m^2-11m+5>0\Leftrightarrow \hoac{&m<\dfrac{5}{6}\\	&m>1.}$$
			Vậy tập hợp các giá trị $m$ cần tìm là $m\in \left(-\infty;\dfrac{5}{6}
			\right)\cup \left(1;+\infty\right)$.
		\end{enumerate}
		
	}
\end{bt}


\begin{dang}{Viết phương trình đường tròn}
	\textbf{Phương pháp:} Để viết phương trình đường tròn ta thường đi theo một trong hai hướng
	\begin{itemize}
		\item Tìm tâm $I(a;b)$ và bán kính $R$. Khi đó phương trình đường tròn là $(x-a)^2+(y-b)^2=R^2$.
		\item Gọi phương trình của đường tròn là $x^2+y^2+2ax+2by+c=0$.\\
		Từ điều kiện của đề bài đưa đến hệ phương trình với ẩn số $a$, $b$, $c$.\\
		Giải hệ phương trình tìm $a$, $b$, $c$, từ đó ta có phương trình đường tròn.
	\end{itemize} 
\end{dang}

\begin{vd}%[Quan Văn Ón, BG10-KNTT-Tập 2]%[0H3Y2-2]
	Trong mặt phẳng $Oxy$, đường tròn $(C)$ tâm $I(-2;5)$ bán kính $R = 7$. Viết phương trình đường tròn $(C)$.
	\loigiai
	{
		Phương trình đường tròn $(C)$ tâm $I(-2;5)$ bán kính $R = 7$ là $(x + 2)^2 + (y - 5)^2 = 49$.
	}
\end{vd}

\begin{vd}%[Quan Văn Ón, BG10-KNTT-Tập 2]%[0H3B2-2]
	Viết phương trình đường tròn $(C)$ có tâm $I(1;-2)$ và đi qua $A(-2;2)$.
	\loigiai{
		\immini{
			Đường tròn $(C)$ có
			\begin{itemize}
				\item Tâm là điểm $I(1;-2)$.
				\item Bán kính $R = IA = \sqrt{(-2-1)^2 + (2 + 2)^2} = 5$.
			\end{itemize}
			Vậy phương trình đường tròn $(C)$ là $(x - 1)^2 + (y + 2)^2 = 25$.
		}{
			\begin{tikzpicture}[>=stealth,line join=round,line cap=round,font=\footnotesize,scale=1]
				\clip (-2.1,-2.1) rectangle (2.3,2.1);
				\path
				(0,0) coordinate (I)
				($ (I) + (0:2) $) coordinate (A);
				\draw (I) circle (2);
				
				\foreach \d/\g in{I/-90,A/-45} 
				\fill[black] (\d) circle (1.3pt) node at ($(\d)+(\g:3mm)$){$\d$};
				\draw (I)--(A);
				\fill ($(I)!0.5!(A)$) node[above]{$R$};
			\end{tikzpicture}
		}
	}
\end{vd}

\begin{vd}%[Quan Văn Ón, BG10-KNTT-Tập 2]%[0H3Y2-2]
	Viết phương trình đường tròn $(C)$ có đường kính $AB$, với $A(-1;-3)$, $B(-3;5)$.
	\loigiai{
		Trung điểm của $AB$ là $I(-2;1)$ và đường kính $AB = \sqrt{(-3+1)^2 + (5 + 3)^2} = 2\sqrt{27}$.
		\immini{
			Đường tròn $(C)$ có
			\begin{itemize}
				\item Tâm là điểm $I(-2;1)$.
				\item Bán kính $R = \dfrac{AB}{2} = \dfrac{2\sqrt{27}}{2} = \sqrt{27}$.
			\end{itemize}
			Vậy phương trình đường tròn $(C)$ là $(x + 2)^2 + (y - 1)^2 = 27$.
		}{
			\begin{tikzpicture}[>=stealth,line join=round,line cap=round,font=\footnotesize,scale=1]
				\clip (-2.5,-2.1) rectangle (2.5,2.1);
				\path
				(0,0) coordinate (I)
				($ (I) + (180:2) $) coordinate (A)
				($ (I) + (0:2) $) coordinate (B);
				\draw (I) circle (2);
				
				\foreach \d/\g in{I/-90,A/-135,B/-45} 
				\fill[black] (\d) circle (1.3pt) node at ($(\d)+(\g:3mm)$){$\d$};
				\draw (A)--(B);
			\end{tikzpicture}
		}
	}
\end{vd}

\begin{vd}%[Quan Văn Ón, BG10-KNTT-Tập 2]%[0H3B2-2]
	Viết phương trình đường tròn $(C)$ có tâm $I(1;3)$ và tiếp xúc với đường thẳng $\Delta\colon x + 2y + 3 = 0$.
	\loigiai{
		\immini{
			Đường tròn $(C)$ tâm $I(1;3)$ và tiếp xúc với đường thẳng $\Delta\colon x + 2y + 3 = 0$ nên có bán kính
			$$ R = \mathrm{d}\left(I,\Delta\right)=\dfrac{\left|1 + 2\cdot 3 + 3 \right| }{\sqrt{1^2 + 2^2}}=\dfrac{10}{\sqrt{5}} = 2\sqrt{5}.$$
			Vậy phương trình đường tròn $(C)$ là $(x-1)^2 + (y-3)^2 = 20$.
		}{
			\begin{tikzpicture}[>=stealth,line join=round,line cap=round,font=\footnotesize,scale=1]
				\clip (-3,-3) rectangle (3,2.1);
				\path
				(0,0) coordinate (I)
				(-2.5,-2) coordinate (A)
				(2.5,-2) coordinate (B)
				($(A)!(I)!(B)$) coordinate (H);
				\draw (I) circle (2);
				
				\foreach \d/\g in{I/90} 
				\fill[black] (\d) circle (1.3pt) node at ($(\d)+(\g:3mm)$){$\d$};
				\draw (A)--(B) (I)--(H);
				\draw pic [draw, angle radius=2mm] {right angle=I--H--B};
				\fill ($(I)!0.5!(H)$) node[left]{$R$};
				\fill (B) node[below]{$\Delta$};
			\end{tikzpicture}
		}
	}
\end{vd}

\begin{vd}%[Quan Văn Ón, BG10-KNTT-Tập 2]%[0H3B2-2]
	Viết phương trình đường tròn $(C)$ có tâm $I(1;-2)$ và tiếp xúc với trục $Ox$.
	\loigiai{
		Nhắc lại: Trục $Ox\colon y = 0$.\\
		Đường tròn $(C)$ tâm $I(1;-2)$ và tiếp xúc với trục $Ox\colon y = 0$ nên có bán kính
		$$ R = \mathrm{d}\left(I,\Delta\right)=\dfrac{\left| -2 \right| }{\sqrt{0^2 + 1^2}} = 2.$$
		Vậy phương trình đường tròn $(C)$ là $(x-1)^2 + (y+2)^2 = 4$.
	}
\end{vd}

\begin{vd}%[Quan Văn Ón, BG10-KNTT-Tập 2]%[0H3B2-5]
	Trong mặt phẳng $Oxy$, viết phương trình đường tròn có tâm nằm trên đường thẳng $y=x$ và đi qua hai điểm $A(3;0)$, $B(4;3)$.
	\loigiai{
		Gọi $I$ là tâm đường tròn. Vì $I$ thuộc đường thẳng $y=x$ nên $I(a; a)$. Ta có 
		\begin{eqnarray*}
			AI=BI &\Leftrightarrow & AI^2=BI^2\\
			&\Leftrightarrow &(a-3)^2+a^2=(a-4)^2+(a-3)^2\\
			&\Leftrightarrow & a^2-6a+9+a^2=a^2-8a+16+a^2-6a+9\\
			&\Leftrightarrow & a=2.
		\end{eqnarray*}
		Vậy $I(2; 2)$ và bán kính $R=AI=\sqrt{(3-2)^2+2^2}=\sqrt{5}$.\\
		Phương trình đường tròn cần lập là $(x-2)^2+(y-2)^2=5$.
	}
\end{vd}

\begin{vd}%[Quan Văn Ón, BG10-KNTT-Tập 2]%[0H3B2-2]
	Lập phương trình đường tròn $(C)$ đi qua ba điểm $A(-1;1)$, $B(0;-2)$, $C(0;2)$.
	\loigiai{
		\textbf{Cách 1:}\\
		Gọi tâm đường tròn $(C)$ là điểm $I(a;b)$.\\
		Ta có $IA = IB = IC \Leftrightarrow IA^2 = IB^2 = IC^2$.\\
		Vì $IA^2 = IB^2$, $IB^2 = IC^2$ nên
		\begin{eqnarray*}
			& & \heva{&(-1-a)^2 + (1-b)^2 = (0-a)^2 + (-2-b)^2\\&(0-a)^2 + (-2-b)^2 = (0-a)^2 + (2-b)^2}\\
			&\Leftrightarrow& \heva{&a^2 + b^2 + 2a - 2b + 2 = a^2 + b^2 + 4b + 4\\&a^2 + b^2 + 4b + 4 = a^2 + b^2 - 4b + 4}\\
			&\Leftrightarrow& \heva{&a - 3b = 1\\&b = 0}\\
			&\Leftrightarrow& \heva{&a = 1\\&b = 0.}
		\end{eqnarray*}
		Đường tròn tâm $I(1;0)$ bản kính $R = IC = \sqrt{(0-1)^2 + (2-0)^2} = \sqrt{5}$.\\
		Vậy phương trình đường tròn $(C)$ là $(x - 1)^2 + y^2 = 5$.\\
		\textbf{Cách 2:}\\
		Phương trình của đường tròn $(C)$ có dạng là $x^2 + y^2 - 2ax - 2by + c = 0$.\\
		Ta có
		\begin{eqnarray*}
			\heva{&A(-1;1) \in (C)\\&B(0;-2) \in (C)\\&C(0;2) \in (C)} \Leftrightarrow \heva{&(-1)^2 + 1^2 - 2a\cdot (-1) - 2b\cdot 1 + c = 0\\&0^2 + (-2)^2 - 2a\cdot 0 - 2b\cdot (-2) + c = 0\\&0^2 + 2^2 - 2a\cdot 0 - 2b\cdot 2 + c = 0} &\Leftrightarrow& \heva{&2a - 2b + c = -2\\&4b + c = -4\\&-4b + c = -4}\\
			&\Leftrightarrow& \heva{&a = 1\\&b = 0\\&c = -4.}
		\end{eqnarray*}
		Vậy phương trình đường tròn $(C)$ là $x^2 + y^2 - 2x - 4 = 0$.
	}
\end{vd}

\subsubsection{Bài tập tự luyện}

\begin{bt}%[Quan Văn Ón, BG10-KNTT-Tập 2]%[0H3B2-2]
	Lập phương trình đường tròn $(C)$ trong các trường hợp sau
	\begin{enumerate}
		\item $(C)$ có tâm $I(1;3)$ và bán kính $R = 2$.
		\item $(C)$ có tâm $I(3;5)$ và qua điểm $A(7;2)$.
		\item $(C)$ có đường kính $AB$ với $A(1;1)$, $B(7;5)$.
	\end{enumerate} 
	\loigiai{
		\begin{enumerate}
			\item Phương trình đường tròn $(C)$ có tâm $I(1;3)$ và bán kính $R = 2$ là $(x - 1)^2 + (y - 3)^2 = 4$.
			\item Đường tròn $(C)$ có
			\begin{itemize}
				\item Tâm là điểm $I(3;5)$.
				\item Bán kính $R = IA = \sqrt{(7-3)^2 + (2-5)^2} = 5$. 
			\end{itemize}
			Vậy phương trình đường tròn $(C)$ là $(x - 3)^2 + (y - 5)^2 = 25$.
			\item Trung điểm của $AB$ là $I(4;3)$ và đường kính $AB = \sqrt{(7 - 1)^2 + (5 - 3)^2} = 2\sqrt{10}$.\\
			Đường tròn $(C)$ có
			\begin{itemize}
				\item Tâm là điểm $I(4;3)$.
				\item Bán kính $R = \dfrac{AB}{2} = \dfrac{2\sqrt{10}}{2} = \sqrt{10}$.
			\end{itemize}
			Vậy phương trình đường tròn $(C)$ là $(x - 4)^2 + (y - 3)^2 = 10$.
		\end{enumerate}
	}
\end{bt}

\begin{bt}%[Quan Văn Ón, BG10-KNTT-Tập 2]%[0H3B2-5]
	Lập phương trình đường tròn $(C)$ trong các trường hợp sau
	\begin{enumerate}
		\item $(C)$ có tâm $I(2;-1)$ và tiếp xúc với đường thẳng $\Delta \colon 3x - 4y - 20 = 0$.
		\item $(C)$ qua hai điểm $A(2;3)$, $B(-2;1)$ và có tâm nằm trên trục hoành.
	\end{enumerate} 
	\loigiai{
		\begin{enumerate}
			\item Đường tròn $(C)$ tâm $I(2;-1)$ và tiếp xúc với đường thẳng $\Delta\colon 3x - 4y - 20 = 0$ nên có bán kính
			$$ R = \mathrm{d}\left(I,\Delta\right)=\dfrac{\left|3\cdot 2 - 4\cdot (-1) - 20 \right| }{\sqrt{3^2 + (-4)^2}}=\dfrac{10}{5} = 2.$$
			Vậy phương trình đường tròn $(C)$ là $(x-2)^2 + (y+1)^2 = 4$.
			\item Gọi $I(a;b)$ là tâm đường tròn $(C)$. Vì $I$ nằm trên trục hoành nên $I(a,0)$. Ta có
			\begin{eqnarray*}
				AI = BI &\Leftrightarrow& AI^2 = BI^2\\
				&\Leftrightarrow& (a - 2)^2 + (0 - 3)^2 = (a + 2)^2 + (0 - 1)^2\\
				&\Leftrightarrow& a^2 - 4a + 4 + 9 = a^2 + 4a + 4 + 1\\
				&\Leftrightarrow& a = 1.
			\end{eqnarray*}
			Vậy $(C)$ có tâm $I(1;0)$ và bán kính $R = AI = \sqrt{(1-2)^2 + (0-3)^2} = \sqrt{10}$ nên có phương trình $(x - 1)^2 + y^2 = 10$.
		\end{enumerate}
	}
\end{bt}

\begin{bt}%[Quan Văn Ón, BG10-KNTT-Tập 2]%[0H3B2-5]
	Viết phương trình đường tròn $(C)$ có tâm thuộc đường thẳng $\Delta \colon x + y - 1 = 0$ và đi qua hai điểm $A(6;2)$, $B(-1;3)$.
	\loigiai{
		Gọi $I(a;b)$ là tâm đường tròn $(C)$. Vì $I$ đường thẳng $\Delta \colon x + y - 1 = 0$ nên $a + b - 1 = 0$ hay $b = -a + 1$, do đó $I(a; -a + 1)$. Ta có
		\begin{eqnarray*}
			AI = BI &\Leftrightarrow& AI^2 = BI^2\\
			&\Leftrightarrow& (a - 6)^2 + (-a + 1 - 2)^2 = (a + 1)^2 + (-a + 1 - 3)^2\\
			&\Leftrightarrow& a^2 - 12a + 36 + a^2 + 2a + 1 = a^2 + 2a + 1 + a^2 + 4a + 4\\
			&\Leftrightarrow& a = 2.
		\end{eqnarray*}
		Vậy $(C)$ có tâm $I(2;-1)$ và bán kính $R = AI = \sqrt{(2-6)^2 + (-1-2)^2} = 5$ nên có phương trình $(x-2)^2 + (y+1)^2 = 25$.
	}
\end{bt}

\begin{bt}%[Quan Văn Ón, BG10-KNTT-Tập 2]%[0H3B2-2]
	Lập phương trình đường tròn $(C)$ đi qua ba điểm
	\begin{enumerate}
		\item $A(2;6)$, $B(-6;2)$, $C(-1;-3)$.
		\item $A(1;2)$. $B(5;2)$, $C(1;-3)$.
	\end{enumerate}
	\loigiai{
		\begin{enumerate}
			\item Gọi tâm đường tròn $(C)$ là điểm $I(a;b)$.\\
			Ta có $AI = IB = CI \Leftrightarrow AI^2 = BI^2 = CI^2$.\\
			Vì $AI^2 = BI^2$, $BI^2 = CI^2$ nên
			\begin{eqnarray*}
				& & \heva{&(a-2)^2 + (b-6)^2 = (a+6)^2 + (b-2)^2\\&(a+6)^2 + (b-2)^2 = (a+1)^2 + (b+3)^2}\\
				&\Leftrightarrow& \heva{&a^2 + b^2 - 4a - 12b + 40 = a^2 + b^2 + 12a - 4b + 40\\&a^2 + b^2 + 12a - 4b + 40 = a^2 + b^2 + 2a + 6b + 10}\\
				&\Leftrightarrow& \heva{&2a + b = 0\\&a - b = -3}\\
				&\Leftrightarrow& \heva{&a = -1\\&b = 2.}
			\end{eqnarray*}
			Đường tròn tâm $I(-1;2)$ bản kính $R = AI = \sqrt{(-1-2)^2 + (2-6)^2} = 5$.\\
			Vậy phương trình đường tròn $(C)$ là $(x + 1)^2 + (y - 2)^2 = 25$.
			\item Gọi tâm đường tròn $(C)$ là điểm $I(a;b)$.\\
			Ta có $AI = IB = CI \Leftrightarrow AI^2 = BI^2 = CI^2$.\\
			Vì $AI^2 = BI^2$, $BI^2 = CI^2$ nên
			\begin{eqnarray*}
				& & \heva{&(a-1)^2 + (b-2)^2 = (a-5)^2 + (b-2)^2\\&(a-5)^2 + (b-2)^2 = (a-1)^2 + (b+3)^2}\\
				&\Leftrightarrow& \heva{&a^2 + b^2 - 2a - 4b + 5 = a^2 + b^2 - 10a - 4b + 29\\&a^2 + b^2 - 10a - 4b + 29 = a^2 + b^2 - 2a + 6b + 10}\\
				&\Leftrightarrow& \heva{&a = 3\\&8a + 10b = 19}\\
				&\Leftrightarrow& \heva{&a = 3\\&b = -\dfrac{1}{2}.}
			\end{eqnarray*}
			Đường tròn tâm $I\left( 3;-\dfrac{1}{2} \right)$ bản kính $R = AI = \sqrt{\left( 3 - 1 \right)^2 + \left( -\dfrac{1}{2} - 2 \right)^2} = \dfrac{\sqrt{41}}{2}$.\\
			Vậy phương trình đường tròn $(C)$ là $(x - 3)^2 + \left(y + \dfrac{1}{2} \right)^2 = \dfrac{41}{4}$.
		\end{enumerate}
	}
\end{bt}

\begin{bt}%[Quan Văn Ón, BG10-KNTT-Tập 2]%[0H3B2-4]
	Lập phương trình đường tròn $(C)$ trong các trường hợp sau
	\begin{enumerate}
		\item $(C)$ có tâm $I(2;-5)$ và tiếp xúc với $Ox$.
		\item $(C)$ có tâm $I(1;3)$ và tiếp xúc với $Oy$.
		\item $(C)$ tiếp xúc cả hai trục tọa độ và có tâm nằm trên đường thẳng $\Delta\colon 4x - 2y - 8 = 0$.
		\item $(C)$ tiếp xúc cả hai trục tọa độ và qua $M(2;1)$.
		\item $(C)$ qua $A(9;9)$ và tiếp xúc với trục $Ox$ tại $M(6;0)$.
		\item $(C)$ tiếp xúc với trục $Ox$ tại $A(2;0)$ và khoảng cách từ tâm của $(C)$ đến $B(6;4)$ bằng $5$.
	\end{enumerate}
	\loigiai{
		\begin{enumerate}
			\item Vì $(C)$ có tâm $I(2;-5)$ và tiếp xúc với $Ox$ nên bán kính $R = |-5| = 5$.\\
			Vậy phương trình đường tròn $(C)$ là $(x - 2)^2 + (y + 5)^2 = 25$.
			\item Vì $(C)$ có tâm $I(1;3)$ và tiếp xúc với $Oy$ nên bán kính $R = |1| = 1$.\\
			Vậy phương trình đường tròn $(C)$ là $(x - 1)^2 + (y - 3)^2 = 1$.
			\item Gọi tâm đường tròn $(C)$ là điểm $I(a;b)$.\\
			Vì $(C)$ tiếp xúc cả hai trục tọa độ nên $R = |a| = |b|$.\\
			\textbf{TH 1:} Nếu $a = b$ thì $I(a;a)$.\\
			Vì $I(a;a) \in \Delta\colon 4x - 2y - 8 = 0$ nên $4a - 2a - 8 = 0 \Rightarrow a = 4$.\\
			Do đó $I(4;4)$ và $R = |a| = 4$.\\
			Suy ra phương trình đường tròn $(C)$ là $(x-4)^2 + (y-4)^2 = 16$.\\
			\textbf{TH 2:} Nếu $a = -b$ thì $I(a;-a)$.\\
			Vì $I(a;-a) \in \Delta\colon 4x - 2y - 8 = 0$ nên $4a - 2\cdot (-a) - 8 = 0 \Rightarrow a = \dfrac{4}{3}$.\\
			Do đó $I\left( \dfrac{4}{3}; -\dfrac{4}{3} \right)$ và $R = \left|\dfrac{4}{3} \right|  = \dfrac{4}{3}$.\\
			Suy ra phương trình đường tròn $(C)$ là $\left( x - \dfrac{4}{3} \right)^2 + \left( y + \dfrac{4}{3} \right)^2 = \dfrac{16}{9}$.
			\item Gọi tâm đường tròn $(C)$ là điểm $I(a;b)$.\\
			Vì $(C)$ tiếp xúc cả hai trục tọa độ nên $R = |a| = |b|$.\\
			\textbf{TH 1:} Nếu $a = b$ thì $I(a;a)$.\\
			Vì $(C)$ qua $M(2;1)$ nên bán kính $R = MI$ hay $R^2 = MI^2$ do đó
			$$ a^2 = (a - 2)^2 + (a - 1)^2 \Leftrightarrow a^2 - 6a + 5 = 0 \Leftrightarrow \hoac{&a = 1\\&a = 5.}$$
			\begin{itemize}
				\item Với $a = 1$, khi đó $I(1;1)$ và bán kính $R = 1$.\\
				Suy ra phương trình đường tròn $(C_1)$ là $(x - 1)^2 + (y - 1)^2 = 1$.
				\item Với $a = 5$, khi đó $I(5;5)$ và bán kính $R = 5$.\\
				Suy ra phương trình đường tròn $(C_2)$ là $(x - 5)^2 + (y - 5)^2 = 25$.
			\end{itemize}
			\textbf{TH 2:} Nếu $a = -b$ thì $I(a;-a)$.\\
			Vì $(C)$ qua $M(2;1)$ nên bán kính $R = MI$ hay $R^2 = MI^2$ do đó
			$$ a^2 = (a - 2)^2 + (-a - 1)^2 \Leftrightarrow a^2 - 2a + 5 = 0 \Leftrightarrow (a - 1)^2 + 4 = 0 \textrm{ (vô lý).} $$
			\item Gọi tâm đường tròn $(C)$ là điểm $I(a;b)$.\\
			Vì $(C)$ tiếp xúc với $Ox$ tại $M(6;0)$ nên $a = 6$ hay $I(6;b)$.\\
			Mặt khác, $A$, $M$ cùng thuộc đường tròn nên 
			\begin{eqnarray*}
				AI^2 = BI^2 \Leftrightarrow (6-6)^2 + (b-0)^2 = (6-9)^2 + (b-9)^2 \Leftrightarrow b = 5.
			\end{eqnarray*}
			Vậy đường tròn $(C)$ tâm $I(6;5)$ và bán kính $R = |b| = 5$ có phương trình là $(x - 6)^2 + (y - 5)^2 = 25$.
			\item Gọi tâm đường tròn $(C)$ là điểm $I(a;b)$.\\
			Vì $(C)$ tiếp xúc với $Ox$ tại $A(2;0)$ nên $a = 2$ hay $I(2;b)$.\\
			Mặt khác, khoảng cách từ tâm của $(C)$ đến $B(6;4)$ bằng $5$ nên $BI = 5$ hay
			$$ BI^2 = 25 \Leftrightarrow (6 - 6)^2 + (b - 4)^2 = 25 \Leftrightarrow b^2 - 8b - 9 = 0 \Leftrightarrow \hoac{&b = -1\\&b = 9.} $$
			\begin{itemize}
				\item Với $b = -1$, khi đó $I(2;-1)$ và bán kính $R = |b| = 1$.\\
				Suy ra phương trình đường tròn $(C_1)$ là $(x - 2)^2 + (y + 1)^2 = 1$.
				\item Với $b = 9$, khi đó $I(2;9)$ và bán kính $R = |b| = 9$.\\
				Suy ra phương trình đường tròn $(C_1)$ là $(x - 2)^2 + (y - 9)^2 = 81$.
			\end{itemize}
		\end{enumerate}
	}
\end{bt}

\begin{dang}{Phương trình tiếp tuyến của đường tròn}
	\begin{enumerate}
		\item Cho điểm $M(x_0;y_0)$ thuộc đường tròn $(C)\colon (x-a)^2+(y-b)^2=R^2$ (tâm $I(a;b)$, bán kính $R$). Khi đó, tiếp tuyến $\Delta$ của $(C)$ tại $M(x_0;y_0)$ có véc-tơ pháp tuyến $\overrightarrow{MI}=(a-x_0;b-y_0)$ và phương trình $\Delta\colon (a-x_0)(x-x_0)+(b-y_0)(y-y_0)=0$.
		\begin{center}
			\begin{tikzpicture}[scale=1, font=\footnotesize, line join=round, line cap=round, >=stealth]
				\path
				(0,0)coordinate(I)
				(0,-2)coordinate(M)
				(-3,-2)coordinate(A)
				(3,-2)coordinate(B)
				;
				\draw [draw=black] (I) circle (2);
				\draw(I)--(M) (A)--(B)node[above]{$\Delta$};
				\foreach \x/\y/\z in {I/M/A}\pic[draw,angle radius=1.5mm]{right angle=\x--\y--\z};
				\foreach \x/\i in {I/90,M/-90}{\path ($(\x)+(\i:3mm)$) node{$\x$};}
				\foreach \x in {I,M}{\draw[fill=black] (\x) circle (1pt);}
			\end{tikzpicture}
		\end{center}
		\item Lập phương trình tiếp tuyến $\Delta$ với đường tròn $(C)$, khi biết $\Delta$ đi qua một điểm không thuộc đường tròn $(C)$. Khi đó ta sử dụng điều kiện đường thẳng $\Delta$ là tiếp tuyến của đường tròn $(C)$ khi và chỉ khi $\mathrm{d}\left(I,\Delta\right) =R$.	
	\end{enumerate}	
\end{dang}


\begin{vd}%[Vương Quyền, BG10-KNTT-Tập 2]%[0H3B2-3]
	Cho đường tròn $(C) \colon x^2+y^2-2x-2y-11=0$. Tiếp tuyến của $(C)$ tại điểm $M(4;-1)$ thuộc $(C)$ có phương trình là
	\loigiai{
		Đường tròn $(C)$ có tâm $I(1;1)$. Phương trình tiếp tuyến của đường tròn tại $M(4;-1)$ là 
		$$(4-1)(x-4)+(-1-1)(y+1)=0 \Leftrightarrow 3x-2y-14=0.$$}
\end{vd}

\begin{vd}%[Vương Quyền, BG10-KNTT-Tập 2]%[0H3B2-3]
	Trong mặt phẳng tọa độ $Oxy$, cho đường tròn $(\mathscr{C})\colon x^2+y^2-3x-y=0$. Viết phương trình tiếp tuyến của $(\mathscr{C})$ tại $M(1;-1)$ thuộc $(C)$.
	\loigiai
	{
		Đường tròn $(\mathscr{C})$ có tâm $I\left(\dfrac{3}{2};\dfrac{1}{2}\right)$ và bán kính $R=\sqrt{\left(\dfrac{3}{2}\right)^2+\left(\dfrac{1}{2}\right)^2-0} = \dfrac{\sqrt{10}}{2}$.\\
		Tiếp tuyến của đường tròn $(\mathscr{C})$ tại $M(1;-1)$ có véc-tơ pháp tuyến là $\overrightarrow{IM}=\left(-\dfrac{1}{2};-\dfrac{3}{2}\right)$ hay $\overrightarrow{n}=(1;3)$.\\
		Phương trình tiếp tuyến của $(\mathscr{C})$ tại $M(1;-1)$ là
		\[1(x-1)+3(y+1)=0 \Leftrightarrow x+3y+2=0.\]
	}
\end{vd}

\begin{vd}%[Vương Quyền, BG10-KNTT-Tập 2]%[0H3B2-3]
	Cho đường tròn $(C)$ có phương trình $x^2+y^2+4x-2y-4=0$. Từ $O(0;0)$ kẻ được bao nhiêu đường thẳng tiếp xúc với $(C)$?
	\loigiai{
		Đường tròn $(C)$ có tâm $I(-2;1)$ và bán kính $R=1$. \\
		Ta có $IO=\sqrt{(0+2)^2+(0-1)^2}=\sqrt{5}>R$ 
		$\Rightarrow$ Điểm $O$ nằm ngoài đường tròn $(C)$. \\
		Vậy từ $O$ kẻ được hai đường thẳng tiếp xúc với $(C)$.}
\end{vd}

\begin{vd}%[Vương Quyền, BG10-KNTT-Tập 2]%[0H3B2-3]
	Trong mặt phẳng $Oxy$, cho đường thẳng $d \colon 2x-y-5=0$ và hai điểm $A(1;2)$ và $B(4;1)$.
	\begin{enumerate}
		\item Viết phương trình đường tròn $(C)$ có tâm thuộc đường thẳng $d$ và đi qua hai điểm $A,B$.
		\item Viết phương trình tiếp tuyến của đường tròn $(C)$ biết tiếp tuyến vuông góc với đường thẳng $d' \colon x+y+2019=0$.
	\end{enumerate}
	\loigiai{
		\begin{enumerate}
			\item Gọi $K$ là tâm của đường tròn $(C)$, vì $K \in d \Rightarrow K(a;2a-5)$.\\
			Ta có $KA=KB \Rightarrow (a-1)^2+(2a-7)^2=(a-4)^2+(2a-6)^2$ \\
			$\Leftrightarrow a=1$. Vậy $K(1;-3)$ và $R=KA=5$.\\
			Phương trình đường tròn $(C) \colon (x-1)^2+(y+3)^2=25$.
			\item  Do tiếp tuyến của $(C)$ vuông góc với $d'$ nên phương trình tiếp tuyến có dạng: $\Delta \colon x-y+m=0$.\\
			Khi đó $\mathrm{d}\left(K; \Delta \right)=R \Leftrightarrow \dfrac{|1+3+m|}{\sqrt{2}}=5 \Leftrightarrow |m+4|=5\sqrt{2} \Leftrightarrow m=-4\pm 5\sqrt{2}$.\\
			Các phương trình tiếp tuyến cần tìm là $x-y-4\pm 5\sqrt{2}=0$
		\end{enumerate}
	}
\end{vd}

\begin{vd}%[Vương Quyền, BG10-KNTT-Tập 2]%[0H3B2-3]
	Với những giá trị nào của $m$ thì đường thẳng $\Delta \colon 4x+3y+m=0$ tiếp xúc với đường tròn $(C) \colon x^2+y^2-9=0$.
	\loigiai{
		Đường tròn $(C)$ có tâm $O(0;0)$ và bán kính $R=3$.\\
		Đường thẳng $\Delta$ tiếp xúc với đường tròn $(C)$ khi và chỉ khi
		\[\mathrm{d}\left(O, \Delta \right)=R 
		\Leftrightarrow \dfrac{|m|}{\sqrt{4^2+3^2}}=3 \Leftrightarrow |m|=15 \Leftrightarrow m=\pm 15.\]}
\end{vd}

\begin{vd}%[Vương Quyền, BG10-KNTT-Tập 2]%[0H3B2-3]
	Viết phương trình tiếp tuyến $(\Delta)$ của đường tròn $(C): (x-1)^2+(y-2)^2=8$ biết tiếp tuyến đi qua điểm $M(3;-2)$.
	\loigiai{
		Đường tròn $(C)$ có tâm $I(1;2)$ và bán kính $R=\sqrt{8}$.
		\\ Ta có $IM=\sqrt{(3-1)^2 + (-2-2)^2}=2\sqrt{5}$.
		\\ Gọi phương trình tiếp tuyến $(\Delta)$ của $(C)$ và đi qua $M(3;-2)$ là $a(x-3)+b(y+2)=0$ ($a^2+b^2\neq 0$).
		\\ Ta có $\mathrm{d}(I,\Delta)=\dfrac{|a(1-3)+b(2+2)|}{\sqrt{a^2+b^2}}=\sqrt{8} \Leftrightarrow \dfrac{|-2a+4b|}{\sqrt{a^2+b^2}}=\sqrt{8}$.
		\\ Phương trình trên tương đương với
		\begin{align*}
			&|-2a+4b|=\sqrt{8a^2+8b^2}
			\\ \Leftrightarrow & (2a-4b)^2=8a^2+8b^2
			\\ \Leftrightarrow & 8b^2 -16ab -4a^2 =0
			\\ \Leftrightarrow &2b^2 - 4ab -a^2=0
			\\ \Leftrightarrow & \left[ \begin{array}{ll}
				b=\dfrac{2+\sqrt{6}}{2}a \\ b=\dfrac{2-\sqrt{6}}{2}a.
			\end{array} \right.
		\end{align*}
		\begin{itemize}
			\item Nếu $b=\dfrac{2+\sqrt{6}}{2}a$ thì ta chọn $a=2 \Rightarrow b=2+\sqrt{6}$. 
			\\ Khi đó phương trình của tiếp tuyến $(\Delta)$ là: 
			$$2(x-3)+(2+\sqrt{6})(y+2)=0 \text{  hay  } 2x+(2+\sqrt{6})y+2\sqrt{6}-2=0.$$
			\item Nếu $b=\dfrac{2-\sqrt{6}}{2}a$ thì ta chọn $a=2 \Rightarrow b=2-\sqrt{6}$. 
			\\ Khi đó phương trình của tiếp tuyến $(\Delta)$ là: 
			$$2(x-3)+(2-\sqrt{6})(y+2)=0 \text{  hay  } 2x+(2-\sqrt{6})y-2\sqrt{6}-2=0.$$
		\end{itemize}
	}
\end{vd}

\subsubsection{Bài tập tự luyện}


\begin{bt}%[Vương Quyền, BG10-KNTT-Tập 2]%[0H3B2-3]
	Trong mặt phẳng tọa độ $Oxy$, viết phương trình tiếp tuyến của đường tròn $$(C)\colon (x+1)^2+(y-3)^2=25$$ tại điểm $M(-4;7)$.
	\loigiai{
		Đường tròn $(C)$ có tâm $I(-1;3)$.\\
		Tiếp tuyến tại $M(-4;7)$ nhận véc-tơ pháp tuyến $\overrightarrow{IM}=(-3;4)$.\\
		Phương trình tiếp tuyến của $(C)$ tại $M$ là $-3x+4y-40=0$.
	}
\end{bt}

\begin{bt}%[Vương Quyền, BG10-KNTT-Tập 2]%[0H3B2-3]
	Trong mặt phẳng tọa độ $Oxy$, viết phương trình tiếp tuyến $\Delta$ của đường tròn $$(C)\colon (x - 1)^2 + (y - 3)^2 = 25$$ tại điểm  $N(4; -1)$. 
	\loigiai{
		Ta có đường tròn $(C)$ có tâm $I(1;3)$, tiếp tuyến $\Delta$ có một véc-tơ pháp tuyến là $\overrightarrow{IN}=(3;-4)$.\\
		Phương trình $\Delta\colon 3(x-4)^2-4(y+1)^2=0\Leftrightarrow 3x-4y-16=0$.
	}
\end{bt}

\begin{bt}%[Vương Quyền, BG10-KNTT-Tập 2]%[0H3K2-3]
	Trong mặt phẳng tọa độ $Oxy$, cho đường tròn $(\mathrm{C}): x^2+y^2+4x+4y-17=0$. Viết phương trình tiếp tuyến $\Delta$ của  $(\mathrm{C})$ biết $\Delta$ vuông góc với đường thẳng $d\colon 3x-4y+1=0$.
	\loigiai{
		Đường tròn $(C)$ có tâm $I(-2;-2)$, bán kính $R=5$.\\
		$\Delta\perp d$ suy ra $\Delta\colon 4x+3y+c=0$.\\
		Ta có $\Delta$ là tiếp tuyến của đường tròn suy ra $$d(I,\Delta)=R\Leftrightarrow \dfrac{|4(-2)+3(-2)+c|}{\sqrt{4^2+3^2}}=5\Leftrightarrow |-14+c|=25\Leftrightarrow \hoac{&c=39\\ &c=-11.}$$
		Suy ra $\Delta\colon 4x+3y+39=0$ và $\Delta\colon 4x+3y-11=0$.
	}
\end{bt}

\begin{bt}%[Vương Quyền, BG10-KNTT-Tập 2]%[0H3K2-3]
	Trong mặt phẳng tọa độ $Oxy$, viết phương trình tiếp tuyến với $(C): (x-1)^2+(y+2)^2=10$, biết tiếp tuyến song song với đường thẳng $d:x+3y-5=0$.
	\loigiai{
		Đường tròn $(C)$ có tâm $I(1;-2)$ và bán kính $R=\sqrt{10}$.\\
		Vì tiếp tuyến $\Delta$ của $(C)$ song song với $d$ nên $\Delta$ có dạng $x+3y+m=0$ với $m\neq -5$.\\
		Vì $\Delta$ tiếp xúc $(C)$ nên $\mathrm{\,d}(I;\Delta)=R \Leftrightarrow \dfrac{|1+3\cdot(-2)+m|}{\sqrt{1^2+3^2}}=\sqrt{10} \Leftrightarrow m=15$ hoặc $m=-5$ (loại).\\
		Vậy tiếp tuyến cần tìm có phương trình $x+3y+15=0$.
	}
\end{bt}

\begin{bt}%[Võ Thị Quỳnh Trang]%[0H3K2-3]
	Viết phương trình tiếp tuyến của đường tròn $(C): (x-3)^2+y^2=9$ biết tiếp tuyến đi qua điểm $M(3;5)$.
	\loigiai{Đường tròn $(C)$ có tâm $I(3;0)$ và bán kính $R=3$. 
		\\ Ta có $IM=\sqrt{0^2+5^2}=5 > R=3$. 
		\\ Gọi tiếp tuyến $(\Delta)$ của đường tròn $(C)$ và đi qua $M$ là $a(x-3)+b(y-5)=0$ với $a^2+b^2>0$.
		\\ Ta có \begin{align*}
			\mathrm{d}(I,\Delta) = R &\Rightarrow \dfrac{|-5b|}{\sqrt{a^2+b^2}}=3
			\\ &\Rightarrow |5b|=3\sqrt{a^2+b^2}
			\\ &\Rightarrow b=\pm \dfrac{3}{4}a.
		\end{align*}
		Nếu $b=-\dfrac{3}{4}a$ thì ta chọn $a=4,b=-3$. Khi đó phương trình tiếp tuyến $(\Delta)$ là $4x-3y+3=0$.
		\\ 	Nếu $b=\dfrac{3}{4}a$ thì ta chọn $a=4,b=3$. Khi đó phương trình tiếp tuyến $(\Delta)$ là $4x+3y-27=0$.
	}
\end{bt}

\begin{bt}%[Vương Quyền, BG10-KNTT-Tập 2]%[0H3K2-3]
	Cho hai đường tròn $(C_1): x^2+y^2+2x-2y-3=0$ và $(C_2): x^2+y^2-4x-14y+33=0$.
	\begin{enumerate}
		\item Chứng minh rằng $(C_1)$ và $(C_2)$ tiếp xúc với nhau.
		\item Viết phương trình tiếp tuyến chung của hai đường tròn tại tiếp điểm.
	\end{enumerate}
	\loigiai{
		\begin{enumerate}
			\item Đường tròn $(C_1)$ có tâm $I(-1;1)$ và bán kính $R_1=\sqrt{5}$. \\
			Đường tròn $(C_2)$ có tâm $J(2;7)$ và bán kính $R_2=2\sqrt{5}$.
			\\ Ta có $IJ=\sqrt{(2+1)^2+(7-1)^2}=3\sqrt{5}=R_1+R_2$. Do đó $(C_1)$ tiếp xúc ngoài với $(C_2)$.
			\item Gọi $M$ là tiếp điểm của $(C_1)$ và $(C_2)$. 
			\\ Khi đó ta có $\overrightarrow{IJ}=3\overrightarrow{IM} \Rightarrow \overrightarrow{OM}= \dfrac{1}{3}\overrightarrow{OJ} +\dfrac{2}{3}\overrightarrow{OI}$.
			\\ Suy ra $M\left( 0;3 \right) \Rightarrow \overrightarrow{IM}=(1;2)$.
			\\ Phương trình tiếp tuyến chung của hai đường tròn tại $M$ là $x+2(y-3)=0$ hay $x+2y-6=0$.
	\end{enumerate}}
\end{bt}


\begin{bt}%[Vương Quyền, BG10-KNTT-Tập 2]%[0H3G2-3]
	Trong mặt phẳng tọa độ $Oxy$, cho đường tròn $(C): x^2+y^2-6x+4y-7=0$ và điểm $A(5; 4)$ nằm ngoài đường tròn. Gọi tiếp điểm của tiếp tuyến kẻ từ $A$ đến đường tròn là $T_1$, $T_2$, với hoành độ $T_1$ nhỏ hơn hoành độ $T_2$. Tìm tọa độ của véc-tơ $\overrightarrow{T_1T_2}$. 
	\loigiai{
		Đường tròn có tâm $I(3; -2)$, bán kính $R = 2\sqrt{5}$.\\
		Gọi véc-tơ pháp tuyến của tiếp tuyến qua $A$ là $\overrightarrow{n}(a;b), ab \neq 0$. Ta có phương trình tiếp tuyến $\Delta: ax+by-5a-4b=0$.\\
		Do $d$ là tiếp tuyến nên 
		\begin{eqnarray*}
			&&d(I,\Delta)=\dfrac{|3a-2b-5a-4b|}{\sqrt{a^2+b^2}}=2\sqrt{5}
			\\&\Leftrightarrow &(a+3b)^2=5(a^2+b^2) \Leftrightarrow \hoac{&a=2b\\&a=-\dfrac{b}{2}.}
		\end{eqnarray*}
		Với $a=2b$, chọn $b=1$, ta có $a = 2$, phương trình tiếp tuyến $2x+y-14=0$, suy ra toạ độ tiếp điểm $(7;0)$.\\
		Với $a=-\dfrac{b}{2}$, chọn $b=-2$, ta có $a = 1$, phương trình tiếp tuyến $x-2y+3=0$, suy ra toạ độ tiếp điểm $(1;2)$.\\
		Vậy các tiếp điểm là $T_1(1;2), T_2(7;0)$ nên $\overrightarrow{T_1T_2}=(6;-2)$.
	}
\end{bt}

\subsection{Bài tập trắc nghiệm}
\subsubsection{Bài tập trắc nghiệm cơ bản}
\Opensolutionfile{ansbook}[ans/ansbook-0H4-7-TN]
\Opensolutionfile{ans}[ans/ans-0H7-21-TN]
\begin{ex}%[0H7YK-1]
	Trong mặt phẳng tọa độ $Oxy$, cho đường tròn có phương trình $(x-3)^2+(y+2)^2=5$. Xác định tâm $I$ và bán kính $R$ của đường tròn trên?
	\choice
	{$I(-3;2)$, $R = \sqrt{5}$}
	{\True $I(3;-2)$, $R = \sqrt{5}$}
	{$I(-3;2)$, $R = 5$}
	{$I(3;-2)$, $R =5$}
	\loigiai{
		Tâm $I(3;-2)$ và $R=\sqrt{5}$.
	}
\end{ex}

\begin{ex}%[0H7YK-2]
	Trong mặt phẳng tọa độ $Oxy$, đường tròn $(C)$ có tọa độ tâm $I(-2 ; 4)$ và bán kính $R = 4$ có phương trình là 
	\choice
	{\True $(C)\colon (x + 2)^2 + (y - 4)^2 = 16$}
	{$(C)\colon (x - 2)^2 + (y + 4)^2 = 16$}
	{$(C)\colon (x + 2)^2 + (y - 4)^2 = 4$}
	{$(C)\colon (x - 2)^2 + (y + 4)^2 = 4$}
	\loigiai{
		Phương trình đường tròn cần tìm là $(C)\colon (x+2)^2+(y-4)^2=16$.
	}
\end{ex}

\begin{ex}%[0H7YK-2]
	Phương trình nào là phương trình của đường tròn có tâm $I(3;-4)$ và đường kính bằng $4$?
	\choice
	{\True $(x-3)^2+(y+4)^2=4$}
	{$(x+3)^2+(y-4)^2=16$}
	{$(x+3)^2+(y-4)^{2}=4$}
	{$(x-3)^2+(y+4)^2=16$}
	\loigiai{
		Phương trình đường tròn có tâm $I(3;-4)$ và bán kính $R=2$ là $(x-3)^2+(y+4)^2=4$.
	}
\end{ex}

\begin{ex}%[0H7YK-2]
	Đường tròn tâm $I(2;0)$ và đi qua điểm $A(-1;7)$ có phương trình là
	\choice
	{$(x+2)^2+y^2=\sqrt{58}$}
	{$(x-2)^2+y^2=\sqrt{58}$}
	{$(x+2)^2+y^2=58$}
	{\True $(x-2)^2+y^2=58$}
	\loigiai{
		Ta có $R^2=IA^2=(-1-2)^2+(7-0)^2=58$.\\
		Vậy phương trình đường tròn là $(x-2)^2+y^2=58$.	
	}
\end{ex}

\begin{ex}%[0H7BK-2]
	Đường tròn đường kính $AB$ với $A(3;-1)$, $B(1;-5)$ có phương trình là
	\choice
	{$(x+2)^2+(y-3)^2=5$}
	{$(x+1)^2+(y+2)^2=17$}
	{$(x-2)^2+(y+3)^2=\sqrt{5}$}
	{\True $(x-2)^2+(y+3)^2=5$}
	\loigiai{
		Gọi $I$ là tâm của đường tròn. Suy ra, $I$ là trung điểm $AB$ nên $I(2;-3)$. \\
		Bán kính $R=IA=\sqrt{(3-2)^2+(-1+3)^2}=\sqrt{5}$. \\
		Phương trình đường tròn cần tìm có dạng $(C)\colon (x-2)^2+(y+3)^2=5$.
	}
\end{ex}

\begin{ex}%[0H7YK-1]
	Trong mặt phẳng tọa độ $Oxy$, bán kính $R$ của đường tròn $x^2+y^2-2x+4y+1=0$ là
	\choice
	{\True $R=2$}
	{$R=4$}
	{$R=1$}
	{$R=3$}
	\loigiai{
		Ta có $a=1, b=-2$ và $c=1$ nên $R=\sqrt{1^2+2^2-1}=2$.
	}
\end{ex}

\begin{ex}%[0H7YK-1]
	Trong các phương trình sau, phương trình nào là phương trình của một đường tròn?
	\choice
	{$x^2+y^2+2x-4y+9= 0$}
	{$x^2+y^2-6x+4y+13=0$}
	{\True $2x^2+2y^2-8x-4y-6=0$}
	{$5x^2+4y^2+x-4y+1=0$}
	\loigiai{
		$\bullet~$ Loại đáp án $5x^2+4y^2+x-4y+1=0$ vì không có dạng $x^2+y^2-2ax-2by+c=0$.\\
		$\bullet~$ Xét đáp án
		$x^2+y^2+2x-4y+9= 0\Rightarrow a=-1,b=2,c=-9\Rightarrow a^2+b^2-c<0\Rightarrow$ loại.\\
		$\bullet~$ Xét đáp án
		$x^2+y^2-6x+4y+13=0\Rightarrow a=3,b=-2,c=13\Rightarrow a^2+b^2-c<0\Rightarrow$ loại.\\
		$\bullet~$ Xét đáp án
		$2x^2+2y^2-8x-4y-6=0\Leftrightarrow x^2+y^2-4x-2y-3=0\Rightarrow  a=2$, $b=1$, $c=-3 \Rightarrow a^2+b^2-c>0$.
	}
\end{ex}

\begin{ex}%[0H7YK-1]
	Tìm tất cả các giá trị của $m$ để phương trình $x^2+y^2-4x+2y+m=0$ là phương trình đường tròn?
	\choice
	{$m=6$}
	{$m=25$}
	{\True $m< 5$}
	{$m>5$}
	\loigiai{
		Ta có $a=2$, $b=-1$ và $c=m$.\\
		Để phương trình đã cho là phương trình đường tròn thì 
		$$a^2+b^2-c>0 \Leftrightarrow 2^2+(-1)^2-m>0 \Leftrightarrow m <5.$$	
		Vậy $m<5$ thì phương trình đã cho là phương trình đường tròn.
	}
\end{ex}

\begin{ex}%[0H7YK-4]
	Cho đường tròn $(C): x^2+y^2-4x+3=0$. Mệnh đề nào sau đây \textbf{sai}?
	\choice
	{$(C)$ có tâm $I(2;0)$}
	{$(C)$ có bán kính $R=1$}
	{$(C)$ cắt trục $Ox$ tại hai điểm phân biệt}
	{\True $(C)$ cắt trục $Oy$ tại hai điểm phân biệt}
	\loigiai{
		Ta có $a=2$, $b=0$, $c=3$.\\
		Đường tròn $(C)$ có tâm $I(2;0)$ và bán kính $R=\sqrt{2^2+0^2-3}=1$.\\
		Mặt khác, $\mathrm{d}\left(I, Ox\right) = 0<R$ nên đường tròn $(C)$ cắt trục $Ox$ tại hai điểm phân biệt.	
	}
\end{ex}

\begin{ex}%[0H7YK-1]
	Cho đường cong $(C_m): x^2+y^2-8x+10y+m=0$. Với giá trị nào của $m$ thì $(C_m)$ là đường tròn có bán kính bằng $7$?
	\choice
	{$m=4$}
	{$m=8$}
	{\True $m=-8$}
	{$m=-4$}
	\loigiai{
		Ta có $a=4$, $b=-5$, $c=m$. Do đó, $(C_m)$ là đường tròn có bán kính bằng $7$ thì 
		$$a^2+b^2-c=7^2 \Leftrightarrow 4^2+(-5)^2-m=49 \Leftrightarrow m=-8.$$
		Vậy $m=-8$ thỏa yêu cầu bài toán.
	}
\end{ex}

\begin{ex}%[0H7BK-1]
	Tìm tọa độ tâm $I$ của đường tròn đi qua ba điểm $A(0;4)$, $B(2;4)$, $C(4;0)$?
	\choice
	{$I(0;0)$}
	{$I(1;0)$}
	{$I(3;2)$}
	{\True $I(1;1)$}
	\loigiai{
		Gọi đường tròn qua ba điểm là $(C)\colon x^2+y^2-2ax-2by+c=0$.\\
		Vì $A,B,C\in (C)\Leftrightarrow \heva{
			& 16-8b+c=0 \\ 
			& 20-4a-8b+c=0 \\ 
			& 16-8a+c=0 \\}\Leftrightarrow \heva{
			& a=1 \\ 
			& b=1 \\ 
			& c=-8 \\}\Rightarrow I\left(1;1\right)$.
	}
\end{ex}

\begin{ex}%[0H7BK-2]
	Phương trình đường tròn qua ba điểm $A(0;4)$, $B(2;4)$, $C(4;0)$ là
	\choice
	{$x^2+y^2-2x-2y+8=0$}
	{$x^2+y^2+2x+2y+8=0$}
	{\True $x^2+y^2-2x-2y-8=0$}
	{$Ix^2+y^2+2x+2y-8=0$}
	\loigiai{
		Gọi đường tròn qua ba điểm là $(C)\colon x^2+y^2-2ax-2by+c=0$.\\
		Vì $A,B,C\in (C)\Leftrightarrow \heva{
			& 16-8b+c=0 \\ 
			& 20-4a-8b+c=0 \\ 
			& 16-8a+c=0 \\}\Leftrightarrow \heva{
			& a=1 \\ 
			& b=1 \\ 
			& c=-8.}$\\
		Vậy phương trình đường tròn $(C) \colon x^2+y^2-2x-2y-8=0$.
	}
\end{ex}

\begin{ex}%[0H7BK-4]
	Với những giá trị nào của $m$ thì đường thẳng $\left(\Delta\right)\colon 4x+3y+m=0$ tiếp xúc với đường tròn $\left(C\right)\colon x^2+y^2=9$?
	\choice
	{$m=3$ và $m=-3$}
	{$m=-3$}
	{$m=-3$}
	{\True $m=15$ và $m=-15$}
	\loigiai{Đường tròn $\left(C\right)$ có tâm $I(0,0)$ và $R=3$.\\
		Vì đường thẳng $\left(\Delta\right)\colon 4x+3y+m=0$ tiếp xúc với đường tròn $\left(C\right)\colon x^2+y^2=9$ nên
		$$\mathrm{d}\left(I,\Delta\right)= R \Leftrightarrow  \dfrac{|m|}{\sqrt{4^2+3^2}} = 3 \Leftrightarrow  |m|=15
		\Leftrightarrow m=\pm 15.$$
		Vậy $m=\pm 15$.
	}
\end{ex}

\begin{ex}%[0H7BK-4]
	Trong mặt phẳng tọa độ $Oxy$, cho $I(1;2)$ và $d: 2x-y+5=0$. Phương trình đường tròn có tâm $I$ và tiếp xúc với đường thẳng $d$ có dạng
	\choice
	{\True $(x-1)^2+(y-2)^2=5$}
	{$(x-1)^2+(y-2)^2=\sqrt{5}$}
	{$(x+1)^2+(y+2)^2=5$}
	{$(x+1)^2+(y+2)^2=\sqrt{5}$} 
	\loigiai{
		Bán kính đường tròn là $R=\mathrm{d}(I;d)=\dfrac{\left|2\cdot 1-2+5\right|}{\sqrt{2^2+(-1)^2}}=\sqrt{5}$.\\
		Vậy phương trình của đường tròn cần tìm có dạng $(x-1)^2+(y-2)^2=5$.
	}
\end{ex}

\begin{ex}%[0H7YK-4]
	Trong mặt phẳng tọa độ $Oxy$ cho điểm $I(2;-3)$. Phương trình đường tròn có tâm $I$ và tiếp xúc với trục hoành có dạng
	\choice
	{\True $(x-2)^2+(y+3)^2=9$}
	{$(x+2)^2+(y-3)^2=9$}
	{$(x-2)^2+(y+3)^2=4$}
	{$(x+2)^2+(y-3)^2=4$}
	\loigiai{
		Bán kính đường tròn là $R=\mathrm{d}(I,Ox)=|-3|=3$.\\
		Vậy phương trình của đường tròn cần tìm có dạng $(x-2)^2+(y+3)^2=9$.	
	}
\end{ex}

\begin{ex}%[0H7YK-3]
	Trong mặt phẳng $Oxy$, cho đường tròn $(C):(x-3)^2+(y+1)^2=13$. Phương trình tiếp tuyến của đường tròn $(C)$ tại điểm $A\left(1;2\right)$ là
	\choice
	{\True $2x-3y+4=0$}
	{$2x+3y+4=0$}
	{$2x-3y-4=0$}
	{$2x+3y-4=0$}
	\loigiai{
		Đường tròn $(C)$ có tâm $I\left(3;-1\right)$ nên tiếp tuyến tại $A$ có VTPT là
		$\overrightarrow{n}=\overrightarrow{IA}=\left(-2;3\right)$.\\
		Phương trình tiếp tuyến dạng: $-2\left(x-1\right)+3\left(y-2\right)=0\Leftrightarrow -2x+3y-4=0 \Leftrightarrow 2x-3y+4=0$.}
\end{ex}


\begin{ex}%[0H7YK-3]
	Trong mặt phẳng $Oxy$, cho đường tròn $(C):x^2+y^2-3x-y=0$. Phương trình tiếp tuyến $d$ của đường tròn $(C)$ tại điểm $N\left(1;-1\right)$ là
	\choice
	{$d\colon x+3y-2=0$}
	{$d \colon x-3y+4=0$}
	{$d\colon x-3y-4=0$}
	{\True $d\colon x+3y+2=0$}
	\loigiai{
		Đường tròn $(C)$ có tâm $I\left(\dfrac{3}{2};\dfrac{1}{2}\right)$ nên tiếp tuyến tại $N$ có VTPT là
		$\overrightarrow{n}=\overrightarrow{IN}=\left(\dfrac{-1}{2};\dfrac{-3}{2}\right)$.\\
		Phương trình tiếp tuyến dạng: $\dfrac{-1}{2}\left(x-1\right)+\dfrac{-3}{2}\left(y+1\right)=0\Leftrightarrow x+3y+2=0$.}
\end{ex}

\begin{ex}%[0H7BK-4]
	Trong mặt phẳng tọa độ $Oxy$, cho đường tròn $(C)\colon (x-1)^2+(y-2)^2=9$. Đường thẳng $d$ đi qua điểm $I(1;2)$ cắt $(C)$ tại hai điểm $M ,N$. Tính độ dài của $MN$.
	\choice
	{$MN = 1$}
	{$MN = 2$}
	{$MN = 3$}
	{\True $MN = 6$}
	\loigiai{
		Đường tròn $(C) \colon (x-1)^2+(y-2)^2=9$ có tâm $I(1;2)$, bán kính $R=3$.\\
		Do đó đường thẳng $d$ đi qua điểm $I(1;2)$ (qua tâm) cắt $(C)$ tại hai điểm $M ,N$ thì $MN$ là đường kính của $(C)$.\\
		Vậy độ dài của $MN=2R=6$.
	}
\end{ex}

\begin{ex}%[0H7YK-4]
	Trong mặt phẳng tọa độ $Oxy$, cho đường tròn $(C) \colon (x-1)^2 + y^2=25$ và điểm $M(2,\sqrt{3})$. Số tiếp tuyến của đường tròn $(C)$ kẻ từ $M$ là
	\choice
	{\True $0$}
	{$1$}
	{$2$}
	{Vô số}
	\loigiai{
		Đường tròn $(C)$ có tâm $I(1;0)$ và bán kính $R=5$.\\
		Vì $IM=\sqrt{(2-1)^2+(\sqrt{3}-0)^2}=2<R$ nên $M$ nằm trong đường tròn.\\
		Vậy số tiếp tuyến của đường tròn $(C)$ kẻ từ $M$ là $0$.
		
	}	
\end{ex}

\begin{ex}%[0H7YK-4]
	Trong mặt phẳng tọa độ $Oxy$, cho đường tròn $(C) \colon x^2+y^2-2x+4y+1=0$ và điểm $M(-2,2)$. Số tiếp tuyến của đường tròn $(C)$ kẻ từ $M$ là
	\choice
	{ $0$}
	{$1$}
	{\True $2$}
	{Vô số}
	\loigiai{
		Đường tròn $(C)$ có tâm $I(1;-2)$ và bán kính $R=2$.\\
		Vì $IM=\sqrt{(-2-1)^2+(2+2)^2}=5>R$ nên $M$ nằm ngoài đường tròn.\\
		Vậy số tiếp tuyến của đường tròn $(C)$ kẻ từ $M$ là $2$.
		
	}	
\end{ex}

\Closesolutionfile{ans}
% \begin{center}
% 	\textbf{ĐÁP ÁN}
% \end{center}
% 	\indapan{10}{ans/ans-0H7-21-TN} 

\subsubsection{Bài tập trắc nghiệm nâng cao}
\Opensolutionfile{ans}[ans/ans-0H7-21-TNNC]
\begin{ex}%[0H7BK-1]
	Cho phương trình $x^2+y^2-2mx-4\left(m-2\right)y+6-m=0$. Tìm điều kiện của $m$ để phương trình đã cho là phương trình đường tròn?
	\choice
	{$m\in \mathbb{R}$}
	{\True $m\in \left(-\infty;1\right)\cup \left(2;+\infty \right)$}
	{$m\in \left(-\infty;1\right]\cup \left[2;+\infty \right)$}
	{$m\in \left(-\infty;\dfrac{1}{3}\right)\cup \left(2;+\infty \right)$}
	\loigiai{
		Ta có $a=m$, $b=2\left(m-2\right)$, $c=6-m$.
		Phương trình đã cho là phương trình đường tròn khi 
		$$a^2+b^2-c>0 \Leftrightarrow m^2+4(m-2)^2-6+m>0  \Leftrightarrow 5m^2-15m+10>0\Leftrightarrow \hoac{
			& m<1 \\ 
			& m>2.}$$
		Vậy $m\in \left(-\infty;1\right)\cup \left(2;+\infty \right)$ thì phương trình đã cho là phương trình đường tròn.
	}
\end{ex}

\begin{ex}%[0H7BK-3]
	Viết phương trình tiếp tuyến của đường tròn $(C) \colon {\left(x-3\right)}^2+{\left(y+1\right)}^2=5$, biết tiếp tuyến song song với đường thẳng $d \colon 2x+y+7=0$?
	\choice
	{$2x+y+1=0$ hoặc $2x+y-1=0$}
	{\True $2x+y=0$ hoặc $2x+y-10=0$}
	{$2x+y+10=0$ hoặc $2x+y-10=0$}
	{$2x+y=0$ hoặc $2x+y+10=0$}
	\loigiai{
		Đường tròn $(C)$ có tâm $I\left(3;-1\right)$ bán kính $R=\sqrt{5}$. \\
		Tiếp tuyến song song với $d\colon 2x+y+7=0$ nên có dạng
		$\Delta \colon 2x+y+c=0$,  $\left(c\neq 7\right)$.\\
		Vì $\Delta$ là tiếp tuyến của đường tròn nên 
		$$\mathrm{d}\left(I;\Delta \right)=R\Leftrightarrow \dfrac{\left| c+5\right|}{\sqrt{5}}=\sqrt{5}\Leftrightarrow \left| c+5\right|=5 \Leftrightarrow  \hoac{
			& c=0 \text{ (thỏa điều kiện)}\\ 
			& c=-10  \text{ (thỏa điều kiện)}.}$$
		Vậy các tiếp tuyến thỏa yêu cầu bài toán là $2x+y=0$; $2x+y-10=0$.
	}
\end{ex}

\begin{ex}%[0H7KK-3]
	Trong mặt phẳng với hệ trục tọa độ $Oxy$, cho đường tròn $(C)\colon x^2+y^2-2x+8y+1=0$ và đường thẳng $d\colon 5x+12y-6=0$. Phương trình các đường thẳng song song với $d$ và tiếp xúc với $(C)$ là
	\choice
	{$5x+12y-95=0$ và $5x+12y-9=0$}
	{$5x+12y+95=0$ và $5x+12y+9=0$}
	{$5x+12y-95=0$ và $5x+12y+9=0$}
	{\True $5x+12y+95=0$ và $5x+12y-9=0$}
	\loigiai{
		Đường tròn $ (C) $ có tâm $ I(1;-4) $ và bán kính $ R=4 $.\\
		Tiếp tuyến của $(C)$ song song với $d$ nên có dạng $\Delta \colon 5x+12y+c=0 $, ($ c\neq -6 $).\\
		Vì $\Delta$ là tiếp tuyến của đường tròn nên
		$$ 
		\mathrm{d}(I,\Delta)=R\Leftrightarrow \dfrac{|-43+m|}{13}=4\Leftrightarrow |m-43|=52\Leftrightarrow\hoac{&m=-9 \text{ (thỏa điều kiện)}\\ &m=95 \text{ (thỏa điều kiện).}}
		$$
		Vậy các tiếp tuyến thỏa yêu cầu bài toán là $5x+12y+95=0$; $5x+12y-9=0$.
	}
\end{ex}

\begin{ex}%[0H7BK-1]
	Trong mặt phẳng tọa độ $Oxy$, cho đường tròn $(\mathrm{C}): x^2+y^2+4x+4y-17=0$. Viết phương trình tiếp tuyến $\Delta$ của  $(\mathrm{C})$ biết $\Delta$ vuông góc với đường thẳng $d\colon 3x-4y+1=0$.
	\choice
	{$4x+3y+39=0$ và $4x+3y-10=0$}
	{$4x-3y+39=0$ và $4x-3y-11=0$}
	{\True $4x+3y+39=0$ và $4x+3y-11=0$}
	{$4x+3y-39=0$ và $4x+3y-10=0$}
	\loigiai{
		Đường tròn $(C)$ có tâm $I(-2;-2)$, bán kính $R=5$.\\
		Tiếp tuyến vuông góc với $d$ nên có dạng $\Delta\colon 4x+3y+c=0$.\\
		Vì $\Delta$ là tiếp tuyến của đường tròn nên
		$$d(I,\Delta)=R\Leftrightarrow \dfrac{|-14+c|}{5}=5\Leftrightarrow |-14+c|=25\Leftrightarrow \hoac{&c=39\\ &c=-11.}$$
		Vậy các tiếp tuyến thỏa yêu cầu bài toán là $\Delta\colon 4x+3y+39=0$; $\Delta\colon 4x+3y-11=0$.
	}
\end{ex}

\begin{ex}%[0H7BK-1]
	Trong mặt phẳng tọa độ $Oxy$, cho $A(1;2)$, $B(-3;1)$, $C(4;-2)$. Tập hợp các điểm $M$ thỏa mãn hệ thức $MA^2+MB^2=MC^2$ là
	\choice
	{Đường tròn tâm $I(-5;6)$ bán kính $R=\sqrt{66}$}
	{Đường tròn tâm $I(-6;5)$ bán kính $R=\sqrt{34}$}
	{\True Đường tròn tâm $I(-6;5)$ bán kính $R=\sqrt{66}$}
	{Đường tròn tâm $I(-5;6)$ bán kính $R=\sqrt{34}$}
	\loigiai{
		Giả sử $M(x;y)$. Theo giả thiết ta có
		\begin{eqnarray*}
			&&MA^2+MB^2=MC^2\\
			&\Leftrightarrow &(x-1)^2+(y-2)^2+(x+3)^2+(y-1)^2=(x-4)^2+(y+2)^2\\
			&\Leftrightarrow& x^2+y^2 +12x -10y-5=0 \\
			&\Leftrightarrow &(x+6)^2+(y-5)^2=66.
		\end{eqnarray*}
		Vậy tập các điểm $M$ là đường tròn tâm $I(-6;5)$, bán kính $R=\sqrt{66}$.
	}
\end{ex}

\begin{ex}%[0H7KK-5]
	Đường tròn $(C)$ đi qua hai điểm $A\left(-1;2\right)$, $B\left(-2;3\right)$ và có tâm $I$ thuộc đường thẳng $\Delta \colon 3x-y+10=0.$ Phương trình của đường tròn $(C)$ là
	\choice
	{$(x+3)^2+(y-1)^2=\sqrt{5}$}
	{$(x-3)^2+(y+1)^2=\sqrt{5}$}
	{$(x-3)^2+(y+1)^2=5$}
	{\True $(x+3)^2+(y-1)^2=5$}
	
	\loigiai{
		Ta có $I\in \Delta \Rightarrow I\left(m;3m+10\right)$.
		\begin{eqnarray*}
			&&AI^2=BI^2 \text{ (cùng bằng } R^2) \\
			& \Leftrightarrow & \left(m+1\right)^2+\left(3m+8\right)^2=\left(m+2\right)^2+\left(3a+7\right)^2\\
			& \Leftrightarrow & 4m+12=0\\
			& \Leftrightarrow & m=-3.
		\end{eqnarray*}
		Do đó, đường tròn $(C)$ có $\heva{&\text{tâm } I(-3;1)\\& \text{bán kính } R=AI=\sqrt{(-3+1)^2+(-9+8)^2}=\sqrt{5}.}$\\
		Vậy đường tròn $(C)\colon {\left(x+3\right)}^2+{\left(y-1\right)}^2=5.$
	}
\end{ex}

\begin{ex}%[0H7KK-5]
	Đường tròn $(C)$ có tâm $I$ thuộc đường thẳng $d\colon x+3y+8=0$, đi qua điểm $A\left(-2;1\right)$ và tiếp xúc với đường thẳng $\Delta:3x-4y+10=0$. Phương trình của đường tròn $(C)$ là:
	\choice
	{${\left(x-2\right)}^2+{\left(y+2\right)}^2=25$}
	{${\left(x+5\right)}^2+{\left(y+1\right)}^2=16$}
	{${\left(x+2\right)}^2+{\left(y+2\right)}^2=9$}
	{\True ${\left(x-1\right)}^2+{\left(y+3\right)}^2=25$}
	\loigiai{
		Ta có $I\in d \Rightarrow I(-3m-8;m)$.
		\begin{eqnarray*}
			&&AI^2=\left[\mathrm{d}(I,\Delta)\right]^2 \text{ (cùng bằng } R^2) \\
			& \Leftrightarrow & \left(-3m-6\right)^2+\left(m-1\right)^2=\dfrac{\left[3(-3m-8)-4m+10\right]^2}{3^2+(-4)^2}\\
			& \Leftrightarrow & 10m^2+34m+37=\dfrac{169m^2+364m+196}{25}\\
			& \Leftrightarrow & 81m^2+486m+729=0\\
			& \Leftrightarrow & m=-3.
		\end{eqnarray*}
		Do đó, đường tròn $(C)$ có $\heva{&\text{tâm } I(1;-3)\\& \text{bán kính } R=AI=\sqrt{(9-6)^2+(-3-1)^2}=5.}$\\
		Vậy phương trình đường tròn là ${\left(x-1\right)}^2+{\left(y+3\right)}^2=25$.
	}
\end{ex}
\begin{ex}%[0H7KK-5]
	Đường tròn $(C)$ đi qua hai điểm $A\left(-1;1\right), B\left(3;3\right)$ và tiếp xúc với đường thẳng $\Delta:3x-4y+8=0$. Viết phương trình đường tròn $(C)$, biết tâm của $(C)$ có hoành độ nhỏ hơn $5.$
	\choice
	{\True $(x-3)^2+(y+2)^2=25$}
	{$(x-1)^2+(y-2)^2=5$}
	{$(x+1)^2+(y-6)^2=25$}
	{$x^2+(y-4)^2=10$}
	\loigiai{
		Gọi $M$ là trung điểm $AB$. Suy ra $M\left(1;2\right)$.\\
		Trung trực của đoạn thẳng $AB$ qua $M$ và nhận $\overrightarrow{AB}=(4;2)$ làm VTPT nên có dạng 
		$$d\colon 4(x-1)+2(y-2)=0 \Leftrightarrow 2x+y-4=0.$$
		Gọi $I$ là tâm của đường tròn $(C)$. Vì $(C)$ qua $A$ và $B$ nên $I\in d \Rightarrow I\left(m;-2m+4\right)$ với $m<5$.\\ 
		Mặt khác $(C)$ tiếp xúc với đường thẳng $\Delta$ nên
		\begin{eqnarray*}
			&&AI^2=\left[\mathrm{d}(I,\Delta)\right]^2 \text{ (cùng bằng } R^2) \\
			& \Leftrightarrow & \left(m+1\right)^2+\left(-2m+3\right)^2=\dfrac{\left[3m-4(-2m+4)+8\right]^2}{3^2+(-4)^2}\\
			& \Leftrightarrow & 5m^2-10m+10=\dfrac{121m^2-176m+64}{25}\\
			& \Leftrightarrow & 4m^2-74m+186=0\\
			& \Leftrightarrow & \hoac{&m=3 \text{ (thỏa điều kiện)}\\& m=\dfrac{31}{2} \text{ (không thỏa điều kiện).}}
		\end{eqnarray*}
		Do đó, đường tròn $(C)$ có $\heva{&\text{tâm } I(3;-2)\\& \text{bán kính } R=AI=\sqrt{(3+1)^2+(-6+3)^2}=5.}$\\
		Vậy phương trình đường tròn là ${\left(x-3\right)}^2+{\left(y+2\right)}^2=25.$
	}
\end{ex}

\begin{ex}%[0H7KK-3]
	Trong mặt phẳng $Oxy$, cho đường tròn $(C)\colon x^2+(y-4)^2=10$. Tìm tất cả các tiếp tuyến của $(C)$, biết rằng tiếp tuyến đi qua điểm $M(-4;2)$?
	\choice
	{$x+3y-2=0$ và $3x-y-14=0$}
	{\True $x+3y-2=0$ và $3x-y+14=0$}
	{$x+3y+2=0$ và $3x-y+14=0$}
	{$x+3y+2=0$ và $3x-y-14=0$}
	\loigiai{
		Đường tròn $(C)$ có tâm $I\left(0;4\right)$, bán kính $R=\sqrt{10}$. \\
		Tiếp tuyến của $(C)$ qua $M$ có dạng
		$\Delta \colon ax+by+4a-2b=0$ với $\left(a^2+b^2\neq 0\right)$.	\\ 
		Vì $\Delta$ là tiếp tuyến của đường tròn nên
		$$d\left[I;\Delta \right]=R\Leftrightarrow \dfrac{\left| 4a+2b\right|}{\sqrt{a^2+b^2}}=\sqrt{10} \Leftrightarrow 6a^2+16ab-6b^2=0\Leftrightarrow \hoac{
			& a=\dfrac{1}{3}b \Rightarrow b=3, a=1 \\ 
			& a=-3b \Rightarrow b=-1, a=3.}$$
		Vậy các tiếp tuyến thỏa yêu cầu bài toán là $\Delta\colon x+3y-2=0$; $\Delta\colon 3x-y+14=0$.
	}
\end{ex}

\begin{ex}%[0H7KK-3]
	Viết phương trình tiếp tuyến $\Delta $ của đường tròn $(C) \colon x^2+y^2-4x-4y+4=0$, biết tiếp tuyến đi qua điểm $B\left(4;6\right)$.
	\choice
	{$\Delta:x-4=0$ hoặc $\Delta:3x+4y-36=0$}
	{$\Delta:x-4=0$ hoặc $\Delta:y-6=0$}
	{$\Delta:y-6=0$ hoặc $\Delta:3x+4y-36=0$}
	{\True $\Delta:x-4=0$ hoặc $\Delta:3x-4y+12=0$}
	\loigiai{
		Đường tròn $(C)$ có tâm $I\left(2;2\right)$, bán kính $R=2$.\\
		Tiếp tuyến của $(C)$ qua $B$ có dạng
		$\Delta \colon ax+by-4a-6b=0$ với $\left(a^2+b^2\neq 0\right)$.	\\ 
		Vì $\Delta$ là tiếp tuyến của đường tròn nên
		$$d\left[I;\Delta \right]=R\Leftrightarrow \dfrac{\left| 2a+4b\right|}{\sqrt{a^2+b^2}}=2\Leftrightarrow b\left(3b+4a\right)=0\Leftrightarrow \hoac{
			& b=0\Rightarrow a=1,b=0 \\ 
			& 3b=-4a\Rightarrow a=3,b=-4.}$$
		Vậy các tiếp tuyến thỏa yêu cầu bài toán là $\Delta\colon x-4=0$; $\Delta\colon 3x-4y+12=0$.
	}
\end{ex}

\Closesolutionfile{ans}
\Closesolutionfile{ansbook}
% \begin{center}
% 	\textbf{ĐÁP ÁN}
% \end{center}
% \indapan{10}{ans/ans-0H7-21-TNNC}
