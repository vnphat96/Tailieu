\setcounter{dang}{0}
\section{Các số đặc trưng đo độ phân tán}
\subsection{Tóm tắt lý thuyết}
\subsubsection{Khoảng biến thiên và khoảng tứ phân vị}
\begin{boxdn}
	\textbf{Khoảng biến thiên}, kí hiệu là $R$, là hiệu số giữa giá trị lớn nhất và giá trị nhỏ nhất trong mẫu số liệu, tức là \fbox{$R=\max - \min$}.
\end{boxdn}
\begin{boxdn}
	\textbf{Khoảng tứ phân vị}, kí hiệu là $ \Delta Q $, là hiệu số giữa tứ phân vị thứ ba và tứ phân vị thứ nhất, tức là: \fbox{$ \Delta Q=Q_3-Q_1 $}.
\end{boxdn}
\begin{itemize}
	\item Khoảng biến thiên và khoảng tứ phân vị là các số đo độ phân tán của mẫu số liệu. Khoảng biến thiên, khoảng tứ phân vị càng lớn thì mẫu số liệu càng phân tán.
	\item Về bản chất, khoảng tứ phân vị là khoảng biến thiên của $50 \%$ số liệu chính giữa của mẫu số liệu đã sắp xếp.
	\item Một số tài liệu gọi khoảng biến thiên là biên độ và khoảng tứ phân vị là độ trải giữa.
	\item Khoảng biến thiên bị ảnh hưởng bởi các số liệu bất thường, trong khi đó khoảng tứ phân vị thì không.
\end{itemize}
\subsubsection{Phương sai và độ lệch chuẩn}
\begin{boxdn}
	Cho mẫu số liệu $x_{1}, x_{2}, \ldots, x_{n}$, có số trung bình là $\overline{x}$ thì với mỗi giá trị $x_{i}$, độ lệch của nó so với giá trị trung bình là $x_{i}-\overline{x}$.
	\begin{itemize}
		\item \textbf{Phương sai} là giá trị \fbox{$s^2=\dfrac{\left(x_{1}-\overline{x}\right)^{2}+\left(x_{2}-\overline{x}\right)^{2}+\ldots+\left(x_{n}-\overline{x}\right)^{2}}{n}$}.\\
		\item Căn bậc hai của phương sai, $s=\sqrt{s^{2}}$, được gọi là \textbf{độ lệch chuẩn}.
	\end{itemize}
\end{boxdn}
\textbf{Ý nghĩa:} Phương sai, độ lệch chuẩn là các số đo độ phân tán của mẫu số liệu. Phương sai, độ lệch chuẩn càng lớn thì mẫu số liệu càng phân tán.
\subsubsection{Phát hiện số liệu bất thường}
\begin{itemize}
	\item  Trong mẫu số liệu thống kê, có khi gặp những giá trị quá lớn 
	hoặc quá nhỏ so với đa số các giá trị khác. Những giá trị này được gọi 
	là \textbf{giá trị bất thường}.
	% \begin{center}
	% 	\begin{tikzpicture}[>=stealth, font=\footnotesize, line join=round, line cap=round,transform shape,scale=1]
	% 		\begin{scope}[scale=1]
	% 			%% Vẽ hộp
	% 			\foreach \t in {7, 8.5, 2, 11} \draw (\t,0.5)--(\t,2);
	% 			\draw (5.2,0.5)--(8.5,0.5) (5.2,2)--(8.5,2);
	% 			\draw (2,1.25)--(11,1.25);
	% 			\fill[black!20!white] (5.2,0.5) rectangle (8.5,2);
	% 			\draw[color=yellow, thick] (7,0.5)--(7,2);
	% 			%% ghi các giá trị phía dưới
	% 			\foreach \e/\f  in {2/(Q_1-1{,}5 \cdot  
	% 				\Delta_{Q}),7/Q_2,5.2/Q_1,8.5/Q_3,11/(Q_3+1{,}5 \cdot 
	% 				\Delta_{Q})}
	% 			{
	% 				\node[shift={(-90:4mm)}] at (\e,0.5){$\f$};
	% 			}
	% 			%% Vẽ giá rị bất thường
	% 			\foreach \btd  in {0.2,0.6}	
	% 			\draw (2-\btd,1.25) circle (0.08);
	% 			\foreach \btt  in {0.2,0.4}	
	% 			\draw (11+\btt,1.25) circle (0.08);
	% 			\draw [thick,->](2-0.2,2.2)--(2-0.2,1.35);
	% 			\draw [thick,->](11+0.4,2.2)--(11+0.4,1.35);
	% 			\node [text = red,
	% 			fill = green!30!white,rounded corners=8pt] at (2 -0.5,2.5){Các giá trị bất thường};
	% 			\node [text = red,
	% 			fill = green!30!white,rounded corners=8pt]  at (11 +0.5,2.5){Các giá trị bất thường};
	% 			%%% Vẽ delta Q
	% 			\draw (5.2,2 +0.2)--node[above]{$\Delta_{Q}$}(8.5,2+0.2);
	% 			\draw (5.2,2 +0.1)--(5.2,2 +0.3) (8.5,2 +0.1)--(8.5,2 +0.3);
	% 		\end{scope}
	% 		\draw (-1,-1) rectangle (14,2+1);
	% 	\end{tikzpicture}
	% \end{center}
	\item Các số liệu \textbf{lớn hơn} $Q_3+1{,}5 \cdot 
	\Delta_{Q}$ hoặc \textbf{bé hơn} $Q_1-1{,}5 \cdot 
	\Delta_{Q}$ được xem là giá trị bất thường.\\
	Hay các số liệu \textbf{không} thuộc đoạn $\left[Q_1-1{,}5 \cdot \Delta_{Q}; Q_3+1{,}5 \cdot \Delta_{Q} \right]$ là các số liệu bất thường.
\end{itemize}

\subsection{Các dạng toán}
%Mục 1: Khoảng biến thiên và khoảng tứ phân vị
\begin{dang}{Tìm khoảng biến thiên và so sánh độ phân tán của một hoặc nhiều mẫu số liệu}
	% Để tìm khoảng biến thiên của một mẫu số liệu ta thực hiện các bước giải như sau:
	% \begin{itemize}
	% 	\item \textbf{\underline{Bước 1} } Tìm giá trị lớn nhất và giá trị nhỏ nhất của mẫu số liệu.
	% 	\item \textbf{\underline{Bước 2} } Khoảng biến thiên của mẫu số liệu được tính bằng hiệu số giữa giá trị lớn nhất và giá trị nhỏ nhất tìm được ở trên.
	% \end{itemize}
	% \begin{itemize}
	% 	\item \textbf{\underline{Bước 3} } Nếu bài toán cho nhiều mẫu số liệu và yêu cầu so sánh độ phân tán của chúng dựa vào khoảng biến thiên thì hãy chú ý rằng khoảng biến thiên càng lớn thì mẫu số liệu cảng phân tán.
	% \end{itemize}
\end{dang}
\viduminhhoa	
\begin{vd}%[Khuyến Kỷ Luật,  BG10-2022]
	Cân nặng (tính theo đơn vị kg) của $10$ học sinh được ghi lại như sau:
	\begin{center}
		49 \quad 57 \quad 66 \quad 45 \quad 50 \quad 41 \quad 57 \quad 42 \quad 55 \quad 52
	\end{center}
	Hãy tìm khoảng biến thiên của mẫu số liệu trên.
	\loigiai{
		Mẫu số liệu có giá trị lớn nhất và giá trị nhỏ nhất lần lượt là $ 66 $ và $ 41 $.\\
		Do đó khoảng biến thiên của mẫu số liệu đã cho là $ R=66-41=25 $.
	}
\end{vd}

\begin{vd}%[Khuyến Kỷ Luật,  BG10-2022]
	Chiều cao (tính theo đơn vị m) của các bạn học sinh trong một lớp học được thống kê và ghi lại trong bảng dưới đây:
	\begin{longtable}{|l|c|c|c|c|c|c|c|}
		\hline
		Chiều cao & 1{,}6 & 1{,}61 & 1{,}62 & 1{,}63 & 1{,}64 & 1{,}65\\
		\hline
		Số lượng & 3 & 5 & 8 & 9 & 7 & 6\\
		\hline
	\end{longtable}
	Hãy tìm khoảng biến thiên của mẫu số liệu trên
	\loigiai{
		Mẫu số liệu có giá trị lớn nhất và giá trị nhỏ nhất lần lượt là $ 1{,}65 $ và $ 1{,}61 $.\\
		Do đó khoảng biến thiên của mẫu số liệu đã cho là $ R=1{,}65-1{,}61=0{,}04 $.
	}
\end{vd}

\begin{vd}%[Khuyến Kỷ Luật,  BG10-2022]
	Điểm kiểm tra môn Toán của các bạn học sinh Tổ 1 và Tổ 2 lớp 10C như sau:
	\begin{longtable}{ccccccccc}
		Tổ 1: & 6 & 9 & 4 & 2 & 7 & 9 & 6 & 10\\
		Tổ 2: & 4 & 5 & 6 & 3 & 9 & 5 & 8 & 4\\
	\end{longtable}
	Hãy tìm các khoảng biến thiên trong hai mẫu số liệu. Căn cứ vào số liệu này, hãy chỉ ra tổ nào học đồng đều hơn.
	\loigiai{
		Mẫu số liệu ``Tổ 1'' có giá trị lớn nhất và giá trị nhỏ nhất lần lượt là $ 10 $ và $ 2 $.\\
		Do đó khoảng biến thiên của mẫu số liệu đã cho là $ R_1=10-2=8 $.\\
		Mẫu số liệu ``Tổ 2'' có giá trị lớn nhất và giá trị nhỏ nhất lần lượt là $ 66 $ và $ 41 $.\\
		Do đó khoảng biến thiên của mẫu số liệu đã cho là $ R_2=9-3=6 $.\\
		Mẫu số liệu ``Tổ 2'' có\\
		Do $R_1>R_2$ nên ta nói các bạn Tổ 2 học đều hơn các bạn Tổ 1.	
	}
\end{vd} 



%Mục 2: Phương sai và độ lệch chuẩn
\begin{dang}{Tính phương sai và độ lệch chuẩn}
	a) Tính phương sai: để tính phương sai $s^{2}$ của một mẫu số liệu $\left\{x_{1}; x_{2}; \ldots ; x_{n}\right\}$ ta thực hiện một trong các cách sau:\\
	\begin{itemize}
		\item Tính số trung bình:
		$\bar{x}=\dfrac{1}{n} \displaystyle\sum\limits_{k=1}^{n} x_{i}$.
		\item Tính các độ lệch:
		$x_{i}-\overline{x}(i=\overline{1, n})$.
		\item Tính phương sai theo công thức:
		$$s^{2}=\dfrac{1}{n} \displaystyle\sum\limits_{i=1}^{n}\left(x_{i}-\overline{x}\right)^{2}=\dfrac{\left(x_{1}-\overline{x}\right)^{2}+\left(x_{2}-\overline{x}\right)^{2}+\ldots+\left(x_{n}-\overline{x}\right)^{2}}{n}.$$
	\end{itemize}
	b) Tính độ lệch chuẩn: Độ lệch chuẩn $s$ bằng căn bậc $2$ của phương sai
\end{dang}
\viduminhhoa	
\begin{vd}%[Lê Minh Thiện Anh]%[0D5B14-1]
	Sản lượng lúa (đơn vị là tạ) của $40$ thửa ruộng thí nghiệm có cùng diện tích được trình bày trong bảng tần số sau đây
	\begin{center}
		\begin{tabular}{|c|c|c|c|c|c|c|}
			\hline
			Sản lượng $(x)$ & $20$ & $21$ & $22$ & $23$ & $24$ &      \\ \hline
			Tần số $(n)$    & $5$  & $8$  & $11$ & $10$ & $6$  & $n=40$ \\ \hline
		\end{tabular}
	\end{center}
	a) Tính sản lượng trung bình của $40$ thửa ruộng.\\
	b) Tính phương sai và độ lệch chuẩn.
	\loigiai{a) Số trung bình sản lượng của $40$ thửa ruộng là
		$$
		\overline{x}=\dfrac{5\cdot20+821+11\cdot22+10\cdot23+6\cdot24}{40}=22{,}1 \text { (tạ). }
		$$
		b) Tính phương sai.\\
		\textbf{Cách 1:} $s^{2}=\dfrac{1}{n} \displaystyle\sum\limits_{i=1}^{5}\left(x_{1}-\overline{x}\right)^{2}=$\\
		$=\dfrac{1}{40}\left[5(20-22{,}1)^{2}+8(21-22{,}1)^{2}+11(22-22{,}1)^{2}+10(23-22,1)^{2}+6(24-22,1)^{2}\right]$\\
		$=\dfrac{1}{40}\left[5(2{,}1)^{2}+8(1{,}1)^{2}+11(0{,}1)^{2}+10(0{,}9)^{2}+6(1{,}9)^{2}\right]=\dfrac{6160}{4000}=1{,}54$.\\
		\textbf{Cách 2:}
		$\displaystyle\sum\limits_{i=1}^{5} n_{i} x_{i}=5\cdot20+8\cdot21+11\cdot22+10\cdot23+6\cdot24=884$;\\
		$\displaystyle\sum\limits_{i=1}^{5} n_{i} x_{i}^{2}=5\cdot20^{2}+8\cdot21^{2}+11\cdot22^{2}+10\cdot23^{2}+6\cdot24^{2}=19598$.\\
		Do đó $s^{2}=\dfrac{1}{n} \displaystyle\sum\limits_{i=1}^{n} n_{i} x_{i}^{2}-\dfrac{1}{n^{2}}\left(\displaystyle\sum\limits_{i=1}^{n} n_{i} x_{i}\right)^{2}=\dfrac{1}{40}\cdot19598-\dfrac{1}{40^{2}} \cdot884^{2}=1{,}54$.\\
		Tính độ lệch chuẩn: $s=\sqrt{s^{2}}=\sqrt{1{,}54}\approx1{,}24$.
	}
\end{vd}

\begin{vd}%[Lê Minh Thiện Anh]%[0D5B14-1]
	$100$ học sinh tham dự kì thi học sinh giỏi Toán (thang diểm là $20$). Kết quả được cho trong bảng sau:
	\begin{center}
		\begin{tabular}{|c|c|c|c|c|c|c|c|c|c|c|c|c|}
			\hline Điểm & $9$ & $10$ & $11$ & $12$ & $13$ & $14$ & $15$ & $16$ & $17$ & $18$ & $19$ & \\
			\hline Tần số & $1$ & $1$ & $3$ & $5$ & $8$ & $13$ & $19$ & $24$ & $14$ & $10$ & $2$ & $n=100$ \\
			\hline
		\end{tabular}
	\end{center}
	a) Tính số trung bình.\\
	b) Tìm phương sai và độ lệch chuẩn.
	\loigiai{ 
		a)	Tính số trung bình.\\
		$\displaystyle\sum\limits_{i=1}^{11} n_{i} x_{i}=1\cdot9+1\cdot10+3\cdot11+5\cdot12+8\cdot13+13\cdot14+19\cdot15+24\cdot16+14\cdot17+10\cdot18+2\cdot19=1523$\\
		$\Rightarrow$ số trung bình là $\overline{x}=\dfrac{1523}{100}=15{,}23$.\\
		b) Ta có $\displaystyle\sum\limits_{i=1}^{11} n_{i} x_{i}=1523$ và $\displaystyle\sum\limits_{i=1}^{11} n_{i} x_{i}^{2}=23591 \Rightarrow$ phương sai là
		$$s^{2}=\dfrac{1}{n} \sum_{i=1}^{n} n_{i} x_{i}^{2}-\dfrac{1}{n^{2}}\left(\displaystyle\sum\limits_{i=1}^{n} n_{i} x_{i}\right)^{2}=\dfrac{1}{100}\cdot23591-\dfrac{1}{100^{2}}\cdot(1523)^{2}\approx3{,}96.$$
		Độ lệch chuẩn là $s=\sqrt{s^{2}}\approx\sqrt{3{,}96}\approx1{,}99$.	
	}
\end{vd}




%Mục 3: Phát hiện số liệu bất thường
\begin{dang}{Tìm các số liệu bất thường của mẫu số liệu}
	Để tìm các số liệu bất thường của một mẫu số liệu ta thực hiện các bước giải như sau:
	\begin{itemize}
		\item \textbf{\underline{Bước 1} } Tìm các tứ phân vị thứ nhất, tứ phân vị thứ hai, tứ phân vị thứ 3 và khoảng tứ phân vị $\Delta_{Q}=Q_3 - Q_1$ của mẫu số liệu.
		\item \textbf{\underline{Bước 2} } Tìm ra đoạn $\left[Q_1-1{,}5 \cdot 
		\Delta_{Q}; Q_3+1{,}5 \cdot 
		\Delta_{Q} \right]$.
		\item \textbf{\underline{Bước 3} } Tìm các số liệu \textbf{không} thuộc đoạn $\left[Q_1-1{,}5 \cdot	\Delta_{Q}; Q_3+1{,}5 \cdot \Delta_{Q} \right]$ là các số liệu bất thường.
	\end{itemize}
\end{dang}
\viduminhhoa	
\begin{vd}%[Phan Quốc Trí,  BG10-2022]%[0D5B3-5]
	Điểm kiểm tra môn Toán của $10$ học sinh sau:\\
	$1$ \quad$7$ \quad$10$ \quad$7$ \quad$7$ \quad$6$ \quad$9$ \quad$8$ \quad$10$ \quad$8$\\
	Hãy tìm các số liệu bất thường trong mẫu số liệu trên.
	\loigiai{
		Sắp xếp mẫu số liệu theo thứ tự không giảm từ trái qua phải như sau
		$$1\quad 6\quad 7\quad 7\quad 7\quad 8\quad 8 \quad 9 \quad 10\quad 10$$
		Vì $n=10$ chẵn nên $Q_2$ là trung bình cộng của hai số liệu chính giữa. Ta có $Q_2 = \dfrac{8+7}{2}= 7{,}5$.\\
		Tứ phân vị dưới $Q_1=7$, tứ phân vị trên $Q_3= 9$ và $\Delta_{Q} = Q_3-Q_1 = 2$.\\
		% Ta có $\left[Q_1-1{,}5 \cdot  \Delta_{Q}; Q_3+1{,}5 \cdot  \Delta_{Q} \right] =\left[4;12\right]$. Biểu đồ hộp cho mẫu số liệu là \\
		% \begin{center}
		% 	\begin{tikzpicture}[>=stealth, font=\footnotesize, line join=round, line cap=round,transform shape,scale=1]
		% 		\def\d{1}  \def\c{13} % điểm đầu, điểm cuối
		% 		\def\sc{1} % điều chỉnh khoảng cách các dấu gạch
				
		% 		\def\qm{7} %Q2
		% 		\def\qh{7.5} %Q1
		% 		\def\qb{9}  %Q3
				
		% 		\def\cd{4}   % Q2-1.5 delta
		% 		\def\ct{12} %% Q2+1.5 delta
		% 		\def\cao{2}  %Chiều cao của hộp
		% 		\begin{scope}[scale=\sc]
		% 			%Vẽ trục dưới
		% 			\draw (\d,0) -- (\c,0);
		% 			\draw[snake=ticks,segment length=\sc cm] (\d,0) -- (\c,0);
		% 			%% ghi số lên trục
		% 			\foreach \t/\s  in {1/1,2/2,3/3,4/4,5/5, 6/6,7/7, 8/8, 9/9, 10/10}
		% 			{
		% 				\node[shift={(-90:3mm)}] at (\t,0){$\s$};
		% 				%	\node[shift={(-90:20mm)}] at (\t,0){$\t$};
		% 			}
		% 			%% Vẽ hộp
		% 			\foreach \r in {\qh, \qb, \cd, \ct} \draw (\r,0.5)--(\r,\cao);
		% 			\draw (\qm,0.5)--(\qb,0.5) (\qm,\cao)--(\qb,\cao);
		% 			\draw (\cd,1.25)--(\ct,1.25);
		% 			\fill[black!20!white] (\qm,0.5) rectangle (\qb,\cao);
		% 			\draw[color=yellow, thick] (\qh,0.5)--(\qh,\cao);
		% 			%% ghi các giá trị phía trên
		% 			\foreach \e/\f  in {\cd/4,\qh/7.5,\qm/7,\qb/9,\ct/12}
		% 			{
		% 				\node[shift={(90:3mm)}] at (\e,\cao){$\f$};
		% 			}
		% 			%% Vẽ giá rị bất thường
		% 			\foreach \bt/\gt  in {1/1}
		% 			{
		% 				\node[shift={(90:3mm)}] at (\bt,\cao){$\gt$};
		% 				\draw (\bt,1.25) circle (0.051);
		% 			}
		% 		\end{scope}
		% 	\end{tikzpicture}
		% \end{center}
		Vì $1\notin \left[4;12\right]$ nên là số liệu bất thường trong mẫu số liệu trên.
		
	}
\end{vd}

\baitaptl
%Mục 1:
\begin{bt}%[Khuyến Kỷ Luật,  BG10-2022]
	Hai chữ số cuối số điện thoại của $ 10 $ người được thống kế như sau:
	\begin{longtable}{cccccccccc}
		23 & 58 & 42 & 11 & 69 & 50 & 13 & 57 & 61 & 72
	\end{longtable}
	Hãy tìm khoảng biến thiên của mẫu số liệu trên.
	\loigiai{
		Khoảng biến thiên của mẫu số liệu là $R=72-11=61$.		
	}
\end{bt}

\begin{bt}%[Khuyến Kỷ Luật,  BG10-2022]
	Tuổi thọ trung bình người dân của $ 11 $ nước được thống kê như sau:
	\begin{longtable}{ccccccccccc}
		69 & 77 & 75 & 83 & 65 & 75 & 74 & 68 & 73 & 72 & 71
	\end{longtable}
	Hãy tìm khoảng biến thiên của mẫu số liệu trên.
	\loigiai{
		Khoảng biến thiên của mẫu số liệu là $R=83-65=18$.		
	}
\end{bt}

\begin{bt}%[Khuyến Kỷ Luật,  BG10-2022]
	Thời gian làm câu đầu tiên trong đề thi tuyển sinh vào lớp 10 tại một trường của các bạn học sinh được thống kê và ghi lại trong bảng sau
	\begin{longtable}{|c|c|c|c|c|c|c|c|c|c|}
		\hline 
		Thời gian & 9 & 10 & 11 & 12 & 13 & 14 & 15 \\ 
		\hline 
		Số lượng học sinh & 45 & 46 & 57 & 63 & 70 & 61 & 50 \\ 
		\hline 
	\end{longtable}
	Hãy tìm khoảng biến thiên của mẫu số liệu thống kê trên.
	\loigiai{
		Khoảng biến thiên của mẫu số liệu là $R=15-9=6$.		
	}
\end{bt}

\begin{bt}%[Khuyến Kỷ Luật,  BG10-2022]
	Điểm thi học kì 2 môn Toán và Ngữ văn của một nhóm học sinh được ghi lại như sau
	\begin{longtable}{|l|c|c|c|c|c|c|c|c|c|}
		\hline 
		Toán & 9 & 8{,}5 & 7 & 6{,}3 & 5 & 9{,}5 & 8 \\ 
		\hline 
		Ngữ văn & 6 & 6{,}5 & 8 & 7{,}3 & 5{,}5 & 8{,}3 & 6{,}5 \\ 
		\hline 
	\end{longtable}
	Hãy tìm khoảng biến thiên của hai mẫu số liệu trên. Từ đó chỉ ra mẫu số liệu có độ phân tán lớn hơn.
	\loigiai{
		Khoảng biến thiên của mẫu số liệu ``Toán'' là $ R_1=9{,}5-5=4{,}5 $.\\
		Khoảng biến thiên của mẫu số liệu ``Ngữ văn'' là $ R_2=8{,}3-5{,}5=2{,}8 $.\\
		Do $R_1>R_2$ nên mẫu số liệu ``Toán'' có độ phân tán lớn hơn mẫu số liệu ``Ngữ văn''.
	}
\end{bt}

%Mục 2: Phương sai và độ lệch chuẩn
\begin{bt}%[Lê Minh Thiện Anh]%[0D5B14-1]
	Một xạ thủ tập bắn, xạ thủ đó đã bắn $30$ viên đạn vào bia. Kết quả được cho trong bảng sau
	\begin{center}
		\begin{tabular}{|l|c|c|c|c|c|c|}
			\hline
			Điểm   & 6 & 7 & 8 & 9 & 10 &      \\ \hline
			Tần số & 3 & 4 & 8 & 9 & 6  & $n=30$ \\ \hline
		\end{tabular}
	\end{center}
	a) Tính điểm trung bình của xạ thủ.\\
	b) Tìm phương sai và độ lệch chuẩn.
	\loigiai{ 
		a)	Điểm trung bình của xạ thủ.\\
		$$\overline{x}=\dfrac{1}{30}\left(3\cdot6+4\cdot7+8\cdot8+9\cdot9+6\cdot10 \right)\approx 8{,}37.$$
		b) Phương sai\\
		$$s^2=\dfrac{1}{30}\left[ 3\cdot{\left( 6-8{,}37 \right)^2}+4\cdot{\left( 7-8{,}37 \right)^2}+8\cdot{\left( 8-8{,}37 \right)^2}+9\cdot{\left( 9-8{,}37 \right)^2}+6\cdot{\left( 10-8{,}37 \right)^2} \right]\approx1{,}50.$$
		Độ lệch chuẩn là $s=\sqrt{s^{2}}\approx\sqrt{1{,}50}\approx1{,}22$.	
	}
\end{bt}

\begin{bt}%[Lê Minh Thiện Anh]%[0D5B14-1]
	Hai lớp 10A1, 10A2 của một trường Trung học phổ thông X đồng thời làm bài thi môn Toán theo cùng một đề thi. Kết quả thi được trình bày ở hai bảng phân bố tần số sau đây\\
	Điểm thi Toán của lớp 10A1
	\begin{center}
		\begin{tabular}{|c|c|c|c|c|c|c|c|} 
			\hline Điểm thi & 5 & 6 & 7 & 8 &9 &10&Cộng \\ 
			\hline Tần số & 3 & 7 & 12 & 14 &3 &1&40 \\ 
			\hline 
		\end{tabular}
	\end{center}
	Điểm thi Toán của lớp 10A2
	\begin{center}
		\begin{tabular}{|c|c|c|c|c|c|} 
			\hline Điểm thi & 6 & 7 & 8 & 9 & Cộng \\ 
			\hline Tần số & 8 & 18 & 10 & 4 & 40 \\ 
			\hline 
		\end{tabular}
	\end{center}
	a) Tính phương sai, độ lệch chuẩn của các bảng phân bố tần số đã cho.\\
	b) Xét xem kết quả làm bài thi của môn Toán ở lớp nào đồng đều hơn?
	\loigiai{a) Trong dãy số liệu về điểm thi của lớp 10A1 ta có
		$$
		\overline{x}=\dfrac{1}{40}(3\cdot5+7\cdot6+12\cdot7+14\cdot8+3\cdot9+1\cdot10) = 7{,}25 \text { điểm.}
		$$
		Phương sai $$s_{1}^{2}=\dfrac{1}{40}\left[3(5-7{,}25)^{2}+7 (6-7{,}25)^{2}+12(7-7{,}25)^{2}+14(8-7{,}25)^{2}+3(9-7{,}25)^{2}+1 (10-7{,}25)^{2}\right] \approx 1{,}3.$$
		Độ lệch chuẩn $s_{1} \approx 1{,}14$.\\
		Trong dãy số liệu về điểm thi của lớp 10A2 ta có\\
		$$\overline{y}=\dfrac{1}{40}(8\cdot6+18\cdot7+10\cdot8+4\cdot9) = 7{,}25 \text { điểm }.$$
		Phương sai $$s_{2}^{2}=\dfrac{1}{40}\left[8(6-7{,}25)^{2}+18 (7-7{,}25)^{2}+10(8-7{,}25)^{2}+4(9-7{,}25)^{2}\right] \approx 0{,}8.$$
		Độ lệch chuẩn $s_{2} \approx 0{,}9$.\\
		b) Các số liệu thống kê có cùng đơn vị đo, $\overline{x} = \overline{y} = 7{,}25$; $s_{1}^{2}>s_{2}^{2}$, suy ra điểm số của các bài thi ở lớp 10A2 là đồng đều hơn.	
	}
\end{bt}


%Mục 3: Phát hiện số liệu bất thường
\begin{bt}%[Phan Quốc Trí,  BG10-2022]%[0D5B3-5]
	Tuổi thọ của $30$ bóng đèn được thắp thử (đơn vị: giờ) có kết quả như trong bảng sau:
	\begin{center}	
		\begin{tabular}{|c|c|c|c|c|c|c|c|c|c|}
			\hline 
			1180 & 1179 & 1187 & 1190 & 1187 & 1198 & 1568 & 1178 & 1185 & 1184 \\ 
			\hline 
			1178 & 1180 & 1185 & 1179 & 1180 & 1198 & 1179 & 1198 & 1569 & 1191 \\ 
			\hline 
			1185 & 1184 & 1179 & 1180 & 1184 & 1198 & 1180 & 1178 & 1179 & 1178 \\ 
			\hline 
		\end{tabular} 
	\end{center}
	Hãy tìm các số liệu bất thường trong mẫu số liệu trên.
	\loigiai{
		Sắp xếp các số liệu trong mẫu theo thứ tự không giảm ta có
		\begin{center}
			\begin{tabular}{|c|c|c|c|c|c|c|c|c|c|c|c|c|}
				\hline Số giờ 			& $1178$ & $1179$ & $1180$ & $1184$ & $1185$ & $1187$ &$1190$ &$1191$ &$1198$&$1568$ & $1569$ \\
				\hline Số bóng đèn  & $4$   &$ 5$    & $5$    & $3$ 	& $3$ & $2$ &$1$ &$1$ & $4$& $1$& $1$ \\
				\hline
			\end{tabular}
		\end{center} 
		Từ bảng số liệu ta tìm được số trung vị $Q_2=1184$, tứ phân vị dưới $Q_1= 1179$, tứ phân vị trên $Q_3= 1190$ và khoảng tứ phân vị $\Delta_{Q} = 1190 - 1179 = 11$.\\
		Ta có $\left[Q_1-1{,}5 \cdot  \Delta_{Q}; Q_3+1{,}5 \cdot  \Delta_{Q} \right] =\left[1162{,}5;1206{,}5\right]$.\\
		Từ đó ta có các số $1568$ và $1569$ là các số liệu bất thường của mẫu số liệu.		
	}
\end{bt}


\begin{bt}%[Phan Quốc Trí,  BG10-2022]%[0D5B3-5]
	Điều tra thời gian hoàn thành một sản phẩm của $20$ công nhân, người ta thu được mẫu số liệu sau (thời gian tính bằng phút)
	\begin{center}	
		\begin{tabular}{|c|c|c|c|c|c|c|c|c|c|}
			\hline 
			7 & 12 & 13 & 15 & 11 & 13 & 16 & 18 & 19 & 21 \\ 
			\hline 
			23 & 21 & 15 & 17 & 16 & 15 & 20 & 13 & 16 & 29 \\ 
			\hline 
		\end{tabular} 
	\end{center}
	Hãy tìm các số liệu bất thường trong mẫu số liệu trên.
	\loigiai{
		Sắp xếp các số liệu trong mẫu theo thứ tự không giảm ta có
		\begin{center}
			\begin{tabular}{|c|c|c|c|c|c|c|c|c|c|c|c|c|c|c|}
				\hline Số giờ 			& $7$ & $11$ & $12$ & $13$ & $15$ & $16$ &$17$ &$18$ &$19$&$20$ & $21$  & $23$ & $29$ \\
				\hline Số lần xuất hiện  & $1$ &$ 1$ & $1$  & $3$ & $3$ & $3$ 	 &$1$ &$1$ & $1$& $1$& $2$& $1$& $1$ \\
				\hline
			\end{tabular}
		\end{center} 
		Từ bảng số liệu ta tìm được số trung vị $Q_2=16$, tứ phân vị dưới $Q_1= 13$, tứ phân vị trên $Q_3= 19{,}5$ và khoảng tứ phân vị $\Delta_{Q} = 19{,}5 - 13 = 6{,}5$.\\
		Ta có $\left[Q_1-1{,}5 \cdot  \Delta_{Q}; Q_3+1{,}5 \cdot  \Delta_{Q} \right] =\left[3{,}25;29{,}25\right]$.\\
		Từ đó ta có mẫu số liệu trên không có số liệu bất thường.		
	}
\end{bt}

% \begin{bt}%[Phan Quốc Trí,  BG10-2022]%[0D5B3-5]
% 	Kết quả kiểm tra môn Toán của lớp $10$A có $21$ học sinh, thể hiện ở bảng dưới đây
% 	\begin{center}
% 		\begin{tabular}{|c|c|c|c|c|c|c|c|c|c|c|c|c|c|c|c|c|c|c|c|c|}
% 			\hline $10$ & $6$ & $7$ & $7$ & $1$ & $7$ & $6$ & $9$ & $9$ & $10$ & $8$ & $8$ & $7$ & $8$ & $6$ & $7$ & $5$ & $6$ & $7$ & $8$ & $9$\\
% 			\hline
% 		\end{tabular}
% 	\end{center}
% 	Hãy tìm các số liệu bất thường trong mẫu số liệu trên.
% 	\loigiai{
% 		Sắp xếp các số liệu trong mẫu theo thứ tự không giảm ta có
% 		\begin{center}
% 			\begin{tabular}{|c|c|c|c|c|c|c|c|c|c|c|c|c|c|c|c|c|c|c|c|c|}
% 				\hline $1$ & $5$ & $6$ & $6$ & $6$ & $6$ & $7$ & $7$ & $7$ & $7$ & $7$ & $7$  & $8$ & $8$ & $8$ & $8$ & $9$ & $9$& $9$&  $10$& $10$ \\
% 				\hline
% 			\end{tabular}
% 		\end{center}
% 		Từ bảng số liệu ta tìm được số trung vị $Q_2=7$, tứ phân vị dưới $Q_1= 6$, tứ phân vị trên $Q_3= 8{,}5$ và khoảng tứ phân vị $\Delta_{Q} = 8{,}5 - 6 = 2{,}5$.\\
% 		Ta có $\left[Q_1-1{,}5 \cdot  \Delta_{Q}; Q_3+1{,}5 \cdot  \Delta_{Q} \right] =\left[2{,}5;12{,}25\right]$.\\
% 		Từ đó ta có $1$ là số liệu bất thường trong mẫu số liệu.		
% 	}
% \end{bt}

% \begin{bt}%[Phan Quốc Trí,  BG10-2022]%[0D5B3-5]
% 	Một cảnh sát giao thông ghi tốc độ (đơn vị: km/h) của $25$ chiếc xe qua trạm như sau:
% 	\begin{center}
% 		\begin{tabular}{ccccccccccccccc}
% 			20 & 41 & 41 & 80 & 40 & 52 & 52 & 52 & 60 & 55 & 60 & 60 & 62  \\  
% 			60 & 65 & 60 & 65 & 135 & 70 & 70 & 65 & 75 & 75 & 70 & 55 &  \\  
% 		\end{tabular} 
% 	\end{center}
% 	Hãy tìm các số liệu bất thường trong mẫu số liệu trên.
% 	\loigiai{
% 		Sắp xếp các số liệu trong mẫu theo thứ tự không giảm ta có
% 		\begin{center}
% 			\begin{tabular}{|c|c|c|c|c|c|c|c|c|c|c|c|c|}
% 				\hline 
% 				Tốc độ    & $20$ & $40$& $41$& $52$& $55$& $60$& $62$& $65$& $70$ & $75$& $80$& $135$\\
% 				\hline 
% 				Số lần xuất hiện  & $1$ & $1$ & $2 $ & $3$& $2$& $5$& $1$& $3$& $3$& $2$ & $1$& $1$\\
% 				\hline
% 			\end{tabular}
% 		\end{center} 
% 		Từ bảng số liệu ta tìm được số trung vị $Q_2=60$, tứ phân vị dưới $Q_1= 52$, tứ phân vị trên $Q_3= 70$ và khoảng tứ phân vị $\Delta_{Q} = 70 - 52 = 18$.\\
% 		Ta có $\left[Q_1-1{,}5 \cdot  \Delta_{Q}; Q_3+1{,}5 \cdot  \Delta_{Q} \right] =\left[25;97\right]$.\\
% 		Từ đó ta có $135$ là số liệu bất thường trong mẫu số liệu.		
% 	}
% \end{bt}


% \begin{bt}%[Phan Quốc Trí,  BG10-2022]%[0D5B3-5]
% 	Thống kê điểm thi môn Toán của $450$ học sinh trong một kì thi ở một trường trung học phổ thông. Người ta được bảng số liệu như sau
% 	\begin{center}
% 		\begin{tabular}{|c|c|c|c|c|c|c|c|c|c|c|}
% 			\hline
% 			Điểm             &	1   & 2  & 3  & 4  & 5   & 6  & 7  & 8  & 9  &10   \\  	\hline
% 			Số học sinh  &   1 & 1 & 1 & 1 & 120 & 200 & 119 & 5 & 1 & 1  \\  	\hline
% 		\end{tabular} 
% 	\end{center}
% 	Hãy tìm các số liệu bất thường trong mẫu số liệu trên.
% 	\loigiai{
% 		Từ bảng số liệu ta tìm được số trung vị $Q_2=6$, tứ phân vị dưới $Q_1= 5$, tứ phân vị trên $Q_3= 7$ và khoảng tứ phân vị $\Delta_{Q} = 7 - 5 = 2$.\\
% 		Ta có $\left[Q_1-1{,}5 \cdot  \Delta_{Q}; Q_3+1{,}5 \cdot  \Delta_{Q} \right] =\left[2;10\right]$.\\
% 		Từ đó ta có $1$ là số liệu bất thường trong mẫu số liệu.		
% 	}
% \end{bt}

\subsection{Bài tập trắc nghiệm}

\Opensolutionfile{ansbook}[ans/ansbook-2D1-2-TN]
\Opensolutionfile{ans}[ans/ans-2D1-2-TN]
%Mục 1
	\begin{ex}%[Khuyến Kỷ Luật,  BG10-2022]
	Hãy tìm khoảng biến thiên của mẫu số liệu thống kê sau:
	\begin{longtable}{cccccccccccc}
		22 & 26 & 31 & 15 & 12 & 4 & 18
		& 93 & 17 & 64 & 10
	\end{longtable}
	\choice
	{$33$}
	{$83$}
	{\True $89$}
	{$97$}
	\loigiai{
		Khoảng biến thiên của mẫu số liệu là $R=93-4=89$.
	}
\end{ex}

\begin{ex}%[Khuyến Kỷ Luật,  BG10-2022]
	Hai chữ số cuối giải đặc biệt Xổ số miền Bắc trong $9$ ngày được ghi lại như sau:
	\begin{longtable}{ccccccccc}
		16 & 11 & 25 & 28 & 45 & 42 & 24 & 33 & 11
	\end{longtable}
	Hãy tìm khoảng biến thiên của mẫu số liệu trên.
	\choice
	{$18$}
	{\True $34$}
	{$56$}
	{$27$}
	\loigiai{
		Khoảng biến thiên của mẫu số liệu là $R=45-11=34$.
	}
\end{ex}

\begin{ex}%[Khuyến Kỷ Luật,  BG10-2022]
	Mẫu số liệu nào dưới đây có khoảng biến thiên là $13$?
	\choice
	{$11$, $28$, $56$, $12$}
	{$6$, $12$, $33$, $23$, $11$}
	{$25$, $9$, $13$, $10$}
	{\True Tất cả đều sai}
	\loigiai{
		Khoảng biến thiên của các mẫu số liệu lần lượt là
		\begin{itemize}
			\item $R_1=56-11=45$.
			\item $R_2=33-6=26$.
			\item $R_3=25-9=14$.
		\end{itemize}
	}
\end{ex}

\begin{ex}%[Khuyến Kỷ Luật,  BG10-2022]
	Mẫu số liệu nào dưới đây có khoảng biến thiên là $53$?
	\choice
	{$18$, $57$, $11$, $26$}
	{\True $44$, $2$, $55$, $46$, $27$}
	{$21$, $3$, $55$, $89$}
	{$4$, $16$, $23$, $20$}
	\loigiai{
		Khoảng biến thiên của các mẫu số liệu lần lượt là
		\begin{itemize}
			\item $R_1=57-11=46$.
			\item $R_2=55-2=53$.
			\item $R_3=89-3=86$.
			\item $R_4=23-4=19$.
		\end{itemize}
	}
\end{ex}

\begin{ex}%[Khuyến Kỷ Luật,  BG10-2022]
	Số lượng học sinh có điểm Toán tổng kết cuối học kì I trên $ 8 $ ở mỗi lớp của một trường được tổng kết như trong bảng dưới đây
	\begin{longtable}{ccccccccc}
		16 & 11 & 15 & 18 & 21 & 12 & 24 & 23 & 11\\
		8 & 9 & 11 & 6 & 27 & 22 & 20 & 35 & 18
	\end{longtable}
	Hãy tìm khoảng biến thiên của mẫu số liệu trên
	\choice
	{$11$}
	{\True $29$}
	{$37$}
	{$25$}
	\loigiai{
		Khoảng biến thiên của mẫu số liệu là $R=35-6=29$.
	}
\end{ex}

\begin{ex}%[Khuyến Kỷ Luật,  BG10-2022]
	Hãy tìm khoảng biến thiên của mẫu số liệu thống kê được cho ở bảng sau
	\begin{longtable}{|l|c|c|c|c|c|c|c|}
		\hline
		Giá trị & 6 & 7 & 8 & 9 & 10\\
		\hline
		Tần số & 15 & 18 & 11 & 32 & 19\\
		\hline
	\end{longtable}
	\choice
	{\True $4$}
	{$5$}
	{$6$}
	{$7$}
	\loigiai{
		Khoảng biến thiên của mẫu số liệu là $R=10-6=4$.
	}
\end{ex}

\begin{ex}%[Khuyến Kỷ Luật,  BG10-2022]
	Sải cánh (tính theo đơn vị cm) của $90$ con chim sẻ được thống kê và ghi lại trong bảng dưới đây:
	\begin{longtable}{|l|c|c|c|c|c|c|c|}
		\hline
		Sải cánh & 18 & 19 & 20 & 21 & 22 & 23 & 24\\
		\hline
		Số lượng & 6 & 11 & 19 & 20 & 15 & 12 & 7\\
		\hline
	\end{longtable}
	\choice
	{$5$}
	{\True $6$}
	{$7$}
	{$8$}
	\loigiai{
		Khoảng biến thiên của các mẫu số liệu lần lượt là
		\begin{itemize}
			\item $R_1=17-3=14$.
			\item $R_2=43-21=22$.
			\item $R_3=24-11=13$.
			\item $R_4=33-13=20$.
		\end{itemize}
	}
\end{ex}

\begin{ex}%[Khuyến Kỷ Luật,  BG10-2022]
	Trong một tuần, nhiệt độ cao nhất trong ngày (đơn vị $^\circ$C) tại hai thành phố Hà Nội và TP Hồ Chí Minh được cho như sau:
	\begin{longtable}{p{3.5cm}cccccccc}
		Hà Nội: & 28 & 27 & 30 & 29 & 27 & 24 & 25\\
		TP Hồ Chí Minh: & 31 & 33 & 32 & 33 & 29 & 32 & 34\\
	\end{longtable}
	Dựa vào khoảng biến thiên của hai mẫu số liệu, hãy chỉ ra mẫu số liệu nào có độ phân tán lớn hơn.
	\choice
	{\True Mẫu số liệu ``Hà Nội'' có độ phân tán lớn hơn mẫu số liệu ``TP Hồ Chí Minh''}
	{Mẫu số liệu ``TP Hồ Chí Minh'' có độ phân tán lớn hơn mẫu số liệu ``Hà Nội''}
	{Hai mẫu số liệu có độ phân tán bằng nhau}
	{Tất cả đều sai}
	\loigiai{
		Mẫu số liệu ``Hà Nội'' có giá trị lớn nhất và giá trị nhỏ nhất lần lượt là $ 30 $ và $ 24 $.\\
		Do đó khoảng biến thiên của mẫu số liệu đã cho là $ R_1=30-24=6 $.\\
		Mẫu số liệu ``TP Hồ Chí Minh'' có giá trị lớn nhất và giá trị nhỏ nhất lần lượt là $ 33 $ và $ 29 $.\\
		Do đó khoảng biến thiên của mẫu số liệu đã cho là $ R_2=33-29=4 $.\\
		Do $R_1>R_2$ nên mẫu số liệu ``Hà Nội'' có độ phân tán lớn hơn mẫu số liệu ``TP Hồ Chí Minh''.	
	}
\end{ex}

\begin{ex}%[Khuyến Kỷ Luật,  BG10-2022]
	Tuổi và giới tính của những đứa trẻ trong một khu trung cư được cho bởi bảng sau
	\begin{longtable}{p{1.5cm}cccccccc}
		Nam: & 10 & 4 & 1 & 6 & 2 & 8 & 5\\
		Nữ: & 2 & 3 & 6 & 4 & 1 & 7 & \\
	\end{longtable}
	Dựa vào khoảng biến thiên của hai mẫu số liệu ``Nam'' và ``Nữ'', hãy chỉ ra mẫu số liệu nào có độ phân tán lớn hơn.
	\choice
	{\True Mẫu số liệu ``Nam'' có độ phân tán lớn hơn mẫu số liệu ``Nữ''}
	{Mẫu số liệu ``Nam'' có độ phân tán lớn hơn mẫu số liệu ``Nữ''}
	{Hai mẫu số liệu có độ phân tán bằng nhau}
	{Tất cả đều sai}
	\loigiai{
		Khoảng biến thiên của mẫu số liệu ``Nam'' là $ R_1=10-1=9 $.\\
		Khoảng biến thiên của mẫu số liệu ``Nữ'' là $ R_2=7-1=6 $.\\
		Do $R_1>R_2$ nên mẫu số liệu ``Nam'' có độ phân tán lớn hơn mẫu số liệu ``Nữ''.	
	}
\end{ex}

\begin{ex}%[Khuyến Kỷ Luật,  BG10-2022]
	Chỉ số IQ và EQ tương ứng của một nhóm học sinh được đo và ghi lại ở bảng sau
	\begin{longtable}{ccccccccc}
		IQ: & 95 & 110 & 90 & 105 & 88 & 100 & 111\\
		EQ: & 90 & 105 & 98 & 100 & 93 & 96 & 103\\
	\end{longtable}
	Dựa vào khoảng biến thiên của hai mẫu số liệu ``IQ'' và ``EQ'', hãy chỉ ra mẫu số liệu nào có độ phân tán lớn hơn.
	\choice
	{\True Mẫu số liệu ``IQ'' có độ phân tán lớn hơn mẫu số liệu ``EQ''}
	{Mẫu số liệu ``IQ'' có độ phân tán lớn hơn mẫu số liệu ``EQ''}
	{Hai mẫu số liệu có độ phân tán bằng nhau}
	{Tất cả đều sai}
	\loigiai{
		Khoảng biến thiên của mẫu số liệu ``IQ'' là $ R_1=111-88=43 $.\\
		Khoảng biến thiên của mẫu số liệu ``EQ'' là $ R_2=103-90=13 $.\\
		Do $R_1>R_2$ nên mẫu số liệu ``IQ'' có độ phân tán lớn hơn mẫu số liệu ``EQ''.
	}
\end{ex}

%Mục 2
\begin{ex}%[Bài giảng Toán 10]%[0D5Y14-2]%[]
	Độ lệch chuẩn là
	\choice
	{Bình phương của phương sai}
	{Một nửa của phương sai}
	{\True Căn bậc hai của phương sai}
	{Căn bậc ba của phương sai}
	\loigiai{	
		Căn bậc hai của phương sai được gọi là độ lệch chuẩn.
	}
\end{ex}

\begin{ex}%[Bài giảng Toán 10]%[0D5Y14-2]%[]
	Đại lượng đo mức độ biến động, chênh lệch giữa các giá trị trong mẫu số liệu thống kê gọi là
	\choice
	{Độ lệch chuẩn}
	{Số trung vị}
	{\True Phương sai}
	{Tần số}
	\loigiai{	
		Đại lượng đo mức độ biến động, chênh lệch giữa các giá trị trong mẫu số liệu thống kê gọi là phương sai.
	}
\end{ex}

\begin{ex}%[Bài giảng Toán 10]%[0D5B14-1]%[]
	Cho dãy số liệu thống kê: $1$, $2$, $3$, $4$, $5$, $6$, $7$, $8$. Độ lệch chuẩn của dãy số liệu thống kê gần bằng
	\choice
	{\True $2{,}30$}
	{$3{,}30$}
	{$4{,}30$}
	{$5{,}30$}
	\loigiai{	
		Ta có
		$$\overline{x}=\dfrac{1}{8}(1+2+3+4+5+6+7+8)=4{,}5.$$
		Phương sai 
		\begin{eqnarray*}
			s^{2}&=& \dfrac{1}{8}\left[(1-4{,}5)^{2}+ (2-4{,}5)^{2}+(3-4{,}5)^{2}+(4-4{,}5)^{2}+(5-4{,}5)^{2}+(6-4{,}5)^{2}+(7-4{,}5)^{2}+(8-4{,}5)^{2}\right] \\
			&= & 5{,}25.
		\end{eqnarray*}
		Độ lệch chuẩn $s=\sqrt{s^2}\approx 2{,}30$.
	}
\end{ex}

\begin{ex}%[Bài giảng Toán 10]%%[0D5B14-1]%[]
	Cho mẫu số liệu $\{10, 8, 6, 2, 4\}$. Độ lệch chuẩn của mẫu là
	\choice
	{\True $2{,}8$}
	{$8$}
	{$6$}
	{$2{,}4$}
	\loigiai{	
		Ta có
		$$\overline{x}=\dfrac{1}{5}(2+4+6+8+10)=6.$$
		Phương sai 
		$$s^{2}=\dfrac{1}{5}\left[(2-6)^{2}+ (4-6)^{2}+(6-6)^{2}+(8-6)^{2}+(10-6)^{2}\right]=8.$$
		Độ lệch chuẩn $s=\sqrt{s^2}\approx 2{,}8$.
	}
\end{ex}

\begin{ex}%[Bài giảng Toán 10]%[0D5B14-1]%[]
	Cho mẫu số liệu thống kê $\{2, 4, 6, 8, 10\}$. Phương sai của mẫu số liệu trên là bao nhiêu?
	\choice
	{$6$}
	{\True $8$}
	{$10$}
	{$40$}
	\loigiai{	
		Ta có
		$$\overline{x}=\dfrac{1}{5}(2+4+6+8+10)=6.$$
		Phương sai 
		$$s^{2}=\dfrac{1}{5}\left[(2-6)^{2}+ (4-6)^{2}+(6-6)^{2}+(8-6)^{2}+(10-6)^{2}\right]=8.$$
	}
\end{ex}

\begin{ex}%[Bài giảng Toán 10]%[0D5B14-1]%[]
	Số ôtô đi qua một cây cầu trong một tuần đếm được như sau: $83$; $74$; $71$; $79$; $83$; $69$; $92$. Phương sai và độ lệch chuẩn lần lượt là
	\choice
	{$78{,}71 - 8{,}87$}
	{$52{,}99 - 7{,}28$}
	{$61{,}82 - 7{,}86$}
	{\True $55{,}63 - 7{,}46$}
	\loigiai{	
		Ta có
		$$\overline{x}=\dfrac{1}{7}(69+71+74+79+83\cdot2+92)\approx 78{,}7.$$
		Phương sai
		\begin{eqnarray*}
			s^{2}&=& \dfrac{1}{7}\left[(69-78{,}7)^{2}+ (71-78{,}7)^{2}+(74-78{,}7)^{2}+(79-78{,}7)^{2}+2\cdot(83-78{,}7)^{2}+(92-78{,}7)^{2}\right]\\
			&\approx & 55{,}63.
		\end{eqnarray*}
		Độ lệch chuẩn $s=\sqrt{s^2}\approx 7{,}46$.
		
	}
\end{ex}

\begin{ex}%[Bài giảng Toán 10]%[0D5B14-1]%[]
	Tiền thưởng (triệu đồng) cho cán bộ và nhân viên trong công ty được trình bày trong bảng tần số sau đây
	\begin{center}
		\begin{tabular}{|c|c|c|c|c|c|c|}
			\hline
			Tiền thưởng $(x)$ & $2$ & $3$ & $4$ & $5$ & $6$ &      \\ \hline
			Tần số $(n)$    & $5$  & $15$  & $10$ & $6$ & $7$  & $n=43$ \\ \hline
		\end{tabular}
	\end{center}
	Phương sai là
	\choice
	{\True $1{,}59$}
	{$1{,}58$}
	{$1{,}61$}
	{$1{,}57$}
	\loigiai{	
		Ta có
		$$\overline{x}=\dfrac{1}{43}(5\cdot2+15\cdot3+10\cdot4+6\cdot5+7\cdot6)\approx 3{,}88.$$
		Phương sai 
		$$s^{2}=\dfrac{1}{43}\left[5(2-3{,}88)^{2}+15 (3-3{,}88)^{2}+10(4-3{,}88)^{2}+6(5-3{,}88)^{2}+7(6-3{,}88)^{2}\right]\approx1{,}59.$$	
	}
\end{ex}

\begin{ex}%[Bài giảng Toán 10]%[0D5B14-1]%[]
	Tiền thưởng (triệu đồng) cho cán bộ và nhân viên trong công ty được trình bày trong bảng tần số sau đây
	\begin{center}
		\begin{tabular}{|c|c|c|c|c|c|c|}
			\hline
			Tiền thưởng $(x)$ & $2$ & $3$ & $4$ & $5$ & $6$ &      \\ \hline
			Tần số $(n)$    & $5$  & $15$  & $10$ & $6$ & $7$  & $n=43$ \\ \hline
		\end{tabular}
	\end{center}
	Độ lệch chuẩn là
	\choice
	{\True $1{,}26$}
	{$1{,}27$}
	{$1{,}25$}
	{$1{,}24$}
	\loigiai{	
		Ta có
		$$\overline{x}=\dfrac{1}{43}(5\cdot2+15\cdot3+10\cdot4+6\cdot5+7\cdot6)\approx 3{,}88.$$
		Phương sai 
		$$s^{2}=\dfrac{1}{43}\left[5(2-3{,}88)^{2}+15 (3-3{,}88)^{2}+10(4-3{,}88)^{2}+6(5-3{,}88)^{2}+7(6-3{,}88)^{2}\right]\approx1{,}59.$$	
		Độ lệch chuẩn $s=\sqrt{s^2}\approx 1{,}26$.
	}
\end{ex}

% \begin{ex}%[Bài giảng Toán 10]%[0D5B14-1]%[]
% 	Điểm thi toán lớp $10$A được trình bày trong bảng tần số sau đây
% 	\begin{center}
% 		\begin{tabular}{|c|c|c|c|c|c|c|}
% 			\hline
% 			Điểm thi $(x)$ & $6$ & $7$ & $8$ & $9$ &  \\ \hline
% 			Tần số $(n)$    & $8$  & $18$  & $10$ & $4$ &  $n=40$ \\ \hline
% 		\end{tabular}
% 	\end{center}
% 	Độ lệch chuẩn là
% 	\choice
% 	{\True $0{,}89$}
% 	{$0{,}88$}
% 	{$0{,}87$}
% 	{$0{,}86$}
% 	\loigiai{	
% 		Ta có
% 		$$\overline{x}=\dfrac{1}{40}(8\cdot6+18\cdot7+10\cdot8+4\cdot9)= 7{,}25.$$
% 		Phương sai 
% 		$$s^{2}=\dfrac{1}{40}\left[8(6-7{,}25)^{2}+18 (7-7{,}25)^{2}+10(8-7{,}25)^{2}+4(9-7{,}25)^{2}\right]\approx0{,}79.$$	
% 		Độ lệch chuẩn $s=\sqrt{s^2}\approx 0{,}89$.
% 	}
% \end{ex}

% \begin{ex}%[Bài giảng Toán 10]%[0D5B14-1]%[]
% 	Cho dãy số liệu thống kê $38$; $18$; $20$; $25$; $18$; $15$; $20$; $22$; $31$.\\
% 	Phương sai của dãy số liệu trên là
% 	\choice
% 	{\True $47,3$}
% 	{$50$}
% 	{$42$}
% 	{$43$}
% 	\loigiai{	
% 		Ta có
% 		$$\overline{x}=\dfrac{1}{9}(15+2\cdot18+2\cdot20+22+25+31+38)= 23.$$
% 		Phương sai 
% 		$$s^{2}=\dfrac{1}{9}\left[(15-23)^{2}+2 (18-23)^{2}+2(20-23)^{2}+(22-23)^{2}+(25-23)^{2}+(31-23)^{2}+(38-23)^{2}\right]\approx47{,}3.$$	
% 	}
% \end{ex}

% \begin{ex}%[Bài giảng Toán 10]%[0D5B14-1]%[]
% 	Trên con đường A, trạm kiểm soát đã ghi lại tốc độ của $30$ chiếc ô tô được trình bày trong bảng tần số sau đây
% 	\begin{center}
% 		\begin{tabular}{|l|c|c|c|c|c|c|c|c|c|c|c|c|c|c|c|c|c|}
% 			\hline
% 			Vận tốc & 60 & 62 & 63 & 65 & 68 & 69 & 70 & 73 & 75 & 76 & 80 & 82 & 83 & 84 & 85 & 88 & 90 \\ \hline
% 			Tần số  & 2  & 2  & 1  & 2  & 3  & 1  & 2  & 2  & 3  & 2  & 3  & 1  & 1  & 2  & 1  & 1  & 1  \\ \hline
% 		\end{tabular}
% 	\end{center}
% 	Phương sai của tốc độ ô tô trên con đường A là
% 	\choice
% 	{$74,77$}
% 	{$75,36$}
% 	{$73,63$}
% 	{\True $72,1$}
% 	\loigiai{	
% 		Ta có $\displaystyle\sum\limits_{i=1}^{17} n_{i} x_{i}=2209$ và $\displaystyle\sum\limits_{i=1}^{17} n_{i} x_{i}^{2}=164819 \Rightarrow$ phương sai là
% 		$$s^{2}=\dfrac{1}{n} \sum_{i=1}^{n} n_{i} x_{i}^{2}-\dfrac{1}{n^{2}}\left(\displaystyle\sum\limits_{i=1}^{n} n_{i} x_{i}\right)^{2}=\dfrac{1}{30}\cdot164819-\dfrac{1}{30^{2}}\cdot(2209)^{2}\approx72{,}1.$$
% 	}
% \end{ex}

% \begin{ex}%[Bài giảng Toán 10]%[0D5B14-1]%[]
% 	Trên con đường A, trạm kiểm soát đã ghi lại tốc độ của $30$ chiếc ô tô được trình bày trong bảng tần số sau đây
% 	\begin{center}
% 		\begin{tabular}{|l|c|c|c|c|c|c|c|c|c|c|c|c|c|c|c|c|c|}
% 			\hline
% 			Vận tốc & 60 & 62 & 63 & 65 & 68 & 69 & 70 & 73 & 75 & 76 & 80 & 82 & 83 & 84 & 85 & 88 & 90 \\ \hline
% 			Tần số  & 2  & 2  & 1  & 2  & 3  & 1  & 2  & 2  & 3  & 2  & 3  & 1  & 1  & 2  & 1  & 1  & 1  \\ \hline
% 		\end{tabular}
% 	\end{center}	
% 	\choice
% 	{$8,68$}
% 	{$8,65$}
% 	{$8,58$}
% 	{\True $8,49$}
% 	\loigiai{	
% 		Ta có $\displaystyle\sum\limits_{i=1}^{17} n_{i} x_{i}=2209$ và $\displaystyle\sum\limits_{i=1}^{17} n_{i} x_{i}^{2}=164819 \Rightarrow$ phương sai là
% 		$$s^{2}=\dfrac{1}{n} \sum_{i=1}^{n} n_{i} x_{i}^{2}-\dfrac{1}{n^{2}}\left(\displaystyle\sum\limits_{i=1}^{n} n_{i} x_{i}\right)^{2}=\dfrac{1}{30}\cdot164819-\dfrac{1}{30^{2}}\cdot(2209)^{2}\approx72{,}1.$$
% 		Độ lệch chuẩn là $s=\sqrt{s^{2}}\approx\sqrt{72{,}1}\approx8{,}49$.
% 	}
% \end{ex}

% \begin{ex}%[Bài giảng Toán 10]%[0D5B14-1]%[]
% 	Số lượng khách đến tham quan một điểm du lịch trong $12$ tháng như sau
% 	\begin{center}
% 		\begin{tabular}{|l|c|c|c|c|c|c|c|c|c|c|c|c|}
% 			\hline
% 			Tháng    & 1   & 2   & 3   & 4   & 5   & 6   & 7   & 8   & 9   & 10  & 11  & 12  \\ \hline
% 			Số khách & 430 & 550 & 430 & 520 & 550 & 515 & 550 & 110 & 520 & 430 & 550 & 880 \\ \hline
% 		\end{tabular}
% 	\end{center}
% 	Độ lệch chuẩn là
% 	\choice
% 	{$567,56$}
% 	{\True $163,84$}
% 	{$171,13$}
% 	{$147,30$}
% 	\loigiai{	
% 		Ta có
% 		$$\overline{x}=\dfrac{1}{12}(3\cdot430+4\cdot550+2\cdot520+515+110+880)\approx  503.$$
% 		Phương sai
% 		\begin{eqnarray*}
% 			s^{2}&=& \dfrac{1}{12}\left[3(430-503)^{2}+4 (550-503)^{2}+2(520-503)^{2}+(515-503)^{2}+(110-503)^{2}+(880-503)^{2}\right]\\
% 			&\approx  & 26843{,}58.
% 		\end{eqnarray*}	
% 		Độ lệch chuẩn $s=\sqrt{s^2}\approx 163{,}84$.
% 	}
% \end{ex}

% %Mục 3
% \begin{ex}%[Phan Quốc Trí,  BG10-2022]%[0D5B3-5]
% 	Một mẫu số liệu thống kê có các tứ phân vị lần lượt là $Q_1=22$, $Q_2= 27$, $Q_3= 32$. Giá trị nào sau đây là giá trị bất thường của mẫu số liệu?
% 	\choice
% 	{$30$}
% 	{$8$}
% 	{\True $6$}
% 	{$46$} 
% 	\loigiai{
% 		Ta có $\Delta_{Q} = Q_3 - Q_1 = 32-22=10$. Do đó 	$\left[Q_1-1{,}5 \cdot  \Delta_{Q}; Q_3+1{,}5 \cdot  \Delta_{Q} \right] =\left[7;47\right]$.\\
% 		Do $6 \notin \left[7;47\right]$ nên là một giá trị bất thường của mẫu số liệu.
% 	}
% \end{ex}


% \begin{ex}%[Phan Quốc Trí,  BG10-2022]%[0D5B3-5]
% 	Hãy tìm các giá trị bất thường của mẫu số liệu thống kê sau
% 	\begin{center}	
% 		\begin{tabular}{cccccccc}
% 			7 & 19 & 6 & 12 & 5 & 17 & 6 & 13  
% 		\end{tabular} 
% 	\end{center}
% 	\choice
% 	{$5$; $6$}
% 	{$5$; $6$; $19$}
% 	{\True Không có số liệu bất thường}
% 	{$5$; $19$} 
% 	\loigiai{
% 		Sắp xếp các số liệu trong mẫu theo thứ tự không giảm ta có
% 		\begin{center}
% 			\begin{tabular}{cccccccc}
% 				5 & 6 & 6 & 7 & 12 & 13 & 17 & 19  
% 			\end{tabular} 
% 		\end{center} 
% 		Từ bảng số liệu ta tìm được số trung vị $Q_2=\dfrac{7+12}{2}=9{,}5$, tứ phân vị dưới $Q_1= 6$, tứ phân vị trên $Q_3= 15$ và khoảng tứ phân vị $\Delta_{Q} = 15 - 6 = 9$.\\
% 		Ta có $\left[Q_1-1{,}5 \cdot  \Delta_{Q}; Q_3+1{,}5 \cdot  \Delta_{Q} \right] =\left[-7{,}5;28{,}5\right]$.\\
% 		Từ đó ta có mẫu số liệu trên không có số liệu bất thường.			
% 	}
% \end{ex}

% \begin{ex}%[Phan Quốc Trí,  BG10-2022]%[0D5B3-5]
% 	Hãy tìm các giá trị bất thường của mẫu số liệu thống kê sau
% 	\begin{center}	
% 		\begin{tabular}{cccccccccc}
% 			$20$ & $52$ & $86$ & $80$ & $44$ & $49$ & $57$ & $41$& $44$ & $55$ 
% 		\end{tabular} 
% 	\end{center}
% 	\choice
% 	{$80$; $86$}
% 	{$41$; $80$; $86$}
% 	{\True $80$; $20$; $86$}
% 	{$86$} 
% 	\loigiai{
% 		Sắp xếp các số liệu trong mẫu theo thứ tự không giảm ta có
% 		\begin{center}	
% 			\begin{tabular}{cccccccccc}
% 				$20$ & $41$ & $44$ & $44$ & $49$ & $52$ & $55$ & $57$& $80$ & $86$ 
% 			\end{tabular} 
% 		\end{center}
% 		Từ bảng số liệu ta tìm được số trung vị $Q_2=\dfrac{49+52}{2}=50{,}5$, 
% 		tứ 
% 		phân vị dưới $Q_1= 44$, tứ phân vị trên $Q_3= 57$ và khoảng tứ phân vị 
% 		$\Delta_{Q} = 57 - 44 = 13$.\\
% 		Ta có $\left[Q_1-1{,}5 \cdot  \Delta_{Q}; Q_3+1{,}5 \cdot  \Delta_{Q} 
% 		\right] =\left[24{,}5;76{,}5\right]$.\\
% 		Từ đó ta có $80$; $20$ và $86$ là các số liệu bất thường.			
% 	}
% \end{ex}
\begin{ex}%[Phan Quốc Trí,  BG10-2022]%[0D5B3-5]
	Một mẫu số liệu thống kê có các tứ phân vị lần lượt là $Q_1=53$, $Q_2= 55$, 
	$Q_3= 61$. Giá trị nào sau đây \textbf{không} phải là giá trị bất thường 
	của mẫu số liệu?
	\choice
	{$40$}
	{$80$}
	{\True $73$}
	{$73{,}5$} 
	\loigiai{
		Ta có $\Delta_{Q} = Q_3 - Q_1 = 8$. Do đó 	$\left[Q_1-1{,}5 
		\cdot  \Delta_{Q}; Q_3+1{,}5 \cdot  \Delta_{Q} \right] 
		=\left[41;73\right]$.\\
		Do $73 \in \left[41;73\right]$ nên không phải là một giá trị bất thường 
		của mẫu số liệu.
	}
\end{ex}

\begin{ex}%[Phan Quốc Trí,  BG10-2022]%[0D5B3-5]
	Một mẫu số liệu thống kê có các tứ phân vị lần lượt là $Q_1=3$, $Q_2= 7$, 
	$Q_3= 12$. Giá trị nào sau đây  là giá trị bất thường 
	của mẫu số liệu?
	\choice
	{$22$}
	{$-8{,}5$}
	{\True $26$}
	{$25{,}5$} 
	\loigiai{
		Ta có $\Delta_{Q} = Q_3 - Q_1 = 9$. Do đó 	$\left[Q_1-1{,}5 
		\cdot  \Delta_{Q}; Q_3+1{,}5 \cdot  \Delta_{Q} \right] 
		=\left[-10{,}5;25{,}5\right]$.\\
		Do $26 \notin \left[-10{,}5;25{,}5\right]$ nên  là một giá 
		trị bất thường của mẫu số liệu.
	}
\end{ex}

% \begin{ex}%[Phan Quốc Trí,  BG10-2022]%[0D5B3-5]
% 	Hãy tìm các giá trị bất thường của mẫu số liệu thống kê sau
% 	\begin{center}	
% 		\begin{tabular}{cccccccccccc}
% 			$10$ & $59$ & $67$ & $72$ & $73$ & $76$  & $88$ & $92$ & $106$& $111$ & $115$ & $169$ 
% 		\end{tabular} 
% 	\end{center}
% 	\choice
% 	{$169$}
% 	{$115$; $169$}
% 	{$111$; $169$}
% 	{\True $10$; $169$} 
% 	\loigiai{
% 		Từ bảng số liệu ta tìm được số trung vị $Q_2=\dfrac{76+88}{2}=82$, 
% 		tứ 
% 		phân vị dưới $Q_1= 69{,}5$, tứ phân vị trên $Q_3= 103{,}5$ và khoảng tứ phân vị 
% 		$\Delta_{Q} = 103{,}5 - 69{,}5 = 34$.\\
% 		Ta có $\left[Q_1-1{,}5 \cdot  \Delta_{Q}; Q_3+1{,}5 \cdot  \Delta_{Q} 
% 		\right] =\left[18{,}5;154{,}5\right]$.\\
% 		Từ đó ta có $10$ và $169$ là các số liệu bất thường.			
% 	}
% \end{ex}

% \begin{ex}%[Phan Quốc Trí,  BG10-2022]%[0D5B3-5]
% 	Cho mẫu số liệu thống kê sau 
% 	\begin{center}	
% 		\begin{tabular}{cccccccccc}
% 			$-3$ & $5$ & $10$ & $12$ & $14$ & $18$  & $24$ & $26$ & $49$& $60$  
% 		\end{tabular} 
% 	\end{center}
% 	Phát biểu nào sau đây là đúng?
% 	\choice
% 	{$-3$ là giá trị bất thường duy nhất}
% 	{\True $60$ là giá trị bất thường duy nhất}
% 	{Không có giá trị bất thường trong mẫu số liệu}
% 	{Mẫu số liệu có nhiều giá trị bất thường} 
% 	\loigiai{
% 		Từ bảng số liệu ta tìm được số trung vị $Q_2=\dfrac{14+18}{2}=16$, 
% 		tứ 
% 		phân vị dưới $Q_1= 10$, tứ phân vị trên $Q_3= 26$ và khoảng tứ phân vị 
% 		$\Delta_{Q} = 26-10= 16$.\\
% 		Ta có $\left[Q_1-1{,}5 \cdot  \Delta_{Q}; Q_3+1{,}5 \cdot  \Delta_{Q} 
% 		\right] =\left[-14;50\right]$.\\
% 		Từ đó ta có $60$  là các số liệu bất thường duy nhất của mẫu số liệu.			
% 	}
% \end{ex}

% \begin{ex}%[Phan Quốc Trí,  BG10-2022]%[0D5B3-5]
% 	Cho mẫu số liệu thống kê sau 
% 	\begin{center}	
% 		\begin{tabular}{cccccccc}
% 			$10$ & $21$ & $21$ & $23$ & $25$ & $26$  & $28$ & $42$   
% 		\end{tabular} 
% 	\end{center}
% 	Phát biểu nào sau đây là đúng?
% 	\choice
% 	{$10$ là giá trị bất thường duy nhất}
% 	{$42$ là giá trị bất thường duy nhất}
% 	{Không có giá trị bất thường trong mẫu số liệu}
% 	{\True Mẫu số liệu có nhiều giá trị bất thường} 
% 	\loigiai{
% 		Từ bảng số liệu ta tìm được số trung vị $Q_2=\dfrac{23+25}{2}=24$, 
% 		tứ 
% 		phân vị dưới $Q_1= 21$, tứ phân vị trên $Q_3= 27$ và khoảng tứ phân vị 
% 		$\Delta_{Q} = 27-21= 6$.\\
% 		Ta có $\left[Q_1-1{,}5 \cdot  \Delta_{Q}; Q_3+1{,}5 \cdot  \Delta_{Q} 
% 		\right] =\left[12;36\right]$.\\
% 		Từ đó ta có $10$; $42$  là các số liệu bất thường của mẫu số liệu.			
% 	}
% \end{ex}

% \begin{ex}%[Phan Quốc Trí,  BG10-2022]%[0D5B3-5]
% 	Cho mẫu số liệu thống kê sau 
% 	\begin{center}	
% 		\begin{tabular}{ccccccc}
% 			$52$ & $47$ & $55$ & $81$ & $61$ & $49$  & $59$    
% 		\end{tabular} 
% 	\end{center}
% 	Phát biểu nào sau đây là đúng?
% 	\choice
% 	{\True $81$ là giá trị bất thường duy nhất}
% 	{$47$ là giá trị bất thường duy nhất}
% 	{Không có giá trị bất thường trong mẫu số liệu}
% 	{Mẫu số liệu có nhiều giá trị bất thường} 
% 	\loigiai{
% 		Sắp xếp bảng số liệu theo thứ tự không giảm ta có
% 		\begin{center}	
% 			\begin{tabular}{ccccccc}
% 				$47$ & $49$ & $52$ & $55$ & $59$ & $61$  & $81$    
% 			\end{tabular} 
% 		\end{center}
% 		Từ bảng số liệu ta tìm được số trung vị $Q_2=55$, 
% 		tứ 
% 		phân vị dưới $Q_1= 49$, tứ phân vị trên $Q_3= 61$ và khoảng tứ phân vị 
% 		$\Delta_{Q} = 49-61= 12$.\\
% 		Ta có $\left[Q_1-1{,}5 \cdot  \Delta_{Q}; Q_3+1{,}5 \cdot  \Delta_{Q} 
% 		\right] =\left[31;79\right]$.\\
% 		Từ đó ta có $81$  là  số liệu bất thường của mẫu số liệu.			
% 	}
% \end{ex}

% \begin{ex}%[Phan Quốc Trí,  BG10-2022]%[0D5B3-5]
% 	Cho mẫu số liệu thống kê sau 
% 	\begin{center}	
% 		\begin{tabular}{ccccccc}
% 			$8$ & $10$ & $13$ & $13$ & $14$ & $16$  & $27$    
% 		\end{tabular} 
% 	\end{center}
% 	Phát biểu nào sau đây là đúng?
% 	\choice
% 	{$8$ là giá trị bất thường duy nhất}
% 	{\True $27$ là giá trị bất thường duy nhất}
% 	{Không có giá trị bất thường trong mẫu số liệu}
% 	{Mẫu số liệu có nhiều giá trị bất thường} 
% 	\loigiai{
% 		Từ bảng số liệu ta tìm được số trung vị $Q_2=13$, 
% 		tứ 
% 		phân vị dưới $Q_1= 10$, tứ phân vị trên $Q_3= 16$ và khoảng tứ phân vị 
% 		$\Delta_{Q} = 16-10= 6$.\\
% 		Ta có $\left[Q_1-1{,}5 \cdot  \Delta_{Q}; Q_3+1{,}5 \cdot  \Delta_{Q} 
% 		\right] =\left[1;25\right]$.\\
% 		Từ đó ta có $27$  là số liệu bất thường của mẫu số liệu.			
% 	}
% \end{ex}

% \begin{ex}%[Phan Quốc Trí,  BG10-2022]%[0D5B3-5]
% 	Cho mẫu số liệu thống kê sau 
% 	\begin{center}	
% 		\begin{tabular}{cccccccc}
% 			$44$ & $51$ & $36$ & $19$ & $40$ & $69$  & $49$& $46$    
% 		\end{tabular} 
% 	\end{center}
% 	Phát biểu nào sau đây là đúng?
% 	\choice
% 	{$19$ là giá trị bất thường duy nhất}
% 	{$69$ là giá trị bất thường duy nhất}
% 	{Không có giá trị bất thường trong mẫu số liệu}
% 	{\True Mẫu số liệu có nhiều giá trị bất thường} 
% 	\loigiai{
% 		Sắp xếp bảng số liệu theo thứ tự không giảm ta có
% 		\begin{center}	
% 			\begin{tabular}{cccccccc}
% 				$19$ & $36$ & $40$ & $44$ & $46$ & $49$  & $51$& $69$    
% 			\end{tabular} 
% 		\end{center}	
% 		Từ bảng số liệu ta tìm được số trung vị $Q_2=\dfrac{44+46}{2}=45$, 
% 		tứ 
% 		phân vị dưới $Q_1= 38$, tứ phân vị trên $Q_3= 50$ và khoảng tứ phân vị 
% 		$\Delta_{Q} = 50-38= 12$.\\
% 		Ta có $\left[Q_1-1{,}5 \cdot  \Delta_{Q}; Q_3+1{,}5 \cdot  \Delta_{Q} 
% 		\right] =\left[20;68\right]$.\\
% 		Do đó không có giá trị  bất thường trong mẫu số liệu.			
% 	}
% \end{ex}

% \begin{ex}%[Phan Quốc Trí,  BG10-2022]%[0D5B3-5]
% 	Cho mẫu số liệu thống kê sau 
% 	\begin{center}	
% 		\begin{tabular}{cccccccc}
% 			$20$ & $22$ & $22$ & $25$ & $28$ & $32$  & $34$& $43$    
% 		\end{tabular} 
% 	\end{center}
% 	Phát biểu nào sau đây là đúng?
% 	\choice
% 	{$20$ là giá trị bất thường duy nhất}
% 	{$43$ là giá trị bất thường duy nhất}
% 	{\True Không có giá trị bất thường trong mẫu số liệu}
% 	{Mẫu số liệu có nhiều giá trị bất thường} 
% 	\loigiai{
% 		Từ bảng số liệu ta tìm được số trung vị $Q_2=\dfrac{25+28}{2}=26{,}5$, 
% 		tứ 
% 		phân vị dưới $Q_1= 22$, tứ phân vị trên $Q_3= 33$ và khoảng tứ phân vị 
% 		$\Delta_{Q} = 33-22= 11$.\\
% 		Ta có $\left[Q_1-1{,}5 \cdot  \Delta_{Q}; Q_3+1{,}5 \cdot  \Delta_{Q} 
% 		\right] =\left[5{,}5;49{,}5\right]$.\\
% 		Từ đó ta có $19$, $69$  là số liệu bất thường của mẫu số liệu.			
% 	}
% \end{ex}


% \subsection*{Bài tập tự luận (bổ sung)}
% \begin{bt}%[Ngọc Thy Tô]%[Bài giảng Toán 10]%[0D5B14-1]
% 	Cho mẫu số liệu gồm $15$ số dương không hoàn toàn giống nhau. Các số đo độ phân tán (khoảng biến thiên, khoảng tứ phân vị, độ lệch chuẩn) sẽ thay đổi như thế nào nếu
% 	\begin{enumerate}
% 		\item Nhân mỗi giá trị của mẫu số liệu với $3$.
% 		\item Cộng mỗi giá trị của mẫu số liệu với $3$.
% 	\end{enumerate}
% 	\loigiai{
% 		Giả sử $15$ số liệu được sắp xếp theo thứ tự không giảm là $x_1$; $x_2$; $\ldots$; $x_{15}$.
% 		\begin{enumerate}
% 			\item Nhân mỗi giá trị của mẫu số liệu với $3$. Ta có
% 			\begin{itemize}
% 				\item Khoảng biến thiên $R=3x_{15}-3x_1=3\left( x_{15}-x_1\right)$.
% 				\item Ta có $Q_2=3x_8$; $Q_1=3x_4$; $Q_3=3x_{12}$. Khoảng tứ phân vị: $\Delta_Q=Q_3-Q_1=3x_{12}-3x_8=3\left(x_{12}-x_8\right)$.
% 				\item Độ lệch chuẩn $s=\sqrt{\dfrac{1}{15} \sum\limits_{i=1}^{15} \left(3x_i-3\overline{x}\right)^2}=3\sqrt{\dfrac{1}{15} \sum\limits_{i=1}^{15} \left(x_i-\overline{x}\right)^2}$.
% 			\end{itemize}
% 			Vậy khi nhân mỗi giá trị của mẫu số liệu với $3$ thì các số đo độ phân tán (khoảng biến thiên, khoảng tứ phân vị, độ lệch chuẩn) sẽ tăng lên $3$ lần.
% 			\item Cộng mỗi giá trị của mẫu số liệu với $3$.
% 			\begin{itemize}
% 				\item Khoảng biến thiên $R=x_{15}+3-\left(x_1+3\right) =x_{15}-x_1$.
% 				\item Ta có $Q_2=x_8+3$; $Q_1=x_4+3$; $Q_3=x_{12}+3$.\\ Khoảng tứ phân vị: $\Delta_Q=Q_3-Q_1=x_{12}+3-\left(x_8+3\right)=x_{12}-x_8$.
% 				\item Độ lệch chuẩn $s=\sqrt{\dfrac{1}{15} \sum\limits_{i=1}^{15} \left(x_i+3-\left(\overline{x}+3\right)\right)^2}=\sqrt{\dfrac{1}{15} \sum\limits_{i=1}^{15} \left(x_i-\overline{x}\right)^2}$.
% 			\end{itemize}
% 		\end{enumerate}
% 		Vậy khi cộng mỗi giá trị của mẫu số liệu với $3$ thì các số đo độ phân tán (khoảng biến thiên, khoảng tứ phân vị, độ lệch chuẩn) sẽ không thay đổi.
% 	}
% \end{bt}
% \begin{bt}%[Ngọc Thy Tô]%[Bài giảng Toán 10]%[0D5Y14-1]
% 	Sản lượng lúa (đơn vị là tạ) của $40$ thửa ruộng thí nghiệm có cùng diện tích được trình bày trong bảng tần số dưới đây:
% 	\begin{center}
% 		\begin{tabular}{|c|c|c|c|c|c|}
% 			\hline 
% 			Sản lượng & $20$ & $21$ & $22$ & $23$ & $24$   \\ 
% 			\hline 
% 			Số thửa ruộng & $5$ & $8$ & $11$ & $10$ & $6$  \\ 
% 			\hline 
% 		\end{tabular} 
% 	\end{center}
% 	a) Tính sản lượng trung bình của $40$ thửa ruộng?\\
% 	b) Tính phương sai và độ lệch chuẩn.
% 	\loigiai{
% 		a) Số trung bình của sản lượng của $40$ thửa ruộng là
% 		$$\overline{x}=\dfrac{5\cdot 20+8\cdot 21 +11\cdot 22+10\cdot 23+6\cdot 24}{40}=22{,}1 \text{ tạ }.$$
% 		b) Tính phương sai\\
% 		Ta có
% 		$$s^2=\dfrac{1}{40}\left[ 5(20-22{,}1)^2+8(21-22{,}1)^2+11(22-22{,}1)^2+10(23-22{,}1)^2+6(24-22{,}1)^2\right]=\dfrac{6160}{4000}.$$
% 		Hay $s^2=1{,}54$.\\
% 		Tính độ lệch chuẩn $s=\sqrt{s^2}=\sqrt{1{,}54}\approx 1{,}24.$
% 	}
% \end{bt}

% \begin{bt}%[Ngọc Thy Tô]%[Bài giảng Toán 10]%[0D5Y14-1]
% 	Số máy tính bán được trong $7$ tháng liên tiếp của một cửa hàng được ghi lại trong bảng sau:
% 	\begin{center}
% 		\begin{tabular}{|c|c|c|c|c|c|c|}
% 			\hline 
% 			$83$ & $79$ & $92$ & $71$ & $69$ & $83$ & $74$ \\ 
% 			\hline 
% 		\end{tabular} 
% 	\end{center}
% 	\begin{enumerate}
% 		\item Tính khoảng biến thiên, khoảng tứ phân vị của mẫu số liệu.
% 		\item Tính số trung bình, phương sai và độ lệch chuẩn.
% 	\end{enumerate}
% 	\loigiai{
% 		\begin{enumerate}
% 			\item Các số liệu được sắp xếp lại theo thứ tự không giảm: $69$ $71$ $74$ $79$ $83$ $83$ $92$.\\
% 			Khoảng biến thiên: $R=92-69=23$.\\
% 			Khoảng tứ phân vị $\Delta_Q=Q_3-Q_1=83-71=12$.
% 			\item Số trung bình là $\overline{x}=\dfrac{83+79+92+71+69+83+74}{7}\approx 78{,}71$.\\
% 			Ta có 
% 			\begin{eqnarray*}
% 				&s^2&=\dfrac{(69-87{,}71)^2+(71-87{,}71)^2+(74-87{,}71)^2+(79-87{,}71)^2+2\cdot (83-87{,}71)^2+(92-87{,}71)^2}{7}\\
% 				&&=55{,}63.
% 			\end{eqnarray*}
% 			Vậy $s=\sqrt{55{,}63}\approx 7{,}46$.
% 		\end{enumerate}	
		
% 	}
% \end{bt}

% \begin{bt}%[Ngọc Thy Tô]%[Bài giảng Toán 10]%[0D5Y14-1]
% 	Kết quả thi kết thúc học kì một của bạn Hoa được ghi lại trong bảng sau:
% 	\begin{center}
% 		\begin{tabular}{|c|c|c|c|c|c|}
% 			\hline 
% 			Văn & Địa & Lý & Hóa & Toán & Anh văn \\ 
% 			\hline 
% 			$6,0$ & $8,0$ & $7,5$ & $8,5$ & $7,0$ & $7,5$ \\ 
% 			\hline 
% 		\end{tabular} 
% 	\end{center}
% 	Tìm số trung bình, phương sai và độ lệch chuẩn.
% 	\loigiai{
% 		\begin{itemize}
% 			\item Số trung bình
% 			$$\overline{x}=\dfrac{6+7+7{,}5\cdot 2+8+8{,}5}{6}\approx 7{,}42.$$
% 			\item Phương sai
% 			$$s^2=\dfrac{(6-7{,}42)^2+(7-7{,}42)^2+2\cdot (7{,}5-7{,}42)^2+(8-7{,}42)^2+(8{,}5-7{,}42)^2+}{6}\approx 0{,}62.$$
% 			\item Độ lệch chuẩn $s=\sqrt{s^2}=\sqrt{0{,}62}\approx 0{,}79$.
% 		\end{itemize}
% 	}
% \end{bt}

% \begin{bt}%[Ngọc Thy Tô]%[Bài giảng Toán 10]%[0D5Y14-1]
% 	Trong sổ theo dõi bán hàng ở một cửa hàng bán xe máy có bảng sau:
% 	\begin{center}
% 		\begin{tabular}{|c|c|c|c|c|c|c|}
% 			\hline 
% 			Số ngày & $0$ & $1$ & $2$ & $3$ & $4$ & $5$ \\ 
% 			\hline 
% 			Số xe bán  & $2$ & $13$ & $15$ & $12$ & $7$ & $3$\\ 
% 			\hline 
% 		\end{tabular} 
% 	\end{center}
% 	\begin{enumerate}
% 		\item Tính khoảng biến thiên, khoảng tứ phân vị của mẫu số liệu.
% 		\item Tính số trung bình, phương sai và độ lệch chuẩn.
% 	\end{enumerate}
% 	\loigiai{
% 		\begin{enumerate}
% 			\item Khoảng biến thiên $R=15-2=13$.\\
% 			Ta có $Q_1=3$; $Q_3=13$. Khoảng tứ phân vị $\Delta_Q=Q_3-Q_1=13-3=10$.
% 			\item Số trung bình $\overline{x}=\dfrac{3\cdot 5+7\cdot 4+12\cdot 3+13+15\cdot 2}{15}\approx 8{,}13$.\\
% 			Phương sai $s^2=\dfrac{5\cdot (3-8{,}13)^2+4\cdot (7-8{,}13)^2+3\cdot (12-8{,}13)^2+(13-8{,}13)^2+2\cdot (15-8{,}13)^2}{15}\approx 19{,}98$.\\
% 			Độ lệch chuẩn $s=\sqrt{19{,}98}\approx 4{,}47$.			
% 		\end{enumerate}
% 	}
% \end{bt}

% \begin{bt}%[Ngọc Thy Tô]%[Bài giảng Toán 10]%[0D5B14-1]
% 	Bảng sau đây ghi lại tốc độ (km/h) của $20$ chiếc ôtô.
% 	\begin{center}
% 		\begin{tabular}{|c|c|c|c|c|c|c|c|c|c|}
% 			\hline 
% 			$40$ & $65$ & $70$ & $68$ & $62$ & $75$ & $80$ & $83$& $82 $& $69$ \\ 
% 			\hline 
% 			$73$ & $75$ & $85$ & $72$ & $67$ & $88$ & $110$ & $85$ & $72$ & $63$ \\ 
% 			\hline 
% 		\end{tabular} 
% 	\end{center}
% 	Hãy tìm các giá trị bất thường (nếu có) của mẫu số liệu trên.
% 	\loigiai{
% 		Các số liệu được sắp xếp lại theo thứ tự không giảm
% 		\begin{center}
% 			\begin{tabular}{cccccccccccccccccccc}
% 				$40$& $62$& $63$& $65$& $67$& $68$& $69$& $70$& $72$& $72$& $73$& $75$& $75$& $80$& $82$& $83$& $85$& $85$& $88$& $110$
% 			\end{tabular} 
% 		\end{center}
% 		Từ mẫu số liệu ta tính được $Q_1=67,5$; $Q_3=82,5$.\\
% 		Do đó khoảng tứ phân vị là: $\Delta_Q=82,5-67,5=15$.\\
% 		Ta có $Q_1-1,5\cdot\Delta_Q=45$; $Q_3+1,5\cdot\Delta_Q=105$.\\
% 		vậy trong mẫu số liệu có hai giá trị được xem là bất thường là $40$ km/h (nhỏ hơn $45$ km/h) và $110$ km/h (lớn hơn $105$ km/h).
% 	}
% \end{bt}

% \begin{bt}%[Ngọc Thy Tô]%[Bài giảng Toán 10]%[0D5K14-1]
% 	Trên hai con đường $A$ và $B$, trạm kiểm soát đã ghi lại tốc độ (km/h) của $30$ chiếc ô tô trên mỗi con đường như sau:\\
% 	Con đường $A$:
% 	\begin{center}
% 		\begin{tabular}{|lcccccccccccccc|}
% 			\hline 
% 			$60$&$65$&$70$&$68$&$62$&$75$&$80$&$83$&$82$&$69$&$73$&$75$&$85$&$72$&$67$\\
% 			$88$&$90$&$85$&$72$&$63$&$75$&$76$&$85$&$84$&$70$&$61$&$60$&$65$&$73$&$76$\\ 
% 			\hline 
% 		\end{tabular}\\
% 	\end{center}	
% 	Con đường $B$:
% 	\begin{center}
% 		\begin{tabular}{|lcccccccccccccc|}
% 			\hline 
% 			$76$&$64$&$58$&$82$&$72$&$70$&$68$&$75$&$63$&$67$&$74$&$70$&$79$&$74$&$60$\\
% 			$80$&$73$&$75$&$71$&$68$&$72$&$73$&$79$&$80$&$63$&$62$&$71$&$70$&$69$&$63$\\
% 			\hline 
% 		\end{tabular}\\
% 	\end{center}
% 	\begin{enumerate}
% 		\item Tính khoảng biến thiên, khoảng tứ phân vị của mẫu số liệu.
% 		\item Tính số trung bình, phương sai và độ lệch chuẩn của tốc độ ôtô trên mỗi con đường $A$, $B$.
% 		\item Theo em thì chạy xe trên con đường nào an toàn hơn?
% 	\end{enumerate}
% 	\loigiai{
% 		Bảng tần số của các số liệu về tốc độ của xe đi trên con đường $A$ (Bảng $A$)
% 		\begin{center}
% 			\begin{tabular}{|c|c|c|c|c|c|c|c|c|c|c|c|c|c|c|c|c|c|c|c|c|}
% 				\hline 
% 				Vận tốc & $60$ & $61$ & $62$ & $63$ & $65$ & $67$ & $68$ & $69$ & $70$ & $72$ & $73$ & $75$ & $76$ & $80$ & $82$ & $83$ & $84$ & $85$ & $88$ & $90$ \\ 
% 				\hline 
% 				Số lượng xe & $2$ & $1$ & $1$ & $1$ & $2$ & $1$ & $1$ & $1$ & $2$ & $2$ & $2$ & $3$ & $2$ & $1$ & $1$ & $1$ & $1$ & $3$ & $1$ & $1$\\ 
% 				\hline 
% 			\end{tabular} 
% 		\end{center}
% 		Bảng tần số của các số liệu về tốc độ của xe đi trên con đường $B$ (Bảng $B$)
% 		\begin{center}
% 			\begin{tabular}{|c|c|c|c|c|c|c|c|c|c|c|c|c|c|c|c|c|c|c|}
% 				\hline 
% 				Vận tốc & $58$ & $60$ & $62$ & $63$ & $64$ & $67$ & $68$ & $69$ & $70$ & $71$ & $72$ & $73$ & $74$ & $75$ & $76$ & $79$ & $80$ & $82$  \\ 
% 				\hline 
% 				Số lượng xe & $1$ & $1$ & $1$ & $3$ & $1$ & $1$ & $2$ & $1$ & $3$ & $2$ & $2$ & $2$ & $2$ & $2$ & $1$ & $2$ & $2$ & $1$ \\ 
% 				\hline 
% 			\end{tabular} 
% 		\end{center}
% 		\begin{enumerate}
% 			\item Khoảng biến thiên: $R_A=90-60=30$; $R_B=82-58=24$.\\
% 			Khoảng tứ phân vị của Bảng $A$\\
% 			Ta có $Q_1=67$; $Q_3=82$. Suy ra $\Delta_Q=82-67=15$.\\
% 			Khoảng tứ phân vị của Bảng $B$\\
% 			Ta có $Q_1=67$; $Q_3=75$. Suy ra $\Delta_Q=75-67=8$.
% 			\item Với bảng $A$. Ta có: $\overline{x}_A\approx 73,63$ km/h, $s^2_A \approx 74,77$, $s_A \approx 8,65$ km/h.\\
% 			Với bảng $B$. Ta có: $\overline{x}_B \approx 70,7$ km/h, $s^2_B\approx 38,21$, $s_B\approx 6,18$ km/h.
% 			\item Nhận xét: Trên con đường $B$, tốc độ trung bình và độ lệch chuẩn đều nhỏ hơn trên con đường $A$. Do đó chạy xe trên con đường $B$ sẽ an toàn hơn trên con đường $A$.
% 		\end{enumerate}
% 	}
% \end{bt}

% \begin{bt}%[Ngọc Thy Tô]%[Bài giảng Toán 10]%[0D5B14-1]
% 	Hai lớp 10A và 10B của một trường THPT cùng làm bài thi môn Toán, chung một đề thi. Kết quả thi được trình bày ở hai bảng phân bố tần số sau đây
% 	\begin{center}
% 		\begin{tabular}{|c|c|c|c|c|c|c|c|c|}
% 			\hline 
% 			Điểm & $3$ & $5$ & $6$ & $7$ & $8$ & $9$ & $10$ & Cộng \\ 
% 			\hline 
% 			Lớp 10A & $7$ & $9$ & $3$ & $3$ & $7$ & $12$ & $4$ & $45$ \\ 
% 			\hline 
% 		\end{tabular}
% 	\end{center}
	
% 	\begin{center}
% 		\begin{tabular}{|c|c|c|c|c|c|c|c|c|}
% 			\hline 
% 			Điểm & $4$ & $5$ & $6$ & $7$ & $8$ & $9$ & $10$ & Cộng \\ 
% 			\hline 
% 			Lớp 10B & $6$ & $6$ & $7$ & $8$ & $9$ & $5$ & $4$ & $45$ \\ 
% 			\hline 
% 		\end{tabular} 
% 	\end{center}
% 	\begin{enumerate}
% 		\item Hãy tính số trung bình, phương sai, độ lệch chuẩn từ các bảng phân bố tần số đã cho (làm tròn đến chữ số thập phân thứ hai).
% 		\item Xét xem kết quả bài thi môn Toán của lớp nào đồng đều hơn?
% 	\end{enumerate}
% 	\loigiai{
% 		\begin{enumerate}
% 			\item \textbf{Lớp 10A:}\\
% 			Số trung bình $\overline{x}_A=\dfrac{3\cdot 7 + 5\cdot 9+ \cdots + 10\cdot 4}{45}\approx 6,87$.\\
% 			Phương sai $s^2_A=\dfrac{(3-\overline{x}_A)^2\cdot 7 + \ldots + (10-\overline{x}_A)^2\cdot 4}{45}\approx 5,38$. \\
% 			Độ lệch chuẩn $s_A=\sqrt{s_A^2}\approx 2,32$. \\
% 			\textbf{Lớp 10B:}\\
% 			Số trung bình $\overline{x}_B\approx 6,87$. \\
% 			Phương sai $s_B^2\approx 3,69$. \\
% 			Độ lệch chuẩn $s_B\approx 1,92$. 
% 			\item Kết quả bài thi môn Toán của lớp 10B đồng đều hơn vì $s_B<s_A$. 
% 		\end{enumerate}
% 	}
% \end{bt}

% \begin{bt}%[Ngọc Thy Tô]%[Bài giảng Toán 10]%[0D5Y14-1]
% 	Bảng số liệu sau cho ta lãi (quy tròn) hàng tháng của một cửa hàng $A$ trong năm $2006$ (đơn vị là triệu đồng).
% 	\begin{center}
% 		\begin{tabular}{|c|c|c|c|c|c|c|c|c|c|c|c|c|}
% 			\hline 
% 			Tháng & $1$ & $2$ & $3$ & $4$ & $5$ & $6$ & $7$ & $8$ & $9$ & $10$ & $11$ & $12$ \\ 
% 			\hline 
% 			Lãi & $12$ & $15$ & $18$ & $13$ & $18$ & $16$ & $17$ & $14$ & $18$ & $17$ & $20$ & $17$ \\ 
% 			\hline 
% 		\end{tabular} 
% 	\end{center}
% 	Tìm số  trung bình. Tìm phương sai và độ lệch chuẩn.
% 	\loigiai{
% 		Trung bình cộng: $16,25$; phương sai $s^2\approx 5,02$; độ lệch chuẩn $2,24$.
% 	}
% \end{bt}

% \begin{bt}%[Ngọc Thy Tô]%[Bài giảng Toán 10]%[0D5G14-1]
% 	Cho biểu đồ biểu diễn kết quả học tập của học sinh trong một lớp qua một bài kiểm tra.
% 	\begin{center}
% 		\begin{tikzpicture}[>=stealth,scale=0.7, line join = round, line cap = round]
% 			\tkzInit[xmin=-1,xmax=14,ymin=-0.9,ymax=9.4]
% 			\tkzClip
% 			\draw[->,line width = 1pt] (0,0)--(11,0) node[below right]{$x$ (điểm)};
% 			\draw[->,line width = 1pt] (0,0)node[below]{$O$} --(0,9) node[left]{$m$};
% 			\draw (1,0) node[below]{$1$} circle (1pt);
% 			\draw (2,0) node[below]{$2$} circle (1pt);
% 			\draw (3,0) node[below]{$3$} circle (1pt);
% 			\draw (4,0) node[below]{$4$} circle (1pt);
% 			\draw (5,0) node[below]{$5$} circle (1pt);
% 			\draw (6,0) node[below]{$6$} circle (1pt);
% 			\draw (7,0) node[below]{$7$} circle (1pt);
% 			\draw (8,0) node[below]{$8$} circle (1pt);
% 			\draw (9,0) node[below]{$9$} circle (1pt);
% 			\draw (10,0) node[below]{$10$} circle (1pt);
% 			\draw (0,1) node[left]{$1$} circle (1pt);
% 			\draw (0,2) node[left]{$2$} circle (1pt);
% 			\draw (0,4) node[left]{$4$} circle (1pt);
% 			\draw (0,6) node[left]{$6$} circle (1pt);
% 			\draw (0,7) node[left]{$7$} circle (1pt);
% 			\draw (0,8) node[left]{$8$} circle (1pt);
% 			\tkzDefPoints{-1/0/A, 0/0/O, 1/0/B, 2/0/C, 3/0/D, 4/0/E, 5/0/F, 6/0/G, 7/0/H, 8/0/I, 9/0/J, 10/0/K, 1/0.5/B', 2/1/C', 3/2/D', 4/4/E', 5/2/F', 6/7/G', 7/8/H', 8/6/I', 9/2/J', 10/1/K', 0/1/M, 0/2/N, 0/4/P, 0/6/Q, 0/7/R, 0/8/S}
% 			\tkzDrawSegments[dashed](C,C' D,D' E,E' F,F' G,G' H,H' I,I' J,J' K,K')
% 			\tkzDrawSegments[dashed](M,K' N,J' P,E' Q,I' R,G' S,H')
% 			\draw (1,0)--(2,1)--(3,2)--(4,4)--(5,2)--(6,7)--(7,8)--(8,6)--(9,2)--(10,1);
% 		\end{tikzpicture}
% 	\end{center}
% 	Từ biểu đồ trên hãy
% 	\begin{enumerate}
% 		\item Viết mẫu số liệu thống kê kết quả học tập của học sinh một lớp nhận được từ biểu đồ đã cho.
% 		\item Tìm khoảng biến thiên của mẫu số liệu đó.
% 		\item Tìm khoảng tứ phân vị trong mẫu số liệu đó.
% 		\item Tính phương sai và độ lệch chuẩn của mẫu số liệu đó.
% 	\end{enumerate}
% 	\loigiai{
% 		\begin{enumerate}
% 			\item Bảng mẫu số liệu thống kê
% 			\begin{center}
% 				\begin{tabular}{|c|c|c|c|c|c|c|c|c|c|c|}
% 					\hline 
% 					Điểm & $1$ & $2$ & $3$ & $4$ & $5$ & $6$ & $7$ & $8$ & $9$ & $10$ \\ 
% 					\hline 
% 					Số lượng & $0$ & $1$ & $2$ & $4$ & $2$ & $7$ & $8$ & $6$ & $2$ & $1$ \\ 
% 					\hline 
% 				\end{tabular} 
% 			\end{center}
% 			\item Khoảng biến thiên $R=10-2=8$.
% 			\item Khoảng tứ phân vị \\
% 			Ta có $Q_1=5$; $Q_3=8$. Vậy $\Delta_Q=8-5=3$.
% 			\item Điểm trung bình $\overline{x}=\dfrac{2\cdot 1+ 3\cdot 2+4\cdot 4+5\cdot 2+6\cdot 7+7\cdot 8+8\cdot 6+9\cdot 2+10\cdot 1}{33}\approx 6{,}3$.
% 			\begin{center}
% 				\begin{tabular}{|c|c|c|}
% 					\hline 
% 					Giá trị & Độ lệch & Bình phương độ lệch\\ 
% 					\hline 
% 					$2$ & $6{,}3-2=4{,}3$ & $18{,}49$ \\ 
% 					\hline 
% 					$3$ & $6{,}3-3=3{,}3$ & $10{,}89$\\ 
% 					\hline
% 					$4$ & $6{,}3-4=2{,}3$ & $5{,}29$\\ 
% 					\hline
% 					$5$ & $6{,}3-5=1{,}3$ & $1{,}69$ \\ 
% 					\hline
% 					$6$ & $6{,}3-6=0{,}3$ & $0{,}09$\\ 
% 					\hline
% 					$7$ & $6{,}3-7=-0{,}7$ & $0{,}49$\\ 
% 					\hline
% 					$8$ & $6{,}3-8=-1{,}7$ & $2{,}89$\\ 
% 					\hline
% 					$9$ & $6{,}3-9=-2{,}7$ & $7{,}29$\\ 
% 					\hline
% 					$10$ & $6{,}3-10=-3{,}7$ & $13{,}69$\\ 
% 					\hline
% 				\end{tabular}
% 			\end{center}
% 			Phương sai $$s^2=\dfrac{18{,}49+2\cdot 10{,}89+4\cdot 5{,}29+2\cdot 1{,}69+7\cdot 0{,}09+8\cdot 0{,}49+6\cdot 2{,}89+2\cdot 7{,}29+13{,}69}{33}\approx 68{,}82.$$
% 			Độ lệch chuẩn: $s=\sqrt{68{,}82}\approx 8{,}3$.
% 		\end{enumerate}
% 	}
% \end{bt}
% \subsection*{Câu hỏi trắc nghiệm (bổ sung)}
% \begin{ex}%[Ngọc Thy Tô]%[Bài giảng Toán 10]%[0D5Y14-1]
% 	Cho dãy số liệu thống kê: $1,2,3,4,5,6,7$.
% 	Phương sai của các số liệu thống kê đã cho là.
% 	\choice
% 	{$1$}
% 	{$2$}
% 	{$3$}
% 	{\True $4$}
% 	\loigiai{
% 		Giá trị trung bình $\overline{x}=\dfrac{1+2+3+4+5+6+7}{7}=4$.\\
% 		Phương sai $s^2=\dfrac{(4-1)^2+(4-2)^2+(4-3)^2+(4-4)^2+(5-4)^2+(6-4)^2+(4-7)^2}{7}=4$.
% 	}
% \end{ex}

% \begin{ex}%[Ngọc Thy Tô]%[Bài giảng Toán 10]%[0D5Y14-1]
% 	Sản lượng lúa (đơn vị là tạ) của 40 thửa ruộng thí nghiệm có cùng diện tích được trình bày trong bảng tần số sau đây.
% 	\begin{center}
% 		\begin{tabular}{|>{\centering\arraybackslash}p{4cm}|>{\centering\arraybackslash}p{1.5cm}|>{\centering\arraybackslash}p{1.5cm}|>{\centering\arraybackslash}p{1.5cm}|>{\centering\arraybackslash}p{1.5cm}|>{\centering\arraybackslash}p{1.5cm}|>{\centering\arraybackslash}p{2cm}|}
% 			\hline  {\bf Sản lượng}&$20$  &$21$  &$22$  &$23$  &$24$  &  \\ 
% 			\hline  {\bf Số thửa ruộng}&$5$  &$8$  &$11$  &$10$  &$6$  & $N=40$  \\ 
% 			\hline 
% 		\end{tabular} 
% 	\end{center}
% 	Tính độ lệch chuẩn.
% 	\choice
% 	{$s\approx 1,34$ (tạ)}
% 	{\True $s\approx 1,24$ (tạ)}
% 	{$s\approx 1,54$ (tạ)}
% 	{$s\approx 1,64$ (tạ)}
% 	\loigiai{
% 		Giá trị trung bình $\overline{x}=\dfrac{20\cdot 5+21\cdot 8+22\cdot 11+23\cdot 10+24\cdot 6}{40}=22{,}1$.\\
% 		Phương sai \\
% 		$S^2=\dfrac{5\cdot (22{,}1-20)^2+8\cdot (21{,}1-21)^2+11\cdot (2{,}1-22)^2+10\cdot (22{,}1-23)^2+6\cdot (22{,}1-24)^2}{40}=1{,}54$.\\
% 		Độ lệch chuẩn $s=\sqrt{1{,}54}\approx 1{,}24$.
% 	}
% \end{ex}

% \begin{ex}%[Ngọc Thy Tô]%[Bài giảng Toán 10]%[0D5Y14-1]
% 	Tiền thưởng (đơn vị là triệu đồng) cho cán bộ và nhân viên trong một công ty được trình bày trong bảng phân bố tần số sau đây.
% 	\begin{center}
% 		\begin{tabular}{|>{\centering\arraybackslash}p{4cm}|>{\centering\arraybackslash}p{1.5cm}|>{\centering\arraybackslash}p{1.5cm}|>{\centering\arraybackslash}p{1.5cm}|>{\centering\arraybackslash}p{1.5cm}|>{\centering\arraybackslash}p{1.5cm}|>{\centering\arraybackslash}p{2cm}|}
% 			\hline  {\bf Tiền thưởng (triệu đồng)}&$2$  &$3$  &$4$  &$5$  &$6$  & Cộng \\ 
% 			\hline  {\bf Số cán bộ và nhân viên}&$5$  &$15$  &$10$  &$6$  &$7$  & $43$  \\ 
% 			\hline 
% 		\end{tabular} 
% 	\end{center}
% 	Tính độ lệch chuẩn.
% 	\choice
% 	{$s\approx 1{,}23$ (triệu đồng)}
% 	{$s\approx 1{,}24$ (triệu đồng)}
% 	{$s\approx 1{,}25$ (triệu đồng)}
% 	{\True $s\approx 1{,}26$ (triệu đồng)}
% 	\loigiai{
% 		Giá trị trung bình $\overline{x}=\dfrac{2\cdot 5+3\cdot 15+4\cdot 10+5\cdot 6+6\cdot 7}{43}=\dfrac{167}{43}$.\\
% 		Phương sai $s^2=\dfrac{5\cdot\left( \frac{167}{43}-2\right)^2+15\cdot\left( \frac{167}{43}-3\right)^2+10\cdot\left( \frac{167}{43}-4\right)^2+6\cdot\left( \frac{167}{43}-5\right)^2+7\cdot\left( \frac{167}{43}-6\right)^2}{43}=1{,}59$.\\
% 		Độ lệch chuẩn $s=\sqrt{1{,}59}\approx 1{,}26$.
% 	}
% \end{ex}


% \begin{ex}%[Ngọc Thy Tô]%[Bài giảng Toán 10]%[0D5Y14-1]
% 	Cho dãy số liệu thống kê: $1$, $2$, $3$, $4$, $5$, $6$, $7$. Tìm khoảng biến thiên của mẫu số liệu.
% 	\choice
% 	{$R=7$}
% 	{$R=4$}
% 	{$R=8$}
% 	{\True $R=6$}
% 	\loigiai{
% 		Khoảng biến thiên của mẫu số liệu $R=7-1=6$.
% 	}
% \end{ex}

% \begin{ex}%[Ngọc Thy Tô]%[Bài giảng Toán 10]%[0D5Y14-1]
% 	Cho biết giá trị thành phẩm quy ra tiền (nghìn đồng) trong một tuần lao động của 7 công nhân là $$180, 190, 190, 200, 210, 210, 220.$$  Tìm khoảng tứ phân vị của mẫu số liệu.
% 	\choice
% 	{$190$}
% 	{\True $20$}
% 	{$210$}
% 	{$200$}
% 	\loigiai{
% 		Ta có $Q_1=190$; $Q_3=210$.\\
% 		Khoảng tứ phân vị $\Delta_Q=Q_3-Q_1=210-190=20$.
% 	}
% \end{ex}

% \begin{ex}%[Ngọc Thy Tô]%[Bài giảng Toán 10]%[0D5B14-1]
% 	Tiền thưởng (triệu đồng) cho cán bộ và nhân viên trong một công ty cho bởi bảng phân bố tần số sau\\
% 	\begin{center}
% 		\begin{tabular}{|c|c|c|c|c|c|c|c|c|c|c|c|} 
% 			\hline Tiền thưởng &2&8&4&5&6\\
% 			\hline
% 			Số cán bộ và nhân viên  &5&15&10&6&7\\
% 			\hline
% 		\end{tabular}
% 	\end{center}
% 	Phương sai của bảng số liệu trên thuộc khoảng nào dưới đây?
% 	\choice
% 	{\True $(4{,}1; 4{,}2)$}
% 	{$(4{,}2; 4{,}3)$}
% 	{$(4{,}3; 4{,}4)$}
% 	{$(4{,}4; 4{,}5)$}
% 	\loigiai{
% 		Giá trị trung bình $\overline{x}=\dfrac{2\cdot 5+8\cdot 15+4\cdot 10+5\cdot 6+6\cdot 7}{43}=\dfrac{242}{43}$.\\
% 		Phương sai \\
% 		$s^2=\dfrac{5\cdot\left( \frac{242}{43}-2\right)^2+15\cdot\left( \frac{242}{43}-8\right)^2+10\cdot\left( \frac{242}{43}-4\right)^2+6\cdot\left( \frac{242}{43}-5\right)^2+7\cdot\left( \frac{242}{43}-6\right)^2}{43}\approx 4{,}19$.\\
% 		Vậy $s^2\in (4{,}1; 4{,}2)$.
% 	}
% \end{ex}

% \begin{ex}%[Ngọc Thy Tô]%[Bài giảng Toán 10]%[0D5Y14-1]
% 	Khách đến tham quan một điểm du lịch trong mỗi tháng được thống kê trong bảng sau đây.
% 	\begin{center}
% 		\begin{tabular}{|c|c|c|c|c|c|c|c|c|c|c|c|c|} 
% 			\hline Tháng &1&2&3&4&5&6&7&8&9&10&11&12\\
% 			\hline
% 			Khách &430&560&450&550&760&430&525&110&635&450&800&950\\
% 			\hline
% 		\end{tabular}
% 	\end{center}
% 	Tính  độ lệch chuẩn $s$ của bảng trên.
% 	\choice
% 	{$s\approx 211$}
% 	{$s\approx 209{,}3$}
% 	{ $s\approx 403{,}54$}
% 	{\True $s \approx 207{,}51$}
% 	\loigiai{
% 		Giá trị trung bình $\overline{x}=\dfrac{110+430\cdot 2+450\cdot 2+525+550+560+635+760+800+950}{12}=\dfrac{3325}{6}$.\\
% 		Phương sai
% 		\begin{eqnarray*}
% 			&s^2&=\dfrac{\left( \frac{3325}{6}-110\right)^2+2\cdot\left( \frac{3325}{6}-430\right)^2+2\cdot\left( \frac{3325}{6}-450\right)^2+\left( \frac{3325}{6}-525\right)^2+\left( \frac{3325}{6}-560\right)^2}{12}\\
% 			&&+\dfrac{\left( \frac{3325}{6}-635\right)^2+\left(\frac{3325}{6}-760\right)^2+\left( \frac{3325}{6}-800\right)^2+\left( \frac{3325}{6}-950\right)^2}{12}\\
% 			&&=43060{,}3588.
% 		\end{eqnarray*}
% 		Độ lệch chuẩn $s=\sqrt{43060{,}3588}\approx 207{,}51$.
% 	}
% \end{ex}

% \begin{ex}%[Ngọc Thy Tô]%[Bài giảng Toán 10]%[0D5Y14-2]
% 	Độ lệch chuẩn bằng
% 	\choice
% 	{bình phương của phương sai}
% 	{\True căn bậc hai số học của phương sai}
% 	{một nửa của phương sai}
% 	{hai lần phương sai}
% 	\loigiai{
% 		Độ lệch chuẩn bằng căn bậc hai số học của phương sai.
% 	}
% \end{ex}

% \begin{ex}%[Ngọc Thy Tô]%[Bài giảng Toán 10]%[0D5Y14-1]
% 	Sản lượng lúa (đơn vị là tạ) của $40$ thửa ruộng thí nghiệm có cùng diện tích được trình bày trong bảng tấn số sau đây
% 	\begin{center}
% 		\begin{tabular}{|c|c|c|c|c|c|c|c|c|c|c|c|c|} 
% 			\hline Sản lượng  &20&21&22&23&24\\
% 			\hline
% 			Số thửa ruộng  &5&8&11&10&6\\
% 			\hline
% 		\end{tabular}
% 	\end{center}
% 	Tính khoảng tứ phân vị của mẫu số liệu.
% 	\choice
% 	{$3$}
% 	{$4$}
% 	{ $1$}
% 	{\True $s2$}
% 	\loigiai{
% 		Ta có $Q_1=21$; $Q_3=23$.\\
% 		Khoảng tứ phân vị $\Delta_Q=23-21=2$.
% 	}
% \end{ex}

% \begin{ex}%[Ngọc Thy Tô]%[Bài giảng Toán 10]%[0D5Y14-2]
% 	Chọn khẳng định {\bf sai} trong các khẳng định sau.
% 	\choice
% 	{Phương sai luôn là một số không âm}
% 	{\True Phương sai không có đơn vị}
% 	{Phương sai càng lớn thì độ phân tán càng lớn}
% 	{Độ lệch chuẩn càng lớn thì độ phân tán càng lớn}
% \end{ex}

% \begin{ex}%[Ngọc Thy Tô]%[Bài giảng Toán 10]%[0D5Y14-1]
% 	Có $100$ học sinh tham dự kì thi học sinh giỏi Toán (thang điểm là $20$). Kết quả được cho trong bảng sau đây.
% 	\begin{center}
% 		\begin{tabular}{|>{\centering\arraybackslash}p{2,5cm}|>{\centering\arraybackslash}p{0.8cm}|>{\centering\arraybackslash}p{0.8cm}|>{\centering\arraybackslash}p{0.8cm}|>{\centering\arraybackslash}p{0.8cm}|>{\centering\arraybackslash}p{0.8cm}|>{\centering\arraybackslash}p{0.8cm}|>{\centering\arraybackslash}p{0.8cm}|>{\centering\arraybackslash}p{0.8cm}|>{\centering\arraybackslash}p{0.8cm}|>{\centering\arraybackslash}p{0.8cm}|>{\centering\arraybackslash}p{0.8cm}|>{\centering\arraybackslash}p{2cm}|}
% 			\hline  
% 			{\bf Điểm}&$9$  &$10$  &$11$  &$12$  &$13$  &$14$ &$15$ &$16$ &$17$ &$18$ &$19$  \\ 
% 			\hline  
% 			{\bf Số học sinh}&$1$  &$1$  &$3$  &$5$  &$8$  &$13$ &$19$ &$24$ &$14$ &$10$ &$2$   \\ 
% 			\hline 
% 		\end{tabular} 
% 	\end{center}
% 	Tính độ lệch chuẩn.
% 	\choice
% 	{$s \approx 1{,}76$ (điểm)}
% 	{$s \approx 1{,}77$ (điểm)}
% 	{$s \approx 1{,}78$ (điểm)}
% 	{\True $s \approx 1{,}79$ (điểm)}
% 	\loigiai{
% 		Giá trị trung bình\\
% 		$\overline{x}=\dfrac{9+10+11+12\cdot 5+13\cdot 8+14\cdot 13+15\cdot 19+16\cdot 24+17\cdot 14+18\cdot 10+19\cdot 2}{100}=15{,}23$.\\
% 		Phương sai
% 		\begin{eqnarray*}
% 			&s^2&=\dfrac{(15,23-9)^2+(15,23-10)^2+(15,23-11)^2+5\cdot (15,23-12)^2+8\cdot (15,23-13)^2}{100}\\
% 			&&+\dfrac{13\cdot (15,23-14)^2+19\cdot (15,23-15)^2+24\cdot (15,23-16)^2+14\cdot (15,23-17)^2}{100}\\
% 			&&+\dfrac{10\cdot (15,23-18)^2+2\cdot (15,23-19)^2}{100}\\
% 			&&\approx 3{,}19.
% 		\end{eqnarray*}
% 		Độ lệch chuẩn $s=\sqrt{3{,}19}\approx 1{,}79$.
% 	}
% \end{ex}

% \begin{ex}%[Ngọc Thy Tô]%[Bài giảng Toán 10]%[0D5Y14-1]
% 	Một nhà nghiên cứu ghi lại tuổi của $30$ bệnh nhân mắc bệnh đau mắt hột. Kết quả thu được mẫu số liệu như sau\\
% 	$21$ $17$ $22$ $18$ $20$ $17$ $15$ $13$ $15$ $20$ $15$ $12$ $18$ $17$ $25$ $17$ $21$ $15$ $12$ $18$ $16$ $23$ $14$ $18$ $19$ $13$ $16$ $19$ $18$ $17$.\\
% 	Tính khoảng biến thiên của mẫu số liệu.
% 	\choice
% 	{$25$}
% 	{\True $13$}
% 	{ $26$}
% 	{ $12$}
% 	\loigiai{
% 		Khoảng biên thiên $R=25-12=13$.
% 	}
% \end{ex}

% \begin{ex}%[Ngọc Thy Tô]%[Bài giảng Toán 10]%[0D5B14-1]
% 	Điểm trung bình từng môn học của hai học sinh An và Bình trong năm học vừa qua được cho trong bảng sau.
% 	\begin{center}
% 		\begin{tabular}{|c|c|c|}
% 			\hline  
% 			{\bf Môn}&{\bf Điểm của An}&{\bf Điểm của Bình} \\
% 			\hline 
% 			Toán & $8$& $8,\!5$ \\
% 			Vật Lý & $7,\!5$& $9,\!5$ \\
% 			Hóa học & $7,\!8$& $9,\!5$ \\
% 			Sinh học & $8,\!3$& $8,\!5$ \\
% 			Ngữ văn & $7$& $5$ \\
% 			Lịch sử & $8$& $5,\!5$ \\
% 			Địa lý & $8,\!2$& $6$ \\
% 			Tiếng Anh & $9$& $9$ \\
% 			Thể dục & $8$& $9$ \\
% 			Công nghệ & $8,\!3$& $8,\!5$ \\
% 			Giáo dục công dân & $9$& $10$ \\
% 			\hline
% 		\end{tabular} 
% 	\end{center}
% 	Hỏi ai ``học lệch'' hơn?
% 	\choice
% 	{An}
% 	{\True Bình}
% 	{Mức độ học lệch của hai người như nhau}
% 	{Chưa đủ cơ sở kết luận}
% 	\loigiai{
% 		\begin{itemize}
% 			\item Đối với bạn An\\
% 			Điểm trung bình $\overline{x}=\dfrac{7+7{,}5+7{,}8+8\cdot 3+8{,}2+8{,}3\cdot 2+9\cdot 2}{11}=8{,}1$.\\
% 			Phương sai
% 			\begin{eqnarray*}
% 				&s^2&=\dfrac{(8{,}1-7)^2+(8{,}1-7{,}5)^2+(8{,}1-7{,}8)^2+2\cdot (8{,}1-8)^2+(8{,}1-8{,}2)^2}{11}\\
% 				&&+\dfrac{2\cdot (8{,}1-8{,}3)^2+2\cdot (8{,}1-9)^2}{11}\\
% 				&&\approx 0{,}31.
% 			\end{eqnarray*}
% 			\item Đối với bạn Bình\\
% 			Điểm trung bình $\overline{x}=\dfrac{5+5{,}5+6+8{,}5\cdot 3+9\cdot 2+9{,}5\cdot 2+10}{11}=8{,}9$.\\
% 			Phương sai
% 			\begin{eqnarray*}
% 				&s^2&=\dfrac{(8{,}9-5)^2+(8{,}9-5{,}5)^2+(8{,}9-6)^2+3\cdot (8{,}9-8{,}5)^2+2\cdot(8{,}9-9)^2+(8{,}9-10)^2}{11}\\
% 				&&\approx 3{,}35.
% 			\end{eqnarray*}
% 		\end{itemize}
% 		Dựa vào phương sai suy ra bạn Bình học lệch hơn.
% 	}	
% \end{ex}

% \begin{ex}%[Ngọc Thy Tô]%[Bài giảng Toán 10]%[0D5G14-1]
% 	Bảng sau đây cho ta biết số cuốn sách mà học sinh của một lớp ở trường Trung học phổ thông đã đọc trong năm $2016$.
% 	\begin{center}
% 		\begin{tabular}{|>{\centering\arraybackslash}p{3cm}|>{\centering\arraybackslash}c|>{\centering\arraybackslash}c|>{\centering\arraybackslash}c|>{\centering\arraybackslash}c|>{\centering\arraybackslash}c|>{\centering\arraybackslash}c|>{\centering\arraybackslash}p{2cm}|}
% 			\hline
% 			Số sách & $1$ & $2$ & $3$ & $4$ & $5$ & $6$  & Cộng \\
% 			\hline
% 			Số học sinh  & $10$ & $x$ & $8$ & $6$ & $y$ & $3$ & $40$ \\
% 			\hline
% 		\end{tabular}
% 	\end{center}
% 	Tính $x$ và $y$, biết rằng phương sai của bảng số liệu $s^2 \approx 2,52$.
% 	\choice
% 	{$x=7,y=6$}
% 	{$x=6,y=7$}
% 	{\True $x=8,y=5$}
% 	{$x=5,y=8$}
% 	\loigiai{
% 		Ta có $10+x+8+6+y+3=40\Leftrightarrow x+y=13\Leftrightarrow y=13-x$.\\
% 		Giá trị trung bình $\overline{x}=\dfrac{10+2x+3\cdot 8+4\cdot 6+5y+6\cdot 3}{40}=\dfrac{2x+5y+76}{40}=\dfrac{121-3x}{40}$.\\
% 		Phương sai 
% 		\begin{eqnarray*}
% 			&&s^2\approx 2{,}52\\
% 			&\Leftrightarrow&\dfrac{10\cdot\left(\frac{121-3x}{40}-1\right)^2+x\cdot\left(\frac{121-3x}{40}-2\right)^2+8\cdot\left(\frac{121-3x}{40}-3\right)^2+6\cdot\left(\frac{121-3x}{40}-4\right)^2}{40}\\
% 			&&+\dfrac{(13-x)\cdot\left(\frac{121-3x}{40}-5\right)^2+3\cdot\left(\frac{121-3x}{40}-6\right)^2}{40}\approx 2{,}52\\
% 			&\Leftrightarrow&x=8\Rightarrow y=5.
% 		\end{eqnarray*}
% 	}
% \end{ex}

% \begin{ex}%[Ngọc Thy Tô]%[Bài giảng Toán 10]%[0D5K14-1]
% 	Cho dãy số liệu thống kê: $x$, $21$, $22$, $23$, $24$, $y$. Tìm $x$, $y$ biết số trung bình cộng bằng $22,5$ và khoảng biến thiên của mẫu số liệu bằng $5$. 
% 	\choice
% 	{$x=-25$ và $y=-20$}
% 	{$x=-20$ và $y=-25$}
% 	{\True $x=20$ và $y=25$}
% 	{$x=25$ và $y=20$}
% 	\loigiai{Ta có: 
% 		$\heva{&\overline{x}=\frac{x+21+22+23+24+y}{6}=22,5\\&y-x=5}\Leftrightarrow\heva{&x=20\\&y=25.}$}
% \end{ex}

\Closesolutionfile{ans}
\Closesolutionfile{ansbook}