\section*{Đề kiểm tra Chương 9}
\subsection*{Đề số 1}
\setcounter{ex}{0}\setcounter{bt}{0}
\Opensolutionfile{ans}[ans/ans-KT-901]
\noindent\textbf{I. PHẦN TRẮC NGHIỆM}
\begin{ex}%[Nguyễn Kiều Nhã Tú- BG Toán 10]%[1D2Y5-4]
	Xếp ngẫu nhiên ba bạn An, Bình, Cường đứng thành một hàng dọc. Xác suất để An không đứng cuối hàng là
	\choice
	{\True $\dfrac{2}{3}$}
	{$\dfrac{1}{3}$}
	{$\dfrac{3}{5}$}
	{$\dfrac{2}{5}$}	
	\loigiai
	{
		Số phần tử của không gian mẫu: $n(\Omega)=3!=6$.\\
		Gọi $A$ là biến cố: \lq\lq  An không đứng cuối hàng\rq\rq \, thì $n(A)=4$.\\
		Xác suất: $\mathrm{P}(A)=\dfrac{n(A)}{n(\Omega)} =\dfrac{4}{6}=\dfrac{2}{3}$.
	}
\end{ex}
\begin{ex}%[Nguyễn Kiều Nhã Tú- BG Toán 10]%[1D2Y5-4]
	Xếp ngẫu nhiên ba bạn An, Bình, Cường đứng thành một hàng dọc. Xác suất để Bình và Cường đứng cạnh nhau là
	\choice
	{$\dfrac{1}{4}$}
	{\True $\dfrac{2}{3}$}
	{$\dfrac{2}{5}$}
	{$\dfrac{1}{2}$}
	\loigiai
	{
		Số phần tử của không gian mẫu: $n(\Omega)=3!=6$.\\
		Gọi $A$ là biến cố: \lq\lq  Bình và Cường đứng cạnh nhau\rq\rq. \\
		Các cách xếp để Bình và Cường đứng cạnh nhau là: An - Bình - Cường; An- Cường - Bình; Bình - Cường - An; Cường - Bình - An.\\
		Do đó  $n(A)=4$.\\
		Xác suất: $\mathrm{P}(A)=\dfrac{n(A)}{n(\Omega)} =\dfrac{4}{6}=\dfrac{2}{3}$.		
	}
\end{ex}
\begin{ex}%[Nguyễn Kiều Nhã Tú- BG Toán 10]%[1D2Y5-4]
	Xếp ngẫu nhiên ba bạn An, Bình, Cường đứng thành một hàng dọc. Xác xuất để Bình đứng trước An là
	\choice
	{$\dfrac{1}{4}$}
	{\True $\dfrac{2}{3}$}
	{$\dfrac{2}{5}$}
	{$\dfrac{1}{2}$}
	\loigiai
	{
		Số phần tử của không gian mẫu: $n(\Omega)=3!=6$.\\
		Gọi $A$ là biến cố: \lq\lq  Bình đứng trước An\rq\rq \, 
		Các cách xếp để Bình đứng trước An là: Cường - Bình - An; Bình - Cường - An; Bình - An - Cường; Cường - Bình - An.\\
		thì $n(A)=3$.\\
		Xác suất: $\mathrm{P}(A)=\dfrac{n(A)}{n(\Omega)} =\dfrac{3}{6}=\dfrac{1}{2}$.			
	}
\end{ex}
\begin{ex}%[Nguyễn Kiều Nhã Tú- BG Toán 10]%[1D2Y5-2]
	Một cái túi đựng $3$ viên bi đỏ, $5$ viên bi xanh và $6$ viên bi vàng có kích thước và khối lượng như nhau. Chọn ngẫu nhiên $3$ viên bi. Xác suất để chọn được $3$ viên bi màu đỏ là
	\choice
	{\True $\dfrac{1}{364}$}
	{$\dfrac{1}{14}$}
	{$\dfrac{1}{182}$}
	{$\dfrac{1}{95}$}
	\loigiai
	{
		Số phần tử của không gian mẫu: $n(\Omega)=\mathrm{C}^3_{14}=364$.\\
		Gọi $A$ là biến cố: \lq\lq  Chọn được $3$ viên màu đỏ\rq\rq \, thì $n(A)=1$.\\
		Xác suất: $\mathrm{P}(A)=\dfrac{n(A)}{n(\Omega)} =\dfrac{1}{364}$.	
	}	
\end{ex}
\begin{ex}%[Nguyễn Kiều Nhã Tú- BG Toán 10]%[1D2Y5-2]
	Gieo hai con xúc xắc cân đối và đồng chất. Xác suất để có đúng $1$ con xúc xắc xuất hiện mặt $6$ chấm là
	\choice
	{$\dfrac{11}{36}$}
	{$\dfrac{1}{3}$}
	{\True $\dfrac{5}{18}$}
	{$\dfrac{4}{9}$}
	\loigiai
	{
		Số phần tử của không gian mẫu: $n(\Omega)=6\cdot 6=36$.\\
		Gọi $A$ là biến cố: \lq\lq  Có đúng $1$ con xúc xắc xuất hiện mặt $6$ chấm\rq\rq .\\
		Ta có 2 trường hợp, trường hợp con xúc xắc thứ nhất xuất hiện mặt 6 và con còn lại là các mặt không phải 6 chấm là $ 1 \cdot 5 =5 $, trường hợp con xúc xắc thứ hai xuất hiện 6 chấm và con đầu tiên xuất hiện mặt khác là $ 5 \cdot 1 = 5 $.\\
		Vậy $n(A)=10$.\\
		Xác suất: $\mathrm{P}(A)=\dfrac{n(A)}{n(\Omega)} =\dfrac{10}{36}=\dfrac{5}{18}$.	
	}	
\end{ex}
\begin{ex}%[Nguyễn Kiều Nhã Tú- BG Toán 10]%[1D2Y5-5]
	Gieo hai con xúc xắc cân đối. Xác suất để tổng số chấm xuất hiện trên hai con xúc xắc nhỏ hơn hoặc bằng $7$ là
	\choice
	{$\dfrac{11}{36}$}
	{\True $\dfrac{7}{12}$}
	{$\dfrac{5}{11}$}
	{$\dfrac{4}{9}$}
	\loigiai
	{
		Không gian mẫu: $n(\Omega)=6\cdot 6=36$.\\
		Gọi $a$, $b$ lần lượt là số chấm của hai con xúc xắc, ta có $a+b\le 7$.\\
		\textbf{Trường hợp 1:} $a$ giống $b$, suy ra $(a,b)\in \left\{(1,1);(2,2);(3,3)\right\}$.\\
		\textbf{Trường hợp 2:} $a\ne b$.
		\begin{itemize}
			\item $a=1\Rightarrow b$ có $5$ khả năng.
			\item $a=2\Rightarrow b$ có $4$ khả năng.
			\item $a=3\Rightarrow b$ có $3$ khả năng.
			\item $a=4\Rightarrow b$ có $3$ khả năng.
			\item $a=5\Rightarrow b$ có $2$ khả năng.
			\item $a=6\Rightarrow b$ có $1$ khả năng.			
		\end{itemize}
		Do đó tổng cộng có $21$ khả năng.\\
		Gọi $A$ là biến cố: \lq\lq  Tổng số chấm xuất hiện trên hai con xúc xắc nhỏ hơn hoặc bằng $7$\rq\rq \, thì $n(A)=21$.\\
		Xác suất: $\mathrm{P}(A)=\dfrac{n(A)}{n(\Omega)} =\dfrac{21}{36}=\dfrac{7}{12}$.			
	}	
\end{ex}
\begin{ex}%[Nguyễn Kiều Nhã Tú- BG Toán 10]%[1D2Y5-2]
	Chọn ngẫu nhiên $5$ số trong tập $S=\{1 ; 2 ; \ldots ; 20\}$. Xác suất để cả $5$ số được chọn không vượt quá $10$ gần với kết quả nào sau đây nhất? 
	\choice
	{\True $0{,}016$}
	{$0{,}013$}
	{$0{,}014$}
	{$0{,}015$}
	\loigiai
	{
		Không gian mẫu: $n(\Omega)=\mathrm{C}^5_{20}$.\\
		Gọi $A$ là biến cố: \lq\lq  Cả $5$ số được chọn không vượt quá $10$\rq\rq \, thì $n(A)=\mathrm{C}^5_{10}$.\\
		Xác suất: $\mathrm{P}(A)=\dfrac{n(A)}{n(\Omega)} =\dfrac{\mathrm{C}^5_{10}}{\mathrm{C}^5_{20}}\approx 0{,}016$.			
	}	
\end{ex}
\begin{ex}%[Nguyễn Kiều Nhã Tú- BG Toán 10]%[1D2B5-2]
	Chọn ngẫu nhiên $5$ học sinh trong một danh sách được đánh số thứ tự từ $1$ đến $199$. Xác suất để cả $5$ học sinh được chọn có số thứ tự nhỏ hơn $100$ gần với kết quả nào sau đây nhất?
	\choice
	{$0{,}028$}
	{\True $0{,}029$}
	{$0{,}027$}
	{$0{,}026$}
	\loigiai
	{
		Không gian mẫu: $n(\Omega)=\mathrm{C}^5_{199}$.\\
		Gọi $A$ là biến cố: \lq\lq  Cả $5$ học sinh được chọn có số thứ tự nhỏ hơn $100$\rq\rq \, thì $n(A)=\mathrm{C}^5_{99}$.\\
		Xác suất: $\mathrm{P}(A)=\dfrac{n(A)}{n(\Omega)} =\dfrac{\mathrm{C}^5_{99}}{\mathrm{C}^5_{199}}\approx 0{,}029$.		
	}	
\end{ex}
\begin{ex}%[Nguyễn Kiều Nhã Tú- BG Toán 10]%[1D2B5-2]
	Chọn ngẫu nhiên $5$ học sinh trong một danh sách được đánh số thứ tự từ $1$ đến $199$. Xác suất để cả $5$ học sinh được chọn có số thứ tự lớn hơn $149$ xấp xỉ là
	\choice
	{$0{,}00089$}
	{$0{,}00083$}
	{$0{,}00088$}
	{\True $0{,}00086$}
	\loigiai
	{
		Không gian mẫu: $n(\Omega)=\mathrm{C}^5_{199}$.\\
		Gọi $A$ là biến cố: \lq\lq  Cả $5$ học sinh được chọn có số thứ tự lớn hơn $149$\rq\rq \, thì $n(A)=\mathrm{C}^5_{50}$.\\
		Xác suất: $\mathrm{P}(A)=\dfrac{n(A)}{n(\Omega)} =\dfrac{\mathrm{C}^5_{50}}{\mathrm{C}^5_{199}}\approx 0{,}00086$.			
	}	
\end{ex}
\begin{ex}%[Nguyễn Kiều Nhã Tú- BG Toán 10]%[1D2B5-2]
	Một túi đựng $3$ viên bi trắng và $5$ viên bi đen có kích thước và khối lượng như nhau. Chọn ngẫu nhiên $3$ viên bi. Xác suất để trong $3$ viên bi đó có cả bi trắng và bi đen là
	\choice
	{$\dfrac{13}{15}$}
	{$\dfrac{9}{11}$}
	{$\dfrac{43}{56}$}
	{\True $\dfrac{45}{56}$}
	\loigiai
	{
		Số phần tử của không gian mẫu: $n(\Omega)=\mathrm{C}^3_{8}$.\\
		Gọi $A$ là biến cố: "Trong $3$ viên bi đó có cả bi trắng và bi đen".\\
		Suy ra $\overline{A}$ là biến cố: "Trong $3$ viên bi đó đều là bi trắng hoặc bi đen" thì $n(\overline{A})=\mathrm{C}^3_{5}+\mathrm{C}^3_{3}$.\\
		Vậy xác suất cần tìm là $\mathrm{P}(A)=1-\mathrm{P}(\overline{A})=1-\dfrac{n\left(\overline{A}\right)}{n(\Omega)}=1-\dfrac{\mathrm{C}^3_{5}+\mathrm{C}^3_{3}}{\mathrm{C}^3_{8}}=1-\dfrac{11}{56}=\dfrac{45}{56}$.
	}	
\end{ex}
\begin{ex}%[Nguyễn Kiều Nhã Tú- BG Toán 10]%[1D2B5-2]
	Mũi tên của bánh xe trong trò chơi \lq\lq  Chiếc nón kì diệu\rq\rq có thể dừng lại ở một trong $7$ vị trí. Người chơi được quay $3$ lần. Xác suất để mũi tên dừng lại ở ba vị trí khác nhau là
	\choice
	{\True $\dfrac{30}{49}$}
	{$\dfrac{29}{50}$}
	{$\dfrac{3}{5}$}
	{$\dfrac{7}{11}$}
	\loigiai
	{
		Số phần tử của không gian mẫu $n(\Omega)=7^3$.\\
		Gọi  $A$ là biến cố "Mũi tên dừng lại ở ba vị trí khác nhau".\\
		Khi đó $n(A)=A_{7}^{3}$.\\
		Vậy xác suất cần tính $ \mathrm{P}(A)=\dfrac{n(A)}{n(\Omega)}=\dfrac{30}{49}$. 		
	}	
\end{ex}
\begin{ex}%[Nguyễn Kiều Nhã Tú- BG Toán 10]%[1D2B5-2]
	Gieo đồng thời hai con xúc xắc cân đối và đồng chất. Xác suất để số chấm xuất hiện trên hai con xúc xắc hơn kém nhau $2$ là
	\choice
	{$\dfrac{5}{22}$}
	{$\dfrac{1}{5}$}
	{\True $\dfrac{2}{9}$}
	{$\dfrac{7}{34}$}
	\loigiai
	{
		Số phần tử của không gian mẫu $n(\Omega)=6^2$.\\
		Gọi  $A$ là biến cố: \lq\lq  Số chấm xuất hiện trên hai con xúc xắc hơn kém nhau $2$ \rq\rq.\\
		Các trường hợp có thể xảy ra là $\{1 ; 3\},\{2 ; 4\},\{3 ; 5\},\{4 ; 6\},\{3 ; 1\},\{4 ; 2\},\{5 ; 3\},\{6 ; 4\}$.\\
		Khi đó $n(A)=8$.\\
		Vậy xác suất cần tính $ \mathrm{P}(A)=\dfrac{n(A)}{n(\Omega)}=\dfrac{2}{9}$.		
	}	
\end{ex}
\begin{ex}%[1D2B5-2]
	Chọn ngẫu nhiên hai số từ tập hợp $S=\{1 ; 2 ; \ldots ; 19\}$ rồi nhân hai số đó với nhau. Xác suất để kết quả là một số lẻ là
	\choice
	{$\dfrac{9}{19}$}
	{$\dfrac{10}{19}$}
	{$\dfrac{4}{19}$}
	{\True $\dfrac{5}{19}$}
	\loigiai
	{
		Số phần tử của không gian mẫu $n(\Omega)=C_{19}^{2}$.\\
		Gọi $A$ là biến cố: \lq\lq  Tích hai số là số lẻ\rq\rq.\\
		Khi đó hai số cần chọn đều là số lẻ.\\
		Ta có $n(A)=C_{10}^{2}$.\\
		Vậy xác suất cần tính $ \mathrm{P}(A)=\dfrac{n(A)}{n(\Omega)}=\dfrac{2}{9}$ 	.	
	}	
\end{ex}
\begin{ex}%[Nguyễn Kiều Nhã Tú- BG Toán 10]%[1D2B5-2]
	Gieo ba con xúc xắc cân đối và đồng chất. Xác suất để số chấm xuất hiện trên mặt của ba con xúc xắc khác nhau là
	\choice
	{\True $\dfrac{5}{9}$}
	{$\dfrac{4}{9}$}
	{$\dfrac{7}{9}$}
	{$\dfrac{2}{9}$}
	\loigiai
	{
		Số phần tử của không gian mẫu $n(\Omega)=6^3$.\\
		Gọi  $A$ là biến cố: \lq\lq  Số chấm xuất hiện trên mặt của ba con xúc xắc khác nhau\rq\rq.\\
		Xét số chấm xuất hiện trên mặt của ba con xúc xắc giống nhau có $6$ trường hợp.\\
		Xét số chấm xuất hiện trên mặt của hai con xúc sắc giống nhau có $C_{3}^{2}\cdot 6\cdot 5$ trường hợp.\\
		Khi đó xác suất cần tính $ \mathrm{P}(A)=1-\dfrac{6+C_{3}^{2} \cdot 6 \cdot 5}{6^{3}}=\dfrac{5}{9}$. 		
	}	
\end{ex}
\begin{ex}%[Nguyễn Kiều Nhã Tú- BG Toán 10]%[1D2B5-2]
	Một khách sạn có $6$ phòng đơn. Có $10$ khách thuê phòng trong đó có $6$ nam và $4$ nữ. Người quản lí chọn ngẫu nhiên $6$ người cho nhận phòng. Xác suất để cả $6$ người là nam là
	\choice
	{$\dfrac{11}{210}$}
	{$\dfrac{1}{105}$}
	{\True $\dfrac{1}{210}$}
	{$\dfrac{7}{210}$}
	\loigiai
	{
		Số phần tử của không gian mẫu: $n(\Omega)=\mathrm{C}^6_{10}$.\\	
		Gọi $A$ là biến cố: \lq\lq  cả $6$ người là nam \rq\rq thì $n(A)=\mathrm{C}^6_{6}$.\\
		Khi đó xác suất cần tính $\mathrm{P}(A)=\dfrac{n(A)}{n(\Omega)} =\dfrac{\mathrm{C}^6_{6}}{\mathrm{C}^6_{10}}=\dfrac{1}{210}$.		
	}	
\end{ex}
\begin{ex}%[Nguyễn Kiều Nhã Tú- BG Toán 10]%[1D2Y5-1]
	Cho $A$ và $\overline{A}$ là hai biến cố đối nhau. Hãy chọn khẳng định đúng.
	\choice
	{$\mathrm{P}(A)=1+\mathrm{P}\left(\overline{A}\right)$}
	{$\mathrm{P}(A)=\mathrm{P}\left(\overline{A}\right)$}
	{\True $\mathrm{P}(A)=1-\mathrm{P}\left(\overline{A}\right)$}
	{$\mathrm{P}(A)+\mathrm{P}\left(\overline{A}\right)=0$}
	\loigiai{
		Theo định nghĩa, ta có $ \mathrm{P}(A)=1-\mathrm{P}\left(\overline{A}\right) $.
	}
\end{ex}
\begin{ex}%[Nguyễn Kiều Nhã Tú- BG Toán 10]%[1D2B5-1]
	Một xưởng sản xuất có $n$ máy, trong đó có một số máy hỏng. Gọi $A_k$ là biến cố: \lq\lq  Máy thứ $k$ bị hỏng \rq\rq, $k=1,2,\ldots,n$. Biến cố $A$: \lq\lq  Cả $n$ máy đều tốt\rq\rq \, là
	\choice
	{$A=A_1\cdot A_2\cdots A_n$}
	{$A=\overline{A_1\cdot A_2\cdots  A_{n-1}\cdot A_n}$}
	{$A=A_1\cdot A_2\cdots A_{n-1}\cdot\overline{A_n}$}
	{\True $A=\overline{A_1}\cdot\overline{A_2}\cdots \overline{A_n}$}
	\loigiai{
		Ta có $A_k$ là biến cố: \lq\lq  Máy thứ $k$ bị hỏng \rq\rq, $k=1,2,\cdots,n$.\\
		Do đó $A=\overline{A_1}\cdot\overline{A_2}\cdots \overline{A_n}, k=1,2,\cdots,n$.
	}
\end{ex}
\begin{ex}%[Nguyễn Kiều Nhã Tú- BG Toán 10]%[1D2B5-1]
	Một xạ thủ bắn liên tục $4$ phát đạn vào bia. Gọi $A_k$ là các biến cố \lq\lq  Xạ thủ bắn trúng lần thứ $k$ \rq\rq với $k=1$, $2$, $3$, $4$. Xác suất biến cố $A$: \lq\lq  Lần thứ tư mới bắn trúng bia\rq\rq \, là
	\choice
	{$\mathrm{P}(A)=\mathrm{P}(\overline{A_1})\cdot \mathrm{P}(\overline{A_2})\cdot \mathrm{P}(\overline{A_3})\cdot \mathrm{P}(\overline{A_4})$}
	{\True $\mathrm{P}(A)=\mathrm{P}(\overline{A_1})\cdot \mathrm{P}(\overline{A_2})\cdot \mathrm{P}(\overline{A_3})\cdot \mathrm{P}(A_4)$}
	{$\mathrm{P}(A)=\mathrm{P}(A_1)\cdot \mathrm{P}(A_2)\cdot \mathrm{P}(A_3)\cdot \mathrm{P}(A_4)$}
	{$\mathrm{P}(A)=\mathrm{P}(A_1)\cdot \mathrm{P}(\overline{A_2})\cdot \mathrm{P}(\overline{A_3})\cdot \mathrm{P}(A_4)$}
	\loigiai{
		Xác suất biến cố $A$ là $\mathrm{P}(A)=\mathrm{P}(\overline{A_1})\cdot \mathrm{P}(\overline{A_2})\cdot \mathrm{P}(\overline{A_3})\cdot \mathrm{P}(A_4)$.
	}
\end{ex}
\begin{ex}%[Nguyễn Kiều Nhã Tú- BG Toán 10]%[1D2B5-1]
	Một xạ thủ bắn liên tục $4$ phát đạn vào bia. Gọi $A_k$ là các biến cố \lq\lq  Xạ thủ bắn trúng lần thứ $k$\rq\rq \, với $k=1,2,3,4$. Xác suất biến cố $B$: \lq\lq  Bắn trúng bia ít nhất một lần\rq\rq \, là
	\choice
	{$\mathrm{P}(B)=\mathrm{P}(A_1 \cup A_2 \cup A_3 \cap A_4)$}
	{$\mathrm{P}(B)=\mathrm{P}(A_1 \cap A_2 \cup A_3 \cup A_4)$}
	{$\mathrm{P}(B)=\mathrm{P}(A_1 \cup A_2 \cap A_3 \cup A_4)$}
	{\True $\mathrm{P}(B)=\mathrm{P}(A_1 \cup A_2 \cup A_3 \cup A_4)$}
	\loigiai{
		Ta có: $\overline{A_k}$ là biến cố lần thứ $k$ ( $k=1,2,3,4$ ) bắn không trúng bia. \\
		Do đó: $B=A_1 \cup A_2 \cup A_3 \cup A_4$ với $i,j,k,m \in \left\{1,2,3,4\right\}$ và đôi một khác nhau.\\
		Xác suất biến cố $\mathrm{P}(B)=\mathrm{P}(A_1 \cup A_2 \cup A_3 \cup A_4)$.
	}
\end{ex}
\begin{ex}%[Nguyễn Kiều Nhã Tú- BG Toán 10]%[1D2K5-2]
	Trên giá sách có $4$ quyển sách Toán, $3$ quyển Lý, $2$ quyển sách Hóa. Lấy ngẫu nhiên ba quyển sách. Xác suất để ba quyển lấy ra có ít nhất một quyển là sách Toán bằng
	\choice
	{$\dfrac{1}{21}$} 
	{$\dfrac{2}{7}$}
	{$\dfrac{5}{42}$}
	{\True $\dfrac{37}{42}$}
	\loigiai{
		Gọi $ A $ là biến cố \lq\lq  Lấy ba quyển sách ra có ít nhất một quyển là sách Toán  \rq\rq.
		Số cách chọn $3$ quyển sách tùy ý trong $9$ quyển sách là $\mathrm{C}_{9}^3=84$.\\
		Số cách chọn $ 3 $ quyển sách mà không có sách Toán là $ \mathrm{C}_{5}^3 $.\\
		Số cách chọn $3$ quyển sách trong đó có ít nhất $1$ quyển toán là $\mathrm{C}_9^3-\mathrm{C}_{5}^3 =74$.\\
		Xác suất cần tìm là $\mathrm{P}(A)=\dfrac{37}{42}$.
	}
\end{ex}
\begin{ex}%[Nguyễn Kiều Nhã Tú- BG Toán 10]%[1D2Y5-4]
	Xếp ngẫu nhiên ba bạn An, Bình, Cường đứng trên thành một hàng dọc. Số phần tử không gian mẫu là
	\choice
	{\True $6$}
	{$4$}
	{$2$}
	{$3$}	
	\loigiai{
		Số phần tử của không gian mẫu: $n(\Omega)=3!=6$.
	}
\end{ex}
\begin{ex}%[Nguyễn Kiều Nhã Tú- BG Toán 10]%[1D2Y5-4]
	Xét phép thử $T$: \lq\lq  Gieo một con súc sắc\rq\rq\;  và biến cố $B$: \lq\lq  Số chấm trên mặt xuất hiện là một số lẻ\rq\rq. Tập hợp nào dưới đây mô tả biến cố $B$?
	\choice
	{\True $\{1;3;5\}$}
	{$\{2;3;4\}$}
	{$\{2;4;6\}$}
	{$\{2;3;6\}$}	
	\loigiai{
		Các phần tử của biến cố $B$ là $\Omega=\{1;3;5\}$.
	}
\end{ex}
\begin{ex}%[Nguyễn Kiều Nhã Tú- BG Toán 10]%[1D2Y5-4]
	Xét phép thử $T$: \lq\lq  Gieo một con súc sắc\rq\rq\
	và  biến cố $C$: \lq\lq  Số chấm xuất hiện trên mặt là số nguyên tố\rq\rq. Tập hợp nào dưới đây mô tả biến cố $C$?
	\choice
	{\True $\{2;3;5\}$}
	{$\{2;3;6\}$}
	{$\{1;3;5\}$}
	{$\{2;3;4\}$}	
	\loigiai{
		Các phần tử của biến cố $C$ là $\Omega=\{2;3;5\}$.
	}
\end{ex}
\begin{ex}%[Nguyễn Kiều Nhã Tú- BG Toán 10]%[1D2Y5-4]
	Xét phép thử $T$: \lq\lq  Gieo ba đồng xu phân biệt\rq\rq. Số phần tử của không gian mẫu đó là
	\choice
	{$6$}
	{\True $8$}
	{$5$}
	{$3$}	
	\loigiai{
		Kí hiệu $S$ - là mặt sấp, $N$ - là mặt ngửa. Khi đó không gian mẫu của phép thử là 
		$$\Omega=\left\lbrace SSS;SSN;SNS;SNN;NSS;NSN;NNS;NNN\right\rbrace .$$
		Vậy số phần tử của không gian mẫu là $8$.
	}
\end{ex}
\begin{ex}%[Nguyễn Kiều Nhã Tú- BG Toán 10]%[1D2Y4-1]
	Xét phép thử: \lq\lq  Gieo hai đồng xu phân biệt\rq\rq. Nếu kí hiệu $S$ để chỉ đồng xu xuất hiện mặt \lq\lq  sấp\rq\rq \,, kí hiệu $N$
	để chỉ đồng xu xuất hiện mặt \lq\lq  ngửa\rq\rq \, thì không gian mẫu của phép thử trên là
	\choice
	{$\Omega=\{SS;NN\}$}
	{ $\Omega=\{SN;NS\}$}
	{$\Omega=\{SS;SN;NN\}$}
	{\True $\Omega=\{SS;SN;NS;NN\}$}	
	\loigiai{
		$$\Omega=\{SS;SN;NS;NN\}.$$
	}
\end{ex}
\begin{ex}%[Nguyễn Kiều Nhã Tú- BG Toán 10]%[1D2Y4-1]
	Xét phép thử $T$: \lq\lq  Gieo một con súc sắc\rq\rq\; có không gian mẫu là $\Omega=\{1 ; 2 ; 3 ; 4 ; 5 ; 6\}$. Xét biến cố $A$: \lq\lq  Số chấm trên mặt xuất hiện là số chẵn\rq\rq.	Số phần tử thuận lợi của biến cố $A$ là
	\choice
	{$6$}
	{ $94$}
	{$5$}
	{\True $3$}	
	\loigiai{
		Các kết quả được gọi là \textit{kết quả thuận lợi cho} $A$ được mô tả bởi: $\Omega_{A}=\{2;4;6\}$ là một tập con của $\Omega \Rightarrow$ Số phần tử thuận lợi của biến cố $A$ là $n(A)=3$.
	}
\end{ex}
\begin{ex}%[Nguyễn Kiều Nhã Tú- BG Toán 10]%[1D2B5-2]
	Gieo ngẫu nhiên một con súc sắc cân đối và đồng chất. Tính xác suất biến cố $A$: \lq\lq  Xuất hiện mặt có số chấm lẻ\rq\rq.
	\choice
	{$\dfrac{1}{6}$}
	{ $\dfrac{1}{4}$}
	{$\dfrac{1}{3}$}
	{\True $\dfrac{1}{2}$}	
	\loigiai{
		Các phần tử không gian mẫu là $\Omega=\{1; 2; 3; 4; 5; 6\}\Rightarrow n(\Omega)=6$.\\
		Các phần tử của biến cố $A$ là $\{1; 3; 5\} \Rightarrow n\left( A  \right) =3$.\\
		Do đó xác suất cần tìm của biến cố $A$ là
		$$P(A)=\dfrac{n(A)}{n(\Omega)}=\dfrac{3}{6}=\dfrac{1}{2}.$$
	}
\end{ex}
\begin{ex}%[Nguyễn Kiều Nhã Tú- BG Toán 10]%[1D2Y5-4]
	Gieo ngẫu nhiên một con súc sắc cân đối và đồng chất. Tính xác suất biến cố $B$: \lq\lq  Xuất hiện mặt có số chấm chia hết cho $3$\rq\rq.
	\choice
	{$\dfrac{1}{6}$}
	{ $\dfrac{1}{4}$}
	{\True$\dfrac{1}{3}$}
	{ $\dfrac{1}{2}$}	
	\loigiai{
		Các phần tử không gian mẫu là $\Omega=\{1; 2; 3; 4; 5\}\Rightarrow n(\Omega)=6$.\\
		Các phần tử của biến cố $B$ là $\{3; 6\} \Rightarrow n\left( B \right) =2$.\\
		Do đó xác suất cần tìm của biến cố $B$ là
		$$P(B)=\dfrac{n(B)}{n(\Omega)}=\dfrac{2}{6}=\dfrac{1}{3}.$$
	}
\end{ex}
\begin{ex}%[Nguyễn Kiều Nhã Tú- BG Toán 10]%[1D2Y5-4]
	Gieo ngẫu nhiên một con súc sắc cân đối và đồng chất. Tính xác suất biến cố $C$:  \lq\lq  Mặt xuất hiện có số chấm lớn hơn $2$\rq\rq.
	\choice
	{$\dfrac{1}{6}$}
	{\True  $\dfrac{2}{3}$}
	{$\dfrac{1}{3}$}
	{$\dfrac{1}{2}$}	
	\loigiai{
		Các phần tử không gian mẫu là $\Omega=\{1; 2; 3; 4; 5; 6\}\Rightarrow n(\Omega)=6$.\\
		Các phần tử của biến cố $C$ là $\{3; 4; 5; 6\} \Rightarrow n\left( C \right) =4$.\\
		Do đó xác suất cần tìm của biến cố $C$ là
		$$P(C)=\dfrac{n(C)}{n(\Omega)}=\dfrac{4}{6}=\dfrac{2}{3}.$$
	}
\end{ex}
\begin{ex}%[Nguyễn Kiều Nhã Tú- BG Toán 10]%	[1D2B5-2]
	Từ hộp chứa $4$ quả cầu trắng, $6$ quả cầu xanh kích thước và khối lượng như nhau. Lấy ngẫu nhiên $3$ quả cầu. Tính xác suất để $3$ quả cầu lấy được có đúng $1$ màu.
	\choice
	{\True $\dfrac{1}{5}$}
	{ $\dfrac{1}{4}$}
	{$\dfrac{1}{2}$}
	{$\dfrac{2}{5}$}	
	\loigiai{
		Chọn $3$ quả cầu trong $10$ quả cầu, suy ra số phần tử không gian mẫu là $n(\Omega)=\mathrm{C}_{10}^3=120$.\\
		Gọi $A$ là biến cố: \lq\lq  ba quả cầu lấy được cùng màu \rq\rq.\\
		TH1:  Chọn $3$ quả cầu màu trắng có $\mathrm{C}_4^3$ cách.\\
		TH2:  Chọn $3$ quả cầu màu xanh có $\mathrm{C}_6^3$ cách.\\
		Theo quy tắc cộng $\Rightarrow n(A)=\mathrm{C}_4^3+\mathrm{C}_6^3=24$.\\
		Do đó xác suất cần tìm của biến cố $A$ là
		$$P(A)=\dfrac{n(A)}{n(\Omega)}=\dfrac{24}{120}=\dfrac{1}{5}.$$
	}
\end{ex}
\begin{ex}%[Nguyễn Kiều Nhã Tú- BG Toán 10]%[1D2B5-2]
	Từ hộp chứa $5$ quả cầu trắng, $4$ quả cầu xanh có kích thước và khối lượng như nhau. Lấy ngẫu nhiên $3$ quả cầu. Tính xác suất để $3$ quả cầu lấy được có đúng $1$ màu.
	\choice
	{\True $\dfrac{1}{6}$}
	{ $\dfrac{1}{4}$}
	{$\dfrac{1}{2}$}
	{$\dfrac{2}{5}$}	
	\loigiai{
		Chọn $3$ quả cầu trong $9$ quả cầu, suy ra số phần tử không gian mẫu là $n(\Omega)=\mathrm{C}_{9}^3=84$.\\
		Gọi $A$ là biến cố: \lq\lq  ba quả lấy được có đúng một màu \rq\rq.\\
			TH1: Chọn $3$ quả cầu màu trắng có $\mathrm{C}_5^3$ cách.\\
			TH2: Chọn $3$ quả cầu màu xanh có $\mathrm{C}_4^3$ cách.\\
		Theo quy tắc cộng $\Rightarrow n(A)=\mathrm{C}_5^3+\mathrm{C}_4^3=14$.\\
		Do đó xác suất cần tìm của biến cố $A$ là
		$$P(A)=\dfrac{n(A)}{n(\Omega)}=\dfrac{14}{84}=\dfrac{1}{6}.$$
	}
\end{ex}
\begin{ex}%[Nguyễn Kiều Nhã Tú- BG Toán 10]%[1D2B5-2]
	Một hộp đựng $5$ bi đỏ, $6$ bi xanh và $7$ bi trắng có khối lượng và kích thước như nhau. Chọn ngẫu nhiên $6$ viên bi từ hộp đó. Tính xác suất để $6$ bi được chọn có cùng màu.
	\choice
	{\True $\dfrac{2}{4641}$}
	{ $\dfrac{1}{4641}$}
	{$\dfrac{1}{1547}$}
	{$\dfrac{2}{1547}$}	
	\loigiai{
		Chọn $6$ viên bi trong $18$ viên bi, suy ra số phần tử không gian mẫu là $n(\Omega)=\mathrm{C}_{18}^6=18564$.\\
		Gọi $A$ là biến cố: \lq\lq  $6$ bi được chọn có cùng màu \rq\rq.\\
	TH1:		 Chọn $6$ viên bi màu xanh có $\mathrm{C}_6^6=1$ cách.\\
	TH2:  Chọn $6$ viên bi màu trắng có $\mathrm{C}_7^6=7$ cách.\\
		Theo quy tắc cộng $\Rightarrow n(A)=1+7=8$ cách.\\
		Do đó xác suất cần tìm của biến cố $A$ là
		$$P(A)=\dfrac{n(A)}{n(\Omega)}=\dfrac{8}{18564}=\dfrac{2}{4641}.$$
	}
\end{ex}
\begin{ex}%[Nguyễn Kiều Nhã Tú- BG Toán 10]%[1D2B5-2]
	Có $7$ học sinh nam và $5$ học sinh nữ tập trung ngẫu nhiên theo một hàng dọc. Tính xác suất để người đứng ở đầu hàng và cuối hàng đều là học sinh nam.
	\choice
	{\True $\dfrac{7}{22}$}
	{ $\dfrac{7}{11}$}
	{$\dfrac{4}{11}$}
	{$\dfrac{3}{11}$}	
	\loigiai{
		Ta có $n(\Omega)=12!$.\\
		Gọi A là biến cố : \lq\lq  người đứng ở đầu hàng và cuối hàng đều là học sinh nam\rq\rq.\\
		Chọn hai bạn nam đứng đầu và cuối hàng có $\mathrm{A}^2_7$ cách.\\
		Hoán vị $10$ bạn còn lại ở giữa hàng có $10!$ cách.\\
		Suy ra $n(A)=\mathrm{A}^2_7\cdot 10!$.\\
		Vậy $P(A)=\dfrac{\mathrm{A}^2_7\cdot 10!}{12!}=\dfrac{7}{22}$.	
	}
\end{ex}
\begin{ex}%[Nguyễn Kiều Nhã Tú- BG Toán 10]%[1D2K5-2]
	Có $6$ học sinh lớp $11$ và $3$ học sinh lớp $12$ xếp ngẫu nhiên vào $9$ ghế thành một dãy. Tính xác suất để xếp được $3$ học sinh lớp $12$ xen kẽ giữa $6$ học sinh lớp $11$.
	\choice
	{$\dfrac{1}{2}$}
	{\True  $\dfrac{5}{42}$}
	{$\dfrac{5}{21}$}
	{$\dfrac{1}{4}$}	
	\loigiai{
		Ta có $n(\Omega)=9!$.\\
		Biến cố A: \lq\lq  $3$ học sinh lớp $12$ xen kẽ giữa $6$ học sinh lớp $11$\rq\rq.\\
		Xếp $6$ học sinh lớp $11$ thành hàng và sắp thứ tự có $6!$ cách, giữa	$6$ học sinh lớp $11$ tạo thành $5$ khoảng trống.\\
		Chọn $3$ khoảng trống cho học sinh lớp $12$ và sắp thứ tự có $\mathrm{A}^3_5$ cách.\\
		Vậy $P(A)=\dfrac{6!\cdot \mathrm{A}^3_5}{9!}=\dfrac{5}{42}$.
	}
\end{ex}
\begin{ex}%[Nguyễn Kiều Nhã Tú- BG Toán 10]%[1D2K5-5]
	Trong kì thi thử THPT Quốc Gia, An làm để thi trắc nghiệm môn Toán. Đề thi gồm $50$ câu hỏi, mỗi câu có $4$ phương án trả lời, trong đó chỉ có một phương án đúng; trả lời đúng mỗi câu được $0{,}2$ điểm. An trả lời hết các câu hỏi và chắc chắn đúng $45$ câu, $5$ câu còn lại An chọn ngẫu nhiên. Tính xác suất để điểm thi của An không dưới $9,5$ điểm.
	\choice
	{$\dfrac{43}{512}$}
	{\True  $\dfrac{45}{512}$}
	{$\dfrac{43}{128}$}
	{$\dfrac{45}{128}$}	
	\loigiai{
		Gọi $A$ là biến cố An chọn đúng đáp án ở một câu.\\
		$\overline{A}$ là biến cố An chọn sai đáp án ở nột câu.\\
		$B$ là biến cố điểm thi của An không dưới $9,5$ điểm.\\
		Khi chọn ngẫu nhiên $1$ trong $4$ đáp án thì\\
		Xác suất trả lời đúng là $\mathrm P(A) = \dfrac{1}{4}$ và xác suất trả lời sai là $\mathrm P(\overline{A}) = \dfrac{3}{4}$.\\
		Mỗi câu trả lời đúng được $0{,}2$ điểm. Để điểm thi của An không dưới $9,5$ điểm thì An cần trả lời đúng $48$ câu. Trong đó, chắc chắn đúng $45$ câu nên ta cần đúng thêm $3$ câu nữa.\\
		Khi đó $\mathrm P(B) = \left( \dfrac{1}{4} \right)^3 \cdot \left( \dfrac{3}{4} \right)^2 \cdot \mathrm{C}_5^3=\dfrac{45}{512}$.
	}
	
\end{ex}



\noindent\textbf{II. PHẦN TỰ LUẬN}
\begin{bt}%[Nguyễn Kiều Nhã Tú- BG Toán 10]%[1D2B5-2]
	Một chiếc hộp đựng $6$ quả cầu trắng, $4$ quả cầu đỏ và $2$ quả cầu đen, có kích thước và khối lượng như nhau. Chọn ngẫu nhiên $6$ quả cầu. Tính xác suất để chọn được $3$ quả trắng, $2$ quả đỏ và $1$ quả đen.
	\loigiai
	{
		$n(\Omega)=\mathrm{C}_{12}^{6}=924$. Gọi $E$ là biến cố: "Chọn được $3$ quả trắng, $2$ quả đỏ và $1$ quả đen.\\
		Chọn $3$ quả cầu trắng từ $6$ quả cầu trắng, có $\mathrm{C}_6^3=20$ cách chọn;\\
		Chọn $2$ quả cầu đỏ từ $4$ quả cầu đỏ, có $\mathrm{C}^2_4=6$ cách chọn;\\
		Chọn $1$ quả cầu đen từ $2$ quả cầu đen, có $2$ cách chọn.\\
		Vậy $n(E)=20\cdot6\cdot 2=240$. Do đó $\mathrm{P}(E)=\dfrac{240}{924}=\dfrac{20}{77}$.
	}	
\end{bt}
\begin{bt}%[Nguyễn Kiều Nhã Tú- BG Toán 10]%[1D2B5-2]
	Từ hộp chứa $4$ quả cầu trắng, $6$ quả cầu xanh kích thước và khối lượng như nhau. Lấy ngẫu nhiên $3$ quả cầu. Tính xác suất để $3$ quả cầu lấy được có đúng $1$ màu.
	\loigiai{
		Chọn $3$ quả cầu trong $10$ quả cầu, suy ra số phần tử không gian mẫu là $n(\Omega)=\mathrm{C}_{10}^3=120$.\\
		Gọi $A$ là biến cố: \lq\lq  ba quả cầu lấy có cùng màu \rq\rq.\\
		\textbf{Trường hợp 1.}	 Chọn $3$ quả cầu màu trắng có $\mathrm{C}_4^3$ cách.\\
		\textbf{Trường hợp 2.}	Chọn $3$ quả cầu màu xanh có $\mathrm{C}_6^3$ cách.
	
		Theo quy tắc cộng $\Rightarrow n(A)=\mathrm{C}_4^3+\mathrm{C}_6^3=24$.\\
		Do đó xác suất cần tìm của biến cố $A$ là
		$$P(A)=\dfrac{n(A)}{n(\Omega)}=\dfrac{24}{120}=\dfrac 15.$$
	}
\end{bt}
\begin{bt}%[Nguyễn Kiều Nhã Tú- BG Toán 10]%[1D2K5-5]
	Một máy bay có $5$ động cơ gồm $3$ động cơ bên cánh phải và $2$ động cơ bên cánh trái. Mỗi động cơ bên cánh phải có xác suất bị hỏng là $0{,}09$, mỗi động cơ bên cánh trái có xác suất bị hỏng là $0{,}04$. Các động cơ hoạt động độc lập với nhau. Máy bay chỉ thực hiện được chuyến bay an toàn nếu ít nhất $2$ động cơ làm việc. Tìm xác suất để máy bay thực hiện được chuyến bay an toàn.
	\loigiai{
		Gọi $A_1$, $A_2$ và $A_3$ lần lượt là biến cố \lq\lq  mỗi động cơ bên phải hoạt động\rq\rq.\\
		\hspace*{0.6cm} $B_1$, $B_2$ lần lượt là biến cố \lq\lq   mỗi động cơ bên trái hoạt động\rq\rq.\\  
		\hspace*{0.6cm} $H$ là biến cố \lq\lq  có ít nhất hai động cơ làm viêc\rq\rq.\\
		Suy ra, $\overline{H}$ là biến cố có nhiều nhất một động cơ hoạt động.\\
		\textbf{Trường hợp 1.} Không có động cơ nào hoạt động: $\overline{A_1A_2A_3B_1B_2}$.\\
		\textbf{Trường hợp 2.} Có một động cơ bên phải hoạt động: $\left(\overline{A_1A_2}A_3 \cup A_1\overline{A_2A_3} \cup A_2\overline{A_3A_1}\right)\overline{B_1B_2}$.\\
		\textbf{Trường hợp 3.} Có một động cơ bên trái hoạt động: $\overline{A_1A_2A_3}\left(\overline{B_1}B_2 \cup B_1\overline{B_2}\right)$.\\
		Suy ra $\mathrm{P}(\overline{H})=0{,}09^3 \cdot 0{,}04^2+(0{,}09^2 \cdot (1-0{,}09)\cdot 3 \cdot 0{,}04^2)+0{,}09^3 \cdot 0{,}04 \cdot 0{,}05 \cdot 2=0{,}00000394632$.\\
		Vậy $\mathrm{P}(H)=1-\mathrm{P}(\overline{H})=0{,}9999605368$.
	}
\end{bt}
\Closesolutionfile{ans}
\Closesolutionfile{ansbook}
% \indapan{10}{ans/ans-KT-901}