\setcounter{bt}{0}
\setcounter{vd}{0}
\setcounter{section}{1}
\section{Các số đặc trưng đo xu thế trung tâm}
\subsection{Tóm tắt lí thuyết}
\subsubsection{Số trung bình cộng (số trung bình)}
	\begin{boxdn}
		Số trung bình cộng của mẫu số liệu $x_1,x_2,\ldots,x_n$, kí hiệu là $\overline{x}$, được tính bằng công thức
	\begin{align*}
	\overline{x} & =\dfrac{x_1+x_2+\cdots +x_n}{n}\\
	& =\dfrac{n_1x_1 + n_2x_2 + \ldots + n_kx_k}{n}\\
	& =f_1x_1+f_2x_2+\cdots +f_kx_k		
	\end{align*}
	\end{boxdn}
\subsubsection{Trung vị}
\begin{boxdn}
	Sắp thứ tự mẫu số liệu gồm $n$ số liệu thành một dãy không giảm (hoặc không tăng).
	\begin{itemize}
		\item Nếu $n$ là lẻ thì số liệu đứng ở vị trí thứ $\dfrac{n+1}{2}$ (số đứng chính giữa MSL) gọi là trung vị.
		\item Nếu $n$ là chẵn thì số trung bình cộng của hai số liệu đứng ở vị trí thứ $\dfrac{n}{2}$ và $\dfrac{n}{2}+1$ gọi là trung vị.
	\end{itemize}
	Trung vị kí hiệu là $M_e$.
\end{boxdn}
\textbf{Ý nghĩa}
\begin{itemize}
    \item Trung vị là giá trị chia đôi trong mẫu số liệu theo số giá trị. Trung vị không bị ảnh hưởng bởi giá trị bất thường trong khi đó số trung bình cộng bị ảnh hưởng bởi giá trị bất thường.
    \item Nếu những số liệu trong mẫu có sự chênh lệch lớn thì ta nên chọn thêm trung vị làm đại diện cho mẫu số liệu đó nhằm điều chỉnh một số hạn chế khi sử dụng số trung bình cộng. Những kết luận về đối tượng thống kê rút ra khi đó sẽ tin cậy hơn.
\end{itemize}

\subsubsection{Tứ phân vị}
\begin{boxdn}
    Sắp thứ tự mẫu số liệu gồm $n$ số liệu thành một dãy không giảm.\\
    Tứ phân vị của mẫu số liệu trên là bộ ba giá trị tứ phân vị thứ nhất, tứ phân vị thé hai và tứ phân vị thứ ba, ba giá trị này chia mẫu số liệu thành bốn phần có số phần tử bằng nhau.
    \begin{itemize}
        \item Tứ phân vị $Q_2$ bằng trung vị.
        \item Nếu $n$ là chẵn thì tứ phân vị thứ nhất $Q_1$ bằng trung vị của nửa dãy phía dưới và tứ phân vị thứ ba $Q_3$ bằng trung vị của nửa dãy phía trên.
        \item Nếu $n$ là lẻ thì tứ phân vị thứ nhất $Q_1$ bằng trung vị của nửa dãy phía dưới (không bao gồm $Q_2$) và tứ phân vị thứ ba $Q_3$ bằng trung vị của nửa dãy phía trên (không bao gồm $Q_2$).
    \end{itemize}
\end{boxdn}
Các điểm $Q_1$, $Q_2$, $Q_3$ chia dãy dữ liệu đã sắp xếp theo thứ tự từ nhỏ đến lớn thành $4$ phần, mỗi phần đều chứa $\dfrac14=25\%$ số giá trị.
    % \begin{center}
    % 	\begin{tikzpicture}[scale=0.9, font=\footnotesize, line join=round, line cap=round, >=stealth]
    % 		\path (0,0) coordinate(A) ++(0:2) coordinate(Q_1) ++(0:3.5) coordinate(Q_2) ++(0:2.5) coordinate(Q_3) ++(0:1.5) coordinate(B);
    % 		\draw (A)node[text width=1.5cm,align=center,below=5pt] {Giá trị\\nhỏ nhất} -- (Q_1) node[below=5pt]{$Q_1$} -- (Q_2) node[below=5pt]{$Q_2$} -- (Q_3) node[below=5pt]{$Q_3$} -- (B)node[below=5pt, text width=1.5 cm, align=center]{Giá trị\\lớn nhất};
    % 		\foreach \i in {A,Q_1,Q_2,Q_3,B} \draw (\i)++(-90:0.07) -- ++(90:0.14);
    % 		\draw[decoration={brace}, decorate] ($(A)+(0,0.2)$)--($(Q_1)+(0,0.2)$) node[sloped,midway,above]{$25\%$};
    % 		\draw[decoration={brace,mirror}, decorate,yscale=-1] ($(Q_1)+(0,0.2)$)--($(Q_2)+(0,0.2)$) node[sloped,midway,below]{$25\%$};
    % 		\draw[decoration={brace,mirror}, decorate,yscale=-1] ($(Q_2)+(0,0.2)$)--($(Q_3)+(0,0.2)$) node[sloped,midway,below]{$25\%$};	
    % 		\draw[decoration={brace}, decorate] ($(Q_3)+(0,0.2)$)--($(B)+(0,0.2)$) node[sloped,midway,above]{$25\%$};
    % 	\end{tikzpicture}
    % \end{center}

\textbf{Ý nghĩa}
Trong thực tế, có những mẫu số liệu mà nhiều số liệu trong mẫu đó vẫn còn sự chênh lệch lớn so với trung vị. Ta nên chọn thêm những số khác cùng làm đại diện cho mẫu đó. Bằng cách lấy thêm trung vị của từng dãy số liệu tách ra bởi trung vị của mẫu nói trên, ta nhận được tứ phân vị đại diện cho mẫu số liệu đó.
    % \item Bộ ba giá trị $Q_1$, $Q_2$, $Q_3$ trong tứ phân vị phản ánh độ phân tán của mẫu số liệu. Nhưng mỗi giá trị $Q_1$, $Q_2$, $Q_3$ lại đo xu thế trung tâm của phần số liệu tương ứng của mẫu đó.

\subsubsection{Mốt}
    \begin{boxdn}
		\textbf{Mốt} của một mẫu số liệu là giá trị xuất hiện nhiều nhất (tần số lớn nhất) trong mẫu số liệu đó.
	\end{boxdn}
\begin{note}
	Một mẫu số liệu có thể có một hoặc nhiều mốt
\end{note}
% \textbf{Ý nghĩa}
% \begin{itemize}
%     \item Mốt của một mẫu số liệu đặc trưng cho số lần lặp đi lặp lại nhiều nhất tại một vị trí của mẫu số liệu đó. Dựa vào mốt, ta có thể đưa ra những kết luận (có ích) về đối tượng thống kê.
%     \item Có thể dùng mốt để đo xu thế trung tâm của mẫu số liệu khi mẫu số liệu có nhiều giá trị trùng nhau.
% \end{itemize}

%%%%%%%Thầy Nguyễn Cảnh Dũng
\setcounter{dang}{0}
\subsection{Các dạng toán}
\setcounter{subsubsection}{0}
\begin{dang}{Số trung bình}
	
\end{dang}
\begin{vd}%[Nguyễn Cảnh Dũng, dự án SGK10]
	Trong một cuộc thi tìm hiểu lịch sử địa phương (thang điểm 10), một lớp học tham gia cuộc thi và đạt được số điểm như sau:
	\begin{center}
		\begin{tabular}{|c|c|c|c|c|}
			\hline
			Số học sinh & 5 & 12 & 10 & 3\\
			\hline
			Số điểm & 5 & 6 & 7 & 9\\
			\hline
		\end{tabular}
	\end{center}
	Hỏi trung bình mỗi học sinh trong lớp đạt bao nhiêu điểm trong cuộc thi?
	\loigiai{
		\begin{itemize}
			\item Số học sinh của lớp là: $5+12+10+3=30$ học sinh.
			\item Điểm trung bình mà mỗi bạn đạt được trong cuộc thi là: \\[.5em]
			\centerline{$\dfrac{5\cdot5 + 12\cdot6 + 10\cdot7 + 3\cdot9}{30}\approx 6.47$ điểm.}
		\end{itemize}
	}
\end{vd}
\begin{vd}%[Nguyễn Cảnh Dũng, dự án SGK10]
	Khi nghiên cứu tuổi thọ của một loại bóng đèn, người ta đã chọn tùy ý 10 bóng đèn trong một lô hàng và bật sáng liên tục cho đến khi nó tự tắt. Tuổi thọ của các bóng đèn (tính theo giờ) của các bóng đèn được ghi lại trong bảng sau:
	\begin{center}
		\begin{tabular}{|c|c|c|c|c|}
			\hline
			Số bóng đèn & 2 & 3 & 4 & 1\\
			\hline
			Tuổi thọ (giờ) & 1150 & 1160 & 1170 & 1180\\
			\hline
		\end{tabular}
	\end{center}
	Hỏi tuổi thọ trung bình của các bóng đèn trong lô hàng là bao nhiêu?
	\loigiai{
		Tuổi thọ trung bình của các bóng đèn trong lô hàng là: \\[.5em]
		\centerline{$\dfrac{2\cdot1150 + 3\cdot1160 + 4\cdot1170 + 1\cdot1180}{10}=1164$ giờ.}
	}
\end{vd}
\begin{vd}%[Nguyễn Cảnh Dũng, dự án SGK10]
	Trong đợt kiểm tra quân sự thường niên tại một đơn vị, ở bộ môn bắn súng AK mỗi người phải bắn 5 phát súng vào bia. Thang điểm bắn là: $0, 4,5,6,7,8,9,10$. Ở 4 lần bắn trước đó, anh Nam đã đạt được số điểm như sau:
	\begin{center}
	\begin{tabular}{|c|c|c|c|c|c|}
			\hline
			Lần bắn & 1 & 2 & 3 & 4 & 5\\
			\hline
			Số điểm & 8 & 7 & 0 & 9 &10 \\
			\hline
	\end{tabular}
	\end{center}
	Biết rằng để vượt qua bài kiểm tra, mỗi người phải đạt điểm trung bình trong các lần bắn từ $6.5$ điểm trở lên. Tính số điểm ít nhất mà anh Nam cần đạt được trong lần bắn thứ 5 để vượt qua bài kiểm tra.
	\loigiai{
		\begin{itemize}
			\item Gọi điểm số trong lần bắn thứ 5 của anh Nam là $n$ điểm, $n \in \{0,4,5,6,7,8,9,10\}$.\\
			Khi đó điểm số trung bình mà anh Nam đạt được trong 5 lần bắn là:\\[.5em]
			\centerline{$\dfrac{8+7+0+9+n}{5}$}.
			\item Theo yêu cầu đề bài, ta có: $\dfrac{8+7+0+9+n}{5} \ge 6.5 \Leftrightarrow n \ge 8.5$.\\
			Vì $n \in \{0,4,5,6,7,8,9,10\}$ nên số điểm ít nhất mà anh Nam cần đạt được trong lần bắn thứ 5 để vượt qua bài điểm tra là $9$ điểm.
		\end{itemize}
	}
\end{vd}
\begin{dang}{Số trung vị}
	
\end{dang}
\begin{vd}%[Nguyễn Cảnh Dũng, dự án SGK10]
	Một nhóm gồm 7 học sinh tham gia một cuộc thi và đạt được số điểm như sau: $89$, $69$, $65$, $0$, $80$, $0$, $90$.
	Hãy tìm trung vị của mẫu số liệu trên.
	\loigiai{
		\begin{itemize}
			\item Sắp xếp dãy số liệu theo thứ tự không giảm: $0$, $0$, $65$, $69$, $80$, $89$, $90$.
			\item Dãy trên có 7 giá trị, giá trị chính giữa là $69$. Vậy trung vị của dãy số liệu trên bằng $69$.
		\end{itemize}
	}
\end{vd}
\begin{vd}%[Nguyễn Cảnh Dũng, dự án SGK10]
	Số áo bán được trong một cửa hàng trong một quý được ghi lại trong bảng sau:
	\begin{center}
		\begin{tabular}{|c|c|c|c|c|c|c|c|}
			\hline
			Cỡ số & 36 & 37 & 38 & 39 & 40 & 41 & 42\\
			\hline
			Số áo bán được & 13 & 45 & 126 & 110 & 126 & 40 & 5\\
			\hline
		\end{tabular}
	\end{center}
	Hãy tìm trung vị của mẫu số liệu trên.
	\loigiai{
		\begin{itemize}
			\item Số giá trị của dãy số liệu trên là: $13+45+126+110+126+40+5=465$.
			\item Sắp xếp dãy số liệu theo thứ tự không giảm ta được:\\
			 $36$, $\ldots$, $36$, $37$, $\ldots$, $37$, $38$, $\ldots$, $38$, $39$, $\ldots$, $39$, $40$, $\ldots$, $40$, $41$, $\ldots$, $41$, $42$, $\ldots$, $42$,
			\item Dãy trên có $465$ giá trị, giá trị chính giữa là giá trị thứ $233$, giá trị này bằng $39$. Vậy trung vị của dãy số liệu trên bằng $39$.
		\end{itemize}
	}
\end{vd}
\begin{dang}{Tứ phân vị}
	
\end{dang}
\begin{vd}%[Nguyễn Cảnh Dũng, dự án SGK10]
	Số tấn hàng bán ra được trong 6 tháng đầu năm của một công ty được cho như sau: $4$, $7$, $9$, $11$, $12$, $20$. Tìm tứ phân vị dưới của mẫu số liệu trên.
	\loigiai{
		\begin{itemize}
			\item Sắp xếp mẫu số liệu trên theo thứ tự không giảm: $4$, $7$, $9$, $11$, $12$, $20$.
			\item Mẫu số liệu trên có 6 giá trị, nên trung vị là trung bình cộng của hai số hạng chính giữa: $Q_2=\dfrac{9+11}{2}=10$.
			\item Như vậy, nửa số liệu bên trái $Q_2$ là: $4$, $7$, $9$. Mẫu số liệu này có 3 giá trị, nên trung vị là số hạng chính giữa, vậy $Q_1=7$.
			\item Vậy tứ phân vị dưới của mẫu số liệu đã cho là $Q_1=7$.
		\end{itemize}
	}
\end{vd}
\begin{vd}%[Nguyễn Cảnh Dũng, dự án SGK10]
	Số buổi nghỉ học trong một năm của một nhóm học sinh được cho như sau: $5$, $8$, $10$, $11$, $15$, $18$, $23$. Tìm tứ phân vị trên của mẫu số liệu đã cho.
	\loigiai{
		\begin{itemize}
			\item Sắp xếp mẫu số liệu trên theo thứ tự không giảm: $5$, $8$, $10$, $11$, $15$, $18$, $23$.
			\item Mẫu số liệu trên có 7 giá trị, nên trung vị là số hạng chính giữa: $Q_2=11$.
			\item Nửa số liệu bên phải $Q_2$ là: $15$, $18$, $23$. Mẫu số liệu này có 3 giá trị, nên trung vị là số hạng chính giữa, vậy $Q_3=18$.
			\item Vậy tứ phân vị trên của mẫu số liệu đã cho là $Q_3=18$.
		\end{itemize}
	}
\end{vd}
\begin{dang}{Mốt}
\end{dang}
\begin{vd}%[Nguyễn Cảnh Dũng, dự án SGK10]%
	Giá thành của một sản phẩm (tính theo đơn vị nghìn đồng) của $20$ cơ sở sản xuất được cho bởi bảng sau:
	\begin{center}
		\begin{tabular}{|c|c|c|c|c|c|c|c|c|c|}
			\hline
			15 & 25 & 25 & 30 & 20 & 25 & 35 & 30 & 25 & 30\\
			\hline
			25 & 20 & 35 & 30 & 15 & 25 & 25 & 20 & 25 & 25\\
			\hline
		\end{tabular}
	\end{center}
	Tìm mốt của mẫu số liệu trên.
	\loigiai{
		\begin{itemize}
			\item Lập bảng tần số (số lần xuất hiện của giá trị) cho mẫu số liệu đã cho, ta được
			\begin{center}
				\begin{tabular}{|c|c|c|c|c|c|}
					\hline
					Giá trị & 15 & 20 & 25 & 30 & 35\\
					\hline
					Tần số & 2 & 3 & 9 & 4 & 2\\
					\hline
				\end{tabular}
			\end{center}
			\item Dựa vào bảng tần số, ta thấy giá trị $25$ xuất hiện nhiều nhất ($9$ lần) nên mốt của dấu hiệu là $25$.
		\end{itemize}
	}
\end{vd}
\begin{vd}%[Nguyễn Cảnh Dũng, dự án SGK10]
	Số cân nặng của $20$ học sinh được ghi lại như sau:
	\begin{center}
		\begin{tabular}{|c|c|c|c|c|c|c|c|c|c|}
			\hline
			28 & 35 & 29 & 37 & 30 & 35 & 37 & 30 & 35 & 29\\
			\hline
			30 & 37 & 35 & 35 & 42 & 28 & 35 & 29 & 37 & 20\\
			\hline
		\end{tabular}
	\end{center}
	Tìm mốt của mẫu số liệu trên.
	\loigiai{
		\begin{itemize}
			\item Lập bảng tần số (số lần xuất hiện của giá trị) cho mẫu số liệu đã cho, ta được
			\begin{center}
				\begin{tabular}{|c|c|c|c|c|c|c|c|}
					\hline
					Giá trị & 20 & 28 & 29 & 30 & 35 & 37 & 42\\
					\hline
					Tần số & 1 & 2 & 3 & 3 & 6 & 4 & 1\\
					\hline
				\end{tabular}
			\end{center}
			\item Dựa vào bảng tần số, ta thấy giá trị $35$ xuất hiện nhiều nhất ($6$ lần) nên mốt của dấu hiệu là $35$.
		\end{itemize}
	}
\end{vd}
%%Thầy PT Sinh
\subsection{Bài tập tự luận}
\begin{bt}%[PT Sinh, dự án SGK10]%[BG10-2022]
	Tìm số trung bình, trung vị, mốt và tứ phân vị của mỗi mẫu số liệu sau đây:
	\begin{listEX}[1]
		\item Số điểm mà năm vận động viên bóng rổ ghi được trong một trận đấu"
		\[9\quad 8\quad 15\quad 8\quad20\]\vspace{-10mm}
		\item Giá của một số loại giày (đơn vị nghìn đồng):
		\[350\quad 300\quad 650\quad 300\quad 450\quad 500\quad 300\quad 250\]\vspace{-10mm}
		\item Số kênh được chiếu của một số hãng truyền hình cáp:
		\[36\quad 38\quad 33\quad 34\quad 32\quad 30\quad 34\quad 35\]\vspace{-10mm}
	\end{listEX}
	\loigiai{	
		\begin{listEX}[1]
			\item Sắp xếp lại mẫu số liệu: $8\quad 8\quad 9\quad 15\quad20$.\\
			Số trung bình: $\overline{x}=\dfrac{8+8+9+15+20}{5}=12$.\\
			Trung vị: $Q=9$.\\
			Mốt: $M_O=8$.\\
			Tứ phân vị: $Q_1=\dfrac{8+8}{2}=8$, $Q_2=9$, $Q_1=\dfrac{15+20}{2}=17{,}5$.
			\item Sắp xếp lại mẫu số liệu: $250\quad 300\quad 300\quad 300\quad 350\quad 450\quad 500\quad 650$.\\
			Số trung bình: $\overline{x}=\dfrac{250+300+300+300+350+450+500+650}{8}=387{,}5$.\\
			Trung vị: $Q=\dfrac{300+350}{2}=325$.\\
			Mốt: $M_O=300$.\\
			Tứ phân vị: $Q_1=300$, $Q_2=325$, $Q_1=\dfrac{450+500}{2}=475$.
			\item Sắp xếp lại mẫu số liệu: $30\quad 32\quad 33\quad 34\quad 34\quad 35\quad 36\quad 38$.\\
			Số trung bình: $\overline{x}=\dfrac{30+32+33+34+34+35+36+38}{8}=34$.\\
			Trung vị: $Q=34$.\\
			Mốt: $M_O=34$.\\
			Tứ phân vị: $Q_1=\dfrac{32+33}{2}=32{,}5$, $Q_2=34$, $Q_1=\dfrac{35+36}{2}=35{,}5$.
		\end{listEX}
	}
\end{bt}

\begin{bt}%[PT Sinh, dự án SGK10]%[BG10-2022]
	Hãy chọn số đặc trưng đo xu thế trung tâm của mỗi mẫu số liệu sau. Giải thích và tính giá trị của số đặc trưng đó.
	\begin{enumerate}
	\item Số mặt trăng đã biết của các hành tinh:\\
	% \begin{center}
		\begin{tikzpicture}
			\matrix[matrix of nodes,nodes in empty cells,
			row sep=-\pgflinewidth,column sep=-\pgflinewidth,
			nodes={minimum height=8mm,minimum width=12mm,draw,anchor=center},
			column 1/.style={nodes={minimum width=36mm,font=\bf}},
			row 1/.style={nodes={minimum height=20mm}},
			]{
				Hành tinh & \node[align=center]{Thủy\\ tinh}; &\node[align=center]{Kim\\ tinh};
				&\node[align=center]{Trái\\ đất};&\node[align=center]{Hỏa\\ tinh};
				&\node[align=center]{Mộc\\ tinh};&\node[align=center]{Thổ\\ tinh};
				&\node[align=center]{Thiên\\vương\\ tinh};&\node[align=center]{Hải\\vương\\ tinh};\\ 
				Số mặt trăng &0&0&1&2&63&34&27&13\\
			};
		\end{tikzpicture}
	% \end{center}
	(Theo \textit{NASA})
	\item Số đường chuyền thành công trong một trận đấu của một cầu thủ bóng đá:
	\[32\quad 24\quad 20\quad 14\quad 23.\]\vspace{-10mm}
	\item Chỉ số IQ của một nhóm học sinh: $60\quad 72\quad 63\quad 83\quad 68\quad 74\quad 90\quad 86\quad 74\quad 80$.
	\item Các sai số trong một phép đo: $10\quad 15\quad 18\quad 15\quad 14\quad 13\quad 42\quad 15\quad 12\quad 14\quad 42$.
	\end{enumerate}
	\loigiai{	
		\begin{enumerate}
			\item Sắp xếp lại số liệu: $0\quad 0\quad 1\quad 2\quad 13\quad 27\quad 34\quad 63$.\\
			Trung vị là $Q=\dfrac{2+13}{2}=7{,}5$.\\
			Ta không chọn số trung bình vì số trung bình là $17{,}5$ chênh lệch với $63$ lớn. Mốt cũng thế.
			\item Các số liệu bài cho không chênh lệch quá lớn với số trung bình nên ta chọn số trung bình.\\
			Số đường chuyền trung bình là: $\overline{x}=\dfrac{32+24+20+14+23}{5}=22{,}6$.
			\item Các số liệu bài cho không chênh lệch quá lớn với số trung bình nên ta chọn số trung bình.\\
			IQ trung bình là $\overline{x}=\dfrac{60+72+63+83+68+74\cdot 2+90+86+80}{10}=75$.
			\item Ta thấy $42$ chênh lệch lớn với các số còn lại nên ta chọn Mốt để đo xu thế trung tâm.\\
			Mốt là $M_O=15$ (tần số là $3$).
		\end{enumerate}
	}
\end{bt}

\begin{bt}%[PT Sinh, dự án SGK10]%[BG10-2022]
	Một bác sĩ mắt ghi lại tuổi của $30$ bệnh nhân mắc bệnh đau mắt hột. Kết quả thu được mẫu số liệu như sau
	\begin{center}
		\begin{tabular}{ccccccccccccccc}
			21    & 17    & 22    & 18    & 20    & 17    & 15    & 13    & 15    & 20 	  & 15    & 12    & 18    & 17    & 25 \\
			17    & 21    & 15    & 12    & 18    & 16    & 23    & 14    & 18    & 19    & 13    & 16    & 19    & 18    & 17 \\
			
		\end{tabular}
	\end{center}	
	Tính mốt $M_O$ của bảng số liệu đã cho. 
	\loigiai{
		Từ bảng số liệu trên ta suy ra bảng phân bố tần số tuổi của 30 bệnh nhân đau mắt hột như sau\\
		\begin{center}
			\begin{tabular}{|c|c|c|c|c|c|c|c|c|c|c|c|c|c|c|}
			\hline
			Tuổi  & 12    & 13    & 14    & 15    & 16    & 17    & 18    & 19    & 20    & 21    & 22    & 23    & 25    & Tổng \\
			\hline
			Tần số & 2     & 2     & 1     & 4     & 2     & 5     & 5     & 2     & 2     & 2     & 1     & 1     & 1     & 30 \\
			\hline
			\end{tabular}
		\end{center}
		Ta thấy tuổi $17$ và $18$ có tần số bằng $5$ là lớn nhất.\\ 
		Do đó bảng số liệu có hai mốt là: $17 $ và $18$.
	}
\end{bt}

\begin{bt}%[PT Sinh, dự án SGK10]%[BG10-2022]
	Điểm bài kiểm tra một tiết môn toán của $40$ học sinh lớp $11A1$ được thống kê bằng bảng số liệu dưới đây
	\begin{center}
		\begin{tabular}{|c|c|c|c|c|c|c|c|c|c|}
		\hline
		Điểm &  3 &  4 &  5 &  6 &  7 &  8 &  9 &  10 & Cộng \\
		\hline
		Số học sinh &  2 &  3 &  $3n-8$ &  $2n+4$ &  3 &  2 &  4 &  5 & 40 \\
		\hline
		\end{tabular}\\
	\end{center}
	Trong đó $n\in \mathbb{N},n\ge 4$. Tính mốt của bảng số liệu thống kê đã cho. 
	\loigiai{
		Vì tổng các số liệu thống kê bằng $40$ nên ta có: $5n+15=40\Leftrightarrow n=5$.\\ 
		Với $n=5$ ta có bảng phân bố tần số\\
		\begin{center}
			\begin{tabular}{|c|c|c|c|c|c|c|c|c|c|}
			\hline
			Điểm &  3 &  4 &  5 &  6 &  7 &  8 &  9 &  10 & Cộng \\
			\hline
			Số học sinh &  2 &  3 &  7 &  14 &  3 &  2 &  4 &  5 & 40 \\
			\hline
			\end{tabular}\\
		\end{center}
		Vậy mốt của bảng số liệu là: $M_O=6$.
	}
\end{bt}

\begin{bt}%[PT Sinh, dự án SGK10]%[BG10-2022]
	Quan sát $9$ con chuột chạy qua một mê hồn trận và ghi lại thời gian (tính bằng phút) của chúng trong bảng sau:
	\begin{center}
		\begin{tabular}{|c|c|c|c|c|c|c|c|c|c|}
			\hline
			Con chuột & $1$ & $2$ & $3$ & $4$ & $5$ & $6$ & $7$ & $8$ & $9$ \\
			\hline
			Thời gian (phút) & $1$ & $2{,}5$ & $3$ & $1{,}5$ & $2$ & $1{,}25$ & $1$ & $0{,}9$ & $30$ \\
			\hline
		\end{tabular}
	\end{center}
	\begin{listEX}[1]
		\item Tính số trung bình, số trung vị và mốt của thời gian chuột ra khỏi mê hồn trận?
		\item Trong trường hợp này nên chọn đại lượng nào để thể hiện xu thế trung bình của mẫu?
	\end{listEX}
	\loigiai{
		\begin{listEX}[1]
			\item Số trung bình: $\overline{x}=\dfrac{1+2,5+\ldots +30}{9}\approx 4{,}79$.\\
			Sắp xếp dãy số liệu theo thứ tự không giảm:
			$0{,}9\quad 1\quad 1\quad 1{,}25\quad 1{,}5\quad 2\quad 2{,}5\quad 3\quad 30$.\\
			Số trung vị: $M_e=x_5=1{,}5$.\\Mốt: $M_0=1$.
			\item Trong trường hợp này ta nên chọn số trung vị để thể hiện xu thế trung bình của mẫu.
		\end{listEX}
	}
\end{bt}
%%Thầy Lam Nguyễn
\subsection{Bài tập trắc nghiệm}
\Opensolutionfile{ansbook}[ans/ansbook-2D1-1-TN]
\Opensolutionfile{ans}[ans/ans-2D1-1-TN]
\setcounter{ex}{0}
\begin{ex}%[Lam Nguyen, dự án SGK 10]%[0D5B3]
	Điều tra về số con của $40$ gia đình ở khu vực, kết quả thu được như sau:
	\begin{center}
		\begin{tabular}{|c|c|c|c|c|c|c|}
			\hline 
			Giá trị (số con) & $0$ &$1$ & $2$ &$3$ & $4$ & Tổng \\ 
			\hline 
			Tần số & $5$ & $9$ & $19$ & $5$ & $2$ & $N=40$ \\ 
			\hline 
		\end{tabular}
	\end{center}
	Số trung bình $\overline{x}$ của mẫu số liệu trên là
	\choice
	{$\overline{x}=2{,}75$}
	{$\overline{x}=1$}
	{\True $\overline{x}=1{,}75$}
	{$\overline{x}=3$}
	\loigiai{
		$\overline{x}=\dfrac{0\cdot 2+1\cdot 10+2\cdot 17+3\cdot 6+4\cdot 2}{40}=1{,}75$
	}
\end{ex} 
\begin{ex}%[Lam Nguyen, dự án SGK 10]]%[0D5K3]
	Kết quả điểm kiểm tra môn Toán của $40$ học sinh lớp $10A$ được trình bày ở bảng sau:
	\begin{center}
		\begin{tabular}{|c|c|c|c|c|c|c|c|c|}
			\hline 
			Điểm&$4$  &$5$  &$6$  &$7$  &$8$  &$9$  &$10$  &Cộng  \\ 
			\hline 
			Tần số&$2$  &$8$  &$7$  &$10$  &$8$  &$3$  &$2$  &$40$  \\ 
			\hline 
		\end{tabular} 
	\end{center}
	Tính số trung bình cộng của bảng trên. (làm tròn kết quả đến một chữ số thập phân).
	\choice
	{\True $6,\!8$}
	{$6,\!4$}
	{$7,\!0$}
	{$6,\!7$}
	\loigiai{
		$$\overline{x}=\dfrac{4\cdot 2+ 5\cdot 8+ 6\cdot 7+ 7\cdot 10+ 8\cdot 8+ 9\cdot 3+ 10\cdot 2}{40}\approx 6,\!8.$$
	}
\end{ex}
\begin{ex}%[Lam Nguyen, dự án SGK 10]%0D5Y3]
	Tiền thưởng (triệu đồng) của cán bộ và nhân viên trong một công ty được cho ở bảng sau:
	\begin{center}
		\begin{tabular}{|c|c|c|c|c|c|c|}
			\hline 
			Tiền lương	&2	&3	&4	&5	&6	&Cộng\\
			\hline
			Tần số	&5	&15	&10	&6	&4	&40\\
			\hline 
		\end{tabular}
	\end{center}
	Tính tiền thưởng trung bình.
	\choice
	{\True $3725000$ đồng}
	{$3745000$ đồng}
	{$3715000$ đồng}
	{$3625000$ đồng}
	\loigiai{
		Số trung bình tiền thưởng trong bảng phân bố trên là:
		$\overline{x}=\dfrac{1}{40}(2\cdot 5+3\cdot15+4\cdot10+5\cdot6+6\cdot4)=3{,}725$ triệu đồng.}
\end{ex}
 \begin{ex}%[Lam Nguyen, dự án SGK 10]% [0D5B3-2]
	Để được cấp chứng chỉ A- Anh văn của một trung tâm ngoại ngữ, học viên phải trải qua $6$ lần kiểm tra trắc nghiệm, thang điểm mỗi lần kiểm tra là $100$, và phải đạt điểm trung bình từ $70$ điểm trở lên. Qua $5$ lần thi Minh đạt điểm trung bình là $64,5$ điểm. Hỏi trong lần kiểm tra cuối cùng Minh phải đạt ít nhất là bao nhiêu điểm để được cấp chứng chỉ?
	\choice
	{\True $97,5$}
	{$96,5$}
	{$94,5$}
	{$93,5$}
	\loigiai{
		Gọi $x$ là số điểm trong lần kiểm tra cuối mà Minh cần đạt được để được cấp chứng chỉ
		Ta có số điểm qua 5 lần thi của Minh là $64,5.5=322,5$ 
		Suy ra $\dfrac{x+322,5}{6}=70\Leftrightarrow x=97,5$.
} \end{ex}	
	\begin{ex}%[Lam Nguyen, dự án SGK 10]%[0D5B3-2]
	Học sinh tỉnh $A$ (gồm lớp $11$ và lớp $12$) tham dự kì thi học sinh giỏi Toán của Tỉnh (thang điểm $20$) và điểm trung bình của họ là $10$ . Biết rằng số học sinh lớp $11$ nhiều hơn só học sinh lớp $12$ là $50\%$ và điểm trung bình của khới $12$ cao hơn điểm trung bình của khối 11 là $50 \%$. Điểm trung bình của khối 12 là
	\choice
	{$10$}
	{$11,25$}
	{\True $12,5$}
	{$15$}
	\loigiai{\begin{itemize}
			\item Gọi số học sinh lớp 12 là $n$. Theo bài ra, số học sinh lớp 11 sẽ là $1,5 n$. Gọi điểm trung bình của học sinh lớp $11$ là $a$. Theo bài ra, điểm trung bình của học sinh lớp $12$ là $1,5 a$.
			\item Tổng số điểm của học sinh lớp $11$ là $S=a \cdot 1,5 n=1,5 an$.
			\item Tổng số điểm của học sinh lớp $12$ là $T=(1,5 a) n=1,5 a n$.
			Vậy tổng số điểm của học sinh lớp $11$ và $12$ là $1,5 a n+1,5 a n=3 a n$.
			
			\item Mặt, ta có tổng số học sinh lớp $11$ và $12$ là $n+1,5 n=2,5 n$ và điểm trung bình của lớp $11$ và $12$ là $10$ . Do đó, tổng số điểm của học sinh lớp $11$ và $12$ là $10 \cdot(2,5 n)=25 n$.
			\item Từ đó ta có $3 a n=25 n$ hay $a=\dfrac{25}{3}$,\\
			Vậy điểm trung bình của học sinh lớp 12 là $1,5 a=1,5 \cdot \dfrac{25}{3}=12,5$.
			
		\end{itemize}
		
	}
\end{ex}
\begin{ex}%[Lam Nguyen, dự án SGK 10]%[0D5B3-2]
	Điểm thi học kì của một học sinh như sau $4$; $6$; $2$; $7$; $3$; $5$; $9$; $8$; $7$; $10$; $9$. Số trung bình và số trung vị lần lượt là
	\choice
	{$7$ và $6$}
	{\True $6{,}(36)$ và $7$}
	{$6{,}22$ và $7$}
	{$6$ và $6$}
	\loigiai
	{Ta có bảng phân bố tần số điểm của học sinh trên như sau
		\begin{center}
			{\renewcommand{\arraystretch}{1.3}
				\begin{tabular}{|>{\centering\arraybackslash}m{2cm}|>{\centering\arraybackslash}m{1cm}|>{\centering\arraybackslash}m{1cm}|>{\centering\arraybackslash}m{1cm}|>{\centering\arraybackslash}m{1cm}|>{\centering\arraybackslash}m{1cm}|>{\centering\arraybackslash}m{1cm}|>{\centering\arraybackslash}m{1cm}|>{\centering\arraybackslash}m{1cm}|>{\centering\arraybackslash}m{1cm}|}
					\hline
					Điểm & 2 & 3 & 4 & 5 & 6 & 7 & 8 & 9 & 10\\
					\hline
					Tần số & 1 & 1 & 1 & 1 & 1 & 2 & 1 & 2 & 1 \\
					\hline
			\end{tabular}}
		\end{center}
		\noindent
		Số trung bình cộng là $\dfrac{2+3+4+5+6+14+8+18+10}{11}=6{,}(36)$.\\
		Số trung vị là số thứ $6$ trong dãy sắp thứ tự. Số trung vị là $7$.}
\end{ex}
\begin{ex}%[Lam Nguyen, dự án SGK 10]%[0D5B4]
	Cho các số liệu thống kê về sản lượng chè thu được trong một năm (kg/sào) của $20$ hộ gia đình
	\begin{center}
		\begin{tabular}{|c|c|c|c|c|c|c|c|c|c|}
			\hline 
			$111$ & $112$ & $112$ & $113$ & $114$ & $114$ & $115$ & $114$ & $115$ & $116$\\ 
			\hline
			$112$ & $113$ & $113$ & $114$ & $115$ & $114$ & $116$ & $117$ & $113$ & $115$\\
			\hline
		\end{tabular}
	\end{center}
Số trung vị của bảng số liệu thống kê trên là
\choice
{$113$}
{\True $114$}
{$116$}
{$115$}
\loigiai{Do có $20$ giá trị biến lượng nên số trung vị của bảng số liệu là $$ M_e=\dfrac{1}{2}(x_{10}+x_{11})=\dfrac{1}{2}(114+114)=114\textrm{kg/sào}.$$
}
\end{ex}

\begin{ex}%[Lam Nguyen, dự án SGK 10]%%[0D5B3]
	Điểm học kì một của một học sinh được cho bởi  bảng số liệu sau (Đơn vị: điểm)
	\begin{center}
		\begin{tabular}{|c|c|c|c|c|c|c|c|c|}
			\hline
			5& 6 &6&7& 7 &8 &8& 8,5&9\\
			\hline
		\end{tabular}
	\end{center}
	Số trung vị của bảng trên là
	\choice
	{\True $7$}
	{$8$}
	{$9$}
	{$11$}
	\loigiai{ Ta có $N=9$ là số lẻ. Số liệu thứ $\dfrac{N+1}{2} = 5$ là số trung vị. Do đó số trung vị là 
		$M_e = 7$ (Điểm).
	}
\end{ex}

\begin{ex}%[Lam Nguyen, dự án SGK 10]%%[0D5Y3-2]
	Thống kê điểm kiểm tra môn Lịch sử của 45 học sinh lớp $ 10A $ như sau
	\begin{center}
		\begin{tabular}{|c|c|c|c|c|c|c|}
			\hline
			Điểm & 5 & 6 & 7 & 8 & 9 & 10 \\
			\hline
			Số học sinh & 2 & 11 & 9 & 16 & 4 & 3 \\
			\hline
		\end{tabular}
	\end{center}
	Số trung vị trong điểm các bài kiểm tra đó là
	\choice
	{$ 7{,}5 $ điểm}
	{$ 7{,}4 $ điểm}
	{\True $ 8 $ điểm}
	{$ 8{,}5 $ điểm}
	\loigiai
	{Số bài kiểm tra lớp $ 10A $ là $ 2+11+9+16+4+3=45 $ bài.\\
		Số trung vị là điểm bài thứ 23 đó là bài điểm 8.
	}
\end{ex}
\begin{ex}%[Lam Nguyen, dự án SGK 10]%%[0D5Y3-2]
	Cho bảng số liệu thống kê chiều cao của một nhóm học sinh như sau
	\begin{center}
		\begin{tabular}{|c|c|c|c|c|c|c|c|c|c|c|c|c|c|c|}
			\hline
			$151$	&$152$&$153$&$154$&$155$&$160$&$160$&$162$&$163$&$165$&$165$&$165$&$166$&$167$&$167$\\
			\hline
		\end{tabular}
	\end{center}
	Số trung vị của bảng số liệu nói trên là
	\choice
	{$160$}
	{\True $162$}
	{$167$}
	{$161$}
	\loigiai{
		Bảng giá trị trên có $15$ giá trị được xếp theo thứ tự tăng dần và số thứ $8$ có giá trị $162$. Do đó số trung vị của bảng số liệu nói trên là $162$.
	}
\end{ex}

\begin{ex}%[Lam Nguyen, dự án SGK 10]%
	Cho mẫu số liệu $5 ; 13 ; 5 ; 7 ; 10 ; 2 ; 3$. Tứ phân vị thứ nhất, thứ hai, thứ ba lần lượt là
	\choice
	{\True $3;5;10$}
	{$5;3;10$}
	{$10;3;5$}
	{$10;5;3$}
	\loigiai{
		Sắp xếp lại mẫu số liệu theo thứ tự không giảm, ta được: $2 ; 3 ; 5 ; 5 ; 7 ; 10 ; 13$.
		\begin{itemize}
			\item Vì cỡ mẫu là $n=7$, là số lẻ, nên giá trị tứ phân vị thứ hai là $Q_{2}=5$.
			\item Tứ phân vị thứ nhất là trung vị của mẫu: $2 ; 3 ; 5$. Do đó $Q_{1}=3$.
			\item Tứ phân vị thứ ba là trung vị của mẫu: $7 ; 10 ; 13$. Do đó $Q_{3}=10$.
		\end{itemize}
	}	
	
\end{ex}
\begin{ex}%[Lam Nguyen, dự án SGK 10]%
	Cho mẫu số liệu  $2 ; 3 ; 10 ; 13 ; 5 ; 15 ; 5 ; 7$. Tứ phân vị thứ nhất, thứ hai, thứ ba lần lượt là
	\choice
	{ $11,5; \,6; \,4$}
	{\True $4; \,6; \,11,5$}
	{$6; \,4; \,11,5$}
	{$6; \,11,5;\, 4$}
	\loigiai{
		Sắp xếp lại mẫu số liệu theo thứ tự không giảm, ta được: $2 ; 3 ; 5 ; 5 ; 7 ; 10 ; 13$.
		\begin{itemize}
			\item Vì cỡ mẫu là $n=8$, là số chẵn, nên giá trị tứ phân vị thứ hai là
			$$
			Q_{2}=\dfrac{1}{2}(5+7)=6.
			$$
			\item Tứ phân vị thứ nhất là trung vị của mẫu: $2 ; 3 ; 5 ; 5$. Do đó $Q_{1}=4$.
			\item Tứ phân vị thứ ba là trung vị của mẫu: $7 ; 10 ; 13 ; 15$. Do đó $Q_{3}=11,5$.
		\end{itemize}
	}	
	
\end{ex}
\begin{ex}%[Lam Nguyen, dự án SGK 10]%
	Cho mẫu số liệu $21 ; 35 ; 17 ; 43 ; 8 ;59 ;72 ; 119$. Tứ phân vị thứ nhất, thứ hai, thứ ba lần lượt là
	\choice
	{ \True $19; \,39; \,65,5$}
	{ $26; \,43; \,65,5$}
	{$39; \,19; \,65,5$}
	{$43; \,26;\, 65,5$}
	\loigiai{
		Sắp xếp lại mẫu số liệu theo thứ tự không giảm, ta được: $8 ; 17 ; 21 ; 35 ; 43 ; 59 ; 72 ; 119$.
		\begin{itemize}
			\item Vì cỡ mẫu là $n=8$, là số chẵn, nên giá trị tứ phân vị thứ hai là
			$$
			Q_{2}=\dfrac{1}{2}(35+43)=39.
			$$
			\item Tứ phân vị thứ nhất là trung vị của mẫu: $8 ; 17 ; 21 ; 35$. Do đó $Q_{1}=19$.
			\item Tứ phân vị thứ ba là trung vị của mẫu:  $43 ; 59 ; 72 ; 119$. Do đó $Q_{3}=65,5$.
		\end{itemize}
	}	
\end{ex}
\begin{ex}%[Lam Nguyen, dự án SGK 10]%%[0D5Y3]
	Các giá trị xuất hiện nhiều nhất trong mẫu dữ liệu được gọi là
	\choice
	{\True Mốt}
	{Số trung vị}
	{Số trung bình}
	{Độ lệch chuẩn}
	\loigiai
	{
		Theo định nghĩa, ta có: Các giá trị xuất hiện nhiều nhất trong mẫu dữ liệu được gọi là mốt.
	}
\end{ex}
\begin{ex}%[Lam Nguyen, dự án SGK 10]%%[0D5B2]
	Cho bảng phân bố tần số\\
	\centerline{\it Tiền thưởng (triệu đồng) cho cán bộ và nhân viên trong một công ty}
	\begin{center}
		\begin{tabular}{|c|c|c|c|c|c|c|}
			\hline 
			Tiền thưởng & 2 & 3 & 4 & 5 & 6 & Cộng\\ 
			\hline 
			Tần số & 5 & 15 & 10 & 6 & 7 & 43\\ 
			\hline 
		\end{tabular} 
	\end{center}
	Mốt của bảng phân bố tần số đã cho là
	\choice
	{$5$ triệu đồng}
	{$6$ triệu đồng}
	{\True $3$ triệu đồng}
	{$2$ triệu đồng}
	\loigiai{
		Mốt là số liệu có tần số lớn nhất, ở đây mốt là $3$ triệu đồng.
	}	
\end{ex}

\begin{ex}%[Lam Nguyen, dự án SGK 10]%%0D5Y3]
	Tiền thưởng (triệu đồng) của cán bộ và nhân viên trong một công ty được cho ở bảng sau:
	\begin{center}
		\begin{tabular}{|c|c|c|c|c|c|c|}
			\hline 
			Tiền lương	&1	&2	&3	&4	&5	&Cộng\\
			\hline 
			Tần số	&10	&12	&11	&15	&2	&50 \\
			\hline 
		\end{tabular}
	\end{center}
	Tính mốt $M_O.$
	\choice
	{\True $M_O=4$}
	{$M_O=5$}
	{$M_O=15$}
	{$M_O=11$}
	\loigiai{
		Trong bảng phân bố giá trị tiền lương thì 4 triệu có tần số lớn nhất là 15 nên mốt là:
		$M_O=4$.}
\end{ex}



\begin{ex}%[Lam Nguyen, dự án SGK 10]%%[Nguyễn Khắc Hưởng]%[0D5G3]
	Điểm kiểm tra môn Toán của $35$ học sinh lớp 10A được thống kê trong bảng phân bố tần số sau đây (thang điểm $10$):
	\begin{center}
		\begin{tabular}{|c|c|c|c|c|c|c|c|c|c|c|c|c|}
			\hline 
			\bf Điểm &$0$ &$1$&$2$&$3$&$4$&$5$&$6$&$7$&$8$&$9$&$10$& Cộng\\ 
			\hline
			\bf Tần số &$2$ &$1$&$2$&$1$&$2$&$3$&$x$&$5$&$y$&$4$&$3$& $n=35$\\
			\hline
		\end{tabular}
	\end{center}
	Biết rằng mẫu số liệu trên có $2$ mốt. Giá trị của $x \cdot y$ là
	\choice
	{\True $36$}
	{$35$}
	{$27$}
	{$32$}
	\loigiai{Tổng số học sinh là $35$ nên $x+y=12$, suy ra có ít nhất một trong hai số $x$ hoặc $y$ không nhỏ hơn $6$. Vì mẫu số liệu có $2$ mốt nên $x=y=6$ thỏa mãn.
	}
\end{ex}
\begin{ex}%[Lam Nguyen, dự án SGK 10]%%[0D5K3]
	Cho bảng phân bố tần số sau
	\begin{center}
		\newcolumntype{P}[1]{>{\centering\arraybackslash}p{#1}}
		\begin{tabular}{|c|P{0.6in}|P{0.6in}|P{0.6in}|P{0.6in}|P{0.6in}|}
			\hline
			Giá trị & $x_1$    & $x_2$    & $x_3$    & $x_4$    & $x_5$ \\
			\hline
			Tần số &   3    &      5     &  $n+6$    & $20-n$    & 9 \\
			\hline
		\end{tabular}
	\end{center}
	Trong đó $ n $ là số tự nhiên và giá trị $x_4 $ là mốt duy nhất của bảng số liệu thống kê đã cho. Hãy tìm số $n$?
	\choice
	{\True $n \in [0;7)$}
	{$n \in [0;8)$}
	{$n \in (0;7)$}
	{$n \in (0;7]$}
	\loigiai{
		Từ giả thiết $x_4$ là mốt duy nhất của bảng số liệu thống kê đã cho nên ta có\\
		$\heva{& 20-n>9 \\   & 20-n>n+6}\Leftrightarrow \heva{ & n<11 \\   & n<7}\Leftrightarrow n<7 $.\\
		Vì $n$ là số tự nhiên nên các giá trị $n$ thỏa mãn là: $0\le n<7$.
	}
\end{ex} 


\begin{ex}%[Lam Nguyen, dự án SGK 10]%%[0D5K3]
	Cho bảng phân bố tần số
	\begin{center}
		\newcolumntype{P}[1]{>{\centering\arraybackslash}p{#1}}
		\begin{tabular}{|c|P{0.6in}|P{0.6in}|P{0.6in}|P{0.6in}|P{0.6in}|}
			\hline
			Giá trị & $x_1$    & $x_2$    & $x_3$    & $x_4$    & $x_5$   \\
			\hline
			Tần số & 2    & $x+y$     & $2x-y$    & 5    & 6\\
			\hline
		\end{tabular}
	\end{center}
	với $x, y$ là các số tự nhiên. Có tất cả các cặp số $(x;y)$ để $x_5$ là mốt của bảng số liệu đã cho?
	\choice
	{$13$}
	{$12$}
	{\True $14$}
	{$16$}
	\loigiai{
		Điều kiện để $x_5$ là mốt của bảng số liệu đã cho là:
		$\heva{ &0\le x+y\le 6& \quad (1)\\  &0\le 2x-y\le 6& \quad (2)\\}$.\\
		Từ $(1)$ suy ra $ y\le 6-x $ và $ x\le 6-y $.\\
		Từ $(2)$ suy ra $ 2x-6\le y $ và $ \dfrac{y}{2}\le x $.\\
		Do đó $ \heva{&2x-6\le 6-x \\  &\dfrac{y}{2}\le 6-y \\ }\Leftrightarrow \heva{ &0\le x\le 4 \\  &0\le y\le 4 \\ }$.\\
		Từ đó tìm được $ 14 $ cặp số   thỏa mãn là:\\
		$ \left(0;0 \right),\left(1;0 \right),\left(1;1 \right),\left(1;2 \right),\left(2;0 \right),\left(2;1 \right),\left(2;2 \right)$\\
		$ \left(3;0 \right),\left(3;1 \right),\left(3;2 \right),\left(3;3 \right),\left(4;2 \right),\left(2;3 \right),\left(2;4 \right)$.
	}
\end{ex}
\begin{ex}%[Lam Nguyen, dự án SGK 10]%%[0D5G3]
	Cho bảng phân bố tần số
	\begin{center}
		\newcolumntype{P}[1]{>{\centering\arraybackslash}p{#1}}
		\begin{tabular}{|c|P{0.6in}|P{0.6in}|P{0.6in}|P{0.6in}|P{0.6in}|}
			\hline
			Giá trị & $x_1$    & $x_2$    & $x_3$    & $x_4$    & $x_5$   \\
			\hline
			Tần số & 6    & $3x+y$     & $3y-3x$    & $x+y$    & 4\\
			\hline
		\end{tabular}
	\end{center}
	với $x, y$ là các số tự nhiên. Có bao nhiêu cặp số $(x;y)$ để bảng số liệu thống kê đã cho có mốt là $3$ giá trị khác nhau?
	\choice
	{$2$}
	{\True $1$}
	{$3$}
	{$4$}
	\loigiai{
		\textsl{Trường hợp 1:} các giá trị $x_1,x_2,x_3$ là mốt khi $ \heva{ &3x+y= 6\\  &3y-3x=6\\ &x+y<6} \Leftrightarrow \heva{ &x=1\\  &y=3\\ &x+y<6}\Leftrightarrow \heva{ &x=1\\  &y=3}$.\\ 
		\textsl{Trường hợp 2:} các giá trị $x_1,x_2,x_4$ là mốt khi $ \heva{ &3x+y= 6\\&x+y=6 \\ &3y-3x<6 } \Leftrightarrow \heva{ &x=0\\  &y=6\\ &y-x<2}$.\\ Hệ vô nghiệm vì $\heva{ &x=0\\  &y=6}$ không thỏa mãn bất phương trình $y-x<2$.\\
		\textsl{Trường hợp 3:} các giá trị $x_1,x_3,x_4$ là mốt khi $ \heva{ &x+y= 6\\&3y-3x=6 \\ &3x+y<6 } \Leftrightarrow \heva{ &x=2\\  &y=4\\ &3x+y<6}$.\\ Hệ vô nghiệm vì $\heva{ &x=2\\  &y=4}$ không thỏa mãn bất phương trình $3x+y<6$.\\
		\textsl{Trường hợp 4:} các giá trị $x_2,x_3,x_4$ là mốt khi $ \heva{ &3x+y=3y-3x\\&3x+y=x+y \\ &x+y>6 } \Leftrightarrow \heva{ &x=0\\  &y=0\\ &x+y>6}$. \\
		Hệ vô nghiệm vì $\heva{ &x=0\\  &y=0}$ không thỏa mãn bất phương trình  $x+y>6$.\\
		Vậy chỉ có $\heva{ &x=1\\  &y=3}$ thỏa mãn yêu cầu của bài toán.
	}
\end{ex}


\Closesolutionfile{ans}
\Closesolutionfile{ansbook}
% % \indapan{10}{ans/ans-2D1-1-TN}