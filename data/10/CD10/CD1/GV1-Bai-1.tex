\section{HỆ PHƯƠNG TRÌNH BẬC NHẤT BA ẨN}
\subsection{Tóm tắt lí thuyết}
\subsubsection{Hệ phương trình bậc nhất ba ẩn}
\begin{tomtat}
	\begin{itemize}
		\item Phương trình bậc nhất ba ẩn có dạng tổng quát là 
		\[ax+by+cz=d,\]
		trong đó $x$, $y$, $z$ là ba ẩn; $a$, $b$, $c$, $d$ là các hệ số và $a$, $b$, $c$ không đồng thời bằng không.\\
		Mỗi bộ ba số $(x_0;y_0;z_0)$ thỏa mãn $ax_0+by_0+cz_0=d$ gọi là một nghiệm của phương trình bậc nhất ba ẩn đã cho.
		\item Hệ phương trình bậc nhất ba ẩn là hệ gồm một số phương trình bậc nhất ba ẩn. Mỗi nghiệm chung của các phương trình đó được gọi là một nghiệm của hệ phương trình đã cho.
		\item Hệ ba phương trình bậc nhất ba ẩn có dạng tổng quát là
		 \[\heva{&a_1x+b_1y+c_1z=d_1\\&a_2x+b_2y+c_2z=d_2\\&a_3x+b_3y+c_3z=d_3}\]
		 trong đó $x$, $y$, $z$ là ba ẩn; các chữ số còn lại là các hệ số. Ở đây, trong mỗi phương trình, ít nhất một trong các hệ số $a_i$, $b_i$, $c_i$ $(i=1,2,3)$ phải khác $0$.
	\end{itemize}
\end{tomtat}
\begin{vd}%KNTT
	Hệ phương trình nào dưới đây là hệ phương trình bậc nhất ba ẩn? Kiểm tra xem mỗi bộ ba số $(1;1;2)$, $(-1;3;0)$ có phải là một nghiệm của hệ phương trình bậc nhất ba ẩn đó không.
\begin{listEX}[3]
	\item $\heva{&2x-2y+x=-7\\&x+2y-2z=5\\&-x^2+3y-2z=-2;}$
	\item $\heva{&2xy+y=1\\&2x+3y+5z=-2\\&-4x-7y+z-4;}$
	\item $\heva{&x+3y+2z=8\\&2x+2y+z=6\\&3x+y+z=6.}$
\end{listEX}
	\loigiai{
	Hệ phương trình ở câu a) không phải là hệ phương trình bậc nhất ba ẩn vì phương trình thứ ba chứa $x^2$.\\
	Hệ phương trình ở câu b) không phải là hệ phương trình bậc nhất ba ẩn vì phương trình thứ nhất chứa $xy$.\\
	Hệ phương trình ở câu c) là hệ phương trình bậc nhất ba ẩn. 
	\begin{itemize}
		\item Thay $x=1$; $y=1$; $z=2$ vào vế trái của từng phương trình của hệ ở câu c) và so sánh với vế phải, ta được:
		\begin{itemize}
			\item Phương trình thứ nhất: $1+3\cdot 1+2\cdot 2=8$ (thỏa mãn);
			\item Phương trình thứ hai: $2\cdot 1+2\cdot 1+2=6$ (thỏa mãn);
			\item Phương trình thứ ba: $3\cdot 1+1+2=6$ (thỏa mãn).
		\end{itemize}
		Vậy $(1;1;2)$ là một nghiệm của hệ phương trình.
		\item Thay $x=-1$; $y=3$; $z=0$ vào vế trái của từng phương trình của hệ ở câu c) và so sánh với vế phải, ta được:
		\begin{itemize}
			\item Phương trình thứ nhất: $(-1)+3\cdot 3+2\cdot 0=8$ (thỏa mãn);
			\item Phương trình thứ hai: $2\cdot (-1)+2\cdot 3+0=4\neq 6$ (không thỏa mãn).
			\end{itemize}
	Vậy $(-1;3;0)$ không phải nghiệm của hệ phương trình.
	\end{itemize}
	}
\end{vd}
\subsubsection{Giải hệ phương trình bậc nhất bằng ba ẩn bằng phương pháp Gauss}
\begin{tomtat}
	Để giải hệ phương trình dạng tam giác, trước hết ta giải từ phương trình chứa một ẩn, sau đó thay giá trị tìm được của ẩn này vào phương trình chứa hai ẩn để tìm giá trị của ẩn thứ hai, cuối cùng thay các giá trị tìm được vào phương trình còn lại để tìm giá trị của ẩn thứ ba.
\end{tomtat}
\begin{vd}
	Giải hệ phương trình 
	\[\heva{&x+y+3z=10\\&y-z=3\\&2z=4.}\]
	\loigiai{
	Từ phương trình thứ ba ta có $z=2$. Thay $z=2$ vào phương trình thứ hai ta được $y-2=3$ hay $y=5$. Thay $y=5$ và $z=2$ vào phương trình thứ nhất ta được $x+5+3\cdot 2=10$ hay $x=-1$.\\
	Vậy nghiệm của hệ phương trình đã cho là $(-1;5;2)$.
}
\end{vd}
\begin{tomtat}
	Để giải một hệ phương trình bậc nhất ba ẩn, ta đưa hệ đó về một hệ đơn giản hơn (thường có dạng tam giác), bằng cách sử dụng các phép biến đổi sau đây:
	\begin{itemize}
		\item Nhân hai vế của một phương trình của hệ với một số khác $0$.
		\item Đổi vị trí hai phương trình của hệ.
		\item Cộng mỗi vế của một phương trình (sau khi đã nhân với một số khác $0$) với vế tương ứng của phương trình khác để được phương trình mới có số ẩn ít hơn.
	\end{itemize}
Từ đó có thể giải hệ đã cho. Phương pháp này được gọi là \textbf{phương pháp Gauss}
\begin{nx}
	Hệ phương trình bậc nhất ba ẩn có thể có nghiệm duy nhất, vô nghiệm hoặc có vô số nghiệm.
\end{nx}
\end{tomtat}
\begin{vd}
	Giải các hệ phương trình sau bằng phương pháp Gauss
\begin{listEX}[3]
	\item $\heva{&x-y+2z=4\\&2x+y-z=-1\\&x+y+z=5;}$
	\item $\heva{&x-2y+3z=10\\&2x+3y-z=2\\&x+5y-4z=1;}$
	\item $\heva{&2x-y+3z=1\\&x+y+2z=1\\&5x+2y+9z=4.}$
\end{listEX}
	\loigiai{
	\begin{enumerate}
		\item Nhân hai vế của phương trình thứ nhất của hệ với $(-2)$ và cộng với phương trình thứ hai theo từng vế tương ứng ta được hệ phương trình 
		\[\heva{&x-y+2z=4\\&3y-5z=-9\\&x+y+z=5.}\]
		Nhân hai vế của phương trình thứ nhất của hệ với $(-1)$ và cộng với phương trình thứ ba theo từng vế tương ứng ta được hệ phương trình
		\[\heva{&x-y+2z=4\\&3y-5z=-9\\&2y-z=1.}\]
		Nhân cả hai vế của phương trình thứ hai với $\dfrac{-2}{3}$ và cộng với phương trình thứ ba theo từng vế tương ứng ta được hệ phương trình 
		\[\heva{&x-y+2z=4\\&3y-5z=-9\\&\dfrac{7}{3}z=7.}\]
		Từ phương trình thứ ba, ta có $z=3$. Thế vào phương trình thứ hai ta được $3y-5\cdot 3=-9$ hay $y=2$. Thay $z=3$ và $y=2$ vào phương trình đầu tiên, ta được $x-2+2\cdot 3=4$ hay $x=0$.\\
		Vậy nghiệm của hệ phương trình đã cho là $(0;2;3)$.
		\item Nhân hai vế của phương trình thứ nhất của hệ với $(-2)$ và cộng với phương trình thứ hai theo từng vế tương ứng ta được hệ phương trình 
		\[\heva{&x-2y+3z=10\\&7y-7z=-18\\&x+5y-4z=1}\]
		Nhân hai vế của phương trình thứ nhất của hệ với $(-1)$ và cộng với phương trình thứ ba theo từng vế tương ứng ta được hệ phương trình
		\[\heva{&x-2y+3z=10\\&7y-7z=-18\\&7y-7z=-9}\]
		Từ hai phương trình thứ hai và thứ ba, suy ra $-18=-9$, điều này vô lí. \\
		Vậy hệ phương trình đã cho vô nghiệm.
		\item Đổi chỗ phương trình thứ nhất và phương trình thứ hai ta được hệ phương trình 
		\[\heva{&x+y+2z=1\\&2x-y+3z=1\\&5x+2y+9z=4.}\]
		Nhân hai vế của phương trình thứ nhất của hệ với $(-2)$ và cộng với phương trình thứ hai theo từng vế tương ứng ta được hệ phương trình 
		\[\heva{&x+y+2z=1\\&-3y-z=-1\\&5x+2y+9z=4.}\]
		Nhân hai vế của phương trình thứ nhất của hệ với $(-5)$ và cộng với phương trình thứ ba theo từng vế tương ứng ta được hệ phương trình 
		\[\heva{&x+y+2z=1\\&-3y-z=-1\\&-3y-z=-1.}\]
		Nhận thấy phương trình thứ hai và thứ ba của hệ giống nhau, ta được hệ tương đương dạng hình thang
		\[\heva{&x+y+2z=1\\&-3y-z=-1.}\]
		Rút $z$ theo $y$ từ phương trình thứ hai của hệ ta được $z=-3y+1$. Thế vào phương trình thứ nhất ta được $x+y+2(-3y+1)=1$ hay $x=5y-1$.\\
		Vậy hệ phương trình đã cho có vô số nghiệm và tập nghiệm của hệ là \[ S=\left\{(5y-1;y;-3y+1)\ \middle | \ y\in\mathbb{R}\right\}.\]
	\end{enumerate}	
}
\end{vd}
\begin{vd}%ứng dụng
	Ba bạn Lan, Anh và Khoa đi chợ mua trái cây. Bạn Lan mua $2$ kí cam và $3$ kí ổi hết $295$ nghìn đồng, bạn Khoa mua $4$ kí táo và $1$ kí ổi hết $345$ nghìn đồng và bạn Anh mua $2$ kí táo, $3$ kí cam và $1$ kí ổi hết $355$ nghìn đồng. Hỏi giá một kí mỗi loại cam, táo và ổi là bao nhiêu? 
	\loigiai{
Gọi $x$, $y$, $z$ (nghìn đồng) lần lượt là giá của một kí mỗi loại cam, táo và ổi (điều kiện $x,y,z\geq 0$).\\
	Bạn Lan mua $2$ kí cam và $3$ kí ổi hết $295$ nghìn đồng nên ta có \[2x+3z=295.\]
	Bạn Khoa mua $4$ kí táo và $1$ kí ổi hết $345$ nghìn đồng nên ta có \[4y+z=345.\]
	bạn Anh mua $2$ kí táo, $3$ kí cam và $1$ kí ổi hết $355$ nghìn đồng nên ta có \[3x+2y+z=355.\]
	Do đó, ta có hệ phương trình bậc nhất ba ẩn
	\[\heva{&2x+3z=295\\&4y+z=345\\&3x+2y+z=355.}\]
	Giải hệ phương trình trên ta được $x=50$, $y=70$ và $z=65$.\\
	Vậy giá của một kí cam là $50$ nghìn đồng, giá của một kí táo là $70$ nghìn đồng và giá của một kí ổi là $65$ nghìn đồng.
}
\end{vd}
\subsubsection{Tìm nghiệm của hệ phương trình bậc nhất ba ẩn bằng máy tính cầm tay}
\begin{tomtat}
	Ta có thể dùng máy tính cầm tay để giải hệ phương trình bậc nhất ba ẩn. Sau khi mở máy, ta lần lượt thực hiện các thao tác sau:
	\begin{itemize}
		\item Vào chương trình giải hệ phương trình  nhất ba ẩn, ấn
		\begin{itemize}
			\item Đối với máy tính CASIO fx-570VN PLUS: \fbox{MODE} \fbox{$5$} \fbox{$2$}. 
			\item Đối với máy tính CASIO fx-580VNX: \fbox{MENU} \fbox{$9$} \fbox{$1$} \fbox{$3$}. 
		\end{itemize}
	\item Nhập các hệ số để giải hệ phương trình.
	\end{itemize}
\end{tomtat}
\begin{vd}
	Dùng máy tính cầm tay tìm nghiệm của các hệ phương trình sau:
	\begin{listEX}[3]
	\item $\heva{&x-y+z=-3\\&3x+2y+3z=6\\&2x-y-4z=3;}$
	\item $\heva{&x-3y+z=5\\&-2x+y+2z=5\\&x+2y-3z=2;}$
	\item $\heva{&5x+y-4z=2\\&3x+3y-2z=4\\&x-y-z=-1.}$
	\end{listEX}
\loigiai{
Vào chương trình giải hệ phương trình  nhất ba ẩn, ấn
\begin{itemize}
	\item Đối với máy tính CASIO fx-570VN PLUS: \fbox{MODE} \fbox{$5$} \fbox{$2$}. 
	\item Đối với máy tính CASIO fx-580VNX: \fbox{MENU} \fbox{$9$} \fbox{$1$} \fbox{$3$}. 
\end{itemize}
\begin{enumerate}
	\item Nhập các hệ số để giải hệ phương trình:
	\begin{itemize}
		\item Nhập hệ số của phương trình thứ nhất: \fbox{$1$} \fbox{$=$} \fbox{$-$} \fbox{$1$} \fbox{$=$} \fbox{$1$} \fbox{$=$} \fbox{$-$} \fbox{$3$} \fbox{$=$}
		\item Nhập hệ số của phương trình thứ hai: \fbox{$3$} \fbox{$=$} \fbox{$2$} \fbox{$=$} \fbox{$3$} \fbox{$=$} \fbox{$6$} \fbox{$=$}
		\item Nhập hệ số của phương tình thứ ba: \fbox{$2$} \fbox{$=$} \fbox{$-$} \fbox{$1$} \fbox{$=$} \fbox{$-$} \fbox{$4$} \fbox{$=$} \fbox{$3$} \fbox{$=$}
	\end{itemize}
Ấn tiếp phím \fbox{$=$}, ta thấy màn hình hiện $x=1$.\\
Ấn tiếp phím \fbox{$=$}, ta thấy màn hình hiện $y=3$.\\
Ấn tiếp phím \fbox{$=$}, ta thấy màn hình hiện $z=-1$.\\
Vậy nghiệm của hệ phương trình là $(1;3;-1)$.
\item Nhập các hệ số để giải hệ phương trình:
	\begin{itemize}
	\item Nhập hệ số của phương trình thứ nhất: \fbox{$1$} \fbox{$=$} \fbox{$-$} \fbox{$3$} \fbox{$=$} \fbox{$1$} \fbox{$=$} \fbox{$5$}  \fbox{$=$}
	\item Nhập hệ số của phương trình thứ hai: \fbox{$-$} \fbox{$2$} \fbox{$=$} \fbox{$1$} \fbox{$=$} \fbox{$2$} \fbox{$=$} \fbox{$5$} \fbox{$=$}
	\item Nhập hệ số của phương tình thứ ba: \fbox{$1$} \fbox{$=$} \fbox{$2$} \fbox{$=$} \fbox{$-$} \fbox{$3$} \fbox{$=$} \fbox{$2$} \fbox{$=$} 
\end{itemize}
Ấn tiếp phím \fbox{$=$}, ta thấy màn hình hiện No Solution.\\
Vậy phương trình đã cho vô nghiệm.
\item Nhập các hệ số để giải hệ phương trình:
	\begin{itemize}
	\item Nhập hệ số của phương trình thứ nhất: \fbox{$5$} \fbox{$=$} \fbox{$1$} \fbox{$=$} \fbox{$-$} \fbox{$4$} \fbox{$=$} \fbox{$2$}  \fbox{$=$}
	\item Nhập hệ số của phương trình thứ hai: \fbox{$3$} \fbox{$=$} \fbox{$3$} \fbox{$=$} \fbox{$-$} \fbox{$2$} \fbox{$=$} \fbox{$4$} \fbox{$=$}
	\item Nhập hệ số của phương tình thứ ba: \fbox{$1$} \fbox{$=$} \fbox{$-$} \fbox{$1$} \fbox{$=$} \fbox{$-$} \fbox{$1$} \fbox{$=$} \fbox{$-$} \fbox{$1$} \fbox{$=$}
\end{itemize}
Ấn tiếp phím \fbox{$=$}, ta thấy màn hình hiện Infinite Solution.\\
Vậy phương trình đã cho có vô số nghiệm.
\end{enumerate}
}
\end{vd}
\subsection{Bài tập luyện tập}
%Bài tập phần 1
\begin{bt}%KNTT
Hệ nào dưới đây là hệ phương trình bậc nhất ba ẩn? Kiểm tra xem bộ ba số $(-3;2;-1)$ có phải là nghiệm của hệ phương trình bậc nhất ba ẩn đó không.
\begin{listEX}[2]
	\item $\heva{&x+2y-3z=1\\&2x-3y+7z=15\\&3x^2-4y+z=-3;}$
	\item $\heva{&-x+y+z=4\\&2x+y-3z=-1\\&3x-2z=-7.}$
\end{listEX}
	\loigiai{
		Hệ phương trình ở câu a) không phải hệ phương trình bậc nhất ba ẩn vì phương trình thứ ba chứa $x^2$.\\
		Hệ phương trình ở câu b) là hệ phương trình bậc nhất ba ẩn. Thay $x=-3$; $y=2$; $z=-1$ vào vế trái của từng phương trình của hệ ở câu c) và so sánh với vế phải, ta được:
		\begin{itemize}
			\item Phương trình thứ nhất: $-(-3)+2+(-1)=4$ (thỏa mãn);
			\item Phương trình thứ hai: $2\cdot (-3)+2-3\cdot (-1)=-1$ (thỏa mãn);
			\item Phương trình thứ ba: $3\cdot (-3)-2\cdot(-1)=-7$ (thỏa mãn).
		\end{itemize}
		Vậy $(-3;2;-1)$ là một nghiệm của hệ phương trình.
	}
\end{bt}
\begin{bt}%CTST
	Hệ phương trình nào dưới đây là hệ phương trình bậc nhất ba ẩn? Mỗi bộ ba số $(1;5;2)$, $(1;1;1)$ và $(-1;2;3)$ có là nghiệm của hệ phương trình bậc nhất ba ẩn đó không?
	\begin{listEX}[2]
		\item $\heva{&4x-2y+z=5\\&4xz-5y+2z=-7\\&-x+3y+2z=3;}$
		\item $\heva{&x+2z=5\\&2x-y+z=-1\\&3x-2y=-7.}$
			\end{listEX}
		\loigiai{
			Hệ phương trình ở câu a) không phải là hệ phương trình bậc nhất ba ẩn vì phương trình thứ hai chứa $xz$.\\
			Hệ phương trình ở câu b) là hệ phương trình bậc nhất ba ẩn.
			\begin{itemize}
				\item Thay $x=1$; $y=5$; $z=2$ vào vế trái của từng phương trình của hệ ở câu c) và so sánh với vế phải, ta được:
				\begin{itemize}
					\item Phương trình thứ nhất: $1+2\cdot 2=5$ (thỏa mãn);
					\item Phương trình thứ hai: $2\cdot 1-5+2=-1$ (thỏa mãn);
					\item Phương trình thứ ba: $3\cdot 1-2\cdot 5=-7$ (thỏa mãn).
				\end{itemize}
				Vậy $(1;5;2)$ là một nghiệm của hệ phương trình.
				\item Thay $x=1$; $y=1$; $z=1$ vào vế trái của từng phương trình của hệ ở câu c) và so sánh với vế phải, ta được phương trình thứ nhất: $1+2\cdot 1=3\neq 5$ (không thỏa mãn).
				Vậy $(1;1;1)$ là nghiệm của hệ phương trình.
				\item Thay $x=-1$; $y=2$; $z=3$ vào vế trái của từng phương trình của hệ ở câu c) và so sánh với vế phải, ta được:
				\begin{itemize}
					\item Phương trình thứ nhất: $(-1)+2\cdot 3=5$ (thỏa mãn);
					\item Phương trình thứ hai: $2\cdot (-1)-2+3=-1$ (thỏa mãn);
					\item Phương trình thứ ba: $3\cdot (-1)-2\cdot 2=-7$ (thỏa mãn).
				\end{itemize}
				Vậy $(-1;2;3)$ là một nghiệm của hệ phương trình.
			\end{itemize}
		}
\end{bt}
%Bài tập phần 2
\begin{bt}%KNTT
	Giải hệ phương trình 
	\[\heva{&2x=3\\&x+y=2\\&2x-2y+z=-1.} \]
\end{bt}
\loigiai{Ta có
	\allowdisplaybreaks
	$$\begin{aligned}
	& \heva{&2x=3\\&x+y=2\\&2x-2y+x=-1}\Leftrightarrow \heva{&x=\dfrac{3}{2}\\&x+y=2\\&2x-2y+z=-1}\Leftrightarrow \heva{&x=\dfrac{3}{2}\\&\dfrac{3}{2}+y=2\\&2x-2y+z=-1}\\
	\Leftrightarrow & 
	\heva{&x=\dfrac{3}{2}\\&y=\dfrac{1}{2}\\&2x-2y+z=-1} \Leftrightarrow
	\heva{&x=\dfrac{3}{2}\\&y=\dfrac{1}{2}\\&2\cdot\dfrac{3}{2}-2\cdot\dfrac{1}{2}+z=-1}\Leftrightarrow
	\heva{&x=\dfrac{3}{2}\\&y=\dfrac{1}{2}\\&z=-3.}
	\end{aligned}$$
	Vậy hệ phương trình đã cho có nghiệm là $\left(\dfrac{3}{2};\dfrac{1}{2};-3\right)$.
}
\begin{bt}
	Giải các hệ phương trình sau:
	\begin{listEX}[3]
		\item $\heva{&2x+y-3z=3\\&x+y+3z=2\\&3x-2y+z=-1;}$
		\item $\heva{&4x+y+3z=-3\\&2x+y-z=1\\&5x+2y=1;}$
		\item $\heva{&x+2z=-2\\&2x+y-z=1\\&4x+y+3z=-3.}$
	\end{listEX}
\loigiai{
\begin{enumerate}
	\item Ta có
	$$\begin{aligned}
	& \heva{&2x+y-3z=3\\&x+y+3z=2\\&3x-2y+z=-1}\Leftrightarrow \heva{&x+y+3z=2\\&2x+y-3z=3\\&3x-2y+z=-1} \Leftrightarrow \heva{&x+y+3z=2\\&-y-9z=-1\\&-5y-8z=-7}\\
	\Leftrightarrow &  \heva{&x+y+3z=2\\&-y-9z=-1\\&37z=-2}\Leftrightarrow \heva{&x+y+3z=2\\&-y-9z=-1\\&z=-\dfrac{2}{37}} \Leftrightarrow \heva{&x+y+3z=2\\&y=\dfrac{55}{37}\\&z=-\dfrac{2}{37}} \Leftrightarrow \heva{&x=\dfrac{25}{37}\\&y=\dfrac{55}{37}\\&z=-\dfrac{2}{37}.}\end{aligned}	$$
	Vậy hệ phương trình có nghiệm là $\left(\dfrac{25}{37};\dfrac{55}{37};-\dfrac{2}{37}\right)$.
	\item Ta có
	$$\begin{aligned}
	& \heva{&4x+y+3z=-3\\&2x+y-z=1\\&5x+2y=1}\Leftrightarrow \heva{&4x+y+3z=-3\\&6x+3y-3z=3\\&5x+2y=1} \\
	\Leftrightarrow& \heva{&4x+y+3z=-3\\&10x+4y=0\\&5x+2y=1}	\Leftrightarrow  \heva{&4x+y+3z=-3\\&5x+2y=0\\&5x+2y=1.}
	\end{aligned}$$
	Từ hai phương trình cuối, ta suy ra $0=1$,  điều này vô lí.\\
	Vậy hệ phương trình đã cho vô nghiệm.
	\item Ta có
	$$\heva{&x+2z=-2\\&2x+y-z=1\\&4x+y+3z=-3} \Leftrightarrow \heva{&x+2z=-2\\&y-5z=5\\&y-5z=5} \Leftrightarrow \heva{&x=-2z-2\\&y=5z+5.}$$
	Vậy hệ phương trình đã cho có vô số nghiệm và tập nghiệm của nó là
	\[S=\left\{(-2z-2;5z+5;z)\ \middle |\ z\in\mathbb{R}\right\}.\]
\end{enumerate}
}
\end{bt}
\begin{bt}%KNTT
	Hà mua văn phòng phẩm cho nhóm bạn cùng lớp gồm Hà, Lan và Minh hết tổng cộng $820$ nghìn đồng. Hà quên không lưu hóa đơn của mỗi bạn, nhưng nhớ được rằng số tiền trả cho Lan ít hơn một nửa số tiền trả cho Hà là 5 nghìn đồng, số tiền trả cho Minh nhiều hơn số tiền trả cho Lan là 210 nghìn đồng. Hỏi mỗi bạn Lan và Minh phải trả cho Hà bao nhiêu tiền?
	\loigiai{
Gọi số tiền mua văn phòng phẩm của Hà, Lan và Minh lần lượt là $x$, $y$, $z$ (nghìn đồng), với điều kiện $x,y,z\geq 0$.\\
Vì tổng số tiền phải trả của cả ba bạn là $820$ nghìn đồng nên 
\[x+y+z=820.\]
Số tiền trả cho Lan ít hơn một nửa số tiền trả cho Hà là 5 nghìn đồng nên  
\[\dfrac{1}{2}x-y=5.\]
Số tiền trả cho Minh nhiều hơn số tiền trả cho Lan là 210 nghìn đồng nên 
\[-y+z=210.\]
Do đó, ta có hệ phương trình
\[\heva{&x+y+z=820\\&\dfrac{1}{2}x-y=5\\&-y+z=210}\Leftrightarrow \heva{&x+y+z=820\\&x=2y+10\\&z=y+210}	\Leftrightarrow \heva{&4y+220=820\\&x=2y+10\\&z=y+210} \Leftrightarrow \heva{&y=150\\&x=310\\&z=360.}\]
Vậy số tiền mua văn phòng phẩm của Hà, Lan và Minh lần lượt là $310$ nghìn đồng, $150$ nghìn đồng và $360$ nghìn đồng.
}
\end{bt}
\begin{bt}%CTST
Giải các hệ phương trình sau bằng phương pháp Gauss:
\begin{listEX}[3]
	\item $\heva{&x-2y=1\\&x+2y-z=-2\\&x-3y+z=3;}$
	\item $\heva{&3x-y+2z=2\\&x+2y-z=1\\&2x-3y+3z=2;}$
	\item $\heva{&x-y+z=0\\&x-4y+2z=-1\\&4x-y+3z=1.}$
\end{listEX}
\loigiai{
\begin{enumerate}
	\item 
	    Ta có
	    \[\begin{aligned}
	   & \heva{&x-2y=1\\&x+2y-z=-2\\&x-3y+z=3} \Leftrightarrow \heva{&x-2y=1\\&4y-z=-3\\&-y+z=2} \Leftrightarrow \heva{&x-2y=1\\&4y-z=-3\\&3z=5}\\
	   \Leftrightarrow & \heva{&x-2y=1\\&4y-z=-3\\&z=\dfrac{5}{3}} \Leftrightarrow \heva{&x-2y=1\\&y=-\dfrac{1}{3}\\&z=\dfrac{5}{3}} \Leftrightarrow \heva{&x=\dfrac{1}{3}\\&y=-\dfrac{1}{3}\\&z=\dfrac{5}{3}.}
	    \end{aligned}\]
	    Vậy hệ phương trình có nghiệm là $\left(\dfrac{1}{3};-\dfrac{1}{3};\dfrac{5}{3}\right)$.
	    \item Ta có 
	    \[\heva{&3x-y+2z=2\\&x+2y-z=1\\&2x-3y+3z=2} \Leftrightarrow \heva{&x+2y-z=1\\&3x-y+2z=2\\&2x-3y+3z=2} \Leftrightarrow \heva{&x+2y-z=1\\&-7y+5z=-1\\&-7x+5z=0.}\]
	    Từ phương trình thứ hai và thứ ba, suy ra $-1=0$, điều này vô lí.\\
	    Vậy phương trình đã cho vô nghiệm.
	    \item Ta có
	    \[ \heva{&x-y+z=0\\&x-4y+2z=-1\\&4x-y+3z=1} \Leftrightarrow
	    \heva{&x-y+z=0\\&-3y+z=-1\\&3y-z=1} \Leftrightarrow \heva{&x-y+z=0\\&z=3y-1} \Leftrightarrow \heva{&x=-2y+1\\&z=3y-1.}\]
	   Vậy hệ phương trình đã cho có vô số nghiệm và tập nghiệm của nó là
	   \[ S=\left\{(-2y+1;y;3y-1) \ \middle |\ y\in \mathbb{R} \right\}.\]
\end{enumerate}
}
\end{bt}
\begin{bt}
	Tìm phương trình của parabol $(P)\colon y=ax^2+bx+c$ $(a\neq 0)$, biết $(P)$ đi qua ba điểm $A(0;-1)$, $B(1;-2)$ và $C(2;-1)$.
	\loigiai{
$(P)$ đi qua $A(0;-1)$ nên $a\cdot 0^2+b\cdot 0 +c=-1$ hay $c=-1$.\\
$(P)$ đi qua $B(1;-2)$ nên $a\cdot 1^2+b\cdot 1+c=-2$ hay $a+b+c=-2$.\\
$(P)$ đi qua $C(2;-1)$ nên $a\cdot 2^2+b\cdot 2+c=-2$ hay $4a+2b+c=-1$.\\
Do đó ta có hệ phương trình 
\[\heva{&c=-1\\&a+b+c=-2\\&4a+2b+c=-1.}\]
Giải hệ này ta được $a=1$; $b=-2$; $c=-1$.\\
Vậy phương trình của $(P)$ là $y=x^2-2x-1$.	
}
\end{bt}
\begin{bt}%CD
	Giải các hệ phương trình sau:
	\begin{listEX}[3]
		\item $\heva{&4x+y-3z=11\\&2x-3y+2z=9\\&x+y+z=-3;}$
		\item $\heva{&x+2y+6z=5\\&-x+y-2z=3\\&x-4y-2z=13;}$
		\item $\heva{&x+y-3z=-1\\&y-z=0\\&-x+2y=1.}$
	\end{listEX}
\loigiai{
\begin{enumerate}
	\item Ta có
	\[\begin{aligned}
	&\heva{&4x+y-3z=11\\&2x-3y+2z=9\\&x+y+z=-3} \Leftrightarrow \heva{&x+y+z=-3\\&2x-3y+2z=9\\&4x+y-3z=11} \Leftrightarrow \heva{&x+y+z=-3\\&-3y-7z=23\\&-5y=15}\\
	 \Leftrightarrow &\heva{&x+y+z=-3\\&-3y-7z=23\\&y=-3} %\Leftrightarrow \heva{&x+y+z=-3\\&-3\cdot(-3)-7z=-2\\&y=-3} 
	 \Leftrightarrow \heva{&x+y+z=-3\\&z=-2\\&y=-3} 
	 %\Leftrightarrow \heva{&x-3-2=-3\\&y=-3\\&z=-2}
	 \Leftrightarrow\heva{&x=2\\&y=-3\\&z=-2.}
	\end{aligned}\]
	Vậy phương trình đã cho có nghiệm là $(2;-3;-2)$.
	\item Ta có 
	\[\heva{&x+2y+6z=5\\&-x+y-2z=3\\&x-4y-2z=13}\Leftrightarrow \heva{&x+2y+6z=5\\&3y+4z=8\\&-6y-8z=8}\Leftrightarrow \heva{&x+2y+6z=5\\&3y+4z=8\\&3y+4z=-4.} \]
	Từ phương trình thứ hai và thứ ba, suy ra $8=-4$, điều này vô lí.\\
	Vậy phương trình đã cho vô nghiệm.
	\item Ta có
	\[\heva{&x+y-3z=-1\\&y-z=0\\&-x+2y=1}\Leftrightarrow \heva{&x+y-3z=-1\\&y-z=0\\&3y-3z=0}\Leftrightarrow  \heva{&x+y-3z=-1\\&y=z}\Leftrightarrow \heva{&x=2z-1\\&y=z.}\]
	Vậy hệ phương trình đã cho có vô số nghiệm và tập nghiệm của nó là
	\[S=\left\{(2z-1;z;z)\ \middle | \ z\in\mathbb{R}\right\}.\]
\end{enumerate}
}
\end{bt}
%Bài tập phần 3
\begin{bt}%CTST
	Sử dụng máy tính cầm tay, tìm nghiệm của các hệ phương trình sau:
	\begin{listEX}[3]
		\item $\heva{&2x+y-z=-1\\&x+3y+2z=2\\&3x+3y-3z=-5;}$
		\item $\heva{&2x-3y+2z=5\\&x+2y-3z=4\\&3x-y-z=2;}$
		\item $\heva{&x-y-z=-1\\&2x-y+z=-1\\&-4x+3y+z=3}$
	\end{listEX}
\loigiai{
\begin{enumerate}
	\item Nghiệm của hệ phương trình là $\left(\dfrac{2}{3};-\dfrac{2}{3};\dfrac{5}{3}\right)$.
	\item Hệ phương trình vô nghiệm.
	\item Hệ phương trình có vô số nghiệm.
\end{enumerate}
}
\end{bt}
\begin{bt}%CD
	Sử dụng máy tính cầm tay để tìm nghiệm của hệ phương trình:
	\[\heva{&2x-3y+4z=-5\\&-4x+5y-z=6\\&3x+4y-3z=7.}\]
	\loigiai{
	Nghiệm của hệ phương trình là $\left(\dfrac{22}{101};\dfrac{131}{101};-\dfrac{39}{101}\right)$.
}
\end{bt}
\begin{bt}%CTST
	Ba bạn Nhân, Nghĩa và Phúc đi vào căng tin của trường. Nhân mua một li trà sữa, một li nước trái cây, hai cái bánh ngọt và trả $90\,000$ đồng. Nghĩa mua một li trà sữa, ba cái bánh ngọt và trả $50\,000$ đồng. Phúc mua một li trà sữa, hai li nước trái cây, ba cái bánh ngọt và trả $140\,000$ đồng. Gọi $x$, $y$, $z$ lần lượt là giá tiền của một li trà sữa, một li nước trái cây và một cái bánh ngọt tại căng tin đó.
	\begin{enumerate}
		\item Lập các hệ thức thể hiện mối liên hệ giữa $x$, $y$ và $z$.
		\item Tìm giá tiền của một li trà sữa, một li nước trái cây và một cái bánh ngọt tại căng tin đó.
	\end{enumerate}
\loigiai{
	\begin{enumerate}
		\item Vì Nhân mua một li trà sữa, một li nước trái cây, hai cái bánh ngọt và trả $90\,000$ đồng nên ta có 
		\[x+y+2z=90000.\]
		Vì Nghĩa mua một li trà sữa, ba cái bánh ngọt và trả $50\,000$ đồng nên ta có
		\[x+3z=50000.\]
		Vì Phúc mua một li trà sữa, hai li nước trái cây, ba cái bánh ngọt và trả $140\,000$ đồng nên ta có
		\[x+2y+3z=140000\]
		Từ đó, ta có hệ phương trình bậc nhất ba ẩn:
		\[\heva{&x+y+2z=90000\\&x+3z=50000\\&x+2y+3z=140000.}\]
		\item Sử dụng máy tính cầm tay để giải hệ phương trình trên, ta được $(35000;45000;5000)$ là nghiệm của hệ phương trình.\\
		Vậy giá tiền của một li trà sữa, một li nước trái cây và một cái bánh ngọt tại căng tin đó lần lượt là $35\,000$ đồng, $45\,000$ đồng và $5\,000$ đồng.
	\end{enumerate}
}
\end{bt}
\begin{bt}%KNTT
	Tại một quốc gia, khoảng 400 loài động vật nằm trong danh sách các loài có nguy cơ tuyệt chủng. Các nhóm động vật có vú, chim và cá chiếm $55\%$ các loài có nguy cơ tuyệt chủng. Nhóm chim chiếm nhiều hơn $0{,}7\%$ so với nhóm cá, nhóm cá chiếm nhiều hơn $1{,}5\%$ so với động vật có vú. Hỏi mỗi nhóm động vật có vú, chim và cá chiếm bao nhiều phần trăm trong các loài có nguy cơ tuyệt chủng?
	\loigiai{
		Gọi $x$, $y$, $z$ lần lượt là số phần trăm nhóm động vật có vú, chim và cá có nguy cơ tuyệt chủng (điều kiện $x,y,z\geq0$).\\
		Các nhóm động vật có vú, chim và cá chiếm $55\%$ các loài có nguy cơ tuyệt chủng nên
		\[x+y+z=55.\]
		Nhóm chim chiếm nhiều hơn $0{,}7\%$ so với nhóm cá nên
		\[y-z=0{,}7.\]
		Nhóm cá chiếm nhiều hơn $1{,}5\%$ so với động vật có vú nên
		\[z-x=1{,}5.\]
		Do đó, ta có hệ phương trình
		\[\heva{&x+y+z=55\\&y-z=0{,}7\\&-x+z=1{,}5.}\]
		Giải hệ phương trình này, ta được $x=17{,}1$; $y=19{,}3$ và $z=18{,}6$.\\
		Vậy nhóm động vật có vú chiếm $17{,}1\%$; nhóm chim chiếm $19{,}3\%$ và nhóm cá chiếm $18{,}6\%$ các loài có nguy cơ tuyệt chủng.
	}
\end{bt}