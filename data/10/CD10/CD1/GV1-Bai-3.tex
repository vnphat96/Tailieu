\section{BÀI TẬP HỆ PHƯƠNG TRÌNH BẬC NHẤT 3 ẨN}
% \subsection{Bài tập tự luận}
\begin{bt} %[0D3Y3-3]
Trong các hệ phương trình sau, hệ nào là hệ phương trình bậc nhất ba ẩn? Mỗi bộ ba số $(-1;0;1)$, $\left(\dfrac{1}{2} ;-\dfrac{1}{2} ;-1\right)$ có là nghiệm của các hệ phương trinh bậc nhất ba ẩn đó không?
\begin{listEX}[3]
\item $\heva{&2x-y+z=-1 \\& -x+2y=1 \\ &3y-2z=-2};$
\item $\heva{& 4x-2y+z=2 \\& 8x+3z=1  \\ & -6y+2z=1};$
\item $\heva{&3x-2y+zx=2 \\& xy-y+2z=1 \\ & x+2y-3yz=-2}.$
\end{listEX}
\loigiai{
a) và b) là các hệ phương trình bậc nhất ba ẩn vì các phương trình trong hệ phương trình a) và b) đều có dạng $ax+by+cz=d$ trong đó $a^2+b^2+c^2>0$.\\
c) không phải hê phương trình bậc nhất ba ẩn vì chứa ẩn $zx,xy,yz$.\\
$\circ$ Bộ ba số $(-1 ; 0 ; 1)$ là nghiệm của hệ a) vì khi thay bộ số này vào từng phương trình của a) thì chúng đều có nghiệm đúng. \\
Thật vậy, 
$2\cdot (-1)-0+1=-1$, $-(-1)+2\cdot0=1$, $3\cdot 0-2
\cdot 1=-2$.\\
$\circ$ Bộ ba số $\left(\dfrac{1}{2} ;-\dfrac{1}{2} ;-1\right)$ không là nghiệm của hệ a) vì khi thay bộ số này vào phương trình thứ nhất của hệ ta được $2 \cdot \dfrac{1}{2}-\left(-\dfrac{1}{2}\right)+(-1)=-1$, đây là đẳng thức sai.\\
$\circ$ Bộ ba số $(-1;0;1)$ không là nghiệm của hệ b) vì khi thay bộ số này vào phương trình thứ nhất của hệ ta được $4 \cdot (-1)-2\cdot 0+1=2$, đây là đẳng thức sai.\\
$\circ$ Bộ ba số $\left(\dfrac{1}{2} ;-\dfrac{1}{2} ;-1\right)$ là nghiệm của hệ b) vì khi thay bộ số này vào từng phương trình thì chúng đều có nghiệm đúng.\\
Thật vậy, 
$4 \cdot \dfrac{1}{2}-2\left(-\dfrac{1}{2}\right)+(-1)=2$, $8 \cdot \dfrac{1}{2}+3 \cdot(-1)=1$, $-6\left(-\dfrac{1}{2}\right)+2 \cdot(-1)=1$.}
\end{bt}
\begin{bt} %[0D3B3-3]
Giải các hệ phương trình sau bằng phương pháp Gauss:
\begin{listEX}[3]
\item $\heva{&x-2 y+z=3 \\& -y+z=2 \\ &y+2 z=1};$
\item $\heva{& 3x-2y-4z =3 \\& 4x+6y-z =17  \\ & x+2y=5};$
\item $\heva{&x+y+z=1 \\& 3 x-y-z=4 \\ & x+5 y+5 z=-1}.$
\end{listEX}
\loigiai{
a) $\heva{&x-2 y+z=3 \\& -y+z=2 \\ &y+2 z=1} \Leftrightarrow \heva{&x-2 y+z=3 \\& -y+z=2 \\ &3z=3} \Leftrightarrow \heva{&x-2 y+z=3 \\& -y+1=2 \\ &z=1}\Leftrightarrow \heva{&x-2\cdot (-1)+1=3 \\& y=-1 \\ &z=1}$\\
$\Leftrightarrow \heva{&x=0 \\& y=-1 \\ &z=1}$.\\
Vậy hệ phương trình đã cho có nghiệm duy nhất là $(0;-1;1)$.\\
b) $\heva{& 3x-2y-4z =3 \\& 4x+6y-z =17  \\ & x+2y=5} \Leftrightarrow \heva{& 3x-2y-4z =3 \\& -13x-26y=-65  \\ & x+2y=5}\Leftrightarrow \heva{& 3x-2y-4z =3 \\& x+2y=5  \\ & x+2y=5}\Leftrightarrow \heva{& 3x-2y-4z =3 \\& x+2y=5}.$\\
Từ phương trình thứ hai ta có $x=-2y+5$, thay vào phương trình thứ nhất ta được $z=-2y+3$.\\
Vậy hệ phương trình đã cho có vô số nghiệm dạng $(-2y+5;y;-2y+3)$.\\
c) $\heva{&x+y+z=1 \\& 3x-y-z=4 \\ & x+5y+5z=-1} \Leftrightarrow \heva{&x+y+z=1 \\& 4y+4z=-1\\ & x+5y+5z=-1} \Leftrightarrow \heva{&x+y+z=1 \\& 4y+4z=-1\\ & -4y-4z=2} \Leftrightarrow \heva{&x+y+z=1 \\& 4y+4z=-1\\ & 0y+0z=1}.$\\
Vì phương trình thứ ba của hệ vô nghiệm nên hệ đã cho vô nghiệm.}
\end{bt}
\begin{bt}%%[0D3B3-3]
	Giải các hệ phương trình sau:
		\begin{enumerate}
	\item $\heva{&x+y+z=6 \\&x+2y+3z=14 \\&3x-2y-z=-4}$;
	\item $\heva{&2x-2y+z=6 \\&3x+2y+5z=7 \\&7x+3y-6z=1}$;
	\item $\heva{&2x+y-6z=1 \\&3x+2y-5z=5 \\&7x+4y-17z=7}$;
	\item $\heva{&5x+2y-7z=6 \\&2x+3y+2z=7 \\&9x+8y-3z=1}$.
		\end{enumerate}
	\loigiai{
		\begin{enumerate}
			\item $\heva{&x+y+z=6 \\&x+2y+3z=14 \\&3x-2y-z=-4}\Leftrightarrow \heva{&x+y+z=6 \\&2x+y=4 \\&4x-y=2}\Leftrightarrow \heva{&x+y+z=6 \\&2x+y=4 \\&x=1}\Leftrightarrow \heva{&x=1 \\&y=2 \\&z=3}$.\\	
			Vậy hệ phương trình có nghiệm là $\left( x;y;z \right)=\left( 1;2;3 \right)$.
			\item $\heva{&2x-2y+z=6 \\&3x+2y+5z=7 \\&7x+3y-6z=1}\Leftrightarrow \heva{&2x-2y+z=6 \\&7x-12y=23 \\&19x-9y=37}\Leftrightarrow \heva{&2x-2y+z=6 \\&7x-12y=23 \\&-55x=-79}\Leftrightarrow \heva{&x=\dfrac{79}{55} \\&y=-\dfrac{178}{165} \\&z=\dfrac{32}{33}}$.
			Vậy hệ phương trình có nghiệm là $(x;y;z)=\left( \dfrac{79}{55};-\dfrac{178}{165};\dfrac{32}{33} \right)$.
			\item $\heva{&2x+y-6z=1 \\&3x+2y-5z=5 \\&7x+4y-17z=7}\Leftrightarrow \heva{&2x+y-6z=1 \\&-8x-7y=-25 \\&-8x-7y=-25}\Leftrightarrow \heva{&x=x_0 \\&y=\dfrac{25-8x_0}{7} \\&z=\dfrac{6x_0+18}{42}}\left( x_0\in \mathbb{R} \right)$.	\\
			Vậy hệ phương trình có vô số nghiệm dạng $\left( x;y;z \right)=\left( x_0;\dfrac{25-8x_0}{7};\dfrac{6x_0+18}{42} \right)\left( x_0\in \mathbb{R} \right)$.	
			\item $\heva{&5x+2y-7z=6 \\&2x+3y+2z=7 \\&9x+8y-3z=1}\Leftrightarrow \heva{&5x+2y-7z=6 \\&24x+25y=61 \\&-48x-50y=11}\Leftrightarrow \heva{&5x+2y-7z=6 \\&24x+25y=61 \\&0x+0y=133.}$.\\
			Vậy hệ phương trình đã cho vô nghiệm.
		\end{enumerate}
	}
\end{bt}

\begin{bt}%
Tìm các số thực $A$, $B$ và $C$ thỏa mãn $\dfrac{1}{x^3+1}=\dfrac{A}{x+1}+\dfrac{Bx+C}{x^2-x+1}$.
	\loigiai{
Ta có:
\begin{eqnarray*}
\dfrac{A}{x+1}+\dfrac{Bx+C}{x^2-x+1}&=&\dfrac{A.\left( x^2-x+1 \right)+\left( Bx+C \right)\left( x+1 \right)}{\left( x+1 \right)\left( x^2-x+1 \right)}\\
&=&\dfrac{\left( A+B \right)x^2+\left( -A+B+C \right)x+A+C}{x^3+1}.
\end{eqnarray*}
Vì $\dfrac{1}{x^3+1}=\dfrac{A}{x+1}+\dfrac{Bx+C}{x^2-x+1}$ nên ta suy ra\\
 \[\heva{&A+B=0 \\&-A+B+C=0 \\&A+C=1}\Leftrightarrow \heva{&A=\dfrac{1}{3} \\&B=-\dfrac{1}{3} \\&C=\dfrac{2}{3}.}\]
Vậy $A=\dfrac{1}{3},B=-\dfrac{1}{3}$ và $C=\dfrac{2}{3}$.
}
\end{bt}

\begin{bt}%%[0D2K3-2]
	Tìm parabol $y=ax^2+bx+c$ trong mỗi trường hợp sau:
	\begin{enumerate}
		\item Parabol đi qua ba điểm $A\left( 2;-1 \right),B\left( 4;3 \right)$ và $C\left( -1;8 \right)$.
		\item Parabol nhận đường thẳng $x=\dfrac{5}{2}$ làm trục đối xứng và đi qua hai điểm $M(1;0),N(5;-4)$.
	\end{enumerate}
	\loigiai{
	\begin{enumerate}
	\item Parabol đi qua ba điểm $A( 2;-1),B( 4;3 )$ và $C( -1;8 )$ nên ta có hệ: $\heva{&4a+2b+c=-1 \\&16a+4b+c=3 \\&a-b+c=8}$.\\
	Giải hệ trên ta được $a=1,b=-4,c=3$.\\
	Vậy parabol cần tìm là  $y=x^2-4x+3$.
	\item Parabol nhận đường thẳng $x=\dfrac{5}{2}$ làm trục đối xứng và đi qua hai điểm $M(1;0),\ N(5;-4)$ nên ta có hệ:
	$\heva{&-\dfrac{b}{2a}=\dfrac{5}{2} \\&a+b+c=0 \\&25a+5b+c=-4}\Leftrightarrow \heva{&5a+b=0 \\&a+b+c=0 \\&25a+5b+c=-4}$.\\
	Giải hệ trên ta được $a=-1,b=5$ và $c=-4$.\\
	Vậy parabol cần tìm là  $y=-x^2+5x-4$.
	\end{enumerate}
	}
\end{bt}
\begin{bt} %[0D3K3-3]
Tìm phương trình của parabol $(P) \colon y=ax^2+bx+c$ $(a \neq 0)$, biết:
\begin{enumerate}[a)]
\item Parabol $(P)$ cắt trục hoành tại hai điểm phân biệt có hoành độ lần lượt là $x=-2$; $x=1$ và đi qua điểm $M(-1 ; 3)$;
\item Parabol $(P)$ cắt trục tung tại điểm có tung độ $y=-2$ và hàm số đạt giá trị nhỏ nhất bằng $-4$ taii $x=2$.
\end{enumerate}
\loigiai{
a) $(P)$ cắt trục hoành tại hai điểm phân biệt có hoành độ lần lượt là $x=-2$; $x=1$ nên ta có hệ hai phương trình bậc nhất ba ẩn:
$$\heva{&a\cdot (-2)^2 + b\cdot (-2) + c=0\\&0 = a \cdot 1^2 + b\cdot 1 + c=0} \Leftrightarrow \heva{&4a-2b+c=0 \,\, (1)\\&a+b+c=0 \,\, (2)}$$
$(P)$ đi qua điểm $\mathrm{M}(-1; 3)$ nên $3=a\cdot (-1)^2+b\cdot (-1)+ c \Leftrightarrow a-b+c=3 \,\, (3)$.\\
Từ $(1), (2)$ và $(3)$ ta có hệ phương trình: $\heva{&4a-2b+c=0\\&a+b+c=0\\&a-b+c=3}$.\\
Giải hệ này ta được $a=-\dfrac{3}{2}$, $b=-\dfrac{3}{2}$, $c=3$.\\
Vậy phương trình của $(P)$ là $y=-\dfrac{3}{2} x^2-\dfrac{3}{2} x+3$.\\
b) $(P)$ cắt trục tung tại điểm có tung độ $=-2$ nên $a \cdot 0^2+b \cdot 0+c=-2$ hay $c=-2 \,\, (1)$.\\
Hàm số đạt giá trị nhỏ nhất bằng $-4$ tại $x=2$ nên 
$$\heva{&-\dfrac{b}{2a}=2\\&a\cdot 2^2+b\cdot 2+c=-4} \Leftrightarrow
\heva{&4a+b=0 \,\, (2) \\&4a+2b+c=-4 \,\, (3)}$$
Từ $(1), (2)$ và $(3)$ ta có hệ phương trình $\heva{&c=-2\\&4a+b=0\\&4a+2b+c=-4}$.\\
Giải hệ này ta được $a=\dfrac{1}{2}, b=-2, c=-2$.\\
Vậy phương trình của $(P)$ là $y=\dfrac{1}{2}x^2-2x-2$.}
\end{bt}
% \begin{bt}%%[0H3K2-2]
% 	Trong mặt phẳng tọa độ, viết phương trình đường tròn đi qua ba điểm $A( 0;1)$, $B(2;3 )$ và $C(4;1)$.
% 	\loigiai{
% 	Phương trình đường tròn có dạng: $x^2+y^2-2ax-2by+c=0$.	\\
% 	Đường tròn đi qua ba điểm $A( 0;1),\ B(2;3 )$ và $C(4;1)$ nên ta có hệ:\\
% 	$\heva{&0^2+1^2-2.0.a-2.1.b+c=0 \\&2^2+3^2-2.2.a-2.3.b+c=0 \\&4^2+1^2-2.4.a-2.1.b+c=0}\Leftrightarrow \heva{&-2b+c=-1 \\&-4a-6b+c=-13 \\&-8a-2b+c=-17}\Leftrightarrow \heva{&a=2 \\&b=1 \\&c=1}$.\\
% Vậy phương trình đường tròn cần tìm là $x^2+y^2-4x-2y+1=0$.}
% \end{bt}

\begin{bt} %[0D3T3-3]
Một viên lam ngọc và hai viên hoàng ngọc trị giá gấp $3$ lần một viên ngọc bích. Còn bảy viên lam ngọc và một viên hoàng ngọc trị giá gấp $8$ lần một viên ngọc bích. Biết giá tiền của bộ ba viên ngọc này là $270$ triệu đồng. Tính giá tiền mỗi viên ngọc.
\loigiai{
Gọi giá tiễn mỗi viên lam ngọc, hoàng ngọc, ngọc bích lần lượt là $x$, $y$, $z$ (triệu đồng).\\
Điều kiện: $0<x,y,z<270$.\\
Theo đề bài ta có:\\
$\circ$ Một viên lam ngọc và hai viên hoàng ngọc trị giá gấp $3$ lần một viên ngọc bích, suy ra $x+2y=3z$ hay $x+2y-3z=0 \,\, (1)$.\\
$\circ$ Bảy viên lam ngọc và một viên hoàng ngọc trị giá gấp $8$ lần một viên ngọc bích, suy ra $7x+y=8z$ hay $7x+y-8z=0 \,\, (2)$.\\
$\circ$ Giá tiền của bộ ba viên ngọc là $270$ triệu đồng , suy ra $x+y+z=270 \,\, (3)$.\\
Từ $(1), (2)$ và $(3)$ ta có hệ phương trình $\heva{&x+2y-3z=0\\&7x+y-8z=0\\&x+y+z=270}$.\\
Giải hệ này ta được $x=90, y=90, z=90$.\\
Vậy giá tiền mỗi viên ngọc đều là $90$ triệu đồng.}
\end{bt}
\begin{bt} %[0D3T3-3]
Bốn ngư dân góp vốn mua chung một chiếc thuyền. Số tiền người đầu tiên đóng góp bằng một nửa tổng số tiền của những người còn lại. Người thứ hai đóng góp bằng $\dfrac{1}{3}$ tổng số tiền của những người còn lại. Người thứ ba đóng góp bằng $\dfrac{1}{4}$ tổng số tiền của những người còn lại. Người thứ tư đóng góp $130$ triệu đồng. Chiếc thuyền này được mua giá bao nhiêu?
\loigiai{
Gọi số tiền người thứ nhất, người thứ hai, người thứ ba đóng góp lần lượt là $x$, $y$, $z$ (triệu đồng).\\
Điều kiện: $x,y,z >0$.\\
Theo đề bài ta có:\\
$\circ$ Số tiền người đầu tiên đóng góp bằng một nửa tổng số tiền của những người còn lại, suy ra $x=\dfrac{1}{2}(y+z+130)$ hay $2x-y-z=130 \,\, (1)$.\\
$\circ$ Người thứ hai đóng góp bằng $\dfrac{1}{3}$ tổng số tiền của những người còn lại, suy ra $y=\dfrac{1}{3}(x+z+130)$ hay $-x+3y-z=130 \,\, (2)$.\\
$\circ$ Người thứ ba đóng góp bằng $\dfrac{1}{4}$ tổng số tiền của những người còn lại, suy ra $z=\dfrac{1}{4}(x+y+130)$ hay $-x-y+4z=130 \,\, (3)$.\\
Từ $(1), (2)$ và $(3)$ ta có hệ phương trình $\heva{&2x-y-z=130\\&-x+3y-z=130\\&-x-y+4z=130}$.\\
Giải hệ này ta được $x=200$, $y=150$, $z=120$.\\
Suy ra tổng số tiền là $200+150+120+130=600$ (triệu đồng).\\
Vậy chiếc thuyền này được mua với giá $600$ triệu đồng.}
\end{bt}
\begin{bt} %[0D3T3-3]
Một quỹ đầu tư dự kiến dành khoản tiền $1,2$ tỉ đồng để đầu tư vào cồ phiếu. Để thấy được mức độ rủi ro, các cổ phiếu được phân thành ba loại: rủi ro cao, rủi ro trung bình và rủi ro thấp. Ban Giám đốc của quỹ ước tính các cổ phiếu rủi ro cao, rủi ro trung bình và rủi ro thấp sẽ có lợi nhuận hằng năm lần lượt là $15 \%$, $10 \%$ và $6 \%$. Nếu đặt ra mục tiêu đầu tư có lợi nhuận trung bình là $9 \%$ /năm trên tổng số vốn đầu tư, thì quỹ nên đầu tư bao nhiêu tiền vào mỗi loại cổ phiếu? Biết rằng, để an toàn, khoản đầu tư vào các cổ phiếu rủi ro thấp sẽ gấp đôi tổng các khoản đầu tư vào các cổ phiếu thuộc hai loại còn lại.
\loigiai{
Gọi số tiền nên đầu tư vào mỗi loại cổ phiếu rủi ro cao, rủi ro trung bình và rủi ro thấp lần lượt là $x$, $y$, $z$ (tỉ đồng).\\
Điều kiện: $0 \le x,y,z  \le 1,2$.\\
Theo đề bài ta có:\\
$\circ$ Tổng số tiền đầu tư là $1,2$ tỉ đồng, suy ra $x+y+z=1,2 \,\, (1)$.\\
$\circ$ Mục tiêu đầu tư có lợi nhuận trung bình là $9 \%$ /năm trên tổng số vốn đầu tư, suy ra $15 \% x+10 \% y+6 \% z=9 \% \cdot 1,2$ hay $15x+10y+6z=10,8 \,\, (2)$.\\
$\circ$ Khoản đầu tư vào các cổ phiếu rủi ro thấp sẽ gấp đôi tổng các khoản đầu tư vào các cổ phiếu thuộc hai loại còn lại, suy ra $z=2(x+y)$ hay $2x+2y-z=0 \,\, (3)$.\\
Từ $(1), (2)$ và $(3)$ ta có hệ phương trình $\heva{&x+y+z=1,2\\&15x+10y+6z=10,8\\&2x+2y-z=0}$.\\
Giải hệ này ta được $x=0,4$, $y=0$, $z=0,8$.\\
Vậy số tiền nên đầu tư vào mỗi loại cổ phiếu rủi ro cao, rủi ro trung bình và rủi ro thấp lần lượt là $0,4$ tỉ đồng, $0$ đồng, $0,8$ tỉ đồng.}
\end{bt}
\begin{bt} %[0D3T3-3]
Ba loại tế bào $A, B, C$ thực hiện số lần nguyên phân lần lượt là $3,4,5$ và tổng số tế bào con tạo ra là $216$. Biết rằng khi chưa thực hiện nguyên phân, số tế bào loại $C$ bằng trung bình cộng số tế bào loại $A$ và loại $B$. Sau khi thực hiện nguyên phân, tổng số tế bào con loại $A$ và loại $B$ được tạo ra ít hơn số tế bào con loại $C$ được tạo ra là $40$. Tính số tế bào con mỗi loại lúc ban đầu.
\loigiai{
Gọi số tế bào con ban đầu mỗi loại $A$, $B$, $C$ lần lượt là $x$, $y$, $z$.\\
Điều kiện: $x,y,z \in \mathbb{Z}^+$, $0<x,y,z <216$.\\
Theo đề bài ta có:\\
$\circ$ Ba loại tế bào $A$, $B$, $C$ thực hiện số lần nguyên phân lần lượt là $3,4,5$, suy ra số tế bào con mối loại $A$, $B$, $C$ lần lượt là $2^3 x, 2^4 y, 2^5 z$ hay $8x$, $16y$, $32z$.\\
$\circ$ Tổng số tế bào con tạo ra là $216$ , suy ra $8x+16y+32z=216$ hay $x+2y+4z=27 \,\ (1)$.\\
$\circ$ Khi chưa thực hiện nguyên phân, số tế bào loại $C$ bằng trung bình cộng số tế bào loại $A$ và loại $B$, suy ra $z=\dfrac{1}{2}(x+y)$ hay $x+y-2z=0 \,\, (2)$.\\
$\circ$ Sau khi thực hiện nguyên phân, tổng số tế bào con loại $A$ và loại $B$ được tạo ra ít hơn số tế bào con loại $C$ được tạo ra là $40$, suy ra $8x+16y=32z-40$ hay $x+2y-4z=-5 \,\, (3)$.\\
Từ $(1), (2)$ và $(3)$ ta có hệ phương trình $\heva{&x+2y+4z=27\\&x+y-2z=0\\&x+2y-4z=-5}$.\\
Giải hệ này ta được $x=5$, $y=3$, $z=4$.\\
Vậy số tế bào con ban đầu mỗi loại $A$, $B$, $C$ lần lượt là $5$, $3$ và $4$.}
\end{bt}
\begin{bt}%%[0D3K3-5]
	Một đoàn xe chở $225$ tấn gạo tiếp tế cho đồng bào vùng bị lũ lụt. Đoàn xe có $36$ chiếc gồm $3$ loại: xe chở $5$ tấn, xe chở $7$ tấn và xe chở $10$ tấn. Biết rằng tổng số hai loại xe chở $5$ tấn và $7$ tấn nhiều gấp ba lần số xe chở $10$ tấn. Hỏi mỗi loại xe có bao nhiêu chiếc?
	\loigiai{
		Gọi $x,y,z$ lần lượt là số xe chở $5$ tấn, xe chở $7$ tấn và xe chở $10$ tấn ($x,y,z\in \mathbb{N}; 0<x,y,z<36$).\\
		Theo đề ra ta có hệ phương trình: $\heva{&x+y+z=36 \\&x+y=3z \\&5x+7y+10z=255}$.\\
	Giải hệ trên ta được: $x=12,y=15,z=9$.\\
	Vậy đoàn xe có $12$ xe loại $5$ tấn, $15$ xe loại $7$ tấn và $9$ xe loại $10$ tấn.
	}
\end{bt}
\begin{bt}%%%[0D3K3-5]
	Bác An là chủ cửa hàng kinh doanh cà phê cho những người sành cà phê. Bác có ba loại cà phê nổi tiếng của Việt Nam: Arabica, Robusta và Moka với giá bán lần lượt là $302$ nghìn đồng/kg, $280$ nghìn đồng/ kg và $260$ nghìn đồng/ kg. Bác muốn trộn ba loại cà phê này để được một hỗn hợp cà phê, sau đó đóng thành các gói $1$kg, bán với giá $300$ nghìn đồng/ kg và lượng cà phê Moka gấp đôi lượng cà phê Robusta trong mỗi gói. Hỏi bác cần trộn ba loài cà phê theo tỉ lệ nào?
	\loigiai{
		Gọi $x,y,z$ lần lượt là tỉ lệ pha trộn cà phê Arabica, Robusta và Moka ($0\le x,y,z\le 1$).\\
		Theo đề ra ta có hệ phương trình: $\heva{&x+y+z=1 \\&z=2y \\&320x+280y+260z=300}$.\\
	Giải hệ trên ta được: $x=\dfrac{5}{8},y=\dfrac{1}{8},z=\dfrac{2}{8} \cdot$ \\
	Vậy tỉ lệ pha trộn cà phê Arabica, Robusta và Moka lần lượt là $\dfrac{5}{8},\ \dfrac{1}{8}$ và $\dfrac{2}{8}\cdot$ 
	}
\end{bt}
\begin{bt}%%%[0D3K3-5]
%	\immini{
		Bác Việt có $12$ ha đất canh tác để trồng ba loại cây: ngô, khoai tây và đậu tương. Chi phí trồng $1$ ha ngô là $4$ triệu đồng, $1$ ha khoai tây là $3$ triệu đồng và $1$ ha đậu tương là $4,5$ triệu đồng. Do nhu cầu thị trường, bác đã trồng khoai tây trên phần diện tích gấp đôi diện tích trồng ngô. Tổng chi phí trồng $3$ loại cây trên là $45,25$ triệu đồng. Hỏi diện tích trồng mỗi loại cây là bao nhiêu?
%	}
%	{\includegraphics{images/Picture1}
%	}
	\loigiai{
		Gọi diện tích trồng ngô, khoai tây, đậu tương lần lượt là $x,y,z$(ha).\\
		Điều kiện $0<x<12,\ 0<y<12,\ 0<z<12$.\\
		Từ dữ kiện bài toán ta lập được hệ phương trình: $\heva{&x+y+z=12 \\&y=2x \\&4x+3y+4,5z=45,25}$\\
	Giải hệ trên ta có $\heva{&x=2,5 \\&y=5 \\&z=4,5}$.\\
	Vậy diện tích trồng ngô, khoai tây, đậu tương của bác Việt lần lượt là: $2,5$(ha), $5$(ha), $4,5$(ha).
	}
\end{bt}

\begin{bt} %[0D3T3-3]
\immini{Cho sơ đồ mạch điện như Hình 1. Biết rằng $R = R_1 = R_2 = 5 \, \Omega$. Hãy tính các cường độ dòng điện $I$, $I_1$ và $I_2$.}{\begin{tikzpicture}
 \draw 
 (0,0.4)--(0,-0.4)
 (0.2,0.2)--(0.2,-0.2)
 (0.2,0)--(2,0)
 (2,-0.2) rectangle (3,0.2)
 (3,0)--(4,0)--(4,3)--(3.5,3)
 (3.5,4)--(3.5,2)--(1,2)
 (1,2.2) rectangle (0,1.8)
 (0,2)--(-1,2) 
 (3.5,4)--(1,4)
 (1,4.2) rectangle (0,3.8)
 (0,4)--(-1,4)--(-1,2)
 (-1,3)--(-2,3)--(-2,0)--(0,0)
 ;
 \draw
 (0.1,-0.7) node {$4 \,V$}
 (2.5,-0.5) node {$R$}
 (0.5,1.5) node {$R_2$}
 (0.5,4.2) node[above] {$R_1$}
 (-1,4) node[above] {$I_1$}
  (-1,1.5) node {$I_2$}
  (-1.5,3) node[above] {$I$}
 ; 
 \draw[->]
 (-1,4)--(-0.5,4)
 (-1,2)--(-0.5,2)
 (-2,3)--(-1.5,3)
 ;
 \end{tikzpicture}}
\loigiai{
Điều kiện: $I,I_1,I_2 >0$.\\
Tổng cường độ dòng điện ra vào vào tại điểm $B$ bằng nhau nên ta có $I = I_1 + I_2 \,\, (1)$.\\
Hiệu điện thế giữa hai điểm $A$ và $C$ được tính bởi:
$U_{AC} = IR + I_1R_1 = 5I + 5I_1$, suy ra $5I + 5I_1 = 4 \,\, (2)$.\\
Hiệu điện thế giữa hai điểm $B$ và $C$ được tính bởi:
$U_{BC} = I_1R_1 = I_2R_2$, suy ra $5I_1 = 5I_2$ hay $I_1 = I_2 \,\, (3)$.\\
Từ $(1), (2)$ và $(3)$ ta có hệ phương trình $\heva{&I - I_1 - I_2 =0\\&5I + 5I_1 = 4\\&I_1 - I_2=0}$.\\
Giải hệ này ta được $I=\dfrac{8}{15}$, $I_1=\dfrac{4}{15}$, $I_2=\dfrac{4}{15}$.}
\end{bt}
\begin{bt} %[0D3T3-3]
Cho $A$, $B$ và $C$ là ba dung dịch cùng loại acid có nồng độ khác nhau. Biết rằng nếu trộn ba dung dịch mỗi loại $100$ ml thì được dung dịch nồng độ $0,4$ M (mol/lít); nếu trộn $100$ ml dung dịch $A$ với $200$ ml dung dịch $B$ thì được dung dịch nồng độ $0,6$ M; nếu trộn $100$ ml dung dịch $B$ với $200$ ml dung dịch $C$ thì được dung dịch nồng độ $0,3$ M. Mỗi dung dịch $A$, $B$ và $C$ có nồng độ bao nhiêu?
\loigiai{Gọi nồng độ của mỗi dung dịch $A$, $B$, $C$ lần lượt là $x$, $y$, $z$ (M).\\
Điều kiện: $x,y,z >0$.\\
Theo đề bài ta có:\\
$\circ$ Nếu trộn ba dung dịch mỗi loại $100$ ml thì được dung dịch nồng độ $0,4$ M, suy ra $\dfrac{0,1x+0,1y+0,1z}{0,1+0,1+0,1}=0,4$ hay $x+y+z=1,2 \,\, (1)$.\\
$\circ$ Nếu trộn $100$ ml dung dịch $A$ với $200$ ml dung dịch $B$ thì được dung dịch nồng độ $0,6$ M, suy ra $\dfrac{0,1x+0,2y}{0,1+0,2}=0,6$ hay $x+2y=1,8 \,\, (2)$.\\
$\circ$ Nếu trộn $100$ ml dung dịch $B$ với $200$ ml dung dịch $C$ thì được dung dịch nồng độ $0,3$ M, suy ra $\dfrac{0,1y+0,2z}{0,1+0,2}=0,3$ hay $y+2z=0,9 \,\ (3)$.\\
Từ $(1), (2)$ và $(3)$ ta có hệ phương trình $\heva{&x+y+z=1,2\\&x+2y=1,8\\&y+2z=0,9}$.\\
Giải hệ này ta được $x=0,4$, $y=0,7$, $z=0,1$.\\
Vậy nồng độ của mỗi dung dịch $A$, $B$, $C$ lần lượt là $0,4$ M; $0,7$ M; $0,1$ M.}
\end{bt}
\begin{bt} %[0D3T3-3]
Xăng sinh học E5 là hỗn hợp xăng không chì truyền thống và cồn sinh học (bio – ethanol). Trong loại xăng này chứa $5 \%$ cồn sinh học. Khi động cơ đốt cháy lượng cồn trên thì xảy ra phản ứng hoá học
$$C_2H_6O + O_2 \buildrel {{t^\circ}} \over \longrightarrow  CO_2 + H_2O.$$
Cân bằng phương trình hoá học trên.
\loigiai{
Gọi $x$, $y$, $z$, $t$ lần lượt là bốn hệ số nguyên dương thoả mãn cân bằng phương trình phản ứng hoá học:
$$xC_2H_6O + yO_2 \buildrel {{t^\circ}} \over \longrightarrow  zCO_2 + tH_2O.$$
Điều kiện: $x,y,z \in \mathbb{Z}^+$.\\
Số nguyên tử $C$ ở hai vế bằng nhau, ta có $2x = z \,\,(1)$.\\
Số nguyên tử $H$ ở hai vế bằng nhau, ta có $6x = 2t$ hay $3x = t \,\,(2)$.\\
Số nguyên tử $O$ ở hai vế bằng nhau, ta có $x + 2y = 2z + t \,\,(3)$.\\
Thay $(1)$ và $(2)$ vào $(3)$ ta được $x+2y=2\cdot 2x+3x$ hay $y=3x$.\\
Vậy $y=3x$, $z=2x$, $t=3x$.\\
Để phương trình có hệ số đơn giản, ta chọn $x=1$, khi đó $y=3$, $z=2$, $t=3$.\\
Vậy phương trình cân bằng phản ứng hóa học là $C_2H_6O + 3O_2 \buildrel {{t^\circ}} \over \longrightarrow  2CO_2 + 3H_2O.$}
\end{bt}
\begin{bt}%%%[0D3K3-5]
	Cân bằng phương trình phản ứng hóa học sau $FeS_2+O_2\to Fe_2O_3+SO_2$
	\loigiai{
		Gọi $x,y,z,t$ là hệ số cân bằng lần lượt đứng trước $FeS_2,\ O_2,\ Fe_2O_3,\ SO_2$.\\
		Khi đó phương trình phản ứng có dạng $xFeS_2+yO_2\to zFe_2O_3+tSO_2$.\\
		Vì số nguyên tử của $Fe,S,O$ trước và sau phản ứng bằng nhau nên ta có hệ phương trình
	\[\heva{&x=2z \\&2x=t \\&2y=3z+2t}\Leftrightarrow \heva{&z=\dfrac{1}{2}x \\&t=2x \\&y=\dfrac{11}{4}x}\]
	 Chọn $x=4$ ta có $y=11,\ z=2,\ t=8$.\\
		Suy ra ta cân bằng phương trình hóa học như sau $4FeS_2+11O_2\to 2Fe_2O_3+8SO_2$.
	}
\end{bt}
\begin{bt}%%%[0D3K3-5]
	Bạn Mai có ba lọ dung dịch chứa một loại acid. Dung dịch $A$ chứa $10\%,$ dung dịch $B$ chứa $30\%$ và dung dịch $B$ chứa $50\%$ Bạn Mai lấy từ mỗi lọ dung dịch và hòa với nhau để có $50g$ hỗn hợp chứa $32\%$ acid này và lượng dung dịch loại $C$ lấy nhiều gấp đôi dung dịch loại $A$. Tính lượng dung dịch mỗi loại bạn Mai đã lấy.
	\loigiai{
		Gọi lượng dung dịch loại $A,\ B,\ C$ mà Mai đã lần lượt lấy ra là $x,\ y,\ z\ (0<x,y,z<50)$.
		Theo bài ra ta có hệ phương trình: $\heva{&x+y+z=50 \\&z=2x \\&\dfrac{1}{10}x+\dfrac{3}{10}y+\dfrac{5}{10}z=\dfrac{32}{100}.50}\Leftrightarrow \heva{&x+y+z=50 \\&z=2x \\&\dfrac{1}{10}x+\dfrac{3}{10}y+\dfrac{5}{10}z=16}$\\
	Giải hệ trên ta có $\heva{&x=5 \\&y=35 \\&z=10}$.\\
	Vậy dung dịch loại $A,\ B,\ C$ mà Mai đã lần lượt lấy ra là: $5$(g), $35$(g), $10$(g).
	}
\end{bt}
\begin{bt}%%[0D3K3-3]
	Cho đoạn mạch như hình vẽ
	\begin{center}
		\begin{circuitikz}[european,c/.style={circle,fill,inner sep=1pt}]
		\draw (0,0) coordinate (A) to(3,0)coordinate (B)  -- (3,1) to[R=$ R_1 $, i>_=$I_1$]  (7,1)
		(B) -- (3,-1)to[R=$ R_2 $, i>_=$I_2$]  (7,-1)--(7,1)
		(A) -- (0,-2.5)  to[battery2,l_=U] (8,-2.5) ;
		\draw (7,0) coordinate (C) to (8,0) to[R=$ R_3 $, i>_=$I_3$] (8,-2.5) ;
		%to[R=$ R_3$ ,i_=$I_1$] (8,0) -- (7,0) coordinate (C);
		\path foreach \p/\g in {}{(\p)node[c]{}+(\g:3.5mm) node{$\p$}};
	\end{circuitikz}
	\end{center}
	Biết $R_1=36\Omega,R_2=45\Omega,I_3=1,5\mathrm{A}$ là cường độ dòng điện trong mạch chính và hiệu điện thế giữa hai hai đầu đoạn mạch $U=60\mathrm{V} $ Gọi $I_1,I_2$ là cường độ dòng điện mạch rẽ. Tính $I_1,I_2$ và $R_3$.
	\loigiai{
		Gọi $U_1,U_2,U_3,U_{12}$ lần lượt là hiệu điện thế giữa hai đầu $R_1,R_2,R_3$ và đoạn mạch mắc song song.\\
		Khi đó từ sơ đồ mạch điện ta có: $\heva{&U_1=U_2=U_{12} \\&U_{12}+U_3=60}(*)$.\\
		Vì $R_1,R_2$ mắc song song nên $R_{12}=\dfrac{R_1.R_2}{R_1+R}=\dfrac{36.45}{36+45}=20$.\\
		Mặt khác $I_{12}=I_3=1,5$( mắc nối tiếp), suy ra $ {U_{12}}=I_{12}.R_{12}=1,5.20=30$.\\
		Theo $\left( * \right)$ ta suy ra $\heva{&U_1=U_2=U_{12}=30 \\&U_3=60-U_{12}=30}\Rightarrow \heva{&I_1=\dfrac{U_1}{R_1}=\dfrac{30}{36}=\dfrac{5}{6} \\&I_1=\dfrac{U_2}{R_2}=\dfrac{30}{45}=\dfrac{2}{3} \\&R_3=\dfrac{U_3}{I_3}=\dfrac{30}{1,5}=20}$ .\\
		Vậy $\heva{&I_1=\dfrac{5}{6}\left( A \right) \\&I_1=\dfrac{2}{3}\left( A \right) \\&R_3=20\left( \Omega \right)}$.
	}
\end{bt}
\begin{bt}%%%[0D3K3-5]
Giải bài toán dân gian sau
	\begin{align*}
	& \text{Em đi chợ phiên}\\
	& \text{Anh gửi một tiền}\\
	& \text{Cam, thanh yên, quýt}\\
	& \text{Không nhiều thì ít}\\
	& \text{Mua đủ một trăm}\\
	& \text{Cam ba đồng một}\\
	& \text{Quýt một đồng năm}\\
	& \text{Thanh yên tươi tốt}\\
	& \text{Năm đồng một trái}
	\end{align*}
	Hỏi mỗi thứ mua bao nhiêu trái, biết một tiền bằng $60$ đồng?
\loigiai{
	Gọi số cam, quýt, thanh yên lần lượt là: $x,\ y,\ z$ (quả), $( x,y,z\in{\mathbb{N}^{*}},x,y,z<100 )$.\\
	Theo đề bài ta lập được hệ phương trình: $\heva{&x+y+z=100\left( 1 \right) \\&3x+\dfrac{1}{5}y+5z=60\left( 2 \right)}$\\
	Từ $\left( 1 \right),\left( 2 \right)$ suy ra: $7x+12z=100\Leftrightarrow 7\left( x-16 \right)=-12\left( z+1 \right)$.\\
	Vì vậy $\heva{&x-16=-12k \\&z+1=7k}\left( k\in \mathbb{Z} \right)\Leftrightarrow \heva{&x=-12k+16 \\&z=7k-1}$.\\
	Để $x,z$ nguyên dương thì $k=1$ từ đó tìm được $x=4,y=90,z=6$.\\
	Vậy có $4$ quả cam, $90$ quả quýt và $6$ quả thanh yên.
		}
\end{bt}
\Closesolutionfile{ans}