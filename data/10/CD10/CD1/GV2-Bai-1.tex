\section{HỆ PHƯƠNG TRÌNH BẬC NHẤT BA ẨN}
\subsection{Tóm tắt lý thuyết}
(Phần này GV bỏ thầy cô hoàn thành tài để được tài liệu hoàn thiện)
\subsection{Bài tập rèn luyện}
\begin{bt}
	Kiểm tra xem mỗi bộ số $(x;\,y;\,z)$ đã cho có là nghiệm của hệ phương trình tương ứng hay không.
	\begin{enumerate}[a)]
		\item $\heva{&x+3y+2z=1\\&5x-y+3z=16\\&-3x+7y+z=-14}$ \qquad $(0;\,3;\,-2)$, $(12;\,5;\,-13)$, $(1;\,-2;\,3)$;
		\item $\heva{&3x-y+4z=-10\\&-x+y+2z=6\\&2x-y+z=-8}$ \qquad $(-2;\,4;\,0)$, $(0;\,-3;\,10)$, $(1;\,-1;\,5)$;
		\item $\heva{&x+y+z=100\\&5x+3y+\dfrac{1}{3}z=100}$ \qquad $(4;\,18;\,78)$, $(8;\,11;\,81)$, $(12;\,4;\,84)$.
	\end{enumerate}
	\loigiai{
		\begin{enumerate}[a)]
			\item Bộ $(1;\,-2;\,3)$ là nghiệm của hệ phương trình $\heva{&x+3y+2z=1\\&5x-y+3z=16\\&-3x+7y+z=-14.}$			
			\item Bộ $(-2;\,4;\,0)$ là nghiệm của hệ phương trình $\heva{&3x-y+4z=-10\\&-x+y+2z=6\\&2x-y+z=-8.}$
			\item Cả $3$ bộ $(4;\,18;\,78)$, $(8;\,11;\,81)$, $(12;\,4;\,84)$ là nghiệm của hệ phương trình $\heva{&x+y+z=100\\&5x+3y+\dfrac{1}{3}z=100.}$
		\end{enumerate}  
	}
\end{bt}

\begin{bt}
	Giải hệ phương trình
	\begin{enumerate}[a)]
		\item $\heva{&x-2y+4z=4\\&3y-z=2\\&2z=-10.}$
		\item $\heva{&4x+3y-5z=-7\\&2y=4\\&y+z=3.}$
		\item $\heva{&x+y+2z=0\\&3x+2y=2\\&x=10.}$
	\end{enumerate}
	\loigiai{
		\begin{enumerate}[a)]
			\item Ta có $\heva{&x-2y+4z=4\\&3y-z=2\\&2z=-10} \Leftrightarrow \heva{&x=22\\&y=-1\\&z=-5.}$
			\item Ta có $\heva{&4x+3y-5z=-7\\&2y=4\\&y+z=3} \Leftrightarrow \heva{&x=-2\\&y=2\\&z=1.}$
			\item Ta có $\heva{&x+y+2z=0\\&3x+2y=2\\&x=10} \Leftrightarrow \heva{&z=2\\&y=-14\\&x=10.}$
		\end{enumerate}
	}
\end{bt}

\begin{bt}
	Giải hệ phương trình
	\begin{enumerate}[a)]
		\item $\heva{&3x-y-2z=5\\&2x+y+3z=6\\&6 x-y-4z=9.}$
		\item $\heva{&x+2y+6z=5\\&-x+y-2z=3\\&x-4y-2z=1.}$
		\item $\heva{&x+4y-2z=2\\&-3x+y+z=-2\\&5x+7y-5z=6.}$
	\end{enumerate}
	\loigiai{
		\begin{enumerate}[a)]
			\item Ta có $\heva{&3x-y-2z=5\\&2x+y+3z=6\\&6 x-y-4z=9} \Leftrightarrow \heva{&3x-y-2z=5\\&5y+13z=8\\&y=-1} \Leftrightarrow \heva{&x=2\\&y=-1\\&z=1.}$
			\item Ta có $\heva{&x+2y+6z=5\\&-x+y-2z=3\\&x-4y-2z=1} \Leftrightarrow \heva{&x+2y+6z=5\\&3y+4z=8\\&3y+4z=3} \Leftrightarrow \heva{&x+2y+6z=5\\&3y+4z=8\\&0=5}$ suy ra hệ phương trình vô nghiệm.
			\item Ta có $\heva{&x+4y-2z=2\\&-3x+y+z=-2\\&5x+7y-5z=6} \Leftrightarrow \heva{&x+4y-2z=2\\&13y-5z=4\\&13y-5z=4}$ suy ra hệ phương trình vô số nghiệm $(x;\,y;\,z)$ thỏa mãn $\heva{&x+4y-2z=2\\&13y-5z=4.}$
		\end{enumerate}
	}
\end{bt}

\begin{bt}
	Tìm số đo ba góc của một tam giác, biết tổng số đo của góc thứ nhất và góc thứ hai bằng hai lần số đo của góc thứ ba, số đo của góc thứ nhất lớn hơn số đo của góc thứ ba là $20^{\circ}$.
	\loigiai{
		Gọi $3$ góc của tam giác lần lượt là $x$, $y$, $z$.\\
		Ta có $\heva{&x+y+z=180^\circ\\&x+y-2z=0^\circ\\&x-z=20^\circ} \Leftrightarrow \heva{&x+y+z=180^\circ\\&z=60^\circ\\&x-z=20^\circ} \Leftrightarrow \heva{&x=80^\circ\\&y=40^\circ\\&z=60^\circ.}$
	}
\end{bt}

\begin{bt}
	Bác Thanh chia số tiền $1$ tỉ đồng của mình cho ba khoản đầu tư. Sau một năm, tổng số tiền lãi thu được là $84$ triệu đồng. Lãi suất cho ba khoản đầu tư lần lượt là $6\%$, $8\%$, $15\%$ và số tiền đầu tư cho khoản thứ nhất bằng tổng số tiền đầu tư cho khoản thứ hai và thứ ba. Tính số tiền bác Thanh đầu tư cho mỗi khoản.
	\loigiai{
		Gọi $3$ khoản đầu tư lần lượt là $x$, $y$, $z$ triệu đồng.\\
		Ta có $\heva{&x+y+z=1000\\&6x+8y+15z=8400\\&x-y-z=0} \Leftrightarrow \heva{&x+y+z=1000\\&2y+9z=2400\\&y+z=500} \Leftrightarrow \heva{&x+y+z=1000\\&2y+9z=2400\\&z=200} \Leftrightarrow \heva{&x=500\\&y=300\\&z=200.}$
	}
\end{bt}

\begin{bt}
	Khi một quả bóng được đá lên, nó sẽ đạt độ cao nào đó rồi rơi xuống. Biết quỹ đạo chuyển động của quả bóng là một parabol và độ cao $h$ của quả bóng được tính bởi công thức $h=\dfrac{1}{2}at^{2}+v_{0}t+h_{0}$, trong đó độ cao $h$ và độ cao ban đầu $h_{0}$ được tính bằng mét, $t$ là thời gian của chuyển động tính bằng giây, $a$ là gia tốc của chuyển động tính bằng $\mathrm{m}/\mathrm{s}^{2}$, $v_{0}$ là vận tốc ban đầu được tính bằng $\mathrm{m}/\mathrm{s}$. Tìm $a$, $v_{0}$, $h_{0}$ biết sau $0,5$ giây quả bóng đạt được độ cao $6,075 \mathrm{~m}$; sau $1$ giây quả bóng đạt độ cao $8,5 \mathrm{~m}$; sau $2$ giây quả bóng đạt độ cao $6\mathrm{~m}$.
	\loigiai{
		Ta có $\heva{&h(0,5)=6,075\\&h(1)=8,5\\&h(2)=6} \Leftrightarrow \heva{&0,125a+0,5v_{0}+h_{0}=6,075\\&0,5a+v_{0}+h_{0}=8,5\\&2a+2v_{0}+h_{0}=6} \Leftrightarrow \heva{&a=-9,8\\&v_{0}=12,2\\&h_{0}=1,2.}$
	}
\end{bt}

\begin{bt}
	Một cửa hàng bán đồ nam gồm áo sơ mi, quần âu và áo phông. Ngày thứ nhất bán được $22$ áo sơ mi, $12$ quần âu và $18$ áo phông, doanh thu là $12580000$ đồng. Ngày thứ hai bán được $16$ áo sơ mi, $10$ quần âu và $20$ áo phông, doanh thu là $10800000$ đồng. Ngày thứ ba bán được $24$ áo sơ mi, $15$ quần âu và $12$ áo phông, doanh thu là $12960000$ đồng. Hỏi giá bán mỗi áo sơ mi, mỗi quần âu và mỗi áo phông là bao nhiêu? Biết giá từng loại trong ba ngày không thay đổi.
	\loigiai{
		Gọi giá bán mỗi áo sơ mi, mỗi quần âu và mỗi áo phông lần lượt là $x$, $y$, $z$ triệu đồng.\\
		Ta có $\heva{&22x+12y+18z=12580000\\&16x+10y+20z=10800000\\&24x+15y+12z=12960000} \Leftrightarrow \heva{&x=250000\\&y=320000\\&z=180000.}$
	}
\end{bt}

\begin{bt}
	Ba nhãn hiệu bánh quy là $A$, $B$, $C$ được cung cấp bởi một nhà phân phối. Với tỉ lệ thành phần dinh dưỡng theo khối lượng, bánh quy nhãn hiệu $A$ chứa $20\%$ protein, bánh quy nhãn hiệu $B$ chứa $28\%$ protein và bánh quy nhãn hiệu $C$ chứa $30\%$ protein. Một khách hàng muốn mua một đơn hàng như sau
	\begin{itemize}
		\item Mua tổng cộng $224$ cái bánh quy bao gồm cả ba nhãn hiệu $A$, $B$, $C$.
		\item Lượng protein trung bình của đơn hàng này (gồm cả ba nhãn hiệu $A$, $B$, $C$) là $25\%$.
		\item Lượng bánh nhãn hiệu $A$ gấp đôi lượng bánh nhãn hiệu $C$.
	\end{itemize}
	Tính lượng bánh quy mỗi loại mà khách hàng đó đặt mua.
	\loigiai{
		Gọi lượng bánh quy mỗi loại mà khách hàng đó đặt mua lần lượt là $x$, $y$, $z$.\\
		Ta có $\heva{&x+y+z=224\\&20x+28y+30z=25\cdot224\\&x-2z=0} \Leftrightarrow \heva{&x=96\\&y=80\\&z=48.}$
	}
\end{bt}

\begin{bt}
	Sử dụng máy tính cầm tay để tìm nghiệm của các hệ phương trình sau
	\begin{enumerate}[a)]
		\item $\heva{&-x+2y-3z=2\\&2x+y+2z=-3\\&-2x-3y+z=5.}$
		\item $\heva{&x-3y+z=1\\&5y-4z=0\\&x+2y-3z=-1.}$
		\item $\heva{&x+y-3z=-1\\&3x-5y-z=-3\\&-x+4y-2z=1.}$
	\end{enumerate}
	\loigiai{
		\begin{enumerate}[a)]
			\item $\heva{&-x+2y-3z=2\\&2x+y+2z=-3\\&-2x-3y+z=5} \Leftrightarrow \heva{&x=-4\\&y=\dfrac{11}{7}\\&z=\dfrac{12}{7}.}$
			\item $\heva{&x-3y+z=1\\&5y-4z=0\\&x+2y-3z=-1}$ hệ phương trình vô nghiệm.
			\item $\heva{&x+y-3z=-1\\&3x-5y-z=-3\\&-x+4y-2z=1}$ hệ phương trình vô số nghiệm.
		\end{enumerate}	
	}
\end{bt}


%%%%%%%%%%%%%%%%%%%%%%%%%%%%%%%%%%%%%%%%%%

%\subsection{CHÂN TRỜI SÁNG TẠO}

%\subsubsection{BÀI TẬP}

\begin{bt}
	Trong các hệ phương trình sau, hệ nào là hệ phương trình bậc nhất ba ẩn? Mỗi bộ ba số $(-1;\,2;\,1)$, $(-1,5;\,0,25;\,-1,25)$ có là nghiệm của hệ phương trình bậc nhất ba ẩn đó không?
	\begin{enumerate}[a)]
		\item $\heva{&3x-2y+z=-6\\&-2x+y+3z=7\\&4x-y+7z=1.}$
		\item $\heva{&5x-2y+3z=4\\&3x+2y-z=2\\&x-3y+2z=-1.}$
		\item $\heva{&2x-4y-3z=\dfrac{-1}{4}\\&3x+8y-4z=\dfrac{5}{2}\\&2x+3y-2z=\dfrac{1}{4}.}$
	\end{enumerate}
	\loigiai{
		\begin{enumerate}[a)]
			\item Bộ $(-1;\,2;\,1)$ là nghiệm của hệ phương trình $\heva{&3x-2y+z=-6\\&-2x+y+3z=7\\&4x-y+7z=1.}$			
			\item Cả $2$ bộ $(-1;\,2;\,1)$, $(-1,5;\,0,25;\,-1,25)$ không là nghiệm của hệ phương trình $\heva{&5x-2y+3z=4\\&3x+2y-z=2\\&x-3y+2z=-1.}$
			\item Bộ $(-1,5;\,0,25;\,-1,25)$ là nghiệm của hệ phương trình $\heva{&2x-4y-3z=\dfrac{-1}{4}\\&3x+8y-4z=\dfrac{5}{2}\\&2x+3y-2z=\dfrac{1}{4}.}$
		\end{enumerate}
	}
\end{bt}

\begin{bt}
	Giải các hệ phương trình sau bằng phương pháp Gauss
	\begin{enumerate}[a)]
		\item $\heva{&2x+3y=4\\&x-3y=2\\&2x+y-z=3.}$
		\item $\heva{&x+y+z=2\\&x+3y+2z=8\\&3x-y+z=4.}$
		\item $\heva{&x-y+5z=-2\\&2x+y+4z=2\\&x+2y-z=4.}$
	\end{enumerate}
	\loigiai{
		\item Ta có $\heva{&2x+3y=4\\&x-3y=2\\&2x+y-z=3} \Leftrightarrow \heva{&2x+3y=4\\&y=0\\&2y+z=1} \Leftrightarrow \heva{&x=2\\&y=0\\&z=1.}$
		\item Ta có $\heva{&x+y+z=2\\&x+3y+2z=8\\&3x-y+z=4} \Leftrightarrow \heva{&x+y+z=2\\&2y+z=6\\&4y+2z=2}$ hệ phương trình vô nghiệm.
		\item Ta có $\heva{&x-y+5z=-2\\&2x+y+4z=2\\&x+2y-z=4} \Leftrightarrow \heva{&x-y+5z=-2\\&y-2z=2\\&y-2z=2}$ hệ phương trình vô số nghiệm. 
	}
\end{bt}

\begin{bt}
	Sử dụng máy tính cầm tay, tìm nghiệm của các hệ phương trình sau
	\begin{enumerate}[a)]
		\item $\heva{&x-5z=2\\&3x+y-4z=3\\&-x+2y+z=-1.}$
		\item $\heva{&2x-y+z=3\\&x+2y-z=1\\&3x+y-2z=2.}$
		\item $\heva{&x+2y-z=1\\&2x+y-2z=2\\&4x-7y-4z=4.}$
	\end{enumerate}
	\loigiai{
		\begin{enumerate}[a)]
			\item $\heva{&x-5z=2\\&3x+y-4z=3\\&-x+2y+z=-1} \Leftrightarrow \heva{&x=\dfrac{17}{26}\\&y=-\dfrac{1}{26}\\&z=-\dfrac{7}{26}.}$
			\item $\heva{&2x-y+z=3\\&x+2y-z=1\\&3x+y-2z=2} \Leftrightarrow \heva{&x=\dfrac{6}{5}\\&y=\dfrac{2}{5}\\&z=1.}$
			\item $\heva{&x+2y-z=1\\&2x+y-2z=2\\&4x-7y-4z=4}$ hệ vô số nghiệm.
		\end{enumerate}
	}
\end{bt}

\begin{bt}
	Tìm phương trình của parabol $(P)\colon y=ax^{2}+bx+c \quad (a\neq 0)$, biết
	\begin{enumerate}[a)]
		\item Parabol $(P)$ có trục đối xứng $x=1$ và đi qua hai điểm $A(1;\,-4)$, $B(2;\,-3)$;
		\item Parabol $(P)$ có đỉnh $I\left(\dfrac{1}{2};\,\dfrac{3}{4}\right)$ và đi qua điểm $M(-1;\,3)$.
	\end{enumerate}
	\loigiai{
		Ta có $\heva{&-\dfrac{a}{2b}=1\\&a\cdot 1^2+b\cdot 1+c=-4\\&a\cdot 2^2+b\cdot 2+c=-3} \Leftrightarrow \heva{&a+2b=0\\&a+b+c=-4\\&4a+2b+c=-3} \Leftrightarrow \heva{&a=\dfrac{2}{5}\\&b=-\dfrac{1}{5}\\&c=-\dfrac{21}{5}.}$
	}
\end{bt}

\begin{bt}
	Một đại lí bán ba loại gas $A$, $B$, $C$ với giá bán mỗi bình gas lần lượt là $520000$ đồng, $480000$ đồng, $420000$ đồng. Sau một tháng, đại lí đã bán được $1299$ bình gas các loại với tổng doanh thu đạt $633960000$ đồng. Biết rằng trong tháng đó, đại lí bán được số bình gas loại $B$ bằng một nửa tổng số bình gas loại $A$ và $C$. Tính số bình gas mỗi loại mà đại lí bán được trong tháng đó.
	\loigiai{
		Gọi $x$, $y$, $z$ lần lượt là số bình ga loại $A$, $B$, $C$.\\
		Ta có $\heva{&x+y+z=1299\\&52x+48y+42z=63396\\&x-2y+z=0} \Leftrightarrow \heva{&x=624\\&y=433\\&z=242.}$
	}
\end{bt}


%%%%%%%%%%%%%%%%%%%%%%%%%%%%%%%%%%%%%%%%%
%\subsection{KẾT NỐI TRI THỨC}

%\subsubsection{BÀI TẬP}

\begin{bt}
	Hệ nào dưới đây là hệ phương trình bậc nhất ba ẩn? Kiểm tra xem bộ ba số $(2;\,0;\,-1)$ có phải là nghiệm của hệ phương trình bậc nhất ba ẩn đó không.
	\begin{enumerate}[a)]
		\item $\heva{&x-2z=4\\&2x+y-z=5\\&-3x+2y=-6.}$
		\item $\heva{&x-2y+3z=7\\&2x-y^{2}+z=2\\&x+2y=-1.}$
	\end{enumerate}
	\loigiai{
		\begin{enumerate}[a)]
			\item Ta có $\heva{&x-2z=4\\&2x+y-z=5\\&-3x+2y=-6}$ là hệ phương trình bậc nhất ba ẩn và bộ $(2;\,0;\,-1)$ không phải là nghiệm.
			\item Ta có $\heva{&x-2y+3z=7\\&2x-y^{2}+z=2\\&x+2y=-1}$ không là hệ phương trình bậc nhất ba ẩn.
		\end{enumerate}
	}
\end{bt}

\begin{bt}
	Giải các hệ phương trình sau
	\begin{enumerate}[a)]
	\item $\heva{&x-y-3z=20\\&x-z=3\\&x+3z=-7.}$
	\item $\heva{&x-y-3z=20\\&x-z=3\\&x+3z=-7.}$
	\end{enumerate}
	\loigiai{
		\begin{enumerate}[a)]
			\item Ta có $\heva{&2x-y-z=20\\&x+y=-5\\&x=10} \Leftrightarrow \heva{&x=10\\&y=-15\\&z=15.}$
			\item $\heva{&x-y-3z=20\\&x-z=3\\&x+3z=-7} \Leftrightarrow \heva{&x=\dfrac{1}{2}\\&y=-12\\&z=-\dfrac{5}{2}.}$
		\end{enumerate}
	}
\end{bt}

\begin{bt}
	Giải các hệ phương trình sau bằng phương pháp Gauss
	\begin{enumerate}[a)]
		\item $\heva{&2x-y-z=2\\&x+y=3\\&x-y+z=2.}$
		\item $\heva{&3x-y-z=2\\&x+2y+z=5\\&-x+y=2.}$
		\item $\heva{&x-3y-z=-6\\&2x-y+2z=6\\&4x-7y=-6.}$
		\item $\heva{&x-3y-z=-6\\&2x-y+2z=6\\&4x-7y=3.}$
		\item $\heva{&3x-y-7z=2\\&4x-y+z=11\\&-5x-y-9z=-22.}$
		\item $\heva{&2x-3y-4z=-2\\&5x-y-2z=3\\&7x-4y-6z=1.}$
	\end{enumerate}
	Kiểm tra lại kết quả tìm được bằng cách sử dụng máy tính cầm tay.
	\loigiai{
		\begin{enumerate}[a)]
			\item Ta có $\heva{&2x-y-z=2\\&x+y=3\\&x-y+z=2} \Leftrightarrow \heva{&2x-y-z=2\\&3y+z=4\\&y-3z=-2} \Leftrightarrow \heva{&2x-y-z=2\\&3y+z=4\\&z=1} \Leftrightarrow \heva{&x=2\\&y=1\\&z=1.}$
			\item Ta có $\heva{&3x-y-z=2\\&x+2y+z=5\\&-x+y=2} \Leftrightarrow \heva{&3x-y-z=2\\&7y+4z=13\\&2y-z=8} \Leftrightarrow \heva{&3x-y-z=2\\&7y+4z=13\\&z=-2} \Leftrightarrow \heva{&x=-1\\&y=-3\\&z=-2.}$
			\item Ta có $\heva{&x-3y-z=-6\\&2x-y+2z=6\\&4x-7y=-6} \Leftrightarrow \heva{&x-3y-z=-6\\&5y+4z=18\\&5y+4z=18}$ hệ phương trình vô số nghiệm.
			\item Ta có $\heva{&x-3y-z=-6\\&2x-y+2z=6\\&4x-7y=3} \Leftrightarrow \heva{&x-3y-z=-6\\&5y+4z=18\\&5y+4z=27}$ hệ phương trình vô nghiệm.
			\item Ta có $\heva{&3x-y-7z=2\\&4x-y+z=11\\&-5x-y-9z=-22} \Leftrightarrow \heva{&3x-y-7z=2\\&y+31z=25\\&2y+18z=-76} \Leftrightarrow \heva{&3x-y-7z=2\\&y+31z=25\\&31z=24} \Leftrightarrow \heva{&x=\dfrac{87}{31}\\&y=1\\&z=\dfrac{24}{31}.}$
			\item Ta có $\heva{&2x-3y-4z=-2\\&5x-y-2z=3\\&7x-4y-6z=1} \Leftrightarrow \heva{&2x-3y-4z=-2\\&13y+16z=16\\&13y+16z=16}$ hệ vô số nghiệm.
		\end{enumerate}
	}
\end{bt}

\begin{bt}
	Ba người cùng làm việc cho một công ty với vị trí lần lượt là quản lí kho, quản lí văn phòng và tài xế xe tải. Tổng tiền lương hằng năm của người quản lí kho và người quản lí văn phòng là $164$ triệu đồng, còn của người quản lí kho và tài xế xe tải là $156$ triệu đồng. Mỗi năm, người quản lí kho lĩnh lương nhiều hơn tài xế xe tải $8$ triệu đồng. Hỏi lương hằng năm của mỗi người là bao nhiêu?
	\loigiai{
		Gọi $x$, $y$, $z$ lần lượt là lương hằng năm của người quản lí kho, quản lí văn phòng và tài xế xe tải.\\
		Ta có $\heva{&x+y=164\\&x+z=156\\&x-z=8} \Leftrightarrow \heva{&x=82\\&y=82\\&z=74.}$
	}
\end{bt}

\begin{bt}
	Năm ngoái, người ta có thể mua ba mẫu xe ô tô của ba hãng $X$, $Y$, $Z$ với tổng số tiền là $2,8$ tỉ đồng. Năm nay, do lạm phát, để mua ba chiếc xe đó cần $3,018$ tỉ đồng. Giá xe ô tô của hãng $X$ tăng $8\%$, của hãng $Y$ tăng $5\%$ và của hãng $Z$ tăng $12\%$. Nếu trong năm ngoái giá chiếc xe của hãng $Y$ thấp hơn $200$ triệu đồng so với giá chiếc xe của hãng $X$ thì giá của mỗi chiếc xe trong năm ngoái là bao nhiêu?
	\loigiai{
		Gọi $x$, $y$, $z$ lần lượt là giá của ba mẫu xe ô tô của ba hãng $X$, $Y$, $Z$.\\
		Ta có $\heva{&x+y+z=2,8\\&1,08x+1,05y+1,12z=3,018\\&x-y=0,2} \Leftrightarrow \heva{&x=1,2\\&y=1\\&z=0,6.}$
	}
\end{bt}

\begin{bt}
	Cho hệ ba phương trình bậc nhất ba ẩn $\heva{&a_{1}x+b_{1}y+c_{1}z=d_{1}\\&a_{2}x+b_{2}y+c_{2}z=d_{2}\\&a_{3}x+b_{3}y+c_{3}z=d_{3}.}$
	\begin{enumerate}[a)]
		\item Giả sử $\left(x_{0};\,y_{0};\,z_{0}\right)$ và $\left(x_{1};\,y_{1};\,z_{1}\right)$ là hai nghiệm phân biệt của hệ phương trình trên. Chứng minh rằng \linebreak $\left(\dfrac{x_{0}+x_{1}}{2};\,\dfrac{y_{0}+y_{1}}{2};\,\dfrac{z_{0}+z_{1}}{2}\right)$ cũng là một nghiệm của hệ.
		\item Sử dụng kết quả của câu a) chứng minh rằng, nếu hệ phương trình bậc nhất ba ẩn có hai nghiệm phân biệt thì nó sẽ có vô số nghiệm.
	\end{enumerate}
	\loigiai{
		\begin{enumerate}[a)]
			\item Ta có $\left(x_{0};\,y_{0};\,z_{0}\right)$ và $\left(x_{1};\,y_{1};\,z_{1}\right)$ là hai nghiệm phân biệt của hệ phương trình $\heva{&a_{1}x+b_{1}y+c_{1}z=d_{1}\\&a_{2}x+b_{2}y+c_{2}z=d_{2}\\&a_{3}x+b_{3}y+c_{3}z=d_{3}}$ suy ra
				\[\heva{&a_{1}x_{0}+b_{1}y_{0}+c_{1}z_{0}=d_{1}\\&a_{2}x_{0}+b_{2}y_{0}+c_{2}z_{0}=d_{2}\\&a_{3}x_{0}+b_{3}y_{0}+c_{3}z_{0}=d_{3}} \quad \text{và} \quad \heva{&a_{1}x_{1}+b_{1}y_{1}+c_{1}z_{1}=d_{1}\\&a_{2}x_{1}+b_{2}y_{1}+c_{2}z_{1}=d_{2}\\&a_{3}x_{1}+b_{3}y_{1}+c_{3}z_{1}=d_{3}.}\]
				Cộng vế với vế các phương trình tương ứng trong hai hệ và chia hai vế cho $2$ ta được
				\[\heva{&a_{1}\left(\dfrac{x_{0}+x_{1}}{2}\right)+b_{1}\left(\dfrac{y_{0}+y_{1}}{2}\right)+c_{1}\left(\dfrac{z_{0}+z_{1}}{2}\right)=d_{1}\\&a_{2}\left(\dfrac{x_{0}+x_{1}}{2}\right)+b_{2}\left(\dfrac{y_{0}+y_{1}}{2}\right)+c_{2}\left(\dfrac{z_{0}+z_{1}}{2}\right)=d_{2}\\&a_{3}\left(\dfrac{x_{0}+x_{1}}{2}\right)+b_{3}\left(\dfrac{y_{0}+y_{1}}{2}\right)+c_{3}\left(\dfrac{z_{0}+z_{1}}{2}\right)=d_{3}.}\]
				Vậy $\left(\dfrac{x_{0}+x_{1}}{2};\,\dfrac{y_{0}+y_{1}}{2};\,\dfrac{z_{0}+z_{1}}{2}\right)$ cũng là một nghiệm của hệ.
			\item Nếu hệ phương trình bậc nhất ba ẩn có hai nghiệm phân biệt thì ta sử dụng kết quả của câu a) suy ra hệ sẽ có thêm nghiệm thứ ba, thứ tư, $\ldots$ Do đó hệ sẽ có vô số nghiệm. 
		\end{enumerate}
	}
\end{bt}