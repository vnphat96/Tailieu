\subsection{Bài tập tự luận}
%CÁNH DIỀU
\begin{bt}
	\immini{Cho mạch điện như \emph{Hình 3}. Biết $U=20$ V, $r_1=1\ \Omega$, $r_2=0,5\ \Omega$, $R=2\ \Omega$. Tìm cường độ dòng điện $I_1,I_2,I$ trong mỗi nhánh.}
	{
\begin{circuitikz}[american voltages,c/.style={circle,fill,inner sep=1pt}]
	\draw (-5,0) to ++(2,0) [generic, l=$r_2$]to ++(1,0) [battery1, l=U]   to ++(2,0) --(0,0);
	\draw (-5,0)--(-5,-2) (0,0)--(0,-2);
	\draw (0,-1) -- (-2,-1) [generic] to ++(-1,0) -- (-5,-1);
	\draw (0,-2) -- (-2,-2) [generic, l=$R$] to ++(-1,0) -- (-5,-2);
	\draw[->] (-3,0)--(-4,0) node[above] {$I_2$} ;
	\draw[->] (-5,-1)--(-4,-1) node[above] {$I_1$};
	\draw[->]   (-5,-2)--(-4,-2)  node[above] {$I$};
	\draw (-2.5,-0.6) node {$r_1$};
	\draw (-0.7,0) node[above] {\scriptsize $-$} (-1.3,0) node[above] {\scriptsize $+$}; 
	\draw (-2.5, -3) node{\textit{Hình 3}.};
\end{circuitikz}	
}
	\loigiai{
		Cường độ dòng điện của đoạn mạch mắc song song là $I_1+I$. Ta có $I_2=I_1+I$ hay $I+I_1-I_2=0$.\\
		Hiệu điệu thế của đoạn mạch mắc song song là $U_1=I_1r_1=IR$ hay $I_1=2I$, do đó ta có $2I-I_1=0$.\\
		Hiệu điện thế toàn mạch là $U=U_2+U_1=I_2r_2+I_1r_1$ hay $20=0,5I_2+I_1$. Ta có hệ
		$$\heva{&I+I_1-I_2=0\\&2I-I_1=0\\&I_1+0,5I_2=20.}$$
		Giải hệ phương trình trên ta được $I=\dfrac{40}{7}$ (A), $I_1=\dfrac{80}{7}$ (A) và $I_2=\dfrac{120}{7}$ (A).
	}
\end{bt}
\begin{bt}
\immini{Cho mạch điện như \emph{Hình 4}. Biết $U=24$ V, $\text{Đ}_1:12\ \mathrm{V}-6\ \mathrm{W}$, $\text{Đ}_2: 12\ \mathrm{V}-12\ \mathrm{W}$, $R=3\ \Omega$.
	\begin{enumerate}[a)]
		\item Tính điện trở của mỗi bóng đèn.
		\item Tính cường độ dòng điện qua các bóng đèn và qua điện trở $R$.
	\end{enumerate}}
{
	\begin{circuitikz}[european,c/.style={circle,fill,inner sep=1pt}]
	\draw (-5,0)--(-2,0) [battery1, l=U]   to ++(1,0) --(0,0);
	\draw (-5,0)-- (-5,-1) [R=$R$] to ++(0,-1) -- (-5,-3);
	\draw (-5,-3)--(-4,-3) -- (-4,-2) [lamp, l=$\text{Đ}_1$] to ++(2,0) -- (-1,-2);
	\draw (-4,-3)--(-4,-4) [lamp,l=$\text{Đ}_2$] to++(3,0) -- (-1,-4)--(-1,-3);
	\draw (-1,-2) -- (-1,-3) -- (0,-3)--(0,0);
	\draw (-1.3,0) node[above] {\scriptsize $-$} (-1.7,0) node[above] {\scriptsize $+$}; 
	\draw (-2.5, -5) node{\textit{Hình 4}.};
	\draw[->] (-5,-3)--(-4.5,-3);
	\draw[->]  (-2.5,-2)--(-2,-2);
	\draw[->]  (-4,-4)--(-3.5,-4);
\end{circuitikz}
}
	\loigiai{
		\begin{enumerate}[a)]
			\item Điện trở của $\text{Đ}_1$ là $R_1=\dfrac{U_{1}^2}{P_{1}}=\dfrac{12^2}{6}=24$ (V).\\
			Điện trở của $\text{Đ}_2$ là $R_2=\dfrac{U_{2}^2}{P_{2}}=\dfrac{12^2}{12}=12$ (V).
			\item Cường độ dòng diện của đoạn mạch song song $I_1+I_2$. Ta có phương trình $I=I_1+I_2$ hay $I-I_1-I_2=0$.\\
			Hiệu điện thế đoạn mạch song song là $U_1=I_1R_1=I_2R_2$ hay $24I_1=12I_2\Leftrightarrow 2I_1-I_2=0$.\\
			Hiệu điện thế của đoạn mạch là $U=U_1+U_2=I_1R_1+I_2R_2$ hay $24=24I_1+12I_2\Leftrightarrow 2I_1+I_2=2$. Ta có hệ phương trình
			$$\heva{&I-I_1-I_2=0\\&2I_1-I_2=0\\&2I_1+I_2=2.}$$
			Giải hệ phương trình trên ta được $I=\dfrac{3}{2}$ (A), $I_1=\dfrac{1}{2}$ (A) và $I=1$ (A).
		\end{enumerate}
	}
\end{bt}
\begin{bt}
	Tìm các hệ số $x,y,z$ để cân bằng mỗi phương trình sau:
	\begin{enumerate}[a)]
		\item $x\mathrm{KClO}_3\xrightarrow{t^0} y\mathrm{KCl}+z\mathrm{O}_2$;
		\item $x\mathrm{FeCl}_2+y\mathrm{Cl}_2\xrightarrow{t^0} z\mathrm{FeCl}_3$;
		\item $x\mathrm{Fe}+y\mathrm{O}_2\xrightarrow{t^0} z\mathrm{Fe}_2\mathrm{O}_3$;
		\item $x\mathrm{Na}_2\mathrm{SO}_3+2\mathrm{KMnO}_4+y\mathrm{NaHSO}_4\xrightarrow{t^0} z\mathrm{Na}_2\mathrm{SO}_4+2\mathrm{MnSO}_4+\mathrm{K}_2\mathrm{SO}_4+3\mathrm{H}_2\mathrm{O}$.
	\end{enumerate}
	\loigiai{
		\begin{enumerate}[a)]
			\item Theo định luật bảo toàn nguyên tố đối với $\mathrm{K}$ và $O$ ta có $x=y$ hay $x-y=0$ và $3x=2z$ hay $3x-2z=0$. Ta có hệ phương trình sau
			$$\heva{&x-y=0\\&3x-2z=0.}$$
			Chọn $x=2$, từ hệ trên ta được $y=2$ và $z=3$. Vậy ta có phương trình sau cân bằng là $$2\mathrm{KClO}_3\xrightarrow{t^0} 2\mathrm{KCl}+3\mathrm{O}_2.$$
			\item Theo định luật bảo toaàn nguyên tố đối với $\mathrm{Fe}$ và $\mathrm{Cl}$ ta có $x=z$ hay $x-z=0$ và $2x+2y=3z$ hay $2x+2y-3z=0$. Ta có hệ phương trình sau
			$$\heva{&x-z=0\\&2x+2y-3z=0.}$$
			Chọn $z=2$, từ hệ trên ta có $x=2$ và $y=1$. Vậy ta có phương trình sau cân bằng là $$2\mathrm{FeCl}_2+\mathrm{Cl}_2\xrightarrow{t^0} 2\mathrm{FeCl}_3.$$
			\item Theo định luật bảo toàn nguyên tố đối với $\mathrm{Fe}$ và $\mathrm{O}$ ta có $x=2z$ hay $x-2z=0$ và $2y=3z$ hay $2y-3z=0$. Ta có hệ phương trình
			$$\heva{&x-2z=0\\&2y-3z=0.}$$
			Chọn $z=2$, từ hệ trên ta có $x=4$ và $y=3$. Vậy ta có phương trình sau cân bằng là $$4\mathrm{Fe}+3\mathrm{O}_2\xrightarrow{t^0} 2\mathrm{Fe}_2\mathrm{O}_3.$$
			\item Theo định luật bảo toàn nguyên tố đối với $\mathrm{Na}$, $\mathrm{H}$ và $\mathrm{O}$ ta có $2x+y=2z$ hay $2x+y-2z=0$ và $y=6$ và $3x+8+4y=4z+8+4+3$ hay $3x+4y-4z=7$. Ta có hệ phương trình sau
			$$\heva{&2x+y-2z=0\\&y=6\\&x+y-z=3}\Leftrightarrow \heva{&x-z=-3\\&3x-4z=-17\\&y=6}\Leftrightarrow \heva{&x=5\\&y=6\\&z=8.}$$
			Vậy ta có phương trình sau cân bằng là $$5\mathrm{Na}_2\mathrm{SO}_3+2\mathrm{KMnO}_4+6\mathrm{NaHSO}_4\xrightarrow{t^0} 8\mathrm{Na}_2\mathrm{SO}_4+2\mathrm{MnSO}_4+\mathrm{K}_2\mathrm{SO}_4+3\mathrm{H}_2\mathrm{O}.$$
		\end{enumerate}
	}
\end{bt}
\begin{bt}
	Một giáo viên dạy Hóa tạo $1000$ g dung dịch $\mathrm{HCl}$ $25\%$ từ ba loại dung dịch $\mathrm{HCl}$ có nồng độ lần lượt là $10\%$, $20\%$ và $30\%$. Tính khối lượng dung dịch mỗi loại. Biết rằng lượng $\mathrm{HCl}$ có trong dung dịch $10\%$ bằng $\dfrac{1}{4}$ lượng $\mathrm{HCl}$ có trong dung dịch $20\%$.
	\loigiai{
		Gọi khối lượng dung dịch $\mathrm{HCl}$ có nồng độ $10 \%, 20 \%$ và $30 \%$ lần lượt là $x$, $y$ và $z$. Theo đề bài ta có
		\begin{align}\label{41}
			x+y+z=1000.
		\end{align}
		Do dung dịch mới có nồng độ $25 \%$ nên ta có
		\begin{align}\label{42}
			\dfrac{10 \% x+20 \% y+30 \% z}{1000}=25 \% \Leftrightarrow 10 x+20 y+30 z=25000 \Leftrightarrow x+2 y+3 z=2500.
		\end{align}
		Lượng $\mathrm{HCl}$ có trong dung dịch $10 \%$ bằng $\dfrac{1}{4}$ lượng $\mathrm{HCl}$ có trong dung dịch $20 \%$ nên
		\begin{align}\label{43}
			10 \% x=\frac{1}{4} 20 \% y \Leftrightarrow 2 x-y=0.
		\end{align}
		Từ \eqref{41}, \eqref{42} và \eqref{43} ta có hệ phương trình
		$$\heva{&x+y+z=1000 \\
			&x+2 y+3 z=2500 \\
			&2 x-y=0.}
		$$
		Giải hệ này ta được $x=125, y=250, z=625$.
		Vậy khối lượng dung dịch $\mathrm{HCl}$ có nồng độ $10 \%, 20 \%$ và $30 \%$ lần lượt là $125 \mathrm{~g}, 250 \mathrm{~g}, 625 \mathrm{~g}$.
	}
\end{bt}
\begin{bt}
	Tổng số hạt $p,n,e$ trong hai nguyên tử kim loại $A$ và $B$ là $177$. Trong đó số hạt mang điện nhiều hơn số hạt không mang điện là $47$. Số hạt mang điện của nguyên tử $B$ nhiều hơn của nguyên tử $A$ là $8$. Xác định số hạt proton trong một nguyên tử $A$.
	\loigiai{
		Tổng số hạt $p,n,e$ trong hai nguyên tử $A$ và $B$ là $177$ nên ta có phương trình 
		\begin{align}\label{pt1}
			2Z_A+N_A+2Z_B+N_B=177.
		\end{align}
		Do số hạt mang điện nhiều hơn số hạt không mang điện là $47$ nên ta có phương trình
		\begin{align}\label{pt2}
			2Z_A+2Z_B-N_A-N_B=47.
		\end{align}
		Do số hạt mang điện của nguyên tử $B$ nhiều hơn của nguyên tử $A$ là $8$ nên ta có phương trình
		\begin{align}\label{pt3}
			2Z_B-2Z_A=8\Leftrightarrow Z_A-Z_B=-4.
		\end{align}
		Lấy phương trình \eqref{pt1} cộng với phương trình \eqref{pt2} vế theo vế ta được
		\begin{align}\label{pt4}
			4Z_A+4Z_B=224\Leftrightarrow Z_A+Z_B=56.
		\end{align}
		Từ \eqref{pt3} và \eqref{pt4} ta có hệ $$\heva{&Z_A-Z_B=4\\&Z_A+Z_B=56}\Leftrightarrow \heva{&Z_A=26\\&Z_B=30.}$$
		Vậy số hạt proton trong nguyên tử $A$ là $26$.
	}
\end{bt}

\begin{bt}
	Một phân tử DNA có khối lượng là $72\cdot 10^4$ đvC và có $2826$ liên kết hyđro. Mạch $2$ có số nu loại $A$ bằng $2$ lần số nu loại $T$ và bằng $3$ lần số nu loại $X$. Xác định số nucleotit mỗi loại trên từng mạch của phân tử DNA đó. Biết rằng một nu có khối lượng trung bình là $300$ đvC.
	\loigiai{
		Kí hiệu $A$, $G$, $T$, $X$ lần lượt là tổng số nu loại $A$, $G$, $T$, $X$ của phân tử DNA.\\
		$N$ là tổng số nu của phân tử DNA.\\
		$A_1$, $G_1$, $T_1$, $X_1$ lần lượt là tổng số nu loại $A$, $G$, $T$, $X$ của mạch 1.\\
		$A_2$, $G_2$, $T_2$, $X_2$ lần lượt là tổng số nu loại $A$, $G$, $T$, $X$ của mạch 2.\\
		Vì phân tử DNA có khối lượng là $72 \cdot 10^4$ đvC, mà một nu có khối lượng trung bình là $300$ đvC nên tổng số nu của phân tử DNA là $N=\dfrac{72 \cdot 10^4}{300} = 2400$.\\
		$\Rightarrow G+A = \dfrac{N}{2}=1200$.\\
		Phân tử DNA có $2826$ liên hết hyđro nên $2A + 3G = 2826$.\\
		Khi đó, ta có hệ phương trình 
		\begin{eqnarray*}
			\heva{&G+A=1200\\&2A + 3G = 2826} \Leftrightarrow \heva{&A=774\\&G=426} \Rightarrow \heva{&A=T=774\\&G=X=426.}
		\end{eqnarray*}
		Mạch 2 có số nu loại $A$ bằng $2$ lần số nu loại $T$ và bằng $3$ lần số nu loại $X$ nên\\ ta có $A_2 = 2T_2$, $A_2 = 3X_2$\\
		hay $A_2 - 2T_2 = 0$, $A_2 - 3X_2 = 0$.\\
		Mặt khác, vì $A_1 = T_2$ nên $A_2 + T_2 = A_2 + A_1 = A = 774$.\\
		Ta có hệ phương trình $\heva{&A_2 - 2T_2 = 0\\&A_2 - 3X_2 = 0\\&A_2 + T_2 =774} \Leftrightarrow \heva{&A_2=516\\&T_2=258\\&X_2=172.}$\\
		Suy ra số nu loại $G$ của mạch 2 là $G_2 = 1200 - (516 + 258 + 172) = 254$.\\
		Ở mạch 1, ta có $A_1 = T_2 = 258$, $T_1 = A_2 = 516$, $G_1 = X_2 = 172$, $X_1 = G_2 = 254$.
	}
\end{bt}

\begin{bt}
	Tìm đa thức bậc ba $f(x) = ax^3 + bx^2 + cx + 1$ (với $a \neq  0$) biết $f(-1) =  -2$, $f(1) = 2$, $f(2) = 7$.
	\loigiai{
		Ta có \\
		$f(-1) =  -2 \Leftrightarrow a\cdot (-1)^3 + b\cdot(-1)^2 + c\cdot(-1) + 1 = -2 \Leftrightarrow -a+b-c=-3$.\\
		$f(1) =  2 \Leftrightarrow a\cdot 1^3 + b\cdot 1^2 + c\cdot 1 + 1 = 2 \Leftrightarrow a+b+c=1$.\\
		$f(2) =  7 \Leftrightarrow a\cdot 2^3 + b\cdot 2^2 + c\cdot 2 + 1 = 7 \Leftrightarrow 8a+4b+2c=6$.\\
		Ta có hệ phương trình
		\begin{eqnarray*}
			\heva{&-a+b-c=-3\\&a+b+c=1\\&8a+4b+2c=6} \Leftrightarrow \heva{&a=1\\&b=-1\\&c=1.}
		\end{eqnarray*}
		Vậy $f(x)=x^3-x^2+x+1$.
	}
\end{bt}

\begin{bt}
	Ba lớp $10A$, $10B$, $10C$ trồng được $164$ cây bạch đàn và $316$ cây thông. Mỗi học sinh lớp $10A$ trồng được $3$ cây bạch đàn và $2$ cây thông; mỗi học sinh lớp $10B$ trồng được $2$ cây bạch đàn và $3$ cây thông; mỗi học sinh lớp $10C$ trồng được $5$ cây thông. Hỏi mỗi lớp có bao nhiêu học sinh? Biết số học sinh lớp $10A$ bằng trung bình cộng số học sinh lớp $10B$ và $10C$.
	\loigiai{
		Gọi số học sinh của ba lớp $10A$, $10B$, $10C$ lần lượt là $x$, $y$, $z$ (học sinh) với ($x, y, z \in \mathbb{N^*}$).\\
		Theo đề bài ta có hệ phương trình
		\begin{eqnarray*}
			\heva{&3x+2y+0z=164\\&2x+3y+5z=316\\&x=\dfrac{y+z}{2}} \Leftrightarrow \heva{&3x+2y=164\\&2x+3y+5z=316\\&2x-y-z=0} \Leftrightarrow \heva{&x=32\\&y=34\\&z=30.}
		\end{eqnarray*}	
		Vậy số học sinh của ba lớp $10A$, $10B$, $10C$ lần lượt là $32$, $34$, $30$ học sinh.
	}
\end{bt}

\begin{bt}
	Độ cao $h$ trong chuyển động của một vật được tính bởi công thức $h = \dfrac{1}{2}at^2 + v_0t + h_0$, với độ cao $h$ và độ cao ban đầu $h_0$ được tính bằng mét, $t$ là thời gian của chuyển động tính bằng giây, $a$ là gia tốc của chuyển động tính bằng m/s$^2$, $v_0$ là vận tốc ban đầu tính bằng m/s. Tìm $a$, $v_0$, $h_0$. Biết rằng sau $1$s và $3$s vật cùng đạt được độ cao $50{,}225$m; sau $2$s vật đạt độ cao $55{,}125$m.
	\loigiai{
		Theo đề bài ta có hệ phương trình
		\begin{eqnarray*}
			\heva{&\dfrac{1}{2}a\cdot 1^2 + v_0\cdot 1 + h_0=50{,}225\\&\dfrac{1}{2}a\cdot 3^2 + v_0\cdot 3 + h_0=50{,}225\\&\dfrac{1}{2}a\cdot 2^2 + v_0\cdot 2 + h_0=55{,}125} \Leftrightarrow \heva{&\dfrac{1}{2}a+v_0+h_0=50{,}225\\&\dfrac{9}{2}a+3v_0+h_0=50{,}225\\&2a+2v_0+h_0=55{,}125} \Leftrightarrow \heva{&a=-9{,}8\\&v_0=19{,}6\\&h_0=35{,}525.}
		\end{eqnarray*}		
		Vậy $a=-9{,}8$ m/s$^2$, $v_0=19{,}6$ m/s, $h_0=35{,}525$ m.
	}
\end{bt}

\begin{bt}
	Một ngân hàng muốn đầu tư số tiền tín dụng là $100$ tỉ đồng thu được vào ba nguồn: mua trái phiếu với mức sinh lời 8$\%$/năm, cho vay thu lãi suất 10$\%$/năm và đầu tư bất động sản với mức sinh lời 12$\%$/năm. Theo điều kiện của quỹ tín dụng đề ra là tổng số tiền đầu tư vào trái phiếu và cho vay phải gấp ba lần số tiền đầu tư vào bất động sản. Nếu ngân hàng muốn thu được mức thu nhập $9{,}6$ tỉ đồng hằng năm thì nên đầu tư như thế nào vào ba nguồn đó?
	\loigiai{
		Gọi số tiền đầu tư trái phiếu, cho vay, bất động sản lần lượt là $x$, $y$, $z$ (tỉ đồng).\\
		Theo đề bài ta có $x + y + z = 100$.\\
		Tổng số tiền đầu tư vào trái phiếu và cho vay gấp ba lần số tiền đầu tư vào bất động sản, do đó $x + y = 3z$ hay $x + y - 3z = 0$.\\
		Lãi suất cho ba khoản đầu tư lần lượt là 8$\%$, 10$\%$, 12$\%$ và tổng số tiền lãi thu được là $9{,}6$ tỉ đồng nên ta có $8\%x + 10\%y + 12\%z = 9{,}6$.\\
		Khi đó, ta có hệ phương trình 
		\begin{eqnarray*}
			\heva{&x + y + z = 100\\&x + y - 3z = 0\\&8\%x + 10\%y + 12\%z = 9{,}6} \Leftrightarrow \heva{&x=45\\&y=30\\&z=25.}
		\end{eqnarray*}
		Vậy số tiền đầu tư trái phiếu, cho vay, bất động sản lần lượt là $45$ tỉ đồng, $30$ tỉ đồng, $25$ tỉ đồng.
	}
\end{bt}


%KET NOI TRI THUC VA CUOC SONG

\begin{bt}%Bài 1.7%Nguyễn Nhật Lệ
	Cho hàm cung và hàm cầu của ba mặt hàng như sau
	\begin{center}
		$Q_{S_1}=-4+x;\ Q_{D_1}=70-x-2y-6z;$\\
		$Q_{S_2}=-3+y;\ Q_{D_2}=76-3x-y-4z;$\\
		$Q_{S_3}=-6+3z;\ Q_{D_3}=70-2x-3y-2z.$
	\end{center}
	Hãy xác định giá trị cân bằng cung - cầu của ba mặt hàng.
	\loigiai{
		Hệ phương trình cân bằng cung - cầu của ba mặt hàng là
		$$\heva{&Q_{S_1}=Q_{D_1}\\&Q_{S_2}=Q_{D_2}\\&Q_{S_3}=Q_{D_3}}\Leftrightarrow \heva{&-4+x=70-x-2y-6z\\&-3+y=76-3x-y-4z\\&-6+3z=70-2x-3y-2x}\Leftrightarrow \heva{&2x+2y+6z=74\\&3x+2y+4z=79\\&2x+3y+5z=76}\Leftrightarrow\heva{&x=15\\&y=7\\&z=5}.$$
		Vậy giá mặt hàng thứ nhất là $15$, mặt hàng thứ hai là $7$, mặt hàng thứ ba là $5$ hợp lí nhất.
	}
\end{bt}
\begin{bt}%Bài 1.8%Nguyễn Nhật Lệ
	Em Hà so sánh tuổi của mình với chị Mai và anh Nam. Tuổi của anh Nam gấp ba lần tuổi của em Hà. Cách đây bảy năm tuổi của chị Mai bằng nửa số tuổi của anh Nam. Ba năm nữa tuổi của anh Nam bằng tổng số tuổi của chị Mai và em Hà. Hỏi tuổi của mỗi người là bao nhiêu? 
	\loigiai{
		Gọi tuổi của anh Nam, chị Mai, em Hà lần lượt là $x,\ y\ z$ (tuổi) ($x,\ y,\ z>0$).\\
		Vì tuổi của anh Nam gấp ba lần tuổi của em Hà, ta có: $x=3z$.\\
		Cách đây $7$ năm, tuổi của chị Mai bằng nửa số tuổi anh Nam, ta có: $y-7=\dfrac{1}{2}(x-7)$.\\
		Ba năm nữa tuổi của anh Nam bằng tổng số tuổi của chị Mai và em Hà, nên ta có: $x+3=(y+3)+(z+3)$.\\
		Khi đó ta có hệ phương trình
		$$\heva{&x=3z\\&y-7=\dfrac{1}{2}(x-7)\\&x+3=(y+3)+(z+3)}\Leftrightarrow \heva{&x-3z=0\\&x-2y=-7\\&x-y-z=3}\Leftrightarrow\heva{&x=39\\&y=23\\&z=13}.$$
		Vậy anh Nam 39 tuổi, chị Mai 23 tuổi, em Hà 13 tuổi.
	}
\end{bt}
\begin{bt}%Bài 1.9%Nguyễn Nhật Lệ
	Bác Việt có 330 740 nghìn đồng, bác chia số tiền này thành ba phần và đem đầu tư vào ba hình thức: Phần thứ nhất bác đầu tư vào chứng khoán với lãi thu được $4\%$ một năm; phần thứ hai bác mua vàng thu lãi $5\%$ một năm và phần thứ ba bác gửi tiết kiện với lãi suất $6\%$ một năm. Sau một năm, kể cả gốc và lãi bác thu được ba món tiền bằng nhau? Hỏi tổng số tiền cả gốc và lãi bác thu được sau một năm là bao nhiêu?
	\loigiai{
		Gọi số tiền mà bác Việt đầu tư vào chứng khoán, vàng, gửi tiết kiệm lần lượt là $x,\ y,\ z$\ (nghìn đồng) ($x,\ y,\ z>0)$.\\
		Theo bài ra, tổng số tiền bác Việt có là $x+y+z=330740$.\\
		Sau một năm, cả gốc lẫn lãi thu được ba món tiền bằng nhau nên ta có
		$$x+4\%x=y+5\%y=z+6\%z\Leftrightarrow 1{,}04x=1
		{,}05y=1{,}06z$$
		Từ đó ta có hệ phương trình 
		$$\heva{&x+y+z=330740\\&1{,}04x-1{,}05y=0\\&1{,}05y-1{,}06z=0}\Leftrightarrow \heva{&x=111300\\&y=110240\\&z=109200}.$$
		Vậy bác Việt đầu tư 111 300 nghìn đồng vào chứng khoán, 110 240 nghìn đồng vào vàng và 109 200 nghìn đồng để gửi tiết kiệm.
	}
\end{bt}
\begin{bt}%Bài 1.10%Nguyễn Nhật Lệ
	Một tuyến cáp treo có ba loại vé sau đây: vé đi lên giá $250$ nghìn đồng; vé đi xuống giá $200$ nghìn đồng và vé hai chiều giá $400$ nghìn đồng. Một ngày nhà ga cáp treo thu được tổng số tiền là 251 triệu đồng. Tìm số vé bán ra mỗi loại, biết rằng nhân viên quản lí cáp treo đếm được 680 lượt người đi lên và 250 lượt người đi xuống.
	\loigiai{
		Gọi số vé đi lên, đi xuống, vé hai chiều bán ra lần lượt là $x,\ y,\ z\ (x,\ y,\ z>0)$.\\
		Theo bài ra, tổng số tiền là $251000000$ triệu đồng, khi đó ta có $250000x+200000y+400000z=251000000$ (1)\\
		Tổng số lượt người đi lên là $x+z=680$ (2)\\
		Tổng số lượt người đi xuống là $y+z=520$ (3)\\
		Từ (1),\ (2),\ (3) ta có hệ phương trình 
		$$\heva{&250000x+200000y+400000z=251000000\\&x+z=680\\&y+z=520}\Leftrightarrow\heva{&x=220\\&y=40\\&z=460}.$$
		Vậy số vé bán ra loại đi lên, đi xuống và hai chiều lần lượt là $220,\ 60,\ 460$.
	}
\end{bt}

\begin{bt}
	Ba lớp $10A$, $10B$, $10C$ của một trường trung học phổ thông gồm $128$ em cùng tham gia lao động trồng cây. Tính trung bình, mỗi em lớp $10A$ trồng được $3$ cây xoan và $4$ cây bạch đàn; mỗi em lớp $10B$ trồng được $2$ cây xoan và $5$ cây bạch đàn; mỗi em lớp $10C$ trồng được $6$ cây xoan. Cả ba lớp trồng được tổng cộng $476$ cây xoan và $375$ cây bạch đàn. Hỏi mỗi lớp có bao nhiêu em.
	\loigiai{ Gọi $x$, $y$, $z$ lần lượt là số học sinh lớp $10A$, $10B$, $10C$ ($x,\ y,\ z >0$).\\
		Vì tổng số học sinh ba lớp là $128$ nên $x+y+z=128$.\\
		Số cây xoan và bạch đàn lớp $10A$ trồng được lần lượt là $3x$, $4x$.\\
		Số cây xoan và bạch đàn lớp $10B$ trồng được lần lượt là $2y$, $5y$.\\
		Số cây xoan lớp $10C$ trồng được là $6z$.\\
		Theo đề ta có
		$$\heva{&x+y+z=128\\&3x+2y+6z=476\\&4x+5y=375} \Leftrightarrow \heva{&x=40\\&y=43\\&z=45.}$$
		Vậy số học sinh lớp $10A$, $10B$, $10C$ lần lượt là $40$, $43$, $45$ học sinh.
	}
\end{bt}
\begin{bt}
	Cân bằng phương trình phản ứng hóa học đốt cháy methane trong oxygen
	\begin{center}
		$\mathrm{CH}_4+\mathrm{O}_2 \rightarrow \mathrm{CO}_2+\mathrm{H}_2 \mathrm{O}$.
	\end{center}
	\loigiai{Gọi $x,\ y,\ z$ lần lượt là hệ số cân bằng của $CH_4$, $O_2$ và $H_2O$.\\
		Vì số nguyên tử $C$ ở 2 vế phương trình là như nhau nên ta có hệ số cân bằng của $CO_2$ bằng $x$.
		\begin{center}
			$x\mathrm{CH}_4+y\mathrm{O}_2 \rightarrow x\mathrm{CO}_2+z\mathrm{H}_2 \mathrm{O}$.
		\end{center}
		Số nguyên tử $\mathrm{O}$ ở hai vế bằng nhau nên $2y=2x+z$.\\
		Số nguyên tử $\mathrm{H}$ ở hai vế bằng nhau nên $4x=2z$.\\
		Ta có hệ phương trình
		$$\heva{&2y=2x+z &\quad (1)\\&4x=2z. & \quad (2)}$$
		Chọn $x=1$, từ (2) suy ra $z=2$, từ (1) suy ra $y=2$.\\
		Vậy phương trình cân bằng phản ứng hóa học là $$\mathrm{CH}_4+2\mathrm{O}_2 \rightarrow \mathrm{CO}_2+2\mathrm{H}_2 \mathrm{O}.$$
	}
\end{bt}
\begin{bt}
	Cho một đoạn mạch như \emph{Hình 1.2}. Gọi $I$ là cường độ dòng điện mạch chính, $I_1$, $I_2$, $I_3$ là cường độ dòng điện mạch rẽ. Cho biết $R_1 =6\ \Omega$, $R_2=8\ \Omega$, $I=3 \ A$ và $I_3=2 \ A$. Tính điện trở $R_3$ và hiệu điện thế $U$ giữa hai đầu đoạn mạch.
	\begin{center}
			\begin{circuitikz}[european,c/.style={circle,fill,inner sep=1pt}]
			\draw (0,0) coordinate (A) to(3,0) coordinate (B)  -- (3,1) to [R=$ R_1 $]  (5,1) to [R=$ R_2 $] (7,1)
			(B) -- (3,-1)to[R=$ R_3 $]  (7,-1)--(7,1)
			(A) -- (0,-2.5)  to[battery2,l_=U] (8,-2.5) ;
			\draw (7,0) coordinate (C) to (8,0)--(8,-2.5) ;
			\draw (1.5,0.2) node{$I$} (3.1,1.2) node {$I_1$} (5,1.2) node{$I_2$} (4,-0.7) node{$I_3$};
			\path foreach \p/\g in {}{(\p)node[c]{}+(\g:3.5mm) node{$\p$}};
			\draw (4,-4) node{\textit{Hình 1.2}};
		\end{circuitikz}
	\end{center}
	\loigiai{
		Vì $R_1$, $R_2$ mắc nối tiếp nên $I_1=I_2$.\\
		Từ sơ đồ mạch điện ta có hệ phương trình
		$$\heva{&I_1+I_3=I\\&R_1I_1+R_2I_2=U\\&R_3I_3=U} \Leftrightarrow \heva{&I_1+2=3\\&6I_1+8I_2=U\\&2R_3=U}\Leftrightarrow \heva{&I_1=1\\&U=14\\&R_3=7.}$$
		Vậy điện trở $R_3=7\ \Omega$, hiệu điện thế giữa hai đầu mạch $U=14 \ V$.
	}
\end{bt}
\begin{bt}
	Mỗi giai đoạn phát triển của thực vật cần phân bón với tỉ lệ $N$, $P$, $K$ nhất định. Bác An làm vườn muốn bón phân cho một cây cảnh với tỉ lệ $N:P:K$ cân bằng nhau. Bác An có ba bao phân bón:
	\begin{center}
		\begin{tabular}{ll}
			& Bao $1$ có tỉ lệ $N:P:K$ là $12:7:12$.\\
			& Bao $2$ có tỉ lệ $N:P:K$ là $6:30:25$.\\
			& Bao $3$ có tỉ lệ $N:P:K$ là $30:16:11$.
		\end{tabular}
	\end{center}
	Hỏi phải trộn ba loại phân bón trên với tỉ lệ bao nhiêu để có hỗn hợp phân bón tỉ lệ $N:P:K$ là $15:15:15$?
	Chú ý rằng trên mỗi bao phân người ta thường viết tỉ lệ $N:P:K$ nhất định. Chẳng hạn trên bao phân $1$ ghi tỉ lệ $N:P:K$ là $12:7:12$ nghĩa là hàm lượng đạm $N$ (nitơ) chiếm $12 \%$, lân $P$ (tức là $P_2O_5$) chiếm $7 \%$ và kali $K$ (tức là $K_2O$) chiếu $12\%$, còn các loại khác chiếm $100\%-(12\%+7\%+12\%)=69\%$.
	\loigiai{Giả sử bác An cần trộn $1$ kg phân bón với khối lượng ba loại phân bón này lần lượt là $x,\ y,\ z$ (kg).\\
		Khi đó, tổng khối lượng phân đạm $N$ trong $1$ kg này là $12\% x+6\%y+30\%z.$\\
		Tổng khối lượng phân lân $P$ trong $1$ kg này là $7\%x+30\%y+16\%z$.\\
		Tổng khối lượng phân kali $K$ trong $1$ kg này là $12\%x+25\%y+11\%z$.\\
		Vì hỗn hợp phân bón mới có tỉ lệ $N:P:K$ là $15:15:15$ nên ta có
		
		$$\heva{&12\%x+6\%y+30\%z=15\%\\&7\%x+30\%y+16\%z=15\%\\&12\%x+25\%y+11\%z=15\%} \Leftrightarrow \heva{&12x+6y+30z=15\\&7x+30y+16z=15\\&12x+25y+11z=15} \Leftrightarrow \heva{&x=0,5\\&y=0,25\\&z=0,25.}$$
		Vậy tỉ lệ bao 1: bao 2: bao 3 là $0,5:0,25:0,25$ hay $2:1:1$.
	}
\end{bt}

%CHÂN TRỜI SÁNG TẠO
\begin{bt}
	Một đại lí bán ba mẫu máy điều hòa $A$, $B$ và $C$, với giá bán mỗi chiếc theo từng mẫu lần lượt là $8$ triệu đồng, $10$ triệu đồng và $12$ triệu đồng. Tháng trước, đại lí bán được 100 chiếc gồm cả ba mẫu và thu được số tiền là $980$ triệu đồng. Tính số lượng máy điều hòa mỗi mẫu đại lí bán được trong tháng trước, biết rằng số tiền thu được từ bán máy điều hòa mẫu $A$ và mẫu $C$ là bằng nhau. 
	\loigiai{
		Gọi $x$, $y$, $z$ lần lượt là số máy điều hòa của các mẫu $A$, $B$, $C$ mà đại lí bán được trong tháng trước ($x,y,z\in \mathbb{N}$).\\
		Đại lí bán được 100 chiếc gồm cả ba mẫu, suy ra: $x+y+z=100$.\\
		Đại lí thu được số tiền là $980$ triệu đồng, suy ra: $8x+10y+12z=980$.\\
		Số tiền thu được từ bán máy điều hòa mẫu $A$ và mẫu $C$ là bằng nhau, suy ra: $8x=12z$.\\
		Như vậy, ta có hệ phương trình 
		$$\heva{&x+y+z=100 \\ &8x+10y+12z=980 \\ &8x-12z=0} \Leftrightarrow \heva{&x=30 \\ &y=50 \\ &z=20.}$$
		Vậy tháng trước đại lí bán được $30$ máy mẫu $A$, $50$ máy mẫu $B$ và $20$ máy mẫu $C$.
	}
\end{bt}


\begin{bt}
	Nhân dịp kỉ niệm ngày thành lập Đoàn Thanh niên Cộng sản Hồ Chí Minh, một trường Trung học phổ thông đã tổ chức cho học sinh tham gia các trò chơi. Ban tổ chức đã chọn $100$ bạn và chia thành ba nhóm $A$, $B$, $C$ để tham gia trò chơi thứ nhất. Sau khi trò chơi kết thúc, ban tổ chức chuyển $\dfrac{1}{3}$ số bạn ở nhóm $A$ sang nhóm $B$; $\dfrac{1}{2}$ số bạn ở nhóm $B$ sang nhóm $C$; số bạn chuyển từ nhóm $C$ sang nhóm $A$ và $B$ đều bằng $\dfrac{1}{3}$ số bạn ở nhóm $C$ ban đầu. Tuy nhiên, người ta nhận thấy số bạn ở mỗi nhóm là không đổi qua hai trò chơi. Ban tổ chức đã chia mỗi nhóm bao nhiêu bạn?
	\loigiai{
		Gọi $x$, $y$, $z$ lần lượt là số học sinh ở mỗi nhóm $A$, $B$, $C$ lúc ban đầu ($x,y,z\in\mathbb{N}$).\\
		Ban đầu có tổng cộng $100$ bạn, suy ra: $x+y+z=100$.\\
		Số bạn ở nhóm $A$ không đổi qua hai trò chơi, suy ra: $x-\dfrac{1}{3}x+\dfrac{1}{3}z=x$.\\
		Số bạn ở nhóm $B$ không đổi qua hai trò chơi, suy ra: $y+\dfrac{1}{3}x-\dfrac{1}{2}y+\dfrac{1}{3}z=y$.\\
		Số bạn ở nhóm $C$ không đổi qua hai trò chơi, suy ra: $z+\dfrac{1}{2}y-\dfrac{1}{3}z-\dfrac{1}{3}z=z$.\\
		Như vậy, ta có hệ phương trình
		$$ \heva{&x+y+z=10 \\ &x-\dfrac{1}{3}x+\dfrac{1}{3}z=x \\ &y+\dfrac{1}{3}x-\dfrac{1}{2}y+\dfrac{1}{3}z=y \\ &z+\dfrac{1}{2}y-\dfrac{1}{3}z-\dfrac{1}{3}z=z} 
		\Leftrightarrow \heva{&x+y+z=100 \\ &x-z=0 \\ &3y-4z=0} 
		\Leftrightarrow \heva{&x=30 \\ &y=40 \\ &z=30.}$$
		Vậy nhóm $A$ có $30$ bạn, nhóm $B$ có $40$ bạn, nhóm $C$ có $30$ bạn. 
	}
\end{bt}


\begin{bt}
	Một cửa hàng giải khát chỉ phục vụ ba loại sinh tố: xoài, bơ và mãng cầu. Để pha mỗi li (cốc) sinh tố này đều cần dùng đến sữa đặc, sữa tươi và sữa chua với công thức cho ở bảng sau.
	\begin{center}
		\begin{tabular}{|>{\centering\arraybackslash}m{3.5cm}|>{\centering\arraybackslash}m{3.5cm}|>{\centering\arraybackslash}m{3.5cm}|>{\centering\arraybackslash}m{3.5cm}|}
			\hline
			Sinh tố (li) 
			& Sữa đặc (m$l$) 
			& Sữa tươi (m$l$) 
			& Sữa chua (m$l$) \\
			\hline 
			Xoài 
			& $20$
			& $100$
			& $30$ \\
			\hline 
			Bơ
			& $10$
			& $120$
			&  $20$\\
			\hline 
			Mãng cầu 
			& $20$
			& $100$
			& $20$\\
			\hline
		\end{tabular}
	\end{center}
	Ngày hôm qua cửa hàng đã dùng hết $2l$ sữa đặc; $12,8l$ sữa tươi và $2,9l$ sữa chua. Cửa hàng đã bán được bao nhiêu li sinh tố mỗi loại trong ngày hôm qua?
	\loigiai{
		Gọi $x$, $y$, $z$ lần lượt là số li sinh tố xoài, bơ và dừa mà cửa hàng bán được ngày hôm qua ($x,y,z\in\mathbb{N}$).\\
		Cửa hàng đã dùng hết $2l$ sữa đặc, suy ra: $20x+10y+20z=2000$.\\
		Cửa hàng đã dùng hết $12,8l$ sữa tươi, suy ra: $100x+120y+100z=12800$.\\
		Cửa hàng đã dùng hết $2,9l$ sữa chua, suy ra: $30x+20y+20z=2900$.\\	
		Như vậy, ta có hệ phương trình 
		$$\heva{&20x+10y+20z=2000 \\ &100x+120y+100z=12800 \\ &30x+20y+20z=2900} \Leftrightarrow \heva{&x=50 \\ &y=40 \\ &z=30.}$$
		Vậy cửa hàng đã bán được $50$ li sinh tố xoài, $40$ li sinh tố bơ, $30$ li sinh tố mãng cầu.
	}
\end{bt}


\begin{bt}
	Ba tế bào $A$, $B$, $C$ sau một số lần nguyên phân tạo ra $168$ tế bào con. Biết số tế bào $A$ tạo ra gấp bốn lần số tế bào $B$ tạo ra và số lần nguyên phân của tế bào $C$ nhiều hơn số lần nguyên phân của tế bào $B$ là bốn lần. Tính số lần nguyên phân của mỗi tế bào.
	\loigiai{
		Gọi $x$, $y$, $z$ lần lượt là số lần nguyên phân của các tế nào $A$, $B$, $C$ ($x,y,z\in\mathbb{N}$).\\
		Ba tế bào $A$, $B$, $C$ sau một số lần nguyên phân tạo ra $168$ tế bào con, suy ra: $2^x + 2^y + 2^z = 168$. \\
		Số tế bào $A$ tạo ra gấp bốn lần số tế bào $B$ tạo ra, suy ra: $2^x = 4\cdot 2^y$.\\
		Số lần nguyên phân của tế bào $C$ nhiều hơn số lần nguyên phân của tế bào $B$ là bốn lần, suy ra: $z=y+4 \Leftrightarrow 16\cdot 2^y - 2^z = 0$.\\
		Như vậy, ta có hệ phương trình
		$$\heva{&2^x + 2^y + 2^z = 168 \\ &2^x - 4\cdot 2^y=0 \\ &16\cdot 2^y - 2^z = 0}
		\Leftrightarrow \heva{&2^x=32 \\ &2^y=8 \\ &2^z=128}
		\Leftrightarrow \heva{&x=5 \\ &y=3 \\ &z=7.}$$
		Vậy số lần nguyên phân của các tế nào $A$, $B$, $C$ lần lượt là $5$, $3$, $7$.
	}
\end{bt}


\begin{bt}
	Cho sơ đồ mạch điện như Hình 3. Biết $R_1 = 4\Omega$, $R_2 = 4\Omega$ và $R_3 = 8\Omega$. Tìm các cường độ dòng điện $I_1$, $I_2$ và $I_3$. 
	\begin{center}
	\begin{center}
		\begin{circuitikz}[european,c/.style={circle,fill,inner sep=1pt}]
			\draw (0,0) coordinate (A) to [R=$R_1$,i>_=$I_1$] (2,0)--(3,0)-- (3,1) to[R=$ R_2 $, i>_=$I_2$]  (7,1)
			(B) -- (3,-1)to[R=$ R_3 $, i>_=$I_3$]  (7,-1)--(7,1)
			(A) -- (0,-2.5)  to[battery2,l_=4 V] (8,-2.5) ;
			\draw (7,0) -- (8,0) -- (8,-2.5) ;
			\path foreach \p/\g in {}{(\p)node[c]{}+(\g:3.5mm) node{$\p$}};
			\draw (4,-4) node{Hình 3.};
		\end{circuitikz}
	\end{center}
	\end{center}
	\loigiai{
		Ta có hệ phương trình 
		$$\heva{&I_1 = I_2 + I_3 \\ &I_2 R_2 = I_3 R_3 \\ &I_1 R_1 + I_2 R_2 = 4} 
		\Leftarrow \heva{&I_1 - I_2 - I_3 = 0 \\ &4I_2 - 8I_3 = 0 \\ &4I_1 + 4I_2 = 4}
		\Leftrightarrow \heva{&I_1=\dfrac{3}{5} \\ &I_2=\dfrac{2}{5} \\ &I_3=\dfrac{1}{5}.}$$
		Vậy $I_1=\dfrac{3}{5}$A, $I_2=\dfrac{2}{5}$A, $I_3=\dfrac{1}{5}$A.
	}
\end{bt}


\begin{bt}
	Cân bằng phương trình phản ứng khi đốt cháy khi methane trong oxygen:
	$$\text{CH}_4+\text{O}_2 \xrightarrow{t^o}{  \text{CO}_2+\text{H}_2\text{O}}.$$
	\loigiai{Gọi $x,y,z$ là các hệ số của $\text{CH}_4$, $\text{O}_2$, $\text{H}_2\text{O}$ trên phương trình, khi đó theo định luật bảo toàn nguyên tố hệ số của $\text{CO}_2$ cũng là $x$.\\
		Ta lại  có: $4x=2z$ hay $4x-2z=0$ và $2y=2x+z$ hay $2x-2y+z=0$.\\
		Ta có hệ phương trình $$\heva{&4x-2z=0\\&2x-2y+z=0}$$
		Chọn $y=2$, khi đó hệ trở thành $\heva{&4x-2z=0\\&2x+z=4}\Leftrightarrow \heva{&x=1\\&y=2\\&z=2}$.\\
		Vậy ta có phương trình
		$$\text{CH}_4+2\text{O}_2 \xrightarrow{t^o}{  \text{CO}_2+2\text{H}_2\text{O}}.$$
	}
\end{bt}

\begin{bt}
	Một nhà máy có ba bộ phận cắt, may, đóng gói để sản xuất ba loại sản phẩm: áo thun, áo sơ mi, áo khoác. Thời gian (tính bằng phút) của mỗi bộ phận để sản xuất 10 cái áo mỗi loại được thể hiện trong bảng sau: 
	\begin{center}
		\begin{tabular}{|c|c|c|c|}
			\hline
			\multirow{2}{*}{Bộ phận}&\multicolumn{3}{c|}{Thời gian (tính bằng phút) để sản xuất 10 cái}\\ \cline{2-4}
			& Áo thun & Áo sơ mi & Áo khoác \\ \hline
			Cắt & 9 & 12 & 15 \\ \hline
			May & 22 & 24 & 28 \\ \hline
			Đóng Gói &6 &8 & 8 \\ \hline
		\end{tabular}
	\end{center}
	Các bộ phận cắt, may và đóng gói có tối đa $80$, $160$ và $48$ giờ lao động tương ứng mỗi ngày. Hãy lập kế hoạch sản xuất để nhà máy hoạt động hết công suất.
	\loigiai{Đổi: 80 giờ $=$ 4800 phút, 160 giờ $=$ 9600 phút, 48 giờ $=$ 2880 phút.\\
		Nhà máy hoạt động hết công suất nghĩa là sử dụng hết thời gian lao động tối đa.\\
		Gọi số lượng áo thun, áo sơ mi, áo khoác cần sản xuất để máy hoạt động hết công suất lần lượt là $x$, $y$, $z$ ($x$, $y$, $z$ nguyên dương).\\
		Dựa vào bảng trên ta có hệ phương trình: $\heva{&9x+12y+15z=4800\\&22x+24y+28z=9600\\&6x+8y+8z=2880}\Leftrightarrow \heva{&x=80\\&y=140\\&z=160}$.\\
		Vậy số lượng áo thun, áo sơ mi, áo khoác cần sản xuất để nhà máy hoạt động hết công suất lần lượt là 80, 140, 160.}
\end{bt}

\begin{bt}
	Bà Hà có 1 tỉ đồng để đàu tư vào cổ phiếu, trái phiếu và gửi tiết kiệm ngân hàng. Cổ phiếu sinh lợi nhuận $12\%$/năm, trong khi trái phiếu và gửi tiết kiệm ngân hàng cho lãi suất lần lượt là $8\%$/năm và $4\%$/năm. Bà Hà đã quy định rằng số tiền gửi tiết kiệm ngân hàng phải bằng tổng của $20\%$ số tiền đầu tư và cổ phiếu và $10\%$ số tiền đầu tư vào trái phiếu. Bà Hà nên phân bố vốn của mình như thế nào để nhận được 100 triệu đồng tiền lãi từ các khoản đầu tư đó trong năm đầu tiên?
	\loigiai{Gọi số tiền bà Hà nên đầu tư và cổ phiếu, trái phiếu và gửi tiết kiệm ngân hàng lần lượt là $x$, $y$, $z$ (triệu đồng).\\
		Theo đề bài ta có:
		\\
		Bà Hà có 1 tỉ đồng nên $x+y+z=1000$.\\
		Số tiền gửi tiết kiệm ngân hàng bằng tổng của $20\%$ số tiền đầu tư và cổ phiếu và $10\%$ số tiền đầu tư vào trái phiếu nên $z=20\% x+10\% y$ hay $2x+y-10z=0$.\\
		Số tiền lãi là 100 triệu đồng, suy ra $12\%x+5\%y+4\%z=100$ hay $3x+2y+z=2500$. Do đó ta có hệ phương trình:
		$$\heva{&x+y+z=1000\\&2x+y-10z=0\\&3x+2y+z=2500}\Leftrightarrow \heva{&x=650\\&y=200\\&z=150}.$$
		Vậy số tiền bà Hà nên đầu tư vào cổ phiếu, trái phiếu và gửi tiết kiệm ngân hàng lần lượt là 650 triệu đồng, 200 triệu đồng, 150 triệu đồng.
	}
\end{bt}

\begin{bt}
	Trên thị trường có ba loại sản phẩm A, B, C với giá mỗi tấn sản phẩm tương ứng là $x, y, z$ (đơn vị: triệu đồng, $x\ge 0, y\ge 0, z\ge 0$). Lượng cung và lượng cầu của mỗi sản phẩm được cho trong bảng dưới đây:
	\begin{center}
		\begin{tabular}{|c|c|c|}
			\hline
			Sản phẩm & Lượng cung & Lượng cầu\\ \hline
			A & $Q_{S_A}=4x-y-z-5$ & $Q_{D_A}=-2x+y+z+9$\\ \hline 
			B & $Q_{S_B}=-x+4y-z-5$ & $Q_{D_B}=x-2y+z+3$\\ \hline 
			C & $Q_{S_C}=-x-y+4z-1$ & $Q_{D_C}=x+y-2z-1$\\ \hline 	
		\end{tabular}
	\end{center}
	Tìm giá của mỗi sản phẩm để thị trường cân bằng.
	\loigiai{Thị trường cân bằng khi $\heva{&Q_{S_A}=Q_{D_A}\\&Q_{S_B}=Q_{D_B}\\&Q_{S_C}=Q_{D_C}}$
		$$\Leftrightarrow \heva{&4x-y-z-5= -2x+y+z+9 \\&-x+4y-z-5=x-2y+z+3   \\&-x-y+4z-1=x+y-2z-1}\Leftrightarrow \heva{&6x-2y-2z=14\\&2x-6y+2z=-8\\&2x+2y-6z=0}\Leftrightarrow \heva{&x=4,5\\&y=3,75\\&z=2,75}.$$
		Vậy giá mỗi sản phẩm A, B, C để thị trường cân bằng lần lượt là 4,5 triệu đồng; 3,75 triệu đồng; 2,75 triệu đồng.}
\end{bt}

\begin{bt}
	Vé vào xem một vở kịch có ba mức giá khác nhau thùy theo khu vực ngồi trong nhà hát. Số lượng vé bán ra và doanh thu của ba suất diễn được cho bởi bảng sau: 
	\begin{center}
		\begin{tabular}{|c|c|c|c|c|}
			\hline
			\multirow{2}{*}{Suất diễn }&\multicolumn{3}{c|}{Số vé bán được cái}&\multirow{2}{*}{Doanh thu (triệu đồng)}\\ \cline{2-4}
			& Khu vực 1 & Khu vực 2 & Khu vực 3&  \\ \hline
			10h00-12h00 & 210 & 152 & 125 &212,7 \\ \hline
			15h00-17h00 & 225 & 165 & 118 &224,4 \\ \hline
			20h00-22h00 &254 &186 & 130 &252,2 \\ \hline
		\end{tabular}
	\end{center}
	Tìm giá vé ứng với mỗi khu vực ngồi trong nhà hát.
	\loigiai{Gọi giá vé ứng với mỗi khu vực 1, khu vực 2, khu vực 3 lần lượt là $x$, $y$, $z$ (triệu đồng).\\
		Dựa vào bảng trên ta có hệ phương trình: $\heva{&210x+152y+125z=212,7\\&225x+165y+118z=22,4\\&254x+186y+130z=252,2}\Leftrightarrow \heva{&x=0,4\\&y=0,6\\&z=0,3}$.\\
		Vậy giá vé tương ứng với mỗi khu vực 1, khu vực 2, khu vực 3 lần lượt là 400 nghìn đồng, 600 nghìn đồng và 300 nghìn đồng.
	}
\end{bt}

