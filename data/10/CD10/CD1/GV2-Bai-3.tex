
% \subsection{BÀI TẬP TRONG SÁCH GIÁO KHOA}
\begin{bt} %[0D3T3-3]
Trên thị trường hàng hoá có ba loại sản phẩm $A$, $B$, $C$ với giá mỗi tấn tương ứng là $x$, $y$, $z$ (đơn vị: triệu đồng, $x \ge 0$, $y \ge 0$, $z \ge 0$). Lượng cung và lượng cầu của mỗi sản phẩm được cho trong bảng dưới đây:
\begin{center}
\begin{tabular}{|p{2cm}|p{5cm}|p{5cm}|}
\hline
Sản phẩm & Lượng cung & Lượng cầu \\
\hline
$A$ & $Q_{S_A}=-60+4x-2z$ & $Q_{D_A}=137-3x+y$ \\
\hline
$B$ & $Q_{S_B}=-30-x+5y-z$ & $Q_{D_B}=131+x-4y+z$ \\
\hline
$C$ & $Q_{S_C}=-30-2x+3z$ & $Q_{D_C}=157+y-2z$ \\
\hline
\end{tabular}
\end{center}
Tìm giá của mỗi sản phẩm để thị trường cân bằng.
\loigiai{
Thị trường cân bằng khi$\heva{&Q_{S_A}=Q_{D_A}\\&Q_{S_B}=Q_{D_B}\\&Q_{S_C}=Q_{D_C}}$\\
$\Leftrightarrow \heva{&-60+4x-2z=137-3x+y\\&-30-x+5y-z=131+x-4y+z\\&-30-2x+3z=157+y-2z}\Leftrightarrow \heva{&7x-y-2z=197\\&2x-9y+2z=-161\\&2x+y-5z=-187}\Leftrightarrow \heva{&x=54\\&y=45\\&z=68}$\\
Vậy giá mỗi sản phẩm $A$, $B$, $C$ lần lượt là $54$, $45$ và $68$ triệu đồng.}
\end{bt}
\begin{bt} %[0D3T3-3]
Giải bài toán cổ sau:
\begin{center}
\textit{Trăm trâu, trăm cỏ\\
Trâu đứng ăn năm\\
Trâu nằm ăn ba\\
Lụ khụ trâu già\\
Ba con một bó}
\end{center}
Hỏi có bao nhiêu con trâu đứng, trâu nằm, trâu già?
\loigiai{
Gọi số trâu đứng, trâu nằm, trâu già lần lượt là $x$, $y$, $z$ ($x$, $y$, $z \in \mathbb{Z}^+$).\\
Theo đề bài ta có hệ phương trình: $\heva{&x+y+z=100\\&5x+3y+\dfrac{1}{3}z=100}\,\, (*)$.\\
$(*) \Leftrightarrow \heva{&x+y=100-z\\&15x+9y=300-z} \Leftrightarrow \heva{&x=\dfrac{-300+4z}{3}\\&y=\dfrac{600-7z}{3}} \Leftrightarrow \heva{&x=\dfrac{4z}{3}-100\\&y=200-\dfrac{-7z}{3}}$.\\
Vì $x>0$ nên $\dfrac{4z}{3}-100>0 \Rightarrow z >75.$\\
Vì $y>0$ nên $200-\dfrac{-7z}{3}>0 \Rightarrow z<85$.\\
Mà $z \in \mathbb{Z}^+$ nên $z \in \left\lbrace 76;77;...;84\right\rbrace $.\\
Lại có $x \in \mathbb{Z}^+$ nên $\dfrac{4z}{3}-100 \in \mathbb{Z}^+$, suy ra $z \vdots 3 \Rightarrow z \in \left\lbrace 78;81;84 \right\rbrace .$\\
$\circ$ Với $z=78$ thì $x=4$, $y=18$.\\
$\circ$ Với $z=81$ thì $x=8$, $y=11$.\\
$\circ$ Với $z=84$ thì $x=12$, $y=4$.\\
Vậy số trâu đứng, trâu nằm, trâu già theo thứ tự có thể là một trong ba bộ số $(4; 18; 78)$, $(8; 11; 81)$, $(12; 4; 84)$.}
\end{bt}
% \subsection{BÀI TẬP NÂNG CAO}
\begin{bt}%[0D3T3-3]
Trong phân tử $M_2X$ có tồng số hạt $p$, $n$, $e$ là $140$, trong đó số hạt mang điện nhiều hơn số hạt không mang điện là $44$ hạt. Số khối của $M$ lớn hơn số khối của $X$ là $23$. Tổng số hạt $p$, $n$, $e$ trong nguyên tử $M$ nhiều hơn trong nguyên tử $X$ là $34$ hạt. Công thức phân từ của $M_2X$ là
\loigiai{
Gọi số hạt $p$, $n$, $e$ của nguyên tử $M$ và $X$ lần lượt là $p_M$, $n_M$, $e_M$; $p_X$, $n_X$, $e_X$.\\
Trong phân tử $M_2X$ có tồng số hạt $p$, $n$, $e$ là $140$ nên ta có phương trình
$$2(p_M+n_M+e_M)+(p_X+n_X+e_X)=140$$
$$\Leftrightarrow 2(2p_M+n_M)+(2p_X+n_X)=140 \Leftrightarrow 4p_M+2n_M+2p_X+n_X=140 \,\, (1)$$
Số hạt mang điện nhiều hơn số hạt không mang điện là $44$ hạt nên ta có phương trình 
$$4p_M-2n_M+2p_X-n_X=44 \,\, (2)$$
Số khối của $M$ lớn hơn số khối của $X$ là $23$ nên ta có phương trình
$$(p_M+n_M)-(p_X+n_X)=23 \Leftrightarrow p_M+n_M-p_X-n_X=23 \,\, (3)$$
Tổng số hạt $p$, $n$, $e$ trong nguyên tử $M$ nhiều hơn trong nguyên tử $X$ là $34$ hạt nên ta có phương trình 
$$(2p_M+n_M)-(2p_X+n_X)=34 \Leftrightarrow 2p_M+n_M-2p_X-n_X=34 \,\, (4)$$
Từ $(1)$, $(2)$, $(3)$ và $(4)$ ta có hệ phương trình 
$$\heva{&4p_M+2n_M+2p_X+n_X=140\\&4p_M-2n_M+2p_X-n_X=44\\&p_M+n_M-p_X-n_X=23\\&2p_M+n_M-2p_X-n_X=34} \Leftrightarrow \heva{&4p_M+2n_M+2p_X+n_X=140\\&8p_M+4p_X=184\\&5p_M+3n_M+p_X=163\\&6p_M+3n_M=174}$$
$$\Leftrightarrow \heva{&4p_M+2n_M+2p_X+n_X=140\\&p_M=13\\&n_M=20\\&p_X=8} \Leftrightarrow \heva{&p_M=13\\&n_M=20\\&p_X=8\\&n_X=8}$$
Khi đó $M$ là $K$, $X$ là $O$.\\
Vậy công thức phân tử của $M_2X$ là $K_2O$.}
\end{bt}
\begin{bt} %[0D3T3-3]
Tìm các hệ số $x$, $y$, $z$ để cân bằng phương trình sau:
$$xNa_2SO_3 +2KMnO_4 +yNaHSO_4 \longrightarrow  zNa_2SO_4 +2MnSO_4 +K_2SO_4 + 3H_2O.$$
\loigiai{
Số nguyên tử $Na$ ở hai vế bằng nhau, ta có $2x +y= 2z$ hay $2x+y-2z=0 \,\,(1)$.\\
Số nguyên tử $S$ ở hai vế bằng nhau, ta có $x+y= z+3$ hay $x+y-z= 3 \,\,(2)$.\\
Số nguyên tử $O$ ở hai vế bằng nhau, ta có $3x + 4y +8= 4z + 15$ hay $3x+4y-4z=7 \,\,(3)$.\\
Từ $(1)$, $(2)$ và $(3)$ ta có hệ phương trình $$\heva{&2x+y-2z=0\\&x+y-z= 3\\&3x+4y-4z=7}$$
Giải hệ phương trình ta được $x=5$, $y=6$, $z=8$.\\
Vậy phương trình cân bằng phản ứng hóa học là $5Na_2SO_3 +2KMnO_4 +6NaHSO_4 \longrightarrow  8Na_2SO_4 +2MnSO_4 +K_2SO_4 + 3H_2O.$}
\end{bt}
\Closesolutionfile{ans}