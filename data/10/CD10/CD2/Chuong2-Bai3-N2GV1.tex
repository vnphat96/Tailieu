\setcounter{section}{2}
\section{PHƯƠNG PHÁP QUY NẠP TOÁN HỌC}
\subsection{TÓM TẮT LÝ THUYẾT}
\subsubsection{Phương pháp quy nạp toán học}
Để chứng minh một mệnh đề đúng với mọi $n\in\mathbb{N}^*$ bằng phương pháp quy nạp toán học, ta thực hiện các bước sau:
\begin{enumerate}
	\item[] Bước 1: Kiểm tra mệnh đề đúng với $n=1$.
	\item[] Bước 2: Giả sử mệnh đề đúng với $n=k\ge 1$ (giả thiết quy nạp).
	\item[] Bước 3: Cần chứng minh mệnh đề đúng với $n=k+1$.
\end{enumerate}
\begin{note}
	Trong trường hợp chứng minh một mệnh đề đúng với mọi số tự nhiên $n\ge p$ ($p$ là số tự nhiên) thì thuật toán là 
	\begin{enumerate}
		\item[] Bước 1: Kiểm tra mệnh đề đúng với $n=p$.
		\item[] Bước 2: Giả sử mệnh đề đúng với $n=k\ge 1$ (giả thiết quy nạp)
		\item[] Bước 3: Cần chứng minh mệnh đề đúng với $n=k+1$.
	\end{enumerate}
\end{note}
\subsection{VÍ DỤ}
% ví dụ 1
\begin{vd}
	Chứng minh rằng $n^3+2 \mathrm{n}$ chia hết cho 3 với mọi $n \in \mathbb{N}^*$.
	\loigiai{
		Bước 1 . Với $n=1$, ta có $1^3+2 \cdot 1=3\ \vdots\ 3$. Do đó khẳng định đúng với $n=1$.\\
		Bước 2. Giả sử khẳng định đúng với $n=k \geq 1$, nghĩa là có  $k^3+2 k\ \vdots\  3$.\\
		Bước 3 . Ta cần chứng minh đẳng thức đúng với $n=k+1$, nghĩa là cần chứng minh 
		$$
		(k+1)^3+2(k+1)\ \vdots\ 3.
		$$
		Sử dụng giả thiết quy nạp, ta có 
		$$
		(k+1)^3+2(k+1)=k^3+3 k^2+3 k+1+2 k+2=\left(k^3+2 k\right)+\left(3 k^2+3 k+3\right).
		$$
		Vì $\left(k^3+2 k\right)$ và $\left(3 k^2+3 k+3\right)$ đều chia hết cho 3 nên $\left(k^3+2 k\right)+\left(3 k^2+3 k+3\right)\ \vdots\ 3$  hay $(k+1)^3+2(k+1)\ \vdots\  3$.
		Vậy khẳng định đúng với $n=k+1$.
		Theo nguyên lí quy nạp toán học, khẳng định đúng với mọi số tự nhiên $n \geq 1$.   }
\end{vd}

% ví dụ 2
\begin{vd}
	Chứng minh rằng đẳng thức sau đúng với mọi $n \in \mathbb{N}^*$ 
	$$1+q+q^2+q^3+q^4+\ldots+q^{n-1}=\dfrac{1-q^n}{1-q} \quad(q \neq 1).$$
	\loigiai{
		Bước 1. Với $n=1$, ta có $q^{1-1}=q^0=1=\dfrac{1-q}{1-q}=\dfrac{1-q^1}{1-q}$. Do đó đẳng thức đúng với $n=1$.\\
		Bước 2. Giả sử đẳng thức đúng với $n=k \geq 1$, nghĩa là có 
		$$
		1+q+q^2+q^3+q^4+\ldots+q^{k-1}=\dfrac{1-q^k}{1-q}.
		$$
		Bước 3 . Ta cần chứng minh đẳng thức đúng với $n=k+1$, nghĩa là cần chứng minh 
		$$
		1+q+q^2+q^3+q^4+\ldots+q^{k-1}+q^{(k+1)-1}=\dfrac{1-q^{k+1}}{1-q}
		$$
		Sử dụng giả thiết quy nạp, ta có 
		$$
		\begin{aligned}
			&1+q+q^2+q^3+q^4+\ldots+q^{k-1}+q^{(k+1)-1} \\
			=&\dfrac{1-q^k}{1-q}+q^{(k+1)-1}=\dfrac{1-q^k}{1-q}+q^k=\dfrac{1-q^k+q^k(1-q)}{1-q}
			=&\dfrac{1-q^k+q^k-q^{k+1}}{1-q}
			=\dfrac{1-q^{k+1}}{1-q} .
		\end{aligned}
		$$
		Vậy đẳng thức đúng với $n=k+1$.
		Theo nguyên lí quy nạp toán học, đẳng thức đúng với mọi số tự nhiên $n \geq 1$.  } 
\end{vd}

% ví dụ 3
\begin{vd}
	Chứng minh rằng trong mặt phẳng, $n$ đường thẳng khác nhau cùng đi qua một điểm chia mặt phẳng thành $2 n$ phần $\left(n \in \mathbb{N}^*\right)$.
	\loigiai{
		Bước 1. Với $n=1$, ta có rõ ràng một đường thẳng chia mặt phẳng thành 2 phần.\\
		Bước 2. Giả sử khẳng định đúng với $n=k \geq 1$, nghĩa là có  $k$ đường thẳng khác nhau đi qua một điểm chia mặt phẳng ra thành $2k$ phần.\\
		Bước 3 . Ta cần chứng minh khẳng định đúng với $n=k+1$, nghĩa là cần chứng minh  $(k+1)$ đường thẳng khác nhau đi qua một điểm chia mặt phẳng ra thành $2(k+1)$ phần.\\
		Sử dụng giả thiết quy nạp, ta có \\
		Nếu dựng đường thẳng đi qua điểm đã cho và không trùng với đường thẳng nào trong số những đường thẳng còn lại, thì ta nhận thêm 2 phần của mặt phẳng. \\
		Như vậy tổng số phần mặt phẳng là của $2k$ cộng thêm 2 , nghĩa là $2(k+1)$.\\
		Vậy khẳng định đúng với $n=k+1$.
		Theo nguyên lí quy nạp toán học, khẳng định đúng với mọi số tự nhiên $n \geq 1$.   }
\end{vd}

\subsection{MỘT SỐ ỨNG DỤNG CỦA PHƯƠNG PHÁP QUY NẠP TOÁN HỌC}
\setcounter{vd}{0}
\subsubsection{Chứng minh tính chất chia hết}
	\begin{vd}
		Chứng minh rằng, với mọi $n \in \mathbb{N}^*$, ta có $5^{2 n}-1$ chia hết cho 24.
		\loigiai{
			Bước 1 . Với $n=1$, ta có $5^{2 \cdot 1}-1=24$ $\vdots\ 24$. Do đó khẳng định đúng với $n=1$.\\
			Bước 2. Giả sử khẳng định đúng với $n=k \geq 1$, nghĩa là có  $5^{2k}-1: 24$.\\
			Bước 3. Ta cần chứng minh đẳng thức đúng với $n=k+1$, nghĩa là cần chứng minh  $5^{2(k+1)}-1\ \vdots\ 24$.\\
			Sử dụng giả thiết quy nạp, ta có  $
			5^{2(k+1)}-1=5^{2 k+2}-1=25 \cdot 5^{2 k}-1=24 \cdot 5^{2 k}+\left(5^{2 k}-1\right)$.\\
			Vì $24\cdot 5^{2 k}$ và $\left(5^{2 k}-1\right)$ đều chia hết cho 24 nên $24\cdot 5^{2 k}+\left(5^{2 k}-1\right)\ \vdots\ 24$ hay $5^{2(k+1)}-1\ \vdots\ 24$.\\
			Vậy khẳng định đúng với $n=k+1$.\\
			Theo nguyên lí quy nạp toán học, khẳng định đúng với mọi số tự nhiên $n \geq 1$.   }
	\end{vd}
\subsubsection{Chứng minh đẳng thức}
	\begin{vd}
		Chứng minh đẳng thức sau đúng với mọi $n \in \mathbb{N}^*$
		$$1\cdot 2+2\cdot 3+3\cdot 4+\ldots+n \cdot(n+1)=\dfrac{n(n+1)(n+2)}{3}.$$
		\loigiai{
			Bước 1 . Với $n=1$, ta có $1(1+1)=2=\dfrac{1(1+1)(1+2)}{3}$. Do đó đẳng thức đúng với $n=1$.\\
			Bước 2. Giả sử đẳng thức đúng với $n=k \geq 1$, nghĩa là có 
			$$1\cdot2+2\cdot3+3\cdot 4+\ldots+k \cdot (k+1)=\dfrac{k(k+1)(k+2)}{3}.$$
			Bước 3. Ta cần chứng minh đẳng thức đúng với $n=k+1$, nghĩa là cần chứng minh 
			$$
			1\cdot 2+2\cdot 3+3\cdot 4+\ldots+k \cdot (k+1)+(k+1)[(k+1)+1]=\dfrac{(k+1)[(k+1)+1][(k+1)+2]}{3}.
			$$ 
			Sử dụng giả thiết quy nạp, ta có 
			$$
			\begin{aligned}
				&1\cdot2+2\cdot3+3\cdot4+\ldots+k \cdot(k+1)+(k+1)[(k+1)+1] \\
				=&\dfrac{k(k+1)(k+2)}{3}+(k+1)(k+2) \\
				=&\dfrac{k(k+1)(k+2)}{3}+\dfrac{3(k+1)(k+2)}{3} \\
				=&\dfrac{(k+1)(k+2)(k+3)}{3} \\
				=&\dfrac{(k+1)[(k+1)+1][(k+1)+2]}{3}.
			\end{aligned}
			$$
			Vậy đẳng thức đúng với $n=k+1$.\\
			Theo nguyên lí quy nạp toán học, đẳng thức đúng với mọi số tự nhiên $n \geq 1$. }
	\end{vd}
\subsubsection{Chứng minh bất đẳng thức}
	\begin{vd}
		Chứng minh rằng bất đẳng thức $2^{n+1}>n^2+n+2,\forall n \ge 3$.
		\loigiai{
			Bước 1. Với $n=3$, ta có $2^{3+1}=16>14=3^2+3+2$. Do đó bất đẳng thức đúng với $n=3$.\\
			Bước 2. Giả sử bất đẳng thức đúng với $n=k \geq 3$, nghĩa là có  $2^{k+1}>k^2+k+2$.\\
			Bước 3. Ta cần chứng minh đẳng thức đúng với $\mathrm{n}=\mathrm{k}+1$,\\ nghĩa là cần chứng minh 
			$2^{(k+1)+1}>(k+1)^2+(k+1)+2$.\\
			Sử dụng giả thiết quy nạp, với lưu ý $k \geq 3$, ta có 
			$$
			2^{(k+1)+1}=2 \cdot 2^{k+1}>2(k 2+k+2)=2 k^2+2 k+4=k^2+k^2+2 k+4>k^2+k+2 k+4
			$$
			$$
			=\left(k^2+2 k+1\right)+(k+1)+2=(k+1)^2+(k+1)+2 .
			$$
			Vậy bất đẳng thức đúng với $n=k+1$.\\
			Theo nguyên lí quy nạp toán học, bất đẳng thức đúng với mọi số tự nhiên $n \geq 3$.  }    
	\end{vd}
\subsubsection{Chứng minh công thức lãi kép}
	\begin{vd}
		(Công thức lãi kép) Một khoản tiền $\mathrm{A}$ đồng (gọi là vốn) được gửi tiết kiệm có kì hạn ở một ngân hàng theo thể thức lãi kép (tiền lãi sau mỗi kì hạn nếu không rút ra thì được cộng vào vốn của kì kế tiếp). Giả sử lãi suất theo kì là $r$ không đổi qua các kì hạn, người gửi không rút tiền vốn và lãi trong suốt các kì hạn đề cập sau đây. Gọi $T_n$ là tổng số tiền vốn và lãi của người gửi sau kì hạn thứ $n$ $\left(n \in \mathbb{N}^*\right)$.
		\begin{enumerate}
			\item  Tính $T_1, T_2, T_3$.
			\item  Từ đó, dự đoán công thức tính $T_n$ và chứng minh công thức đó bằng phương pháp quy nạp toán học.     
		\end{enumerate}
		\loigiai{
			\begin{enumerate}
				\item ~\\
				- Tổng số tiền (cả vốn lẫn lãi) $T_1$ nhận được sau kì thứ 1 là  $T_1=A+A r=A(1+r)$.\\
				- Tổng số tiền (cả vốn lẫn lãi) $T_2$ nhận được sau kì thứ 2 là  $$T_2=A(1+r)+A(1+r) r=A(1+r)(1+r)=A(1+r)^2.$$
				- Tổng số tiền (cả vốn lẫn lãi) $T_3$ nhận được sau kì thứ 3 là  $$T_3=A(1+r)^2+A(1+r)^2 r=A(1+r)^3.$$
				\item  Từ câu a) ta có thể dự đoán $\mathrm{T}_{\mathrm{n}}=\mathrm{A}(1+\mathrm{r})^{\mathrm{n}}$.
				Ta chứng minh bằng quy nạp toán học.\\
				Bước 1. Với $n=1$ ta có $T_1=A(1+r)=A(1+r)^1$.
				Như vậy khẳng định đúng cho trường hợp $\mathrm{n}=1$.
				Bước 2. Giả sử khẳng định đúng với $\mathrm{n}=\mathrm{k} \geq 1$, tức là ta có  $T_k=A(1+r)^k$.\\
				Bước 3. Ta sẽ chứng minh rằng khẳng định cũng đúng với $n=k+1$, nghĩa là ta sẽ chứng minh  $$T_{k+1}=A(1+r)^{k+1}.$$
				Thật vậy,
				tổng số tiền (cả vốn lẫn lãi) $T_{k+1}$ nhận được sau kì thứ $(k+1)$ là 
				$$\mathrm{T}_{\mathrm{k}+1}=\mathrm{A}(1+r)^{\mathrm{k}}+\mathrm{A}(1+r)^{\mathrm{k}} \cdot \mathrm{r}=\mathrm{A}(1+r)^{\mathrm{k}}(1+r)=\mathrm{A}(1+r)^{\mathrm{k}+1}.$$
				Vậy khẳng định đúng với $\mathrm{n}=\mathrm{k}+1$.\\
				Theo nguyên lí quy nạp toán học, khẳng định đúng với mọi số tự nhiên $n \geq 1$.\\
				Vậy $T_n=A(1+r)^n$ với mọi số tự nhiên $n \geq 1$. 
		\end{enumerate}}
	\end{vd}
\subsubsection{Dự đoán công thức tổng hữu hạn và chứng minh bằng phương pháp quy nạp}
	\begin{vd}
		Cho dãy số $\left(u_n\right)$ xác định bởi: $\left\{\begin{array}{l}u_1=1 \\ u_n=2 u_{n-1}+3 \end{array}\right.$, $\forall n \geq 2$.
		\begin{enumerate}
			\item Viết năm số hạng đầu của dãy.
			\item Chứng minh rằng $u_n=2^{n+1}-3$.
		\end{enumerate}
		\loigiai{
			\begin{enumerate}
				\item Ta có 5 số hạng đầu của dãy là \\ $u_1=1 ; u_2=2 u_1+3=5 ; u_3=2 u_2+3=13 ; u_4=2 u_3+3=29$; $u_5=2 u_4+3=61$.
				\item  Ta chứng minh bài toán bằng phương pháp quy nạp\\
				- Với $n=1 \Rightarrow u_1=2^{1+1}-3=1 \Rightarrow$ bài toán đúng với $n=1$.\\
				- Giả sử $u_k=2^{k+1}-3$, ta chứng minh $u_{k+1}=2^{k+2}-3$.\\
				Thật vậy, theo công thức truy hồi ta có  $u_{k+1}=2 u_k+3=2\left(2^{k+1}-3\right)+3=2^{k+2}-3$.     
		\end{enumerate}} 
	\end{vd}