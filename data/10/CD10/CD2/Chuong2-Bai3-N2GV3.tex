\begin{bt}%[1D3K1-1]
	Chứng minh rằng $n^{3}-n+3$ chia hết cho $3$ với mọi số tự nhiên $n \geq 1$.
	\loigiai{ Ta chứng minh khẳng định trên bằng quy nạp theo $ n $, với $n \geq 1$.\\
		Với $n = 1$, ta thấy $n^{3}-n+3=3 \, \vdots \, 3$. Vậy khẳng định đúng với $n=1$.\\
		Giả sử khẳng định đúng với $n=k \geq 1$, tức là $k^{3}-k+3 \, \vdots \, 3$.\\
		Ta cần chứng minh khẳng định đúng với $n=k+1$, tức là chứng minh $(k+1)^{3} - (k+1) + 3 \, \vdots \, 3$. Thật vậy, theo giả thiết quy nạp, ta có
		\begin{center}
			$(k+1)^{3} - (k+1) + 3 = \left( k^{3} - k + 3 \right) + 3k^{2} + 3k + 3 \, \vdots \, 3$
		\end{center}
		Vậy khẳng định đúng với mọi số tự nhiên $n \geq 1$.
	}
\end{bt}

\begin{bt}%[1D3K1-1]
	Chứng minh rằng $n^{2} - n + 41$ là số lẻ với mọi số nguyên dương $n$.
	\loigiai{ Ta chứng minh $n^{2} - n + 41 \not{\vdots}\, 2$ bằng quy nạp theo $ n $, với $n \geq 1$.\\
		Với $n=1$, ta thấy $n^{2} - n + 41 = 41 \not{\vdots}\, 2$. Vậy khẳng định trên đúng với $n=1$.\\
		Giả sử khẳng định đúng với $n=k\geq 1$, tức là $k^{2}-k+41 \not{\vdots}\, 2$.\\
		Ta cần chứng minh khẳng định đúng với $n=k+1$, tức là chứng minh $(k+1)^{2}-(k+1)+41 \\ \not{\vdots}\, 2$. Thật vậy, theo giả thiết quy nạp, ta có
		\begin{center}
			$(k+1)^{2}-(k+1)+41 = (k^{2}-k+41) + 2k \not{\vdots}\, 2 $.
		\end{center}
		Vậy khẳng định đúng với mọi số nguyên dương $n$.
	}
\end{bt}

\begin{bt}%Bài 2.5%[1D3K1-1]
	Chứng minh rằng nếu $x > -1$ thì $(1+x)^{n} \geq 1+nx$ với mọi số tự nhiên $n$.
	\loigiai{ 
		Xét số thực $x > -1$, ta chứng minh $(1+x)^{n} \geq 1+nx$ bằng quy nạp theo $n$.\\
		Với $n = 0$, ta thấy $(1+x)^{0} \geq 1+0.x$ đúng. Vậy khẳng định đúng với $n=0$.\\
		Giả sử khẳng định đúng với $n=k$, tức là $(1+x)^{k} \geq 1+kx$.\\
		Ta cần chứng minh khẳng định đúng với $n=k+1$, tức là chứng minh $$(1+x)^{k+1} \geq 1+(k+1)x$$ Thật vậy, theo giả thiết quy nạp, ta có
		\begin{center}
			$(1+x)^{k+1} = (1+x)^{k} (1+x) \geq (1+kx)(1+x) = 1 + (k+1)x + kx^{2} \geq 1+ (k+1)x.$\\
		\end{center}
		(Vì $1+x>0$ và $kx^{2} \geq 0$)\\
		Vậy khẳng định đúng với mọi số tự nhiên $n$.
	}
\end{bt}

\begin{bt}%Bài 2.6%[1D3K1-1]
	Cho tổng $S_{n} = \dfrac{1}{1\cdot2} + \dfrac{1}{2\cdot3} + \cdots + \dfrac{1}{n\cdot(n+1)}.$
	\begin{enumerate}
		\item Tính $S_{1}, S_{2}, S_{3}$.
		\item Dự đoán công thức tính tổng $S_{n}$ và chứng minh bằng quy nạp.
	\end{enumerate}
	\loigiai{
		\begin{enumerate}
			\item $S_{1} = \dfrac{1}{1.2} = \dfrac{1}{2}, S_{2} = \dfrac{1}{1.2} + \dfrac{1}{2.3} = \dfrac{2}{3}, S_{3} = \dfrac{3}{4}.$
			\item Ta sẽ chứng minh $S_{n} = \dfrac{n}{n+1}$ bằng quy nạp theo $n \in \mathbb{Z}^{+}$.\\
			Theo câu a), khẳng định đúng với $n=1$.\\
			Giả sử khẳng định đúng với $n=k \geq 1$, tức là $S_{k} = \dfrac{k}{k+1}$.\\
			Ta cần chứng minh khẳng định đúng với $n=k+1$, tức là chứng minh $S_{k+1} = \dfrac{k+1}{k+2}$. Thật vậy, theo giả thiết quy nạp, ta có
		\\	$S_{k+1} = \dfrac{1}{1.2} + \dfrac{1}{2.3} + ... + \dfrac{1}{k(k+1)} + \dfrac{1}{(k+1)(k+2)} = S_{k} + \dfrac{1}{(k+1)(k+2)} =$\\
			$\dfrac{k}{k+1} + \dfrac{1}{k+1} - \dfrac{1}{k+2} = 1 - \dfrac{1}{k+2} = \dfrac{k+1}{k+2}.$\\
			Vậy khẳng định đúng với mọi số nguyên dương $n$.
		\end{enumerate}
	}
\end{bt}

\begin{bt}%[1D3K1-1]
	Sử dụng phương pháp quy nạp toán học, chứng minh rằng số đường chéo của một đa giác $n$ cạnh $(n\geq 4)$ là $\dfrac{n(n-3)}{2}$.
	\loigiai{
		Với $n=4$, đa giác 4 cạnh có đúng $2 \left( = \dfrac{4(4-3)}{2} \right)$ đường chéo. Vậy khẳng định đúng với $n=4$.\\
		Giả sử khẳng định đúng với $n=k \geq 4$, tức là số đường chéo của đa giác $k$ cạnh là $\dfrac{k(k-3)}{2}$.\\
		Ta cần chứng minh khẳng định đúng với $n=k+1$, tức là chứng minh số đường chéo của đa giác $k+1$ cạnh là $\dfrac{(k+1)(k-2)}{2}$.\\
		Thật vậy, theo giả thiết quy nạp, ta có đa giác $k$ cạnh có $\dfrac{k(k-3)}{2}$ đường chéo, bây giờ ta xóa đi một cạnh bất kì và thay vào đó hai cạnh mới, thì đa giác thu được có $k+1$ cạnh, đa giác này có $\dfrac{k(k-3)}{2}$ ban đầu và có thêm $k-1$ đường chéo mới. Như vậy số đường chéo của đa giác $k+1$ cạnh sẽ là $\dfrac{k(k-3)}{2} + k - 1 = \dfrac{k^{2}-k-2}{2} = \dfrac{(k+1)(k-2)}{2}$.\\
		Vậy khẳng định đúng với mọi số nguyên dương $n \geq 4$.
	}
\end{bt}

\begin{bt}%[1D3K1-1]
	Ta sẽ "lập luận" bằng quy nạp toán học để chỉ ra rằng "Mọi con mèo đều có cùng màu". Ta gọi $P(n)$ với $n$ nguyên dương là mệnh đề sau: "Mọi con mèo trong một đàn gồm $n$ con đều có cùng màu". \\
	Bước 1. Với $n=1$ thì mệnh đề $P(1)$ là "Mọi con mèo trong một đàn gồm 1 con đều có cùng màu". Hiển nhiên mệnh đề này đúng!\\
	Bước 2. Giả sử $P(k)$ đúng với một số nguyên dương $k$ nào đó. Xét một đàn mèo gồm $k+1$ con. Gọi chúng là $M_{1}, M_{2}, ..., M_{k+1}$. Bỏ con mèo $M_{k+1}$ ra khỏi đàn, ta nhận được một đàn mèo gồm $k$ con là $M_{1}, M_{2}, ..., M_{k}$. Theo giả thiết quy nạp, các con mèo có cùng màu. Bây giờ, thay vì bỏ con mèo $M_{k+1}$, ta bỏ con mèo $M_{1}$ để có đàn mèo gồm $k$ con là $M_{2}, M_{3}, ..., M_{k+1}$. Vẫn theo giả thiết quy nạp thì các con mèo $M_{2}, M_{3}, ..., M_{k+1}$ có cùng màu. Cuối cùng, đưa con mèo $M_{1}$ trở lại đàn để có đàn mèo ban đầu. Theo các lập luận trên: các con mèo $M_{1}, M_{2}, ..., M_{k}$ có cùng màu và các con mèo $M_{2}, M_{3}, ..., M_{k+1}$ có cùng màu. Từ đó suy ra tất cả các con mèo $M_{1}, M_{2}, ..., M_{k+1}$ đều có cùng màu.\\
	Vậy, theo nguyên lí quy nạp thì $P(n)$ đúng với mọi số nguyên dương $n$. Nói riêng nếu gọi $N$ là số mèo hiện tại trên Trái Đất thì việc $P(N)$ đúng cho thấy tất cả các con mèo (trên Trái Đất) đều có cùng màu!\\
	Tất nhiên là ta có thể tìm được các con mèo khác màu nhau! Theo em thì lập luận trên đây sai ở chỗ nào?\\
	\loigiai{
		Lập luận này sai ở Bước 2 khi $k=2$.\\
		Với $k=2$, tức là đàn mèo có 2 con $M_1, M_2$. Khi đó việc tách đàn mèo này thành hai đàn mèo nhỏ, mỗi đàn 1 con mèo sẽ dẫn đến việc hai tập hợp $\left\{M_1, M_2, \ldots, M_k\right\}$ (lúc này chỉ là $\left\{M_1\right\}$ ) và $\left\{M_2, M_3, \ldots, M_k+1\right\}$ (lúc này chỉ là $\left\{M_2\right\}$ ) không có phần tử giao nhau. Do đó không thể suy ra tất cả các con mèo $M_1, M_2, \ldots, M_{k+1}$ đều có cùng màu.
	}
\end{bt}
\begin{bt}%[1D3K1-1]
	Chứng minh rằng với mọi $ n\in\mathbb{N}^* $, ta có $ 1^2+2^2+3^2+\ldots+n^2=\dfrac{n(n+1)(2n+1)}{6}. \ (*) $
	\loigiai{
		\begin{itemize}
			\item [$\bullet$] Với $ n=1 $, ta có $ VT_{(*)}=VP_{(*)}=1 $. Suy ra, $ (*) $ đúng với $ n=1 $.
			\item [$\bullet$] Giả sử $ (*) $ đúng với $ n=k $, nghĩa là ta có $$ 1^2+2^2+3^2+\ldots+k^2=\dfrac{k(k+1)(2k+1)}{6}. $$
			\item [$\bullet$] Ta cần chứng minh $ 1^2+2^2+3^2+\ldots+(k+1)^2=\dfrac{(k+1)(k+2)(2k+3)}{6}. $\\
			Thật vậy, ta có $ \begin{aligned}[t]
				\underbrace{1^2+2^2+3^2+\ldots+(k)^2}_{\dfrac{k(k+1)(2k+1)}{6}}+(k+1)^2&=\dfrac{k(k+1)(2k+1)}{6}+(k+1)^2\\&=\dfrac{(k+1)(2k^2+k+6k+6)}{6}\\&=\dfrac{(k+1)(k+2)(2k+3)}{6}.
			\end{aligned} $\\
			$ \Rightarrow (*) $ đúng khi $ n=k+1 $.
			\item [$\bullet$] Vậy theo nguyên lý quy nạp, $ (*) $ đúng với mọi số nguyên dương $ n $.
		\end{itemize}
	}
\end{bt}
\begin{bt}
	Chứng minh rằng với mọi $ n\in\mathbb{N}^* $, ta có $$ 1\cdot 2+2\cdot 3+3\cdot 4+\ldots+n(n+1)=\dfrac{n(n+1)(n+2)}{3}. \ (*) $$
	\loigiai{
		\begin{itemize}
			\item [$\bullet$] Với $ n=1 $, ta có $ VT_{(*)}=VP_{(*)}=2 $. Suy ra, $ (*) $ đúng với $ n=1 $.
			\item [$\bullet$] Giả sử $ (*) $ đúng với $ n=k $, nghĩa là ta có $$ 1\cdot 2+2\cdot 3+3\cdot 4+\ldots+k(k+1)=\dfrac{k(k+1)(k+2)}{3}. $$
			\item [$\bullet$] Ta cần chứng minh $ 1\cdot 2+2\cdot 3+3\cdot 4+\ldots+(k+1)(k+2)=\dfrac{(k+1)(k+2)(k+3)}{3}. $\\
			Thật vậy, ta có $$ \begin{aligned}[t]
				\underbrace{1\cdot 2+2\cdot 3+3\cdot 4+\ldots+k(k+1)}_{\dfrac{k(k+1)(k+2)}{3}}+(k+1)(k+2)&=\dfrac{k(k+1)(k+2)}{3}+(k+1)(k+2)\\&=\dfrac{(k+1)(k+2)(k+3)}{3}.
			\end{aligned} $$
			$ \Rightarrow (*) $ đúng khi $ n=k+1 $.
			\item [$\bullet$] Vậy theo nguyên lý quy nạp, $ (*) $ đúng với mọi số nguyên dương $ n $.
		\end{itemize}
	}
\end{bt}
\begin{bt} Chứng minh
\[1+2+2^2+2^3+\ldots +2^{n-1}=2^n-1\quad (*)\]
\loigiai{
\begin{enumerate}[Bước 1.]
	\item Với $ n=1 $, ta có $1=2^1-1$ (đúng).
	\item Giả sử (*) đúng với $ n=k \;\left(k\in \mathbb{N}, k\ge 1\right) $, ta sẽ chứng minh công thức đúng với $ n=k+1 $. Tức là chứng minh:
	\[1+2+2^2+2^3+\ldots +2^{k-1}+2^{k}=2^{k+1}-1\]
	Thật vậy, ta có:
	\[\begin{array}{ll}
		1+2+2^2+2^3+\ldots +2^{k-1}+2^{k}&=(2^k+1)+2^k\\
		&=2.2^k+1\\
		&=2^{k+1}+1
	\end{array}\]
	Do đó, (*) đúng với $ n=k+1 $.
\end{enumerate}
Vậy (*) đúng với mọi $ n\in \mathbb{N}^* $.}
\end{bt}
\begin{bt}%[1D3K1-1]
Chứng minh
\[5^{2n}-1 \text{ chia hết cho } 24 (*)\]
\loigiai{
\begin{enumerate}[Bước 1.]
	\item Với $ n=1 $, ta có $5^2-1=24 \;\vdots \;24$ (đúng).
	\item Giả sử (*) đúng với $ n=k \;\left(k\in \mathbb{N}, k\ge 1\right) $, ta sẽ chứng minh công thức đúng với $ n=k+1 $. Tức là chứng minh:
	\[5^{2(k+1)}-1 \vdots 24\]
	Thật vậy, ta có:
	\[\begin{array}{ll}
		5^{2(k+1)}-1&=5^{2k+2}-1\\
		&=25.5^{2k}-25+24\\
		&=25.(5^{2k}-1) +24
	\end{array}\]
	Vì $\left(5^{2k}-1\right) \vdots 24$ do giả thiết quy nạp, và $ 24 \; \vdots \;24 $, nên:
	\[\left(5^{2(k+1)}-1 \right)\text{ chia hết cho } 24\]
	Do đó, (*) đúng với $ n=k+1 $.
\end{enumerate}
Vậy (*) đúng với mọi $ n\in \mathbb{N}^* $.}
\end{bt}
\begin{bt}%Bài 2.3%[1D3K1-1]
	Chứng minh rằng $n^{3}+5n$ chia hết cho $6$ với mọi số tự nhiên $n \geq 1$.
	\loigiai{ Ta chứng minh khẳng định trên bằng quy nạp theo $ n $, với $n \geq 1$.\\
		Với $n = 1$, ta thấy $n^{3}+5n=6 \, \vdots \, 3$. Vậy khẳng định đúng với $n=1$.\\
		Giả sử khẳng định đúng với $n=k \geq 1$, tức là $k^3+5k \, \vdots \, 6$.\\
		Ta cần chứng minh khẳng định đúng với $n=k+1$, tức là chứng minh $(k+1)^{3} +5(k+1)  \, \vdots \, 6$. Thật vậy, theo giả thiết quy nạp, ta có
		\begin{center}
			$(k+1)^{3} +5(k+1)  = \left( k^{3} +5k + 6+3k(k+1) \right)  \, \vdots \, 6$ (do $k(k+1)$ chia hết cho 2)
		\end{center}
		Vậy khẳng định đúng với mọi số tự nhiên $n \geq 1$.
	}
\end{bt}
\begin{bt}%[1D3G1-1]
	Cho $a, b \geq 0$. Chứng minh rằng bất đẳng thức sau đúng với mọi $n \in \mathbb{N}^*$ :
	$$
	\dfrac{a^n+b^n}{2} \geq\left(\dfrac{a+b}{2}\right)^n
	$$
	\loigiai{Trước hết nhận xét rằng nếu $a=b$ thì bất đẳng thức (5) xảy ra dấu bằng) với mọi $n \in \mathbb{N}^*$.\\
		Giả sử $a \neq b$.\\
		Nếu $n=1$ thì bất đẩng thức (5) đúng và dấu bằng xảy ra.\\
		Ta sẽ chứng minh với $n \geq 2$ thì bất đẳng thức (5) đúng, bằng phương pháp quy nạp. Thật vậy:\\
		Với $n=2$ thì (5) có dạng $\dfrac{a^2+b^2}{2} \geq\left(\dfrac{a+b}{2}\right)^2$ hay $(a-b)^2 \geq 0$.\\
		Rõ ràng bất đẳng thức này đúng và dấu bằng không xảy ra.\\
		Giả sử bất đẳng thức (5) đúng với ${a} \neq {b}$ và ${n}={k} \geq 2$, tức là
		$
		\dfrac{a^k+b^k}{2}>\left(\dfrac{a+b}{2}\right)^k
		$\\
		Nhân hai vế của bất đẳng thức này với $a+b>0$, ta có
		$\dfrac{a^k+b^k}{2} \cdot(a+b)>\dfrac{(a+b)^{k+1}}{2^k}$
		
		hay $ \dfrac{a^{k+1}+a^k b+a b^k+b^{k+1}}{2}>\dfrac{(a+b)^{k+1}}{2^k} (*)$\\
		Vì	$a^{k+1}+b^{k+1}-\left(a^k b+a b^k\right)=a^k(a-b)-b^k(a-b)$
		$=(a-b)\left(a^k-b^k\right)>0 $
		nên $$ a^{k+1}+b^{k+1}>a^k b+a b^k (**).$$
		Từ $ (*) $ và $(*)$ suy ra
		$$
		\dfrac{\left(a^{k+1}+b^{k+1}\right)+\left(a^{k+1}+b^{k+1}\right)}{2}>\dfrac{(a+b)^{k+1}}{2^k}
		$$
		hay $\dfrac{a^{k+1}+b^{k+1}}{2}>\left(\dfrac{a+b}{2}\right)^{k+1}$;
		nghĩa là bất đẳng thức (5) đúng với $n=k+1$.\\
		Vậy bất đẳng thức đã được chứng minh.}
\end{bt}




\begin{bt}%[1D3K1-1]
	Chứng minh rằng bất đẳng thức sau đúng với mọi số tự nhiên $n \geq 2$ :
	$$
	1+\dfrac{1}{2}+\dfrac{1}{3}+\cdots+\dfrac{1}{n}>\dfrac{2 n}{n+1} \quad (*)
	$$
	\loigiai{Với $ n=2 $ thì $ 1+\dfrac{1}{2}>\dfrac{2.2}{2+1} \Leftrightarrow \dfrac{3}{2} > \dfrac{4}{3}$ (Đúng).\\
		Giả sử $ (*) $ đúng với $ n=k \,(k \ge 2) $, tức là:
		$
		1+\dfrac{1}{2}+\dfrac{1}{3}+\cdots+\dfrac{1}{k}>\dfrac{2k}{k+1}
		$\\
		Ta chứng minh $ (*) $ đúng với $ n=k+1$.\\
		Ta có $
		1+\dfrac{1}{2}+\dfrac{1}{3}+\cdots+\dfrac{1}{k+1}=1+\dfrac{1}{2}+\dfrac{1}{3}+\cdots+\dfrac{1}{k}+\dfrac{1}{k+1}>\dfrac{2k}{k+1}+\dfrac{1}{k+1}=\dfrac{2k+1}{k+1}
		$\\
		Mặt khác $ \dfrac{2k+1}{k+1}>\dfrac{2(k+1)}{k+2} (1)$ 
		\\Vì $ (1) \Leftrightarrow (2k+1)(k+2)>2(k+1)(k+1)\Leftrightarrow 2k^2+5k+2>2k^2+4k+2 \Leftrightarrow k>0$ (Đúng).\\
		Do đó, $
		1+\dfrac{1}{2}+\dfrac{1}{3}+\cdots+\dfrac{1}{k+1}>\dfrac{2(k+1)}{k+2}$\\
		Vậy bất đẳng thức đã được chứng minh.}
\end{bt}




\begin{bt}%[1D3K1-1]
	 Trong mặt phẳng, cho đa giác $A_1A_2A_3\ldots A_n $ có $ n $ cạnh $ \left(n\ge 3\right) $. Gọi $ S_n $ là tổng số đo các góc trong của đa giác.
	\begin{enumerate}
		\item Tính $ S_3,~ S_4,~S_5 $ tương ứng với trường hợp tam giác, tứ giác và ngũ giác.
		\item Từ đó, dự đoán công thức tính $ S_n $ và chứng minh công thức đó bằng phương pháp quy nạp toán học.
		
	\end{enumerate}
	\loigiai{
		\begin{enumerate}
			\item $ S_3=180^{\circ},~ S_4=360^{\circ}, ~S_5=540^{\circ} $.
			\item Ta dự đoán $ S_n=\left(n-2\right).180^{\circ} $.\\
			Ta chứng minh công thức trên bằng phương pháp quy nạp toán học.
			\begin{enumerate}[Bước 1.]
				\item Với $ n=3 $, ta có tổng 3 góc trong tam giác bằng $ 180^{\circ}=(3-2).180^{\circ} $. Vậy công thức đúng với $ n=3 $.
				\item Gỉa sử công thức đúng với $ n=k \;\left(k\in \mathbb{N}, k\ge 3\right) $, ta sẽ chứng minh công thức đúng với $ n=k+1 $. Tức là chứng minh:
				\[S_{k+1}=\left[(k+1)-2\right].180^{\circ}=\left(k-1\right).180^{\circ}\]
				Thật vậy, xét đa giác có $ k+1 $ cạnh $ A_1A_2\ldots A_k A_{k+1} $, nối hai đỉnh $ A_1 $ và $ A_k $, ta được đa giác có $ k $ cạnh $ A_1A_2\ldots A_k $. Theo giả thiết quy nạp, ta có tổng các góc trong đa giác $ k $ cạnh này là $ S_k=\left(k-2\right).180^{\circ} $.\\
				Dễ thấy tổng các góc trong đa giác $ A_1A_2\ldots A_k A_{k+1} $ bằng với tổng các góc trong đa giác $ A_1A_2\ldots A_k $ và tổng các góc trong tam giác $ A_1A_kA_{k+1} $, tức là:
				\[S_{k+1}=\left(k-2\right).180^{\circ}+180^{\circ}=\left(k-1\right).180^{\circ}\]
				Do đó, công thức đúng với $ n=k+1 $
			\end{enumerate}
			Vậy công thức đúng với mọi đa giác $ n $ cạnh $ \left(n \ge 3\right) $.
	\end{enumerate}}
	
\end{bt}
\begin{bt}%[1D3K1-1]
	 Hàng tháng, một người gửi vào ngân hàng một khoản tiền tiết kiệm không đổi $a$ đồng. Giả sử lãi suất hằng thàng là $r$ không đổi và theo thể thức lãi kép (tiền lãi của tháng trước được cộng vào vốn của tháng kế tiếp). Gọi $T_n \left( n\ge 1\right)$, là tổng số tiền vốn và lãi của người đó có trong ngân hàng sau $ n $ tháng. 
	\begin{enumerate}
		\item Tính $T_1, T_2, T_3$.
		\item Từ đó, dự đoán công thức tính $T_n$, và chứng minh công thức đó bằng phương pháp quy nạp toán học.
	\end{enumerate}
	\loigiai{
		\begin{enumerate}
			\item 	${T}_{1}=A+Ar=A\left( 1+r \right)$ \\ 
			${T}_{2}=A\left( 1+r \right)+A\left( 1+r \right)r=A\left( 1+r \right)\left( 1+r \right)=A{{\left( 1+r \right)}^{2}}$ \\ 
			${{T}_{3}}=A{{\left( 1+r \right)}^{2}}+A{{\left( 1+r \right)}^{2}}r=A{{\left( 1+r \right)}^{2}}\left( 1+r \right)=A{{\left( 1+r \right)}^{3}}$
			\item Từ câu a) ta dự đoán:
			${{T}_{n}}=A{{\left( 1+r \right)}^{n}} n\ge 1 (2)$
			Ta sẽ chứng minh (2) đúng với mọi $n\in {{\mathbb{N}}^{*}}$ bằng phương pháp quy nạp toán học.
			\begin{enumerate}[Bước 1.]
				\item Ở câu a) ta đã biết (2) đúng với $n=1$.
				\item Giả sử (2) đúng với $n=k\ge 1$, tức là tổng tiền vốn và lãi của người đó sau kì hạn thứ k là ${{T}_{k}}=A{{\left( 1+r \right)}^{k}}$.
				Số tiền trên là vốn của kì hạn thứ $k+1$. Do đó số tiền vốn và lãi của người đó sau kì hạn thứ $k+1$ là:
				\[\begin{array}{ll}
					{{T_k}}&{ = A{{\left( {1 + r} \right)}^k} + A{{\left( {1 + r} \right)}^k}r} \\ 
					{}&{ = A{{\left( {1 + r} \right)}^{k + 1}}} 
				\end{array}\]
				Do đó (2) đúng với $n=k+1$.
			\end{enumerate}
			Theo nguyên lí quy nạp công thức (2) đúng với mọi $n\in {{\mathbb{N}}^{*}}$.
		\end{enumerate}
	}
\end{bt}
\Closesolutionfile{ans}