\subsection{Bài tập tự luận}

\begin{bt}%[1D2B3-2]
	Tìm hệ số của $x^8$ trong khai triển của $(2x + 3)^{10}$.
	\loigiai{
	Số hạng chứa $x^k$ trong khai triển của $(2x + 3)^{10}$ là $T_{k+1} = \mathrm{C}_{10}^{10-k} \cdot (2x)^k 3^{10-k}$.\\
	Số hạng chứa $x^8$ ứng với $k = 8$, tức là số hạng $\mathrm{C}_{10}^2 \cdot (2x)^8 \cdot 3^2 = 103680 x^8$.\\
	Vậy hệ số của $x^8$ trong khai triển của $(2x + 3)^{10}$ là $103680$.
	}
\end{bt}

\begin{bt}%[1D2B3-2]
	Biết hệ số của $x^2$ trong khai triển của $(1 - 3x)^n$ là $90$. Tìm $n$.
	\loigiai{
		Số hạng chứa $x^2$ trong khai triển của $(1 - 3x)^n$ là $T_{k+1} = \mathrm{C}_n^k \cdot (-3x)^k =(-3)^k \cdot  \mathrm{C}_n^k \cdot x^k$.\\
		Suy ra hệ số của $x^2$ trong khai triển của $(1 - 3x)^n$ ứng với $ k=2$ là $(-3)^2 \mathrm{C}_n^2$.\\
		Ta có: $(-3)^2\mathrm{C}_n^2 = 90 \Leftrightarrow 9 \cdot \dfrac{n(n - 1)}{2} = 90 \Leftrightarrow n(n - 1) = 20 \Leftrightarrow \hoac{&n = 4\\ &n = 5.}$
	}
\end{bt}

\begin{bt}%[1D2B3-2]
	Từ khai triển biểu thức $(3x - 5)^4$ thành đa thức, hãy tính tổng các hệ số của đa thức nhận được.
	\loigiai{
		Ta có 
		\allowdisplaybreaks
		\begin{eqnarray*}
			f(x)&=&(3x - 5)^4\\
			&=&\mathrm{C}_{2n}^0 + \mathrm{C}_{2n}^2 + \mathrm{C}_{2n}^4 + \cdots + \mathrm{C}_{2n}^{2n} = 2^{2021}\\
			&=&\mathrm{C}_4^0 \cdot (3x)^4 + \mathrm{C}_4^1 \cdot (3x)^3 (-5) + \mathrm{C}_4^2 \cdot (3x)^2 (-5)^2 + \mathrm{C}_4^3 \cdot (3x) (-5)^3 + \mathrm{C}_4^4 \cdot (-5)^4.
		\end{eqnarray*}
		Suy ra tổng các hệ số của khai triển là
		$$S = \mathrm{C}_4^0 \cdot 3^4 + \mathrm{C}_4^1 \cdot 3^3 \cdot (-5) + \mathrm{C}_4^2 \cdot 3^2 \cdot (-5)^2 + \mathrm{C}_4^3 \cdot 3 \cdot (-5)^3 + \mathrm{C}_4^4 \cdot (-5)^4 = f(1) = (3 - 5)^4 = 16.$$
	}
\end{bt}

\begin{bt}%[1D2B3-2]
	Tìm hệ số của $x^5$ trong khai triển thành đa thức của biểu thức $$x(1 - 2x)^5 + x^2(1 + 3x)^{10}.$$
	\loigiai{
		Hệ số của $x^5$ trong khai triển thành đa thức của biểu thức $x(1 - 2x)^5 + x^2(1 + 3x)^{10}$ là
		$$\mathrm{C}_5^4 \cdot (-2)^4 + \mathrm{C}_{10}^3 \cdot 3^3 = 3320.$$
	}
\end{bt}

\begin{bt}%[1D2B3-3]
	Tính tổng sau đây:
	$$\mathrm{C}_{2021}^0 - 2\mathrm{C}_{2021}^1 + 2^2\mathrm{C}_{2021}^2 - 2^3 \mathrm{C}_{2021}^3 + \cdots - 2^{2021} \mathrm{C}_{2021}^{2021}.$$
	\loigiai{
		Ta có $$\mathrm{C}_{2021}^0 - 2\mathrm{C}_{2021}^1 + 2^2 \mathrm{C}_{2021}^2 - 2^3 \mathrm{C}_{2021}^3 + \cdots - 2^{2021} \mathrm{C}_{2021}^{2021} = (1 - 2)^{2021} = -1.$$
	}
\end{bt}
%Bài 2.16
\begin{bt}%[1D2B3-1]
	Tìm số tự nhiên $n$ thỏa mãn $$\mathrm{C}_{2n}^0 + \mathrm{C}_{2n}^2 + \mathrm{C}_{2n}^4 + \cdots + \mathrm{C}_{2n}^{2n} = 2^{2021}.$$
	\loigiai{
		Xét khai triển
		$$2^{2n} = (1 + 1)^{2n} = \mathrm{C}_{2n}^0 + \mathrm{C}_{2n}^1 + \mathrm{C}_{2n}^2 + \cdots + \mathrm{C}_{2n}^k + \cdots + \mathrm{C}_{2n}^{2n} \;\;(1).$$
		$$0^{2n} = (1 - 1)^{2n} = \mathrm{C}_{2n}^0 - \mathrm{C}_{2n}^1 + \mathrm{C}_{2n}^2 - \cdots + (-1)^k \mathrm{C}_{2n}^k + \cdots + \mathrm{C}_{2n}^{2n} \;\; ( 2 ).$$
		Ta có : 
		\allowdisplaybreaks
		\begin{eqnarray*}
			\mathrm{C}_{2n}^0 + \mathrm{C}_{2n}^2 + \mathrm{C}_{2n}^4 + \cdots + \mathrm{C}_{2n}^{2n}&=&\mathrm{C}_{2n}^1 + \mathrm{C}_{2n}^3 + \mathrm{C}_{2n}^5 + \cdots + \mathrm{C}_{2n}^{2n-1}\\
			&\Rightarrow&\mathrm{C}_{2n}^0 + \mathrm{C}_{2n}^2 + \mathrm{C}_{2n}^4 + \cdots + \mathrm{C}_{2n}^{2n} = 2^{2021}\\
			&\Leftrightarrow&2n - 1 = 2021\\
			&\Leftrightarrow& n = 1011.
		\end{eqnarray*}
	}
\end{bt}
%Bài 2.17
\begin{bt}%[1D2B3-1]
	Tìm số nguyên dương $n$ sao cho $\mathrm{C}_{n}^0 + 2\mathrm{C}_{n}^1 + 4\mathrm{C}_{n}^2 + \cdots + 2^n\mathrm{C}_{n}^2 = 243$.
	\loigiai{
		Ta có :  $$\mathrm{C}_{n}^0 + 2\mathrm{C}_{n}^1 + 4\mathrm{C}_{n}^2 + \cdots + 2^n\mathrm{C}_{n}^2 = 243 \Leftrightarrow \left(1 + 2\right)^n = 243 \Leftrightarrow 3^n = 3^5 \Leftrightarrow n = 5.$$
	}
\end{bt}
%Bài 2.18KNTT
\begin{bt}%[1D2K3-1]
	Biết rằng $\left(2 + x\right)^{100} = a_0 + a_1x + a_2x^2 + \cdots + a_{100}x^{100}$. Với giá trị nào của $k $ $\left(0 \le k \le 100 \right)$ thì $a_k$ lớn nhất? 
	\loigiai{
		Ta có :  $\left(2 + x\right)^{100} = \mathrm{C}_{100}^0 2^{100} + \mathrm{C}_{100}^1 2^{99}x + \mathrm{C}_{100}^2 2^{98}x^2 + \cdots + \mathrm{C}_{100}^{99} 2x^{99} + \mathrm{C}_{100}^{100}x^{100}$.\\
		Suy ra $a_k = \mathrm{C}_{100}^k 2^{100-k}$.
		\allowdisplaybreaks
		\begin{eqnarray*}
			a_k \;\;\text{lớn nhất}&\Leftrightarrow&\heva{a_k \geq a_{k - 1}\\ a_k \geq a_{k + 1}}\\
			&\Leftrightarrow&\heva{\mathrm{C}_{100}^k 2^{100 - k} \geq \mathrm{C}_{100}^{k - 1} 2^{100 - k + 1} \\\mathrm{C}_{100}^k 2^{100 - k} \geq \mathrm{C}_{100}^{k + 1} 2^{100 - k - 1}}\\
			&\Leftrightarrow& \heva{\dfrac{100!}{k!(100 - k)!} \cdot 2^{100 - k} \geq \dfrac{100!}{(k - 1)!(100 - k + 1)!} \cdot 2^{100 - k + 1} \\\dfrac{100!}{k!(100 - k)!} \cdot 2^{100 - k} \geq \dfrac{100!}{(k + 1)!(100 - k - 1)!} \cdot 2^{100 - k + 1}}\\
			&\Leftrightarrow& \heva{\dfrac{1}{k} \geq \dfrac{2}{201 - k} \\\dfrac{2}{100 - k} \geq \dfrac{1}{k + 1}}\\
			&\Leftrightarrow& \heva{100 - 2k \geq 2k \\ 2(k + 1) \geq 100 - k}\\
			&\Leftrightarrow& \dfrac{98}{3} \leq k \leq \dfrac{101}{3}\\
			&\Rightarrow& k = 33.
		\end{eqnarray*}
		Vậy $k = 33$ thì $a_{33} = \mathrm{C}_{100}^{33} \cdot 2^{67}$ lớn nhất.
	}
\end{bt}
%Bài 1 CTST
\begin{bt}%[1D2B3-1]
	Khai triển biểu thức:
	\begin{listEX}[2]
		\item $(x - 2y)^6$.
		\item $(3x - 1)^5$.
	\end{listEX}
	\loigiai{
	\begin{listEX}[1]
		\item Ta có
		\allowdisplaybreaks
		\begin{eqnarray*}
			(x - 2y)^6 &=&\mathrm{C}_6^0 x^6 -  \mathrm{C}_6^1 x^5 2y + \mathrm{C}_6^2 x^4 (2y)^2 - \mathrm{C}_6^3 x^3 (2y)^3 + \mathrm{C}_6^4 x^2 (2y)^4 - \mathrm{C}_6^5 x (2y)^5 + \mathrm{C}_6^6 (2y)^6\\
			&=& x^6 -  12x^5y + 60 x^4 y^2 - 160 x^3 y^3 + 240 x^2 y^4 - 192 x y^5 + 64 y^6.
		\end{eqnarray*}
		\item Ta có
		\allowdisplaybreaks
		\begin{eqnarray*}
			(3x - 1)^5 &=&\mathrm{C}_5^0 (3x)^5 -  \mathrm{C}_5^1 (3x)^4 + \mathrm{C}_5^2 (3x)^3 - \mathrm{C}_5^3 (3x)^2 + \mathrm{C}_5^4 3x - \mathrm{C}_5^5\\
			&=& 243x^5 -  405x^4 + 270x^3 - 90x^2 + 15x - 1.
		\end{eqnarray*}
	\end{listEX}
	}
\end{bt}
%Bài 2 CTST
\begin{bt}%[1D2B3-2]
	Tìm hệ số của $x^{10}$ trong khai triển của biểu thức $(2 - x)^{12}$.
	\loigiai{
		Số hạng chứa $x^k$ trong khai triển của $(2 - x)^{12}$ là $T = \mathrm{C}_{12}^{12-k} \cdot 2^k \cdot (-x)^{12-k}$.\\
		Số hạng chứa $x^{10}$ ứng với $k = 2$, tức là số hạng $\mathrm{C}_{12}^{10} \cdot 2^2 \cdot x^{10} = 264 x^{10}$.\\
		Vậy hệ số của $x^{10}$ trong khai triển của $(2 - x)^{12}$ là $264$.
	}
\end{bt}
%Bài 3 CTST
\begin{bt}%[1D2B3-2]
	Biết rằng $a$ là một số thực khác $0$ và trong khai triển của $(ax + 1)^6$, hệ số của $x^4$ gấp bốn lần hệ số của $x^2$. Tìm giá trị của $a$.
	\loigiai{
		Số hạng chứa $x^k$ trong khai triển của $(ax + 1)^6$ là $T = \mathrm{C}_6^{6 - k} \cdot (ax)^k$.\\
		Số hạng chứa $x^4$ ứng với $k = 4$, tức là số hạng $\mathrm{C}_6^2 \cdot a^4x^4 = 15a^4x^4$.\\
		Số hạng chứa $x^2$ ứng với $k = 2$, tức là số hạng $\mathrm{C}_6^4 \cdot a^2x^2 = 15a^2x^2$.\\
		Do hệ số của $x^4$ gấp bốn lần hệ số của $x^2$ nên 
		\[15a^4 = 15a^2 \Leftrightarrow 15a^2(a^2 - 1) = 0 \Leftrightarrow \hoac{&a = 0 \\&
		a = 1 \;\; \text{(nhận)} \\& a = -1.}\]
		Vậy $a = 1$.
	}
\end{bt}
%Bài 4 CTST
\begin{bt}%[1D2B3-2]
	Biết rằng hệ số của $x^2$ trong khai triển của $(1 + 3x)^n$ là $90$. Tìm giá trị của $n$.
	\loigiai{
		Số hạng chứa $x^2$ trong khai triển của $(1 + 3x)^n$ là $T = \mathrm{C}_n^k \cdot (3x)^k = 3^k \cdot  \mathrm{C}_n^k \cdot x^k$.\\
		Suy ra hệ số của $x^2$ trong khai triển của $(1 + 3x)^n$ ứng với $k = 2$ là $3^2 \mathrm{C}_n^2$.\\
		Ta có: $3^2 \mathrm{C}_n^2 = 90 \Leftrightarrow 9 \cdot \dfrac{n(n - 1)}{2} = 90 \Leftrightarrow n(n - 1) = 20 \Leftrightarrow \hoac{&n = 4\\ &n = 5.}$
	}
\end{bt}
%Bài 5 CTST
\begin{bt}%[1D2K3-3]
	Chứng minh công thức nhị thức Newton (công thức (1), trang 35) bằng phương pháp quy nạp toán học.\\
	Chứng minh rằng: $\left(a + b\right)^n = \mathrm{C}_n^0 a^n + \mathrm{C}_n^1 a^{n-1}b + \mathrm{C}_n^2 a^{n-2}b^2 + \cdots + \mathrm{C}_n^k a^{n-k}b^k + \cdots + \mathrm{C}_n^n b^n \;\; \left(1\right)$.\\
	\loigiai{
	Chứng minh:
	Ta chứng minh $(1)$ bằng phương pháp quy nạp theo $n$.\\
	•	Khi $n = 1$, ta có : $\left(a + b\right)^1 = a + b = \mathrm{C}_1^0 a + \mathrm{C}_1^1b$.\\
	Vậy công thức $(1)$ đúng với $n = 1$.\\
	•	Với giả thiết $(1)$ là đúng với $n = m$, tức là ta có :	
	$$\left(a + b\right)^m = \mathrm{C}_m^0 a^m + \mathrm{C}_m^1 a^{m-1}b + \cdots + \mathrm{C}_m^{m-1} ab^{m-1} + \mathrm{C}_m^m b^m.$$			
	Ta sẽ chứng minh\\
	$$\left(a + b\right)^{m+1} = \mathrm{C}_{m+1}^0 a^{m+1} + \mathrm{C}_{m+1}^1 a^mb +...+C_{m+1}^mab^m+C_{m+1}^{m+1}b^{m+1} \;\; (2).$$
	Thật vậy, ta có 
	\allowdisplaybreaks
	\begin{eqnarray*}
		&&\left(a + b\right)^{m+1} \\
		&=&\left(a + b\right)^m \left(a + b\right)\\
		&=& \left(\mathrm{C}_m^0 a^m + \mathrm{C}_m^1 a^{m-1}b + \cdots + \mathrm{C}_m^{m-1}ab^{m-1} + \mathrm{C}_m^m b^m\right)\left(a + b\right)\\
		&=& \left(\mathrm{C}_m^0 a^m + \mathrm{C}_m^1a^{m-1}b + \cdots  + \mathrm{C}_m^mb^m\right)a + \left(\mathrm{C}_m^0 a^m + \mathrm{C}_m^1a^{m-1}b + \cdots + \mathrm{C}_m^m b^m\right)b\\
		&=& \mathrm{C}_m^0a^{m+1} + \left(\mathrm{C}_m^1 + \mathrm{C}_m^0\right)a^mb + \cdots + \left(\mathrm{C}_m^k + \mathrm{C}_m^{k-1}\right)a^{m+1-k}b^k + \cdots + \mathrm{C}_m^mb^m.
	\end{eqnarray*}
	+ Vì  $\mathrm{C}_m^0 = 1 = \mathrm{C}_{m+1}^0, \mathrm{C}_m^m = 1 = \mathrm{C}_{m+1}^{m+1}, \mathrm{C}_m^k + \mathrm{C}_m^{k-1} = \mathrm{C}_{m+1}^k$ nên ta có (2).\\
	Vậy công thức nhị thức Newton là đúng với mọi số nguyên dương $n$.\\
	\textbf{Chú ý:}  Số hạng thứ $\left(k + 1\right)$ trong khai triển của $\left(a + b\right)^n$ thành dạng (1) là $$T_{k+1} = \mathrm{C}_n^k a^{n-k}b^k.$$
	}
\end{bt}
%Bài 6 CTST
\begin{bt}%[1D2B3-2]
	Biết rằng $(3x- 1)^7 = a_0 + a_1 x+ a_2 x^2 + a_3 x^3 + a_4 x4 + a_5 x^5 + a_6 x^6 + a_7 x^7$.\\ Hãy tính:
	\begin{listEX}[2]
		\item $a_0 + a_1 + a_2 + a_3 + a_4 + a_5 + a_6 + a_7$;
		\item $a_0 + a_2 + a_4 + a_6$.
	\end{listEX}
	\loigiai{
		\begin{listEX}[1]
			\item Với $x = 1$ ta có $2^7 = a_0 + a_1 + a_2 + a_3 + a_4 + a_5 + a_6 + a_7 .\qquad (1)$
			\item Với $x = -1$ ta có $(-4)^7 = a_0 - a_1 + a_2 - a_3 + a_4 - a_5 + a_6 - a_7. \qquad(2)$\\
			Từ $(1)$ và $(2)$ ta có $2^7 + (-4)^7 = 2(a_0 + a_2 + a_4 + a_6) \Leftrightarrow a_0 + a_2 + a_4 + a_6 = -8128$.
		\end{listEX}
	}
\end{bt}
%Bài 7 CTST
\begin{bt}%[1D2B3-1]
	Một tập hợp có $12$ phần tử thì có tất cả bao nhiêu tập hợp con?
	\loigiai{
		Một tập hợp có $12$ phần tử thì có tất cả $2^{12}$ tập hợp con.
	}
\end{bt}
%Bài 8 CTST
\begin{bt}%[1D2K3-1]
	Từ $15$ bút chì màu có màu khác nhau đôi một.
	\begin{listEX}[1]
		\item Có bao nhiêu cách chọn ra một số bút chì màu, tính cả trường hợp không chọn cái nào?
		\item Có bao nhiêu cách chọn ra ít nhất $8$ bút chì màu?
	\end{listEX}
	\loigiai{
		\begin{listEX}[1]
			\item Số cách chọn ra một số bút chì màu theo yêu cầu bài toán chính là số tập con của tập hợp gồm $15$ phần tử.\\
			Vậy số cách chọn là $n = \mathrm{C}_{15}^0 + \mathrm{C}_{15}^1 + \cdots + \mathrm{C}_{15}^{15} = 2^{15} = 32768 $.
			\item Do $n = \mathrm{C}_{15}^0 + \mathrm{C}_{15}^1 + \cdots + \mathrm{C}_{15}^{15} = 2(\mathrm{C}_{15}^0 + \mathrm{C}_{15}^1 + \cdots + \mathrm{C}_{15}^{8})$.\\
			Suy ra số cách chọn ra ít nhất $8$ bút chì màu là
			\[\mathrm{C}_{15}^8 + \mathrm{C}_{15}^9 + \mathrm{C}_{15}^{10} + \mathrm{C}_{15}^{11} + \mathrm{C}_{15}^{12} + \mathrm{C}_{15}^{13} + \mathrm{C}_{15}^{14} + \mathrm{C}_{15}^{15} = \dfrac{n}{2} = 16384.\]
		\end{listEX}
	}
\end{bt}

