\begin{bt}
	Khai triển các biểu thức sau
	\begin{listEX}[3]
		\item $(2x+y)^6$;
		\item $(x-3y)^6$;
		\item $(x-1)^n$;
		\item $(x+2)^n$;
		\item $(x+y)^{2n}$;
		\item $(x-y)^{2n}$,
	\end{listEX}
	trong đó $n$ là số nguyên dương.
	\loigiai{
	\begin{listEX}
		\item Ta có
		\begin{eqnarray*}
			(2x+y)^6&=& \mathrm{C}_6^0 (2x)^6+\mathrm{C}_6^1 (2x)^5 y+\mathrm{C}_6^2 (2x)^4 y^2 +\mathrm{C}_6^3 (2x)^3 y^3 +\mathrm{C}_6^4 (2x)^2 y^4 +\mathrm{C}_6^5 2x y^5+\mathrm{C}_6^6 y^6\\
			&=&2^6 x^6+\mathrm{C}_6^1 2^5 x^5y + \mathrm{C}_6^2 2^4 x^4y^2+ \mathrm{C}_6^3 2^3 x^3y^3+ \mathrm{C}_6^4 2^2 x^2y^4 +\mathrm{C}_6^5 2xy^5+ \mathrm{C}_6^6 y^6\\
			&=& 64 x^6 + 192 x^5 y + 240 x^4 y^2 + 160 x^3 y^3 + 60 x^2 y^4 + 12 x y^5 + y^6.
		\end{eqnarray*}
		\item Ta có
		\begin{eqnarray*}
			(x-3y)^6&=&\mathrm{C}_6^0 x^6+\mathrm{C}_6^1 x^5 (-3y)+\mathrm{C}_6^2 x^4 (-3y)^2 +\mathrm{C}_6^3 x^3 (-3y)^3\\
			& &+\mathrm{C}_6^4 x^2 (-3y)^4 +\mathrm{C}_6^5 x (-3y)^5+\mathrm{C}_6^6 (-3y)^6\\
			&=& \mathrm{C}_6^0 x^6+\mathrm{C}_6^1 (-3) x^5 y+\mathrm{C}_6^2 x^4 (-3)^2 y^2 +\mathrm{C}_6^3 x^3 (-3)^3 y^3 \\
			& &+\mathrm{C}_6^4 x^2 (-3)^4 y^4 +\mathrm{C}_6^5 x (-3)^5 y^5+\mathrm{C}_6^6 (-3)^6 y^6\\
			&=&x^6 - 18 x^5 y + 135 x^4 y^2 - 540 x^3 y^3 + 1215 x^2 y^4 - 1458 x y^5 + 729 y^6
		\end{eqnarray*}
		\item Ta có
		\begin{eqnarray*}
			(x-1)^n &=& \mathrm{C}_n^0 x^n+ \mathrm{C}_n^1 x^{n-1} (-1)+\mathrm{C}_n^2 x^{n-2} (-1)^2+\cdots +\mathrm{C}_n^{n-1} x (-1)^{n-1}+\mathrm{C}_n^n (-1)^n\\
			&=&x^n- \mathrm{C}_n^1 x^{n-1}+\mathrm{C}_n^2 x^{n-2} -\cdots +(-1)^{n-1} \mathrm{C}_n^{n-1} x + (-1)^n
		\end{eqnarray*}
		\item Ta có 
		\begin{eqnarray*}
			(x+2)^n&=&\mathrm{C}_n^0 x^n+ \mathrm{C}_n^1 x^{n-1} 2+\mathrm{C}_n^2 x^{n-2} 2^2+\cdots +\mathrm{C}_n^{n-1} x 2^{n-1}+\mathrm{C}_n^n 2^n\\
			&=&x^n+2\mathrm{C}_n^1 x^{n-1}+4\mathrm{C}_n^2 x^{n-2} +\cdots +2^{n-1}\mathrm{C}_n^{n-1} x + 2^n.
		\end{eqnarray*}
		\item Ta có
		\begin{eqnarray*}
			(x+y)^{2n}&=& \mathrm{C}_{2n}^0 x^{2n} +\mathrm{C}_{2n}^1 x^{2n-1} y+\mathrm{C}_{2n}^2 x^{2n-2} y^2+\cdots+\mathrm{C}_{2n}^{2n-1} x y^{2n-1}+\mathrm{C}_{2n}^{2n} y^{2n}\\
			&=&  x^{2n} +\mathrm{C}_{2n}^1 x^{2n-1} y+\mathrm{C}_{2n}^2 x^{2n-2} y^2+\cdots+\mathrm{C}_{2n}^{2n-1} x y^{2n-1}+ y^{2n}.
		\end{eqnarray*}
		\item Ta có
		\begin{eqnarray*}
			(x-y)^{2n}&=& \mathrm{C}_{2n}^0 x^{2n} +\mathrm{C}_{2n}^1 x^{2n-1} (-y)+\mathrm{C}_{2n}^2 x^{2n-2} (-y)^2+\cdots+\mathrm{C}_{2n}^{2n-1} x (-y)^{2n-1}+\mathrm{C}_{2n}^{2n} (-y)^{2n}\\
			&=&  x^{2n} -\mathrm{C}_{2n}^1 x^{2n-1} y+\mathrm{C}_{2n}^2 x^{2n-2} y^2-\cdots-\mathrm{C}_{2n}^{2n-1} x y^{2n-1}+ y^{2n}.
		\end{eqnarray*}
	\end{listEX}
}
\end{bt}
\begin{bt}
	Tính
	\begin{listEX}
		\item $S=\mathrm{C}_{2022}^0 9^{2022}+\mathrm{C}_{2022}^1 9^{2021}+\cdots+\mathrm{C}_{2022}^k 9^{2022-k}+\cdots+\mathrm{C}_{2022}^{2021} 9+\mathrm{C}_{2022}^{2022}$.
		\item $T=\mathrm{C}_{2022}^0 4^{2022}-\mathrm{C}_{2022}^1 4^{2021}\cdot 3+\cdots-\mathrm{C}_{2022}^{2021} 4\cdot 3^{2021}+\mathrm{C}_{2022}^{2022} 3^{2022}$.
	\end{listEX}
\loigiai{
\begin{listEX}
	\item Ta có
	\begin{eqnarray*}
		S&=&\mathrm{C}_{2022}^0 9^{2022}+\mathrm{C}_{2022}^1 9^{2021}+\cdots+\mathrm{C}_{2022}^k 9^{2022-k}+\cdots+\mathrm{C}_{2022}^{2021} 9+\mathrm{C}_{2022}^{2022}\\
		&=& \mathrm{C}_{2022}^0 9^{2022}+\mathrm{C}_{2022}^1 9^{2021}\cdot 1+\cdots+\mathrm{C}_{2022}^k 9^{2022-k}\cdot 1^k+\cdots+\mathrm{C}_{2022}^{2021} 9\cdot 1^{2021}+\mathrm{C}_{2022}^{2022}\cdot 1^{2022}\\
		&=& (9+1)^{2020}=10^{2022}.
	\end{eqnarray*}
	\item Ta có 
	\begin{eqnarray*}
		T&=&\mathrm{C}_{2022}^0 4^{2022}-\mathrm{C}_{2022}^1 4^{2021}\cdot 3+\cdots-\mathrm{C}_{2022}^{2021} 4\cdot 3^{2021}+\mathrm{C}_{2022}^{2022} 3^{2022}\\
		&=&\mathrm{C}_{2022}^0 4^{2022}+\mathrm{C}_{2022}^1 4^{2021}\cdot (-3)+\cdots+\mathrm{C}_{2022}^{2021} 4\cdot (-3)^{2021}+\mathrm{C}_{2022}^{2022} (-3)^{2022}\\
		&=& (4-3)^{2022}=1^{2022}=1.
	\end{eqnarray*}
\end{listEX}
}
\end{bt}
\begin{bt}
	Chứng minh
	$$\mathrm{C}_n^0 3^n+\mathrm{C}_n^1 3^{n-1}+\cdots +\mathrm{C}_n^k 3^{n-k}+\cdots +\mathrm{C}_n^{n-1} 3+\mathrm{C}_n^n=\mathrm{C}_n^0+\mathrm{C}_n^1 3+\cdots +\mathrm{C}_n^k 3^k +\cdots +\mathrm{C}_{n}^{n-1} 3^{n-1} +\mathrm{C}_n^n 3^n$$
	với $0\le k\le n$; $k$, $n\in \mathbb{N}$.
	\loigiai{
	\begin{itemize}
		\item Ta có
		\begin{eqnarray*}
			& &\mathrm{C}_n^0 3^n+\mathrm{C}_n^1 3^{n-1}+\cdots +\mathrm{C}_n^k 3^{n-k}+\cdots +\mathrm{C}_n^{n-1} 3+\mathrm{C}_n^n\\
			&=& \mathrm{C}_n^0 3^n+\mathrm{C}_n^1 3^{n-1}\cdot 1+\cdots +\mathrm{C}_n^k 3^{n-k}\cdot 1^k+\cdots +\mathrm{C}_n^{n-1} 3\cdot 1^{n-1}+\mathrm{C}_n^n\cdot 1^n\\
			&=& (3+1)^n=4^n.\qquad (1)
		\end{eqnarray*}
		\item Ta có
		\begin{eqnarray*}
			& & \mathrm{C}_n^0+\mathrm{C}_n^1 3+\cdots +\mathrm{C}_n^k 3^k +\cdots +\mathrm{C}_{n}^{n-1} 3^{n-1} +\mathrm{C}_n^n 3^n\\
			&=& \mathrm{C}_n^0\cdot 1^n+\mathrm{C}_n^1 \cdot 1^{n-1}\cdot 3+\cdots +\mathrm{C}_n^k \cdot 1^{n-k}\cdot 3^k +\cdots +\mathrm{C}_{n}^{n-1} \cdot 1\cdot 3^{n-1} +\mathrm{C}_n^n 3^n\\
			&=& (1+3)^n=4^n.\qquad (2)
		\end{eqnarray*}
	\end{itemize}
	Từ $(1)$, $(2)$ vậy 
	$$\mathrm{C}_n^0 3^n+\mathrm{C}_n^1 3^{n-1}+\cdots +\mathrm{C}_n^k 3^{n-k}+\cdots +\mathrm{C}_n^{n-1} 3+\mathrm{C}_n^n=\mathrm{C}_n^0+\mathrm{C}_n^1 3+\cdots +\mathrm{C}_n^k 3^k +\cdots +\mathrm{C}_{n}^{n-1} 3^{n-1} +\mathrm{C}_n^n 3^n.$$
}
\end{bt}
\begin{bt}
	Xác định hệ số của 
	\begin{listEX}
		\item $x^{12}$ trong khai triển của $(x+4)^{30}$;
		\item $x^{10}$ trong khai triển của $(3+2x)^{30}$;
		\item $x^{15}$ và $x^{16}$ trong khai triển của $\left(\dfrac{2x}{3}-\dfrac{1}{7}\right)^{51}$.
	\end{listEX}
\loigiai{
\begin{listEX}
	\item Số hạng chứa $x^{12}$ trong khai triển là $\mathrm{C}_{30}^{18} x^{12} 4^{18}$.\\
	 Hệ số của $x^{12}$ là $\mathrm{C}_{30}^{18} 4^{18}$.
	\item Số hạng chứa $x^{10}$ trong khai triển là $\mathrm{C}_{30}^{10} 3^{20} (2x)^{10}$ hay $\mathrm{C}_{30}^{10} 3^{20} 2^{10} x^{10}$. \\
	Hệ số của $x^{10}$ là $\mathrm{C}_{30}^{10} 3^{20} 2^{10}$.
	\item Số hạng chứa $x^{15}$ trong khai triển là $\mathrm{C}_{51}^{36} \left(\dfrac{2x}{3}\right)^{15} \left(-\dfrac{1}{7}\right)^{36}$ hay $\mathrm{C}_{51}^{36} \dfrac{2^{15}}{3^{15}\cdot 7^{36}}\cdot x^{15}$.\\
	Hệ số của $x^{15}$ là $\mathrm{C}_{51}^{36} \dfrac{2^{15}}{3^{15}\cdot 7^{36}}$.\\
	Số hạng chứa $x^{16}$ trong khai triển là $\mathrm{C}_{51}^{35} \left(\dfrac{2x}{3}\right)^{16} \left(-\dfrac{1}{7}\right)^{35}$ hay $-\mathrm{C}_{51}^{35} \dfrac{2^{16}}{3^{16}\cdot 7^{35}}\cdot x^{16}$.\\
	Hệ số của $x^{16}$ là $-\mathrm{C}_{51}^{35} \dfrac{2^{16}}{3^{16}\cdot 7^{35}}$.
\end{listEX}
}
\end{bt}
\begin{bt}
	Xét khai triển của $\left(x+\dfrac{5}{2}\right)^{12}$.
	\begin{listEX}
		\item Xác định hệ số của $x^7$.
		\item Nêu hệ số của $x^k$ với $k\in \mathbb{N}$, $k\le 12$.
	\end{listEX}
\loigiai{
\begin{listEX}
	\item Số hạng chứa $x^7$ trong khai triển là $\mathrm{C}_{12}^5 x^7 \left(\dfrac{5}{2}\right)^5$.\\
	 Hệ số của $x^7$ là $\mathrm{C}_{12}^5 \left(\dfrac{5}{2}\right)^5$.
	 \item Số hạng chứa $x^k$ trong khai triển là $\mathrm{C}_{12}^{12-k} x^k \left(\dfrac{5}{2}\right)^{12-k}$.\\
	 Hệ số của $x^k$ là $\mathrm{C}_{12}^{12-k}\left(\dfrac{5}{2}\right)^{12-k}$.
\end{listEX}
}
\end{bt}
\begin{bt}
	Xét khai triển của $\left(\dfrac{x}{2}+\dfrac{1}{5}\right)^{21}$.
	\begin{listEX}
		\item Xác định hệ số của $x^{10}$.
		\item Nêu hệ số của $x^k$ với $k\in \mathbb{N}$, $k\le 21$.
	\end{listEX}
\loigiai{
\begin{listEX}
	\item Số hạng chứa $x^{10}$ trong khai triển là $\mathrm{C}_{21}^{11} \left(\dfrac{x}{2}\right)^{10} \left(\dfrac{1}{5}\right)^{11}$.\\
	Hệ số của $x^{10}$ là $\mathrm{C}_{21}^{11} \left(\dfrac{1}{2}\right)^{10} \left(\dfrac{1}{5}\right)^{11}$ hay $\mathrm{C}_{21}^11 \dfrac{1}{2^{10} \cdot 5^{11}}$.
	\item Số hạng chứa $x^k$ trong khai triển là $\mathrm{C}_{21}^{21-k} \left(\dfrac{x}{2}\right)^k \left(\dfrac{1}{5}\right)^{21-k}$. \\
	Hệ số của $x^k$ là $\mathrm{C}_{21}^{21-k} \dfrac{1}{2^k\cdot 5^{21-k}}$.
\end{listEX}
}
\end{bt}
\begin{bt}
	Tìm hệ số lớn nhất trong khai triển của
	\begin{listEX}[2]
		\item $(a+b)^8$
		\item $(a+b)^9$
	\end{listEX}
\loigiai{
\begin{listEX}
	\item Ta có $\mathrm{C}_8^0<\mathrm{C}_8^1<\mathrm{C}_8^2<\mathrm{C}_8^3<\mathrm{C}_8^4$ và $\mathrm{C}_8^4>\mathrm{C}_8^5>\mathrm{C}_8^6>\mathrm{C}_8^7>\mathrm{C}_8^8$.\\
	Vậy hệ số lớn nhất trong khai triển $(a+b)^8$ là $\mathrm{C}_8^4$.
	\item Ta có $\mathrm{C}_9^0<\mathrm{C}_9^1<\mathrm{C}_9^2<\mathrm{C}_9^3<\mathrm{C}_9^4=\mathrm{C}_9^5$ và $\mathrm{C}_9^5>\mathrm{C}_9^6>\mathrm{C}_9^7>\mathrm{C}_9^8>\mathrm{C}_9^9$.\\
	Vậy hệ số lớn nhất trong khai triển $(a+b)^9$ là $\mathrm{C}_9^4$ và $\mathrm{C}_9^5$.
\end{listEX}
}
\end{bt}
\begin{bt}
	Chứng minh công thức nhị thức Newton bằng phương pháp quy nạp
	$$(a+b)^n=\mathrm{C}_n^0 a^n+\mathrm{C}_n^1 a^{n-1} b+\cdots+\mathrm{C}_n^{n-1} a b^{n-1}+\mathrm{C}_n^n b^n$$
	với mọi $n\in \mathbb{N}^*$, $n\ge 2$.
	\loigiai{
	\begin{itemize}
		\item Với $n=1$, ta có $(a+b)^1=a+b=\mathrm{C}_1^0 a^1 b^0+\mathrm{C}_1^1 a^0 b^1$.\\
		Vậy công thức đúng với $n=1$.
		\item Với $n=k$ là số nguyên dương tùy ý mà công thức đúng, ta phải chứng minh công thức cũng đúng với $n=k+1$, tức là
		$$(a+b)^{k+1}=\mathrm{C}_{k+1}^0a^{k+1}+\mathrm{C}_{k+1}^1 a^{(k+1)-1} b+\cdots +\mathrm{C}_{k+1}^{(k+1)-1} ab^{(k+1)-1}+\mathrm{C}_{k+1}^{k+1} b^{k+1}.$$
		Thật vậy, theo  giả thiết ta có 
		$$(a+b)^k=\mathrm{C}_k^0 a^k+\mathrm{C}_k^1 a^{k-1} b+\cdots+\mathrm{C}_k^{k-1} ab^{k-1}+\mathrm{C}_k^k b^k.$$
		Khi đó
		\begin{eqnarray*}
			(a+b)^{k+1}&=& (a+b)(a+b)^k=a(a+b)^k+b(a+b)^k\\
			&=& a\left(\mathrm{C}_k^0 a^k+\mathrm{C}_k^1 a^{k-1} b+\cdots+\mathrm{C}_k^{k-1} ab^{k-1}+\mathrm{C}_k^k b^k\right)\\
			& &+b\left(\mathrm{C}_k^0 a^k+\mathrm{C}_k^1 a^{k-1} b+\cdots+\mathrm{C}_k^{k-1} ab^{k-1}+\mathrm{C}_k^k b^k\right)\\
			&=&\left(\mathrm{C}_k^0 a^{k+1}+\mathrm{C}_k^1 a^k b+\cdots+\mathrm{C}_k^{k-1} a^2 b^{k-1}+\mathrm{C}_k^k ab^k\right)\\
			& &+\left(\mathrm{C}_k^0 a^kb+\mathrm{C}_k^1 a^{k-1} b^2+\cdots+\mathrm{C}_k^{k-1} ab^k+\mathrm{C}_k^k b^{k+1}\right)\\
			&=&\mathrm{C}_k^0 a^{k+1} +\left(\mathrm{C}_k^1+\mathrm{C}_k^0 \right) a^k b+ \left(\mathrm{C}_k^1+\mathrm{C}_k^2\right) a^{k-1} b^2 \\
			& &+\cdots+ \left(\mathrm{C}_k^{k-2}+\mathrm{C}_k^{k-1}\right) a^2 b^{k-1} + \left(\mathrm{C}_k^{k-1}+\mathrm{C}_k^k\right) ab^k+\mathrm{C}_k^k b^{k+1}\\
			&=& a^{k+1}+\mathrm{C}_{k+1}^1 a^k b+\mathrm{C}_{k+1}^2 a^{k-1} b^2+\cdots+ \mathrm{C}_{k+1}^{k-1} a^2 b^{k-1}+\mathrm{C}_{k+1}^k ab^k+b^{k+1}\\
			&=& \mathrm{C}_{k+1}^0a^{k+1}+\mathrm{C}_{k+1}^1 a^{(k+1)-1} b+\cdots +\mathrm{C}_{k+1}^{(k+1)-1} ab^{(k+1)-1}+\mathrm{C}_{k+1}^{k+1} b^{k+1}.
		\end{eqnarray*}
		Vậy công thức đúng với $n=k+1$.
	\end{itemize}
Do đó, theo nguyên lí quy nạp toán học, đẳng thức đúng với mọi $n\in \mathbb{N^*}$. 
}
\end{bt}
\begin{bt}
	Bằng phương pháp quy nạp, chứng minh
	\begin{listEX}
		\item $n^5-n$ chia hết cho $5$ với mọi $n\in \mathbb{N}^*$;
		\item $n^7-n$ chia hết cho $7$ với mọi $n\in \mathbb{N^*}$.
	\end{listEX}
\loigiai{
\begin{listEX}
	\item 
	\begin{itemize}
		\item Với $n=1$, ta có $1^5-1$ chia hết cho $5$ là mệnh đề đúng.
		\item Với $n=k$ là một số nguyên dương tùy ý mà mệnh đề đúng, ta phải chứng minh mệnh đề cũng đúng với $n=k+1$, tức là 
		$$(k+1)^5-(k+1)\text{ chia hết cho }5.$$
		Thật vậy, theo giả thiết quy nạp ta có $k^5-k$ chia hết cho $5$.\\
		Khi đó
		\begin{eqnarray*}
			(k+1)^5-(k+1)&=& (k^5+5k^4+10k^3+10k^2+5k+1)-(k+1)\\
			&=& (k^5-k) + (5k^4+10k^3+10k^2+5k).
		\end{eqnarray*}
		Mà $(k^5-k)$ và $(5k^4+10k^3+10k^2+5k)$ đều chia hết cho $5$, do đó
		$$(k^5-k) + (5k^4+10k^3+10k^2+5k)\text{ chia hết cho }5.$$
		Vậy mệnh đề đúng với $n=k+1$.
	\end{itemize}
	Do đó, theo nguyên lí quy nạp toán học, đẳng thức đúng với mọi $n\in \mathbb{N^*}$.
	\item 
	\begin{itemize}
		\item Với $n=1$, ta có $1^7-1$ chia hết cho $7$ là mệnh đề đúng.
		\item Với $n=k$ là một số nguyên dương tùy ý mà mệnh đề đúng, ta phải chứng minh mệnh đề cũng đúng với $n=k+1$, tức là 
		$$(k+1)^7-(k+1)^7\text{ chia hết cho }7.$$
		Khi đó
		\begin{eqnarray*}
			(k+1)^7-(k+1)^7&=& (k^7+7k^6+21k^5+35k^4+35k^3+21k^2+7k+1)-(k+1)\\
			&=& (k^7-k) + (7k^6+21k^5+35k^4+35k^3+21k^2+7k).
		\end{eqnarray*}
		Mà $(k^7-k)$ và $(7k^6+21k^5+35k^4+35k^3+21k^2+7k)$ đều chia hết cho $7$, do đó
		$$(k^7-k) + (7k^6+21k^5+35k^4+35k^3+21k^2+7k)\text{ chia hết cho }7.$$
		Vậy mệnh đề đúng với $n=k+1$.
	\end{itemize}
	Do đó, theo nguyên lí quy nạp toán học, đẳng thức đúng với mọi $n\in \mathbb{N^*}$.
\end{listEX}
}
\end{bt}
\begin{bt}
	Cho tập hợp $A=\{x_1;x_2;x_3;\ldots;x_n\}$ có $n$ phần tử. Tính số tập hợp con của $A$.
	\loigiai{
	Vì $A$ có $n$ phần tử nên số tập hợp con có $k$ phần tử $(k\le n)$ của tập hợp $A$ là $\mathrm{C}_n^k$.\\
	Vậy tổng số tập con của tập hợp $A$ là
	$$\mathrm{C}_n^0+\mathrm{C}_n^1+\mathrm{C}_n^2+\cdots+\mathrm{C}_n^{n-1}+\mathrm{C}_n^n.$$
	Lại có $\mathrm{C}_n^0+\mathrm{C}_n^1+\mathrm{C}_n^2+\cdots+\mathrm{C}_n^{n-1}+\mathrm{C}_n^n=2^n$.\\
	Vậy tập hợp $A$ có tất cả $2^n$ tập con.
}
\end{bt}
\begin{bt}
	Một nhóm gồm $10$ học sinh tham gia chiến dịch Mùa hè xanh. Nhà trường muốn chọn ra một đội công tác có ít nhất hai học sinh trong những học sinh trên. Hỏi có bao nhiêu cách lập đội công tác như thế?
	\loigiai{
	Đội cộng tác có thể có từ $2$ đến $10$ học sinh.\\
	Nếu đội cộng tác có $k$ học sinh thì ta có $\mathrm{C}_{10}^k$ cách chọn.\\
	Như vậy tổng số cách chọn là $\mathrm{C}_{10}^2+\mathrm{C}_{10}^3+\cdots +\mathrm{C}_{10}^{10}$.\\
	Mà
	\begin{eqnarray*}
		& &\mathrm{C}_{10}^0+\mathrm{C}_{10}^1+\mathrm{C}_{10}^2+\mathrm{C}_{10}^3+\cdots +\mathrm{C}_{10}^{10}=2^{10}=1024\\
		&\Rightarrow& \mathrm{C}_{10}^2+\mathrm{C}_{10}^3+\cdots +\mathrm{C}_{10}^{10}=1024-\mathrm{C}_{10}^0-\mathrm{C}_{10}^1=1013.
	\end{eqnarray*}
	Vậy có $1013$ cách lập đội.
}
\end{bt}
\begin{bt}
	Để tham gia một cuộc thi làm bánh, bạn Tiếm làm $12$ chiếc bánh có màu khác nhau và chọn ra số nguyên dương chẵn chiếc bánh để cho vào hộp trưng bày. Hỏi bạn Tiến có bao nhiêu cách để chọn bánh cho vào hộp trưng bày đó?
	\loigiai{
	Số bánh bạn Tiến có thể chọn để cho vào hộp có thể là $2$, $4$, $6$, $8$, $10$ hoặc $12$.\\
	Vậy tổng số cách chọn là $\mathrm{C}_{12}^2+\mathrm{C}_{12}^4+\cdots+\mathrm{C}_{12}^{12}$.\\
	Mà 
	\begin{eqnarray*}
		& &\mathrm{C}_{12}^0+\mathrm{C}_{12}^2+\mathrm{C}_{12}^4+\cdots+\mathrm{C}_{12}^{12}=2^{2\cdot 6-1}\\
		&\Rightarrow &\mathrm{C}_{12}^2+\mathrm{C}_{12}^4+\cdots+\mathrm{C}_{12}^{12}=2048-1=2047.
	\end{eqnarray*}
	Vậy có $2047$ cách chọn.
}
\end{bt}
\begin{bt}
	Bác Thành muốn mua quà cho con nhân dịp sinh nhật nên đã đến một cửa hàng đồ chơi. Bác dự định chọn một trong năm loại đồ chơi. Ở cửa hàng, mỗi loại đồ chơi đó chỉ có $10$ sản phẩm khác nhau bày bán. Biết rằng nếu mua bộ trực thăng điểu kiển từ xa, bác sẽ chỉ mua $1$ sản phẩm; nếu mua bộ đồ chơi lego, bác sẽ mua $3$ sản phẩm khác nhau; nếu mua bộ lắp ghép robot chạy bằng năng lượng mặt trời, bác sẽ mua $5$ sản phẩm khác nhau; nếu mua rubik, bác sẽ mua $7$ sản phẩm khác nhau; còn nếu mua mô hình khủng long, bác sẽ mua $9$ sản phẩm khác nhau. Bác Thành có bao nhiêu cách chọn quà sinh nhật cho con?
	\loigiai{
	Số cách chọn nếu bác Thành mua
	\begin{itemize}
		\item Bộ trực thăng điều kiển từ xa là $\mathrm{C}_{10}^1$.
		\item Bộ đồ chơi lego là $\mathrm{C}_{10}^3$.
		\item Bộ lắp ghép robot chạy bằng năng lượng mặt trời là $\mathrm{C}_{10}^5$.
		\item Rubik là $\mathrm{C}_{10}^7$.
		\item Mô hình khủng long là $\mathrm{C}_{10}^9$.
	\end{itemize}
	Tổng số cách chọn là $\mathrm{C}_{10}^1+\mathrm{C}_{10}^3+\mathrm{C}_{10}^5+\mathrm{C}_{10}^7+\mathrm{C}_{10}^9=2^{2\cdot 5-1}=2^9=512$.\\
	Vậy có $512$ cách chọn.
}
\end{bt}
\begin{bt}
	Giả sử tính trạng ở một loài cây được quy định do tác động cộng gộp của $n$ cặp alen phân li độc lập $A_1a_1$, $A_2a_2$, $\ldots$, $A_na_n$. Cho cây $F_1$ dị hợp về $n$ cặp alen giao phối với nhau. Tỉ lệ phân li kiểu hình của $F_2$ là hệ số của khai triển nhị thức Newton $(a+b)^{2n}$, nghĩa là tỉ lệ phân li kiểu hình của $F_2$ là $\mathrm{C}_{2n}^0\colon \mathrm{C}_{2n}^1\colon \mathrm{C}_{2n}^2\colon \ldots\colon \mathrm{C}_{2n}^{2n-2}\colon \mathrm{C}_{2n}^{2n-1}\colon \mathrm{C}_{2n}^{2n}$.\\
	Cho biết một loại cây có tính trạng được quy định bởi tác động cộng gộp của $4$ cặp alen phân li độc lập. Tìm tỉ lệ phân li kiểu hình của $F_2$ nếu cây $F_1$ dị hợp về $4$ cặp alen giao phối với nhau.
	\loigiai{
		Với $n=2$, tỉ lệ phân li kiểu hình của $F_2$ nếu cây $F_1$ dị hợp về $4$ cặp alen giao phối với nhau là
		$$\mathrm{C}_{2\cdot 4}^0\colon \mathrm{C}_{2\cdot 4}^1\colon \mathrm{C}_{2\cdot 4}^2 \colon \cdots \colon \mathrm{C}_{2\cdot 4}^{2\cdot 4}.$$
		Hay 
		$$\mathrm{C}_8^0\colon \mathrm{C}_8^1\colon \mathrm{C}_8^2\colon \mathrm{C}_8^3\colon \mathrm{C}_8^4\colon \mathrm{C}_8^5\colon \mathrm{C}_8^6\colon \mathrm{C}_8^7\colon \mathrm{C}_8^8.$$
	}
\end{bt}
\begin{bt}
	Sử dụng tam giác Pascal, viết khai triển
	\begin{listEX}[2]
		\item $(x-1)^5$;
		\item $(2x-3y)^4$.
	\end{listEX}
\loigiai{
\begin{listEX}
	\item Dựa vào hàng $5$ của tam giác Pascal, ta có
	$$(a+b)^5=a^5+5a^4b+10a^3b^2+10a^2b^3+5ab^4+b^5.$$
	Với $a=x$, $b=-1$, thay vào ta được
	\begin{eqnarray*}
		(x-1)^5&=&x^5+5x^4\cdot (-1)+10 x^3 (-1)^2+10 x^2 (-1)^3 +5x(-1)^4+(-1)^5\\
		&=&x^5 - 5 x^4 + 10 x^3 - 10 x^2 + 5 x - 1.
	\end{eqnarray*}
	\item Dựa vào hàng $4$ của tam giác Pascal, ta có
	$$(a+b)^4=a^4+4a^3b+6a^2b^2+4ab^3+b^4.$$
	Với $a=2x$, $b=-3y$, thay vào ta được
	\begin{eqnarray*}
		(2x-3y)^4&=& (2x)^4+4(2x)^3(-3y)+6(2x)^2(-3y)^2+4(2x)(-3y)^3+(-3y)^4\\
		&=&16 x^4 - 96 x^3 y + 216 x^2 y^2 - 216 x y^3 + 81 y^4.		
	\end{eqnarray*}
\end{listEX}
}
\end{bt}
\begin{bt}
	Viết khai triển theo nhị thức Newton
	\begin{listEX}[2]
		\item $(x+y)^6$;
		\item $(1-2x)^5$.
	\end{listEX}
\loigiai{
	\begin{listEX}
		\item Theo công thức nhị thức Newton, ta có
		\begin{eqnarray*}
			(x+y)^6&=& \mathrm{C}_6^0 x^6+\mathrm{C}_6^1 x^5y+\mathrm{C}_6^2 x^4y^2+\mathrm{C}_6^3 x^3 y^3+\mathrm{C}_6^4 x^2 y^4+\mathrm{C}_6^5 x y^5+\mathrm{C}_6^6 y^6\\
			&=& x^6+\mathrm{C}_6^1 x^5y+\mathrm{C}_6^2 x^4y^2+\mathrm{C}_6^3 x^3 y^3+\mathrm{C}_6^4 x^2 y^4+\mathrm{C}_6^5 x y^5+ y^6.
		\end{eqnarray*}
		\item Theo công thức nhị thức Newton, ta có
		\begin{eqnarray*}
			(1-2x)^5&=& \mathrm{C}_5^0 1^5 +\mathrm{C}_5^1 (-2x)^1+\mathrm{C}_5^2 (-2x)^2+\mathrm{C}_5^3 (-2x)^3+\mathrm{C}_5^4 (-2x)^4+ \mathrm{C}_5^5 (-2x)^5\\
			&=& -32 x^5 + 80 x^4 - 80 x^3 + 40 x^2 - 10 x + 1.
		\end{eqnarray*}
	\end{listEX}
}
\end{bt}