%%\section{PHƯƠNG PHÁP QUY NẠP TOÁN HỌC}
\subsection{Bài tập tự luận}
\begin{bt}%[0D2K1-1]
Cho $ S_n=1+2+2^2+ \ldots + 2^n$ và $ T_n=2^{n+1}-1 $, với $ n \in \mathbb{N}^* $.
\begin{enumerate}
\item So sánh $ S_1 $ và $ T_1 $; $ S_2 $ và $ T_2 $; $ S_3 $ và $ T_3 $.
\item Dự đoán công thức tính $ S_n $ và chứng minh bằng phương pháp quy nạp toán học.
\end{enumerate}
\loigiai{
\begin{enumerate}
\item Ta có $ S_1=3 $, $ T_1=3 $ suy ra $ S_1 = T_1 $.\\
$ S_2= 1+2+2^2=7$, $ T_2=2^3-1 =7$ suy ra $ S_2=T_2 $.\\
$ S_3=1+2+2^2+2^3=15 $, $ T_3=2^4-1=15 $ suy ra $ S_3=T_3 $.
\item Dự đoán $ S_n= 1+2+2^2+ \ldots + 2^n = 2^{n+1}-1 $, với $ n \in \mathbb{N}^*$.\\
Chứng minh bằng quy nạp toán học.\\
Cho $ n=1 $ ta được $ S_1=3 $ (đúng).\\
Giả sử đẳng thức đúng với $ n=k>1 $, $ n\in  \mathbb{N}^*$ tức là $$ S_k=1+2+2^2+\ldots +2^k =2^{k+1}-1.$$ 
Ta chứng minh đẳng thức đúng với $ n=k+1 $, $ n\in  \mathbb{N}^*$ tức là $$ S_{k+1} =1 + 2 + 2^2 + \ldots + 2^{k+1}=2^{k+2}-1.$$
Thật vậy $ \begin{aligned}[t]
S_{k+1}&=\left( 1 + 2 + 2^2 + \ldots +  2^k\right)  + 2^{k+1}\\
 & = 2^{k+1}-1 + 2^{k+1}\\
 & =2\cdot 2^{k+1}-1\\
 &=2^{k+2}-1.
\end{aligned} $\\
Vậy $ S_n= 1+2+2^2+ \ldots + 2^n = 2^{n+1}-1 $, với $ n \in \mathbb{N}^*$.
\end{enumerate}
}
\end{bt}
%%=======
\begin{bt}%[0D2K1-1]
Cho $ S_n=1+\dfrac{1}{2}+\dfrac{1}{2^2}+ \ldots + \dfrac{1}{2^n}$ và $ T_n = 2 -\dfrac{1}{2^n} $, với $ n \in \mathbb{N}^* $.
\begin{enumerate}
\item So sánh $ S_1 $ và $ T_1 $; $ S_2 $ và $ T_2 $; $ S_3 $ và $ T_3 $.
\item Dự đoán công thức tính $ S_n $ và chứng minh bằng phương pháp quy nạp toán học.
\end{enumerate}
\loigiai{
\begin{enumerate}
	\item Ta có $ S_1=1+\dfrac{1}{2}=\dfrac{3}{2} $, $ T_1=2-\dfrac{1}{2}=\dfrac{3}{2} $ suy ra $ S_1 = T_1 $.\\
	$ S_2= 1+\dfrac{1}{2} + \dfrac{1}{2^2}=\dfrac{7}{4}$, $ T_2=2-\dfrac{1}{2^2}=\dfrac{7}{4}$ suy ra $ S_2=T_2 $.\\
	$ S_3=1+\dfrac{1}{2}+\dfrac{1}{2^2}+\dfrac{1}{2^3}=\dfrac{15}{8} $, $ T_3=2-\dfrac{1}{2^3}=\dfrac{15}{8} $ suy ra $ S_3=T_3 $.
	\item Dự đoán $ S_n=1+\dfrac{1}{2}+\dfrac{1}{2^2}+ \ldots + \dfrac{1}{2^n} = 2 -\dfrac{1}{2^n}  $, với $ n \in \mathbb{N}^*$.\\
	Chứng minh bằng quy nạp toán học.\\
	Cho $ n=1 $ ta được $ S_1=\dfrac{3}{2} $ (đúng).\\
	Giả sử đẳng thức đúng với $ n=k>1 $, $ n\in  \mathbb{N}^*$ tức là $$ S_k=1+\dfrac{1}{2}+\dfrac{1}{2^2}+ \ldots + \dfrac{1}{2^k} = 2 -\dfrac{1}{2^k}.$$ 
	Ta chứng minh đẳng thức đúng với $ n=k+1 $, $ n\in  \mathbb{N}^*$ tức là $$ S_{k+1} =1+\dfrac{1}{2}+\dfrac{1}{2^2}+ \ldots + \dfrac{1}{2^{k+1}} = 2 -\dfrac{1}{2^{k+1}}.$$
	Thật vậy $ \begin{aligned}[t]
	S_{k+1}&=\left(1+\dfrac{1}{2}+\dfrac{1}{2^2}+ \ldots + \dfrac{1}{2^{k}}\right)  + \dfrac{1}{2^{k+1}}\\
	& =  2 -\dfrac{1}{2^k} + \dfrac{1}{2^{k+1}}\\
	& =2 - \dfrac{1}{2^{k+1}}.
	\end{aligned} $\\
	Vậy $ S_n=1+\dfrac{1}{2}+\dfrac{1}{2^2}+ \ldots + \dfrac{1}{2^n} = 2 -\dfrac{1}{2^n}  $, với $ n \in \mathbb{N}^*$.
\end{enumerate}
}
\end{bt}
%%============
\begin{bt}%[0D2G1-1]
	Cho $ S_n=\dfrac{1}{1\cdot 5} + \dfrac{1}{5 \cdot 9} + \dfrac{1}{9 \cdot 13} + \ldots + \dfrac{1}{(4n-3)(4n+1)}$, với $ n \in \mathbb{N}^* $.
	\begin{enumerate}
		\item Tính $ S_1 $, $ S_2 $, $ S_3 $, $ S_4 $.
		\item Dự đoán công thức tính $ S_n $ và chứng minh bằng phương pháp quy nạp toán học.
	\end{enumerate}
\loigiai{
	\begin{enumerate}
		\item Ta có $ S_1=\dfrac{1}{1\cdot 5}=\dfrac{1}{5} $, $ S_2= \dfrac{1}{1\cdot 5} + \dfrac{1}{5 \cdot 9}=\dfrac{2}{9}$, $ S_3= \dfrac{1}{1\cdot 5} + \dfrac{1}{5 \cdot 9}+\dfrac{1}{9\cdot 13}=\dfrac{3}{13}$, $ S_4= \dfrac{1}{1\cdot 5} + \dfrac{1}{5 \cdot 9}+\dfrac{1}{9\cdot 13}+\dfrac{1}{13\cdot 17}=\dfrac{4}{17}$.
		\item Ta có $ \dfrac{1}{(4n-3)(4n+1)} =\dfrac{1}{4}\cdot \left( \dfrac{1}{4n-3}-\dfrac{1}{4n+1}\right) $ nên $$ S_n=\dfrac{1}{4} \cdot \left(1-\dfrac{1}{5}+\dfrac{1}{5}-\dfrac{1}{9}+\ldots +\dfrac{1}{4n-3}-\dfrac{1}{4n+1} \right) =\dfrac{1}{4}\cdot \left(1-\dfrac{1}{4n+1} \right)  =\dfrac{n}{4n+1}.$$
		Dự đoán $ S_n=\dfrac{1}{1\cdot 5} + \dfrac{1}{5 \cdot 9} + \dfrac{1}{9 \cdot 13} + \ldots + \dfrac{1}{(4n-3)(4n+1)}=\dfrac{n}{4n+1} $, với $ n \in \mathbb{N}^*$.\\
		Cho $ n=1 $ ta được $ S_1=\dfrac{1}{5} $ (đúng).\\
		Giả sử đẳng thức đúng với $ n=k \in \mathbb{N}^* , n>1$, tức là $$ S_k =\dfrac{1}{1\cdot 5} + \dfrac{1}{5 \cdot 9} + \dfrac{1}{9 \cdot 13} + \ldots + \dfrac{1}{(4k-3) \cdot (4k+1)}=\dfrac{k}{4k+1}.$$
		Ta chứng minh đẳng thức đúng với $ n=k+1,  n \in \mathbb{N}^* $, tức là $$ S_{k+1}=\dfrac{1}{1\cdot 5} + \dfrac{1}{5 \cdot 9} + \dfrac{1}{9 \cdot 13} + \ldots + \dfrac{1}{(4k-3) \cdot (4k+1)} +\dfrac{1}{(4k+1) \cdot (4k+5)} =\dfrac{k+1}{4k+5}.$$
		Thật vậy, $\begin{aligned}[t]
		S_{k+1}&=\left( \dfrac{1}{1\cdot 5} + \dfrac{1}{5 \cdot 9} + \dfrac{1}{9 \cdot 13} + \ldots + \dfrac{1}{(4k-3) \cdot (4k+1)}\right)  +\dfrac{1}{(4k+1) \cdot (4k+5)}\\
		& = \dfrac{k}{4k+1}+\dfrac{1}{(4k+1) \cdot (4k+5)}\\
		& =\dfrac{k+1}{4k+5}.
		\end{aligned}$
	Vậy $ S_n=\dfrac{1}{1\cdot 5} + \dfrac{1}{5 \cdot 9} + \dfrac{1}{9 \cdot 13} + \ldots + \dfrac{1}{(4n-3)(4n+1)}=\dfrac{n}{4n+1} $, với $ n \in \mathbb{N}^*$.
	\end{enumerate}
}
\end{bt}
%%========
\begin{bt}%[0D2K1-1]
Cho $ q $ là số thực khác $ 1 $. Chứng minh $ 1+q+q^2+ \ldots + q^{n-1}=\dfrac{1-q^n}{1-q} $, với $ n \in \mathbb{N}^* $.
\loigiai{
Cho $ n=1 $ ta được $ q^0=\dfrac{1-q}{1-q} $ (đúng).\\
Giả sử đẳng thức đúng với $ n=k>1, n\in \mathbb{N}^* $, tức là $$1+q+q^2+ \ldots + \dfrac{1-q^k}{1-q}  , \text{với} \; n \in \mathbb{N}^*, n>1. $$
Ta chứng minh đẳng thức đúng với $ n=k+1 $, $ n\in \mathbb{N}^* $ tức là $$ 1+q+q^2+\ldots +q^{k-1}+q^k=\dfrac{1-q^{k+1}}{1-q}, \text{với}\; k \in \mathbb{N}^*.$$
Thật vậy, $\begin{aligned}[t]
\text{VT} &=\left(1+q+q^2+\ldots +q^{k-1}\right)+q^k\\
&=\dfrac{1-q^k}{1-q}+q^k\\
&=\dfrac{1-q^{k+1}}{1-q} =\text{VP}, \text{với}\; k \in \mathbb{N}^*  
\end{aligned}$\\
Vậy $ 1+q+q^2+ \ldots + q^{n-1}=\dfrac{1-q^n}{1-q} $, với $ n \in \mathbb{N}^* $.
}
\end{bt}
%=======
\begin{bt}%[0D2K1-1]
Chứng minh với mọi $ n \in \mathbb{N}^* $, ta có 
\begin{enumerate}
\item  $ 4^n+15n-1 $ chia hết cho $ 9 $;
\item $ 13^n-1 $ chia hết cho $ 6 $.
\end{enumerate}
\loigiai{
\begin{enumerate}
\item Cho $ n=1 $ ta được $ 4^1+15\cdot 1-1=18 $ chia hết cho $ 9 $ (đúng).\\
Giả sử mệnh đề đúng với $ n=k \in \mathbb{N}^* $, tức là $ 4^k+15k-1 $ chia hết cho $ 9 $.\\
Ta chứng minh mệnh đề đúng với $ n=k+1 $, $ n \in \mathbb{N}^* $ tức là $ 4^{k+1}+15(k+1)-1 $ chia hết cho $ 9 $.\\
Thật vậy $ 4^{k+1}+15(k+1)-1=4\cdot 4^k+15k +14 = 4\left( 4^k+15k-1\right)-45k+18  $ chia hết cho $ 9 $.\\
Vậy $ 4^n+15n-1 $ chia hết cho $ 9 $.
\item Cho $ n=1 $ ta được $ 13^1-1=12 $ chia hết cho $ 6 $.\\
Giả sử mệnh đề đúng với $ n=k \in \mathbb{N}^* $, tức là $ 13^k-1 $ chia hết cho $ 6 $.\\
Ta chứng minh mệnh đề đúng với $ n=k+1 $, $ n \in \mathbb{N}^* $, tức là $ 13^{k+1}-1 $ chia hết cho $ 6 $.\\
Thật vậy, $ 13^{k+1}-1=13\cdot 13^k -1 = 13\left(13^k-1 \right) +12 $ chia hết cho $ 6 $.\\
Vậy $ 13^n-1 $ chia hết cho $ 6 $.
\end{enumerate}
}
\end{bt}
%%==========
\begin{bt}%[0D2K1-1]
Chứng minh $ n^n>\left(n+1 \right)^{n-1}  $ với $ n \in \mathbb{N}^* $, $ n \ge 2 $.
\loigiai{
Cho $ n=2 $ ta được $ 2^2> (2+1)^{2-1}$ (đúng).\\
Giả sử bất đẳng thức đúng với $ n=k>2 $, $ n \in \mathbb{N}^* $ tức là $ k^k>(k+1)^{k-1} \Rightarrow \dfrac{k^k}{(k+1)^{k-1} }>1$, với $ k \in \mathbb{N}^* $, $ k>2 $.\\
Ta chứng minh bất đẳng thức đúng với $ n=k+1 $, tức là $ \left( k+1\right)^{k+1} > (k+2)^k $, với $ k \in \mathbb{N}^* $, $ k>2 $.\\
Thật vậy $\begin{aligned}[t]
&(k+1)^2>k(k+2)\\
 \Rightarrow &(k+1)^{2k} >\left[ k(k+2)\right]^k\\
  \Rightarrow &\dfrac{(k+1)^{2k}}{(k+2)^k}>k^k\\
   \Rightarrow &\dfrac{(k+1)^{k+1}}{(k+2)^k}>\dfrac{k^k}{(k+1)^{k-1}}>1\\
    \Rightarrow &(k+1)^{k+1}>(k+2)^k.
\end{aligned}
 $\\
Vậy $ n^n>\left(n+1 \right)^{n-1}  $ với $ n \in \mathbb{N}^* $, $ n \ge 2 $.
}
\end{bt}
\begin{bt}%[0D2G1-1]
Chứng minh $a^n-b^n=(a-b)\left( a^{n-1}+a^{n-2}b+ \ldots +ab^{n-2}+b^{n-1}\right)$ với $ n\in \mathbb{N}^* $.
\loigiai{
Khi $ n=1 $ ta có $ a^1-b^1=a-b $ (luôn đúng).\\
Giả sử đẳng thức đúng với $ n=k $, $ n \in \mathbb{N}^* $, tức là $a^k-b^k=(a-b)\left( a^{k-1}+a^{k-2}b+ \ldots +ab^{k-2}+b^{k-1}\right)$ với $ k \in \mathbb{N}^* $.\\
Ta chứng minh đẳng thức đúng với $ n=k+1 $, $ n \in \mathbb{N}^* $ tức là $$ a^{k+1} - b^{k+1}=(a-b)\left[ a^{(k+1)-1}+a^{(k+1)-2}b+ \ldots +ab^{(k+1)-2}+b^{(k+1)-1} \right]. $$
Thật vậy $ \begin{aligned}[t]
a^{k+1}-b^{k+1}&=a \cdot a^k - b \cdot b^k\\
&= a \cdot a^k - a \cdot b^k + a \cdot b^k - b \cdot b^k\\
&= a\left( a^k - b^k\right) + b^k (a-b)\\
&= a\left[(a-b)\left( a^{k-1}+a^{k-2}b+ \ldots +ab^{k-2}+b^{k-1}\right) \right]  + b^k(a-b)\\
&=(a-b)\left[ a^{(k+1)-1}+a^{(k+1)-2}b+ \ldots +ab^{(k+1)-2}+b^{(k+1)-1} \right] .
\end{aligned} $\\
Vậy $a^n-b^n=(a-b)\left( a^{n-1}+a^{n-2}b+ \ldots +ab^{n-2}+b^{n-1}\right)$ với $ n\in \mathbb{N}^* $.
}
\end{bt}
\begin{bt}%[0D2G1-1]
Cho tam đều màu xanh (Hình thứ nhất).
\begin{enumerate}
\item Nêu quy luật chọn tam giác đều màu trắng ở Hình thứ hai.
\item Nêu quy luật chọn tam giác đều màu trắng ở Hình thứ ba.\\
\begin{tikzpicture}[>=stealth,line join=round,line cap=round,font=\footnotesize,scale=1,line width=0.8pt]
\coordinate (A) at (0,0);
\coordinate (B) at (-60:3);
\coordinate (C) at (-120:3);
%%====
\coordinate (A1) at ($ (A)+(5,0) $);
\coordinate (B1) at ($ (B)+(5,0) $);
\coordinate (C1) at ($ (C)+(5,0) $);
\coordinate(M1) at($ (A1)!0.5!(B1) $);
\coordinate(N1) at($ (B1)!0.5!(C1) $);
\coordinate(P1) at($ (C1)!0.5!(A1) $);
%%%============
\coordinate (A2) at ($ (A1)+(5,0) $);
\coordinate (B2) at ($ (B1)+(5,0) $);
\coordinate (C2) at ($ (C1)+(5,0) $);
\coordinate(M2) at($ (A2)!0.5!(B2) $);
\coordinate(N2) at($ (B2)!0.5!(C2) $);
\coordinate(P2) at($ (C2)!0.5!(A2) $);
\coordinate(M3) at($ (A2)!0.5!(M2) $);
\coordinate(M4) at($ (M2)!0.5!(P2) $);
\coordinate(M5) at($ (A2)!0.5!(P2) $);
\coordinate(N3) at($ (M2)!0.5!(B2) $);
\coordinate(N4) at($ (B2)!0.5!(N2) $);
\coordinate(N5) at($ (M2)!0.5!(N2) $);
\coordinate(P3) at($ (P2)!0.5!(N2) $);
\coordinate(P4) at($ (N2)!0.5!(C2) $);
\coordinate(P5) at($ (C2)!0.5!(P2) $);
%%===================
\draw [fill=green] (A)--(B)--(C)--cycle;
\node [below] at ($ (B)!0.5!(C) $) {\text{Hình thứ nhất}};
\draw [fill=green] (A1)--(B1)--(C1)--cycle;
\node [below] at ($ (B1)!0.5!(C1) $) {\text{Hình thứ hai}};
\draw [fill=white] (M1)--(N1)--(P1)--cycle;
\draw [fill=green] (A2)--(B2)--(C2)--cycle;
\draw [fill=white] (M2)--(N2)--(P2)--cycle;
\draw [fill=white] (M3)--(M4)--(M5)--cycle;
\draw [fill=white] (N3)--(N4)--(N5)--cycle;
\draw [fill=white] (P3)--(P4)--(P5)--cycle;
\node [below] at ($ (B2)!0.5!(C2) $) {\text{Hình thứ ba}};
\end{tikzpicture}
\item Nêu quy luật tiếp tục chọn các tam giác đều màu trắng từ Hình thứ tư và các tam giác đều màu trắng ở những hình sau đó.
\item Tính số tam giác đều màu xanh lần lượt trong các Hình thứ nhất, Hình thứ hai, Hình thứ ba.
\item Dự đoán số tam giác đều màu xanh trong hình thứ $ n $. Chứng minh kết quả đó bằng quy nạp toán học.
\end{enumerate}
\loigiai{
\begin{enumerate}
\item Tam giác màu trắng được tạo thành từ các đường trung bình của tam giác màu xanh.
\item Cứ mỗi tam giác màu xanh ở hình thứ hai ta được một tam giác màu trắng ở hình thứ ba.
\item Quy luật chọn các tam giác màu trắng ở hình thứ ba, hình thứ tư và các hình tiếp theo là $ 1; 4; 4+3\cdot 3=13; 13 + 12\cdot 3 = 49; 49+48\cdot 3 \ldots $ 
\item Số tam giác màu xanh ở hình thứ nhất là $ 1 $, số tam giác màu xanh ở hình thứ hai là $ 3 $, số tam giác màu xanh ở hình thứ ba là $ 9 $.
\item Dự đoán số tam giác đều màu xanh trong hình thứ $ n $ là $T_n = 3^n $.
\end{enumerate}
}
\end{bt}
%=======
\begin{bt}%[0D2G1-1]
	Quan sát Hình 6.
		\begin{enumerate}
			\item Nêu quy luật sắp xếp các chấm trắng và đen xen kẽ nhau khi xếp các chấm đó từ góc trên bên trái xuống góc dưới bên phải (tạo thành hình vuông).
			\item Giả sử hình vuông thứ $ n $ có mỗi cạnh chứa $ n $ chấm. Tính tổng số chấm được xếp trong hình vuông (kể cả trên cạnh). Chứng minh kết quả đó bằng phương pháp quy nạp toán học. 
		\end{enumerate}
	\begin{center}
		\begin{tikzpicture}[>=stealth,line join=round,line cap=round,font=\footnotesize,scale=0.8,line width=0.8pt]
		\coordinate (A1) at (0,0);
		\draw (A1) circle[radius=0.2cm];
		\foreach \so in {2,3,4,5,6,7}
		{
		\coordinate (A\so) at ($ (A1)+(0,\so -1) $);
		\draw (A\so) circle[radius=0.2cm]; % Hoặc dùng lệnh \draw (0,0) circle (1cm);
		\coordinate (A\so) at ($ (A1)-(\so - 1,0) $);
		\draw (A\so) circle[radius=0.2cm]; 
		}
	%%=========
\coordinate (B1) at (135:1.414);
		\draw [fill](B1) circle[radius=0.2cm];
	\foreach \so in {2,3,4,5,6}
	{
		\coordinate (B\so) at ($ (B1)+(0,\so -1) $);
		\draw [fill](B\so) circle[radius=0.2cm]; % Hoặc dùng lệnh \draw (0,0) circle (1cm);
		\coordinate (B\so) at ($ (B1)-(\so - 1,0) $);
		\draw [fill](B\so) circle[radius=0.2cm]; 
	}
%%=========
\coordinate (C1) at (135:2.8284);
\draw (C1) circle [radius=0.2cm];
\foreach \so in {2,3,4,5}
{
	\coordinate (C\so) at ($ (C1)+(0,\so -1) $);
	\draw (C\so) circle[radius=0.2cm]; % Hoặc dùng lệnh \draw (0,0) circle (1cm);
	\coordinate (C\so) at ($ (C1)-(\so - 1,0) $);
	\draw (C\so) circle[radius=0.2cm]; 
}
%%===========
\coordinate (D1) at (135:4.2426);
\draw [fill](D1) circle[radius=0.2cm];
\foreach \so in {2,3,4}
{
	\coordinate (D\so) at ($ (D1)+(0,\so -1) $);
	\draw [fill](D\so) circle[radius=0.2cm]; % Hoặc dùng lệnh \draw (0,0) circle (1cm);
	\coordinate (D\so) at ($ (D1)-(\so - 1,0) $);
	\draw [fill](D\so) circle[radius=0.2cm]; 
}
%%=====
\coordinate (E1) at (135:5.6568);
\draw (E1) circle[radius=0.2cm];
\foreach \so in {2,3}
{
	\coordinate (E\so) at ($ (E1)+(0,\so -1) $);
	\draw (E\so) circle[radius=0.2cm]; % Hoặc dùng lệnh \draw (0,0) circle (1cm);
	\coordinate (E\so) at ($ (E1)-(\so - 1,0) $);
	\draw (E\so) circle[radius=0.2cm]; 
}
%%%
%%=====
\coordinate (F1) at (135:7.071);
\draw [fill](F1) circle[radius=0.2cm];
\foreach \so in {2}
{
	\coordinate (F\so) at ($ (F1)+(0,\so -1) $);
	\draw [fill](F\so) circle[radius=0.2cm]; % Hoặc dùng lệnh \draw (0,0) circle (1cm);
	\coordinate (F\so) at ($ (F1)-(\so - 1,0) $);
	\draw [fill](F\so) circle[radius=0.2cm]; 
}
%	
\coordinate (G1) at (135:8.485);
\draw (G1) circle[radius=0.2cm];
		\end{tikzpicture}
\end{center}
\loigiai{
\begin{enumerate}
\item Số chấm trắng và số chấm đen được sắp xếp kín hai cạnh của một hình vuông.
\item Tổng số chấm trắng tính từ góc trên bên trái xuống góc dưới bên phải của một hình vuông có độ dài cạnh $ 2n $ chấm là $$ S_1=(2\cdot 1-1)+(2\cdot 3-1) +(2\cdot 5 -1) + \ldots + [2\cdot (2n-1)-1]$$
Tổng số chấm đen tính từ góc trên bên trái xuống góc dưới bên phải của một hình vuông có độ dài cạnh $ 2n $ chấm là
$$S_2=(2\cdot 2 -1)+(2\cdot 4 -1)+...+(2\cdot 2n -1) .$$
Tổng số chấm tính từ góc trên bên trái xuống góc dưới bên phải của một hình vuông có độ dài cạnh $ 2n $ chấm là
$$S=S_1+S_2=(2\cdot 1 -1) +(2\cdot 2 -1 ) +\ldots + (2\cdot 2n -1) =2\left( 1+2+3+\ldots +2n\right) - 2n.$$
Tổng số chấm tính từ góc trên bên trái xuống góc dưới bên phải của một hình vuông có độ dài cạnh $ n $ chấm là
$$\dfrac{S}{4}=\dfrac{1}{2}\cdot \left(1+2+3+\ldots +2n \right)-\dfrac{n}{2}.$$
Ta cũng có tổng số chấm trong hình vuông mỗi cạnh chứa $ n $ chấm là $ n^2 $.\\
Như vậy $ \dfrac{1}{2}\cdot \left(1+2+3+\ldots +2n \right)-\dfrac{n}{2} = n^2 $.\\
Chứng minh bằng quy nạp.\\
Cho $ n=1 $ ta được $ \dfrac{1}{2}\cdot (1+2)-\dfrac{1}{2}=1$ (đúng).\\
Giả sử đẳng thức đúng khi $ n=k \in \mathbb{N}^* $, tức là $ \dfrac{1}{2}\cdot \left(1+2+3+\ldots +2k \right)-\dfrac{k}{2} = k^2 $.\\
Ta chứng minh đẳng thức đúng khi $ n=k+1, n\in \mathbb{N}^* $, tức là $$ \dfrac{1}{2}\cdot \left[1+2+3+\ldots +2k+(2k+1)+2(k+1) \right]-\dfrac{k+1}{2} = (k+1)^2.$$
Thật vậy $ \begin{aligned}[t]
\text{VT}&=\dfrac{1}{2}\cdot (1+2+3+\ldots +2k)+2k+\dfrac{3}{2}-\dfrac{k}{2} -\dfrac{1}{2}\\
&=k^2+2k+1\\
&=(k+1)^2=\text{VP}.
\end{aligned} $\\
Vậy $ \dfrac{1}{2}\cdot \left(1+2+3+\ldots +2n \right)-\dfrac{n}{2} = n^2 $.
\end{enumerate}
}
\end{bt}
%%==========
\begin{bt}%[0D2G1-1]
Giả sử năm đầu tiên, cô Hạnh gửi vào ngân hàng $ A $ (đồng) với lãi suất $ r\% $/ năm. Hết năm đầu tiên, cô Hạnh không rút tiền ra và gửi thêm $ A $ (đồng) nữa. Hết năm thứ hai, cô Hạnh cũng không rút tiền ra và lại gửi thêm $ A $ (đồng) nữa. Cứ tiếp tục như vậy cho những năm sau. Chứng minh số tiền cả vốn lẫn lãi mà cô Hạnh có được sau $ n $ (năm) là $ T_n=\dfrac{A(100+r)}{r}\left[\left( 1+\dfrac{r}{100}\right)^n-1  \right] $ (đồng), nếu trong khoảng thời gian này lãi suất không thay đổi.
\loigiai{
Hết năm đầu tiên số tiền cô Hạnh có được là $A\left(1+\dfrac{r}{100} \right)$.\\
Đầu năm thứ hai số tiền cô Hạnh có được là $A\left(1+\dfrac{r}{100} \right)+A$.\\
Cuối năm thứ hai số tiền cô Hạnh có được là $\left[ A\left(1+\dfrac{r}{100} \right)+A\right] \cdot \left( 1+\dfrac{r}{100}\right) =A\left(1+\dfrac{r}{100} \right)^2+A\left(1+\dfrac{r}{100} \right)$.\\
Đầu năm thứ ba số tiền cô Hạnh có được là $ A\left(1+\dfrac{r}{100} \right)^2+A\left(1+\dfrac{r}{100} \right)+A $.\\
Cuối năm ba số tiền cô Hạnh có được là $ \left[ A\left(1+\dfrac{r}{100} \right)^2+A\left(1+\dfrac{r}{100} \right)+A\right] \cdot \left( 1+\dfrac{r}{100}\right) =A\left(1+\dfrac{r}{100} \right)^3+A\left(1+\dfrac{r}{100} \right)^2+A\left(1+\dfrac{r}{100} \right)$.\\
\vdots \\
Cuối năm thứ $ n $ số tiền cô Hạnh có được là \\
\allowdisplaybreaks
\begin{eqnarray*}
T_n&=& A\left(1+\dfrac{r}{100} \right)^n+A\left(1+\dfrac{r}{100} \right)^{n-1}+\ldots +A\left(1+\dfrac{r}{100} \right)^2+A \left(1+\dfrac{r}{100} \right)\\
&=& A \left[ \left(1+\dfrac{r}{100} \right)^n+\left(1+\dfrac{r}{100} \right)^{n-1}+\ldots +\left(1+\dfrac{r}{100} \right)^2+\left(1+\dfrac{r}{100} \right)\right] \\
&=& \dfrac{A(100+r)}{r}\left[\left( 1+\dfrac{r}{100}\right)^n-1  \right]. 
\end{eqnarray*}
}
\end{bt}
%%%=========
\begin{bt}%[0D2G1-1]
Một người gửi số tiền $ A $ (đồng) vào ngân hàng. Biểu lãi suất của ngân hàng như sau: Chia mỗi năm thành $ m $ kì hạn và lãi suất $ r\% $/năm. Biết rằng nếu không rút tiền ra khỏi ngân hàng thì cứ sau mỗi kì hạn, số tiền lãi sẽ được nhập vào vốn ban đầu. Chứng minh số tiền nhận được (bao gồm cả vốn lẫn lãi) sau $ n $ (năm) gửi là $ S_n=A\left( 1+\dfrac{r}{100m}\right)^{m\cdot n}  $ (đồng), nếu trong khoảng thời gian này người gửi không rút tiền ra và lãi suất không thay đổi.
\loigiai{
Số tiền người đó có được sau kì hạn đầu tiên là $ A\left( 1+\dfrac{r}{100m}\right) $.\\
Số tiền người đó có được sau kì hạn thứ hai là $ A\left( 1+\dfrac{r}{100m}\right)\cdot \left( 1+\dfrac{r}{100m}\right)=A\cdot \left( 1+\dfrac{r}{100m}\right)^2$.\\
\vdots \\
Số tiền người đó có được sau năm đầu tiên (m kì hạn) là $A\left( 1+\dfrac{r}{100m}\right)^m $.\\
Số tiền người đó có được sau năm đầu tiên và kì hạn đầu tiên của năm thứ hai là $A\left( 1+\dfrac{r}{100m}\right)^m \left( 1+\dfrac{r}{100m}\right)=A\left( 1+\dfrac{r}{100m}\right)^{m+1} $.\\
\vdots \\
Số tiền người đó có được sau $ n $ năm là $S_n = A\left( 1+\dfrac{r}{100m}\right)^{\overbrace {m +\ldots +m}^{\text{n số m}}} =A\left( 1+\dfrac{r}{100m}\right)^{m\cdot n}$.
}
\end{bt}
\begin{bt}%[0D2K1-1]
	Sử dụng phương pháp quy nạp toán học, chứng minh các đẳng thức sau đúng với mọi số tự nhiên $ n\ge 1 $.
	\begin{enumerate}
	\item $ 2+4+6+\ldots + 2n=n(n+1) $;
	\item $ 1^2+2^2+3^2+\ldots +n^2=\dfrac{n(n+1)(2n+1)}{6} $.
	\end{enumerate}
\loigiai{
\begin{enumerate}
\item Cho $ n=1 $ ta được $ 2=1\cdot (1+1) $ (đúng).\\
Giả sử đẳng thức đúng khi $ n=k, n \in \mathbb{N}^*, n>1 $, tức là $ 2+4+6+\ldots +2k=k(k+1) $.\\
Ta chứng minh đẳng thức đúng khi $ n=k+1 $, $ n \in \mathbb{N}^*, n>1 $, tức là $$ 2+4+6+\ldots+2k+2(k+1)=(k+1)(k+2). $$
Thật vậy, $\text{với}\; k \in \mathbb{N}^*, k>1 $ ta có $\begin{aligned}[t]
	VT & = 2+4+6+\ldots+2k+2(k+1) \\
	& = k(k+1)+2(k+1)\\
	&=(k+1)(k+2)=VP.
\end{aligned}$\\
Vậy $ 2+4+6+\ldots + 2n=n(n+1) $, với $ n \in \mathbb{N}^* $, $ n>1 $.
\item Cho $ n=1 $ ta được $ 1^3=\dfrac{1(1+1)(2\cdot 1+1)}{6} $ (đúng).\\
Giả sử đẳng thức đúng khi $ n=k, n \in \mathbb{N}^*, n>1 $, tức là $$ 1^2+2^2+3^2+\ldots +n^2=\dfrac{k(k+1)(2k+1)}{6}. $$\\
Ta chứng minh đẳng thức đúng khi $ n=k+1 $, $ n \in \mathbb{N}^*, n>1 $, tức là $$  1^2+2^2+3^2+\ldots +k^2 +(k+1)^2=\dfrac{(k+1)(k+2)(2k+3)}{6} $$
Thật vậy, $\text{với}\; k \in \mathbb{N}^*, k>1 $ ta có $\begin{aligned}[t]
VT & = 1^2+2^2+3^2+\ldots +k^2 +(k+1)^2\\
& = \dfrac{k(k+1)(2k+1)}{6}+(k+1)^2 \\
&=\dfrac{(k+1)(k+2)(2k+3)}{6}=VP.
\end{aligned}$\\
Vậy $ 1^2+2^2+3^2+\ldots +n^2=\dfrac{n(n+1)(2n+1)}{6} $, với $ n \in \mathbb{N}^* $, $ n\ge1 $.
\end{enumerate}
}
\end{bt}
%%=======
\begin{bt}%[0D2K1-1]
Mỗi khẳng định sau là đúng hay sai? Nếu em nghĩ là nó đúng, hãy chứng minh nó. Nếu em nghĩ là nó sai, hãy đưa ra một phản ví dụ.
\begin{enumerate}
\item $ p(n)=n^2-n+11 $ là số nguyên tố với mọi số tự nhiên $ n $.
\item $ n^2>n $ với mọi số tự nhiên $ n\ge 2 $.
\end{enumerate}	
\loigiai{
\begin{enumerate}
\item Mệnh đề đã cho là mệnh đề đúng.\\
Chứng minh mệnh đề bằng quy nạp.\\
Cho $ n=1 $ ta được $ 1^2-1+11=11 $ là số nguyên tố (đúng).\\
Giả sử mệnh đề đúng với $ n=k $, $ k \in \mathbb{N}^* $, $ k>1 $, tức là $ k^2-k+11 $ là số nguyên tố.\\
Ta chứng minh mệnh đề đúng với $ n=k+1 $, $ k \in \mathbb{N} $, $ k>1 $ tức là $ (k+1)^2-(k+1)+11 $ là số nguyên tố,\\
Thật vậy, đặt $ u=k+1 $, $ u\in \mathbb{N} $, $ u>1 $ thì  $ (k+1)^2-(k+1)+11 =u^2-u+11$.\\
Theo giả thuyết quy nạp thì $ u^2-u+11 $, $ u\in \mathbb{N} $, $ u>1 $ là số nguyên tố.\\
Vậy $ p(n)=n^2-n+11 $ là số nguyên tố với mọi số tự nhiên $ n $.
\item Mệnh đề đã cho là mệnh đề đúng.\\
Chứng minh bằng phương pháp quy nạp.\\
Cho $ n=2 $ ta được $ 2^2>2 $ (đúng).\\
Giả sử mệnh đề đúng với $ n=k $, $k\in \mathbb{N}, k>2 $, tức là $ k^2>k $.\\
Ta chứng minh mệnh đề đúng với $ n=k+1 $, $ k\in \mathbb{N}, k>2 $, tức là $ (k+1)^2>k+1 $.\\
Thật vậy, $ (k+1)^2=k^2+2k+1>3k+1>k+1 $, $ k \in \mathbb{N} $, $ k>2 $.\\
Vậy  $ n^2>n $ với mọi số tự nhiên $ n\ge 2 $.
\end{enumerate}
}
\end{bt}
\Closesolutionfile{ans}