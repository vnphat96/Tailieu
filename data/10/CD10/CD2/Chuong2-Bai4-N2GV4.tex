\section{NHỊ THỨC NEWTON}

\subsection{Lí thuyết}

\subsubsection{Công thức nhị thức Niu-tơn}

\begin{dn}
	Cho $a$, $b\in\mathbb{R}$ và $n\in\mathbb{N}^*$, ta có
	\begin{center}
		$(a+b)^n=\mathrm{C}_n^0a^n+\mathrm{C}_n^1a^{n-1}b+\mathrm{C}_n^2a^{n-2}b^2+\cdots +\mathrm{C}_n^{n-1}ab^{n-1}+\mathrm{C}_n^nb^n.$ \quad $(1)$
	\end{center}
	Công thức $(1)$ gọi là \textbf{công thức nhị thức Newton}, gọi tắt là \textbf{nhị thức Newton}.
\end{dn}

\begin{nx}
	\begin{itemize}
		\item Trong khai triển $(a \pm b)^{n} $ có $n+1$ số hạng và các hệ số của cặp số hạng cách đều số hạng đầu và số hạng cuối thì bằng nhau: $\mathrm{C}^{k}_n = \mathrm{C}^{n-k}_n $ .
		\item Số hạng tổng quát dạng: $T_{n+1} = \mathrm{C}^{k}_n \cdot a^{n-k} \cdot b^{k} $ và số hạng thứ $N$ thì $k = N - 1$.
		\item Trong khai triển  $(a - b)^{n} $ thì dấu đan nhau, nghĩa là: $+$, rồi $-$, rồi $+$,...
		\item Số mũ của $a$ giảm dần, số mũ của $b$ tăng dần nhưng tổng số mũ $a$ và $b$ bằng $n$.
	\end{itemize}
\end{nx}

\begin{vd}%[1D2B3-1]
	Hãy khai triển $\left(a+b\right)^6$.
	\loigiai{
		\allowdisplaybreaks
		\begin{eqnarray*}
			\left(a+b\right)^6&=&\mathrm{C}_6^0a^6+\mathrm{C}_6^1a^5b+\mathrm{C}_6^2a^4b^2+\mathrm{C}_6^3a^3b^3+\mathrm{C}_6^4a^2b^4+\mathrm{C}_6^5ab^5+\mathrm{C}_6^6b^6\\
			&=&a^6+6a^5b+15a^4b^2+31a^3b^3+15a^2b^4+6ab^5+b^6.
		\end{eqnarray*}
	}
\end{vd}

\begin{vd}%[1D2B3-1]
	Hãy khai triển  $\left(x+1\right)^5$.
	\loigiai{
		\allowdisplaybreaks
		\begin{eqnarray*}
			\left(x-1\right)^5 &=&\mathrm{C}_5^0x^5+\mathrm{C}_5^1x^4(-1)+\mathrm{C}_5^2x^3(-1)^2+\mathrm{C}_5^3x^2(-1)^3+\mathrm{C}_5^4x(-1)^4+\mathrm{C}_5^5(-1)^5\\
			&=&x^5-5x^4+10x^3-10x^2+5x-1.
		\end{eqnarray*}
	}
\end{vd}



\subsubsection{Tam giác Pascal}

\begin{dn}
	Các hệ số của các khai triển $(a+b)^0$, $(a+b)^1$, $(a+b)^2$, \dots, $(a+b)^n$ có thể xếp thành một tam giác gọi là tam giác Pascal.
	\begin{longtable}{m{0.4\textwidth}m{0.4\textwidth}}
		$\begin{array}{lllllllll}
			n=0\colon & 1 & & & & & & &\\
			n=1\colon & 1 & 1 & & & & & &\\
			n=2\colon & 1 & 2 & 1 & & & & &\\
			n=3\colon & 1 & 3 & 3 & 1 & & & &\\
			n=4\colon & 1 & 4 & 6 & 4 & 1 & & &\\
			n=5\colon & 1 & 5 & 10 & 10 & 5 & 1 & &\\
			n=6\colon & 1 & 6 & 15 & 31 & 15 & 6 & 1 &\\
			n=7\colon & 1 & 7 & 21 & 35 & 35 & 21 & 7 & 1\\
		\end{array}$
		& \centerline{\bf HẰNG ĐẲNG THỨC PASCAL}\newline
		$$\mathrm{C}_{n-1}^{k-1}+\mathrm{C}_{n-1}^k=\mathrm{C}_n^k.$$
	\end{longtable}
\end{dn}


\begin{vd}%[1D2B3-1]
	Sử dụng tam giác Pascal hãy khai triển  $(a+2b)^5$.
	\loigiai{
		\allowdisplaybreaks
		\begin{eqnarray*}
			(a+2b)^5&=&a^5+5\cdot a^4\cdot 2b+10\cdot a^3\cdot (2b)^2+10\cdot a^2\cdot (2b)^3+5\cdot a \cdot (2b)^4+(2b)^5\\
			&=& a^5+10a^4b+40a^3b^2+80a^2b^3+80ab^4+32b^5.
		\end{eqnarray*}
	}
\end{vd}

\begin{vd}%[1D2B3-1]
	Sử dụng tam giác Pascal hãy khai triển  $\left(2x-1\right)^6$.
	\loigiai{
		\allowdisplaybreaks
		\begin{eqnarray*}
			\left(2x-1\right)^6 &=&(2x)^6+6\cdot (2x)^5\left(-1\right)+15\cdot (2x)^4\left(-1\right)^2+31\cdot (2x)^3\left(-1\right)^3\\
			&&+15\cdot (2x)^2\left(-1\right)^4+6\cdot (2x)\left(-1\right)^5+\left(-1\right)^6\\
			&=&64x^6-192x^5+240x^4-160x^3+60x^2-12x+1.
		\end{eqnarray*}
	}
\end{vd}

\subsection{Vận dụng công thức nhị thức Newton}

\begin{vd}%[1D2B3-2]
	Xác định hệ số của $x^8y^9$ trong khai triển  $(2x-3y)^{17}$.
	\loigiai
	{		
		Số hạng tổng quát trong khai triển $(2x-3y)^{17}$ là
		$\mathrm{C}_{17}^k(2x)^{17-k}(-3y)^k=\mathrm{C}_{17}^k2^{17-k}(-3)^kx^{17-k}y^k$.\\
		Để có số hạng chứa $x^8y^9$ thì $k=9$.\\
		Vậy hệ số của số hạng chứa $x^8y^9$ là $\mathrm{C}_{17}^9\cdot 2^8\cdot (-3)^9=-2^8 3^9\mathrm{C}_{17}^9$.
	}
\end{vd}

\begin{vd}%[1D2B3-2]
	Cho $a$ là một số thực dương. Biết rằng trong khai khiển $(5x+a)^{10}$, hệ số của $x^5$ là $252$. Hãy tìm giá trị của $a$.
	\loigiai
	{		
		Số hạng tổng quát trong khai triển $(5x+a)^{10}$ là
		$\mathrm{C}_{10}^k\cdot (5x)^{10-k}\cdot (a)^k=\mathrm{C}_{10}^k\cdot 5^{10-k}\cdot a^k\cdot x^{10-k}$.\\
		Để có số hạng chứa $x^5$ thì $k=5$. Hệ số của số hạng này là $ \mathrm{C}_{10}^5\cdot 5^5\cdot a^5=252\Leftrightarrow a=\dfrac{1}{5}$.\\
		Vậy $a=\dfrac{1}{5}$ là giá trị cần tìm.
	}
\end{vd}

\begin{vd}%[1D2K3-3]
	Chứng minh
	$\mathrm{C}_{2n}^0 + \mathrm{C}_{2n}^2 + \cdots + \mathrm{C}_{2n}^{2n}=\mathrm{C}_{2n}^1 + \mathrm{C}_{2n}^3 + \cdots + \mathrm{C}_{2n}^{2n-1}=2^{2n-1}$.
	\loigiai{
		Xét $ (1+1)^{2n}=\displaystyle\sum\limits_{k=0}^{2n} \mathrm{C}_{2n}^k \cdot 1^{2n-k} \cdot 1^k =\mathrm{C}_{2n}^0 + \mathrm{C}_{2n}^1 + \mathrm{C}_{2n}^2 + \mathrm{C}_{2n}^3 + \mathrm{C}_{2n}^4 + \cdots + \mathrm{C}_{2n}^{2n} $.\tagEX{1} \noindent
		Xét $ (1-1)^{2n}=\displaystyle\sum\limits_{k=0}^{2n} \mathrm{C}_{2n}^k \cdot 1^{2n-k} \cdot (-1)^k =\mathrm{C}_{2n}^0 - \mathrm{C}_{2n}^1 + \mathrm{C}_{2n}^2 - \mathrm{C}_{2n}^3 + \mathrm{C}_{2n}^4 + \cdots + \mathrm{C}_{2n}^{2n} $.\tagEX{2} \noindent
		Lấy $ (1) $ cộng $ (2) $ ta được
		\allowdisplaybreaks
		\begin{eqnarray*}
			&& 2^{2n} + 0^{2n} =2 \left( \mathrm{C}_{2n}^0 + \mathrm{C}_{2n}^2 + \mathrm{C}_{2n}^4 + \mathrm{C}_{2n}^6 + \cdots + \mathrm{C}_{2n}^{2n} \right) \\ 
			&\Leftrightarrow& \mathrm{C}_{2n}^0 + \mathrm{C}_{2n}^2 + \mathrm{C}_{2n}^4 + \mathrm{C}_{2n}^6 + \cdots + \mathrm{C}_{2n}^{2n} =2^{2n-1}. 
		\end{eqnarray*}
		Lấy $ (1) $ trừ $ (2) $ ta được
		\allowdisplaybreaks
		\begin{eqnarray*}
			&& 2^{2n} - 0^{2n} =2 \left( \mathrm{C}_{2n}^1 + \mathrm{C}_{2n}^3 + \mathrm{C}_{2n}^5 + \mathrm{C}_{2n}^7 + \cdots + \mathrm{C}_{2n}^{2n-1} \right) \\ 
			&\Leftrightarrow&   \mathrm{C}_{2n}^1 + \mathrm{C}_{2n}^3 + \mathrm{C}_{2n}^5 + \mathrm{C}_{2n}^7 + \cdots + \mathrm{C}_{2n}^{2n-1} =2^{2n-1}. 
		\end{eqnarray*}
		Vậy $\mathrm{C}_{2n}^0 + \mathrm{C}_{2n}^2 + \cdots + \mathrm{C}_{2n}^{2n}=\mathrm{C}_{2n}^1 + \mathrm{C}_{2n}^3 + \cdots + \mathrm{C}_{2n}^{2n-1}=2^{2n-1}$.
	}
\end{vd}


\begin{vd}%[1D2B3-3]
	Tính tổng $S=\mathrm{C}_{5}^{0}+2\mathrm{C}_{5}^{1}+2^{2}\mathrm{C}_{5}^{2}+ \cdots+2^5\mathrm{C}_{5}^{5}$. 
	\loigiai{
		Ta có
		$(a+b)^{5}=\mathrm{C}_{5}^{0}a^{5}+\mathrm{C}_{5}^{1}a^{4}b+\mathrm{C}_{5}^{2}a^{3}b^{2}+\mathrm{C}_{5}^{3}a^{2}b^3 +\mathrm{C}_{5}^{4}ab^{4}+\mathrm{C}_{5}^{5}b^{5}$.\\
		Cho $a=1$, $b=2$, ta có
		\[3^{5}=\mathrm{C}_{5}^{0}+2\mathrm{C}_{5}^{1}+2^2\mathrm{C}_{5}^{2}+ 2^3\mathrm{C}_{5}^{3}+2^4\mathrm{C}_{5}^{4}+2^5\mathrm{C}_{5}^{5}.\]
		Vậy $S=3^5$.
	}
\end{vd}
\Closesolutionfile{ans}