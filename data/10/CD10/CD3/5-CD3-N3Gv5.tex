\section{PARABOL}
\subsection{Các dạng toán}
\begin{dang}{Hình dạng của parabol}
	Cho parabol có phương trình chính tắc $y^2=2px$ $(p>0)$. Khi đó:
	\begin{itemize}
		\item Parabol có một trục đối xứng là $Ox$ (đi qua tiêu điểm và vuông góc với đường chuẩn).
		\item Giao điểm $O(0;0)$ của parabol và trục đối xứng được gọi là đỉnh của parabol.
		\item Tham số tiêu $p$ gấp đôi khoảng cách giữa đỉnh $O(0;0)$ và tiêu điểm $F\left(\dfrac{p}{2};0\right)$.
		\item Trong phương trình chính tắc, các điểm thuộc parabol đều có hoành độ không âm.
	\end{itemize}
\end{dang}
%Ví dụ 1
\begin{vd}
	Trong mặt phẳng tọa độ $Oxy$ parabol $(P)$ có phương trình chính tắc và đi qua điểm $A(6;6)$. Tìm tham số tiêu và tiêu điểm của $(P)$.
	\loigiai{
		Phương trình chính tắc của parabol có dạng $y^2=2px$ $(p>0)$.\\
		Vì parabol đi qua $A(6;6)$ nên $6^2=2p\cdot 6\Rightarrow p=\dfrac{1}{3}$.\\
		Tiêu điểm của parabol là $F\left(\dfrac{1}{6};0\right)$.\\
	}
\end{vd}
%ví dụ 2
\begin{vd}
	Trong mặt phẳng tọa độ $Oxy$ parabol $(P)$ có phương trình chính tắc $y^2=4x$. Tìm tham số tiêu và tiêu điểm của $(P)$.
	\loigiai{
		Phương trình chính tắc của parabol có dạng $y^2=2px$ $(p>0)$.\\
		Suy ra $2p=4\Rightarrow p=2$.
		Parabol có tham số tiêu $p=2$ và tiêu điểm $F(1;0)$.
	}
\end{vd}
%Ví dụ 3
\begin{vd}
	Trong mặt phẳng tọa độ $Oxy$ parabol $(P)$ có phương trình chính tắc và đi qua điểm $A(1;4)$. Tìm tham số tiêu và phương trình đường chuẩn của $(P)$.
	\loigiai{
		Phương trình chính tắc của parabol có dạng $y^2=2px$ $(p>0)$.\\
		Vì parabol đi qua $A(1;4)$ nên $4^2=2p\cdot 1\Rightarrow p=4$.\\
		Tham số tiêu của parabol là $p=4$, phương trình đường chuẩn của parabol là $x+4=0$.\\
	}
\end{vd}
%Bài 1
\begin{bt}
	Cho parabol có phương trình chính tắc $y^2=8x$. Tìm tham số tiêu và phương trình đường chuẩn của parabol đo.
	\loigiai{
	Phương trình chính tắc của parabol có dạng $y^2=2px$ $(p>0)$.\\
	Ta có $2p=8\Rightarrow p=4$.\\
	Tham số tiêu của parabol là $p=4$, phương trình đường chuẩn của parabol là $x+4=0$.\\
}
\end{bt}
%Bài 2
\begin{bt}
	Trong mặt phẳng $Oxy$, cho parabol có phương trình chính tắc và biết đường chuẩn $\Delta\colon x+2=0$. Xác định tham số tiêu và phương trình chính tắc của parabol.
	\loigiai{
	Phương trình chính tắc parabol có dạng $y=2px$ $(p>0)$.\\
	Vì parabol có đường chuẩn $\Delta\colon x+2=0$ nên $p=2$.\\
	Vậy parabol có phương trình $y^2=4x$.
	}
\end{bt}
%Bài 3
\begin{bt}
	Tìm tham số tiêu và đường chuẩn của parabol $y^2=24x$.
	\loigiai{
		Ta có $2p=24\Rightarrow p=12$.\\
		Tham số tiêu là $p=12$, đường chuẩn $\Delta\colon x+12=0$.
	}
\end{bt}

%=====================================================
\begin{dang}{Bán kính qua tiêu, tâm sai và đường chuẩn}
	Cho parabol có phương trình chính tắc $y^2=2px$ $(p>0)$. Khi đó:
	\begin{itemize}
		\item Parabol có tiêu điểm $F\left(\dfrac{p}{2};0\right)$ và đường chuẩn $\Delta\colon x=-\dfrac{p}{2}$.
		\item Với điểm $M(x;y)$ thuộc parabol, đoạn thẳng $MF$ được gọi là bán kính qua tiêu của $M$ và có độ dài $MF=x+\dfrac{p}{2}$.
		\item Với mọi điểm $M(x;y)$ thuộc parabol, tỉ số $\dfrac{MF}{\mathrm{d}(M,\Delta)}$ luôn bằng $1$. Ta nói parabol có tâm sai bằng $1$.
	\end{itemize}
\end{dang}

%Ví dụ 1
\begin{vd}
	Cho parabol có phương trình $y^2=6x$. Tìm tọa độ tiêu điểm và phương trình đường chuẩn của parabol.
	\loigiai{
	Ta có $2p=6$, suy ra $p=3$.\\
			Tiêu điểm của parabol là $F\left(\dfrac{3}{2};0\right)$.\\
			Đường chuẩn của parabol có phương trình là $x=-\dfrac{3}{2}$.
	}
\end{vd}
%Ví dụ 2
\begin{vd}
	Cho parabol có phương trình $y^2=8x$. 
	\begin{enumerate}
		\item Tìm tọa độ tiêu điểm và phương trình đường chuẩn của parabol.
		\item Tính bán kính qua tiêu của điểm $M$ thuộc parabol biết điểm $M$ có tung độ bằng $4$.
	\end{enumerate}
	\loigiai{
		\begin{enumerate}
			\item Ta có $2p=8\Leftrightarrow p=4$.\\
			Tiêu điểm của parabol là $F\left(2;0\right)$.\\
			Đường chuẩn của parabol có phương trình là $x=-2$.
			\item Ta có $M(x;4)$ thuộc parabol nên ta có $4^2=8x\Leftrightarrow x=2$.\\
			Bán kính qua tiêu điểm $M(2;4)$ là $MF=x+\dfrac{p}{2}=2+2=4$.
		\end{enumerate}
	}
\end{vd}
%Ví dụ 3
\begin{vd}
	Chứng minh rằng trong các điểm thuộc parabol thì đỉnh parabol có khoảng cách tới tiêu điểm nhỏ nhất và khoảng cách đó bằng một nửa tham số tiêu.
	\loigiai{
		Giả sử parabol có phương trình chính tắc là $y^2=2px$, $(p>0)$. Với điểm $M\left(x_0;y_0\right)$ bất kì thuộc parabol, ta có $x_0\ge 0$. Do đó, theo công thức bán kính qua tiêu, ta có $MF=x_0+\dfrac{p}{2}\ge \dfrac{p}{2}$. Dấu đẳng thức xảy ra khi và chỉ khi $x_0=0$ (và do đó $y_0=0)$, tức là $M$ trùng với đỉnh $O(0;0)$ của parabol. Từ đó, ta nhận được điều phải chứng minh.
	}
\end{vd}
%Ví dụ 4
\begin{vd}
	Một sao chổi chuyển động theo quỹ đạo parabol nhận tâm Mặt Trời làm tiêu điểm. Khoảng cách ngắn nhất từ sao chổi đến tâm Mặt Trời là $106~\mathrm{km}$. 
	\begin{enumerate}
		\item Lập phương trình chính tắc của quỹ đạo theo đơn vị kilômét. 
		\item Hỏi khi sao chổi nằm trên đường vuông góc với trục đối xứng của quỹ đạo tại tâm Mặt Trời, thì khoảng cách từ sao chổi đến tâm Mặt Trời là bao nhiêu kilômét?
	\end{enumerate}
	\loigiai{
	\immini{\begin{enumerate}
			\item Gọi sao chổi là $A$. Phương trình chính tắc của parabol $(P)$ có dạng $y^2=2px$ $(p>0)$. Khi đó tiêu điểm $F\left(\dfrac{p}{2};0\right)$ là tâm Mặt Trời.\\
			Ta có khoảng cách ngắn nhất của $AF$ là $OF=\dfrac{p}{2}=106\Rightarrow p=212$.\\
			Vậy phương trình chính tắc của parabol $(P)$ là $y^2=424x$.
			\item Vì sao chổi $A$ nằm trên đường thẳng đi qua tiêu điểm và vuông góc với trục đối xứng của $(P)$ nên ta có $A\left(106;y\right)$.\\
			Khoảng cách từ sao chổi $A$ đến tâm Mặt Trời chính lá bán kính qua tiêu điểm $M$ của parabol, do đó khoảng cách là $MF=x+\dfrac{p}{2}=106+\dfrac{212}{2}=212~\mathrm{km}$.
		\end{enumerate}
		
	}{\begin{tikzpicture}[scale=1, font=\footnotesize, line join=round, line cap=round, >=stealth]
			\def\p{2}
			\path 
			(\p/2,0) coordinate (F)
			(0,0) coordinate (O)
			;
			\draw[->] (-1,0) -- (4,0) node[below] {$x$};
			\draw[->] (0,-2) -- (0,3) node[left] {$y$};
			%			\draw (d1)--(d2) node[left] {$\Delta$};
			\begin{scope}[rotate=-90]
				\path (-1.8,2.5) coordinate (A);
				\draw[dashed] (-1.8,2.5)--(1.8,2.5);
				\draw (-1.8,2.5)  parabola bend (0,0.01) (1.8,2.5) node[right]{$(P)$};
			\end{scope}
			%				\draw (2.5,0)node[above right]{$H$} circle(1pt);
			\foreach \p/\g in {O/-135, F/45, A/0}
			\draw[fill=black] (\p) circle (1pt) node[shift=(\g:3mm)] {$\p$};
	\end{tikzpicture}}		
	}
\end{vd}
%Bài 1
\begin{bt}%[]
	Cho parabol có phương trình $y^2=12x$.
	\begin{enumerate}
		\item Tìm tiêu điểm và đường chuẩn của parabol.
		\item Tính bán kính qua tiêu của điểm $M$ thuộc parabol và có hoành độ bằng $5$.
	\end{enumerate}
	\loigiai{
		\begin{enumerate}
				\item Ta có $2p=12$, suy ra $p=6$.\\
				Tiêu điểm của parabol là $F\left(3;0\right)$.\\
				Đường chuẩn của parabol có phương trình là $x=-3$.
				\item Ta có $M(5;y)$ thuộc parabol bán kính qua tiêu điểm $M$ là $MF=x+\dfrac{p}{2}=5+6=11$.	
		\end{enumerate}	
	}
\end{bt}
%Bài 2
\begin{bt}%[]
	Tìm tọa độ tiêu điểm và phương trình đường chuẩn của các parabol sau
	\begin{enumerate}
		\item $(P_1)\colon y^2=7x$.
		\item $(P_2)\colon y^2=\dfrac{1}{3}x$.
		\item $(P_3)\colon y^2=\sqrt{2}x$.
	\end{enumerate}
	\loigiai{
		\begin{enumerate}
			\item $(P_1)\colon y^2=7x$.\\
			Ta có $2p=7\Rightarrow p=\dfrac{7}{2}$.\\
			Tiêu điểm là $M\left(\dfrac{7}{4};0\right)$.\\
			Phương trình đường chuẩn là $x=-\dfrac{7}{4}$.
			\item $(P_2)\colon y^2=\dfrac{1}{3}x$.\\
			Ta có $2p=\dfrac{1}{3}\Rightarrow p=\dfrac{1}{6}$.\\
			Tiêu điểm là $M\left(\dfrac{1}{12};0\right)$.\\
			Phương trình đường chuẩn là $x=-\dfrac{1}{12}$.
			\item $(P_3)\colon y^2=\sqrt{2}x$.\\
			Ta có $2p=\sqrt{2}\Rightarrow p=\dfrac{\sqrt{2}}{2}$.\\
			Tiêu điểm là $M\left(\dfrac{\sqrt{2}}{4};0\right)$.\\
			Phương trình đường chuẩn là $x=-\dfrac{\sqrt{2}}{4}$.
		\end{enumerate}	
	}
\end{bt}
%Bài 3
\begin{bt}%[]
	Tính bán kính qua tiêu của điểm đã cho trên các parabol sau:
	\begin{enumerate}
		\item Điểm $M_1(3;-6)$ trên $(P_1)\colon y^2=12x$.
		\item Điểm $M_2(6;1)$ trên $(P_2)\colon y^2=\dfrac{1}{6}x$.
		\item Điểm $M_3\left(\sqrt{3};\sqrt{3}\right)$ trên $(P_3)\colon y^2=\sqrt{3}x$.
	\end{enumerate}
	\loigiai{
		\begin{enumerate}
			\item Ta có $2p=12\Rightarrow p=6$. Do dó bán kính qua tiêu của $M_1$ là $M_1F=x+\dfrac{p}{2}=3+3=6$.
			\item Ta có $2p=\dfrac{1}{6}\Rightarrow p=\dfrac{1}{12}$. Do dó bán kính qua tiêu của $M_2$ là $M_2F=x+\dfrac{p}{2}=6+\dfrac{1}{24}=\dfrac{145}{24}$.
			\item Ta có $2p=\sqrt{3}\Rightarrow p=\dfrac{\sqrt{3}}{2}$. Do dó bán kính qua tiêu của $M_3$ là $M_3F=x+\dfrac{p}{2}=\sqrt{3}+\dfrac{\sqrt{3}}{4}=\dfrac{5\sqrt{3}}{4}$.
		\end{enumerate}	
	}
\end{bt}
%Bài 4
\begin{bt}%[]
	Trong mặt phẳng tọa độ $Oxy$, parabol $(P)$ có phương trình chính tắc và đi qua điểm $M\left(3;3\sqrt{2}\right)$. Tính bán kính qua tiêu và khoảng cách từ tiêu điểm tới đường chuẩn của $(P)$.
	\loigiai{
	Phương chình chính tắc của $(P)$ là $y^2=2px$ $(p>0)$.\\
	Vì $M\left(3;3\sqrt{2}\right)\in (P)$ nên $\left(3\sqrt{2}\right)^2=2p\cdot 3\Leftrightarrow p=3$.\\
	Bán kính qua tiêu của $M\left(3;3\sqrt{2}\right)$ là $MF=x+\dfrac{p}{2}=3+\dfrac{3}{2}=\dfrac{9}{2}$.\\
	Khoảng cách từ tiêu điểm tới đường chuẩn của $(P)$ là $2\cdot \dfrac{p}{2}=p=3$.
	}
\end{bt}
%Bài 5
\begin{bt}%[]
	Trong mặt phăng $Oxy$, cho điểm $A\left(\dfrac{1}{4};0\right)$ và đường thẳng $d\colon x+\dfrac{1}{4}=0$. Viết phương trình của đường $(P)$ là tập hợp tâm $M(x;y)$ của các đường tròn $(C)$ di động nhưng luôn luôn đi qua $A$ và tiếp xúc với $d$.
	\loigiai{Giả sử đường tròn $(C)$ có bán kính $R$.\\
		Ta có $A\in (C)\Leftrightarrow AM=R\Leftrightarrow \left(x-\dfrac{1}{4}\right)^2+y^2=R^2$.\quad(1)\\
		Mặt khác ta có $\mathrm{d}\left(M,d\right)=R\Rightarrow \left|x+\dfrac{1}{4}\right|=R\Leftrightarrow \left(x+\dfrac{1}{4}\right)^2=R^2$.\quad(2)\\
		Từ (1) và (2) suy ra $\begin{aligned}[t]&\left(x-\dfrac{1}{4}\right)^2+y^2=\left(x+\dfrac{1}{4}\right)^2\\
			\Leftrightarrow~ &x^2-\dfrac{1}{2}x+\dfrac{1}{16}+y^2=x^2+\dfrac{1}{2}x+\dfrac{1}{16}\\
			\Leftrightarrow~&y^2=x.
			\end{aligned}$\\
	Vậy $(P)\colon y^2=x$.
}
\end{bt}
%Bài 6
\begin{bt}%[]
	Cho parabol $(P)$. Trên $(P)$ lấy hai điểm $M$, $N$ sao cho đoạn thẳng $MN$ đi qua tiêu điểm $F$ của $(P)$. Chứng minh rằng khoảng cách từ trung điểm $I$ của đoạn thẳng $MN$ đến đường chuẩn $\Delta$ của $(P)$ bằng $\dfrac{1}{2}MN$ và đường tròn đường kính $MN$ tiếp xúc với $\Delta$.
	\loigiai{
		\immini{
	Hạ $MM'$, $NN'$, $II'$ vuông góc với đường chuẩn $\Delta$.\\
	Hình thang vuông $MNN'M'$ có $II'$ là đường trung bình nên 
	\[\mathrm{d}(I,\Delta)=II'=\dfrac{1}{2}\left(MM'+NN'\right)\]
	Mà $M$, $N$ thuộc parabol $(P)$ nên
	\[NF=NN' ~\text{và}~ MF=MM'\]	
	Do đó $MM'+NN'=MF+NF$.\\
	Vậy $\mathrm{d}(I,\Delta)=\dfrac{1}{2}MN$, suy ra đường tròn đường kính $MN$ tiếp xúc đường chuẩn $\Delta$.
	}{\begin{tikzpicture}[scale=1, font=\footnotesize, line join=round, line cap=round, >=stealth]
				\def\p{2}
				\path 
				(\p/2,0) coordinate (F)
				(0,0) coordinate (O)
				(0.75,-1) coordinate (M)
				(-\p/2,2.5) coordinate (d1)
				(-\p/2,-2.5) coordinate (d2)
				($(M)!2.32!(F)$) coordinate (N)
				($(M)!0.5!(N)$) coordinate (I)
				($(d1)!(M)!(d2)$) coordinate (M')
				($(d1)!(N)!(d2)$) coordinate (N')
				($(d1)!(I)!(d2)$) coordinate (I')
				;
				\draw[->] (-2,0) -- (4,0) node[below] {$x$};
				\draw[->] (0,-3) -- (0,3) node[left] {$y$};
				\draw (d1)--(d2) node[left] {$\Delta$};
				\begin{scope}[rotate=-90]
					\draw 
					(-1.8,2.5) node[above]{$(P)$} parabola bend (0,0.01) (1.8,2.5);
				\end{scope}
				%		\draw (M) let \p1=($(M)-(K)$) in circle ({veclen(\x1,\y1)});
				\draw (N)--(M) (M)--(M') (N)--(N') (I)--(I');
				\foreach \p/\g in {O/-135, F/-45, M/-100, N/100, I'/180, M'/180, N'/180, I/20}
				\draw[fill=black] (\p) circle (1pt) node[shift=(\g:3mm)] {$\p$};
		\end{tikzpicture}}
	}
\end{bt}
%Bài 7
\begin{bt}%[]
	Hãy so sánh bán kính qua tiêu của điểm $M$ trên parabol $(P)$ với bán kính của đường tròn tâm $M$, tiếp xúc với đường chuẩn của $(P)$.
	\loigiai{
	\immini{Giả sử parabol $(P)$ có phương trình chính tắc $y^2=2px$ $(p>0)$.\\
	Bán kính qua tiêu của điểm $M(x_0;y_0)\in (P)$ là $MF=x_0+\dfrac{p}{2}$.\\
	Đường tròn tâm $M(x_0;y_0)$ tiếp xúc đường chuẩn $\Delta\colon x+\dfrac{p}{2}=0$ của $(P)$ có bán kính $R=\mathrm{d}(M,\Delta)$.\\
	Ta có $\mathrm{d}(M,\Delta)=\dfrac{x_0+\dfrac{p}{2}}{1}=x_0+\dfrac{p}{2}=MF$.
	}{\begin{tikzpicture}[scale=1, font=\footnotesize, line join=round, line cap=round, >=stealth]
		\def\p{2}
		\path 
		(\p/2,0) coordinate (F)
		(0,0) coordinate (O)
		(1.2,-1.25) coordinate (M)
		(-\p/2,2.5) coordinate (d1)
		(-\p/2,-2.5) coordinate (d2)
		($(d1)!(M)!(d2)$) coordinate (K)
		;
		\draw[->] (-2,0) -- (4,0) node[below] {$x$};
		\draw[->] (0,-3) -- (0,3) node[left] {$y$};
		\draw (d1)--(d2) node[left] {$\Delta$};
		\begin{scope}[rotate=-90]
		\draw 
		(-1.8,2.5) node[above]{$(P)$} parabola bend (0,0.01) (1.8,2.5);
		\end{scope}
%		\draw (M) let \p1=($(M)-(K)$) in circle ({veclen(\x1,\y1)});
		\draw (F)--(M)--(K);
		\foreach \p/\g in {O/-135, F/45, M/-100, K/180}
		\draw[fill=black] (\p) circle (1pt) node[shift=(\g:3mm)] {$\p$};
		\end{tikzpicture}}
	}
\end{bt}
%Bài 8
\begin{bt}%[]
	Một sao chổi $A$ chuyển động theo quỹ đạo có dạng một parabol $(P)$ nhận tâm Mặt Trời là tiêu điểm. Cho biết khoảng cách ngắn nhất giữa sao chổi $A$ và tâm Mặt Trời là khoảng $112~\mathrm{km}$. 
	\begin{enumerate}
		\item Viết phương trình chính tắc của parabol $(P)$.
		\item Tính khoảng cách giữa sao chổi $A$ và tâm Mặt Trời khi sao chổi nằm trên đường thẳng đi qua tiêu điểm và vuông góc với trục đối xứng của $(P)$.
	\end{enumerate}
	\loigiai{
		\immini{\begin{enumerate}
			\item Phương trình chính tắc của parabol $(P)$ có dạng $y^2=2px$ $(p>0)$. Khi đó tiêu điểm $F\left(\dfrac{p}{2};0\right)$ là tâm Mặt Trời.\\
			Ta có khoảng cách ngắn nhất của $AF$ là $OF=\dfrac{p}{2}=120\Rightarrow p=240$.\\
			Vậy phương trình chính tắc của parabol $(P)$ là $y^2=480x$.
			\item Vì sao chổi $A$ nằm trên đường thẳng đi qua tiêu điểm và vuông góc với trục đối xứng của $(P)$ nên ta có $A\left(120;y\right)$.\\
			Khoảng cách từ sao chổi $A$ đến tâm Mặt Trời chính lá bán kính qua tiêu điểm $M$ của parabol, do đó khoảng cách là $MF=x+\dfrac{p}{2}=120+\dfrac{240}{2}=240~\mathrm{km}$.
		\end{enumerate}
		
	}{\begin{tikzpicture}[scale=1, font=\footnotesize, line join=round, line cap=round, >=stealth]
				\def\p{2}
				\path 
				(\p/2,0) coordinate (F)
				(0,0) coordinate (O)
				;
				\draw[->] (-1,0) -- (4,0) node[below] {$x$};
				\draw[->] (0,-2) -- (0,3) node[left] {$y$};
				%			\draw (d1)--(d2) node[left] {$\Delta$};
				\begin{scope}[rotate=-90]
					\path (-1.8,2.5) coordinate (A);
					\draw[dashed] (-1.8,2.5)--(1.8,2.5);
					\draw (-1.8,2.5)  parabola bend (0,0.01) (1.8,2.5) node[right]{$(P)$};
				\end{scope}
%				\draw (2.5,0)node[above right]{$H$} circle(1pt);
				\foreach \p/\g in {O/-135, F/45, A/0}
				\draw[fill=black] (\p) circle (1pt) node[shift=(\g:3mm)] {$\p$};
		\end{tikzpicture}}	
	}
\end{bt}
%Bài 9
\begin{bt}%[]
	Mặt cắt của gương phản chiếu của một đèn pha có dạng một parabol $(P)$ có phương trình chính tắc $y^2=6x$. Tính khoảng cách từ điểm $M\left(1;\sqrt{6}\right)$ trên gương đến tiêu điểm của $(P)$, (với đơn vị trên hệ trục tọa độ là xentimét).
	\loigiai{
		Khoảng cách từ điểm $M\left(1;\sqrt{6}\right)$ đến tiêu điểm của $(P)$ chình là bán kính qua tiêu của điểm $M$.\\
		Ta có $2p=6\Rightarrow p=3$.\\
		Do đó $MF=x+\dfrac{p}{2}=1+\dfrac{3}{2}=2{,}5~\mathrm{cm}$.
	}
\end{bt}


\Closesolutionfile{ans}