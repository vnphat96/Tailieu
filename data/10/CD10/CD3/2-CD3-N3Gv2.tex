\begin{dang}{Bài toán tìm điểm trên Elip}
	Khi làm bài cần chú ý các tính chất sau của elip:
\begin{enumerate}
	\item Elip nhận hai trục $Ox,Oy$ làm trục đối xứng. 
	\item Tâm sai của elip $e=\dfrac{c}{a}$.
	\item Bán kính qua tiêu của điểm $M(x;y)\in (E)$: $MF_1=a+ex; MF_2=a-ex$.
	\item Đường chuẩn của elip:\\Đường thẳng $d_1: x+\dfrac{a}{e}=0$ được gọi là đường chuẩn của elip, ứng với tiêu điểm $F_1(-c;0)$.\\Đường thẳng $d_2: x-\dfrac{a}{e}=0$ được gọi là đường chuẩn của elip, ứng với tiêu điểm $F_2(c;0).$      
\end{enumerate}
\end{dang}

\subsubsection{Các ví dụ}
\begin{vd}%[0H3B3-3]
Trong mặt phẳng $Oxy$, cho elip $(E)\colon \dfrac{x^2}{8}+\dfrac{y^2}{2}=1$. Tìm tất cả các điểm thuộc elip có tọa độ là số nguyên.
\loigiai
{Gọi $M(x;y)$ là điểm cần tìm.\\
Vì $M\in (E)\colon \dfrac{x^2}{8}+\dfrac{y^2}{2}=1 \Rightarrow x^2\le 8$ nên $|x|$ sẽ nhận các giá trị là $0$, $1$, và $2$.
\begin{enumerate}[$\bullet$]
	\item Nếu $|x|=0\Rightarrow |y|=\sqrt{2}$ (loại).
	\item Nếu $|x|=1\Rightarrow |y|=\dfrac{\sqrt{7}}{2}$ (loại).
	\item Nếu $|x|=2 \Rightarrow |y|=1$ (thỏa mãn).
\end{enumerate}	
Khi đó có tất cả 4 điểm thuộc $(E)$ có tọa độ nguyên là $(2;1)$, $(-2;1)$, $(-2;-1)$, $(2;-1)$.
}
\end{vd}

\begin{vd}%[0H3B3-3]
Trong mặt phẳng $Oxy$, cho eip $(E)\colon \dfrac{x^2}{16}+\dfrac{y^2}{9}=1$ với $F_1$, $F_2$ là hai tiêu cự và hoành độ $F_1$ là số âm. Tìm tất cả điểm $M$ thuộc elip sao cho độ dài $MF_1=4$.
\loigiai{
Gọi $M(x;y)$ là điểm thuộc $(E)\colon \dfrac{x^2}{16}+\dfrac{y^2}{9}=1$. Ta có $a=4,$ $b=3$ và $c=\sqrt{7}$.\\
Khi đó $MF_1 = 4+\dfrac{\sqrt{7}}{4}x$. Mà $MF_1 = 4 \Leftrightarrow 4+\dfrac{\sqrt{7}}{4}x=4 \Leftrightarrow x=0$.\\
Thay vào elip ta tìm được $y=\pm 3$.\\
Vậy có 2 điểm thuộc $(E)$ thỏa mãn là $M_1(0;3)$ và $M_2(0;-3)$.
}
\end{vd}

\begin{vd}%[0H3B3-3]
	Trong mặt phẳng $Oxy$, cho eip $(E)\colon \dfrac{x^2}{16}+\dfrac{y^2}{9}=1$ với $F_1$, $F_2$ là hai tiêu điểm. Tìm tất cả điểm $M$ thuộc elip sao cho $\widehat{F_1MF_2}=60^\circ$.
	\loigiai{
Ta có $a=4,$ $b=3$ và $c=\sqrt{7}$. Gọi $M(x;y)$ là điểm thuộc elip, ta có
$$
MF_1 = 4+\dfrac{\sqrt{7}}{4}x; \qquad MF_2=4-\dfrac{\sqrt{7}}{4}x; \qquad F_1F_2=2\sqrt{7}.
$$
Áp dụng định lý cosin trong tam giác $MF_1F_2$, ta có
$$
\begin{array}{lll}
&F_1F_2^2=MF_1^2+MF_2^2-2\cdot MF_1\cdot MF_2 \cdot \cos \widehat{F_1MF_2} \\
\Leftrightarrow & 28 = 2\left(16+\dfrac{7}{16}x^2\right)-\left(4+\dfrac{\sqrt{7}}{4}x\right)\left(4-\dfrac{\sqrt{7}}{4}x\right) \\
\Leftrightarrow &x^2=\dfrac{64}{7}\Leftrightarrow  x=\pm \dfrac{8}{\sqrt{7}}.
\end{array}
$$
Vậy có 4 điểm thỏa mãn là $\left(\dfrac{8}{\sqrt{7}};\dfrac{3\sqrt{3}}{7}\right)$; $\left(\dfrac{8}{\sqrt{7}};-\dfrac{3\sqrt{3}}{7}\right)$; $\left(-\dfrac{8}{\sqrt{7}};\dfrac{3\sqrt{3}}{7}\right)$; $\left(-\dfrac{8}{\sqrt{7}};-\dfrac{3\sqrt{3}}{7}\right)$.
}
\end{vd}

\subsubsection{Bài tập vận dụng}
\begin{bt}%[0H3B3-3]
Trong mặt phẳng tọa độ $Oxy$, cho elip $(E): \dfrac{x^2}{4}+\dfrac{y^2}{3}=1$. Tìm tất cả các điểm $M$ thuộc elip có tọa độ là số nguyên.
\loigiai{
Gọi $M(x;y)$ là điểm cần tìm.\\
Vì $M\in (E)\colon \dfrac{x^2}{4}+\dfrac{y^2}{3}=1 \Rightarrow x^2\le 4$ nên $|x|$ sẽ nhận các giá trị là $0$, $1$, và $2$.
\begin{enumerate}[$\bullet$]
	\item Nếu $|x|=0\Rightarrow |y|=\sqrt{3}$ (loại).
	\item Nếu $|x|=1\Rightarrow |y|=\dfrac{3}{2}$ (loại).
	\item Nếu $|x|=2 \Rightarrow |y|=0$ (thỏa mãn).
\end{enumerate}	
Khi đó có tất cả 2 điểm thuộc $(E)$ có tọa độ nguyên là $(2;0)$, $(-2;0)$.
}
\end{bt}

\begin{bt}%[0H3B3-3]
Trong mặt phẳng $Oxy$, cho elip $(E)\colon \dfrac{x^2}{9}+\dfrac{y^2}{5}=1$. Tìm điểm $M$ trên elip sao cho $MF_1=2MF_2$ với $F_1$, $F_2$ là hai tiểu điểm (hoành độ của $F_1$ âm).
\loigiai{
Ta có $a=3$, $b=\sqrt{5}$ và $c=2$. Gọi $M(x;y)$ là điểm thuộc elip, ta có
$$
MF_1=3+\dfrac{2}{3}x;\qquad MF_2=3-\dfrac{2}{3}x.
$$
Theo đề bài, ta có
$$
MF_1=2MF_2 \Leftrightarrow 3+\dfrac{2}{3}x=6-\dfrac{4}{3}x \Leftrightarrow 2x=3 \Leftrightarrow x=\dfrac{3}{2}\Rightarrow y=\pm\dfrac{\sqrt{15}}{2}.
$$
Vậy có 2 điểm thuộc $(E)$ thỏa mãn là $\left(\dfrac{3}{2};\dfrac{\sqrt{15}}{2}\right)$; $\left(\dfrac{3}{2};-\dfrac{\sqrt{15}}{2}\right)$.
}
\end{bt}

\begin{bt}%[0H3B3-3]
Trong mặt phẳng $Oxy$, cho eip $(E)\colon \dfrac{x^2}{25}+\dfrac{y^2}{9}=1$ với $F_1$, $F_2$ là hai tiêu điểm. Tìm tất cả điểm $M$ thuộc elip sao cho $\widehat{F_1MF_2}=90^\circ$.
\loigiai{
	Ta có $a=5,$ $b=3$ và $c=4$. Gọi $M(x;y)$ là điểm thuộc elip, ta có
	$$
	MF_1 = 5+\dfrac{4}{5}x; \qquad MF_2=5-\dfrac{4}{5}x; \qquad F_1F_2=8.
	$$
	Áp dụng định lý cosin trong tam giác $MF_1F_2$, ta có
	$$
	\begin{array}{lll}
		&F_1F_2^2=MF_1^2+MF_2^2 \\
		\Leftrightarrow & 64 = 2\left(25+\dfrac{16}{25}x^2\right)\\
		\Leftrightarrow &x^2=\dfrac{175}{16}\Leftrightarrow  x=\pm \dfrac{5\sqrt{7}}{4}.
	\end{array}
	$$
	Vậy có 4 điểm thỏa mãn là $\left(\dfrac{5\sqrt{7}}{4};\dfrac{9}{4}\right)$; $\left(\dfrac{5\sqrt{7}}{4};-\dfrac{9}{4}\right)$; $\left(-\dfrac{5\sqrt{7}}{4};\dfrac{9}{4}\right)$; $\left(-\dfrac{5\sqrt{7}}{4};-\dfrac{9}{4}\right)$.
}
\end{bt}

\begin{dang}{Bài toán thực tế}
	Khi làm bài cần chú ý các tính chất sau của elip:
\begin{enumerate}
	\item Elip nhận hai trục $Ox,Oy$ làm trục đối xứng. 
	\item Tâm sai của elip $e=\dfrac{c}{a}$.
	\item Bán kính qua tiêu của điểm $M(x;y)\in (E)$: $MF_1=a+ex; MF_2=a-ex$.
	\item Đường chuẩn của elip:\\Đường thẳng $d_1: x+\dfrac{a}{e}=0$ được gọi là đường chuẩn của elip, ứng với tiêu điểm $F_1(-c;0)$.\\Đường thẳng $d_2: x-\dfrac{a}{e}=0$ được gọi là đường chuẩn của elip, ứng với tiêu điểm $F_2(c;0).$      
\end{enumerate}	
\end{dang}

\subsubsection{Các ví dụ}
\begin{vd}%[0H3K3-4]
Mặt trăng chuyển động quanh Trái Đất theo quỹ đạo là một elip $(E)$ mà Trái Đất là một tiểu điểm. $(E)$ có độ dài trục lớn và độ dài trục bé lần lượt là (khoảng) $768800$ km và $767619$ km (\textit{Nguồn:} Ron Larson (2014), Precalculus: Real mathematics, Real People, Cengage). Tính khoảng cách ngắn nhất và khoảng cách dài nhất từ Trái Đất đến Mặt Trăng.
\loigiai{
Giả sử $(E)$ có phương trình chính tắc là $\dfrac{x^2}{a^2}+\dfrac{y^2}{b^2}=1$, trong đó
$$
a=768800:2=384400\; (\text{km})\; \text{và}\; b=767619:2=383809,5\; (\text{km})
$$
Ta có $c=\sqrt{a^2-b^2}=\sqrt{384400^2-383809,5^2}\approx 21298,54$ km.\\
Khoảng cách ngắn nhất từ Trái Đất đến Mặt Trăng là
$$
a-c\approx 384400-21298,54=363101,46\; (\text{km}).
$$
Khoảng cách dài nhất từ Trái Đất đến Mặt Trăng là
$$
a+c\approx 384400+21298,54=405698,54\; (\text{km}).
$$
}
\end{vd}

\begin{vd}%[0H3K3-4]
Trái Đất chuyển động quanh Mặt Trời theo một quỹ đạo là đường elip mà Mặt Trời là một tiêu điểm. Biết elip này có bán trục lớn $a\approx 149598261$ km và tâm sai $e\approx 0,017$. Tìm khoảng cách nhỏ nhất và lớn nhất giữa Trái Đất và Mặt Trời (kết quả được làm tròn đến hàng đơn vị).
\loigiai{
Ta có $e=\dfrac{c}{a}\Rightarrow c=a\cdot e \approx 2543170.437$ km.\\
Khoảng cách nhỏ nhất giữa Trái Đất và Mặt Trời là $a-c\approx 147055091$ km.\\
Khoảng cách lớn nhất giữa Trái Đất và Mặt Trời là $a+c\approx 152141431$ km.
}
\end{vd}

\begin{vd}%[0H3K3-4]
Ngày 04/10/1957, Liên Xô đã phóng thành công vệ tinh nhân tạo đầu tiên vào không gian, vệ tinh mang tên Sputnik I. Vệ tinh đó có quỹ đạo hình elip $(E)$ nhìn tâm Trái Đất là một tiêu điểm. Cho biết khoảng cách xa nhất giữa vệ tinh và tâm Trái Đất là $7310$ km và khoảng cách gần nhất giữa vệ tinh và tâm Trái Đất là $6586$ km. Tìm tâm sai của quỹ đạo chuyển động của vệ tinh Sputnik I.
\loigiai{
Theo đề bài, ta có $\heva{&a-c=6586 \\&a+c=7310}\Rightarrow \heva{&a=6948\\&c=362}\Rightarrow e=\dfrac{c}{a}\approx 0,052$.
}
\end{vd}

\subsubsection{Bài tập vận dụng}
\begin{bt}%[0H3K3-4] 
Một phòng thì thầm có trần vòm elip với hai tiêu điểm ở độ cao $1,6$ m (so với mặt đất) và cách nhau $16m$. Đỉnh của mái vòm cao $7.6$ m. Hỏi âm thanh thì thầm từ một tiêu điểm thì sau bao nhiêu giây đến được tiêu điểm kia? Biết vận tốc âm thanh là $343,2$ m/s và làm tròn đáp số tới 4 chữ số sau dấu phẩy.
\loigiai{
\begin{center}
\begin{tikzpicture}[smooth, line join=round, line cap=round, font=\scriptsize,scale=0.8,>=latex]
\path
(0,0) coordinate (A)
(0,-1) coordinate (B)
(12,0) coordinate (D)
($(B)+(D)-(A)$) coordinate (C);
\draw (D) arc (0:180:6 and 3.5) (A)--(B)--(C)--(D)--cycle;
\draw[<->] (-1,-1)--(-1,0);
\draw[<->] (6,-1)--(6,3.5);
\draw (-1,-0.5) node[left] {$1.6$ m} (6,1.5) node[left]{$7.6$ m};
\draw[->] (10,0)--(3,3.031);
\draw[->] (3,3.031)--(2,0);
\end{tikzpicture}
\end{center}
Giả sử phương trình chính tắc của elip này là $\dfrac{x^2}{a^2}+\dfrac{y^2}{b^2}=1$ ($a>b>0$).\\
Dựa vào hình vẽ, ta thấy $2c=16\Rightarrow c=8$.\\
Và $b=7.6-1.6=6\Rightarrow a=\sqrt{b^2+c^2}=10$.\\
Âm thanh đi từ một tiêu điểm qua điểm $M(x;y)$ trên trần vòm rồi đến tiêu điểm kia.\\
 Do đó quãng đường mà âm thanh đã đi được là $MF_1+MF_2=2a=20$ m.\\
Thời gian mà âm thanh đã đi là $\dfrac{20}{343,2}\approx 0,0583$ giây.\\
Vậy âm thanh thì thầm từ một tiêu điểm thì sau khoảng $0,0583$ giây sẽ đến được tiêu điểm kia.
}
\end{bt}

\begin{bt}%[0H3K3-4] 
	\immini
	{
		Ông Hoàng có một mảnh vườn hình Elip có chiều dài trục lớn và trục nhỏ lần lượt là $60$ m và $30$ m. Ông chia mảnh vườn ra làm hai nửa bằng một đường tròn tiếp xúc trong với Elip để làm mục đích sử dụng khác nhau (xem hình vẽ). Nửa bên trong đường tròn ông trồng cây lâu năm, nửa bên ngoài đường tròn ông trồng hoa màu.
	}
	{
		\begin{tikzpicture}[line join=round, line cap=round,thick,scale=0.75]
			\tikzset{label style/.style={font=\footnotesize}}
			\tkzDefPoints{0/0/O}
			\draw (O) circle(1.5cm);
			\draw (O) ellipse (3cm and 1.5cm);
		\end{tikzpicture}
	}
	\noindent Tính tỉ số diện tích $T$ giữa phần trồng cây lâu năm so với diện tích trồng hoa màu. Biết diện tích hình Elip được tính theo công thức $S=\pi ab$ với $a$, $b$ lần lượt là nửa độ dài trục lớn và nửa độ dài trục bé. Biết độ rộng của đường Elip là không đáng kể.
	\loigiai{
		\immini
		{
			Xét hệ trục tọa độ $Oxy$ như \textbf{hình vẽ}.\\
			Khi đó đường tròn có bán kính là $R=15$ và elip có nửa độ dài trục lớn là $a=30$, nửa độ dài trục bé là $b=15$.\\
			Diện tích đường tròn là $S_{(C)}=\pi R^2=\pi \cdot 15^2=225\pi$.\\
			Diện tích Elip là $S_{(E)}=\pi ab=\pi 30\cdot 15=450\pi$.\\
			Diện tích nửa bên ngoài đường tròn trồng hoa màu là\\ $S=S_{(E)}-S_{(C)}=450\pi -225\pi =225\pi$.\\
			Vậy tỉ số diện tích $T=\dfrac{225\pi}{225\pi}=1$.
		}
		{
			\begin{tikzpicture}[scale=0.6,>=stealth, font=\footnotesize, line join=round, line cap=round]
				\tikzset{label style/.style={font=\footnotesize}}
				\draw[->] (-4,0)--(4,0) node [below]{$x$};
				\draw[->] (0,-2)--(0,2.5) node [left]{$y$};
				\draw[fill=black] (0,0) circle(1pt) node at (0,0) [below left]{$O$};
				\draw[fill=black] (3,0) circle(1pt) node at (3,0) [below right]{\footnotesize$ 30$};
				\draw[fill=black] (-3,0) circle(1pt) node at (-3,0) [below left]{\footnotesize$-30$};
				\draw[fill=black] (0,1.5) circle(1pt) node at (0,1.5) [above right]{\footnotesize$15$};
				\draw[fill=black] (0,-1.5) circle(1pt) node at (0,-1.5) [below right]{\footnotesize$-15$};
				\clip (-4,-2) rectangle (4,2.5);
				\draw (O) circle(1.5cm);
				\draw (O) ellipse (3cm and 1.5cm);
			\end{tikzpicture}
		}
	}
\end{bt}

\begin{bt}%[0H3K3-4]
	Ta biết rằng Mặt Trăng chuyển động quanh trái đất theo một quỹ đạo là một elip mà Trái Đất là một tiêu điểm. Elip đó có chiều dài trục lớn và trục nhỏ lần lượt là $769 266$ km và $768 106$ km. Tính khoảng cách lớn nhất từ Trái Đất đến Mặt Trăng (làm tròn đến hàng đơn vị).
	\loigiai{
		Gọi phương trình của elip là $\dfrac{x^2}{a^2} + \dfrac{y^2}{b^2} = 1$.
		\noindent Theo giả thiết, $2a = 769 266 \Rightarrow a = 384 633$ (km); $2b = 768 106 \Rightarrow b = 384 053$ (km).\\
		Vậy $c = \sqrt{a^2 - b^2} = 21115$ (km).
		\noindent Khoảng cách lớn nhất từ Trái Đất đến Mặt Trăng bằng $a + c = 405 748$ (km).
	}
\end{bt}