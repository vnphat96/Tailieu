\section{ÔN TẬP CHUYÊN ĐỀ 3}
\subsection{KIẾN THỨC TRỌNG TÂM}
\subsubsection{Elip}
\begin{itemize}
\item Elip có dạng chính tắc $\dfrac{x^2}{a^2}+\dfrac{y^2}{b^2}=1$ có độ dài trục lớn, độ dài trục nhỏ lần lượt là $2a, 2b$,  $c=\sqrt{a^2-b^2}$ với $F_1(-c;0), F_2(c;0) $ là hai tiêu điểm.
\item Với điểm $M(x;y)$ thuộc elip ta có $MF_1=a+\dfrac{c}{a}x, MF_2=a-\dfrac{c}{a} x$ được gọi là bán kính qua tiêu của $M$, $e=\dfrac{c}{a}$ là tâm sai của elip.
\item Đường thẳng $\Delta_1: x=-\dfrac{a}{e}$ và $\Delta_2: x=\dfrac{a}{e}$ được gọi là các đường chuẩn tương ứng với $F_1$ và $F_2$ của elip.
\item Elip có độ dài trục lớn $2a$, độ dài trục nhỏ $2b$, có phương trình chính tắc $\dfrac{x^2}{a^2}+\dfrac{x^2}{b^2}=1$.
\end{itemize}
\subsubsection{Hypebol}
\begin{itemize}
\item Hypebol có dạng chính tắc $\dfrac{x^2}{a^2}-\dfrac{y^2}{b^2}=1$ có độ dài trục thực, độ dài trục ảo lần lượt là $2a, 2b$,  $c=\sqrt{a^2+b^2}$ với $F_1(-c;0), F_2(c;0) $ là hai tiêu điểm.
\item Với điểm $M(x;y)$ thuộc hypebol ta có $MF_1=a+\dfrac{c}{a}x, MF_2=a-\dfrac{c}{a} x$ được gọi là bán kính qua tiêu của $M$, $e=\dfrac{c}{a}$ là tâm sai của hypebol.
\item Đường thẳng $\Delta_1: x=-\dfrac{a}{e}$ và $\Delta_2: x=\dfrac{a}{e}$ được gọi là các đường chuẩn tương ứng với $F_1$ và $F_2$ của hypebol.
\item Hypebol có độ dài trục thực $2a$ và độ dài trục ảo $2b$ có phương trình chính tắc $\dfrac{x^2}{a^2}-\dfrac{x^2}{b^2}=1$. 
\end{itemize}
\subsubsection{Parabol}
\begin{itemize}
	\item Parabol có phương trình chính tắc $y^2=2px, p>0$ có tiêu điểm $F\left( \dfrac{p}{2};0  \right) $ và đường chuẩn $\Delta: x=-\dfrac{p}{2}$, với mỗi điểm $M(x,y)$ thuộc parabol, đoạn $MF= x+\dfrac{p}{2}$ được gọi là bán kính qua tiêu của $M$, parabol có tâm sai bằng 1.
	\item Parabol có tiêu điểm $F\left( \dfrac{p}{2};0\right) $ có phương trình chính tắc $y^2=2px$, $(p>0)$. 
\end{itemize}
\subsection{CÁC DẠNG BÀI TẬP} 
\begin{dang}{Xác định các yếu tố của đường conic} 
	\begin{itemize}
		\item Elip có dạng chính tắc $\dfrac{x^2}{a^2}+\dfrac{y^2}{b^2}=1$ có độ dài trục lớn, độ dài trục nhỏ lần lượt là $2a, 2b$,  $c=\sqrt{a^2-b^2}$ với $F_1(-c;0), F_2(c;0) $ là hai tiêu điểm. Với điểm $M(x;y)$ thuộc elip ta có $MF_1=a+\dfrac{c}{a}x, MF_2=a-\dfrac{c}{a} x$ được gọi là bán kính qua tiêu của $M$, $e=\dfrac{c}{a}$ là tâm sai của elip, $\Delta_1: x=-\dfrac{a}{e}$ và $\Delta_2: x=\dfrac{a}{e}$ được gọi là các đường chuẩn tương ứng với $F_1$ và $F_2$ của elip.
		\item 	Hypebol có dạng chính tắc $\dfrac{x^2}{a^2}-\dfrac{y^2}{b^2}=1$ có độ dài trục thực, độ dài trục ảo lần lượt là $2a, 2b$,  $c=\sqrt{a^2+b^2}$ với $F_1(-c;0), F_2(c;0) $ là hai tiêu điểm. Với điểm $M(x;y)$ thuộc elip ta có $MF_1=a+\dfrac{c}{a}x, MF_2=a-\dfrac{c}{a} x$ được gọi là bán kính qua tiêu của $M$, $e=\dfrac{c}{a}$ là tâm sai của hypebol, $\Delta_1: x=-\dfrac{a}{e}$ và $\Delta_2: x=\dfrac{a}{e}$ được gọi là các đường chuẩn tương ứng với $F_1$ và $F_2$ của hypebol. 
		\item Parabol có phương trình chính tắc $y^2=2px, p>0$ có tiêu điểm $F\left( \dfrac{p}{2};0  \right) $ và đường chuẩn $\Delta: x=-\dfrac{p}{2}$, với mỗi điểm $M(x,y)$ thuộc parabol, đoạn $MF= x+\dfrac{p}{2}$ được gọi là bán kính qua tiêu của $M$, parabol có tâm sai bằng 1.
	\end{itemize}
\end{dang} 
%-----_VD elip
\begin{vd}
	Cho elip $\dfrac{x^2}{64}+\dfrac{y^2}{39} =1$. Tìm tâm sai, đường chuẩn và các tiêu điểm của elip.
	\loigiai{Ta có $a^2=64$, $b^2=39$. Suy ra $a=8$, $b=\sqrt{39}$ và $c=\sqrt{a^2-b^2}=5$. \\
	Vậy elip có 2 tiêu điểm là $F_1(-5;0)$ và $F_2(5;0)$, tâm sai $e=\dfrac{c}{a} =\dfrac{5}{8}$.\\
	Hai đường chuẩn $\Delta_1: x=-\dfrac{64}{5}, \Delta_2:x=\dfrac{64}{5}$ tương ứng lần lượt với tiêu điểm $F_1$, $F_2$.              }  
\end{vd}
%--------VD hypebol
\begin{vd}
	Tìm tâm sai và đường chuẩn của hypebol $\dfrac{x^2}{64}-\dfrac{y^2}{17} =1$.
	\loigiai{   Ta có $a^2=64, b^2=17$. Suy ra $a=8, b=\sqrt{17}$ và $c=\sqrt{a^2+b^2}=9$. Do đó hypebol có tâm sai $e=\dfrac{c}{a} =\dfrac{9}{8}$, hai đường chuẩn $\Delta_1: x=-\dfrac{64}{9}, \Delta_2:x=\dfrac{64}{9}$ tương ứng với lần lượt tiêu điểm $F_1(-9;0), F_2(9;0)$.                    }
	
\end{vd}
%-------------VD parabol
\begin{vd}
	Cho parabol có phương trình $y^2=4x$.
	\begin{enumerate}[a)]
		\item Tìm tọa độ tiêu điểm parabol và phương trình đường chuẩn.
		\item Tìm bán kính qua tiêu của điểm $M$ thuộc parabol và có hoành độ bằng 3.
	\end{enumerate}
\loigiai{        \begin{enumerate}[a)]
		\item $2p=4\Leftrightarrow p=2$. Do đó parabol có tiêu điểm $F(1;0)$ và đường chuẩn là $x=-1$. 
		\item Bán kính qua tiêu: $MF=3+1=4.$
\end{enumerate}        }
\end{vd}

%----------_BT elip
\begin{bt}
	Cho elip $\dfrac{x^2}{12}+\dfrac{y^2}{4}=1.$
	\begin{enumerate}
		\item Xác định các đỉnh và độ dài các trục của elip.
		\item Xác định tâm sai và các đường chuẩn của elip.
		\item Tìm các bán kính qua tiêu của điểm $M$ huộc elip, biết điểm $M$ có hoành độ $-3$.
	\end{enumerate}
\loigiai{       	\begin{enumerate}
		\item $a^2=12, b^2=4$. Suy ra $a=2\sqrt{3}, b=2$. Các đỉnh của elip là $(-2\sqrt{3};0), (2\sqrt{3};0), (0;-2); (0;2)   $, độ dài trục lớn là $2a=4\sqrt{3}$, độ dài trục nhỏ là $2b=4$.
		\item $c=\sqrt{a^2-b^2}=2\sqrt{2}$. Tâm sai $e=\dfrac{c}{a}=\dfrac{\sqrt{6}}{3}$.\\
		Hai đường chuẩn $\Delta_1: x=-3\sqrt{2}, \Delta_2: x=3\sqrt{2}$. 
		\item Bán kính qua tiêu $MF_1=\sqrt{12}+ \dfrac{\sqrt{6}}{3} \cdot (-3)=-\sqrt{6}+2\sqrt{3} , MF_2 =  \sqrt{6}+2\sqrt{3}.  $
\end{enumerate}        }
\end{bt}
%----BT hypebol
\begin{bt}
	Trong mặt phẳng tọa độ, cho hypebol có phương trình chính tắc $\dfrac{x^2}{9}-\dfrac{y^2}{4}=1$. Xác định tọa độ các đỉnh, độ dài các trục, tâm sai và phương trình các đường chuẩn của hypebol.
	\loigiai{          $a^2=9, b^2=4$. Suy ra $a=3, b=2$. Tọa độ hai đỉnh là $A_1(-3;0)  , A_2(3;0)$. Độ dài trục thực, độ dài trục ảo lần lượt là $6, 4$. Ta có: $c=\sqrt{a^2+b^2}=\sqrt{13}$ nên tâm sai $e=\dfrac{c}{a}=\dfrac{\sqrt{13} }{3}  $. Hai đường chuẩn $\Delta_1:x=-\dfrac{9\sqrt{13}}{3}, \Delta_2:x=\dfrac{9\sqrt{13}}{3}$.      }.
\end{bt}
%------------BT parabol
\begin{bt}
	Cho parabol có phương trình $y^2=12x$. Tìm tiêu điểm và đường chuẩn của parabol. Tìm bán kính qua tiêu của điểm $M$ thuộc parabol có hoành độ bằng 5.
	\loigiai{        $2p=12\Leftrightarrow p=6$.  Do đó parabol có tiêu điểm $F_1(3;0) $ và đường chuẩn là $x=-3$. Bán kính qua tiêu: $MF=5+3=8.$                }
\end{bt}
%----------------------------------- 
\begin{dang}{Viết phương trình chính tắc của các đường conic}

\begin{itemize}
	\item Elip có độ dài trục lớn $2a$, độ dài trục nhỏ $2b$ có phương trình chính tắc $\dfrac{x^2}{a^2}+\dfrac{x^2}{b^2}=1$.
	\item Hypebol có độ dài trục thực $2a$ và độ dài trục ảo $2b$ có phương trình chính tắc $\dfrac{x^2}{a^2}-\dfrac{x^2}{b^2}=1$.
	\item Parabol có tiêu điểm $F\left( \dfrac{p}{2};0\right) $ có phương trình chính tắc $y^2=2px (p>0)$. 
\end{itemize}
\end{dang}	 
\begin{vd}
	Cho elip với độ lớn trục lớn là 20, độ lớn trục nhỏ là 10. Viết phương trình chính tắc của elip.
	\loigiai{  $a=\dfrac{20}{2}=10$, b=$\dfrac{10}{2}=5$. Phương trình chính tắc của elip là $\dfrac{x^2}{100}+\dfrac{y^2}{25}=1$.               }
\end{vd}

\begin{vd}
	Cho hypecbol với độ lớn trục thực là 16, độ lớn trục ảo là 12. Viết phương trình chính tắc của hypebol.
		\loigiai{  $a=\dfrac{16}{2}=8$, $b=\dfrac{12}{2}=6$. Phương trình chính tắc của elip là $\dfrac{x^2}{64}-\dfrac{y^2}{16}=1$.               }
\end{vd}

\begin{vd}
	Lập phương trình chính tắc của parabol có khoảng cách từ đỉnh đến tiêu điểm bằng 3. 
	\loigiai{     Khoảng cách giữa tiêu điểm $F\left(\dfrac{p}{2} ;0\right) $ với đỉnh $O(0;0)$ là 3 nên $\dfrac{p}{2}=3\Rightarrow p=6$.\\
	Vậy parabol có phương trình chính tắc $y^2=12x$.                 }
\end{vd}

\begin{bt}
	Viết phương trình chính tắc của elip trong trường hợp độ dài trục lớn bằng 10 và tiêu cự bằng 6.

\loigiai{  Độ dài trục lớn bằng 10 nên $a=5$, tiêu cự bằng 6 nên $c=3$, $b=\sqrt{a^2-c^2}=\sqrt{5^2 - 3^2} = 4 $. Phương trình chính tắc của elip là $\dfrac{x^2}{25}+\dfrac{y^2}{16}=1.$     }           
\end{bt}

\begin{bt}
	Viết phương trình trình chính tắc của elip biết độ dài trục lớn bằng 8 và tâm sai bằng $\dfrac{\sqrt{3}}{2}$.
	\loigiai{       Độ dài trục lớn bằng 8 nên $a=4$. Tâm sai bằng $\dfrac{\sqrt{3}}{2}$ nên $e=\dfrac{c}{a}\Rightarrow  \dfrac{\sqrt{3}}{2} = \dfrac{c}{4} \Rightarrow c=2\sqrt{3}$, $b=\sqrt{a^2 - c^2 }= \sqrt{4^2 - (2\sqrt{3})^2 }=2$. Phương trình chính tắc của elip là $\dfrac{x^2}{16}+\dfrac{y^2}{4}=1.$            }
\end{bt}
\begin{bt}
	 Trong mặt phẳng tọa độ, hypebol $(H)$ có phương trình chính tắc. Lập phương trình chính tắc của $(H)$ trong mỗi trường hợp sau.
	 \begin{enumerate}[a)]
	 	\item $(H)$ có nửa trục thực 4, tiêu cự bằng 10;
	 	\item $(H)$ có tiêu cự $2\sqrt{13}$, một đường tiệm cận là $y=\dfrac{2}{3}x$.
	 \end{enumerate}
 \loigiai{      \begin{enumerate}
 		\item Hypebol có nửa trục thực bằng 4 nên $a=4$, hypebol có tiêu cự bằng 10 nên $2c=10\Rightarrow c=5\Rightarrow b^2=c^2 - a^2 = 5^2 - 4^2 = 9$. \\
 		Vậy phương trình chính tắc của hypebol đã cho là $\dfrac{x^2}{16}-\dfrac{y^2}{9}=1$.  
 		\item Hypebol có tiêu cự $2\sqrt{13}\Rightarrow 2c=2\sqrt{13}\Rightarrow c=\sqrt{13}$. Hypebol có một đường tiệm cận là $y=\dfrac{2}{3}x\Rightarrow \dfrac{b}{a}=\dfrac{2}{3}\Rightarrow \dfrac{b}{2}=\dfrac{a}{3}\Rightarrow \dfrac{b^2}{4}=\dfrac{a^2}{9}=\dfrac{b^2+a^2}{13}=\dfrac{c^2}{13}=1\Rightarrow a^2 =9$ và $b^2=4$.\\
 		Vậy phương trình chính tắc của hypebol cần tìm là $\dfrac{x^2}{9}-\dfrac{y^2}{4}=1$.
 		\item 
 \end{enumerate}         }
\end{bt}




