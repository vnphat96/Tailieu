\begin{dang}{Bài toán tìm điểm trên các đường conic}
	Từ giả thiết bài toán ta tìm các phương trình, hệ phương trình liên quan tới tọa độ của điểm trên đường conic. Giải các phương trình, hệ phương trình đó để tìm ra tọa độ của điểm trên conic.
\end{dang}
\begin{vd}%[0H3K3-3]%[Đào Trung Kiên]
	Trong mặt phẳng $Oxy$, cho elip $(E): 9x^2+16y^2=144$. Tìm tất cả điểm $M$ thuộc elip sao cho góc $\widehat{F_1MF_2}$ bằng $60^\circ$.
	\loigiai{
		Phương trình chính tắc của elip $(E): \dfrac{x^2}{16}+\dfrac{y^2}{9}=1$. Khi đó $a=4, b=3, c=\sqrt{7}$.
		\\
		Gọi $M(x;y)$ là điểm cần tìm. Ta có
		$$MF_1=4+\dfrac{\sqrt{7}}{4}x;\quad MF_2=4-\dfrac{\sqrt{7}}{4}x;\quad  F_1F_2=2\sqrt{7} $$
		Áp dụng định lí Cô-sin trong tam giác $F_1MF_2$, ta  được
	\begin{eqnarray*}
		& &F_1F_2^2=MF_1^2+MF_2^2-2MF_1.MF_2.\cos \widehat{F_1MF_2}\\
		&\Leftrightarrow & 28 = 2\left(16+\dfrac{7}{16}x^2 \right) -\left( 4+\dfrac{\sqrt{7}}{4}x\right) \left( 4-\dfrac{\sqrt{7}}{4}x\right) \Leftrightarrow x^2 = \dfrac{64}{7}\\
		&\Leftrightarrow&  x= \dfrac{8}{\sqrt{7}} \text{ hoặc }  x=-\dfrac{8}{\sqrt{7}}.
	\end{eqnarray*}
Vậy có $4$ điểm thỏa đề bài là $\left( \dfrac{8}{\sqrt{7}} ; \dfrac{3\sqrt{3}}{\sqrt{7}}\right) ; \left(- \dfrac{8}{\sqrt{7}} ; \dfrac{3\sqrt{3}}{\sqrt{7}}\right) ; \left( \dfrac{8}{\sqrt{7} };- \dfrac{3\sqrt{3}}{\sqrt{7}}\right) ; \left( -\dfrac{8}{\sqrt{7} };- \dfrac{3\sqrt{3}}{\sqrt{7}}\right) $.}
\end{vd}

\begin{vd}%[0H3K3-3]%[Đào Trung Kiên]
	Trong mặt phẳng $Oxy$, cho elip  $(E):\dfrac{x^2}{8}+\dfrac{y^2}{2}=1$. Tìm tất cả các điểm thuộc elip có tọa độ là số nguyên.
	\loigiai{
		Gọi $M(x;y)$ là điểm cần tìm. \\
		Khi đó
		$\dfrac{x^2}{8}+\dfrac{y^2}{2}=1 \Rightarrow |x| \leq 2\sqrt{2}$ nên $|x|$ chỉ có thể nhận các giá trị $0, 1$ hoặc $2$. \\
		Với $|x|=0$ ta có $|y|=\sqrt{2}$ (loại).\\
		Với $|x|=1$ ta có $|y|=\dfrac{\sqrt{7}}{2}$ (loại).\\
		Với $|x|=2$ ta có $|y|=1$.\\
		Vậy tất cả các điểm thuộc $(E)$ có tọa đô nguyên là $(2;1), (2;-1), (-2;1), (-2;-1)$.
	}
\end{vd}

\begin{vd}%[0H3G3-3]%[Đào Trung Kiên]
	Trong mặt phẳng $Oxy$, cho elip $(E): \dfrac{x^2}{9}+\dfrac{y^2}{4}=1$. Gọi $M$ là điểm di động trên elip. Gọi $H, K$ lần lượt là hình chiếu của $M$ lên các trục tọa độ $Ox$ và $Oy$. Tìm tất cả điểm $M$ để tứ giác $OHMK$ có diện tích lớn nhất.
\loigiai{
		Gọi $M(x;y)$ là điểm cần tìm. Dễ thấy tứ giác $OHMK$ là hình chữ nhật. Khi đó,
		$$S_{OHMK}= OH. OK = |x|.|y| = |xy|.$$
		Mặt khác, ta có $$1=\dfrac{x^2}{9}+\dfrac{y^2}{4} \geq 2\sqrt{\dfrac{x^2}{9}.\dfrac{y^2}{4}} = \dfrac{1}{3}|xy|\Rightarrow |xy|\leq3$$
		Dấu bằng xảy ra khi và chỉ khi $\dfrac{x^2}{9}=\dfrac{y^2}{4} = \dfrac{1}{2} \Leftrightarrow x=\pm \dfrac{3\sqrt{2}}{2} \text{ và } y= \pm \sqrt{2}$.
		\\
		Vậy, tứ giác $OHMK$ có diện tích lớn nhất khi $M$ nằm tại một trong các điểm $$\left(\dfrac{3\sqrt{2}}{2}; \sqrt{2} \right); \left(\dfrac{3\sqrt{2}}{2}; -\sqrt{2} \right); \left(-\dfrac{3\sqrt{2}}{2}; \sqrt{2} \right); \left(-\dfrac{3\sqrt{2}}{2}; -\sqrt{2} \right).$$}
\end{vd}

\begin{vd}%[0H3G3-3]%[Đào Trung Kiên]
Cho $(P): y^2=8x$. Tìm các điểm $M$ trên $(P)$ sao cho khoảng cách từ $M$ đến đường chuẩn bằng $3$.
	\loigiai{ Đường chuẩn $(\Delta)$ có phương trình $x=-2$. Điểm $M(x_0; y_0)\in (P)$ thì khoảng cách từ $M$ đến đường chuẩn bằng $|x_0+2|$.\\
			Điểm $M$ cần tìm thỏa mãn $|x_0+2|=3\Leftrightarrow \hoac{&x_0=1\\&x_0=-5 (\text{ loại}).}$.\\
			Từ đó ta có $x_0=1$ suy ra có hai điểm thỏa mãn yêu cầu bài toán là $M_1\left(1; 2\sqrt{2}\right)$ và $M_2\left(1; -2\sqrt{2}\right)$.
			}
\end{vd}


\begin{vd}%[0H3G3-3]%[Đào Trung Kiên]
	Cho $(H)\colon \dfrac{x^2}{16}-\dfrac{y^2}{9}=1$. Tìm điểm $M$ trên $(H)$ sao cho $M$ nhìn $2$ tiêu điểm dưới một góc vuông.
	\loigiai{Từ giả thiết suy ra $c=5$ nên $(H)$ có hai tiêu điểm $F_1(-5; 0)$ và $F_2(5; 0)$.\\
 Với $M(x; y)$ thuộc $(H)$ thì $MF_1=\left|4+\dfrac{5}{4}x\right|$ và $MF_2=\left|4-\dfrac{5}{4}x\right|$.\\
	$M$ nhìn hai tiêu điểm dưới một góc vuông thì theo định lý Pitago ta có $$F_1F_2^2=MF_1^2+MF_2^2\Leftrightarrow 10^2=\left(4+\dfrac{5}{4}x\right)^2+\left(4-\dfrac{5}{4}x\right)^2\Leftrightarrow x=\pm \dfrac{4\sqrt{34}}{5}.$$
	Với $x=\pm\dfrac{4\sqrt{34}}{5}$ suy ra $y=\pm \dfrac{9}{5}$ nên ta có bốn điểm thỏa mãn yêu cầu bài toán là $M_1\left(-\dfrac{4\sqrt{34}}{5};\dfrac{9}{5}\right)$, $M_2\left(-\dfrac{4\sqrt{34}}{5};-\dfrac{9}{5}\right)$, $M_3\left(\dfrac{4\sqrt{34}}{5};\dfrac{9}{5}\right)$ và $M_4\left(\dfrac{4\sqrt{34}}{5};-\dfrac{9}{5}\right)$
	}
\end{vd}



\begin{bt}%[0H3K3-3]%[Đào Trung Kiên]
	Trong mặt phẳng $Oxy$, cho elip $(E):\dfrac{x^2}{16}+\dfrac{y^2}{7}	=1$. Gọi $F_1, F_2$ lần lượt là tiêu điểm bên trái và bên phải của elip. Tìm tất cả điểm $M$ thuộc elip sao cho
	\begin{enumerate}
		\item $MF_1=3$.
		\item $MF_1=3MF_2$.
	\end{enumerate}
	\loigiai{
		Ta có $a=4; b=\sqrt{7}; c=3$. Gọi $M(x;y)$ là điểm cần tìm. Khi đó
		$$MF_1=4+\dfrac{3}{4}x;\quad MF_2=4-\dfrac{3}{4}x.$$
		\begin{enumerate}
			\item Theo đề bài, ta có
			$$MF_1=3\Leftrightarrow 4+\dfrac{3}{4}x=3 \Leftrightarrow x= -\dfrac{4}{3}.$$
			Thay vào phương trình của elip, ta được $y=\pm \dfrac{2\sqrt{14}}{3}$.\\
			Vậy, điểm cần tìm là $\left(-\dfrac{4}{3}; \dfrac{2\sqrt{14}}{3} \right); \left(-\dfrac{4}{3};- \dfrac{2\sqrt{14}}{3} \right)$.
			\item Theo đề bài, ta có
			$$MF_1=3MF_2\Leftrightarrow 4+\dfrac{3}{4}x=3\left(4-\dfrac{3}{4}x\right)\Leftrightarrow x= \dfrac{ 8}{3}.$$
			Thay vào phương trình của elip, ta được $y=\pm \dfrac{\sqrt{35}}{3}$.\\
			Điểm cần tìm là $\left(\dfrac{8}{3}; \dfrac{\sqrt{35}}{3} \right); \left(\dfrac{8}{3}; -\dfrac{\sqrt{35}}{3} \right)$.
		\end{enumerate}
	}
\end{bt}

\begin{bt}%[0H3K3-3]%[Đào Trung Kiên]
	Trong mặt phẳng $Oxy$, cho elip $(E): x^2+5y^2-20=0$, có $F_1, F_2$ lần lượt là tiêu điểm bên trái và bên phải. Tìm tất cả điểm $M$ thuộc elip sao cho:
	\begin{enumerate}
		\item  $\widehat{F_1MF_2} = 90^\circ$.
		\item $\widehat{F_1MF_2} = 120^\circ$.
	\end{enumerate}
	\loigiai{
		Phương trình chính tắc $(E):\dfrac{x^2}{20}+\dfrac{y^2}{4}=1$.
		\\
		Ta có $a=2\sqrt{5}; b=2; c=4; F_1F_2=2c=8$. Gọi $M(x;y)$ là điểm cần tìm. Khi đó
		$$MF_1=2\sqrt{5}+\dfrac{2}{\sqrt{5}}x ; MF_2=2\sqrt{5}-\dfrac{2}{\sqrt{5}}x.$$
		
		\begin{enumerate}
			\item Theo đề bài, ta có $$F_1F_2^2= MF_1^2+MF_2^2  \Leftrightarrow 64=\left(2\sqrt{5}+\dfrac{2}{\sqrt{5}}x  \right)^2 +\left(2\sqrt{5}-\dfrac{2}{\sqrt{5}}x  \right)^2 \Leftrightarrow x^2= 15 \Leftrightarrow x=\pm \sqrt{15}.$$
			Vậy điểm cần tìm là $\left(\sqrt{15};1 \right); \left(\sqrt{15};-1 \right); \left(-\sqrt{15};1 \right); \left(-\sqrt{15};-1 \right)$.
			\item 
			Áp dụng định lí Cô-sin trong tam giác $MF_1F_2$ ta được
			\begin{alignat*}{2}
				& F_1F_2^2= MF_1^2+MF_2^2 - 2MF_1.MF_2.\cos \widehat{F_1MF_2}\\
				\Leftrightarrow\ & 64=\left(2\sqrt{5}+\dfrac{2}{\sqrt{5}}x  \right)^2 +\left(2\sqrt{5}-\dfrac{2}{\sqrt{5}}x  \right)^2 +\left(2\sqrt{5}+\dfrac{2}{\sqrt{5}}x  \right).\left(2\sqrt{5}-\dfrac{2}{\sqrt{5}}x  \right)\\
				\Leftrightarrow\ & x^2= 5 \Leftrightarrow x=\pm \sqrt{5}.
			\end{alignat*}
			
			Vậy điểm cần tìm là $\left(\sqrt{5};\sqrt{3} \right); \left(\sqrt{5};-\sqrt{3} \right); \left(-\sqrt{5};\sqrt{3} \right); \left(-\sqrt{5};-\sqrt{3} \right)$.
	\end{enumerate}}
\end{bt} 


\begin{bt}%[0H3G3-3]%[Đào Trung Kiên]
	Trong mặt phẳng $Oxy$, cho elip $(E): \dfrac{x^2}{9}+\dfrac{y^2}{4}=1$ và điểm $A(-3;0)$. Tìm tất cả các điểm $B,C$ thuộc elip sao cho tam giác $ABC$ nhận điểm $I(-1;0)$ làm tâm đường tròn ngoại tiếp.
	\loigiai{
		Gọi $(C)$ là đường tròn ngoại tiếp $\Delta ABC$, có tâm là $I(-1;0)$ và bán kính $IA= 2$. Khi đó, phương trình đường tròn $(C)$ là $(x+1)^2+y^2=4$.
		Do $B,C$ thuộc elip nên tọa độ của $B,C$ là nghiệm của hệ phương trình
		$$\heva{&(x+1)^2+y^2=4\\& \dfrac{x^2}{9}+\dfrac{y^2}{4}=1} \Leftrightarrow \heva{& x=-3 \text{ hoặc } x=-\dfrac{3}{5}\\& y^2=4-(x+1)^2 }$$
		Với $x=-3$ thì $y=0$. Suy ra $A,B,C$ trùng nhau (loại).\\
		Với $x=-\dfrac{3}{5}$ thì $y=\pm \dfrac{4\sqrt{6}}{5}$. Vậy, tất cả các điểm thỏa bài toán là 
		$$B \left(-\dfrac{3}{5};\dfrac{4\sqrt{6}}{5} \right); C\left(-\dfrac{3}{5};-\dfrac{4\sqrt{6}}{5} \right) \text{ hoặc }B \left(-\dfrac{3}{5};-\dfrac{4\sqrt{6}}{5} \right); C\left(-\dfrac{3}{5};\dfrac{4\sqrt{6}}{5} \right) . $$ 
	}
\end{bt}

\begin{bt}%[0H3G3-3]%[Đào Trung Kiên]
	Cho elip $(E):\dfrac{x^2}{4}+\dfrac{y^2}{1}=1$. Tìm tọa độ các điểm $A$ và $B$ thuộc $(E)$ có hoành độ dương sao cho tam giác $OAB$ cân tại $O$ và có diện tích lớn nhất.
	\loigiai{
		Gọi $A(x;y)$ thì $B(x;-y)$ với $x>0$. Ta có $AB=2|y|=\sqrt{4-x^2}$.\\
		Gọi $H$ là trung điểm của $AB$ thì $OH=x$ suy ra $$S_{OAB}=\dfrac{1}{2}OH.AB=\dfrac{1}{2}x\sqrt{4-x^2}=\dfrac{1}{2}\sqrt{x^2(4-x^2)}\leq 1.	$$
		Đẳng thức xảy ra khi $x=\sqrt{2}$.\\
		Vậy ta có $A\left(\sqrt{2};\dfrac{\sqrt{2}}{2}\right)$, $B\left(\sqrt{2};-\dfrac{\sqrt{2}}{2}\right)$ hoặc $A\left(\sqrt{2};-\dfrac{\sqrt{2}}{2}\right)$, $B\left(\sqrt{2};\dfrac{\sqrt{2}}{2}\right)$
	}
\end{bt}

\begin{bt}%[0H3K3-3]%[Đào Trung Kiên]
	Cho $(H)\colon \dfrac{x^2}{16}-\dfrac{y^2}{9}=1$. Tìm điểm $M$ trên $(H)$ sao cho $M$ nhìn $2$ tiêu điểm dưới một góc $60^{\circ}$.
	\loigiai{$(H)$ có hai tiêu điểm $F_1(-5; 0)$ và $F_2(5; 0)$. Với $M(x; y)$ thuộc $(H)$ thì $MF_1=\left|4+\dfrac{5}{4}x\right|$ và $MF_2=\left|4-\dfrac{5}{4}x\right|$.\\
		$M$ nhìn hai tiêu điểm dưới một góc $60^\circ$ thì theo định lý Cô-sin ta có
		\begin{eqnarray*}
		& &F_1F_2^2=MF_1^2+MF_2^2+MF_1\cdot MF_2\\
		&\Leftrightarrow& 10^2=\left(4+\dfrac{5}{4}x\right)^2+\left(4-\dfrac{5}{4}x\right)^2+\left(4+\dfrac{5}{4}x\right)\left(4-\dfrac{5}{4}x\right)\Leftrightarrow x=\pm \dfrac{8\sqrt{13}}{5}.	
		\end{eqnarray*}		
		Với $x=\pm\dfrac{8\sqrt{13}}{5}$ suy ra $y=\pm \dfrac{9\sqrt{3}}{5}$ nên ta có bốn điểm thỏa mãn yêu cầu bài toán là $M_1\left(-\dfrac{8\sqrt{13}}{5};\dfrac{9\sqrt{3}}{5}\right)$, $M_2\left(-\dfrac{8\sqrt{13}}{5};-\dfrac{9\sqrt{3}}{5}\right)$, $M_3\left(\dfrac{8\sqrt{13}}{5};\dfrac{9\sqrt{3}}{5}\right)$ và $M_4\left(\dfrac{8\sqrt{13}}{5};-\dfrac{9\sqrt{3}}{5}\right)$.
	}
\end{bt}

\begin{dang}{Bài toán thực tế}
Ta đưa các bài toán thực tế về các mô hình toán học sử dụng các đường cônic, sau đó dùng kiến thức liên quan tới đường cônic giải bài toán.
\end{dang}


\begin{vd}%[0H3K3-4]%[Đào Trung Kiên]
Một cổng vòm hình bán elip trên đường có chiều cao $3$ m và chiều rộng $12$ m. Chiếc xe tải có chiều rộng $3$ m và chiều cao $2{,}7$ m. Hỏi xe tải có đi qua được cổng vòm không?
	\loigiai{
		\begin{center}
			\begin{tikzpicture}[line join=round, line cap=round,>=stealth,thick]
				\tikzset{label style/.style={font=\footnotesize}}
				\draw[->] (-7.1,0)--(7.1,0) node[below left] {$x$};
				\draw[->] (0,-1.1)--(0,4.1) node[below left] {$y$};
				\draw (0,0) node [below left] {$O$};
			\foreach \x in {-6,-1.5,1.5,6}
			\draw[thin] (\x,1pt)--(\x,-1pt) node [below] {$\x$};
			\foreach \y in {3}
			\draw[thin] (1pt,\y)--(-1pt,\y) node [above left] {$\y$};				\draw[dashed,thin](1.5,0)--(1.5,2.9)--(0,2.9);
				\draw[dashed,thin](-1.5,0)--(-1.5,2.9)--(0,2.9);
				\begin{scope}
					\clip (-7,-1) rectangle (7,4);
					\draw[samples=200,domain=-6:6,smooth,variable=\x] plot (\x,{sqrt(9-(\x)*(\x)/4)});
				\end{scope}
			\end{tikzpicture}
		\end{center}
	Vì xe tải có chiều rộng $3$ m, để qua được cổng thì chiều cao của cổng tính từ giữa sang hai bên $1.5 m$ phải lớn hơn chiều cao $2{,}7$ m của xe.\\	
	Với $a=6$ và $b=3$ nên elip có phương trình $\dfrac{x^2}{6^2}+\dfrac{y^2}{3^2}=1$.\\	
	Với $x=1{,}5$ ta tìm được $y=\dfrac{\sqrt{135}}{4}\approx 2{,}9$ m.\\
	Vậy xe tải đi qua được cổng đó.
	}
\end{vd}

\begin{vd}%[0H3B3-4]%[Đào Trung Kiên]
Khoảng cách lớn nhất và nhỏ nhất giữa mặt trời và trái đất là $152\cdot 10^6 $km và $94{,}5 \cdot 10^6$ km. Biết rằng trái đất quay quanh mặt trời theo quỹ đạo là đường elip mà mặt trời ở vị trí của một tiêu điểm. Tính khoảng cách từ mặt trời tới tiêu điểm còn lại.
	\loigiai{
	\immini{Giả sử trái đất quay quanh mặt trời theo quỹ đạo hình elip với trục lớn $AA'$, mặt trời ở vị trí tiêu điểm $F_2$.
		$$
		\begin{aligned}
			&AF_2=94{,}5 \cdot 10^6 \mathrm{~km}, F_2A'=152 \cdot 10^6 \mathrm{~km} \\
			&a+c=152 \cdot 10^6 \\
			&a-c=94{,}5 \cdot 10^6
		\end{aligned}
		$$
	Từ đó suy ra $2c=57.5 \cdot 10^6=575 \cdot 10^5 \mathrm{~km}$.\\
	Vậy khoảng cách từ mặt trời tới tiêu điểm còn lại bằng $575 \cdot 10^5$ km.
	}{\begin{tikzpicture}[line join=round, line cap=round,>=stealth,thick,scale=0.8]
\draw[thick] (0,0) ellipse (4 cm and 3 cm);	
			\coordinate (A') at (-4,0);
			\coordinate (A) at (4,0);
			\coordinate (F_2) at (3,0);
			\coordinate (F_1) at (-3,0);
			\draw (A)--(A');
			\foreach \x/\g in {A/0,A'/-180,F_1/-90,F_2/-90}
			\fill[black] 	(\x) circle (2pt)
			($(\g:4mm)+(\x)$) node {$\x$};
			
\end{tikzpicture}}
}
\end{vd}


\begin{vd}%[0H3B3-4]%[Đào Trung Kiên]
	Một trạm ăng ten viễn thông dạng parabol có tiêu điểm cách đỉnh $2$ m. Tìm chiều rộng của ăng ten ở vị trí cách đỉnh $3$ m.
\loigiai{
\immini{Từ giả thiết ta có phương trình của parabol là $y^2=8x$.\\
Khi đó đường thẳng $x=3$ cắt parabol tại hai điểm $A$, $B$ ta có chiều rộng của ăng ten ở vị trí cách đỉnh $3$ m chính bằng $AB$.\\
Ta tính được $AB=4\sqrt{6}$ m.}{\begin{tikzpicture}[line join=round, line cap=round,>=stealth,thick,scale=0.8]
		\tikzset{label style/.style={font=\footnotesize}}
		\draw[->] (-2.1,0)--(6.1,0) node[below left] {$x$};
		\draw[->] (0,-6.1)--(0,6.1) node[below left] {$y$};
		\draw (0,0) node [below left] {$O$};
		\draw[<->, dashed] (3,4.9)--(3,-4.9);
		\draw[thick] (2,0) node[above] {$F$};
		\draw (3,4.9)node[right]{$A$} (3,-4.9) node[right]{$B$};
		\foreach \x in {2, 3}
		\draw[thin] (\x,1pt)--(\x,-1pt) node [below right] {$\x$};
	\begin{scope}
			\clip (-2,-6) rectangle (5,6);
			\draw[samples=200,domain=0:5,smooth,variable=\x] plot (\x,{2*sqrt(2*\x)});
			\draw[samples=200,domain=0:5,smooth,variable=\x] plot (\x,{-2*sqrt(2*\x)});
	\end{scope}
\end{tikzpicture}}
}
\end{vd}

\begin{bt}%[0H3B3-4]%[Đào Trung Kiên]
	Một trạm ăng ten viễn thông dạng parabol có tiêu điểm cách đỉnh $1$ m. Tìm chiều rộng của ăng ten ở vị trí cách đỉnh $3$ m.
	\loigiai{
		\immini{Từ giả thiết ta có phương trình của parabol là $y^2=4x$.\\
			Khi đó đường thẳng $x=3$ cắt parabol tại hai điểm $A$, $B$ ta có chiều rộng của ăng ten ở vị trí cách đỉnh $3$ m chính bằng $AB$.\\
			Ta tính được $AB=4\sqrt{3}$ m.}{\begin{tikzpicture}[line join=round, line cap=round,>=stealth,thick,scale=0.8]
				\tikzset{label style/.style={font=\footnotesize}}
				\draw[->] (-2.1,0)--(5.1,0) node[below left] {$x$};
				\draw[->] (0,-5.1)--(0,5.1) node[below left] {$y$};
				\draw (0,0) node [below left] {$O$};
				\draw[<->, dashed] (3,3.46)--(3,-3.46);
				\draw[thick] (1,0) node[above] {$F$};
				\draw (3,3.46)node[right]{$A$} (3,-3.46) node[right]{$B$};
				\foreach \x in {1, 3}
				\draw[thin] (\x,1pt)--(\x,-1pt) node [below right] {$\x$};
				\begin{scope}
					\clip (-2,-6) rectangle (5,5);
					\draw[samples=200,domain=0:5,smooth,variable=\x] plot (\x,{2*sqrt(\x)});
					\draw[samples=200,domain=0:5,smooth,variable=\x] plot (\x,{-2*sqrt(\x)});
				\end{scope}
		\end{tikzpicture}}
	}
\end{bt}

\begin{bt}%[0H3B3-4]%[Đào Trung Kiên]
	Một vệ tinh quay quanh trái đất theo quỹ đạo là một hình elip, trong đó tâm của trái đất là một tiêu điểm. Khoảng cách lớn nhất và nhỏ nhất giữa vệ tinh đó và trái đất là $2000$ km và $900$ km. Tính tiêu cự elip đó.
	\loigiai{
		\immini{Giả sử vệ tinh quay quanh mặt trời theo quỹ đạo hình elip với trục lớn $AA'$, trái đất ở vị trí tiêu điểm $F_2$.
			$$
			\begin{aligned}
				&AF_2=900 \mathrm{~km}, F_2A'=2000 \mathrm{~km} \\
				&a+c=2000 \\
				&a-c=900
			\end{aligned}
			$$
			Từ đó suy ra $2c=1100 \mathrm{~km}$.\\
			Vậy tiêu cự của quỹ đạo elip bằng $1100$ km.
		}{\begin{tikzpicture}[line join=round, line cap=round,>=stealth,thick,scale=0.8]
				\draw[thick] (0,0) ellipse (4 cm and 3 cm);	
				\coordinate (A') at (-4,0);
				\coordinate (A) at (4,0);
				\coordinate (F_2) at (3,0);
				\coordinate (F_1) at (-3,0);
				\draw (A)--(A');
				\foreach \x/\g in {A/0,A'/-180,F_1/-90,F_2/-90}
				\fill[black] 	(\x) circle (2pt)
				($(\g:4mm)+(\x)$) node {$\x$};
				
		\end{tikzpicture}}
	}
\end{bt}

\begin{bt}%[0H3K3-4]%[Đào Trung Kiên]
	Một cổng vòm hình bán elip trên đường có chiều cao $3m$ m và chiều rộng $8$ m. Hỏi xe tải có chiều ngang $2$ m thì chiều cao bao nhiêu sẽ đi qua được cổng đó?
	\loigiai{
		\begin{center}
			\begin{tikzpicture}[line join=round, line cap=round,>=stealth,thick]
				\tikzset{label style/.style={font=\footnotesize}}
				\draw[->] (-5.1,0)--(5.1,0) node[below left] {$x$};
				\draw[->] (0,-1.1)--(0,4.1) node[below left] {$y$};
				\draw (0,0) node [below left] {$O$};
				\foreach \x in {-4,-1,1,4}
				\draw[thin] (\x,1pt)--(\x,-1pt) node [below] {$\x$};
				\foreach \y in {3}
				\draw[thin] (1pt,\y)--(-1pt,\y) node [above left] {$\y$};				\draw[dashed,thin](1,0)--(1,2.9)--(0,2.9);
				\draw[dashed,thin](-1,0)--(-1,2.9)--(0,2.9);
				\begin{scope}
					\clip (-7,-1) rectangle (7,4);
					\draw[samples=200,domain=-4:4,smooth,variable=\x] plot (\x,{sqrt(9-9*(\x)*(\x)/16)});
				\end{scope}
			\end{tikzpicture}
		\end{center}
		Vì xe tải có chiều rộng $2$ m, để qua được cổng thì chiều cao xe phải nhỏ hơn chiều cao  của cổng tính từ giữa sang hai bên $1$ m.\\	
		Với $a=4$ và $b=3$ nên elip có phương trình $\dfrac{x^2}{4^2}+\dfrac{y^2}{3^2}=1$.\\	
		Với $x=1$ ta tìm được $y=\dfrac{\sqrt{135}}{4}\approx 2{,}9$ m.\\
		Vậy chiều cao của xe tải đi qua được cổng đó phải nhỏ hơn $2{,}9$ m.
	}
\end{bt}

\Closesolutionfile{ans}