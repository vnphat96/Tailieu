\section{BẤT PHƯƠNG TRÌNH BẬC NHẤT HAI ẨN}

\subsection{TÓM TẮT LÝ THUYẾT}
\subsubsection{Bất phương trình bậc nhất hai ẩn}
\begin{itemize}
	\item [\iconMT] Bất phương trình bậc nhất hai ẩn $x$, $y$ có dạng tổng quát là 
		\boxmini{$ax+by\le c \quad (1)$}
	trong đó $a$, $b$, $c$ là những số thực đã cho, $a$ và $b$ không đồng thời bằng $0$, $x$ và $y$ là các ẩn số.
	\item [\iconMT] Nghiệm của bất phương trình là những cặp số $(x_0;y_0)$ thỏa mãn (1).
	\item [\iconMT] Các dạng khác $ax+by<c$; $ax+by\ge c$; $ax+by>c$,...
\end{itemize}

\subsubsection{Biểu diễn tập nghiệm của bất phương trình bậc nhất hai ẩn}
\begin{itemize}
	\item [\iconMT] Các bất phương trình bậc nhất hai ẩn thường có vô số nghiệm. Để mô tả tập nghiệm của chúng, ta sử dụng phương pháp biểu diễn hình học.
	\item [\iconMT] Giả sử muốn biểu diễn miền nghiệm của bất phương trình $ax+by \le c \quad(2)$, ta thực hiện các bước như sau:
	\begin{boxdn}
		\begin{itemize}
			\item [\ding{172}] Trên mặt phẳng tọa độ $Oxy$, vẽ đường thẳng $\Delta \colon ax+by=c$.
			\item [\ding{173}] Lấy một điểm $M_0\left(x_0;y_0\right)$ không thuộc $\Delta$.
			\item [\ding{174}] Thay $(x_0;y_0)$ vào (2), sẽ có một trong hai trường hợp xảy ra: 
			\begin{itemize}
				\item Nếu mệnh đề đúng thì miền nghiệm phải chứa $M_0$. Suy ra nửa mặt phẳng bờ $\Delta$ chứa $M_0$ là miền nghiệm của $(2)$.
				\item Nếu mệnh đề sai thì miền nghiệm không chứa $M_0$. Suy ra nửa mặt phẳng bờ $\Delta$ không chứa $M_0$ là miền nghiệm của $(2)$.
			\end{itemize}
		\end{itemize}
	\end{boxdn}
\end{itemize}
\begin{vidu}
	Biểu diễn hình học tập nghiệm của bất phương trình $2x+y\le 3 \quad(\star)$, ta làm như sau:\\
	\immini{
		\begin{itemize}
			\item [$\bullet$] Vẽ đường thẳng $\Delta\colon 2x+y=3$.
			\item [$\bullet$] Lấy gốc tọa độ $O(0;0)$, ta thấy $O\notin \Delta $. 
			\item [$\bullet$] Thay tọa độ $O$ vào $(\star)$: $2 \cdot 0+0<3$ (thỏa). Suy ra nửa mặt phẳng bờ $\Delta$ chứa gốc tọa độ $O$ là miền nghiệm của bất phương trình đã cho (miền không bị tô đậm trong hình vẽ)
		\end{itemize}
	}
	{
		\begin{tikzpicture}[line join=round, line cap=round, >=stealth,font=\footnotesize, scale=0.8]
			\fill [pattern=north east lines,pattern color=gray] (-0.5,4)--(3,4)--(3,-1) -- (2,-1)--cycle;
			\draw[samples=100,smooth,domain=-0.5:2,red] plot(\x,{-2*(\x)+3})node[left] {$\Delta$};
			\draw[->](-2,0)--(3.2,0) node[below] {$x$};
			\draw[->](0,-1)--(0,4.2) node[right] {$y$};
			\node (0,0) [below left]{$ O $};
			\foreach \x in {-1,...,2}
			\draw[shift={(\x,0)},color=black] (0pt,2pt) -- (0pt,-2pt);
			\foreach \y in {1,...,3}
			\draw[shift={(0,\y)},color=black] (2pt,0pt) -- (-2pt,0pt);
			\draw[fill=black] (0,3) circle(1pt) node[left]{$3$};
			\draw[fill=black] (1.5,0)circle(1pt) node[below left]{\tiny $\dfrac{3}{2}$};
		\end{tikzpicture}
	}
\end{vidu}
\begin{vidu}
	Biểu diễn hình học tập nghiệm của bất phương trình $x-2y> 0 \quad(\star)$, ta làm như sau:\\
	\immini{
		\begin{itemize}
			\item [$\bullet$] Vẽ đường thẳng $\Delta\colon x-2y=0$.
			\item [$\bullet$] Lấy điểm $M(1;0)$, ta thấy $M\notin \Delta $. 
			\item [$\bullet$] Thay tọa độ $M$ vào $(\star)$: $1 \cdot 1-2\cdot0>0$ (thỏa). Suy ra miền nghiệm của bất phương trình là nửa mặt phẳng không bị gạch trong hình vẽ bên và không kể đường thẳng $\Delta$.
		\end{itemize}
	}
	{
		\begin{tikzpicture}[line join=round, line cap=round, >=stealth,font=\footnotesize, scale=0.8]
			\fill [pattern=north west lines,pattern color=gray] (-3,-1.5)--(-3,3)-- (4,3)-- (4,2)--cycle;
			\draw[samples=100,smooth,domain=-3:4,red] plot(\x,{0.5*(\x)})node[below] {$\Delta$};
			\draw[->](-3,0)--(4.5,0) node[below] {$x$};
			\draw[->](0,-2)--(0,3.3) node[right] {$y$};
			\node (0,0) [below right]{$ O $};
			\foreach \x in {-2,...,3}
			\draw[shift={(\x,0)},color=black] (0pt,2pt) -- (0pt,-2pt);
			\foreach \y in {-1,...,2}
			\draw[shift={(0,\y)},color=black] (2pt,0pt) -- (-2pt,0pt);
			\draw[fill=black] (1,0)circle(1pt) node[below]{$M$};
		\end{tikzpicture}
	}
\end{vidu}
\begin{note}
	Miền nghiệm của bất phương trình $ax_0+by_0\leq c$ bỏ đi đường thẳng $ax+by=c$ là miền nghiệm của bất phương trình $ax_0+by_0<c$.
\end{note}
	
\subsection{RÈN LUYỆN KĨ NĂNG GIẢI TOÁN}
\begin{dang}{Nghiệm của bất phương trình bậc nhất hai ẩn}	
\end{dang}
\begin{vd}%[0D4Y4-1]
	Cho bất phương trình $2x-y<0$ . 
	\begin{tasks}(1)
		\task Trong các cặp số $(-1;2)$, $\left(2;0\right)$, $(0;1)$, $\left(3;-2\right)$, $(-1;-2)$, cặp nào là nghiệm của bất phương trình, cặp nào không phải là nghiệm của bất phương trình?
		\task Tìm tất cả giá trị nguyên dương của $m$ để cặp số $(m;8)$ là nghiệm của bất phương trình đã cho. 
	\end{tasks}
	\loigiai{
		Bằng cách thử trực tiếp, các cặp $(-1;2)$, $(0;1)$ là nghiệm, các cặp còn lại không phải là nghiệm của bất phương trình.
	}
\end{vd}

\begin{vd}
	Cho bất phương trình $x+y \le 10$. 
	\begin{tasks}(1)
		\task  Trong các cặp số $(-1;0)$, $\left(10;3\right)$, $(7;3)$, $\left(3;-2\right)$, $(4;7)$, cặp nào là nghiệm của bất phương trình đã cho?
		\task Tìm tất cả giá trị nguyên dương của $m$ để cặp số $(m;4)$ là nghiệm của bất phương trình đã cho.
	\end{tasks}
\end{vd}

\begin{dang}{Biểu diễn miền nghiệm của bất phương trình bậc nhất hai ẩn}	
\end{dang}
\begin{vd}
	Biểu diễn hình học tập nghiệm của bất phương trình bậc nhất hai ẩn $3x + y \ge 3$.
	\loigiai{
		\immini{
			Vẽ đường thẳng $d: 3x + y = 3 $.\\
			Thay tọa độ điểm $O(0;0)$ vào vế trái phương trình đường thẳng $(d)$, ta được: $0 < 3$.\\
			Vậy miền nghiệm của bất phương trình là nửa mặt phẳng không chứa điểm $O$, kể cả bờ $(d)$. (Trên hình là nửa mặt phẳng không bị gạch bỏ).
		}{
			\begin{tikzpicture}[scale=.7]
				%---------- vẽ hệ trục tọa độ
				\draw[->] (-3.25,0)--(3.25,0) node[below right] {$x$};
				\draw[->] (0,-1.75)--(0,4.75) node[right] {$y$};
				\node (0,0) [below left] {$ O $};
				%---------- đoạn chắn trên trục
				\foreach \x in {-3,-2,-1,1,2,3}
				\draw[shift={(\x,0)},color=black] (0pt,2pt) -- (0pt,-2pt);
				\node at (1.2,.5) {$1$};
				\foreach \y in {-1,1,2,3,4}
				\draw[shift={(0,\y)},color=black] (2pt,0pt) -- (-2pt,0pt);
				\node at (.5,2.8) {$-2$};
				
				%---------- vẽ hàm
				\draw [thick, domain=-.5:1.5, samples=100] plot (\x, {-3*\x + 3});
				\node at (2,-1) {$(d)$};
				
				%---------- vẽ miền nghiệm
				\tkzDefPoints{-.5/4.5/A, -3/4.5/B, -3/-1.5/C,1.5/-1.5/D}
				\tkzDrawPolygon[ pattern=north east lines,opacity=.3](A,B,C,D)
			\end{tikzpicture}
	}}
\end{vd}

\begin{vd}
	Biểu diễn hình học tập nghiệm của bất phương trình bậc nhất hai ẩn $2x - 4y < 8$.
	\loigiai{
		\immini{
			Vẽ đường thẳng $d: 2x - 4y =8$.\\
			Thay tọa độ điểm $O(0;0)$ vào vế trái phương trình đường thẳng $(d)$, ta được: $0 < 8$.\\
			Vậy miền nghiệm của bất phương trình là nửa mặt phẳng chứa điểm $O$. (Trên hình là nửa mặt phẳng không bị gạch bỏ).
		}{
			\begin{tikzpicture}[scale=.7]
				%----------------- Vẽ hệ trục tọa độ
				\draw[->] (-2.25,0)--(8.25,0) node[below right] {$x$};
				\draw[->] (0,-3.25)--(0,1.25) node[right] {$y$};
				\node (0,0) [below left] {$ O $};
				%----------------- Vẽ đoạn chắn trên trục
				\foreach \x in {-2,-1,1,2,3,4,5,6,7,8}
				\draw[shift={(\x,0)},color=black] (0pt,2pt) -- (0pt,-2pt);
				\node at (3.8,0.5) {$4$};
				\foreach \y in {-3,-2,-1,1}
				\draw[shift={(0,\y)},color=black] (2pt,0pt) -- (-2pt,0pt);
				\node at (-0.5,-1.8) {$-2$};
				
				%------------- Vẽ hàm
				\draw [thick, domain=-2:6, samples=100] plot (\x, {(1/2)*\x - 2});
				\node at (4.5,.75) {$(d)$};
				
				%---------------- Vẽ miền nghiệm
				\tkzDefPoints{6/1/A, -2/-3/B, 8/-3/C, 8/1/D}
				\tkzDrawPolygon[ pattern=north east lines,opacity=.3](A,B,C,D)
			\end{tikzpicture}
	}}
\end{vd}

\begin{vd}
	Phần nửa mặt phẳng không bị gạch (không kể đường thẳng $d$) ở mỗi hình sau là miền nghiệm của bất phương trình nào?
	\begin{tasks}(3)
		\task \begin{tikzpicture}[line join=round, line cap=round, >=stealth,font=\footnotesize, scale=0.5]
			\fill [pattern=north west lines,pattern color=tsblue] (-3,-2)--(-3,3)-- (0,3)-- (0,-2)--cycle;
			\draw[->](-3,0)--(3,0) node[below] {$x$};
			\draw[->](0,-2)--(0,3) node[right] {$y$};
			\node (0,0) [below right]{$ O $};
			\foreach \x in {-2,...,2}
			\draw[shift={(\x,0)},color=black] (0pt,2pt) -- (0pt,-2pt);
			\foreach \y in {-1,...,2}
			\draw[shift={(0,\y)},color=black] (2pt,0pt) -- (-2pt,0pt);
		\end{tikzpicture}
		\task \begin{tikzpicture}[line join=round, line cap=round, >=stealth,font=\footnotesize, scale=0.5]
			\fill[pattern=north east lines,pattern color=myblue] (-3,2.5)--(-3,-2)-- (4,-2)-- (4,-1)--cycle;
			\draw[samples=100,smooth,domain=4:-3,red] plot(\x,{-0.5*(\x)+1})node[above] {$d$};
			\draw[->](-3,0)--(4.5,0) node[below] {$x$};
			\draw[->](0,-2)--(0,3.3) node[right] {$y$};
			\node (0,0) [below right]{$ O $};
			\foreach \x in {-2,...,3}
			\draw[shift={(\x,0)},color=black] (0pt,2pt) -- (0pt,-2pt);
			\foreach \y in {-1,...,2}
			\draw[shift={(0,\y)},color=black] (2pt,0pt) -- (-2pt,0pt);
			\node[right] at (0,1.2) {$1$};
			\node[above] at (2,0) {$2$};
		\end{tikzpicture}
	\task \begin{tikzpicture}[line join=round, line cap=round, >=stealth,font=\footnotesize, scale=0.5]
		\fill [pattern=north west lines,pattern color=blue!60!green] (-1,-2)--(-3,-2)-- (-3,3)-- (1.5,3)--cycle;
		\draw[samples=100,smooth,domain=-1:1.5,red] plot(\x,{2*(\x)})node[right] {$d$};
		\draw[->](-3,0)--(3,0) node[below] {$x$};
		\draw[->](0,-2)--(0,3.3) node[right] {$y$};
		\node (0,0) [below right]{$O$};
		\foreach \x in {-2,...,2}
		\draw[shift={(\x,0)},color=black] (0pt,2pt) -- (0pt,-2pt);
		\foreach \y in {-1,...,2}
		\draw[shift={(0,\y)},color=black] (2pt,0pt) -- (-2pt,0pt);
		\draw[dashed](1,0)--(1,2)--(0,2);
		\draw[fill=black] (1,2)circle(1.5pt) node[right]{$M$};
	\end{tikzpicture}
	\end{tasks}
\loigiai{
\begin{enumerate}[a)]
	\item Đường giới hạn của miền nghiệm là trục tung có phương trình là $y=0$.\\
	Do miền nghiệm là phần bên phải trục tung nên ta có bất phương trình tương ứng là $y>0$.
	\item Gọi $d \colon y =ax+b$ là đường thẳng giới hạn. Theo hình vẽ thì $d$ qua hai điểm $(0;1)$ và $(2;0)$ nên ta có hệ
	$$\heva{&a \cdot 0 +b=1\\&a \cdot 2 +b=0} \Leftrightarrow \heva{&a=-\dfrac{1}{2}\\&b=1}.$$
	Vậy $d \colon y=-\dfrac{1}{2}x+1 \Leftrightarrow x+2y=2$.\\
	Do điểm $O(0;0)$ không thuộc miền nghiệm nên ta có bất phương trình tương ứng là $x+2y>2$.
	\item Gọi $d \colon y =ax+b$ là đường thẳng giới hạn. Theo hình vẽ thì $d$ qua hai điểm $(0;0)$ và $(1;2)$ nên ta có hệ
	$$\heva{&a \cdot 0 +b=0\\&a \cdot 1 +b=2} \Leftrightarrow \heva{&a=2\\&b=0}.$$
	Vậy $d \colon y=2x \Leftrightarrow 2x-y=0$.\\
	Do điểm $A(2;0)$ thuộc miền nghiệm nên ta có bất phương trình tương ứng là $2x-y>0$.
\end{enumerate}}
\end{vd}
\subsection{VẬN DỤNG, THỰC TIỄN}
\begin{dang}{Các bài toán thực tiễn}
\end{dang}
\begin{vd}
	Một gian hàng trưng bày bàn và ghế rộng $60\,m^2.$ Diện tích để kê một chiếc ghế là $0,5\,m^2$ , một chiếc bàn là $1,2\,m^2$ . Gọi $x$ là số chiếc ghế, $y$ là số chiếc bàn được kê.
	\begin{tasks}(1)
		\task Viết bất phương trình bậc nhất hai ẩn $x,\,y$ cho phần mặt sàn để kê bàn và ghế, biết diện tích mặt sàn dành cho lưu thông tối thiểu là $12\,m^2.$
		\task Chỉ ra ba nghiệm của bất phương trình trên.
	\end{tasks}
\loigiai{
	\begin{enumerate}[a)]
		\item Diện tích kê $x$ chiếc ghế là $0,5x$ m$^2$, ($x\in\mathbb{N^*}$).\\
		Diện tích kê $y$ chiếc ghế là $1,2y$ m$^2$, ($y\in\mathbb{N^*}$).\\
		Diện tích mặt sàn tối đa có thể kê bàn, ghế là $60-12=48$ m$^2$.\\
		Do đó ta có bất phương trình $0,5x+1,2y\le 48$.
		\item Cho $x=10$, ta được $1,2y \le 43$. Có thể chọn 3 giá trị $y_1=10$, $y_2=11$, $y_3=12$ thỏa mãn. Ba nghiệm của bất phương trình trên là $(10;10)$, $(10;11)$, $(10;12)$
	\end{enumerate}
	}
\end{vd}

\begin{vd}%[0D4B4]
	Công ty viễn thông Mobifone tính phí $1$ nghìn đồng mỗi phút gọi nội mạng, $2$ nghìn đồng mỗi phút gọi ngoại mạng. Mỗi tháng Minh gọi điện thoại hết từ $200$ đến $300$ nghìn đồng. Viết bất phương trình bậc nhất hai ẩn mô tả cho số tiền điện thoại trả cho ($x$) phút gọi nội mạng và ($y$) phút gọi ngoại mạng trong một tháng.
	\loigiai{
		Số tiền điện thoại trả cho $x$ phút gọi nội mạng là $x$ nghìn đồng.\\
		Số tiền điện thoại trả cho $y$ phút gọi nội mạng là $2y$ nghìn đồng.\\
		Mỗi tháng Minh gọi điện thoại hết từ $200$ đến $300$ nghìn đồng nên ta có 
		\[200\le x+2y\le 300.\]
	}
\end{vd}

\begin{vd}
	Nhân ngày Quốc tế Thiếu Nhi 1-6, một rạp chiếu phim phục vụ các khán giả một bộ phim hoạt hình. Vé được bán ra có hai loại:
	\begin{itemize}
		\item [$\bullet$] Loại 1 (dành cho trẻ từ 6-13 tuổi): 50 000 đồng/vé.
		\item [$\bullet$] Loại 2 (dành cho người trên 13 tuổi): 100 000 đồng/vé.
	\end{itemize}
	Người ta tính toán rằng, để không phải bù lỗ thì số tiền vé thu được ở rạp chiếu phim này phải đạt tối thiểu 20 triệu đồng. Hỏi số lượng vé bán được trong những trường hợp nào thì rạp chiếu phim phải bù lỗ?
	\loigiai{
	\immini{Gọi $x$ là số lượng vé loại 1 bán được, $y$ là số lượng vé loại 2 bán được $(x,y \in \mathbb{N})$.
	\begin{itemize}
		\item [$\bullet$] Số tiền bán vé thu được là $50x+100y$ (nghìn đồng).
		\item [$\bullet$] Người ta sẽ phải bù lỗ trong trường hợp $50x+100y<20 000$ hay $x+2y<400 \quad (1)$.
	\end{itemize}
	Miền nghiệm của (1) là những điểm có tọa độ nguyên nằm trong tam giác $OAB$. Vậy, nếu bán được $x$ vé loại 1 và $y$ vé loại 2 mà điểm $(x;y)$ nằm trong tam giác $OAB$ thì rạp phim phải bù lỗ.}{
	\begin{tikzpicture}[line join=round, line cap=round, >=stealth,font=\footnotesize, scale=0.5]
		\draw [pattern=dots,pattern color=myblue] (-3,3.5)--(5,3.5)-- (5,-0.5)--cycle;
		\draw[samples=100,smooth,domain=5:-3,red] plot(\x,{-0.5*(\x)+2})node[above] {$d$};
		\draw [color=white] (-3,3.5)--(5,3.5)-- (5,-0.5);
		\draw[->](-3,0)--(5.2,0) node[right] {$x$};
		\draw[->](0,-1)--(0,3.7) node[right] {$y$};
		\node (0,0) [below left]{$ O $};
		\foreach \x in {-2,...,3}
		\draw[shift={(\x,0)},color=black] (0pt,2pt) -- (0pt,-2pt);
		\foreach \y in {1,...,2}
		\draw[shift={(0,\y)},color=black] (2pt,0pt) -- (-2pt,0pt);
		\node[left] at (0,1.8) {$200$};
		\node[below] at (3.8,0) {$400$};
		\node[right] at (0,2.2) {$A$};
		\node[above] at (4.1,0) {$B$};
	\end{tikzpicture}
	}	
}
\end{vd}

\subsection{BÀI TẬP TỰ LUYỆN}

\begin{bt}%[0D4B4-1]
	Biểu diễn miền nghiệm của các bất phương trình sau:
	\begin{enumEX}{2}
		\item[a)] $x-2y-1>0$;
		\item[b)] $x+y-1 \le 0$.
	\end{enumEX}
	\loigiai{
		\begin{enumerate}
			\item[a)] $x-2y-1>0$.
			\immini{
				Vẽ đường thẳng $\Delta \colon x-2y-1=0$ đi qua hai điểm $A(1;0)$ và $B\left(0;-\dfrac{1}{2}\right)$.\\
				Xét gốc tọa độ $O(0;0)$.\\
				Ta thấy $O \notin \Delta$ và $0-2 \cdot 0 -1<0$.\\
				Do đó miền nghiệm của bất phương trình là nửa mặt phẳng không kể bờ $\Delta$, không chứa gốc tọa độ $O$ (miền không gạch chéo trên hình bên).
			}{\vspace{-0.5cm}
				\begin{tikzpicture}[scale=1, font=\footnotesize, line join=round, line cap=round, >=stealth]
					\draw[->] (-3.6,0)--(3.8,0) node [below]{$x$};
					\draw[->] (0,-2)--(0,1.8) node [left]{$y$};
					\node at (0,0) [above left=-3pt]{$O$};
					\clip (-3.6,-2) rectangle (3.8,1.8);
					\fill[pattern=north west lines,opacity=0.6] plot[domain=-3.6:3.8](\x,{1/2*(\x)-1/2})--(3.8,1.8)--(-3.6,1.8)--cycle;
					\draw plot[domain=-3.6:3.8](\x,{1/2*(\x)-1/2})node[shift={(-120:10pt)}]{$\Delta$};
					\draw[fill=black] (1,0)circle(1pt) +(-5:8pt)node{$A$};
					\draw[fill=black] (0,-1/2)circle(1pt) +(-35:8pt)node{$B$};
					\node at (1,0)[shift={(-90:8pt)}]{$1$};
					\node at (0,-1/2)[shift={(-125:13pt)}]{$-\tfrac{1}{2}$};
				\end{tikzpicture}
			}
			\item[b)] $x+y-1\le 0$. 
			\immini{
				Vẽ đường thẳng $\Delta \colon x+y-1=0$ đi qua hai điểm $A(1;0)$ và $B(0;1)$.\\
				Xét gốc tọa độ $O(0;0)$.\\
				Ta thấy $O \notin \Delta $ và $0+0-1<0$.\\
				Do đó miền nghiệm của bất phương trình là nửa mặt phẳng kể cả bờ $\Delta$, chứa gốc tọa độ $O$ (miền không gạch chéo trên hình bên).
			}{\vspace{-0.5cm}
				\begin{tikzpicture}[scale=1, font=\footnotesize, line join=round, line cap=round, >=stealth]
					\draw[->] (-3.6,0)--(3.8,0) node [below]{$x$};
					\draw[->] (0,-2)--(0,1.8) node [left]{$y$};
					\node at (0,0) [below left=-3pt]{$O$};
					\clip (-3.6,-2) rectangle (3.8,1.8);
					\fill[pattern=north east lines,opacity=0.6] plot[domain=-3.6:3.8](\x,{-(\x)+1})--(3.8,1.8)--cycle;
					\draw plot[domain=-3.6:3](\x,{-(\x)+1})node[shift={(153:15pt)}]{$\Delta$};
					\draw[fill=black] (1,0)circle(1pt) +(65:8pt)node{$A$};
					\draw[fill=black] (0,1)circle(1pt) +(15:8pt)node{$B$};
					\node at (1,0)[shift={(-90:8pt)}]{$1$};
					\node at (0,1)[shift={(185:8pt)}]{$1$};
				\end{tikzpicture}
			}
		\end{enumerate}
	}
\end{bt}
\begin{bt}%[0D4B4]
	Biểu diễn hình học tập nghiệm của bất phương trình bậc nhất hai ẩn $3x - y \le 0$.
	\loigiai{
		\immini{
			Vẽ đường thẳng $d: 3x - y = 0 $.\\
			Thay tọa độ điểm $M(0;2)$ vào vế trái phương trình đường thẳng $(d)$, ta được: $-2 < 0$.\\
			Vậy miền nghiệm của bất phương trình là nửa mặt phẳng không chứa điểm $M$, kể cả bờ $(d)$. (Trên hình là nửa mặt phẳng không bị gạch bỏ).
		}{
			\begin{tikzpicture}
				%---------------------- Vẽ hệ trục tọa độ
				\draw[->] (-2.25,0)--(2.25,0) node[below right] {$x$};
				\draw[->] (0,-1.25)--(0,3.25) node[right] {$y$};
				\node (0,0) [above right]{$ O $};
				%----------------------- Vẽ đoạn chắn trên trục
				\foreach \x in {-2,-1,1}
				\draw[shift={(\x,0)},color=black] (0pt,2pt) -- (0pt,-2pt);
				%\node at (3.8,0.5) {$4$};
				\foreach \y in {-1,1,2,3}
				\draw[shift={(0,\y)},color=black] (2pt,0pt) -- (-2pt,0pt);
				%\node at (-0.5,-1.8) {$-2$};
				
				%--------------------- Vẽ hàm
				\draw [thick, domain=-.33:1, samples=100] plot (\x, {3*\x});
				\node at (.5,2.5) {$(d)$};
				
				%---------------------- Điểm M
				\fill (0,2) circle (2pt) node[left]{$M(0;2)$};
				
				%----------------------Vẽ miền nghiệm
				\tkzDefPoints{-.33/-.99/A, 2/-1/B, 2/3/C, 1/3/D}
				\tkzDrawPolygon[ pattern=north east lines,opacity=.3](A,B,C,D)
			\end{tikzpicture}
	}}
\end{bt}

\begin{bt}
	Phần nửa mặt phẳng không bị gạch (không kể đường thẳng $d$) ở mỗi hình sau là miền nghiệm của bất phương trình nào?
	\begin{enumEX}[a)]{3}
		\item \begin{tikzpicture}[line join=round, line cap=round, >=stealth,font=\footnotesize, scale=0.5]
			\fill[pattern=north west lines,pattern color=tsblue] (-3,0)--(-3,-2)-- (3,-2)-- (3,0)--cycle;
			\draw[->](-3,0)--(3,0) node[below] {$x$};
			\draw[->](0,-2)--(0,3) node[right] {$y$};
			\node (0,0) [below right]{$ O $};
			\foreach \x in {-2,...,2}
			\draw[shift={(\x,0)},color=black] (0pt,2pt) -- (0pt,-2pt);
			\foreach \y in {-1,...,2}
			\draw[shift={(0,\y)},color=black] (2pt,0pt) -- (-2pt,0pt);
		\end{tikzpicture}
		\item \begin{tikzpicture}[line join=round, line cap=round, >=stealth,font=\footnotesize, scale=0.5]
			\fill [pattern=north east lines,pattern color=myblue] (-1,3)--(-3,3)-- (-3,-2)-- (4,-2)--cycle;
			\draw[samples=100,smooth,domain=4:-1,red] plot(\x,{-(\x)+2})node[above] {$d$};
			%\draw [color=white] (-1,3)--(-3,3)-- (-3,-2)-- (4,-2);
			\draw[->](-3,0)--(4.5,0) node[below] {$x$};
			\draw[->](0,-2)--(0,3.3) node[right] {$y$};
			\node (0,0) [below right]{$ O $};
			\foreach \x in {-2,...,3}
			\draw[shift={(\x,0)},color=black] (0pt,2pt) -- (0pt,-2pt);
			\foreach \y in {-1,...,2}
			\draw[shift={(0,\y)},color=black] (2pt,0pt) -- (-2pt,0pt);
			\node[right] at (0,2.2) {$2$};
			\node[above] at (2.1,0) {$2$};
		\end{tikzpicture}
		\item \begin{tikzpicture}[line join=round, line cap=round, >=stealth,font=\footnotesize, scale=0.5]
			\fill [pattern=north west lines,pattern color=blue!60!green] (-3,-1.5)--(-3,3)-- (3,3)-- (3,1.5)--cycle;
			\draw[samples=100,smooth,domain=-3:3,red] plot(\x,{0.5*(\x)})node[right] {$d$};
			\draw[->](-3,0)--(3,0) node[below] {$x$};
			\draw[->](0,-2)--(0,3.3) node[right] {$y$};
			\node (0,0) [below right]{$O$};
			\foreach \x in {-2,...,2}
			\draw[shift={(\x,0)},color=black] (0pt,2pt) -- (0pt,-2pt);
			\foreach \y in {-1,...,2}
			\draw[shift={(0,\y)},color=black] (2pt,0pt) -- (-2pt,0pt);
			\draw[dashed](2,0)--(2,1)--(0,1);
			\draw[fill=black] (2,1)circle(1.5pt) node[above]{$M$};
		\end{tikzpicture}
	\end{enumEX}
\begin{enumerate}[a)]
	\item Đường giới hạn của miền nghiệm là trục hoành có phương trình là $x=0$.\\
	Do miền nghiệm là phần bên trên trục hoành nên ta có bất phương trình tương ứng là $x>0$.
	\item Gọi $d \colon y =ax+b$ là đường thẳng giới hạn. Theo hình vẽ thì $d$ qua hai điểm $(0;2)$ và $(2;0)$ nên ta có hệ
	$$\heva{&a \cdot 0 +b=2\\&a \cdot 2 +b=0} \Leftrightarrow \heva{&a=-1\\&b=2}.$$
	Vậy $d \colon y=-x+2 \Leftrightarrow x+y=2$.\\
	Do điểm $O(0;0)$ không thuộc miền nghiệm nên ta có bất phương trình tương ứng là $x+y>2$.
	\item Gọi $d \colon y =ax+b$ là đường thẳng giới hạn. Theo hình vẽ thì $d$ qua hai điểm $(0;0)$ và $(2;1)$ nên ta có hệ
	$$\heva{&a \cdot 0 +b=0\\&a \cdot 2 +b=1} \Leftrightarrow \heva{&a=\dfrac{1}{2}\\&b=0}.$$
	Vậy $d \colon y=\dfrac{1}{2}x \Leftrightarrow x-2y=0$.\\
	Do điểm $A(2;0)$ thuộc miền nghiệm nên ta có bất phương trình tương ứng là $x-2y>0$.
\end{enumerate}
\end{bt}

\begin{bt}%[0D4B4]
	Hà mang $95000$ đồng ra chợ mua hoa cúc và hoa hồng. Một bông hoa cúc có giá $4000$ đồng, một bông hoa hồng có giá $7000$ đồng. Viết bất phương trình bậc nhất hai ẩn cho số tiền mà Hà phải chi để mua $x$ bông hoa cúc và $y$ bông hoa hồng.  
	\loigiai{
		Ta có $x, y\in\mathbb{N}^*$.\\
		Giá của $x$ bông hoa cúc là $4000x$ đồng, giá của $y$ bông hoa hồng là $7000y$ đồng.\\
		Vì số tiền Hà mang đi là $95000$ đồng nên ta có bất phương trình 
		\[4000x+7000y\le 95000\Leftrightarrow 4x+7y\le 95.\] 
	}
\end{bt}

\begin{bt}%[0D4K4]
	Mỗi ngày Nga đều dành không quá $30$ phút để đọc cả $2$ cuốn sách A, B. Nga đọc được $3$ trang sách A trong $2$ phút, đọc được $2$ trang sách B trong $1$ phút. Gọi $x$, $y$ lần lượt là số phút đọc sách A và số phút đọc sách B. Tìm điều kiện của $x$ và $y$ để Nga đọc được ít nhất $35$ trang sách trong một ngày.
	\loigiai{
		Gọi $x$, $y$ lần lượt là số phút đọc sách A và số phút đọc sách B trong một ngày, $x, y>0$. Tổng số phút đọc sách không quá $30$ phút nên $x+y\le 30$.\\
		Số trang sách A đọc được sau $x$ phút là $\dfrac{3x}{2}$.
		Số trang sách B đọc được sau $y$ phút là $2y$.\\
		Nga đọc được ít nhất $35$ trang sách trong một ngày khi và chỉ khi $\dfrac{3x}{2}+2y\ge 35$.\\
		Vậy $x,y$ cần thỏa mãn các điều kiện $\heva{&x,y>0\\&x+y\le30\\&\dfrac{3x}{2}+2y\ge35.}$
	}
\end{bt}

\begin{bt}%[0D4K4]
	Một cửa hàng bán hai loại trà sữa, trong đó $4$ cốc loại $1$ có giá $100000$ đồng, $1$ cốc loại $2$ có giá $30000$ đồng. Muốn có lãi theo dự tính thì mỗi ngày cửa hàng phải bán được ít nhất $5$ triệu đồng tiền hàng. Hỏi số cốc trà sữa bán được trong một ngày trong những trường hợp nào thì cửa hàng có lãi như dự tính?
	\loigiai{
		Gọi $x$, $y$ lần lượt là số cốc trà sữa loại $1$, loại $2$ bán được ($x, y\in\mathbb{N}$).\\
		Tổng số tiền bán trà sữa là $25x+30y$ nghìn đồng.\\
		Cửa hàng có lãi như dự tính trong trường hợp số tiền bán trà sữa thu được trong một ngày không nhỏ hơn $5$ triệu đồng, tức là 
		\[25x+30y\ge 5000.\quad\quad (1)\]
		\immini{
			Miền nghiệm của bất phương trình (1) được xác định như sau\\
			+/ Vẽ đường thẳng $d\colon 25x+30y=5000$.\\
			+/ Chọn gốc tọa độ $O(0;0)$ và tính $25\cdot0+30\cdot0<500$.\\
			Do đó miền nghiệm của bất phương trình (1) là nửa mặt phẳng bờ $d$, không chứa gốc tọa độ $O$, lấy cả đường thẳng $d$.\\
			Gọi $A$, $B$ lần lượt là giao điểm của $d$ và $Ox$, $Oy$. Khi đó, nếu bán được $x$ cốc trà sữa loại $1$ và $y$ cốc trà sữa loại $2$ mà điểm $(x;y)$ nằm ở góc phần tư thứ nhất đồng thời nằm ngoài miền tam giác $OAB$ (có thể nằm trên cạnh $AB$) (phần gạch chéo) thì cửa hàng sẽ có lãi như dự tính.
		}
		{
			\begin{tikzpicture}[scale=.7,>=stealth]
				\draw[->] (0,0) -- (5.3,0)node[below]{$x$};
				\draw[->,color=black] (0,0) -- (0,5.3)node[left]{$y$};
				\node[below left] at (0,0){$O$};
				\node[left] at (0,3){$\dfrac{5000}{3}$};
				\node[above right] at (0,3){B};
				\node[below] at (4,0){$200$};
				\node[above] at (4,0){A};
				\clip(0,0) rectangle (5.3,5.3);
				\fill[pattern=north east lines](4,0)-- (5.3,0) -- (5.3,5.3) -- (0,5.3)--(0,3)-- cycle;
				\draw[line width=1.2pt,smooth,samples=100,domain=0:4] plot(\x,{-0.75 *(\x) +3});
			\end{tikzpicture}
		}
	}
\end{bt}
\begin{bt}%[0D4K4]
	Một rạp chiếu phim $2$D phục vụ khán giả một bộ phim mới với $2$ loại vé khác nhau. Vé loại $1$ (từ thứ $2$ đến thứ $5$) giá $80000$ đồng/vé, vé loại $2$ (từ thứ $6$ đến chủ nhật và ngày lễ) giá $100000$ đồng/vé. Để không phải bù lỗ thì số tiền vé thu được ở rạp chiếu phim này phải đạt tối thiểu $150$ triệu đồng. Hỏi số lượng vé bán được trong những trường hợp nào thì rạp chiếu phim phải bù lỗ?
	\loigiai{
		Gọi $x$, $y$ lần lượt là số vé loại $1$, loại $2$ bán được ($x, y\in\mathbb{N}$).\\
		Tổng số tiền bán vé là $80x+100y$ nghìn đồng.\\
		Rạp chiếu phim phải bù lỗ trong trường hợp số tiền bán vé nhỏ hơn $150$ triệu đồng, tức là 
		\[80x+100y<150000\Leftrightarrow 4x+5y<7500.\quad\quad (1)\]
		\immini{
			Miền nghiệm của bất phương trình (1) được xác định như sau\\
			+/ Vẽ đường thẳng $d\colon 4x+5y=7500$.\\
			+/ Chọn gốc tọa độ $O(0;0)$ và tính $4\cdot0+5\cdot0<7500$.\\
			Do đó miền nghiệm của bất phương trình (1) là nửa mặt phẳng bờ $d$, chứa gốc tọa độ $O$, không kể đường thẳng $d$.\\
			Gọi $A$, $B$ lần lượt là giao điểm của $d$ và $Ox$, $Oy$. Khi đó, nếu bán được $x$ vé loại $1$ và $y$ vé loại $2$ mà điểm $(x;y)$ nằm trong miền tam giác $OAB$ không kể cạnh $AB$ thì rạp chiếu phim sẽ phải bù lỗ.
		}
		{
			\begin{tikzpicture}[scale=.7,>=stealth]
				\draw[->] (0,0) -- (5.3,0)node[below]{$x$};
				\draw[->,color=black] (0,0) -- (0,5.3)node[left]{$y$};
				\node[below left] at (0,0){$O$};
				\node[left] at (0,3){$1500$};
				\node[above right] at (0,3){B};
				\node[below] at (4,0){$1875$};
				\node[above] at (4,0){A};
				\clip(0,0) rectangle (5.3,5.3);
				\fill[pattern=north east lines](4,0)-- (5.3,0) -- (5.3,5.3) -- (0,5.3)--(0,3)-- cycle;
				\draw[line width=1.2pt,smooth,samples=100,domain=0:4] plot(\x,{-0.75 *(\x) +3});
			\end{tikzpicture}
		}	
	}
\end{bt}


\begin{bt}%[0D4K4]
	Một bác nông dân cần trồng lúa và khoai trên diện tích đất $6$ ha, với lượng phân bón dự trữ là $100$ kg và sử dụng tối đa $120$ ngày công. Để trồng $1$ ha lúa cần sử dụng $20$ kg phân bón, $10$ ngày công với lợi nhuận là $30$ triệu đồng; để trồng $1$ ha khoai cần sử dụng $10$ kg phân bón, $30$ ngày công với lợi nhuận là $60$ triệu đồng. Biết bác nông dân đã trồng $x$ (ha) lúa và $y$ (ha) khoai. Tìm giá trị của $x$ để bác nông dân đạt được lợi nhuận cao nhất.
	\loigiai{
		Theo bài toán, ta có:\\
		$ \heva{& x+y=6\\&20x+10y\leq 100\\&10x+30y\leq 120\\&
			T=30x+60y \longrightarrow Max}$ 
		$\Leftrightarrow \heva{& y=6-x\\&x\leq 4\\ &x\geq 3\\& T=24x+360 \longrightarrow Max}$
		$\Leftrightarrow \heva{&y=6-x\\&3\leq x\leq 4\\& T=24x+360 \longrightarrow Max.}$\\
		Vì $T=24x+360$ là hàm số bậc nhất và có hệ số $a=24>0$ nên $T$ đạt GTLN tại $x=4$.\\
		Vậy $x=4$ là giá trị cần tìm.
	}
\end{bt}