\section{ĐỀ TRẮC NGHIỆM ÔN TẬP CUỐI CHƯƠNG}
\Opensolutionfile{ans}[ans/ansTL-CD4]
\setcounter{ex}{0}
\subsubsection{Đề số 1}

\begin{ex}%[0D1Y4-1]
	Sử dụng các kí hiệu ``khoảng'', ``nửa khoảng'' và ``đoạn'' để viết lại tập
	hợp $$A=\left\{x \in \mathbb{R}|4 \leq x \leq 9\right\}.$$
	\choice
	{$A=(4;9]$}
	{\True $A=[4;9]$}
	{$A=[4;9)$}
	{$(4;9)$}
	\loigiai{
		Ta có $A=\left\{x \in \mathbb{R}|4 \leq x \leq 9\right\}$ nên
		$A=[4;9]$.}
\end{ex}

\begin{ex}%[0D1Y1-3]
	Tìm mệnh đề phủ định của mệnh đề sau \lq \lq $\forall x \in \mathbb{R},x^2 \geq 0$ \rq \rq.
	\choice
	{\lq \lq $\forall x \in \mathbb{R},x^2<0$ \rq \rq}
	{\lq \lq $\forall x \in \mathbb{R},x^2>0$ \rq \rq}
	{\True \lq \lq $\exists x \in \mathbb{R},x^2<0$ \rq \rq}
	{\lq \lq $\exists x \in \mathbb{R},x^2 \geq 0$ \rq \rq}
	\loigiai{
		Mệnh đề phủ định cần tìm là \lq \lq $\exists x \in \mathbb{R},x^2<0$ \rq \rq.}
\end{ex}

\begin{ex}%[0D1Y2-2]
	Cho tập hợp $M=\{1;2;3;4;5\}$. Số các tập con của $M$ luôn chứa cả ba phần tử $1$, $3$, $5$ là
	\choice
	{$3$}
	{$2$}
	{\True $4$}
	{$8$}
	\loigiai{
		Các tập con của $M$ luôn chứa cả ba phần tử $1$, $3$, $5$ là $\{1;3;5\}$, $\{1;3;5;2\}$, $\{1;3;5;4\}$, $\{1;3;5;2;4\}$.
	}
\end{ex}

\begin{ex}%[0D1Y1]
	Mệnh đề nào sau đây là đúng?
	\choice
	{\True $\exists n\in\mathbb{N}: n^2=n$}
	{$\forall n\in\mathbb{N}: n^2>0$}
	{$\forall n\in\mathbb{N}: n^2+1$ là số lẻ}
	{$\exists n\in\mathbb{N}: n^2-2=0$}
	\loigiai{
	Xét mệnh đề $\exists n\in\mathbb{N}: n^2=n$. Với $n=1$ thì $1^2=1$ nên đây là mệnh đề đúng.
	}
\end{ex}

\begin{ex}%[0D1B1]
	Trong các câu khẳng định sau, câu nào là mệnh đề \textbf{sai}?
	\choice
	{Tổng $3$ góc trong của một tam giác bằng $180^\circ$}
	{\True Nếu tam giác $ABC$ thỏa mãn $AB^2+AC^2=BC^2$ thì tam giác $ABC$ vuông tại $B$}
	{2 là số nguyên tố}
	{Nếu một phương trình bậc hai có biệt thức $\Delta$ không âm thì nó có nghiệm}
	\loigiai{
		Nếu tam giác $ABC$ thỏa mãn $AB^2+AC^2=BC^2$ thì tam giác $ABC$ vuông tại $A$ nên khẳng định tam giác vuông tại $B$ là sai.
	}
\end{ex}

\begin{ex}%[0D1B3]
	Cho hai tập hợp $A = \left\{ 0, 1, 2,  3, 4, 5 \right\}$ và $B = \left\{-2, 1, 4, 6\right \}$. Tìm tập hợp $A \setminus B$.
	\choice
	{\True $\left\{0, 2, 3, 5\right\}$}
	{ $\left\{0, 1, 2, 3, 4\right\}$}
	{$\left\{1, 4\right\} $}
	{$\left\{-2, 0, 1, 2, 3, 4, 5, 6\right\}$}
	\loigiai{
		Với $A \setminus B$ thì ta lấy những phần tử thuộc $A$ mà không thuộc $B$. Suy ra
		$$A \setminus B=\left\{0, 2, 3, 5\right\}.$$
	}
\end{ex}

\begin{ex}%[0D1B2-1]
	Hỏi tập hợp $A=\{k^2+1\mid k \in \mathbb{Z},|k| \leq 2\}$ có bao nhiêu phần tử?
	\choice
	{\True $3$}
	{$5$}
	{$2$}
	{$1$}
	\loigiai{
		Ta có $|k| \leq 2,k \in \mathbb{Z} \Leftrightarrow k \in \{\pm 1;0;\pm 2\}$.\\
		\begin{listEX}[2]
			\item [$\bullet$] Với $ k= \pm 2$, suy ra $k^2+1=5$.
			\item [$\bullet$] Với $ k= \pm 1$, suy ra $k^2+1=2$.
			\item [$\bullet$] Với $ k= 0$, suy ra $k^2+1=1$.
	\end{listEX}}
Vậy, tập $A$ có 3 phần tử.
\end{ex}

\begin{ex}%[0D1B5]
	Kết quả làm tròn số $x=76324,7533695$ đến hàng phần chục nghìn là
	\choice
	{$x\approx 76324,75337$}
	{$x\approx 76324,75336$}
	{$x\approx 76324,7533$}
	{\True $x\approx 76324,7534$}
	\loigiai{
	}
\end{ex}

\begin{ex}%[0D1B1]
	Cho mệnh đề chứa biến $P(x):``x+15 \le x^2,\;x \in \mathbb{R}"$. Trong các mệnh đề sau, mệnh đề nào đúng?
	\choice
	{$P(3) $}
	{$P(4) $}
	{$P(0) $}
	{\True $P(5) $}
	\loigiai{
	Thử trực tiếp các giá trị của biến
	\begin{listEX}[2]
			\item [$\bullet$] Với $x=3$ thì $3+15 \le 3^2$ (sai).
		\item [$\bullet$] Với $x=4$ thì $4+15 \le 4^2$ (sai).
		\item [$\bullet$] Với $x=0$ thì $0+15 \le 0^2$ (sai).
		\item [$\bullet$] Với $x=5$ thì $5+15 \le 5^2$ (đúng).
	\end{listEX}
Suy ra $P(5)$ đúng.
	}
\end{ex}

\begin{ex}%[0D1B3-2]
	Tập hợp nào sau đây chỉ gồm các số vô tỷ?
	\choice
	{\True $\mathbb{R} \setminus \mathbb{Q}$}
	{$\mathbb{Q} \setminus \mathbb{N}^*$}
	{$\mathbb{Q} \setminus \mathbb{Z}$}
	{$\mathbb{R} \setminus \{0\}$}
	\loigiai{
		Ta có $\mathbb{R} \setminus \mathbb{Q}=I$ là tập hợp các số vô tỷ.}
\end{ex}

\begin{ex}%[0D1B4-1]
	Tập hợp $(-\infty;2] \cap (-6;+\infty)$ bằng tập nào dưới đây?
	\choice
	{\True $(-6;2]$}
	{$(-\infty;+\infty)$}
	{$[-6;2]$}
	{$(-6;2)$}
	\loigiai{
		Ta có $(-\infty;2] \cap (-6;+\infty)=(-6;2]$.
	}
\end{ex}

\begin{ex}%[0D1B3]
	Cho hai tập hợp $A=\left\{x\big| x\in\mathbb{R}\right\}$ và $B=(0;+\infty)$. Tìm tập hợp $A\setminus B$.
	\choice
	{$(0;+\infty)$}
	{$(-\infty;0)$}
	{$[0;+\infty)$}
	{\True $(-\infty;0]$}
	\loigiai{
	}
\end{ex}

\begin{ex}%[0D1B2]
	Tập hợp $A=\left\{x\in \mathbb{R}\big| 0<x<2 \right\}$ bằng tập hợp nào dưới đây?
	\choice
	{$[0; 2]$}
	{$\left\{0; 2\right\}$}
	{$(0; 2]$}
	{\True $(0; 2)$}
	\loigiai{
	}
\end{ex}

\begin{ex}%[0D1B1-2]
	Trong các mệnh đề sau, mệnh đề nào \textbf{sai}?
	\choice
	{Nếu hai tam giác bằng nhau thì hai tam giác đó đồng dạng}
	{Nếu hai tam giác bằng nhau thì bán kính đường tròn ngoại tiếp của hai tam giác đó bằng nhau}
	{Nếu hai tam giác bằng nhau thì hai tam giác đó diện tích bằng nhau}
	{\True Nếu hai tam giác có bán kính đường tròn ngoại tiếp bằng nhau thì hai tam giác đó bằng nhau}
	\loigiai{
		Có thể lấy ví dụ về hai tam giác vuông cùng có cạnh huyền bằng $a$ nhưng các cạnh góc vuông không bằng nhau, suy ra nhận xét Nếu hai tam giác có bán kính đường tròn ngoại tiếp bằng nhau thì hai tam giác đó bằng nhau là sai.}
\end{ex}

\begin{ex}%[0D1B3-1]
	Cho tập hợp $X=(-\infty;2] \cap (-6;+\infty)$. Khẳng định nào sau đây là đúng?
	\choice
	{\True $X=(-6;2]$}
	{$X=(-\infty;+\infty)$}
	{$(-6;+\infty)$}
	{$X=(-\infty;2]$}
	\loigiai{
		Ta có: $X=(-\infty;2] \cap (-6;+\infty)=(-6;2]$.}
\end{ex}

\begin{ex}%[0D1B4]
	Cho các tập hợp $A=(-2;15)$ và  $B=(3;+ \infty )$. Khi đó $A \cup B$ là tập hợp nào sau đây?
	\choice
	{$[15;+\infty ) $}
	{$(-2;3] $}
	{\True $(-2;+ \infty ) $}
	{$(3;15) $}
	\loigiai{
	}
\end{ex}

\begin{ex}%[0D1B2]
	Hãy viết tập hợp $A=\left\{x\in \mathbb{R}\big| 2x^2-3x+1=0\right\}$ dưới dạng liệt kê các phần tử.
	\choice
	{$A=\left\{\dfrac{1}{2}\right\}$}
	{\True $A=\left\{1; \dfrac{1}{2}\right\}$}
	{$A=\left(\dfrac{1}{2}; 1\right)$}
	{$A=\left\{-1; \dfrac{1}{2}\right\}$}
	\loigiai{
		Xét $2x^2-3x+1=0 \Leftrightarrow x=1$ hoặc $x=\dfrac{1}{2}$. Vậy $A=\left\{1; \dfrac{1}{2}\right\}$.
	}
\end{ex}

\begin{ex}%[0D1K4]
	Cho 3 tập hợp $A=(-\infty;1]$, $B=[-2;2]$ và $C=(0;5)$. Tìm tập hợp $P=(A\cap B)\cup (A\cap C)$.
	\choice
	{$P=\left[1;2\right] $}
	{$P= \left(-2;5\right)$}
	{\True$P=\left[-2;1\right]$}
	{$ P=\left(0;1\right]$}
	\loigiai{
	}
\end{ex}

\begin{ex}%[0D1K4]
	Cho các tập hợp $A=(-3;3), B=(-2;+\infty )$ và $C=\left( -\infty; \dfrac{1}{2} \right)$. Khi đó tập hợp $A\cap B \cap C$ là
	\choice
	{$ \bigg\lbrace x \in \mathbb{R} \big| -2<x\le \dfrac{1}{2} \bigg\rbrace$}
	{\True $\bigg\lbrace x \in \mathbb{R} \big| -2<x<\dfrac{1}{2} \bigg\rbrace $}
	{$\bigg\lbrace x \in \mathbb{R} \big| -3<x<\dfrac{1}{2} \bigg\rbrace $}
	{$\bigg\lbrace x \in \mathbb{R} \big| -2\le x\le \dfrac{1}{2} \bigg\rbrace $}
	\loigiai{
	}
\end{ex}

\begin{ex}%[0D1K1]
	Trong các mệnh đề sau, mệnh đề nào \textbf{sai} ?
	\choice
	{ $\sqrt{23}<5\Rightarrow -2\sqrt{23}>-2\cdot5$}
	{ \True $-\pi <-2\Leftrightarrow {{\pi }^2}<4$}
	{ $\sqrt{23}<5\Rightarrow 2\sqrt{23}<2\cdot5$}
	{ $\pi <4\Leftrightarrow {{\pi }^2}<16$}
	\loigiai{
	}
\end{ex}

\begin{ex}%[0D1K2]
	Tìm mệnh đề phủ định của mệnh đề
	$\exists x \in \mathbb{R},\ \exists y \in \mathbb{R}  $ : $x^2-y^2 > 10^{1000}$.
	\choice
	{$\forall x \in \mathbb{R},\ \forall y \in \mathbb{R}  $ : $x^2-y^2  <  10^{1000}$}
	{$\forall x \in \mathbb{R},\ \forall y \in \mathbb{R}  $ : $x^2-y^2 > 10^{1000}$}
	{\True $\forall x \in \mathbb{R},\ \forall y \in \mathbb{R}  $ : $x^2-y^2  \leq  10^{1000}$}
	{$\exists x \in \mathbb{R},\ \exists y \in \mathbb{R}  $ : $x^2-y^2 < 10^{1000}$}
	\loigiai{
	}
\end{ex}

\begin{ex}%[0D1K3]
	Cho hai tập hợp $A=\left\{ x\in \mathbb{R}\big|\left(x^2-1 \right)\left( x^2-3x-4 \right)=0 \right\}$ và $B=\left\{ x\in \mathbb{Z}\big|\left| x \right|\leq 2 \right\}$. Tìm tập hợp $A\cup B$.
	\choice
	{\True $\left\{ -2,-1,0,1,2, 4\right\}$}
	{$\left\{-1, 1 \right\}$}
	{$\left\{ -2,-1,0,1,2\right\}$}
	{$\left\{-2, 0, 2\right\}$}
	\loigiai{
		Xét 
		\begin{itemize}
			\item [$\bullet$] $\left(x^2-1 \right)\left( x^2-3x-4 \right)=0 \Leftrightarrow \hoac{& x^2-1=0\\& x^2-3x-4=0}  \Leftrightarrow \hoac{& x= \pm 1\\& x=-1,\, x=4}$. Suy ra $A=\{-1;1;4\}$.
			\item [$\bullet$] $x\in \mathbb{Z}$ và $\left| x \right|\leq 2$ thì $x \in \{\pm 2; \pm 1;0\}$. Suy ra $B=\{-2;-1;0;1;2\}$.
		\end{itemize}
		Khi đó $A\cup B= \left\{ -2,-1,0,1,2, 4 \right\}$.
	}
\end{ex}

\begin{ex}%[0D1G3]
	Một đơn vị thiên văn xấp xỉ bằng $ 1,496.10^8 $ km. Một trạm vũ trụ di chuyển với vận tốc trung bình là $ 15000 $ m/s. Hỏi trạm vũ trụ đó phải mất xấp xỉ bao nhiêu giờ (làm tròn đến hàng đơn vị) mới đi được một đơn vị thiên văn?
	\choice
	{ $ 277$ h}
	{ $3  $ h}
	{ $ 9977300$ h}
	{\True $ 2771 $ h}
	\loigiai{
		Một đơn vị thiên văn xấp xỉ bằng :
		$ 1,496.10^8$ $km$=$1,496.10^{11} $ m\\
		Vận tốc trung bình của một trạm vũ trụ là : $ 1,5.10^4 $ m/s\\
		Số giờ trạm vũ trụ đi hết một đơn vị thiên văn là : $ \dfrac{1,496.10^{11}}{1,5.10^4\times 3600}\approx 2771 $ h.
	}
\end{ex}

\begin{ex}%[0D1B4-1]
	Cho $A=(-\infty;2m-7)$ và $B=(13m+1;+\infty)$. Số nguyên $m$ nhỏ nhất thỏa mãn $A \cap B=\varnothing$ là
	\choice
	{$2$}
	{$-1$}
	{\True $0$}
	{$1$}
	\loigiai{
		Ta có $A \cap B=\varnothing \Leftrightarrow 2m-7 \leq 13m+1 \Leftrightarrow m \geq -\dfrac{8}{11}$.\\
		Trong các tham số $m \geq -\dfrac{8}{11}$ thì $m_0=0$ là số nguyên nhỏ nhất thoả mãn $A \cap B=\varnothing$.}
\end{ex}

\begin{ex}%[0D1Y3-1]
	Cho hai tập khác rỗng $A=(m-1;4]$, $B=(-2;2m+2)$ với $m \in \mathbb{R}$. Xác định $m$ để $A \cap B \ne \varnothing$.
	\choice
	{\True $-2<m<5$}
	{$m<5$}
	{$m>-3$}
	{$-3<m<5$}
	\loigiai{
		$A$ và $B$ là hai tập hợp khác rỗng nên $$\heva{&m-1<4 \\& 2m+2>-2}\Leftrightarrow \heva{&m<5 \\& m>-2}\Leftrightarrow -2<m<5 \quad (1).$$
		Để $A \cap B \ne \varnothing$ thì $$m-1<2m+2 \Leftrightarrow m>-3 \quad (2).$$
		Từ (1) và (2) suy ra $-2<m<5$.}
\end{ex}
\centerline{\textbf{---HẾT---}}
\Closesolutionfile{ans}
