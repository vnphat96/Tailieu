%Câu 1
\begin{ex}
	Trong các câu sau, có bao nhiêu câu là mệnh đề:
    \begin{tasks}(2)
        \task $2+3=6$.
        \task $7-3+6>8$.
        \task Bạn đang đi đâu đấy?
        \task $2$ là một số lẻ.
        \task $2+x=8$.
    \end{tasks}
	
	\choice
	{$1$}
	{$2$}
	{\True $3$}
	{$4$}
	\loigiai{
		Câu a, b, d là một mệnh đề.\\
		Câu c là một câu hỏi nên không là mệnh đề.\\
		Câu e là mệnh đề chứa biến
	}
\end{ex}
%Câu 2
\begin{ex}
	Trong các phát biểu sau, đâu là mệnh đề chứa biến:
	\choice
	{$x^2 \ge 0$ với $x\in \mathbb{R}$}
	{\True $2x^2-3x+1=0$ với $x\in \mathbb{R}$}
	{$4+x^2<0$ với $x\in \mathbb{R}$}
	{$3+4=7$}
	\loigiai{
		Phương án A và D là các mệnh đề đúng.\\
		Phương án C là một mệnh đề sai.\\
		Phương án B là một mệnh đề chứa biến
	}
\end{ex}
%Câu 3
\begin{ex}
	Trong các mệnh đề sau, đâu là mệnh đề đúng:
	\choice
	{\True $\forall x\in \mathbb{R},x>3\Rightarrow x^2>9$}
	{$\forall x\in \mathbb{R},x> -3\Rightarrow x^2>9$}
	{$\forall x\in \mathbb{R},x^2>9\Rightarrow x>3$}
	{$\forall x\in \mathbb{R},x^2>9\Rightarrow x>-3$}
	\loigiai{
		Mệnh đề đúng là $\forall x\in \mathbb{R},x>3\Rightarrow x^2>9$
	}
\end{ex}
%Câu 4
\begin{ex}
	Trong các mệnh đề sau mệnh đề nào đúng?
	\choice
	{$\dfrac{3}{2}$ là số nguyên}
	{$2$ là số chính phương}
	{\True $2$ là số nguyên tố}
	{$2023$ chia hết cho $3$}
	\loigiai{
		Số $2$ là số tự nhiện lớn hơn $1$ chỉ có một ước lớn hơn 1 là chính nó nên $2$ là số nguyên tố
	}
\end{ex}
%Câu 5
\begin{ex}
	Tìm mệnh đề phủ định của mệnh đề sau: $100$ là số chẵn.
	\choice
	{$100$ có phải là số chẵn không?}
	{$100$ là số chính phương}
	{\True $100$ không phải là số chẵn}
	{$100$ là số nguyên tố}
	\loigiai{
		Mệnh đề phủ định của mệnh đề đã cho là: $100$ không phải là số chẵn
	}
\end{ex}
%Câu 6
\begin{ex}
	Tìm mệnh đề phủ định của mệnh đề sau: $\exists x\in \mathbb{N}\colon 1-x^2\ge 0$.
	\choice
	{$\forall x\in \mathbb{N}\colon 1-x^2\le 0$}
	{$\exists x\in \mathbb{N}\colon 1-x^2<0$}
	{$\forall x\in \mathbb{N}\colon 1-x^2\ge 0$}
	{\True $\forall x\in \mathbb{N}\colon 1-x^2<0$}
	\loigiai{
		Ta có: phủ định của mệnh đề \lq\lq$\exists x\in X$, $P(x)$ " là mệnh đề \lq\lq $\forall x\in X,\overline{P(x)}$\rq\rq.\\
		Nên mệnh đề phủ định của mệnh đề đã cho là: \lq\lq$\forall x\in \mathbb{N} & \colon 1-x^2<0$\rq\rq
	}
\end{ex}
%Câu 7
\begin{ex}
	Cho mệnh đề $P$:\lq\lq Hai số nguyên chia hết cho $7$\rq\rq và mệnh đề $Q$:\lq\lq Tổng của chúng chia hết cho $7$\rq\rq. Phát biểu mệnh đề $P\Rightarrow Q$.
	\choice
	{Nếu hai số nguyên chia hết cho $7$ thì tổng của chúng không chia hết cho $7$}
	{\True Nếu hai số nguyên chia hết cho $7$ thì tổng của chúng chia hết cho $7$}
	{Nếu hai số nguyên không chia hết cho $7$ thì tổng của chúng không chia hết cho $7$}
	{Nếu tổng của hai số nguyên chia hết cho $7$ thì hai số nguyên đó chia hết cho $7$}
	\loigiai{
		Mệnh đề $P$:\lq\lq Hai số nguyên chia hết cho $7$\rq\rq.\\
		Mệnh đề $Q$:\lq\lq Tổng của chúng chia hết cho $7$\rq\rq.\\
		Mệnh đề $P\Rightarrow Q$ có dạng: \lq\lq Nếu $P$ thì $Q$\rq\rq.\\
		Vậy mệnh đề $P\Rightarrow Q$: \lq\lq Nếu hai số nguyên chia hết cho $7$ thì tổng của chúng chia hết cho $7$\rq\rq
	}
\end{ex}
%Câu 8
\begin{ex}
	Cho số tự nhiên $n$. Xét mệnh đề: \lq\lq Nếu số tự nhiên $n$ có chữ số tận cùng bằng $5$ thì $n$ chia hết cho $5$\rq\rq. Mệnh đề đảo của mệnh đề đó là
	\choice
	{Nếu số tự nhiên $n$ có chữ số tận cùng bằng $5$ thì $n$ không chia hết cho $5$}
	{Nếu số tự nhiên $n$ chia hết cho $5$ thì $n$ không có chữ số tận cùng bằng $5$}
	{Nếu số tự nhiên $n$ không chia hết cho $5$ thì $n$ có chữ số tận cùng bằng $5$}
	{\True Nếu số tự nhiên $n$ chia hết cho $5$ thì $n$ có chữ số tận cùng bằng $5$}
	\loigiai{
		Đặt mệnh đề $P$:\lq\lq Số tự nhiên $n$ có chữ số tận cùng bằng $5$\rq\rq.\\
		Mệnh đề $Q$: \lq\lq Số tự nhiên $n$ chia hết cho $5$\rq\rq.\\
		Mệnh đề: \lq\lq Nếu số tự nhiên $n$ có chữ số tận cùng bằng $5$ thì $n$ chia hết cho $5$\rq\rq có dạng $P\Rightarrow Q$ nên mệnh đề đảo của nó có dạng $Q\Rightarrow P$: \lq\lq Nếu số tự nhiên $n$ chia hết cho $5$ thì $n$ có chữ số tận cùng bằng $5$\rq\rq
	}
\end{ex}
%Câu 9
\begin{ex}
	Cho tam giác $ABC$. Xét mệnh đề $P$:\lq\lq Tam giác $ABC$ cân và có một góc bằng $60^\circ $\rq\rq và mệnh đề $Q$:\lq\lq Tam giác $ABC$ đều\rq\rq. Cách phát biểu nào sau đây không thể dùng để phát biểu mệnh đề $P\Leftrightarrow Q$?
	\choice
	{Tam giác $ABC$ cân và có một góc bằng $60^\circ $ tương đương tam giác $ABC$ đều}
	{Tam giác $ABC$ cân và có một góc bằng $60^\circ $ khi và chỉ khi tam giác $ABC$ đều}
	{Tam giác $ABC$ cân và có một góc bằng $60^\circ $ nếu và chỉ nếu tam giác $ABC$ đều}
	{\True Tam giác $ABC$ cân và có một góc bằng $60^\circ $ là điều kiện đủ để tam giác $ABC$ đều}
	\loigiai{
		Mệnh đề $P\Leftrightarrow Q$ có thể phát biểu ở những dạng sau:\\
		1. $P$ tương đương $Q$.\\
		2. $P$ khi và chỉ khi $Q$.\\
		3. $P$ nếu và chỉ nếu $Q$.\\
		4. $P$ là điều kiện cần và đủ để có $Q$.\\
		Vậy cách phát biểu ở phương án $D$ không dùng để phát biểu mệnh đề $P\Leftrightarrow Q$
	}
\end{ex}
%Câu 10
\begin{ex}
	Mệnh đề: $''\exists x\in \mathbb{R}\colon x^2=2022''$ khẳng định rằng
	\choice
	{Bình phương của mọi số thực bằng $2022$}
	{\True Có ít nhất một số thực mà bình phương của nó bằng $2022$}
	{Chỉ có một số thực bình phương bằng $2022$}
	{Nếu ${x}$ là số thực thì $x^2=2022$}
	\loigiai{
		Mệnh đề \lq\lq $''\exists x\in \mathbb{R}\colon x^2=2022''$ khẳng định rằng: \lq\lq Có ít nhất một số thực mà bình phương của nó bằng $2022$.\rq\rq
	}
\end{ex}
%Câu 11
\begin{ex}
	Mệnh đề \lq\lq $\forall n\in \mathbb{N},\exists x\in \mathbb{Z}\colon x^n\, \vdots\, 3$\rq\rq có nghĩa là
	\choice
	{Tồn tại số tự nhiên $n$ sao cho với mọi số nguyên $x$ luôn thỏa mãn $x^n$ chia hết cho 3}
	{Với mọi số tự nhiên $n$ luôn tồn tại \mathrm{\,d}uy nhất số nguyên $x$ thỏa mãn $x^n$ chia hết cho 3}
	{\True Với mọi số tự nhiên $n$ luôn tồn tại số nguyên $x$ thỏa mãn $x^n$ chia hết cho 3}
	{Tồn tại \mathrm{\,d}uy nhất số tự nhiên $n$ sao cho với mọi số nguyên $x$ luôn thỏa mãn $x^n$ chia hết cho 3}
	\loigiai{
		Mệnh đề \lq\lq$\forall n\in \mathbb{N},\exists x\in \mathbb{Z}\colon x^n\, \vdots\, 3$\rq\rq có nghĩa là với mọi số tự nhiên $n$ luôn tồn tại số nguyên $x$ thỏa mãn $x^n$ chia hết cho 3
	}
\end{ex}
%Câu 12
\begin{ex}
	Gọi $S$ là tập nghiệm của phương trình $(x+2)(2x-1)(x-3)=0$. Khẳng định nào sau đây sai?
	\choice
	{$-2\in S$. B. $3\in S$}
	{\True $2\in S$}
	{$\dfrac{1}{2}\in S$}
	\loigiai{
		Ta có $(x+2)(2x-1)(x-3)=0\Leftrightarrow \hoac{& x=-2 \\& x=\dfrac{1}{2} \\& x=3}$, suy ra $S=\left\{ -2;\dfrac{1}{2};3 \right\}$.\\
		Vậy $2\notin S$
	}
\end{ex}
%Câu 13
\begin{ex}
	Cho tập hợp $A=\left\{ x\in \mathbb{N}\mid (2x+6)(x-3)=0 \right\}$. Số phần tử của tập hợp $A$ là
	\choice
	{$0$}
	{\True $1$}
	{$3$}
	{$2$}
	\loigiai{
		Ta có $(2x+6)(x-3)=0\Leftrightarrow \hoac{& 2x+6=0 \\& x-3=0}\Leftrightarrow \hoac{& x=-3\notin \mathbb{N} \\& x=3\in \mathbb{N}}$.\\
		Vậy $A=\{3\}$ nên tập $A$ có 1 phần tử
	}
\end{ex}
%Câu 14
\begin{ex}
	Cho tập hợp $A=\left\{ -1;0;1;2;3 \right\}$. Số tập con gồm 2 phần tử của tập $A$ là
	\choice
	{$20$}
	{\True 10}
	{$12$}
	{$15$}
	\loigiai{
		Các tập con gồm 2 phần tử của tập hợp $A$ là: $\left\{ -1;0 \right\},\left\{ -1;1 \right\},\left\{ -1;2 \right\},\left\{ -1;3 \right\},\left\{ 0;1 \right\},\left\{ 0;2 \right\},\left\{ 0;3 \right\},\left\{ 1;2 \right\},\left\{ 1;3 \right\},\left\{ 2;3 \right\}$.\\
		Vậy có 10 tập con gồm 2 phần tử của tập $A$
	}
\end{ex}
%Câu 15
\begin{ex}
	Cho tập hợp $A=\{ x\in \mathbb{N} \mid x $ là số nguyên tố nhỏ hơn $10\}$. Tập $A$ bằng tập hợp nào sau đây?
	\choice
	{$Q=\left\{ \,1;2;3;5;7 \right\}$}
	{$M=\left\{ 1;3;4;5 \right\}$}
	{$P=\left\{ 0;2;3;5;7 & \right\}$}
	{\True $N=\left\{ 2;3;5;7 \right\}$}
	\loigiai{
		Ta có $A=\left\{ x\in \mathbb{N} \mid x$ là số nguyên tố nhỏ hơn $ 10 \right\}=\left\{ 2;3;5;7 \right\}$.\\
		Vậy $A=N$
	}
\end{ex}
%Câu 16
\begin{ex}
	Cho hai tập hợp $A=\left\{ 1;3;5;7;9 \right\}$, $B=\left\{ 0;1;2;4;5;6;8 \right\}$. Tìm tập hợp $C=A\cup B$.
	\choice
	{$C=\left\{ 3;7;9 \right\}$}
	{$C=\left\{ 1;5 \right\}$}
	{$C=\left\{ 1;3;5;7;9 \right\}$}
	{\True $C=\left\{ 0;1;2;3;4;5;6;7;8;9 \right\}$}
	\loigiai{
		Ta có $C=\left\{ 0;1;2;3;4;5;6;7;8;9 \right\}$
	}
\end{ex}
%Câu 17
\begin{ex}
	Cho hai tập hợp $A=\left\{ 1\,;2;5 \right\}$ và $B=\left\{ 1;3\,;4;5 \right\}$. Tập hợp $A\cap B$ là tập nào dưới đây?
	\choice
	{$\left\{ 3;4 \right\}$}
	{$\{2\}$}
	{$\left\{ 1;3\,;4;5 \right\}$}
	{\True $\left\{ 1;5 \right\}$}
	\loigiai{
		Ta có $A\cap B=\left\{ 1;5 \right\}$
	}
\end{ex}
%Câu 18
\begin{ex}
	Cho các tập hợp $A=\left[-5\,;\dfrac{1}{2}\right]$, $B=\left(-3;+\infty\right)$. Khi đó tập hợp $A\cap B$ bằng:
	\choice
	{$\left\{ x\in \mathbb{R}|-3\le x\le \dfrac{1}{2} \right\}$}
	{\True $\left\{ x\in \mathbb{R}|-3<x\le \dfrac{1}{2} \right\}$}
	{$\left\{ x\in \mathbb{R}|-5<x\le \dfrac{1}{2} \right\}$}
	{$\left\{ x\in \mathbb{R}|-3\le x<\dfrac{1}{2} \right\}$}
	\loigiai{
		Ta có $A\cap B=\left(-3;\dfrac{1}{2}\right]=\left\{ x\in \mathbb{R}|-3<x\le \dfrac{1}{2} \right\}$
	}
\end{ex}
%Câu 19
\begin{ex}
	Cho hai tập hợp $A=\left\{ 1;2;3;4;5 \right\}$, $B=\left\{ 4;5;6;7 \right\}$. Xác định tập hợp $T=A\setminus B$.
	\choice
	{\True $T=\left\{ 1;2;3 \right\}$}
	{$T=\left\{ 4;5 \right\}$}
	{$T=\left\{ 6;7 \right\}$}
	{$T=\left\{ 1;2;3;4;5;6;7 \right\}$}
	\loigiai{
		Ta có $x\in A\setminus B\Leftrightarrow \heva{& x\in A \\& x\notin B}$ $\Leftrightarrow \heva{& x\in \left\{ 1;2;3;4;5 \right\} \\& x\notin \left\{ 4;5;6;7 \right\}}\Leftrightarrow x\in \left\{ 1;2;3 \right\}$.\\
		Suy ra $T=A\setminus B=\left\{ 1;2;3 \right\}$
	}
\end{ex}
%Câu 20
\begin{ex}
	Cho hai tập hợp $A=\left\{ 1;2;3;4;5 \right\}$, $B=\left\{ 3;4;5 \right\}$.
	Biết $B\subset A$, xác định tập hợp $T=C_AB$.
	\choice
	{$T=\left\{ 1;2;3 \right\}$}
	{$T=\left\{ 3;4;5 \right\}$}
	{\True $T=\left\{ 1;2 \right\}$}
	{$T=\left\{ 1;2;3;4;5 \right\}$}
	\loigiai{
		Ta có $T=C_AB=A\setminus B$\\
		$x\in A\setminus B\Leftrightarrow \heva{& x\in A \\& x\notin B}$ $\Leftrightarrow \heva{& x\in \left\{ 1;2;3;4;5 \right\} \\& x\notin \left\{ 3;4;5 \right\}}\Leftrightarrow x\in \left\{ 1;2 \right\}$.\\
		Suy ra $T=C_AB=A\setminus B=\left\{ 1;2 \right\}$
	}
\end{ex}
%Câu 21
\begin{ex}
	Cho các mệnh đề:
	$A\colon $\lq\lq$2$ là số tự nhiên lẻ\rq\rq.
	$B\colon $\lq\lq$5$ là số nguyên tố\rq\rq.
	$C\colon $\lq\lq$16$ là số chính phương\rq\rq.
	Trong các mệnh đề trên, có bao nhiêu mệnh đề đúng?
	\choice
	{$0$}
	{$3$}
	{$1$}
	{\True $2$}
	\loigiai{
		Mệnh đề $A$ là mệnh đề sai; mệnh đề $B$ và mệnh đề $C$ là các mệnh đề đúng
	}
\end{ex}
%Câu 22
\begin{ex}
	Phủ định của mệnh đề \lq\lq$\exists x\in \mathbb{R},5x-3x^2=0$\rq\rq là mệnh đề
	\choice
	{\lq\lq$\exists x\in \mathbb{R},5x-3x^2\ne 0$\rq\rq}
    {\lq\lq$\forall x\in \mathbb{R},5x-3x^2=0$\rq\rq}
	{\True \lq\lq$\forall x\in \mathbb{R},5x-3x^2\ne 0$\rq\rq}
	{\lq\lq$\exists x\in \mathbb{R},5x-3x^2\ge 0$\rq\rq}
	\loigiai{
		Phủ định của mệnh đề \lq\lq$\exists x\in \mathbb{R},5x-3x^2=0$\rq\rq là mệnh đề \lq\lq$\forall x\in \mathbb{R},5x-3x^2\ne 0$\rq\rq
	}
\end{ex}
%Câu 23
\begin{ex}
	Cho mệnh đề $P\colon \forall x\in \mathbb{R},x^2+x+1>0$. Mệnh đề phủ định của mệnh đề $P$ là:
	\choice
	{$\forall x\in \mathbb{R},x^2+x+1<0$}
	{$\forall x\in \mathbb{R},x^2+x+1\le 0$}
	{\True $\exists x\in \mathbb{R},x^2+x+1\le 0$}
	{$\not\exists x\in \mathbb{R},x^2+x+1>0$}
	\loigiai{
		Phủ định của mệnh đề $\forall x\in \mathbb{R},x^2+x+1>0$ là mệnh đề $\exists x\in \mathbb{R},x^2+x+1\le 0$
	}
\end{ex}
%Câu 24
\begin{ex}
	Trong các mệnh đề sau, mệnh đề nào sai?
	\choice
	{Tam giác $ABC$ cân và có một góc bằng $60^\circ $ tương đương tam giác $ABC$ đều}
	{Tam giác $ABC$ có ba góc bằng $60^\circ $ khi và chỉ khi tam giác $ABC$ đều}
	{Tam giác $ABC$ có ba cạnh bằng nhau nếu và chỉ nếu tam giác $ABC$ đều}
	{\True Tam giác $ABC$ cân là điều kiện cần và đủ để tam giác $ABC$ đều}
	\loigiai{
		\lq\lqNếu tam giác $ABC$ cân thì tam giác $ABC$ đều\rq\rq là mệnh đề sai. Vậy mệnh đề ở phương án D là mệnh đề sai
	}
\end{ex}
%Câu 25
\begin{ex}
	Mệnh đề nào sau đây có mệnh đề đảo đúng?
	\choice
	{Hai góc đối đỉnh thì bằng nhau}
	{Nếu một số chia hết cho 6 thì cũng chia hết cho 3}
	{\True Nếu một phương trình bậc hai có biệt thức $\triangle $ âm thì phương trình đó vô nghiệm}
	{Nếu $a=b$ thì $a^2=b^2$}
	\loigiai{
		Mệnh đề đảo của đáp án A: Hai góc bằng nhau thì đối đỉnh, là 1 mệnh đề sai.\\
		Mệnh đề đảo của đáp án B: Nếu một số chia hết cho $3$ thì cũng chia hết cho $6$, là một mệnh đề sai.\\
		Mệnh đề đảo của đáp án C: Nếu một phương trình bậc hai vô nghiệm thì nó có biệt thức $\triangle $ âm, là một mệnh đề đúng.\\
		Mệnh đề đảo của đáp án D: Nếu $a^2=b^2$ thì $a=b$, là một mệnh đề sai
	}
\end{ex}
%Câu 26
\begin{ex}
	Viết mệnh đề sau bằng cách sử dụng kí hiệu $\forall $ hoặc $\exists $: \lq\lqCó ít nhất một số thực mà bình phương của nó bằng 3\rq\rq.
	\choice
	{$\forall x\in \mathbb{Q},x^2=3$}
	{\True $\exists x\in \mathbb{R},x^2=3$. C. $\forall x\in \mathbb{R},x^2=3$}
	{$\exists x\in \mathbb{Q},x^2=3$}
	\loigiai{
		Đáp án A: Bình phương của mọi số hữu tỉ đều bằng $3$.\\
		Đáp án C: Bình phương của mọi số thực đều bằng $3$.\\
		Đáp án D: Có ít nhất một số hữu tỉ mà bình phương của nó bằng $3$
	}
\end{ex}
%Câu 27
\begin{ex}
	Kí hiệu $X$ là tập hợp các cầu thủ $x$ trong đội tuyển bóng rổ, $P(x)$ là mệnh đề chứa biến \lq\lq$x$ cao trên $180 cm$\rq\rq. Mệnh đề $''\forall x\in X$, $P(x)''$ khẳng định rằng:
	\choice
	{\True Mọi cầu thủ trong đội tuyển bóng rổ đều cao trên $180 cm$}
	{Trong số các cầu thủ của đội tuyển bóng rổ có một số cầu thủ cao trên $180 cm$}
	{Bất cứ ai cao trên $180 cm$ đều là cầu thủ của đội tuyển bóng rổ}
	{Có một số người cao trên $180 cm$ là cầu thủ của đội tuyển bóng rổ}
	\loigiai{
		Mọi cầu thủ trong đội tuyển bóng rổ đều cao trên $180\,cm\Leftrightarrow ''\forall x\in X$, $P(x)''$
	}
\end{ex}
%Câu 28
\begin{ex}
	Cho tập hợp $A=\left\{ x\in \mathbb{Z}\left| |x-1|<3 \right. \right\}$. Có bao nhiêu tập hợp con của tập hợp$A$ có đúng $4$ phần tử.
	\choice
	{$3$. B. $4$}
	{\True $5$}
	{$6$}
	\loigiai{
		Ta có $|x-1|<3\Leftrightarrow -3<x-1<3\Leftrightarrow -2<x<4$, mà $x\in \mathbb{Z}\Rightarrow x\in \left\{ -1;0;1;2;3 \right\}$.\\
		Suy ra $A=\left\{ x\in \mathbb{Z}\left| |x-1|<3 \right. \right\}=\left\{ -1;0;1;2;3 \right\}$. Các tập hợp con có đúng $4$ phần tử của tập hợp$A$ là: $\left\{ -1;0;1;2 \right\},\left\{ -1;0;1;3 \right\},\left\{ -1;0;2;3 \right\},\left\{ 0;1;2;3 \right\},\left\{ -1;1;2;3 \right\}$.\\
		Vậy có $5$ tập hợp con của tập hợp$A$ có đúng $4$ phần tử
	}
\end{ex}
%Câu 29
\begin{ex}
	Cho tập hợp $A=\left[1-m; & 4-m\right]$,$B=\left[7-4m;+\infty\right)$($m$ là tham số). Tìm tất cả giá trị của $m$ để $A\cap B\ne \varnothing $.
	\choice
	{\True $m\ge 1$}
	{$m\le 1$}
	{$m>1$}
	{$m\ge 2$}
	\loigiai{
		Ta có $A\cap B=\varnothing \Leftrightarrow 4-m\,< & 7-4m\Leftrightarrow 3m<3\Leftrightarrow m<1$.\\
		Vậy $A\cap B\ne \varnothing \Leftrightarrow m\ge 1$
	}
\end{ex}
%Câu 30
\begin{ex}
	Cho hai tập hợp $A=\left\{ \left. x\in \mathbb{N} \mid4x<13 \right\}$ và $B=\left\{ \left. x\in \mathbb{Z} \midx^2<2 \right\}$. Tìm $A\cup B$.
	\choice
	{$A\cup B=\left\{ 0;1;2 \right\}$}
	{\True $A\cup B=\left\{ -1;0;1;2;3 \right\}$}
	{$A\cup B=\left\{ -1;0;1 \right\}$}
	{$A\cup B=\left\{ -1;1;2 \right\}$}
	\loigiai{
		Ta có $A=\left\{ \left. x\in \mathbb{N} \mid4x<13 \right\}=\left\{ 0;1;2;3 \right\}$ và $B=\left\{ \left. x\in \mathbb{Z} \midx^2<2 \right\}=\left\{ -1;0;1 \right\}$.\\
		Do đó, $A\cup B=\left\{ -1;0;1;2;3 \right\}$
	}
\end{ex}
%Câu 31
\begin{ex}
	Cho hai tập hợp $A=\left\{ x\in \mathbb{R}|3x-1\ge 2;3-x>1 \right\},B=[0;3]$.
	Khẳng định nào sau đây là đúng?
	\choice
	{$C_BA=\left\{ 0;2;3 \right\}$}
	{$C_BA=[2;3]$}
	{$C_BA=[0;1)$}
	{\True $C_BA=[0;1)\cup [2;3]$}
	\loigiai{
	Ta có: $A=[1;2),B=[0;3]\Rightarrow C_BA=[0;1)\cup [2;3]$.\\
	Vậy đáp án đúng là}\\
	{
	}
\end{ex}
%Câu 32
\begin{ex}
	Cho hai tập hợp $A=\left\{ x\in \mathbb{R}|3x-1\ge 2;4-x\ge 1 \right\},B=[0;2]$.
	Khẳng định nào sau đây là đúng?
	\choice
	{$A\setminus B=[0;1)$}
	{$A\setminus B=[2;3]$}
	{\True $A\setminus B=(2;3]$}
	{$A\setminus B=[0;1)\cup (2;3]$}
	\loigiai{
	Ta có: $A=[1;3],B=[0;2]\Rightarrow A\setminus B=(2;3]$.\\
	Vậy đáp án đúng là}\\
	{
	}
\end{ex}
%Câu 33
\begin{ex}
	Cho tập $A\setminus B=\left\{ 1;2;3 \right\}$, $A\cap B=\left\{ 5{,}6 \right\}$. Số phần tử của tập hợp $A$ là
	\choice
	{$4$}
	{\True $5$}
	{$6$}
	{$3$}
	\loigiai{
		Ta có $A=\left(A\setminus B\right)\cup \left(A\cap B\right)=\left\{ 1;2;3 \right\}\cup \left\{ 5;6 \right\}=\left\{ 1;2;3;5;6 \right\}$.\\
		Vậy $A$ có 5 phần tử
	}
\end{ex}
%Câu 34
\begin{ex}
	Cho các tập hợp $A=(-3;10];B=(0;5)$. Số phần tử của tập $\left(A\setminus B\right)\cap \mathbb{Z}$ là
	\choice
	{$7$}
	{$8$}
	{\True $9$}
	{$13$}
	\loigiai{
		Ta có $A\setminus B=(-3;0]\cup [5;10]\Rightarrow \left(A\setminus B\right)\cap \mathbb{Z}=\left\{ -2;-1;0;5;6;7;8;9;10 \right\}$.\\
		Vậy $\left(A\setminus B\right)\cap \mathbb{Z}$ có 9 phần tử
	}
\end{ex}
%Câu 35
\begin{ex}
	Cho hai tập hợp $A=\left\{ x\in \mathbb{Z}||x+3|\le 5 \right\}$ và $B=\left\{ x\in \mathbb{N}|4-x\ge 2x-8 \right\}$. Có bao nhiêu số nguyên dương thuộc tập hợp $A\cap B$?
	\choice
	{\True $2$}
	{$3$}
	{$1$}
	{$4$}
	\loigiai{
		Ta có:\\
		+) $|x+3|\le 5\Leftrightarrow -5\le x+3\le 5\Leftrightarrow -8\le x\le 2$.\\
		$A=\left\{ -8;-7;-6;-5;-4;-3;-2;-1;0;1;2 \right\}$.\\
		+) $4-x\ge 2x-8\Leftrightarrow 3x\le 12\Leftrightarrow x\le 4$.\\
		$B=\left\{ 0;1;2;3;4 \right\}$.\\
		Suy ra $A\cap B=\left\{ 0;1;2 \right\}$.\\
		Vậy có $2$ số nguyên dương thuộc tập hợp $A\cap B$
	}
\end{ex}
%Câu 36
\begin{ex}
	Trong các mệnh đề sau, mệnh đề nào đúng?
	\choice
	{$\forall x\in \mathbb{R},x<4\Rightarrow x^2<16$}
	{$\exists n\in \mathbb{N},n^3-n$ không chia hết cho 3}
	{$\exists k\in \mathbb{Z},k^2+k+1$ là một số chẵn}
	{\True $\forall x\in \mathbb{Z},\dfrac{2x^3-6x^2+x-3}{2x^2+1}\in \mathbb{Z}$}
	\loigiai{
		+) Mệnh đề $\forall x\in \mathbb{R},x<4\Rightarrow x^2<16$ sai vì khi $x=-5<4$ thì $x^2=25>16$.\\
		+) Mệnh đề \lq\lq $\exists n\in \mathbb{N},n^3-n$ \rq\rq không chia hết cho 3 sai vì $n^3-n=n(n-1)(n+1)$ là tích của ba số tự nhiên liên tiếp nên luôn chia hết cho 3.\\
		+) Mệnh đề \lq\lq $\exists k\in \mathbb{Z},k^2+k+1$ \rq\rq là một số chẵn sai vì $k^2+k+1=k(k+1)+1$ luôn không chia hết cho 2.\\
		+) Mệnh đề \lq\lq $\forall x\in \mathbb{Z},\dfrac{2x^3-6x^2+x-3}{2x^2+1}\in \mathbb{Z}$ \rq\rq đúng vì $\dfrac{2x^3-6x^2+x-3}{2x^2+1}=x-3$ thuộc $\mathbb{Z}$ với mọi $x\in \mathbb{Z}$
	}
\end{ex}
%Câu 37
\begin{ex}
	Cho các mệnh đề $P$: \lq\lqSố $4$ là số chẵn\rq\rq, $Q$: \lq\lqSố $4$ chia hết cho 2\rq\rq, $R$: \lq\lqSố $4$ là số nguyên tố\rq\rq. Xét các mệnh đề sau, hỏi có bao nhiêu mệnh đề đúng?
	\lq\lq$P\Rightarrow \overline{Q}$\rq\rq; \lq\lq$\overline{Q}\Leftrightarrow \overline{R}$\rq\rq; \lq\lq$\left(P\Rightarrow Q\right)\Rightarrow R$\rq\rq; \lq\lq$\left(\overline{P}\Rightarrow Q\right)\Leftrightarrow Q$\rq\rq.
	\choice
	{$0$}
	{$3$}
	{$2$}
	{\True $1$}
	\loigiai{
		Nhận xét:\\
		+) Mệnh đề $P$ đúng thì mệnh đề $\overline{P}$ sai và ngược lại.\\
		+) Mệnh đề \lq\lq$P\Rightarrow Q$\rq\rq chỉ sai khi $P$ đúng, $Q$ sai.\\
		+) Mệnh đề \lq\lq$P\Leftrightarrow Q$\rq\rq đúng khi cả hai mệnh đề \lq\lq$P\Rightarrow Q$\rq\rq và \lq\lq$Q\Rightarrow P$\rq\rq cùng đúng
	}
\end{ex}
%Câu 38
\begin{ex}
	Cho tập hợp $A=\left\{ 1;2 \right\}$ và tập hợp $B=\left\{ x\in \mathbb{R}\left| x^2+(m+2)x-2m-8=0 \right. \right\}$. Có bao nhiêu giá trị nguyên của tham số $m$ sao cho $B\subset A$.
	\choice
	{$0$}
	{$3$}
	{\True $2$}
	{$1$}
	\loigiai{
		Ta có: $x^2+(m+2)x-2m-8=0\Leftrightarrow (x-2)(x+m+4)=0\Leftrightarrow \hoac{& x=2 \\& x=-m-4}\Rightarrow B=\left\{ 2;-m-4 \right\}$.\\
		Giả thiết: $B\subset A$ $\Leftrightarrow \hoac{& -m-4=1 \\& -m-4=2}\Leftrightarrow \hoac{& m=-5 \\& m=-6}$ (thỏa mãn).\\
		Vậy có $2$ giá trị thỏa mãn
	}
\end{ex}
%Câu 39
\begin{ex}
	Cho hai tập hợp $A=\left\{ x\in \mathbb{N}|x^2-4x-5=0 \right\}$, $B=\left\{ x\in \mathbb{R}|(x-1)\left(x^2-4\right)=0 \right\}$. Tập hợp $A\cup B$ bằng
	\choice
	{$\left\{ 1;2;-2 \right\}$}
	{$\left\{ -1;5;1;2;-2 \right\}$}
	{$\left\{ 5;1 \right\}$}
	{\True $\left\{ 5;1;2;-2 \right\}$}
	\loigiai{
		Ta có: $x^2-4x-5=0\Leftrightarrow \hoac{& x=-1\notin \mathbb{N} \\& x=5\in \mathbb{N}}\Rightarrow A=\{5\}$.\\
		Ta có: $(x-1)\left(x^2-4\right)=0\Leftrightarrow \hoac{& x-1=0 \\& x^2-4=0}\Leftrightarrow \hoac{& x=1\in \mathbb{R} \\& x=2\in \mathbb{R} \\& x=-2\in \mathbb{R}}\Rightarrow B=\left\{ 1;2;-2 \right\}$.\\
		Khi đó: $A\cup B=\left\{ 5;1;2;-2 \right\}$
	}
\end{ex}
%Câu 40
\begin{ex}
	Cho hai tập hợp $A=\left\{ 1;3 \right\}$, $B=\left\{ x\in \mathbb{R}|x^2-mx+m-1=0 \right\}$. Với giá trị nào của $m$ thì $A\setminus B=\{3\}$?
	\choice
	{$m\ne 2$}
	{$m=4$}
	{\True $m\ne 4$}
	{$m=2$}
	\loigiai{
		Ta có: $x^2-mx+m-1=0\Leftrightarrow (x-1)(x+1-m)=0\Leftrightarrow \hoac{& x=1 \\& x=m-1}$.\\
		Suy ra $B=\left\{ 1;m-1 \right\}$.\\
		Khi đó, $A\setminus B=\{3\}\Leftrightarrow m-1\ne 3\Leftrightarrow m\ne 4$
	}
\end{ex}
%Câu 41
\begin{ex}
	Một cuộc khảo sát thói quen sử dụng mạng xã hội của học sinh lớp $10A$ đưa ra những thông tin sau:
	• Có $28$ học sinh sử dụng Facebook.
	• Có $29$ học sinh sử dụng Instagram.
	• Có $19$ học sinh sử dụng Twitter.
	• Có $14$ học sinh sử dụng Facebook và Instagram.
	• Có $12$ học sinh sử dụng Facebook và Twitter.
	• Có $10$ học sinh sử dụng Instagram và Twitter.
	• Có $8$ học sinh sử dụng cả $3$ loại mạng xã hội trên.
	Biết rằng các học sinh tham gia khảo sát đều sử dụng ít nhất một loại mạng xã hội. Hỏi có bao nhiêu học sinh lớp 10A tham gia khảo sát?
	\choice
	{$52$}
	{$50$}
	{\True $48$}
	{$46$}
	\loigiai{
		Gọi $F$, $I,T$ lần lượt là tập hợp học sinh sử dụng Facebook, Instagram, Twitter.\\
		Theo giả thiết ta có:\\
		$n(F)=28$; $n(I)=29$; $n(T)=19$; $n\left(F\cap I\right)=14$; $n\left(F\cap T\right)=12$; $n\left(I\cap T\right)=10$,\\
		$n\left(F\cap I\cap T\right)=8$.\\
		Ta có:\\
		$n\left(F\cup I\cup T\right)=n(F)+n(I)+n(T)-n\left(F\cap I\right)-n\left(I\cap T\right)-n\left(F\cap T\right)+n\left(F\cap I\cap T\right)$.\\
		Hay $n\left(F\cup I\cup T\right)=28+29+19-14-12-10+8=48$.\\
		Vậy có $48$ học sinh tham gia khảo sát
	}
\end{ex}
%Câu 42
\begin{ex}
	Cho các tập hợp sau: $X=\left\{ 1;2;3;4;5;6;7;8;9 \right\}$; $A=\left\{ 1;3;4;5;8;9 \right\}$; $B=\left\{ 2;4;5;7;9 \right\}$. Khẳng định nào sau đây là đúng?
	\choice
	{$X\setminus \left(A\cup B\right)=\left(X\setminus A\right)\cup \left(X\setminus B\right)$}
	{\True $X\setminus \left(A\cap B\right)=\left(X\setminus A\right)\cup \left(X\setminus B\right)$}
	{$X\cup \left(A\cap B\right)=\left(X\cup A\right)\cap \left(X\cap B\right)$}
	{$X\cap \left(A\cup B\right)=\left(X\cup A\right)\cup \left(X\cap B\right)$}
	\loigiai{
		Ta có $A\cap B=\left\{ 4;5;9 \right\}\Rightarrow X\setminus \left(A\cap B\right)=\left\{ 1;2;3;6;7;8 \right\}(1)$.\\
		Lại có $X\setminus A=\left\{ 2;6;7 \right\}$, $X\setminus B=\left\{ 1;3;6;8 \right\}\Rightarrow \left(X\setminus A\right)\cup \left(X\setminus B\right)=\left\{ 1;2;3;6;7;8 \right\}(2)$.\\
		Từ $(1),(2)\Rightarrow X\setminus \left(A\cap B\right)=\left(X\setminus A\right)\cup \left(X\setminus B\right)$. Chọn đáp án}\\
	{
	}
\end{ex}
%Câu 43
\begin{ex}
	Cho tập hợp $A=\left\{ x\in \mathbb{R}|\dfrac{3}{|x+7|}>\dfrac{1}{3} \right\}$ và tập hợp $B=\left\{ x\in \mathbb{R}|1\le |x|\le 5 \right\}$. Tập hợp $\left(A\cup B\right)\setminus \left(A\cap B\right)$ có tất cả bao nhiêu phần tử là số nguyên?
	\choice
	{$13$}
	{\True $14$}
	{$15$}
	{$16$}
	\loigiai{
	Ta có: $A=(-16;-7)\cup (-7;2)$, $B=[-5;-1]\cup [1;5]$\\
	$A\cup B=(-16;-7)\cup (-7;5]$, $A\cap B=[-5;-1]\cup [1;2)$\\
	$\left(A\cup B\right)\setminus \left(A\cap B\right)=(-16;-7)\cup (-7;-5)\cup (-1;1)\cup [2;5]$.\\
	Vậy tập hợp $\left(A\cup B\right)\setminus \left(A\cap B\right)$ có 14 phần tử là số nguyên là $-15;-14; \cdot \cdot \cdot ;-8;-6;0;2;3;4;5$
	}
\end{ex}
%Câu 44
\begin{ex}
	Cho hai tập hợp $A=[-5;2]$ và $B=\left(m-2;m+3\right]$. Số giá trị nguyên của tham số $m$ để $A\cap B\ne \varnothing $ là
	\choice
	{\True $12$}
	{$11$}
	{$13$}
	{$10$}
	\loigiai{
		Ta có $A\cap B\doteq \varnothing \Leftrightarrow \hoac{& m-2\ge 2 \\& m+3<-5}\Leftrightarrow \hoac{& m\ge 4 \\& m<-8}$.\\
		Vậy $A\cap B\ne \varnothing \Leftrightarrow -8\le m<4$. Suy ra số giá trị nguyên của $m$ để $A\cap B\ne \varnothing $ là $12$
	}
\end{ex}
%Câu 45
\begin{ex}
	Cho khoảng $A=\left(1;m+7\right)$ và nửa khoảng $B=\left[2m+3;13\right)$ ($m$ là tham số). Gọi $S$ là tập hợp tất cả các số nguyên $m$ sao cho $A\cup B=(1;13)$. Tổng các phần tử của tập hợp $S$ là
	\choice
	{\True $10$}
	{$9$}
	{$-5$}
	{$21$}
	\loigiai{
		Điều kiện đối với $m$ để tồn tại khoảng $A$ và nửa khoảng $B$ là $\heva{& m+7>1 \\& 2m+3<13}\Leftrightarrow -6<m<5$ $\left(*\right)$.\\
		Khi đó\\
		$A\cup B=(1;13)\Leftrightarrow \heva{& 2m+3>1 \\& 2m+3\le m+7 \\& m+7\le 13}\Leftrightarrow \heva{& m>-1 \\& m\le 4 \\& m\le 6}\Leftrightarrow -1<m\le 4$.\\
		Kết hợp $\left(*\right)$, ta được $-1<m\le 4$.\\
		Vì $m\in \mathbb{Z}$ nên tập hợp các số nguyên $m$ thỏa mãn yêu cầu của bài toán là $S=\left\{ 0;1;2;3;4 \right\}$.\\
		Vậy tổng các phần tử của tập hợp $S$ bằng $10$
	}
\end{ex}
%Câu 46
\begin{ex}
	Cho tập hợp $A=\left\{ \left(x ; y\right)|x^2-25=y(y+6);x, y\in \mathbb{Z} \right\}$,
	$B=\left\{ \left(\text{5 };-6\right) ; \left(-5 ;-6\right) \right\}$ và tập hợp $M$. Biết $A\cup B=M$, số phần tử của tập hợp $M$ là
	\choice
	{$2$}
	{$4$}
	{$8$}
	{\True $6$}
	\loigiai{
	Ta có $x^2-25=y(y+6)\Leftrightarrow x^2-{{(y+3)}^2}=16\Leftrightarrow \left(|x|+|y+3|\right)\left(|x|-|y+3|\right)=16\left(*\right)$.\\
	Vì $|x|+|y+3|\ge 0$ nên từ $\left(*\right)$ suy ra $|x|-|y+3|\ge 0$.\\
	Lại có: $|x|+|y+3|\ge |x|-|y+3|$ và $x, y\in \mathbb{Z}$.\\
	Do đó $\left(|x|+|y+3|\right)\left(|x|-|y+3|\right)=16$ khi các trường hợp sau xảy ra:\\
	*$\heva{& |x|+|y+3|=16 \\& |x|-|y+3|=1}$ $\Leftrightarrow \heva{& |x|=\dfrac{17}{2} \\& |y+3|=\dfrac{15}{2}}$ (loại do $x, y\in \mathbb{Z}$).\\
	*$\heva{& |x|+|y+3|=8 \\& |x|-|y+3|=2}\Leftrightarrow \heva{& |x|=5 \\& |y+3|=3}\Leftrightarrow \heva{& x=\pm 5 \\& y+3=\pm 3}\Leftrightarrow \heva{& x=\pm 5 \\& \hoac{& y=0 \\& y=-6}} $(thỏa mãn $x, y\in \mathbb{Z}$).\\
	*$\heva{& |x|+|y+3|=4 \\& |x|-|y+3|=4}\Leftrightarrow \heva{& |x|=4 \\& |y+3|=0}\Leftrightarrow \heva{& x=\pm 4 \\& y=-3}$ (thỏa mãn $x, y\in \mathbb{Z}$).\\
	Khi đó $A=\left\{ \left(5 ; 0\right) ; \left(5 ; -6\right) ; \left(-5 ; 0\right) ; \left(-5 ; -6\right) ; \left(4 ; -3\right) ; \left(-4 ; -3\right) \right\}$.\\
	Mặt khác: $B=\left\{ \left(\text{5 };-6\right) ; \left(-5 ;-6\right) \right\}$ và $A\cup B=M$ nên $M=\left\{ \left(5 ; 0\right) ; \left(5 ; -6\right) ; \left(-5 ; 0\right) ; \left(-5 ; -6\right) ; \left(4 ; -3\right) ; \left(-4 ; -3\right) \right\}$.\\
	Vậy số phần tử của tập hợp $M$ bằng $6$
	}
\end{ex}
%Câu 47
\begin{ex}
	Lớp 10A có $40$ học sinh, trong đó có $10$ bạn học sinh giỏi Toán, $15$ bạn học sinh giỏi Lý và $19$ bạn không giỏi môn học nào trong hai môn Toán, Lý. Hỏi lớp 10A có bao nhiêu bạn học sinh vừa giỏi Toán vừa giỏi Lý?
	\choice
	{$7$}
	{$10$}
	{\True $4$}
	{$17$}
	\loigiai{
		`Số học sinh giỏi Toán hoặc Lý là: $40-19=21$.\\
		Số học sinh chỉ giỏi môn Lý là: $21-10=11$.\\
		Số học sinh chỉ giỏi môn Toán là: $21-15=6$.\\
		Suy ra số học sinh giỏi cả hai môn Toán và Lý là: $21-11-6=4$
	}
\end{ex}
%Câu 48
\begin{ex}
	Cho các tập hợp khác rỗng $A=\left(m-18;2m+7\right)$, $B=\left(m-12;21\right)$ và $C=(-15;15)$. Có bao nhiêu giá trị nguyên của tham số $m$ để $A\setminus B\subset C$.
	\choice
	{\True $5$}
	{$3$}
	{$1$}
	{$4$}
	\loigiai{
		+) Để $A$, $B$ là các tập hợp khác rỗng $\Leftrightarrow \heva{& m-18<2m+7 \\& m-12<21}\Leftrightarrow \heva{& m>-25 \\& m<33}\Leftrightarrow -25<m<33$.\\
		+) TH1: $2m+7\le m-12\Leftrightarrow m\le -19$.\\
		Ta có $A\setminus B=\left(m-18;2m+7\right)$. $A\setminus B\subset C\Leftrightarrow \heva{& m-18\ge -15 \\& 2m+7\le 15}\Leftrightarrow \heva{& m\ge 3 \\& m\le 4}\Leftrightarrow 3\le m\le 4$ (Loại).\\
		+) TH2: $m-12<2m+7\le 21\Leftrightarrow -19<m\le 7$.\\
		Ta có $A\setminus B=\left(m-18;m-12\right]$. $A\setminus B\subset C\Leftrightarrow \heva{& m-18\ge -15 \\& m-12<15}\Leftrightarrow \heva{& m\ge 3 \\& m<27}\Leftrightarrow 3\le m<27$.\\
			Kết hợp điều kiện suy ra $3\le m\le 7$.\\
			+) TH3: $2m+7>21\Leftrightarrow m>7$.\\
			Ta có $A\setminus B=\left(m-18;m-12\right]\cup \left[21;2m+7\right)$.\\
		$A\setminus B\subset C\Leftrightarrow \heva{& m-18\ge -15 \\& 2m+7\le 15}\Leftrightarrow \heva{& m\ge 3 \\& m\le 4}\Leftrightarrow 3\le m\le 4$ (Loại).\\
			Với $3\le m\le 7$ thì $A\setminus B\subset C$ nên có 5 giá trị nguyên của $m$ thỏa mãn
	}
\end{ex}
%Câu 49
\begin{ex}
	Cho các tập $A=\left[-1;5\right]$, $B=\left\{ x\in \mathbb{R}\colon \,|x|\le 2 \right\}$, $C=\left\{ x\in \mathbb{R}\colon \,x^2-9>0 \right\}$ và $D=\left[m;2m+1\right]$. Tính tổng các giá trị của $m$ sao cho $\left(\left(A\cup B\right)\setminus C\right)\cap D$ là một đoạn có độ dài bằng 1.
	\choice
	{$0$}
	{$1$}
	{\True $2$}
	{$-1$}
	\loigiai{
		+) $x\in \mathbb{R}\colon \,|x|\le 2\Leftrightarrow -2\le x\le 2$. Suy ra $B=\left[-2;2\right]$ $\Rightarrow A\cup B=\left[-2;5\right]$.\\
		+) $x\in \mathbb{R}\colon \,x^2-9>0\Leftrightarrow (x-3)(x+3)>0\Leftrightarrow \hoac{& \heva{& x-3>0 \\& x+3>0}\\& \heva{& x-3<0 \\& x+3<0}} \Leftrightarrow \hoac{& x>3 \\& x<-3}$\\
		Suy ra $C=\left(-\infty ;-3\right)\cup \left(3;+\infty\right)$ $\Rightarrow \left(A\cup B\right)\setminus C=\left[-2;3\right]$.\\
		+) Vì $\left(A\cup B\right)\setminus C$ là một đoạn có độ dài bằng 5 nên để $\left(\left(A\cup B\right)\setminus C\right)\cap D$ là một đoạn có độ dài bằng 1 thì sẽ xảy ra các trường hợp sau:\\
		TH1: $-2\le m\le 3\le 2m+1\Leftrightarrow \heva{& -2\le m\le 3 \\& m\ge 1}\Leftrightarrow 1\le m\le 3$.\\
		Khi đó: $\left(\left(A\cup B\right)\setminus C\right)\cap D=\left[m;3\right]$.\\
		Đoạn có độ dài bằng 1 khi và chỉ khi $3-m=1\Leftrightarrow m=2$ (Thoả mãn).\\
		TH2: $m\le -2\le 2m+1\le 3\Leftrightarrow \heva{& m\le -2 \\& -\dfrac{3}{2}\le m\le 1}\Leftrightarrow m\in \varnothing $.\\
		TH3: $-2\le m\le 2m+1\le 3\Leftrightarrow \heva{& m\ge -2 \\& -1\le m\le 1}\Leftrightarrow -1\le m\le 1$.\\
		Khi đó: $\left(\left(A\cup B\right)\setminus C\right)\cap D=\left[m;2m+1\right]$.\\
		Đoạn có độ dài bằng 1 khi và chỉ khi $2m+1-m=1\Leftrightarrow m=0$ (Thoả mãn).\\
		Vậy tổng các giá trị $m$ thoả mãn bằng 2
	}
\end{ex}
%Câu 50
\begin{ex}
	Cho hai tập hợp $A=\left\{ x\in \mathbb{R}\left| |mx-3|=mx-3 \right. \right\}$, $B=\left\{ x\in \mathbb{R}\left| x^2-4=0 \right. \right\}$. Tìm $m$ để $B\setminus A=B$.
	\choice
	{\True $-\dfrac{3}{2}<m<\dfrac{3}{2}$}
	{$-\dfrac{3}{2}\le m\le \dfrac{3}{2}$}
	{$m<\dfrac{3}{2}$}
	{$m\ge -\dfrac{3}{2}$}
	\loigiai{
		Ta có: $x\in A\Leftrightarrow mx-3\ge 0$.\\
		$x\in B\Leftrightarrow \hoac{& x=2 \\\\
				x=-2 } $.\\
		Cách 1:\\
		Ta có: ${B\setminus A=B}{\Leftrightarrow B\cap A=\varnothing }{\Leftrightarrow \hoac{& \begin{aligned}
							& m=0 \\& \heva{& m>0 \\& \dfrac{3}{m}>2} \end{aligned} \\\\
						\heva{& m<0 \\& \dfrac{3}{m}<-2} } }{\Leftrightarrow \hoac{& \begin{aligned}
							& m=0 \\& 0<m<\dfrac{3}{2} \end{aligned} \\\\
						-\dfrac{3}{2}<m<0 } }{\Leftrightarrow -\dfrac{3}{2}<m<\dfrac{3}{2}}$.\\
		Cách 2: ${B\setminus A=B}\Leftrightarrow \heva{& 2\notin A \\& -2\notin A}$ $\Leftrightarrow \heva{& 2m-3<0 \\& -2m-3<0}$ ${\Leftrightarrow -\dfrac{3}{2}<m<\dfrac{3}{2}}$
	}
\end{ex}
