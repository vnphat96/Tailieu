\section*{Đề kiểm tra Chương 5}
\subsection*{Đề số 1}
\setcounter{ex}{0}\setcounter{bt}{0}
\Opensolutionfile{ans}[ans/ans-KT-501]
\noindent\textbf{I. PHẦN TRẮC NGHIỆM}
\begin{ex}%[Dự án BG10-2022 - Đề KT]%[Đỗ Viết Lân]%[0D1Y5-1]
	Cho $a$ là số gần đúng của số đúng $\overline{a}$. Khi đó $\Delta_a = |\overline{a} - a|$ được gọi là 
	\choice
	{số quy tròn của $\overline{a}$}
	{sai số tương đối của số gần đúng $a$}
	{\True sai số tuyệt đối của số gần đúng $a$}
	{số quy tròn của $a$}
	\loigiai{$\Delta_a = |\overline{a} - a|$ được gọi là sai số tuyệt đối của số gần đúng $a$.}
\end{ex}

\begin{ex}%[Dự án BG10-2022 - Đề KT]%[Đỗ Viết Lân]%[0D1B5-2]
	Trong các số dưới đây, giá trị gần đúng của $\sqrt{30}-5$ với sai số tuyệt đối bé nhất là
	\choice
	{$0{,}476$}
	{\True $0{,}477$}
	{$0{,}478$}
	{$0{,}479$}
	\loigiai{Dùng máy tính cầm tay ta tính được giá trị gần đúng của $\sqrt{30}-5$ là $0{,}477225$ nên giá trị gần đúng của nó với sai số tuyệt đối bé nhất trong bốn đáp án trên là $0{,}477$. }
\end{ex}

\begin{ex}%[Dự án BG10-2022 - Đề KT]%[Đỗ Viết Lân]%[0D1B5-2]
	Nếu lấy $3,1416$ làm giá trị gần đúng cho $\pi$ thì sai số tuyệt đối không vượt quá 
	\choice{$0, 0002$}
	{$0, 0003$}
	{\True $0, 0001$}
	{$0, 0004$}
	\loigiai{Ta có $\pi = 3, 141592654...$. Do $3, 1415 < \pi = 3, 141592654... < 3, 1416$ nên ta có 
		\begin{equation*}
			\Delta = |3,1416 - \pi| < |3,1416 - 3, 1415| = 0, 0001.
		\end{equation*}
	}
\end{ex}

\begin{ex}%[Dự án BG10-2022 - Đề KT]%[Đỗ Viết Lân]%[0D1B5-2]
	Cho giá trị gần đúng của $\dfrac{3}{7}$ là $0{,}429$ thì sai số tuyệt đối không vượt quá 
	\choice{$0{,}002$}
	{\True $0{,}001$}
	{$ 0{,}003$}
	{$0{,}004$}
	\loigiai{
		Ta có $\dfrac{3}{7} = 0{,}428571...$.\\
		Do $0{,}428 < \dfrac{3}{7} = 0{,}428571... < 0{,}429$ nên $\Delta = \left|0, 429 - \dfrac{3}{7}\right| < |0{,}429 - 0{,}428| = 0{,}001$.  
	}
\end{ex}

\begin{ex}%[Dự án BG10-2022 - Đề KT]%[Đỗ Viết Lân]%[0D1K5-2]
	Một vật có thể tích $V = 180{,}37 \text{\ cm}^3 \pm 0,05 \text{\ cm}^3$. Nếu lấy $180{,}37 \text{\ cm}^3$ làm giá trị gần đúng cho $V$ thì sai số tương đối của giá trị gần đúng đó không vượt quá 
	\choice{\True $0{,}03 \%$}
	{$0{,}01 \%$}
	{$0{,}02 \%$}
	{$0{,}001 \%$}
	\loigiai{Ta có $\delta = \dfrac{\Delta}{|V|} \leq \dfrac{d}{|V|} = \dfrac{0,05}{180{,}37} \approx 0{,}03 \%$. 
	}
\end{ex}

\begin{ex}%[Dự án BG10-2022 - Đề KT]%[Đỗ Viết Lân]%[0D1K5-2]
	Số $\overline{a}$ được cho bởi giá trị gần đúng $a = 5{,}7824$ với sai số tương đối không vượt quá $0{,}05\%$. Khi đó, sai số tuyệt đối của $a$ không vượt quá
	\choice{\True $0{,}0028912$}
	{$0{,}0027912$}
	{$0{,}0026912$}
	{$0{,}0025912$}
	\loigiai{Ta có $\delta_a = \dfrac{\Delta_a}{|a|} \leq 0{,}0005 \Rightarrow \Delta_a \leq 0{,}0005 \cdot 5{,}7824 = 0{,}0028912$.
	}
\end{ex}

%dạng 2
\begin{ex}%[Dự án BG10-2022 - Đề KT]%[Đỗ Viết Lân]%[0D1Y5-1]
	Cho số $a$ là số gần đúng của $\overline{a}$ với độ chính xác $d$. Mệnh đề nào sau đây là mệnh đề đúng?
	\choice
	{$\overline{a}=a+d$}
	{$\overline{a}=a-d$}
	{$\overline{a}=a$}	
	{\True $\overline{a}=a\pm d$}
	\loigiai{
		Nếu $a$ là số gần đúng của $\overline{a}$ với độ chính xác $d$ thì $\overline{a}=a\pm d$.
	}
\end{ex}

\begin{ex}%[Dự án BG10-2022 - Đề KT]%[Đỗ Viết Lân]%[0D1B5-2]
Trong một buổi thực hành về phép đo, một học sinh đo chiều dài của phòng học và thu được kết quả là $\overline{a}=7{,}4863$ m $\pm 0{,}001$ m. Tìm số qui tròn của số gần đúng $7{,}4863$.
\choice
{$7{,}48$}
{$7{,}5$}
{$7{,}4$}
{\True $7{,}49$}
\loigiai
{
Số $\overline{a}$ có độ chính xác là $d=0{,}001$ (độ chính xác đến hàng phần nghìn) nên ta qui tròn số đến hàng phần trăm.
Vậy số qui tròn của số gần đúng $7{,}4863$ là $7{,}49$.
}
\end{ex}


%dạng 3
\begin{ex}%[Dự án BG10-2022 - Đề KT]%[Đỗ Viết Lân]%[0D1Y5-1]
	Đo độ cao của một ngọn cây là $h=347{,}53$ m $\pm$ $0{,}2$ m. Hãy viết số quy tròn của số gần đúng $347{,}53$.
	\choice
	{$345$}
	{\True $348$}
	{$347{,}6$}
	{$347{,}5$}
	\loigiai{
		Vì độ chính xác đến hàng phần chục nên ta quy tròn số $347{,}53$ đến hàng đơn vị.\\
		Vậy số quy tròn là $348$.
	}
\end{ex}

\begin{ex}%[Dự án BG10-2022 - Đề KT]%[Đỗ Viết Lân]%[0D1Y5-1]
Khi sử dụng máy tính cầm tay với $10$ chữ số thập phân ta được $\sqrt{8}= 2{,}828427125$. Số quy tròn đến hàng phần trăm của $\sqrt{8}$ là
\choice
{\True $2{,}83$}{$2{,}8$}{$2{,}82$}{$2{,}828$}
\loigiai{
Số quy tròn đến hàng phần trăm của $\sqrt{8}$ là $2{,}83$.
}
\end{ex}

\begin{ex}%[Dự án BG10-2022 - Đề KT]%[Đỗ Viết Lân]%[0D1Y5-1]
Tìm số quy tròn đến chữ số hàng phần trăm của số $54732{,}14752498$.
	\choice
	{\True$54732{,}15$}
	{$54700$}
	{$54732{,}148$}
	{$54732{,}1$}	
\loigiai{
	Số $54732{,}14752498$ quy tròn đến hàng phần trăm là $54732{,}15$ (do chữ số đằng sau hàng quy tròn là $7$).
}
\end{ex}

\begin{ex}%[Dự án BG10-2022 - Đề KT]%[Đỗ Viết Lân]%[0D1Y5-1]
Cho số gần đúng $a=2841275$ với độ chính xác $d=300$. Hãy viết số quy tròn của số $a$.
\choice
{\True $ 2841000$}
{$ 2841280$}
{$ 2841300$}
{ $ 2842000$}
\loigiai{
Vì độ chính xác đến hàng trăm ($d=300$) nên ta quy tròn $a$	đến hàng nghìn theo quy tắc làm tròn. \\
Vậy số quy tròn của $a$ là $2841000$.
}
\end{ex}

%Xu the trung tam

\begin{ex}%[Dự án BG10-2022 - Đề KT]%[Đỗ Viết Lân]%[0D5B3-1]
	Thời gian chạy $ 50 $ m của $ 20 $ học sinh được ghi lại trong bảng dưới đây
	\begin{center}
		\begin{tabular}{|l|c|c|c|c|c|}
			\hline
			Thời gian (giây) & 8,3 & 8,4 & 8,5 & 8,7 & 8,8 \\
			\hline
			Số bạn & 2 & 3 & 9 & 5 & 1 \\
			\hline
		\end{tabular}
	\end{center}
	Số trung bình cộng thời gian chạy của $20$ học sinh là
	\choice
	{ $8{, }54$}
	{$8{, }5$}
	{\True $8{, }53$}
	{$4$}
	\loigiai{
	Số trung bình cộng thời gian chạy của $ 20 $ học sinh là
	\[ \dfrac{8{,}3\cdot 2+8{,}4\cdot 3+8{,}5\cdot 9+8{,}7\cdot 5+8{,}8\cdot 1}{20}=8{,}53.\]
	}
\end{ex}

\begin{ex}%[Dự án BG10-2022 - Đề KT]%[Đỗ Viết Lân]%[0D5B3-1]
	Nhiệt độ trung bình (${}^{\circ}$C) hàng tháng trong năm $2020$ của tỉnh $A$ được ghi lại trong bảng sau
	\begin{center}
		\begin{tabular}{|c|c|c|c|c|c|c|c|c|c|c|c|c|}
			\hline
			Tháng&$1$&$2$&$3$&$4$&$5$&$6$&$7$&$8$&$9$&$10$&$11$&$12$\\
			\hline
			Nhiệt độ&$25$&$27$&$28$&$28$&$29$&$30$&$30$&$30$&$28$&$26$&$27$&$27$\\
			\hline
		\end{tabular}
	\end{center}
	Nhiệt độ trung bình của tỉnh $A$ trong năm $2020$ gần nhất với giá trị nào dưới đây?
	\choice
	{$27^{\circ}$ C}
	{\True $27{,}9^{\circ}$ C}
	{$28^{\circ}$ C}
	{$27{,}8^{\circ}$ C}
	\loigiai{
		Nhiệt độ trung bình của tỉnh $A$ trong năm $2020$ là 
		\begin{align*}
			\dfrac{25+27+28+28+29+30+30+30+28+26+27+27}{12}\approx 27{,}917^\circ \text{ C}. 
		\end{align*}
		
	}
\end{ex}

\begin{ex}%[Dự án BG10-2022 - Đề KT]%[Đỗ Viết Lân]%[0D5Y3-1]
		Một học sinh có điểm các bài kiểm tra Toán như sau: 8; 4; 9; 8; 6; 6; 9; 9; 9. Điểm trung bình môn Toán của học sinh đó (làm tròn đến 1 chữ số thập phân) là
		\choice
		{$ 7{,}3 $}
		{\True $ 7{,}6 $}
		{$ 8{,}5 $}
		{$ 6{,}8 $}
		\loigiai
		{Điểm trung bình môn Toán của học sinh đó là $ \dfrac{8+4+9+8+6+6+9+9+9}{9}=\dfrac{68}{9} \approx 7{,}6 $.
		}
	\end{ex}

\begin{ex}%[Dự án BG10-2022 - Đề KT]%[Đỗ Viết Lân]%[0D5B3-2]
	Cho bảng số liệu ghi lại điểm của $ 40 $ học sinh trong bài kiểm tra 1 tiết môn toán
	\begin{center}
		\begin{tabular}{|c|c|c|c|c|c|c|c|c|c|}
			\hline
			Điểm & 3 & 4 & 5 & 6 & 7 & 8 & 9 & 10 & Cộng  \\
			\hline
			Số học sinh & 2 & 3 & 7 & 18 & 3 & 2 & 4 & 1 & 40 \\
			\hline
		\end{tabular}
	\end{center}
	Số trung vị là
	\choice
	{$7$}
	{$6{, }5$}
	{\True $6$}
	{$5$}
	\loigiai{
	Có $ 40 $ học sinh nên số trung vị là số trung bình cộng của điểm của hai em có điểm xếp thứ $ 20 $ và $ 21 $. Theo bảng số liệu, hai em này đều được $ 6 $ điểm nên số trung vị là $ 6 $.
	}
\end{ex}

\begin{ex}%[Dự án BG10-2022 - Đề KT]%[Đỗ Viết Lân]%[0D5B3-2]
	Tiền lương hàng tháng của $7$ nhân viên trong một công ty du lịch lần lượt là $6{,}5$; $8{,}4$; $6{,}9$; $7{,}2$; $2{,}5$; $6{,}7$, $3{,}0$ (đơn vị: triệu đồng). Số trung vị của dãy số liệu thống kê trên bằng 
	\choice
	{\True $6{,}7$ triệu đồng}
	{$7{,}2$ triệu đồng}
	{$6{,}8$ triệu đồng}
	{$6{,}9$ triệu đồng}
	\loigiai{
		Sắp xếp tăng dần số tiền lương hàng tháng của $7$ nhân viên ta được dãy sau
		\begin{center}
			\begin{tabular}{|c|c|c|c|c|c|c|}
				\hline
				$2{,}5$ &$3{,}0$ &$6{,}5$ &$6{,}7$ &$6{,}9$ &$7{,}2$ &$8{,}4$ \\
				\hline
			\end{tabular}
		\end{center}
		Số đứng chính giữa của dãy số liệu trên là số trung vị. Vậy số trung vị của dãy số liệu trên là $6{,}7$.
	}
\end{ex}

\begin{ex}%[Dự án BG10-2022 - Đề KT]%[Đỗ Viết Lân]%[0D5Y3-2]
	Cho bảng số liệu thống kê chiều cao của một nhóm học sinh như sau
	\begin{center}
		\begin{tabular}{|c|c|c|c|c|c|c|c|c|c|c|c|c|c|c|}
			\hline
			$151$	&$152$&$153$&$154$&$155$&$160$&$160$&$162$&$163$&$165$&$165$&$165$&$166$&$167$&$167$\\
			\hline
		\end{tabular}
	\end{center}
	Số trung vị của bảng số liệu nói trên là
	\choice
	{$160$}
	{\True $162$}
	{$167$}
	{$161$}
	\loigiai{
		Bảng giá trị trên có $15$ giá trị được xếp theo thứ tự tăng dần và số thứ $8$ có giá trị $162$. Do đó số trung vị của bảng số liệu nói trên là $162$.
	}
\end{ex}

\begin{ex}%[Dự án BG10-2022 - Đề KT]%[Đỗ Viết Lân]%[0D5B3-3]
Thống kê điểm thi của $30$ em học sinh đứng đầu kì thi học sinh giỏi Toán (thang điểm 20), kết quả được cho trong  bảng sau đây
\begin{center}
	\begin{tabular}{|c|c|c|c|c|c|}
	\hline
	Điểm & 16 & 17 & 18 & 19 &  \\
	\hline
	Số học sinh & 9 & 11 & 7 & 3 & N=30 \\
	\hline
\end{tabular}
\end{center}
Mốt của bảng phân bố đã cho là 
	\choice
	{$19$}
	{$3$}
	{\True $17$}
	{$11$}
\loigiai{
Theo lí thuyết $M_0=17$.
	}
\end{ex}

\begin{ex}%[Dự án BG10-2022 - Đề KT]%[Đỗ Viết Lân]%[0D5B3-3]
	Số áo bán được trong một quý ở cửa hàng bán áo sơ mi nam được thống kê như sau
	\begin{center}
		\begin{tabular}{|c|c|c|c|c|c|c|c|}
			\hline
			Cỡ áo &$36$ &$37$ &$38$ &$39$ &$40$ &$41$ &$42$\\
			\hline
			Số áo bán được &$13$ &$45$ &$126$ &$125$ &$110$ &$40$ &$12$\\
			\hline
		\end{tabular}
	\end{center}
	Giá trị mốt của bảng phân bố tần số trên bằng
	\choice
	{\True $38$}
	{$126$}
	{$42$}
	{$12$}
	\loigiai{
		Dựa vào bảng phân bố tần số, ta thấy cỡ áo $38$ có tần số lớn nhất nên giá trị mốt của bảng phân bố tần số trên bằng $38$. 
	}
\end{ex}

\begin{ex}%[Dự án BG10-2022 - Đề KT]%[Đỗ Viết Lân]%[0D5Y3-3]
Thống kê điểm kiểm tra môn toán (thang điểm $10$) của một nhóm gồm $8$ học sinh ta có bảng số liệu sau
\begin{center}
\begin{tabular}{|c|c|c|c|c|c|c|c|c|}
\hline
Tên học sinh&An&Bình&Chi&Dung&Minh&Nam&Thuận&Hòa\\
\hline
Điểm&$9$&$8$&$7$&$10$&$8$&$9$&$8$&$7$\\
\hline
\end{tabular}
\end{center}
Giá trị Mốt của dãy số liệu thống kê trên là
\choice
{\True $M_0=8$}
{$M_0=9$}
{$M_0=10$}
{$M_0=7$}
\loigiai{
Ta có bảng thống kê sau
\begin{center}
\begin{tabular}{|c|c|c|c|c|c|}
\hline
Giá trị&$7$&$8$&$9$&$10$&\\
\hline
Số học sinh&$2$&$3$&$2$&$1$&$N=8$\\
\hline
\end{tabular}
\end{center}
Dựa vào bảng ta có $M_0=8$.
}
\end{ex}

\begin{ex}%[Dự án BG10-2022 - Đề KT]%[Đỗ Viết Lân]%[0D5Y3-5]
Có bao nhiêu giá trị của mẫu số liệu nằm giữa $Q_1$ và $Q_3$?
\choice
{$25\%$}
{\True $50\%$}
{$75\%$}
{$100\%$}
\loigiai{
Theo lý thuyết có $50\%$ giá trị của mẫu số liệu nằm giữa $Q_1$ và $Q_3$.
}
\end{ex}

\begin{ex}%[Dự án BG10-2022 - Đề KT]%[Đỗ Viết Lân]%[0D5Y3-5]
	Cho mẫu số liệu $21 ; 35 ; 17 ; 43 ; 8 ;59 ;72 ; 119$. Tứ phân vị thứ nhất, thứ hai, thứ ba lần lượt là
	\choice
	{ \True $19; \,39; \,65,5$}
	{ $26; \,43; \,65,5$}
	{$39; \,19; \,65,5$}
	{$43; \,26;\, 65,5$}
	\loigiai{
		Sắp xếp lại mẫu số liệu theo thứ tự không giảm, ta được: $8 ; 17 ; 21 ; 35 ; 43 ; 59 ; 72 ; 119$.
		\begin{itemize}
			\item Vì cỡ mẫu là $n=8$, là số chẵn, nên giá trị tứ phân vị thứ hai là
			$$
			Q_{2}=\dfrac{1}{2}(35+43)=39.
			$$
			\item Tứ phân vị thứ nhất là trung vị của mẫu: $8 ; 17 ; 21 ; 35$. Do đó $Q_{1}=19$.
			\item Tứ phân vị thứ ba là trung vị của mẫu:  $43 ; 59 ; 72 ; 119$. Do đó $Q_{3}=65,5$.
		\end{itemize}
	}	
\end{ex}

\begin{ex}%[Dự án BG10-2022 - Đề KT]%[Đỗ Viết Lân]%[0D5Y3-5]
	Cho mẫu số liệu  $2 ; 3 ; 10 ; 13 ; 5 ; 15 ; 5 ; 7$. Tứ phân vị thứ nhất, thứ hai, thứ ba lần lượt là
	\choice
	{ $11,5; \,6; \,4$}
	{\True $4; \,6; \,11,5$}
	{$6; \,4; \,11,5$}
	{$6; \,11,5;\, 4$}
	\loigiai{
		Sắp xếp lại mẫu số liệu theo thứ tự không giảm, ta được: $2 ; 3 ; 5 ; 5 ; 7 ; 10 ; 13$.
		\begin{itemize}
			\item Vì cỡ mẫu là $n=8$, là số chẵn, nên giá trị tứ phân vị thứ hai là
			$$
			Q_{2}=\dfrac{1}{2}(5+7)=6.
			$$
			\item Tứ phân vị thứ nhất là trung vị của mẫu: $2 ; 3 ; 5 ; 5$. Do đó $Q_{1}=4$.
			\item Tứ phân vị thứ ba là trung vị của mẫu: $7 ; 10 ; 13 ; 15$. Do đó $Q_{3}=11,5$.
		\end{itemize}
	}	
	
\end{ex}

\begin{ex}%[Dự án BG10-2022 - Đề KT]%[Đỗ Viết Lân]%[0D5Y3-6]
	Hãy tìm khoảng biến thiên của mẫu số liệu thống kê sau:
	\begin{longtable}{p{1cm}p{1cm}p{1cm}p{1cm}p{1cm}p{1cm}p{1cm}p{1cm}p{1cm}p{1cm}p{1cm}p{1cm}}
		22 & 26 & 31 & 15 & 12 & 4 & 18
		& 93 & 17 & 64 & 10
	\end{longtable}
	\choice
	{$33$}
	{$83$}
	{\True $89$}
	{$97$}
	\loigiai{
		Khoảng biến thiên của mẫu số liệu là $R=93-4=89$.
	}
\end{ex}

\begin{ex}%[Dự án BG10-2022 - Đề KT]%[Đỗ Viết Lân]%[0D5Y3-6]
	Mẫu số liệu nào dưới đây có khoảng biến thiên là $13$?
	\choice
	{$11$, $28$, $56$, $12$}
	{$6$, $12$, $33$, $23$, $11$}
	{$25$, $9$, $13$, $10$}
	{\True Tất cả đều sai}
	\loigiai{
		Khoảng biến thiên của các mẫu số liệu lần lượt là
		\begin{itemize}
			\item $R_1=56-11=45$.
			\item $R_2=33-6=26$.
			\item $R_3=25-9=14$.
		\end{itemize}
	}
\end{ex}

\begin{ex}%[Dự án BG10-2022 - Đề KT]%[Đỗ Viết Lân]%[0D5Y3-6]
	Tuổi và giới tính của những đứa trẻ trong một khu trung cư được cho bởi bảng sau
	\begin{longtable}{p{1.5cm}p{1cm}p{1cm}p{1cm}p{1cm}p{1cm}p{1cm}p{1cm}p{1cm}}
		Nam: & 10 & 4 & 1 & 6 & 2 & 8 & 5\\
		Nữ: & 2 & 3 & 6 & 4 & 1 & 7 & \\
	\end{longtable}
	Dựa vào khoảng biến thiên của hai mẫu số liệu ``Nam'' và ``Nữ'', hãy chỉ ra mẫu số liệu nào có độ phân tán lớn hơn.
	\choice
	{\True Mẫu số liệu ``Nam'' có độ phân tán lớn hơn mẫu số liệu ``Nữ''}
	{Mẫu số liệu ``Nam'' có độ phân tán lớn hơn mẫu số liệu ``Nữ''}
	{Hai mẫu số liệu có độ phân tán bằng nhau}
	{Tất cả đều sai}
	\loigiai{
		Khoảng biến thiên của mẫu số liệu ``Nam'' là $ R_1=10-1=9 $.\\
		Khoảng biến thiên của mẫu số liệu ``Nữ'' là $ R_2=7-1=6 $.\\
		Do $R_1>R_2$ nên mẫu số liệu ``Nam'' có độ phân tán lớn hơn mẫu số liệu ``Nữ''.	
	}
\end{ex}

\begin{ex}%[Dự án BG10-2022 - Đề KT]%[Đỗ Viết Lân]%[0D5Y3-7]
	Cho mẫu số liệu  $2 ; 3 ; 10 ; 13 ; 5 ; 15 ; 5 ; 7$. Khoảng tứ phân vị của mẫu số liệu là
	\choice
	{ $7$}
	{\True $7{,}5$}
	{$5{,}5$}
	{$2$}
	\loigiai{
		Sắp xếp lại mẫu số liệu theo thứ tự không giảm, ta được: $2 ; 3 ; 5 ; 5 ; 7 ; 10 ; 13$.
		\begin{itemize}
			\item Vì cỡ mẫu là $n=8$, là số chẵn, nên giá trị tứ phân vị thứ hai là
			$$
			Q_{2}=\dfrac{1}{2}(5+7)=6.
			$$
			\item Tứ phân vị thứ nhất là trung vị của mẫu: $2 ; 3 ; 5 ; 5$. Do đó $Q_{1}=4$.
			\item Tứ phân vị thứ ba là trung vị của mẫu: $7 ; 10 ; 13 ; 15$. Do đó $Q_{3}=11,5$.
		\end{itemize}
		Khoảng tứ phân vị của mẫu số liệu là $\Delta_Q = Q_3 - Q_1 = 7{,}5$.
	}	
	
\end{ex}

\begin{ex}%[Dự án BG10-2022 - Đề KT]%[Đỗ Viết Lân]%[0D5B4-1]
Cho dãy số liệu thống kê $2,3,4,5,6,7,8$. Phương sai của các số liệu thống kê đã cho là
\choice
{$6$}
{\True $4$}
{$5$}
{$7$}
\loigiai{
\begin{itemize}
\item Giá trị trung bình của dãy số liệu là 
\[\overline{m}=\dfrac{2+3+4+5+6+7=8}{7}=5.\]
\item Phương sai là $$s^2=\dfrac{(2-5)^2+(3-5)^2+(4-5)^2+(5-5)^2+(6-5)^2+(7-5)^2+(8-5)^2}{7}=4.$$
\end{itemize}
}
\end{ex}

\begin{ex}%[Dự án BG10-2022 - Đề KT]%[Đỗ Viết Lân]%[0D5B4-1]
	Tính phương sai của dãy số liệu $1;3;3;5;7;9;10;11;11;11$.
	\choice
	{$\dfrac{71}{10}$}
	{$\dfrac{\sqrt{1329}}{10}$}
	{$\dfrac{\sqrt{710}}{10}$}
	{\True $\dfrac{1329}{100}$}
\loigiai{
	Ta có trung bình cộng của dãy số liệu trên là $7{,}1$. Suy ra phương sai của nó bằng
	\[\dfrac{(1-7{,}1)^2+2(3-7{,}1)^2+(5-7{,}1)^2+(7-7{,}1)^2+(9-7{,}1)^2+(10-7{,}1)^2+3(11-7{,}1)^2}{10}=\dfrac{1329}{100}.\]
}
\end{ex}

\begin{ex}%[Dự án BG10-2022 - Đề KT]%[Đỗ Viết Lân]%[0D5B4-1]
		Cho mẫu số liệu thống kê $\{1;2;3;4;5;6;7;8;9 \}$. Tính (gần đúng) độ lệch chuẩn của mẫu số liệu trên? 
		\choice
		{$2{,}45$}
		{\True $2{,}58$}
		{$6{,}67$}
		{$6{,}0$}
		\loigiai{
		Ta có giá trị trung bình $\overline{x}=\dfrac{1+2+3+4+5+6+7+8+9}{9}=5$.\\
		Do đó độ lệch chuẩn
		$$s=\sqrt{\dfrac{(1-5)^2+(2-5)^2+(3-5)^2+\cdots+(9-5)^2}{9}}=\dfrac{2\sqrt{15}}{3}\approx 2{,}58.$$
		}
\end{ex}
	
\begin{ex}%[Dự án BG10-2022 - Đề KT]%[Đỗ Viết Lân]%[0D5B4-1]
    Cho mẫu số liệu $\{10,8,6,2,4\}$. Độ lệch chuẩn của mẫu là
    \choice
    {$8$}
    {$2,4$}
    {\True $2,8$}
    {$6$}
    \loigiai{
        Giá trị trung bình của dãy số liệu là $\overline{x}=\dfrac{10+8+6+4+2}{5}=6$.\\
        Độ lệch chuẩn của dãy số liệu là
        $$\delta_x=\sqrt{\dfrac{(10-6)^2+(8-6)^2+(4-6)^2+(2-6)^2}{5}}\simeq2,8.$$
    }
\end{ex}

\begin{ex}%[Dự án BG10-2022 - Đề KT]%[Đỗ Viết Lân]%[0D5B3-5]
	Một mẫu số liệu thống kê có các tứ phân vị lần lượt là $Q_1=53$, $Q_2= 55$, $Q_3= 61$. Giá trị nào sau đây \textbf{không} phải là giá trị bất thường 
	của mẫu số liệu?
	\choice
	{$40$}
	{$80$}
	{\True $73$}
	{$73{,}5$} 
	\loigiai{
		Ta có $\Delta_{Q} = Q_3 - Q_1 = 8$. Do đó 	$\left[Q_1-1{,}5 
		\cdot  \Delta_{Q}; Q_3+1{,}5 \cdot  \Delta_{Q} \right] 
		=\left[41;73\right]$.\\
		Do $73 \in \left[41;73\right]$ nên không phải là một giá trị bất thường 
		của mẫu số liệu.
	}
\end{ex}

\begin{ex}%[Dự án BG10-2022 - Đề KT]%[Đỗ Viết Lân]%[0D5B3-5]
	Một mẫu số liệu thống kê có các tứ phân vị lần lượt là $Q_1=3$, $Q_2= 7$, 
	$Q_3= 12$. Giá trị nào sau đây  là giá trị bất thường 
	của mẫu số liệu?
	\choice
	{$22$}
	{$-8{,}5$}
	{\True $26$}
	{$25{,}5$} 
	\loigiai{
		Ta có $\Delta_{Q} = Q_3 - Q_1 = 9$. Do đó 	$\left[Q_1-1{,}5 
		\cdot  \Delta_{Q}; Q_3+1{,}5 \cdot  \Delta_{Q} \right] 
		=\left[-10{,}5;25{,}5\right]$.\\
		Do $26 \notin \left[-10{,}5;25{,}5\right]$ nên  là một giá 
		trị bất thường của mẫu số liệu.
	}
\end{ex}

\begin{ex}%[Dự án BG10-2022 - Đề KT]%[Đỗ Viết Lân]%[0D5B3-5]
	Hãy tìm các giá trị bất thường của mẫu số liệu thống kê sau
	\begin{center}	
		\begin{tabular}{cccccccccccc}
			$10$ & $59$ & $67$ & $72$ & $73$ & $76$  & $88$ & $92$ & $106$& $111$ & $115$ & $169$ 
		\end{tabular} 
	\end{center}
	\choice
	{$169$}
	{$115$; $169$}
	{$111$; $169$}
	{\True $10$; $169$} 
	\loigiai{
		Từ bảng số liệu ta tìm được số trung vị $Q_2=\dfrac{76+88}{2}=82$, 
		tứ 
		phân vị dưới $Q_1= 69{,}5$, tứ phân vị trên $Q_3= 103{,}5$ và khoảng tứ phân vị 
		$\Delta_{Q} = 103{,}5 - 69{,}5 = 34$.\\
		Ta có $\left[Q_1-1{,}5 \cdot  \Delta_{Q}; Q_3+1{,}5 \cdot  \Delta_{Q} 
		\right] =\left[18{,}5;154{,}5\right]$.\\
		Từ đó ta có $10$ và $169$ là các số liệu bất thường.			
	}
\end{ex}




\noindent\textbf{II. PHẦN TỰ LUẬN}

\begin{bt}%[Dự án BG10-2022 - Đề KT]%[Đỗ Viết Lân]%[0D1Y5-3]
Sử dụng cùng lúc 3 thiết bị khác nhau để đo thành tích của một vận động viên, người ta thu được kết quả như sau
\begin{center}
\begin{tabular}{|c|c|c|c|}
\hline 
Thiết bị & A & B & C \\ 
\hline 
Kết quả & $9{,}592 \pm 0{,}004$ & $9{,}593\pm 0{,}005$ & $9{,}598 \pm 0{,}006$ \\ 
\hline 
\end{tabular} 
\end{center}
Tính sai số tương đối của từng thiết bị và cho biết thiết bị nào có sai số tương đối nhỏ nhất.
\loigiai{
Xét kết quả của thiết bị A, ta có $\delta_A \leq \dfrac{0{,}004}{9{,}592} \approx 4{,}170 \cdot 10^{-2} \%$\\
Xét kết quả của thiết bị B, ta có $\delta_B \leq \dfrac{0{,}005}{9{,}593} \approx 5{,}212 \cdot 10^{-2} \%$\\
Xét kết quả của thiết bị C, ta có $\delta_C \leq \dfrac{0{,}006}{9{,}589} \approx 6{,}257 \cdot 10^{-2} \%$\\
Thiết bị A có sai số tương đối nhỏ nhất.
}
\end{bt}

\begin{bt}%[Dự án BG10-2022 - Đề KT]%[Đỗ Viết Lân]%[0D5Y3-3]
	Số đôi giày bán ra trong Quý $IV$ năm $2020$ của một cửa hàng được thống kê trong bảng tần số sau:\\
	\begin{center}
	\begin{tabular}{|c|c|c|c|c|c|c|c|c|}
	\hline 
	Cỡ giày &37&38&39&40&41&42&43&44 \\ 
	\hline 
	Số đôi giày bán được &40&48&52&70&54&47&28&3 \\ 
	\hline 
	\end{tabular} 
	\end{center}
	\begin{listEX}[1]
		\item Mốt của mẫu số liệu trên là bao nhiêu?
		\item Cửa hàng đó nên nhập về nhiều hơn cỡ giày nào để bán trong tháng tiếp theo?
	\end{listEX}
	\loigiai{	
		\begin{listEX}[1]
			\item Ta thấy giá trị $40$ có tần số $70$ lớn nhất, do đó mốt của mẫu số liệu trên là: $M_O=40$.
			\item Cửa hàng nên nhập về nhiều hơn cỡ giày $40$ để bán trong tháng tiếp theo.
		\end{listEX}
	}
\end{bt}

\begin{bt}%[Dự án BG10-2022 - Đề KT]%[Đỗ Viết Lân]%[0D5B3-6]
	Cho dãy số liệu thống kê: $x$, $21$, $22$, $23$, $24$, $y$. Tìm $x$, $y$ biết số trung bình cộng bằng $22,5$ và khoảng biến thiên của mẫu số liệu bằng $5$. 
	\loigiai{Ta có: 
		$\heva{&\overline{x}=\frac{x+21+22+23+24+y}{6}=22,5\\&y-x=5}\Leftrightarrow\heva{&x=20\\&y=25.}$}
\end{bt}

\begin{bt}%[Dự án BG10-2022 - Đề KT]%[Đỗ Viết Lân]%[0D5B3-5]
	Một cảnh sát giao thông ghi tốc độ (đơn vị: km/h) của $25$ chiếc xe qua trạm như sau:
	\begin{center}
		\begin{tabular}{ccccccccccccccc}
			20 & 41 & 41 & 80 & 40 & 52 & 52 & 52 & 60 & 55 & 60 & 60 & 62  \\  
			60 & 65 & 60 & 65 & 135 & 70 & 70 & 65 & 75 & 75 & 70 & 55 &  \\  
		\end{tabular} 
	\end{center}
	Hãy tìm các số liệu bất thường trong mẫu số liệu trên.
	\loigiai{
		Sắp xếp các số liệu trong mẫu theo thứ tự không giảm ta có
		\begin{center}
			\begin{tabular}{|c|c|c|c|c|c|c|c|c|c|c|c|c|}
				\hline 
				Tốc độ    & $20$ & $40$& $41$& $52$& $55$& $60$& $62$& $65$& $70$ & $75$& $80$& $135$\\
				\hline 
				Số lần xuất hiện  & $1$ & $1$ & $2 $ & $3$& $2$& $5$& $1$& $3$& $3$& $2$ & $1$& $1$\\
				\hline
			\end{tabular}
		\end{center} 
		Từ bảng số liệu ta tìm được số trung vị $Q_2=60$, tứ phân vị dưới $Q_1= 52$, tứ phân vị trên $Q_3= 70$ và khoảng tứ phân vị $\Delta_{Q} = 70 - 52 = 18$.\\
		Ta có $\left[Q_1-1{,}5 \cdot  \Delta_{Q}; Q_3+1{,}5 \cdot  \Delta_{Q} \right] =\left[25;97\right]$.\\
		Từ đó ta có $135$ là số liệu bất thường trong mẫu số liệu.		
	}
\end{bt}
\Closesolutionfile{ans}
\Closesolutionfile{ansbook}
% \indapan{10}{ans/ans-KT-501}
