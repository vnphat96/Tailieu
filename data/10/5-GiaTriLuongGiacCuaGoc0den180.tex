\def\tenchude{GTLG CỦA MỘT GÓC TỪ $\mathbf{0^\circ}$ ĐẾN $\mathbf{180^\circ}$}
\setcounter{section}{0}
\section{GTLG CỦA MỘT GÓC TỪ $\mathbf{0^\circ}$ ĐẾN $\mathbf{180^\circ}$}
\subsection{Tóm tắt lí thuyết}
\subsubsection{Khái niệm}
	\immini
	{
Điểm $M(x_0;y_0)$ nằm trên nửa đường tròn đơn vị sao cho $\widehat{xOM}=\alpha$. Khi đó
\begin{itemize}
	\item $\sin\alpha=y_0$;
	\item $\cos\alpha=x_0$;
	\item $\tan\alpha=\dfrac{\sin\alpha}{\cos\alpha}$ với $(\alpha\ne 90^\circ)$;
	\item $\cot\alpha=\dfrac{\cos\alpha}{\sin\alpha}$ với ($\alpha\ne 0^\circ,180^\circ$).
\end{itemize}
}
{
\begin{tikzpicture}[scale=1.5, font=\footnotesize,line join=round, line cap=round, >=stealth, x=2cm,y=2cm]
	\def\x{0.5}
	\def\a{60}
	\pgfmathsetmacro{\y}{\x*tan(\a)}
	\draw[->] (-1.2,0)--(1.2,0) node [below]{$x$};
	\draw[->] (0,-0.2)--(0,1.2) node [left]{$y$};
	\node at (0,0) [below left]{$O$};
	\draw (1,0) arc (0:180:2cm);
	\fill (1,0) node[below]{$1$} circle(1pt);
	\fill (-1,0) node[below]{$-1$} circle(1pt);
	\fill (0,1) node[above left]{$1$} circle(1pt);
	%	\fill (\x,0) node[below]{$x_0$} circle(1pt);
	\fill (\x,0) node[below]{$x_0$} circle(1pt);
	%	\fill (\a:1) node[above right]{$M$} circle(1pt);
	\fill (\a:1) node[above]{$M$} circle(1pt);
	\fill (0,\y) node[left]{$y_0$} circle(1pt);
	\draw[dashed] (\x,0)|-(0,\y);
	\draw (0,0)--(\a:1);
	\draw (0.3,0) arc (0:\a:0.6cm);
	\node at (0,0)[shift={(25:0.2)}]{$\alpha$};
\end{tikzpicture}
}
\subsubsection{Dấu của giá trị lượng giác.}
\begin{center}
	\renewcommand\arraystretch{1.2} %tăng độ rộng
	% \renewcommand{\tabcolsep}{6mm} %tăng chiều dài
	\begin{tabular}{|c|l l r|l r c|}
		\hline
		Góc $\alpha$ & $0^\circ$ && \multicolumn{2}{c}{$90^\circ$} && $180^\circ$\\
		\hline
		$\sin\alpha$ && $+$ &&& $+$ &\\
		\hline
		$\cos\alpha$ && $+$ &&& $-$ &\\
		\hline
		$\tan\alpha$ && $+$ &&& $-$ &\\
		\hline
		$\cot\alpha$ && $+$ &&& $-$ &\\
		\hline
	\end{tabular}
\end{center}
\subsubsection{Bảng giá trị lượng giác của một số góc đặc biệt cần nhớ}
\begin{center}
	\renewcommand{\arraystretch}{2}%
	\begin{tabular}{|c|c|c|c|c|c|c|c|c|c|}
		\hline
		$\alpha$ & $0^\circ$ & $30^\circ$ & $45^\circ$ & $60^\circ$ & $90^\circ$ & $120^\circ$ & $135^\circ$ & $150^\circ$& $180^\circ$ \\
		\hline
		$\sin\alpha$& $0$ & $\dfrac{1}{2}$ & $\dfrac{\sqrt{2}}{2}$  &  $\dfrac{\sqrt{3}}{2}$ & $1$ & $\dfrac{\sqrt{3}}{2}$ & $\dfrac{\sqrt{2}}{2}$ & $\dfrac{1}{2}$ & $0$ \\
		\hline
		$\cos\alpha$& $1$ & $\dfrac{\sqrt{3}}{2}$ & $\dfrac{\sqrt{2}}{2}$  &  $\dfrac{1}{2}$& $0$& $\dfrac{-1}{2}$&$\dfrac{-\sqrt{2}}{2}$&$\dfrac{-\sqrt{3}}{2}$&$-1$ \\
		\hline
		$\tan\alpha$& $0$ & $\dfrac{\sqrt{3}}{3}$ & $1$  &  $\sqrt{3}$& $\parallel$& $-\sqrt{3}$&$-1$&$\dfrac{-\sqrt{3}}{3}$& $0$\\
		\hline
		$\cot\alpha$& $\parallel$ & $\sqrt{3}$ & $1$  &  $\dfrac{\sqrt{3}}{3}$& $0$& $\dfrac{-\sqrt{3}}{3}$&$-1$&$-\sqrt{3}$& $\parallel$\\
		\hline
	\end{tabular}
\end{center}

\subsubsection{Tính chất}
\begin{multicols}{2}
\begin{enumerate}
	\item Giá trị lượng giá của hai góc phụ nhau
	\begin{itemize}
		\item $\sin(90^\circ-\alpha)=\cos\alpha$.
		\item $\cos(90^\circ-\alpha)=\sin\alpha$.
		\item $\tan(90^\circ-\alpha)=\cot\alpha$.
		\item $\cot(90^\circ-\alpha)=\tan\alpha$.
	\end{itemize}
	\item Giá trị lượng giác của hai góc bù nhau
	\begin{itemize}
		\item $\sin(180^\circ-\alpha)=\sin\alpha$.
		\item $\cos(180^\circ-\alpha)=-\cos\alpha$.
		\item $\tan(180^\circ-\alpha)=-\tan\alpha$.
		\item $\cot(180^\circ-\alpha)=-\cot\alpha$.
	\end{itemize}
\end{enumerate}
\end{multicols}
\begin{enumerate}
\item[c)] Hệ thức cơ bản
	\begin{itemize}
		\item $\sin^2\alpha+\cos^2\alpha=1$.
		\item $1+\tan^2\alpha=\dfrac{1}{\cos^2\alpha}$ với $(\alpha\ne 90^\circ)$.
		\item $1+\cot^2\alpha=\dfrac{1}{\sin^2\alpha}$ với $(0^\circ<\alpha< 180^\circ)$.
		\item $\tan\alpha\cdot\cot\alpha=1$ với $(0^\circ<\alpha< 180^\circ, \alpha\ne 90^\circ)$.
	\end{itemize}
\end{enumerate}
\subsection{Các dạng toán}
%\setcounter{subsection}{1}% Reset lại số đếm subsection
\begin{dang}{Tính giá trị biểu thức lượng giác. Chứng minh đẳng thức lượng giác}
	Áp dụng các công thức lượng giác
\end{dang}
\subsubsection{Ví dụ minh hoạ}
\begin{vd}%[0H2Y1]
	Tính giá trị biểu thức sau
	\begin{tasks}(2)
		\task $A= 2\cos 30^\circ+3\sin 120^\circ$.
		\task $B=a\cos60^{\circ}+2a\tan45^{\circ}-3a\sin30^{\circ}$.
	\end{tasks}
	\loigiai{
		\begin{enumerate}[a)]
			\item
			\item Ta có $B=\dfrac{1}{2}a+2a-\dfrac{1}{2}.3a=a$.
		\end{enumerate}
		
	}
\end{vd}
\begin{vd}%[0H2Y1]
	Cho $x=30^{\circ}$. Tính $A=\sin (2x)-3\cos x$.
	\loigiai{
		$A=\sin 2.(30^{\circ})-3\cos30^{\circ}=\sin60^{\circ}-3\cos30^{\circ}=\dfrac{\sqrt{3}}{2}-3\dfrac{\sqrt{3}}{2}=-\sqrt{3}$.
	}
\end{vd}

\begin{vd}
	Biết $\sin15^\circ=\dfrac{\sqrt{3}-1}{2\sqrt{2}}$. Tính giá trị biểu thức $P=\sin165^\circ+\cos75^\circ$.
\end{vd}

\begin{vd}
Không dùng máy tính, tính giá trị của các biểu thức sau
\begin{enumerate}
	\item $A=\sin 45^\circ\cot 135^\circ+\cos 60^\circ\cdot\sin 150^\circ-\cos 30^\circ\cdot\sin 120^\circ$.
	\item $B=\tan 135^\circ+\cot 60^\circ\cot 30^\circ-\tan 60^\circ\tan 150^\circ$.
	\item $C=2\sin 60^\circ\tan 150^\circ-\cos 180^\circ\cdot \cot 45^\circ$.
\end{enumerate}	
\loigiai{
\begin{enumerate}
	\item Ta có $\sin 45^\circ=-\cos 135^\circ=\dfrac{\sqrt{2}}{2}$, $\cos 60^\circ=\sin 150^\circ=\dfrac{1}{2}$ và $\cos 30^\circ=\sin 120^\circ=\dfrac{\sqrt{3}}{2}$.\\
	Từ đó suy ra
	$A=\dfrac{\sqrt{2}}{2}\cdot\left(\dfrac{-\sqrt{2}}{2}\right)+\dfrac{1}{2}\cdot\dfrac{1}{2}-\dfrac{\sqrt{3}}{2}\cdot\dfrac{\sqrt{3}}{2}=\dfrac{-1}{2}+\dfrac{1}{4}-\dfrac{3}{4}=-1$.
	\item Do $\tan 135^\circ=-1$, $\cot 60^\circ=\dfrac{\sqrt{3}}{3}$, $\cot 30^\circ=\tan 60^\circ=\sqrt{3}$ và $\tan 150^\circ=\dfrac{-\sqrt{3}}{3}$ nên
	$$B=-1+\dfrac{\sqrt{3}}{3}\cdot \sqrt{3}-\sqrt{3}\cdot\left(\dfrac{-\sqrt{3}}{3}\right)=1.$$
	\item Ta có $\sin 60^\circ=\dfrac{\sqrt{3}}{2}$, $\tan 150^\circ=\dfrac{-\sqrt{3}}{3}$, $\cos 180^\circ=-1$ và $\cot 45^\circ=1$.\\
	Suy ra $C=2\cdot\dfrac{\sqrt{3}}{2}\cdot\left(\dfrac{-\sqrt{3}}{3}\right)-(-1)\cdot 1=0$.
\end{enumerate}	
\textbf{Chú ý.} Nếu để ý đến mối liên hệ giữa các góc có trong biểu thức, như các góc bù nhau, các góc phụ nhau, thì ta có thể giải bài toán theo cách sau
\begin{enumerate}
	\item Do $135^\circ=180^\circ-45^\circ$, $150^\circ=180^\circ-30^\circ$, $120^\circ=180^\circ-60^\circ$ nên 
	\allowdisplaybreaks
	\begin{eqnarray*}
		A&=&\sin 45^\circ\cdot(-\cos 45^\circ)+\cos 60^\circ\cdot\sin 30^\circ-\cos 30^\circ\cdot\sin 60^\circ\\
		&=&\dfrac{\sqrt{2}}{2}\cdot\left(\dfrac{-\sqrt{2}}{2}\right)+\dfrac{1}{2}\cdot\dfrac{1}{2}-\dfrac{\sqrt{3}}{2}\cdot\dfrac{\sqrt{3}}{2}=-\dfrac{1}{2}+\dfrac{1}{4}-\dfrac{3}{4}=-1.
	\end{eqnarray*}
\item Do $135^\circ=180^\circ-45^\circ$, $60^\circ=90^\circ-30^\circ$, $150^\circ=180^\circ-30^\circ$ nên
$$B=-1+1-\tan 60^\circ\cdot(-\tan 30^\circ)=1.$$
\item Do $150^\circ=180^\circ-30^\circ$ nên
\allowdisplaybreaks
\begin{eqnarray*}
	C&=&2\sin 60^\circ\cdot(-\tan 30^\circ)-\cos 180^\circ\cdot\cot 45^\circ\\
	&=&2\cdot\dfrac{\sqrt{3}}{2}\cdot\left(\dfrac{-\sqrt{3}}{3}\right)-(-1)\cdot 1=0.
\end{eqnarray*}
\end{enumerate}
}
\end{vd}

\baitaptl
\begin{bt}
Tính giá trị của các biểu thức
\begin{enumerate}
	\item $A= \sin 45^\circ+2 \sin 60^\circ+\tan 120^\circ+\cos 135^\circ$;
	\item $B= \tan 45^\circ \cdot \cot 135^\circ-\sin 30^\circ \cdot \cos 120^\circ-\sin 60^\circ \cdot \cos 150^\circ$;
	\item $C=\cos^2 5^\circ+\cos^2 25^\circ+\cos^2 45^\circ+\cos^2 65^\circ+\cos^2 85^\circ$;
	\item $D= \dfrac{12}{1+\tan^273^\circ} -4\tan 75^\circ\cdot\cot 105^\circ+12\sin^2 107^\circ-2 \tan 40^\circ \cdot \cos 60^\circ \cdot \tan 50^\circ$;
	\item $E=4 \tan 32^\circ \cdot \cos 60^\circ \cdot \cot 148^\circ+\dfrac{5 \cot^2 108^\circ}{1+\tan^2 18^\circ}+5\sin^272^\circ$.
\end{enumerate}	
\loigiai{
\begin{enumerate}
	\item 
	\allowdisplaybreaks
	\begin{eqnarray*}
		A&=& \sin 45^\circ+2 \sin 60^\circ+\tan 120^\circ+\cos 135^\circ\\
		&=&\dfrac{\sqrt{2}}{2}+2\cdot\dfrac{\sqrt{3}}{2}-\sqrt{3}-\dfrac{\sqrt{2}}{2}\\
		&=&\sqrt{3}-\sqrt{3}=0.
	\end{eqnarray*}
	\item 
	\allowdisplaybreaks
	\begin{eqnarray*}
		B&=& \tan 45^\circ \cdot \cot 135^\circ-\sin 30^\circ \cdot \cos 120^\circ-\sin 60^\circ \cdot \cos 150^\circ\\
		&=&1\cdot (-1)-\dfrac{1}{2}\cdot\left(\dfrac{-1}{2}\right)-\dfrac{\sqrt{3}}{2}\cdot \left(\dfrac{-\sqrt{3}}{2}\right)\\
		&=&-1+\dfrac{1}{4}+\dfrac{3}{4}=0.
	\end{eqnarray*}
	\item Do $5^\circ=90^\circ-85^\circ$, $25^\circ=90^\circ-65^\circ$ nên 
	\allowdisplaybreaks
	\begin{eqnarray*}
		C&=&\cos^25^\circ+\cos^2 25^\circ+\cos^2 45^\circ+\cos^2 65^\circ+\cos^2 85^\circ\\
		&=&\sin^285^\circ+\cos^285^\circ+\sin^225^\circ+\cos^225^\circ+\cos^245^\circ\\
		&=&1+1+\left(\dfrac{\sqrt{2}}{2}\right)^2=2+\dfrac{1}{2}=\dfrac{5}{2}.
	\end{eqnarray*}
	\item 
	\allowdisplaybreaks
	\begin{eqnarray*}
		D&=& \dfrac{12}{1+\tan^273^\circ} -4\tan 75^\circ\cdot\cot 105^\circ+12\sin^2 107^\circ-2 \tan 40^\circ \cdot \cos 60^\circ \cdot \tan 50^\circ\\
		&=&12\cos^273^\circ-4\tan75^\circ\cdot\cot(180^\circ-75^\circ)+12\sin^2(180^\circ-73^\circ)-2\tan(90^\circ-50)\cos60^\circ\tan 50^\circ\\
		&=&12\cos^273^\circ+4\tan75^\circ\cdot\cot75^\circ+12\sin^273^\circ-2\cot 50^\circ\cdot \tan 50^\circ\cdot \cos 60^\circ\\
		&=&12+4-1=15.
	\end{eqnarray*}
	\item Ta có do $148^\circ=180^\circ-32^\circ$, $108^\circ=180^\circ-72^\circ$ và $18^\circ=90^\circ-72^\circ$ nên
	\allowdisplaybreaks
	\begin{eqnarray*}
		E&=&4 \tan 32^\circ \cdot \cos 60^\circ \cdot \cot 148^\circ+\dfrac{5 \cot^2 108^\circ}{1+\tan^2 18^\circ}+5\sin^272^\circ\\
		&=&-4 \tan 32^\circ \cdot \cos 60^\circ \cdot \cot 32^\circ+5\cot^2108^\circ\cdot\cos^218^\circ+5\sin^272^\circ\\
		&=&-4\cdot\dfrac{1}{2}+5\cot^2108^\circ\cdot\sin^272^\circ+5\sin^272^\circ\\
		&=&-2+5\sin^272^\circ\cdot\left(1+\cot^2108^\circ\right)\\
		&=&-2+5\sin^272^\circ\cdot \dfrac{1}{\sin^2 108^\circ}\\
		&=&-2+5=3.
	\end{eqnarray*}
\end{enumerate}	
}
\end{bt}
\begin{bt}%[0H2K1]
	Tính giá trị các biểu thức sau:
	\begin{tasks}(1)
		\task $A=\sin^2 10^{\circ}+\sin^2 20^{\circ}+\dots+\sin^2 170^{\circ}+\sin^2 180^{\circ}$.
		\task $B=\tan 10^{\circ}.\tan 20^{\circ}\dots\tan 80^{\circ}$.
		\task $C=\cot 20^{\circ}+\cot 40^{\circ}+\dots +\cot 140^{\circ}+\cot160^{\circ}$.
	\end{tasks}
	\loigiai{
		\begin{enumerate}[a)]
			\item Ta có $\sin 10^{\circ}=\sin170^{\circ},\ \sin20^{\circ}=\sin160^{\circ},\dots$, suy ra $C= 2\bigl(\sin^2 10^{\circ}+\sin^2 20^{\circ}+\dots+\sin^2 80^{\circ}\bigr)+\sin^2 90^{\circ}$. Mặt khác ta có $\sin 80^{\circ}=\cos 10^{\circ},\ \sin 70^{\circ}=\cos 20^{\circ},\dots$, có 4 cặp như vậy nên ta tính được $A=5$.
			\item $\tan 10^{\circ}=\cot 80^{\circ}$, $\tan 20^{\circ}=\cot 70^{\circ}$, $\tan 30^{\circ}=\cot 60^{\circ}$, $\tan 40^{\circ}=\cot 50^{\circ}$. Do đó, ta tính được $B=1$.
			\item $\cot20^{\circ}=-\cot160^{\circ},\ \cot40^{\circ}=-\cot140^{\circ},\dots$ nên ta tính được $C=0$.
		\end{enumerate}
	}
\end{bt}
\begin{bt}
	Chứng minh rằng
	\begin{enumerate}
		\item $\sin^4\alpha+\cos^4\alpha=1-2\sin^2\alpha\cdot\cos^2\alpha$;
		\item $\sin^6\alpha+\cos^6\alpha=1-3\sin^2\alpha\cdot\cos^2\alpha$;
		\item $\sqrt{\sin^4\alpha+6\cos^2\alpha+3}+\sqrt{\cos^4\alpha+4\sin^2\alpha}=4$.
	\end{enumerate}
\loigiai{
	\begin{enumerate}
	\item Ta có
	\allowdisplaybreaks
	\begin{eqnarray*}
		\sin^4\alpha+\cos^4\alpha&=&(\sin^2\alpha)^2+(\cos^2\alpha)^2\\
		&=&(\sin^2\alpha)^2+(\cos^2\alpha)^2+2\sin^2\alpha\cdot\cos^2\alpha-2\sin^2\alpha\cdot\cos^2\alpha\\
		&=&\left(\sin^2\alpha+\cos^2\alpha\right)^2-2\sin^2\alpha\cdot\cos^2\alpha\\
		&=&1-2\sin^2\alpha\cdot\cos^2\alpha.
	\end{eqnarray*}
		\item Ta có
		\allowdisplaybreaks
	\begin{eqnarray*}
		\sin^6\alpha+\cos^6\alpha&=&(\sin^2\alpha)^3+(\cos^2\alpha)^3\\
		&=&\left(\sin^2\alpha+\cos^2\alpha\right)\cdot\left(\sin^4\alpha-\sin^2\alpha\cdot\cos^2\alpha+\cos^4\alpha\right)\\
		&=&\left(\sin^2\alpha+\cos^2\alpha\right)^2-3\sin^2\alpha\cdot\cos^2\alpha\\
		&=&1-3\sin^2\alpha\cdot\cos^2\alpha.
	\end{eqnarray*}
	\item \allowdisplaybreaks
	\begin{eqnarray*}
		&&\sqrt{\sin^4\alpha+6\cos^2\alpha+3}+\sqrt{\cos^4\alpha+4\sin^2\alpha}\\
		&=&\sqrt{\sin^4\alpha+6(1-\sin^2\alpha)+3}+\sqrt{\cos^4\alpha+4(1-\cos^2\alpha)}\\
		&=&\sqrt{\sin^4\alpha-6\sin^2\alpha+9}+\sqrt{\cos^4\alpha-4\cos^2\alpha+4}\\
		&=&\sqrt{(3-\sin^2\alpha)^2}+\sqrt{(2-\cos^2\alpha)}\\
		&=&3-\sin^2\alpha+2-\cos^2\alpha=5-(\sin^2\alpha + \cos^2 \alpha)=4.
	\end{eqnarray*}
\end{enumerate}
}
\end{bt}

\begin{bt}%[0H2B1]
	Cho $A, B, C$ là các góc của tam giác. Chứng minh các đẳng thức sau:
	\begin{tasks}(2)
		\task $\sin\left(A+B\right)=\sin C.$
		\task $\cos\left(A+B\right)+\cos C=0.$
		\task $\sin\dfrac{A+B}{2}=\cos\dfrac{C}{2}.$
		\task $\tan\left(A-B+C\right)=-\tan2B.$
	\end{tasks}
	\loigiai{ Do $A, B, C$ là các góc của tam giác nên ta có $A+B+C=180^{\circ}$.
		\begin{enumerate}[a)]
			\item Ta có $A+B+C=180^{\circ}\Leftrightarrow A+B=180^{\circ}-C.$\\
			Từ đó suy ra $\sin\left(A+B\right)=\sin \left(180^{\circ}-C\right)=\sin C.$
			\item Ta có $A+B+C=180^{\circ}\Leftrightarrow A+B=180^{\circ}-C.$\\
			Từ đó suy ra $\cos\left(A+B\right)=\cos \left(180^{\circ}-C\right)=-\cos C \Rightarrow \cos\left(A+B\right)+\cos C=0.$
			\item Ta có $A+B+C=180^{\circ}\Leftrightarrow \dfrac{A+B}{2}=\dfrac{180^{\circ}-C}{2}=90^{\circ}-\dfrac{C}{2}.$\\
			Từ đó suy ra $\sin\dfrac{A+B}{2}=\sin\left(90^{\circ}-\dfrac{C}{2}\right)= \cos\dfrac{C}{2}.$
			\item Ta có $\tan\left(A-B+C\right)=\tan\left(A+B+C-2B\right)=\tan\left(180^{\circ}-2B\right)=-\tan2B.$
		\end{enumerate}
	}
	\end{bt}
\begin{dang}{Tìm các GTLG khi biết một GTLG của góc}
Áp dụng tính chất về dấu của GTLG của một góc và các công thức lượng giác cơ bản.
\end{dang}
\viduminhhoa
\begin{vd}%[0H2B1-2]%[Nguyễn Tiến]%Ví dụ 1.
	\text{}
	\begin{enumerate}
		\item Cho $\sin\alpha=\dfrac{1}{3}$ với $90^\circ<\alpha<180^\circ$. Tính $\cos\alpha$ và $\tan\alpha$.
		\item Cho $\cos\alpha=-\dfrac{2}{3}$ và $\sin\alpha>0$. Tính $\sin\alpha$ và $\cot\alpha$.
		\item Cho $\tan\alpha=-2\sqrt{2}$, tính giá trị lượng giác còn lại.
	\end{enumerate}
	\loigiai{
		\begin{enumerate}
			\item Vì $90^\circ<\alpha<180^\circ$ nên $\cos\alpha<0$, mặt khác $\sin^2\alpha+\cos^2\alpha=1$ suy ra
			$$\cos\alpha=-\sqrt{1-\sin^2\alpha}=-\sqrt{1-\dfrac{1}{9}}=-\dfrac{2\sqrt{2}}{3}.$$
			Do đó $\tan\alpha=\dfrac{\sin\alpha}{\cos\alpha}=\dfrac{\dfrac{1}{3}}{-\dfrac{2\sqrt{2}}{3}}=-\dfrac{1}{2\sqrt{2}}$.
			\item Vì $\sin^2\alpha+\cos^2\alpha=1$ và $\sin\alpha>0$, nên $\sin\alpha=\sqrt{1-\cos^2\alpha}=\sqrt{1-\dfrac{4}{9}}=\dfrac{\sqrt{5}}{3}$.\\
			Ta có $\cot\alpha=\dfrac{\cos\alpha}{\sin\alpha}=\dfrac{-\dfrac{2}{3}}{\dfrac{\sqrt{5}}{3}}=-\dfrac{2}{\sqrt{5}}$.
			\item Vì $\tan\alpha=-2\sqrt{2}<0\Rightarrow\cos\alpha<0$.\\
			Ta có $\tan^2\alpha+1=\dfrac{1}{\cos^2\alpha}$, suy ra $\cos\alpha=-\sqrt{\dfrac{1}{\tan^2+1}}=-\sqrt{\dfrac{1}{8+1}}=-\dfrac{1}{3}$.\\
			Do đó $\tan\alpha=\dfrac{\sin\alpha}{\cos\alpha}\Rightarrow\sin\alpha=\tan\alpha\cdot\cos\alpha=-2\sqrt{2}\cdot\left(-\dfrac{1}{3}\right)=\dfrac{2\sqrt{2}}{3}$.\\
			$\Rightarrow\cot\alpha=\dfrac{\cos\alpha}{\sin\alpha}=\dfrac{-\dfrac{1}{3}}{\dfrac{2\sqrt{2}}{3}}=-\dfrac{1}{2\sqrt{2}}$.
		\end{enumerate}
	}
\end{vd}
\begin{vd}%[0H2B1-3]%[Nguyễn Tiến]%Ví dụ 2.
	\begin{enumerate}
		\item Cho $\cos\alpha=\dfrac{3}{4}$ với $0^\circ<\alpha<90^\circ$. Tính $A=\dfrac{\tan\alpha+3\cot\alpha}{\tan\alpha+\cot\alpha}$.
		\item Cho $\tan\alpha=\sqrt{2}$. Tính $B=\dfrac{\sin\alpha-\cos\alpha}{\sin^3\alpha+3\cos^3\alpha+2\sin\alpha}$.
	\end{enumerate}
	\loigiai{
		\begin{enumerate}
			\item Ta có $A=\dfrac{\tan\alpha+3\dfrac{1}{\tan\alpha}}{\tan\alpha+\dfrac{1}{\tan\alpha}}=\dfrac{\tan^2\alpha+3}{\tan^2\alpha+1}=\dfrac{\dfrac{1}{\cos^2\alpha}+2}{\dfrac{1}{\cos^2\alpha}}=1+2\cos^2\alpha$.\\
			Suy ra $A=1+2\cdot\dfrac{9}{16}=\dfrac{17}{8}$.
			\item Ta có $B=\dfrac{\dfrac{\sin\alpha}{\cos^3\alpha}-\dfrac{\cos\alpha}{\cos^3\alpha}}{\dfrac{\sin^3\alpha}{\cos^3\alpha}+\dfrac{3\cos^3\alpha}{\cos^3\alpha}+\dfrac{2\sin\alpha}{\cos^3\alpha}}=\dfrac{\tan\alpha\left(\tan^2\alpha+1\right)-\left(\tan^2\alpha+1\right)}{\tan^3\alpha+3+2\tan\alpha\left(\tan^2\alpha+1\right)}$.\\
			Suy ra $B=\dfrac{\sqrt{2}(2+1)-(2+1)}{2\sqrt{2}+3+2\sqrt{2}(2+1)}=\dfrac{3(\sqrt{2}-1)}{3+8\sqrt{2}}$.
		\end{enumerate}
	}
\end{vd}
\baitaptl
\begin{bt}
Cho góc $\alpha$, $0^\circ<\alpha<180^\circ$ thỏa mãn $\cos\alpha=\dfrac{-1}{3}$.
\begin{enumerate}
	\item Tính $\tan\alpha$.
	\item Tính giá trị của biểu thức $P=\tan\alpha+2\cot\alpha$.
\end{enumerate}	
\loigiai{
\begin{enumerate}
	\item Do $\cos\alpha=\dfrac{-1}{3}<0$ nên $\alpha$ là góc tù và $\tan\alpha=-\sqrt{\dfrac{1}{\cos^2\alpha}-1}=-2\sqrt{2}$.
	\item Do $\tan\alpha\cot\alpha=1$ và $\tan\alpha=-2\sqrt{2}$ nên $\cot\alpha=\dfrac{-\sqrt{2}}{4}$ và bởi vậy $$P=-2\sqrt{2}+2\cdot\left(\dfrac{-\sqrt{2}}{4}\right)=\dfrac{-5\sqrt{2}}{4}.$$
\end{enumerate}
\textbf{Nhận xét.} Khi tính $\tan\alpha$ từ $\cos\alpha$ nhờ đẳng thức $1+\tan^2\alpha=\dfrac{1}{\cos^2\alpha}$ sai lầm thường gặp của học sinh là mặc định coi $\tan\alpha=\sqrt{\dfrac{1}{\cos^2\alpha}-1}$ mà quên mất $\tan\alpha<0$ khi $\alpha$ là góc tù.
}
\end{bt}
\begin{bt}
Cho góc $\alpha$ thỏa mãn $0^\circ<\alpha<180^\circ$ và $\tan\alpha=2$. Tính giá trị của các biểu thức sau
\begin{enumerate}
	\item $G=2\sin\alpha+\cos\alpha$;
	\item $H=\dfrac{2\sin\alpha+\cos\alpha}{\sin\alpha-\cos\alpha}$.
\end{enumerate}	
\loigiai{
	\begin{enumerate}
		\item Do $\alpha$ thỏa mãn $0^\circ<\alpha<180^\circ$ và $\tan\alpha=2$ nên $\sin\alpha>0$ và $\cos\alpha>0$.\\
		Ta có $\cos\alpha=\sqrt{\dfrac{1}{1+\tan^2\alpha}}=\sqrt{\dfrac{1}{1+4}}=\dfrac{\sqrt{5}}{5}$.\\
		Từ đó $\sin\alpha=\tan\alpha\cdot\cos\alpha=\dfrac{2\sqrt{5}}{5}$.\\
		Vậy $G=2\sin\alpha+\cos\alpha=\dfrac{4\sqrt{5}}{5}+\dfrac{\sqrt{5}}{5}=\sqrt{5}$.
		\item Ta có $H=\dfrac{2\sin\alpha+\cos\alpha}{\sin\alpha-\cos\alpha}=\dfrac{2\tan\alpha+1}{\tan\alpha-1}=5$.
	\end{enumerate}
}
\end{bt}

\begin{bt}
Cho góc $\alpha$ với $90^\circ<\alpha<180^\circ$ thỏa mãn $\sin\alpha=\dfrac{3}{4}$. Tính giá trị của biểu thức $F=\dfrac{\tan\alpha+2\cot\alpha}{\tan\alpha+\cot\alpha}$.
\loigiai{
Do $\alpha\in (90^\circ;180^\circ)$ nên $\cos\alpha<0$.\\
Ta có $\cos\alpha=-\sqrt{1-\sin^2\alpha}=-\sqrt{1-\left(\dfrac{3}{4}\right)^2}=\dfrac{-\sqrt{7}}{4}$.\\
Suy ra $\tan\alpha=\dfrac{\sin\alpha}{\cos\alpha}=\dfrac{-3\sqrt{7}}{7}$ và $\cot\alpha=\dfrac{1}{\tan\alpha}=\dfrac{-\sqrt{7}}{3}$.\\
Vậy $F=\dfrac{\tan\alpha+2\cot\alpha}{\tan\alpha+\cot\alpha}=\dfrac{23}{16}$.
}	
\end{bt}

\begin{bt}
Cho góc $\alpha$ thỏa mãn $0^\circ<\alpha<180^\circ$ và $\tan\alpha=\sqrt{2}$. Tính giá trị của các biểu thức sau $$K=\dfrac{\sin^3\alpha+\sin\alpha\cdot\cos^2\alpha+2\sin^2\alpha\cdot\cos\alpha-4\cos^3\alpha}{\sin\alpha-\cos\alpha}.$$
\loigiai{
Ta có \allowdisplaybreaks
\begin{eqnarray*}
	K&=&\dfrac{\sin^3\alpha+\sin\alpha\cdot\cos^2\alpha+2\sin^2\alpha\cdot\cos\alpha-4\cos^3\alpha}{\sin\alpha-\cos\alpha}\\
	&=&\dfrac{\cos^3\alpha\left(\tan^3\alpha+\tan\alpha+2\tan^2\alpha-4\right)}{\cos^3\alpha\left(\tan\alpha\cdot(1+\tan^2\alpha)-(1+\tan^2\alpha)\right)}\\
	&=&\dfrac{\tan^3\alpha+\tan\alpha+2\tan^2\alpha-4}{(\tan\alpha-1)(1+\tan^2\alpha)}\\
	&=&\dfrac{2\sqrt{2}+\sqrt{2}+2\cdot 2-4}{(\sqrt{2}-1)(1+2)}\\
	&=&\dfrac{\sqrt{2}}{\sqrt{2}-1}=2+\sqrt{2}.
\end{eqnarray*}
}
\end{bt}
\subsection{Câu hỏi trắc nghiệm}
\Opensolutionfile{ans}[ans/ans-0D3-5-TN]
\begin{ex}%[0H2Y1-2]%[Nguyễn Tiến]%Câu 1.
	Giá trị của $\cos 60^\circ+\sin 30^\circ$ bằng bao nhiêu?
	\choice
	{$\dfrac{\sqrt{3}}{2}$}
	{$\sqrt{3}$}
	{$\dfrac{\sqrt{3}}{3}$}
	{\True $1$}
	\loigiai{
		Ta có $\cos 60^\circ+\sin 30^\circ=\dfrac{1}{2}+\dfrac{1}{2}=1$.
	}
\end{ex}
\begin{ex}%[0H2Y1-2]%[Nguyễn Tiến]%Câu 2.
	Giá trị của $\tan 30^\circ+\cot 30^\circ$ bằng bao nhiêu?
	\choice
	{\True $\dfrac{4}{\sqrt{3}}$}
	{$\dfrac{1+\sqrt{3}}{3}$}
	{$\dfrac{2}{\sqrt{3}}$}
	{$2$}
	\loigiai{
		Ta có $\tan 30^\circ+\cot 30^\circ=\dfrac{\sqrt{3}}{3}+\sqrt{3}=\dfrac{4\sqrt{3}}{3}$.
	}
\end{ex}
\begin{ex}%[0H2Y1-2]%[Nguyễn Tiến]%Câu 3.
	Trong các đẳng thức sau đây, đẳng thức nào \textbf{sai}?
	\choice
	{$\sin 0^\circ+\cos 0^\circ=1$}
	{$\sin 90^\circ+\cos 90^\circ=1$}
	{$\sin 180^\circ+\cos 180^\circ=-1$}
	{\True $\sin 60^\circ+\cos 60^\circ=1$}
	\loigiai{
		Ta có $\sin 60^\circ=\dfrac{\sqrt{3}}{2}$, $\cos 60^\circ=\dfrac{1}{2}$ nên đẳng thức sai là ``$\sin 60^\circ+\cos 60^\circ=1$''.
	}
\end{ex}
\begin{ex}%[0H2Y1-2]%[Nguyễn Tiến]%Câu 4.
	Trong các khẳng định sau, khẳng định nào \textbf{sai}?
	\choice
	{$\cos 60^\circ=\sin 30^\circ$}
	{\True $\cos 60^\circ=\sin 120^\circ$}
	{$\cos 30^\circ=\sin 120^\circ$}
	{$\sin 60^\circ=-\cos 120^\circ$}
	\loigiai{
		Ta có cặp góc $60^\circ$, $120^\circ$ bù nhau nên khẳng định sai là ``$\cos 60^\circ=\sin 120^\circ$''.
	}
\end{ex}
\begin{ex}%[0H2Y1-2]%[Nguyễn Tiến]%Câu 5.
	Đẳng thức nào sau đây \textbf{sai}?
	\choice
	{$\sin 45^\circ+\sin 45^\circ=\sqrt{2}$}
	{$\sin 30^\circ+\cos 60^\circ=1$}
	{$\sin 60^\circ+\cos 150^\circ=0$}
	{\True $\sin 120^\circ+\cos 30^\circ=0$}
	\loigiai{
		Ta có $\sin 120^\circ=\cos 30^\circ=\dfrac{\sqrt{3}}{2}$ nên đẳng thức sai là ``$\sin 120^\circ+\cos 30^\circ=0$''.
	}
\end{ex}
\begin{ex}%[0H2Y1-2]%[Nguyễn Tiến]%Câu 6.
	Giá trị $\cos 45^\circ+\sin 45^\circ$ bằng bao nhiêu?
	\choice
	{$1$}
	{\True $\sqrt{2}$}
	{$\sqrt{3}$}
	{$0$}
	\loigiai{
		Ta có $\cos 45^\circ=\sin 45^\circ=\dfrac{\sqrt{2}}{2}$ nên $\cos 45^\circ+\sin 45^\circ=\sqrt{2}$.
	}
\end{ex}
\begin{ex}%[0H2Y1-2]%[Nguyễn Tiến]%Câu 7.
	Trong các đẳng thức sau, đẳng thức nào \textbf{đúng}?
	\choice
	{$\sin\left( 180^\circ-\alpha\right)=-\cos\alpha$}
	{$\sin\left(180^\circ-\alpha\right)=-\sin\alpha$}
	{\True $\sin\left(180^\circ-\alpha\right)=\sin\alpha$}
	{$\sin\left(180^\circ-\alpha\right)=\cos\alpha$}
	\loigiai{
		Theo tính chất của cặp góc bù nhau thì ``$\sin\left(180^\circ-\alpha\right)=\sin\alpha$''.
	}
\end{ex}
\begin{ex}%[0H2Y1-2]%[Nguyễn Tiến]%Câu 8.
	Trong các đẳng thức sau, đẳng thức nào \textbf{sai}?
	\choice
	{\True $\sin 0^\circ+\cos 0^\circ=0$}
	{$\sin 90^\circ+\cos 90^\circ=1$}
	{$\sin 180^\circ+\cos 180^\circ=-1$}
	{$\sin 60^\circ+\cos 60^\circ=\dfrac{\sqrt{3}+1}{2}$}
	\loigiai{
		Ta có $\sin 0^\circ=0$, $\cos 0^\circ=1$ nên đẳng thức sai là ``$\sin 0^\circ+\cos 0^\circ=0$''.
	}
\end{ex}
\begin{ex}%[0H2Y1-2]%[Nguyễn Tiến]%Câu 9.
	Cho $\alpha$ là góc tù. Điều khẳng định nào sau đây là \textbf{đúng}?
	\choice
	{$\sin\alpha<0$}
	{$\cos\alpha>0$}
	{\True $\tan\alpha<0$}
	{$\cot\alpha>0$}
	\loigiai{
		Góc tù có điểm biểu diễn thuộc góc phần tư thứ II, suy ra $\tan\alpha<0$.
	}
\end{ex}
\begin{ex}%[0H2B1-2]%[Nguyễn Tiến]%Câu 10.
	Giá trị của $E=\sin 36^\circ\cos 6^\circ-\sin 126^\circ\cos 84^\circ$ là
	\choice
	{\True $\dfrac{1}{2}$}
	{$\dfrac{\sqrt{3}}{2}$}
	{$1$}
	{$-1$}
	\loigiai{
		Ta có
		\allowdisplaybreaks
		\begin{eqnarray*}
			E&= & \sin 36^\circ\cos 6^\circ-\sin\left(90^\circ+36^\circ\right)\cos\left(90^\circ-6^\circ\right)\\
			&= & \sin 36^\circ\cos 6^\circ-\cos 36^\circ\sin 6^\circ=\sin 30^\circ=\dfrac{1}{2}.
		\end{eqnarray*}
	}
\end{ex}
\begin{ex}%[0H2B1-2]%[Nguyễn Tiến]%Câu 11.
	Giá trị của biểu thức $A=\sin^2 51^\circ+\sin^2 55^\circ+\sin^2 39^\circ+\sin^2 35^\circ$ là
	\choice
	{$3$}
	{$4$}
	{$1$}
	{\True $2$}
	\loigiai{
		Ta có
		\allowdisplaybreaks
		\begin{eqnarray*}
			A&= & \left(\sin^2 51^\circ+\sin^2 39^\circ\right)+\left(\sin^2 55^\circ+\sin^2 35^\circ\right)\\
			&= & \left(\sin^2 51^\circ+\cos^2 51^\circ\right)+\left(\sin^2 55^\circ+\cos^2 55^\circ\right)=2.
		\end{eqnarray*}
	}
\end{ex}
\begin{ex}%[0H2K1-2]%[Nguyễn Tiến]%Câu 12.
	Giá trị của biểu thức $A=\tan 1^\circ\tan 2^\circ\tan 3^\circ\cdots\tan 88^\circ\tan 89^\circ$ là
	\choice
	{$0$}
	{$2$}
	{$3$}
	{\True $1$}
	\loigiai{
		Ta có $A=\left(\tan 1^\circ\cdot\tan 89^\circ\right)\cdot\left(\tan 2^\circ\cdot\tan 88^\circ\right)\cdots\left(\tan 44^\circ\cdot\tan 46^\circ\right)\cdot\tan 45^\circ=1$.
	}
\end{ex}
\begin{ex}%[0H2K1-2]%[Nguyễn Tiến]%Câu 13.
	Tổng $\sin^2 2^\circ+\sin^2 4^\circ+\sin^2 6^\circ+\cdots +\sin^2 84^\circ+\sin^2 86^\circ+\sin^2 88^\circ$ bằng
	\choice
	{$21$}
	{$23$}
	{\True $22$}
	{$24$}
	\loigiai{
		Ta có
		\allowdisplaybreaks
		\begin{eqnarray*}
			S&= & \sin^2 2^\circ+\sin^2 4^\circ+\sin^2 6^\circ+\cdots +\sin^2 84^\circ+\sin^2 86^\circ+\sin^2 88^\circ\\
			&= & \left(\sin^2 2^\circ+\sin^2 88^\circ\right)+\left(\sin^2 4^\circ+\sin^2 86^\circ\right)+\cdots +\left(\sin^2 44^\circ+\sin^2 46^\circ\right)\\
			&= & \left(\sin^2 2^\circ+\cos^2 2^\circ\right)+\left(\sin^2 4^\circ+\cos^2 4^\circ\right)+\cdots +\left(\sin^2 44^\circ+\cos^2 44^\circ\right)=22.
		\end{eqnarray*}
	}
\end{ex}
\begin{ex}%[0H2K1-2]%[Nguyễn Tiến]%Câu 14.
	Giá trị của $A=\tan 5^\circ\cdot\tan 10^\circ\cdot\tan 15^\circ\cdots\tan 80^\circ\cdot\tan 85^\circ$ là
	\choice
	{$2$}
	{\True $1$}
	{$0$}
	{$-1$}
	\loigiai{
		Ta có
		\allowdisplaybreaks
		\begin{eqnarray*}
			A&= & \left(\tan 5^\circ\cdot\tan 85^\circ\right)\cdot\left(\tan 10^\circ\cdot\tan 80^\circ\right)\cdots\left(\tan 40^\circ\tan 50^\circ\right)\cdot\tan 45^\circ\\
			&= & \left(\tan 5^\circ\cdot\cot 5^\circ\right)\cdot\left(\tan 10^\circ\cdot\cot 10^\circ\right)\cdots\left(\tan 40^\circ\cot 40^\circ\right)\cdot\tan 45^\circ =1.
		\end{eqnarray*}
	}
\end{ex}
\begin{ex}%[0H2B1-2]%[Nguyễn Tiến]%Câu 15.
	Giá trị của $B=\cos^2 73^\circ+\cos^2 87^\circ+\cos^2 3^\circ+\cos^2 17^\circ$ là
	\choice
	{$\sqrt{2}$}
	{\True $2$}
	{$-2$}
	{$1$}
	\loigiai{
		Ta có 
		\allowdisplaybreaks
		\begin{eqnarray*}
			B&= & \left(\cos^2 73^\circ+\cos^2 17^\circ\right)+\left(\cos^2 87^\circ+\cos^2 3^\circ\right)\\
			&= & \left(\cos^2 73^\circ+\sin^2 73^\circ\right)+\left(\cos^2 87^\circ+\sin^2 87^\circ\right)=2.
		\end{eqnarray*}
	}
\end{ex}
\begin{ex}%Câu 1.%[Nguyễn Chiến Thắng - TLDH7]%[0H2K1-2]
	Cho $\cos x=\dfrac 12$. Tính biểu thức $P=3\sin^2x+4\cos^2x$ 
	\choice
	{\True $\dfrac{13}{4}$}
	{$\dfrac{7}{4}$}
	{$\dfrac{11}{4}$}
	{$\dfrac{15}{4}$}
	\loigiai{
		Ta có $P=3\sin^2x+4\cos^2x=3\left(\sin^2x+\cos^2x\right)+\cos^2x=3+\left(\dfrac 12\right)^2=\dfrac{13}4$.}
\end{ex}
\begin{ex}%Câu 2.%[Nguyễn Chiến Thắng - TLDH7]%[0H2K1-2]
	Biết $\cos\alpha=\dfrac 13$. Giá trị đúng của biểu thức $P=\sin^2\alpha+3\cos^2\alpha$ là 
	\choice
	{$\dfrac{1}{3}$}
	{$\dfrac{10}{9}$}
	{\True $\dfrac{11}{9}$}
	{$\dfrac{4}{3}$}
	\loigiai{
		Ta có:	$\cos\alpha=\dfrac 13\Rightarrow P=\sin^2\alpha+3cos^2\alpha=\left(\sin^2\alpha+cos^2\alpha\right)+2cos^2\alpha=1+2cos^2\alpha=\dfrac{11}9$.}
\end{ex}
\begin{ex}%Câu 3.%[Nguyễn Chiến Thắng - TLDH7]%[0H2K1-2]
	Cho biết $\tan\alpha=\dfrac{1}{2}$. Tính $\cot\alpha$. 
	\choice
	{\True $\cot\alpha=2$}
	{$\cot\alpha=\sqrt{2}$}
	{$\cot\alpha=\dfrac{1}{4}$}
	{$\cot\alpha=\dfrac{1}{2}$}
	\loigiai{
		Ta có	$\tan\alpha\cdot\cot\alpha=1\Rightarrow\cot\alpha=\dfrac{1}{\tan\alpha}=2$.}
\end{ex}
\begin{ex}%Câu 4.%[Nguyễn Chiến Thắng - TLDH7]%[0H2K1-2]
	Cho biết $\cos\alpha=-\dfrac{2}{3}$ và $0<\alpha<\dfrac{\pi}{2}$. Tính $\tan\alpha$?
	\choice
	{$\dfrac{5}{4}$}
	{$-\dfrac{5}{2}$}
	{$\dfrac{\sqrt{5}}{2}$}
	{\True $-\dfrac{\sqrt{5}}{2}$}
	\loigiai{
		Do $0<\alpha<\dfrac{\pi}{2}\Rightarrow\tan\alpha<0$. \\
		Ta có: $1+\tan^2\alpha=\dfrac 1{\cos^2\alpha}\Leftrightarrow\tan^2\alpha=\dfrac 54\Rightarrow\tan\alpha=-\dfrac{\sqrt 5}2$.}
\end{ex}
\begin{ex}%Câu 5.%[Nguyễn Chiến Thắng - TLDH7]%[0H2K1-2]
	Cho $\alpha$ là góc tù và $\sin\alpha=\dfrac{5}{13}$. Giá trị của biểu thức $3\sin\alpha+2\cos\alpha$ là
	\choice
	{$3$}
	{\True $-\dfrac{9}{13}$}
	{$-3$}
	{$\dfrac{9}{13}$}
	\loigiai{
		Ta có $\cos^2\alpha=1-\sin^2\alpha=\dfrac{144}{169}\Rightarrow\cos\alpha=\pm\dfrac{12}{13}$.\\
		Do $\alpha$ là góc tù nên $\cos\alpha<0$, từ đó $\cos\alpha=-\dfrac{12}{13}$.\\
		Như vậy $3\sin\alpha+2\cos\alpha=3\cdot\dfrac{5}{13}+2\left(-\dfrac{12}{13}\right)=-\dfrac{9}{13}$.}
\end{ex}
\begin{ex}%Câu 6.%[Nguyễn Chiến Thắng - TLDH7]%[0H2K1-2]
	Cho biết $\sin\alpha+\cos\alpha=a$. Giá trị của $\sin\alpha\cdot\cos\alpha$ bằng bao nhiêu?
	\choice
	{$\sin\alpha\cdot\cos\alpha=a^2$}
	{$\sin\alpha\cdot\cos\alpha=2a$}
	{$\sin\alpha\cdot\cos\alpha=\dfrac{1-a^2}{2}$}
	{\True $\sin\alpha\cdot\cos\alpha=\dfrac{a^2-1}{2}$}
	\loigiai{
		$a^2=\left(\sin\alpha+\cos\alpha\right)^2=1+2\sin\alpha\cos\alpha\Rightarrow\sin\alpha\cos\alpha=\dfrac{a^2-1}{2}$.}
\end{ex}
\begin{ex}%Câu 7.%[Nguyễn Chiến Thắng - TLDH7]%[0H2K1-2]
	Cho biết $\cos\alpha=-\dfrac{2}{3}$. Tính giá trị của biểu thức $E=\dfrac{\cot\alpha+3\tan\alpha}{2\cot\alpha+\tan\alpha}$?
	\choice
	{$-\dfrac{19}{13}$}
	{\True $\dfrac{19}{13}$}
	{$\dfrac{25}{13}$}
	{$-\dfrac{25}{13}$}
	\loigiai{
		Ta có	$E=\dfrac{\cot\alpha+3\tan\alpha}{2\cot\alpha+\tan\alpha}=\dfrac{1+3\tan^2\alpha}{2+\tan^2\alpha}=\dfrac{3\left(\tan^2\alpha+1\right)-2}{1+\left(1+\tan^2\alpha\right)}=\dfrac{\dfrac 3{\cos^2\alpha}-2}{\dfrac 1{\cos^2\alpha}+1}=\dfrac{3-2\cos^2\alpha}{1+\cos^2\alpha}=\dfrac{19}{13}$.}
\end{ex}
\begin{ex}%Câu 8.%[Nguyễn Chiến Thắng - TLDH7]%[0H2K1-2]
	Cho biết $\cot\alpha=5$. Tính giá trị của $E=2\cos^2\alpha+5\sin\alpha\cos\alpha+1$?
	\choice
	{$\dfrac{10}{26}$}
	{$\dfrac{100}{26}$}
	{$\dfrac{50}{26}$}
	{\True $\dfrac{101}{26}$}
	\loigiai{
		$E=\sin^2\alpha\left(2\cot^2\alpha+5\cot\alpha+\dfrac{1}{\sin^2\alpha}\right)=\dfrac{1}{1+\cot^2\alpha}\left(3\cot^2\alpha+5\cot\alpha+1\right)=\dfrac{101}{26}$.}
\end{ex}
\begin{ex}%Câu 9.%[Nguyễn Chiến Thắng - TLDH7]%[0H2K1-2]
	Cho $\cot\alpha=\dfrac{1}{3}$. Giá trị của biểu thức $A=\dfrac{3\sin\alpha+4\cos\alpha}{2\sin\alpha-5\cos\alpha}$ là 
	\choice
	{$-\dfrac{15}{13}$}
	{$-13$}
	{$\dfrac{15}{13}$}
	{\True $13$}
	\loigiai{
		Ta có	$A=\dfrac{3\sin\alpha+4\sin\alpha\cdot\cot\alpha}{2\sin\alpha-5\sin\alpha\cdot\cot\alpha}=\dfrac{3+4\cot\alpha}{2-5\cot\alpha}=13$.}
\end{ex}
\begin{ex}%Câu 10.%[Nguyễn Chiến Thắng - TLDH7]%[0H2K1-2]
	Cho biết $\cos\alpha=-\dfrac{2}{3}$. Giá trị của biểu thức $E=\dfrac{\cot\alpha-3\tan\alpha}{2\cot\alpha-\tan\alpha}$ bằng bao nhiêu?
	\choice
	{$-\dfrac{25}{3}$}
	{$-\dfrac{11}{13}$}
	{\True $-\dfrac{11}{3}$}
	{$-\dfrac{25}{13}$}
	\loigiai{
		Ta có	$E=\dfrac{\cot\alpha-3\tan\alpha}{2\cot\alpha-\tan\alpha}=\dfrac{1-3\tan^2\alpha}{2-\tan^2\alpha}=\dfrac{4-3\left(\tan^2\alpha+1\right)}{3-\left(1+\tan^2\alpha\right)}=\dfrac{4-\dfrac{3}{\cos^2\alpha}}{3-\dfrac{1}{\cos^2\alpha}}=\dfrac{4\cos^2\alpha-3}{3\cos^2\alpha-1}=-\dfrac{11}{3}$.}
\end{ex}
\begin{ex}%Câu 11.%[Nguyễn Chiến Thắng - TLDH7]%[0H2K1-2]
	Biết $\sin a+\cos a=\sqrt{2}$. Hỏi giá trị của $\sin^4a+\cos^4a$ bằng bao nhiêu?
	\choice
	{$\dfrac{3}{2}$}
	{\True $\dfrac{1}{2}$}
	{$-1$}
	{$0$}
	\loigiai{
		Ta có: $\sin a+\cos a=\sqrt{2}\Rightarrow 2=\left(\sin a+\cos a\right)^2\Rightarrow\sin a\cdot\cos a=\dfrac{1}{2}$.\\
		$\sin^4a+\cos^4a=\left(\sin^2a+\cos^2a\right)-2\sin^2a\cos^2a=1-2\left(\dfrac{1}{2}\right)^2=\dfrac{1}{2}$.}
\end{ex}
\begin{ex}%Câu 12.%[Nguyễn Chiến Thắng - TLDH7]%[0H2K1-2]
	Cho $\tan\alpha+\cot\alpha=m$. Tìm $m$ để $\tan^2\alpha+\cot^2\alpha=7$. 
	\choice
	{$m=9$}
	{$m=3$}
	{$m=-3$}
	{\True $m=\pm 3$}
	\loigiai{
		Ta có	$7=\tan^2\alpha+\cot^2\alpha=\left(\tan\alpha+\cot\alpha\right)^2-2\Rightarrow m^2=9\Leftrightarrow m=\pm 3$.}
\end{ex}
\begin{ex}%Câu 13.%[Nguyễn Chiến Thắng - TLDH7]%[0H2K1-2]
	Cho biết $3\cos\alpha-\sin\alpha=1$, $0^{\circ}<\alpha<90^{\circ}$ Giá trị của $\tan\alpha$ bằng
	\choice
	{\True $\tan\alpha=\dfrac{4}{3}$}
	{$\tan\alpha=\dfrac{3}{4}$}
	{$\tan\alpha=\dfrac{4}{5}$}
	{$\tan\alpha=\dfrac{5}{4}$}
	\loigiai{
		Ta có $3\cos\alpha-\sin\alpha=1\Leftrightarrow 3\cos\alpha=\sin\alpha+1\to 9\cos^2\alpha=\left(\sin\alpha+1\right)^2$ \\
		$ \Leftrightarrow 9\cos^2\alpha=\sin^2\alpha+2\sin\alpha+1\Leftrightarrow 9\left(1-\sin^2\alpha\right)=\sin^2\alpha+2\sin\alpha+1 $ \\
		$ \Leftrightarrow 10\sin^2\alpha+2\sin\alpha-8=0\Leftrightarrow\hoac{&\sin\alpha=-1\\&\sin\alpha=\dfrac{4}{5}} $.
		\begin{itemize}
			\item 	$\sin\alpha=-1 $: không thỏa mãn vì $0^{\circ}<\alpha<90^{\circ}$.
			\item 	$\sin\alpha=\dfrac{4}{5}\Rightarrow\cos\alpha=\dfrac{3}{5}\Rightarrow\tan\alpha=\dfrac{\sin\alpha}{\cos\alpha}=\dfrac{4}{3}$.
		\end{itemize}
	}
\end{ex}
\begin{ex}%Câu 14.%[Nguyễn Chiến Thắng - TLDH7]%[0H2K1-2]
	Cho biết $2\cos\alpha+\sqrt{2}\sin\alpha=2$, $0^{\circ}<\alpha<90^{\circ}$. Tính giá trị của $\cot\alpha$. 
	\choice
	{$\cot\alpha=\dfrac{\sqrt{5}}{4}$}
	{$\cot\alpha=\dfrac{\sqrt{3}}{4}$}
	{\True $\cot\alpha=\dfrac{\sqrt{2}}{4}$}
	{$\cot\alpha=\dfrac{\sqrt{2}}{2}$}
	\loigiai{
		Ta có $2\cos\alpha+\sqrt{2}\sin\alpha=2\Leftrightarrow\sqrt{2}\sin\alpha=2-2\cos\alpha\to 2\sin^2\alpha=\left(2-2\cos\alpha\right)^2$.\\
		$\begin{aligned}&\Leftrightarrow 2\sin^2\alpha=4-8\cos\alpha+4\cos^2\alpha\Leftrightarrow 2\left(1-\cos^2\alpha\right)=4-8\cos\alpha+4\cos^2\alpha\\&\Leftrightarrow 6\cos^2\alpha-8\cos\alpha+2=0\Leftrightarrow\hoac{&\cos\alpha=1\\&\cos\alpha=\dfrac{1}{3}}.\end{aligned}$ 
		\begin{itemize}
			\item 	$\cos\alpha=1$: không thỏa mãn vì $0^{\circ}<\alpha<90^{\circ}$.
			\item 	$\cos\alpha=\dfrac{1}{3}\Rightarrow\sin\alpha=\dfrac{2\sqrt{2}}{3}\Rightarrow\cot\alpha=\dfrac{\cos\alpha}{\sin\alpha}=\dfrac{\sqrt{2}}{4}$.
		\end{itemize}
	}
\end{ex}
\begin{ex}%Câu 15.%[Nguyễn Chiến Thắng - TLDH7]%[0H2G1-2]
	Cho biết $\cos\alpha+\sin\alpha=\dfrac{1}{3}$. Giá trị của $P=\sqrt{\tan^2\alpha+\cot^2\alpha}$ bằng bao nhiêu?
	\choice
	{$P=\dfrac{5}{4}$}
	{\True $P=\dfrac{7}{4}$}
	{$P=\dfrac{9}{4}$}
	{$P=\dfrac{11}{4}$}
	\loigiai{
		Ta có $\cos\alpha+\sin\alpha=\dfrac{1}{3}\to\left(\cos\alpha+\sin\alpha\right)^2=\dfrac{1}{9}\Leftrightarrow 1+2\sin\alpha\cos\alpha=\dfrac{1}{9}\Leftrightarrow\sin\alpha\cos\alpha=-\dfrac{4}{9}$.\\
		Ta có $P=\sqrt{\tan^2\alpha+\cot^2\alpha}=\sqrt{\left(\tan\alpha+\cot\alpha\right)^2-2\tan\alpha\cot\alpha}=\sqrt{\left(\dfrac{\sin\alpha}{\cos\alpha}+\dfrac{\cos\alpha}{\sin\alpha}\right)^2-2}$.\\
		$=\sqrt{\left(\dfrac{\sin^2\alpha+\cos^2\alpha}{\sin\alpha\cos\alpha}\right)^2-2}=\sqrt{\left(\dfrac{1}{\sin\alpha\cos\alpha}\right)^2-2}=\sqrt{\left(-\dfrac{9}{4}\right)^2-2}=\dfrac{7}{4}$.}
\end{ex}
\begin{ex}%Câu 16.%[Nguyễn Chiến Thắng - TLDH7]%[0H2G1-2]
	Cho biết $\sin\alpha-\cos\alpha=\dfrac{1}{\sqrt{5}}$. Giá trị của $P=\sqrt{\sin^4\alpha+\cos^4\alpha}$ bằng bao nhiêu?
	\choice
	{$P=\dfrac{\sqrt{15}}{5}$}
	{\True $P=\dfrac{\sqrt{17}}{5}$}
	{$P=\dfrac{\sqrt{19}}{5}$}
	{$P=\dfrac{\sqrt{21}}{5}$}
	\loigiai{
		Ta có $\sin\alpha-\cos\alpha=\dfrac{1}{\sqrt{5}}\to\left(\sin\alpha-\cos\alpha\right)^2=\dfrac{1}{5}\Leftrightarrow 1-2\sin\alpha\cos\alpha=\dfrac{1}{5}\Leftrightarrow\sin\alpha\cos\alpha=\dfrac{2}{5}$.\\
		$P=\sqrt{\sin^4\alpha+\cos^4\alpha}=\sqrt{\left(\sin^2\alpha+\cos^2\alpha\right)^2-2\sin^2\alpha\cos^2\alpha} =\sqrt{1-2\left(\sin\alpha cos\alpha\right)^2}=\dfrac{\sqrt{17}}{5}$.}
\end{ex}
\Closesolutionfile{ans}
