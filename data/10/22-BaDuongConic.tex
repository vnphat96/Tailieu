
\section{BA ĐƯỜNG CONIC}
\subsection{Tóm tắt lí thuyết}



\subsection{Các dạng toán}
\begin{dang}{Bài toán liên quan đến elip}
	%{\bf Quy tắc 1. Lập bảng biến thiên suy ra kết luận về cực trị}
	\begin{itemize}
		\item Các công thức, đồ thị liên quan đến elip.
		\item 
	\end{itemize}
\end{dang}

\begin{bt}%[BG2-2022-Huỳnh Xuân Tín]
	Mặt Trăng chuyển động theo một quỹ đạo là hình elip nhận tâm Trái Đất là một tiêu điểm. Các khoảng cách lớn nhất và nhỏ nhất từ các vị trí của Mặt Trăng đến tâm Trái Đất tương ứng là $400000$ km và $363000$ km (theo nssdc.gsfc.nasa.gov). Tìm tâm sai của quỹ đạo elip.
	\loigiai{
\immini{Gọi phương trình quỹ đạo elip có dạng $\dfrac{x^2}{a^2}+\dfrac{y^2}{b^2}=1$.\\
	Theo tính chất elip và theo giả thiết thì $A_1F_=363000$ và $A_2F_1=400000$.\\
Khi đó $\heva{&a-c=363000\\&a+c=400000}\Rightarrow\heva{&a=381500\\&c=18500.}$\\
Ta có $b=\sqrt{a^2-b^2}=381051{,}177$.\\
Vậy tâm sai của quỹ đạo là $e=\dfrac{c}{a}=0{,}0485499$.}{\begin{tikzpicture}[scale=1.2, font=\footnotesize, line join=round, line cap=round,>=stealth]
		\def \xmin{-2.7};
		\def \xmax{2.7};
		\def \ymin{-1.9};
		\def \ymax{2.0};
		\def \a{2.3};
		\def \b{1.5};
		\pgfmathsetmacro{\c}{sqrt((\a)^2-(\b)^2)};
		\draw[->] (\xmin, 0.) -- (\xmax+0.5,0.) node[anchor=north] {$x$};
		\draw[->] (0.,\ymin) -- (0.,\ymax) node[anchor=west] {$y$};
		%\clip(\xmin,\ymin) rectangle (\xmax,\ymax);
		\draw (0,0) ellipse ({\a} and {\b});
		\coordinate[label=above: Mặt Trăng] (M) at ($(0,0) + (60: {\a}  and {\b})$);
		\draw ({-\c},0) node[above, xshift=10pt] {$F_1$}circle(1pt) --(M)circle(1pt)--({\c},0)node[below, xshift=-5pt] {$F_2$}circle(1pt);
		\draw[fill=black] (0,0) circle (1pt) node[below left] {$O$};
			\draw[fill=black] ({-\a},0) circle (1pt) node[below left] {$A_1$};
			\draw[fill=black] ({\a},0) circle (1pt) node[below right] {$A_2$};
				\draw[fill=black] ({-\c},0) circle (1pt) node[below] {Trái Đất};
	\end{tikzpicture}
}}
\end{bt}


\begin{bt}%[BG2-2022-Huỳnh Xuân Tín]
	Với tâm sai khoảng $0{,}244$ quỹ đạo của sao Diêm Vương "dẹt" so với quỹ đạo của tám hình tinh trong hệ Mặt Trời. Nửa độ dài trục lớn của elip quỹ đạo là khoảng $590635\cdot 10^6$ km. Tìm khoảng cách gần nhất và khoảng cách xa nhất giữa sao Diêm Vương và tâm Mặt Trời ( tiêu điểm cuat quỹ đạo) (theo nssdc.gsfc.nasa.gov). 
	\loigiai{
Theo giả thiết ta có
 $$\heva{&e=0{,}244\\&a=590635\cdot 10^6}\Rightarrow c=e\cdot a=0{,}244\cdot 590635\cdot 10^6=144114{,}94\cdot 10^6.$$		
 Vậy khoảng cách gần nhất giữa sao Diêm Vương và tâm Mặt Trời $a-c=590635\cdot 10^6-144114{,}94\cdot 10^6=446520{,}06\cdot 10^6$ km.\\
 Khoảng cách xa nhất giữa sao Diêm Vương và tâm Mặt Trời $a+-c=590635\cdot 10^6+144114{,}94\cdot 10^6=734749{,}94\cdot 10^6$ km.
	}
\end{bt}


\begin{bt}%[BG2-2022-Huỳnh Xuân Tín]
\immini{Một phòng thì thầm có trần vòm elip với hai tiêu điểm ở độ cao $1{,6}$ m (so với mặt  sàn) và cách nhau $16$ m. Đỉnh của mái vòm cao $7{,}6$ m (hình bên). Hỏi âm thanh thì thầm từ một tiêu điểm thì sau bao nhiêu giây đến được tiêu điểm kia? Biết vận tốc âm thanh là $343{,}2$ m/s và làm tròn đáp số tới $4$ chữ số sau dấu phẩy.}{\begin{tikzpicture}[scale=1.2, font=\footnotesize, line join=round, line cap=round,>=stealth]
		\def \xmin{-2.7};
		\def \xmax{2.7};
		\def \ymin{-1.9};
		\def \ymax{2.0};
		\def \a{2.3};
		\def \b{1.5};
		\pgfmathsetmacro{\c}{sqrt((\a)^2-(\b)^2)};
		%	\draw[->] (\xmin, 0.) -- (\xmax+0.5,0.) node[anchor=north] {$x$};
		%	\draw[->] (0.,\ymin) -- (0.,\ymax) node[anchor=west] {$y$};
		%\clip(\xmin,\ymin) rectangle (\xmax,\ymax);
		%\draw (0,0) ellipse ({\a} and {\b});
		\coordinate[] (M) at ($(0,0) + (60: {\a}  and {\b})$);
		\draw[->] ({-\c},0)circle(1pt) --(M)circle(1pt)--({\c},0)circle(1pt);
	%	\draw[fill=black] (0,0) circle (1pt) node[below left] {$O$};
		%\draw[fill=black] ({-\a},0) circle (1pt) node[above left] {$A_1$};
	%	\draw[fill=black] ({\a},0) circle (1pt) node[above right] {$A_2$};
		\draw ({\a},0) arc (0:180:2.3cm and 1.5cm);
		\draw (-3.3,0)--(2.7,0);
		\draw (-3.3,-1)--(2.7,-1);
		\draw[<->][dashed] (-2.7,0)--(-2.7,-1); 
		\draw[fill=black] (-2.7,-0.5) node[left] {$1{,}6$m};	
		\draw[<->][dashed] (0,-1)--(0,1.5); 
		\draw[fill=black] (0,-0.5) node[right] {$7{,}6$m};			
\end{tikzpicture}}
	\loigiai{
\immini{Gọi phương trình quỹ đạo elip có dạng $\dfrac{x^2}{a^2}+\dfrac{y^2}{b^2}=1$.\\
	Ta có $2c=16\Rightarrow c=8$ và $b=7{,}6-1{,}6=6\Rightarrow a=\sqrt{b^2+c^2}=10$.\\
	Gọi một điểm $M$ bất kì trên mái vòm. Khi đó đường đi của âm thanh là $MF_1+MF_2=2a=20$.\\
	Vậy thời gian âm  thanh thì thầm từ một tiêu điểm đến được tiêu điểm kia là $20:343{,}2=0,0583$.
}{\begin{tikzpicture}[scale=1.2, font=\footnotesize, line join=round, line cap=round,>=stealth]
		\def \xmin{-2.7};
		\def \xmax{2.7};
		\def \ymin{-1.9};
		\def \ymax{2.0};
		\def \a{2.3};
		\def \b{1.5};
		\pgfmathsetmacro{\c}{sqrt((\a)^2-(\b)^2)};
	%	\draw[->] (\xmin, 0.) -- (\xmax+0.5,0.) node[anchor=north] {$x$};
	%	\draw[->] (0.,\ymin) -- (0.,\ymax) node[anchor=west] {$y$};
		%\clip(\xmin,\ymin) rectangle (\xmax,\ymax);
		%\draw (0,0) ellipse ({\a} and {\b});
		\coordinate[label=above: $M$] (M) at ($(0,0) + (60: {\a}  and {\b})$);
		\draw[->] ({-\c},0) node[below, xshift=10pt] {$F_1$}circle(1pt) --(M)circle(1pt)--({\c},0)node[below, xshift=-5pt] {$F_2$}circle(1pt);
		\draw[fill=black] (0,0) circle (1pt) node[below left] {$O$};
		\draw[fill=black] ({-\a},0) circle (1pt) node[above left] {$A_1$};
		\draw[fill=black] ({\a},0) circle (1pt) node[above right] {$A_2$};
		\draw ({\a},0) arc (0:180:2.3cm and 1.5cm);
\draw (-3.3,0)--(2.7,0);
\draw (-3.3,-1)--(2.7,-1);
\draw[<->][dashed] (-2.7,0)--(-2.7,-1); 
\draw[fill=black] (-2.7,-0.5) node[left] {$1{,}6$m};	
\draw[<->][dashed] (0,-1)--(0,1.5); 
\draw[fill=black] (0,-0.5) node[right] {$7{,}6$m};			
	\end{tikzpicture}
}		
	}
\end{bt}

\begin{dang}{Bài toán liên quan đến hypebol}
	%{\bf Quy tắc 1. Lập bảng biến thiên suy ra kết luận về cực trị}
	\begin{itemize}
		\item Các công thức, đồ thị liên quan đến hypebol. 
	\end{itemize}
\end{dang}

\begin{bt}%[BG2-2022-Huỳnh Xuân Tín]
\immini{Một sao chổi đi qua hệ Mặt Trời theo quỹ đạo là một nhánh hypebol nhận tâm Mặt Trời là một tiêu điểm, khoảng cách gần nhất từ sao chổi này đến tâm Mặt Trời là $3\cdot 10^8$ km và tâm sai của quỹ đạo hypebol là $3{,}6$ (hình bên). Hãy lập phương trình chính tắc của hypebol chứa quỹ đạo, với một đơn vị đo trên mặt phẳng tọa độ tương ứng với $10^8$ km trên thực tế.}{\begin{tikzpicture}[scale=0.7, font=\footnotesize, line join=round, line cap=round, >=stealth,
		declare function={f(\a)=sqrt((1+(\a)^2/0.4671)*0.5329);},
		declare function={g(\b)=sqrt((1+(\b)^2/6.8684)*2.1316);}
		]
		%\draw[->] (-1,0)--(6,0) node [below]{$x$};
		%\draw[->] (0,-3)--(0,3) node [left]{$y$};
		\node at (0,0) [below left]{$O$};
		\fill (1.65197,1.38742) circle (1.5pt) node[above right]{Sao chổi};
		\draw[dashed] (3,0)node[above right]{$F_1$}--(1.65197,1.38742)--(0,0);
		\draw[fill=black] (3,0)circle (1pt) node[below] {Mặt Trời};
		\clip (-3.9,-2.9) rectangle (3.9,2.9);
		%	\draw[smooth] plot[domain=-3:3,variable=\a] ({-f(\a)},{\a});
		%	\draw[smooth] plot[domain=-3:3,variable=\a] ({f(\a)},{\a});
		%\draw[smooth] plot[domain=-3:3,variable=\b] ({-g(\b)},{\b});
		\draw[smooth] plot[domain=-3:3,variable=\b] ({g(\b)},{\b});
		%	\node at (3,2.3) [below]{$(H_1)$};
		%\node at (1.5,2.9) [below]{$(H_2)$};
		
\end{tikzpicture}}		
	\loigiai{
\immini{Chọn hệ trục tọa độ như hình vẽ (với một đơn vị đo trên mặt phẳng tọa độ tương ứng với $10^8$ km trên thực tế). Khi đó quỹ đạo có phương trình dạng $\dfrac{x^2}{a^2}-\dfrac{y^2}{b^2}=1$.\\
Ta có $\heva{&c-a=3\\&e=3{,}6}\Rightarrow\heva{&c-a=3\\&\dfrac{c}{a}=3{,}6}\Rightarrow\heva{&c=18\\&a=5.}$\\
Khi đó $b=\sqrt{c^2-a^2}=\sqrt{18^2-5^2}=\sqrt{299}$.\\
Vậy phương trình quỹ đạo hypebol cần tìm là $\dfrac{x^2}{25}-\dfrac{y^2}{299}=1$.}{\begin{tikzpicture}[scale=0.7, font=\footnotesize, line join=round, line cap=round, >=stealth,
		declare function={f(\a)=sqrt((1+(\a)^2/0.4671)*0.5329);},
		declare function={g(\b)=sqrt((1+(\b)^2/6.8684)*2.1316);}
		]
		\draw[->] (-1,0)--(6,0) node [below]{$x$};
		\draw[->] (0,-3)--(0,3) node [left]{$y$};
		\node at (0,0) [below left]{$O$};
		\fill (1.65197,1.38742) circle (1.5pt) node[above right]{Sao chổi};
		\draw[dashed] (3,0)node[above right]{$F_1$}--(1.65197,1.38742)--(0,0);
			\draw[fill=black] (3,0)circle (1pt) node[below] {Mặt Trời};
		\clip (-3.9,-2.9) rectangle (3.9,2.9);
	%	\draw[smooth] plot[domain=-3:3,variable=\a] ({-f(\a)},{\a});
	%	\draw[smooth] plot[domain=-3:3,variable=\a] ({f(\a)},{\a});
		%\draw[smooth] plot[domain=-3:3,variable=\b] ({-g(\b)},{\b});
		\draw[smooth] plot[domain=-3:3,variable=\b] ({g(\b)},{\b});
	%	\node at (3,2.3) [below]{$(H_1)$};
	%\node at (1.5,2.9) [below]{$(H_2)$};

\end{tikzpicture}}		
	}
\end{bt}

\begin{bt}%[BG2-2022-Huỳnh Xuân Tín]
\immini{Bốn trạm phát tín hiệu vô tuyến có vị trí $A$, $B$, $C$, $D$ theo thứ tự đó thẳng hàng và cách đều với khoảng cách $200$ km (hình bên). Tại một thời điểm, bốn trạm cùng phát tín hiệu với vận tốc $292000$ km/s. Một tàu thủy nhận được tín hiệu từ trạm $C$ trước $0{,}0005$ s so với tín hiệu từ trạm $B$ và nhận được tín hiệu từ trạm $D$ sớm $0{,} 0001$ s so với tín hiệu từ trạm $A$.}{\begin{tikzpicture}[scale=1.2, font=\footnotesize, line join=round, line cap=round, >=stealth]
		\draw[->] (-4,0)--(4,0) node [below]{$x$};
		\draw[->] (0,-1)--(0,3) node [left]{$y$};
		\node at (0,0) [below left]{$O$};
		\fill (-3,0) circle (1.5pt) node[below]{$A$};
		\fill (-1,0) circle (1.5pt) node[below]{$B$};
		\fill (1,0) circle (1.5pt) node[below]{$C$};
		\fill (3,0) circle (1.5pt) node[below]{$D$};
		\fill (2,2) circle (1.5pt) node[above]{$M$};
		\draw[dashed] (-3,0)--(2,2)--(-1,0) (3,0)--(2,2)--(1,0);
\end{tikzpicture}}
\begin{enumerate}
	\item Tính hiệu các khoảng cách từ tàu đến các trạm $B$, $C$.
	\item Tính hiệu các khoảng cách từ tàu đến các trạm $A$, $D$.
	\item Chọn hệ trục tọa độ $Oxy$ như hình bên(1 đơn vị trên mặt phẳng tọa độ ứng với $100$ km trên thực tế). Hãy lập phương trình chính tắc của hai hypebol đi qua vị trí $M$ của tàu. 
\end{enumerate} 
	\loigiai{
\begin{center}
		\begin{tikzpicture}[scale=1.2, font=\footnotesize, line join=round, line cap=round, >=stealth]
		\draw[->] (-4,0)--(4,0) node [below]{$x$};
		\draw[->] (0,-1)--(0,3) node [left]{$y$};
		\node at (0,0) [below left]{$O$};
		\fill (-3,0) circle (1.5pt) node[below]{$A$};
		\fill (-1,0) circle (1.5pt) node[below]{$B$};
		\fill (1,0) circle (1.5pt) node[below]{$C$};
		\fill (3,0) circle (1.5pt) node[below]{$D$};
		\fill (2,2) circle (1.5pt) node[above]{$M$};
		\draw[dashed] (-3,0)--(2,2)--(-1,0) (3,0)--(2,2)--(1,0);
	\end{tikzpicture}
\end{center}
\begin{enumerate}
	\item Chọn hệ trục như hình trên, và giả sử con tàu đang ở nhánh bên phải trục $Oy$.	Khi đó hai điểm $B$ và $C$ là hai tiêu điểm của một quỹ đạo hypebol. Do đó $MB-MC=2a=2OC=200$ km.
	\item Chọn hệ trục như hình trên, và giả sử con tàu đang ở nhánh bên phải trục $Oy$.	Khi đó hai điểm $A$ và $D$ là hai tiêu điểm của một quỹ đạo hypebol. Do đó $MB-MC=2a=2OD=600$ km.
	\item Chọn hệ trục tọa độ $Oxy$ như hình bên(1 đơn vị trên mặt phẳng tọa độ ứng với $100$ km trên thực tế). 
	\begin{itemize}
		\item Xét hypebol có hai tiêu điểm là $B$, $C$ suy ra $c=1$. Ta có một tàu thủy nhận được tín hiệu từ trạm $C$ trước $0{,}0005$ s so với tín hiệu từ trạm $B$ suy ra
		\[\dfrac{MB}{2920}-\dfrac{MC}{2920}=0{,}0005\Rightarrow MB-MC= \dfrac{73}{50}=2a\Rightarrow a=\dfrac{73}{100}.\]
		Khi đó $b=\sqrt{c^2-a^2}=\dfrac{\sqrt{4671}}{100}$.\\
		Vậy hypebol nhận $B$, $C$ là, tiêu điểm $H_1\colon \dfrac{x^2}{\dfrac{5329}{10000}}-\dfrac{y^2}{\dfrac{4671}{10000}}=1$.
		\item Xét hypebol có hai tiêu điểm là $A$, $D$ suy ra $c=3$. Ta có  nhận được tín hiệu từ trạm $D$ sớm $0{,} 001$ s so với tín hiệu từ trạm $A$ suy ra
		\[\dfrac{MA}{2920}-\dfrac{MD}{2920}=0{,}0001\Rightarrow MB-MC= \dfrac{73}{250}=2a\Rightarrow a=\dfrac{73}{500}.\]
		Khi đó $b=\sqrt{c^2-a^2}=\dfrac{\sqrt{2244671}}{500}$.\\
		Vậy hypebol nhận $A$, $D$ là, tiêu điểm $H_2\colon \dfrac{x^2}{\dfrac{5329}{250000}}-\dfrac{y^2}{\dfrac{2244671}{250000}}=1$.\\
		Khi đó tọa độ điểm $M$ là giao điểm của hai hypebol $H_1$ và $H_2$.
	\end{itemize}
\end{enumerate} 
	
	}
\end{bt}

\begin{dang}{Bài toán liên quan đến parabol}
	%{\bf Quy tắc 1. Lập bảng biến thiên suy ra kết luận về cực trị}
	\begin{itemize}
		\item Các công thức, đồ thị liên quan đến parabol.
	\end{itemize}
\end{dang}

% \begin{bt}%[BG2-2022-Huỳnh Xuân Tín]
% \immini{Xét đèn có bát đáy parabol với kích thước được thể hiện như hình bên. Dây tóc bóng đèn được đặt ở vị trí tiêu điểm. Tính khoảng cách từ dây tóc tới đỉnh bát đáy.}{{\includegraphics[scale=0.8]{"Pic/22-01"}}}
% 	\loigiai{
% \immini{Chọn hệ trục như hình vẽ, parabol có dạng $y^2=2px$.\\
% Ta có $OF=\dfrac{p}{2}$, $AB=30$, $OH=20$.\\
% Khi đó tọa độ điểm $A$ là $A(20;15)$ thuộc parabol nên
% \[15^2=2p\cdot 20\Rightarrow p=\dfrac{45}{8}.\]
% Vậy tiêu điểm của parabol là $P\left(\dfrac{45}{16};0\right)$.\\
% Do đó khoảng cách từ dây tóc tới đỉnh bát đáy là $\dfrac{45}{16}$. }{\begin{tikzpicture}[>=stealth,x=1cm,y=1cm,scale=.6]
% 		\coordinate[label=below left:$\scriptsize O$](O) at (0,0);
% 		\draw[->] (-1,0) -- (6,0) node[below] { $x$};
% 		\draw[->] (0,-4) -- (0,5) node[left] {$y$};
% 		\draw[thick,samples=150,smooth,domain=0:4] plot(\x,{2*\x^.5});
% 		\draw[thick,samples=150,smooth,domain=0:4] plot(\x,{-2*\x^.5});
% 		\fill (3,0)node[above right]{$F$} circle(1.5pt);
% 		\fill (4,4)node[above right]{$A$} circle(1.5pt);
% 			\fill (4,0)node[above right]{$H$} circle(1.5pt);
% 		\fill (4,-4)node[above right]{$B$} circle(1.5pt);
% 		\draw[dashed](4,4)--(4,-4);
% 		\draw(5,-2)--(1,-2)--(3,0)--(1,2)--(5,2);
% 		\draw(5,-1)--(.25,-1)--(3,0)--(.25,1)--(5,1);
% 		\draw[->](.25,1)--(4,1);
% 		\draw[->](1,2)--(4,2);
% 		\draw[->](.25,-1)--(4,-1);
% 		\draw[->](1,-2)--(4,-2);
% 		\draw[->](3,0)--(.25,1);
% 		\draw[->](3,0)--(.25,-1);
% 		\draw[->](3,0)--(1,2);
% 		\draw[->](3,0)--(1,-2);		
% \end{tikzpicture}}		
% 	}
% \end{bt}
% \textbf{Bài tập tương tự}
% \begin{bt}%[BG2-2022-Huỳnh Xuân Tín]
% \immini{Ăng-ten vệ tinh parabol ở hình bên có đầu thu đặt tại tiêu điểm, đường kính ăng-ten là $240$ cm, khoảng cách từ vị trí đặt đầu thu đến miệng ăng-ten là $130$ cm. Tính khoảng cách từ vị trí đặt đàu thu tới đỉnh ăng-ten.}{{\includegraphics[scale=0.8]{"Pic/22-02"}}}

% \end{bt}

\begin{bt}%[BG2-2022-Huỳnh Xuân Tín]
	Quỹ đạo chuyển động của sao chổi Halley là một elip, nhận Mặt Trời là một tiêu điểm, có tâm sai bằng $0{,}967$.
	\begin{enumerate}
		\item Giải thích vì sao ta có thể coi bất kì hình elip nào với tâm sai bằng $0{,}967$ là hình ảnh thu nhỏ của sao chổi Halley.
		\item Biết khoảng cách gần nhất từ sao chổi Halley đến tâm Mặt Trời là khoảng $88\cdot 10^6$ km, tính khoảng cách xa nhất (theo nssdc.gsfc.nasa.gov)
	\end{enumerate}
	
\end{bt}

\begin{bt}%[BG2-2022-Huỳnh Xuân Tín]%
	Một tàu vũ trụ nằm trong một quỹ đạo tròn và ở độ cao $148$km so với bờ mặt Trái Đất. Sau khi đạt được vận tốc cần thiết để thoát khỏa lực hấp dẫn của Trái Đất, tàu vũ trụ sẽ đi theo quỹ đạo parabol với tâm Trái Đất là tiêu điểm; điểm khởi đầu của quỹ đạo này là đỉnh parabol quỹ đạo.
	\begin{enumerate}
		\item Viết phương trình chính tắc của parabol quỹ đạo ($1$ đơn vị đo trên mặt phẳng tọa độ ứng với $1$ km thực tế, lấy bán kính Trái Đất là $6371$ km).
		\item Giải thích vì sao, kể từ khi đi vào quỹ đạo parabol, càng ngày, tàu vũ trụ càng cách xa Trái Đất.
	\end{enumerate}

\end{bt}

\begin{bt}%[BG2-2022-Huỳnh Xuân Tín]
	Khúc cua của một con đường có dạng hình parabol, điểm đầu vào khúc cua là $A$, điểm cuối là $B$, khoảng cách $AB=400$ m. Đỉnh của parabol $(P)$ của khúc cua cách đường  thẳng $AB$ một khoảng $20$ m và cách đều $A$, $B$.
	\begin{enumerate}
		\item Lập phương trình chính tắc của $(P)$, với $1$ đơn vị đo trong mặt phẳng tọa độ tương ứng với $1$ m thự tế.
		\item Lập phương trình chính tắc của $(P)$, với $1$ đơn vị đo trong mặt phẳng tọa độ tương ứng với $1$ km thự tế.
	\end{enumerate}
	
\end{bt}

