\section*{BT ÔN TẬP HÀM SỐ}
\setcounter{ex}{0}\setcounter{bt}{0}
\Opensolutionfile{ans}[ans/ansOC3-CD-2]

\begin{ex}%[Phan Anh]%[0K6B1-1]
    Điểm nào sau đây thuộc đồ thị hàm số $y=\dfrac{1}{x-1}$?
    \choice
    {\True $M_1(2;1)$}
    {$M_2(1;1)$}
    {$M_3(2;0)$}
    {$M_4(0;-2)$}
    \loigiai{
        Xét điểm $M_1$, thay $x=2$ và $y=1$
        vào hàm số $y=\dfrac{1}{x-1}$ ta được $1=\dfrac{1}{2-1}$ ta thấy đúng nên nhận $M_1$.}
\end{ex}

\begin{ex}%[Phan Anh]%[0K6B1-1]
    Cho hàm số $f(x)=\left\{\begin{array}{*{35}{l}}
    \dfrac{2}{x-1} &, x\in(-\infty;0) \\
    \sqrt{x+1} &, x\in[0;2] \\
    x^2-1 &, x\in(2;5]
    \end{array}\right.$. Tính giá trị của $f(4)$.
    \choice
    {$f(4)=\dfrac{2}{3}$}
    {\True $f(4)=15$}
    {$f(4)=\sqrt{5}$}
    {Không tính được}
    \loigiai{Do $4\in(2;5]$ nên $f(4)=4^2-1=15$.}
\end{ex}

\begin{ex}%[Phan Anh]%[0K6B1-1]
    Cho hàm số $f(x)=\left\{\begin{array}{*{35}{l}}
    \dfrac{2\sqrt{x+2}-3}{x-1} &, x\ge 2 \\
    x^2 +1 &, x<2
    \end{array}\right.$. Tính $P=f(2)+f(-2)$.
    \choice
    {$P=\dfrac{8}{3}$}
    {$P=4$}
    {\True $P=6$}
    {$P=\dfrac{5}{3}$}
    \loigiai{\begin{itemize}
            \item Khi $x\ge 2$ thì $f(2)=\dfrac{2\sqrt{2+2}-3}{2-1}=1$.
            \item Khi $x<2$ thì $f(-2)=(-2)^2+1=5$.
        \end{itemize}
        Vậy $f(2)+f(-2)=6$.}
\end{ex}

\begin{ex}%[Phan Anh]%[0K6B1-2]
    Tìm tập xác định $\mathscr{D}$ của hàm số $y=\dfrac{2x-1}{(2x+1)(x-3)}$.
    \choice
    {$\mathscr{D}=(3;+\infty)$}
    {\True $\mathscr{D}=\mathbb{R}\setminus\left\{-\dfrac{1}{2};3\right\}$}
    {$\mathscr{D}=\left(-\dfrac{1}{2};+\infty\right)$}
    {$\mathscr{D}=\mathbb{R}$}
    \loigiai{
        Hàm số xác định khi $\heva{
            & 2x+1\ne 0 \\ 
            & x-3\ne 0}\Leftrightarrow \heva{
            & x\ne-\dfrac{1}{2} \\ 
            & x\ne 3.}$\\
        Vậy tập xác định của hàm số là $ \mathscr{D}=\mathbb{R}\setminus\left\{-\dfrac{1}{2};3\right\}$}
\end{ex}

\begin{ex}%[Phan Anh]%[0K6B1-2]
    Tìm tập xác định $\mathscr{D}$ của hàm số $y=\dfrac{3x-1}{2x-2}$.
    \choice
    {$\mathscr{D}=\mathbb{R}$}
    {$\mathscr{D}=(1;+\infty)$}
    {\True $\mathscr{D}=\mathbb{R}\setminus\{1\}$}
    {$\mathscr{D}=[1;+\infty)$}
    \loigiai{
        Hàm số xác định khi $2x-2\ne0\Leftrightarrow x\ne1$.\\
        Vậy tập xác định của hàm số là $\mathscr{D}=\mathbb{R}\setminus\{1\}$.}
\end{ex}

\begin{ex}%[Phan Anh]%[0K6B1-2]
    Tìm tập xác định $\mathscr{D}$ của hàm số $y=\dfrac{x^2+1}{x^2+3x-4}$.
    \choice
    {$\mathscr{D}=\{1;-4\}$}
    {\True $\mathscr{D}=\mathbb{R}\setminus\{1;-4\}$}
    {$\mathscr{D}=\mathbb{R}\setminus\{1;4\}$}
    {$\mathscr{D}=\mathbb{R}$}
    \loigiai{
        Hàm số xác định khi $x^2+3x-4\ne 0\Leftrightarrow \heva{
            & x\ne 1 \\ 
            & x\ne-4}.$\\
        Vậy tập xác định của hàm số là $\mathscr{D}=\mathbb{R}\setminus\{1;-4\}$.}
\end{ex}

\begin{ex}%[Phan Anh]%[0K6B1-2]
    Tìm tập xác định $\mathscr{D}$ của hàm số $y=\sqrt{\sqrt{x^2+2x+2}-(x+1)}$.
    \choice
    {$\mathscr{D}=\left(-\infty;-1\right)$}
    {$\mathscr{D}=\left[-1;+\infty \right)$}
    {$\mathscr{D}=\mathbb{R}\setminus\left\{-1\right\}$}
    {\True $\mathscr{D}=\mathbb{R}$}
    \loigiai{
        Hàm số xác định khi $\begin{aligned}[t]
        &\sqrt{x^2+2x+2}-(x+1)\ge 0\Leftrightarrow \sqrt{(x+1)^2+1}\ge x+1\\
        \Leftrightarrow&\, \hoac{
            & \heva{
                & x+1<0 \\ 
                & (x+1)^2+1\ge 0}\\ 
            & \heva{
                & x+1\ge 0 \\ 
                & (x+1)^2+1\ge(x+1)^2}}\Leftrightarrow \hoac{
            & x+1<0 \\ 
            & x+1\ge 0}\Leftrightarrow x\in \mathbb{R}.
        \end{aligned}$\\
        Vậy tập xác định của hàm số là $\mathscr{D}=\mathbb{R}$.}
\end{ex}

\begin{ex}%[Phan Anh]%[0K6B1-2]
    Tìm tập xác định $\mathscr{D}$ của hàm số $y=\dfrac{x}{x-\sqrt{x}-6}$.
    \choice
    {$\mathscr{D}=\left[0;+\infty \right)\setminus\left\{3\right\}$}
    {\True $\mathscr{D}=\left[0;+\infty \right)\setminus\left\{9\right\}$}
    {$\mathscr{D}=\left[0;+\infty \right)\setminus\left\{\sqrt{3}\right\}$}
    {$\mathscr{D}=\mathbb{R}\setminus\left\{9\right\}$}
    \loigiai{
        Hàm số xác định khi $\heva{
            & x\ge 0 \\ 
            & x-\sqrt{x}-6\ne 0}\Leftrightarrow \heva{
            & x\ge 0 \\ 
            & \sqrt{x}\ne 3}\Leftrightarrow \heva{
            & x\ge 0 \\ 
            & x\ne 9.}$\\
        Vậy tập xác định của hàm số là $\mathscr{D}=\left[0;+\infty \right)\setminus\left\{9\right\}$.}
\end{ex}

\begin{ex}%[Phan Anh]%[0K6B1-2]
    Tìm tập xác định $\mathscr{D}$ của hàm số $y=\sqrt{6-x}+\dfrac{2x+1}{1+\sqrt{x-1}}$.
    \choice
    {$\mathscr{D}=(1;+\infty)$}
    {\True $\mathscr{D}=[1;6]$}
    {$\mathscr{D}=\mathbb{R}$}
    {$\mathscr{D}=(1;6)$}
    \loigiai{
        Hàm số xác định khi $\heva{
            & 6-x\ge 0 \\ 
            & x-1\ge 0 \\ 
            & 1+\sqrt{x-1}\ne 0\left(\text{luôn đúng} \right)}\Leftrightarrow \heva{
            & x\le 6 \\ 
            & x\ge 1}\Leftrightarrow 1\le x\le 6$.\\
        Vậy tập xác định của hàm số là $\mathscr{D}=[1;6]$.}
\end{ex}

\begin{ex}%[Phan Anh]%[0K6B1-4]
    Cho hàm số $f(x)=4-3x$. Khẳng định nào sau đây đúng?
    \choice
    {Hàm số đồng biến trên $\left(-\infty;\dfrac{4}{3}\right)$}
    {\True Hàm số nghịch biến trên $\left(\dfrac{4}{3};+\infty \right)$}
    {Hàm số đồng biến trên $\mathbb{R}$}
    {Hàm số đồng biến trên $\left(\dfrac{3}{4};+\infty \right)$}
    \loigiai{
        TXĐ: $\mathscr{D}=\mathbb{R}$. \\Với mọi $x_1,x_2\in \mathbb{R}$ và $x_1<x_2$, ta có
        $f\left(x_1\right)-f\left(x_2\right)=\left(4-3x_1\right)-\left(4-3x_2\right)=-3\left(x_1-x_2\right)>0.$\\
        Suy ra $f\left(x_1\right)>f\left(x_2\right)$. Do đó, hàm số nghịch biến trên $\mathbb{R}$.\\
        Mà $\left(\dfrac{4}{3};+\infty \right)\subset \mathbb{R}$ nên hàm số cũng nghịch biến trên $\left(\dfrac{4}{3};+\infty \right)$.}
\end{ex}

\begin{ex}%[Phan Anh]%[0K6B1-2]
    Tìm tập xác định $\mathscr{D}$ của hàm số $y=\dfrac{2018}{\sqrt[3]{x^2-3x+2}-\sqrt[3]{x^2-7}}$.
    \choice
    {\True $\mathscr{D}=\mathbb{R}\setminus\left\{3\right\}$}
    {$\mathscr{D}=\mathbb{R}$}
    {$\mathscr{D}=\left(-\infty;1\right)\cup \left(2;+\infty \right)$}
    {$\mathscr{D}=\mathbb{R}\setminus\left\{0\right\}$}
    \loigiai{
        Hàm số xác định khi $\begin{aligned}[t]
        &\sqrt[3]{x^2-3x+2}-\sqrt[3]{x^2-7}\ne 0\Leftrightarrow \sqrt[3]{x^2-3x+2}\ne \sqrt[3]{x^2-7}\\
        \Leftrightarrow&\,x^2-3x+2\ne x^2-7\Leftrightarrow 9\ne 3x\Leftrightarrow x\ne 3.
        \end{aligned}$\\
        Vậy tập xác định của hàm số là $\mathscr{D}=\mathbb{R}\setminus\left\{3\right\}$.}
\end{ex}

\begin{ex}%[Phan Anh]%[0K6B1-1]
    Cho hàm số $y=f(x)=|-5x|$. Khẳng định nào sau đây là \textbf{sai}?
    \choice
    {$f(-1)=5$}
    {$f(2)=10$}
    {$f(-2)=10$}
    {\True $f\left(\dfrac{1}{5}\right)=-1$}
    \loigiai{Ta có
        \begin{itemize}
            \item $f(-1)=|-5\cdot(-1)|=|5|=5$.
            \item $f(2)=|-5\cdot2|=|-10|=10$.
            \item $f(-2)=|-5\cdot(-2)|=|10|=10$.
            \item $f\left(\dfrac{1}{5}\right)=\left|-5\cdot\dfrac{1}{5}\right|=|-1|=1$
        \end{itemize}
        Cách khác: Vì hàm đã cho là hàm trị tuyệt đối nên không âm. Do đó $f\left(\dfrac{1}{5}\right)=-1$ là sai.}
\end{ex}

\begin{ex}%[Phan Anh]%[0K6B1-2]
    Tìm tập xác định $\mathscr{D}$ của hàm số $y=\dfrac{\sqrt{x+1}}{x^2-x-6}$.
    \choice
    {$\mathscr{D}=\left\{3\right\}$}
    {\True $\mathscr{D}=\left[-1;+\infty \right)\setminus\left\{3\right\}$}
    {$\mathscr{D}=\mathbb{R}$}
    {$\mathscr{D}=\left[-1;+\infty \right)$}
    \loigiai{
        Hàm số xác định khi $\heva{
            & x+1\ge 0 \\ 
            & x^2-x-6\ne 0}\Leftrightarrow \heva{
            & x\ge-1 \\ 
            & x\ne 3 \\ 
            & x\ne-2}\Leftrightarrow \heva{
            & x\ge-1 \\ 
            & x\ne 3.}$\\
        Vậy tập xác định của hàm số là $\mathscr{D}=[-1;+\infty)\setminus\left\{3\right\}$.}
\end{ex}

\begin{ex}%[Phan Anh]%[0K6B1-2]
    Tìm tập xác định $\mathscr{D}$ của hàm số $y=\dfrac{x+1}{(x+1)(x^2+3x+4)}$.
    \choice
    {$\mathscr{D}=\mathbb{R}\setminus\left\{1\right\}$}
    {$\mathscr{D}=\left\{-1\right\}$}
    {\True $\mathscr{D}=\mathbb{R}\setminus\left\{-1\right\}$}
    {$\mathscr{D}=\mathbb{R}$}
    \loigiai{
        Hàm số xác định khi $\heva{
            & x+1\ne 0 \\ 
            & x^2+3x+4\ne 0}\Leftrightarrow x\ne-1$.\\
        Vậy tập xác định của hàm số là $\mathscr{D}=\mathbb{R}\setminus\left\{-1\right\}$.}
\end{ex}

\begin{ex}%[Phan Anh]%[0K6B1-2]
    Tìm tập xác định $\mathscr{D}$ của hàm số $y=\sqrt{6-3x}-\sqrt{x-1}$.
    \choice
    {$\mathscr{D}=\left(1;2\right)$}
    {\True $\mathscr{D}=\left[1;2\right]$}
    {$\mathscr{D}=\left[1;3\right]$}
    {$\mathscr{D}=\left[-1;2\right]$}
    \loigiai{
        Hàm số xác định khi $\heva{
            & 6-3x\ge 0 \\ 
            & x-1\ge 0}\Leftrightarrow \heva{
            & x\le 2 \\ 
            & x\ge 1}\Leftrightarrow 1\le x\le 2$.\\
        Vậy tập xác định của hàm số là $\mathscr{D}=\left[1;2\right]$.}
\end{ex}

\begin{ex}%[Phan Anh]%[0K6B1-2]
    Tìm tập xác định $\mathscr{D}$ của hàm số $y=\sqrt{x^2-2x+1}+\sqrt{x-3}$.
    \choice
    {$\mathscr{D}=(-\infty;3]$}
    {$\mathscr{D}=[1;3]$}
    {\True $\mathscr{D}=[3;+\infty)$}
    {$\mathscr{D}=(3;+\infty)$}
    \loigiai{
        Hàm số xác định khi $\heva{
            & x^2-2x+1\ge 0 \\ 
            & x-3\ge 0}\Leftrightarrow \heva{
            & {\left(x-1\right)}^2\ge 0 \\ 
            & x-3\ge 0}\Leftrightarrow \heva{
            & x\in \mathbb{R} \\ 
            & x\ge 3}\Leftrightarrow x\ge 3$.\\
        Vậy tập xác định của hàm số là $\mathscr{D}=\left[3;+\infty \right)$.}
\end{ex}

\begin{ex}%[Phan Anh]%[0K6B1-4]
    Xét tính đồng biến, nghịch biến của hàm số $f(x)=\dfrac{x-3}{x+5}$ trên khoảng $\left(-\infty;-5\right)$ và trên khoảng $\left(-5;+\infty \right)$. Khẳng định nào sau đây đúng?
    \choice
    {Hàm số nghịch biến trên $\left(-\infty;-5\right)$, đồng biến trên $\left(-5;+\infty \right)$}
    {Hàm số đồng biến trên $\left(-\infty;-5\right)$, nghịch biến trên $\left(-5;+\infty \right)$}
    {Hàm số nghịch biến trên các khoảng $\left(-\infty;-5\right)$ và $\left(-5;+\infty \right)$}
    {\True Hàm số đồng biến trên các khoảng $\left(-\infty;-5\right)$ và $\left(-5;+\infty \right)$}
    \loigiai{
        Ta có
        \begin{eqnarray*}
        f\left(x_1\right)-f\left(x_2\right)&=&\left(\dfrac{x_1-3}{x_1+5}\right)-\left(\dfrac{x_2-3}{x_2+5}\right)\\
        &=&\dfrac{\left(x_1-3\right)\left(x_2+5\right)-\left(x_2-3\right)\left(x_1+5\right)}{\left(x_1+5\right)\left(x_2+5\right)}\\
        &=&\dfrac{8\left(x_1-x_2\right)}{\left(x_1+5\right)\left(x_2+5\right)}.
        \end{eqnarray*} 
         Với mọi $x_1, x_2\in \left(-\infty;-5\right)$ và $x_1<x_2$. Ta có $\heva{& x_1<-5 \\& x_2<-5}\Leftrightarrow \heva{&x_1+5<0 \\& x_2+5<0.}$\\
        Suy ra $\dfrac{f\left(x_1\right)-f\left(x_2\right)}{x_1-x_2}=\dfrac{8}{\left(x_1+5\right)\left(x_2+5\right)}>0\Rightarrow f(x)$ đồng biến trên $\left(-\infty;-5\right)$.\\
    Với mọi $x_1, x_2\in \left(-5;+\infty \right)$ và $x_1<x_2$. Ta có $\heva{
            & x_1>-5 \\ 
            & x_2>-5 \\}\Leftrightarrow \heva{
            & x_1+5>0 \\ 
            & x_2+5>0 \\}$.\\
        Suy ra $\dfrac{f\left(x_1\right)-f\left(x_2\right)}{x_1-x_2}=\dfrac{8}{\left(x_1+5\right)\left(x_2+5\right)}>0\Rightarrow f(x)$ đồng biến trên $\left(-5;+\infty \right)$.}
\end{ex}

\begin{ex}%[Phan Anh]%[0K6B1-2]
    Tìm tập xác định $\mathscr{D}$ của hàm số $y=\dfrac{2x+1}{x^3-3x+2}$.
    \choice
    {$\mathscr{D}=\mathbb{R}\setminus\left\{1;2\right\}$}
    {\True $\mathscr{D}=\mathbb{R}\setminus\left\{-2;1\right\}$}
    {$\mathscr{D}=\mathbb{R}\setminus\left\{-2\right\}$}
    {$\mathscr{D}=\mathbb{R}$}
    \loigiai{
        Hàm số xác định khi $\begin{aligned}[t]
        &x^3-3x+2\ne 0\Leftrightarrow (x-1)(x^2+x-2)\ne 0\\
        \Leftrightarrow&\,\heva{
            & x-1\ne 0 \\ 
            & x^2+x-2\ne 0}\Leftrightarrow \heva{
            & x\ne 1 \\ 
            & \heva{
                & x\ne 1 \\ 
                & x\ne-2}}\Leftrightarrow \heva{
            & x\ne 1 \\ 
            & x\ne-2.}
        \end{aligned}$\\
        Vậy tập xác định của hàm số là $\mathscr{D}=\mathbb{R}\setminus\left\{-2;1\right\}$.}
\end{ex}

\begin{ex}%[Phan Anh]%[0K6B1-2]
    Tìm tập xác định $\mathscr{D}$ của hàm số $y=\dfrac{\sqrt{2-x}+\sqrt{x+2}}{x}$.
    \choice
    {$\mathscr{D}=[-2;2]$}
    {$\mathscr{D}=(-2;2)\setminus\left\{0\right\}$}
    {\True $\mathscr{D}=[-2;2]\setminus\left\{0\right\}$}
    {$\mathscr{D}=\mathbb{R}$}
    \loigiai{
        Hàm số xác định khi $\heva{
            & 2-x\ge 0 \\ 
            & x+2\ge 0 \\ 
            & x\ne 0}\Leftrightarrow \heva{
            & x\le 2 \\ 
            & x\ge-2 \\ 
            & x\ne 0.}$\\
        Vậy tập xác định của hàm số là $\mathscr{D}=\left[-2;2\right]\setminus\left\{0\right\}$.}
\end{ex}

\begin{ex}%[Phan Anh]%[0K6B1-2]
    Tìm tập xác định $\mathscr{D}$ của hàm số $y=\sqrt{x+2}-\sqrt{x+3}$.
    \choice
    {$\mathscr{D}=[-3;+\infty)$}
    {\True $\mathscr{D}=\left[-2;+\infty \right)$}
    {$\mathscr{D}=\mathbb{R}$}
    {$\mathscr{D}=\left[2;+\infty \right)$}
    \loigiai{
        Hàm số xác định khi $\heva{
            & x+2\ge 0 \\ 
            & x+3\ge 0 \\}\Leftrightarrow \heva{
            & x\ge-2 \\ 
            & x\ge-3 \\}\Leftrightarrow x\ge-2$.\\
        Vậy tập xác định của hàm số là $\mathscr{D}=\left[-2;+\infty \right)$.}
\end{ex}

\begin{ex}%[Phan Anh]%[0K6B1-2]
    Tìm tập xác định $\mathscr{D}$ của hàm số $y=\dfrac{x+4}{\sqrt{x^2-16}}$.
    \choice
    {$\mathscr{D}=\left(-\infty;-2\right)\cup \left(2;+\infty \right)$}
    {$\mathscr{D}=\mathbb{R}$}
    {\True $\mathscr{D}=\left(-\infty;-4\right)\cup \left(4;+\infty \right)$}
    {$\mathscr{D}=\left(-4;4\right)$}
    \loigiai{Hàm số xác định khi $x^2-16>0\Leftrightarrow x^2>16\Leftrightarrow \hoac{
            & x>4 \\ 
            & x<-4}$.\\
        Vậy tập xác định của hàm số là $\mathscr{D}=\left(-\infty;-4\right)\cup \left(4;+\infty \right)$.}
\end{ex}

\begin{ex}%[Phan Anh]%[0K6B1-2]
    Tìm tập xác định $\mathscr{D}$ của hàm số $y=\dfrac{\sqrt{3x-2}+6x}{\sqrt{4-3x}}$.
    \choice
    {\True $\mathscr{D}=\left[\dfrac{2}{3};\dfrac{4}{3}\right)$}
    {$\mathscr{D}=\left[\dfrac{3}{2};\dfrac{4}{3}\right)$}
    {$\mathscr{D}=\left[\dfrac{2}{3};\dfrac{3}{4}\right)$}
    {$\mathscr{D}=\left(-\infty;\dfrac{4}{3}\right)$}
    \loigiai{
        Hàm số xác định khi $\heva{
            & 3x-2\ge 0 \\ 
            & 4-3x>0}\Leftrightarrow \heva{
            & x\ge \dfrac{2}{3} \\ 
            & x<\dfrac{4}{3}}\Leftrightarrow \dfrac{2}{3}\le x<\dfrac{4}{3}$.\\
        Vậy tập xác định của hàm số là $\mathscr{D}=\left[\dfrac{2}{3};\dfrac{4}{3}\right)$.}
\end{ex}

\begin{ex}%[Phan Anh]%[0K6B1-2]
    Tìm tập xác định $\mathscr{D}$ của hàm số $y=\dfrac{\sqrt{x-1}+\sqrt{4-x}}{\left(x-2\right)\left(x-3\right)}$.
    \choice
    {$\mathscr{D}=\left[1;4\right]$}
    {$\mathscr{D}=\left(1;4\right)\setminus\left\{2;3\right\}$}
    {\True $\mathscr{D}=\left[1;4\right]\setminus\left\{2;3\right\}$}
    {$\mathscr{D}=\left(-\infty;1\right]\cup \left[4;+\infty \right)$}
    \loigiai{
        Hàm số xác định khi $\heva{
            & x-1\ge 0 \\ 
            & 4-x\ge 0 \\ 
            & x-2\ne 0 \\ 
            & x-3\ne 0}\Leftrightarrow \heva{
            & x\ge 1 \\ 
            & x\le 4 \\ 
            & x\ne 2 \\ 
            & x\ne 3}\Leftrightarrow \heva{
            & 1\le x\le 4 \\ 
            & x\ne 2 \\ 
            & x\ne 3.}$\\
        Vậy tập xác định của hàm số là $\mathscr{D}=\left[1;4\right]\setminus\left\{2;3\right\}$.}
\end{ex}

\begin{ex}%[Phan Anh]%[0K6B1-4]
    Xét sự biến thiên của hàm số $f(x)=\dfrac{3}{x}$ trên khoảng $(0;+\infty)$. Khẳng định nào sau đây đúng?
    \choice
    {Hàm số đồng biến trên khoảng $\left(0;+\infty \right)$}
    {\True Hàm số nghịch biến trên khoảng $\left(0;+\infty \right)$}
    {Hàm số vừa đồng biến, vừa nghịch biến trên khoảng $\left(0;+\infty \right)$}
    {Hàm số không đồng biến, cũng không nghịch biến trên khoảng $\left(0;+\infty \right)$}
    \loigiai{
        Ta có $f\left(x_1\right)-f\left(x_2\right)=\dfrac{3}{x_1}-\dfrac{3}{x_2}=\dfrac{3\left(x_2-x_1\right)}{x_1x_2}=-\dfrac{3\left(x_1-x_2\right)}{x_1x_2}.$\\
        Với mọi $x_1, x_2\in \left(0;+\infty \right)$ và $x_1<x_2$. Ta có $\heva{
            & x_1>0 \\ 
            & x_2>0 \\}\Rightarrow x_1\cdot x_2>0$.\\
        Suy ra $\dfrac{f\left(x_1\right)-f\left(x_2\right)}{x_1-x_2}=-\dfrac{3}{x_1x_2}<0\Rightarrow f(x)$ nghịch biến trên $\left(0;+\infty \right)$.}
\end{ex}

\begin{ex}%[Phan Anh]%[0K6B1-2]
    Tìm tập xác định $\mathscr{D}$ của hàm số $y=\dfrac{\sqrt[3]{x-1}}{x^2+x+1}$.
    \choice
    {$\mathscr{D}=\left(1;+\infty \right)$}
    {$\mathscr{D}=\left\{1\right\}$}
    {\True $\mathscr{D}=\mathbb{R}$}
    {$\mathscr{D}=\left(-1;+\infty \right)$}
    \loigiai{
        Hàm số xác định khi $x^2+x+1\ne 0$ luôn đúng với mọi $x\in \mathbb{R}$.\\
        Vậy tập xác định của hàm số là $\mathscr{D}=\mathbb{R}$.}
\end{ex}

\begin{ex}%[Phan Anh]%[0K6B1-2]
    Tìm tập xác định $\mathscr{D}$ của hàm số $y=\dfrac{x+1}{(x-3)\sqrt{2x-1}}$.
    \choice
    {$\mathscr{D}=\mathbb{R}$}
    {$\mathscr{D}=\left(-\dfrac{1}{2};+\infty \right)\setminus\left\{3\right\}$}
    {$\mathscr{D}=\left[\dfrac{1}{2};+\infty \right)\setminus\left\{3\right\}$}
    {\True $\mathscr{D}=\left(\dfrac{1}{2};+\infty \right)\setminus\left\{3\right\}$}
    \loigiai{
        Hàm số xác định khi $\heva{
            & x-3\ne 0 \\ 
            & 2x-1>0}\Leftrightarrow \heva{
            & x\ne 3 \\ 
            & x>\dfrac{1}{2}.}$\\
        Vậy tập xác định của hàm số là $\mathscr{D}=\left(\dfrac{1}{2};+\infty \right)\setminus\left\{3\right\}$.}
\end{ex}

\begin{ex}%[Phan Anh]%[0K6B1-4]
    Xét sự biến thiên của hàm số $f(x)=x+\dfrac{1}{x}$ trên khoảng $\left(1;+\infty \right)$. Khẳng định nào sau đây đúng?
    \choice
    {\True Hàm số đồng biến trên khoảng $\left(1;+\infty \right)$}
    {Hàm số nghịch biến trên khoảng $\left(1;+\infty \right)$}
    {Hàm số vừa đồng biến, vừa nghịch biến trên khoảng $\left(1;+\infty \right)$}
    {Hàm số không đồng biến, cũng không nghịch biến trên khoảng $\left(1;+\infty \right)$}
    \loigiai{
        Ta có
        $f\left(x_1\right)-f\left(x_2\right)=\left(x_1+\dfrac{1}{x_1}\right)-\left(x_2+\dfrac{1}{x_2}\right)=\left(x_1-x_2\right)+\left(\dfrac{1}{x_1}-\dfrac{1}{x_2}\right)=\left(x_1-x_2\right)\left(1-\dfrac{1}{x_1x_2}\right).$\\
        Với mọi $x_1, x_2\in \left(1;+\infty \right)$ và $x_1<x_2$. Ta có $\heva{
            & x_1>1 \\ 
            & x_2>1 \\}\Rightarrow x_1\cdot x_2>1\Rightarrow \dfrac{1}{x_1\cdot x_2}<1.$\\
        Suy ra $\dfrac{f\left(x_1\right)-f\left(x_2\right)}{x_1-x_2}=1-\dfrac{1}{x_1x_2}>0\Rightarrow f(x)$ đồng biến trên $\left(1;+\infty \right)$.}
\end{ex}

\begin{ex}%[Phan Anh]%[0K6B1-4]
    Xét tính đồng biến, nghịch biến của hàm số $f(x)=x^2-4x+5$ trên khoảng $\left(-\infty;2\right)$ và trên khoảng $\left(2;+\infty \right)$. Khẳng định nào sau đây đúng?
    \choice
    {\True Hàm số nghịch biến trên $\left(-\infty;2\right)$, đồng biến trên $\left(2;+\infty \right)$}
    {Hàm số đồng biến trên $\left(-\infty;2\right)$, nghịch biến trên $\left(2;+\infty \right)$}
    {Hàm số nghịch biến trên các khoảng $\left(-\infty;2\right)$ và $\left(2;+\infty \right)$}
    {Hàm số đồng biến trên các khoảng $\left(-\infty;2\right)$ và $\left(2;+\infty \right)$}
    \loigiai{
        Ta có $f\left(x_1\right)-f\left(x_2\right)=\left(x_1^2-4x_1+5\right)-\left(x_2^2-4x_2+5\right)$
        $=\left(x_1^2-x_2^2\right)-4\left(x_1-x_2\right)=\left(x_1-x_2\right)\left(x_1+x_2-4\right)$.
         Với mọi $x_1, x_2\in \left(-\infty;2\right)$ và $x_1<x_2$. Ta có $\heva{
            & x_1<2 \\ 
            & x_2<2 \\}\Rightarrow x_1+x_2<4$.\\
        Suy ra $\dfrac{f\left(x_1\right)-f\left(x_2\right)}{x_1-x_2}=\dfrac{\left(x_1-x_2\right)\left(x_1+x_2-4\right)}{x_1-x_2}=x_1+x_2-4<0$.\\
        Vậy hàm số nghịch biến trên $\left(-\infty;2\right)$.\\
     Với mọi $x_1, x_2\in \left(2;+\infty \right)$ và $x_1<x_2$. Ta có $\heva{
            & x_1>2 \\ 
            & x_2>2 \\}\Rightarrow x_1+x_2>4$.\\
        Suy ra $\dfrac{f\left(x_1\right)-f\left(x_2\right)}{x_1-x_2}=\dfrac{\left(x_1-x_2\right)\left(x_1+x_2-4\right)}{x_1-x_2}=x_1+x_2-4>0$.\\
        Vậy hàm số đồng biến trên $\left(2;+\infty \right)$.}
\end{ex}

\begin{ex}%[Phan Anh]%[0K6B1-2]
    Tìm tập xác định $\mathscr{D}$ của hàm số $y=\dfrac{\sqrt{x+2}}{x\sqrt{x^2-4x+4}}$.
    \choice
    {\True $\mathscr{D}=[-2;+\infty)\setminus\left\{0;2\right\}$}
    {$\mathscr{D}=\mathbb{R}$}
    {$\mathscr{D}=[-2;+\infty)$}
    {$\mathscr{D}=(-2;+\infty)\setminus\left\{0;2\right\}$}
    \loigiai{
        Hàm số xác định khi $\heva{
            & x+2\ge 0 \\ 
            & x\ne 0 \\ 
            & x^2-4x+4>0}\Leftrightarrow \heva{
            & x+2\ge 0 \\ 
            & x\ne 0 \\ 
            & (x-2)^2>0}\Leftrightarrow \heva{
            & x\ge-2 \\ 
            & x\ne 0 \\ 
            & x\ne 2.}$\\
        Vậy tập xác định của hàm số là $\mathscr{D}=\left[-2;+\infty \right)\setminus\left\{0;2\right\}$.}
\end{ex}

\begin{ex}%[Phan Anh]%[0K6B1-1]
    Điểm nào sau đây \textbf{không} thuộc đồ thị hàm số $y=\dfrac{\sqrt{x^2-4x+4}}{x}$?
    \choice
    {$A\left(2;0\right)$}
    {$B\left(3;\dfrac{1}{3}\right)$}
    {\True $C\left(1;-1\right)$}
    {$D\left(-1;-3\right)$}
    \loigiai{Thay từng đáp án vào hàm số $y=\dfrac{\sqrt{x^2-4x+4}}{x}$.
        \begin{itemize}
            \item Với $x=2$ và $y=0$, ta được $0=\dfrac{\sqrt{2^2-4.2+4}}{2}$ (đúng).
            \item Với $x=3$ và $y=\dfrac{1}{3}$, ta được $\dfrac{1}{3}=\dfrac{\sqrt{3^2-4\cdot3+4}}{3}$ (đúng).
            \item Với thay $x=1$ và $y=-1$, ta được $-1=\dfrac{\sqrt{1^2-4\cdot1+4}}{1}\Leftrightarrow-1=1$ (sai).
        \end{itemize}}
\end{ex}


\begin{ex}%[Phan Anh]%[0K6K1-2]
    Tìm tập xác định $\mathscr{D}$ của hàm số $f(x)=\left\{\begin{array}{*{35}{l}}
    \dfrac{1}{2-x} &;x\ge 1 \\
    \sqrt{2-x} &;x<1.
    \end{array}\right.$
    \choice
    {$\mathscr{D}=\mathbb{R}$}
    {$\mathscr{D}=\left(2;+\infty \right)$}
    {$\mathscr{D}=\left(-\infty;2\right)$}
    {\True $\mathscr{D}=\mathbb{R}\setminus\left\{2\right\}$}
    \loigiai{
        Hàm số xác định khi $\hoac{
            & \heva{
                & x\ge 1 \\ 
                & 2-x\ne 0} \\ 
            & \heva{
                & x<1 \\ 
                & 2-x\ge 0}}\Leftrightarrow \hoac{
            & \heva{
                & x\ge 1 \\ 
                & x\ne 2} \\ 
            & \heva{
                & x<1 \\ 
                & x\le 2}}\Leftrightarrow \hoac{
            & \heva{
                & x\ge 1 \\ 
                & x\ne 2} \\ 
            & x<1.}$\\
        Vậy xác định của hàm số là $\mathscr{D}=\mathbb{R}\setminus\left\{2\right\}$.}
\end{ex}

\begin{ex}%[Phan Anh]%[0K6K1-2]
    Tìm tất cả các giá trị thực của tham số $m$ để hàm số $y=\dfrac{x+2m+2}{x-m}$ xác định trên $\left(-1;0\right)$.
    \choice
    {$\hoac{
            & m>0 \\ 
            & m<-1}$}
    {$m\le-1$}
    {\True $\hoac{
            & m\ge 0 \\ 
            & m\le-1}$}
    {$m\ge 0$}
    \loigiai{
        Hàm số xác định khi $x-m\ne 0\Leftrightarrow x\ne m$.
        Tập xác định của hàm số là $\mathscr{D}=\mathbb{R}\setminus\left\{m\right\}$.\\
        Hàm số xác định trên $\left(-1;0\right)$ khi và chỉ khi $m\notin \left(-1;0\right)\Leftrightarrow \hoac{
            & m\ge 0 \\ 
            & m\le-1.}$}
\end{ex}

\begin{ex}%[Phan Anh]%[0K6K1-4]
    Tìm tất cả các giá trị thực của tham số $m$ để hàm số $y=-x^2+\left(m-1\right)x+2$ nghịch biến trên khoảng $\left(1;2\right)$.
    \choice
    {$m<5$}
    {$m>5$}
    {\True $m<3$}
    {$m>3$}
    \loigiai{
        Với mọi $x_1\ne x_2$, ta có\\
        $\dfrac{f\left(x_1\right)-f\left(x_2\right)}{x_1-x_2}=\dfrac{\left[-x_1^2+\left(m-1\right)x_1+2\right]-\left[-x_2^2+\left(m-1\right)x_2+2\right]}{x_1-x_2}=-\left(x_1+x_2\right)+m-1.$\\
        Để hàm số nghịch biến trên $\left(1;2\right)\Leftrightarrow-\left(x_1+x_2\right)+m-1<0$, với mọi $x_1,x_2\in \left(1;2\right)$\\
        $\Leftrightarrow m<\left(x_1+x_2\right)+1$, với mọi $x_1,x_2\in \left(1;2\right)$
        $\Leftrightarrow m<\left(1+1\right)+1=3$.}
\end{ex}

\begin{ex}%[Phan Anh]%[0K6K1-2]
    Tìm tất cả các giá trị thực của tham số $m$ để hàm số $y=\sqrt{x-m+1}+\dfrac{2x}{\sqrt{-x+2m}}$ xác định trên khoảng $(-1;3)$.
    \choice
    {\True Không có giá trị $m$ thỏa mãn}
    {$m\ge 2$}
    {$m\ge 3$}
    {$m\ge 1$}
    \loigiai{
        Hàm số xác định khi $\heva{
            & x-m+1\ge 0 \\ 
            &-x+2m>0}\Leftrightarrow \heva{
            & x\ge m-1 \\ 
            & x<2m.}$\\
        Tập xác định của hàm số là $\mathscr{D}=\left[m-1;2m\right)$ với điều kiện $m-1<2m\Leftrightarrow m>-1$.\\
        Hàm số đã cho xác định trên $\left(-1;3\right)$ khi và chỉ khi $\left(-1;3\right)\subset \left[m-1;2m\right)$\\
        $\Leftrightarrow m-1\le-1<3\le 2m\Leftrightarrow \heva{
            & m\le 0 \\ 
            & m\ge \dfrac{3}{2}.}$\\
        Vậy không có giá trị $m$ thỏa bài toán.}
\end{ex}

\begin{ex}%[Phan Anh]%[0K6K1-2]
    Tìm tập xác định $\mathscr{D}$ của hàm số $y=\dfrac{|x|}{|x-2|+\left|x^2+2x\right|}$.
    \choice
    {\True $\mathscr{D}=\mathbb{R}$}
    {$\mathscr{D}=\mathbb{R}\setminus\left\{-2;0\right\}$}
    {$\mathscr{D}=\mathbb{R}\setminus\left\{-2;0;2\right\}$}
    {$\mathscr{D}=\left(2;+\infty \right)$}
    \loigiai{
        Hàm số xác định khi $|x-2|+\left|x^2+2x\right|\ne0$.\\
        Xét phương trình $|x-2|+\left|x^2+2x\right|=0\Leftrightarrow \heva{
            & |x-2|=0 \\ 
            & \left|x^2+2x\right|=0}\Leftrightarrow \heva{
            & x=2 \\ 
            & x=0\vee x=-2.}$\\
        Vậy không có giá trị $x$ làm cho $|x-2|+\left| x^2+2x\right|=0$, do đó $|x-2|+\left| x^2+2x\right|\ne 0$ đúng với mọi $x\in \mathbb{R}$. Vậy tập xác định của hàm số là $\mathscr{D}=\mathbb{R}$.}
\end{ex}

\begin{ex}%[Phan Anh]%[0K6K1-4]
    \immini{Cho đồ thị hàm số $y=x^3$ như hình bên. Khẳng định nào sau đây \textbf{sai}?
    \choice
    {Hàm số đồng biến trên khoảng $\left(-\infty;0\right)$}
    {Hàm số đồng biến trên khoảng $\left(0;+\infty \right)$}
    {Hàm số đồng biến trên khoảng $\left(-\infty;+\infty \right)$}
    {\True Hàm số đồng biến tại gốc tọa độ $O$}}
{\begin{tikzpicture}[>=stealth,scale=0.6]
    \draw[->](-2,0)--(2,0)node[above]{$x$};
    \draw[->](0,-3)--(0,3)node[right]{$y$};
    \draw[smooth,samples=100,domain=-1.4:1.4]plot(\x,{(\x)^3});
    \fill (0,0)node[above left]{$O$}circle(1.2pt);
    \end{tikzpicture}}
    \loigiai{Dựa vào đồ thị, ta thấy hàm số đồng biến trên toàn miền xác định. Nhưng không thể đồng biến chỉ tại đúng một điểm.}
\end{ex}

\begin{ex}%[Phan Anh]%[0K6K1-2]
    Tìm tất cả các giá trị thực của tham số $m$ để hàm số $y=\dfrac{2x+1}{\sqrt{x^2-6x+m-2}}$ xác định trên $\mathbb{R}$.
    \choice
    {$m\ge 11$}
    {\True $m>11$}
    {$m<11$}
    {$m\le 11$}
    \loigiai{
        Hàm số xác định khi $x^2-6x+m-2>0\Leftrightarrow {\left(x-3\right)}^2+m-11>0$.\\
        Hàm số xác định với $\forall x\in \mathbb{R}\Leftrightarrow (x-3)^2+m-11>0$ đúng với mọi $x\in \mathbb{R}$
        $\Leftrightarrow m-11>0\Leftrightarrow m>11$.}
\end{ex}

\begin{ex}%[Phan Anh]%[0K6K1-2]
    Tìm tất cả các giá trị thực của tham số $m$ để hàm số $y=\sqrt{x-m}+\sqrt{2x-m-1}$ xác định trên $(0;+\infty)$.
    \choice
    {$m\le 0$}
    {$m\ge 1$}
    {$m\le 1$}
    {\True $m\le-1$}
    \loigiai{
        Hàm số xác định khi $\heva{
            & x-m\ge 0 \\ 
            & 2x-m-1\ge 0}\Leftrightarrow \heva{
            & x\ge m \\ 
            & x\ge \dfrac{m+1}{2}}\,(*)$.
        \begin{itemize}
            \item Nếu $m\ge \dfrac{m+1}{2}\Leftrightarrow m\ge 1$ thì $\left(*\right)\Leftrightarrow x\ge m$.\\
            Tập xác định của hàm số là $\mathscr{D}=\left[m;+\infty \right)$.
            Khi đó, hàm số xác định trên $\left(0;+\infty \right)$ khi và chỉ khi $\left(0;+\infty \right)\subset \left[m;+\infty \right)\Leftrightarrow m\le 0$
            $\Rightarrow $ Không thỏa mãn điều kiện $m\ge 1$.
            \item Nếu $m\le \dfrac{m+1}{2}\Leftrightarrow m\le 1$ thì $\left(*\right)\Leftrightarrow x\ge \dfrac{m+1}{2}$.\\
            Tập xác định của hàm số là $\mathscr{D}=\left[\dfrac{m+1}{2};+\infty \right)$.
            Khi đó, hàm số xác định trên $\left(0;+\infty \right)$
            khi và chỉ khi $\left(0;+\infty \right)\subset \left[\dfrac{m+1}{2};+\infty \right)\Leftrightarrow \dfrac{m+1}{2}\le 0\Leftrightarrow m\le-1$.\\
            $\Rightarrow $ Thỏa mãn điều kiện $m\le 1$.
        \end{itemize}
        Vậy $m\le-1$ thỏa yêu cầu bài toán.}
\end{ex}

\begin{ex}%[Phan Anh]%[0K6K1-4]
    \immini{Cho hàm số $y=f(x)$ có tập xác định là $\left[-3;3\right]$ và đồ thị của nó được biểu diễn bởi hình bên. Khẳng định nào sau đây là đúng?
    \choice
    {\True Hàm số đồng biến trên khoảng $\left(-3;-1\right)$ và $\left(1;3\right)$}
    {Hàm số đồng biến trên khoảng $\left(-3;-1\right)$và $\left(1;4\right)$}
    {Hàm số đồng biến trên khoảng $\left(-3;3\right)$}
    {Hàm số nghịch biến trên khoảng $\left(-1;0\right)$}}
    {\begin{tikzpicture}[>=stealth,scale=0.7]
        \draw[->](-4,0)--(4,0)node[above]{$x$};
        \draw[->](0,-2)--(0,5)node[right]{$y$};
        \draw (-3,-1)--(-1,1)--(0,1)node[above left]{$1$}--(3,4);
        \draw[dashed](-3,0)node[above]{$-3$}--(-3,-1)--(0,-1)node[right]{$-1$};
        \draw[dashed](-1,0)node[below]{$-1$}--(-1,1);
        \draw[dashed](3,0)node[below]{$3$}--(3,4)--(0,4)node[left]{$4$};
        \fill (-3,0)circle(1.2pt) (-3,-1)circle(1.2pt) (0,-1)circle(1.2pt) (-1,0)circle(1.2pt) (-1,1)circle(1.2pt) (0,1)circle(1.2pt) (3,0)circle(1.2pt) (3,4)circle(1.2pt) (0,4)circle(1.2pt) (0,0)node[above right]{$O$}circle(1.2pt);
        \end{tikzpicture}}  
    \loigiai{
        Trên khoảng $\left(-3;-1\right)$ và $\left(1;3\right)$ đồ thị hàm số đi lên từ trái sang phải\\
        $\Rightarrow $ Hàm số đồng biến trên khoảng $\left(-3;-1\right)$ và $\left(1;3\right).$}
\end{ex}

\begin{ex}%[Phan Anh]%[0K6K1-4]
    Có bao nhiêu giá trị nguyên của tham số $m$ thuộc đoạn $\left[-3;3\right]$ để hàm số $f(x)=\left(m+1\right)x+m-2$ đồng biến trên $\mathbb{R}$?
    \choice
    {$7$}
    {$5$}
    {\True $4$}
    {$3$}
    \loigiai{
        Tập xác định $\mathscr{D}=\mathbb{R}.$\\
        Với mọi $x_1,x_2\in\mathscr{D}$ và $x_1<x_2$. \\Ta có
        $f\left(x_1\right)-f\left(x_2\right)=\left[\left(m+1\right)x_1+m-2\right]-\left[\left(m+1\right)x_2+m-2\right]=\left(m+1\right)\left(x_1-x_2\right).$\\
        Suy ra $\dfrac{f\left(x_1\right)-f\left(x_2\right)}{x_1-x_2}=m+1$.\\
        Để hàm số đồng biến trên $\mathbb{R}$ khi và chỉ khi
        $m+1>0\Leftrightarrow m>-1\xrightarrow{m\in \left[-3;3\right]}{m\in \mathbb{Z}}\Rightarrow m\in \left\{0;1;2;3\right\}$.\\
        Vậy có 4 giá trị nguyên của $m$ thỏa mãn.}
\end{ex}

\begin{ex}%[Phan Anh]%[0K6K1-2]
    Tìm tập xác định $\mathscr{D}$ của hàm số $y=\dfrac{2x-1}{\sqrt{x|x-4|}}$.
    \choice
    {$\mathscr{D}=\mathbb{R}\setminus\left\{0;4\right\}$}
    {$\mathscr{D}=\left(0;+\infty \right)$}
    {$\mathscr{D}=\left[0;+\infty \right)\setminus\left\{4\right\}$}
    {\True $\mathscr{D}=\left(0;+\infty \right)\setminus\left\{4\right\}$}
    \loigiai{
        Hàm số xác định khi $x|x-4|>0\Leftrightarrow \heva{
            & \left| x-4\right|\ne 0 \\ 
            & x>0}\Leftrightarrow \heva{
            & x\ne 4 \\ 
            & x>0.}$\\
        Vậy tập xác định của hàm số là $\mathscr{D}=\left(0;+\infty \right)\setminus\left\{4\right\}$.}
\end{ex}

\begin{ex}%[Phan Anh]%[0K6K1-2]
    Tìm tất cả các giá trị thực của tham số $m$ để hàm số $y=\dfrac{mx}{\sqrt{x-m+2}-1}$ xác định trên $(0;1)$.
    \choice
    {$m\in \left(-\infty;\dfrac{3}{2}\right]\cup \left\{2\right\}$}
    {$m\in \left(-\infty;-1\right]\cup \left\{2\right\}$}
    {$m\in \left(-\infty;1\right]\cup \left\{3\right\}$}
    {\True $m\in \left(-\infty;1\right]\cup \left\{2\right\}$}
    \loigiai{
        Hàm số xác định khi $\heva{
            & x-m+2\ge 0 \\ 
            & \sqrt{x-m+2}-1\ne 0}\Leftrightarrow \heva{
            & x\ge m-2 \\ 
            & x\ne m-1.}$
    \\ Tập xác định của hàm số là $\mathscr{D}=\left[m-2;+\infty \right)\setminus\left\{m-1\right\}$.\\
        Hàm số xác định trên $\left(0;1\right)$ khi và chỉ khi $\left(0;1\right)\subset \left[m-2;+\infty \right)\setminus\left\{m-1\right\}$\\
        $\Leftrightarrow \hoac{
            & m-2\le 0<1\le m-1 \\ 
            & m-1\le 0}\Leftrightarrow \hoac{
            & \heva{
                & m\le 2 \\ 
                & m\ge 2} \\ 
            & m\le 1}\Leftrightarrow \hoac{
            & m=2 \\ 
            & m\le 1.}$}
\end{ex}

\begin{ex}%[Phan Anh]%[0K6K1-2]
    Tìm tập xác định $\mathscr{D}$ của hàm số $y=\dfrac{\sqrt{5-3\left| x\right|}}{x^2+4x+3}$.
    \choice
    {\True $\mathscr{D}=\left[-\dfrac{5}{3};\dfrac{5}{3}\right]\setminus\left\{-1\right\}$}
    {$\mathscr{D}=\mathbb{R}$}
    {$\mathscr{D}=\left(-\dfrac{5}{3};\dfrac{5}{3}\right)\setminus\left\{-1\right\}$}
    {$\mathscr{D}=\left[-\dfrac{5}{3};\dfrac{5}{3}\right]$}
    \loigiai{
        Hàm số xác định khi $\heva{
            & 5-3\left| x\right|\ge 0 \\ 
            & x^2+4x+3\ne 0}\Leftrightarrow \heva{
            & \left| x\right|\le \dfrac{5}{3} \\ 
            & x\ne-1 \\ 
            & x\ne-3}\Leftrightarrow \heva{
            &-\dfrac{5}{3}\le x\le \dfrac{5}{3} \\ 
            & x\ne-1 \\ 
            & x\ne-3}\Leftrightarrow \heva{
            &-\dfrac{5}{3}\le x\le \dfrac{5}{3} \\ 
            & x\ne-1.}$\\
        Vậy tập xác định của hàm số là $\mathscr{D}=\left[-\dfrac{5}{3};\dfrac{5}{3}\right]\setminus\left\{-1\right\}$.}
\end{ex}

\begin{ex}%[Phan Anh]%[0K6K1-2]
    Tìm tập xác định $\mathscr{D}$ của hàm số $f(x)=\left\{\begin{array}{*{35}{l}}
    \dfrac{1}{x} &;x\ge 1 \\
    \sqrt{x+1} &;x<1.
    \end{array}\right.$
    \choice
    {$\mathscr{D}=\left\{-1\right\}$}
    {$\mathscr{D}=\mathbb{R}$}
    {\True $\mathscr{D}=\left[-1;+\infty \right)$}
    {$\mathscr{D}=\left[-1;1\right)$}
    \loigiai{
        Hàm số xác định khi $\hoac{
            & \heva{
                & x\ge 1 \\ 
                & x\ne 0} \\ 
            & \heva{
                & x<1 \\ 
                & x+1\ge 0}}\Leftrightarrow \hoac{
            & x\ge 1 \\ 
            & \heva{
                & x<1 \\ 
                & x\ge-1.}}$\\
        Vậy xác định của hàm số là $\mathscr{D}=\left[-1;+\infty \right)$.}
\end{ex}

\begin{ex}%[Phan Anh]%[0K6K1-4]
    Cho hàm số $f(x)=\sqrt{2x-7}$. Khẳng định nào sau đây đúng?
    \choice
    {Hàm số nghịch biến trên $\left(\dfrac{7}{2};+\infty \right)$}
    {\True Hàm số đồng biến trên $\left(\dfrac{7}{2};+\infty \right)$}
    {Hàm số đồng biến trên $\mathbb{R}$}
    {Hàm số nghịch biến trên $\mathbb{R}$}
    \loigiai{
        Tập xác định là $\mathscr{D}=\left[\dfrac{7}{2};+\infty \right)$ nên ta loại đáp án C và D.\\
        Xét $f\left(x_1\right)-f\left(x_2\right)=\sqrt{2x_1-7}-\sqrt{2x_2-7}=\dfrac{2\left(x_1-x_2\right)}{\sqrt{2x_1-7}+\sqrt{2x_2-7}}.$\\
        Với mọi $x_1, x_2\in \left(\dfrac{7}{2};+\infty \right)$ và $x_1<x_2$, ta có $\dfrac{f\left(x_1\right)-f\left(x_2\right)}{x_1-x_2}=\dfrac{2}{\sqrt{2x_1-7}+\sqrt{2x_2-7}}>0.$\\
        Vậy hàm số đồng biến trên $\left(\dfrac{7}{2};+\infty \right)$.}
\end{ex}\Closesolutionfile{ans}