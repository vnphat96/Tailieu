\begin{dang}
  {LÝ THUYẾT} 
\end{dang}
% \Opensolutionfile{ans}[ans/ans10D1-TN-1]

% \begin{ex}%[0D6Y1-5]
% Khẳng định nào sau đây là {\bf đúng} khi nói về {\bf ``đường tròn định hướng''}?
% \choice
% {Mỗi đường tròn là một đường tròn định hướng}
% {Mỗi đường tròn đã chọn một điểm là gốc đều là một đường tròn định hướng}
% {Mỗi đường tròn đã chọn một chiều chuyển động và một điểm là gốc đều là một đường tròn định hướng}
% {\True Mỗi đường tròn trên đó ta đã chọn một chiều chuyển động gọi là chiều dương và chiều ngược lại được gọi là chiều âm là một đường tròn định hướng}
% \loigiai{
    % Theo kiến thức cơ bản SGK trang $134$ ở dòng $2$.
% }
% \end{ex}

\begin{ex}%[0D6Y1-5]
Quy ước chiều dương của một góc lượng giác là:
\choice
{Luôn cùng chiều quay kim đồng hồ}
{\True Luôn ngược chiều quay kim đồng hồ}
{Có thể cùng chiều quay kim đồng hồ mà cũng có thể là ngược chiều quay kim đồng hồ}
{Không cùng chiều quay kim đồng hồ và cũng không ngược chiều quay kim đồng hồ}
\loigiai{
% Theo SGK cơ bản trang $134$ ở dòng $6$.
}
\end{ex}

\begin{ex}%[0D6B1-5]
Trên đường tròn lượng giác, mỗi cung lượng giác $\overset{\curvearrowright}{AB}$ xác định:
\choice
{Một góc lượng giác tia đầu $OA$, tia cuối $OB$}
{Hai góc lượng giác tia đầu $OA$, tia cuối $OB$}
{Bốn góc lượng giác tia đầu $OA$, tia cuối $OB$}
{\True Vô số góc lượng giác tia đầu $OA$, tia cuối $OB$}
\loigiai{
    % Theo SGK cơ bản trang $134$ ở dòng cuối.
}
\end{ex}

% \begin{ex}%[0D6B1-5]
% Khẳng định nào sau đây là {\bf đúng} khi nói về {\bf ``góc lượng giác''}?
% \choice
% {Trên đường tròn tâm $O$ bán kính $R=1$, góc hình học $AOB$ là góc lượng giác}
% {Trên đường tròn tâm $O$ bán kính $R=1$, góc hình học $AOB$ có phân biệt điểm đầu $A$ và điểm cuối $B$ là góc lượng giác}
% {Trên đường tròn định hướng, góc hình học $AOB$ là góc lượng giác}
% {\True Trên đường tròn định hướng, góc hình học $AOB$ có phân biệt điểm đầu $A$ và điểm cuối $B$ là góc lượng giác}
% \loigiai{
% Theo SGK cơ bản trang $135$, mục $2$.
% }
% \end{ex}

\begin{ex}%[0D6B1-5]
Khẳng định nào sau đây là {\bf đúng} khi nói về {\bf ``đường tròn lượng giác''}?
\choice
{Mỗi đường tròn là một đường tròn lượng giác}
{Mỗi đường tròn có bán kính $R=1$ là một đường tròn lượng giác}
{Mỗi đường tròn có bán kính $R=1$, tâm trùng với gốc tọa độ là một đường tròn lượng giác}
{\True Mỗi đường tròn có bán kính $R=1$, tâm trùng với gốc tọa độ, được định hướng và lấy điểm $A(1;0)$ làm điểm gốc là một đường tròn lượng giác}
\loigiai{
    % Theo SGK cơ bản trang $135$, mục $3$.
}
\end{ex}

\begin{dang} {ĐỔI TỪ ĐỘ SANG RADIAN VÀ NGƯỢC LẠI}
\end{dang}

\begin{ex}%[0D6B1-5]
Trên đường tròn cung có số đo $1$ rad là?
\choice
{Cung có độ dài bằng 1}
{Cung tương ứng với góc ở tâm $60^{\circ}$}
{Cung có độ dài bằng đường kính}
{\True Cung có độ dài bằng nửa đường kính}
\loigiai{
    Cung có độ dài bằng bán kính (nửa đường kính) thì có số đó bằng $1$ rad.
}
\end{ex}

\begin{ex}%[0D6B1-1]
Khẳng định nào sau đây là đúng?
\choice
{$\pi$ rad = $1^{\circ}$}
{$\pi$ rad = $60^{\circ}$}
{\True $\pi$ rad =$180^{\circ}$}
{$\pi $ rad =${\left(\dfrac{180}{\pi}\right)}^{\circ}$}
\loigiai{
$\pi$ rad tướng ứng với $180^{\circ}$.
}
\end{ex}

\begin{ex}%[0D6B1-1]
Khẳng định nào sau đây là đúng?
\choice
{$1$ rad $=1^{\circ}$}
{$1$ rad $=60^{\circ}$}
{$1$ rad $=180^{\circ}$}
{\True $1$ rad $={\left(\dfrac{180}{\pi}\right)}^{\circ}$}
\loigiai{
    Ta có $\pi$ rad tướng ứng với $180^{\circ}$.
    Suy ra $1$ rad tương ứng với $x^{\circ}$. Vậy $x=\dfrac{180 \cdot 1}{\pi}$. 
}
\end{ex}

\begin{ex}%[0D6B1-1]
Nếu một cung tròn có số đo là $a^{\circ}$ thì số đo radian của nó là
\choice
{$180\pi a$}
{$\dfrac{180\pi}{a}$}
{\True $\dfrac{a\pi}{180}$}
{$\dfrac{\pi}{180a}$}
\loigiai{
Áp dụng công thức $\alpha = \dfrac{a \cdot \pi}{180}$ với $\alpha $ tính bằng radian, $a$ tính bằng độ. 
}
\end{ex}

\begin{ex}%[0D6B1-1]
Nếu một cung tròn có số đo là $3a^{\circ}$ thì số đo radian của nó là
\choice
{\True $\dfrac{a\pi}{60}$}
{$\dfrac{a\pi}{180}$}
{$\dfrac{180}{a\pi}$}
{$\dfrac{60}{a\pi}$}
\loigiai{
Áp dụng công thức $\alpha = \dfrac{a \cdot \pi}{180}$ với $\alpha $ tính bằng radian, $a$ tính bằng độ.\\
Trong trường hợp này là $3a\xrightarrow{}\alpha =\dfrac{3a\cdot \pi}{180}=\dfrac{a\pi}{60}$. 
}
\end{ex}

\begin{ex}%[0D6B1-1]
Đổi số đo của góc $70^{\circ}$ sang đơn vị radian.
\choice
{$\dfrac{70}{\pi}$}
{$\dfrac{7}{18}$}
{\True $\dfrac{7\pi}{18}$}
{$\dfrac{7}{18\pi}$}
\loigiai{
    Áp dụng công thức $\alpha =\dfrac{a \cdot \pi}{180}$ với $\alpha $ tính bằng radian, $a$ tính bằng độ.\\    
    Ta có $\alpha =\dfrac{a \cdot \pi}{180}=\dfrac{70\pi}{180}=\dfrac{7\pi}{18}$. 
    
    %{\bf Cách khác.} Bấm máy tính:\\    
    %Bước 1. Bấm \key{qw4} để chuyển về chế độ radian.\\    
    %Bước 2. Bấm \key{70x=qB1=}. Màn hình hiện ra kết quả bất ngờ.
}
\end{ex}

\begin{ex}%[0D6Y1-1]
Đổi số đo của góc $108^{\circ}$ sang đơn vị radian.
\choice
{\True $\dfrac{3\pi}{5}$}
{$\dfrac{\pi}{10}$}
{$\dfrac{3\pi}{2}$}
{$\dfrac{\pi}{4}$}
\loigiai{
Tương tự như câu trên. 
}
\end{ex}

\begin{ex}%[0D6B1-1]
Đổi số đo của góc $45^{\circ}32'$ sang đơn vị radian với độ chính xác đến hàng phần nghìn.
\choice
{$0{,}7947$}
{$0{,}7948$}
{\True $0{,}795$}
{$0{,}794$}
\loigiai{
  Áp dụng công thức $\alpha =\dfrac{a \cdot \pi}{180}$ với $\alpha $ tính bằng radian, $a$ tính bằng độ.\\
Trước tiên ta đổi $45^{\circ}32'={\left(45+\dfrac{32}{60}\right)}^{\circ}$.\\
Áp dụng công thức, ta được $\alpha =\dfrac{\left(45+\dfrac{32}{60}\right) \cdot \pi}{180}=0{,}7947065861.$ 

%{\bf Cách khác.} Bấm máy tính:\\
%Bước 1. Bấm \key{q w 4} để chuyển về chế độ radian.\\
%Bước 2. Bấm \key{45 x 32 x = q B 1 =}. Màn hình hiện ra kết quả bất ngờ.
}
\end{ex}

\begin{ex}%[0D6B1-1]
Đổi số đo của góc $40^{\circ}25'$ sang đơn vị radian với độ chính xác đến hàng phần trăm.
\choice
{$0{,}705$}
{$0{,}70$}
{$0{,}7054$}
{\True $0{,}71$}
\loigiai{
  Áp dụng công thức $\alpha =\dfrac{a \cdot \pi}{180}$ với $\alpha $ tính bằng radian, $a$ tính bằng độ.\\
Trước tiên ta đổi $40^{\circ}25'={\left(40+\dfrac{25}{60}\right)}^{\circ}$.\\
Áp dụng công thức, ta được $\alpha =\dfrac{\left(40+\dfrac{25}{60}\right) \cdot \pi}{180}=\dfrac{97\pi}{432}=0{,}705403906.$ 

%{\bf Cách khác.} Bấm máy tính:\\
%Bước 1. Bấm \key{q w 4} để chuyển về chế độ radian.\\
%Bước 2. Bấm \key{40 x 25 x = q B 1 = n}. Màn hình hiện ra kết quả bất ngờ.
}
\end{ex}

\begin{ex}%[0D6B1-1]
Đổi số đo của góc $-125^{\circ}45'$ sang đơn vị radian.
\choice
{\True $-\dfrac{503\pi}{720}$}
{$\dfrac{503\pi}{720}$}
{$\dfrac{251\pi}{360}$}
{$-\dfrac{251\pi}{360}$}
\loigiai{
Tương tự như câu trên. 
}
\end{ex}

\begin{ex}%[0D6B1-1]
Đổi số đo của góc $\dfrac{\pi}{12}$ rad sang đơn vị độ, phút, giây.
\choice
{\True $15^{\circ}$}
{$10^{\circ}$}
{$6^{\circ}$}
{$5^{\circ}$}
\loigiai{
  Từ công thức $\alpha =\dfrac{a \cdot \pi}{180}\Rightarrow a={\left(\dfrac{\alpha \cdot 180}{\pi}\right)}^{\circ}$ với $\alpha $ tính bằng radian, $a$ tính bằng độ.\\
Ta có $a={\left(\dfrac{\alpha \cdot 180}{\pi}\right)}^{\circ}={\left(\dfrac{\dfrac{\pi}{12} \cdot 180}{\pi}\right)}^{\circ}=15^{\circ}$. 

%{\bf Cách khác.} Bấm máy tính:\\
%Bước 1. Bấm \key{qw3} để chuyển về chế độ độ, phút, giây.\\
%Bước 2. Bấm \key{(qLP12)qB2=}.
%Màn hình hiện ra kết quả bất ngờ.
}
\end{ex}

\begin{ex}%[0D6B1-1]
Đổi số đo của góc $-\dfrac{3\pi}{16}$ rad sang đơn vị độ, phút, giây.
\choice
{$33^{\circ}45'$}
{$-29^{\circ}30'$}
{\True $-33^{\circ}45'$}
{$-32^{\circ}55$}
\loigiai{
  Ta có $a={\left(\dfrac{\alpha \cdot 180}{\pi}\right)}^{\circ}={\left(\dfrac{-\dfrac{3\pi}{16} \cdot 180}{\pi}\right)}^{\circ}={\left(-\dfrac{135}{4}\right)}^{\circ}=-33^{\circ}45'.$ 
  
%{\bf Cách khác.} Bấm máy tính:\\
%Bước 1. Bấm \key{qw3} để chuyển về chế độ độ, phút, giây.\\
%Bước 2. Bấm \key{(z3qLP16)qB2=nx}.
}
\end{ex}

\begin{ex}%[0D6B1-1]
Đổi số đo của góc $-5$ rad sang đơn vị độ, phút, giây.
\choice
{$-286^{\circ}44'28''$}
{\True $-286^{\circ}28'44''$}
{$-286^{\circ}$}
{$286^{\circ}28'44''$}
\loigiai{
      Ta có $a={\left(\dfrac{\alpha \cdot 180}{\pi}\right)}^{\circ}={\left(\dfrac{-5.180}{\pi}\right)}^{\circ}=-286^{\circ}28'44''.$ 
      
%{\bf Cách khác.} Bấm máy tính:\\
%Bước 1. Bấm \key{qw3} để chuyển về chế độ độ, phút, giây.\\
%Bước 2. Bấm \key{z 5 qB2=x}.
}
\end{ex}

\begin{ex}%[0D6B1-1]
Đổi số đo của góc $\dfrac{3}{4}$ rad sang đơn vị độ, phút, giây.
\choice
{$42^{\circ}9{7}'1{8}''$}
{$42^{\circ}5{8}'$}
{$42^{\circ}9{7}'$}
{\True $42^{\circ}5{8}'1{8}''$}
\loigiai{
    Tương tự như câu trên. 
}
\end{ex}

\begin{ex}%[0D6B1-1]
Đổi số đo của góc $-2$ rad sang đơn vị độ, phút, giây.
\choice
{$-114^{\circ}5{9}'1{5}''$}
{$-114^{\circ}3{5}'$}
{\True $-114^{\circ}3{5}'2{9}''$}
{$-114^{\circ}5{9}'$}
\loigiai{
Tương tự như câu trên. 
}
\end{ex}

\begin{dang}
    { ĐỘ DÀI CUNG TRÒN}
\end{dang}

\begin{ex}%[0D6Y1-2] 
Mệnh đề nào sau đây là đúng?
\choice
{\True Số đo của cung tròn tỉ lệ với độ dài cung đó}
{Độ dài của cung tròn tỉ lệ với bán kính của nó}
{Số đo của cung tròn tỉ lệ với bán kính của nó}
{Độ dài của cung tròn tỉ lệ nghịch với số đo của cung đó}
\loigiai{
Từ công thức $\ell =R\alpha \Rightarrow \ell $ và $\alpha $ tỷ lệ nhau. 
}
\end{ex}

\begin{ex}%[0D6B1-2] 
Tính độ dài $\ell $ của cung trên đường tròn có bán kính bằng $20$ cm và số đo $\dfrac{\pi}{16}.$
\choice
{\True $\ell =3{,}93$ cm}
{$\ell =2{,}94$ cm}
{$\ell =3{,}39$ cm}
{$\ell =1{,}49$ cm}
\loigiai{
Áp dụng công thức $\ell =R\alpha =20 \cdot \dfrac{\pi}{16}\approx 3{,}93$ cm. 
}
\end{ex}

\begin{ex}%[0D6B1-2] 
Tính độ dài của cung trên đường tròn có số đo $1{,}5$ và bán kính bằng $20$ cm.
\choice
{\True $30$ cm}
{$40$ cm}
{$20$ cm}
{$60$ cm}
\loigiai{
Ta có $\ell =\alpha R=1{,}5 \cdot 20=30$cm. 
}
\end{ex}

\begin{ex}%[0D6B1-2] 
Một đường tròn có đường kính bằng $20$ cm. Tính độ dài của cung trên đường tròn có số đo $35^{\circ}$(lấy $2$ chữ số thập phân).
\choice
{$6{,}01$ cm}
{\True $6{,}11$ cm}
{$6{,}21$ cm}
{$6{,}31$ cm}
\loigiai{
Cung có số đo $35^{\circ}$ thì có số đó radian là $\alpha =\dfrac{a\pi}{180}=\dfrac{35\pi}{180}=\dfrac{7\pi}{36}$. 
Bán kính đường tròn $R=\dfrac{20}{2}=10$ cm. 
Suy ra $\ell =\alpha R=\dfrac{7\pi}{36} \cdot 10\approx 6{,}11$ cm. 
}
\end{ex}

\begin{ex}%[0D6B1-2] 
Tính số đo cung có độ dài của cung bằng $\dfrac{40}{3}$ cm trên đường tròn có bán kính $20$ cm.
\choice
{$1,5$ rad}
{\True $0,67$ rad}
{$80^{\circ}$}
{$88^{\circ}$}
\loigiai{
Ta có $\ell =\alpha R\Leftrightarrow \alpha =\dfrac{\ell}{R}=\dfrac{\dfrac{40}{3}}{20}=\dfrac{2}{3}\approx 0,67$ rad. 
}
\end{ex}

\begin{ex}%[0D6B1-2]
Một cung tròn có độ dài bằng $2$ lần bán kính. Số đo $ radian$ của cung tròn đó là
\choice
{$1$}
{\True $2$}
{$3$}
{$4$}
\loigiai{
$\ell =\alpha R\Leftrightarrow \alpha =\dfrac{\ell}{R}=\dfrac{2R}{R}=2$ rad. 
}
\end{ex}

\begin{ex}%[0D6B1-2]
Trên đường tròn bán kính $R$, cung tròn có độ dài bằng $\dfrac{1}{6}$ độ dài nửa đường tròn thì có số đo (tính bằng radian) là
\choice
{$\pi /2$}
{$\pi /3$}
{$\pi /4$}
{\True $\pi /6$}
\loigiai{
Ta có $\ell =\alpha R\Leftrightarrow \alpha =\dfrac{\ell}{R}=\dfrac{\dfrac{1}{6}\pi R}{R}=\dfrac{\pi}{6}$. 
}
\end{ex}

\begin{ex}%[0D6B1-2]
Một cung có độ dài $10$ cm, có số đo bằng radian là $2{,}5$ thì đường tròn của cung đó có bán kính là
\choice
{$2{,}5$ cm}
{$3{,}5$ cm}
{\True $4$ cm}
{$4{,}5$ cm}
\loigiai{
Ta có $l=R\alpha \Leftrightarrow R=\dfrac{l}{\alpha}=\dfrac{10}{2{,}5}=4$. 
}
\end{ex}

\begin{ex}%[0D6K1-2]
Bánh xe đạp của người đi xe đạp quay được $2$ vòng trong $5$ giây. Hỏi trong $2$ giây, bánh xe quay được 1 góc bao nhiêu độ.
\choice
{\True $\dfrac{8}{5}\pi$}
{$\dfrac{5}{8}\pi$}
{$\dfrac{3}{5}\pi$}
{$\dfrac{5}{3}\pi$}
\loigiai{
    Trong $2$ giây bánh xe đạp quay được $\dfrac{2 \cdot 2}{5}=\dfrac{4}{5}$ vòng tức là quay được cung có độ dài là $l=\dfrac{4}{5} \cdot 2\pi R=\dfrac{8}{5}\pi R$. \\
Ta có $l=R\alpha \Leftrightarrow \alpha =\dfrac{l}{R}=\dfrac{\dfrac{8}{5}\pi R}{R}=\dfrac{8}{5}\pi.$ 
}
\end{ex}

\begin{ex}%[0D6K1-2]
Một bánh xe có $72$ răng. Số đo góc mà bánh xe đã quay được khi di chuyển $10$ răng là
\choice
{$30^{\circ}$}
{$40^{\circ}$}
{$50^{\circ}$}
{$60^{\circ}$} 
\loigiai{
  $72$ răng có chiều dài là $2\pi R$ nên $10$ răng có chiều dài $l=\dfrac{10 \cdot 2\pi R}{72}=\dfrac{5\pi}{18}R$.\\ 
Theo công thức $l=R\alpha \Leftrightarrow \alpha =\dfrac{l}{R}=\dfrac{\dfrac{5}{18}\pi R}{R}=\dfrac{5}{18}\pi $ mà $a=\dfrac{180\alpha}{\pi}=\dfrac{180 \cdot \dfrac{5}{18}\pi}{\pi}=50^{\circ}$.\\
{\bf Cách khác.} $72$ răng tương ứng với $360^{\circ}$ nên $10$ răng tương ứng với $\dfrac{10 \cdot 360}{72}=50^{\circ}$.
}
\end{ex}
\begin{dang}
    { GÓC LƯỢNG GIÁC}
\end{dang}

\begin{ex}%[0D6B1-3] 
Cho góc lượng giác $\left(Ox,Oy\right)=22^{\circ}30'+k360^{\circ}.$ Với giá trị $k$ bằng bao nhiêu thì góc $\left(Ox,Oy\right)=1822^{\circ}30'$? 
\choice
{$k\in \varnothing$}
{$k=3$}
{$k=-5$}
{\True $k=5$}
\loigiai{
Theo đề $\left(Ox,Oy\right)=1822^{\circ}30'\Rightarrow{}22^{\circ}30'+k{.{360}^{\circ}}=1822^{\circ}30'\Rightarrow{}k=5.$
}
\end{ex}

\begin{ex}%[0D6B1-3] 
Cho góc lượng giác $\alpha =\dfrac{\pi}{2}+k2\pi $. Tìm $k$ để $10\pi <\alpha <11\pi.$
\choice
{$k=4$}
{\True $k=5$}
{$k=6$}
{$k=7$}
\loigiai{
Ta có $10\pi <\alpha <11\pi \Rightarrow{}\dfrac{19\pi}{2}<k2\pi <\dfrac{21\pi}{2}\Rightarrow{}k=5.$ 
}
\end{ex}

\begin{ex}%[0D6B1-4]
Một chiếc đồng hồ, có kim chỉ giờ $OG$ chỉ số $9$ và kim phút $OP$ chỉ số $12$. Số đo của góc lượng giác $\left(OG,OP\right)$ là
\choice
{\True $\dfrac{\pi}{2}+k2\pi,k\in \mathbb{Z}$}
{$-270^{\circ}+k360^{\circ},k\in \mathbb{Z}.$}
{$270^{\circ}+k360^{\circ},k\in \mathbb{Z}$}
{$\dfrac{9\pi}{10}+k2\pi,k\in \mathbb{Z}$}
\loigiai{
Góc lượng giác $\left(OG,OP\right)$ chiếm $\dfrac{1}{4}$ đường tròn. Số đo là $\dfrac{1}{4}.2\pi+k2\pi $, $k\in \mathbb{Z}$.
}
\end{ex}

\begin{ex}% [0D6B1-3] 
Trên đường tròn lượng giác có điểm gốc là $A$. Điểm $M$ thuộc đường tròn sao cho cung lượng giác $AM$ có số đo $45^{\circ}$. Gọi $N$ là điểm đối xứng với $M$ qua trục $Ox$, số đo cung lượng giác $AN$ bằng
\choice
{$-45^{\circ}$}
{$315^{\circ}$}
{$45^{\circ}$ hoặc $315^{\circ}$}
{\True $-45^{\circ}+k360^{\circ},k\in \mathbb{Z}$}
\loigiai{
Vì số đo cung $AM$ bằng $45^{\circ}$ nên $\widehat{AOM}=45^{\circ}$, $N$ là điểm đối xứng với $M$ qua trục $Ox$ nên $\widehat{AON}=45^{\circ}$. Do đó số đo cung $AN$ bằng $45^{\circ}$ nên số đo cung lượng giác $AN$ có số đo là $-45^{\circ}+k360^{\circ},k\in \mathbb{Z}$.

}
\end{ex}

\begin{ex}%[0D6B1-3] 
Trên đường tròn với điểm gốc là $A$. Điểm $M$ thuộc đường tròn sao cho cung lượng giác $AM$ có số đo $60^{\circ}$. Gọi $N$ là điểm đối xứng với điểm $M$ qua trục $Oy$, số đo cung $AN$ là
\choice
{\True $120^{\circ}$}
{$-240^{\circ}$}
{$-120^{\circ}$ hoặc $240^{\circ}$}
{$120^{\circ}+k360^{\circ},k\in \mathbb{Z}$}
\loigiai{
\immini{Ta có $\widehat{AOM}=60^{\circ}$, $\widehat{MON}=60^{\circ}$ .\\
Nên $\widehat{AON}=120^{\circ}$. \\
Khi đó số đo cung $AN$ bằng $120^{\circ}$.
}{
    \begin{tikzpicture}[>=stealth, scale= 1.3]
		\draw[->] (-1.4,0) -- (1.4,0) node[above] {$x$};
		\draw[->] (0,-1.4) -- (0,1.4) node[right] {$y$};
		\tkzDefPoints{0/0/O, 1/0/A}
		\tkzDefPointBy[rotation = center O angle 60](A) \tkzGetPoint{M}
		\tkzDefPointBy[rotation = center O angle 120](A) \tkzGetPoint{N}
                  	\tkzDrawCircle[radius](O,A)
                  	\tkzDrawSegments(O,M O,N M,N)
		\tkzDrawPoints(O,M,N,A)
		\tkzLabelPoints[below left](O)
		\tkzLabelPoints[below right](A)
		\tkzLabelPoints[above left](N)
		\tkzLabelPoints[above right](M)
		\end{tikzpicture} 
}
}
\end{ex}

\begin{ex}%[0D6B1-3] 
Trên đường tròn lượng giác với điểm gốc là $A$. Điểm $M$ thuộc đường tròn sao cho cung lượng giác $AM$ có số đo $75^{\circ}$. Gọi $N$ là điểm đối xứng với điểm $M$ qua gốc tọa độ $O$, số đo cung lượng giác $AN$ bằng
\choice
{$255^{\circ}$}
{$-105^{\circ}$}
{$-105^{\circ}$ hoặc $255^{\circ}$}
{\True $-105^{\circ}+k360^{\circ},k\in \mathbb{Z}$}
\loigiai{
\immini{
    Ta có $\widehat{AOM}=75^{\circ}$, $\widehat{MON}=180^{\circ}$.\\
    Nên cung lượng giác $AN$ có số đo bằng \\
    $-105^{\circ}+k360^{\circ},k\in \mathbb{Z}$. \\
    

}{
    \begin{tikzpicture}[>=stealth, scale= 1.3]
		\draw[->] (-1.4,0) -- (1.4,0) node[above] {$x$};
		\draw[->] (0,-1.4) -- (0,1.4) node[right] {$y$};
		\tkzDefPoints{0/0/O, 1/0/A}
		\tkzDefPointBy[rotation = center O angle 75](A) \tkzGetPoint{M}
		\tkzDefPointBy[symmetry = center O](M)\tkzGetPoint{N}
                  	\tkzDrawCircle[radius](O,A)
                  	\tkzDrawSegments(O,M O,N M,N)
		\tkzDrawPoints(O,M,N,A)
		\tkzLabelPoints[below left](O,N)
		\tkzLabelPoints[below right](A)
		\tkzLabelPoints[above right](M)
		\end{tikzpicture} 
}
}
\end{ex}

\begin{ex}%[0D6B1-3] 
Cho bốn cung (trên một đường tròn định hướng): $\alpha =-\dfrac{5\pi}{6},\beta =\dfrac{\pi}{3}$, $\gamma =\dfrac{25\pi}{3},\delta =\dfrac{19\pi}{6}$. Các cung nào có điểm cuối trùng nhau?
\choice
{$\alpha $ và $\beta $; $\gamma $ và $\delta $}
{\True $\beta $ và $\gamma $; $\alpha $ và $\delta $}
{$\alpha,\beta,\gamma $}
{$\beta,\gamma,\delta $}
\loigiai{
  Ta có $\delta-\alpha =4\pi \Rightarrow $ hai cung $\alpha $ và $\delta $ có điểm cuối trùng nhau. \\
Và $\gamma-\beta =8\pi \Rightarrow $ hai cung $\beta $ và $\gamma $ có điểm cuối trùng nhau. \\
{\bf Cách khác.} Gọi $A,B,C,D$ là điểm cuối của các cung $\alpha,\beta,\gamma,\delta $.\\
Biểu diễn các cung trên đường tròn lượng giác ta có $B\equiv C,A\equiv D.$ 
}
\end{ex}

\begin{ex}%[0D6B1-3] 
Các cặp góc lượng giác sau ở trên cùng một đường tròn đơn vị, cùng tia đầu và tia cuối. Hãy nêu kết quả {\bf SAI} trong các kết quả sau đây
\choice
{$\dfrac{\pi}{3}$ và $-\dfrac{35\pi}{3}$}
{\True $\dfrac{\pi}{10}$ và $\dfrac{152\pi}{5}$}
{$-\dfrac{\pi}{3}$ và $\dfrac{155\pi}{3}$}
{$\dfrac{\pi}{7}$ và $\dfrac{281\pi}{7}$}
\loigiai{
Cặp góc lượng giác $a$ và $b$ ở trên cùng một đường tròn đơn vị, cùng tia đầu và tia cuối. Khi đó $a=b+k2\pi $, $k\in \mathbb{Z}$ hay $k=\dfrac{a-b}{2\pi}$.\\
Dễ thấy, ở {\bf đáp án B} vì $k=\dfrac{\dfrac{\pi}{10}-\dfrac{152\pi}{5}}{2\pi}=-\dfrac{303}{20}\notin \mathbb{Z}$. 
}
\end{ex}

\begin{ex}%[0D6K1-3]
Trên đường tròn lượng giác gốc $A$, cung lượng giác nào có các điểm biểu diễn tạo thành tam giác đều?
\choice
{\True $\dfrac{k2\pi}{3}$}
{$k\pi $}
{$\dfrac{k\pi}{2}$}
{$\dfrac{k\pi}{3}$}
\loigiai{
Tam giác đều có góc ở đỉnh là $60^{\circ}$ nên góc ở tâm là $120^{\circ}$ tương ứng $\dfrac{k2\pi}{3}$. 
}
\end{ex}

\begin{ex}%[0D6K1-3]
Trên đường tròn lượng giác gốc $A$, cung lượng giác nào có các điểm biểu diễn tạo thành hình vuông?
\choice
{\True $\dfrac{k\pi}{2}$}
{$k\pi $}
{$\dfrac{k2\pi}{3}$}
{$\dfrac{k\pi}{3}$ }
\loigiai{
    \immini{Hình vẽ tham khảo (hình vẽ bên).
Hình vuông $CDEF$ có góc $\widehat{DCE}$ là $45^{\circ}$ 
nên góc ở tâm là $90^{\circ}$ tương ứng $\dfrac{k\pi}{2}.$ 

}{
    \begin{tikzpicture}[>=stealth, scale= 1.3]
		\draw[->] (-1.4,0) -- (1.4,0) node[above] {$x$};
		\draw[->] (0,-1.4) -- (0,1.4) node[left] {$y$};
		\tkzDefPoints{0/0/O, 1/0/A, 0/1/B}
		\tkzDefPointBy[rotation = center O angle 30](A) \tkzGetPoint{F}
		\tkzDefPointBy[symmetry = center O](F)\tkzGetPoint{D}
		\tkzDefPointBy[rotation = center O angle 90](F) \tkzGetPoint{C}
		\tkzDefPointBy[symmetry = center O](C)\tkzGetPoint{E}
		\tkzDefPointBy[symmetry = center O](A)\tkzGetPoint{A'}
		\tkzDefPointBy[symmetry = center O](B)\tkzGetPoint{B'}
                  	\tkzDrawCircle[radius](O,A)
                  	\tkzDrawSegments(O,D E,C E,F C,F D,C E,D)
		\tkzLabelPoints[below left](D,B')
		\tkzLabelPoints[below right](A,E)
		\tkzLabelPoints[above right](F,B,O)
		\tkzLabelPoints[above left](C,A')
		\end{tikzpicture} 
}
}
\end{ex}

% \Closesolutionfile{ans}
% \begin{center}
% \bf ĐÁP ÁN
% \end{center}
% \begin{multicols}{10}
% \input{ans/ans10D1-TN-1}
% \end{multicols}