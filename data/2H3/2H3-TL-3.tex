\Opensolutionfile{ans}[ans/ans-2H3-TL-3]
\setcounter{dang}{0}
\setcounter{vd}{0}
\section{PTĐT TRONG KHÔNG GIAN}
	\subsection{Tóm tắt lý thuyết}
	\begin{tomtat}
	\subsubsection{PTTS của đường thẳng}
	\begin{dn}
	PTTS của đường thẳng $d$ qua $M(x_0;y_0;z_0)$ và có véc-tơ chỉ phương $\vec{a}=(a_1;a_2;a_3)$ là phương trình có dạng $ \begin{cases}
	x=x_0+a_1t\\
	y=y_0+a_2t\\
	z=z_0+a_3t
	\end{cases}$ trong đó $ t $ là tham số.
	\end{dn}
	\begin{note}
	Phương trình chính tắc của đường thẳng $d$ qua $M(x_0;y_0;z_0)$ và có véc-tơ chỉ phương $\vec{a}=(a_1;a_2;a_3)$ là $d: \dfrac{x-x_0}{a_1}=\dfrac{y-y_0}{a_2}=\dfrac{z-z_0}{a_3}$ với $abc\ne 0$.
	\end{note}
	\subsubsection{Điều kiện để hai đường thẳng song song, cắt nhau, chéo nhau}	
	Cho đường thẳng $d_1$ có véc-tơ chỉ phương $\vec{u}=(u_1;u_2;u_3)$ và đi qua điểm $M_1$ và đường thẳng $d_2$ có véc-tơ chỉ phương $\vec{v}=(v_1;v_2;v_3)$ và đi qua điểm $M_2$.
	\begin{eqnarray*}
	d_1 \equiv d_2&\Leftrightarrow& \heva{&\left[\vec{u};\vec{v}\right]=\vec{0}\\&\left[\vec{u};\vec{M_1M_2}\right]=\vec{0}}.\\
	d_1 \parallel d_2&\Leftrightarrow&\heva{&\left[\vec{u};\vec{v}\right]=\vec{0}\\&\left[\vec{u};\vec{M_1M_2}\right]\ne \vec{0}}.\\
	d_1 \text{ cắt } d_2&\Leftrightarrow&\heva{&\left[\vec{u};\vec{v}\right]\ne\vec{0}\\&\left[\vec{u};\vec{v}\right]\cdot\vec{M_1M_2}= \vec{0}}\\
	d_1 \text{ chéo } d_2&\Leftrightarrow&\heva{&\left[\vec{u};\vec{v}\right]\ne\vec{0}\\&\left[\vec{u};\vec{v}\right]\cdot\vec{M_1M_2}\ne \vec{0}}.\\
	\end{eqnarray*}
	\subsubsection{Góc}
	\begin{enumerate}
	\item Cho đường thẳng $d$ có véc-tơ chỉ phương $\vec{u}$ và mặt phẳng $ (P) $ có véc-tơ pháp tuyến $ \vec{n} $. Góc giữa đường thẳng $d$ và mặt phẳng $ (P) $: $$\sin\varphi=|\cos\left(\vec{u},\vec{n}\right)| \quad \left(0\leq \varphi \leq \dfrac{\pi}{2}\right)$$
	\item Cho hai đường thẳng chéo nhau $d_1$ có véc-tơ chỉ phương $\vec{u}$ và $d_2$ có véc-tơ chỉ phương $\vec{v}$. Góc giữa hai đường thẳng $ d_1 $ và $ d_2 $:
	$$\left(d_1,d_2\right)=\dfrac{\left|\left[\vec{u};\vec{v}\right]\cdot\vec{M_1M_2}\right|}{|\left[\vec{u};\vec{v}\right]|} $$
	\end{enumerate}
	\subsubsection{Khoảng cách}
	\begin{enumerate}
	\item Cho đường thẳng $d$ có véc-tơ chỉ phương $\vec{u}$, đi qua điểm $M_0$ và điểm $ M $. Khoảng cách từ điểm $ M $ đến đường thẳng $ d $:
	$$ d\left(M,d\right)=\dfrac{|\left[\vec{u};\vec{M_0M}\right]|}{|\vec{u}|}$$
	\item Cho hai đường thẳng chéo nhau $d_1$ có véc-tơ chỉ phương $\vec{u}$, đi qua điểm $M_1 $ và $d_2$ có véc-tơ chỉ phương $\vec{v}$, đi qua điểm $M_2$. Khoảng cách giữa $ d_1 $ và $ d_2 $:
	$$\left(d_1,d_2\right)=\dfrac{\left|\left[\vec{u};\vec{v}\right]\cdot\vec{M_1M_2}\right|}{|\left[\vec{u};\vec{v}\right]|}$$
	\end{enumerate}
	\end{tomtat}
 \subsection{Dạng toán và bài tập}
	\begin{dang}{Xác định vectơ chỉ phương}%dang 1
		\begin{itemize}
			\item Đường thẳng $(d)$ đi qua hai điểm $A$, $B$, khi đó véc-tơ $\overrightarrow{AB}$ là một chỉ phương của $(d)$.
			\item Đường thẳng $(d)$ song song với đường thẳng $(l)$, khi đó véc-tơ chỉ phương của $(l)$ cũng là một chỉ phương của $(d)$.
			\item Đương thẳng $(d)$ vuông góc với mặt phẳng $(\alpha)$, khi đó véc-tơ pháp tuyến của $(\alpha)$ là một chỉ phương của $(d)$.
			\item Đường thẳng $(d)$ là giao tuyến của $(P)\colon A_1x+B_1y+C_1z+D_1=0$, mặt phẳng $(Q)\colon A_2x+B_2y+C_2z+D_2=0$ có véc-tơ chỉ phương của $(d)$, $\overrightarrow{u}=\left[\overrightarrow{n}_{P},\overrightarrow{n}_{Q}\right] $
			\item Đường thẳng $(d)$ đi qua điểm $M$ và vuông góc hai đường thẳng $(d_1)$, $(d_2)$. Khi đó ta gọi $\overrightarrow{u}$ là một véc-tơ chỉ phương của $(d)$ thì $\heva{&\overrightarrow{u}\perp\overrightarrow{u_1}\\&\overrightarrow{u}\perp\overrightarrow{u_2}}$ với $\overrightarrow{u_1}$, $\overrightarrow{u_2}$ lần lượt là chỉ phương của $(d_1)$, $(d_2)$ nên ta chọn $\overrightarrow{u}=\left[ \overrightarrow{u_1},\overrightarrow{u_2}\right] $.
			\item Đường thẳng $d$ đi qua điểm $M$, cắt và vuông góc với một đường thẳng $d_1$ cho trước. Gọi $H$ là hình chiếu vuông góc của $M$ lên đường thẳng $d_1$ cho trước . Dựa vào điều kiện $\overrightarrow{MH}\cdot\overrightarrow{u}_{l}=0$ ta tìm được $H$. Khi đó $\overrightarrow{MH} $ là  VTCP cần tìm.
			\item Đường thẳng đi qua điểm $M$, vuông góc với $(d_1)$ và cắt $(d_2)$.Gọi $K$ là giao điểm của $(d)$ và $(d_2)$. Ta có $MK\perp (d_1)$ nên $\overrightarrow{MK}\cdot\overrightarrow{u}_{d_1}=0$, từ đó ta tìm được véc-tơ $\overrightarrow{MK}$ chính là chỉ phương của $(d)$.
			\item Đường thẳng  $d$ đi qua điểm $M$ cắt cả hai đường thẳng $(d_1)$ và $(d_2)$.	Gọi $(a)$ là mặt phẳng chứa $(d_1)$ và đi qua điểm $M$, $(b)$ là mặt phẳng chứa $(d_2)$ và đi qua điểm $M$. Khi đó đường thẳng giao tuyến của hai mặt phẳng $(a)$ và $(b)$ là đường thẳng $(d)$ cần tìm.
			\item Đường thẳng $(d)$ nằm trong mặt phẳng $(P)$ cắt cả hai đường thẳng $(d_1)$, $(d_2)$.Ta cần tìm điểm $M$ là giao điểm của $(P)$ và $(d_1)$, điểm $N$ là giao điểm của $(P)$ và $(d_2)$. Khi đó đường thẳng $(d)$ đi qua hai điểm $M$, $N$ là đường thẳng cần tìm.
			\end{itemize}
	\end{dang}
	\subsubsection{Ví dụ minh hoạ}
	\begin{vd}%[Thi thử L2, Kinh Môn, Hải Dương 2018]%[2H3B3-1]%[Nguyễn Phúc Đức, 12EX-7]
	Trong KG $Oxyz$, cho đường thẳng $d:\dfrac{x+8}{4}=\dfrac{y-5}{-2}=\dfrac{z}{1}$. Xác định một véc-tơ chỉ phương của đường thẳng $d$.
	\loigiai{Tọa độ véc-tơ chỉ phương của đường thẳng $d$ là $\overrightarrow{u}=(4;-2;1)$.
	}
	\end{vd}
	\begin{vd}%[KSCL lần 2, Chuyên Lam Sơn Thanh Hóa - 2018]%[2H3B3-1]%[Vũ Văn Trường, dự án(12EX-7)]
	Trong KG $Oxyz$, cho hai mặt phẳng $(P): 3x-2y+2z-5=0$, $(Q): 4x+5y-z+1=0$. Gọi $ d $ là giao tuyến của hai mặt phẳng $(P)$ và $(Q)$. Xác định một véc-tơ chỉ phương của đường thẳng $d$.
	\loigiai{
	Ta có $\overrightarrow{n}=(3;-2;2)$ và $\overrightarrow{n'}=(4;5;-1)$ lần lượt là các véc-tơ pháp tuyến của các mặt phẳng $(P), (Q)$. Do đó $\left[\overrightarrow{n}, \overrightarrow{n'}\right]=(-8;11;23)$ là một véc-tơ chỉ phương của giao tuyến $ d $ của $(P)$ và $(Q)$.
	}
	\end{vd}
	\begin{vd}%[KSCL L2, Yên Phong 2, Bắc Ninh 2018]%[2H3B3-1]%[Khuất Văn Thanh, 12EX-8]
	Trong KG $Oxyz$ cho hai mặt phẳng $(P)\colon -x-2y+5z-2017=0$, $(Q)\colon 2x-y+3z+2018=0$. Gọi $\Delta$ là giao tuyến của $(P)$ và $(Q)$. Xác định một véc-tơ chỉ phương của đường thẳng $\Delta$?
	\loigiai{
	$(P)$ có véc-tơ pháp tuyến $\vec{n}_P=(-1;-2;5)$.\\
	$(Q)$ có véc-tơ pháp tuyến $\vec{n}_Q=(2;-1;3)$.\\
	Suy ra $\left[\vec{n}_P,\vec{n}_Q\right]=(-1;13;5)\neq\vec{0}$.\\
	Vậy $\Delta$ có véc-tơ chỉ phương là $\vec{u}_{\Delta}=\left[\vec{n}_P,\vec{n}_Q\right]=(-1;13;5)$.
	}
	\end{vd}
	\begin{vd}%[2-GHK2-96-ThithuTHTT-Lan7]%[2H3B3-1]%[Phạm Tuấn, 12EX-9]
	Trong KG $Oxyz$ cho đường thẳng $d \colon \dfrac{x+3}{2} = \dfrac{y-1}{1}=\dfrac{z-1}{-3}$. Xác định véc-tơ chỉ phương của $ d' $ là hình chiếu vuông góc của $d$ trên mặt phẳng $(Oyz)$.
	\loigiai{
	Chọn $A(-3;1;1), B(-1;2;-2)$ thuộc $d$, ta có các điểm $A'(0;1;1)$, $B'(0;2;-2)$ là hình chiếu vuông góc của $A,B$ trên mặt phẳng $(Oyz)$, khi đó $\overrightarrow{u} =\overrightarrow{A'B'} = (0;1;-3)$.
	}
	\end{vd}
	\subsubsection{Bài tập trắc nghiệm}
	\begin{ex}%[Thi thử lần 1, chuyên KHTN, 2018]%[Trần Chiến, dự án EX5]%[2H3B3-1]
	Trong KG $Oxyz$, cho đường thẳng $d: \heva{& x=1-t\\&y=-2-2t\\& z=1+t}. $ Véc-tơ nào dưới đây là véc-tơ chỉ phương của $d$? 
	\choice
	{$\vec{n}=(1;-2;1)$}
	{$\vec{n}=(1;2;1)$}
	{\True $\vec{n}=(-1;-2;1)$}
	{$\vec{n}=(-1;2;1)$}
	\loigiai{Từ PTTS, suy ra véc-tơ chỉ phương là $\vec{n}=(-1;-2;1)$.	
	}
	\end{ex}
	\begin{ex}%[Toán Học Tuổi Trẻ-Lần 6-2018]%[2H3B3-1]%[Phạm Tuấn, 12EX-7]
	Trong KG $Oxyz$, cho tam giác $ABC$ với $A(1;1;1)$, $B(-1;1;0)$, $C(1;3;2)$. Đường trung tuyến xuất phát từ đỉnh $A$ của tam giác $ABC$ nhận véc-tơ $\overrightarrow{a}$ nào dưới đây làm một véc-tơ chỉ phương?
	\choice
	{$\overrightarrow{a}=(1;1;0)$}
	{$\overrightarrow{a}=(-2;2;2)$}
	{$\overrightarrow{a}=(-1;2;1)$}
	{\True $\overrightarrow{a}=(-1;1;0)$}
	\loigiai{
	Tọa độ trung điểm của $BC$ là $M(0;2;1)$, $\overrightarrow{AM}=(-1;1;0)$. 
	}
	\end{ex}
	\begin{ex}%[Đề TT, Chuyên Lê Khiết, Quảng Ngãi 2018]%[2H3B3-1]%[Lê Quốc Hiệp, dự án(12EX-7)]
	Trong KG $Oxyz$, cho đường thẳng $d:\dfrac{x-2}{-1}=\dfrac{1-y}{2}=\dfrac{z}{1}$. Véc-tơ nào dưới đây là véc-tơ chỉ phương của đường thẳng $d$?
	\choice
	{$\vec{m}=(-1;2;1)$}
	{$\vec{n}=(1;2;1)$}
	{$\vec{p}=(-1;2;-1)$}
	{\True $\vec{q}=(1;2;-1)$}
	\loigiai{ $d$ có véc-tơ chỉ phương $\vec{u}=(-1;-2;1)=-\vec{q}$.}
	\end{ex}
	\begin{ex}%[Đề TT lần 1, Chuyên Nguyễn Thị Minh Khai, Sóc Trăng 2018]%[2H3B3-1]%[Hung Tran,12EX-8]
	Trong không gian Oxyz , cho đường thẳng $d\colon\heva{x=&1-2t\\y=&-2+4t\\z=&1}$. Đường thẳng $d$ có một véc-tơ chỉ phương là
	\choice
	{$\vec{u_4}=(-2;4;1)$}
	{$\vec{u_1}=(2;4;0)$}
	{\True $\vec{u_2}=(1;-2;0)$}
	{$\vec{u_3}=(1;-2;1)$}
	\loigiai{Đường thẳng $d$ có véc-tơ chỉ phương là $\vec{u}=(-2;4;0)$ nên có một véc-tơ chỉ phương là\\ $\vec{u_2}=(1;-2;0)$.}
	\end{ex}
	\begin{ex}%[TT lần 2, Đinh Tiên Hoàng - Ninh Bình - 2018]%[2H3B3-1]%[Vũ Văn Trường, dự án(12EX-8)]
	Trong KG $Oxyz$, cho đường thẳng $\Delta\colon \dfrac{x-1}{1}=\dfrac{y-2}{3}=\dfrac{z-3}{-1}$. Gọi $\Delta'$ là đường thẳng đối xứng với đường thẳng $\Delta$ qua $(Oxy)$. Tìm một véc-tơ chỉ phương của đường thẳng $\Delta'$.
	\choice
	{$\overrightarrow{u}=(-1;3;-1)$}
	{$\overrightarrow{u}=(1;2;-1)$}
	{$\overrightarrow{u}=(1;3;0)$}
	{\True $\overrightarrow{u}=(1;3;1)$}
	\loigiai{
	Đường thẳng $\Delta$ cắt mặt phẳng $(Oxy)$ tại điểm $A(4;11;0)$.\\
	Ta thấy $B(1;2;3)\in \Delta$ và $B'(1;2;-3)$ là điểm đối xứng của điểm $B$ qua mặt phẳng $(Oxy)$.\\
	Đường thẳng $\Delta'$ đi qua các điểm $A, B'$. Ta có $\overrightarrow{AB'}=(-3;-9;-3)$, từ đó suy ra $\overrightarrow{u}=(1;3;1)$ là một véc-tơ chỉ phương của đường thẳng $\Delta'$.
	}
	\end{ex}
	\begin{ex}%[Thi thử L1, Đặng Thúc Hứa, Nghệ An, 2018]%[2H3B3-1]%[Nguyễn Văn Vũ, 12EX-8]
	Trong KG $Oxyz$, cho hai điểm $A(1;2;2)$, $B(3;-2;0)$. Một véc-tơ chỉ phương của đường thẳng $AB$ là
	\choice
	{\True $\overrightarrow{u}=(-1;2;1)$}
	{$\overrightarrow{u}=(1;2;-1)$}
	{$\overrightarrow{u}=(2;-4;2)$}
	{$\overrightarrow{u}=(2;4;-2)$}
	\loigiai{Đường thẳng $AB$ có một véc-tơ chỉ phương là $\overrightarrow{AB}=(2;-4;-2)$. Nhận thấy $\overrightarrow{u}=(-1;2;1)$ cùng phương với $\overrightarrow{AB}$ nên $\overrightarrow{u}=(-1;2;1)$ cũng là một véc-tơ chỉ phương của đường thẳng $AB$.
	}
	\end{ex}
	\begin{ex}%[Đề thi thử - Trường THPT chuyên Lương Thế Vinh - Đồng Nai - Lần 1 - 2018]%[2H3B3-1]%[Kim Minh Bui - 12EX8]
	Trong KG $Oxyz$, cho đường thẳng $d:\begin{cases}x=1-2t \\ y=3 \\ z=5+3t \end{cases}$. Trong các vec-tơ sau, vec-tơ nào là một vec-tơ chỉ phương của đường thẳng $d$?
	\choice
	{$\overrightarrow{a}_1=(1;3;5)$}
	{$\overrightarrow{a}_2=(2;3;3)$}
	{\True $\overrightarrow{a}_3=(-2;0;3)$}
	{$\overrightarrow{a}_1=(-2;3;3)$}
	\loigiai{Đường thẳng $d:\begin{cases}x=1-2t \\ y=3 \\ z=5+3t \end{cases}$ nhận $\overrightarrow{a}_3=(-2;0;3)$ làm một vec-tơ chỉ phương.}
	\end{ex}
	\begin{ex}%[Đề KSCL học kỳ 2 Toán 12 năm học 2017 – 2018 sở GD và ĐT Nam Định] %[Trần Tuấn Việt, 12EX-9-2018]%[2H3B3-1]
	Trong KG $Oxyz$, cho tam giác $ABC$ có phương trình đường phân giác trong góc $A$ là $\dfrac{x}{1} = \dfrac{y - 6}{-4} = \dfrac{z - 6}{-3}$. Biết rằng điểm $M(0; 5; 3)$ thuộc đường thẳng $AB$ và điểm $N(1; 1; 0)$ thuộc đường thẳng $AC$. Véc-tơ nào sau đây là véc-tơ chỉ phương của đường thẳng $AC$?
	\choice
	{$\overrightarrow{u}(1; 2; 3)$}
	{$\overrightarrow{u}(0; -2; 6)$}
	{$\overrightarrow{u}(0; 1; -3)$}
	{\True $\overrightarrow{u}(0; 1; 3)$}
	\loigiai{$\bullet$ Hình chiếu $H$ của $M$ trên đường phân giác trong góc $A$ có tọa độ: $H\left(\dfrac{1}{2}; 4; \dfrac{9}{2}\right)$.\\
	$\bullet$ $M'$ là điểm đối xứng của $M$ qua $H$. Từ đây ta tìm được tọa độ $M'(1; 3; 6)$.\\
	$\bullet$ Véc-tơ chỉ phương của đường thẳng $AC$ chính là véc-tơ $\vec{NM'} = \left(0; 2; 6\right)$.\\
	Suy ra, đường thẳng $AC$ có một véc-tơ chỉ phương là $(0; 1; 3)$.
	}
	\end{ex}
	\begin{ex}%[2-GHK2-96-ThithuTHTT-Lan7]%[2H3B3-1]%[Phạm Tuấn, 12EX-9]
	Trong KG $Oxyz$ cho đường thẳng $d \colon \dfrac{x+3}{2} = \dfrac{y-1}{1}=\dfrac{z-1}{-3}$. Hình chiếu vuông góc của $d$ trên mặt phẳng $(Oyz)$ là một đường thẳng có véc-tơ chỉ phương là
	\choice
	{$\overrightarrow{u}=(0;1;3)$}
	{\True $\overrightarrow{u}=(0;1;-3)$}
	{$\overrightarrow{u}=(2;1;-3)$}
	{$\overrightarrow{u}=(2;0;0)$}
	\loigiai{
	Chọn $A(-3;1;1), B(-1;2;-2)$ thuộc $d$, ta có các điểm $A'(0;1;1)$, $B'(0;2;-2)$ là hình chiếu vuông góc của $A,B$ trên mặt phẳng $(Oxy)$, khi đó $\overrightarrow{u} =\overrightarrow{A'B'} = (0;1;-3)$.
	}
	\end{ex}
	\begin{ex}%[Học kì 2, THPT Chương Mỹ B, 2018]%[Vương Quyền, dự án (12EX-9)]%[2H3B3-1]
	Trong KG $Oxyz$, cho các đường thẳng có phương trình sau 
	\begin{align*}
	(d_1)\colon\heva{& x=2+2t\\& y=-3t\\& z=-3+5t},
	\quad (d_2)\colon\heva{& x=2-4t\\&y=6t\\& z=-3-10t},
	\quad (d_3)\colon\heva{& x=4+2t\\& y=3-6t\\& z=2+5t}
	\end{align*}
	Trong các đường thẳng trên, đường thẳng nào đi qua điểm $M(2;0;-3)$ và nhận véc-tơ $\overrightarrow{a}=(2;-3;5)$ làm véc-tơ chỉ phương?
	\choice
	{\True Chỉ có $d_1$, $d_2$}
	{Chỉ có $d_1$, $d_3$}
	{Chỉ có $d_1$}
	{Chỉ có $d_2$}
	\loigiai{
	Ta có $\overrightarrow{u}_{d_1}=(2;-3;5)=\overrightarrow{a}$; $\overrightarrow{u}_{d_2}=(-4;6;-10)=-2(2;-3;5)=\overrightarrow{a}$; $\overrightarrow{u}_{d_3}=(2;-6;5)\neq \overrightarrow{a}$.\\
	Do đó $\overrightarrow{u}_{d_1}$, $\overrightarrow{u}_{d_2}$ cùng phương với $\overrightarrow{a}$ và $\overrightarrow{u}_{d_3}$ không cùng phương với $\overrightarrow{a}$.\\
	Thay $M(2;0;-3)$ vào $d_1$; $d_2$ được $t=0\Rightarrow M\in d_1$; $d_2$.
	}
	\end{ex}
	\begin{ex}%[Học kì 2, THPT Chương Mỹ B, 2018]%[Vương Quyền, dự án (12EX-9)]%[2H3B3-1]
	Trong KG $Oxyz$, cho đường thẳng $d\colon\dfrac{x-1}{5}=\dfrac{y-2}{-8}=\dfrac{z+3}{7}$. véc-tơ nào dưới đây là một véc-tơ chỉ phương của đường thẳng $d$?
	\choice
	{$\overrightarrow{u}_2=(-1;-2;3)$}
	{$\overrightarrow{u}_4=(7;-8;5)$}
	{\True $\overrightarrow{u}_3=(5;-8;7)$}
	{$\overrightarrow{u}_1=(1;2;-3)$}
	\loigiai{
	Đường thẳng $d\colon\dfrac{x-1}{5}=\dfrac{y-2}{-8}=\dfrac{z+3}{7}$ có véc-tơ chỉ phương là $\overrightarrow{u}=(5;-8;7)=\overrightarrow{u}_3$.
	}
	\end{ex}
	\begin{ex}%[TT Chuyên Hùng Vương Bình Dương L5, 2018]%[Lê Minh An, 12EX-10]%[2H3B3-1]
	Cho đường thẳng $d\colon \dfrac{x-1}{2}=\dfrac{3-y}{3}=\dfrac{z+1}{-2}$. Một véc-tơ chỉ phương của đường thẳng $d$ là
	\choice{\True $\overrightarrow{u}=(2;3;-2)$}
	{\True $\overrightarrow{u}=(2;-3;-2)$}
	{$\overrightarrow{u}=(-2;-3;-2)$}
	{$\overrightarrow{u}=(2;-3;2)$}
	\loigiai{
	Ta có $d\colon \dfrac{x-1}{2}=\dfrac{3-y}{3}=\dfrac{z+1}{-2}\Leftrightarrow \dfrac{x-1}{2}=\dfrac{y-3}{-3}=\dfrac{z+1}{-2}$.\\
	Do đó $d$ có một véc-tơ chỉ phương là $\overrightarrow{u}=(2;-3;-2)$.
	}
	\end{ex}
	\begin{ex}%[TT lần 2, cụm các trường THPT chuyên Bắc Bộ - 2018]%[Đinh Bích Hảo, dự án(12EX-10)]%[2H3B3-1]
	Trong KG $Oxyz$, cho mặt phẳng $(P)\colon (m^2+1)x-(2m^2-2m+1)y+(4m+2)z-m^2+2m=0$ luôn chứa một đường thẳng $\Delta $ cố định khi $m$ thay đổi. Đường thẳng $d$ đi qua $M(1;-1;1)$ vuông góc $(\Delta)$ và cách $O$ một khoảng lớn nhất có véc-tơ chỉ phương $\overrightarrow{u}=(-1;b;c)$. Tính $b^2-c$?
	\choice
	{$2$}
	{$23$}
	{\True $19$}
	{$-1$}
	\loigiai{Cho $m=0$ có mặt phẳng $(P_{0})\colon x-y+2z=0$, suy ra $\overrightarrow{n}=(1;-1;2)$.\\
	Cho $m=1$ có mặt phẳng $(P_{1})\colon 2x-y+6z+1=0$ suy ra $\overrightarrow{n'}=(2;-1;6)$.\\
	Suy ra $\Delta $ có véc-tơ chỉ phương $\overrightarrow{u_1}=\left[\overrightarrow{n};\overrightarrow{n'}\right]=(-4;-2;1)$.\\
	Gọi $H$ là hình chiếu của $O$ trên $d$ thì $OH$ là khoảng cách từ $O$ đến $d$.\\
	Ta có $OH\leq OM$.\\
	Do đó yêu cầu bài toán tương đương với $d \perp OM$.\\
	Vậy $d$ có một véc-tơ chỉ phương $\overrightarrow{u}=\left[\overrightarrow{u_1};\overrightarrow{OM}\right]=(-1;5;6) \Rightarrow b^2-c=25-6=19$.
	}
	\end{ex}
	\begin{ex}%[HK2 (2017-2018), THPT Tân Hiệp, Kiên Giang]%[Bùi Mạnh Tiến, dự án (12EX-9)]%[2H3B3-1]
	Trong KG $Oxyz$, cho $2$ đường thẳng $\Delta_1\colon \heva{&x=3+t\\&y=1+t\\&z=1+2t}(t\in \mathbb{R})$; $\Delta_2\colon \dfrac{x+2}{2}=\dfrac{y-2}{5}=\dfrac{z}{-1}$ và điểm $M(0;3;0)$. Đường thẳng $d$ đi qua $M$, cắt $\Delta_1$ và vuông góc với $\Delta_2$ có một véc-tơ chỉ phương là $\vec{u}=\left(4;a;b\right)$. Tính $T=a+b$
	\choice
	{$T=-2$}
	{$T=4$}
	{$T=-4$}
	{\True $T=2$}
	\loigiai{
	\immini{
	Gọi $(P)$ là mặt phẳng chứa $M$ và $\Delta_1$.\\
	Lấy $A(3;1;1)\in \Delta_1$.\\
	Mặt phẳng $(P)$ có véc-tơ pháp tuyến vuông góc với các véc-tơ $\overrightarrow{MA}=(3;-2;1)$ và $\overrightarrow{u}_{\Delta_1}=(1;1;2)$.\\
	Ta có $ \left[\overrightarrow{MA},\overrightarrow{u}_{\Delta_1}\right]=(-5;-5;5)$.
	}
	{
	\begin{tikzpicture}
	\clip (-0.2,-0.2) rectangle (7.2,3.2);
	\tkzDefPoints{2/2/A, 0/0/B, 5/0/C, 7/2/D, 2/1.6/E, 5/0.4/F, 2/0.4/I, 4.6/1.4/J, 5.6/3/G}
	\tkzDefPointBy[projection=onto E--F](G) \tkzGetPoint{H}
	\tkzInterLL(G,H)(C,B)\tkzGetPoint{K}
	\tkzInterLL(I,J)(E,F)\tkzGetPoint{M}
	\tkzDrawSegments(A,B B,C C,D D,A E,F I,J G,H)
	\tkzDrawSegments[dashed](H,K)
	\tkzMarkAngle(C,B,A)
	\tkzDrawPoints(M,E);
	\tkzLabelSegment[below](I,M){$\Delta_1$}
	\tkzLabelSegment[right](G,H){$\Delta_2$}
	\tkzLabelSegment[above](E,M){$d$}
	\draw ($(B)+(0.3,0)$) node[above right] {$P$} (E) node[below] {$M$};
	\end{tikzpicture}	
	}
	\flushleft
	Một trong các véc-tơ pháp tuyến của mặt phẳng $(P)$ là $\overrightarrow{n}_{(P)}=(1;1;-1)$.\\
	Đường thẳng $d$ nằm trong mặt phẳng $(P)$ và vuông góc với $\Delta_2$ có $\overrightarrow{u}_d=\left[\overrightarrow{n}_{(P)},\overrightarrow{u}_{\Delta_2}\right]=(4;-1;3)$.
	Vậy $a=-1;b=3\Rightarrow T=a+b=2$.
	}
	\end{ex}
	\begin{dang}{Viết PTĐT}%dang 2
		\begin{itemize}
			\item Tìm  một điểm $M(x_0; y_0; z_0)$ thuộc đường thẳng $d$.
			\item Tìm một vec-tơ chỉ phương của $d$ là $\overrightarrow{u}=(a;b;c)$. (Chú ý  xem cách tìm ở dạng 1).
			\item PTTS của $d$  là $  \begin{cases}
			x=x_0+at\\
			y=y_0+bt\\
			z=z_0+ct
			\end{cases}$ trong đó $ t $ là tham số.
		\end{itemize}
		\begin{note}
	Phương trình chính tắc của đường thẳng $d$ qua $M(x_0;y_0;z_0)$ và có véc-tơ chỉ phương $\vec{u}=(a;b;c)$ là $d: \dfrac{x-x_0}{a}=\dfrac{y-y_0}{b}=\dfrac{z-z_0}{c}$ với $abc\ne 0$.
	\end{note}
	\end{dang}
\setcounter{subsubsection}{0}
\setcounter{vd}{0}
\setcounter{ex}{0}
	\subsubsection{Ví dụ minh hoạ}
	\begin{vd}%[NguyenvanSang]%[2H3Y3]
	Trong không gian với hệ trục $Oxyz$, viết PTTS của đường thẳng $d$ đi qua điểm $M(1,2,3)$ và có véc-tơ chỉ phương $\vec{a}=(1;3;2)$.
	\loigiai{
	PTTS của đường thẳng $d$ đi qua điểm $M(1;2;3)$ và có véc-tơ chỉ phương $\vec{a}=(1;3;2)$ là $\left\{\begin{aligned}& x=1+t \\
	& y=2+3t \\
	& z=3+2t. 
	\end{aligned}\right.$
	}
	\end{vd}
	\begin{vd}%[NguyenvanSang]%[2H3Y3]
	Trong KG $Oxyz$, viết phương trình chính tắc của đường thẳng đi qua hai điểm $A(1;2;-3)$ và $B(3;-1;1)$.
	\loigiai{
	Ta có $\overrightarrow{AB}=(2;-3;4)$ nên phương trình chính tắc là $\dfrac{x - 1}{2} = \dfrac{y - 2}{ - 3} = \dfrac{z + 3}{4}.$
	}
	\end{vd}
	\begin{vd}%[NguyenvanSang]%[2H3Y3]
	Trong KG $Oxyz$, cho $3$ điểm $A(1;2;3)$, $B(2;3;4)$ và $C(0;0;1)$. Viết phương trình chính tắc của đường thẳng qua điểm $C$ và nhận $\overrightarrow{AB}$ làm véc-tơ chỉ phương.
	\loigiai{
	Ta có $\overrightarrow{AB}=(1;1;1)	$. Suy ra phương trình chính tắc là: $\dfrac{x}{1}=\dfrac{y}{1}=\dfrac{z-1}{1}$.
	}
	\end{vd}
	\begin{vd}%[NguyenvanSang]%[2H3Y3]
	Trong KG $Oxyz$, viết phương trình chính tắc đường thẳng đi qua điểm $A(2;3;0)$ và vuông góc với mặt phẳng $(P): x + 3y - z + 5 = 0.$
	\loigiai{Ta có đường thẳng cần tìm vuông góc với mặt phẳng $(P)$ nên có véc-tơ chỉ phương là véc-tơ pháp tuyến của mặt phẳng $(P)$ là $\vec{u}=\vec{n}_{(P)}=(1;3;-1)$. Suy ra phương trình chính tắc là $$ \dfrac{x-2}{1}=\dfrac{y-3}{3}=\dfrac{z}{-1}.$$
	}
	\end{vd}
	\begin{vd}%[Hồ Như Vương]%[2H3Y3]
	Trong KG $Oxyz$, viết PTTS đường thẳng đi qua $A\left(3;5;7\right)$ và song song với $d:\dfrac{x-1}{2}=\dfrac{y-2}{3}=\dfrac{z-3}{4}$.
	\loigiai{
	Gọi $\Delta $ là đường thẳng thỏa yêu cầu bài toán.
	Ta có $\Delta $ có véc-tơ chỉ phương là $\vec{u}=\left(2;3;4\right)$ và qua $A\left(3;5;7\right)$$\Rightarrow \left(\Delta \right):\left\{\begin{aligned}& x=3+2t \\
	& y=5+3t \\
	& z=7+4t. 
	\end{aligned}\right.$
	}
	\end{vd}
	\begin{vd}%[Mai Hà Lan]%[2H3B3]
	Viết PTĐT $d$ đi qua điểm $A(1;-1;1)$ và song song với hai mặt phẳng $(P): x + y -3z -1 = 0$ và $(Q): -2x + y -4z + 1 = 0$. 
	\loigiai{
	Mặt phẳng $(P)$ , $(Q)$ lần lượt có véc tơ pháp tuyến là $\vec{n_1} = (1;1;-3)$ và $\vec{n_2} = (-2;1;-4)$. Vì $d$ song song với $(P)$ và $(Q)$ nên véc tơ chỉ phương của $d$ là $\vec{u} = [\vec{n_1}, \vec{n_2}] = (-1;10;3)$.
	Đường thẳng $d$ đi qua điểm $A (1;-1;1)$ và có một véc tơ chỉ phương là $ \vec{u} = (-1;10;3)$, nên $d$ có PTTS là
	$$ \heva{& x = 1 - t\\ & y = -1 + 10t\\ &z = 1 + 3t.} $$
	}
	\end{vd}
	\begin{vd}%[Mai Hà Lan]%[2H3B3]
	Cho điểm $A(2;-5;-1)$ và mặt phẳng $(P): x-y-z+9=0$, đường thẳng $d: \dfrac{x-1}{2} = \dfrac{y-1}{1} = \dfrac{z+2}{3}$. Lập phương trình của đường thẳng $\Delta$ qua $A$, song song với $(P)$ và vuông góc với $d$.
	\loigiai{
	Ta có $(P)$ có một véc tơ pháp tuyến là $\vec{n} = (1;-1;-1)$, đường thẳng $d$ có một véc tơ chỉ phương là $\vec{u} = (2;1;3)$, nên đường thẳng $\Delta$ có véc tơ chỉ phương là $\left[ \vec{u}, \vec{n} \right] = (-2; -5; 3)$. Suy ra $\Delta$ có phương trình $\dfrac{x-2}{-2} = \dfrac{y+5}{-5} = \dfrac{z +1}{3}$.
	}
	\end{vd}
	\begin{vd}%[Nguyễn Phúc Đức]%[2H3B3]
	Trong không gian cho đường thẳng $(d_1):\heva{&x=t\\&y=1-4t\\&z=2+6t}$ và $(d_2):\heva{&x=2t\\&y=1+t\\&z=2-5t}$.\\
	Viết PTĐT $(d)$ đi qua $M(1;-1;2)$ và vuông góc với cả hai đường thẳng $(d_1)$ và $(d_2)$.
	\loigiai{
	Véc tơ chỉ phương của $(d_1)$ và $(d_2)$ lần lượt là : $\vec{u_1}=(1;-4;6)$ và $\vec{u_2}=(2;1;-5)$.\\
	Gọi $\vec{u}$ là một véc tơ chỉ phương của $(d)$, ta có : \\
	$\heva{&\vec{u}\perp \vec{u_1}\\&\vec{u}\perp \vec{u_2}}\Rightarrow \vec{u}=\left[\vec{u_1};\vec{u_2}\right]=(14;17;9)$.\\
	Khi đó, đường thẳng $(d)$ thỏa mãn:\\
	$(d):\heva{&\mbox{qua }M(1;-1;2)\\&\mbox{có VTCP }\vec{u}=(14;17;9)}\Leftrightarrow (d):\heva{&x=1+14t\\&y=-1+17t\\&z=2+9t}$.
	}
	\end{vd}
	\begin{vd}%[Lê Quốc Hiệp]%[2H3B3]
	Trong không gian với hệ trục toạ độ $Oxyz$, cho điểm $A(1;2;-6)$, đường thẳng $d_1:\dfrac{x}{1}=\dfrac{y-6}{4}=\dfrac{z}{2}$ và đường thẳng $d_2:\heva{&x=1-t\\&y=2+t\\&z=1+4t}$. Viết PTĐT $d$ đi qua điểm $A$ đồng thời cắt cả hai đường thẳng $d_1$ và $d_2$.
	\loigiai{
	Đường thẳng $d_1$ có véc-tơ chỉ phương $\vec{u}_1=(1;4;2)$ và đi qua $M(0;6;0)$.\\
	Gọi $(\alpha)$ là mặt phẳng đi qua $A$ và chứa đường thẳng $d_1$.\\
	véc-tơ pháp tuyến của $(\alpha)$ là $\vec{n}_{\alpha}=\left[\overrightarrow{MA},\vec{u}_1\right]=(16;-8;8)$, chọn $\vec{n}_{\alpha}=(2;-1;1)$.\\
	Suy ra, $(\alpha):2x-y+z+6=0$.\\
	Gọi $B=(\alpha) \cap d_2$. Xét phương trình $2(1-t)-(2+t)+(1+4t)+6=0 \Leftrightarrow t=-7$.\\
	Suy ra $B(8;-5;-27)$.\\
	Đường thẳng $d$ có véc-tơ chỉ phương là $\vec{u}_d=\overrightarrow{AB}=(7;-7;-21)$, chọn $\vec{u}_d=(1;-1;-3)$.\\
	Vậy phương trình của đường thẳng $d:\dfrac{x-1}{1}=\dfrac{y-2}{-1}=\dfrac{z+6}{-3}$.
	}
	\end{vd}
	\begin{vd}%[Lê Quốc Hiệp]%[2H3B3]
	Trong không gian với hệ trục toạ độ $Oxyz$, cho điểm $A(1;2;0)$, đường thẳng $d_1:\dfrac{x+2}{3}=\dfrac{y-3}{-1}=\dfrac{z-1}{-1}$ và đường thẳng $d_2:\heva{&x=2+t\\&y=1+2t\\&z=3+2t}$. Viết PTĐT $d$ đi qua điểm $A$, vuông góc với đường thẳng $d_1$ và cắt đường thẳng $d_2$.
	\loigiai{
	Đường thẳng $d_1$ có véc-tơ chỉ phương $\vec{u}_1=(3;-1;-1)$.\\
	Gọi $(\alpha)$ là mặt phẳng đi qua $A$ và vuông góc đường thẳng $d_1$.\\
	Véc-tơ pháp tuyến của $(\alpha)$ là $\vec{n}_{\alpha}=\vec{u}_1=(3;-1;-1)$.\\
	Suy ra, $(\alpha):3x-y-z-1=0$.\\
	Gọi $B=(\alpha) \cap d_2$. Xét phương trình $3(2+t)-(1+2t)-(3+2t)-1=0 \Leftrightarrow t=1$.\\
	Suy ra $B(3;3;5)$.\\
	Đường thẳng $d$ có véc-tơ chỉ phương là $\vec{u}_d=\overrightarrow{AB}=(2;1;5)$.\\
	Vậy phương trình của đường thẳng $d:\dfrac{x-1}{2}=\dfrac{y-2}{1}=\dfrac{z}{5}$.
	}
	\end{vd}
	\begin{vd}%[Lê Đình Mẫn]%[2H3B3]
	Trong KG $Oxyz$, cho điểm $A(1;2;-2)$ và đường thẳng $d_1:\left\{ 
	\begin{aligned}&x=2t\\&y=1+t\\&z=-t\end{aligned}\right.$. Viết PTĐT $d$ đi qua điểm $A$, vuông góc và cắt đường thẳng $d_1$.
	\loigiai{Gọi $M(2t;1+t;-t)$ là giao điểm của $d$ và $d_1$. Vì $d\perp d_1$ nên\\ $\overrightarrow{AM}\cdot \overrightarrow{u}_{d_1}=0\Leftrightarrow (2t-1)\cdot 2+(1+t-2)\cdot 1+(-t+2)\cdot (-1)=0\Leftrightarrow 6t-5=0\Leftrightarrow t=\dfrac{5}{6}$.\\
	Suy ra $\overrightarrow{AM}=\left(\dfrac{2}{3};-\dfrac{1}{6};\dfrac{7}{6}\right)$. Từ đó ta có PTĐT $d$ là $\left\{\begin{aligned}
	&x=1+4t\\&y=2-t\\&z=-2+7t
	\end{aligned}\right.$.}
	\end{vd}
	\begin{vd}%[Lê Đình Mẫn]%[2H3B3]
	Trong KG $Oxyz$, cho $(P):y+2z=0$, $d_1:\dfrac{x-1}{-1}=\dfrac{y}{1}=\dfrac{z}{4}$ và\newline $d_2:\left\{\begin{aligned}&x=2-t\\&y=4+2t\\&z=1\end{aligned}\right.$. Viết PTĐT $d$ nằm trong mặt phẳng $(P)$ đồng thời cắt cả hai đường thẳng $d_1$ và $d_2$.
	\loigiai{Ta có $\overrightarrow{n}_{(P)}=(0;1;2)$, $\overrightarrow{u}_{d_1}=(-1;1;4)$, $\overrightarrow{u}_{d_2}=(-1;2;0)$. Kiểm tra $\overrightarrow{n}_{(P)}\cdot\overrightarrow{u}_{d_1}\ne 0$, $\overrightarrow{n}_{(P)}\cdot\overrightarrow{u}_{d_2}\ne 0$ nên $(P)\cap d_1\equiv A(1;0;0)$ và $(P)\cap d_2\equiv B(5;-2;1)$. Từ đó ta có $\overrightarrow{AB}=(4;-2;1)$.\\
	Vậy, đường thẳng $d$ có phương trình $\dfrac{x-1}{4}=\dfrac{y}{-2}=\dfrac{z}{1}$.}
	\end{vd}
	\begin{vd}%[Lê Đình Mẫn]%[2H3B3]
	Trong KG $Oxyz$, cho các đường thẳng $d':\dfrac{x}{2}=\dfrac{y-1}{-1}=\dfrac{z-1}{2}$,\newline $d_1:\dfrac{x+1}{1}=\dfrac{y-1}{-1}=\dfrac{z-1}{2}$ và $d_2:\left\{\begin{aligned}&x=2+3t\\&y=-1+2t\\&z=-3+t\end{aligned}\right.$. Viết PTĐT $d$ song song với đường thẳng $d'$ đồng thời cắt cả hai đường thẳng $d_1$ và $d_2$.
	\loigiai{Gọi $M(-1+t_1;1-t_1;1+2t_1)\in d\cap d_1$, $N(2+3t_2;-1+2t_2;-3+t_2)\in d\cap d_2$.\\
	Ta có $\overrightarrow{MN}=(3t_2-t_1+3;2t_2+t_1-2;t_2-2t_1-4)$. Vì $d\parallel d'$ nên $\overrightarrow{MN}$ cùng phương với $\overrightarrow{u}_{d'}$.\\
	Từ đó ta tìm được $t_1=-\dfrac{51}{5}, t_2=\dfrac{8}{5}$ và tính được $M\left(-\dfrac{56}{5};\dfrac{56}{5};-\dfrac{97}{5}\right), \overrightarrow{MN}=(18;-9;18)$.\\
	Vậy $d:\dfrac{x+\dfrac{56}{5}}{2}=\dfrac{y-\dfrac{56}{5}}{-1}=\dfrac{z+\dfrac{97}{5}}{2}$.}
	\end{vd}
	\begin{vd}%[Đinh Mạnh Hùng]%[2H3K3]
	Viết phương trình đường vuông góc chung của hai đường thẳng sau $d_1: \dfrac{x}{1}=\dfrac{y}{1}=\dfrac{z}{1}; d_2:\dfrac{x-1}{1}=\dfrac{y-1}{2}=\dfrac{z-1}{3}$.
	\loigiai{Gọi đường vuông góc chung là $d$, ta có ngay $\overrightarrow{u_d}=[\overrightarrow{u_{d_1}},\overrightarrow{u_{d_2}}]=(1;-2;1)$.\\Gọi hai giao điểm của $d$ với $d_1; d_2$ lần lượt là $M_1; M_2$, ta có $M_{1}(t;t;t),M_{2}(1+t';1+2t';1+3t')$. Do $\overrightarrow{u_d}\parallel \overrightarrow{M_1M_2}$, ta tìm được 
	$\begin{cases}
	t=1\\t'=0
	\end{cases}$.\\ 
	Vậy $M_1(1;1;1)$, phương trình $d: \dfrac{x-1}{1}=\dfrac{y-1}{-2}=\dfrac{z-1}{1}$. 
	}
	\end{vd}
	\begin{vd}%[Đinh Mạnh Hùng]%[2H3K3]
	Viết phương trình hình chiếu vuông góc $d'$ của đường thẳng $d:\dfrac{x-1}{2}=\dfrac{y-2}{1}=\dfrac{z-1}{2}$ lên mặt phẳng $(P): x+y+z+1=0$ .
	\loigiai{Giao điểm của $(P)$ và $d$ là $M(x;y;z)$. Ta tìm được $M(-1;1;-1)$, cần tìm thêm hình chiếu vuông góc của một điểm khác trên $d$ xuống $(P)$.\\
	Ta có $A(1;2;1)$ thuộc $d$, đường thẳng qua $A$ và vuông góc với $(P)$ là $\heva{x=1+t\\y=2+t\\z=1+t}$, từ đây ta xác định toạ độ hình chiếu của $A$ lên $(P)$ là $A'\left(-\dfrac{2}{3};\dfrac{1}{3};-\dfrac{2}{3}\right)$.\\
	Hình chiếu vuông góc $d'$ của đường thẳng $d$ trên mặt phẳng $(P)$ là đường thẳng đi qua các điểm $M, A'$.\\
	Ta có $\overrightarrow{MA'}=\left(\dfrac{1}{3};-\dfrac{2}{3};\dfrac{1}{3}\right)$, do đó $MA'$ là đường thẳng đi qua điểm $A(-1;1;-1)$ và có véc-tơ chỉ phương $\overrightarrow{u}=(1;-2;1)$.\\
	$d'$ có phương trình: $\heva{&x=-1+t\\&y=1-2t\\&z=-1+t}$
	}
	\end{vd}
\begin{vd}%[Phan Anh Tiến- dự án 12-EX-1-DCHT][2H3K3-2]
	Viết PTĐT $(d)$ là hình chiếu vuông góc của $(a)$ lên mặt phẳng $(P)$, với $(a)\colon\dfrac{x+1}{1}=\dfrac{y-1}{1}=\dfrac{z-3}{1}$ và $(P)\colon x-y+z-3=0$.
	\dapso{$(d)\colon\heva{&x=t\\&y=2t+1\\&z=t+4.}$
	}
	\loigiai{Gọi $(\alpha)$ là mặt phẳng chứa $(a)$ và vuông góc với $(P)$ khi đó véc-tơ pháp tuyến của $(\alpha)$ là $\overrightarrow{n}_{\alpha}=\left[ \overrightarrow{n}_{P},\overrightarrow{u}_{a}\right] =(-2;0;2)$ với $\overrightarrow{n}_{P}=(1;-1;1)$, $\overrightarrow{u}_{a}=(1;1;1)$. Suy ra phương trình $(\alpha)\colon -x+z-4=0$.\\
		Đường thẳng $(d)$ là giao tuyến của $(\alpha)$ và $(P)$, tọa độ các điểm chung của hai mặt phẳng là nghiệm của hệ $$\heva{&-x+z-4=0=0\\&x-y+z-3=0}\Leftrightarrow \heva{&x=t\\&y=2t+1\\&z=t+4.}$$\\
		Vậy phương trình $(d)\colon\heva{&x=t\\&y=2t+1\\&z=t+4.}$
	}
\end{vd}
\subsubsection{Bài tập trắc nghiệm}
	\begin{ex}%[Đề KSCL học kỳ 2 Toán 12 năm học 2017 – 2018 sở GD và ĐT Nam Định] %[Trần Tuấn Việt, 12EX-9-2018]%[2H3B3-2]
	Trong KG $Oxyz$, cho đường thẳng $\Delta$ đi qua điểm $M(2; 0; -1)$ và véc-tơ chỉ phương $\overrightarrow{a} = (4; -6; 2)$. PTTS của $\Delta$ là
	\choice
	{$\heva{&x = -2 + 4t\\&y=-6t\\&z=1+2t}$}
	{$\heva{&x = -2+2t\\&y=-3t\\&z=1+t}$}
	{$\heva{&x = 4+2t\\&y=-6-3t\\&z=2+t}$}
	{\True $\heva{&x=2+2t\\&y=-3t\\&z=-1+t}$}
	\loigiai{$\bullet$ $\Delta\colon \heva{&x = 2 + 4t\\&y = - 6t\\&z = -1 + 2t}$. Đặt $2t = t'$ ta có $\Delta\colon \heva{&x=2+2t'\\&y=-3t'\\&z=-1+t'}$.
	}
	\end{ex}
\begin{ex}%[TT, THPT Đông Thụy Anh, Thái Bình, 2017-2018]%[Nguyễn Tiến Thùy, 12EX7]%[2H3B3-2]
	Trong KG $Oxyz$, đường thẳng đi qua điểm $M(1;2;3)$ và song song với trục $Oy$ có PTTS là
	\choice
	{$d: \heva{
	& x=1+t \\ 
	& y=2 \\ 
	& z=3
	},\, t\in\mathbb{R}$} 
	{\True $d: \heva{
	& x=1 \\ 
	& y=2+2t \\ 
	& z=3
	},\, t\in\mathbb{R}$} 
	{$d: \heva{
	& x=1 \\ 
	& y=2 \\ 
	& z=3+t
	},\, t\in\mathbb{R}$} 
	{$d: \heva{
	& x=1-t \\ 
	& y=2+t \\ 
	& z=3-t
	} ,\, t\in\mathbb{R}$}
	\loigiai{
	Đường thẳng qua $M(1;2;3)$ song song với trục $Oy$ có véc-tơ chỉ phương là $\overrightarrow{j}=(0;1;0)$ nên nó có PTTS
	$$d: \heva{
	& x=1 \\ 
	& y=2+t \\ 
	& z=3
	},\, t\in\mathbb{R}.$$
	}
\end{ex}
\begin{ex}%[Thi HK2, Sở GD\&ĐT Đồng Tháp, 2018]%[2H3B3-2]%[Trần Hòa, dự án (12EX-8)]
	Trong KG $Oxyz$, cho đường thẳng $(d)\colon \heva{x&=1-t\\y&=-1+2t\\z&=2-t}\,(t\in \mathbb{R})$. Đường thẳng đi qua điểm $M(0;1;-1)$ và song song với đường thẳng $(d)$ có phương trình là
	\choice
	{\True $\dfrac{x}{1}=\dfrac{y-1}{-2}=\dfrac{z+1}{1}$}
	{$\dfrac{x+1}{1}=\dfrac{y-2}{-1}=\dfrac{z+1}{2}$}
	{$\dfrac{x}{-1}=\dfrac{y+1}{2}=\dfrac{z-1}{-1}$}
	{$\dfrac{x-1}{1}=\dfrac{y+2}{-1}=\dfrac{z-1}{2}$}
	\loigiai{
	Rõ ràng $M\notin (d)$.\\
	Đường thẳng $(d)$ có một véc-tơ chỉ phương là $\vec{u}=(-1;2;-1)$.\\
	Đường thẳng đi qua $M(0;1;-1)$ và song song với đường thẳng $(d)$ có phương trình là $\dfrac{x}{1}=\dfrac{y-1}{-2}=\dfrac{z+1}{1}$.
	}
\end{ex}
\begin{ex}%[Thi HK2, THPT Lý Thái Tổ, Hà Nội , 2018] %[2H3B3-2]%[Trần Tuấn Việt, dự án(12EX-8)
	Trong KG $Oxyz$ cho hai mặt phẳng $(P): 2x + y - z - 3 = 0$ và $(Q): x + y + z - 1 = 0$. Phương trình chính tắc đường thẳng giao tuyến của hai mặt phẳng $(P)$ và $(Q)$ là 
	\choice
	{$\dfrac{x+1}{-2} = \dfrac{y - 2}{-3} = \dfrac{z - 1}{1}$}
	{\True $\dfrac{x}{2} = \dfrac{y - 2}{-3} = \dfrac{z + 1}{1}$}
	{$\dfrac{x - 1}{2} = \dfrac{y + 2}{3} = \dfrac{z + 1}{1}$}
	{$\dfrac{x}{2} = \dfrac{y + 2}{-3} = \dfrac{z - 1}{-1}$}
	\loigiai{ Xét hệ phương trình $\heva{& 2x + y -z - 3 = 0\\& x + y + z - 1 = 0} \Leftrightarrow \heva{x - 2z - 2 &= 0\\x + y + z - 1 & = 0} \Leftrightarrow \heva{x & = 2z + 2\\ y &= -3z - 1}.$
	Đặt $z = t$ ta suy ra $x = 2t + 2, y = -3t - 1$. Từ đó ta thu được PTĐT $d:$ 
	$\dfrac{x - 2}{2} = \dfrac{y + 1}{-3} = \dfrac{z}{1} $. Xét điểm $A(2; -1; 0) \in d,$ ta thấy $A$ chỉ thuộc đường thẳng $\dfrac{x}{2} = \dfrac{y - 2}{-3} = \dfrac{z + 1}{1}$.
	}
\end{ex}
\begin{ex}%[HK2, THTH ĐHSP Tp. HCM, 2018]%[2H3B3-2]%[Vinhhop Tran, 12EX-8]
	Trong không gian tọa độ $Oxyz,$ viết phương trình chính tắc của đường thẳng $d$ đi qua $A(1; 2; -1)$ và vuông góc với mặt phẳng $(P)\colon x+2y-3z+1=0.$
	\choice
	{$d\colon \dfrac{x+1}{1}=\dfrac{y+2}{-2}=\dfrac{z-1}{-3}$}
	{$d\colon \dfrac{x+1}{1}=\dfrac{y+2}{2}=\dfrac{z-1}{-3}$}
	{$d\colon \dfrac{x-1}{1}=\dfrac{y-2}{2}=\dfrac{z+1}{3}$}
	{\True $d\colon \dfrac{x-1}{-1}=\dfrac{y-2}{-2}=\dfrac{z+1}{3}$}
	\loigiai{$d$ đi qua $A(1; 2; -1)$ và nhận véc-tơ pháp tuyến của $(P)$ làm véc-tơ chỉ phương nên có phương trình là $\dfrac{x-1}{-1}=\dfrac{y-2}{-2}=\dfrac{z+1}{3}.$
	}
\end{ex}
\begin{ex}%[TT Cụm 5 Trường Chuyên ĐB Sông Hồng, 2018]%[2H3B3-2]%[Lê Minh An, 12Ex-8]
	Trong không gian với hệ tọa độ $Oxy$, cho điểm $A(1;2;3)$ và mặt phẳng $(P)\colon 2x+y-4z+1=0$. Đường thẳng $(d)$ qua điểm $A$, song song với mặt phẳng $(P)$, đồng thời cắt trục $Oz$. Viết PTTS của đường thẳng $(d)$.
	\choice{$\heva{& x=1+5t\\ & y=2-6t\\ & z=3+t}$}
	{\True $\heva{& x=t\\ & y=2t\\ & z=2+t}$}
	{$\heva{& x=1+3t\\ & y=2+2t\\ & z=3+t}$}
	{$\heva{& x=1-t\\ & y=2+6t\\ & z=3+t}$}
	\loigiai{
	Gọi $B=d\cap Oz\Rightarrow B(0;0;b)\Rightarrow \overrightarrow{AB}=(-1;-2;b-3)$.\\
	Lại có $d\parallel (P)$ nên $\overrightarrow{AB}\perp \overrightarrow{n}_{(P)}=(2;1;-4)$. Do đó $$\overrightarrow{AB}\cdot\overrightarrow{n}_{(P)}=0\Leftrightarrow -2-2-4b+12=0\Leftrightarrow b=2.$$
	Suy ra $\overrightarrow{AB}=(-1;-2-1)$. Do đó, $(d)$ là đường thẳng qua $B(0;0;2)$ và nhận $\overrightarrow{u}=(1;2;1)$ làm véc-tơ chỉ phương. Nên $(d)$ có phương trình $\heva{& x=t\\ & y=2t\\ & z=2+t}$.
	}
\end{ex}
\begin{ex}%[2H3B3-2]%[Đề thi thử THPT Quốc gia 2018 môn Toán trường THPT Kim Liên – Hà Nội lần 2]%[Võ Thanh Phong, dự án 12EX-8]
	Trong KG $Oxyz$, cho điểm $M\left(-1;1;2\right)$ và hai đường thẳng $d\colon \dfrac{x-2}{3} = \dfrac{y+3}{2} = \dfrac{z-1}{1}$, $d'\colon \dfrac{x+1}{1} = \dfrac{y}{3} = \dfrac{z}{-2}$. Phương trình nào dưới đây là PTĐT đi qua điểm $M$, cắt $d$ và vuông góc với $d'$.
	\choice
	{$\heva{
	&x =-1+3t\\
	&y = 1+t\\
	&z = 2
	}$}
	{\True$\heva{
	&x =-1+3t\\
	&y = 1-t\\
	&z = 2
	}$}
	{$ \heva{
	&x = 1+3t\\
	&y = 1-t\\
	&z = 2
	}$}
	{$ \heva{
	&x =-1-7t\\
	&y = 1+7t\\
	&z = 2+7t
	}$}
	\loigiai{
	Gọi $\Delta $ là đường thẳng đi qua điểm $M$, cắt $d$ và vuông góc với $d'$.\\
	Giả sử $\Delta \cap d = A \Rightarrow A\left(2+3t;-3+2t;1+t\right)$.\\
	$\overrightarrow {AM} = \left(3+3t;-4+2t;-1+t\right)$.\\
	$\Delta \perp d'\Rightarrow \overrightarrow {AM}\cdot \overrightarrow {u_{d'}} = 0 \Leftrightarrow 3+3t+3\left(-4+2t\right)-2\left(-1+t\right) = 0 \Leftrightarrow 7t = 7 \Leftrightarrow t = 1$.\\
	$ \Rightarrow A\left(5;-1;2\right),\overrightarrow {AM} = \left(6;-2;0\right) = 2\left(3;-1;0\right)$.\\
	Vậy PTĐT $\Delta\colon \heva{
	&x =-1+3t\\
	&y = 1-t\\
	&z = 2}$.
	}
\end{ex}
\begin{ex}%	[Thi thử L2, Chuyên DHSP Ha Noi, 2018 ]%[2H3B3-2]%[MyNguyen, Dự án (12EX-8)]
	Trong KG $Oxyz$, cho điểm $A(1;2;3)$ và hai mặt phẳng $(P)\colon 2x+3y=0, (Q)\colon 3x+4y=0$. Đường thẳng đi qua $A$ và song song với hai mặt phẳng $(P), (Q)$ có phương trình là	
	\choice
	{$\heva{& x=t\\ & y=2\\& z=3+t}$}
	{$\heva{& x=1\\ & y=t\\& z=3}$}
	{$\heva{& x=1+t\\ & y=2+t\\& z=3+t}$}
	{\True $\heva{& x=1\\ & y=2\\& z=t}$}
	\loigiai{Gọi $\Delta $ là đường thẳng cần tìm. Mặt phẳng $(P) $ có một véc-tơ pháp tuyến là $\vec{n}_1 = (2;3;0)$ và $(Q)$ có một véc-tơ pháp tuyến là $\vec{n}_2= (3;4;0)$. Ta có $\left[\vec{n}_1, \vec{n}_2\right] =(0;0;2)$. Khi đó, $\Delta $ đi qua điểm $A$ và nhận véc-tơ $\vec{u} =(0;0;1)$ làm véc-tơ chỉ phương. PTĐT $\Delta$ là $\heva{& x=1\\ & y=2\\& z=3+t}, t\in \mathbb{R}$.\\
	Với $t=-3$ thì điểm $B(1;2;0)$ thuộc $\Delta$. Viết lại PTĐT $\Delta \colon \heva{& x=1\\ & y=2\\& z=t}$.}
\end{ex}
\begin{ex}%[TT, Chuyên Lê Quý Đôn, Lai Châu, 2018]%[2H3B3-2]%[Nguyễn Tiến Thùy, 12EX-8]
	Trong KG $Oxyz$, gọi $\Delta$ là đường thẳng đi qua điểm $M(2;0;-3)$ và vuông góc với mặt phẳng $(\alpha): 2x-3y+5z+4$. Viết phương trình chính tắc của đường thẳng $\Delta$.
	\choice
	{$\Delta: \dfrac{x+2}{1}=\dfrac{y}{-3}=\dfrac{z-3}{5}$}
	{$\Delta: \dfrac{x+2}{2}=\dfrac{y}{-3}=\dfrac{z-3}{5}$}
	{$\Delta: \dfrac{x-2}{2}=\dfrac{y}{3}=\dfrac{z+3}{5}$}
	{\True $\Delta: \dfrac{x-2}{2}=\dfrac{y}{-3}=\dfrac{z+3}{5}$}
	\loigiai{
	Mặt phẳng $(\alpha)$ có véc-tơ pháp tuyến $\overrightarrow{n}(2;-3;5)$.\\
	Đường thẳng $\Delta$ qua $M(2;0;-3)$ và vuông góc với mặt phẳng $(\alpha)$ nên $\Delta$ có véc-tơ chỉ phương $\overrightarrow{u}_\Delta=\overrightarrow{n}(2;-3;5)$.\\
	Từ đó ta có phương trình chính tắc của đường thẳng $\Delta$
	\begin{center}
	$\dfrac{x-2}{2}=\dfrac{y}{-3}=\dfrac{z+3}{5}.$
	\end{center}
	}
\end{ex}
\begin{ex}%[2H3B3-2]%[2-TT-43-SoBacGiang]%[DoVuMinhThang]
	Trong không gian với hệ trục tọa độ $Oxyz$, cho điểm $M(2;1;0)$ và đường thẳng $\Delta$: $\dfrac{x-1}{2}=\dfrac{y+1}{1}=\dfrac{z}{-1}$. PTTS của đường thẳng $d$ đi qua điểm $M$, cắt và vuông góc với $\Delta$ là
	\choice{\True $d: \heva{&x=2+t \\ &y=1-4t \\ &z=-2t}$}
	{$d: \heva{&x=2-t \\ &y=1+t \\ &z=t}$}
	{$d: \heva{&x=1+t \\ &y=-1-4t \\ &z=2t}$}
	{$d: \heva{&x=2+2t \\ &y=1+t \\ &z=-t}$} 
	\loigiai{
	Gọi $A(1+2t;-1+t;-t)$ là điểm thuộc $\Delta$. Ta có $\vec{MA}=(2t-1;t-2;-t)$; $\vec{u}_{\Delta}=(2;1;-1)$.
	\[
	MA\perp \Delta \Leftrightarrow \vec{MA}\cdot \vec{u}_{\Delta}=0\Leftrightarrow 2(2t-1)+(t-2)-1(-t)=0\Leftrightarrow t=\dfrac{2}{3}.
	\]
	Suy ra một vec-tơ chỉ phương của $d$ là $\vec{u}_d=(1;-4;-2)$.
	}
\end{ex}
\begin{ex}%[HKII, Tam Quan - Bình Định, 2018]%[Nguyễn Hồng Điệp, 12EX-9-2018]%[2H3B3-2]
	PTTS của đường thẳng đi qua điểm $A\left(1;4;7\right)$ và vuông góc với mặt phẳng $(P)\colon x+2y-2z-3=0$ là
	\choice
	{\True $\heva{
	& x=1+2t \\ 
	& y=4+4t \\ 
	& z=7-4t}$}
	{$\heva{
	& x=-4+t \\ 
	& y=3+2t \\ 
	& z=-1-2t}$}
	{$\heva{
	& x=1+4t \\ 
	& y=4+3t \\ 
	& z=7+t}$}
	{$\heva{
	& x=1+t \\ 
	& y=2+4t \\ 
	& z=-2+7t}$}
	\loigiai{
	$(P)$ có véc-tơ pháp tuyến là $\overrightarrow{n}=(1;2;-2)$.	\\
	Do đường thẳng $d$ song song mặt phẳng $(P)$ nên $d$ có véc-tơ chỉ phương là $ \overrightarrow{u}=(1;2;-2)$.\\
	PTTS $d\colon \heva{
	& x=1+2t \\ 
	& y=4+4t \\ 
	& z=7-4t}$.
	} 
\end{ex}
\begin{ex}%[Thi thử THPT QG THPT Bình Giang, Hải Dương, 2017-2018]%[Nhật Thiện 12EX10]%[2H3B3-2]
	Trong KG $Oxyz$, cho điểm $M(-1;1;3)$ và hai đường thẳng \break$\Delta\colon\dfrac{x-1}{3}=\dfrac{y+3}{2}=\dfrac{z-1}{1}, \Delta '\colon\dfrac{x+1}{1}=\dfrac{y}{3}=\dfrac{z}{-2}$. Phương trình nào dưới đây là PTĐT đi qua $M$, vuông góc với $\Delta $ và $\Delta '$. 
	\choice
	{\True $\left\{\begin{aligned}& x=-1-t \\
	& y=1+t \\
	& z=3+t 
	\end{aligned}\right.$	}
	{$\left\{\begin{aligned}& x=-t \\
	& y=1+t \\
	& z=3+t 
	\end{aligned}\right.$	}
	{$\left\{\begin{aligned}& x=-1-t \\
	& y=1-t \\
	& z=3+t 
	\end{aligned}\right.$	}
	{$\left\{\begin{aligned}& x=-1-t \\
	& y=1+t \\
	& z=1+3t 
	\end{aligned}\right.$}
	\loigiai{
	$\Delta$ có véc-tơ chỉ phương $\vec{u}_1=(3;2;1)$, $\Delta'$ có véc-tơ chỉ phương $\vec{u}_2=(1;3;-2)$. Gọi $d$ là đường thẳng cần tìm, khi đó $d$ có véc-tơ chỉ phương là $[\vec{u}_1,\vec{u}_2]=(-7;7;7)$ hay $\vec{u}=(-1;1;1)$.\\
	PTĐT $d$ là $\heva{&x=-1-t\\ &y=1+t\\ &z=3+t}$.
	}
\end{ex}
\begin{ex}%[Thi thử L2, Nguyễn Khuyến, Nam Định, 2018]%[Nguyễn Tài Tuệ, dự án EX10]%[2H3B3-2]
	Phương trình nào sau đây là PTĐT đi qua hai điểm $A\left({-1;3;2}\right),B\left({1;4;-2}\right)$?
	\choice
	{$\left\{\begin{aligned}& x=-1-2t \\& y=3+t \\& z=2+4t 
	\end{aligned}\right. \,\, t\in \mathbb{R}$
	}
	{$\dfrac{x-1}{2}=\dfrac{y+3}{1}=\dfrac{z+2}{-4}$}
	{$\dfrac{x+1}{2}=\dfrac{y-3}{1}=\dfrac{z-2}{4}$}
	{\True $\dfrac{x-1}{-2}=\dfrac{y-4}{-1}=\dfrac{z+2}{4}$}
	\loigiai{
	Ta có $ \vec{BA}=(-2;-1;4) $ là vec-tơ chỉ phương của đường thẳng $ AB .$\\
	Đường thẳng $ AB $ đi qua điểm $ B $ nên có phương trình chính tắc là $\dfrac{x-1}{-2}=\dfrac{y-4}{-1}=\dfrac{z+2}{4}$.
	}
\end{ex}
\begin{ex}%[Đề KSCL, Số 2 An Nhơn, Bình Định 2018]%[Nguyễn Thị Kiều Ngân, dự án 12EX-10]%[2H3B3-2]
	Trong KG $Oxyz$, cho ba đường thẳng $d_1 \colon \dfrac{x-3}{-1} =\dfrac{y-3}{-2} =\dfrac{z+2}{1}$; $d_2 \colon \dfrac{x-5}{-3} =\dfrac{y+1}{2} =\dfrac{z-2}{1}$ và $\Delta \colon \dfrac{x+1}{1} =\dfrac{y-3}{2} =\dfrac{z-1}{3}$. Đường thẳng song song với $\Delta$, cắt $d_1$ và $d_2$ có phương trình là
	\choice
	{$\dfrac{x-1}{3} =\dfrac{y+1}{2} =\dfrac{z}{1}$}
	{$\dfrac{x-2}{1} =\dfrac{y-3}{2} =\dfrac{z-1}{3}$}
	{$\dfrac{x-3}{1} =\dfrac{y-3}{2} =\dfrac{z+2}{3}$}
	{\True $\dfrac{x-1}{1} =\dfrac{y+1}{2} =\dfrac{z}{3}$}
	\loigiai{
	Đường thẳng $d$ cắt $d_1$ tại $M(3-t;3-2t;-2+t)$.\\
	Đường thẳng $d$ cắt $d_2$ tại $N(5-3s;-1+2s;2+s)$.\\
	Đường thẳng $d$ có một véc-tơ chỉ phương là $\vv{MN}=(2-3s+t;-4+2s+2t;4+s-t)$.\\
	Vì $d$ song song $\Delta$ nên $d$ cũng có véc-tơ chỉ phương là $\vv{u}=(1;2;3)$.\\
	Khi đó $\vv{MN}$ cùng phương $\vv{u}$, suy ra 
	\begin{eqnarray*}
	& & \dfrac{2-3s+t}{1} =\dfrac{-4+2s+2t}{2} =\dfrac{4+s-t}{3}\\
	& \Leftrightarrow & \heva{&4-6s+2t=-4+2s+2t\\&-12+6s+6t=8+2s-2t}
	\Leftrightarrow \heva{&s=1\\&t=2.}
	\end{eqnarray*}
	Do đó đường thẳng $d$ qua $M(1;-1;0)$ và nhận $\vv{u}=(1;2;3)$ làm véc-tơ chỉ phương.\\
	Vậy phương trình chính tắc của $d$ là $\dfrac{x-1}{1} =\dfrac{y+1}{2} =\dfrac{z}{3}$.
	}
\end{ex}
\begin{ex}%[TT lần 2 - Chuyên Lê Hồng Phong Nam Định - 2018]%[Nguyễn Văn Vũ, 12EX-10]%[2H3B3-2]
	Trong KG $Oxyz$, cho điểm $A(-4; -2; 4)$ và đường thẳng $d:\heva{&x =-3 + 2t\\&y = 1 - t\\&
	z = -1 + 4t}$. Viết PTĐT $\Delta$ đi qua $A$ cắt và vuông góc với đường thẳng $d$.
	\choice
	{\True $\Delta: \heva{&
	x = -4 + 3t\\&
	y = -2 + 2t\\&
	z = 4 - t}$ }
	{ $\Delta: \heva{&
	x = -4 + 3t\\&
	y = -2 - t\\&
	z = 4 - t}$}
	{ $\Delta: \heva{&
	x = -4 - 3t\\&
	y = -2 + 2t\\&
	z = 4 - t}$}
	{ $\Delta:\heva{&
	x = -4 + t\\&
	y = -2 + t\\&
	z = 4 + t}$}
	\loigiai{
	Gọi $H$ là hình chiếu của $A$ lên đường thẳng $d$. Ta có
	$H(-3+2t;1-t;-1+4t)$.\\
	Suy ra $\vec{AH}\cdot \vec{u}_d =0 \Leftrightarrow t=1$.\\
	$\vec{AH}=(3;2;-1)$. Vậy ptdt là $\Delta: \heva{&
	x = -4 + 3t\\&
	y = -2 + 2t\\&
	z = 4 - t}$ .
	}
\end{ex}
\begin{ex}%[Đề Thi thử, Sở GD-ĐT Quảng Bình 2018]%[Đỗ Đường Hiếu, (dự án 12EX-10)]%[2H3B3-2]
	Trong KG $Oxyz$, đường thẳng $\Delta$ đi qua $A(1;2;-1)$ và song song với đường thẳng $d\colon \dfrac{x-3}{1}=\dfrac{y-3}{3}=\dfrac{z}{2}$ có phương trình là 
	\choice
	{\True $\dfrac{x-1}{-2}=\dfrac{y-2}{-6}=\dfrac{z+1}{-4}$}
	{$\dfrac{x+1}{1}=\dfrac{y+2}{3}=\dfrac{z-1}{2}$}
	{$\dfrac{x-1}{1}=\dfrac{y-2}{-3}=\dfrac{z+1}{-2}$}
	{$\dfrac{x-1}{2}=\dfrac{y-2}{3}=\dfrac{z+1}{1}$}
	\loigiai{
	Đường thẳng $d\colon \dfrac{x-3}{1}=\dfrac{y-3}{3}=\dfrac{z}{2}$ đi qua điểm $M(3;3;0)$ và có véc-tơ chỉ phương $\overrightarrow{u}=(1;3;2)$.\\
	Đường thẳng $\Delta\colon \dfrac{x-1}{-2}=\dfrac{y-2}{-6}=\dfrac{z+1}{-4}$ đi qua $A(1;2;-1)$ và có véc-tơ chỉ phương $\overrightarrow{v}=(-2;-6;-4)$, mặt khác véc-tơ $\overrightarrow{v}$ cùng phương với véc-tơ $\overrightarrow{u}$, điểm $A$ không thuộc $d$ nên đường thẳng $\Delta$ song song với đường thẳng $d$.
	}
\end{ex}
\begin{ex}%[Thi Thử Sở Đà Nẵng 2018]%[Phan Hoàng Anh - 12-EX-10]%[2H3B3-2]
	Trong KG $Oxyz$, cho hai mặt phẳng $(P)\colon 3x-y-3z+2=0$ và $(Q)\colon -4x+y+2z+1=0$. PTĐT đi qua gốc tọa độ $O$ và song song với hai mặt phẳng $(P)$, $(Q)$ là
	\choice
	{$\dfrac{x}{1}=\dfrac{y}{-1}=\dfrac{z}{6}$}
	{$\dfrac{x}{1}=\dfrac{y}{-6}=\dfrac{z}{-1}$}
	{$\dfrac{x}{1}=\dfrac{y}{1}=\dfrac{z}{6}$}
	{\True $\dfrac{x}{1}=\dfrac{y}{6}=\dfrac{z}{-1}$}
	\loigiai{Mặt phẳng $(P)$ có véc-tơ pháp tuyến là $\overrightarrow{n}_P=(3;-1;-3)$.\\
	Mặt phẳng $(Q)$ có véc-tơ pháp tuyến là $\overrightarrow{n}_Q=(-4;1;2)$.\\
	Đường thẳng $d$ song song với cả $(P)$ và $(Q)$ nên có véc-tơ chỉ phương là $\overrightarrow{u}=\left[\overrightarrow{n}_P,\overrightarrow{n}_Q\right]=(1;6;-1)$.\\
	Do $d$ đi qua gốc tọa độ $O$ nên phương trình của $d$ là $\dfrac{x}{1}=\dfrac{y}{6}=\dfrac{z}{-1}$.}
\end{ex}
\begin{ex}%[TT L4, chuyen Quang Trung, Bình Phước, 2018]%[Lê Mạnh Thắng, 12EX-10]%[2H3B3-2]
	Trong KG $Oxyz$, cho đường thẳng $d\colon \dfrac{x-1}{1}=\dfrac{y-1}{-1}=\dfrac{z}{3}$ và mặt phẳng $(P)\colon x+3y+z=0$. Đường thẳng $\Delta$ đi qua $M(1;1;2)$, song song với mặt phẳng $(P)$ đồng thời cắt đường thẳng $d$ có phương trình là
	\choice
	{$\dfrac{x-3}{1}=\dfrac{y+1}{-1}=\dfrac{z-9}{2}$}
	{$\dfrac{x+2}{1}=\dfrac{y+1}{-1}=\dfrac{z-6}{2}$}
	{$\dfrac{x-1}{-1}=\dfrac{y-1}{2}=\dfrac{z-2}{1}$}
	{\True $\dfrac{x-1}{1}=\dfrac{y-1}{-1}=\dfrac{z-2}{2}$}
	\loigiai{
	Mặt phẳng $(P)$ có $1$ véc-tơ pháp tuyến $\overrightarrow{n}_P=(1;3;1)$.\\
	Giả sử đường thẳng $\Delta$ cắt $d$ tại điểm $N$.
	\begin{itemize}
	\item $N\in d \,\Rightarrow\, N(1+t;1-t;3t)$ $\,\Rightarrow\, \overrightarrow{MN}=(t;-t;3t-2)$.
	\item $\Delta \parallel (P) \,\Rightarrow\, \overrightarrow{MN}\cdot \overrightarrow{n}_P=0 
	\,\Leftrightarrow\, 1\cdot t+3\cdot (-t)+1\cdot (3t-2)=0
	\,\Leftrightarrow\, t=2$.
	\end{itemize}	
	Do đó, $\Delta$ có $1$ véc-tơ chỉ phương là $\overrightarrow{MN}=(2;-2;4)=2(1;-1;2)$.\\
	Suy ra, $\Delta\colon \dfrac{x-1}{1}=\dfrac{y-1}{-1}=\dfrac{z-2}{2}$.	
	}
\end{ex}
\begin{ex}%[12-TN-BGD-2]%[Lê Đình Mẫn]%[2H3B3-2]
	Trong KG $Oxyz$, cho đường thẳng $d \colon \dfrac{x}{2}=\dfrac{y-3}{1}=\dfrac{z-2}{-3}$ và mặt phẳng $(P) \colon x-y+2z-6=0$. Đường thẳng nằm trong mặt phẳng $(P)$, cắt và vuông góc với $d$ có phương trình
	\choice
	{\True $\dfrac{x+2}{1}=\dfrac{y-2}{7}=\dfrac{z-5}{3}$}
	{$\dfrac{x-2}{1}=\dfrac{y-4}{7}=\dfrac{z+1}{3}$}
	{$\dfrac{x-2}{1}=\dfrac{y+2}{7}=\dfrac{z+5}{3}$}
	{$\dfrac{x+2}{1}=\dfrac{y+4}{7}=\dfrac{z-1}{3}$}
	\loigiai{
	PTTS của $d \colon \heva{&x=2t\\&y=3+t\\&z=2-3t.}$\\
	Gọi $M(2t;3+t;2-3t)$ là giao điểm của $d$ và $(P)$.\\
	$M$ thuộc $(P)$ nên $2t-(3+t)+2(2-3t)-6=0\Leftrightarrow t=-1$, suy ra $M(-2;2;5)$.\\
	$d$ có véc-tơ chỉ phương của $\overrightarrow{u}_d=(2;1;-3)$,	$(P)$ có véc-tơ pháp tuyến $\overrightarrow{n}_{(P)}=(1;-1;2)$.\\
	Gọi $\Delta$ là đường thẳng nằm trong $(P)$, cắt và vuông góc với $d$. Suy ra $\Delta$ đi qua $M(-2;2;5)$ và có véc-tơ chỉ phương $\overrightarrow{u}_{\Delta}=\left[\overrightarrow{n}_{(P)},\overrightarrow{u}_d\right]=(1;7;3)$.\\
	Vậy $\Delta$ có phương trình là $\dfrac{x+2}{1}=\dfrac{y-2}{7}=\dfrac{z-5}{3}$.
	}
\end{ex}
\begin{ex}%[2H3B3-2]%[Trần Nhân Kiệt - Đề 1]
	Trong không gian $Oxyz,$ cho mặt phẳng $(P) \colon 2x-y+3z+10=0$ và điểm $M(2;-1;2)$. Viết PTĐT $d$ đi qua điểm $M$, vuông góc với trục $Oy$ và song song với mặt phẳng $(P)$.
	\choice
	{$\heva{&x=2+2t\\&y=-1-t\\&z=2+3t}$}
	{$\heva{&x=2+3t\\&y=-1-2t\\&z=2}$}
	{\True $\heva{&x=2+3t\\&y=-1\\&z=2-2t}$}
	{$\heva{&x=2\\&y=-1+3t\\&z=2-2t}$}
	\loigiai
	{
	Ta có $\vec{j}=(0;1;0)$ là véc-tơ chỉ phương của trục $Oy$ và $\vec{n}_{(P)}=(2;-1;3)$ là véc-tơ pháp tuyến của mặt phẳng $(P)$.\\
	Suy ra $\left[\vec{j},\vec{n}_{(Q)}\right]=(3;0;-2)$.\\
	Vì $d$ vuông góc với trục $Oy$ và song song với mặt phẳng $(P)$ nên $\vv{u}=(3;0;-2)$ là véc-tơ chỉ phương của đường thẳng $d$.\\
	PTĐT $d$ là $\heva{&x=2+3t\\&y=-1\\&z=2-2t,} \,t \in \mathbb{R}$.
	}
\end{ex}	
	\begin{dang}{Tìm tọa độ điểm liên quan đến đường thẳng}%dang 3
		\begin{enumerate}
			\item 
		
		Tìm hình chiếu vuông góc $H$ của điểm $M$ trên đường thẳng $d$ ta làm như sau:
		\begin{itemize}
			\item Tìm véc-tơ chỉ phương $\overrightarrow{u}_d$ của đường thẳng $d$.
			\item Viết PTTS của đường thẳng $d\colon  \left\{\begin{aligned}
			&x=x_0+at\\
			&y=y_0+bt\\
			&z=z_0+ct\\
			\end{aligned}\right.$ (*).
			\item Sử dụng (*) để ghi thành toạ độ cho $H$, tức là $H(x_0+at\,;y_0+bt\,;z_0+ct)$.
			\item Tính $\overrightarrow{MH}$ theo $t$. Cho $\overrightarrow{MH}\cdot \overrightarrow{u}_d=0$, tìm $t$. Sau đó tìm được $H$.
		\end{itemize}
		\begin{center}
			\begin{tikzpicture}[scale=0.8,>=stealth, font=\footnotesize, line join=round, line cap=round]
			\clip (-0.6,-0.3) rectangle (3.5,5);
			\tkzDefPoints{0/0/A,3/5/B,0.4/4.3/M}
			\tkzDefPointBy[projection=onto A--B](M)\tkzGetPoint{H}
			\tkzDrawSegments(A,B M,H)
			\tkzDrawPoints[fill=black](H,M)
			\tkzLabelPoints[left](M)
			\tkzLabelPoints[below right=-2pt](H)
			\tkzMarkRightAngles[size=0.2](M,H,A)
			\tkzLabelPoint[above left](A){$d$}
			\end{tikzpicture} \qquad\qquad
			\begin{tikzpicture}[scale=0.8,>=stealth, font=\footnotesize, line join=round, line cap=round]
			\clip (-0.6,-0.3) rectangle (4.5,5);
			\tkzDefPoints{0/0/A,3/5/B,0.4/4.3/M}
			\tkzDefPointBy[projection=onto A--B](M)\tkzGetPoint{H}
			\coordinate (M') at ($(H)!-1!(M)$);
			\tkzDrawSegments(A,B M,M')
			\tkzDrawPoints[fill=black](H,M,M')
			\tkzLabelPoints[left](M)
			\tkzLabelPoints[below=3pt](H)
			\tkzMarkRightAngles[size=0.2](M,H,A)
			\tkzLabelPoints[right](M')
			\tkzLabelPoint[above left](A){$d$}
			\end{tikzpicture}
		\end{center}
		\begin{note}
			Nếu đề bài có thêm yêu cầu tìm điểm $M'$ đối xứng với $M$ qua $d$:
			\begin{itemize}
				\item  Ta vẫn phải giải bài toán tìm hình chiếu vuông góc $H$ của $M$ trên $d$ (như trên).
				\item Tiếp tục cho $M'$ đối xứng với $M$ qua $H$ và tìm được $M'$ bởi công thức: $$ \left\{\begin{aligned}
				&x_{M'}=2x_H-x_M\\
				&y_{M'}=2y_H-y_M\\
				&z_{M'}=2z_H-z_M.\\
				\end{aligned}\right.$$
			\end{itemize}
		\end{note}
	\item Để tìm hình chiếu vuông góc $H$ của điểm $M$ trên mặt phẳng $(P)$ ta làm như sau:
	\begin{itemize}
		\item Gọi $d$ là đường thẳng đi qua $M$ và vuông góc với $(P)$. Viết PTTS của $d$ $$d\colon  \left\{\begin{aligned}
		&x=x_M+at\\
		&y=y_M+bt\\
		&z=z_M+ct\\
		\end{aligned}\right. (*),\text{ trong đó } \overrightarrow{n}=(a;b;c) \text{ là véc-tơ pháp tuyến của } (P).$$
		\item Sử dụng (*) để ghi thành toạ độ cho $H$, tức là $H(x_M+at\,;y_M+bt\,;z_M+ct)$.
		\item Thay toạ độ của $H$ vào phương trình của $(P)$ để tìm $t$, từ đó tìm được $H$.
	\end{itemize}
	\begin{center}
		\begin{tikzpicture}[scale=0.8,>=stealth, font=\footnotesize, line join=round, line cap=round]
		\tkzDefPoints{0/0/A,4.5/0/B,3/-1.5/C}
		\coordinate (D) at ($(A)+(C)-(B)$);
		\coordinate (H) at ($(A)!0.5!(C)$);
		\coordinate (M) at ($(H)+(0,1.4)$);
		\coordinate (d) at ($(H)+(0,2.5)$);
		\tkzDrawPoints[fill=black](H,M)
		\tkzDrawSegments(H,d)
		\tkzDrawPolygon(A,B,C,D)
		\tkzLabelPoints[left](M,H)
		\tkzLabelPoints[right](d)
		\tkzMarkAngles[size=0.9cm,arc=l](C,D,A)
		\tkzLabelAngles[pos=0.6,rotate=0](C,D,A){$P$}
		\end{tikzpicture} \qquad\qquad
		\begin{tikzpicture}[scale=0.8,>=stealth, font=\footnotesize, line join=round, line cap=round]
		\tkzDefPoints{0/0/A,4.5/0/B,3/-1.5/C}
		\coordinate (D) at ($(A)+(C)-(B)$);
		\coordinate (H) at ($(A)!0.5!(C)$);
		\coordinate (M) at ($(H)+(0,1.4)$);
		\coordinate (M') at ($(H)-(0,1.4)$);
		\coordinate (d) at ($(H)+(0,2.5)$);
		\tkzInterLL(M,M')(C,D)\tkzGetPoint{x}
		\tkzDrawPoints[fill=black](H,M,M')
		\tkzDrawSegments(H,d x,M')
		\tkzDrawSegments[dashed](H,x)
		\tkzDrawPolygon(A,B,C,D)
		\tkzLabelPoints[left](M',M,H)
		\tkzLabelPoints[right](d)
		\tkzMarkAngles[size=0.9cm,arc=l](C,D,A)
		\tkzLabelAngles[pos=0.6,rotate=0](C,D,A){$P$}
		\end{tikzpicture}
	\end{center}
	\begin{note}
		Nếu đề bài có thêm yêu cầu tìm điểm $M'$ đối xứng với $M$ qua $(P)$:
		\begin{itemize}
			\item  Ta vẫn phải giải bài toán tìm hình chiếu vuông góc $H$ của $M$ trên $(P)$ (như trên).
			\item Tiếp tục cho $M'$ đối xứng với $M$ qua $H$ và tìm được $M'$ bởi công thức: $$ \left\{\begin{aligned}
			&x_{M'}=2x_H-x_M\\
			&y_{M'}=2y_H-y_M\\
			&z_{M'}=2z_H-z_M.\\
			\end{aligned}\right.$$
		\end{itemize}
	\end{note}
\end{enumerate}
	\end{dang}
\setcounter{subsubsection}{0}
\setcounter{vd}{0}
\setcounter{ex}{0}
	\subsubsection{Ví dụ minh hoạ}
	\begin{vd}%[Lê Phúc Lữ lần 2, WTB]%[2H3B3-3]
	Trong KG $Oxyz$, cho đường thẳng $(d)$ có phương trình $\heva{&x = 1 + 2t \\ &y = 2 + 3t \\ &z = 3 + 4t}$. Hỏi điểm nào trong các điểm sau nằm trên $(d)$: $A(-1; -2; -3)$, $B(2; 3; 4)$, $C(-1; -1; -1)$, $D(0; 0; 0)$?
	\loigiai{
	Thay $t = -1$ vào PTĐT $(d)$, ta được $x = y = z = -1$.
	}
	\end{vd}
	\begin{vd}%[HK2 (2017-2018), THPT Tân Hiệp, Kiên Giang]%[Bùi Mạnh Tiến, dự án (12EX-9)]%[2H3B3-3]
	Trong KG $Oxyz$, cho đường thẳng $\Delta\colon \heva{&x=-1+3t\\&y=1+t\\&z=3t} \left(t\in \mathbb{R}\right)$ và hai điểm $A(5;0;2),B(2;-5;3)$. Tìm điểm $M$ thuộc $\Delta$ sao cho $\triangle ABM$ vuông tại $A$.
	\loigiai{
	Điểm $M$ thuộc đường thẳng $\Delta$ nên $M\left(-1+3t;1+t;3t\right)$.\\
	Ta có $\overrightarrow{AM}=\left(3t-6;t+1;3t-2\right)$ và $\overrightarrow{AB}=\left(-3;-5;1\right)$.\\
	Tam giác $ABM$ vuông tại $M$ khi và chỉ khi 
	\begin{center}
	$\overrightarrow{AB}\perp \overrightarrow{AM}\Leftrightarrow \overrightarrow{AB}\cdot \overrightarrow{AM}=0\Leftrightarrow -3(3t-6)-5(t+1)+3t-2=0\Leftrightarrow t=1$. 
	\end{center}
	Khi đó tọa độ điểm $M(2;2;3)$.
	}
	\end{vd}
	\begin{vd}%[Đề HK2-2018, Chuyên Phan Ngọc Hiển-Cà Mau]%[Phùng Hoàng Em, dự án EX9]%[2H3B3-3]
	Trong KG $Oxyz$, cho điểm $M(2;0;1)$ và đường thẳng $d \colon \dfrac{x-1}{1}=\dfrac{y}{2}=\dfrac{z-2}{1}$. Tìm tọa độ hình chiếu vuông góc của $M$ lên đường thẳng $d$.
	\loigiai{
	Gọi $(P)$ là mặt phẳng đi qua $M\left(2 \text{;}0 \text{;1}\right)$ và vuông góc với đường thẳng $d$. Suy ra $(P)$ nhận $\vec{u_d}=(1;2;1)$ làm véc-tơ pháp tuyến. \\
	Phương trình mặt phẳng $(P):(x-2)+2y+z-1=0 \Leftrightarrow x+2y+z-3=0$. \\
	Gọi $H$ là hình chiếu vuông góc của $M$ lên đường thẳng $d$, suy ra $H=d \cap (P)$. \\
	Tọa độ điểm $H$ là nghiệm của hệ $\heva{& \dfrac{x-1}{1}=\dfrac{y}{2}=\dfrac{z-2}{1}\\&x+2y+z-3=0}\Leftrightarrow \heva{&2x-y=2\\&y-2z=-4\\&x+2y+z-3=0}\Leftrightarrow \heva{&x=1\\&y=0\\&z=2.}$}
	\end{vd}
	\begin{vd}%[HK2, THTH ĐHSP Tp. HCM, 2018]%[2H3B3-3]%[Vinhhop Tran, 12EX-8]
	Trong không gian tọa độ $Oxyz,$ tìm tọa độ điểm $M'$ đối xứng với điểm $M(1; 4; -2)$ qua đường thẳng $(d)\colon \heva{x=&1+2t,\\y=&-1-t,\\z=&2t.}$
	\loigiai{Gọi $H$ là hình chiếu của $M$ lên $(d)$ thì $H$ chính là trung điểm của $MM'$. Vì $H$ thuộc $(d)$ nên tọa độ $H$ có dạng $H(1+2t; -1-t; 2t).$ Từ $MH\perp (d)$ ta suy ra $t=-1$. Do đó, $H(-1; 0; -2)$ và $M'(-3; -4; -2).$
	}
	\end{vd}
	\begin{vd}%[TT-THTT-Lan8-2018]%[Phạm Tuấn, 12-EX-10-2018]%[2H3B3-3]%20
	Trong KG $Oxyz$, cho $A(3;4;1)$, $B(-3;-2;-2)$. Đường thẳng qua $A$ và $B$ cắt mặt phẳng $(Oxy)$ tại $M$. Tính tỉ số $k=\dfrac{MA}{MB}$.
	\choice
	{$k=-\dfrac{1}{2}$}
	{$k=2$}
	{$k=-2$}
	{\True $k=\dfrac{1}{2}$}
	\loigiai{
	$M \in (Oxy) \Rightarrow M(a;b;0)$. Điểm $M$ thuộc đường thẳng đi qua $AB$ cho nên $$\dfrac{a-3}{6}=\dfrac{b-4}{6}=\dfrac{0-1}{3} \Rightarrow a=1;b=2$$
	Ta có $\overrightarrow{AM} =(-2;-2;-1)$, $\overrightarrow{AB}=(-6;-6;-3)$, $\overrightarrow{AM}=\dfrac{1}{3} \overrightarrow{AB}$, suy ra $k=\dfrac{1}{2}$.
	}
	\end{vd}
	\subsubsection{Bài tập trắc nghiệm}
	\begin{ex}%[Thi HK2, Sở GD\&ĐT Đồng Tháp, 2018]%[2H3B3-3]%[Trần Hòa, dự án (12EX-8)]
	Trong KG $Oxyz$, cho đường thẳng $(\Delta) \colon \dfrac{x+1}{1}=\dfrac{y+4}{2}=\dfrac{z}{1}$ và điểm $A(2;0;1)$. Hình chiếu vuông góc của $A$ trên $(\Delta)$ là điểm nào dưới đây?
	\choice
	{$Q(2;2;3)$}
	{$M(-1;4;-4)$}
	{$N(0;-2;1)$}
	{\True $P(1;0;2)$}
	\loigiai{
	Đường thẳng $(\Delta)$ đi qua $M(-1;-4;0)$, có véc-tơ chỉ phương $\vec{u}_{\Delta}=(1;2;1)$.\\
	PTTS của đường thẳng $(\Delta)\colon 
	\begin{cases}
	x=-1+t\\
	y=-4+2t\\
	z=t.
	\end{cases}$\\
	Gọi $P$ là hình chiếu vuông góc của $A$ trên $(\Delta)$. Khi đó $P\in (\Delta) \Rightarrow P(-1+t;-4+2t;t)$.\\
	Ta có $\overrightarrow{AP}=(-3+t;-4+2t;t-1)$.\\
	Vì $\overrightarrow{AP} \perp \vec{u}_{\Delta}$ nên $\overrightarrow{AP} \cdot \vec{u}_{\Delta} = 0 \Leftrightarrow 1\cdot (-3+t) + 2\cdot(-4+2t) + 1\cdot (t-1) = 0 \Leftrightarrow t = 2 \Rightarrow P(1;0;2)$.
	}
	\end{ex}
	\begin{ex}%[Thi HK2, THPT Lý Thái Tổ, Hà Nội , 2018] %[2H3B3-3]%[Trần Tuấn Việt, dự án(12EX-8)
	Trong KG $Oxyz$, cho điểm $A(1; 1; 1)$ và đường thẳng $d: \heva{6 - 4t\\-2-t\\-1+2t}(t \in \mathbb{R})$. Hình chiếu của $A$ trên $d$ có tọa độ là
	\choice
	{$(-2; 3; 1)$}
	{\True $(2; -3; 1)$}
	{$(2; 3; 1)$}
	{$(2; -3; -1)$}
	\loigiai{Gọi $H(6 - 4t; -2 - t; -1 + 2t)$ là hình chiếu của $A$ lên đường thẳng $d$. Ta có $\overrightarrow{AH} = (5 - 4t; -3 - t; -2 + 2t)$ và $\overrightarrow{u_d} = (4, 1, -2)$. Do $AH \perp d$ nên 
	$\overrightarrow{AH}. \overrightarrow{u}_d = 0$ \\
	$\Leftrightarrow 4(5 - 4t) + 1 (-3 - t) - 2(-2 + 2t) = 0$\\
	$\Leftrightarrow t = 1.$ Vậy tọa độ điểm $H$ là $H(2; -3; 1)$.
	}
	\end{ex}
	\begin{ex}%[2-TT-33-BenTre-VinhPhuc-18]%[2H3B3-3]%[Đoàn Nhật Thiện]%[12EX-8]
	Trong KG $Oxyz$, cho đường thẳng $d\colon \dfrac{x-12}{4}=\dfrac{y-9}{3}=\dfrac{z-1}{1}$ và mặt phẳng $(P)\colon 3x+5y-z-2=0$ cắt nhau tại điểm $M(a;b;c)$ khi đó $a+b+c$ có giá trị là
	\choice
	{$5 $}
	{\True $-2 $}
	{$2 $}
	{$3 $}
	\loigiai{
	Tọa độ giao điểm của $d$ và $(P)$ thỏa hệ $$\left\{ \begin{aligned}&\dfrac{x-12}{4}=\dfrac{y-9}{3}\\&\dfrac{y-9}{3}=\dfrac{z-1}{1}\\&3x+5y-z-2=0\end{aligned} \right. \Leftrightarrow \left\{ \begin{aligned} & 3x-4y=0\\ &y-3z=6\\ &3x+5y-z=2 \end{aligned} \right. \Leftrightarrow \left\{ \begin{aligned} &x=0\\ &y=0\\ &z=-2 \end{aligned} \right.\Rightarrow M(0;0;-2).$$ Khi đó $a+b+c=-2$.
	}
	\end{ex}
	\begin{ex}%[Thi thử L1, Chuyên Hà Tĩnh, 2018]%[2H3B3-3]%[Nguyễn Thế Út, dự án(12EX-8)]
	Trong không gian $ Oxyz $, cho đường thẳng $ d\colon \dfrac{x+1}{1}=\dfrac{y+3}{2}=\dfrac{z+2}{2} $ và điểm $ A(3;2;0) $. Điểm đối xứng với điểm $ A $ qua đường thẳng $ d $ có tọa độ là
	\choice
	{\True $ (-1;0;4) $}
	{$ (7;1;-1) $}
	{$ (2;1;-2) $}
	{$ (0;2;-5) $}
	\loigiai{
	Gọi $M(-1+t;-3+2t;-2+2t)\in d\Rightarrow \vec{AH}=(t-4;2t-5;2t-2)$. Véc-tơ chỉ phương của $d$ là $\vec{u}=(1;2;2)$.\\ Vì $\vec{AH}\perp \vec{u}$ nên $\vec{AH}\cdot \vec{u}=0\Leftrightarrow 1(t-4)+2(2t-5)+2(2t-2)=0\Leftrightarrow t=2.$ \\ Suy ra $M(1;1;2)$, gọi $A'(x;y;z)$ là điểm đối xứng của $A$ qua $d$ thì $\heva{&x=2\cdot 1-3=-1\\&y=2\cdot 1-2=0\\&z=2\cdot 2-0=4.}$\\ Do đó $ A'(-1;0;4).$
	}
	\end{ex}
	\begin{ex}%[Thi thử L2, THPT Huỳnh Thúc Kháng Khánh Hòa, 2018]%[Dương Xuân Lợi, dự án 12(Ex-9)]%[2H3B3-3]
	Gọi $H(a; b; c)$ là hình chiếu của $A(2; -1; 1)$ lên đường thẳng $(d) \colon \heva{&x=1\\ &y=4+2t\\&z=-2t}$. Đẳng thức nào dưới đây đúng?
	\choice
	{$a + 2b + 3c = 10$}
	{$a + 2b + 3c = 5$}
	{$a + 2b + 3c = 8$}
	{\True $a + 2b + 3c = 12$}
	\loigiai{
	Vì $H \in (d)$ $\Rightarrow H(1;4+2t;-2t)$, $\overrightarrow{AH}=(-1;5+2t;-1-2t)$.\\
	$(d)$ có vtcp $\vec{u}=(0;2;-2)$.
	$$\overrightarrow{AH}\cdot \vec{u}=0 \Leftrightarrow (-1)0 +(5+2t)2+(-1-2t)(-2)=0 \Leftrightarrow 8t+12=0 \Leftrightarrow t=\dfrac{-3}{2}.$$
	Suy ra $H(1;1;3)$.\\
	Vậy $a+2b+3c=12$.	
	}
	\end{ex}
	\begin{ex}%[Đề thi học kì II, Ngô Quyền-Quảng Ninh, 2018]%[Trần Ngọc Minh, dự án 12EX-9-2018]%[2H3B3-3]
	Trong KG $Oxyz$, cho mặt phẳng $(P): x+y-z-1=0$ và đường thẳng $d:\dfrac{x-1}{2}=\dfrac{y-2}{1}=\dfrac{z-3}{2}$. Tìm giao điểm $M$ của $d$ và $(P)$.
	\choice 
	{$M(3;-3;-5)$}
	{$M(3;3;-5$)}
	{\True $M(3;3;5)$}
	{$M(-3;-3;-5)$} 
	\loigiai{ 
	Tọa độ giao điểm $M$ thỏa mãn hệ phương trình 
	$\left \lbrace \begin{aligned} &x+y-z-1=0\\ 
	&\dfrac{x-1}{2}=\dfrac{y-2}{1}=\dfrac{z-3}{2} \\ \end{aligned} \right. .$\\ 
	Giải hệ trên, được nghiệm $(3;3;5)$. Từ đó điểm $M(3;3;5)$.
	} 
	\end{ex} 
	\begin{ex}%[Thi HK2, THPT An Phước, Ninh Thuận, 2018]% [Nguyễn Kim Đông, 12-Ex9]%[2H3B3-3]
	Trong không gian với hệ tọa độ $Oxyz,$ cho đường thẳng $d:\dfrac{x-4}{2}=\dfrac{y-5}{3}=\dfrac{z-6}{4}.$ Điểm nào dưới đây thuộc đường thẳng $d$?
	\choice
	{\True $M\left(2;2;2\right)$}
	{$M\left(2;2;4\right)$}
	{$M\left(2;3;4\right)$}
	{$M\left(2;2;10\right)$}
	\loigiai{Vì $\dfrac{2-4}{2}=\dfrac{2-5}{3}=\dfrac{2-6}{4}=-1$ nên $M\left(2;2;2\right)$ thuộc đường thẳng $d$.}
	\end{ex}
	\begin{ex}%[Thi HK2, THPT An Phước, Ninh Thuận, 2018]% [Nguyễn Kim Đông, 12-Ex9]%[2H3B3-3]
	Trong KG $Oxyz$, cho đường thẳng $d$ có phương trình $\dfrac{x-1}{3}=\dfrac{y+2}{2}=\dfrac{z-3}{-4}$. Điểm nào sau đây \textbf{không} thuộc đường thẳng $d$?
	\choice
	{$Q\left(-2;-4;7\right)$}
	{\True $P\left(7;2;1\right)$}
	{$M\left(1;-2;3\right)$}
	{$N\left(4;0;-1\right)$}
	\loigiai{Vì $\dfrac{7-1}{3}=\dfrac{2+2}{2}\neq \dfrac{1-3}{-4}$ nên $P\left(7;2;1\right)$ không thuộc đường thẳng $d$.}
	\end{ex}
	\begin{ex}%[Học kì 2, THPT Chương Mỹ B, 2018]%[Vương Quyền, dự án (12EX-9)]%[2H3B3-3]
	Trong KG $Oxyz$, cho điểm $M(2;-6;3)$ và đường thẳng $d\colon\heva{& x=1+3t\\& y=-2-2t\\& z=t}$. Tìm tọa độ hình chiếu vuông góc $H$ của $M$ trên $d$.
	\choice
	{$H(1;2;1)$}
	{$H(1;-2;0)$}
	{\True $H(4;-4;1)$}
	{$H(2;2;-2)$}
	\loigiai{
	Đường thẳng $d$ có véc-tơ chỉ phương $\overrightarrow{u}=(3;-2;1)$.\\
	Phương trình mặt phẳng $(P)$ đi qua $M$ và vuông góc với $d$ là: $3x-2y+z-21=0$.\\
	Gọi $H(1+3t;-2-2t;t)\in d\Rightarrow H\in (P)\Rightarrow 3(1+3t)-2(-2-2t)+t-21=0\Leftrightarrow t=1$.\\
	$\Rightarrow H(4;-4;1)$.
	}
	\end{ex} 
	\begin{ex}%[HK2 (2017-2018), Sở Giáo Dục Lâm Đồng]%[Lê Quốc Hiệp dự án EX9]%[2H3B3-3]
	Trong KG $Oxyz$, cho đường thẳng $d\colon\heva{&x=1+5t\\&y=2t\\&z=-3+t}$. Điểm nào dưới đây \textbf{không thuộc} đường thẳng $d$?
	\choice
	{$M(-4;-2;-4)$}
	{$N(1;0;-3)$}
	{\True $P(6;2;2)$}
	{$Q(51;20;7)$}
	\loigiai
	{
	Thế tọa độ $P(6;2;2)$ vào PTĐT $d\colon\heva{&x=1+5t\\&y=2t\\&z=-3+t}$ ta được hệ phương trình 
	\[
	\heva{&6=1+5t\\&2=2t\\&2=-3+t} \Leftrightarrow \heva{&t=1\\&t=1\\&t=-5}\text{ (vô nghiệm).}
	\]
	Vậy $P(6;2;2)$ không thuộc $d$.
	}
	\end{ex}
	\begin{dang}{Góc}%dang 4
	\begin{enumerate}
	\item Cho đường thẳng $d$ có véc-tơ chỉ phương $\vec{u}$ và mặt phẳng $ (P) $ có véc-tơ pháp tuyến $ \vec{n} $. Góc giữa đường thẳng $d$ và mặt phẳng $ (P) $: $$\sin\varphi=|\cos\left(\vec{u},\vec{n}\right)| \quad \left(0\leq \varphi \leq \dfrac{\pi}{2}\right)$$
	\item Cho hai đường thẳng chéo nhau $d_1$ có véc-tơ chỉ phương $\vec{u}$ và $d_2$ có véc-tơ chỉ phương $\vec{v}$. Góc giữa hai đường thẳng $ d_1 $ và $ d_2 $:
	$$\left(d_1,d_2\right)=\dfrac{\left|\left[\vec{u};\vec{v}\right]\cdot\vec{M_1M_2}\right|}{|\left[\vec{u};\vec{v}\right]|} $$
	\end{enumerate}
	\end{dang}
\setcounter{subsubsection}{0}
\setcounter{vd}{0}
\setcounter{ex}{0}
	\subsubsection{Ví dụ minh hoạ}
	\begin{vd}%[Phan Quốc Trí]%[2H3B3-4]
	Trong KG $Oxyz$, góc giữa đường thẳng $d:\dfrac{x-3}{2}=\dfrac{y+1}{1}=\dfrac{z-3}{1}$ và mặt phẳng $(P):x+2y-z+5=0$ là
	\choice
	{$30^{\circ}$}
	{\True $45^{\circ}$}
	{$60^{\circ}$}
	{$90^{\circ}$}
	\loigiai{
	Gọi $\varphi$ là góc giữa $d$ và $(P)$. Ta có 
	$$\sin \varphi = \dfrac{\left| 2\cdot 1+1\cdot 2+1\cdot (-1)\right|}{\sqrt{2^2+1^2+1^2} \cdot \sqrt{1^2 +2^2+(-1)^2}}=\dfrac{\sqrt{2}}{2} \Rightarrow \varphi = 45^{\circ}.$$
	}
	\end{vd}
	\begin{vd}%[Phan Quốc Trí]%[2H3B3-4]
	Trong KG $Oxyz$, góc giữa hai đường thẳng $d:\dfrac{x}{1}=\dfrac{y+1}{-1}=\dfrac{z-1}{2}$ và $d': \dfrac{x+1}{2}=\dfrac{y}{1}=\dfrac{z-3}{1}$ là 
	\choice
	{$30^{\circ}$}
	{$45^{\circ}$}
	{\True $60^{\circ}$}
	{$90^{\circ}$}
	\loigiai{
	Gọi $\varphi$ là góc giữa hai đường thẳng $d$ và $d'$. Ta có
	$$\cos \varphi = \dfrac{\left| 1\cdot 2 + 1 \cdot (-1)+2\cdot 1 \right|}{\sqrt{1^2+(-1)^2+2^2}\cdot \sqrt{2^2+1^2+1^2}} = \dfrac{1}{2} \Rightarrow \varphi = 60^{\circ}.$$
	}
	\end{vd}
	\begin{vd}%[Phan Quốc Trí]%[2H3B3-4]
	Trong KG $Oxyz$, góc giữa hai đường thẳng $d:\heva{&x=1-t\\&y=t \\&z=0}$ và \break $d': \dfrac{x}{-2}=\dfrac{y}{1}=\dfrac{z-1}{-2}$ là 
	\choice
	{$30^{\circ}$}
	{\True $45^{\circ}$}
	{$60^{\circ}$}
	{$90^{\circ}$}
	\loigiai{
	Gọi $\varphi$ là góc giữa hai đường thẳng $d$ và $d'$. Ta có
	$$\cos \varphi = \dfrac{\left| (-1)\cdot (-2) + 1 \cdot 1+0\cdot (-2) \right|}{\sqrt{(-1)^2+1^2+0^2}\cdot \sqrt{(-2)^2+1^2+(-2)^2}} = \dfrac{\sqrt{2}}{2} \Rightarrow \varphi = 45^{\circ}.$$
	}
	\end{vd}
	\begin{vd}%[Thi thử L3, Chuyên Lam Sơn - Thanh Hóa, 2018]%[Trần Chiến, dự án EX10]%[2H3K3-4]
	Trong không gian với hệ trục tọa độ $Oxyz,$ cho điểm $A(1;0;6).$ Biết rằng có hai điểm $M, N$ phân biệt thuộc trục $Ox$ sao cho các đường thẳng $AM, AN$ cùng tạo với đườntg thẳng chứa trục $Ox$ một góc $45^\circ .$ Tính ổng các hoành độ hai điểm $M, N$. 
	\loigiai{
	Đặt $M(t;0;0) \Rightarrow \overrightarrow {AM}= (t-1;0;-6), \overrightarrow {u}_{Ox}=(1;0;0).$\\
	Áp dụng công thức góc giữa hai đường thẳng ta có: $$\cos45^\circ = \dfrac{|t-1|}{\sqrt{(t-1)^2 +36}}=\dfrac{1}{\sqrt{2}}\Rightarrow(t-1)^2 =36 \Leftrightarrow \left[ \begin{gathered}
	t=7\hfill \\
	t=-5.\hfill \\ 
	\end{gathered}\right.$$
	Hai điểm $M(7;0;0), N(-5;0;0),$ tổng hoành độ là: $7+(-5)=2.$
	}
	\end{vd}
	\subsubsection{Bài tập trắc nghiệm}
	\begin{ex}%[Đề HK2-2018, Marie Curie-HCM]%[Võ Đức Trí - EX9]%[2H3B3-4]
	Trong KG $Oxyz$, mặt phẳng $(P) \colon 3x+4y+5z+8=0$ và đường thẳng $d$ là giao tuyến của hai mặt phẳng $(\alpha) \colon x-2y+1=0$, $(\beta) \colon x-2z-3=0$. Góc giữa $d$ và $(P)$ bằng
	\choice
	{$45^\circ$}
	{$90^\circ$}
	{$30^\circ$}
	{\True $60^\circ$}
	\loigiai{
	$(P)$, $(\alpha)$, $(\beta)$ có véc-tơ pháp tuyến lần lượt là $\overrightarrow{n}_P=(3;4;5)$, $\overrightarrow{n}_\alpha = (1;-2;0)$, $\overrightarrow{n}_\beta=(1;0;-2)$. Véc-tơ chỉ phương của $d$ là $\overrightarrow{u}=\left[\overrightarrow{n}_\alpha, \overrightarrow{n}_\beta\right] = (4;2;2)$. Gọi $\varphi$ là góc giữa $d$ và $(P)$, ta có:
	$$\sin \varphi = \dfrac{\left| \overrightarrow{n}_P\cdot \overrightarrow{u} \right|}{\left| \overrightarrow{n}_P\right| \cdot \left| \overrightarrow{u} \right|}=\dfrac{\sqrt{3}}{2} \Rightarrow \varphi=60^\circ.$$
	}
	\end{ex}
	\begin{ex}%[2H3B3-4]%[Đề thi thử L2, THPT Nguyễn Quang Diêu, 2018]%[Đỗ Đường Hiếu, 12EX-9]
	Trong không gian với hệ trục tọa độ $Oxyz$, cho điểm $H(2;-1;-2)$ là hình chiếu vuông góc của gốc tọa độ $O$ xuống mặt phẳng $(P)$, số đo góc giữa mặt phẳng $(P)$ và mặt phẳng $(Q)\colon x-y-11=0$ bằng bao nhiêu?
	\choice
	{\True $45^\circ$}
	{$30^\circ$}
	{$90^\circ$}
	{$60^\circ$}
	\loigiai{
	Vì $H(2;-1;-2)$ là hình chiếu vuông góc của gốc tọa độ $O$ xuống mặt phẳng $(P)$ nên mặt phẳng $(P)$ có véc-tơ pháp tuyến $\overrightarrow{n}_P=\overrightarrow{OH}=(2;-1;-2)$.\\
	Mặt phẳng $(Q)$ có véc-tơ pháp tuyến $\overrightarrow{n}_Q=(1;-1;0)$.\\
	Gọi $\varphi$ là số đo góc giữa mặt phẳng $(P)$ và mặt phẳng $(Q)$, ta có
	$$\cos\varphi =\dfrac{\left|\overrightarrow{n}_P\cdot \overrightarrow{n}_Q\right|}{\left|\overrightarrow{n}_P\right|\cdot \left|\overrightarrow{n}_Q\right|}=\dfrac{\left|2\cdot 1+ (-1)\cdot (-1)+(-2)\cdot 0\right|}{\sqrt{2^2+(-1)^2+(-2)^2}\cdot \sqrt{1^2+(-1)^2+0^2}}=\dfrac{\sqrt{2}}{2}.$$
	Suy ra $\varphi =45^\circ$.
	}
	\end{ex}
	\begin{ex}%[Phan Quốc Trí]%[2H3B3-4]
	Trong KG $Oxyz$, góc giữa đường thẳng $d:\dfrac{x-3}{2}=\dfrac{y+1}{1}=\dfrac{z-3}{1}$ và mặt phẳng $(P):x+2y-z+5=0$ là
	\choice
	{$30^{\circ}$}
	{\True $45^{\circ}$}
	{$60^{\circ}$}
	{$90^{\circ}$}
	\loigiai{
	Gọi $\varphi$ là góc giữa $d$ và $(P)$. Ta có 
	$$\sin \varphi = \dfrac{\left| 2\cdot 1+1\cdot 2+1\cdot (-1)\right|}{\sqrt{2^2+1^2+1^2} \cdot \sqrt{1^2 +2^2+(-1)^2}}=\dfrac{\sqrt{2}}{2} \Rightarrow \varphi = 45^{\circ}.$$
	}
	\end{ex}
	\begin{ex}%[Phan Quốc Trí]%[2H3B3-4]
	Trong KG $Oxyz$, góc giữa đường thẳng $d:\dfrac{x-2}{1}=\dfrac{y+3}{2}=\dfrac{z-1}{1}$ và mặt phẳng $(P):-x+y+2z+5=0$ là
	\choice
	{$30^{\circ}$}
	{\True $45^{\circ}$}
	{$60^{\circ}$}
	{$90^{\circ}$}
	\loigiai{
	Gọi $\varphi$ là góc giữa $d$ và $(P)$. Ta có 
	$$\sin \varphi = \dfrac{\left| 1\cdot (-1)+2\cdot 1+1\cdot 2 \right|}{\sqrt{1^2+2^2+1^2} \cdot \sqrt{(-1)^2 +1^2+2^2}}=\dfrac{1}{2} \Rightarrow \varphi = 30^{\circ}.$$
	}
	\end{ex}
	\begin{ex}%[Phan Quốc Trí]%[2H3B3-4]
	Trong KG $Oxyz$, góc giữa hai đường thẳng $d:\dfrac{x}{1}=\dfrac{y+1}{-1}=\dfrac{z-1}{2}$ và $d': \dfrac{x+1}{2}=\dfrac{y}{1}=\dfrac{z-3}{1}$ là 
	\choice
	{$30^{\circ}$}
	{$45^{\circ}$}
	{\True $60^{\circ}$}
	{$90^{\circ}$}
	\loigiai{
	Gọi $\varphi$ là góc giữa hai đường thẳng $d$ và $d'$. Ta có
	$$\cos \varphi = \dfrac{\left| 1\cdot 2 + 1 \cdot (-1)+2\cdot 1 \right|}{\sqrt{1^2+(-1)^2+2^2}\cdot \sqrt{2^2+1^2+1^2}} = \dfrac{1}{2} \Rightarrow \varphi = 60^{\circ}.$$
	}
	\end{ex}
	\begin{ex}%[Phan Quốc Trí]%[2H3B3-4]
	Trong KG $Oxyz$, góc giữa hai đường thẳng $d:\heva{&x=1-t\\&y=t \\&z=0}$ và \break $d': \dfrac{x}{-2}=\dfrac{y}{1}=\dfrac{z-1}{-2}$ là 
	\choice
	{$30^{\circ}$}
	{\True $45^{\circ}$}
	{$60^{\circ}$}
	{$90^{\circ}$}
	\loigiai{
	Gọi $\varphi$ là góc giữa hai đường thẳng $d$ và $d'$. Ta có
	$$\cos \varphi = \dfrac{\left| (-1)\cdot (-2) + 1 \cdot 1+0\cdot (-2) \right|}{\sqrt{(-1)^2+1^2+0^2}\cdot \sqrt{(-2)^2+1^2+(-2)^2}} = \dfrac{\sqrt{2}}{2} \Rightarrow \varphi = 45^{\circ}.$$
	}
	\end{ex}
	\begin{ex}%[HK2, THTH ĐHSP Tp. HCM, 2018]%[2H3B3-4]%[Vinhhop Tran, 12EX-8]
	Trong không gian tọa độ $Oxyz,$ cho đường thẳng $\Delta\colon \dfrac{x-1}{-2}=\dfrac{y+1}2=\dfrac{z-2}{-1}$ và mặt phẳng $(P)\colon 2x-y-2z+1=0.$ Gọi $\alpha$ là góc giữa đường thẳng $\Delta$ và mặt phẳng $(P).$ Khẳng định nào sau đây đúng?
	\choice
	{$\cos \alpha=\dfrac49$}
	{$\cos \alpha=-\dfrac49$}
	{\True $\sin \alpha=\dfrac49$}
	{$\sin \alpha=-\dfrac49$}
	\loigiai{$\Delta$ có một véc-tơ chỉ phương là $\vec{u}=(-2; 2; -1)$, $(P)$ có một véc-tơ pháp tuyến là $\vec{n}=(2; -1; -2).$ Từ đó, $\sin \alpha=\left|\cos (\vec{u}, \vec{n})\right|=\left|\dfrac{\vec{u}\cdot \vec{n}}{|\vec{u}||\vec{n}|}\right|=\dfrac49.$
	}
	\end{ex}
	\begin{ex}%[HK2, THTH ĐHSP Tp. HCM, 2018]%[2H3B3-4]%[Vinhhop Tran, 12EX-8]
	Trong không gian tọa độ $Oxyz,$ cho đường thẳng $\Delta\colon \dfrac{x-1}{-2}=\dfrac{y+1}1=\dfrac{z-2}{3}$ và mặt phẳng $(\alpha)\colon 4x-2y-6z+5=0.$ Khẳng định nào sau đây đúng?
	\choice
	{$\Delta$ song song với $(\alpha)$}
	{$\Delta$ nằm trên $(\alpha)$}
	{\True $\Delta$ vuông góc với $(\alpha)$}
	{$\Delta$ cắt và không vuông góc với $(\alpha)$}
	\loigiai{$\Delta$ có một véc-tơ chỉ phương là $\vec{u}=(-2; 1; 3)$, $(\alpha)$ có một véc-tơ pháp tuyến là $\vec{n}=(2; -1; -3).$ Dễ thấy $\vec u=-\vec n$ nên $\Delta$ vuông góc với $(\alpha).$
	}
	\end{ex}
	\begin{ex}%[TT, Chuyên Lê Quý Đôn, Lai Châu, 2018]%[2H3B3-4]%[Nguyễn Tiến Thùy, 12EX-8]
	Trong KG $Oxyz$, cho đường thẳng $d: \heva{&x=5+t\\ &y=-2+t\\ &z=4+\sqrt{2}t},\, (t\in\mathbb{R})$ và mặt phẳng $(P): x-y+\sqrt{2}z-7=0$. Hãy xác định góc giữa đường thẳng $d$ và mặt phẳng $(P)$.
	\choice
	{$90^\circ$}
	{$45^\circ$}
	{\True $30^\circ$}
	{$60^\circ$}
	\loigiai{
	Đường thẳng $d$ có véc-tơ chỉ phương $\overrightarrow{u}(1;1;\sqrt{2})$.\\
	Mặt phẳng $(P)$ có véc-tơ pháp tuyến $\overrightarrow{n}(1;-1;\sqrt{2})$.\\
	Gọi $\varphi$ là góc giữa đường thẳng và mặt phẳng, khi đó ta có
	$$\sin\varphi=\dfrac{|\overrightarrow{u}\cdot\overrightarrow{n}|}{|\overrightarrow{u}|\cdot|\overrightarrow{n}|}=\dfrac{|1\cdot1+1\cdot(-1)+\sqrt{2}\cdot\sqrt{2}|}{\sqrt{1^2+1^2+\sqrt{2}^2}\cdot\sqrt{1^2+(-1)^2+\sqrt{2}^2}}=\dfrac{1}{2}.$$
	Từ đó suy ra $\varphi=30^\circ$.
	}
	\end{ex}
	\begin{ex}%[HK2 (2017-2018), THPT Chuyên Lê Hồng Phong, TP.HCM]%[Huỳnh Xuân Tín dự án (EX9)]%[2H3B3-4]
	Trong không gian với hệ trục tọa độ $Oxyz$, cho mặt phẳng $(P) \colon x-y+2z+1=0$ và đường thẳng $d: \dfrac{x-1}{1}=\dfrac{y}{2}=\dfrac{z+1}{-1}$. Tính góc giữa đường thẳng $d$ và mặt phẳng $(P)$.
	\choice
	{$60^{\circ} $}
	{$120^{\circ} $}
	{$150^{\circ} $}
	{\True $30^{\circ} $}
	\loigiai{
	Ta có $\vec{u}_{d}=(1;2;-1)$ và $\vec{n}_{(P)}=(1;-1;2)$.\\
	Do đó $\cos(\vec{u}_d;\vec{n}_{(P)})=\dfrac{|1-2-2|}{\sqrt{6} \cdot \sqrt{6}}=\dfrac{1}{2}$, suy ra góc giữa đường thẳng $d$ và mặt phẳng $(P)$ bằng $90^{\circ}-60^{\circ}=30^{\circ}$.
	}
	\end{ex}
	\begin{ex}%[Đề HK2-2018, Marie Curie-HCM]%[Võ Đức Trí - EX9]%[2H3B3-4]
	Trong KG $Oxyz$, mặt phẳng $(P) \colon 3x+4y+5z+8=0$ và đường thẳng $d$ là giao tuyến của hai mặt phẳng $(\alpha) \colon x-2y+1=0$, $(\beta) \colon x-2z-3=0$. Góc giữa $d$ và $(P)$ bằng
	\choice
	{$45^\circ$}
	{$90^\circ$}
	{$30^\circ$}
	{\True $60^\circ$}
	\loigiai{
	$(P)$, $(\alpha)$, $(\beta)$ có véc-tơ pháp tuyến lần lượt là $\overrightarrow{n}_P=(3;4;5)$, $\overrightarrow{n}_\alpha = (1;-2;0)$, $\overrightarrow{n}_\beta=(1;0;-2)$. Véc-tơ chỉ phương của $d$ là $\overrightarrow{u}=\left[\overrightarrow{n}_\alpha, \overrightarrow{n}_\beta\right] = (4;2;2)$. Gọi $\varphi$ là góc giữa $d$ và $(P)$, ta có:
	$$\sin \varphi = \dfrac{\left| \overrightarrow{n}_P\cdot \overrightarrow{u} \right|}{\left| \overrightarrow{n}_P\right| \cdot \left| \overrightarrow{u} \right|}=\dfrac{\sqrt{3}}{2} \Rightarrow \varphi=60^\circ.$$
	}
	\end{ex}
	\begin{ex}%[2H3B3-4]%[Đề thi thử L2, THPT Nguyễn Quang Diêu, 2018]%[Đỗ Đường Hiếu, 12EX-9]
	Trong không gian với hệ trục tọa độ $Oxyz$, cho điểm $H(2;-1;-2)$ là hình chiếu vuông góc của gốc tọa độ $O$ xuống mặt phẳng $(P)$, số đo góc giữa mặt phẳng $(P)$ và mặt phẳng $(Q)\colon x-y-11=0$ bằng bao nhiêu?
	\choice
	{\True $45^\circ$}
	{$30^\circ$}
	{$90^\circ$}
	{$60^\circ$}
	\loigiai{
	Vì $H(2;-1;-2)$ là hình chiếu vuông góc của gốc tọa độ $O$ xuống mặt phẳng $(P)$ nên mặt phẳng $(P)$ có véc-tơ pháp tuyến $\overrightarrow{n}_P=\overrightarrow{OH}=(2;-1;-2)$.\\
	Mặt phẳng $(Q)$ có véc-tơ pháp tuyến $\overrightarrow{n}_Q=(1;-1;0)$.\\
	Gọi $\varphi$ là số đo góc giữa mặt phẳng $(P)$ và mặt phẳng $(Q)$, ta có
	$$\cos\varphi =\dfrac{\left|\overrightarrow{n}_P\cdot \overrightarrow{n}_Q\right|}{\left|\overrightarrow{n}_P\right|\cdot \left|\overrightarrow{n}_Q\right|}=\dfrac{\left|2\cdot 1+ (-1)\cdot (-1)+(-2)\cdot 0\right|}{\sqrt{2^2+(-1)^2+(-2)^2}\cdot \sqrt{1^2+(-1)^2+0^2}}=\dfrac{\sqrt{2}}{2}.$$
	Suy ra $\varphi =45^\circ$.
	}
	\end{ex}
	\begin{dang}{Khoảng cách}%dang 5
	\begin{enumerate}
	\item Cho đường thẳng $d$ có véc-tơ chỉ phương $\vec{u}$, đi qua điểm $M_0$ và điểm $ M $. Khoảng cách từ điểm $ M $ đến đường thẳng $ d $:
	$$ d\left(M,d\right)=\dfrac{|\left[\vec{u};\vec{M_0M}\right]|}{|\vec{u}|}$$
	\item Cho hai đường thẳng chéo nhau $d_1$ có véc-tơ chỉ phương $\vec{u}$, đi qua điểm $M_1 $ và $d_2$ có véc-tơ chỉ phương $\vec{v}$, đi qua điểm $M_2$. Khoảng cách giữa $ d_1 $ và $ d_2 $:
	$$\left(d_1,d_2\right)=\dfrac{\left|\left[\vec{u};\vec{v}\right]\cdot\vec{M_1M_2}\right|}{|\left[\vec{u};\vec{v}\right]|}$$
	\end{enumerate}
\end{dang}
\setcounter{subsubsection}{0}
\setcounter{vd}{0}
\setcounter{ex}{0}
\subsubsection{Ví dụ minh hoạ}
\begin{vd}%[Phan Quốc Trí]%[2H3B3-5]
	Trong KG $Oxyz$, tính khoảng cách từ điểm $A(1;-2;3)$ đến đường thẳng $(\Delta )\colon \dfrac{x-10}{5}=\dfrac{y-2}{1}=\dfrac{z+2}{1}$.
	\loigiai{
	Đường thẳng $\Delta$ đi qua điểm $M(10;2;-2)$ và có véc-tơ chỉ phương $\overrightarrow{u}=(5;1;1)$.\\ Ta có $\overrightarrow{MA}=(-9;-4;5)$ và $\left[ \overrightarrow{MA},\overrightarrow{u} \right]=(-9;34;11)$, do đó
	$$\mathrm{d}\left(M, \Delta \right) = \dfrac{\left| \left[ \overrightarrow{MA},\overrightarrow{u} \right] \right|}{\left| \overrightarrow{u} \right|} = \sqrt{\dfrac{1358}{27}}.$$
	}
\end{vd}
\begin{vd}%[Phan Quốc Trí]%[2H3B3-5]
	Trong KG $Oxyz$, khoảng cách từ điểm $N(-2;1;-1)$ đến đường thẳng $\Delta \colon \dfrac{x-1}{1}=\dfrac{y-2}{2}=\dfrac{z+2}{-2}$ bằng bao nhiêu?
	\loigiai{
	Đường thẳng $\Delta$ đi qua điểm $M(1;2;-2)$ và có véc-tơ chỉ phương $\overrightarrow{u}=(1;2;-2)$.\\ Ta có $\overrightarrow{MN}=(-3;-1;1)$ và $\left[ \overrightarrow{MN},\overrightarrow{u} \right]=(0;-5;5)$, do đó
	$$\mathrm{d}\left(N, \Delta \right) = \dfrac{\left| \left[ \overrightarrow{MN},\overrightarrow{u} \right] \right|}{\left| \overrightarrow{u} \right|} = \dfrac{5\sqrt{2}}{3}.$$
	}
\end{vd}
\begin{vd}%[Phan Quốc Trí]%[2H3B3-5]
	Trong KG $Oxyz$, cho hai đường thẳng $\Delta_1 \colon \dfrac{x-1}{1}=\dfrac{y-2}{2}=\dfrac{z-3}{3}$ và $\Delta_2 \colon \heva{&x=2-t\\&y=-1+t\\&z=t}$. Tính khoảng cách giữa hai đường thẳng $\Delta_1$ và $\Delta_2 $.
	\loigiai{
	\begin{itemize}
	\item $\Delta_1$ có véc-tơ chỉ phương $\overrightarrow{u}=(1;2;3)$ và đi qua điểm $M_0(1;2;3)$
	\item $\Delta_2$ có véc-tơ chỉ phương $\overrightarrow{u'}=(-1;1;1)$ và đi qua điểm $M'_0(2;-1;0)$
	\end{itemize}
	Ta có $\overrightarrow{M_0M'_0} = (1;-3;-3)$ và $\left[ \overrightarrow{u},\overrightarrow{u'} \right]=(-1;-4;3)$. Khoảng cách giữa $\Delta_1$ và $\Delta_2$ là $$\mathrm{d}(\Delta_1; \Delta_2)=\dfrac{\left|\left[ \overrightarrow{u},\overrightarrow{u'} \right]\cdot \overrightarrow{M_0M'_0} \right|}{\left|\left[ \overrightarrow{u},\overrightarrow{u'} \right] \right|}=\dfrac{\left| 1\cdot(-1)-3\cdot(-4)-3\cdot 3 \right|}{\sqrt{(-1)^2+(-4)^2+3^2}}=\dfrac{\sqrt{26}}{13}.$$
	}
\end{vd}
\begin{vd}%[Đề thi học kì II, Ngô Quyền-Quảng Ninh, 2018]%[Trần Ngọc Minh, dự án 12EX-9-2018]%[2H3B3-5]
	Trong KG $Oxyz$, cho $A(1;3;-2)$ và $B(3;5;-12)$. Đường thẳng $AB$ cắt mặt phẳng $Oyz$ tại $N$. Tính tỉ số $\dfrac{BN}{AN}$.
	\loigiai{ 
	Ta có $\dfrac{BN}{AN}=\dfrac{d(B,Oyz)}{d(A,Oyz)}=\dfrac{3}{1}=3.$
	} 
\end{vd} 
\subsubsection{Bài tập trắc nghiệm}
\begin{ex}%[Đề Thi thử lần 3,THTT năm 2018]%[Đỗ Đường Hiếu, 12-EX-5(ID6)]%[2H3B3-5]
	Trong không gian với hệ trục tọa độ $Oxyz$, tính khoảng cách từ điểm $M(1;3;2)$ đến đường thẳng $\heva{&x=1+t\\&y=1+t\\&z=-t}.$
	\choice
	{$\sqrt{2}$}
	{$2$}
	{\True $2\sqrt{2}$}
	{$3$}
	\loigiai{
	Phương trình mặt phẳng đi qua điểm $M(1;3;2)$ và vuông góc với đường thẳng $\heva{&x=1+t\\&y=1+t\\&z=-t}$ là
	$$ 1\cdot (x-1)+1\cdot (y-3)+(-1)\cdot (z-2)=0 \Leftrightarrow x+y-z-2=0. $$
	Hình chiếu vuông góc $H$ của $M$ trên đường thẳng đã cho có tọa độ là nghiệm $(x;y;z)$ của hệ
	$$\heva{&x=1+t\\&y=1+t\\&z=-t\\&x+y-z-2=0}\Leftrightarrow \heva{&x=1\\&y=1\\&z=0\\&t=0.} $$
	Như vậy $H=(1;1;0)$, do đó khoảng cách từ điểm $M$ đến đường thẳng đã cho là
	$$ \mathrm{d}=MH=\sqrt{(1-1)^2+(1-3)^2+(0-2)^2}=2\sqrt{2}. $$
	}
\end{ex}
\begin{ex}%[Đề HK2, 2018, THPT Đa Phúc - Hà Nội]%[Nguyễn Hữu Nhân, Dự án EX9]%[2H3B3-5]
	Trong KG $Oxyz$, cho hai đường thẳng $d \colon \heva{&x=1-t \\ &y=t \\&z=-t}, t \in \mathbb{R}$ và $d' \colon \heva{&x=2t' \\ &y=-1+t' \\& z=t'},t' \in \mathbb{R}$. Khoảng cách giữa hai đường thẳng $d$ và $d'$ là
	\choice{\True $\dfrac{1}{\sqrt{14}}$}
	{$\sqrt{7}$}
	{$\sqrt{14}$}
	{$\dfrac{1}{\sqrt{7}}$}
	\loigiai{Đường thẳng $d$ đi qua điểm $A(1;0;0)$ và có véc-tơ chỉ phương $\overrightarrow{u}_d =(-1;1;-1)$. \\
	Đường thẳng $d'$ đi qua điểm $B(0;-1;0)$ và có véc-tơ chỉ phương $\overrightarrow{u}_{d'} = (2;1;1)$. \\
	$\overrightarrow{AB} = (-1;-1;0)$. \\
	$\left[ \overrightarrow{u}_d , \overrightarrow{u}_{d'} \right] =\left( \begin{vmatrix} 1 & -1 \\ 1 & 1 \end{vmatrix}; \begin{vmatrix} -1 & -1 \\ 1 & 2	\end{vmatrix}; \begin{vmatrix} -1 & 1 \\ 2 & 1	\end{vmatrix}\right) =\left( 2; -1; -3\right)$. \\
	Khoảng cách giữa hai đường thẳng $d$ và $d'$ là 
	\begin{align*}
	\mathrm{d}\left( d,d' \right) = \dfrac{\left| \left[ \overrightarrow{u}_d , \overrightarrow{u}_{d'} \right]\cdot \overrightarrow{AB} \right|}{ \left| \left[ \overrightarrow{u}_d , \overrightarrow{u}_{d'} \right] \right| } = \dfrac{\left| 2\cdot (-1) + (-1) \cdot (-1) + (-3)\cdot 0\right|}{\sqrt{2^2 +(-1)^2+(-3)^2}} =\dfrac{1}{\sqrt{14}}.
	\end{align*}
	}
\end{ex}
\begin{ex}%[Đề HK2, 2018, THPT Đa Phúc - Hà Nội]%[Nguyễn Hữu Nhân, Dự án EX9]%[2H3B3-5]
	Trong không gian với hê tọa độ $Oxyz$, cho điểm $A(1;2;3)$. Khoảng cách từ $A$ đến trục $Oy$ bằng
	\choice{$10$}{\True $\sqrt{10}$}{$3$}{$2$}
	\loigiai{Trục $Oy$ có véc-tơ chỉ phương $\overrightarrow{u}=(0;1;0)$. Ta có $\overrightarrow{OA} = (1;2;3)$.\\
	Do đó $\left[ \overrightarrow{OA}, \overrightarrow{u}\right] =\left( \begin{vmatrix}
	2 & 3 \\ 1& 0 \end{vmatrix}; \begin{vmatrix}
	3 & 1 \\ 0 & 0	\end{vmatrix}; \begin{vmatrix}
	1 & 2 \\ 0 & 1	\end{vmatrix} \right) = (-3;0;1).$	\\
	Khoảng cách từ $A(1;2;3)$ đến trục $Oy$ bằng $\dfrac{\left| \left[ \overrightarrow{OA}, \overrightarrow{u}\right] \right|}{\left| \overrightarrow{u} \right|} = \dfrac{\sqrt{(-3)^2+0^2+1^2}}{1} = \sqrt{10}.$}
\end{ex}
\begin{ex}%[HK2, THPT Nguyễn Trãi - Hà Nội, 2018]%[Dương BùiĐức, 12EX9]%[2H3B3-5]
	Trong không gian với hệ tọa độ $ Oxyz $, cho đường thẳng $ d\colon \dfrac{x}{2}=\dfrac{y}{-1}=\dfrac{z+1}{1} $ và mặt phẳng $ (P)\colon x-2y-2z+5=0 $. Điểm $ A $ nào dưới đây thuộc $ d $ và thỏa mãn khoảng cách từ $ A $ đến mặt phẳng $ (P) $ bằng $ 3 $?
	\choice
	{$A(4;-2;1) $}
	{$A(2;-1;0) $}
	{$A(-2;1;-2) $}
	{\True $A(0;0;-1) $}
	\loigiai{
	Vì $ A\in (d) $ nên ta có tọa độ điểm $ A(2a;-a;a-1) $. Khoảng cách từ $ A $ đến $ (P) $ là
	\[ 
	\dfrac{|2a+2a-2(a-1)+5|}{\sqrt{9}}=3\Leftrightarrow |2a+9|=9\Leftrightarrow\hoac{&a=0\\ &a=-\dfrac{9}{2}.}
	\]
	Với $ a=0\Rightarrow A(0;0;-1) $.
	}
\end{ex}
\begin{ex}%[HK2 (2017 - 2018), THPT Gia Định, Hồ Chí Minh]%[Lê Văn Thiện, dự án(12EX-9)]%[2H3B3-5]
	Trong KG $Oxyz$, tính khoảng cách giữa đường thẳng $d\colon \dfrac{x-1}{2}=\dfrac{y+2}{-4}=\dfrac{z-4}{3}$ và trục $Ox$.
	\choice
	{ $1$}
	{$4$}
	{ $3$}
	{\True $2$}
	\loigiai{ 
	Đường thẳng $d$ có có vec-tơ chỉ phương $\overrightarrow{u}_d=(2;-4;3)$ và đi qua điểm $M(1;-2;4)$.\\
	Trục $Ox$ có vec-tơ chỉ phương $\overrightarrow{u}_{Ox}=(1;0;0)$ và đi qua điểm $N(1;0;0)$.\\
	Khoảng cách giữa đường thẳng $d$ và trục $Ox$ là\\
	$\mathrm{d}[d,Ox]=\dfrac{|[\overrightarrow{u}_d, \overrightarrow{u}_{Ox}]\cdot\overrightarrow{MN}|}{|[\overrightarrow{u}_d, \overrightarrow{u}_{Ox}]|} = \dfrac{|(0;3;4)\cdot (0;2;-4)|}{|(0;3;4)|}=2.$	
	}
\end{ex}
\begin{ex}%[HK2 (2017-2018), SGD Lạng Sơn]%[Lê Thanh Nin, dự án ex9]%[2H3B3-5]
	Trong KG $Oxyz$, cho điểm $A(3;2;1)$. Tính khoảng cách từ $A$ đến trục $Oy$.
	\choice
	{$2$}
	{\True $\sqrt{10}$}
	{$3$}
	{$10$}
	\loigiai{
	Hình chiếu của $ A $ lên $ Oy $ là $H(0;2;0)$.
	Vậy khoảng cách từ $A$ đến trục $Oy$ bằng $$AH=\sqrt{(0-3)^2+(2-2)^2+(0-1)^2}=\sqrt{10}.$$
	}
\end{ex}
\begin{ex}%[TT lần 5, Chuyên Thái Bình, 2018]%[Vũ Nguyễn Hoàng Anh, 12EX9]%[2H3B3-5]
	Trong không gian với hệ trục tọa độ $Oxyz$, cho ba điểm $A(2;0;0)$, $B(0;3;1)$, $C(-1;4;2)$. Độ dài đường cao từ đỉnh $A$ của tam giác $ABC$ là
	\choice
	{$\sqrt{6}$}
	{\True $\sqrt{2}$}
	{$\dfrac{\sqrt{3}}{2}$}
	{$\sqrt{3}$}
	\loigiai{
	Ta có $\overrightarrow{CB}=(1;-1;-1)$, $\overrightarrow{BA}=(2;-3;-1)$, suy ra $\left[ \overrightarrow{BA}, \overrightarrow{CB} \right]=(2;1;1)$. Độ dài đường cao từ đỉnh $A$ của tam giác $ABC$ là
	\begin{align*}
	d\left( A,BC \right)= \dfrac{\left| \left[ \overrightarrow{BA}, \overrightarrow{CB} \right] \right|}{\left| \overrightarrow{CB} \right|}=\dfrac{\sqrt{6}}{\sqrt{3}}=\sqrt{2}.
	\end{align*}
	}
\end{ex}
\begin{ex}%[TT-Đặng Thúc Hứa- Nghệ An lần 2 - 2018]%[PhanMinhTâm, Ex10]%[2H3B3-5]
	Trong KG $Oxyz$, khoảng cách $h$ từ điểm $A(-4;3;2)$ đến trục $Ox$ là
	\choice
	{$h=4$}
	{\True $h=\sqrt{13}$}
	{$h=3$}
	{$h=2\sqrt{5}$}
	\loigiai{Ta có $ H(-4;0;0) $ là hình chiếu của điểm $ A(-4;3;2) $ trên trục $ Ox. $\\
	Khoảng cách từ $ A $ đến trục $ Ox $ là $AH=\sqrt{13}$.
	}
\end{ex}
\begin{ex}%[TT L4, chuyen Quang Trung, Bình Phước, 2018]%[Lê Mạnh Thắng, 12EX-10]%[2H3B3-5]
	Trong KG $Oxyz$, cho đường thẳng $d\colon \dfrac{x-1}{1}=\dfrac{y}{-1}=\dfrac{z}{2}$ và điểm $A(1;6;0)$. Tìm giá trị nhỏ nhất của độ dài $MA$ với $M\in d$.
	\choice
	{$5\sqrt{3}$}
	{$6$}
	{$4\sqrt{2}$}
	{\True $\sqrt{30}$}
	\loigiai{
	Ta có $M\in d \;\Rightarrow\; M(1+t;-t;2t) \;\Rightarrow\; \overrightarrow{AM}=(t;-t-6;2t)$. Khi đó,\vspace*{-6pt}
	\begin{center}
	$MA=\sqrt{t^2+(t+6)^2+(2t)^2}=\sqrt{6t^2+12t+36}=\sqrt{6(t+1)^2+30}\geqslant \sqrt{30}$.
	\end{center}	
	Suy ra, giá trị nhỏ nhất của $MA$ bằng $\sqrt{30}$ khi $t=-1$ hay là $M(0;1;-2)$.
	}
\end{ex}
\begin{ex}%[Phan Quốc Trí]%[2H3B3-5]
	Trong KG $Oxyz$, tính khoảng cách từ điểm $A(1;-2;3)$ đến đường thẳng $(\Delta )\colon \dfrac{x-10}{5}=\dfrac{y-2}{1}=\dfrac{z+2}{1}$.
	\choice
	{$\sqrt{\dfrac{1361}{27}}$}
	{$7$}
	{\True $\sqrt{\dfrac{1358}{27}}$}
	{$\dfrac{13}{12}$}
	\loigiai{
	Đường thẳng $\Delta$ đi qua điểm $M(10;2;-2)$ và có véc-tơ chỉ phương $\overrightarrow{u}=(5;1;1)$.\\ Ta có $\overrightarrow{MA}=(-9;-4;5)$ và $\left[ \overrightarrow{MA},\overrightarrow{u} \right]=(-9;34;11)$, do đó
	$$\mathrm{d}\left(M, \Delta \right) = \dfrac{\left| \left[ \overrightarrow{MA},\overrightarrow{u} \right] \right|}{\left| \overrightarrow{u} \right|} = \sqrt{\dfrac{1358}{27}}.$$
	}
\end{ex}
\begin{ex}%[Phan Quốc Trí]%[2H3B3-5]
	Trong KG $Oxyz$, khoảng cách từ điểm $N(-2;1;-1)$ đến đường thẳng $\Delta \colon \dfrac{x-1}{1}=\dfrac{y-2}{2}=\dfrac{z+2}{-2}$ bằng bao nhiêu?
	\choice
	{$\dfrac{5\sqrt{2}}{2}$}
	{$\dfrac{\sqrt{2}}{3}$}
	{\True $\dfrac{5\sqrt{2}}{3}$}
	{$\dfrac{5}{3}$}
	\loigiai{
	Đường thẳng $\Delta$ đi qua điểm $M(1;2;-2)$ và có véc-tơ chỉ phương $\overrightarrow{u}=(1;2;-2)$.\\ Ta có $\overrightarrow{MN}=(-3;-1;1)$ và $\left[ \overrightarrow{MN},\overrightarrow{u} \right]=(0;-5;5)$, do đó
	$$\mathrm{d}\left(N, \Delta \right) = \dfrac{\left| \left[ \overrightarrow{MN},\overrightarrow{u} \right] \right|}{\left| \overrightarrow{u} \right|} = \dfrac{5\sqrt{2}}{3}.$$
	}
\end{ex}
\begin{ex}%[Đề HK2, 2018, THPT Đa Phúc - Hà Nội]%[Nguyễn Hữu Nhân, Dự án EX9]%[2H3B3-5]
	Trong KG $Oxyz$, cho hai đường thẳng $d \colon \heva{&x=1-t \\ &y=t \\&z=-t}, t \in \mathbb{R}$ và $d' \colon \heva{&x=2t' \\ &y=-1+t' \\& z=t'},t' \in \mathbb{R}$. Khoảng cách giữa hai đường thẳng $d$ và $d'$ là
	\choice{\True $\dfrac{1}{\sqrt{14}}$}
	{$\sqrt{7}$}
	{$\sqrt{14}$}
	{$\dfrac{1}{\sqrt{7}}$}
	\loigiai{Đường thẳng $d$ đi qua điểm $A(1;0;0)$ và có véc-tơ chỉ phương $\overrightarrow{u}_d =(-1;1;-1)$. \\
	Đường thẳng $d'$ đi qua điểm $B(0;-1;0)$ và có véc-tơ chỉ phương $\overrightarrow{u}_{d'} = (2;1;1)$. \\
	$\overrightarrow{AB} = (-1;-1;0)$. \\
	$\left[ \overrightarrow{u}_d , \overrightarrow{u}_{d'} \right] =\left( \begin{vmatrix} 1 & -1 \\ 1 & 1 \end{vmatrix}; \begin{vmatrix} -1 & -1 \\ 1 & 2	\end{vmatrix}; \begin{vmatrix} -1 & 1 \\ 2 & 1	\end{vmatrix}\right) =\left( 2; -1; -3\right)$. \\
	Khoảng cách giữa hai đường thẳng $d$ và $d'$ là 
	\begin{align*}
	\mathrm{d}\left( d,d' \right) = \dfrac{\left| \left[ \overrightarrow{u}_d , \overrightarrow{u}_{d'} \right]\cdot \overrightarrow{AB} \right|}{ \left| \left[ \overrightarrow{u}_d , \overrightarrow{u}_{d'} \right] \right| } = \dfrac{\left| 2\cdot (-1) + (-1) \cdot (-1) + (-3)\cdot 0\right|}{\sqrt{2^2 +(-1)^2+(-3)^2}} =\dfrac{1}{\sqrt{14}}.
	\end{align*}
	}
\end{ex}
\begin{ex}%[Đề HK2, 2018, THPT Đa Phúc - Hà Nội]%[Nguyễn Hữu Nhân, Dự án EX9]%[2H3B3-5]
	Trong không gian với hê tọa độ $Oxyz$, cho điểm $A(1;2;3)$. Khoảng cách từ $A$ đến trục $Oy$ bằng
	\choice{$10$}{\True $\sqrt{10}$}{$3$}{$2$}
	\loigiai{Trục $Oy$ có véc-tơ chỉ phương $\overrightarrow{u}=(0;1;0)$. Ta có $\overrightarrow{OA} = (1;2;3)$.\\
	Do đó $\left[ \overrightarrow{OA}, \overrightarrow{u}\right] =\left( \begin{vmatrix}
	2 & 3 \\ 1& 0 \end{vmatrix}; \begin{vmatrix}
	3 & 1 \\ 0 & 0	\end{vmatrix}; \begin{vmatrix}
	1 & 2 \\ 0 & 1	\end{vmatrix} \right) = (-3;0;1).$	\\
	Khoảng cách từ $A(1;2;3)$ đến trục $Oy$ bằng $\dfrac{\left| \left[ \overrightarrow{OA}, \overrightarrow{u}\right] \right|}{\left| \overrightarrow{u} \right|} = \dfrac{\sqrt{(-3)^2+0^2+1^2}}{1} = \sqrt{10}.$}
\end{ex}
\begin{ex}%[HK2 (2017 - 2018), THPT Gia Định, Hồ Chí Minh]%[Lê Văn Thiện, dự án(12EX-9)]%[2H3B3-5]
	Trong KG $Oxyz$, tính khoảng cách giữa đường thẳng $d\colon \dfrac{x-1}{2}=\dfrac{y+2}{-4}=\dfrac{z-4}{3}$ và trục $Ox$.
	\choice
	{ $1$}
	{$4$}
	{ $3$}
	{\True $2$}
	\loigiai{ 
	Đường thẳng $d$ có có vec-tơ chỉ phương $\overrightarrow{u}_d=(2;-4;3)$ và đi qua điểm $M(1;-2;4)$.\\
	Trục $Ox$ có vec-tơ chỉ phương $\overrightarrow{u}_{Ox}=(1;0;0)$ và đi qua điểm $N(1;0;0)$.\\
	Khoảng cách giữa đường thẳng $d$ và trục $Ox$ là\\
	$\mathrm{d}[d,Ox]=\dfrac{|[\overrightarrow{u}_d, \overrightarrow{u}_{Ox}]\cdot\overrightarrow{MN}|}{|[\overrightarrow{u}_d, \overrightarrow{u}_{Ox}]|} = \dfrac{|(0;3;4)\cdot (0;2;-4)|}{|(0;3;4)|}=2.$	
	}
\end{ex}
\begin{ex}%[HK2 (2017-2018), SGD Lạng Sơn]%[Lê Thanh Nin, dự án ex9]%[2H3B3-5]
	Trong KG $Oxyz$, cho điểm $A(3;2;1)$. Tính khoảng cách từ $A$ đến trục $Oy$.
	\choice
	{$2$}
	{\True $\sqrt{10}$}
	{$3$}
	{$10$}
	\loigiai{
	Hình chiếu của $ A $ lên $ Oy $ là $H(0;2;0)$.
	Vậy khoảng cách từ $A$ đến trục $Oy$ bằng $$AH=\sqrt{(0-3)^2+(2-2)^2+(0-1)^2}=\sqrt{10}.$$
	}
\end{ex}
\begin{dang}{Vị trí tương đối giữa hai đường thẳng, giữa đường thẳng và mặt phẳng }%Dạng 6
	Để xét vị trí tương đối của hai đường thẳng $d_1$ và $d_2$ , ta thực hiện theo các bước:
	\begin{enumerate}
	\item
	\begin{itemize} 
	\item Với đường thẳng $d_1$ chỉ ra vtcp $\vec{a}=(a_1;a_2;a_3)$ và điểm $M_1 \in d_1$.
	\item Với đường thẳng $d_2$ chỉ ra vtcp $\vec{b}=(b_1;b_2;b_3)$ và điểm $M_2 \in d_2$.
	\end{itemize}
	\item 	Kiểm tra:
	\begin{itemize}
	\item Nếu $\vec{a}$, $\vec{b}$, $\vec{M_1M_2}$ cùng phương thì kết luận $d_1$ và $d_2$ trùng nhau.
	\item Nếu $\vec{a}$, $\vec{b}$ cùng phương và không cùng phương với $\vec{M_1M_2}$ thì kết luận $d_1$ và $d_2$ song song với nhau.
	\item Nếu $\vec{a}$, $\vec{b}$ không cùng phương, thực hiện bước 3.
	\end{itemize}
	\item 	Xác định $[\vec{a}, \vec{b}]\cdot\vec{M_1M_2} $, khi đó:
	\begin{itemize}
	\item Nếu $[\vec{a}, \vec{b}]\cdot\vec{M_1M_2}=0 $ thì kết luận $d_1$ và $d_2$ cắt nhau.
	\item Nếu $[\vec{a}, \vec{b}]\cdot\vec{M_1M_2} \neq 0$ thì kết luận $d_1$ và $d_2$ chéo nhau.
	\end{itemize}
	\end{enumerate}
	\end{dang}
\setcounter{subsubsection}{0}
\setcounter{vd}{0}
\setcounter{ex}{0}
	\subsubsection{Ví dụ minh hoạ}
	\begin{vd}%[Đề thi thử QG lần 1- Sở Bình Phước -2018]%[2H3B3-6]%[Trịnh Văn Xuân -Ex-8]
	Trong không gian với hệ trục tọa độ $Oxyz$, cho hai đường thẳng $(\Delta_1)\colon\heva{
	& x=-3+2t \\ 
	& y=1-t \\ 
	& z=-1+4t \\ }$ và $(\Delta_2)\colon \dfrac{x+4}{3}=\dfrac{y+2}{2}=\dfrac{z-4}{-1}$. Xét vị trí tương đối của $ d_1 $ và $ d_2 $.
	\loigiai{
	PTTS của $(\Delta_2):\heva{
	& x=-4+3{t}' \\ 
	& y=-2+2{t}' \\ 
	& z=4-{t}' \\ }$.\\
	Véc-tơ chỉ phương của $(\Delta_1)$ và $(\Delta_2)$ lần lượt là $\overrightarrow{u_1}=(2;-1;4)$ và $\overrightarrow{u_2}=(3;2;-1)$.\\
	Do $\overrightarrow{u_1}\cdot\overrightarrow{u_2}=2\cdot3+(-1)\cdot2+4\cdot(-1)=0$ nên $(\Delta_1)\perp (\Delta_2)$.\\
	Xét hệ phương trình $\heva{
	&-3+2t=-4+3{t}' \\ 
	& 1-t=-2+2{t}' \\ 
	&-1+4t=4-{t}' \\ }\Leftrightarrow \heva{
	& 2t-3{t}'=-1 \\ 
	& t+2{t}'=3 \\ 
	& 4t+{t}'=5 \\ }\Leftrightarrow \heva{
	& t=1 \\ 
	& {t}'=1 \\ }$.\\
	Vậy $(\Delta_1)$ cắt và vuông góc với $(\Delta_2)$.}
	\end{vd}
	\begin{vd}%[Thi thử THPT QG 2018, Sở GD Hà Tĩnh 2018]%[Lê Mạnh Thắng, 12EX-9]%[2H3B3-6]
	Trong KG $Oxyz$, cho đường thẳng $d\colon \dfrac{x+1}{1}=\dfrac{y}{-1}=\dfrac{z-1}{-3}$ và mặt phẳng $(P)\colon 3x-3y+2z+1=0$. Xét vị trí tương đối của $ d$ và $(P) $.
	\loigiai{
	Ta có $d$ có 1 véc-tơ chỉ phương là $\overrightarrow{u}_d=(1;-1;-3)$ và $(P)$ có 1 véc-tơ pháp tuyến là $\overrightarrow{n}_P=(3;-3;2)$.\\
	Nhận thấy $\overrightarrow{u}_d\cdot \overrightarrow{n}_P=1\cdot 3+(-1)\cdot (-3)+(-3)\cdot 2 = 0$ $\Rightarrow d \parallel (P)$ hoặc $d \subset (P)$.\\
	Lấy $A(-1;0;1)\in d$. Thay vào phương trình của $(P)$ ta được $3\cdot (-1)-3\cdot 0+2\cdot 1+1=0$\\
	$\Rightarrow A\in (P)$. Suy ra $d$ nằm trong $(P)$.\\
	\textbf{Cách khác}.\\
	Viết lại đường thẳng $d$ ở dạng tham số $\heva{&x=-1+t\\&y=-t\\&z=1-3t}$.\\
	Xét phương trình $3\cdot (-1+t)-3\cdot (-t)+2\cdot (1-3t)+1=0 \Leftrightarrow 0=0$. Kết luận phương trình có vô số nghiệm $\Rightarrow d \subset (P)$.
	}
	\end{vd}
	\begin{vd}%[Đề thi học kì II, Ngô Quyền-Quảng Ninh, 2018]%[Trần Ngọc Minh, dự án 12EX-9-2018]%[2H3B3-6]
	Trong KG $Oxyz$, cho hai đường thẳng $d: \left \lbrace \begin{aligned} &x=1+mt\\ &y=t \\&z=-1+2t \end{aligned} \right. (t \in \mathbb{R})$ và $d': \left \lbrace \begin{aligned} &x=1-t'\\ &y=2+2t'\\&z=3-t' \end{aligned} \right. (t' \in \mathbb{R})$. Giá trị của $m$ để hai đường thẳng $d$ và $d'$ cắt nhau là
	\loigiai{ 
	Đường thẳng $d$ đi qua $A(1;0;-1)$, có véc-tơ chỉ phương $\overrightarrow{u_1}=(m;1;2)$.\\
	Đường thẳng $d'$ đi qua $B(1;2;3)$, có véc-tơ chỉ phương $\overrightarrow{u_2}=(-1;2;-1)$.\\ 
	Ta có $[\overrightarrow{u_1}, \overrightarrow{u_2}]=(-5;m-2;2m+1)$ và $\overrightarrow{AB}=(0;2;4)$.\\
	Hai đường thẳng $d$ và $d'$ cắt nhau $\Leftrightarrow [\overrightarrow{u_1}, \overrightarrow{u_2}]\cdot AB=0\Leftrightarrow m=0$. 
	} 
	\end{vd}
	\begin{ex}%[Thi HK2, Sở GD Bình Dương, 2018]%[2H3B3-6]%[Trần Bá Huy, 12EX-8-2018]
	Trong KG $Oxyz$, cho mặt phẳng $(P)\colon 3x+5y-z-2=0$ và đường thẳng $d\colon \dfrac{x-12}{4}=\dfrac{y-9}{3}=\dfrac{z-1}{1}$. Tọa độ giao điểm $M$ của $d$ và $(P)$ là
	\loigiai{
	PTTS của đường thẳng $d\colon\heva{& x=12+4t\\ & y=9+3t\\ & z=1+t}$.\\
	Tọa độ giao điểm $M$ của $d$ và $(P)$ là nghiệm của hệ
	{\allowdisplaybreaks
	\begin{align*}
	& \heva{& x=12+4t\\ & y=9+3t\\ & z=1+t\\ & 3x+5y-z-2=0}\\
	\Rightarrow & 3(12+4t)+5(9+3t)-(1+t)-2=0\Leftrightarrow 26t=-78\Leftrightarrow t=-3\\
	\Rightarrow & \heva{& x=0\\ & y=0\\ & z=-2.}
	\end{align*}}
	Vậy tọa độ điểm $M(0;0;-2)$.
	}
	\end{ex}
	\begin{vd}%[Đề HK2 Khối 12, Lý Thái Tổ, Bắc Ninh 2018]%[Nguyễn Thị Kiều Ngân, dự án 12EX9]%[2H3B3-6]
	Trong KG $Oxyz$, cho hai điểm $A(10;2;-2)$ và $B(5;1;-3)$. Tìm tất cả các giá trị của tham số $m$ để đường thẳng $AB$ vuông góc với mặt phẳng $(P) \colon 10x+2y+mz+11=0$.
	\loigiai{
	Ta có $\vv{AB}=(-5;-1;-1)$.\\
	Mặt phẳng $(P) \colon 10x+2y+mz+11=0$ có véc-tơ pháp tuyến là $\vv{n} =(10;2;m)$.\\
	Đường thẳng $AB$ vuông góc với mặt phẳng $(P)$ khi và chỉ khi $\vv{n}$ cùng phương với $\vv{AB}$\\
	$\Rightarrow \dfrac{10}{-5} =\dfrac{2}{-1} =\dfrac{m}{-1} \Rightarrow m=2$.
	}
	\end{vd}
	\begin{vd}%[KSCL Lớp 12, Phả Lại - Hải Dương - 2018]%[Vũ Văn Trường, dự án(12EX-9)]%[2H3B3-6]
	Trong KG $Oxyz$, cho điểm $M(0;-3;1)$ và đường thẳng $d\colon\heva{& x=-1+3t \\& y=1-2t \\& z=3+t}$. Mặt phẳng $(P)$ đi qua điểm $M$ và vuông góc với đường thẳng $d$ có phương trình
	\loigiai{$d$ có véc-tơ chỉ phương $\overrightarrow{u}=(3;-2;1)$.\\
	Mặt phẳng $(P)$ qua điểm $M(0;-3;1)$ và có véc-tơ pháp tuyến $\overrightarrow{n}=\overrightarrow{u}$ nên có phương trình:\\
	$3(x-0)-2(y+3)+1(z-1)=0$ hay $3x-2y+z-7=0$.
	}
	\end{vd}
	\subsubsection{Bài tập trắc nghiệm}
	\setcounter{ex}{0}
	\begin{ex}%[TT, Đại Học Ngoại Thương - Hà Nội, 2018]%[2H3B3-6]%[Dương BùiĐức, 12EX-7]
	Trong KG $Oxyz$, cho hai đường thẳng $d:\dfrac{x+1}{2}=\dfrac{1-y}{-m}=\dfrac{2-z}{-3}$ và $d_{1}:\dfrac{x-3}{1}=\dfrac{y}{1}=\dfrac{z-1}{1}$. Tìm tất cả các giá trị của $m$ để $d\perp d_{1}$.
	\choice
	{$m=-1$}
	{$m=1$}
	{\True $m=-5$}
	{$m=5$}
	\loigiai{
	Véc-tơ chỉ phương của $d$ và $d_{1}$ lần lượt là $(2;m;3)$ và $(1;1;1)$.\\
	Để $d\perp d_{1}$ thì $2+m+3=0\Rightarrow m=-5$.
	}
	\end{ex}
	\begin{ex}%[TT L2, Tây Thụy Anh Thái Bình, 2018]%[2H3B3-6]%[Học Toán, (12EX-8)]
	Trong KG $Oxyz$ cho đường thẳng $d\colon \dfrac{x-1}{2}=\dfrac{y+2}{-1}=\dfrac{z+1}{1}.$ Trong các mặt phẳng dưới đây mặt phẳng nào vuông góc với đường thẳng $d$?
	\choice
	{\True $4x-2y+2z+4=0$}
	{$4x+2y+2z+4=0$}
	{$2x-2y+2z+4=0$}
	{$4x-2y-2z-4=0$}
	\loigiai{
	Đường thẳng $d$ có vec-tơ chỉ phương $\overrightarrow{u}=(2;-1;1)$.\\
	Xét mặt phẳng $4x-2y+2z+4=0$ có vec-tơ pháp tuyến $\overrightarrow{n}=(4;-2;2)\Rightarrow\overrightarrow{n}=2\overrightarrow{v}.$\\
	$\Rightarrow d$ vuông góc với mặt phẳng có phương trình $4x-2y+2z+4=0$.
	}
	\end{ex}
	\begin{ex}%[Thi thử Lần 1, Thanh Chương 3 Nghệ An, 2018]%[2H3B3-6]%[Đỗ Viết Lân, 12EX8]
	Trong KG $Oxyz$, viết phương trình mặt phẳng $(P)$ đi qua hai điểm $A(2;1;3)$, $B(1;-2;1)$ và song song với đường thẳng $d\colon \heva{&x=-1+t\\&y=2t\\&z=-3-2t}.$
	\choice
	{$2x+y+3z+19=0$}
	{\True $10x-4y+z-19=0$}
	{$2x+y+3z-19=0$}
	{$10x-4y+z+19=0$}
	\loigiai{
	$\overrightarrow{AB} = (-1;-3;-2)$ và $\vec{u}_d = (1;2;-2)$.\\
	Do $AB$ nằm trong $(P)$ và $d$ song song với $(P)$ nên $\vec{n}_{(P)} = \left[\overrightarrow{AB},\vec{u}_d \right] = (10;-4;1)$.\\
	Từ đó $(P)\colon 10(x-2)-4(y-1)+(z-3)=0\Leftrightarrow 10x-4y+z-19=0$.
	}
	\end{ex}
	\begin{ex} %[Đề KSCL học kỳ 2 Toán 12 năm học 2017 – 2018 sở GD và ĐT Nam Định] %[Trần Tuấn Việt, 12EX-9-2018]%[2H3B3-6]
	Trong KG $Oxyz$, cho ba điểm $A(1; 2; -1)$, $B(-3; 4; 3)$, $C(3; 1; -3)$. Số điểm $D$ sao cho $4$ điểm $A$, $B$, $C$, $D$ là $4$ đỉnh của một hình bình hành là
	\choice
	{$3$}
	{$1$}
	{$2$}
	{\True $0$}
	\loigiai{Ta có $\vec{AB} = (-4; 2; 4)$, $\vec{AC} = (2; -1; -2)$. Suy ra $A, B, C$ thẳng hàng. Do đó không có điểm $D$ nào thỏa mãn $A,B,C,D$ là $4$ đỉnh của hình bình hành. 
	}
	\end{ex}
	\begin{ex}%[TT Sở GD Bắc Ninh, 2018]%[Lê Minh An, 12Ex-9]%[2H3B3-6]
	Trong KG $Oxyz$, tìm tất cả giá trị tham số $m$ để đường thẳng $d\colon \dfrac{x-1}{1}=\dfrac{y}{2}=\dfrac{z-1}{1}$ song song với mặt phẳng $(P)\colon 2x+y-m^2z+m=0$.
	\choice{$m\in\{-2;2\}$}
	{$m\in\varnothing$}
	{\True $m=-2$}
	{$m=2$}
	\loigiai{
	$d$ qua điểm $M(1;0;1)$ và có VTCP là $\overrightarrow{u}=(1;2;1)$, $(P)$ có VTPT là $\overrightarrow{n}=(2;1;-m^2)$.\\
	Vì $d\parallel (P)$ nên $\overrightarrow{u}\perp \overrightarrow{n}\Leftrightarrow \overrightarrow{u}\cdot\overrightarrow{n}=0\Leftrightarrow m=\pm 2$.
	\begin{itemize}
	\item Với $m=2$, $(P)\colon 2x+y-4z+2=0\Rightarrow M\in (P)$ (loại).
	\item Với $m=-2$, $(P)\colon 2x+y-4z-2=0\Rightarrow M\notin (P)$ (thỏa mãn).
	\end{itemize}
	}
	\end{ex}
	\begin{ex}%[Đề thi thử THPT Cẩm Xuyên, Hà Tĩnh, lần 2, 2017-2018]%[Nguyễn Bình Nguyên-EX9]%[2H3B3-6]
	Trong KG $Oxyz$, cho mặt phẳng $(\alpha) \colon y+2z-1=0$. Khẳng định nào sau đây \textbf{sai}?
	\choice 
	{ $ (\alpha) \perp (Oyz) $}
	{ $(\alpha)$ cắt $(Oxy) $}
	{ \True $ (\alpha) \perp Ox $}
	{ $(\alpha) \parallel Ox $}
	\loigiai{
	Mặt phẳng $(\alpha)$ có véc-tơ pháp tuyến $ \vec{n} =(0;1;2)$ \\
	Ta thấy véc-tơ $\vec{n}$ và véc-tơ $\vec{i} =(1;0;0)$ không cùng phương với nhau. \\
	Suy ra $ (\alpha) \perp Ox$ không xảy ra. 
	} 
	\end{ex}
	\begin{ex}%[HK2, THPT Chuyên Amsterdam, Hà Nội]%[Bùi Quốc Hoàn, dự án EX9]%[2H3B3-6]
	Trong KG $Oxyz$, cho ba điểm $A\left(2;1;-1\right)$, $B\left(0;- 1; 3\right)$, $C\left(1;2;1\right)$. Mặt phẳng $\left(P\right)$ qua $B$ và vuông góc với $AC$ có phương trình là 
	\choice
	{$x + y + 2z + 5 = 0$}
	{\True $x - y - 2z + 5 = 0$}
	{$x - y + 2z + 5 = 0$}
	{$x + y - 2z + 5 = 0$}
	\loigiai{ Ta có $\overrightarrow{AC}\left(- 1; 1; 2\right)$
	do giả thiết suy ra $\overrightarrow{AC}$ là véc-tơ pháp tuyến của mặt phẳng $\left(P\right)$. Khi đó phương trình của $\left(P\right)$ là 
	$$(-1)\left(x - 0\right) + \left(y + 1\right) + 2\left(z - 3\right)\Leftrightarrow x - y - 2z + 5 = 0$$	
	}
	\end{ex}
	\begin{ex}%[Đề thi học kì II, Ngô Quyền-Quảng Ninh, 2018]%[Trần Ngọc Minh, dự án 12EX-9-2018]%[2H3B3-6]
	Trong KG $Oxyz$, cho hai đường thẳng $d: \dfrac{x-1}{3}=\dfrac{y-2}{4}=\dfrac{z-3}{5}$ và $d':\dfrac{x-4}{6}=\dfrac{y-6}{8}=\dfrac{z-8}{10}$. Mệnh đề nào sau đây là đúng?
	\choice 
	{$d$ vuông góc với $d'$}
	{$d$ song song với $d'$}
	{\True $d$ trùng với $d'$}
	{$d$ và $d'$ chéo nhau} 
	\loigiai{ 
	Đường thẳng $d$ có véc-tơ chỉ phương $\overrightarrow{u_1}=(3;4;5)$ và đi qua điểm $A(1;2;3)$.\\
	Đường thẳng $d'$ có véc-tơ chỉ phương $\overrightarrow{u_2}=(6;8;10)$.\\
	Dễ thấy $\overrightarrow{u_1}$ và $\overrightarrow{u_2}$ cùng phương và điểm $A \in d'$.\\ 
	Vậy $d$ và $d'$ trùng nhau.
	} 
	\end{ex} 
	\begin{ex}%[HK2 (2017-2018), THPT LÊ QUÝ ĐÔN, HÀ NỘI]%[Trần Hòa, dự án EX9]%[2H3B3-6]
	Trong KG $Oxyz$, cho điểm $A(0;1;1)$ và hai đường thẳng $d_1\colon \heva{&x=-1\\&y=-1+t\\&z=t}$ và $d_2\colon \dfrac{x-1}{3}=\dfrac{y-2}{1}=\dfrac{z}{1}$. Gọi $d$ là đường thẳng đi qua điểm $A$, cắt đường thẳng $d_1$ và vuông góc với đường thẳng $d_2$. Đường thẳng $d$ đi qua điểm nào trong các điểm dưới đây?
	\choice%14
	{$N(2;1-5)$}
	{$Q(3;2;5)$}
	{$P(-2;-3;11)$}
	{\True $M(1;0;-1)$}
	\loigiai{
	Gọi $B=d_1\cap d$. $B\in d_1\Rightarrow B(-1;-1+t;t)$. $\vec{AB}=(-1;t-2;t-1)$. $d_2$ có một véc-tơ chỉ phương $\vec{u}=(3;1;1)$. Do $d\perp d_2$ nên $\vec{u}\cdot \vec{AB}=0\Leftrightarrow -3+t-2+t-1=0\Leftrightarrow t=3\Rightarrow \vec{AB}=(-1;1;2)$.\\
	Có $\vec{AN}=(2;0;6)$, $\vec{AQ}=(3;1;4)$, $\vec{AP}=(-2;-4;10)$, $\vec{AM}=(1;-1;-2)$.\\
	Suy ra đường thẳng $d$ đi qua $M$.
	}
	\end{ex}
	\begin{ex}%[Đề HK2 Khối 12, Lý Thái Tổ, Bắc Ninh 2018]%[Nguyễn Thị Kiều Ngân, dự án 12EX9]%[2H3B3-6]
	Trong KG $Oxyz$,cho đường thẳng $d\colon \dfrac{x-2}{-1} =\dfrac{y-8}{1} =\dfrac{z+4}{-1}$ và mặt phẳng $(P)\colon x+y+z-3=0$. Tọa độ giao điểm của đường thẳng $d$ và mặt phẳng $(P)$ là
	\choice
	{$(2;8;-4)$}
	{$(0;10;-7)$}
	{\True $(-1;11;-7)$}
	{$(5;5;-1)$}
	\loigiai{
	Tọa độ giao điểm của đường thẳng $d$ và mặt phẳng $(P)$ thỏa hệ\\
	$\heva{&\dfrac{x-2}{-1} =\dfrac{y-8}{1} =\dfrac{z+4}{-1}\\&x+y+z-3=0}
	\Leftrightarrow \heva{&\dfrac{x-2}{-1} =\dfrac{y-8}{1} \\&\dfrac{y-8}{1} =\dfrac{z+4}{-1}\\&x+y+z-3=0}
	\Leftrightarrow \heva{&x+y=10\\&-y-z=-4\\&x+y+z=3}
	\Leftrightarrow \heva{&x=-1\\&y=11\\&z=-7.}$\\
	Vậy tọa độ giao điểm là $(-1;11;-7)$.
	}
	\end{ex}
	\begin{ex}%[Đề HK2, 2018, Sở Đồng Nai]%[Trần Quang Thạnh, dự án EX9]%[2H3B3-6]
	Trong KG $Oxyz$, cho mặt phẳng $(P) \colon 2x+y+z+3=0$ và đường thẳng $d: \dfrac{x}{2}=\dfrac{y}{1}=\dfrac{z+2}{m}$, với $m\neq 0$. Tìm $m$ để $d$ song song $(P)$.
	\choice
	{$m=5 $}
	{\True $m=-5 $}
	{$m=1 $}
	{$m=-1 $}
	\loigiai{
	Mặt phẳng $(P)$ có véc-tơ pháp tuyến là $\vec{n}=(2;1;1)$ và đường thẳng $d$ có véc-tơ chỉ phương là $\vec{u}=(2;1;m)$.\\
	Vì $M\in d$ và $M\notin (P)$ nên $d\parallel (P) \Leftrightarrow \vec{n}\cdot \vec{u}=0 \Leftrightarrow m=-5$. 
	}
	\end{ex}
	\begin{ex}%[HK2 (2017-2018), THPT Tân Hiệp, Kiên Giang]%[Bùi Mạnh Tiến, dự án (12EX-9)]%[2H3B3-6]
	Trong KG $Oxyz$, cho đường thẳng $d_1\colon \heva{&x=1+t\\&y=2-t\\&z=3t}(t\in \mathbb{R})$ và đường thẳng $d_2\colon \heva{&x=2s\\&y=1-2s\\&z=6s}(s\in \mathbb{R})$. Chọn khẳng định \textbf{đúng}.
	\choice
	{$d_1,d_2$ chéo nhau}
	{$d_1,d_2$ cắt nhau}
	{\True $d_1\parallel d_2$}
	{$d_1\equiv d_2$}
	\loigiai{
	Ta có $\overrightarrow{u}_{d_1}=(1;-1;3)$, $\overrightarrow{u}_{d_2}=(2;-2;6)=2\overrightarrow{u_{d_1}}\Rightarrow \hoac{&d_1\parallel d_2\\&d_1\equiv d_2.}$\\
	Lấy $A(0;1;0)\in d_2$. Dễ thấy $A\notin d_1\Rightarrow d_1\parallel d_2$.
	}
	\end{ex}
	\begin{ex}%[HK2 (2017-2018), THPT Tân Hiệp, Kiên Giang]%[Bùi Mạnh Tiến, dự án (12EX-9)]%[2H3B3-6]
	Trong KG $Oxyz$, cho mặt phẳng $(\alpha)\colon x+y-2z+1=0$ đi qua điểm $M(1;-2;0)$ và cắt đường thẳng $d\colon \heva{&x=11+2t\\&y=2t\\&z=-4t} (t\in \mathbb{R})$ tại $N$. Tính độ dài đoạn $MN$.
	\choice
	{$7\sqrt{6}$}
	{$3\sqrt{11}$}
	{$\sqrt{10}$}
	{\True $4\sqrt{5}$}
	\loigiai{
	Điểm $N\in (d)\Rightarrow N(11+2t;2t;-4t)$. Mặt khác $N\in (\alpha)$ nên
	\begin{center}
	$11+2t+2t-2(-4t)+1=0\Leftrightarrow t=-1$.
	\end{center}
	Điểm $N(9;-2;4)\Rightarrow \overrightarrow{MN}=(8;0;4)\Rightarrow MN=4\sqrt{5}$.
	}
	\end{ex}
	\begin{ex}%[HK2-Sở Bến Tre - 2018]%[Phan Hoàng Anh - EX9]%[2H3B3-6]
	Trong không gian với hệ trục tọa độ $Oxyz$, cho mặt phẳng $(P)\colon2x-y-z+3=0$, và đường thẳng $\Delta\colon\dfrac{x+1}{1}=\dfrac{y-1}{-2}=\dfrac{z}{2}$. Xét vị trí tương đối của $(P)$ và $\Delta$. 
	\choice
	{$(P)$ và $\Delta$ chéo nhau}
	{$(P)$ song song $\Delta$}
	{$(P)$ chứa $\Delta$}
	{\True $(P)$ cắt $\Delta$}
	\loigiai{\begin{itemize}
	\item Mặt phẳng $(P)$ có véc-tơ pháp tuyến là $\overrightarrow{n}=(2;-1;-1)$.
	\item Đường thẳng $\Delta$ đi qua điểm $M(-1;1;0)$ và có véc-tơ chỉ phương là $\overrightarrow{u}=(1;-2;2)$.
	\end{itemize}
	Ta có $\overrightarrow{n}\cdot\overrightarrow{u}=2+2-2=2\ne0$, nên suy ra $\Delta$ cắt $(P)$.}
	\end{ex}
	\begin{ex}%[HK2-Sở Bến Tre - 2018]%[Phan Hoàng Anh - EX9]%[2H3B3-6]
	Trong KG $Oxyz$, cho đường thẳng $d\colon\heva{&x=3+2t\\&y=5-3mt\\&z=-1+t.}$ và mặt phẳng $(P)\colon4x-4y+2z-5=0$. Giá trị nào của $m$ để đường thẳng $d$ vuông góc với mặt phẳng $(P)$. 
	\choice
	{$m=\dfrac{3}{2}$}
	{\True $m=\dfrac{2}{3}$}
	{$m=-\dfrac{5}{6}$}
	{$m=\dfrac{5}{6}$}
	\loigiai{\begin{itemize}
	\item Mặt phẳng $(P)$ có véc-tơ pháp tuyến là $\overrightarrow{n}=(4;-4;2)$.
	\item Đường thẳng $d$ có véc-tơ chỉ phương là $\overrightarrow{u}=(2;-3m;1)$.
	\end{itemize}
	Đường thẳng $d$ vuông góc với mặt phẳng $(P)$ khi và chỉ khi $\overrightarrow{n}$ cùng phương với $\overrightarrow{u}$\\
	$\Leftrightarrow\dfrac{2}{4}=\dfrac{-3m}{-4}=\dfrac{1}{2}\Leftrightarrow3m=2\Leftrightarrow m=\dfrac{2}{3}$.}
	\end{ex}
	\begin{dang}{Bài toán liên quan giữa đường thẳng - mặt phẳng - mặt cầu}%Dạng 7
	\end{dang}
 \setcounter{subsubsection}{0}
 \setcounter{vd}{0}
 \setcounter{ex}{0}
	\subsubsection{Ví dụ minh hoạ}
	\begin{vd}%[Đề thi hết học kì 2, Bình Minh, Ninh Bình 2018]%[Nguyễn Tuấn Anh, dự án EX9]%[2H3B3-7]
	Trong KG $Oxyz$, viết phương trình mặt phẳng $(P)$ đi qua hai điểm $A(2;1;3), B(1;-2;1)$ và song song với đường thẳng $d\colon \heva{&x=-1+t\\ &y=2t\\ &z=-3-2t.}$
	\loigiai{
	Ta có $\overrightarrow{AB}=(-1;-3;-2)$, đường thẳng $d$ nhận $\overrightarrow{u}=(1;2;-2)$ làm véc-tơ chỉ phương.\\
	Theo giả thiết mặt phẳng $(P)$ qua $A(2;1;3)$ và nhận $\overrightarrow{n}=[\overrightarrow{AB},\overrightarrow{u}]=(10;-4;1)$ làm véc-tơ pháp tuyến.\\
	Phương trình mặt phẳng $(P)$ là
	$$ 10(x-2)-4(y-1)+(z-3)=0\Leftrightarrow 10x-4y+z-19=0. $$
	}
	\end{vd}
	\begin{vd}%[HK2, THTH ĐHSP Tp. HCM, 2018]%[Vinhhop Tran, 12EX-8]%[2H3B3-7]
	Trong không gian tọa độ $Oxyz,$ cho điểm $A(1; 0; 0)$ và đường thẳng $d\colon \dfrac{x-1}{2}=\dfrac{y+2}1=\dfrac{z-1}{2}.$ Viết phương trình mặt phẳng $(P)$ chứa điểm $A$ và đường thẳng $d$.
	\loigiai{
	$d$ đi qua điểm $M(1; -2; 1)$ và có một véc-tơ chỉ phương là $\vec{u}(2; 1; 2).$ Do $(P)$ chứa $A$ và $d$ nên $(P)$ có một véc-tơ pháp tuyến là $\vec n=\left[\vec{AM},\vec u\right]=(-5; 2; 4).$ Suy ra phương trình của $(P)$ là $5x-2y-4z-5=0.$
	}
	\end{vd}
\begin{vd} %[Nguyễn Quốc Thịnh, dự án 12-EX-DCHT-1-DCHT-2019]% [2H3G3-2]
	Trong không gian với hệ trục $Oxyz$, cho đường thẳng $ \left(d\right) \colon \dfrac{x-2}{-1}=\dfrac{y-1}{-2}=\dfrac{z-1}{1}$ và mặt cầu $ \left(S\right) \colon {\left(x+1\right)}^{2}+{\left(y-2\right)}^{2}+{\left(z-1\right)}^{2}=25$. Viết PTĐT $ \Delta $ đi qua điểm $ M\left(-1;-1;-2\right)$ cắt đường thẳng $ \left(d\right)$ và cắt mặt cầu $ \left(S\right)$ tại hai điểm $ A$ và $ B$ sao cho $ AB=8$.
	\dapso{$ \Delta \colon \dfrac{x+1}{2}=\dfrac{x+1}{2}=\dfrac{z+2}{1}$}
	\loigiai{
		$ \left(S\right) \colon {\left(x+1\right)}^{2}+{\left(y-2\right)}^{2}+{\left(z+1\right)}^{2}=25$ có tâm $ I\left(-1;2;1\right)$ và bán kính $ R=5$.
		$ AB=8\Leftrightarrow d\left(I,\Delta \right)=3$.\\
		Gọi $ N\left(2-t;1-2t;1+t\right)$ là giao điểm của $ \left(d\right)$ và $ \Delta $.\\
		Khi đó một vtcp của $ \Delta $ là $ \overrightarrow{MN}=\left(3-t;2-2t;3+t\right)$.
		\[ d\left(I,\Delta \right)=3\Leftrightarrow \dfrac{\left|\left[\overrightarrow{IM}.\overrightarrow{MN}\right]\right|}{\left|\overrightarrow{MN}\right|}=3\Leftrightarrow \dfrac{\sqrt{{\left(3-3t\right)}^{2}+{\left(9-3t\right)}^{2}+{\left(3t-9\right)}^{2}}}{\sqrt{{\left(3-t\right)}^{2}+{\left(2-2t\right)}^{2}+{\left(3+t\right)}^{2}}}=3 \Leftrightarrow t=-1\]
		Vậy $ \overrightarrow{MN}=\left(4;4;2\right)$ nên chọn vtcp của $ \Delta $ là $ \vec{u}=\left(2;2;1\right)$.\\
		Vậy PTĐT $ \Delta \colon \dfrac{x+1}{2}=\dfrac{x+1}{2}=\dfrac{z+2}{1}$.}
\end{vd}
\begin{vd}%[Nguyen Phu Thach, 12-EX-1-DCHT]%[2H3B3-6]%[2H3K3-7]
	Cho mặt cầu $(S):x^2+y^2+z^2-2x+2y+4z-3=0$ và hai đường thẳng $\Delta_1:\heva{&x+2y-2=0\\&x-2z=0}$ và $\Delta_2:\dfrac{x-1}{-1}=\dfrac{y}{1}=\dfrac{z}{-1}$.
	\begin{enumerate}
		\item Chứng minh $\Delta_1$ và $\Delta_2$ chéo nhau.
		\item Viết phương trình tiếp diện của mặt cầu $(S)$ biết tiếp diện đó song song với hai đường thẳng $\Delta_1$ và $\Delta_2$.
		\dapso{ $(\alpha_1):y+z+3+3\sqrt{2}=0$ và  $(\alpha_2):y+z+3-3\sqrt{2}=0.$}
	\end{enumerate}
	\loigiai{
		\begin{enumerate}
			\item Đặt $x=2t$ thì ta đưa $\Delta_1$ về dạng tham số là
			$\heva{&x=2t\\&y=1-t\\&z=t}$.\\
			$\Delta_1$ qua $A(0;1;0)$ và có véc-tơ chỉ phương $\overrightarrow{u}_1=(2;-1;1)$, $\Delta_2$ qua $B(1;0;0)$ và có véc-tơ chỉ phương $\overrightarrow{u}_2=(-1;1;-1)$. Ta có $\overrightarrow{AB}=(1;-1;0)$. Xét
			$$[\overrightarrow{u}_1,\overrightarrow{u}_2]\cdot \overrightarrow{AB}=(0;1;1)\cdot (1;-1;0)=-1\neq 0.$$
			Vậy $\Delta_1$ và $\Delta_2$ chéo nhau.
			\item Mặt cầu $(S)$ có tâm $I(1;-1;-2)$ và bán kính $R=3$.\\
			Gọi $(\alpha)$ là tiếp diện cần tìm. Vì $(\alpha)$ song song với $\Delta_1$ và $\Delta_2$ nên $(\alpha)$ nhận
			$ [\overrightarrow{u}_1,\overrightarrow{u}_2]=(0;1;1)$ làm véc-tơ pháp tuyến. Suy ra phương trình $(\alpha)$ có dạng
			$$y+z+m=0.$$
			Vì $(\alpha)$ tiếp xúc với $(S)$ nên
			$$\mathrm{d}(I,(\alpha))=R\Leftrightarrow \dfrac{|-1+(-2)+m|}{\sqrt{1^2+1^2}}=3\Leftrightarrow\dfrac{|m-3|}{\sqrt{2}}=3\Leftrightarrow \hoac{&m=3+3\sqrt{2}\\&m=3-3\sqrt{2}.}$$
			Vậy $(S)$ có hai tiếp diện là $(\alpha_1):y+z+3+3\sqrt{2}=0$ và  $(\alpha_2):y+z+3-3\sqrt{2}=0.$
		\end{enumerate}
	}
\end{vd}
\begin{vd}%[Trần Tuấn Việt,Dự án 12EX1-DCHT 2018]%[2H3K1-1]%Bài 162
	Trong KG $Oxyz$, tìm điểm $M \in d\colon \dfrac{x - 3}{2} = \dfrac{y - 2}{1} = \dfrac{z - 1}{-2}$ sao cho mặt phẳng đi qua $M$ và vuông góc với đường thẳng $\mathrm{d}$ cắt mặt cầu $(S)\colon x^2 + y^2 + z^2 - 2x + 2y - 4z -19 = 0$ theo một đường tròn có chu vi bằng $8\pi$. \dapso{$M(-1; 0; 5)$ và $M(3; 2; 1)$}
	\loigiai{Phương trình mặt cầu $(S)$ được viết lại thành $(S)\colon (x - 1)^2 + (y + 1)^2 + (z - 2)^2 = 25$. Suy ra, tâm và bán kính của $(S)$ lần lượt là $I(1; -1; 2)$ và $R = 5$.\\
		Gọi $(P)$ là mặt phẳng đi qua $M$ và vuông góc với $d$, phương trình mặt phẳng $(P)$ có dạng: $(P)\colon 2x + y - 2z + m = 0$.\\
		Gọi $(C)$ là đường tròn tạo thành từ giao của $(S)$ và $(P)$. Theo đề bài, ta tìm được bán kính $r$ của $(C)$ như sau:
		$$2\pi r = 8\pi \Leftrightarrow r = 4.$$
		Suy ra, khoảng cách từ tâm $I$ đến mặt phẳng $P$ bằng $\sqrt{R^2 - r^2} = \sqrt{5^2 - 4^2} = 3$. Ta thu được hệ thức
		\begin{align*}
		\mathrm{d}(I; P) = 3 & \Leftrightarrow  \dfrac{\left|2\cdot 1 + 1 \cdot (-1) -2 \cdot 2 + m\right|}{3} = 3\\
		& \Leftrightarrow  \left| m - 3\right|  =  9 \\
		& \Leftrightarrow \hoac{& m =12\\&m = -6}.
		\end{align*}
		
		\begin{itemize}
			\item Với $m = 12$, tọa độ điểm $M$ là nghiệm của hệ: $\heva{& x = 3 + 2t\\& y = 2 + t\\& z = 1 - 2t\\& 2x + y - 2z + 12 = 0} \Leftrightarrow \heva{&t = -2\\&x = -1\\&y = 0\\&z = 5} \Rightarrow M(-1; 0; 5)$.
			\item Với $m = -6$, tọa độ điểm $M$ là nghiệm của hệ: $\heva{& x = 3 + 2t\\& y = 2 + t\\& z = 1 - 2t\\& 2x + y - 2z - 6 = 0} \Leftrightarrow \heva{&t = 0\\&x = 3\\&y = 2\\&z = 1} \Rightarrow M(3; 2; 1)$.
		\end{itemize}
		\noindent Vậy có hai điểm $M$ thỏa mãn bài toán là $M(-1; 0; 5)$ và $M(3; 2; 1)$.
	}
\end{vd}
\begin{vd}%[Trần Tuấn Việt,Dự án 12EX1-DCHT 2018]%[2H3K1-1]%Bài 168
	Trong KG $Oxyz$, cho mặt phẳng $(P)\colon x - 2y + 2z - 1 = 0$ và hai đường thẳng $\Delta _1\colon \dfrac{x - 4}{1} = \dfrac{y + 1}{2} = \dfrac{z}{2}$, $\Delta _2\colon \heva{& x = 3\\&y = 1 + t\\&z = 2 + t}$. Xác định tọa độ điểm $M \in \Delta _1$ sao cho khoảng cách từ $M$ đến đường thẳng $\Delta _2$ và khoảng cách từ $M$ đến mặt phẳng $(P)$ bằng nhau. \dapso{$M(5; 1; 2)$ và $M(2; -5; -4)$}
	\loigiai{Giả sử $M(m + 4; 2m - 1; 2m)$ là điểm cần tìm thuộc đường thẳng $\Delta _1$. Ta có
		$$\mathrm{d}(M; P) = \dfrac{\left|m + 4 - 4m + 2 + 4m - 1\right|}{3} = \dfrac{\left|m + 5\right|}{3} .$$
		Gọi $H (3; 1 + h; 2 + h)$ là hình chiếu của điểm $M$ trên đường thẳng $\Delta _2$.\\
		Ta có $\overrightarrow{MH} = (-1 - m; h - 2m + 2; h - 2m + 2)$, từ đây ta lập được hệ phương trình
		\begin{align*}
		\heva{& MH = \mathrm{d}(M; P)\\& \overrightarrow{MH} \cdot \overrightarrow{u}_{\Delta_2} = 0} & \Leftrightarrow \heva{& (m + 1)^2 + (h - 2m + 2)^2 + (h - 2m + 2)^2 = \dfrac{(m + 5)^2}{9}\\& 2h - 4m + 4 = 0}\\
		& \Leftrightarrow \heva{& (m + 1)^2 = \dfrac{(m + 5)^2}{9} \\& h = 2m - 2} \\
		& \Leftrightarrow \heva{& \hoac{& m = 1\\& m = -2}\\& h = 2m - 2.}
		\end{align*} 
		Vậy tọa độ điểm $M$ cần tìm là $M(5; 1; 2)$ và $M(2; -5; -4)$.
	}
\end{vd}
\begin{vd}%[Dự án 12EX1-DCHT,Chu Duc Minh]%[2H3K3-3]
	Trong không gian với hệ trục tọa độ $Oxyz$, cho đường thẳng $\Delta \colon \dfrac{x-2}{1} = \dfrac{y + 1}{-2} = \dfrac{z}{-1}$ và mặt phẳng $(P) \colon x + y + z -3 = 0$. Gọi $I$ là giao điểm của $\Delta$ và $(P)$. Tìm tọa độ điểm $M$ thuộc $(P)$ sao cho $MI$ vuông góc với $\Delta$ và $MI = 4\sqrt{14}$. 
	\dapso{$M(5; 9; -11), M(-3; -7; 13)$.}
	\loigiai{
		\begin{itemize}
			\item $I \in \Delta$ nên $I(2+t; -1;ư-2t; - t)$. Ta có $I \in (P) \Leftrightarrow (2 + t) + (-1-2t) + (-t) - 3 = 0 \Leftrightarrow t = 1$. Suy ra $I(1; 1; 1)$. 
			\item Giả sử $M(x;y;z)$. Đường thẳng $\Delta$ có VTCP $\overrightarrow{u} = (1; -2; -1)$. Ta có $\overrightarrow{IM} = (x -1; y - 1; z - 1)$. \\
			Theo giả thiết $\heva{&M \in (P)\\&IM \perp \Delta} \Leftrightarrow \heva{& x + y + z - 3 = 0\\ & (x - 1) - 2(y - 1)- (z - 1) = 0} \Leftrightarrow \heva{& y = 2x - 1\\&z = 4- 3x.}$
			\item Ta có 
			\begin{align*}
			IM = 4\sqrt{14} &\Leftrightarrow (x -1)^2 + (y - 1)^2 + (z - 1)^2 = 224 \Leftrightarrow (x-1)^2 + 4(x-1)^2 + 9(x - 1)^2 = 224 \\
			&\Leftrightarrow (x-1)^2 = 16 \Leftrightarrow \hoac{& x = 5\\ &x = -3.}
			\end{align*}
			\item Vậy $M(5; 9; -11)$ hoặc $M(-3; -7; 13)$. 
		\end{itemize}
	}
\end{vd}
\begin{vd}%[Nguyễn Ngọc Tâm, Dự án DCHT 1]%[2H3K3-7]
	Viết phương trình mặt cầu $\left( {S} \right)$ có tâm $I$ và tiếp xúc với đường thẳng $\Delta$, với:
	\begin{listEX}[2]
		\item $I\left( {1;2;3} \right)$, $\Delta \colon \dfrac{x}{1}=\dfrac{y+2}{-2}=\dfrac{z}{2}$
		\dapso{$\left( {S} \right)$ là $(S) \colon \left( {x-1} \right)^2+\left( {y-2} \right)^2+\left( {z-3} \right)^2=\dfrac{233}{9}$}
		\item $I\left( {-2;3;-1} \right)$, $\Delta \colon \dfrac{x-1}{1}=\dfrac{y+1}{1}=\dfrac{z+2}{-2}$
		\dapso{$(S) \colon \left( {x+2} \right)^2+\left( {y-3} \right)^2+\left( {z+1} \right)^2=\dfrac{155}{6}$}
	\end{listEX}
	\loigiai{
		\begin{enumerate}
			\item 
			Gọi $R$ là bán kính mặt cầu $\left( {S} \right)$.
			\\$\Delta$ đi qua $A\left( {0;-2;0} \right)$ và có một véc-tơ chỉ phương là $\overrightarrow{u}=\left( {1;-2;2} \right)$, $\left[ {\overrightarrow{u},\overrightarrow{IA}} \right]=\left( {14;1;-6} \right)$.
			\\Do $\left( {S} \right)$ tiếp xúc với $\Delta$ nên $R=\mathrm{d}(I;\Delta)=\dfrac{\left| {\left[ {\overrightarrow{u},\overrightarrow{IA}} \right]} \right|}{\left| {\overrightarrow{u}} \right|}=\dfrac{\sqrt{{14}^2+1^2+\left( {-6} \right)^2}}{\sqrt{1^2+\left( {-2} \right)^2+2^2}}=\dfrac{\sqrt{233}}{3}$.
			\\Phương trình mặt cầu $\left( {S} \right)$ là $(S) \colon \left( {x-1} \right)^2+\left( {y-2} \right)^2+\left( {z-3} \right)^2=\dfrac{233}{9}$.
			\item 
			Gọi $R$ là bán kính mặt cầu $\left( {S} \right)$.
			\\$\Delta$ đi qua $A\left( {1;-1;-2} \right)$ và có một véc-tơ chỉ phương là $\overrightarrow{u}=\left( {1;1;-2} \right)$, $\left[ {\overrightarrow{u},\overrightarrow{IA}} \right]=\left( {-9;-5;-7} \right)$.
			\\Do $\left( {S} \right)$ tiếp xúc với $\Delta$ nên $R=\mathrm{d}(I;\Delta)=\dfrac{\left| {\left[ {\overrightarrow{u},\overrightarrow{IA}} \right]} \right|}{\left| {\overrightarrow{u}} \right|}=\dfrac{\sqrt{\left( {-9} \right)^2+\left( {-5} \right)^2+\left( {-7} \right)^2}}{\sqrt{1^2+1^2+\left( {-2} \right)^2}}=\dfrac{\sqrt{930}}{6}$.
			\\Phương trình mặt cầu $\left( {S} \right)$ là $(S) \colon \left( {x+2} \right)^2+\left( {y-3} \right)^2+\left( {z+1} \right)^2=\dfrac{155}{6}$.
		\end{enumerate}
	}
\end{vd}
\begin{vd}%[Nguyễn Ngọc Tâm, Dự án DCHT 1]%[2H3K3-7]
	Trong không gian với hệ trục toạ độ $Oxyz$, cho đường thẳng $d:\dfrac{x+1}{1}=\dfrac{y}{2}=\dfrac{z-2}{1}$ và điểm $I\left( {0;0;3} \right)$. Viết phương trình mặt cầu $\left( {S} \right)$ có tâm $I$ và cắt đường thẳng $d$ tại hai điểm $A$, $B$ sao cho tam giác $IAB$ vuông tại $I$.
	\dapso{$(S) \colon x^2+y^2+\left( {z-3} \right)^2=\dfrac{8}{3}$.}
	\loigiai{
		Đường thẳng $d$ qua $C\left( {-1;0;2} \right)$ và có một véc-tơ chỉ phương là $\overrightarrow{u}=\left( {1;2;1} \right)$. $\left[ {\overrightarrow{IC},\overrightarrow{u}} \right]=\left( {2;0;-2} \right)$
		Gọi $H$ là hình chiếu của $I$ xuống đường thẳng $AB$. Ta có $IH=\mathrm{d}(I,d)=\dfrac{\left| {\left[ {\overrightarrow{IC},\overrightarrow{u}} \right]} \right|}{\left| {\overrightarrow{u}} \right|}=\dfrac{\sqrt{2^2+0^2+\left( {-2} \right)^2}}{\sqrt{1^2+2^2+1^2}}=\dfrac{2\sqrt{3}}{3}$. 
		\\Tam giác $IAB$ vuông tại $I$, đường cao $IH$: $\dfrac{1}{IA^2}+\dfrac{1}{IB^2}=\dfrac{1}{IH^2} \Leftrightarrow \dfrac{1}{IA^2}=\dfrac{1}{2IH^2}\Leftrightarrow IA=IH\sqrt{2}=\dfrac{2\sqrt{6}}{3}$.
		\\Phương trình mặt cầu $\left( {S} \right)$ tâm $I$, bán kính $IA$ là $(S) \colon x^2+y^2+\left( {z-3} \right)^2=\dfrac{8}{3}$.}
\end{vd}
\begin{vd}%[Nguyễn Ngọc Tâm, Dự án DCHT 1]%[2H3]
	Cho điểm $I\left( {3;4;0} \right)$ và đường thẳng $\Delta \colon \dfrac{x-1}{1}=\dfrac{y-2}{1}=\dfrac{z+1}{-4}$. Viết phương trình mặt cầu $\left( {S} \right)$ có tâm $I$ và cắt đường thẳng $\Delta$ tại hai điểm $A$, $B$ sao cho $S_{\triangle IAB}=12$.
	\dapso{$(S) \colon \left( {x-3} \right)^2+\left( {y-4} \right)^2+z^2=25$.}
	\loigiai{
		Đường thẳng $\Delta$ đi qua điểm $C\left( {1;2;-1} \right)$ và có một véc-tơ chỉ phương là $\overrightarrow{u}=\left( {1;1;-4} \right)$. $\left[ {\overrightarrow{IC},\overrightarrow{u}} \right]=\left( {-9;-9;0} \right)$.
		\\Gọi $H$ là hình chiếu của $I$ xuống đường thẳng $\Delta \Rightarrow$ $H$ là trung điểm của $AB$. \\Ta có $IH=\mathrm{d}(I,\Delta)=\dfrac{\left| {\left[ {\overrightarrow{IC},\overrightarrow{u}} \right]} \right|}{\left| {\overrightarrow{u}} \right|}=\dfrac{\sqrt{\left( {-9} \right)^2+\left( {-9} \right)^2+0^2}}{\sqrt{1^2+1^2+\left( {-4} \right)^2}}=3$.
		\\$S_{\triangle IAB}=\dfrac{1}{2} \cdot IH \cdot AB \Leftrightarrow AB=\dfrac{2 \cdot S_{\triangle IAB}}{IH}=8 \Leftrightarrow HA=\dfrac{AB}{2}=4$.
		\\Xét tam giác $IHA$ vuông tại $H$: $IA=\sqrt{IH^2+HA^2}=\sqrt{3^2+4^2}=5.$
		\\Phương trình mặt cầu $\left( {S} \right)$ tâm $I$, bán kính $IA$ là $(S) \colon \left( {x-3} \right)^2+\left( {y-4} \right)^2+z^2=25$.}
\end{vd}
	\subsubsection{Bài tập trắc nghiệm}
	\begin{ex}%[Đề thi thử - trường THPT chuyên Tiền Giang - Lần 1 - 2018]%[2H3B3-7]%[Thầy Dương Phước Sang và Thầy Tuấn Nguyễn, dự án(12EX-7)]
	Trong hệ tọa độ $Oxyz$, cho điểm $A(2;1;1)$ và mặt phẳng $(P):2x-y+2z+1=0$. Phương trình của mặt cầu tâm $A$ và tiếp xúc với mặt phẳng $(P)$ là
	\choice
	{$(x-2)^2+(y-1)^2+(z-1)^2=9$}
	{$(x-2)^2+(y-1)^2+(z-1)^2=2$}
	{\True $(x-2)^2+(y-1)^2+(z-1)^2=4$}
	{$(x-2)^2+(y-1)^2+(z-1)^2=36$}
	\loigiai{
	Mặt cầu $(S)$ có bán kính $R=\mathrm{d}(A;(P))= \dfrac{|2.2-1+2.1+1|}{\sqrt{2^2+(-1)^2+2^2}}=2$ và tâm $A(2;1;1)$\\
	$\Rightarrow (S):(x-2)^2+(y-1)^2+(z-1)^2=4$.}
	\end{ex}
	\begin{ex}%[Thi thử L1, Phổ thông Năng khiếu, HCM, 2018]%[2H3B3-7]%[Nguyễn Thành Khang, 12-EX-8-2018]
	Trong không gian với hệ toạ độ $Oxyz$, cho $(P)\colon x-2y+2z-5=0$, $A(-3;0;1)$, $B(1;-1;3)$. Viết PTĐT $d$ qua $A$, song song với $(P)$ sao cho khoảng cách từ $B$ đến $d$ là lớn nhất.
	\choice
	{$\dfrac{x+3}{1}=\dfrac{y}{-1}=\dfrac{z-1}{2}$}
	{$\dfrac{x+3}{3}=\dfrac{y}{-2}=\dfrac{z-1}{2}$}
	{$\dfrac{x-1}{1}=\dfrac{y}{-2}=\dfrac{z-1}{2}$}
	{\True $\dfrac{x+3}{2}=\dfrac{y}{-6}=\dfrac{z-1}{-7}$}
	\loigiai{
	\immini{
	Vì $\left(-3-2\cdot 0+2\cdot 1-5\right)\left(1-2\cdot (-1)+2\cdot 3-5\right)<0$ nên hai điểm $A, B$ khác phía so với $(P)$.\\
	Gọi $H$ là hình chiếu của $B$ lên $d$.\\
	Ta có: $BH \le BA$ nên khoảng cách $BH$ từ $B$ đến $d$ lớn nhất khi và chỉ khi $H$ trùng $A$.\\
	Khi đó $AB \perp d$.\\
	véc-tơ pháp tuyến của $(P)$ là $\vec{n}=\left(1;-2;2\right), \vec{AB}=(4;-1;2)$.\\
	véc-tơ chỉ phương của $d$ là $\vec{u}=\left[\vec{n},\vec{AB}\right]=(-2;6;7)$.\\
	Mà $d$ qua $A(-3;0;1)$ nên PTĐT $d$ là: $\dfrac{x+3}{2}=\dfrac{y}{-6}=\dfrac{z-1}{-7}$.
	}{
	\begin{tikzpicture}
	\tkzInit[ymin=-0.5,ymax=7,xmin=-2,xmax=4.1]
	\tkzClip
	\tkzDefPoints{0/0/B,0/6/H,2/6/A,1.5/4.5/C,0/4.5/D,0.75/2.25/E,0/2.25/F,-1/6/h,4/6/a,-1.5/2.25/X,-0.75/4.5/Y,4/4.5/P,3.25/2.25/T}
	\tkzDefMidPoint(A,B)\tkzGetPoint{M}
	\tkzDefMidPoint(B,H)\tkzGetPoint{N}
	\tkzDrawSegments(h,a A,M H,N B,E B,F X,Y Y,D C,P P,T T,X)
	\tkzMarkAngle[size=0.75](C,P,T)
	\tkzDrawSegments[dashed](C,D M,E N,F) 
	\tkzLabelSegment[pos=0.9](h,a){$d$}
	\tkzLabelPoints[above](H,A)
	\tkzLabelPoints[below](B)
	\tkzLabelPoints[below left](P)
	\tkzMarkRightAngle(A,H,B)
	\end{tikzpicture}
	}
	}
	\end{ex}
	\begin{ex}%[KSCL L2, Yên Phong 2, Bắc Ninh 2018]%[2H3B3-7]%[Khuất Văn Thanh, 12EX-8]
	Trong KG $Oxyz$ cho đường thẳng $d\colon \heva{&x=2+t\\&y=-3+2t\\&z=1+3t} ~t\in\mathbb{R}$. Gọi $d'$ là hình chiếu vuông góc của $d$ trên mặt phẳng tọa độ $Oxz$. Viết PTĐT $d'$.
	\choice
	{$\heva{&x=2+t\\&y=3-2t\\&z=1+3t}~(t\in\mathbb{R})$}
	{$\heva{&x=0\\&y=-3+2t\\&z=1+3t}~(t\in\mathbb{R})$}
	{$\heva{&x=2+t\\&y=-3+2t\\&z=0}~(t\in\mathbb{R})$}
	{\True $\heva{&x=2+t\\&y=0\\&z=1+3t}~(t\in\mathbb{R})$}
	\loigiai{
	$d$ đi qua $M(2;-3;1)$ và có véc-tơ chỉ phương $\vec{u}=(1;2;3)$.\\
	Mặt phẳng $(Oxz)$ có véc-tơ pháp tuyến $\vec{n}=(0;1;0)$ và có phương trình $y=0$.\\
	Suy ra $\left[\vec{u},\vec{n}\right]=(-3;0;1)$.\\
	Gọi $H$ là hình chiếu vuông góc của $M$ trên $Oxz\Rightarrow H(2;0;1)$.\\
	Suy ra $d'$ là đường thẳng qua $H(2;0;1)$ và nhận véc-tơ $\vec{u'}=\left[\vec{n},\left[\vec{u},\vec{n}\right]\right]=(1;0;3)$ làm véc-tơ chỉ phương.\\
	Vậy phương trình của $d'\colon \heva{&x=2+t\\&y=0\\&z=1+3t}~(t\in\mathbb{R})$.
	}
	\end{ex}
	\begin{ex}%[Thi thử L2 -THPT-TranPhu-HaTinh]%[2H3B3-7]%[Nguyễn Tài Tuệ,Dự án EX8]
	Trong không gian với hệ trục tọa độ $ Oxyz$, cho đường thẳng $ d \colon \heva{&x=2+2t\\&y=1+t\\&z=4-t}$. Mặt phẳng đi qua $ A(2;-1;1) $ và vuông góc với đường thẳng $ d $ có phương trình là
	\choice
	{\True $ 2x +y-z-2=0$}
	{$ x+3y-2z-3=0 $}
	{$ x-3y-2z+3=0 $}
	{$ x+3y-2z-5=0 $}
	\loigiai{
	Véc-tơ chỉ phương của đường thẳng $ (d) $ là $ \vec{u}=(2;1;-1) $.\\	
	Mặt phẳng $ (P) $ đi qua $ A(2;-1;1) $ nhận $ \vec {u} $ là véc-tơ pháp tuyến có phương trình\\ $ 2(x-2)+1(y+1)-1(z-1)=0 \Leftrightarrow 2x+y-z-2=0.$
	}
	\end{ex}
	\begin{ex}%[Đề GHK2 T12, Thủ Đức, TP. Hồ Chí Minh 2018]%[2H3B3-7]%[Đinh Bích Hảo, 12EX-8]
	Trong KG $Oxyz$, cho mặt cầu $(S)\colon x^2+y^2+z^2-4x-2y+4z=0$ và mặt phẳng $(P)\colon x+2y-2z+1=0$. Gọi $(Q)$ là mặt phẳng song song với $(P)$ và tiếp xúc với mặt cầu $(S)$. Phương trình của mặt phẳng $(Q)$ là
	\choice
	{\True $(Q): x+2y-2z-17=0 $}
	{$(Q): 2x+2y-2z+19=0 $}
	{$(Q): x+2y-2z-35=0 $}
	{ $(Q): x+2y-2z+1=0 $}
	\loigiai{
	Ta gọi $I$ và $R$ lần lượt là tâm và bán kính mặt cầu $(S)$. Khi đó $I(2;1;-2)$ và $R=3$.\\
	Mặt phẳng $(Q) \parallel (P)$ nên $(Q)$ có dạng $x+2y-2z+d=0$. \\Vì $(Q)$ tiếp xúc với mặt cầu $(S)$ nên $d(I,(Q))=R$, suy ra $\dfrac{|2+2+4+d|}{\sqrt{1+2^2+(-2)^2}}=3 \Rightarrow d=1$ (loại vì trùng mặt phẳng $(P)$) hoặc $d=-17$.\\ Vậy $(Q): x+2y-2z-17=0 $.
	}
	\end{ex}
	\begin{ex}%[Thi HK2, Sở GD\&ĐT Đồng Tháp, 2018]%[2H3B3-7]%[Trần Hòa, dự án (12EX-8)]
	Trong KG $Oxyz$, cho đường thẳng $(d)\colon \dfrac{x+2}{1}=\dfrac{y-2}{-1}=\dfrac{z+3}{2}$ và điểm $A(1;-2;3)$. Mặt phẳng qua $A$ và vuông góc với đường thẳng $(d)$ có phương trình là
	\choice
	{\True $x-y+2z-9=0$}
	{$x-2y+3z-14=0$}
	{$x-y+2z+9=0$}
	{$x-2y+3z-9=0$}
	\loigiai{
	Đường thẳng $d$ có véc-tơ chỉ phương $\vec{u}=(1;-1;2)$.\\
	Vì mặt phẳng $(P)$ đi qua $A$ và vuông góc với đường thẳng $d$ nên $(P)$ có véc-tơ pháp tuyến $\vec{n}=(1;-1;2)$. Vậy phương trình mặt phẳng $(P)$ là $(x-1)-(y+2)+2(z-3)=0\Leftrightarrow x-y+2z-9=0$.
	}
	\end{ex}
	\begin{ex}%[Thi HK2, THPT Lý Thái Tổ, Hà Nội , 2018] %[2H3B3-7]%[Trần Tuấn Việt, dự án(12EX-8)
	Trong KG $Oxyz$, cho mặt phẳng $(P)$ có phương trình $(P): -x + 3z - 2 = 0$. Tìm đáp án đúng 
	\choice
	{\True $(P) \parallel Oy$}
	{$(P) \parallel xOz$}
	{$(P) \supset Oy$}
	{$(P) \parallel Ox$}
	\loigiai{Ta có $\overrightarrow{n}_P = (-1; 0; 3)$, $\overrightarrow{u}_{Oy} = (0; 1; 0)$. Dễ có tích vô hướng $\overrightarrow{n}_P . \overrightarrow{u_{Oy}} = 0$. Suy ra $(P) \parallel Oy$. 
	}
	\end{ex}
	\begin{ex}%[Thi HK2, THPT Lý Thái Tổ, Hà Nội , 2018] %[2H3B3-7]%[Trần Tuấn Việt, dự án(12EX-8)
	Trong KG $Oxyz$ cho đường thẳng $d: \heva{x &= 1 - 3t\\ y &= 2t \\z &= -2 - mt}$ và mặt phẳng $(P): 2x - y - 2z - 6 = 0$. Giá trị của $m$ để $d \subset (P)$ là 
	\choice
	{\True $m = 4$}
	{$m = -4$}
	{$m = 2$}
	{$m = -2$}
	\loigiai{Để $d \subset (P)$ thì phương trình $2(1 - 3t) - (2t) - 2(-2 - mt) - 6 = 0$ đúng với $\forall t \in \mathbb{R}$. \\
	$\Leftrightarrow t(-8 + 2m) = 0$ đúng với $\forall t \in \mathbb{R}$. \\
	$\Leftrightarrow m = 4$.
	}
	\end{ex}
	\begin{ex}%[HK2, THTH ĐHSP Tp. HCM, 2018]%[2H3B3-7]%[Vinhhop Tran, 12EX-8]
	Trong không gian tọa độ $Oxyz,$ cho điểm $A(1; 0; 0)$ và đường thẳng $d\colon \dfrac{x-1}{2}=\dfrac{y+2}1=\dfrac{z-1}{2}.$ Viết phương trình mặt phẳng $(P)$ chứa điểm $A$ và đường thẳng $d$.
	\choice
	{$(P)\colon 5x+2y+4z-5=0$}
	{$(P)\colon 2x+y+2z-1=0$}
	{$(P)\colon 2x+2y+z-2=0$}
	{\True $(P)\colon 5x-2y-4z-5=0$}
	\loigiai{$d$ đi qua điểm $M(1; -2; 1)$ và có một véc-tơ chỉ phương là $\vec{u}(2; 1; 2).$ Do $(P)$ chứa $A$ và $d$ nên $(P)$ có một véc-tơ pháp tuyến là $\vec n=\left[\vec{AM},\vec u\right]=(-5; 2; 4).$ Suy ra phương trình của $(P)$ là $5x-2y-4z-5=0.$
	}
	\end{ex}
	\begin{ex}%[HK2, THTH ĐHSP Tp. HCM, 2018]%[2H3B3-7]%[Vinhhop Tran, 12EX-8]
	Trong không gian tọa độ $Oxyz,$ cho đường thẳng $d\colon \dfrac{x-1}{2}=\dfrac{y}1=\dfrac{z}{-2}$ và hai điểm $A(2; 1; 0),$ $ B(-2; 3; 2).$ Viết phương trình mặt cầu $(S)$ có tâm thuộc $d$ và đi qua hai điểm $A, B$.
	\choice
	{\True $(S)\colon (x+1)^2+(y+1)^2+(z-2)^2=17$}
	{$(S)\colon (x-1)^2+(y-1)^2+(z+2)^2=17$}
	{$(S)\colon (x-3)^2+(y-1)^2+(z+2)^2=5$}
	{$(S)\colon (x+3)^2+(y+1)^2+(z-2)^2=33$}
	\loigiai{Gọi tâm mặt cầu là $I(1+2t; t; -2t)$. Từ $IA = IB$ ta suy ra $$(2t-1)^2+(t-1)^2+(-2t)^2=(2t+3)^2+(t-3)^2+(-2t-2)^2\Leftrightarrow t=-1.$$ Suy ra $I(-1; -1; 2)$ và $AI^2=17.$ Vậy mặt cầu cần tìm là $(S)\colon (x+1)^2+(y+1)^2+(z-2)^2=17.$
	}
	\end{ex}
	\begin{ex}%[Đề thi thử QG lần 1- Sở Bình Phước -2018]%[2H3B3-7]%[Trịnh Văn Xuân -Ex-8]
	Trong không gian với hệ toạ độ $Oxyz$, cho hai đường thẳng $d_1$, $d_2$ lần lượt có phương trình $d_1\colon \dfrac{x-2}{2}=\dfrac{y-2}{1}=\dfrac{z-3}{3}$, $d_2\colon\dfrac{x-1}{2}=\dfrac{y-2}{-1}=\dfrac{z-1}{4}$. Phương trình mặt phẳng cách đều hai đường thẳng $d_1$, $d_2$ là
	\choice
	{\True $14x-4y-8z+3=0$}
	{$14x-4y-8z-1=0$}
	{$14x-4y-8z+1=0$}
	{$14x-4y-8z-3=0$}
	\loigiai{
	Ta có $d_1$ đi qua $A(2;2;3)$, có véc-tơ chỉ phương $\vec{u}_{1}=(2;1;3)$, $d_2$ đi qua $B(1;2;1)$ và có $\vec{u}_{2}=(2;-1;4)$.\\
	Do $(P)$ cách đều $d_1,d_2$ nên $(P)$ song song với $d_1$, $d_2$\\ $\Rightarrow{\vec{n}}_P=\left[\vec{u}_{1},\vec{u}_{2}\right]=(7;-2;-4)$\\
	PT mặt phẳng $(P)$ có dạng: $7x-2y-4z+D=0$\\
	Do $(P)$ cách đều $d_1$, $d_2$ suy ra $\mathrm{d}(A,(P))=\mathrm{d}(B,(P))$\\
	$\Leftrightarrow \dfrac{\left| 7\cdot2-2\cdot2-4\cdot3+d\right|}{\sqrt{69}}=\dfrac{\left| 7\cdot1-2\cdot2-4\cdot1+d\right|}{\sqrt{69}}\Leftrightarrow |D-2|=| D-1|\Leftrightarrow D=\dfrac{3}{2}$\\
	Phương trình mặt phẳng $P$: $14x-4y-8z+3=0$}
	\end{ex}
	\begin{ex}%[Thi thử lần 1, Thanh Chương 1, Nghệ An 2018][2H3B3-7]%[Lê Mạnh Thắng, 12EX-8]
	Trong không gian với hệ tọa độ $Oxyz,$ cho đường thẳng $d$ có phương trình $\heva{&x=1+2t \\ &y=t \\ &z=2-t}$. Gọi đường thẳng $d'$ là hình chiếu vuông góc của đường thẳng $d$ trên mặt phẳng $(Oxy)$. Đường thẳng $d'$ có một véc-tơ chỉ phương là
	\choice
	{$\overrightarrow{u}_1=(2;0;1)$}
	{$\overrightarrow{u}_3=(1;1;0)$}
	{$\overrightarrow{u}_2=(-2;1;0)$}
	{\True $\overrightarrow{u}_4=(2;1;0)$}
	\loigiai{
	Gọi $I$ là giao điểm của $d$ và mp$(Oxy)\colon z=0$ $\Rightarrow I(5;2;0)$.\\
	Lấy $A(3;1;1)\in d$. Gọi $A'$ là hình chiếu vuông góc của $A$ trên $(Oxy)$ $\Rightarrow A'(3;1;0)$.\\
	Vì $d'$ là hình chiếu vuông góc của $d$ trên $(Oxy)$ nên $d'$ có $1$ véc-tơ chỉ phương là $\overrightarrow{IA'}=(-2;-1;0)$.
	}
	\end{ex}
	\begin{ex}%[KSCL (2017-2018) lần 4,Thanh Miện 2,Hải Dương]%[Chu Đức Minh, dự án EX9]%[2H3B3-7]
	Trong KG $Oxyz$, mặt phẳng song song với 2 đường thẳng $\Delta_1 \colon \dfrac{x - 2}{2} = \dfrac{y + 1}{-3} = \dfrac{z}{4}$ và $\Delta_2 \colon \heva{& x = 2 + t\\ & y = 3 + 2t \\ & z = 1- t}$ có 1 véc-tơ pháp tuyến là 
	\choice
	{$\vec{n} = (-5; 6; -7)$}
	{$\vec{n} = (5; - 6 ; 7)$}
	{\True $\vec{n} = (-5; 6; 7)$}
	{$\vec{n} = (-5; -6; 7)$}
	\loigiai{
	$\bullet$ $\vec{u_1} = (2; - 3; 4)$ và $\vec{u_2} = (1; 2; -1)$. \\
	$\bullet$ $(P)$ có véc-tơ pháp tuyến là $\vec{n} = \vec{u_1} \wedge \vec{u_2} = (-5; 6; 7)$. 
	}
	\end{ex}
	\begin{ex}%[TT Sở GD Bắc Ninh, 2018]%[Lê Minh An, 12Ex-9]%[2H3B3-7]
	Trong KG $Oxyz$, cho mặt cầu $(S)\colon (x-1)^2+(y+1)^2+z^2=8$ và hai đường thẳng $d_1\colon \dfrac{x+1}{1}=\dfrac{y-1}{1}=\dfrac{z-1}{2}$, $d_2\colon\dfrac{x+1}{1}=\dfrac{y}{1}=\dfrac{z}{1}$. Viết phương trình tất cả các mặt phẳng tiếp xúc với mặt cầu $(S)$ đồng thời song song với $d_1$, $d_2$.
	\choice{$x-y+2=0$}
	{$x-y+2=0$ hoặc $x-y+6=0$}
	{\True $x-y-6=0$}
	{$x-y+6=0$}
	\loigiai{
	Ta có $(S)$ có tâm $I(1;-1;0)$, bán kính $R=2\sqrt{2}$.\\
	$d_1$ qua $M(-1;1;1)$ và có véc-tơ chỉ phương $\overrightarrow{u}_1=(1;1;2)$. $d_2$ qua $N(-1;0;0)$ và có véc-tơ chỉ phương $\overrightarrow{u}_2=(1;1;1)$.\\
	Mặt phẳng $(P)$ song song với $d_1$, $d_2$ nên có véc-tơ pháp tuyến $\overrightarrow{n}=[\overrightarrow{u}_1,\overrightarrow{u}_2]=(1;-1;0)$, đo đó $(P)$ có phương trình $x-y+d=0$.\\
	Lại có $(P)$ tiếp xúc $(S)$ nên $\mathrm{d}(I,(P))=2\sqrt{2}\Leftrightarrow \hoac{& d=2\\ & d=-6}$.
	\begin{itemize}
	\item Với $d=2$, $(P)\colon x-y+2=0\Rightarrow M\in(P)$ (loại).
	\item Với $d=-6$, $(P)\colon x-y-6=0\Rightarrow M,N\notin (P)$ (thỏa mãn).
	\end{itemize}
	}
	\end{ex}
	\begin{ex}%[Đề thi học kì II, Ngô Quyền-Quảng Ninh, 2018]%[Trần Ngọc Minh, dự án 12EX-9-2018]%[2H3B3-7]
	Trong KG $Oxyz$, cho mặt cầu $S(I;R)$ có tâm $I(1;1;3)$ và bán kính $R=\sqrt{10}$. Hỏi có bao nhiêu giao điểm giữa mặt cầu $(S)$ với các trục $Ox, Oy, Oz$?
	\choice 
	{$1$}
	{$2$}
	{\True $4$}
	{$6$}
	\loigiai{ 
	Dễ thấy mặt cầu tiếp xúc với trục $Ox$ tại $A(1;0;0)$ và tiếp xúc $Oy$ tại $B(0;1;0)$, cắt trục $Oz$ tại hai điểm phân biệt, tất cả các điểm trên không trùng gốc tọa độ (do $O$ không nằm trên mặt cầu). Vậy có tất cả bốn giao điểm. 
	} 
	\end{ex} 
	\begin{ex}%[12 HK2 năm học 2017 – 2018 Đà Nẵng]%[Mai Sương 12-EX-9-2018]%[2H3B3-7]
	Trong KG $Oxyz$, cho điểm $I(3;4;-5)$ và mặt phẳng $(P)$ có phương trình $2x+6y-3z+4=0$. Phương trình mặt cầu $(S)$ có tâm $I$ và tiếp xúc với mặt phẳng $(P)$ là
	\choice
	{$(x-3)^2+(y-4)^2+(z+5)^2=\dfrac{361}{49}$}
	{\True $(x-3)^2+(y-4)^2+(z+5)^2=49$}
	{$(x+3)^2+(y+4)^2+(z-5)^2=49$}
	{$(x+3)^2+(y+4)^2+(z-5)^2=\dfrac{361}{49}$}
	\loigiai{
	Mặt cầu $(S)$ có tâm $I(3;4;-5)$ và tiếp xúc với mặt phẳng $(P):2x+6y-3z+4=0$ có bán kính
	$$R=\mathrm{d}(I,(P))=\dfrac{|2\cdot 3 + 6\cdot 4 -3\cdot (-5) +4|}{\sqrt{2^2+6^2+(-3)^2}}=7.$$
	Vậy $(S)\colon(x-3)^2+(y-4)^2+(z+5)^2=49.$
	}
	\end{ex}
\begin{ex}%[Thi thử L2, Võ Thành Trinh An Giang, 2018]%[Trần Như Ngọc, dự án (12EX-10)]%[2H3B3-7]
	Trong KG $Oxyz$, cho đường thẳng $\Delta: \dfrac{x + 2}{1}=\dfrac{y - 1}{1}=\dfrac{z - 2}{2}$ và mặt phẳng $(P): x + y + z=0. $ Đường thẳng ${\Delta}'$ là hình chiếu của đường thẳng $\Delta $ lên mặt phẳng $(P). $ Một véc-tơ chỉ phương $\overrightarrow{u}$ của đường thẳng ${\Delta}'$ là
	\choice
	{\True $\overrightarrow{u}=\left(1; 1; - 2\right)$}
	{$\overrightarrow{u}=\left(1; - 1; 0\right)$}
	{$\overrightarrow{u}=\left(1; 0; - 1\right)$}
	{$\overrightarrow{u}=\left(1; - 2; 1\right)$}
	\loigiai{
	Gọi $(Q)$ là mặt phẳng chứa $\Delta $ và vuông góc với $(P)$. Suy ra, véc-tơ pháp tuyến của $(Q)$ là $\vec{n}_Q=\left[\vec{u}_{\Delta}, \vec{n}_P\right]=\left(- 1; 1; 0\right)$.\\
	Gọi $\vec{u}$ là véc-tơ chỉ phương của đường thẳng $\Delta'$. Ta có $\heva{ & \vec{u}\bot \vec{n}_P \\ & \vec{u}\bot \vec{n}_Q} \Rightarrow \vec{u}=\left[\vec{n}_P, \vec{n}_Q\right]=\left(1; 1; - 2\right)$.}
	\end{ex}

	\begin{ex}%[TT lần 2, cụm các trường THPT chuyên Bắc Bộ - 2018]%[Đinh Bích Hảo, dự án(12EX-10)]%[2H3B3-7]
	Trong KG $Oxyz$, cho $A(0;1;-1), B(-2;3;1)$ và mặt cầu $(S)\colon x^2+y^2+z^2+2x-4y=0$. Đường thẳng $AB$ và mặt cầu $(S)$ có bao nhiêu điểm chung?
	\choice
	{$0$}
	{$1$}
	{\True $2$}
	{Vô số}
	\loigiai{
	Mặt cầu có tâm $I(-1;2;0)$ và PTĐT $AB\colon \heva{&x=-t\\&y=1+t\\&z=-1+t}$.\\
	Ta thấy $I$ thuộc $AB$, suy ra đường thẳng $AB$ và mặt cầu $(S)$ có $2$ điểm chung.
	}
	\end{ex}
	\begin{ex}%[12-TN-BGD-1]%[Phạm Lâm]%[2H3B3-7]
	Trong KG $Oxyz$, cho đường thẳng $d\colon\dfrac{x-2}{3}=\dfrac{y+1}{1}=\dfrac{z+5}{-1}$ và mặt phẳng $(P)\colon2x-3y+z-6=0$. Đường thẳng nằm trong mặt phẳng $(P)$, cắt và vuông góc với $d$ có phương trình là
	\choice
	{$\dfrac{x+4}{2}=\dfrac{y+3}{5}=\dfrac{z+3}{11}$}
	{\True $\dfrac{x-8}{2}=\dfrac{y-1}{5}=\dfrac{z+7}{11}$}
	{$\dfrac{x-4}{2}=\dfrac{y-3}{5}=\dfrac{z-3}{11}$}
	{$\dfrac{x+8}{2}=\dfrac{y+1}{5}=\dfrac{z-7}{11}$}
	\loigiai{
	Ta có $\overrightarrow{n}_{P}=(2;-3;1)$, $\overrightarrow{u}_{d}=(3;1;-1)$.\\
	Gọi đường thẳng cần tìm là $\Delta$. Theo đề $\overrightarrow{u}_{\Delta}=\left[\overrightarrow{u}_{d};\overrightarrow{n}_{P} \right]=(2;5;11)$.\\
	Tọa độ giao điểm $A$ của $d$ và $(P)$ là nghiệm của hệ $\heva{x&=2+3t\\y&=-1+t\\z&=-5-t\\2x&-3y+z-6=0}\Leftrightarrow \heva{t&=2\\x&=8\\y&=1\\z&=-7.}$\\
	Phương trình của đường thẳng $\Delta$ đi qua $A(8;1;-7)$ có vec-tơ chỉ phương $\overrightarrow{u}_{\Delta}$ là
	$$\dfrac{x-8}{2}=\dfrac{y-1}{5}=\dfrac{z+7}{11}.$$
	}
	\end{ex}
	\begin{ex}%[12-TN-BGD-4]%[Trịnh Xuân]%[2H3B3-7]
	Trong không gian với hệ trục tọa độ $Oxyz$, cho điểm $A(0;0;2)$ và đường thẳng $d\colon \dfrac{x-1}{2}=\dfrac{y-1}{-1}=\dfrac{z}{1}$. Phương trình mặt phẳng $(P)$ đi qua $A$ và vuông góc với $d$ là
	\choice
	{$2x-y+2z+5=0$}
	{$2x-y+z+2=0$}
	{\True $2x-y+z-2=0$}
	{$2x-y+z-4=0$}
	\loigiai{
	vì $d\perp (P)$ nên một véc-tơ pháp tuyến của $(P)$ có tọa độ là $\vec{n}_{(P)}=(2;-1;1)$.\\
	Vậy phương trình mặt phẳng $(P)$ là $2(x)-y+(z-2)=0\Rightarrow 2x-y+z-2=0$.
	}
	\end{ex}
	\begin{dang}{Các bài toán cực trị}
	\end{dang}
\setcounter{subsubsection}{0}
\setcounter{vd}{0}
\setcounter{ex}{0}
	\subsubsection{Ví dụ minh hoạ}
	\begin{vd}%[2H3G3-8]
	Trong KG $Oxyz$, cho điểm $E(2;1;3)$, mặt phẳng $(P)\colon 2x+2y-z-3=0$ và mặt cầu $(S)\colon (x-3)^2+(y-2)^2+(z-5)^2=36$. Gọi $\Delta$ là đường thẳng đi qua $E$, nằm trong $(P)$ và cắt $(S)$ tại hai điểm có khoảng cách nhỏ nhất. Biết $\Delta$ có một vectơ chỉ phương $\vec{u}=(2018;y_0;z_0)$. Tính $T=z_0-y_0$.
	\loigiai{
	\begin{center}
	\begin{tikzpicture}[>=stealth,scale= 1.3]
	\def\R{2}
	\def\r{0.4}
	\pgfmathsetmacro\l{\R*sin(60)}
	\tkzDefPoints{0/0/I,2/0/M,-2/0/N,-4/ -1.58/P}
	\tkzDefPointBy[rotation= center I angle -30](M) \tkzGetPoint{M'}
	\tkzDefPointBy[rotation= center I angle 30](N) \tkzGetPoint{N'}
	\tkzDefPointBy[rotation= center I angle -10](M) \tkzGetPoint{M''}
	\tkzDefPointBy[rotation= center I angle 10](N) \tkzGetPoint{N''}
	\tkzDefPointBy[rotation= center I angle 53](N) \tkzGetPoint{Q}
	\tkzDefPointBy[rotation= center I angle -53](M) \tkzGetPoint{W}
	\coordinate (J') at ($(M'')!1.05!(N'')$);
	\coordinate (J'') at ($(N'')!1.3!(M'')$);
	\tkzDefMidPoint(M',N')\tkzGetPoint{H}
	\coordinate (A) at ($(H)+({\l*cos(40)},{\r*sin(40)})$);
	\coordinate (B) at ($(H)+({\l*cos(-70)},{\r*sin(-70)})$);
	\tkzDefMidPoint(A,B)\tkzGetPoint{E}
	\coordinate (J) at ($(E)!2.5!(B)$);
	\coordinate (K) at ($(E)!2.5!(A)$);
	\coordinate (L) at ($(J'')+(P)-(J')$);
	\tkzDrawPoints[size=4](M',A,I,N',H,B,E)
	\tkzDrawSegments[dashed, line width=0.8pt](A,B I,E H,A H,B H,E I,H A,K M'',N'')
	\tkzDrawSegments(J',N'' J'',M'' J',P P,L L,J'')
	\draw (B)--(J);
	\draw[line width=1pt] (N') arc (180:360: \l cm and \r cm);
	\draw[dashed,line width=1pt] (N') arc (180:0: \l cm and \r cm); 
	\tkzDrawArc[dashed,line width=0.6pt](I,N'')(Q)
	\tkzDrawArc[dashed,line width=0.6pt](I,W)(M'')
	\tkzDrawArc[line width=0.6pt](I,Q)(W)
	\tkzDrawArc[line width=0.6pt](I,M'')(N'')
	\draw (J) node[right]{\small $\Delta$};
	\tkzLabelPoints[font=\footnotesize ,xshift=-8,yshift=1](A,E,B)
	\tkzLabelPoints[below,font=\footnotesize ](H);
	\tkzLabelPoints[left,font=\footnotesize](I);
	\tkzMarkAngles[arc=l,size=0.8 cm](L,P,J')
	\tkzLabelAngle[pos=0.6](L,P,J'){$P$}
	\end{tikzpicture}
	\end{center}
	Mặt cầu $(S)$ có tâm $I(3;2;5)$ và bán kính $R=6$.\\
	$IE=\sqrt{1^2+1^2+2^2}=\sqrt{6}<R\Rightarrow $ điểm $E$ nằm trong mặt cầu $(S)$.\\
	Gọi $H$ là hình chiếu của $I$ trên mặt phẳng $(P)$, $A$ và $B$ là hai giao điểm của $\Delta$ với $(S)$.\\
	Khi đó, $AB$ nhỏ nhất $\Leftrightarrow AB\perp HE$, mà $AB\perp IH$ nên $AB\perp (HIE)\Rightarrow AB\perp IE$.\\
	Suy ra $\vec{u}_{\Delta}=\left[\vec{n}_{(P)},\vv{EI}\right]=(5;-5;0)=5(1;-1;0)$.\\
	Suy ra $\vec{u}=(2018;-2018;0)$, do đó $T=z_0-y_0=2018$.
	}
	\end{vd}
	\begin{vd}%[Sở GD\&ĐT Hà Nội-2017]%[2H3G3-8]
	Trong KG $Oxyz$, cho điểm $M\left(\dfrac{1}{2};\dfrac{\sqrt{3}}{2};0\right)$ và mặt cầu $(S)\colon x^2+y^2+z^2=8$. Đường thẳng $d$ thay đồi, đi qua điểm $M$, cắt mặt cầu $(S)$ tại hai điểm phân biệt. Tính diện tích lớn nhất $S$ của tam giác $OAB$.
	\loigiai{
	\begin{center}
	\begin{tikzpicture}[>=stealth,scale= 1.3,font=\footnotesize ]
	\tkzDefPoints{0/0/O,2/0/L,-2/0/Q}
	\tkzDefPointBy[rotation= center O angle -30](L) \tkzGetPoint{A}
	\tkzDefPointBy[rotation= center O angle 30](Q) \tkzGetPoint{B}
	\tkzDefMidPoint(A,B)\tkzGetPoint{H}
	\coordinate (M) at ($(A)!.7!(B)$);
	\draw (O) circle (2cm);
	\tkzDrawPoints(A,B,O,M,H)
	\tkzDrawSegments[dashed](O,A O,B A,B O,H O,M)
	\tkzLabelPoints[below](A,B,M,H)
	\tkzLabelPoints[above left](O)
	\tkzMarkRightAngles[size=0.1](O,H,A)
	\end{tikzpicture}
	\end{center}
	Mặt cầu $(S)$ có tâm $O(0;0;0)$ và bán kính $R=2\sqrt{2}$.\\
	Vì $OM=1<R$ nên $M$ thuộc miền trong của mặt cầu $(S)$. Gọi $A,B$ là giao điểm của đường thẳng với mặt cầu. Gọi $H$ là chân đường cao hạ từ $O$ của tam giác $OAB$.\\
	Đặt $OH=x$, ta có $0<x\le OM=1$, đồng thời $HA=\sqrt{R^2-OH^2}=\sqrt{8-x^2}$. Vậy diện tích tam giác $OAB$ là\\
	$S_{\Delta OAB}=\dfrac{1}{2}OH.AB=OH.HA=x\sqrt{8-x^2}$.\\
	Khảo sát hàm số $f(x)=x\sqrt{8-x^2}$ trên $(0;1]$, ta đượcc $\max\limits_{(0;1]}f(x)=f(1)=\sqrt{7}$.\\
	Vậy giá trị lớn nhất của $S_{\Delta OAB}=\sqrt{7}$, đạt được khi $x=1$ hay $M\equiv H$, nói cách khác là $d\perp OM$.
	}
	\end{vd}
	\begin{vd}%[2H3G3-8]
	Trong KG $Oxyz$, cho điểm $E(1;1;2)$, mặt phẳng $(P)\colon x+y+z-4=0$ và mặt cầu $(S)\colon x^2+y^2+z^2=9$. Gọi $\Delta$ là đường thẳng qua $E$, nằm trong mặt phẳng $(P)$ và cắt mặt cầu $(S)$ tại hai điểm có khoảng cách nhỏ nhất. Lập phương trình của $\Delta$.
	\loigiai{
	\begin{center}
	\begin{tikzpicture}[>=stealth,scale= 1.3]
	\def\R{2}
	\def\r{0.4}
	\pgfmathsetmacro\l{\R*sin(60)}
	\tkzDefPoints{0/0/I,2/0/M,-2/0/N,-4/ -1.58/P}
	\tkzDefPointBy[rotation= center I angle -30](M) \tkzGetPoint{M'}
	\tkzDefPointBy[rotation= center I angle 30](N) \tkzGetPoint{N'}
	\tkzDefPointBy[rotation= center I angle -10](M) \tkzGetPoint{M''}
	\tkzDefPointBy[rotation= center I angle 10](N) \tkzGetPoint{N''}
	\tkzDefPointBy[rotation= center I angle 53](N) \tkzGetPoint{Q}
	\tkzDefPointBy[rotation= center I angle -53](M) \tkzGetPoint{W}
	\coordinate (J') at ($(M'')!1.05!(N'')$);
	\coordinate (J'') at ($(N'')!1.3!(M'')$);
	\tkzDefMidPoint(M',N')\tkzGetPoint{H}
	\coordinate (A) at ($(H)+({\l*cos(40)},{\r*sin(40)})$);
	\coordinate (B) at ($(H)+({\l*cos(-70)},{\r*sin(-70)})$);
	\tkzDefMidPoint(A,B)\tkzGetPoint{E}
	\coordinate (J) at ($(E)!2.5!(B)$);
	\coordinate (K) at ($(E)!2.5!(A)$);
	\coordinate (L) at ($(J'')+(P)-(J')$);
	%\draw (I) circle (2cm);
	\tkzDrawPoints[size=4](M',A,I,N',H,B,E)
	\tkzDrawSegments[dashed, line width=0.8pt](A,B I,E H,A H,B H,E I,H A,K M'',N'')
	\tkzDrawSegments(J',N'' J'',M'' J',P P,L L,J'')
	\draw (B)--(J);
	\draw[dashed,line width=0.8pt] (N') arc (180:0: \l cm and \r cm); 
	\tkzDrawArc[dashed,line width=0.6pt](I,N'')(Q)
	\tkzDrawArc[dashed,line width=0.6pt](I,W)(M'')
	\tkzDrawArc[line width=0.6pt](I,Q)(W)
	\tkzDrawArc[line width=0.6pt](I,M'')(N'')
	\draw (J) node[right]{\small $\Delta$};
	\tkzLabelPoints[font=\footnotesize ,xshift=-8,yshift=1](A,E,B)
	\tkzLabelPoints[below,font=\footnotesize ](H);
	\tkzLabelPoints[left,font=\footnotesize](I);
	\tkzMarkAngles[arc=l,size=0.8 cm](L,P,J')
	\tkzLabelAngle[pos=0.6](L,P,J'){$P$}
	\end{tikzpicture}
	\end{center}
	Mặt cầu $(S)$ có tâm $I(0;0;0)$ và bán kính $R=3$.\\
	$IE=\sqrt{1^2+1^2+2^2}=\sqrt{6}<R\Rightarrow $ điểm $E$ nằm trong mặt cầu $(S)$.\\
	Gọi $H$ là hình chiếu của $I$ lên $(P)$, $A,B$ là hai giao điểm của $\Delta$ với $(S)$.\\
	Khi đó, $AB$ nhỏ nhất khi $\Leftrightarrow AB\perp HE$, mà $AB\perp OH$ nên $AB\perp \left(HOE\right)\Rightarrow AB\perp OE$.\\
	Suy ra $\vec{u}_{\Delta}=\left[\vec{n}_{(P)},\vv{IE}\right]=(-1;1;0)$. Vậy phương trình của $\Delta$ là $\heva{&x=1-t\\&y=1+t\\&z=2}$.
	}
	\end{vd}
	\begin{vd}%[2H3G3-8]
	Trong KG $Oxyz$, cho điểm $E(1;1;1)$, mặt cầu $(S)\colon x^2+y^2+z^2=4$ và mặt phẳng $(P)\colon x-3y+5z-3=0$. Gọi $\Delta$ là đường thẳng qua $E$, nằm trong $(P)$ và cắt $(S)$ tại hai điểm $A,B$ sao cho tam giác $OAB$ đều. Lập phương trình của $\Delta$.
	\loigiai{
	\begin{center}
	\begin{tikzpicture}[>=stealth,scale= 1.3]
	\def\R{2}
	\def\r{0.4}
	\pgfmathsetmacro\l{\R*sin(60)}
	\tkzDefPoints{0/0/I,2/0/M,-2/0/N,-4/ -1.58/P}
	\tkzDefPointBy[rotation= center I angle -30](M) \tkzGetPoint{M'}
	\tkzDefPointBy[rotation= center I angle 30](N) \tkzGetPoint{N'}
	\tkzDefPointBy[rotation= center I angle -10](M) \tkzGetPoint{M''}
	\tkzDefPointBy[rotation= center I angle 10](N) \tkzGetPoint{N''}
	\tkzDefPointBy[rotation= center I angle 53](N) \tkzGetPoint{Q}
	\tkzDefPointBy[rotation= center I angle -53](M) \tkzGetPoint{W}
	\coordinate (J') at ($(M'')!1.05!(N'')$);
	\coordinate (J'') at ($(N'')!1.3!(M'')$);
	\tkzDefMidPoint(M',N')\tkzGetPoint{H}
	\coordinate (A) at ($(H)+({\l*cos(40)},{\r*sin(40)})$);
	\coordinate (B) at ($(H)+({\l*cos(-70)},{\r*sin(-70)})$);
	\tkzDefMidPoint(A,B)\tkzGetPoint{E}
	\coordinate (J) at ($(E)!2.5!(B)$);
	\coordinate (K) at ($(E)!2.5!(A)$);
	\coordinate (L) at ($(J'')+(P)-(J')$);
	\tkzDrawPoints[size=4](M',A,I,N',H,B,E)
	\tkzDrawSegments[dashed, line width=0.8pt](A,B I,E H,A H,B H,E I,H A,K M'',N'')
	\tkzDrawSegments(J',N'' J'',M'' J',P P,L L,J'')
	\draw (B)--(J);
	\draw[line width=0.8pt] (N') arc (180:360: \l cm and \r cm);
	\draw[dashed,line width=0.8pt] (N') arc (180:0: \l cm and \r cm); 
	\tkzDrawArc[dashed,line width=0.6pt](I,N'')(Q)
	\tkzDrawArc[dashed,line width=0.6pt](I,W)(M'')
	\tkzDrawArc[line width=0.6pt](I,Q)(W)
	\tkzDrawArc[line width=0.6pt](I,M'')(N'')
	\draw (J) node[right]{\small $\Delta$};
	\tkzLabelPoints[font=\footnotesize ,xshift=-8,yshift=1](A,E,B)
	\tkzLabelPoints[below,font=\footnotesize ](H);
	\tkzLabelPoints[left,font=\footnotesize](I);
	\tkzMarkAngles[arc=l,size=0.8 cm](L,P,J')
	\tkzLabelAngle[pos=0.6](L,P,J'){$P$}
	\end{tikzpicture}
	\end{center}
	Mặt cầu $(S)$ có tâm $I(0;0;0)$ và bán kính $R=2$.\\
	$IE=\sqrt{1^2+1^2+1^2}=\sqrt{3}<R\Rightarrow$ điểm $E$ nằm trong mặt cầu $(S)$.\\
	Gọi $K$ là hình chiếu của $I$ lên $AB$. Vì $\Delta IAB$ đều nên $OK=\dfrac{IA\sqrt{3}}{2}=\dfrac{R\sqrt{3}}{2}=\sqrt{3}=IE$.\\
	Suy ra $K\equiv E$. Do đó $AB\perp IE$. Suy ra $\vec{u}_{\Delta}=\left[\vec{u}_{(P)},\vv{IE}\right]=(8;-4;-4)=4(2;-1;-1)$.\\
	Vậy phương trình của $\Delta$ là $d\colon \dfrac{x-1}{2}=\dfrac{y-1}{-1}=\dfrac{z-1}{-1}$. 
	}
	\end{vd}
	\subsubsection{Bài tập trắc nghiệm}
	\begin{ex}%[Đề thi thử lần 1, Hoàng Văn Thụ, Hòa Bình, 2018]%[Trần Quang Thạnh, dự án ID6, 12EX-6-2018]%[2H3K3-8]
	Trong hệ tọa độ $Oxyz$, cho hai điểm $A(-1;2;1), B(1;2;-3)$ và đường thẳng $d \colon \dfrac{x+1}{2}=\dfrac{y-5}{2}=\dfrac{z}{-1}$. Tìm véc-tơ chỉ phương $\vec{u}$ của đường thẳng $\Delta$ đi qua $A$ và vuông góc với $d$ đồng thời cách $B$ một khoảng lớn nhất.
	\choice
	{\True $\vec{u}=(4;-3;2) $}
	{$\vec{u}=(2;0;-4) $}
	{$\vec{u}=(2;2;-1) $}
	{$\vec{u}=(1;0;2) $}
	\loigiai{Gọi $H$ là hình chiếu của $B$ lên $\Delta$. Ta có $\mathrm{\,d}(A;\Delta)=AH\leq AB$ và đẳng thức xảy ra khi $H\equiv A$.\\
	Do đó, nếu gọi $\vec{u}$ là véc-tơ chỉ phương của $\Delta$ thì $\vec{u}\perp \vec{AB}$ và $\vec{u}\perp \vec{u_d}$, với $\vec{AB}=(2;0;-4)$ và $\vec{u_d}=(2;2;-1)$.\\
	Khi đó, ta có thể chọn $\vec{AB}\wedge \vec{u_d}=(8;-6;4)=2(4;-3;2)$ làm véc-tơ chỉ phương của $\Delta$.
	}
	\end{ex}
	\begin{ex}%[Đề thi thử lần 1, Hoàng Văn Thụ, Hòa Bình, 2018]%[Trần Quang Thạnh, dự án ID6, 12EX-6-2018]%[2H3K3-8]
	Trong KG $Oxyz$, cho hai điểm $M(0;1;3), N(10;6;0)$ và mặt phẳng $(P) \colon x-2y+2z-10=0$. Biết rằng tồn tại điểm $I(-10;a;b)$ thuộc $(P)$ sao cho $|IM-IN|$ đạt giá trị lớn nhất. Tính $T=a+b$.
	\choice
	{$T=5 $}
	{$T=1 $}
	{\True $T=2 $}
	{$T=6 $}
	\loigiai{Thay tọa độ điểm $M$ và $N$ vào vế trái phương trình mặt phẳng $(P)$, ta có $(0-2+3-10)\cdot (10-12-10)>0$ nên hai điểm $M, N$ nằm cùng phía đối với mặt phẳng $(P)$.\\
	Khi đó ta có $|IM-IN|\leq MN$ và đẳng thức xảy ra khi $I=MN\cap (P)$.\\
	PTTS của đường thẳng $MN$ là $\heva{x&=10t\\y&=1+5t\\z&=3-3t.}$\\
	Tọa độ giao điểm của $MN$ và $(P)$ là nghiệm hệ phương trình $$\heva{x&=10t\\y&=1+5t\\z&=3-3t\\x&-2y+2z-10=0} \Rightarrow \heva{x&=-10\\y&=-4\\z&=6.}$$
	Vậy $T=a+b=2$.
	}
	\end{ex}
	\begin{ex}%[Thi HK2, Sở GD\&ĐT Đồng Tháp, 2018]%[2H3K3-8]%[Trần Hòa, dự án (12EX-8)]
	Trong KG $Oxyz$, cho mặt cầu $(S) \colon (x-1)^2 = (y+2)^2 +z^2 = 4$ có tâm $I$ và mặt phẳng $(P)\colon 2x-y+2z+2=0$. Tìm tọa độ điểm $M$ thuộc $(P)$ sao cho đoạn thẳng $IM$ ngắn nhất.
	\choice
	{\True $\left(-\dfrac{1}{3};-\dfrac{4}{3};-\dfrac{4}{3}\right)$}
	{$\left(-\dfrac{11}{9};-\dfrac{8}{9};-\dfrac{2}{9}\right)$}
	{$(1;-2;2)$}
	{$(1;-2;-3)$}
	\loigiai{
	Mặt cầu $(S)$ có tâm $I(1;-2;0)$ và bán kính $R=2$.\\
	Khoảng cách từ $I$ đến mặt phẳng $(P)$ ngắn nhất khi và chỉ khi $M$ là hình chiếu vuông góc của $I$ lên mặt phẳng $(P)$.\\
	Đường thẳng đi qua $I$ và vuông góc với mặt phẳng $(P)$ có PTTS là
	$\begin{cases}
	x=1+2t\\
	y=-2-t\\
	z=2t
	\end{cases}$. Khi đó tọa độ điểm $M$ là nghiệm của hệ phương trình\\
	$\begin{cases}
	x=1+2t\\
	y=-2-t\\
	z=2t\\
	2x-y+2z+2=0
	\end{cases}$
	$\Leftrightarrow$
	$\begin{cases}
	x=1+2t\\
	y=-2-t\\
	z=2t\\
	2(1+2t)-(2-t)+2(2t)+2=0
	\end{cases}$
	$\Leftrightarrow$
	$\begin{cases}
	x=-\dfrac{1}{3}\\
	y=-\dfrac{4}{3}\\
	z=-\dfrac{4}{3}\\
	t=-\dfrac{2}{3}.
	\end{cases}$\\
	Vậy tọa độ điểm $M$ là $\left(-\dfrac{1}{3};-\dfrac{4}{3};-\dfrac{4}{3}\right)$.
	}
	\end{ex}
	\begin{ex}%[Thi học kì 2, Sở Giáo Dục và Đào Tạo Bạc Liêu, 2018]%[2H3K3-8]%[Đặng Viết Quân, dự án (12EX-8)]
	Trong KG $Oxyz$, cho hai điểm $E(1;-2;4), F(1;-2;-3)$. Gọi $M$ là điểm thuộc mặt phẳng $(Oxy)$ sao cho tổng $ME+MF$ có giá trị nhỏ nhất. Tìm tọa độ điểm $M$.
	\choice 
	{$M(-1;2;0)$}
	{$M(-1;-2;0)$}
	{\True $M(1;-2;0)$}
	{$M(1;2;0)$}
	\loigiai{
	Mặt phẳng $(Oxy)$ có phương trình $z=0$.\\
	Thay tọa độ điểm $E,F$ vào phương trình mặt phẳng $(Oxy)$ ta được $4\cdot (-3)<0$. Vậy $E,F$ nằm khác phía so với mặt phẳng $(Oxy)$.\\
	$ME+MF$ nhỏ nhất khi $3$ điểm $M,E,F$ thẳng hàng hay $\vec{ME}$ cùng phương với $\vec{MN}$.\\
	Do $M\in (Oxy)$ nên đặt $M(a;b;0)$.\\
	$\vec{ME}=(1-a;-2-b;4)$, $\vec{EF}=(0;0;-7)$.\\
	Từ đó ta được $\heva{&1-a=0\\&-2-b=0}\Leftrightarrow \heva{&a=1\\&b=-2}$.\\
	Vậy $M(1;-2;0)$.
	}
	\end{ex}
	\begin{ex}%[Đề KSCL trường THPT chuyên Hùng Vương, Phú Thọ, năm 2018, lần 4]%[Nguyễn Thành Khang, 12-Ex-9]%[2H3K3-8]
	Trong không gian với hệ toạ độ $Oxyz$, cho mặt phẳng $(P)\colon x-2y+z-1=0$ và điểm $A(0;-2;3)$, $B(2;0;1)$. Điểm $M(a;b;c)$ thuộc $(P)$ sao cho $MA+MB$ nhỏ nhất. Giá trị của $a^2+b^2+c^2$ bằng
	\choice
	{$\dfrac{41}{4}$}
	{\True $\dfrac{9}{4}$}
	{$\dfrac{7}{4}$}
	{$3$}
	\loigiai{
	\immini{
	Ta có $\left(0-2\cdot (-2)+3-1\right)\left(2-2\cdot 0+1-1\right)=12>0$, nên $A,B$ nằm cùng phía với mặt phẳng $(P)$.\\
	Gọi $C$ là điểm đối xứng của $A$ qua mặt phẳng $(P)$.\\
	Khi đó ta có $MA+MB=MC+MB\ge BC$. Dẫn tới $MA+MB$ nhỏ nhất khi và chỉ khi $M$ trùng với $I$ là giao điểm của $BC$ với mặt phẳng $(P)$.\\
	PTĐT $AC$ là $\dfrac{x}{1}=\dfrac{y+2}{-2}=\dfrac{z-3}{1}$.\\
	Toạ độ giao điểm $H$ của $AC$ với mặt phẳng $(P)$ là nghiệm của hệ $\heva{&x-2y+z-1=0 \\ &\dfrac{x}{1}=\dfrac{y+2}{-2}=\dfrac{z-3}{1}}\Leftrightarrow \heva{&x=-1 \\ &y=0 \\ &z=2}$, hay $H(-1;0;2)$.
	}{
	\begin{tikzpicture}
	\tkzInit[xmin=-2,xmax=4.1,ymin=-0.5,ymax=8]
	\tkzClip
	\tkzDefPoints{0/0/C,0/7/A,2/6/B,1.5/4.5/G,0/4.5/D,0.75/2.25/E,0/2.25/F,-1.5/2.25/X,-0.75/4.5/Y,4/4.5/P,3.25/2.25/T,2.5/3/M}
	\tkzDefMidPoint(B,C)\tkzGetPoint{I}
	\tkzDefMidPoint(C,A)\tkzGetPoint{H}
	\tkzInterLL(A,M)(B,I)\tkzGetPoint{K}
	\tkzInterLL(C,M)(X,T)\tkzGetPoint{J}
	\tkzInterLL(B,M)(Y,P)\tkzGetPoint{L}
	\tkzDrawSegments(B,K A,H C,E C,F X,Y Y,D L,P P,T T,X A,M B,M C,J A,I)
	\tkzDrawPoints(A,B,C,H,I,M)
	\tkzMarkAngles[size=0.75](G,P,T)
	\tkzDrawSegments[dashed](D,L K,E H,F J,M)
	\tkzLabelPoints[above](B,A)
	\tkzLabelPoints[below](C)
	\tkzLabelPoints[below right](I,M)
	\tkzLabelPoints[below left](P)
	\tkzLabelPoints[left](H)
	\end{tikzpicture}
	}
	\noindent $H$ là trung điểm $AC$ nên toạ độ $C$ là $C(-2;2;1)$.\\
	Đường thẳng $BC$ đi qua $B(2;0;1)$ có véc-tơ chỉ phương $\vec{BC}=(-4;2;0)$ là $\heva{&x=2-4t \\ &y=2t \\ &z=1}$.\\
	Toạ độ $I$ là nghiệm của hệ $\heva{&x=2-4t \\ &y=2t \\ &z=1 \\ &x-2y+z-1=0}\Leftrightarrow \heva{&t=\dfrac{1}{4} \\ &x=1 \\ &y=\dfrac{1}{2} \\ &z=1}$, hay $I\left(1;\dfrac{1}{2};1\right)$.\\
	Vậy ta có $a=1;b=\dfrac{1}{2};c=1$, dẫn tới $a^2+b^2+c^2=\dfrac{9}{4}$.
	}
	\end{ex}
	\begin{ex}%[Đề KSCL Toán 12 THPT năm học 2017 – 2018 sở GD và ĐT Thanh Hóa]%[Phạm Doãn Lê Bình, 12-Ex-9]%[2H3K3-8]
	Trong không gian với hệ trục tọa độ $Oxyz$, cho điểm $A(2;-1;-2)$ và đường thẳng $(d)$ có phương trình $\dfrac{x-1}{1} = \dfrac{y-1}{-1} = \dfrac{z-1}{1}$. Gọi $(P)$ là mặt phẳng đi qua điểm $A$, song song với đường thẳng $(d)$ và khoảng cách từ đường thẳng $(d)$ tới mặt phẳng $(P)$ là lớn nhất. Khi đó, mặt phẳng $(P)$ vuông góc với mặt phẳng nào sau đây?
	\choice
	{$x-y-z-6=0$}
	{$x+3y+2z+10=0$}
	{$x-2y-3z-1=0$}
	{\True $3x+z+2=0$}
	\loigiai{ 
	\immini{Đường thẳng $(d)$ có véc-tơ chỉ phương $\vec{u}=(1;-1;1)$.\\
	Gọi $K$ là hình chiếu vuông góc của $A$ trên $d$.\\
	Do $K \in d$ nên $K(1+t; 1-t; 1+t)$ ($t\in \mathbb{R}$).\\
	$\vec{AK}= (t-1; 2-t; t+3)$.
	}{
	\begin{tikzpicture}[scale=0.8,>=stealth]
	\tkzDefPoint(0,0){a}
	\tkzDefShiftPoint[a](0:6){b}
	\tkzDefShiftPoint[a](-130:2.5){d}
	\tkzDefShiftPoint[a](-70:1.5){A}
	\tkzDefShiftPoint[A](60:3){K}
	\tkzDefPointBy[translation = from a to b](d)\tkzGetPoint{c}
	\tkzDefShiftPoint[K](0:3){k1}
	\tkzDefShiftPoint[K](0:-3){k2}
	\tkzDefShiftPoint[K](-90:2){H}
	\tkzInterLL(A,K)(a,b)\tkzGetPoint{a1}
	\tkzInterLL(H,K)(a,b)\tkzGetPoint{a2}
	\tkzDrawSegments(a,a1 a2,b b,c c,d d,a k1,k2 A,K A,H K,H)
	\tkzDrawSegments[dashed](a1,a2)
	\tkzDrawPoints(A,K,H) 
	\tkzLabelPoints[above](K)
	\tkzLabelPoints[right](H)
	\tkzLabelPoints[left](A)
	\tkzMarkRightAngles(A,K,k2 K,H,A)
	\end{tikzpicture}
	}
	Ta có $AK \perp d \Leftrightarrow \vec{AK} \perp \vec{u} \Leftrightarrow \vec{AK} \cdot \vec{u} = 0$.\\
	$\Leftrightarrow t-1-(2-t)+t+3 = 0 \Leftrightarrow t = 0$. Vậy $K(1;1;1)$.\\
	Ta có $\mathrm{d}((d),(P)) = \mathrm{d}(K,(P)) = KH \le KA = 14.$\\
	$\max \mathrm{d}((d),(P)) = \sqrt{14} \Leftrightarrow (P)$ đi qua $A$ và có véc-tơ pháp tuyến $\vec{KA} = (-1;2;3)$.\\
	Suy ra mặt phẳng $(P)$ vuông góc với mặt phẳng $3x+z+2=0$ vì có tích vô hướng hai véc-tơ pháp tuyến bằng $0$.
	}
	\end{ex}
	\begin{ex}%[2-GHK2-96-ThithuTHTT-Lan7]%[2H3K3-8]%[Phạm Tuấn, 12EX-9]
	Trong KG $Oxyz$ cho điểm $A(3;-1;0)$ và đường thẳng $d\colon\dfrac{x-2}{-1}=\dfrac{y+1}{2} = \dfrac{z-1}{1}$. Mặt phẳng $(\alpha)$ chứa $d$ sao cho khoảng cách từ $A$ đến $(\alpha)$ lớn nhất có phương trình là 
	\choice
	{\True $x+y-z=0$}
	{$x+y-z-2=0$}
	{$x+y-z+1=0$}
	{$-x+2y+z+5=0$}
	\loigiai{
	Gọi $K$ là hình chiếu vuông góc của $A$ trên $d$, dễ thấy $\mathrm{d}(A,(\alpha)) \leq AK$, dấu $"="$ xảy ra khi $\overrightarrow{AK}$ là véc-tơ pháp tuyến của $(\alpha)$. \\
	Ta có $K(2-t;2t-1;t+1)$, $\overrightarrow{AK}=(-t-1;2t;t+1)$, $\overrightarrow{AK} \cdot \overrightarrow{u}_{\mathrm{d}} =0 \Leftrightarrow t=-\dfrac{1}{3}\Rightarrow\overrightarrow{AK}= \left ( -\dfrac{2}{3};-\dfrac{2}{3};\dfrac{2}{3}\right )$. \\
	Phương trình mặt phẳng $(\alpha)$ là $(x-2)+(y+1)-(z-1)=0 \Leftrightarrow x+y-z=0$. 
	}
	\end{ex}
	\begin{ex}%[Đề HK2, 2018, THPT Đa Phúc - Hà Nội]%[Nguyễn Hữu Nhân, Dự án EX9]%[2H3K3-8]
	Trong KG $Oxyz$, cho hai điểm $A(0;1;1)$, $B(1;2;1)$ và đường thẳng $d \colon \dfrac{x}{1}= \dfrac{y+1}{-1} =\dfrac{z-2}{-2}$. Hoành độ của điểm $M$ thuộc $d$ sao cho diện tích tam giác $MAB$ có giá trị nhỏ nhất có giá trị bằng
	\choice{$2$}{\True $0$}{$-1$}{$1$}
	\loigiai{
	Điểm $M$ thuộc $d$ có tọa độ là $\left(t;-1-t;2-2t\right)$ với $t \in \mathbb{R}$.
	\\ Khi đó $\overrightarrow{MA} = \left(-t; t+2; 2t-1\right)$ và $\overrightarrow{MB} = \left(1-t; t+3;2t-1\right)$.
	\\ Ta có $\left[\overrightarrow{MA}, \overrightarrow{MB}\right] =\left( \begin{vmatrix}
	t+2 & 2t-1 \\ t+3 & 2t-1
	\end{vmatrix}; \begin{vmatrix} 2t-1 & -t \\ 2t-1 & 1-t	\end{vmatrix}; \begin{vmatrix} -t & t+2 \\ 1-t & 3+t	\end{vmatrix} \right) = \left( -2t+1; 2t-1; -2t-2\right).$
	\\ Ta có $S_{\triangle MAB} = \dfrac{1}{2} \left| \left[\overrightarrow{MA}, \overrightarrow{MB}\right] \right| = \dfrac{1}{2} \sqrt{2(2t-1)^2+ (2t+2)^2} =\dfrac{1}{2} \sqrt{12t^2+6} \geq \dfrac{\sqrt{6}}{2}$.
	\\ Dấu đẳng thức xảy ra khi $t=0$. Do đó $\triangle MAB$ có diện tích nhỏ nhất bằng $\dfrac{\sqrt{6}}{2}$ khi $M(0;-1;2)$.}
	\end{ex}
	\begin{ex}%[Đề thi học kỳ 2 Toán 12 chuyên Lê Hồng Phong Nam Định, 2018]%[Nguyễn Sỹ, dự án(12EX-9)]%[2H3K3-8]
	Trong KG $Oxyz$, cho đường thẳng $\Delta \colon \dfrac{x-1}{2}=\dfrac{y}{1}=\dfrac{z+2}{-1}$ và hai điểm $A(0;-1;3), B(1;-2;1)$. Tìm tọa độ điểm $M$ thuộc đường thẳng $\Delta$ sao cho $MA^2+2MB^2$ đạt giá trị nhỏ nhất.
	\choice
	{$M(1;0;-2)$}
	{$M(3;1;-3)$}
	{$M(5;2;-4)$}
	{\True $M(-1;-1;-1)$}
	\loigiai
	{ Vì $M \in \Delta$ nên $M\left( 1+2t;t;-2-t\right)$.\\
	Ta có 
	\begin{eqnarray*}
	MA^2+2MB^2&=&(2t+1)^2+(t+1)^2+(t+5)^2 +2 \left[ (2t)^2+ (t+2)^2 +(t+3)^2\right]\\
	&= & 18t^2 +36t +53\\
	&= & 18(t+1)^2+35 \geq 35, \forall t \in \mathbb{R}.
	\end{eqnarray*}
	Vậy giá trị nhỏ nhất của $MA^2+2MB^2$ là $35$, xảy ra khi $t=-1$, khi đó $M(-1;-1;-1)$.
	}
	\end{ex}
	\begin{ex}%[HK2 (2017 - 2018), THPT Gia Định, Hồ Chí Minh]%[Lê Văn Thiện, dự án(12EX-9)]%[2H3K3-8]
	Trong KG $Oxyz$, cho mặt cầu $(S)\colon x^2+y^2+z^2-4x-2z-4=0$ và mặt phẳng $(P)\colon 2x+y-2z+61=0$. Điểm $M$ thay đổi trên $(S)$, điểm $N$ thay đổi trên $(P)$. Độ dài nhỏ nhất của $MN$ bằng
	\choice
	{$24$}
	{$21$}
	{$3$}
	{\True $18$}
	\loigiai{ 
	\immini{Mặt cầu có tâm $I=(2,0,1)$ và bán kính là $R=3$.\\
	Khoảng cách từ $I$ xuống mặt phẳng $(P)$ là 
	$$\mathrm{d}[I,(P)]=\dfrac{|2\cdot 2+1\cdot 0-2\cdot 1+61|}{\sqrt{2^2+1+2^2}}=21>R,$$ 
	nên mặt phẳng $(P)$ và mặt cầu $(S)$ không có điểm chung.
	}
	{
	\begin{tikzpicture}[scale=0.9]
	\tkzInit[ymin=-1.5,ymax=3.0,xmin=-2.5,xmax=2.0]
	\tkzClip
	\tkzDefPoints{0/1.5/I, 0/0/t,-1.6/-0.8/N, 1.6/-0.8/q}
	\tkzDrawCircle(I,t)
	\tkzInterLL(I,t)(N,q) \tkzGetPoint{J}
	\tkzInterLC(I,q)(I,t) \tkzGetSecondPoint{M}
	\tkzDrawSegments[black, thick](I,N N,q I,J I,M M,N)
	\tkzMarkRightAngle(I,J,q)
	\tkzLabelPoints[above](I)
	\tkzLabelPoints[below](J,M,N)
	\foreach \X in {I,M,N,J}
	{\draw[fill=black] (\X) circle (1.0pt);} 
	\end{tikzpicture}
	}
	Gọi $J$ là hình chiếu của $I$ trên mặt phẳng $(P)$ ta có
	$MN+MI \ge IN \ge IJ$
	suy ra $MN\ge IJ-R$. \\
	Do đó $MN_{\min}=IJ-R$.\\
	Mặt khác $IJ=\mathrm{d}[I,(P)]=21$.\\
	Vậy $MN_{\min}=IJ-R=21-3=18$.
	}
	\end{ex}

\Closesolutionfile{ans}
\begin{center}
	\begin{tabular}{|c|c|c|c|}
		\hline 
		1 & 2 & 3&4 \\ \hline
		5 & 6 & 7&8 \\ \hline
		9 & 10 & 11&12 \\ \hline
	\end{tabular}
	\begin{center}
		PHIẾU ĐÓNG HỌC PHÍ CỦA HỌC SINH: 
	\end{center}
\end{center}