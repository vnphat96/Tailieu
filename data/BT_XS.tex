\section{XÁC SUẤT}
\subsection{LÝ THUYẾT TÓM TẮT}
\begin{enumerate}
\item Biến cố
\begin{itemize}
 \item Không gian mẫu $\Omega$: là tập các kết quả có thể xảy ra của một phép thử.
 \item Biến cố $A$: là tập các kết quả của phép thử làm xảy ra $A$. Do đó $\Omega_A \subset \Omega$.
 \item Biến cố không: $\varnothing$. Biến cố chắc chắn: $\Omega$
 \item Biến cố đối của $A$: $\overline{A}=\Omega \setminus A$
 \item Hợp hai biến cố: $A \cup B$ 
 \item Giao hai biến cố: $A \cap B$ (hoặc $A.B$)
 \item Hai biến cố xung khắc: $A \cap B = \varnothing$
 \item Hai biến cố độc lập: nếu việc xảy ra biến cố này không ảnh hưởng đến việc xảy ra biến cố kia.
\end{itemize}
\item Xác suất
\begin{itemize}
\item Xác suất của biến cố: $P(A) = \dfrac{|\Omega_A|}{|\Omega|}$
\item $0 \le P(A) \le 1$; $P(\Omega) = 1$; $P(\varnothing) = 0$
\item Qui tắc cộng: Nếu $A \cap B = \varnothing$ thì $P(A \cup B) = P(A) + P(B)$.\\
 Mở rộng: $A$, $B$ bất kì: $P(A \cup B) = P(A) + P(B) - P(A. B)$
 \item $P(\overline{A}) = 1 - P(A)$
\item Qui tắc nhân: Nếu $A$, $B$ độc lập thì $P(A. B) = P(A) \cdot P(B)$
\end{itemize}
\end{enumerate}
\subsection{BÀI TẬP}
\Opensolutionfile{ans}[ans/ans-BT_XS]
\begin{dang}{XÁC ĐỊNH PHÉP THỬ, KHÔNG GIAN MẪU VÀ BIẾN CỐ}
\textit{Phương pháp:} Để xác định không gian mẫu và biến cố ta thường sử dụng các cách sau\\
Cách 1: Liệt kê các phần tử của không gian mẫu và biến cố rồi chúng ta đếm.\\
Cách 2: Sử dụng các quy tắc đếm để xác định số phần tử của không gian mẫu và biến cố.\\
\end{dang}
%Câu 1
\begin{ex}
Trong các thí nghiệm sau thí nghiệm nào không phải là phép thử ngẫu nhiên:
\choice
{Gieo đồng tiền xem nó mặt ngửa hay mặt sấp}
{Gieo $3$ đồng tiền và xem có mấy đồng tiền lật ngửa}
{Chọn bất kì 1 học sinh trong lớp và xem là nam hay nữ}
{\True Bỏ hai viên bi xanh và ba viên bi đỏ trong một chiếc hộp, sau đó lấy từng viên một để đếm xem có tất cả bao nhiêu viên bi}
\loigiai{
Phép thử ngẫu nhiên là phép thử mà ta chưa biết được kết quả là gì.\\
Đáp án D không phải là phép thử vì ta biết chắc chắn kết quả chỉ có thể là một số cụ thể số bi xanh và số bi đỏ.
}
\end{ex}
%Câu 2
\begin{ex}
Gieo 3 đồng tiền là một phép thử ngẫu nhiên có không gian mẫu là:
\choice
{$\left\{ NN,NS,SN,SS \right\}$}
{$\left\{ NNN, SSS, NNS, SSN, NSN, SNS \right\}$}
{\True $\left\{ NNN,SSS,NNS,SSN,NSN,SNS,NSS,SNN \right\}$}
{$\left\{ NNN,SSS,NNS,SSN,NSS,SNN \right\}$}
\loigiai{
Liệt kê các phần tử.
}
\end{ex}
%Câu 3
\begin{ex}
Gieo một đồng tiền và một con súcsắc. Số phần tử của không gian mẫu là:
\choice
{$24$}
{\True $12$}
{$6$}
{$8$}
\loigiai{
Mô tả không gian mẫu ta có: $\Omega =\left\{ S1;S2;S3;S4;S5;S6;N1;N2;N3;N4;N5;N6 \right\}$.
}
\end{ex}
%Câu 4
\begin{ex}
Gieo 2 con súc sắc và gọi kết quả xảy ra là tích số hai nút ở mặt trên. Số phần tử của không gian mẫu là:
\choice
{$9$}
{\True $18$}
{$29$}
{$39$}
\loigiai{
Mô tả không gian mẫu ta có: $\Omega =\left\{ 1;2;3;4;5;6;8;9;10;12;15;16;18;20;24;25;30;36 \right\}$.
}
\end{ex}
%Câu 5
\begin{ex}
Gieo con súc sắc hai lần. Biến cố A là biến cố để sau hai lần gieo có ít nhất một mặt 6 chấm :
\choice
{$A=\left\{ (1;6),(2;6),(3;6),(4;6),(5;6) \right\}$}
{$A=\left\{ \left(1{,}6\right),\left(2{,}6\right),\left(3{,}6\right),\left(4{,}6\right),\left(5{,}6\right),\left(6{,}6\right) \right\}$}
{\True $A=\left\{ \left(1{,}6\right),\left(2{,}6\right),\left(3{,}6\right),\left(4{,}6\right),\left(5{,}6\right),\left(6{,}6\right),\left(6{,}1\right),\left(6{,}2\right),\left(6{,}3\right),\left(6{,}4\right),\left(6{,}5\right) \right\}$}
{$A=\left\{ \left(6{,}1\right),\left(6{,}2\right),\left(6{,}3\right),\left(6{,}4\right),\left(6{,}5\right) \right\}$}
\loigiai{
Liệt kê ta có: $A=\left\{ \left(1{,}6\right),\left(2{,}6\right),\left(3{,}6\right),\left(4{,}6\right),\left(5{,}6\right),\left(6{,}6\right),\left(6{,}1\right),\left(6{,}2\right),\left(6{,}3\right),\left(6{,}4\right),\left(6{,}5\right) \right\}$.
}
\end{ex}
%Câu 6
\begin{ex}
Gieo đồng tiền hai lần. Số phần tử của biến cố để mặt ngửa xuất hiện đúng $1$ lần là:
\choice
{\True $2$}
{$4$}
{$5$}
{$6$}
\loigiai{
Liệt kê ta có: $A=\left\{ NS\cdot SN \right\}$
}
\end{ex}
%Câu 7
\begin{ex}
Gieo ngẫu nhiên $2$ đồng tiền thì không gian mẫu của phép thử có bao nhiêu biến cố:
\choice
{\True $4$}
{$8$}
{$12$}
{$16$}
\loigiai{
Mô tả không gian mẫu ta có: $\Omega =\left\{ SS;SN;NS;NN \right\}$
}
\end{ex}
%Câu 8
\begin{ex}
Cho phép thử có không gian mẫu $\Omega =\left\{ 1{,}2,3{,}4,5{,}6 \right\}$. Các cặp biến cố không đối nhau là:
\choice
{$A=\{1\}$ và $B=\left\{ 2{,}3,4{,}5,6 \right\}$}
{$C\left\{ 1{,}4,5 \right\}$ và $D=\left\{ 2{,}3,6 \right\}$}
{\True $E=\left\{ 1{,}4,6 \right\}$ và $F=\left\{ 2{,}3 \right\}$}
{$\Omega $ và $\varnothing $}
\loigiai{
Cặp biến cố không đối nhau là $E=\left\{ 1{,}4,6 \right\}$ và $F=\left\{ 2{,}3 \right\}$ do $E\cap F=\varnothing $ và $E\cup F\ne \Omega $
}
\end{ex}
%Câu 9
\begin{ex}
Một hộp đựng $10$ thẻ, đánh số từ $1$ đến $10$. Chọn ngẫu nhiên $3$ thẻ. Gọi $A$ là biến cố để tổng số của $3$ thẻ được chọn không vượt quá $8$. Số phần tử của biến cố $A$ là:
\choice
{$2$}
{$3$}
{\True $4$}
{$5$}
\loigiai{
Liệt kê ta có: $A=\left\{ (1;2;3);(1;2;4);(1;2;5);(1;3;4) \right\}$
}
\end{ex}
%Câu 10
\begin{ex}
Gieo một đồng tiền 5 lần. Xác định và tính số phần tử của không gian mẫu
\choice
{$|\Omega|=8$}
{$|\Omega|=16$}
{\True $|\Omega|=32$}
{$|\Omega|=64$}
\loigiai{
Kết quả của 5 lần gieo là dãy $abcde$ với $a,b,c,d,e$ nhận một trong hai giá trị N hoặc S. Do đó số phần tử của không gian mẫu: $|\Omega|=2.2\cdot 2.2\cdot 2=32$
}
\end{ex}
%Câu 11
\begin{ex}
Gieo một đồng tiền 5 lần. Xác định và tính số phần tử của biến cố A: \lq\lq  Lần đầu tiên xuất hiện mặt ngửa\rq\rq 
\choice
{\True $|\Omega_A|=16$}
{$|\Omega_A|=18$}
{$|\Omega_A|=20$}
{$|\Omega_A|=22$}
\loigiai{
Lần đầu tiên xuất hiện mặt sấp nên $a$ chỉ nhận giá trị S; $b,c,d,e$ nhận S hoặc N nên $|\Omega_A|=1.2\cdot 2.2\cdot 2=16$
}
\end{ex}
%Câu 12
\begin{ex}
Gieo một đồng tiền 5 lần. Xác định và tính số phần tử của biến cố B: \lq\lq  Mặt sấp xuất hiện ít nhất một lần\rq\rq 
\choice
{\True $|\Omega_B|=31$}
{$|\Omega_B|=32$}
{$|\Omega_B|=33$}
{$|\Omega_B|=34$}
\loigiai{
Kết quả 5 lần gieo mà không có lần nào xuất hiện mặt sấp là 1\\
Vậy $|\Omega_B|=32-1=31$
}
\end{ex}
%Câu 13
\begin{ex}
Gieo một đồng tiền 5 lần. Xác định và tính số phần tử của biến cố C: \lq\lq  Số lần mặt sấp xuất hiện nhiều hơn mặt ngửa\rq\rq 
\choice
{$|\Omega_C|=19$}
{$|\Omega_C|=18$}
{\True $|\Omega_C|=17$}
{$|\Omega_C|=20$}
\loigiai{
Kết quả của 5 lần gieo mà mặt N xuất hiện đúng một lần: $\mathrm{C}_5^1$\\
Kết quả của 5 lần gieo mà mặt N xuất hiện đúng hai lần: $\mathrm{C}_5^2$\\
Số kết quả của 5 lần gieo mà số lần mặt S xuất hiện nhiều hơn số lần mặt N là: $|\Omega_C|=32-\mathrm{C}_5^2-\mathrm{C}_5^1=17$
}
\end{ex}
%Câu 14
\begin{ex}
Có 100 tấm thẻ được đánh số từ 1 đến 100. Lấy ngẫu nhiên 5 thẻ. Tính số phần tử của không gian mẫu
\choice
{\True $|\Omega|=\mathrm{C}_{100}^5$}
{$|\Omega|=A_{100}^5$}
{$|\Omega|=\mathrm{C}_{100}^1$}
{$|\Omega|=A_{100}^1$}
\loigiai{
Ta có $|\Omega|=\mathrm{C}_{100}^5$
}
\end{ex}
%Câu 15
\begin{ex}
Có 100 tấm thẻ được đánh số từ 1 đến 100. Lấy ngẫu nhiên 5 thẻ. Tính số phần tử của biến cố A: \lq\lq  Số ghi trên các tấm thẻ được chọn là số chẵn\rq\rq 
\choice
{$|\Omega_A|=A_{50}^5$}
{$|\Omega_A|=A_{100}^5$}
{\True $|\Omega_A|=\mathrm{C}_{50}^5$}
{$|\Omega_A|=\mathrm{C}_{100}^5$}
\loigiai{
Trong 100 tấm thẻ có 50 tấm được ghi các số chẵn, do đó $|\Omega_A|=\mathrm{C}_{50}^5$
}
\end{ex}
%Câu 16
\begin{ex}
Có 100 tấm thẻ được đánh số từ 1 đến 100. Lấy ngẫu nhiên 5 thẻ. Tính số phần tử của biến cố B: \lq\lq  Có ít nhất một số ghi trên thẻ được chọn chia hết cho 3\rq\rq .
\choice
{$|\Omega_B|=\mathrm{C}_{100}^5+\mathrm{C}_{67}^5$}
{$|\Omega_B|=\mathrm{C}_{100}^5-\mathrm{C}_{50}^5$}
{$|\Omega_B|=\mathrm{C}_{100}^5+\mathrm{C}_{50}^5$}
{\True $|\Omega_B|=\mathrm{C}_{100}^5-\mathrm{C}_{67}^5$}
\loigiai{
Từ 1 đến 100 có 33 số chia hết cho 3. Do đó, số cách chọn 5 tấm thẻ mà không có tấm thẻ nào ghi số chia hết cho 3 là: $\mathrm{C}_{67}^5$\\
Vậy $|\Omega_B|=\mathrm{C}_{100}^5-\mathrm{C}_{67}^5$
}
\end{ex}
%Câu 17
\begin{ex}
Trong một chiếc hộp đựng 6 viên bi đỏ, 8 viên bi xanh, 10 viên bi trắng. Lấy ngẫu nhiên 4 viên bi. Tính số phần tử của không gian mẫu
\choice
{\True 10626}
{14241}
{14284}
{31311}
\loigiai{
Ta có: $|\Omega|=\mathrm{C}_{24}^4=10626$
}
\end{ex}
%Câu 18
\begin{ex}
Trong một chiếc hộp đựng 6 viên bi đỏ, 8 viên bi xanh, 10 viên bi trắng. Lấy ngẫu nhiên 4 viên bi. Tính số phần tử của biến cố A: \lq\lq  4 viên bi lấy ra có đúng hai viên bi màu trắng\rq\rq 
\choice
{$|\Omega_A|=4245$}
{$|\Omega_A|=4295$}
{\True $|\Omega_A|=4095$}
{$|\Omega_A|=3095$}
\loigiai{
Số cách chọn 4 viên bi có đúng hai viên bị màu trắng là: $\mathrm{C}_{10}^2\cdot \mathrm{C}_{14}^2=4095$\\
Suy ra: $|\Omega_A|=4095$
}
\end{ex}
%Câu 19
\begin{ex}
Trong một chiếc hộp đựng 6 viên bi đỏ, 8 viên bi xanh, 10 viên bi trắng. Lấy ngẫu nhiên 4 viên bi. Tính số phần tử của biến cố B: \lq\lq  4 viên bi lấy ra có ít nhất một viên bi màu đỏ\rq\rq 
\choice
{$|\Omega_B|=7366$}
{$|\Omega_B|=7563$}
{\True $|\Omega_B|=7566$}
{$|\Omega_B|=7568$}
\loigiai{
Số cách lấy 4 viên bi mà không có viên bi màu đỏ được chọn là: $\mathrm{C}_{18}^4$\\
Suy ra : $|\Omega_B|=\mathrm{C}_{24}^4-\mathrm{C}_{18}^4=7566$
}
\end{ex}
%Câu 20
\begin{ex}
Trong một chiếc hộp đựng 6 viên bi đỏ, 8 viên bi xanh, 10 viên bi trắng. Lấy ngẫu nhiên 4 viên bi. Tính số phần tử của biến cố C: \lq\lq  4 viên bi lấy ra có đủ 3 màu\rq\rq 
\choice
{$|\Omega_C|=4859$}
{$|\Omega_C|=58552$}
{\True $|\Omega_C|=5859$}
{$|\Omega_C|=8859$}
\loigiai{
Số cách lấy 4 viên bi chỉ có một màu là: $\mathrm{C}_6^4+\mathrm{C}_8^4+\mathrm{C}_{10}^4$\\
Số cách lấy 4 viên bi có đúng hai màu là:\\
$\mathrm{C}_{14}^4+\mathrm{C}_{18}^4+\mathrm{C}_{14}^4-2(\mathrm{C}_6^4+\mathrm{C}_8^4+\mathrm{C}_{10}^4)$\\
Số cách lấy 4 viên bị có đủ ba màu là:\\
$\mathrm{C}_{24}^4-(\mathrm{C}_{14}^4+\mathrm{C}_{18}^4+\mathrm{C}_{14}^4)+(\mathrm{C}_6^4+\mathrm{C}_8^4+\mathrm{C}_{10}^4)=5859$\\
Suy ra $|\Omega_C|=5859$
}
\end{ex}
%Câu 21
\begin{ex}
Một xạ thủ bắn liên tục 4 phát đạn vào bia. Gọi $A_k$ là các biến cố \lq\lq  xạ thủ bắn trúng lần thứ $k$\rq\rq  với $k=1{,}2,3{,}4$. Hãy biểu diễn các biến cố sau qua các biến cố $A_1,A_2,A_3,A_4$\\
$A$: \lq\lq Lần thứ tư mới bắn trúng bia’’; $B$: \lq\lq Bắn trúng bia ít nhất một lần’’; $C$: \lq\lq  Chỉ bắn trúng bia hai lần’’
\choice
{$A=\overline{A_1}\cap \overline{A_2}\cap A_3\cap A_4$, $B=A_1\cup A_2\cup A_3\cap A_4$, $C=A_i\cup A_j\cap \overline{A_k}\cap \overline{A_m}$,$i,j,k,m\in \left\{ 1{,}2,3{,}4 \right\}$ và đôi một khác nhau}
{$A=A_1\cap \overline{A_2}\cap \overline{A_3}\cap A_4$, $B=A_1\cap A_2\cup A_3\cup A_4$, $C=A_i\cup A_j\cup \overline{A_k}\cup \overline{A_m}$,$i,j,k,m\in \left\{ 1{,}2,3{,}4 \right\}$ và đôi một khác nhau}
{$A=\overline{A_1}\cap A_2\cap \overline{A_3}\cap A_4$, $B=A_1\cup A_2\cap A_3\cup A_4$, $C=A_i\cap A_j\cup \overline{A_k}\cup \overline{A_m}$,$i,j,k,m\in \left\{ 1{,}2,3{,}4 \right\}$ và đôi một khác nhau}
{\True $A=\overline{A_1}\cap \overline{A_2}\cap \overline{A_3}\cap A_4$, $B=A_1\cup A_2\cup A_3\cup A_4$, $C=A_i\cap A_j\cap \overline{A_k}\cap \overline{A_m}$,$i,j,k,m\in \left\{ 1{,}2,3{,}4 \right\}$ và đôi một khác nhau}
\loigiai{
Ta có: $\overline{A_k}$ là biến cố lần thứ $k$ ($k=1{,}2,3{,}4$) bắn không trúng bia.\\
Do đó:\\
$A=\overline{A_1}\cap \overline{A_2}\cap \overline{A_3}\cap A_4$\\
$B=A_1\cup A_2\cup A_3\cup A_4$\\
$C=A_i\cap A_j\cap \overline{A_k}\cap \overline{A_m}$ với $i,j,k,m\in \left\{ 1{,}2,3{,}4 \right\}$ và đôi một khác nhau}
\end{ex}
\begin{dang}{TÌM XÁC SUẤT CỦA BIẾN CỐ}
Phương pháp: Tính xác suất của biến cố theo định nghĩa cổ điển ta sử dụng công thức :$P(A)=\dfrac{|\Omega_A|}{|\Omega|}$
\end{dang}
%Câu 22
\begin{ex}
Cho A là một biến cố liên quan phép thử T. Mệnh đề nào sau đây là mệnh đề đúng ?
\choice
{$P(A)$ là số lớn hơn 0}
{\True $P(A)=1-P\left(\overline{A}\right)$}
{$P(A)=0\Leftrightarrow A=\Omega $}
{$P(A)$ là số nhỏ hơn 1}
\loigiai{
 Loại trừ :A ;B ;C đều sai
}
\end{ex}
%Câu 23
\begin{ex}
Gieo đồng tiền hai lần. Xác suất để sau hai lần gieo thì mặt sấp xuất hiện ít nhất một lần
\choice
{$\dfrac{1}{4}$}
{$\dfrac{1}{2}$}
{\True $\dfrac{3}{4}$}
{$\dfrac{1}{3}$}
\loigiai{
Số phần tử không gian mẫu:$n\left(\Omega\right)=2.2=4$\\
Biến cố xuất hiện mặt sấp ít nhất một lần: $A=\left\{ SN;NS;SS \right\}$\\
Suy ra $P(A)=\dfrac{|\Omega_A|}{n\left(\Omega\right)}=\dfrac{3}{4}$
}
\end{ex}
%Câu 24
\begin{ex}
Gieo đồng tiền $5$ lần cân đối và đồng chất. Xác suất để được ít nhất một lần xuất hiện mặt sấp là:
\choice
{\True $\dfrac{31}{32}$}
{$\dfrac{21}{32}$}
{$\dfrac{11}{32}$}
{$\dfrac{1}{32}$}
\loigiai{
Phép thử : Gieo đồng tiền $5$ lần cân đối và đồng chất\\
Ta có $n\left(\Omega\right)=2^5=32$\\
Biến cố $A$ : Được ít nhất một lần xuất hiện mặt sấp\\
$\overline{A}$ : Tất cả đều là mặt ngửa\\
$n\left({\bar{A}}\right)=1$\\
$\Rightarrow |\Omega_A|=n\left(\Omega\right)-n\left({\bar{A}}\right)=31$\\
$\Rightarrow p(A)=\dfrac{|\Omega_A|}{n\left(\Omega\right)}=\dfrac{31}{32}$
}
\end{ex}
%Câu 25
\begin{ex}
Gieo đồng tiền $5$ lần cân đối và đồng chất. Xác suất để được ít nhất một đồng tiền xuất hiện mặt sấp là
\choice
{\True $\dfrac{31}{32}$}
{$\dfrac{21}{32}$}
{$\dfrac{11}{32}$}
{$\dfrac{1}{32}$}
\loigiai{
$n\left(\Omega\right)=2^5=32$.\\
$\text{A}$: \lq\lq được ít nhất một đồng tiền xuất hiện mặt sấp\rq\rq .\\
Xét biến cố đối $\bar{A}$: \lq\lq không có đồng tiền nào xuất hiện mặt sấp\rq\rq .\\
$\bar{A}=\left\{ \left(N,N,N,N,N\right) \right\}$, có $n\left({\bar{A}}\right)=1$.\\
Suy ra $|\Omega_A|=32-1=31$.\\
KL: $P(A)=\dfrac{|\Omega_A|}{n\left(\Omega\right)}=\dfrac{31}{32}$
}
\end{ex}
%Câu 26
\begin{ex}
Gieo ngẫu nhiên một đồng tiền cân đối và đồng chất bốn lần. Xác suất để cả bốn lần gieo đều xuất hiện mặt sấp là:
\choice
{$\dfrac{4}{16}$}
{$\dfrac{2}{16}$}
{\True $\dfrac{1}{16}$}
{$\dfrac{6}{16}$}
\loigiai{
Gọi A là biến cố: \lq\lq cả bốn lần gieo đều xuất hiện mặt sấp.\rq\rq \\
-Không gian mẫu: $2^4=16$.\\
-$|\Omega_A|=1.1\cdot 1.1=1$.\\
=>$P(A)=\dfrac{|\Omega_A|}{\left| \Omega \right|}=\dfrac{1}{16}$.
}
\end{ex}
%Câu 27
\begin{ex}
Gieo một đồng tiền liên tiếp $2$ lần. Số phần tử của không gian mẫu $|\Omega|$ là?
\choice
{$1$}
{$2$}
{\True $4$}
{$8$}
\loigiai{
$|\Omega|=2.2=4$.\\
(lần 1 có 2 khả năng xảy ra- lần 2 có 2 khả năng xảy ra)
}
\end{ex}
%Câu 28
\begin{ex}
Gieo một đồng tiền liên tiếp 3 lần. Tính xác suất của biến cố $A$:\rq\rq lần đầu tiên xuất hiện mặt sấp\rq\rq 
\choice
{\True $P(A)=\dfrac{1}{2}$}
{$P(A)=\dfrac{3}{8}$}
{$P(A)=\dfrac{7}{8}$}
{$P(A)=\dfrac{1}{4}$}
\loigiai{
Xác suất để lần đầu xuất hiện mặt sấp là $\dfrac{1}{2}$.Lần 2 và 3 thì tùy ý nên xác suất là 1.\\
Theo quy tắc nhân xác suất: $P(A)=\dfrac{1}{2}\cdot 1.1=\dfrac{1}{2}$
}
\end{ex}
%Câu 29
\begin{ex}
Gieo một đồng tiền liên tiếp 3 lần. Tính xác suất của biến cố $A$:\rq\rq kết quả của 3 lần gieo là như nhau\rq\rq 
\choice
{$P(A)=\dfrac{1}{2}$}
{$P(A)=\dfrac{3}{8}$}
{$P(A)=\dfrac{7}{8}$}
{\True $P(A)=\dfrac{1}{4}$}
\loigiai{
Lần đầu có thể ra tùy ý nên xác suất là 1.Lần 2 và 3 phải giống lần 1 xác suất là $\dfrac{1}{2}$.\\
Theo quy tắc nhân xác suất: $P(A)=1\cdot \dfrac{1}{2}\cdot \dfrac{1}{2}=\dfrac{1}{4}$
}
\end{ex}
%Câu 30
\begin{ex}
Gieo một đồng tiền liên tiếp 3 lần. Tính xác suất của biến cố $A$:\rq\rq có đúng 2 lần xuất hiện mặt sấp\rq\rq 
\choice
{$P(A)=\dfrac{1}{2}$}
{\True $P(A)=\dfrac{3}{8}$}
{$P(A)=\dfrac{7}{8}$}
{$P(A)=\dfrac{1}{4}$}
\loigiai{
Chọn 2 trong 3 lần để xuất hiện mặt sấp có $\mathrm{C}_3^2=3$ cách.\\
2 lần xuất hiện mặt sấp có xác suất mỗi lần là $\dfrac{1}{2}$. Lần xuất hiện mặt ngửa có xác suất là $\dfrac{1}{2}$.\\
Vậy: $P(A)=3\cdot \dfrac{1}{2}\cdot \dfrac{1}{2}\cdot \dfrac{1}{2}=\dfrac{3}{8}$
}
\end{ex}
%Câu 31
\begin{ex}
Gieo một đồng tiền liên tiếp 3 lần. Tính xác suất của biến cố $A$:\rq\rq ít nhất một lần xuất hiện mặt sấp\rq\rq 
\choice
{$P(A)=\dfrac{1}{2}$}
{$P(A)=\dfrac{3}{8}$}
{\True $P(A)=\dfrac{7}{8}$}
{$P(A)=\dfrac{1}{4}$}
\loigiai{
Ta có: $\overline{A}$:\rq\rq không có lần nào xuất hiện mặt sấp\rq\rq  hay cả 3 lần đều mặt ngửa.\\
Theo quy tắc nhân xác suất: $P(\overline{A})=\dfrac{1}{2}\cdot \dfrac{1}{2}\cdot \dfrac{1}{2}=\dfrac{1}{8}$. Vậy: $P(A)=1-P(\overline{A})=1-\dfrac{1}{8}=\dfrac{7}{8}$
}
\end{ex}
%Câu 32
\begin{ex}
Gieo một đồng tiền cân đối và đồng chất bốn lần. Xác suất để cả bốn lần xuất hiện mặt sấp là:
\choice
{$\dfrac{4}{16}$}
{$\dfrac{2}{16}$}
{\True $\dfrac{1}{16}$}
{$\dfrac{6}{16}$}
\loigiai{
Mỗi lần suất hiện mặt sấp có xác suất là $\dfrac{1}{2}$.\\
Theo quy tắc nhân xác suất: $P(A)=\dfrac{1}{2}\cdot \dfrac{1}{2}\cdot \dfrac{1}{2}\cdot \dfrac{1}{2}=\dfrac{1}{16}$
}
\end{ex}
%Câu 33
\begin{ex}
Gieo ngẫu nhiên đồng thời bốn đồng xu. Tính xác xuất để ít nhất hai đồng xu lật ngửa, ta có kết quả
\choice
{$\dfrac{10}{9}$}
{$\dfrac{11}{12}$}
{\True $\dfrac{11}{16}$}
{$\dfrac{11}{15}$}
\loigiai{
Do mỗi đồng xu có một mặt sấp và một mặt ngửa nên $n\left(\Omega\right)=2.2\cdot 2.2=16$.\\
Gọi $A$ là biến cố: \lq\lq Có nhiều nhất một đồng xu lật ngửa\rq\rq . Khi đó, ta có hai trường hợp\\
Trường hợp 1. Không có đồng xu nào lật ngửa $\Rightarrow $ có một kết quả.\\
Trường hợp 2. Có một đồng xu lật ngửa $\Rightarrow $ có bốn kết quả.\\
Vậy xác suất để ít nhất hai đồng xu lật ngửa là\\
$P=1-P(A)=1-\dfrac{1+4}{16}=\dfrac{11}{16}$.
}
\end{ex}
%Câu 34
\begin{ex}
Gieo một con súc sắc. Xác suất để mặt chấm chẵn xuất hiện là:
\choice
{$0{,}2$}
{$0{,}3$}
{$0{,}4$}
{\True $0{,}5$}
\loigiai{
Không gian mẫu:$\Omega =\left\{ 1;2;3;4;5;6 \right\}$\\
Biến cố xuất hiện mặt chẵn: $A=\left\{ 2;4;6 \right\}$\\
Suy ra $P(A)=\dfrac{|\Omega_A|}{n\left(\Omega\right)}=\dfrac{1}{2}$
}
\end{ex}
%Câu 35
\begin{ex}
Gieo ngẫu nhiên một con súc sắc. Xác suất để mặt $6$ chấm xuất hiện:
\choice
{\True $\dfrac{1}{6}$}
{$\dfrac{5}{6}$}
{$\dfrac{1}{2}$}
{$\dfrac{1}{3}$}
\loigiai{
Không gian mẫu:$\Omega =\left\{ 1;2;3;4;5;6 \right\}$\\
Biến cố xuất hiện: $A=\{6\}$\\
Suy ra $P(A)=\dfrac{|\Omega_A|}{n\left(\Omega\right)}=\dfrac{1}{6}$
}
\end{ex}
%Câu 36
\begin{ex}
Gieo ngẫu nhiên hai con súc sắc cân đối và đồng chất. Xác suất để sau hai lần gieo kết quả như nhau là:
\choice
{$\dfrac{5}{36}$}
{\True $\dfrac{1}{6}$}
{$\dfrac{1}{2}$}
{1}
\loigiai{
Số phần tử của không gian mẫu:$n\left(\Omega\right)=6.6=36$\\
Biến cố xuất hiện hai lần như nhau: $A=\left\{ (1;1);(2;2);(3;3);(4;4);(5;5);(6;6) \right\}$\\
Suy ra $P(A)=\dfrac{|\Omega_A|}{n\left(\Omega\right)}=\dfrac{6}{36}=\dfrac{1}{6}$
}
\end{ex}
%Câu 37
\begin{ex}
Một con súc sắc cân đối đồng chất được gieo $5$ lần. Xác suất để tổng số chấm ở hai lần gieo đầu bằng số chấm ở lần gieo thứ ba:
\choice
{$\dfrac{10}{216}$}
{\True $\dfrac{15}{216}$}
{$\dfrac{16}{216}$}
{$\dfrac{12}{216}$}
\loigiai{
Số phần tử không gian mẫu:$n\left(\Omega\right)=6.6\cdot 6.6\cdot 6=6^5$\\
Bộ kết quả của $3$ lần gieo thỏa yêu cầu là:
\begin{itemize}    
\item (1;1;2), (1;2;3), (2;1;3);
\item (1;3;4), (3;1;4), (2;2;4);
\item (1;4;5), (4;1;5), (2;3;5);
\item (3;2;5), (1;5;6), (5;1;6);
\item (2;4;6), (4;2;6), (3;3;6).
\end{itemize}
Nên $|\Omega_A|=15.6\cdot 6$.\\
Suy ra $P(A)=\dfrac{|\Omega_A|}{n\left(\Omega\right)}=\dfrac{15.6\cdot 6}{6^5}=\dfrac{15}{216}$.
}
\end{ex}
%Câu 38
\begin{ex}
Gieo $3$ con súc sắc cân đối và đồng chất. Xác suất để số chấm xuất hiện trên $3$ con súc sắc đó bằng nhau:
\choice
{$\dfrac{5}{36}$}
{$\dfrac{1}{9}$}
{$\dfrac{1}{18}$}
{\True $\dfrac{1}{36}$}
\loigiai{
Phép thử : Gieo ba con súc sắc cân đối và đồng chất\\
Ta có $n\left(\Omega\right)=6^3=216$\\
Biến cố $A$ : Số chấm trên ba súc sắc bằng nhau\\
$|\Omega_A|=6$\\
$\Rightarrow p(A)=\dfrac{|\Omega_A|}{n\left(\Omega\right)}=\dfrac{1}{36}$
}
\end{ex}
%Câu 39
\begin{ex}
Gieo $2$ con súc sắc cân đối và đồng chất. Xác suất để tổng số chấm xuất hiện trên hai mặt của $2$ con súc sắc đó không vượt quá $5$ là:
\choice
{$\dfrac{2}{3}$}
{$\dfrac{7}{18}$}
{$\dfrac{8}{9}$}
{\True $\dfrac{5}{18}$}
\loigiai{
Phép thử : Gieo hai con súc sắc đồng chất\\
Ta có $n\left(\Omega\right)=6^2=36$\\
Biến cố $A$ : Được tổng số chấm của hai súc sắc không quá $5$. Khi đó ta được các trường hợp là $(1;1),(1;2),(1;3),(1;4),(2;1),(2;2),(2;3),(3;1),(3;2);(4;1)$\\
$\Rightarrow |\Omega_A|=10$\\
$\Rightarrow p(A)=\dfrac{|\Omega_A|}{n\left(\Omega\right)}=\dfrac{5}{18}$
}
\end{ex}
%Câu 40
\begin{ex}
Gieo hai con súc sắc. Xác suất để tổng số chấm trên hai mặt chia hết cho $3$ là
\choice
{$\dfrac{13}{36}$}
{$\dfrac{11}{36}$}
{$\dfrac{1}{6}$}
{\True $\dfrac{1}{3}$}
\loigiai{
Số phần tử của không gian mẫu $n\left(\Omega\right)=6^2=36$.\\
Biến cố $\text{A}$: \lq\lq tổng số chấm trên hai mặt chia hết cho $3$\rq\rq .\\
$A=\left\{ \left(1{,}2\right);\left(1{,}5\right);\left(2{,}1\right);\left(2{,}4\right);\left(3{,}3\right);\left(3{,}6\right);\left(4{,}2\right);\left(4{,}5\right);\left(5{,}1\right);\left(5{,}4\right);\left(6{,}3\right);\left(6{,}6\right) \right\}$.\\
$|\Omega_A|=12$. KL: $P(A)=\dfrac{|\Omega_A|}{n\left(\Omega\right)}=\dfrac{12}{23}=\dfrac{1}{3}$
}
\end{ex}
%Câu 41
\begin{ex}
Gieo $3$ con súc sắc cân đối và đồng chất. Xác suất để số chấm xuất hiện trên $3$ con súc sắc đó bằng nhau:
\choice
{$\dfrac{5}{36}$}
{$\dfrac{1}{9}$}
{$\dfrac{1}{18}$}
{\True $\dfrac{1}{36}$}
\loigiai{
$n\left(\Omega\right)=6^3=216$.\\
$\text{A}$: \lq\lq số chấm xuất hiện trên $3$ con súc sắc đó bằng nhau\rq\rq .\\
$A=\left\{ \left(1{,}1,1\right);\left(2{,}2,2\right);\left(3{,}3,3\right);\left(4{,}4,4\right);\left(5{,}5,5\right);\left(6{,}6,6\right) \right\}$.\\
$|\Omega_A|=6$.\\
KL: $P(A)=\dfrac{|\Omega_A|}{n\left(\Omega\right)}=\dfrac{6}{216}=\dfrac{1}{36}$
}
\end{ex}
%Câu 42
\begin{ex}
Một con xúc sắc cân đối và đồng chất được gieo ba lần. Gọi $P$ là xác suất để tổng số chấm xuất hiện ở hai lần gieo đầu bằng số chấm xuất hiện ở lần gieo thứ ba. Khi đó $P$ bằng:
\choice
{$\dfrac{10}{216}$}
{\True $\dfrac{15}{216}$}
{$\dfrac{16}{216}$}
{$\dfrac{12}{216}$}
\loigiai{
$|\Omega|=6.6\cdot 6=216$. Gọi $A$:\rq\rq tổng số chấm xuất hiện ở hai lần gieo đầu bằng số chấm xuất hiện ở lần gieo thứ ba\rq\rq .\\
Ta chỉ cần chọn 1 bộ 2 số chấm ứng với hai lần gieo đầu sao cho tổng của chúng thuộc tập $\{1;2;3;4;5;6\}$ và số chấm lần gieo thứ ba sẽ là tổng hai lần gieo đầu.\\
Liệt kê ra ta có:\\
$ \!\!\{\!\!\text{ (1;1);(1;2);(1;3);(1;4);(1;5);(2;1);(2;2);(2;3);(2;4);(3;1);(3;2);(3;3);(4;1);(4;2);(5;1) }\!\!\}\!\! $\\
Do đó $|\Omega_A|=15$. Vậy $P(A)=\dfrac{15}{216}$
}
\end{ex}
%Câu 43
\begin{ex}
Gieo hai con súc xắc cân đối và đồng chất. Xác suất để hiệu số chấm trên mặt xuất hiện của hai con súc xắc bằng 2 là:
\choice
{$\dfrac{1}{12}$}
{\True $\dfrac{1}{9}$}
{$\dfrac{2}{9}$}
{$\dfrac{5}{36}$}
\loigiai{
$|\Omega|=6.6=36$. Gọi $A$:\rq\rq hiệu số chấm trên mặt xuất hiện của hai con súc xắc bằng 2\rq\rq .\\
Các hiệu có thể bằng 2 là:\\
$3-1=2,4-2=2,5-3=2,6-4=2$.\\
Do đó $|\Omega_A|=4$. Vậy $P(A)=\dfrac{4}{36}=\dfrac{1}{9}$
}
\end{ex}
%Câu 44
\begin{ex}
Gieo hai con súc xắc cân đối và đồng chất. Xác suất để tổng số chấm trên mặt xuất hiện của hai con súc xắc bằng 7 là:
\choice
{$\dfrac{2}{9}$}
{\True $\dfrac{1}{6}$}
{$\dfrac{7}{36}$}
{$\dfrac{5}{36}$}
\loigiai{
$|\Omega|=6.6=36$. Gọi $A$:\rq\rq tổng số chấm trên mặt xuất hiện của hai con súc xắc bằng 7\rq\rq .\\
$A= \!\!\{\!\!\text{ (1;6);(2;5);(3;4);(4;3);(5;2);(6;1) }\!\!\}\!\! $.\\
Do đó $|\Omega_A|=6$. Vậy $P(A)=\dfrac{6}{36}=\dfrac{1}{6}$
}
\end{ex}
%Câu 45
\begin{ex}
Gieo một con súc xắc cân đối và đồng chất hai lần. Xác suất để ít nhất một lần xuất hiện mặt sáu chấm là:
\choice
{$\dfrac{12}{36}$}
{\True $\dfrac{11}{36}$}
{$\dfrac{6}{36}$}
{$\dfrac{8}{36}$}
\loigiai{
$|\Omega|=6.6=36$. Gọi $A$:\rq\rq ít nhất một lần xuất hiện mặt sáu chấm\rq\rq .\\
Khi đó $\overline{A}$:\rq\rq không có lần nào xuất hiện mặt sáu chấm\rq\rq .\\
Ta có$n(\overline{A})=5.5=25$. Vậy $P(A)=1-P(\overline{A})=1-\dfrac{25}{36}=\dfrac{11}{36}$
}
\end{ex}
%Câu 46
\begin{ex}
Gieo ba con súc xắc cân đối và đồng chất. Xác suất để số chấm xuất hiện trên ba con như nhau là:
\choice
{$\dfrac{12}{216}$}
{$\dfrac{1}{216}$}
{\True $\dfrac{6}{216}$}
{$\dfrac{3}{216}$}
\loigiai{
Lần đầu có thể ra tùy ý nên xác suất là 1. Lần 2 và 3 phải giống lần 1 xác suất là $\dfrac{1}{6}$.\\
Theo quy tắc nhân xác suất: $P(A)=1\cdot \dfrac{1}{6}\cdot \dfrac{1}{6}=\dfrac{1}{36}=\dfrac{6}{216}$
}
\end{ex}
%Câu 47
\begin{ex}
Một con súc sắc đồng chất được đổ $6$ lần. Xác suất để được một số lớn hơn hay bằng $5$ xuất hiện ít nhất $5$ lần là
\choice
{$\dfrac{31}{23328}$}
{\True $\dfrac{41}{23328}$}
{$\dfrac{51}{23328}$}
{$\dfrac{21}{23328}$}
\loigiai{
Ta có $n\left(\Omega\right)=6.6\cdot 6.6\cdot 6.6=6^6$.\\
Có các trường hợp sau:\\
1. Số bằng $5$ xuất hiện đúng $5$ lần $\Rightarrow $ có $30$ kết quả thuận lợi.\\
2. Số bằng $5$ xuất hiện đúng $6$ lần $\Rightarrow $ có $1$ kết quả thuận lợi.\\
3. Số bằng $6$ xuất hiện đúng $5$ lần $\Rightarrow $ có $30$ kết quả thuận lợi.\\
4. Số bằng $6$ xuất hiện đúng $6$ lần $\Rightarrow $ có $1$ kết quả thuận lợi.\\
Vậy xác suất để được một số lớn hơn hay bằng $5$ xuất hiện ít nhất $5$ lần là\\
$P=\dfrac{30+1+30+1}{6^6}=\dfrac{31}{23328}$.
}
\end{ex}
%Câu 48
\begin{ex}
Gieo ngẫu nhiên hai con súc sắc cân đối, đồng chất. Xác suất của biến cố \lq\lq Tổng số chấm của hai con súc sắc bằng 6\rq\rq  là
\choice
{$\dfrac{5}{6}$}
{$\dfrac{7}{36}$}
{$\dfrac{11}{36}$}
{\True $\dfrac{5}{36}$}
\loigiai{
Gọi A là biến cố: \lq\lq Tổng số chấm của hai con súc sắc bằng 6.\rq\rq \\
-Không gian mẫu: $6^2=36$.\\
-Ta có $1+5=6{,}2+4=6{,}3+3=6{,}4+2=6{,}5+1=6$.\\
=>$|\Omega_A|=5$.\\
=>$P(A)=\dfrac{|\Omega_A|}{\left| \Omega \right|}=\dfrac{5}{36}$.
}
\end{ex}
%Câu 49
\begin{ex}
Gieo một con súc sắc cân đối và đồng chất $6$ lần độc lập. Tính xác xuất để không lần nào xuất hiện mặt có số chấm là một số chẵn ?
\choice
{$\dfrac{1}{36}$}
{\True $\dfrac{1}{64}$}
{$\dfrac{1}{32}$}
{$\dfrac{1}{72}$}
\loigiai{
Số phần tử của không gian mẫu là: $\left| \Omega \right|=6^6$.\\
Số phần tử của không gian thuận lợi là: $\left| \Omega _A \right|=3^6$\\
Xác suất biến cố $A$ là : $P(A)=\dfrac{1}{64}$
}
\end{ex}
%Câu 50
\begin{ex}
Gieo một con súc sắc cân đối và đồng chất hai lần. Xác suất để tổng số chấm xuất hiện là một số chia hết cho $5$ là:
\choice
{$\dfrac{6}{36}$}
{$\dfrac{4}{36}$}
{$\dfrac{8}{36}$}
{\True $\dfrac{7}{36}$}
\loigiai{
Số phần tử của không gian mẫu là: $\left| \Omega \right|=6^2$.\\
Số phần tử của không gian thuận lợi là: $\left| \Omega _A \right|=7$\\
Xác suất biến cố $A$ là : $P(A)=\dfrac{7}{36}$
}
\end{ex}
%Câu 51
\begin{ex}
Gieo hai con súc sắc. Xác suất để tổng hai mặt bằng $11$ là.
\choice
{\True $\dfrac{1}{18}$}
{$\dfrac{1}{6}$}
{$\dfrac{1}{8}$}
{$\dfrac{2}{15}$}
\loigiai{
Số phần tử của không gian mẫu là: $\left| \Omega \right|=6^2=36$.\\
Gọi A là biến cố để tổng hai mặt là $11$, các trường hợp có thể xảy ra của A là $A=\left\{ (5;6);(6;5) \right\}$.\\
Số phần tử của không gian thuận lợi là: $\left| \Omega _A \right|=2$.\\
Xác suất biến cố $A$ là : $P(A)=\dfrac{1}{18}$
}
\end{ex}
%Câu 52
\begin{ex}
Gieo hai con súc sắc. Xác suất để tổng hai mặt bằng $7$ là.
\choice
{$\dfrac{1}{2}$}
{$\dfrac{7}{12}$}
{\True $\dfrac{1}{6}$}
{$\dfrac{1}{3}$}
\loigiai{
Số phần tử của không gian mẫu là: $\left| \Omega \right|=6^2=36$.\\
Gọi A là biến cố để tổng hai mặt là $7$, các trường hợp có thể xảy ra của A là $A=\left\{ (1;6);(6;1);(2;5);(5;2);(3;4);(4;3) \right\}$.\\
Số phần tử của không gian thuận lợi là: $\left| \Omega _A \right|=6$.\\
Xác suất biến cố $A$ là : $P(A)=\dfrac{1}{6}$
}
\end{ex}
%Câu 53
\begin{ex}
Gieo hai con súc sắc. Xác suất để tổng hai mặt chia hết cho $3$ là.
\choice
{$\dfrac{13}{36}$}
{$\dfrac{11}{36}$}
{\True $\dfrac{1}{3}$}
{$\dfrac{2}{3}$}
\loigiai{
Số phần tử của không gian mẫu là: $\left| \Omega \right|=6^2=36$.\\
Gọi A là biến cố để tổng hai mặt chia hết cho $3$, các trường hợp có thể xảy ra của A là $A=\left\{ (1;5);(5;1);(1;2);(2;1);(2;4);(4;2);(3;6);(6;3);(3;3);(6;6);(4;5);(5;4) \right\}$.\\
Số phần tử của không gian thuận lợi là: $\left| \Omega _A \right|=12$.\\
Xác suất biến cố $A$ là : $P(A)=\dfrac{1}{3}$
}
\end{ex}
%Câu 54
\begin{ex}
Gieo ba con súc sắc. Xác suất để được nhiều nhất hai mặt 5 là.
\choice
{$\dfrac{5}{72}$}
{$\dfrac{1}{216}$}
{$\dfrac{1}{72}$}
{\True $\dfrac{215}{216}$}
\loigiai{
Số phần tử của không gian mẫu là: $\left| \Omega \right|=6^3$.\\
Số phần tử của không gian thuận lợi là: $\left| \Omega _A \right|=6^3-1$\\
Xác suất biến cố $A$ là : $P(A)=1-P(B)=1-\dfrac{1}{216}=\dfrac{215}{216}$
}
\end{ex}
%Câu 55
\begin{ex}
Gieo một con súc sắc có sáu mặt các mặt $1{,}2,3{,}4$ được sơn đỏ, mặt $5{,}6$ sơn xanh. Gọi $A$ là biến cố được số lẻ, $B$ là biến cố được nút đỏ (mặt sơn màu đỏ). Xác suất của $A \cup B$  là:
\choice
{$\dfrac{1}{4}$}
{\True $\dfrac{1}{3}$}
{$\dfrac{3}{4}$}
{$\dfrac{2}{3}$}
\loigiai{
Số phần tử của không gian mẫu là: $\left| \Omega \right|=6$.\\
Số phần tử của không gian thuận lợi là: $\left| \Omega_{A\cap B} \right|=2$\\
Xác suất biến cố $P\left(A\cap B\right)=\dfrac{1}{3}$
}
\end{ex}
%Câu 56
\begin{ex}
Gieo hai con súc sắc. Xác suất để tổng số chấm trên hai mặt chia hết cho $3$ là:
\choice
{$\dfrac{13}{36}$}
{$\dfrac{11}{36}$}
{\True $\dfrac{1}{3}$}
{$\dfrac{1}{6}$}
\loigiai{
Số phần tử không gian mẫu:$n\left(\Omega\right)=6.6=36$\\
Biến cố tổng hai mặt chia hết cho $3$ là:\\
$A=\left\{ (1;2);(1;5);(2;1);(2;4);(3;3);(3;6);(4;2);(4;5);(5;1);(5;4);(6;3);(6;6) \right\}$\\
nên $|\Omega_A|=12$.\\
Suy ra $P(A)=\dfrac{|\Omega_A|}{n\left(\Omega\right)}=\dfrac{12}{36}=\dfrac{1}{3}$
}
\end{ex}
%Câu 57
\begin{ex}
Gieo ba con súc sắc. Xác suất để nhiều nhất hai mặt $5$ là:
\choice
{$\dfrac{5}{72}$}
{$\dfrac{1}{216}$}
{$\dfrac{1}{72}$}
{\True $\dfrac{215}{216}$}
\loigiai{
Số phần tử không gian mẫu:$n\left(\Omega\right)=6.6\cdot 6=216$\\
Biến cố có ba mặt $5$ là: $\overline{A}=\left\{ (5;5;5) \right\}$ nên $n\left(\overline{A}\right)=1$.\\
Suy ra $P(A)=1-P\left(\overline{A}\right)=1-\dfrac{n\left(\overline{A}\right)}{n\left(\Omega\right)}=\dfrac{215}{216}$
}
\end{ex}
%Câu 58
\begin{ex}
Gieo một con súc sắc $3$ lần. Xác suất để được mặt số hai xuất hiện cả $3$ lần là:
\choice
{$\dfrac{1}{172}$}
{$\dfrac{1}{18}$}
{$\dfrac{1}{20}$}
{\True $\dfrac{1}{216}$}
\loigiai{
Số phần tử không gian mẫu:$n\left(\Omega\right)=6.6\cdot 6=216$\\
Số phần tử của biến cố xuất hiện mặt số hai ba lần: $|\Omega_A|=1$\\
Suy ra $P(A)=\dfrac{|\Omega_A|}{n\left(\Omega\right)}=\dfrac{1}{216}$
}
\end{ex}
%Câu 59
\begin{ex}
Rút ra một lá bài từ bộ bài $52$ lá. Xác suất để được lá bích là:
\choice
{$\dfrac{1}{13}$}
{\True $\dfrac{1}{4}$}
{$\dfrac{12}{13}$}
{$\dfrac{3}{4}$}
\loigiai{
Số phần tử không gian mẫu:$n\left(\Omega\right)=52$\\
Số phần tử của biến cố xuất hiện lá bích: $|\Omega_A|=13$\\
Suy ra $P(A)=\dfrac{|\Omega_A|}{n\left(\Omega\right)}=\dfrac{13}{52}=\dfrac{1}{4}$
}
\end{ex}
%Câu 60
\begin{ex}
Rút ra một lá bài từ bộ bài $52$ lá. Xác suất để được lá át (A) là:
\choice
{$\dfrac{12}{13}$}
{$\dfrac{1}{169}$}
{\True $\dfrac{1}{13}$}
{$\dfrac{51}{169}$}
\loigiai{
Số phần tử không gian mẫu:$n\left(\Omega\right)=52$\\
Số phần tử của biến cố xuất hiện lá ách: $|\Omega_A|=4$\\
Suy ra $P(A)=\dfrac{|\Omega_A|}{n\left(\Omega\right)}=\dfrac{4}{52}=\dfrac{1}{13}$
}
\end{ex}
%Câu 61
\begin{ex}
Rút ra một lá bài từ bộ bài $52$ lá. Xác suất để được lá ách (A) hay lá rô là:
\choice
{$\dfrac{1}{52}$}
{$\dfrac{34}{52}$}
{\True $\dfrac{4}{13}$}
{$\dfrac{17}{52}$}
\loigiai{
Số phần tử không gian mẫu:$n\left(\Omega\right)=52$\\
Số phần tử của biến cố xuất hiện lá ách hay lá rô: $|\Omega_A|=4+12=16$\\
Suy ra $P(A)=\dfrac{|\Omega_A|}{n\left(\Omega\right)}=\dfrac{16}{52}=\dfrac{4}{13}$
}
\end{ex}
%Câu 62
\begin{ex}
Rút ra một lá bài từ bộ bài $52$ lá. Xác suất để được lá bồi (J) màu đỏ hay lá $5$ là:
\choice
{$\dfrac{1}{13}$}
{\True $\dfrac{3}{26}$}
{$\dfrac{3}{13}$}
{$\dfrac{1}{238}$}
\loigiai{
Số phần tử không gian mẫu:$n\left(\Omega\right)=52$\\
Số phần tử của biến cố xuất hiện lá bồi đỏ hay lá 5: $|\Omega_A|=2+4=6$\\
Suy ra $P(A)=\dfrac{|\Omega_A|}{n\left(\Omega\right)}=\dfrac{6}{52}=\dfrac{3}{26}$
}
\end{ex}
%Câu 63
\begin{ex}
Rút ra một lá bài từ bộ bài $52$ lá. Xác suất để được một lá rô hay một lá hình người (lá bồi, đầm, già) là:
\choice
{$\dfrac{17}{52}$}
{\True $\dfrac{11}{26}$}
{$\dfrac{3}{13}$}
{$\dfrac{3}{13}$}
\loigiai{
Số phần tử không gian mẫu:$n\left(\Omega\right)=52$\\
Số phần tử của biến cố xuất hiện lá hình người hay lá rô: $|\Omega_A|=4+4+4+(13-3)=22$\\
Suy ra $P(A)=\dfrac{|\Omega_A|}{n\left(\Omega\right)}=\dfrac{22}{52}=\dfrac{11}{26}$
}
\end{ex}
%Câu 64
\begin{ex}
Rút một lá bài từ bộ bài gồm $52$ lá. Xác suất để được lá bích là
\choice
{$\dfrac{1}{13}$}
{\True $\dfrac{1}{4}$}
{$\dfrac{12}{13}$}
{$\dfrac{3}{4}$}
\loigiai{
Bộ bài gồm có $13$ lá bài bích. Vậy xác suất để lấy được lá bích là\\
$P=\dfrac{\mathrm{C}_{13}^1}{\mathrm{C}_{52}^1}=\dfrac{13}{52}=\dfrac{1}{4}$.
}
\end{ex}
%Câu 65
\begin{ex}
Rút một lá bài từ bộ bài gồm $52$ lá. Xác suất để được lá $10$ hay lá át là
\choice
{\True $\dfrac{2}{13}$}
{$\dfrac{1}{169}$}
{$\dfrac{4}{13}$}
{$\dfrac{3}{4}$}
\loigiai{
Trong bộ bài có bốn lá $10$ và bốn lá át nên xác suất để lấy được lá $10$ hay lá át là\\
$P=\dfrac{\mathrm{C}_8^1}{\mathrm{C}_{52}^1}=\dfrac{8}{52}=\dfrac{2}{13}$.
}
\end{ex}
%Câu 66
\begin{ex}
Rút một lá bài từ bộ bài gồm $52$ lá. Xác suất để được lá át hay lá rô là
\choice
{$\dfrac{1}{52}$}
{$\dfrac{2}{13}$}
{\True $\dfrac{4}{13}$}
{$\dfrac{17}{52}$}
\loigiai{
Trong bộ bài có ba lá át (không tính lá át rô) và $13$ lá rô nên xác suất để lấy được lá át hay lá rô là\\
$P=\dfrac{\mathrm{C}_{16}^1}{\mathrm{C}_{52}^1}=\dfrac{16}{52}=\dfrac{4}{13}$.
}
\end{ex}
%Câu 67
\begin{ex}
Rút một lá bài từ bộ bài gồm $52$ lá. Xác suất để được lá át (A) hay lá già (K) hay lá đầm (Q) là
\choice
{$\dfrac{1}{2197}$}
{$\dfrac{1}{64}$}
{$\dfrac{1}{13}$}
{\True $\dfrac{3}{13}$}
\loigiai{
Trong bộ bài có bốn lá át (A), bốn lá già (K) và bốn lá đầm (Q) nên xác suất để lấy được lá át (A) hay lá già (K) hay lá đầm (Q) là\\
$P=\dfrac{\mathrm{C}_{12}^1}{\mathrm{C}_{52}^1}=\dfrac{12}{52}=\dfrac{3}{13}$.
}
\end{ex}
%Câu 68
\begin{ex}
Rút một lá bài từ bộ bài gồm $52$ lá. Xác suất để được lá bồi (J) màu đỏ hay lá $5$ là
\choice
{$\dfrac{1}{13}$}
{\True $\dfrac{3}{26}$}
{$\dfrac{3}{13}$}
{$\dfrac{1}{238}$}
\loigiai{
Trong bộ bài có hai lá bồi (J) màu đỏ và bốn lá $5$ nên xác suất để lấy được lá bồi (J) màu đỏ hay lá $5$ là\\
$P=\dfrac{\mathrm{C}_6^1}{\mathrm{C}_{52}^1}=\dfrac{6}{52}=\dfrac{3}{26}$.
}
\end{ex}
%Câu 69
\begin{ex}
Từ các chữ số $1,2$, $4,6$, $8,9$ lấy ngẫu nhiên một số. Xác suất để lấy được một số nguyên tố là:
\choice
{$\dfrac{1}{2}$}
{$\dfrac{1}{3}$}
{$\dfrac{1}{4}$}
{\True $\dfrac{1}{6}$}
\loigiai{
Số phần tử không gian mẫu:$n\left(\Omega\right)=6$\\
Biến cố số lấy được là số nguyên tố là: $A=\{2\}$ nên $|\Omega_A|=1$.\\
Suy ra $P(A)=\dfrac{|\Omega_A|}{n\left(\Omega\right)}=\dfrac{1}{6}$
}
\end{ex}
%Câu 70
\begin{ex}
Cho hai biến cố $A$ và $B$ có $P(A)=\dfrac{1}{3},P(B)=\dfrac{1}{4},P(A\cup B)=\dfrac{1}{2}$. Ta kết luận hai biến cố $A$ và $B$ là:
\choice
{Độc lập}
{\True Không xung khắc}
{Xung khắc}
{Không rõ}
\loigiai{
Ta có: $P\left(A\cup B\right)=P(A)+P(B)-P\left(A\cap B\right)$ nên $P\left(A\cap B\right)=\dfrac{1}{12}\ne 0$\\
Suy ra hai biến cố $A$ và $B$ là hai biến cố không xung khắc
}
\end{ex}
%Câu 71
\begin{ex}
Một túi chứa $2$ bi trắng và $3$ bi đen. Rút ra $3$ bi. Xác suất để được ít nhất $1$ bi trắng là:
\choice
{$\dfrac{1}{5}$}
{$\dfrac{1}{10}$}
{\True $\dfrac{9}{10}$}
{$\dfrac{4}{5}$}
\loigiai{
Số phần tử của không gian mẫu: $n\left(\Omega\right)=\mathrm{C}_5^3=10$\\
Số khả năng để có không có bi trắng là: $n\left(\overline{A}\right)=\mathrm{C}_3^3=1$\\
Suy ra $P(A)=1-\dfrac{n\left(\overline{A}\right)}{n\left(\Omega\right)}=1-\dfrac{1}{10}=\dfrac{9}{10}$
}
\end{ex}
%Câu 72
\begin{ex}
Một hộp đựng $4$ bi xanh và $6$ bi đỏ lần lượt rút $2$ viên bi. Xác suất để rút được một bi xanh và 1 bi đỏ là:
\choice
{$\dfrac{2}{15}$}
{$\dfrac{6}{25}$}
{$\dfrac{8}{25}$}
{\True $\dfrac{4}{15}$}
\loigiai{
Phép thử : Rút lần lượt hai viên bi\\
Ta có $n\left(\Omega\right)=9.10=90$\\
Biến cố $A$ : Rút được một bi xanh, một bi đỏ\\
$|\Omega_A|=4.6=24$\\
$\Rightarrow p(A)=\dfrac{|\Omega_A|}{n\left(\Omega\right)}=\dfrac{4}{15}$
}
\end{ex}
%Câu 73
\begin{ex}
Một bình đựng $5$ quả cầu xanh và $4$ quả cầu đỏ và $3$ quả cầu vàng. Chọn ngẫu nhiên $3$ quả cầu. Xác suất để được $3$ quả cầu khác màu là:
\choice
{$\dfrac{3}{5}$}
{$\dfrac{3}{7}$}
{\True $\dfrac{3}{11}$}
{$\dfrac{3}{14}$}
\loigiai{
Phép thử : Rút ngẫu nhiên ba quả cầu\\
Ta có $n\left(\Omega\right)=\mathrm{C}_{12}^3=220$\\
Biến cố $A$ : Rút được ba qua cầu khác màu\\
$|\Omega_A|=5.4\cdot 3=60$\\
$\Rightarrow p(A)=\dfrac{|\Omega_A|}{n\left(\Omega\right)}=\dfrac{3}{11}$
}
\end{ex}
%Câu 74
\begin{ex}
Một bình đựng $4$ quả cầu xanh và $6$ quả cầu trắng. Chọn ngẫu nhiên $3$ quả cầu. Xác suất để được $3$ quả cầu toàn màu xanh là:
\choice
{$\dfrac{1}{20}$}
{\True $\dfrac{1}{30}$}
{$\dfrac{1}{15}$}
{$\dfrac{3}{10}$}
\loigiai{
Phép thử : Chọn ngẫu nhiên ba quả cầu\\
Ta có $n\left(\Omega\right)=\mathrm{C}_{10}^3=120$\\
Biến cố $A$ : Được ba quả toàn màu xanh\\
$\Rightarrow |\Omega_A|=\mathrm{C}_4^3=4$\\
$\Rightarrow p(A)=\dfrac{|\Omega_A|}{n\left(\Omega\right)}=\dfrac{1}{30}$
}
\end{ex}
%Câu 75
\begin{ex}
Một bình đựng $4$ quả cầu xanh và $6$ quả cầu trắng. Chọn ngẫu nhiên $4$ quả cầu. Xác suất để được $2$ quả cầu xanh và $2$ quả cầu trắng là:
\choice
{$\dfrac{1}{20}$}
{\True $\dfrac{3}{7}$}
{$\dfrac{1}{7}$}
{$\dfrac{4}{7}$}
\loigiai{
Phép thử : Chọn ngẫu nhiên bốn quả cầu\\
Ta có $n\left(\Omega\right)=\mathrm{C}_{10}^4=210$\\
Biến cố $A$ : Được hai quả xanh, hai quả trắng\\
$\Rightarrow |\Omega_A|=\mathrm{C}_4^2\cdot \mathrm{C}_6^2=90$\\
$\Rightarrow p(A)=\dfrac{|\Omega_A|}{n\left(\Omega\right)}=\dfrac{3}{7}$
}
\end{ex}
%Câu 76
\begin{ex}
Một hộp đựng $4$ bi xanh và $6$ bi đỏ lần lượt rút $2$ viên bi. Xác suất để rút được một bi xanh và một bi đỏ là
\choice
{$\dfrac{4}{15}$}
{$\dfrac{6}{25}$}
{$\dfrac{8}{25}$}
{\True $\dfrac{8}{15}$}
\loigiai{
$n\left(\Omega\right)=\mathrm{C}_{10}^2=45$.\\
$\text{A}$: \lq\lq rút được một bi xanh và một bi đỏ\rq\rq .\\
+ Rút $1$ bi xanh từ $4$ bi xanh, có $\mathrm{C}_4^1=4$ (cách).\\
+ Rút $1$ bi đỏ từ $6$ bi đỏ, có $\mathrm{C}_6^1=6$ (cách).\\
+ Vậy số cách $\mathrm{C}_4^1\cdot \mathrm{C}_6^1=24$.\\
KL: $P(A)=\dfrac{|\Omega_A|}{n\left(\Omega\right)}=\dfrac{24}{45}=\dfrac{8}{15}$
}
\end{ex}
%Câu 77
\begin{ex}
Một bình đựng $5$ quả cầu xanh và $4$ quả cầu đỏ và $3$ quả cầu vàng. Chọn ngẫu nhiên $3$ quả cầu. Xác suất để được $3$ quả cầu khác màu là
\choice
{$\dfrac{3}{5}$}
{$\dfrac{3}{7}$}
{\True $\dfrac{3}{11}$}
{$\dfrac{3}{14}$}
\loigiai{
$n\left(\Omega\right)=\mathrm{C}_{12}^3=220$.\\
$\text{A}$: \lq\lq chọn được $3$ quả cầu khác màu\rq\rq .\\
Chỉ có trường hợp: $1$ quả cầu xanh, $1$ quả cầu đỏ, $1$ quả cầu vàng, có $|\Omega_A|=\mathrm{C}_5^1\cdot \mathrm{C}_4^1\cdot \mathrm{C}_3^1=60$.\\
KL: $P(A)=\dfrac{|\Omega_A|}{n\left(\Omega\right)}=\dfrac{60}{220}=\dfrac{3}{11}$
}
\end{ex}
%Câu 78
\begin{ex}
Một bình đựng $4$ quả cầu xanh và $6$ quả cầu trắng. Chọn ngẫu nhiên $3$ quả cầu. Xác suất để được $3$ quả cầu toàn màu xanh là
\choice
{$\dfrac{1}{20}$}
{\True $\dfrac{1}{30}$}
{$\dfrac{1}{15}$}
{$\dfrac{3}{10}$}
\loigiai{
$n\left(\Omega\right)=\mathrm{C}_{10}^3=120$.\\
$\text{A}$: \lq\lq được $3$ quả cầu toàn màu xanh\rq\rq  có $|\Omega_A|=\mathrm{C}_4^3=4$.\\
KL: $P(A)=\dfrac{|\Omega_A|}{n\left(\Omega\right)}=\dfrac{4}{120}=\dfrac{1}{30}$
}
\end{ex}
%Câu 79
\begin{ex}
Một bình đựng $4$ quả cầu xanh và $6$ quả cầu trắng. Chọn ngẫu nhiên $4$ quả cầu. Xác suất để được $2$ quả cầu xanh và $2$ quả cầu trắng là
\choice
{$\dfrac{1}{20}$}
{\True $\dfrac{3}{7}$}
{$\dfrac{1}{7}$}
{$\dfrac{4}{7}$}
\loigiai{
$n\left(\Omega\right)=\mathrm{C}_{10}^4=210$.\\
$\text{A}$: \lq\lq được $2$ quả cầu xanh và $2$ quả cầu trắng\rq\rq  có $\mathrm{C}_4^2\cdot \mathrm{C}_6^2=90$.\\
KL: $P(A)=\dfrac{|\Omega_A|}{n\left(\Omega\right)}=\dfrac{90}{210}=\dfrac{3}{7}$
}
\end{ex}
%Câu 80
\begin{ex}
Một hộp chứa $4$ viên bi trắng, $5$ viên bi đỏ và $6$ viên bi xanh. Lấy ngẫu nhiên từ hộp ra $4$ viên bi. Xác suất để $4$ viên bi được chọn có đủ ba màu và số bi đỏ nhiều nhất là
\choice
{\True $P=\dfrac{\mathrm{C}_4^1\mathrm{C}_5^2\mathrm{C}_6^1}{\mathrm{C}_{15}^4}$}
{$P=\dfrac{\mathrm{C}_4^1\mathrm{C}_5^3\mathrm{C}_6^2}{\mathrm{C}_{15}^2}$}
{$P=\dfrac{\mathrm{C}_4^1\mathrm{C}_5^2\mathrm{C}_6^1}{\mathrm{C}_{15}^2}$}
{$P=\dfrac{\mathrm{C}_4^1\mathrm{C}_5^2\mathrm{C}_6^1}{\mathrm{C}_{15}^2}$}
\loigiai{
Số phần tử không gian mẫu: $n\left(\Omega\right)=\mathrm{C}_{15}^4$.
Gọi $A$ là biến cố cần tìm. Khi đó: $|\Omega_A|=\mathrm{C}_4^1\cdot \mathrm{C}_5^2\cdot \mathrm{C}_6^1$ (vì số bi đỏ nhiều nhất là 2)
Xác suất của biến cố $A$ là $P(A)=\dfrac{|\Omega_A|}{n\left(\Omega\right)}=\dfrac{\mathrm{C}_4^1\cdot \mathrm{C}_5^2\cdot \mathrm{C}_6^1}{\mathrm{C}_{15}^4}$
}
\end{ex}
%Câu 81
\begin{ex}
Một hộp có $5$ bi đen, $4$ bi trắng. Chọn ngẫu nhiên$2$ bi. Xác suất$2$ bi được chọn có đủ hai màu là
\choice
{$\dfrac{5}{324}$}
{\True $\dfrac{5}{9}$}
{$\dfrac{2}{9}$}
{$\dfrac{1}{18}$}
\loigiai{
Số phần tử không gian mẫu: $n\left(\Omega\right)=\mathrm{C}_9^2=36$.\\
(bốc 2 bi bất kì từ 9 bi trong hộp).\\
Gọi $A$: \lq\lq hai bi được chọn có đủ hai màu \rq\rq . Ta có: $|\Omega_A|=\mathrm{C}_5^1\cdot \mathrm{C}_4^1=20$.\\
(chọn 1 bi đen từ 5 bi đen - chọn 1 bi trắng từ 4 bi trắng).\\
Khi đó: $P(A)=\dfrac{|\Omega_A|}{n\left(\Omega\right)}=\dfrac{20}{36}=\dfrac{5}{9}$
}
\end{ex}
%Câu 82
\begin{ex}
Một bình chứa 16 viên bi với 7 viên bi trắng, 6 viên bi đen và 3 viên bi đỏ. Lấy ngẫu nhiên 3 viên bi. Tính xác suất lấy được cả 3 viên bi đỏ.
\choice
{\True $\dfrac{1}{560}$}
{$\dfrac{9}{40}$}
{$\dfrac{1}{28}$}
{$\dfrac{143}{280}$}
\loigiai{
$|\Omega|=\mathrm{C}_{16}^3=560$. Gọi $A$:\rq\rq lấy được 3 viên bi đỏ\rq\rq .\\
Ta có $|\Omega_A|=1$. Vậy $P(A)=\dfrac{1}{560}$
}
\end{ex}
%Câu 83
\begin{ex}
Một bình chứa 16 viên bi với 7 viên bi trắng, 6 viên bi đen và 3 viên bi đỏ. Lấy ngẫu nhiên 3 viên bi. Tính xác suất lấy được cả 3 viên bi không đỏ.
\choice
{$\dfrac{1}{560}$}
{$\dfrac{9}{40}$}
{$\dfrac{1}{28}$}
{\True $\dfrac{143}{280}$}
\loigiai{
$|\Omega|=\mathrm{C}_{16}^3=560$. Gọi $A$:\rq\rq lấy được 3 viên bi đỏ\rq\rq  thì $A$:\rq\rq lấy được 3 viên bi trắng hoặc đen\rq\rq \\
Có $7+6=13$ viên bi trắng hoặc đen. Ta có $|\Omega_A|=\mathrm{C}_{13}^3=286$. Vậy $P(A)=\dfrac{286}{560}=\dfrac{143}{280}$
}
\end{ex}
%Câu 84
\begin{ex}
Một bình chứa 16 viên bi với 7 viên bi trắng, 6 viên bi đen và 3 viên bi đỏ. Lấy ngẫu nhiên 3 viên bi. Tính xác suất lấy được cả 1 viên bi trắng, 1 viên bi đen, 1 viên bi đỏ.
\choice
{$\dfrac{1}{560}$}
{\True $\dfrac{9}{40}$}
{$\dfrac{1}{28}$}
{$\dfrac{143}{280}$}
\loigiai{
$|\Omega|=\mathrm{C}_{16}^3=560$. Gọi $A$:\rq\rq lấy được 1 viên bi trắng, 1 viên vi đen, 1 viên bi đỏ\rq\rq \\
Ta có $|\Omega_A|=7.6\cdot 3=126$. Vậy $P(A)=\dfrac{126}{560}=\dfrac{9}{40}$
}
\end{ex}
%Câu 85
\begin{ex}
Từ một hộp chứa ba quả cầu trắng và hai quả cầu đen lấy ngẫu nhiên hai quả. Xác suất để lấy được cả hai quả trắng là:
\choice
{\True $\dfrac{9}{30}$}
{$\dfrac{12}{30}$}
{$\dfrac{10}{30}$}
{$\dfrac{6}{30}$}
\loigiai{
$|\Omega|=\mathrm{C}_5^2=10$. Gọi $A$:\rq\rq Lấy được hai quả màu trắng\rq\rq .\\
Ta có $|\Omega_A|=\mathrm{C}_3^2=3$. Vậy $P(A)=\dfrac{3}{10}=\dfrac{9}{30}$
}
\end{ex}
%Câu 86
\begin{ex}
Một bình đựng $5$ viên bi xanh và $3$ viên bi đỏ (các viên bi chỉ khác nhau về màu sắc). Lấy ngẫu nhiên một viên bi, rồi lấy ngẫu nhiên một viên bi nữa. Khi tính xác suất của biến cố \lq\lq Lấy lần thứ hai được một viên bi xanh\rq\rq , ta được kết quả
\choice
{\True $\dfrac{5}{8}$}
{$\dfrac{5}{9}$}
{$\dfrac{5}{7}$}
{$\dfrac{4}{7}$}
\loigiai{
Gọi $A$ là biến cố \lq\lq Lấy lần thứ hai được một viên bi xanh\rq\rq . Có hai trường hợp xảy ra\\
Trường hợp 1. Lấy lần thứ nhất được bi xanh, lấy lần thứ hai cũng được một bi xanh. Xác suất trong trường hợp này là $P_1=\dfrac{5}{8}\cdot \dfrac{4}{7}=\dfrac{5}{14}$.\\
Trường hợp 2. Lấy lần thứ nhất được bi đỏ, lấy lần thứ hai được bi xanh. Xác suất trong trường hợp này là $P_2=\dfrac{3}{8}\cdot \dfrac{5}{7}=\dfrac{15}{56}$.\\
Vậy $P(A)=P_1+P_2=\dfrac{5}{14}+\dfrac{15}{56}=\dfrac{35}{56}=\dfrac{5}{8}$.
}
\end{ex}
%Câu 87
\begin{ex}
Một hộp có 5 viên bi đỏ và 9 viên bi xanh. Chọn ngẫu nhiên 2 viên bi. Xác suất để chọn được 2 viên bi khác màu là:
\choice
{$\dfrac{14}{45}$}
{\True $\dfrac{45}{91}$}
{$\dfrac{46}{91}$}
{$\dfrac{15}{22}$}
\loigiai{
Gọi A là biến cố: \lq\lq chọn được 2 viên bi khác màu.\lq\lq \\
-Không gian mẫu: $\left| \Omega \right|=\mathrm{C}_{14}^2=91$..\\
-$|\Omega_A|=\mathrm{C}_5^1\cdot \mathrm{C}_9^1=45$.\\
=>$P(A)=\dfrac{|\Omega_A|}{\left| \Omega \right|}=\dfrac{45}{91}$.
}
\end{ex}
%Câu 88
\begin{ex}
Một hộp chứa ba quả cầu trắng và hai quả cầu đen. Lấy ngẫu nhiên đồng thời hai quả. Xác suất để lấy được cả hai quả trắng là:
\choice
{$\dfrac{2}{10}$}
{\True $\dfrac{3}{10}$}
{$\dfrac{4}{10}$}
{$\dfrac{5}{10}$}
\loigiai{
Gọi A là biến cố: \lq\lq lấy được cả hai quả trắng.\rq\rq \\
-Không gian mẫu: $\mathrm{C}_5^2=10$.\\
-$|\Omega_A|=\mathrm{C}_3^2=3$.\\
=>$P(A)=\dfrac{|\Omega_A|}{\left| \Omega \right|}=\dfrac{3}{10}$.
}
\end{ex}
%Câu 89
\begin{ex}
Một hộp chứa sáu quả cầu trắng và bốn quả cầu đen. Lấy ngẫu nhiên đồng thời bốn quả. Tính xác suất sao cho có ít nhất một quả màu trắng?
\choice
{$\dfrac{1}{21}$}
{$\dfrac{1}{210}$}
{\True $\dfrac{209}{210}$}
{$\dfrac{8}{105}$}
\loigiai{
Gọi A là biến cố: \lq\lq trong bốn quả được chọn có ít nhất 1 quả trắng.\rq\rq \\
-Không gian mẫu: $\mathrm{C}_{10}^4=210$.\\
-$\overline{A}$ là biến cố: \lq\lq trong bốn quả được chọn không có 1 quả trắng nào.\rq\rq \\
=>$n\left(\overline{A}\right)=\mathrm{C}_4^4=1$.\\
=>$P\left(\overline{A}\right)=\dfrac{n\left(\overline{A}\right)}{\left| \Omega \right|}=\dfrac{1}{210}$.\\
=> $P(A)=1-P\left(\overline{A}\right)=1-\dfrac{1}{210}=\dfrac{209}{210}$.
}
\end{ex}
%Câu 90
\begin{ex}
Có hai hộp đựng bi. Hộp I có 9 viên bi được đánh số $1, 2, \ldots , 9$. Lấy ngẫu nhiên mỗi hộp một viên bi. Biết rằng xác suất để lấy được viên bi mang số chẵn ở hộp II là $\dfrac{3}{10}$. Xác suất để lấy được cả hai viên bi mang số chẵn là:
\choice
{$\dfrac{2}{15}$}
{\True $\dfrac{1}{15}$}
{$\dfrac{4}{15}$}
{$\dfrac{7}{15}$}
\loigiai{
Gọi X là biến cố: \lq\lq lấy được cả hai viên bi mang số chẵn. \lq\lq \\
Gọi A là biến cố: \lq\lq lấy được viên bi mang số chẵn ở hộp I \lq\lq \\
=>$P(A)=\dfrac{\mathrm{C}_4^1}{\mathrm{C}_9^1}=\dfrac{4}{9}$.\\
Gọi B là biến cố: \lq\lq lấy được viên bi mang số chẵn ở hộp II \lq\lq $P(B)=\dfrac{3}{10}$.\\
Ta thấy biến cố A, B là 2 biến cố độc lập nhau, theo công thức nhân xác suất ta có:\\
$P(X)=P\left(A.B\right)=P(A)\cdot P(B)=\dfrac{4}{9}\cdot \dfrac{3}{10}=\dfrac{1}{15}$.
}
\end{ex}
%Câu 91
\begin{ex}
Một hộp chứa $5$ viên bi màu trắng, $15$ viên bi màu xanh và $35$ viên bi màu đỏ. Lấy ngẫu nhiên từ hộp ra 7 viên bi. Xác suất để trong số 7 viên bi được lấy ra có ít nhất 1 viên bi màu đỏ là:
\choice
{$\mathrm{C}_{35}^1$}
{\True $\dfrac{\mathrm{C}_{55}^7-\mathrm{C}_{20}^7}{\mathrm{C}_{55}^7}$}
{$\dfrac{\mathrm{C}_{35}^7}{\mathrm{C}_{55}^7}$}
{$\mathrm{C}_{35}^1\cdot \mathrm{C}_{20}^6$}
\loigiai{
Gọi A là biến cố: \lq\lq trong số $7$ viên bi được lấy ra có ít nhất 1 viên bi màu đỏ.\rq\rq \\
-Không gian mẫu: $\mathrm{C}_{55}^7$.\\
-$\overline{A}$ là biến cố: \lq\lq trong số 7 viên bi được lấy ra không có viên bi màu đỏ nào.\rq\rq \\
=>$n\left(\overline{A}\right)=\mathrm{C}_{20}^7$.\\
=>$|\Omega_A|=\Omega -n\left(\overline{A}\right)=\mathrm{C}_{55}^7-\mathrm{C}_{20}^7$.\\
=> $P(A)=\dfrac{\mathrm{C}_{55}^7-\mathrm{C}_{20}^7}{\mathrm{C}_{55}^7}$.
}
\end{ex}
%Câu 92
\begin{ex}
Trong một túi có 5 viên bi xanh và 6 viên bi đỏ; lấy ngẫu nhiên từ đó ra 2 viên bi. Khi đó xác suất để lấy được ít nhất một viên bi xanh là:
\choice
{$\dfrac{8}{11}$}
{$\dfrac{2}{11}$}
{\True $\dfrac{3}{11}$}
{$\dfrac{9}{11}$}
\loigiai{
Gọi A là biến cố: \lq\lq Lấy được ít nhất một viên bi xanh.\rq\rq \\
-Không gian mẫu: $\Omega =\mathrm{C}_{11}^2=55$.\\
-$\overline{A}$ là biến cố: \lq\lq Kông lấy được viên bi xanh nào.\rq\rq \\
=>$n\left(\overline{A}\right)=\mathrm{C}_6^2=15$.\\
=>$P\left(\overline{A}\right)=\dfrac{n\left(\overline{A}\right)}{\left| \Omega \right|}=\dfrac{15}{55}=\dfrac{3}{11}$.\\
=>$P(A)=1-P\left(\overline{A}\right)=1-\dfrac{3}{11}=\dfrac{8}{11}$.
}
\end{ex}
%Câu 93
\begin{ex}
Một bình đựng $12$ quả cầu được đánh số từ 1 đến 12. Chọn ngẫu nhiên bốn quả cầu. Xác suất để bốn quả cầu được chọn có số đều không vượt quá 8.
\choice
{$\dfrac{56}{99}$}
{$\dfrac{7}{99}$}
{\True $\dfrac{14}{99}$}
{$\dfrac{28}{99}$}
\loigiai{
Gọi A là biến cố: \lq\lq bốn quả cầu được chọn có số đều không vượt quá 8.\rq\rq \\
-Không gian mẫu: $\left| \Omega \right|=\mathrm{C}_{12}^4=495$.\\
-$|\Omega_A|=\mathrm{C}_8^4=70$.\\
=>$P(A)=\dfrac{|\Omega_A|}{\left| \Omega \right|}=\dfrac{70}{495}=\dfrac{14}{99}$.
}
\end{ex}
%Câu 94
\begin{ex}
Một bình chứa $16$ viên bi với $7$ viên bi trắng, $6$ viên bi đen, 3 viên bi đỏ. Lấy ngẫu nhiên 3 viên bi. Tính xác suất lấy được 1 viên bi trắng, 1 viên bi đen, 1 viên bi đỏ.
\choice
{$\dfrac{1}{560}$}
{$\dfrac{1}{16}$}
{\True $\dfrac{9}{40}$}
{$\dfrac{143}{240}$}
\loigiai{
Gọi A là biến cố: \lq\lq lấy được 1 viên bi trắng, 1 viên bi đen, 1 viên bi đỏ.\rq\rq \\
-Không gian mẫu: $\left| \Omega \right|=\mathrm{C}_{16}^3=560$.\\
-$|\Omega_A|=\mathrm{C}_7^1\cdot \mathrm{C}_6^1\cdot \mathrm{C}_3^1=126$.\\
=>$P(A)=\dfrac{|\Omega_A|}{\left| \Omega \right|}=\dfrac{126}{560}=\dfrac{9}{40}$.
}
\end{ex}
%Câu 95
\begin{ex}
Có $3$ viên bi đỏ và $7$ viên bi xanh, lấy ngẫu nhiên $4$ viên bi. Tính xác suất để lấy được $2$ bi đỏ và $2$ bi xanh ?
\choice
{$\dfrac{12}{35}$}
{$\dfrac{126}{7920}$}
{\True $\dfrac{21}{70}$}
{$\dfrac{4}{35}$}
\loigiai{
Số phần tử của không gian mẫu là: $\left| \Omega \right|=\mathrm{C}_{10}^4=210$.\\
Số phần tử của không gian thuận lợi là: $\left| \Omega _A \right|=\mathrm{C}_3^2\cdot \mathrm{C}_7^2=63$\\
Xác suất biến cố $A$ là : $P(A)=\dfrac{21}{70}$
}
\end{ex}
%Câu 96
\begin{ex}
Một bình đựng $8$ viên bi xanh và $4$ viên bi đỏ. Lấy ngẫu nhiên $3$ viên bi. Xác suất để có được ít nhất hai viên bi xanh là bao nhiêu?
\choice
{$\dfrac{28}{55}$}
{$\dfrac{14}{55}$}
{$\dfrac{41}{55}$}
{\True $\dfrac{42}{55}$}
\loigiai{
Số phần tử của không gian mẫu là: $\left| \Omega \right|=\mathrm{C}_{12}^3$.\\
Số phần tử của không gian thuận lợi là: $\left| \Omega _A \right|=\mathrm{C}_8^3+\mathrm{C}_8^2\cdot \mathrm{C}_4^1$\\
Xác suất biến cố $A$ là : $P(A)=\dfrac{42}{55}$
}
\end{ex}
%Câu 97
\begin{ex}
Bạn Tít có một hộp bi gồm $2$ viên đỏ và $8$ viên trắng. Bạn Mít cũng có một hộp bi giống như của bạn Tít. Từ hộp của mình, mỗi bạn lấy ra ngẫu nhiên $3$ viên bi. Tính xác suất để Tít và Mít lấy được số bi đỏ như nhau
\choice
{\True $\dfrac{11}{25}$}
{$\dfrac{1}{120}$}
{$\dfrac{7}{15}$}
{$\dfrac{12}{25}$}
\loigiai{
Số phần tử của không gian mẫu là: $\left| \Omega \right|=\mathrm{C}_{10}^3\cdot \mathrm{C}_{10}^3=14400$.\\
Số phần tử của không gian thuận lợi là: $\left| \Omega _A \right|=\left(\mathrm{C}_2^1\cdot \mathrm{C}_8^2\right)^2+\left(\mathrm{C}_2^2\cdot \mathrm{C}_8^1\right)^2+\left(\mathrm{C}_8^3\right)^2=6336$\\
Xác suất biến cố $A$ là : $P(A)=\dfrac{11}{25}$
}
\end{ex}
%Câu 98
\begin{ex}
Một hộp có $5$ viên bi đỏ và $9$ viên bi xanh. Chọn ngẫu nhiên $2$ viên bi. Xác suất để chọn được $2$ viên bi khác màu là:
\choice
{$\dfrac{14}{45}$}
{\True $\dfrac{45}{91}$}
{$\dfrac{46}{91}$}
{$\dfrac{15}{22}$}
\loigiai{
Số phần tử của không gian mẫu là: $\left| \Omega \right|=\mathrm{C}_{14}^2=91$.\\
Số phần tử của không gian thuận lợi là: $\left| \Omega _A \right|=\mathrm{C}_{14}^2-\mathrm{C}_5^2-\mathrm{C}_9^2=45$.\\
Xác suất biến cố $A$ là : $P(A)=\dfrac{45}{91}$
}
\end{ex}
%Câu 99
\begin{ex}
Một hộp chứa 5 bi xanh và 10 bi đỏ. Lấy ngẫu nhiên 3 bi. Xác suất để được đúng một bi xanh là:
\choice
{\True $\dfrac{45}{91}$}
{$\dfrac{2}{3}$}
{$\dfrac{3}{4}$}
{$\dfrac{200}{273}$}
\loigiai{
Số phần tử của không gian mẫu là: $\left| \Omega \right|=\mathrm{C}_{15}^3$.\\
Gọi A là biến cố để được đúng một bi xanh.\\
Số phần tử của không gian thuận lợi là: $\left| \Omega _A \right|=\mathrm{C}_5^1\cdot \mathrm{C}_{10}^2$.\\
Xác suất biến cố $A$ là : $P(A)=\dfrac{45}{91}$}
\end{ex}
%Câu 100:
\begin{ex}
Một bình chứa $2$ bi xanh và $3$ bi đỏ. Rút ngẫu nhiên $3$ bi. Xác suất để được ít nhất một bi xanh là.
\choice
{$\dfrac{1}{5}$}
{$\dfrac{1}{10}$}
{\True $\dfrac{9}{10}$}
{$\dfrac{4}{5}$}
\loigiai{
Số phần tử của không gian mẫu là: $\left| \Omega \right|=\mathrm{C}_5^3$.\\
Gọi A là biến cố để được ít nhất một bi xanh.\\
Số phần tử của không gian thuận lợi là: $\left| \Omega _A \right|=\mathrm{C}_5^3-\mathrm{C}_3^3$.\\
Xác suất biến cố $A$ là : $P(A)=\dfrac{9}{10}$}
\end{ex}
%Câu 101
\begin{ex}
Một hộp chứa 7 bi xanh, 5 bi đỏ, 3 bi vàng. Xác suất để trong lần thứ nhất bốc được một bi mà không phải là bi đỏ là:
\choice
{$\dfrac{1}{3}$}
{\True $\dfrac{2}{3}$}
{$\dfrac{10}{21}$}
{$\dfrac{11}{21}$}
\loigiai{
+ Số phần tử của không gian mẫu là : $n\left(\Omega\right)=15$\\
+ Gọi biến cố A \lq\lq  lần thứ nhất bốc được một bi mà không phải bi đỏ \rq\rq \\
Ta có : $|\Omega_A|=10$\\
Vậy xác suất biến cố A: $P(A)=\dfrac{n\left(\Omega\right)}{|\Omega_A|}=\dfrac{10}{15}=\dfrac{2}{3}$\\
Chưa tô đậm A, B, C D trong đáp án}
\end{ex}
%Câu 102
\begin{ex}
Một chứa 6 bi đỏ, 7 bi xanh. Nếu chọn ngẫu nhiên 5 bi từ hộp này. Thì xác suất đúng đến phần trăm để có đúng 2 bi đỏ là:
\choice
{0{,}14}
{\True 0{,}41}
{0{,}28}
{0{,}34}
\loigiai{
+ Số phần tử của không gian mẫu là : $n\left(\Omega\right)=\mathrm{C}_{13}^5$\\
+ Gọi biến cố A \lq\lq  5 bi được chọn có đúng 2 bi đỏ \rq\rq \\
Ta có : $|\Omega_A|=\mathrm{C}_7^2\cdot \mathrm{C}_6^3$\\
Vậy xác suất biến cố A: $P(A)=\dfrac{n\left(\Omega\right)}{|\Omega_A|}=\dfrac{175}{429}=0{,}41$\\
Chưa tô đậm A, B, C D trong đáp án}
\end{ex}
%Câu 103
\begin{ex}
Một hộp chứa 6 bi xanh, 7 bi đỏ. Nếu chọn ngẫu nhiên 2 bi từ hộp này. Thì xác suất để được 2 bi cùng màu là:
\choice
{\True 0{,}46}
{0{,}51}
{0{,}55}
{0{,}64}
\loigiai{
+ Số phần tử của không gian mẫu là : $n\left(\Omega\right)=\mathrm{C}_{13}^2$\\
+ Gọi biến cố A \lq\lq  hai viên bi được chọn cùng màu\rq\rq \\
Ta có : $|\Omega_A|=\mathrm{C}_6^2+\mathrm{C}_7^2$\\
Vậy xác suất biến cố A: $P(A)=\dfrac{n\left(\Omega\right)}{|\Omega_A|}=\dfrac{6}{13}=0{,}46$\\
Chưa tô đậm A, B, C D trong đáp án}
\end{ex}
%Câu 104
\begin{ex}
Một hộp chứa 3 bi xanh, 2 bi đỏ, 4 bi vàng. Lấy ngẫu nhiên 3 bi. Xác suất để đúng một bi đỏ là:
\choice
{$\dfrac{1}{3}$}
{$\dfrac{2}{5}$}
{\True $\dfrac{1}{2}$}
{$\dfrac{3}{5}$}
\loigiai{
+ Số phần tử của không gian mẫu là : $n\left(\Omega\right)=\mathrm{C}_9^3$\\
+ Gọi biến cố A \lq\lq  ba viên bi được chọn có đúng 1 viên bi đỏ \rq\rq \\
Ta có: $|\Omega_A|=2\cdot \mathrm{C}_7^2$\\
Vậy xác suất biến cố A: $P(A)=\dfrac{n\left(\Omega\right)}{|\Omega_A|}=\dfrac{1}{2}$}
\end{ex}
%Câu 105
\begin{ex}
Có 3 chiếc hộp. Hộp A chứa 3 bi đỏ, 5 bi trắng. Hộp B chứa 2 bi đỏ, hai bi vàng. Hộp C chứa 2 bi đỏ, 3 bi xanh. Lấy ngẫu nhiên một hộp rồi lấy một bi từ hộp đó. Xác suất để được một bi đỏ là:
\choice
{$\dfrac{1}{8}$}
{$\dfrac{1}{6}$}
{$\dfrac{2}{15}$}
{\True $\dfrac{17}{40}$}
\loigiai{
Lấy ngẫu nhiên một hộp\\
Gọi $C{}_1$ là biến cố lấy được hộp A\\
Gọi $\mathrm{C}_2$ là biến cố lấy được hộp B\\
Gọi $\mathrm{C}_3$ là biến cố lấy được hộp C\\
Vậy $P\left(\mathrm{C}_1\right)=P\left(\mathrm{C}_2\right)=P\left(\mathrm{C}_3\right)=\dfrac{1}{3}$\\
Gọi C là biến cố \lq\lq  lấy ngẫu nhiên một hộp, trong hộp đó lại lấy ngẫu nhiên một viên bi và được bi đỏ \rq\rq  là\\
$C=\left(C\cap \mathrm{C}_1\right)\cup \left(C\cap \mathrm{C}_2\right)\cup \left(C\cap \mathrm{C}_3\right)\Rightarrow P(C)=P\left(C\cap \mathrm{C}_1\right)+P\left(C\cap C{}_2\right)+P\left(C\cap \mathrm{C}_3\right)$\\
$=\dfrac{1}{3}\cdot \dfrac{3}{8}+\dfrac{1}{3}\cdot \dfrac{2}{4}+\dfrac{1}{3}\cdot \dfrac{2}{5}=\dfrac{17}{40}$\\
Chưa tô đậm A, B, C D trong đáp án, bài này không có trong chương trình phổ thông}
\end{ex}
%Câu 106
\begin{ex}
Một hộp chứa 3 bi đỏ, 2 bi vàng và 1 bi xanh. Lần lượt lấy ra ba bi và không bỏ lại. Xác suất để được bi thứ nhất đỏ, nhì xanh, ba vàng là:
\choice
{$\dfrac{1}{60}$}
{\True $\dfrac{1}{20}$}
{$\dfrac{1}{120}$}
{$\dfrac{1}{2}$}
\loigiai{
Xác suất để được bi thứ nhất đỏ, nhì xanh, ba vàng là:$\dfrac{3.1\cdot 2}{6.5\cdot 4}=\dfrac{1}{20}$}
\end{ex}
%Câu 107
\begin{ex}
Một hộp chứa 3 bi xanh và 2 bi đỏ. Lấy một bi lên xem rồi bỏ vào, rồi lấy một bi khác. Xác suất để được cả hai bi đỏ là:
\choice
{$\dfrac{4}{25}$}
{$\dfrac{1}{25}$}
{\True $\dfrac{2}{5}$}
{$\dfrac{1}{5}$}
\loigiai{
Lấy một bi lên xem rồi bỏ vào, rồi lấy một bi khác. Xác suất để được cả hai bi đỏ là: $\dfrac{2.2}{5.5}=\dfrac{4}{25}$}
\end{ex}
%Câu 108
\begin{ex}
Có hai chiếc hộp. Hộp thứ nhất chứa 1 bi xanh, 3 bi vàng. Hộp thứ nhì chứa 2 bi xanh, 1 bi đỏ. Lấy từ mỗi hộp một bi. Xác suất để được hai bi xanh là:
\choice
{$\dfrac{2}{3}$}
{$\dfrac{2}{7}$}
{\True $\dfrac{1}{6}$}
{$\dfrac{11}{12}$}
\loigiai{
Xác suất để được hai bi xanh là:$\dfrac{1.2}{4.3}=\dfrac{1}{6}$}
\end{ex}
%Câu 109
\begin{ex}
Mộthộpcó$5$ bi đen, $4$ bi trắng. Chọn ngẫu nhiên$2$ bi. Xác suất$2$ bi được chọn đều cùng màu là:
\choice
{$\dfrac{1}{4}$}
{$\dfrac{1}{9}$}
{\True $\dfrac{4}{9}$}
{$\dfrac{5}{9}$}
\loigiai{
Xác suất$2$ bi được chọn đều cùng màu là:$\dfrac{\mathrm{C}_5^2+\mathrm{C}_4^2}{\mathrm{C}_9^2}=\dfrac{4}{9}$}
\end{ex}
%Câu 110
\begin{ex}
Một hộp đựng $9$ thẻ được đánh số từ $1$ đến $9$. Rút ngẫu nhiên hai thẻ và nhân hai số ghi trên hai thẻ với nhau. Xác suất để tích hai số ghi trên hai thẻ là số lẻ là:
\choice
{$\dfrac{1}{9}$}
{\True $\dfrac{5}{18}$}
{$\dfrac{3}{18}$}
{$\dfrac{7}{18}$}
\loigiai{
Phép thử : Chọn ngẫu nhiên hai thẻ\\
Ta có $n\left(\Omega\right)=\mathrm{C}_9^2=36$\\
Biến cố $A$ : Rút được hai thẻ có tích là số lẻ\\
$|\Omega_A|=\mathrm{C}_5^2=10$\\
$\Rightarrow p(A)=\dfrac{|\Omega_A|}{n\left(\Omega\right)}=\dfrac{5}{18}$}
\end{ex}
%Câu 111
\begin{ex}
Cho $100$ tấm thẻ được đánh số từ $1$ đến $100$, chọn ngẫu nhiên $3$ tấm thẻ. Xác suất để chọn được $3$ tấm thẻ có tổng các số ghi trên thẻ là số chia hết cho $2$ là
\choice
{$P=\dfrac{5}{6}$}
{\True $P=\dfrac{1}{2}$}
{$P=\dfrac{5}{7}$}
{$P=\dfrac{3}{4}$}
\loigiai{
Số phần tử của không gian mẫu là $n\left(\Omega\right)=\mathrm{C}_{100}^3=161700$.\\
(bốc ngẫu nhiên 3 tấm thẻ từ 100 tấm thẻ).\\
Gọi $A$: \lq\lq tổng các số ghi trên thẻ là số chia hết cho $2$\rq\rq .\\
$|\Omega_A|=\mathrm{C}_{50}^3+\mathrm{C}_{50}^1\mathrm{C}_{50}^2=80850\Rightarrow P(A)=\dfrac{|\Omega_A|}{n\left(\Omega\right)}=\dfrac{1}{2}$.\\
(bốc 3 tấm thẻ đánh số chẵn từ 50 tấm thể đánh số chẵn hoặc 1 tấm thẻ đánh số chẵn từ 50 thẻ đánh số chẵn và 2 tấm thẻ đánh số lẻ từ 50 tấm thẻ đánh số lẻ)}
\end{ex}
%Câu 112
\begin{ex}
Một tổ học sinh gồm có$6$ nam và$4$ nữ. Chọn ngẫu nhiên$3$ em. Tính xác suất$3$ em được chọn có ít nhất 1 nữ
\choice
{\True $\dfrac{5}{6}$}
{$\dfrac{1}{6}$}
{$\dfrac{1}{30}$}
{$\dfrac{1}{2}$}
\loigiai{
Xác suất$3$ em được chọn có ít nhất 1 nữ là: $\dfrac{\mathrm{C}_{10}^3-\mathrm{C}_6^3}{\mathrm{C}_{10}^3}=\dfrac{5}{6}$}
\end{ex}
%Câu 113
\begin{ex}
Một tổ có 7 nam và 3 nữ. Chọn ngẫu nhiên 2 người. Tính xác suất sao cho 2 người được chọn đều là nữ.
\choice
{\True $\dfrac{1}{15}$}
{$\dfrac{2}{15}$}
{$\dfrac{7}{15}$}
{$\dfrac{8}{15}$}
\loigiai{
$|\Omega|=\mathrm{C}_{10}^2=45$\\
Gọi $A$:\rq\rq 2 người được chọn là nữ\rq\rq . Ta có $|\Omega_A|=\mathrm{C}_3^2=3$. Vậy $P(A)=\dfrac{3}{45}=\dfrac{1}{15}$}
\end{ex}
%Câu 114
\begin{ex}
Một tổ có 7 nam và 3 nữ. Chọn ngẫu nhiên 2 người. Tính xác suất sao cho 2 người được chọn không có nữ nào cả.
\choice
{$\dfrac{1}{15}$}
{$\dfrac{2}{15}$}
{\True $\dfrac{7}{15}$}
{$\dfrac{8}{15}$}
\loigiai{
$|\Omega|=\mathrm{C}_{10}^2=45$\\
Gọi $A$:\rq\rq 2 người được chọn không có nữ\rq\rq  thì $A$:\rq\rq 2 người được chọn đều là nam\rq\rq .\\
Ta có $|\Omega_A|=\mathrm{C}_7^2=21$. Vậy $P(A)=\dfrac{21}{45}=\dfrac{7}{15}$}
\end{ex}
%Câu 115
\begin{ex}
Một tổ có 7 nam và 3 nữ. Chọn ngẫu nhiên 2 người. Tính xác suất sao cho 2 người được chọn có ít nhất một nữ.
\choice
{$\dfrac{1}{15}$}
{$\dfrac{2}{15}$}
{$\dfrac{7}{15}$}
{\True $\dfrac{8}{15}$}
\loigiai{
$|\Omega|=\mathrm{C}_{10}^2=45$\\
Gọi $A$:\rq\rq 2 người được chọn có ít nhất 1 nữ\rq\rq  thì $\overline{A}$:\rq\rq 2 người được chọn không có nữ\rq\rq  hay\\
$\overline{A}$:\rq\rq 2 người được chọn đều là nam\rq\rq .\\
Ta có $n(\overline{A})=\mathrm{C}_7^2=21$. Do đó $P(\overline{A})=\dfrac{21}{45}$ suy ra $P(A)=1-P(\overline{A})=1-\dfrac{21}{45}=\dfrac{24}{45}=\dfrac{8}{15}$}
\end{ex}
%Câu 116
\begin{ex}
Một tổ có 7 nam và 3 nữ. Chọn ngẫu nhiên 2 người. Tính xác suất sao cho 2 người được chọn có đúng một người nữ.
\choice
{$\dfrac{1}{15}$}
{$\dfrac{2}{15}$}
{\True $\dfrac{7}{15}$}
{$\dfrac{8}{15}$}
\loigiai{
$|\Omega|=\mathrm{C}_{10}^2=45$. Gọi $A$:\rq\rq 2 người được chọn có đúng 1 nữ\rq\rq \\
Chọn 1 nữ có 3 cách, chọn 1 nam có 7 cách suy ra $|\Omega_A|=7.3=21$. Do đó $P(A)=\dfrac{21}{45}=\dfrac{7}{15}$}
\end{ex}
%Câu 117
\begin{ex}
Có 5 nam, 5 nữ xếp thành một hàng dọc. Tính xác suất để nam, nữ đứng xen kẽ nhau.
\choice
{$\dfrac{1}{125}$}
{\True $\dfrac{1}{126}$}
{$\dfrac{1}{36}$}
{$\dfrac{13}{36}$}
\loigiai{
Gọi A là biến cố: \lq\lq nam, nữ đứng xen kẽ nhau.\lq\lq \\
-Không gian mẫu: $\left| \Omega \right|=10!$.\\
-Số cách xếp để nam đứng đầu và nam nữ đứng xen kẽ nhau là: $5!\cdot 5!$\\
-Số cách xếp để nam đứng đầu và nam nữ đứng xen kẽ nhau là: $5!\cdot 5!$\\
=>$|\Omega_A|=5!\cdot 5!+5!\cdot 5!=28800$.\\
=>$P(A)=\dfrac{|\Omega_A|}{\left| \Omega \right|}=\dfrac{28800}{10!}=\dfrac{1}{126}$}
\end{ex}
%Câu 118
\begin{ex}
Lớp 11A1 có 41 học sinh trong đó có 21 bạn nam và 20 bạn nữ. Thứ 2 đầu tuần lớp phải xếp hàng chào cờ thành một hàng dọc. Hỏi có bao nhiêu cách sắp xếp để 21 bạn nam xen kẽ với 20 bạn nữ?
\choice
{$P_{41}$}
{$P_{21}-P_{20}$}
{\True $2\cdot P_{21} \cdot P_{20}$}
{$P_{21}+P_{20}$}
\loigiai{
-Số cách xếp để nam đứng đầu và nam, nữ đứng xen kẽ nhau là: ${{P}_{21}}\cdot {{P}_{20}}$.\\
-Số cách xếp để nam đứng đầu và nam, nữ đứng xen kẽ nhau là: ${{P}_{21}}\cdot {{P}_{20}}$.\\
=> Số cách sắp xếp để $21$ bạn nam xen kẽ với 20 bạn nữ là:\\
${{P}_{21}}\cdot {{P}_{20}}+{{P}_{21}}\cdot {{P}_{20}}=2\cdot {{P}_{21}}\cdot {{P}_{20}}$}
\end{ex}
%Câu 119
\begin{ex}
Một lớp có 20 học sinh nam và 18 học sinh nữ. Chọn ngẫu nhiên một học sinh. Tính xác suất chọn được một học sinh nữ.
\choice
{$\dfrac{1}{38}$}
{$\dfrac{10}{19}$}
{\True $\dfrac{9}{19}$}
{$\dfrac{19}{9}$}
\loigiai{
Gọi A là biến cố: \lq\lq chọn được một học sinh nữ.\rq\rq \\
-Không gian mẫu: $\left| \Omega \right|=\mathrm{C}_{38}^1=38$.\\
-$|\Omega_A|=\mathrm{C}_{18}^1=18$.\\
=>$P(A)=\dfrac{|\Omega_A|}{\left| \Omega \right|}=\dfrac{18}{38}=\dfrac{9}{19}$}
\end{ex}
%Câu 120
\begin{ex}
Một tổ học sinh có $7$ nam và $3$ nữ. Chọn ngẫu nhiên 2 người. Tính xác suất sao cho 2 người được chọn có đúng một người nữ.
\choice
{$\dfrac{1}{15}$}
{\True $\dfrac{7}{15}$}
{$\dfrac{8}{15}$}
{$\dfrac{1}{5}$}
\loigiai{
Gọi A là biến cố: \lq\lq 2 người được chọn có đúng một người nữ.\rq\rq \\
-Không gian mẫu: $\left| \Omega \right|=\mathrm{C}_{10}^2=45$.\\
-$|\Omega_A|=\mathrm{C}_3^1\cdot \mathrm{C}_7^1=21$.\\
=>$P(A)=\dfrac{|\Omega_A|}{\left| \Omega \right|}=\dfrac{21}{45}=\dfrac{7}{15}$}
\end{ex}
%Câu 121
\begin{ex}
Chọn ngẫu nhiên một số có $2$ chữ số từ các số $00$ đến $99$. Xác suất để có một con số tận cùng là $0$ là:
\choice
{\True $0{,}1$}
{$0{,}2$}
{$0{,}3$}
{$0{,}4$}
\loigiai{
Phép thử : Chọn một số có hai chữ số bất kì\\
Ta có $n\left(\Omega\right)=\mathrm{C}_{100}^1=100$\\
Biến cố $A$ : Chọn số có số tận cùng là $0$\\
$|\Omega_A|=\mathrm{C}_{10}^1=10$\\
$\Rightarrow p(A)=\dfrac{|\Omega_A|}{n\left(\Omega\right)}=0{,}1$}
\end{ex}
%Câu 122
\begin{ex}
Chọn ngẫu nhiên một số có hai chữ số từ các số $00$ đến $99$. Xác suất để có một con số lẻ và chia hết cho $9$:
\choice
{$0{,}12$}
{$0{,}6$}
{\True $0{,}06$}
{$0{,}01$}
\loigiai{
Phép thử : Chọn một số có hai chữ số bất kì\\
Ta có $n\left(\Omega\right)=\mathrm{C}_{100}^1=100$\\
Biến cố $A$ : Chọn số lẻ và chia hết cho $9$ là các số $09;81;27;63;45;99$\\
$|\Omega_A|=6$\\
$\Rightarrow p(A)=\dfrac{|\Omega_A|}{n\left(\Omega\right)}=0{,}06$}
\end{ex}
%Câu 123
\begin{ex}
Sắp $3$ quyển sách Toán và $3$ quyển sách Vật Lí lên một kệ dài. Xác suất để $2$ quyển sách cùng một môn nằm cạnh nhau là:
\choice
{$\dfrac{1}{5}$}
{\True $\dfrac{9}{10}$}
{$\dfrac{1}{20}$}
{$\dfrac{2}{5}$}
\loigiai{
Phép thử : Sắp ba quyển toán, ba quyển lí lên kệ dài\\
Ta có $n\left(\Omega\right)=6!=720$\\
Biến cố $A$ : Có hai quyển sách cùng môn nằm cạnh nhau\\
$\overline{A}$ : Các quyển sách cùng môn không nằm cạnh nhau\\
Có $n\left(\overline{A}\right)=2.3!\cdot 3!=72$\\
$|\Omega_A|=n\left(\Omega\right)-n\left(\overline{A}\right)=648$\\
$\Rightarrow p(A)=\dfrac{|\Omega_A|}{n\left(\Omega\right)}=\dfrac{9}{10}$}
\end{ex}
%Câu 124
\begin{ex}
Sắp $3$ quyển sách Toán và $3$ quyển sách Vật Lí lên một kệ dài. Xác suất để $2$ quyển sách cùng một môn nằm cạnh nhau là
\choice
{$\dfrac{1}{5}$}
{\True $\dfrac{1}{10}$}
{$\dfrac{1}{20}$}
{$\dfrac{2}{5}$}
\loigiai{
$n\left(\Omega\right)=6!=720$.\\
$\text{A}$: \lq\lq Xếp $2$ quyển sách cùng một môn nằm cạnh nhau\rq\rq . Số sách toán, số sách lý là số lẻ nên không thể xếp cùng môn nằm rời thành cặp (hoặc bội$2$) được. Do đó, phải xếp chúng cạnh nhau\\
+ Xếp vị trí nhóm sách toán - lý, có $2!$ (cách).\\
+ Ứng với mỗi cách trên, xếp vị trí của 3 sách toán, có $3!$ (cách); xếp vị trí của 3 sách lý, có $3!$ (cách).\\
+ Vậy số cách $|\Omega_A|=2!\cdot 3!\cdot 3!=72$.\\
KL: $P(A)=\dfrac{|\Omega_A|}{n\left(\Omega\right)}=\dfrac{72}{720}=\dfrac{1}{10}$}
\end{ex}
%Câu 125
\begin{ex}
Giải bóng chuyền VTVcup có $12$ đội tham gia trong đó có $9$ đội nước ngoài và $3$ đội của Việt Nam. Ban tổ chức bốc thăm ngẫu nhiên để chia thành $3$ bảng đấu A, B, C mỗi bảng $4$ đội. Xác suất để $3$ đội Việt Nam nằm ở $3$ bảng đấu khác nhau là
\choice
{$P=\dfrac{2\mathrm{C}_9^3\mathrm{C}_6^3}{\mathrm{C}_{12}^4\mathrm{C}_8^4}$}
{\True $P=\dfrac{6\mathrm{C}_9^3\mathrm{C}_6^3}{\mathrm{C}_{12}^4\mathrm{C}_8^4}$}
{$P=\dfrac{3\mathrm{C}_9^3\mathrm{C}_6^3}{\mathrm{C}_{12}^4\mathrm{C}_8^4}$}
{$P=\dfrac{\mathrm{C}_9^3\mathrm{C}_6^3}{\mathrm{C}_{12}^4\mathrm{C}_8^4}$}
\loigiai{
+ Số phần tử không gian mẫu: $n\left(\Omega\right)=\mathrm{C}_{12}^4\cdot \mathrm{C}_8^4\cdot \mathrm{C}_4^4.3!$.\\
(bốc 4 đội từ 12 đội vào bảng A - bốc 4 đội từ 8 đội còn lại vào bảng B - bốc 4 đội từ 4 đội còn lại vào bảng C - hoán vị 3 bảng)\\
Gọi $A$: \lq\lq $3$ đội Việt Nam nằm ở $3$ bảng đấu\rq\rq \\
Khi đó: $|\Omega_A|=\mathrm{C}_9^3\cdot \mathrm{C}_6^3\cdot \mathrm{C}_3^3.3!\cdot 3!$.\\
(bốc 3 đội NN từ 9 đội NN vào bảng A - bốc 3 đội NN từ 6 đội NN còn lại vào bảng B - bốc 3 đội NN từ 3 đội NN còn lại vào bảng C - hoán vị 3 bảng - bốc 1 đội VN vào mỗi vị trí còn lại của 3 bảng)\\
Xác suất của biến cố $A$ là $P(A)=\dfrac{|\Omega_A|}{n\left(\Omega\right)}=\dfrac{\mathrm{C}_9^3\cdot \mathrm{C}_6^3\cdot \mathrm{C}_3^3.3!\cdot 3!}{\mathrm{C}_{12}^4\cdot \mathrm{C}_8^4\cdot \mathrm{C}_4^4.3!}=\dfrac{6\cdot \mathrm{C}_9^3\cdot \mathrm{C}_6^3}{\mathrm{C}_{12}^4\cdot \mathrm{C}_8^4}$}
\end{ex}
%Câu 126
\begin{ex}
Gọi $S$ là tập hợp tất cả số tự nhiên có $4$ chữ số phân biệt. Chọn ngẫu nhiên một số từ $S$. Xác suất chọn được số lớn hơn $2500$ là
\choice
{$P=\dfrac{13}{68}$}
{$P=\dfrac{55}{68}$}
{\True $P=\dfrac{68}{81}$}
{$P=\dfrac{13}{81}$}
\loigiai{
Số có $4$ chữ số có dạng: $\overline{abcd}$.\\
Số phần tử của không gian mẫu: $n(S)=9.9\cdot 8.7=4536$.\\
Gọi $A$: \lq\lq  tập hợp các số tự nhiên có $4$ chữ số phân biệt và lớn hơn $2500$.\rq\rq \\
TH1. $a>2$\\
Chọn $a$: có $7$ cách chọn.\\
Chọn $b$: có $9$ cách chọn.\\
Chọn $c$: có $8$ cách chọn.\\
Chọn $d$: có $7$ cách chọn.\\
Vậy trường hợp này có: $7.9\cdot 8.7=3528$ (số).\\
TH2. $a=2,b>5$\\
Chọn $a$: có $1$ cách chọn.\\
Chọn $b$: có $4$ cách chọn.\\
Chọn $c$: có $8$ cách chọn.\\
Chọn $d$: có $7$ cách chọn.\\
Vậy trường hợp này có: $1.4\cdot 8.7=224$ (số).\\
TH3. $a=2,b=5,c>0$\\
Chọn $a$: có $1$ cách chọn.\\
Chọn $b$: có $1$ cách chọn.\\
Chọn $c$: có $7$ cách chọn.\\
Chọn $d$: có $7$ cách chọn.\\
Vậy trường hợp này có: $1.1\cdot 7.7=49$ (số).\\
TH4. $a=2,b=5,c=0,d>0$\\
Chọn $a$: có $1$ cách chọn.\\
Chọn $b$: có $1$ cách chọn.\\
Chọn $c$: có $1$ cách chọn.\\
Chọn $d$: có $7$ cách chọn.\\
Vậy trường hợp này có: $1.1\cdot 1.7=7$ (số).\\
Như vậy: $|\Omega_A|=3528+224+49+7=3808$.\\
Suy ra: $P(A)=\dfrac{|\Omega_A|}{n(S)}=\dfrac{3508}{4536}=\dfrac{68}{81}$}
\end{ex}
%Câu 127
\begin{ex}
Trong giải bóng đá nữ ở trường THPT có $12$ đội tham gia, trong đó có hai đội của hai lớp 12A2 và 11A6. Ban tổ chức tiến hành bốc thăm ngẫu nhiên để chia thành hai bảng đấu A, B mỗi bảng $6$ đội. Xác suất để $2$ đội của hai lớp 12A2 và 11A6 ở cùng một bảng là
\choice
{$P=\dfrac{4}{11}$}
{$P=\dfrac{3}{22}$}
{$P=\dfrac{5}{11}$}
{\True $P=\dfrac{5}{22}$}
\loigiai{
Số phần tử của không gian mẫu là $n\left(\Omega\right)=\mathrm{C}_{12}^6\cdot \mathrm{C}_6^6.2!=1848$.\\
(bốc 6 đội từ 12 đội vào bảng A - bốc 6 đội từ 6 đội còn lại vào bảng B - hoán vị 2 bảng)\\
Gọi $A$: \lq\lq $2$ đội của hai lớp $12\text{A}2$ và $11\text{A}6$ ở cùng một bảng\rq\rq .\\
$|\Omega_A|=\mathrm{C}_{10}^4.2!=420$.\\
(bốc 4 đội từ 10 đội (không tính hai lớp $12\text{A}2$ và$11\text{A}6$) vào bảng đã xếp hai đội của hai lớp $12\text{A}2$ và$11\text{A}6$- 6 đội còn lại vào một bảng - hoán vị hai bảng).\\
$\Rightarrow P(A)=\dfrac{|\Omega_A|}{n\left(\Omega\right)}=\dfrac{420}{1848}=\dfrac{5}{22}$}
\end{ex}
%Câu 128
\begin{ex}
Cho đa giác đều $12$ đỉnh. Chọn ngẫu nhiên $3$ đỉnh trong $12$ đỉnh của đa giác. Xác suất để $3$ đỉnh được chọn tạo thành tam giác đều là
\choice
{\True $P=\dfrac{1}{55}$}
{$P=\dfrac{1}{220}$}
{$P=\dfrac{1}{4}$}
{$P=\dfrac{1}{14}$}
\loigiai{
Số phần tử không gian mẫu: $n\left(\Omega\right)=\mathrm{C}_{12}^3=220$.\\
(chọn 3 đỉnh bất kì từ 12 đỉnh của đa giác ta được một tam giác)\\
Gọi $A$: \lq\lq $3$ đỉnh được chọn tạo thành tam giác đều \rq\rq .\\
Chia $12$ đỉnh thành $3$ phần. Mỗi phần gồm $4$ đỉnh liên tiếp nhau. Mỗi đỉnh của tam giác đều ứng với một phần ở trên.Chỉ cần chọn 1 đỉnh thì 2 đỉnh còn lại xác định là duy nhất.\\
Ta có: $|\Omega_A|=\mathrm{C}_4^1=4$.\\
Khi đó: $P(A)=\dfrac{|\Omega_A|}{n\left(\Omega\right)}=\dfrac{4}{220}=\dfrac{1}{55}$}
\end{ex}
%Câu 129
\begin{ex}
Gọi $S$ là tập hợp tất cả các số tự nhiên có $6$ chữ số phân biệt được lấy từ các số $1$,$2$,$3$,$4$,$5$,$6$,$7$,$8$,$9$. Chọn ngẫu nhiên một số từ $S$. Xác suất chọn được số chỉ chứa $3$ số lẻ là
\choice
{$P=\dfrac{16}{42}$}
{$P=\dfrac{16}{21}$}
{\True $P=\dfrac{10}{21}$}
{$P=\dfrac{23}{42}$}
\loigiai{
Số phần tử không gian mẫu: $n\left(\Omega\right)=A_9^6=60480$.\\
(mỗi số tự nhiên $\overline{abcdef}$ thuộc $S$ là một chỉnh hợp chập 6 của 9- số phần tử của $S$ là số chỉnh hợp chập 6 của 9).\\
Gọi $A$: \lq\lq số được chọn chỉ chứa $3$ số lẻ\rq\rq . Ta có: $|\Omega_A|=\mathrm{C}_5^3\cdot A_6^3\cdot A_4^3=28800$.\\
(bốc ra 3 số lẻ từ 5 số lẻ đã cho- chọn ra 3 vị trí từ 6 vị trí của số $\overline{abcdef}$ xếp thứ tự 3 số vừa chọn - bốc ra 3 số chẵn từ 4 số chẵn đã cho xếp thứ tự vào 3 vị trí còn lại của số $\overline{abcdef}$)\\
Khi đó: $P(A)=\dfrac{|\Omega_A|}{n\left(\Omega\right)}=\dfrac{28800}{60480}=\dfrac{10}{21}$}
\end{ex}
%Câu 130
\begin{ex}
Trên giá sách có 4 quyến sách toán, 3 quyến sách lý, 2 quyến sách hóa. Lấy ngẫu nhiên 3 quyển sách. Tính xác suất để 3 quyển lấy thuộc 3 môn khác nhau.
\choice
{\True $\dfrac{2}{7}$}
{$\dfrac{1}{21}$}
{$\dfrac{37}{42}$}
{$\dfrac{5}{42}$}
\loigiai{
$|\Omega|=\mathrm{C}_9^3=84$. Gọi $A$: "3 quyển lấy được thuộc 3 môn khác nhau"\\
Ta có $|\Omega_A|=4.3\cdot 2=24$. Vậy $P(A)=\dfrac{24}{84}=\dfrac{2}{7}$}
\end{ex}
%Câu 131
\begin{ex}
Trên giá sách có 4 quyến sách toán, 3 quyến sách lý, 2 quyến sách hóa. Lấy ngẫu nhiên 3 quyển sách. Tính xác suất để 3 quyển lấy ra đều là môn toán.
\choice
{$\dfrac{2}{7}$}
{\True $\dfrac{1}{21}$}
{$\dfrac{37}{42}$}
{$\dfrac{5}{42}$}
\loigiai{
$|\Omega|=\mathrm{C}_9^3=84$. Gọi $A$:\rq\rq 3 quyển lấy ra đều là môn toán\rq\rq \\
Ta có $|\Omega_A|=\mathrm{C}_4^3=4$. Vậy $P(A)=\dfrac{4}{84}=\dfrac{1}{21}$}
\end{ex}
%Câu 132
\begin{ex}
Trên giá sách có 4 quyến sách toán, 3 quyến sách lý, 2 quyến sách hóa. Lấy ngẫu nhiên 3 quyển sách. Tính xác suất để 3 quyển lấy ra có ít nhất 1 quyển là môn toán.
\choice
{$\dfrac{2}{7}$}
{$\dfrac{1}{21}$}
{\True $\dfrac{37}{42}$}
{$\dfrac{5}{42}$}
\loigiai{
$|\Omega|=\mathrm{C}_9^3=84$. Gọi $A$:\rq\rq 3 quyển lấy ra có ít nhất 1 quyển là môn toán\rq\rq \\
Khi đó $\overline{A}$:\rq\rq 3 quyển lấy ra không có quyển nào môn toán\rq\rq  hay $\overline{A}$:\rq\rq 3 quyển lấy ra là môn lý hoặc hóa\rq\rq .\\
Ta có $3+2=5$ quyển sách lý hoặc hóa. $n(\overline{A})=\mathrm{C}_5^3=10$. Vậy $P(A)=1-P(\overline{A})=1-\dfrac{10}{84}=\dfrac{37}{42}$}
\end{ex}
%Câu 133
\begin{ex}
Một hộp đựng 11 tấm thẻ được đánh số từ 1 đến 11. Chọn ngẫu nhiên 6 tấm thẻ. Gọi $P$ là xác suất để tổng số ghi trên 6 tấm thẻ ấy là một số lẻ. Khi đó $P$ bằng:
\choice
{$\dfrac{100}{231}$}
{$\dfrac{115}{231}$}
{$\dfrac{1}{2}$}
{\True $\dfrac{118}{231}$}
\loigiai{
$|\Omega|=\mathrm{C}_{11}^6=462$. Gọi $A$:\rq\rq tổng số ghi trên 6 tấm thẻ ấy là một số lẻ\rq\rq .\\
Từ 1 đến 11 có 6 số lẻ và 5 số chẵn.Để có tổng là một số lẻ ta có 3 trường hợp.\\
Trường hợp 1:\\
Chọn được 1 thẻ mang số lẻ và 5 thẻ mang số chẵn có: $6\cdot \mathrm{C}_5^5=6$ cách.\\
Trường hợp 2:\\ 
Chọn được 3 thẻ mang số lẻ và 3 thẻ mang số chẵn có: $\mathrm{C}_6^3\cdot \mathrm{C}_5^3=200$ cách.\\
Trường hợp 3:\\
Chọn được 5 thẻ mang số lẻ và 1 thẻ mang số chẵn có: $\mathrm{C}_6^5.5=30$ cách.\\
Do đó $|\Omega_A|=6+200+30=236$. Vậy $P(A)=\dfrac{236}{462}=\dfrac{118}{231}$}
\end{ex}
%Câu 134
\begin{ex}
Chọn ngẫu nhiên 6 số nguyên dương trong tập $\{1;2;\cdots;10 \}$ và sắp xếp chúng theo thứ tự tăng dần. Gọi $P$ là xác suất để số 3 được chọn và xếp ở vị trí thứ 2. Khi đó $P$ bằng:
\choice
{$\dfrac{1}{60}$}
{$\dfrac{1}{6}$}
{\True $\dfrac{1}{3}$}
{$\dfrac{1}{2}$}
\loigiai{
$|\Omega|=\mathrm{C}_{10}^6=210$. Gọi $A$:\rq\rq số 3 được chọn và xếp ở vị trí thứ 2\rq\rq .\\
Trong tập đã cho có 2 số nhỏ hơn số 3, có 7 số lớn hơn số 3.\\
+ Chọn 1 số nhỏ hơn số 3 ở vị trí đầu có: 2 cách.\\
+ Chọn số 3 ở vị trí thứ hai có: 1 cách.\\
+ Chọn 4 số lớn hơn 3 và sắp xếp theo thứ tự tăng dần có: $\mathrm{C}_7^4=35$ cách.\\
Do đó $|\Omega_A|=2.1\cdot 35=70$. Vậy $P(A)=\dfrac{70}{210}=\dfrac{1}{3}$}
\end{ex}
%Câu 135
\begin{ex}
Có ba chiếc hộp $A,B,C$ mỗi chiếc hộp chứa ba chiếc thẻ được đánh số 1, 2, 3. Từ mỗi hộp rút ngẫu nhiên một chiếc thẻ. Gọi $P$ là xác suất để tổng số ghi trên ba tấm thẻ là 6. Khi đó $P$ bằng:
\choice
{$\dfrac{1}{27}$}
{$\dfrac{8}{27}$}
{\True $\dfrac{7}{27}$}
{$\dfrac{6}{27}$}
\loigiai{
$|\Omega|=3.3\cdot 3=27$. Gọi $A$:\rq\rq tổng số ghi trên ba tấm thẻ là 6\rq\rq .\\
Để tổng số ghi trên ba tấm thẻ là 6 thì có các tổng sau:\\
$1+2+3=6$, khi đó hoán vị 3 phần tử 1, 2, 3 ta được $3!=6$ cách.\\
$2+2+2=6$, khi đó ta có 1 cách.\\
Do đó $|\Omega_A|=6+1=7$. Vậy $P(A)=\dfrac{7}{27}$}
\end{ex}
%Câu 136
\begin{ex}
Có 5 người đến nghe một buổi hòa nhạc. Số cách xếp 5 người này vào một hàng có 5 ghế là:
\choice
{\True $120$}
{$100$}
{$130$}
{$125$}
\loigiai{
Số cách sắp xếp là số hoán vị của tập có 5 phần tử: $P_5=5!=120$}
\end{ex}
%Câu 137
\begin{ex}
Xác suất bắn trúng mục tiêu của một vận động viên khi bắn một viên đạn là $0{,}6$. Người đó bắn hai viên đạn một cách độc lập. Xác suất để một viên trúng mục tiêu và một viên trượt mục tiêu là:
\choice
{$0{,}4$}
{$0{,}6$}
{\True $0{,}48$}
{$0{,}24$}
\loigiai{
Có thể lần 1 bắn trúng hoặc lần 2 bắn trúng.Chọn lần để bắn trúng có 2 cách.\\
Xác suất để 1 viên trúng mục tiêu là $0{,}6$. Xác suất để 1 viên trượt mục tiêu là $1-0{,}6=0{,}4$.\\
Theo quy tắc nhân xác suất: $P(A)=2.0{,}6.0{,}4=0{,}48$}
\end{ex}
%Câu 138
\begin{ex}
Hai xạ thủ độc lập với nhau cùng bắn vào một tấm bia. Mỗi người bắn một viên. Xác suất bắn trúng của xạ thủ thứ nhất là $0{,}7$; của xạ thủ thứ hai là $0{,}8$. Gọi $X$ là số viên đạn bắn trúng bia. Tính kì vọng của$X$:
\choice
{$1{,}75$}
{\True $1{,}5$}
{$1{,}54$}
{$1{,}6$}
\loigiai{
Xác suất để 2 người không bắn trúng bia là: $P=0{,}3.0{,}2=0{,}06$\\
Xác suất để 2 người cùng bắn trúng bia là: $P=0{,}7.0{,}8=0{,}56$\\
Xác suất để đúng 1 người cùng bắn trúng bia là: $P=1-0{,}06-0{,}56=0{,}38$\\
Ta có bảng phân bố xác suất của biến ngẫu nhiên rời rạc $X$.
$X$	0	1	2
$P$	$0{,}06$	$0{,}38$	$0{,}56$
Vậy kỳ vọng xủa $X$ là: $E(X)=0.0{,}06+1.0{,}38+2.0{,}56=1{,}5$}
\end{ex}
%Câu 139
\begin{ex}
Với số nguyên $k$ và $n$ sao cho $1\le k<n$. Khi đó
\choice
{\True $\dfrac{n-2k-1}{k+1}\cdot \mathrm{C}_n^k$ là một số nguyên với mọi $k$ và $n$}
{$\dfrac{n-2k-1}{k+1}\cdot \mathrm{C}_n^k$ là một số nguyên với mọi giá trị chẵn của $k$ và $n$}
{$\dfrac{n-2k-1}{k+1}\cdot \mathrm{C}_n^k$ là một số nguyên với mọi giá trị lẻ của $k$ và $n$}
{$\dfrac{n-2k-1}{k+1}\cdot \mathrm{C}_n^k$ là một số nguyên nếu $\hoac{& k=1 \\& n=1 }$}
\loigiai{
Ta có
\begin{align*}
 & \dfrac{n-2k-1}{k+1}\cdot \mathrm{C}_n^k=\dfrac{(n-k)-(k+1)}{k+1}\cdot \mathrm{C}_n^k=\dfrac{n-k}{k+1}\cdot \mathrm{C}_n^k-\mathrm{C}_n^k=\dfrac{n-k}{k+1}\cdot \dfrac{n!}{k!\cdot (n-k)!}-\mathrm{C}_n^k \\& =\dfrac{n!}{(k+1)!\cdot \left(n-(k+1)\right)!}-\mathrm{C}_n^k=\mathrm{C}_n^{k+1}-\mathrm{C}_n^k\cdot 
\end{align*}
Do $1\le k<n\Rightarrow k+1\le n\Rightarrow \mathrm{C}_n^{k+1}$ luôn tồn tại với mọi số nguyên $k$ và $n$ sao cho $1\le k<n$.\\
Mặt khác $\mathrm{C}_n^{k+1}$ và $\mathrm{C}_n^k$ là các số nguyên dương nên $\mathrm{C}_n^{k+1}-\mathrm{C}_n^k$ cũng là một số nguyên}
\end{ex}
%Câu 140
\begin{ex}
Một nhóm gồm $8$ nam và $7$ nữ. Chọn ngẫu nhiên $5$ bạn. Xác suất để trong $5$ bạn được chọn có cả nam lẫn nữ mà nam nhiều hơn nữ là:
\choice
{$\dfrac{60}{143}$}
{\True $\dfrac{238}{429}$}
{$\dfrac{210}{429}$}
{$\dfrac{82}{143}$}
\loigiai{
Gọi A là biến cố: \lq\lq 5 bạn được chọn có cả nam lẫn nữ mà nam nhiều hơn nữ \lq\lq \\
-Không gian mẫu: $\left| \Omega \right|=\mathrm{C}_{15}^5$.\\
-Số cách chọn 5 bạn trong đó có 4 nam, 1 nữ là: $\mathrm{C}_8^4 \cdot \mathrm{C}_7^1$.\\
-Số cách chọn 5 bạn trong đó có 3 nam, 2 nữ là: $\mathrm{C}_8^3 \cdot \mathrm{C}_7^2$.\\
=>$|\Omega_A|=\mathrm{C}_8^4\cdot \mathrm{C}_7^1+\mathrm{C}_8^3\cdot \mathrm{C}_7^2=1666$\\
=>$P(A)=\dfrac{|\Omega_A|}{\left| \Omega \right|}=\dfrac{1666}{\mathrm{C}_{15}^5}=\dfrac{238}{429}$
}
\end{ex}
%Câu 141
\begin{ex}
Có 2 hộp bút chì màu. Hộp thứ nhất có có 5 bút chì màu đỏ và 7 bút chì màu xanh. Hộp thứ hai có có 8 bút chì màu đỏ và 4 bút chì màu xanh. Chọn ngẫu nhiên mỗi hộp một cây bút chì. Xác suất để có 1 cây bút chì màu đỏ và 1 cây bút chì màu xanh là:
\choice
{\True $\dfrac{19}{36}$}
{$\dfrac{17}{36}$}
{$\dfrac{5}{12}$}
{$\dfrac{7}{12}$}
\loigiai{
Gọi A là biến cố: \lq\lq có 1 cây bút chì màu đỏ và 1 cây bút chì màu xanh\lq\lq \\
-Không gian mẫu: $\left| \Omega \right|=\mathrm{C}_{12}^1\cdot \mathrm{C}_{12}^1=144$.\\
-Số cách chọn được 1 bút đỏ ở hộp 1, 1 bút xanh ở hộp 2 là: $\mathrm{C}_5^1\cdot \mathrm{C}_4^1$.\\
-Số cách chọn được 1 bút đỏ ở hộp 2, 1 bút xanh ở hộp 1 là: $\mathrm{C}_8^1\cdot \mathrm{C}_7^1$.\\
=>$|\Omega_A|=\mathrm{C}_5^1\cdot \mathrm{C}_4^1+\mathrm{C}_8^1\cdot \mathrm{C}_7^1=76$.\\
=>$P(A)=\dfrac{|\Omega_A|}{\left| \Omega \right|}=\dfrac{76}{144}=\dfrac{19}{36}$}
\end{ex}
%Câu 142
\begin{ex}
Một lô hàng gồm 1000 sản phẩm, trong đó có 50 phế phẩm. Lấy ngẫu nhiên từ lô hàng đó 1 sản phẩm. Xác suất để lấy được sản phẩm tốt là:
\choice
{0{,}94}
{0{,}96}
{\True 0{,}95}
{0{,}97}
\loigiai{
Gọi A là biến cố: \lq\lq lấy được 1 sản phẩm tốt.\lq\lq \\
-Không gian mẫu: $\left| \Omega \right|=\mathrm{C}_{100}^1=100$..\\
-$|\Omega_A|=\mathrm{C}_{950}^1=950$.\\
=>$P(A)=\dfrac{|\Omega_A|}{\left| \Omega \right|}=\dfrac{950}{100}=0{,}95$}
\end{ex}
%Câu 143
\begin{ex}
Ba người cùng bắn vào 1 bia. Xác suất để người thứ nhất, thứ hai,thứ ba bắn trúng đích lần lượt là 0{,}8 ; 0{,}6; 0{,}5. Xác suất để có đúng 2 người bắn trúng đích bằng:
\choice
{0.24}
{0.96}
{\True 0.46}
{0.92}
\loigiai{
Gọi X là biến cố: \lq\lq có đúng 2 người bắn trúng đích \lq\lq \\
Gọi A là biến cố: \lq\lq người thứ nhất bắn trúng đích \lq\lq =>$P(A)=0{,}8;P\left(\overline{A}\right)=0{,}2$.\\
Gọi B là biến cố: \lq\lq người thứ hai bắn trúng đích \lq\lq =>$P(B)=0{,}6;P\left(\overline{B}\right)=0{,}4$.\\
Gọi C là biến cố: \lq\lq người thứ ba bắn trúng đích \lq\lq =>$P(C)=0{,}5;P\left(\overline{C}\right)=0{,}5$.\\
Ta thấy biến cố A, B, C là 3 biến cố độc lập nhau, theo công thức nhân xác suất ta có:\\
$P(X)=P\left(A.B\cdot \overline{C}\right)+P\left(A\cdot \overline{B}\cdot C\right)+P\left(\overline{A}\cdot B\cdot C\right)=0{,}8.0{,}6.0{,}5+0{,}8.0{,}4.0{,}5+0{,}2.0{,}6.0{,}5=0{,}46$}
\end{ex}
%Câu 144
\begin{ex}
Cho tập $A=\left\{ 1;2;3;4;5;6 \right\}$. Từ tập $A$ có thể lập được bao nhiêu số tự nhiên có 3 chữ số khác nhau. Tính xác suất biến cố sao cho tổng 3 chữ số bằng 9
\choice
{$\dfrac{1}{20}$}
{$\dfrac{3}{20}$}
{\True $\dfrac{9}{20}$}
{$\dfrac{7}{20}$}
\loigiai{
Gọi A là biến cố: \lq\lq  số tự nhiên có tổng 3 chữ số bằng 9.\lq\lq \\
-Số số tự nhiên có 3 chữ số khác nhau có thể lập được là: $A_6^3=120$.\\
=>Không gian mẫu: $\left| \Omega \right|=120$.\\
-Ta có $1+2+6=9;1+3+5=9;2+3+4=9$.\\
=>Số số tự nhiên có 3 chữ số khác nhau có tổng bằng 9 là:$3!+3!+3!=18$.\\
=>$|\Omega_A|=18$.\\
 =>$P(A)=\dfrac{|\Omega_A|}{\left| \Omega \right|}=\dfrac{18}{120}=\dfrac{3}{20}$}
\end{ex}
%Câu 145
\begin{ex}
Có bốn tấm bìa được đánh số từ 1 đến 4. Rút ngẫu nhiên ba tấm. Xác suất của biến cố \lq\lq Tổng các số trên ba tấm bìa bằng 8\rq\rq  là
\choice
{$1$}
{\True $\dfrac{1}{4}$}
{$\dfrac{1}{2}$}
{$\dfrac{3}{4}$}
\loigiai{
Gọi A là biến cố: \lq\lq Tổng số trên tấm bìa bằng 8.\rq\rq \\
-Không gian mẫu: $\mathrm{C}_4^3=4$.\\
-Ta có $1+3+4=8$.\\
=>$|\Omega_A|=1$.\\
=>$P(A)=\dfrac{|\Omega_A|}{\left| \Omega \right|}=\dfrac{1}{4}$}
\end{ex}
%Câu 146
\begin{ex}
Một người chọn ngẫu nhiên hai chiếc giày từ bốn đôi giày cỡ khác nhau. Xác suất để hai chiếc chọn được tạo thành một đôi là:
\choice
{$\dfrac{4}{7}$}
{$\dfrac{3}{14}$}
{\True $\dfrac{2}{7}$}
{$\dfrac{5}{28}$}
\loigiai{
Gọi A là biến cố: \lq\lq hai chiếc chọn được tạo thành một đôi.\rq\rq \\
-Không gian mẫu: $\mathrm{C}_8^2=28$.\\
-Ta có chiếc giày thứ nhất có 8 cách chọn, chiếc giày thứ 2 có 1 cách chọn để cùng đôi với chiếc giày thứ nhất.\\
=>$|\Omega_A|=8.1=8$.\\
=>$P(A)=\dfrac{|\Omega_A|}{\left| \Omega \right|}=\dfrac{8}{28}=\dfrac{2}{7}$}
\end{ex}
%Câu 147
\begin{ex}
Một tiểu đội có 10 người được xếp ngẫu nhiên thành hàng dọc, trong đó có anh A và anh B. Xác suất để A và B đứng liền nhau bằng:
\choice
{$\dfrac{1}{6}$}
{$\dfrac{1}{4}$}
{\True $\dfrac{1}{5}$}
{$\dfrac{1}{3}$}
\loigiai{
Gọi A là biến cố: \lq\lq A và B đứng liền nhau.\rq\rq \\
-Không gian mẫu: $10!$.\\
-$|\Omega_A|=2!\cdot 9!$.\\
=>$P(A)=\dfrac{|\Omega_A|}{\left| \Omega \right|}=\dfrac{2!\cdot 9!}{10!}=\dfrac{1}{5}$}
\end{ex}
%Câu 148
\begin{ex}
Một đề thi có 20 câu hỏi trắc nghiệm khách quan, mỗi câu hỏi có 4 phương án lựa chọn, trong đó chỉ có một phương án đúng. Khi thi, một học sinh đã chọn ngẫu nhiên một phương án trả lời với mỗi câu của đề thi đó. Xác suất để học sinh đó trả lời không đúng cả 20 câu là:
\choice
{$\dfrac{1}{4}$}
{$\dfrac{3}{4}$}
{$\dfrac{1}{20}$}
{\True $\left(\dfrac{3}{4}\right)^20$}
\loigiai{
Gọi A là biến cố: \lq\lq học sinh đó trả lời không đúng cả 20 câu.\rq\rq \\
-Không gian mẫu: $\Omega ={4^{20}}$.\\
-$|\Omega_A|={3^{20}}$.\\
=>$P(A)=\dfrac{|\Omega_A|}{\left| \Omega \right|}=\dfrac{3^{20}}{4^{20}}=\left(\dfrac{3}{4}\right)^20$}
\end{ex}
%Câu 149
\begin{ex}
Hai người độc lập nhau ném bóng vào rổ. Mỗi người ném vào rổ của mình một quả bóng. Biết rằng xác suất ném bóng trúng vào rổ của từng người tương ứng là $\dfrac{1}{5}$ và $\dfrac{2}{7}$. Gọi $A$ là biến cố: \lq\lq Cả hai cùng ném bóng trúng vào rổ\rq\rq . Khi đó, xác suất của biến cố $A$ là bao nhiêu?
\choice
{$p(A)=\dfrac{12}{35}$}
{$p(A)=\dfrac{1}{25}$}
{$p(A)=\dfrac{4}{49}$}
{\True $p(A)=\dfrac{2}{35}$}
\loigiai{
Gọi A là biến cố: \lq\lq Cả hai cùng ném bóng trúng vào rổ. \lq\lq \\
Gọi X là biến cố: \lq\lq người thứ nhất ném trúng rổ.\lq\lq =>$P(X)=\dfrac{1}{5}$.\\
Gọi Y là biến cố: \lq\lq người thứ hai ném trúng rổ.\lq\lq =>$P(Y)=\dfrac{2}{7}$.\\
Ta thấy biến cố X, Y là 2 biến cố độc lập nhau, theo công thức nhân xác suất ta có:\\
$P(A)=P\left(X\cdot Y\right)=P(X)\cdot P(Y)=\dfrac{1}{5}\cdot \dfrac{2}{7}=\dfrac{2}{35}$}
\end{ex}
%Câu 150
\begin{ex}
Chọn ngẫu nhiên một số tự nhiên nhỏ hơn 30. Tính xác suất biến cố A: \lq\lq số được chọn là số nguyên tố\rq\rq .
\choice
{$p(A)=\dfrac{11}{30}$}
{$p(A)=\dfrac{10}{29}$}
{\True $p(A)=\dfrac{1}{3}$}
{$p(A)=\dfrac{1}{2}$}
\loigiai{
Gọi A là biến cố: \lq\lq số được chọn là số nguyên tố.\rq\rq \\
-Không gian mẫu: $\Omega =\mathrm{C}_{30}^1=30$.\\
-Trong dãy số tự nhiên nhỏ hơn 30 có 10 số nguyên tố.\\
=>$|\Omega_A|=\mathrm{C}_{10}^1=10$.\\
=>$P(A)=\dfrac{|\Omega_A|}{\left| \Omega \right|}=\dfrac{10}{30}=\dfrac{1}{3}$}
\end{ex}
%Câu 151
\begin{ex}
Một lô hàng có $100$ sản phẩm, biết rằng có $8$ sản phẩm hỏng. Người kiểm định lấy ra ngẫu nhiên từ đó $5$ sản phẩm. Tính xác suất của biến cố A: "Người đó lấy được đúng 2 sản phẩm hỏng".
\choice
{$P(A)=\dfrac{2}{25}$}
{\True $P(A)=\dfrac{229}{6402}$}
{$P(A)=\dfrac{1}{50}$}
{$P(A)=\dfrac{1}{2688840}$}
\loigiai{
Gọi A là biến cố: \lq\lq Người đó lấy được đúng 2 sản phẩm hỏng.\rq\rq \\
-Không gian mẫu: $\Omega =\mathrm{C}_{100}^5$.\\
-$|\Omega_A|=\mathrm{C}_8^2\cdot \mathrm{C}_{92}^3$.\\
=>$P(A)=\dfrac{|\Omega_A|}{\left| \Omega \right|}=\dfrac{299}{6402}$}
\end{ex}
%Câu 152
\begin{ex}
Hai xạ thủ bắn mỗi người một viên đạn vào bia, biết xác suất bắn trúng vòng 10 của xạ thủ thứ nhất là $0{,}75$ và của xạ thủ thứ hai là $0{,}85$. Tính xác suất để có ít nhất một viên trúng vòng 10.
\choice
{$0{,}9625$}
{$0{,}325$}
{\True $0{,}6375$}
{$0{,}0375$}
\loigiai{
Gọi A là biến cố: \lq\lq có ít nhất một viên trúng vòng 10.\rq\rq \\
-$\overline{A}$ là biến cố: \lq\lq Không viên nào trúng vòng 10.\rq\rq \\
=>$P\left(\overline{A}\right)=\left(1-0{,}75\right)\cdot \left(1-0{,}85\right)=0{,}0375$.\\
=>$P(A)=1-P\left(\overline{A}\right)=1-0{,}0375=0{,}9625$}
\end{ex}
%Câu 153
\begin{ex}
Bài kiểm tra môn toán có 20 câu trắc nghiệm khách quan; mỗi câu có 4 lựa chọn và chỉ có một phương án đúng. Một học sinh không học bài nên làm bài bằng cách lựa chọn ngẫu nhiên một phương án trả lời. Tính xác suất để học sinh đó trả lời sai cả 20 câu ?
\choice
{$\left(0{,}25\right)^20$}
{$1-\left(0{,}75\right)^20$}
{$1-\left(0{,}25\right)^20$}
{\True $(0{,}75)^20$}
\loigiai{
Gọi A là biến cố: \lq\lq Học sinh đó trả lời sai cả 20 câu.\rq\rq \\
-Trong một câu, xác suất học sinh trả lời sai là: $\dfrac{3}{4}=0{,}75$.\\
=>$P(A)=\left(0{,}75\right)^20$}
\end{ex}
%Câu 154
\begin{ex}
Cho $A$ và $\overline{A}$ là hai biến cố đối nhau. Chọn câu đúng.
\choice
{$P(A)=1+P\left(\overline{A}\right)$}
{$P(A)=P\left(\overline{A}\right)$}
{\True $P(A)=1-P\left(\overline{A}\right)$}
{$P(A)+P\left(\overline{A}\right)=0$}
\loigiai{ }
\end{ex}
%Câu 155
\begin{ex}
Chọn ngẫu nhiên hai số tự nhiên có 4 chữ số khác nhau. Tính xác suất chọn được ít nhất một số chẵn. (lấy kết quả ở hàng phần nghìn)
\choice
{$0{,}652$}
{$0{,}256$}
{$0{,}756$}
{\True $0{,}922$}
\loigiai{
Gọi A là biến cố: \lq\lq chọn được ít nhất một số chẵn.\rq\rq \\
-Số số tự nhiên có 4 chữ số là: $9.10.10.10=9000$.\\
=>Không gian mẫu: $\left| \Omega \right|=\mathrm{C}_{9000}^2$.\\
- Số số tự nhiên lẻ có 4 chữ số khác nhau là:$5.9\cdot 8.7=2520$.\\
=>$n\left(\overline{A}\right)=\mathrm{C}_{2520}^2$.\\
=>$P\left(\overline{A}\right)=\dfrac{n\left(\overline{A}\right)}{\left| \Omega \right|}=\dfrac{\mathrm{C}_{2520}^2}{\mathrm{C}_{9000}^2}=0{,}078$.\\
=>$P(A)=1-P\left(\overline{A}\right)=1-0{,}078=0{,}922$}
\end{ex}
%Câu 156
\begin{ex}
Gieo một đồng tiền liên tiếp $3$ lần. Gọi $A$ là biến cố \lq\lq có ít nhất một lần xuất hiện mặt sấp \rq\rq . Xác suất của biến cố $A$ là
\choice
{$P(A)=\dfrac{1}{2}$}
{$P(A)=\dfrac{3}{8}$}
{\True $P(A)=\dfrac{7}{8}$}
{$P(A)=\dfrac{1}{4}$}
\loigiai{
Số phần tử của không gian mẫu là: $\left| \Omega \right|=2^3=8$.\\
Số phần tử của không gian thuận lợi là: $\left| \Omega _A \right|=2^3-1=7$.\\
Xác suất biến cố $A$ là : $P(A)=\dfrac{7}{8}$.}
\end{ex}
%Câu 157
\begin{ex}
Trên giá sách có $4$ quyển sách Toán, $3$ quyển sách Vật lý, $2$ quyển sách Hoá học. Lấy ngẫu nhiên $3$ quyển sách trên kệ sách ấy. Tính xác suất để$3$ quyển được lấy ra đều là sách Toán.
\choice
{$\dfrac{2}{7}$}
{\True $\dfrac{1}{21}$}
{$\dfrac{37}{42}$}
{$\dfrac{5}{42}$}
\loigiai{
Số phần tử của không gian mẫu là: $\left| \Omega \right|=\mathrm{C}_9^3=84$.\\
Số phần tử của không gian thuận lợi là: $\left| \Omega _A \right|=\mathrm{C}_4^3=4$\\
Xác suất biến cố $A$ là : $P(A)=\dfrac{1}{21}$}
\end{ex}
%Câu 158
\begin{ex}
Có$5$ tờ $20.000$ đ và 3 tờ $50.000$ đ. Lấy ngẫu nhiên $2$ tờ trong số đó. Xác suất để lấy được $2$ tờ có tổng giá trị lớn hơn $70.000$ đ là
\choice
{$\dfrac{15}{28}$}
{$\dfrac{3}{8}$}
{$\dfrac{4}{7}$}
{\True $\dfrac{3}{28}$}
\loigiai{
Số phần tử của không gian mẫu là: $\left| \Omega \right|=\mathrm{C}_8^2=28$.\\
Số phần tử của không gian thuận lợi là: $\left| \Omega _A \right|=\mathrm{C}_3^2=3$\\
Xác suất biến cố $A$ là : $P(A)=\dfrac{3}{28}$}
\end{ex}
%Câu 159
\begin{ex}
Có $8$ người trong đó có vợ chồng anh X được xếp ngẫu nhiên theo một hàng ngang. Tính xác suất để vợ chồng anh X ngồi gần nhau ?
\choice
{$\dfrac{1}{64}$}
{$\dfrac{1}{25}$}
{$\dfrac{1}{8}$}
{\True $\dfrac{1}{4}$}
\loigiai{
Số phần tử của không gian mẫu là: $\left| \Omega \right|=8!$.\\
Số phần tử của không gian thuận lợi là: $\left| \Omega _A \right|=2!\cdot 7!$\\
Xác suất biến cố $A$ là : $P(A)=\dfrac{1}{4}$}
\end{ex}
%Câu 160
\begin{ex}
Rút ra ba quân bài từ mười ba quân bài cùng chất rô $\left\{ 2;3;4;\ldots ;J;Q;K;A \right\}$. Tính xác suất để trong ba quân bài đó không có cả$J$ và $Q$?
\choice
{$\dfrac{5}{26}$}
{$\dfrac{11}{26}$}
{\True $\dfrac{25}{26}$}
{$\dfrac{1}{26}$}
\loigiai{
Số phần tử của không gian mẫu là: $\left| \Omega \right|=\mathrm{C}_{13}^3$.\\
Số phần tử của không gian thuận lợi là: $\left| \Omega _A \right|=\mathrm{C}_{11}^3-\mathrm{C}_{11}^2$\\
Xác suất biến cố $A$ là : $P(A)=\dfrac{25}{26}$}
\end{ex}
%Câu 161
\begin{ex}
Một nhóm gồm $8$ nam và $7$ nữ. Chọn ngẫu nhiên $5$ bạn. Xác suất để trong $5$ bạn được chọn có cả nam lẫn nữ mà nam nhiều hơn nữ là:
\choice
{$\dfrac{60}{143}$}
{\True $\dfrac{238}{429}$}
{$\dfrac{210}{429}$}
{$\dfrac{82}{143}$}
\loigiai{
Số phần tử của không gian mẫu là: $\left| \Omega \right|=\mathrm{C}_{15}^5$.\\
Số phần tử của không gian thuận lợi là: $\left| \Omega _A \right|=\mathrm{C}_8^4\mathrm{C}_7^1+\mathrm{C}_8^3\mathrm{C}_7^2$\\
Xác suất biến cố $A$ là : $P(A)=\dfrac{238}{429}$}
\end{ex}
%Câu 162
\begin{ex}
Cho hai đường thẳng song song $d_1,d_2$. Trên $d_1$ có $6$ điểm phân biệt được tô màu đỏ, trên $d_2$ có $4$ điểm phân biệt được tô màu xanh. Xét tất cả các tam giác được tạo thành khi nối các điểm đó với nhau. Chọn ngẫu nhiên một tam giác, khi đó xác suất để thu được tam giác có hai đỉnh màu đỏ là:
\choice
{$\dfrac{2}{9}$}
{$\dfrac{3}{8}$}
{$\dfrac{5}{9}$}
{\True $\dfrac{5}{8}$}
\loigiai{
Số phần tử của không gian mẫu là: $\left| \Omega \right|=\mathrm{C}_6^2\cdot \mathrm{C}_4^1+\mathrm{C}_6^1\cdot \mathrm{C}_4^2=96$.\\
Số phần tử của không gian thuận lợi là: $\left| \Omega _A \right|=\mathrm{C}_6^2\cdot \mathrm{C}_4^1=60$.\\
Xác suất biến cố $A$ là : $P(A)=\dfrac{5}{8}$}
\end{ex}
%Câu 163
\begin{ex}
Có hai hộp bút chì màu. Hộp thứ nhất có có$5$ bút chì màu đỏ và $7$ bút chì màu xanh. Hộp thứ hai có có $8$ bút chì màu đỏ và $4$ bút chì màu xanh. Chọn ngẫu nhiên mỗi hộp một cây bút chì. Xác suất để có $1$ cây bút chì màu đỏ và $1$ cây bút chì màu xanh là:
\choice
{\True $\dfrac{19}{36}$}
{$\dfrac{17}{36}$}
{$\dfrac{5}{12}$}
{$\dfrac{7}{12}$}
\loigiai{
Số phần tử của không gian mẫu là: $\left| \Omega \right|=\mathrm{C}_{12}^1\cdot \mathrm{C}_{12}^1=144$.\\
Số phần tử của không gian thuận lợi là: $\left| \Omega _A \right|=\mathrm{C}_5^1\cdot \mathrm{C}_4^1+\mathrm{C}_7^1\cdot \mathrm{C}_8^1=76$.\\
Xác suất biến cố $A$ là : $P(A)=\dfrac{19}{36}$}
\end{ex}
%Câu 164
\begin{ex}
Một lô hàng gồm$1000$ sản phẩm, trong đó có $50$ phế phẩm. Lấy ngẫu nhiên từ lô hàng đó $1$ sản phẩm. Xác suất để lấy được sản phẩm tốt là:
\choice
{$0{,}94$}
{$0{,}96$}
{\True $0{,}95$}
{$0{,}97$}
\loigiai{
Số phần tử của không gian mẫu là: $\left| \Omega \right|=1000$.\\
Sản phẩm tốt: $1000-50=950$. Số phần tử của không gian thuận lợi là: $\left| \Omega _A \right|=950$.\\
Xác suất biến cố $A$ là : $P(A)=0{,}95$}
\end{ex}
%Câu 165
\begin{ex}
Ba người cùng bắn vào $1$ bia Xác suất để người thứ nhất, thứ hai,thứ ba bắn trúng đích lần lượt là $0{,}8$; $0{,}6$;$0{,}5$. Xác suất để có đúng $2$ người bắn trúng đích bằng:
\choice
{$0{,}24$}
{$0{,}96$}
{\True $0{,}46$}
{$0{,}92$}
\loigiai{
Xác suất để người thứ nhất, thứ hai, thứ ba bán trúng đích lần lượt là: $P\left(A_1\right)=0{,}8$; $P\left(A_2\right)=0{,}6$ ; $P\left(A_1\right)=0{,}5$\\
Xác suất để có đúng hai người bán trúng đích bằng: $P\left(A_1\right)\cdot P\left(A_2\right)\cdot \overline{P\left(A_3\right)}+P\left(A_1\right)\cdot \overline{P\left(A_2\right)}\cdot P\left(A_3\right)+\overline{P\left(A_1\right)}\cdot P\left(A_2\right)\cdot P\left(A_3\right)=0{,}46$}
\end{ex}
%Câu 166
\begin{ex}
Cho tập $A=\left\{ 1;2;3;4;5;6 \right\}$. Từ tập $A$ có thể lập được bao nhiêu số tự nhiên có $3$ chữ số khác nhau. Tính xác suất biến cố sao cho tổng $3$ chữ số bằng $9$.
\choice
{$\dfrac{1}{20}$}
{\True $\dfrac{3}{20}$}
{$\dfrac{9}{20}$}
{$\dfrac{7}{20}$}
\loigiai{
Số phần tử của không gian mẫu là: $\left| \Omega \right|=A_6^3=120$.\\
Số phần tử của không gian thuận lợi là: $\left| \Omega _A \right|=3P_3=18$(Do 3 cặp số $\left\{ 1;2;6 \right\}$,$\left\{ 1;3;5 \right\}$, $\left\{ 2;3;4 \right\}$)\\
Xác suất biến cố $A$ là : $P(A)=\dfrac{3}{20}$}
\end{ex}
%Câu 167
\begin{ex}
Có $5$ nam, $5$ nữ xếp thành một hàng dọc. Tính xác suất để nam, nữ đứng xen kẻ nhau
\choice
{$\dfrac{1}{125}$}
{\True $\dfrac{1}{126}$}
{$\dfrac{1}{36}$}
{$\dfrac{13}{36}$}
\loigiai{
Số phần tử của không gian mẫu là: $\left| \Omega \right|=10!=3628800$.\\
Số phần tử của không gian thuận lợi là: $\left| \Omega _A \right|=2.5!\cdot 5!=28800$\\
Xác suất biến cố $A$ là : $P(A)=\dfrac{1}{126}$}
\end{ex}
%Câu 168
\begin{ex}
Cho $X$ là tập hợp chứa $6$ số tự nhiên lẻ và $4$ số tự nhiên chẵn. Chọn ngẫu nhiên từ $X$ ra ba số tự nhiên. Xác suất để chọn được ba số có tích là một số chẵn là
\choice
{$P=\dfrac{\mathrm{C}_4^3}{\mathrm{C}_{10}^3}$}
{$P=1-\dfrac{\mathrm{C}_4^3}{\mathrm{C}_{10}^3}$}
{$P=\dfrac{\mathrm{C}_6^3}{\mathrm{C}_{10}^3}$}
{\True $P=1-\dfrac{\mathrm{C}_6^3}{\mathrm{C}_{10}^3}$}
\loigiai{
Số phần tử của không gian mẫu là: $\left| \Omega \right|=\mathrm{C}_{10}^3$.\\
Số phần tử của không gian chọn được ba số có tích là một số lẻ: $\mathrm{C}_6^3$ .\\
Xác suất biến cố chọn được ba số có tích là một số chẵn là : $P=1-\dfrac{\mathrm{C}_6^3}{\mathrm{C}_{10}^3}$}
\end{ex}
%Câu 169
\begin{ex}
Bạn Xuân là một trong 15 người. Chọn 3 người trong đó để lập một ban đại diện. Xác suất đúng đến mười phần nghìn để Xuân là một trong ba người được chọn là.
\choice
{\True 0{,}2000}
{0{,}00667}
{0{,}0022}
{0{,}0004}
\loigiai{
Số phần tử của không gian mẫu là: $\left| \Omega \right|=\mathrm{C}_{15}^3$.\\
Gọi A là biến cố để được để Xuân là một trong ba người được chọn.\\
Số phần tử của không gian thuận lợi là: $\left| \Omega _A \right|=1\cdot \mathrm{C}_{14}^2$.\\
Xác suất biến cố $A$ là : $P(A)=0{,}2000$}
\end{ex}
%Câu 170
\begin{ex}
Một ban đại diện gồm 5 người được thành lập từ 10 người có tên sau đây: Liên, Mai, Mộc, Thu, Miên, An, Hà, Thanh, Mơ, Kim. Xác suất để đúng 2 người trong ban đại diện có tên bắt đầu bằng chữ M là.
\choice
{$\dfrac{1}{42}$}
{$\dfrac{1}{4}$}
{\True $\dfrac{10}{21}$}
{$\dfrac{25}{63}$}
\loigiai{
Số phần tử của không gian mẫu là: $\left| \Omega \right|=\mathrm{C}_{10}^5$.\\
Gọi A là biến cố để để đúng 2 người trong ban đại diện có tên bắt đầu bằng chữ M.\\
Có $4$ người có tên bắt đầu bằng chữ M. Chọn $2$ người trong $4$ người đó có $\mathrm{C}_4^2$ cách.\\
Số phần tử của không gian thuận lợi là: $\left| \Omega _A \right|=\mathrm{C}_4^2\cdot \mathrm{C}_6^3$.\\
Xác suất biến cố $A$ là : $P(A)=\dfrac{10}{21}$}
\end{ex}
%Câu 171
\begin{ex}
Một ban đại diện gồm 5 người được thành lập từ 10 người có tên sau đây: Liên, Mai, Mộu, Thu, Miên, An, Hà, Thanh, Mơ, Kim. Xác suất để ít nhất 3 người trong ban đại diện có tên bắt đầu bằng chữ M là:
\choice
{$\dfrac{5}{252}$}
{$\dfrac{1}{24}$}
{$\dfrac{5}{21}$}
{\True $\dfrac{11}{42}$}
\loigiai{
+ Số phần tử của không gian mẫu là : $n\left(\Omega\right)=\mathrm{C}_{10}^5$\\
+ Gọi biến cố A \lq\lq Có ít nhất 3 người trong ban đại diện có tên bắt đầu từ chữ M\rq\rq \\
 Ta có $|\Omega_A|=\mathrm{C}_4^3\cdot \mathrm{C}_6^2+\mathrm{C}_6^1$\\
Vậy xác suất biến cố A: $P(A)=\dfrac{n\left(\Omega\right)}{|\Omega_A|}=\dfrac{11}{42}$\\
Chưa tô đậm A, B, C D trong đáp án, Lời giải: nhầm}
\end{ex}
%Câu 172
\begin{ex}
Lớp 12 có 9 học sinh giỏi, lớp 11 có 10 học sinh giỏi, lớp 10 có 3 học sinh giỏi. Chọn ngẫu nhiên 2 trong các học sinh đó. Xác suất để 2 học sinh được chọn từ cùng mọt lớp là:
\choice
{$\dfrac{2}{11}$}
{\True $\dfrac{4}{11}$}
{$\dfrac{3}{11}$}
{$\dfrac{5}{11}$}
\loigiai{
+ Số phần tử của không gian mẫu là : $n\left(\Omega\right)=\mathrm{C}_{22}^2$\\
+ Gọi biến cố A \lq\lq hai em được chọn ở cùng một lớp\rq\rq \\
Ta có : $|\Omega_A|=\mathrm{C}_9^2+\mathrm{C}_{10}^2+\mathrm{C}_3^2$\\
Vậy xác suất biến cố A: $P(A)=\dfrac{n\left(\Omega\right)}{|\Omega_A|}=\dfrac{4}{11}$.\\
Chưa tô đậm A, B, C D trong đáp án}
\end{ex}
%Câu 173
\begin{ex}
Bạn Tân ở trong một lớp có 22 học sinh. Chọn ngẫu nhiên 2 em trong lớp để đi xem văn nghệ. Xác suất để Tân được đi xem là:
\choice
{19{,}6\%}
{18{,}2\%}
{9{,}8\%}
{\True 9{,}1\%}
\loigiai{
+ Số phần tử của không gian mẫu là : $n\left(\Omega\right)=\mathrm{C}_{22}^2$\\
+ Gọi biến cố A \lq\lq  hai em trong lớp trong đó có Tân được chọn xem văn nghệ\rq\rq \\
Ta có : $|\Omega_A|=21$\\
Vậy xác suất biến cố A: $P(A)=\dfrac{n\left(\Omega\right)}{|\Omega_A|}=9{,}1\%$\\
}
\end{ex}
%Câu 174
\begin{ex}
Bốn quyển sách được đánh dấu bằng những chữ cái: U, V, X, Y được xếp tuỳ ý trên một kệ sách dài. Xác suất để chúng được xếp theo thứ tự bản chữ cái là:
\choice
{$\dfrac{1}{4}$}
{$\dfrac{1}{6}$}
{\True $\dfrac{1}{24}$}
{$\dfrac{1}{256}$}
\loigiai{
+ Số phần tử của không gian mẫu là : $n\left(\Omega\right)=P_4$\\
+ Gọi biến cố A \lq\lq  xếp thứ tự theo bản chữ cái \rq\rq \\
Ta có :$|\Omega_A|=1$\\
Vậy xác suất biến cố A: $P(A)=\dfrac{n\left(\Omega\right)}{|\Omega_A|}=\dfrac{1}{P_4}=\dfrac{1}{24}$\\
Chưa tô đậm A, B, C D trong đáp án}
\end{ex}
%Câu 175
\begin{ex}
Trong nhóm 60 học sinh có 30 học sinh thích học Toán, 25 học sinh thích học Lý và 10 học sinh thích cả Toán và Lý. Chọn ngẫu nhiên 1 học sinh từ nhóm này. Xác suất để được học sinh này thích học ít nhất là một môn Toán hoặc Lý?
\choice
{$\dfrac{4}{5}$}
{\True $\dfrac{3}{4}$}
{$\dfrac{2}{3}$}
{$\dfrac{1}{2}$}
\loigiai{
Gọi A là tập hợp \lq\lq học sinh thích học Toán\rq\rq \\
Gọi B là tập hợp \lq\lq học sinh thích học Lý\rq\rq \\
Gọi C là tập hợp \rq\rq  học sinh thích học ít nhất một môn \lq\lq \\
Ta có $|\Omega_C|=n\left(A\cup B\right)=|\Omega_A|+|\Omega_B|-n\left(A\cap B\right)=30+25-10=45$\\
Vậy xác suất để được học sinh này thích học ít nhất là một môn Toán hoặc Lý là: $P(C)=\dfrac{|\Omega_C|}{n\left(\Omega\right)}=\dfrac{45}{60}=\dfrac{3}{4}$}
\end{ex}
%Câu 176
\begin{ex}
Trên một kệ sách có 10 sách Toán, 5 sách Lý. Lần lượt lấy 3 cuốn sách mà không để lại trên kệ. Tính xác suất để được hai cuốn sách đầu là Toán và cuốn thứ ba là Lý là:
\choice
{$\dfrac{18}{91}$}
{\True $\dfrac{15}{91}$}
{$\dfrac{7}{45}$}
{$\dfrac{8}{15}$}
\loigiai{
+ Số phần tử của không gian mẫu là : $n\left(\Omega\right)=15.14.13$\\
+ Gọi biến cố A \lq\lq hai cuốn sách đầu là Toán và cuốn thứ ba là Lý\rq\rq \\
Ta có $|\Omega_A|=10.9\cdot 5$\\
Vậy xác suất biến cố A: $P(A)=\dfrac{n\left(\Omega\right)}{|\Omega_A|}=\dfrac{15}{91}$.\\
}
\end{ex}
%Câu 177
\begin{ex}
Cho $A$, $B$ là hai biến cố xung khắc. Biết P(A) = $\dfrac{1}{5}$, $P(A \cup B) = \dfrac{1}{3}$. Tính $P(B)$.
\choice
{$\dfrac{3}{5}$}
{$\dfrac{8}{15}$}
{\True $\dfrac{2}{15}$}
{$\dfrac{1}{15}$}
\loigiai{
A, B là hai biến cố xung khắc\\
$P\left(A\cup B\right)=P(A)+P(B)\Rightarrow P(B)=\dfrac{1}{3}-\dfrac{1}{5}=\dfrac{2}{15}$\\
}
\end{ex}
%Câu 178
\begin{ex}
Cho $A$, $B$ là hai biến cố. Biết $P(A) = \dfrac{1}{2}$, $P(B) =\dfrac{3}{4}$. $P(A \cap B) = \dfrac{1}{4}$. Biến cố $A \cup B$ là biến cố
\choice
{Sơ đẳng}
{\True Chắc chắn}
{Không xảy ra}
{Có xác suất bằng $\dfrac{1}{8}$}
\loigiai{
$A$, $B$ là hai biến cố bất kỳ ta luôn có: $P\left(A\cup B\right)=P(A)+P(B)-P\left(A\cap B\right)=\dfrac{1}{2}+\dfrac{3}{4}-\dfrac{1}{4}=1$\\
Vậy $A\cup B$ là biến cố chắc chắn}
\end{ex}
%Câu 179
\begin{ex}
$A$, $B$ là hai biến cố độc lập. Biết $P(A)=\dfrac{1}{4}$, $P\left(A\cap B\right)=\dfrac{1}{9}$. Tính $P(B)$
\choice
{$\dfrac{7}{36}$}
{$\dfrac{1}{5}$}
{\True $\dfrac{4}{9}$}
{$\dfrac{5}{36}$}
\loigiai{
$A$, $B$ là hai biến cố độc lập nên:$P\left(A\cap B\right)=P(A)\cdot P(B)\Leftrightarrow \dfrac{1}{9}=\dfrac{1}{4}\cdot P(B)\Leftrightarrow P(B)=\dfrac{4}{9}$}
\end{ex}
%Câu 180
\begin{ex}
$A$, $B$ là hai biến cố độc lập. $P(A)=0{,}5$. $P\left(A\cap B\right)=0{,}2$. Xác suất $P\left(A\cup B\right)$ bằng:
\choice
{$0{,}3$}
{$0{,}5$}
{$0{,}6$}
{\True $0{,}7$}
\loigiai{
$A$, $B$ là hai biến cố độc lập nên:$P\left(A\cap B\right)=P(A)\cdot P(B)\Leftrightarrow P(B)=0{,}4$\\
$P\left(A\cup B\right)=P(A)+P(B)-P\left(A\cap B\right)=0{,}7$}
\end{ex}
%Câu 181
\begin{ex}
Cho $P(A)=\dfrac{1}{4}$, $P\left(A\cup B\right)=\dfrac{1}{2}$. Biết $A$, $B$ là hai biến cố xung khắc, thì $P(B)$ bằng:
\choice
{$\dfrac{1}{3}$}
{$\dfrac{1}{8}$}
{\True $\dfrac{1}{4}$}
{$\dfrac{3}{4}$}
\loigiai{
$A$, $B$ là hai biến cố xung khắc: $P\left(A\cup B\right)=P(A)+P(B)\Leftrightarrow P(B)=\dfrac{1}{4}$}
\end{ex}
%Câu 182
\begin{ex}
Cho $P(A)=\dfrac{1}{4}$, $P\left(A\cup B\right)=\dfrac{1}{2}$. Biết $A$, $B$ là hai biến cố độc lập, thì $P(B)$ bằng:
\choice
{\True $\dfrac{1}{3}$}
{$\dfrac{1}{8}$}
{$\dfrac{1}{4}$}
{$\dfrac{3}{4}$}
\loigiai{
Ta có A, B là biến cố độc lập nên ta có $P\left(A\cup B\right)=P(A)+P(B)-P(A\cap B)$\\
Vậy $P(B)=\dfrac{1}{3}$}
\end{ex}
%Câu 183
\begin{ex}
Trong một kì thi có $60\%$ thí sinh đỗ. Hai bạn $A$, $B$ cùng dự kì thi đó. Xác suất để chỉ có một bạn thi đỗ là:
\choice
{$0{,}24$}
{$0{,}36$}
{$0{,}16$}
{\True $0{,}48$}
\loigiai{
Ta có: $P(A)=P(B)=0{,}6\Rightarrow P\left(\overline{A}\right)=P\left(\overline{B}\right)=0{,}4$\\
Xác suất để chỉ có một bạn thi đỗ là:$P=P\left(\overline{A}\right)\cdot P(B)+P(A)\cdot P\left(\overline{B}\right)=0{,}48$}
\end{ex}
%Câu 184
\begin{ex}
Một xưởng sản xuất cón máy, trong đó có một số máy hỏng. Gọi$A_k$ là biến cố : \lq\lq  Máy thứ $k$ bị hỏng\rq\rq . $k=1{,}2,\ldots ,n$. Biếncố$A$: \lq\lq  Cả $n$ đều tốt đều tốt \lq\lq  là
\choice
{$A=A_1A_2\ldots A_n$}
{$A=\bar{A}_1\bar{A}_2\cdots \bar{A}_{n-1} A_n$}
{$A=A_1A_2\ldots A_{n-1}\bar{A}_n$}
{\True $A=\bar{A}_1\bar{A}_2\ldots \bar{A}_n$}
\loigiai{
Ta có: $A_k$ làbiếncố : \lq\lq  Máy thứ$k$ bị hỏng\rq\rq. $k=1{,}2,\ldots,n$.\\
Nên: $\overline{A_k}$ là biến cố : \lq\lq  Máy thứ $k$ tốt \rq\rq $k=1{,}2,\ldots,n$.\\
Biếncố$A$: \lq\lq  Cả$n$ đều tốt đều tốt \lq\lq  là:$A=\bar{A}_1\bar{A}_2 \ldots \bar{A}_n$}
\end{ex}
%Câu 185
\begin{ex}
Cho phép thử có không gian mẫu$\Omega =\left\{ 1{,}2,3{,}4,5{,}6 \right\}$. Các cặp biến cố không đố inhau là:
\choice
{$A=\left\{ 1~ \right\}~$ và $B=\left\{ 2{,}3,4{,}5,6 \right\}$}
{$C=\left\{ 1{,}4,5 \right\}$ và $D=\left\{ 2{,}3,6 \right\}$}
{\True $E=\left\{ 1{,}4,6 \right\}$ và $F=\left\{ 2{,}3 \right\}$}
{$\Omega ~~$ và$\varnothing $}
\loigiai{
Theo định nghĩa hai biến cố đối nhau là hai biến cố giao nhau bằng rỗng và hợp nhau bằng không gian mẫu.\\
Mà $\heva{& E\cap F=\varnothing \\& E\cup F\ne \Omega }$ nên $E,F$ không đối nhau}
\end{ex}
%Câu 186
\begin{ex}
Một người bỏ ngẫu nhiên bốn lá thư vào 4 bì thư đã được ghi địa chỉ. Tính xác suất của biến cố sau \lq\lq  Có ít nhất một lá thư bỏ đúng phong bì của nó\rq\rq .
\choice
{$P(A)=\dfrac{5}{8}$}
{$P(A)=\dfrac{3}{8}$}
{$P(A)=\dfrac{1}{8}$}
{$P(A)=\dfrac{7}{8}$}
\loigiai{
Số cách bỏ 4 lá thư vào 4 bì thư là: $\left| \Omega \right|=4!=24$\\
Kí hiệu 4 lá thư là: $L_1,L_2,L_3,L_4$ và bộ $\left(L_1,L_2,L_3,L_4\right)$ là một hóa vị của các số $1{,}2,3{,}4$ trong đó $L_i=i(i=\overline{1{,}4}$) nếu lá thư $L_i$ bỏ đúng địa chỉ.\\
Ta xét các khả năng sau\\
$\bullet $ có 4 lá thư bỏ đúng địa chỉ: $(1{,}2,3{,}4)$ nên có 1 cách bỏ\\
$\bullet $ có 2 là thư bỏ đúng địa chỉ:\\
 +) số cách bỏ 2 lá thư đúng địa chỉ là: $\mathrm{C}_4^2$\\
 +) khi đó có 1 cách bỏ hai là thư còn lại\\
Nên trường hợp này có: $\mathrm{C}_4^2=6$ cách bỏ.\\
$\bullet $ Có đúng 1 lá thư bỏ đúng địa chỉ:\\
Số cách chọn lá thư bỏ đúng địa chỉ: 4 cách\\
Số cách chọn bỏ ba lá thư còn lại: $2.1=2$ cách\\
Nên trường hợp này có: $4.2=8$ cách bỏ.\\
Do đó: $\left| \Omega _A \right|=1+6+8=15$\\
Vậy $P(A)=\dfrac{\left| \Omega _A \right|}{\left| \Omega \right|}=\dfrac{15}{24}=\dfrac{5}{8}$}
\end{ex}
%Câu 187
\begin{ex}
Một đoàn tàu có 7 toa ở một sân ga. Có 7 hành khách từ sân ga lên tàu, mỗi người độc lập với nhau và chọn một toa một cách ngẫu nhiên. Tìm xác suất của các biến cố A: \lq\lq  Một toa 1 người, một toa 2 người, một toa có 4 người lên và bốn toa không có người nào cả\rq\rq 
\choice
{$P(A)=\dfrac{450}{1807}$}
{$P(A)=\dfrac{40}{16807}$ }
{\True $P(A)=\dfrac{450}{16807}$}
{$P(A)=\dfrac{450}{1607}$}
\loigiai{
Số cách lên toa của 7 người là: $\left| \Omega \right|=7^7$.\\
Ta tìm số khả năng thuận lợi của A như sau\\
$\bullet $ Chọn 3 toa có người lên: $A_7^3$\\
$\bullet $ Với toa có 4 người lên ta có: $\mathrm{C}_7^4$ cách chọn\\
$\bullet $ Với toa có 2 người lên ta có: $\mathrm{C}_3^2$ cách chọn\\
$\bullet $ Người cuối cùng cho vào toa còn lại nên có 1 cách\\
Theo quy tắc nhân ta có: $\left| \Omega _A \right|=A_7^3\cdot \mathrm{C}_7^4\cdot \mathrm{C}_3^2$\\
Do đó: $P(A)=\dfrac{\left| \Omega _A \right|}{\left| \Omega \right|}=\dfrac{450}{16807}$}
\end{ex}
%Câu 188
\begin{ex}
Một đoàn tàu có 7 toa ở một sân ga. Có 7 hành khách từ sân ga lên tàu, mỗi người độc lập với nhau và chọn một toa một cách ngẫu nhiên. Tìm xác suất của các biến cố B: \lq\lq  Mỗi toa có đúng một người lên\rq\rq .
\choice
{$P(B)=\dfrac{6!}{7^7}$}
{$P(B)=\dfrac{5!}{7^7}$}
{$P(B)=\dfrac{8!}{7^7}$}
{\True $P(B)=\dfrac{7!}{7^7}$}
\loigiai{
Số cách lên toa của 7 người là: $\left| \Omega \right|=7^7$.\\
Mỗi một cách lên toa thỏa yêu cầu bài toán chính là một hoán vị của 7 phần tử nên ta có: $\left| \Omega _B \right|=7!$\\
Do đó: $P(B)=\dfrac{\left| \Omega _B \right|}{\left| \Omega \right|}=\dfrac{7!}{7^7}$.
}
\end{ex}
\begin{dang}{CÁC QUY TẮT TÍNH XÁC SUẤT}
1. Quy tắc cộng xác suất\\
Nếu hai biến cố A và B xung khắc thì $P(A\cup B)=P(A)+P(B)$\\
$\bullet $ Mở rộng quy tắc cộng xác suất\\
Cho $k$ biến cố $A_1,A_2,\ldots ,A_k$ đôi một xung khắc. Khi đó:\\
$P(A_1\cup A_2\cup \ldots \cup A_k)=P(A_1)+P(A_2)+\ldots +P(A_k)$.\\
$\bullet $ $P(\overline{A})=1-P(A)$\\
$\bullet $ Giải sử A và B là hai biến cố tùy ý cùng liên quan đến một phép thử. Lúc đó: $P(A\cup B)=P(A)+P(B)-P(AB)$.\\
2. Quy tắc nhân xác suất\\
$\bullet $ Ta nói hai biến cố A và B độc lập nếu sự xảy ra (hay không xảy ra) của A không làm ảnh hưởng đến xác suất của B. \\
$\bullet $ Hai biến cố A và B độc lập khi và chỉ khi $P(AB)=P(A)\cdot P(B)$.\\
\underline{Bài toán 01}\\
Tính xác suất bằng quy tắc cộng\\
Phương pháp: Sử dụng các quy tắc đếm và công thức biến cố đối, công thức biến cố hợp.\\
$\bullet $ $P(A\cup B)=P(A)+P(B)$ với A và B là hai biến cố xung khắc\\
$\bullet $ $P(\overline{A})=1-P(A)$.\\
\underline{Bài toán 02}\\
Tính xác suất bằng quy tắc nhân\\
Phương pháp:\\
Để áp dụng quy tắc nhân ta cần:\\
 $\bullet $ Chứng tỏ $A$ và $B$ độc lập\\
 $\bullet $ Áp dụng công thức: $P(AB)=P(A) \cdot P(B)$
\end{dang}
%Câu 189
\begin{ex}
Một con súc sắc không đồng chất sao cho mặt bốn chấm xuất hiện nhiều gấp 3 lần mặt khác, các mặt còn lại đồng khả năng. Tìm xác suất để xuất hiện một mặt chẵn
\choice
{\True $P(A)=\dfrac{5}{8}$}
{$P(A)=\dfrac{3}{8}$}
{$P(A)=\dfrac{7}{8}$}
{$P(A)=\dfrac{1}{8}$}
\loigiai{
Gọi $A_i$ là biến cố xuất hiện mặt $i$ chấm $(i=1{,}2,3{,}4,5{,}6)$\\
Ta có $P(A_1)=P(A_2)=P(A_3)=P(A_5)=P(A_6)=\dfrac{1}{3}P(A_4)=x$\\
Do $\sum\limits_{k=1}^6{P(A_k)=1\Rightarrow 5x+3x=1\Rightarrow x=\dfrac{1}{8}}$\\
Gọi A là biến cố xuất hiện mặt chẵn, suy ra $A=A_2\cup A_4\cup A_6$\\
Vì cá biến cố $A_i$ xung khắc nên:\\
$P(A)=P(A_2)+P(A_4)+P(A_6)=\dfrac{1}{8}+\dfrac{3}{8}+\dfrac{1}{8}=\dfrac{5}{8}$}
\end{ex}
%Câu 190
\begin{ex}
Gieo một con xúc sắc 4 lần. Tìm xác suất của biến cố A: \lq\lq Mặt 4 chấm xuất hiện ít nhất một lần \rq\rq 
\choice
{\True $P(A)=1-\left(\dfrac{5}{6}\right)^4$}
{$P(A)=1-{{\left(\dfrac{1}{6}\right)}^4}$}
{$P(A)=3-{{\left(\dfrac{5}{6}\right)}^4}$}
{$P(A)=2-{{\left(\dfrac{5}{6}\right)}^4}$}
\loigiai{
Gọi $A_i$ là biến cố \lq\lq mặt 4 chấm xuất hiện lần thứ $i$\rq\rq  với $i=1,2,3,4$.\\
Khi đó: $\overline{A_i}$ là biến cố \lq\lq Mặt 4 chấm không xuất hiện lần thứ $i$\rq\rq \\
Và $P\left(\overline{A_i}\right)=1-P(A_i)=1-\dfrac{1}{6}=\dfrac{5}{6}$\\
Ta có: $\overline{A}$ là biến cố: \lq\lq  không có mặt 4 chấm xuất hiện trong 4 lần gieo\rq\rq \\
Và $\overline{A}=\overline{A_1}\cdot \overline{A_2}\cdot \overline{A_3}\cdot \overline{A_4}$. Vì các $A_i$ độc lập với nhau nên ta có\\
$P(\overline{A})=P\left(\overline{A_1}\right)P\left(\overline{A_2}\right)P\left(\overline{A_3}\right)P\left(\overline{A_4}\right)=\left(\dfrac{5}{6}\right)^4$\\
Vậy $P(A)=1-P\left(\overline{A}\right)=1-{\left(\dfrac{5}{6}\right)}^4$}
\end{ex}
%Câu 191
\begin{ex}
Gieo một con xúc sắc 4 lần. Tìm xác suất của biến cố B: \lq\lq  Mặt 3 chấm xuất hiện đúng một lần\rq\rq 
\choice
{\True $P(A)=\dfrac{5}{324}$}
{$P(A)=\dfrac{5}{32}$}
{$P(A)=\dfrac{5}{24}$}
{$P(A)=\dfrac{5}{34}$}
\loigiai{
Gọi $B_i$ là biến cố \lq\lq  mặt 3 chấm xuất hiện lần thứ $i$\rq\rq  với $i=1{,}2,3{,}4$\\
Khi đó: $\overline{B_i}$ là biến cố \lq\lq  Mặt 3 chấm không xuất hiện lần thứ $i$\rq\rq \\
Ta có: $A=\overline{B_1}\cdot B_2\cdot B_3\cdot B_4\cup B_1\cdot \overline{B_2}\cdot B_3\cdot B_4\cup B_1\cdot B_2\cdot \overline{B_3}\cdot B_4\cup B_1\cdot B_2\cdot B_3\cdot \overline{B_4}$\\
Suy ra $P(A)=P\left(\overline{B_1}\right)P\left(B_2\right)P\left(B_3\right)P\left(B_4\right)+P\left(B_1\right)P\left(\overline{B_2}\right)P\left(B_3\right)P\left(B_4\right)$\\
 $+P\left(B_1\right)P\left(B_2\right)P\left(\overline{B_3}\right)P\left(B_4\right)+P\left(B_1\right)P\left(B_2\right)P\left(B_3\right)P\left(\overline{B_4}\right)$\\
Mà $P\left(B_i\right)=\dfrac{1}{6}$, $P\left(\overline{B_i}\right)=\dfrac{5}{6}$.\\
Do đó: $P(A)=4 \cdot \left(\dfrac{1}{6}\right)^3\cdot \dfrac{5}{6}=\dfrac{5}{324}$}
\end{ex}
%Câu 192
\begin{ex}
Một hộp đựng 4 viên bi xanh,3 viên bi đỏ và 2 viên bi vàng.Chọn ngẫu nhiên 2 viên bi. Tính xác suất để chọn được 2 viên bi cùng màu
\choice
{\True $P(X)=\dfrac{5}{18}$}
{$P(X)=\dfrac{5}{8}$}
{$P(X)=\dfrac{7}{18}$}
{$P(X)=\dfrac{11}{18}$}
\loigiai{
Gọi A là biến cố "Chọn được 2 viên bi xanh"; B là biến cố "Chọn được 2 viên bi đỏ", C là biến cố "Chọn được 2 viên bi vàng" và X là biến cố "Chọn được 2 viên bi cùng màu".\\
Ta có $X=A\cup B\cup C$ và các biến cố $A,B,C$ đôi một xung khắc.\\
Do đó, ta có: $P(X)=P(A)+P(B)+P(C)$.\\
Mà: $P(A)=\dfrac{\mathrm{C}_4^2}{\mathrm{C}_9^2}=\dfrac{1}{6};P(B)=\dfrac{\mathrm{C}_3^2}{\mathrm{C}_9^2}=\dfrac{1}{12};P(C)=\dfrac{\mathrm{C}_2^2}{\mathrm{C}_9^2}=\dfrac{1}{36}$\\
Vậy $P(X)=\dfrac{1}{6}+\dfrac{1}{12}+\dfrac{1}{36}=\dfrac{5}{18}$}
\end{ex}
%Câu 193
\begin{ex}
Một hộp đựng 4 viên bi xanh,3 viên bi đỏ và 2 viên bi vàng.Chọn ngẫu nhiên 2 viên bi. Tính xác suất để chọn được 2 viên bi khác màu
\choice
{\True $P(\overline{X})=\dfrac{13}{18}$}
{$P(\overline{X})=\dfrac{5}{18}$}
{$P(\overline{X})=\dfrac{3}{18}$}
{$P(\overline{X})=\dfrac{11}{18}$}
\loigiai{
Biến cố "Chọn được 2 viên bi khác màu" chính là biến cố $\overline{X}$.\\
Vậy $P(\overline{X})=1-P(X)=\dfrac{13}{18}$}
\end{ex}
%Câu 194
\begin{ex}
Xác suất sinh con trai trong mỗi lần sinh là 0{,}51.Tìm các suất sao cho 3 lần sinh có ít nhất 1 con trai
\choice
{\True $P(A)=\approx 0{,}88$}
{$P(A)=\approx 0{,}23$}
{$P(A)=\approx 0{,}78$}
{$P(A)=\approx 0{,}32$}
\loigiai{
Gọi $A$ là biến cố ba lần sinh có ít nhất 1 con trai, suy ra $\overline{A}$ là xác suất 3 lần sinh toàn con gái.\\
Gọi $B_i$ là biến cố lần thứ i sinh con gái ($i=1{,}2,3$)\\
Suy ra $P(B_1)=P(B_2)=P(B_3)=0{,}49$\\
Ta có: $\overline{A}=B_1\cap B_2\cap B_3$\\
$\Rightarrow P(A)=1-P\left(\overline{A}\right)=1-P\left(B_1\right)P\left(B_2\right)P\left(B_3\right)=1-\left(0{,}49\right)^3\approx 0{,}88$}
\end{ex}
%Câu 195
\begin{ex}
Hai cầu thủ sút phạt đền.Mỗi nười đá 1 lần với xác suất làm bàm tương ứng là 0{,}8 và 0{,}7.Tính xác suất để có ít nhất 1 cầu thủ làm bàn
\choice
{$P(X)=0{,}42$}
{\True $P(X)=0{,}94$}
{$P(X)=0{,}234$}
{$P(X)=0{,}9$}
\loigiai{
Gọi A là biến cố cầu thủ thứ nhất làm bàn}
{là biến cố cầu thủ thứ hai làm bàn\\
X là biến cố ít nhất 1 trong hai cầu thủ làm bàn\\
Ta có: $X=(A\cap \overline{B})\cup \left(\overline{A}\cap B\right)\cup \left(A\cap B\right)$\\
$\Rightarrow P(X)=P(A)\cdot P(\overline{B})+P(B)\cdot P(\overline{A})+P(A)\cdot P(B)=0{,}94$}
\end{ex}
%Câu 196
\begin{ex}
Một đề trắc nghiệm gồm 20 câu, mỗi câu có 4 đáp án và chỉ có một đáp án đúng. Bạn An làm đúng 12 câu, còn 8 câu bạn An đánh hú họa vào đáp án mà An cho là đúng. Mỗi câu đúng được 0{,}5 điểm. Hỏi Anh có khả năng được bao nhiêu điểm?
\choice
{\True $6+\dfrac{1}{4^7}$}
{$5+\dfrac{1}{4^2}$}
{$6+\dfrac{1}{4^2}$}
{$5+\dfrac{1}{4^7}$}
\loigiai{
An làm đúng 12 câu nên có số điểm là $12.0{,}5=6$\\
Xác suất đánh hú họa đúng của mỗi câu là $\dfrac{1}{4}$, do đó xác suất để An đánh đúng 8 câu còn lại là: ${{\left(\dfrac{1}{4}\right)}^8}=\dfrac{1}{4^8}$\\
Vì 8 câu đúng sẽ có số điểm $8.0{,}5=4$\\
Nên số điểm có thể của An là: $6+\dfrac{1}{4^8}\cdot 4=6+\dfrac{1}{4^7}$}
\end{ex}
%Câu 197
\begin{ex}
Một hộp đựng 40 viên bi trong đó có 20 viên bi đỏ, 10 viên bi xanh, 6 viên bi vàng,4 viên bi trắng. Lấy ngẫu nhiên 2 bi, tính xác suất biến cố :\\
 A: \lq\lq 2 viên bi cùng màu\rq\rq .
\choice
{$P(A)=\dfrac{4}{195}$}
{$P(A)=\dfrac{6}{195}$}
{$P(A)=\dfrac{4}{15}$}
{\True $P(A)=\dfrac{64}{195}$}
\loigiai{
Ta có: $\left| \Omega \right|=\mathrm{C}_{40}^2$\\
Gọi các biến cố: D: \lq\lq lấy được 2 bi viên đỏ\rq\rq  ta có: $\left| \Omega _D \right|=\mathrm{C}_{20}^2=190$;\\
 X: \lq\lq lấy được 2 bi viên xanh\rq\rq  ta có: $\left| \Omega _X \right|=\mathrm{C}_{10}^2=45$;\\
 V: \lq\lq lấy được 2 bi viên vàng\rq\rq  ta có: $\left| \Omega _V \right|=\mathrm{C}_6^2=15$;\\
 T: \lq\lq  lấy được 2 bi màu trắng\rq\rq  ta có: $\left| \Omega _T \right|=\mathrm{C}_4^2=6$.\\
Ta có $\text{D},\text{ X},\text{ V},\text{ T}$ là các biến cố đôi một xung khắc và $A=D\cup X\cup V\cup T$\\
$P(A)=P\left(\text{D}\right)+P(X)+P(V)+P(T)=\dfrac{256}{\mathrm{C}_{40}^2}=\dfrac{64}{195}$}
\end{ex}
%Câu 198
\begin{ex}
Một cặp vợ chồng mong muốn sinh bằng đựơc sinh con trai (Sinh được con trai rồi thì không sinh nữa, chưa sinh được thì sẽ sinh nữa). Xác suất sinh được con trai trong một lần sinh là $0{,}51$. Tìm xác suất sao cho cặp vợ chồng đó mong muốn sinh được con trai ở lần sinh thứ 2.
\choice
{$P(C)=0{,}24$}
{$P(C)=0{,}299$}
{$P(C)=0{,}24239$}
{\True $P(C)=0{,}2499$}
\loigiai{
Gọi A là biến cố : \lq\lq  Sinh con gái ở lần thứ nhất\rq\rq , ta có:\\
$P(A)=1-0{,}51=0{,}49$.\\
Gọi B là biến cố: \lq\lq  Sinh con trai ở lần thứ hai\rq\rq , ta có: $P(B)=0{,}51$\\
Gọi C là biến cố: \lq\lq Sinh con gái ở lần thứ nhất và sinh con trai ở lần thứ hai\rq\rq \\
Ta có: $C=AB$, mà $A,B$ độc lập nên ta có:\\
$P(C)=P(AB)=P(A)\cdot P(B)=0{,}2499$}
\end{ex}
%Câu 199
\begin{ex}
Một hộp đựng 10 viên bi trong đó có 4 viên bi đỏ,3 viên bi xanh,2 viên bi vàng,1 viên bi trắng. Lấy ngẫu nhiên 2 bi tính xác suất biến cố : A: \lq\lq 2 viên bi cùng màu\rq\rq 
\choice
{$P(C)=\dfrac{1}{9}$}
{\True $P(C)=\dfrac{2}{9}$}
{$P(C)=\dfrac{4}{9}$}
{$P(C)=\dfrac{1}{3}$}
\loigiai{
Ta có:$|\Omega|=\mathrm{C}_{10}^2$\\
Gọi các biến cố: D: \lq\lq lấy được 2 viên đỏ\rq\rq  ; X: \lq\lq lấy được 2 viên xanh\rq\rq  ;\\
V: \lq\lq lấy được 2 viên vàng\rq\rq \\
Ta có D, X, V là các biến cố đôi một xung khắc và $C=D\cup X\cup V$\\
$P(C)=P\left(\text{D}\right)+P(X)+P(V)=\dfrac{2}{5}+\dfrac{\mathrm{C}_3^2}{45}+\dfrac{1}{15}=\dfrac{10}{45}=\dfrac{2}{9}$}
\end{ex}
%Câu 200
\begin{ex}
Chọn ngẫu nhiên một vé xổ số có 5 chữ số được lập từ các chữ số từ 0 đến 9. Tính xác suất của biến cố X: \lq\lq lấy được vé không có chữ số 2 hoặc chữ số 7\rq\rq 
\choice
{\True $P(X)=0{,}8533$}
{$P(X)=0{,}85314$}
{$P(X)=0{,}8545$}
{$P(X)=0{,}853124$}
\loigiai{
Ta có $|\Omega|=10^5$\\
Gọi A: \lq\lq lấy được vé không có chữ số 2\rq\rq; B: \lq\lq lấy được vé số không có chữ số 7\rq\rq \\
Suy ra $|\Omega_A|=|\Omega_B|=9^5\Rightarrow P(A)=P(B)={{\left(0{,}9\right)}^5}$\\
Số vé số trên đó không có chữ số 2 và 7 là: $8^5$, suy ra $n(A\cap B)=8^5$\\
$\Rightarrow P(A\cap B)={{(0{,}8)}^5}$\\
Do $X=A\cup B$ $\Rightarrow P(X)=P\left(A\cup B\right)=P(A)+P(B)-P\left(A\cap B\right)=0{,}8533$}
\end{ex}
%Câu 201
\begin{ex}
Cho ba hộp giống nhau, mỗi hộp 7 bút chỉ khác nhau về màu sắc\\
Hộp thứ nhất : Có 3 bút màu đỏ, 2 bút màu xanh, 2 bút màu đen\\
Hộp thứ hai : Có 2 bút màu đỏ, 2 màu xanh, 3 màu đen\\
Hộp thứ ba : Có 5 bút màu đỏ, 1 bút màu xanh, 1 bút màu đen\\
Lấy ngẫu nhiên một hộp, rút hú họa từ hộp đó ra 2 bút\\
Tính xác suất của biến cố A: \lq\lq Lấy được hai bút màu xanh\rq\rq 
\choice
{$P(A)=\dfrac{1}{63}$}
{$P(A)=\dfrac{2}{33}$}
{\True $P(A)=\dfrac{2}{66}$}
{$P(A)=\dfrac{2}{63}$}
\loigiai{
Gọi $X_i$ là biến cố rút được hộp thứ i, $i=1{,}2,3\Rightarrow P\left(X_i\right)=\dfrac{1}{3}$\\
Gọi $A_i$ là biến cố lấy được hai bút màu xanh ở hộp thứ i, $i=1{,}2,3$\\
Ta có: $P\left(A_1\right)=P\left(A_2\right)=\dfrac{1}{\mathrm{C}_7^2},P\left(A_3\right)=0$.\\
Vậy $P(A)=\dfrac{1}{3}\left(2\cdot \dfrac{1}{\mathrm{C}_7^2}+0\right)=\dfrac{2}{63}$}
\end{ex}
%Câu 202
\begin{ex}
Cho ba hộp giống nhau, mỗi hộp 7 bút chỉ khác nhau về màu sắc\\
Hộp thứ nhất : Có 3 bút màu đỏ, 2 bút màu xanh, 2 bút màu đen\\
Hộp thứ hai : Có 2 bút màu đỏ, 2 màu xanh, 3 màu đen\\
Hộp thứ ba : Có 5 bút màu đỏ, 1 bút màu xanh, 1 bút màu đen\\
Lấy ngẫu nhiên một hộp, rút hú họa từ hộp đó ra 2 bút\\
Tính xác suất của xác suất B: \lq\lq Lấy được hai bút không có màu đen\rq\rq 
\choice
{$P(B)=\dfrac{1}{63}$}
{$P(B)=\dfrac{3}{63}$}
{$P(B)=\dfrac{13}{63}$}
{\True $P(B)=\dfrac{31}{63}$}
\loigiai{
Gọi $X_i$ là biến cố rút được hộp thứ i, $i=1{,}2,3\Rightarrow P\left(X_i\right)=\dfrac{1}{3}$\\
Gọi $B_i$ là biến cố rút hai bút ở hộp thứ i không có màu đen.\\
$P\left(B_1\right)=\dfrac{\mathrm{C}_5^2}{\mathrm{C}_7^2},P\left(B_2\right)=\dfrac{\mathrm{C}_4^2}{\mathrm{C}_7^2},P\left(B_3\right)=\dfrac{\mathrm{C}_6^2}{\mathrm{C}_7^2}$\\
Vậy có $P(B)=\dfrac{1}{3}\left(\dfrac{\mathrm{C}_5^2+\mathrm{C}_4^2+\mathrm{C}_6^2}{\mathrm{C}_7^2}\right)=\dfrac{31}{63}$}
\end{ex}
%Câu 203
\begin{ex}
Cả hai xạ thủ cùng bắn vào bia. Xác suất người thứ nhất bắn trúng bia là 0{,}8; người thứ hai bắn trúng bia là 0{,}7. Hãy tính xác suất để cả hai người cùng bắn trúng.
\choice
{\True $P(A)=0{,}56$}
{$P(A)=0{,}6$}
{$P(A)=0{,}5$}
{$P(A)=0{,}326$}
\loigiai{
Gọi $A_1$ là biến cố \lq\lq Người thứ nhất bắn trúng bia\rq\rq \\
 $A_2$ là biến cố \lq\lq Người thứ hai bắn trúng bia\rq\rq \\
Gọi A là biến cố \lq\lq cả hai người bắng trúng\rq\rq , suy ra $A=A_1\cap A_2$\\
Vì $A_1,A_2$ là độc lập nên $P(A)=P(A_1)P(A_2)=0{,}8.0{,}7=0{,}56$}
\end{ex}
%Câu 204
\begin{ex}
Cả hai xạ thủ cùng bắn vào bia. Xác suất người thứ nhất bắn trúng bia là 0{,}8; người thứ hai bắn trúng bia là 0{,}7. Hãy tính xác suất để cả hai người cùng không bắn trúng.
\choice
{$P(B)=0{,}04$}
{\True $P(B)=0{,}06$}
{$P(B)=0{,}08$}
{$P(B)=0{,}05$}
\loigiai{
Gọi B là biến cố "Cả hai người bắn không trúng bia".\\
Ta thấy $B=\overline{A_1}\overline{A_2}$. Hai biến cố $\overline{A_1}$ và $\overline{A_2}$ là hai biến cố độc lập nên\\
$P(B)=P\left(\overline{A_1}\right)P\left(\overline{A_2}\right)=\left[1-P(A_1)\right]\left[1-P(A_2)\right]=0{,}06$}
\end{ex}
%Câu 205
\begin{ex}
Cả hai xạ thủ cùng bắn vào bia. Xác suất người thứ nhất bắn trúng bia là 0{,}8; người thứ hai bắn trúng bia là 0{,}7. Hãy tính xác suất để có ít nhất một người bắn trúng.
\choice
{$P(C)=0{,}95$}
{$P(C)=0{,}97$}
{\True $P(C)=0{,}94$}
{$P(C)=0{,}96$}
\loigiai{
Gọi C là biến cố "Có ít nhất một người bắn trúng bia", khi đó biến cố đối của B là biến cố C. 
Do đó $P(C)=1-P(D)=1-0{,}06=0{,}94$}
\end{ex}
%Câu 206
\begin{ex}
Một chiếc máy có hai động cơ I và II hoạt động độc lập với nhau.Xác suất để động cơ I và động cơ II chạy tốt lần lượt là $0{,}8$ và $0{,}7$. Hãy tính xác suất để cả hai động cơ đều chạy tốt.
\choice
{\True $P(C)=0{,}56$}
{$P(C)=0{,}55$}
{$P(C)=0{,}58$}
{$P(C)=0{,}50$}
\loigiai{
Gọi A là biến cố "Động cơ I chạy tốt", B là biến cố "Động cơ II chạy tốt" C là biến cố "Cả hai động cơ đều chạy tốt".Ta thấy A, B là hai biến cố độc lập với nhau và $C=AB$.\\
Ta có $P(C)=P(AB)=P(A)P(B)=0{,}56$}
\end{ex}
%Câu 207
\begin{ex}
Một chiếc máy có hai động cơ I và II hoạt động độc lập với nhau.Xác suất để động cơ I và động cơ II chạy tốt lần lượt là $0{,}8$ và $0{,}7$. Hãy tính xác suất để cả hai động cơ đều không chạy tốt;
\choice
{$P(D)=0{,}23$}
{$P(D)=0{,}56$}
{\True $P(D)=0{,}06$}
{$P(D)=0{,}04$}
\loigiai{
Gọi D là biến cố "Cả hai động cơ đều chạy không tốt".Ta thấy $D=\overline{A}\overline{B}$. Hai biến cố $\overline{A}$ và $\overline{B}$ độc lập với nhau nên\\
$P(D)=\left(1-P(A)\right)\left(1-P(B)\right)=0{,}06$}
\end{ex}
%Câu 208
\begin{ex}
Một chiếc máy có hai động cơ I và II hoạt động độc lập với nhau.Xác suất để động cơ I và động cơ II chạy tốt lần lượt là $0{,}8$ và $0{,}7$. Hãy tính xác suất để có ít nhất một động cơ chạy tốt.
\choice
{$P(K)=0{,}91$}
{$P(K)=0{,}34$}
{$P(K)=0{,}12$}
{\True $P(K)=0{,}94$}
\loigiai{
Gọi K là biến cố "Có ít nhất một động cơ chạy tốt", khi đó biến cố đối của K là biến cố D. Do đó $P(K)=1-P(D)=0{,}94$}
\end{ex}
%Câu 209
\begin{ex}
Có hai xạ thủ I và xạ tám xạ thủ II. Xác suất bắn trúng của I là $0{,}9$; xác suất của II là $0{,}8$ lấy ngẫu nhiên một trong hai xạ thủ, bắn một viên đạn. Tính xác suất để viên đạn bắn ra trúng đích.
\choice
{$P(A)=0{,}4124$}
{$P(A)=0{,}842$}
{$P(A)=0{,}813$}
{\True $P(A)=0{,}82$}
\loigiai{
Gọi $B_i$ là biến cố \lq\lq Xạ thủ thứ $i$ bắn trúng đích \rq\rq, $i=1{,}2$. Ta có :\\
$P\left(B_i\right)=\dfrac{2}{10}$,$P\left(B_2\right)=\dfrac{8}{10}\And P\left(A/B_1\right)=0{,}9P\left(A/B_2\right)=0{,}8$\\
Nên $P(A)=P\left(B_1\right)P\left(A/B_1\right)+P\left(B_2\right)P\left(A/B_2\right)=\dfrac{2}{10}\cdot \dfrac{9}{10}+\dfrac{8}{10}\cdot \dfrac{8}{10}=0{,}82$}
\end{ex}
%Câu 210
\begin{ex}
Bốn khẩu pháo cao xạ A,B,C,D cùng bắn độc lập vào một mục tiêu.Biết xác suất bắn trúng của các khẩu pháo tương ứng là $P(A)=\dfrac{1}{2}\cdot P(B)-\dfrac{2}{3},P(C)=\dfrac{4}{5},P(D)=\dfrac{5}{7}$.Tính xác suất để mục tiêu bị bắn trúng
\choice
{$P(D)=\dfrac{14}{105}$}
{$P(D)=\dfrac{4}{15}$}
{$P(D)=\dfrac{4}{105}$}
{\True $P(D)=\dfrac{104}{105}$}
\loigiai{
Tính xác suất mục tiêu không bị bắn trúng: $P(H)=\dfrac{1}{2}\cdot \dfrac{1}{3}\cdot \dfrac{1}{5}\cdot \dfrac{2}{7}=\dfrac{1}{105}$\\
Vậy xác suất trúng đích $P(D)=1-\dfrac{1}{105}=\dfrac{104}{105}$}
\end{ex}
%Câu 211
\begin{ex}
Một hộp đựng 10 viên bi trong đó có 4 viên bi đỏ,3 viên bi xanh, 2 viên bi vàng,1 viên bi trắng.Lấy ngẫu nhiên 2 bi tính xác suất biến cố A: \lq\lq 2 viên lấy ra màu đỏ\rq\rq 
\choice
{$|\Omega_A|=\dfrac{\mathrm{C}_4^2}{\mathrm{C}_{10}^2}$}
{$|\Omega_A|=\dfrac{\mathrm{C}_5^2}{\mathrm{C}_{10}^2}$}
{$|\Omega_A|=\dfrac{\mathrm{C}_4^2}{\mathrm{C}_8^2}$}
{\True $|\Omega_A|=\dfrac{\mathrm{C}_7^2}{\mathrm{C}_{10}^2}$}
\loigiai{
$\Omega =\mathrm{C}_{10}^2$\\
$|\Omega_A|=\mathrm{C}_4^2\Rightarrow P(A)=\dfrac{\mathrm{C}_4^2}{\mathrm{C}_{10}^2}$}
\end{ex}
%Câu 212
\begin{ex}
Một hộp đựng 10 viên bi trong đó có 4 viên bi đỏ,3 viên bi xanh, 2 viên bi vàng,1 viên bi trắng.Lấy ngẫu nhiên 2 bi tính xác suất biến cố B: \lq\lq 2 viên bi một đỏ,1 vàng\rq\rq 
\choice
{$|\Omega_B|=\dfrac{8}{55}$}
{$|\Omega_B|=\dfrac{2}{5}$}
{$|\Omega_B|=\dfrac{8}{15}$}
{\True $|\Omega_B|=\dfrac{8}{45}$}
\loigiai{
$\Omega =\mathrm{C}_{10}^2$\\
$|\Omega_B|=\mathrm{C}_4^1\cdot \mathrm{C}_2^1\Rightarrow P(B)=\dfrac{\mathrm{C}_4^1\cdot \mathrm{C}_2^1}{\mathrm{C}_{10}^2}=\dfrac{8}{45}$}
\end{ex}
%Câu 213
\begin{ex}
Một hộp đựng 10 viên bi trong đó có 4 viên bi đỏ,3 viên bi xanh, 2 viên bi vàng,1 viên bi trắng.Lấy ngẫu nhiên 2 bi tính xác suất biến cố C: \lq\lq 2 viên bi cùng màu\rq\rq 
\choice
{$P(C)=\dfrac{7}{9}$}
{$P(C)=\dfrac{1}{9}$}
{$P(C)=\dfrac{5}{9}$}
{\True $P(C)=\dfrac{2}{9}$}
\loigiai{
$\Omega =\mathrm{C}_{10}^2$\\
Đ là biến cố 2 viên đỏ,X là biến cố 2 viên xanh,V là biến cố 2 viên vàng\\
Đ, X, V là các biến cố đôi một xung khắc\\
$P(C)=P\left(\text{D}\right)+P(X)+P(V)=\dfrac{2}{5}+\dfrac{\mathrm{C}_3^2}{45}+\dfrac{1}{15}=\dfrac{10}{45}=\dfrac{2}{9}$}
\end{ex}
%Câu 214
\begin{ex}
Gieo ngẫu nhiên một con xúc xắc 6 lần.Tính xác suất để một số lớn hơn hay bằng 5 xuất hiện ít nhất 5 lần trong 6 lần gieo
\choice
{$\dfrac{23}{729}$}
{$\dfrac{13}{79}$}
{$\dfrac{13}{29}$}
{\True $\dfrac{13}{729}$}
\loigiai{
Gọi A là biến cố một số lớn hơn hay bẳng 5 chấm trong mỗi lần gieo.A xảy ra,con xúc xắc xuất hiện mặt 5,chấm hoặc 6 chấm ta có $P(A)=\dfrac{2}{6}=\dfrac{1}{3}$.\\
Trong 6 lần gieo xác suất để biến cố A xảy ra đúng 6 lần $P\left(A.A\cdot A.A\cdot A.A\right)=\left(\dfrac{1}{3}\right)^6$\\
Xác suất để được đúng 5 lần xuất hiện A và 1 lần không xuất hiện A theo một thứ tự nào đó $\left(\dfrac{1}{3}\right)^5\cdot \dfrac{2}{3}$\\
Vì có 6 cách để biến cố này xuất hiện : $6\cdot {{\left(\dfrac{1}{3}\right)}^5}\cdot \dfrac{2}{3}=\dfrac{12}{729}$\\
Vậy xác xuất để A xuất hiện ít nhất 5 lần là $\dfrac{12}{729}+{{\left(\dfrac{1}{3}\right)}^6}=\dfrac{13}{729}$}
\end{ex}
%Câu 215
\begin{ex}
Một người bắn liên tiếp vào một mục tiêu khi viên đạn trúng mục tiêu thì thôi (các phát súng độc lập nhau). Biết rằng xác suất trúng mục tiêu của mỗi lần bắn như nhau và bằng 0{,}6.Tính xác suất để bắn đến viên thứ 4 thì ngừng bắn
\choice
{$P(H)=0{,}03842$}
{$P(H)=0{,}384$}
{$P(H)=0{,}03384$}
{\True $P(H)=0{,}0384$}
\loigiai{
Gọi $A_i$ là biến cố trúng đích lần thứ 4\\
H là biến cố bắn lần thứ 4 thì ngừng $H=\overline{A_1}\cap \overline{A_2}\cap \overline{A_3}\cap A_4$\\
$P(H)=0{,}4.0{,}4.0{,}4.0{,}6=0{,}0384$}
\end{ex}
%Câu 216
\begin{ex}
Chọn ngẫu nhiên một vé xổ số có 5 chữ số được lập từ các chữ số từ 0 đến 9. Tính xác suất của biến cố X: \lq\lq lấy được vé không có chữ số 1 hoặc chữ số 2\rq\rq .
\choice
{$P(X)=0{,}8534$}
{$P(X)=0{,}84$}
{$P(X)=0{,}814$}
{\True $P(X)=0{,}8533$}
\loigiai{
Ta có $\left| \Omega \right|=10^5$\\
Gọi A: \lq\lq lấy được vé không có chữ số 1\rq\rq, B: \lq\lq lấy được vé số không có chữ số 2\rq\rq \\
Suy ra $\left| \Omega _A \right|=\left| \Omega _B \right|=9^5\Rightarrow P(A)=P(B)=\left(0{,}9\right)^5$\\
Số vé số trên đó không có chữ số 1 và 2 là: $8^5$, suy ra $\left| {{\Omega }_{A\cap B}} \right|=8^5$\\
Nên ta có: $P(A\cap B)={{(0{,}8)}^5}$\\
Do $X=A\cup B$.\\
Vậy $P(X)=P\left(A\cup B\right)=P(A)+P(B)-P\left(A\cap B\right)=0{,}8533$}
\end{ex}
%Câu 217
\begin{ex}
Một máy có 5 động cơ gồm 3 động cơ bên cánh trái và hai động cơ bên cánh phải. Mỗi động cơ bên cánh phải có xác suất bị hỏng là $0{,}09$, mỗi động cơ bên cánh trái có xác suất bị hỏng là $0{,}04$. Các động cơ hoạt động độc lập với nhau. Máy bay chỉ thực hiện được chuyến bay an toàn nếu có ít nhất hai động cơ làm việc. Tìm xác suất để máy bay thực hiện được chuyến bay an toàn.
\choice
{\True $P(A)=0{,}9999074656$}
{$P(A)=0{,}981444$}
{$P(A)=0{,}99074656$}
{$P(A)=0{,}91414148$}
\loigiai{
Gọi A là biến cố: \lq\lq Máy bay bay an toàn\rq\rq .\\
Khi đó $\overline{A}$ là biến cố: \lq\lq Máy bay bay không an toàn\rq\rq .\\
Ta có máy bay bay không an toàn khi xảy ra một trong các trường hợp sau\\
TH 1. Cả 5 động cơ đều bị hỏng\\
Ta có xác suất để xảy ra trường hợp này là: ${{\left(0{,}09\right)}^3}\cdot {{\left(0{,}04\right)}^2}$\\
TH 2. Có một động cơ ở cánh phải hoạt động và các động cơ còn lại đều bị hỏng. Xác suất để xảy ra trường hợp này là: $3\cdot {{\left(0{,}09\right)}^2}\cdot 0{,}91\cdot {{(0{,}04)}^2}$\\
TH 3. Có một động cơ bên cánh trái hoạt động, các động cơ còn lại bị hỏng\\
Xác suất xảy ra trường hợp này là: $2.0{,}04.0{,}96\cdot {{(0{,}09)}^3}$\\
$P\left(\overline{A}\right)={{\left(0{,}09\right)}^3}\cdot {{\left(0{,}04\right)}^2}+3\cdot {{\left(0{,}09\right)}^2}\cdot 0{,}91\cdot {{(0{,}04)}^2}+2.0{,}04.0{,}96\cdot {{(0{,}09)}^3}$\\
 $=0,{{925344.10}^{-4}}$.\\
Vậy $P(A)=1-P\left(\overline{A}\right)=0{,}9999074656$}
\end{ex}
%Câu 218
\begin{ex}
Ba cầu thủ sút phạt đến 11m, mỗi người đá một lần với xác suất làm bàn tương ứng là $x$, $y$ và $0{,}6$ (với $x>y$). Biết xác suất để ít nhất một trong ba cầu thủ ghi bàn là $0{,}976$ và xác suất để cả ba cầu thủ đều ghi ban là $0{,}336$. Tính xác suất để có đúng hai cầu thủ ghi bàn.
\choice
{\True $P(C)=0{,}452$}
{$P(C)=0{,}435$}
{$P(C)=0{,}4525$}
{$P(C)=0{,}4245$}
\loigiai{
Gọi $A_i$ là biến cố \lq\lq người thứ $i$ ghi bàn\rq\rq  với $i=1{,}2,3$.\\
Ta có các $A_i$ độc lập với nhau và $P\left(A_1\right)=x, P\left(A_2\right)=y, P\left(A_3\right)=0{,}6$.\\
Gọi A là biến cố: \lq\lq  Có ít nhất một trong ba cầu thủ ghi bàn\rq\rq \\
 B: \lq\lq  Cả ba cầu thủ đều ghi bàn\rq\rq \\
 C: \lq\lq Có đúng hai cầu thủ ghi bàn\rq\rq \\
Ta có: $\overline{A}=\overline{A_1}\cdot \overline{A_2}\cdot \overline{A_3}\Rightarrow P\left(\overline{A}\right)=P\left(\overline{A_1}\right)\cdot P\left(\overline{A_2}\right)\cdot P\left(\overline{A_3}\right)=0{,}4(1-x)(1-y)$\\
Nên $P(A)=1-P\left(\overline{A}\right)=1-0{,}4(1-x)(1-y)=0{,}976$\\
Suy ra$(1-x)(1-y)=\dfrac{3}{50}\Leftrightarrow xy-x-y=-\dfrac{47}{50}$ (1).\\
Tương tự: $B=A_1\cdot A_2\cdot A_3$, suy ra:\\
$P(B)=P\left(A_1\right)\cdot P\left(A_2\right)\cdot P\left(A_3\right)=0{,}6xy=0{,}336$ hay là $xy=\dfrac{14}{25}$ (2)\\
Từ (1) và (2) ta có hệ: $\heva{& xy=\dfrac{14}{25} \\& x+y=\dfrac{3}{2} }$, giải hệ này kết hợp với $x>y$ ta tìm được\\
$x=0{,}8$ và $y=0{,}7$.\\
Ta có: $C=\overline{A_1}A_2A_3+A_1\overline{A_2}A_3+A_1A_2\overline{A_3}$\\
Nên $P(C)=(1-x)y\cdot 0{,}6+x(1-y)\cdot 0{,}6+xy\cdot 0{,}4=0{,}452$}
\end{ex}
%Câu 219
\begin{ex}
Một bài trắc nghiệm có 10 câu hỏi, mỗi câu hỏi có 4 phương án lựa chọn trong đó có 1 đáp án đúng. Giả sử mỗi câu trả lời đúng được 5 điểm và mỗi câu trả lời sai bị trừ đi 2 điểm. Một học sinh không học bài nên đánh hú họa một câu trả lời. Tìm xác suất để học sinh này nhận điểm dưới 1.
\choice
{$P(A)=0{,}7124$}
{\True $P(A)=0{,}7759$}
{$P(A)=0{,}7336$}
{$P(A)=0{,}783$}
\loigiai{
Ta có xác suất để học sinh trả lời câu đúng là $\dfrac{1}{4}$ và xác suất trả lời câu sai là $\dfrac{3}{4}$.\\
Gọi $x$ là số câu trả lời đúng, khi đó số câu trả lời sai là $10-x$\\
Số điểm học sinh này đạt được là : $4x-2(10-x)=6x-20$\\
Nên học sinh này nhận điểm dưới 1 khi $6x-20<1\Leftrightarrow x<\dfrac{21}{6}$\\
Mà $x$ nguyên nên $x$ nhận các giá trị: $0{,}1,2{,}3$.\\
Gọi $A_i$ ($i=0{,}1,2{,}3$) là biến cố: \lq\lq Học sinh trả lời đúng $i$ câu\rq\rq , $B$ là biến cố: \lq\lq  Học sinh nhận điểm dưới 1\rq\rq \\
Suy ra: $A=A_0\cup A_1\cup A_2\cup A_3$ và $P(A)=P(A_0)+P(A_1)+P(A_2)+P(A_3)$\\
Mà: $P(A_i)=\mathrm{C}_{10}^i\cdot {{\left(\dfrac{1}{4}\right)}^i}{{\left(\dfrac{3}{4}\right)}^{10-i}}$ nên $P(A)=\sum\limits_{i=0}^3{\mathrm{C}_{10}^i\cdot {{\left(\dfrac{1}{4}\right)}^i}{{\left(\dfrac{3}{4}\right)}^{10-i}}}=0{,}7759$
}
\end{ex}
\Closesolutionfile{ans}