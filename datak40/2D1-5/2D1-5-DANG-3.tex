\begin{dang}{Tương giao của hai hàm cụ thể}
    Xác định tọa độ giao điểm của hai đồ thị $y=f(x)$ và $y=g(x)$:
\begin{itemize}
    \item \textbf{Bước 1: } Giải phương trình hoành độ giao điểm $f(x)=g(x)$, tìm các nghiệm $x_0 \in \mathscr{D}_f \cap \mathscr{D}_g$.
    \item \textbf{Bước 2: } Với $x_0$ vừa tìm, thay vào một trong hai hàm số ban đầu để tìm $y_0$.
    \item \textbf{Bước 3: } Kết luận giao điểm $(x_0;y_0)$.
\end{itemize}
\end{dang}
\begin{vd}
Tìm tọa độ giao điểm điểm của
\begin{listEX}[2]
    \item Đồ thị hàm số $y=x^2-2$ và đồ thị $y=2x+1$.
    \item Đồ thị hàm số $y=\dfrac{x-3}{2x}$ và trục hoành.
\end{listEX}
\loigiai{}
\end{vd}
\begin{vd}
Tìm $m$ đề hàm số $y=x^2+2x-m$ cắt trục hoành tại hai điểm phân biệt.
\loigiai{}
\end{vd}
\BTTN
\Opensolutionfile{ans}[ans/2D1-5-DANG-3]
\begin{ex}%[Hoàng Blue - DA1]%[2D1B5-4]
%câu 1
Số giao điểm của đồ thị hàm số $y=\dfrac{x-2}{4x}$ với trục hoành là bao nhiêu?
\choice
{\True $1$}
{$2$}
{$3$}
{$4$}
\loigiai{
    Vì đồ thị hàm số cắt trục hoành nên ta có $\dfrac{x-2}{4x} = 0\Rightarrow x= 2.$\\
    Vậy số giao điểm của đồ thị hàm số $y=\dfrac{x-2}{4x}$ và trục hoành là $1$ điểm.
}
\end{ex}
\begin{ex}%[Hoàng Blue - DA1]%[2D1B5-4]
%câu 2
Đồ thị hàm số $y=x^3-16x^2+13x+2$ cắt trục tung tại điểm nào sau đây?
\choice
{$M(1;0)$}
{$N(-1;0)$}
{\True $P(0;2)$}
{$O(0;0)$}
\loigiai{
    Đồ thị hàm số $y=x^3-16x^2+13x+2$ cắt trục tung nên giao điểm có hoành độ $x=0\Rightarrow y =2$.\\
    Vậy đồ thị cắt trục hoành tại điểm $P(0;2)$.
}
\end{ex}
\begin{ex}%[Hoàng Blue - DA1]%[2D1B5-4]
%câu 3
Số giao điểm của đồ thị hàm số $y=x^3-3x+2$ và trục hoành là bao nhiêu?
\choice
{$3$ điểm}
{\True $2$ điểm}
{$1$ điểm}
{$0$ điểm}
\loigiai{
    Vì đồ thị hàm số cắt trục hoành nên ta có $x^3-3x+2 = 0\Leftrightarrow \hoac{&x=-2\\&x=1.}$\\
    Vậy số giao điểm của đồ thị hàm số $y=x^3-3x+2$ và trục hoành là $2$ điểm.
}
\end{ex}
\begin{ex}%[Hoàng Blue - DA1]%[2D1B5-4]
%câu 4
Tọa độ giao điểm của đường thẳng $y=x+1$ với đồ thị hàm số $y=\dfrac{x+1}{x-2}$ là
\choice
{$A(4;2)$, $B(0;-1)$}
{$C(-1;3)$}
{$D(3;-1)$}
{\True $E(-1;0)$, $J(3;4)$}
\loigiai{
    Phương trình hoành độ giao điểm ($\forall x \ne 2$) là
    \begin{eqnarray*}
        \dfrac{x+1}{x-2}=x+1 &\Leftrightarrow& x+1 = (x-2)(x+1)\\
        &\Leftrightarrow& x+1 = x^2 - x - 2\\
        &\Leftrightarrow&x^2-2x-3=0\\
        &\Leftrightarrow& \hoac{&x=-1\\&x=3.}
    \end{eqnarray*}
    \begin{itemize}
        \item Với $x=-1\Rightarrow y=0$.
        \item Với $x=3 \Rightarrow y = 4$.
    \end{itemize}
    Vậy tọa độ giao điểm của đường thẳng $y=x+1$ với đồ thị hàm số $y=\dfrac{x+1}{x-2}$ là $E(-1;0)$, $J(3;4)$.
}
\end{ex}
\begin{ex}%[Hoàng Blue - DA1]%[2D1B5-4]
%câu 5
Biết đồ thị hàm số $y=x^4-3x^2+5$ và đường thẳng $y=9$ cắt nhau tại hai điểm phân biệt $A(x_1;y_1)$ và $B(x_2; y_2)$. Giá trị của $x_1+x_2$ bằng
\choice
{$3$}
{\True $0$}
{$18$}
{$5$}
\loigiai{
    Phương trình hoành độ giao điểm là $x^4-3x^2+5 = 9\Leftrightarrow x^4 - 3x^2 - 4\Leftrightarrow \hoac{&x^2=-1\\&x^2=4}\Leftrightarrow x= \pm 2$.\\
    Vậy $x_1+x_2 = -2+2=0$.
}
\end{ex}
\begin{ex}%[Hoàng Blue - DA1]%[2D1B5-4]
%câu 6
Tọa độ giao điểm của đường thẳng $y=x+2$ với đồ thị hàm số $y=x^3+3x+2$ là
\choice
{$(0;3)$}
{$(0;-2)$}
{$(2;0)$}
{\True $(0;2)$}
\loigiai{
    Ta có phương trình hoành độ giao điểm là\\
    \[x+2 = x^3+3x+2 \Leftrightarrow x^3+2x=0\Leftrightarrow x=0.\]
    Với $x=0\Rightarrow y = 2$.\\
    Vậy tọa độ giao điểm cần tìm là $(0;2)$.
}
\end{ex}
\begin{ex}%[Hoàng Blue - DA1]%[2D1B5-4]
%câu 7
Đồ thị hàm số $y=\dfrac{2x+1}{x+1}$ cắt các trục tọa độ tại $A$, $B$. Độ dài đoạn $AB$ bằng
\choice
{\True $\dfrac{\sqrt{5}}{2}$}
{$\dfrac{1}{2}$}
{$\dfrac{\sqrt{2}}{2}$}
{$\dfrac{5}{4}$}
\loigiai{
    Đồ thị hàm số cắt trục tung nên $y=1$, đồ thị hàm số cắt trục hoành ($\forall x \ne -1$) nên $\dfrac{2x+1}{x+1}=0\Rightarrow x= -\dfrac{1}{2}$.\\
    Vậy $A(0;1)$, $B\left( -\dfrac{1}{2};0\right)$ là tọa độ hai điểm cần tìm.\\
    $\vec{AB}=\left(-\dfrac{1}{2};-1 \right)$$\Rightarrow \left|\vec{AB}\right| =\sqrt{\left( -\dfrac{1}{2}\right)^2+(-1)^2}=\dfrac{\sqrt{5}}{2}$.
}
\end{ex}
\begin{ex}%[Hoàng Blue - DA1]%[2D1B5-4]
%câu 7
Đồ thị hàm số $y=x^2-x$ và đồ thị hàm số $y=5+\dfrac{3}{x}$ cắt nhau tại hai điểm $A$ và $B$. Độ dài đoạn $AB$ bằng
\choice
{$8\sqrt{5}$}
{$25$}
{\True $4\sqrt{2}$}
{$10\sqrt{2}$}
\loigiai{
    Ta có phương trình hoành độ giao điểm $x^2-x=5+\dfrac{3}{x}\Leftrightarrow x^3 - x^2 -5x-3=0\Leftrightarrow \hoac{&x=3\\&x=-1}$.\\
    $\Rightarrow A(3;6)$, $B(-1;2)$. \\
    $\vec{AB} = (-4,-4)\Rightarrow |\vec{AB}|=\sqrt{(-4)^2+(-4)^2} = 4\sqrt{2}$.
}
\end{ex}
\begin{ex}%[Sơn Bùi - DA1]%[2D1Y5-4]
Đồ thị hàm số $y=\dfrac{2x-1}{x+5}$ và đường thẳng $y=x-1$ cắt nhau tại hai điểm phân biệt $A$, $B$. Tìm hoành độ trung điểm $I$ của đoạn thẳng $AB$.
\choice
{$x_I=1$}
{$x_I=-2$}
{$x_I=2$}
{\True $x_I=-1$}
\loigiai{Phương trình hoành độ giao điểm
    $$\dfrac{2x-1}{x+5}=x-1\Leftrightarrow \heva{&x^2+2x-6=0\\&x\neq -5}\Leftrightarrow x^2+2x-6=0.$$
    Phương trình trên có hai nghiệm $x_A$, $x_B$ nên theo định lí Vi1et, ta có $x_A+x_B=-2$.\\
    Suy ra $x_I=\dfrac{x_A+x_B}{2}=\dfrac{-2}{2}=-1$.
}
\end{ex}
%%==========Câu 10
\begin{ex}%[Sơn Bùi - DA1]%[2D1Y5-4]
Gọi $M$, $N$ lần lượt là giao điểm của đường thẳng $y=x+1$ và đường cong $y=\dfrac{2x+4}{x-1}$. Tìm tọa độ trung điểm $I$ của đoạn thẳng $MN$.
\choice
{\True $I(1;2)$}
{$I(-2;-3)$}
{$I(1;3)$}
{$I(2;3)$}
\loigiai{Phương trình hoành độ giao điểm
    $$\dfrac{2x+4}{x-1}=x+1\Leftrightarrow \heva{&x^2-2x-5=0\\&x\neq 1}\Leftrightarrow x^2-2x-5=0.$$
    Phương trình trên có hai nghiệm $x_M$, $x_N$ nên theo định lý Viét, ta có $x_M+x_N=2$.\\
    Suy ra $\heva{&x_I=\dfrac{x_M+x_N}{2}=1\\&y_I=\dfrac{y_M+y_N}{2}=\dfrac{x_M+1+x_N+1}{2}=2}\Rightarrow I(1;2).$
}
\end{ex}
\begin{ex}%[2D1B5-4]
Đường thẳng $y=x-1$ cắt đồ thị hàm số $y=x^3-x^2+x-1$ tại hai điểm. Tìm tổng tung độ các giao điểm đó.
\choice
{$-3 $}
{$2 $}
{$0 $}
{\True $-1 $}
\loigiai{Phương trình hoành độ giao điểm
    $$x^3-x^2+x-1=x-1\Leftrightarrow \left[ \begin{aligned} &x=1 \Rightarrow y=0\\ &x=0 \Rightarrow y=-1.\end{aligned} \right.$$
    Tổng tung độ các giao điểm là $0+(-1)=-1$.
}
\end{ex}
\begin{ex}%[2D1Y5-4]
Số giao điểm của đồ thị hàm số $y=(x-1)(x^2-3x+2)$ và trục hoành là
\choice
{$0$}
{$1$}
{\True $2$}
{$3$}
\loigiai{
    Phương trình $y=0$ có hai nghiệm là $x=1$ và $x=2$.
}
\end{ex}
\begin{ex}
Đồ thị hàm số $y=x^3-3x^2+2x-1$ cắt đồ thị hàm số $y=x^2-3x+1$ tại hai điểm phân biệt $A,B$. Tính độ dài $AB$.
\choice
{$AB=3$}
{$AB=2\sqrt2$}
{$AB=2$}
{\True $AB=1$}
\loigiai{
    Phương trình hoành độ giao điểm $$x^3-3x^2+2x-1=x^2-3x+1\Leftrightarrow x^3-4x^2+5x-2=0\Leftrightarrow \hoac{& x=1\\& x=2}\Rightarrow \hoac{& y=-1\\& y=-1}.$$
    Không mất tính tổng quát, ta giả sử $A(1;-1),B(2;-1)$. Suy ra $\vec{AB}=(1;0)\Rightarrow AB=1$.
}
\end{ex}
\begin{ex}
Đồ thị của hàm số $ y = \dfrac{x - 1}{x+1} $ cắt hai trục $ Ox $ và $ Oy $ tại $ A $ và $ B $. Khi đó diện tích của tam giác $ OAB $ (với $ O $ là gốc tọa độ) bằng
\choice
{$ 1 $}
{$ \dfrac{1}{4} $}
{$ 2 $}
{\True $ \dfrac{1}{2} $}
\loigiai{
    Ta có $ A(1;0), B(0;-1) $. Diện tích $ S_{\triangle OAB} = \dfrac{OA\cdot OB}{2} = \dfrac{1}{2} $.
}
\end{ex}
\begin{ex}
Biết đường thẳng $y=x-2$ cắt đồ thị hàm số $ y=\dfrac{x}{x-1} $ tại $ 2 $ điểm phân biệt $ A, $ $ B. $ Tìm hoành độ trọng tâm tam giác $OAB$ với $O$ là gốc tọa độ.
\choice
{$ \dfrac{2}{3} $}
{$ 2 $}
{\True $ \dfrac{4}{3} $}
{$ 4 $}
\loigiai{
    Xét phương trình hoành độ giao điểm $ x-2=\dfrac{x}{x-1} $ (Điều kiện $ x\neq 1 $).
    $ \Rightarrow (x-2)(x-1)=x\Leftrightarrow x^2-4x+2=0 \, (1).$
    Khi đó $ A(x_1;x_1-2), $ $ B(x_2;x_2-2) $ với $ x_1, x_2 $ là $ 2 $ nghiệm của phương trình $ (1) $ thỏa mãn
    $ \heva{&x_1+x_2=4\\&x_1.x_2=2}. $ Gọi $ G\left(x_G;y_G\right) $ là trọng tâm tam giác $ OAB. $
    $ \Rightarrow x_G=\dfrac{0+x_1+x_2}{3}=\dfrac{4}{3}. $
}
\end{ex}
\begin{ex}%[2D1B5]
Gọi $ M, N $ là giao điểm của đường thẳng $ y = x + 1 $ và đường cong $ y = \dfrac{2x+4}{x-1} $. Tìm hoành độ trung điểm của đoạn thẳng $ MN. $
\choice
{$ x = -1 $}
{\True $ x = 1 $}
{$ x = -2 $}
{ $ x = 2$}
\loigiai
{Xét phương trình hoành độ giao điểm $ x+ 1 = \dfrac{2x +4}{x-1} \Leftrightarrow \heva{&x \ne 1\\ &x^2 - 2x- 5 = 0}$\\
    $ \Rightarrow x_M + x_N = 2 \Rightarrow x_I = \dfrac{x_M+x_N}{2} = 1.$}
\end{ex}
\begin{ex}
Cho hàm số $y=\dfrac{2x}{x+1}$ có đồ thị $(C)$. Gọi $A,B$ là giao điểm của đường thẳng $d:y=x$ với đồ thị $(C)$. Tính độ dài đoạn $AB$.
\choice
{\True $AB=\sqrt{2}$}
{ $AB=\dfrac{\sqrt{2}}{2}$}
{$AB=1$}
{ $AB=2$}
\loigiai{
    Phương trình hoành độ giao điểm\\
    $\dfrac{2x}{x+1}=x,\left({x\ne -1}\right)\Rightarrow x^2-x=0\Rightarrow \left[{\begin{aligned}&{x=0\Rightarrow y=0\Rightarrow A\left({0;0}\right)} \\ &{x=1\Rightarrow y=1\Rightarrow B\left({1;1}\right)} \\ \end{aligned}}\right.$\\
    Vậy $AB=\sqrt{2}$.
}
\end{ex}
%%==========Câu 11
\begin{ex}%[Sơn Bùi - DA1]%[2D1Y5-4]
Đồ thị hàm số $y=x^4+x^2$ và đồ thị hàm số $y=-x^2-1$ có bao nhiêu điểm chung?
\choice
{\True $0$}
{$1$}
{$2$}
{$3$}
\loigiai{Phương trình hoành độ giao điểm
    $$x^4+x^2=-x^2-1\Leftrightarrow x^4+2x^2+1=0.$$
    Phương trình trên vô nghiệm vì $x^4+2x^2+1\ge 1>0$ với mọi $x\in\mathbb{R}$.\\
    Vậy hai đồ thị đã cho không có điểm chung.
}
\end{ex}
%%==========Câu 12
\begin{ex}%[Sơn Bùi - DA1]%[2D1Y5-4]
Hai đồ thị $(\mathrm{C})\colon y=x^3-2x+2$ và $(\mathrm{C'})\colon y=3x^2-x-1$ có bao nhiêu giao điểm?
\choice
{$0$}
{$1$}
{$2$}
{\True $3$}
\loigiai{Phương trình hoành độ giao điểm
    $$x^3-2x+2=3x^2-x-1\Leftrightarrow x^3-3x^2-x+3=0\Leftrightarrow\hoac{&x=1\\&x=-1\\&x=3.}$$
    Vậy hai đồ thị đã cho có $3$ điểm chung.}
\end{ex}
%%==========Câu 13
\begin{ex}%[Sơn Bùi - DA1]%[2D1Y5-4]
Đồ thị hàm số $y=x^4-2x^2$ cắt trục hoành tại bao nhiêu điểm?
\choice
{$0$}
{$2$}
{$4$}
{\True $3$}
\loigiai{Phương trình hoành độ giao điểm $x^4-2x^2=0\Leftrightarrow\hoac{&x=0\\&x=\sqrt{2}\\&x=-\sqrt{2}.}$\\
    Vậy đồ thị hàm số đã cho cắt trục hoành tại $3$ điểm phân biệt.}
\end{ex}
%%==========Câu 14
\begin{ex}%[Sơn Bùi - DA1]%[2D1Y5-4]
Đường thẳng $y=x+1$ cắt đồ thị hàm số $y=\dfrac{2x+2}{x-1}$ tại hai điểm phân biệt $A(x_1;y_1)$ và $B(x_2;y_2)$. Tính $S=y_1+y_2$.
\choice
{$S=1$}
{\True $S=4$}
{$S=3$}
{$S=0$}
\loigiai{Phương trình hoành độ giao điểm
    $$\dfrac{2x+2}{x-1}=x+1\Leftrightarrow \heva{&(x+1)(x-3)=0\\&x\neq 1}\Leftrightarrow\heva{&x_1=-1\Rightarrow y_1=0\\&x_2=3\Rightarrow y_2=4}\Rightarrow S=y_1+y_2=4.$$
}
\end{ex}
%%==========Câu 15
\begin{ex}%[Sơn Bùi - DA1]%[2D1Y5-4]
Tìm tung độ giao điểm của đồ thị $(\mathrm{C})\colon y=\dfrac{2x-3}{x+3}$ và đường thẳng $(d)\colon y=x-1$.
\choice
{$3$}
{\True $-1$}
{$1$}
{$-3$}
\loigiai{Phương trình hoành độ giao điểm
    $$\dfrac{2x-3}{x+3}=x-1\Leftrightarrow \heva{&2x-3=x^2+2x-3\\&x\neq -3}\Leftrightarrow \heva{&x^2=0\\&x\neq -3}\Leftrightarrow x=0\Rightarrow y=-1.$$
}
\end{ex}
%%==========Câu 16
\begin{ex}%[Sơn Bùi - DA1]%[2D1Y5-4]
Đồ thị hàm số $y=x^3+2x^2-x+1$ và đồ thị hàm số $y=x^2-x+3$ có bao nhiêu điểm chung phân biệt?
\choice
{Có $2$ điểm chung}
{Không có điểm chung}
{Có $3$ điểm chung}
{\True Có $1$ điểm chung}
\loigiai{Phương trình hoành độ giao điểm
    $$x^3+2x^2-x+1=x^2-x+3\Leftrightarrow x^3+x^2-2=0\Leftrightarrow x=1.$$
    Vậy hai đồ thị đã cho có $1$ điểm chung.
}
\end{ex}
%%==========Câu 17
\begin{ex}%[Sơn Bùi - DA1]%[2D1Y5-4]
Biết rằng đồ thị các hàm số $y=x^3+\dfrac{5}{4}x-2$ và $y=x^2+x-2$ tiếp xúc nhau tại điểm $M(x_0;y_0)$. Tìm $x_0$.
\choice
{$x_0=\dfrac{3}{2}$}
{\True $x_0=\dfrac{1}{2}$}
{$x_0=\dfrac{-5}{2}$}
{$x_0=\dfrac{3}{4}$}
\loigiai{Hệ điều kiện tiếp xúc
    $$\heva{&x^3+\dfrac{5}{4}x-2=x^2+x-2\\&3x^2+\dfrac{5}{4}=2x+1}\Leftrightarrow x=\dfrac{1}{2}.$$
    Vậy đồ thị hàm số đã cho tiếp xúc trục hoành tại điểm có hoành độ $x_0=\dfrac{1}{2}.$
}
\end{ex}
%%==========Câu 18
\begin{ex}%[Sơn Bùi - DA1]%[2D1Y5-4]
Số giao điểm của đồ thị hàm số $y=x^4-7x^2-6$ và $y=x^3-13x$ là bao nhiêu?
\choice
{$1$}
{$2$}
{\True $3$}
{$4$}
\loigiai{Phương trình hoành độ giao điểm
    $$x^4-7x^2-6=x^3-13x\Leftrightarrow x^4-x^3-7x^2+13x-6=0\Leftrightarrow \hoac{&x=1\\&x=2.}$$
    Vậy hai đồ thị đã cho có $2$ giao điểm.
}
\end{ex}
%%==========Câu 19
\begin{ex}%[Sơn Bùi - DA1]%[2D1Y5-4]
Đồ thị hàm số $y=x^4-2x^2+2$ và đồ thị của hàm số $y=-x^2+4$ có tất cả bao nhiêu điểm chung phân biệt?
\choice
{$0$}
{$4$}
{$1$}
{\True $2$}
\loigiai{Phương trình hoành độ giao điểm
    $$x^4-2x^2+2=-x^2+4\Leftrightarrow x^4-x^2-2=0\Leftrightarrow \hoac{&x^2=2\\&x^2=-1\text{ (vô nghiệm)}}\Leftrightarrow\hoac{&x=\sqrt{2}\\&x=-\sqrt{2}.}$$
    Vậy đồ thị hai hàm số đã cho có $2$ điểm chung phân biệt.
}
\end{ex}
%%==========Câu 20
\begin{ex}%[Sơn Bùi - DA1]%[2D1Y5-4]
Biết đường thẳng $y=2x+4$ cắt đồ thị hàm số $y=x^3+x^2-4$ tại điểm duy nhất $(x_0;y_0)$. Tìm $x_0+y_0$.
\choice
{$x_0+y_0=6$}
{$x_0+y_0=2$}
{\True $x_0+y_0=10$}
{$x_0+y_0=8$}
\loigiai{Phương trình hoành độ giao điểm
    $$x^3+x^2-4=2x+4\Leftrightarrow x_0=2\Rightarrow y_0=8.$$
    Vậy $x_0+y_0=2+8=10$.
}
\end{ex}
%%==========Câu 21
\begin{ex}%[Sơn Bùi - DA1]%[2D1Y5-4]
Parabol $y=x^2+2x$ cắt đường cong $y=-x^3+3x^2+2x-1$ tại bao nhiêu điểm?
\choice
{$1$}
{$2$}
{\True $3$}
{$0$}
\loigiai{Phương trình hoành độ giao điểm
    $$x^2+2x=-x^3+3x^2+2x-1\Leftrightarrow \hoac{&x=1\\&x=\dfrac{1\pm\sqrt{5}}{2}.}$$
    Vậy hai đồ thị đã cho cắt nhau tại 3 điểm.
}
\end{ex}
%%==========Câu 22
\begin{ex}%[Sơn Bùi - DA1]%[2D1Y5-4]
Đồ thị hàm số nào sau đây cắt trục tung tại điểm có tung độ âm?
\choice
{$y=\dfrac{4x+1}{x+2}$}
{\True $y=\dfrac{3x+4}{x-1}$}
{$y=\dfrac{-2x+3}{x+1}$}
{$y=\dfrac{2x-3}{3x-1}$}
\loigiai{Đồ thị hàm số $y=\dfrac{3x+4}{x-1}$ cắt trục tung tại điểm có tung độ $y=-4<0$.
}
\end{ex}
%%==========Câu 23
\begin{ex}%[Sơn Bùi - DA1]%[2D1Y5-4]
Số giao điểm của đồ thị hàm số $y=\dfrac{x^2-2x+3}{x-1}$ và đường thẳng $y=3x-6$ là
\choice
{$3$}
{$0$}
{$1$}
{\True $2$}
\loigiai{Phương trình hoành độ giao điểm
    $$\dfrac{x^2-2x+3}{x-1}=3x-6\Leftrightarrow\heva{&x^2-2x+3=x^2-7x+3\\&x\neq 1}\Leftrightarrow x=\dfrac{7\pm\sqrt{37}}{2}.$$
    Vậy hai đồ thị đã cho có $2$ giao điểm.
}
\end{ex}
%%==========Câu 24
\begin{ex}%[Sơn Bùi - DA1]%[2D1Y5-4]
Đồ thị của hàm số $y=x^4-x^2+2$ và đồ thị của hàm số $y=4$ có bao nhiêu điểm chung?
\choice
{$0$}
{$1$}
{$4$}
{\True $2$}
\loigiai{Phương trình hoành độ giao điểm
    $$x^4-x^2+2=4\Leftrightarrow \hoac{&x^2=2\\&x^2=-1\text{ (vô nghiệm )}}\Leftrightarrow x=\pm\sqrt{2}.$$
    Vậy hai đồ thị đã cho có $2$ điểm chung.
}
\end{ex}
%%==========Câu 25
\begin{ex}%[Sơn Bùi - DA1]%[2D1Y5-4]
Số giao điểm của đường cong $y=x^4+5x^2-2$ và trục hoành là bao nhiêu?
\choice
{\True $2$}
{$0$}
{$4$}
{$3$}
\loigiai{Phương trình hoành độ giao điểm
    $$x^4+5x^2-2=0\Leftrightarrow \hoac{&x^2=\dfrac{-5-\sqrt{33}}{2}\\&x^2=\dfrac{-5+\sqrt{33}}{2}\text{ (vô nghiệm)}}\Rightarrow x=\pm\sqrt{\dfrac{-5+\sqrt{33}}{2}}.$$
    Vậy đồ thị hàm số đã cho cắt trục hoành tại $2$ điểm.
}
\end{ex}
\begin{ex}%[Sơn Bùi - DA1]%[2D1B5-4]
Tìm tất cả giá trị của tham số $m$ để đồ thị hàm số $y=\dfrac{2x+m}{x+1}$ cắt đường thẳng $y=1-x$ tại $2$ điểm phân biệt.
\choice
{$m\in(-\infty;2]$}
{\True $m\in(-\infty;2)$}
{$m\in(-\infty;-2)$}
{$m\in(2;+\infty)$}
\loigiai{Phương trình hoành độ giao điểm
    $$\dfrac{2x+m}{x+1}=1-x\Leftrightarrow \heva{&2x+m=1-x^2\\&x\neq -1}\Leftrightarrow\heva{&x^2+2x+m-1=0\quad(1)\\&x\neq -1.}$$
    Hai đồ thị đã cho cắt nhau tại hai điểm phân biệt $\Leftrightarrow (1)$ có hai nghiệm phân biệt khác $-1$\\
    $\Leftrightarrow\heva{&\Delta=4-4(m-1)>0\\&m\neq 2}\Leftrightarrow m<2.$}
\end{ex}
%%==========Câu 28
\begin{ex}%[Sơn Bùi - DA1]%[2D1B5-4]
Tìm tất cả giá trị của tham số $m$ để đồ thị $y=\dfrac{x}{x-1}$ cắt đường thẳng $y=-x+m$ tại hai điểm phân biệt.
\choice
{$1<m<4$}
{$m<0$ hoặc $m>2$}
{\True $m<0$ hoặc $m>4$}
{$m<1$ hoặc $m>4$}
\loigiai{Phương trình hoành độ giao điểm
    $$\dfrac{x}{x-1}=-x+m\Leftrightarrow \heva{&x^2+-mx+m=0\quad (1)\\&x\neq 1.}$$
    Hai đồ thị đã cho cắt nhau tại hai điểm phân biệt $\Leftrightarrow (1)$ có hai nghiệm phân biệt khác $1$\\
    $\Leftrightarrow \heva{&\Delta=m^2-4m>0\\&1-m+m\neq 0}\Leftrightarrow \hoac{&m>4\\&m<0.}$
}
\end{ex}
%%==========Câu 29
\begin{ex}%[Sơn Bùi - DA1]%[2D1B5-4]
Cho hàm số $y=\dfrac{2x+3}{x+2}$ có đồ thị $(\mathrm{C})$ và đường thẳng $d\colon y=x+m$. Tìm tất cả giá trị của tham số $m$ để $d$ cắt $(\mathrm{C})$ tại hai điểm phân biệt.
\choice
{$m>2$}
{$m<6$}
{$m=2$}
{\True $m<2$ hoặc $m>6$}
\loigiai{Phương trình hoành độ giao điểm
    $$\dfrac{2x+3}{x+2}=x+m\Leftrightarrow\heva{&2x+3=(x+m)(x+2)\\&x\neq -2}\Leftrightarrow \heva{&x^2+mx+2m-3=0\quad (1)\\&x\neq -2.}$$
    Hai đồ thị đã cho cắt nhau tại hai điểm phân biệt $\Leftrightarrow (1)$ có hai nghiệm phân biệt khác $-2$\\
    $\Leftrightarrow\heva{&\Delta=m^2-4(2m-3)>0\\&(-2)^2-2m+2m-3\neq 0}\Leftrightarrow\hoac{&m>6\\&m<2.}$
}
\end{ex}
%%==========Câu 30
\begin{ex}%[Sơn Bùi - DA1]%[2D1K5-4]
Tìm tất cả giá trị thực của tham số $m$ để đồ thị hàm số $y=(x-m)\left(2x^2+x-3m\right)$ cắt trục hoành tại $3$ điểm phân biệt.
\choice
{$m\in\mathbb{R}\setminus\{0;1\}$}
{$m\in\left(-\infty;\dfrac{1}{24}\right)\setminus\{0;1\}$}
{\True $m\in\left(-\dfrac{1}{24};+\infty\right)\setminus\{0;1\}$}
{$m\in\left(-\dfrac{1}{24};+\infty\right)$}
\loigiai{Đồ thị hàm số đã cho cắt trục hoành tại 3 điểm phân biệt\\
    $\Leftrightarrow (x-m)\left(2x^2+x-3m\right)=0$ có 3 nghiệm phân biệt\\
    $\Leftrightarrow 2x^2+x-3m=0$ có 2 nghiệm phân biệt khác $m$\\
    $\Leftrightarrow \heva{&\Delta=1+24m>0\\&2m^2-2m\neq 0}\Leftrightarrow\heva{&m>\dfrac{-1}{24}\\&m\neq 0\\&m\neq 1.}$
}
\end{ex}
\Closesolutionfile{ans}