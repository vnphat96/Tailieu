\begin{dang}{Tìm $m$ đểS bậc ba có cực trị thỏa yêu cầu cho trước}
    \begin{itemize}
        \item Tìm $m$ để hàm số bậc ba $y=ax^3+bx^2+cx+d$ đạt cực trị tại $x=x_0$.
%        \begin{multicols}{2}
            \begin{itemize}
                \item 	$x=x_0$ là điểm cực đại $\Leftrightarrow \heva{& y'(x_0)=0 \\ & y''(x_0)<0}$
                \item 	$x=x_0$ là điểm cực tiểu $\Leftrightarrow \heva{& y'(x_0)=0 \\ & y''(x_0)>0}$
            \end{itemize}
%        \end{multicols}
%        \item Tìm $m$ để hàm bậc ba, hàm trùng phương có điểm cực trị thỏa yêu cầu sau
%        \begin{itemize}
%            \item Hàm bậc ba $y=ax^3+bx^2+cx+d$ có 2 cực trị (1 CĐ và 1CT) $\Leftrightarrow $ $ b^2-3ac>0$.
%            \item Hàm bậc ba $y=ax^3+bx^2+cx+d$ không có cực trị  $\Leftrightarrow $ $ b^2-3ac\le 0$.
%            \item Hàm trùng phương $y=ax^4+bx^2+c$ có 3 cực trị $\Leftrightarrow$ $ab<0$.
%            \item Hàm trùng phương $y=ax^4+bx^2+c$ có 1 cực trị $\Leftrightarrow$ $ab\ge0$.
%        \end{itemize}
    \end{itemize}
\end{dang}
%\dotlineans{34}{vd}
\begin{vd}
    Tìm $m$ để hàm số
    \begin{listEX}[1]
        \item  $ y=\dfrac{x^3}{3}+mx^2+(m^2-4)x+2 $ đạt cực đại tại $ x=1 $.
        \item $ y=x^3-2mx^2+m^2x+2 $ đạt cực tiểu tại $ x=1 $.
    \end{listEX}
    \loigiai{
        \begin{listEX}
            \item $ y'=x^2+2mx+(m^2-4) $ nên $ y'(1)=1+2m+m^2-4=m^2+2m-3 $.\\
            Hàm số đạt cực đại tại $ x=1 \Rightarrow y'(1)=0\Leftrightarrow m^2+2m-3=0 \Leftrightarrow m=1$ hoặc $ m=-3 $.\\
            $ y''=2x+2m \Rightarrow y''(1)=2+2m$.\\
            Với $ m=1 $ thì $ y''(1)=4>0 $ nên $ x=1 $ là điểm cực tiểu (loại).\\
            Với $ m=-3 $ thì $ y''(1)=-4<0 $ nên $ x=1 $ là điểm cực đại (nhận).\\
            Vậy $ m=-3 $ thỏa yêu cầu bài toán.
            \item $ y'=3x^2-4mx+m^2 $ nên $ y'(1)=3-4m+m^2 $.\\
            Hàm số đạt cực tiểu tại $ x=1 \Rightarrow y'(1)=0 \Leftrightarrow 3-4m+m^2=0 \Leftrightarrow m=1$ hoặc $ m=3 $.\\
            $ y''=6x-4m \Rightarrow y''(1)=6-4m$.\\
            Với $ m=1 $ thì $ y''(1)=2>0 $ nên $ x=1 $ là điểm cực tiểu (nhận).\\
            Với $ m=3 $ thì $ y''(1)=-6<0 $ nên $ x=1 $ là điểm cực đại (loại).\\
            Vậy $ m=1 $ thỏa yêu cầu bài toán.
        \end{listEX}
    }
\end{vd}
%\dotlineans{36}{vd}
\begin{vd}%[2D1B2-7]
    Tìm $m$ để hàm số
    \begin{listEX}[2]
%        \item $ y=x^4-(3m-1)x^2+m-2 $ có ba điểm cực trị.
%        \item $ y=x^4-(m-2)x^2 $ có đúng một điểm cực trị.
        \item $ y=x^3-mx^2+2mx-1 $ không có cực trị.
        \item 	$ y=x^3-2mx^2+mx-1 $ có cực trị.
    \end{listEX}
    \loigiai{
        \begin{listEX}
%            \item 	$ y'=3x^2-4mx+m $ có $ \Delta'=4m^2-3m $.\\
%            Hàm số bậc 3 có cực trị $ \Leftrightarrow y'=0$ có hai nghiệm phân biệt $\Leftrightarrow 4m^2-3m>0 \Leftrightarrow m<0 $ hoặc $ m>\dfrac{3}{4} $.
%            \item Xét $ m=0 $. Khi đó $y=x^2+x-1$ là hàm bậc hai chỉ có 1 cực trị nên loại $ m=0 $.\\
%            Xét $ m \ne 0 $.\\
%            $ y'=mx^2-2(m-1)x+(m+1) $ có $ \Delta'=(m-1)^2-m(m+1)=-3m+1 $.\\
%            Hàm số bậc 3 có cực trị $ \Leftrightarrow y'=0$ có hai nghiệm phân biệt $\Leftrightarrow -3m+1>0 \Leftrightarrow m<\dfrac{1}{3}$.
            \item $ y'=3x^2-2mx+2m $ có $ \Delta'=m^2-6m $.\\
            Hàm bậc 3 không có cực trị $ \Leftrightarrow y'=0 $ vô nghiệm hoặc có nghiệm kép $ \Leftrightarrow m^2-6m \le 0 \Leftrightarrow 0<m<6 $.
            \item 	Hàm trùng phương có 3 cực trị $ \Leftrightarrow ab<0 \Leftrightarrow -(3m-1)<0 \Leftrightarrow m>\dfrac{1}{3}$.
            \item Hàm trùng phương có 1 cực trị $ \Leftrightarrow ab \ge 0 \Leftrightarrow -(m-2) \ge 0 \Leftrightarrow m \le 2$.
        \end{listEX}
    }
\end{vd}
\begin{vd}%[2D1B2-4]
    Tìm $m$ để hàm số $ y=\dfrac{2}{3}x^3-mx^2-2(3m^2-1)x+\dfrac{2}{3} $ có hai điểm cực trị $ x_1 $, $ x_2 $ sao cho $ x_1x_2+2(x_1+x_2)=1 $.
    \loigiai{$ y'=2x^2-2mx-2(3m^2-1) $ có $ \Delta'=m^2+4(3m^2-1)=13m^2-4$.\\
        Hàm số có 2 điểm cực trị $ \Leftrightarrow y'=0 $ có 2 nghiệm phân biệt $ \Leftrightarrow 13m^2-4>0 \Leftrightarrow m<-\dfrac{2\sqrt{13}}{13} $ hoặc $ m>\dfrac{2\sqrt{13}}{13} $.\\
        Hai điểm cực trị $ x_1 $, $ x_2 $ cũng là hai nghiệm của $ y'=0 $ nên theo Viete $ x_1+x_2=-(3m^2-1) $ và $ x_1x_2=m $.\\
        $ x_1x_2+2(x_1+x_2)=1 \Leftrightarrow m-2(3m^2-1)-1=0 \Leftrightarrow -6m^2+m=0 \Leftrightarrow m=0 $ (loại) hoặc $ m=6 $ (nhận).\\
        Vậy $ m=6 $ thỏa yêu cầu bài toán.
    }
\end{vd}
\BTTN
\Opensolutionfile{ans}[ans/2D1-2-DANG-2]
\begin{ex}%[2D1B2-3]
    Cho hàm số $y=\dfrac{1}{3}x^3-m x^2+\left(m^2-4\right)x+3$. Giá trị của tham số $m$ để hàm số đạt cực đại tại $x=3$ là
    \choice
    {$m=1$}
    {$m=-1$}
    {\True $m=5$}
    {$m=-7$}
    \loigiai
    {
        Tập xác định $\mathscr D=\mathbb{R}$.\\
        Ta có $y'=x^2-2mx+(m^2-4)$ và $y''=2x-2m$.\\
        Để hàm số đã cho đạt cực đại tại điểm $x=3$ thì $y'(3)=0
        \Leftrightarrow m^2-6m+5=0 \Leftrightarrow\hoac{& m=1\\& m=5.}$\\
        Thử lại:
        \begin{itemize}
            \item Với $m=1$ thì $y''=2x-2$, suy ra $y''(3)=4>0$.\\
            Nên $x=3$ là điểm cực tiểu của hàm số đã cho (không thỏa mãn).\\
            $\Rightarrow m=1$ (loại).
            \item Với $m=5$ thì $y''=2x-10$, suy ra $y''(3)=-4<0$.\\
            Nên $x=3$ là điểm cực đại của hàm số đã cho (thỏa mãn).\\
            $\Rightarrow m=5$ (nhận).
        \end{itemize}
    }
\end{ex}
\begin{ex}%[2D1B2-3]
    Tất cả các giá trị của tham số $m$ để hàm số $y=x^3-m x^2+(2m-3)x-3$ đạt cực đại tại $x=1$ là
    \choice
    {$m=3$}
    {$m<3$}
    {\True $m>3$}
    {$m\le 3$}
    \loigiai
    {
        Ta có $y'=3x^2-2mx+2m-3$.\\
        \textbf{Nhận xét:} Hàm số đã cho là hàm bậc ba. Do đó hàm số đạt cực đại tại $x=1$
        $$\Leftrightarrow\heva{& y'(1)=0 \\ & y''(1)<0} \Leftrightarrow \heva{& 3\cdot1^2-2m\cdot 1+2m-3=0 \\ & 6\cdot1-2m<0}\Leftrightarrow m>3.$$
    }
\end{ex}
\begin{ex}%[2D1B2-3]
    Cho hàm số $y=x^3+2m x^2+m^2x-3$. Với giá trị nào của $m$ thì hàm số đạt cực tiểu tại $x=1$?
    \choice
    {\True $m=-1$}
    {$m=-3$}
    {$m=1$}
    {$m=3$}
    \loigiai
    {
        Ta có $y'=3x^2+4mx+m^2$, $y''=6x+4m$.\\
        Hàm số đạt cực tiểu tại $x=1\Rightarrow y'(1)=0\Leftrightarrow \hoac{& m=-1 \\ & m=-3.}$\\
        Với $m=-1$, ta có $y''(1)=2>0$, do đó hàm số đạt cực tiểu tại $x=1$. Vậy nhận $m=-1$.\\
        Với $m=-3$, ta có $y''(1)=-6<0$, do đó hàm số đạt cực đại tại $x=1$. Vậy loại $m=-3$.
    }
\end{ex}
\begin{ex}%[2D1B2-4]
    Biết $M(0;2)$, $N(2;-2)$ là các điểm cực trị của đồ thị hàm số $y=a x^3+b x^2+c x+d$. Giá trị của hàm số tại $x=-2$ bằng
    \choice
    {$2$}
    {$22$}
    {$6$}
    {\True $-18$}
    \loigiai
    {
        \begin{itemize}
            \item Ta có $y' = 3ax^2 + 2bx + c$.
            Vì $M, \; N$ là hai điểm cực trị của đồ thị hàm số, nên ta sẽ tìm được hệ phương trình sau
            \[\left\{
            \begin{array}{rlrlrlll}
                0^3.a & +& 0^2.b & +& 0.c & +& d & = 2\\
                2^3.a & +& 2^2.b & +& 2.c & +& d & = -2\\
                3.0^2.a & +& 2.0.b & +& c & & & = 0\\
                3.2^2.a & +& 2.2.b & +& c & & & = 0\\
            \end{array}
            \right. \tag{1} \]
            Giải hệ phương trình $(1)$, ta có các nghiệm $a= 1, \; b=-3, \; c=0, \; d=2$.\\
            Do đó, hàm số cần tìm là $y=x^3 - 3x^2 + 2$.
            \item Từ đó, ta tính được giá trị của hàm số tại $x=-2$ là $y(-2)=-18$.
        \end{itemize}
    }
\end{ex}
\begin{ex}%[2D1B2-7]
    Tập hợp tất cả các giá trị thực của tham số $ m $ để hàm số $ y=x^3+x^2-mx+3 $ có hai điểm cực trị là
    \choice
    {$\left(-\infty;-\dfrac{1}{3}\right]$}
    {$\left[-\dfrac{1}{3};+\infty \right)$}
    {$\left(-\infty;-\dfrac{1}{3}\right)$}
    {\True $\left(-\dfrac{1}{3};+\infty\right)$}
    \loigiai{
        $ y'=3x^2+2x-m $ có $ \Delta'=1+3m$.\\
        Hàm số bậc 3 có 2 điểm cực trị $ \Leftrightarrow y'=0$ có 2 nghiệm phân biệt $ \Leftrightarrow 1+3m>0 \Leftrightarrow m>-\dfrac{1}{3} $.
    }
\end{ex}
\begin{ex}%[2D1B2-7]
    Có tất cả bao nhiêu giá trị nguyên của tham số $ m $ để hàm số $ y=\dfrac{1}{3}x^3+mx^2-4mx+3 $ không có cực trị?
    \choice
    {$0$}
    {\True $5$}
    {Vô số}
    {$3$}
    \loigiai{$ y'=x^2+2mx-4m $ có $ \Delta'=m^2+4m $.\\
        Hàm số bậc 3 không có cực trị $ \Leftrightarrow y'=0 $ vô nghiệm hoặc có nghiệm kép $ \Leftrightarrow m^2+4m \le 0 \Leftrightarrow -4 \le m \le 0 $ (có 5 giá trị nguyên).
    }
\end{ex}
\begin{ex}%[2D1K2-4]
    Cho hàm số $y = x^3 - 3x^2 +(m+1)x + 2$. Có bao nhiêu giá trị nguyên của $m \in (-10;10)$ để hàm số có 2 điểm cực trị?
    \choice
    {$9$}
    {$10$}
    {\True $11$}
    {$12$}
    \loigiai{
        Tập xác định $\mathscr{D}=\mathbb{R}$.\\
        $y’= 3x^2-6 x+m+1$.\\
        Hàm số đã cho có hai điểm cực trị \\
        $\Leftrightarrow y' = 3x^2-6 x+m+1 =0$ có hai nghiệm phân biệt
        \[\Leftrightarrow \heva{&a = 3 \neq 0\\ &\Delta = (-6)^2 - 4\cdot 3\cdot (m+1) >0} \Leftrightarrow 24-12 m >0 \Leftrightarrow  m <2.\]\\
        Vì $m\in \mathrm{Z}$ và $m\in (-10;10)$ nên $m\in \{-9;-8;-7;-6;-5;-4;-3;-2;-1;0;1\}$.}
\end{ex}
\begin{ex}%[2D1K2-4]
    Cho hàm số $y = x^3 - 3x^2 +2 mx + m$. Có bao nhiêu giá trị nguyên của $m$ thỏa mãn $|m|< 5$ sao cho
    hàm số có cực trị?
    \choice
    {$5$}
    {\True $6$}
    {$9$}
    {$8$}
    \loigiai{
        Tập xác định $\mathscr{D}=\mathbb{R}$.\\
        $y’= 3x^2-6 x+2m$.\\
        Hàm số đã cho có hai điểm cực trị \\
        $\Leftrightarrow y' = 3x^2-6 x+2m =0$ có hai nghiệm phân biệt
        \[\Leftrightarrow \heva{&a = 3 \neq 0\\ &\Delta = (-6)^2 - 4\cdot 3\cdot 2m >0} \Leftrightarrow 36-24 m >0 \Leftrightarrow  m <\dfrac{3}{2}.\]\\
        Vì $m\in \mathrm{Z}$ và $|m|< 5$ nên $m\in \{-4;-3;-2;-1;0;1\}$.}
\end{ex}
\begin{ex}%[2D1K2-4]
    Tất cả các giá trị của tham số $m$ sao cho hàm số  $y =  m x^3 - 3 m x^2 +  3 x + 1$  có điểm cực đại nằm bên trái điểm cực tiểu là
    \choice
    {$0 < m < 1$}
    {\True $m > 1$}
    {$  m <1$}
    {$m < 0 \vee m>1 $}
    \loigiai{
        Tập xác định $\mathscr{D}=\mathbb{R}$.\\
        $y’=  3m x^2 - 6m x+  3 $.\\
        Để hàm số có điểm cực đại nằm bên trái điểm cực tiểu, bảng biến thiên có dạng
        \begin{center}
            \begin{tikzpicture}
                \tkzTabInit[lgt=1,espcl=2]%
                {$x$/1,%
                    $y'$ /1,%
                    $y$/2}%
                {$-\infty$ , $x_1$ , $x_2$ , $+\infty$}%
                \tkzTabLine{ ,+, 0 ,-, 0 ,+, }
                %\tkzTabSlope{1//+\infty,3/-1 /+1}
                \tkzTabVar %
                {- / $-\infty$ ,
                    + / $y_{\text{CĐ}}$ ,
                    - / $y_{\text{CT}}$ ,
                    + / $+\infty$ }
            \end{tikzpicture}
        \end{center}
        Từ bảng biến thiên suy ra
        \[ \heva{&a = 3m > 0\\ &\Delta' = (-3m)^2 -      3  \cdot 3 m > 0}
        \Leftrightarrow \heva{&m > 0 \\ &  m < 0\vee m>1}\Leftrightarrow  m>1.\]\\
        Vậy $m>1$.}
\end{ex}
\begin{ex}%[2D1B2-4]
    Cho hàm số $y=x^3-3x^2+m x-1$ có hai cực trị $x_1$, $x_2$ thỏa mãn $x_1^2+x_2^2=6$. Khi đó $m$ bằng
    \choice
    {\True $-1$}
    {$3$}
    {$1$}
    {$-3$}
    \loigiai
    {
        Ta có $y'=3x^2-6x+m$. Hàm số có hai điểm cực trị khi $\Delta>0\Leftrightarrow m<3$.\\
        Theo định lí Vi-ét ta có $\heva{& x_1+x_2=2 \\ & x_1x_2=m.}$\\
        Ta có $x_1^2+x_2^2=6\Leftrightarrow (x_1+x_2)^2-2x_1x_2=6\Leftrightarrow 2^2-2m=6\Leftrightarrow m=-1$ (thỏa mãn).
    }
\end{ex}
\begin{ex}%[2D1K2-4]
    Tìm các giá trị của tham số $ m $ để đồ thị hàm số $ y = x^3-3mx^2+2 $ có 2
    điểm cực trị $ A $ và $ B $ sao cho $ 3 $ điểm $ A $, $ B $ và $ M(1; -2) $
    thẳng hàng ?
    \choice
    {$ m = \sqrt{2} $}
    {$ m = -\sqrt{2} $}
    {$ m=2 $}
    {\True $m= \pm \sqrt{2} $}
    \loigiai{
        Ta có
        $$ y' = 3x^2-6mx \Rightarrow y' = 0 \Leftrightarrow \hoac{& x = 0 \\
            & x = 2m} $$
        Để hàm số có hai cực trị thì phương trình $ y'=0 $ có $ 2 $ nghiệm phân
        biệt $ \Leftrightarrow m \neq 0 $. \hfill$(*)$ \\
        Tọa độ hai điểm cực trị là $ A(0;2) $, $ B(2m;-4m^3+2) $.\\
        Khi đó $ A $, $ B $, $ M $ thẳng hàng $ \Leftrightarrow \exists k \neq
        0 $: $ \vec{AM} = k\vec{AB}  \Leftrightarrow \heva{& 2km = 1
            \\ & -4km^3 = -4} \Leftrightarrow m = \pm \sqrt{2}$.\\
        Kết hợp với $ (*) $ suy ra $ m = \pm \sqrt{2}$ thỏa mãn
        điều kiện bài toán.
    }
\end{ex}
\begin{ex}%[2D1B2-3]
    Với giá trị nào của tham số $m$ thì hàm số $y=x^3-mx^2+(2m-3)x-3$ đạt cực đại tại $x=1$?
    \choice
    {$m\le 3$}
    {$m=3$}
    {$m<3$}
    {\True $m>3$}
    \loigiai{
        Tập xác định $\mathscr{D}=\mathbb{R}$.\\
        Ta có $y'=3x^2-2mx+(2m-3)$ và $y''=6x-2m$.\\
        Hàm số đạt cực đại tại $x=1$ khi và chỉ khi
        $$\heva{&y'(1)=0 \\ &y''(1)<0}\Leftrightarrow \heva{&3-2m+2m-3=0 \\ &6-2m<0} \Leftrightarrow m>3.$$
    }
\end{ex}


\begin{ex}%[2D1B2-3]
    Biết điểm $M(0;4)$ là điểm cực đại của đồ thị hàm số $f(x)=x^3+ax^2+bx+a^2$. Tính $f(3)$.
    \choice
    {$f(3)=17$}
    {$f(3)=49$}
    {$f(3)=34$}
    {\True $f(3)=13$}
    \loigiai{
        Ta có $f'(x)=3x^2+2ax+b$ và $f''(x)=6x+2a$.\\
        Vì $M(0;4)$ là điểm cực đại của đồ thị hàm số nên $\heva{&f(0)=4\\&f'(0)=0\\&f''(0)<0}\Rightarrow\heva{&a^2=4\\&b=0\\&a<0}\Rightarrow\heva{&a=-2\\&b=0.}$ \\
        Suy ra $ f(x)=x^3-2x^2+4 $. Vậy $f(3)=13$.}
\end{ex}

\begin{ex}%[2D1B2-3]
    Giả $a$, $b$, $c$ là các số thực thỏa mãn đồ thị hàm số $y=x^3+ax^2+bx+c$ đi qua điểm $(1; 0)$ và có điểm cực trị $(-2; 0)$. Tính giá trị biểu thức $T=a^2+b^2+c^2$.
    \choice
    {\True $25$}
    {$-1$}
    {$7$}
    {$14$}
    \loigiai{
        Ta có $y'=3x^2+2ax+b$.\\
        Đồ thị hàm số $y=x^3+ax^2+bx+c$ đi qua điểm $(1; 0)$ nên $a+b+c=-1$.\\
        Đồ thị hàm số có điểm cực trị $(-2; 0)$ nên $\heva{&4a-2b+c=8\\&y'(-2)=0}\Leftrightarrow\heva{&4a-2b+c=8\\&-4a+b=-12.}$ \\
        Xét hệ phương trình $\heva{&a+b+c=-1\\&4a-2b+c=8\\&-4a+b=-12}\Leftrightarrow\heva{&a=3\\&b=0\\&c=-4.}$ \\
        Vậy $T=a^2+b^2+c^2 =25$.}
\end{ex}

\begin{ex}%[2D1B2-3]
    Tìm tất cả các giá trị thực của tham số $m$ để hàm số $y=mx^3+x^2+\left(m^2-6\right)x+1$ đạt cực tiểu tại $x=1$.
    \choice
    {\True $m=1$}
    {$m=-4$}
    {$m=-2$}
    {$m=2$}
    \loigiai{
        Ta có $y'=3mx^2+2x+m^2-6$ suy ra $y''=6mx+2$.\\
        Hàm số đạt cực tiểu tại $x=1\Leftrightarrow\heva{&f'(1)=0\\&f''(1)>0}$ \\
        $ \Leftrightarrow\heva{&3m+2+m^2-6=0\\&6m+2>0}\Leftrightarrow\heva{&m^2+3m-4=0\\&6m+2>0}\Leftrightarrow m=1 $.}
\end{ex}
\begin{ex}%[2D1B2-3]
    Tìm tất cả các giá trị thực của tham số $m$ để hàm số $y=-x^3-2x^2+mx+1$ đạt cực tiểu tại $x=-1$.
    \choice
    {$m<-1$}
    {$m>-1$}
    {$m \ne -1$}
    {\True $m=-1$}
    \loigiai{
        Ta có $y'=-3x^2-4x+m$.\\
        Hàm số đạt cực tiểu tại $x=-1$, suy ra $y'(-1)=0 \Leftrightarrow 1+m=0 \Leftrightarrow m=-1$.\\
        Mặt khác $y''(-1)=2 >0$.\\
        Vậy hàm số đạt cực tiểu tại $x=-1$ khi và chỉ khi $m=-1$.
    }
\end{ex}
\begin{ex}%[2D1B2-4]
    Tìm tất cả các giá trị của tham số $m$ để hàm số $y=-x^3-3x^2+mx+2$ có cực đại và cực tiểu.
    \choice
    {$m\geq 3$}
    {$m\geq -3$}
    {$m>3$}
    {\True $m>-3$}
    \loigiai{
        Ta có $y'=-3x^2-6x+m$. Hàm số đã cho có cực đại và cực tiểu khi và chỉ khi phương trình $y'=0$ có $2$ nghiệm phân biệt $\Leftrightarrow \Delta'>0\Leftrightarrow 9+3m>0\Leftrightarrow m>-3$.
    }
\end{ex}
\begin{ex}%[2D1B2-4]
    Số các giá trị nguyên của tham số $m$ để hàm số $y=x^3-mx^2+\left(m^2-2m\right)x$ có cực tiểu là
    \choice
    {$0$}
    {\True $2$}
    {$1$}
    {$3$}
    \loigiai{
        Ta có $y'=3x^2-2mx+m^2-2m$.\\
        Hàm số đã cho có cực tiểu $\Leftrightarrow$ Phương trình $y'=0$ có hai nghiệm phân biệt $$\Leftrightarrow \Delta'=m^2-3\left(m^2-2m\right)>0\Leftrightarrow -2m^2+6m>0\Leftrightarrow 0<m<3.$$
        Do $m$ là số nguyên nên $m=1$ và $m=2$.\\
        Vậy có hai giá trị nguyên của tham số $m$.
    }
\end{ex}

\begin{ex}%[2D1B2-4]
    Cho hàm số $y=x^3-3(m+1)x^2+3(7m-3)x$. Số giá trị nguyên của tham số $m$ để hàm số không có cực trị là
    \choice
    {$1$}
    {\True $4$}
    {$2$}
    {$3$}
    \loigiai{
        Hàm số bậc $3$ không có cực trị khi và chỉ khi phương trình $y'=0 \Leftrightarrow 3x^2-6(m+1)x+3(7m-3)=0$ có nghiệm kép hoặc vô nghiệm hay
        $$\Delta' \le 0 \Leftrightarrow 9(m+1)^2-9(7m-3)\le 0 \Leftrightarrow m^2-5m+4 \le 0 \Leftrightarrow 1 \le m \le 4.$$
        Mà $m \in \mathbb{Z}$ nên $ m \in \{1;2;3;4\}$.\\
        Vậy có $4$ giá trị nguyên của $m$ thỏa mãn yêu cầu bài toán.
    }
\end{ex}
\begin{ex}%[2D1B2-4]
    Tìm các giá trị của tham số $m$ để hàm số $y=\dfrac{1}{3}{{x}^{3}}+m{{x}^{2}}+(m+6)x-2m-1$ có cực đại và cực tiểu.
    \loigiai{
        Ta có $y'=x^2+2mx+m+6$.
        \begin{eqnarray*}
            \text{Hàm số có cực đại và cực tiểu} &\Leftrightarrow& y'=0 \text{ có hai nghiệm phân biệt}\\
            &\Leftrightarrow& x^2+2mx+m+6=0 \text{ có hai nghiệm phân biệt}\\
            &\Leftrightarrow& \Delta' >0\\
            &\Leftrightarrow& m^2-m-6>0\\
            &\Leftrightarrow& \hoac{&m<-2\\&m>3.}
        \end{eqnarray*}
        Vậy $\hoac{&m<-2\\&m>3}$ thỏa đề.
    }
\end{ex}

\begin{ex}%[2D1B2-4]
    Tìm tất cả các giá trị thực của $m$ để hàm số $y=x^3-3x^2+(m+1)x+2$ có hai điểm cực trị.
    \choice{\True $m<2$}
    {$m\le 2$}
    {$m>2$}
    {$m<-4$}
    \loigiai{
        $y'=3x^2-6x+m+1$, $\Delta'=6-3m$. Để hàm số có hai điểm cực trị thì $y'=0$ phải có hai nghiệm phân biệt tức là $\Delta>0\Leftrightarrow m<2$.
    }
\end{ex}


\begin{ex}%[2D1B2-4]
    Với giá trị nào của tham số $m$ để hàm số $y=\dfrac{1}{3}x^3-\dfrac{1}{2}(1+m)x^2+m x+m$ không có cực trị?
    \choice
    {$m=0$}
    {\True $m=1$}
    {$m=4$}
    {$m=2$}
    \loigiai
    {
        Ta có $y'=x^2-(m+1)x+m$. Hàm số không có cực trị $\Leftrightarrow \Delta<0\Leftrightarrow (m+1)^2-4m\le 0\Leftrightarrow m=1$.
    }
\end{ex}
\Closesolutionfile{ans}
\indapan{10}{ans/2D1-2-DANG-2}