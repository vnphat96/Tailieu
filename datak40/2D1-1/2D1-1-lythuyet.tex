%\section{TÍNH ĐƠN ĐIỆU CỦA HÀM SỐ}
\subsection{TÓM TẮT LÝ THUYẾT}
\subsubsection{Định nghĩa}
\immini{Cho hàm số $y=f(x)$ xác định trên $K$ với $K$ là khoảng hoặc đoạn hoặc nửa khoảng.
\begin{itemize}
    \item Hàm số $y=f(x)$ {\bfseries đồng biến} ({\it tăng}) trên $K$ nếu $\forall x_1$, $x_2\in K$, $x_1<x_2$ thì $f(x_1)<f(x_2)$.
    \item Hàm số $y=f(x)$ {\bfseries nghịch biến} ({\it giảm}) trên $K$ nếu $\forall x_1$, $x_2\in K$, $x_1<x_2$ thì $f(x_1)>f(x_2)$.
\end{itemize}
{\it Hàm số đồng biến hoặc nghịch biến trên $K$ được gọi chung là {\bf\itshape đơn điệu} trên $K$.}
\begin{nx}
Nếu $\forall x_1, x_2 \in K$ và $x_1 \ne x_2$ thì hàm số
\begin{itemize}
\item $f(x)$ đồng biến trên $K \Leftrightarrow \dfrac{f\left(x_2\right)-f\left(x_1\right)}{x_2-x_1}>0$.
\item $f(x)$ nghịch biến trên $K \Leftrightarrow \dfrac{f\left(x_2\right)-f\left(x_1\right)}{x_2-x_1}<0$.
\item Nếu hàm số \textbf{đồng biến} trên $K$ thì \textbf{đồ thị đi lên} từ trái sang phải.
\end{itemize}
\begin{itemize}
\item Nếu hàm số \textbf{nghịch biến} trên $K$ thì \textbf{đồ thị đi xuống} từ trái sang phải
\end{itemize}
\end{nx}
\subsubsection{Tính đơn điệu và dấu của đạo hàm}
\begin{dl}
    Giả sử hàm số $y=f(x)$ có đạo hàm trên khoảng $K$.
    \begin{itemize}
        \item Nếu $f'(x)>0, \forall x \in K$ thì hàm số đồng biến trên khoảng $K$.
        \item Nếu $f'(x)<0, \forall x \in K$ thì hàm số nghịch biến trên khoảng $K$.
        \item Nếu $f'(x)=0, \forall x \in K$ thì hàm số không đổi trên khoảng $K$.
    \end{itemize}
\end{dl}
    }{
\begin{tikzpicture}[>=stealth,scale=1]
    \draw[->] (-.5,0)--(0,0)node[below left]{$O$} --(3,0)node[below]{$x$};
    \draw[->] (0,-.5) -- (0,2) node[right]{$y$};
    \draw
    [postaction={
        decorate,
        decoration={
            raise=2pt,
            text along path,
            text align=center,
            text={{đồ}ng bi{ế}n}
        }
    },line width=1pt
    ]
    plot[domain=0.5:2.5] (\x,{(.39)*(\x)^2-(.41)*(\x)+.75});
    \draw[dashed] (.5,0)node[below]{$a$} -- (.5,.6425) (2.5,0)node[below]{$b$} -- (2.5,2.1625);
    \path (current bounding box.south) node[below=2mm]{
        \begin{tikzpicture}[>=stealth,scale=1]
            \draw[->] (-.5,0)--(0,0)node[below left]{$O$} --(3,0)node[below]{$x$};
            \draw[->] (0,-.5) -- (0,2) node[right]{$y$};
            \draw
            [postaction={
                decorate,
                decoration={
                    raise=2pt,
                    text along path,
                    text align=center,
                    text={ngh{ị}ch bi{ế}n}
                }
            },line width=1pt
            ]
            plot[domain=.5:2.5] (\x,{-.29*(\x)^2+.43*(\x)+1.56});
            \draw[dashed] (.5,0)node[below]{$a$} -- (.5,1.7025) (2.5,0)node[below]{$b$} -- (2.5,.8225);
        \end{tikzpicture}
    };
\end{tikzpicture}
}
\textbf{Định lí mở rộng:} Nếu $f'(x) \ge 0, \forall x \in K$ (hoặc $f'(x) \le 0, \forall x \in K$) và $f'(x)=0$ chỉ tại một số điểm hữu hạn của $K$ thì hàm số đồng biến (nghịch biến) trên khoảng $K$.
\begin{note}
    Nếu $K$ là một đoạn hoặc nửa khoảng thì phải bổ sung giả thiết \lq\lq hàm số $y=f(x)$ liên tục trên đoạn hoặc nửa khoảng đó \rq\rq. Chẳng hạn:\\
    Nếu hàm số $y=f(x)$ liên tục trên $[a;b]$ và có đạo hàm $f'(x)>0, \forall x \in (a;b)$ thì hàm số đồng biến trên đoạn $[a;b]$.
\end{note}
