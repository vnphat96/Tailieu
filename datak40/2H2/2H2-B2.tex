\setcounter{section}{1}
\section{TỌA ĐỘ CỦA VÉC TƠ TRONG KHÔNG GIAN}
\subsection{LÝ THUYẾT CẦN NHỚ}
\subsubsection{Hệ tọa độ trong không gian}
Trong không gian, ba trục $O x$, $O y$, $O z$ đôi một vuông góc với nhau tại gốc $O$ của mỗi trục. Gọi $\vec{i}$, $\vec{j}$, $\vec{k}$ lần lượt là các véc-tơ đơn vị trên các trục $O x$, $O y$, $O z$.
\immini{
	\begin{itemize}
		\item  Hệ ba trục như vậy được gọi là hệ trục toạ độ Descartes vuông góc $Oxyz$, hay đơn giản là hệ toạ độ $Oxyz$. Điểm $O$ được gọi là gốc toạ độ.
		\item  Các mặt phẳng $(O x y)$, $(O y z)$, $(O z x)$ đôi một vuông góc với nhau được gọi là các mặt phẳng toạ độ.
		\item  ${\vec{i}^2} = {\vec{j}^2} = {\vec{k}^2} = 1$ \\
		và $\vec{i} \cdot \vec{j} = \vec{j} \cdot \vec{k} = \vec{k} \cdot \vec{i}  = 0$
	\end{itemize}
}{\hspace{1cm}
	\begin{tikzpicture}[>=stealth,line join=round,line cap=round,scale=1]
		\def\a{3.0}
		\path
		(0,0) coordinate (A1)
		(\a,0) coordinate (A2)
		(\a,\a) coordinate (A3)
		(0,\a) coordinate (A4);
		\foreach \i in {1,...,4}
		\path (A\i)+(45:.75) coordinate (B\i);
		\draw (B1)--(A1) (B1)--(B2) (B1)--(B4);
		%	\draw(A4)--(B4)--(B3)--(B2)--(A2) (A3)--(B3)
		%	(A1)--(A2)--(A3)--(A4)--cycle;
		\draw[-stealth] (B1)--(B2)node[right]{$y$};
		\draw[-stealth] (B1)--(B4)node[above]{$z$};
		\draw[dashed](B1)--+(45:0.85)[dashed](B1)--+(180:0.85)(B1)--+(270:0.85);
		\draw[-stealth] (A1)--+(-135:.95)node[below]{$x$};
		\draw[-stealth,blue] (B1)--+(0:.95)node[above]{$\vec{j}$};
		\draw[-stealth,blue] (B1)--+(90:.85)node[above left]{$\vec{k}$};
		\draw[-stealth,blue] (B1)--+(-135:0.95)node[right]{$\vec{i}$};
		\fill(B1)circle(1pt) node[below right]{$O$};
	\end{tikzpicture}}
Không gian với hệ toạ độ $Oxyz$ còn được gọi là không gian $Oxyz$.
\subsubsection{Tọa độ của điểm}
Trong KG $Oxyz$, cho điểm $M$. Tọa độ điểm $M$ được xác định như sau:
\immini{
	\begin{itemize}
		\item Xác định hình chiếu $M_1$ của điểm $M$ trên mặt phẳng $Oxy$. Trong mặt phẳng tọa độ $Oxy$, tìm hoành độ $a$, tung độ $b$ của điểm $M_1$.
		\item Xác định hình chiếu $P$ của điểm $M$ trên trục cao $Oz$, điểm $P$ ứng với số $c$ trên trục $Oz$. Số $c$ là cao độ của điểm $M$.
	\end{itemize}
	Bộ số $(a;b;c)$ là toạ độ của điểm $M$ trong không gian với hệ toạ độ $Oxyz$, kí hiệu là $M(a;b;c)$.
}{
	\begin{tikzpicture}[scale=0.6, font=\small,>=stealth]
		\path
		(0,0) coordinate (O)
		(-2,-2) coordinate (H)
		(3,-2) coordinate (M_1)
		(5,0) coordinate (K)
		(3,1) coordinate (M)
		(0,3) coordinate (P)
		;
		\draw[->] (0,0)--(6.7,0) node[below]{$y$};
		\draw[->] (0,0)--(-3,-3) node[below]{$x$};
		\draw[->] (0,0)--(0,4.3) node[left]{$z$};
		\draw[dashed] (P)node[left]{$c$}--(M)--(M_1)--(H)node[left]{$a$} (O)--(M_1)--(K)node[above]{$b$} (O)--(M);
		\foreach \x/\g in {O/160,M_1/-90,M/30,H/-80,K/-70,P/30}\draw[fill=black] (\x) circle (.05) +(\g:.5)node{\small$\x$};
		\foreach \x/\y/\z in {M_1/H/O,M_1/K/O,M/P/O}{\path pic[draw,angle radius=5pt]{right angle= \x--\y--\z};}
	\end{tikzpicture}
}
\subsubsection{Tọa độ của vectơ}
Trong KG $Oxyz$:
\immini{
	\begin{itemize}
		\item Toạ độ của điểm $M$ cũng là toạ độ của vectơ $\overrightarrow{OM}$.
		\item Cho $\vec{u}$. Dựng điểm $M(a;b;c)$ thỏa $\vec{OM}=\vec{u}$ thì tọa độ của điểm $M$ là tọa độ của $\vec{u}$. Theo hình vẽ thì
		      $$\vec{u}=\vec{OM}=\vec{OH}+\vec{OK}+\vec{OP}=a\vec{i}+b\vec{j}+c\vec{k}.$$
		      Suy ra
		      $$\vec{u}=\left(a;b;c \right)\Leftrightarrow \vec{u}=a\vec{i}+b\vec{j}+c\vec{k}. $$
	\end{itemize}
}{
	\begin{tikzpicture}[scale=0.6, font=\small,>=stealth]
		\path
		(0,0) coordinate (O)
		(-2,-2) coordinate (H)
		(3,-2) coordinate (M_1)
		(5,0) coordinate (K)
		(3,1) coordinate (M)
		(0,3) coordinate (P)
		;
		\draw[->] (0,0)--(6.7,0) node[below]{$y$};
		\draw[->] (0,0)--(-3,-3) node[below]{$x$};
		\draw[->] (0,0)--(0,4.3) node[left]{$z$};
		\draw[-stealth,blue,thick] (O)--(-1,-1)node[above]{$\vec{i}$};
		\draw[-stealth,blue,thick](O)--(1,0)node[below right]{$\vec{j}$};
		\draw[-stealth,blue,thick] (O)--(0,1)node[above left]{$\vec{k}$};
		\draw[dashed] (P)node[left]{$c$}--(M)--(M_1)--(H)node[left]{$a$} (O)--(M_1)--(K)node[above]{$b$};
		\draw[thick,->](O)--(M)node[midway,sloped,above,scale=1]{$\vec{u}$};
		\foreach \x/\g in {O/160,M_1/-90,M/30,H/-80,K/-70,P/30}\draw[fill=black] (\x) circle (.05) +(\g:.5)node{\small$\x$};
		\foreach \x/\y/\z in {M_1/H/O,M_1/K/O,M/P/O}{\path pic[draw,angle radius=5pt]{right angle= \x--\y--\z};}
	\end{tikzpicture}}
\begin{note}
	Tọa độ các véc tơ đơn vị lần lượt là: $\vec{i}=(1;0;0)$,\quad $\vec{j}=(0;1;0)$,\quad $\vec{k}=(0;0;1)$.
\end{note}
\subsection{PHÂN LOẠI VÀ PHƯƠNG PHÁP GIẢI TOÁN}

\begin{dang}{Tọa độ điểm, tọa độ vec tơ}
	\indamm{Khi xác định tọa độ điểm, tọa độ véc tơ ta chú ý các kết quả sau:}
	\begin{enumerate}
		\item $\vec{u}=a\vec{i}+b\vec{j}+c\vec{k} \Leftrightarrow \vec{u}=\big(a;b;c\big)$.
		\item $\vec{u}\big(u_1;u_2;u_3\big)=\vec{v}\big(v_1;v_2;v_3\big) \Leftrightarrow \heva{&u_1=v_1\\&u_2=v_2\\&u_3=v_3}$
		\item $\vec{OM}=(a;b;c)$ thì $M\big(a;b;c\big)$.
		\item $\vec{AB}=\big(x_B-x_A;y_B-y_A;z_B-z_A \big).$
		\item Chiếu điểm $M(a;b;c)$ lên mặt phẳng tọa độ (hoặc trục tọa độ) thì "thành phần bị khuyết" bằng $0$. Chẳng hạn: $M(1;2;3)$ chiếu lên $(Oxy)$ thì $z=0$. Suy ra hình chiếu là $M_1(1;2;0)$.
		\item Tứ giác $ABCD$ là hình bình hành khi và chỉ khi $$\vec{AD}=\vec{BC}$$
	\end{enumerate}
\end{dang}
\BTTL
\begin{vd}
	Trong KG $Oxyz$, cho $A(3 ;-2 ;-1)$. Gọi $ A_1, A_2, A_3$ lần lượt là hình chiếu của điểm $A$ trên các mặt phẳng toạ độ $(Oxy),(Oyz),(Oxz)$. Tìm toạ độ của các điểm $ A_1, A_2, A_3$.
	\loigiai{
		Toạ độ của các điểm $ A_1=(3 ;-2 ;0)$.\\
		Toạ độ của các điểm $ A_2=(3 ;0 ;-1)$.\\
		Toạ độ của các điểm $ A_3=(0 ;-2 ;-1)$
	}
\end{vd}

\begin{vd}
	Trong KG $Oxyz$, cho $A(-2;3;4)$. Gọi $H, K, P$ lần lượt là hình chiếu của điểm $A$ trên các trục $Ox, Oy, Oz$. Tìm tọa độ của các điểm $H,K,P$.
	\loigiai{
		Tìm tọa độ của các điểm $H=(-2;0;0)$.\\
		Tìm tọa độ của các điểm $K=(0;3;0)$.\\
		Tìm tọa độ của các điểm $P=(0;0;4)$.
	}
\end{vd}

\begin{vd} Trong KG $Oxyz$, cho $A(1; 1;-2)$, $B(4; 3; 1)$ và $C(-1;-2; 2)$.
	\begin{tasks}
		\task Tìm tọa độ của véctơ $\overrightarrow{A B}$.
		\task Tìm toạ độ của điểm $D$ sao cho $ABCD$ là hình bình hành.
	\end{tasks}
	\loigiai{
		\begin{enumerate}
			\item Ta có $
				      \overrightarrow{AB}=(4-1; 3-1; 1-(-2))=(3; 2; 3) .
			      $
			\item Gọi tọa độ của điểm $D$ là $\left(x_D; y_D; z_D\right)$, ta có
			      $
				      \overrightarrow{DC}=\left(-1-x_D;-2-y_D; 2-z_D\right) .
			      $\\
			      Tứ giác $A B C D$ là hình bình hành khi và chỉ khi
			      $$
				      \overrightarrow{DC}=\overrightarrow{A B} \Leftrightarrow\heva{&
					      - 1 - x _ { D } = 3 \\&
					      - 2 - y _ { D } = 2 \\&
					      2 - z _ { D } = 3.}
				      \Leftrightarrow \heva{&
					      x_D=-4 \\&
					      y_D=-4 \\&
					      z_D=-1.}$$
			      Vậy $D(-4;-4;-1)$.
		\end{enumerate}}
\end{vd}

\begin{vd}
	Trong KG $Oxyz$, cho hình hộp $ABCD \cdot A'B'C'D'$ có $A(4;6;-5)$, $B(5;7;-4)$, $C(5;6;-4)$, $D'(2;0;2)$. Tìm tọa độ các đỉnh còn lại của hình hộp $ABCD\cdot A'B'C'D'$.
	\loigiai{
		\begin{center}
			\begin{tikzpicture}[scale=0.7, font=\small, line join=round, line cap=round, >=stealth]
				\def\bc{4} % cạnh BC
				\def\ba{3} % cạnh BA
				\def\gocB{35} % góc B của đáy
				\coordinate[label=below left:$B(5;7;-4)$] (B) at (0,0);
				\coordinate[label=above left:$A(4;6;-5)$] (A) at (\gocB:\ba);
				\coordinate[label=below:$C(5;6;-4)$] (C) at (\bc,0);
				\coordinate[label=right:$D$] (D) at ($(C)-(B)+(A)$);
				\coordinate[label=above left:$A'$] (A') at ($(A)+(90:\bc)$);
				\coordinate[label=left:$B'$] (B') at ($(B)-(A)+(A')$);
				\coordinate[label=below right:$C'$] (C') at ($(C)-(A)+(A')$);
				\coordinate[label=right:$D'(2;0;2)$] (D') at ($(D)-(A)+(A')$);
				\draw (B')--(B)--(C)--(D)--(D')--(A')--(B')--(C')--(D') (C)--(C');
				\draw[dashed] (A')--(A)--(D) (A)--(B);
				\foreach \diem in {A,B,C,D,A',B',C',D'}	\fill (\diem)circle(1.5pt);
			\end{tikzpicture}
		\end{center}
		Ta có  $\overrightarrow{AD}=\overrightarrow{BC}\Leftrightarrow \heva{x_D&=x_A-x_B+x_C\\y_D&=y_A-y_B+y_C\\z_D&=z_A-z_B+z_C}\Leftrightarrow \heva{x_D&=4\\y_D&=5\\z_D&=-5}$. Suy ra $D(4;5;-5)$.\\
		Do đó $\overrightarrow{DD'}=(2-4;0-5;2-(-5)) =(-2;-5;7)$.\\
		Theo tính chất của hình hộp ta có $\overrightarrow{AA'}=\overrightarrow{BB'}=\overrightarrow{CC'}=\overrightarrow{DD'}=(-2;-5;7)$. Suy ra tọa độ đỉnh còn lại của hình hộp là $A'=(2;1;2)$, $B'(3;2;3)$, $C'(3;1;3)$.
	}
\end{vd}

\BTTN
\setcounter{ex}{0}
\Opensolutionfile{ans}[ans/2H2-B2-d1-1]

\begin{ex}
	Trong KG $Oxyz$, cho $\overrightarrow{a}=-2\overrightarrow{i}+3\overrightarrow{j}+5\overrightarrow{k}$. Toạ độ của véc-tơ $\overrightarrow{a}$ là
	\choice
	{$(2;-3;-5)$}
	{$(2;3;-5)$}
	{\True $(-2;3;5)$}
	{$(2;3;5)$}
	\loigiai{
		Toạ độ của véc-tơ $\overrightarrow{a}$ là $(-2;3;5)$.}
\end{ex} 

\begin{ex}
	Trong KG $Oxyz$, cho véc-tơ $\overrightarrow{u}=3\overrightarrow{i}+4\overrightarrow{k}-\overrightarrow{j}$. Tọa độ của véc-tơ $\overrightarrow{u}$ là
	\choice
	{\True $(3;-1;4)$}
	{$(3;4;-1)$}
	{$(4;-1;3)$}
	{$(4;3;-1)$}
	\loigiai
	{
		Tọa độ của véc-tơ $\overrightarrow{u}$ là $(3;-1;4)$.
	}
\end{ex} 

\begin{ex}
	Trong KG $Oxyz$, điểm nào sau đây thuộc trục $Oz$?
	\choice
	{$M(1;0;0)$}
	{$M(1;0;2)$}
	{$M(1;2;0)$}
	{\True $M(0;0;-2)$}
	\loigiai{
		Ta có $M(0;0;-2) \in Oz$.
	}
\end{ex} 

\begin{ex}%[An Do - Dự án 2H3-LVD]%[2H3Y1-1]%
	Trong KG $Oxyz$, cho điểm $M$ thỏa $\vec{OM} = 2\vec{i} + \vec{j}$. Tọa độ điểm $M$ là
	\choice
	{$M(0;2;1)$}
	{$M(1;2;0)$}
	{$M(2;0;1)$}
	{\True$M(2;1;0)$}
	\loigiai{
		Tọa độ $\vec{OM} = 2\vec{i} + \vec{j} = (2;0;0) + (0;1;0) = (2;1;0)$.\\
		Vậy $M (2;1;0)$.
	}
\end{ex} 

\begin{ex}%[2H3Y1-1]%[Đoàn Mạnh Hùng]%
	Trong KG $Oxyz$, cho vectơ $\overrightarrow{OA}=\overrightarrow{j}-2\overrightarrow{k}$. Tọa độ điểm $A$ là
	\choice
	{$(1;0;-2)$}
	{\True $(0;1;-2)$}
	{$(0;-1;2)$}
	{$(1;-2;0)$}
	\loigiai{
		Ta có $\overrightarrow{OA}=\vec{j}-2\vec{k}\Leftrightarrow A(0;1;-2)$.
	}
\end{ex} 

\begin{ex}
	Trong không gian $O x y z$, xác định toạ độ của điểm $A$ biết $A$ nằm trên tia $O x$ và $O A=2$.
	\choice
	{$A(0;0;2)$}
	{$A(2;2;0)$}
	{$A(0;2;0)$}
	{\True $A(2;0;0)$}
	\loigiai{$A$ nằm trên tia $O x$ và $O A=2$ nên $A(2;0;0)$.
	}
\end{ex} 

\begin{ex}
	Trong không gian $O x y z$, xác định toạ độ của điểm $A$ biết $A$ nằm trên tia đối của tia $O y$ và $O A=3$.
	\choice
	{$A(0;3;0)$}
	{\True $A(0;-3;0)$}
	{$A(0;-9;0)$}
	{$A(3;-3;0)$}
	\loigiai{
		$A$ nằm trên tia đối của tia $O y$ và $O A=3$ nên $A(0;-3;0)$.}
\end{ex} 

\begin{ex}
	Trong KG $Oxyz$, cho hai điểm $A(1;-1;2)$ và $B(2;1;-4)$. Véc-tơ $\vec{AB}$ có tọa độ là
	\choice
	{$(-1;-2;6)$}
	{$(3;0;-2)$}
	{$(1;0;-6)$}
	{\True $(1;2;-6)$}
	\loigiai{
		Ta có $\vec{AB} = (1;2;-6)$.
	}
\end{ex} 

\begin{ex}
	Trong không gian $ Oxyz $, cho hai điểm $ A(1;3;-2) $, $ B(3;-2;4) $. Véc-tơ $ \overrightarrow{AB} $ có tọa độ là
	\choice
	{\True $ (2;5;6) $}
	{$ (4;1;2) $}
	{\True $ (2;-5;6) $}
	{$ (-2;5;6) $}
	\loigiai{
		Véc-tơ $ \overrightarrow{AB} $ có tọa độ là $ (2;-5;6) $.}
\end{ex} 

\begin{ex}
	Cho hai điểm $A$, $B$ thỏa mãn $\vec{OA} = (2;-1; 3)$ và  $\vec{OB}= (5;2;-1)$. Tìm tọa độ véc-tơ $\vec{AB}$.
	\choice
	{$\vec{AB} =(2;-1;3)$}
	{\True  $\vec{AB} =(3;3;-4)$}
	{$\vec{AB} = (7;1;2)$}
	{$\vec{AB} =(3;-3;4)$}
	\loigiai{
		$\vec{AB} = \vec{OB} - \vec{OA} = (5-2;2+1;-1-3)=(3;3;-4)$.}
\end{ex} 

\begin{ex}
	Trong KG $Oxyz$, cho hai điểm $M$ và $N$ biết $M(2;1;-1)$ và $\vv{MN}=(-1;2-3)$. Tọa độ $N$ là
	\choice
	{$N(1;-3;-4)$}
	{\True $N(1;3;-4)$}
	{$N(-1;3;-4)$}
	{$N(1;3;4)$}
	\loigiai
	{
		Gọi $N(x,y,z)$, khi đó ta có $\heva{&x-2=-1\\&y-1=2\\&z+1=-3}\Leftrightarrow \heva{&x=1\\&y=3\\&z=-4}\Rightarrow N(1;3;-4)$.\\
	}
\end{ex} 

\begin{ex}
	Hình chiếu vuông góc của điểm $A(3;-4;5)$ trên mặt phẳng $(Oxz)$ là điểm
	\choice
	{$M(3;0;0)$}
	{$M(0;-4;5)$}
	{$M(0;0;5)$}
	{\True $M(3;0;5)$}
	\loigiai{
		Hình chiếu vuông góc của điểm $A(3;-4;5)$ trên mặt phẳng $(Oxz)$ là điểm $M(3;0;5)$.}
\end{ex} 

\begin{ex}
	Hình chiếu vuông góc của điểm $A(1;2;3)$ trên mặt phẳng $(Oxy)$ là điểm
	\choice
	{$M(0;0;3)$}
	{\True $N(1;2;0)$}
	{$Q(0;2;0)$}
	{$P(1;0;0)$}
	\loigiai
	{
		Hình chiếu vuông góc của điểm $A(1;2;3)$ trên mặt phẳng $(Oxy)$ là điểm $N(1;2;0)$.
	}
\end{ex} 

\begin{ex}
	Hình chiếu vuông góc của điểm $M(2;1;-3)$ lên mặt phẳng $(Oyz)$ có tọa độ là
	\choice
	{$(2;0;0)$}
	{$(2;1;0)$}
	{\True $(0;1;-3)$}
	{$(2;0;-3)$}
	\loigiai{
		Điểm thuộc $(Oyz)$ có tọa độ $(0;y;z)$ nên hình chiếu của $M$ lên $(Oyz)$ có tọa độ là $(0;-1;3)$.
	}
\end{ex} 

\begin{ex}
	Hình chiếu vuông góc của điểm $A(3;2;1)$ trên trục $Ox$ có tọa độ là
	\choice
	{$(0;2;1)$}
	{$(0;2;0)$}
	{\True $(3;0;0)$}
	{$(0;0;1)$}
	\loigiai{
		Hình chiếu vuông góc của điểm $A(3;2;1)$ lên trục $Ox$ là $A'(3;0;0)$.
	}
\end{ex} 

\begin{ex}
	Hình chiếu của điểm $M(2;3;-2)$ trên trục $Oy$ có tọa độ là
	\choice
	{$ (2;0;0) $}
	{\True $ (0;3;0) $}
	{$ (0;0;-2) $}
	{$ (2;0;-2) $}
	\loigiai{
		Hình chiếu của điểm $M(2;3;-2)$ trên trục $Oy$ có tọa độ là $(0;3;0)$.
	}
\end{ex} 

\begin{ex}
	\immini{Trong KG $Oxyz$, cho hình bình hành $ABCD$ với $A(-2;3;1)$, $B(3;0;-1)$, $C(6;5;0)$. Tọa độ đỉnh $D$ là
		\choice
		{$D(11;2;2)$}
		{\True $D(1;8;2)$}
		{$D(11;2;-2)$}
		{$D(1;8;-2)$}}{
		\begin{tikzpicture}[scale=0.7, font=\small,>=stealth]
			\path
			%	Vẽ mp
			(0,0) coordinate (A)
			(1,1.5) coordinate (B)
			(4,0) coordinate (D)
			($(D)+(B)-(A)$)coordinate (C)
			;
			\draw (A)--(B)--(C)--(D)--(A);
			\foreach \x/\g in {A/-90,B/90,C/0,D/-90}\draw[fill=black] (\x) circle (.05) +(\g:.5)node{\small$\x$};
		\end{tikzpicture}
	}
	\loigiai{
		Ta có $\heva{& x_D = x_A+x_C-x_B = 1 \\ & y_D = y_A +y_C -y_B = 8 \\ & z_D = z_A + z_C - z_B = 2}\Rightarrow D(1;8;2)$.
	}
\end{ex} 

\begin{ex}
	\immini{Trong KG $Oxyz$, cho các điểm $A(1;0;3)$, $B(2;3;-4)$,$C(-3;1;2)$. Tìm tọa độ điểm $D$ sao cho tứ giác $ABCD$ là hình bình hành.
		\choice
		{$D(4;2;9) $}
		{$D(-2;4;-5) $}
		{$D(6;2;-3) $}
		{\True $(-4;-2;9) $}}{
		\begin{tikzpicture}[scale=0.7, font=\small,>=stealth]
			\path
			%	Vẽ mp
			(0,0) coordinate (A)
			(1,1.5) coordinate (B)
			(4,0) coordinate (D)
			($(D)+(B)-(A)$)coordinate (C)
			;
			\draw (A)--(B)--(C)--(D)--(A);
			\foreach \x/\g in {A/-90,B/90,C/0,D/-90}\draw[fill=black] (\x) circle (.05) +(\g:.5)node{\small$\x$};
		\end{tikzpicture}}
	\loigiai{
		Gọi $D(x;y;z) \Rightarrow \vec{CD}=(x+3;y-1;z-2)$ và $\vec{BA}=(-1;-3;7)$.\\
		Để tứ giác $ABCD$ là hình bình hành ta có $\vec{BA}=\vec{CD}$ $\Rightarrow \heva{&x+3=-1\\&y-1=-3\\&z-2=7} \Rightarrow D(-4;-2;9)$.
	}
\end{ex} 

\begin{ex}
	\immini{Cho hình hộp $A B C D . A' B' C' D'$ có $A(1 ; 0 ; 1)$, $B(2 ; 1 ; 2)$, $D(1 ;-1 ; 1), C'(4 ; 5 ;-5)$. Tìm tọa độ đỉnh $C$ của hình hộp.
		\haicot
		{$C(2;0;2)$}
		{$C(2;0;2)$}
		{$C(2;0;2)$}
		{$C(2;0;2)$}}{
		\begin{tikzpicture}[scale=0.65, font=\small, line join=round, line cap=round, >=stealth]
			\def\bc{4} % cạnh BC
			\def\ba{2} % cạnh BA
			\def\gocB{35} % góc B của đáy
			\coordinate[label=below left:$B$] (B) at (0,0);
			\coordinate[label=above left:$A$] (A) at (\gocB:\ba);
			\coordinate[label=below:$C$] (C) at (\bc,0);
			\coordinate[label=right:$D$] (D) at ($(C)-(B)+(A)$);
			\coordinate[label=above left:$A'$] (A') at ($(A)+(100:\bc)$);
			\coordinate[label=left:$B'$] (B') at ($(B)-(A)+(A')$);
			\coordinate[label=below right:$C'$] (C') at ($(C)-(A)+(A')$);
			\coordinate[label=right:$D'$] (D') at ($(D)-(A)+(A')$);
			\draw (B')--(B)--(C)--(D)--(D')--(A')--(B')--(C')--(D') (C)--(C');
			\draw[dashed] (A')--(A)--(D) (A)--(B);
			\foreach \diem in {A,B,C,D,A',B',C',D'}	\fill (\diem)circle(1.5pt);
		\end{tikzpicture}}
	\loigiai{
		Ta có $\overrightarrow{AB}=\overrightarrow{DC}\Leftrightarrow \heva{& 2-1=x_C-1\\ & 1-0=y_C-(-1) \\ & 2-1=z_C-1} \Leftrightarrow \heva{&x_C = 2 \\ & y_C = 0 \\ & z_C=2}\Rightarrow C(2;0;2)$.
	}
\end{ex} 


\begin{ex} %[2H2H2-2]
	Cho hình hộp $A B C D . A' B' C' D'$ có $A(1 ; 0 ; 1)$, $B(2 ; 1 ; 2)$, $D(1 ;-1 ; 1), C'(4 ; 5 ;-5)$. Tìm tọa độ đỉnh $A'$ của hình hộp.
	\choice
	{$A'(-1;-5;8)$}
	{$A'(-1;-5;8)$}
	{$A'(-1;-5;8)$}
	{$A'(-1;-5;8)$}
	\loigiai{
		Ta có
		\begin{itemize}
			\item $\overrightarrow{AB}=\overrightarrow{DC}\Leftrightarrow \heva{& 2-1=x_C-1\\ & 1-0=y_C-(-1) \\ & 2-1=z_C-1} \Leftrightarrow \heva{&x_C = 2 \\ & y_C = 0 \\ & z_C=2}\Rightarrow C(2;0;2)$;
			\item $\overrightarrow{AA'}=\overrightarrow{CC'}\Leftrightarrow \heva{& x_{A'}-1=2-4\\ & y_{A'}-0=0-5 \\ & z_{A'}-1=2-(-5)} \Leftrightarrow \heva{& x_{A'} = -1 \\ & y_{A'} = -5 \\ & z_{A'}=8}\Rightarrow A'(-1;-5;8)$;
		\end{itemize}
	}
\end{ex} 


\begin{ex}%[2H2H2-2]
	\immini{Cho hình hộp $A B C D . A' B' C' D'$ có $A(1 ; 0 ; 1)$, $B(2 ; 1 ; 2)$, $D(1 ;-1 ; 1), C'(4 ; 5 ;-5)$. Tìm tọa độ đỉnh $D'$ của hình hộp.
		\haicot
		{$D'(-1;-6;8)$}
		{$D'(-1;-6;8)$}
		{$D'(-1;-6;8)$}
		{$D'(-1;-6;8)$}}{
		\begin{tikzpicture}[scale=0.65, font=\small, line join=round, line cap=round, >=stealth]
			\def\bc{4} % cạnh BC
			\def\ba{2} % cạnh BA
			\def\gocB{35} % góc B của đáy
			\coordinate[label=below left:$B$] (B) at (0,0);
			\coordinate[label=above left:$A$] (A) at (\gocB:\ba);
			\coordinate[label=below:$C$] (C) at (\bc,0);
			\coordinate[label=right:$D$] (D) at ($(C)-(B)+(A)$);
			\coordinate[label=above left:$A'$] (A') at ($(A)+(100:\bc)$);
			\coordinate[label=left:$B'$] (B') at ($(B)-(A)+(A')$);
			\coordinate[label=below right:$C'$] (C') at ($(C)-(A)+(A')$);
			\coordinate[label=right:$D'$] (D') at ($(D)-(A)+(A')$);
			\draw (B')--(B)--(C)--(D)--(D')--(A')--(B')--(C')--(D') (C)--(C');
			\draw[dashed] (A')--(A)--(D) (A)--(B);
			\foreach \diem in {A,B,C,D,A',B',C',D'}	\fill (\diem)circle(1.5pt);
		\end{tikzpicture}}
	\loigiai{
		Ta có
		\begin{itemize}
			\item $\overrightarrow{AB}=\overrightarrow{DC}\Leftrightarrow \heva{& 2-1=x_C-1\\ & 1-0=y_C-(-1) \\ & 2-1=z_C-1} \Leftrightarrow \heva{&x_C = 2 \\ & y_C = 0 \\ & z_C=2}\Rightarrow C(2;0;2)$;
			\item $\overrightarrow{DD'}=\overrightarrow{CC'}\Leftrightarrow \heva{& x_{D'}-1=2-4\\ & y_{D'}-(-1)=0-5 \\ & z_{D'}-1=2-(-5)} \Leftrightarrow \heva{& x_{D'} = -1 \\ & y_{D'} = -6 \\ & z_{D'}=8}\Rightarrow D'(-1;-6;8)$.
		\end{itemize}
	}
\end{ex} 

\Closesolutionfile{ans}
\BTTF
\Opensolutionfile{ans}[ans/2H2-B2-d1-2]
\begin{ex}
	Trong KG $Oxyz$, cho $\vec{a}=\vec{i}+3\vec{k}-4\vec{j}$ và $\vec{b}=\big(m-n;4m-6n;n^2-3m+2\big)$, với $m$, $n$ là tham số.
	\choiceTF
	{Tọa độ $\vec{a}=\big(1;3;-4\big)$}
	{\True Dựng điểm $A$ thỏa $\vec{OA}=\vec{a}$ thì $A(1;-4;3)$}
	{Tồn tại giá trị của $m$ và $n$ để $\vec{b}=\vec{0}$}
	{\True Nếu $\vec{a}=\vec{b}$ thì $m+n=9$}
	\loigiai{
		\begin{enumerate}[a)]
			\item Tọa độ $\vec{a}=\big(1;-4;3\big)$.
			\item Khi $\vec{OA}=\vec{a}$ thì tọa độ $\vec{a}$ cũng là tọa độ điểm $A$. Suy ra $A(1;-4;3)$.
			\item $\vec{b}=\vec{0} \Leftrightarrow \heva{&m-n=0\\&4m-6n=0\\&n^2-3m+2=0} \Leftrightarrow \heva{&m=0\\&n=0\\&n^2-3m+2=0}$ (vô nghiệm).\\
			      Vậy, không tồn tại $m$, $n$ để $\vec{b}=\vec{0}$.
			\item $\vec{a}=\vec{b} \Leftrightarrow \heva{&m-n=1\\&4m-6n=-4\\&n^2-3m+2=3} \Leftrightarrow \heva{&m=5\\&n=4}$.\\
			      Suy ra $m+n=9$.
		\end{enumerate}}
\end{ex} 
\begin{ex}
	\immini{Trong KG $Oxyz$, cho $\vec{a}=(2;2;0)$, $\vec{b}=2\vec{j}+2\vec{k}$. Dựng $\vec{OA}=\vec{a}$ và $\vec{OB}=\vec{b}$.
		\choiceTF
		{$\vec{a}=2\vec{i}+2\vec{k}$}
		{\True Toạ độ $\vec{b}=(0;2;2)$}
		{\True Toạ độ $\vec{AB}=(-2;2;0)$}
		{Góc $\widehat{AOB}=45^\circ$}}{
		\begin{tikzpicture}[scale=0.5, font=\small,>=stealth]
			\path
			(0,0) coordinate (O)
			(-2,-2) coordinate (H)
			(4,0) coordinate (K)
			(0,3.5) coordinate (P)
			($(P)+(H)-(O)$)coordinate (A)
			($(P)+(K)-(O)$)coordinate (B)
			;
			\draw[->] (0,0)--(6.7,0) node[below]{$y$};
			\draw[->] (0,0)--(-3,-3) node[below]{$x$};
			\draw[->] (0,0)--(0,5) node[left]{$z$};
			\draw[-stealth,blue,thick] (O)--(-1,-1)node[above]{$\vec{i}$};
			\draw[-stealth,blue,thick](O)--(1,0)node[below right]{$\vec{j}$};
			\draw[-stealth,blue,thick] (O)--(0,1)node[above right]{$\vec{k}$};
			\draw[dashed] (H)--(A)--(P)--(B)--(K);
			\draw[thick,->](O)--(A)node[midway,sloped,above,scale=1]{$\vec{a}$};
			\draw[thick,->](O)--(B)node[midway,sloped,below,scale=1]{$\vec{b}$};
			\foreach \x/\g in {O/-90,A/180,B/10}\draw[fill=black] (\x) circle (.05) +(\g:.5)node{\small$\x$};
		\end{tikzpicture}}
	\loigiai{
		\immini{
			\begin{enumerate}[a)]
				\item Ta có $\vec{a}=(2;0;2)\Rightarrow \vec{a}=2\vec{i}+2\vec{k} $.
				\item Ta có $\vec{b}=2\vec{j}+2\vec{k} \Rightarrow \vec{b}=(0;2;2)$.
				\item Ta có $\vec{OA}=\vec{a}$ thì toạ độ véc tơ $\vec{a}$ cũng chính là toạ độ $A$. Suy ra $A(2;0;2)$. Tương tự $B(0;2;2)$. Từ đây, ta tính được
				      $$\vec{AB}=(-2;2;0).$$
				\item Nhận xét $OHMK.PANB$ là hình lập phương. Suy ra $\triangle OAB$ đều. Vậy $\widehat{AOB}=60^\circ$.
			\end{enumerate}}{
			\begin{tikzpicture}[scale=0.8, font=\small,>=stealth]
				\path
				(0,0) coordinate (O)
				(-2,-2) coordinate (H)
				(4,0) coordinate (K)
				(0,3.5) coordinate (P)
				($(P)+(H)-(O)$)coordinate (A)
				($(P)+(K)-(O)$)coordinate (B)
				($(H)+(K)-(O)$)coordinate (M)
				($(A)+(B)-(P)$)coordinate (N)
				;
				\draw[->] (0,0)--(6.7,0) node[below]{$y$};
				\draw[->] (0,0)--(-3,-3) node[below]{$x$};
				\draw[->] (0,0)--(0,4.3) node[left]{$z$};
				\draw[-stealth,blue,thick] (O)--(-1,-1)node[above]{$\vec{i}$};
				\draw[-stealth,blue,thick](O)--(1,0)node[below right]{$\vec{j}$};
				\draw[-stealth,blue,thick] (O)--(0,1)node[above right]{$\vec{k}$};
				\draw[dashed] (H)--(A)--(P)--(B)--(K)--(M)--(H) (A)--(N)--(B) (M)--(N);
				\draw[thick,->](O)--(A)node[midway,sloped,above,scale=1]{$\vec{a}$};
				\draw[thick,->](O)--(B)node[midway,sloped,above,scale=1]{$\vec{b}$};
				\foreach \x/\g in {O/-90,A/180,B/10,H/-90,M/-90,K/20,P/30,N/0}\draw[fill=black] (\x) circle (.05) +(\g:.5)node{\small$\x$};
			\end{tikzpicture}
		}
	}
\end{ex} 

\begin{ex}
	\immini{Trong không gian $O x y z$, cho hình hộp $O A B C . O' A' B' C'$ có $A(1 ; 1 ;-1)$, $B(0 ; 3 ; 0)$, $\vec{BC'}=(2 ;-6 ; 6)$. Gọi $H$, $K$ lần lượt là trọng tâm của tam giác $OA'O'$ và $CB'C'$.
	\choiceTF
		{\True Tọa độ điểm $C'$ là $(2;-3;6)$}
		{\True Tọa độ điểm $O'$ là $(3;-5;5)$}
		{Tọa độ véc tơ $\vec{AB'}=(-2;3;-6)$}
		{Tọa độ véc tơ $\vec{HK}=(-1;2;-1)$}}
		{\begin{tikzpicture}[scale=0.5, font=\small,>=stealth]
			\path
			%	Vẽ mp
			(0,0) coordinate (O)
			(-1.5,-1) coordinate (A)
			(5,0) coordinate (C)
			($(A)+(C)-(O)$)coordinate (B)
			($(O)+(-.5,4)$)coordinate (O')
			($(A)+(C)-(O)$)coordinate (B)
			($(A)+(O')-(O)$)coordinate (A')
			($(A')+(B)-(A)$)coordinate (B')
			($(B')+(C)-(B)$)coordinate (C')
			;
			\draw (B)--(A)--(A')--(B')--(B)--(C)--(C')--(O')--(A') (B')--(C');
			\draw[thick,->] (B)--(C');
			\draw[dashed] (C)--(O)--(O') (O)--(A);
			\foreach \x/\g in {O/170,A/-90,B/-90,C/0,O'/90,A'/180,B'/-5,C'/10}\draw[fill=black] (\x) circle (.05) +(\g:.4)node{\small$\x$};
		\end{tikzpicture}}
	\loigiai{
		\begin{enumerate}[a)]
			\item Gọi $C'(x;y;z)$. Ta có $$\vec{BC'}=(2 ;-6 ; 6) \Rightarrow \heva{&x-0=2\\&y-3=-6\\&z-0=6} \Leftrightarrow \heva{&x=2\\&y=-3\\&z=6}$$
			      Vậy $C(2;-3;6)$.
			\item Gọi $O'(x;y;z)$. Theo hình vẽ thì
			      $$\vec{AO'}=\vec{BC'} \Leftrightarrow \heva{&x-1=2\\&y-1=-6\\&z+1=6} \Leftrightarrow \heva{&x=3\\&y=-5\\&z=5}$$
			      Vậy $O'(3;-5;5)$.
			\item Theo hình vẽ thì $\vec{AB'}=\vec{OC'}=(2;-3;6)$.
			\item Ta có $\vec{HK}=\vec{AB}=(-1;2;1)$.
		\end{enumerate}
	}
\end{ex} 
\Closesolutionfile{ans}

\begin{dang}{Tọa độ hóa một số hình không gian}
	\begin{listEX}[1]
		\item [\ding{172}] Chọn một điểm mà từ đó có ba đường đôi một vuông góc nhau làm gốc tọa độ.
		\item [\ding{173}] Xây dựng tọa độ các điểm trên hình đã cho tương ứng với hệ trục vừa chọn.
		\item [\ding{173}] Tọa độ các điểm đặc biệt:
		\begin{listEX}[3]
			\item [$\bullet$] $M \in Ox \Rightarrow M(x;0;0)$.
			\item [$\bullet$] $M \in Oy \Rightarrow M(0;y;0)$.
			\item [$\bullet$] $M \in Oz \Rightarrow M(0;0;z)$.
			\item [$\bullet$] $M \in (Oxy) \Rightarrow M(x;y;0)$.
			\item [$\bullet$] $M \in (Oxz) \Rightarrow M(x;0;z)$.
			\item [$\bullet$] $M \in (Oyz) \Rightarrow M(0;y;z)$.
		\end{listEX}
	\end{listEX}
\end{dang}
\BTTL
\begin{vd}
	\immini{Cho hình hộp chữ nhật $ABCD.A'B'C'D'$ có cạnh $AB=AA'=2$, $AD=4$. Gọi $E$ là tâm của hình chữ nhật $ABCD$, $F$ là trung điểm $AC'$. Với hệ toạ độ $Oxyz$ được thiết lập như hình bên (gốc tọa độ $O$ trùng với $A$), hãy xác định tọa độ các đỉnh của hình hộp chữ nhật và tọa độ hai điểm $E$, $F$.
	}{
		\begin{tikzpicture}[scale=0.7, font=\small,>=stealth]
			\path
			(0,0) coordinate (A)
			(-2,-2) coordinate (B)
			(5,0) coordinate (D)
			(0,3) coordinate (A')
			($(B)+(D)-(A)$)coordinate (C)
			($(A')+(B)-(A)$)coordinate (B')
			($(B')+(C)-(B)$)coordinate (C')
			($(A')+(D)-(A)$)coordinate (D')
			($(A)!0.5!(C)$)coordinate (E)
			($(A)!0.5!(C')$)coordinate (F)
			;
			\draw[->] (D)--(6.7,0) node[below]{$y$};
			\draw[->] (B)--(-3,-3) node[below]{$x$};
			\draw[->] (A')--(0,4.5) node[left]{$z$};
			\draw (B')--(B)--(C)--(D)--(D')--(A')--(B')--(C')--(D') (C)--(C');
			\draw[dashed] (A')--(A)--(B)--(D)--(A)--(C)--(A') (A)--(C');
			\draw[-stealth,blue,thick] (O)--(-0.6,-0.6)node[above]{$\vec{i}$};
			\draw[-stealth,blue,thick](O)--(1,0)node[below right]{$\vec{j}$};
			\draw[-stealth,blue,thick] (O)--(0,0.7)node[left]{$\vec{k}$};
			\foreach \x/\g in {A/-90,B/180,C/-70,D/10,A'/40,B'/180,C'/10,D'/0,E/-90,F/-10}\draw[fill=black] (\x) circle (.04) +(\g:.5)node{\small$\x$};
		\end{tikzpicture}}
	\loigiai{}
\end{vd}

\begin{vd}%[2H2V2-2]
	\immini{
		Một máy bay $M$ đang cất cánh từ phi trường. Với hệ toạ độ $Oxyz$ được thiết lập như Hình bên, cho biết $M$ là vị trí của máy bay với $OM=14$, $\widehat{NOB}=32^\circ$, $\widehat{MOC}=65^\circ$. Tính toạ độ điểm $M$.
	}{
		\begin{tikzpicture}[scale=0.6, font=\small,>=stealth]
			\path
			(0,0) coordinate (O)
			(-2,-2) coordinate (A)
			(3,-2) coordinate (N)
			(5,0) coordinate (B)
			(3,1) coordinate (M)
			(0,3) coordinate (C)
			;
			\draw[->] (0,0)--(6,0) node[below]{$y$};
			\draw[->] (0,0)--(-3,-3) node[below]{$x$};
			\draw[->] (0,0)--(0,4) node[left]{$z$};
			\draw[dashed] (C)--(M)--(N)--(A) (O)--(N)--(B);
			\draw[fill=blue] (M)circle (0.15)--(O)node[midway,sloped,scale=1,above]{$14$};
			\foreach \x/\g in {O/160,N/-90,M/30,A/-80,B/-70,C/30}\draw[fill=black] (\x) circle (.05) +(\g:.5)node{\small$\x$};
			\foreach \x/\y/\z in {N/A/O,N/B/O,M/C/O}{\path pic[draw,angle radius=5pt]{right angle= \x--\y--\z};}
			\draw pic["$65^\circ$",draw,angle eccentricity=1.9,angle radius=0.3cm]{angle=M--O--C};
			\draw pic["$32^\circ$",draw,angle eccentricity=1.9,angle radius=0.4cm]{angle=N--O--B};
		\end{tikzpicture}
	}
	\loigiai{
	\immini{
	Ta có:\\
	$OC=OM\cos 65^\circ\approx 5{,}9$.\\
	$ON=CN=OM\sin 65^\circ\approx 12{,}7$.\\
	$OB=ON\cos 32^\circ\approx 10{,}8$.\\
	$OA=BN=ON\sin 32^\circ\approx 6{,}7$.\\
	Vì $OANB$ là hình chữ nhật nên $\vec{ON}=\vec{OA}+\vec{OB}$.\\
	Vì $OCMN$ là hình chữ nhật nên $$\vec{OM}=\vec{OC}+\vec{ON}=\vec{OA}+\vec{OB}+\vec{OC}=6{,}7\vec{i}+10{,}8\vec{j}+5{,}9\vec{k}.$$
	Do đó $M(6{,}7; 10{,}8; 5{,}9)$.
	}{
	\begin{tikzpicture}[scale=1, font=\small, line join=round, line cap=round, >=stealth]
		\def\x{2.5}
		\def\y{4}
		\def\z{3}
		\def\gocXY{-150} % góc B của đáy
		\coordinate[label=above left:$O$] (O) at (0,0);
		\coordinate[label=below:$x$] (x) at (\gocXY:\x);
		\coordinate[label=below:$y$] (y) at (\y,0);
		\coordinate[label=right:$z$] (z) at (0,\z);
		\def\vtdv{1}
		\coordinate (i) at (\gocXY:\vtdv);
		\coordinate (j) at (\vtdv,0);
		\coordinate (k) at (0,\vtdv);
		\coordinate[label=above left:$A$] (A) at (\gocXY:0.7*\x);
		\coordinate[label=above:$B$] (B) at (0.8*\y,0);
		\coordinate[label=left:$C$] (C) at (0,0.8*\z);
		\coordinate[label=below:$N$] (N) at ($(A)+(B)$);
		\coordinate[label=right:$M$] (M) at ($(N)+(C)$);
		\draw[->] (O)--(x);
		\draw[->] (O)--(y);
		\draw[->] (O)--(z);
		\draw[->] (O)--(M);
		\draw[->, red] (O)--(i) node[left]{$\vec{i}$};
		\draw[->, red] (O)--(j) node[above]{$\vec{j}$};
		\draw[->, red] (O)--(k) node[left]{$\vec{k}$};
		\draw[dashed] (A)--(N)--(B) (O)--(N) (C)--(M)--(N);
		\foreach \diem in {A,B,C,O,N,M}	\fill (\diem)circle(1.5pt);
		\foreach \A/\B/\C in {O/C/M,N/B/y,N/A/O}
		\draw pic[draw=black,angle radius=6pt] {right angle = \A--\B--\C};
		\draw pic[draw,% double,% nét đôi
				blue,angle radius=5mm,angle eccentricity=2.5,"$32^\circ$"] {angle = N--O--y};
		\draw pic[draw,% double,% nét đôi
				blue,angle radius=3mm,angle eccentricity=2.5,"$65^\circ$"] {angle = M--O--C};
	\end{tikzpicture}
	}
	}
\end{vd}

\BTTN
\Opensolutionfile{ans}[ans/2H2-B2-d2-1]

\begin{ex}
	\immini{Hình bên mô tả một sân cầu lông với kích thước theo tiêu chuẩn quốc tế. Với hệ toạ độ $Oxyz$ được thiết lập như hình bên (đơn vị trên mỗi trục là mét), giả sử $AB$ là một trụ cầu lông để căng lưới, hãy xác định tọa độ của $B$.
		\choice
		{$\big(6,1;6,7;1,55\big)$}
		{\True $\big(6,7;6,1;1,55\big)$}
		{$\big(6,1;0;1,55\big)$}
		{$\big(0;6,7;1,55\big)$}
	}{
		\begin{tikzpicture}[scale=0.55, font=\small,>=stealth]
			\path
			%	Vẽ mp
			(0,0) coordinate (O)
			(8,0) coordinate (M)
			(10,2) coordinate (N)
			(2,2) coordinate (K)
			(4,1) coordinate (F)
			(4,0) coordinate (E)
			(6,3) coordinate (B)
			(6,2) coordinate (A)
			(2.3,-0.7) coordinate (I)
			($(A)!0.5!(B)$)coordinate (C)
			($(E)!0.5!(F)$)coordinate (D)
			;
			\draw[->] (M)--(10.5,0) node[below]{$x$};
			\draw[->] (K)--(3,3) node[above]{$y$};
			\draw[->] (O)--(0,4.5) node[left]{$z$};
			\draw[fill=green!20] (O)--(M)--(N)--(K)--cycle;
			\draw[pattern=north west lines] (C)--(B)--(F)--(D)--cycle;
			\draw (O)--(M)--(N)--(K)--(O) (E)--(F) (A)--(B);
			\draw[<->,dashed] (0,-0.3)--(8,-0.3)node[midway,sloped,below]{\scriptsize$13,40$ m};
			\draw[<->,dashed] (8.3,0)--(10.3,2)node[midway,right]{\scriptsize$6,10$ m};
			\draw[<->,dashed] (6.3,2)--(6.3,3)node[midway,right]{\scriptsize$1,55$ m};
			\foreach \x/\g in {O/180,A/-80,B/90}\draw[fill=black] (\x) circle (.05) +(\g:.5)node{\small$\x$};
		\end{tikzpicture}}
	\loigiai{
		\begin{itemize}
			\item Gọi toạ độ điểm $A$ là $\left(x_A;y_A;z_A\right)$. Vì chiều rộng của sân là $6,1 \mathrm{~m}$ nên $x_A=6,1$. Do một nửa chiều dài của sân là $6,7 \mathrm{~m}$ nên $y_A=6,7$. Điểm $A$ thuộc mặt phẳng $(Oxy)$ nên $z_A=0$. Vì vậy, điểm $A$ có tọa độ là $(6,1;6,7;0)$.
			\item Độ dài đoạn thẳng $AB$ là $1,55 \mathrm{~m}$ nên điểm $B$ có toạ độ là $(6,1;6,7;1,55)$.
		\end{itemize}
		Vậy ta có: $\overrightarrow{AB}=(6,1-6,1;6,7-6,7;1,55-0)$, tức là $\overrightarrow{AB}=(0;0;1,55)$.
	}
	\loigiai{
	}
\end{ex} 

\begin{ex}
	\immini{Cho hình lập phương $ABCD.A'B'C'D'$ có cạnh bằng 2. Với hệ toạ độ $Oxyz$ được thiết lập như hình bên (gốc tọa độ $O$ trùng với điểm $A$), tọa độ điểm $B'$ là
		\haicot
		{$B(0;2;0)$}
		{$B(2;2;2)$}
		{$B(2;2;0)$}
		{\True $B(2;0;2)$}
	}{
		\begin{tikzpicture}[scale=0.5, font=\small,>=stealth]
			\path
			(0,0) coordinate (A)
			(-2,-2) coordinate (B)
			(5,0) coordinate (D)
			(0,3) coordinate (A')
			($(B)+(D)-(A)$)coordinate (C)
			($(A')+(B)-(A)$)coordinate (B')
			($(B')+(C)-(B)$)coordinate (C')
			($(A')+(D)-(A)$)coordinate (D')
			;
			\draw[->] (D)--(6.7,0) node[below]{$y$};
			\draw[->] (B)--(-3,-3) node[below]{$x$};
			\draw[->] (A')--(0,4.5) node[left]{$z$};
			\draw (B')--(B)--(C)--(D)--(D')--(A')--(B')--(C')--(D') (C)--(C');
			\draw[dashed] (A')--(A)--(B) (A)--(D);
			\foreach \x/\g in {A/-90,B/180,C/-70,D/40,A'/40,B'/180,C'/10,D'/20}\draw[fill=black] (\x) circle (.04) +(\g:.6)node{\small$\x$};
		\end{tikzpicture}
	}
	\loigiai{
	}
\end{ex} 
\begin{ex}
	\immini{Cho hình lập phương $ABCD.A'B'C'D'$ có cạnh bằng 2. Với hệ toạ độ $Oxyz$ được thiết lập như hình bên (gốc tọa độ $O$ trùng với điểm $A$), tọa độ điểm $C'$ là
		\haicot
		{$C'(2;2;0)$}
		{\True $C'(2;2;2)$}
		{$C'(2;2;0)$}
		{$C'(2;0;2)$}
	}{
		\begin{tikzpicture}[scale=0.65, font=\small,>=stealth]
			\path
			(0,0) coordinate (A)
			(-2,-2) coordinate (B)
			(5,0) coordinate (D)
			(0,3) coordinate (A')
			($(B)+(D)-(A)$)coordinate (C)
			($(A')+(B)-(A)$)coordinate (B')
			($(B')+(C)-(B)$)coordinate (C')
			($(A')+(D)-(A)$)coordinate (D')
			;
			\draw[->] (D)--(6.7,0) node[below]{$y$};
			\draw[->] (B)--(-3,-3) node[below]{$x$};
			\draw[->] (A')--(0,4.5) node[left]{$z$};
			\draw (B')--(B)--(C)--(D)--(D')--(A')--(B')--(C')--(D') (C)--(C');
			\draw[dashed] (A')--(A)--(B) (A)--(D);
			\foreach \x/\g in {A/-90,B/180,C/-70,D/40,A'/40,B'/180,C'/10,D'/20}\draw[fill=black] (\x) circle (.04) +(\g:.6)node{\small$\x$};
		\end{tikzpicture}
	}
	\loigiai{
	}
\end{ex} 


\begin{ex}
	\immini{Cho hình chóp tứ giác đều $S.ABCD$ có cạnh đáy bằng $a\sqrt{2}$, cạnh bên bằng $a\sqrt{5}$. Gọi $O$ là tâm của hình vuông $ABCD$. Với hệ toạ độ $Oxyz$ được thiết lập như hình bên (gốc tọa độ $O$ trùng với tâm hình vuông $ABCD$), tọa độ $\vec{SC}$ là
		\choice
		{$\vec{SC}=(2a;0;-2a)$}
		{$\vec{SC}=(2a;-a;-2a)$}
		{\True $\vec{SC}=(a;0;-2a)$}
		{$\vec{SC}=(a;0;2a)$}}{
		\begin{tikzpicture}[scale=0.65, font=\small,>=stealth]
			\path
			(0,0) coordinate (A)
			(-3,-2) coordinate (B)
			(5,0) coordinate (D)
			($(B)+(D)-(A)$)coordinate (C)
			($(A)!0.5!(C)$)coordinate (O)
			($(O)+(0,4)$)coordinate (S)
			;
			\draw[->] (D)--(7,0.5) node[below]{$y$};
			\draw[->] (C)--(3,-3) node[below]{$x$};
			\draw[->] (S)--(1,4.5) node[left]{$z$};
			\draw (C)--(D)--(S)--(C)--(B)--(S);
			\draw[dashed] (S)--(A)--(D)--(B)--(A)--(C) (S)--(O);
			\pic[draw,thin,angle radius=2mm] {right angle = C--O--D};
			\foreach \x/\g in {A/180,B/-90,C/-100,D/-30,S/10,O/-90}\draw[fill=black] (\x) circle (.04) +(\g:.6)node{\small$\x$};
		\end{tikzpicture}}
	\loigiai{
	}
\end{ex} 

\begin{ex}%[2H2H2-2]
	\immini{
		Cho tứ diện $SABC$ có $ABC$ là tam giác vuông tại $B$, $BC=3$, $BA=2$, $SA$ vuông góc với mặt phẳng $(ABC)$ và có độ dài bằng $2$. Với hệ toạ độ $Oxyz$ được thiết lập như hình bên (gốc tọa độ $O$ trùng với điểm $B$), tìm khẳng định \textbf{sai}.
		\haicot
		{$A(0; 2; 0)$}
		{$B(0; 0; 0)$}
		{$C(0; 0; 3)$}
		{\True $S(-2; 2; 2)$}
	}{
		\begin{tikzpicture}[scale=1, font=\small, line join=round, line cap=round, >=Stealth]
			\path
			(0:0) coordinate (B)
			(20:4) coordinate (x)
			(90:3) coordinate (z)
			(130:2) coordinate (y)
			($(B)!.7!(x)$) coordinate (C)
			($(B)!4/6!(y)$) coordinate (A)
			($(B)!3/5!(z)$) coordinate (H)
			($(A)+(H)-(B)$) coordinate (S)
			;
			\draw[->] (B)--(x);
			\draw[->] (B)--(y);
			\draw[->] (B)--(z);
			\draw[dashed] 	(A)--(C)
			;
			\draw (A)--(S) (B)--(S)--(C);
			\pic[draw,angle radius=2mm]{right angle=C--B--A}
			pic[draw,angle radius=2mm]{right angle=C--B--H}
			pic[draw,angle radius=2mm]{right angle=H--B--A}
			;
			\foreach \x/\g in {B/-90,x/90,y/180,z/0,C/-90,A/210,H/0,S/90}
			\draw[fill=black] 	(\x)
			($(\g:.2)+(\x)$) node {$\x$};
		\end{tikzpicture}
	}
	\loigiai{
	}
\end{ex} 

\begin{ex}%[2H2H2-2]
	\immini{Cho hình chóp $S.ABC$ có đáy $ABC$ là tam giác đều cạnh bằng $2$, $SA$ vuông góc với đáy và $SA =1$. Với hệ toạ độ $Oxyz$ được thiết lập như hình bên (gốc tọa độ $O$ trùng với trung điểm của đoạn $BC$), hãy tìm toạ độ điểm $S$.
		\haicot
		{$S(0;\sqrt{3};1)$}
		{$S(0;\sqrt{3};1)$}
		{$S(0;\sqrt{3};1)$}
		{$S(0;\sqrt{3};1)$}
	}{
		\begin{tikzpicture}[ font = \small, scale =1,>=stealth]
			\path
			(0:0) coordinate (A)
			++(0:4) coordinate (C)
			++(-160:3)coordinate(B)
			(A)++(90:2) coordinate (S)
			($(B)!1/2!(C)$) coordinate (O)
			($(S)+(O)-(A)$) coordinate (H)
			($(O)!1.4!(A)$) coordinate (y) node[above]{$y$}
			($(O)!1.7!(C)$) coordinate (x) node[above]{$x$}
			($(O)!1.4!(H)$) coordinate (z) node[right]{$z$}
			(intersection of S--C and O--H) coordinate (t)
			;
			\draw (S)--(A)--(B)--(C) (S)--(B) (O)--(z) (S)--(H) (C)--(t)
			;
			\draw[dashed] (C)--(A)--(O) (S)--(t)
			;
			\draw[->] (C)--(x);
			\draw[->] (A)--(y);
			\draw[->] (H)--(z);
			\pic[draw,thin,angle radius=2mm] {right angle = A--O--B}
			pic[draw,thin,angle radius=2mm] {right angle = O--H--S}
			pic[draw,thin,angle radius=2mm] {right angle = H--O--C}
			;
			\foreach \x/\g in {A/-100,C/-50,B/-90,O/-70,H/0,S/90}
			\fill (\x) circle (1pt)
			+(\g:3mm) node{$\x$};
		\end{tikzpicture}
	}
	\loigiai{
	}
\end{ex} 
\begin{ex}%[2H2V2-6]
	\immini{Ở một sân bay, vị trí của máy bay được xác định bởi điểm $M$ Trong KG $Oxyz$ như hình bên. Gọi $H$ là hình chiếu vuông góc của $M$ xuống mặt phẳng $(Oxy)$. Cho biết $OM = 50$, $\left(\overrightarrow{i},\overrightarrow{OH}\right) = 64^\circ$, $\left(\overrightarrow{OH},\overrightarrow{OM}\right) = 48^\circ$. Tìm toạ độ của điểm $M$.
		\choice
		{$M(14{,}7; 30{,}1; 37{,}2)$}
		{$M(14{,}7; 30{,}1; 37{,}2)$}
		{$M(14{,}7; 30{,}1; 37{,}2)$}
		{$M(14{,}7; 30{,}1; 37{,}2)$}
	}{
		\begin{tikzpicture}[scale=0.85, font=\small,>=stealth]
			\path
			(0,0) coordinate (O)
			(-2,-2) coordinate (A)
			(3,-2) coordinate (H)
			(5,0) coordinate (B)
			(3,1) coordinate (M)
			(0,3) coordinate (C)
			;
			\draw[->] (0,0)--(6,0) node[below]{$y$};
			\draw[->] (0,0)--(-3,-3) node[below]{$x$};
			\draw[->] (0,0)--(0,4) node[left]{$z$};
			\draw[dashed] (C)--(M)--(H)--(A) (O)--(H)--(B);
			\draw[fill=blue] (M)circle (0.15)--(O)node[midway,sloped,scale=1,above]{$50$};
			\foreach \x/\g in {O/160,H/-90,M/30,A/-80,B/-70,C/30}\draw[fill=black] (\x) circle (.05) +(\g:.5)node{\small$\x$};
			\foreach \x/\y/\z in {H/A/O,H/B/O,M/C/O}{\path pic[draw,angle radius=5pt]{right angle= \x--\y--\z};}
			\draw pic["\scriptsize$48^\circ$",draw,angle eccentricity=1.9,angle radius=0.45cm]{angle=H--O--M};
			\draw pic["\scriptsize$64^\circ$",draw,angle eccentricity=1.9,angle radius=0.3cm]{angle=A--O--H};
		\end{tikzpicture}
	}
	\loigiai{
		\immini{
			Tam giác $OMH$ vuông tại $H$, $OM = 50$; $\widehat{MOH} = 48^\circ$ nên ta có
			\begin{itemize}
				\item [$\bullet$] $OH = OM\cdot \cos 48 \approx 33{,}5$
				\item [$\bullet$] $OC = MH = OM \cdot \sin 48 \approx 37{,}2$.
			\end{itemize}
			Tam giác $OAH$ vuông tại $A$, $OH = 33{,}5$; $\widehat{AOH} = 64^\circ$ nên ta có
			\begin{itemize}
				\item [$\bullet$] $OA = OH\cdot \cos 64 \approx 14{,}7$,
				\item [$\bullet$] $OB = AH = OH\cdot \sin 64 \approx 30{,}1$.
			\end{itemize}
			Suy ra
			\begin{eqnarray*}
				\overrightarrow{OM} & = & \overrightarrow{OC} + \overrightarrow{OH} = \overrightarrow{OC} + \overrightarrow{OA}+\overrightarrow{OB} \\
				& = & 14{,}7\overrightarrow{i}+30{,}1\overrightarrow{j}+37{,}2\overrightarrow{k}.
			\end{eqnarray*}
			Vậy $M(14{,}7; 30{,}1; 37{,}2)$.
		}{
			\begin{tikzpicture}[scale=0.85, font=\small,>=stealth]
				\path
				(0,0) coordinate (O)
				(-2,-2) coordinate (A)
				(3,-2) coordinate (H)
				(5,0) coordinate (B)
				(3,1) coordinate (M)
				(0,3) coordinate (C)
				;
				\draw[->] (0,0)--(6,0) node[below]{$y$};
				\draw[->] (0,0)--(-3,-3) node[below]{$x$};
				\draw[->] (0,0)--(0,4) node[left]{$z$};
				\draw[dashed] (C)--(M)--(H)--(A) (O)--(H)--(B);
				\draw[fill=blue] (M)circle (0.15)--(O)node[midway,sloped,scale=1,above]{$50$};
				\foreach \x/\g in {O/160,H/-90,M/30,A/-80,B/-70,C/30}\draw[fill=black] (\x) circle (.05) +(\g:.5)node{\small$\x$};
				\foreach \x/\y/\z in {H/A/O,H/B/O,M/C/O}{\path pic[draw,angle radius=5pt]{right angle= \x--\y--\z};}
				\draw pic["\scriptsize$48^\circ$",draw,angle eccentricity=1.9,angle radius=0.45cm]{angle=H--O--M};
				\draw pic["\scriptsize$64^\circ$",draw,angle eccentricity=1.9,angle radius=0.3cm]{angle=A--O--H};
			\end{tikzpicture}
		}
	}
\end{ex} 

\Closesolutionfile{ans}
\BTTF
\Opensolutionfile{ans}[ans/2H2-B2-d2-2]
\begin{ex}
	\immini{Cho hình chóp $S.ABCD$ có đáy $ABCD$ là hình chữ nhật, $AB=1$, $AD=2$, $SA$ vuông góc với mặt đáy và $SA=3$. Với hệ toạ độ $Oxyz$ được thiết lập như sau: Gốc tọa độ $O$ trùng với điểm $A$, các véc tơ $\vec{AB}$, $\vec{AD}$, $\vec{AS}$ lần lượt cùng hướng với $\vec{i}$, $\vec{j}$ và $\vec{k}$. Xét tính đúng sai của các khẳng định sau
		\choiceTF
		{\True Tọa độ $D(0;2;0)$}
		{Tọa độ $C(1;2;3)$}
		{\True Tọa độ $S(2;0;0)$}
		{Tọa độ $I(1;1;0)$}
	}{
		\begin{tikzpicture}[scale=0.6, font=\small,>=stealth]
			\path
			(0,0) coordinate (A)
			(-2,-2) coordinate (B)
			(5,0) coordinate (D)
			($(B)+(D)-(A)$)coordinate (C)
			($(A)!0.5!(C)$)coordinate (I)
			($(A)+(0,3)$)coordinate (S)
			;
			\draw (C)--(D)--(S)--(C)--(B)--(S);
			\draw[dashed] (S)--(A)--(D)--(B)--(A)--(C);
			\pic[draw,thin,angle radius=2mm] {right angle = B--A--D};
			\foreach \x/\g in {A/180,B/-90,C/-100,D/-80,S/90,I/-90}\draw[fill=black] (\x) circle (.04) +(\g:.5)node{\small$\x$};
		\end{tikzpicture}	}
	\loigiai{
		\immini{
			Với hệ trục đã chọn như hình vẽ thì
			\begin{enumerate}[a)]
				\item Điểm $D \in Oy$ và $AD=2$ nên $D(0;2;0)$.
				\item Điểm $C \in (Oxy)$ và có hình chiếu lên $Ox$, $Oy$ lần lượt là điểm $B$ và $D$.\\
				      Do $AB=1$ và $AD=2$ nên $C(2;2;0)$.
				\item Điểm $S \in Oz$ và $AS=3$ nên $S(0;0;3)$.
				\item Điểm $I \in (Oxy)$ và và có hình chiếu lên $Ox$, $Oy$ lần lượt là trung điểm của $AB$ và $AD$ nên $I(0,5;1;0)$.
			\end{enumerate}}{
			\begin{tikzpicture}[scale=0.6, font=\small,>=stealth]
				\path
				(0,0) coordinate (A)
				(-2,-2) coordinate (B)
				(5,0) coordinate (D)
				($(B)+(D)-(A)$)coordinate (C)
				($(A)!0.5!(C)$)coordinate (I)
				($(A)+(0,3)$)coordinate (S)
				;
				\draw[->] (D)--(7,0) node[below]{$y$};
				\draw[->] (B)--(-3,-3) node[below]{$x$};
				\draw[->] (S)--(0,4) node[left]{$z$};
				\draw (C)--(D)--(S)--(C)--(B)--(S);
				\draw[dashed] (S)--(A)--(D)--(B)--(A)--(C);
				\pic[draw,thin,angle radius=2mm] {right angle = B--A--D};
				\foreach \x/\g in {A/180,B/-90,C/-100,D/-80,S/170,I/-90}\draw[fill=black] (\x) circle (.04) +(\g:.5)node{\small$\x$};
			\end{tikzpicture}}
	}
\end{ex} 


\begin{ex}
	\immini{Cho hình lập phương $ABCD.A'B'C'D'$ có cạnh bằng $2$. Với hệ toạ độ $Oxyz$ được thiết lập như hình bên (gốc tọa độ $O$ trùng với tâm hình vuông $ABCD$), hãy xét tính đúng sai của các khẳng định sau:
		\choiceTF
		{Tọa độ $A(-1;0;0)$}
		{\True $\vec{AC'}=(2\sqrt{2};0;2)$}
		{\True Tọa độ $D'(0;\sqrt{2};2)$}
		{$\vec{BD'}=(0;0;2)$}
	}{
		\begin{tikzpicture}[scale=0.45, font=\small,>=stealth]
			\path
			(0,0) coordinate (A)
			(-2,-2) coordinate (B)
			(6,0) coordinate (D)
			(0,4) coordinate (A')
			($(B)+(D)-(A)$)coordinate (C)
			($(A')+(B)-(A)$)coordinate (B')
			($(B')+(C)-(B)$)coordinate (C')
			($(A')+(D)-(A)$)coordinate (D')
			($(A)!0.5!(C)$)coordinate (O)
			($(A')!0.5!(C')$)coordinate (O')
			;
			\draw[->] (D)--(8,0.5) node[below]{$y$};
			\draw[->] (C)--(6,-3) node[below]{$x$};
			\draw[->] (O')--(2,5.5) node[left]{$z$};
			\draw (B')--(B)--(C)--(D)--(D')--(A')--(B')--(C')--(D') (C)--(C')--(A') (B')--(D');
			\draw[dashed] (A')--(A)--(B)--(D)--(A)--(C) (O)--(O');
			\pic[draw,thin,angle radius=2mm] {right angle = C--O--D};
			\foreach \x/\g in {A/-90,B/180,C/-70,D/-40,A'/40,B'/180,C'/10,D'/20,O/-90}\draw[fill=black] (\x) circle (.04) +(\g:.65)node{\small$\x$};
		\end{tikzpicture}
	}
	\loigiai{
		Độ dài $AC=2\sqrt{2}$. Với hệ trục $Oxyz$ đã chọn như hình vẽ thì
		\begin{enumerate}[a)]
			\item Điểm $A \in Ox$, nằm ngược chiều dương và $OA=\sqrt{2}$ nên $A(-\sqrt{2};0;0)$.
			\item Tọa độ $C'(\sqrt{2};0;2)$. Suy ra $\vec{AC'}=(2\sqrt{2};0;2)$.
			\item Điểm $D'$ có hình chiếu vuông góc xuống $(Oxy)$ là điểm $D(0;\sqrt{2};0)$ và $DD'=2$ nên $D'(0;\sqrt{2};2)$.
			\item Tọa độ $B(0;-\sqrt{2};0)$, $D'(0;\sqrt{2};2)$. Suy ra $\vec{BD'}=(0;2\sqrt{2};2)$.
		\end{enumerate}
	}
\end{ex} 

\begin{ex}
	\immini{Cho hình lăng trụ $ABC.A'B'C'$ có đáy $ABC$ là tam giác đều cạnh bằng $2$ như hình vẽ. Hình chiếu vuông góc của $A'$ lên $(ABC)$ trùng với trung điểm cạnh $AB$, góc $\widehat{A'AO}=60^\circ$. Với hệ toạ độ $Oxyz$ được thiết lập như hình bên (gốc tọa độ $O$ trùng với trung điểm của đoạn $BC$), hãy xét tính đúng sai của các khẳng định sau:
		\choiceTF
		{\True Tọa độ điểm $A(-1;0;0)$}
		{\True Tọa độ điểm $C(0;\sqrt{3};0)$}
		{Tọa độ điểm $A'(0;-1;\sqrt{3})$}
		{\True Tọa độ điểm $C'\big(1;\sqrt{3};\sqrt{3}\big)$}
	}{
		\begin{tikzpicture}[scale=0.7, font=\small,>=stealth]
			\path
			(0,0) coordinate (A)
			(2,-2) coordinate (B)
			(5,0) coordinate (C)
			(1,4) coordinate (A')
			($(A)!0.5!(B)$)coordinate (O)
			($(A')+(B)-(A)$)coordinate (B')
			($(A')+(C)-(A)$)coordinate (C')
			;
			\draw[->] (B)--(3,-3) node[below]{$x$};
			\draw[->] (C)--(7,0.5) node[below]{$y$};
			\draw[->] (A')--(1,5) node[left]{$z$};
			\draw (B)--(A)--(A')--(B')--(B)--(C)--(C')--(B') (A')--(C') (O)--(A');
			\draw[dashed] (A)--(C)--(O);
			\pic[draw,thin,angle radius=2mm] {right angle = C--O--B};
			\pic[draw,thin,angle radius=2mm] {right angle = A'--O--B};
			\pic[draw,thin,angle radius=2mm] {right angle = A'--O--C};
			\foreach \x/\g in {A/180,B/-90,C/-90,A'/180,B'/-20,C'/0,O/-100}\draw[fill=black] (\x) circle (.05) +(\g:.5)node{\small$\x$};
		\end{tikzpicture}}
	\loigiai{
		Độ dài $OC=2.\dfrac{\sqrt{3}}{2}=\sqrt{3}$. $OA'=OA.\tan60^\circ=\sqrt{3}$. Với hệ trục $Oxyz$ đã chọn như hình vẽ trên thì
		\begin{enumerate}[a)]
			\item Điểm $A \in Ox$, nằm ngược chiều dương và $OA=1$ nên $A(-1;0;0)$.
			\item Điểm $A' \in Oy$, nằm cùng chiều dương và $OC=\sqrt{3}$ nên $C(0;\sqrt{3};0)$.
			\item $A' \in Oz$, nằm cùng chiều dương và $OA'=\sqrt{3}$ nên $A'(0;0;\sqrt{3})$.
			\item Gọi $C'(x;y;z)$. Ta có
			      $$\vec{A'C'}=\vec{AC} \Leftrightarrow\heva{&x-0=1\\&y-0=\sqrt{3}\\&z-\sqrt{3}=0}\Leftrightarrow\heva{&x=1\\&y=\sqrt{3}\\&z=\sqrt{3}}.$$
		\end{enumerate}
	}
\end{ex} 


\Closesolutionfile{ans}
