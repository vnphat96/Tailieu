PHẦN I. Câu trắc nghiệm nhiều phương án lựa chọn. Thí sinh trả lời từ câu 1 đến câu 12. Mỗi câu hỏi 
 thí sinh chỉ chọn một phương án.
%Câu 1
\begin{ex}
Trong không gian cho vectơ $\vec{AB}$. Khi đó:
\choice
{Giá của vectơ $\vec{AB}$ là $\vec{AB}$}
{Giá của vectơ$\vec{AB}$ là $\left| \vec{AB} \right|$}
{\True Giá của vectơ$\vec{AB}$ là đường thẳng $AB$}
{Giá của vectơ $\vec{AB}$ là đoạn thẳng $AB$}
\loigiai{
A sai vì giá của vectơ là đường thẳng không phải là một vectơ.\\
B sai vì giá của vectơ là đường thẳng không phải là độ dài.\\
D sai vì giá của vectơ là đường thẳng không phải là một đoạn thẳng
}
\end{ex}
%Câu 2
\begin{ex}
 Cho hình hộp chữ nhật $ABCD \cdot A'B'C'D'$. Trong các vectơ dưới đây, vectơ nào cùng phương với vectơ $\vec{AB}$?
{\centering\color{red} HINH O DAY}
\choice
{Vectơ$\vec{AD}$}
{Vectơ$\vec{CC'}$}
{Vectơ$\vec{BD}$}
{\True Vectơ$\vec{CD}$}
\loigiai{
Giá của vectơ $\vec{AB}$ là đường thẳng $AB$\\
Giá của vectơ $\vec{CD}$ là đường thẳng $CD$\\
Mà $AB \parallel CD$\\
Do đó vectơ $\vec{AB}$ cùng phương với vectơ $\vec{CD}$
}
\end{ex}
%Câu 3
\begin{ex}
 Hình ảnh dưới đây là phân độ của 8 hướng trên la bàn. Mệnh đề nào sau đây sai?
{\centering\color{red} HINH O DAY}
\choice
{Hai vectơ $\vec{a}$ và $\vec{c}$ cùng phương}
{Hai vectơ $\vec{a}$ và $\vec{c}$ ngược hướng}
{Hai vectơ $\vec{b}$ và $\vec{d}$ cùng phương}
{\True Hai vectơ $\vec{a}$ và $\vec{c}$ cùng hướng}
\loigiai{
Phương án D sai vì hai vectơ $\vec{a}$ và $\vec{c}$ ngược hướng
}
\end{ex}
%Câu 4
\begin{ex}
 Cho hình hộp $ABCD \cdot A'B'C'D'$. Vectơ $\vec{u}=\vec{A'A}+\vec{A'B'}+\vec{A'D'}$ bằng vectơ nào dưới đây?
\choice
{\True $\vec{A'C}$}
{$\vec{CA'}$}
{$\vec{AC'}$}
{$\vec{C'A}$}
\loigiai{
{\centering\color{red} HINH O DAY}\\
 Do $A'B'BA$ là hình bình hành nên $\vec{A'A}+\vec{A'B'}=\vec{A'B}$. Lại có, $A'BCD'$ cũng là hình bình hành nên $\vec{A'B}+\vec{A'D'}=\vec{A'C}$. Vậy $\vec{A'A}+\vec{A'B'}+\vec{A'D'}=\vec{A'C}$
}
\end{ex}
%Câu 5
\begin{ex}
 Cho hình lăng trụ tam giác $ABC \cdot A'B'C'$. Đặt $\vec{AA'}=\vec{a},\vec{AB}=\vec{b},\vec{AC}=\vec{c},\vec{BC}=\vec{d}$. Trong các biểu thức vec tơ sau đây, biểu thức nào là đúng?
\choice
{$\vec{a}=\vec{b}+\vec{c}$}
{$\vec{a}+\vec{b}+\vec{c}+\vec{d}=\vec{0}$}
{\True $\vec{b}-\vec{c}+\vec{d}=\vec{0}$}
{$\vec{a}+\vec{b}+\vec{c}=\vec{d}$}
\loigiai{
{\centering\color{red} HINH O DAY}\\
Ta có: $\vec{b}-\vec{c}+\vec{d}=\vec{AB}-\vec{AC}+\vec{BC}=\vec{CB}+\vec{BC}=\vec{0}$
}
\end{ex}
%Câu 6
\begin{ex}
 Cho lập phương $ABCD \cdot A'B'C'D'$ có độ dài mỗi cạnh bằng 1. Tính độ dài của vectơ $\vec{AC}+\vec{C'D'}$.
{\centering\color{red} HINH O DAY}
\choice
{$\sqrt{3}$}
{$\sqrt{2}$}
{\True $1$}
{$2\sqrt{2}$}
\loigiai{
Ta có: $A'C'CA$ là hình chữ nhật nên $\vec{A'C'}=\vec{AC}$.\\
Khi đó, $\vec{AC}+\vec{C'D'}=\vec{A'C'}+\vec{C'D'}=\vec{A'D'}$. Vậy $\left| \vec{AC}+\vec{C'D'} \right|=\left| \vec{A'D'} \right|=A'D'=1$
}
\end{ex}
%Câu 7
\begin{ex}
Cho $O$ là tâm hình bình hành $ABCD$. Hỏi vectơ $\left(\vec{AO}-\vec{DO}\right)$ bằng vectơ nào?
\choice
{$\vec{BA}$}
{\True $\vec{AD}$}
{$\vec{DC}$}
{$\vec{AC}$}
\loigiai{
{\centering\color{red} HINH O DAY}\\
Ta có: $\vec{AO}-\vec{DO}=\vec{AO}+\vec{OD}=\vec{AD}$. Chọn B
}
\end{ex}
%Câu 8
\begin{ex}
Cho ba điểm phân biệt $A$, $B,C$. Nếu $\vec{AB}=-3\vec{AC}$ thì đẳng thức nào dưới đây đúng?
\choice
{$\vec{BC}=-4\vec{AC}$}
{$\vec{BC}=-2\vec{AC}$}
{$\vec{BC}=2\vec{AC}$}
{\True $\vec{BC}=4\vec{AC}$}
\loigiai{
Ta có: $\vec{AB}=-3\vec{AC}\Leftrightarrow \vec{CB}-\vec{CA}=-3\vec{AC}\Leftrightarrow \vec{AC}+3\vec{AC}=-\vec{CB}\Leftrightarrow \vec{BC}=4\vec{AC}$. Chọn D
}
\end{ex}
%Câu 9
\begin{ex}
Cho tam giác $ABC$ có điểm $O$ thỏa mãn: $\left| \vec{OA}+\vec{OB}-2\vec{OC} \right|=\left| \vec{OA}-\vec{OB} \right|$. Khẳng định nào sau đây là đúng?
\choice
{Tam giác $ABC$ đều }
{Tam giác $ABC$ cân tại $C$}
{\True Tam giác $ABC$ vuông tại $C$ }
{Tam giác $ABC$ cân tại $B$}
\loigiai{
Gọi $M$ là trung điểm $AB$, ta có $\vec{OA}+\vec{OB}=2\vec{OM}$.\\
Do đó, $\left| \vec{OA}+\vec{OB}-2\vec{OC} \right|=\left| \vec{OA}-\vec{OB} \right|\Leftrightarrow \left| 2\vec{OM}-2\vec{OC} \right|=\left| \vec{BA} \right|\Leftrightarrow 2\left| \vec{CM} \right|=BA\Leftrightarrow CM=\dfrac{1}{2}BA \cdot (1)$\\
Vì $M$ là trung điểm $AB$ nên$CM$ là đường trung tuyến của $\triangle ABC$, Từ (1) suy ra, tam giác $\triangle ABC$ vuông tại $C$
}
\end{ex}
%Câu 10
\begin{ex}
 Cho hình hộp $ABCD \cdot A'B'C'D'$. Đẳng thức nào dưới đây là đúng?
\choice
{$\vec{AC'}=\vec{AB}+\vec{AD}+\vec{AC}$}
{$\vec{AC'}=\vec{AA'}+\vec{AD}+\vec{AC}$}
{\True $\vec{AC'}=\vec{AB'}+\vec{AD}$}
{$\vec{AC'}=\vec{AC}+\vec{AB}+\vec{AA'}$}
\loigiai{
{\centering\color{red} HINH O DAY}\\
 Do $AB'C'D$ là hình bình hành nên $\vec{AC'}=\vec{A'B'}+\vec{AD}$. Chọn đáp án là C
}
\end{ex}
%Câu 11
\begin{ex}
 Cho hình lập phương $ABCD \cdot A'B'C'D'$ cạnh bằng $a$. Tích vô hướng của hai vectơ $\vec{DD'}$ và $\vec{A'C'}$ bằng
\choice
{$\sqrt{2}a^2$}
{$a^2$}
{$-\sqrt{2}a^2$}
{\True $0$}
\loigiai{
{\centering\color{red} HINH O DAY}\\
Ta có: $\vec{A'C'}=\vec{A'D'}+\vec{D'C'}$, mà tứ giác $ADD'A'$ và $DCC'D'$ là hình vuông nên $\vec{DD'} \cdot \vec{A'D'}=\vec{DD'} \cdot \vec{D'C'}=0$. Do đó $\vec{DD'} \cdot \left(\vec{A'D'}+\vec{D'C'}\right)=0$. Chọn đáp án D
}
\end{ex}
%Câu 12
\begin{ex}
 Cho lập phươn g $ABCD \cdot A'B'C'D'$ có độ dài cạnh bằng $a$. Tính độ dài của vectơ $\vec{AD'}+\vec{BA'}$.
{\centering\color{red} HINH O DAY}
\choice
{$\sqrt{3}a$}
{$\sqrt{2}a$}
{\True $\sqrt{6}a$}
{$2\sqrt{3}a$}
\loigiai{
Gọi $O'$ là tâm của hình vuông $A'B'C'D'$.\\
Ta có $ABC'D'$ là hình bình hành nên $\vec{AD'}=\vec{BC'}$, do đó $\vec{BA'}+\vec{AD'}=\vec{BA'}+\vec{BC'}=2\vec{BO'}$.\\
Tam giác $BA'C'$ là tam giác đều cạnh $a\sqrt{2}$ nên $BO'=\dfrac{\sqrt{3}}{2}a\sqrt{2}=\dfrac{\sqrt{6}}{2}a$.\\
Từ đó độ dài của vectơ $\vec{AD'}+\vec{BA'}$ bằng $\sqrt{6}a$.\\
{\centering\color{red} HINH O DAY}\\
PHẦN II. Câu trắc nghiệm đúng sai. Thí sinh trả lời từ câu 1 đến câu 4. Trong mỗi ý a), b), c), d) ở mỗi câu, thí sinh chọn đúng hoặc sai.%Câu 1\\
\begin{ex}\\
Cho hình chóp $S \cdot ABCD$ có đáy $ABCD$ là hình chữ nhật. Biết rằng: cạnh $AB=a$, $AD=2a$, cạnh bên $SA=2a$ và vuông góc với mặt đáy. Gọi $M$, $N$ lần lượt là trung điểm của các cạnh $SB$, $SD$. Các mệnh đề sau đúng hay sai ?\\
a) Hai vectơ $\vec{AB\,}$, $\vec{CD\,}$ là hai vectơ cùng phương, cùng hướng.\\
b) Góc giữa hai vectơ $\vec{SC\,}$ và $\vec{AC\,}$ bằng $60^\circ $.\\
c) Tích vô hướng $\vec{AM\,}\cdot \,\vec{AB\,}=\dfrac{a^2}{2}$.\\
d) Độ dài của vectơ $\vec{AM\,}-\vec{AN\,}$ là $\dfrac{a\sqrt{3}}{2}$}\\
\loigiai{
a) Sai.\\
b) Đúng.\\
c) Sai.\\
{\centering\color{red} HINH O DAY}\\
a) Ta thấy: $ABCD$ là hình chữ nhật nên $AB//CD$.\\
Suy ra: hai vectơ $\vec{AB\,}$, $\vec{CD\,}$ là hai vectơ cùng phương, ngược hướng.\\
Mệnh đề a) sai.\\
b) Ta có: $ABCD$ là hình chữ nhật nên: $AC=\sqrt{AB^2+AD^2}=a\sqrt{5}$.\\
Hình chóp $S \cdot ABCD$ có $SA$ vuông góc với mặt đáy nên tam giác $SAC$ là tam giác vuông tại $A$. Suy ra: $\tan \widehat{SCA\,}=\dfrac{SA}{AC}=\dfrac{2a}{a\sqrt{5}}\Rightarrow \widehat{SCA\,}\approx 41^\circ 48’ $.\\
Ta có: $\left(\vec{SC\,},\vec{AC\,}\right)=\left(\vec{CS\,},\vec{CA\,}\right)=\widehat{SCA}\approx 41^\circ 48’ $.\\
Mệnh đề b) sai.\\
c) Hình chóp $S \cdot ABCD$ có $SA$ vuông góc với mặt đáy nên tam giác $SAB$ là tam giác vuông tại $A$.\\
Suy ra: $SB=\sqrt{SA^2+AB^2}=a\sqrt{5}$.\\
Trong tam giác $SAB$ vuông tại $A$ có $AM$ là đường trung tuyến nên: \\
$AM=\dfrac{1}{2}SB=\dfrac{a\sqrt{5}}{2}$.\\
Lại có: $M$ là trung điểm của $SB$ nên $MB=\dfrac{1}{2}SB=\dfrac{a\sqrt{5}}{2}$. \\
Ta tính được: $\cos \widehat{MAB\,}=\dfrac{MA^2+AB^2-MB^2}{2MA \cdot \,AB}=\dfrac{\sqrt{5}}{5}$.\\
Mà: $\left(\vec{AM\,},\vec{AB\,}\right)=\widehat{MAB\,}$, suy ra: \\
$\vec{AM\,}\cdot \,\vec{AB\,}=\left| \vec{AM\,} \right| \cdot \,\left| \vec{AB\,} \right| \cdot \,\cos \left(\vec{AM\,},\vec{AB\,}\right)=\dfrac{a\sqrt{5}}{2} \cdot \,a \cdot \,\dfrac{\sqrt{5}}{5}=\dfrac{a^2}{2}$.\\
Mệnh đề c) đúng.\\
d) Ta có: $M$, $N$ lần lượt là trung điểm của các cạnh $SB$, $SD$ nên $MN$ là đường trung bình của tam giác $SBD$. Do đó: $MN=\dfrac{1}{2}BD=\sqrt{AB^2+AD^2}=\dfrac{a\sqrt{5}}{2}$.\\
Suy ra: $\left| \vec{AM\,}-\vec{AN\,} \right|=\left| \vec{MN\,} \right|=\dfrac{a\sqrt{5}}{2}$. \\
Mệnh đề d) sai
}
\end{ex}
%Câu 2
\begin{ex}
Cho hình lập phương $ABCD \cdot A’ B’ C’ D’ $ có cạnh bằng $a$. Trên các cạnh $AA’ $, $CC’ $ lần lượt lấy các điểm $M$, $N$ sao cho $AM=\dfrac{2}{3}AA’ $, $CN=NC’ $. Các mệnh đề sau đúng hay sai ?
a) Góc giữa hai vectơ $\vec{AN\,}$ và $\vec{AC\,}$ bằng $60^\circ $.
b) Độ dài của vectơ $\vec{MN\,}+\vec{AM\,}$ là $\dfrac{3a}{2}$.
c) Tích vô hướng $\vec{AN\,}\cdot \,\vec{AC\,}=a^2$.
d) Tích vô hướng $\vec{MN\,}\cdot \,\vec{A’ C’ \,}=2a^2$}
\loigiai{
a) Sai.\\
b) Đúng.\\
c) Sai.\\
d) Đúng.\\
{\centering\color{red} HINH O DAY}\\
a) Ta có: $AC=\sqrt{AB^2+AC^2}=a\sqrt{2}$.\\
Lại có: $CN=NC’ $ nên $CN=NC’ =\dfrac{a}{2}$.\\
$ABCD \cdot A’ B’ C’ D’ $ là hình lập phương nên tam giác $NAC$ là tam giác vuông tại $C$.\\
Suy ra: $\tan NAC=\dfrac{CN}{AC}=\dfrac{\sqrt{2}}{4}\Rightarrow \widehat{NAC\,}\approx 19^\circ 28’ $\\
Ta có: $\left(\vec{AN\,},\vec{AC\,}\right)=\widehat{NAC\,}\approx 19^\circ 28’ $.\\
Mệnh đề a) sai.\\
b) Trong tam giác $NAC$ vuông tại $C$ có: $AN=\sqrt{AC^2+CN^2}=\dfrac{3a}{2}$.\\
Ta có: $\left| \vec{MN\,}+\vec{AM\,} \right|=\left| \vec{AN\,} \right|=\dfrac{3a}{2}$.\\
Mệnh đề b) đúng.\\
c) Ta có: $\tan \widehat{NAC\,}=\dfrac{\sqrt{2}}{4}\Rightarrow \cos \widehat{NAC\,}=\dfrac{2\sqrt{2}}{3}$ (Do $\widehat{NAC\,}<90^\circ $).\\
Do đó: $\vec{AN\,}\cdot \,\vec{AC\,}=\left| \vec{AN\,} \right| \cdot \,\left| \vec{AC\,} \right| \cdot \,\cos \left(\vec{AN\,},\vec{AC\,}\right)=\dfrac{3a}{2} \cdot \,a\sqrt{2} \cdot \,\dfrac{2\sqrt{2}}{3}=2a^2$.\\
Mệnh đề c) sai.\\
d) Trên cạnh $CC’ $ lấy điểm $M’ $ sao cho: $\dfrac{CM’ }{CC’ }=\dfrac{2}{3}$.\\
Suy ra: $\heva{& NM’ =NC’ -M’ C’ =\dfrac{a}{6} \\& MM’ //AC \\\\
 MM’ =AC=a\sqrt{2}} $.\\
Ta có: $\cos \widehat{NMM’ \,}=\dfrac{NM^2+M’ M^2-M’ N^2}{2 \cdot \,NM \cdot \,M’ M}=\dfrac{6\sqrt{146}}{73}$.\\
Mặt khác: $\left(\vec{MN\,},\vec{A’ C’ \,}\right)=\left(\vec{MN\,},\vec{MM’ \,}\right)=\widehat{NMM’ \,}$.\\
Tam giác $MNM’ $ vuông tại $M’ $ có: $MN=\sqrt{M’ N^2+M’ M^2}=\dfrac{a\sqrt{73}}{6}$.\\
Do đó: $\vec{MN\,}\cdot \,\vec{A’ C’ \,}=\left| \vec{MN\,} \right| \cdot \,\left| \vec{A’ C’ \,} \right| \cdot \,\cos \left(\vec{MN\,},\vec{A’ C’ \,}\right)=2a^2$.\\
Mệnh đề d) đúng
}
\end{ex}
%Câu 3
\begin{ex}
Cho hình lăng trụ đứng $ABC\,A'B'C'$ đáy là tam giác đều cạnh $2a,AA'=a\sqrt{3}$. $H$, $K$ lần lượt là trung điểm $BC$, $B'C'$. Các mệnh đề sau đúng hay sai ?
a) Hai vectơ $\vec{AH\,}$, $\vec{KA'\,}$ là hai vectơ cùng phương, cùng hướng.
b) Góc giữa hai vectơ $\vec{A'H\,}$ và $\vec{AH\,}$ bằng $60^\circ $.
c) Tích vô hướng $\vec{AK\,}\cdot \,\vec{AB'\,}=\dfrac{5a^2}{2}$.
d) Độ dài của vectơ $\vec{AK\,}+\vec{AH\,}$ là $\dfrac{a\sqrt{3}}{2}$}
\loigiai{

d) Sai
e) Sai
f) sai
g) Đúng
{\centering\color{red} HINH O DAY}\\
e) Ta có tam giác $\triangle ABC,\triangle A'B'C'$ đều cạnh $2a$ suy ra $A'K=AH=a\sqrt{3}$\\
Xét tứ giác $AA'KH$ có $AA'=KH=AH=A'K=a\sqrt{3}\,,AA'\bot AH$ suy ra tứ giác $AA'KH$ là hình vuông , từ đó dễ thấy hai vectơ $\vec{AH\,}$, $\vec{KA'\,}$ là hai vecto cùng phương ngược hướng.\\
Mệnh đề a) sai.\\
f) Ta có: $AA'KH$ là hình vuông suy ra $\widehat{A'HA}=45^\circ $\\
Có : $A'A\bot AH\Rightarrow \triangle A'AH$ vuông tại $A\Rightarrow \left(\vec{A'H\,},\vec{AH\,}\right)=\widehat{A'HA}=45^\circ $\\
Mệnh đề b) sai.\\
g) Ta có $\triangle AB'C'$ cân tại $A$, suy ra $AK\bot B'C'$, $AK=a\sqrt{6},B'K=a$\\
$AB'=\sqrt{AB^2+BB'^2}=\sqrt{4a^2+3a^2}=a\sqrt{7}$\\
Xét $\triangle AKB'$ có $\operatorname{Cos}\widehat{KAB'}=\dfrac{AK}{AB'}=\dfrac{a\sqrt{6}}{a\sqrt{7}}=\sqrt{\dfrac{6}{7}}$.\\
$\vec{AK} \cdot \vec{AB'}=AK \cdot AB' \cdot \operatorname{Cos}\widehat{KAB'}=a\sqrt{6} \cdot a\sqrt{7} \cdot \sqrt{\dfrac{6}{7}}=6a^2$\\
Mệnh đề c) sai.\\
h) Gọi $I$ là trung điểm $HK\Rightarrow IH=\dfrac{a\sqrt{3}}{2}$, $AI=\sqrt{IH^2+AH^2}=\sqrt{\dfrac{3a^2}{4}+3a^2}=\dfrac{a\sqrt{15}}{2}$.\\
Ta có : $\left| \vec{AK}+\vec{AH} \right|=\left| 2 \cdot \vec{AI} \right|=2AI=a\sqrt{15}$.\\
Mệnh đề d) đúng
}
\end{ex}
%Câu 4
\begin{ex}
Cho tứ diện đều $ABCD$ cạnh $a$. $E$ là điểm trên đoạn $CD$ sao cho $ED=2CE$. Các mệnh đề sau đúng hay sai ?
a) Có $6$ vectơ (khác vectơ $\vec{0}$) có điểm đầu và điểm cuối được tạo thành từ các đỉnh của tứ diện.
b) Góc giữa hai vectơ $\vec{AB\,}$ và $\vec{BC\,}$ bằng $60^\circ $.
c) Nếu $\vec{BE\,}=m\vec{BA\,}+n\vec{BC\,}+p\vec{BD\,}$ thì $m+n+p=\dfrac{2}{3}$.
d) Tích vô hướng $\vec{AD} \cdot \vec{BE}=\dfrac{a^2}{6}$}
\loigiai{
 {\centering\color{red} HINH O DAY}\\
a) Các vectơ (khác vectơ $\vec{0}$) có điểm đầu và điểm cuối được tạo thành từ các đỉnh của tứ diện là: $\vec{AB\,}$,$\vec{AC\,}$,$\vec{AD\,}$,$\vec{BA\,}$,$\vec{BC\,}$,$\vec{B\text{D}\,}$,$\vec{CA\,}$,$\vec{CB\,}$,$\vec{C\text{D}\,}$,$\vec{DA\,}$,$\vec{DB\,}$,$\vec{DC\,}$. Do đó có $12$ vectơ thỏa mãn yêu cầu. Vậy mệnh đề sai.\\
b) $(\vec{AB\,},\vec{BC})=180}^\circ}-(\vec{BA\,},\vec{BC})={{180}^\circ}-\widehat{ABC}={{120}^\circ}$. Vậy mệnh đề sai.\\
c) $\vec{BE\,}=\vec{BC\,}+\vec{CE\,}=\vec{BC\,}+\dfrac{1}{3}\vec{CD\,}=\vec{BC\,}+\dfrac{1}{3}\left(\vec{BD\,}-\vec{BC\,}\right)=\dfrac{2}{3}\vec{BC\,}+\dfrac{1}{3}\vec{BD\,}$.\\
Do đó $m=0$,$n=\dfrac{2}{3}$,$p=\dfrac{1}{3}$. Suy ra $m+n+p=1$.\\
Vậy mệnh đề sai.\\
d) \\
Ta có: $\vec{BE}=\vec{AE}-\vec{AB}=\left(\vec{AC}+\vec{CE}\right)-\vec{AB}=\vec{AC}+\dfrac{1}{3}\vec{CD}-\vec{AB}$\\
$=\vec{AC}+\dfrac{1}{3}\left(\vec{AD}-\vec{AC}\right)-\vec{AB}=\dfrac{2}{3}\vec{AC}+\dfrac{1}{3}\vec{AD}-\vec{AB}$\\
Suy ra: $\vec{AD} \cdot \vec{BE}=\vec{AD} \cdot \left(\dfrac{2}{3}\vec{AC}+\dfrac{1}{3}\vec{AD}-\vec{AB}\right)=\dfrac{2}{3} \cdot \vec{AD} \cdot \vec{AC}+\dfrac{1}{3} \cdot {{\vec{AD}}^2}-\vec{AD} \cdot \vec{AB}$\\
$=\dfrac{2}{3} \cdot a \cdot a \cdot \cos 60^\circ +\dfrac{1}{3}a^2-a \cdot a \cdot \cos 60^\circ =\dfrac{a^2}{6}$.\\
Vậy mệnh đề đúng. \\
PHẦN III. Câu trắc nghiệm trả lời ngắn. Thí sinh trả lời từ câu 1 đến câu 6.%Câu 1\\
\begin{ex}\\
 Cho tứ diện $ABCD$. Trên các cạnh $AD$ và $BC$ lần lượt lấy $M$, $N$ sao cho $AM=3MD$, $BN=3NC$. Gọi $P$, $Q$ lần lượt là trung điểm của $AD$ và $BC$. Phân tích vectơ $\vec{MN}$ theo hai vectơ $\vec{PQ}$ và $\vec{DC}$ ta được $\vec{MN}=a\vec{PQ}+b\vec{DC}$ . Tính $a+2b$}\\
\loigiai{
{\centering\color{red} HINH O DAY}\\
Do $AM=3MD$, $BN=3NC$ và $P$, $Q$ lần lượt là trung điểm của $AD$ và $BC$ nên $M$, $N$ lần lượt là trung điểm của $PD$ và $QC$. \\
Ta có $\heva{& \vec{MN}=\vec{MP}+\vec{PQ}+\vec{QN} \\& \vec{MN}=\vec{MD}+\vec{DC}+\vec{CN}}\Rightarrow 2\vec{MN}=\vec{PQ}+\vec{DC}\Rightarrow \vec{MN}=\dfrac{1}{2}\left(\vec{PQ}+\vec{DC}\right)$\\
$\Rightarrow a=\dfrac{1}{2};\ b=\dfrac{1}{2}\Rightarrow a+2b=\dfrac{3}{2}$
}
\end{ex}
%Câu 2
\begin{ex}
 Cho hình chóp $S \cdot ABCD$ có đáy $ABCD$ là hình bình hành. Một mặt phẳng $\left(\alpha\right)$ cắt các cạnh $SA$, $SB$, $SC$, $SD$ lần lượt tại $A',B',C',D'$. Giá trị của biểu thức 
$P=\dfrac{SA}{SA'}+\dfrac{SC}{SC'}-\dfrac{SB}{SB'}-\dfrac{SD}{SD'}$ bằng bao nhiêu ?}
\loigiai{
{\centering\color{red} HINH O DAY}\\
Gọi $O$ là tâm của hình bình hành $ABCD$ thì $\vec{SA}+\vec{SC}=\vec{SB}+\vec{SD}=2\vec{SO}$\\
$\Leftrightarrow \dfrac{SA}{SA'}\vec{SA'}+\dfrac{SC}{SC'}\vec{SC'}=\dfrac{SB}{SB'}\vec{SB'}+\dfrac{SD}{SD'}\vec{SD'}$\\
Do $A',B',C',D'$ đồng phẳng nên $\Rightarrow \dfrac{SA}{SA'}+\dfrac{SC}{SC'}=\dfrac{SB}{SB'}+\dfrac{SD}{SD'}\Rightarrow P=\dfrac{SA}{SA'}+\dfrac{SC}{SC'}-\dfrac{SB}{SB'}-\dfrac{SD}{SD'}=0$
}
\end{ex}
%Câu 3
\begin{ex}
 Cho hình lập phương $ABCD \cdot A'B'C'D'$ có cạnh bằng $2$. Tính $\vec{AB} \cdot \vec{A'C'}$.
{\centering\color{red} HINH O DAY}}
\loigiai{
{\centering\color{red} HINH O DAY}\\
Ta có: $\left(\vec{AB},\vec{A'C'}\right)=\left(\vec{AB},\vec{AC}\right)=45^\circ $.\\
Khi đó: $\vec{AB} \cdot \vec{A'C'}=AB \cdot A'C' \cdot \cos \left(\vec{AB},\vec{A'C'}\right)=2 \cdot 2\sqrt{2} \cdot \cos 45^\circ =4$
}
\end{ex}
%Câu 4
\begin{ex}
 Cho tứ diện $ABCD$, gọi $M$, $N$ lần lượt là trung điểm của $BC$ và $AD$, biết $AB=a$, $CD=a$, $MN=\dfrac{a\sqrt{3}}{2}$. Tìm số đo góc giữa hai đường thẳng $AB$ và $CD$}
\loigiai{
{\centering\color{red} HINH O DAY}\\
Gọi $I$ là trung điểm của $AC$.\\
Ta có $\heva{& IM \parallel AB \\& IN \parallel CD}\Rightarrow \widehat{\left(AB$, $CD\right)}=\widehat{\left(IM$, $IN\right)}$.\\
Đặt $\widehat{MIN}=\alpha $. Xét tam giác $IMN$, có: $IM=\dfrac{AB}{2}=\dfrac{a}{2},IN=\dfrac{CD}{2}=\dfrac{a}{2},MN=\dfrac{a\sqrt{3}}{2}$.\\
Theo định lý côsin, có $\cos \alpha =\dfrac{IM^2+IN^2-MN^2}{2 \cdot IM \cdot IN}=-\dfrac{1}{2}<0$.\\
$\Rightarrow \widehat{MIN}=120^\circ \Rightarrow \widehat{\left(AB$, $CD\right)}=60^\circ $
}
\end{ex}
%Câu 5
\begin{ex}
 Cho hình lập phương $B'C$ có đường chéo $A'C=\dfrac{3}{16}$. Gọi $O$ là tâm hình vuông $ABCD$ và điểm $20$ thỏa mãn: $\vec{OS}=\vec{OA}+\vec{OB}+\vec{OC}+\vec{OD}+\vec{OA'}+\vec{OB'}+\vec{OC'}+\vec{OD'}$. Khi đó độ dài của đoạn $OS$ bằng $\dfrac{a\sqrt{3}}{b}$ với $a,b\in \mathbb{N}$ và $\dfrac{a}{b}$ là phân số tối giản. Tính giá trị của biểu thức $P=a^2+b^2$.
 Lời giải
{\centering\color{red} HINH O DAY}
Ta có: $A'C^2=A'A^2+AC^2=3A'A^2\Rightarrow A'A=\dfrac{A'C}{\sqrt{3}}=\dfrac{\sqrt{3}}{16}$.
Gọi $O'$ là tâm của hình vuông $A'B'C'D'$.
Lại có : $\vec{OS}=\vec{OA}+\vec{OB}+\vec{OC}+\vec{OD}+\vec{OA'}+\vec{OB'}+\vec{OC'}+\vec{OD'}$
$=\left(\vec{OA}+\vec{OC}\right)+\left(\vec{OB}+\vec{OD}\right)+\left(\vec{OA'}+\vec{OC'}\right)+\left(\vec{OB'}+\vec{OD'}\right)$
$=2\vec{OO'}+2\vec{OO'}=4\vec{OO'}$
Suy ra $OS=\left| \vec{OS} \right|=\left| 4\vec{OO'} \right|=4OO'=4 \cdot \dfrac{\sqrt{3}}{16}=\dfrac{\sqrt{3}}{4}$.
Khi đó $a=1,b=4\Rightarrow P=a^2+b^2=17$
}
\end{ex}
%Câu 6
\begin{ex}
 Khi chuyển động trong không gian, máy bay luôn chịu tác động của 4 lực chính: lực đẩy của động cơ, lực cản của không khí, trọng lực và lực nâng khí động học(hình ảnh 2.20).
{\centering\color{red} HINH O DAY}
 Lực cản của không khí ngược hướng với lực đẩy của động cơ và có độ lớn tỉ lệ thuận với bình phương vận tốc máy bay. Một chiếc máy bay tăng vận tốc từ 900(km/h) lên 920(km/h), trong quá trình tăng tốc máy bay giữ nguyên hướng bay. Lực cản của không khí khi máy bay đạt vận tốc 900(km/h) và 920(km/h) lần lượt biểu diễn bởi hai véc tơ $\vec{{{F_1}$ và $\vec{F_2}$ với $\vec{F_1}=k\vec{F_2}(k\in \mathbb{R};k>0)$. Tính giá trị của $k$ (làm tròn kết quả đến chữ số thập phân thứ hai).
 Lời giải
 Vì trong quá trình máy bay tăng vận tốc từ 900(km/h) lên 900(km/h), máy bay giữ nguyên hướng bay nên hai véc tơ $\vec{F_1}$ và $\vec{F_2}$ có cùng hướng và $\vec{F_1}=k\vec{F_2}(k>0)$.
 Gọi $v_1,v_2$ lần lượt là vận tốc của chiếc máy bay khi đạt 900(km/h) và 920(km/h).
 Suy ra $v_1=900$(km/h), $v_2=920$(km/h).
 Vì lực cản của không khí ngược hướng với lực đẩy của động cơ và có độ lớn tỉ lệ thuận với bình phương vận tốc máy bay nên $\left| \dfrac{\vec{F_1}}{\vec{F_2}} \right|=\dfrac{v_1^2}{v_2^2}=\dfrac{900}^2}}{{{920}^2}}=\dfrac{2025}{2116}\Rightarrow \left| \vec{{{F_1} \right|=\dfrac{2025}{2116}\left| \vec{F_2} \right|\Rightarrow \vec{F_1}=\dfrac{2025}{2116}\vec{F_2}$.
 Từ đó suy ra: $k=\dfrac{2025}{2116}\approx 0{,}96$
}
\end{ex}