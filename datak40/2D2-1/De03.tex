PHẦN I. Câu trắc nghiệm nhiều phương án lựa chọn. Thí sinh trả lời từ câu 1 đến câu 12. Mỗi câu hỏi thí sinh chỉ chọn một phương án.
%Câu 1
\begin{ex}
	Tìm mệnh đề sai trong các mệnh đề sau:
	\choice
	{\True Hai vectơ được gọi là cùng phương nếu chúng có giá song song với nhau}
	{Nếu hai vectơ cùng phương thì chúng cùng hướng hoặc ngược hướng}
	{Hai vectơ được gọi là bằng nhau nếu chúng cùng độ dài và cùng hướng}
	{Nếu vectơ $\vec{a}$ và vectơ $\vec{b}$ cùng bằng vectơ $\vec{c}$ thì hai vectơ $\vec{a}$ và vectơ $\vec{b}$ bằng nhau}
	\loigiai{
		Hai vectơ được gọi là cùng phương nếu chúng có giá song song hoặc trùng nhau.
	}
\end{ex}
%Câu 2
\begin{ex}
	Cho hình hộp chữ nhật $ABCD.A'B'C'D'$. Khi đó, vectơ bằng vectơ $\vec{AB}$ là
	\choice
	{\True $\vec{D'C'}$}
	{$\vec{BA}$}
	{$\vec{CD}$}
	{$\vec{B'A'}$}
	\loigiai{
		\immini{
			Dễ thấy vectơ bằng với vectơ $\vec{AB}$ là vectơ nào $\vec{D'C'}$ vì chúng cùng hướng và có cùng độ dài.
		}{\begin{tikzpicture}[line cap=round,line join=round, >=stealth,scale=.6]
				\def \a{-1.5} \def \b{-1}\def \c{4.5} \def \h{4}
				\path (.5,.5)coordinate(A) 
				+(\a,\b)coordinate(B)
				+(\c,0)coordinate(D)
				($(B)+(D)-(A)$)coordinate(C)
				+(0,\h) coordinate(C')
				($(B)+(C')-(C)$)coordinate(B')
				($(A)+(C')-(C)$)coordinate(A')
				($(D)+(C')-(C)$)coordinate(D');
				\draw [dashed] (A)--(B)(D)--(A)--(A');
				\draw (B')--(B)--(C)(B')--(C')--(C)--(D)--(D')--(A')--(B')(C')--(D');
				\foreach \x/\g in {A/135,B/-135,C/-45,D/0,A'/135,B'/180,C'/-20,D'/0}\fill[red] (\x) circle (1pt)+(\g:3mm) node[black]{$\x$};
		\end{tikzpicture}}
	}
\end{ex}
%Câu 3
\begin{ex}
	Cho hình hộp $ABCD.A'B'C'D'$. Vectơ nào dưới đây cùng phương với vectơ $\vec{AB}$?
	\choice
	{\True $\vec{CD}$}
	{$\vec{B'C'}$}
	{$\vec{AD}$}
	{$\vec{AC'}$}
	\loigiai{
		\immini{
			Vectơ cùng phương với $\vec{AB}$ là $\vec{CD}$, vì hai vectơ này có giá song song với nhau.}
		{\begin{tikzpicture}[line cap=round,line join=round, >=stealth,scale=.6]
				\def \a{-1.5} \def \b{-1}\def \c{4.5} \def \h{4}
				\path (.5,.5)coordinate(A) 
				+(\a,\b)coordinate(B)
				+(\c,0)coordinate(D)
				($(B)+(D)-(A)$)coordinate(C)
				+(0,\h) coordinate(C')
				($(B)+(C')-(C)$)coordinate(B')
				($(A)+(C')-(C)$)coordinate(A')
				($(D)+(C')-(C)$)coordinate(D');
				\draw [dashed] (A)--(B)(D)--(A)--(A');
				\draw (B')--(B)--(C)(B')--(C')--(C)--(D)--(D')--(A')--(B')(C')--(D');
				\foreach \x/\g in {A/135,B/-135,C/-45,D/0,A'/135,B'/180,C'/-20,D'/0}\fill[red] (\x) circle (1pt)+(\g:3mm) node[black]{$\x$};
		\end{tikzpicture}}
	}
\end{ex}
%Câu 4
\begin{ex}
	Cho hình hộp $ABCD.A'B'C'D'$. Mệnh đề nào sau đây sai?
	\choice
	{$\vec{AC'}=\vec{AB}+\vec{AD}+\vec{AA'}$}
	{\True $\vec{BC'}=\vec{BC}+\vec{BD}+\vec{BB'}$}
	{$\vec{DB'}=\vec{DA}+\vec{DC}+\vec{DD'}$}
	{$\vec{BD'}=\vec{BA}+\vec{BC}+\vec{BB'}$}
	\loigiai{
		Theo quy tắc hình hộp, ta có mệnh đề sai là $\vec{BC'}=\vec{BC}+\vec{BD}+\vec{BB'}$.
	}
\end{ex}
%Câu 5
\begin{ex}
	\immini{Cho hình tứ diện $ABCD$. Gọi $M$, $N$ lần lượt là trung điểm $AB$, $CD$, $I$ là trung điểm của đoạn $MN$. Mệnh đề nào sau đây sai?
		\choice
		{\True $\vec{AN}=\left(\vec{AD}+\vec{AC}\right)$}
		{$\vec{IN}+\vec{IM}=\vec{0}$}
		{$\vec{MA}+\vec{MB}=\vec{0}$}
		{$\vec{NC}+\vec{ND}=\vec{0}$}
	}{
		\begin{tikzpicture}[scale=.6, font=\footnotesize, line join=round, line cap=round, >=stealth]
			\def\ac{4} % cạnh AC
			\def\ab{2} % cạnh AB
			\def\ad{4} % cạnh AD
			\def\gocA{50} % góc A của đáy
			\path
			(0,0) coordinate (A)
			(\ac,0) coordinate (C)
			(-\gocA:\ab) coordinate (B)
			(65:\ad) coordinate (D)
			($(A)!.5!(B)$) coordinate (M)
			($(C)!.5!(D)$) coordinate (N)
			($(M)!.5!(N)$) coordinate (I)
			;
			\draw (A)--(B)--(C)--(D)--cycle (D)--(B);
			\draw[dashed] (A)--(C) (M)--(N);
			\foreach \x/\g in {A/180,B/-90,C/0,D/90,M/-135,N/45,I/135}\fill (\x) circle (1pt)+(\g:3mm) node[black]{$\x$};
	\end{tikzpicture}}
	\loigiai{
		Đáp án B đúng: Vì $I$ là trung điểm $MN$ nên ta có: $\vec{IN}+\vec{IM}=\vec{0}$.\\
		Đáp án C đúng: Vì $M$ là trung điểm $AB$ nên ta có: $\vec{MA}+\vec{MB}=\vec{0}$.\\
		Đáp án D đúng. Vì N là trung điểm CD nên ta có $\vec{NC}+\vec{ND}=\vec{0}$.
	}
\end{ex}
%Câu 6
\begin{ex}
	\immini{Cho hình lập phương $ABCD.A'B'C'D'$ có cạnh bằng $a$. Hãy tìm mệnh đề đúng trong những mệnh đề sau đây
		\choice
		{\True $\vec{AB}+\vec{AD}+\vec{AA'}=\vec{AC'}$}
		{$\vec{AD}+\vec{DB'}=\vec{B'A}$}
		{$\vec{AB}+\vec{AD}=\vec{BD}$}
		{$\vec{AC}-\vec{AB'}=\vec{CB'}$}
	}{
		\begin{tikzpicture}[line cap=round,line join=round, >=stealth,scale=.6]
			\def \a{-1.5} \def \b{-1}\def \c{4} \def \h{4}
			\path (.5,.5)coordinate(A) 
			+(\a,\b)coordinate(B)
			+(\c,0)coordinate(D)
			($(B)+(D)-(A)$)coordinate(C)
			+(0,\h) coordinate(C')
			($(B)+(C')-(C)$)coordinate(B')
			($(A)+(C')-(C)$)coordinate(A')
			($(D)+(C')-(C)$)coordinate(D');
			\draw [dashed] (A)--(B)(D)--(A)--(A');
			\draw (B')--(B)--(C)(B')--(C')--(C)--(D)--(D')--(A')--(B')(C')--(D');
			\foreach \x/\g in {A/135,B/-135,C/-45,D/0,A'/135,B'/180,C'/-20,D'/0}\fill[red] (\x) circle (1pt)+(\g:3mm) node[black]{$\x$};
	\end{tikzpicture}}
	\loigiai{
		Theo quy tắc hình hộp ta có $\vec{AB}+\vec{AD}+\vec{AA'}=\vec{AC'}$.
	}
\end{ex}
%Câu 7
\begin{ex}
	Cho tứ diện $ABCD$, có bao nhiêu vectơ có điểm đầu là $A$ và điểm cuối là một trong các đỉnh còn lại của tứ diện?
	\choice
	{$1$}
	{\True $3$}
	{$2$}
	{$4$}
	\loigiai{
		Có ba vectơ là: $\vec{AB},\vec{AC},\vec{AD}$.
	}
\end{ex}
%Câu 8
\begin{ex}
	Cho hình hộp $ABCD.A'B'C'D'$. Hai vectơ nào sau đây cùng phương?
	\choice
	{$\vec{A'B}$ và $\vec{A'B'}$}
	{$\vec{B'C'}$ và $\vec{CD}$}
	{$\vec{AB}$ và $\vec{B'C'}$}
	{\True $\vec{AB}$ và $\vec{D'C'}$}
	\loigiai{
		\immini{
			Hai vectơ $\vec{AB}$ và $\vec{D'C'}$ có giá song song nên cùng phương.}
		{\begin{tikzpicture}[scale=.7, font=\footnotesize, line join=round, line cap=round, >=stealth]
				\def\bc{4} % cạnh BC
				\def\ba{2} % cạnh BA
				\def\h{3.5} % đường cao
				\def\gocB{35} % góc B của đáy
				\coordinate[label=below left:$B$] (B) at (0,0);
				\coordinate[label=above left:$A$] (A) at (\gocB:\ba);
				\coordinate[label=below:$C$] (C) at (\bc,0);
				\coordinate[label=right:$D$] (D) at ($(C)-(B)+(A)$);
				\coordinate[label=above left:$A'$] (A') at ($(A)+(80:\h)$);
				\coordinate[label=left:$B'$] (B') at ($(B)-(A)+(A')$);
				\coordinate[label=below right:$C'$] (C') at ($(C)-(A)+(A')$);
				\coordinate[label=right:$D'$] (D') at ($(D)-(A)+(A')$);
				\draw (B')--(B)--(C)--(D)--(D')--(A')--(B')--(C')--(D') (C)--(C');
				\draw[dashed] (A)--(D) (A')--(A)--(B);
				\foreach \diem in {A,B,C,D,A',B',C',D'}	\fill (\diem)circle(1.5pt);
		\end{tikzpicture}}
	}
\end{ex}
%Câu 9
\begin{ex}
	\immini{
		Cho tứ diện $ABCD$. Gọi $M$, $N$ lần lượt là trung điểm của $AB$, $CD$, $G$ là trung điểm của MN. Vectơ $\vec{GA}+\vec{GB}+\vec{GC}+\vec{GD}$ bằng Vectơ nào sau đây
		\choice
		{$4\vec{MG}$}
		{$\vec{GD}$}
		{\True $\vec{0 \cdot }$}
		{$\vec{MN}$}
	}{\begin{tikzpicture}[scale=.6, font=\footnotesize, line join=round, line cap=round, >=stealth]
			\def\ac{4} % cạnh AC
			\def\ab{2} % cạnh AB
			\def\ad{4} % cạnh AD
			\def\gocA{50} % góc A của đáy
			\path
			(0,0) coordinate (A)
			(\ac,0) coordinate (C)
			(-\gocA:\ab) coordinate (B)
			(65:\ad) coordinate (D)
			($(A)!.5!(B)$) coordinate (M)
			($(C)!.5!(D)$) coordinate (N)
			($(M)!.5!(N)$) coordinate (G)
			;
			\draw (A)--(B)--(C)--(D)--cycle (D)--(B);
			\draw[dashed] (A)--(C) (M)--(N);
			\foreach \x/\g in {A/180,B/-90,C/0,D/90,M/-135,N/45,G/-45}\fill (\x) circle (1pt)+(\g:3mm) node[black]{$\x$};
	\end{tikzpicture}}
	\loigiai{
		$\vec{GA}+\vec{GB}+\vec{GC}+\vec{GD}=\left(\vec{GA}+\vec{GB}\right)+\left(\vec{GC}+\vec{GD}\right)=2\vec{GM}+2\vec{GN}=2\left(\vec{GM}+\vec{GN}\right)=\vec{0}$.
	}
\end{ex}
%Câu 10
\begin{ex}
	Cho hình lập phương $ABCD.A'B'C'D'$. Chọn mệnh đề đúng?
	\choice
	{$\vec{AC}=\vec{C'A'}$}
	{$\vec{AB}+\vec{AD}+\vec{AC}=\vec{AA'}$}
	{$\vec{AB}=\vec{CD}$}
	{\True $\vec{AB}+\vec{C'D'}=\vec{0}$}
	\loigiai{
		Ta có $\vec{AB}+\vec{C'D'}=\vec{AB}+\vec{CD}=\vec{AB}+\vec{BA}=\vec{0}$.
	}
\end{ex}
%Câu 11
\begin{ex}
	Cho hình lăng trụ $ABC.A'B'C'$, $M$ là trung điểm của $BB'$. Đặt $\vec{CA}=\vec{a}$, $\vec{CB}=\vec{b}$, $\vec{AA'}=\vec{c}$. Khẳng định nào sau đây đúng?
	\choice
	{$\vec{AM}=\vec{b}+\vec{c}-\dfrac{1}{2}\vec{a}$}
	{$\vec{AM}=\vec{a}-\vec{c}+\dfrac{1}{2}\vec{b}$}
	{$\vec{AM}=\vec{a}+\vec{c}-\dfrac{1}{2}\vec{b}$}
	{\True $\vec{AM}=\vec{b}-\vec{a}+\dfrac{1}{2}\vec{c}$}
	\loigiai{
		\immini{
			Ta có\\
			$\vec{AM}=\vec{AB}+\vec{BM}=\vec{CB}-\vec{CA}+\dfrac{1}{2}\vec{BB'}$\\
			$=\vec{b}-\vec{a}+\dfrac{1}{2}\vec{AA'}=\vec{b}-\vec{a}+\dfrac{1}{2}\vec{c}$.
		}{
			\begin{tikzpicture}[scale=.6, font=\footnotesize, line join=round, line cap=round, >=stealth]
				\def\ac{4} % cạnh AC
				\def\ab{2} % cạnh AB
				\def\ben{4} % cạnh bên
				\def\gocnghieng{75} % góc nghiêng cạnh bên
				\def\gocA{50} % góc A của đáy
				\coordinate[label=left:$A$] (A) at (0,0);
				\coordinate[label=right:$C$] (C) at (\ac,0);
				\coordinate[label=below left:$B$] (B) at (-\gocA:\ab);
				\coordinate[label=left:$A'$] (A') at ($(A)+(\gocnghieng:\ben)$);
				\coordinate[label=below left:$B'$] (B') at ($(B)-(A)+(A')$);
				\coordinate[label=right:$C'$] (C') at ($(C)-(A)+(A')$);
				\draw (A')--(A)--(B)--(C)--(C')--(A')--(B')--(C') (B)--(B');
				\draw[dashed] (A)--(C);
				\foreach \diem in {A,B,C,A',B',C'} \fill (\diem)circle(1.5pt);
			\end{tikzpicture}
		}
	}
\end{ex}
%Câu 12
\begin{ex}
	\immini{
		Cho tứ diện $ABCD$ có $M$, $N$ lần lượt là trung điểm các cạnh $AC$ và $BD$. Gọi $G$ là trung điểm của đoạn thẳng $MN$. Hãy chọn khẳng định sai
		\choice
		{$\vec{GA}+\vec{GC}=2\vec{GM}$}
		{$\vec{GB}+\vec{GD}=\vec{MN}$}
		{$\vec{GA}+\vec{GB}+\vec{GC}+\vec{GD}=\vec{0}$}
		{\True $2\vec{NM}=\vec{AB}+\vec{CD}$}
	}{\begin{tikzpicture}[scale=.6, font=\footnotesize, line join=round, line cap=round, >=stealth]
			\def\ab{4}
			\def\ac{2}
			\def\ad{4}
			\def\gocA{55}
			\path
			(0,0) coordinate (A)
			(\ab,0) coordinate (B)
			(-\gocA:\ac) coordinate (C)
			(65:\ad) coordinate (D)
			($(A)!.5!(C)$) coordinate (M)
			($(B)!.5!(D)$) coordinate (N)
			($(M)!.5!(N)$) coordinate (G)
			;
			\draw (A)--(C)--(B)--(D)--cycle (D)--(C);
			\draw[dashed] (A)--(B) (M)--(N);
			\foreach \x/\g in {A/180,B/0,C/-90,D/90,M/-135,N/45,G/-45}\fill (\x) circle (1pt)+(\g:3mm) node[black]{$\x$};
	\end{tikzpicture}}
	\loigiai{
		\begin{itemize}
			\item $\vec{GA}+\vec{GC}=2\vec{GM}$ đúng vì $M$ là trung điểm $AC$.\\
			\item $\vec{GB}+\vec{GD}=\vec{MN}$ đúng vì $\vec{GB}+\vec{GD}=2\vec{GN}=\vec{MN}$\\
			\item $\vec{GA}+\vec{GB}+\vec{GC}+\vec{GD}=\vec{0}$ đúng vì $\vec{GA}+\vec{GB}+\vec{GC}+\vec{GD}=2\left(\vec{GM}+\vec{GN}\right)=\vec{0}$.\\
			\item $2\vec{NM}=\vec{AB}+\vec{CD}$ sai vì $\vec{AB}+\vec{CD} =\left(\vec{AM}+\vec{MN}+\vec{NB}\right)+\left(\vec{CM}+\vec{MN}+\vec{ND}\right) =2\vec{MN}+\vec{0}+\vec{0}=2\vec{MN}$.
		\end{itemize}
	}
\end{ex}
%Câu 13
\begin{ex}
	Cho tứ diện đều $SABC$ có cạnh $a$. Gọi $M$, $N$ lần lượt là trung điểm $SA$, $BC$. Các mệnh đề sau đúng hay sai?
	\begin{center}
		\begin{tikzpicture}[scale=1, font=\footnotesize, line join=round, line cap=round, >=stealth]
			\def\ac{4} % cạnh AC
			\def\ab{2} % cạnh AB
			\def\as{4} % cạnh AS
			\def\gocA{50} % góc A của đáy
			\path
			(0,0) coordinate (A)
			(\ac,0) coordinate (C)
			(-\gocA:\ab) coordinate (B)
			(70:\as) coordinate (S)
			($(S)!.5!(A)$) coordinate (M)
			($(B)!.5!(C)$) coordinate (N)
			;
			\draw (A)--(B)--(C)--(S)--cycle (S)--(B);
			\draw (S)--(A)node[midway,above left]{$a$};
			\draw[dashed] (A)--(C) (M)--(N);
			\foreach \x/\g in {A/180,B/-90,C/0,S/90}\fill (\x) circle (1pt)+(\g:3mm) node[black]{$\x$};
		\end{tikzpicture}
	\end{center}
	\choiceTF
	{\True Độ dài của vectơ $\vec{SA}$ bằng $a$.}
	{\True $\vec{SA} \cdot \vec{SB}=\dfrac{a^2\sqrt{3}}{2}$}
	{$\vec{SB}+\vec{AB}+\vec{SC}+\vec{AC}=4\vec{MN}$}
	{Gọi $I$ là trọng tâm của tứ diện. Khoảng cách từ $I$ đến $(ABC)$ bằng $\dfrac{3a\sqrt{6}}{4}$}
	\loigiai{
		\begin{center}
			\begin{tikzpicture}[scale=1, font=\footnotesize, line join=round, line cap=round, >=stealth]
				\def\ac{4} % cạnh AC
				\def\ab{2} % cạnh AB
				\def\as{4} % cạnh AS
				\def\gocA{50} % góc A của đáy
				\path
				(0,0) coordinate (A)
				(\ac,0) coordinate (C)
				(-\gocA:\ab) coordinate (B)
				(70:\as) coordinate (S)
				($(S)!.5!(A)$) coordinate (M)
				($(B)!.5!(C)$) coordinate (N)
				($(M)!.5!(N)$) coordinate (I)
				($(A)!2/3!(N)$) coordinate (G)
				;
				\draw (A)--(B)--(C)--(S)--cycle (S)--(B) (S)--(N) (M)--(B);
				\draw (S)--(A)node[midway,above left]{$a$};
				\draw[dashed] (A)--(C)--(M)--(N)--(A) (S)--(G) ;
				\foreach \x/\g in {A/180,B/-90,C/0,S/90}\fill (\x) circle (1pt)+(\g:3mm) node[black]{$\x$};
			\end{tikzpicture}
		\end{center}
		\begin{enumerate}[a)]
			\item $|\vec{SA}|=SA=a$.
			\item $\vec{SA} \cdot \vec{SB}=\left| \vec{SA} \right| \cdot \left| \vec{SB} \right| \cdot \sin \widehat{ASB}=a \cdot a \cdot \sin 60^\circ=\dfrac{a^2\sqrt{3}}{2}$.
			\item Do $N$ là trung điểm của $BC$ nên $\vec{SB}+\vec{SC}=2\vec{SN}$ và $\vec{AB}+\vec{AC}=2\vec{MB}$.\\
			Suy ra $\vec{SB}+\vec{SC}+\vec{AB}+\vec{AC}=2\left(\vec{SN}+\vec{AN}\right)$\\
			Do $M$ là trung điểm của $SA$ nên $\vec{NA}+\vec{NS}=2\vec{NM}\Leftrightarrow \vec{AN}+\vec{SN}=2\vec{MN}$.\\
			Do đó $\vec{SB}+\vec{SC}+\vec{AB}+\vec{AC}=2 \cdot 2 \cdot \vec{MN}=4\vec{MN}$.
			\item Gọi $G$ là trọng tâm tam giác $ABC$.\\
			Do tứ diện $SABC$ là tứ diện đều và $I$ là trọng tâm tứ diện nên $d\left(I,(ABC)\right)=IG$\\
			Tam giác $ABC$ đều cạnh $a$, $N$ là trung điểm của $BC$, suy ra $AN=\dfrac{a\sqrt{3}}{2}$.\\
			Do $G$ là trọng tâm tam giác$ABC$ nên $AG=\dfrac{2}{3}AN=\dfrac{a\sqrt{3}}{3}$.\\
			Do tứ diện $SABC$ là tứ diện đều nên $SG\bot (ABC)\Rightarrow SG\bot AG$.\\
			Tam giác $SAG$ vuông tại $G$ nên $SG=\sqrt{SA^2-AG^2}=\sqrt{a^2-\dfrac{a^2}{3}}=\dfrac{a\sqrt{6}}{3}$.\\
			Do $I$ là trọng tâm tứ diện$SABC$ nên $IG=\dfrac{1}{4}SG=\dfrac{1}{4} \cdot \dfrac{a\sqrt{6}}{3}=\dfrac{a\sqrt{6}}{12}$.\\
			Vậy $d\left(I,(ABC)\right)=\dfrac{a\sqrt{6}}{12}$.
		\end{enumerate}
	}
\end{ex}
%Câu 14
\begin{ex}
	Cho hình lập phương $ABCD.A'B'C'D'$ có cạnh $a$. Gọi $M$ là trung điểm $AD$. Các mệnh đề sau đúng hay sai?
	\choiceTF
	{$\vec{A'B'}=\vec{CD}$}
	{\True $\vec{DC'}=\vec{DC}+\vec{DD'}$}
	{\True $\vec{AB'} \cdot \vec{CD'}=0$}
	{$\vec{B'M} =\vec{BB'}+\vec{A'B'
	}+\dfrac{1}{2}\vec{B'C'}$}
	\loigiai{
		\begin{center}
			\begin{tikzpicture}[line join = round, line cap = round, thick, font = \small, scale = .7]
				\def \canh{4}
				\path 
				(0:0) coordinate (D')
				+(90:\canh) coordinate (D)
				+(0:\canh) coordinate (C')
				+(40:.6*\canh) coordinate (A')
				($(C')+(D)-(D')$) coordinate (C)
				($(D)+(A')-(D')$) coordinate (A)
				($(C')+(A')-(D')$) coordinate (B')
				($(C)+(A)-(D)$) coordinate (B)
				($(A)!.5!(D)$) coordinate (M);
				\draw[dashed] 
				(A')--(A) (A')--(B') (A')--(D') (B')--(M)--(C')
				;
				\draw 
				(A)--(B)--(B')--(C')--(D')--(D)--cycle
				(D)--(C)--(B) (C)--(C') 
				;
				\foreach \x/\g in {D'/-90,C'/-90,D/180,A'/135,C/-45,A/90,B'/0,B/90,M/135}
				\fill (\x) circle (1.5pt)
				+(\g:3mm) node {$\x$};
				\end{tikzpicture}
		\end{center}
		\begin{enumerate}[a)]
			\item $\vec{A'B'}=-\vec{CD}$.
			\item $\vec{DC}+\vec{DD'}=\vec{DC}+\vec{CC'}=\vec{DC'}$.
			\item $\vec{AB'} \cdot \vec{CD'}=\vec{AB'} \cdot \vec{BA'}=0$
			\item $\begin{aligned}[t]
			\vec{B'M} & =\vec{B'B}+\vec{BM} =\vec{BB'}+\dfrac{1}{2}\left(\vec{BA}+\vec{BD}\right) =\vec{BB'}+\dfrac{1}{2}\left(\vec{B'A'}+\vec{B'D'}\right) \\
			& =\vec{BB'}+\dfrac{1}{2}\left(\vec{B'A'}+\vec{B'A'}+\vec{B'C'}\right) =\vec{BB'}+\vec{B'A'}+\dfrac{1}{2}\vec{B'C'} \end{aligned}$
		\end{enumerate}	
	}
\end{ex}
%Câu 15 
\begin{ex}
	Cho tứ diện $ABCD$ có cạnh $a$. Gọi $M$, $N$ lần lượt là trung điểm của $AB$, $CD$. Các mệnh đề sau đúng hay sai?
	\choiceTF
	{$\vec{AB}$ và $\vec{CD}$ cùng hướng}
	{\True $\vec{EA}+\vec{EB}+\vec{EC}+\vec{ED}=\vec{0}$ với $E$ là trung điểm $MN$}
	{\True $\vec{AB} \cdot \vec{CD}+\vec{AC} \cdot \vec{DB}+\vec{AD} \cdot \vec{BC}=\vec{0}$}
	{\True Điểm $I$ xác định bởi $P=3\vec{IA}^2+\vec{IB}^2+\vec{IC}^2+\vec{ID}^2$ có giá trị nhỏ nhất. Khi đó giá trị nhỏ nhất của $P$ là $2a^2$}
	\loigiai{
		\begin{center}
			\begin{tikzpicture}[line join = round, line cap = round, thick, font = \small, scale = 1]
				\path 
				(0:0) coordinate (B)
				+(0:5) coordinate (C)
				+(-45:2) coordinate (D)
				+(70:4) coordinate (A)
				($(A)!.5!(B)$) coordinate (M)
								($(C)!.5!(D)$) coordinate (N)
								($(M)!.5!(N)$) coordinate (E)
								($(B)!2/3!(N)$) coordinate (G)
								($(A)!.5!(G)$) coordinate (O)
				;
				\draw[dashed] 
				(B)--(C) (A)--(G) (M)--(N)
				;
				\draw 
				(A)--(B)--(D)--(C)--cycle
				(A)--(D)
				;
				\foreach \x/\g in {B/180,C/0,D/-90,G/0,A/90,M/135,N/-45,E/0,O/0}
				\fill (\x) circle (1.5pt)
				+(\g:3mm) node {$\x$};
				\end{tikzpicture}
		\end{center}
		\begin{enumerate}[a)]
			\item $\vec{AB}$ và $\vec{CD}$ ngược hướng.
			\item Vì $M$ là trung điểm $AB$ nên $\vec{EA}+\vec{EB}=2\vec{EM}$, $N$ là trung điểm $CD$ nên $\vec{EC}+\vec{ED}=2\vec{EN}$.\\
			Ta có $\vec{EA}+\vec{EB}+\vec{EC}+\vec{ED}=2\left(\vec{EM}+\vec{EN}\right)=\vec{0}$.
			\item $\begin{aligned}[t]
				&\vec{AB} \cdot \vec{CD}+\vec{AC} \cdot \vec{DB}+\vec{AD} \cdot \vec{BC}=\left(\vec{AC}+\vec{CB}\right) \cdot \vec{CD}+\vec{AC} \cdot \vec{DB}+\vec{AD} \cdot \vec{BC}\\
				 = & \vec{AC} \cdot \left(\vec{CD}+\vec{DB}\right)+\vec{AD} \cdot \vec{BC}+\vec{CB \cdot }\vec{CD}=\vec{AC} \cdot \vec{CB}+\vec{AD} \cdot \vec{BC}+\vec{CB \cdot }\vec{CD} \\
				 = &\vec{CB}\left(\vec{AC}-\vec{AD}\right)+\vec{CB \cdot }\vec{CD}=\vec{0} 
				\end{aligned}$
			\item Gọi $O$ là điểm thoả mãn hệ thức $3\vec{OA}+\vec{OB}+\vec{OC}+\vec{OD}=\vec{0}$ suy ra $O$ cố định vì $A$, $B,C$, $D$ cố định. Ta có
			\begin{align*}
				P& =3\vec{IA}^2+\vec{IB}^2+\vec{IC}^2+\vec{ID}^2 \\
				& =3\left(\vec{IO}+\vec{OA}\right)^2+\left(\vec{IO}+\vec{OB}\right)^2+\left(\vec{IO}+\vec{OC}\right)^2+\left(\vec{IO}+\vec{OD}\right)^2 \\
				& =6IO^2+3OA^2+OB^2+OC^2+OD^2+2\vec{IO}\left(3\vec{OA}+\vec{OB}+\vec{OC}+\vec{OD}\right) \\
				& =6IO^2+3OA^2+OB^2+OC^2+OD^2.
			\end{align*}
			Do đó để $P$ nhỏ nhất thì $I$ trùng với $O$. Gọi $G$ là trọng tâm tam giác $BCD$.\\
			Vì $3\vec{OA}+\vec{OB}+\vec{OC}+\vec{OD}=3\vec{OA}+\left(\vec{OB}+\vec{OC}+\vec{OD}\right) =3\vec{OA}+3\vec{OG}$ nên $\vec{OA}+\vec{OG}=\vec{0}$.\\			
			Suy ra $O$ là trung điểm của $AG$.\\
			Ta có $BG=\dfrac{2}{3} \cdot \dfrac{a\sqrt{3}}{2}=\dfrac{a}{\sqrt{3}}\Rightarrow AG=\sqrt{AB^2-BG^2}=\sqrt{a^2-{{\left(\dfrac{a}{\sqrt{3}}\right)}^2}}=\dfrac{a\sqrt{2}}{\sqrt{3}}$\\
			$\Rightarrow OA=\dfrac{1}{2}AG=\dfrac{a}{\sqrt{6}}\Rightarrow OA^2=\dfrac{a^2}{6}$.\\
			Lại có $OD^2=OC^2=OB^2=OG^2+BG^2=\dfrac{a^2}{6}+\dfrac{a^2}{3}=\dfrac{a^2}{2}$.\\
			Vậy giá trị nhỏ nhất là $P=3 \cdot \dfrac{a^2}{6}+3 \cdot \dfrac{a^2}{2}=2a^2$ khi $I$ trùng với $O$.
		\end{enumerate}
		
	}
\end{ex}
%Câu 16
\begin{ex}
	Cho tứ diện đều $ABCD$ cạnh $a$ có $G$ là trọng tâm của tam giác $BCD$ và $I$ là điểm thuộc đoạn thẳng $AG$ sao cho $\vec{AI}=3\vec{IG}$. Các mệnh đề sau đúng hay sai?
	\choiceTF
	{$\vec{GA}+\vec{GB}+\vec{GC}=\vec{0}$}
	{\True $\vec{IB}+\vec{IC}+\vec{ID}=3\vec{IG}$}
	{\True $\vec{GB}+\vec{GC}+\vec{GD}=\vec{IA}+\vec{IB}+\vec{IC}+\vec{ID}$}
	{\True $\vec{IB}=\dfrac{3}{4}\vec{AB}-\dfrac{1}{4}\vec{AC}-\dfrac{1}{4}\vec{AD}$}
	\loigiai{
		\begin{center}
			\begin{tikzpicture}[line join = round, line cap = round, thick, font = \small, scale = .7]
				\path 
				(0:0) coordinate (B)
				+(0:5) coordinate (C)
				+(-45:2) coordinate (D)
				+(70:4) coordinate (A)
								($(C)!.5!(D)$) coordinate (N)
								($(B)!2/3!(N)$) coordinate (G)
				;
				\draw[dashed] 
				(B)--(C) (A)--(G)
				;
				\draw 
				(A)--(B)--(D)--(C)--cycle
				(A)--(D)
				;
				\foreach \x/\g in {B/180,C/0,D/-90,G/0,A/90}
				\fill (\x) circle (1.5pt)
				+(\g:3mm) node {$\x$};
				\end{tikzpicture}
		\end{center}
		\begin{enumerate}
			\item $G$ là trọng tâm của tam giác $BCD$ nên $\vec{GB}+\vec{GC}+\vec{GD}=\vec{0}$.
			\item $\vec{IB}+\vec{IC}+\vec{ID}=\vec{IG}+\vec{GB}+\vec{IG}+\vec{GC}+\vec{IG}+\vec{GD}=3\vec{IG}+\left(\vec{GB}+\vec{GC}+\vec{GD}\right)=3\vec{IG}$.
			\item $\vec{GB}+\vec{GC}+\vec{GD}=\vec{0}\vec{IA}+\vec{IB}+\vec{IC}+\vec{ID}=\vec{IA}+3\vec{IG}=\vec{IA}+\vec{AI}=\vec{0}$.
			\item $\vec{AI}=3\vec{IG}\Leftrightarrow \vec{IA}=-\dfrac{3}{4}\vec{AG}$.\\
			$\vec{IB}=\vec{IA}+\vec{AB}=-\dfrac{3}{4}\vec{AG}+\vec{AB}=-\dfrac{3}{4} \cdot \dfrac{1}{3}\left(\vec{AB}+\vec{AC}+\vec{AD}\right)+\vec{AB}=\dfrac{3}{4}\vec{AB}-\dfrac{1}{4}\vec{AC}-\dfrac{1}{4}\vec{AD}$.
		\end{enumerate}
	}
\end{ex}
%Câu 17
\begin{ex}
	Cho tứ diện $ABCD$. Gọi $E$ là trung điểm $AD$, $F$ là trung điểm $BC$. Ta có $\vec{AB}+\vec{DC}= k\vec{EF}$. Tìm giá trị của $k$.
	\loigiai{
		\SA{2}
		Do $E$ là trung điểm $AD$, $F$ là trung điểm $BC$ nên $\vec{EA}+\vec{ED}=\vec{0}$; $\vec{FB}+\vec{FC}=-\left(\vec{BF}+\vec{CF}\right)=\vec{0}$.\\
		Có $\heva{& \vec{AB}=\vec{AE}+\vec{EF}+\vec{FB} \\& \vec{DC}=\vec{DE}+\vec{EF}+\vec{FB}}\Rightarrow \vec{AB}+\vec{DC}=2\vec{EF}$.
	}
\end{ex}
%Câu 18
\begin{ex}
	Cho hình hộp chữ nhật $ABCD.A'B'C'D'$ có $AB=2$, $AD=3$. Độ dài vectơ $\vec{B'D'}$ bằng bao nhiêu (làm tròn đến hàng phần trăm)?
	\loigiai{
		\SA{3,61}
		Ta có: $\left| \vec{B'D'} \right|=B'D'=BD=\sqrt{AB^2+AD^2}=\sqrt{13}$.\\
		Vậy độ dài vectơ $\vec{B'D'}$ bằng $\sqrt{13} \approx 3,61$.
	}
\end{ex}
%Câu 19
\begin{ex}
	Cho hình lập phương $ABCD.A'B'C'D'$. Góc giữa hai vectơ $\vec{A'B}$ và $\vec{AC'}$ bằng
	\loigiai{
		\SA{90}
		$\vec{A'B}=\vec{A'A}+\vec{AB}=\vec{AB}-\vec{AA'}$.\\
		$\vec{AC'}=\vec{AB}+\vec{AD}+\vec{AA'}$.\\
		$\Rightarrow \vec{A'B} \cdot \vec{AC'} = \left(\vec{AB}-\vec{AA'}\right) \cdot \left(\vec{AB}+\vec{AD}+\vec{AA'}\right)={{\vec{AB}}^2}-{{\vec{AA'}}^2}=0$.\\
		$\Rightarrow$ Góc giữa hai vectơ $\vec{A'B}$ và $\vec{AC'}$ bằng $90^\circ$.
	}
\end{ex}
%Câu 20
\begin{ex}
	Cho hình chóp $S.ABC$ có $SA$, $SB$, $SC$ đôi một vuông góc nhau và $SA=SB=SC=a$. Gọi $M$ là trung điểm của $AB$. Góc giữa hai vectơ $\vec{SM}$ và $\vec{BC}$ bằng
	\loigiai{
		\shortans{120}	
		Ta có $\cos \left(\vec{SM},\vec{BC}\right)=\dfrac{\vec{SM} \cdot \vec{BC}}{|\vec{SM}| \cdot |\vec{BC}|}=\dfrac{\vec{SM} \cdot \vec{BC}}{SM \cdot BC}$.\\
		\begin{align*}
			\vec{SM} \cdot \vec{BC} & =\dfrac{1}{2}\left(\vec{SA}+\vec{SB}\right) \cdot \left(\vec{SC}-\vec{SB}\right)\\
			& =\dfrac{1}{2}\left(\vec{SA} \cdot \vec{SC}-\vec{SA} \cdot \vec{SB}+\vec{SB} \cdot \vec{SC}-\vec{SB} \cdot \vec{SB}\right) \\
			& =-\dfrac{1}{2}\vec{SB} \cdot \vec{SB}=-\dfrac{1}{2}SB^2=-\dfrac{a^2}{2}.
		\end{align*}
		Tam giác $SAB$ và $SBC$ vuông cân tại $S$ nên $AB=BC=a\sqrt{2}$.\\
		$\Rightarrow SM=\dfrac{AB}{2}=\dfrac{a\sqrt{2}}{2}$.\\
		Do đó $\cos \left(\vec{SM},\vec{BC}\right)=\dfrac{-\dfrac{a^2}{2}}{\dfrac{a\sqrt{2}}{2} \cdot a\sqrt{2}}=-\dfrac{1}{2}$. Suy ra $\left(\vec{SM},\vec{BC}\right)={120}^\circ$.
	}
\end{ex}

%Câu 22
\begin{ex}
	Cho hình hộp $ABCD.A'B'C'D'$. Xét các điểm $M$, $N$ lần lượt thuộc các đường thẳng $A'C$, $C'D$ sao cho đường thẳng $MN$ song song với đường thẳng $BD'$. Khi đó tỉ số $\dfrac{MN}{BD'}$ bằng
	\loigiai{
		\shortans{0,25}
		\begin{center}
			\begin{tikzpicture}[line join = round, line cap = round, thick, font = \small, scale = 1]
				\path 
				(0:0) coordinate (D)
				+(100:3) coordinate (D')
				+(0:3) coordinate (C)
				+(35:2) coordinate (A)
				($(C)+(D')-(D)$) coordinate (C')
				($(D')+(A)-(D)$) coordinate (A')
				($(C)+(A)-(D)$) coordinate (B)
				($(C')+(A')-(D')$) coordinate (B')
				($(C)!1/4!(A')$) coordinate (M)
				($(C')!.5!(D)$) coordinate (N)
				;
				\draw[dashed] 
				(A')--(A) (A)--(B) (A)--(D) (B)--(D') (A')--(C) (M)--(N)
				;
				\draw 
				(A')--(B')--(B)--(C)--(D)--(D')--cycle
				(C')--(B') (C')--(D') (D)--(C')--(C)
				;
				\foreach \x/\g in {D/-90,C/-90,D'/180,A/-60,C'/-45,A'/90,B/0,B'/90,M/-135,N/135}
				\fill (\x) circle (1.5pt)
				+(\g:3mm) node {$\x$};
				\end{tikzpicture}
		\end{center}
		Đặt $\vec{BA}=\vec{x}$, $\vec{BB'}=\vec{y}$, $\vec{BC}=\vec{z}$.\\
		Do $\vec{CM}$, $\vec{CA'}$ là hai vectơ cùng phương $\Rightarrow \exists \,k\in \mathbb{R}\colon \,\vec{CM}=k \cdot \vec{CA'}$.\\
		Và $\vec{C'N}$, $\vec{C'D}$ là hai vectơ cùng phương $\Rightarrow \exists \,h\in \mathbb{R}\colon \,\vec{C'N}=h \cdot \vec{C'D}$.\\
		Ta có: $\vec{BD'}=\vec{BA}+\vec{BC}+\vec{BB'}=\vec{x}+\vec{y}+\vec{z}$, \hfill (1)\\
		$\begin{aligned}[t]
			\vec{MN} & =\vec{CN}-\vec{CM}=\vec{CC'}+\vec{C'N}-\vec{CM}=\vec{CC'}+h \cdot \vec{C'D}-k \cdot \vec{CA'} \\
			& =\vec{y}+h \cdot (-\vec{y}+\vec{x})-k \cdot \left(\vec{y}-\vec{z}+\vec{x}\right)=(h-k) \cdot \vec{x}+(1-h-k) \cdot \vec{y}+k \cdot \vec{z}
		\end{aligned}$ \hfill (2)\\
		Do $MN\parallel B'D$ nên tồn tại $t\in \mathbb{R} \colon \vec{MN}=t \cdot \vec{BD'}$.\\
		Từ (1) và (2) ta có$\heva{& h-k=t \\& 1-h-k=t \\& k=t}\Leftrightarrow \heva{& k=t \\& h=2t \\& 1-3t=t}\Rightarrow t=\dfrac{1}{4}\Rightarrow \vec{MN}=\dfrac{1}{4}\vec{BD'}$.\\
		Vậy $\dfrac{MN}{BD'}=\dfrac{1}{4}=0,25$.
	}
\end{ex}