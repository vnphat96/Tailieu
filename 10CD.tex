\documentclass[10pt,a4paper,onecolumn,titlepage,twoside,openany]{book}
% \usepackage[utf8]{vietnam}
%\usepackage{fouriernc}
%%%%%%%%%%%%% KIỂU MÀU VÀ ĐỘ RỘNG NOTE
\def\kieumau{Y} %Y: Màu; N: đen-trắng
% \def\kieumau{N} %Y: Màu; N: đen-trắng
\def\leftnote{5} %Độ rộng cột Note
%%%%%%%%%%%%% ĐN CƠ BẢN
\input{cautrucDT/color\kieumau} %MÀU
%=====================================
% Khai báo nhóm Tex (cơ bản)
%=====================================
\usepackage{amsmath,amssymb,mathrsfs,maybemath,xlop,polynom,slashbox}
\usepackage{yhmath} %\let\widering\relax %cần khi sd với font fouriernc

\usepackage{enumerate}
\usepackage{tikz} 
\usepackage{tkz-euclide}
%\usepackage{ex_tkz-euclide}
%\usetkzobj{all}
\usepackage{tikz-3dplot}
\usepackage{tkz-tab}
\usepackage{pifont} %kí hiệu đặc biệt
% \usepackage{xcolor}
%\usepackage{bbding}
%\usepackage{array}
\usepackage{tasks}
% \usepackage{casiovn}
%==========
\usetikzlibrary{math,through,calc,intersections,angles,quotes,shapes,shapes.geometric,arrows,patterns,snakes,matrix,chains,arrows.meta,decorations.shapes,decorations.fractals,decorations.markings,shadows}
\usetikzlibrary{positioning,decorations.text,decorations.pathmorphing}% Để uốn cong văn bản 
\usetikzlibrary{shadings,fadings} %ĐỔ BÓNG
\usepackage{pgfplots}
\usepackage{pgfornament}
\usepgfplotslibrary{fillbetween}
\pgfplotsset{compat=1.9}
\usepackage[hidelinks,unicode]{hyperref}
\usepackage{currfile}
\usepackage[outline]{contour} %viền
\usepackage{fontawesome} % Gói kí hiệu
\usepackage{lipsum} %Lấy text
\usepackage{tabularx}
%%---------
%\usepackage{setspace}
%\usepackage{scrextend}
\usepackage{varwidth}
%===========Bảng
\usepackage{longtable,multirow,makecell}
\usepackage{diagbox}
\renewcommand{\tabcolsep}{3mm}
\newcolumntype{C}[1]{>{\centering\arraybackslash}p{#1}}
\newcolumntype{L}[1]{>{\raggedright\arraybackslash}p{#1}}
%-----------Trang vb


%%%%%%%%%%%%% Các thông số trang tài liệu
\def\tren{1.5}\def\duoi{1.5}\def\trai{1.25}\def\phai{0.75} %cách lề
\def\topset{0.75} %kc giữa đáy header và vùng vb
\def\botset{0.75} %kc giữa đỉnh footer và vùng vb
%\usepackage{ifoddpage}
\pgfmathsetmacro{\mepphai}{\phai+\leftnote} 
%\usepackage[top=\tren cm, bottom=\duoi cm, left=\trai cm, right=\mepphai cm] {geometry}
%%%%%%%%%%%%%
%---------------------------------Các thông số trang tài liệu
\pgfmathsetmacro{\so}{\leftnote - 0.5} 
\usepackage[top=\tren cm, bottom=\duoi cm, left=\trai cm, right=\mepphai cm,
marginparwidth=\so cm, marginparsep=5mm,
%,headsep=6mm
%,footskip=10mm
] {geometry}
%-------------------------------------
\usepackage{marginnote}
\setlength{\marginparwidth}{\so cm}
\renewcommand*{\marginfont}{\small}
%--------------Gói trắc nghiệm EX-TEST
% \usepackage[dethi]{ex_test}
\usepackage[loigiai]{ex_test} 
% \usepackage[solcolor]{ex_test}
%----Lời giải, Hiền thị tên EX; Dấu kết thúc
\font\damEX=ugqb8v at 11pt
\def\loigiaiEX{\color{\mauLG}\damEX\strut\faCommenting\ Lời giải.}
%lời giải EXS
\def\loigiaiEXS{\loigiaiEX{\fontsize{8}{16}\selectfont\color{\maucham}\dotfill}}
%--
\renewcommand{\nameex}{\damEX\color{\mauEX} CÂU}
\newtheorem{EX}{\nameex} %MÔI TRƯỜNG PHỤ CHO TÁCH CÂU
\def\mauVuong{cyan}
\def\qedEX{\color{\mauVuong}\ensuremath{\square}}
%--------------Cài đặt lại dòng kẻ \dotline
\renewcommand{\dotlineEX}[1]{
	\def\numlinedot{#1}
	\par
	\foreach \dotline in{1,...,\numlinedot}
	{
		\noindent
		\fontsize{8}{16}\selectfont
		\color{\maucham}\dotfill
		\par
	}
}
% sd cho \dongcham
\newcommand{\dotlineEXS}[1]{
	\def\numlinedot{#1}
	\foreach \dotline in{1,...,\numlinedot}
	{
		\noindent
		\fontsize{8}{16}\selectfont
		\color{\maucham}\dotfill
		\par
	}
}
%---------- Khai báo viết tắt, in đáp án
\newcommand{\hoac}[1]{ %hệ hoặc
	\left[\begin{aligned}#1\end{aligned}\right.}
\newcommand{\heva}[1]{ %hệ và
	\left\{\begin{aligned}#1\end{aligned}\right.}
%--In đáp án
\newcommand{\indapan}[2]{
	\addcontentsline{toc}{subsection}{\sf Bảng đáp án} % đưa MT vào mục lục
	\begin{center}
		\begin{tikzpicture}%
			\node[thick,scale=1,fill=\mauEX!2,draw=\maufoot,minimum width=3.5cm,minimum height=0.1cm,rounded corners=2mm]{\fontfamily{qag}\fontsize{11}{11}\selectfont\bfseries\color{\mauEX} BẢNG ĐÁP ÁN};
		\end{tikzpicture}%
	\end{center}
	\inputansbox{#1}{#2}
}
%----------
\usepackage{esvect}
\def\vec{\vv} %vecto
\def\overrightarrow{\vv}
%Lệnh song song
\DeclareSymbolFont{symbolsC}{U}{txsyc}{m}{n}
\DeclareMathSymbol{\varparallel}{\mathrel}{symbolsC}{9}
\DeclareMathSymbol{\parallel}{\mathrel}{symbolsC}{9}
%--------------------------
% HEADER AND FOOTER STYLING
%--------------------------
%--------------------------
\newcommand{\myfancyhead}{% trên và chấm trái
		\boldmath
\begin{tikzpicture}[remember picture,overlay,>=stealth]
		\path ([yshift=-\tren cm+0.5*\topset cm]current page.north west) coordinate (AA)
		++(\paperwidth,0)coordinate (BB); 
\checkoddpage\ifoddpage %nếu trang lẻ
		%-----đường kẻ
		\draw[\maufoot, line width=2pt] 
		([xshift=\trai cm]AA) --([xshift=-\phai cm]BB);
		%-----bên phải
		\node[text=\maufoot, anchor=south east,inner sep=0pt] at ([xshift=-\phai cm,yshift=4pt]BB){
			\fontfamily{qag}\fontsize{8.5pt}{12pt}\selectfont 
			{\color{\mauSO}\faMapMarker}\,\, \diachi\,\,{\color{\mauSO}\faMapMarker}
		};
		%-----bên trái
		\node[text=\maufoot, anchor=south west,inner sep=0pt] at ([xshift=\trai cm,yshift=4pt]AA){
			\fontfamily{qag}\fontsize{10pt}{10pt}\selectfont\bfseries\faEdit\, \tenchuyende
		};
		%----- Kẻ đứng
		\draw[\maufoot] ([xshift=-\mepphai cm+2.5mm]BB)--([yshift=\duoi cm-0.75*\botset cm,xshift=-\mepphai cm+2.5mm]current page.south east);
		%--		
		\path ([yshift=-\tren cm+0.5*\topset cm-0.5cm,xshift=-\phai cm-0.5*\leftnote cm+2.5mm]current page.north east) coordinate (DDD); 
		\begin{scope}
			\clip ([yshift=-\tren cm+0.5*\topset cm-1pt,xshift=-\phai cm]current page.north east) rectangle ([yshift=\duoi cm-0.5*\botset cm,xshift=-\mepphai cm+5mm]current page.south east);% cắt chấm
			\node[inner sep =0pt,scale=1,anchor=north] at ([yshift=0cm,xshift=0pt]DDD) {
				\parbox{\leftnote cm}{\centering
					\def\maucham{\maufoot}\dotlineEX{60}
				}
			};
			%--note dưới
			\node[inner sep =6pt, text=white,scale=1,anchor=north,fill=\maufoot] (noteduoi) at ([yshift=2.5mm]DDD) {
				\parbox{\leftnote cm-5mm-12pt}{ \fontsize{11}{1}\fontfamily{qag}\selectfont\bfseries\centering
					QUICK NOTE
				}
			};
			\draw[\maufoot, line width=0.4pt] ([yshift=-2pt]noteduoi.south west)--([yshift=-2pt]noteduoi.south east);
		\end{scope}
\else %chẵn
		%-----đường kẻ
		\draw[\maufoot, line width=2pt] 
		([xshift=\phai cm]AA) --([xshift=-\trai cm]BB);
		%-----bên trái
		\node[text=\maufoot, anchor=south west,inner sep=0pt] at ([xshift=\phai cm,yshift=4pt]AA){
			\fontfamily{qag}\fontsize{8.5pt}{12pt}\selectfont 
			{\color{\mauSO}\faMapMarker}\,\, \diachi\,\,{\color{\mauSO}\faMapMarker}
		};
		%-----bên phải
		\node[text=\maufoot, anchor=south east,inner sep=0pt] at ([xshift=-\trai cm,yshift=4pt]BB){
			\fontfamily{qag}\fontsize{10pt}{10pt}\selectfont\bfseries\faEdit\, \tenchuyende
		};
		%----- Kẻ đứng
		\draw[\maufoot] ([xshift=\mepphai cm-2.5mm]AA)--([yshift=\duoi cm-0.75*\botset cm,xshift=\mepphai cm-2.5mm]current page.south west);
		%--		
		\path ([yshift=-\tren cm+0.5*\topset cm-0.5cm,xshift=\phai cm+0.5*\leftnote cm-2.5mm]current page.north west) coordinate (DDD); 
		\begin{scope}
			\clip ([yshift=-\tren cm+0.5*\topset cm-1pt,xshift=\phai cm]current page.north west) rectangle ([yshift=\duoi cm-0.5*\botset cm,xshift=\mepphai cm-5mm]current page.south west);% cắt chấm
			\node[inner sep =0pt,scale=1,anchor=north] at ([yshift=0cm,xshift=0pt]DDD) {
				\parbox{\leftnote cm}{\centering
					\def\maucham{\maufoot}\dotlineEX{60}
				}
			};
			%--note dưới
			\node[inner sep =6pt, text=white,scale=1,anchor=north,fill=\maufoot] (noteduoi) at ([yshift=2.5mm]DDD) {
				\parbox{\leftnote cm-5mm-12pt}{ \fontsize{11}{1}\fontfamily{qag}\selectfont\bfseries\centering
					QUICK NOTE
				}
			};
			\draw[\maufoot, line width=0.4pt] ([yshift=-2pt]noteduoi.south west)--([yshift=-2pt]noteduoi.south east);
		\end{scope}
\fi
\end{tikzpicture}%
}
% trên mục lục
\newcommand{\headmucluc}{%
	\boldmath
\begin{tikzpicture}[remember picture,overlay,>=stealth]
	\path ([yshift=-\tren cm+0.5*\topset cm]current page.north west) coordinate (AA)
	++(\paperwidth,0)coordinate (BB); 
\checkoddpage\ifoddpage %nếu trang lẻ
	%-----đường kẻ
	\draw[\maufoot, line width=2pt] 
	([xshift=\trai cm]AA) --([xshift=-\phai cm]BB);
	%-----bên phải
	\node[text=\maufoot, anchor=south east,inner sep=0pt] at ([xshift=-\phai cm,yshift=4pt]BB){
		\fontfamily{qag}\fontsize{8.5pt}{12pt}\selectfont 
		{\color{\mauSO}\faMapMarker}\,\, \diachi\,\,{\color{\mauSO}\faMapMarker}
	};
	%-----bên trái
	\node[text=\maufoot, anchor=south west,inner sep=0pt] at ([xshift=\trai cm,yshift=4pt]AA){
		\fontfamily{qag}\fontsize{12pt}{12pt}\selectfont\bfseries\faEdit\, \tenchuyende
	};
\else %chẵn
	%-----đường kẻ
	\draw[\maufoot, line width=2pt] 
	([xshift=\phai cm]AA) --([xshift=-\trai cm]BB);
	%-----bên trái
	\node[text=\maufoot, anchor=south west,inner sep=0pt] at ([xshift=\phai cm,yshift=4pt]AA){
		\fontfamily{qag}\fontsize{8.5pt}{12pt}\selectfont 
		{\color{\mauSO}\faMapMarker}\,\, \diachi\,\,{\color{\mauSO}\faMapMarker}
	};
	%-----bên phải
	\node[text=\maufoot, anchor=south east,inner sep=0pt] at ([xshift=-\trai cm,yshift=4pt]BB){
		\fontfamily{qag}\fontsize{12pt}{12pt}\selectfont\bfseries\faEdit\, \tenchuyende
	};
\fi
\end{tikzpicture}%
}
%===========================
\newcommand{\myfancyfoot}{% dưới
	\begin{tikzpicture}[remember picture,overlay]
	\path ([yshift=\duoi cm-0.75*\botset cm]current page.south west) coordinate (AA)
	++(\paperwidth,0)coordinate (BB); 
	\checkoddpage\ifoddpage %nếu trang lẻ
		%---kẻ
		\draw[\maufoot, line width=2pt] ([xshift=2*\trai cm+4pt]AA)--([xshift=-\phai cm+3pt]BB);
		%-----bên trái
		\fill[fill=\maufoot, rounded corners=2mm] ([xshift=2*\trai cm,yshift=0.25 cm]AA) rectangle +(-3*\trai cm,-0.5cm);
		%-----trang
		\node[anchor=west,text=white,inner sep=0pt,xshift=-0.75cm] at ([xshift=2*\trai cm]AA) {\fontfamily{put}\bfseries\thepage};
		%-----tên tg
		\node[anchor=west,text=\maufoot,inner sep=0pt,fill=white] at ([xshift=2*\trai cm]AA){\fontfamily{qag}\fontsize{9pt}{1pt}\selectfont\bfseries \,\,\, \tentacgia \,\,\, };
	\else %chẵn
		%---kẻ
		\draw[\maufoot, line width=2pt] ([xshift=-2*\trai cm+4pt]BB)--([xshift=\phai cm-3pt]AA);
		%-----bên trái
		\fill[fill=\maufoot, rounded corners=2mm] ([xshift=-2*\trai cm,yshift=0.25 cm]BB) rectangle +(3*\trai cm,-0.5cm);
		%-----trang
		\node[anchor=east,text=white,inner sep=0pt,xshift=0.75cm] at ([xshift=-2*\trai cm]BB) {\fontfamily{put}\bfseries\thepage};
		%-----tên tg
		\node[anchor=east,text=\maufoot,inner sep=0pt,fill=white] at ([xshift=-2*\trai cm]BB){\fontfamily{qag}\fontsize{9pt}{1pt}\selectfont\bfseries \,\,\, \tentacgia \,\,\,};
	\fi
	\end{tikzpicture}%
}
%======================Head chapter theo note, fullwidth
%----------------------
\usepackage{changepage}
\strictpagecheck
\usepackage{lastpage}
\usepackage{fancyhdr,lastpage}
\pagestyle{fancy}
\fancyhf{}
\fancypagestyle{plain}{
	\fancyhead[LO,RE]{\headmucluc}
	\fancyfoot[LO,RE]{\myfancyfoot}
}
\fancyhead[LO,RE]{\myfancyhead}
\fancyfoot[LO,RE]{\myfancyfoot}
\renewcommand{\footrulewidth}{0pt}
\renewcommand{\headrulewidth}{0pt}
%--------------4.2
\usepackage[most]{tcolorbox}
\colorlet{tcbcol@back}{tcbcolback}
\colorlet{tcbcol@frame}{tcbcolframe}
%---------------------------------------------------------------
% ĐỊNH NGHĨA SECTION. SUBSECTION, SUBSUBSECTION ... THEO Ý RIÊNG
%---------------------------------------------------------------
\usepackage[explicit]{titlesec} % để gọi #1
\usepackage{titledot} % gói lệnh chứa cả titlesec và titletoc
%=====================================
\setcounter{secnumdepth}{4} %độ sâu
\renewcommand{\thechapter}{\Roman{chapter}}
\renewcommand{\thesection}{\arabic{section}}
\renewcommand{\thesubsection}{\Alph{subsection}}
\renewcommand{\thesubsubsection}{\arabic{subsubsection}}
%--------------Tròn
\newcommand{\tron}[1]{
	\begin{tikzpicture}[baseline=(A.base)]%
		\node[circle,draw=\mauSUBSEC,line width=0.5pt,fill=white,inner sep=2pt,outer sep=1pt] (A) {\color{white} #1};
		\node[circle,draw=none,fill=\mauSUBSEC,inner sep=1pt,outer sep=1pt] (A) {\color{white} #1};
	\end{tikzpicture}%
}
%================= Đn chương
\font\fontchap=ugqb8v at 21pt
\titlespacing{\chapter}{0cm}{0cm}{0.5cm}[0cm] %1: , 2: Trên, 3: dưới
\titleformat{\chapter}[display]
{\fontsize{20pt}{20pt}\fontfamily{qag}\selectfont\bfseries\color{\mauCHUONG}} %định dạng chung
{\fontsize{16pt}{20pt}\selectfont\chaptername\, \thechapter.} %đánh số
{1mm}
{\fontchap\centering\MakeUppercase{#1}}
[\vspace{0cm}]
%============================Mục lục - Chapter*
\titleformat{name=\chapter,numberless}[display]
{\fontsize{14pt}{16pt}\fontfamily{qag}\selectfont\bfseries\color{\mauCHUONG}} %định dạng chung
{}
{-1em}
{%
	\begin{tikzpicture}
		%-----Nội dung
		\node[inner sep=0pt,right] (ndchuong) at (0,0){\fontchap \MakeUppercase{#1}};
%		%-----Đường kẻ ngang
%		\begin{scope}
%			\clip (0,-0.75) rectangle +(\textwidth,1.5);
%			\draw[\mauchuong,line width=2pt] (ndchuong.south east)++(10pt,8pt) --++(\linewidth,0);
%		\end{scope}
	\end{tikzpicture}
}
[
\vspace{-3mm}
%\thispagestyle{empty}
]
%--------Đn Section---------------------------
%\titlespacing*{\section}{0cm}{0cm}{0cm}[0cm]
\titleformat
{\section}
{\color{\mauSEC}\fontfamily{qag}\fontsize{16pt}{1pt}\selectfont\bfseries\centering}
{Bài\,\thesection.}
{3mm}
{\MakeUppercase{#1}}
[]
%-------Đn subsection---------------------------
\titlespacing{\subsection}{0cm}{0cm}{0cm}[0cm]
\titleformat{\subsection}
{\normalfont\fontsize{15pt}{20pt}\fontfamily{put}\selectfont\bfseries\color{\mausubsec}}
{\thesubsection.}
{3mm}
{\MakeUppercase{#1}}
[]
%----------ĐN subsubsection-----------------------
\titlespacing{\subsubsection}{0pt}{0mm}{0mm}[0cm]
\titleformat{\subsubsection}
{\fontsize{13pt}{18pt}\fontfamily{put}\selectfont\bfseries\color{\mausubsubsec}}
{\thesubsubsection.}
{3mm}
{#1}
[]
%----------ĐN paragraph-----------------------
\titlespacing{\paragraph}{0pt}{0mm}{0mm}[0cm]
\titleformat{\paragraph}
{\fontsize{11.5pt}{17pt}\fontfamily{put}\selectfont\bfseries\color{\mausubsubsec}}
{\theparagraph.}
{3mm}
{#1}
[]
%============================
\def\itemKN{\color{\mauitemKN}\faCheckSquareO}
\def\itemCI{\color{\mauitemCI}\faCheckCircleO}
%%======= Thiết lập labelitem, labelenumerate
%\renewcommand{\labelitemi}{\color{red}\faCheckSquareO}
\renewcommand{\labelitemi}{\color{\mauitem}\faCheckCircleO}
\renewcommand{\labelitemii}{\color{\mauitem}\bf ---}
\renewcommand{\labelitemiii}{\color{\mauitem}\bf +}
\renewcommand{\labelenumi}{\alph{enumi})}
%\renewcommand{\labelenumii}{\color{blue}\bf\arabic{enumi}.\arabic{enumii}}
%============================
%============================
% Canh chỉnh mục lục chính
\setcounter{secnumdepth}{4} %Độ sâu đánh số
\setcounter{tocdepth}{2} %Độ sâu mục lục
\contentsmargin{0cm}
%~~~~~~~~~~~~~~~~~~~~~
\renewcommand*\l@part[2]{%
	\ifnum \c@tocdepth >-2\relax
	\addpenalty{-\@highpenalty}%
	\addvspace{10pt \@plus\p@}%
	\setlength\@tempdima{3em}%
	\begingroup
	\hypersetup{linkcolor=violet}
	\tikz[remember picture, overlay]{
		\fill[\mauPHAN] (0,0) rectangle +(\textwidth,1);
		\draw (0,0.5) node[right=5pt]
		{\color{white}\fontsize{16pt}{1pt}\fontfamily{qag}\selectfont\bfseries  {\scshape Phần} #1};
	}
	\par\smallskip
	\penalty\@highpenalty
	\endgroup
	\fi
}
%------------------------
%\titlecontents{part}[0pc]
%{\addvspace{10pt}%
%	\color{red!70!black}\fontsize{18pt}{1pt}\fontfamily{qag}\selectfont\bfseries 
%}%
%{}
%{}
%{}%
%~~~~~~~~~~~~~~~~~~~~~
\titlecontents{chapter}[6.5pc] % nd cách trái
{\addvspace{5pt}%
	\color{\mauCHUONG}\fontsize{13pt}{16pt}\fontfamily{put}\selectfont\bfseries
}%
{\contentslabel[\chaptertitlename\,\thecontentslabel.]{6.5pc}} %nhãn
{}
{\hfill\bfseries\thecontentspage
}%
[\vspace*{5pt}]
%~~~~~~~~~~~~~~~~~~~~~
\titlecontents{section}[10pc]
{\addvspace{0pt}\bfseries\color{\mauSEC}}
{\fontsize{12.5pt}{15pt}\selectfont\sffamily\contentslabel[{Bài\,\thecontentslabel.}]{3.5pc}}
{}
{\hfill
	\thecontentspage
}
[]
%~~~~~~~~~~~~~~~~~~~~~
\titlecontents{subsection}[10pc]
{\addvspace{0pt}\color{\mauSUBSEC}}
{\fontsize{12pt}{15pt}\selectfont\sffamily\contentslabel[\tron{\thecontentslabel}]{2.8pc}}
{}
{{\tiny\dotfill}\thecontentspage}
[]
%~~~~~~~~~~~~~~~~~~~~~
%--------------------------
% ĐỊNH NGHĨA CÁC MÔI TRƯỜNG 
%--------------------------
\listenumerate{dn,dl,tc,nx,ex}%xuống dòng khi liệt kê
\theoremstyle{plain} %
\theoremheaderfont{\scshape} %đầu
\theorembodyfont{\normalfont} % thân
\theoremseparator {.} % Ngăn cách
\newtheorem{dn}{\color{\maudn}\faBolt\, Định nghĩa}[section]
%===================================ĐNghĩa
\theoremstyle{plain} %
\theoremheaderfont{\fontfamily{put}\bfseries} %đầu
\theorembodyfont{\normalfont} % thân
\theoremseparator {.} % Ngăn cách
\newtheorem{vd}{\color{\mauVD}\damEX
	%\faToggleOn\ 
	%\faUnlink\ 
	VÍ DỤ}%[section]
\newtheorem{bt}{\color{\mauBT}\damEX BÀI}
%===================================
\theoremstyle{plain} %
\theoremheaderfont{\scshape} %đầu
\theorembodyfont{\slshape} % thân 
\theoremseparator {.} % Ngăn cách 
\newtheorem{dl}{\color{\maudl}\faBolt\, Định lí}[section]
\newtheorem{tc}{\color{\mauhq!70!black}\faBolt\, Tính chất}[section]
\newtheorem{hq}{\color{\mauhq!70!black}\faBolt\, Hệ quả}[section]
%====================================
\theoremstyle{nonumberplain} %ko đánh số, ko xuống dòng
\theoremheaderfont{\scshape} %đầu
\theorembodyfont{\normalfont} %phần thân
\theoremseparator {.} %ngăn cách
\newtheorem{nx}{\color{\mauhq!70!black}\faBolt\, Nhận xét}
\newtheorem{tomtat}{\!\!\!\!\!\!\!\!}
%====================================Hộp
%--------------------Chú ý
\newenvironment{note}
{\begin{tcolorbox}
		[enhanced jigsaw,breakable,pad at break*=1mm,
		opacityback=0,boxrule=0pt,frame hidden,
		left=8mm, right=0pt, bottom=0pt, top=0pt,
		before skip=1mm,
		after skip=1mm,
		underlay unbroken and first={
			\draw ([xshift=0.3cm,yshift=-0.32cm]interior.north west) node[\mauly]{\large\bfseries \faExclamationTriangle};
		},
		fontupper=\it,
		]}
	{\end{tcolorbox}}
%\let\mynote\note
%\renewcommand{\note}{\mynote{\bfseries\color{\mauly}Lưu ý:}} 
%---------------Dạng toán
\newcounter{dang}\setcounter{dang}{0}
\renewcommand{\thedang}{\arabic{dang}}
%---Dạng 1
\newtcolorbox{dang}[1]{
	fonttitle=\fontfamily{qag}\bfseries,%fontupper=\itshape,
	colframe=\maudang,colback=yellow!20,coltitle=white,
	sharp corners, breakable, halign title=center,%adjusted title=center, %canh giữa DẠNG
	before skip=2mm,after skip=3mm,
	left=2mm,right=2mm,top=2mm,bottom=2mm,
	boxrule=1pt,
	title={\faFolderOpen\ Dạng~\stepcounter{dang}\thedang.\ #1}
		\addcontentsline{toc}{subsection}{\it\sffamily \faFolderOpen\ Dạng~\thedang.#1}
		\setcounter{subsubsection}{0}
		\setcounter{vd}{0}
		\setcounter{ex}{0}
		\setcounter{bt}{0}
}
%====================
\setlength{\parindent}{0pt} %không thụt đầu dòng
%--Name
\newcounter{deso}
\font\dam=ugqb8v at 13pt
\font\damTT=ugqb8v at 18pt
%================đn notenam
\def\notename{
		\begin{tikzpicture}[remember picture,overlay,>=stealth]
		\checkoddpage\ifoddpage %nếu trang lẻ
		%--tiêu đề phải
		\path ([yshift=-\tren cm+0.5*\topset cm-0.5cm,xshift=-\phai cm-0.5*\leftnote cm+2.5mm]current page.north east) coordinate (DD); 
		%--
		\fill[white] ([yshift=-\tren cm+0.5*\topset cm-5pt,xshift=\trai cm-2pt]current page.north east) rectangle ([yshift=\duoi cm-0.5*\botset cm-12cm,xshift=-\mepphai cm+3mm]current page.north east);
		\node[inner sep =0pt,anchor=north] (thanhta) at ([yshift=-2mm]DD) {
			\includegraphics[width=4.5cm]{logo/logo.jpg}		
		};	
		\else
		%--tiêu đề phải
		\path ([yshift=-\tren cm+0.5*\topset cm-0.5cm,xshift=\phai cm+0.5*\leftnote cm-2.5mm]current page.north west) coordinate (DD); 
		%--
		\fill[white] ([yshift=-\tren cm+0.5*\topset cm-5pt,xshift=\phai cm-2pt]current page.north west) rectangle ([yshift=\duoi cm-0.5*\botset cm-12cm,xshift=\mepphai cm-3mm]current page.north west);
		\node[inner sep =0pt,anchor=north] (thanhta) at ([yshift=-2mm]DD) {
			\includegraphics[width=4.5cm]{logo/logo.jpg}		
		};
		\fi
		%\draw (thanhta.south) node[below=0pt,xscale=0.8]{\small\normalfont\color{\mauname} Sưu tầm \& Biên tập};
		%---note
		\node[inner sep =6pt, text=black,scale=1,anchor=north,fill=\maufoot!3,draw=\maufoot] (bon) at ([yshift=-1cm]thanhta.south) {
			\parbox{\leftnote cm-5mm-12pt}{ \fontsize{10}{15}\selectfont\normalfont
				\vspace*{2pt}
				\chamngon%
			}
		};
		\draw[\maufoot, line width=5pt] (bon.north west)--(bon.north east);
		\draw[\maufoot!50] ([yshift=9pt,line width=0.4pt]bon.north east)--([yshift=9pt]bon.north west)
		node[fill=white,inner sep=2pt,anchor=south west,yshift=-2pt,xshift=-2pt]{\bfseries\color{\maufoot}ĐIỂM:}
		;
		%--note dưới
		\node[inner sep =6pt, text=white,scale=1,anchor=north,fill=\maufoot] (noteduoi) at ([yshift=-0.25cm]bon.south) {
			\parbox{\leftnote cm-5mm-12pt}{ \fontsize{11}{1}\selectfont\bfseries\centering
				QUICK NOTE
			}
		};
		\draw[\maufoot, line width=0.4pt] ([yshift=-2pt]noteduoi.south west)--([yshift=-2pt]noteduoi.south east);
	\end{tikzpicture}
}
%===================note và nonote
%FULL WIDTH
\def\FULLWIDTH{
	\newpage
	\fancyhead[LO,RE]{\headmucluc}
	\def\notename{}
	\newgeometry{top=\tren cm, bottom=\duoi cm, left=\trai cm, right=\phai cm}
}
\def\NOTE{
	\newpage
	\fancyhead[LO,RE]{\myfancyhead}
	\def\notename{
		\begin{tikzpicture}[remember picture,overlay,>=stealth]
			\checkoddpage\ifoddpage %nếu trang lẻ
			%--tiêu đề phải
			\path ([yshift=-\tren cm+0.5*\topset cm-0.5cm,xshift=-\phai cm-0.5*\leftnote cm+2.5mm]current page.north east) coordinate (DD); 
			%--
			\fill[white] ([yshift=-\tren cm+0.5*\topset cm-5pt,xshift=\trai cm-2pt]current page.north east) rectangle ([yshift=\duoi cm-0.5*\botset cm-12cm,xshift=-\mepphai cm+3mm]current page.north east);
			\node[inner sep =0pt,anchor=north] (thanhta) at ([yshift=-2mm]DD) {
				\includegraphics[width=4.5cm]{logo/logo.jpg}		
			};	
			\else
			%--tiêu đề phải
			\path ([yshift=-\tren cm+0.5*\topset cm-0.5cm,xshift=\phai cm+0.5*\leftnote cm-2.5mm]current page.north west) coordinate (DD); 
			%--
			\fill[white] ([yshift=-\tren cm+0.5*\topset cm-5pt,xshift=\phai cm-2pt]current page.north west) rectangle ([yshift=\duoi cm-0.5*\botset cm-12cm,xshift=\mepphai cm-3mm]current page.north west);
			\node[inner sep =0pt,anchor=north] (thanhta) at ([yshift=-2mm]DD) {
				\includegraphics[width=4.5cm]{logo/logo.jpg}		
			};
			\fi
			%\draw (thanhta.south) node[below=0pt,xscale=0.8]{\small\normalfont\color{\mauname} Sưu tầm \& Biên tập};
			%---note
			\node[inner sep =6pt, text=black,scale=1,anchor=north,fill=\maufoot!3,draw=\maufoot] (bon) at ([yshift=-1cm]thanhta.south) {
				\parbox{\leftnote cm-5mm-12pt}{ \fontsize{10}{15}\selectfont\normalfont
					\vspace*{2pt}
					\chamngon%
				}
			};
			\draw[\maufoot, line width=5pt] (bon.north west)--(bon.north east);
			\draw[\maufoot!50] ([yshift=9pt,line width=0.4pt]bon.north east)--([yshift=9pt]bon.north west)
			node[fill=white,inner sep=2pt,anchor=south west,yshift=-2pt,xshift=-2pt]{\bfseries\color{\maufoot}ĐIỂM:}
			;
			%--note dưới
			\node[inner sep =6pt, text=white,scale=1,anchor=north,fill=\maufoot] (noteduoi) at ([yshift=-0.25cm]bon.south) {
				\parbox{\leftnote cm-5mm-12pt}{ \fontsize{11}{1}\selectfont\bfseries\centering
					QUICK NOTE
				}
			};
			\draw[\maufoot, line width=0.4pt] ([yshift=-2pt]noteduoi.south west)--([yshift=-2pt]noteduoi.south east);
		\end{tikzpicture}
	}
	\newgeometry{top=\tren cm, bottom=\duoi cm, left=\trai cm, right=\mepphai cm}
}
%===================đn name
\newcommand{\name}[4]{
%	\NOTE
%	\newpage
	\setcounter{ex}{0}\setcounter{bt}{0}%\setcounter{EX}{0}
	\boldmath\fontfamily{qag}\selectfont\color{\mauname}
\hoten \dotfill {\fontsize{10}{11}\selectfont \ngaylamde}
	\begin{tcolorbox}[boxrule=0.7pt,arc=0mm,breakable,colframe=\mauSO,colback=\mauname!2,before skip=2mm,after skip=2mm]\color{\mauname}
	\begin{center}
		%%---
		{\damTT \MakeUppercase{#1}}\\[1pt]
		{\dam \MakeUppercase{#2 --- Đề} \stepcounter{deso}\thedeso}\\[1pt]
		% {\dam \MakeUppercase{#2}}\\[1pt]
		{\dam\color{\mauSO} \MakeUppercase{#3}}\\[1pt]
		{\fontsize{10}{10}\selectfont \textit{#4}}%\\[-1mm]
	\end{center}
	\end{tcolorbox}
	%%--- Phần note đầu đề
	\notename
\vspace*{0.5cm}
	\addcontentsline{toc}{section}{\hspace*{-4.2cm}\sf Đề \thedeso: #2 --- #3} % đưa MT vào mục lục
}
%--Sang trang 
\BeforeBeginEnvironment{name}{
	\ifnum\the\value{deso}>0
	\newpage
	\fi
}
%%---Đánh số trang
%\AtEndEnvironment{name}{
%	\ifnum\the\value{deso}=1
%	\pagenumbering{arabic}%đánh số trang dạng 1,2,...
%	\fi
%}
%---------
\def\chap#1{
	\begin{center}
		\fontchap\color{\mauCHUONG} #1
	\end{center}
	\addcontentsline{toc}{chapter}{\hspace*{-2.75cm}#1}
}
%---Hiện bảng ĐA
\newcommand{\hienDA}{
	\renewcommand{\indapan}[2]{
		\addcontentsline{toc}{subsection}{\hspace*{-4.2cm}\sf Bảng đáp án} % đưa MT vào mục lục
		%		\begin{center}
		\par\vspace*{5mm}
		\begin{tikzpicture}%
			\draw (0,0)++(0.5*\textwidth,0) node[thick,scale=1,fill=\mauEX!2,draw=\maufoot,minimum width=3.5cm,minimum height=0.1cm,rounded corners=2mm] {\damEX\color{\mauname} BẢNG ĐÁP ÁN};
		\end{tikzpicture}%
		%		\end{center}
		\vspace*{-2mm}
		\inputansbox{##1}{##2}
	}
}
%---Ẩn bảng ĐA
\newcommand{\anDA}{
	\renewcommand{\indapan}[2]{}
}
%---Dòng chấm từng câu
\newcommand{\dongchamEX}[1]{
%	\hideansEX{ex}
	\anLG
	\AfterEndEnvironment{ex}{%
		\foreach \cauEX/\dongEX in {#1}{
			\ifnum\dongEX=0
			\else
			\ifnum\the\value{ex}=\cauEX
			\par\noindent\loigiaiEXS\par
			\dotlineEX{\dongEX}
			\fi
			\fi
		}
	}
}
%---Dòng chấm nhiều câu
\newcommand{\dongchamEXS}[2]{
%	\hideansEX{ex}
	\anLG
	\AfterEndEnvironment{ex}{%
		\foreach \cauEX in {#1}{
			\ifnum#2=0
			\else
			\ifnum\the\value{ex}=\cauEX
			\par\noindent\loigiaiEXS\par
			\dotlineEXS{#2}
			\fi
			\fi
		}
	}
}
%---Dòng chấm từng câu theo đề
\newcommand{\DEdongchamEX}[2]{
%	\hideansEX{ex}
	\anLG
	\AfterEndEnvironment{ex}{%
		\foreach \cauEX/\dongEX in {#2}{
			\ifnum\dongEX=0
			\else
			\ifnum\the\value{deso}=#1
			\ifnum\the\value{ex}=\cauEX
			\par\noindent\loigiaiEXS\par
			\dotlineEX{\dongEX}
			\fi
			\fi
			\fi
		}
	}
}
%---Dòng chấm nhiều câu theo đề
\newcommand{\DEdongchamEXS}[3]{
%	\hideansEX{ex}
	\anLG
	\AfterEndEnvironment{ex}{%
		\foreach \cauEX in {#2}{
			\ifnum#3=0
			\else
			\ifnum\the\value{deso}=#1
			\ifnum\the\value{ex}=\cauEX
			\par\noindent\loigiaiEXS\par
			\dotlineEX{#3}
			\fi
			\fi
			\fi
		}
	}
}
%---Ẩn LG
\newcommand{\anLG}{
	\renewcommand{\loigiai}[1]{	}%
	% \chooseNSA
	\renewcommand{\TrueTF}{\FalseTF}
	\renewcommand{\TrueEX}{\FalseEX}
	\renewcommand{\writekeyTFone}{\gdef\TrueX{}\gdef\FalseX{}}
	\renewcommand{\writekeyTF}{&&}
}
%---Hiện LG
\newcommand{\hienLG}{
	%Xuất hiện chữ Lời giải trong môi trường onlysolution
	\renewcommand{\loigiai}[1]{%
		\begin{onlysolution}%
			##1
		\end{onlysolution}%
	}%
	%---
	\def\loigiaiEXS{}
	%\choiceTF
	\renewcommand{\writekeyTFone}{\gdef\TrueX{}\gdef\FalseX{\tickF}}
	\renewcommand{\writekeyTF}{%
		&\centering\leavevmode\TrueX%
		&\parbox[t]{\linewidth}{\centering\leavevmode\FalseX}%
			\gdef\TrueX{}\gdef\FalseX{\tickF}%
	}
	\def\kindSA{ShowSAKeyColor}
	\showanswers
	% \SAOPTN{kindSA=oly}
	\renewcommand{\dotlineEXS}[1]{}	
}
%=======Đn các phương án
\def\khoanhtrondapan{
	\renewcommand*\circled[1]{\tikz[baseline=(char.base)]{
			\node[shape=circle,draw=\mauDA,inner sep=1pt] (char) {##1};}}
	\renewcommand{\TrueEX}{\stepcounter{dapan}
		{\squareEX{\textbf{\damEX\color{\mauDA}\Alph{dapan}}}} \ignorespaces}
	\renewcommand{\FalseEX}{\stepcounter{dapan}
		{\circled{\textbf{\damEX\color{\mauDA}\Alph{dapan}}}} \ignorespaces}
	%---Chọn đáp án
	\renewcommand{\circEX}[2][fill=\mauTrue!3,draw=\mauTrue]{%
	\tikz[baseline=(char.base)]{\node[shape=circle,inner sep=1pt,##1] (char) {\color{red}##2};}}
}
%----
\def\khongkhoanhtrondapan{
	\renewcommand{\TrueEX}{\stepcounter{dapan}
		{\squareEX{\textbf{\damEX\color{\mauDA}\Alph{dapan}}}} \ignorespaces}
	\renewcommand{\FalseEX}{\stepcounter{dapan}
		{\textbf{\damEX\color{\mauDA}\Alph{dapan}.}} \ignorespaces}
	%---Chọn đáp án
	\renewcommand{\circEX}[2][fill=\mauTrue!3,draw=\mauTrue]{%
	\tikz[baseline=(char.base)]{\node[shape=circle,inner sep=1pt,##1] (char) {\color{red}##2};}}
}
%=============ĐN HIỆN CÂU EX CẦN THIẾT if
\newcommand{\hienEXS}[2]{
	%\foreach \bdem in {#1,...,#2}{%11-30
	\def\biendau{#1}\def\biencuoi{#2}%
	\pgfmathsetmacro{\sodau}{\fpeval{round(\biendau-1,0)}}
	\pgfmathsetmacro{\socuoi}{\fpeval{round(\biencuoi+1,0)}}
	\setcounter{EX}{#1-1}
	\RenewEnviron{ex}{
		\stepcounter{ex}%
		\ifnum\value{ex}<\socuoi
		\ifnum\value{ex}>\sodau
		\par%
		\begin{EX}
			\BODY% 
		\end{EX}
		\fi\fi
	}
	%
	\AtEndEnvironment{name}{\setcounter{EX}{#1-1}}
	%
	\AtEndEnvironment{EX}{
		\ifnum\the\value{numTrue}=1
		\scantokens{\begin{EXsol}A\end{EXsol}}
		\fi
		\ifnum\the\value{numTrue}=2
		\scantokens{\begin{EXsol}B\end{EXsol}}
		\fi
		\ifnum\the\value{numTrue}=3
		\scantokens{\begin{EXsol}C\end{EXsol}}
		\fi
		\ifnum\the\value{numTrue}=4
		\scantokens{\begin{EXsol}D\end{EXsol}}
		\fi
		\setcounter{numTrue}{0}
	}
}
%%%%%%%%%%%%%%%%%%%%%%%

%============================ Khung
\newenvironment{khung}
{\begin{tcolorbox}[
		enhanced,breakable,
		colback=yellow!10,
		colframe=blue,
		boxrule=0.5pt,
		%		drop fuzzy shadow=gray,
		left=5pt,right=5pt,top=5pt,bottom=5pt,
		arc=0mm
		]}
	{\end{tcolorbox}}
%-----------------------------Mục con = subsub
\newcounter{muccon}
\newcommand{\muccon}[1]{%
	\stepcounter{muccon}
	{%\setcounter{bt}{0}\setcounter{vd}{0}\setcounter{ex}{0}
		%\fontsize{13pt}{15pt}\selectfont
		%		\color{violet!70!black}\sffamily
		\bfseries\sffamily\bfseries\hspace*{0mm}\themuccon.\  
		#1}
}
%----------------------------------------------------

% Hộp định nghĩa
\newenvironment{boxdn}
{\begin{tcolorbox}
		[enhanced jigsaw,breakable,pad at break*=1mm,
		colback=cyan!2,
		%standard jigsaw, 
		opacityback=0, %ko nền
		boxrule=0pt,frame hidden, left=0.7cm, right=0pt, bottom=2pt, top=0pt,
		borderline west={1mm}{0.5cm}{cyan},
		overlay={
			\fill[fill=cyan!20,draw=none] ([xshift=0.6cm]interior.north west) rectangle (interior.south east)
			;
		}
		\setcounter{muccon}{0}
		]}%0mm lề trái
	{\end{tcolorbox}}
%===============================================
\theoremstyle{nonumberbreak} % ko đánh số
\theoremheaderfont{\sffamily\bfseries} %tên
\theorembodyfont{\normalfont} %thân
\theoremsymbol{\ensuremath{_\blacksquare}} %Dấu kết thúc là ô vuông đen.
\theoremseparator {:} % Dấu ngăn cách
\newtheorem{myphantich}{\color{violet}%\faServer\ 
	\faFileText\ PHÂN TÍCH}
%===============================================
\newenvironment{phantich}{\begin{boxdn}\begin{myphantich}}{\end{myphantich}\end{boxdn}}
%-------------- Khung (Trong main này ko sd)
\newtcolorbox[auto counter]{khung4}[1]{enhanced, breakable,
	before skip=1mm,after skip=1mm,
	left=1mm,right=1mm,top=2mm,bottom=1mm,
	colframe=myblue,colback=cyan!0,colbacktitle=cyan!6,coltitle=myblue,colupper=black,sharp corners,
	,boxrule=0.4mm,
	coltext=mauE,
	attach boxed title to top center=
	{yshift=-0.1mm-\tcboxedtitleheight/2,yshifttext=2mm-\tcboxedtitleheight/2},
	varwidth boxed title*=-3cm,
	boxed title style={boxrule=0.3mm,
		frame code={ \path[tcb fill frame] ([xshift=-4mm]frame.west)
			-- (frame.north west) -- (frame.north east) -- ([xshift=4mm]frame.east)
			-- (frame.south east) -- (frame.south west) -- cycle; },
		interior code={ \path[tcb fill interior] ([xshift=-2mm]interior.west)
			-- (interior.north west) -- (interior.north east)
			-- ([xshift=2mm]interior.east) -- (interior.south east) -- (interior.south west)
			-- cycle;} 
	},
	fonttitle=\fontsize{10}{0}
	\bfseries,
	fontupper=\fontsize{10}{0},
	title={#1}
}

\newcommand{\boxmini}[1]{
	\vspace*{-2mm}
	\begin{center}
		\begin{tikzpicture}[outline/.style={draw=##1,thick,fill=##1!3},outline/.default=myblue]
			\node [outline,
			sharp corners] at (0,0) {\fontfamily{qag} \selectfont\bfseries\color{\mauEX} #1};
		\end{tikzpicture}
	\end{center}
	\vspace*{0mm}
}

%-----------------------
\newcommand{\inden}[1]{
	{\fontsize{11pt}{9pt}\sffamily \selectfont\bfseries\color{\maudn} #1}
}
\newcommand{\indam}[1]{
	{\fontsize{11.5pt}{9pt}\sffamily \selectfont\bfseries\color{\maudn} #1}
}
\newcommand{\indamm}[1]{
	{\fontsize{11.5pt}{9pt}\sffamily \selectfont\bfseries\color{\maudl} #1}
}
\newcommand{\ind}[1]{
	{\fontsize{11.5pt}{9pt}\sffamily \selectfont\bfseries\color{\maucham} #1}
}
%%%===Các biểu tượng===
\def\iconGN{{\color{magenta}\faPencilSquareO}}
\def\iconNS{{\color{gray}\faStar}}
\def\iconQS{{\color{magenta}\faFolderOpen}}
\def\iconMT{{\color{magenta!80!black}\faSunO}}
\def\iconX{{\color{red}\faClose}}
\def\iconCH{{\color{myblue}\faCheckCircle}}
\def\iconVD{\faCubes}
\def\iconCV{{\color{myblue}\faCubes}}

\newcommand{\dongcham}[1]{
	\def\sod{#1}
	\pgfmathsetmacro{\sodong}{2*\sod -1} 
	\columnsep=10pt
	\vspace*{-3.5mm}
	\begin{multicols}{2}
		\foreach \dotline in{1,...,\sodong}
		{\noindent\color{gray}{\dotfill}\\[1mm]
		}\noindent\color{gray}{\dotfill}\\[-4mm]
	\end{multicols}
}


\def\TNTF{
    {\bfseries Phần II. Trong mỗi ý a), b), c) và d) ở mỗi câu, học sinh chọn đúng hoặc sai.}
}
\def\TN{
    {\bfseries Phần I. Mỗi câu hỏi học sinh chọn một trong bốn phương án A, B, C, D.}
}
\def\TNSA{
    {\bfseries Phần III. Học sinh điền kết quả vào ô trống.}
}
\def\BTTL{
    \begin{center}
        \fcolorbox{black}{white}{{\bfseries BÀI TẬP TỰ LUẬN TRẢ LỜI NGẮN}}
    \end{center}
}
\def\BTTF{
    \begin{center}
        \fcolorbox{black}{white}{{\bfseries BÀI TẬP TRẮC NGHIỆM ĐÚNG SAI}}
    \end{center}
%    \TNTF
}
\def\BTTN{
    \begin{center}
        \fcolorbox{black}{white}{{\bfseries BÀI TẬP TRẮC NGHIỆM 4 PHƯƠNG ÁN}}
    \end{center}
}
\def\TL{
    {\bfseries Phần II. Câu hỏi tự luận.}
}
 %Khai báo cơ bản
\usepackage{tkz-euclide,circuitikz,casio580x}
%%%%%%%%%%%%% ĐIỀU KHIỂN LỜI GIẢI ,DÒNG CHẤM, ĐÁP SỐ
%------------Dòng chấm bằng chiều dài LG (bật)
% \dotlinefull{ex}
%------------Thay Loi giải bằng n dòng kẻ (bật)
% \dotlineans{2}{ex}
%------------Ẩn lời giải
%\hideansEX{ex}
%------------Dòng chấm tùy ý (ko cần \loigiai{})
%---Nhiều câu cùng dòng chấm (tách dụng lên mọi đề)
%\dongchamEXS{1,...,20}{2}
%\dongchamEXS{21,...,40}{5}
%\dongchamEXS{41,...,50}{10}
%---Nhiều câu cùng dòng chấm (tách dụng lên MỘT ĐỀ đc chọn)
%\DEdongchamEXS{3}{1,...,20}{2} %{3} là đề thứ 3
%---Dòng chấm từng câu, tác dụng lên mọi đề
%\dongchamEX{1/3,2/5,3/7} % câu / số dòng chấm của câu đó
%---Dòng chấm từng câu, tác dụng lên 1 đê
%\DEdongchamEX{3}{1/3,2/5,3/7} % câu / số dòng chấm của câu đó, {3} là đề số 3 
%------------Ẩn đáp số (bật), đáp án
\exitdapso %ẩn đs
%\renewcommand{\indapan}[2]{} %ẩn đáp án
%%%%%%%%%%%%% khung NAME
\def\ngaylamde{Ngày làm đề: ...../...../........} %để {} nếu ko muốn
\def\tenchude{HỆ THỨC LƯỢNG TRONG TAM GIÁC}
% \def\tendethi{ }
\def\tentruong{PHedu}
\def\thoigian{Thời gian làm bài: 90 phút, không kể thời gian phát đề}
%%%%%%%%%%%%% Nội dung head & foot
% \def\diachi{ }
\def\diachi{VNPmath - 0962940819}
\def\tenchuyende{\tenchude}
\def\tentacgia{GV.VŨ NGỌC PHÁT}
\def\chamngon{\lq\lq It's not how much time you have, it's how you use it.\rq\rq
}
%%%%%%%%%%%%% Đn lại A.B.C.D
\khoanhtrondapan
% \khongkhoanhtrondapan
%%%%%%%%%%%%%
\renewcommand{\arraystretch}{1}
\newcommand{\viduminhhoa}{\subsubsection{Ví dụ minh hoạ}}
\newcommand{\baitaptl}{\subsubsection{Bài tập tự luận}}
%%===================================================
%=================BẮT ĐẦU TÀI LIỆU===================
\begin{document}
\renewcommand{\chaptername}{Chương}
\pagenumbering{arabic}%đánh số trang dạng 1,2,...
%====================================================
%==================BẮT ĐẦU TÀI LIỆU==================
%\hienEXS{41}{50} %chỉ hiện câu từ 41 đến 50 của đề
%--------Đề bài
\NOTE \anLG \anDA 
%--
\notename
%%Chương 1
% \setlistsEX{column-sep=-25pt,after-skip=-10pt,after-item-skip=0ex}
\section{Mệnh đề}
\subsection{Tóm tắt lý thuyết}
\begin{tomtat}
\subsubsection{Mệnh đề}
\begin{boxdn}{}
	\textit{Mệnh đề toán học} (gọi tắt là \textit{mệnh đề}) là một khẳng định về một sự kiện toán học \textbf{hoặc đúng hoặc sai}, \textbf{không thể vừa đúng vừa sai}.	
	\begin{itemize}
		\item Mệnh đề thường được kí hiệu bằng các chữ cái in hoa. Ví dụ: Q: \lq\lq  6 chia hết cho 3\rq\rq.
	\end{itemize}
\end{boxdn}

\begin{note}
	\begin{itemize}
		\item Các câu hỏi, câu cảm thán, câu mệnh lệnh không phải là mệnh đề.
		\item Một câu chưa xác định được đúng hay sai nhưng chắc chắn nó chỉ đúng hoặc sai (không thể vừa đúng vừa sai) cũng là một mệnh đề. Ví dụ: \lq\lq  $2^{2023^2+2023+1}+1$ là số nguyên tố\rq\rq\ là một mệnh đề.
		\item Trong thực tế, có những mệnh đề mà tính đúng sai của nó luôn gắn với một thời gian và địa điểm cụ thể: đúng ở thời gian hoặc địa điểm này nhưng sai ở thời gian hoặc địa điểm khác. Nhưng ở bất kì thời gian, địa điểm nào cũng luôn có giá trị chân lí hoặc đúng hoặc sai. Ví dụ: Số 1 là số tự nhiên nhỏ nhất. (Trong một số chương trình, tập số tự nhiên không bao gồm số 0. Tìm hiểu thêm ở topic: \lq\lq  Natural Number\rq\rq\ trên Wikipedia) 
	\end{itemize}
\end{note}
\subsubsection{Mệnh đề chứa biến}
\begin{boxdn}{}
	Những khẳng định mà tính đúng, sai của chúng phụ thuộc vào giá trị của biến gọi là \textit{mệnh đề chứa biến}.
\end{boxdn}
Ví dụ: Cho $P(x): x>x^2$ với $x$ là số thực. Ta chưa khẳng định được tính đúng sai của câu này, do đó nó chưa phải là mệnh đề.\\
Tuy nhiên, khi thay $x$ bởi những giá trị cụ thể thì ta được một mệnh đề, chẳng hạn, $P(2)$ là mệnh đề sai, $P\left(\dfrac{1}{2}\right)$ là mệnh đề đúng.

\subsubsection{Mệnh đề phủ định}

\begin{boxdn}{}
	Cho mệnh đề $P$. Mệnh đề \lq\lq  Không phải $P$\rq\rq\ được gọi là mệnh đề phủ định của $P$ và kí hiệu là $\overline{P}$.
	\begin{itemize}
		\item Mệnh đề $P$ và mệnh đề phủ định $\overline{P}$ là hai khẳng định trái ngược nhau. Nếu $P$ đúng thì $\overline{P}$ sai, nếu $P$ sai thì $\overline{P}$ đúng.
		\item Mệnh đề phủ định của $P$ có thể diễn đạt theo nhiều cách khác nhau. Chẳng hạn, xét mệnh đề $P$: \lq\lq  $2$ là số chẵn\rq\rq. Khi đó, mệnh đề phủ định của $P$ có thể phát biểu là $\overline{P}$: \lq\lq  $2$ không phải là số chẵn\rq\rq\ hoặc \lq\lq  $2$ là số lẻ\rq\rq.
	\end{itemize} 
\end{boxdn}

\subsubsection{Mệnh đề kéo theo và mệnh đề đảo}

\begin{boxdn}{}
	Cho hai mệnh đề $P$ và $Q$. Mệnh đề \lq\lq  Nếu $P$ thì $Q$\rq\rq\ được gọi là mệnh đề kéo theo.
	\begin{itemize}
		\item Kí hiệu là $P\Rightarrow Q.$
		\item Mệnh đề kéo theo chỉ sai khi $P$ đúng $Q$ sai.
		\item $P\Rightarrow Q$ còn được phát biểu là \lq\lq  $P$ kéo theo $Q$\rq\rq, \lq\lq  $P$ suy ra $Q$\rq\rq\ hay \lq\lq  Vì $P$ nên $Q$\rq\rq.
	\end{itemize}
\end{boxdn}

\begin{note}
	Trong toán học, định lí là một mệnh đề đúng, thường có dạng $P\Rightarrow Q$.
	Khi đó ta nói 
	\begin{itemize}
		\item $P$ là giả thiết, $Q$ là kết luận của định lí.
		\item $P$ là $\underline{\textit{điều kiện đủ}}$ để có $Q$, còn $Q$ là $\underline{\textit{điều kiện cần}}$ để có $P$.
	\end{itemize}
\end{note}

% \begin{note}
% 	Trong logic toán học, khi xét giá trị chân lí của mệnh đề $P\Rightarrow Q$ người ta không quan tâm đến mối quan hệ về nội dung của hai mệnh đề $P$, $Q$. Không phân biệt trường hợp $P$ có phải là nguyên nhân để có $Q$ hay không mà chỉ quan tâm đến tính đúng, sai của chúng.
	
% 	Ví dụ: \lq\lq  Nếu mặt trời quay quanh trái đất thì Việt Nam nằm ở châu Âu\rq\rq\ là một mệnh đề đúng. Vì ở đây hai mệnh đề $P$: \lq\lq  Mặt trời quay xung quanh trái đất\rq\rq\ và $Q$: \lq\lq  Việt Nam nằm ở châu Âu\rq\rq\ đều là mệnh đề sai.
% 	(Tìm hiểu thêm ở topic \lq\lq  Mệnh đề toán học\rq\rq trên Wikipedia)
% \end{note}

\begin{boxdn}{}
	Cho mệnh đề kéo theo $P\Rightarrow Q$. Mệnh đề $Q\Rightarrow P$ được gọi là mệnh đề đảo của mệnh đề $P\Rightarrow Q$.
\end{boxdn}

\begin{note}
	Mệnh đề đảo của một mệnh đề đúng không nhất thiết là một mệnh đề đúng.
\end{note}

\subsubsection{Mệnh đề tương đương}

\begin{boxdn}{}
	Cho hai mệnh đề $P$ và $Q$. Mệnh đề có dạng \lq\lq  $P$ nếu và chỉ nếu $Q$\rq\rq\ được gọi là mệnh đề tương đương.
	\begin{itemize}
		\item Kí hiệu là $P \Leftrightarrow Q$.
		\item Mệnh đề $P \Leftrightarrow Q$ đúng khi cả hai mệnh đề $P\Rightarrow Q$ và $Q \Rightarrow P$ cùng đúng hoặc cùng sai. \\
		(Hay $P \Leftrightarrow Q$ đúng khi cả hai mệnh đề $P$ và $Q$ cùng đúng hoặc cùng sai).
		\item $P\Leftrightarrow Q$ còn được phát biểu là \lq\lq  $P$ khi và chỉ khi $Q$\rq\rq, \lq\lq  $P$ tương đương với $Q$\rq\rq, hay \lq\lq  $P$ là điều kiện cần và đủ để có $Q$\rq\rq.
	\end{itemize}
\end{boxdn}

% \begin{note}
% 	Trong logic học, hai mệnh đề $P$, $Q$ tương đương với nhau hoàn toàn không có nghĩa là nội dung của chúng như nhau, mà nó chỉ nói lên rằng chúng có cùng giá trị chân lí (cùng đúng hoặc cùng sai).\\
% 	Ví dụ: \lq\lq  Hình vuông có một góc tù khi và chỉ khi 100 là số nguyên tố\rq\rq\ là một mệnh đề đúng.
% \end{note}

\subsubsection{Mệnh đề có chứa kí hiệu $\forall$ và $\exists$}

\begin{itemize}
	\item Kí hiệu $\forall$ (với mọi): \lq\lq $ \forall x \in X, P(x)$\rq\rq\ hoặc \lq\lq $ \forall x \in X : P(x)$\rq\rq.
	\item Kí hiệu $\exists$ (tồn tại): \lq\lq $ \exists x \in X, P(x)$\rq\rq\ hoặc \lq\lq $ \exists x \in X : P(x)$\rq\rq.
\end{itemize}

\begin{note}\hfil
	\begin{itemize}
		\item Phủ định của mệnh đề \lq\lq $ \forall x \in X, P(x)$\rq\rq\ là mệnh đề \lq\lq $ \exists x\in X, \overline{P(x)}$\rq\rq.
		\item Phủ định của mệnh đề \lq\lq $ \exists x\in X, P(x)$\rq\rq\ là mệnh đề  \lq\lq $ \forall x\in X, \overline{P(x)}$\rq\rq.
	\end{itemize}
\end{note}

\end{tomtat}


\subsection{Các dạng toán}

\begin{dang}{Xác định mệnh đề và xét tính đúng - sai của mệnh đề}
\end{dang}

\subsubsection{Ví dụ minh hoạ}

\begin{vd}%[Thành Đức Trung]%[0D1Y1-1]
	Phát biểu nào sau đây là một mệnh đề toán học?
	\begin{enumerate}
		\item Hà Nội là Thủ đô của Việt Nam.
		\item Số $\pi$ là một số hữu tỉ.
		\item $x=1$ có phải là nghiệm của phương trình $x^2-1=0$ không?
		\item Phương trình $3x^2-5x+2=0$ có nghiệm nguyên.
		\item $5<7-3$.
		\item Đây là cách xử lí khôn ngoan!
	\end{enumerate}
	\loigiai
	{
		\begin{enumerate}
			\item Phát biểu \lq\lq  Hà Nội là Thủ đô của Việt Nam\rq\rq\ là mệnh đề nhưng không phải là mệnh đề toán học.
			\item Phát biểu \lq\lq  Số $\pi$ là một số hữu tỉ\rq\rq\ là một mệnh đề toán học.
			\item Phát biểu \lq\lq  $x=1$ có phải là nghiệm của phương trình $x^2-1=0$ không?\rq\rq\ là một câu hỏi nên không phải là một mệnh đề toán học.
			\item Phát biểu \lq\lq  Phương trình $3x^2-5x+2=0$ có nghiệm nguyên\rq\rq\ là một mệnh đề toán học.
			\item Phát biểu \lq\lq  $5<7-3$\rq\rq\ là một mệnh đề toán học.
			\item Phát biểu \lq\lq  Đây là cách xử lí khôn ngoan!\rq\rq\ là một câu cảm thán nên không phải là một mệnh đề toán học.
		\end{enumerate}
	}
\end{vd}

\begin{vd}%[Thành Đức Trung]%[0D1Y1-2]
	Trong các mệnh đề toán học sau đây, mệnh đề nào là một khẳng định đúng? Mệnh đề nào là một khẳng định sai?
	\begin{enumerate}
		\item $P\colon$\lq\lq  Tổng hai góc đối của một tứ giác nội tiếp bằng $180^{\circ}$\rq\rq.
		\item $Q\colon$\lq\lq  $7$ là số chính phương\rq\rq.
		\item $R\colon$\lq\lq  $1$ là số nguyên tố\rq\rq.
	\end{enumerate}
	\loigiai
	{
		Mệnh đề $P$ là mệnh đề đúng. \\
		Mệnh đề $Q$ và $R$ là mệnh đề sai. \\
	}
\end{vd}

% \begin{vd}%[Thành Đức Trung]%[0D1Y1-2]
% 	Thay dấu \lq\lq ?\rq\rq\ bằng dấu \lq\lq  x\rq\rq\ vào ô thích hợp trong bảng sau
% 	\begin{center}
% 		\begin{tabular}{|>{\centering\arraybackslash}m{5.5cm}|>{\centering\arraybackslash}m{2cm}|>{\centering\arraybackslash}m{2cm}|>{\centering\arraybackslash}m{2cm}|}
% 			\hline
% 			Câu & Không phải MĐ & MĐ đúng & MĐ sai \\
% 			\hline
% 		\end{tabular}
% 		\begin{tabular}{|>{\raggedright\arraybackslash}m{5.5cm}|>{\centering\arraybackslash}m{2cm}|>{\centering\arraybackslash}m{2cm}|>{\centering\arraybackslash}m{2cm}|}
% 			$13$ là số nguyên tố. & ? & ? & ? \\
% 			\hline
% 			Tổng độ dài hai cạnh bất kì của một tam giác nhỏ hơn độ dài cạnh còn lại. & ? & ? & ? \\
% 			\hline
% 			Bạn đã làm bài tập chưa? & ? & ? & ? \\
% 			\hline
% 			Thời tiết hôm nay thật đẹp! & ? & ? & ? \\
% 			\hline
% 			$9>2$. & ? & ? & ? \\
% 			\hline
% 			$27$ chia hết cho $5$. & ? & ? & ? \\
% 			\hline
% 			$2+3=6$. & ? & ? & ? \\
% 			\hline
% 			$36$ là số chính phương. & ? & ? & ? \\
% 			\hline
% 			Chó có khôn hơn lợn không? & ? & ? & ? \\
% 			\hline
% 		\end{tabular}
% 	\end{center}
% 	\loigiai
% 	{
% 		\begin{center}
% 			\begin{tabular}{|>{\centering\arraybackslash}m{5.5cm}|>{\centering\arraybackslash}m{2cm}|>{\centering\arraybackslash}m{2cm}|>{\centering\arraybackslash}m{2cm}|}
% 				\hline
% 				Câu & Không phải mệnh đề & Mệnh đề đúng & Mệnh đề sai \\
% 				\hline
% 			\end{tabular}
% 			\begin{tabular}{|>{\raggedright\arraybackslash}m{5.5cm}|>{\centering\arraybackslash}m{2cm}|>{\centering\arraybackslash}m{2cm}|>{\centering\arraybackslash}m{2cm}|}
% 				$13$ là số nguyên tố. &  & x &  \\
% 				\hline
% 				Tổng độ dài hai cạnh bất kì của một tam giác nhỏ hơn độ dài cạnh còn lại. &  &  & x \\
% 				\hline
% 				Bạn đã làm bài tập chưa? & x &  &  \\
% 				\hline
% 				Thời tiết hôm nay thật đẹp! & x &  &  \\
% 				\hline
% 				$9>2$. &  & x &  \\
% 				\hline
% 				$27$ chia hết cho $5$. &  &  & x \\
% 				\hline
% 				$2+3=6$. &  &  & x \\
% 				\hline
% 				$36$ là số chính phương. &  & x &  \\
% 				\hline
% 				Chó có khôn hơn lợn không? & x &  &  \\
% 				\hline
% 			\end{tabular}
% 		\end{center}
% 	}
% \end{vd}

\subsubsection{Bài tập tự luận}
\begin{bt}%[Thành Đức Trung]%[0D1Y1-1]
	Trong các phát biểu sau, phát biểu nào là mệnh đề toán học?
	\begin{enumerate}
		\item Tích hai số thực trái dấu là một số thực âm.
		\item Mọi số tự nhiên đều là số dương.
		\item Có sự sống ngoài Trái Đất.
		\item Ngày $1$ tháng $5$ là ngày Quốc tế Lao động.
	\end{enumerate}
	\loigiai
	{
		\begin{itemize}
			\item Phát biểu \lq\lq  Tích hai số thực trái dấu là một số thực âm\rq\rq\ là mệnh đề toán học.
			\item Phát biểu \lq\lq  Mọi số tự nhiên đều là số dương\rq\rq\ là mệnh đề toán học.
			\item Phát biểu \lq\lq  Có sự sống ngoài Trái Đất\rq\rq\ là mệnh đề nhưng không là mệnh đề toán học.
			\item Phát biểu \lq\lq  Ngày $1$ tháng $5$ là ngày Quốc tế Lao động\rq\rq\ là mệnh đề nhưng không là mệnh đề toán học.
		\end{itemize}
	}
\end{bt}
\begin{bt}%[Thành Đức Trung]%[0D1Y1-2]
	Xét tính đúng sai của mỗi mệnh đề sau
	\begin{listEX}[2]
		\item $\pi<\dfrac{10}{3}$.
		\item Phương trình $3x+7=0$ có nghiệm.
		\item Tồn tại số cộng với chính nó bằng $0$.
		\item $2022$ là hợp số.
	\end{listEX}
	\loigiai
	{
		\begin{enumerate}
			\item Mệnh đề \lq\lq  $\pi<\dfrac{10}{3}$\rq\rq\ là mệnh đề đúng.
			\item Mệnh đề \lq\lq  Phương trình $3x+7=0$ có nghiệm\rq\rq\ là mệnh đề đúng vì $3x+7=0 \Leftrightarrow x=-\dfrac{7}{3}$.
			\item Mệnh đề \lq\lq  Tồn tại số cộng với chính nó bằng $0$\rq\rq\ là mệnh đề đúng vì $0+0=0$.
			\item Mệnh đề \lq\lq  $2022$ là hợp số\rq\rq\ là mệnh đề đúng vì $2022$ có ít nhất $3$ ước là $1$; $2$ và $2022$.
		\end{enumerate}
	}
\end{bt}

\begin{bt}%[Thành Đức Trung]%[0D1Y1-2]
	Xét tính đúng sai của mỗi mệnh đề sau
	\begin{listEX}[2]
		\item $1993$ chia hết cho $3$.
		\item $\sqrt{12}$ là một số hữu tỉ.
		\item $9$ là một số chính phương.
		\item $|-1997|\leqslant0$.
	\end{listEX}
	\loigiai
	{
		\begin{enumerate}
			\item Mệnh đề \lq\lq  $1993$ chia hết cho $3$\rq\rq\ là mệnh đề sai vì $1993$ chia $3$ dư $1$.
			\item Mệnh đề \lq\lq  $\sqrt{12}$ là một số hữu tỉ\rq\rq\ là mệnh đề sai vì $\sqrt{12}$ là một số vô tỉ.
			\item Mệnh đề \lq\lq  $9$ là một số chính phương\rq\rq\ là mệnh đề đúng vì $\sqrt{9}=3$.
			\item Mệnh đề \lq\lq  $|-1997|\leqslant0$\rq\rq\ là mệnh đề sai vì $|-1997|=1997>0$.
		\end{enumerate}
	}
\end{bt}

\begin{bt}%[Thành Đức Trung]%[0D1Y1-2]
	Xét tính đúng sai của mỗi mệnh đề sau
	\begin{listEX}[3]
		\item $\sqrt{3}+\sqrt{2}=\dfrac{1}{\sqrt{3}-\sqrt{2}}$.
		\item $\left(\sqrt{2}-\sqrt{18}\right)^2\geqslant8$.
		\item $\left(\sqrt{3}+\sqrt{12}\right)^2$ là một số hữu tỉ.
		\item! $x=2$ là một nghiệm của phương trình $\dfrac{x^2-4}{x-2}=0$.
	\end{listEX}
	\loigiai
	{
		\begin{enumerate}
			\item Mệnh đề \lq\lq  $\sqrt{3}+\sqrt{2}=\dfrac{1}{\sqrt{3}-\sqrt{2}}$\rq\rq\ là mệnh đề đúng.
			\item Mệnh đề \lq\lq  $\left(\sqrt{2}-\sqrt{18}\right)^2\geqslant8$\rq\rq\ là mệnh đề đúng vì $\left(\sqrt{2}-\sqrt{18}\right)^2=8$.
			\item Mệnh đề \lq\lq  $\left(\sqrt{3}+\sqrt{12}\right)^2$ là một số hữu tỉ\rq\rq\ là mệnh đề đúng vì $\left(\sqrt{3}+\sqrt{12}\right)^2=27$.
			\item Mệnh đề \lq\lq  $x=2$ là một nghiệm của phương trình $\dfrac{x^2-4}{x-2}=0$\rq\rq\ là mệnh đề sai vì $x=2$ vi phạm điều kiện xác định của phương trình.
		\end{enumerate}
	}
\end{bt}

\begin{bt}%[Thành Đức Trung]%[0D1Y1-2]
	Thay dấu \lq\lq ?\rq\rq\ bằng dấu \lq\lq  x\rq\rq\ vào ô thích hợp trong bảng sau
	\begin{center}
		\begin{tabular}{|>{\centering\arraybackslash}m{5.5cm}|>{\centering\arraybackslash}m{2cm}|>{\centering\arraybackslash}m{2cm}|>{\centering\arraybackslash}m{2cm}|}
			\hline
			Câu & Không phải mệnh đề & Mệnh đề đúng & Mệnh đề sai \\
			\hline
		\end{tabular}
		\begin{tabular}{|>{\raggedright\arraybackslash}m{5.5cm}|>{\centering\arraybackslash}m{2cm}|>{\centering\arraybackslash}m{2cm}|>{\centering\arraybackslash}m{2cm}|}
			Hãy đi nhanh lên! & ? & ? & ? \\
			\hline
			$5+7+4=15$. & ? & ? & ? \\
			\hline
			Phương trình $x^2-3x+2=0$ có nghiệm. & ? & ? & ? \\
			\hline
			$2^{10}-1$ chia hết cho $11$. & ? & ? & ? \\
			\hline
			Có vô số số nguyên tố. & ? & ? & ? \\
			\hline
			Bây giờ là mấy giờ? & ? & ? & ? \\
			\hline
			$\sqrt{5}$ là số vô tỉ. & ? & ? & ? \\
			\hline
		\end{tabular}
	\end{center}
	\loigiai
	{
		\begin{center}
			\begin{tabular}{|>{\centering\arraybackslash}m{5.5cm}|>{\centering\arraybackslash}m{2cm}|>{\centering\arraybackslash}m{2cm}|>{\centering\arraybackslash}m{2cm}|}
				\hline
				Câu & Không phải mệnh đề & Mệnh đề đúng & Mệnh đề sai \\
				\hline
			\end{tabular}
			\begin{tabular}{|>{\raggedright\arraybackslash}m{5.5cm}|>{\centering\arraybackslash}m{2cm}|>{\centering\arraybackslash}m{2cm}|>{\centering\arraybackslash}m{2cm}|}
				Hãy đi nhanh lên! & x &  &  \\
				\hline
				$5+7+4=15$. &  &  & x \\
				\hline
				Phương trình $x^2-3x+2=0$ có nghiệm. &  & x &  \\
				\hline
				$2^{10}-1$ chia hết cho $11$. &  & x &  \\
				\hline
				Có vô số số nguyên tố. &  & x &  \\
				\hline
				Bây giờ là mấy giờ? & x &  &  \\
				\hline
				$\sqrt{5}$ là số vô tỉ. &  & x &  \\
				\hline
			\end{tabular}
		\end{center}
	}
\end{bt}

\begin{dang}{Mệnh đề phủ định, mệnh đề đảo, mệnh đề kéo theo, tương đương}
\end{dang}

\subsubsection{Ví dụ minh hoạ}

\begin{vd}
	Phát biểu mệnh đề phủ định của các mệnh đề sau và cho biết tính đúng sai của mệnh đề phủ định đó.
	\begin{enumerate}
		\item $P\colon$\lq\lq  $\sqrt{5}$ là số hữu tỉ\rq\rq.
		\item $Q\colon $\lq\lq  Tổng ba góc trong một tam giác bằng $180^\circ$\rq\rq.
		\item $R\colon$\lq\lq  $25$ là một số chính phương\rq\rq.
		\item $T\colon $\lq\lq  Hình vuông không phải là hình bình hành\rq\rq.
	\end{enumerate}
	\loigiai{
		\begin{enumerate}
			\item Mệnh đề phủ định của mệnh đề $P$ là $\overline{P}\colon$\lq\lq  $\sqrt{5}$ không phải là số hữu tỉ\rq\rq.\\
			Đây là một mệnh đề đúng vì $\sqrt{5}$ không thể biểu diễn dưới dạng $\dfrac{a}{b}$ với $a$, $b\in \mathbb{Z}$.
			\item Mệnh đề phủ định của mệnh đề $Q$ là $\overline{Q}\colon $\lq\lq  Tổng ba góc trong tam giác không bằng $180^\circ$.\\
			Đây là một mệnh đề sai.
			\item Mệnh đề phủ định của mệnh đề $R$ là $\overrightarrow{R}\colon $\lq\lq  $25$ không phải là một số chính phương\rq\rq.\\
			Đây là một mệnh đề sai.
			\item Mệnh đề phủ định của mệnh đề $T$ là $\overline{T}\colon$\lq\lq  Hình vuông là hình bình hành\rq\rq.\\
			Đây là một mệnh đề đúng.
		\end{enumerate}
	}
\end{vd}

\begin{vd}
	Cho tam giác $ABC$. Xét hai mệnh đề $P\colon $\lq\lq  tam giác $ABC$ vuông\rq\rq\text{} và $Q\colon $\lq\lq  $AB^2+AC^2=BC^2$\rq\rq. Phát biểu và cho biết mệnh đề sau đúng hay sai.
	\begin{enumEX}{2}
		\item $P\Rightarrow Q$.
		\item $Q\Rightarrow P$.
	\end{enumEX}
	\loigiai{
		\begin{enumerate}
			\item Mệnh đề $P\Rightarrow Q$ là \lq\lq  Nếu tam giác $ABC$ vuông thì $AB^2+AC^2=BC^2$.\\
			Mệnh đề $P\Rightarrow Q$ sai vì chưa chắc tam giác $ABC$ đã vuông tại $A$.
			\item Mệnh đề $Q\Rightarrow P$ là \lq\lq  Nếu tam giác $ABC$ có $AB^2+AC^2=BC^2$ thì tam giác vuông\rq\rq.\\
			Mệnh đề $Q\Rightarrow P$ đúng (theo định lí Py-ta-go).
		\end{enumerate}
	}
\end{vd}

\begin{vd}
	Cho $\triangle ABC$ có hai đường trung tuyến $BM$, $CN$. Lập mệnh đề $P\Rightarrow Q$ và mệnh đề đảo của nó, rồi xét tính đúng sai của chúng khi
	\begin{enumerate}
		\item $P\colon$\lq\lq  Góc $A$ tù\rq\rq\text{} và $Q\colon $\lq\lq  Cạnh $BC$ lớn nhất\rq\rq.
		\item $P\colon$\lq\lq  $BM=CN$\rq\rq\text{} và $Q\colon $\lq\lq  tam giác $ABC$ cân\rq\rq.
	\end{enumerate}
	\loigiai{
		\begin{enumerate}
			\item $P\colon$\lq\lq  Góc $A$ tù\rq\rq\text{} và $Q\colon $\lq\lq  Cạnh $BC$ lớn nhất\rq\rq.
			\begin{itemize}
				\item Mệnh đề $P\Rightarrow Q$ là \lq\lq  Nếu góc $A$ tù thì cạnh $BC$ lớn nhất\rq\rq. Đây là mệnh đề đúng.
				\item Mệnh đề $Q\Rightarrow P$ là \lq\lq  Nếu cạnh $BC$ lớn nhất thì $A$ là góc tù\rq\rq. Đây là mệnh đề sai ($A$ vẫn có thể là góc nhọn hoặc góc vuông).
			\end{itemize}
			\item $P\colon$\lq\lq  $BM=CN$\rq\rq\text{} và $Q\colon $\lq\lq  tam giác $ABC$ cân\rq\rq.
			\begin{itemize}
				\item Mệnh đề $P\Rightarrow
				Q$ là \lq\lq  Nếu $BM=CN$ thì tam giác $ABC$ cân\rq\rq. Đây là một mệnh đề đúng.
				\item Mệnh đề $Q\Rightarrow P$ là \lq\lq  Nếu tam giác $ABC$ cân thì $BM=CN$\rq\rq. Đây là một mệnh đề sai vì chưa chắc tam giác $ABC$ đã cân tại $A$.
			\end{itemize}
		\end{enumerate}
	}
\end{vd}

\begin{vd}
	Cho định lí \lq\lq  Nếu $MA\perp MB$ thì $M$ thuộc đường tròn đường kính $AB$\rq\rq. Hãy xác định giả thiết của định lí, kết luận của định lí và dùng thuật ngữ \lq\lq  điều kiện cần\rq\rq, \lq\lq  điều kiện đủ\rq\rq\text{} để phát biểu lại định lí.
	\loigiai{
		Giả thiết của định lí là $MA\perp MB$.\\
		Kết luật của định lí là $M$ thuộc đường  tròn đường kính $AB$.
		\begin{itemize}
			\item Điều kiện cần để $MA\perp MB$ là $M$  thuộc đường tròn đường kính $AB$.
			\item Điều kiện đủ để $M$ thuộc đường tròn đường kính $AB$ là $MA\perp MB$.
		\end{itemize}
	}
\end{vd}

\begin{vd}
	Phát biểu mệnh đề $P\Leftrightarrow Q$ và cho biết tính đúng sai của nó.
	\begin{enumerate}
		\item $P\colon $\lq\lq  Tứ giác $ABCD$ là hình vuông\rq\rq \text{ và }$Q\colon$ \lq\lq  Tứ giác $ABCD$ là hình thoi có $AC=BD$\rq\rq.
		\item $P\colon $\lq\lq  Điểm $M$ nằm trên phân giác của góc $xOy$\rq\rq \text{} và $Q\colon $\lq\lq  Điểm $M$ cách đều hai cạnh $Ox$, $Oy$\rq\rq.
		\item $P\colon$\lq\lq  Tam giác $ABC$ đều\rq\rq \text{} và $Q\colon$\lq\lq  Tam giác $ABC$ có ba đường cao bằng nhau\rq\rq.
	\end{enumerate}
	\loigiai{
		\begin{enumerate}
			\item Mệnh đề tương đương $P\Leftrightarrow Q$ là \lq\lq  Tứ giác $ABCD$ là hình vuông khi và chỉ khi tứ giác $ABCD$ là hình thoi có $AC=BD$\rq\rq.\\
			Mệnh đề $P\Leftrightarrow Q$ đúng vì mệnh đề $P\Rightarrow Q$ và mệnh đề $Q\Rightarrow P$ là hai mệnh đề đúng.
			\item Mệnh đề tương đương $P\Leftrightarrow Q$ là \lq\lq  Điểm $M$ nằm trên phân giác của góc $xOy$ khi và chỉ khi điểm $M$ cách đều hai cạnh $Ox$, $Oy$\rq\rq.\\
			Mệnh đề $P\Leftrightarrow Q$ đúng vì mệnh đề $P\Rightarrow Q$ và $Q\Rightarrow P$ là hai mệnh đề đúng.
			\item Mệnh đề tương đương $P\Leftrightarrow Q$ là \lq\lq  Tam giác $ABC$ đều khi và chỉ khi ba đường cao bằng nhau\rq\rq.\\
			Mệnh đề $P\Leftrightarrow Q$ đúng vì hai mệnh đề $P\Rightarrow Q$ và $Q\Rightarrow P$ là hai mệnh đề đúng. 
		\end{enumerate}
	}
\end{vd}

\subsubsection{Bài tập tự luận}

\begin{bt}
	Phát biểu mệnh đề phủ định của các mệnh đề sau
	\begin{enumerate}
		\item $A\colon$\lq\lq  $2022$ chia hết cho $7$\rq\rq.
		\item $B\colon$\lq\lq  Tích của ba số tự nhiên liên tiếp chia hết cho $6$\rq\rq.
		\item $C\colon $\lq\lq  Phương trình $x^2+x+1=0$ vô nghiệm\rq\rq.
	\end{enumerate}
	\loigiai{
		\begin{enumerate}
			\item Mệnh đề phủ định của mệnh đề $A$ là $\overline{A}\colon$\lq\lq  $2022$ không chia hết cho $7$\rq\rq.
			\item Mệnh đề phủ định của mệnh đề $B$ là $\overline{B}\colon$\lq\lq  Tích của ba số tự nhiên liên tiếp không chia hết cho $6$\rq\rq.
			\item Mệnh đề phủ định của mệnh đề $C$ là $\overline{C}\colon$\lq\lq  Phương trình $x^2-x+1=0$ có nghiệm\rq\rq.
		\end{enumerate}
	}
\end{bt}

\begin{bt}
	Hãy lập mệnh đề phủ định của các mệnh đề sau đây và cho biết các mệnh đề phủ định đó đúng hay sai?
	\begin{enumerate}
		\item $A\colon$\lq\lq  $735$ là số nguyên tố\rq\rq.
		\item $B\colon$\lq\lq  Phương trình $x^2+9x-2011=0$ vô nghiệm\rq\rq.
		\item $C\colon$\lq\lq  Đường tròn có một tâm đối xứng\rq\rq.
		\item $D\colon$\lq\lq  Hai đường thẳng song song không có điểm chung\rq\rq.
	\end{enumerate}
	\loigiai{
		\begin{enumerate}
			\item Phủ định của mệnh đề $A$ là $\overline{A}\colon$\lq\lq  Số $735$ không phải là số nguyên tố\rq\rq. Đây là mệnh đề đúng vì $735\,\vdots\,5$.
			\item Phủ định của mệnh đề $B$ là $\overline{B}\colon$\lq\lq  Phương trình $x^2+9x-2022=0$ có nghiệm\rq\rq. Đây là mệnh đề đúng vì $a=1$ và $c=-2022$ trái dấu.
			\item Phủ định của mệnh đề $C$ là $\overline{C}\colon$\lq\lq  Không phải đường tròn có một tâm đối xứng\rq\rq. Đây là một mệnh đề sai.
			\item Phủ định của mệnh đề $D$ là $\overline{D}\colon$\lq\lq  Hai đường thẳng song song có điểm chung\rq\rq. Đây là mệnh đề sai.
		\end{enumerate}
	}
\end{bt}

\begin{bt}
	Phát biểu mệnh đề đảo của mệnh đề sau và xét tính đúng sai của mệnh đề đảo.
	\begin{enumerate}
		\item Nếu một số chia hết cho $6$ thì số đó chia hết cho $3$.
		\item Nếu một số là số tự nhiên lẻ thì nó là số nguyên tố.
		\item Nếu $\dfrac{AB}{MN}=\dfrac{AC}{MP}$ thì $\triangle ABC\backsim \triangle MNP$.
	\end{enumerate}
	\loigiai{
		\begin{enumerate}
			\item Nếu một số chia hết cho $3$ thì số đó chia hết cho $6$. Đây là mệnh đề sai.
			\item Nếu một số là số nguyên tố thì nó là số lẻ. Đây là mệnh đề sai vì $2$ là số nguyên tố chẵn.
			\item Nếu $\triangle ABC\backsim \triangle MNP$ thì $\dfrac{AB}{MN}=\dfrac{AC}{MP}$. Đây là mệnh đề đúng.
		\end{enumerate}
	}
\end{bt}

\begin{bt}
	Phát biểu mệnh đề đảo của mệnh đề sau và cho biết tính đúng sai của mệnh đề đảo.
	\begin{enumerate}
		\item Nếu hai tam giác bằng nhau thì chúng có diện tích bằng nhau.
		\item Nếu tứ giác $ABCD$ là hình bình hành thì nó có hai cạnh đối song song và bằng nhau.
	\end{enumerate}
	\loigiai{
		\begin{enumerate}
			\item Nếu hai tam giác có diện tích bằng nhau thì nó bằng nhau.\\
			Đây là một mệnh đề sai.
			\item Nếu tứ giác $ABCD$ có hai cạnh đối song song và bằng nhau thì nó là hình bình hành.\\
			Đây là mệnh đề đúng. 
		\end{enumerate}
	}
\end{bt}

\begin{bt}
	Hãy xác định giả thiết, kết luận đồng thời dùng thuật ngữ \lq\lq  điều kiện đủ\rq\rq,\text{} để phát biểu các định lí sau
	\begin{enumerate}
		\item Nếu $a$ và $b$ là hai số hữu tỉ thì tổng $a+b$ cũng là số hữu tỉ.
		\item Nếu một số tự nhiên $n$ có tổng các chữ số chia hết cho $9$ thì nó chia hết cho $9$.
	\end{enumerate}
	\loigiai{
		\begin{enumerate}
			\item Giả thiết của định lí là \lq\lq  $a$ và $b$ là hai số hữu tỉ\rq\rq.\\
			Kết luận của định lí là \lq\lq  tổng $a+b$ là số hữu tỉ\rq\rq.\\
			Phát biểu định lí dưới dạng điều kiện đủ \lq\lq  Điều kiện đủ để tổng $a+b$ là số hữu tỉ là cả hai số $a$ và $b$ đều là số hữu tỉ\rq\rq.
			\item Giả thiết của định lí là \lq\lq  Một số tự nhiên $n$ có tổng các chữ số chia hết cho $9$\rq\rq.\\
			Kết luận của định lí là \lq\lq  $n$ chia hết cho $9$\rq\rq.\\
			Phát biểu định lí dưới dạng điều kiện đủ \lq\lq  Điều kiện đủ để $n$ chia hết cho $9$ là tổng các chữ số của $n$ chia hết cho $9$\rq\rq.
		\end{enumerate}
	}
\end{bt}

\begin{bt}
	Cho định lí \lq\lq  Cho số tự nhiên $n$, nếu $n^5$ chia hết cho $5$ thì $n$ chia hết cho $5$\rq\rq. Định lí này được viết dưới dạng $P\Rightarrow Q$.
	\begin{enumerate}
		\item Hãy xác định các mệnh đề $P$ và $Q$.
		\item Phát biểu định lí trên bằng cách dùng thuật ngữ \lq\lq  điều kiện cần\rq\rq.
		\item Phát biểu định lí trên bằng cách dùng thuật ngữ \lq\lq  điều kiện đủ\rq\rq.
		Hãy phát biểu định lí đảo (nếu có) của định lí trên rồi dùng các thuật ngữ \lq\lq  điều kiện cần và điều kiện đủ\rq\rq\text{} phát biểu gộp cả hai định lí thuận và đảo.
	\end{enumerate}
	\loigiai{
		\begin{enumerate}
			\item $P\colon $\lq\lq  $n$ là số tự nhiên và $n^5$ chia hết cho $5$\rq\rq, $Q\colon $\lq\lq  $n$ chia hết cho $5$\rq\rq.
			\item Với $n$ là số tự nhiên, $n$ chia hết cho $5$ là điều kiện cần để $n^5$ chia hết cho $5$.
			\item Với $n$ là số tự nhiên, $n^5$ chia hết cho $5$ là điều kiện đủ để $n$ chia hết cho $5$.
			\item 
			\begin{itemize}
				\item Định lí đảo \lq\lq  Cho số tự nhiên $n$, nếu $n$ chia hết cho $5$ thì $n^5$ chia hết cho $5$\rq\rq.\\
				\item Phát biểu gộp cả hai định lí \lq\lq  Điều kiện cần và đủ để $n$ chia hết cho $5$ là $n^5$ chia hết cho $5$\rq\rq.
			\end{itemize} 
		\end{enumerate}
	}
\end{bt}

\begin{bt}
	Cho tam giác ABC với trung tuyến $AM$. Xét hai mệnh đề\\
	$P\colon $\lq\lq  Tam giác $ABC$ vuông tại $A$\rq\rq.
	$Q\colon $\lq\lq  Trung tuyến $AM$ bằng một nửa cạnh $BC$\rq\rq
	\begin{enumerate}
		\item Hãy phát biểu mệnh đề $P\Rightarrow Q$. Mệnh đề này đúng hay sai?
		\item Hãy phát biểu mệnh đề $Q\Rightarrow P$. Mệnh đề này đúng hay sai?
		\item Phát biểu mệnh đề $P\Leftrightarrow Q$ và cho biết mệnh đề đó đúng hay sai?
	\end{enumerate}
	\loigiai{
		\begin{enumerate}
			\item Mệnh đề $P\Rightarrow Q$ là \lq\lq  Nếu tam giác $ABC$ vuông tại $A$ thì trung tuyến $AM$ bằng một nửa cạnh $BC$\rq\rq.\\
			Đây là mệnh đề đúng.
			\item Mệnh đề $Q\Rightarrow P$ là \lq\lq  Nếu trung tuyến $AM$ bằng một nửa cạnh $BC$ thì tam giác $ABC$ vuông tại $A$\rq\rq.\\
			Đây là mệnh đề đúng.
			\item Mệnh đề $P\Leftrightarrow Q$ là \lq\lq  Tam giác $ABC$ vuông tại $A$ khi và chỉ khi trung tuyến $AM$ bằng một nửa cạnh $BC$\rq\rq.\\
			Mệnh đề tương đương $P\Leftrightarrow Q$ đúng vì $P\Rightarrow Q$ và $Q\Rightarrow P$ là hai mệnh đề đúng.
		\end{enumerate}
	}
\end{bt}

\begin{bt}
	Phát biểu mệnh đề $P\Rightarrow Q$ và phát biểu mệnh đề đảo, xét tính đúng sai của nó.
	\begin{enumerate}
		\item $P\colon $\lq\lq  Tứ giác $ABCD$ là hình chữ nhật\rq\rq \text{} và $Q\colon $\lq\lq  Tứ giác $ABCD$ có $AC$ và $BD$ cắt nhau tại trung điểm của mỗi đường\rq\rq.
		\item $P\colon$\lq\lq  Hình thang $ABCD$ nội tiếp một đường tròn \rq\rq \text{} và $Q\colon$\lq\lq  Hình thang $ABCD$ cân\rq\rq.
	\end{enumerate}
	\loigiai{
		\begin{enumerate}
			\item Mệnh đề đảo của mệnh đề $P\Rightarrow Q$ là $Q\Rightarrow P\colon $\lq\lq  Nếu tứ giác $ABCD$ có $AC$ và $BD$ cắt nhau tại trung điểm của mỗi đường thì nó là hình chữ nhật\rq\rq.\\
			Đây là một mệnh đề sai vì tứ giác có hai đường chéo cắt nhau tại trung điểm của mỗi đường thì nó chỉ là hình bình hành, chưa đủ điều kiện để là hình chữ nhật.
			\item Mệnh đề đảo của mệnh đề $P\Rightarrow Q$ là $Q\Rightarrow P\colon $\lq\lq  Nếu $ABCD$ là hình thang cân thì $ABCD$ nội tiếp một đường tròn\rq\rq.\\
			Đây là một mệnh đề đúng vì hình thang cân có tổng hai góc đối bằng $180^\circ$.
		\end{enumerate}
	}
\end{bt}

\begin{bt}
	Hãy phát biểu mệnh đề $P\Leftrightarrow Q$ và cho biết mệnh đề đó đúng hay sai nếu biết
	\begin{enumerate}
		\item $P\colon $\lq\lq  $a$ và $b$ cùng chia hết cho $c$\rq\rq\text{} và $Q\colon $\lq\lq  $a+b$ chia hết cho $c$\rq\rq.
		\item $P\colon $\lq\lq  $a$ chia hết cho $3$\rq\rq\text{} và $Q\colon $\lq\lq  $a$ chia hết cho $9$\rq\rq.
		\item $P\colon $\lq\lq  $ABCD$ là hình chữ nhật\rq\rq\text{} và $Q\colon $\lq\lq  Tứ giác $ABCD$ có ba góc vuông\rq\rq.
	\end{enumerate}
	\loigiai{
		\begin{enumerate}
			\item Mệnh đề $P\Leftrightarrow Q\colon $\lq\lq  $a$ và $b$ cùng chia hết cho $c$ nếu và chỉ nếu $a+b$ chia hết cho $c$\rq\rq.\\
			Đây là mệnh đề sai vì mệnh đề $P\Rightarrow Q$ đúng nhưng mệnh đề $Q\Rightarrow P$ là sai.
			\item Mệnh đề $P\Leftrightarrow Q\colon $\lq\lq  $a$ chia hết cho $3$ nếu và chỉ nếu $a$ chia hết cho $9$\rq\rq.\\
			Đây là mệnh đề sai vì mệnh đề $P\Rightarrow Q$ là mệnh đề đúng còn mệnh đề $Q\Rightarrow P$ là mệnh đề sai.
			\item Mệnh đề $P\Leftrightarrow Q\colon $\lq\lq  $ABCD$ là hình chữ nhật khi và chỉ khi nó có ba góc vuông\rq\rq.\\
			Đây là một mệnh đề đúng vì mệnh đề $P\Rightarrow Q$ và $Q\Rightarrow P$ là hai mệnh đề đúng.
		\end{enumerate}
	}
\end{bt}

\begin{dang}{Mệnh đề chứa biến- mệnh đề chứa kí hiệu $\forall$ và $\exists$}
	% Kí hiệu $\forall$ đọc là \lq \lq với mọi\rq \rq.\\
	% Kí hiệu $\exists$ đọc là \lq \lq có một\rq \rq \,(tồn tại một) hay \lq \lq có ít nhất một\rq \rq\,(tồn tại ít nhất một).\\
	% Mối quan hệ giữa $\exists$ và $\forall$.\\
	% Cho mệnh đề \lq \lq $P(x),\, x \in X$\rq \rq.\\
	% Phủ định của mệnh đề \lq \lq $ \forall x \in X,\;P(x)$\rq \rq \;là mệnh đề \lq \lq $\exists x \in X,\;\overline{P(x)}$\rq \rq.\\
	% Phủ định của mệnh đề \lq \lq $ \exists x \in X,\;P(x)$\rq \rq \;là mệnh đề \lq \lq $ \forall x \in X,\;\overline{P(x)}$\rq \rq.
\end{dang}

\subsubsection{Ví dụ minh hoạ}

\begin{vd}%[Nguyễn Cường- BG Toán 10]%[0D1Y1-1]
	Xét câu \lq \lq $n$ là số chẵn\rq \rq. (với $n$ là số nguyên) \\
	Ta chưa khẳng định được tính đúng sai của câu này. Tuy nhiên, với mỗi giá trị của $n$ thuộc tập số nguyên, câu này cho ta một mệnh đề.
	Chẳng hạn,
	\begin{itemize}
		\item Với $n=1$ ta được mệnh đề \lq \lq $1$ là số chẵn\rq \rq\, (đây là mệnh đề sai).
		\item Với $n=2$ ta được mệnh đề \lq \lq $2$ là số chẵn\rq \rq\, (đây là mệnh đề đúng).
	\end{itemize}
	Ta nói rằng câu \lq \lq $n$ là số chẵn\rq \rq\, là một mệnh đề chứa biến.	
\end{vd}
%%==========Ví dụ 2
\begin{vd}%[Nguyễn Cường- BG Toán 10]%[0D1Y1-2]
	Xét câu \lq\lq $x>1$\rq\rq. Hãy tìm hai giá trị thực của $x$, ta nhận được một mệnh đề đúng và một mệnh đề sai.
	\loigiai{\begin{enumerate}
			\item Cho $x=2$ ta được mệnh đề đúng.
			\item Cho $x=0$ ta được mệnh đề sai.
		\end{enumerate}
	}
\end{vd}
%%==========Ví dụ 3
\begin{vd}%[Nguyễn Cường- BG Toán 10]%[0D1Y1-1]
	Trong các câu sau, câu nào là mệnh đề chứa biến?
	\begin{enumerate}
		\item $18$ chia hết cho $9$;
		\item $3n$ chia hết cho $9$.
	\end{enumerate}
	\loigiai{
		\begin{enumerate}
			\item Câu \lq\lq $18$ chia hết cho $9$\rq\rq\,là mệnh đề nhưng không phải là mệnh đề chứa biến.
			\item Câu \lq\lq $3n$ chia hết cho $9$\rq\rq\,là mệnh đề chứa biến, kí hiệu là $P(n)\colon$\lq\lq $3n$ chia hết cho $9$\rq\rq.
		\end{enumerate}
	}
\end{vd}
%%==========Ví dụ 4
\begin{vd}%[Nguyễn Cường- BG Toán 10]%[0D1Y1-5]
	Cho mệnh đề $P\colon$\lq\lq  $\forall x \in \mathbb{N}: x-2>0$\rq\rq. Tìm mệnh đề phủ định của mệnh đề $P$. Xét tính đúng sai của mệnh đề $\overline{P}$.
	\loigiai{
		Ta có $\overline{P}\colon$\lq\lq  $\exists x \in \mathbb{N}: x-2\leq 0$\rq\rq.\\
		Đây là mệnh đề đúng, vì với $x=0$ thì $x-2=-2<0$.
	}
\end{vd}
%%==========Ví dụ 5
\begin{vd}%[Nguyễn Cường- BG Toán 10]%[0D1Y1-5]
	Viết mệnh đề phủ định của mệnh đề sau và xác định tính đúng sai của nó.\break
	$P\colon$ \lq\lq $\exists x\in\mathbb{R}, x^2+1=0$\rq\rq.
	\loigiai{
		Mệnh đề $P$ có thể phát biểu là \lq\lq  Tồn tại một số thực mà bình phương của nó cộng với $1$ bằng $0$\rq\rq.\\
		Phủ định của mệnh đề $P$ là \lq\lq  Không tồn tại một số thực mà bình phương của nó cộng với $1$ bằng $0$\rq\rq.\\
		Tức là \lq\lq  Mọi số thực mà bình phương của nó cộng với $1$ khác $0$\rq\rq.\\
		Ta có thể viết mệnh đề phủ định của $P$ là $\overline{P}\colon$\lq\lq $\forall x\in\mathbb{R}, x^2+1\ne 0$\rq\rq. Mệnh đề phủ định này đúng.
	}
\end{vd}

\subsubsection{Bài tập tự luận}

%%==========Bài 1
% \begin{bt}%[Nguyễn Cường- BG Toán 10]%[0D1Y1-2]
% 	Cho câu \lq\lq $x>5$\rq\rq. Hãy tìm hai giá trị thực của $x$ để từ câu đã cho, ta nhận được một mệnh đề đúng và một mệnh đề sai.
% 	\loigiai{
% 		\begin{enumerate}
% 			\item Cho $x=7$ ta được mệnh đề đúng.
% 			\item Cho $x=5$ ta được mệnh đề sai.
% 		\end{enumerate}
% 	}
% \end{bt}
%%==========Bài 2
\begin{bt}%[Nguyễn Cường- BG Toán 10]%[0D1Y1-5]
	Sử dụng kí hiệu \lq\lq $\forall$\rq\rq \,để viết mỗi mệnh đề sau và xét xem mệnh đề đó là đúng hay sai, giải thích vì sao.
	\begin{enumerate}
		\item $P\colon$\lq\lq  Với mọi số thực $x, x^2+1>0$\rq\rq.
		\item $Q\colon$\lq\lq  Với mọi số tự nhiên $n, n^2+n$ chia hết cho $6$\rq\rq.
	\end{enumerate}
	\loigiai{
		\begin{enumerate}
			\item $P\colon$\lq\lq  Với mọi số thực $x, x^2+1>0$\rq\rq.\\
			Mệnh đề được viết là $P \colon \lq\lq \forall x \in \mathbb{R}, x^2+1>0$\rq\rq.\\
			Xét một số thực $x$ tùy ý, ta phải chứng tỏ rằng $x^2+1>0$.\\
			Thật vậy, ta có $x^2+1 \geq 1>0$.\\
			Vậy mệnh đề $P$ là mệnh đề đúng.
			\item $Q\colon$\lq\lq  Với mọi số tự nhiên $n, n^2+n$ chia hết cho $6$\rq\rq.\\
			Mệnh đề được viết là $Q\colon\lq\lq  \forall n \in \mathbb{N},\left(n^2+n\right) \,\vdots\, 6$\rq\rq.\\
			Để chứng minh mệnh đề $Q$ là sai, ta cần chỉ ra một giá trị cụ thể của $n$ để nhận được mệnh đề sai.\\
			Thật vậy, chọn $n=1$, ta thấy $n^2+n=2$ không chia hết cho $6$.\\
			Vậy mệnh đề $Q$ là mệnh đề sai.
		\end{enumerate}
	}
\end{bt}
%%==========Bài 3
\begin{bt}%[Nguyễn Cường- BG Toán 10]%[0D1Y1-5]
	Sử dụng kí hiệu \lq\lq $\exists$\rq\rq\, để viết mỗi mệnh đề sau và xét xem mệnh đề đó là đúng hay sai, giải thích vì sao.
	\begin{enumerate}
		\item $M\colon$\lq\lq  Tồn tại số thực $x$ sao cho $x^3=-8$\rq\rq.
		\item $N\colon$\lq\lq  Tồn tại số nguyên $x$ sao cho $2x+1=0$\rq\rq.
	\end{enumerate}
	\loigiai{
		\begin{enumerate}
			\item $M\colon$\lq\lq  Tồn tại số thực $x$ sao cho $x^3=-8$\rq\rq.\\
			Mệnh đề được viết là $M\colon\lq\lq \exists x \in \mathbb{R}, x^3=-8$\rq\rq.
			Để chứng tỏ mệnh đề $M$ là đúng, ta cần chỉ ra một giá trị cụ thể của $x$ để nhận được mệnh đề đúng.\\
			Thật vậy, chọn $x=-2$, ta thấy $(-2)^3=-8$.\\
			Vậy mệnh đề $M$ là mệnh đề đúng.\\
			Mệnh đề $N\colon\lq\lq \exists x \in \mathbb{Z}, 2x+1=0$\rq\rq.
			\item $N\colon$\lq\lq  Tồn tại số nguyên $x$ sao cho $2x+1=0$\rq\rq.\\
			Để chứng minh mệnh đề $N$ là sai, ta phải chứng tỏ rằng với số nguyên $x$ tùy ý thì $2x+1 \neq 0$.\\
			Thật vậy, xét một số nguyên $x$ tùy ý, ta có $2x+1 \neq 0$.\\
			Vì thế mệnh đề $N$ là mệnh đề sai.
		\end{enumerate}
	}
\end{bt}
%%==========Bài 4
\begin{bt}%[Nguyễn Cường- BG Toán 10]%[0D1B1-5]
	Bạn An nói \lq\lq  Mọi số thực đều có bình phương là một số không âm\rq\rq.
	Bạn Bình phủ định lại câu nói của bạn An \lq\lq  Có một số thực mà bình phương của nó là một số âm\rq\rq.
	\begin{enumerate}
		\item Sử dụng kí hiệu \lq\lq $\forall$\rq\rq\,để viết mệnh đề của bạn An.
		\item Sử dụng kí hiệu \lq\lq $\exists$\rq\rq\,để viết mệnh đề của bạn Bình.
	\end{enumerate}
	\loigiai{
		\begin{enumerate}
			\item \lq\lq $\forall x\in\mathbb{R}, x^2\ge 0$\rq\rq.
			\item \lq\lq $\exists x\in\mathbb{R}, x^2< 0$\rq\rq.
		\end{enumerate}	
	}
\end{bt}
%%==========Bài 5
\begin{bt}%[Nguyễn Cường- BG Toán 10]%[0D1B1-5]
	Lập mệnh đề phủ định của mỗi mệnh đề sau
	\begin{enumerate}
		\item $\forall x \in \mathbb{R},|x| \geq x$.
		\item $\exists x \in \mathbb{R}, x^2+1=0$.
	\end{enumerate}
	\loigiai{
		\begin{enumerate}
			\item Phủ định của mệnh đề \lq\lq $\forall x \in \mathbb{R},|x| \geq x$\rq\rq\,là mệnh đề \lq\lq $\exists x \in \mathbb{R},|x|<x$\rq\rq.
			\item Phủ định của mệnh đề \lq\lq $\exists x \in \mathbb{R}, x^2+1=0$\rq\rq\,là mệnh đề \lq\lq $\forall x \in \mathbb{R}, x^2+1 \neq 0$\rq\rq.
		\end{enumerate}
	}
\end{bt}
%%==========Bài 6
% \begin{bt}%[Nguyễn Cường- BG Toán 10]%[0D1B1-5]
% 	Phát biểu mệnh đề phủ định của mỗi mệnh đề sau
% 	\begin{enumerate}
% 		\item Tồn tại số nguyên chia hết cho $3$.
% 		\item Mọi số thập phân đều viết được dưới dạng phân số.
% 	\end{enumerate}
% 	\loigiai{
% 		\begin{enumerate}
% 			\item Mọi số nguyên đều không chia hết cho $3$.
% 			\item Tồn tại số thập phân không viết được dưới dạng phân số.
% 		\end{enumerate}	
% 	}
% \end{bt}
%%==========Bài 7
% \begin{bt}%[Nguyễn Cường- BG Toán 10]%[0D1B1-5]
% 	Phát biểu các mệnh đề sau
% 	\begin{enumerate}
% 		\item $\forall x \in \mathbb{R}, x^2 \geq 0$.
% 		\item $\exists x \in \mathbb{R}, \dfrac{1}{x}>x$.
% 	\end{enumerate}
% 	\loigiai{
% 		\begin{enumerate}
% 			\item Mọi số thực đều không âm.
% 			\item Tồn tại số thực sao cho nghịch đảo của số đó lớn hơn chính số đó.
% 		\end{enumerate}	
% 	}
% \end{bt}
%%==========Bài 8
\begin{bt}%[Nguyễn Cường- BG Toán 10]%[0D1B1-5]
	Lập mệnh đề phủ định của mỗi mệnh đề sau và xét tính đúng sai của mỗi mệnh đề phủ định đó
	\begin{enumerate}
		\item $\forall x \in \mathbb{R}, x^2 \neq 2x-2$.
		\item $\forall x \in \mathbb{R}, x^2 \leq 2x-1$.
		\item $\exists x \in \mathbb{R}, x+\dfrac{1}{x} \geq 2$.
		\item $\exists x \in \mathbb{R}, x^2-x+1<0$.
	\end{enumerate}
	\loigiai{
		\begin{enumerate}
			\item $\exists x \in \mathbb{R}, x^2=2x-2$.\\
			Mệnh đề này sai vì phương trình $x^2-2x+2=0$ vô nghiệm trên tập số thực.
			\item $\exists x \in \mathbb{R}, x^2 > 2x-1$.\\
			Mệnh đề này đúng vì với $x=2$ thì $2^2>2\cdot 2-1$.
			\item $\forall x \in \mathbb{R}, x+\dfrac{1}{x}<2$.\\
			Mệnh đề này sai vì với $x=1$ thì $1+\dfrac{1}{1}=2$.
			\item $\forall x \in \mathbb{R}, x^2-x+1\ge 0$.\\
			Mệnh đề này đúng vì $x^2-x+1=\left(x-\dfrac{1}{2}\right)^2+\dfrac{3}{4}> 0$ với mọi $x\in\mathbb{R}$.
		\end{enumerate}	
	}
\end{bt}
%%==========Bài 9
\begin{bt}%[Nguyễn Cường- BG Toán 10]%[0D1B1-5]
	Trong tiết học môn Toán, Nam phát biểu: \lq\lq  Mọi số thực đều có bình phương khác $1$\rq\rq. Mai phát biểu: \lq\lq  Có một số thực mà bình phương của nó bằng $1$\rq\rq.
	\begin{enumerate}
		\item Hãy cho biết bạn nào phát biểu đúng.
		\item Dùng kí hiệu $\forall$, $\exists$ để viết lại các phát biểu của Nam và Mai dưới dạng mệnh đề.
	\end{enumerate}
	\loigiai{
		\begin{enumerate}
			\item Bạn Mai phát biểu là đúng vì có số $1$ bình phương lên bằng $1$.
			\item Nam phát biểu \lq\lq $\forall x\in \mathbb{R}, x^2\ne 1$\rq\rq.\\
			Mai phát biểu \lq\lq $\exists x\in \mathbb{R}, x^2=1$\rq\rq.\\
		\end{enumerate}	
	}
\end{bt}
%%==========Bài 10
\begin{bt}%[Nguyễn Cường- BG Toán 10]%[0D1B1-5]
	Phát biểu bằng lời mệnh đề sau và cho biết mệnh đề đó đúng hay sai.
	$$
	\forall x \in \mathbb{R}, x^2+1 \leq 0
	$$
	\loigiai{
		Mọi số thực bình phương lên và cộng cho một luôn không dương.\\
		Đây là một mệnh đề sai vì $0^2+1=1>0$.	
	}
\end{bt}

\subsection{BÀI TẬP TRẮC NGHIỆM ÔN TẬP CUỐI BÀI}

% \Opensolutionfile{ansbook}[ans/ansbook-0D1-1-TN]
\Opensolutionfile{ans}[ans/ans-0D1-1-TN]

\begin{ex}%[Lương Như Quỳnh]%[0D1Y1-1]
	Phát biểu nào dưới đây là mệnh đề?
	\choice
	{\True $2+3=9$}
	{Phong cảnh đẹp quá!}
	{$5-x=7$}
	{Bây giờ là mấy giờ?}
	\loigiai{
		\lq\lq $2+3=9$\rq\rq\ là mệnh đề sai.\\
		\lq\lq  Phong cảnh đẹp quá!\rq\rq\ không là mệnh đề vì đây là câu cảm thán.\\
		\lq\lq $5-x=7$\rq\rq\ là mệnh đề chứa biến.\\
		\lq\lq  Bây giờ là mấy giờ?\rq\rq\ không là mệnh đề vì đây là câu nghi vấn.
	}
\end{ex}
\begin{ex}%[Lương Như Quỳnh]%[0D1B1-1]
	Các câu sau đây, câu nào {\bf không} là mệnh đề?
	\choice
	{Phương trình $ x^2-x+1=0$ vô nghiệm}
	{\True $x+y>1$}
	{$12$ không là số nguyên tố}
	{Hai phương trình $ x^2-4x+3=0$ và $ 2x^2-\sqrt{x+3}=0$ có nghiệm chung}
	\loigiai{
		\lq\lq  Phương trình $ x^2-x+1=0$ vô nghiệm\rq\rq\ là mệnh đề sai.\\
		\lq\lq  $12$ không là số nguyên tố\rq\rq\ là mệnh đề đúng.\\
		\lq\lq  Hai phương trình $ x^2-4x+3=0$ và $ 2x^2-\sqrt{x+3}=0$ có nghiệm chung\rq\rq\ là mệnh đề đúng.\\
		\lq\lq  $x+y>1$\rq\rq\ là mệnh đề chứa biến.}
\end{ex}
\begin{ex}%[Lương Như Quỳnh]%[0D1B1-4]
	Trong các câu sau, câu nào là mệnh đề \textbf{đúng}?
	\choice
	{Nếu $a\ge b$ thì $a^2\ge b^2$}
	{\True Nếu $a$ chia hết cho $9$ thì $a$ chia hết cho $3$}
	{Nếu bạn tự tin thì bạn thành công}
	{Nếu một tam giác có một góc bằng $60^\circ $ thì tam giác đó đều}
	\loigiai
	{
		\begin{itemize}
			\item Mệnh đề \lq\lq  Nếu $a\ge b$ thì $a^2\ge b^2$\rq\rq\ là một mệnh đề sai vì $b\le a < 0$ thì $a^2\le b^2$ .
			\item Mệnh đề \lq\lq  Nếu $a$ chia hết cho $9$ thì $a$ chia hết cho $3$\rq\rq\ là mệnh đề đúng.\\
			Vì $a$ $\vdots$ $9\Rightarrow \heva{&a=9n, n\in \mathbb{Z}\\&9\hspace{0.15cm}\vdots\hspace{0.15cm} 3}\Rightarrow a$ $\vdots$  $3$.
			\item \lq\lq  Nếu bạn tự tin thì bạn thành công\rq\rq\ chưa là mệnh đề vì chưa khẳng định được tính đúng, sai.
			\item Mệnh đề \lq\lq  Nếu một tam giác có một góc bằng $60^\circ $ thì tam giác đó đều\rq\rq\ là mệnh đề sai vì chưa đủ điều kiện để khẳng định một tam giác là đều.
		\end{itemize}
	}
\end{ex}
\begin{ex}%[Lương Như Quỳnh]%[0D1Y1-2]
	Mệnh đề nào sau đây là \textbf{sai}?
	\choice
	{Phương trình $ x^2+bx+c=0$ có nghiệm $\Leftrightarrow b^2-4c\geqslant 0$}
	{\True $\heva{
			&a>b\\
			&b>c} \Leftrightarrow a>c$}
	{$\Delta ABC$ vuông tại $A\Leftrightarrow \widehat{B}+\widehat{C}=90^\circ$}
	{ $ n^2$ chẵn $\Leftrightarrow n$ chẵn}
	\loigiai{
		Xét mệnh đề $\heva{
			&a>b\\
			&b>c\\
		} \Leftrightarrow a>c$, ta có
		\begin{itemize}
			\item $\heva{
				&a>b\\
				&b>c\\
			} \Rightarrow a>c$ đúng.
			\item $ a>c\Rightarrow \heva{
				&a>b\\
				&b>c.\\
			} $ sai. Chẳng hạn $ a=5$; $c=3$; $b=1$ thì $5>3\Rightarrow \heva{
				&5>1\\
				&1>3} $ vô lý.
		\end{itemize} 
	}
\end{ex}
\begin{ex}%[Lương Như Quỳnh]%[0D1Y1-5]
	Trong các mệnh đề sau, mệnh đề nào \textbf{sai}?
	\choice
	{$\exists x\in \mathbb{R},\,x^2-3x+2=0$}
	{$\forall x\in \mathbb{R},\,x^2+1>0$}
	{\True $\exists x\in \mathbb{R},\,x^2<0$}
	{$\forall x\in \mathbb{R},\,|x+1|\ge 0$}
	\loigiai{
		Mệnh đề \lq\lq $\exists x\in \mathbb{R},x^2<0$\rq\rq\, sai, vì $ x^2\ge 0,\,\forall x\in \mathbb{R}$.
	}
\end{ex}

\begin{ex}%[Lương Như Quỳnh]%[0D1B1-4]
	Trong các mệnh đề sau, mệnh đề nào có mệnh đề đảo \textbf{đúng}?
	\choice
	{Nếu số nguyên $n$ có chữ số tận cùng là $5$ thì số nguyên $n$ chia hết cho $5$}
	{\True Nếu tứ giác $ABCD$ có hai đường chéo cắt nhau tại trung điểm mỗi đường thì tứ giác $ABCD$ là hình bình hành}
	{Nếu tứ giác $ABCD$ là hình chữ nhật thì tứ giác $ABCD$ có hai đường chéo bằng nhau}
	{Nếu tứ giác $ABCD$ là hình thoi thì tứ giác $ABCD$ có hai đường chéo vuông góc với nhau}
	\loigiai
	{
		\begin{itemize}
			\item Mệnh đề đảo của mệnh đề \lq\lq  Nếu số nguyên $n$ có chữ số tận cùng là $5$ thì số nguyên $n$ chia hết cho $5$\rq\rq\, là \lq\lq  Nếu số nguyên $n$ chia hết cho $5$ thì số nguyên $n$ có chữ số tận cùng là $5$ \rq\rq. Mệnh đề này sai vì số nguyên $n$ cũng có thể có chữ số tận cùng là $0$.
			\item Mệnh đề đảo của mệnh đề \lq\lq  Nếu tứ giác $ABCD$ có hai đường chéo cắt nhau tại trung điểm mỗi đường thì tứ giác $ABCD$ là hình bình hành\rq\rq\, là \lq\lq  Nếu tứ giác $ABCD$ là hình bình hành thì tứ giác $ABCD$ có hai đường chéo cắt nhau tại trung điểm mỗi đường\rq\rq. Mệnh đề này đúng.
			\item Mệnh đề đảo của mệnh đề \lq\lq  Nếu tứ giác $ABCD$ là hình chữ nhật thì tứ giác $ABCD$ có hai đường chéo bằng nhau\rq\rq\, là \lq\lq  Nếu tứ giác $ABCD$ có hai đường chéo bằng nhau thì tứ giác $ABCD$ là hình chữ nhất\rq\rq. Mệnh đề này sai vì hình thang cân cũng có hai đường chéo bằng nhau, nhưng không là hình chữ nhật.
			\item Mệnh đề đảo của mệnh đề \lq\lq  Nếu tứ giác $ABCD$ là hình thoi thì tứ giác $ABCD$ có hai đường chéo vuông góc\rq\rq\, là \lq\lq  Nếu tứ giác $ABCD$ có hai đường chéo vuông góc thì tứ giác $ABCD$ là hình thoi\rq\rq. Mệnh đề này sai.
		\end{itemize}
	}
\end{ex}
\begin{ex}%[Lương Như Quỳnh]%[0D1B1-3]
	Trong các mệnh đề sau, mệnh đề nào có mệnh đề đảo là \textbf{sai}?
	\choice
	{Nếu tam giác $ABC$ cân thì tam giác có hai cạnh bằng nhau}
	{Nếu $ a$ chia hết cho $6$ thì $ a$ chia hết cho $2$ và $3$}
	{\True Nếu $ABCD$ là hình bình hành thì $AB$ song song với $CD$}
	{Nếu tứ giác có hai đường chéo vuông góc thì tứ giác đó là hình thoi}
	\loigiai{
		Mệnh đề đảo của mệnh đề \lq\lq  Nếu $ABCD$ là hình bình hành thì $AB$ song song với $CD$\rq\rq\, là \lq\lq  Nếu tứ giác $ABCD$ có $AB$ song song với $CD$ thì $ABCD$ là hình bình hành \rq\rq. Mệnh đề này sai vì tứ giác $ABCD$ có thể là hình thang có hai đáy là $AB$ và $CD$. }
\end{ex}
\begin{ex}%[Lương Như Quỳnh]%[0D1B1-5]
	Cho mệnh đề $P(x)\colon$ \lq\lq $\forall x\in \mathbb{R},\ x^2+x+1>0$\rq\rq. Mệnh đề phủ định của mệnh đề $P(x)$ là
	\choice
	{\lq\lq $\forall x\in \mathbb{R},\ x^2+x+1<0$\rq\rq}
	{\lq\lq $\forall x\in \mathbb{R},\ x^2+x+1\leqslant 0$\rq\rq}
	{\True \lq\lq $\exists x\in \mathbb{R},\ x^2+x+1\leqslant 0$\rq\rq}
	{\lq\lq $x\in \mathbb{R},\ x^2+x+1>0$\rq\rq}
	\loigiai{
		Phủ định của mệnh đề $P(x)$ là $\overline{P(x)}\colon$ \lq\lq $\exists x\in \mathbb{R},\ x^2+x+1\leqslant 0$\rq\rq.}
\end{ex}
\begin{ex}%[Lương Như Quỳnh]%[0D1Y1-3]
	Cho mệnh đề $P\colon$ \lq\lq $\exists x\in \mathbb{R},\, x<\dfrac{1}{x}$\rq\rq. Xác định mệnh đề phủ định của mệnh đề $P$.
	\choice
	{$\overline{P}\colon$ \lq\lq  $\exists x\in \mathbb{R},\, x\ge \dfrac{1}{x}$\rq\rq}
	{$\overline{P}\colon$ \lq\lq  $\forall x\in \mathbb{R},\, x> \dfrac{1}{x}$\rq\rq}
	{\True 	$\overline{P}\colon$ \lq\lq  $\forall x\in \mathbb{R},\, x\ge \dfrac{1}{x}$\rq\rq}
	{$\overline{P}\colon$ \lq\lq  $\exists x\in \mathbb{R},\, x> \dfrac{1}{x}$\rq\rq}
	\loigiai{
		Phủ định của mệnh đề $P\colon$ \lq\lq $\exists x\in \mathbb{R},\, x<\dfrac{1}{x}$\rq\rq\, là mệnh đề $\overline{P}\colon$ \lq\lq  $\forall x\in \mathbb{R},\, x\ge \dfrac{1}{x}$\rq\rq.}
\end{ex}
\begin{ex}%[Lương Như Quỳnh]%[0D1B1-2]
	Cách phát biểu nào sau đây \textbf{không} thể dùng để phát biểu mệnh đề $A \Rightarrow B$?
	\choice
	{Nếu $A$ thì $B$}
	{$A$ kéo theo $B$}
	{$A$ là điều kiện đủ để có $B$}
	{\True $A$ là điều kiện cần để có $B$}
	\loigiai{
		$A$ là điều kiện cần để có $B$ dùng để phát biểu mệnh đề $B \Rightarrow A$.
	}
\end{ex}
\begin{ex}%[Lương Như Quỳnh]%[0D1B1-5]
	Trong các mệnh đề sau đây, mệnh đề nào đúng?
	\choice
	{\True Với mọi số thực $x$, nếu $x <-2$ thì $x^2> 4$}
	{Với mọi số thực $x$, nếu $x^2< 4$ thì $x <-2$}
	{Với mọi số thực $x$, nếu $x <-2$ thì $x^2< 4$}
	{Với mọi số thực $x$, nếu $x^2> 4$ thì $x >-2$}
	\loigiai{
		Mệnh đề \lq\lq  Với mọi số thực $x$, nếu $x^2< 4$ thì $x <-2$\rq\rq\ sai. Chẳng hạn $x=1\Rightarrow{x^2}=1 < 4$ nhưng $1 >-2$.\\
		Mệnh đề \lq\lq  Với mọi số thực $x$, nếu $x <-2$ thì $x^2< 4$\rq\rq\  sai. Chẳng hạn $x=-3 <-2$ nhưng $x^2=9 > 4$.\\
		Mệnh đề \lq\lq  Với mọi số thực $x$, nếu $x^2> 4$ thì $x >-2$\rq\rq\ sai. Chẳng hạn $x=-3\Rightarrow{x^2}=9 > 4$ nhưng $-3 <-2$.}
\end{ex}
\begin{ex}%[Lương Như Quỳnh]%[0D1B1-2]
	Biết $A$ là mệnh đề sai và $B$ là mệnh đề đúng. Mệnh đề nào sau đây đúng?
	\choice
	{$B\Rightarrow A$}
	{$B\Leftrightarrow A$}
	{$\overline{A}\Leftrightarrow \overline{B}$}
	{\True $B\Rightarrow \overline{A}$}
	\loigiai{
		Ta có $\overline{A}$ và $B$ đúng nên $B\Rightarrow \overline{A}$ là mệnh đề đúng.
	}
\end{ex}
\begin{ex}%[Lương Như Quỳnh]%[0D1B1-4]
	Cho $P\Leftrightarrow Q$ là mệnh đề đúng. Khẳng định nào sau đây là \textbf{sai}?
	\choice
	{$\overline{P}\Leftrightarrow Q$ sai}
	{$\overline{P}\Leftrightarrow\overline{Q}$ đúng}
	{$\overline{Q}\Leftrightarrow P$ sai}
	{\True $\overline{P}\Leftrightarrow \overline{Q}$ sai}
	\loigiai
	{
		Ta có $P\Leftrightarrow Q$ đúng nên $P\Rightarrow Q$ đúng và $Q\Rightarrow P$ đúng.\\
		Do đó $\overline P\Rightarrow\overline Q $ đúng và $\overline Q\Rightarrow\overline P $ đúng.\\
		Vậy $\overline P\Leftrightarrow\overline Q $ đúng.
	}
\end{ex}
\begin{ex}%[Lương Như Quỳnh]%[0D1B1-2]
	Cho $A$, $B$, $C$ là ba mệnh đề đúng. Mệnh đề nào sau đây là đúng?
	\choice
	{$A\Rightarrow (B\Rightarrow \overline{C})$}
	{$C\Rightarrow \overline{A}$}
	{$B\Rightarrow (\overline{A\Rightarrow C})$}
	{\True $C\Rightarrow (A\Rightarrow B)$}
	\loigiai{
		Ta có $A$, $B$, $C$ là ba mệnh đề đúng nên 
		\begin{itemize}
			\item $ B\Rightarrow \overline{C} $ sai và $A\Rightarrow (B\Rightarrow \overline{C})$ sai.
			\item $ \overline{A} $ sai và $C\Rightarrow \overline{A}$ sai.
			\item $ \overline{A}\Rightarrow C $ đúng và $B\Rightarrow (\overline{A\Rightarrow C})$ sai.
			\item $ A\Rightarrow B $ đúng và $C\Rightarrow (A\Rightarrow B)$ đúng.
	\end{itemize}}
\end{ex}
\begin{ex}%[Lương Như Quỳnh]%[0D1B1-2]
	Trong các mệnh đề nào sau đây mệnh đề nào \textbf{sai}?
	\choice
	{Hai tam giác bằng nhau khi và chỉ khi chúng đồng dạng và có một góc bằng nhau}
	{Một tứ giác là hình chữ nhật khi và chỉ khi chúng có $3$ góc vuông}
	{\True Một tam giác là vuông khi và chỉ khi nó có một góc bằng tổng hai góc còn lại}
	{Một tam giác là đều khi và chỉ khi chúng có hai đường trung tuyến bằng nhau và có một góc bằng $60^{\circ}$}
	\loigiai{
		Mệnh đề \lq\lq  Một tam giác là vuông khi và chỉ khi nó có một góc bằng tổng hai góc còn lại\rq\rq sai. Chẳng hạn tam giác có $A=60^\circ$, $B=70^\circ$, $C=50^\circ$ nhưng tam giác $ABC$ không là tam giác vuông.}
\end{ex}
\begin{ex}%[Lương Như Quỳnh]%[0D1B1-2]
	Trong các mệnh đề sau, mệnh đề nào là mệnh đề đúng?
	\choice
	{Tổng của hai số tự nhiên là một số chẵn khi và chỉ khi cả hai số đều là số chẵn}
	{Tích của hai số tự nhiên là một số chẵn khi và chỉ khi cả hai số đều là số chẵn}
	{Tổng của hai số tự nhiên là một số lẻ khi và chỉ khi cả hai số đều là số lẻ}
	{\True Tích của hai số tự nhiên là một số lẻ khi và chỉ khi cả hai số đều là số lẻ}
	\loigiai{
		Mệnh đề \lq\lq  Tổng của hai số tự nhiên là một số chẵn khi và chỉ khi cả hai số đều là số chẵn\rq\rq\, sai. Ví dụ: $3+5=8$ là số chẵn nhưng $3$ và $5$ là hai số lẻ.\\
		Mệnh đề \lq\lq  Tích của hai số tự nhiên là một số chẵn khi và chỉ khi cả hai số đều là số chẵn\rq\rq\, sai. Ví dụ: $2\cdot 3=6$ là số chẵn nhưng $3$ là số lẻ.\\
		Mệnh đề \lq\lq  Tổng của hai số tự nhiên là một số lẻ khi và chỉ khi cả hai số đều là số lẻ\rq\rq\, sai. Ví dụ: $1+3=4$ là số chẵn nhưng $1$, $3$ là hai số lẻ.}
\end{ex}
\begin{ex}%[Lương Như Quỳnh]%[0D1Y1-2]
	Cho mệnh đề chứa biến $P(x)\colon$ \lq\lq $x>x^3$\rq\rq. Trong các khẳng định sau, khẳng định nào đúng?
	\choice
	{\True $P(1)$ là mệnh đề sai}
	{$P(1)$ là mệnh đề đúng}
	{$P(1)$ là mệnh đề vừa đúng vừa sai}
	{$P(1)$ không phải là mệnh đề}
	\loigiai{
		Mệnh đề $P(1)\colon\lq\lq  1>1^3$\rq\rq\, sai.}
\end{ex}
\begin{ex}%[Lương Như Quỳnh]%[0D1Y1-2]
	Xét mệnh đề chứa biến $P(x)\colon\lq\lq  x\in\mathbb{R},\ x^2-2x\geqslant 0$\rq\rq. Tìm một giá trị của biến để được mệnh đề đúng.
	\choice
	{$ x=\dfrac{1}{4}$}
	{\True $ x=3$}
	{$ x=1$}
	{$ x=0{,}5$}
	\loigiai{
		\begin{itemize}
			\item Với $x=\dfrac14$ ta có $P\left(\dfrac14\right)\colon\lq\lq \left(\dfrac14\right)^2-2\cdot\dfrac14\geqslant0$\rq\rq\ là mệnh đề sai.
			\item Với $x=3$ ta có $P\left(3\right)\colon\lq\lq 3^2-2\cdot3\geqslant0$\rq\rq\ là mệnh đề đúng.
			\item Với $x=1$ ta có $P\left(1\right)\colon\lq\lq 1^2-2\cdot1\geqslant0$\rq\rq\ là mệnh đề sai.
			\item Với $x=0{,}5$ ta có $P\left(0{,}5\right)\colon\lq\lq 0{,}5^2-2\cdot0{,}5\geqslant0$\rq\rq\ là mệnh đề sai.
		\end{itemize}
	}
\end{ex}

\begin{ex}%[Lương Như Quỳnh]%[0D1B1-5]
	Mệnh đề nào dưới đây {\bf sai}?
	\choice
	{$x\left(1-2x\right)\le\dfrac{1}{8},\, \forall x$}
	{\True $x^2+2+\dfrac{1}{x^2+2}>\dfrac{5}{2},\, \forall x$}
	{$\dfrac{x^2-x+1}{x^2+x+1}\ge\dfrac{1}{3},\, \forall x$}
	{$\dfrac{x}{x^2+1}\le\dfrac{1}{2},\, \forall x$}
	\loigiai{
		Ta có
		\begin{itemize}
			\item $x\left(1-2x\right)\le\dfrac{1}{8}\Leftrightarrow 2\left(x-\dfrac{1}{4}\right)^2\ge 0$ (đúng).
			\item $\dfrac{x^2-x+1}{x^2+x+1}\ge \dfrac{1}{3}\Leftrightarrow \dfrac{3(x^2-x+1)-(x^2+x+1)}{x^2+x+1}\ge 0\Leftrightarrow \dfrac{2(x-1)^2}{x^2+x+1}\ge 0$ (đúng).
			\item $\dfrac{x}{x^2+1}\le \dfrac{1}{2}\Leftrightarrow (x-1)^2\ge 0$ (đúng).
			\item Với $x=0$ dễ thấy $0^2+2+\dfrac{1}{0^2+2}>\dfrac{5}{2}$ sai.
		\end{itemize}
	}
\end{ex}

\begin{ex}%[Lương Như Quỳnh]%[0D1K1-5]
	Mệnh đề nào sau đây {\bf sai}?
	\choice
	{$\forall x \in \mathbb{R},\,3x^2-4x+4>0$}
	{\True $\exists x \in \mathbb{R},\,(x-1)^2+(x+1)^2=0$}
	{$\exists x \in \mathbb{Q},\,x<\dfrac{1}{x}$}
	{$\exists n \in \mathbb{N},\,(1+2+3+ \cdots +n)\ \vdots\ 11$}
	\loigiai{
		\begin{itemize}
			\item Mệnh đề \lq\lq  $\forall x \in \mathbb{R},\,3x^2-4x+4>0$\rq\rq\, đúng vì $3x^2-4x+4=2x^2+(x-2)^2>0,\ \forall x\in\mathbb{R}$.
			\item Mệnh đề \lq\lq  $\exists x \in \mathbb{Q},\,x<\dfrac{1}{x}$\rq\rq\, đúng vì với $x=\dfrac{1}{2}$ thì $x<\dfrac{1}{x}$.
			\item Mệnh đề \lq\lq  $\exists n \in \mathbb{N},\,(1+2+3+ \cdots +n)\ \vdots\ 11$\rq\rq\, đúng vì với $n=10$ thì $1+2+\cdots +10=5\cdot 11\ \vdots\ 11$.
			\item Mệnh đề \lq\lq  $\exists x \in \mathbb{R},\,(x-1)^2+(x+1)^2=0$\rq\rq\, sai vì $(x-1)^2+(x+1)^2>0,\ \forall x\in\mathbb{R}$.
		\end{itemize}
	}
\end{ex}

\Closesolutionfile{ans}
% \Closesolutionfile{ansbook}
% \indapan{10}{ans/ans-0D1-1-TN}
% %\Opensolutionfile{ansbook}[ans/ansbook-BG10-2022-2]
%\Opensolutionfile{ans}[ans/ans-BG10-2022-2]
\setcounter{section}{1}
\section{Tập hợp và các phép toán trên tập hợp}
\subsection{LÝ THUYẾT}
\subsubsection{Tập hợp}
\begin{tcolorbox}
	Có thể mô tả một tập hợp bằng một trong hai cách sau:\\
	\textbf{Cách 1}. Liệt kê các phần tử của tập hợp;\\
	\textbf{Cách 2}. Chỉ ra tính chất đặc trưng cho các phần tử của tập hợp.
\end{tcolorbox}
$a \in S$: phần tử $a$ thuộc tập hợp $S$.
$a \notin S$: phần tử $a$ không thuộc tập hợp $S$.
\begin{note}
	\textbf{Chú ý:}\\
	$\bullet$ Số phần tử của tập hợp $S$ được kí hiệu là $n(S)$.\\
	$\bullet$ Tập hợp không chứa phần tử nào được gọi là tập rỗng, kí hiệu là $\varnothing$.
\end{note}
\subsubsection{Tập hợp con}
\begin{tcolorbox}
	$T \subset S \Leftrightarrow \forall x, (x \in T \Rightarrow x \in S).$ 
\end{tcolorbox} \noindent
\begin{note}
	$\bullet$ Quy ước tập rỗng là tập con của mọi tập hợp.
\end{note}

\begin{tcolorbox}
	\immini {$\bullet$ Người ta thường minh hoạ một tập hợp bằng một hình phẳng được bao quanh bởi một đường kín, gọi là biểu đồ Ven (H.1.2).}
	{\begin{tikzpicture}[>=stealth,line join=round,line cap=round,font=\footnotesize,scale=.7]
			\draw (1,1) circle (2 and 1);
			\coordinate[label=center:$X$] (X)at(-.5,1.3);
			\coordinate[label=center:$ H.1.2$] (X)at(1,-.5);
	\end{tikzpicture}}
	\immini {$\bullet$ Minh hoạ $T$ là một tập con của $S$ như Hình $1.3$.}
	{\begin{tikzpicture}[>=stealth,line join=round,line cap=round,font=\footnotesize,scale=.7]
			\coordinate[label=center:$$] (I)at(.4,1);
			\coordinate[label=center:$$] (M)at(1,1);
			\draw (.4,1) circle (3 and 2);
			\coordinate[label=center:$S$] (X)at(-2,1.5);
			\draw (1,1) circle (2 and 1);
			\coordinate[label=center:$T$] (X)at(1,.5);
			\coordinate[label=center:$ H.1.3$] (X)at(1,-1.5);
			\fill[cyan,opacity=.2] (I) ellipse (3 cm and 2 cm)--cycle;
			\fill[violet,opacity=.4] (M) ellipse (2 cm and 1 cm)--cycle;
	\end{tikzpicture}}
\end{tcolorbox}
\subsubsection{Hai tập hợp bằng nhau}
\begin{tcolorbox}
	$S=T \Leftrightarrow \heva{& S \subset T \\ & T \subset S} \Leftrightarrow \forall x,\ (x\in S \Leftrightarrow x \in T)$
\end{tcolorbox}

\subsubsection{Mối quan hệ giữa các tập hợp số}
$\bullet$ Tập hợp các số tự nhiên $\mathbb{N}=\{0 ; 1 ; 2 ; 3 ; \ldots\}$.\\
$\bullet$ Tập hợp các số nguyên $\mathbb{Z}$ gồm các số tự nhiên và các số nguyên âm:
$\mathbb{Z}=\{\ldots ;-2 ;-1 ; 0 ; 1 ; 2 ; \ldots\}$.
$\bullet$ Tập hợp các số hữu tỉ $\mathbb{Q}$ gồm các số viết được dưới dạng phân số $\dfrac{a}{b}$, với $a, b \in \mathbb{Z}, b \neq 0$. Số hữu tỉ còn được biểu diễn dưới dạng số thập phân hữu hạn hoặc vô hạn tuần hoàn.\\
$\bullet$ Tập hợp các số thực $\mathbb{R}$ gồm các số hữu tỉ và các số vô tỉ. Số vô tỉ là các số thập phân vô hạn không tuần hoàn.
\begin{tcolorbox}
	\immini { Mối quan hệ giữa các tập hợp số:  $\mathbb{N} \subset \mathbb{Z} \subset \mathbb{Q} \subset \mathbb{R}$.}
	{\begin{tikzpicture}[>=stealth,line join=round,line cap=round,font=\footnotesize,scale=.7]
			\path (1.3,1)coordinate[label=center:$$](M) (1.5,.7)coordinate[label=center:$$](N) (1.6,.5)coordinate[label=center:$$](P) (1.7,.3)coordinate[label=center:$$](Q)
			(1.5,-1.5)coordinate[label=center:$H.1.5$](H);
			\draw (1.3,1) circle (3 and 2);
			\draw (1.5,.7) circle (2 and 1.5);
			\draw (1.6,.5) circle (1.5 and 1);
			\draw (1.7,.3) circle (1 and .5);
			\fill[red,opacity=.2] (M) ellipse (3 cm and 2 cm)--cycle;
			\fill[cyan,opacity=.3] (N) ellipse (2 cm and 1.5 cm)--cycle;
			\fill[violet,opacity=.4] (P) ellipse (1.5 cm and 1 cm)--cycle;
			\fill[blue,opacity=.7] (Q) ellipse (1 cm and .5 cm)--cycle;
			\coordinate[label=center:$\mathbb{R}$] (X)at(-1,2);
			\coordinate[label=center:$\mathbb{Q}$] (Q)at(0.2,1.5);
			\coordinate[label=center:$\mathbb{Z}$] (Z)at(1,1.2);
			\coordinate[label=center:$\mathbb{N}$] (N)at(2,.5);
	\end{tikzpicture}}
\end{tcolorbox}
\subsubsection{Các tập con thường dùng của $\mathbb{R}$}
\begin{tcolorbox}
	Một số tập con thường dùng của tập số thực $\mathbb{R}$.\\
	$\bullet$ Khoảng\\
	\immini {$(a ; b)=\{x \in \mathbb{R} \mid a<x<b\}$} {\begin{tikzpicture}[>=stealth,line join=round,line cap=round,font=\footnotesize,scale=1]
			\draw[->] (0,0)--(6,0);
			\path[pattern=north east lines,pattern color=blue] (0,-3pt)rectangle(2,3pt);
			\path[pattern=north east lines,pattern color=blue] (6,-3pt)rectangle(4,3pt);
			\path 
			(2,-0.1)coordinate[label=below:$a$](a) 
			(4,-0.1)coordinate[label=below:$b$](b)
			(2,0)coordinate[label=center:$($](()
			(4,0)coordinate[label=center:$)$](s);
	\end{tikzpicture}}
	\immini {$(a ;+\infty)=\{x \in \mathbb{R} \mid x>a\}$}
	{\begin{tikzpicture}[>=stealth,line join=round,line cap=round,font=\footnotesize,scale=1]
			\draw[->] (0,0)--(6,0);
			\path[pattern=north east lines,pattern color=blue] (0,-3pt)rectangle(2,3pt);
			%\path[pattern=north east lines,pattern color=blue] (6,-3pt)rectangle(4,3pt);
			\path 
			(2,-0.1)coordinate[label=below:$a$](a) 
			%(4,0)coordinate[label=below:$a$](b)
			(2,0)coordinate[label=center:$($](()
			%(4,0)coordinate[label=center:$)$](s)
			;
	\end{tikzpicture}}
	\immini {$(-\infty ; b)=\left\{x \in \mathbb{R} \mid x<b\right\}$}
	{\begin{tikzpicture}[>=stealth,line join=round,line cap=round,font=\footnotesize,scale=1]
			\draw[->] (0,0)--(6,0);
			%\path[pattern=north east lines,pattern color=blue] (0,-3pt)rectangle(2,3pt);
			\path[pattern=north east lines,pattern color=blue] (6,-3pt)rectangle(4,3pt);
			\path 
			%(2,0)coordinate[label=below:$a$](a) 
			(4,-0.1)coordinate[label=below:$b$](b)
			%(2,0)coordinate[label=center:$($](()
			(4,0)coordinate[label=center:$)$](s)
			;
	\end{tikzpicture}}
	\immini {$(-\infty ;+\infty)$}
	{\begin{tikzpicture}[>=stealth,line join=round,line cap=round,font=\footnotesize,scale=1]
			\draw[->] (0,0)--(6,0);
			%\path[pattern=north east lines,pattern color=blue] (0,-3pt)rectangle(2,3pt);
			%\path[pattern=north east lines,pattern color=blue] (6,-3pt)rectangle(4,3pt);
			\path 
			(3,-0.1)coordinate[label=below:$O$](a) 
			%(4,0)coordinate[label=below:$a$](b)
			(3,0)coordinate[label=center:$|$](()
			%(4,0)coordinate[label=center:$)$](s)
			;
	\end{tikzpicture}}
	$\bullet$ Đoạn\\
	\immini {$[a ; b]=\{x \in \mathbb{R} \mid a \leq x \leq b\}$}
	{\begin{tikzpicture}[>=stealth,line join=round,line cap=round,font=\footnotesize,scale=1]
			\draw[->] (0,0)--(6,0);
			\path[pattern=north east lines,pattern color=blue] (0,-3pt)rectangle(2,3pt);
			\path[pattern=north east lines,pattern color=blue] (6,-3pt)rectangle(4,3pt);
			\path 
			(2,-0.1)coordinate[label=below:$a$](a) 
			(4,-0.1)coordinate[label=below:$b$](b)
			(2,0)coordinate[label=center:$[$]()
			(4,0)coordinate[label=center:$\mathrm{]}$](s);
	\end{tikzpicture}}
	$\bullet$ Nửa khoảng\\
	\immini {$[a ; b)=\{x \in \mathbb{R} \mid a \leq x<b\}$}
	{\begin{tikzpicture}[>=stealth,line join=round,line cap=round,font=\footnotesize,scale=1]
			\draw[->] (0,0)--(6,0);
			\path[pattern=north east lines,pattern color=blue] (0,-3pt)rectangle(2,3pt);
			\path[pattern=north east lines,pattern color=blue] (6,-3pt)rectangle(4,3pt);
			\path 
			(2,-0.1)coordinate[label=below:$a$](a) 
			(4,-0.1)coordinate[label=below:$b$](b)
			(2,0)coordinate[label=center:$[$]()
			(4,0)coordinate[label=center:$\mathrm{)}$](s);
	\end{tikzpicture}}
	\immini {$(a ; b]=\{x \in \mathbb{R} \mid a<x \leq b\}$}
	{\begin{tikzpicture}[>=stealth,line join=round,line cap=round,font=\footnotesize,scale=1]
			\draw[->] (0,0)--(6,0);
			\path[pattern=north east lines,pattern color=blue] (0,-3pt)rectangle(2,3pt);
			\path[pattern=north east lines,pattern color=blue] (6,-3pt)rectangle(4,3pt);
			\path 
			(2,-0.1)coordinate[label=below:$a$](a) 
			(4,-0.1)coordinate[label=below:$b$](b)
			(2,0)coordinate[label=center:$($]()
			(4,0)coordinate[label=center:$\mathrm{]}$](s);
	\end{tikzpicture}}
	\immini {$[a ;+\infty)=\{x \in \mathbb{R} \mid x \geq a\}$}
	{\begin{tikzpicture}[>=stealth,line join=round,line cap=round,font=\footnotesize,scale=1]
			\draw[->] (0,0)--(6,0);
			\path[pattern=north east lines,pattern color=blue] (0,-3pt)rectangle(2,3pt);
			%\path[pattern=north east lines,pattern color=blue] (6,-3pt)rectangle(4,3pt);
			\path 
			(2,-0.1)coordinate[label=below:$a$](a) 
			%(4,0)coordinate[label=below:$a$](b)
			(2,0)coordinate[label=center:$[$](()
			%(4,0)coordinate[label=center:$)$](s)
			;
	\end{tikzpicture}}
	\immini {$(-\infty ; b]=\{x \in \mathbb{R} \mid x \leq b\}$}
	{\begin{tikzpicture}[>=stealth,line join=round,line cap=round,font=\footnotesize,scale=1]
			\draw[->] (0,0)--(6,0);
			%\path[pattern=north east lines,pattern color=blue] (0,-3pt)rectangle(2,3pt);
			\path[pattern=north east lines,pattern color=blue] (6,-3pt)rectangle(4,3pt);
			\path 
			%(2,0)coordinate[label=below:$a$](a) 
			(4,-0.1)coordinate[label=below:$b$](b)
			%(2,0)coordinate[label=center:$($](()
			(4,0)coordinate[label=center:$\mathrm{]}$](s)
			;
	\end{tikzpicture}}
\end{tcolorbox}

\subsubsection{Giao của hai tập hợp}
\begin{tcolorbox}
	\immini{Tập hợp gồm các phần tử thuộc cả hai tập hợp $S$ và $T$ gọi là giao của hai tập hợp $S$ và $T$, kí hiệu là $S \cap T$.\\
		$S \cap T=\{x \mid x \in S$ và $x \in T\}$.}
	{\begin{tikzpicture}[>=stealth,line join=round,line cap=round,font=\footnotesize,scale=1]
			\begin{scope}
				\clip (0.5,0) ellipse (1.5 and 1 );
				\fill[pattern = north east lines]	(2.5,0) ellipse (2 and 1.5 );
			\end{scope}
			\draw (0.5,0) ellipse (1.5 and 1 );
			\draw (2.5,0) ellipse (2 and 1.5 );
			\coordinate[label=center:$S\cap T$] (M)at(1.1,.3);
	\end{tikzpicture}}
\end{tcolorbox}
\subsubsection{Hợp của hai tập hợp}
\begin{tcolorbox}
	\immini {Tập hợp gồm các phần tử thuộc tập hợp $S$ hoặc thuộc tập hợp $T$ gọi là hợp của hai tập hợp $S$ và $T$. Kí hiệu là $S \cup T$.
		\[S \cup T=\{x \mid x \in S \text { hoặc } x \in T\}.\]}
	{\begin{tikzpicture}[>=stealth,line join=round,line cap=round,font=\footnotesize,scale=.7]
			\coordinate[label=center:$$] (I)at(0,1);
			\coordinate[label=center:$$] (M)at(1,1);
			\draw[fill,pattern = north east lines] (0,1) circle (1.5 and 1);
			\draw[fill,pattern = north east lines] (1,1) circle (1.5 and 1);
			\coordinate[label=center:$S\cup T$] (X)at(.6,-.5);
			\coordinate[label=center:$S$] (X)at(-1,1);
			\coordinate[label=center:$T$] (X)at(2,1);
	\end{tikzpicture}}
\end{tcolorbox}
\subsubsection{Hiệu của hai tập hợp}
\begin{tcolorbox}
	\immini {$\bullet$ Hiệu của hai tập hợp $S$ và $T$ là tập hợp gồm các phần tử thuộc S nhưng không thuộc $T$, kí hiệu là $S \backslash T$.
		\[S \backslash T=\{x \mid x \in S \text{ và } x \notin T\}\].}
	{\begin{tikzpicture}[>=stealth,line join=round,line cap=round,font=\footnotesize,scale=.8]
			\draw (0.5,0) ellipse (1.5 and 1 );
			\draw (2.5,0) ellipse (1.5 and 1 );
			\fill[pattern = north east lines] (0.5,0) ellipse (1.5 and 1 );
			\fill[fill=white] (2.5,0) ellipse (1.5 and 1 );
			\path 
			(0,.5)coordinate[label=center:$S$](S) (2,.5)coordinate[label=center:$T$](T)
			(1.7,1.5)coordinate[label=center:$S\backslash T$](T) ;
			
	\end{tikzpicture}}
	\immini {$\bullet$ Nếu $T \subset S$ thì $S \backslash T$ được gọi là phần bù của $T$ trong $S$, kí hiệu là $C_{s} T$.}
	{	\begin{tikzpicture}[>=stealth,line join=round,line cap=round,font=\footnotesize,scale=.8]
			\draw (0.5,0) ellipse (1.5 and 1 );
			\draw (.8,0) ellipse (1 and .5 );
			\fill[pattern = north east lines] (0.5,0) ellipse (1.5 and 1 );
			\fill[fill=white] (.8,0) ellipse (1 and .5 );
			\path 
			(-.3,.5)coordinate[label=center:$S$](S) (1,0)coordinate[label=center:$T$](T)
			(1,1.3)coordinate[label=center:$C_S T$](T) ;
			
	\end{tikzpicture}}
\end{tcolorbox}
\subsection{CÁC DẠNG BÀI TẬP}
\begin{dang}{Xác định tập hợp}
	Được mô tả theo 2 cách:
	\begin{enumEX}{1}
		\item  Liệt kê tất cả các phần tử của tập hợp.
		\item  Nêu tính chất đặc trưng.
	\end{enumEX}
\end{dang}
\subsubsection{Ví dụ minh hoạ}
\begin{vd}%[BG10-2022]%[Đỗ Văn Dự]%[0D1Y2-1]
	Cho $D=\{n \in \mathbb{N} \mid n$ là số nguyên tố, $5<n<20\}$.
	\begin{enumEX}{1}
		\item Dùng kí hiệu $\in, \notin$ để viết câu trả lời cho câu hỏi sau: Trong các số $5$; $12$; $17$; $18$, số nào thuộc tập $D$, số nào không thuộc tập $D$?
		\item Viết tập hợp $D$ bằng cách liệt kê các phần tử. Tập hợp $D$ có bao nhiêu phần tử?
	\end{enumEX}
	\loigiai{
		\begin{enumEX}{1}
			\item $5 \notin D$; $12 \notin D$; $17 \in D$; $18 \notin D$.
			\item $D=\{7 ; 11 ; 13 ; 17 ; 19\}$. Tập hợp $D$ có $5$ phần tử.
		\end{enumEX}	
	}
\end{vd}

\begin{vd}%[BG10-2022]%[Đỗ Văn Dự]%[0D1B2-1] 
	Viết mỗi tập hợp sau bằng cách liệt kê các phần tử.
	\begin{enumEX}{2}
		\item $A=\left\{\left. x\in \mathbb{R}\right|\left(2x-x^2\right)\left(3x-2\right)=0\right\}$.
		\item $B=\left\{\left. x\in \mathbb{Z}\right|2x^3-3x^2-5x=0\right\}$.
		\item $C=\left\{\left. x\in \mathbb{Z}\right|2x^2-75x-77=0\right\}$.
		\item $D=\left\{\left. x\in \mathbb{R}\right|(x^2-x-2)(x^2-9)=0\right\}$.
	\end{enumEX}
	\loigiai{
		\begin{enumEX}{1}
			\item Ta giải phương trình\\
			 $\left(2x-x^2\right)\left(2x^2-3x-2\right)=0\Leftrightarrow \hoac{
				& 2x-x^2=0 \\ 
				& 2x^2-3x-2=0}\Leftrightarrow \hoac{
				& x=0\vee x=2 \\ 
				& x=-\dfrac{1}{2}\vee x=2}$.\\
			Do $x\in \mathbb{R}$ nên $A=\left\{-\dfrac{1}{2};0;2\right\}$.
			\item Ta giải phương trình $2x^3-3x^2-5x=0\Leftrightarrow x\left(2x^2-3x-5\right)=0\Leftrightarrow \hoac{&x=0\\&x=-1\\&x=\dfrac{5}{3}}$.\\
			Do $x\in \mathbb{Z}$ nên $B=\left\{0;-1\right\}$.
			\item Ta giải phương trình $2x^2-75x-77=0\Leftrightarrow \hoac{&x=-1\\&x=\dfrac{77}{2}}$.\\
			Do $x\in \mathbb{Z}$ nên $C=\left\{-1\right\}$.
	\end{enumEX}}
\end{vd}

\begin{vd}%[BG10-2022]%[Đỗ Văn Dự]%[0D1B2-1] 
	Viết mỗi tập hợp sau bằng cách liệt kê các phần tử.
	\begin{enumEX}{1}
		\item $A=\left\{\left. n\in {\mathbb{N}}^{*}\right|3<n^2<30\right\}$.
		\item $B=\left\{\left. n\in \mathbb{Z}\right|\left| n\right|<3\right\}$.
		\item $C=\left\{\left. x\right|x=3k\right.$ với $k\in \mathbb{Z}$ và $\left.-4<x<12\right\}$.
		\item $D=\left\{\left. n^2+3\right|n \in \mathbb{N} \text{ và } n<5\right\}$.
	\end{enumEX}
	\loigiai{
		\begin{enumEX}{1}
			\item Với $3<n^2<30$ và $n\in {\mathbb{N}}^{*}$ nên chọn $n=2;3;4;5$.\\
			Vậy $A=\left\{2;3;4;5\right\}$.
			\item  Vì $x<\left| 3\right|\Leftrightarrow-3<x<3$.\\
			Do $x\in \mathbb{Z}$ nên $B=\left\{-2;-1;0;1;2\right\}$.
			\item Ta có $-4<x<12\Leftrightarrow-4<3k<12\Leftrightarrow-\dfrac{4}{3}<k<4$.\\
			Do $k\in \mathbb{Z}$ nên ta chọn $k=\left\{-10;1;2;3\right\}$ suy ra $x=3k=\left\{-3;0;3;6;9\right\}$.\\
			Vậy $C=\left\{-3;0;3;6;9\right\}$.
			\item Vì $n \in \mathbb{N} \text{ và } n<5$ nên chọn  $n=0,1,2;3;4$.\\
			Vậy $A=\left\{3;4;12;19\right\}$.
		\end{enumEX}
	}
\end{vd}

\begin{vd}%[BG10-2022]%[Đỗ Văn Dự]%[0D1K2-1]
	Viết mỗi tập hợp sau bằng cách nêu tính chất đặc trưng.
	\begin{enumEX}{2}
		\item $A=\left\{\dfrac{2}{3};\dfrac{3}{8};\dfrac{4}{15};\dfrac{5}{24};\dfrac{6}{35}\right\}$.
		\item $B=\left\{0;3;8;15;24;35\right\}$.
		\item $C=\left\{-4;1;6;11;16\right\}$.
		\item $D=\left\{1;-2;7\right\}$.
	\end{enumEX}
	\loigiai{
		\begin{enumEX}{2}
			\item $A=\left\{\left. \dfrac{n}{n^2-1}\right|n\in \mathbb{N},2\le n\le 6\right\}$.
			\item $B=\left\{\left. n^2-1\right|n\in \mathbb{N},1\le n\le 6\right\}$.
			\item $C=\left\{\left. n\in \mathbb{N}\right|\right.\left. 5n-4\right\}$.
			\item $D=\left\{\left. x\in \mathbb{R}\right|\left(x-1\right)\left(x+2\right)\left(x-7\right)=0\right\}$.
		\end{enumEX}
	}
\end{vd}
\subsubsection{Bài tập tự luận}
\begin{bt}%[Huỳnh Quy]%[0D1B2-1]
	Liệt kê các phần tử của các tập hợp sau:
	\begin{enumerate}
		\item $A=\left\lbrace n\in \mathbb{N} \mid n<5\right\rbrace$.
		\item $B$ là tập hợp các số tự nhiên lớn hơn $0$ và nhỏ hơn $5$.
		\item $C=\left\lbrace x\in \mathbb{R}\mid (x-1)(x+2)=0\right\rbrace$.
	\end{enumerate}
	\loigiai{
		\begin{enumerate}
			\item $A=\left\lbrace 0;1;2;3;4\right\rbrace$.
			\item $B=\left\lbrace 1;2;3;4\right\rbrace$.
			\item Ta có $(x-1)(x+2)=0 \Leftrightarrow \hoac{&x=1\\&x=-2.}$\\
			Mà $x\in \mathbb{R}$ nên
			$C=\left\lbrace -2;1\right\rbrace$.
		\end{enumerate}
	}
\end{bt}
\begin{bt}%[Huỳnh Quy]%[0D1B2-1]
	Viết các tập hợp sau bằng phương pháp liệt kê:
	\begin{enumerate}
		\item $A=\left\lbrace  x\in \mathbb{Q}\mid (x^2-2x+1)(x^2-5)\right\rbrace=0$.
		\item $B=\left\lbrace x \in \mathbb{N}\mid 5<x^2<40\right\rbrace$.
		\item $C=\left\lbrace x\in \mathbb{Z}\mid x^2<9\right\rbrace$.
		\item $D=\left\lbrace x\in \mathbb{R}\mid \left|2x+1\right|=5\right\rbrace$.
	\end{enumerate}
	\loigiai{
		\begin{enumerate}
			\item Ta có $x\in A\Leftrightarrow\hoac{&x^2-2x+1=0\\&x^2-5=0}\Leftrightarrow\hoac{&x=1\in\mathbb{Q}\\&x=\pm\sqrt{5}\not\in \mathbb{Q}.}$\\
			Vậy $A=\left\lbrace 1\right\rbrace$.
			\item $B=\left\lbrace 3;4;5;6\right\rbrace$.
			\item $C=\left\lbrace -2;-1;0;1;2\right\rbrace$.
			\item Ta có $\left|2x+1\right|=5\Leftrightarrow \hoac{& 2x+1=5 \\ & 2x+1 =-5} \Leftrightarrow \hoac{&x=2\\&x=-3.}$\\
			Vậy $D=\left\lbrace 2;-3\right\rbrace$.
		\end{enumerate}
	}
\end{bt}
\begin{bt}%[Huỳnh Quy]%[0D1B2-1]
	Viết các tập hợp sau bằng cách chỉ ra tính chất đặc trưng cho các phần tử của tập hợp đó.
	\begin{enumerate}
		\item $A=\left\lbrace 0;4;8;12;16;\ldots ;52\right\rbrace$.
		\item $B=\left\lbrace 3;6;9;12;15;\ldots ;51\right\rbrace$.
		\item $C=\left\lbrace 2;5;8;11;14;\ldots ;62\right\rbrace$.
	\end{enumerate}
	\loigiai{
		\begin{enumerate}
			\item $A=\left\lbrace x\in \mathbb{N}\mid 0\le x\le 52 \text{ và } x\;\vdots \;4\right\rbrace$.
			\item $B=\left\lbrace x\in \mathbb{N}\mid 3\le x\le 51 \text{ và } x\;\vdots \;3\right\rbrace$.
			\item $C=\left\lbrace  x\in \mathbb{N}\mid 2\le x\le 62 \text{ và } (x-2)\;\vdots \;3\right\rbrace$.
		\end{enumerate}
	}
\end{bt}
\begin{bt}%[Huỳnh Quy]%[0D1K2-1]
	Viết các tập hợp sau bằng cách chỉ ra tính chất đặc trưng cho các phần tử của tập hợp đó.
	\begin{enumerate}
		\item $A=\left\lbrace 2;3;5;7;11;13;17\right\rbrace$.
		\item $B=\left\lbrace -2;4;-8;16;-32;64\right\rbrace$.
	\end{enumerate}
	\loigiai{
		\begin{enumerate}
			\item $A=\left\lbrace x\in \mathbb{N}\mid x\le 17 \text{ và }x \text{ là số nguyên tố} \right\rbrace$.
			\item $B=\left\lbrace x=(-2)^n\mid n\in \mathbb{N}, 1\le n\le 6 \right\rbrace$.
		\end{enumerate}
	}
\end{bt}

\begin{bt}%[Huỳnh Quy]%[0D1B2-1]
	Tìm một tính chất đặc trưng xác định các phần tử của mỗi tập hợp sau
	\begin{align*}
		A&=\{1; 2; 3; 4; 5; 6; 7; 8; 9\} \\
		B&=\{0; 7; 14; 21; 28\}
	\end{align*}
	\loigiai{
		\begin{align*}
			A&=\{x\in \mathbb{N^*} \mid x\leq 9\} \\
			B&=\{x\in \mathbb{N} \mid x \;\vdots \;7 \text{ và } x\leq 28\}
		\end{align*}
	}
\end{bt}

\begin{dang}{Tập hợp con, xác định tập hợp con}
	Cho tập hợp $A$ gồm $n$ phần tử.
	\begin{enumEX}{1}
		\item  Khi liệt kê tất cả các tập con của $A$, ta liệt kê đầy đủ theo thứ tự:\\		
		\centerline{ $\varnothing$; tập $1$ phần tử; tập $2$ phần tử; tập $3$ phần tử;...; $A$.}
		\item  Số tập con của $A$ là $2^n$.
		\item  Số tập con gồm $k$ phần tử của $A$ là $\mathrm{C}_n^k$.
	\end{enumEX}
\end{dang}
\subsubsection{Ví dụ minh hoạ}
\begin{vd}%[BG10-2022]%[Đỗ Văn Dự]%[0D1Y2-2]
	Cho tập hợp $S=\{2 ; 3 ; 5\}$. Những tập hợp nào sau đây là tập con của $S$?
	$$S_{1}=\{3\};S_{2}=\{0 ; 2\}; S_{3}=\{3 ; 5\}$$.
	\loigiai{
		Các tập hợp $S_{1}=\{3\}$, $S_{3}=\{3 ; 5\}$ là những tập con của $S$.\\
		Tập $S_{2}=\{0 ; 2\}$ không là tập con của $S$.
	}
\end{vd}

\begin{vd}%[BG10-2022]%[Đỗ Văn Dự]%[0D1B2-2]
	Cho tập hợp $A=\left\{2;3;4\right\}$ và $B=\left\{2;3;4;5;6\right\}$.
	\begin{enumEX}{1}
		\item Xác định tất cả tập con có hai phần tử của $A$.
		\item Xác định tất cả tập con có ít hơn hai phần tử của $A$.
		\item Tập $A$ có tất cả bao nhiêu tập con.
		\item Xác định tất cả các tập $X$ thỏa $A \subset X \subset B$.
	\end{enumEX}
		\loigiai{
	\begin{enumEX}{1}		
		\item 	Các tập hợp $S_1=\{2;3\}$, $S_2=\{2;4\}$, $S_3=\{3;4\}$  là những tập con của $A$.\\
			Tập $S_{2}=\{0 ; 2\}$ không là tập con của $S$.
		\item 	Các tập hợp $\varnothing$, $\{2\}$, $\{3\}$, $\{4\}$  là những tập con ít hơn $2$ phần tử của $A$.\\
		\item 	Tập $A$ có tất cả $8$ tập con.
		\item 	 Tất cả các tập $X$ thỏa $A \subset X \subset B$ là $\{2;3;4\}$, $\{2;3;4,5\}$, $\{2;3;4,5,6\}$.	
	\end{enumEX}
		}	
\end{vd}
\subsubsection{Bài tập tự luận}
\begin{bt}%[Huỳnh Quy]%[0D1B2-2]
	Tìm tất cả các tập con của tập $A=\{a,1,2\}$.
	\loigiai{Tập $A$ có $2^3=8$ tập con.
		\begin{itemize}
			\item 0 phần tử: $ \varnothing $.
			\item 1 phần tử: $\{a\}$, $\{1\}$, $\{2\}$.
			\item 2 phần tử: $\{a, 1\}$, $\{a,2\}$, $\{1,2\}$.
			\item 3 phần tử: $\{a,1,2\}$.
	\end{itemize}}
\end{bt}
\begin{bt}%[Huỳnh Quy]%[0D1B2-2]
	Tìm tất cả các tập con có 2 phần tử của tập $A=\{1,2,3,4,5,6\}$.
	\loigiai{$\{1,2\}$,$\{1,3\}$, $\{1,4\}$, $\{1,5\}$, $\{1,6\}$, $\{2,3\}$, $\{2,4\}$, $\{2,5\}$, $\{2,6\}$, $\{3,4\}$, $\{3,5\}$, $\{3,6\}$, $\{4,5\}$, $\{4,6\}$, $\{5,6\}$.}
\end{bt}
\begin{bt}%[Huỳnh Quy]%[0D1B2-2]
	Xác định tập hợp $X$ biết $\{1,2\} \subset X \subset \{1,2,5\}$.
	\loigiai{Ta có
		\begin{itemize}
			\item  Vì $\{1,2\} \subset X$ nên tập hợp $X$ có chứa các phần tử $1,2$.
			\item  Vì $X \subset \{1,2,5\}$ nên các phần tử của tập hợp $X$ có thể là $1,2,5$.
		\end{itemize}
		Khi đó tập hợp $X$ có thể là $\{1,2\}, \{1,2,5\}$.}
\end{bt}
\begin{bt}%[Huỳnh Quy]%[0D1B2-2]
	Xác định tập hợp $X$ biết $\{a,1\} \subset X \subset \{a,b,1,2\}$.
	\loigiai{Ta có
		\begin{itemize}
			\item  Vì $\{a,1\} \subset X$ nên tập hợp $X$ có chứa 2 phần tử là $a,1$.
			\item  Vì $X \subset \{a,b,1,2\}$ nên các phần tử của tập hợp $X$ có thể là $a,b,1,2$.
		\end{itemize}
		Suy ra, tập hợp $X$ có 2 phần tử, 3 phần tử hoặc 4 phần tử.\\
		Khi đó, tập hợp $X$ có thể là $\{a,1\}, \{a,1,2\}, \{a,b,1\}, \{a,b,2\}, \{a,b,1,2\}$.\\
		\underline{\textbf{Cách khác:}} $X=\{a;1\} \cup X'$ với $X' \subset \{b;2\}$. \\
		Vì có $4$ tập hợp $X'$ nên có $4$ tập hợp $X$ thỏa yêu cầu bài toán.}
\end{bt}
\begin{bt}%[Huỳnh Quy]%[0D1K2-2]
	Cho tập hợp $A=\{1;2;3;4;5;6\}$. Tìm tất cả các tập con có $3$ phần tử của tập hợp $A$ sao cho tổng các phần tử này là một số lẻ.
	\loigiai{
		Để tổng của ba số nguyên là một số lẻ thì trong ba số chỉ có một số lẻ hoặc cả ba số đều lẻ. Nói cách khác tập con này của $A$ phải có một số lẻ hoặc ba số lẻ.\\
		Chỉ có một tập con gồm ba số lẻ của $A$ là $\{1;3;5\}$. Các tập con gồm ba số của $A$ trong đó có một số lẻ là: \\
		$\{1;2;4\}$; $\{1;2;6\}$; $\{1;4;6\}$;$\{3;2;4\}$; $\{3;2;6\}$; $\{3;4;6\}$; $\{5;2;4\}$; $\{5;2;6\}$; $\{5;4;6\}$.\\
		\textit{\underline{Nhận xét:} Tổng các số nguyên là một số lẻ khi số số lẻ là số lẻ.}
	}
\end{bt}
\begin{bt}%[Huỳnh Quy]%[0D1K2-2]
	Cho $A=\{n\in\mathbb{N}\mid n \text{ là ước của }2\}$; $B=\{x\in\mathbb{R}\mid (x^2-1)(x-2)(x-4)=0 \}$. Tìm tất cả các tập hợp $X$ sao cho $A\subset X\subset B$.
	\loigiai{ 
		Liệt kê các phần tử của tập hợp $A$ và $B$ ta được : \\
		$A=\{1;2\}$; $B=\{-1;1;2;4\}$.\\
		Muốn tìm tập $X$ thỏa điều kiện $A\subset X\subset B$ đầu tiên ta lấy $X=A$, sau đó ghép thêm các phần tử thuộc $B$ mà không thuộc $A$. Với cách thực hiện như trên, ta có các tập hợp $X$ thỏa mãn yêu cầu bài toán là: $X=A=\{1;2 \}$, rồi ghép thêm vào một phần tử ta được: $\{-1;1;2 \}$;$\{4;1;2 \}$\\
		Ghép thêm vào $A$ hai trong bốn phần tử còn lại của $B$ ta được : $X=B=\{-1;1;2;4\}$
	}
\end{bt}
\begin{dang}{Các phép toán trên tập hợp}
\end{dang}
\subsubsection{Ví dụ minh hoạ}
\begin{vd}%[BG10-2022]%[Đỗ Văn Dự]%[0D1B2-2]
	Cho hai tập hợp:
	$C=\{n \in \mathbb{N} \mid n$ là bội chung của 2 và $3 ; n<30\}$;
	$D=\{n \in \mathbb{N} \mid n$ là bội của $6 ; n<30\}$.
	Chứng minh rằng $C=D$.
	\loigiai{
		Ta có: $C=\{0 ; 6 ; 12 ; 18 ; 24\}$.\\
		$D=\{0 ; 6 ; 12 ; 18 ; 24\}$.\\
		Vậy $C=D$.
	}
\end{vd}

\begin{vd}%[BG10-2022]%[Đỗ Văn Dự]%[0D1Y4-1]
	Viết các tập hợp sau dưới dạng các khoảng, đoạn, nửa khoảng trong $\mathbb{R}$ rồi biểu diễn trên trục số: $C=\{x \in \mathbb{R} \mid 2 \leq x \leq 7\}$; $D=\{x \in \mathbb{R} \mid x<2\}$.
	\loigiai{
		\immini{$C=[2;7]$}{\begin{tikzpicture}[scale=1,>=stealth, font=\footnotesize, line join=round, line cap=round]
				\draw [-stealth] (0,0)--(9,0);
				\path (2,0) node{$[$} (2,-0.1)node[below]{$2$}
				(7,0) node{$]$} (7,-0.1)node[below]{$7$};
				\foreach \x in{0,0.1,...,2} \draw (\x,0)--++(45:.07) (\x,0)--++(-135:.07);
				\foreach \x in{7.1,7.2,...,8.8} \draw (\x,0)--++(45:.07) (\x,0)--++(-135:.07);
		\end{tikzpicture}}
		\immini{$C=(-\infty;2)$}{\begin{tikzpicture}[scale=1,>=stealth, font=\footnotesize, line join=round, line cap=round]
				\draw [-stealth] (0,0)--(9,0);
				\path (2,0) node{$)$} (2,-0.1)node[below]{$2$};
				\foreach \x in{2,2.1,...,8.9} \draw (\x,0)--++(45:.07) (\x,0)--++(-135:.07);
		\end{tikzpicture}}	
		
	}
\end{vd}

\begin{vd}%[BG10-2022]%[Đỗ Văn Dự]%[0D1B4-1]
	\begin{enumerate}
		\item Cho hai tập hợp $C=\{4 ; 7 ; 27\}$ và $D=\{2 ; 4 ; 9 ; 27 ; 36\}$. Hãy xác định tập hợp $C \cap D$.
		\item Cho hai tập hợp $E=[1 ;+\infty)$ và $F=(-\infty ; 3]$. Hãy xác định tập hợp $E \cap F$.
	\end{enumerate}
	\loigiai{
		\begin{enumEX}{1}
			\item Giao của hai tập hợp $C$ và $D$ là $C \cap D=\{4 ; 27\}$.
			\item Giao của hai tập hợp $E$ và $F$ là $E \cap F=[1 ; 3]$.\\
			\begin{tikzpicture}[scale=1,>=stealth, font=\footnotesize, line join=round, line cap=round]
				\draw [-stealth] (-1,0)--(5,0);
				\path (1,0) node{$[$} (1,-0.1)node[below]{$1$} (-1.5,0)node[left]{$[1 ;+\infty)$};
				\foreach \x in{-0.9,-0.8,...,0.9} \draw (\x,0)--++(45:.08) (\x,0)--++(-135:.08);
			\end{tikzpicture}\\
			\begin{tikzpicture}[scale=1,>=stealth, font=\footnotesize, line join=round, line cap=round]
				\draw [-stealth] (-1,0)--(5,0);
				\path (3,0) node{$]$} (3,-0.1)node[below]{$3$} (-1.5,0)node[left]{$(-\infty ; 3]$};
				\foreach \x in{3.1,3.2,...,4.9} \draw (\x,0)--++(-45:.08) (\x,0)--++(135:.08);
			\end{tikzpicture}\\
			\begin{tikzpicture}[scale=1,>=stealth, font=\footnotesize, line join=round, line cap=round]
				\draw [-stealth] (-1,0)--(5,0);
				\path (3,0) node{$]$} (3,-0.1)node[below]{$3$} 
				(1,0) node{$[$} (1,-0.1)node[below]{$1$} (-1.5,0)node[left]{$[1 ;+\infty)\cap(-\infty ; 3]=[1;3]$}
				;
				\foreach \x in{3.1,3.2,...,4.9} \draw (\x,0)--++(-45:.08) (\x,0)--++(135:.08);
				\foreach \x in{-0.9,-0.8,...,0.9} \draw (\x,0)--++(45:.08) (\x,0)--++(-135:.08);
			\end{tikzpicture}
		\end{enumEX}
	}
\end{vd}
\begin{vd}%[BG10-2022]%[Đỗ Văn Dự]%[0D1B3-1]
	Cho hai tập hợp: $C=\{2 ; 3 ; 4 ; 7\}$; $D=\{-1 ; 2 ; 3 ; 4 ; 6\}$. Hãy xác định tập hợp $C \cup D$.
	\loigiai{
		Hợp của hai tập hợp $C$ và $D$ là $C \cup D=\{-1 ; 2 ; 3 ; 4 ; 6 ; 7\}$.
	}
\end{vd}

\begin{vd}%[BG10-2022]%[Đỗ Văn Dự]%[0D1B3-2]
	Cho các tập hợp: $D=\{-2 ; 3 ; 5 ; 6\}$; $E=\{x \mid x$ là số nguyên tố nhỏ hơn 10$\}$; $X=\{x \mid x$ là số nguyên dương nhỏ hơn 10$\}$.
	\begin{enumEX}{1}
		\item Tìm $D \backslash E$ và $E \backslash D$.
		\item $E$ có là tập con của $X$ không? Hãy tìm phần bù của $E$ trong $X$ (nếu có).
	\end{enumEX}
	\loigiai{
		\begin{enumEX}{1}
			\item Ta có: $E=\{2 ; 3 ; 5 ; 7\}$.\\
			Do đó, $D \backslash E=\{-2 ; 6\}$; $E \backslash D=\{2 ; 7\}$.
			\item Ta có: $X=\{1 ; 2 ; 3 ; 4 ; 5 ; 6 ; 7 ; 8 ; 9\}$. Vậy $E$ là tập con của $X$.\\
			Phần bù của $E$ trong $X$ là $X \backslash E=C_{X} E=\{1 ; 4 ; 6 ; 8 ; 9\}$.
		\end{enumEX}	
	}
\end{vd}

\begin{vd}%[BG10-2022]%[Đỗ Văn Dự]%[0D1B3-1]
	Cho hai tập hợp $A=\left\{0;1;2;3;4\right\}$ và $B=\left\{2;3;4;5;6\right\}$.
	\begin{enumEX}{1}
		\item Tìm các tập hợp $A\cup B, A\cap B, A\backslash B, B\backslash A$.
		\item  Tìm các tập $\left(A\backslash B\right)\cup \left(B\backslash A\right), \left(A\backslash B\right)\cap \left(B\backslash A\right)$.
	\end{enumEX}
	\loigiai{
	\begin{enumEX}{1}
			\item Ta có $A\backslash B=\left\{0;1\right\}$, $B\backslash A=\left\{5;6\right\}$, $A\cup B=\left\{0;1;2;3;4;5;6\right\}$, $A\cap B=\left\{2;3;4\right\}$.
			\item Ta có $\left(A\backslash B\right)\cup \left(B\backslash A\right)=\left\{0;1;5;6\right\}$, $\left(A\backslash B\right)\cap \left(B\backslash A\right)=\varnothing $.
	\end{enumEX}
	}
\end{vd}
\subsubsection{Bài tập tự luận}
\begin{bt}%[Huỳnh Quy]%[0D1B3-2]
	Cho hai tập hợp $A=\{ 1;2;3;4;5\}$ và $B=\{ 0;2;4\}$. Xác định $A\cap B$, $A\cup B$.
	\loigiai{
		Ta có $A\cap B=\{2;4\}$ và $A\cup B=\{0;1;2;3;4;5\}$.
	}
\end{bt}
\begin{bt}%[Huỳnh Quy]%[0D1B3-1]
	Cho hai tập hợp $A=\{1;2;3;5;7\}$ và  $B=\{n\in \mathbb{N} |\, n \text{ là ước số của } 12\}$. Tìm $A\cap B$  và $A\cup B$.
	\loigiai{
		Ta có: $B=\{1;2;3;4;6;12\}$.
		Vậy: $A\cap B=\{ 1;2;3\}$ và $A\cup B=\{ 1;2;3;4;5;6;7;12\}$.
	}
\end{bt}
\begin{bt}%[Huỳnh Quy]%[0D1B3-2]
	Cho hai tập hợp $A$ và $B$. Tìm $A\cap B, A\cup B$ biết
	\begin{enumerate}
		\item $A=\{x\mid x\ \text{là ước nguyên dương của 12} \}$ và 	$B=\{x\mid x\ \text{là ước nguyên dương của 18} \}$.
		\item $A=\{x\mid x\ \text{là ước nguyên dương của 27}\}$ và $B=\{x\mid x\ \text{là ước nguyên dương của 15} \}$.
	\end{enumerate}
	\loigiai{
		\begin{enumerate}
			\item $A=\{1;2;4;6;12\}$, $B=\{1;2;3;6;9;18\}$ $\Rightarrow \begin{cases} A\cap B=\{1;2;6\}\\
				A\cup B=\{1;2;3;4;6;9;12;18\} \end{cases}$
			\item $A=\{1;3;9;27\}$, $B=\{1;3;5;15\}$$\Rightarrow \begin{cases} A\cap B=\{1;3\}\\A\cup B=\{1;3;5;9;15;27\}\end{cases}$
		\end{enumerate}
	}
\end{bt}
\begin{bt}%[Huỳnh Quy]%[0D1B3-1]
	Cho $A$ là tập hợp học sinh lớp $12$ của trường Buôn Ma Thuột và $B$ là tập hợp học sinh của trường Buôn Ma Thuột dự kiến sẽ lựa chọn thi khối $A$ vào các trường đại học. Hãy mô tả các học sinh thuộc tập hợp sau
	\begin{enumEX}{2}
		\item $A\cap B$.
		\item $A\cup B$.
	\end{enumEX}
	\loigiai{
		\begin{enumerate}
			\item $A\cap B$ là tập hợp các học sinh lớp 12 thi khối $A$ của trường Buôn Ma Thuột.
			\item $A\cup B$ là tập hợp các học sinh hoặc lớp 12 hoặc học sinh chọn thi khối A của trường Buôn Ma Thuột. 
		\end{enumerate}
	}
\end{bt}
\begin{bt}%[Huỳnh Quy]%[0D1K3-1]
	Cho tập hợp $B=\{ x\in \mathbb{Z}|\, -4< x \le 4 \}$ và $C=\{ x\in \mathbb{Z}|\, x\le a\}$.
	Tìm số nguyên $a$ để tập hợp $B\cap C=\varnothing $.
	\loigiai{
		Ta có $B=\{-3;-2;-1;0;1;2;3;4\}$, $C=\{\ldots,a-1,a\}$.\\
		Để $B\cap C=\varnothing $ thì  $a\le -4$.
	}
\end{bt}
\begin{bt}%[Huỳnh Quy]%[0D1B3-1]
	Xác định tập hợp $A\cap B$ biết 
	$$A=\{x\in\mathbb{N}|\, x \text { là bội của }3 \}, \,\, B=\{x\in\mathbb{N}|\, x\text { là bội của }7\}.$$
	\loigiai{
		Ta có $A\cap B=\{x\in\mathbb{N}|\, x\text { là bội của }3 \text{ và bội của }7 \}= \{x\in\mathbb{N}|\, x\text {  là bội của  }21\} $.
	}
\end{bt}
\begin{bt}%[Huỳnh Quy]%[0D1B3-2]
	Cho $A$ là tập hợp các số tự nhiên chẵn không lớn hơn $10$, 
	$B=\left\{n\in \mathbb{N}|n\le 6\right\}$ và 
	$C=\left\{n\in \mathbb{N}|4\le n\le 10\right\}$. 
	Hãy tìm $A\cap (B\cup C)$.
	\loigiai{
		Ta có $A=\{0;2;4;6;8;10\}$; $B=\{0;1;2;3;4;5;6\}$ và $C=\{4;5;6;7;8;9;10\}$\\
		$B\cup C=\{0; 1; 2; 3; 4; 5; 6; 7; 8; 9; 10\}$ nên $A\cap (B\cup C)=\{0;2; 4; 6; 8; 10\}$.\\
		\underline{\textbf{Cách khác:}} Vì $B \cup C = \{n \in \mathbb{N} | n \ge 10\}$ nên $A \subset (B \cup C)$.\\
		Do đó $A \cap (B \cup C) = A = \{0;2;4;6;8;10\}$.
	}
\end{bt}
\begin{bt}%[Huỳnh Quy]%[0D1B3-1]
	Cho các tập hợp $A=\{x\in\mathbb{N}\mid x<8\}$ và $B=\{x\in\mathbb{Z}\mid  -3\leq x\leq 5\}$. Tìm $A\cap B$; $A\cup B$.
	\loigiai{
		Ta có $A=\{0;1;2;3;4;5;6;7\}$; $B=\{-3;-2;-1;0;1;2;3;4;5\}$.\\
		Vậy $A\cap B=\{0;1;2;3;4;5\}$ và $A\cup B=\{-3;-2;-1;0;1;2;3;4;5;6;7\}$.
	}
\end{bt}
\begin{bt}%[Huỳnh Quy]%[0D1K3-1]
	Cho các tập hợp $A=\{x \in \mathbb{Z}\big| |x-1|<4\}$, $B=\{x \in \mathbb{Z}\big| |x-1|>2\}$. Tìm $A \cap B$.
	\loigiai{
		Ta có $|x-1|<4 \Leftrightarrow -4<x-1<4 \Leftrightarrow -3<x<5$, $A=\{-2;-1;0;1;2;3;4\}$. \\
		Lại có $|x-1|>2 \Leftrightarrow x<-1\vee x>3$, $B=\{\ldots;-3;-2;4;5;6;\ldots\}$ nên $A \cap B=\{-2;4\}$.
	}
\end{bt}
\begin{bt}%[Huỳnh Quy]%[0D1G3-1]
	Cho các tập hợp $A=\{x\in\mathbb{Z}\,|\, 2m-1<x<2m+3\}$, $B=\{x\in\mathbb{Z}\,\big|\, |x|<2\}$. Tìm $m$ để $A\cap B=\varnothing$.
	\loigiai{
		Ta có $B=\{x\in\mathbb{Z}\,|\, -2<x<2\}=\{-1;0;1\}$ và $A=\{2m,\ldots,2m+2\}$.\\
		$A\cap B=\varnothing \Leftrightarrow \hoac{& 2m+2 \le -2 \\ & 2m \ge 2} \Leftrightarrow \hoac{& m \le -2 \\ & m \ge 1.}$
	}
\end{bt}
\begin{bt}%[Huỳnh Quy]%[0D1B4-1]
	Cho $A=\left[-2;4\right],B=\left(2;+\infty\right),C=(-\infty;3)$. Xác định các tập hợp sau đây và biểu diễn chúng trên trục số.
	\begin{enumerate}
		\item $A\cap B, B\cap C$.
		\item $\mathbb{R}\cap A,\mathbb{R}\cap B$.
	\end{enumerate}
	\loigiai{
		\begin{tikzpicture}[scale=1]
			\draw[->,thick](-4,0) --(6,0);
		\end{tikzpicture}\\
		%
		\begin{tikzpicture}[scale=1]
			\draw[->,thick](-4,0) --(6,0) node at (-2,0.7){$-2$} node at (4,0.7){$4$} node at (-2,0) {$\Big [$} node at (4,0) {$\Big ]$}node at (6.7,0){$A$};
			\foreach \b in {-3.8,-3.6,...,-2.2} {\draw[ thick](\b-0.1,-0.2)--(\b+0.1,0.2);}
			\foreach \b in {4.2,4.4,...,5.8} {\draw[ thick](\b-0.1,-0.2)--(\b+0.1,0.2);}
		\end{tikzpicture}\\
		%
		\begin{tikzpicture}[scale=1]
			\draw[->,thick](-4,0) --(6,0) node at (2,0.7){$2$} node at (2,0) {$\Big ($}node at (6.7,0){$B$};
			\foreach \b in {-3.8,-3.6,...,1.8} {\draw[ thick](\b-0.1,-0.2)--(\b+0.1,0.2);}
		\end{tikzpicture}\\
		%
		\begin{tikzpicture}[scale=1]
			\draw[->,thick](-4,0) --(6,0) node at (3,0.7){$3$} node at (3,0) {$\Big )$}node at (6.7,0){$C$};
			\foreach \b in {3.2,3.4,...,5.8} {\draw[ thick](\b-0.1,-0.2)--(\b+0.1,0.2);}
		\end{tikzpicture}\\
		%
		\begin{enumerate}
			\item $A\cap B=\left(2;4\right], B\cap C=\left(2;3\right)$.
			\item $\mathbb{R}\cap A=\left[-2;4 \right],\mathbb{R}\cap B=\left(2;+\infty \right)$.
		\end{enumerate}
	}
\end{bt}
\begin{bt}%[Huỳnh Quy]%[0D1B4-2]
	Cho hai tập hợp $A=\lbrace x\in \mathbb{R}\vert  x \leq 2 \rbrace$, $B=\lbrace x\in \mathbb{R}\vert -2< x \rbrace$. Tìm $A\setminus B, B\setminus A$.
	\loigiai{
		\begin{tikzpicture}[scale=1]
			\draw[->,thick](-4,0) --(6,0) node at (2,0.7){$2$} node at (2,0) {$\Big ]$}node at (6.7,0){$A$};
			\foreach \b in {2.2,2.4,...,5.8} {\draw[ thick](\b-0.1,-0.2)--(\b+0.1,0.2);}
		\end{tikzpicture}\\
		%
		\begin{tikzpicture}[scale=1]
			\draw[->,thick](-4,0) --(6,0) node at (-2,0.7){$-2$} node at (-2,0) {$\Big ($} node at (6.7,0){$B$};
			\foreach \b in {-3.8,-3.6,...,-2.2} {\draw[ thick](\b-0.1,-0.2)--(\b+0.1,0.2);}
		\end{tikzpicture}\\
		$\Rightarrow A\setminus B=\left(-\infty;-2 \right], B\setminus A=\left(2;+\infty \right)$.}
\end{bt}
\begin{bt}%[Huỳnh Quy]%[0D1B4-2] 
	Cho $A=\left[-2;4\right],B=\left(2;+\infty\right),C=(-\infty;3)$. Xác định các tập hợp sau đây và biểu diễn chúng trên trục số.
	\begin{enumerate}
		\item $A\setminus B, B\setminus A$.
		\item $\mathbb{R}\setminus A,\mathbb{R}\setminus B, \mathbb{R}\setminus  C$.
	\end{enumerate}
	\loigiai{
		\begin{tikzpicture}[scale=1]
			\draw[->,thick](-4,0) --(6,0);
		\end{tikzpicture}\\
		%
		\begin{tikzpicture}[scale=1]
			\draw[->,thick](-4,0) --(6,0) node at (-2,0.7){$-2$} node at (4,0.7){$4$} node at (-2,0) {$\Big [$} node at (4,0) {$\Big ]$}node at (6.7,0){$A$};
			\foreach \b in {-3.8,-3.6,...,-2.2} {\draw[ thick](\b-0.1,-0.2)--(\b+0.1,0.2);}
			\foreach \b in {4.2,4.4,...,5.8} {\draw[ thick](\b-0.1,-0.2)--(\b+0.1,0.2);}
		\end{tikzpicture}\\
		%
		\begin{tikzpicture}[scale=1]
			\draw[->,thick](-4,0) --(6,0) node at (2,0.7){$2$} node at (2,0) {$\Big ($}node at (6.7,0){$B$};
			\foreach \b in {-3.8,-3.6,...,1.8} {\draw[ thick](\b-0.1,-0.2)--(\b+0.1,0.2);}
		\end{tikzpicture}\\
		%
		\begin{tikzpicture}[scale=1]
			\draw[->,thick](-4,0) --(6,0) node at (3,0.7){$3$} node at (3,0) {$\Big )$}node at (6.7,0){$C$};
			\foreach \b in {3.2,3.4,...,5.8} {\draw[ thick](\b-0.1,-0.2)--(\b+0.1,0.2);}
		\end{tikzpicture}
		%
		\begin{enumerate}
			\item $A\setminus B=\left[-2;2\right], B\setminus A=\left(4;+\infty\right)$.
			\item $\mathbb{R}\setminus A=\left(\infty;-2\right)\cup\left(4;+\infty \right),\mathbb{R}\setminus B=\left(-\infty;2\right], \mathbb{R}\setminus  C=\left[3;+\infty \right)$.
		\end{enumerate}
	}
\end{bt}
\begin{dang}{Ứng dụng thực tế các phép toán tập hợp}
\end{dang}
\subsubsection{Ví dụ minh hoạ}
\begin{vd}%[BG10-2022]%[Đỗ Văn Dự]%[0D1Y3-3]
	Cho $A$ là tập hợp các học sinh giỏi Toán của trường THPT X và $B$ là tập hợp học sinh giỏi Văn của trường này. Hãy mô tả các học sinh thuộc tập hợp sau\begin{enumEX}{3} 
			\item $A\cup B$.
			\item $A\cap B$.
			\item $A\setminus B$.
			\item $B\setminus A$.
			\item $\left(A\cup B\right)\setminus \left(A\cap B\right)$.
	\end{enumEX}
	\loigiai{
		\immini{
			\begin{enumerate}
				\item $A\cup B$ là tập hợp các học sinh giỏi Toán hoặc giỏi Văn của trường.
				\item $A\cap B$ là tập hợp các học sinh giỏi cả hai môn Toán và Văn của trường.
				\item $A\setminus B$ là tập hợp các học sinh chỉ giỏi Toán, không giỏi Văn.
				\item $B\setminus A$ là tập hợp các học sinh chỉ giỏi Văn, không giỏi Toán.		
				\item $\left(A\cup B\right)\setminus \left(A\cap B\right)$ là tập hợp các học sinh chỉ giỏi Toán hoặc giỏi Văn của trường.	
			\end{enumerate}
		}
		{\begin{tikzpicture}[scale=1, line join=round, line cap=round]	
				\coordinate (A) at (-1.2,0);
				\coordinate (B) at (0.8,0);
				\begin{scope}
					\clip[rotate=20] (A) ellipse (2cm and 1cm); %cat theo elip A
					\fill[rotate=30,pattern=north west lines] (B) ellipse (2cm and 1cm); %to elip B nhung bi cat mat theo phan giao voi elip A
				\end{scope}
				\draw[rotate=20] (A) ellipse (2cm and 1cm) node[fill=white,above]{$A\setminus B$}; %ve lai elip A
				\draw[rotate=30] (B) ellipse (2cm and 1cm) node[fill=white,above,right]{$B\setminus A$}; %ve lai elip B
				\path (-0.2,0.2) node[below]{$A\cap B$};
		\end{tikzpicture}}
	}
\end{vd}
\begin{vd}%[BG10-2022]%[Đỗ Văn Dự]%[0D1B3-3]
	Trong kì thi học sinh giỏi cấp trường, lớp 10C1 có $45$ học sinh trong đó có  $17$ bạn đạt học sinh giỏi Văn, $25$ bạn đạt học sinh giỏi Toán và $13$ bạn học sinh không đạt học sinh giỏi. Tìm số học sinh giỏi cả Văn và Toán của lớp 10C1.
	\loigiai{
		\immini{
			\begin{itemize}			
				\item Gọi $A$, $B$ theo thứ tự là tập hợp các học sinh giỏi Văn và giỏi Toán của lớp. 
				Theo đề ta có $n(A)=17$, $n(B)=25$, $n(A \cup B)= 45-13=32$.
				\item Số học sinh giỏi cả Văn và Toán là $$n(A \cap B )=n(A) + n(B) - n(A \cup B)=25+17-32=10.$$
			\end{itemize}
		}
		{	\begin{tikzpicture}[scale=.8]
				\def\radius{2cm}	
				\coordinate (ceni);
				\coordinate[xshift=.9*\radius] (cenii);
				
				\draw (ceni) circle (.9*\radius);
				\draw (cenii) circle (\radius);
				\draw  ([xshift=-25pt,yshift=15pt]current bounding box.north west) 
				rectangle ([xshift=25pt,yshift=-15]current bounding box.south east);
				
				\node[xshift=-.3*\radius] at (ceni) {$n(B)=15$};
				\node[xshift=.3*\radius] at (cenii) {$n(A)=35$};
				\node[xshift=.4*\radius] at (ceni) {$5$};
	\end{tikzpicture}}}
\end{vd}
\begin{vd}%[BG10-2022]%[Đỗ Văn Dự]%[0D1B3-3]
	Một lớp học có $ 50 $ học sinh trong đó có $ 30 $ em biết chơi bóng chuyền, $25$ em biết chơi bóng đá, $ 10 $ em biết chơi cả bóng đá và bóng chuyền. Hỏi có bao nhiêu em không biết chơi môn nào trong hai môn ở trên?
	\loigiai{
		Gọi tập $A$ là tập hợp các học sinh biết chơi bóng chuyền.
		\\Tập $B$ là tập hợp các học sinh biết chơi bóng đá.
		\\Khi đó số học sinh biết chơi ít nhất một trong hai môn bóng chuyền hoặc bóng đá là 
		$$n(A \cup B)=n(A)+n(B)-n(A \cap B)=30+25-10=45.$$
		Vậy số học sinh không biết chơi môn nào là $50-45=5$. 		
	}	
\end{vd}
\begin{vd}%[BG10-2022]%[Đỗ Văn Dự]%[0D1B3-3]
	Trong số $45$ cán bộ được triệu tập để chuẩn bị công tác cho một cuộc hội nghị quốc tế có $25$ cán bộ phiên dịch tiếng Anh, $15$ cán bộ phiên dịch tiếng Pháp, trong đó có $10$ cán bộ vừa phiên dịch được tiếng Anh, vừa phiên dịch được tiếng Pháp. Hỏi
	\begin{enumerate}
		\item Nhóm có bao nhiêu cán bộ được cấp thẻ đỏ, biết rằng muốn được cấp thẻ đỏ cán bộ đó phải phiên dịch được tiếng Anh hoặc phiên dịch được tiếng Pháp.
		\item Nhóm có bao nhiêu cán bộ không phiên dịch được tiếng Anh và không phiên dịch được tiếng Pháp.
	\end{enumerate}
	\loigiai{
		Gọi $A$, $B$ theo thứ tự là tập hợp các cán bộ phiên dịch tiếng Anh và tập hợp các các bộ phiên dịch tiếng Pháp. 
		Theo đề ta có $n(A)=25$, $n(B)=15$, $n(A \cap B)= 10$.
		\begin{enumerate}				
			\item Tập hợp các cán bộ được cấp thẻ đỏ là $A\cup B$.\\
			Vậy số cán bộ được cấp thẻ đỏ là $n(A \cup B)=n(A)+n(B)-n(A \cap B)=25+15-10=30$.
			\item Tập hợp các cán bộ của nhóm không phiên dịch được tiếng Anh và tiếng Pháp chính là số cán bộ không được cấp thẻ đỏ.\\
			Vậy số cán bộ đó là $45-30=15$.
	\end{enumerate}}
\end{vd}
\begin{vd}%[BG10-2022]%[Đỗ Văn Dự]%[0D1B3-3]
	Lớp $10A$ có $15$ bạn thích môn Văn, $20$ bạn thích môn Toán. Trong số các bạn thích văn hoặc toán có $8$ bạn thích cả $2$ môn. Trong lớp vẫn còn $10$ bạn không thích môn nào trong $2$ môn Văn và Toán. Hỏi lớp $10A$ có bao nhiêu học sinh?
	\loigiai{Ta sử dụng sơ đồ Ven.		
		\begin{center}
			\begin{tikzpicture}
				\draw[] (0.7,0) circle ( 2.5cm);
				\draw[] (0,0) circle ( 1.5cm);
				\draw (1.5,0) circle ( 1.5 cm);
				\draw (-0.5,0) node {$7$};
				\draw (2,0) node {$12$};
				\draw (0.5,0) node {$8$};
				\draw (0.5,1.8) node {$10$};
			\end{tikzpicture}
		\end{center}
		\begin{itemize}
			\item Hình tròn lớn ngoài cùng thể hiện số học sinh cả lớp.\\
			Như vậy, ta có:\\
			\item	Số bạn chỉ thích Văn là $ 15-8=7$(bạn).\\
			\item	Số bạn chỉ thích Toán là $20-8=12$(bạn).\\
			\item	Số học sinh cả lớp là tổng các phần không giao nhau: $7+8+12+10=37$.
		\end{itemize}
	}
\end{vd}

\begin{vd}%[BG10-2022]%[Đỗ Văn Dự]%[0D1B3-3]
	Một lớp có $40$ học sinh, mỗi học sinh đều đăng ký chơi ít nhất $1$ trong $2$ môn thể thao là bóng đá hoặc cầu lông. Có $30$ học sinh có đăng ký môn bóng đá, $25$ học sinh có đăng ký môn cầu lông. Hỏi có bao nhiêu em đăng ký cả $2$ môn.
	\loigiai{
		Gọi $A$ là tập hợp các học sinh đăng kí chơi bóng đá, $B$ là tập học sinh đăng kí chơi cầu lông thì $A\cap B$ là tập hợp các học sinh đăng kí chơi cả hai môn.\\	
		Vậy số học sinh đăng kí chơi cả hai môn là 
		$n(A \cap B)=n(A)+n(B)-n(A \cup B)=30+25-40=15$. }
\end{vd}

\begin{vd}%[BG10-2022]%[Đỗ Văn Dự]%[0D1B3-3]
	Ở xứ sở của thần Thoại ngoài các vị thần thì còn có các sinh vật gồm $27$ con người, $311$ con yêu quái một mắt, $205$ con yêu quái tóc rắn và yêu quái vừa một mắt vừa tóc rắn. Tìm số yêu quái vừa một mắt vừa tóc rắn biết có tổng số sinh vật là 500 con.
	\loigiai{
		\begin{itemize}
			\item Số sinh vật không phải con người là $500-27=473$ (con).
			\item Gọi $A$, $B$ lần lượt là tập hợp yêu quái một mắt và yêu quái tóc rắn. Khi đó $n(A)=311$,  $n(B)=205$, $n(A \cup B)=473$.
			\item Vậy số yêu quái vừa một mắt vừa tóc rắn là $|A \cap B| =311+205-473=43$.
		\end{itemize}
	}
\end{vd}

\begin{vd}%[BG10-2022]%[Đỗ Văn Dự]%[0D1B3-3]
	Mỗi học sinh của lớp $10A$ đều chơi bóng đá hoặc bóng chuyền. Biết rằng có $25$ bạn chơi bóng đá, $20$ bạn chơi bóng chuyền và $10$ bạn chơi cả $2$ môn thể thao. Hỏi lớp $10A$ có bao nhiêu học sinh.
	\loigiai{
		Gọi $A$ là tập hợp các học sinh chơi bóng đá, $B$ là tập các học sinh chơi bóng chuyền. Do đó $A\cap B$ là tập các học sinh chơi cả hai môn.\\
		Theo đề $n(A)=25$, $n(B)=20$, $|A \cap B| =10$.\\
		Vậy số học sinh cả lớp là $|A \cup B| =n(A)+n(B)-n(A \cap B)=25+20-10=35$.}
\end{vd}

\begin{vd}%[BG10-2022]%[Đỗ Văn Dự]%[0D1B3-3]
	Lớp 10A có $45$ học sinh, có $15$ học sinh giỏi và $20$ học sinh xếp hạnh kiểm tốt, trong đó có $10$ bạn vừa học giỏi vừa xếp hạnh kiểm tốt. Các học sinh được học sinh giỏi hoặc hạnh kiểm tốt đều được khen thưởng. Số học sinh được khen thưởng của lớp 10A là là bao nhiêu?
	\loigiai{
		Gọi $A$ là tập hợp các học sinh giỏi, 
		$B$ là tập hợp các học sinh xếp hạnh kiểm tốt.\\
		Khi đó số học sinh được khen thưởng là $n(A \cup B)$.\\
		Vậy số học sinh được khen thưởng là 
		$n(A \cup B)=n(A)+n(B)-n(A \cap B)=15+20-10=25.$		
	}
\end{vd}

\begin{vd}%BT6%[Nguyễn Thành Tuấn]%[0D1K3-3]
	Kết quả thi học kì một của một trường THPT có $48$ thí sinh giỏi môn Toán, $37$ thí sinh giỏi môn Vật Lí,$42$ thí sinh giỏi môn Văn. Biết rằng có $75$ thí sinh giỏi môn Toán hoặc môn Vật lí, $76$ thí sinh giỏi  môn Toán hoặc môn Văn, $66$ thí sinh giỏi môn Vật lí hoặc môn Văn và có $4$ thí sinh giỏi cả ba môn. Hỏi 
	\begin{enumerate}
		\item có bao nhiêu học sinh chỉ giỏi 1 môn.
		\item có bao nhiêu học sinh chỉ giỏi 2 môn.
		\item có bao nhiêu học sinh giỏi ít nhất 1 môn.
	\end{enumerate}
	\loigiai{		
		Gọi $A$, $B$, $C$ theo thứ tự là tập hợp các học sinh giỏi Toán, giỏi Lí và giỏi Văn. Theo đề ta có
		\begin{itemize}
			\item Số học sinh giỏi Toán và Lí là $$n(A \cap B)=n(A)+n(B)-n(A \cup B)=48+37-75=10.$$	
			\item Số học sinh giỏi Toán và Văn là $$n(A \cap C)=n(A)+n(C)-n(A\cup C)=48+42 - 76 = 14 .$$	
			\item Số học sinh giỏi Lí và Văn là $$n(B \cap C)=n(B)+n(C)-n(B\cup C)= 42+37-66=13 .$$	
			\item Số học sinh chỉ giỏi môn Toán $48-10-14+4=28 $.	
			\item Số học sinh chỉ giỏi môn Lí $37-10-13+4=18 $.	
			\item Số học sinh chỉ giỏi môn Văn $42-13-14+4=19 $.
		\end{itemize}
		\begin{center}
			\begin{tikzpicture}[scale=0.6,>=stealth, font=\footnotesize, line join=round, line cap=round]	
				\def\firstcircle{(0,0) circle (2.1cm)}
				\def\secondcircle{(45:2.5cm) circle (1.95cm)}
				\def\thirdcircle{(-15:1.7cm) circle (1.4cm)}
				\colorlet{circle edge}{black!50}
				\colorlet{circle area}{black!20}
				\tikzset{filled/.style={fill=circle area, draw=circle edge, thick},
					outline/.style={draw=circle edge, thick}}
				\draw \firstcircle;
				\draw \secondcircle;
				\draw \thirdcircle;
				\begin{scope}
					\clip \firstcircle;
					\fill[filled] \secondcircle;
				\end{scope}
				\begin{scope}
					\clip \firstcircle;
					\fill[filled] \thirdcircle;
				\end{scope}
				\begin{scope}
					\clip \secondcircle;
					\fill[filled] \thirdcircle;
				\end{scope}  
				\node at (-1,0){48};  
				%\node at (0.5,1){3}; 
				%	\node at (1.2,0.3){1};  
				%	\node at (2.3,0.3){1};      
				%	\node at (1,-0.5){2};   
				\node at (2,-1.2){42};     
				\node at (2.5,2.5){37}; 
				\draw[outline] \firstcircle
				\secondcircle  \thirdcircle;    
				\node at (-2,4) {\textbf{Toán}};
				\node at (2,6) {\textbf{Toán và Lý}};
				\node at (5,5) {\textbf{Lý}};
				\node at (7,1) {\textbf{Văn và Lý}};   
				\node at (4.6,-3) {\textbf{Văn}};
				\node at (-2,-3) {\textbf{Toán và Văn}};
				\draw[dashed] (-2,3.5) -- (-1,1.2) 
				(2,5.5) -- (0.7,0.3) 
				(2,5.5) -- (0.2,0.9)
				(5,4.5) -- (3,2.8)
				(5,0.9) -- (1.2,0.5)
				(5,0.9) -- (2.4,0.2)
				(4,-2.9) -- (2.1,-1.4)
				(-2,-2.5) -- (1.2,0.3)
				(-2,-2.5) -- (0.9,-0.9)	;
		\end{tikzpicture}	
		\end{center}
		
		\begin{enumerate}	
			\item Số học sinh chỉ giỏi đúng 1 môn là $28+18+19=65$. 
			\item Số học sinh chỉ giỏi đúng 2 môn là $10+14+13-4\cdot2=25 $. 
			\item Số học sinh giỏi ít nhất một môn là $65+25+4=94$.
		\end{enumerate}
	}
\end{vd}
\begin{vd}%[BG10-2022]%[Đỗ Văn Dự]%[0D1K3-3]
	Một nhóm học sinh giỏi các bộ môn: Anh, Toán, Văn. Có $18$ em giỏi Văn, $10$ em giỏi Anh, $12$ em giỏi Toán, $3$ em giỏi Văn và Toán, $4$ em giỏi Toán và Anh, $5$ em giỏi Văn và Anh, $2$ em giỏi cả ba môn. Hỏi nhóm đó có bao nhiêu em?
	\loigiai{
		Gọi $A$ là tập hợp những học sinh giỏi Anh,
		$T$ là tập hợp những học sinh giỏi toán,
		$V$ là tập hợp những học sinh giỏi Văn.
		Ta có\\
		$\bullet$ $n\left( V \right)=18,\; n\left( A \right)=10$,\; $n\left( T \right)=12$,\; 
		$n(V\cap T)=3,\;n(T\cap A)=4,\;n(V\cap A)=5$,\; $n(A\cap B\cap C)=2$.\\
		$\bullet$  $\begin{aligned}[t] n(V\cup A\cup T)&=n\left( V \right)+n\left( A \right)+n\left( T \right)-\left[ n(V\cap A)+n(A\cap T)+n(T\cap V) \right]+n\left( V\cap A\cap T \right)\\ &=
		18+10+12-\left[ 3+4+5 \right]+2=30 \end{aligned}$.\\
		Vậy nhóm đó có 30 em.}
\end{vd}

\begin{vd}%[BG10-2022]%[Đỗ Văn Dự]%[0D1K3-3]
	Trong số $42$ học sinh của lớp 10A có $13$ bạn được xếp loại học lực giỏi, $22$ bạn được xếp loại hạnh kiểm tốt, trong đó $7$ bạn vừa học lực giỏi, vừa có hạnh kiểm tốt. Hỏi lớp 10A có bao nhiêu bạn được khen thưởng? Biết rằng muốn được khen thưởng thì bạn đó phải có học lực giỏi hoặc có hạnh kiểm tốt.
	\loigiai{
		Gọi tập hợp các học sinh học lực giỏi là $G$, tập hợp các bạn học sinh hạnh kiểm tốt là $T$. Khi đó tập hợp các bạn học sinh vừa có học lực giỏi là, vừa có hạnh kiểm tốt là $G\cap T$, tập hợp các bạn học sinh đạt học lực giỏi hoặc hạnh kiểm tốt là $G\cup T$. Ta có \\
		$n(G)=13$, $n(T)=22$, $n(G\cap T)=7$.\\
		$n(G\cup T)=n(G)+n(T)-n(G\cap T)=13+22-7=28.$}
\end{vd}

\begin{vd}%[BG10-2022]%[Đỗ Văn Dự]%[0D1K3-3]
	Một nhóm học sinh giỏi các bộ môn: Anh, Toán, Văn. Có $ 18 $ em giỏi Văn, $ 10 $ em giỏi Anh, $ 12 $ em giỏi Toán, $ 3 $ em giỏi Văn và Toán, $ 4 $ em giỏi Toán và Anh, $ 5 $ em giỏi Văn và Anh, $ 2 $ em giỏi cả ba môn. Hỏi nhóm đó có bao nhiêu em?
	\loigiai{
		\begin{center}
			\begin{tikzpicture}[scale=.8]
				\def\radius{2cm}	
				\coordinate (ceni);
				\coordinate[xshift=.8*\radius, yshift=.2*\radius] (cenii);
				\coordinate[xshift=.6*\radius,yshift=-.8*\radius] (ceniii);
				\draw (ceni) circle (.9*\radius);
				\draw (cenii) circle (\radius);
				\draw (ceniii) circle (1.11*\radius);
				
				\node[xshift=-.3*\radius, yshift=.05*\radius] at (ceni) {$8$(V)};
				\node[xshift=.35*\radius, yshift=.25*\radius] at (ceni) {$3$(VT)};
				\node[xshift=.1*\radius, yshift=-.55*\radius] at (ceni) {$5$(VA)};
				\node[xshift=.4*\radius, yshift=-.15*\radius] at (ceni) {{\footnotesize $2$(TVA)}};
				\node[xshift=.2*\radius, yshift=-.55*\radius] at (cenii) {$4$(TA)};
				\node[xshift=1.1*\radius,yshift=.3*\radius] at (ceni) {$12$(T)};
				\node[yshift=-.4*\radius] at (ceniii) {$10$(A)};
			\end{tikzpicture}
		\end{center}
		Ký hiệu $ A $ là tập hợp những học sinh giỏi Anh,\\
		$ T $ là tập hợp những học sinh giỏi Toán,\\
		$ V $ là tập hợp những học sinh giỏi Văn.\\
		$\bullet$ $n(V)=18,\; n(A)=10,\; n(T)=12$,\\
		$\bullet$ $n(T \cap V)=3,\; n(T \cap A)=4,\; n(V \cap A)=5, n(A \cap B \cap C)=2$.\\
		Số học sinh của nhóm là
		\begin{eqnarray*}
			n(V \cup A \cup T)&=&n(V)+n(A)+n(T)-n(V \cap A)-n(T \cap A)-n(T \cap V)+n(A \cap B \cap C)\\
			&=&18+10+12-(3+4+5)+2=30.
		\end{eqnarray*}
		Vậy nhóm đó có $ 30 $ em.
	}
\end{vd}

\begin{vd}%[BG10-2022]%[Đỗ Văn Dự]%[0D1K3-3]
	Có $44$ học sinh giỏi, mỗi em giỏi ít nhất một môn. Có $22$ em giỏi Văn, $25$ em giỏi Toán, $20$ em giỏi Anh. Có $8$ em giỏi đúng hai môn Văn, Toán; Có $7$ em giỏi đúng hai môn Toán, Anh; Có $6$ em giỏi đúng hai môn Anh, Văn. Hỏi có bao nhiêu em giỏi cả ba môn Văn, Toán, Anh?
	\loigiai{
		Ta có \\
		$n\left( V \right)=22,\;n\left( T \right)=25$,\; $n\left( A \right)=20$\\
		$n((V\cap T)\setminus A)=8,\;n((T\cap A)\setminus V)=7,\;n((V\cap A)\setminus T)=6,\; n(V\cup T\cup A)=44$.\\
		$n(V\cup A\cup T)=n\left( V \right)+n\left( A \right)+n\left( T \right)-n(V\cap A)-n(A\cap T)-n(T\cap V)+n\left( V\cap A\cap T \right)$\\
		$  \Leftrightarrow  44=22+20+25-6-7-8+4n\left( V\cap A\cap T \right)$.\\
		$\Rightarrow n\left( V\cap A\cap T \right)=1$.
	}
\end{vd}
\begin{vd}%[BG10-2022]%[Đỗ Văn Dự]%[0D1B3-3]
	Để thành lập đội tuyển học sinh giỏi khối $ 10 $, nhà trường tổ chức thi chọn các môn Toán, Văn, Anh trên tổng số $ 111 $ học sinh. Kết quả có: $ 70 $ học sinh giỏi Toán, $ 65 $ học sinh giỏi Văn, $ 62 $ học sinh giỏi Anh. Trong đó có $ 49 $ học sinh giỏi cả hai môn Văn và Toán, $ 32 $ học sinh giỏi cả hai môn Toán và Anh, $ 34 $ học sinh giỏi cả hai môn Văn và Anh. Xác định số học sinh giỏi cả ba môn Văn, Toán, Anh. Biết rằng có $ 6 $ học sinh không đạt yêu cầu cả ba môn.
	\loigiai{
		\immini
		{Có $ 111-6=105 $ học sinh thi đạt ít nhất $ 1 $ môn.\\
			Gọi $ A $ là số học sinh giỏi môn Toán và Tiếng Anh nhưng không giỏi Văn.\\
			Gọi $ B $ là số học sinh giỏi môn Toán và Văn nhưng không giỏi Tiếng Anh.\\
			Gọi $ C $ là số học sinh giỏi môn Văn và Tiếng Anh nhưng không giỏi Toán.\\
			Gọi $ D $ là số học sinh giỏi cả ba môn. Ta có hệ:\\
			$\begin{cases}
				B+D=49\\
				A+D=32\\
				C+D=34\\
				70+65+62-(A+B+C+2D)=105
			\end{cases}\\
			\Rightarrow 92=32-D+49-D+34-D+2D\\
			\Rightarrow D=23$.\\
			Vậy có $ 23 $ học sinh giỏi cả ba môn.
		}
		{
			\begin{tikzpicture}[scale=.8]
				\def\radius{2cm}	
				\coordinate (ceni);
				\coordinate[xshift=.8*\radius, yshift=.2*\radius] (cenii);
				\coordinate[xshift=.6*\radius,yshift=-.8*\radius] (ceniii);
				\draw (ceni) circle (.9*\radius);
				\draw (cenii) circle (\radius);
				\draw (ceniii) circle (\radius);
				
				\node[xshift=.3*\radius, yshift=.25*\radius] at (ceni) {A};
				\node[xshift=.1*\radius, yshift=-.5*\radius] at (ceni) {B};
				\node[xshift=.45*\radius, yshift=-.25*\radius] at (ceni) {D};
				\node[xshift=.1*\radius, yshift=-.55*\radius] at (cenii) {C};
		\end{tikzpicture}}
	}
\end{vd}
\subsubsection{Bài tập tự luận}
\begin{bt}%[Huỳnh Quy]%[0D1B3-3]
	Mỗi học sinh của lớp $10A$ đều chơi bóng đá hoặc bóng chuyền. Biết rằng có $25$ bạn chơi bóng đá, $20$ bạn chơi bóng chuyền và $10$ bạn chơi cả $2$ môn thể thao. Hỏi lớp $10A$ có bao nhiêu học sinh.
	\loigiai{
		Ngoài sơ đồ Ven ta có thể dùng công thức số phần tử. Gọi $A$ là tập hợp các học sinh chơi bóng đá, $B$ là tập các học sinh chơi bóng chuyền. Do đó $A\cap B$ là tập các học sinh chơi cả hai môn. Ta có
		$$n(A)=25, n(B)=20, n(A \cap B) =10.$$
		Số học sinh cả lớp là số phần tử của tập $A \cup B$ nên $n(A \cup B) = 25+20-10=35$ (học sinh).
	}
\end{bt}
\begin{bt}%[Huỳnh Quy]%[0D1B3-3]
	Lớp $10{{B}_{1}}$ có $7$ học sinh giỏi Toán, $5$ học sinh giỏi Lý, $6$ học sinh giỏi Hóa, $3$ học sinh giỏi cả Toán và Lý, $4$ học sinh giỏi cả Toán và Hóa, $2$ học sinh giỏi cả Lý và Hóa, $1$ học sinh giỏi cả $3$ môn Toán, Lý, Hóa. Tính số học sinh giỏi ít nhất một môn (Toán, Lý, Hóa) của lớp $10{{B}_{1}}$.
	\loigiai{
		Ta dùng biểu đồ Ven để giải:
		\begin{center}
			\begin{tikzpicture}
				\def\firstcircle{(0,0) circle (1.5cm)}
				\def\secondcircle{(45:2.5cm) circle (2cm)}
				\def\thirdcircle{(-15:1.7cm) circle (1.4cm)}
				\colorlet{circle edge}{black!50}
				\colorlet{circle area}{black!20}
				\tikzset{filled/.style={fill=circle area, draw=circle edge, thick},
					outline/.style={draw=circle edge, thick}}
				\draw \firstcircle;
				\draw \secondcircle;
				\draw \thirdcircle;
				\begin{scope}
					\clip \firstcircle;
					\fill[filled] \secondcircle;
				\end{scope}
				\begin{scope}
					\clip \firstcircle;
					\fill[filled] \thirdcircle;
				\end{scope}
				\begin{scope}
					\clip \secondcircle;
					\fill[filled] \thirdcircle;
				\end{scope}  
				\node at (-1,0){1};  
				\node at (0.5,1){2}; 
				\node at (1,0.3){1};  
				\node at (2,0.3){1};      
				\node at (1,-0.5){3};   
				\node at (2,-1){1};     
				\node at (3,2.5){1}; 
				\draw[outline] \firstcircle
				\secondcircle  \thirdcircle;    
				\node at (-2,4) {\textbf{Toán}};
				\node at (2,6) {\textbf{Giỏi Toán + Lý}};
				\node at (5,5) {\textbf{Lý}};
				\node at (7,1) {\textbf{Giỏi Hóa + Lý}};   
				\node at (5,-3) {\textbf{Hóa}};
				\node at (-2,-3) {\textbf{Giỏi Toán + Hóa}};
				\draw[->] (-2,3.5) -- (-1,1.2) 
				(2,5.5) -- (0.7,0.3) 
				(2,5.5) -- (0.2,0.9)
				(5,4.5) -- (3.6,2.5)
				(5,0.9) -- (1.2,0.5)
				(5,0.9) -- (2.4,0.2)
				(4,-2.9) -- (2,-1.8)
				(-2,-2.5) -- (1.2,0)
				(-2,-2.5) -- (0.9,-0.9)
				;
			\end{tikzpicture}
		\end{center}
		Nhìn vào biểu đồ, số học sinh giỏi ít nhất $1$ trong $3$ môn là: $1+2+1+3+1+1+1=10$}
\end{bt}
\begin{bt}%[Huỳnh Quy]%[0D1B3-3]
	Lớp $10{{\text{A}}_{1}}$ có $7$ học sinh giỏi Toán, $5$ học sinh giỏi Lý, $6$ học sinh giỏi Hóa, $3$ học sinh giỏi cả Toán và Lý, $4$ học sinh giỏi cả Toán và Hóa, $2$ học sinh giỏi cả Lý và Hóa, $1$ học sinh giỏi cả $3$ môn Toán, Lý, Hóa. Tính số học sinh giỏi đúng hai môn học của lớp $10{{\text{A}}_{1}}$.
	\loigiai{
		\begin{center}
			\begin{tikzpicture}
				\def\firstcircle{(0,0) circle (1.5cm)}
				\def\secondcircle{(45:2.5cm) circle (2cm)}
				\def\thirdcircle{(-15:1.7cm) circle (1.4cm)}
				\colorlet{circle edge}{black!50}
				\colorlet{circle area}{black!20}
				\tikzset{filled/.style={fill=circle area, draw=circle edge, thick},
					outline/.style={draw=circle edge, thick}}
				\draw \firstcircle;
				\draw \secondcircle;
				\draw \thirdcircle;
				\begin{scope}
					\clip \firstcircle;
					\fill[filled] \secondcircle;
				\end{scope}
				\begin{scope}
					\clip \firstcircle;
					\fill[filled] \thirdcircle;
				\end{scope}
				\begin{scope}
					\clip \secondcircle;
					\fill[filled] \thirdcircle;
				\end{scope}  
				\node at (-1,0){1};  
				\node at (0.5,1){2}; 
				\node at (1,0.3){1};  
				\node at (2,0.3){1};      
				\node at (1,-0.5){3};   
				\node at (2,-1){1};     
				\node at (3,2.5){1}; 
				\draw[outline] \firstcircle
				\secondcircle  \thirdcircle;    
				\node at (-2,4) {\textbf{Toán}};
				\node at (2,6) {\textbf{Giỏi Toán + Lý}};
				\node at (5,5) {\textbf{Lý}};
				\node at (7,1) {\textbf{Giỏi Hóa + Lý}};   
				\node at (5,-3) {\textbf{Hóa}};
				\node at (-2,-3) {\textbf{Giỏi Toán + Hóa}};
				\draw[->] (-2,3.5) -- (-1,1.2) 
				(2,5.5) -- (0.7,0.3) 
				(2,5.5) -- (0.2,0.9)
				(5,4.5) -- (3.6,2.5)
				(5,0.9) -- (1.2,0.5)
				(5,0.9) -- (2.4,0.2)
				(4,-2.9) -- (2,-1.8)
				(-2,-2.5) -- (1.2,0)
				(-2,-2.5) -- (0.9,-0.9)
				;
			\end{tikzpicture}
		\end{center}
		Dựa vào biểu đồ ven trên, ta có số học sinh giỏi đúng hai môn học là $2+1+3=6$}
\end{bt}

\subsection{Bài tập trắc nghiệm}
\Opensolutionfile{ans}[ans/ans-0D1-2-TN]
\begin{ex}%[Bài giảng Toán 10 - 2022]%[Nhật Thiện]%[0D1Y3-1]
	Cho hai tập hợp $X=\{1;3;5;8\}$, $Y=\{3;5;7;9\}$. Tập hợp $X\cup Y$ bằng tập hợp nào sau đây?
	\choice
	{$\{1;3;5\}$}
	{$\{3;5\}$}
	{$\{1;7;9\}$}
	{\True $\{1;3;5;7;8;9\}$}
	\loigiai{
		Ta có $X\cup Y=\{1;3;5;7;8;9\}$.
	}
\end{ex}
\begin{ex}%[Bài giảng Toán 10 - 2022]%[Nhật Thiện]%[0D1Y3-2]
	Cho hai tập hợp $A=\{1;2;3;4;5\}$ và $B=\{0;2;4;6;8\}$. Tìm $A\setminus B$.
	\choice
	{$A\setminus B=\{2;4\}$}
	{\True $A\setminus B=\{1;3;5\}$}
	{$A\setminus B=\{0;1;3;5\}$}
	{$A\setminus B=\{0;6;8\}$}
	\loigiai{
		Ta có $A\setminus B=\{1;3;5\}$.
	}
\end{ex}
\begin{ex}%[Bài giảng Toán 10 - 2022]%[Nhật Thiện]%[0D1B3-1]
	Cho hai tập hợp $ A=\left\lbrace x\in \mathbb{R}\,\mid\,(2x-x^2)(x-1)=0 \right\rbrace$, $ B=\left\lbrace n \in \mathbb{N}\,\mid\,0<n^2<10 \right\rbrace$. Chọn mệnh đề đúng?
	\choice
	{\True $A \cap B =\left\lbrace 1;2 \right\rbrace $}
	{$A \cap B =\left\lbrace 2 \right\rbrace $}
	{ $A \cap B =\left\lbrace 0;1;2;3 \right\rbrace $}
	{$A \cap B =\left\lbrace 0;3 \right\rbrace $}
	\loigiai{
		Ta có
		\begin{itemize}
			\item $ (2x-x^2)(x-1)=0\Leftrightarrow \hoac{& x=0 \\ &x=1\\&x=2 }\Rightarrow A=\left\lbrace 0;1;2 \right\rbrace  $.
			\item  $ B= \left\lbrace 1;2;3 \right\rbrace$.
		\end{itemize}
		Suy ra $ A\cap B= \left\lbrace 1;2 \right\rbrace$.}
\end{ex}
\begin{ex}%[Bài giảng Toán 10 - 2022]%[Nhật Thiện]%[0D1B3-1]
	Cho các tập hợp $A=\left\{ x\in \mathbb{N} \ | \ (4-x^2)(x^2-5x+4)=0 \right\}; B = \left\{ x\in \mathbb{Z} \ | \ x \text{ là ước của } 4 \right\}$. Tập hợp $A \cap B$ là
	\choice
	{$\{-2,1,2,4\}$}
	{\True $\{1,2,4\}$}
	{$\{2,4\}$}
	{$\{-4,-2,-1,1,2,4\}$}
	\loigiai{
		Ta có $(4-x^2)(x^2-5x+4)=0 \Leftrightarrow \hoac{&4-x^2=0\\&x^2-5x+4=0} \Leftrightarrow \hoac{&x=2\\&x=-2\\&x=1\\&x=4.}$\\
		Vì $x\in \mathbb{N}$ nên $x \in \{1,2,4\}$.\\
		Do đó $A=\{1,2,4\} \quad (1)$.\\
		Ta có các ước của $4$ là $\pm 1, \pm 2, \pm 4$.\\
		Do đó $B=\{-4,-2,-1,1,2,4\} \quad (2)$.\\
		Từ $(1), (2)$ ta có $A \cap B = \{1,2,4\}$.
	}
\end{ex}
\begin{ex}%[Bài giảng Toán 10 - 2022]%[Nhật Thiện]%[0D1Y4-2]
	Cho hai tập hợp $A = \left(-5;7\right)$ và $B = \left(1;+\infty\right)$. Tìm $A\setminus B$.
	\choice
	{\True $A\setminus B = \left(-5;1\right]$}
	{$A\setminus B = \left(-5;1\right)$}
	{$A\setminus B = \left[7;+\infty\right)$}
	{$A\setminus B = \left(7;+\infty\right)$}
	\loigiai{
		Ta có $A\setminus B = \left(-5;1\right]$.
	}
\end{ex}
\begin{ex}%[Bài giảng Toán 10 - 2022]%[Nhật Thiện]%[0D1Y4-2]
	Cho hai tập hợp $A=\left[ -2;4\right)$ và $B=\left(0;+\infty\right)$. Tìm khẳng định đúng.
	\choice
	{$A\cup B=\left(4;+\infty\right)$}
	{\True $A\cap B=\left(0;4\right)$}
	{$B\setminus A=\left[ -2;+\infty\right)$}
	{$A\setminus B=\left[ -2;0\right)$}
	\loigiai{
		$A\cup B=[-2;+\infty) \rightarrow $ loại.\\
		$A\cap B = (0;4) \rightarrow$ chọn.\\
		$B\setminus A = [4;+\infty) \rightarrow$ loại.\\
		$A\setminus B = [-2;0] \rightarrow$ loại.
	}
\end{ex}
\begin{ex}%[Bài giảng Toán 10 - 2022]%[Nhật Thiện]%[0D1B3-2]
	Cho $A$ là tập hợp các hình thoi, $B$ là tập hợp các hình chữ nhật và $C$ là tập hợp các hình vuông. Khi đó
	\choice
	{\True $A\cap B=C $}
	{$A\setminus B=C$}
	{$B\setminus A=C$}
	{$A\cup B=C$}
	\loigiai{
		Ta có hình thoi có hai cạnh kề vuông góc khi và chỉ khi nó là hình vuông.\\
		Hình chữ nhật có hai cạnh kề bằng nhau khi và chỉ khi nó là hình vuông.
	}
\end{ex}
\begin{ex}%[Bài giảng Toán 10 - 2022]%[Nhật Thiện]%[0D1B3-2]
	Cho hai tập hợp $M=\{1;2;3;5\}$ và $N=\{2;6;-1\}$. Xét các khẳng định
	\begin{enumEX}[(I)]{3}
		\item $M\cap N=\{2\}$
		\item $N\setminus M=\{1;3;5\}$
		\item $M\cup N=\{1;2;3;5;6;-1\}.$
	\end{enumEX}
	Có bao nhiêu khẳng định đúng trong ba khẳng định nêu trên?
	\choice
	{$0$}
	{$3$}
	{$1$}
	{\True $2$}
	\loigiai{
		Ta có $M\cap N=\{2\}$; $N\setminus M=\{6;-1\}$ và $M\cup N=\{1;2;3;5;6;-1\}$.\\
		Vậy có $2$ khẳng định đúng là (I) và (III).
	}
\end{ex}
\begin{ex}%[Bài giảng Toán 10 - 2022]%[Nhật Thiện]%[0D1B3-2]
	Cho hai tập hợp $A=\{2;4;6;8\}$ và $B$ là tập hợp các số tự nhiên nhỏ hơn $10$. Phần bù của $A$ trong $B$ là
	\choice
	{\True $\{0;1;3;5;7;9\}$}
	{$[0;10) \setminus \{2;4;6;8\}$}
	{$\varnothing$}
	{$\{1;3;5;7;9\}$}
	\loigiai{
		Vì $B$ là tập hợp các số tự nhiên nhỏ hơn $10$ nên $B=\{0;1;2;3;4;5;6;7;8;9\}$.\\
		Khi đó $\mathrm{C}_B A=\{0;1;3;5;7;9\}$.
	}
\end{ex}
\begin{ex}%[Bài giảng Toán 10 - 2022]%[Nhật Thiện]%[0D1B4-2]
	Cho hai tập hợp $C_{\mathbb{R}}A = (0;+\infty)$ và $C_{\mathbb{R}}B =(-\infty;-5)\cup  (-2;+\infty)$. Xác định tập $A \cup B$.
	\choice
	{$A \cap B = (-2;0)$}
	{$A \cap B = (-5;-2)$}
	{$A \cap B = (-5;0]$}
	{\True $A \cap B = [-5;-2]$}
	\loigiai{
		Ta có $C_{\mathbb{R}}A \cup C_{\mathbb{R}}B = C_{\mathbb{R}} (A \cap B) = (-\infty;-5)\cup  (-2;+\infty)$, suy ra $A \cap B = [-5;-2]$.
	}
\end{ex}

\begin{ex}%[Bài giảng Toán 10 - 2022]%[Nhật Thiện]%[0D1B4-1]
	Hình vẽ nào dưới đây biểu diễn cho tập hợp $[-2;1]\cap (0;1)$?
	\choice
	{\begin{tikzpicture}[scale=.5,line join=round]
			\draw[->](-3,0)->(5,0); %Ve truc so
			\IntervalLR{-3}{-1}
			\IntervalGRF{}{}{(}{0}
			\IntervalLR{3}{5}
			\IntervalGRF{]}{1}{}{}
			\def\skipInterval{0.5cm}%khoang cach dat nhan
	\end{tikzpicture}}
	{\begin{tikzpicture}[scale=.5,line join=round]
			\draw[->](-3,0)->(5,0); %Ve truc so
			\IntervalLR{-3}{-1}
			\IntervalGRF{}{}{[}{-2}
			\IntervalLR{3}{5}
			\IntervalGRF{)}{1}{}{}
			\def\skipInterval{0.5cm}%khoang cach dat nhan
	\end{tikzpicture}}
	{\True \begin{tikzpicture}[scale=.5,line join=round]
			\draw[->](-3,0)->(5,0); %Ve truc so
			\IntervalLR{-3}{-1}
			\IntervalGRF{}{}{(}{0}
			\IntervalLR{3}{5}
			\IntervalGRF{)}{1}{}{}
			\def\skipInterval{0.5cm}%khoang cach dat nhan
	\end{tikzpicture}}
	{\begin{tikzpicture}[scale=.5,line join=round]
			\draw[->](-3,0)->(5,0); %Ve truc so
			\IntervalLR{-3}{-1}
			\IntervalGRF{}{}{[}{0}
			\IntervalLR{3}{5}
			\IntervalGRF{]}{1}{}{}
			\def\skipInterval{0.5cm}%khoang cach dat nhan
	\end{tikzpicture}}
	\loigiai{
		Ta có $[-2;1]\cap (0;1)=(0;1)$.}
\end{ex}
\begin{ex}%[Bài giảng Toán 10 - 2022]%[Nhật Thiện]%[0D1K3-2]
	Cho hai tập $A=\left\{x\in \mathbb{Z}\left| \dfrac{x+5}{x+1}\in \mathbb{Z}\right. \right\}$ và $B=\{x\in \mathbb{N}\mid x^2-4x+3=0\}$. Có bao nhiêu tập hợp $X$ thỏa mãn $B\subset X \subset A$?
	\choice
	{$64$}
	{\True $16$}
	{$8$}
	{$32$}
	\loigiai{
		Ta có $\dfrac{x+5}{x+1}=1+\dfrac{4}{x+1}$.\\
		Vì $x\in\mathbb{Z}$ và $\dfrac{x+5}{x+1}\in \mathbb{Z}$ nên $\dfrac{4}{x+1}\in \mathbb{Z}\Leftrightarrow x+1\in \{1;2;4;-1;-2;-4\} \Leftrightarrow x\in\{0;1;3;-2;-3;-5\}$.\\
		Do đó $A=\{-5;-3;-2;0;1;3\}$.\\
		Vì $x^2-4x+3=0\Leftrightarrow \hoac{&x=1\\&x=3}$ nên $B=\{1;3\}$.\\
		Ta có $B\subset X \subset A\Leftrightarrow \{1;3\}\subset X \subset \{-5;-3;-2;0;1;3\}$.\\
		Suy ra số tập $X$ đúng bằng số tập con của tập $\{-5;-3;-2;0\}$.\\
		Vậy số tập $X$ là $2^4=16$.
	}
\end{ex}
\begin{ex}%[Bài giảng Toán 10 - 2022]%[Nhật Thiện]%[0D1G3-2]
	Cho tập hợp $X=\{3;-4;5\}$ có hai tập con $A$ và $B$ (số phần tử của tập $B$ ít hơn số phần tử của tập $A$). Có bao nhiêu cặp $(A;B)$ mà $\{3;-4\}\cup (A\setminus B)=X$?
	\choice
	{$12$}
	{$10$}
	{\True $11$}
	{$15$}
	\loigiai{
		Từ giả thiết $\{3;-4\}\cup (A\setminus B)=X\Rightarrow 5\in (A\setminus B)\Rightarrow \heva{&5\in A\\&5\notin B.}$\\
		Vì số phần tử của tập $B$ ít hơn số phần tử của tập $A$ nên tập $B$ có không quá $2$ phần tử.\\
		Các khả năng có thể xảy ra và thỏa mãn là
		\begin{itemize}
			\item TH1: $A=\{3;-4;5\}$ và $B$ bằng một trong các tập sau $\varnothing$, $\{3\}$, $\{-4\}$, $\{3;-4\}$.
			\item TH2: $A=\{-4;5\}$ và $B$ bằng một trong các tập sau $\varnothing$, $\{3\}$, $\{-4\}$.
			\item TH3: $A=\{3;5\}$ và $B$ bằng một trong các tập sau $\varnothing$, $\{3\}$, $\{-4\}$.
			\item TH4: $A={5}$ và $B=\varnothing$.
		\end{itemize}
		Vậy tất cả có $11$ kết quả thỏa mãn.
	}
\end{ex}
\begin{ex}%[Bài giảng Toán 10 - 2022]%[Nhật Thiện]%[0D1K4-2]
	Tìm điều kiện của tham số $m$ để $A\cap B$ là một khoảng, biết $A ( m ; m + 2 )$, $B ( 4 ; 7 )$.
	\choice
	{$4 \leq m<7$}
	{\True $2<m<7$}
	{$2 \leq m<7$}
	{$2< m < 4$}
	\loigiai{
		Để $A \cap B = \varnothing$ thì $\hoac {&m+2 \leq 4 \\ & m \geq 7 } \Leftrightarrow \hoac{&m \leq 2\\&m\geq 7.}$ \\
		Do đó, để $A\cap B$ là một khoảng thì $2<m<7$.}
\end{ex}
\begin{ex}%[Bài giảng Toán 10 - 2022]%[Nhật Thiện]%[0D1K4-2]
	Cho hai tập hợp $A=(m-1;5)$ và $B=(3;+\infty)$. Tìm tất cả các giá trị thực của tham số $m$ để $A\setminus B=\varnothing$.
	\choice
	{$4\leq m\leq 6$}
	{$m=4$}
	{\True $m\geq 4$}
	{$4\leq m<6$}
	\loigiai
	{Ta có $A\setminus B=\varnothing\Leftrightarrow 3\leq m-1\Leftrightarrow m\geq 4$.
	}
\end{ex}
\begin{ex}%[Bài giảng Toán 10 - 2022]%[Nhật Thiện]%[0D1G4-1]
	Cho hai tập hợp $A=[-5;8)$ và $B=[-m;m+2]$. Tìm tất cả các giá trị của $m$ để $A \cap B \ne \varnothing$.
	\choice
	{$m\in(-8;6)$}
	{$m\in[-7;+\infty)$}
	{$m\in(-8;+\infty)$}
	{\True $m\in(-1;+\infty)$}
	\loigiai{
		$A \cap B \ne \varnothing \Leftrightarrow \heva{&-m < m + 2\\&-m < 8\\&m+2 \ge -5}\Leftrightarrow m > -1$.
	}
\end{ex}
\begin{ex}%[Nguyễn Vương Hiển]%[0D1Y2-1]
	Tập hợp $A=\left\{x\in \mathbb{R}\big| x^2-6x+8=0\right\}$ có bao nhiêu phần tử?
	\choice
	{$0$}
	{$1$}
	{\True $2$}
	{vô số}
	\loigiai{
		$x^2-6x+8=0\Leftrightarrow \hoac{&x=2\\&x=4.}$\\
		Vậy tập hợp $A$ có $2$ phần tử.
	}
\end{ex}
\begin{ex}%[Nguyễn Vương Hiển]%[0D1B2-1]
	Tập hợp $A=\left\{x\in \mathbb{Z}^{+}\big|x^2-x=0\right\}$ có bao nhiêu phần tử?
	\choice
	{\True $1$}
	{$2$}
	{$0$}
	{$3$}
	\loigiai{
	$x^2-x=0\Leftrightarrow \hoac{&x=0\\&x=1.}$\\
	Vì $x\in\mathbb{Z}^{+}$ nên $A=\{1\}$.
	Vậy tập hợp $A$ có $1$ phần tử.
	}
\end{ex}
\begin{ex}%[Nguyễn Vương Hiển]%[0D1Y2-1]
	Hãy viết tập hợp $A=\left\{x\in \mathbb{R}\big| x^2-6x+8=0\right\}$ dưới dạng liệt kê các phần tử.
	\choice
	{\True $A=\left\{2;4\right\}$}
	{$A=\left\{6;8\right\}$}
	{$A=\left\{-2;2\right\}$}
	{$A=\left(2; 4\right)$}
	\loigiai{
		Ta có $x^2-6x+8=0\Leftrightarrow \hoac{&x=2\\&x=4}$ nên $A=\left\{2; 4\right\}$.	
	}
\end{ex}
\begin{ex}%[Nguyễn Vương Hiển]%[0D1Y2-1]
	Trong các mệnh đề sau, mệnh đề nào đúng?
	\choice{\True ``$x\in [-4;1)\Leftrightarrow -4 \le x <1$''}
	{``$x\in [-4;1)\Leftrightarrow -4 < x \le 1 $''}
	{``$x\in [-4;1)\Leftrightarrow -4 \le x \le 1$''}
	{``$x\in [-4;1)\Leftrightarrow -4 < x <1$''}
	\loigiai{ Ta có ``$x\in [-4;1)\Leftrightarrow -4 \le x <1$''.
	}
\end{ex}
\begin{ex}%[Nguyễn Vương Hiển]%[0D1Y2-1]
	Số tập con của tập hợp $X=\{x\in\mathbb{Z}\ |\ 2x^2-5x+2=0\}$ là?
	\choice
	{$1$}
	{$3$}
	{\True $2$}
	{$4$}
	\loigiai{
		Ta có $2x^2-5x+2=0\Leftrightarrow \hoac{&x=2\\ &x=\dfrac{1}{2}}$, mà $x\in\mathbb{Z}$ nên $x=2$.\\ Vậy $X=\{2\}$ nên có $2$ tập con.
	}
\end{ex}
\begin{ex}%[Nguyễn Vương Hiển]%[0D1Y2-1]
	Tập hợp $A=\{1;2;3;4;5;6\}$ được viết dưới dạng chỉ ra tính chất đặc trưng cho các phần tử của nó là
	\choice
	{$A=\left\{n\in \mathbb{N}\big| 1<n\le 6\right\}$}
	{$A=\left\{n\in \mathbb{N}\big| n\le 6\right\}$}
	{\True $A=\left\{n\in \mathbb{N}\big| 0<n\le 6\right\}$}
	{$A=\left\{n\in \mathbb{N}\big| 0<n<6\right\}$}
	\loigiai{
		Ta có $A=\{1;2;3;4;5;6\}=\left\{n\in \mathbb{N}\big| 0<n\le6\right\}$.
	}
\end{ex}
\begin{ex}%[Nguyễn Vương Hiển]%[0D1Y3-1]
	Cho hai tập hợp $X=\left\{ 7, 2, 8, 4, 9, 12 \right\}$ và $Y=\left\{ 1, 3, 7, 4 \right\}$. Tìm tập hợp $X\cap Y$.
	\choice
	{$\left\{ 1, 2, 3, 4, 8, 9, 7, 12 \right\}$}
	{$\left\{ 2, 8, 9, 12 \right\}$}
	{\True $\left\{ 4, 7 \right\}$}
	{$\left\{ 1, 3 \right\}$}
	\loigiai{
		Ta có	$X\cap Y=\{4,7\}$.
	}
\end{ex}
\begin{ex}%[Nguyễn Vương Hiển]%[0D1Y3-1]
	Cho hai tập hợp $X=\left\{ 2, 4, 6, 9 \right\}$ và $Y=\left\{ 1, 2, 3, 4 \right\}$. Tìm tập hợp $X \cup Y$.
	\choice
	{$\left\{1, 3 \right\}$	}
	{$\left\{6, 9 \right\}$}
	{\True $\left\{1, 2, 3, 4, 6, 9 \right\}$}
	{$\left\{2, 4 \right\}$}
	\loigiai{
		Ta có $X \cup Y=\left\{1, 2, 3, 4, 6, 9 \right\}$.
	}
\end{ex}
\begin{ex}%[Nguyễn Vương Hiển]%[0D1Y3-2]
	Cho hai tập hợp $X=\left\{0, 1, 2, 3, 4\right\}$ và $Y=\left\{ 2, 3, 4, 5, 6 \right\}$. Tìm tập hợp $X\setminus Y$.
	\choice
	{$\left\{ 0 \right\}$}
	{\True $\left\{ 0, 1 \right\}$}
	{$\left\{ 1, 2 \right\}$}
	{$\left\{ 1, 5 \right\}$}
	\loigiai{
		Ta có $X\setminus Y=\left\{ 0, 1 \right\}$.
	}
\end{ex}
\begin{ex}%[Nguyễn Vương Hiển]%[0D1Y3-2]
	Cho hai tập hợp $A = \left\{0, 2, 4, 6, 8\right\}$ và $B = \left\{0, 2, 4\right\}$. Tìm tập hợp $C_{A}B$.
	\choice
	{$\left\{0, 2, 4, 6\right\}$}
	{$\left\{0, 2, 4, 8\right\}$}
	{$\left\{2, 4\right\}$}
	{\True $\left\{6, 8\right\}$}
	\loigiai{
		Ta có $B\subset A$ và $A\setminus B=\{6,8\}\Rightarrow C_{A}B=\{6, 8\}$.
	}
\end{ex}
\begin{ex}%[Nguyễn Vương Hiển]%[0D1B4-1]
	Cho $A=(-\infty;-2], B=[3;+\infty)$ và $C=(0;4)$. Khi đó tập $(A\cup B)\cap C$ là
	\choice
	{$(-\infty;-2)\cup[3;+\infty)$}
	{$(-\infty;-2]\cup (3;+\infty)$}
	{\True $[3;4)$}
	{$[3;4]$}
	\loigiai{
		$(A\cup B)=(-\infty;-2]\cup [3;+\infty)$.\\
		Vậy $(A\cup B)\cap C =[3;4)$.
	}
\end{ex}
\begin{ex}%[Nguyễn Vương Hiển]%[0D1B4-1]
	Cho hai tập hợp $A=(-3;4]$ và $B=(-\sqrt 2;+\infty)$. Tập hợp $A\cap B$ là
	\choice
	{\True $(-\sqrt 2;4]$}
	{$(-3;+\infty)$}
	{$(-3;-\sqrt 2]$}
	{$(4;+\infty)$}
	\loigiai{
		Ta có $A\cap B=(-\sqrt 2;4]$.
	}
\end{ex}
\begin{ex}%[Nguyễn Vương Hiển]%[0D1K4-1]
	Cho hai tập hợp $A=\left\{ x\in \mathbb{R}\big|x+2\geq 0 \right\}$ và $B=\left\{ x\in \mathbb{R}\big|5-x\geq 0 \right\}$. Tìm tập hợp $A\cap B$. 
	\choice
	{\True $\left[ -2;5 \right]$}
	{$\left[ -2;6 \right]$}
	{$\left[ -5;2 \right]$}
	{$\left( -2;+\infty  \right)$}
	\loigiai{
		Ta có $\heva{& A=\left\{ x\in \mathbb{R}\big|x+2\geq 0 \right\}=[-2;+\infty)\\&B=\left\{ x\in \mathbb{R}\big|5-x\geq 0 \right\}=(-\infty;5].}$\\
		Khi đó $A\cap B=[-2;+\infty)\cap (-\infty;5]=[-2;5]$.
	}
\end{ex}
\begin{ex}%[Nguyễn Vương Hiển]%[0D1K4-1]
	Cho các tập hợp $M = [1; 4]$, $N = (2; 6)$ và $P = (1; 2)$. Tìm tập hợp $(M \cap N) \cap P$.
	\choice
	{$[0; 4]$}
	{$[5; + \infty  )$}
	{$(- \infty  ; 1)$}
	{\True $\varnothing $}
	\loigiai{
		Ta có $M\cap N=(2;4]\Rightarrow M\cap N\cap P=(2;4]\cap (1;2)=\varnothing$.
	}
\end{ex}
\begin{ex}%[Nguyễn Vương Hiển]%[0D1G3-3]
	Lớp $10A$ có $10$ học sinh giỏi Văn, $15$ học sinh giỏi Sử, $5$ học sinh giỏi cả $2$ môn Văn, Sử và $2$ học sinh không giỏi môn nào. Hỏi lớp $10A$ có bao nhiêu học sinh?
	\choice
	{$20$ }
	{\True $22$}
	{$25$}
	{$28$}
	\loigiai{
		Số học sinh giỏi một môn Văn: $10-5=5$(học sinh).\\
		Số học sinh giỏi một môn Sử: $15-5=10$(học sinh).\\
		Số học sinh lớp $10A$: $2+5+10+5=22$(học sinh).
	}
\end{ex}
\begin{ex}%[Nguyễn Vương Hiển]%[0D1G3-3]
	Để phục vụ cho công việc tiêm vắc-xin phòng chống Covid-19, Sở y tế đã huy động $30$ cán bộ đo huyết áp, $25$ cán bộ tiêm vắc-xin. Trong đó có $12$ cán bộ làm được cả $2$ công việc đo huyết áp và tiêm vắc-xin. Hỏi Sở y tế đã huy động tất cả bao nhiêu cán bộ cho công việc tiêm vắc-xin phòng chống Covid-19?
	\choice{$42$}
	{$31$}
	{$55$}
	{\True $43$}
	\loigiai{
		Số cán bộ được huy động là: $30+25-12=43$ cán bộ.
	}
\end{ex}

\Closesolutionfile{ans}
% % \indapan{10}{ans/ans-0D1-2-TN}
% \Closesolutionfile{ansbook}
% \section*{Đề kiểm tra Chương 1}
\subsection*{Đề số 1}
\setcounter{ex}{0}\setcounter{bt}{0}
\Opensolutionfile{ans}[ans/ans-KT-101]
\noindent\textbf{I. PHẦN TRẮC NGHIỆM}
%[Phan Quốc Trí, Bai Giảng T10(2022)]%
\begin{ex}%[Phan Quốc Trí, Bai Giảng T10(2022)]%[0D1Y1-1]
	Trong các câu sau, câu nào không phải là mệnh đề?
	\choice
	{Hình bình hành có bốn cạnh bằng nhau}
	{\True Chúc bạn may mắn}
	{Số $8$ là số chính phương}
	{Cà Mau là tên một tỉnh của nước Việt Nam}
	\loigiai{
		Mệnh đề là một câu khẳng định \textbf{đúng} hoặc một câu khẳng định \textbf{sai}.
	}
\end{ex}
\begin{ex}%[Phan Quốc Trí, Bai Giảng T10(2022)]%[0D1Y1-1]
	Trong các câu sau, câu nào là mệnh đề?
	\choice
	{$x^2+x=2$}
	{Hôm nay trời đẹp quá!}
	{$2n+1$ chia hết cho 3}
	{\True Số 15 là một số nguyên tố}
	\loigiai{
\begin{itemize}
	\item $x^2+x=2$  là một khẳng định, nhưng không là mệnh đề. 
	\item  Hôm nay trời đẹp quá! không phải là mệnh đề.
	\item Câu $2n+1$ chia hết cho 3 là một khẳng định, nhưng không là mệnh đề. 
	\item  Số 15 là một số nguyên tố là một mệnh đề sai.
\end{itemize}
	}
\end{ex}
\begin{ex}%[Phan Quốc Trí, Bai Giảng T10(2022)]%[0D1Y1-2]
	Trong các mệnh đề sau, mệnh đề nào là mệnh đề \textbf{sai}?
	\choice
	{\True $\sqrt{3}$ là số nguyên}
	{$6$ chia hết cho $2$}
	{$5$ chia hết cho $5$}
	{$30$ là một số chẵn}
	\loigiai{
		Trong các mệnh đề đã cho, mệnh đề sai là \lq\lq$\sqrt{3}$ là số nguyên\rq\rq.
	}
\end{ex}
\begin{ex}%[Phan Quốc Trí, Bai Giảng T10(2022)]%[0D1Y1-2]
	Cho mệnh đề chứa biến $P(x): 3x+5\le x^2$ với $x$ là số thực. Mệnh đề nào sau đây là đúng?
	\choice
	{$P(3)$}
	{$P(4)$}
	{$P(1)$}
	{\True $P(5)$}
	\loigiai{
		$P(3)= 3.3+5\leqslant 3^2 \Leftrightarrow 14\leqslant 9$ là mệnh đề sai.\\
		$P(4) =3.4+5\leqslant 4^2 \Leftrightarrow 17\leqslant 16$ là mệnh đề sai.\\
		$P(1)=3.1+5\leqslant 1^2 \Leftrightarrow 8\leqslant 1$ là mệnh đề sai.\\
		$P(5) =3.5+5\leqslant 5^2 \Leftrightarrow 20\leqslant 25$ là mệnh đề đúng.}
\end{ex}
\begin{ex}%[Phan Quốc Trí, Bai Giảng T10(2022)]%[0D1Y1-3]
	Cho mệnh đề $P\colon$ \lq\lq  $9$ là số chia hết cho $3$\rq\rq. Mệnh đề phủ định của mệnh đề $P$ là
	\choice
	{$\overline{P}\colon$\lq\lq  $9$ là ước của $3$\rq\rq}
	{$\overline{P}\colon$\lq\lq  $9$ là bội của $3$\rq\rq}
	{\True $\overline{P}\colon$ \lq\lq  $9$ là số không chia hết cho $3$\rq\rq}
	{$\overline{P}\colon$\lq\lq  $9$ là số lớn hơn $3$\rq\rq}
	\loigiai{Mệnh đề $P\colon$\lq\lq  $9$ là số chia hết cho $3$\rq\rq      có mệnh đề phủ định là $\overline{P}\colon$ \lq\lq  $9$ là số không chia hết cho $3$\rq\rq.}
\end{ex}
\begin{ex}%[Phan Quốc Trí, Bai Giảng T10(2022)]%[0D1Y1-3]
	Cho mệnh đề \lq\lq$\forall x\in \mathbb{R},\, x^2+1>0$\rq \rq. Mệnh đề phủ định của mệnh đề đã cho là
	\choice
	{\lq\lq$\forall x\in \mathbb{R},\, x^2+1\leq 0$\rq \rq}
	{\lq\lq$\forall x\in \mathbb{R},\, x^2+1<0$\rq \rq}
	{\True \lq\lq$\exists x\in \mathbb{R},\, x^2+1\leq 0$\rq \rq}
	{\lq\lq$\exists x\in \mathbb{R},\, x^2+1>0$\rq \rq}
	\loigiai{
		Mệnh đề phủ định của mệnh đề \lq\lq$\forall x\in \mathbb{R},\, x^2+1>0$\rq \rq\, là \lq\lq$\exists x\in \mathbb{R},\, x^2+1\leq 0$\rq \rq.}
\end{ex}
\begin{ex}%[Phan Quốc Trí, Bai Giảng T10(2022)]%[0D1Y1-4]
	Cho mệnh đề $P\colon$ ``Tam giác $ABC$ cân tại $A$'', mệnh đề $Q\colon$ ``$AB=AC$''. Phát biểu mệnh đề ``$P$ kéo theo $Q$'' là
	\choice
	{Nếu $AB=AC$ thì tam giác $ABC$ cân tại $A$}
	{\True Nếu tam giác $ABC$ cân tại $A$ thì $AB=AC$}
	{Tam giác $ABC$ cân tại $B$ là điều kiện cần và đủ để $AB=AC$}
	{Tam giác $ABC$ cân tại $A$ khi và chỉ khi $AB=AC$}
	\loigiai{
		Mệnh đề ``$P$ kéo theo $Q$'' là ``Nếu tam giác $ABC$ cân tại $A$ thì $AB=AC$''.
	}
\end{ex}
\begin{ex}%[Phan Quốc Trí, Bai Giảng T10(2022)]%[0D1Y1-4]
	Cho mệnh đề $P$: \lq\lq Nếu tam giác có hai đường trung tuyến bằng nhau thì đó là tam giác cân\rq\rq. Mệnh đề nào sau đây là mệnh đề đảo của $P$?
	\choice
	{Tam giác có hai đường trung tuyến bằng nhau thì nó là tam giác cân}
	{\True Nếu tam giác $ABC$ cân thì tam giác đó có hai đường trung tuyến bằng nhau }
	{Tam giác là tam giác cân khi và chỉ khi nó có hai đường trung tuyến bằng nhau}
	{Tam giác là tam giác cân nếu nó có hai đường trung tuyến bằng nhau}
	\loigiai{
Mệnh đề đảo là \lq \lq Nếu tam giác $ABC$ cân thì tam giác đó có hai đường trung tuyến bằng nhau.\rq \rq		
	}
\end{ex}
\begin{ex}%[Phan Quốc Trí, Bai Giảng T10(2022)]%[0D1Y1-5]
	Mệnh đề "Bình phương mọi số thực đều không âm" mô tả mệnh đề nào dưới đây?
	\choice
	{"$\forall n\in\mathbb{N}: n^2\geq 0$"}
	{"$\exists x\in\mathbb{R}:x^2\geq 0$"}
	{\True "$\forall x\in\mathbb{R}: x^2\geq 0$"}
	{"$\forall x\in\mathbb{R}:x^2>0$"}
	\loigiai{
	Mệnh đề trên được viết lại dạng: \lq \lq $\forall x\in\mathbb{R}: x^2\geq 0$\rq \rq	
	}
\end{ex}
\begin{ex}%[Phan Quốc Trí, Bai Giảng T10(2022)]%[0D1B1-5]
	Mệnh đề nào sau đây là đúng?
	\choice
	{$\forall n\in\mathbb{N}\colon n^2>n$}
	{$\forall x\in\mathbb{R}\colon x^2<2$}
	{$\forall x\in\mathbb{Z}\colon 2x>1$}
	{\True $\exists x\in\mathbb{R}\colon x^2>x$}
	\loigiai
	{
		\begin{itemize}
			\item Mệnh đề ``$\forall n\in\mathbb{N}\colon n^2>n$'' sai vì với $n=1$ thì $n^2=1=n$.
			\item Mệnh đề ``$\forall x\in\mathbb{R}\colon x^2<2$'' sai vì với $x=2$ thì $x^2=4>2$.
			\item Mệnh đề ``$\forall x\in\mathbb{Z}\colon 2x>1$'' sai vì với $x=0$ thì $2x=0<1$.
			\item Mệnh đề ``$\exists x\in\mathbb{R}\colon x^2>x$'' đúng vì với $x=2$ thì $x^2=4$ nên $x^2>x$.
		\end{itemize}
	}
\end{ex}
\begin{ex}%[Phan Quốc Trí, Bai Giảng T10(2022)]%[0D1Y2-1]
	Hãy liệt kê các phần tử của tập hợp $X = \left\{ x \in \mathbb{Z} | 2x^2-5x+3=0 \right\}$.
	\choice
	{$X = \left\{1;\dfrac{3}{2} \right\}$}
	{\True $X = \{ 1 \}$}
	{$X = \left\{ \dfrac{3}{2} \right\}$}
	{$X = \varnothing$}
	\loigiai{
		Ta có $2x^2-5x+3=0\Leftrightarrow \hoac{&x=1\in \mathbb{Z}\\&x=\dfrac{3}{2}\notin \mathbb{Z}}$.\\
		Vậy $X = \{ 1 \}$.		
	}
\end{ex}
\begin{ex}%[Phan Quốc Trí, Bai Giảng T10(2022)]%[0D1B2-1]
	Viết tập hợp $A=\lbrace x \in \mathbb{Z}|x^2<17 \rbrace $ theo cách liệt kê các phần tử, ta được tập hợp nào sau đây?
	\choice{\True $ \lbrace -4;-3;-2;-1;0;1;2;3;4 \rbrace $}
	{$ \lbrace 1;2;3;4 \rbrace $}
	{$\lbrace 0;1;2;3;4  \rbrace $}
	{$ \lbrace -4; -3;-2;-1 \rbrace $}
	\loigiai{Ta có $x^2<17\Leftrightarrow \left|x\right|<\sqrt{17}\Leftrightarrow -\sqrt{17}<x<\sqrt{17}$.\\
		Vì $x\in \mathbb{Z}$ nên $A=\lbrace -4;-3;-2;-1;0;1;2;3;4 \rbrace $.
	}
\end{ex}

\begin{ex}%[Phan Quốc Trí, Bai Giảng T10(2022)]%[0D1Y2-1]
	Cho tập hợp $A=\left\{ x \in \mathbb{N}| x^2+2x-3=0 \right\}$. Mệnh đề nào sau đây là đúng?
	\choice
	{\True $-3 \notin A$}
	{$A=\left\{1;-3\right\}$}
	{$1 \notin A$}
	{$A=\left\{1;3\right\}$}
	\loigiai{
		Ta có $x^2+2x-3=0 \Leftrightarrow \hoac{&x=1\\&x=-3}$. Do $x \in \mathbb{N}$ nên $A=\left\{1\right\}$. Do đó $-3 \notin A$.
	}
\end{ex}
\begin{ex}%[Phan Quốc Trí, Bai Giảng T10(2022)]%[0D1Y2-2]
	Cho tập $A=\{a;b;5\}$. Số tập con của tập $A$ là
	\choice
	{$5$}
	{\True $8$}
	{$7$}
	{$4$}
	\loigiai{Tập con của $A$ là $\varnothing, \{a\}, \{b\},\{5\}, \{a;b\}, \{a;5\}, \{b;5\}, \{a;b;5\} $. Vậy số tập con của $A$ là $8$.
	}
\end{ex}
\begin{ex}%[Phan Quốc Trí, Bai Giảng T10(2022)]%[0D1Y2-2]
	Có bao nhiêu tập $ X $ thỏa mãn $ \{a;b\} \subset X \subset \{1;2;a;b\}$?
	\choice
	{$3$}
	{$2$}
	{\True $4$}
	{$5$}
	\loigiai{
		Các tập $ X $ thỏa mãn là $ \{a;b\} $, $ \{1;a;b\} $, $ \{2;a;b\} $, $ \{1;2;a;b\} $.
	}
\end{ex}
\begin{ex}%[Phan Quốc Trí, Bai Giảng T10(2022)]%[0D1B4-1]
	Cho tập hợp $A=\{x\in\mathbb{R}|-3<x\le 3\}$. Mệnh đề nào dưới đây đúng?
	\choice
	{$A=\{-2;-1;0;1;2;3\}$}
	{\True $A=(-3;3]$}
	{$A=[-3;3]$}
	{$A=[-3;3)$}
	\loigiai{
		Từ giả thiết, có $A=(-3;3]$.
	}
\end{ex}
\begin{ex}%[Phan Quốc Trí, Bai Giảng T10(2022)]%[0D1B2-2]
	Cho tập hợp $A = \left\{x \in \mathbb{N}\big| x^2 + 8x + 15 = 0\right\}$. Khẳng định nào sau đây đúng?
	\choice
	{$A = \left\{-3;-5\right\}$}
	{\True $A =\varnothing$}
	{$A = \left\{\varnothing\right\}$}
	{$A = \left\{0\right\}$}
	\loigiai{
		Phương trình $x^2+8x+15=0$ có hai nghiệm $x_1=-3$, $x_2=-5$. Tuy nhiên $x_1, x_2\notin \mathbb{N}$. Vậy $A=\varnothing$.
	}
\end{ex}
\begin{ex}%[Phan Quốc Trí, Bai Giảng T10(2022)]%[0D1B2-2]
	Gọi $A$ là tập hợp tất cả các hình bình hành và $B$ là tập hợp tất cả các hình chữ nhật. Trong các kết luận sau, kết luận nào đúng?
	\choice
	{$A \subset B$}
	{\True $B \subset A$}
	{$A=B$}
	{$A \cap B=\varnothing$}
	\loigiai{
		Ta có hình chữ nhật là hình bình hành có một góc vuông nên $B \subset A$.
	}
\end{ex}

\begin{ex}%[Phan Quốc Trí, Bai Giảng T10(2022)]%[0D1Y2-2]
	Khẳng định nào sau đây là đúng?
	\choice
	{$ \mathbb{R}\subset\mathbb{Q} $}
	{$ \mathbb{Z}\subset\mathbb{N} $}
	{$ \mathbb{Q}\subset\mathbb{Z} $}
	{\True $ \mathbb{N}\subset\mathbb{R} $}
	\loigiai{
		Ta có $ \mathbb{N}\subset\mathbb{Z}\subset\mathbb{Q}\subset\mathbb{R} $.
	}
\end{ex}
\begin{ex}%[Phan Quốc Trí, Bai Giảng T10(2022)]%[0D1Y3-1]
	Cho hai tập hợp $X=\left\{1;3;5;8\right\}$ và $Y=\left\{3;5;7;9\right\}$. Tập hợp $X\cup Y$ bằng
	\choice
	{$\left\{1;7;9\right\}$}
	{$\left\{3;5\right\}$}
	{$\left\{1;3;5\right\}$}
	{\True $\left\{1;3;5;7;8;9\right\}$}
	\loigiai{
		Ta có $X\cup Y=\left\{1;3;5;7;8;9\right\}$.
	}
\end{ex}
\begin{ex}%[Phan Quốc Trí, Bai Giảng T10(2022)]%[0D1Y3-1]
	Cho $A=\{2;3;6;7\}, B=\{3;6;8\}$. Tập hợp $A\cap B$ bằng
	\choice
	{$\{3;6;8\}$}
	{$\{2;3;6;7;8\}$}
	{\True $\{3;6\}$}
	{$\{2;7\}$}
	\loigiai{
		Ta có $A\cap B=\{3;6\}.$
	}
\end{ex}
\begin{ex}%[Phan Quốc Trí, Bai Giảng T10(2022)]%[0D1Y3-2]
	Cho hai tập hợp $A=\{2;4;6;9\}$, $B=\{1;2;3;4\}$. Tập $A \setminus B$ bằng tập hợp nào sau đây?
	\choice
	{$\{2;4\}$}
	{$\{1;3\}$}
	{\True $\{6;9\}$}
	{$\{6;9;1;3\}$}
	\loigiai{
		Ta có: $A \setminus B=\{6;9\}$.}
\end{ex}

\begin{ex}%[Phan Quốc Trí, Bai Giảng T10(2022)]%[0D1Y3-2] 
	Cho tập $ X=\left\lbrace 0;1;2;3;4;5 \right\rbrace$ và tập $ A=\left\lbrace 0;2;4 \right\rbrace  $. Tìm phần bù của $ A $ trong $ X $.	
	\choice
	{$ \varnothing $}
	{$ \left\lbrace 2;4 \right\rbrace $ }
	{$ \left\lbrace 0;1;3 \right\rbrace  $}
	{\True $ \left\lbrace 1;3;5 \right\rbrace  $}
	\loigiai
	{
		Ta có phần bù của $ A $ trong $ X $ bằng tập $ X\setminus A=\left\lbrace 1;3;5 \right\rbrace  $.	
	}
\end{ex}
\begin{ex}%[Phan Quốc Trí, Bai Giảng T10(2022)]%[0D1B3-1]
	Cho hai tập hợp $A=(-3;3)$ và $B=(0;+\infty)$. Tìm  $A \cup B$.
	\choice
	{\True $ A \cup B = (-3;+\infty) $}
	{$ A \cup B = [-3;+\infty) $}
	{$ A \cup B = [-3;0) $}
	{$ A \cup B = (0;3) $}
	\loigiai{
		\immini{Tập hợp $A=(-3;3)$ có biểu diễn là}{ 
			\begin{tikzpicture}[scale=1, font=\footnotesize, line join = round, line cap = round, >=stealth]
				\draw[thick,->] (-2,0)node[below=6pt]{$-\infty$} -- (0,0) node[scale=1.5]{\bf ( } node[below=6pt]{$-3$} -- (3,0) node[scale=1.5]{\bf )} node[below=6pt]{$3$} -- (5,0)node[below=6pt]{$+\infty$};
				\foreach \i in {1,...,10} 
				\draw ($(0,0)-(.2*\i,0)$) node[scale=.6]{/};
				\draw ($(0,0)$) node[scale=.6]{/};
				\foreach \i in {1,...,9} 
				\draw ($(3,0)+(.2*\i,0)$) node[scale=.6]{/};
				\draw ($(3,0)$) node[scale=.6]{/};
		\end{tikzpicture}}
		\immini{Tập hợp $B=(0;+\infty)$ có biểu diễn là}{
			\begin{tikzpicture}[scale=1, font=\footnotesize, line join = round, line cap = round, >=stealth]
				\draw[thick,->] (-2,0)node[below=6pt]{$-\infty$}--(1,0) node[scale=1.5]{\bf (} node[below=6pt]{$0$}-- (5,0)node[below=6pt]{$+\infty$};
				\foreach \i in {1,...,15} 
				\draw ($(1,0)-(.2*\i,0)$) node[scale=.6,rotate=60]{/};
				\draw ($(1,0)$) node[scale=.6,rotate=60]{/};
				%	\foreach \i in {1,...,5} 
				%	\draw ($(4,0)+(.2*\i,0)$) node[scale=.6,rotate=60]{/};
		\end{tikzpicture}}
		\noindent Do đó $A \cup B=(-3;+\infty)$.
	}
\end{ex}
\begin{ex}%[Phan Quốc Trí, Bai Giảng T10(2022)]%[0D1B3-1]
	Cho tập hợp $X=(-\infty;2] \cap (-6;+\infty)$. Khẳng định nào sau đây là đúng?
	\choice
	{\True $X=(-6;2]$}
	{$(-6;+\infty)$}
	{$X=(-\infty;+\infty)$}
	{$X=(-\infty;2]$}
	\loigiai{
		Ta có: $X=(-\infty;2] \cap (-6;+\infty)=(-6;2]$.}
\end{ex}

\begin{ex}%[Phan Quốc Trí, Bai Giảng T10(2022)]%[0D1B3-2]
	Cho tập hợp $ A=[-2;3] $ và $ B=(1;5] $. Khi đó $ A\setminus B $ là
	\choice
	{$(-2;1]$}
	{$(-2;-1)$}
	{$[-2;1) $}
	{\True $[-2;1]$}
	\loigiai{
		Ta có $ A\setminus B =[-2;3]\setminus(1;5]= [-2;1]$. 
	}
\end{ex}

 \begin{ex}%[Phan Quốc Trí, Bai Giảng T10(2022)]%[0D1B3-2]
 	Cho tập hợp $A=\left\{x \in \mathbb{R} |  0 \le x+2 <5 \right\}$. Tập hợp $C_{\mathbb{R}}A$ bằng
 	\choice
 	{$(-\infty;-2)$}
 	{$(-\infty;-2] \cup (3;+\infty)$}
 	{\True $(-\infty;-2) \cup [3;+\infty)$}
 	{$[3;+\infty)$}
 	\loigiai{
 		Ta có: $C_{\mathbb{R}}A=(-\infty;-2) \cup [3;+\infty)$.}
 \end{ex}
\begin{ex}%[Phan Quốc Trí, Bai Giảng T10(2022)]%[0D1B3-3]
	Một lớp học có $25$ học sinh giỏi môn Toán, $23$ học sinh giỏi môn Lý, $14$ học sinh giỏi cả môn Toán và Lý và có $6$ học sinh không giỏi môn nào cả. Hỏi lớp đó có bao nhiêu học sinh?
	\choice
	{$26$}
	{$54$}
	{$68$}
	{\True $40$}
	\loigiai
	{
		Vì có $25$ học sinh giỏi môn Toán và $14$ học sinh giỏi cả môn Toán và Lý nên có $25-14=11$ học sinh chỉ giỏi môn Toán mà không giỏi môn Lý. \\
		Vì có $23$ học sinh giỏi môn Lý và $14$ học sinh giỏi cả môn Toán và Lý nên có $23-14=9$ học sinh chỉ giỏi môn Lý mà không giỏi môn Toán. \\
		Vậy lớp đó có $11+9+14+6=40$ học sinh.
	}
\end{ex}
\begin{ex}%[Phan Quốc Trí, Bai Giảng T10(2022)]%[0D1B3-3]
	Mỗi học sinh của lớp $10A$ đều chơi bóng đá hoặc bóng chuyền. Biết rằng có $25$ bạn chơi bóng đá, $20$ bạn chơi bóng chuyền và $10$ bạn chơi cả $2$ môn thể thao. Hỏi lớp $10A$ có bao nhiêu học sinh.
	\choice
	{$30$}
	{$55$}
	{$45$}
	{\True $35$}
	\loigiai{
		Ngoài sơ đồ Ven ta có thể dùng công thức số phần tử. Gọi $A$ là tập hợp các học sinh chơi bóng đá, $B$ là tập các học sinh chơi bóng chuyền. Do đó $A\cap B$ là tập các học sinh chơi cả hai môn. Ta có
		$$|A|=25, |B|=20, |A \cap B| =10.$$
		Số học sinh cả lớp là số phần tử của tập $A \cup B$. Theo công thức ta có $|A \cup B| = 25+20-10=35$ (học sinh).
	}
\end{ex}
\begin{ex}%[Phan Quốc Trí, Bai Giảng T10(2022)]%[0D1B3-1] 
	Cho các tập hợp $M=[-3;6]$ và $N=(-\infty; -2)\cup (3;+\infty)$. Khi đó $M\cap N$ là
	\choice
	{$(-\infty;-2)\cup (3;6)$}
	{$(-\infty;-2)\cup [3;+\infty)$}
	{\True $[-3;-2)\cup (3;6]$}
	{$(-3;-2)\cup (3;6)$}
	\loigiai{
		Biểu diễn trục số:  
		\begin{center}
			\begin{tikzpicture}
				\draw[->](-4,0)->(7,0);
				\IntervalLR{-4}{-3}
				\IntervalGR{}{}{\big[}{-3}
				\IntervalLR{6}{6.9}
				\IntervalGR{\big]}{6}{}{}
				\IntervalLR{-2}{3}
				\def\skipInterval{0.5cm}
				\IntervalGLF{\big)}{-2}{\big(}{3}
			\end{tikzpicture}
		\end{center}
		Từ hình vẽ, ta có
		Khi đó: $M\cap N= [-3;-2)\cup (3;6]$.}
\end{ex}

\begin{ex}%[Phan Quốc Trí, Bai Giảng T10(2022)]%%[Trần Ngọc Minh]%[301-320-Huỳnh Thanh Tiến]%[0D1B4-1]
	Tập hợp $(1;2)\cap\mathbb{N}$ là tập hợp nào sau đây?
	\choice
	{$\{1;2\}$}
	{$\{1\}$}
	{\True $\varnothing$}
	{$\{2\}$}
	\loigiai{
		Ta có $\mathbb{N}=\{0,1,2,\ldots\}$.
		Do đó 	$(1;2)\cap\mathbb{N}=\varnothing$.
	}
\end{ex}	
\begin{ex}%[Phan Quốc Trí, Bai Giảng T10(2022)]%[0D1B4-1]
	Cho $A=(-5;1]$, $B=[3;+\infty)$, $C=(-\infty;-2)$. Khẳng định nào sau đây đúng?
	\choice
	{$A\cap C=[-5;-2]$}
	{$A\cup B=(-5;+\infty)$}
	{$B\cup C=(-\infty;+\infty)$}
	{$\True B\cap C= \varnothing$}
	\loigiai{
		\begin{center}
			\begin{tikzpicture}
				\draw[->](-1,0)->(5,0);
				\IntervalLR{-1}{3}
				\def\skipInterval{0.5cm}
				\IntervalGRF{}{}{\big[}{3}
				\IntervalLR{4}{4.8}
				\def\skipInterval{0.5cm}
			\end{tikzpicture}\\
			\begin{tikzpicture}
				\draw[->](-1,0)->(5,0);
				\IntervalLR{-1}{1/2}
				\def\skipInterval{0.5cm}
				\IntervalLR{2}{4.9}
				\def\skipInterval{0.5cm}
				\IntervalGRF{\big)}{-2}{}{}
			\end{tikzpicture}
		\end{center}
		Từ biểu diễn tập nghiệm của $B$ và $C$ ta thấy $B\cap C= \varnothing$.
	}
\end{ex}
\begin{ex}%[Phan Quốc Trí, Bai Giảng T10(2022)]%%[0-HK2-2021, Trường THPT Trần Phú, Hải Phòng, năm học 2017 - 2018]%[Trần Quang Thạnh]%[0D1Y4-2]
	Cho tập hợp $A=[-2;3]$ và $B=(-2;5]$. Khi đó $A\setminus B$ là
	\choice
	{$[-2;5] $}
	{$(-2;-1) $}
	{$(3;5) $}
	{\True $\left\{-2\right\} $}
	\loigiai{
		Ta có $A\setminus B=\left\{-2\right\}$.
	}
\end{ex}
\begin{ex}%[Phan Quốc Trí, Bai Giảng T10(2022)]%[0D1B4-2]
	Cho các tập $A=\left\{x\in\mathbb{R}\mid x\ge -1\right\}$; $B=\left\{x\in\mathbb{R}\mid x<3\right\}$. Tập hợp $\mathbb{R}\setminus\left(A\cap B\right)$ là
	\choice
	{$[-1;3)$}
	{$(-\infty;-1]\cup(3;+\infty)$}
	{\True $(-\infty;-1)\cup[3;+\infty)$}
	{$(-1;3]$}
	\loigiai{
		Ta có $A\cap B=[-1;3)$, suy ra $\mathbb{R}\setminus\left(A\cap B\right)=(-\infty;-1)\cup[3;+\infty)$.
	}
\end{ex}
\begin{ex}%[Phan Quốc Trí, Bai Giảng T10(2022)]%[0D1K2-2]
	Tìm tất cả các giá trị của $m$ để đoạn $[m;m+3]$ là tập con của nửa khoảng $(-2;9]$.
	\choice
	{$-2\le m\le 6$}
	{$-2\le m<6$}
	{\True $-2<m\le 6$}
	{$-2<m<6$}
	\loigiai{
		Đoạn $[m;m+3]$ là tập con của nửa khoảng $(-2;9]$ khi và chỉ khi $\heva{& -2<m \\& m+3\le 9}\Leftrightarrow \heva{& -2<m \\& m\le 6}\Leftrightarrow -2<m\le 6.$
	}
\end{ex}



\noindent\textbf{II. PHẦN TỰ LUẬN}
\begin{bt}%[Phan Quốc Trí, Bai Giảng T10(2022)]%[0D1B4-1] 
	Cho $A=\left \{x \in \mathbb{R} \Big|  x \geq 3\right \}$ và $B=(-2 ; 7]$.  Tìm các tập hợp $A \cap B$, $A \cup B$.
	\loigiai{
		Ta có $A=\left \{x \in \mathbb{R} \Big|  x \geq 3\right \} = [3;+\infty )$. \\
		$A \cap B =[3;+\infty ) \cap (-2 ; 7] =    [3;7 ]$. \\
		$A \cup B = [3;+\infty ) \cup (-2 ; 7] = (-2;+\infty)$. 
	}
\end{bt}
\begin{bt}%[Phan Quốc Trí, Bai Giảng T10(2022)]%[0D1B3-2]
	Cho hai tập hợp $A=\left\{ 0;2 \right\}$ và $B=\left\{ 0;1;2;3;4 \right\}$. Tìm tất cả các tập hợp $X$ thỏa mãn $A\cup X=B$.
	\loigiai { 
		Vì $A\cup X=B$ nên $X$ chắc chắn có chứa các phần tử $1;3;4$\\
		Các tập $X$ có thể là $\left\{ 1;3;4 \right\},\,\left\{ 1;3;4;0 \right\},\,\left\{ 1;3;4;2 \right\},\,\left\{ 1;3;4;0;2 \right\}$}
\end{bt}
\begin{bt}%[Phan Quốc Trí, Bai Giảng T10(2022)]%[0D1K4-1]
	Cho hai tập hợp $A=(2m-1;m+3)$, $B=(-4;5)$. Tìm $m$ để
 $A\cap B=\varnothing $.
	\loigiai{
		Điều kiện: $2m-1<m+3 \Leftrightarrow m<4$.\\
	 Ta có  $A\cap B=\varnothing $ khi và chỉ khi $\left[
			\begin{aligned}
				&m+3\leqslant -4\\
				&2m-1\geqslant 5
			\end{aligned}\right. \Leftrightarrow \left[
			\begin{aligned}
				&m\leqslant -7\\
				&m\geqslant 3.
			\end{aligned}\right. $\\
			Đối chiếu điều kiện, ta được $m\leqslant -7$ hoặc $3\leqslant m<4$ thỏa yêu cầu bài toán.
	}
\end{bt}
\begin{bt}%[Phan Quốc Trí, Bai Giảng T10(2022)]%[0D1B3-3]
Lớp 10A có $45$ học sinh, trong đó có $18$ học sinh tham gia cuộc thi vẽ đồ họa trên máy tính, $24$ học sinh tham gia cuộc thi tin học văn phòng cấp trường và $9$ học sinh không tham gia cả hai cuộc thi này. Hỏi lớp 10A có bao nhiêu học sinh tham gia đồng thời cả hai cuộc thi?	
	\loigiai{
\immini{
Gọi $A$ là tập hợp các học sinh tham gia cuộc thi vẽ đồ họa trên máy tính. Suy ra $n(A)= 18$.\\
$B$ là tập hợp các học sinh tham gia  cuộc thi tin học văn phòng cấp trường. Suy ra $n(B)= 24$.\\
Ta có $A \cap B$ là tập hợp các học sinh tham gia đồng thời cả hai cuộc thi.\\
 $A \cup B$ là tập hợp các học sinh tham gia cuộc thi vẽ đồ họa trên máy tính hoặc tham gia  cuộc thi tin học văn phòng cấp trường. 
}{
	\begin{tikzpicture}
		\draw[] (0,0) circle ( 1.0cm);
		\draw (1.0,0) circle ( 1.0 cm);
		\draw (-0.5,0) node {$18$};
		\draw (1.5,0) node {$24$};
		\draw (-1.3,-0.75) node {$9$};
		\draw (1.5,1.5) node {$45$};
		\node[rectangle,
		draw = lightgray,
		minimum width = 4cm, 
		minimum height = 2.5cm] (r) at (0.5,0) {};
	\end{tikzpicture}
}
$$n \left(A \cup B \right)= 45-9=36.$$	
$$n \left( A \cap B \right) = n(A)+ n(B)-n(A \cup B)=18+24-36= 6.$$
Vậy có $6$  học sinh tham gia đồng thời cả hai cuộc thi.
	}
\end{bt}
\Closesolutionfile{ans}

\newpage
\begin{indapan}{10}
{ans/ans-KT-101}
\end{indapan}

\section*{Đề kiểm tra Chương 1}
\subsection*{Đề số 2}
\setcounter{ex}{0}\setcounter{bt}{0}
\Opensolutionfile{ans}[ans/ans-KT-102]
\noindent\textbf{I. PHẦN TRẮC NGHIỆM}
%[Thi thử, Sở GD và ĐT - Điện Biên, 2018]%[Dương BùiĐức, 12EX10]%[2D2Y3-2]
\begin{ex}%[Phan Quốc Trí, Bai Giảng T10(2022)]%[0D1Y1-1]
	Trong các câu sau, câu nào không phải là mệnh đề?
	\choice
	{Hình bình hành có bốn cạnh bằng nhau}
	{\True Chúc bạn may mắn}
	{Số $8$ là số chính phương}
	{Cà Mau là tên một tỉnh của nước Việt Nam}
	\loigiai{
		Mệnh đề là một câu khẳng định \textbf{đúng} hoặc một câu khẳng định \textbf{sai}.
	}
\end{ex}
\begin{ex}%[Phan Quốc Trí, Bai Giảng T10(2022)]%[0D1Y1-1]
	Trong các câu sau, câu nào là mệnh đề?
	\choice
	{$x^2+x=2$}
	{Hôm nay trời đẹp quá!}
	{$2n+1$ chia hết cho 3}
	{\True Số 15 là một số nguyên tố}
	\loigiai{
		\begin{itemize}
			\item $x^2+x=2$  là một khẳng định, nhưng không là mệnh đề. 
			\item  Hôm nay trời đẹp quá! không phải là mệnh đề.
			\item Câu $2n+1$ chia hết cho 3 là một khẳng định, nhưng không là mệnh đề. 
			\item  Số 15 là một số nguyên tố là một mệnh đề sai.
		\end{itemize}
	}
\end{ex}
\begin{ex}%[Phan Quốc Trí, Bai Giảng T10(2022)]%[0D1Y1-2]
	Trong các mệnh đề sau, mệnh đề nào là mệnh đề \textbf{sai}?
	\choice
	{\True $\sqrt{3}$ là số nguyên}
	{$6$ chia hết cho $2$}
	{$5$ chia hết cho $5$}
	{$30$ là một số chẵn}
	\loigiai{
		Trong các mệnh đề đã cho, mệnh đề sai là \lq\lq$\sqrt{3}$ là số nguyên\rq\rq.
	}
\end{ex}
\begin{ex}%[Phan Quốc Trí, Bai Giảng T10(2022)]%[0D1Y1-2]
	Cho mệnh đề chứa biến $P(x): 3x+5\le x^2$ với $x$ là số thực. Mệnh đề nào sau đây là đúng?
	\choice
	{$P(3)$}
	{$P(4)$}
	{$P(1)$}
	{\True $P(5)$}
	\loigiai{
		$P(3)= 3.3+5\leqslant 3^2 \Leftrightarrow 14\leqslant 9$ là mệnh đề sai.\\
		$P(4) =3.4+5\leqslant 4^2 \Leftrightarrow 17\leqslant 16$ là mệnh đề sai.\\
		$P(1)=3.1+5\leqslant 1^2 \Leftrightarrow 8\leqslant 1$ là mệnh đề sai.\\
		$P(5) =3.5+5\leqslant 5^2 \Leftrightarrow 20\leqslant 25$ là mệnh đề đúng.}
\end{ex}
\begin{ex}%[Phan Quốc Trí, Bai Giảng T10(2022)]%[0D1Y1-3]
	Cho mệnh đề $P\colon$ \lq\lq  $9$ là số chia hết cho $3$\rq\rq. Mệnh đề phủ định của mệnh đề $P$ là
	\choice
	{$\overline{P}\colon$\lq\lq  $9$ là ước của $3$\rq\rq}
	{$\overline{P}\colon$\lq\lq  $9$ là bội của $3$\rq\rq}
	{\True $\overline{P}\colon$ \lq\lq  $9$ là số không chia hết cho $3$\rq\rq}
	{$\overline{P}\colon$\lq\lq  $9$ là số lớn hơn $3$\rq\rq}
	\loigiai{Mệnh đề $P\colon$\lq\lq  $9$ là số chia hết cho $3$\rq\rq      có mệnh đề phủ định là $\overline{P}\colon$ \lq\lq  $9$ là số không chia hết cho $3$\rq\rq.}
\end{ex}
\begin{ex}%[Phan Quốc Trí, Bai Giảng T10(2022)]%[0D1Y1-3]
	Cho mệnh đề \lq\lq$\forall x\in \mathbb{R},\, x^2+1>0$\rq \rq. Mệnh đề phủ định của mệnh đề đã cho là
	\choice
	{\lq\lq$\forall x\in \mathbb{R},\, x^2+1\leq 0$\rq \rq}
	{\lq\lq$\forall x\in \mathbb{R},\, x^2+1<0$\rq \rq}
	{\True \lq\lq$\exists x\in \mathbb{R},\, x^2+1\leq 0$\rq \rq}
	{\lq\lq$\exists x\in \mathbb{R},\, x^2+1>0$\rq \rq}
	\loigiai{
		Mệnh đề phủ định của mệnh đề \lq\lq$\forall x\in \mathbb{R},\, x^2+1>0$\rq \rq\, là \lq\lq$\exists x\in \mathbb{R},\, x^2+1\leq 0$\rq \rq.}
\end{ex}
\begin{ex}%[Phan Quốc Trí, Bai Giảng T10(2022)]%[0D1Y1-4]
	Cho mệnh đề $P\colon$ ``Tam giác $ABC$ cân tại $A$'', mệnh đề $Q\colon$ ``$AB=AC$''. Phát biểu mệnh đề ``$P$ kéo theo $Q$'' là
	\choice
	{Nếu $AB=AC$ thì tam giác $ABC$ cân tại $A$}
	{\True Nếu tam giác $ABC$ cân tại $A$ thì $AB=AC$}
	{Tam giác $ABC$ cân tại $B$ là điều kiện cần và đủ để $AB=AC$}
	{Tam giác $ABC$ cân tại $A$ khi và chỉ khi $AB=AC$}
	\loigiai{
		Mệnh đề ``$P$ kéo theo $Q$'' là ``Nếu tam giác $ABC$ cân tại $A$ thì $AB=AC$''.
	}
\end{ex}
\begin{ex}%[Phan Quốc Trí, Bai Giảng T10(2022)]%[0D1Y1-4]
	Cho mệnh đề $P$: \lq\lq Nếu tam giác có hai đường trung tuyến bằng nhau thì đó là tam giác cân\rq\rq. Mệnh đề nào sau đây là mệnh đề đảo của $P$?
	\choice
	{Tam giác có hai đường trung tuyến bằng nhau thì nó là tam giác cân}
	{\True Nếu tam giác $ABC$ cân thì tam giác đó có hai đường trung tuyến bằng nhau }
	{Tam giác là tam giác cân khi và chỉ khi nó có hai đường trung tuyến bằng nhau}
	{Tam giác là tam giác cân nếu nó có hai đường trung tuyến bằng nhau}
	\loigiai{
		Mệnh đề đảo là \lq \lq Nếu tam giác $ABC$ cân thì tam giác đó có hai đường trung tuyến bằng nhau.\rq \rq		
	}
\end{ex}
\begin{ex}%[Phan Quốc Trí, Bai Giảng T10(2022)]%[0D1Y1-5]
	Mệnh đề "Bình phương mọi số thực đều không âm" mô tả mệnh đề nào dưới đây?
	\choice
	{"$\forall n\in\mathbb{N}: n^2\geq 0$"}
	{"$\exists x\in\mathbb{R}:x^2\geq 0$"}
	{\True "$\forall x\in\mathbb{R}: x^2\geq 0$"}
	{"$\forall x\in\mathbb{R}:x^2>0$"}
	\loigiai{
		Mệnh đề trên được viết lại dạng: \lq \lq $\forall x\in\mathbb{R}: x^2\geq 0$\rq \rq	
	}
\end{ex}
\begin{ex}%[Phan Quốc Trí, Bai Giảng T10(2022)]%[0D1B1-5]
	Mệnh đề nào sau đây là đúng?
	\choice
	{$\forall n\in\mathbb{N}\colon n^2>n$}
	{$\forall x\in\mathbb{R}\colon x^2<2$}
	{$\forall x\in\mathbb{Z}\colon 2x>1$}
	{\True $\exists x\in\mathbb{R}\colon x^2>x$}
	\loigiai
	{
		\begin{itemize}
			\item Mệnh đề ``$\forall n\in\mathbb{N}\colon n^2>n$'' sai vì với $n=1$ thì $n^2=1=n$.
			\item Mệnh đề ``$\forall x\in\mathbb{R}\colon x^2<2$'' sai vì với $x=2$ thì $x^2=4>2$.
			\item Mệnh đề ``$\forall x\in\mathbb{Z}\colon 2x>1$'' sai vì với $x=0$ thì $2x=0<1$.
			\item Mệnh đề ``$\exists x\in\mathbb{R}\colon x^2>x$'' đúng vì với $x=2$ thì $x^2=4$ nên $x^2>x$.
		\end{itemize}
	}
\end{ex}
\begin{ex}%[Phan Quốc Trí, Bai Giảng T10(2022)]%[0D1Y2-1]
	Hãy liệt kê các phần tử của tập hợp $X = \left\{ x \in \mathbb{Z} | 2x^2-5x+3=0 \right\}$.
	\choice
	{$X = \left\{1;\dfrac{3}{2} \right\}$}
	{\True $X = \{ 1 \}$}
	{$X = \left\{ \dfrac{3}{2} \right\}$}
	{$X = \varnothing$}
	\loigiai{
		Ta có $2x^2-5x+3=0\Leftrightarrow \hoac{&x=1\in \mathbb{Z}\\&x=\dfrac{3}{2}\notin \mathbb{Z}}$.\\
		Vậy $X = \{ 1 \}$.		
	}
\end{ex}
\begin{ex}%[Phan Quốc Trí, Bai Giảng T10(2022)]%[0D1B2-1]
	Viết tập hợp $A=\lbrace x \in \mathbb{Z}|x^2<17 \rbrace $ theo cách liệt kê các phần tử, ta được tập hợp nào sau đây?
	\choice{\True $ \lbrace -4;-3;-2;-1;0;1;2;3;4 \rbrace $}
	{$ \lbrace 1;2;3;4 \rbrace $}
	{$\lbrace 0;1;2;3;4  \rbrace $}
	{$ \lbrace -4; -3;-2;-1 \rbrace $}
	\loigiai{Ta có $x^2<17\Leftrightarrow \left|x\right|<\sqrt{17}\Leftrightarrow -\sqrt{17}<x<\sqrt{17}$.\\
		Vì $x\in \mathbb{Z}$ nên $A=\lbrace -4;-3;-2;-1;0;1;2;3;4 \rbrace $.
	}
\end{ex}

\begin{ex}%[Phan Quốc Trí, Bai Giảng T10(2022)]%[0D1Y2-1]
	Cho tập hợp $A=\left\{ x \in \mathbb{N}| x^2+2x-3=0 \right\}$. Mệnh đề nào sau đây là đúng?
	\choice
	{\True $-3 \notin A$}
	{$A=\left\{1;-3\right\}$}
	{$1 \notin A$}
	{$A=\left\{1;3\right\}$}
	\loigiai{
		Ta có $x^2+2x-3=0 \Leftrightarrow \hoac{&x=1\\&x=-3}$. Do $x \in \mathbb{N}$ nên $A=\left\{1\right\}$. Do đó $-3 \notin A$.
	}
\end{ex}
\begin{ex}%[Phan Quốc Trí, Bai Giảng T10(2022)]%[0D1Y2-2]
	Cho tập $A=\{a;b;5\}$. Số tập con của tập $A$ là
	\choice
	{$5$}
	{\True $8$}
	{$7$}
	{$4$}
	\loigiai{Tập con của $A$ là $\varnothing, \{a\}, \{b\},\{5\}, \{a;b\}, \{a;5\}, \{b;5\}, \{a;b;5\} $. Vậy số tập con của $A$ là $8$.
	}
\end{ex}
\begin{ex}%[Phan Quốc Trí, Bai Giảng T10(2022)]%[0D1Y2-2]
	Có bao nhiêu tập $ X $ thỏa mãn $ \{a;b\} \subset X \subset \{1;2;a;b\}$?
	\choice
	{$3$}
	{$2$}
	{\True $4$}
	{$5$}
	\loigiai{
		Các tập $ X $ thỏa mãn là $ \{a;b\} $, $ \{1;a;b\} $, $ \{2;a;b\} $, $ \{1;2;a;b\} $.
	}
\end{ex}
\begin{ex}%[Phan Quốc Trí, Bai Giảng T10(2022)]%[0D1B4-1]
	Cho tập hợp $A=\{x\in\mathbb{R}|-3<x\le 3\}$. Mệnh đề nào dưới đây đúng?
	\choice
	{$A=\{-2;-1;0;1;2;3\}$}
	{\True $A=(-3;3]$}
	{$A=[-3;3]$}
	{$A=[-3;3)$}
	\loigiai{
		Từ giả thiết, có $A=(-3;3]$.
	}
\end{ex}
\begin{ex}%[Phan Quốc Trí, Bai Giảng T10(2022)]%[0D1B2-2]
	Cho tập hợp $A = \left\{x \in \mathbb{N}\big| x^2 + 8x + 15 = 0\right\}$. Khẳng định nào sau đây đúng?
	\choice
	{$A = \left\{-3;-5\right\}$}
	{\True $A =\varnothing$}
	{$A = \left\{\varnothing\right\}$}
	{$A = \left\{0\right\}$}
	\loigiai{
		Phương trình $x^2+8x+15=0$ có hai nghiệm $x_1=-3$, $x_2=-5$. Tuy nhiên $x_1, x_2\notin \mathbb{N}$. Vậy $A=\varnothing$.
	}
\end{ex}
\begin{ex}%[Phan Quốc Trí, Bai Giảng T10(2022)]%[0D1B2-2]
	Gọi $A$ là tập hợp tất cả các hình bình hành và $B$ là tập hợp tất cả các hình chữ nhật. Trong các kết luận sau, kết luận nào đúng?
	\choice
	{$A \subset B$}
	{\True $B \subset A$}
	{$A=B$}
	{$A \cap B=\varnothing$}
	\loigiai{
		Ta có hình chữ nhật là hình bình hành có một góc vuông nên $B \subset A$.
	}
\end{ex}

\begin{ex}%[Phan Quốc Trí, Bai Giảng T10(2022)]%[0D1Y2-2]
	Khẳng định nào sau đây là đúng?
	\choice
	{$ \mathbb{R}\subset\mathbb{Q} $}
	{$ \mathbb{Z}\subset\mathbb{N} $}
	{$ \mathbb{Q}\subset\mathbb{Z} $}
	{\True $ \mathbb{N}\subset\mathbb{R} $}
	\loigiai{
		Ta có $ \mathbb{N}\subset\mathbb{Z}\subset\mathbb{Q}\subset\mathbb{R} $.
	}
\end{ex}
\begin{ex}%[Phan Quốc Trí, Bai Giảng T10(2022)]%[0D1Y3-1]
	Cho hai tập hợp $X=\left\{1;3;5;8\right\}$ và $Y=\left\{3;5;7;9\right\}$. Tập hợp $X\cup Y$ bằng
	\choice
	{$\left\{1;7;9\right\}$}
	{$\left\{3;5\right\}$}
	{$\left\{1;3;5\right\}$}
	{\True $\left\{1;3;5;7;8;9\right\}$}
	\loigiai{
		Ta có $X\cup Y=\left\{1;3;5;7;8;9\right\}$.
	}
\end{ex}
\begin{ex}%[Phan Quốc Trí, Bai Giảng T10(2022)]%[0D1Y3-1]
	Cho $A=\{2;3;6;7\}, B=\{3;6;8\}$. Tập hợp $A\cap B$ bằng
	\choice
	{$\{3;6;8\}$}
	{$\{2;3;6;7;8\}$}
	{\True $\{3;6\}$}
	{$\{2;7\}$}
	\loigiai{
		Ta có $A\cap B=\{3;6\}.$
	}
\end{ex}
\begin{ex}%[Phan Quốc Trí, Bai Giảng T10(2022)]%[0D1Y3-2]
	Cho hai tập hợp $A=\{2;4;6;9\}$, $B=\{1;2;3;4\}$. Tập $A \setminus B$ bằng tập hợp nào sau đây?
	\choice
	{$\{2;4\}$}
	{$\{1;3\}$}
	{\True $\{6;9\}$}
	{$\{6;9;1;3\}$}
	\loigiai{
		Ta có: $A \setminus B=\{6;9\}$.}
\end{ex}

\begin{ex}%[Phan Quốc Trí, Bai Giảng T10(2022)]%[0D1Y3-2] 
	Cho tập $ X=\left\lbrace 0;1;2;3;4;5 \right\rbrace$ và tập $ A=\left\lbrace 0;2;4 \right\rbrace  $. Tìm phần bù của $ A $ trong $ X $.	
	\choice
	{$ \varnothing $}
	{$ \left\lbrace 2;4 \right\rbrace $ }
	{$ \left\lbrace 0;1;3 \right\rbrace  $}
	{\True $ \left\lbrace 1;3;5 \right\rbrace  $}
	\loigiai
	{
		Ta có phần bù của $ A $ trong $ X $ bằng tập $ X\setminus A=\left\lbrace 1;3;5 \right\rbrace  $.	
	}
\end{ex}
\begin{ex}%[Phan Quốc Trí, Bai Giảng T10(2022)]%[0D1B3-1]
	Cho hai tập hợp $A=(-3;3)$ và $B=(0;+\infty)$. Tìm  $A \cup B$.
	\choice
	{\True $ A \cup B = (-3;+\infty) $}
	{$ A \cup B = [-3;+\infty) $}
	{$ A \cup B = [-3;0) $}
	{$ A \cup B = (0;3) $}
	\loigiai{
		\immini{Tập hợp $A=(-3;3)$ có biểu diễn là}{ 
			\begin{tikzpicture}[scale=1, font=\footnotesize, line join = round, line cap = round, >=stealth]
				\draw[thick,->] (-2,0)node[below=6pt]{$-\infty$} -- (0,0) node[scale=1.5]{\bf ( } node[below=6pt]{$-3$} -- (3,0) node[scale=1.5]{\bf )} node[below=6pt]{$3$} -- (5,0)node[below=6pt]{$+\infty$};
				\foreach \i in {1,...,10} 
				\draw ($(0,0)-(.2*\i,0)$) node[scale=.6]{/};
				\draw ($(0,0)$) node[scale=.6]{/};
				\foreach \i in {1,...,9} 
				\draw ($(3,0)+(.2*\i,0)$) node[scale=.6]{/};
				\draw ($(3,0)$) node[scale=.6]{/};
		\end{tikzpicture}}
		\immini{Tập hợp $B=(0;+\infty)$ có biểu diễn là}{
			\begin{tikzpicture}[scale=1, font=\footnotesize, line join = round, line cap = round, >=stealth]
				\draw[thick,->] (-2,0)node[below=6pt]{$-\infty$}--(1,0) node[scale=1.5]{\bf (} node[below=6pt]{$0$}-- (5,0)node[below=6pt]{$+\infty$};
				\foreach \i in {1,...,15} 
				\draw ($(1,0)-(.2*\i,0)$) node[scale=.6,rotate=60]{/};
				\draw ($(1,0)$) node[scale=.6,rotate=60]{/};
				%	\foreach \i in {1,...,5} 
				%	\draw ($(4,0)+(.2*\i,0)$) node[scale=.6,rotate=60]{/};
		\end{tikzpicture}}
		\noindent Do đó $A \cup B=(-3;+\infty)$.
	}
\end{ex}
\begin{ex}%[Phan Quốc Trí, Bai Giảng T10(2022)]%[0D1B3-1]
	Cho tập hợp $X=(-\infty;2] \cap (-6;+\infty)$. Khẳng định nào sau đây là đúng?
	\choice
	{\True $X=(-6;2]$}
	{$(-6;+\infty)$}
	{$X=(-\infty;+\infty)$}
	{$X=(-\infty;2]$}
	\loigiai{
		Ta có: $X=(-\infty;2] \cap (-6;+\infty)=(-6;2]$.}
\end{ex}

\begin{ex}%[Phan Quốc Trí, Bai Giảng T10(2022)]%[0D1B3-2]
	Cho tập hợp $ A=[-2;3] $ và $ B=(1;5] $. Khi đó $ A\setminus B $ là
	\choice
	{$(-2;1]$}
	{$(-2;-1)$}
	{$[-2;1) $}
	{\True $[-2;1]$}
	\loigiai{
		Ta có $ A\setminus B =[-2;3]\setminus(1;5]= [-2;1]$. 
	}
\end{ex}

\begin{ex}%[Phan Quốc Trí, Bai Giảng T10(2022)]%[0D1B3-2]
	Cho tập hợp $A=\left\{x \in \mathbb{R} |  0 \le x+2 <5 \right\}$. Tập hợp $C_{\mathbb{R}}A$ bằng
	\choice
	{$(-\infty;-2)$}
	{$(-\infty;-2] \cup (3;+\infty)$}
	{\True $(-\infty;-2) \cup [3;+\infty)$}
	{$[3;+\infty)$}
	\loigiai{
		Ta có: $C_{\mathbb{R}}A=(-\infty;-2) \cup [3;+\infty)$.}
\end{ex}
\begin{ex}%[Phan Quốc Trí, Bai Giảng T10(2022)]%[0D1B3-3]
	Một lớp học có $25$ học sinh giỏi môn Toán, $23$ học sinh giỏi môn Lý, $14$ học sinh giỏi cả môn Toán và Lý và có $6$ học sinh không giỏi môn nào cả. Hỏi lớp đó có bao nhiêu học sinh?
	\choice
	{$26$}
	{$54$}
	{$68$}
	{\True $40$}
	\loigiai
	{
		Vì có $25$ học sinh giỏi môn Toán và $14$ học sinh giỏi cả môn Toán và Lý nên có $25-14=11$ học sinh chỉ giỏi môn Toán mà không giỏi môn Lý. \\
		Vì có $23$ học sinh giỏi môn Lý và $14$ học sinh giỏi cả môn Toán và Lý nên có $23-14=9$ học sinh chỉ giỏi môn Lý mà không giỏi môn Toán. \\
		Vậy lớp đó có $11+9+14+6=40$ học sinh.
	}
\end{ex}
\begin{ex}%[Phan Quốc Trí, Bai Giảng T10(2022)]%[0D1B3-3]
	Mỗi học sinh của lớp $10A$ đều chơi bóng đá hoặc bóng chuyền. Biết rằng có $25$ bạn chơi bóng đá, $20$ bạn chơi bóng chuyền và $10$ bạn chơi cả $2$ môn thể thao. Hỏi lớp $10A$ có bao nhiêu học sinh.
	\choice
	{$30$}
	{$55$}
	{$45$}
	{\True $35$}
	\loigiai{
		Ngoài sơ đồ Ven ta có thể dùng công thức số phần tử. Gọi $A$ là tập hợp các học sinh chơi bóng đá, $B$ là tập các học sinh chơi bóng chuyền. Do đó $A\cap B$ là tập các học sinh chơi cả hai môn. Ta có
		$$|A|=25, |B|=20, |A \cap B| =10.$$
		Số học sinh cả lớp là số phần tử của tập $A \cup B$. Theo công thức ta có $|A \cup B| = 25+20-10=35$ (học sinh).
	}
\end{ex}
\begin{ex}%[Phan Quốc Trí, Bai Giảng T10(2022)]%[0D1B3-1] 
	Cho các tập hợp $M=[-3;6]$ và $N=(-\infty; -2)\cup (3;+\infty)$. Khi đó $M\cap N$ là
	\choice
	{$(-\infty;-2)\cup (3;6)$}
	{$(-\infty;-2)\cup [3;+\infty)$}
	{\True $[-3;-2)\cup (3;6]$}
	{$(-3;-2)\cup (3;6)$}
	\loigiai{
		Biểu diễn trục số:  
		\begin{center}
			\begin{tikzpicture}
				\draw[->](-4,0)->(7,0);
				\IntervalLR{-4}{-3}
				\IntervalGR{}{}{\big[}{-3}
				\IntervalLR{6}{6.9}
				\IntervalGR{\big]}{6}{}{}
				\IntervalLR{-2}{3}
				\def\skipInterval{0.5cm}
				\IntervalGLF{\big)}{-2}{\big(}{3}
			\end{tikzpicture}
		\end{center}
		Từ hình vẽ, ta có
		Khi đó: $M\cap N= [-3;-2)\cup (3;6]$.}
\end{ex}


\begin{ex}%[Phan Quốc Trí, Bai Giảng T10(2022)]%%[Trần Ngọc Minh]%[301-320-Huỳnh Thanh Tiến]%[0D1B4-1]
	Tập hợp $(1;2)\cap\mathbb{N}$ là tập hợp nào sau đây?
	\choice
	{$\{1;2\}$}
	{$\{1\}$}
	{\True $\varnothing$}
	{$\{2\}$}
	\loigiai{
		Ta có $\mathbb{N}=\{0,1,2,\ldots\}$.
		Do đó 	$(1;2)\cap\mathbb{N}=\varnothing$.
	}
\end{ex}	
\begin{ex}%[Phan Quốc Trí, Bai Giảng T10(2022)]%[0D1B4-1]
	Cho $A=(-5;1]$, $B=[3;+\infty)$, $C=(-\infty;-2)$. Khẳng định nào sau đây đúng?
	\choice
	{$A\cap C=[-5;-2]$}
	{$A\cup B=(-5;+\infty)$}
	{$B\cup C=(-\infty;+\infty)$}
	{$\True B\cap C= \varnothing$}
	\loigiai{
		\begin{center}
			\begin{tikzpicture}
				\draw[->](-1,0)->(5,0);
				\IntervalLR{-1}{3}
				\def\skipInterval{0.5cm}
				\IntervalGRF{}{}{\big[}{3}
				\IntervalLR{4}{4.8}
				\def\skipInterval{0.5cm}
			\end{tikzpicture}\\
			\begin{tikzpicture}
				\draw[->](-1,0)->(5,0);
				\IntervalLR{-1}{1/2}
				\def\skipInterval{0.5cm}
				\IntervalLR{2}{4.9}
				\def\skipInterval{0.5cm}
				\IntervalGRF{\big)}{-2}{}{}
			\end{tikzpicture}
		\end{center}
		Từ biểu diễn tập nghiệm của $B$ và $C$ ta thấy $B\cap C= \varnothing$.
	}
\end{ex}
\begin{ex}%[Phan Quốc Trí, Bai Giảng T10(2022)]%%[0-HK2-2021, Trường THPT Trần Phú, Hải Phòng, năm học 2017 - 2018]%[Trần Quang Thạnh]%[0D1Y4-2]
	Cho tập hợp $A=[-2;3]$ và $B=(-2;5]$. Khi đó $A\setminus B$ là
	\choice
	{$[-2;5] $}
	{$(-2;-1) $}
	{$(3;5) $}
	{\True $\left\{-2\right\} $}
	\loigiai{
		Ta có $A\setminus B=\left\{-2\right\}$.
	}
\end{ex}
\begin{ex}%[Phan Quốc Trí, Bai Giảng T10(2022)]%[0D1B4-2]
	Cho các tập $A=\left\{x\in\mathbb{R}\mid x\ge -1\right\}$; $B=\left\{x\in\mathbb{R}\mid x<3\right\}$. Tập hợp $\mathbb{R}\setminus\left(A\cap B\right)$ là
	\choice
	{$[-1;3)$}
	{$(-\infty;-1]\cup(3;+\infty)$}
	{\True $(-\infty;-1)\cup[3;+\infty)$}
	{$(-1;3]$}
	\loigiai{
		Ta có $A\cap B=[-1;3)$, suy ra $\mathbb{R}\setminus\left(A\cap B\right)=(-\infty;-1)\cup[3;+\infty)$.
	}
\end{ex}
\begin{ex}%[Phan Quốc Trí, Bai Giảng T10(2022)]%[0D1B2-2]
	Tìm tất cả các giá trị của $m$ để đoạn $[m;m+3]$ là tập con của nửa khoảng $(-2;9]$.
	\choice
	{$-2\le m\le 6$}
	{$-2\le m<6$}
	{\True $-2<m\le 6$}
	{$-2<m<6$}
	\loigiai{
		Đoạn $[m;m+3]$ là tập con của nửa khoảng $(-2;9]$ khi và chỉ khi $\heva{& -2<m \\& m+3\le 9}\Leftrightarrow \heva{& -2<m \\& m\le 6}\Leftrightarrow -2<m\le 6.$
	}
\end{ex}



\noindent\textbf{II. PHẦN TỰ LUẬN}
\begin{bt}%[Phan Quốc Trí, Bai Giảng T10(2022)]%[0D1B4-1] 
	Cho $A=\left \{x \in \mathbb{R} \Big|  x \geq 3\right \} ; B=(-2 ; 7]$.  Tìm các tập hợp $A \cap B, A \cup B$.
	\loigiai{
		Ta có $A=\left \{x \in \mathbb{R} \Big|  x \geq 3\right \} = [3;+\infty )$. \\
		$A \cap B =[3;+\infty ) \cap (-2 ; 7] =    [3;7 ]$. \\
		$A \cup B = [3;+\infty ) \cup (-2 ; 7] = (-2;+\infty)$. 
	}
\end{bt}
\begin{bt}%[Phan Quốc Trí, Bai Giảng T10(2022)]%[0D1B3-2]
	Cho hai tập hợp $A=\left\{ 0;2 \right\}$ và $B=\left\{ 0;1;2;3;4 \right\}$. Tìm tất cả các tập hợp $X$ thỏa mãn $A\cup X=B$.
	\loigiai { 
		Vì $A\cup X=B$ nên $X$ chắc chắn có chứa các phần tử $1;3;4$\\
		Các tập $X$ có thể là $\left\{ 1;3;4 \right\},\,\left\{ 1;3;4;0 \right\},\,\left\{ 1;3;4;2 \right\},\,\left\{ 1;3;4;0;2 \right\}$}
\end{bt}
\begin{bt}%[Phan Quốc Trí, Bai Giảng T10(2022)]%[0D1B4-1]
	Cho hai tập hợp $A=(2m-1;m+3)$, $B=(-4;5)$. Tìm $m$ để
	$A\cap B=\varnothing $.
	\loigiai{
		Điều kiện: $2m-1<m+3 \Leftrightarrow m<4$.
		Để $A\cap B=\varnothing $ khi và chỉ khi $\left[
		\begin{aligned}
			&m+3\leqslant -4\\
			&2m-1\geqslant 5
		\end{aligned}\right. \Leftrightarrow \left[
		\begin{aligned}
			&m\leqslant -7\\
			&m\geqslant 3
		\end{aligned}\right. $.\\
		Đối chiếu điều kiện, ta được $m\leqslant -7$ hoặc $3\leqslant m<4$ thỏa yêu cầu bài toán.
	}
\end{bt}
\begin{bt}%[Phan Quốc Trí, Bai Giảng T10(2022)]%[0D1B3-3]
	Lớp 10A có $45$ học sinh, trong đó có $18$ học sinh tham gia cuộc thi vẽ đồ họa trên máy tính, $24$ học sinh tham gia cuộc thi tin học văn phòng cấp trường và $9$ học sinh không tham gia cả hai cuộc thi này. Hỏi lớp 10A có bao nhiêu học sinh tham gia đồng thời cả hai cuộc thi?	
	\loigiai{
		\immini{
			Gọi $A$ là tập hợp các học sinh tham gia cuộc thi vẽ đồ họa trên máy tính. Suy ra $n(A)= 18$.\\
			$B$ là tập hợp các học sinh tham gia  cuộc thi tin học văn phòng cấp trường. Suy ra $n(B)= 24$.\\
			Ta có $A \cap B$ là tập hợp các học sinh tham gia đồng thời cả hai cuộc thi.\\
			$A \cup B$ là tập hợp các học sinh tham gia cuộc thi vẽ đồ họa trên máy tính hoặc tham gia  cuộc thi tin học văn phòng cấp trường. 
		}{
			\begin{tikzpicture}
				\draw[] (0,0) circle ( 1.0cm);
				\draw (1.0,0) circle ( 1.0 cm);
				\draw (-0.5,0) node {$18$};
				\draw (1.5,0) node {$24$};
				\draw (-1.3,-0.75) node {$9$};
				\draw (1.5,1.5) node {$45$};
				\node[rectangle,
				draw = lightgray,
				minimum width = 4cm, 
				minimum height = 2.5cm] (r) at (0.5,0) {};
			\end{tikzpicture}
		}
		$$n \left(A \cup B \right)= 45-9=36.$$	
		$$n \left( A \cap B \right) = n(A)+ n(B)-n(A \cup B)=18+24-36= 6.$$
		Vậy có $6$  học sinh tham gia đồng thời cả hai cuộc thi.
	}
\end{bt}

\Closesolutionfile{ans}

\newpage
\begin{indapan}{10}
	{ans/ans-KT-102}
\end{indapan}




\section*{Đề kiểm tra Chương 1}
\subsection*{Đề số 3}
\setcounter{ex}{0}\setcounter{bt}{0}
\Opensolutionfile{ans}[ans/ans-KT-103]
\noindent\textbf{I. PHẦN TRẮC NGHIỆM}
%[Thi thử, Sở GD và ĐT - Điện Biên, 2018]%[Dương BùiĐức, 12EX10]%[2D2Y3-2]
\begin{ex}%[Phan Quốc Trí, Bai Giảng T10(2022)]%[0D1Y1-1]
	Câu nào trong các câu sau \textbf{không phải} là mệnh đề?
	\choice
	{\True $\pi$ có phải là một số vô tỷ không?}
	{$2+2=5$}
	{$\sqrt{2}$ là một số hữu tỷ}
	{$\dfrac{4}{2}=2$}
	\loigiai{
		\lq\lq $\pi$ có phải là một số vô tỷ không?\rq\rq\, là câu hỏi, nên không phải là mệnh đề.
	}
\end{ex}
\begin{ex}%[Phan Quốc Trí, Bai Giảng T10(2022)]%[0D1Y1-1]
	Phát biểu nào sau đây là một mệnh đề?
	\choice
	{Mùa thu Hà Nội đep quá!}
	{\True Hà Nội là thủ đô của Việt Nam}
	{Bạn có đi học không?}
	{Đề thi môn Toán khó quá!}
	\loigiai{
		Hà Nội là thủ đô của Việt Nam là một khẳng định đúng.
	}
\end{ex}
\begin{ex}%[Phan Quốc Trí, Bai Giảng T10(2022)]%[0D1Y1-2]
	Trong các mệnh đề sau, mệnh đề nào \textbf{sai}?
	\choice
	{$-2 \in \mathbb{Q}$}
	{$\sqrt{2} \in \mathbb{R}$}
	{\True $\dfrac{1}{2} \in \mathbb{Z}$}
	{$2 \in \mathbb{N}$}
	\loigiai{
		Mệnh đề sai là  $\dfrac{1}{2} \in \mathbb{Z}$.
	}
\end{ex}
\begin{ex}%[Phan Quốc Trí, Bai Giảng T10(2022)]%[0D1Y1-2]
	Cho mệnh đề chứa biến $P(n) \colon$ "$n^2-1$ chia hết cho $4$" với $n$ là số nguyên. Khẳng định nào sau đây đúng?
	\choice
	{$P(5)$ đúng và $P(2)$ đúng}
	{\True $P(5)$ đúng và $P(2)$ sai}
	{$P(5)$ sai và $P(2)$ sai}
	{$P(5)$ sai và $P(2)$ đúng}
	\loigiai{
		Ta có:\\
		$P(5) \colon$ "$5^2-1$ chia hết cho $4$" tức là "$24$ chia hết cho $4$", nên $P(5)$ đúng.\\
		$P(2) \colon$ "$2^2-1$ chia hết cho $4$" tức là "$3$ chia hết cho $4$", nên $P(2)$ sai.\\
		Vậy $P(5)$ đúng và $P(2)$ đúng.}
\end{ex}
\begin{ex}%[Phan Quốc Trí, Bai Giảng T10(2022)]%[0D1Y1-3]
	Phủ định của mệnh đề: ``$\pi$ là số vô tỷ'' là
	\choice
	{$\pi$ là số nguyên}
	{$\pi$ là số dương}
	{$\pi$ là số thực}
	{\True $\pi$ không phải là số vô tỷ}	
	\loigiai{Mệnh đề phủ định là ``$\pi$ không phải là số vô tỷ''.}
\end{ex}
\begin{ex}%[Phan Quốc Trí, Bai Giảng T10(2022)]%[0D1Y1-3]
	Mệnh đề phủ định của mệnh đề \lq \lq $\exists x\in \mathbb{N}\colon x^2-4=0$\rq \rq \ là
	\choice
	{\lq \lq $\forall x\in \mathbb{N}\colon x^2-4=0$\rq \rq }
	{\lq \lq $\forall x\in \mathbb{N}\colon x^2-4>0$\rq \rq }
	{\lq \lq $\exists x\in \mathbb{N}\colon x^2-4\ne 0$\rq \rq }
	{\True \lq \lq $\forall x\in \mathbb{N}\colon x^2-4\ne 0$\rq \rq}
	\loigiai{
		Ta có mệnh đề phủ định của mệnh đề \lq\lq $\exists x \in X, P(x)$\rq\rq\ là \lq\lq $\forall x \in X, \overline{P(x)}$\rq\rq.\\
		Nên mệnh đề phủ định của mệnh đề \lq \lq $\exists x\in \mathbb{N}\colon x^2-4=0$\rq \rq \ là \lq \lq $\forall x\in \mathbb{N}\colon x^2-4\ne 0$\rq \rq.
	}
\end{ex}
\begin{ex}%[Phan Quốc Trí, Bai Giảng T10(2022)]%[0D1Y1-4]
	Cho mệnh đề $P\colon$ ``$b^2\ge 4ac$'', mệnh đề $Q\colon$ ``Phương trình $ax^2 +bx+c=0$ vô nghiệm'' (với $a,b,c$ là số thực và $a\ne 0$). Phát biểu mệnh đề ``$P$ kéo theo $Q$'' là
	\choice
	{Nếu Phương trình $ax^2 +bx+c=0$ vô nghiệm thì $b^2\ge 4ac$ (với $a,b,c$ là số thực và $a\ne 0$)}
	{\True Nếu $b^2\ge 4ac$ thì phương trình $ax^2 +bx+c=0$ vô nghiệm (với $a,b,c$ là số thực và $a\ne 0$) }
	{$b^2\ge 4ac$ là điều kiện cần và đủ để phương trình $ax^2 +bx+c=0$ vô nghiệm (với $a,b,c$ là số thực và $a\ne 0$)}
	{$b^2\ge 4ac$ khi và chỉ khi phương trình $ax^2 +bx+c=0$ vô nghiệm (với $a,b,c$ là số thực và $a\ne 0$)}
	\loigiai{
		Mệnh đề ``$P$ kéo theo $Q$'' là ``Nếu $b^2\ge 4ac$ thì phương trình $ax^2 +bx+c=0$ vô nghiệm (với $a,b,c$ là số thực và $a\ne 0$)''.
	}
\end{ex}
\begin{ex}%[Phan Quốc Trí, Bai Giảng T10(2022)]%[0D1Y1-4]
	Cho mệnh đề $P$: \lq\lq Nếu tam giác $ABC$ có hai góc bằng $60^{\circ}$ thì đó là tam giác$ABC$ đều\rq\rq. Mệnh đề nào sau đây là mệnh đề đảo của $P$?
	\choice
	{Tam giác $ABC$ có hai góc bằng $60^{\circ}$ thì đó là tam giác$ABC$ đều}
	{\True Nếu tam giác $ABC$ đều thì tam giác đó có hai góc bằng $60^{\circ}$}
	{Tam giác $ABC$ đều khi và chỉ khi tam giác đó có hai góc bằng $60^{\circ}$}
	{Tam giác$ABC$ đều nếu nó có hai góc bằng $60^{\circ}$}
	\loigiai{
		Mệnh đề đảo là \lq \lq Nếu tam giác $ABC$ đều thì tam giác đó có hai góc bằng $60^{\circ}$.\rq \rq		
	}
\end{ex}
\begin{ex}%[Phan Quốc Trí, Bai Giảng T10(2022)]%[0D1Y1-5]
	Mệnh đề "Có ít nhất một số tự nhiên khác 0" mô tả mệnh đề nào dưới đây?
	\choice{"$\forall n\in\mathbb{N}: n\neq 0$"}
	{"$\exists x\in\mathbb{N}:x=0$"}
	{"$\exists x\in\mathbb{Z}: x\neq 0$"}
	{\True "$\exists x\in\mathbb{N}:x\neq 0$"}
	\loigiai{
		Mệnh đề được viết lại là: "$\exists x\in\mathbb{N}:x\neq 0$".}
\end{ex}
\begin{ex}%[Phan Quốc Trí, Bai Giảng T10(2022)]%[0D1Y1-5]
	Mệnh đề nào sau đây đúng?
	\choice{$\forall n\in\mathbb{N}: n > 0$}
	{\True $\exists m\in\mathbb{Z}:2m=m$}
	{$\forall x\in\mathbb{R}:x^2> 0$}
	{$\exists k\in\mathbb{Q}:k^2=2$}
	\loigiai{
		Mệnh đề $\forall n\in\mathbb{N}: n > 0$ là mệnh đề sai, ví dụ $n=0$.\\
		Mệnh đề $\exists m\in\mathbb{Z}:2m=m$ là mệnh đề đúng, vì tồn tại $m=0$ thỏa mãn.\\
		Mệnh đề $\forall x\in\mathbb{R}:x^2> 0$ là mệnh đề sai, ví dụ $x=0$.\\
		Mệnh đề $\exists k\in\mathbb{Q}:k^2=2$ là mệnh đề sai.}
\end{ex}
\begin{ex}%[Phan Quốc Trí, Bai Giảng T10(2022)]%[0D1Y2-1]
	Hãy liệt kê các phần tử của tập $X=\left\{ x \in \mathbb{Q}| (x^2-x-6)(x^2-5)=0 \right\}$.
	\choice
	{$X=\left\{ \sqrt{5};3 \right\}$}
	{\True $X=\left\{ -2;3 \right\}$}
	{$X=\left\{ -\sqrt{5};-2;\sqrt{5}; 3 \right\}$}
	{$X=\left\{ -\sqrt{5};\sqrt{5}  \right\}$}
	\loigiai{
		Ta có 
		\begin{itemize}
			\item $x^2-x-6 = 0 \Leftrightarrow \hoac{&x=3 \in \mathbb{Q} \\&x=-2 \in \mathbb{Q}. }$
			\item $x^2-5 =0 \Leftrightarrow \hoac{&x= \sqrt{5} \notin \mathbb{Q} \\& x= - \sqrt{5} \notin \mathbb{Q}.}$
		\end{itemize}
		Do đó $X=\left\{ -2;3 \right\}$.
	}
\end{ex}
\begin{ex}%[Phan Quốc Trí, Bai Giảng T10(2022)]%[0D1Y2-1]
	Hãy liệt kê các phần tử của tập hợp $X=\left\{x \in \mathbb{N} \ |  \ x \leq 3\right\}$
	\choice
	{$X=\left[0;3\right]$}
	{\True $X=\left\{0;1;2;3\right\}$}
	{$X=\left\{1;2;3\right\}$}
	{$X=\left\{0  \longrightarrow 3\right\}$}
	\loigiai{ 
		Liệt kê các phần tử của tập hợp $X=\left\{x \in \mathbb{N} \ | \ x \leq 3\right\}=\left\{0;1;2;3\right\}$.
	}
\end{ex}

\begin{ex}%[Phan Quốc Trí, Bai Giảng T10(2022)]%[0D1Y2-1]
	Cho tập hợp $A=\{x\in \mathbb{R}|x^2-2x+5=0\}$. Mệnh đề nào sau đây đúng?
	\choice
	{\True $A=\varnothing$}
	{$A=0$}
	{$A=\{-1\}$}
	{$A=\{0\}$}
	\loigiai{
		Vì phương trình $x^2-2x+5=0$ vô ngiệm nên $A=\varnothing$.	
	}	
\end{ex}
\begin{ex}%[Phan Quốc Trí, Bai Giảng T10(2022)]%[0D1Y2-2]
	Cho tập hợp $A=\{a;b;c;d\}$. Số tập hợp con của $A$ có hai phần tử là
	\choice
	{\True $6$}
	{$7$}
	{$8$}
	{$5$}
	\loigiai{
		Sô tập con có hai phần tử của tập $A$ là $6$ đó là các tập $\{a;b\}$, $\{a;c\}$, $\{a;d\}$, $\{b;c\}$, $\{b;d\}$, $\{c;d\}$.}
\end{ex}
\begin{ex}%[Phan Quốc Trí, Bai Giảng T10(2022)]%[0D1Y2-2]
	Có bao nhiêu tập $A$ để $\{m; n\} \subset A \subset \{m; n; x; y\}$? 
	\choice
	{$2$}
	{$3$}
	{$1$}
	{\True $4$}
	\loigiai{
		Các tập $A$ thỏa mãn là $\{m; n\}$, $\{m; n; x\}$, $\{m; n; y\}$ và $\{m; n; x; y\}$. 
	}
\end{ex}
\begin{ex}%[Phan Quốc Trí, Bai Giảng T10(2022)]%[0D1B4-1]
	Cho tập hợp $X = \left\{x \in \mathbb{R}\big|- 2\le x \le 5\right\}$. Khẳng định nào sau đây đúng?
	\choice{$X = \left(-2;5\right)$}
	{$X = \left\{-2;5\right\}$}
	{$X = \left[-2;5\right)$}
	{\True $X = \left[-2;5\right]$}
	\loigiai{
		$\left\{x \in \mathbb{R}\big|- 2\le x \le 5\right\}= \left[-2;5\right]$.
	}
\end{ex}
\begin{ex}%[Phan Quốc Trí, Bai Giảng T10(2022)]%[0D1B2-2]
	Cho tập hợp $A = \left\{x \in \mathbb{N}\big| x^2 + 3x  = 0\right\}$. Khẳng định nào sau đây đúng?
	\choice
	{$A = \left\{-3;0\right\}$}
	{$A =\varnothing$}
	{$A = \left\{\varnothing\right\}$}
	{\True $ A = \left\{0\right\}$}
	\loigiai{
		Phương trình $x^2+3x=0$ có hai nghiệm $x_1=-3$, $x_2=0$. Tuy nhiên chỉ $x_1=0 \in \mathbb{N}$. Vậy $A=\left\{0\right\} $.
	}
\end{ex}
\begin{ex}%[Phan Quốc Trí, Bai Giảng T10(2022)]%[0D1B2-2]
	Gọi $A$ là tập hợp tất cả các tam giác cân và $B$ là tập hợp tất cả các tam giác đều. Trong các kết luận sau, kết luận nào đúng?
	\choice
	{$A \subset B$}
	{\True $B \subset A$}
	{$A=B$}
	{$A \cap B=\varnothing$}
	\loigiai{
		Ta có tam giác đều là một tam giác cân và có một góc $60^{\circ}$ nên $B \subset A$.
	}
\end{ex}

\begin{ex}%[Phan Quốc Trí, Bai Giảng T10(2022)]%[0D1B2-2]
	Cho $\mathbb{N}$, $\mathbb{Z}$, $\mathbb{Q}$, $\mathbb{R}$ là các tập hợp số. Mệnh đề nào sau đây \textbf{sai}?
	\choice
	{$\mathbb{Q} \subset \mathbb{R}$}
	{$\mathbb{N} \subset \mathbb{Z} \subset \mathbb{Q} \subset \mathbb{R}$}
	{$\mathbb{N} \subset \mathbb{Z} \subset \mathbb{Q}$}
	{\True $\mathbb{R} \subset \mathbb{Z}$}
	\loigiai{
		Theo mối quan hệ giữa các tập hợp số, ta có $\mathbb{N} \subset \mathbb{Z} \subset \mathbb{Q} \subset \mathbb{R}$.
	}
\end{ex}
\begin{ex}%[Phan Quốc Trí, Bai Giảng T10(2022)]%[0D1Y3-1]
	Cho tập hợp $M=\{1;2;3\}$ và $N=\{1;a;b\}$. Tìm $M\cup N$.
	\choice
	{\True $M\cup N=\{1;2;3;a;b\}$}
	{$M\cup N=\{2;3;a;b\}$}
	{$M\cup N=\{1\}$}
	{$M\cup N=\{2;3\}$}
	\loigiai{
		$M\cup N=\{1;2;3;a;b\}$.
	}
\end{ex}
\begin{ex}%[Phan Quốc Trí, Bai Giảng T10(2022)]%[0D1Y3-1]
	Cho hai tập hợp $A=\{a;c;d;e\}$ và $B=\{c;d;f;1;2\}$. Khi đó $A\cap B$ là	
	\choice
	{$\{1;2;f\}$}
	{$\{c;d;a;e;f;1;2\}$}
	{\True $\{c;d;\}$}
	{$\{a;e\}$}
	\loigiai{
		Ta có $A\cap B=\{c;d\}$.	
	}
\end{ex}
\begin{ex}%[Phan Quốc Trí, Bai Giảng T10(2022)]%[0D1Y3-2]
	Cho hai tập hợp $A= \left\{0, 1, 2,3,4,5,7\right\}$  và $B= \left\{2,3,4,5,6\right\}$. Tập hợp $A \setminus B$ bằng 
	\choice
	{$ \left\{0,1,2,7\right\}$}
	{$\left\{0,7\right\}$}
	{\True  $\left\{0,1,7\right\}$}
	{$\left\{0,1,6,7\right\}$}
	\loigiai{
		Ta có \[A \setminus B=\left\{0, 1, 7\right\}.\]
	}
\end{ex}

\begin{ex}%[Phan Quốc Trí, Bai Giảng T10(2022)]%[0D1Y3-2]
	Cho hai tập hợp $M=\{0;2;4;6;8;10\}$ và $N=\{0;1;2;3;4;5;6;7;8;9;10\}$. Hãy tìm phần bù của $M$ trong tập hợp $N$.
	\choice
	{$C_NM=N $}
	{$C_NM=M $}
	{\True $C_NM=\{1;3;5;7;9\} $}
	{$C_NM=\{0;2;6;8\} $}
	\loigiai{
		Vì $C_NM$ là tập hợp tất cả các phần tử của $N$ nhưng không là phần tử của $M$, do đó $C_NM= \{1;3;5;7;9\} $.
	}
\end{ex}
\begin{ex}%[Phan Quốc Trí, Bai Giảng T10(2022)]%[0D1B4-1]
	Cho $A=(-\infty;5]$ và $B=(0;+\infty)$. Tập hợp $A\cap B$ là
	\choice
	{\True $(0;5]$}
	{$[0;5)$}
	{$(0;5)$}
	{$(-\infty;+\infty)$}
	\loigiai
	{
		\begin{center}
			\begin{tikzpicture}[scale=1, font=\footnotesize, >=stealth]
				\draw[->] (0,0)--(9,0);
				\draw (3,0) node{$\Big($} + (90:.5) node{$0$} (6,0) node{$\Big]$} + (90:.5) node{$5$};
				\fill[pattern=north east lines] (0,-3pt) rectangle (3,3pt) (6,-3pt) rectangle (9,3pt);
				\begin{scope}[on background layer]\path[white]node{MDD-138};\end{scope}
			\end{tikzpicture}
		\end{center}
		Ta có $A\cap B = (-\infty;5]\cap (0;+\infty) = (0;5]$.
	}
\end{ex}
\begin{ex}%[Phan Quốc Trí, Bai Giảng T10(2022)]%[0D1B4-1]
	Cho tập hợp $A=[-2;5)$ và $B=[0;+\infty)$. Tìm $A\cup B$.
	\choice
	{$A\cup B=[0;5)$}
	{$A\cup B=[-2;0)$}
	{\True $A\cup B=[-2;+\infty)$}
	{$A\cup B[5;+\infty)$}
	\loigiai{
		Ta có 	$A\cup B=[-2;+\infty)$.
	}
\end{ex}

\begin{ex}%[Phan Quốc Trí, Bai Giảng T10(2022)]%[0D1B4-2]
	Tập hợp $\left( -2;3 \right)\setminus \left[1;6 \right]$ bằng tập hợp nào sau đây?
	\choice
	{$\left(-2;1\right]$}
	{$\left(-3;-2\right)$}
	{$\left(-2;6\right)$}
	{\True $\left(-2;1\right)$}
	\loigiai{Ta có $\left(-2;3\right)\setminus \left[1;6\right]=\left(-2;1\right)$.
	}
\end{ex}

\begin{ex}%[Phan Quốc Trí, Bai Giảng T10(2022)]%[0D1B4-2]
	Cho tập hợp $A=\left[ -2;\,\sqrt{5} \right)$. Tập hợp $C_{\mathbb{R}}A$ bằng
	\choice
	{$\left( -\infty ;\,-2 \right]\cup \left[ \sqrt{5};\,+\infty \right)$}
	{\True $\left( -\infty ;\,-2 \right)\cup \left[ \sqrt{5};\,+\infty \right)$}
	{$\left( -\infty ;\,-2 \right]\cup \left( \sqrt{5};\,+\infty \right)$}
	{$\left( -\infty ;\,-2 \right)\cup \left( \sqrt{5};\,+\infty \right)$}
	\loigiai{
		Ta có $C_{\mathbb{R}}A=\Bbb{R}\setminus A=\left( -\infty ;\,-2 \right)\cup \left[ \sqrt{5};\,+\infty \right).$
	}
\end{ex}
\begin{ex}%[Phan Quốc Trí, Bai Giảng T10(2022)]%[0D1B3-3]
	Một lớp có $40$ học sinh, trong đó có $24$ học sinh giỏi Toán, $18$ học sinh giỏi Văn và $10$ học
	sinh không giỏi môn nào trong hai môn Toán và Văn. Hỏi lớp đó có bao nhiêu học sinh giỏi cả hai môn
	Toán và Văn?
	\choice
	{\True $12$ học sinh}
	{$8$ học sinh}
	{$10$ học sinh}
	{$14$ học sinh}
	\loigiai{
		Số học sinh giỏi ít nhất một trong hai môn Toán và Văn là $40-10=30$.\\
		Do có $24$ học sinh giỏi Toán, $18$ học sinh giỏi Văn nên số học sinh giỏi cả hai môn là
		\[24+18-30=12.\]
	}	
\end{ex}
\begin{ex}%[Phan Quốc Trí, Bai Giảng T10(2022)]%[0D1B3-3]
	Mỗi học sinh lớp 10A phải học ít nhất một trong hai môn ngoại ngữ tiếng Anh hoặc tiếng Nhật. Biết lớp 10A có $51$ bạn học sinh trong đó có $31$ bạn học tiếng Anh và $27$ bạn học tiếng Nhật. Hỏi lớp 10A có bao nhiêu bạn học cả tiếng Anh và tiếng Nhật?
	\choice
	{\True $7$}
	{$9$}
	{$5$}
	{$12$}
	\loigiai{
		Số học sinh học cả tiếng Anh và tiếng Nhật của lớp 10A là $31+27-51=7$ bạn.
	}
\end{ex}
\begin{ex}%[Phan Quốc Trí, Bai Giảng T10(2022)]%[0D1B4-1]
	Cho hai tập hợp $A=(-4;5) \cup (7;9)$ và $B=(2;8)$. Tìm $A\cap B$ ta được
	\choice
	{$A \cap B=(7;8)$}
	{$A \cap B=(2;5)$}
	{\True $A \cap B=(2;5) \cup (7;8)$}
	{$A \cap B=[2;5] \cup [7;8]$}
	\loigiai{
		Ta có $A\cap B = \left[(-4;5) \cup (7;9)\right] \cap (2;8)=\left[(-4;5) \cap (2;8)\right] \cup \left[(7;9) \cap (2;8)\right]=(2;5)\cup (7;8)$.	
	}
\end{ex}

\begin{ex}%[Phan Quốc Trí, Bai Giảng T10(2022)]%[0D1B4-1]
	Cho hai tập $A=(-2; 4] \cap \mathbb{Z}$, $B=[-5; 7] \cap \mathbb{N}^{*}$. Số phần tử của tập hợp $A \cup B$ là
	\choice
	{\True $9$}
	{$13$}
	{$10$}
	{$8$}
	\loigiai{
		$A=(-2; 4] \cap \mathbb{Z}=\{-1;0;1;2;3;4\}$, $B=[-5; 7] \cap \mathbb{N}^{*}=\{1;2;3;4;5;6;7\}$.\\
		Ta có $A \cup B=\{-1;0;1;2;3;4;5;6;7\}$. Vậy tập hợp $A \cup B$ có $9$ phần tử.
	}
\end{ex}
\begin{ex}%[Phan Quốc Trí, Bai Giảng T10(2022)]%[0D1B4-1]
	Cho các tập hợp $A=\left[-2;2\right]$, $B=\left(1;5\right]$ và $ C=\left[0;3\right)$. Khi đó tập $\left(A \setminus B\right)\cap C$ là
	\choice
	{\True $\left[0;1\right]$}
	{$\left\{0;1\right\}$}
	{$\left[0;1\right)$}
	{$\left(0;1\right]$}
	\loigiai{
		Ta có $ A\setminus B=\left[-2;1\right]$$\Rightarrow\left(A\setminus B\right)\cap C=\left[0;1\right]$.
	}
\end{ex}
\begin{ex}%[Phan Quốc Trí, Bai Giảng T10(2022)]%[0D1Y4-2]
	Cho tập hợp $A=(-2;1]$ và $B=[-2;8)$. Khi đó $A\setminus B$ bằng
	\choice
	{$\left\{ 2 \right\}$}
	{$(1;8) $}
	{$\left\{ 8 \right\}$}
	{\True $\varnothing$}
	\loigiai{
		Ta có $A\setminus B=\varnothing$.
	}
\end{ex}
\begin{ex}%[Phan Quốc Trí, Bai Giảng T10(2022)]%[0D1B4-2]
	Cho $A=\left\{x\in \mathbb{R}|x+1\ge 0\right\}$, $B=\left\{x\in \mathbb{R}|4-x\ge 0\right\}$. Khi đó $A\backslash B$ là 
	\choice
	{$\left[-1; 4\right]$}
	{$\left[4;+\infty\right)$}
	{\True $\left(4;+\infty\right)$}
	{$\left(-\infty;-1\right)$}
	\loigiai{
		$A=\left\{x\in \mathbb{R}|x+1\ge 0\right\}=\left[-1;+\infty\right)$; $B=\left\{x\in \mathbb{R}|4-x\ge 0\right\}=\left(-\infty; 4\right]$.\\
		Nên $A\backslash B=\left(4;+\infty\right)$.
	}
\end{ex}
\begin{ex}%[Phan Quốc Trí, Bai Giảng T10(2022)]%[0D1K4-2]
	Cho tập hợp $A=[m;m+1]$, $B=[1;3]$. Tập hợp tất cả các giá trị của $m$ để $A\subset B$ là
	\choice
	{$m\leq 1$ hoặc $m\geq 2$}
	{\True $1\leq m\leq 2$}
	{$1<m<2$}
	{$0\leq m\leq 2$}
	\loigiai{Để $A\subset B$ thì $\heva{&m\geq 1\\&m+1\leq 3}\Leftrightarrow 1\leq m\leq 2$.}
\end{ex}
\noindent\textbf{II. PHẦN TỰ LUẬN}
\begin{bt}%[Phan Quốc Trí, Bai Giảng T10(2022)]%[0D1B4-1] 
	Cho $A=\left \{x \in \mathbb{R} \Big|  -1<x \le 7 \right \} ; B=(-\infty ; 2]$.  Tìm các tập hợp $A \cap B, A \cup B$.
	\loigiai{
		Ta có $A = (-1;7]$, $A \cap B = (-1;2]$ và 
		$A \cup B = (-\infty ; 7] $. 
	}
\end{bt}
\begin{bt}%[Phan Quốc Trí, Bai Giảng T10(2022)]%[0D1B3-2]
	Cho hai tập hợp $A=\left\{ a;b \right\}$ và $B=\left\{ a;b;c;2;3;4 \right\}$. Tìm tất cả các tập hợp $X$ thỏa mãn $A\cup X=B$.
	\loigiai { 
		Vì $A\cup X=B$ nên $X$ chắc chắn có chứa các phần tử $c;2;3;4$\\
		Các tập $X$ có thể là $\left\{ c;2;3;4 \right\},\,\left\{ c;2;3;4;a \right\},\,\left\{ c;2;3;4;b \right\},\,\left\{ c;2;3;4; a;b \right\}$.
	}
\end{bt}
\begin{bt}%[Phan Quốc Trí, Bai Giảng T10(2022)]%[0D1K4-1]
	Cho hai tập khác rỗng $A=(m-1;4]$; $B=(-2;2m+2)$, $m\in\mathbb{R}$. Tìm tất cả các giá trị của $m$ để $A\cap B\neq\varnothing$.
	\loigiai{
		Điều kiện: $\heva{&m-1<4\\&2m+2>-2} \Leftrightarrow -2<m<5$.\\
		Ta có	$A\cap B \ne \varnothing \Leftrightarrow 2m+2> m-1\Leftrightarrow m>-3$.\\
		Kết hợp với điều kiện ta có  $A\cap B \ne \varnothing\Leftrightarrow -3 < m < 5$.
	}
\end{bt}

\begin{bt}%[Phan Quốc Trí, Bai Giảng T10(2022)]%[0D1B3-3]
	Trong số $40$ học sinh của lớp 10A có $12$ bạn được xếp loại học lực giỏi, $19$ bạn được xếp loại hạnh kiểm tốt, trong đó có $8$ bạn vừa có học lực giỏi, vừa có hạnh kiểm tốt. Để được khen thưởng thì bạn đó phải có học lực giỏi hoặc có hạnh kiểm tốt. Hỏi có bao nhiêu bạn \textbf{không} được khen thưởng?
	\loigiai{
		Gọi $A$ là tập hợp các học sinh được xếp loại học lực giỏi, $B$ là tập hợp các học sinh được xếp loại hạnh kiểm tốt.\\
		Ta có 
		\begin{itemize}
			\item $A\cup B$ là tập hợp các học sinh được xếp loại học lực giỏi hoặc hạnh kiểm tốt.
			\item $A\cap B$ là tập hợp các học sinh được xếp loại học lực giỏi và hạnh kiểm tốt.
			\item Theo đề  $$n(A)=12, n(B)=19, n(A\cap B)=8.$$
			\item Do đó $$n(A\cup B=n(A)+n(B)-n(A \cap B))=23.$$
		\end{itemize}
		Vậy lớp 10A có $23$ học sinh được khen thưởng và $17$ học sinh \textbf{không} được khen thưởng.}
\end{bt}

\Closesolutionfile{ans}

\newpage
\begin{indapan}{10}
	{ans/ans-KT-103}
\end{indapan}

%%Chương 2
% \section{Bất phương trình bậc nhất hai ẩn}
\setcounter{dang}{0}
\subsection{Tóm tắt lý thuyết}
\subsubsection{Bất phương trình bậc nhất hai ẩn}
Bất phương trình bậc nhất hai ẩn $x$, $y$ có dạng tổng quát là 
\begin{center}
\fbox{$ax+by\le c \text{ (hoặc } ax+by<c; ax+by\ge c; ax+by>c),$}
\end{center}
trong đó $a$, $b$, $c$ là những số thực, $a$ và $b$ không đồng thời bằng $0$, $x$ và $y$ là các ẩn số.

\subsubsection{Biểu diễn tập nghiệm của bất phương trình bậc nhất hai ẩn}
Cũng như bất phương trình bậc nhất một ẩn, các bất phương trình bậc nhất hai ẩn có vô số nghiệm và để mô tả tập nghiệm của chúng, ta sử dụng phương pháp biểu diễn hình học.\\ 
Trong mặt phẳng tọa độ $Oxy$, tập hợp các điểm có tọa độ là nghiệm của bất phương trình được gọi là \textbf{miền nghiệm} của nó.\\
Quy tắc thực hành biểu diễn miền nghiệm của bất phương trình $ax+by \le c$ như sau (tương tự cho bất phương trình $ax+by\ge c$)
\begin{itemize}
	\item{\bf Bước 1:} Trên mặt phẳng tọa độ $Oxy$, vẽ đường thẳng $\Delta \colon ax+by=c$.
	\item{\bf Bước 2:} Lấy một điểm $M_0\left(x_0;y_0\right)$ không thuộc $\Delta$ (ta thường lấy gốc tọa độ $O$).
	\item{\bf Bước 3:} Tính $ax_0+by_0$ và so sánh $ax_0+by_0$ với $c$.
	\item{\bf Bước 4:} Kết luận,
	\begin{itemize}
		\item Nếu $ax_0+by_0<c$ thì nửa mặt phẳng bờ $\Delta$ chứa $M_0$ là miền nghiệm của $ax_0+by_0\leq c$.
		\item Nếu $ax_0+by_0>c$ thì nửa mặt phẳng bờ $\Delta$ không chứa $M_0$ là miền nghiệm của $ax_0+by_0\le c$.
	\end{itemize}
\end{itemize}

\begin{note}
	Miền nghiệm của bất phương trình $ax_0+by_0\leq c$ bỏ đi đường thẳng $ax+by=c$ là miền nghiệm của bất phương trình $ax_0+by_0<c$.
\end{note}
\subsection{Các dạng toán}
\begin{dang}{Bất phương trình bậc nhất hai ẩn và bài toán liên quan}
\end{dang}
\viduminhhoa
\begin{vd}%[Dự án Tex hóa đề thi lớp 10-11-Nhóm Word-T-Begin]%[Lê Vũ Hải-Phản biện: Trần Quốc Tráng]%[0D4Y4-1]%
	Cho bất phương trình: $2x-y<0$ . Trong các cặp số $(-1;2)$, $\left(2;0\right)$, $(0;1)$, $\left(3;-2\right)$, $(-1;-2)$, cặp nào là nghiệm của bất phương trình, cặp nào không phải là nghiệm của bất phương trình?
	\loigiai{
		Bằng cách thử trực tiếp, các cặp $(-1;2)$, $(0;1)$ là nghiệm, các cặp còn lại không phải là nghiệm của bất phương trình.
	}
\end{vd}
\begin{vd}%[Dự án Chuyển tex 10-11, Cao Thành Thái]%[0D4B4-1]
	Biểu diễn hình học tập nghiệm của bất phương trình $2x+y\le 3$.
	\loigiai
	{
		\immini{
			Vẽ đường thẳng $\Delta\colon 2x+y=3$.\\
			Lấy gốc tọa độ $O(0;0)$, ta thấy $O\notin \Delta $ và có $2 \cdot 0+0<3$ nên nửa mặt phẳng bờ $\Delta$ chứa gốc tọa độ $O$ là miền nghiệm của bất phương trình đã cho (miền không bị tô đậm trong hình vẽ).
		}
		{
			\begin{tikzpicture}[line join=round, line cap=round, >=stealth,font=\footnotesize, scale=0.8]
				\draw[->](-2,0)--(3,0) node[below right] {$x$};
				\draw[->](0,-1)--(0,4) node[right] {$y$};
				\clip (-2,-1) rectangle (3,4);
				\node (0,0) [below left]{$ O $};
				\foreach \x in {-1,...,2}
				\draw[shift={(\x,0)},color=black] (0pt,2pt) -- (0pt,-2pt);
				\foreach \y in {1,...,3}
				\draw[shift={(0,\y)},color=black] (2pt,0pt) -- (-2pt,0pt);
				\draw[samples=100,smooth,domain=-2:3] plot(\x,{-2*(\x)+3});
				\draw [pattern=north west lines] (-1,5)--(3,5)--(3,-1) -- (2,-1)--cycle;
				\draw[fill=black] (0,3) circle(1pt) node[left]{$3$};
				\draw[fill=black] (1.5,0)circle(1pt) node[below left]{\tiny $\dfrac{3}{2}$};
			\end{tikzpicture}
		}
	}
\end{vd}
\begin{vd}%[Mai Hà Lan]%[0D4B4-1]
	\begin{enumerate}
		\item Biểu diễn hình học tập nghiệm của bất phương trình $-2x + 3y > 0$.
		\item Cho hai điểm $A(2;1)$ và $B(3; 3)$, hỏi hai điểm này cùng phía hay khác phía đối với bờ $(d)$.
	\end{enumerate}
	\loigiai{
		\immini{
			\begin{enumerate}
				\item Vẽ đường thẳng $d: -2x + 3y = 0$.\\
				Thay tọa độ điểm $M(1;0)$ vào vế trái phương trình đường thẳng $(d)$, ta được: $-2 < 0$.\\
				Vậy miền nghiệm của bất phương trình là nửa mặt phẳng không chứa điểm $M$. (Trên hình là nửa mặt phẳng không bị gạch bỏ).
				\item Thế tọa độ điểm $A$ vào vế trái của phương trình đường thẳng $(d)$ ta được $-2 \cdot 2 + 3 \cdot 1 = -1 < 0$.\hfill $(1)$\\
				Thế tọa độ điểm $B$ vào vế trái của phương trình đường thẳng $(d)$ ta được $-2 \cdot 3 + 3 \cdot 3 = 3 > 0$. \hfill $(2)$ \\
				Từ $(1)$ và $(2)$ suy ra hai điểm nằm ở hai phía đối với bời $(d)$.
			\end{enumerate}
		}{
			\begin{tikzpicture}
				%---------------------- Vẽ hệ trục tọa độ
				\draw[->] (-2.25,0)--(4.25,0) node[below right] {$x$};
				\draw[->] (0,-0.755)--(0,2.25) node[right] {$y$};
				\node (0,0) [below right]{$ O $};
				%----------------------- Vẽ đoạn chắn trên trục
				\foreach \x in {-2,-1,1,2,3,4}
				\draw[shift={(\x,0)},color=black] (0pt,2pt) -- (0pt,-2pt);
				%\node at (3.8,0.5) {$4$};
				\foreach \y in {1,2}
				\draw[shift={(0,\y)},color=black] (2pt,0pt) -- (-2pt,0pt);
				%\node at (-0.5,-1.8) {$-2$};
				
				%--------------------- Vẽ hàm
				\draw [thick, domain=-1:3, samples=100] plot (\x, {(2/3) * \x});
				\node at (2,1.75) {$(d)$};
				
				%---------------------- Điểm M
				\fill (1,0) circle (2pt) node[below right]{$M(1;0)$};
				
				%----------------------Vẽ miền nghiệm
				\tkzDefPoints{-1/-.66/A, 4/-.66/B, 4/2/C, 3/2/D}
				\tkzDrawPolygon[ pattern=north east lines,opacity=.3](A,B,C,D)
			\end{tikzpicture}
	}}
\end{vd}
\begin{vd}%[Mai Hà Lan]%[0D4G4-1]
	\begin{enumerate}
		\item Biểu diễn hình học tập nghiệm của bất phương trình $x + y -3 < 0$.
		\item Tìm điều kiện của $m$ và $n$ để mọi điểm thuộc đường thẳng $(d')$: $ (m^2-2)x - y + m + n= 0 $ đều là nghiệm của bất phương trình trên.
	\end{enumerate}
	\loigiai{
		\immini{
			\begin{itemize}
				\item[a)] Vẽ đường thẳng $d: x + y = 3 $.\\
				Thay tọa độ điểm $O(0;0)$ vào vế trái phương trình đường thẳng $(d)$, ta được: $0< 3$.\\
				Vậy miền nghiệm của bất phương trình là nửa mặt phẳng chứa điểm $O$. (Trên hình là nửa mặt phẳng không bị gạch bỏ).
				\item[b)] Để mọi điểm thuộc đường thẳng  $(d')$ đều là nghiệm của bất phương trình thì điều kiện cần là $(d')$ phải song song với $(d)$. Ta có $d:  y = -x + 3$ và $d': y = (m^2-2)x +  m + n$. Để $(d)$ song song $(d')$ thì $\heva{& m^2 - 2 = -1 \\& m + n \ne 3} \Leftrightarrow \hoac{&\heva{&m=1\\&n\ne 2}\\&\heva{&m=-1\\&n\ne 4}}$\\
				Với $\heva{&m=1\\&n\ne 2}$ thì ta được $d': y = - x + n + 1$. Để thỏa yêu cầu bài toán thì điều kiện đủ là đường thẳng $(d')$ là đồ thị của đường thẳng $(d)$ khi $(d)$ tịnh tiến xuống dưới theo trục $Oy$. Tức $n + 1 < 3 \Leftrightarrow n < 2$.
			\end{itemize}
		}{
			\begin{tikzpicture}[scale=.8]
				%---------------------- Vẽ hệ trục tọa độ
				\draw[->] (-2.25,0)--(4.25,0) node[below right] {$x$};
				\draw[->] (0,-1.25)--(0,4.25) node[right] {$y$};
				\node (0,0) [below left]{$ O $};
				%----------------------- Vẽ đoạn chắn trên trục
				\foreach \x in {-2,-1,1,2,3,4}
				\draw[shift={(\x,0)},color=black] (0pt,2pt) -- (0pt,-2pt);
				\node at (2.75,-0.4) {$3$};
				\foreach \y in {-1,1,2,3,4}
				\draw[shift={(0,\y)},color=black] (2pt,0pt) -- (-2pt,0pt);
				\node at (-0.35,2.75) {$3$};
				
				%--------------------- Vẽ hàm
				\draw [thick, domain=-1:4, samples=100] plot (\x, {3-\x});
				\node at (1.2,1.2) {$(d)$};
				
				%----------------------Vẽ miền nghiệm
				\tkzDefPoints{-1/4/A, 4/4/B, 4/-1/C}
				\tkzDrawPolygon[pattern=north east lines,opacity=.3](A,B,C)
			\end{tikzpicture}
	}}
\end{vd}
\baitaptl
\begin{bt}%[Dự án Chuyển tex 10-11, Cao Thành Thái]%[0D4B4-1]%
	Biểu diễn hình học tập nghiệm của bất phương trình $2x+y\le 3$.
	\loigiai
	{
		\immini{
			Vẽ đường thẳng $\Delta\colon 2x+y=3$.\\
			Lấy gốc tọa độ $O(0;0)$, ta thấy $O\notin \Delta $ và có $2 \cdot 0+0<3$ nên nửa mặt phẳng bờ $\Delta$ chứa gốc tọa độ $O$ là miền nghiệm của bất phương trình đã cho (miền không bị tô đậm trong hình vẽ).
		}
		{
			\begin{tikzpicture}[line join=round, line cap=round, >=stealth,font=\footnotesize, scale=0.8]
				\draw[->](-2,0)--(3,0) node[below right] {$x$};
				\draw[->](0,-1)--(0,4) node[right] {$y$};
				\clip (-2,-1) rectangle (3,4);
				\node (0,0) [below left]{$ O $};
				\foreach \x in {-1,...,2}
				\draw[shift={(\x,0)},color=black] (0pt,2pt) -- (0pt,-2pt);
				\foreach \y in {1,...,3}
				\draw[shift={(0,\y)},color=black] (2pt,0pt) -- (-2pt,0pt);
				\draw[samples=100,smooth,domain=-2:3] plot(\x,{-2*(\x)+3});
				\draw [pattern=north west lines] (-1,5)--(3,5)--(3,-1) -- (2,-1)--cycle;
				\draw[fill=black] (0,3) circle(1pt) node[left]{$3$};
				\draw[fill=black] (1.5,0)circle(1pt) node[below left]{\tiny $\dfrac{3}{2}$};
			\end{tikzpicture}
		}
	}
\end{bt}
\begin{bt}%[0D4B4-1]%
	Biểu diễn hình học tập nghiệm của bất phương trình bậc nhất hai ẩn $2x - 4y < 8$.
	\loigiai{
		\immini{
			Vẽ đường thẳng $d: 2x - 4y =8$.\\
			Thay tọa độ điểm $O(0;0)$ vào vế trái phương trình đường thẳng $(d)$, ta được: $0 < 8$.\\
			Vậy miền nghiệm của bất phương trình là nửa mặt phẳng chứa điểm $O$. (Trên hình là nửa mặt phẳng không bị gạch bỏ).
		}{
			\begin{tikzpicture}[scale=.7]
				%----------------- Vẽ hệ trục tọa độ
				\draw[->] (-2.25,0)--(8.25,0) node[below right] {$x$};
				\draw[->] (0,-3.25)--(0,1.25) node[right] {$y$};
				\node (0,0) [below left] {$ O $};
				%----------------- Vẽ đoạn chắn trên trục
				\foreach \x in {-2,-1,1,2,3,4,5,6,7,8}
				\draw[shift={(\x,0)},color=black] (0pt,2pt) -- (0pt,-2pt);
				\node at (3.8,0.5) {$4$};
				\foreach \y in {-3,-2,-1,1}
				\draw[shift={(0,\y)},color=black] (2pt,0pt) -- (-2pt,0pt);
				\node at (-0.5,-1.8) {$-2$};
				
				%------------- Vẽ hàm
				\draw [thick, domain=-2:6, samples=100] plot (\x, {(1/2)*\x - 2});
				\node at (4.5,.75) {$(d)$};
				
				%---------------- Vẽ miền nghiệm
				\tkzDefPoints{6/1/A, -2/-3/B, 8/-3/C, 8/1/D}
				\tkzDrawPolygon[ pattern=north east lines,opacity=.3](A,B,C,D)
			\end{tikzpicture}
	}}
\end{bt}
\begin{bt}%[Mai Hà Lan]%[0D4B4]
	Biểu diễn hình học tập nghiệm của bất phương trình bậc nhất hai ẩn $3x - y \le 0$.
	\loigiai{
		\immini{
			Vẽ đường thẳng $d: 3x - y = 0 $.\\
			Thay tọa độ điểm $M(0;2)$ vào vế trái phương trình đường thẳng $(d)$, ta được: $-2 < 0$.\\
			Vậy miền nghiệm của bất phương trình là nửa mặt phẳng không chứa điểm $M$, kể cả bờ $(d)$. (Trên hình là nửa mặt phẳng không bị gạch bỏ).
		}{
			\begin{tikzpicture}
				%---------------------- Vẽ hệ trục tọa độ
				\draw[->] (-2.25,0)--(2.25,0) node[below right] {$x$};
				\draw[->] (0,-1.25)--(0,3.25) node[right] {$y$};
				\node (0,0) [above right]{$ O $};
				%----------------------- Vẽ đoạn chắn trên trục
				\foreach \x in {-2,-1,1}
				\draw[shift={(\x,0)},color=black] (0pt,2pt) -- (0pt,-2pt);
				%\node at (3.8,0.5) {$4$};
				\foreach \y in {-1,1,2,3}
				\draw[shift={(0,\y)},color=black] (2pt,0pt) -- (-2pt,0pt);
				%\node at (-0.5,-1.8) {$-2$};
				
				%--------------------- Vẽ hàm
				\draw [thick, domain=-.33:1, samples=100] plot (\x, {3*\x});
				\node at (.5,2.5) {$(d)$};
				
				%---------------------- Điểm M
				\fill (0,2) circle (2pt) node[left]{$M(0;2)$};
				
				%----------------------Vẽ miền nghiệm
				\tkzDefPoints{-.33/-.99/A, 2/-1/B, 2/3/C, 1/3/D}
				\tkzDrawPolygon[ pattern=north east lines,opacity=.3](A,B,C,D)
			\end{tikzpicture}
	}}
\end{bt}
\begin{bt}%[Mai Hà Lan]%[0D4K4]
	a) Biểu diễn hình học tập nghiệm của bất phương trình bậc nhất hai ẩn $\dfrac{x}{3} + \dfrac{y}{6} < 1$.\\
	b) Tìm điểm $A$ thuộc miền nghiệm của bất phương trình trên. Biết rằng điểm $A$ là giao điểm của parabol $(P)$ có dạng $y = x^2 - 5x +4$ và trục hoành. 
	\loigiai{
		\immini{
			\begin{itemize}
				\item[a)] $\dfrac{x}{3} + \dfrac{y}{6} < 1 \Leftrightarrow 2x + y  <6$\\
				Vẽ đường thẳng $d: 2x + y = 6$.\\
				Thay tọa độ điểm $O(0;0)$ vào vế trái phương trình đường thẳng $(d)$, ta được: $0 < 6$.\\
				Vậy miền nghiệm của bất phương trình là nửa mặt phẳng chứa điểm $O$. (Trên hình là nửa mặt phẳng không bị gạch bỏ).
				\item[b)] Điểm $A$ nằm trên parabol $(P)$ có dạng $y = x^2 - 5x +4$ và trục hoành nên hoành độ của $A$ là nghiệm của phương trình $x^2 - 5x + 4 = 0 \Leftrightarrow \hoac{& x = 1\\ & x = 4.}$ \\
				Suy ra ta được hai điểm $(1;0)$ và $(4;0)$. Lần lượt thế tọa độ từng điểm vào vế trái của phương trình đường thẳng $(d)$, do $A$ thuộc miền nghiệm của bất phương trình đã cho nên ta được $A$ có tọa độ là $(1;0)$.
			\end{itemize}
		}{
			\begin{tikzpicture}[scale=.6]
				%---------------------- Vẽ hệ trục tọa độ
				\draw[->] (-2.25,0)--(6.25,0) node[below right] {$x$};
				\draw[->] (0,-3.25)--(0,8.25) node[right] {$y$};
				\node (0,0) [below left]{$ O $};
				%----------------------- Vẽ đoạn chắn trên trục
				\foreach \x in {-2,-1,1,2,3,4,5,6}
				\draw[shift={(\x,0)},color=black] (0pt,2pt) -- (0pt,-2pt);
				\node at (2.75,-0.4) {$3$};
				\node at (.75,-0.4) {$1$};
				\node at (4.25,-0.4) {$4$};
				\foreach \y in {-3,-2,-1,1,2,3,4,5,6,7,8}
				\draw[shift={(0,\y)},color=black] (2pt,0pt) -- (-2pt,0pt);
				\node at (0.5,6.25) {$6$};
				
				%--------------------- Vẽ hàm
				\draw [thick, domain=-1:4.5, samples=100] plot (\x, {-2*\x + 6});
				\node at (3.5,-2.75) {$(d)$};
				
				%---------------------- Vẽ (P)
				\draw [thick, domain=-.7:5.7, samples=100] plot (\x, {(\x)^2 - 5*\x + 4});
				\node at (1,-2) {$(P)$};
				
				%----------------------Vẽ miền nghiệm
				\tkzDefPoints{-1/8/A, 6/8/B, 6/-3/C, 4.5/-3/D}
				\tkzDrawPolygon[pattern=north east lines,opacity=.3](A,B,C,D)
			\end{tikzpicture}
	}}
\end{bt}
\begin{bt}%[Lê Xuân Dũng]%[0D4B4]%[0D4K4]
	Cho bất phương trình $2x+y-1 \le 0$.
	
	a) Biểu diễn miền nghiệm của bất phương trình đã cho trong mặt phẳng tọa độ $Oxy.$
	
	b) Tìm tất cả giá trị  tham số $m$ để điểm $M(m,1)$ nằm trong miền nghiệm của bất phương trình đã và biểu diễn tập
	hợp $M$ tìm được trong cùng hệ trục tọa độ $Oxy$ ở câu a).
	\loigiai{
		\immini{a) Đường thẳng $(d){:} \, 2x+y-1=0$ có đồ thị như hình vẽ bên.
			Ta có $2.0+0-1 <0.$ Do đó, miền nghiệm là đường thẳng $(d)$ và miền không gạch chéo như hình vẽ bên (Miền chứa gốc tọa độ).
			
			b) Để $M$ là một nghiệm thì $2m+1-1 \le 0\Leftrightarrow m\le 0.$ 
			Vì $M$ nằm trên đường thẳng $(\Delta): y = 1.$ Do đó, tập hợp tất cả điểm $M$
			là nghiệm của bất phương trình trình đã cho là tia $At$ như hình vẽ.
		}{\begin{tikzpicture}[scale=0.7,thick,>=stealth']
				\draw[->] (-1.3,0) -- (3.3,0)node[above]{$x$};
				\foreach \x in {}
				\draw[shift={(\x,0)},color=black] (0pt,2pt) -- (0pt,-2pt) node[below] {\footnotesize $\x$};
				\draw[->,color=black] (0,-2) -- (0,3.3)node[right]{$y$};
				\foreach \y in {}
				\draw[shift={(0,\y)},color=black] (2pt,0pt) -- (-2pt,0pt) node[left] {\footnotesize $\y$};
				\node[below left] at (0,0) {$O$};
				\node[below left] at (0,1) {$A$};
				\node[below] at (1,-1.3) {$d$};
				\node[above right] at (0,1) {$1$};
				\node[below] at (0.5,0) {$\frac{1}{2}$};
				\clip(-1.3,-2) rectangle (3,3);
				\node[below] at (-1.2,1) {$t$};
				%	\fill[pattern=north east lines] (-1,2.66667) -- (-1,4) -- (4,4) -- (4,-0.6667) -- (3,0) -- cycle;
				%\draw[line width=1.2pt,smooth,samples=100,domain=-1:4] plot(\x,{2-0.666667*(\x)});
				\fill[pattern=north east lines,pattern color=blue!30] (-1,3)--(0,1)--(0.5,0)--(1.5,-2)--(3,-2)--(3,0)--(3,3)-- cycle;
				\draw[line width=1.2pt,smooth,samples=100,domain=-3:3] plot(\x,{1-2*(\x)});
				\draw[line width=1.2pt][-] (0,1)--(-1.3,1);
		\end{tikzpicture}}
	}
\end{bt}

\begin{bt}%[Lê Xuân Dũng]%[0D4B4]%[0D4K4]
	Cho bất phương trình $x-2y+4m > 0.$
	
	a) Tùy theo giá trị tham số $m,$ hãy biểu diễn tập nghiệm của bất phương trình đã cho
	trong hệ trục tọa độ $Oxy.$
	
	b) Gọi $A,B$ lần lượt  là giao của đường thẳng $x-2y+4m=0$ với trục hoành và trục tung. 
	Tìm tất cả các giá trị của tham số $m$ để tập nghiệm của bất phương trình đã cho chứa điểm $C(2;1)$ 
	sao cho diện tích tam giác $ABC$ bằng $4.$
	\loigiai{
		\immini{a) Xét đường thẳng  $(d_m){:} \, x-2y+4m=0$ có đồ thị như hình vẽ bên.
			Ta có $0-2.0+4m = 4m.$ Do đó, với mọi $m \ne 0$ miền nghiệm luôn chứa gốc tọa độ.
			Nếu $m=0$ thì miền nghiệm chứa điểm $(1;0).$ Vậy với mọi $m$ miền nghiệm
			là miền không gạch chéo như hình vẽ bên.
			
			b) Để $C$ là một nghiệm của bất phương trình đã cho thì $2-2+4m > 0\Leftrightarrow m> 0.$ 
			Khi đó, $OC \parallel (d_m),$ suy ra $S_{\Delta ABC}=S_{\Delta ABO} = 4m^2.$
			Theo giả thiết, ta có $4m^2 = 4 \Leftrightarrow m=1.$
		}{\begin{tikzpicture}[scale=0.6,thick,>=stealth']
				\draw[->] (-5.0,0) -- (4.3,0)node[above]{$x$};
				\foreach \x in {1,2}
				\draw[shift={(\x,0)},color=black] (0pt,2pt) -- (0pt,-2pt) node[below] {\footnotesize $\x$};
				\draw[->,color=black] (0,-1) -- (0,3.3)node[right]{$y$};
				\foreach \y in {}
				\draw[shift={(0,\y)},color=black] (2pt,0pt) -- (-2pt,0pt) node[left] {\footnotesize $\y$};
				\node[below right] at (0,0) {$O$};
				\node[below ] at (-4,0) {$A$};
				\node[above left] at (-4,0) {$-4m$};
				\node[right=0.3] at (0,2) {$B$};
				\node[above left] at (0,2) {$2m$};
				\node[above] at (2,1) {$C$}; 
				\node at (-0.3,1) {$1$}; 
				\draw[fill]  (2,1) circle (1.5pt) (-4,0) circle (1.5pt) (0,2) circle (1.5pt) (2,1) circle (1.5pt);
				\draw  (-4,0)--(2,1)--(0,2);
				\draw[dashed] (2,1)--(2,0) (2,1)--(0,1);
				\clip(-5.0,-2) rectangle (4.3,3.0);	
				\fill[pattern=north east lines,pattern color=blue!30] (-5,3)--(-5,0)--(-5,-0.5)--(-4,0)--(0,2)--(2,3)-- cycle;
				\draw[line width=1.2pt,smooth,samples=100,domain=-5.0:4.3] plot(\x,{2+0.5*(\x)});
				\draw[line width=1.2pt,smooth,samples=100,domain=-2.0:4.3] plot(\x,{0.5*(\x)});
		\end{tikzpicture}}
	}
\end{bt}
\begin{dang}{Bài toán thực tế liên quan}
	
\end{dang}
\viduminhhoa
\begin{vd}%[Nguyện Ngô]%[0D4B4]
	Hà mang $95000$ đồng ra chợ mua hoa cúc và hoa hồng. Một bông hoa cúc có giá $4000$ đồng, một bông hoa hồng có giá $7000$ đồng. Viết bất phương trình bậc nhất hai ẩn cho số tiền mà Hà phải chi để mua $x$ bông hoa cúc và $y$ bông hoa hồng.  
	\loigiai{
		Ta có $x, y\in\mathbb{N}^*$.\\
		Giá của $x$ bông hoa cúc là $4000x$ đồng, giá của $y$ bông hoa hồng là $7000y$ đồng.\\
		Vì số tiền Hà mang đi là $95000$ đồng nên ta có bất phương trình 
		\[4000x+7000y\le 95000\Leftrightarrow 4x+7y\le 95.\] 
	}
\end{vd}

\begin{vd}%[Nguyện Ngô]%[0D4K4]
	Mỗi ngày Nga đều dành không quá $30$ phút để đọc cả $2$ cuốn sách A, B. Nga đọc được $3$ trang sách A trong $2$ phút, đọc được $2$ trang sách B trong $1$ phút. Gọi $x$, $y$ lần lượt là số phút đọc sách A và số phút đọc sách B. Tìm điều kiện của $x$ và $y$ để Nga đọc được ít nhất $35$ trang sách trong một ngày.
	\loigiai{
		Gọi $x$, $y$ lần lượt là số phút đọc sách A và số phút đọc sách B trong một ngày, $x, y>0$. Tổng số phút đọc sách không quá $30$ phút nên $x+y\le 30$.\\
		Số trang sách A đọc được sau $x$ phút là $\dfrac{3x}{2}$.
		Số trang sách B đọc được sau $y$ phút là $2y$.\\
		Nga đọc được ít nhất $35$ trang sách trong một ngày khi và chỉ khi $\dfrac{3x}{2}+2y\ge 35$.\\
		Vậy $x,y$ cần thỏa mãn các điều kiện $\heva{&x,y>0\\&x+y\le30\\&\dfrac{3x}{2}+2y\ge35.}$
	}
\end{vd}

\begin{vd}%[Nguyện Ngô]%[0D4K4]
	Một cửa hàng bán hai loại trà sữa, trong đó $4$ cốc loại $1$ có giá $100000$ đồng, $1$ cốc loại $2$ có giá $30000$ đồng. Muốn có lãi theo dự tính thì mỗi ngày cửa hàng phải bán được ít nhất $5$ triệu đồng tiền hàng. Hỏi số cốc trà sữa bán được trong một ngày trong những trường hợp nào thì cửa hàng có lãi như dự tính?
	\loigiai{
		Gọi $x$, $y$ lần lượt là số cốc trà sữa loại $1$, loại $2$ bán được ($x, y\in\mathbb{N}$).\\
		Tổng số tiền bán trà sữa là $25x+30y$ nghìn đồng.\\
		Cửa hàng có lãi như dự tính trong trường hợp số tiền bán trà sữa thu được trong một ngày không nhỏ hơn $5$ triệu đồng, tức là 
		\[25x+30y\ge 5000.\quad\quad (1)\]
		\immini{
			Miền nghiệm của bất phương trình (1) được xác định như sau\\
			+/ Vẽ đường thẳng $d\colon 25x+30y=5000$.\\
			+/ Chọn gốc tọa độ $O(0;0)$ và tính $25\cdot0+30\cdot0<500$.\\
			Do đó miền nghiệm của bất phương trình (1) là nửa mặt phẳng bờ $d$, không chứa gốc tọa độ $O$, lấy cả đường thẳng $d$.\\
			Gọi $A$, $B$ lần lượt là giao điểm của $d$ và $Ox$, $Oy$. Khi đó, nếu bán được $x$ cốc trà sữa loại $1$ và $y$ cốc trà sữa loại $2$ mà điểm $(x;y)$ nằm ở góc phần tư thứ nhất đồng thời nằm ngoài miền tam giác $OAB$ (có thể nằm trên cạnh $AB$) (phần gạch chéo) thì cửa hàng sẽ có lãi như dự tính.
		}
		{
			\begin{tikzpicture}[scale=.7,>=stealth]
				\draw[->] (0,0) -- (5.3,0)node[below]{$x$};
				\draw[->,color=black] (0,0) -- (0,5.3)node[left]{$y$};
				\node[below left] at (0,0){$O$};
				\node[left] at (0,3){$\dfrac{5000}{3}$};
				\node[above right] at (0,3){B};
				\node[below] at (4,0){$200$};
				\node[above] at (4,0){A};
				\clip(0,0) rectangle (5.3,5.3);
				\fill[pattern=north east lines](4,0)-- (5.3,0) -- (5.3,5.3) -- (0,5.3)--(0,3)-- cycle;
				\draw[line width=1.2pt,smooth,samples=100,domain=0:4] plot(\x,{-0.75 *(\x) +3});
			\end{tikzpicture}
		}
	}
\end{vd}
\baitaptl
\begin{bt}%[Nguyện Ngô]%[0D4B4]
	Giá sách của Hoa có thể chứa được khối lượng sách tối đa là $4$ kg. Hoa xếp cả hai loại sách (loại $1$ và loại $2$) vào giá. Sách loại $1$ có khối lượng $100$ gam mỗi cuốn và sách loại $2$ có khối lượng $200$ gam mỗi cuốn. Viết bất phương trình bậc nhất hai ẩn cho khối lượng của $x$ cuốn loại $1$ và $y$ cuốn loại $2$ có thể được xếp lên giá sách.
	\loigiai{
		Ta có $4$ kg $=4000$ gam.\\
		Khối lượng của $x$ cuốn sách loại $1$ là $100x$ gam.
		Khối lượng của $y$ cuốn sách loại $2$ là $200y$ gam.\\
		Hoa xếp cả hai loại sách nên $x, y\in\mathbb{N}^*$.
		Vì giá sách của Hoa có thể chứa được khối lượng sách tối đa là $4$ kg nên số cuốn sách ($x$ cuốn loại $1$ và $y$ cuốn loại $2$) có thể được xếp lên giá sách thỏa mãn bất phương trình 
		$100x+200y\le 4000\Leftrightarrow x+2y\le 40$.
	}
\end{bt}

\begin{bt}%[Nguyện Ngô]%[0D4B4]
	Công ty viễn thông Mobifone tính phí $1$ nghìn đồng mỗi phút gọi nội mạng, $2$ nghìn đồng mỗi phút gọi ngoại mạng. Mỗi tháng Minh gọi điện thoại hết từ $200$ đến $300$ nghìn đồng. Viết bất phương trình bậc nhất hai ẩn mô tả cho số tiền điện thoại trả cho ($x$) phút gọi nội mạng và ($y$) phút gọi ngoại mạng trong một tháng.
	\loigiai{
		Số tiền điện thoại trả cho $x$ phút gọi nội mạng là $x$ nghìn đồng.\\
		Số tiền điện thoại trả cho $y$ phút gọi nội mạng là $2y$ nghìn đồng.\\
		Mỗi tháng Minh gọi điện thoại hết từ $200$ đến $300$ nghìn đồng nên ta có 
		\[200\le x+2y\le 300.\]
	}
\end{bt}

\begin{bt}%[Nguyện Ngô]%[0D4K4]
	Bạn An giải $10$ bài Toán trong $20$ phút thì đúng được $80\%$ số bài Toán, giải $12$ bài Lý trong $15$ phút thì đúng được $\dfrac{3}{4}$ số bài Lý. Viết bất phương trình bậc nhất hai ẩn cho thời gian giải $x$ bài Toán đúng và $y$ bài Lý đúng, biết thời gian giải ít hơn $150$ phút.   
	\loigiai{
		Sau $20$ phút An làm đúng được $10\cdot 80\%=8$ bài Toán.\\
		Suy ra thời gian An giải đúng $x$ bài Toán là $\dfrac{20x}{8}=\dfrac{5x}{2}$ phút.\\
		Sau $15$ phút An làm đúng được $12\cdot \dfrac{3}{4}=9$ bài Lý.\\
		Suy ra thời gian An giải đúng $y$ bài Lý là $\dfrac{15y}{9}=\dfrac{5y}{3}$ phút.\\
		Vì thời gian giải ít hơn $150$ phút nên ta có 
		\[\dfrac{5x}{2}+\dfrac{5x}{3}<150\Leftrightarrow 3x+2y<180.\] 
	}
\end{bt}

\begin{bt}%[Nguyện Ngô]%[0D4K4]
	Một gian hàng trưng bày bàn và ghế rộng $100$ m$^2$. Diện tích để kê một chiếc ghế là $1$ m$^2$, một chiếc bàn là $2$ m$^2$ và diện tích mặt sàn dành cho lưu thông tối thiểu là $24$ m$^2$. Gọi $x$ là số chiếc ghế, $y$ là số chiếc bàn được kê, hãy viết bất phương trình bậc nhất hai ẩn $x$, $y$ cho phần mặt sàn để kê bàn và ghế và chỉ ra hai nghiệm của bất phương trình.
	\loigiai{
		Diện tích kê $x$ chiếc ghế là $x$ m$^2$, ($x\in\mathbb{N^*}$).\\
		Diện tích kê $y$ chiếc ghế là $2y$ m$^2$, ($y\in\mathbb{N^*}$).\\
		Diện tích mặt sàn tối đa có thể kê bàn, ghế là $100-24=76$ m$^2$.\\
		Do đó ta có bất phương trình $x+2y\le 76$.\\
		Cho $x=26$, ta có $26+2y\le 76\Leftrightarrow y\le 25$.\\ 
		Lần lượt chọn $y=23$, $y=24$ ta được hai nghiệm của bất phương trình là $(26;23)$ và $(26;24)$.
	}
\end{bt}

\begin{bt}%[Nguyện Ngô]%[0D4K4]
	Một rạp chiếu phim $2$D phục vụ khán giả một bộ phim mới với $2$ loại vé khác nhau. Vé loại $1$ (từ thứ $2$ đến thứ $5$) giá $80000$ đồng/vé, vé loại $2$ (từ thứ $6$ đến chủ nhật và ngày lễ) giá $100000$ đồng/vé. Để không phải bù lỗ thì số tiền vé thu được ở rạp chiếu phim này phải đạt tối thiểu $150$ triệu đồng. Hỏi số lượng vé bán được trong những trường hợp nào thì rạp chiếu phim phải bù lỗ?
	\loigiai{
		Gọi $x$, $y$ lần lượt là số vé loại $1$, loại $2$ bán được ($x, y\in\mathbb{N}$).\\
		Tổng số tiền bán vé là $80x+100y$ nghìn đồng.\\
		Rạp chiếu phim phải bù lỗ trong trường hợp số tiền bán vé nhỏ hơn $150$ triệu đồng, tức là 
		\[80x+100y<150000\Leftrightarrow 4x+5y<7500.\quad\quad (1)\]
		\immini{
			Miền nghiệm của bất phương trình (1) được xác định như sau\\
			+/ Vẽ đường thẳng $d\colon 4x+5y=7500$.\\
			+/ Chọn gốc tọa độ $O(0;0)$ và tính $4\cdot0+5\cdot0<7500$.\\
			Do đó miền nghiệm của bất phương trình (1) là nửa mặt phẳng bờ $d$, chứa gốc tọa độ $O$, không kể đường thẳng $d$.\\
			Gọi $A$, $B$ lần lượt là giao điểm của $d$ và $Ox$, $Oy$. Khi đó, nếu bán được $x$ vé loại $1$ và $y$ vé loại $2$ mà điểm $(x;y)$ nằm trong miền tam giác $OAB$ không kể cạnh $AB$ thì rạp chiếu phim sẽ phải bù lỗ.
		}
		{
			\begin{tikzpicture}[scale=.7,>=stealth]
				\draw[->] (0,0) -- (5.3,0)node[below]{$x$};
				\draw[->,color=black] (0,0) -- (0,5.3)node[left]{$y$};
				\node[below left] at (0,0){$O$};
				\node[left] at (0,3){$1500$};
				\node[above right] at (0,3){B};
				\node[below] at (4,0){$1875$};
				\node[above] at (4,0){A};
				\clip(0,0) rectangle (5.3,5.3);
				\fill[pattern=north east lines](4,0)-- (5.3,0) -- (5.3,5.3) -- (0,5.3)--(0,3)-- cycle;
				\draw[line width=1.2pt,smooth,samples=100,domain=0:4] plot(\x,{-0.75 *(\x) +3});
			\end{tikzpicture}
		}	
	}
\end{bt}


\begin{bt}%[Nguyện Ngô]%[0D4K4]
	Một bác nông dân cần trồng lúa và khoai trên diện tích đất $6$ ha, với lượng phân bón dự trữ là $100$ kg và sử dụng tối đa $120$ ngày công. Để trồng $1$ ha lúa cần sử dụng $20$ kg phân bón, $10$ ngày công với lợi nhuận là $30$ triệu đồng; để trồng $1$ ha khoai cần sử dụng $10$ kg phân bón, $30$ ngày công với lợi nhuận là $60$ triệu đồng. Biết bác nông dân đã trồng $x$ (ha) lúa và $y$ (ha) khoai. Tìm giá trị của $x$ để bác nông dân đạt được lợi nhuận cao nhất.
	\loigiai{
		Theo bài toán, ta có:\\
		$ \heva{& x+y=6\\&20x+10y\leq 100\\&10x+30y\leq 120\\&
			T=30x+60y \longrightarrow Max}$ 
		$\Leftrightarrow \heva{& y=6-x\\&x\leq 4\\ &x\geq 3\\& T=24x+360 \longrightarrow Max}$
		$\Leftrightarrow \heva{&y=6-x\\&3\leq x\leq 4\\& T=24x+360 \longrightarrow Max.}$\\
		Vì $T=24x+360$ là hàm số bậc nhất và có hệ số $a=24>0$ nên $T$ đạt GTLN tại $x=4$.\\
		Vậy $x=4$ là giá trị cần tìm.
	}
\end{bt}
\subsection{Câu hỏi trắc nghiệm}
% \Opensolutionfile{ansbook}[ans/ansbook-BPTbacnhathaian]
\Opensolutionfile{ans}[ans/ans-BPTbacnhathaian]
\begin{ex}%[0D4Y4-1]%
	Trong các bất phương trình sau đây, đâu là bất phương trình bậc nhất hai ẩn
	\choice
	{$2x^2-3x \geq 1$}
	{\True $2x+y\leq 1$}
	{$3x+1\leq 0$}
	{$3x+y=1$}
	\loigiai{
		Theo định nghĩa $2x+y\leq 1$ là bất phương trình bậc nhất hai ẩn.
	}
\end{ex}
\begin{ex}%[Dự án Tex Khối 10-11 W-T-Begin lần 4]%[Biên soạn: Tuan Nguyen, Phản biện: Phan Văn Thành]%[0D4B4-1]%Câu 1
	Cho bất phương trình $2x+3y-6 \leq 0\quad (1)$. Chọn khẳng định đúng trong các khẳng định sau.
	\choice
	{Bất phương trình $(1)$ chỉ có một nghiệm duy nhất}
	{Bất phương trình $(1)$ vô nghiệm}
	{\True Bất phương trình $(1)$ luôn có vô số nghiệm}
	{Bất phương trình $(1)$ có tập nghiệm là $\mathbb{R}$}
	\loigiai{
		Trên mặt phẳng tọa độ, đường thẳng $(d) \colon 2x+3y-6=0$ chia mặt phẳng thành hai nửa mặt phẳng.\\
		Chọn điểm $O(0;0)$ không thuộc đường thẳng đó. Ta thấy $(x;y)=(0;0)$ là nghiệm của bất phương trình đã cho.\\ Vậy miền nghiệm của bất phương trình là nửa mặt phẳng bờ $(d)$ chứa điểm $O(0;0)$ kể cả $(d)$.\\
		Vậy bất phương trình $(1)$ luôn có vô số nghiệm.}
\end{ex}
\begin{ex}%[Dự án Tex Khối 10-11 W-T-Begin lần 4]%[Biên soạn: Tuan Nguyen, Phản biện: Phan Văn Thành]%[0D4B4-1]%Câu 5
	Trong các cặp số sau đây, cặp nào \textbf{không} là nghiệm của bất phương trình $x-4y+1 \geq 0$?
	\choice
	{$(-1;0)$}
	{$(-2;-1)$}
	{\True $(-1;3)$}
	{$(0;0)$}
	\loigiai{
		Ta có $(-1)-4\cdot 3+1\ge 0$ là mệnh đề sai nên cặp số $(-1;3)$ không là nghiệm của của bất phương trình trên.
	}
\end{ex}
\begin{ex}%[Dự án Tex hóa đề thi lớp 10-11-Nhóm Word-T-Begin]%[Nguyễn Trung Kiên. Phản biện: Trần Nhân Kiêt]%[0D4Y4-1]%
	Miền nghiệm của bất phương trình $4(x-1)+5(y-3)>2x-9$ là nửa mặt phẳng chứa điểm nào?
	\choice
	{$(0;0)$}
	{$(1;1)$}
	{$(-1;1)$}
	{\True $(2;5)$}
	\loigiai{
		Ta có $4(x-1)+5(y-3)>2x-9\Leftrightarrow 4x-4+5y-15>2x-9\Leftrightarrow 2x+5y-10>0$.\\
		Dễ thấy tại điểm $(2;5)$ ta có $2\cdot 2+5\cdot 5-10>0$ (đúng).}
\end{ex}
\begin{ex}%[0D4B4-1]%
	Điểm nào sau đây thuộc miền nghiệm của bất phương trình $x+y-2>0$?
	\choice
	{\True $(2;1)$}
	{$(0;0)$ }
	{$(1;0)$ }
	{$(0;1)$ }
	\loigiai{
		\immini{
			Tập hợp các điểm biểu diễn nghiệm của bất phương trình $x+y-2>0$  là nửa mặt phẳng bờ là đường thẳng $y=x+2$  và không chứa gốc tọa độ.
			Từ đó ta có điểm $(2;1)$  thuộc miền nghiệm của bất phương trình.
		}
		{\begin{tikzpicture}[>=stealth, scale=0.7]
				\draw[->,line width = 1pt] (-2.5,0)--(0,0) node[below right]{$O$}--(4,0) node[below]{$x$};
				\draw[->,line width = 1pt] (0,-2.5) --(0,3.5) node[right]{$y$};
				\foreach \x in {-2,-1,1,2,3}{
					\draw (\x,0) node[below]{$\x$} circle (1pt);
					\draw (0,\x) node[left]{$\x$} circle (1pt);
				}
				\draw [pattern = horizontal lines, thick, domain=-1:4.0, samples=100] plot (\x, {-(\x)+2}) node[right]{$d$};
				\draw[pattern = horizontal lines,opacity=.3, line width = 1.2pt,draw=none] plot[domain=-1:4.0] (\x, {-(\x)+2})--(-2,-2)--(-2,3)--cycle;
				\clip (-2.5,-2.5) rectangle (4.0,3.5);
				\draw (2,1) node[right]{$M(2;1)$} circle(2pt);
				\draw[dashed] (2,0)--(2,1)--(0,1);
			\end{tikzpicture}
		}
	}
\end{ex}
\begin{ex}%[0D4Y4-1]%
	Điểm $A(-1;3)$ thuộc miền của bất phương trình
	\choice
	{$x+3y<0$}
	{$3x-y>0$}
	{\True  $-3x+2y-4>0$}
	{$2x-y+4>0$}
	\loigiai{
		Thay tọa độ $A(-1;3)$ vào các bất phương trình:
		\begin{itemize}
			\item[•] Với bất phương trình $x+3y<0$, ta có $(-1)+3\cdot 3<0$ sai.
			\item[•] Với bất phương trình $3x-y>0$, ta có $3\cdot (-1)-3>0$ sai.
			\item[•] Với bất phương trình $-3x+2y-4>0$, ta có $-3\cdot (-1)+2\cdot 3-4>0$ đúng.
			\item[•] Với bất phương trình $2x-y+4>0$, ta có $2\cdot (-1)-3+4>0$ sai.
		\end{itemize}
		Vậy $A(-1;3)$ thuộc miền nghiệm bất phương trình $-3x+2y-4>0$.
	}
\end{ex}
\begin{ex}%[Nguyễn Trung Hiếu]%[781-810 Phạm Quốc Toàn]%[0D4K4-1]%
	Tìm tất cả các số thực $a$ sao cho miền nghiệm của bất phương trình $x\le a$ chứa điểm $M(-1;0)$.
	\choice
	{$a>-1$}
	{\True $a \ge -1$}
	{$a>0$}
	{$a\ge 0$}
	\loigiai{Để $M(-1;0)$ thuộc miền nghiệm của bất phương trình $x\le a$ thì $a \geq -1$.
	}
\end{ex}
\begin{ex}%[0-GHK2-2021, THPT Nguyễn Trường Tộ, 2020-2021]%[Vô Văn Tự]%[0D4B4-1]%
	Cho đường thẳng $d\colon 7x-9y+2=0$ chia mặt phẳng toạ độ làm hai nửa  mặt phẳng, trong đó miền nghiệm của bất phương trình $7x-9y+2>0$ là nửa mặt phẳng
	\choice
	{có bờ là đường thẳng $d$ và không chứa điểm $O(0;0)$}
	{\True không có bờ $d$ và chứa điểm $O(0;0)$}
	{có bờ là đường thẳng $d$ và chứa điểm $O(0;0)$}
	{không chứa bờ $d$ và không chứa điểm $O(0;0)$}
	\loigiai{
		Ta có toạ độ điểm $O(0;0)$ thoả mãn bất phương trình $7x-9y+2>0$ nên miền nghiệm của bất phương trình $7x-9y+2>0$ là nửa mặt phẳng không có bờ $d$ và chứa điểm $O(0;0)$.
	}
\end{ex}
\begin{ex}%[Word: Nguyễn Văn Mến, LaTeX: Nguyễn Tài Tuệ, PB: Nguyễn Tấn Linh]%[0D4B4-1]%
	\immini{ Phần gạch chéo trong hình vẽ dưới đây (không bao gồm đường thẳng d) là miền nghiệm cuả bất phương trình bậc nhất hai ẩn nào sau đây?
		\choice
		{$2x-y<0$}
		{$x-2y<2$}
		{\True $2y-x<-2$}
		{$2x-y>1$}}{\begin{tikzpicture}[line join=round, line cap=round,>=stealth]
			\tikzset{label style/.style={font=\footnotesize}}
			\begin{scope}
				\clip (-2.5,-3) rectangle (3,2);
				\fill[pattern=north east lines] (-4.5,-3.25)--(7,-3.25)--(7,2.5)--cycle;
				\draw (6,2)--(-4,-3) node [pos=0.45, above, sloped] {};
			\end{scope}
			\draw[->] (-2.5,0)--(3,0) node[below]{$x$};
			\draw[->] (0,-3)--(0,2) node[left]{$y$};
			\draw (0,0) node[below left]{$O$};
			\foreach \x in {2}
			\draw[thin] (\x,1pt)--(\x,-1pt) node [below] {$\x$};
			\foreach \y in {-1}
			\draw[thin] (1pt,\y)--(-1pt,\y) node [left] {$\y$};
	\end{tikzpicture}}
	\loigiai{
		%Fb tác giả: Nguyễn Văn Mến\\
		Đường thẳng d đi qua hai điểm $A(0;-1)$ và $B(2;0)$ nên có phương trình là $y=\dfrac{1}{2}x-1$.\\
		Lại có điểm $O(0;0)$ không thuộc vào miền nghiệm nên $y<\dfrac{1}{2}x-1$ (vì $0<\dfrac{1}{2} \cdot 0-1$ \textbf{không đúng}).\\
		Hay $2y<x-2 \Leftrightarrow 2y-x<-2$.}
\end{ex}
\begin{ex}%[0D4K4-1]%
	\immini{ Bất phương trình nào sau đây có miền nghiệm (phần không gạch sọc) như hình vẽ bên?
		\choice
		{\True $2x-y+1<0$}
		{$x-y+1<0$ }
		{$2x-3y+1<0$ }
		{$2x-y-1<0$ }
	}
	{
		\begin{tikzpicture}[>=stealth, scale=0.7]
			\draw[->,line width = 1pt] (-3,0)--(0,0) node[below right]{$O$}--(4,0) node[below]{$x$};
			\draw[->,line width = 1pt] (0,-2) --(0,3.5) node[right]{$y$};
			\foreach \x in {-2,-1,1,2,3}{
				\draw (\x,0) node[below]{$\x$} circle (1pt);
				\draw (0,\x) node[left]{$\x$} circle (1pt);
			}
			\draw [pattern = north west lines, thick, domain=-1.5:1, samples=100] plot (\x, {2*(\x)+1}) node[right]{$d$};
			\draw[pattern = north east lines,opacity=.3, line width = 1.2pt,draw=none] plot[domain=-1.5:1] (\x, {2*(\x)+1})--(3,3)--(3,-2)--cycle;
		\end{tikzpicture}
	}
	\loigiai{Tập hợp các điểm biểu diễn nghiệm của bất phương trình $2x-y+1<0$ là nửa mặt phẳng bờ là đường thẳng $y=2x+1$ và không chứa gốc tọa độ.
		Từ đó ta có điểm $(2;1)$ thuộc miền nghiệm của bất phương trình.
	}
\end{ex}
\begin{ex}%[0D4B4-1]%
	Miền nghiệm của bất phương trình $x+y \leq 2$ là phần không bị gạch sọc của hình vẽ nào trong các hình sau?
	\choice
	{
		\begin{tikzpicture}[scale=1, font=\footnotesize, line join=round, line cap=round, >=stealth]
			\def\xmin{-1}\def\xmax{3.0}\def\ymin{-1}\def\ymax{3.0}
			\draw[->] (\xmin-0.2,0)--(\xmax+0.2,0) node[below] {\footnotesize $x$};
			\draw[->] (0,\ymin-0.2)--(0,\ymax+0.2) node[right] {$y$};
			\draw (0,0) node [below left] {$O$};
			\foreach \x in {-1,1,2,3}\draw (\x,0.1)--(\x,-0.1) node [below] {\footnotesize $\x$};
			\foreach \y in {-1,1,2,3}\draw (0.1,\y)--(-0.1,\y) node [left] {\footnotesize $\y$};
			\clip (\xmin,\ymin) rectangle (\xmax,\ymax);
			\draw[pattern = north west lines,smooth,samples=200,domain=\xmin:\xmax] plot (\x,{-1*(\x)+2});
			\draw[pattern = north east lines,opacity=.3, line width = 1.2pt,draw=none] plot[domain=\xmin:\xmax] (\x, {-1*(\x)+2})--(-1,-1)--(-1,3)--cycle;
		\end{tikzpicture}
	}
	{
		\begin{tikzpicture}[scale=1, font=\footnotesize, line join=round, line cap=round, >=stealth]
			\def\xmin{-3.0}\def\xmax{1}\def\ymin{-1}\def\ymax{3.0}
			\draw[->] (\xmin-0.2,0)--(\xmax+0.2,0) node[below] {\footnotesize $x$};
			\draw[->] (0,\ymin-0.2)--(0,\ymax+0.2) node[right] {$y$};
			\draw (0,0) node [below left] {$O$};
			\foreach \x in {-3,-2,-1,1}\draw (\x,0.1)--(\x,-0.1) node [below] {\footnotesize $\x$};
			\foreach \y in {-1,1,2,3}\draw (0.1,\y)--(-0.1,\y) node [left] {\footnotesize $\y$};
			\clip (\xmin,\ymin) rectangle (\xmax,\ymax);
			\draw[pattern = north west lines,smooth,samples=200,domain=\xmin:\xmax] plot (\x,{1*(\x)+2});
			\draw[pattern = north east lines,opacity=.3, line width = 1.2pt,draw=none] plot[domain=\xmin:\xmax] (\x, {1*(\x)+2})--(1,3)--(1,-1)--cycle;
		\end{tikzpicture}
	}
	{\True
		\begin{tikzpicture}[scale=1, font=\footnotesize, line join=round, line cap=round, >=stealth]
			\def\xmin{-1}\def\xmax{3.0}\def\ymin{-1}\def\ymax{3.0}
			\draw[->] (\xmin-0.2,0)--(\xmax+0.2,0) node[below] {\footnotesize $x$};
			\draw[->] (0,\ymin-0.2)--(0,\ymax+0.2) node[right] {$y$};
			\draw (0,0) node [below left] {$O$};
			\foreach \x in {-1,1,2,3}\draw (\x,0.1)--(\x,-0.1) node [below] {\footnotesize $\x$};
			\foreach \y in {-1,1,2,3}\draw (0.1,\y)--(-0.1,\y) node [left] {\footnotesize $\y$};
			\clip (\xmin,\ymin) rectangle (\xmax,\ymax);
			\draw[pattern = north west lines,smooth,samples=200,domain=\xmin:\xmax] plot (\x,{-1*(\x)+2});
			\draw[pattern = north east lines,opacity=.3, line width = 1.2pt,draw=none] plot[domain=\xmin:\xmax] (\x, {-1*(\x)+2})--(3,3)--(-1,3)--cycle;
		\end{tikzpicture}
	}
	{
		\begin{tikzpicture}[scale=1, font=\footnotesize, line join=round, line cap=round, >=stealth]
			\def\xmin{-3.0}\def\xmax{1}\def\ymin{-1}\def\ymax{3.0}
			\draw[->] (\xmin-0.2,0)--(\xmax+0.2,0) node[below] {\footnotesize $x$};
			\draw[->] (0,\ymin-0.2)--(0,\ymax+0.2) node[right] {$y$};
			\draw (0,0) node [below left] {$O$};
			\foreach \x in {-3,-2,-1,1}\draw (\x,0.1)--(\x,-0.1) node [below] {\footnotesize $\x$};
			\foreach \y in {-1,1,2,3}\draw (0.1,\y)--(-0.1,\y) node [left] {\footnotesize $\y$};
			\clip (\xmin,\ymin) rectangle (\xmax,\ymax);
			\draw[pattern = north west lines,smooth,samples=200,domain=\xmin:\xmax] plot (\x,{1*(\x)+2});
			\draw[pattern = north east lines,opacity=.3, line width = 1.2pt,draw=none] plot[domain=\xmin:\xmax] (\x, {1*(\x)+2})--(-3,3)--(-3,-1)--cycle;
		\end{tikzpicture}
	}
	\loigiai{
		\immini{
			Biểu diễn miền nghiệm trên mặt phẳng $Oxy$:\\
			- Vẽ đường thẳng $d: x+y=2$.\\
			- Lấy điểm $O(0;0)$ thay tọa độ vào ta có $0+0 \leq 2$ đúng.\\
			Vậy miền nghiệm bất phương trình là nửa mặt phẳng chứa điểm $O(0;0)$ và có bờ là đường thẳng $d$, kể cả đường thẳng $d$.
		}{
			\begin{tikzpicture}[scale=1, font=\footnotesize, line join=round, line cap=round, >=stealth]
				\def\xmin{-1}\def\xmax{3.0}\def\ymin{-1}\def\ymax{3.0}
				\draw[->] (\xmin-0.2,0)--(\xmax+0.2,0) node[below] {\footnotesize $x$};
				\draw[->] (0,\ymin-0.2)--(0,\ymax+0.2) node[right] {$y$};
				\draw (0,0) node [below left] {$O$};
				\foreach \x in {-1,1,2,3}\draw (\x,0.1)--(\x,-0.1) node [below] {\footnotesize $\x$};
				\foreach \y in {-1,1,2,3}\draw (0.1,\y)--(-0.1,\y) node [left] {\footnotesize $\y$};
				\clip (\xmin,\ymin) rectangle (\xmax,\ymax);
				\draw[pattern = north west lines,smooth,samples=200,domain=\xmin:\xmax] plot (\x,{-1*(\x)+2});
				\draw[pattern = north east lines,opacity=.3, line width = 1.2pt,draw=none] plot[domain=\xmin:\xmax] (\x, {-1*(\x)+2})--(3,3)--(-1,3)--cycle;
			\end{tikzpicture}
		}
	}
\end{ex}
\begin{ex}%[0D4K4-1]%Câu 24%[Dự án Tex hóa đề thi lớp 10-11-Nhóm Word-T-Begin-Lần 4]%[Lê Quốc Dũng\& Phản biện: Thanh Phong]%
	Cho bất phương trình $2x+3y-2<0$. Miền nghiệm của bất phương trình là
	\choice
	{\True nửa mặt phẳng chứa điểm $O$ có bờ là đường thẳng $2x+3y-2=0$ (không kể bờ)}
	{nửa mặt phẳng chứa điểm $O$ có bờ là đường thẳng $2x+3y-2=0$ (kể cả bờ)}
	{nửa mặt phẳng không chứa điểm $O$ có bờ là đường thẳng $2x+3y-2=0$ (không kể bờ)}
	{nửa mặt phẳng không chứa điểm $O$ có bờ là đường thẳng $2x+3y-2=0$ (kể cả bờ)}
	\loigiai{
		\immini{Vẽ đường thẳng $2x+3y-2=0$.\\
			Xét điểm $O(0;0)$ không thuộc đường thẳng $2x+3y-2=0$.\\
			Ta có $P=2 \cdot 0+3 \cdot 0-2<0$.\\
			Vậy nửa mặt phẳng chứa điểm $O$ có bờ là đường thẳng $2x+3y-2=0$ (không kể bờ) là miền nghiệm của bất phương trình.}{
			\begin{tikzpicture}[scale=0.8, font=\footnotesize, line join=round, line cap=round, >=stealth]
				\clip(-2,-1) rectangle (3,2);
				\draw[line width=0.8pt,dash pattern=on 3pt off 3pt,fill=black,pattern=north east lines,pattern color=black](-4.08,3.38)--(-4.08,-7.81)--(4.74,-7.81)--(4.74,-2.49)--(-4.08,3.38);
				\draw [->,line width=0.4pt] (-2,0) -- (3,0);
				\draw [->,line width=0.4pt] (0,-1) -- (0,2);
				\begin{scriptsize}
					\draw (-0.3,1.8) node {$y$};
					\draw (2.95,-0.2) node {$x$};
					\draw (-0.3,-0.3) node {$O$};
					\draw [fill=black] (1,0) circle (1pt);
					\draw (1,0.25 ) node {$1$};
					\draw (0,1) node[right] {$2x+3y-2<0$};
				\end{scriptsize}
	\end{tikzpicture}}}
\end{ex}
\begin{ex} %[Trần Ngọc Lam]%[1181-1200 Trần Chiến; PB: Nguyễn Tài Tuệ]%[0D4K4-1]%
	Miền nghiệm của bất phương trình $x-2y+5<0$ là
	\choice
	{ \True Nửa mặt phẳng không chứa gốc tọa độ, bờ là đường thẳng $y=\dfrac{1}{2}x+\dfrac{5}{2}$ (không bao gồm đường thẳng)}
	{ Nửa mặt phẳng chứa gốc tọa độ, bờ là đường thẳng $y=\dfrac{1}{2}x+\dfrac{5}{2}$ (không bao gồm đường thẳng)}
	{ Nửa mặt phẳng không chứa gốc tọa độ, bờ là đường thẳng $y=\dfrac{1}{2}x+\dfrac{5}{2}$ (bao gồm đường thẳng)}
	{ Nửa mặt phẳng chứa gốc tọa độ, bờ là đường thẳng $y=\dfrac{1}{2}x+\dfrac{5}{2}$ (không bao gồm đường thẳng)}
	\loigiai{
		\immini{
			Thay tọa độ điểm $ O(0;0) $ vào phương trình đường thẳng ta thấy không thỏa mãn.\\
			Do đó miền nghiệm là nửa mặt phẳng không chứa gốc tọa độ bờ là đường thẳng $ y=\dfrac{1}{2}x+\dfrac{5}{2} $( không bao gồm đường thẳng, như hình vẽ).}{
			\begin{tikzpicture}[thick,>=stealth,scale=0.7]
				\draw[->] (-4,0) -- (3,0) node[below]{\small $x$};
				\draw[->] (0,-2) -- (0,3) node[right]{\small $y$};
				\foreach \x in {1}
				\draw[shift={(\x,0)}] (0pt,2pt) -- (0pt,-2pt) node[right] {\footnotesize $\x$};
				\draw (0pt,-10pt) node[right] {\footnotesize $O$};
				\fill[black] (0,0) circle(2pt);
				\clip(-4,-2) rectangle (3,3);
				\draw[very thick, blue, smooth, domain=-4:3]
				plot(\x,{(\x+5)/2});
				\fill[pattern=north west lines] (-4,0.5)--(-4,3)--(1,3)--cycle;
			\end{tikzpicture}
		}
		
	}
\end{ex}
\begin{ex}%[Đỗ Vũ Minh Thắng]%[751-780 Lê Quốc Bảo]%[0D4K4-1]%
	Cặp điểm nào sau đây thuộc miền nghiệm của bất phương trình $3(x+\sqrt{2}y-\sqrt{3})>8(\sqrt{3}x+2y-\sqrt{2})$?
	\choice{$A(2;-2)$ và $B(2;2)$}
	{\True $C(-\sqrt{3};-\sqrt{2})$ và $D(\sqrt{2};-1-\sqrt{5})$}
	{$E(\sqrt{2};\sqrt{2})$ và $F(\sqrt{5}; 1)$}
	{$G(-\sqrt{2};2+\sqrt{3})$ và $H(1;4)$}
	\loigiai{Ta có $3(x+\sqrt{2}y-\sqrt{3})>8(\sqrt{3}x+2y-\sqrt{2}) \Leftrightarrow \left(3-8\sqrt{3}\right) x + \left(3\sqrt{2}-16\right) y -3\sqrt{3}+8\sqrt{2} >0$.\\
		Thay điểm $C(-\sqrt{3};-\sqrt{2})$ vào bất phương trình trên, ta có
		$$\left(3-8\sqrt{3}\right) \cdot (-\sqrt{3}) + \left(3\sqrt{2}-16\right) \cdot (-\sqrt{2}) -3\sqrt{3}+8\sqrt{2} = 18 - 6\sqrt{3} + 24\sqrt{2} >0 \text{ (đúng).}$$
		Thay điểm $D(\sqrt{2};-1-\sqrt{5})$ vào bất phương trình trên, ta có
		$$\left(3-8\sqrt{3}\right) \cdot (\sqrt{2}) + \left(3\sqrt{2}-16\right) \cdot (-1-\sqrt{5}) -3\sqrt{3}+8\sqrt{2}>0 \text{ (đúng).}$$
		Nên cặp điểm $C$, $D$ thuộc miền nghiệm của bất phương trình trên.
	}
\end{ex}
\begin{ex}%[0D4K4-1]%
	Giao miền nghiệm của ba bất phương trình $y\geq 0; 3x-2y\geq -6; 3x+4y\leq 12$ tạo thành một tam giác có diện tích bằng
	\choice
	{$18$}
	{\True $9$}
	{$6$}
	{$12$}
	\loigiai{
		\immini{
			Vẽ các đường thẳng $d_1: y=0; d_2: 3x-2y=6; d_3: 3x+4y=12$.\\
			- Lấy điểm $O(0;0)$ thế vào vế trái $d_2$ ta được $3\cdot 0-2\cdot 0 \geq -6$ đúng. Vậy miền nghiệm bất phương trình $3x-2y\geq -6$ chứa $O$ có bờ là $d_2$.\\
			- Lấy điểm $O(0;0)$ thế vào vế trái $d_3$ ta được $3\cdot 0+4\cdot 0 \leq 12$ đúng. Vậy miền nghiệm bất phương trình $3x+4y\leq 12$ chứa $O$ có bờ là $d_3$.\\
			Gọi $A, B, C$ là ba đỉnh của tam giác. Ta có $A(-2;0);B(0;3),C(4;0)$.\\
			Ta có $BO=3;AC=6$. Diện tích tam giác $ABC$ là
			\[S=\dfrac{1}{2}BO\cdot AC = \dfrac{1}{2}\cdot 3\cdot 6 = 9. \]
		}{
			\begin{tikzpicture}[scale=1, font=\footnotesize, line join=round, line cap=round, >=stealth]
				\def\xmin{-3.0}\def\xmax{5.0}\def\ymin{-1.0}\def\ymax{4.0}
				\draw[->] (\xmin-0.2,0)--(\xmax+0.2,0) node[below] {\footnotesize $x$};
				\draw[->] (0,\ymin-0.2)--(0,\ymax+0.2) node[right] {$y$};
				\draw (0,0) node [below left] {$O$};
				\foreach \x in {-3,-2,-1,1,2,3,4,5}\draw (\x,0.1)--(\x,-0.1) node [below] {\footnotesize $\x$};
				\foreach \y in {-1,1,2,3,4}\draw (0.1,\y)--(-0.1,\y) node [left] {\footnotesize $\y$};
				\clip (\xmin,\ymin) rectangle (\xmax,\ymax);
				\draw[pattern = north west lines,smooth,samples=200,domain=\xmin:\xmax] plot (\x,{1.5*(\x)+3});
				\draw[pattern = north east lines,opacity=.3, line width = 1.2pt,draw=none] plot[domain=\xmin:\xmax] (\x, {1.5*(\x)+3})--(-3,4)--(-3,-1)--cycle;
				\draw[pattern = north west lines,smooth,samples=200,domain=\xmin:\xmax] plot (\x,{-0.75*(\x)+3});
				\draw[pattern = north east lines,opacity=.3, line width = 1.2pt,draw=none] plot[domain=\xmin:\xmax] (\x, {-0.75*(\x)+3})--(5,-1)--(5,4)--cycle;
				\draw[pattern = north west lines,smooth,samples=200,domain=\xmin:\xmax] plot (\x,{0*(\x)});
				\draw[pattern = north east lines,opacity=.3, line width = 1.2pt,draw=none] plot[domain=\xmin:\xmax] (\x, {0*(\x)})--(5,-1)--(-3,-1)--cycle;
			\end{tikzpicture}
		}
	}
\end{ex}
\begin{ex}%[0D4K4-1]%
	Giao miền nghiệm của ba bất phương trình $x+4y\geq 8; -x+2y\leq 4; x+y\leq 5$ tạo thành một tam giác có chu vi bằng
	\choice
	{\True $\sqrt{17}+\sqrt{5}+2\sqrt{2}$}
	{$\sqrt{17}+\sqrt{5}+\sqrt{2}$}
	{$\sqrt{17}+2\sqrt{5}+\sqrt{2}$}
	{$\sqrt{17}+2\sqrt{5}+2\sqrt{2}$}
	\loigiai{
		\immini{
			Vẽ các đường thẳng $d_1: x+4y=8; d_2: -x+2y=4; d_3: x+y=5$.\\
			- Lấy điểm $O(0;0)$ thế vào vế trái $d_1$ ta được $3\cdot 0+4\cdot 0 \geq 8$ sai. Vậy miền nghiệm bất phương trình $x+4y\geq 8$ không chứa $O$ có bờ là $d_1$.\\
			- Lấy điểm $O(0;0)$ thế vào vế trái $d_2$ ta được $-0+2\cdot 0 \leq 4$ đúng. Vậy miền nghiệm bất phương trình $3-x+2y\leq 4$ chứa $O$ có bờ là $d_2$.\\
			- Lấy điểm $O(0;0)$ thế vào vế trái $d_3$ ta được $ 0+ 0 \leq 5$ đúng. Vậy miền nghiệm bất phương trình $x+y\leq 5$ chứa $O$ có bờ là $d_3$.\\
			Gọi $A, B, C$ là ba đỉnh của tam giác. Ta có $A(0;2);B(4;1),C(2;3)$.\\
			Ta có:\\
			$AB=\sqrt{(4-0)^2+(1-2)^2}=\sqrt{17}$. \\
			$AC=\sqrt{(2-0)^2+(3-2)^2}=\sqrt{5}$.\\
			$BC=\sqrt{(2-4)^2+(3-1)^2}=2\sqrt{2}$.\\
			Chu vi tam giác $ABC$ là
			\[2P = \sqrt{17}+\sqrt{5}+2\sqrt{2} . \]
		}{
			\begin{tikzpicture}[scale=1, font=\footnotesize, line join=round, line cap=round, >=stealth]
				\def\xmin{-1}\def\xmax{6.0}\def\ymin{-1}\def\ymax{6.0}
				\draw[->] (\xmin-0.2,0)--(\xmax+0.2,0) node[below] {\footnotesize $x$};
				\draw[->] (0,\ymin-0.2)--(0,\ymax+0.2) node[right] {$y$};
				\draw (0,0) node [below left] {$O$};
				\foreach \x in {-1,1,2,3,4,5,6}\draw (\x,0.1)--(\x,-0.1) node [below] {\footnotesize $\x$};
				\foreach \y in {-1,1,2,3,4,5,6}\draw (0.1,\y)--(-0.1,\y) node [left] {\footnotesize $\y$};
				\clip (\xmin,\ymin) rectangle (\xmax,\ymax);
				\draw[pattern = north west lines,smooth,samples=200,domain=\xmin:\xmax] plot (\x,{-1*(\x)+5});
				\draw[pattern = north east lines,opacity=.3, line width = 1.2pt,draw=none] plot[domain=\xmin:\xmax] (\x, {-1*(\x)+5})--(6,6)--(-1,6)--cycle;
				\draw[pattern = north west lines,smooth,samples=200,domain=\xmin:\xmax] plot (\x,{0.5*(\x)+2});
				\draw[pattern = north east lines,opacity=.3, line width = 1.2pt,draw=none] plot[domain=\xmin:\xmax] (\x, {0.5*(\x)+2})--(6,6)--(-1,6)--cycle;
				\draw[pattern = north west lines,smooth,samples=200,domain=\xmin:\xmax] plot (\x,{-0.25*(\x)+2});
				\draw[pattern = north east lines,opacity=.3, line width = 1.2pt,draw=none] plot[domain=\xmin:\xmax] (\x, {-0.25*(\x)+2})--(6,-1)--(-1,-1)--cycle;
				\tkzDefPoint(0,2){A}
				\tkzDefPoint(4,1){B}
				\tkzDefPoint(2,3){C}
				\tkzDrawPoints[color=blue](A,B,C)
				\tkzLabelPoints[right](A)
				\tkzLabelPoints[right](B)
				\tkzLabelPoints[above](C)
				\draw[dashed](2,0)--(2,3)--(0,3);
				\draw[dashed](4,0)--(4,1)--(0,1);
			\end{tikzpicture}
		}
	}
\end{ex}
\begin{ex}%[Đỗ Vũ Minh Thắng]%[751-780 Lê Quốc Bảo]%[0D4K4-1]%
	Tìm tất cả các giá trị thực của tham số $m$ để bất phương trình $3x+my-7 \geq 0$ có miền nghiệm chứa điểm $A(\sqrt{2};1)$.
	\choice{$m\in [3\sqrt{2}-7; +\infty )$}
	{$m\in (-\infty ;3\sqrt{2}-7)$}
	{$m\in (-\infty ;7-3\sqrt{3})$}
	{\True $m\in [7-3\sqrt{2}; +\infty )$}
	\loigiai{Vì điểm $A(\sqrt{2};1)$ thuộc miền nghiệm của bất phương trình đã cho, nên
		$$3 \cdot \sqrt{2} + m \cdot 1 - 7 \geq 0 \Leftrightarrow m \geq 7- 3 \cdot \sqrt{2}.$$
	}
\end{ex}
\begin{ex}%[Đỗ Vũ Minh Thắng]%[781-810 Phạm Quốc Toàn]%[0D4K4-1]%
	%781
	Cho bất phương trình $mx+\sqrt{2}y-1<0$ với $m$ là tham số thực. Điểm nào dưới đây luôn luôn \textbf{không} thuộc miền nghiệm của bất phương trình đã cho?
	\choice
	{$E(m;m^2)$}
	{$F(2m^2;m)$}
	{\True $G(0;1+m^2)$}
	{$H(0;-1-m^2)$}
	\loigiai{Điểm $E (m; m^2)$ không thỏa mãn vì $m^2+\sqrt{2} m^2 -1<0 \Leftrightarrow - \dfrac{1}{\sqrt{1 + \sqrt{2}}} < m <  \dfrac{1}{\sqrt{1 + \sqrt{2}}}$. \\
		Điểm $F(2m^2;m)$ không thỏa mãn vì $2 m^3 + \sqrt{2} m-1<0$ (bất phương trình này luôn có nghiệm). \\
		Điểm $H(0;-1-m^2)$ không thỏa mãn vì $m.0 +\sqrt{2} (-1 - m^2) -1<0 \Leftrightarrow \sqrt{2} m^2 > -1 - \sqrt{2}$ (thỏa mãn với mọi $m$). \\
		Với điểm $G(0;1+m^2)$, ta có $mx+\sqrt{2}y-1=m.0 +\sqrt{2} (1 +m^2) - 1 = \sqrt{2} m^2 + (\sqrt{2} - 1) > 0$ với mọi $m \in \mathbb{R}$. Vậy điểm $G$ không thuộc miền nghiệm của bất phương trình đã cho.
	}
\end{ex}
\begin{ex}%[Nguyễn Phúc Đức]%[781-810 Phạm Quốc Toàn]%[0D4K4-1]%
	%782
	Với giá trị nào của $m$ thì điểm $A(1-m;m)$ {\bf không thuộc} miền nghiệm của bất phương trình $2x-3(y-x)>4$.
	\choice
	{$0\leq m \leq 1$}
	{$m<\dfrac{1}{8}$}
	{$\dfrac{1}{8}\leq m\leq 1$}
	{\True $m \geq \dfrac{1}{8}$}
	\loigiai{ $A(1-m;m)$ không thuộc miền nghiệm của bất phương trình $2x-3(y-x)>4$ khi tọa độ của nó không thỏa mãn bất phương trình, tức là $2(1-m)-3(m+m-1) \leq 4$ hay $m \geq \dfrac{1}{8}$.}
\end{ex}
\begin{ex}%[Bài thi mẫu khảo sát, ĐHQG TP Hồ Chí Minh, 2019]%[Ngụy Như Thái, 12EX4]%[0D4K4-3]%
	Một bác nông dân cần trồng lúa và khoai trên diện tích đất $6$ ha, với lượng phân bón dự trữ là $100$ kg và sử dụng tối đa $120$ ngày công. Để trồng $1$ ha lúa cần sử dụng $20$ kg phân bón, $10$ ngày công với lợi nhuận là $30$ triệu đồng; để trồng $1$ ha khoai cần sử dụng $10$ kg phân bón, $30$ ngày công với lợi nhuận là $60$ triệu đồng. Để đạt lợi nhuận cao nhất, bác nông dân đã trồng $x$ (ha) lúa và $y$ (ha) khoai. Giá trị của $x$ là
	\choice
	{$2$}
	{$3$}
	{\True  $4$}
	{$5$}
	\loigiai{
		Theo bài toán, ta có:\\
		$ \heva{& x+y=6\\&20x+10y\leq 100\\&10x+30y\leq 120\\&
			T=30x+60y \longrightarrow Max}$
		$\Leftrightarrow \heva{& y=6-x\\&x\leq 4\\ &x\geq 3\\& T=24x+360 \longrightarrow Max}$
		$\Leftrightarrow \heva{&y=6-x\\&3\leq x\leq 4\\& T=24x+360 \longrightarrow Max.}$\\
		Vì $T=24x+360$ là hàm số bậc nhất và có hệ số $a=24>0$ nên nó đạt GTLN tại $x=4$.\\
		Vậy $x=4$ là giá trị cần tìm.
	}
\end{ex}

\begin{ex}%[BG Toán 10-2022]%[Trần Nhân Kiệt]%[0D4T4-3]
	Một người thợ mộc tốn $6$ giờ để làm một cái bàn và $4$ giờ để làm một cái ghế. Gọi $x$, $y$ lần lượt là số bàn và số ghế mà người thợ mộc sản xuất trong một tuần. Viết bất phương trình biểu thị mối liên hệ giữa $x$ và $y$ biết trong một tuần người thợ mộc có thể làm tối đa $50$ giờ.
	\choice
	{\True $3x+2y\le 25$}
	{$3x+2y> 25$}
	{$3x+2y\ge 25$}
	{$3x+2y< 25$}
	\loigiai{
		Trong một tuần, số giờ làm ra $x$ cái bàn là $6x$ và số giờ làm ra $y$ cái ghế là $4y$.\\
		Vì trong một tuần người thợ mộc làm tối đa $50$ giờ nên 
		$$6x+4y\le 50\Leftrightarrow 3x+2y\le 25.$$
	}
\end{ex}
\begin{ex}%[BG Toán 10-2022]%[Trần Nhân Kiệt]%[0D4T4-3]
	Một gian hàng trưng bày bàn và ghế rộng $60$ m$^2$. Diện tích để kê một chiếc ghế là $0{,}6$ m$^2$, một chiếc bàn là $1{,}3$ m$^2$. Gọi $x$ là số chiếc ghế, $y$ là số chiếc bàn được kê. Viết bất phương trình bậc nhất hai ẩn $x$, $y$ cho phần mặt sàn để kê bàn và ghế, biết diện tích mặt sàn dành cho lưu thông tối thiểu là $10$ m$^2$.
	\choice
	{$0{,}6x+1{,}3y\ge 50$}
	{\True $0{,}6x+1{,}3y\le 50$}
	{$1{,}3x+0{,}6y\le 50$}
	{$1{,}3x+0{,}6y\ge 50$}
	\loigiai{
		Diện tích để kê $x$ chiếc ghế và $y$ chiếc bàn là $0{,}6x+1{,}3y$.\\
		Vì diện tích mặt sàn dành cho lưu thông tối thiểu là $10$ m$^2$ nên diện tích để kê $x$ chiếc ghế và $y$ chiếc bàn tối đa là $50$ m$^2$.\\
		Do đó $0{,}6x+1{,}3y\le 50$.
	}
\end{ex}
\begin{ex}%[BG Toán 10-2022]%[Trần Nhân Kiệt]%[0D4T4-3]
	Bạn Nam đang sưu tầm các đồng tiền vàng và bạc để vào một các túi, trọng lượng tối đa mà túi chứa được là $20$ gam. Mỗi đồng xu vàng nặng khoảng $14$ gam, mỗi đồng xu bạc nặng khoảng $7$ gam. Bất phương trình nào sau đây mô tả số đồng tiền vàng ($x$) và số đồng tiền bạc ($y$) có thể được chứa trong túi?
	\choice
	{$7x+14y\le 20$}
	{$7x+14y>20$}
	{\True $14x+7y\le 20$}
	{$14x+7y>20$}
	\loigiai{
		Khối lượng của $x$ đồng tiền vàng là $14x$ gam.\\
		Khối lượng của $y$ đồng tiền bạc là $7y$ gam.\\	
		Số đồng tiền vàng và bạc có thể chứa trong túi khi $14x+7y\le 20$.
	}
\end{ex}
\begin{ex}%[BG Toán 10-2022]%[Trần Nhân Kiệt]%[0D4T4-3]
	Trong $1$ lạng ($100$ g) thịt bò chứa khoảng $26$ g protein và $1$ lạng cá rô phi chứa khoảng $20$ g protein. Trung bình trong một ngày, một người đàn ông cần tối thiểu $52$ g protein. Gọi $x$, $y$ lần lượt là số lạng thịt bò và số lạng cá rô phi mà một người đàn ông nên ăn trong một ngày. Viết bất phương trình bậc nhất hai ẩn $x$, $y$ để biểu diễn lượng protein cần thiết cho một người đàn ông trong một ngày.
	\choice
	{$26x+20y\le 52$}
	{$26x+20y< 52$}
	{\True $13x+10y\ge 26$}
	{$13x+10y> 26$}
	\loigiai{
		Trong $x$ lạng thịt bò chứa $26x$ g protein.\\
		Trong $y$ lạng cá rô phi chứa $20y$ g protein.\\
		Do đó lượng protein cần thiết trong một ngày của một người đàn ông là 
		$$26x+20y\ge 52\Leftrightarrow 13x+10y\ge 26.$$
	}
\end{ex}
\begin{ex}%[BG Toán 10-2022]%[Trần Nhân Kiệt]%[0D4T4-3]
	Công ty viễn thông Viettel có gói cước Hi School tính phí là $1190$ đồng mỗi phút gọi nội mạng và $1390$ đồng mỗi phút gọi ngoại mạng. Một bạn học sinh đăng kí gói cước trên và sử dụng $x$ phút gọi nội mạng, $y$ phút gọi ngoại mạng trong một tháng. Viết bất phương trình bậc nhất hai ẩn $x$, $y$ để mô tả số tiền bạn đó phải trả trong một tháng ít hơn $100$ nghìn đồng.
	\choice
	{$119x+139y\ge 10000$}
	{$139x+119y< 10000$}
	{$119x+139y\le 10000$}
	{\True $119x+139y< 10000$}
	\loigiai{
		Trong một tháng, số tiền gọi nội mạng là $1190 x$ đồng và số tiền gọi ngoại mạng là $1390y$ đồng.\\
		Tổng số tiền trong một tháng bạn học sinh phải trả là $1190x+1390y$.\\
		Để số tiền trong một tháng phải trả ít hơn $100$ nghìn đồng thì 
		$$1190x+1390y< 100000\Leftrightarrow 119y+139y< 10000.$$
		
	}
\end{ex}
\begin{ex}%[BG Toán 10-2022]%[Trần Nhân Kiệt]%[0D4T4-3]
	Nhân ngày Quốc tế Thiếu nhi $1-6$, một rạp chiếu phim phục vụ các khán giả một bộ phim hoạt hình. Vé được bán ra có hai loại: loại $1$ dành cho trẻ từ $6-13$ tuổi, giá vé là $50000$ đồng/vé và loại $2$ dành cho người trên $13$ tuổi, giá vé là $80000$ đồng/vé. Gọi $x$ là số vé loại $1$ và $y$ là số vé loại $2$ bán được. Viết bất phương trình bậc nhất hai ẩn $x$, $y$ để biểu diễn điều kiện sao cho số tiền bán vé thu được tối thiểu $10$ triệu đồng.
	\choice
	{$5x+8y\ge 100$}
	{$5x+8y> 1000$}
	{$8x+5y\ge 1000$}
	{\True $5x+8y\ge 1000$}
	\loigiai{
		Số tiền thu được từ $x$ vé loại $1$ là $50000x$ và số tiền thu được từ $y$ vé loại $2$ là $80000y$.\\
		Tổng số tiền bán vé thu được là $50000x+80000y$.\\
		Để số tiền bán vé thu được tối thiểu $10$ triệu đồng thì 
		$$50000x+80000y\ge 10000000\Leftrightarrow 5x+8y\ge 1000.$$
	}
\end{ex}

\begin{ex}%[BG Toán 10-2022]%[Trần Nhân Kiệt]%[0D4T4-3]
	Ngoài giờ học, bạn Nam làm thêm việc phụ bán cơm được $15$ nghìn đồng/một giờ và phụ bán tạp hóa được $10$ nghìn đồng/một giờ. Gọi $x$, $y$ lần lượt là số giờ phụ bán cơm và phụ bán tạp hóa trong mỗi tuần. Viết bất phương trình bậc nhất hai ẩn $x$ và $y$ sao cho Nam kiếm thêm tiền mỗi tuần được ít nhất là $900$ nghìn đồng.
	\choice
	{$3x+2y\le 180$}
	{$3x+2y> 180$}
	{\True $3x+2y\ge 180$}
	{$3x+2y< 180$}
	\loigiai{
		Số tiền từ việc phụ bán cơm là $15x$ nghìn đồng và số tiền từ việc phụ bán tạp hóa là $10y$ nghìn đồng.\\
		Số tiền Nam kiếm được mỗi tuần là $15x+10y$.\\
		Để số tiền Nam kiếm được mỗi tuần ít nhất là $900$ nghìn đồng thì 
		$$15x+10y\ge 900\Leftrightarrow 3x+2y\ge 180.$$
	}
\end{ex}


\begin{ex}%[BG Toán 10-2022]%[Trần Nhân Kiệt]%[0D4T4-3]
	Anh A muốn thuê một chiếc ô tô (có người lái) trong một tuần. Giá thuê xe như sau: từ thứ hai đến thứ sáu phí cố định là $900$ nghìn đồng/ngày và phí tính theo quãng đường di chuyển là $10$ nghìn đồng/km còn thứ bảy và chủ nhật thì phí cố định là $1200$ nghìn đồng/ngày và phí tính theo quãng đường di chuyển là $15$ nghìn đồng/km. Gọi $x$, $y$ lần lượt là số km mà anh A đi trong các ngày từ thứ hai đến thứ sáu và trong hai ngày cuối tuần. Viết bất phương trình biểu thị mối liên hệ giữa $x$ và $y$ sao cho tổng số tiền anh A phải trả không quá $20$ triệu đồng.
	\choice
	{$10x+15y\le 20000$}
	{$2x+3y\ge 2720$}
	{$10x+15y\ge 20000$}
	{\True $2x+3y\le 2720$}
	\loigiai{
		Số tiền thuê xe của anh A từ thứ hai đến thứ sáu là $900\cdot 5+10x$ nghìn đồng và hai ngày thứ bảy, chủ nhật là $1200\cdot 2+15y$ nghìn đồng.\\
		Để số tiền anh A phải trả không quá $20$ triệu đồng thì 
		$$(900\cdot 5+10x)+(1200\cdot 2+15y)\le 20000\Leftrightarrow 2x+3y\le 2720.$$
	}
\end{ex}

\begin{ex}%[BG Toán 10-2022]%[Trần Nhân Kiệt]%[0D4T4-3]
	Một cửa hàng làm kệ sách và bàn làm việc. Mỗi kệ sách cần $4$ giờ hoàn thiện. Mỗi bàn làm việc cần $3$ giờ hoàn thiện. Mỗi tháng cửa hàng có tối đa $240$ giờ làm việc. Hãy biểu diễn đồ thị mô tả số giờ làm việc trong mỗi tháng của cửa hàng theo số kệ sách hoàn thiện ($x$) và số bàn hoàn thiện ($y$).
	\choice
	{\centering \begin{tikzpicture}[scale=.7,font=\footnotesize, line join=round, line cap=round, >=stealth]
			\draw[->] (0,0) -- (5.3,0)node[below]{$x$};
			\draw[->,color=black] (0,0) -- (0,5.3)node[left]{$y$};
			\node[above left] at (0,0){$O$};
			\node[below] at (4,0){$80$};
			\node[left] at (0,3){$60$};
			\clip(0,0) rectangle (5.3,5.3);
			\fill[pattern=north east lines](4,0)-- (5.3,0) -- (5.3,5.3) -- (0,5.3)--(0,3)-- cycle;
			\draw[line width=1.2pt,smooth,samples=100,domain=0:4] plot(\x,{-0.75 *(\x) +3});
	\end{tikzpicture}}
	{\centering \begin{tikzpicture}[scale=.7,font=\footnotesize, line join=round, line cap=round, >=stealth]
			\draw[->] (0,0) -- (5.3,0)node[below]{$x$};
			\draw[->,color=black] (0,0) -- (0,5.3)node[left]{$y$};
			\node[above left] at (0,0){$O$};
			\node[below] at (4,0){$80$};
			\node[left] at (0,3){$60$};
			\clip(0,0) rectangle (5.3,5.3);
			\fill[pattern=north east lines](0,0)-- (4,0) -- (0,3) -- cycle;
			\draw[line width=1.2pt,smooth,samples=100,domain=0:4] plot(\x,{-0.75 *(\x) +3});
	\end{tikzpicture}}
	{\True \centering \begin{tikzpicture}[scale=.7,font=\footnotesize, line join=round, line cap=round, >=stealth]
			\draw[->] (0,0) -- (5.3,0)node[below]{$x$};
			\draw[->,color=black] (0,0) -- (0,5.3)node[left]{$y$};
			\node[above left] at (0,0){$O$};
			\node[below] at (3,0){$60$};
			\node[left] at (0,4){$80$};
			\clip(0,0) rectangle (5.3,5.3);
			\fill[pattern=north east lines](3,0)-- (5.3,0) -- (5.3,5.3) -- (0,5.3)--(0,4)-- cycle;
			\draw[line width=1.2pt,smooth,samples=100,domain=0:3] plot(\x,{-1.3333 *(\x) +4});
	\end{tikzpicture}}
	{\centering \begin{tikzpicture}[scale=.7,font=\footnotesize, line join=round, line cap=round, >=stealth]
			\draw[->] (0,0) -- (5.3,0)node[below]{$x$};
			\draw[->,color=black] (0,0) -- (0,5.3)node[left]{$y$};
			\node[above left] at (0,0){$O$};
			\node[below] at (3,0){$60$};
			\node[left] at (0,4){$80$};
			\clip(0,0) rectangle (5.3,5.3);
			\fill[pattern=north east lines](0,0)-- (4,0) -- (0,3)-- cycle;
			\draw[line width=1.2pt,smooth,samples=100,domain=0:3] plot(\x,{-1.3333 *(\x) +4});
	\end{tikzpicture}}
	\loigiai{
		\immini{Ta có bất phương trình $4x+3y \le 240$ mô tả số giờ làm việc trong mỗi tháng của cửa hàng. Biểu diễn nghiệm của bất phương trình như sau
		}
		{
			\centering \begin{tikzpicture}[scale=.7,font=\footnotesize, line join=round, line cap=round, >=stealth]
				\draw[->] (0,0) -- (5.3,0)node[below]{$x$};
				\draw[->,color=black] (0,0) -- (0,5.3)node[left]{$y$};
				\node[above left] at (0,0){$O$};
				\node[below] at (3,0){$60$};
				\node[left] at (0,4){$80$};
				\clip(0,0) rectangle (5.3,5.3);
				\fill[pattern=north east lines](3,0)-- (5.3,0) -- (5.3,5.3) -- (0,5.3)--(0,4)-- cycle;
				\draw[line width=1.2pt,smooth,samples=100,domain=0:3] plot(\x,{-1.3333 *(\x) +4});
			\end{tikzpicture}
		}
	}
\end{ex}


\begin{ex}%[BG Toán 10-2022]%[Trần Nhân Kiệt]%[0D4T4-3]
	Một gia đình cần $x$ kg thịt bò và $y$ kg thịt lợn trong một ngày, giá tiền $1$ kg thịt bò là $200$ nghìn đồng, $1$ kg thịt lợn là $60$ nghìn đồng. Biểu diễn đồ thị mô tả chi phí gia đình đó mua thịt bò và thịt lợn mỗi ngày để số tiền bỏ ra trong một ngày không quá $300$ nghìn đồng.
	\choice
	{\True \begin{tikzpicture}[scale=.7,font=\footnotesize, line join=round, line cap=round, >=stealth]
			\tikzset{label style/.style={font=\footnotesize}}
			\begin{scope}
				\clip (0,0) rectangle (4,6);
				\fill[pattern=north east lines] (-1,8.33)--(5,8.33)--(5,-11.67)--cycle;
				\draw (-0.3,6)--(1.5,0) node [pos=0.45, above, sloped] {};
			\end{scope}
			\draw[->] (0,0)--(4,0) node[below]{$x$};
			\draw[->] (0,0)--(0,6) node[left]{$y$};
			\draw (0,0) node[below left]{$O$};
			\draw (1.5,0) node[below]{$1{,}5$};
			\foreach \y in {5}
			\draw[thin] (1pt,\y)--(-1pt,\y) node [left] {$\y$};
	\end{tikzpicture}}
	{\begin{tikzpicture}[scale=.7,font=\footnotesize, line join=round, line cap=round, >=stealth]
			\tikzset{label style/.style={font=\footnotesize}}
			\begin{scope}
				\clip (0,0) rectangle (4,6);
				\fill[pattern=north east lines] (-1,8.33)--(-1,-11.67)--(5,-11.67)--cycle;
				\draw (-0.3,6)--(1.5,0) node [pos=0.45, above, sloped] {};
			\end{scope}
			\draw[->] (0,0)--(4,0) node[below]{$x$};
			\draw[->] (0,0)--(0,6) node[left]{$y$};
			\draw (0,0) node[below left]{$O$};
			\draw (1.5,0) node[below]{$1{,}5$};
			\foreach \y in {5}
			\draw[thin] (1pt,\y)--(-1pt,\y) node [left] {$\y$};
	\end{tikzpicture}}
	{\begin{tikzpicture}[scale=.7,font=\footnotesize, line join=round, line cap=round, >=stealth]
			\tikzset{label style/.style={font=\footnotesize}}
			\begin{scope}
				\clip (0,0) rectangle (4,6);
				\fill[pattern=north east lines] (-1,8.33)--(5,8.33)--(5,-11.67)--cycle;
				\draw (-0.3,6)--(1.5,0) node [pos=0.45, above, sloped] {};
			\end{scope}
			\draw[->] (0,0)--(4,0) node[below]{$x$};
			\draw[->] (0,0)--(0,6) node[left]{$y$};
			\draw (0,0) node[below left]{$O$};
			\draw (1.5,0) node[below]{$1$};
			\foreach \y in {5}
			\draw[thin] (1pt,\y)--(-1pt,\y) node [left] {$\y$};
	\end{tikzpicture}}
	{\begin{tikzpicture}[scale=.7,font=\footnotesize, line join=round, line cap=round, >=stealth]
			\tikzset{label style/.style={font=\footnotesize}}
			\begin{scope}
				\clip (0,0) rectangle (4,6);
				\fill[pattern=north east lines] (-1,8.33)--(-1,-11.67)--(5,-11.67)--cycle;
				\draw (-0.3,6)--(1.5,0) node [pos=0.45, above, sloped] {};
			\end{scope}
			\draw[->] (0,0)--(4,0) node[below]{$x$};
			\draw[->] (0,0)--(0,6) node[left]{$y$};
			\draw (0,0) node[below left]{$O$};
			\draw (1.5,0) node[below]{$1$};
			\foreach \y in {5}
			\draw[thin] (1pt,\y)--(-1pt,\y) node [left] {$\y$};
	\end{tikzpicture}}
	\loigiai{
		Số tiền mua thịt bò là $200x$ và số tiền mua thịt lợn là $60y$.\\
		Tổng số tiền trong một ngày mua thịt lợn và thịt bò là $200x+60y$.\\
		Để chi phí mua thịt bò và thịt lợn mỗi ngày không quá $300$ nghìn đồng thì 
		$$200x+60y\le 300\Leftrightarrow 10x+3y\le 15.$$
		Khi đó biểu diễn đồ thị mô tả chi phí là
		\begin{center}
			\begin{tikzpicture}[scale=.7,font=\footnotesize, line join=round, line cap=round, >=stealth]
				\tikzset{label style/.style={font=\footnotesize}}
				\begin{scope}
					\clip (0,0) rectangle (4,6);
					\fill[pattern=north east lines] (-1,8.33)--(5,8.33)--(5,-11.67)--cycle;
					\draw (-0.3,6)--(1.5,0) node [pos=0.45, above, sloped] {};
				\end{scope}
				\draw[->] (0,0)--(4,0) node[below]{$x$};
				\draw[->] (0,0)--(0,6) node[left]{$y$};
				\draw (0,0) node[below left]{$O$};
				\draw (1.5,0) node[below]{$1{,}5$};
				\foreach \y in {5}
				\draw[thin] (1pt,\y)--(-1pt,\y) node [left] {$\y$};
			\end{tikzpicture}
		\end{center}
	}
\end{ex}
\Closesolutionfile{ans}
% \Closesolutionfile{ansbook}
% \indapan{10}{ans/ans-BPTbacnhathaian}
% \subsection{Lời giải câu hỏi trắc nghiệm}
% \input{ans/ansbook-BPTbacnhathaian}
% \section{Hệ bất phương trình bậc nhất hai ẩn}
\setcounter{dang}{0}
\subsection{Tóm tắt lý thuyết}
\subsubsection{Khái niệm hệ bất phương trình bậc nhất hai ẩn}
\begin{boxdn}
	\textbf{\textit{Hệ bất phương trình bậc nhất hai ẩn}} là hệ gồm hai hay nhiều bất phương trình bậc nhất hai ẩn $x$, $y$. Mỗi nghiệm chung của tất cả các bất phương trình đó được gọi là một nghiệm của hệ bất phương trình đã cho.\\
	Trên mặt phẳng toạ độ $Oxy$, tập hợp các điểm $(x_0;y_0)$ có tọa độ là nghiệm của hệ bất phương trình bậc nhất hai ẩn được gọi là \textbf{\textit{miền nghiệm}} của hệ bất phương trình đó.
\end{boxdn}
\subsubsection{Biểu diễn miền nghiệm của hệ bất phương trình bậc nhất hai ẩn}
\begin{boxdn}
	Để \textbf{\textit{biểu diễn miền nghiệm}} của hệ bất phương trình bậc nhất hai ẩn trên mặt phẳng toạ độ $Oxy$, ta thực hiện như sau:
	\begin{itemize}
	\item Trên cùng mặt phẳng tọa độ, biểu diễn miền nghiệm của mỗi bất phương trình của hệ.
	\item Phần giao của các miền nghiệm là miền nghiệm của hệ bất phương trình.
	\end{itemize}
\end{boxdn}
\begin{note}
	Miền mặt phẳng tọa độ bao gồm một đa giác lồi và phần nằm bên trong đa giác đó được gọi là một miền đa giác.
\end{note}
\subsubsection{Tìm giá trị lớn nhất và giá trị nhỏ nhất của biểu thức $\mathbf{F=ax+by}$ trên một miền đa giác}
Hệ bất phương trình giúp ta mô tả được nhiều bài toán thực tế để tìm ra cách giải quyết tối ưu. Chúng thường được đưa về bài toán tìm giá trị lớn nhất (GTLN) hoặc giá trị nhỏ nhất (GTNN) của biểu thức $F=ax+by$ trên một miền đa giác.\\
Người ta chứng minh được $F$ đạt giá trị lớn nhất hoặc nhỏ nhất tại một trong các đỉnh của đa giác.
\subsection{Các dạng toán}
\begin{dang}{Biểu diễn hình học của tập nghiệm}
	
\end{dang}

\viduminhhoa
\begin{vd}%[0D4B4]
	Biểu diễn hình học tập nghiệm của hệ bất phương trình bậc nhất hai ẩn sau
	$$\left\{\begin{aligned}
		x+y &> 1\\
		x-y &<2 \\
	\end{aligned}\right.$$
	\loigiai
	{\immini{
			Vẽ các đường thẳng
			\begin{gather*}
				d_1: x+y=1,
				d_2: x-y=2.
			\end{gather*}
			Vì điểm $M(0,2)$ có tọa độ thỏa mãn các bất phương trình trong hệ nên ta tô đậm các nửa mặt phẳng bờ $d_1,d_2$ không chứa $M$. 
			
			Miền không bị tô đậm trong hình vẽ và không chứa các tia giới hạn miền là miền nghiệm của hệ đã cho.
		}{
			\begin{tikzpicture}[line cap=round, line join=round, font=\footnotesize, >=stealth, scale=1]
				\tikzset{label style/.style={font=\footnotesize}}
				\draw[color=white,fill=black!10] (-2,-2) rectangle (4,3);
				\draw[fill=white,color=white] (-2,3) -- (1.5,-.5) -- (4,2) -- (4,3) -- cycle;
				\draw[->] (0,-2) -- (0,3) node[left]{\scriptsize $y$};
				\draw[->] (-2,0) -- (4,0) node[above]{\scriptsize $x$};
				\draw (0.15,0.1) node[below left]{\scriptsize $O$};
				\draw (-2,3) -- (3,-2) node[right]{\scriptsize $d_1$};
				\draw (0,-2) -- (4,2) node[right]{\scriptsize $d_2$};
				\draw[dashed] (0,-0.5)node[left]{\scriptsize $-\dfrac{1}{2}$} -- (1.5,-.5) node[right]{\scriptsize $I$} -- (1.5,0) node[above]{\scriptsize $\dfrac{3}{2}$};
				\draw (0,2) node[right]{\scriptsize $M$} (0,2) node[left]{\scriptsize $2$} (-0.05,2) -- (0.05,2);
			\end{tikzpicture}
		}
	}
\end{vd}
\begin{vd}%[0D4B4]
	Biểu diễn hình học tập nghiệm của hệ bất phương trình bậc nhất hai ẩn sau
	$$\left\{\begin{aligned}
		x+y &< 2\\
		x-y &>1 \\
		y &>-1
	\end{aligned}\right.$$
	\loigiai
	{\immini{
			Vẽ các đường thẳng
			\begin{gather*}
				d_1: x+y=2,\\
				d_2: x-y=1,\\
				d_3: y=-1.
			\end{gather*}
			Vì điểm $M\biggl(\dfrac{3}{2},0\biggr)$ có tọa độ thỏa mãn các bất phương trình trong hệ nên ta tô đậm các nửa mặt phẳng bờ $d_1,d_2,d_3$ không chứa $M$. Miền không bị tô đậm trong hình vẽ, không bao gồm các đoạn giới hạn miền là miền nghiệm của hệ đã cho.
		}{
			\begin{tikzpicture}[line cap=round, line join=round, font=\footnotesize, >=stealth, scale=1]
				\tikzset{label style/.style={font=\footnotesize}}
				\draw[color=white,fill=black!10] (-2,-2) rectangle (5,3);
				\path (-1,3) coordinate (A1);
				\path (4,-2) coordinate (B1);
				\path (-1,-2) coordinate (A2);
				\path (4,3) coordinate (B2);
				\path (-2,-1) coordinate (A3);
				\path (5,-1) coordinate (B3);
				\path (intersection of A1--B1 and A2--B2) coordinate (A);
				\path (intersection of A1--B1 and A3--B3) coordinate (B);
				\path (intersection of A2--B2 and A3--B3) coordinate (C);
				\draw[fill=white] (A) -- (B) -- (C) -- cycle;
				\draw[->] (0,-2) -- (0,3) node[right]{\scriptsize $y$};
				\draw[->] (-2,0) -- (5,0) node[above]{\scriptsize $x$};
				\draw (0,0) node[above left]{\scriptsize $O$};
				\draw (A1) -- (B1) node[right]{\scriptsize $d_1$} (A2) -- (B2)node[right]{\scriptsize $d_2$} (A3) -- (B3)node[right]{\scriptsize $d_3$};
				\draw[dashed] (B)node[below]{\scriptsize $B$} -- (B |- 0,0) node[above]{\scriptsize $3$}
				(A-| 0,0)node[left]{\scriptsize $\dfrac{3}{2}$} -- (A)node[above]{\scriptsize $A$} -- (A|- 0,0)node[below]{\scriptsize $\dfrac{3}{2}$};
				\draw (1.5,0)node[above right]{\scriptsize $M$} (C) node[below right]{\scriptsize $C$} (C) node[above left]{\scriptsize $-1$};
			\end{tikzpicture}
		}
	}
\end{vd}
\begin{vd}%[0D4B4]
	Biểu diễn hình học tập nghiệm của hệ bất phương trình bậc nhất hai ẩn sau
	$$\left\{\begin{aligned}
		2x+5y &> 2\\
		x-3y &\geq 1 \\
		x+y &<3
	\end{aligned}\right.$$
	\loigiai
	{\immini{
			Vẽ các đường thẳng
			\begin{gather*}
				d_1: 2x+5y=2,\\
				d_2: x-3y=1,\\
				d_3: x+y=3.
			\end{gather*}
			Vì điểm $M(2,0)$ có tọa độ thỏa mãn các bất phương trình trong hệ nên ta tô đậm các nửa mặt phẳng bờ $d_1,d_2,d_3$ không chứa $M$. 
			
			Miền không bị tô đậm trong hình vẽ có chứa đoạn $AC$ và không chứa các điểm $A,C$, không chứa các đoạn $AB,BC$ là miền nghiệm của hệ đã cho.
		}{
			\begin{tikzpicture}[line cap=round, line join=round, font=\footnotesize, >=stealth, scale=1]
				\tikzset{label style/.style={font=\footnotesize}}
				\draw[color=white,fill=black!10] (-1,-2) rectangle (6,3.2);
				\path (-1,0.8) coordinate (A1);
				\path (6,-2) coordinate (B1);
				\path (-1,-2/3) coordinate (A2);
				\path (6,5/3) coordinate (B2);
				\path (-0.2,3.2) coordinate (A3);
				\path (5,-2) coordinate (B3);
				\path (intersection of A1--B1 and A2--B2) coordinate (A);
				\path (intersection of A1--B1 and A3--B3) coordinate (B);
				\path (intersection of A2--B2 and A3--B3) coordinate (C);
				\draw[fill=white] (A) -- (B) -- (C) -- cycle;
				\draw[->] (0,-2) -- (0,3.2) node[right]{\scriptsize $y$};
				\draw[->] (-1,0) -- (6,0) node[above]{\scriptsize $x$};
				\draw (0.15,0.1) node[below left]{\scriptsize $O$};
				\draw (A1) -- (B1) node[right]{\scriptsize $d_1$} (A2) -- (B2)node[right]{\scriptsize $d_2$} (A3) -- (B3)node[right]{\scriptsize $d_3$};
				\draw[dashed] (B -| 0,0) node[left]{\scriptsize $-\dfrac{4}{3}$}--(B)node[below]{\scriptsize $B$} -- (B |- 0,0) node[above]{\scriptsize $\dfrac{13}{3}$}
				(C-| 0,0) node[left]{\scriptsize $\dfrac{1}{2}$} -- (C)node[above]{\scriptsize $C$} -- (C|- 0,0)node[below]{\scriptsize $\dfrac{5}{2}$};
				\draw (2,0)node[above]{\scriptsize $2$} (2,0.05)  -- (2,-0.05) node[below]{\scriptsize $M$} (A) node[below]{\scriptsize $A$} (A) node[above]{\scriptsize $1$};	
			\end{tikzpicture}
		}
	}
\end{vd}
\begin{vd}%[0D4B4]
	Biểu diễn hình học tập nghiệm của hệ bất phương trình bậc nhất hai ẩn sau
	$$\left\{\begin{aligned}
		2x+y &\geq 2\\
		x-2y &\leq 1 \\
		y &\leq 2\\
		x &\leq 3
	\end{aligned}\right.$$
	\loigiai
	{\immini{
			Vẽ các đường thẳng
			\begin{gather*}
				d_1: 2x+y=2,\\
				d_2: x-2y=1,\\
				d_3: y=2,
				d_4: x=3.
			\end{gather*}
			Vì điểm $M(2,1)$ có tọa độ thỏa mãn các bất phương trình trong hệ nên ta tô đậm các nửa mặt phẳng bờ $d_1,d_2,d_3,d_4$ không chứa $M$. Miền không bị tô đậm trong hình vẽ là miền nghiệm của hệ đã cho bao gồm các đoạn thẳng xác định miền.
		}{\begin{tikzpicture}[>=stealth]
				\draw[color=white,fill=black!10] (-2,-2) rectangle (5,3);
				\path (-1/2,3) coordinate (A1);
				\path (2,-2) coordinate (B1);
				\path (-2,-3/2) coordinate (A2);
				\path (5,2) coordinate (B2);
				\path (-2,2) coordinate (A3);
				\path (5,2) coordinate (B3);
				\path (3,-2) coordinate (A4);
				\path (3,3) coordinate (B4);
				\path (intersection of A1--B1 and A2--B2) coordinate (A);
				\path (intersection of A1--B1 and A3--B3) coordinate (B);
				\path (intersection of A3--B3 and A4--B4) coordinate (C);
				\path (intersection of A2--B2 and A4--B4) coordinate (D);
				\draw[fill=white] (A) -- (B) -- (C) -- (D) -- cycle;
				\draw[->] (0,-2) -- (0,3) node[right]{\scriptsize $y$};
				\draw[->] (-2,0) -- (5,0) node[above]{\scriptsize $x$};
				\draw (0.15,0.1) node[below left]{\scriptsize $O$};
				\draw (A1) -- (B1) node[right]{\scriptsize $d_1$} (A2)node[left]{\scriptsize $d_2$} -- (B2) (A3)node[left]{\scriptsize $d_3$} -- (B3) (A4) -- (B4) node[right]{\scriptsize $d_4$};
				\draw[dashed] (A) node[above]{\scriptsize $1$} (A) node[below]{\scriptsize $A$} (B) node[below left]{\scriptsize $2$} (B) node[above right]{\scriptsize $B$} (C) node[above right]{\scriptsize $C$} (C|- 0,0) node[below right]{\scriptsize $3$} (D) node[below right]{\scriptsize $D$} -- (D-| 0,0) node[left]{\scriptsize $1$} (2,1) node[above]{\scriptsize $M$} -- (2,0)node[below]{\scriptsize $2$};
			\end{tikzpicture}
		}
	}
\end{vd}

\baitaptl
\begin{bt}%[0D4B4]
	Biểu diễn hình học tập nghiệm của hệ bất phương trình bậc nhất hai ẩn sau
	$$\left\{\begin{aligned}
		x+2y &\geq 1\\
		3x-y & \leq 2 \\
	\end{aligned}\right.$$
	\loigiai{
		\immini{
			Vẽ hai đường thẳng
			\begin{gather*}
				d_1:x+2y=1,\\
				d_2:3x-y=2
			\end{gather*}
			trên cùng một hệ trục tọa độ $Oxy$. Dễ dàng kiểm tra được điểm $O$ thuộc miền nghiệm của cả hai bất phương trình nên ta có miền nghiệm của hệ bất phương trình là miền không bị tô đậm bao gồm cả bờ.
		}{
			\begin{tikzpicture}[line cap=round, line join=round, font=\footnotesize, >=stealth, scale=1]
				\tikzset{label style/.style={font=\footnotesize}}
				\fill[black!10] (-1,-2) rectangle (3,1);
				\fill[white] (-1,1)--(5/7,1/7)--(0,-2)--(-1,-2)--cycle;
				\draw[->] (-1.2,0)--(3,0) node[below]{$x$};
				\draw[->] (0,-2)--(0,1) node[left]{$y$};
				\draw (-1,1)--(3,-1) node[below]{$d_1$}
				(0,-2)--(1,1) node[right]{$d_2$};
				\draw[fill=black](0,0) circle (1pt) node[below left]{$O$}
				(0,2/3) circle (1pt) node[right]{$\frac{2}{3}$};
				\foreach\x in {-1,1,2} \draw[fill=black] (\x,0) circle (1pt) node[below]{$\x$};
				\foreach\y in {-2,-1} \draw[fill=black] (0,\y) circle (1pt) node[left]{$\y$};
			\end{tikzpicture}	
			
		}
	}
\end{bt}

\begin{bt}%[0D4B4]
	Biểu diễn hình học tập nghiệm của hệ bất phương trình bậc nhất hai ẩn sau
	$$\left\{\begin{aligned}
		x-2y &< 1\\
		x+3y &<-2 \\
		-x+y &<2
	\end{aligned}\right.$$
	\loigiai{
		\immini{
			Vẽ các đường thẳng
			\begin{gather*}
				d_1:x-2y=1\\ d_2:x+2y=-2\\ d_3:-x+y=2
			\end{gather*}
			trên cùng mặt phẳng tọa độ $Oxy$. Ta kiểm tra được điểm $M(-2;-1)$ thuộc miền nghiệm của hệ bất phương trình nên ta có miền nghiệm của hệ bất phương trình là miền trong tam giác $ABC$ không kể các cạch.
		}{
			\begin{tikzpicture}[line cap=round, line join=round, font=\scriptsize, >=stealth, scale=1]
				\tikzset{label style/.style={font=\scriptsize}}
				\fill[black!10] (-6,-4) rectangle (2,1);
				\path (-2,0) coordinate (A)
				(-5,-3) coordinate (B)
				(-1/5,-3/5) coordinate (C)
				(-2,-1) coordinate (M);
				\fill[white] (A)--(B)--(C)--cycle;
				\draw[->] (-6,0)--(2,0) node[below]{$x$};
				\draw[->] (0,-4)--(0,1) node[left]{$y$};
				\draw[fill=black] (0,0) circle (1pt) node[above right]{$O$}
				(-5,0) circle (1pt) node[above]{$-5$}
				(-2,0) circle (1pt) node[above]{$-2$}
				(0,-3) circle (1pt) node[right]{$-3$}
				(0,-1) circle (1pt) node[right]{$-1$};
				\draw[smooth] plot[domain=-6:2] (\x,{ \x/2-1/2 }) node[above]{$d_1$}
				plot[domain=-5:2] (\x,{ -\x/3-2/3 }) node[below]{$d_2$}
				plot[domain=-6:-1] (\x,{ \x+2}) node[left]{$d_3$};
				\foreach\x/\y in {A/-45,B/-60,C/135,M/-145} \draw[fill=black] (\x) circle (1pt)+(\y:0.3) node{$\x$};
				\draw[dashed,fill=black] (-1/5,0) circle (1pt) node[above]{$\frac{-1}{5}$}-- (C) (C)--(0,-3/5) circle (1pt) node[right]{$\frac{-3}{5}$};
				\draw[dashed] (-5,0)--(B)--(0,-3)
				(A)--(M)--(0,-1);
			\end{tikzpicture}	
			
		} 
	}
\end{bt}

\begin{bt}%[0D4B4]
	Biểu diễn hình học tập nghiệm của hệ bất phương trình bậc nhất hai ẩn sau
	$$\left\{\begin{aligned}
		3x+y & \leq 5\\
		x+y & \leq 4 \\
		x &\geq 0\\
		y &\geq 0
	\end{aligned}\right.$$
	\loigiai{
		\immini{
			Vẽ các đường thẳng $d_1:3x+y=5$ và $d_2:x+y=4$ lên cùng hệ trục tọa độ. Ta thấy điểm $M(1;1)$ thỏa mãn tất cả các bất phương trình của hệ, do đó tập nghiệm của hệ bất phương trình đã cho là miền trong tứ giác $ABCO$ kể cả các cạnh.
		}{
			\begin{tikzpicture}[line cap=round, line join=round, font=\scriptsize, >=stealth, scale=1,y=0.7cm]
				\tikzset{label style/.style={font=\footnotesize}}
				\fill[black!10] (-1,-1) rectangle (5,6);
				\path (0,4) coordinate (A)
				(0.5,3.5) coordinate (B)
				(5/3,0) coordinate (C)
				(0,0) coordinate (O)
				(1,1) coordinate (M);
				\fill[white] (A)--(B)--(C)--(O)--cycle;
				\draw[->] (-1,0)--(5,0) node[below]{$x$};
				\draw[->] (0,-1)--(0,6) node[right]{$y$};
				\draw[fill=black]
				(0.5,0) circle (1pt) node[below]{$\frac{1}{2}$}
				(1,0) circle (1pt) node[below]{$1$}
				(5/3,0) circle (1pt) node[above right]{$\frac{5}{3}$}
				(0,3.5) circle (1pt) node[left]{$\frac{7}{2}$}
				(0,1) circle (1pt) node[left]{$1$}
				(0,4) circle (1pt) node[left]{$4$};
				\draw[smooth] plot[domain=-1:5] (\x,{ 4-\x }) node[above]{$d_2$}
				plot[domain=-0.33:2] (\x,{ 5-3*\x }) node[right]{$d_1$};
				\foreach\x/\y in {A/30,B/-135,C/-135,M/90,O/-135} \draw[fill=black] (\x) circle (1pt)+(\y:0.3) node{$\x$};
				\draw[dashed] (0,3.5)--(B)--(0.5,0)
				(1,0)--(M)--(0,1);
			\end{tikzpicture}	
			
		}
	}
\end{bt}

\begin{dang}{Tìm cực trị của biểu thức $F=ax+by$ trên một miền đa giác}
	\begin{enumerate}
		\item Bài toán: \\
		Tìm giá trị lớn nhất, giá trị nhỏ nhất của biểu thức $F=ax+by$ ($a$, $b$ là hai số đã cho không đồng thời bằng $0$) với $x$, $ y$ thỏa mãn hệ bất phương trình bậc nhất hai ẩn (có miền nghiệm là miền đa giác $A_1 A_2 \ldots A_i A_{i+1} \ldots A_n$).
		\item Người ta chứng minh được: Biểu thức $F=ax+by$  có giá trị nhỏ nhất, giá trị lớn nhất tại một trong các đỉnh của đa giác $A_1 A_2 \ldots A_i A_{i+1} \ldots A_n$.
		\item Phương pháp: 
		\begin{itemize}
			\item Bước 1. Tìm miền đa giác $A_1 A_2 \ldots A_i A_{i+1} \ldots A_n$ là miền nghiệm của hệ bất phương trình.
			\item Bước 2. Tìm tọa độ các đỉnh $A_1$, $A_2$, $\ldots$, $A_n$.
			\item Bước 3. Tính $F\left(x_i ; y_i\right)$ trong đó $A_i\left(x_i;y_i\right)$ với $i=1,2,\ldots,\ n$.
			\item Bước 4. Kết luận\\
			Giá trị lớn nhất $M=\max \limits_{i=1,2,\ldots n} F\left(x_i; y_i\right)$.\\
			Giá trị nhỏ nhất $m=\min\limits_{i=1,2,\ldots n} F\left(x_i; y_i\right)$.
		\end{itemize}
	\end{enumerate}
\end{dang}
\viduminhhoa
\begin{vd}%[BG10-2022]%[Toanvo]%[0D4G4-4]
	Cho cặp số $\left(x;y\right)$  là nghiệm của hệ  $\heva{&3x-y \geq -1\\&2x+y \leq 6\\&x+3y>3}$. Tìm giá trị lớn nhất và nhỏ nhất của biểu thức $f\left(x;y\right)=2x-3y+1$.
	\loigiai{
		\begin{itemize}
			\item Trước hết ta biểu diễn miền nghiệm của hệ (*):
			\begin{itemize}
				\item  Vẽ các đường thẳng $d_1 \colon 3x-y=-1$; $d_2 \colon 2x+y=6$; $d_3 \colon x+3y=3$.
				\item  Điểm $M(1;1)$ có tọa độ thỏa mãn tất cả các bất phương trình trong hệ nên ta tô đậm các nửa mặt phẳng bờ $d_1;d_2;d_3$ không chứa điểm $M$. Miền không bị tô đậm là hình tam giác $ABC$, tính cả ba cạnh $AB,BC,CA$ trong hình vẽ dưới là miền nghiệm của hệ bất phương trình đã cho.
			\end{itemize}
			\begin{center}
				\begin{tikzpicture}
					\draw[->,line width=0.8pt] (-1,0)--(6.5,0) node[right]{$x$};
					\draw[->,line width=0.8pt] (0,-1)--(0,5) node[above left]{$y$};
					
					\clip (-1,-1) rectangle (6.5,5);
					
					
					\begin{scope}
						\clip (-2,-5)--(-2,7)--(2,7)--(-2,-5);
						\foreach \x in {-1,-.85,...,20}{
							\pgfmathsetmacro{\y}{-2*\x+8}
							\draw[gray] (\x,\y)--++(150:10);
						}			
					\end{scope}
					
					\begin{scope}
						\clip (-2,10)--(6,10)--(6,-6)--(-2,10);
						\foreach \x in {-1,-.9,...,20}{
							\pgfmathsetmacro{\y}{-2*\x+1}
							\draw[gray] (\x,\y)--++(1500:10);
						}			
					\end{scope}
					
					\pgfmathsetmacro{\hsgg}{-1/3}
					\begin{scope}
						\clip (-3,2)--(-3,-3)--(12,-3)--(-3,2);
						\foreach \x in {-1,-.95,...,20}{
							\pgfmathsetmacro{\y}{-4*\x-3}
							\draw[gray] (\x,\y)--++(10:10);
						}			
					\end{scope}
					
					
					
					\draw[thick,smooth,domain=-1:6.5,blue] plot (\x,{3*\x+1});
					\draw[thick,smooth,domain=-1:6.5,red] plot (\x,{-2*\x+6});
					\draw[thick,smooth,domain=-1:6.5,green] plot (\x,{\hsgg*\x+1});
					
					\path (2.5,4.5)--(3,4.5) node[fill=white,pos=.5,sloped]{$d_{1} \colon 3x-y=-1$};
					\path (1,4)--(3,0) node[fill=white,pos=.5,sloped,above]{$d_{2} \colon 2x+y=6$};
					\path (1,-1)--(2,-1) node[fill=white,pos=.5,sloped,above]{$d_{3} \colon x+3y=3$};
					
					\foreach \i in {-1,1,2,3,4}{
						\fill (0,\i) circle (1.3pt);
					}
					\foreach \i in {-1,0,1,2,3,4,5,6}{
						\fill (\i,0) circle (1.3pt);
					}
					\fill (1,1) circle (1.3pt) (1,4) circle (1.3pt);
					
					\path 
					(0,1) node[below left,fill=white]{$A$}
					(1,1) node[above]{$M$}
					(3,0) node[above right,fill=white]{$C$}
					(1.1,4) node[right,fill=white]{$B$}
					(0,0) node[below left,fill=white]{$O$};
				\end{tikzpicture}
			\end{center}
			\item Tìm tọa độ các điểm $A,B,C$:
			\begin{itemize}
				\item $A=d_1 \cap d_3$ nên tọa độ thỏa mãn $\heva{	&3x-y=-1\\
					&x+3y=3} \Leftrightarrow \heva{&x=0\\
					&y=1}$. Vậy $A(0;1)$.
				\item  $B=d_1 \cap d_2$ nên tọa độ thỏa mãn $\heva{&3x-y=-1\\
					&2x+y=6} \Leftrightarrow \heva{&x=1\\
					&y=4} $. Vậy $B(1;4)$.
				\item $C=d_2 \cap d_3$ nên tọa độ thỏa mãn $\heva{&2x+y=6\\
					&x+3y=3}\Leftrightarrow \heva{&x=3\\
					&y=0} $. Vậy $C(3;0)$.
			\end{itemize}
		\end{itemize}
		Tính giá trị của $f(x;y)=2x-3y+1$ tại tất cả các đỉnh của tam giác $ABC$: 
		\begin{center}
			\begin{tabular}{|p{5cm}||p{2cm}|p{2cm}|p{2cm}|}
				\hline
				$(x;y)$ & $A(0;1)$ & $B(1;4)$ & $C(3;0)$ \\
				\hline
				$f(x;y)=2x-3y+1$  & $-2$  & $-9$ & $7$ \\
				\hline
			\end{tabular}
		\end{center}
		Suy ra $\min f(x;y)=f(1;4)=-9$ và $\max f(x;y)=f(3;0)=7$.\\
		
	}
\end{vd}
\begin{vd}%[BG10-2022]%[Toanvo]%[0D4G4-4]
	Quảng cáo sản phẩm trên truyền hình là một hoạt động quan trọng trong kinh doanh của các doanh nghiệp.\\
	Theo Thông báo số $10/2019$, giá quảng cáo trên VTV1 là $30$ triệu đồng cho $15$ giây/$1$ lần quảng cáo vào khoảng $20$h$30$; là $6$ triệu đồng cho $15$ giây/$1$ lần quảng cáo vào khung giờ $16$h$00-17$h$00$.\\
	Một công ty dự định chi không quá $900$ triệu đồng để quảng cáo trên VTV1 với yêu cầu quảng cáo về số lần phát như sau: ít nhất $10$ lần quảng cáo vào khoảng
	$20$h$30$ và không quá $50$ lần quảng cáo vào khung giờ $16$h$00-17$h$00$. 
	\loigiai{
		Gọi $x, \,y$ lần lượt là số lần phát quảng cáo vào khoảng $20$h$30$  và vào khung giờ $16$h$00-17$h$00$. Theo giả thiết, ta có: $x \in \mathbb{N},\, y \in \mathbb{N},\, x \geq 10,0 \leq y \leq 50$.\\
		Tổng số lần phát quảng cáo là $T=x+y$.\\
		Số tiền công ty cần chi là $30 x+6 y$ (triệu đồng).\\
		Do công ty dự định chi không quá $900$ triệu đồng nên $30 x+6 y \leq 900$ hay $5 x+y \leq 150$.\\
		Ta có hệ bất phương trình: $\heva{5 x+y \leq 150 \\ x \geq 10 \\ 0 \leq y \leq 50.} $ \hfill(I)	
		Bài toán đưa về tìm $x, y$ là nghiệm của hệ bất phương trình (I) sao cho $T=x+y$ có giá trị lớn nhất.\\
		Trước hết, ta xác định miền nghiệm của hệ bất phương trình (I).\\
		Miền nghiệm của hệ bất phương trình (I) là miền tứ giác $A B C D$ với $A(30 ; 0), B(20 ; 50)$, $C(10 ; 50), D(10 ; 0)$ (Hình vẽ).
		\immini{
			Người ta chứng minh được: Biểu thức $T=x+y$ đạt được giá trị lớn nhất tại một trong các đỉnh của tứ giác $A B C D$.
			Tính giá trị của biểu thức $T=x+y$ tại cặp số $(x ; y)$ là toạ độ các đỉnh của tứ giác $A B C D$ rồi so sánh các giá trị đó. Ta được $T$ đạt giá trị lớn nhất khi $x=20, y=50$ ứng với tọa độ đỉnh $B$.\\
			Vậy để phát được số lần quảng cáo nhiều nhất thì số lần phát quảng cáo vào khoảng $20$h$30$  và vào khung giờ $16$h$00-17$h$00$  lần lượt là $20$ và $50$ lần.
		}{\begin{tikzpicture}[scale=0.07,line join=round, line cap=round,>=stealth,thick]
				\tikzset{label style/.style={font=\footnotesize}}
				\begin{scope}
					
					\clip (-20,-20) rectangle (60,60);
					\fill[pattern=crosshatch dots] (-21,255)--(61,255)--(61,-155)--cycle;
					\fill[pattern=crosshatch dots] (10,-20)--(-20,-20)--(-20,60)--(10,60)--cycle;
					\fill[pattern=crosshatch dots] (-20,0)--(-20,-20)--(60,-20)--(60,0)--cycle;
					\fill[pattern=crosshatch dots] (-20,50)--(-20,60)--(60,60)--(60,50)--cycle;
					\draw (18,60)--(34,-20) ;
					\draw (10,-20)--(10,60) ;
					\draw (-20,50)--(60,50) ;
				\end{scope}
				\draw[->] (-20,0)--(60,0) node[below]{$x$};
				\draw[->] (0,-20)--(0,60) node[left]{$y$};
				\draw (0,0) node[below left]{$O$};
				\path (10,50) coordinate(C) (10,0) coordinate(D) (30,0) coordinate(A) (20,50) coordinate(B);
				\foreach \i in {A,B,C,D} \fill (\i) circle(20pt) ($(\i)+(55:5)$)node{$\i$};
				\foreach \x in {10,20,30}
				\draw[thin] (\x,1pt)--(\x,-1pt) node [below right] {$\x$};
				\foreach \y in {50}
				\draw[thin] (1pt,\y)--(-1pt,\y) node [below left] {$\y$};
		\end{tikzpicture}}
	}
\end{vd}
\begin{vd}%[BG10-2022]%[Toanvo]%[0D4G4-4]
	Một hộ nông dân dự định trồng đậu và cà trên diện tích $8$ ha. Nếu trồng đậu thì cần $20$ công và thu $3$ triệu đồng trên diện tích mỗi ha, nếu trồng cà thì cần $30$ công và thu $4$ triệu đồng trên diện tích mỗi ha. Hỏi cần trồng mỗi loại cây trên với diện tích là bao nhiêu để thu về được nhiều tiền nhất, biết rằng tổng số công không quá $180$.
	\loigiai{
		Gọi diện tích để trồng đậu là $x$ (ha); diện tích để trồng cà là  $y$ (ha). ( điều kiện: $0 \leq x,y \leq 8$ ).\\
		Tổng số diện tích sử dụng là $x+y$.\\
		Tổng số công cần sử dụng là $20x+30y$. \\
		Ta có hệ bất phương trình
		$$\heva{&0\leq x\leq8\\&0\leq y\leq8\\&x+y\leq8\\&20x+30y\leq180}\Leftrightarrow\heva{&0\leq x\leq8\\&0\leq y\leq8\\&x+y\leq8\\&2x+3y\leq18.}$$
		Vẽ các đường thẳng thẳng $(d_1) \colon x+y=8,\ (d_2) \colon 2x+3y=18,\ (d_3) \colon x=8,\ (d_4) \colon y=8$ ta được miền
		nghiệm của hệ bất phương trình là phần tô đậm như hình vẽ
		\begin{center}
			\begin{tikzpicture}[line join = round, line cap = round, >=stealth, font=\footnotesize, scale=.5]
				\tikzset{label style/.style={font=\footnotesize}}
				\def \xmin{-6}
				\def \xmax{15}
				\def \ymin{-5}
				\def \ymax{14}
				\tkzDefPoints{0/6/A, 6/2/B, 8/0/C, 0/0/D, 0/8/E, 9/0/F, 8/8/G}
				\draw[gray!20](\xmin,\ymin) grid (\xmax,\ymax);
				\draw [->](\xmin,0)--(0,0)node[below left]{$O$}--(\xmax,0)node[above]{$x$};
				\draw [->](0,\ymin)--(0,\ymax)node[right]{$y$};
				\foreach \x in {-5,5,10} \draw[shift={(\x,0)}] (0pt,2pt)--(0pt,-2pt) node[below]{$\x$};
				\foreach \y in {5,10} \draw [shift={(0,\y)}] (2pt,0pt)--(-2pt,0pt) node[left]{$\y$};
				\tkzDrawLines[add=0.5 and 0.5](E,C A,F C,G E,G)
				\tkzDrawPoints[fill=black](A,B,C,D)
				\tkzLabelPoints[right](A)
				\tkzLabelPoints[above](B)
				\tkzLabelPoints[below left](C)
				\tkzLabelPoints[below right](D)
				\tkzFillPolygon[color=cyan,fill opacity=.5](A,B,C,D)
				\tkzLabelLine[pos=1.7,above right](B,E){$(d_1)$}
				\tkzLabelLine[pos=1.8,above right](B,A){$(d_2)$}
				\tkzLabelLine[pos=1.5,right](C,G){$(d_3)$}
				\tkzLabelLine[pos=1.5,above](E,G){$(d_4)$}
			\end{tikzpicture}
		\end{center}
		Ta có 
		{\allowdisplaybreaks
			\begin{eqnarray*}
				&&A(0;6)=(d_2) \cap Oy,B(6;2)=(d_1) \cap (d_2)\\
				&&C(8;0)=(d_1) \cap Ox,D \equiv O(0;0).
			\end{eqnarray*}
		}
		Số tiền thu về là $f(x;y)=3x+4y$ (triệu đồng).
		\begin{center}
			\renewcommand\arraystretch{1.6}
			\renewcommand{\tabcolsep}{6mm}
			\begin{tabular}{|c|c|c|c|c|}
				\hline 
				$M(x;y)$& $A$ & $B$ & $C$ & $D$ \\ 
				\hline 
				$f(x,y)=3x+4y$& $24$ & $26$ & $24$ & $0$ \\ 
				\hline 
			\end{tabular} 
		\end{center}
		Do đó $f(x,y)$ đạt giá trị lớn nhất tại $B(6;2)$.\\
		Vậy để thu được nhiều tiền nhất thì cần trồng 6 ha đậu và 2 ha cà.
	}	
\end{vd}
\baitaptl
\begin{bt}%[BG10-2022]%[Toanvo]%[0D4G4-4]
	Một gia đình cần ít nhất $900$ đơn vị protein và $400$ đơn vị lipit trong thức ăn mỗi ngày. Mỗi kg thịt bò chứa $800$ đơn vị protein và $200$ đơn vị lipit. Mỗi kg thịt lợn chứa $600$ đơn vị protein và $400$ đơn vị lipit. Biết rằng mỗi ngày gia đình này chỉ mua tối đa $1{,}5$ kg thịt bò và $1$ kg thịt lợn, giá tiền $1$ kg thịt bò là $200$ nghìn đồng, $1$ kg thịt lợn là $100$ nghìn đồng. Hỏi gia đình đó phải mua bao nhiêu kg thịt mỗi loại để số tiền bỏ ra là ít nhất.
	\loigiai{
		Gọi số kg thịt bò cần mua là $x$ (kg); số kg thịt lợn cần mua là $y$ (kg). Điều kiện: $0 \leq x \leq 1{,}5, 0 \leq y \leq 1$.\\
		Khi đó số đơn vị protein là $800x+600y$.\\
		Số đơn vị lipit là $200x+400y$.\\
		Ta có hệ bất phương trình $$\heva{&0\leq x\leq 1{,}5\\&0\leq y\leq 1\\&800x+600y\geq 900\\&200x+400y\geq200}\Leftrightarrow\heva{&0\leq x\leq 1{,}5\\&0\leq y\leq 1\\&8x+6y\geq9\\&x+2y\geq2.}$$	
		Vẽ các đường thẳng $(d_1)\colon x=1{,}5$, $(d_2)\colon y=1$, $(d_3)\colon 8x+6y=9$, $(d_4)\colon x+2y=2$. Ta được miền nghiệm của hệ bất phương trình là phần tô đậm trong hình vẽ.
		\begin{center}
			\begin{tikzpicture}[line join = round, line cap = round, >=stealth, font=\footnotesize, scale=1.2]
				\tikzset{label style/.style={font=\footnotesize}}
				\def \xmin{-3.5}
				\def \xmax{5.5}
				\def \ymin{-1.5}
				\def \ymax{6}
				\tkzDefPoints{0.375/1/A, 1.5/1/B, 1.5/0.25/C, 0.6/0.7/D}
				\draw[gray!20](\xmin,\ymin) grid (\xmax,\ymax);
				\draw [->](\xmin,0)--(0,0)node[below left]{$O$}--(\xmax,0)node[above]{$x$};
				\draw [->](0,\ymin)--(0,\ymax)node[right]{$y$};
				\foreach \x in {-3,-2,-1,1,2,3,4,5} \draw[shift={(\x,0)}] (0pt,2pt)--(0pt,-2pt) node[below]{$\x$};
				\foreach \y in {-1,2,3,4,5} \draw [shift={(0,\y)}] (2pt,0pt)--(-2pt,0pt) node[left]{$\y$};
				\draw [shift={(0,1)}] (2pt,0pt)--(-2pt,0pt) node[below left]{$1$};
				\tkzDrawLine[add=6 and 2](B,C)
				%\tkzDrawLine[add=0.5 and 0.5](B,C A,B A,D D,C)
				\tkzDrawLine[add=3 and 3](A,B)
				\tkzDrawLine[add=12 and 6](A,D)
				\tkzDrawLines[add=4 and 2](D,C)
				\tkzDrawPoints[fill=black](A,B,C,D)
				\tkzLabelPoints[above right](A,B,C)
				\tkzLabelPoints[below](D)
				\tkzFillPolygon[color=cyan,fill opacity=.5](A,B,C,D)
				\tkzLabelLine[pos=7,right](C,B){$(d_1)$}
				\tkzLabelLine[pos=3.5,above right](A,B){$(d_2)$}
				\tkzLabelLine[pos=12,above right](D,A){$(d_3)$}
				\tkzLabelLine[pos=5,above right](C,D){$(d_4)$}
			\end{tikzpicture}
		\end{center}
		Ta có 
		{\allowdisplaybreaks
			\begin{eqnarray*}
				&&A\left(\dfrac{3}{8};1\right)=(d_3) \cap (d_2), B(1{,}5;1)=(d_1) \cap (d_2),\\
				&&C(1{,}5;0{,}25)=(d_1) \cap (d_4), D\left(\dfrac{3}{5};\dfrac{7}{10}\right)=(d_3) \cap (d_4).
			\end{eqnarray*}
		}
		Số tiền bỏ ra là $f(x,y)=200x+100y$ (nghìn đồng).
		\begin{center}
			\renewcommand\arraystretch{1.6}
			\renewcommand{\tabcolsep}{6mm}
			\begin{tabular}{|c|c|c|c|c|}
				\hline 
				$M(x;y)$& $A$ & $B$ & $C$ & $D$ \\ 
				\hline 
				$f(x,y)=200x+100y$& $175$ & $400$ & $325$ & $190$ \\ 
				\hline 
			\end{tabular} 
		\end{center}
		Do đó $f(x,y)$ đạt giá trị nhỏ nhất tại $A\left(\dfrac{3}{8};1\right)$.\\
		Vậy để số tiền bỏ ra nhỏ nhất thì cần mua $\dfrac{3}{8}$ kg thịt bò và $1$ kg thịt lợn.
	}
\end{bt}
\begin{bt}%[BG10-2022]%[Toanvo]%[0D4G4-4]
	Người ta định dùng hai loại nguyên liệu để chiết xuất ít nhất $120$ kg hóa chất $A$ và $9$ kg hóa chất $B$. Từ mỗi tấn nguyên liệu loại I giá $4$ triệu đồng có thể chiết xuất được $20$ kg chất $A$ và $0{,}6$ kg chất $B$. Từ mỗi tấn nguyên liệu loại II giá $3$ triệu đồng có thể chiết xuất được $10$ kg chất $A$ và $1{,}5$ kg chất $B$. Hỏi phải dùng bao nhiêu tấn nguyên liệu mỗi loại để chi phí mua nguyên liệu là ít nhất. Biết rằng cơ sở cung cấp nguyên liệu chỉ có thể cung cấp không quá $10$ tấn nguyên liệu loại I và không quá $9$ tấn nguyên liệu loại II. 
	\loigiai{
		Gọi số tấn nguyên liệu loại I cần sử dụng là $x$ (tấn); số tấn nguyên liệu loại II cần sử dụng là  $y$ (tấn).\\
		Điều kiện: $0 \leq x \leq 10$, $0 \leq y \leq 9$.\\
		Khi đó số kg chất $A$ thu được là $20x+10y$, số kg chất $B$ thu được là $0{,}6x+1{,}5y$.\\
		Ta có hệ bất phương trình $$\heva{&0\leq x\leq 10\\&0\leq y\leq 9\\&20x+10y\geq120\\&0{,}6x+1{,}5y\geq9}\Leftrightarrow \heva{&0\leq x \leq10\\&0\leq y \leq 9\\&2x+y\geq 12\\&2x+5y\geq30.}$$
		Vẽ các đường thẳng $(d_1) \colon x=10$, $(d_2) \colon y=9$, $(d_3) \colon 2x+y=12$, $(d_4) \colon 2x+5y=30$. \\
		Ta có miền nghiệm của hệ bất phương trình là phần tô màu như hình vẽ:
		\begin{center}
			\begin{tikzpicture}[line join = round, line cap = round, >=stealth, font=\footnotesize, scale=.5]
				\tikzset{label style/.style={font=\footnotesize}}
				\def \xmin{-8}
				\def \xmax{17}
				\def \ymin{-5}
				\def \ymax{15}
				\tkzDefPoints{1.5/9/A, 10/9/B, 10/2/C, 3.75/4.5/D}
				\draw[gray!20](\xmin,\ymin) grid (\xmax,\ymax);
				\draw [->](\xmin,0)--(0,0)node[below left]{$O$}--(\xmax,0)node[above]{$x$};
				\draw [->](0,\ymin)--(0,\ymax)node[right]{$y$};
				\foreach \x in {-5,5,15} \draw[shift={(\x,0)}] (0pt,2pt)--(0pt,-2pt) node[below]{$\x$};
				\draw[shift={(10,0)}] (0pt,2pt)--(0pt,-2pt) node[below left]{$10$};
				\foreach \y in {5,10} \draw [shift={(0,\y)}] (2pt,0pt)--(-2pt,0pt) node[left]{$\y$};
				\tkzDrawLine[add=.8 and .8](C,B)
				\tkzDrawLine[add=.8 and .5](A,B)
				\tkzDrawLine[add=2 and 1.2](D,A)
				\tkzDrawLines[add=1 and 1.5](C,D)
				\tkzDrawPoints[fill=black](A,B,C,D)
				\tkzLabelPoints[above right](A,B,C)
				\tkzLabelPoints[below](D)
				\tkzFillPolygon[color=cyan,fill opacity=.5](A,B,C,D)
				\tkzLabelLine[pos=1.7,right](C,B){$(d_1)$}
				\tkzLabelLine[pos=1.3,above right](A,B){$(d_2)$}
				\tkzLabelLine[pos=2.2,below left](D,A){$(d_3)$}
				\tkzLabelLine[pos=2.4,below left](C,D){$(d_4)$}
			\end{tikzpicture}
		\end{center}
		Ta có 
		{\allowdisplaybreaks
			\begin{eqnarray*}
				&&(d_2) \cap (d_3)=A\left(\dfrac{3}{2};9\right),\ (d_2) \cap (d_1)=B(10;9),\\
				&&(d_1) \cap (d_4)=C(10;2),\ (d_4) \cap (d_3)=D\left(\dfrac{15}{4};\dfrac{9}{2}\right).
			\end{eqnarray*}
		}
		Chi phí mua nguyên liệu cần bỏ ra là $f(x,y)=4x+3y$ (triệu đồng).
		\begin{center}
			\renewcommand\arraystretch{1.6}
			\renewcommand{\tabcolsep}{6mm}
			\begin{tabular}{|c|c|c|c|c|}
				\hline 
				$M(x;y)$& $A$ & $B$ & $C$ & $D$ \\ 
				\hline 
				$f(x,y)=4x+3y$& $3$ & $67$ & $46$ & $28{,}5$ \\
				\hline 
			\end{tabular} 
		\end{center}
		Do đó $f(x,y)$ đạt giá trị nhỏ nhất tại $D\left(\dfrac{15}{4};\dfrac{9}{2}\right)$.\\
		Vậy để chi phí nguyên liệu là ít nhất ta cần sử dụng $\dfrac{15}{4}=3{,}75$ tấn nguyên liệu loại I và $\dfrac{9}{2}=4{,}5$ tấn nguyên liệu loại II.
	}
\end{bt}
\begin{bt}%[BG10-2022]%[Toanvo]%[0D4G4-4]
	Có ba nhóm máy $A$, $B$, $C$ dùng để sản xuất ra hai loại sản phẩm I và II. Để sản xuất một đơn vị sản phẩm mỗi loại phải lần lượt dùng các máy thuộc các nhóm khác nhau. Số máy trong một nhóm và số máy của từng nhóm cần thiết để sản xuất ra một đơn vị sản phẩm thuộc mỗi loại  được cho trong bảng sau:
	\begin{center}
		\begin{tabular}{|c|c|p{2.5cm}|p{2.5cm}|}
			\hline 
			\multirow{2}{*}{Nhóm}& \multirow{2}{*}{Số máy trong mỗi nhóm}  & \multicolumn{2}{p{5cm}|}{Số máy trong từng nhóm để sản xuất ra một đơn vị sản phẩm} \\ 
			\cline{3-4} 
			&  & Loại I & Loại II  \\ 
			\hline 
			A & 10 & 2 & 2 \\ 
			\hline 
			B & 4 & 0 & 2 \\ 
			\hline 
			C & 12 & 2 & 4 \\ 
			\hline 
		\end{tabular} 
	\end{center}
	Một đơn vị sản phẩm I lãi ba nghìn đồng, một đơn vị sản phẩm loại II lãi năm nghìn đồng. Hãy lập phương án để việc sản xuất hai loại sản phẩm trên có lãi cao nhất. 
	\loigiai{
		Gọi số sản phẩm loại I cần sản xuất là $x$; số sản phẩm loại II cần sản xuất là $y$. Điều kiện: $x,y \geq 0$.\\
		Số máy nhóm $A$ cần sử dụng là $2x+2y$.\\
		Số máy nhóm $B$ cần sử dụng là $2y$.\\
		Số máy nhóm $C$ cần sử dụng là $2x+4y$.\\
		Ta có hệ bất phương trình $$\heva{&x\geq0\\&y\geq0\\&2x+2y\leq10\\&2y\leq4\\&x+2y\leq6}\Leftrightarrow\heva{&x\geq0\\&0\leq y \leq 2\\&x+y\leq5\\&x+2y\leq6.}$$
		Vẽ các đường thẳng $(d_1) \colon y=2$, $(d_2) \colon x+y=5$, $(d_3) \colon x+2y=6$. Ta có miền nghiệm của bất phương trình là phần tô màu như hình vẽ:
		\begin{center}
			\begin{tikzpicture}[line join = round, line cap = round, >=stealth, font=\footnotesize, scale=1.2]
				\tikzset{label style/.style={font=\footnotesize}}
				\def \xmin{-2.5}
				\def \xmax{8}
				\def \ymin{-2}
				\def \ymax{6.5}
				\tkzDefPoints{0/2/A, 2/2/B, 4/1/C, 5/0/D, 0/0/E}
				\draw[gray!20](\xmin,\ymin) grid (\xmax,\ymax);
				\draw [->](\xmin,0)--(0,0)node[below left]{$O$}--(\xmax,0)node[above]{$x$};
				\draw [->](0,\ymin)--(0,\ymax)node[right]{$y$};
				\foreach \x in {-2,-1,1,2,3,4,5,6,7} \draw[shift={(\x,0)}] (0pt,2pt)--(0pt,-2pt) node[below]{$\x$};
				\foreach \y in {-1,1,3,4,5,6} \draw [shift={(0,\y)}] (2pt,0pt)--(-2pt,0pt) node[left]{$\y$};
				\draw [shift={(0,2)}] (2pt,0pt)--(-2pt,0pt) node[below left]{$2$};
				\tkzDrawLine[add=1 and 2.7](A,B)
				%\tkzDrawLine[add=0.5 and 0.5](B,C A,B A,D D,C)
				\tkzDrawLine[add=1.5 and 5](D,C)
				\tkzDrawLine[add=2 and 2](C,B)
				%\tkzDrawLines[add=4 and 2](D,C)
				\tkzDrawPoints[fill=black](A,B,C,D,E)
				\tkzLabelPoints[above right](A,B,C,D)
				%\tkzLabelPoints[below](D)
				\tkzLabelPoints[below right](E)
				\tkzFillPolygon[color=cyan,fill opacity=.5](A,B,C,D,E)
				\tkzLabelLine[pos=3.6,above](A,B){$(d_1)$}
				\tkzLabelLine[pos=6,below left](D,C){$(d_2)$}
				\tkzLabelLine[pos=3,below left](C,B){$(d_3)$}
				%\tkzLabelLine[pos=5,above right](C,D){$(d_4)$}
			\end{tikzpicture}
		\end{center} 
		Ta có 
		{\allowdisplaybreaks
			\begin{eqnarray*}
				&&(d_1) \cap Oy=A(0;2), (d_1) \cap (d_3)=B(2;2), (d_2)\cap (d_3)=C(4;1)\\
				&&(d_2) \cap Ox=D(5;0), E \equiv O=(0;0). 
			\end{eqnarray*}
		}
		Lãi suất thu được là $f(x,y)=3x+5y$ (nghìn đồng).
		\begin{center}
			\renewcommand\arraystretch{1.6}
			\renewcommand{\tabcolsep}{6mm}
			\begin{tabular}{|c|c|c|c|c|c|}
				\hline 
				$M(x;y)$& $A$ & $B$ & $C$ & $D$ & $E$ \\ 
				\hline 
				$f(x,y)=3x+5y$& $10$ & $16$ & $17$ & $15$ & $0$ \\
				\hline 
			\end{tabular} 
		\end{center}
		Do đó $f(x,y)$ đạt giá trị lớn nhất tại $C(4;1)$.\\
		Vậy phương án sản xuất 4 sản phẩm loại I và $1$ sản phẩm loại II sẽ cho lãi cao nhất.
	}
\end{bt}
\begin{bt}%[BG10-2022]%[Toanvo]%[0D4G4-4]
	Một nhà khoa học nghiên cứu về tác động phối hợp của vitamin $A$ và vitamin $B$ đối với cơ thể con người. Kết quả như sau:
	\begin{enumerate}
		\item  Một người có thể tiếp nhận được mỗi ngày không quá $600$ đơn vị vitamin $A$ và không quá $500$ đơn vị vitamin $B$.
		\item Một người mỗi ngày cần từ $400$ đến $1000$ đơn vị vitamin cả $A$ lẫn $B$.
		\item  Do tác động phối hợp của hai loại vitamin, mỗi ngày số đơn vị vitamin $B$ phải nhiều hơn $\dfrac{1}{2}$ số đơn vị vitamin $A$ nhưng không nhiều hơn ba lần số đơn vị vitamin $A$. Biết giá một đơn vị vitamin $A$ là $9$ đồng và giá một đơn vị vitamin $B$ là $7{,}5$ đồng.
	\end{enumerate}
	Tìm phương án dùng vitamin $A$ và vitamin $B$ thỏa mãn các điều kiện trên sao cho số tiền phải trả ít nhất.
	\loigiai{
		Gọi số đơn vị vitamin $A$ cần dùng là $x$; số đơn vị vitamin $B$ cần dùng là $y$.\\
		Điều kiện: $0 \leq x \leq 600$, $0 \leq y \leq 500$.\\
		Tổng số đơn vị vitamin $A$ và vitamin $B$ cần dùng là $x+y$.\\
		Ta có hệ bất phương trình $\heva{&0\leq x\leq 600\\&0\leq y\leq 500\\&400\leq x+y\leq 1000\\&\dfrac{x}{2}\leq y\leq 3x.}$\\
		Vẽ các đường thẳng $(d_1) \colon x=600$, $(d_2) \colon y=500$, $(d_3) \colon x+y=400$, $(d_4) \colon x+y=1000$, \\
		$(d_5) \colon \dfrac{x}{2}-y=0$, $(d_6) \colon y=3x$.\\
		Ta có miền nghiệm của hệ bất phương trình như hình vẽ:
		\begin{center}
			\begin{tikzpicture}[line join = round, line cap = round, >=stealth, font=\footnotesize, scale=1.2]
				\tikzset{label style/.style={font=\footnotesize}}
				\def \xmin{-3.5}
				\def \xmax{5.5}
				\def \ymin{-1}
				\def \ymax{6}
				\tkzDefPoints{0.833333/2.5/A, 2.5/2.5/B, 3/2/C, 3/1.5/D, 1.33333/0.66666/E, 0.5/1.5/F}
				\draw[gray!20](\xmin,\ymin) grid (\xmax,\ymax);
				\draw [->](\xmin,0)--(0,0)node[above left]{$O$}--(\xmax,0)node[above]{$x$};
				\draw [->](0,\ymin)--(0,\ymax)node[right]{$y$};
				\draw[shift={(-3,0)}] (0pt,2pt)--(0pt,-2pt) node[below]{$-600$};
				\draw[shift={(-2,0)}] (0pt,2pt)--(0pt,-2pt) node[below]{$-400$};
				\draw[shift={(-1,0)}] (0pt,2pt)--(0pt,-2pt) node[below]{$-200$};
				\draw[shift={(1,0)}] (0pt,2pt)--(0pt,-2pt) node[below]{$200$};
				\draw[shift={(2,0)}] (0pt,2pt)--(0pt,-2pt) node[below]{$400$};
				\draw[shift={(3,0)}] (0pt,2pt)--(0pt,-2pt) node[below right]{$600$};
				\draw[shift={(4,0)}] (0pt,2pt)--(0pt,-2pt) node[below]{$800$};
				\draw[shift={(5,0)}] (0pt,2pt)--(0pt,-2pt) node[below]{$1000$};
				\draw[shift={(0,1)}] (2pt,0pt)--(-2pt,0pt) node[left]{$200$};
				\draw[shift={(0,2)}] (2pt,0pt)--(-2pt,0pt) node[left]{$400$};
				\draw[shift={(0,3)}] (2pt,0pt)--(-2pt,0pt) node[left]{$600$};
				\draw[shift={(0,4)}] (2pt,0pt)--(-2pt,0pt) node[left]{$800$};
				\draw[shift={(0,5)}] (2pt,0pt)--(-2pt,0pt) node[left]{$1000$};
				\tkzDrawLine[add=4 and 7](D,C)
				\tkzDrawLine[add=1.5 and 2](B,A)
				\tkzDrawLine[add=1.6 and 4](E,F)
				\tkzDrawLines[add=4.5 and 6](C,B)
				\tkzDrawLines[add=1.7 and 1](E,D)
				\tkzDrawLines[add=2.2 and 3](F,A)
				\tkzDrawPoints[fill=black](A,B,C,D,E,F)
				\tkzLabelPoints[above right](A,B,C)
				\tkzLabelPoints[below right](D)
				\tkzLabelPoints[below](E)
				\tkzLabelPoints[left](F)
				\tkzFillPolygon[color=cyan,fill opacity=.5](A,B,C,D,E,F)
				\tkzLabelLine[pos=7.5,right](D,C){$(d_1)$}
				\tkzLabelLine[pos=3,above](B,A){$(d_2)$}
				\tkzLabelLine[pos=5,above right](E,F){$(d_3)$}
				\tkzLabelLine[pos=7,above](C,B){$(d_4)$}
				\tkzLabelLine[pos=2,below](E,D){$(d_5)$}
				\tkzLabelLine[pos=4,left](F,A){$(d_6)$}
			\end{tikzpicture}
		\end{center} 
		Ta có 
		{\allowdisplaybreaks
			\begin{eqnarray*}
				&&(d_1) \cap (d_6)=A\left(\dfrac{500}{3};500\right), (d_4) \cap (d_2)=B(500;500), (d_1) \cap (d_4)=C(600;400),\\
				&&(d_1) \cap (d_5)=D(600;300), (d_3) \cap (d_5)=E\left(\dfrac{800}{3};\dfrac{400}{3}\right), (d_3) \cap (d_6)=F(100;300).
			\end{eqnarray*}
		}
		Số tiền phải trả là $f(x,y)=9x+7{,}5y$ (nghìn đồng).
		\begin{center}
			\renewcommand\arraystretch{1.6}
			\renewcommand{\tabcolsep}{6mm}
			\begin{tabular}{|c|c|c|c|c|c|c|}
				\hline 
				$M(x;y)$& $A$ & $B$ & $C$ & $D$ & $E$ & $F$ \\ 
				\hline 
				$f(x,y)=9x+7{,}5y$& $5250$ & $8250$ & $8400$ & $7650$ & $3400$ & $3150$ \\
				\hline 
			\end{tabular} 
		\end{center}
		Do đó $f(x,y)=9x+7{,}5y$ đạt giá trị nhỏ nhất tại $F(100;300)$.\\
		Vậy phương án dùng mỗi ngày $100$ đơn vị vitamin $A$ và $300$ vitamin $B$ thì số tiền phải trả là ít nhất.
	}
\end{bt}
\subsection{Bài tập trắc nghiệm}
\Opensolutionfile{ans}[ans/ans-0D2-bieu-dien-hinh-hoc-tap-nghiem]
\begin{ex}%[0D4Y4-2]
	Điểm nào sau đây thuộc miền nghiệm của hệ bất phương trình $\heva{&2x+7y-3>0\\&x-2y\ge 0}$?
	\choice
	{$P(-1;-5)$}
	{$O(0;0)$}
	{$M(3;-1)$}
	{\True $N(2;0)$}
	\loigiai{
		Thay lần lượt tọa độ các điểm vào hệ bất phương trình, ta thấy tọa độ điểm $N$ thỏa mãn.\\
		Vậy điểm $N(2;0)$ thuộc miền nghiệm của hệ bất phương trình.
	}
\end{ex}
\begin{ex}%[0D4Y4-4]
	Miền nghiệm của hệ bất phương trình $\heva{&2x-5y-1>0\\&2x+y+5>0\\&x+y+1<0}$ chứa điểm nào trong các điểm sau?
	\choice
	{$(0;0)$}
	{$(1;0)$}
	{\True $(0;-2)$}
	{$(0;2)$}
	\loigiai{Thay điểm $(0;-2)$ vào hệ bất phương trình, ta có 
		$\heva{&2 \cdot 0 - 5 \cdot (-2)-1=9>0\\&2 \cdot 0 + (-2) +5=3>0\\&0+(-2)+1=-1<0}$ (đúng).
	}
\end{ex}

\begin{ex}%[0D4Y4-4]
	Miền nghiệm của hệ bất phương trình $\heva{& x-y\ge 3 \\ & 2x+y<4}$ chứa điểm nào trong các điểm sau?
	\choice
	{\True $(1;-3)$}
	{$(-2;1)$}
	{$(3;-2)$}
	{$(4;1)$}
	\loigiai{Thay điểm $(1;-3)$ vào hệ bất phương trình, ta có
		$\heva{& 1-(-3) = 4 \ge 3 \\ & 2 \cdot 1 + (-3) = -2 <4}$ (đúng).
	}
\end{ex}

\begin{ex}%[0D4B4-4]
	Miền nghiệm của hệ bất phương trình $\heva{& 2x-y>0\\ & x+y\ge -1 \\ & x-y<-2}$ \textbf{không} chứa điểm nào trong các điểm sau?
	\choice
	{$(5;8)$}
	{$(6;9)$}
	{$(4;7)$}
	{\True $(3,4)$}
	\loigiai{Thay điểm $(3;4)$ vào hệ bất phương trình, ta có
		$\heva{& 2 \cdot 3 - 4 = 2>0 \\ & 3+4=7 \ge -1 \\ & 3-4 = -1<-2}$ (sai).
	}
\end{ex}

\begin{ex}%[0D4B4-4]
	Điểm nào sau đây thuộc miền nghiệm của hệ bất phương trình $\heva{& 2x+3y-1>0\\& 5x-y+4<0.}$
	\choice
	{$(0;0)$}
	{$(-2;0)$}
	{$(-1;-4)$}
	{\True $(-3;4)$}
	\loigiai{
		Thay tọa độ từng điểm vào mỗi hệ bất phương trình.
		\begin{itemize}
			\item Với điểm $(0;0)$ ta được $2\cdot 0+3 \cdot 0-1=-1<0$ (sai) nên không thỏa mãn bất phương trình đầu.
			\item Với điểm $(-2;0)$ ta được $2 \cdot (-2)+3\cdot 0-1=-5<0$ (sai) nên không thỏa mãn bất phương trình đầu.
			\item Với điểm $(-1;-4)$ ta được $2 \cdot (-1)+3 \cdot (-4)-1=-15<0$ (sai) nên không thỏa mãn bất phương trình đầu.
			\item Với điểm $(-3;4)$ ta được $\heva{&2 \cdot(-3)+3\cdot 4-1=5>0 \\ &5 \cdot (-3)-4+4=-15<0}$ (đúng) thỏa mãn cả hai bất phương trình của hệ.
		\end{itemize}
	}
\end{ex}

\begin{ex}%[0D4B4-4]
	Cho hệ bất phương trình $\heva{&y\geq0\\&3x+2y-6<0}$ có miền nghiệm $S$ và bốn điểm $O(0;0)$, $A(2;3)$, $B(-1;1)$, $C(-1;3)$. Trong các điểm đã cho, có bao nhiêu điểm thuộc $S$?
	\choice
	{$1$}
	{$2$}
	{\True $3$}
	{$4$}
	\loigiai{Thay điểm $O(0;0)$ vào hệ bất phương trình, ta có 
		$\heva{&0\geq0\\&3\cdot 0+2 \cdot 0-6=-6<0}$ (đúng).\\
		Thay điểm $A(2;3)$ vào hệ bất phương trình, ta có 
		$\heva{&3\geq 0\\&3\cdot 2+2 \cdot 3-6=6<0}$ (sai).\\
		Thay điểm $B(-1;1)$ vào hệ bất phương trình, ta có 
		$\heva{&1\geq 0\\&3\cdot (-1)+2 \cdot 1-6=-7<0}$ (đúng).\\
		Thay điểm $C(-1;3)$ vào hệ bất phương trình, ta có 
		$\heva{&3\geq 0\\&3\cdot (-1)+2 \cdot 3-6=-3<0}$ (đúng).
	}
\end{ex}

\begin{ex}%[0D4B4-4]
	Xét hệ bất phương trình $\heva{& x+y\le 2\\ & x-2y\ge -1 \\ & y\ge 1}$ và bốn điểm $A(1;1)$, $B(2;1)$, $C(0;1)$, $D(-2;0)$. Trong các điểm trên, có bao nhiêu điểm thuộc miền nghiệm của hệ bất phương trình đã cho?
	\choice
	{\True $1$}
	{$2$}
	{$3$}
	{$4$}
	\loigiai{Thay điểm $A(1;1)$ vào hệ bất phương trình, ta có 
		$\heva{& 1+1 =2 \le 2\\ & 1-2 \cdot 1 =-1 \ge -1 \\ & 1\ge 1}$ (đúng).\\
		Thay điểm $B(2;1)$ vào hệ bất phương trình, ta có 
		$\heva{& 2+1 =3 \le 2\\ & 2-2 \cdot 1 =0 \ge -1 \\ & 1\ge 1}$ (sai).\\
		Thay điểm $C(0;1)$ vào hệ bất phương trình, ta có 
		$\heva{& 0+1 =1 \le 2\\ & 0-2 \cdot 1 =-2 \ge -1 \\ & 1\ge 1}$ (sai).\\
		Thay điểm $D(-2;0)$ vào hệ bất phương trình, ta có 
		$\heva{& -2+0 =-2 \le 2\\ & -2-2 \cdot 0 =-2 \ge -1 \\ & 0\ge 1}$ (sai).\\
	}
\end{ex}

\begin{ex}%[0D4B4-4]
	Cặp số $(x;y)$ nào sau đây là một nghiệm của hệ bất phương trình $\heva{&2x+3y-1>0\\&5x-y+4\leq0}$?
	\choice
	{\True $(0;4)$}
	{$(0;0)$}
	{$(-2;-4)$}
	{$(-3;-4)$}
	\loigiai{Thay cặp số $(0;4)$ vào hệ bất phương trình đã cho, ta có
		$\heva{&2 \cdot 0 + 3 \cdot 4-1=11>0\\&5 \cdot 0-4+4 = 0 \leq 0}$
		(đúng).
	} 
\end{ex}

\begin{ex}%[0D4B4-4]
	Trong các cặp số $(x;y)$ sau, cặp số nào \textbf{không là} nghiệm của hệ bất phương trình $$\heva{&x-y-2\leq0\\&3x-2y+2>0}?$$
	\choice
	{$(x;y)=(0;0)$}
	{$(x;y)=(1;1)$}
	{\True $(x;y)=(-1;1)$}
	{$(x;y)=(-1;-1)$}
	\loigiai{Thay cặp số $(-1;1)$ vào hệ bất phương trình đã cho, ta có
		$\heva{&-1-1-2=-4 \leq 0\\&3 \cdot (-1)-2 \cdot 1+2=-3>0}$ (sai).
	} 
\end{ex}

\begin{ex}%[0D4B4-4]
	Cặp số $(x;y)=(0;0)$ \textbf{không} là nghiệm của hệ bất phương trình nào trong các hệ bất phương trình sau?
	\def\dotEX{}	
	\choice
	{$\heva{& 2x-y<1 \\& x\ge 0 \\& y\le 1}$}
	{\True $\heva{& 2x+y<1 \\& x\ge 0 \\& y<0}$}
	{$\heva{& 2x-y<1 \\& x\ge 0 \\& y\ge 0}$}
	{$\heva{& 2x+y<1 \\& x\le 0 \\& y<1}$}
	\loigiai{Thay cặp số $(0;0)$ vào hệ bất phương trình $\heva{& 2x+y<1 \\& x\ge 0 \\& y<0}$, ta có
		$\heva{& 2 \cdot 0+0=0<1 \\& 0 \ge 0 \\& 0<0}$ (sai).
	} 
\end{ex}

\begin{ex}%[0D4B4-4]
	Điểm nào sau đây thuộc miền nghiệm của hệ bất phương trình $\heva{&5x+3y-19\leq 0 \\ &12x-5y-13\geq 0}$?
	\choice
	{\True $N(1+\sqrt{2};\sqrt{2})$}
	{$N(1+\sqrt{2};2+\sqrt{2})$}
	{$N(1;3+\sqrt{2})$}
	{$N(5+\sqrt{2};\sqrt{2})$}
	\loigiai{Thay điểm $N(1+\sqrt{2};\sqrt{2})$ vào hệ bất phương trình đã cho, ta có
		$$\heva{&5 \cdot \left(1+\sqrt{2} \right)+3 \cdot \sqrt{2}-19 = -14+8\sqrt{2} \leq 0 \\ &12 \cdot \left(1+\sqrt{2} \right)-5 \sqrt{2} -13 = -1+ 7 \sqrt{2}\geq 0} \text{ (đúng).}$$
	} 
\end{ex}

\begin{ex}%[0D4B4-4]
	Cặp số $(x;y)=(-1;3)$ là nghiệm của hệ bất phương trình nào trong các hệ bất phương trình sau?	
	\def\dotEX{}
	\choice
	{$\heva{& x-y\le 2\\& 3x+2y\ge 2\\& y\le 0\\& x<0}$}
	{$\heva{& x-y\le 2\\& 3x+y\ge 2\\& y\le 0\\& x<0}$}
	{$\heva{& x-y\le 2\\& 3x+y\ge 2\\& y\ge 0\\& x<0}$}
	{\True $\heva{& x-y\le 2\\& 3x+2y\ge 2\\& y\ge 0\\& x<0}
		$}
	\loigiai{Thay cặp số $(-1;3)$ vào hệ bất phương trình $\heva{& x-y\le 2\\& 3x+2y\ge 2\\& y\ge 0\\& x<0}$, ta có
		$\heva{& -1-3 =-4 \le 2\\& 3 \cdot (-1)+2 \cdot 3 = 3 \ge 2\\& 3 \ge 0\\& -1<0}$ (đúng).
	} 
\end{ex}

\begin{ex}%[0D4B4-4]
	Hệ bất phương trình $\heva{& y\le x+1 \\& y+x>3}$ nhận cặp số $(x;y)$ nào sau đây làm nghiệm của nó?
	\choice
	{$(x;y)=(2;1)$}
	{\True $(x;y)=(2;3)$}
	{$(x;y)=(3;0)$}
	{$(x;y)=(1;3)$}
	\loigiai{Thay cặp số $(2;3)$ vào hệ bất phương trình đã cho, ta có $\heva{&  3 \le 2+1=3 \\& 3+2=5>3}$ (đúng).
	} 
\end{ex}

\begin{ex}%[0D4K4-2]
	Cho hệ $\heva{&2x+3y<5\\&x+\dfrac{3}{2}y<5.}$  Gọi $S_1$ là tập nghiệm của bất phương trình $2x+3y<5$, $S_2$ là tập nghiệm của bất phương trình $x+\dfrac{3}{2}y<5$ và $S$ là tập nghiệm của hệ thì
	\choice
	{\True $S \subset S_2$}
	{$S_2\subset S_1 $}
	{$S_2\subset S $}
	{$S=S_1 \cup S_2$}
	\loigiai{Ta có $x+\dfrac{3}{2}y<5 \Leftrightarrow 2x + 3y < 10$. Do vậy với mọi cặp $(x_0, y_0)$ thỏa mãn bất phương trình $2x + 3y < 5$ đều thỏa mãn bất phương trình $2x + 3y < 10$.\\
		Vậy tập nghiệm của hệ ban đầu sẽ là tập con của bất phương trình $x+\dfrac{3}{2}y<5$. 
	}
\end{ex}

\begin{ex}%[0D4K4-2]
	Cho hệ phương trình $\heva{3x+y&>4 \hspace{0.5cm}(1)\\x+\dfrac{1}{3}y&>4\hspace{0.5cm}(2).}$ Gọi $S_1$ là tập nghiệm của bất phương trình $(1)$, $S_2$ là nghiệm của bất phương trình $(2)$ và $S$ là tập nghiệm của hệ bất phương trình đã cho. Khẳng định nào sau đay là đúng?
	\choice
	{$S_1\subset S_2$}
	{\True $S_2\subset S_1$}
	{$S_2\cup S=S_1$}
	{$S_1\subset S$}
	\loigiai{Lần lượt biểu diễn tập nghiệm của bất phương trình $(1)$, bất phương trình $(2)$ và hệ bất phương trình ta có các hình vẽ sau:\\
		\begin{tabular}{ccc}
			\begin{tikzpicture}[scale=0.5, font=\footnotesize, line join=round, line cap=round, >=stealth,xscale=1.5]
				\draw[->] (-1.2,0) -- (4.9,0)node[above]{$x$};
				\foreach \x in {-1,1,2,3,4}
				\draw[shift={(\x,0)},color=black] (0pt,2pt) -- (0pt,-2pt) node[below] {\footnotesize $\x$};
				\draw[->,color=black] (0,-1.2) -- (0,13.2)node[right]{$y$};
				\foreach \y in {1,2,3,4,5,6,7,8,9,10,11,12}
				\draw[shift={(0,\y)},color=black] (2pt,0pt) -- (-2pt,0pt) node[right] {\footnotesize $\y$};
				\node[above left] at (0,0){$O$};
				\clip(-2,-1) rectangle (5,13);
				\fill[pattern=north west lines] (1.666666,-1) -- (-1,-1) -- (-1,7) -- cycle;
				\draw[line width=1.2pt,smooth,samples=100,domain=-1:3] plot(\x,{4-3*(\x)});
			\end{tikzpicture}&\begin{tikzpicture}[scale=0.5, font=\footnotesize, line join=round, line cap=round, >=stealth,xscale=1.5]
				\draw[->] (-1.2,0) -- (4.9,0)node[above]{$x$};
				\foreach \x in {-1,1,2,3,4}
				\draw[shift={(\x,0)},color=black] (0pt,2pt) -- (0pt,-2pt) node[below] {\footnotesize $\x$};
				\draw[->,color=black] (0,-1.2) -- (0,13.2)node[right]{$y$};
				\foreach \y in {1,2,3,4,5,6,7,8,9,10,11,12}
				\draw[shift={(0,\y)},color=black] (2pt,0pt) -- (-2pt,0pt) node[right] {\footnotesize $\y$};
				\node[above left] at (0,0){$O$};
				\clip(-2,-1) rectangle (5,13);
				\fill[pattern=north east lines] (4.33333,-1) -- (-1,-1) -- (-1,13) -- (-0.333333,13) -- cycle;
				\draw[line width=1.2pt,smooth,samples=100,domain=-1:13] plot(\x,{12-3*(\x)});
			\end{tikzpicture}&\begin{tikzpicture}[scale=0.5, font=\footnotesize, line join=round, line cap=round, >=stealth,xscale=1.5]
				\draw[->] (-1.2,0) -- (4.9,0)node[above]{$x$};
				\foreach \x in {-1,1,2,3,4}
				\draw[shift={(\x,0)},color=black] (0pt,2pt) -- (0pt,-2pt) node[below] {\footnotesize $\x$};
				\draw[->,color=black] (0,-1.2) -- (0,13.2)node[right]{$y$};
				\foreach \y in {1,2,3,4,5,6,7,8,9,10,11,12}
				\draw[shift={(0,\y)},color=black] (2pt,0pt) -- (-2pt,0pt) node[right] {\footnotesize $\y$};
				\node[above left] at (0,0){$O$};
				\clip(-2,-1) rectangle (5,13);
				\fill[pattern=north east lines] (4.33333,-1) -- (-1,-1) -- (-1,13) -- (-0.333333,13) -- cycle;
				\fill[pattern=north west lines] (1.666666,-1) -- (-1,-1) -- (-1,7) -- cycle;
				\draw[line width=1.2pt,smooth,samples=100,domain=-1:13] plot(\x,{12-3*(\x)});
				\draw[line width=1.2pt,smooth,samples=100,domain=-1:3] plot(\x,{4-3*(\x)});
			\end{tikzpicture}\\
			$S_1$&$S_2$&$S$	
		\end{tabular}\\
		Dựa vào miền nghiệm của mỗi bất phương trình và của hệ ta thấy $S_2\subset S_1$ và $S=S_2$.}
\end{ex}

\begin{ex}%[0D4K4-2]
	Tìm số thực $a$ sao cho miền nghiệm của hệ bất phương trình $\heva{&x\ge 0\\&y\geq0\\&ax-3y\geq-12}$ là một tam giác có diện tích bằng $6$.
	\choice
	{\True $a=-4$} 
	{$a=4$}
	{$a=6$}
	{$a=12$}
	\loigiai
	{\immini{
			Xét $ax-3y=-12$.\\
			Với $x=0\Rightarrow y=4$.\\ Với $y=0\Rightarrow x=-\dfrac{12}{a}$.\\
			Do $x\ge 0$ suy ra $-\dfrac{12}{a}\ge 0$ hay $a<0$.\\
			Dựa vào hình bên ta có:\\
			$S$=$\dfrac{1}{2}\cdot 4\cdot \dfrac{-12}{a}=6$\\
			$\Rightarrow$ $a= -4$.\\
		}{
			\begin{tikzpicture}[scale=0.8, font=\footnotesize, line join=round, line cap=round, >=stealth]
				\draw[->] (-1,0) -- (4.5,0)node[above]{$x$};
				\foreach \x in {1,2,3,4}
				\draw[shift={(\x,0)},color=black] (0pt,2pt) -- (0pt,-2pt) node[below] {\footnotesize $\x$};
				\draw[->,color=black] (0,-1) -- (0,4.4)node[right]{$y$};
				\foreach \y in {1,2,3,4}
				\draw[shift={(0,\y)},color=black] (2pt,0pt) -- (-2pt,0pt) node[left]{\footnotesize $\y$};
				\node[below left] at (0,0){$O$};
				\node[below left] at (4,1.5){$\dfrac{-12}{a}$};
				\clip(-1,-1) rectangle (4.3,4.3);
				\fill[pattern=north east lines] (0,4) -- (3.25,0) -- (0,0) -- cycle;
				\draw[line width=1.2pt,smooth,samples=100,domain=-1:4] plot(\x,{4-1.25*(\x)});
				%\fill[pattern=north east lines] (-1,0) -- (-1,-1) -- (4,-1) -- (4,0)-- cycle;
			\end{tikzpicture}
		}
	}
\end{ex}

\begin{ex}%[0D4K4-2]
	Tính diện tích $S$ của miền nghiệm hệ bất phương trình $\heva{& y+x\le 3 \\ & y-x\le 3 \\ & y\ge -1}$.
	\choice
	{$S=8$}
	{$S=25$}
	{\True $S=16$}
	{$S=12$}
	\loigiai{
		\immini{
			Miền nghiệm là miền tam giác như hình vẽ.\\
			Diện tích $S=\dfrac{1}{2}.8.4=16$
		}
		{
			\begin{tikzpicture}[scale=0.5, font=\footnotesize, line join=round, line cap=round, >=stealth]
				\def\xmin{-5} \def\xmax{5}
				\def\ymin{-2} \def\ymax{4}
				\clip(\xmin,\ymin) rectangle (\xmax,\ymax);
				\tkzDefPoints{\xmax/\ymax/A1,\xmin/\ymax/A2,\xmin/\ymin/A3,\xmax/\ymin/A4}
				\fill[pattern=north east lines,pattern color=black!60] (A1)--(A2)--(A3)--(A4)--cycle;
				\tkzDefPoints{-4/-1/M,0/3/I,4/-1/N}
				\fill[color=white] (M)--(I)--(N)--cycle;
				\draw[domain=-5:5] plot(\x,{3-(\x)}) plot(\x,{3+(\x)}) plot(\x,{-1});
				\begin{scriptsize}
					\draw[->](\xmin,0)--(\xmax,0); \draw(\xmax-0.1,0) node[below]{$x$};
					\draw[->](0,\ymin)--(0,\ymax); \draw(0,\ymax-0.2) node[right]{$y$};
					\foreach \x in {-4,4}
					\draw (\x,0.05) -- ++(0,-0.1) node [above] {$\x$};
					\draw node [above right]{$O$} (0,3) 
					node[left]{$3$} (0,-1) 
					node[below left]{$-1$};
					\draw[dashed] (-4,0)--(M) (4,0)--(N);
				\end{scriptsize}
			\end{tikzpicture}
		}
	}
\end{ex}

\begin{ex}%[0D4K4-2]
	Tính diện tích $S$ của miền nghiệm của hệ bất phương trình $\heva{& x\ge -3 \\ & y+x\le 8 \\ & y-x\ge -2}$.
	\choice
	{$S=48$}
	{\True $S=64$}
	{$S=81$}
	{$S=49$}
	\loigiai{
		\immini{
			Miền nghiệm là miền tam giác như hình vẽ.\\
			Diện tích $S=\dfrac{1}{2}.16.8=64$.
		}
		{
			\begin{tikzpicture}[scale=0.3, font=\footnotesize, line join=round, line cap=round, >=stealth]
				\def\xmin{-6} \def\xmax{9}
				\def\ymin{-6} \def\ymax{12}
				\clip(\xmin,\ymin) rectangle (\xmax,\ymax);
				\tkzDefPoints{\xmax/\ymax/A1,\xmin/\ymax/A2,\xmin/\ymin/A3,\xmax/\ymin/A4}
				\fill[pattern=north east lines,pattern color=black!60] (A1)--(A2)--(A3)--(A4)--cycle;
				\tkzDefPoints{-3/-5/M,-3/11/N,5/3/I}
				\fill[color=white] (M)--(N)--(I)--cycle;
				\draw[domain=-5:5] plot(\x,{8-(\x)}) plot(\x,{(\x)-2});
				\draw (-3,\ymin)--(-3,\ymax);
				\begin{scriptsize}
					\draw[->](\xmin,0)--(\xmax,0); \draw(\xmax-0.1,0) node[below]{$x$};
					\draw[->](0,\ymin)--(0,\ymax); \draw(0,\ymax-0.2) node[right]{$y$};
					\foreach \x in {-3,5}
					\draw (\x,0.05) -- ++(0,-0.1) node [above right] {$\x$};
					\foreach \x in {-5,3,11}
					\draw (0.05,\x) -- ++(-0.1,0) node [above left] {$\x$};
					\draw node [above right]{$O$};
					\draw[dashed] (M)--(0,-5) (N)--(0,11) (0,3)--(I)--(5,0);
				\end{scriptsize}
			\end{tikzpicture}
		}
	}
\end{ex}

\begin{ex}%[0D4K4-2]
	Tính chu vi $P$ của miền nghiệm hệ bất phương trình $\heva{& x\ge -3 \\ & x\le 6 \\& y\le 5 \\ & y\ge -6}$.
	\choice
	{$P=38$}
	{$P=36$}
	{$P=42$}
	{\True $P=40$}
	\loigiai{
		\immini{
			Miền nghiệm là miền hình chữ nhật như hình vẽ.\\
			Chu vi $P=2(11+9)=40$.
		}
		{
			\begin{tikzpicture}[scale=0.3, font=\footnotesize, line join=round, line cap=round, >=stealth]
				\def\xmin{-5} \def\xmax{8}
				\def\ymin{-8} \def\ymax{7}
				\clip(\xmin,\ymin) rectangle (\xmax,\ymax);
				\tkzDefPoints{\xmax/\ymax/A1,\xmin/\ymax/A2,\xmin/\ymin/A3,\xmax/\ymin/A4}
				\fill[pattern=north east lines,pattern color=black!60] (A1)--(A2)--(A3)--(A4)--cycle;
				\tkzDefPoints{-3/-6/M,-3/5/N,6/5/P,6/-6/Q}
				\fill[color=white] (M)--(N)--(P)--(Q)--cycle;
				\draw (-3,\ymin)--(-3,\ymax) (6,\ymin)--(6,\ymax) (\xmin,5)--(\xmax,5) (\xmin,-6)--(\xmax,-6);
				\begin{scriptsize}
					\draw[->](\xmin,0)--(\xmax,0); \draw(\xmax-0.1,0) node[below]{$x$};
					\draw[->](0,\ymin)--(0,\ymax); \draw(0,\ymax-0.2) node[right]{$y$};
					\foreach \x in {-3,6}
					\draw (\x,0.05) -- ++(0,-0.1) node [above right] {$\x$};
					\foreach \x in {-6,5}
					\draw (0.05,\x) -- ++(-0.1,0) node [above left] {$\x$};
					\draw node [above right]{$O$};
				\end{scriptsize}
			\end{tikzpicture}
		}	
	}
\end{ex}

\begin{ex}%[0D4K4-2]
	Tìm giá trị của số thực $a$ sao cho miền nghiệm của hệ bất phương trình $\heva{& x\le a\\& x\ge 0\\& y\ge 0\\& y\le 2}$ có diện tích bằng $6$.
	\choice
	{$a=-3$}
	{$a=8$}
	{\True $a=3$}
	{$a=-8$}
	\loigiai{
		\immini{
			Từ giả thiết suy ra $a>0$.\\
			Diện tích $S=2a=6$. Do đó $a=3$.
		}
		{
			\begin{tikzpicture}[scale=0.6, font=\footnotesize, line join=round, line cap=round, >=stealth]
				\def\xmin{-1} \def\xmax{4}
				\def\ymin{-1} \def\ymax{3}
				\clip(\xmin,\ymin) rectangle (\xmax,\ymax);
				\tkzDefPoints{\xmax/\ymax/A1,\xmin/\ymax/A2,
					\xmin/\ymin/A3,\xmax/\ymin/A4}
				\fill[pattern=north east lines,pattern color=black!60] (A1)--(A2)--(A3)--(A4)--cycle;
				\tkzDefPoints{0/0/O,0/2/A,3/2/B,3/0/C}
				\fill[color=white] (O)--(A)--(B)--(C)--cycle;
				\draw (\xmin,2)--(\xmax,2) (3,\ymin)--(3,\ymax);
				\tkzDrawPoints(A,B,C)
				\begin{scriptsize}
					\draw[->](\xmin,0)--(\xmax,0); \draw(\xmax-0.1,0) node[below]{$x$};
					\draw[->](0,\ymin)--(0,\ymax); \draw(0,\ymax-0.2) node[right]{$y$};
					\draw node [below left]{$O$} 
					(A) node [below left]{$2$} 
					(C) node [below left]{$a$};
				\end{scriptsize}
			\end{tikzpicture}
		}	
	}
\end{ex}

\begin{ex}%[0D4K4-2]
	Tìm giá trị của số thực $a$ sao cho miền nghiệm của hệ bất phương trình $\heva{& x-y\ge a\\& x\le 0\\& y\ge 0}$ là một tam giác có diện tích bằng $2$.
	\choice
	{$a=2$}
	{\True $a=-2$}
	{$a=\sqrt{2}$}
	{$a=-\sqrt{2}$}
	\loigiai{
		\immini{
			Do $a\le 0$, $y\ge 0$ suy ra $x-y\le 0$
			suy ra $a<0$.\\
			Diện tích $S=\dfrac{1}{2} a^2 =2$. Do đó $a=-2$.
		}
		{
			\begin{tikzpicture}[scale=0.6, font=\footnotesize, line join=round, line cap=round, >=stealth]
				\def\xmin{-3} \def\xmax{1}
				\def\ymin{-1} \def\ymax{3}
				\clip(\xmin,\ymin) rectangle (\xmax,\ymax);
				\tkzDefPoints{\xmax/\ymax/A1,\xmin/\ymax/A2,
					\xmin/\ymin/A3,\xmax/\ymin/A4}
				\fill[pattern=north east lines,pattern color=black!60] (A1)--(A2)--(A3)--(A4)--cycle;
				\tkzDefPoints{0/0/O,0/2/A,-2/0/B}
				\fill[color=white] (O)--(A)--(B)--cycle;
				\draw[domain=-3:1] plot(\x,{(\x)+2});
				\tkzDrawPoints(A,B)
				\begin{scriptsize}
					\draw[->](\xmin,0)--(\xmax,0); \draw(\xmax-0.1,0) node[below]{$x$};
					\draw[->](0,\ymin)--(0,\ymax); \draw(0,\ymax-0.2) node[right]{$y$};
					\draw node [below left]{$O$} 
					(A) node [left]{$-a$} 
					(B) node [above]{$a$};
				\end{scriptsize}
			\end{tikzpicture}
		}	
	}
\end{ex}

\begin{ex}%[0D4K4-2]
	Tìm giá trị của số thực $m$ sao cho miền nghiệm của hệ bất phương trình $\heva{& x+my\le 2\\& x\ge 0\\& y\ge 0}$ là một tam giác có diện tích bằng $4$.
	\choice
	{$m=2$}
	{$m=4$}
	{$m=\dfrac{1}{4}$}
	{\True $m=\dfrac{1}{2}$}
	\loigiai{
		\immini{
			
			Diện tích $S=\dfrac{1}{2}\cdot 2\cdot \dfrac{2}{m}=4$.\\ Do đó $m=\dfrac{1}{2}$.
		}
		{
			\begin{tikzpicture}[scale=0.6, font=\footnotesize, line join=round, line cap=round, >=stealth]
				\def\xmin{-1} \def\xmax{3}
				\def\ymin{-1} \def\ymax{5}
				\clip(\xmin,\ymin) rectangle (\xmax,\ymax);
				\tkzDefPoints{\xmax/\ymax/A1,\xmin/\ymax/A2,
					\xmin/\ymin/A3,\xmax/\ymin/A4}
				\fill[pattern=north east lines,pattern color=black!60] (A1)--(A2)--(A3)--(A4)--cycle;
				\tkzDefPoints{0/0/O,0/4/A,2/0/B}
				\fill[color=white] (O)--(A)--(B)--cycle;
				\draw[domain=-1:3] plot(\x,{4-2*(\x)});
				\tkzDrawPoints(A,B)
				\begin{scriptsize}
					\draw[->](\xmin,0)--(\xmax,0); \draw(\xmax-0.1,0) node[below]{$x$};
					\draw[->](0,\ymin)--(0,\ymax); \draw(0,\ymax-0.2) node[right]{$y$};
					\draw node [below left]{$O$} 
					(A) node [left]{$\dfrac{2}{m}$} 
					(B) node [above]{$2$};
				\end{scriptsize}
			\end{tikzpicture}
		}	
	}
\end{ex}

\begin{ex}%[0D4K4-2]
	Tìm giá trị của số thực $m$ sao cho miền nghiệm của hệ bất phương trình $\heva{& x\ge 0\\& x\le 2\\& y\le -1\\& y\ge m}$ có chu vi bằng $8$.
	\choice
	{\True $m=-3$}
	{$m=2$}
	{$m=3$}
	{$m=-2$}
	\loigiai{
		\immini{
			Từ giả thiết suy ra $m<-1$ hay $-1-m>0$.\\
			Chu vi $P=2(-1-m+2)=8$.\\ Do đó $m=-3$.
		}
		{
			\begin{tikzpicture}[scale=0.6, font=\footnotesize, line join=round, line cap=round, >=stealth]
				\def\xmin{-1} \def\xmax{3}
				\def\ymin{-4} \def\ymax{1}
				\clip(\xmin,\ymin) rectangle (\xmax,\ymax);
				\tkzDefPoints{\xmax/\ymax/A1,\xmin/\ymax/A2,
					\xmin/\ymin/A3,\xmax/\ymin/A4}
				\fill[pattern=north east lines,pattern color=black!60] (A1)--(A2)--(A3)--(A4)--cycle;
				\tkzDefPoints{0/-1/A,0/-3/B,2/-3/C,2/-1/D}
				\fill[color=white] (A)--(B)--(C)--(D)--cycle;
				\draw (\xmin,-1)--(\xmax,-1) (\xmin,-3)--(\xmax,-3) (2,\ymin)--(2,\ymax);
				\tkzDrawPoints(A,B,C,D)
				\begin{scriptsize}
					\draw[->](\xmin,0)--(\xmax,0); \draw(\xmax-0.1,0) node[below]{$x$};
					\draw[->](0,\ymin)--(0,\ymax); \draw(0,\ymax-0.2) node[right]{$y$};
					\draw node [below left]{$O$} 
					(A) node [below left]{$-1$} 
					(0,-3) node [below left]{$m$} 				
					(2,0) node [above left]{$2$};
				\end{scriptsize}
			\end{tikzpicture}
		}	
	}
\end{ex}

\begin{ex}%[0D4K4-2]
	Tìm giá trị của số thực dương $m$ sao cho miền nghiệm của hệ bất phương trình $\heva{& x\ge 0\\& y\ge 0\\& 2x+3y\le 12\\& mx+y\ge 2}$ có diện tích bằng $8$.
	\choice
	{$m=2$}
	{$m=3$}
	{$m=\dfrac{1}{3}$}
	{\True $m=\dfrac{1}{2}$}
	\loigiai{
		\immini{
			Diện tích cần tìm $S_{ABCD}=S_{OBC}-S_{OAD}$.\\
			Do đó $S_{OAD}=S_{OBC}-S_{ABCD}=12-8=4=\dfrac{1}{2}.2.\dfrac{2}{m}$.\\
			Suy ra $m=\dfrac{1}{2}$.
		}
		{
			\begin{tikzpicture}[scale=0.6, font=\footnotesize, line join=round, line cap=round, >=stealth]
				\def\xmin{-2} \def\xmax{8}
				\def\ymin{-2} \def\ymax{5}
				\clip(\xmin,\ymin) rectangle (\xmax,\ymax);
				\tkzDefPoints{\xmax/\ymax/A1,\xmin/\ymax/A2,
					\xmin/\ymin/A3,\xmax/\ymin/A4}
				\fill[pattern=north east lines,pattern color=black!60] (A1)--(A2)--(A3)--(A4)--cycle;
				\tkzDefPoints{0/2/A,0/4/B,6/0/C,4/0/D}
				\fill[color=white] (A)--(B)--(C)--(D)--cycle;
				\draw[domain=-2:8] plot(\x,{4-2*(\x)/3}) plot(\x,{2-(\x)/2});
				\tkzDrawPoints(A,B,C,D)
				\begin{scriptsize}
					\draw[->](\xmin,0)--(\xmax,0); \draw(\xmax-0.1,0) node[below]{$x$};
					\draw[->](0,\ymin)--(0,\ymax); \draw(0,\ymax-0.2) node[right]{$y$};
					\draw node [below left]{$O$} 
					(A) node [below left]{$A(0;2)$} 
					(B) node [below left]{$B(0;4)$}
					(C) node [above right]{$C(6;0)$} 
					(D) node [below left]{$D(\dfrac{2}{m};0)$};;
				\end{scriptsize}
			\end{tikzpicture}
		}	
	}
\end{ex}

\begin{ex}%[0D4K4-3]
	Ngoài giờ học, bạn Nam làm thêm việc phụ bán cơm được $15$ nghìn đồng/một giờ và phụ bán tạp hóa được $10$ nghìn đồng/một giờ. Nam không thể làm thêm việc nhiều hơn $15$ giờ mỗi tuần. Gọi $x$, $y$ lần lượt là số giờ phụ bán cơm và phụ bán tạp hóa. Hệ bất phương trình nào sau đây xác định số giờ để làm mỗi việc nếu Nam muốn kiếm được ít nhất $100$ nghìn đồng mỗi tuần?
	\def\dotEX{}
	\choice
	{$\heva{& x+y\ge 15\\& 15x+10y\ge 100.}$}
	{$\heva{& x+y\le 15\\& 15x+10y>100.}$}
	{\True $\heva{& x+y\le 15\\& 15x+10y\ge 100.}$}
	{$\heva{& x+y>15\\& 15x+10y<100.}$}
	\loigiai{
		Gọi $x$, $y$ lần lượt là số giờ phụ bán cơm và phụ bán tạp hóa, tổng số giờ này không được nhiều hơn $15$ giờ nên $x+y\le 15$.\\
		Số tiền kiếm được sau $x$ giờ phục vụ cơm là $15x$.\\
		Số tiền kiếm được sau $y$ giờ bán tạp hóa là $10y$.\\
		Để Nam kiếm được ít nhất 100 nghìn đồng mỗi tuần thì $15x+10y\ge 100.$\\
		Vậy ta có hệ $\heva{& x+y\le 15\\& 15x+10y\ge 100.}$
	}
\end{ex}

\begin{ex}%[0D4K4-3]
	Để trở thành một thành viên của ban nhạc thì một sinh viên phải đạt điểm trung bình các môn học ít nhất là $7{,}0$ và phải có tối thiểu $5$ lần thực hành sau giờ học. Gọi $x$ là điểm trung bình các môn học và $y$ là số lần thực hành sau giờ học, hãy chọn hệ bất phương trình thể hiện tốt nhất tình huống này.
	\def\dotEX{}
	\choice
	{\True $\heva{& x\ge 7 \\ & y\ge 5.}$}
	{$\heva{& x\le 7 \\ & y\le 5.}$}
	{$\heva{& x<7 \\ & y<5.}$}
	{$\heva{& x>7 \\ & y>5.}$}
	\loigiai{
		Theo đề điểm trung bình các môn học ít nhất là $7,0$, tức là $x\ge 7$.\\
		Học sinh phải có tối thiểu $5$ lần thực hành sau giờ học, tức là $y\ge 5.$	\\
		Vậy ta có hệ $\heva{& x\ge 7 \\ & y\ge 5.}$
	}
\end{ex}

\begin{ex}%[0D4K4-4]
	Cho hệ bất phương trình $\heva{&x-y>0\\&2x+7y<0}$ có tập nghiệm $S$. Chọn khẳng định đúng trong các khẳng định sau.
	\choice
	{\True $(1;-1)\in S$}
	{$(1;-\dfrac{1}{2})\notin S$}
	{$(4;-1)\in S$}
	{$(-\dfrac{1}{2};-\dfrac{2}{7})\in S$}
	\loigiai{Bằng cách thay từng cặp giá trị vào hệ bất phương trình ta thấy chỉ có $(1;-1)$ và $(1;-\dfrac{1}{2})$ thỏa mãn. Vậy $(1;-1)\in S$ là đúng.}
\end{ex}

\begin{ex}%[0D4K4-4]
	Điểm $A\left (0;\dfrac{5}{3}\right )$ luôn thuộc miền nghiệm của bất phương trình nào trong các bất phương trình dưới đây (với $m$ là tham số thực)?
	\choice
	{\True $(m^2-4)x+3y-5\leq 0$}
	{$(m^2-4)x+3y-5> 0$}
	{$(m^2-4)x+3y-5< 0$}
	{$(m^2-4)x+3y+7\leq 0$}
	\loigiai{Thay điểm $A\left (0;\dfrac{5}{3}\right )$ vào bất phương trình $(m^2-4)x+3y-5\leq 0$, ta có
		$$(m^2-4) \cdot 0+3 \cdot \dfrac{5}{3} -5 = 0 \leq 0 \text{ (đúng).}$$
		Thay điểm $A$ vào lần lượt các bất phương trình ở các phương án còn lại, ta thấy không thỏa mãn.
	}
\end{ex}

\begin{ex}%[0D4K4-4]
	Hình vẽ dưới đây là biểu diễn hình học tập nghiệm của hệ bất phương trình nào? (với miền nghiệm là miền \textbf{không} gạch sọc và chứa bờ)
	\begin{center}
		\begin{tikzpicture}[scale=1, font=\footnotesize, line join=round, line cap=round, >=stealth]
			\clip(-4,-2.5) rectangle (4,4);
			\def\xmin{-4} \def\xmax{4}
			\def\ymin{-4} \def\ymax{4}
			\tkzDefPoints{\xmax/\ymax/A,\xmin/\ymax/B,\xmin/\ymin/C,\xmax/\ymin/D}
			%Nhập hệ số a,b,c đt d1: x+y-1=0-----------Đường thẳng MN
			\def\hsa{3} \def\hsb{4} \def\hsc{-8}
			\ifnum\hsb=0{
				\tkzDefPoint(-\hsc/ \hsa,\ymin){M};
				\tkzDefPoint(-\hsc/ \hsa,\ymax){N};}
			\else{ 
				\tkzDefPoint(\xmin,-\hsa*\xmin/\hsb-\hsc/\hsb){M};
				\tkzDefPoint(\xmax,-\hsa*\xmax/\hsb-\hsc/\hsb){N};}\fi
			%Nhập hệ số a,b,c đt d2: 2x-3y-1=0---------Đường thẳng PQ
			\def\hsa{5} \def\hsb{-12} \def\hsc{-3}
			\ifnum\hsb=0{
				\tkzDefPoint(-\hsc/ \hsa,\ymin){P};
				\tkzDefPoint(-\hsc/ \hsa,\ymax){Q};}
			\else{ 
				\tkzDefPoint(\xmin,-\hsa*\xmin/\hsb-\hsc/\hsb){P};
				\tkzDefPoint(\xmax,-\hsa*\xmax/\hsb-\hsc/\hsb){Q};}\fi
			\tkzInterLL(M,N)(P,Q) \tkzGetPoint{I}
			\tkzInterLL(M,N)(A,B) \tkzGetPoint{M'}
			\fill[pattern=north east lines,pattern color=blue!30] (C) rectangle (A);
			%\fill[white] (P)--(I)--(N)--(A)--(B)--cycle;
			\fill[white] (Q)--(I)--(M)--(A)--cycle;
			\tkzDrawSegment(M,N)
			\tkzDrawSegment(P,Q)
			\draw[->](\xmin,0)--(\xmax,0); \draw(\xmax-0.1,0) node[below]{$x$};
			\draw[->](0,\ymin)--(0,\ymax); \draw(0,\ymax-0.2) node[right]{$y$};
			\begin{scriptsize}
				\foreach \x in {-3,-2,-1,1,2,3} {
					\draw (\x,0.05) -- ++(0,-0.1) node [below] {$\x$};
					\draw (0.05,\x) -- ++(-0.1,0) node [left] {$\x$}; }
				\draw node [above left]{$O$};
				\draw (2.67,0) node [below] {$\dfrac{8}{3}$};
				\draw (0,-0.25) node [below left] {$-0,25$};
			\end{scriptsize}
			\draw [dashed] (0,1)--(3,1)--(3,0);
		\end{tikzpicture}
	\end{center}
	\def\dotEX{}
	\choice
	{\True $\heva{&3x+4y-8 \geq 0 \\ &5x-12y-3\leq 0.}$}
	{$\heva{&3x+4y-8 \leq 0 \\ &5x-12y-3\leq 0.}$}
	{$\heva{&3x+4y-8 \geq 0 \\ &5x-12y-3 \geq 0.}$}
	{$\heva{&3x+4y-3 \geq 0 \\ &5x-12y-8 \leq 0.}$}
	\loigiai{Xét các điểm $A(0; 2)$, $B (3; 1)$ và $C \left(0; - \dfrac{1}{4}\right)$ thuộc bờ. \\
		Điểm $B (3; 1)$ không thỏa mãn bất phương trình $3x+4y-8 \leq 0$ nên loại $\heva{&3x+4y-8 \leq 0 \\ &5x-12y-3\leq 0.}$ \\ 
		Điểm $A (0; 2)$ không thỏa mãn bất phương trình $5x - 12 y - 3 \geq 0$ nên loại $\heva{&3x+4y-8 \geq 0 \\ &5x-12y-3 \geq 0.}$\\
		Điểm $C \left(0; - \dfrac{1}{4}\right)$ không thỏa mãn bất phương trình $3x + 4y - 3 \geq 0$ nên loại $\heva{&3x+4y-3 \geq 0 \\ &5x-12y-8 \leq 0.}$
	}
\end{ex}

\begin{ex}%[0D4K4-4]
	\immini{Phần mặt phẳng không bị gạch, kể cả phần biên của nó trên đường thẳng $y=0$ trong hình vẽ bên là miền nghiệm của hệ bất phương trình nào?
		\def\dotEX{}
		\choice
		{$ \heva{&y\leq0\\ &2x+y>1.}$}
		{\True $ \heva{&x+y<2\\&y\geq0.}$}
		{$\heva{&2x-2y>6\\&2x+y\geq1.}$}
		{$ \heva{&y\leq0\\&x+y<1.}$}
	}{\begin{tikzpicture}[scale=1, font=\footnotesize, line join=round, line cap=round, >=stealth]
			\draw[->] (-1,0) -- (3.5,0)node[above]{$x$};
			\foreach \x in {1,2,3}
			\draw[shift={(\x,0)},color=black] (0pt,2pt) -- (0pt,-2pt) node[below] {\footnotesize $\x$};
			\draw[->,color=black] (0,-1) -- (0,3.5)node[right]{$y$};
			\foreach \y in {1,2,3}
			\draw[shift={(0,\y)},color=black] (2pt,0pt) -- (-2pt,0pt) node[left] {\footnotesize $\y$};
			\node[below left] at (0,0) {$O$};
			\clip(-1,-1) rectangle (3.5,3.5);
			\fill[pattern=north east lines] (-1,3) -- (3,3) -- (3,-1) -- cycle;
			\draw[line width=1.2pt,smooth,samples=100,domain=-1:4] plot(\x,{2-(\x)});
			\fill[pattern=north east lines] (-1,0)--(-1,-1) --(3,-1)--(3,0)-- cycle;
		\end{tikzpicture}
	}
	\loigiai{Phần không bị gạch nằm phía trên trục hoành nên nó là miền nghiệm của bất phương trình $y \geq 0$ $(1)$. \\ 
		Điểm $A (0; 1)$ thỏa mãn bất phương trình $x + y < 2$ nên miền không bị gạch chính là miền nghiệm của bất phương trình $x +y < 2$ $(2)$. \\
		Từ $(1)$ và $(2)$ suy ra phần mặt phẳng không bị gạch, kể cả phần biên của nó trên đường thẳng $y=0$ trong hình vẽ bên là miền nghiệm của hệ bất phương trình $ \heva{&x+y<2\\&y\geq0.}$
	}
\end{ex}

\begin{ex}%[0D4K4-4]
	\immini{Phần mặt phẳng không bị gạch, kể cả phần biên của nó trên đường thẳng $d$ trong hình vẽ bên là miền nghiệm của hệ bất phương trình nào?
		\def\dotEX{}
		\choice
		{$ \heva{&2x+3y\leq6\\ &2x+y>2.}$}
		{$ \heva{&x-2y<1\\&3x+2y\leq3.}$}
		{\True $\heva{&2x+3y<6\\&2x+y\leq2.}$}
		{$ \heva{&2x-3y\leq6\\&2x+y<1.}$}
	}{\begin{tikzpicture}[scale=1, font=\footnotesize, line join=round, line cap=round, >=stealth]
			\draw[->] (-1,0) -- (4.3,0)node[above]{$x$};
			\foreach \x in {1,2,3,4}
			\draw[shift={(\x,0)},color=black] (0pt,2pt) -- (0pt,-2pt) node[below] {\footnotesize $\x$};
			\draw[->,color=black] (0,-1) -- (0,4)node[right]{$y$};
			\foreach \y in {1,2,3}
			\draw[shift={(0,\y)},color=black] (2pt,0pt) -- (-2pt,0pt) node[left] {\footnotesize $\y$};
			\node[below left] at (0,0) {$O$};
			\node[below left] at (1,-0.5) {$d$};
			\node at (0.5,0.6) {$d$};
			\clip(-1,-1) rectangle (4,3.5);
			\fill[pattern=north east lines] (-1,2.66667) -- (-1,4) -- (4,4) -- (4,-0.6667) -- (3,0) -- cycle;
			\draw[line width=1.2pt,smooth,samples=100,domain=-1:4] plot(\x,{2-0.666667*(\x)});
			\fill[pattern=north east lines] (-0,75,3.5)--(0,2) --(0.5,1)--(1,0)--(1.5,-1)--(4,-1)--(4,0)--(4,4)-- cycle;
			\draw[line width=1.2pt,smooth,samples=100,domain=-1:4] plot(\x,{2-2*(\x)});
		\end{tikzpicture}
	}	
	\loigiai{Đường thẳng $d$ có phương trình $2x + y = 2$.\\ 
		Đường thẳng $\Delta$ đi qua hai điểm $(3; 0)$ và $(0; 2)$ có phương trình là $2x + 3y = 6$. \\
		Tại điểm $A (0; 1)$, ta có $2. 0 + 1 = 1 < 2$, suy ra điểm $A$ thuộc miền nghiệm của bất phương trình $2x + y < 2$ $(1)$. Tương tự, ta cũng kiểm tra được rằng điểm $A$ cũng thuộc miền nghiệm của bất phương trình $2x + 3y < 6$ $(2)$. \\
		Vậy phần mặt phẳng không bị gạch, kể cả phần biên của nó trên đường thẳng $d$ trong hình vẽ bên là miền nghiệm của hệ bất phương trình $\heva{&2x+3y<6\\&2x+y\leq2.}$
	}
\end{ex}

\begin{ex}%[0D4K4-4]
	\immini{Cho hệ bất phương trình $\heva{& x \geq -2\\& y \geq -2\\ & x+y<2.}$ Biết rằng  $A$, $B$, $C$ là giao điểm của hai trong ba đường thẳng $x=-2$, $y=-2$, $x+y=2$ \textit{(được cho như hình vẽ)}. Khẳng định nào dưới đây là đúng?}
	{
		\begin{tikzpicture}[scale=0.5, font=\footnotesize, line join=round, line cap=round, >=stealth]
			\clip(-3.5,-4) rectangle (5,5);
			\fill[pattern=north east lines] (-5,-4) -- (-5,5) -- (5,5) -- (5,-4) -- cycle;
			\fill[white] (-2,-2) -- (-2,4) -- (4,-2) -- cycle;
			\draw[violet,line width=2pt,samples=100] (-2,-2) -- (-2,4) -- (4,-2)--(-2,-2);
			\draw[->] (-3.5,0) -- (5,0); \draw (4.7,0)node[above]{$x$};
			\foreach \x in {-2,2,4}
			\draw[shift={(\x,0)},color=black] (0pt,2pt) -- (0pt,-2pt) node[below left] {\footnotesize $\x$};
			\draw[->,color=black] (0,-4.2) -- (0,5); \draw (0,4.7) node[left]{$y$};
			\foreach \y in {-2,2,4}
			\draw[shift={(0,\y)},color=black] (2pt,0pt) -- (-2pt,0pt) node[above right] {\footnotesize $\y$};
			\node[above left] at (0,0){$O$};
			\node[below left] at (-2,4){$A$};
			\node[above right] at (4,-2){$B$};
			\node[below left] at (-2,-2){$C$};
			\draw[dashed] (4,0)--(4,-2);
			\draw[dashed] (-2,4)--(0,4);
			\draw[line width=1.2pt,smooth,samples=100,domain=-3:4] (-2,-4)--(-2,5);
			\draw[line width=1.2pt,smooth,samples=100,domain=-5:5] plot(\x,{-2-0*(\x)});
			\draw[line width=1.2pt,smooth,samples=100,domain=-3:5] plot(\x,{2-(\x)});
		\end{tikzpicture}
	}
	\choice
	{Miền nghiệm của hệ bất phương trình là miền tam giác $ABC$ bao gồm cả các cạnh $AB$, $AC$, $BC$ }
	{\True Miền nghiệm của hệ bất phương trình là miền tam giác $ABC$ bao gồm các cạnh  $AC$, $BC$ ngoại trừ điểm $A$, điểm $B$}
	{Miền nghiệm của hệ bất phương trình là miền tam giác $ABC$ bao gồm các cạnh $AB$, $AC$, $BC$ ngoại trừ điểm $A$, điểm $B$, điểm $C$}
	{Miền nghiệm của hệ bất phương trình là miền tam giác $ABC$ bao gồm các cạnh  $AB$, $BC$ ngoại trừ điểm $A$, điểm $C$}
	
	\loigiai{Ta thấy điểm $O$, điểm $C$, cạnh $AC$ (ngoại trừ điểm $A$), cạnh $BC$ (ngoại trừ điểm $B$) thuộc miền nghiệm của cả ba bất phương trình. Đường thẳng $x+y=2$ (chứa cạnh $AB$) không thuộc miền nghiệm của bất phương trình $x+y<2$. Vậy \textit{miền nghiệm của hệ bất phương trình trên là miền tam giác $ABC$ bao gồm các cạnh  $AC$, $BC$ ngoại trừ điểm $A$, điểm $B$.}
	}
\end{ex}

\begin{ex}%[0D4K4-4]
	\immini{Miền không bị gạch chéo (kể cả đường thẳng $d_1$ và $d_2$) là miền nghiệm của hệ bất phương trình nào sau đây? 
		\def\dotEX{}
		\choice
		{$\heva{x-y&\leq -2\\-2x-y&\geq -2.}$}
		{$\heva{x+y&\leq 2\\-2x-y&\geq -2.}$}
		{\True $\heva{x+y&\geq -2\\-2x+y&\geq -2.}$}
		{$\heva{-x-y&\leq -2\\2x-y&\geq -2.}$}}
	{\begin{tikzpicture}[scale=0.7, font=\footnotesize, line join=round, line cap=round, >=stealth]
			\clip(-3,-4) rectangle (3,3);
			\fill[pattern=north west lines] (-3,-4) rectangle (3,3);
			\fill[white] (0,-2) -- (-3,1) -- (-3,3) -- (2.5,3)  -- cycle;
			\draw[->] (-3,0) -- (3,0)node[above left]{$x$};
			\foreach \x in {-2,-1,1,2,3}
			\draw[shift={(\x,0)},color=black] (0pt,2pt) -- (0pt,-2pt) node[below] {\footnotesize $\x$};
			\draw[->,color=black] (0,-4) -- (0,3)node[below right]{$y$};
			\foreach \y in {-3,-2,-1,1,2}
			\draw[shift={(0,\y)},color=black] (2pt,0pt) -- (-2pt,0pt) node[left] {\footnotesize $\y$};
			\node[below left] at (0,0){$O$};
			\node[left] at (2,2){$d_2$};
			\node[right] at (-3,1){$d_1$};
			\draw[line width=1.2pt,smooth,samples=100,domain=-3:3] plot(\x,{-2-(\x)});
			\draw[line width=1.2pt,smooth,samples=100,domain=-3:3] plot(\x,{2*(\x)-2});
	\end{tikzpicture}}
	\loigiai{Đường thẳng $d_1$ đi qua hai điểm $(-2; 0)$ và $(0; -2)$ nên có phương trình là $x+y=-2$. Đường thẳng $d_2$ đi qua hai điểm $(1; 0)$ và $(0; -2)$ nên có phương trình là $-2x+y=-2$. \\
		Điểm $O (0; 0)$ thỏa mãn hệ bất phương trình $\heva{x+y&\geq -2\\-2x+y&\geq -2}$ nên phần không bị gạch chính là miền nghiệm của hệ bất phương trình trên. 
	}
\end{ex}

\begin{ex}%[0D4K4-4]
	\immini{
		Miền tam giác không bị gạch kể cả $3$ cạnh của nó trong hình bên là miền nghiệm của hệ bất phương trình nào?
		\def\dotEX{}
		\choice
		{$\heva{&y\geq0\\&5x-4y\leq10\\&5x+4y\leq10.}$}
		{$\heva{&x\geq0\\&4x-5y\leq10\\&5x+4y\leq10.}$}
		{\True $\heva{&x\geq0\\&5x-4y\leq10\\&4x+5y\leq10.}$}
		{$\heva{&x>0\\&5x-4y\leq10\\&4x+5y\leq10.}$}
	}{\begin{tikzpicture}[scale=1, font=\footnotesize, line join=round, line cap=round, >=stealth]
			\draw[->] (-1,0) -- (4.3,0)node[above]{$x$};
			\foreach \x in {1,2,3,4}
			\draw[shift={(\x,0)},color=black] (0pt,2pt) -- (0pt,-2pt) node[below] {\footnotesize $\x$};
			\draw[->,color=black] (0,-3) -- (0,4.3)node[right]{$y$};
			\foreach \y in {-2,-1,1,2,3,4}
			\draw[shift={(0,\y)},color=black] (2pt,0pt) -- (-2pt,0pt) node[left] {\footnotesize $\y$};
			\node[below left] at (0,0) {$O$};
			\clip(-1,-3) rectangle (4,4);
			\fill[pattern=north east lines] (-0.4,-3) -- (4,-3) -- (4,2.5)-- cycle;
			\draw[line width=1.2pt,smooth,samples=100,domain=-1:4] plot(\x,{-2.5+1.25*(\x)});
			\fill[pattern=north east lines] (-1,2.8)--(-1,4) --(4,4)--(4,-1.2)-- cycle;
			\draw[line width=1.2pt,smooth,samples=100,domain=-1:4] plot(\x,{2-0.8*(\x)});
			\fill[pattern=north east lines] (0,4)--(-1,4) --(-1,-3)--(0,-3)-- cycle;
		\end{tikzpicture}
	}
	\loigiai{Miền không bị gạch nằm bên phải trục tung nên là miền nghiệm của bất phương trình $x \geq 0$. \\
		Gọi $A (x_0; y_0)$ là một đỉnh của tam giác (điểm $A$ không nằm trên trục $Oy$). Dựa vào hình vẽ ta thấy $x_0 > 2$, $y_0 > 0$. Từ đó suy ra $5 x_0 + 4 y_0 > 10$. Vậy điểm $A$ không thuộc miền nghiệm của bất phương trình $5x + 4y \leq 10$. \\
		Vậy miền tam giác không bị gạch kể cả ba cạnh của nó trong hình bên là miền nghiệm của hệ bất phương trình $\heva{&x\geq0\\&5x-4y\leq10\\&4x+5y\leq10.}$ 
	}
\end{ex}

\begin{ex}%[0D4K4-4]
	\immini{Miền tam giác $ABC$ kể cả ba cạnh là miền nghiệm của hệ bất phương trình nào trong bốn hệ sau?
		\def\dotEX{}
		\choice
		{$\heva{y&\geq 0\\2x-3y&\geq 6\\x+y&\leq 3.} $}
		{\True $\heva{x&\geq 0\\-2x+3y&\geq -6\\x+y&\leq 3.} $}
		{$\heva{x&\geq 0\\-2x+3y&\leq -6\\x+y&\leq 3.} $}
		{$\heva{y&\geq 0\\2x-3y&\leq -6\\x+y&\leq 3.} $}
	}
	{\begin{tikzpicture}[scale=0.7, font=\footnotesize, line join=round, line cap=round, >=stealth,xscale=1]
			\fill[pattern=north east lines] (-2,-3) -- (-2,4) -- (4,4) -- (4,-3) -- cycle;
			\fill[white] (0,-2) -- (3,0) -- (0,3) -- cycle;
			\draw[violet,line width=2pt,samples=100] (0,-2) -- (3,0) -- (0,3)--(0,-2);
			\draw[->] (-2,0) -- (4.3,0)node[above]{$x$};
			\foreach \x in {3}
			\draw[shift={(\x,0)},color=black] (0pt,2pt) -- (0pt,-2pt) node[below] {\footnotesize $\x$};
			\draw[->,color=black] (0,-3) -- (0,4.3)node[right]{$y$};
			\foreach \y in {-2,3}
			\draw[shift={(0,\y)},color=black] (2pt,0pt) -- (-2pt,0pt) node[right] {\footnotesize $\y$};
			\node[above left] at (0,0){$O$};
			\node[below left] at (0,3){$A$};
			\node[above] at (3,0){$B$};
			\node[above left] at (0,-2){$C$};
			\clip(-2,-3) rectangle (4,4);
			\draw[line width=1.2pt,smooth,samples=100,domain=-3:4] plot(\x,{3-(\x)});
			\draw[line width=1.2pt,smooth,samples=100,domain=-3:4] plot(\x,{2/3*(\x)-2});
	\end{tikzpicture}}
	\loigiai{Ta thấy cạnh $AC$ thuộc đường thẳng $x=0$ và miền tam giác $ABC$ thuộc miền nghiệm của bất phương trình $x\geq 0$.\\
		Cạnh $AB$ thuộc đường thẳng $x+y=3$ và miền tam giác $ABC$ thuộc miền nghiệm của bất phương trình $x+y\leq 3$.\\
		Cạnh $BC$ thuộc đường thẳng $-2x+3y=-6$ và miền tam giác $ABC$ thuộc miền nghiệm của bất phương trình $-2x+3y\geq -6$.\\
		Vậy miền tam giác $ABC$ kể cả ba cạnh là miền nghiệm của hệ bất phương trình:  $\heva{x&\geq 0\\-2x+3y&\geq -6\\x+y&\leq 3.} $}
\end{ex}

\begin{ex}%[0D4K4-4]
	Phần mặt phẳng không bị gạch, kể cả phần biên của nó nằm trên đường thẳng $d$ trong hình vẽ nào sau đây là miền nghiệm của hệ bất phương trình $\heva{&y<0\\&2x+y\leq4.}$
	\choice
	{\begin{tikzpicture}[scale=.7, font=\footnotesize, line join=round, line cap=round, >=stealth]
			\draw[->] (-1,0) -- (4.3,0)node[above]{$x$};
			\foreach \x in {1,2,3,4}
			\draw[shift={(\x,0)},color=black] (0pt,2pt) -- (0pt,-2pt) node[below] {\footnotesize $\x$};
			\draw[->,color=black] (0,-1) -- (0,4.3)node[right]{$y$};
			\foreach \y in {1,2,3,4}
			\draw[shift={(0,\y)},color=black] (2pt,0pt) -- (-2pt,0pt) node[left]{\footnotesize $\y$};
			\node[below left] at (0,0){$O$};
			\node[below left] at (0.5,-0.2){$d$};
			\clip(-1,-1) rectangle (4.3,4.3);
			\fill[pattern=north east lines] (-0.25,-1) -- (-1,-1) -- (-1,4.3) -- (2.4,4.3) -- cycle;
			\draw[line width=1.2pt,smooth,samples=100,domain=-1:4] plot(\x,{-0.5+2*(\x)});
			\fill[pattern=north east lines] (-1,0) -- (-1,4.3) -- (4,4.3) -- (4,0)-- cycle;
	\end{tikzpicture}}
	{\True \begin{tikzpicture}[scale=.7, font=\footnotesize, line join=round, line cap=round, >=stealth]
			\draw[->] (-1,0) -- (4.3,0)node[above]{$x$};
			\foreach \x in {1,2,3,4}
			\draw[shift={(\x,0)},color=black] (0pt,2pt) -- (0pt,-2pt) node[below] {\footnotesize $\x$};
			\draw[->,color=black] (0,-1) -- (0,4.3)node[right]{$y$};
			\foreach \y in {1,2,3,4}
			\draw[shift={(0,\y)},color=black] (2pt,0pt) -- (-2pt,0pt) node[left]{\footnotesize $\y$};
			\node[below left] at (0,0){$O$};
			\node[below] at (2,-0.5){$d$};
			\clip(-1,-1) rectangle (4,4.3);
			\fill[pattern=north east lines] (-0.15,4.3) -- (4,4.3) -- (4,-1) -- (2.5,-1)-- cycle;
			\draw[line width=1.2pt,smooth,samples=100,domain=-1:4] plot(\x,{4-2*(\x)});
			\fill[pattern=north east lines] (-1,0) -- (-1,4.3) -- (4,4.3) -- (4,0)-- cycle;
	\end{tikzpicture}}
	{\begin{tikzpicture}[scale=.7, font=\footnotesize, line join=round, line cap=round, >=stealth]
			\draw[->] (-1,0) -- (4.3,0)node[above]{$x$};
			\foreach \x in {1,2,3,4}
			\draw[shift={(\x,0)},color=black] (0pt,2pt) -- (0pt,-2pt) node[below] {\footnotesize $\x$};
			\draw[->,color=black] (0,-1) -- (0,4.3)node[right]{$y$};
			\foreach \y in {1,2,3,4}
			\draw[shift={(0,\y)},color=black] (2pt,0pt) -- (-2pt,0pt) node[left]{\footnotesize $\y$};
			\node[below left] at (0,0){$O$};
			\node[below left] at (1,2){$d$};
			\clip(-1,-1) rectangle (4,4.3);
			\fill[pattern=north east lines] (-0.15,4.3) -- (4,4.3) -- (4,-1) -- (2.5,-1)-- cycle;
			\draw[line width=1.2pt,smooth,samples=100,domain=-1:4] plot(\x,{4-2*(\x)});
			\fill[pattern=north east lines] (-1,0) -- (-1,-1) -- (4,-1) -- (4,0)-- cycle;
	\end{tikzpicture}}
	{\begin{tikzpicture}[scale=.7, font=\footnotesize, line join=round, line cap=round, >=stealth]
			\draw[->] (-1,0) -- (4.3,0)node[above]{$x$};
			\foreach \x in {1,2,3,4}
			\draw[shift={(\x,0)},color=black] (0pt,2pt) -- (0pt,-2pt) node[below] {\footnotesize $\x$};
			\draw[->,color=black] (0,-1) -- (0,4.3)node[right]{$y$};
			\foreach \y in {1,2,3,4}
			\draw[shift={(0,\y)},color=black] (2pt,0pt) -- (-2pt,0pt) node[left]{\footnotesize $\y$};
			\node[below left] at (0,0){$O$};
			\node[below left] at (1.2,1){$d$};
			\clip(-1,-1) rectangle (4.3,4.3);
			\fill[pattern=north east lines] (-0.25,-1) -- (-1,-1) -- (-1,4.3) -- (2.4,4.3) -- cycle;
			\draw[line width=1.2pt,smooth,samples=100,domain=-1:4] plot(\x,{-0.5+2*(\x)});
			\fill[pattern=north east lines] (-1,0) -- (-1,-1) -- (4,-1) -- (4,0)-- cycle;
	\end{tikzpicture}}
	\loigiai{Miền nghiệm của bất phương trình $y<0$ nằm bên dưới trục hoành $(1)$. \\
		Đường thẳng đi qua hai điểm $(0; 4)$ và $(2; 0)$ có phương trình là $2x + y =4$ (đây chính là đường thẳng $d$) $(2)$. \\
		Từ $(1)$ và $(2)$ suy ra hình B biểu diễn miền nghiệm của hệ bất phương trình đã cho. 
	}
\end{ex}

\begin{ex}%[0D4K4-4]
	\immini{Hệ bất phương trình nào sau đây có miền nghiệm là phần mặt phẳng không bị gạch có hai bờ là hai đường thẳng $a$ và $b$ như hình bên?
		\def\dotEX{}		
		\choice
		{\True $\heva{& 2x+y\le 2\\& 2x+y\ge -2.}$}
		{$\heva{& 2x+y\le -2\\& 2x+y\ge 2.}$}
		{$\heva{& 2x-y\le 2\\& 2x-y\ge -2.}$}
		{$\heva{& 2x-y\le -2\\& 2x-y\ge 2.}$}
	}
	{
		\begin{tikzpicture}[scale=0.6, font=\footnotesize, line join=round, line cap=round, >=stealth]
			\def\xmin{-3} \def\xmax{3}
			\def\ymin{-3} \def\ymax{3}
			\clip(\xmin,\ymin) rectangle (\xmax,\ymax);
			\tkzDefPoints{\xmax/\ymax/A1,\xmin/\ymax/A2,\xmin/\ymin/A3,\xmax/\ymin/A4}
			\tkzDefPoints{-2.5/3/M,.5/-3/N, -.5/3/P,2.5/-3/Q}
			\fill[pattern=north east lines,pattern color=blue!60] (M)--(A2)--(A3)--(N)--cycle (P)--(Q)--(A4)--(A1)--cycle;
			\draw (M)--(N) (P)--(Q);
			\begin{scriptsize}
				\draw[->](\xmin,0)--(\xmax,0); \draw(\xmax-0.1,0) node[below]{$x$};
				\draw[->](0,\ymin)--(0,\ymax); \draw(0,\ymax-0.2) node[right]{$y$};
				\foreach \x in {-1,1}
				\draw (\x,0.05) -- ++(0,-0.1) node [above] {$\x$};
				\foreach \x in {-2,2}
				\draw (0.05,\x) -- ++(-0.1,0) node [left] {$\x$};
				\draw node [above right]{$O$}
				(1,-2.5) node [below left]{$a$}
				(1.8,-2.5) node [below right]{$b$};
			\end{scriptsize}
		\end{tikzpicture}
	}
	\loigiai{Đường thẳng $a$ đi qua hai điểm $(-1; 0)$ và $(0; -2)$ nên có phương trình là $2x + y =-2$. \\
		Đường thẳng $b$ đi qua hai điểm $(1; 0)$ và $(0; 2)$ nên có phương trình là $2x + y =2$. \\
		Điểm $O(0; 0)$ thuộc miền nghiệm của bất phương trình $2x +y \leq 2$ và $2x + y \geq -2$. \\
		Vậy miền không bị gạch là miền nghiệm của hệ bất phương trình $\heva{& 2x+y\le 2\\& 2x+y\ge -2.}$ 
	}
\end{ex}

\begin{ex}%[0D4K4-4]
	\immini{Hệ bất phương trình nào sau đây có miền nghiệm là phần mặt phẳng không bị gạch như hình bên (kể cả các điểm nằm trên hai đường thẳng $a$, $b$ và không thuộc miền bị gạch)?
		\def\dotEX{}
		\choice
		{$\heva{& 2x+y\le 2\\& -2x+y\ge 2.}$}
		{$\heva{& 2x+y\ge 2\\& -2x+y\ge 2.}$}
		{$\heva{& 2x+y\ge 2\\& -2x+y\ge -2.}$}
		{\True $\heva{& 2x+y\le 2\\& -2x+y\le 2.}$}
	}
	{
		\begin{tikzpicture}[scale=0.6, font=\footnotesize, line join=round, line cap=round, >=stealth]
			\def\xmin{-3} \def\xmax{3}
			\def\ymin{-2} \def\ymax{4}
			\clip(\xmin,\ymin) rectangle (\xmax,\ymax);
			\tkzDefPoints{\xmax/\ymax/A1,\xmin/\ymax/A2,\xmin/\ymin/A3,\xmax/\ymin/A4}
			\tkzDefPoints{-2.5/-3/M,2.5/-3/N,0/2/I}
			\fill[pattern=north east lines,pattern color=blue!60]   (M)--(A3)--(A2)--(A1)--(A4)--(N)--(I)--cycle;
			\draw[domain=-3:3] plot(\x,{2-2*(\x)}) plot(\x,{2+2*(\x)});
			\begin{scriptsize}
				\draw[->](\xmin,0)--(\xmax,0); \draw(\xmax-0.1,0) node[below]{$x$};
				\draw[->](0,\ymin)--(0,\ymax); \draw(0,\ymax-0.2) node[right]{$y$};
				\foreach \x in {-1,1}
				\draw (\x,0.05) -- ++(0,-0.1) node [below] {$\x$};
				\foreach \x in {-1,2}
				\draw (0.05,\x) -- ++(-0.1,0) node [left] {$\x$};
				\draw node [above right]{$O$}
				(-1.6,-1.7) node {$a$}
				(1.6,-1.7) node {$b$};
			\end{scriptsize}
		\end{tikzpicture}
	}
	\loigiai{Đường thẳng $a$ đi qua hai điểm $(-1; 0)$ và $(0; 2)$ nên có phương trình là $-2x + y =2$. \\
		Đường thẳng $b$ đi qua hai điểm $(1; 0)$ và $(0; 2)$ nên có phương trình là $2x + y =2$. \\
		Điểm $O(0; 0)$ thuộc miền nghiệm của bất phương trình $-2x +y \leq 2$ và $2x + y \leq 2$. \\
		Vậy miền không bị gạch là miền nghiệm của hệ bất phương trình $\heva{& 2x+y\le 2\\& -2x+y\le 2.}$}
\end{ex}

\begin{ex}%[0D4K4-4]
	\immini{Hệ bất phương trình nào sau đây có miền nghiệm là phần mặt phẳng không bị gạch như hình bên (kể cả các điểm nằm trên các đường thẳng $a$, $b$, $c$ và không thuộc miền bị gạch)?
		\def\dotEX{}
		\choice
		{$\heva{& 2x+y\ge 1\\& x-2y\ge 2\\& x\le 0.}$}
		{$\heva{& 2x-y\le 1\\& x-2y\ge 2\\& y\le 0.}$}
		{\True $\heva{& 2x+y\le 1\\& x-2y\ge 2\\& x\ge 0.}$}
		{$\heva{& 2x-y\le 1\\& x-2y\ge 2\\& x\ge 0.}$}
	}
	{
		\begin{tikzpicture}[scale=0.8, font=\footnotesize, line join=round, line cap=round, >=stealth]
			\def\xmin{-2} \def\xmax{3}
			\def\ymin{-3} \def\ymax{2}
			\clip(\xmin,\ymin) rectangle (\xmax,\ymax);
			\tkzDefPoints{\xmax/\ymax/A1,\xmin/\ymax/A2,\xmin/\ymin/A3,\xmax/\ymin/A4}
			\tkzDefPoints{0/-3/M,0/-1/N,2/-3/P}
			\tkzDefPoint(4/5,-3/5){I}
			\fill[pattern=north east lines,pattern color=blue!60]   (M)--(N)--(I)--(P)--(A4)--(A1)--(A2)--(A3)--cycle;
			\draw[domain=-3:3] plot(\x,{1-2*(\x)}) plot(\x,{(\x)/2-1});		
			\begin{scriptsize}
				\draw[->](\xmin,0)--(\xmax,0); \draw(\xmax-0.1,0) node[below]{$x$};
				\draw[->](0,\ymin)--(0,\ymax); \draw(0,\ymax-0.2) node[right]{$y$};
				\foreach \x in {-1,1,2}
				\draw (\x,0.05) -- ++(0,-0.1) node [below] {$\x$};
				\foreach \x in {-2,-1,1}
				\draw (0.05,\x) -- ++(-0.1,0) node [left] {$\x$};
				\draw node [below left]{$O$}
				(0.2,-2.5) node {$a$}
				(1.5,-2.5) node {$b$}
				(0.5,-1) node {$c$};
			\end{scriptsize}
		\end{tikzpicture}
	}
	\loigiai{Đường thẳng $a$ có phương trình $x=0$. \\
		Đường thẳng $b$ đi qua hai điểm $\left(\dfrac{1}{2}; 0 \right)$ và $(0; 1)$ nên có phương trình là $2x + y =1$. \\
		Đường thẳng $c$ đi qua hai điểm $\left(2; 0 \right)$ và $(0; -1)$ nên có phương trình là $x - 2y =2$. \\
		Điểm $(0; -2)$ thuộc miền nghiệm của bất phương trình $x \geq 0$, $2x +y \leq 1$ và $x - 2y \geq 2$. \\
		Vậy miền không bị gạch là miền nghiệm của hệ bất phương trình $\heva{& 2x+y\le 1\\ & x-2y\ge 2\\ & x\ge 0.}$
	}
\end{ex}

\begin{ex}%[0D4K4-4]
	Tìm tất cả các số thực $a$, $b$ sao cho miền nghiệm của hệ bất phương trình $\heva{& x\ge a\\& y<b}$ chứa điểm $M(-1;1)$.
	\choice
	{$a\ge -1$; $b\le 1$}
	{$a<-1$; $b\ge 1$}
	{\True $a\le -1$; $b>1$}
	{$a\le -1$; $b<1$}
	\loigiai{Để $M(-1;1)$ thuộc miền nghiệm của bất phương trình $\heva{& x \ge a \\& y<b}$ thì $a \leq -1$ và $b > 1$. 
	}
\end{ex}

\begin{ex}%[0D4K4-4]
	Tìm tất cả các giá trị của $m$ để đường thẳng $y=m$ có điểm chung với miền nghiệm của hệ bất phương trình $\heva{&x\geq -2\\&y\geq -2\\& x+y\leq 2.}$ 
	\choice
	{$m\geq -2$}
	{$m\leq 4$}
	{\True $-2\leq m\leq 4$}
	{ $-2<m<4$}
	\loigiai{Miền nghiệm của hệ bất phương trình đã cho là miền tam giác $ABC$ và các cạnh (như hình vẽ).\\
		\begin{center}
			\begin{tikzpicture}[scale=1, font=\footnotesize, line join=round, line cap=round, >=stealth]
				\fill[pattern=north east lines] (-5,-4) -- (-5,5) -- (5,5) -- (5,-4) -- cycle;
				\fill[white] (-2,-2) -- (-2,4) -- (4,-2) -- cycle;
				\draw[violet,line width=2pt,samples=100] (-2,-2) -- (-2,4) -- (4,-2)--(-2,-2);
				\draw[->] (-5.2,0) -- (5.3,0)node[above]{$x$};
				\foreach \x in {-2,2,4}
				\draw[shift={(\x,0)},color=black] (0pt,2pt) -- (0pt,-2pt) node[below left] {\footnotesize $\x$};
				\draw[->,color=black] (0,-4.2) -- (0,5.3)node[right]{$y$};
				\foreach \y in {-2,2,4}
				\draw[shift={(0,\y)},color=black] (2pt,0pt) -- (-2pt,0pt) node[above right] {\footnotesize $\y$};
				\draw[color=red] (-5,1) -- (5,1); %node[above ]{$y=m$};
				\node[above right] at (-2, 1){$y=m$};
				\node[above left] at (0,0){$O$};
				\node[below left] at (-2,4){$A$};
				\node[above right] at (4,-2){$B$};
				\node[below left] at (-2,-2){$C$};
				\clip(-5,-4) rectangle (5,5);
				%	\draw[dashed] (4,0)--(4,-2);
				\draw[dashed] (-2,4)--(0,4);
				\draw[line width=1.2pt,smooth,samples=100,domain=-3:4] (-2,-4)--(-2,5);
				\draw[line width=1.2pt,smooth,samples=100,domain=-5:5] plot(\x,{-2-0*(\x)});
				\draw[line width=1.2pt,smooth,samples=100,domain=-3:5] plot(\x,{2-(\x)});
			\end{tikzpicture}
		\end{center}
		Ba đỉnh của tam giác là $A(-2;4)$, $B(4;-2)$ và $C(-2;-2)$.\\ 
		Ta thấy điểm thấp nhất và cao nhất của miền nghiệm lần lượt có tung độ là $y=-2$ và $y=4$. Mặt khác $y=m$ là đường thẳng song song với trục $Ox$.\\
		Vậy để đường thẳng $y=m$ có điểm chung với miền nghiệm thì $-2\leq m\leq 4$.}
\end{ex}

\begin{ex}%[0D4K4-4]
	Cho hệ bất phương trình $\heva{&(a-2)x+(a-4)y\geq 2 \\ &(a+1)x+(3a+2)y\leq -1}$ với $a\in \mathbb{R}$, $a\neq 0$ và $a\neq \dfrac{1}{2}$. Điểm nào sau đây luôn thuộc miền nghiệm của hệ bất phương trình đã cho?
	\choice
	{$M\left (\dfrac{-3}{2a-1};\dfrac{7}{2a-1}\right )$}
	{$N\left (\dfrac{-7}{2a-1};\dfrac{-3}{2a-1}\right )$}
	{\True $P\left (\dfrac{7}{2a-1};\dfrac{-3}{2a-1}\right )$}
	{$P\left (\dfrac{7}{2a-1};\dfrac{3}{2a-1}\right )$}
	\loigiai{
		Dễ dàng nhận thấy rằng	nếu có một điểm luôn thuộc miền nghiệm của bất phương trình đã cho thì điểm đó phải là nghiệm của hệ $\heva{&(a-2)x+(a-4)y= 2 \\ &(a+1)x+(3a+2)y= -1} \Leftrightarrow \heva{&x=\dfrac{7}{2a-1} \\ & y=\dfrac{-3}{2a-1}.}$
	}
\end{ex}

\begin{ex}%[0D4K4-4]
	Miền nghiệm của hệ bất phương trình $\heva{&\sqrt{2}x+\sqrt{3}y-1\leq 0 \\ &\sqrt{3}x-\sqrt{2}y+1\geq 0 \\ &y\geq -4}$ là 
	\choice
	{\True tam giác vuông kể cả các điểm nằm trên ba cạnh của tam giác}
	{tam giác đều kể cả các điểm nằm trên ba cạnh của tam giác}
	{tam giác tù kể cả các điểm nằm trên ba cạnh của tam giác}
	{tam giác cân (không vuông) kể cả các điểm nằm trên ba cạnh của tam giác}
	\loigiai{
		\begin{center}
			\begin{tikzpicture}[scale=1, font=\footnotesize, line join=round, line cap=round, >=stealth]
				\clip(-5,-5) rectangle (7,1.5);
				\fill[pattern=north east lines,pattern color=blue!30] (-5,-5) rectangle (7,1.5);
				\fill[white] (-0.0636,0.6293)--(5.606,-4)--(-3.8433,-4)--cycle;
				\draw[->](-5,0)--(7,0); \draw(7-0.1,0) node[below]{$x$};
				\draw[->](0,-5)--(0,1.5); \draw(0,1.5-0.2) node[right]{$y$};
				\foreach \x in {-4,-3,-2,-1,1,2,3,4,5,6} {
					\draw (\x,0.05) -- ++(0,-0.1) node [below] {$\x$};
					\draw (0.05,\x) -- ++(-0.1,0) node [left] {$\x$}; }
				\draw node [below left]{$O$};
				\draw [domain=-4:7] plot(\x,{(-0.8165*\x)+0.5774});
				\draw plot(\x,{(1.2247*\x)+0.7071});
				\draw [domain=-5:7] plot(\x,{-4});
			\end{tikzpicture}	
		\end{center}
		Hai đường thẳng $\sqrt{2}x+\sqrt{3}y-1= 0$ và $\sqrt{3}x-\sqrt{2}y+1=0$ vuông góc với nhau nên miền nghiệm là tam giác vuông, kể cả các điểm nằm trên ba cạnh của tam giác.
	}
\end{ex}

\begin{ex}%[0D4G4-4]
	Miền nghiệm của bất phương trình $\vert x+y \vert +\vert x-y \vert \leq 4$ là
	\choice
	{một hình vuông (không kể biên)}
	{một hình chữ nhật (không phải là hình vuông và không kể biên)}
	{một hình chữ nhật (không phải là hình vuông và kể cả biên)}
	{\True một hình vuông (kể cả biên)}
	\loigiai{
		Để phá dấu giá trị tuyệt đối, ta xét dấu của $x+y$ và $x-y$, có $4$ trường hợp sau đây
		\immini{$(1)$ $\heva{&x+y\geq 0 \\&x-y\geq 0 \\&2x\leq 4 }$; $(2)$ $\heva{&x+y> 0 \\&x-y< 0 \\&2y\leq 4 }$; \\
			$(3)$ $\heva{&x+y< 0 \\&x-y> 0 \\&-2y\leq 4 }$ và $(4)$ $\heva{&x+y< 0 \\&x-y< 0 \\&-2x\leq 4 .}$ 
		}{
			\begin{tikzpicture}[scale=0.8, font=\footnotesize, line join=round, line cap=round, >=stealth]
				\clip(-3,-3) rectangle (3,3);
				\fill[pattern=north east lines,pattern color=blue!30] (-3,-3) rectangle (3,3);
				\fill[white] (-2,-2)--(-2,2)--(2,2)--(2,-2)--cycle;
				\draw[->](-3,0)--(3,0); \draw(3-0.1,0) node[below]{$x$};
				\draw[->](0,-3)--(0,3); \draw(0,3-0.2) node[right]{$y$};
				\begin{scriptsize}
					\foreach \x in {-2,-1,1,2} {
						\draw (\x,0.05) -- ++(0,-0.1) node [below] {$\x$};
						\draw (0.05,\x) -- ++(-0.1,0) node [left] {$\x$}; }
					\draw node [below left]{$O$};
					\draw (2,2) node [below right] {$A$};
					\draw (2,-2) node [above right] {$B$};
					\draw (-2,-2) node [below right] {$C$};
					\draw (-2,2) node [above right] {$D$};
				\end{scriptsize}
				
				\draw plot(\x,{\x)});
				\draw plot(\x,{-\x)});
				\draw plot(\x,{2});
				\draw plot(\x,{-2});
				\draw (-2,3)--(-2,-3)(2,3)--(2,-3);
			\end{tikzpicture}	
		}	
		
		Giải bốn hệ bất phương trình này rồi kết hợp lại ta được miền nghiệm của bất phương trình đã cho là hình vuông $ABCD$ với $A(2;2)$, $B(2;-2)$; $C(-2;-2)$; $D(-2;2)$ kể cả biên.
	}
\end{ex}
\begin{ex}%[0D4K4-2]
	Tìm giá trị lớn nhất $M$ của biểu thức $z=3x+2y$ biết rằng $x$, $y$ thỏa mãn hệ bất phương trình $\heva{&x\ge 0, \ y\ge 0 \\& x+2y\le 4 \\& x-y\le 1.}$
	\choice
	{\True $M=8$}
	{$M=10$}
	{$M=6$}
	{$M=9$}
	\loigiai{
		\immini{
			Miền nghiệm là tứ giác như hình vẽ.\\
			$z$ lớn nhất là $8$ tại đỉnh $(2;1)$.
		}
		{
			\begin{tikzpicture}[scale=0.7, font=\footnotesize, line join=round, line cap=round, >=stealth]
				\def\xmin{-1} \def\xmax{5}
				\def\ymin{-1.5} \def\ymax{3}
				\clip(\xmin,\ymin) rectangle (\xmax,\ymax);
				\tkzDefPoints{0/0/O,1/0/M,2/1/N,0/2/P}
				\tkzDrawPoints(O,M,N,P)
				\fill[pattern=north east lines,pattern color=blue!60] (O)--(M)--(N)--(P)--cycle;
				\draw[domain=-1:5] plot(\x,{2-(\x)/2}) plot(\x,{(\x)-1});
				\begin{scriptsize}
					\draw[->](\xmin,0)--(\xmax,0); \draw(\xmax-0.1,0) node[below]{$x$};
					\draw[->](0,\ymin)--(0,\ymax); \draw(0,\ymax-0.2) node[right]{$y$};
					\foreach \x in {1,2,4}
					\draw (\x,0.05) -- ++(0,-0.1) node [below] {$\x$};
					\foreach \x in {1,2}
					\draw (0.05,\x) -- ++(-0.1,0) node [left] {$\x$};
					\draw node [below left]{$O$};
				\end{scriptsize}
			\end{tikzpicture}
		}	
	}
\end{ex}

\begin{ex}%[0D4K4-2]
	Tìm giá trị lớn nhất của biểu thức $F(x;y)=x-y-1$ với $x$, $y$ thỏa mãn hệ $\heva{&x-2y+2\geq0\\&3x+8y-24\leq0\\&x\geq0,\ y\geq0.}$
	\choice
	{$ 5 $}
	{$ 6 $}
	{\True $ 7 $}
	{$ 8 $}
	\loigiai{
		\immini{
			Dễ thấy rằng: miền nghiệm của hệ đã cho là hình tứ giác OABC trên hình vẽ (Kể cả biên), trong đó các đỉnh của tứ giác có tọa độ: $O (0;0)$, $A (0;1)$, $B \left(\dfrac{16}{7};\dfrac{15}{7} \right)$, $C (8;0)$. 
		}
		{\begin{tikzpicture}[scale=1, font=\footnotesize, line join=round, line cap=round, >=stealth]
				\draw[->] (-3,0) -- (8.3,0)node[above]{$x$};
				\foreach \x in {-3,-2,-1,1,2,3,4,5,6,7,8}
				\draw[shift={(\x,0)},color=black] (0pt,2pt) -- (0pt,-2pt) node[below] {\footnotesize $\x$};
				\draw[->,color=black] (0,-1) -- (0,4.3)node[right]{$y$};
				\foreach \y in {-1,1,2,3,4}
				\draw[shift={(0,\y)},color=black] (2pt,0pt) -- (-2pt,0pt) node[left] {\footnotesize $\y$};
				\clip(-3,-1) rectangle (8.3,4);
				\fill[pattern=north east lines] (-2,3.75) -- (-3,4) -- (8,4) -- (8,0)-- cycle;
				\draw[line width=1.2pt,smooth,samples=100,domain=-3:9] plot(\x,{3-0.375*(\x)});
				\fill[pattern=north east lines] (-3,4)--(-3,-0.5)--(6,4)--(8,4)-- cycle;
				\draw[line width=1.2pt,smooth,samples=100,domain=-3:9] plot(\x,{1+0.5*(\x)});
				\fill[pattern=north east lines] (0,4)--(-3,4) --(-3,-3)--(0,-3)-- cycle;
				\fill[pattern=north east lines] (-3,0)--(-3,-3) --(8,-3)--(8,0)-- cycle;
				\color{red}{\node[below] at (2.285,2) {$B$};
					\node[below right] at (0,1) {$A$};
					\node[below left] at (8,0) {$C$};
					\node[above right] at (0,0) {$O$};
				}
			\end{tikzpicture}
		}
		\noindent Ta biết rằng giá trị lớn nhất của biểu thức $F(x;y)$ sẽ đạt được tại các đỉnh của tứ giác, do đó ta tính giá trị của $F(x;y)$ tại các đỉnh này.
		$F(0;0)=-1$, $F(0;1)=-2$, $F\left(\dfrac{16}{7};\dfrac{15}{7} \right)=-\dfrac{6}{7}$, $F(8;0)=7$. \\
		Vậy giá trị lớn nhất của biểu thức thỏa mãn hệ là $F(8;0)=7$.
	}
\end{ex}

\begin{ex}%[0D4K4-2]
	Tìm giá trị lớn nhất $a$ và giá trị nhỏ nhất $b$ của $F(x;y)=3x+9y$ với $(x;y)$ là nghiệm của hệ bất phương trình $\heva{&x-y+1 \leq 0\\&2x-y+4 \geq 0\\&x+y+1 \geq 0\\&2x+y-4 \leq 0.}$
	\choice
	{$a=21,b=1$}
	{$a=21,b=-3$}
	{$a=36,b=1$}
	{\True $a=36,b=-3$}
	\loigiai{ Ta đã biết giá trị lớn nhất và giá trị nhỏ nhất của biểu thức $F(x;y)$ đạt được tại các điểm $(1;2)$, $(0;4)$, $(-1;0)$, $\left(\dfrac{-5}{3};\dfrac{2}{3} \right)$ theo trên hình vẽ minh họa. Thử lại ta thấy giá trị lớn nhất $a=36$ tại $(x;y)=(0;4)$, giá trị nhỏ nhất $b=-3$ tại $(x;y)=(-1;0)$.\\
		\begin{center}
			\begin{tikzpicture}[scale=1, font=\footnotesize, line join=round, line cap=round, >=stealth]
				\draw[->] (-3.5,0) -- (3.3,0)node[above]{$x$};
				\foreach \x in {-3,-2,-1,1,2,3}
				\draw[shift={(\x,0)},color=black] (0pt,2pt) -- (0pt,-2pt) node[below] {\footnotesize $\x$};
				\draw[->,color=black] (0,-1.3) -- (0,4.3)node[right]{$y$};
				\foreach \y in {-1,1,2,3,4}
				\draw[shift={(0,\y)},color=black] (2pt,0pt) -- (-2pt,0pt) node[left] {\footnotesize $\y$};
				\clip(-3,-1) rectangle (3.3,4.3);
				\fill[pattern=north east lines] (-2.5,-1) -- (-3.3,-1) -- (-3.3,4.3) -- (0.15,4.3)-- cycle;
				\draw[line width=1.2pt,smooth,samples=100,domain=-3:9] plot(\x,{1+(\x)});
				\fill[pattern=north east lines] (-0.15,4.3)--(3.3,4.3)--(3.3,-1)--(2.5,-1)-- cycle;
				\draw[line width=1.2pt,smooth,samples=100,domain=-3:9] plot(\x,{4+2*(\x)});
				\fill[pattern=north east lines] (3.3,4.3)--(3.3,-1) --(-2,-1)-- cycle;
				\draw[line width=1.2pt,smooth,samples=100,domain=-3:9] plot(\x,{-1-(\x)});
				\fill[pattern=north east lines] (0,-1)--(-3,-1) --(-3,2)-- cycle;
				\draw[line width=1.2pt,smooth,samples=100,domain=-3:9] plot(\x,{4-2*(\x)});
			\end{tikzpicture}
		\end{center}
		\begin{center}
			Hình vẽ minh họa
		\end{center}
	}
\end{ex}

\begin{ex}%[0D4K4-2]
	Cho hệ bất phương trình $\heva{&0\leq x\leq 5 \\ &0\leq y \leq 10 \\ &5x+3y\geq 15 \\ &-x+y\geq 2}$ và biểu thức $P(x;y)=2x-2y+3$ với $(x;y)$ thuộc miền nghiệm của hệ bất phương trình đã cho. Tìm giá trị nhỏ nhất của $P$.
	\choice
	{\True $-17$}
	{$-34$}
	{$-7$}
	{$-14$}
	\loigiai{
		\immini{
			Miền nghiệm của hệ bất phương trình cho ở giả thiết bài toán được biểu diễn như hình trên, với miền nghiệm là hình ngũ giác màu trắng, kể cả biên. $P$ chỉ có thể đạt giá trị nhỏ nhất tại các đỉnh của ngũ giác, các đỉnh đó có tọa độ lần lượt là $A(0;10)$, $B(5;10)$, $C(5;7)$, $D\left (\dfrac{9}{8}; \dfrac{25}{8}\right )$, $E(0;5)$. Thay tọa độ các đỉnh vào $P$ ta tìm được giá trị nhỏ nhất của $P$ bằng $-17$ tại $x=0$, $y=10$.
		}{
			\begin{tikzpicture}[scale=0.6, font=\footnotesize, line join=round, line cap=round, >=stealth]
				\clip(-3,-1) rectangle (6,11);
				\def\xmin{-3} \def\xmax{6}
				\def\ymin{-1} \def\ymax{11}
				\tkzDefPoints{\xmax/\ymax/A,\xmin/\ymax/B,\xmin/\ymin/C,\xmax/\ymin/D}
				%Nhập hệ số a,b,c đt d1: x+y-1=0-----------Đường thẳng MN
				\def\hsa{5} \def\hsb{3} \def\hsc{-15}
				\ifnum\hsb=0{
					\tkzDefPoint(-\hsc/ \hsa,\ymin){M};
					\tkzDefPoint(-\hsc/ \hsa,\ymax){N};}
				\else{ 
					\tkzDefPoint(\xmin,-\hsa*\xmin/\hsb-\hsc/\hsb){M};
					\tkzDefPoint(\xmax,-\hsa*\xmax/\hsb-\hsc/\hsb){N};}\fi
				%Nhập hệ số a,b,c đt d2: 2x-3y-1=0---------Đường thẳng PQ
				\def\hsa{-1} \def\hsb{1} \def\hsc{-2}
				\ifnum\hsb=0{
					\tkzDefPoint(-\hsc/ \hsa,\ymin){P};
					\tkzDefPoint(-\hsc/ \hsa,\ymax){Q};}
				\else{ 
					\tkzDefPoint(\xmin,-\hsa*\xmin/\hsb-\hsc/\hsb){P};
					\tkzDefPoint(\xmax,-\hsa*\xmax/\hsb-\hsc/\hsb){Q};}\fi
				%Nhập hệ số a,b,c đt d1: x+y-1=0-----------Đường thẳng RS
				\def\hsa{1} \def\hsb{0} \def\hsc{-5}
				\ifnum\hsb=0{
					\tkzDefPoint(-\hsc/ \hsa,\ymin){R};
					\tkzDefPoint(-\hsc/ \hsa,\ymax){S};}
				\else{ 
					\tkzDefPoint(\xmin,-\hsa*\xmin/\hsb-\hsc/\hsb){R};
					\tkzDefPoint(\xmax,-\hsa*\xmax/\hsb-\hsc/\hsb){S};}\fi
				%Nhập hệ số a,b,c đt d1: x+y-1=0-----------Đường thẳng UV
				\def\hsa{0} \def\hsb{1} \def\hsc{-10}
				\ifnum\hsb=0{
					\tkzDefPoint(-\hsc/ \hsa,\ymin){U};
					\tkzDefPoint(-\hsc/ \hsa,\ymax){V};}
				\else{ 
					\tkzDefPoint(\xmin,-\hsa*\xmin/\hsb-\hsc/\hsb){U};
					\tkzDefPoint(\xmax,-\hsa*\xmax/\hsb-\hsc/\hsb){V};}\fi
				\tkzInterLL(M,N)(P,Q) \tkzGetPoint{I}
				\tkzInterLL(P,Q)(R,S) \tkzGetPoint{K}
				\fill[pattern=north east lines,pattern color=blue!30] (C) rectangle (A);
				\fill[white] (I)--(K)--(5,10)--(0,10)--(0,5)--cycle;
				\tkzDrawSegment(M,N)
				\tkzDrawSegment(P,Q)
				\tkzDrawSegment(R,S)
				\tkzDrawSegment(U,V)
				\draw[->](\xmin,0)--(\xmax,0); \draw(\xmax-0.1,0) node[below]{$x$};
				\draw[->](0,\ymin)--(0,\ymax); \draw(0,\ymax-0.2) node[right]{$y$};
				\begin{scriptsize}
					\foreach \x in {-3,-2,-1,1,2,3,4,5,6,7,8,9,10} {
						\draw (\x,0.05) -- ++(0,-0.1) node [below] {$\x$};
						\draw (0.05,\x) -- ++(-0.1,0) node [left] {$\x$}; }
					\draw node [below left]{$O$};
				\end{scriptsize}
				
			\end{tikzpicture}
		}	
	}
\end{ex}

\begin{ex}%[0D4K4-2]
	Tìm giá trị nhỏ nhất của biểu thức $F=y-x$ trên miền xác định bởi hệ $\heva{y-2x&\leq 2\\2y-x&\geq 4\\x+y&\leq 5.}$
	\choice
	{\True $\min F=1$ khi $x=2$, $y=3$}
	{$\min F=2$ khi $x=0$, $y=2$}
	{$\min F=3$ khi $x=1$, $y=4$}
	{$\min F=-1$ khi $x=2$, $y=1$}
	\loigiai{Biểu diễn miền nghiệm của bất phương trình đã cho ta được miền nghiệm là tam giác $ABC$ với tọa độ các đỉnh là: $A(0;2)$, $B(2;3)$, $C(1;4)$.
		\begin{center}
			\begin{tikzpicture}[scale=1, font=\footnotesize, line join=round, line cap=round, >=stealth]
				\fill[pattern=north east lines] (-5,-2) -- (-5,6) -- (6,6) -- (6,-2) -- cycle;
				\fill[white] (0,2)--(1,4) -- (2,3) -- cycle;
				\draw[->] (-5.2,0) -- (6.3,0)node[above]{$x$};
				\foreach \x in {-4,-1,4,5}
				\draw[shift={(\x,0)},color=black] (0pt,2pt) -- (0pt,-2pt) node[below] {\footnotesize $\x$};
				\draw[->,color=black] (0,-2.1) -- (0,6.3)node[right]{$y$};
				\foreach \y in {2,5}
				\draw[shift={(0,\y)},color=black] (2pt,0pt) -- (-2pt,0pt) node[above right] {\footnotesize $\y$};
				\node[above right] at (0,0){$O$};
				\node[below right] at (0,2){$A$};
				\node[below] at (2,3){$B$};
				\node[left] at (1,4){$C$};
				\clip(-5,-2) rectangle (6,6);
				\draw[line width=1.2pt,smooth,samples=100,domain=-2:6] plot(\x,{2+2*(\x)});
				\draw[line width=1.2pt,smooth,samples=100,domain=-5:6] plot(\x,{0.5*(\x)+2});
				\draw[line width=1.2pt,smooth,samples=100,domain=-2:6] plot(\x,{5-(\x)});
			\end{tikzpicture}
		\end{center}
		Tính giá trị của biểu thức $F=y-x$ tại tọa độ các đỉnh ta có:\\
		Tại $A(0;2)$: $F=y-x=2$.\\
		Tại $B(2;3)$: $F=y-x=1$.\\
		Tại $C(1;4)$: $F=y-x=3$.\\
		Vậy $\min F=1$ khi $x=2$, $y=3$.
	}
\end{ex}

\begin{ex}%[0D4K4-2]
	Tìm giá trị nhỏ nhất $T$ của biểu thức $z=5x+7y$ biết rằng $x$, $y$ là các số không âm thỏa mãn hệ bất phương trình $\heva{&2x+3y\ge 6 \\ & 3x-y\le 15 \\& -x+y\le 4 \\& 2x+5y\le 27.}$
	\choice
	{$T=12$}
	{\True $T=14$}
	{$T=28$}
	{$T=18$}
	\loigiai{
		\immini{
			Miền nghiệm là miền lục giác có tọa độ các đỉnh lần lượt là:\\ $(0;2)$, $(0;4)$, $(1;5)$, $(6;3)$, $(5;0)$, $(3;0)$.\\
			Giá trị nhỏ nhất $T=14$ đạt tại đỉnh $(0;2)$.
		}
		{
			\begin{tikzpicture}[scale=0.5, font=\footnotesize, line join=round, line cap=round, >=stealth]
				\def\xmin{-2} \def\xmax{7}
				\def\ymin{-1.5} \def\ymax{6}
				\clip(\xmin,\ymin) rectangle (\xmax,\ymax);
				\tkzDefPoints{\xmax/\ymax/A1,\xmin/\ymax/A2,\xmin/\ymin/A3,\xmax/\ymin/A4}
				\fill[pattern=north east lines,pattern color=blue!60] (A1)--(A2)--(A3)--(A4)--cycle;
				\tkzDefPoints{0/2/A,0/4/B,1/5/C,6/3/D,5/0/E,3/0/F}
				\tkzDrawPoints(A,B,C,D,E,F)
				\fill[color=white] (A)--(B)--(C)--(D)--(E)--(F)--cycle;
				\draw[domain=-2:7] plot(\x,{2-2*(\x)/3}) plot(\x,{3*(\x)-15}) plot(\x,{(\x)+4}) plot(\x,{27/5-2*(\x)/5});
				\begin{scriptsize}
					\draw[->](\xmin,0)--(\xmax,0); \draw(\xmax-0.1,0) node[below]{$x$};
					\draw[->](0,\ymin)--(0,\ymax); \draw(0,\ymax-0.2) node[right]{$y$};
					\foreach \x in {2,4,6}
					\draw (\x,0.05) -- ++(0,-0.1) node [below] {$\x$};
					\foreach \x in {2,4}
					\draw (0.05,\x) -- ++(-0.1,0) node [left] {$\x$};
					\draw node [below left]{$O$};
				\end{scriptsize}
			\end{tikzpicture}
		}	
	}
\end{ex}

\begin{ex}%[0D4K4-2]
	Tìm các cặp số $(x;y)$ thỏa mãn hệ bất phương trình $\heva{&0\le y\leq 2\\&y\leq x\\&x+y\leq 5\\&x\leq 4}$ sao cho biểu thức $S=2x+y$ đạt giá trị lớn nhất.
	\choice
	{$(x;y)=(4;0)$}
	{\True $(x;y)=(4;1)$}
	{$(x;y)=(3;2)$}
	{$(x;y)=(2;2)$}
	\loigiai{Miền nghiệm của hệ bất phương trình đã cho là ngũ giác $OABCD$ (bao gồm các điểm trên các cạnh)
		\begin{center}
			\begin{tikzpicture}[scale=1, font=\footnotesize, line join=round, line cap=round, >=stealth]
				\fill[pattern=north east lines] (-2,-2) -- (-2,6) -- (6,6) -- (6,-2) -- cycle;
				\fill[white] (0,0) -- (2,2) -- (3,2) --(4,1)--(4,0)-- cycle;
				\draw[->] (-2.1,0) -- (6.3,0)node[above]{$x$};
				\foreach \x in {2,3,4,5}
				\draw[shift={(\x,0)},color=black] (0pt,2pt) -- (0pt,-2pt) node[below left] {\footnotesize $\x$};
				\draw[->,color=black] (0,-2.1) -- (0,6.3)node[right]{$y$};
				\foreach \y in {2,5}
				\draw[shift={(0,\y)},color=black] (2pt,0pt) -- (-2pt,0pt) node[above right] {\footnotesize $\y$};
				\node[above left] at (0,0){$O$};
				\node[below] at (2,2){$A$};
				\node[below] at (3,2){$B$};
				\node[above right] at (4,1){$C$};
				\node[above left] at (4,0){$D$};
				\clip(-2,-2) rectangle (6,6);
				\draw[line width=1.2pt,smooth,samples=100,domain=-2:6] plot(\x,{2-0*(\x)});
				\draw[line width=1.2pt,smooth,samples=100,domain=-2:6] plot(\x,{(\x)});
				\draw[line width=1.2pt,smooth,samples=100,domain=-2:6] plot(\x,{5-(\x)});
				\draw[line width=1.2pt,smooth,samples=100,domain=-2:6] (4,-2)--(4,6);
			\end{tikzpicture}
		\end{center}	
		Tọa độ các đỉnh là: $O(0;0)$, $A(2;2)$, $B(3;2)$, $C(4;1)$, $D(4;0)$.\\
		Lần lượt tính giá trị của biểu thức tại các cặp số là tọa độ các đỉnh, suy ra biểu thức $S=2x+y$ đạt giá trị lớn nhất với cặp số $(4;1)$ ứng với tọa độ đỉnh $C$.
	}
\end{ex}

\begin{ex}%[0D4K4-3]
	Khẩu phần dinh dưỡng hàng ngày cho người ăn kiêng cần cung cấp ít nhất $300$ calo, $36$ đơn vị vitamin $A$ và $90$ đơn vị vitamin $C$. Một tách thức uống $X$ có giá $5$ nghìn đồng và cung cấp $60$ calo, $12$ đơn vị vitamin $A$ và $10$ đơn vị vitamin $C$. Một tách thức uống $Y$ có giá $6$ nghìn đồng và cung cấp $60$ calo, $6$ đơn vị vitamin $A$ và $30$ đơn vị vitamin $C$. Mỗi ngày nên uống bao nhiêu tách mỗi loại để có được chi phí tối ưu và vẫn đáp ứng được yêu cầu dinh dưỡng hàng ngày?
	\choice
	{$1$ tách loại $X$, $4$ tách loại $Y$}
	{\True $3$ tách loại $X$, $2$ tách loại $Y$}
	{$2$ tách loại $X$, $3$ tách loại $Y$}
	{$4$ tách loại $X$, $1$ tách loại $Y$}
	\loigiai{
		\immini{
			Ta có hệ: $\heva{& 60x+60y\ge 300 \\ & 12x+6y\ge 36 \\ & 10x+30y \ge 90 \\& x\ge 0 \\ & y\ge 0}$\\
			Giá $C=5x+6y$.\\
			Miền nghiệm như hình vẽ. Các đỉnh là:\\
			$M(0;6)$, $N(1;4)$, $P(3;2)$, $Q(9;0)$.\\
			$C$ nhỏ nhất tại đỉnh $P(3;2)$.\\
			Vậy nên uống $3$ tách loại $X$ và $2$ tách loại $Y$.
		}
		{
			\begin{tikzpicture}[scale=0.5, font=\footnotesize, line join=round, line cap=round, >=stealth]
				\def\xmin{-2} \def\xmax{12}
				\def\ymin{-2} \def\ymax{8}
				\clip(\xmin,\ymin) rectangle (\xmax,\ymax);
				\tkzDefPoints{\xmax/\ymax/A1,\xmin/\ymax/A2,\xmin/\ymin/A3,\xmax/\ymin/A4}
				\tkzDefPoints{0/6/M,1/4/N,3/2/P,9/0/Q}
				\tkzDrawPoints(M,N,P,Q) \tkzLabelPoints[above right](M,N,P,Q)
				
				\fill[pattern=north east lines,pattern color=black!60] (A2)--(A3)--(A4)--(\xmax,0)--(Q)--(P)--(N)--(M)--(0,\ymax)--cycle;
				\draw[domain=-2:12] plot(\x,{5-(\x)}) plot(\x,{6-2*(\x)}) plot(\x,{3-(\x)/3});
				\begin{scriptsize}
					\draw[->](\xmin,0)--(\xmax,0); \draw(\xmax-0.1,0) node[below]{$x$};
					\draw[->](0,\ymin)--(0,\ymax); \draw(0,\ymax-0.2) node[right]{$y$};
					\foreach \x in {2,4,6,8,10}
					\draw (\x,0.05) -- ++(0,-0.1) node [below] {$\x$};
					\foreach \x in {2,4,6,8}
					\draw (0.05,\x) -- ++(-0.1,0) node [left] {$\x$};
					\draw node [below left]{$O$};
				\end{scriptsize}
			\end{tikzpicture}
		}	
	}
\end{ex}

\begin{ex}%[0D4K4-3]
	Một gia đình cần ít nhất $900$ đơn vị prô-tê-in và $400$ đơn vị li-pít trong thức ăn mỗi ngày. Mỗi kí-lô-gam thịt bò chứa $800$ đơn vị prô-tê-in và $200$ đơn vị li-pít. Mỗi kí-lô-gam thịt lợn chứa $600$ đơn vị prô-tê-in và $400$ đơn vị li-pít. Biết rằng gia đình này chỉ mua tối đa $1,6$ kg thịt bò và $1,1$ kg thịt lợn; giá tiền $1$ kg thịt bò là $45000$ đồng, $1$ kg thịt lợn là $35000$ đồng. Hỏi gia đình đó phải mua bao nhiêu kí-lô-gam thịt mỗi loại để số tiền bỏ ra là ít nhất?
	\choice
	{$0{,}3$ kg thịt bò và $1{,}1$ kg thịt lợn}
	{\True $0{,}6$ kg thịt bò và $0{,}7$ kg thịt lợn}
	{$1{,}6$ kg thịt bò và $1{,}1$ kg thịt lợn}
	{$0{,}6$ kg thịt lợn và $0{,}7$ kg thịt bò}
	\loigiai{Gọi $x$ và $y$ lần lượt là số kí-lô-gam thịt bò và thịt lợn mà gia đình đó mua mỗi ngày ($0\leq x\leq 1{,}6;0\leq y\leq 1{,}1$).\\
		Khi đó chi phí để mua số thịt trên là: $F=45000x+35000y$ đồng.\\
		Trong $x$ kg thịt bò chứa $800x$ đơn vị prô-tê-in và $200x$ đơn vị li-pít.\\
		Trong $y$ kg thịt lợn chứa $600x$ đơn vị prô-tê-in và $400y$ đơn vị li-pít.\\
		Suy ra, số đơn vị prô-tê-in và số đơn li-pít lần lượt là $800x+600y$ đơn vị và $200x+400y$ đơn vị.
		Do gia đình này cần ít nhất $900$ đơn vị prô-tê-in và $400$ đơn vị li-pít trong thức ăn mỗi ngày nên ta có hệ bất phương trình sau: 
		$$\heva{&800x+600y\geq 900\\&200x+400y\geq 400\\&0\leq x\leq 1{,}6\\&0\leq y\leq 1{,}1}\Leftrightarrow \heva{&8x+6y\geq 9\\&x+2y\geq 2\\&0\leq x\leq 1{,}6\\&0\leq y\leq 1{,}1.}$$	
		\immini{Bài toán trở thành: Tìm GTNN của $F=45000x+35000y$ với $x$, $y$ thỏa hệ trên.\\
			Giải hệ bất phương trình trên, ta có miền nghiệm là tứ giác $ABCD$ (hình bên) với tọa độ các đỉnh là: $A(1{,}6;1{,}1)$, $B(1{,}6;0{,}2)$, $C(0{,}6;0{,}7)$, $D(0{,}3;1{,}1)$.\\
			Khi đó: \\Tại $A(1{,}6;1{,}1)$: $F=110500$\\
			Tại $B(1{,}6;0{,}2)$: $F=79000$\\
			Tại $C(0{,}6;0{,}7)$: $F=51500$\\
			Tại $D(0{,}3;1{,}1)$: $F=52000$.
		}
		{\begin{tikzpicture}[scale=1, font=\footnotesize, line join=round, line cap=round, >=stealth]
				\fill[pattern=north east lines] (-1,-1) -- (-1,2) -- (3,2) -- (3,-1) -- cycle;
				\fill[white] (1.6,1.1) -- (1.6,0.2) -- (0.6,0.7) --(0.3,1.1)-- cycle;
				\draw[->] (-1.3,0) -- (3.3,0)node[above]{$x$};
				\foreach \x in {1,2}
				\draw[shift={(\x,0)},color=black] (0pt,2pt) -- (0pt,-2pt) node[below left] {\footnotesize $\x$};
				\draw[->,color=black] (0,-1.2) -- (0,2.3)node[right]{$y$};
				\foreach \y in {1}
				\draw[shift={(0,\y)},color=black] (2pt,0pt) -- (-2pt,0pt) node[above right] {\footnotesize $\y$};
				\node[above left] at (0,0){$O$};
				\node[above left] at (1.6,1.1){$A$};
				\node[above right] at (1.6,0.2){$B$};
				\node[below left] at (0.6,0.7){$C$};
				\node[above right] at (0.3,1.1){$D$};
				\clip(-1,-1) rectangle (3,2);
				\draw[smooth,samples=100,domain=-1:3] plot(\x,{1.5-1.333333*(\x)});
				\draw[smooth,samples=100,domain=-1:3] plot(\x,{1-0.5*(\x)});
				\draw[,smooth,samples=100,domain=-1:3] plot(\x,{1.1-0*(\x)});
				\draw[smooth,samples=100,domain=-1:3] (1.6,-1)--(1.6,3);
			\end{tikzpicture}
		}
		Suy ra, $F$ nhỏ nhất khi $(x;y)=(0{,}6;0{,}7)$. Do đó gia đình này cần mua $0{,}6$ kg thịt bò và $0{,}7$ kg thịt lợn.
	}
\end{ex}

\begin{ex}%[0D4K4-4]
	Một cửa hàng làm kệ sách và bàn làm việc. Mỗi kệ sách cần $5$ giờ chế biến gỗ và $4$ giờ hoàn thiện. Mỗi bàn làm việc cần $10$ giờ chế biến gỗ và $3$ giờ hoàn thiện. Mỗi tháng cửa hàng có $600$ giờ lao động để chế biến gỗ và $240$ giờ để hoàn thiện. Lợi nhuận của mỗi kệ sách là $400$ nghìn đồng và mỗi bàn là $750$ nghìn đồng. Có bao nhiêu sản phẩm mỗi loại cần được làm mỗi tháng để thu được lợi nhuận tối đa?
	\choice
	{$24000$}
	{$45000$}
	{\True $45600$}
	{$46000$}
	\loigiai{
		\immini{Ta có hệ: $\heva{& 5x+10y\le 600 \\ & 4x+3y\le 240 \\ & x\ge 0 \\ & y\ge 0}$\\
			Lợi nhuận: $P=400x+750y$.\\
			Miền nghiệm của hệ là miền tứ giác $OABC$ với:\\
			$A(0;60)$, $B(24;48)$, $C(60;0)$.\\
			Lợi nhuận tối đa $P_{max}=P(B)=45600$.
		}
		{
			\begin{tikzpicture}[scale=0.05, font=\footnotesize, line join=round, line cap=round, >=stealth]
				\def\xmin{-10} \def\xmax{140}
				\def\ymin{-10} \def\ymax{100}
				\clip(\xmin,\ymin) rectangle (\xmax,\ymax);
				\tkzDefPoints{\xmax/\ymax/A1,\xmin/\ymax/A2,\xmin/\ymin/A3,\xmax/\ymin/A4}
				\fill[pattern=north east lines,pattern color=black!60] (A1)--(A2)--(A3)--(A4)--cycle;
				\tkzDefPoints{0/0/O,0/60/A,24/48/B,60/0/C}
				\fill[color=white] (O)--(A)--(B)--(C)--cycle;
				\tkzDrawPoints(A,B,C)
				\draw[domain=-10:140] plot(\x,{60-(\x)/2}) plot(\x,{80-4*(\x)/3});
				\begin{scriptsize}
					\draw[->](\xmin,0)--(\xmax,0); \draw(\xmax-4,0) node[below]{$x$};
					\draw[->](0,\ymin)--(0,\ymax); \draw(0,\ymax-4) node[right]{$y$};
					\draw node [below left]{$O$}
					(A) node [above right]{$A$}
					(B) node [below left]{$B$}
					(C) node [above right]{$C$};
				\end{scriptsize}
			\end{tikzpicture}
		}
	}
\end{ex}

\begin{ex}%[0D4G4-4]
	Cho hệ bất phương trình $\heva{&\vert x-1 \vert \leq 2 \\ &\vert y+1 \vert \leq 3}$ và biểu thức $P(x;y)=3x+2y-5$ với $(x;y)$ thuộc miền nghiệm của hệ bất phương trình đã cho. Tìm giá trị lớn nhất của $P$.
	\choice
	{$16$}
	{$-16$}
	{\True $8$}
	{$-8$}
	\loigiai{
		$$\heva{&\vert x-1 \vert \leq 2 \\ &\vert y+1 \vert \leq 3} 
		\Leftrightarrow
		\heva{&-1\leq x\leq 3 \\ &-4\leq y\leq 2.}$$
		Miền nghiệm là hình chữ nhật $ABCD$ với $A(3;2)$, $B(3;-4)$, $C(-1;-4)$ và $D(-1;2)$. Giá trị lớn nhất của $P$ đạt được tại đỉnh $A(3;2)$ và $P(3;2)=8$.	
	}
\end{ex}

\begin{ex}%[0D4K4-3]
	Người ta dự định dùng hai loại nguyên liệu để chiết xuất ít nhất $140$ kg chất A và $9$ kg chất B. Từ mỗi tấn nguyên liệu loại I giá $4$ triệu đồng, có thể chiết xuất được $20$ kg chất A và $0{,}6$ kg chất B. Từ mỗi tấn nguyên liệu loại II giá $3$ triệu đồng có thể chiết xuất được $10$ kg chất A và $1{,}5$ kg chất B. Hỏi phải dùng bao nhiêu tấn nguyên liệu mỗi loại để chi phí mua nguyên liệu là ít nhất, biết rằng cơ sở cung cấp nguyên liệu chỉ có thể cung cấp không quá $10$ tấn nguyên liệu loại I và không quá $9$ tấn nguyên liệu loại II?
	\choice
	{$2{,}5$ tấn loại I và $9$ tấn loại II}
	{$10$ tấn loại I và $9$ tấn loại II}
	{$10$ tấn loại I và $2$ tấn loại II}
	{\True $5$ tấn loại I và $4$ tấn loại II}
	\loigiai{
		Gọi $x$, $y$ lần lượt là số tấn nguyên liệu loại I và loại II phải dùng.\\
		Từ bài toán ta đưa được hệ bất phương trình:
		$\heva{&0 \leq x \leq 10\\&0 \leq y \leq 9\\&2x+y\geq 14\\&2x+5y\geq 30} (*)$
		\immini{
			Tổng chi phí là $F(x;y)=4x+3y$\\
			Ta tìm $x$, $y$ thỏa mãn hệ $(*)$ sao cho $F(x;y)$ nhỏ nhất.\\
			Ta biết giá trị nhỏ nhất đạt tại các điểm $A(5;4)$, $B(10;2)$, $C(10;9)$, $D(3;9)$.\\
			Thử lại thấy $F(5;4)$=32 là giá trị nhỏ nhất.\\
			Vậy cần 5 tấn nguyên liệu loại I và 4 tấn nguyên liệu loại II.
		}{
			\begin{tikzpicture}[scale=0.4, font=\footnotesize, line join=round, line cap=round, >=stealth]
				\draw[->] (-1,0) -- (15.7,0)node[above]{$x$};
				\foreach \x in {,,,4,,,7,,,10,,,,,15}
				\draw[shift={(\x,0)},color=black] (0pt,2pt) -- (0pt,-2pt) node[below left] {\footnotesize $\x$};
				\draw[->,color=black] (0,-1) -- (0,14.7)node[right]{$y$};
				\foreach \y in {,2,,,,6,,,9,,,,,14}
				\draw[shift={(0,\y)},color=black] (2pt,0pt) -- (-2pt,0pt) node[above left] {\footnotesize $\y$};
				\node[below left] at (0,0) {$O$};
				\clip(-1,-1) rectangle (15.3,14.3);
				\fill[pattern=north east lines] (-0.15,14.3) -- (-1,14.3) -- (-1,-1) -- (7.5,-1)-- cycle;
				\draw[smooth,samples=100,domain=-1:9] plot(\x,{14-2*(\x)});
				\fill[pattern=north east lines] (-1,6.4)--(-1,-1)--(15.3,-1)--(15.3,-0.12)-- cycle;
				\draw[smooth,samples=100,domain=-1:15.3] plot(\x,{6-0.4*(\x)});
				\fill[pattern=north east lines] (-1,9)--(-1,14.3) --(15.3,14.3)--(15.3,9)-- cycle;
				\draw[smooth,samples=100,domain=-1:15.3] plot(\x,{9+0*(\x)});
				\fill[pattern=north east lines] (10,14.3)--(15.3,14.3) --(15.3,-1)--(10,-1)-- cycle;
				\draw[smooth,samples=100](10,-1)--(10,15.3);
				\fill[pattern=north east lines] (0,4)--(-3,4) --(-3,-3)--(0,-3)-- cycle;
				\draw[smooth,samples=100,domain=-3:9] plot(\x,{0*(\x)});
				\node[above left] at (5.5,4.5) {$A$};
				\node[left] at (10,3) {$B$};
				\node[below left] at (10,9) {$C$};
				\node[below right] at (2.75,9) {$D$};	
			\end{tikzpicture}
			
		}
	}
\end{ex}
\begin{ex}%[0D4K4-4]
	Giá trị nhỏ nhất $F_{\text {min }}$ của biểu thức $F(x ; y)=y-x$ trên miền xác định bởi hệ $\heva{&y-2 x \leq 2 \\ &2 y-x \geq 4  \\ &x+y \leq 5}$ là
	\choice
	{\True $F_{\min }=1$}
	{$F_{\min }=2$}
	{$F_{\min }=3$}
	{$F_{\min }=4$}
	\loigiai{
		Ta có $\heva{&y-2 x \leq 2 \\ &2 y-x \geq 4  \\ &x+y \leq 5} \Leftrightarrow \heva{&y-2 x -2\leq 0 \\ &2 y-x-4 \geq 0  \\ &x+y-5 \leq 0.}\quad(*)$\\
		Trong mặt phẳng tọa độ $ Oxy $, vẽ các đường thẳng $d_1\colon y-2x-2=0 $, $ d_2\colon 2y-x-4=0 $, $ d_3\colon x+y-5=0 $. Khi đó miền nghiệm của hệ bất phương trình $(*)$ là phần mặt phẳng (tam giác $ABC$ kể cả biên) như hình vẽ. Xét các đỉnh của miền khép kín tạo bởi hệ $(*)$ là $A(0 ; 2)$, $B(2 ; 3)$, $C(1 ; 4)$. 
		\begin{center}
			\begin{tikzpicture}[line cap=round,line join=round,>=stealth,x=1.0cm,y=1.0cm,scale=.7]
				\clip(-2,-1) rectangle (6,6);
				\draw[->] (-2,0) -- (6,0) node[above left]{$x$};
				\draw[->] (0,-1) -- (0,6)node[below right]{$y$};
				\foreach \x in{1,2}\draw[fill=black] (\x,0) circle (.5pt) node[below]{\footnotesize $\x$};
				\foreach \y in{2,3,4,5}\draw[fill=black] (0,\y)circle (.5pt) node[left]{\footnotesize $\y$};
				\draw[smooth,samples=100,domain=-2:6] plot(\x,{(--2--2*\x)/1});
				\draw[smooth,samples=100,domain=-2:6] plot(\x,{(--4--1*\x)/2});
				\draw[smooth,samples=100,domain=-2:6] plot(\x,{(--5-1*\x)/1});
				\fill[pattern=north east lines] (-2,-1) -- (-2,6) -- (6,6) -- (6,-1) -- cycle;
				\draw [fill=white] (0,2) -- (1,4) -- (2,3) -- cycle;
				\draw [fill=black] (0,0)node[above left]{$O$}circle(.5pt) (0,2)node[below right]{$A$}circle(.5pt) (1,4)node[right]{$B$}circle(.5pt) (2,3)node[above]{$C$}circle(1.2pt);
				\draw [fill=black] (1.7,5.5)node[right]{$d_1$}circle(.5pt) (4,4.8)node[right]{$d_2$}circle(.5pt) (4.5,.5)node[above]{$d_3$}circle(.5pt);
				\draw [dashed] (0,4) -- (1,4) -- (1,0) (0,3) -- (2,3) -- (2,0);
			\end{tikzpicture}
		\end{center}
		Ta có $\heva{
			&F(0 ; 2)=2 \\
			&F(2 ; 3)=1 \\
			&F(1 ; 4)=3
		}\longrightarrow F_{\min }=1$.
	}
\end{ex}
\begin{ex}%[0D4K4-3]
	Một nhà máy sản xuất, sử dụng ba loại máy đặc chủng để sản xuất sản phẩm $A$ và sản phẩm $B$ trong một chu trình sản xuất. Đề sản xuất một tấn sản phẩm $A$ lãi $4$ triệu đồng người ta sử dụng máy $I$ trong 1 giờ, máy II trong $2$ giờ và máy III trong $3$ giờ. Để sản xuất ra một tấn sản phẩm $B$ lãi được $3$ triệu đồng người ta sử dụng máy I trong $6$ giờ, máy II trong $3$ giờ và máy III trong $2$ giờ. Biết rằng máy $I$ chỉ hoạt động không quá $36$ giờ, máy hai hoạt động không quá $23$ giờ và máy III hoạt động không quá $27$ giờ. Hãy lập kế hoạch sản xuất cho nhà máy để tiền lãi được nhiều nhất. 
	\choice
	{Sản xuất $9$ tấn sản phẩm $A$ và không sản xuất sản phẩm $B$}
	{\True Sản xuất $7$ tấn sản phẩm $A$ và $3$ tấn sản phẩm $B$}
	{Sản xuất $\dfrac{45}{8}$ tấn sản phẩm $A$ và $\dfrac{81}{16}$ tấn sản phẩm $B$}
	{Sản xuất $6$ tấn sản phẩm $B$ và không sản xuất sản phẩm $A$}
	\loigiai{
		Gọi $x \geq 0$, $y \geq 0$ (tấn) là sản lượng cần sản xuất của sản phẩm $A$ và sản phẩm $B$. Ta có:\\
		$x+6 y$ là thời gian hoạt động của máy I,\\
		$2 x+3 y$ là thời gian hoạt động của máy II,\\
		$3 x+2 y$ là thời gian hoạt động của máy III.\\
		Số tiền lãi của nhà máy: $f(x;y)=4 x+3 y$ (triệu đồng).\\
		Bài toán trở thành: Tìm $(x;y)$ thỏa mãn $\heva{&x+6 y \leq 36 \\ &2 x+3 y \leq 23 \\ &3 x+2 y \leq 27 \\ & x \ge 0,\ y \ge 0}$ để $f(x;y)=4 x+3 y $ đạt giá trị lớn nhất.\\
		Miền nghiệm của hệ trên là ngũ giác $OABCD$ (kể cả bờ).
		\begin{center}
			\begin{tikzpicture}[line cap=round,line join=round,>=stealth,x=1.0cm,y=1.0cm,scale=.7]
				\clip(-1,-1.5) rectangle (10,7);
				\draw[->] (-1,0) -- (10,0) node[above left]{$x$};
				\draw[->] (0,-1.5) -- (0,7)node[below right]{$y$};
				\draw[smooth,samples=100,domain=-1:10] plot(\x,{(--36-1*\x)/6});
				\draw[smooth,samples=100,domain=-1:10] plot(\x,{(--23-2*\x)/3});
				\draw[smooth,samples=100,domain=-1:10] plot(\x,{(--27-3*\x)/2});
				\fill[pattern=north east lines] (-1,-1.5) -- (-1,7) -- (10,7) -- (10,-1.5) -- cycle;
				\draw [fill=white] (0,0)--(0,6)--(3.33,5.44) -- (7,3) -- (9,0)-- cycle;
				\fill (0,0)node[above left]{$O$}circle(1.2pt) (0,6)node[above left]{$A$}circle(1.2pt) (3.33,5.44)node[above right]{$B$}circle(1.2pt) (7,3)node[above right]{$C$}circle(1.2pt) (9,0)node[below left]{$D$}circle(1.2pt) ;
				% \draw[fill=black] (5.63,0) node[below]{$\dfrac{45}{8}$} (0,5.06)node[left]{$\dfrac{81}{16}$};
				% \draw[dashed] (5.63,0) |- (5.63,5.06) |- (0,5.06);
			\end{tikzpicture}
		\end{center}
		Thay toạ độ các điểm $O(0;0)$, $A\left(0;6\right)$, $B\left(\dfrac{10}{3};\dfrac{49}{9}\right)$, $C(7;3)$, $D(9;0)$ vào $f(x;y)$ ta được 
		$f(0;0)=0$, $f\left(\dfrac{10}{3};\dfrac{49}{9}\right)=\dfrac{89}{3}$, $f(7;3)=37$, $f(9,0)=36$.\\
		Suy ra $f(x;y)$ lớn nhất khi $(x;y)=\left(7;3\right)$. \\
		Như vậy để tiền lãi được nhiều nhất thì sản xuất $7$ tấn sản phẩm $A$ và $3$ tấn sản phẩm $B$.}
\end{ex}
\begin{ex}%[0D4K4-1]
	Biểu thức $F=y-x$ đạt giá trị nhỏ nhất với điều kiện 
	$\heva{
		&-2x+y\le -2 \\
		&x-2y\le 2 \\
		&x+y\le 5 \\
		&x\ge 0}$ tại điểm $S(x;y)$ có toạ độ là
	\choice
	{\True $(4;1)$}
	{$(3;1)$}
	{$(2;1)$}
	{$(1;1)$}
	\loigiai{
		Biểu diễn miền nghiệm của hệ bất phương trình $\heva{
			&-2x+y\le -2 \\
			&x-2y\le 2 \\
			&x+y\le 5 \\
			&x\ge 0}$ trên hệ trục tọa độ như dưới đây:
		\begin{center}
			\begin{tikzpicture}[scale=1, font=\footnotesize, line join=round, line cap=round, >=stealth]
				%=========== Vẽ lưới và hệ trục tọa độ ===========%
				\def \xmin{-1.5} \def \xmax{5.5}
				\def \ymin{-1.5} \def \ymax{4.5} 
				\def \f(#1){2*(#1)-2}
				\def \g(#1){((#1)-2)/2}
				\def \h(#1){(5-(#1))}
				\begin{scope}
					\clip (\xmin,\ymin) rectangle (\xmax,\ymax);
					\draw [] plot [domain=\xmin:\xmax](\x,{\f(\x)}) node [below right=0.5mm]{$f(x)$};
					\draw [] plot [domain=\xmin:\xmax](\x,{\g(\x)});
					\draw [] plot [domain=\xmin:\xmax](\x,{\h(\x)});
				\end{scope}
				\fill[pattern=north east lines] (\xmin,\ymin) -- (\xmax,\ymin) -- (\xmax,\ymax) -- (\xmin,\ymax) -- cycle;
				\draw [fill=white] (0.67,-0.67) -- (2.33,2.66) -- (4,1) -- cycle;
				\draw[-stealth,blue] (\xmin,0)--(\xmax,0) node [below]{$x$};
				\draw[-stealth,blue] (0,\ymin)--(0,\ymax) node[left]{$y$};
				\draw [blue,fill](0,0) circle(1pt) node[below left]{$O$};
				\draw [blue] (0,0.2)-|(0.2,0);
				\draw [fill](0.67,-0.67) circle(1pt) node[above left] {$A$};
				\draw [fill](2.33,2.66) circle(1pt) node[above] {$B$};
				\draw [fill](4,1) circle(1pt) node[above] {$C$};
			\end{tikzpicture}
		\end{center}
		Nhận thấy biết thức $F=y-x$ chỉ đạt giá trị nhỏ nhất tại các điểm $A$, $B$ hoặc $C$.\\
		Chỉ $C(4;1)$ có tọa độ nguyên nên thỏa mãn.
		Vậy $\min F=-3$ khi $x=4$, $y=1$.
	}
\end{ex}

\begin{ex}%[0D4G4-2]
	Giá trị lớn nhất của biểu thức $F(x; y)=x+2y$, với điều kiện $\heva{
		&0\le y\le 4 \\
		&x\ge 0 \\
		&x-y-1\le 0 \\
		&x+2y-10\le 0}$ là
	\choice
	{$6$}
	{$8$}
	{\True $10$}
	{$12$}
	\loigiai{
		Vẽ các đường thẳng $d_1 \colon y=4$;
		$d_2 \colon x-y-1=0$; $d_3 \colon x+2y-10=0$;
		$Ox \colon y=0$; $Oy \colon x=0$.
		\begin{center}
			\begin{tikzpicture}[scale=0.8, font=\footnotesize, line join=round, line cap=round, >=stealth]
				%=========== Vẽ lưới và hệ trục tọa độ ===========%
				\def \xmin{-1.5} \def \xmax{5}
				\def \ymin{-1.5} \def \ymax{4.5} 
				\fill [pattern=north east lines] (\xmin,\ymin) rectangle (\xmax,\ymax);
				\fill [white] (0,0)--(1,0)--(4,3)--(2,4)--(0,4)--cycle;
				\draw[-stealth,blue] (\xmin,0)--(\xmax,0) node [below]{$x$};
				\draw[-stealth,blue] (0,\ymin)--(0,\ymax) node[left]{$y$};
				\draw [fill](0,0) circle(1pt) node[above right]{$O$};
				\draw [blue] (0,0.2)-|(0.2,0);
				\def \f(#1){4}
				\def \g(#1){(#1)-1}
				\def \h(#1){(10-(#1))/2}
				\begin{scope}
					\clip (\xmin,\ymin) rectangle (\xmax,\ymax);
					\draw [] plot [domain=\xmin:\xmax](\x,{\f(\x)}) node [below right=0.5mm]{$f(x)$};
					\draw [] plot [domain=\xmin:\xmax](\x,{\g(\x)});
					\draw [] plot [domain=\xmin:\xmax](\x,{\h(\x)});
				\end{scope}
				\draw (0,4) circle(1pt) node[below right] {$A$};
				\draw (1,0) circle(1pt) node[above] {$B$};
				\draw (4,3) circle(1pt) node[left] {$C$};
				\draw (2,4) circle(1pt) node[below] {$D$};
			\end{tikzpicture}
		\end{center}
		Các đường thẳng trên đôi một cắt nhau tại $A(0;4)$, $O(0;0)$, $B(1;0)$, $C(4;3)$, $D(2;4)$.\\
		Vì điểm $M_0(1;1)$ có toạ độ thoả mãn tất cả các bất phương trình trong hệ nên ta tô đậm các nửa mặt phẳng bờ $d_1$, $d_2$, $d_3$, $Ox$, $Oy$ không chứa điểm $M_0$.\\
		Miền không bị tô đậm là đa giác $OADCB$ kể cả các cạnh (hình bên) là miền nghiệm của hệ bất phương trình đã cho.\\
		Kí hiệu $F(A)=F\left(x_A;y_A\right)=x_A+2y_A$, ta có 
		$F(A)=8$, $F(O)=0$, $F(B)=1$, $F(C)=10$; $F(D)=10$; $0<1<8<10$.\\
		Giá trị lớn nhất cần tìm là $10$.
	}
\end{ex}
\Closesolutionfile{ans}

%%CĐ1:Hệ pt
% \section{HỆ PHƯƠNG TRÌNH BẬC NHẤT BA ẨN}
\subsection{Tóm tắt lí thuyết}
\subsubsection{Hệ phương trình bậc nhất ba ẩn}
\begin{tomtat}
	\begin{itemize}
		\item Phương trình bậc nhất ba ẩn có dạng tổng quát là 
		\[ax+by+cz=d,\]
		trong đó $x$, $y$, $z$ là ba ẩn; $a$, $b$, $c$, $d$ là các hệ số và $a$, $b$, $c$ không đồng thời bằng không.\\
		Mỗi bộ ba số $(x_0;y_0;z_0)$ thỏa mãn $ax_0+by_0+cz_0=d$ gọi là một nghiệm của phương trình bậc nhất ba ẩn đã cho.
		\item Hệ phương trình bậc nhất ba ẩn là hệ gồm một số phương trình bậc nhất ba ẩn. Mỗi nghiệm chung của các phương trình đó được gọi là một nghiệm của hệ phương trình đã cho.
		\item Hệ ba phương trình bậc nhất ba ẩn có dạng tổng quát là
		 \[\heva{&a_1x+b_1y+c_1z=d_1\\&a_2x+b_2y+c_2z=d_2\\&a_3x+b_3y+c_3z=d_3}\]
		 trong đó $x$, $y$, $z$ là ba ẩn; các chữ số còn lại là các hệ số. Ở đây, trong mỗi phương trình, ít nhất một trong các hệ số $a_i$, $b_i$, $c_i$ $(i=1,2,3)$ phải khác $0$.
	\end{itemize}
\end{tomtat}
\begin{vd}%KNTT
	Hệ phương trình nào dưới đây là hệ phương trình bậc nhất ba ẩn? Kiểm tra xem mỗi bộ ba số $(1;1;2)$, $(-1;3;0)$ có phải là một nghiệm của hệ phương trình bậc nhất ba ẩn đó không.
\begin{listEX}[3]
	\item $\heva{&2x-2y+x=-7\\&x+2y-2z=5\\&-x^2+3y-2z=-2;}$
	\item $\heva{&2xy+y=1\\&2x+3y+5z=-2\\&-4x-7y+z-4;}$
	\item $\heva{&x+3y+2z=8\\&2x+2y+z=6\\&3x+y+z=6.}$
\end{listEX}
	\loigiai{
	Hệ phương trình ở câu a) không phải là hệ phương trình bậc nhất ba ẩn vì phương trình thứ ba chứa $x^2$.\\
	Hệ phương trình ở câu b) không phải là hệ phương trình bậc nhất ba ẩn vì phương trình thứ nhất chứa $xy$.\\
	Hệ phương trình ở câu c) là hệ phương trình bậc nhất ba ẩn. 
	\begin{itemize}
		\item Thay $x=1$; $y=1$; $z=2$ vào vế trái của từng phương trình của hệ ở câu c) và so sánh với vế phải, ta được:
		\begin{itemize}
			\item Phương trình thứ nhất: $1+3\cdot 1+2\cdot 2=8$ (thỏa mãn);
			\item Phương trình thứ hai: $2\cdot 1+2\cdot 1+2=6$ (thỏa mãn);
			\item Phương trình thứ ba: $3\cdot 1+1+2=6$ (thỏa mãn).
		\end{itemize}
		Vậy $(1;1;2)$ là một nghiệm của hệ phương trình.
		\item Thay $x=-1$; $y=3$; $z=0$ vào vế trái của từng phương trình của hệ ở câu c) và so sánh với vế phải, ta được:
		\begin{itemize}
			\item Phương trình thứ nhất: $(-1)+3\cdot 3+2\cdot 0=8$ (thỏa mãn);
			\item Phương trình thứ hai: $2\cdot (-1)+2\cdot 3+0=4\neq 6$ (không thỏa mãn).
			\end{itemize}
	Vậy $(-1;3;0)$ không phải nghiệm của hệ phương trình.
	\end{itemize}
	}
\end{vd}
\subsubsection{Giải hệ phương trình bậc nhất bằng ba ẩn bằng phương pháp Gauss}
\begin{tomtat}
	Để giải hệ phương trình dạng tam giác, trước hết ta giải từ phương trình chứa một ẩn, sau đó thay giá trị tìm được của ẩn này vào phương trình chứa hai ẩn để tìm giá trị của ẩn thứ hai, cuối cùng thay các giá trị tìm được vào phương trình còn lại để tìm giá trị của ẩn thứ ba.
\end{tomtat}
\begin{vd}
	Giải hệ phương trình 
	\[\heva{&x+y+3z=10\\&y-z=3\\&2z=4.}\]
	\loigiai{
	Từ phương trình thứ ba ta có $z=2$. Thay $z=2$ vào phương trình thứ hai ta được $y-2=3$ hay $y=5$. Thay $y=5$ và $z=2$ vào phương trình thứ nhất ta được $x+5+3\cdot 2=10$ hay $x=-1$.\\
	Vậy nghiệm của hệ phương trình đã cho là $(-1;5;2)$.
}
\end{vd}
\begin{tomtat}
	Để giải một hệ phương trình bậc nhất ba ẩn, ta đưa hệ đó về một hệ đơn giản hơn (thường có dạng tam giác), bằng cách sử dụng các phép biến đổi sau đây:
	\begin{itemize}
		\item Nhân hai vế của một phương trình của hệ với một số khác $0$.
		\item Đổi vị trí hai phương trình của hệ.
		\item Cộng mỗi vế của một phương trình (sau khi đã nhân với một số khác $0$) với vế tương ứng của phương trình khác để được phương trình mới có số ẩn ít hơn.
	\end{itemize}
Từ đó có thể giải hệ đã cho. Phương pháp này được gọi là \textbf{phương pháp Gauss}
\begin{nx}
	Hệ phương trình bậc nhất ba ẩn có thể có nghiệm duy nhất, vô nghiệm hoặc có vô số nghiệm.
\end{nx}
\end{tomtat}
\begin{vd}
	Giải các hệ phương trình sau bằng phương pháp Gauss
\begin{listEX}[3]
	\item $\heva{&x-y+2z=4\\&2x+y-z=-1\\&x+y+z=5;}$
	\item $\heva{&x-2y+3z=10\\&2x+3y-z=2\\&x+5y-4z=1;}$
	\item $\heva{&2x-y+3z=1\\&x+y+2z=1\\&5x+2y+9z=4.}$
\end{listEX}
	\loigiai{
	\begin{enumerate}
		\item Nhân hai vế của phương trình thứ nhất của hệ với $(-2)$ và cộng với phương trình thứ hai theo từng vế tương ứng ta được hệ phương trình 
		\[\heva{&x-y+2z=4\\&3y-5z=-9\\&x+y+z=5.}\]
		Nhân hai vế của phương trình thứ nhất của hệ với $(-1)$ và cộng với phương trình thứ ba theo từng vế tương ứng ta được hệ phương trình
		\[\heva{&x-y+2z=4\\&3y-5z=-9\\&2y-z=1.}\]
		Nhân cả hai vế của phương trình thứ hai với $\dfrac{-2}{3}$ và cộng với phương trình thứ ba theo từng vế tương ứng ta được hệ phương trình 
		\[\heva{&x-y+2z=4\\&3y-5z=-9\\&\dfrac{7}{3}z=7.}\]
		Từ phương trình thứ ba, ta có $z=3$. Thế vào phương trình thứ hai ta được $3y-5\cdot 3=-9$ hay $y=2$. Thay $z=3$ và $y=2$ vào phương trình đầu tiên, ta được $x-2+2\cdot 3=4$ hay $x=0$.\\
		Vậy nghiệm của hệ phương trình đã cho là $(0;2;3)$.
		\item Nhân hai vế của phương trình thứ nhất của hệ với $(-2)$ và cộng với phương trình thứ hai theo từng vế tương ứng ta được hệ phương trình 
		\[\heva{&x-2y+3z=10\\&7y-7z=-18\\&x+5y-4z=1}\]
		Nhân hai vế của phương trình thứ nhất của hệ với $(-1)$ và cộng với phương trình thứ ba theo từng vế tương ứng ta được hệ phương trình
		\[\heva{&x-2y+3z=10\\&7y-7z=-18\\&7y-7z=-9}\]
		Từ hai phương trình thứ hai và thứ ba, suy ra $-18=-9$, điều này vô lí. \\
		Vậy hệ phương trình đã cho vô nghiệm.
		\item Đổi chỗ phương trình thứ nhất và phương trình thứ hai ta được hệ phương trình 
		\[\heva{&x+y+2z=1\\&2x-y+3z=1\\&5x+2y+9z=4.}\]
		Nhân hai vế của phương trình thứ nhất của hệ với $(-2)$ và cộng với phương trình thứ hai theo từng vế tương ứng ta được hệ phương trình 
		\[\heva{&x+y+2z=1\\&-3y-z=-1\\&5x+2y+9z=4.}\]
		Nhân hai vế của phương trình thứ nhất của hệ với $(-5)$ và cộng với phương trình thứ ba theo từng vế tương ứng ta được hệ phương trình 
		\[\heva{&x+y+2z=1\\&-3y-z=-1\\&-3y-z=-1.}\]
		Nhận thấy phương trình thứ hai và thứ ba của hệ giống nhau, ta được hệ tương đương dạng hình thang
		\[\heva{&x+y+2z=1\\&-3y-z=-1.}\]
		Rút $z$ theo $y$ từ phương trình thứ hai của hệ ta được $z=-3y+1$. Thế vào phương trình thứ nhất ta được $x+y+2(-3y+1)=1$ hay $x=5y-1$.\\
		Vậy hệ phương trình đã cho có vô số nghiệm và tập nghiệm của hệ là \[ S=\left\{(5y-1;y;-3y+1)\ \middle | \ y\in\mathbb{R}\right\}.\]
	\end{enumerate}	
}
\end{vd}
\begin{vd}%ứng dụng
	Ba bạn Lan, Anh và Khoa đi chợ mua trái cây. Bạn Lan mua $2$ kí cam và $3$ kí ổi hết $295$ nghìn đồng, bạn Khoa mua $4$ kí táo và $1$ kí ổi hết $345$ nghìn đồng và bạn Anh mua $2$ kí táo, $3$ kí cam và $1$ kí ổi hết $355$ nghìn đồng. Hỏi giá một kí mỗi loại cam, táo và ổi là bao nhiêu? 
	\loigiai{
Gọi $x$, $y$, $z$ (nghìn đồng) lần lượt là giá của một kí mỗi loại cam, táo và ổi (điều kiện $x,y,z\geq 0$).\\
	Bạn Lan mua $2$ kí cam và $3$ kí ổi hết $295$ nghìn đồng nên ta có \[2x+3z=295.\]
	Bạn Khoa mua $4$ kí táo và $1$ kí ổi hết $345$ nghìn đồng nên ta có \[4y+z=345.\]
	bạn Anh mua $2$ kí táo, $3$ kí cam và $1$ kí ổi hết $355$ nghìn đồng nên ta có \[3x+2y+z=355.\]
	Do đó, ta có hệ phương trình bậc nhất ba ẩn
	\[\heva{&2x+3z=295\\&4y+z=345\\&3x+2y+z=355.}\]
	Giải hệ phương trình trên ta được $x=50$, $y=70$ và $z=65$.\\
	Vậy giá của một kí cam là $50$ nghìn đồng, giá của một kí táo là $70$ nghìn đồng và giá của một kí ổi là $65$ nghìn đồng.
}
\end{vd}
\subsubsection{Tìm nghiệm của hệ phương trình bậc nhất ba ẩn bằng máy tính cầm tay}
\begin{tomtat}
	Ta có thể dùng máy tính cầm tay để giải hệ phương trình bậc nhất ba ẩn. Sau khi mở máy, ta lần lượt thực hiện các thao tác sau:
	\begin{itemize}
		\item Vào chương trình giải hệ phương trình  nhất ba ẩn, ấn
		\begin{itemize}
			\item Đối với máy tính CASIO fx-570VN PLUS: \fbox{MODE} \fbox{$5$} \fbox{$2$}. 
			\item Đối với máy tính CASIO fx-580VNX: \fbox{MENU} \fbox{$9$} \fbox{$1$} \fbox{$3$}. 
		\end{itemize}
	\item Nhập các hệ số để giải hệ phương trình.
	\end{itemize}
\end{tomtat}
\begin{vd}
	Dùng máy tính cầm tay tìm nghiệm của các hệ phương trình sau:
	\begin{listEX}[3]
	\item $\heva{&x-y+z=-3\\&3x+2y+3z=6\\&2x-y-4z=3;}$
	\item $\heva{&x-3y+z=5\\&-2x+y+2z=5\\&x+2y-3z=2;}$
	\item $\heva{&5x+y-4z=2\\&3x+3y-2z=4\\&x-y-z=-1.}$
	\end{listEX}
\loigiai{
Vào chương trình giải hệ phương trình  nhất ba ẩn, ấn
\begin{itemize}
	\item Đối với máy tính CASIO fx-570VN PLUS: \fbox{MODE} \fbox{$5$} \fbox{$2$}. 
	\item Đối với máy tính CASIO fx-580VNX: \fbox{MENU} \fbox{$9$} \fbox{$1$} \fbox{$3$}. 
\end{itemize}
\begin{enumerate}
	\item Nhập các hệ số để giải hệ phương trình:
	\begin{itemize}
		\item Nhập hệ số của phương trình thứ nhất: \fbox{$1$} \fbox{$=$} \fbox{$-$} \fbox{$1$} \fbox{$=$} \fbox{$1$} \fbox{$=$} \fbox{$-$} \fbox{$3$} \fbox{$=$}
		\item Nhập hệ số của phương trình thứ hai: \fbox{$3$} \fbox{$=$} \fbox{$2$} \fbox{$=$} \fbox{$3$} \fbox{$=$} \fbox{$6$} \fbox{$=$}
		\item Nhập hệ số của phương tình thứ ba: \fbox{$2$} \fbox{$=$} \fbox{$-$} \fbox{$1$} \fbox{$=$} \fbox{$-$} \fbox{$4$} \fbox{$=$} \fbox{$3$} \fbox{$=$}
	\end{itemize}
Ấn tiếp phím \fbox{$=$}, ta thấy màn hình hiện $x=1$.\\
Ấn tiếp phím \fbox{$=$}, ta thấy màn hình hiện $y=3$.\\
Ấn tiếp phím \fbox{$=$}, ta thấy màn hình hiện $z=-1$.\\
Vậy nghiệm của hệ phương trình là $(1;3;-1)$.
\item Nhập các hệ số để giải hệ phương trình:
	\begin{itemize}
	\item Nhập hệ số của phương trình thứ nhất: \fbox{$1$} \fbox{$=$} \fbox{$-$} \fbox{$3$} \fbox{$=$} \fbox{$1$} \fbox{$=$} \fbox{$5$}  \fbox{$=$}
	\item Nhập hệ số của phương trình thứ hai: \fbox{$-$} \fbox{$2$} \fbox{$=$} \fbox{$1$} \fbox{$=$} \fbox{$2$} \fbox{$=$} \fbox{$5$} \fbox{$=$}
	\item Nhập hệ số của phương tình thứ ba: \fbox{$1$} \fbox{$=$} \fbox{$2$} \fbox{$=$} \fbox{$-$} \fbox{$3$} \fbox{$=$} \fbox{$2$} \fbox{$=$} 
\end{itemize}
Ấn tiếp phím \fbox{$=$}, ta thấy màn hình hiện No Solution.\\
Vậy phương trình đã cho vô nghiệm.
\item Nhập các hệ số để giải hệ phương trình:
	\begin{itemize}
	\item Nhập hệ số của phương trình thứ nhất: \fbox{$5$} \fbox{$=$} \fbox{$1$} \fbox{$=$} \fbox{$-$} \fbox{$4$} \fbox{$=$} \fbox{$2$}  \fbox{$=$}
	\item Nhập hệ số của phương trình thứ hai: \fbox{$3$} \fbox{$=$} \fbox{$3$} \fbox{$=$} \fbox{$-$} \fbox{$2$} \fbox{$=$} \fbox{$4$} \fbox{$=$}
	\item Nhập hệ số của phương tình thứ ba: \fbox{$1$} \fbox{$=$} \fbox{$-$} \fbox{$1$} \fbox{$=$} \fbox{$-$} \fbox{$1$} \fbox{$=$} \fbox{$-$} \fbox{$1$} \fbox{$=$}
\end{itemize}
Ấn tiếp phím \fbox{$=$}, ta thấy màn hình hiện Infinite Solution.\\
Vậy phương trình đã cho có vô số nghiệm.
\end{enumerate}
}
\end{vd}
\subsection{Bài tập luyện tập}
%Bài tập phần 1
\begin{bt}%KNTT
Hệ nào dưới đây là hệ phương trình bậc nhất ba ẩn? Kiểm tra xem bộ ba số $(-3;2;-1)$ có phải là nghiệm của hệ phương trình bậc nhất ba ẩn đó không.
\begin{listEX}[2]
	\item $\heva{&x+2y-3z=1\\&2x-3y+7z=15\\&3x^2-4y+z=-3;}$
	\item $\heva{&-x+y+z=4\\&2x+y-3z=-1\\&3x-2z=-7.}$
\end{listEX}
	\loigiai{
		Hệ phương trình ở câu a) không phải hệ phương trình bậc nhất ba ẩn vì phương trình thứ ba chứa $x^2$.\\
		Hệ phương trình ở câu b) là hệ phương trình bậc nhất ba ẩn. Thay $x=-3$; $y=2$; $z=-1$ vào vế trái của từng phương trình của hệ ở câu c) và so sánh với vế phải, ta được:
		\begin{itemize}
			\item Phương trình thứ nhất: $-(-3)+2+(-1)=4$ (thỏa mãn);
			\item Phương trình thứ hai: $2\cdot (-3)+2-3\cdot (-1)=-1$ (thỏa mãn);
			\item Phương trình thứ ba: $3\cdot (-3)-2\cdot(-1)=-7$ (thỏa mãn).
		\end{itemize}
		Vậy $(-3;2;-1)$ là một nghiệm của hệ phương trình.
	}
\end{bt}
\begin{bt}%CTST
	Hệ phương trình nào dưới đây là hệ phương trình bậc nhất ba ẩn? Mỗi bộ ba số $(1;5;2)$, $(1;1;1)$ và $(-1;2;3)$ có là nghiệm của hệ phương trình bậc nhất ba ẩn đó không?
	\begin{listEX}[2]
		\item $\heva{&4x-2y+z=5\\&4xz-5y+2z=-7\\&-x+3y+2z=3;}$
		\item $\heva{&x+2z=5\\&2x-y+z=-1\\&3x-2y=-7.}$
			\end{listEX}
		\loigiai{
			Hệ phương trình ở câu a) không phải là hệ phương trình bậc nhất ba ẩn vì phương trình thứ hai chứa $xz$.\\
			Hệ phương trình ở câu b) là hệ phương trình bậc nhất ba ẩn.
			\begin{itemize}
				\item Thay $x=1$; $y=5$; $z=2$ vào vế trái của từng phương trình của hệ ở câu c) và so sánh với vế phải, ta được:
				\begin{itemize}
					\item Phương trình thứ nhất: $1+2\cdot 2=5$ (thỏa mãn);
					\item Phương trình thứ hai: $2\cdot 1-5+2=-1$ (thỏa mãn);
					\item Phương trình thứ ba: $3\cdot 1-2\cdot 5=-7$ (thỏa mãn).
				\end{itemize}
				Vậy $(1;5;2)$ là một nghiệm của hệ phương trình.
				\item Thay $x=1$; $y=1$; $z=1$ vào vế trái của từng phương trình của hệ ở câu c) và so sánh với vế phải, ta được phương trình thứ nhất: $1+2\cdot 1=3\neq 5$ (không thỏa mãn).
				Vậy $(1;1;1)$ là nghiệm của hệ phương trình.
				\item Thay $x=-1$; $y=2$; $z=3$ vào vế trái của từng phương trình của hệ ở câu c) và so sánh với vế phải, ta được:
				\begin{itemize}
					\item Phương trình thứ nhất: $(-1)+2\cdot 3=5$ (thỏa mãn);
					\item Phương trình thứ hai: $2\cdot (-1)-2+3=-1$ (thỏa mãn);
					\item Phương trình thứ ba: $3\cdot (-1)-2\cdot 2=-7$ (thỏa mãn).
				\end{itemize}
				Vậy $(-1;2;3)$ là một nghiệm của hệ phương trình.
			\end{itemize}
		}
\end{bt}
%Bài tập phần 2
\begin{bt}%KNTT
	Giải hệ phương trình 
	\[\heva{&2x=3\\&x+y=2\\&2x-2y+z=-1.} \]
\end{bt}
\loigiai{Ta có
	\allowdisplaybreaks
	$$\begin{aligned}
	& \heva{&2x=3\\&x+y=2\\&2x-2y+x=-1}\Leftrightarrow \heva{&x=\dfrac{3}{2}\\&x+y=2\\&2x-2y+z=-1}\Leftrightarrow \heva{&x=\dfrac{3}{2}\\&\dfrac{3}{2}+y=2\\&2x-2y+z=-1}\\
	\Leftrightarrow & 
	\heva{&x=\dfrac{3}{2}\\&y=\dfrac{1}{2}\\&2x-2y+z=-1} \Leftrightarrow
	\heva{&x=\dfrac{3}{2}\\&y=\dfrac{1}{2}\\&2\cdot\dfrac{3}{2}-2\cdot\dfrac{1}{2}+z=-1}\Leftrightarrow
	\heva{&x=\dfrac{3}{2}\\&y=\dfrac{1}{2}\\&z=-3.}
	\end{aligned}$$
	Vậy hệ phương trình đã cho có nghiệm là $\left(\dfrac{3}{2};\dfrac{1}{2};-3\right)$.
}
\begin{bt}
	Giải các hệ phương trình sau:
	\begin{listEX}[3]
		\item $\heva{&2x+y-3z=3\\&x+y+3z=2\\&3x-2y+z=-1;}$
		\item $\heva{&4x+y+3z=-3\\&2x+y-z=1\\&5x+2y=1;}$
		\item $\heva{&x+2z=-2\\&2x+y-z=1\\&4x+y+3z=-3.}$
	\end{listEX}
\loigiai{
\begin{enumerate}
	\item Ta có
	$$\begin{aligned}
	& \heva{&2x+y-3z=3\\&x+y+3z=2\\&3x-2y+z=-1}\Leftrightarrow \heva{&x+y+3z=2\\&2x+y-3z=3\\&3x-2y+z=-1} \Leftrightarrow \heva{&x+y+3z=2\\&-y-9z=-1\\&-5y-8z=-7}\\
	\Leftrightarrow &  \heva{&x+y+3z=2\\&-y-9z=-1\\&37z=-2}\Leftrightarrow \heva{&x+y+3z=2\\&-y-9z=-1\\&z=-\dfrac{2}{37}} \Leftrightarrow \heva{&x+y+3z=2\\&y=\dfrac{55}{37}\\&z=-\dfrac{2}{37}} \Leftrightarrow \heva{&x=\dfrac{25}{37}\\&y=\dfrac{55}{37}\\&z=-\dfrac{2}{37}.}\end{aligned}	$$
	Vậy hệ phương trình có nghiệm là $\left(\dfrac{25}{37};\dfrac{55}{37};-\dfrac{2}{37}\right)$.
	\item Ta có
	$$\begin{aligned}
	& \heva{&4x+y+3z=-3\\&2x+y-z=1\\&5x+2y=1}\Leftrightarrow \heva{&4x+y+3z=-3\\&6x+3y-3z=3\\&5x+2y=1} \\
	\Leftrightarrow& \heva{&4x+y+3z=-3\\&10x+4y=0\\&5x+2y=1}	\Leftrightarrow  \heva{&4x+y+3z=-3\\&5x+2y=0\\&5x+2y=1.}
	\end{aligned}$$
	Từ hai phương trình cuối, ta suy ra $0=1$,  điều này vô lí.\\
	Vậy hệ phương trình đã cho vô nghiệm.
	\item Ta có
	$$\heva{&x+2z=-2\\&2x+y-z=1\\&4x+y+3z=-3} \Leftrightarrow \heva{&x+2z=-2\\&y-5z=5\\&y-5z=5} \Leftrightarrow \heva{&x=-2z-2\\&y=5z+5.}$$
	Vậy hệ phương trình đã cho có vô số nghiệm và tập nghiệm của nó là
	\[S=\left\{(-2z-2;5z+5;z)\ \middle |\ z\in\mathbb{R}\right\}.\]
\end{enumerate}
}
\end{bt}
\begin{bt}%KNTT
	Hà mua văn phòng phẩm cho nhóm bạn cùng lớp gồm Hà, Lan và Minh hết tổng cộng $820$ nghìn đồng. Hà quên không lưu hóa đơn của mỗi bạn, nhưng nhớ được rằng số tiền trả cho Lan ít hơn một nửa số tiền trả cho Hà là 5 nghìn đồng, số tiền trả cho Minh nhiều hơn số tiền trả cho Lan là 210 nghìn đồng. Hỏi mỗi bạn Lan và Minh phải trả cho Hà bao nhiêu tiền?
	\loigiai{
Gọi số tiền mua văn phòng phẩm của Hà, Lan và Minh lần lượt là $x$, $y$, $z$ (nghìn đồng), với điều kiện $x,y,z\geq 0$.\\
Vì tổng số tiền phải trả của cả ba bạn là $820$ nghìn đồng nên 
\[x+y+z=820.\]
Số tiền trả cho Lan ít hơn một nửa số tiền trả cho Hà là 5 nghìn đồng nên  
\[\dfrac{1}{2}x-y=5.\]
Số tiền trả cho Minh nhiều hơn số tiền trả cho Lan là 210 nghìn đồng nên 
\[-y+z=210.\]
Do đó, ta có hệ phương trình
\[\heva{&x+y+z=820\\&\dfrac{1}{2}x-y=5\\&-y+z=210}\Leftrightarrow \heva{&x+y+z=820\\&x=2y+10\\&z=y+210}	\Leftrightarrow \heva{&4y+220=820\\&x=2y+10\\&z=y+210} \Leftrightarrow \heva{&y=150\\&x=310\\&z=360.}\]
Vậy số tiền mua văn phòng phẩm của Hà, Lan và Minh lần lượt là $310$ nghìn đồng, $150$ nghìn đồng và $360$ nghìn đồng.
}
\end{bt}
\begin{bt}%CTST
Giải các hệ phương trình sau bằng phương pháp Gauss:
\begin{listEX}[3]
	\item $\heva{&x-2y=1\\&x+2y-z=-2\\&x-3y+z=3;}$
	\item $\heva{&3x-y+2z=2\\&x+2y-z=1\\&2x-3y+3z=2;}$
	\item $\heva{&x-y+z=0\\&x-4y+2z=-1\\&4x-y+3z=1.}$
\end{listEX}
\loigiai{
\begin{enumerate}
	\item 
	    Ta có
	    \[\begin{aligned}
	   & \heva{&x-2y=1\\&x+2y-z=-2\\&x-3y+z=3} \Leftrightarrow \heva{&x-2y=1\\&4y-z=-3\\&-y+z=2} \Leftrightarrow \heva{&x-2y=1\\&4y-z=-3\\&3z=5}\\
	   \Leftrightarrow & \heva{&x-2y=1\\&4y-z=-3\\&z=\dfrac{5}{3}} \Leftrightarrow \heva{&x-2y=1\\&y=-\dfrac{1}{3}\\&z=\dfrac{5}{3}} \Leftrightarrow \heva{&x=\dfrac{1}{3}\\&y=-\dfrac{1}{3}\\&z=\dfrac{5}{3}.}
	    \end{aligned}\]
	    Vậy hệ phương trình có nghiệm là $\left(\dfrac{1}{3};-\dfrac{1}{3};\dfrac{5}{3}\right)$.
	    \item Ta có 
	    \[\heva{&3x-y+2z=2\\&x+2y-z=1\\&2x-3y+3z=2} \Leftrightarrow \heva{&x+2y-z=1\\&3x-y+2z=2\\&2x-3y+3z=2} \Leftrightarrow \heva{&x+2y-z=1\\&-7y+5z=-1\\&-7x+5z=0.}\]
	    Từ phương trình thứ hai và thứ ba, suy ra $-1=0$, điều này vô lí.\\
	    Vậy phương trình đã cho vô nghiệm.
	    \item Ta có
	    \[ \heva{&x-y+z=0\\&x-4y+2z=-1\\&4x-y+3z=1} \Leftrightarrow
	    \heva{&x-y+z=0\\&-3y+z=-1\\&3y-z=1} \Leftrightarrow \heva{&x-y+z=0\\&z=3y-1} \Leftrightarrow \heva{&x=-2y+1\\&z=3y-1.}\]
	   Vậy hệ phương trình đã cho có vô số nghiệm và tập nghiệm của nó là
	   \[ S=\left\{(-2y+1;y;3y-1) \ \middle |\ y\in \mathbb{R} \right\}.\]
\end{enumerate}
}
\end{bt}
\begin{bt}
	Tìm phương trình của parabol $(P)\colon y=ax^2+bx+c$ $(a\neq 0)$, biết $(P)$ đi qua ba điểm $A(0;-1)$, $B(1;-2)$ và $C(2;-1)$.
	\loigiai{
$(P)$ đi qua $A(0;-1)$ nên $a\cdot 0^2+b\cdot 0 +c=-1$ hay $c=-1$.\\
$(P)$ đi qua $B(1;-2)$ nên $a\cdot 1^2+b\cdot 1+c=-2$ hay $a+b+c=-2$.\\
$(P)$ đi qua $C(2;-1)$ nên $a\cdot 2^2+b\cdot 2+c=-2$ hay $4a+2b+c=-1$.\\
Do đó ta có hệ phương trình 
\[\heva{&c=-1\\&a+b+c=-2\\&4a+2b+c=-1.}\]
Giải hệ này ta được $a=1$; $b=-2$; $c=-1$.\\
Vậy phương trình của $(P)$ là $y=x^2-2x-1$.	
}
\end{bt}
\begin{bt}%CD
	Giải các hệ phương trình sau:
	\begin{listEX}[3]
		\item $\heva{&4x+y-3z=11\\&2x-3y+2z=9\\&x+y+z=-3;}$
		\item $\heva{&x+2y+6z=5\\&-x+y-2z=3\\&x-4y-2z=13;}$
		\item $\heva{&x+y-3z=-1\\&y-z=0\\&-x+2y=1.}$
	\end{listEX}
\loigiai{
\begin{enumerate}
	\item Ta có
	\[\begin{aligned}
	&\heva{&4x+y-3z=11\\&2x-3y+2z=9\\&x+y+z=-3} \Leftrightarrow \heva{&x+y+z=-3\\&2x-3y+2z=9\\&4x+y-3z=11} \Leftrightarrow \heva{&x+y+z=-3\\&-3y-7z=23\\&-5y=15}\\
	 \Leftrightarrow &\heva{&x+y+z=-3\\&-3y-7z=23\\&y=-3} %\Leftrightarrow \heva{&x+y+z=-3\\&-3\cdot(-3)-7z=-2\\&y=-3} 
	 \Leftrightarrow \heva{&x+y+z=-3\\&z=-2\\&y=-3} 
	 %\Leftrightarrow \heva{&x-3-2=-3\\&y=-3\\&z=-2}
	 \Leftrightarrow\heva{&x=2\\&y=-3\\&z=-2.}
	\end{aligned}\]
	Vậy phương trình đã cho có nghiệm là $(2;-3;-2)$.
	\item Ta có 
	\[\heva{&x+2y+6z=5\\&-x+y-2z=3\\&x-4y-2z=13}\Leftrightarrow \heva{&x+2y+6z=5\\&3y+4z=8\\&-6y-8z=8}\Leftrightarrow \heva{&x+2y+6z=5\\&3y+4z=8\\&3y+4z=-4.} \]
	Từ phương trình thứ hai và thứ ba, suy ra $8=-4$, điều này vô lí.\\
	Vậy phương trình đã cho vô nghiệm.
	\item Ta có
	\[\heva{&x+y-3z=-1\\&y-z=0\\&-x+2y=1}\Leftrightarrow \heva{&x+y-3z=-1\\&y-z=0\\&3y-3z=0}\Leftrightarrow  \heva{&x+y-3z=-1\\&y=z}\Leftrightarrow \heva{&x=2z-1\\&y=z.}\]
	Vậy hệ phương trình đã cho có vô số nghiệm và tập nghiệm của nó là
	\[S=\left\{(2z-1;z;z)\ \middle | \ z\in\mathbb{R}\right\}.\]
\end{enumerate}
}
\end{bt}
%Bài tập phần 3
\begin{bt}%CTST
	Sử dụng máy tính cầm tay, tìm nghiệm của các hệ phương trình sau:
	\begin{listEX}[3]
		\item $\heva{&2x+y-z=-1\\&x+3y+2z=2\\&3x+3y-3z=-5;}$
		\item $\heva{&2x-3y+2z=5\\&x+2y-3z=4\\&3x-y-z=2;}$
		\item $\heva{&x-y-z=-1\\&2x-y+z=-1\\&-4x+3y+z=3}$
	\end{listEX}
\loigiai{
\begin{enumerate}
	\item Nghiệm của hệ phương trình là $\left(\dfrac{2}{3};-\dfrac{2}{3};\dfrac{5}{3}\right)$.
	\item Hệ phương trình vô nghiệm.
	\item Hệ phương trình có vô số nghiệm.
\end{enumerate}
}
\end{bt}
\begin{bt}%CD
	Sử dụng máy tính cầm tay để tìm nghiệm của hệ phương trình:
	\[\heva{&2x-3y+4z=-5\\&-4x+5y-z=6\\&3x+4y-3z=7.}\]
	\loigiai{
	Nghiệm của hệ phương trình là $\left(\dfrac{22}{101};\dfrac{131}{101};-\dfrac{39}{101}\right)$.
}
\end{bt}
\begin{bt}%CTST
	Ba bạn Nhân, Nghĩa và Phúc đi vào căng tin của trường. Nhân mua một li trà sữa, một li nước trái cây, hai cái bánh ngọt và trả $90\,000$ đồng. Nghĩa mua một li trà sữa, ba cái bánh ngọt và trả $50\,000$ đồng. Phúc mua một li trà sữa, hai li nước trái cây, ba cái bánh ngọt và trả $140\,000$ đồng. Gọi $x$, $y$, $z$ lần lượt là giá tiền của một li trà sữa, một li nước trái cây và một cái bánh ngọt tại căng tin đó.
	\begin{enumerate}
		\item Lập các hệ thức thể hiện mối liên hệ giữa $x$, $y$ và $z$.
		\item Tìm giá tiền của một li trà sữa, một li nước trái cây và một cái bánh ngọt tại căng tin đó.
	\end{enumerate}
\loigiai{
	\begin{enumerate}
		\item Vì Nhân mua một li trà sữa, một li nước trái cây, hai cái bánh ngọt và trả $90\,000$ đồng nên ta có 
		\[x+y+2z=90000.\]
		Vì Nghĩa mua một li trà sữa, ba cái bánh ngọt và trả $50\,000$ đồng nên ta có
		\[x+3z=50000.\]
		Vì Phúc mua một li trà sữa, hai li nước trái cây, ba cái bánh ngọt và trả $140\,000$ đồng nên ta có
		\[x+2y+3z=140000\]
		Từ đó, ta có hệ phương trình bậc nhất ba ẩn:
		\[\heva{&x+y+2z=90000\\&x+3z=50000\\&x+2y+3z=140000.}\]
		\item Sử dụng máy tính cầm tay để giải hệ phương trình trên, ta được $(35000;45000;5000)$ là nghiệm của hệ phương trình.\\
		Vậy giá tiền của một li trà sữa, một li nước trái cây và một cái bánh ngọt tại căng tin đó lần lượt là $35\,000$ đồng, $45\,000$ đồng và $5\,000$ đồng.
	\end{enumerate}
}
\end{bt}
\begin{bt}%KNTT
	Tại một quốc gia, khoảng 400 loài động vật nằm trong danh sách các loài có nguy cơ tuyệt chủng. Các nhóm động vật có vú, chim và cá chiếm $55\%$ các loài có nguy cơ tuyệt chủng. Nhóm chim chiếm nhiều hơn $0{,}7\%$ so với nhóm cá, nhóm cá chiếm nhiều hơn $1{,}5\%$ so với động vật có vú. Hỏi mỗi nhóm động vật có vú, chim và cá chiếm bao nhiều phần trăm trong các loài có nguy cơ tuyệt chủng?
	\loigiai{
		Gọi $x$, $y$, $z$ lần lượt là số phần trăm nhóm động vật có vú, chim và cá có nguy cơ tuyệt chủng (điều kiện $x,y,z\geq0$).\\
		Các nhóm động vật có vú, chim và cá chiếm $55\%$ các loài có nguy cơ tuyệt chủng nên
		\[x+y+z=55.\]
		Nhóm chim chiếm nhiều hơn $0{,}7\%$ so với nhóm cá nên
		\[y-z=0{,}7.\]
		Nhóm cá chiếm nhiều hơn $1{,}5\%$ so với động vật có vú nên
		\[z-x=1{,}5.\]
		Do đó, ta có hệ phương trình
		\[\heva{&x+y+z=55\\&y-z=0{,}7\\&-x+z=1{,}5.}\]
		Giải hệ phương trình này, ta được $x=17{,}1$; $y=19{,}3$ và $z=18{,}6$.\\
		Vậy nhóm động vật có vú chiếm $17{,}1\%$; nhóm chim chiếm $19{,}3\%$ và nhóm cá chiếm $18{,}6\%$ các loài có nguy cơ tuyệt chủng.
	}
\end{bt}
% \section{HỆ PHƯƠNG TRÌNH BẬC NHẤT BA ẨN}
\subsection{Tóm tắt lý thuyết}
(Phần này GV bỏ thầy cô hoàn thành tài để được tài liệu hoàn thiện)
\subsection{Bài tập rèn luyện}
\begin{bt}
	Kiểm tra xem mỗi bộ số $(x;\,y;\,z)$ đã cho có là nghiệm của hệ phương trình tương ứng hay không.
	\begin{enumerate}[a)]
		\item $\heva{&x+3y+2z=1\\&5x-y+3z=16\\&-3x+7y+z=-14}$ \qquad $(0;\,3;\,-2)$, $(12;\,5;\,-13)$, $(1;\,-2;\,3)$;
		\item $\heva{&3x-y+4z=-10\\&-x+y+2z=6\\&2x-y+z=-8}$ \qquad $(-2;\,4;\,0)$, $(0;\,-3;\,10)$, $(1;\,-1;\,5)$;
		\item $\heva{&x+y+z=100\\&5x+3y+\dfrac{1}{3}z=100}$ \qquad $(4;\,18;\,78)$, $(8;\,11;\,81)$, $(12;\,4;\,84)$.
	\end{enumerate}
	\loigiai{
		\begin{enumerate}[a)]
			\item Bộ $(1;\,-2;\,3)$ là nghiệm của hệ phương trình $\heva{&x+3y+2z=1\\&5x-y+3z=16\\&-3x+7y+z=-14.}$			
			\item Bộ $(-2;\,4;\,0)$ là nghiệm của hệ phương trình $\heva{&3x-y+4z=-10\\&-x+y+2z=6\\&2x-y+z=-8.}$
			\item Cả $3$ bộ $(4;\,18;\,78)$, $(8;\,11;\,81)$, $(12;\,4;\,84)$ là nghiệm của hệ phương trình $\heva{&x+y+z=100\\&5x+3y+\dfrac{1}{3}z=100.}$
		\end{enumerate}  
	}
\end{bt}

\begin{bt}
	Giải hệ phương trình
	\begin{enumerate}[a)]
		\item $\heva{&x-2y+4z=4\\&3y-z=2\\&2z=-10.}$
		\item $\heva{&4x+3y-5z=-7\\&2y=4\\&y+z=3.}$
		\item $\heva{&x+y+2z=0\\&3x+2y=2\\&x=10.}$
	\end{enumerate}
	\loigiai{
		\begin{enumerate}[a)]
			\item Ta có $\heva{&x-2y+4z=4\\&3y-z=2\\&2z=-10} \Leftrightarrow \heva{&x=22\\&y=-1\\&z=-5.}$
			\item Ta có $\heva{&4x+3y-5z=-7\\&2y=4\\&y+z=3} \Leftrightarrow \heva{&x=-2\\&y=2\\&z=1.}$
			\item Ta có $\heva{&x+y+2z=0\\&3x+2y=2\\&x=10} \Leftrightarrow \heva{&z=2\\&y=-14\\&x=10.}$
		\end{enumerate}
	}
\end{bt}

\begin{bt}
	Giải hệ phương trình
	\begin{enumerate}[a)]
		\item $\heva{&3x-y-2z=5\\&2x+y+3z=6\\&6 x-y-4z=9.}$
		\item $\heva{&x+2y+6z=5\\&-x+y-2z=3\\&x-4y-2z=1.}$
		\item $\heva{&x+4y-2z=2\\&-3x+y+z=-2\\&5x+7y-5z=6.}$
	\end{enumerate}
	\loigiai{
		\begin{enumerate}[a)]
			\item Ta có $\heva{&3x-y-2z=5\\&2x+y+3z=6\\&6 x-y-4z=9} \Leftrightarrow \heva{&3x-y-2z=5\\&5y+13z=8\\&y=-1} \Leftrightarrow \heva{&x=2\\&y=-1\\&z=1.}$
			\item Ta có $\heva{&x+2y+6z=5\\&-x+y-2z=3\\&x-4y-2z=1} \Leftrightarrow \heva{&x+2y+6z=5\\&3y+4z=8\\&3y+4z=3} \Leftrightarrow \heva{&x+2y+6z=5\\&3y+4z=8\\&0=5}$ suy ra hệ phương trình vô nghiệm.
			\item Ta có $\heva{&x+4y-2z=2\\&-3x+y+z=-2\\&5x+7y-5z=6} \Leftrightarrow \heva{&x+4y-2z=2\\&13y-5z=4\\&13y-5z=4}$ suy ra hệ phương trình vô số nghiệm $(x;\,y;\,z)$ thỏa mãn $\heva{&x+4y-2z=2\\&13y-5z=4.}$
		\end{enumerate}
	}
\end{bt}

\begin{bt}
	Tìm số đo ba góc của một tam giác, biết tổng số đo của góc thứ nhất và góc thứ hai bằng hai lần số đo của góc thứ ba, số đo của góc thứ nhất lớn hơn số đo của góc thứ ba là $20^{\circ}$.
	\loigiai{
		Gọi $3$ góc của tam giác lần lượt là $x$, $y$, $z$.\\
		Ta có $\heva{&x+y+z=180^\circ\\&x+y-2z=0^\circ\\&x-z=20^\circ} \Leftrightarrow \heva{&x+y+z=180^\circ\\&z=60^\circ\\&x-z=20^\circ} \Leftrightarrow \heva{&x=80^\circ\\&y=40^\circ\\&z=60^\circ.}$
	}
\end{bt}

\begin{bt}
	Bác Thanh chia số tiền $1$ tỉ đồng của mình cho ba khoản đầu tư. Sau một năm, tổng số tiền lãi thu được là $84$ triệu đồng. Lãi suất cho ba khoản đầu tư lần lượt là $6\%$, $8\%$, $15\%$ và số tiền đầu tư cho khoản thứ nhất bằng tổng số tiền đầu tư cho khoản thứ hai và thứ ba. Tính số tiền bác Thanh đầu tư cho mỗi khoản.
	\loigiai{
		Gọi $3$ khoản đầu tư lần lượt là $x$, $y$, $z$ triệu đồng.\\
		Ta có $\heva{&x+y+z=1000\\&6x+8y+15z=8400\\&x-y-z=0} \Leftrightarrow \heva{&x+y+z=1000\\&2y+9z=2400\\&y+z=500} \Leftrightarrow \heva{&x+y+z=1000\\&2y+9z=2400\\&z=200} \Leftrightarrow \heva{&x=500\\&y=300\\&z=200.}$
	}
\end{bt}

\begin{bt}
	Khi một quả bóng được đá lên, nó sẽ đạt độ cao nào đó rồi rơi xuống. Biết quỹ đạo chuyển động của quả bóng là một parabol và độ cao $h$ của quả bóng được tính bởi công thức $h=\dfrac{1}{2}at^{2}+v_{0}t+h_{0}$, trong đó độ cao $h$ và độ cao ban đầu $h_{0}$ được tính bằng mét, $t$ là thời gian của chuyển động tính bằng giây, $a$ là gia tốc của chuyển động tính bằng $\mathrm{m}/\mathrm{s}^{2}$, $v_{0}$ là vận tốc ban đầu được tính bằng $\mathrm{m}/\mathrm{s}$. Tìm $a$, $v_{0}$, $h_{0}$ biết sau $0,5$ giây quả bóng đạt được độ cao $6,075 \mathrm{~m}$; sau $1$ giây quả bóng đạt độ cao $8,5 \mathrm{~m}$; sau $2$ giây quả bóng đạt độ cao $6\mathrm{~m}$.
	\loigiai{
		Ta có $\heva{&h(0,5)=6,075\\&h(1)=8,5\\&h(2)=6} \Leftrightarrow \heva{&0,125a+0,5v_{0}+h_{0}=6,075\\&0,5a+v_{0}+h_{0}=8,5\\&2a+2v_{0}+h_{0}=6} \Leftrightarrow \heva{&a=-9,8\\&v_{0}=12,2\\&h_{0}=1,2.}$
	}
\end{bt}

\begin{bt}
	Một cửa hàng bán đồ nam gồm áo sơ mi, quần âu và áo phông. Ngày thứ nhất bán được $22$ áo sơ mi, $12$ quần âu và $18$ áo phông, doanh thu là $12580000$ đồng. Ngày thứ hai bán được $16$ áo sơ mi, $10$ quần âu và $20$ áo phông, doanh thu là $10800000$ đồng. Ngày thứ ba bán được $24$ áo sơ mi, $15$ quần âu và $12$ áo phông, doanh thu là $12960000$ đồng. Hỏi giá bán mỗi áo sơ mi, mỗi quần âu và mỗi áo phông là bao nhiêu? Biết giá từng loại trong ba ngày không thay đổi.
	\loigiai{
		Gọi giá bán mỗi áo sơ mi, mỗi quần âu và mỗi áo phông lần lượt là $x$, $y$, $z$ triệu đồng.\\
		Ta có $\heva{&22x+12y+18z=12580000\\&16x+10y+20z=10800000\\&24x+15y+12z=12960000} \Leftrightarrow \heva{&x=250000\\&y=320000\\&z=180000.}$
	}
\end{bt}

\begin{bt}
	Ba nhãn hiệu bánh quy là $A$, $B$, $C$ được cung cấp bởi một nhà phân phối. Với tỉ lệ thành phần dinh dưỡng theo khối lượng, bánh quy nhãn hiệu $A$ chứa $20\%$ protein, bánh quy nhãn hiệu $B$ chứa $28\%$ protein và bánh quy nhãn hiệu $C$ chứa $30\%$ protein. Một khách hàng muốn mua một đơn hàng như sau
	\begin{itemize}
		\item Mua tổng cộng $224$ cái bánh quy bao gồm cả ba nhãn hiệu $A$, $B$, $C$.
		\item Lượng protein trung bình của đơn hàng này (gồm cả ba nhãn hiệu $A$, $B$, $C$) là $25\%$.
		\item Lượng bánh nhãn hiệu $A$ gấp đôi lượng bánh nhãn hiệu $C$.
	\end{itemize}
	Tính lượng bánh quy mỗi loại mà khách hàng đó đặt mua.
	\loigiai{
		Gọi lượng bánh quy mỗi loại mà khách hàng đó đặt mua lần lượt là $x$, $y$, $z$.\\
		Ta có $\heva{&x+y+z=224\\&20x+28y+30z=25\cdot224\\&x-2z=0} \Leftrightarrow \heva{&x=96\\&y=80\\&z=48.}$
	}
\end{bt}

\begin{bt}
	Sử dụng máy tính cầm tay để tìm nghiệm của các hệ phương trình sau
	\begin{enumerate}[a)]
		\item $\heva{&-x+2y-3z=2\\&2x+y+2z=-3\\&-2x-3y+z=5.}$
		\item $\heva{&x-3y+z=1\\&5y-4z=0\\&x+2y-3z=-1.}$
		\item $\heva{&x+y-3z=-1\\&3x-5y-z=-3\\&-x+4y-2z=1.}$
	\end{enumerate}
	\loigiai{
		\begin{enumerate}[a)]
			\item $\heva{&-x+2y-3z=2\\&2x+y+2z=-3\\&-2x-3y+z=5} \Leftrightarrow \heva{&x=-4\\&y=\dfrac{11}{7}\\&z=\dfrac{12}{7}.}$
			\item $\heva{&x-3y+z=1\\&5y-4z=0\\&x+2y-3z=-1}$ hệ phương trình vô nghiệm.
			\item $\heva{&x+y-3z=-1\\&3x-5y-z=-3\\&-x+4y-2z=1}$ hệ phương trình vô số nghiệm.
		\end{enumerate}	
	}
\end{bt}


%%%%%%%%%%%%%%%%%%%%%%%%%%%%%%%%%%%%%%%%%%

%\subsection{CHÂN TRỜI SÁNG TẠO}

%\subsubsection{BÀI TẬP}

\begin{bt}
	Trong các hệ phương trình sau, hệ nào là hệ phương trình bậc nhất ba ẩn? Mỗi bộ ba số $(-1;\,2;\,1)$, $(-1,5;\,0,25;\,-1,25)$ có là nghiệm của hệ phương trình bậc nhất ba ẩn đó không?
	\begin{enumerate}[a)]
		\item $\heva{&3x-2y+z=-6\\&-2x+y+3z=7\\&4x-y+7z=1.}$
		\item $\heva{&5x-2y+3z=4\\&3x+2y-z=2\\&x-3y+2z=-1.}$
		\item $\heva{&2x-4y-3z=\dfrac{-1}{4}\\&3x+8y-4z=\dfrac{5}{2}\\&2x+3y-2z=\dfrac{1}{4}.}$
	\end{enumerate}
	\loigiai{
		\begin{enumerate}[a)]
			\item Bộ $(-1;\,2;\,1)$ là nghiệm của hệ phương trình $\heva{&3x-2y+z=-6\\&-2x+y+3z=7\\&4x-y+7z=1.}$			
			\item Cả $2$ bộ $(-1;\,2;\,1)$, $(-1,5;\,0,25;\,-1,25)$ không là nghiệm của hệ phương trình $\heva{&5x-2y+3z=4\\&3x+2y-z=2\\&x-3y+2z=-1.}$
			\item Bộ $(-1,5;\,0,25;\,-1,25)$ là nghiệm của hệ phương trình $\heva{&2x-4y-3z=\dfrac{-1}{4}\\&3x+8y-4z=\dfrac{5}{2}\\&2x+3y-2z=\dfrac{1}{4}.}$
		\end{enumerate}
	}
\end{bt}

\begin{bt}
	Giải các hệ phương trình sau bằng phương pháp Gauss
	\begin{enumerate}[a)]
		\item $\heva{&2x+3y=4\\&x-3y=2\\&2x+y-z=3.}$
		\item $\heva{&x+y+z=2\\&x+3y+2z=8\\&3x-y+z=4.}$
		\item $\heva{&x-y+5z=-2\\&2x+y+4z=2\\&x+2y-z=4.}$
	\end{enumerate}
	\loigiai{
		\item Ta có $\heva{&2x+3y=4\\&x-3y=2\\&2x+y-z=3} \Leftrightarrow \heva{&2x+3y=4\\&y=0\\&2y+z=1} \Leftrightarrow \heva{&x=2\\&y=0\\&z=1.}$
		\item Ta có $\heva{&x+y+z=2\\&x+3y+2z=8\\&3x-y+z=4} \Leftrightarrow \heva{&x+y+z=2\\&2y+z=6\\&4y+2z=2}$ hệ phương trình vô nghiệm.
		\item Ta có $\heva{&x-y+5z=-2\\&2x+y+4z=2\\&x+2y-z=4} \Leftrightarrow \heva{&x-y+5z=-2\\&y-2z=2\\&y-2z=2}$ hệ phương trình vô số nghiệm. 
	}
\end{bt}

\begin{bt}
	Sử dụng máy tính cầm tay, tìm nghiệm của các hệ phương trình sau
	\begin{enumerate}[a)]
		\item $\heva{&x-5z=2\\&3x+y-4z=3\\&-x+2y+z=-1.}$
		\item $\heva{&2x-y+z=3\\&x+2y-z=1\\&3x+y-2z=2.}$
		\item $\heva{&x+2y-z=1\\&2x+y-2z=2\\&4x-7y-4z=4.}$
	\end{enumerate}
	\loigiai{
		\begin{enumerate}[a)]
			\item $\heva{&x-5z=2\\&3x+y-4z=3\\&-x+2y+z=-1} \Leftrightarrow \heva{&x=\dfrac{17}{26}\\&y=-\dfrac{1}{26}\\&z=-\dfrac{7}{26}.}$
			\item $\heva{&2x-y+z=3\\&x+2y-z=1\\&3x+y-2z=2} \Leftrightarrow \heva{&x=\dfrac{6}{5}\\&y=\dfrac{2}{5}\\&z=1.}$
			\item $\heva{&x+2y-z=1\\&2x+y-2z=2\\&4x-7y-4z=4}$ hệ vô số nghiệm.
		\end{enumerate}
	}
\end{bt}

\begin{bt}
	Tìm phương trình của parabol $(P)\colon y=ax^{2}+bx+c \quad (a\neq 0)$, biết
	\begin{enumerate}[a)]
		\item Parabol $(P)$ có trục đối xứng $x=1$ và đi qua hai điểm $A(1;\,-4)$, $B(2;\,-3)$;
		\item Parabol $(P)$ có đỉnh $I\left(\dfrac{1}{2};\,\dfrac{3}{4}\right)$ và đi qua điểm $M(-1;\,3)$.
	\end{enumerate}
	\loigiai{
		Ta có $\heva{&-\dfrac{a}{2b}=1\\&a\cdot 1^2+b\cdot 1+c=-4\\&a\cdot 2^2+b\cdot 2+c=-3} \Leftrightarrow \heva{&a+2b=0\\&a+b+c=-4\\&4a+2b+c=-3} \Leftrightarrow \heva{&a=\dfrac{2}{5}\\&b=-\dfrac{1}{5}\\&c=-\dfrac{21}{5}.}$
	}
\end{bt}

\begin{bt}
	Một đại lí bán ba loại gas $A$, $B$, $C$ với giá bán mỗi bình gas lần lượt là $520000$ đồng, $480000$ đồng, $420000$ đồng. Sau một tháng, đại lí đã bán được $1299$ bình gas các loại với tổng doanh thu đạt $633960000$ đồng. Biết rằng trong tháng đó, đại lí bán được số bình gas loại $B$ bằng một nửa tổng số bình gas loại $A$ và $C$. Tính số bình gas mỗi loại mà đại lí bán được trong tháng đó.
	\loigiai{
		Gọi $x$, $y$, $z$ lần lượt là số bình ga loại $A$, $B$, $C$.\\
		Ta có $\heva{&x+y+z=1299\\&52x+48y+42z=63396\\&x-2y+z=0} \Leftrightarrow \heva{&x=624\\&y=433\\&z=242.}$
	}
\end{bt}


%%%%%%%%%%%%%%%%%%%%%%%%%%%%%%%%%%%%%%%%%
%\subsection{KẾT NỐI TRI THỨC}

%\subsubsection{BÀI TẬP}

\begin{bt}
	Hệ nào dưới đây là hệ phương trình bậc nhất ba ẩn? Kiểm tra xem bộ ba số $(2;\,0;\,-1)$ có phải là nghiệm của hệ phương trình bậc nhất ba ẩn đó không.
	\begin{enumerate}[a)]
		\item $\heva{&x-2z=4\\&2x+y-z=5\\&-3x+2y=-6.}$
		\item $\heva{&x-2y+3z=7\\&2x-y^{2}+z=2\\&x+2y=-1.}$
	\end{enumerate}
	\loigiai{
		\begin{enumerate}[a)]
			\item Ta có $\heva{&x-2z=4\\&2x+y-z=5\\&-3x+2y=-6}$ là hệ phương trình bậc nhất ba ẩn và bộ $(2;\,0;\,-1)$ không phải là nghiệm.
			\item Ta có $\heva{&x-2y+3z=7\\&2x-y^{2}+z=2\\&x+2y=-1}$ không là hệ phương trình bậc nhất ba ẩn.
		\end{enumerate}
	}
\end{bt}

\begin{bt}
	Giải các hệ phương trình sau
	\begin{enumerate}[a)]
	\item $\heva{&x-y-3z=20\\&x-z=3\\&x+3z=-7.}$
	\item $\heva{&x-y-3z=20\\&x-z=3\\&x+3z=-7.}$
	\end{enumerate}
	\loigiai{
		\begin{enumerate}[a)]
			\item Ta có $\heva{&2x-y-z=20\\&x+y=-5\\&x=10} \Leftrightarrow \heva{&x=10\\&y=-15\\&z=15.}$
			\item $\heva{&x-y-3z=20\\&x-z=3\\&x+3z=-7} \Leftrightarrow \heva{&x=\dfrac{1}{2}\\&y=-12\\&z=-\dfrac{5}{2}.}$
		\end{enumerate}
	}
\end{bt}

\begin{bt}
	Giải các hệ phương trình sau bằng phương pháp Gauss
	\begin{enumerate}[a)]
		\item $\heva{&2x-y-z=2\\&x+y=3\\&x-y+z=2.}$
		\item $\heva{&3x-y-z=2\\&x+2y+z=5\\&-x+y=2.}$
		\item $\heva{&x-3y-z=-6\\&2x-y+2z=6\\&4x-7y=-6.}$
		\item $\heva{&x-3y-z=-6\\&2x-y+2z=6\\&4x-7y=3.}$
		\item $\heva{&3x-y-7z=2\\&4x-y+z=11\\&-5x-y-9z=-22.}$
		\item $\heva{&2x-3y-4z=-2\\&5x-y-2z=3\\&7x-4y-6z=1.}$
	\end{enumerate}
	Kiểm tra lại kết quả tìm được bằng cách sử dụng máy tính cầm tay.
	\loigiai{
		\begin{enumerate}[a)]
			\item Ta có $\heva{&2x-y-z=2\\&x+y=3\\&x-y+z=2} \Leftrightarrow \heva{&2x-y-z=2\\&3y+z=4\\&y-3z=-2} \Leftrightarrow \heva{&2x-y-z=2\\&3y+z=4\\&z=1} \Leftrightarrow \heva{&x=2\\&y=1\\&z=1.}$
			\item Ta có $\heva{&3x-y-z=2\\&x+2y+z=5\\&-x+y=2} \Leftrightarrow \heva{&3x-y-z=2\\&7y+4z=13\\&2y-z=8} \Leftrightarrow \heva{&3x-y-z=2\\&7y+4z=13\\&z=-2} \Leftrightarrow \heva{&x=-1\\&y=-3\\&z=-2.}$
			\item Ta có $\heva{&x-3y-z=-6\\&2x-y+2z=6\\&4x-7y=-6} \Leftrightarrow \heva{&x-3y-z=-6\\&5y+4z=18\\&5y+4z=18}$ hệ phương trình vô số nghiệm.
			\item Ta có $\heva{&x-3y-z=-6\\&2x-y+2z=6\\&4x-7y=3} \Leftrightarrow \heva{&x-3y-z=-6\\&5y+4z=18\\&5y+4z=27}$ hệ phương trình vô nghiệm.
			\item Ta có $\heva{&3x-y-7z=2\\&4x-y+z=11\\&-5x-y-9z=-22} \Leftrightarrow \heva{&3x-y-7z=2\\&y+31z=25\\&2y+18z=-76} \Leftrightarrow \heva{&3x-y-7z=2\\&y+31z=25\\&31z=24} \Leftrightarrow \heva{&x=\dfrac{87}{31}\\&y=1\\&z=\dfrac{24}{31}.}$
			\item Ta có $\heva{&2x-3y-4z=-2\\&5x-y-2z=3\\&7x-4y-6z=1} \Leftrightarrow \heva{&2x-3y-4z=-2\\&13y+16z=16\\&13y+16z=16}$ hệ vô số nghiệm.
		\end{enumerate}
	}
\end{bt}

\begin{bt}
	Ba người cùng làm việc cho một công ty với vị trí lần lượt là quản lí kho, quản lí văn phòng và tài xế xe tải. Tổng tiền lương hằng năm của người quản lí kho và người quản lí văn phòng là $164$ triệu đồng, còn của người quản lí kho và tài xế xe tải là $156$ triệu đồng. Mỗi năm, người quản lí kho lĩnh lương nhiều hơn tài xế xe tải $8$ triệu đồng. Hỏi lương hằng năm của mỗi người là bao nhiêu?
	\loigiai{
		Gọi $x$, $y$, $z$ lần lượt là lương hằng năm của người quản lí kho, quản lí văn phòng và tài xế xe tải.\\
		Ta có $\heva{&x+y=164\\&x+z=156\\&x-z=8} \Leftrightarrow \heva{&x=82\\&y=82\\&z=74.}$
	}
\end{bt}

\begin{bt}
	Năm ngoái, người ta có thể mua ba mẫu xe ô tô của ba hãng $X$, $Y$, $Z$ với tổng số tiền là $2,8$ tỉ đồng. Năm nay, do lạm phát, để mua ba chiếc xe đó cần $3,018$ tỉ đồng. Giá xe ô tô của hãng $X$ tăng $8\%$, của hãng $Y$ tăng $5\%$ và của hãng $Z$ tăng $12\%$. Nếu trong năm ngoái giá chiếc xe của hãng $Y$ thấp hơn $200$ triệu đồng so với giá chiếc xe của hãng $X$ thì giá của mỗi chiếc xe trong năm ngoái là bao nhiêu?
	\loigiai{
		Gọi $x$, $y$, $z$ lần lượt là giá của ba mẫu xe ô tô của ba hãng $X$, $Y$, $Z$.\\
		Ta có $\heva{&x+y+z=2,8\\&1,08x+1,05y+1,12z=3,018\\&x-y=0,2} \Leftrightarrow \heva{&x=1,2\\&y=1\\&z=0,6.}$
	}
\end{bt}

\begin{bt}
	Cho hệ ba phương trình bậc nhất ba ẩn $\heva{&a_{1}x+b_{1}y+c_{1}z=d_{1}\\&a_{2}x+b_{2}y+c_{2}z=d_{2}\\&a_{3}x+b_{3}y+c_{3}z=d_{3}.}$
	\begin{enumerate}[a)]
		\item Giả sử $\left(x_{0};\,y_{0};\,z_{0}\right)$ và $\left(x_{1};\,y_{1};\,z_{1}\right)$ là hai nghiệm phân biệt của hệ phương trình trên. Chứng minh rằng \linebreak $\left(\dfrac{x_{0}+x_{1}}{2};\,\dfrac{y_{0}+y_{1}}{2};\,\dfrac{z_{0}+z_{1}}{2}\right)$ cũng là một nghiệm của hệ.
		\item Sử dụng kết quả của câu a) chứng minh rằng, nếu hệ phương trình bậc nhất ba ẩn có hai nghiệm phân biệt thì nó sẽ có vô số nghiệm.
	\end{enumerate}
	\loigiai{
		\begin{enumerate}[a)]
			\item Ta có $\left(x_{0};\,y_{0};\,z_{0}\right)$ và $\left(x_{1};\,y_{1};\,z_{1}\right)$ là hai nghiệm phân biệt của hệ phương trình $\heva{&a_{1}x+b_{1}y+c_{1}z=d_{1}\\&a_{2}x+b_{2}y+c_{2}z=d_{2}\\&a_{3}x+b_{3}y+c_{3}z=d_{3}}$ suy ra
				\[\heva{&a_{1}x_{0}+b_{1}y_{0}+c_{1}z_{0}=d_{1}\\&a_{2}x_{0}+b_{2}y_{0}+c_{2}z_{0}=d_{2}\\&a_{3}x_{0}+b_{3}y_{0}+c_{3}z_{0}=d_{3}} \quad \text{và} \quad \heva{&a_{1}x_{1}+b_{1}y_{1}+c_{1}z_{1}=d_{1}\\&a_{2}x_{1}+b_{2}y_{1}+c_{2}z_{1}=d_{2}\\&a_{3}x_{1}+b_{3}y_{1}+c_{3}z_{1}=d_{3}.}\]
				Cộng vế với vế các phương trình tương ứng trong hai hệ và chia hai vế cho $2$ ta được
				\[\heva{&a_{1}\left(\dfrac{x_{0}+x_{1}}{2}\right)+b_{1}\left(\dfrac{y_{0}+y_{1}}{2}\right)+c_{1}\left(\dfrac{z_{0}+z_{1}}{2}\right)=d_{1}\\&a_{2}\left(\dfrac{x_{0}+x_{1}}{2}\right)+b_{2}\left(\dfrac{y_{0}+y_{1}}{2}\right)+c_{2}\left(\dfrac{z_{0}+z_{1}}{2}\right)=d_{2}\\&a_{3}\left(\dfrac{x_{0}+x_{1}}{2}\right)+b_{3}\left(\dfrac{y_{0}+y_{1}}{2}\right)+c_{3}\left(\dfrac{z_{0}+z_{1}}{2}\right)=d_{3}.}\]
				Vậy $\left(\dfrac{x_{0}+x_{1}}{2};\,\dfrac{y_{0}+y_{1}}{2};\,\dfrac{z_{0}+z_{1}}{2}\right)$ cũng là một nghiệm của hệ.
			\item Nếu hệ phương trình bậc nhất ba ẩn có hai nghiệm phân biệt thì ta sử dụng kết quả của câu a) suy ra hệ sẽ có thêm nghiệm thứ ba, thứ tư, $\ldots$ Do đó hệ sẽ có vô số nghiệm. 
		\end{enumerate}
	}
\end{bt}
% \section{ỨNG DỤNG HỆ PHƯƠNG TRÌNH BẬC NHẤT BA ẨN}
\subsection{Các dạng toán và ví dụ}
\begin{dang}{Giải bài toán bằng cách lập hệ phương trình}
	Để giải bài toán bằng cách lập hệ phương trình bậc nhất ba ẩn, ta thực hiện các bước sau:\\
	\textbf{Bước 1: }Lập hệ phương trình.\\
	Chọn ẩn là những đại lượng chưa biết.\\
	Dựa trên ý nghĩa của các đại lượng chưa biết, đặt điều kiện cho ẩn.\\
	Dựa vào dữ kiện của bài toán, lập hệ phương trình với các ẩn.\\
	\textbf{Bước 2:} Giải hệ phương trình.\\
	\textbf{Bước 3: }Kiểm tra điều kiện của nghiệm và kết luận.
	\end{dang}
\begin{vd}%[0D3B3-5]
	Một cửa hàng bán áo sơ mi, quần nam và váy nữ. Ngày thứ nhất bán được $21$ áo, $21$ quần và $18$ váy, doanh thu là $5349000$ đồng. Ngày thứ hai bán được $16$ áo, $24$ quần và $12$ váy, doanh thu là $5600000$ đồng. Ngày thứ ba bán được $24$ áo, $15$ quần và $12$ váy, doanh thu là $5259000$ đồng. Khi đó giá bán mỗi áo, mỗi quần và mỗi váy bằng bao nhiêu?	\loigiai
	{		Gọi giá tiền mỗi chiếc áo, quần và váy lần lượt là $x$, $y$, $z$ \big(đồng, $ x$, $y$, $z > 0$\big) \\
		Theo đề bài ta có hệ phương trình $ \heva{& 21x + 21y + 18z = 5349000 \\ & 16x + 24y + 12z = 5600000 \\ & 24x + 15y + 12z = 5259000 } \Leftrightarrow \heva{& x = 98000 \\ & y = 125000 \\ & z = 86000.} $ \\
		Vậy giá tiền mỗi chiếc áo, quần và váy lần lượt là $ 98000$, $125000$, $86000 $ \big(đồng\big).
	}
\end{vd}
\begin{vd}%[0D3B3-5]
	Ba cô Lan, Hương và Thúy cùng thêu một loại áo giống nhau. Số áo của Lan thêu trong $1$ giờ ít hơn tổng số áo của Hương và Thúy thêu trong $1$ giờ là $5$ áo. Tổng số áo của Lan thêu trong $4$ giờ và Hương thêu trong $3$ giờ nhiều hơn số áo của Thúy thêu trong $5$ giờ là $30$ áo. Số áo của Lan thêu trong $2$ giờ cộng với số áo của Hương thêu trong $5$ giờ và số áo của Thúy thêu trong $3$ giờ tất cả được $76$ áo. Hỏi trong $1$ giờ mỗi cô thêu được mấy áo?
	\loigiai{
		Gọi $ x, y, x $	lần lượt là số áo của Lan, Hương, Thúy thêu được trong $ 1 $ giờ. Điều kiện là $ x, y, z \in \mathbb{N^*} $. \\
		Theo đề bài ta có hệ phương trình: $$ \heva{& x = y + z -5 \\& 4x + 3y - 5z = 30 \\& 2x + 5y + 3z =76} \Leftrightarrow \heva{& x - y - z = 5 \\& 4x + 3y - 5z = 30 \\& 2x + 5y + 3z =76} \Leftrightarrow \heva{& x = 9 \\& y = 8 \\& z = 6.}$$ 
		Vậy trong $1$ giờ Lan thêu được $9$ áo, Hương thêu được $8$ áo, Thúy thêu được  $6$ áo.
	}
\end{vd}
\begin{vd}%[0D3B3-5]
	Ba phân số đều có tử số là $1$ và tổng của ba phân số đó bằng $1$. Hiệu của phân số thứ nhất và phân số thứ hai bằng phân số thứ ba, còn tổng của phân số thứ nhất và phân số thứ hai bằng $5$ lần phân số thứ ba. Khi đó tích ba phân số đó bằng bao nhiêu?
	\loigiai
	{
		Gọi ba phân số lần lượt là $ \dfrac{1}{x}$, $\dfrac{1}{y}$, $\dfrac{1}{z}$. \\
		Theo đề bài ta có hệ phương trình: $$\heva{& \dfrac{1}{x} + \dfrac{1}{y} + \dfrac{1}{z} = 1 \\ & \dfrac{1}{x} - \dfrac{1}{y} = \dfrac{1}{z} \\ & \dfrac{1}{x} + \dfrac{1}{y} = \dfrac{5}{z}} \Leftrightarrow \heva{& \dfrac{1}{x} = \dfrac{1}{2} \\ & \dfrac{1}{y} = \dfrac{1}{3} \\ & \dfrac{1}{z} =\dfrac{1}{6}.} $$ 
		Vậy $ \dfrac{1}{x} \cdot \dfrac{1}{y} \cdot \dfrac{1}{z} = \dfrac{1}{36}$.
	}
\end{vd}
\begin{dang}{Ứng dụng trong giải bài toán Vật Lý, Hóa Học, Sinh Học.}
	
	\end{dang}
\begin{vd}%[0D3B3-5]
\immini{Cho mạch điện như hình vẽ. Điện trở $R_1=200 \Omega$; hiệu điện thế giữa hai điểm $A$ và $B$ giữ không đổi là $U_{AB}=6 \mathrm{~V}$. Điện trở của ampe kế bằng không, vôn kế có điện trở hữu hạn $R_V$ chưa biết. Số chỉ ampe kế là $10 \mathrm{~mA}$, số chỉ của vôn kế là $4{,}5 \mathrm{~V}$. Tìm giá trị điện trở $R_2$ và điện trở của vôn kế $R_V$?(chiều dòng điện qua ampe kế từ $C$ đến $D$)}{
\begin{tikzpicture}
	\tikzset{
		every node/.style={font=\footnotesize,transform shape},
		%
		ampe/.pic={\draw (0,0)circle(0.25) (0,0)node[align=center]{A};},
		%
		vol/.pic={\draw (0,0)circle(0.2) (0,0)node[align=center]{V};},
		% Điện tro
		dientro2/.pic={\draw (-.1,-.2)--(-.1,.2)--(.1,.2)--(.1,-.2)--cycle %(0,0)node[align=center]{Hình\\chữ nhật}
			;},
		% Điện tro
		dientro/.pic={\draw (-.2,-.1)--(-.2,.1)--(.2,.1)--(.2,-.1)--cycle %(0,0)node[align=center]{Hình\\chữ nhật}
			;},
		% Điện tro
		dientro1/.pic={\draw (-.2,-.1)--(-.2,.1)--(.2,.1)--(.2,-.1)--cycle %(0,0)node[align=center]{Hình\\chữ nhật}
			;},
		% Điện tro
		dientro3/.pic={\draw (-.2,-.1)--(-.2,.1)--(.2,.1)--(.2,-.1)--cycle %(0,0)node[align=center]{Hình\\chữ nhật}
			;},
	}
	%
	\draw
	(0,0)pic[local bounding box =am]{ampe}
	(2,-.5)pic[local bounding box =dt3]{dientro3}
	(-1,2)pic[local bounding box =dt]{dientro}
	(1.5,2)pic[local bounding box =dt1]{dientro1}
	(0,1)pic[local bounding box =dt2]{dientro2}
	(-.5,-.5)pic[local bounding box =vol]{vol}
	;
	%
	\draw[](0,-.5)--(am)--(dt2)node[right]{$R_2$}--(0,2)node[above]{$C$}--(dt1)--(3,2)--(3,-.5)--(dt3)node[above]{$R_2$}--(0,-.5)node[below]{$D$}--(vol)--(-2,-.5)--(-2,2)--(dt)node[above]{$R_1$}--(0,2);
	\draw[-latex](0,2)--(0.5,2)node[above]{$I_3$};
	\draw[-latex](-2,-.5)--(-1.5,-.5)node[above]{$I_4$};
	\draw[-latex](-2,2)--(-1.5,2)node[above]{$I_1$};
	\draw[-latex](0.5,-.5)--(1,-.5)node[above]{$I_5$};
	\draw(-2,-.5)--(-2,-1.5)--(-.25,-1.5)node[above left]{$A$}circle(1pt) (.25,-1.5)node[above right]{$B$}circle(1pt)--(3,-1.5)--(3,-.5) (0,-1.5)node[below]{$U$} (0.3,-1.5)node[below]{+} (-0.3,-1.5)node[below]{-} (dt1)node[above]{$R_1$};
\end{tikzpicture}
}
\loigiai{Ta có $	U_{D B}=U-U_{A D}=1{,}5 \mathrm{~V}$.\\
	Do dòng điện đi theo chiều từ $C$ tới $D$
	$$
	\begin{aligned}
		U_{A D} &=U_{A C}+U_{C D} \\
		U_{A D} &=I_1 R_1+I_2 R_2 \\
		4{,}5=& I_1 \cdot 200+0{,}01 \cdot R_2 \quad(1)
			\end{aligned}$$
và $$			\begin{aligned}
		& U_{D B}=U_{D C}+U_{C B} \\
		& U_{D B}=-I_2 R_2+I_3 R_1 \\
		& 1,5=0,01 \cdot R_2+I_3 \cdot 200\quad(2)
	\end{aligned}
	$$
	Tại nút $C$ có:
	$	I_1=I_2+I_3=0{,}01+I_3$.\quad(3)\\
Từ  (1), (2), (3) có hệ ba phương trình ba ẩn $I_1$, $I_3$, $R_2$
	$$
	\left\{\begin{array} { l } 
		{ 4 , 5 = I _ { 1 } \cdot 2 0 0 + 0 , 0 1 \cdot R _ { 2 } } \\
		{ 1 , 5 = 0 , 0 1 \cdot R _ { 2 } + I _ { 3 } \cdot 2 0 0 } \\
		{ I _ { 1 } = I _ { 2 } + I _ { 3 } = 0 , 0 1 + I _ { 3 } }
	\end{array} \Leftrightarrow \left\{\begin{array}{l}
		I_1=0{,}02 \mathrm{~A} \\
		I_3=0{,}01 \mathrm{~A} \\
		R_2=50 \Omega.
	\end{array}\right.\right.
	$$
Vậy $I_5=\dfrac{U_{D B}}{R_2}=\dfrac{1{,}5}{50}=0{,}03 \mathrm{~A}$; $I_4=I_5-I_2=0{,}02 \mathrm{~A}$;
		$ R_V=\dfrac{U_{A D}}{I_4}=\dfrac{4{,}5}{0{,}02}=250 \Omega$.
}
	\end{vd}
\begin{vd}%[0D3B3-5]
	Cân bằng phương trình sau
$\mathrm{CH}_4+\mathrm{O}_2\rightarrow\mathrm{CO}_2+\mathrm{H}_2 \mathrm{O}$.
\loigiai{
Để cân bằng phản ứng, ta cần tìm các số nguyên dương $x, y, z, t$ sao cho $$x \mathrm{CH}_4+\mathrm{yO}_2\rightarrow\mathrm{zCO}_2+\mathrm{tH}_2 \mathrm{O}.$$
Đối với mỗi nguyên tố, số nguyên tử ở vế phải và vế trái phải bằng nhau, ta có\\
Carbon: $x=z$;
Hydrogen: $4 x=2 t$;
Oxygen: $2 y=2 z+t$.\\
Từ đó ta có hệ phương trình:
$\heva{&x-z=0\\&4 x-2 t=0\\&2 y-2 z-t=0.}$\\
Giải hệ, ta có nghiệm tổng quát
$x=\dfrac{t}{2}$, $y=t$, $z=\dfrac t2$, với $t$ là số thực tùy ý.\\
Số nguyên dương $t$ nhỏ nhất để $x, y, z, t$ là nguyên dương là $t=2$.\\
Do đó cân bằng được phương trình phản ứng:
$\mathrm{CH}_4+2 \mathrm{O}_2\rightarrow\mathrm{CO}_2+2 \mathrm{H}_2 \mathrm{O}$.
}
	\end{vd}
\begin{vd}%[0D3B3-5]
	Cần 3 thành phần khác nhau $\mathrm{A}, \mathrm{B}$ và $\mathrm{C}$, để sản xuất một lượng hợp chất hóa học nào đó. $\mathrm{A}, \mathrm{B}$ và C phải được hòa tan trong nước một cách riêng biệt trước khi chúng kết hợp lại để tạo ra hợp chất hóa học. Biết rằng nếu kết hợp dung dịch chứa $\mathrm{A}$ với tỉ lệ $1{,}5 \mathrm{~g} / \mathrm{cm}$ với dung dịch chứa $\mathrm{B}$ với tỉ lệ $3{,}6 \mathrm{~g} / \mathrm{cm}$ và dung dịch chứa $\mathrm{C}$ với tỉ lệ $5{,}3 \mathrm{~g} / \mathrm{cm}$ thì tạo ra $25{,}07 \mathrm{~g}$ hợp chất hóa học đó. Nếu tỉ lệ của $\mathrm{A}, \mathrm{B}, \mathrm{C}$ trong phương án này thay đổi thành tương ứng $2{,}5$; $4{,}3$ và $2{,}4 \mathrm{~g} / \mathrm{cm}$ (trong khi thể tích là giống nhau), khi đó $22{,}36 \mathrm{~g}$ chất hóa học sẽ được tạo ra. Cuối cùng, nếu tỉ lệ tương ứng là $2{,}7$; $5{,}5$ và $3{,}2 \mathrm{~g} / \mathrm{cm}$, thì sẽ tạo ra $28{,}14 \mathrm{~g}$ hợp chất. Thể tích của dung dịch chứa $\mathrm{A}$, $\mathrm{B}$ và $\mathrm{C}$ là bao nhiêu?
\loigiai{
	Gọi $x$, $y$, $z$ tương ứng là thể tích $(\mathrm{cm})$ của phương án chứa $\mathrm{A}$, $\mathrm{B}$ và $\mathrm{C}$. \\
	Khi đó $1{,}5x$ là khối lượng của $\mathrm{A}$ trong trường hợp đầu, $3{,}6y$ là khối lượng của $\mathrm{B}$ và $5{,}3 z$ là khối lượng của $\mathrm{C}$.\\
	 Cộng lại với nhau, ba khối lượng này sẽ tạo ra $25{,}07$ g.\\
	 Do đó: $1{,}5 x+3{,}6 y+5{,}32=25{,}07$.\\
	Tương tự cho hai trường hợp còn lại, ta có hệ phương trình :
	$$
	\left\{\begin{array}{l}
		1{,}5 x+3{,}6 y+5{,}32=25{,}07 \\
		2{,}5 x+4{,}3 y+2{,}4 z=22{,}36 \\
		2{,}7 x+5{,}5 y+3{,}22=28{,}14.
	\end{array}\right.
	$$
	Giải hệ trên cho ta nghiệm là $x=1{,}5$; $y=3{,}1$; $z=2{,}2$.
	}
	\end{vd}
\begin{dang}{Ứng dụng trong giải bài toán kinh tế}
	
	\end{dang}
\begin{vd}%[0D3B3-5]
	Xét thị trường chè, cà phê và ca cao. Gọi $x, y$ và $z$ lần lượt là giá của $1 \mathrm{~kg}$ chè, $1 \mathrm{~kg}$ cà phê và $1 \mathrm{~kg}$ ca cao (đơn vị: nghìn đồng, $x \geq 0$, $y \geq 0$, $z \geq 0$ ). Các lượng cung và lượng cầu của mỗi sản phẩm được cho như bảng sau:\begin{center}
	\begin{tabular}{|l|c|c|}
		\hline \multicolumn{1}{|c|}{ Sản phẩm } & Lượng cung & Lượng cầuu \\
		\hline Chè & $Q_{S_1}=-380+x+y$ & $Q_{D_1}=350-x-z$ \\
		\hline Cà phê & $Q_{S_2}=-405+x+2 y-z$ & $Q_{D_2}=760-2 y-z$ \\
		\hline Ca cao & $Q_{S_3}=-350-2 x+3 z$ & $Q_{D_3}=145-x+y-z$ \\
		\hline
	\end{tabular}
\end{center}
	Tìm giá của mỗi kilôgam chè, cà phê và ca cao để thị trường cân bằng.
	\loigiai{
Thị trường cân bằng khi $\left\{\begin{array}{l}Q_{S_1}=Q_{D_1} \\ Q_{S_2}=Q_{D_2} \\ Q_{S_3}=Q_{D_3}\end{array}\right.$ hay $\left\{\begin{array}{l}2 x+y+z=730 \\ x+4 y=1165 \\ -x-y+4 z=495.\end{array}\right.$.\\
Giải hệ phương trình, ta được: $x=125, y=260, z=220$.\\
Vậy giá của mỗi kilôgam chè, cà phê, ca cao lần lượt là 125000 đồng, 260000 đồng, 220000 đồng.	
}
	\end{vd}
\begin{vd}%[0D3B3-5]
	Một đoàn xe tải chở $290$ tấn xi măng cho một công trình xây đập thuỷ điện. Đoàn xe có $57$ chiếc gồm ba loại, xe chở 3 tấn, xe chở 5 tấn và xe chở $7{,}5$ tấn. Nếu dùng tất cả xe $7{,}5$ tấn chở ba chuyến thì được số xi măng bằng tổng số xi măng do xe 5 tấn chở ba chuyến và xe 3 tấn chở hai chuyến. Hỏi số xe mỗi loại?
	\loigiai{
	Gọi $x$ là số xe tải chở 3 tấn, $y$ là số xe tải chở 5 tấn và $z$ là số xe tải chở 7,5 tấn. Điều kiện $x, y, z$ nguyên dương.
	Theo giả thiết của bài toán ta có
	$$
\heva{
		x + y + z & = 5 7  \\
		3 x + 5 y + 7{,}5 z  & = 2 9 0  \\
		2 2 {,}5 z  &  = 6 x + 1 5 y 
	} \Leftrightarrow \heva{
		x+y+z & =57 \\
		3 x+5 y+7{,}5 z & =290 \\
		-2 x-5 y+7{,}5 z & =0.
}	$$
	Giải hệ phương trình ta được $x=20$, $y=19$, $z=18$.\\
	Vậy có $20$ xe chở 3 tấn, $19$ xe chở $5$ tấn, $18$ xe chở $7{,}5$ tấn
}
\end{vd}
\subsection{Bài tập rèn luyện}
%CTST
\begin{bt}%[0D3B3-5]
Ba vận động viên Hùng, Dũng và Mạnh tham gia thi đấu nội dung ba môn phối hợp: chạy, bơi và đạp xe, trong đó tốc độ trung bình của họ trên mỗi chặng đua được cho ở bảng dưới đây.\begin{center}
\begin{tabular}{|c|c|c|c|}
	\hline \multirow{2}{*}{ Vận động viên } & \multicolumn{3}{|c|}{ Tốc độ trung bình (km/h) } \\
	\cline { 2 - 4 } & Chạy & Bơi & Đạp xe \\
	\hline Hùng & $12{,}5$ & $3{,}6$ & $48$ \\
	\hline Dũng & $12$ & $3{,}75$ & $45$ \\
	\hline Mạnh & $12{,}5$ & $4$ & $45$ \\
	\hline
\end{tabular}
\end{center}
Biết tổng thời gian thi đấu ba môn phối hợp của Hùng là $1$ giờ $1$ phút $30$ giây, của Dũng là $1$ giờ $3$ phút $40$ giây và của Mạnh là $1$ giờ $1$ phút $55$ giây. Tính cự li của mỗi chặng đua.
\loigiai{Đổi: $1$ giờ $1$ phút $30$ giây $=\dfrac{41}{40}$ h, $1$ giờ $3$ phút $40$ giây $=\dfrac{191}{180}$ h, $1$ giờ $1$ phút $55$ giây $=\dfrac{743}{720}$ h. Gọi cự li của mỗi chặng đua chạy, bơi và đạp xe lần lượt là $x$, $y$, $z$ (km).\\
	Dựa vào bảng trên ta có hệ phương trình: $\left\{\begin{array}{l}\dfrac{x}{12{,}5}+\dfrac{y}{3{,}6}+\dfrac{z}{48}=\dfrac{41}{40} \\ \dfrac{x}{12}+\dfrac{y}{3{,}75}+\dfrac{z}{45}=\dfrac{191}{180} \\ \dfrac{x}{12{,}5}+\dfrac{y}{4}+\dfrac{z}{45}=\dfrac{743}{720}.\end{array}\right.$\\
	Giải hệ này ta được $x=5$, $y=0{,}75$, $z=20$.\\
	Vậy cự li của mỗi chặng đua chạy, bơi và đạp xe lần lượt là $5$ km; $0{,}75$ km; $20$ km.
}
\end{bt}
\begin{bt}%[0D3B3-5]
	Một nhà hoá học có ba dung dịch cùng một loại acid nhưng với nồng độ khác nhau là $10 \%$, $20 \%$ và $40 \%$. Trong một thí nghiệm, để tạo ra $100 \mathrm{~ml} $ dung dịch nồng độ $18 \%$, nhà hoá học đã sử dụng lượng dung dịch nồng độ $10 \%$ gấp bốn lần lượng dung dịch nồng độ $40 \%$. Tính số mililít dung dịch mỗi loại mà nhà hoá học đó đã sử dụng trong thí nghiệm này.
	\loigiai{Gọi $x$, $y$, $z$ lần lượt là số mililít dung dịch acid có nồng độ $10 \%$, $20 \%$ và $40 \%$ đã sử dụng trong thí nghiệm $(x \geq 0$, $y \geq 0$, $z \geq 0)$.\\
		Theo đề bài, ta có hệ phương trình:
		$\left\{\begin{array}{l}x+y+z=100 \\ x=4 z \\ 0{,}1 x+0{,}2 y+0{,}4 z=0{,}18\cdot100\end{array}\right.$\\
		 hay $\left\{\begin{array}{l}x+y+z=100 \\ x-4 z=0 \\ x+2 y+4 z=180 .\end{array}\right.$\\
		Giải hệ phương trình, ta được $x=40 ; y=50$ và $z=10$.\\
		Vậy lượng acid có nồng độ $10 \%, 20 \%$ và $40 \%$ cần sử dụng lần lượt là $40 \mathrm{~ml}$, $50 \mathrm{~ml}$ và $10 \mathrm{ml}$.
	}
	\end{bt}
\begin{bt}%[0D3B3-5]
	Ba loại tế bào $A$, $B$, $C$ thực hiện số lần nguyên phân lần lượt là $3$, $4$, $7$ và tổng số tế bào con tạo ra là $480$. Biết rằng khi chưa thưc hiện nguyên phân, số tế bào loại $B$ bằng tổng số tế bào loại $A$ và loại $C$. Sau khi thực hiện nguyên phân, tổng số tế bào con loại $A$ và loại $C$ được tạo ra gấp năm lần số tế bào con loại $B$ được tạo ra. Tính số tế bào con mỗi loại lúc ban đầu.
	\loigiai{
	Gọi $x$, $y$, $z$ lần lượt là số tế bào loại $A$, $B$, $C$ lúc ban đầu $(x$, $y$, $z \in \mathbb{N})$.\\
	Ba loại tế bào $A$, $B$, $C$ thực hiện số lần nguyên phân lần lượt là $3$, $4$, $7$ và tổng số tế bào con tạo ra là $480$, ta có $x \cdot 2^3+y \cdot 2^4+z \cdot 2^7=480$ hay $x+2 y+16 z=60$.\\
	Khi chưa thực hiện nguyên phân số tế bào loại $B$ bằng tổng số tế bào loại $A$ và loại $C$, ta có $y=x+z$.\\
	Sau khi thực hiện nguyên phân, tổng số tế bào con loại $A$ và loại $C$ được tạo ra gấp năm lần số tế bào con loại $B$ được tạo ra, ta có $x \cdot 2^3+z \cdot 2^7=5 \cdot y \cdot 2^4$ hay $x+16 z=10 y$.\\
	Từ đó, ta có hệ phương trình
	$$
	\left\{\begin{array}{l}
		x+2 y+16 z=60 \\
		x-y+z=0 \\
		x-10 y+16 z=0.
	\end{array}\right.
	$$
	Giải hệ phương trình, ta được: $x=2$, $y=5$, $z=3$.
	Vậy số tế bào $A$, $B$, $C$ lúc ban đầu lần lượt là $2$, $5$, $3$ tế bào.
}
	\end{bt}
\begin{bt}%[0D3B3-5]
\immini{Cho sơ đồ mạch điện như Hình 2. Tính các cường độ dòng điện $I_1$, $I_2$ và $I_3$.}{\begin{tikzpicture}
		\tikzset{
			every node/.style={font=\footnotesize,transform shape},
			%
			ampe/.pic={\draw (0,0)circle(0.25) (0,0)node[align=center]{A};},
			%
			vol/.pic={\draw (0,0)circle(0.2) (0,0)node[align=center]{V};},
			% Điện tro
			dientro2/.pic={\draw (-.1,-.2)--(-.1,.2)--(.1,.2)--(.1,-.2)--cycle %(0,0)node[align=center]{Hình\\chữ nhật}
				;},
			% Điện tro
			dientro/.pic={\draw (-.2,-.1)--(-.2,.1)--(.2,.1)--(.2,-.1)--cycle %(0,0)node[align=center]{Hình\\chữ nhật}
				;},
			% Điện tro
			dientro1/.pic={\draw (-.2,-.1)--(-.2,.1)--(.2,.1)--(.2,-.1)--cycle %(0,0)node[align=center]{Hình\\chữ nhật}
				;},
		% nguồn điện
		nguondien1/.pic={\draw  (-.1,-.1)--(.1,-.1) (-.2,.1)--(.2,.1)
			%(0,0)node[align=center]{Hình\\chữ nhật}
			;},
	% nguồn điện
nguondien2/.pic={\draw  (-.1,-.1)--(.1,-.1) (-.2,.1)--(.2,.1)
	%(0,0)node[align=center]{Hình\\chữ nhật}
	;},
			% Điện tro
			dientro3/.pic={\draw (-.2,-.1)--(-.2,.1)--(.2,.1)--(.2,-.1)--cycle %(0,0)node[align=center]{Hình\\chữ nhật}
				;},
		}
		%
		\draw

	%	(2,-.5)pic[local bounding box =dt3]{dientro3}
		(0,2)pic[local bounding box =dt]{dientro}
		(0,1)pic[local bounding box =dt1]{dientro1}
	(0,0)pic[local bounding box =dt3]{dientro3}
		(1,.5)pic[local bounding box =nd1]{nguondien1}
(1,1.5)pic[local bounding box =nd2]{nguondien1}
		;
		%
	\draw[](dt3)node[above]{$4\Omega$}--(1,0)--(nd1)node[right]{ $5$ V}--(nd2)node[right]{ $4$ V}--(1,2)--(dt)node[above]{$16\Omega$}--(-1,2)--(-1,0)--(dt3) (-1,1)--(dt1)node[above]{$8\Omega$}--(1,1);
		\draw[-latex](-0.5,2)--(-0.8,2)node[above]{$I_1$};
		\draw[-latex](-.5,1)--(-0.8,1)node[above]{$I_2$};
		\draw[-latex](0.5,0)--(0.7,0)node[above]{$I_3$};
		%\draw[-latex](0.5,-.5)--(1,-.5)node[above]{$I_5$};
		\draw(0,0)node[below]{Hình 2};
	%	\draw(nd)--(dt1);
\end{tikzpicture}}
	\loigiai{
	Theo đề bài, ta có hệ phương trình: $\left\{\begin{array}{l}I_1+I_2=I_3 \\ 16 I_1-8 I_2=4 \\ 8 I_2+4 I_3=5.\end{array}\right.$\\
	Giải hệ phương trình, ta được: $I_1=\dfrac{11}{28} \mathrm{~A}$; $I_2=\dfrac{2}{7} \mathrm{~A}$; $I_3=\dfrac{19}{28} \mathrm{~A}$.}
	\end{bt}
\begin{bt}%[0D3B3-5]
	Để mở rộng sản suất, một công ty đã vay $800$ triệu đồng từ ba ngân hàng $A, B$ và $C$, với lãi suất cho vay theo năm lần lượt là $6 \%, 8 \%$ và $9 \%$. Biết rằng tổng số tiền lãi năm đầu tiên công ty phải trả cho ba ngân hàng là $60$ triệu đồng và số tiền lãi công ty trả cho hai ngân hàng $A$ và $C$ là bằng nhau. Tính số tiền công ty đã vay từ mỗi ngân hàng.
	\loigiai{
	Gọi $x$, $y$, $z$ lần lượt là số tiền công ty đã vay từ các ngân hàng $A$, $B$, $C$ (đơn vị: triệu đồng, $x \geq 0$, $y \geq 0$, $z \geq 0$).\\
	Theo đề bài, ta có hệ phương trình $\left\{\begin{array}{l}x+y+z=800 \\ 0{,}06 x+0{,}08 y+0{,}09 z=60 \\ 0{,}06 x=0{,}09 z.\end{array}\right.$\\
	Giải hệ phương trình, ta được $x=300$, $y=300$, $z=200$.\\
	Vậy số tiền công ty đã vay từ ba ngân hàng $A$, $B$, $C$ lần lượt là $300$ triệu đồng, $300$ triệu đồng, $200$ triệu đồng.
}
	\end{bt}
\begin{bt}%[0D3B3-5]
	Bác Nhân có $650$ triệu đồng dự định gửi tiết kiệm vào các ngân hàng $A$, $B$ và $C$. Biết các ngân hàng $A$, $B$, $C$ trả lãi suất lần lượt là $8 \% /$năm, $7{,}5 \% /$năm và $7 \% /$năm. Để phù hợp với nhu cầu, bác Nhân mong muốn sau một năm, tồng số tiền lãi bác nhận được là $50$ triệu đồng và số tiền bác gửi vào ngân hàng $B$ lớn hơn số tiền gửi vào ngân hàng $C$ là $100$ triệu đồng. Hãy tính giúp bác Nhân số tiền gửi vào mỗi ngân hàng sao cho đáp ứng được yêu cầu của bác.
	\loigiai{Gọi $x$, $y$, $z$ lần lượt là số tiền bác Nhân đã gửi vào các ngân hàng $A$, $B$, $C$ (đơn vị: triệu đồng, $x \geq 0, y \geq 0, z \geq 0$ ).\\
		Theo bài ra, ta có hệ phương trình $\heva{&x+y+z=650 \\ &0{,}08 x+0{,}075 y+0{,}07 z=50 \\ &y-z=100 .}$\\
		Giải hệ phương trình, ta được $x=350$, $y=200$, $z=100$.\\
		Vậy số tiền nên gưii vào các ngân hàng $A$, $B$, $C$ lần lượt là $350$ triệu đồng, $200$ triệu đồng, $100$ triệu đồng.
	}
		\end{bt}
\begin{bt}%[0D3B3-5]
	Một công ty sản xuất ba loại phân bón:\begin{itemize}
	\item Loại $\mathrm{A}$ có chứa $18 \%$ nitơ, $4 \%$ photphat và $5 \%$ kali;
	\item Loại $\mathrm{B}$ có chứa $20 \%$ nitơ, $4 \%$ photphat và $4 \%$ kali;
	\item Loại $\mathrm{C}$ có chứa $24 \%$ nitơ, $3 \%$ photphat và $6 \%$ kali.
	\end{itemize}
	Công ty sản xuất bao nhiêu kilôgam mỗi loại phân bón trên? Biết rằng công ty đã dùng hết $26400 \mathrm{~kg}$ nitơ, $4900 \mathrm{~kg}$ photphat, $6200 \mathrm{~kg}$ kali.
	\loigiai{
	Gọi $x, y, z$ lần lượt là số kilôgam mỗi loại phân bón $A, B, C$ mà công ty đã sản xuất $(x \geq 0$, $y \geq 0$, $z \geq 0)$.\\
	Theo đề bài, ta có hệ phương trình $\left\{\begin{array}{l}0{,}18 x+0{,}2 y+0{,}24 z=26400 \\ 0{,}04 x+0{,}04 y+0{,}03 z=4900 \\ 0{,}05 x+0{,}04 y+0{,}06 z=6200.\end{array}\right.$\\
	Giải hệ phương trinh, ta được $x=40000$, $y=60000$, $z=30000$.\\
	Vậy khối lượng mỗi loại phân bón $A$, $B$, $C$ mà công ty đã sản xuất lần lượt là $40000 \mathrm{~kg}$, $60000 \mathrm{~kg}$, $30000 \mathrm{~kg}$.
}
	\end{bt}
%KEtNoi
\begin{bt}%[0D3B3-5]
	Cân bằng phương trình phản ứng hoá học đốt cháy octane trong oxygen
	$$
	\mathrm{C}_8 \mathrm{H}_{18}+\mathrm{O}_2 \rightarrow \mathrm{CO}_2+\mathrm{H}_2 \mathrm{O}
	$$
	\loigiai{
	Giả sử $x, y, z, t$ là các số thoả mãn cân bằng
	$$
	x \mathrm{C}_8 \mathrm{H}_{18}+y \mathrm{O}_2 \rightarrow z \mathrm{CO}_2+t \mathrm{H}_2 \mathrm{O} \text {. }
	$$
	Ta có hệ phương trình $\left\{\begin{array}{l}8 x=z \\ 18 x=2 t \\ 2 y=2 z+t.\end{array}\right.$\\
	Giải hệ ta được $z=8 x, t=9 x$ và $y=\dfrac{25}{2} x$.\\
	Chọn $x=2$, được cân bằng
	$$
	2 \mathrm{C}_8 \mathrm{H}_{18}+25 \mathrm{O}_2 \rightarrow 16 \mathrm{CO}_2+18 \mathrm{H}_2 \mathrm{O}.
	$$
}
	\end{bt}
\begin{bt}%[0D3B3-5]
 Xét thị trường hải sản gồm ba mặt hàng là cua, tôm và cá. Kí hiệu $x$, $y$, $z$ lần lượt là giá $1 \mathrm{~kg}$ cua, $1 \mathrm{~kg}$ tôm và $1 \mathrm{~kg}$ cá (đơn vị nghìn đồng). Kí hiệu $Q_{S_1}, Q_{S_2}$ và $Q_{S_3}$ là lượng cua, tôm và cá mà người bán bằng lòng bán với giá $x, y$ và $z$. Kí hiệu $Q_{D_1}$, $Q_{D_2}$ và $Q_{D_3}$ tương ứng là lượng cua, tôm và cá mà người mua bằng lòng mua với giá $x$, $y$ và $z$. Cụ thể các hàm này được cho bởi
	$$
	\begin{aligned}
		&Q_{S_1}=-300+x ; Q_{D_1}=1300-3 x+4 y-z \\
		&Q_{S_2}=-450+3 y ; Q_{D_2}=1150+2 x-5 y-z \\
		&Q_{S_3}=-400+2 z ; Q_{D_3}=900-2 x-3 y+4 z.
	\end{aligned}
	$$
	Tìm mức giá cua, tôm và cá mà người bán và người mua cùng hài lòng.
	\loigiai{
	Hệ phương trình cân bằng cung - cầu $\left\{\begin{array}{l}-300+x=1300-3 x+4 y-z \\ -450+3 y=1150+2 x-5 y-z \\ -400+2 z=900-2 x-3 y+4z.\end{array}\right.$\\
	Giải hệ ta được $x=600, y=300$, $z=400$.\\
	Vậy giá của mỗi kg cua, tôm, cá lần lượt là $600$ nghìn đồng, $300$ nghìn đồng, $400$ nghìn đồng.	
}
	\end{bt}
% \subsection{Bài tập tự luận}
%CÁNH DIỀU
\begin{bt}
	\immini{Cho mạch điện như \emph{Hình 3}. Biết $U=20$ V, $r_1=1\ \Omega$, $r_2=0,5\ \Omega$, $R=2\ \Omega$. Tìm cường độ dòng điện $I_1,I_2,I$ trong mỗi nhánh.}
	{
\begin{circuitikz}[american voltages,c/.style={circle,fill,inner sep=1pt}]
	\draw (-5,0) to ++(2,0) [generic, l=$r_2$]to ++(1,0) [battery1, l=U]   to ++(2,0) --(0,0);
	\draw (-5,0)--(-5,-2) (0,0)--(0,-2);
	\draw (0,-1) -- (-2,-1) [generic] to ++(-1,0) -- (-5,-1);
	\draw (0,-2) -- (-2,-2) [generic, l=$R$] to ++(-1,0) -- (-5,-2);
	\draw[->] (-3,0)--(-4,0) node[above] {$I_2$} ;
	\draw[->] (-5,-1)--(-4,-1) node[above] {$I_1$};
	\draw[->]   (-5,-2)--(-4,-2)  node[above] {$I$};
	\draw (-2.5,-0.6) node {$r_1$};
	\draw (-0.7,0) node[above] {\scriptsize $-$} (-1.3,0) node[above] {\scriptsize $+$}; 
	\draw (-2.5, -3) node{\textit{Hình 3}.};
\end{circuitikz}	
}
	\loigiai{
		Cường độ dòng điện của đoạn mạch mắc song song là $I_1+I$. Ta có $I_2=I_1+I$ hay $I+I_1-I_2=0$.\\
		Hiệu điệu thế của đoạn mạch mắc song song là $U_1=I_1r_1=IR$ hay $I_1=2I$, do đó ta có $2I-I_1=0$.\\
		Hiệu điện thế toàn mạch là $U=U_2+U_1=I_2r_2+I_1r_1$ hay $20=0,5I_2+I_1$. Ta có hệ
		$$\heva{&I+I_1-I_2=0\\&2I-I_1=0\\&I_1+0,5I_2=20.}$$
		Giải hệ phương trình trên ta được $I=\dfrac{40}{7}$ (A), $I_1=\dfrac{80}{7}$ (A) và $I_2=\dfrac{120}{7}$ (A).
	}
\end{bt}
\begin{bt}
\immini{Cho mạch điện như \emph{Hình 4}. Biết $U=24$ V, $\text{Đ}_1:12\ \mathrm{V}-6\ \mathrm{W}$, $\text{Đ}_2: 12\ \mathrm{V}-12\ \mathrm{W}$, $R=3\ \Omega$.
	\begin{enumerate}[a)]
		\item Tính điện trở của mỗi bóng đèn.
		\item Tính cường độ dòng điện qua các bóng đèn và qua điện trở $R$.
	\end{enumerate}}
{
	\begin{circuitikz}[european,c/.style={circle,fill,inner sep=1pt}]
	\draw (-5,0)--(-2,0) [battery1, l=U]   to ++(1,0) --(0,0);
	\draw (-5,0)-- (-5,-1) [R=$R$] to ++(0,-1) -- (-5,-3);
	\draw (-5,-3)--(-4,-3) -- (-4,-2) [lamp, l=$\text{Đ}_1$] to ++(2,0) -- (-1,-2);
	\draw (-4,-3)--(-4,-4) [lamp,l=$\text{Đ}_2$] to++(3,0) -- (-1,-4)--(-1,-3);
	\draw (-1,-2) -- (-1,-3) -- (0,-3)--(0,0);
	\draw (-1.3,0) node[above] {\scriptsize $-$} (-1.7,0) node[above] {\scriptsize $+$}; 
	\draw (-2.5, -5) node{\textit{Hình 4}.};
	\draw[->] (-5,-3)--(-4.5,-3);
	\draw[->]  (-2.5,-2)--(-2,-2);
	\draw[->]  (-4,-4)--(-3.5,-4);
\end{circuitikz}
}
	\loigiai{
		\begin{enumerate}[a)]
			\item Điện trở của $\text{Đ}_1$ là $R_1=\dfrac{U_{1}^2}{P_{1}}=\dfrac{12^2}{6}=24$ (V).\\
			Điện trở của $\text{Đ}_2$ là $R_2=\dfrac{U_{2}^2}{P_{2}}=\dfrac{12^2}{12}=12$ (V).
			\item Cường độ dòng diện của đoạn mạch song song $I_1+I_2$. Ta có phương trình $I=I_1+I_2$ hay $I-I_1-I_2=0$.\\
			Hiệu điện thế đoạn mạch song song là $U_1=I_1R_1=I_2R_2$ hay $24I_1=12I_2\Leftrightarrow 2I_1-I_2=0$.\\
			Hiệu điện thế của đoạn mạch là $U=U_1+U_2=I_1R_1+I_2R_2$ hay $24=24I_1+12I_2\Leftrightarrow 2I_1+I_2=2$. Ta có hệ phương trình
			$$\heva{&I-I_1-I_2=0\\&2I_1-I_2=0\\&2I_1+I_2=2.}$$
			Giải hệ phương trình trên ta được $I=\dfrac{3}{2}$ (A), $I_1=\dfrac{1}{2}$ (A) và $I=1$ (A).
		\end{enumerate}
	}
\end{bt}
\begin{bt}
	Tìm các hệ số $x,y,z$ để cân bằng mỗi phương trình sau:
	\begin{enumerate}[a)]
		\item $x\mathrm{KClO}_3\xrightarrow{t^0} y\mathrm{KCl}+z\mathrm{O}_2$;
		\item $x\mathrm{FeCl}_2+y\mathrm{Cl}_2\xrightarrow{t^0} z\mathrm{FeCl}_3$;
		\item $x\mathrm{Fe}+y\mathrm{O}_2\xrightarrow{t^0} z\mathrm{Fe}_2\mathrm{O}_3$;
		\item $x\mathrm{Na}_2\mathrm{SO}_3+2\mathrm{KMnO}_4+y\mathrm{NaHSO}_4\xrightarrow{t^0} z\mathrm{Na}_2\mathrm{SO}_4+2\mathrm{MnSO}_4+\mathrm{K}_2\mathrm{SO}_4+3\mathrm{H}_2\mathrm{O}$.
	\end{enumerate}
	\loigiai{
		\begin{enumerate}[a)]
			\item Theo định luật bảo toàn nguyên tố đối với $\mathrm{K}$ và $O$ ta có $x=y$ hay $x-y=0$ và $3x=2z$ hay $3x-2z=0$. Ta có hệ phương trình sau
			$$\heva{&x-y=0\\&3x-2z=0.}$$
			Chọn $x=2$, từ hệ trên ta được $y=2$ và $z=3$. Vậy ta có phương trình sau cân bằng là $$2\mathrm{KClO}_3\xrightarrow{t^0} 2\mathrm{KCl}+3\mathrm{O}_2.$$
			\item Theo định luật bảo toaàn nguyên tố đối với $\mathrm{Fe}$ và $\mathrm{Cl}$ ta có $x=z$ hay $x-z=0$ và $2x+2y=3z$ hay $2x+2y-3z=0$. Ta có hệ phương trình sau
			$$\heva{&x-z=0\\&2x+2y-3z=0.}$$
			Chọn $z=2$, từ hệ trên ta có $x=2$ và $y=1$. Vậy ta có phương trình sau cân bằng là $$2\mathrm{FeCl}_2+\mathrm{Cl}_2\xrightarrow{t^0} 2\mathrm{FeCl}_3.$$
			\item Theo định luật bảo toàn nguyên tố đối với $\mathrm{Fe}$ và $\mathrm{O}$ ta có $x=2z$ hay $x-2z=0$ và $2y=3z$ hay $2y-3z=0$. Ta có hệ phương trình
			$$\heva{&x-2z=0\\&2y-3z=0.}$$
			Chọn $z=2$, từ hệ trên ta có $x=4$ và $y=3$. Vậy ta có phương trình sau cân bằng là $$4\mathrm{Fe}+3\mathrm{O}_2\xrightarrow{t^0} 2\mathrm{Fe}_2\mathrm{O}_3.$$
			\item Theo định luật bảo toàn nguyên tố đối với $\mathrm{Na}$, $\mathrm{H}$ và $\mathrm{O}$ ta có $2x+y=2z$ hay $2x+y-2z=0$ và $y=6$ và $3x+8+4y=4z+8+4+3$ hay $3x+4y-4z=7$. Ta có hệ phương trình sau
			$$\heva{&2x+y-2z=0\\&y=6\\&x+y-z=3}\Leftrightarrow \heva{&x-z=-3\\&3x-4z=-17\\&y=6}\Leftrightarrow \heva{&x=5\\&y=6\\&z=8.}$$
			Vậy ta có phương trình sau cân bằng là $$5\mathrm{Na}_2\mathrm{SO}_3+2\mathrm{KMnO}_4+6\mathrm{NaHSO}_4\xrightarrow{t^0} 8\mathrm{Na}_2\mathrm{SO}_4+2\mathrm{MnSO}_4+\mathrm{K}_2\mathrm{SO}_4+3\mathrm{H}_2\mathrm{O}.$$
		\end{enumerate}
	}
\end{bt}
\begin{bt}
	Một giáo viên dạy Hóa tạo $1000$ g dung dịch $\mathrm{HCl}$ $25\%$ từ ba loại dung dịch $\mathrm{HCl}$ có nồng độ lần lượt là $10\%$, $20\%$ và $30\%$. Tính khối lượng dung dịch mỗi loại. Biết rằng lượng $\mathrm{HCl}$ có trong dung dịch $10\%$ bằng $\dfrac{1}{4}$ lượng $\mathrm{HCl}$ có trong dung dịch $20\%$.
	\loigiai{
		Gọi khối lượng dung dịch $\mathrm{HCl}$ có nồng độ $10 \%, 20 \%$ và $30 \%$ lần lượt là $x$, $y$ và $z$. Theo đề bài ta có
		\begin{align}\label{41}
			x+y+z=1000.
		\end{align}
		Do dung dịch mới có nồng độ $25 \%$ nên ta có
		\begin{align}\label{42}
			\dfrac{10 \% x+20 \% y+30 \% z}{1000}=25 \% \Leftrightarrow 10 x+20 y+30 z=25000 \Leftrightarrow x+2 y+3 z=2500.
		\end{align}
		Lượng $\mathrm{HCl}$ có trong dung dịch $10 \%$ bằng $\dfrac{1}{4}$ lượng $\mathrm{HCl}$ có trong dung dịch $20 \%$ nên
		\begin{align}\label{43}
			10 \% x=\frac{1}{4} 20 \% y \Leftrightarrow 2 x-y=0.
		\end{align}
		Từ \eqref{41}, \eqref{42} và \eqref{43} ta có hệ phương trình
		$$\heva{&x+y+z=1000 \\
			&x+2 y+3 z=2500 \\
			&2 x-y=0.}
		$$
		Giải hệ này ta được $x=125, y=250, z=625$.
		Vậy khối lượng dung dịch $\mathrm{HCl}$ có nồng độ $10 \%, 20 \%$ và $30 \%$ lần lượt là $125 \mathrm{~g}, 250 \mathrm{~g}, 625 \mathrm{~g}$.
	}
\end{bt}
\begin{bt}
	Tổng số hạt $p,n,e$ trong hai nguyên tử kim loại $A$ và $B$ là $177$. Trong đó số hạt mang điện nhiều hơn số hạt không mang điện là $47$. Số hạt mang điện của nguyên tử $B$ nhiều hơn của nguyên tử $A$ là $8$. Xác định số hạt proton trong một nguyên tử $A$.
	\loigiai{
		Tổng số hạt $p,n,e$ trong hai nguyên tử $A$ và $B$ là $177$ nên ta có phương trình 
		\begin{align}\label{pt1}
			2Z_A+N_A+2Z_B+N_B=177.
		\end{align}
		Do số hạt mang điện nhiều hơn số hạt không mang điện là $47$ nên ta có phương trình
		\begin{align}\label{pt2}
			2Z_A+2Z_B-N_A-N_B=47.
		\end{align}
		Do số hạt mang điện của nguyên tử $B$ nhiều hơn của nguyên tử $A$ là $8$ nên ta có phương trình
		\begin{align}\label{pt3}
			2Z_B-2Z_A=8\Leftrightarrow Z_A-Z_B=-4.
		\end{align}
		Lấy phương trình \eqref{pt1} cộng với phương trình \eqref{pt2} vế theo vế ta được
		\begin{align}\label{pt4}
			4Z_A+4Z_B=224\Leftrightarrow Z_A+Z_B=56.
		\end{align}
		Từ \eqref{pt3} và \eqref{pt4} ta có hệ $$\heva{&Z_A-Z_B=4\\&Z_A+Z_B=56}\Leftrightarrow \heva{&Z_A=26\\&Z_B=30.}$$
		Vậy số hạt proton trong nguyên tử $A$ là $26$.
	}
\end{bt}

\begin{bt}
	Một phân tử DNA có khối lượng là $72\cdot 10^4$ đvC và có $2826$ liên kết hyđro. Mạch $2$ có số nu loại $A$ bằng $2$ lần số nu loại $T$ và bằng $3$ lần số nu loại $X$. Xác định số nucleotit mỗi loại trên từng mạch của phân tử DNA đó. Biết rằng một nu có khối lượng trung bình là $300$ đvC.
	\loigiai{
		Kí hiệu $A$, $G$, $T$, $X$ lần lượt là tổng số nu loại $A$, $G$, $T$, $X$ của phân tử DNA.\\
		$N$ là tổng số nu của phân tử DNA.\\
		$A_1$, $G_1$, $T_1$, $X_1$ lần lượt là tổng số nu loại $A$, $G$, $T$, $X$ của mạch 1.\\
		$A_2$, $G_2$, $T_2$, $X_2$ lần lượt là tổng số nu loại $A$, $G$, $T$, $X$ của mạch 2.\\
		Vì phân tử DNA có khối lượng là $72 \cdot 10^4$ đvC, mà một nu có khối lượng trung bình là $300$ đvC nên tổng số nu của phân tử DNA là $N=\dfrac{72 \cdot 10^4}{300} = 2400$.\\
		$\Rightarrow G+A = \dfrac{N}{2}=1200$.\\
		Phân tử DNA có $2826$ liên hết hyđro nên $2A + 3G = 2826$.\\
		Khi đó, ta có hệ phương trình 
		\begin{eqnarray*}
			\heva{&G+A=1200\\&2A + 3G = 2826} \Leftrightarrow \heva{&A=774\\&G=426} \Rightarrow \heva{&A=T=774\\&G=X=426.}
		\end{eqnarray*}
		Mạch 2 có số nu loại $A$ bằng $2$ lần số nu loại $T$ và bằng $3$ lần số nu loại $X$ nên\\ ta có $A_2 = 2T_2$, $A_2 = 3X_2$\\
		hay $A_2 - 2T_2 = 0$, $A_2 - 3X_2 = 0$.\\
		Mặt khác, vì $A_1 = T_2$ nên $A_2 + T_2 = A_2 + A_1 = A = 774$.\\
		Ta có hệ phương trình $\heva{&A_2 - 2T_2 = 0\\&A_2 - 3X_2 = 0\\&A_2 + T_2 =774} \Leftrightarrow \heva{&A_2=516\\&T_2=258\\&X_2=172.}$\\
		Suy ra số nu loại $G$ của mạch 2 là $G_2 = 1200 - (516 + 258 + 172) = 254$.\\
		Ở mạch 1, ta có $A_1 = T_2 = 258$, $T_1 = A_2 = 516$, $G_1 = X_2 = 172$, $X_1 = G_2 = 254$.
	}
\end{bt}

\begin{bt}
	Tìm đa thức bậc ba $f(x) = ax^3 + bx^2 + cx + 1$ (với $a \neq  0$) biết $f(-1) =  -2$, $f(1) = 2$, $f(2) = 7$.
	\loigiai{
		Ta có \\
		$f(-1) =  -2 \Leftrightarrow a\cdot (-1)^3 + b\cdot(-1)^2 + c\cdot(-1) + 1 = -2 \Leftrightarrow -a+b-c=-3$.\\
		$f(1) =  2 \Leftrightarrow a\cdot 1^3 + b\cdot 1^2 + c\cdot 1 + 1 = 2 \Leftrightarrow a+b+c=1$.\\
		$f(2) =  7 \Leftrightarrow a\cdot 2^3 + b\cdot 2^2 + c\cdot 2 + 1 = 7 \Leftrightarrow 8a+4b+2c=6$.\\
		Ta có hệ phương trình
		\begin{eqnarray*}
			\heva{&-a+b-c=-3\\&a+b+c=1\\&8a+4b+2c=6} \Leftrightarrow \heva{&a=1\\&b=-1\\&c=1.}
		\end{eqnarray*}
		Vậy $f(x)=x^3-x^2+x+1$.
	}
\end{bt}

\begin{bt}
	Ba lớp $10A$, $10B$, $10C$ trồng được $164$ cây bạch đàn và $316$ cây thông. Mỗi học sinh lớp $10A$ trồng được $3$ cây bạch đàn và $2$ cây thông; mỗi học sinh lớp $10B$ trồng được $2$ cây bạch đàn và $3$ cây thông; mỗi học sinh lớp $10C$ trồng được $5$ cây thông. Hỏi mỗi lớp có bao nhiêu học sinh? Biết số học sinh lớp $10A$ bằng trung bình cộng số học sinh lớp $10B$ và $10C$.
	\loigiai{
		Gọi số học sinh của ba lớp $10A$, $10B$, $10C$ lần lượt là $x$, $y$, $z$ (học sinh) với ($x, y, z \in \mathbb{N^*}$).\\
		Theo đề bài ta có hệ phương trình
		\begin{eqnarray*}
			\heva{&3x+2y+0z=164\\&2x+3y+5z=316\\&x=\dfrac{y+z}{2}} \Leftrightarrow \heva{&3x+2y=164\\&2x+3y+5z=316\\&2x-y-z=0} \Leftrightarrow \heva{&x=32\\&y=34\\&z=30.}
		\end{eqnarray*}	
		Vậy số học sinh của ba lớp $10A$, $10B$, $10C$ lần lượt là $32$, $34$, $30$ học sinh.
	}
\end{bt}

\begin{bt}
	Độ cao $h$ trong chuyển động của một vật được tính bởi công thức $h = \dfrac{1}{2}at^2 + v_0t + h_0$, với độ cao $h$ và độ cao ban đầu $h_0$ được tính bằng mét, $t$ là thời gian của chuyển động tính bằng giây, $a$ là gia tốc của chuyển động tính bằng m/s$^2$, $v_0$ là vận tốc ban đầu tính bằng m/s. Tìm $a$, $v_0$, $h_0$. Biết rằng sau $1$s và $3$s vật cùng đạt được độ cao $50{,}225$m; sau $2$s vật đạt độ cao $55{,}125$m.
	\loigiai{
		Theo đề bài ta có hệ phương trình
		\begin{eqnarray*}
			\heva{&\dfrac{1}{2}a\cdot 1^2 + v_0\cdot 1 + h_0=50{,}225\\&\dfrac{1}{2}a\cdot 3^2 + v_0\cdot 3 + h_0=50{,}225\\&\dfrac{1}{2}a\cdot 2^2 + v_0\cdot 2 + h_0=55{,}125} \Leftrightarrow \heva{&\dfrac{1}{2}a+v_0+h_0=50{,}225\\&\dfrac{9}{2}a+3v_0+h_0=50{,}225\\&2a+2v_0+h_0=55{,}125} \Leftrightarrow \heva{&a=-9{,}8\\&v_0=19{,}6\\&h_0=35{,}525.}
		\end{eqnarray*}		
		Vậy $a=-9{,}8$ m/s$^2$, $v_0=19{,}6$ m/s, $h_0=35{,}525$ m.
	}
\end{bt}

\begin{bt}
	Một ngân hàng muốn đầu tư số tiền tín dụng là $100$ tỉ đồng thu được vào ba nguồn: mua trái phiếu với mức sinh lời 8$\%$/năm, cho vay thu lãi suất 10$\%$/năm và đầu tư bất động sản với mức sinh lời 12$\%$/năm. Theo điều kiện của quỹ tín dụng đề ra là tổng số tiền đầu tư vào trái phiếu và cho vay phải gấp ba lần số tiền đầu tư vào bất động sản. Nếu ngân hàng muốn thu được mức thu nhập $9{,}6$ tỉ đồng hằng năm thì nên đầu tư như thế nào vào ba nguồn đó?
	\loigiai{
		Gọi số tiền đầu tư trái phiếu, cho vay, bất động sản lần lượt là $x$, $y$, $z$ (tỉ đồng).\\
		Theo đề bài ta có $x + y + z = 100$.\\
		Tổng số tiền đầu tư vào trái phiếu và cho vay gấp ba lần số tiền đầu tư vào bất động sản, do đó $x + y = 3z$ hay $x + y - 3z = 0$.\\
		Lãi suất cho ba khoản đầu tư lần lượt là 8$\%$, 10$\%$, 12$\%$ và tổng số tiền lãi thu được là $9{,}6$ tỉ đồng nên ta có $8\%x + 10\%y + 12\%z = 9{,}6$.\\
		Khi đó, ta có hệ phương trình 
		\begin{eqnarray*}
			\heva{&x + y + z = 100\\&x + y - 3z = 0\\&8\%x + 10\%y + 12\%z = 9{,}6} \Leftrightarrow \heva{&x=45\\&y=30\\&z=25.}
		\end{eqnarray*}
		Vậy số tiền đầu tư trái phiếu, cho vay, bất động sản lần lượt là $45$ tỉ đồng, $30$ tỉ đồng, $25$ tỉ đồng.
	}
\end{bt}


%KET NOI TRI THUC VA CUOC SONG

\begin{bt}%Bài 1.7%Nguyễn Nhật Lệ
	Cho hàm cung và hàm cầu của ba mặt hàng như sau
	\begin{center}
		$Q_{S_1}=-4+x;\ Q_{D_1}=70-x-2y-6z;$\\
		$Q_{S_2}=-3+y;\ Q_{D_2}=76-3x-y-4z;$\\
		$Q_{S_3}=-6+3z;\ Q_{D_3}=70-2x-3y-2z.$
	\end{center}
	Hãy xác định giá trị cân bằng cung - cầu của ba mặt hàng.
	\loigiai{
		Hệ phương trình cân bằng cung - cầu của ba mặt hàng là
		$$\heva{&Q_{S_1}=Q_{D_1}\\&Q_{S_2}=Q_{D_2}\\&Q_{S_3}=Q_{D_3}}\Leftrightarrow \heva{&-4+x=70-x-2y-6z\\&-3+y=76-3x-y-4z\\&-6+3z=70-2x-3y-2x}\Leftrightarrow \heva{&2x+2y+6z=74\\&3x+2y+4z=79\\&2x+3y+5z=76}\Leftrightarrow\heva{&x=15\\&y=7\\&z=5}.$$
		Vậy giá mặt hàng thứ nhất là $15$, mặt hàng thứ hai là $7$, mặt hàng thứ ba là $5$ hợp lí nhất.
	}
\end{bt}
\begin{bt}%Bài 1.8%Nguyễn Nhật Lệ
	Em Hà so sánh tuổi của mình với chị Mai và anh Nam. Tuổi của anh Nam gấp ba lần tuổi của em Hà. Cách đây bảy năm tuổi của chị Mai bằng nửa số tuổi của anh Nam. Ba năm nữa tuổi của anh Nam bằng tổng số tuổi của chị Mai và em Hà. Hỏi tuổi của mỗi người là bao nhiêu? 
	\loigiai{
		Gọi tuổi của anh Nam, chị Mai, em Hà lần lượt là $x,\ y\ z$ (tuổi) ($x,\ y,\ z>0$).\\
		Vì tuổi của anh Nam gấp ba lần tuổi của em Hà, ta có: $x=3z$.\\
		Cách đây $7$ năm, tuổi của chị Mai bằng nửa số tuổi anh Nam, ta có: $y-7=\dfrac{1}{2}(x-7)$.\\
		Ba năm nữa tuổi của anh Nam bằng tổng số tuổi của chị Mai và em Hà, nên ta có: $x+3=(y+3)+(z+3)$.\\
		Khi đó ta có hệ phương trình
		$$\heva{&x=3z\\&y-7=\dfrac{1}{2}(x-7)\\&x+3=(y+3)+(z+3)}\Leftrightarrow \heva{&x-3z=0\\&x-2y=-7\\&x-y-z=3}\Leftrightarrow\heva{&x=39\\&y=23\\&z=13}.$$
		Vậy anh Nam 39 tuổi, chị Mai 23 tuổi, em Hà 13 tuổi.
	}
\end{bt}
\begin{bt}%Bài 1.9%Nguyễn Nhật Lệ
	Bác Việt có 330 740 nghìn đồng, bác chia số tiền này thành ba phần và đem đầu tư vào ba hình thức: Phần thứ nhất bác đầu tư vào chứng khoán với lãi thu được $4\%$ một năm; phần thứ hai bác mua vàng thu lãi $5\%$ một năm và phần thứ ba bác gửi tiết kiện với lãi suất $6\%$ một năm. Sau một năm, kể cả gốc và lãi bác thu được ba món tiền bằng nhau? Hỏi tổng số tiền cả gốc và lãi bác thu được sau một năm là bao nhiêu?
	\loigiai{
		Gọi số tiền mà bác Việt đầu tư vào chứng khoán, vàng, gửi tiết kiệm lần lượt là $x,\ y,\ z$\ (nghìn đồng) ($x,\ y,\ z>0)$.\\
		Theo bài ra, tổng số tiền bác Việt có là $x+y+z=330740$.\\
		Sau một năm, cả gốc lẫn lãi thu được ba món tiền bằng nhau nên ta có
		$$x+4\%x=y+5\%y=z+6\%z\Leftrightarrow 1{,}04x=1
		{,}05y=1{,}06z$$
		Từ đó ta có hệ phương trình 
		$$\heva{&x+y+z=330740\\&1{,}04x-1{,}05y=0\\&1{,}05y-1{,}06z=0}\Leftrightarrow \heva{&x=111300\\&y=110240\\&z=109200}.$$
		Vậy bác Việt đầu tư 111 300 nghìn đồng vào chứng khoán, 110 240 nghìn đồng vào vàng và 109 200 nghìn đồng để gửi tiết kiệm.
	}
\end{bt}
\begin{bt}%Bài 1.10%Nguyễn Nhật Lệ
	Một tuyến cáp treo có ba loại vé sau đây: vé đi lên giá $250$ nghìn đồng; vé đi xuống giá $200$ nghìn đồng và vé hai chiều giá $400$ nghìn đồng. Một ngày nhà ga cáp treo thu được tổng số tiền là 251 triệu đồng. Tìm số vé bán ra mỗi loại, biết rằng nhân viên quản lí cáp treo đếm được 680 lượt người đi lên và 250 lượt người đi xuống.
	\loigiai{
		Gọi số vé đi lên, đi xuống, vé hai chiều bán ra lần lượt là $x,\ y,\ z\ (x,\ y,\ z>0)$.\\
		Theo bài ra, tổng số tiền là $251000000$ triệu đồng, khi đó ta có $250000x+200000y+400000z=251000000$ (1)\\
		Tổng số lượt người đi lên là $x+z=680$ (2)\\
		Tổng số lượt người đi xuống là $y+z=520$ (3)\\
		Từ (1),\ (2),\ (3) ta có hệ phương trình 
		$$\heva{&250000x+200000y+400000z=251000000\\&x+z=680\\&y+z=520}\Leftrightarrow\heva{&x=220\\&y=40\\&z=460}.$$
		Vậy số vé bán ra loại đi lên, đi xuống và hai chiều lần lượt là $220,\ 60,\ 460$.
	}
\end{bt}

\begin{bt}
	Ba lớp $10A$, $10B$, $10C$ của một trường trung học phổ thông gồm $128$ em cùng tham gia lao động trồng cây. Tính trung bình, mỗi em lớp $10A$ trồng được $3$ cây xoan và $4$ cây bạch đàn; mỗi em lớp $10B$ trồng được $2$ cây xoan và $5$ cây bạch đàn; mỗi em lớp $10C$ trồng được $6$ cây xoan. Cả ba lớp trồng được tổng cộng $476$ cây xoan và $375$ cây bạch đàn. Hỏi mỗi lớp có bao nhiêu em.
	\loigiai{ Gọi $x$, $y$, $z$ lần lượt là số học sinh lớp $10A$, $10B$, $10C$ ($x,\ y,\ z >0$).\\
		Vì tổng số học sinh ba lớp là $128$ nên $x+y+z=128$.\\
		Số cây xoan và bạch đàn lớp $10A$ trồng được lần lượt là $3x$, $4x$.\\
		Số cây xoan và bạch đàn lớp $10B$ trồng được lần lượt là $2y$, $5y$.\\
		Số cây xoan lớp $10C$ trồng được là $6z$.\\
		Theo đề ta có
		$$\heva{&x+y+z=128\\&3x+2y+6z=476\\&4x+5y=375} \Leftrightarrow \heva{&x=40\\&y=43\\&z=45.}$$
		Vậy số học sinh lớp $10A$, $10B$, $10C$ lần lượt là $40$, $43$, $45$ học sinh.
	}
\end{bt}
\begin{bt}
	Cân bằng phương trình phản ứng hóa học đốt cháy methane trong oxygen
	\begin{center}
		$\mathrm{CH}_4+\mathrm{O}_2 \rightarrow \mathrm{CO}_2+\mathrm{H}_2 \mathrm{O}$.
	\end{center}
	\loigiai{Gọi $x,\ y,\ z$ lần lượt là hệ số cân bằng của $CH_4$, $O_2$ và $H_2O$.\\
		Vì số nguyên tử $C$ ở 2 vế phương trình là như nhau nên ta có hệ số cân bằng của $CO_2$ bằng $x$.
		\begin{center}
			$x\mathrm{CH}_4+y\mathrm{O}_2 \rightarrow x\mathrm{CO}_2+z\mathrm{H}_2 \mathrm{O}$.
		\end{center}
		Số nguyên tử $\mathrm{O}$ ở hai vế bằng nhau nên $2y=2x+z$.\\
		Số nguyên tử $\mathrm{H}$ ở hai vế bằng nhau nên $4x=2z$.\\
		Ta có hệ phương trình
		$$\heva{&2y=2x+z &\quad (1)\\&4x=2z. & \quad (2)}$$
		Chọn $x=1$, từ (2) suy ra $z=2$, từ (1) suy ra $y=2$.\\
		Vậy phương trình cân bằng phản ứng hóa học là $$\mathrm{CH}_4+2\mathrm{O}_2 \rightarrow \mathrm{CO}_2+2\mathrm{H}_2 \mathrm{O}.$$
	}
\end{bt}
\begin{bt}
	Cho một đoạn mạch như \emph{Hình 1.2}. Gọi $I$ là cường độ dòng điện mạch chính, $I_1$, $I_2$, $I_3$ là cường độ dòng điện mạch rẽ. Cho biết $R_1 =6\ \Omega$, $R_2=8\ \Omega$, $I=3 \ A$ và $I_3=2 \ A$. Tính điện trở $R_3$ và hiệu điện thế $U$ giữa hai đầu đoạn mạch.
	\begin{center}
			\begin{circuitikz}[european,c/.style={circle,fill,inner sep=1pt}]
			\draw (0,0) coordinate (A) to(3,0) coordinate (B)  -- (3,1) to [R=$ R_1 $]  (5,1) to [R=$ R_2 $] (7,1)
			(B) -- (3,-1)to[R=$ R_3 $]  (7,-1)--(7,1)
			(A) -- (0,-2.5)  to[battery2,l_=U] (8,-2.5) ;
			\draw (7,0) coordinate (C) to (8,0)--(8,-2.5) ;
			\draw (1.5,0.2) node{$I$} (3.1,1.2) node {$I_1$} (5,1.2) node{$I_2$} (4,-0.7) node{$I_3$};
			\path foreach \p/\g in {}{(\p)node[c]{}+(\g:3.5mm) node{$\p$}};
			\draw (4,-4) node{\textit{Hình 1.2}};
		\end{circuitikz}
	\end{center}
	\loigiai{
		Vì $R_1$, $R_2$ mắc nối tiếp nên $I_1=I_2$.\\
		Từ sơ đồ mạch điện ta có hệ phương trình
		$$\heva{&I_1+I_3=I\\&R_1I_1+R_2I_2=U\\&R_3I_3=U} \Leftrightarrow \heva{&I_1+2=3\\&6I_1+8I_2=U\\&2R_3=U}\Leftrightarrow \heva{&I_1=1\\&U=14\\&R_3=7.}$$
		Vậy điện trở $R_3=7\ \Omega$, hiệu điện thế giữa hai đầu mạch $U=14 \ V$.
	}
\end{bt}
\begin{bt}
	Mỗi giai đoạn phát triển của thực vật cần phân bón với tỉ lệ $N$, $P$, $K$ nhất định. Bác An làm vườn muốn bón phân cho một cây cảnh với tỉ lệ $N:P:K$ cân bằng nhau. Bác An có ba bao phân bón:
	\begin{center}
		\begin{tabular}{ll}
			& Bao $1$ có tỉ lệ $N:P:K$ là $12:7:12$.\\
			& Bao $2$ có tỉ lệ $N:P:K$ là $6:30:25$.\\
			& Bao $3$ có tỉ lệ $N:P:K$ là $30:16:11$.
		\end{tabular}
	\end{center}
	Hỏi phải trộn ba loại phân bón trên với tỉ lệ bao nhiêu để có hỗn hợp phân bón tỉ lệ $N:P:K$ là $15:15:15$?
	Chú ý rằng trên mỗi bao phân người ta thường viết tỉ lệ $N:P:K$ nhất định. Chẳng hạn trên bao phân $1$ ghi tỉ lệ $N:P:K$ là $12:7:12$ nghĩa là hàm lượng đạm $N$ (nitơ) chiếm $12 \%$, lân $P$ (tức là $P_2O_5$) chiếm $7 \%$ và kali $K$ (tức là $K_2O$) chiếu $12\%$, còn các loại khác chiếm $100\%-(12\%+7\%+12\%)=69\%$.
	\loigiai{Giả sử bác An cần trộn $1$ kg phân bón với khối lượng ba loại phân bón này lần lượt là $x,\ y,\ z$ (kg).\\
		Khi đó, tổng khối lượng phân đạm $N$ trong $1$ kg này là $12\% x+6\%y+30\%z.$\\
		Tổng khối lượng phân lân $P$ trong $1$ kg này là $7\%x+30\%y+16\%z$.\\
		Tổng khối lượng phân kali $K$ trong $1$ kg này là $12\%x+25\%y+11\%z$.\\
		Vì hỗn hợp phân bón mới có tỉ lệ $N:P:K$ là $15:15:15$ nên ta có
		
		$$\heva{&12\%x+6\%y+30\%z=15\%\\&7\%x+30\%y+16\%z=15\%\\&12\%x+25\%y+11\%z=15\%} \Leftrightarrow \heva{&12x+6y+30z=15\\&7x+30y+16z=15\\&12x+25y+11z=15} \Leftrightarrow \heva{&x=0,5\\&y=0,25\\&z=0,25.}$$
		Vậy tỉ lệ bao 1: bao 2: bao 3 là $0,5:0,25:0,25$ hay $2:1:1$.
	}
\end{bt}

%CHÂN TRỜI SÁNG TẠO
\begin{bt}
	Một đại lí bán ba mẫu máy điều hòa $A$, $B$ và $C$, với giá bán mỗi chiếc theo từng mẫu lần lượt là $8$ triệu đồng, $10$ triệu đồng và $12$ triệu đồng. Tháng trước, đại lí bán được 100 chiếc gồm cả ba mẫu và thu được số tiền là $980$ triệu đồng. Tính số lượng máy điều hòa mỗi mẫu đại lí bán được trong tháng trước, biết rằng số tiền thu được từ bán máy điều hòa mẫu $A$ và mẫu $C$ là bằng nhau. 
	\loigiai{
		Gọi $x$, $y$, $z$ lần lượt là số máy điều hòa của các mẫu $A$, $B$, $C$ mà đại lí bán được trong tháng trước ($x,y,z\in \mathbb{N}$).\\
		Đại lí bán được 100 chiếc gồm cả ba mẫu, suy ra: $x+y+z=100$.\\
		Đại lí thu được số tiền là $980$ triệu đồng, suy ra: $8x+10y+12z=980$.\\
		Số tiền thu được từ bán máy điều hòa mẫu $A$ và mẫu $C$ là bằng nhau, suy ra: $8x=12z$.\\
		Như vậy, ta có hệ phương trình 
		$$\heva{&x+y+z=100 \\ &8x+10y+12z=980 \\ &8x-12z=0} \Leftrightarrow \heva{&x=30 \\ &y=50 \\ &z=20.}$$
		Vậy tháng trước đại lí bán được $30$ máy mẫu $A$, $50$ máy mẫu $B$ và $20$ máy mẫu $C$.
	}
\end{bt}


\begin{bt}
	Nhân dịp kỉ niệm ngày thành lập Đoàn Thanh niên Cộng sản Hồ Chí Minh, một trường Trung học phổ thông đã tổ chức cho học sinh tham gia các trò chơi. Ban tổ chức đã chọn $100$ bạn và chia thành ba nhóm $A$, $B$, $C$ để tham gia trò chơi thứ nhất. Sau khi trò chơi kết thúc, ban tổ chức chuyển $\dfrac{1}{3}$ số bạn ở nhóm $A$ sang nhóm $B$; $\dfrac{1}{2}$ số bạn ở nhóm $B$ sang nhóm $C$; số bạn chuyển từ nhóm $C$ sang nhóm $A$ và $B$ đều bằng $\dfrac{1}{3}$ số bạn ở nhóm $C$ ban đầu. Tuy nhiên, người ta nhận thấy số bạn ở mỗi nhóm là không đổi qua hai trò chơi. Ban tổ chức đã chia mỗi nhóm bao nhiêu bạn?
	\loigiai{
		Gọi $x$, $y$, $z$ lần lượt là số học sinh ở mỗi nhóm $A$, $B$, $C$ lúc ban đầu ($x,y,z\in\mathbb{N}$).\\
		Ban đầu có tổng cộng $100$ bạn, suy ra: $x+y+z=100$.\\
		Số bạn ở nhóm $A$ không đổi qua hai trò chơi, suy ra: $x-\dfrac{1}{3}x+\dfrac{1}{3}z=x$.\\
		Số bạn ở nhóm $B$ không đổi qua hai trò chơi, suy ra: $y+\dfrac{1}{3}x-\dfrac{1}{2}y+\dfrac{1}{3}z=y$.\\
		Số bạn ở nhóm $C$ không đổi qua hai trò chơi, suy ra: $z+\dfrac{1}{2}y-\dfrac{1}{3}z-\dfrac{1}{3}z=z$.\\
		Như vậy, ta có hệ phương trình
		$$ \heva{&x+y+z=10 \\ &x-\dfrac{1}{3}x+\dfrac{1}{3}z=x \\ &y+\dfrac{1}{3}x-\dfrac{1}{2}y+\dfrac{1}{3}z=y \\ &z+\dfrac{1}{2}y-\dfrac{1}{3}z-\dfrac{1}{3}z=z} 
		\Leftrightarrow \heva{&x+y+z=100 \\ &x-z=0 \\ &3y-4z=0} 
		\Leftrightarrow \heva{&x=30 \\ &y=40 \\ &z=30.}$$
		Vậy nhóm $A$ có $30$ bạn, nhóm $B$ có $40$ bạn, nhóm $C$ có $30$ bạn. 
	}
\end{bt}


\begin{bt}
	Một cửa hàng giải khát chỉ phục vụ ba loại sinh tố: xoài, bơ và mãng cầu. Để pha mỗi li (cốc) sinh tố này đều cần dùng đến sữa đặc, sữa tươi và sữa chua với công thức cho ở bảng sau.
	\begin{center}
		\begin{tabular}{|>{\centering\arraybackslash}m{3.5cm}|>{\centering\arraybackslash}m{3.5cm}|>{\centering\arraybackslash}m{3.5cm}|>{\centering\arraybackslash}m{3.5cm}|}
			\hline
			Sinh tố (li) 
			& Sữa đặc (m$l$) 
			& Sữa tươi (m$l$) 
			& Sữa chua (m$l$) \\
			\hline 
			Xoài 
			& $20$
			& $100$
			& $30$ \\
			\hline 
			Bơ
			& $10$
			& $120$
			&  $20$\\
			\hline 
			Mãng cầu 
			& $20$
			& $100$
			& $20$\\
			\hline
		\end{tabular}
	\end{center}
	Ngày hôm qua cửa hàng đã dùng hết $2l$ sữa đặc; $12,8l$ sữa tươi và $2,9l$ sữa chua. Cửa hàng đã bán được bao nhiêu li sinh tố mỗi loại trong ngày hôm qua?
	\loigiai{
		Gọi $x$, $y$, $z$ lần lượt là số li sinh tố xoài, bơ và dừa mà cửa hàng bán được ngày hôm qua ($x,y,z\in\mathbb{N}$).\\
		Cửa hàng đã dùng hết $2l$ sữa đặc, suy ra: $20x+10y+20z=2000$.\\
		Cửa hàng đã dùng hết $12,8l$ sữa tươi, suy ra: $100x+120y+100z=12800$.\\
		Cửa hàng đã dùng hết $2,9l$ sữa chua, suy ra: $30x+20y+20z=2900$.\\	
		Như vậy, ta có hệ phương trình 
		$$\heva{&20x+10y+20z=2000 \\ &100x+120y+100z=12800 \\ &30x+20y+20z=2900} \Leftrightarrow \heva{&x=50 \\ &y=40 \\ &z=30.}$$
		Vậy cửa hàng đã bán được $50$ li sinh tố xoài, $40$ li sinh tố bơ, $30$ li sinh tố mãng cầu.
	}
\end{bt}


\begin{bt}
	Ba tế bào $A$, $B$, $C$ sau một số lần nguyên phân tạo ra $168$ tế bào con. Biết số tế bào $A$ tạo ra gấp bốn lần số tế bào $B$ tạo ra và số lần nguyên phân của tế bào $C$ nhiều hơn số lần nguyên phân của tế bào $B$ là bốn lần. Tính số lần nguyên phân của mỗi tế bào.
	\loigiai{
		Gọi $x$, $y$, $z$ lần lượt là số lần nguyên phân của các tế nào $A$, $B$, $C$ ($x,y,z\in\mathbb{N}$).\\
		Ba tế bào $A$, $B$, $C$ sau một số lần nguyên phân tạo ra $168$ tế bào con, suy ra: $2^x + 2^y + 2^z = 168$. \\
		Số tế bào $A$ tạo ra gấp bốn lần số tế bào $B$ tạo ra, suy ra: $2^x = 4\cdot 2^y$.\\
		Số lần nguyên phân của tế bào $C$ nhiều hơn số lần nguyên phân của tế bào $B$ là bốn lần, suy ra: $z=y+4 \Leftrightarrow 16\cdot 2^y - 2^z = 0$.\\
		Như vậy, ta có hệ phương trình
		$$\heva{&2^x + 2^y + 2^z = 168 \\ &2^x - 4\cdot 2^y=0 \\ &16\cdot 2^y - 2^z = 0}
		\Leftrightarrow \heva{&2^x=32 \\ &2^y=8 \\ &2^z=128}
		\Leftrightarrow \heva{&x=5 \\ &y=3 \\ &z=7.}$$
		Vậy số lần nguyên phân của các tế nào $A$, $B$, $C$ lần lượt là $5$, $3$, $7$.
	}
\end{bt}


\begin{bt}
	Cho sơ đồ mạch điện như Hình 3. Biết $R_1 = 4\Omega$, $R_2 = 4\Omega$ và $R_3 = 8\Omega$. Tìm các cường độ dòng điện $I_1$, $I_2$ và $I_3$. 
	\begin{center}
	\begin{center}
		\begin{circuitikz}[european,c/.style={circle,fill,inner sep=1pt}]
			\draw (0,0) coordinate (A) to [R=$R_1$,i>_=$I_1$] (2,0)--(3,0)-- (3,1) to[R=$ R_2 $, i>_=$I_2$]  (7,1)
			(B) -- (3,-1)to[R=$ R_3 $, i>_=$I_3$]  (7,-1)--(7,1)
			(A) -- (0,-2.5)  to[battery2,l_=4 V] (8,-2.5) ;
			\draw (7,0) -- (8,0) -- (8,-2.5) ;
			\path foreach \p/\g in {}{(\p)node[c]{}+(\g:3.5mm) node{$\p$}};
			\draw (4,-4) node{Hình 3.};
		\end{circuitikz}
	\end{center}
	\end{center}
	\loigiai{
		Ta có hệ phương trình 
		$$\heva{&I_1 = I_2 + I_3 \\ &I_2 R_2 = I_3 R_3 \\ &I_1 R_1 + I_2 R_2 = 4} 
		\Leftarrow \heva{&I_1 - I_2 - I_3 = 0 \\ &4I_2 - 8I_3 = 0 \\ &4I_1 + 4I_2 = 4}
		\Leftrightarrow \heva{&I_1=\dfrac{3}{5} \\ &I_2=\dfrac{2}{5} \\ &I_3=\dfrac{1}{5}.}$$
		Vậy $I_1=\dfrac{3}{5}$A, $I_2=\dfrac{2}{5}$A, $I_3=\dfrac{1}{5}$A.
	}
\end{bt}


\begin{bt}
	Cân bằng phương trình phản ứng khi đốt cháy khi methane trong oxygen:
	$$\text{CH}_4+\text{O}_2 \xrightarrow{t^o}{  \text{CO}_2+\text{H}_2\text{O}}.$$
	\loigiai{Gọi $x,y,z$ là các hệ số của $\text{CH}_4$, $\text{O}_2$, $\text{H}_2\text{O}$ trên phương trình, khi đó theo định luật bảo toàn nguyên tố hệ số của $\text{CO}_2$ cũng là $x$.\\
		Ta lại  có: $4x=2z$ hay $4x-2z=0$ và $2y=2x+z$ hay $2x-2y+z=0$.\\
		Ta có hệ phương trình $$\heva{&4x-2z=0\\&2x-2y+z=0}$$
		Chọn $y=2$, khi đó hệ trở thành $\heva{&4x-2z=0\\&2x+z=4}\Leftrightarrow \heva{&x=1\\&y=2\\&z=2}$.\\
		Vậy ta có phương trình
		$$\text{CH}_4+2\text{O}_2 \xrightarrow{t^o}{  \text{CO}_2+2\text{H}_2\text{O}}.$$
	}
\end{bt}

\begin{bt}
	Một nhà máy có ba bộ phận cắt, may, đóng gói để sản xuất ba loại sản phẩm: áo thun, áo sơ mi, áo khoác. Thời gian (tính bằng phút) của mỗi bộ phận để sản xuất 10 cái áo mỗi loại được thể hiện trong bảng sau: 
	\begin{center}
		\begin{tabular}{|c|c|c|c|}
			\hline
			\multirow{2}{*}{Bộ phận}&\multicolumn{3}{c|}{Thời gian (tính bằng phút) để sản xuất 10 cái}\\ \cline{2-4}
			& Áo thun & Áo sơ mi & Áo khoác \\ \hline
			Cắt & 9 & 12 & 15 \\ \hline
			May & 22 & 24 & 28 \\ \hline
			Đóng Gói &6 &8 & 8 \\ \hline
		\end{tabular}
	\end{center}
	Các bộ phận cắt, may và đóng gói có tối đa $80$, $160$ và $48$ giờ lao động tương ứng mỗi ngày. Hãy lập kế hoạch sản xuất để nhà máy hoạt động hết công suất.
	\loigiai{Đổi: 80 giờ $=$ 4800 phút, 160 giờ $=$ 9600 phút, 48 giờ $=$ 2880 phút.\\
		Nhà máy hoạt động hết công suất nghĩa là sử dụng hết thời gian lao động tối đa.\\
		Gọi số lượng áo thun, áo sơ mi, áo khoác cần sản xuất để máy hoạt động hết công suất lần lượt là $x$, $y$, $z$ ($x$, $y$, $z$ nguyên dương).\\
		Dựa vào bảng trên ta có hệ phương trình: $\heva{&9x+12y+15z=4800\\&22x+24y+28z=9600\\&6x+8y+8z=2880}\Leftrightarrow \heva{&x=80\\&y=140\\&z=160}$.\\
		Vậy số lượng áo thun, áo sơ mi, áo khoác cần sản xuất để nhà máy hoạt động hết công suất lần lượt là 80, 140, 160.}
\end{bt}

\begin{bt}
	Bà Hà có 1 tỉ đồng để đàu tư vào cổ phiếu, trái phiếu và gửi tiết kiệm ngân hàng. Cổ phiếu sinh lợi nhuận $12\%$/năm, trong khi trái phiếu và gửi tiết kiệm ngân hàng cho lãi suất lần lượt là $8\%$/năm và $4\%$/năm. Bà Hà đã quy định rằng số tiền gửi tiết kiệm ngân hàng phải bằng tổng của $20\%$ số tiền đầu tư và cổ phiếu và $10\%$ số tiền đầu tư vào trái phiếu. Bà Hà nên phân bố vốn của mình như thế nào để nhận được 100 triệu đồng tiền lãi từ các khoản đầu tư đó trong năm đầu tiên?
	\loigiai{Gọi số tiền bà Hà nên đầu tư và cổ phiếu, trái phiếu và gửi tiết kiệm ngân hàng lần lượt là $x$, $y$, $z$ (triệu đồng).\\
		Theo đề bài ta có:
		\\
		Bà Hà có 1 tỉ đồng nên $x+y+z=1000$.\\
		Số tiền gửi tiết kiệm ngân hàng bằng tổng của $20\%$ số tiền đầu tư và cổ phiếu và $10\%$ số tiền đầu tư vào trái phiếu nên $z=20\% x+10\% y$ hay $2x+y-10z=0$.\\
		Số tiền lãi là 100 triệu đồng, suy ra $12\%x+5\%y+4\%z=100$ hay $3x+2y+z=2500$. Do đó ta có hệ phương trình:
		$$\heva{&x+y+z=1000\\&2x+y-10z=0\\&3x+2y+z=2500}\Leftrightarrow \heva{&x=650\\&y=200\\&z=150}.$$
		Vậy số tiền bà Hà nên đầu tư vào cổ phiếu, trái phiếu và gửi tiết kiệm ngân hàng lần lượt là 650 triệu đồng, 200 triệu đồng, 150 triệu đồng.
	}
\end{bt}

\begin{bt}
	Trên thị trường có ba loại sản phẩm A, B, C với giá mỗi tấn sản phẩm tương ứng là $x, y, z$ (đơn vị: triệu đồng, $x\ge 0, y\ge 0, z\ge 0$). Lượng cung và lượng cầu của mỗi sản phẩm được cho trong bảng dưới đây:
	\begin{center}
		\begin{tabular}{|c|c|c|}
			\hline
			Sản phẩm & Lượng cung & Lượng cầu\\ \hline
			A & $Q_{S_A}=4x-y-z-5$ & $Q_{D_A}=-2x+y+z+9$\\ \hline 
			B & $Q_{S_B}=-x+4y-z-5$ & $Q_{D_B}=x-2y+z+3$\\ \hline 
			C & $Q_{S_C}=-x-y+4z-1$ & $Q_{D_C}=x+y-2z-1$\\ \hline 	
		\end{tabular}
	\end{center}
	Tìm giá của mỗi sản phẩm để thị trường cân bằng.
	\loigiai{Thị trường cân bằng khi $\heva{&Q_{S_A}=Q_{D_A}\\&Q_{S_B}=Q_{D_B}\\&Q_{S_C}=Q_{D_C}}$
		$$\Leftrightarrow \heva{&4x-y-z-5= -2x+y+z+9 \\&-x+4y-z-5=x-2y+z+3   \\&-x-y+4z-1=x+y-2z-1}\Leftrightarrow \heva{&6x-2y-2z=14\\&2x-6y+2z=-8\\&2x+2y-6z=0}\Leftrightarrow \heva{&x=4,5\\&y=3,75\\&z=2,75}.$$
		Vậy giá mỗi sản phẩm A, B, C để thị trường cân bằng lần lượt là 4,5 triệu đồng; 3,75 triệu đồng; 2,75 triệu đồng.}
\end{bt}

\begin{bt}
	Vé vào xem một vở kịch có ba mức giá khác nhau thùy theo khu vực ngồi trong nhà hát. Số lượng vé bán ra và doanh thu của ba suất diễn được cho bởi bảng sau: 
	\begin{center}
		\begin{tabular}{|c|c|c|c|c|}
			\hline
			\multirow{2}{*}{Suất diễn }&\multicolumn{3}{c|}{Số vé bán được cái}&\multirow{2}{*}{Doanh thu (triệu đồng)}\\ \cline{2-4}
			& Khu vực 1 & Khu vực 2 & Khu vực 3&  \\ \hline
			10h00-12h00 & 210 & 152 & 125 &212,7 \\ \hline
			15h00-17h00 & 225 & 165 & 118 &224,4 \\ \hline
			20h00-22h00 &254 &186 & 130 &252,2 \\ \hline
		\end{tabular}
	\end{center}
	Tìm giá vé ứng với mỗi khu vực ngồi trong nhà hát.
	\loigiai{Gọi giá vé ứng với mỗi khu vực 1, khu vực 2, khu vực 3 lần lượt là $x$, $y$, $z$ (triệu đồng).\\
		Dựa vào bảng trên ta có hệ phương trình: $\heva{&210x+152y+125z=212,7\\&225x+165y+118z=22,4\\&254x+186y+130z=252,2}\Leftrightarrow \heva{&x=0,4\\&y=0,6\\&z=0,3}$.\\
		Vậy giá vé tương ứng với mỗi khu vực 1, khu vực 2, khu vực 3 lần lượt là 400 nghìn đồng, 600 nghìn đồng và 300 nghìn đồng.
	}
\end{bt}


% \section{BÀI TẬP HỆ PHƯƠNG TRÌNH BẬC NHẤT 3 ẨN}
% \subsection{Bài tập tự luận}
\begin{bt} %[0D3Y3-3]
Trong các hệ phương trình sau, hệ nào là hệ phương trình bậc nhất ba ẩn? Mỗi bộ ba số $(-1;0;1)$, $\left(\dfrac{1}{2} ;-\dfrac{1}{2} ;-1\right)$ có là nghiệm của các hệ phương trinh bậc nhất ba ẩn đó không?
\begin{listEX}[3]
\item $\heva{&2x-y+z=-1 \\& -x+2y=1 \\ &3y-2z=-2};$
\item $\heva{& 4x-2y+z=2 \\& 8x+3z=1  \\ & -6y+2z=1};$
\item $\heva{&3x-2y+zx=2 \\& xy-y+2z=1 \\ & x+2y-3yz=-2}.$
\end{listEX}
\loigiai{
a) và b) là các hệ phương trình bậc nhất ba ẩn vì các phương trình trong hệ phương trình a) và b) đều có dạng $ax+by+cz=d$ trong đó $a^2+b^2+c^2>0$.\\
c) không phải hê phương trình bậc nhất ba ẩn vì chứa ẩn $zx,xy,yz$.\\
$\circ$ Bộ ba số $(-1 ; 0 ; 1)$ là nghiệm của hệ a) vì khi thay bộ số này vào từng phương trình của a) thì chúng đều có nghiệm đúng. \\
Thật vậy, 
$2\cdot (-1)-0+1=-1$, $-(-1)+2\cdot0=1$, $3\cdot 0-2
\cdot 1=-2$.\\
$\circ$ Bộ ba số $\left(\dfrac{1}{2} ;-\dfrac{1}{2} ;-1\right)$ không là nghiệm của hệ a) vì khi thay bộ số này vào phương trình thứ nhất của hệ ta được $2 \cdot \dfrac{1}{2}-\left(-\dfrac{1}{2}\right)+(-1)=-1$, đây là đẳng thức sai.\\
$\circ$ Bộ ba số $(-1;0;1)$ không là nghiệm của hệ b) vì khi thay bộ số này vào phương trình thứ nhất của hệ ta được $4 \cdot (-1)-2\cdot 0+1=2$, đây là đẳng thức sai.\\
$\circ$ Bộ ba số $\left(\dfrac{1}{2} ;-\dfrac{1}{2} ;-1\right)$ là nghiệm của hệ b) vì khi thay bộ số này vào từng phương trình thì chúng đều có nghiệm đúng.\\
Thật vậy, 
$4 \cdot \dfrac{1}{2}-2\left(-\dfrac{1}{2}\right)+(-1)=2$, $8 \cdot \dfrac{1}{2}+3 \cdot(-1)=1$, $-6\left(-\dfrac{1}{2}\right)+2 \cdot(-1)=1$.}
\end{bt}
\begin{bt} %[0D3B3-3]
Giải các hệ phương trình sau bằng phương pháp Gauss:
\begin{listEX}[3]
\item $\heva{&x-2 y+z=3 \\& -y+z=2 \\ &y+2 z=1};$
\item $\heva{& 3x-2y-4z =3 \\& 4x+6y-z =17  \\ & x+2y=5};$
\item $\heva{&x+y+z=1 \\& 3 x-y-z=4 \\ & x+5 y+5 z=-1}.$
\end{listEX}
\loigiai{
a) $\heva{&x-2 y+z=3 \\& -y+z=2 \\ &y+2 z=1} \Leftrightarrow \heva{&x-2 y+z=3 \\& -y+z=2 \\ &3z=3} \Leftrightarrow \heva{&x-2 y+z=3 \\& -y+1=2 \\ &z=1}\Leftrightarrow \heva{&x-2\cdot (-1)+1=3 \\& y=-1 \\ &z=1}$\\
$\Leftrightarrow \heva{&x=0 \\& y=-1 \\ &z=1}$.\\
Vậy hệ phương trình đã cho có nghiệm duy nhất là $(0;-1;1)$.\\
b) $\heva{& 3x-2y-4z =3 \\& 4x+6y-z =17  \\ & x+2y=5} \Leftrightarrow \heva{& 3x-2y-4z =3 \\& -13x-26y=-65  \\ & x+2y=5}\Leftrightarrow \heva{& 3x-2y-4z =3 \\& x+2y=5  \\ & x+2y=5}\Leftrightarrow \heva{& 3x-2y-4z =3 \\& x+2y=5}.$\\
Từ phương trình thứ hai ta có $x=-2y+5$, thay vào phương trình thứ nhất ta được $z=-2y+3$.\\
Vậy hệ phương trình đã cho có vô số nghiệm dạng $(-2y+5;y;-2y+3)$.\\
c) $\heva{&x+y+z=1 \\& 3x-y-z=4 \\ & x+5y+5z=-1} \Leftrightarrow \heva{&x+y+z=1 \\& 4y+4z=-1\\ & x+5y+5z=-1} \Leftrightarrow \heva{&x+y+z=1 \\& 4y+4z=-1\\ & -4y-4z=2} \Leftrightarrow \heva{&x+y+z=1 \\& 4y+4z=-1\\ & 0y+0z=1}.$\\
Vì phương trình thứ ba của hệ vô nghiệm nên hệ đã cho vô nghiệm.}
\end{bt}
\begin{bt}%%[0D3B3-3]
	Giải các hệ phương trình sau:
		\begin{enumerate}
	\item $\heva{&x+y+z=6 \\&x+2y+3z=14 \\&3x-2y-z=-4}$;
	\item $\heva{&2x-2y+z=6 \\&3x+2y+5z=7 \\&7x+3y-6z=1}$;
	\item $\heva{&2x+y-6z=1 \\&3x+2y-5z=5 \\&7x+4y-17z=7}$;
	\item $\heva{&5x+2y-7z=6 \\&2x+3y+2z=7 \\&9x+8y-3z=1}$.
		\end{enumerate}
	\loigiai{
		\begin{enumerate}
			\item $\heva{&x+y+z=6 \\&x+2y+3z=14 \\&3x-2y-z=-4}\Leftrightarrow \heva{&x+y+z=6 \\&2x+y=4 \\&4x-y=2}\Leftrightarrow \heva{&x+y+z=6 \\&2x+y=4 \\&x=1}\Leftrightarrow \heva{&x=1 \\&y=2 \\&z=3}$.\\	
			Vậy hệ phương trình có nghiệm là $\left( x;y;z \right)=\left( 1;2;3 \right)$.
			\item $\heva{&2x-2y+z=6 \\&3x+2y+5z=7 \\&7x+3y-6z=1}\Leftrightarrow \heva{&2x-2y+z=6 \\&7x-12y=23 \\&19x-9y=37}\Leftrightarrow \heva{&2x-2y+z=6 \\&7x-12y=23 \\&-55x=-79}\Leftrightarrow \heva{&x=\dfrac{79}{55} \\&y=-\dfrac{178}{165} \\&z=\dfrac{32}{33}}$.
			Vậy hệ phương trình có nghiệm là $(x;y;z)=\left( \dfrac{79}{55};-\dfrac{178}{165};\dfrac{32}{33} \right)$.
			\item $\heva{&2x+y-6z=1 \\&3x+2y-5z=5 \\&7x+4y-17z=7}\Leftrightarrow \heva{&2x+y-6z=1 \\&-8x-7y=-25 \\&-8x-7y=-25}\Leftrightarrow \heva{&x=x_0 \\&y=\dfrac{25-8x_0}{7} \\&z=\dfrac{6x_0+18}{42}}\left( x_0\in \mathbb{R} \right)$.	\\
			Vậy hệ phương trình có vô số nghiệm dạng $\left( x;y;z \right)=\left( x_0;\dfrac{25-8x_0}{7};\dfrac{6x_0+18}{42} \right)\left( x_0\in \mathbb{R} \right)$.	
			\item $\heva{&5x+2y-7z=6 \\&2x+3y+2z=7 \\&9x+8y-3z=1}\Leftrightarrow \heva{&5x+2y-7z=6 \\&24x+25y=61 \\&-48x-50y=11}\Leftrightarrow \heva{&5x+2y-7z=6 \\&24x+25y=61 \\&0x+0y=133.}$.\\
			Vậy hệ phương trình đã cho vô nghiệm.
		\end{enumerate}
	}
\end{bt}

\begin{bt}%
Tìm các số thực $A$, $B$ và $C$ thỏa mãn $\dfrac{1}{x^3+1}=\dfrac{A}{x+1}+\dfrac{Bx+C}{x^2-x+1}$.
	\loigiai{
Ta có:
\begin{eqnarray*}
\dfrac{A}{x+1}+\dfrac{Bx+C}{x^2-x+1}&=&\dfrac{A.\left( x^2-x+1 \right)+\left( Bx+C \right)\left( x+1 \right)}{\left( x+1 \right)\left( x^2-x+1 \right)}\\
&=&\dfrac{\left( A+B \right)x^2+\left( -A+B+C \right)x+A+C}{x^3+1}.
\end{eqnarray*}
Vì $\dfrac{1}{x^3+1}=\dfrac{A}{x+1}+\dfrac{Bx+C}{x^2-x+1}$ nên ta suy ra\\
 \[\heva{&A+B=0 \\&-A+B+C=0 \\&A+C=1}\Leftrightarrow \heva{&A=\dfrac{1}{3} \\&B=-\dfrac{1}{3} \\&C=\dfrac{2}{3}.}\]
Vậy $A=\dfrac{1}{3},B=-\dfrac{1}{3}$ và $C=\dfrac{2}{3}$.
}
\end{bt}

\begin{bt}%%[0D2K3-2]
	Tìm parabol $y=ax^2+bx+c$ trong mỗi trường hợp sau:
	\begin{enumerate}
		\item Parabol đi qua ba điểm $A\left( 2;-1 \right),B\left( 4;3 \right)$ và $C\left( -1;8 \right)$.
		\item Parabol nhận đường thẳng $x=\dfrac{5}{2}$ làm trục đối xứng và đi qua hai điểm $M(1;0),N(5;-4)$.
	\end{enumerate}
	\loigiai{
	\begin{enumerate}
	\item Parabol đi qua ba điểm $A( 2;-1),B( 4;3 )$ và $C( -1;8 )$ nên ta có hệ: $\heva{&4a+2b+c=-1 \\&16a+4b+c=3 \\&a-b+c=8}$.\\
	Giải hệ trên ta được $a=1,b=-4,c=3$.\\
	Vậy parabol cần tìm là  $y=x^2-4x+3$.
	\item Parabol nhận đường thẳng $x=\dfrac{5}{2}$ làm trục đối xứng và đi qua hai điểm $M(1;0),\ N(5;-4)$ nên ta có hệ:
	$\heva{&-\dfrac{b}{2a}=\dfrac{5}{2} \\&a+b+c=0 \\&25a+5b+c=-4}\Leftrightarrow \heva{&5a+b=0 \\&a+b+c=0 \\&25a+5b+c=-4}$.\\
	Giải hệ trên ta được $a=-1,b=5$ và $c=-4$.\\
	Vậy parabol cần tìm là  $y=-x^2+5x-4$.
	\end{enumerate}
	}
\end{bt}
\begin{bt} %[0D3K3-3]
Tìm phương trình của parabol $(P) \colon y=ax^2+bx+c$ $(a \neq 0)$, biết:
\begin{enumerate}[a)]
\item Parabol $(P)$ cắt trục hoành tại hai điểm phân biệt có hoành độ lần lượt là $x=-2$; $x=1$ và đi qua điểm $M(-1 ; 3)$;
\item Parabol $(P)$ cắt trục tung tại điểm có tung độ $y=-2$ và hàm số đạt giá trị nhỏ nhất bằng $-4$ taii $x=2$.
\end{enumerate}
\loigiai{
a) $(P)$ cắt trục hoành tại hai điểm phân biệt có hoành độ lần lượt là $x=-2$; $x=1$ nên ta có hệ hai phương trình bậc nhất ba ẩn:
$$\heva{&a\cdot (-2)^2 + b\cdot (-2) + c=0\\&0 = a \cdot 1^2 + b\cdot 1 + c=0} \Leftrightarrow \heva{&4a-2b+c=0 \,\, (1)\\&a+b+c=0 \,\, (2)}$$
$(P)$ đi qua điểm $\mathrm{M}(-1; 3)$ nên $3=a\cdot (-1)^2+b\cdot (-1)+ c \Leftrightarrow a-b+c=3 \,\, (3)$.\\
Từ $(1), (2)$ và $(3)$ ta có hệ phương trình: $\heva{&4a-2b+c=0\\&a+b+c=0\\&a-b+c=3}$.\\
Giải hệ này ta được $a=-\dfrac{3}{2}$, $b=-\dfrac{3}{2}$, $c=3$.\\
Vậy phương trình của $(P)$ là $y=-\dfrac{3}{2} x^2-\dfrac{3}{2} x+3$.\\
b) $(P)$ cắt trục tung tại điểm có tung độ $=-2$ nên $a \cdot 0^2+b \cdot 0+c=-2$ hay $c=-2 \,\, (1)$.\\
Hàm số đạt giá trị nhỏ nhất bằng $-4$ tại $x=2$ nên 
$$\heva{&-\dfrac{b}{2a}=2\\&a\cdot 2^2+b\cdot 2+c=-4} \Leftrightarrow
\heva{&4a+b=0 \,\, (2) \\&4a+2b+c=-4 \,\, (3)}$$
Từ $(1), (2)$ và $(3)$ ta có hệ phương trình $\heva{&c=-2\\&4a+b=0\\&4a+2b+c=-4}$.\\
Giải hệ này ta được $a=\dfrac{1}{2}, b=-2, c=-2$.\\
Vậy phương trình của $(P)$ là $y=\dfrac{1}{2}x^2-2x-2$.}
\end{bt}
% \begin{bt}%%[0H3K2-2]
% 	Trong mặt phẳng tọa độ, viết phương trình đường tròn đi qua ba điểm $A( 0;1)$, $B(2;3 )$ và $C(4;1)$.
% 	\loigiai{
% 	Phương trình đường tròn có dạng: $x^2+y^2-2ax-2by+c=0$.	\\
% 	Đường tròn đi qua ba điểm $A( 0;1),\ B(2;3 )$ và $C(4;1)$ nên ta có hệ:\\
% 	$\heva{&0^2+1^2-2.0.a-2.1.b+c=0 \\&2^2+3^2-2.2.a-2.3.b+c=0 \\&4^2+1^2-2.4.a-2.1.b+c=0}\Leftrightarrow \heva{&-2b+c=-1 \\&-4a-6b+c=-13 \\&-8a-2b+c=-17}\Leftrightarrow \heva{&a=2 \\&b=1 \\&c=1}$.\\
% Vậy phương trình đường tròn cần tìm là $x^2+y^2-4x-2y+1=0$.}
% \end{bt}

\begin{bt} %[0D3T3-3]
Một viên lam ngọc và hai viên hoàng ngọc trị giá gấp $3$ lần một viên ngọc bích. Còn bảy viên lam ngọc và một viên hoàng ngọc trị giá gấp $8$ lần một viên ngọc bích. Biết giá tiền của bộ ba viên ngọc này là $270$ triệu đồng. Tính giá tiền mỗi viên ngọc.
\loigiai{
Gọi giá tiễn mỗi viên lam ngọc, hoàng ngọc, ngọc bích lần lượt là $x$, $y$, $z$ (triệu đồng).\\
Điều kiện: $0<x,y,z<270$.\\
Theo đề bài ta có:\\
$\circ$ Một viên lam ngọc và hai viên hoàng ngọc trị giá gấp $3$ lần một viên ngọc bích, suy ra $x+2y=3z$ hay $x+2y-3z=0 \,\, (1)$.\\
$\circ$ Bảy viên lam ngọc và một viên hoàng ngọc trị giá gấp $8$ lần một viên ngọc bích, suy ra $7x+y=8z$ hay $7x+y-8z=0 \,\, (2)$.\\
$\circ$ Giá tiền của bộ ba viên ngọc là $270$ triệu đồng , suy ra $x+y+z=270 \,\, (3)$.\\
Từ $(1), (2)$ và $(3)$ ta có hệ phương trình $\heva{&x+2y-3z=0\\&7x+y-8z=0\\&x+y+z=270}$.\\
Giải hệ này ta được $x=90, y=90, z=90$.\\
Vậy giá tiền mỗi viên ngọc đều là $90$ triệu đồng.}
\end{bt}
\begin{bt} %[0D3T3-3]
Bốn ngư dân góp vốn mua chung một chiếc thuyền. Số tiền người đầu tiên đóng góp bằng một nửa tổng số tiền của những người còn lại. Người thứ hai đóng góp bằng $\dfrac{1}{3}$ tổng số tiền của những người còn lại. Người thứ ba đóng góp bằng $\dfrac{1}{4}$ tổng số tiền của những người còn lại. Người thứ tư đóng góp $130$ triệu đồng. Chiếc thuyền này được mua giá bao nhiêu?
\loigiai{
Gọi số tiền người thứ nhất, người thứ hai, người thứ ba đóng góp lần lượt là $x$, $y$, $z$ (triệu đồng).\\
Điều kiện: $x,y,z >0$.\\
Theo đề bài ta có:\\
$\circ$ Số tiền người đầu tiên đóng góp bằng một nửa tổng số tiền của những người còn lại, suy ra $x=\dfrac{1}{2}(y+z+130)$ hay $2x-y-z=130 \,\, (1)$.\\
$\circ$ Người thứ hai đóng góp bằng $\dfrac{1}{3}$ tổng số tiền của những người còn lại, suy ra $y=\dfrac{1}{3}(x+z+130)$ hay $-x+3y-z=130 \,\, (2)$.\\
$\circ$ Người thứ ba đóng góp bằng $\dfrac{1}{4}$ tổng số tiền của những người còn lại, suy ra $z=\dfrac{1}{4}(x+y+130)$ hay $-x-y+4z=130 \,\, (3)$.\\
Từ $(1), (2)$ và $(3)$ ta có hệ phương trình $\heva{&2x-y-z=130\\&-x+3y-z=130\\&-x-y+4z=130}$.\\
Giải hệ này ta được $x=200$, $y=150$, $z=120$.\\
Suy ra tổng số tiền là $200+150+120+130=600$ (triệu đồng).\\
Vậy chiếc thuyền này được mua với giá $600$ triệu đồng.}
\end{bt}
\begin{bt} %[0D3T3-3]
Một quỹ đầu tư dự kiến dành khoản tiền $1,2$ tỉ đồng để đầu tư vào cồ phiếu. Để thấy được mức độ rủi ro, các cổ phiếu được phân thành ba loại: rủi ro cao, rủi ro trung bình và rủi ro thấp. Ban Giám đốc của quỹ ước tính các cổ phiếu rủi ro cao, rủi ro trung bình và rủi ro thấp sẽ có lợi nhuận hằng năm lần lượt là $15 \%$, $10 \%$ và $6 \%$. Nếu đặt ra mục tiêu đầu tư có lợi nhuận trung bình là $9 \%$ /năm trên tổng số vốn đầu tư, thì quỹ nên đầu tư bao nhiêu tiền vào mỗi loại cổ phiếu? Biết rằng, để an toàn, khoản đầu tư vào các cổ phiếu rủi ro thấp sẽ gấp đôi tổng các khoản đầu tư vào các cổ phiếu thuộc hai loại còn lại.
\loigiai{
Gọi số tiền nên đầu tư vào mỗi loại cổ phiếu rủi ro cao, rủi ro trung bình và rủi ro thấp lần lượt là $x$, $y$, $z$ (tỉ đồng).\\
Điều kiện: $0 \le x,y,z  \le 1,2$.\\
Theo đề bài ta có:\\
$\circ$ Tổng số tiền đầu tư là $1,2$ tỉ đồng, suy ra $x+y+z=1,2 \,\, (1)$.\\
$\circ$ Mục tiêu đầu tư có lợi nhuận trung bình là $9 \%$ /năm trên tổng số vốn đầu tư, suy ra $15 \% x+10 \% y+6 \% z=9 \% \cdot 1,2$ hay $15x+10y+6z=10,8 \,\, (2)$.\\
$\circ$ Khoản đầu tư vào các cổ phiếu rủi ro thấp sẽ gấp đôi tổng các khoản đầu tư vào các cổ phiếu thuộc hai loại còn lại, suy ra $z=2(x+y)$ hay $2x+2y-z=0 \,\, (3)$.\\
Từ $(1), (2)$ và $(3)$ ta có hệ phương trình $\heva{&x+y+z=1,2\\&15x+10y+6z=10,8\\&2x+2y-z=0}$.\\
Giải hệ này ta được $x=0,4$, $y=0$, $z=0,8$.\\
Vậy số tiền nên đầu tư vào mỗi loại cổ phiếu rủi ro cao, rủi ro trung bình và rủi ro thấp lần lượt là $0,4$ tỉ đồng, $0$ đồng, $0,8$ tỉ đồng.}
\end{bt}
\begin{bt} %[0D3T3-3]
Ba loại tế bào $A, B, C$ thực hiện số lần nguyên phân lần lượt là $3,4,5$ và tổng số tế bào con tạo ra là $216$. Biết rằng khi chưa thực hiện nguyên phân, số tế bào loại $C$ bằng trung bình cộng số tế bào loại $A$ và loại $B$. Sau khi thực hiện nguyên phân, tổng số tế bào con loại $A$ và loại $B$ được tạo ra ít hơn số tế bào con loại $C$ được tạo ra là $40$. Tính số tế bào con mỗi loại lúc ban đầu.
\loigiai{
Gọi số tế bào con ban đầu mỗi loại $A$, $B$, $C$ lần lượt là $x$, $y$, $z$.\\
Điều kiện: $x,y,z \in \mathbb{Z}^+$, $0<x,y,z <216$.\\
Theo đề bài ta có:\\
$\circ$ Ba loại tế bào $A$, $B$, $C$ thực hiện số lần nguyên phân lần lượt là $3,4,5$, suy ra số tế bào con mối loại $A$, $B$, $C$ lần lượt là $2^3 x, 2^4 y, 2^5 z$ hay $8x$, $16y$, $32z$.\\
$\circ$ Tổng số tế bào con tạo ra là $216$ , suy ra $8x+16y+32z=216$ hay $x+2y+4z=27 \,\ (1)$.\\
$\circ$ Khi chưa thực hiện nguyên phân, số tế bào loại $C$ bằng trung bình cộng số tế bào loại $A$ và loại $B$, suy ra $z=\dfrac{1}{2}(x+y)$ hay $x+y-2z=0 \,\, (2)$.\\
$\circ$ Sau khi thực hiện nguyên phân, tổng số tế bào con loại $A$ và loại $B$ được tạo ra ít hơn số tế bào con loại $C$ được tạo ra là $40$, suy ra $8x+16y=32z-40$ hay $x+2y-4z=-5 \,\, (3)$.\\
Từ $(1), (2)$ và $(3)$ ta có hệ phương trình $\heva{&x+2y+4z=27\\&x+y-2z=0\\&x+2y-4z=-5}$.\\
Giải hệ này ta được $x=5$, $y=3$, $z=4$.\\
Vậy số tế bào con ban đầu mỗi loại $A$, $B$, $C$ lần lượt là $5$, $3$ và $4$.}
\end{bt}
\begin{bt}%%[0D3K3-5]
	Một đoàn xe chở $225$ tấn gạo tiếp tế cho đồng bào vùng bị lũ lụt. Đoàn xe có $36$ chiếc gồm $3$ loại: xe chở $5$ tấn, xe chở $7$ tấn và xe chở $10$ tấn. Biết rằng tổng số hai loại xe chở $5$ tấn và $7$ tấn nhiều gấp ba lần số xe chở $10$ tấn. Hỏi mỗi loại xe có bao nhiêu chiếc?
	\loigiai{
		Gọi $x,y,z$ lần lượt là số xe chở $5$ tấn, xe chở $7$ tấn và xe chở $10$ tấn ($x,y,z\in \mathbb{N}; 0<x,y,z<36$).\\
		Theo đề ra ta có hệ phương trình: $\heva{&x+y+z=36 \\&x+y=3z \\&5x+7y+10z=255}$.\\
	Giải hệ trên ta được: $x=12,y=15,z=9$.\\
	Vậy đoàn xe có $12$ xe loại $5$ tấn, $15$ xe loại $7$ tấn và $9$ xe loại $10$ tấn.
	}
\end{bt}
\begin{bt}%%%[0D3K3-5]
	Bác An là chủ cửa hàng kinh doanh cà phê cho những người sành cà phê. Bác có ba loại cà phê nổi tiếng của Việt Nam: Arabica, Robusta và Moka với giá bán lần lượt là $302$ nghìn đồng/kg, $280$ nghìn đồng/ kg và $260$ nghìn đồng/ kg. Bác muốn trộn ba loại cà phê này để được một hỗn hợp cà phê, sau đó đóng thành các gói $1$kg, bán với giá $300$ nghìn đồng/ kg và lượng cà phê Moka gấp đôi lượng cà phê Robusta trong mỗi gói. Hỏi bác cần trộn ba loài cà phê theo tỉ lệ nào?
	\loigiai{
		Gọi $x,y,z$ lần lượt là tỉ lệ pha trộn cà phê Arabica, Robusta và Moka ($0\le x,y,z\le 1$).\\
		Theo đề ra ta có hệ phương trình: $\heva{&x+y+z=1 \\&z=2y \\&320x+280y+260z=300}$.\\
	Giải hệ trên ta được: $x=\dfrac{5}{8},y=\dfrac{1}{8},z=\dfrac{2}{8} \cdot$ \\
	Vậy tỉ lệ pha trộn cà phê Arabica, Robusta và Moka lần lượt là $\dfrac{5}{8},\ \dfrac{1}{8}$ và $\dfrac{2}{8}\cdot$ 
	}
\end{bt}
\begin{bt}%%%[0D3K3-5]
%	\immini{
		Bác Việt có $12$ ha đất canh tác để trồng ba loại cây: ngô, khoai tây và đậu tương. Chi phí trồng $1$ ha ngô là $4$ triệu đồng, $1$ ha khoai tây là $3$ triệu đồng và $1$ ha đậu tương là $4,5$ triệu đồng. Do nhu cầu thị trường, bác đã trồng khoai tây trên phần diện tích gấp đôi diện tích trồng ngô. Tổng chi phí trồng $3$ loại cây trên là $45,25$ triệu đồng. Hỏi diện tích trồng mỗi loại cây là bao nhiêu?
%	}
%	{\includegraphics{images/Picture1}
%	}
	\loigiai{
		Gọi diện tích trồng ngô, khoai tây, đậu tương lần lượt là $x,y,z$(ha).\\
		Điều kiện $0<x<12,\ 0<y<12,\ 0<z<12$.\\
		Từ dữ kiện bài toán ta lập được hệ phương trình: $\heva{&x+y+z=12 \\&y=2x \\&4x+3y+4,5z=45,25}$\\
	Giải hệ trên ta có $\heva{&x=2,5 \\&y=5 \\&z=4,5}$.\\
	Vậy diện tích trồng ngô, khoai tây, đậu tương của bác Việt lần lượt là: $2,5$(ha), $5$(ha), $4,5$(ha).
	}
\end{bt}

\begin{bt} %[0D3T3-3]
\immini{Cho sơ đồ mạch điện như Hình 1. Biết rằng $R = R_1 = R_2 = 5 \, \Omega$. Hãy tính các cường độ dòng điện $I$, $I_1$ và $I_2$.}{\begin{tikzpicture}
 \draw 
 (0,0.4)--(0,-0.4)
 (0.2,0.2)--(0.2,-0.2)
 (0.2,0)--(2,0)
 (2,-0.2) rectangle (3,0.2)
 (3,0)--(4,0)--(4,3)--(3.5,3)
 (3.5,4)--(3.5,2)--(1,2)
 (1,2.2) rectangle (0,1.8)
 (0,2)--(-1,2) 
 (3.5,4)--(1,4)
 (1,4.2) rectangle (0,3.8)
 (0,4)--(-1,4)--(-1,2)
 (-1,3)--(-2,3)--(-2,0)--(0,0)
 ;
 \draw
 (0.1,-0.7) node {$4 \,V$}
 (2.5,-0.5) node {$R$}
 (0.5,1.5) node {$R_2$}
 (0.5,4.2) node[above] {$R_1$}
 (-1,4) node[above] {$I_1$}
  (-1,1.5) node {$I_2$}
  (-1.5,3) node[above] {$I$}
 ; 
 \draw[->]
 (-1,4)--(-0.5,4)
 (-1,2)--(-0.5,2)
 (-2,3)--(-1.5,3)
 ;
 \end{tikzpicture}}
\loigiai{
Điều kiện: $I,I_1,I_2 >0$.\\
Tổng cường độ dòng điện ra vào vào tại điểm $B$ bằng nhau nên ta có $I = I_1 + I_2 \,\, (1)$.\\
Hiệu điện thế giữa hai điểm $A$ và $C$ được tính bởi:
$U_{AC} = IR + I_1R_1 = 5I + 5I_1$, suy ra $5I + 5I_1 = 4 \,\, (2)$.\\
Hiệu điện thế giữa hai điểm $B$ và $C$ được tính bởi:
$U_{BC} = I_1R_1 = I_2R_2$, suy ra $5I_1 = 5I_2$ hay $I_1 = I_2 \,\, (3)$.\\
Từ $(1), (2)$ và $(3)$ ta có hệ phương trình $\heva{&I - I_1 - I_2 =0\\&5I + 5I_1 = 4\\&I_1 - I_2=0}$.\\
Giải hệ này ta được $I=\dfrac{8}{15}$, $I_1=\dfrac{4}{15}$, $I_2=\dfrac{4}{15}$.}
\end{bt}
\begin{bt} %[0D3T3-3]
Cho $A$, $B$ và $C$ là ba dung dịch cùng loại acid có nồng độ khác nhau. Biết rằng nếu trộn ba dung dịch mỗi loại $100$ ml thì được dung dịch nồng độ $0,4$ M (mol/lít); nếu trộn $100$ ml dung dịch $A$ với $200$ ml dung dịch $B$ thì được dung dịch nồng độ $0,6$ M; nếu trộn $100$ ml dung dịch $B$ với $200$ ml dung dịch $C$ thì được dung dịch nồng độ $0,3$ M. Mỗi dung dịch $A$, $B$ và $C$ có nồng độ bao nhiêu?
\loigiai{Gọi nồng độ của mỗi dung dịch $A$, $B$, $C$ lần lượt là $x$, $y$, $z$ (M).\\
Điều kiện: $x,y,z >0$.\\
Theo đề bài ta có:\\
$\circ$ Nếu trộn ba dung dịch mỗi loại $100$ ml thì được dung dịch nồng độ $0,4$ M, suy ra $\dfrac{0,1x+0,1y+0,1z}{0,1+0,1+0,1}=0,4$ hay $x+y+z=1,2 \,\, (1)$.\\
$\circ$ Nếu trộn $100$ ml dung dịch $A$ với $200$ ml dung dịch $B$ thì được dung dịch nồng độ $0,6$ M, suy ra $\dfrac{0,1x+0,2y}{0,1+0,2}=0,6$ hay $x+2y=1,8 \,\, (2)$.\\
$\circ$ Nếu trộn $100$ ml dung dịch $B$ với $200$ ml dung dịch $C$ thì được dung dịch nồng độ $0,3$ M, suy ra $\dfrac{0,1y+0,2z}{0,1+0,2}=0,3$ hay $y+2z=0,9 \,\ (3)$.\\
Từ $(1), (2)$ và $(3)$ ta có hệ phương trình $\heva{&x+y+z=1,2\\&x+2y=1,8\\&y+2z=0,9}$.\\
Giải hệ này ta được $x=0,4$, $y=0,7$, $z=0,1$.\\
Vậy nồng độ của mỗi dung dịch $A$, $B$, $C$ lần lượt là $0,4$ M; $0,7$ M; $0,1$ M.}
\end{bt}
\begin{bt} %[0D3T3-3]
Xăng sinh học E5 là hỗn hợp xăng không chì truyền thống và cồn sinh học (bio – ethanol). Trong loại xăng này chứa $5 \%$ cồn sinh học. Khi động cơ đốt cháy lượng cồn trên thì xảy ra phản ứng hoá học
$$C_2H_6O + O_2 \buildrel {{t^\circ}} \over \longrightarrow  CO_2 + H_2O.$$
Cân bằng phương trình hoá học trên.
\loigiai{
Gọi $x$, $y$, $z$, $t$ lần lượt là bốn hệ số nguyên dương thoả mãn cân bằng phương trình phản ứng hoá học:
$$xC_2H_6O + yO_2 \buildrel {{t^\circ}} \over \longrightarrow  zCO_2 + tH_2O.$$
Điều kiện: $x,y,z \in \mathbb{Z}^+$.\\
Số nguyên tử $C$ ở hai vế bằng nhau, ta có $2x = z \,\,(1)$.\\
Số nguyên tử $H$ ở hai vế bằng nhau, ta có $6x = 2t$ hay $3x = t \,\,(2)$.\\
Số nguyên tử $O$ ở hai vế bằng nhau, ta có $x + 2y = 2z + t \,\,(3)$.\\
Thay $(1)$ và $(2)$ vào $(3)$ ta được $x+2y=2\cdot 2x+3x$ hay $y=3x$.\\
Vậy $y=3x$, $z=2x$, $t=3x$.\\
Để phương trình có hệ số đơn giản, ta chọn $x=1$, khi đó $y=3$, $z=2$, $t=3$.\\
Vậy phương trình cân bằng phản ứng hóa học là $C_2H_6O + 3O_2 \buildrel {{t^\circ}} \over \longrightarrow  2CO_2 + 3H_2O.$}
\end{bt}
\begin{bt}%%%[0D3K3-5]
	Cân bằng phương trình phản ứng hóa học sau $FeS_2+O_2\to Fe_2O_3+SO_2$
	\loigiai{
		Gọi $x,y,z,t$ là hệ số cân bằng lần lượt đứng trước $FeS_2,\ O_2,\ Fe_2O_3,\ SO_2$.\\
		Khi đó phương trình phản ứng có dạng $xFeS_2+yO_2\to zFe_2O_3+tSO_2$.\\
		Vì số nguyên tử của $Fe,S,O$ trước và sau phản ứng bằng nhau nên ta có hệ phương trình
	\[\heva{&x=2z \\&2x=t \\&2y=3z+2t}\Leftrightarrow \heva{&z=\dfrac{1}{2}x \\&t=2x \\&y=\dfrac{11}{4}x}\]
	 Chọn $x=4$ ta có $y=11,\ z=2,\ t=8$.\\
		Suy ra ta cân bằng phương trình hóa học như sau $4FeS_2+11O_2\to 2Fe_2O_3+8SO_2$.
	}
\end{bt}
\begin{bt}%%%[0D3K3-5]
	Bạn Mai có ba lọ dung dịch chứa một loại acid. Dung dịch $A$ chứa $10\%,$ dung dịch $B$ chứa $30\%$ và dung dịch $B$ chứa $50\%$ Bạn Mai lấy từ mỗi lọ dung dịch và hòa với nhau để có $50g$ hỗn hợp chứa $32\%$ acid này và lượng dung dịch loại $C$ lấy nhiều gấp đôi dung dịch loại $A$. Tính lượng dung dịch mỗi loại bạn Mai đã lấy.
	\loigiai{
		Gọi lượng dung dịch loại $A,\ B,\ C$ mà Mai đã lần lượt lấy ra là $x,\ y,\ z\ (0<x,y,z<50)$.
		Theo bài ra ta có hệ phương trình: $\heva{&x+y+z=50 \\&z=2x \\&\dfrac{1}{10}x+\dfrac{3}{10}y+\dfrac{5}{10}z=\dfrac{32}{100}.50}\Leftrightarrow \heva{&x+y+z=50 \\&z=2x \\&\dfrac{1}{10}x+\dfrac{3}{10}y+\dfrac{5}{10}z=16}$\\
	Giải hệ trên ta có $\heva{&x=5 \\&y=35 \\&z=10}$.\\
	Vậy dung dịch loại $A,\ B,\ C$ mà Mai đã lần lượt lấy ra là: $5$(g), $35$(g), $10$(g).
	}
\end{bt}
\begin{bt}%%[0D3K3-3]
	Cho đoạn mạch như hình vẽ
	\begin{center}
		\begin{circuitikz}[european,c/.style={circle,fill,inner sep=1pt}]
		\draw (0,0) coordinate (A) to(3,0)coordinate (B)  -- (3,1) to[R=$ R_1 $, i>_=$I_1$]  (7,1)
		(B) -- (3,-1)to[R=$ R_2 $, i>_=$I_2$]  (7,-1)--(7,1)
		(A) -- (0,-2.5)  to[battery2,l_=U] (8,-2.5) ;
		\draw (7,0) coordinate (C) to (8,0) to[R=$ R_3 $, i>_=$I_3$] (8,-2.5) ;
		%to[R=$ R_3$ ,i_=$I_1$] (8,0) -- (7,0) coordinate (C);
		\path foreach \p/\g in {}{(\p)node[c]{}+(\g:3.5mm) node{$\p$}};
	\end{circuitikz}
	\end{center}
	Biết $R_1=36\Omega,R_2=45\Omega,I_3=1,5\mathrm{A}$ là cường độ dòng điện trong mạch chính và hiệu điện thế giữa hai hai đầu đoạn mạch $U=60\mathrm{V} $ Gọi $I_1,I_2$ là cường độ dòng điện mạch rẽ. Tính $I_1,I_2$ và $R_3$.
	\loigiai{
		Gọi $U_1,U_2,U_3,U_{12}$ lần lượt là hiệu điện thế giữa hai đầu $R_1,R_2,R_3$ và đoạn mạch mắc song song.\\
		Khi đó từ sơ đồ mạch điện ta có: $\heva{&U_1=U_2=U_{12} \\&U_{12}+U_3=60}(*)$.\\
		Vì $R_1,R_2$ mắc song song nên $R_{12}=\dfrac{R_1.R_2}{R_1+R}=\dfrac{36.45}{36+45}=20$.\\
		Mặt khác $I_{12}=I_3=1,5$( mắc nối tiếp), suy ra $ {U_{12}}=I_{12}.R_{12}=1,5.20=30$.\\
		Theo $\left( * \right)$ ta suy ra $\heva{&U_1=U_2=U_{12}=30 \\&U_3=60-U_{12}=30}\Rightarrow \heva{&I_1=\dfrac{U_1}{R_1}=\dfrac{30}{36}=\dfrac{5}{6} \\&I_1=\dfrac{U_2}{R_2}=\dfrac{30}{45}=\dfrac{2}{3} \\&R_3=\dfrac{U_3}{I_3}=\dfrac{30}{1,5}=20}$ .\\
		Vậy $\heva{&I_1=\dfrac{5}{6}\left( A \right) \\&I_1=\dfrac{2}{3}\left( A \right) \\&R_3=20\left( \Omega \right)}$.
	}
\end{bt}
\begin{bt}%%%[0D3K3-5]
Giải bài toán dân gian sau
	\begin{align*}
	& \text{Em đi chợ phiên}\\
	& \text{Anh gửi một tiền}\\
	& \text{Cam, thanh yên, quýt}\\
	& \text{Không nhiều thì ít}\\
	& \text{Mua đủ một trăm}\\
	& \text{Cam ba đồng một}\\
	& \text{Quýt một đồng năm}\\
	& \text{Thanh yên tươi tốt}\\
	& \text{Năm đồng một trái}
	\end{align*}
	Hỏi mỗi thứ mua bao nhiêu trái, biết một tiền bằng $60$ đồng?
\loigiai{
	Gọi số cam, quýt, thanh yên lần lượt là: $x,\ y,\ z$ (quả), $( x,y,z\in{\mathbb{N}^{*}},x,y,z<100 )$.\\
	Theo đề bài ta lập được hệ phương trình: $\heva{&x+y+z=100\left( 1 \right) \\&3x+\dfrac{1}{5}y+5z=60\left( 2 \right)}$\\
	Từ $\left( 1 \right),\left( 2 \right)$ suy ra: $7x+12z=100\Leftrightarrow 7\left( x-16 \right)=-12\left( z+1 \right)$.\\
	Vì vậy $\heva{&x-16=-12k \\&z+1=7k}\left( k\in \mathbb{Z} \right)\Leftrightarrow \heva{&x=-12k+16 \\&z=7k-1}$.\\
	Để $x,z$ nguyên dương thì $k=1$ từ đó tìm được $x=4,y=90,z=6$.\\
	Vậy có $4$ quả cam, $90$ quả quýt và $6$ quả thanh yên.
		}
\end{bt}
\Closesolutionfile{ans}
% 
% \subsection{BÀI TẬP TRONG SÁCH GIÁO KHOA}
\begin{bt} %[0D3T3-3]
Trên thị trường hàng hoá có ba loại sản phẩm $A$, $B$, $C$ với giá mỗi tấn tương ứng là $x$, $y$, $z$ (đơn vị: triệu đồng, $x \ge 0$, $y \ge 0$, $z \ge 0$). Lượng cung và lượng cầu của mỗi sản phẩm được cho trong bảng dưới đây:
\begin{center}
\begin{tabular}{|p{2cm}|p{5cm}|p{5cm}|}
\hline
Sản phẩm & Lượng cung & Lượng cầu \\
\hline
$A$ & $Q_{S_A}=-60+4x-2z$ & $Q_{D_A}=137-3x+y$ \\
\hline
$B$ & $Q_{S_B}=-30-x+5y-z$ & $Q_{D_B}=131+x-4y+z$ \\
\hline
$C$ & $Q_{S_C}=-30-2x+3z$ & $Q_{D_C}=157+y-2z$ \\
\hline
\end{tabular}
\end{center}
Tìm giá của mỗi sản phẩm để thị trường cân bằng.
\loigiai{
Thị trường cân bằng khi$\heva{&Q_{S_A}=Q_{D_A}\\&Q_{S_B}=Q_{D_B}\\&Q_{S_C}=Q_{D_C}}$\\
$\Leftrightarrow \heva{&-60+4x-2z=137-3x+y\\&-30-x+5y-z=131+x-4y+z\\&-30-2x+3z=157+y-2z}\Leftrightarrow \heva{&7x-y-2z=197\\&2x-9y+2z=-161\\&2x+y-5z=-187}\Leftrightarrow \heva{&x=54\\&y=45\\&z=68}$\\
Vậy giá mỗi sản phẩm $A$, $B$, $C$ lần lượt là $54$, $45$ và $68$ triệu đồng.}
\end{bt}
\begin{bt} %[0D3T3-3]
Giải bài toán cổ sau:
\begin{center}
\textit{Trăm trâu, trăm cỏ\\
Trâu đứng ăn năm\\
Trâu nằm ăn ba\\
Lụ khụ trâu già\\
Ba con một bó}
\end{center}
Hỏi có bao nhiêu con trâu đứng, trâu nằm, trâu già?
\loigiai{
Gọi số trâu đứng, trâu nằm, trâu già lần lượt là $x$, $y$, $z$ ($x$, $y$, $z \in \mathbb{Z}^+$).\\
Theo đề bài ta có hệ phương trình: $\heva{&x+y+z=100\\&5x+3y+\dfrac{1}{3}z=100}\,\, (*)$.\\
$(*) \Leftrightarrow \heva{&x+y=100-z\\&15x+9y=300-z} \Leftrightarrow \heva{&x=\dfrac{-300+4z}{3}\\&y=\dfrac{600-7z}{3}} \Leftrightarrow \heva{&x=\dfrac{4z}{3}-100\\&y=200-\dfrac{-7z}{3}}$.\\
Vì $x>0$ nên $\dfrac{4z}{3}-100>0 \Rightarrow z >75.$\\
Vì $y>0$ nên $200-\dfrac{-7z}{3}>0 \Rightarrow z<85$.\\
Mà $z \in \mathbb{Z}^+$ nên $z \in \left\lbrace 76;77;...;84\right\rbrace $.\\
Lại có $x \in \mathbb{Z}^+$ nên $\dfrac{4z}{3}-100 \in \mathbb{Z}^+$, suy ra $z \vdots 3 \Rightarrow z \in \left\lbrace 78;81;84 \right\rbrace .$\\
$\circ$ Với $z=78$ thì $x=4$, $y=18$.\\
$\circ$ Với $z=81$ thì $x=8$, $y=11$.\\
$\circ$ Với $z=84$ thì $x=12$, $y=4$.\\
Vậy số trâu đứng, trâu nằm, trâu già theo thứ tự có thể là một trong ba bộ số $(4; 18; 78)$, $(8; 11; 81)$, $(12; 4; 84)$.}
\end{bt}
% \subsection{BÀI TẬP NÂNG CAO}
\begin{bt}%[0D3T3-3]
Trong phân tử $M_2X$ có tồng số hạt $p$, $n$, $e$ là $140$, trong đó số hạt mang điện nhiều hơn số hạt không mang điện là $44$ hạt. Số khối của $M$ lớn hơn số khối của $X$ là $23$. Tổng số hạt $p$, $n$, $e$ trong nguyên tử $M$ nhiều hơn trong nguyên tử $X$ là $34$ hạt. Công thức phân từ của $M_2X$ là
\loigiai{
Gọi số hạt $p$, $n$, $e$ của nguyên tử $M$ và $X$ lần lượt là $p_M$, $n_M$, $e_M$; $p_X$, $n_X$, $e_X$.\\
Trong phân tử $M_2X$ có tồng số hạt $p$, $n$, $e$ là $140$ nên ta có phương trình
$$2(p_M+n_M+e_M)+(p_X+n_X+e_X)=140$$
$$\Leftrightarrow 2(2p_M+n_M)+(2p_X+n_X)=140 \Leftrightarrow 4p_M+2n_M+2p_X+n_X=140 \,\, (1)$$
Số hạt mang điện nhiều hơn số hạt không mang điện là $44$ hạt nên ta có phương trình 
$$4p_M-2n_M+2p_X-n_X=44 \,\, (2)$$
Số khối của $M$ lớn hơn số khối của $X$ là $23$ nên ta có phương trình
$$(p_M+n_M)-(p_X+n_X)=23 \Leftrightarrow p_M+n_M-p_X-n_X=23 \,\, (3)$$
Tổng số hạt $p$, $n$, $e$ trong nguyên tử $M$ nhiều hơn trong nguyên tử $X$ là $34$ hạt nên ta có phương trình 
$$(2p_M+n_M)-(2p_X+n_X)=34 \Leftrightarrow 2p_M+n_M-2p_X-n_X=34 \,\, (4)$$
Từ $(1)$, $(2)$, $(3)$ và $(4)$ ta có hệ phương trình 
$$\heva{&4p_M+2n_M+2p_X+n_X=140\\&4p_M-2n_M+2p_X-n_X=44\\&p_M+n_M-p_X-n_X=23\\&2p_M+n_M-2p_X-n_X=34} \Leftrightarrow \heva{&4p_M+2n_M+2p_X+n_X=140\\&8p_M+4p_X=184\\&5p_M+3n_M+p_X=163\\&6p_M+3n_M=174}$$
$$\Leftrightarrow \heva{&4p_M+2n_M+2p_X+n_X=140\\&p_M=13\\&n_M=20\\&p_X=8} \Leftrightarrow \heva{&p_M=13\\&n_M=20\\&p_X=8\\&n_X=8}$$
Khi đó $M$ là $K$, $X$ là $O$.\\
Vậy công thức phân tử của $M_2X$ là $K_2O$.}
\end{bt}
\begin{bt} %[0D3T3-3]
Tìm các hệ số $x$, $y$, $z$ để cân bằng phương trình sau:
$$xNa_2SO_3 +2KMnO_4 +yNaHSO_4 \longrightarrow  zNa_2SO_4 +2MnSO_4 +K_2SO_4 + 3H_2O.$$
\loigiai{
Số nguyên tử $Na$ ở hai vế bằng nhau, ta có $2x +y= 2z$ hay $2x+y-2z=0 \,\,(1)$.\\
Số nguyên tử $S$ ở hai vế bằng nhau, ta có $x+y= z+3$ hay $x+y-z= 3 \,\,(2)$.\\
Số nguyên tử $O$ ở hai vế bằng nhau, ta có $3x + 4y +8= 4z + 15$ hay $3x+4y-4z=7 \,\,(3)$.\\
Từ $(1)$, $(2)$ và $(3)$ ta có hệ phương trình $$\heva{&2x+y-2z=0\\&x+y-z= 3\\&3x+4y-4z=7}$$
Giải hệ phương trình ta được $x=5$, $y=6$, $z=8$.\\
Vậy phương trình cân bằng phản ứng hóa học là $5Na_2SO_3 +2KMnO_4 +6NaHSO_4 \longrightarrow  8Na_2SO_4 +2MnSO_4 +K_2SO_4 + 3H_2O.$}
\end{bt}
\Closesolutionfile{ans}

%%Chương 4
\def\tenchude{GTLG CỦA MỘT GÓC TỪ $\mathbf{0^\circ}$ ĐẾN $\mathbf{180^\circ}$}
\setcounter{section}{0}
\section{GTLG CỦA MỘT GÓC TỪ $\mathbf{0^\circ}$ ĐẾN $\mathbf{180^\circ}$}
\subsection{Tóm tắt lí thuyết}
\subsubsection{Khái niệm}
	\immini
	{
Điểm $M(x_0;y_0)$ nằm trên nửa đường tròn đơn vị sao cho $\widehat{xOM}=\alpha$. Khi đó
\begin{itemize}
	\item $\sin\alpha=y_0$;
	\item $\cos\alpha=x_0$;
	\item $\tan\alpha=\dfrac{\sin\alpha}{\cos\alpha}$ với $(\alpha\ne 90^\circ)$;
	\item $\cot\alpha=\dfrac{\cos\alpha}{\sin\alpha}$ với ($\alpha\ne 0^\circ,180^\circ$).
\end{itemize}
}
{
\begin{tikzpicture}[scale=1.5, font=\footnotesize,line join=round, line cap=round, >=stealth, x=2cm,y=2cm]
	\def\x{0.5}
	\def\a{60}
	\pgfmathsetmacro{\y}{\x*tan(\a)}
	\draw[->] (-1.2,0)--(1.2,0) node [below]{$x$};
	\draw[->] (0,-0.2)--(0,1.2) node [left]{$y$};
	\node at (0,0) [below left]{$O$};
	\draw (1,0) arc (0:180:2cm);
	\fill (1,0) node[below]{$1$} circle(1pt);
	\fill (-1,0) node[below]{$-1$} circle(1pt);
	\fill (0,1) node[above left]{$1$} circle(1pt);
	%	\fill (\x,0) node[below]{$x_0$} circle(1pt);
	\fill (\x,0) node[below]{$x_0$} circle(1pt);
	%	\fill (\a:1) node[above right]{$M$} circle(1pt);
	\fill (\a:1) node[above]{$M$} circle(1pt);
	\fill (0,\y) node[left]{$y_0$} circle(1pt);
	\draw[dashed] (\x,0)|-(0,\y);
	\draw (0,0)--(\a:1);
	\draw (0.3,0) arc (0:\a:0.6cm);
	\node at (0,0)[shift={(25:0.2)}]{$\alpha$};
\end{tikzpicture}
}
\subsubsection{Dấu của giá trị lượng giác.}
\begin{center}
	\renewcommand\arraystretch{1.2} %tăng độ rộng
	% \renewcommand{\tabcolsep}{6mm} %tăng chiều dài
	\begin{tabular}{|c|l l r|l r c|}
		\hline
		Góc $\alpha$ & $0^\circ$ && \multicolumn{2}{c}{$90^\circ$} && $180^\circ$\\
		\hline
		$\sin\alpha$ && $+$ &&& $+$ &\\
		\hline
		$\cos\alpha$ && $+$ &&& $-$ &\\
		\hline
		$\tan\alpha$ && $+$ &&& $-$ &\\
		\hline
		$\cot\alpha$ && $+$ &&& $-$ &\\
		\hline
	\end{tabular}
\end{center}
\subsubsection{Bảng giá trị lượng giác của một số góc đặc biệt cần nhớ}
\begin{center}
	\renewcommand{\arraystretch}{2}%
	\begin{tabular}{|c|c|c|c|c|c|c|c|c|c|}
		\hline
		$\alpha$ & $0^\circ$ & $30^\circ$ & $45^\circ$ & $60^\circ$ & $90^\circ$ & $120^\circ$ & $135^\circ$ & $150^\circ$& $180^\circ$ \\
		\hline
		$\sin\alpha$& $0$ & $\dfrac{1}{2}$ & $\dfrac{\sqrt{2}}{2}$  &  $\dfrac{\sqrt{3}}{2}$ & $1$ & $\dfrac{\sqrt{3}}{2}$ & $\dfrac{\sqrt{2}}{2}$ & $\dfrac{1}{2}$ & $0$ \\
		\hline
		$\cos\alpha$& $1$ & $\dfrac{\sqrt{3}}{2}$ & $\dfrac{\sqrt{2}}{2}$  &  $\dfrac{1}{2}$& $0$& $\dfrac{-1}{2}$&$\dfrac{-\sqrt{2}}{2}$&$\dfrac{-\sqrt{3}}{2}$&$-1$ \\
		\hline
		$\tan\alpha$& $0$ & $\dfrac{\sqrt{3}}{3}$ & $1$  &  $\sqrt{3}$& $\parallel$& $-\sqrt{3}$&$-1$&$\dfrac{-\sqrt{3}}{3}$& $0$\\
		\hline
		$\cot\alpha$& $\parallel$ & $\sqrt{3}$ & $1$  &  $\dfrac{\sqrt{3}}{3}$& $0$& $\dfrac{-\sqrt{3}}{3}$&$-1$&$-\sqrt{3}$& $\parallel$\\
		\hline
	\end{tabular}
\end{center}

\subsubsection{Tính chất}
\begin{multicols}{2}
\begin{enumerate}
	\item Giá trị lượng giá của hai góc phụ nhau
	\begin{itemize}
		\item $\sin(90^\circ-\alpha)=\cos\alpha$.
		\item $\cos(90^\circ-\alpha)=\sin\alpha$.
		\item $\tan(90^\circ-\alpha)=\cot\alpha$.
		\item $\cot(90^\circ-\alpha)=\tan\alpha$.
	\end{itemize}
	\item Giá trị lượng giác của hai góc bù nhau
	\begin{itemize}
		\item $\sin(180^\circ-\alpha)=\sin\alpha$.
		\item $\cos(180^\circ-\alpha)=-\cos\alpha$.
		\item $\tan(180^\circ-\alpha)=-\tan\alpha$.
		\item $\cot(180^\circ-\alpha)=-\cot\alpha$.
	\end{itemize}
\end{enumerate}
\end{multicols}
\begin{enumerate}
\item[c)] Hệ thức cơ bản
	\begin{itemize}
		\item $\sin^2\alpha+\cos^2\alpha=1$.
		\item $1+\tan^2\alpha=\dfrac{1}{\cos^2\alpha}$ với $(\alpha\ne 90^\circ)$.
		\item $1+\cot^2\alpha=\dfrac{1}{\sin^2\alpha}$ với $(0^\circ<\alpha< 180^\circ)$.
		\item $\tan\alpha\cdot\cot\alpha=1$ với $(0^\circ<\alpha< 180^\circ, \alpha\ne 90^\circ)$.
	\end{itemize}
\end{enumerate}
\subsection{Các dạng toán}
%\setcounter{subsection}{1}% Reset lại số đếm subsection
\begin{dang}{Tính giá trị biểu thức lượng giác. Chứng minh đẳng thức lượng giác}
	Áp dụng các công thức lượng giác
\end{dang}
\subsubsection{Ví dụ minh hoạ}
\begin{vd}%[0H2Y1]
	Tính giá trị biểu thức sau
	\begin{tasks}(2)
		\task $A= 2\cos 30^\circ+3\sin 120^\circ$.
		\task $B=a\cos60^{\circ}+2a\tan45^{\circ}-3a\sin30^{\circ}$.
	\end{tasks}
	\loigiai{
		\begin{enumerate}[a)]
			\item
			\item Ta có $B=\dfrac{1}{2}a+2a-\dfrac{1}{2}.3a=a$.
		\end{enumerate}
		
	}
\end{vd}
\begin{vd}%[0H2Y1]
	Cho $x=30^{\circ}$. Tính $A=\sin (2x)-3\cos x$.
	\loigiai{
		$A=\sin 2.(30^{\circ})-3\cos30^{\circ}=\sin60^{\circ}-3\cos30^{\circ}=\dfrac{\sqrt{3}}{2}-3\dfrac{\sqrt{3}}{2}=-\sqrt{3}$.
	}
\end{vd}

\begin{vd}
	Biết $\sin15^\circ=\dfrac{\sqrt{3}-1}{2\sqrt{2}}$. Tính giá trị biểu thức $P=\sin165^\circ+\cos75^\circ$.
\end{vd}

\begin{vd}
Không dùng máy tính, tính giá trị của các biểu thức sau
\begin{enumerate}
	\item $A=\sin 45^\circ\cot 135^\circ+\cos 60^\circ\cdot\sin 150^\circ-\cos 30^\circ\cdot\sin 120^\circ$.
	\item $B=\tan 135^\circ+\cot 60^\circ\cot 30^\circ-\tan 60^\circ\tan 150^\circ$.
	\item $C=2\sin 60^\circ\tan 150^\circ-\cos 180^\circ\cdot \cot 45^\circ$.
\end{enumerate}	
\loigiai{
\begin{enumerate}
	\item Ta có $\sin 45^\circ=-\cos 135^\circ=\dfrac{\sqrt{2}}{2}$, $\cos 60^\circ=\sin 150^\circ=\dfrac{1}{2}$ và $\cos 30^\circ=\sin 120^\circ=\dfrac{\sqrt{3}}{2}$.\\
	Từ đó suy ra
	$A=\dfrac{\sqrt{2}}{2}\cdot\left(\dfrac{-\sqrt{2}}{2}\right)+\dfrac{1}{2}\cdot\dfrac{1}{2}-\dfrac{\sqrt{3}}{2}\cdot\dfrac{\sqrt{3}}{2}=\dfrac{-1}{2}+\dfrac{1}{4}-\dfrac{3}{4}=-1$.
	\item Do $\tan 135^\circ=-1$, $\cot 60^\circ=\dfrac{\sqrt{3}}{3}$, $\cot 30^\circ=\tan 60^\circ=\sqrt{3}$ và $\tan 150^\circ=\dfrac{-\sqrt{3}}{3}$ nên
	$$B=-1+\dfrac{\sqrt{3}}{3}\cdot \sqrt{3}-\sqrt{3}\cdot\left(\dfrac{-\sqrt{3}}{3}\right)=1.$$
	\item Ta có $\sin 60^\circ=\dfrac{\sqrt{3}}{2}$, $\tan 150^\circ=\dfrac{-\sqrt{3}}{3}$, $\cos 180^\circ=-1$ và $\cot 45^\circ=1$.\\
	Suy ra $C=2\cdot\dfrac{\sqrt{3}}{2}\cdot\left(\dfrac{-\sqrt{3}}{3}\right)-(-1)\cdot 1=0$.
\end{enumerate}	
\textbf{Chú ý.} Nếu để ý đến mối liên hệ giữa các góc có trong biểu thức, như các góc bù nhau, các góc phụ nhau, thì ta có thể giải bài toán theo cách sau
\begin{enumerate}
	\item Do $135^\circ=180^\circ-45^\circ$, $150^\circ=180^\circ-30^\circ$, $120^\circ=180^\circ-60^\circ$ nên 
	\allowdisplaybreaks
	\begin{eqnarray*}
		A&=&\sin 45^\circ\cdot(-\cos 45^\circ)+\cos 60^\circ\cdot\sin 30^\circ-\cos 30^\circ\cdot\sin 60^\circ\\
		&=&\dfrac{\sqrt{2}}{2}\cdot\left(\dfrac{-\sqrt{2}}{2}\right)+\dfrac{1}{2}\cdot\dfrac{1}{2}-\dfrac{\sqrt{3}}{2}\cdot\dfrac{\sqrt{3}}{2}=-\dfrac{1}{2}+\dfrac{1}{4}-\dfrac{3}{4}=-1.
	\end{eqnarray*}
\item Do $135^\circ=180^\circ-45^\circ$, $60^\circ=90^\circ-30^\circ$, $150^\circ=180^\circ-30^\circ$ nên
$$B=-1+1-\tan 60^\circ\cdot(-\tan 30^\circ)=1.$$
\item Do $150^\circ=180^\circ-30^\circ$ nên
\allowdisplaybreaks
\begin{eqnarray*}
	C&=&2\sin 60^\circ\cdot(-\tan 30^\circ)-\cos 180^\circ\cdot\cot 45^\circ\\
	&=&2\cdot\dfrac{\sqrt{3}}{2}\cdot\left(\dfrac{-\sqrt{3}}{3}\right)-(-1)\cdot 1=0.
\end{eqnarray*}
\end{enumerate}
}
\end{vd}

\baitaptl
\begin{bt}
Tính giá trị của các biểu thức
\begin{enumerate}
	\item $A= \sin 45^\circ+2 \sin 60^\circ+\tan 120^\circ+\cos 135^\circ$;
	\item $B= \tan 45^\circ \cdot \cot 135^\circ-\sin 30^\circ \cdot \cos 120^\circ-\sin 60^\circ \cdot \cos 150^\circ$;
	\item $C=\cos^2 5^\circ+\cos^2 25^\circ+\cos^2 45^\circ+\cos^2 65^\circ+\cos^2 85^\circ$;
	\item $D= \dfrac{12}{1+\tan^273^\circ} -4\tan 75^\circ\cdot\cot 105^\circ+12\sin^2 107^\circ-2 \tan 40^\circ \cdot \cos 60^\circ \cdot \tan 50^\circ$;
	\item $E=4 \tan 32^\circ \cdot \cos 60^\circ \cdot \cot 148^\circ+\dfrac{5 \cot^2 108^\circ}{1+\tan^2 18^\circ}+5\sin^272^\circ$.
\end{enumerate}	
\loigiai{
\begin{enumerate}
	\item 
	\allowdisplaybreaks
	\begin{eqnarray*}
		A&=& \sin 45^\circ+2 \sin 60^\circ+\tan 120^\circ+\cos 135^\circ\\
		&=&\dfrac{\sqrt{2}}{2}+2\cdot\dfrac{\sqrt{3}}{2}-\sqrt{3}-\dfrac{\sqrt{2}}{2}\\
		&=&\sqrt{3}-\sqrt{3}=0.
	\end{eqnarray*}
	\item 
	\allowdisplaybreaks
	\begin{eqnarray*}
		B&=& \tan 45^\circ \cdot \cot 135^\circ-\sin 30^\circ \cdot \cos 120^\circ-\sin 60^\circ \cdot \cos 150^\circ\\
		&=&1\cdot (-1)-\dfrac{1}{2}\cdot\left(\dfrac{-1}{2}\right)-\dfrac{\sqrt{3}}{2}\cdot \left(\dfrac{-\sqrt{3}}{2}\right)\\
		&=&-1+\dfrac{1}{4}+\dfrac{3}{4}=0.
	\end{eqnarray*}
	\item Do $5^\circ=90^\circ-85^\circ$, $25^\circ=90^\circ-65^\circ$ nên 
	\allowdisplaybreaks
	\begin{eqnarray*}
		C&=&\cos^25^\circ+\cos^2 25^\circ+\cos^2 45^\circ+\cos^2 65^\circ+\cos^2 85^\circ\\
		&=&\sin^285^\circ+\cos^285^\circ+\sin^225^\circ+\cos^225^\circ+\cos^245^\circ\\
		&=&1+1+\left(\dfrac{\sqrt{2}}{2}\right)^2=2+\dfrac{1}{2}=\dfrac{5}{2}.
	\end{eqnarray*}
	\item 
	\allowdisplaybreaks
	\begin{eqnarray*}
		D&=& \dfrac{12}{1+\tan^273^\circ} -4\tan 75^\circ\cdot\cot 105^\circ+12\sin^2 107^\circ-2 \tan 40^\circ \cdot \cos 60^\circ \cdot \tan 50^\circ\\
		&=&12\cos^273^\circ-4\tan75^\circ\cdot\cot(180^\circ-75^\circ)+12\sin^2(180^\circ-73^\circ)-2\tan(90^\circ-50)\cos60^\circ\tan 50^\circ\\
		&=&12\cos^273^\circ+4\tan75^\circ\cdot\cot75^\circ+12\sin^273^\circ-2\cot 50^\circ\cdot \tan 50^\circ\cdot \cos 60^\circ\\
		&=&12+4-1=15.
	\end{eqnarray*}
	\item Ta có do $148^\circ=180^\circ-32^\circ$, $108^\circ=180^\circ-72^\circ$ và $18^\circ=90^\circ-72^\circ$ nên
	\allowdisplaybreaks
	\begin{eqnarray*}
		E&=&4 \tan 32^\circ \cdot \cos 60^\circ \cdot \cot 148^\circ+\dfrac{5 \cot^2 108^\circ}{1+\tan^2 18^\circ}+5\sin^272^\circ\\
		&=&-4 \tan 32^\circ \cdot \cos 60^\circ \cdot \cot 32^\circ+5\cot^2108^\circ\cdot\cos^218^\circ+5\sin^272^\circ\\
		&=&-4\cdot\dfrac{1}{2}+5\cot^2108^\circ\cdot\sin^272^\circ+5\sin^272^\circ\\
		&=&-2+5\sin^272^\circ\cdot\left(1+\cot^2108^\circ\right)\\
		&=&-2+5\sin^272^\circ\cdot \dfrac{1}{\sin^2 108^\circ}\\
		&=&-2+5=3.
	\end{eqnarray*}
\end{enumerate}	
}
\end{bt}
\begin{bt}%[0H2K1]
	Tính giá trị các biểu thức sau:
	\begin{tasks}(1)
		\task $A=\sin^2 10^{\circ}+\sin^2 20^{\circ}+\dots+\sin^2 170^{\circ}+\sin^2 180^{\circ}$.
		\task $B=\tan 10^{\circ}.\tan 20^{\circ}\dots\tan 80^{\circ}$.
		\task $C=\cot 20^{\circ}+\cot 40^{\circ}+\dots +\cot 140^{\circ}+\cot160^{\circ}$.
	\end{tasks}
	\loigiai{
		\begin{enumerate}[a)]
			\item Ta có $\sin 10^{\circ}=\sin170^{\circ},\ \sin20^{\circ}=\sin160^{\circ},\dots$, suy ra $C= 2\bigl(\sin^2 10^{\circ}+\sin^2 20^{\circ}+\dots+\sin^2 80^{\circ}\bigr)+\sin^2 90^{\circ}$. Mặt khác ta có $\sin 80^{\circ}=\cos 10^{\circ},\ \sin 70^{\circ}=\cos 20^{\circ},\dots$, có 4 cặp như vậy nên ta tính được $A=5$.
			\item $\tan 10^{\circ}=\cot 80^{\circ}$, $\tan 20^{\circ}=\cot 70^{\circ}$, $\tan 30^{\circ}=\cot 60^{\circ}$, $\tan 40^{\circ}=\cot 50^{\circ}$. Do đó, ta tính được $B=1$.
			\item $\cot20^{\circ}=-\cot160^{\circ},\ \cot40^{\circ}=-\cot140^{\circ},\dots$ nên ta tính được $C=0$.
		\end{enumerate}
	}
\end{bt}
\begin{bt}
	Chứng minh rằng
	\begin{enumerate}
		\item $\sin^4\alpha+\cos^4\alpha=1-2\sin^2\alpha\cdot\cos^2\alpha$;
		\item $\sin^6\alpha+\cos^6\alpha=1-3\sin^2\alpha\cdot\cos^2\alpha$;
		\item $\sqrt{\sin^4\alpha+6\cos^2\alpha+3}+\sqrt{\cos^4\alpha+4\sin^2\alpha}=4$.
	\end{enumerate}
\loigiai{
	\begin{enumerate}
	\item Ta có
	\allowdisplaybreaks
	\begin{eqnarray*}
		\sin^4\alpha+\cos^4\alpha&=&(\sin^2\alpha)^2+(\cos^2\alpha)^2\\
		&=&(\sin^2\alpha)^2+(\cos^2\alpha)^2+2\sin^2\alpha\cdot\cos^2\alpha-2\sin^2\alpha\cdot\cos^2\alpha\\
		&=&\left(\sin^2\alpha+\cos^2\alpha\right)^2-2\sin^2\alpha\cdot\cos^2\alpha\\
		&=&1-2\sin^2\alpha\cdot\cos^2\alpha.
	\end{eqnarray*}
		\item Ta có
		\allowdisplaybreaks
	\begin{eqnarray*}
		\sin^6\alpha+\cos^6\alpha&=&(\sin^2\alpha)^3+(\cos^2\alpha)^3\\
		&=&\left(\sin^2\alpha+\cos^2\alpha\right)\cdot\left(\sin^4\alpha-\sin^2\alpha\cdot\cos^2\alpha+\cos^4\alpha\right)\\
		&=&\left(\sin^2\alpha+\cos^2\alpha\right)^2-3\sin^2\alpha\cdot\cos^2\alpha\\
		&=&1-3\sin^2\alpha\cdot\cos^2\alpha.
	\end{eqnarray*}
	\item \allowdisplaybreaks
	\begin{eqnarray*}
		&&\sqrt{\sin^4\alpha+6\cos^2\alpha+3}+\sqrt{\cos^4\alpha+4\sin^2\alpha}\\
		&=&\sqrt{\sin^4\alpha+6(1-\sin^2\alpha)+3}+\sqrt{\cos^4\alpha+4(1-\cos^2\alpha)}\\
		&=&\sqrt{\sin^4\alpha-6\sin^2\alpha+9}+\sqrt{\cos^4\alpha-4\cos^2\alpha+4}\\
		&=&\sqrt{(3-\sin^2\alpha)^2}+\sqrt{(2-\cos^2\alpha)}\\
		&=&3-\sin^2\alpha+2-\cos^2\alpha=5-(\sin^2\alpha + \cos^2 \alpha)=4.
	\end{eqnarray*}
\end{enumerate}
}
\end{bt}

\begin{bt}%[0H2B1]
	Cho $A, B, C$ là các góc của tam giác. Chứng minh các đẳng thức sau:
	\begin{tasks}(2)
		\task $\sin\left(A+B\right)=\sin C.$
		\task $\cos\left(A+B\right)+\cos C=0.$
		\task $\sin\dfrac{A+B}{2}=\cos\dfrac{C}{2}.$
		\task $\tan\left(A-B+C\right)=-\tan2B.$
	\end{tasks}
	\loigiai{ Do $A, B, C$ là các góc của tam giác nên ta có $A+B+C=180^{\circ}$.
		\begin{enumerate}[a)]
			\item Ta có $A+B+C=180^{\circ}\Leftrightarrow A+B=180^{\circ}-C.$\\
			Từ đó suy ra $\sin\left(A+B\right)=\sin \left(180^{\circ}-C\right)=\sin C.$
			\item Ta có $A+B+C=180^{\circ}\Leftrightarrow A+B=180^{\circ}-C.$\\
			Từ đó suy ra $\cos\left(A+B\right)=\cos \left(180^{\circ}-C\right)=-\cos C \Rightarrow \cos\left(A+B\right)+\cos C=0.$
			\item Ta có $A+B+C=180^{\circ}\Leftrightarrow \dfrac{A+B}{2}=\dfrac{180^{\circ}-C}{2}=90^{\circ}-\dfrac{C}{2}.$\\
			Từ đó suy ra $\sin\dfrac{A+B}{2}=\sin\left(90^{\circ}-\dfrac{C}{2}\right)= \cos\dfrac{C}{2}.$
			\item Ta có $\tan\left(A-B+C\right)=\tan\left(A+B+C-2B\right)=\tan\left(180^{\circ}-2B\right)=-\tan2B.$
		\end{enumerate}
	}
	\end{bt}
\begin{dang}{Tìm các GTLG khi biết một GTLG của góc}
Áp dụng tính chất về dấu của GTLG của một góc và các công thức lượng giác cơ bản.
\end{dang}
\viduminhhoa
\begin{vd}%[0H2B1-2]%[Nguyễn Tiến]%Ví dụ 1.
	\text{}
	\begin{enumerate}
		\item Cho $\sin\alpha=\dfrac{1}{3}$ với $90^\circ<\alpha<180^\circ$. Tính $\cos\alpha$ và $\tan\alpha$.
		\item Cho $\cos\alpha=-\dfrac{2}{3}$ và $\sin\alpha>0$. Tính $\sin\alpha$ và $\cot\alpha$.
		\item Cho $\tan\alpha=-2\sqrt{2}$, tính giá trị lượng giác còn lại.
	\end{enumerate}
	\loigiai{
		\begin{enumerate}
			\item Vì $90^\circ<\alpha<180^\circ$ nên $\cos\alpha<0$, mặt khác $\sin^2\alpha+\cos^2\alpha=1$ suy ra
			$$\cos\alpha=-\sqrt{1-\sin^2\alpha}=-\sqrt{1-\dfrac{1}{9}}=-\dfrac{2\sqrt{2}}{3}.$$
			Do đó $\tan\alpha=\dfrac{\sin\alpha}{\cos\alpha}=\dfrac{\dfrac{1}{3}}{-\dfrac{2\sqrt{2}}{3}}=-\dfrac{1}{2\sqrt{2}}$.
			\item Vì $\sin^2\alpha+\cos^2\alpha=1$ và $\sin\alpha>0$, nên $\sin\alpha=\sqrt{1-\cos^2\alpha}=\sqrt{1-\dfrac{4}{9}}=\dfrac{\sqrt{5}}{3}$.\\
			Ta có $\cot\alpha=\dfrac{\cos\alpha}{\sin\alpha}=\dfrac{-\dfrac{2}{3}}{\dfrac{\sqrt{5}}{3}}=-\dfrac{2}{\sqrt{5}}$.
			\item Vì $\tan\alpha=-2\sqrt{2}<0\Rightarrow\cos\alpha<0$.\\
			Ta có $\tan^2\alpha+1=\dfrac{1}{\cos^2\alpha}$, suy ra $\cos\alpha=-\sqrt{\dfrac{1}{\tan^2+1}}=-\sqrt{\dfrac{1}{8+1}}=-\dfrac{1}{3}$.\\
			Do đó $\tan\alpha=\dfrac{\sin\alpha}{\cos\alpha}\Rightarrow\sin\alpha=\tan\alpha\cdot\cos\alpha=-2\sqrt{2}\cdot\left(-\dfrac{1}{3}\right)=\dfrac{2\sqrt{2}}{3}$.\\
			$\Rightarrow\cot\alpha=\dfrac{\cos\alpha}{\sin\alpha}=\dfrac{-\dfrac{1}{3}}{\dfrac{2\sqrt{2}}{3}}=-\dfrac{1}{2\sqrt{2}}$.
		\end{enumerate}
	}
\end{vd}
\begin{vd}%[0H2B1-3]%[Nguyễn Tiến]%Ví dụ 2.
	\begin{enumerate}
		\item Cho $\cos\alpha=\dfrac{3}{4}$ với $0^\circ<\alpha<90^\circ$. Tính $A=\dfrac{\tan\alpha+3\cot\alpha}{\tan\alpha+\cot\alpha}$.
		\item Cho $\tan\alpha=\sqrt{2}$. Tính $B=\dfrac{\sin\alpha-\cos\alpha}{\sin^3\alpha+3\cos^3\alpha+2\sin\alpha}$.
	\end{enumerate}
	\loigiai{
		\begin{enumerate}
			\item Ta có $A=\dfrac{\tan\alpha+3\dfrac{1}{\tan\alpha}}{\tan\alpha+\dfrac{1}{\tan\alpha}}=\dfrac{\tan^2\alpha+3}{\tan^2\alpha+1}=\dfrac{\dfrac{1}{\cos^2\alpha}+2}{\dfrac{1}{\cos^2\alpha}}=1+2\cos^2\alpha$.\\
			Suy ra $A=1+2\cdot\dfrac{9}{16}=\dfrac{17}{8}$.
			\item Ta có $B=\dfrac{\dfrac{\sin\alpha}{\cos^3\alpha}-\dfrac{\cos\alpha}{\cos^3\alpha}}{\dfrac{\sin^3\alpha}{\cos^3\alpha}+\dfrac{3\cos^3\alpha}{\cos^3\alpha}+\dfrac{2\sin\alpha}{\cos^3\alpha}}=\dfrac{\tan\alpha\left(\tan^2\alpha+1\right)-\left(\tan^2\alpha+1\right)}{\tan^3\alpha+3+2\tan\alpha\left(\tan^2\alpha+1\right)}$.\\
			Suy ra $B=\dfrac{\sqrt{2}(2+1)-(2+1)}{2\sqrt{2}+3+2\sqrt{2}(2+1)}=\dfrac{3(\sqrt{2}-1)}{3+8\sqrt{2}}$.
		\end{enumerate}
	}
\end{vd}
\baitaptl
\begin{bt}
Cho góc $\alpha$, $0^\circ<\alpha<180^\circ$ thỏa mãn $\cos\alpha=\dfrac{-1}{3}$.
\begin{enumerate}
	\item Tính $\tan\alpha$.
	\item Tính giá trị của biểu thức $P=\tan\alpha+2\cot\alpha$.
\end{enumerate}	
\loigiai{
\begin{enumerate}
	\item Do $\cos\alpha=\dfrac{-1}{3}<0$ nên $\alpha$ là góc tù và $\tan\alpha=-\sqrt{\dfrac{1}{\cos^2\alpha}-1}=-2\sqrt{2}$.
	\item Do $\tan\alpha\cot\alpha=1$ và $\tan\alpha=-2\sqrt{2}$ nên $\cot\alpha=\dfrac{-\sqrt{2}}{4}$ và bởi vậy $$P=-2\sqrt{2}+2\cdot\left(\dfrac{-\sqrt{2}}{4}\right)=\dfrac{-5\sqrt{2}}{4}.$$
\end{enumerate}
\textbf{Nhận xét.} Khi tính $\tan\alpha$ từ $\cos\alpha$ nhờ đẳng thức $1+\tan^2\alpha=\dfrac{1}{\cos^2\alpha}$ sai lầm thường gặp của học sinh là mặc định coi $\tan\alpha=\sqrt{\dfrac{1}{\cos^2\alpha}-1}$ mà quên mất $\tan\alpha<0$ khi $\alpha$ là góc tù.
}
\end{bt}
\begin{bt}
Cho góc $\alpha$ thỏa mãn $0^\circ<\alpha<180^\circ$ và $\tan\alpha=2$. Tính giá trị của các biểu thức sau
\begin{enumerate}
	\item $G=2\sin\alpha+\cos\alpha$;
	\item $H=\dfrac{2\sin\alpha+\cos\alpha}{\sin\alpha-\cos\alpha}$.
\end{enumerate}	
\loigiai{
	\begin{enumerate}
		\item Do $\alpha$ thỏa mãn $0^\circ<\alpha<180^\circ$ và $\tan\alpha=2$ nên $\sin\alpha>0$ và $\cos\alpha>0$.\\
		Ta có $\cos\alpha=\sqrt{\dfrac{1}{1+\tan^2\alpha}}=\sqrt{\dfrac{1}{1+4}}=\dfrac{\sqrt{5}}{5}$.\\
		Từ đó $\sin\alpha=\tan\alpha\cdot\cos\alpha=\dfrac{2\sqrt{5}}{5}$.\\
		Vậy $G=2\sin\alpha+\cos\alpha=\dfrac{4\sqrt{5}}{5}+\dfrac{\sqrt{5}}{5}=\sqrt{5}$.
		\item Ta có $H=\dfrac{2\sin\alpha+\cos\alpha}{\sin\alpha-\cos\alpha}=\dfrac{2\tan\alpha+1}{\tan\alpha-1}=5$.
	\end{enumerate}
}
\end{bt}

\begin{bt}
Cho góc $\alpha$ với $90^\circ<\alpha<180^\circ$ thỏa mãn $\sin\alpha=\dfrac{3}{4}$. Tính giá trị của biểu thức $F=\dfrac{\tan\alpha+2\cot\alpha}{\tan\alpha+\cot\alpha}$.
\loigiai{
Do $\alpha\in (90^\circ;180^\circ)$ nên $\cos\alpha<0$.\\
Ta có $\cos\alpha=-\sqrt{1-\sin^2\alpha}=-\sqrt{1-\left(\dfrac{3}{4}\right)^2}=\dfrac{-\sqrt{7}}{4}$.\\
Suy ra $\tan\alpha=\dfrac{\sin\alpha}{\cos\alpha}=\dfrac{-3\sqrt{7}}{7}$ và $\cot\alpha=\dfrac{1}{\tan\alpha}=\dfrac{-\sqrt{7}}{3}$.\\
Vậy $F=\dfrac{\tan\alpha+2\cot\alpha}{\tan\alpha+\cot\alpha}=\dfrac{23}{16}$.
}	
\end{bt}

\begin{bt}
Cho góc $\alpha$ thỏa mãn $0^\circ<\alpha<180^\circ$ và $\tan\alpha=\sqrt{2}$. Tính giá trị của các biểu thức sau $$K=\dfrac{\sin^3\alpha+\sin\alpha\cdot\cos^2\alpha+2\sin^2\alpha\cdot\cos\alpha-4\cos^3\alpha}{\sin\alpha-\cos\alpha}.$$
\loigiai{
Ta có \allowdisplaybreaks
\begin{eqnarray*}
	K&=&\dfrac{\sin^3\alpha+\sin\alpha\cdot\cos^2\alpha+2\sin^2\alpha\cdot\cos\alpha-4\cos^3\alpha}{\sin\alpha-\cos\alpha}\\
	&=&\dfrac{\cos^3\alpha\left(\tan^3\alpha+\tan\alpha+2\tan^2\alpha-4\right)}{\cos^3\alpha\left(\tan\alpha\cdot(1+\tan^2\alpha)-(1+\tan^2\alpha)\right)}\\
	&=&\dfrac{\tan^3\alpha+\tan\alpha+2\tan^2\alpha-4}{(\tan\alpha-1)(1+\tan^2\alpha)}\\
	&=&\dfrac{2\sqrt{2}+\sqrt{2}+2\cdot 2-4}{(\sqrt{2}-1)(1+2)}\\
	&=&\dfrac{\sqrt{2}}{\sqrt{2}-1}=2+\sqrt{2}.
\end{eqnarray*}
}
\end{bt}
\subsection{Câu hỏi trắc nghiệm}
\Opensolutionfile{ans}[ans/ans-0D3-5-TN]
\begin{ex}%[0H2Y1-2]%[Nguyễn Tiến]%Câu 1.
	Giá trị của $\cos 60^\circ+\sin 30^\circ$ bằng bao nhiêu?
	\choice
	{$\dfrac{\sqrt{3}}{2}$}
	{$\sqrt{3}$}
	{$\dfrac{\sqrt{3}}{3}$}
	{\True $1$}
	\loigiai{
		Ta có $\cos 60^\circ+\sin 30^\circ=\dfrac{1}{2}+\dfrac{1}{2}=1$.
	}
\end{ex}
\begin{ex}%[0H2Y1-2]%[Nguyễn Tiến]%Câu 2.
	Giá trị của $\tan 30^\circ+\cot 30^\circ$ bằng bao nhiêu?
	\choice
	{\True $\dfrac{4}{\sqrt{3}}$}
	{$\dfrac{1+\sqrt{3}}{3}$}
	{$\dfrac{2}{\sqrt{3}}$}
	{$2$}
	\loigiai{
		Ta có $\tan 30^\circ+\cot 30^\circ=\dfrac{\sqrt{3}}{3}+\sqrt{3}=\dfrac{4\sqrt{3}}{3}$.
	}
\end{ex}
\begin{ex}%[0H2Y1-2]%[Nguyễn Tiến]%Câu 3.
	Trong các đẳng thức sau đây, đẳng thức nào \textbf{sai}?
	\choice
	{$\sin 0^\circ+\cos 0^\circ=1$}
	{$\sin 90^\circ+\cos 90^\circ=1$}
	{$\sin 180^\circ+\cos 180^\circ=-1$}
	{\True $\sin 60^\circ+\cos 60^\circ=1$}
	\loigiai{
		Ta có $\sin 60^\circ=\dfrac{\sqrt{3}}{2}$, $\cos 60^\circ=\dfrac{1}{2}$ nên đẳng thức sai là ``$\sin 60^\circ+\cos 60^\circ=1$''.
	}
\end{ex}
\begin{ex}%[0H2Y1-2]%[Nguyễn Tiến]%Câu 4.
	Trong các khẳng định sau, khẳng định nào \textbf{sai}?
	\choice
	{$\cos 60^\circ=\sin 30^\circ$}
	{\True $\cos 60^\circ=\sin 120^\circ$}
	{$\cos 30^\circ=\sin 120^\circ$}
	{$\sin 60^\circ=-\cos 120^\circ$}
	\loigiai{
		Ta có cặp góc $60^\circ$, $120^\circ$ bù nhau nên khẳng định sai là ``$\cos 60^\circ=\sin 120^\circ$''.
	}
\end{ex}
\begin{ex}%[0H2Y1-2]%[Nguyễn Tiến]%Câu 5.
	Đẳng thức nào sau đây \textbf{sai}?
	\choice
	{$\sin 45^\circ+\sin 45^\circ=\sqrt{2}$}
	{$\sin 30^\circ+\cos 60^\circ=1$}
	{$\sin 60^\circ+\cos 150^\circ=0$}
	{\True $\sin 120^\circ+\cos 30^\circ=0$}
	\loigiai{
		Ta có $\sin 120^\circ=\cos 30^\circ=\dfrac{\sqrt{3}}{2}$ nên đẳng thức sai là ``$\sin 120^\circ+\cos 30^\circ=0$''.
	}
\end{ex}
\begin{ex}%[0H2Y1-2]%[Nguyễn Tiến]%Câu 6.
	Giá trị $\cos 45^\circ+\sin 45^\circ$ bằng bao nhiêu?
	\choice
	{$1$}
	{\True $\sqrt{2}$}
	{$\sqrt{3}$}
	{$0$}
	\loigiai{
		Ta có $\cos 45^\circ=\sin 45^\circ=\dfrac{\sqrt{2}}{2}$ nên $\cos 45^\circ+\sin 45^\circ=\sqrt{2}$.
	}
\end{ex}
\begin{ex}%[0H2Y1-2]%[Nguyễn Tiến]%Câu 7.
	Trong các đẳng thức sau, đẳng thức nào \textbf{đúng}?
	\choice
	{$\sin\left( 180^\circ-\alpha\right)=-\cos\alpha$}
	{$\sin\left(180^\circ-\alpha\right)=-\sin\alpha$}
	{\True $\sin\left(180^\circ-\alpha\right)=\sin\alpha$}
	{$\sin\left(180^\circ-\alpha\right)=\cos\alpha$}
	\loigiai{
		Theo tính chất của cặp góc bù nhau thì ``$\sin\left(180^\circ-\alpha\right)=\sin\alpha$''.
	}
\end{ex}
\begin{ex}%[0H2Y1-2]%[Nguyễn Tiến]%Câu 8.
	Trong các đẳng thức sau, đẳng thức nào \textbf{sai}?
	\choice
	{\True $\sin 0^\circ+\cos 0^\circ=0$}
	{$\sin 90^\circ+\cos 90^\circ=1$}
	{$\sin 180^\circ+\cos 180^\circ=-1$}
	{$\sin 60^\circ+\cos 60^\circ=\dfrac{\sqrt{3}+1}{2}$}
	\loigiai{
		Ta có $\sin 0^\circ=0$, $\cos 0^\circ=1$ nên đẳng thức sai là ``$\sin 0^\circ+\cos 0^\circ=0$''.
	}
\end{ex}
\begin{ex}%[0H2Y1-2]%[Nguyễn Tiến]%Câu 9.
	Cho $\alpha$ là góc tù. Điều khẳng định nào sau đây là \textbf{đúng}?
	\choice
	{$\sin\alpha<0$}
	{$\cos\alpha>0$}
	{\True $\tan\alpha<0$}
	{$\cot\alpha>0$}
	\loigiai{
		Góc tù có điểm biểu diễn thuộc góc phần tư thứ II, suy ra $\tan\alpha<0$.
	}
\end{ex}
\begin{ex}%[0H2B1-2]%[Nguyễn Tiến]%Câu 10.
	Giá trị của $E=\sin 36^\circ\cos 6^\circ-\sin 126^\circ\cos 84^\circ$ là
	\choice
	{\True $\dfrac{1}{2}$}
	{$\dfrac{\sqrt{3}}{2}$}
	{$1$}
	{$-1$}
	\loigiai{
		Ta có
		\allowdisplaybreaks
		\begin{eqnarray*}
			E&= & \sin 36^\circ\cos 6^\circ-\sin\left(90^\circ+36^\circ\right)\cos\left(90^\circ-6^\circ\right)\\
			&= & \sin 36^\circ\cos 6^\circ-\cos 36^\circ\sin 6^\circ=\sin 30^\circ=\dfrac{1}{2}.
		\end{eqnarray*}
	}
\end{ex}
\begin{ex}%[0H2B1-2]%[Nguyễn Tiến]%Câu 11.
	Giá trị của biểu thức $A=\sin^2 51^\circ+\sin^2 55^\circ+\sin^2 39^\circ+\sin^2 35^\circ$ là
	\choice
	{$3$}
	{$4$}
	{$1$}
	{\True $2$}
	\loigiai{
		Ta có
		\allowdisplaybreaks
		\begin{eqnarray*}
			A&= & \left(\sin^2 51^\circ+\sin^2 39^\circ\right)+\left(\sin^2 55^\circ+\sin^2 35^\circ\right)\\
			&= & \left(\sin^2 51^\circ+\cos^2 51^\circ\right)+\left(\sin^2 55^\circ+\cos^2 55^\circ\right)=2.
		\end{eqnarray*}
	}
\end{ex}
\begin{ex}%[0H2K1-2]%[Nguyễn Tiến]%Câu 12.
	Giá trị của biểu thức $A=\tan 1^\circ\tan 2^\circ\tan 3^\circ\cdots\tan 88^\circ\tan 89^\circ$ là
	\choice
	{$0$}
	{$2$}
	{$3$}
	{\True $1$}
	\loigiai{
		Ta có $A=\left(\tan 1^\circ\cdot\tan 89^\circ\right)\cdot\left(\tan 2^\circ\cdot\tan 88^\circ\right)\cdots\left(\tan 44^\circ\cdot\tan 46^\circ\right)\cdot\tan 45^\circ=1$.
	}
\end{ex}
\begin{ex}%[0H2K1-2]%[Nguyễn Tiến]%Câu 13.
	Tổng $\sin^2 2^\circ+\sin^2 4^\circ+\sin^2 6^\circ+\cdots +\sin^2 84^\circ+\sin^2 86^\circ+\sin^2 88^\circ$ bằng
	\choice
	{$21$}
	{$23$}
	{\True $22$}
	{$24$}
	\loigiai{
		Ta có
		\allowdisplaybreaks
		\begin{eqnarray*}
			S&= & \sin^2 2^\circ+\sin^2 4^\circ+\sin^2 6^\circ+\cdots +\sin^2 84^\circ+\sin^2 86^\circ+\sin^2 88^\circ\\
			&= & \left(\sin^2 2^\circ+\sin^2 88^\circ\right)+\left(\sin^2 4^\circ+\sin^2 86^\circ\right)+\cdots +\left(\sin^2 44^\circ+\sin^2 46^\circ\right)\\
			&= & \left(\sin^2 2^\circ+\cos^2 2^\circ\right)+\left(\sin^2 4^\circ+\cos^2 4^\circ\right)+\cdots +\left(\sin^2 44^\circ+\cos^2 44^\circ\right)=22.
		\end{eqnarray*}
	}
\end{ex}
\begin{ex}%[0H2K1-2]%[Nguyễn Tiến]%Câu 14.
	Giá trị của $A=\tan 5^\circ\cdot\tan 10^\circ\cdot\tan 15^\circ\cdots\tan 80^\circ\cdot\tan 85^\circ$ là
	\choice
	{$2$}
	{\True $1$}
	{$0$}
	{$-1$}
	\loigiai{
		Ta có
		\allowdisplaybreaks
		\begin{eqnarray*}
			A&= & \left(\tan 5^\circ\cdot\tan 85^\circ\right)\cdot\left(\tan 10^\circ\cdot\tan 80^\circ\right)\cdots\left(\tan 40^\circ\tan 50^\circ\right)\cdot\tan 45^\circ\\
			&= & \left(\tan 5^\circ\cdot\cot 5^\circ\right)\cdot\left(\tan 10^\circ\cdot\cot 10^\circ\right)\cdots\left(\tan 40^\circ\cot 40^\circ\right)\cdot\tan 45^\circ =1.
		\end{eqnarray*}
	}
\end{ex}
\begin{ex}%[0H2B1-2]%[Nguyễn Tiến]%Câu 15.
	Giá trị của $B=\cos^2 73^\circ+\cos^2 87^\circ+\cos^2 3^\circ+\cos^2 17^\circ$ là
	\choice
	{$\sqrt{2}$}
	{\True $2$}
	{$-2$}
	{$1$}
	\loigiai{
		Ta có 
		\allowdisplaybreaks
		\begin{eqnarray*}
			B&= & \left(\cos^2 73^\circ+\cos^2 17^\circ\right)+\left(\cos^2 87^\circ+\cos^2 3^\circ\right)\\
			&= & \left(\cos^2 73^\circ+\sin^2 73^\circ\right)+\left(\cos^2 87^\circ+\sin^2 87^\circ\right)=2.
		\end{eqnarray*}
	}
\end{ex}
\begin{ex}%Câu 1.%[Nguyễn Chiến Thắng - TLDH7]%[0H2K1-2]
	Cho $\cos x=\dfrac 12$. Tính biểu thức $P=3\sin^2x+4\cos^2x$ 
	\choice
	{\True $\dfrac{13}{4}$}
	{$\dfrac{7}{4}$}
	{$\dfrac{11}{4}$}
	{$\dfrac{15}{4}$}
	\loigiai{
		Ta có $P=3\sin^2x+4\cos^2x=3\left(\sin^2x+\cos^2x\right)+\cos^2x=3+\left(\dfrac 12\right)^2=\dfrac{13}4$.}
\end{ex}
\begin{ex}%Câu 2.%[Nguyễn Chiến Thắng - TLDH7]%[0H2K1-2]
	Biết $\cos\alpha=\dfrac 13$. Giá trị đúng của biểu thức $P=\sin^2\alpha+3\cos^2\alpha$ là 
	\choice
	{$\dfrac{1}{3}$}
	{$\dfrac{10}{9}$}
	{\True $\dfrac{11}{9}$}
	{$\dfrac{4}{3}$}
	\loigiai{
		Ta có:	$\cos\alpha=\dfrac 13\Rightarrow P=\sin^2\alpha+3cos^2\alpha=\left(\sin^2\alpha+cos^2\alpha\right)+2cos^2\alpha=1+2cos^2\alpha=\dfrac{11}9$.}
\end{ex}
\begin{ex}%Câu 3.%[Nguyễn Chiến Thắng - TLDH7]%[0H2K1-2]
	Cho biết $\tan\alpha=\dfrac{1}{2}$. Tính $\cot\alpha$. 
	\choice
	{\True $\cot\alpha=2$}
	{$\cot\alpha=\sqrt{2}$}
	{$\cot\alpha=\dfrac{1}{4}$}
	{$\cot\alpha=\dfrac{1}{2}$}
	\loigiai{
		Ta có	$\tan\alpha\cdot\cot\alpha=1\Rightarrow\cot\alpha=\dfrac{1}{\tan\alpha}=2$.}
\end{ex}
\begin{ex}%Câu 4.%[Nguyễn Chiến Thắng - TLDH7]%[0H2K1-2]
	Cho biết $\cos\alpha=-\dfrac{2}{3}$ và $0<\alpha<\dfrac{\pi}{2}$. Tính $\tan\alpha$?
	\choice
	{$\dfrac{5}{4}$}
	{$-\dfrac{5}{2}$}
	{$\dfrac{\sqrt{5}}{2}$}
	{\True $-\dfrac{\sqrt{5}}{2}$}
	\loigiai{
		Do $0<\alpha<\dfrac{\pi}{2}\Rightarrow\tan\alpha<0$. \\
		Ta có: $1+\tan^2\alpha=\dfrac 1{\cos^2\alpha}\Leftrightarrow\tan^2\alpha=\dfrac 54\Rightarrow\tan\alpha=-\dfrac{\sqrt 5}2$.}
\end{ex}
\begin{ex}%Câu 5.%[Nguyễn Chiến Thắng - TLDH7]%[0H2K1-2]
	Cho $\alpha$ là góc tù và $\sin\alpha=\dfrac{5}{13}$. Giá trị của biểu thức $3\sin\alpha+2\cos\alpha$ là
	\choice
	{$3$}
	{\True $-\dfrac{9}{13}$}
	{$-3$}
	{$\dfrac{9}{13}$}
	\loigiai{
		Ta có $\cos^2\alpha=1-\sin^2\alpha=\dfrac{144}{169}\Rightarrow\cos\alpha=\pm\dfrac{12}{13}$.\\
		Do $\alpha$ là góc tù nên $\cos\alpha<0$, từ đó $\cos\alpha=-\dfrac{12}{13}$.\\
		Như vậy $3\sin\alpha+2\cos\alpha=3\cdot\dfrac{5}{13}+2\left(-\dfrac{12}{13}\right)=-\dfrac{9}{13}$.}
\end{ex}
\begin{ex}%Câu 6.%[Nguyễn Chiến Thắng - TLDH7]%[0H2K1-2]
	Cho biết $\sin\alpha+\cos\alpha=a$. Giá trị của $\sin\alpha\cdot\cos\alpha$ bằng bao nhiêu?
	\choice
	{$\sin\alpha\cdot\cos\alpha=a^2$}
	{$\sin\alpha\cdot\cos\alpha=2a$}
	{$\sin\alpha\cdot\cos\alpha=\dfrac{1-a^2}{2}$}
	{\True $\sin\alpha\cdot\cos\alpha=\dfrac{a^2-1}{2}$}
	\loigiai{
		$a^2=\left(\sin\alpha+\cos\alpha\right)^2=1+2\sin\alpha\cos\alpha\Rightarrow\sin\alpha\cos\alpha=\dfrac{a^2-1}{2}$.}
\end{ex}
\begin{ex}%Câu 7.%[Nguyễn Chiến Thắng - TLDH7]%[0H2K1-2]
	Cho biết $\cos\alpha=-\dfrac{2}{3}$. Tính giá trị của biểu thức $E=\dfrac{\cot\alpha+3\tan\alpha}{2\cot\alpha+\tan\alpha}$?
	\choice
	{$-\dfrac{19}{13}$}
	{\True $\dfrac{19}{13}$}
	{$\dfrac{25}{13}$}
	{$-\dfrac{25}{13}$}
	\loigiai{
		Ta có	$E=\dfrac{\cot\alpha+3\tan\alpha}{2\cot\alpha+\tan\alpha}=\dfrac{1+3\tan^2\alpha}{2+\tan^2\alpha}=\dfrac{3\left(\tan^2\alpha+1\right)-2}{1+\left(1+\tan^2\alpha\right)}=\dfrac{\dfrac 3{\cos^2\alpha}-2}{\dfrac 1{\cos^2\alpha}+1}=\dfrac{3-2\cos^2\alpha}{1+\cos^2\alpha}=\dfrac{19}{13}$.}
\end{ex}
\begin{ex}%Câu 8.%[Nguyễn Chiến Thắng - TLDH7]%[0H2K1-2]
	Cho biết $\cot\alpha=5$. Tính giá trị của $E=2\cos^2\alpha+5\sin\alpha\cos\alpha+1$?
	\choice
	{$\dfrac{10}{26}$}
	{$\dfrac{100}{26}$}
	{$\dfrac{50}{26}$}
	{\True $\dfrac{101}{26}$}
	\loigiai{
		$E=\sin^2\alpha\left(2\cot^2\alpha+5\cot\alpha+\dfrac{1}{\sin^2\alpha}\right)=\dfrac{1}{1+\cot^2\alpha}\left(3\cot^2\alpha+5\cot\alpha+1\right)=\dfrac{101}{26}$.}
\end{ex}
\begin{ex}%Câu 9.%[Nguyễn Chiến Thắng - TLDH7]%[0H2K1-2]
	Cho $\cot\alpha=\dfrac{1}{3}$. Giá trị của biểu thức $A=\dfrac{3\sin\alpha+4\cos\alpha}{2\sin\alpha-5\cos\alpha}$ là 
	\choice
	{$-\dfrac{15}{13}$}
	{$-13$}
	{$\dfrac{15}{13}$}
	{\True $13$}
	\loigiai{
		Ta có	$A=\dfrac{3\sin\alpha+4\sin\alpha\cdot\cot\alpha}{2\sin\alpha-5\sin\alpha\cdot\cot\alpha}=\dfrac{3+4\cot\alpha}{2-5\cot\alpha}=13$.}
\end{ex}
\begin{ex}%Câu 10.%[Nguyễn Chiến Thắng - TLDH7]%[0H2K1-2]
	Cho biết $\cos\alpha=-\dfrac{2}{3}$. Giá trị của biểu thức $E=\dfrac{\cot\alpha-3\tan\alpha}{2\cot\alpha-\tan\alpha}$ bằng bao nhiêu?
	\choice
	{$-\dfrac{25}{3}$}
	{$-\dfrac{11}{13}$}
	{\True $-\dfrac{11}{3}$}
	{$-\dfrac{25}{13}$}
	\loigiai{
		Ta có	$E=\dfrac{\cot\alpha-3\tan\alpha}{2\cot\alpha-\tan\alpha}=\dfrac{1-3\tan^2\alpha}{2-\tan^2\alpha}=\dfrac{4-3\left(\tan^2\alpha+1\right)}{3-\left(1+\tan^2\alpha\right)}=\dfrac{4-\dfrac{3}{\cos^2\alpha}}{3-\dfrac{1}{\cos^2\alpha}}=\dfrac{4\cos^2\alpha-3}{3\cos^2\alpha-1}=-\dfrac{11}{3}$.}
\end{ex}
\begin{ex}%Câu 11.%[Nguyễn Chiến Thắng - TLDH7]%[0H2K1-2]
	Biết $\sin a+\cos a=\sqrt{2}$. Hỏi giá trị của $\sin^4a+\cos^4a$ bằng bao nhiêu?
	\choice
	{$\dfrac{3}{2}$}
	{\True $\dfrac{1}{2}$}
	{$-1$}
	{$0$}
	\loigiai{
		Ta có: $\sin a+\cos a=\sqrt{2}\Rightarrow 2=\left(\sin a+\cos a\right)^2\Rightarrow\sin a\cdot\cos a=\dfrac{1}{2}$.\\
		$\sin^4a+\cos^4a=\left(\sin^2a+\cos^2a\right)-2\sin^2a\cos^2a=1-2\left(\dfrac{1}{2}\right)^2=\dfrac{1}{2}$.}
\end{ex}
\begin{ex}%Câu 12.%[Nguyễn Chiến Thắng - TLDH7]%[0H2K1-2]
	Cho $\tan\alpha+\cot\alpha=m$. Tìm $m$ để $\tan^2\alpha+\cot^2\alpha=7$. 
	\choice
	{$m=9$}
	{$m=3$}
	{$m=-3$}
	{\True $m=\pm 3$}
	\loigiai{
		Ta có	$7=\tan^2\alpha+\cot^2\alpha=\left(\tan\alpha+\cot\alpha\right)^2-2\Rightarrow m^2=9\Leftrightarrow m=\pm 3$.}
\end{ex}
\begin{ex}%Câu 13.%[Nguyễn Chiến Thắng - TLDH7]%[0H2K1-2]
	Cho biết $3\cos\alpha-\sin\alpha=1$, $0^{\circ}<\alpha<90^{\circ}$ Giá trị của $\tan\alpha$ bằng
	\choice
	{\True $\tan\alpha=\dfrac{4}{3}$}
	{$\tan\alpha=\dfrac{3}{4}$}
	{$\tan\alpha=\dfrac{4}{5}$}
	{$\tan\alpha=\dfrac{5}{4}$}
	\loigiai{
		Ta có $3\cos\alpha-\sin\alpha=1\Leftrightarrow 3\cos\alpha=\sin\alpha+1\to 9\cos^2\alpha=\left(\sin\alpha+1\right)^2$ \\
		$ \Leftrightarrow 9\cos^2\alpha=\sin^2\alpha+2\sin\alpha+1\Leftrightarrow 9\left(1-\sin^2\alpha\right)=\sin^2\alpha+2\sin\alpha+1 $ \\
		$ \Leftrightarrow 10\sin^2\alpha+2\sin\alpha-8=0\Leftrightarrow\hoac{&\sin\alpha=-1\\&\sin\alpha=\dfrac{4}{5}} $.
		\begin{itemize}
			\item 	$\sin\alpha=-1 $: không thỏa mãn vì $0^{\circ}<\alpha<90^{\circ}$.
			\item 	$\sin\alpha=\dfrac{4}{5}\Rightarrow\cos\alpha=\dfrac{3}{5}\Rightarrow\tan\alpha=\dfrac{\sin\alpha}{\cos\alpha}=\dfrac{4}{3}$.
		\end{itemize}
	}
\end{ex}
\begin{ex}%Câu 14.%[Nguyễn Chiến Thắng - TLDH7]%[0H2K1-2]
	Cho biết $2\cos\alpha+\sqrt{2}\sin\alpha=2$, $0^{\circ}<\alpha<90^{\circ}$. Tính giá trị của $\cot\alpha$. 
	\choice
	{$\cot\alpha=\dfrac{\sqrt{5}}{4}$}
	{$\cot\alpha=\dfrac{\sqrt{3}}{4}$}
	{\True $\cot\alpha=\dfrac{\sqrt{2}}{4}$}
	{$\cot\alpha=\dfrac{\sqrt{2}}{2}$}
	\loigiai{
		Ta có $2\cos\alpha+\sqrt{2}\sin\alpha=2\Leftrightarrow\sqrt{2}\sin\alpha=2-2\cos\alpha\to 2\sin^2\alpha=\left(2-2\cos\alpha\right)^2$.\\
		$\begin{aligned}&\Leftrightarrow 2\sin^2\alpha=4-8\cos\alpha+4\cos^2\alpha\Leftrightarrow 2\left(1-\cos^2\alpha\right)=4-8\cos\alpha+4\cos^2\alpha\\&\Leftrightarrow 6\cos^2\alpha-8\cos\alpha+2=0\Leftrightarrow\hoac{&\cos\alpha=1\\&\cos\alpha=\dfrac{1}{3}}.\end{aligned}$ 
		\begin{itemize}
			\item 	$\cos\alpha=1$: không thỏa mãn vì $0^{\circ}<\alpha<90^{\circ}$.
			\item 	$\cos\alpha=\dfrac{1}{3}\Rightarrow\sin\alpha=\dfrac{2\sqrt{2}}{3}\Rightarrow\cot\alpha=\dfrac{\cos\alpha}{\sin\alpha}=\dfrac{\sqrt{2}}{4}$.
		\end{itemize}
	}
\end{ex}
\begin{ex}%Câu 15.%[Nguyễn Chiến Thắng - TLDH7]%[0H2G1-2]
	Cho biết $\cos\alpha+\sin\alpha=\dfrac{1}{3}$. Giá trị của $P=\sqrt{\tan^2\alpha+\cot^2\alpha}$ bằng bao nhiêu?
	\choice
	{$P=\dfrac{5}{4}$}
	{\True $P=\dfrac{7}{4}$}
	{$P=\dfrac{9}{4}$}
	{$P=\dfrac{11}{4}$}
	\loigiai{
		Ta có $\cos\alpha+\sin\alpha=\dfrac{1}{3}\to\left(\cos\alpha+\sin\alpha\right)^2=\dfrac{1}{9}\Leftrightarrow 1+2\sin\alpha\cos\alpha=\dfrac{1}{9}\Leftrightarrow\sin\alpha\cos\alpha=-\dfrac{4}{9}$.\\
		Ta có $P=\sqrt{\tan^2\alpha+\cot^2\alpha}=\sqrt{\left(\tan\alpha+\cot\alpha\right)^2-2\tan\alpha\cot\alpha}=\sqrt{\left(\dfrac{\sin\alpha}{\cos\alpha}+\dfrac{\cos\alpha}{\sin\alpha}\right)^2-2}$.\\
		$=\sqrt{\left(\dfrac{\sin^2\alpha+\cos^2\alpha}{\sin\alpha\cos\alpha}\right)^2-2}=\sqrt{\left(\dfrac{1}{\sin\alpha\cos\alpha}\right)^2-2}=\sqrt{\left(-\dfrac{9}{4}\right)^2-2}=\dfrac{7}{4}$.}
\end{ex}
\begin{ex}%Câu 16.%[Nguyễn Chiến Thắng - TLDH7]%[0H2G1-2]
	Cho biết $\sin\alpha-\cos\alpha=\dfrac{1}{\sqrt{5}}$. Giá trị của $P=\sqrt{\sin^4\alpha+\cos^4\alpha}$ bằng bao nhiêu?
	\choice
	{$P=\dfrac{\sqrt{15}}{5}$}
	{\True $P=\dfrac{\sqrt{17}}{5}$}
	{$P=\dfrac{\sqrt{19}}{5}$}
	{$P=\dfrac{\sqrt{21}}{5}$}
	\loigiai{
		Ta có $\sin\alpha-\cos\alpha=\dfrac{1}{\sqrt{5}}\to\left(\sin\alpha-\cos\alpha\right)^2=\dfrac{1}{5}\Leftrightarrow 1-2\sin\alpha\cos\alpha=\dfrac{1}{5}\Leftrightarrow\sin\alpha\cos\alpha=\dfrac{2}{5}$.\\
		$P=\sqrt{\sin^4\alpha+\cos^4\alpha}=\sqrt{\left(\sin^2\alpha+\cos^2\alpha\right)^2-2\sin^2\alpha\cos^2\alpha} =\sqrt{1-2\left(\sin\alpha cos\alpha\right)^2}=\dfrac{\sqrt{17}}{5}$.}
\end{ex}
\Closesolutionfile{ans}

 \def\tenchude{HỆ THỨC LƯỢNG TRONG TAM GIÁC}
\setcounter{section}{1}
\setcounter{dang}{0}
\section{HỆ THỨC LƯỢNG TRONG TAM GIÁC}
\subsection{Tóm tắt lý thuyết}
\subsubsection{Định lý cosin}
\immini{Cho tam giác $ ABC$ có $ BC=a$, $ AC=b$ và $ AB=c$.
	\begin{listEX}
		\item[$\bullet$] $ a^2=b^2+c^2-2bc\cdot \cos A \Rightarrow \cos A=$ \dotfill 
		\item[$\bullet$] $ b^2=c^2+a^2-2ca\cdot \cos B \Rightarrow \cos B=$ \dotfill 
		\item[$\bullet$] $ c^2=a^2+b^2-2ab\cdot \cos C \Rightarrow \cos A=$\dotfill 
	\end{listEX}
}
{\begin{tikzpicture}[scale=0.7,font=\footnotesize,line join=round, line cap=round,>=stealth]
		\tkzDefPoints{0/0/B,1/2/A,4/0/C}
		\tkzDrawPoints[fill=black](A,B,C)
		\tkzDefMidPoint(A,B) \tkzGetPoint{c}
		\tkzDefMidPoint(C,B) \tkzGetPoint{a}
		\tkzDefMidPoint(A,C) \tkzGetPoint{b}
		\tkzDrawPolygon(A,B,C)
		\tkzLabelPoints[above](A) 
		\tkzLabelPoints[below](B,C,a)
		\tkzLabelPoints[left](c)
		\tkzLabelPoints[right](b)
\end{tikzpicture}}
\subsubsection{Định lý sin}
\immini{
	Cho tam giác $ ABC$ có $ BC=a,AC=b$, $ AB=c$ và 
	$ R$ là bán kính đường tròn ngoại tiếp. 
	Ta có 
	$$ \dfrac{a}{\sin A}=\dfrac{b}{\sin B}=\dfrac{c}{\sin C}=2R$$
	\begin{note}
		Ghi nhớ: Tỉ lệ "cạnh chia sin góc đối" thì bằng nhau.
	\end{note}
}
{
	\begin{tikzpicture}[scale=0.6,font=\footnotesize,line join=round, line cap=round,>=stealth]
		\tkzDefPoints{0/0/B,1/3/A,4/0/C}
		\tkzCircumCenter(A,B,C) \tkzGetPoint{I}
		\tkzDrawPoints[fill=black](A,B,C,I)
		\tkzDrawCircle(I,A)
		\tkzDefMidPoint(A,B) \tkzGetPoint{c}
		\tkzDefMidPoint(C,B) \tkzGetPoint{a}
		\tkzDefMidPoint(A,C) \tkzGetPoint{b}
		\tkzDefMidPoint(a,c) \tkzGetPoint{R}
		\tkzDrawPolygon(A,B,C)
		\tkzDrawSegments(I,A I,B I,C)
		\tkzLabelPoints[above](A) 
		\tkzLabelPoints[below](B,C,a,I,R)
		\tkzLabelPoints[left](c)
		\tkzLabelPoints[right](b)
\end{tikzpicture}} 

\subsubsection{Công thức tính diện tích tam giác}
Gọi $S$ là diện tích tam giác $ABC$. Ta có
\begin{itemize}
	\item $S=\dfrac{1}{2}a\cdot h_a=\dfrac{1}{2}b\cdot h_b=\dfrac{1}{2}c\cdot h_c$,
	\item $S=\dfrac{1}{2}bc\sin A=\dfrac{1}{2}ca\sin B=\dfrac{1}{2}ab\sin C$,
	\item $S=\dfrac{abc}{4R}$, $S=p\cdot r$, (đọc thêm)
	\item $S=\sqrt{p(p-a)(p-b)(p-c)}$.
\end{itemize}
Trong đó:
\begin{itemize}
	\item [$\bullet$] $ h_a$, $h_b$, $h_c$ là độ dài đường cao lần lượt tương ứng với các cạnh $ BC$, $CA$, $AB$.
	\item [$\bullet$] $ R$ là bán kính đường tròn ngoại tiếp tam giác.
	\item [$\bullet$] $ r$ là bán kính đường tròn nội tiếp tam giác.
	\item [$\bullet$] $ p=\dfrac{a+b+c}{2}$ là nửa chu vi tam giác.
\end{itemize}

\subsection{Các dạng toán}
\begin{dang}{Áp dụng định lý cos}
	\textbf{Nhận dạng định lý:}
	\begin{itemize}
		\item [$\bullet$] Cho tam giác biết trước độ dài hai cạnh và số đo của một góc.
		\item [$\bullet$] Cho tam giác biết trước độ dài ba cạnh.
	\end{itemize}
\end{dang}
\viduminhhoa
\begin{vd}
	Cho tam giác $ABC$ có $b=5, c=7$ và $\cos A=\dfrac{3}{5}$. Tính cạnh $a$ và cosin các góc còn lại của tam giác đó.
	\loigiai{Ta có:
		\begin{align*}
			&a^2=b^2+c^2-2bc\cos A=25+49-2.5.7.\dfrac{3}{5}=32\Rightarrow a=\sqrt{32}=4\sqrt{2} \\
			&\cos B=\dfrac{c^2+a^2-b^2}{2ca}=\dfrac{32+49-25}{56\sqrt{2}}=\dfrac{\sqrt{2}}{2}\\
			&\cos C=\dfrac{a^2+b^2-c^2}{2ab}=\dfrac{32+25-49}{40\sqrt{2}}=\dfrac{8}{40\sqrt{2}}=\dfrac{\sqrt{2}}{10}.
		\end{align*}
	}
\end{vd}
\begin{vd}
	Cho tam giác $ABC$ có $AC = 10 \textrm{cm}, BC = 16 \textrm{cm}$ và $C=120^\circ$, tính độ dài cạnh $AB$.
	\loigiai{Áp dụng định lý hàm số cosin ta có $AB^2=CA^2+CB^2-2CA.CB\cos C$ ta suy ra $AB=\sqrt{516}\, \textrm{cm}$
	}
\end{vd}
\begin{note}
	Cho tam giác $ ABC$ có $ m_a$, $ m_b$, $ m_c$ lần lượt là các trung tuyến kẻ từ $ A$, $ B$, $ C$. 
	Ta có
	\immini{
		\begin{listEX}
			\item [$\bullet$] $ m_a^2=\dfrac{b^2+c^2}{2}-\dfrac{a^2}{4}$.
			\item [$\bullet$] $ m_b^2=\dfrac{a^2+c^2}{2}-\dfrac{b^2}{4}$.
			\item [$\bullet$] $ m_{c}^2=\dfrac{a^2+b^2}{2}-\dfrac{c^2}{4}$.
		\end{listEX}
	}
	{
		\begin{tikzpicture}[scale=0.7,font=\footnotesize,line join=round, line cap=round,>=stealth]
			\tkzDefPoints{0/0/B,1/3/A,6/0/C}
			\tkzDefMidPoint(A,B) \tkzGetPoint{c}
			\tkzDefMidPoint(C,B) \tkzGetPoint{a}
			\tkzDefMidPoint(A,C) \tkzGetPoint{b}
			\coordinate (m_a) at ($ (A)!0.4!(a)$ );
			\coordinate (m_b) at ($ (B)!0.4!(b)$ );
			\coordinate (m_c) at ($ (C)!0.4!(c)$ );
			\tkzDrawPoints[fill=black](A,B,C,a,b,c)
			\tkzDrawPolygon(A,B,C)
			\tkzDrawSegments(a,A b,B c,C)
			\tkzLabelPoints[above](A) 
			\tkzLabelPoints[below](B,C,a,m_b,m_c)
			\tkzLabelPoints[left](c,m_a)
			\tkzLabelPoints[above right](b)
	\end{tikzpicture}}
\end{note}
\begin{vd}
	Cho tam giác $ABC$ có $AB=4~\mathrm{cm}$, $AC=3 ~\mathrm{cm}$ và $BC=6 ~\mathrm{cm}$. Tính
	độ dài trung tuyến kẻ từ $C$ của tam giác $ABC$. 
	\loigiai{	\immini{Độ dài trung tuyến kẻ từ $C$ của tam giác $ABC$ là
			\\
			$ m_{c}^2=\dfrac{a^2+b^2}{2}-\dfrac{c^2}{4}= \dfrac{6^2 + 3^2}{2} - \dfrac{4^2}{4} =\dfrac{37}{2} \Rightarrow m_c = \dfrac{\sqrt{74}}{2}$.	
		}
		{
			\begin{tikzpicture}[scale=0.7,font=\footnotesize,line join=round, line cap=round,>=stealth]
				\tkzDefPoints{0/0/B,1/3/A,6/0/C}
				\tkzDefMidPoint(A,B) \tkzGetPoint{c}
				\tkzDefMidPoint(C,B) \tkzGetPoint{a}
				\tkzDefMidPoint(A,C) \tkzGetPoint{b}
				\coordinate (m_a) at ($ (A)!0.3!(a)$ );
				\coordinate (m_b) at ($ (B)!0.3!(b)$ );
				\coordinate (m_c) at ($ (C)!0.3!(c)$ );
				\tkzDrawPoints[fill=black](A,B,C,a,b,c)
				\tkzDrawPolygon(A,B,C)
				\tkzDrawSegments(a,A b,B c,C)
				\tkzLabelPoints[above](A) 
				\tkzLabelPoints[below](B,C,a,m_b,m_c)
				\tkzLabelPoints[left](c,m_a)
				\tkzLabelPoints[above right](b)
\end{tikzpicture}}}\end{vd}
\begin{vd}
	Cho tam giác $ABC$ có $BC=3, CA=4$ và $AB=6$. Tính cosin của góc có số đo lớn nhất của tam giác đã cho.
	\loigiai{Do $AB>AC>BC$ nên $C>B>A$.\\
		Áp dụng định lý hàm số cosin ta có $\cos C=-\dfrac{11}{24}.$
	}
\end{vd}
\begin{vd}
	\immini
	{
		Hai chiếc tàu thủy cùng xuất phát từ một vị trí $ A$, đi thẳng theo hai hướng tạo với nhau góc $ 60^\circ$. Tàu $ B$ chạy với tốc độ $ 20$ hải lí một giờ. Tàu $ C$ chạy với tốc độ $ 15$ hải lí một giờ. Hỏi sau hai giờ, hai tàu cách nhau bao nhiêu hải lí?
	}
	{
		\begin{tikzpicture}[line join=round, line cap=round,>=stealth,scale=1]
			\foreach \x in {1,2,3}
			\foreach \y in {0,1,2,3,4,5}
			\draw(0.25+\y,\x-0.25)--(\y,\x-0.25) ;
			\foreach \x in {1,2,3}
			\foreach \y in {0,1,2,3,4,5}
			\draw(-0.25+\y,\x-0.75)--(\y,\x-0.75) ;
			\tkzDefPoints{0/0/A,4/0/B,1.5/2.5/C}
			\tkzDefMidPoint(A,B) \tkzGetPoint{40}
			\tkzDefMidPoint(A,C) \tkzGetPoint{30} 
			\tkzDrawPoints[fill=black](A,B,C)
			\tkzDrawSegments(A,B A,C)
			\fill plot [smooth cycle] coordinates{(3.75,0)(4.25,0)(4.3,0.2)(4.15,0.2)(4.15,0.3)(4.05,0.3)(4.05,0.4)(3.95,0.4)(3.95,0.3)(3.85,0.3)(3.85,0.2)(3.7,0.2)} ;
			\fill plot [smooth cycle] coordinates{(1.25,2.5)(1.75,2.5)(1.8,2.7)(1.65,2.7)(1.65,2.8)(1.55,2.8)(1.55,2.9)(1.45,2.9)(1.45,2.8)(1.35,2.8)(1.35,2.7)(1.2,2.7)} ;
			\tkzLabelPoints[below](A,B,C,40)
			\tkzLabelPoints[left](30)
		\end{tikzpicture}
	}
	\loigiai{
		Sau $ 2$ giờ tàu $ B$ đi được $ 40$ hải lí, tàu $ C$ đi được $ 30$ hải lí. \\
		Vậy tam giác $ ABC$ có $ AB=40$, $AC=30$ và $ \widehat{A}=60^\circ$. \\
		Áp dụng định lí cosine vào tam giác $ ABC$, ta có
		$$ a^2=b^2+c^2-2bc\cdot\cos A=30^2+40^2-2\cdot 30\cdot 40\cdot \cos 60^\circ=1300\Rightarrow a\simeq 36.$$ 
		Vậy sau $ 2$ giờ, hai tàu cách nhau khoảng $ 36$ hải lí.}
\end{vd}
\begin{vd}%[0H2B3-2]
	Tam giác $ABC$ có $AB= c$; $BC=a$; $CA= b.$ Các cạnh $a$, $b$, $c$ liên hệ với nhau bởi đẳng thức $b(b^2 -a^2) = c(a^2 -c^2)$. Tính số đo góc $\widehat{BAC}$.
	\loigiai{
		Theo định lý hàm côsin, ta có $$\cos \widehat{BAC}= \dfrac{AB^2 + AC^2 - BC^2}{2 \cdot AB\cdot AC}= \dfrac{c^2 + b^2 -a^2}{2bc}.$$
		Mà 
		\begin{eqnarray*}
			& & b(b^2 -a^2) = c(a^2 -c^2) \\ 
			& \Leftrightarrow & b^3 -a^2 b = a^2 c -c^3 \\
			& \Leftrightarrow &-a^2 (b+c) + (b+c)(b^2 + c^2 -bc)=0\\
			& \Leftrightarrow & b^2 + c^2 - a^2 =bc.
		\end{eqnarray*} 
		Khi đó $\cos \widehat{BAC} = \dfrac{bc}{2bc}= \dfrac{1}{2}$.\\
		Vậy $\widehat{BAC}= 60^\circ$.
	}
\end{vd}
\baitaptl
\begin{bt}
	Cho tam giác $ABC$ có $\widehat{A} = 60^\circ$, $AB=6$, $AC=8$. Tính $BC$.
	\loigiai{Áp dụng định lý cosine trong tam giác $ABC$ ta có $BC^2=AB^2 +AC^2 -2AB\cdot AC \cdot \cos A=6^2 +8^2 -2\cdot6\cdot 8 \cos 60^\circ = 52 \Rightarrow BC=2\sqrt{13}$.}
\end{bt}
\begin{bt}
	Cho tam giác $ABC$ có các cạnh $BC=6$, $CA=4\sqrt{2}$, $AB=2$. Tính $\cos A$ và góc $\widehat{A}$.
	\loigiai{Áp dụng hệ quả của định lý cosine ta có\\
		$\cos A = \dfrac{AB^2+AC^2 -BC^2}{2AB \cdot AC}= \dfrac{2^2 + \left(4\sqrt{2}\right)^2 -6^2}{2\cdot 2\cdot 4\sqrt{2}} =0 \Leftrightarrow \widehat{A}=90^\circ $.}
\end{bt}
\begin{bt}
	Cho tam giác $ABC$ có $AB=6$ cm; $AC =5$ cm và $\widehat{ACB}=60^\circ$. Tính $BC$.
	\loigiai{ Áp dụng định lý cosine trong tam giác ta có\\ $AB^2 =AC^2 +BC^2 -2AC\cdot BC \cdot \cos \widehat{ACB}$ $\Rightarrow 6^2 =5^2 +BC^2 -2\cdot 5 \cdot BC \cos 60^\circ$\\ $\Leftrightarrow BC^2 -5BC-11=0 \Leftrightarrow BC =\dfrac{5+ \sqrt{69}}{2}$.}
\end{bt}
\begin{bt}
	Tam giác $ABC$ có $b= 6$, $c= 8$ và $m_a= 5$. Tính $a$, $\widehat{A}$.
	\loigiai{Áp dụng công thức đường trung tuyến trong tam giác ta có 
		\\
		$m_a^2 = \dfrac{b^2+c^2}{2} -\dfrac{a^2}{4} \Leftrightarrow 5^2 = \dfrac{6^2+8^2}{2} - \dfrac{a^2}{4} \Leftrightarrow a=10$.
	}
\end{bt}
\begin{bt}%[0H2K3]
	Cho tam giác $ABC$, gọi $l_a$ là độ dài đường phân giác trong kẻ từ đỉnh $A$ của tam giác $ABC$. Chứng minh rằng $l_a=\dfrac{bc\sin A}{(b+c)\sin\frac{A}{2}}$.
	\loigiai{
		\immini{
			Gọi $D$ là chân đường phân giác trong kẻ từ đỉnh $A$ của tam giác $ABC$. Ta có $l_a=AD$. Ta có
			$$\begin{aligned}
				&S_{ABC}=S_{ABD}+S_{ACD}\\
				\Leftrightarrow &\dfrac{1}{2}AB.AC.\sin A=\dfrac{1}{2}AB.AD.\sin \frac{A}{2}+\dfrac{1}{2}AC.AD.\sin \frac{A}{2}\\
				\Leftrightarrow & cb\sin A=l_a(c+b)\sin\frac{A}{2}\\
				\Leftrightarrow &l_a=\dfrac{bc\sin A}{(b+c)\sin\frac{A}{2}}
			\end{aligned}$$
		}{
			% \begin{tikzpicture}
			% 	\clip (-3,-3) rectangle (4,2);
			% 	\tkzDefPoints{0/1/A,-1/-2/B,3/-2/C}
			% 	\draw (A)node[above]{$A$}--(B)node[below]{$B$}--(C)node[below]{$C$}--(A);
			% 	\tkzInCenter(A,B,C) \tkzGetPoint{I}
			% 	\tkzInterLL(B,C)(A,I) \tkzGetPoint{D}
			% 	\tkzDrawBisector(B,A,C)
			% 	\tkzLabelPoints[below](D)
			% 	\tkzMarkAngle[size=0.5](D,A,C)
			% 	\tkzMarkAngle[size=0.6](B,A,D)
			% \end{tikzpicture}
		}
	}
\end{bt}
\begin{bt}
	Hai lực $\overrightarrow{f_1}$ và $\overrightarrow{f_2}$ cho trước cùng tác dụng lên một vật và tạo thành góc nhọn $\left(\overrightarrow{f_1},\overrightarrow{f_2}\right)=\alpha$. Hãy lập công thức tính cường độ của hợp lực $\overrightarrow{s}$.
	\loigiai{
		\centerline{
			\begin{tikzpicture}[>=stealth,scale=1]
				\tkzDefPoints{0/0/A,1/2/B}
				\tkzDefShiftPoint[A](0:4){D}
				\tkzDefShiftPoint[B](0:4){C}
				\tkzDrawSegments[->](A,B A,D A,C)
				\tkzDrawSegments(B,C C,D)
				\tkzLabelPoints[left](B,A)
				\tkzLabelPoints[right](C,D)
				\tkzMarkAngles[mkpos=.2,size=.3](D,A,B A,B,C)
				\tkzLabelAngle[pos=.6](D,A,B){\footnotesize $\alpha$}
				\tkzLabelLine[pos=.5,left](A,B){\footnotesize $\overrightarrow{f_1}$}
				\tkzLabelLine[pos=.5,below](A,D){\footnotesize $\overrightarrow{f_2}$}
				\tkzLabelLine[pos=.5,above left](A,C){\footnotesize $\overrightarrow{s}$}
		\end{tikzpicture}}\\
		Đặt $\overrightarrow{AB}=\overrightarrow{f_1}, \overrightarrow{AD}=\overrightarrow{f_2}$ và vẽ hình bình hành $ABCD$.\\
		Khi đó $\overrightarrow{AC}=\overrightarrow{AB}+\overrightarrow{AD}=\overrightarrow{f_1}+\overrightarrow{f_2}=\overrightarrow{s}$.\\
		Vậy $\left|\overrightarrow{s}\right|=\left|\overrightarrow{AC}\right|=\left|\overrightarrow{f_1}+\overrightarrow{f_2}\right|$.\\
		Theo định lí côsin đối với tam giác $ABC$, ta có:\\
		$AC^2=AB^2+BC^2-2.AB.BC.\cos B$, hay $\left|\overrightarrow{s}\right|^2=\left|\overrightarrow{f_1}\right|^2+\left|\overrightarrow{f_2}\right|^2-2\left|\overrightarrow{f_1}\right|.\left|\overrightarrow{f_2}\right|.\cos(180^\circ-\alpha)$.\\
		Do đó: $\left|\overrightarrow{s}\right|=\sqrt{\left|\overrightarrow{f_1}\right|^2+\left|\overrightarrow{f_2}\right|^2+2\left|\overrightarrow{f_1}\right|.\left|\overrightarrow{f_2}\right|.\cos\alpha}$.
	}
\end{bt}
\begin{dang}{Áp dụng định lý sin}
\textbf{Nhận dạng định lý:}
\begin{itemize}
	\item [$\bullet$] Cho tam giác biết trước độ dài hai cạnh và số đo của một góc.
	\item [$\bullet$] Cho tam giác biết trước độ dài một cạnh và số đo của hai góc.
	\item [$\bullet$] Cho tam giác biết trước độ dài một cạnh, số đo góc đối diện và bán kính đường tròn ngoại tiếp tam giác.
\end{itemize}
\end{dang}
\viduminhhoa
\begin{vd}%[Phạm Tuấn]%[0H2B3-1] 
	Cho tam giác $ABC$ có $\widehat{A}= 120^\circ$ và $BC= 10 \mathrm{~cm}$. Tính bán kính đường tròn ngoại tiếp tam giác $ABC$.
	\loigiai{
		Áp dụng định lí sin ta có $R = \dfrac{BC}{2\sin A} = \dfrac{10}{2\sin 120^\circ} = \dfrac{10\sqrt{3}}{3} \mathrm{~cm}$. 
	}
\end{vd}


\begin{vd}%[Phạm Tuấn]%[0H2B3-1] 
	\immini{
		Cho tam giác $ABC$ có $\widehat{A}= 40^\circ$, $\widehat{B}=55^\circ$ và $AB = 100$. Tính độ dài cạnh $BC$ (làm tròn kết quả đến hàng phần mười).
	}
	{
		\begin{tikzpicture}[scale=1, font=\footnotesize, line join=round, line cap=round,>=stealth]
			\path
			(0,0) coordinate (A)
			(4,0) coordinate (B)
			;
			\coordinate (A') at ($(A)+(40:3)$); 
			\coordinate (B') at ($(B)+(-55:3)$); 
			\coordinate (C) at (intersection of A--A' and B--B');
			\draw (A)--(B)--(C)--(A);
			
			\tkzMarkAngle[arc=ll,size=0.5,mark=0](B,A,C)
			\tkzLabelAngle[pos=1.1](B,A,C) {$40^\circ$}
			\tkzMarkAngle[arc=l,size=0.5,mark=0](C,B,A)
			\tkzLabelAngle[pos=0.8](C,B,A) {$55^\circ$}
			\foreach \x/\g in {A/-120,B/-60,C/90} 
			\fill[black] (\x) circle (1pt)+(\g:3mm) node {$\x$};
		\end{tikzpicture}
	}
	\loigiai{
		Ta có $\widehat{C}= 180^\circ -  \widehat{A} -\widehat{B} = 180^\circ - 40^\circ-55^\circ  =85^\circ$. \\
		Áp dụng định lí sin ta có 
		\[
		\dfrac{AB}{\sin C} = \dfrac{BC}{\sin A} \Rightarrow BC = \dfrac{AB\sin A}{\sin C}  = \dfrac{100\sin 40^{\circ}}{\sin 85^{\circ}} \approx 64{,}5. 
		\]
	}
\end{vd}


\begin{vd}%[Phạm Tuấn]%[0H2B3-1] 
	Cho tam giác $ABC$ có $\dfrac{AB}{2} = \dfrac{BC}{3}$ và $\widehat{A} = 45^\circ$. Tính các góc $B$, $C$  của tam giác đó (làm tròn kết quả đến hàng phần mười).
	\loigiai{
		Áp dụng định lí sin ta có
		\[
		\dfrac{AB}{\sin C}  = \dfrac{BC}{\sin A} \Rightarrow \sin C = \dfrac{AB\sin A}{BC} = \dfrac{2\sin 45^\circ}{3} \Rightarrow \widehat{C} \approx 
		28{,}1^\circ.\]
		Khi đó $\widehat{B}= 180^\circ - \widehat{A}- \widehat{C}= 180^\circ - 45^\circ -28{,}1^\circ=106{,}9^\circ $.
	}
\end{vd}


\begin{vd}%[Phạm Tuấn]%[0H2B3-1] 
	Cho tam giác $ABC$ có $\widehat{A}= 30^\circ$, $\widehat{B}=50^\circ$ và bán kính đường tròn ngoại tiếp bằng $10 \mathrm{~cm}$. Tính độ dài các cạnh của tam giác $ABC$ (làm tròn đến hàng phần mười).
	\loigiai{
		Ta có $\widehat{C}= 180^\circ - \widehat{A} - \widehat{B} =  180^\circ  -30^\circ - 50^\circ= 100^\circ$.  \\
		Áp dụng định lí sin
		\begin{align*}
			&  AB = 2R\sin C = 2\cdot 10 \cdot \sin 100^\circ \approx 19{,}7 \mathrm{~cm}; \\
			& BC = 2R\sin A = 2\cdot 10 \cdot \sin 30^\circ  = 10 \mathrm{~cm};  \\
			&AC = 2R\sin B = 2\cdot 10 \cdot \sin 50^\circ \approx 15{,}3 \mathrm{~cm}.
		\end{align*}
	}
\end{vd}


\begin{vd}%[Phạm Tuấn]%[0H2B3-1] 
	Cho tam giác $ABC$. Chứng minh rằng  $\sin^2 A = \sin B \sin C$ khi và chỉ khi $a^2 = bc$. 
	\loigiai{
		Theo định lí sin ta có  $\sin A = \dfrac{a}{2R}$;   $\sin B = \dfrac{b}{2R} $; $\sin C = \dfrac{c}{2R}$. \\
		Do đó 
		\[
		\sin^2 A = \sin B \sin C  \Leftrightarrow \left (\dfrac{a}{2R}\right )^2 =  \dfrac{b}{2R} \cdot  \dfrac{c}{2R} \Leftrightarrow a^2 = bc. 
		\]
	}
\end{vd}

\begin{vd}%[Phạm Tuấn]%[0H2B3-1] 
	Cho tam giác $ABC$.  Biết $AB= 5 \mathrm{~cm}$,  $BC= 6 \mathrm{~cm}$ và $2\sin A = \sin B + \sin C$. Tính độ dài cạnh $AC$. 
	\loigiai{
		Theo định lí sin ta có  $\sin A = \dfrac{BC}{2R}$;   $\sin B = \dfrac{AC}{2R} $; $\sin C = \dfrac{AB}{2R}$. \\
		Do đó 
		\[
		2\sin  A = \sin B +  \sin C  \Leftrightarrow \dfrac{2BC}{2R} =  \dfrac{AC}{2R} +  \dfrac{AB}{2R} \Leftrightarrow 2BC =AC+AB. 
		\]
		Suy ra $AC= 2BC-AB=  12 -5 =7    \mathrm{~cm}$. 
	}
\end{vd}

\baitaptl

\begin{bt}%[Phạm Tuấn]%[0H2B3-1] 
	Cho tam giác $ABC$ có $\widehat{B}= 70^\circ$ và $AC= 15 \mathrm{~cm}$. Tính bán kính đường tròn ngoại tiếp tam giác $ABC$ (làm tròn kết quả đến hàng phần mười).
	\loigiai{
		Áp dụng định lí sin ta có $R = \dfrac{AC}{2\sin B} = \dfrac{15}{2\sin 70^\circ} \approx 8 \mathrm{~cm}$. 
	}
\end{bt}


\begin{bt}%[Phạm Tuấn]%[0H2B3-1] 
	Cho tam giác $ABC$ có $\widehat{B}= 30^\circ$, $\widehat{C}=65^\circ$ và $BC = 50$. Tính độ dài cạnh $AB$ (làm tròn kết quả đến hàng phần mười).
	\loigiai{
		Ta có $\widehat{A}= 180^\circ -  \widehat{B} -\widehat{C} = 180^\circ - 30^\circ-65^\circ  =75^\circ$. \\
		Áp dụng định lí sin ta có 
		\[
		\dfrac{AB}{\sin C} = \dfrac{BC}{\sin A} \Rightarrow AB = \dfrac{BC\sin C}{\sin A}  = \dfrac{50\sin 65^{\circ}}{\sin 75^{\circ}} \approx 46{,}9. 
		\]
	}
\end{bt}

\begin{bt}%[Phạm Tuấn]%[0H2B3-1] 
	Cho tam giác $ABC$ có $\dfrac{BC}{3} = \dfrac{AC}{5}$ và $\widehat{A} = 30^\circ$. Tính các góc $B$, $C$  của tam giác đó (làm tròn kết quả đến hàng phần mười).
	\loigiai{
		Áp dụng định lí sin ta có
		\[
		\dfrac{AC}{\sin B}  = \dfrac{BC}{\sin A} \Rightarrow \sin B = \dfrac{AC\sin A}{BC} = \dfrac{5\sin 30^\circ}{3} \Rightarrow B \approx 
		56{,}4^\circ.\]
		Khi đó $\widehat{C}= 180^\circ - \widehat{A}- \widehat{B}= 180^\circ - 30^\circ -56{,}4^\circ=93{,}6^\circ $.
	}
\end{bt}

\begin{bt}%[Phạm Tuấn]%[0H2B3-2] 
	Cho tam giác $ABC$ thỏa mãn $a \sin B = c \sin A$.  Chứng minh rằng tam giác $ABC$ cân.
	\loigiai{
		Từ giả thiết suy ra $\dfrac{a}{\sin A} = \dfrac{c}{\sin B}$.  \quad (1)\\
		Áp dụng định lí sin ta có  $\dfrac{a}{\sin A} = \dfrac{b}{\sin B}$. \quad (2) \\
		Từ (1) và (2) suy ra $\dfrac{c}{\sin B} = \dfrac{b}{\sin B} \Rightarrow b=c$. \\
		Vậy tam giác $ABC$ cân. 
	}
\end{bt}

\begin{bt}%[Phạm Tuấn]%[0H2B3-2] 
	Cho tam giác $ABC$ thỏa mãn $\sin^2A = \sin^2B + \sin^2 C$.  Chứng minh rằng tam giác $ABC$ vuông. 
	\loigiai{
		Từ định lí sin suy ra $\sin A = \dfrac{a}{2R}$, $\sin B = \dfrac{b}{2R}$, $\sin C = \dfrac{c}{2R}$. \\
		Khi đó 
		\[
		\sin^2A = \sin^2B + \sin^2 C \Leftrightarrow \left (\dfrac{a}{2R}\right )^2 = \left (\dfrac{b}{2R}\right )^2 +\left (\dfrac{c}{2R}\right )^2 \Leftrightarrow a^2=b^2+c^2.
		\]
		Vậy tam giác $ABC$ vuông tại $A$. 
	}
\end{bt}

\begin{bt}%[Phạm Tuấn]%[0H2K3-2] 
	\immini{
		Cho tam giác $ABC$. Gọi $D$ là điểm thuộc miền trong tam giác $ABC$ sao cho $\widehat{BAD} = \widehat{CBD} = \widehat{ACD} = \varphi$.  Chứng minh rằng 
		\[
		\sin^3 \varphi = \sin (A- \varphi)  \sin (B- \varphi)   \sin (C- \varphi) . 
		\]
	}
	{
		\begin{tikzpicture}[scale=1, font=\footnotesize, line join=round, line cap=round,>=stealth]
			\path
			(2,4) coordinate (A)
			(1,1) coordinate (B)
			(5,1) coordinate (C)
			(2.4,1.76) coordinate (D)
			;
			\draw (A)--(B)--(C)--(A)--(D)  (B)--(D)--(C);
			
			\tkzMarkAngle[arc=l, size=0.7cm,mark=0](B,A,D)
			\tkzLabelAngle[pos=1.1](B,A,D) {$\varphi$}
			\tkzMarkAngle[arc=l, size=0.7cm,mark=0](C,B,D)
			\tkzLabelAngle[pos=1.1](C,B,D) {$\varphi$}
			\tkzMarkAngle[arc=l, size=0.7cm,mark=0](A,C,D)
			\tkzLabelAngle[pos=1.1](A,C,D) {$\varphi$}
			\foreach \x/\g in {A/90,B/-120,C/-60,D/50} 
			\fill[black] (\x) circle (1pt)+(\g:3mm) node {$\x$};
		\end{tikzpicture}
	}
	\loigiai{
		Áp dụng định lí sin cho các tam giác $ABD$, $BCD$ và $ACD$ ta nhận được
		\[
		\heva{&\dfrac{BD}{\sin \varphi} = \dfrac{AD}{\sin (B- \varphi)}\\&  \dfrac{CD}{\sin \varphi} = \dfrac{BD}{\sin (C- \varphi)}\\&\dfrac{AD}{\sin \varphi} = \dfrac{CD}{\sin (A- \varphi)}} \Rightarrow \dfrac{BD}{\sin \varphi} \cdot \dfrac{CD}{\sin \varphi}  \cdot \dfrac{AD}{\sin \varphi} = \dfrac{AD}{\sin (B- \varphi)} \cdot \dfrac{BD}{\sin (C- \varphi)} \cdot \dfrac{CD}{\sin (A- \varphi)}.
		\]
		Rút gọn,  ta suy ra $\sin^3 \varphi = \sin (A- \varphi)  \sin (B- \varphi)   \sin (C- \varphi)$. 
	}
\end{bt}
\begin{dang}{Giải tam giác và ứng dụng}
	Giải tam giác là bài toán tìm độ dài tất cả các cạnh và độ lớn tất cả các góc của tam giác.
\end{dang}
\viduminhhoa
\begin{vd}%[Phạm Tuấn]%[0H2B3-1] 
	\immini{
		Cho tam giác $A B C$ có $BC=40\mathrm{~cm}$, $\widehat{B}=30^{\circ}, \widehat{C}=45^{\circ}$. Tính góc $\widehat{A}$ và độ dài các cạnh $A B$, $A C$  của tam giác đó (làm tròn kết quả đến hàng phần mười).
	}
	{
		\begin{tikzpicture}[scale=1, font=\footnotesize, line join=round, line cap=round,>=stealth]
			\path
			(0,0) coordinate (B)
			(4,0) coordinate (C)
			;
			\coordinate (B') at ($(B)+(30:3)$) ; 
			\coordinate (C') at ($(C)+(-45:3)$) ; 
			\coordinate (A) at (intersection of B--B' and C--C');
			\draw (A)--(B)--(C)--(A) ;
			
			\tkzMarkAngle[arc=ll,size=0.6,mark=0](C,B,A)
			\tkzLabelAngle[pos=.9](C,B,A) {$30^\circ$}
			\tkzMarkAngle[arc=l, size=0.6,mark=0](A,C,B)
			\tkzLabelAngle[pos=0.9](A,C,B) {$45^\circ$}
			\foreach \x/\g in {A/90,B/-120,C/-60} 
			\fill[black] (\x) circle (1pt)+(\g:3mm) node {$\x$};
		\end{tikzpicture}
	}
	\loigiai{
		Ta  có $\widehat{A} = 180^\circ - (\widehat{B}+\widehat{C}) = 180^\circ - (30^{\circ}+45^{\circ}) = 105^{\circ}$. \\
		Áp dụng định lí sin ta có 
		\begin{align*}
			&\dfrac{AB}{\sin C} = \dfrac{BC}{\sin A} \Rightarrow AB = \dfrac{BC\sin C}{\sin A}  = \dfrac{40\sin 45^{\circ}}{105^{\circ}} \approx 29{,}3 \mathrm{~(cm)}; \\
			&\dfrac{AC}{\sin B} = \dfrac{BC}{\sin A} \Rightarrow AC = \dfrac{BC\sin B}{\sin A}  = \dfrac{40\sin 30^{\circ}}{105^{\circ}} \approx 20{,}7 \mathrm{~(cm)}. 
		\end{align*}
	}
\end{vd}

\begin{vd}%[Phạm Tuấn]%[0H2B3-1] 
	Cho tam giác $ABC$ có $AB=25$, $AC=20$, $\widehat{A}=120^{\circ}$. Tính cạnh $BC$ và các góc $B$, $C$  của tam giác đó.  
	\loigiai{
		Áp dụng định lí côsin ta có 
		\[ BC^2=AB^2+AC^2-2AB \cdot AC \cos A =25^2+20^2-2 \cdot 25 \cdot 20 \cos 120^\circ  =  1525 \Rightarrow BC=5\sqrt{61} \approx 39.\]
		Áp dụng định lí sin ta có 
		\begin{align*}
			\dfrac{AC}{\sin B} = \dfrac{BC}{\sin A} \Rightarrow \sin B = \dfrac{AC\sin A}{BC}  = \dfrac{20 \sin 120^{\circ}}{5\sqrt{61}} \Rightarrow B \approx 26{,}3^\circ.  
		\end{align*}
		Khi đó $\widehat{C}= 180^\circ - \widehat{A} - \widehat{B} =  180^\circ  -120^\circ - 26{,}3^\circ= 33{,}7^\circ$.
	}
\end{vd}

\begin{vd}%[Phạm Tuấn]%[0H2B3-4]
	\immini{
		Để đo chiều rộng $AB$ của một khúc sông, người ta chọn điểm $C$.  Sau đó,  đo khoảng cách $BC$, các góc $B$ và $C$. Biết rằng $BC = 200$ m, $\widehat{B} = 107^\circ$,  $\widehat{C} = 28^\circ$. Tìm chiều rộng $AB$ của khúc sông đó (làm tròn đến chữ số thập phân thứ nhất).
	}
	{
		\begin{tikzpicture}[scale=1]
			\coordinate (X) at (1,1);
			\foreach \i in {0,...,6}
			\foreach \j  in {0,...,2}  
			\draw[blue] ($(X)+({\i+0.5},{0.6*(\j)+0.4})$)--($(X)+({\i+0.7},{0.6*(\j)+0.4})$);
			\coordinate [label=above left:$A$](A) at (3,3);
			\coordinate [label=below left:$B$](B) at (3,1);
			\coordinate [label=right:$C$](C) at (5,0.5);
			\draw (X)--(8,1) (8,3)--(1,3) (A)--(B)--(C)--(A);
			\foreach \i in {A,B,C} \draw[fill=black] (\i) circle(1.2pt);
		\end{tikzpicture}
	}
	\loigiai{
		Ta có $\widehat{A}= 180^\circ -  \widehat{B} -\widehat{C} =180^\circ- 107^\circ - 28^\circ=55^\circ$. \\
		Áp dụng định lí sin ta có 
		\[
		\dfrac{AB}{\sin C} = \dfrac{BC}{\sin A} \Rightarrow AB = \dfrac{BC\sin C}{\sin A}  = \dfrac{200\sin 28^{\circ}}{\sin 55^{\circ}} \approx 113{,}6\mathrm{~m}. 
		\]
	}
\end{vd}

\begin{vd}%[Phạm Tuấn]%[0H2B3-4]
	\immini{
		Để đo chiều cao $CH$ của một  tháp truyền hình, người ta chọn hai điểm quan sát $A$, $B$ trên mặt đất (hình vẽ).  Biết $\widehat{CAH} =51^\circ$, $\widehat{CBH} =66^\circ$ và $AB=75 \mathrm{~m}$, tính chiều cao của tháp.
	}
	{
		\begin{tikzpicture}[scale=1, font=\footnotesize, line join=round, line cap=round,>=stealth]
			\path
			(1,0) coordinate (A)
			(2,0) coordinate (B)
			(4,3.5) coordinate (C)
			(4,0) coordinate (H)
			(3.75,0) coordinate (U)
			(4.25,0) coordinate (V)
			;
			\draw[thick] (U)--(C)--(V); 
			\draw (B)--(C)--(A)--(V) ;
			\draw[dashed] (C)--(H) ;  
			\draw  
			($(C)!{1}!(U)$)--($(C)!{0.9}!(V)$)
			--($(C)!{0.8}!(U)$)--($(C)!{0.7}!(V)$)
			--($(C)!{0.6}!(U)$)--($(C)!{0.5}!(V)$)
			--($(C)!{0.4}!(U)$)--($(C)!{0.3}!(V)$)
			--($(C)!{0.2}!(U)$)--($(C)!{0.1}!(V)$)
			;
			
			\tkzMarkAngle[arc=l, size=0.6,mark=0](H,A,C)
			\tkzLabelAngle[pos=1](H,A,C) {$51^{\circ}$}
			\tkzMarkAngle[arc=ll, size=0.6,mark=0](H,B,C)
			\tkzLabelAngle[pos=1](H,B,C) {$66^{\circ}$}
			\foreach \x/\g in {A/-90,B/-90,C/90,H/-90} 
			\fill[black] (\x) circle (1pt)+(\g:3mm) node {$\x$};
		\end{tikzpicture}
	}
	
	\loigiai{
		Ta có $\widehat{ACB}=\widehat{CBH} - \widehat{CAH}=  66^\circ -51^\circ= 15^\circ$. \\
		Áp dụng định lí sin ta có
		\[
		\dfrac{AB}{\sin \widehat{ACB}} = \dfrac{BC}{\sin \widehat{CAH} } \Rightarrow BC = \dfrac{AB\sin \widehat{CAH}}{\sin \widehat{ACB}}  = \dfrac{75\sin 51^\circ}{\sin 15^\circ}.
		\]
		Suy ra $CH =BC\sin \widehat{CBH} = \dfrac{75 \sin 51^\circ \sin 66^\circ}{\sin 15^\circ} \approx 205{,}7\mathrm{~m}.$
	}
\end{vd}

\begin{vd}%[Phạm Tuấn]%[0H2B3-4]
	\immini{
		Trên ngọn đồi có một cái tháp cao $120 \mathrm{~m}$. Đỉnh tháp $B$ và chân tháp $C$ nhìn điểm $A$ ở chân đồi dưới các góc tương ứng bằng $35^{\circ}$ và $60^{\circ}$ so với phương thẳng đứng. Xác định chiều cao $H A$ của ngọn đồi. (Làm tròn đến phần mười)
	}
	{
		\begin{tikzpicture}[scale=1, font=\footnotesize, line join=round, line cap=round, >=stealth]
			\path 
			(0,0) coordinate (X)
			(1.1,2) coordinate (Y)
			(2,2) coordinate (Z)
			(5,0) coordinate (A)
			(1.5,5) coordinate (B)
			(1.5,2) coordinate (C)
			(1.5,0) coordinate (K)
			(1.3,2) coordinate (U)
			(1.8,2) coordinate (V) 
			(5,2) coordinate (H);
			\draw  
			($(B)!{1}!(U)$)--($(B)!{0.9}!(V)$)
			--($(B)!{0.8}!(U)$)--($(B)!{0.7}!(V)$)
			--($(B)!{0.6}!(U)$)--($(B)!{0.5}!(V)$)
			--($(B)!{0.4}!(U)$)--($(B)!{0.3}!(V)$)
			--($(B)!{0.2}!(U)$)--($(B)!{0.1}!(V)$)
			;
			\draw[thick]  (X)--(A)--(Z)--(Y)--(X);
			\draw (A)--(B)  (U)--(B)--(V) (Y)--($(Y)!{1.2}!(H)$)  (X)--($(X)!{1.2}!(A)$);
			\draw[->] (A)--(H) ; 
			\draw ($(A)!0.5!(H)$) node[right]{$h$};
			\draw[dashed](B)--(K) (A)--(C);
			
			\tkzMarkAngle[arc=l, size=0.6,mark=0](K,C,A)
			\tkzLabelAngle[pos=0.9](K,C,A) {$60^{\circ}$}
			\tkzMarkAngle[arc=ll, size=0.7,mark=0](K,B,A)
			\tkzLabelAngle[pos=1.1](V,B,A) {$35^{\circ}$}
			\foreach \x/\g in{A/-90,B/90,C/-130,H/90}
			\fill[black](\x) ($(\x)+(\g:3mm)$)node{$\x$}; 
		\end{tikzpicture}
	}
	\loigiai{
		Ta có $\widehat{BAC} = 60^\circ - 35^\circ =25^\circ $;  $\widehat{ACH} = 90^\circ - 60^\circ =30^\circ $\\
		Áp dụng định lí sin ta có
		\[
		\dfrac{AC}{\sin \widehat{ABC}} = \dfrac{BC}{\sin \widehat{BAC}} \Rightarrow AC =  \dfrac{BC\sin \widehat{ABC}}{\sin \widehat{BAC}}
		= \dfrac{120\sin 35^\circ}{\sin 25^\circ}.
		\]
		Suy ra $AH =AC \sin \widehat{ACH} = \dfrac{120 \sin 35^\circ \sin 30^\circ }{\sin 25^\circ} \approx 81{,}4 \mathrm{~m}$.
	}
\end{vd}
\baitaptl

\begin{bt}%[Phạm Tuấn]%[0H2B3-1]
	Cho tam giác $ABC$ có $AB=8$, $BC=10$, $AC=15$.  Tính $\widehat{A} + 2\widehat{C}$ (làm tròn kết quả đến hàng phần mười).
	\loigiai{
		Áp dụng định lí côsin ta có
		\begin{align*}
			&\cos A = \dfrac{AB^2+AC^2-BC^2}{2 \cdot AB \cdot AC} = \dfrac{8^2+15^2-10^2}{2 \cdot 8 \cdot 15 } = \dfrac{63}{80} \Rightarrow \widehat{A} \approx  38{,}04^\circ. \\
			&\cos C = \dfrac{AC^2+BC^2-AB^2}{2 \cdot AC \cdot BC} = \dfrac{15^2+10^2-8^2}{2 \cdot 15 \cdot 10 } = \dfrac{87}{100} \Rightarrow \widehat{C} \approx  29{,}54^\circ.
		\end{align*}
		Suy ra $\widehat{A} + 2\widehat{C} \approx 97{,}1^\circ$. 
	}
\end{bt}

\begin{bt}%[Phạm Tuấn]%[0H2B3-1]
	Cho tam giác $A B C$ có $AB=15\mathrm{~cm}$, $AC=21\mathrm{~cm}$, $\widehat{A}=30^{\circ}$. Tính cạnh $BC$ và các góc $B$, $C$  của tam giác đó (làm tròn kết quả đến hàng phần mười).
	\loigiai{
		Áp dụng định lí côsin ta có 
		\[ BC^2=AB^2+AC^2-2AB \cdot AC \cos A =15^2+21^2-2 \cdot 15 \cdot 21 \cos 30^\circ   \Rightarrow BC \approx 11 \mathrm{~cm}.\]
		Áp dụng định lí sin ta có 
		\begin{align*}
			\dfrac{AC}{\sin B} = \dfrac{BC}{\sin A} \Rightarrow \sin B = \dfrac{AC\sin A}{BC}  = \dfrac{21 \sin 30^{\circ}}{11} \Rightarrow B \approx 72{,}7^\circ.
		\end{align*}
		Khi đó $\widehat{C}= 180^\circ - \widehat{A}- \widehat{B}= 180^\circ - 30^\circ -72{,}7^\circ =77{,}3^\circ $.
	}
\end{bt}

\begin{bt}%[Phạm Tuấn]%[0H2K3-1] 
	\immini{
		Cho tam giác $A B C$ có $AB=15$, $AC=12$, $\widehat{A}=60^{\circ}$. $M$ là điểm thuộc cạnh $AB$  sao cho $AM=2BM$. Tính cạnh $CM$,  góc $\widehat{BCM}$ và bán kính đường tròn ngoại tiếp tam giác  $BCM$ (làm tròn kết quả đến hàng phần mười).
	}
	{
		\begin{tikzpicture}[scale=1, font=\footnotesize, line join=round, line cap=round,>=stealth]
			\path
			(0,0) coordinate (A)
			(4,0) coordinate (B)
			;
			\coordinate (M) at ($(A)!{2/3}!(B)$) ;
			\coordinate (C) at ($(A)+(60:3)$) ; 
			\draw (A)--(B)--(C)--(A) (C)--(M) ;
			
			\tkzMarkAngle[arc=l, size=0.6,mark=0](B,A,C)
			\tkzLabelAngle[pos=1](B,A,C) {$60^{\circ}$}
			\foreach \x/\g in {A/-120,B/-60,C/90,M/-90} 
			\fill[black] (\x) circle (1pt)+(\g:3mm) node {$\x$};
		\end{tikzpicture}
	}
	\loigiai{
		Ta có $AM=2BM \Rightarrow BM = \dfrac{1}{3}AB=5$ và $AM=\dfrac{2}{3}AB=10$. \\
		Áp dụng định lí côsin ta có 
		\begin{align*}
			&CM^2=AM^2+AC^2-2AM \cdot AC \cos A =10^2+12^2-2 \cdot 10 \cdot 12 \cos 60^\circ  =  124 \Rightarrow CM=\sqrt{124} \approx 11{,}1; \\
			&BC^2=AB^2+AC^2-2AB \cdot AC \cos A =15^2+12^2-2 \cdot 15 \cdot 12 \cos 60^\circ  =  189\Rightarrow BC=\sqrt{189}.
		\end{align*}
		Áp dụng định lí côsin ta có 
		\begin{align*}
			&BM^2=CM^2+CB^2-2CM \cdot CB \cos \widehat{BCM}  \\
			\Leftrightarrow~ & \cos \widehat{BCM} = \dfrac{CM^2+CB^2- BM^2}{2CM \cdot CB}  \\
			\Leftrightarrow~ & \cos \widehat{BCM} = \dfrac{124+189- 5^2}{2\sqrt{124} \cdot \sqrt{189}}  \\
			\Rightarrow~& \widehat{BCM} \approx 19{,}8^\circ. 
		\end{align*}
		Áp đụng định lí sin,  ta nhận được bán kính đường tròn ngoại tiếp tam giác $BCM$ là
		\[
		R = \dfrac{BM}{2\sin \widehat{BCM}} \approx 7{,}4. 
		\]
	}
\end{bt}

\begin{bt}%[Phạm Tuấn]%[0H2B3-4]
	\immini{
		Để đo chiều rộng $AB$ của một khúc sông, người ta chọn điểm $C$,  đo khoảng cách $BC$, các góc $B$ và $C$. Biết rằng $BC = 250$ m, $\widehat{B} = 104^\circ$,  $\widehat{C} = 31^\circ$. Tìm chiều rộng $AB$ của khúc sông đó (làm tròn đến chữ số hàng đơn vị).
	}
	{
		\begin{tikzpicture}[scale=1, font=\footnotesize, line join=round, line cap=round,>=stealth]
			\coordinate (X) at (1,1);
			\foreach \i in {0,...,6}
			\foreach \j  in {0,...,2}  
			\draw[blue] ($(X)+({\i+0.5},{0.6*(\j)+0.4})$)--($(X)+({\i+0.7},{0.6*(\j)+0.4})$);
			\coordinate [label=above:$A$](A) at (3,3);
			\coordinate [label=below:$B$](B) at (3,1);
			\coordinate [label=right:$C$](C) at (5,0.5);
			\draw (X)--(8,1) (8,3)--(1,3) (A)--(B)--(C)--(A);
			\foreach \i in {A,B,C} \draw[fill=black] (\i) circle(1.2pt);
		\end{tikzpicture}
	}
	\loigiai{
		Ta có $\widehat{A}= 180^\circ -  \widehat{B} -\widehat{C} =180^\circ- 104^\circ - 31^\circ=45^\circ$. \\
		Áp dụng định lí sin ta có 
		\[
		\dfrac{AB}{\sin C} = \dfrac{BC}{\sin A} \Rightarrow AB = \dfrac{BC\sin C}{\sin A}  = \dfrac{250\sin 31^{\circ}}{\sin 45^{\circ}} \approx 182 \mathrm{~m}. 
		\]
	}
\end{bt}

\begin{bt}%[Phạm Tuấn]%[0H2B3-4]
	\immini{
		Để đo chiều cao $CH$ của một  tháp truyền hình, người ta chọn hai điểm quan sát $A$, $B$ trên mặt đất (hình vẽ).  Biết $\widehat{CAH} =54^\circ$, $\widehat{CBH} =68^\circ$ và $AB=80 \mathrm{~m}$, tính chiều cao của tháp (Làm tròn đến hàng đơn vị).
	}
	{
		\begin{tikzpicture}[scale=1, font=\footnotesize, line join=round, line cap=round,>=stealth]
			\path
			(1,0) coordinate (A)
			(2,0) coordinate (B)
			(4,3.5) coordinate (C)
			(4,0) coordinate (H)
			(3.75,0) coordinate (U)
			(4.25,0) coordinate (V)
			;
			\draw[thick] (U)--(C)--(V); 
			\draw (B)--(C)--(A)--(V) ;
			\draw[dashed] (C)--(H) ;  
			\draw  
			($(C)!{1}!(U)$)--($(C)!{0.9}!(V)$)
			--($(C)!{0.8}!(U)$)--($(C)!{0.7}!(V)$)
			--($(C)!{0.6}!(U)$)--($(C)!{0.5}!(V)$)
			--($(C)!{0.4}!(U)$)--($(C)!{0.3}!(V)$)
			--($(C)!{0.2}!(U)$)--($(C)!{0.1}!(V)$)
			;
			
			\tkzMarkAngle[arc=l, size=0.6,mark=0](H,A,C)
			\tkzLabelAngle[pos=1](H,A,C) {$54^{\circ}$}
			\tkzMarkAngle[arc=ll, size=0.6,mark=0](H,B,C)
			\tkzLabelAngle[pos=1](H,B,C) {$68^{\circ}$}
			\foreach \x/\g in {A/-90,B/-90,C/90,H/-90} 
			\fill[black] (\x) circle (1pt)+(\g:3mm) node {$\x$};
		\end{tikzpicture}
	}
	
	\loigiai{
		Ta có $\widehat{ACB}=\widehat{CBH} - \widehat{CAH}=  68^\circ -54^\circ= 14^\circ$. \\
		Áp dụng định lí sin ta có
		\[
		\dfrac{AB}{\sin \widehat{ACB}} = \dfrac{BC}{\sin \widehat{CAH} } \Rightarrow BC = \dfrac{AB\sin \widehat{CAH}}{\sin \widehat{ACB}}  = \dfrac{80\sin 54^\circ}{\sin 14^\circ}.
		\]
		Suy ra $CH =BC\sin \widehat{CBH} = \dfrac{80 \sin 54^\circ \sin 68^\circ}{\sin 14^\circ} \approx 248 \mathrm{~m}.$
	}
\end{bt}
\begin{dang}{Bài tập tổng hợp}
	
\end{dang}
\viduminhhoa
\begin{vd}%[Chim Khuyên]%[0H2B3-1] 
	Cho tam giác $ABC$ có $\widehat{A}= 60^\circ$ và $AB= 8 \mathrm{~cm}$, $AC= 5 \mathrm{~cm}$. 
	\begin{enumerate}
		\item Tính diện tích của tam giác $ABC$.
		\item Tính độ dài đường cao hạ từ đỉnh $A$ của tam giác $ABC$.
		\item Tính bán kính đường tròn nội tiếp tam giác $ABC$.
	\end{enumerate}
	\loigiai{
		\begin{center}
			\begin{tikzpicture}[scale=1, font=\footnotesize, line join=round, line cap=round,>=stealth]
				\path
				(0,0) coordinate (B)
				(5,0) coordinate (C)
				(3,2) coordinate (A)
				($(B)!(A)!(C)$) coordinate (H)
				;
				\draw (A)--(B)--(C)--(A)--(H);
				\draw pic["$60^{\circ}$", draw=black, angle eccentricity=1.5, angle radius=0.5cm]{angle=B--A--C} ;
				\foreach \x/\g in {A/90,B/-120,C/-60,H/-60} 
				\fill[black] (\x) circle (1pt)+(\g:3mm) node {$\x$};
			\end{tikzpicture}
		\end{center}
		\begin{enumerate}
			\item Áp dụng công thức tính diện tích tam giác ta có\\
			$S_{\triangle ABC}=\dfrac{1}{2}AB\cdot AC \cdot \sin 60^{\circ} = \dfrac{1}{2} \cdot 8 \cdot 5 \cdot \dfrac{\sqrt{3}}{2} = 10\sqrt{3} \mathrm{~cm}^2$. 
			\item Áp dụng định lí cosin trong tam giác $ABC$ ta có \\
			$BC^2= AB^2+AC^2-2 AB\cdot AC \cdot \cos 60^{\circ}= 8^2+5^2-2 \cdot 8\cdot 5 \cdot \dfrac{1}{2} =49 \Rightarrow BC=7$.\\
			$S_{\triangle ABC}=\dfrac{1}{2} AH \cdot BC \Rightarrow AH=\dfrac{2S_{\triangle ABC}}{BC}=\dfrac{2\cdot 10 \sqrt{3}}{7}=\dfrac{20\sqrt{3}}{7} $.
			\item $S_{\triangle  ABC}=pr \Rightarrow r=\dfrac{S_{\triangle  ABC}}{p}=\dfrac{2S_{\triangle  ABC}}{AB+BC+AC}=\dfrac{2 \cdot 10\sqrt{3}}{5+8+7}=\sqrt{3}$.
		\end{enumerate}
	}
\end{vd}
\begin{vd}%[Chim Khuyên]%[0H2B3-1] 
	Cho hình bình hành $ABCD$ có $AB=6, BC=8$ và $\widehat{ABC}=60^{\circ}$. Tính diện tích hình bình hành $ABCD$.
	\loigiai{
		\immini{Ta có \\
			$S_{ABCD}=2S_{\triangle ABC}=2\cdot \dfrac{1}{2} BA \cdot BC \cdot \cos \widehat{ABC} $\\
			$= 6\cdot 8 \cdot \cos 60^{\circ} =24$}{	\begin{tikzpicture}[line cap=round,line join=round, >=stealth,scale=.7]
				\tkzDefPoints{0/0/O,-2/1/A} 
				\coordinate (D) at ($(A)+(6,0)$);
				\tkzDefPointBy[homothety = center O ratio -1](A) \tkzGetPoint{C}   
				\tkzDefPointBy[homothety = center O ratio -1](D) \tkzGetPoint{B}  
				\tkzDrawSegments(A,B B,C C,D D,A) 
				\tkzLabelPoints[left](A,B) 
				\tkzLabelPoints[right](C,D)  
		\end{tikzpicture}}
	}
\end{vd}

\begin{vd}%[Chim Khuyên]%[0H2K3-1] 
	Cho tam giác $ABC$ có $\widehat{A}=120^\circ$, $\widehat{B}=30^\circ$, diện tích tam giác $ABC$ bằng $9\sqrt{3}$. Tính các cạnh của tam giác $ABC$.
	\loigiai
	{
		\immini
		{
			Ta có $\widehat{C}=180^\circ-(\widehat{A}+\widehat{B})=30^\circ$.\\
			Khi đó
			\allowdisplaybreaks
			\begin{eqnarray*}
				&&\heva{& \dfrac{BC}{\sin 120^\circ}=\dfrac{AC}{\sin 30^\circ}=\dfrac{AB}{\sin 30^\circ}\\ & S_{\triangle ABC}=\dfrac{1}{2}\cdot BC\cdot AC\cdot \sin 30^\circ=9\sqrt{3}}\\ &\Leftrightarrow&\heva{& BC=\sqrt{3}AC \\ &BC\cdot AC=36\sqrt{3}\\ &AC=AB} \Leftrightarrow \heva{& BC=6\sqrt{3}  \\ & AC=6\\ &AB=6.}
			\end{eqnarray*}
			Vậy $BC=6\sqrt{3}$, $AC=6$, $AB=6$.
		}
		{
			\begin{tikzpicture}[>=stealth,line join=round,line cap=round,line width=0.6pt,font=\footnotesize,scale=1]
				\coordinate[label=below left:$A$](A) at (0,0);
				\coordinate[label=below right:$B$](B) at (3,0);
				\coordinate [label=above right:$C$](C) at ($(A)!1!120:(B)$); % Quay hướng 60 độ và vị tự tỉ số 1 điểm B tâm A thành điểm B1
				\draw (A)--(B)--(C)--cycle;
				\foreach \x in {C,A,B}\draw[->] pic[draw,blue,angle radius=3mm] {angle = B--A--C};
				\foreach \x in {A,B,C} \fill (\x) circle (1.5pt) ;
				\draw (100:0.7)node[below right]{$120^\circ$};
			\end{tikzpicture}
		}
	}
\end{vd}
\begin{vd}%[Chim Khuyên]%[0H2K3-1]
	Cho tam giác $ABC$ có $AB=2$, $AC=2\sqrt{7}$ và $BC=4$.
	\begin{enumerate}
		\item Tính góc $B$ và diện tích tam giác $ABC$.
		\item Tính độ dài đường phân giác trong của góc $B$ của tam giác $ABC$.
	\end{enumerate}
	\loigiai
	{
		\immini
		{
			\begin{enumerate}
				\item Ta có $\cos B=\dfrac{AB^2+BC^2-AC^2}{2\cdot AB \cdot BC}= \dfrac{4+16-28}{2\cdot 2 \cdot 4}=\dfrac{-1}{2} \Rightarrow \widehat{B}= 120^\circ$.\\
				Và $S_{\triangle ABC}=\dfrac{1}{2}AB\cdot BC\cdot \sin 120^\circ=2\sqrt{3}$.
				\item Gọi $D$ là chân đường phân giác trong của góc $B$.\\
				Ta có
				\allowdisplaybreaks
				\begin{eqnarray*}
					&&S_{\triangle ABC} = S_{\triangle ABD} + S_{\triangle BCD}\\ &\Leftrightarrow& 2\sqrt{3}=\dfrac{1}{2} AB \cdot BD \cdot \sin \widehat{ABD} +\dfrac{1}{2} CB \cdot BD \cdot \sin \widehat{CBD}\\
					&\Leftrightarrow& 2\sqrt{3}=\dfrac{1}{2} \cdot 2 \cdot BD \cdot \sin 60^\circ +\dfrac{1}{2}\cdot 4 \cdot BD \cdot \sin 60^\circ\\
					&\Leftrightarrow & 2\sqrt{3} = \dfrac{3\sqrt{3}}{2}BD\Leftrightarrow BD= \dfrac{4}{3}.
				\end{eqnarray*}
			\end{enumerate}
		}
		{
			\begin{tikzpicture}[>=stealth,line join=round,line cap=round,line width=0.6pt,font=\footnotesize]
				\coordinate[label=below left:$B$](B) at (0,0);
				\coordinate[label=below right:$C$](C) at (4,0);
				\coordinate [label=above right:$A$](A) at ($(B)!0.5!120:(C)$); 
				\coordinate [label=above  right:$D$](D) at ($(B)!0.333!60:(C)$);
				\draw (A)--(B)--(C)--cycle (B)--(D);
				\foreach \x in {A,B,C,D} \fill (\x) circle (1.5pt) ;
				\draw pic[draw,blue,angle radius=4mm] {angle = D--B--A} ($($(B)!4mm!(A)$)!.5!($(B)!4mm!(D)$)$)node[rotate=180]{\scriptsize $|$}; 
				\draw pic[draw,blue,angle radius=4mm] {angle = C--B--D} ($($(B)!4mm!(C)$)!.5!($(B)!4mm!(D)$)$)node[rotate=120]{\scriptsize $|$}; 
				
			\end{tikzpicture}
		}
	}
\end{vd}

\baitaptl

\begin{bt}%[Chim Khuyên]%[0H2B2-8]
	Cho tam giác với ba cạnh $ a=13,b=14,c=15$. Tính diện tích của tam giác và độ dài đường cao $ h_c$.
	\loigiai{
		Ta có $ S=\sqrt{p(p-a)(p-b)(p-c)}=84$
		Lại có 
		$ S=\dfrac{1}{2}h_c \cdot 15 \Rightarrow h_c=11\dfrac{1}{5}. $	
	}
	
\end{bt}

\begin{bt}%[Chim Khuyên]%[0H2K3-1]
	Cho tam giác $ABC$ có $AB=10$, $BC=6$ và góc $\widehat{B}=120^\circ$.
	\begin{enumerate}
		\item Tính $AC$ và diện tích tam giác $ABC$.
		\item Tính đường cao $AH$ và bán kính đường tròn nội tiếp tam giác $ABC$.
		\item Tính độ dài đường phân giác trong $BD$ của tam giác $ABC$.
	\end{enumerate}
	
	\loigiai{
		\begin{enumerate}
			\item Ta có $AC= \sqrt{AB^2+BC^2-2AB\cdot BC \cdot \cos B}=14$
			và $S_{\triangle ABC}=\dfrac{1}{2}\cdot AB \cdot BC \cdot \sin B=15\sqrt{3}$.
			\item Ta có 
			\immini
			{
				$AH=\dfrac{2S_{\triangle ABC}}{BC}=\dfrac{2\cdot 15\sqrt{3}}{14}$\\ và $r=\dfrac{S_{\triangle ABC}}{p}=\dfrac{15\sqrt{3}}{15}=\sqrt{3}$ với $p=\dfrac{6+10+14}{2}=15$.
			}
			{
				\begin{tikzpicture}[>=stealth,line join=round,line cap=round,line width=0.6pt,font=\footnotesize]
					\coordinate[label=below left:$B$](B) at (0,0);
					\coordinate[label=below right:$C$](C) at (3,0);
					\coordinate [label=above right:$A$](A) at ($(B)!1.2!120:(C)$); 
					\coordinate [label=above  right:$D$](D) at ($(B)!0.545!60:(C)$);
					\draw (A)--(B)--(C)--cycle (B)--(D);
					\foreach \x in {A,B,C,D} \fill (\x) circle (1.5pt) ;
					\draw pic[draw,blue,angle radius=4mm] {angle = D--B--A} ($($(B)!4mm!(A)$)!.5!($(B)!4mm!(D)$)$)node[rotate=180]{\scriptsize $|$}; 
					\draw pic[draw,blue,angle radius=4mm] {angle = C--B--D} ($($(B)!4mm!(C)$)!.5!($(B)!4mm!(D)$)$)node[rotate=120]{\scriptsize $|$}; 
					
				\end{tikzpicture}
			}
			\item  Ta có
			\allowdisplaybreaks
			\begin{eqnarray*}
				&&S_{\triangle ABC} = S_{\triangle ABD} + S_{\triangle BCD}\\ &\Leftrightarrow& 15\sqrt{3}=\dfrac{1}{2} AB \cdot BD \cdot \sin \widehat{ABD} +\dfrac{1}{2} CB \cdot BD \cdot \sin \widehat{CBD}\\
				&\Leftrightarrow& 15\sqrt{3}=\dfrac{1}{2}\cdot 10 \cdot BD \cdot \sin 60^\circ +\dfrac{1}{2} \cdot 6 \cdot BD \cdot \sin 60^\circ\\
				&\Leftrightarrow&15\sqrt{3}=4\sqrt{3}\cdot BD\Leftrightarrow BD= \dfrac{15}{4}.
			\end{eqnarray*}
	\end{enumerate}	}
\end{bt}

\begin{bt}%[ Chim Khuyên]%[0H2K3-1]
	Cho tam giác $ABC$ có $AB=2$, $AC=3$ và $\widehat{BAC}=120^\circ$. Tính độ dài $BC$, diện tích tam giác $ABC$, độ dài đường phân giác trong $AD$ của tam giác $ABC$.
	\loigiai
	{
		\immini
		{
			Ta có 
			\begin{itemize}
				\item[$\bullet$] $BC= \sqrt{AB^2+AC^2-2AB\cdot AC \cdot \cos A}=\sqrt{19}$\\ và $S_{\triangle ABC}=\dfrac{1}{2}AB\cdot AC \sin A=\dfrac{3\sqrt{3}}{2}$.
				\item  [$\bullet$] Và 
				\allowdisplaybreaks
				\begin{eqnarray*}
					&&S_{\triangle ABC} = S_{\triangle BAD} + S_{\triangle DAC}\\ &\Leftrightarrow& \dfrac{ 3\sqrt{3}}{2}=\dfrac{1}{2} AB \cdot AD \cdot \sin \widehat{BAD} +\dfrac{1}{2} AC \cdot AD \cdot \sin \widehat{DAC}\\
					&\Leftrightarrow& \dfrac{ 3\sqrt{3}}{2}=\dfrac{1}{2} \cdot 2 \cdot AD \cdot \sin 60^\circ +\dfrac{1}{2}\cdot 3 \cdot AD \cdot \sin 60^\circ\\
					&\Leftrightarrow & \dfrac{ 3\sqrt{3}}{2} = \dfrac{5\sqrt{3}}{4}AD\Leftrightarrow AD= \dfrac{6}{5}.
				\end{eqnarray*}
			\end{itemize}
			
		}
		{
			\begin{tikzpicture}[>=stealth,line join=round,line cap=round,line width=0.6pt,font=\footnotesize,scale=1.3]
				\coordinate[label=below left:$A$](A) at (0,0);
				\coordinate[label=below right:$C$](C) at (3,0);
				\coordinate [label=above right:$B$](B) at ($(A)!0.5!120:(C)$); 
				\coordinate [label=above  right:$D$](D) at ($(A)!0.333!60:(C)$);
				\draw (A)--(B)--(C)--cycle (A)--(D);
				\foreach \x in {A,B,C,D} \fill (\x) circle (1.5pt) ;
				\draw pic[draw,blue,angle radius=4mm] {angle = D--A--B} ($($(A)!3mm!(B)$)!.5!($(A)!3mm!(D)$)$)node[rotate=180]{\scriptsize $|$}; 
				\draw pic[draw,blue,angle radius=4mm] {angle = C--A--D} ($($(A)!3mm!(C)$)!.5!($(A)!3mm!(D)$)$)node[rotate=120]{\scriptsize $|$}; 
				
			\end{tikzpicture}
		}
	}
\end{bt}

\begin{bt}%[Chim Khuyên]%[0H2K3-2]
	Cho tam giác $ABC$ có $AB=c$, $BC=a$, $AC=b$. Gọi $h_a$, $h_b$, $h_c$ lần lượt là các đường cao tương ứng xuất phát từ các đỉnh $A$, $B$, $C$ và $r$ là bán kính đường tròn nội tiếp tam giác $ABC$. Chứng minh $\dfrac{1}{h_a}+\dfrac{1}{h_b}+\dfrac{1}{h_c}=\dfrac{1}{r}$.
	\loigiai{
		Ta có $S=\dfrac{1}{2}ah_a=\dfrac{1}{2}bh_b=\dfrac{1}{2}ch_c \Rightarrow \dfrac{1}{h_a}=\dfrac{a}{2S},\dfrac{1}{h_b}=\dfrac{b}{2S}, \dfrac{1}{h_c}=\dfrac{c}{2S}$ và $S=pr \Rightarrow \dfrac{1}{r}=\dfrac{p}{S}$.
		{\allowdisplaybreaks
			\begin{eqnarray*}
				VT=\dfrac{1}{h_a}+\dfrac{1}{h_b}+\dfrac{1}{h_c}&=&\dfrac{a}{2S}+\dfrac{b}{2S}+\dfrac{c}{2S}\\
				&=&\dfrac{a+b+c}{2S}=\dfrac{2p}{2S}\\
				&=&\dfrac{p}{S}=\dfrac{1}{r}.
		\end{eqnarray*}}
	}
\end{bt}

\begin{bt}%[Chim Khuyên]%[0H2K2-2]
	Cho tam giác $ABC$ không vuông ở $A$, chứng minh $S=\dfrac{1}{4}\left(b^2+c^2-a^2\right)\tan A$.
	\loigiai{
		Ta có 
		{\allowdisplaybreaks
			\begin{eqnarray*}
				S&=&\dfrac{1}{2}bc\sin A\\
				&=&\dfrac{1}{2}bc\cos A \cdot \dfrac{\sin A}{\cos A}\\
				&=&\dfrac{1}{2}bc\cdot \dfrac{b^2+c^2-a^2}{2bc} \cdot \tan A\\
				&=&\dfrac{1}{4}\left(b^2+c^2-a^2\right) \cdot \tan A.	
		\end{eqnarray*}}	
	}
\end{bt}
\subsection{Câu hỏi trắc nghiệm}
	\Opensolutionfile{ansbook}[ans/ansbook-0D3-6-TN]
	\Opensolutionfile{ans}[ans/ans-0D3-6-TN]
	\begin{ex}%[0H2Y3-1]
		Tam giác $ ABC$ có $ AB=5$, $BC=7$, $CA=8$. Số đo góc $ \widehat{A}$ bằng
		\choice
		{$ 90^\circ $}
		{$ 45^\circ $}
		{\True $ 60^\circ $}
		{$ 30^\circ $}
		\loigiai
		{Theo định lí hàm cosine, ta có $ \cos{A}=\dfrac{AB^2+AC^2-BC^2}{2AB\cdot AC}=\dfrac{5^2+8^2-7^2}{2\cdot 5\cdot 8}=\dfrac{1}{2}$.\\
			Do đó, $ \widehat{A}=60^\circ $.}
	\end{ex}
	\begin{ex}%[0H2Y3-1]
		Tam giác $ ABC$ có $ AB=\sqrt{2}$, $AC=\sqrt{3}$ và $ \widehat{C}=45^\circ $. Tính độ dài cạnh $ BC$.
		\choice
		{$ BC=\sqrt{5}$}
		{\True $ BC=\dfrac{\sqrt{6}+\sqrt{2}}{2}$}
		{$ BC=\sqrt{6}$}
		{$ BC=\dfrac{\sqrt{6}-\sqrt{2}}{2}$}
		\loigiai
		{Theo định lí hàm cosine, ta có\\
			$ AB^2=AC^2+BC^2-2\cdot AC\cdot BC\cdot \cos \widehat{C}\Rightarrow {(\sqrt{2} )}^2={(\sqrt{3} )}^2+BC^2-2\cdot \sqrt{3}\cdot BC\cdot \cos 45^\circ $ \\
			$ \Rightarrow BC=\dfrac{\sqrt{6}+\sqrt{2}}{2}$.}
	\end{ex}
	\begin{ex}%[0H2Y3-1]
		Tam giác $ ABC$ có $ AB=2$, $AC=1$ và $ \widehat{A}=60^\circ $. Tính độ dài cạnh $ BC$.
		\choice
		{$ BC=\sqrt{2}$}
		{\True $ BC=\sqrt{3}$}
		{$ BC=1$}
		{$ BC=2$}
		\loigiai
		{Theo định lí hàm cosine, ta có\\
			$ BC^2=AB^2+AC^2-2AB\cdot AC\cdot \cos{A}=2^2+1^2-2\cdot 2\cdot 1\cdot \cos 60^\circ =3$\\
			$\Rightarrow BC=\sqrt{3}$.}
	\end{ex}
	\begin{ex}%[0H2B3-1]
		Tam giác $ ABC$ có $ AB=3$, $AC=6$, $\widehat{BAC}=60^\circ $. Tính độ dài đường cao $ h_a$ của tam giác.
		\choice
		{$ h_a=3\sqrt{3}$}
		{$ h_a=\sqrt{3}$}
		{$ h_a=\dfrac{3}{2}$}
		{\True $ h_a=3$}
		\loigiai
		{Áp dụng định lý hàm số cosine, ta có
			$ BC^2=AB^2+AC^2-2AB\cdot AC\cos A=27\Rightarrow BC=3\sqrt{3}$.\\
			Ta có $ S_{\Delta ABC}=\dfrac{1}{2}\cdot AB\cdot AC\cdot \sin{A}=\dfrac{1}{2}\cdot 3\cdot 6\cdot \sin 60^\circ=\dfrac{9\sqrt{3}}{2}$.\\
			Lại có $ S_{\Delta ABC}=\dfrac{1}{2}\cdot BC\cdot h_a\Rightarrow h_a=\dfrac{2S}{BC}=3$.}
	\end{ex}
	\begin{ex}%[0H2B3-1]
		Tam giác $ ABC$ có $ AB=\dfrac{\sqrt{6}-\sqrt{2}}{2}$, $BC=\sqrt{3}$, $CA=\sqrt{2}$. Gọi $ D$ là chân đường phân giác trong góc $ \widehat{A}$. Khi đó góc $ \widehat{ADB}$ bằng
		\choice
		{$ 90^\circ $}
		{$ 45^\circ $}
		{$ 60^\circ $}
		{\True $ 75^\circ $}
		\loigiai{
			\immini
			{
				Theo định lí hàm cosine, ta có\\
				$ \begin{aligned}
					& \cos \widehat{BAC}=\dfrac{AB^2+AC^2-BC^2}{2\cdot AB\cdot AC}=-\dfrac{1}{2}. \\
					\Rightarrow& \widehat{BAC}=120^\circ \Rightarrow \widehat{BAD}=60^\circ. \\
				\end{aligned}$ \\
				$ \begin{aligned}
					& \cos \widehat{ABC}=\dfrac{AB^2+BC^2-AC^2}{2\cdot AB\cdot BC}=\dfrac{\sqrt{2}}{2}.\\
					\Rightarrow &\widehat{ABC}=45^\circ.
				\end{aligned}$ \\
				Trong $ \Delta ABD$ có $ \widehat{BAD}=60^\circ$, $\widehat{ABD}=45^\circ. \\
				\Rightarrow \widehat{ADB}=75^\circ $.
			}
			{
				\begin{tikzpicture}[scale=1, font=\footnotesize, line join = round, line cap = round,>=stealth]
					\tkzDefPoints{-2/0/B,0/3/A,5/0/C}
					\clip (-2.2,-0.5) rectangle (5.2,3.5);
					\tkzInCenter(A,B,C) \tkzGetPoint{I}
					\tkzInterLL(A,I)(C,B) \tkzGetPoint{D}
					\tkzDrawPoints[fill=black](A,B,C,D)
					\tkzDrawPolygon(A,B,C)
					\tkzDrawSegments(D,A)
					\tkzLabelPoints[above](A)
					\tkzLabelPoints[below](D,B,C)
					\tkzMarkAngles[size=0.5cm,arc=l,mark=||](B,A,D D,A,C)
				\end{tikzpicture}
			}
		}
	\end{ex}
	\begin{ex}%[0H2B3-1]
		Tam giác $ ABC$ có $ AB=4$, $BC=6$, $AC=2\sqrt{7}$. Điểm $ M$ thuộc đoạn $ BC$ sao cho $ MC=2MB$. Tính độ dài cạnh $ AM$.
		\choice
		{$ AM=4\sqrt{2}$}
		{$ AM=3\sqrt{2}$}
		{\True $ AM=2\sqrt{3}$}
		{$ AM=3$}
		\loigiai{
			\immini
			{
				Theo định lí hàm cosine, ta có\\
				$ \cos B=\dfrac{AB^2+BC^2-AC^2}{2\cdot AB\cdot BC}=\dfrac{4^2+6^2-(2\sqrt{7} )^2}{2\cdot 4\cdot 6}=\dfrac{1}{2}$.\\
				Do $ MC=2MB\Rightarrow BM=\dfrac{1}{3}BC=2$.\\
				Theo định lí hàm cosine, ta có\\
				$ \begin{aligned}
					AM^2&=AB^2+BM^2-2\cdot AB\cdot BM\cdot \cos B \\
					& =4^2+2^2-2\cdot 4\cdot 2\cdot \dfrac{1}{2}=12.
				\end{aligned}\\
				\Rightarrow AM=2\sqrt{3}$.
			}
			{
				\begin{tikzpicture}[scale=0.9, font=\footnotesize, line join = round, line cap = round,>=stealth]
					\tkzDefPoints{-2/0/B,0/3/A,5/0/C}
					\coordinate (M) at ($(B)!0. 33!(C)$);
					\tkzDrawPoints[fill=black](A,B,C,M)
					\tkzDrawPolygon(A,B,C)
					\tkzDrawSegments(M,A)
					\tkzLabelPoints[above](A)
					\tkzLabelPoints[below](M,B,C)
				\end{tikzpicture}
		}}
	\end{ex}
	\begin{ex}%[0H2B3-1]
		Cho hình thoi $ ABCD$ cạnh bằng $ 1$ cm và có $ \widehat{BAD}=60^\circ $. Tính độ dài cạnh $ AC$.
		\choice
		{$ AC=2$}
		{\True $ AC=\sqrt{3}$}
		{$ AC=2\sqrt{3}$}
		{$ AC=\sqrt{2}$}
		\loigiai{
			\immini{
				Do $ ABCD$ là hình thoi, có $ \widehat{BAD}=60^\circ \Rightarrow \widehat{ABC}=120^\circ $.\\
				Theo định lí hàm cosine, ta có\\
				$ \begin{aligned}
					AC^2&=AB^2+BC^2-2\cdot AB\cdot BC\cdot \cos \widehat{ABC} \\
					& =1^2+1^2-2\cdot 1\cdot 1\cdot \cos 120^\circ\\
					& =3.\\
					\Rightarrow AC=\sqrt{3}\cdot \\
				\end{aligned}$}
			{
				\begin{tikzpicture}[scale=1, font=\footnotesize, line join = round, line cap = round,>=stealth]
					\tkzDefPoints{-2/0/A,0/3/B}
					\tkzDefPointBy[rotation = center A angle -60](B) \tkzGetPoint{D}
					\coordinate (C) at ($(B)+(D)-(A)$);
					\tkzDrawPoints[fill=black](A,B,C,D)
					\tkzDrawPolygon(A,B,C,D)
					\tkzDrawSegments(C,A D,B)
					\tkzLabelPoints[above](B,C)
					\tkzLabelPoints[below](A,D)
					\tkzMarkAngles[size=0.5cm,arc=l](D,A,B)
			\end{tikzpicture}}
		}
	\end{ex}
	\begin{ex}%[0H2B3-4]
		Khoảng cách từ $A$ đến $B$ không thể đo trực tiếp được vì phải qua một đầm lầy. Người ta xác định được một điểm $C$ mà từ đó có thể nhìn được $A$ và $B$ dưới một góc $78^\circ 24'$. Biết $CA=250$ m, $CB=120$ m. Khoảng cách $AB$ bằng bao nhiêu?
		\choice
		{$266$ m}
		{\True $255$ m}
		{$166$ m}
		{$298$ m}
		\loigiai{
			\immini
			{
				Áp dụng định lí cosine cho $\triangle ABC$, ta có
				$
				\begin{aligned}
					AB^2& =CA^2+CB^2-2CA\cdot CB\cdot \cos C\\
					& =250^2+120^2-2\cdot 250\cdot 120\cdot \cos 78^\circ 24'\\
					& \approx 64835
				\end{aligned}$.\\
				$\Rightarrow AB\approx 255$ (m).
			}
			{
				\begin{tikzpicture}[scale=1, font=\footnotesize, line join=round, line cap=round,>=stealth]
					\tkzInit[xmin=-1,xmax=7,ymin=-1,ymax=3.5]
					\tkzClip
					\tkzDefPoints{0/0/A,6/0/B}
					\tkzDefPointBy[rotation= center A angle 28](B) \tkzGetPoint{a}
					\tkzDefPointBy[rotation= center B angle -74](A) \tkzGetPoint{b}
					\tkzInterLL(A,a)(B,b) \tkzGetPoint{C}
					\tkzDrawPoints[fill=black](A,B,C)
					\tkzDrawSegments(A,B B,C C,A)
					\tkzLabelPoints[above](C) \tkzLabelPoints[above left](A)
					\tkzLabelPoints[above right](B)
					\draw[pattern = north west lines] (2,-0.0) parabola bend (3,-0.75) (4,-0.0)--cycle;
					\tkzLabelSegment[above left](A,C){$250$ m}
					\tkzLabelSegment[above right](C,B){$120$ m}
					\tkzLabelAngle[pos=.8](A,C,B){$78^{\circ}24'$}
					\tkzMarkAngle[size=.5](A,C,B)
				\end{tikzpicture}
			}	
		}
	\end{ex}
	\begin{ex} Cho tam giác $ABC$ có $BC = 2\sqrt{3}$, $AB= \sqrt{6}- \sqrt{2}$, $AC= 2\sqrt{2}$. $AD$ là tia phân giác của góc $\widehat{BAD}$. Tính góc $\widehat{BAD}$.
		\choice
		{\True $60^\circ$}
		{ $90^\circ$}
		{$45^\circ$}
		{$75^\circ$}
		\loigiai{
			Áp dụng hệ quả định lý cosine trong tam giác $ABC$, ta có:
			$\begin{aligned}
				\cos A &=\dfrac{A B^2+A C^2-B C^2}{2\cdot A B\cdot A C}\\ &=\dfrac{(\sqrt 6-\sqrt 2)^2+(2\sqrt 2)^2-(2\sqrt 3)^2}{2\cdot(\sqrt 6-\sqrt 2)\cdot(2\sqrt 2)}\\ &=\dfrac{8-4\sqrt 3+8-12}{2\cdot(\sqrt 6-\sqrt 2)\cdot(2\sqrt 2)}\\ &=\dfrac{4-4\sqrt 3}{-8+8\sqrt 3}=\dfrac{-1}2
			\end{aligned}$
		}	
	\end{ex}
	\begin{ex}
		\immini{Một ô tô muốn đi từ địa điểm H đến địa điểm G, nhưng giữa H và G là một ngọn núi cao nên ô tô phải đi thành 2 đoạn từ H lên K (ô tô leo dốc lên núi) và từ K đến G (ô tô xuống núi). Các đoạn đường tạo thành tam giác $HKG$ với $HK = 15$ km, $KG = 20$ km và $\widehat{HKG}=120^\circ$. Giả sử cứ chạy $1$ km, ô tô tiêu thụ hết $0{,}3$ lít xăng. Giá thành xăng hiện nay là $13050$ đồng một lít xăng. Hỏi ô tô đi từ H đến G hết bao nhiêu tiền xăng?}
		{\begin{tikzpicture}[scale=1, font=\footnotesize, line join=round, line cap=round, >=stealth]
				\coordinate [label =left: $H$ ] (H) at (0,0);
				\coordinate [label =above: $K$ ] (K) at (1.5,3);
				\coordinate [label =right: $G$ ] (G) at (5,0);
				\draw (H)--(K)--(G)--cycle;	
				\draw[fill] (2,-0.2)--(2.1,0.7)--(2.2,1)--(2.4,0.5)--(2.6,1.3)--(3,-0.2)--cycle ;
				\foreach \diem in {H,K,G}\fill (\diem)circle(1.5pt);
			\end{tikzpicture}
		}
		\choice
		{\True $137025$ đồng}
		{ $107025$ đồng}
		{$12278$ đồng}
		{$137000$ đồng}	
		\loigiai{Tổng quãng đường mà ô tô phải đi là$ S = HK + KG = 15 + 20 = 35$ km.\\	
			Ô tô đi hết quãng đường tiêu thụ hết số lít xăng là 	$35 \cdot 0{,}3 = 10{,}5$ lít.\\	
			Ô tô đi từ H đến G hết số tiền xăng là
			$10{,}5 \cdot 13050 = 137025$ đồng.}
	\end{ex}	
	\begin{ex}%[Phạm Tuấn]%[0H2B3-1]
		Cho tam giác $ABC$ có góc $\widehat{B}=45^{\circ}$, $AC= 28$, $BC=25$. Tính số đo góc $A$  của tam giác (làm tròn kết quả đến hàng phần mười).
		\choice
		{$39{,}1^\circ$}
		{$40{,}2^\circ$}
		{\True $39{,}2^\circ$}
		{$40^\circ$}
		\loigiai{
			Áp dụng định lí sin ta có 
			\[ \dfrac{AC}{\sin B}=\dfrac{BC}{\sin A} \Rightarrow \sin A=\dfrac{BC\sin B}{AC}=\dfrac{25\sin 45^\circ}{28} =\dfrac{25\sqrt{2}}{56} \Rightarrow \widehat{A} \approx 39{,}2^\circ.\]
		}
	\end{ex}
	
	
	\begin{ex}%[Phạm Tuấn]%[0H2B3-1]
		Cho tam giác $ABC$ có góc $\widehat{B}=30^{\circ}, \widehat{C}=75^{\circ}$, $AB=20$. Độ dài cạnh $AC$ là
		\choice
		{$20(\sqrt{6}-\sqrt{2})$}
		{\True $10(\sqrt{6}-\sqrt{2})$}
		{$10(\sqrt{6}-1)$}
		{$5(\sqrt{6}+\sqrt{2})$}
		\loigiai{
			Áp dụng định lí sin ta có 
			\[ \dfrac{AC}{\sin B}=\dfrac{AB}{\sin C} \Rightarrow AC=\dfrac{AB\cdot\sin B}{\sin C}=\dfrac{20\cdot \sin 30^\circ}{\sin 75^\circ} =10(\sqrt{6}-\sqrt{2}).\]
		}
	\end{ex}
	
	\begin{ex}%[Phạm Tuấn]%[0H2B3-1] 
		Cho tam giác $ABC$ có $\widehat{B}= 30^\circ$, $\widehat{C}=45^\circ$ và $BC = 30\mathrm{~cm}$. Tính độ dài cạnh $AB$ (làm tròn kết quả đến hàng phần mười).
		\choice
		{$15 (\sqrt{3}+1) \mathrm{~cm}$}
		{$15 (\sqrt{3}-1) \mathrm{~cm}$}
		{$30 (2\sqrt{3}-1) \mathrm{~cm}$}
		{\True $30 (\sqrt{3}-1) \mathrm{~cm}$}
		\loigiai{
			Ta có $\widehat{A}= 180^\circ -  \widehat{B} -\widehat{C} = 180^\circ - 30^\circ-45^\circ  =105^\circ$. \\
			Áp dụng định lí sin ta có 
			\[
			\dfrac{AB}{\sin C} = \dfrac{BC}{\sin A} \Rightarrow AB = \dfrac{BC\sin C}{\sin A}  = \dfrac{30\sin 45^{\circ}}{\sin 105^{\circ}} = 30\sqrt{3}-30 \mathrm{~cm}.
			\]
		}
	\end{ex}
	
	
	\begin{ex}%[Phạm Tuấn]%[0H2B3-1]
		Cho tam giác $ABC$ có $BC= 11$, $\widehat{A} = 30^\circ$. Độ dài  cạnh $AB$ lớn nhất bằng bao nhiêu?
		\choice
		{$11\sqrt{3}$}
		{$\dfrac{22\sqrt{3}}{2}$}
		{\True $22$}
		{$11 (\sqrt{3}+1)$}
		\loigiai{
			Áp dụng định lí sin ta có 
			\begin{align*}
				\dfrac{AB}{\sin C} = \dfrac{BC}{\sin A} \Rightarrow AB = \dfrac{BC\sin C}{\sin A} = \dfrac{11 \sin C}{\sin 30^\circ}   \leq 22.
			\end{align*}
			Đẳng thức xảy ra khi $\widehat{C} = 90^\circ$. \\
			Vậy độ dài cạnh $AB$ lớn nhất bằng $22$.
		}
	\end{ex}
	
	\begin{ex}%[Phạm Tuấn]%[0H2B3-1] 
		Cho tam giác $ABC$ có $\widehat{C}= 30^\circ$ và $AB= 30 \mathrm{~cm}$. Tính bán kính đường tròn ngoại tiếp tam giác $ABC$.
		\choice
		{$30\sqrt{3} \mathrm{~cm}$}
		{$15 \sqrt{3}\mathrm{~cm}$}
		{\True $30 \mathrm{~cm}$}
		{$15 \mathrm{~cm}$}
		\loigiai{
			Áp dụng định lí sin ta có $R = \dfrac{AB}{2\sin C} = \dfrac{30}{2\sin 30^\circ} = 30 \mathrm{~cm}$. 
		}
	\end{ex}
	
	\begin{ex}%[Phạm Tuấn]%[0H2B3-1]
		Cho tam giác $MNK$ có $MN = a$, $MK=3a$, $\widehat{M} = 120^\circ$.  Tính bán kính đường tròn ngoại tiếp $R$ của tam giác $MNK$.
		\choice
		{\True $\dfrac{a\sqrt{39}}{3}$}
		{$\dfrac{a\sqrt{21}}{3}$}
		{$\dfrac{a\sqrt{33}}{3}$}
		{$\dfrac{a\sqrt{42}}{3}$}
		\loigiai{
			Áp dụng định lí côsin ta có
			\[
			NK^2= MN^2+MK^2-2MN \cdot MK \cos M = a^2+9a^2 -2 \cdot a \cdot 3a \cos 120^\circ = 13a^2 \Rightarrow NK=a\sqrt{13}.
			\]
			Áp dụng định lí sin ta có $R= \dfrac{NK}{2\sin M} = \dfrac{a\sqrt{13}}{2\sin 120^\circ} = \dfrac{a\sqrt{39}}{3}$.
		}
	\end{ex}
	
	
	\begin{ex}%[Phạm Tuấn]%[0H2B3-1] 
		\immini{
			Để đo bán kính của một chiếc đĩa cổ chỉ còn lại một phần, các nhà khảo cổ chọn $3$ điểm trên chiếc đĩa (hình vẽ). Biết $\widehat{A}=33^\circ$, $BC=15{,}3 \mathrm{~cm}$, tính bán kính của chiếc đĩa (làm tròn kết quả đến hàng phần mười).
			\choice
			{\True $13{,}8 \mathrm{cm}$}
			{$12{,}6 \mathrm{cm}$}
			{$12{,}9 \mathrm{cm}$}
			{$13{,}1 \mathrm{cm}$}
		}
		{
			\begin{tikzpicture}[scale=1, font=\footnotesize, line join=round, line cap=round,>=stealth]
				\coordinate (O) at (0,0); 
				\coordinate (A) at ($(O)+(131.4:3)$) ; 
				\coordinate (B) at ($(O)+(101.6:3)$) ; 
				\coordinate (C) at ($(O)+(41.7:3)$) ; 
				\coordinate (D) at ($(O)+(140:3)$) ; 
				\coordinate (E) at ($(O)+(90:1)$) ; 
				\coordinate (F) at ($(O)+(18:3)$) ; 
				\draw
				(D) 
				.. controls ++(0:0.5) and ++(150:0.5) .. (E)
				.. controls ++(180:-0.5) and ++(160: 0.5) .. (F)
				;
				\draw (A)--(B)--(C)--(A)  ; 
				\draw  (F) arc (18:140:3);
				\foreach \x/\g in {A/140,B/90,C/60} 
				\fill[black] (\x) circle (1.1pt)+(\g:3mm) node {$\x$};
			\end{tikzpicture}
		}
		\loigiai{
			Áp dụng định lí sin suy ra bán kính của chiếc đĩa là
			\[
			R = \dfrac{BC}{2\sin A} = \dfrac{15,3}{2\sin 33^\circ} \approx 13{,}8 \mathrm{~(cm)}. 
			\]
		}
	\end{ex}
	
	
	\begin{ex}%[Phạm Tuấn]%[0H2B3-2]
		Cho tam giác $ABC$ có $b^2= a^2+c^2+ac$. Khẳng định nào sau đây đúng?
		\choice
		{$\sin^2A = \sin^2B+ \sin^2C + \sin B\sin C$}
		{\True $\sin^2B = \sin^2A+ \sin^2C + \sin A\sin C$}
		{$\widehat{A} = 120^\circ$}
		{$\widehat{A} = 60^\circ$}
		\loigiai{
			Ta có
			\begin{align*}
				&b^2= a^2+c^2+ac \\
				\Leftrightarrow~& (2R\sin B)^2 =  (2R\sin A)^2+(2R\sin C)^2 + (2R\sin A) \cdot (2R\sin C )\\
				\Leftrightarrow~& \sin^2B = \sin^2A+ \sin^2C + \sin A\sin C. 
			\end{align*}
		}
	\end{ex}
	
	\begin{ex}%[Phạm Tuấn]%[0H2B3-2]
		Cho tam giác $ABC$. Khẳng định nào sau đây đúng?
		\choice
		{$\cot A = \dfrac{b^2+c^2-a^2}{2bc}$}
		{$\cot A = \dfrac{b^2+c^2-a^2}{abc}$}
		{$\cot A = \dfrac{R(b^2+c^2-a^2)}{2abc}$}
		{\True $\cot A = \dfrac{R(b^2+c^2-a^2)}{abc}$}
		\loigiai{
			Ta có
			\[
			\cot A = \dfrac{\cos A}{\sin A} = \dfrac{b^2+c^2 - a^2}{2bc \cdot \dfrac{a}{2R}} = \dfrac{R(b^2+c^2-a^2)}{abc}.
			\]
		}
	\end{ex}
	
	\begin{ex}%[Phạm Tuấn]%[0H2B3-1] 
		\immini{
			Cho tam giác $ABCD$ nội tiếp đường tròn tâm $O$.  Biết $\widehat{ACB} = 32^\circ$, $\widehat{ADC} = 75^\circ$ và $BC=8{,}8 \mathrm{~cm}$. Tính bán kính đường tròn đường tròn $(O)$. (Làm tròn kết quả đến hàng phần mười)
			\choice
			{$7{,}8 \mathrm{~cm}$}
			{$7{,}5 \mathrm{~cm}$}
			{$6{,}6 \mathrm{~cm}$}
			{\True $6{,}5 \mathrm{~cm}$}
		}
		{
			\begin{tikzpicture}[scale=1, font=\footnotesize, line join=round, line cap=round,>=stealth]
				\coordinate (O) at (1,1); 
				\coordinate (A) at ($(O)+(133:2.5)$) ; 
				\coordinate (B) at ($(O)+(198:2.5)$) ; 
				\coordinate (C) at ($(O)+(282:2.5)$) ; 
				\coordinate (D) at ($(O)+(0:2.5)$) ; 
				\draw (O) circle[radius=2.5cm];
				\draw (A)--(B)--(C)--(D)--(A)--(C) (B)--(D)  ; 
				\foreach \x/\g in {A/130,B/200,C/-90,D/0,O/-30} 
				\fill[black] (\x) circle (1.1pt)+(\g:3mm) node {$\x$};
			\end{tikzpicture}
		}
		\loigiai{
			Tứ giác $ABCD$ nội tiếp suy ra $\widehat{ADB} = \widehat{ACB} = 32^\circ $ $\Rightarrow \widehat{BCD} = \widehat{ADC} - \widehat{ADB} = 43^\circ$. \\
			Khi đó, bán kính đường tròn tâm $O$ là 
			\[
			R = \dfrac{BC}{2\sin \widehat{BDC}} = \dfrac{8{,}8}{2\sin 43^\circ} \approx 6{,}5 \mathrm{~(cm)}. 
			\]
		}
	\end{ex}
	\begin{ex}%[Phạm Tuấn]%[0H2B3-1]
		Cho tam giác $ABC$ có $AB=12$, $BC=15$, $AC=18$.  Tính $\widehat{A} + 2\widehat{C}$ (làm tròn kết quả đến hàng phần mười).
		\choice
		{$129{,}3^\circ$}
		{$142{,}7^\circ$}
		{$118{,}4^\circ$}
		{\True $138{,}6^\circ$}
		\loigiai{
			Áp dụng định lí côsin ta có
			\begin{align*}
				&\cos A = \dfrac{AB^2+AC^2-BC^2}{2 \cdot AB \cdot AC} = \dfrac{12^2+18^2-15^2}{2 \cdot 12 \cdot 18 } = \dfrac{9}{16} \Rightarrow \widehat{A} \approx  55{,}77^\circ. \\
				&\cos C = \dfrac{AC^2+BC^2-AB^2}{2 \cdot AC \cdot BC} = \dfrac{18^2+15^2-12^2}{2 \cdot 18 \cdot 15 } = \dfrac{3}{4} \Rightarrow \widehat{C} \approx  41{,}4^\circ.
			\end{align*}
			Suy ra $\widehat{A} + 2\widehat{C} \approx 138{,}6^\circ$. 
		}
	\end{ex}
	
	\begin{ex}%[Phạm Tuấn]%[0H2B3-1]
		Cho tam giác $ABC$ có góc $\widehat{A}=60^{\circ}$, $\widehat{B}=45^{\circ}$, $AB=25$. Độ dài cạnh $BC$ gần với giá trị nào nhất dưới đây?
		\choice
		{$22$}
		{\True $22{,}5$}
		{$24{,}5$}
		{$21{,}5$}
		\loigiai{
			Ta có $\widehat{C}=180^\circ - \widehat{A}-\widehat{B} =  180^\circ - 60^{\circ}-45^{\circ} = 75^{\circ}$. \\
			Áp dụng định lí sin ta có 
			\[ \dfrac{BC}{\sin A}=\dfrac{AB}{\sin C} \Rightarrow BC=\dfrac{AB\cdot\sin A}{\sin C}=\dfrac{25\cdot \sin 60^\circ}{\sin 75^\circ} \approx 22{,}4.\]
		}
	\end{ex}
	
	\begin{ex}%[Phạm Tuấn]%[0H2B3-1]
		Cho tam giác $ABC$ có $AB=8$, $AC = 11$, $\widehat{A}=30^\circ$.  Số đo góc $B$ gần với giá trị nào nhất dưới đây?
		\choice
		{$50{,}5^\circ$}
		{$45{,}8^\circ$}
		{$65{,}3^\circ$}
		{\True $55{,}2^\circ$}
		\loigiai{
			Áp dụng định lí côsin ta có 
			\[ BC^2=AB^2+AC^2-2AB \cdot AC \cos A =8^2+11^2-2 \cdot 8 \cdot 11 \cos 30^\circ \Rightarrow BC \approx 6{,}7.\]
			Áp dụng định lí sin ta có 
			\begin{align*}
				\dfrac{AC}{\sin B} = \dfrac{BC}{\sin A} \Rightarrow \sin B = \dfrac{AC\sin A}{BC}  = \dfrac{11 \sin 30^{\circ}}{6{,}7} \Rightarrow \widehat{B} \approx 55{,}2^\circ.  
			\end{align*}
		}
	\end{ex}

	\begin{ex}%[Phạm Tuấn]%[0H2B3-4] 
		\immini{
			Để đo bán kính của một chiếc đĩa cổ chỉ còn lại một phần, các nhà khảo cổ chọn ba điểm trên chiếc đĩa (hình vẽ).  Biết $AB=7{,}1 \mathrm{~cm}$, 
			$BC=16{,}3 \mathrm{~cm}$, $AC=19{,}6 \mathrm{~cm}$,  tính bán kính của chiếc đĩa (làm tròn kết quả đến hàng phần mười).
			\choice
			{$11{,}1 \mathrm{cm}$}
			{$9{,}8 \mathrm{cm}$}
			{\True $10{,}3 \mathrm{cm}$}
			{$10{,}1 \mathrm{cm}$}
		}
		{
			\begin{tikzpicture}[scale=1, font=\footnotesize, line join=round, line cap=round,>=stealth]
				\coordinate (O) at (0,0); 
				\coordinate (A) at ($(O)+(131.4:3)$) ; 
				\coordinate (B) at ($(O)+(101.6:3)$) ; 
				\coordinate (C) at ($(O)+(41.7:3)$) ; 
				\coordinate (D) at ($(O)+(140:3)$) ; 
				\coordinate (E) at ($(O)+(90:1)$) ; 
				\coordinate (F) at ($(O)+(18:3)$) ; 
				\draw
				(D) 
				.. controls ++(0:0.5) and ++(150:0.5) .. (E)
				.. controls ++(180:-0.5) and ++(160: 0.5) .. (F)
				;
				\draw (A)--(B)--(C)--(A)  ; 
				\draw  (F) arc (18:140:3);
				\foreach \x/\g in {A/140,B/90,C/60} 
				\fill[black] (\x) circle (1.1pt)+(\g:3mm) node {$\x$};
			\end{tikzpicture}
		}
		\loigiai{
			Áp dụng định lí côsin ta có
			\begin{align*}
				&BC^2=AB^2+AC^2-2AB \cdot AC \cos A  \\
				\Leftrightarrow~ & \cos A = \dfrac{AB^2+AC^2- BC^2}{2AB \cdot AC}  \\
				\Leftrightarrow~ & \cos A = \dfrac{7{,}1^2+19{,}6^2-16{,}3^2}{2 \cdot 7{,}1 \cdot 19{,}6}\\
				\Rightarrow~& \widehat{A} \approx 52{,}6427^\circ. 
			\end{align*}
			Áp dụng định lí sin suy ra bán kính của chiếc đĩa là
			\[
			R = \dfrac{BC}{2\sin A} = \dfrac{16{,}3}{2\sin 52{,}6427^\circ} \approx 10{,}3 \mathrm{~(cm)}. 
			\]
		}
	\end{ex}
	
	
	\begin{ex}%[Phạm Tuấn]%[0H2B3-4] 
		\immini{
			Để đo khoảng cách từ $A$ đến $B$ ngang qua một đầm lầy, người ta chọn điểm  $C$, sau đó khoảng cách từ $A$ đến $C$ và các góc $A$, $C$. Tính khoảng cách từ $A$ đến $B$ biết $AC=115\mathrm{~m}$, $\widehat{A}=98^\circ$, $\widehat{C}=52^\circ$.
			\choice
			{$188{,}1 \mathrm{~m}$}
			{$190{,}7 \mathrm{~m}$}
			{\True $181{,}2 \mathrm{~m}$}
			{$193{,}6 \mathrm{~m}$}
		}
		{
			\begin{tikzpicture}[scale=1, font=\footnotesize, line join=round, line cap=round,>=stealth]
				\path
				(2,2) coordinate (A)
				(7,2) coordinate (B)
				(1.6,4.5) coordinate (C)
				(2.5,2) coordinate (D)
				(3.5,3.1) coordinate (E)
				(5.5,2.9) coordinate (F)
				(6.4,1.5) coordinate (G)
				(5.2,0.7) coordinate (H)
				(3.5,0.7) coordinate (I)
				(2.6,1.3) coordinate (J)
				;
				\draw[fill=gray!40] 
				(D) 
				.. controls ++(65:0.1) and ++(200: 1) .. (E)
				.. controls ++(200:-0.5) and ++(170: 0.3) .. (F)
				.. controls ++(170:-0.5) and ++(100: 0.3) .. (G)
				.. controls ++(100:-0.3) and ++(30: 0.3) .. (H)
				.. controls ++(30:-0.3) and ++(150: -0.3) .. (I)
				.. controls ++(150:0.3) and ++(130: -0.3) .. (J)
				.. controls ++(130:0.3) and ++(65: -0.1) .. (D)
				;
				\draw[dashed] (A)--(B)--(C) ;
				\draw (A)--(C)   ;
				\foreach \x/\g in {A/-120,B/-60,C/90} 
				\fill[black] (\x) circle (1pt)+(\g:3mm) node {$\x$};
			\end{tikzpicture}
		}
		\loigiai{
			Ta có $\widehat{B} = 180^\circ -  \widehat{A} -\widehat{C}  = 30^\circ $. \\
			Áp dụng định lí sin ta có 
			\begin{align*}
				\dfrac{AB}{\sin C} = \dfrac{AC}{\sin B} \Rightarrow AB = \dfrac{AC\sin C}{\sin B} = \dfrac{115 \sin 52^\circ}{\sin 30^\circ} \approx 181{,}2 \mathrm{~(m).}
			\end{align*}
		}
	\end{ex}
	
	\begin{ex}%[Phạm Tuấn]%[0H2K3-1] 
		\immini{
			Cho tam giác $A B C$ có $AB=8$, $AC=10$, $\widehat{A}=75^{\circ}$. $M$ là điểm thuộc cạnh $BC$  sao cho $CM=2BM$. Bán kính đường tròn ngoại tiếp tam giác  $ABM$  gần nhất với giá trị nào dưới đây?
			\choice
			{$3{,}8$}
			{\True $4{,}1$}
			{$3{,}6$}
			{$3{,}5$}
		}
		{
			\begin{tikzpicture}[scale=1, font=\footnotesize, line join=round, line cap=round,>=stealth]
				\path
				(0,0) coordinate (B)
				(4,0) coordinate (C)
				;
				\coordinate (M) at ($(B)!{1/3}!(C)$) ;
				\coordinate (A) at ($(B)+(66:3.5)$) ; 
				\draw (A)--(B)--(C)--(A) (A)--(M) ;
				\foreach \x/\g in {A/90,B/-120,C/-60,M/-90} 
				\fill[black] (\x) circle (1pt)+(\g:3mm) node {$\x$};
			\end{tikzpicture}
		}
		\loigiai{
			Áp dụng định lí côsin ta có 
			\begin{align*}
				&BC^2=AB^2+AC^2-2AB \cdot AC \cos A =8^2+10^2-2 \cdot 8 \cdot 10 \cos 75^\circ  \Rightarrow BC\approx  11{,}072; \\
				& \cos B = \dfrac{AB^2+BC^2-AC^2}{2AB \cdot BC} \approx  0{,}4888 \Rightarrow \widehat{B} \approx  60{,}4^\circ.
			\end{align*}
			Ta có $CM=2BM \Rightarrow BM = \dfrac{1}{3}BC =  3{,}69$. \\
			Áp dụng định lí côsin ta có 
			\begin{align*}
				AM^2=AB^2+BM^2-2AB \cdot BM \cos B =8^2+3{,}69^2-2 \cdot 8 \cdot 3{,}69 \cdot  0{,}4888 \Rightarrow AM \approx 6{,}983.
			\end{align*}
			Áp đụng định lí sin,  suy ra bán kính đường tròn ngoại tiếp tam giác $ABM$ là
			\[
			R = \dfrac{AM}{2\sin B} = \dfrac{6{,}983}{2\sin 60{,}4^\circ} \approx 4. 
			\]
		}
	\end{ex}
	
	\begin{ex}%[Phạm Tuấn]%[0H2B3-4] 
		\immini{
			Tàu $A$ rời cảng vào lúc 6h00 và chuyển động với vận tốc $30\mathrm{~km/h}$.  Tàu $B$ rời cảng vào lúc 6h30. Vào lúc 9h30 tàu $B$ gặp tàu $A$ tại điểm $C$ (hình vẽ). Giả sử hai tàu chuyển động thẳng và có vận tốc không đổi trong suốt quá trình di chuyển, tính vận tốc tàu $B$ (kết quả làm tròn đến hàng phần mười).
			\choice
			{$42{,}5\mathrm{~km/h}$}
			{\True $44{,}8\mathrm{~km/h}$}
			{$41{,}7\mathrm{~km/h}$}
			{$45{,}4\mathrm{~km/h}$}
		}
		{
			\begin{tikzpicture}[scale=1, font=\footnotesize, line join=round, line cap=round,>=stealth]
				\path
				(0.5,2) coordinate (A)
				(0,0) coordinate (B)
				(4,2) coordinate (C)
				;
				\draw (A)--(B)--(C)--(A) ;
				
				\tkzMarkAngle[arc=ll, size=0.5,mark=0](C,B,A)
				\tkzLabelAngle[pos=0.9](C,B,A) {$55^{\circ}$}
				\tkzMarkAngle[arc=ll, size=0.5,mark=0](B,A,C)
				\tkzLabelAngle[pos=0.9](B,A,C) {$102^{\circ}$}
				\foreach \x/\g in {A/120,B/-90,C/60} 
				\fill[black] (\x) circle (1pt)+(\g:3mm) node {$\x$};
			\end{tikzpicture}
		}
		\loigiai{
			Khoảng cách từ $A$ đến $C$ là $30 \cdot 2{,}5 =75 \mathrm{~km}$. \\
			Áp dụng định lí sin ta có 
			\begin{align*}
				\dfrac{AC}{\sin B} = \dfrac{BC}{\sin A} \Rightarrow BC = \dfrac{AC\sin A}{\sin B}  = \dfrac{75 \sin 102^{\circ}}{\sin 55^{\circ}}.
			\end{align*}
			Suy ra vận tốc của tàu $B$ là  $v= \dfrac{BC}{2} = \dfrac{75 \sin 102^{\circ}}{2\sin 55^{\circ}}  \approx 44{,}8\mathrm{~km/h}$.
		}
	\end{ex}

	\begin{ex}%[0H2Y3-1]
		Chọn công thức đúng trong các đáp án sau
		\choice
		{$S=\dfrac{1}{2}bc\sin B$}
		{\True $S=\dfrac{1}{2}bc\sin A$}
		{$S=\dfrac{1}{2}ab\sin B$}
		{$S=\dfrac{1}{2}ac\sin A$}
		\loigiai{
			Công thức đúng là $S=\dfrac{1}{2}bc\sin A$.
		}
	\end{ex}
	\begin{ex}%[0H2B3-2]
		Cho $\triangle ABC$ với các cạnh $AB=c$, $AC=b$, $BC=a$. Gọi $R$, $r$, $S$ lần lượt là bán kính đường tròn ngoại tiếp, nội tiếp và diện tích của tam giác $ABC$. Trong các phát biểu sau, phát biểu nào \textbf{sai}?
		\choice
		{$S=\dfrac{abc}{4R}$}
		{\True $R=\dfrac{a}{\sin A}$}
		{$S=\dfrac{1}{2}ab\sin C$}
		{$a^2+b^2-c^2=2ab\cos C$}
		\loigiai{Theo định lý Sin trong tam giác, ta có $\dfrac{a}{\sin A}=2R$. Nên mệnh đề \textbf{sai} là ``$R=\dfrac{a}{\sin A}$''.}
	\end{ex}
	
	\begin{ex}%[0H2Y3-1]
		Cho tam giác $ABC$ có $AB=4$, $AC=3$, $\widehat{BAC}=30^\circ$. Khi đó diện tích tam giác $ABC$ bằng
		\choice
		{\True $3$}
		{$4\sqrt{3}$}
		{$6\sqrt{3}$}
		{$6$}
		\loigiai{
			Ta có $S_{ABC}=\dfrac{1}{2}AB\cdot AC\cdot\sin\widehat{BAC}=\dfrac{4\cdot 3\cdot\sin30^\circ}{2}=3$.
		}
	\end{ex}
	\begin{ex}%[0H2B3-1]
		Tìm chu vi tam giác $ABC$, biết $AB=6$ và $2\sin A=3\sin B=4\sin C$.
		\choice
		{\True $26$}
		{$13$}
		{$5\sqrt{26}$}
		{$10\sqrt{6}$}
		\loigiai{
			Từ $2\sin A=3\sin B=4\sin C$ suy ra $2BC=3AC=4AB$.\\
			Mà $AB=6$ nên $AC=8$, $BC=12$. Chu vi tam giác bằng $26$.
		}
	\end{ex}
	\begin{ex}%[0H2B3-1]
		Cho tam giác $ABC$ có $a=13$ m, $b= 14$ m, $c=15$ m. Tính diện tích $S$ của tam giác $ABC$.
		\choice
		{\True $S= 84$ m$^2$}
		{$S= 90$ m$^2$}
		{$S= 76$ m$^2$}
		{$S= 80$ m$^2$}
		\loigiai{
			Ta có $p=\dfrac{a+b+c}{2} =21$ và $S=\sqrt{p(p-a)(p-b)(p-c)}=\sqrt{21(21-13)(21-14)(21-15)} =84$ m$^2$.	
		}
	\end{ex}
	\begin{ex}%[0H2B3-1]
		Cho tam giác $ABC$. Biết $AB=3$, $AC=4$, $BC>5$ và diện tích tam giác $ABC$ bằng $3\sqrt{3}$. Số đo góc $\widehat{BAC}$ bằng
		\choice
		{\True $120^{\circ}$}
		{$60^{\circ}$}
		{$135^{\circ}$}
		{$45^{\circ}$}
		\loigiai{
			Ta có $S_{\triangle ABC}=\dfrac{1}{2}\cdot AB \cdot AC \cdot \sin{\widehat{BAC}}$, suy ra 
			\[\sin{\widehat{BAC}}=\dfrac{2S_{\triangle ABC}}{AB \cdot AC}=\dfrac{2\cdot 3\sqrt{3}}{3 \cdot 4}=\dfrac{\sqrt{3}}{2} \Rightarrow \hoac{&\widehat{BAC}=60^\circ\\&\widehat{BAC}=120^\circ}.\]
			Mặt khác, ta có $\cos{\widehat{BAC}}=\dfrac{AB^2+AC^2-BC^2}{2\cdot AB \cdot AC}<\dfrac{9+16-25}{2 \cdot 3 \cdot 4}=0$.\\
			Vậy $\widehat{BAC}=120^{\circ}$.
		}
	\end{ex}
	\begin{ex}%[0H2K3-1]
		Cho tam giác $ABC$ có $AB=2$, $AC=3$, $BC=4$. Khi đó độ dài đường cao của tam giác $ABC$ kẻ từ $A$ bằng
		\choice
		{$\dfrac{3\sqrt{15}}{2}$}
		{$\dfrac{3\sqrt{15}}{4}$}
		{\True $\dfrac{3\sqrt{15}}{8}$}
		{$3\sqrt{15}$}
		\loigiai{
			Ta có nữa chu vi $p=\dfrac{2+3+4}{2}=\dfrac{9}{2}$.\\
			Suy ra $S_{ABC}=\sqrt{p(p-AB)(p-AC)(p-BC)}=\sqrt{\dfrac{9}{2}\left(\dfrac{9}{2}-2\right)\left(\dfrac{9}{2}-3\right)\left(\dfrac{9}{2}-4\right)}=\dfrac{3\sqrt{15}}{4}$.\\
			Suy ra độ dài đường cao kẻ từ $A$ bằng $\dfrac{2S_{ABC}}{BC}=\dfrac{2\cdot\dfrac{3\sqrt{15}}{4}}{4}=\dfrac{3\sqrt{15}}{8}$.
		}
	\end{ex}
	\begin{ex}%[0H2B3-1]
		Cho tam giác $ABC$ có $AB=9$cm, $AC=12$cm và $BC=15$cm. Khi đó đường trung tuyến $BM$ của tam giác $ABC$ có độ dài là 
		\choice
		{$117$cm}
		{$18{,}82$cm}
		{$10{,}82$cm}
		{\True $7{,}5$cm}
		\loigiai{
			Ta có $m_a^2= \dfrac{2(b^2+c^2)-a^2}{4}=\dfrac{2(12^2+9^2)-15^2}{4}= \dfrac{225}{4} \Rightarrow m_a= 7{,}5$. 		
		}
	\end{ex}
	\begin{ex}%[0H2K3-1]
		Tam giác $ABC$ có các trung tuyến $m_a=10$, $m_b=8$ và $m_c=6$. Tính diện tích $S$ của tam giác $ABC$.
		\choice
		{\True $S=32$}
		{$S=24$}
		{$S=48$}
		{$S=64$}
		\loigiai{
			\immini{
				Gọi $M,N,P$ lần lượt là trung điểm $BC,CA,AB$, $G$ là trọng tâm tam giác $ABC$.\\
				Theo bài ra ta có $AM=10,BN=8,CP=6$.\\
				Lấy $Q$ đối xứng với $G$ qua $M$ thì $BGCQ$ là hình bình hành và ta có $BQ=CG=\dfrac{2CP}{3}=4$, $QG=2GM=\dfrac{2AM}{3}=\dfrac{20}{3}$.\\
				Mà $BG=\dfrac{2BN}{3}=\dfrac{16}{3}$ nên $QG^2=BG^2+BQ^2$ hay $\triangle BGQ$ vuông tại $B$.\\
				Suy ra $S_{BGQ}=\dfrac{BG\cdot BQ}{2}=\dfrac{32}{3}$.\\
				Mà $S_{BGQ}=S_{BGC}=\dfrac{1}{3}S_{ABC}\Rightarrow S_{ABC}=32$.
			}{
				\begin{tikzpicture}[scale=1, font=\footnotesize, line join=round, line cap=round,>=stealth]
					\tkzInit[xmin=-0.5, xmax=5.5, ymin=-0.5, ymax=6.5]
					\tkzClip
					\tkzDefPoints{4/1.5/C,2/0/Q}
					\tkzDefPointBy[rotation=center C angle -90](Q)\tkzGetPoint{g}
					\tkzDefPointBy[homothety=center C ratio 3/4](g)\tkzGetPoint{G}
					\tkzDefPointBy[symmetry=center G](Q)\tkzGetPoint{A}
					\tkzDefMidPoint(G,Q)\tkzGetPoint{M}
					\tkzDefPointBy[symmetry=center M](C)\tkzGetPoint{B}
					\tkzDefMidPoint(A,B)\tkzGetPoint{P}
					\tkzDefMidPoint(A,C)\tkzGetPoint{N}
					\tkzDrawPoints[fill=black](C,G,Q,B,A,P,M,N)
					\tkzDrawSegments(A,B B,C C,A A,Q B,N C,P B,Q C,Q)
					\tkzLabelPoints[above](A)
					\tkzLabelPoints[below](B,Q,C)
					\tkzLabelPoints[right](G,N)
					\tkzLabelPoints[below right](M)
					\tkzLabelPoints[left](P)
				\end{tikzpicture}
			}
		}
	\end{ex}
	\begin{ex}%[0H2K3-4]
		Cho tam giác $ABC$ có chu vi bằng $26$ cm và $\dfrac{\sin A}{2} = \dfrac{\sin B}{6} = \dfrac{\sin C}{5}$. Tính diện tích của tam giác $ABC$. 
		\choice
		{$ 2\sqrt{23} $ (cm$^2$)}
		{$ 6\sqrt{13} $ (cm$^2$)}
		{\True $ 3\sqrt{39} $ (cm$^2$)}
		{$ 5\sqrt{21} $ (cm$^2$)}
		\loigiai{
			Ta có $\dfrac{\sin A}{2} = \dfrac{\sin B}{6} = \dfrac{\sin C}{5} \Leftrightarrow \heva{&\sin B = 3\sin A\\&\sin C = \dfrac{5}{2}\sin A.}$\\
			Mặt khác theo định lí $\sin$ trong tam giác $ABC$ ta có
			$\dfrac{a}{\sin A} = \dfrac{B}{\sin B} = \dfrac{c}{\sin C} \Leftrightarrow \heva{&b = 3a\\&c = \dfrac{5}{2}a.}$\\
			Mà $a + b + c = 26 \Leftrightarrow a + 3a + \dfrac{5}{2}a = 26 \Leftrightarrow \dfrac{13a}{2} = 26 \Leftrightarrow a = 4 \Rightarrow b = 12$ và $c = 10$.\\
			Vậy diện tích tam giác $ABC$ là 
			$$S_{\triangle ABC} = \sqrt{13(13 - 4)(13 - 12)(13 - 10)} = 3\sqrt{39} \,\, (\textrm{cm}^2).$$
		}
	\end{ex}
	\begin{ex}%[0H2B3-1]
		Cho tam giác $ABC$ vuông tại $C$ và $BC=6$, $CA=8$. Tính bán kính đường tròn nội tiếp của tam giác $ABC$.
		\choice
		{\True $2$}
		{$2\sqrt{2}$}
		{$\sqrt{2}$}
		{$4$}
		\loigiai{
			Vì tam giác $ABC$ vuông tại $C$ nên $AB=\sqrt{AC^2+BC^2}=10$ và $S_{ABC}=\dfrac{1}{2}\cdot AC\cdot BC=24$.\\
			Mặt khác $p=\dfrac{6+8+10}{2}=12$, $S_{ABC}=p\cdot r\Rightarrow r=\dfrac{S_{ABC}}{p}=\dfrac{24}{12}=2$.
		}
	\end{ex}
	
	\begin{ex}%[0H2T3-4]
		Từ vị trí $A$ người ta quan sát một cây cao (Hình vẽ). Biết $AH=4$ m, $HB=20$ m, $\widehat{BAC}=45^{\circ}$. Chiều cao của cây gần nhất với giá trị nào sau đây?
		\begin{center}
			\usetikzlibrary{decorations.pathmorphing,shapes}
			\tikzset{
				treetop/.style = {
					decoration={random steps, segment length=0.4mm},
					decorate
				},
				trunk/.style = {
					decoration={random steps, segment length=2mm, amplitude=0.2mm},
					decorate
				}
			}
			\tikzset{
				man/.pic={%
					\fill [rounded corners=1.5] (0,0.4) -- (0,0.4) -- (0.4,0.5) -- (0.4,0.4) --
					(0.325,0.4) -- (0.325,0.7) -- (0.3,0.7) -- (0.3,0) -- (0.225,0) --
					(0.225,0.4) -- (0.175,0.4) -- (0.175,0) -- (0.1,0) -- (0.1,0.7) --
					(0.075,0.7) -- (0.075,0.4) -- cycle;
					\fill (0.2,0.9) circle (0.1);
					\coordinate (-head) at (0.2,1);
					\coordinate (-foot) at (0.2,0);
				}
			}
			\begin{tikzpicture}
				\path 
				(-5,-2.09) coordinate (A)
				(0,1.5) coordinate (C)
				(0.1,-3) coordinate (T)
				(-5,-3) coordinate (H)
				(0,-3) coordinate (B)		
				;
				%\pic[red] at (-6.3,-3) (myman) {man};
				\foreach \w/\f in {0.3/30,0.2/50,0.1/70} {
					\fill [brown!\f!black, trunk] (0,0) ++(-\w/2,0) rectangle +(\w,-3);
				}
				\foreach \n/\f in {1.4/40,1.2/50,1/60,0.8/70,0.6/80,0.4/90} {
					\fill [green!\f!black, treetop] ellipse (\n/1.5 and \n);
				}
				\draw (H)--(T) (A)--(H) (A)--(C) (A)--(B);
				\pic[draw,"$45^{\circ}$", angle eccentricity=1.4,angle radius=0.8cm]{angle=B--A--C};
				\pic[draw,"$ $", angle eccentricity=1.4,angle radius=0.7cm]{angle=B--A--C};
				\draw pic[draw, angle radius=2mm]{right angle=B--H--A};
				\path (H)--(B) node[below,midway,sloped]{$20$ m};
				\foreach \x/\g in {A/180,B/-90,C/90,H/-90} \fill[black] (\x)+(\g:.3) node {$\x$};
			\end{tikzpicture}
		\end{center}
		\choice
		{$14$ m}
		{$15$ m}
		{\True $17$ m}
		{$16$ m}
		\loigiai{\immini{Ta có $AB= \sqrt{AH^2 + BH^2} = \sqrt{4^2+20^2} = 4 \sqrt{26}$.\\
				$\tan \widehat{HAB} = \dfrac{HB}{HA} = \dfrac{20}{4} = 5 \Rightarrow \widehat{HAB} \approx 78{,}69^{\circ}$.\\
				Do $AH \parallel BC$ nên $ \widehat{ABC} = \widehat{HAB} \approx 78{,}69^{\circ}$.\\
				$\widehat{ACB} = 180^{\circ} - 45^{\circ} - \widehat{ABC} \approx 56{,}31^{\circ}$.\\
				Áp dụng định lí hàm số $\sin$ trong tam giác $ABC$ ta có
				$$ \dfrac{BC}{\sin 45^{\circ}} = \dfrac{AB}{\sin 56{,}31^{\circ}} = \dfrac{4 \sqrt{26}}{\sin 56{,}31^{\circ}} \Rightarrow BC \approx  17{,}33.$$	}
			{\begin{tikzpicture}[scale=0.8, font=\footnotesize,line join = round, line cap = round, >=stealth]
					\tkzDefPoints{-5/-2.09/A, 0/1.5/C, 0.1/-3/T, -5/-3/H,0/-3/B}
					\tkzDrawSegments(H,B A,H A,C A,B B,C)
					\tkzMarkAngles[size=0.6,arc=ll](B,A,C)
					\tkzMarkRightAngles(B,H,A)
					\tkzLabelAngles[pos=1.1](B,A,C){$45^{\circ}$}
					\tkzLabelPoints[above](C)
					\tkzLabelPoints[above](A)
					\tkzLabelPoints[below](B)
					\tkzLabelPoints[below](H)
			\end{tikzpicture}}
		}
	\end{ex}
	
	\begin{ex}%[0H2G3-1]
		Một miếng giấy hình tam giác $ABC$ diện tích $S$ có $I$ là trung điểm $BC$ và $O$ là trung điểm của $AI$. Cắt miếng giấy theo một đường thẳng qua $O$, đường thẳng này đi qua $M$, $N$ lần lượt trên các cạnh $AB$, $AC$. Khi đó diện tích miếng giấy chứa điểm $A$ có diện tích thuộc đoạn $\left[mS; nS\right]$. Tính $T = \dfrac{1}{m} + \dfrac{1}{n}$.
		\choice
		{$T = \dfrac{7}{12}$}
		{$T = 12$}
		{\True $T = 7$}
		{$T=\dfrac{12}{7}$}
		\loigiai
		{\immini
			{Ta có $\dfrac{S_{\triangle AMN}}{S_{\triangle ABC}} = \dfrac{AM}{AB}\cdot\dfrac{AN}{AC}$\\
				Dễ thấy $S_{\triangle ABI} = S_{\triangle ACI} = \dfrac{1}{2}\cdot S_{\triangle ABC}$.\\
				Mặt khác   
				\begin{eqnarray*}
					&{  }&\dfrac{S_{\triangle AMO}}{S_{\triangle ABI}} = \dfrac{AO}{AI}\cdot\dfrac{AM}{AB}\\
					&=& \dfrac{1}{2}\cdot\dfrac{AM}{AB}\Rightarrow \dfrac{2\cdot S_{\triangle AMO}}{S_{\triangle ABC}} = \dfrac{1}{2}\cdot\dfrac{AM}{AB}\quad (1)
				\end{eqnarray*}	
			}
			{\begin{tikzpicture}[scale=1, font=\footnotesize, line join=round, line cap=round, >=stealth]
					\clip(-1,-1) rectangle (6.5,4.5);
					\tkzDefPoints{0/0/B,1/4/A, 6/0/C}
					\tkzDefMidPoint(B,C) \tkzGetPoint{I}
					\tkzDefMidPoint(A,I) \tkzGetPoint{O}
					\tkzDefBarycentricPoint(A=2,B=3)
					\tkzGetPoint{M}
					\tkzInterLL(M,O)(A,C)\tkzGetPoint{N}
					\tkzInterLL(B,O)(A,C)\tkzGetPoint{N'}
					\tkzInterLL(C,O)(A,B)\tkzGetPoint{M'}
					\tkzDefLine[parallel = through I](B,N') \tkzGetPoint{c}
					\tkzInterLL(I,c)(A,C)\tkzGetPoint{I'}
					\tkzLabelPoints[above](A)
					\tkzLabelPoints[left](M,M')
					\tkzLabelPoints[above right](N,N',I')
					\tkzLabelPoints[below left](I)
					\tkzLabelPoints[below](B,C)
					\tkzDrawPoints[fill=black](A,B,C,M,N,O,I,M',N',I')
					\tkzDrawSegments(A,B B,C C,A A,I M,N)
					\tkzDrawSegments [dashed](B,N' C,M' I,I')
					\draw ($(O)+(-0.1,-0.35)$) node {$O$};
				\end{tikzpicture}
			}\noindent
			Tương tự  $\dfrac{2\cdot S_{\triangle ANO}}{S_{\triangle ABC}} = \dfrac{1}{2}\cdot\dfrac{AN}{AC}\quad (2)$.	
			Từ $(1)$ và $(2)$ suy ra 
			$$\dfrac{2\cdot S_{\triangle AMN}}{S_{\triangle ABC}} = \dfrac{1}{2}\cdot\left(\dfrac{AM}{AB} + \dfrac{AN}{AC}\right)\Leftrightarrow \dfrac{S_{\triangle AMN}}{S_{\triangle ABC}} = \dfrac{1}{4}\cdot\left(\dfrac{AM}{AB} + \dfrac{AN}{AC}\right)$$
			Theo bất đẳng thức Côsi suy ra 
			$$\dfrac{AM}{AB} + \dfrac{AN}{AC}\geq 2\sqrt{\dfrac{AM}{AB}\cdot \dfrac{AN}{AC}}\Leftrightarrow \left(\dfrac{AM}{AB} + \dfrac{AN}{AC}\right)^2\geq 4\cdot \dfrac{AM}{AB}\cdot \dfrac{AN}{AC}$$
			Đặt $t = \dfrac{S_{\triangle AMN}}{S_{\triangle ABC}}$ điều kiện $t > 0$. Khi đó ta có  $16t^2\geq 4t\Leftrightarrow t\geq\dfrac{1}{4}$ suy ra $S_{\triangle AMN}\geq \dfrac{1}{4}\cdot S_{ABC}$.\\
			Khi $M\equiv B$ suy ra $N\equiv N'$ khi đó $S_{\triangle AMN} =  S_{\triangle ABN'}$.\\
			Mà $S_{\triangle ABN'} = S_{\triangle ABO}+ S_{\triangle AON'}$.\\
			Dễ thấy $S_{\triangle ABO} = \dfrac{1}{2}\cdot S_{\triangle ABI} = \dfrac{1}{4}\cdot S_{\triangle ABC}$.\\
			Mặt khác từ $I$ kẻ $II'\parallel BN'$, khi đó $AN' = N'I' = I'C$ nên
			$$\dfrac{S_{\triangle AON'}}{S_{\triangle AIC}} = \dfrac{AO}{AI}\cdot\dfrac{AN'}{AC} = \dfrac{1}{2}\cdot\dfrac{1}{3} = \dfrac{1}{6}\Rrightarrow S_{\triangle AON'} = \dfrac{1}{6}\cdot S_{\triangle AIC}$$
			Do đó $S_{\triangle AON'} = \dfrac{1}{12}\cdot S_{\triangle ABC}$ nên $S_{\triangle ABN'} = \dfrac{1}{4}\cdot S_{\triangle ABC} + \dfrac{1}{12}\cdot S_{\triangle ABC} = \dfrac{1}{3}\cdot S_{\triangle ABC}$.\\
			Khi $N\equiv C$ suy ra $M\equiv M'$ khi đó $S_{\triangle AMN} =  S_{\triangle ACM'}$.\\
			Chứng minh tương tự, ta có  $S_{\triangle ACM'} = \dfrac{1}{3}\cdot S_{\triangle ABC}$.\\
			Do đó khi $MN$ đi thay đổi qua $O$ suy ra 
			$$\dfrac{1}{4}\cdot S_{\triangle ABC}\leq  S_{\triangle AMN}\leq  \dfrac{1}{3}\cdot S_{\triangle ABC}\Leftrightarrow \dfrac{1}{4}\cdot S\leq S_{\triangle AMN} \leq  \dfrac{1}{3}\cdot S$$
			Do đó $m = \dfrac{1}{4}$ và $n = \dfrac{1}{3}$ nên $T = \dfrac{1}{m} + \dfrac{1}{n} = 4 + 3 = 7$.
		}
	\end{ex}
	\Closesolutionfile{ans}
	\Closesolutionfile{ansbook}
	% % \indapan{10}{ans/ans-0D3-6-TN}
	
%%Chương 3
% \chap{HÀM SỐ VÀ ĐỒ THỊ}
\setcounter{section}{0}
\section{Hàm số}

\subsection{Tóm tắt lý thuyết}
\subsubsection{Hàm số và tập xác định của hàm số}
\begin{dn}{}
	Giả sử $x$ và $y$ là hai đại lượng biến thiên và $x$ nhận giá trị thuộc tập số $\mathscr{D}$.\\
	Nếu với mỗi giá trị của $x$ thuộc tập $\mathscr{D}$, ta xác định được một và chỉ một giá trị tương ứng $y$ thuộc tập số thực $\mathbb{R}$ thì ta có một \textbf{hàm số}. \\
	Ta gọi $x$ là \textbf{biến số} và $y$ là \textbf{hàm số} của $x$.\\
	Tập hợp $\mathscr{D}$ được gọi là \textbf{tập xác định} của hàm số.\\
	Tập hợp $T$ gồm tất cả các giá trị $y$ (tương ứng với $x$ thuộc $\mathscr{D}$) được gọi là \textbf{tập giá trị} của hàm số.
\end{dn}

\subsubsection{Cách cho hàm số}

\begin{listEX}[3]
	\item Cho bằng bảng
	\item Cho bằng biểu đồ
	\item Cho bằng công thức
\end{listEX}

\begin{note}
	Khi cho hàm số bằng công thức mà không chỉ rõ tập xác định của nó thì ta quy ước \textbf{tập xác định} của hàm số $y=f(x)$ là tập hợp tất cả các số thực $x$ để biểu thức $f(x)$ \textbf{có nghĩa}.
\end{note}

\subsubsection{Đồ thị của hàm số}
\begin{dn}{}
	Cho hàm số $y=f(x)$ có tập xác định $\mathscr{D}$. Trên mặt phẳng tọa độ $Oxy$, \textbf{đồ thị} $(C)$ của hàm số là tập hợp tất cả các điểm $M(x;y)$ với $x\in\mathscr{D}$ và $y=f(x)$.\\
	Vậy $(C)=\{M(x;f(x))\mid x\in\mathscr{D}\}$.
\end{dn}

Ta thường gặp trường hợp đồ thị của hàm số $y=f(x)$ là một đường (đường thẳng, đường cong,...). Khi đó, ta nói $y=f(x)$ là \textbf{phương trình} của đường đó.

\subsubsection{Sự biến thiên của hàm số}

\begin{dn}{}
	Hàm số $y=f(x)$ gọi là \textbf{đồng biến (tăng)} trên khoảng $(a;b)$ nếu
	$$\forall x_1, x_2\in (a;b), x_1 < x_2 \Rightarrow f(x_1) < f(x_2).$$
	Hàm số $y=f(x)$ gọi là \textbf{nghịch biến (giảm)} trên khoảng $(a;b)$ nếu
	$$\forall x_1, x_2\in (a;b), x_1 < x_2 \Rightarrow f(x_1) > f(x_2).$$
\end{dn}

%\begin{note}
%	\textbf{Xét chiều biến thiên của một hàm số} là tìm các khoảng đồng biến và các khoảng nghịch biến của hàm số đó.
%\end{note} 

\begin{note}
	Khi hàm số đồng biến trên $(a;b)$ thì đồ thị của nó có dạng đi lên từ trái sang phải.\\
	Khi hàm số nghịch biến trên $(a;b)$ thì đồ thị của nó có dạng đi xuống từ trái sang phải.
\end{note}
\begin{center}
	\begin{tikzpicture}[line cap=round, line join=round, scale=.6]
		\begin{axis}[
				legend pos=outer north east,
				xlabel = $x$,
				ylabel = $y$,
				axis lines = middle
			]
			\addplot [
				domain=-2:2,
				samples=100,
				%color=blue,
			]
			{-x^2+1.5};
			% \addlegendentry{$f(x)$}
		\end{axis}
	\end{tikzpicture}
	\begin{tikzpicture}[line join = round, line cap = round,>=stealth,font=\footnotesize,scale=1]
		\tkzTabInit[nocadre=false,lgt=1.2,espcl=2.5,deltacl=0.6]
		{$x$ /0.6, $f(x)$ /2}
		{$-\infty$,$0$,$+\infty$}
		\tkzTabVar{-/$ $,+/$ $,-/$ $}
	\end{tikzpicture}
\end{center}
\subsection{Các dạng toán}

\begin{dang}{Tập xác định, tập giá trị của hàm số}
\end{dang}

\viduminhhoa

\begin{vd}%[Thành Đức Trung]%[0D2B1-2]
	Tìm tập xác định của các hàm số sau
	\begin{enumerate}
		\begin{minipage}{0.5\linewidth}
			\item $y=\dfrac{\sqrt[3]{x^2-1}}{x^2+2x+3}$.
		\end{minipage} \begin{minipage}{0.5\linewidth}
			\item $y=\dfrac{x}{x-\sqrt{x}-6}$.
		\end{minipage}
		\begin{minipage}{0.5\linewidth}
			\item $y=\sqrt{x+2}-\sqrt{x+3}$.
		\end{minipage} \begin{minipage}{0.5\linewidth}
			\item $y=\heva{ & \dfrac{1}{x} & & \text{khi} \ x\geqslant1 \\ & \sqrt{1-x} & & \text{khi} \ x<0.}$
		\end{minipage}
	\end{enumerate}
	\loigiai{
		\begin{enumerate}
			\item Điều kiện xác định $x^2+2x+3\neq0$ đúng với mọi $x$. \\
			      Vậy tập xác định của hàm số là $\mathscr{D}=\mathbb{R}$.
			\item Điều kiện xác định $\heva{ & x\geqslant0 \\ & x-\sqrt x-6\neq0} \Leftrightarrow \heva{ & x\geqslant0 \\ &\sqrt{x}\neq-2 \\ &\sqrt{x}\neq3} \Leftrightarrow \heva{ & x\geqslant0 \\ & x \neq9.}$\\
			      Vậy tập xác định của hàm số là $\mathscr{D}=[0;+\infty)\setminus\{9\}$.
			\item Điều kiện xác định $\heva{ & x+2\geqslant0 \\ & x+3\geqslant0} \Leftrightarrow \heva{ & x\geqslant-2 \\ & x\geqslant-3} \Leftrightarrow x\geqslant-2$. \\
			      Vậy tập xác định của hàm số là $\mathscr{D}=[-2;+\infty)$.
			\item Khi $x\geqslant1$ thì hàm số là $y=\dfrac{1}{x}$ luôn xác định với $x\geqslant1$. \\
			      Khi $x<0$ thì hàm số là $y=\sqrt{1-x}$ luôn xác định với $x<0$. \\
			      Vậy tập xác định của hàm số là $\mathscr{D}=(-\infty;0)\cup[1;+\infty)$.
		\end{enumerate}
	}
\end{vd}

\begin{vd}%[Thành Đức Trung]%[0D1Y1-2]
	Cho bảng giá trị tương ứng của hai đại lượng $x$ và $y$. Đại lượng $y$ có là hàm số của đại lượng $x$ không? Nếu có, hãy tìm tập xác định và tập giá trị của hàm số đó.
	\begin{enumerate}
		\item
		      \begin{tabular}{|c|c|c|c|c|c|c|c|c|c|}
			      \hline
			      $x$ & $-5$ & $-3$ & $-1$ & $0$ & $1$ & $2$ & $5$ & $8$  & $9$  \\
			      \hline
			      $y$ & $-6$ & $-8$ & $-4$ & $1$ & $3$ & $2$ & $3$ & $12$ & $15$ \\
			      \hline
		      \end{tabular}
		\item
		      \begin{tabular}{|c|c|c|c|c|c|c|c|c|c|}
			      \hline
			      $x$ & $-10$ & $-8$  & $-4$ & $2$ & $3$ & $6$  & $7$  & $6$  & $13$ \\
			      \hline
			      $y$ & $-16$ & $-14$ & $-2$ & $4$ & $5$ & $20$ & $18$ & $24$ & $25$ \\
			      \hline
		      \end{tabular}
	\end{enumerate}
	\loigiai
	{
		\begin{enumerate}
			\item Đại lượng $y$ có là hàm số của đại lượng $x$ vì mỗi giá trị của $x$ có duy nhất một giá trị $y$ tương ứng. \\
			      Tập xác định là $\{-5;-3;-1;0;1;2;5;8;9\}$. \\
			      Tập giá trị là $\{-8;-6;-4;1;2;3;12;15\}$.
			\item Đại lượng $y$ không là hàm số của đại lượng $x$ vì với $x=6$ có hai giá trị $y=20$ và $y=24$.
		\end{enumerate}
	}
\end{vd}

\begin{vd}%[Thành Đức Trung]%[0D2B1-2]
	Tìm tất cả các giá trị thực của tham số $m$ để hàm số $y=\dfrac{x+2m+2}{x-m}$ xác định trên $(-1;0)$.
	\loigiai
	{
		Điều kiện $x-m\neq0 \Leftrightarrow x\neq m$. \\
		Hàm số xác định trên $(-1;0)$ khi và chỉ khi $m\notin(-1;0) \Leftrightarrow \hoac{ & m\leqslant-1 \\ & m\geqslant0.}$
	}
\end{vd}

\begin{vd}%[Thành Đức Trung]%[0D2K1-2]
	Tìm tất cả các giá trị thực của tham số $m$ để hàm số $ y=\sqrt{x-m+1}+\dfrac{2x}{\sqrt{-x+2m}}$ xác định trên khoảng $(-1;3)$.
	\loigiai
	{
	Điều kiện $\heva{ & x-m+1\geqslant0 \\ & -x+2m>0} \Leftrightarrow \heva{ & x\geqslant m-1 \\ & x<2m.}$ \\
	Ta cần $m-1<2m \Leftrightarrow m>-1$. \\
	Suy ra tập xác định là $\mathscr{D}=[m-1;2m)$. \\
	Hàm số xác định trên $(-1;3)$ khi $(-1;3)\subset\mathscr{D} \Leftrightarrow \Leftrightarrow \heva{ & m-1\leqslant-1 \\ & 2m\geqslant3} \Leftrightarrow \heva{ & m\leqslant0 \\ & m\geqslant\dfrac{3}{2}}$: vô nghiệm. \\
	Vậy không có giá trị $m$ thỏa mãn.
	}
\end{vd}

\begin{vd}%[Thành Đức Trung]%[0D2B1-1]
	Cho hàm số $f(x)=\heva{ & x-\sqrt{x^2+m^2} & & \text{khi} \ x<1 \\ & 2x & & \text{khi} \ x\geqslant1}$ với $m$ là tham số. Biết đồ thị hàm số cắt trục tung tại điểm có tung độ bằng $-3$. Tính giá trị biểu thức $P=f(-4)+f(1)$.
	\loigiai
	{
		Vì đồ thị hàm số cắt trục tung tại điểm có tung độ bằng $-3$ nên
		\[f(0)=-3 \Leftrightarrow -\sqrt{m^2}=-3 \Leftrightarrow m^2=9.\]
		Ta có $P=-4-\sqrt{\left(-4\right)^2+9}+2\cdot1=-7$.
	}
\end{vd}

\baitaptl

\begin{bt}%[Thành Đức Trung]%[0D2B1-2]
	Tìm tập xác định của các hàm số sau
	\begin{enumerate}[\indent a)]
		\begin{minipage}{0.5\linewidth}
			\item $y=-x^2$.
		\end{minipage} \begin{minipage}{0.5\linewidth}
			\item $y=\sqrt{2-3x}$.
		\end{minipage}
		\begin{minipage}{0.5\linewidth}
			\item $y=\dfrac{4}{x+1}$.
		\end{minipage} \begin{minipage}{0.5\linewidth}
			\item $y=\heva{ & 1 & & \text{nếu} \ x\in\mathbb{Q} \\ & 0 & & \text{nếu} \ x\in\mathbb{R}\setminus\mathbb{Q}.}$
		\end{minipage}
	\end{enumerate}
	\loigiai
	{
		\begin{enumerate}[\indent a)]
			\item Tập xác định của hàm số là $\mathscr{D}=\mathbb{R}$.
			\item Điều kiện xác định $2-3x\geqslant0 \Leftrightarrow x\leqslant\dfrac{2}{3}$. \\
			      Vậy tập xác định của hàm số là $\mathscr{D}=\left(-\infty;\dfrac{2}{3}\right]$.
			\item Điều kiện xác định $x+1\neq0 \Leftrightarrow x\neq-1$. \\
			      Vậy tập xác định của hàm số là $\mathscr{D}=\mathbb{R}\setminus\{-1\}$.
			\item Khi $x\in\mathbb{Q}$ thì hàm số là $y=1$ luôn xác định với $x\in\mathbb{Q}$. \\
			      Khi $x\in\mathbb{R}\setminus\mathbb{Q}$ thì hàm số là $y=0$ luôn xác định với $x\in\mathbb{R}\setminus\mathbb{Q}$. \\
			      Vậy tập xác định của hàm số là $\mathscr{D}=\mathbb{R}$.
		\end{enumerate}
	}
\end{bt}

\begin{bt}%[Thành Đức Trung]%[0D1Y1-2]
	Theo quyết định số 2019/QĐ-BĐVN ngày 01/11/2018 của Tổng công ty Bưu điện Việt Nam, giá cước dịch vụ Bưu chính phổ cập đối với dịch vụ thư cơ bản và bưu thiếp trong nước có khối lượng đến 250g như trong bảng sau
	\immini
	{
		\begin{enumerate}
			\item Số tiền dịch vụ thư cơ bản phải trả $y$ (đồng) có là hàm số của khối lượng thư cơ bản $x$ (g) hay không? Nếu đúng, hãy xác định những công thức tính $y$.
			\item Tính số tiền phải trả khi bạn Dương gửi thư có khối lượng $150$ g, $200$ g.
		\end{enumerate}
	}
	{
		\begin{tabular}{|l|c|}
			\hline
			Khối lượng đến $250$ g   & Mức cước (đồng) \\
			\hline
			Đến $20$ g               & $4 000$         \\
			\hline
			Trên $20$ g đến $100$ g  & $6 000$         \\
			\hline
			Trên $100$ g đến $250$ g & $8 000$         \\
			\hline
		\end{tabular}
	}
	\loigiai
	{
		\begin{enumerate}
			\item Đại lượng $y$ có là hàm số của đại lượng $x$ vì mỗi giá trị của $x$ có duy nhất một giá trị $y$ tương ứng. \\
			      Ta có $y=\heva{ & 4000 & & \text{nếu} \ x\leqslant20 \\ & 6000 & & \text{nếu} \ 20<x\leqslant100 \\ & 8000 & & \text{nếu} \ 100<x\leqslant250.}$ \\
			\item Số tiền bạn Dương phải trả khi gửi thư có khối lượng $150$ g là $8000$ đồng. \\
			      Số tiền bạn Dương phải trả khi gửi thư có khối lượng $200$ g là $8000$ đồng.
		\end{enumerate}
	}
\end{bt}

\begin{bt}%[Thành Đức Trung]%[0D2K1-2]
	Cho hàm số $y=\sqrt{2x-3m+4}+\dfrac{x}{x+m-1}$ với $m$ là tham số. Tìm $m$ để hàm số có tập xác định là $[0;+\infty)$.
	\loigiai
	{
	Điều kiện xác định $\heva{ & 2x-3m+4\geqslant0 \\ & x+m-1\neq0} \Leftrightarrow \heva{ & x\geqslant\dfrac{3m-4}{2} \\ & x\neq1-m.}$ \\
	Với $1-m\geqslant\dfrac{3m-4}{2} \Leftrightarrow m\leqslant\dfrac{6}{5}$, khi đó tập xác định của hàm số là $\mathscr{D}=\left[\dfrac{3m-4}2;+\infty\right)\setminus\{1-m\}$. \\
	Do đó $m\leqslant\dfrac{6}{5}$ không thỏa mãn yêu cầu bài toán. \\
	Với $m>\dfrac{6}{5}$ khi đó tập xác định của hàm số là $\mathscr{D}=\left[\dfrac{3m-4}2;+\infty\right)$.\\
	Do đó để hàm số có tập xác định là $[0;+\infty) \Leftrightarrow \dfrac{3m-4}{2}=0 \Leftrightarrow m=\dfrac{4}{3}$ (thỏa mãn). \\
	Vậy $m=\dfrac{4}{3}$ là giá trị cần tìm.
	}
\end{bt}

\begin{bt}%[Thành Đức Trung]%[0D2K1-2]
	Tìm tất cả các giá trị thực của tham số $m$ để hàm số $y=\dfrac{mx}{\sqrt{x-m+2}-1}$ xác định trên $(0;1)$.
	\loigiai
	{
	Điều kiện xác định $\heva{ & x-m+2\geqslant0 \\ & \sqrt{x-m+2}-1\neq0} \Leftrightarrow \heva{ & x\geqslant m-2 \\ & x\neq m-1.}$ \\
	Suy ra tập xác định $\mathscr{D}=[m-2;+\infty)\setminus\{m-1\}$. \\
	Hàm số xác định trên $(0;1)$ khi $(0;1)\subset\mathscr{D} \Leftrightarrow \heva{ & m-2\leqslant0 \\ & \hoac{ & m-1\leqslant0 \\ & m-1\geqslant1}} \Leftrightarrow \heva{ & m\leqslant2 \\ & \hoac{ & m\leqslant1 \\ & m\geqslant2}} \Leftrightarrow \hoac{ & m\leqslant1 \\ & m=2.}$
	}
\end{bt}

\begin{bt}%[Thành Đức Trung]%[0D2B1-1]
	Cho hàm số $f(x)=\heva{ & 2x+m & & \text{khi} \ x<3 \\ & x^2+4 & & \text{khi} \ x\geqslant3}$ với $m$ là tham số. Biết đồ thị hàm số cắt trục tung tại điểm có tung độ bằng $4$. Tính giá trị biểu thức $T=f(0)+f(10)$.
	\loigiai
	{
		Vì đồ thị hàm số cắt trục tung tại điểm có tung độ bằng $4$ nên $f(0)=4 \Leftrightarrow m=4$. \\
		Ta có $T=4+10^2+4=108$.
	}
\end{bt}


\begin{dang}{Tính đồng biến nghịch biến của hàm số}

\end{dang}
\viduminhhoa
\begin{vd}%[Bùi Mạnh Tiến]%[0D2B1-3]
	Xét tính đồng biến nghịch biến của hàm số
	\begin{enumerate}
		\item $y=f(x)=x^2-3x+2$ trên khoảng $\left(-\infty;
			      \dfrac{3}{2}\right)$;
		\item $y=g(x)=\dfrac{x-1}{x+1}$ trên khoảng $(-1;+\infty)$;
		\item $y=h(x)=\sqrt{4-3x}$ trên khoảng $\left(-\infty;\dfrac{4}{3}\right)$.
		\item $y=t(x)=|x-2|$ trên các khoảng $(-\infty;2)$ và $(2;+\infty)$.
	\end{enumerate}
	\loigiai
	{
		\begin{enumerate}
			\item Với mọi $x_1$, $x_2\in \left(-\infty;
				      \dfrac{3}{2}\right)$ và $x_1\neq x_2$ ta có $x_1<\dfrac{3}{2}$ và $x_2<\dfrac{3}{2}\Rightarrow x_1+x_2<3$. Khi đó
			      \begin{eqnarray*}
				      P&=&\dfrac{f(x_2)-f(x_1)}{x_2-x_1}\\
				      &=&\dfrac{x_2^2-3x_2+2-(x_1^2-3x_1+2)}{x_2-x_1}\\
				      &=&\dfrac{(x_2-x_1)(x_2+x_1-3)}{x_2-x_1}\\
				      &=&x_1+x_2-3<0.
			      \end{eqnarray*}
			      Do đó $y=f(x)$ là hàm số nghịch biến trên $\left(-\infty;\dfrac{3}{2}\right)$.
			\item Với mọi $x_1$, $x_2\in (-1;+\infty)$ và $x_1\neq x_2$ ta có $x_1>-1$, $x_2>-1\Rightarrow x_1+1>0$, $x_2+1>0$. Khi đó
			      \begin{eqnarray*}
				      P&=&\dfrac{g(x_2)-g(x_1)}{x_2-x_1}\\
				      &=&\dfrac{\dfrac{x_2-1}{x_2+1}-\dfrac{x_1-1}{x_1+1}}{x_2-x_1}\\
				      &=&\dfrac{\dfrac{(x_2-1)(x_1+1)-(x_1-1)(x_2+1)}{(x_2+1)(x_1+1)}}{x_2-x_1}\\
				      &=&\dfrac{2(x_2-x_1)}{(x_2+1)(x_1+1)(x_2-x_1)}\\
				      &=&\dfrac{2}{(x_2+1)(x_1+1)}>0.
			      \end{eqnarray*}
			      Vậy $y=g(x)$ là hàm đồng biến trên $(-1;+\infty)$.
			\item Với mọi $x_1$, $x_2\in \left(-\infty;\dfrac{4}{3}\right)$ và $x_1\neq x_2$ ta có
			      \begin{eqnarray*}
				      P&=&\dfrac{h(x_2)-h(x_1)}{x_2-x_1}\\
				      &=&\dfrac{\sqrt{4-3x_2}-\sqrt{4-3x_1}}{x_2-x_1}=\dfrac{4-3x_2-(4-3x_1)}{(x_2-x_1)(\sqrt{4-3x_2}+\sqrt{4-3x_1})}\\
				      &=&-\dfrac{3}{\sqrt{4-3x_2}+\sqrt{4-3x_1}}<0.
			      \end{eqnarray*}
			      Vậy hàm số đã cho nghịch biến trên khoảng $\left(-\infty;\dfrac{4}{3}\right)$.
			\item Xét biểu thức $P=\dfrac{t(x_2)-t(x_1)}{x_2-x_1}=\dfrac{|x_2-2|-|x_1-2|}{x_2-x_1}$.
			      \begin{itemize}
				      \item Với mọi $x_1$, $x_2\in (-\infty;2)$ và $x_1\neq x_2$ thì $x_1<2$, $x_2<2$ nên $|x_1-2|=2-x_1$ và $|x_2-2|=2-x_2$, do đó
				            \begin{align*}
					            P=\dfrac{2-x_2-(2-x_1)}{x_2-x_1}=\dfrac{x_1-x_2}{x_2-x_1}=-1<0.
				            \end{align*}
				      \item Với mọi $x_1$, $x_2\in (2;+\infty)$ và $x_1\neq x_2$ thì $x_1>2$, $x_2>2$ nên $|x_1-2|=x_1-2$ và $|x_2-2|=x_2-2$, do đó
				            \begin{align*}
					            P=\dfrac{x_2-2-(x_1-2)}{x_2-x_1}=\dfrac{x_2-x_1}{x_2-x_1}=1>0.
				            \end{align*}
			      \end{itemize}
			      Vậy hàm số $y=t(x)$ đồng biến trên khoảng $(2;+\infty)$ và nghịch biến trên khoảng $(-\infty;2)$.
		\end{enumerate}
	}
\end{vd}

\begin{vd}%[Bùi Mạnh Tiến]%[0D2B1-3]
	Tìm tất cả các giá trị của tham số $m$ để hàm số $y=f(x)=(1-3m)x+2m-2$ đồng biến trên tập xác định.
	\loigiai{
		Tập xác định: $\mathscr{D}=\mathbb{R}$.\\
		Gọi $x_1$, $x_2$ là hai giá trị phân biệt tùy ý thuộc $\mathbb{R}$, ta có
		\begin{align*}
			\dfrac{f(x_2)-f(x_1)}{x_2-x_1}=\dfrac{\left[(1-3m)x_2+2m-2\right]-\left[(1-3m)x_1+2m-2\right]}{x_2-x_1}=\dfrac{(1-3m)(x_2-x_1)}{x_2-x_1}=1-3m.
		\end{align*}
		Hàm số đồng biến trên $\mathbb{R}$ khi và chỉ khi $1-3m>0\Leftrightarrow m<\dfrac{1}{3}$.\\
		Vậy $m<\dfrac{1}{3}$.
	}
\end{vd}

\begin{vd}%[Bùi Mạnh Tiến]%[0D2B1-3]
	Cho hàm số $y=f(x)$ có đồ thị như hình vẽ bên dưới
	\begin{center}
		\begin{tikzpicture}[line join = round, line cap = round,>=stealth,font=\footnotesize,scale=1]
			\def\xmin{-2};
			\def\xmax{2};
			\def\ymin{-0.5};
			\def\ymax{4};
			\clip (\xmin,\ymin) rectangle (\xmax+0.15,\ymax+0.15);
			\draw[->] (\xmin,0) -- (\xmax,0) node[below] {$x$};
			\draw[->] (0,\ymin) -- (0,0) node[below left] {$O$} -- (0,\ymax) node[left] {$y$};
			\draw[smooth,samples=100] plot[domain=-1.7:1.7] (\x,{(\x)^4-2*(\x)^2+1});
			\fill (-1,0) circle (1.2pt) node[below] {$-1$};
			\fill (1,0) circle (1.2pt) node[below] {$1$};
		\end{tikzpicture}
	\end{center}
	Xác định các khoảng đồng biến và nghịch biến của hàm số.
	\loigiai{
		Từ đồ thị trên ta thấy
		\begin{itemize}
			\item hàm số đồng biến trên các khoảng $(-1;0)$ và $(1;+\infty)$.
			\item hàm số nghịch biến trên các khoảng $(-\infty;-1)$ và $(0;1)$.
		\end{itemize}
	}
\end{vd}

% \begin{vd}%[Bùi Mạnh Tiến]%[0D2B1-3]
% 	Tìm $m$ để hàm số $y=mx-\sqrt{2-m}$ đồng biến trên $\mathbb{R}$?
% 	\loigiai{
% 		Tập xác định $\mathscr{D}=\mathbb{R}$.\\
% 		Ta chỉ xét với $ 2-m\ge 0\Leftrightarrow m\le 2$. \quad (1)\\
% 		Với mọi $x_1$, $x_2\in \mathbb{R}$, $x_1\neq x_2$. Xét biểu thức
% 		\begin{align*}
% 			P=\dfrac{f(x_2)-f(x_1)}{x_2-x_1}=\dfrac{mx_2-\sqrt{2-m}-\left(mx_1-\sqrt{2-m}\right)}{x_2-x_1}=\dfrac{m(x_2-x_1)}{x_2-x_1}=m.
% 		\end{align*}
% 		Hàm số đồng biến trên $\mathbb{R}$ khi $m>0$. \quad (2)\\
% 		Từ $(1)$ và $(2)$ suy ra $0<m\leq 2$.
% 	}
% \end{vd}
%
%\begin{vd}
%	
%\end{vd}

\baitaptl

\begin{bt}%[Bùi Mạnh Tiến]%[0D2B1-3]
	Xét tính đồng biến nghịch biến của hàm số
	\begin{enumerate}
		\item $y=f(x)=\dfrac{-x+2}{x-1}$ trên khoảng $(-\infty;1)$ và $(1;+\infty)$.
		\item $y=g(x)=\dfrac{x-2}{2x-3}$ trên khoảng $\left(-\infty;\dfrac{3}{2}\right)$ và $\left(\dfrac{3}{2};+\infty\right)$.
	\end{enumerate}
	\loigiai
	{
		\begin{enumerate}
			\item Xét biểu thức $P=\dfrac{f(x_2)-f(x_1)}{x_2-x_1}$.
			      \begin{itemize}
				      \item Với mọi $x_1$, $x_2\in (-\infty;1)$ và $x_1\neq x_2$ thì $x_1<1$, $x_2<1$ do đó $(x_1-1)(x_2-1)>0$. Khi đó
				            \begin{align*}
					            P=\dfrac{\dfrac{-x_2+2}{x_2-1}-\dfrac{-x_1+2}{x_1-1}}{x_2-x_1}=\dfrac{-1}{(x_2-1)(x_1-1)}<0.
				            \end{align*}
				      \item Với mọi $x_1$, $x_2\in (1;+\infty)$ và $x_1\neq x_2$ thì $x_1>1$, $x_2>1$ do đó $(x_1-1)(x_2-1)>0$. Khi đó
				            \begin{align*}
					            P=\dfrac{\dfrac{-x_2+2}{x_2-1}-\dfrac{-x_1+2}{x_1-1}}{x_2-x_1}=\dfrac{-1}{(x_2-1)(x_1-1)}<0.
				            \end{align*}
			      \end{itemize}
			      Vậy hàm số $y=f(x)$ là hàm nghịch biến trên các khoảng $(-\infty;1)$ và $(1;+\infty)$.
			\item Xét biểu thức $P=\dfrac{g(x_2)-g(x_1)}{x_2-x_1}$.
			      \begin{itemize}
				      \item Với mọi $x_1$, $x_2\in \left(-\infty;\dfrac{3}{2}\right)$ và $x_1\neq x_2$ thì $x_1<\dfrac{3}{2}$, $x_2<\dfrac{3}{2}$ do đó $\left(x_1-\dfrac{3}{2}\right)\left(x_2-\dfrac{3}{2}\right)>0$. Khi đó
				            \begin{align*}
					            P=\dfrac{\dfrac{x_2-2}{2x_2-3}-\dfrac{x_1-2}{2x_1-3}}{x_2-x_1}=\dfrac{1}{(2x_2-3)(2x_1-3)}>0.
				            \end{align*}
				      \item Với mọi $x_1$, $x_2\in \left(\dfrac{3}{2};+\infty\right)$ và $x_1\neq x_2$ thì $x_1>\dfrac{3}{2}$, $x_2>\dfrac{3}{2}$ do đó $\left(x_1-\dfrac{3}{2}\right)\left(x_2-\dfrac{3}{2}\right)>0$. Khi đó
				            \begin{align*}
					            P=\dfrac{\dfrac{x_2-2}{2x_2-3}-\dfrac{x_1-2}{2x_1-3}}{x_2-x_1}=\dfrac{1}{(2x_2-3)(2x_1-3)}>0.
				            \end{align*}
			      \end{itemize}
		\end{enumerate}
	}
\end{bt}

\begin{bt}%[Bùi Mạnh Tiến]%[0D2B1-3]
	Dùng định nghĩa xét sự đồng biến nghịch biến của hàm số $y=f(x)=x^2+2x+2$ trên các khoảng $(-\infty;-1)$, $(-1;+\infty)$.
	\loigiai{\\
		Xét biểu thức
		\begin{align*}
			P=\dfrac{f(x_1)-f(x_2)}{x_1-x_2}=\dfrac{(x_1^2+2x_1+2)-(x_2^2+2x_2+2)}{x_1-x_2}=x_1+x_2+2.
		\end{align*}
		\begin{itemize}
			\item Trường hợp $x_1, x_2$ phân biệt cùng thuộc $(-\infty;-1)\Rightarrow x_1<-1$, $x_2<-1$ thì $x_1+x_2+2<0\Leftrightarrow P<0$, suy ra hàm số nghịch biến trên $(-\infty;-1)$.
			\item Trường hợp $x_1, x_2$ phân biệt cùng thuộc $(-1;+\infty)\Rightarrow x_1>-1$, $x_2>-1$ thì $P=x_1+x_2+2>0$, suy ra hàm số đồng biến trên $(-1;+\infty)$.
		\end{itemize}
	}
\end{bt}

% \begin{bt}%[Bùi Mạnh Tiến]%[0D2K1-3]
% 	Dùng định nghĩa xét sự đồng biến nghịch biến của hàm số $y=f(x)=\left|\sqrt{2-x}+1\right|$ trên khoảng $(-\infty;2)$
% 	\loigiai{\\
% 		Gọi $x_1, x_2$ là hai giá trị tùy ý thuộc $(-\infty;2)$, $x_1\neq x_2\Rightarrow 2-x_1>0$, $2-x_2>0$. Xét biểu thức
% 		\begin{eqnarray*}
% 			\dfrac{f(x_1)-f(x_2)}{x_1-x_2}&=&\dfrac{\left|\sqrt{2-x_1}+1\right|-\left|\sqrt{2-x_2}+1\right|}{x_1-x_2}\\
% 			&=&\dfrac{\sqrt{2-x_1}-\sqrt{2-x_2}}{x_1-x_2}\\
% 			&=&\dfrac{(2-x_1)-(2-x_2)}{(x_1-x_2)\left(\sqrt{2-x_1}+\sqrt{2-x_2}\right)}\\
% 			&=&\dfrac{-1}{\sqrt{2-x_1}+\sqrt{2-x_2}}<0.
% 		\end{eqnarray*}
% 		Vậy hàm số đã cho luôn nghịch biến trên khoảng $(-\infty;2)$.
% 	}
% \end{bt}

% \begin{bt}%[Bùi Mạnh Tiến]%[0D2K1-3]
% 	Dùng định nghĩa xét tính đơn điệu của hàm số $y=\dfrac{x}{x^2+1}$ trên các khoảng $(0;1)$, $(1;+\infty)$.
% 	\loigiai{
% 		Xét biểu thức
% 		\begin{eqnarray*}
% 			P&=&\dfrac{f(x_1)-f(x_2)}{x_1-x_2}\\
% 			&=&\dfrac{\dfrac{x_1}{x_1^2+1}-\dfrac{x_2}{x_2^2+1}}{x_1-x_2}\\
% 			&=&\dfrac{x_1(x_2^2+1)-x_2(x_1^2+1)}{(x_1-x_2)(x_1^2+1)(x_2^2+1)}\\
% 			&=&\dfrac{x_1x_2(x_2-x_1)-(x_2-x_1)}{(x_1-x_2)(x_1^2+1)(x_2^2+1)}\\
% 			&=&\dfrac{(1-x_1x_2)(x_1-x_2)}{(x_1-x_2)(x_1^2+1)(x_2^2+1)}\\
% 			&=&\dfrac{1-x_1x_2}{(x_1^2+1)(x_2^2+1)}.
% 		\end{eqnarray*}
% 		\begin{itemize}
% 			\item Trường hợp $x_1, x_2\in (0;1)$ suy ra $0<x_1$ ,$ x_2<1\Rightarrow P=1-x_1x_2>0$, từ đó ta có $$\dfrac{f(x_1)-f(x_2)}{x_1-x_2}>0.$$ \\
% 			      Vậy hàm số đã cho đồng biến trên khoảng $(0;1)$.
% 			\item Trường hợp $x_1, x_2\in (1;+\infty)$ suy ra $x_1$, $x_2>1\Rightarrow P=1-x_1x_2<0$.\\
% 			      Vậy hàm số đã cho nghịch biến trên khoảng $(1;+\infty)$.
% 		\end{itemize}
% 	}
% \end{bt}

\begin{bt}%[Bùi Mạnh Tiến]%[0D2K1-3]
	Tìm tất cả các giá trị của tham số $m$ để hàm số $y=f(x)=(2m-3)x+5-m$ nghịch biến trên tập xác định.
	\loigiai{
		Tập xác định: $\mathscr{D}=\mathbb{R}$.\\
		Gọi $x_1$, $x_2$ là hai giá trị phân biệt tùy ý thuộc $\mathbb{R}$, ta có
		\begin{align*}
			\dfrac{f(x_2)-f(x_1)}{x_2-x_1}=\dfrac{\left[(2m-3)x_2+5-m\right]-\left[(2m-3)x_1+5-m\right]}{x_2-x_1}=\dfrac{(2m-3)(x_2-x_1)}{x_2-x_1}=2m-3.
		\end{align*}
		Hàm số nghịch biến trên $\mathbb{R}$ khi và chỉ khi $2m-3<0\Leftrightarrow m<\dfrac{3}{2}$.\\
		Vậy $m<\dfrac{3}{2}$.
	}
\end{bt}

\begin{bt}%[Bùi Mạnh Tiến]%[0D2B1-3]
	Cho hàm số $y=f(x)$ có đồ thị như hình vẽ bên dưới
	\begin{center}
		\begin{tikzpicture}[line join = round, line cap = round,>=stealth,font=\footnotesize,scale=1]
			\def\xmin{-3.5};
			\def\xmax{3.5};
			\def\ymin{-1.5};
			\def\ymax{4.5};
			\clip (\xmin,\ymin) rectangle (\xmax+0.15,\ymax+0.15);
			\draw[->] (\xmin,0) -- (\xmax,0) node[below] {$x$};
			\draw[->] (0,\ymin) -- (0,0) node[below left] {$O$} -- (0,\ymax) node[left] {$y$};
			\draw (-3,-1) -- (-1,1) -- (1,0) -- (3,4);
			\draw[dashed] (-3,0) -- (-3,-1) -- (0,-1) (-1,0) -- (-1,1) -- (0,1) (3,0) -- (3,4) -- (0,4);
			\fill (-1,0) circle (1.2pt) node[below] {$-1$};
			\fill (-2,0) circle (1.2pt) node[below] {$-2$};
			\fill (-3,0) circle (1.2pt) node[above] {$-3$};
			\fill (0,-1) circle (1.2pt) node[right] {$-1$};
			\fill (0,1) circle (1.2pt) node[right] {$1$};
			\fill (0,4) circle (1.2pt) node[left] {$4$};
			\fill (1,0) circle (1.2pt) node[below] {$1$};
			\fill (3,0) circle (1.2pt) node[below] {$3$};
		\end{tikzpicture}
	\end{center}
	Hãy xác định các khoảng đồng biến và nghịch biến của hàm số trên $(-3;3)$.
	\loigiai
	{
		Dựa vào đồ thị ta thấy hàm số đã cho
		\begin{itemize}
			\item Đồng biến trên các khoảng $(-3,-1)$ và $(1;3)$;
			\item Nghịch biến trên các khoảng $(-1;1)$.
		\end{itemize}
	}
\end{bt}

\begin{bt}%[Bùi Mạnh Tiến]%[0D2K1-3]
	Cho hàm số $y=f(x)$ có bảng biến thiên như hình vẽ
	\begin{center}
		\begin{tikzpicture}[line join = round, line cap = round,>=stealth,font=\footnotesize,scale=1]
			\tkzTabInit[nocadre=false,lgt=1.2,espcl=2.5,deltacl=0.6]
			{$x$ /0.6, $f(x)$ /2}
			{$-\infty$,$0$,$4$,$+\infty$}
			\tkzTabVar{-/$-\infty$,+/$2$,-/$-32$,+/$+\infty$}
		\end{tikzpicture}
	\end{center}
	Chứng minh rằng hàm số $y=g(x)=5x-f(x)$ nghịch biến trên khoảng $(0;4)$.
	\loigiai
	{
		Từ bảng biến thiên ta thấy hàm số $y=f(x)$ nghịch biến trên $(0;4)$ nên với mỗi $x_1$, $x_2\in (0;4)$ và $x_1\neq x_2$ ta có
		\begin{align*}
			\dfrac{f(x_2)-f(x_1)}{x_2-x_1}<0.
		\end{align*}
		Với mỗi $x_1$, $x_2\in (0;4)$ và $x_1\neq x_2$ ta có
		\begin{align*}
			P=\dfrac{g(x_2)-g(x_1)}{x_2-x_1}=\dfrac{5x_2-f(x_2)-(5x_1-f(x_1))}{x_2-x_1}=5-\dfrac{f(x_2)-f(x_1)}{x_2-x_1}>5>0.
		\end{align*}
		Do đó $y=g(x)$ là hàm nghịch biến trên khoảng $(0;4)$.
	}
\end{bt}




\begin{dang}{Bài toán thực tế về hàm số}

\end{dang}
\viduminhhoa
%%==========Ví dụ 1
\begin{vd}%[Nguyễn Cường- BG Toán 10]%[0D2T2-5]
	Một cửa hàng bán sách online sẽ tính chi phí tiền vận chuyển sách khi mua sách như sau
	\begin{itemize}
		\item Số tiền mua sách không quá $2000000$ đồng thì phí vận chuyển là $50000$ đồng.
		\item Số tiền mua sách nhiều hơn $2000000$ đồng thì miễn phí vận chuyển.
	\end{itemize}
	\begin{enumerate}
		\item Viết công thức tính tổng chi phí $C$ mua sách của cửa hàng.
		\item Tính $C(1000000)$, $C(2000000)$, $C(2500000)$.
	\end{enumerate}
	\loigiai{
		\begin{enumerate}
			\item Viết công thức tính tổng chi phí $C$ mua sách của cửa hàng.\\
			      Gọi $x$ là số tiền mua sách. Ta có
			      $$C(x)=\heva{&x+50000&\text{nếu }x\le 2000000\\
					      &x&\text{nếu }x> 2000000.}$$
			\item Tính $C(1000000)$, $C(2000000)$, $C(2500000)$.\\
			      Ta có $C(1000000)=1000000$, $C(2000000)=2000000+50000=2050000$, $C(2500000)=2500000=2500000$.
		\end{enumerate}
	}
\end{vd}
%%==========Ví dụ 2
\begin{vd}%[Nguyễn Cường- BG Toán 10]%[0D2T2-5]
	Tốc độ quy định trên đường cao tốc là $50$ km/h đến $120$ km/h. Giả sử mức phạt tiền là $1000000$ đồng với mỗi km/h nếu tài xế chạy vượt quá tốc độ quy định hoặc dưới tốc độ quy định.
	\begin{enumerate}
		\item Hoàn thành hàm số $F(x)$ về quy định tiền phạt, với $x$ là tốc độ xe chạy.
		\item Tính $F(45)$, $F(60)$, $F(100)$, $F(125)$ và cho biết ý nghĩa của mỗi giá trị này.
	\end{enumerate}
	\loigiai{
		\begin{enumerate}
			\item Hoàn thành hàm số $F(x)$ về quy định tiền phạt, với $x$ km/h là tốc độ xe chạy.
			      $$F(x)=\heva{&1000000(x-120)&\text{nếu }x>120\\
					      &0&\text{nếu }50\le x\le 120\\
					      &1000000(50-x)&\text{nếu }x<50.}$$
			\item Tính $F(45)$, $F(60)$, $F(100)$, $F(125)$ và cho biết ý nghĩa của mỗi giá trị này.
			      \begin{itemize}
				      \item $F(45)=1000000(50-45)=5000000$ đồng, mức phạt $5000000$ đồng do xe chạy không đúng quy định tốc độ tối thiếu.
				      \item $F(60)=F(100)=0$ đồng, do xe đi trong giới hạn vận tốc cho phép.
				      \item $F(125)=1000000(125-120)=5000000$ đồng, mức phạt $5000000$ đồng do xe chạy không đúng quy định tốc độ tối đa.
			      \end{itemize}
		\end{enumerate}
	}
\end{vd}

%%==========Ví dụ 3
\begin{vd}%[Nguyễn Cường- BG Toán 10]%[0D2T2-5]
	Một khách sạn tại Đà Lạt cho thuê phòng với giá tiền $750000$ đồng một ngày đêm cho hai ngày đêm đầu tiên và $500000$ cho mỗi ngày đêm tiếp theo. Tổng số tiền $T$ cần phải trả là một hàm số của số ngày $x$ mà khách ở tại khách sạn.
	\begin{enumerate}
		\item Viết công thức của hàm số $T(x)$.
		\item Tính $T(2)$, $T(4)$, $T(6)$ và cho biết ý nghĩa của mỗi giá trị này.
	\end{enumerate}
	\loigiai{
		\begin{enumerate}
			\item Viết công thức của hàm số $T(x)$.
			      $$T(x)=\heva{&750000x&\text{nếu }1\le x\le 2\\
					      &1500000+500000(x-2)&\text{nếu }x\ge 3.}$$
			\item Tính $T(2)$, $T(4)$, $T(6)$ và cho biết ý nghĩa của mỗi giá trị này.
			      \begin{itemize}
				      \item $T(2)=750000\cdot 2=1500000$ đồng, khách thuê hai ngày đêm nên chi phí là $1500000$ đồng.
				      \item $T(4)=750000\cdot 2+500000(4-2)=2500000$ đồng, khách thuê bốn ngày đêm nên chi phí là $2500000$ đồng.
				      \item $T(6)=750000\cdot 2+500000(6-2)=3500000$ đồng, khách thuê sáu ngày đêm nên chi phí là $3500000$ đồng.
			      \end{itemize}
		\end{enumerate}
	}
\end{vd}
\baitaptl
%%==========Bài 1
\begin{bt}%[Nguyễn Cường- BG Toán 10]%[0D2T2-5]
	Một cửa hàng đồng loạt giảm giá các sản phẩm. Trong đó có chương trình nếu mua một gói kẹo thứ hai trở đi sẽ được giảm $10\%$ so với giá ban đầu là $50000$ đồng.
	\begin{enumerate}
		\item Nếu gọi số gói kẹo đã mua là $x$, số tiền phải trả là $y$. Hãy biểu diễn $y$ theo $x$.
		\item Bạn Thư muốn mua $10$ gói kẹo thì hết bao nhiêu tiền.
	\end{enumerate}

	\loigiai{
		\begin{enumerate}
			\item  Số tiền $y$ theo biến $x$ là $y = 90\%(x - 1)\cdot 50000 + 50000$.\\
			      Vậy $y = 45000x + 5000$.
			\item  Số tiền bạn Thư phải trả cho $10$ gói kẹo là
			      $y = 45000\cdot 10+5000= 455000$ đồng.
		\end{enumerate}
	}
\end{bt}

%%==========Bài 2
\begin{bt}%[Nguyễn Cường- BG Toán 10]%[0D2T2-5]
	Một cửa tiệm sách có một chính sách như sau: Nếu khách hàng đăng kí làm hội viên của cửa hàng thì mỗi năm phải đóng $50\;000$ đồng chi phí và được mướn sách với giá $5\;000$ đồng/cuốn, còn nếu khách hàng không phải hội viên thì sẽ mướn sách với giá $10\;000$ đồng/cuốn.
	Gọi $s$ (đồng) là tổng số tiền mỗi khách hàng phải trả trong một năm và $t$ là số cuốn sách mà khách hàng mướn.
	\begin{enumerate}
		\item Lập hàm số của s theo t đối với khách hàng là hội viên và đối với khách hàng không phải là hội viên.
		\item Trung là một hội viên của cửa hàng sách. Năm ngoái Trung đã trả cho cửa hàng sách tổng cộng $90\;000$ đồng. Hỏi nếu Trung không phải hội viên thì số tiền Trung phải trả là bao nhiêu?
	\end{enumerate}

	\loigiai{
		\begin{enumerate}
			\item Hàm số của $s$ theo $t$ đối với khách hàng là hội viên là $s=50\;000 + 5\;000t$.\\
			      Hàm số của $s$ theo $t$ đối với khách hàng không phải là hội viên là $s=10\;000t$.
			\item Thay $s=90\;000$ vào $s=50\;000 + 5\;000t$, ta được
			      \[50\;000 + 5\;000t = 90\;000 \Rightarrow t = 8.\]
			      Thay $t=8$ vào $s = 10\;000 t$, ta được
			      \[s = 10\;000 \cdot 8 = 80\;000.\]
			      Vậy nếu Trung không phải hội viên thì số tiền Trung phải trả là $80\;000$ đồng.
		\end{enumerate}

	}
\end{bt}

%%==========Bài 3
\begin{bt}%[Nguyễn Cường- BG Toán 10]%[0D2T2-5]
	Một người thuê nhà với giá $5000000$ đồng/tháng và người đó phải trả tiền dịch vụ giới thiệu là $1000000$ đồng (tiền dịch vụ chỉ trả $1$ lần). Gọi $x$ (tháng) là khoảng thời gian người đó thuê nhà, $y$ (đồng) là số tiền người đó phải tốn khi thuê nhà trong $x$ tháng.
	\begin{enumerate}
		\item Tìm một hệ thức liên hệ giữa $y$ và $x$.
		\item Tính số tiền người đó phải tốn sau khi ở $6$ tháng, $1$ năm.
	\end{enumerate}
	\loigiai{
		\begin{enumerate}
			\item Hệ thức liên hệ giữa $y$ và $x$ là $y = 5000000\cdot x + 1000000$.
			\item Số tiền mà người thuê nhà phải trả khi thuê nhà trong
			      \begin{itemize}
				      \item $6$ tháng: $y = 5000000\cdot 6 + 1000000 = 31000000$ (đồng).
				      \item $1$ năm: $y = 5000000\cdot 12 + 1000000 = 61000000$ (đồng).
			      \end{itemize}
		\end{enumerate}
	}
\end{bt}

%%==========Bài 4
\begin{bt}%[Nguyễn Cường- BG Toán 10]%[0D2T2-5]
	Một người vay ngân hàng $30~000~000$ (ba mươi triệu) đồng với lãi suất ngân hàng là $5 \%$ một năm và theo thể thức lãi đơn (tiền lãi không gộp vào chung với vốn).
	\begin{enumerate}
		\item Hãy thiết lập hàm số thể hiện mối liên hệ giữa tổng số tiền nợ $T$ (VNĐ) và số nợ (năm).
		\item Hãy cho biết sau 4 năm, người đó nợ ngân hàng tất cả bao nhiêu tiền?
	\end{enumerate}
	\loigiai{
		\begin{enumerate}
			\item Một người vay ngân hàng $30~000~000$ đồng với lãi suất $5\%$ một năm theo thể thức lãi đơn.
			      \begin{itemize}
				      \item Sau một năm người này nợ thêm: $30~000~000\cdot 5\%=1~500~000$ (đồng).
				      \item Sau $n$ năm người này nợ thêm: $1~500~000\cdot n$ (đồng).
			      \end{itemize}
			      Khi đó tổng số tiền người đó nợ sau $n$ năm là
			      \[1~500~000n+30~000~000\ (\text{đồng}). \]
			      Hàm số thể hiện mối liên hệ giữa tổng số tiền nợ $T$ (đồng) và số nợ $n$ (năm) là
			      \[T=1~500~000n+30~000~000. \]
			\item Thay $n=4$ vào công thức $T=1~500~000n+30~000~000$, ta được
			      \[T=1~500~000\cdot 4+30~000~000=36~000~000\ (\text{đồng}). \]
			      Vậy sau 4 năm người đó nợ $36~000~000$ đồng.
		\end{enumerate}
	}
\end{bt}


%%==========Bài 5
\begin{bt}%[Nguyễn Cường- BG Toán 10]%[0D2T2-5]
	Khách sạn A tại Đà Lạt có mức phí cho mỗi phòng được tính như sau: Mỗi phòng có giá là $300\ 000$ đồng/đêm, với thuế giá trị gia tăng là $8\%$. Do số lượng khách đến Đà Lạt vào dịp Tết tăng nhanh, khách sạn quyết định phụ thu thêm phí dịch vụ là $50\ 000$ đồng cho mỗi phòng và phí này chỉ thu một lần cố định.
	\begin{listEX}[1]
		\item Gọi $x$ là số đêm bạn An ở tại khách sạn A, $y$ là số tiền bạn An phải trả. Hãy viết biểu thức biểu diễn $y$ theo $x$.
		\item Biết bạn An phải trả tổng cộng $1\ 346\ 000$ đồng, hãy tính số đêm mà bạn An ở tại khách sạn A.
	\end{listEX}
	\dapso{a) $y=324\ 000x+50\ 000$, b) $4$ đêm.}
	\loigiai
	{
		\begin{enumerate}
			\item Số tiền bạn An phải trả là $y=300\ 000\cdot (100\%+8\%)x+50\ 000=324\ 000x+50\ 000$ đồng.
			\item Do bạn An phải trả tổng cộng $1\ 346\ 000$ nên $324\ 000x+50\ 000=1\ 346\ 000\Leftrightarrow x=4$.\\
			      Vậy An ở lại $4$ đêm.
		\end{enumerate}
	}
\end{bt}

%%==========Bài 6
\begin{bt}%[Nguyễn Cường- BG Toán 10]%[0D2T2-5]
	Một cửa hàng bán lại bánh $A$ như sau: nếu mua không quá $3$ hộp thì giá $35$ nghìn đồng mỗi hộp, nếu mua nhiều hơn $3$ hộp thì bắt đầu từ hộp thứ tư trở đi giá mỗi hộp sẽ giảm đi $20\%$ giá ban đầu.
	\begin{enumerate}
		\item Viết công thức tính $y$ (số tiền mua bánh) theo $x$ (số hộp bánh mua trong trường hợp nhiều hơn $3$ hộp).
		\item Lan và Hồng đều mua loại bánh $A$ với số hộp nhiều hơn $3$. Hỏi mỗi bạn mua bao nhiêu hộp biết rằng số hộp bánh Lan mua gấp đôi số hộp Hồng mua, đồng thời số tiền mua bánh của Lan nhiều hơn Hồng $140$ nghìn đồng.
	\end{enumerate}
	\loigiai
	{
		\begin{enumerate}
			\item Giá tiền mỗi hộp bánh khi giảm $20\%$ là $80\%\cdot 35\,000=28\,000$ đồng.\\
			      Giá tiền $3$ hộp bánh là $3\cdot 35\,000=105\,000$ đồng.\\
			      Công thức tính $y$ theo $x$ là \[y=28\,000(x-3)+105\,000\Leftrightarrow y=28000x-21000.\]
			\item Gọi $x$ (hộp) là số hộp bánh Hồng mua ($x>3$).\\
			      $2x$ (hộp) là số hộp bánh Lan mua.\\
			      Theo giải thiết, ta có
			      \allowdisplaybreaks
			      \begin{eqnarray*}
				      \left(28000\cdot 2x-21000\right)-\left(28000\cdot x-21000\right)=140000&\Leftrightarrow& 56000x-28000x=140000\\&\Leftrightarrow&28000x=140000\\
				      &\Leftrightarrow&x=5.
			      \end{eqnarray*}
			      Vậy số hộp bánh Hồng mua là $5$ hộp và số hộp bánh Lan mua là $10$ hộp.
		\end{enumerate}
	}
\end{bt}

%%==========Bài 7
\begin{bt}%[Nguyễn Cường- BG Toán 10]%[0D2T2-5]
	Một tiệm bánh có chương trình giảm $5\%$ trên tổng hóa đơn khi mua hàng chỉ trong ngày $09/01/2021$, bạn My mua $5$ hộp bánh bông lan cùng loại trong ngày $09/01/2021$, số tiền bạn phải trả là $37\,\, 250$ đồng. Ngày $12/01/2021$, bạn Uyên mua $6$ hộp bánh bông lan cùng loại với bạn My đã mua thì trả số tiền là $470\,\,000$ đồng. Biết số tiền phải trả (khi chưa có chương trình khuyến mãi) và số hộp bánh bông lan liên hệ bằng công thức: $y = ax + b$, $y$ (đồng) là số tiền phải trả và $x$ là số hộp bánh bông lan cùng loại.
	\begin{enumerate}
		\item Viết hàm số biểu diễn $y$ theo $x$.
		\item Hỏi vào ngày $12/01/2021$, bạn Nhân mua bao nhiêu hộp bánh bông lan cùng loại với bạn My? Biết số tiền Nhân trả là $320\,\,000$ đồng.
	\end{enumerate}
	\loigiai
	{
		\begin{enumerate}
			\item Bạn My mua $5$ hộp bánh bông lan cùng loại trong ngày $09/01/2021$, khi đó có chương trình khuyến mãi $5\%$ hóa đơn, số tiền bạn phải trả là $375\,\,250$ đồng nên ta có: $95\%(5a + b) = 375\,\,250$ hay $4{,}75a + 0{,}95b = 375\,\,250$. $\hfill (1)$\\
			      Ngày $12/01/2021$, bạn Uyên mua $6$ hộp bánh bông lan cùng loại với bạn Uyên thì trả số tiền là $470\,\,000$ đồng nên ta có: $6a + b = 470\,\,000$. $\hfill (2)$\\
			      Từ $(1)$ và $(2)$, ta có hệ phương trình
			      $$ \heva{&4{,}75a + 0{,}95b = 375\,\,250\\&6a + b = 470\,\,000} \Leftrightarrow \heva{&a = 75\,\,000\\&b = 20\,\,000.} $$
			      $\Rightarrow y = 75\,\,000x + 20\,\,000$.
			\item Bạn Nhân mua bánh vào ngày $12/01/2021$ nên không có chương trình khuyến mãi.\\
			      Vì bạn Nhân đã mua bánh hết $320\,\,000$ đồng nên $y = 320\,\,000$. Thay $y = 320\,\,000$ vào $y = 75\,\,000x + 20\,\,000$, ta được
			      $$ 320\,\,000 = 75\,\,000x + 20\,\,000 \Leftrightarrow x = 4. $$
			      Vậy bạn Nhân đã mua $4$ hộp.
		\end{enumerate}
	}
\end{bt}

%%==========Bài 8
\begin{bt}%[Nguyễn Cường- BG Toán 10]%[0D2T2-5]
	Nồng độ cồn trong máu $(BAC)$ được định nghĩa là phần trăm rượu (rượu ethyl hoặc ethanol) trong máu của một người. $BAC$ là $ 0{,}05 \%$ có nghĩa là có $0{,}05$ gam rượu trong $100 \mathrm{ml}$ máu. Càng uống nhiều rượu bia thì nồng độ cồn trong máu càng cao và càng nguy hiểm khi tham gia giao thông. Nồng độ $BAC$ trong máu của một người được thể hiện qua đồ thị sau:
	\begin{center}
		\begin{tikzpicture}[>=stealth, line join=round, line cap=round, font=\footnotesize, scale=1,yscale=.75]
			\draw[->] (-1,0)--(0,0)node[below left]{$O$}--(7,0)node[below]{$t$ (giờ)};
			\draw[->] (0,-1)--(0,5)node[left]{$BAC$ ($\%$)};
			\draw[dashed](3,0)node[below]{$1$}--(3,2)--(0,2)node[left]{$0{,}068$};
			\draw (0,3)node[left]{$0{,}076$}--(6,1);
			\fill[black](3,2)circle (1pt);
		\end{tikzpicture}
	\end{center}
	\begin{listEX}[1]
		\item	Viết công thức biểu thị mối quan hệ giữa nồng độ cồn trong máu $(BAC)$ sau $t$ giờ sử dụng.
		\item Theo nghị định 100/2019/ND-CP về xử phạt vi phạm hành chính, các mức phạt (đối với xe máy). Hỏi sau $3$ giờ, nếu người này tham gia giao thông thì sẽ bị xử phạt ở mức độ nào?
	\end{listEX}
	\begin{center}
		\begin{tabular}{|l|c|}
			\hline
			Mức $1$: Nồng độ cồn chưa vượt quá $50$mg/$100$ml máu        & Phạt tiền từ $02-03$ triệu đồng \\&(tước bằng từ $10-12$ tháng) \\
			\hline
			Mức $2$: Nồng độ cồn  vượt quá $50$mg đến $80$mg/$100$ml máu & Phạt tiền từ $04-05$ triệu đồng \\&(tước bằng từ $16-18$ tháng)\\
			\hline
			Mức $3$: Nồng độ cồn vượt quá $80$mg/$100$ml máu             & Phạt tiền từ $06-08$ triệu đồng \\&(tước bằng từ $22-24$ tháng) \\
			\hline
		\end{tabular}
	\end{center}
	\loigiai{
		\begin{enumerate}
			\item Dựa vào đồ thị ta gọi công thức biểu thị mối liên hệ giữa nồng độ cồn trong máu $(BAC)$ sau $t$ giờ sử dụng có công thức $BAC=at+b$.\\
			      Từ đồ thị ta có hàm số đi qua các điểm $(0;0{,}076)$ và $(1;0{,}068)$ nên ta được
			      $\heva{& BAC=0{,}076 \\ & a=-\dfrac{1}{125}.}$\\
			      Công thức biểu thị mối quan hệ giữa nồng độ cồn trong máu $(BAC)$ sau $t$ giờ sử dụng là $BAC=-\dfrac{1}{125}t+0{,}076$.
			\item Nồng độ cồn trong máu sau $3$ giờ là $BAC=-\dfrac{1}{125}\cdot 3+0{,}076=0{,}052$.\\
			      Do nồng độ cồn trong máu sau $3$ giờ là $0{,}052$mg/$100$ml máu nằm ở mức $2$ nên người này bị phạt tiền từ $04-05$ triệu đồng  và tước bằng từ $16-18$ tháng.
		\end{enumerate}
	}
\end{bt}

%%==========Bài 9
\begin{bt}%[Nguyễn Cường- BG Toán 10]%[0D2T2-5]
	Một cửa hàng cho thuê sách cũ có quy định: Nếu khách hàng là hội viên của cửa hàng thì phải đóng phí $70000$ đồng/năm và được thuê sách với giá $6000$ đồng/quyển, còn nếu khách hàng không là hội viên phải thuê sách với giá $10000$ đồng/quyển. Gọi $y$ (đồng) là tổng số tiền khách hàng phải trả trong một năm và $x$ là số quyển sách thuê trong một năm.
	\begin{listEX}
		\item Lập hàm số của $y$ theo $x$ với khách hàng là hội viên và với khách hàng không là hội viên của cửa hàng.
		\item Anh Nam là một hội viên của cửa hàng, năm vừa rồi anh Nam trả cho cửa hàng tổng cộng
		$322000$ đồng. Hỏi nếu anh Nam không là hội viên của cửa hàng thì năm vừa rồi anh phải trả
		cho cửa hàng bao nhiêu tiền?
	\end{listEX}
	\loigiai
	{
		\begin{enumerate}
			\item Đối với khách hàng hội viên ta có $y=70000+6000x$.\\
			      Đối với khách hàng không hội viên ta có $y=10000x$.
			\item Thế $y=322000$ vào $y=70000+6000x$, ta có $320000=70000+6000x \Leftrightarrow x=42$.\\
			      Thế $x=42$ vào $y=10000x$, ta có $y=420000$.\\
			      Vậy năm vừa rồi nếu không là hội viên anh Nam phải trả $420000$ đồng.
		\end{enumerate}
	}
\end{bt}

%%==========Bài 10
\begin{bt}%[Nguyễn Cường- BG Toán 10]%[0D2T2-5]
	Bạn Bình muốn mua một đôi giày thể thao mới. Hiện tại bạn đang có sẵn một số tiền nhưng không đủ để mua. Vì vậy bạn lên kế hoạch tiết kiệm tiền từ ngày 01/02/2020 đến ngày
	31/03/2020. Tháng Tư, Bình rủ An đến cửa hàng để mua giày. Sau khi mua giày xong, Bình mua hai thêm hai ly trà sữa với giá $30000$ đồng một ly thì Bình còn dư lại $60000$ đồng. Gọi $y$ (đồng) là số tiền bạn Bình có sẵn, $x$ (đồng) là số tiền bạn để dành mỗi ngày từ 01/02/2020 đến 31/03/2020.
	\begin{enumerate}
		\item Lập hàm số $y$ theo $x$ biết giá đôi giày bạn mua là $680000$ đồng.
		\item Biết số tiền bạn Bình có sẵn do ông bà lì xì Tết là $200000$ đồng. Hỏi để có tiền mua giày thì mỗi ngày Bình phải tiết kiệm bao nhiêu tiền?
	\end{enumerate}
	\loigiai{
		\begin{enumerate}
			\item Từ ngày 01/02/2020 đến 31/03/2020 có $60$ ngày, nên ta có\\
			      $y+60x=680000+2\cdot30000+60000\Rightarrow y=800000-60x$.
			\item Ta có $y=200000$ (đồng), mà $x=\dfrac{800000-y}{60}$ $\Rightarrow x=\dfrac{800000-20000}{60}=10000$ (đồng).\\
			      Vậy mỗi ngày Bình phải tiết kiệm $10000$ đồng để có tiền mua giày.
		\end{enumerate}
	}
\end{bt}


\subsection{Câu hỏi trắc nghiệm}

\Opensolutionfile{ansbook}[ans/ansbook-0D1-1-TN]
\Opensolutionfile{ans}[ans/ans-0D1-1-TN]
\begin{ex}%[Phan Anh]%[0D2B1-1]
	Điểm nào sau đây thuộc đồ thị hàm số $y=\dfrac{1}{x-1}$?
	\choice
	{\True $M_1(2;1)$}
	{$M_2(1;1)$}
	{$M_3(2;0)$}
	{$M_4(0;-2)$}
	\loigiai{
		Xét điểm $M_1$, thay $x=2$ và $y=1$
		vào hàm số $y=\dfrac{1}{x-1}$ ta được $1=\dfrac{1}{2-1}$ ta thấy đúng nên nhận $M_1$.}
\end{ex}
\begin{ex}%[Phan Anh]%[0D2B1-1]
	Điểm nào sau đây \textbf{không} thuộc đồ thị hàm số $y=\dfrac{\sqrt{x^2-4x+4}}{x}$?
	\choice
	{$A\left(2;0\right)$}
	{$B\left(3;\dfrac{1}{3}\right)$}
	{\True $C\left(1;-1\right)$}
	{$D\left(-1;-3\right)$}
	\loigiai{Thay từng đáp án vào hàm số $y=\dfrac{\sqrt{x^2-4x+4}}{x}$.
		\begin{itemize}
			\item Với $x=2$ và $y=0$, ta được $0=\dfrac{\sqrt{2^2-4.2+4}}{2}$ (đúng).
			\item Với $x=3$ và $y=\dfrac{1}{3}$, ta được $\dfrac{1}{3}=\dfrac{\sqrt{3^2-4\cdot3+4}}{3}$ (đúng).
			\item Với thay $x=1$ và $y=-1$, ta được $-1=\dfrac{\sqrt{1^2-4\cdot1+4}}{1}\Leftrightarrow-1=1$ (sai).
		\end{itemize}}
\end{ex}
\begin{ex}%[Phan Anh]%[0D2B1-1]
	Cho hàm số $y=f(x)=|-5x|$. Khẳng định nào sau đây là \textbf{sai}?
	\choice
	{$f(-1)=5$}
	{$f(2)=10$}
	{$f(-2)=10$}
	{\True $f\left(\dfrac{1}{5}\right)=-1$}
	\loigiai{Ta có
		\begin{itemize}
			\item $f(-1)=|-5\cdot(-1)|=|5|=5$.
			\item $f(2)=|-5\cdot2|=|-10|=10$.
			\item $f(-2)=|-5\cdot(-2)|=|10|=10$.
			\item $f\left(\dfrac{1}{5}\right)=\left|-5\cdot\dfrac{1}{5}\right|=|-1|=1$
		\end{itemize}
		Cách khác: Vì hàm đã cho là hàm trị tuyệt đối nên không âm. Do đó $f\left(\dfrac{1}{5}\right)=-1$ là sai.}
\end{ex}
\begin{ex}%[Phan Anh]%[0D2B1-1]
	Cho hàm số $f(x)=\left\{\begin{array}{*{35}{l}}
			\dfrac{2}{x-1} & , x\in(-\infty;0) \\
			\sqrt{x+1}     & , x\in[0;2]       \\
			x^2-1          & , x\in(2;5]
		\end{array}\right.$. Tính giá trị của $f(4)$.
	\choice
	{$f(4)=\dfrac{2}{3}$}
	{\True $f(4)=15$}
	{$f(4)=\sqrt{5}$}
	{Không tính được}
	\loigiai{Do $4\in(2;5]$ nên $f(4)=4^2-1=15$.}
\end{ex}
\begin{ex}%[Phan Anh]%[0D2B1-1]
	Cho hàm số $f(x)=\left\{\begin{array}{*{35}{l}}
			\dfrac{2\sqrt{x+2}-3}{x-1} & , x\ge 2 \\
			x^2 +1                     & , x<2
		\end{array}\right.$. Tính $P=f(2)+f(-2)$.
	\choice
	{$P=\dfrac{8}{3}$}
	{$P=4$}
	{\True $P=6$}
	{$P=\dfrac{5}{3}$}
	\loigiai{\begin{itemize}
			\item Khi $x\ge 2$ thì $f(2)=\dfrac{2\sqrt{2+2}-3}{2-1}=1$.
			\item Khi $x<2$ thì $f(-2)=(-2)^2+1=5$.
		\end{itemize}
		Vậy $f(2)+f(-2)=6$.}
\end{ex}
\begin{ex}%[Phan Anh]%[0D2B1-2]
	Tìm tập xác định $\mathscr{D}$ của hàm số $y=\dfrac{3x-1}{2x-2}$.
	\choice
	{$\mathscr{D}=\mathbb{R}$}
	{$\mathscr{D}=(1;+\infty)$}
	{\True $\mathscr{D}=\mathbb{R}\setminus\{1\}$}
	{$\mathscr{D}=[1;+\infty)$}
	\loigiai{
		Hàm số xác định khi $2x-2\ne0\Leftrightarrow x\ne1$.\\
		Vậy tập xác định của hàm số là $\mathscr{D}=\mathbb{R}\setminus\{1\}$.}
\end{ex}
\begin{ex}%[Phan Anh]%[0D2B1-2]
	Tìm tập xác định $\mathscr{D}$ của hàm số $y=\dfrac{2x-1}{(2x+1)(x-3)}$.
	\choice
	{$\mathscr{D}=(3;+\infty)$}
	{\True $\mathscr{D}=\mathbb{R}\setminus\left\{-\dfrac{1}{2};3\right\}$}
	{$\mathscr{D}=\left(-\dfrac{1}{2};+\infty\right)$}
	{$\mathscr{D}=\mathbb{R}$}
	\loigiai{
		Hàm số xác định khi $\heva{
				& 2x+1\ne 0 \\
				& x-3\ne 0}\Leftrightarrow \heva{
				& x\ne-\dfrac{1}{2} \\
				& x\ne 3.}$\\
		Vậy tập xác định của hàm số là $ \mathscr{D}=\mathbb{R}\setminus\left\{-\dfrac{1}{2};3\right\}$}
\end{ex}
\begin{ex}%[Phan Anh]%[0D2B1-2]
	Tìm tập xác định $\mathscr{D}$ của hàm số $y=\dfrac{x^2+1}{x^2+3x-4}$.
	\choice
	{$\mathscr{D}=\{1;-4\}$}
	{\True $\mathscr{D}=\mathbb{R}\setminus\{1;-4\}$}
	{$\mathscr{D}=\mathbb{R}\setminus\{1;4\}$}
	{$\mathscr{D}=\mathbb{R}$}
	\loigiai{
		Hàm số xác định khi $x^2+3x-4\ne 0\Leftrightarrow \heva{
				& x\ne 1 \\
				& x\ne-4}.$\\
		Vậy tập xác định của hàm số là $\mathscr{D}=\mathbb{R}\setminus\{1;-4\}$.}
\end{ex}
\begin{ex}%[Phan Anh]%[0D2B1-2]
	Tìm tập xác định $\mathscr{D}$ của hàm số $y=\dfrac{x+1}{(x+1)(x^2+3x+4)}$.
	\choice
	{$\mathscr{D}=\mathbb{R}\setminus\left\{1\right\}$}
	{$\mathscr{D}=\left\{-1\right\}$}
	{\True $\mathscr{D}=\mathbb{R}\setminus\left\{-1\right\}$}
	{$\mathscr{D}=\mathbb{R}$}
	\loigiai{
		Hàm số xác định khi $\heva{
				& x+1\ne 0 \\
				& x^2+3x+4\ne 0}\Leftrightarrow x\ne-1$.\\
		Vậy tập xác định của hàm số là $\mathscr{D}=\mathbb{R}\setminus\left\{-1\right\}$.}
\end{ex}
\begin{ex}%[Phan Anh]%[0D2B1-2]
	Tìm tập xác định $\mathscr{D}$ của hàm số $y=\dfrac{2x+1}{x^3-3x+2}$.
	\choice
	{$\mathscr{D}=\mathbb{R}\setminus\left\{1;2\right\}$}
	{\True $\mathscr{D}=\mathbb{R}\setminus\left\{-2;1\right\}$}
	{$\mathscr{D}=\mathbb{R}\setminus\left\{-2\right\}$}
	{$\mathscr{D}=\mathbb{R}$}
	\loigiai{
		Hàm số xác định khi
		\begin{align*}
			                & x^3-3x+2\ne 0
			\Leftrightarrow (x-1)(x^2+x-2)\ne 0                         \\
			\Leftrightarrow & \,\heva{                       & x-1\ne 0 \\
			                & x^2+x-2\ne 0}
			\Leftrightarrow \heva{
			                & x\ne 1                                    \\ & \heva{ & x\ne 1 \\
			                & x\ne-2}}\Leftrightarrow \heva{
			                & x\ne                                      \\
			                & x\ne-2.}
		\end{align*}
		Vậy tập xác định của hàm số là $\mathscr{D}=\mathbb{R}\setminus\left\{-2;1\right\}$.
	}
\end{ex}
\begin{ex}%[Phan Anh]%[0D2B1-2]
	Tìm tập xác định $\mathscr{D}$ của hàm số $y=\sqrt{x+2}-\sqrt{x+3}$.
	\choice
	{$\mathscr{D}=[-3;+\infty)$}
	{\True $\mathscr{D}=\left[-2;+\infty \right)$}
	{$\mathscr{D}=\mathbb{R}$}
	{$\mathscr{D}=\left[2;+\infty \right)$}
	\loigiai{
	Hàm số xác định khi $\heva{
			& x+2\ge 0 \\
			& x+3\ge 0 \\}\Leftrightarrow \heva{
			& x\ge-2 \\
			& x\ge-3 \\}\Leftrightarrow x\ge-2$.\\
	Vậy tập xác định của hàm số là $\mathscr{D}=\left[-2;+\infty \right)$.}
\end{ex}
\begin{ex}%[Phan Anh]%[0D2B1-2]
	Tìm tập xác định $\mathscr{D}$ của hàm số $y=\sqrt{6-3x}-\sqrt{x-1}$.
	\choice
	{$\mathscr{D}=\left(1;2\right)$}
	{\True $\mathscr{D}=\left[1;2\right]$}
	{$\mathscr{D}=\left[1;3\right]$}
	{$\mathscr{D}=\left[-1;2\right]$}
	\loigiai{
		Hàm số xác định khi $\heva{
				& 6-3x\ge 0 \\
				& x-1\ge 0}\Leftrightarrow \heva{
				& x\le 2 \\
				& x\ge 1}\Leftrightarrow 1\le x\le 2$.\\
		Vậy tập xác định của hàm số là $\mathscr{D}=\left[1;2\right]$.}
\end{ex}
\begin{ex}%[Phan Anh]%[0D2B1-2]
	Tìm tập xác định $\mathscr{D}$ của hàm số $y=\dfrac{\sqrt{3x-2}+6x}{\sqrt{4-3x}}$.
	\choice
	{\True $\mathscr{D}=\left[\dfrac{2}{3};\dfrac{4}{3}\right)$}
	{$\mathscr{D}=\left[\dfrac{3}{2};\dfrac{4}{3}\right)$}
	{$\mathscr{D}=\left[\dfrac{2}{3};\dfrac{3}{4}\right)$}
	{$\mathscr{D}=\left(-\infty;\dfrac{4}{3}\right)$}
	\loigiai{
	Hàm số xác định khi $\heva{
			& 3x-2\ge 0 \\
			& 4-3x>0}\Leftrightarrow \heva{
			& x\ge \dfrac{2}{3} \\
			& x<\dfrac{4}{3}}\Leftrightarrow \dfrac{2}{3}\le x<\dfrac{4}{3}$.\\
	Vậy tập xác định của hàm số là $\mathscr{D}=\left[\dfrac{2}{3};\dfrac{4}{3}\right)$.}
\end{ex}
\begin{ex}%[Phan Anh]%[0D2B1-2]
	Tìm tập xác định $\mathscr{D}$ của hàm số $y=\dfrac{x+4}{\sqrt{x^2-16}}$.
	\choice
	{$\mathscr{D}=\left(-\infty;-2\right)\cup \left(2;+\infty \right)$}
	{$\mathscr{D}=\mathbb{R}$}
	{\True $\mathscr{D}=\left(-\infty;-4\right)\cup \left(4;+\infty \right)$}
	{$\mathscr{D}=\left(-4;4\right)$}
	\loigiai{Hàm số xác định khi $x^2-16>0\Leftrightarrow x^2>16\Leftrightarrow \hoac{
				& x>4 \\
				& x<-4}$.\\
		Vậy tập xác định của hàm số là $\mathscr{D}=\left(-\infty;-4\right)\cup \left(4;+\infty \right)$.}
\end{ex}
\begin{ex}%[Phan Anh]%[0D2B1-2]
	Tìm tập xác định $\mathscr{D}$ của hàm số $y=\sqrt{x^2-2x+1}+\sqrt{x-3}$.
	\choice
	{$\mathscr{D}=(-\infty;3]$}
	{$\mathscr{D}=[1;3]$}
	{\True $\mathscr{D}=[3;+\infty)$}
	{$\mathscr{D}=(3;+\infty)$}
	\loigiai{
	Hàm số xác định khi $\heva{
			& x^2-2x+1\ge 0 \\
			& x-3\ge 0}\Leftrightarrow \heva{
			& {\left(x-1\right)}^2\ge 0 \\
			& x-3\ge 0}\Leftrightarrow \heva{
			& x\in \mathbb{R} \\
			& x\ge 3}\Leftrightarrow x\ge 3$.\\
	Vậy tập xác định của hàm số là $\mathscr{D}=\left[3;+\infty \right)$.}
\end{ex}
\begin{ex}%[Phan Anh]%[0D2B1-2]
	Tìm tập xác định $\mathscr{D}$ của hàm số $y=\dfrac{\sqrt{2-x}+\sqrt{x+2}}{x}$.
	\choice
	{$\mathscr{D}=[-2;2]$}
	{$\mathscr{D}=(-2;2)\setminus\left\{0\right\}$}
	{\True $\mathscr{D}=[-2;2]\setminus\left\{0\right\}$}
	{$\mathscr{D}=\mathbb{R}$}
	\loigiai{
		Hàm số xác định khi $\heva{
				& 2-x\ge 0 \\
				& x+2\ge 0 \\
				& x\ne 0}\Leftrightarrow \heva{
				& x\le 2 \\
				& x\ge-2 \\
				& x\ne 0.}$\\
		Vậy tập xác định của hàm số là $\mathscr{D}=\left[-2;2\right]\setminus\left\{0\right\}$.}
\end{ex}
\begin{ex}%[Phan Anh]%[0D2B1-2]
	Tìm tập xác định $\mathscr{D}$ của hàm số $y=\dfrac{\sqrt{x+1}}{x^2-x-6}$.
	\choice
	{$\mathscr{D}=\left\{3\right\}$}
	{\True $\mathscr{D}=\left[-1;+\infty \right)\setminus\left\{3\right\}$}
	{$\mathscr{D}=\mathbb{R}$}
	{$\mathscr{D}=\left[-1;+\infty \right)$}
	\loigiai{
	Hàm số xác định khi $\heva{
			& x+1\ge 0 \\
			& x^2-x-6\ne 0}\Leftrightarrow \heva{
			& x\ge-1 \\
			& x\ne 3 \\
			& x\ne-2}\Leftrightarrow \heva{
			& x\ge-1 \\
			& x\ne 3.}$\\
	Vậy tập xác định của hàm số là $\mathscr{D}=[-1;+\infty)\setminus\left\{3\right\}$.}
\end{ex}
\begin{ex}%[Phan Anh]%[0D2B1-2]
	Tìm tập xác định $\mathscr{D}$ của hàm số $y=\sqrt{6-x}+\dfrac{2x+1}{1+\sqrt{x-1}}$.
	\choice
	{$\mathscr{D}=(1;+\infty)$}
	{\True $\mathscr{D}=[1;6]$}
	{$\mathscr{D}=\mathbb{R}$}
	{$\mathscr{D}=(1;6)$}
	\loigiai{
	Hàm số xác định khi $\heva{
			& 6-x\ge 0 \\
			& x-1\ge 0 \\
			& 1+\sqrt{x-1}\ne 0\left(\text{luôn đúng} \right)}\Leftrightarrow \heva{
			& x\le 6 \\
			& x\ge 1}\Leftrightarrow 1\le x\le 6$.\\
	Vậy tập xác định của hàm số là $\mathscr{D}=[1;6]$.}
\end{ex}
\begin{ex}%[Phan Anh]%[0D2B1-2]
	Tìm tập xác định $\mathscr{D}$ của hàm số $y=\dfrac{x+1}{(x-3)\sqrt{2x-1}}$.
	\choice
	{$\mathscr{D}=\mathbb{R}$}
	{$\mathscr{D}=\left(-\dfrac{1}{2};+\infty \right)\setminus\left\{3\right\}$}
	{$\mathscr{D}=\left[\dfrac{1}{2};+\infty \right)\setminus\left\{3\right\}$}
	{\True $\mathscr{D}=\left(\dfrac{1}{2};+\infty \right)\setminus\left\{3\right\}$}
	\loigiai{
		Hàm số xác định khi $\heva{
				& x-3\ne 0 \\
				& 2x-1>0}\Leftrightarrow \heva{
				& x\ne 3 \\
				& x>\dfrac{1}{2}.}$\\
		Vậy tập xác định của hàm số là $\mathscr{D}=\left(\dfrac{1}{2};+\infty \right)\setminus\left\{3\right\}$.}
\end{ex}
\begin{ex}%[Phan Anh]%[0D2B1-2]
	Tìm tập xác định $\mathscr{D}$ của hàm số $y=\dfrac{\sqrt{x+2}}{x\sqrt{x^2-4x+4}}$.
	\choice
	{\True $\mathscr{D}=[-2;+\infty)\setminus\left\{0;2\right\}$}
	{$\mathscr{D}=\mathbb{R}$}
	{$\mathscr{D}=[-2;+\infty)$}
	{$\mathscr{D}=(-2;+\infty)\setminus\left\{0;2\right\}$}
	\loigiai{
	Hàm số xác định khi $\heva{
			& x+2\ge 0 \\
			& x\ne 0 \\
			& x^2-4x+4>0}\Leftrightarrow \heva{
			& x+2\ge 0 \\
			& x\ne 0 \\
			& (x-2)^2>0}\Leftrightarrow \heva{
			& x\ge-2 \\
			& x\ne 0 \\
			& x\ne 2.}$\\
	Vậy tập xác định của hàm số là $\mathscr{D}=\left[-2;+\infty \right)\setminus\left\{0;2\right\}$.}
\end{ex}
%Câu 21
\begin{ex}%[Phan Anh]%[0D2B1-2]
	Tìm tập xác định $\mathscr{D}$ của hàm số $y=\dfrac{x}{x-\sqrt{x}-6}$.
	\choice
	{$\mathscr{D}=\left[0;+\infty \right)\setminus\left\{3\right\}$}
	{\True $\mathscr{D}=\left[0;+\infty \right)\setminus\left\{9\right\}$}
	{$\mathscr{D}=\left[0;+\infty \right)\setminus\left\{\sqrt{3}\right\}$}
	{$\mathscr{D}=\mathbb{R}\setminus\left\{9\right\}$}
	\loigiai{
	Hàm số xác định khi $\heva{
			& x\ge 0 \\
			& x-\sqrt{x}-6\ne 0}\Leftrightarrow \heva{
			& x\ge 0 \\
			& \sqrt{x}\ne 3}\Leftrightarrow \heva{
			& x\ge 0 \\
			& x\ne 9.}$\\
	Vậy tập xác định của hàm số là $\mathscr{D}=\left[0;+\infty \right)\setminus\left\{9\right\}$.}
\end{ex}
\begin{ex}%[Phan Anh]%[0D2B1-2]
	Tìm tập xác định $\mathscr{D}$ của hàm số $y=\dfrac{\sqrt[3]{x-1}}{x^2+x+1}$.
	\choice
	{$\mathscr{D}=\left(1;+\infty \right)$}
	{$\mathscr{D}=\left\{1\right\}$}
	{\True $\mathscr{D}=\mathbb{R}$}
	{$\mathscr{D}=\left(-1;+\infty \right)$}
	\loigiai{
		Hàm số xác định khi $x^2+x+1\ne 0$ luôn đúng với mọi $x\in \mathbb{R}$.\\
		Vậy tập xác định của hàm số là $\mathscr{D}=\mathbb{R}$.}
\end{ex}
\begin{ex}%[Phan Anh]%[0D2B1-2]
	Tìm tập xác định $\mathscr{D}$ của hàm số $y=\dfrac{\sqrt{x-1}+\sqrt{4-x}}{\left(x-2\right)\left(x-3\right)}$.
	\choice
	{$\mathscr{D}=\left[1;4\right]$}
	{$\mathscr{D}=\left(1;4\right)\setminus\left\{2;3\right\}$}
	{\True $\mathscr{D}=\left[1;4\right]\setminus\left\{2;3\right\}$}
	{$\mathscr{D}=\left(-\infty;1\right]\cup \left[4;+\infty \right)$}
	\loigiai{
		Hàm số xác định khi $\heva{
				& x-1\ge 0 \\
				& 4-x\ge 0 \\
				& x-2\ne 0 \\
				& x-3\ne 0}\Leftrightarrow \heva{
				& x\ge 1 \\
				& x\le 4 \\
				& x\ne 2 \\
				& x\ne 3}\Leftrightarrow \heva{
				& 1\le x\le 4 \\
				& x\ne 2 \\
				& x\ne 3.}$\\
		Vậy tập xác định của hàm số là $\mathscr{D}=\left[1;4\right]\setminus\left\{2;3\right\}$.}
\end{ex}
\begin{ex}%[Phan Anh]%[0D2B1-2]
	Tìm tập xác định $\mathscr{D}$ của hàm số $y=\sqrt{\sqrt{x^2+2x+2}-(x+1)}$.
	\choice
	{$\mathscr{D}=\left(-\infty;-1\right)$}
	{$\mathscr{D}=\left[-1;+\infty \right)$}
	{$\mathscr{D}=\mathbb{R}\setminus\left\{-1\right\}$}
	{\True $\mathscr{D}=\mathbb{R}$}
	\loigiai{
		Hàm số xác định khi $\begin{aligned}[t]
				                & \sqrt{x^2+2x+2}-(x+1)\ge 0\Leftrightarrow \sqrt{(x+1)^2+1}\ge x+1 \\
				\Leftrightarrow & \, \hoac{
				                & \heva{
				                & x+1<0                                                             \\
				                & (x+1)^2+1\ge 0}                                                   \\
				                & \heva{
				                & x+1\ge 0                                                          \\
				                & (x+1)^2+1\ge(x+1)^2}}\Leftrightarrow \hoac{
				                & x+1<0                                                             \\
				                & x+1\ge 0}\Leftrightarrow x\in \mathbb{R}.
			\end{aligned}$\\
		Vậy tập xác định của hàm số là $\mathscr{D}=\mathbb{R}$.}
\end{ex}
\begin{ex}%[Phan Anh]%[0D2B1-2]
	Tìm tập xác định $\mathscr{D}$ của hàm số $y=\dfrac{2018}{\sqrt[3]{x^2-3x+2}-\sqrt[3]{x^2-7}}$.
	\choice
	{\True $\mathscr{D}=\mathbb{R}\setminus\left\{3\right\}$}
	{$\mathscr{D}=\mathbb{R}$}
	{$\mathscr{D}=\left(-\infty;1\right)\cup \left(2;+\infty \right)$}
	{$\mathscr{D}=\mathbb{R}\setminus\left\{0\right\}$}
	\loigiai{
		Hàm số xác định khi $\begin{aligned}[t]
				                & \sqrt[3]{x^2-3x+2}-\sqrt[3]{x^2-7}\ne 0\Leftrightarrow \sqrt[3]{x^2-3x+2}\ne \sqrt[3]{x^2-7} \\
				\Leftrightarrow & \,x^2-3x+2\ne x^2-7\Leftrightarrow 9\ne 3x\Leftrightarrow x\ne 3.
			\end{aligned}$\\
		Vậy tập xác định của hàm số là $\mathscr{D}=\mathbb{R}\setminus\left\{3\right\}$.}
\end{ex}
\begin{ex}%[Phan Anh]%[0D2K1-2]
	Tìm tập xác định $\mathscr{D}$ của hàm số $y=\dfrac{|x|}{|x-2|+\left|x^2+2x\right|}$.
	\choice
	{\True $\mathscr{D}=\mathbb{R}$}
	{$\mathscr{D}=\mathbb{R}\setminus\left\{-2;0\right\}$}
	{$\mathscr{D}=\mathbb{R}\setminus\left\{-2;0;2\right\}$}
	{$\mathscr{D}=\left(2;+\infty \right)$}
	\loigiai{
		Hàm số xác định khi $|x-2|+\left|x^2+2x\right|\ne0$.\\
		Xét phương trình $|x-2|+\left|x^2+2x\right|=0\Leftrightarrow \heva{
				& |x-2|=0 \\
				& \left|x^2+2x\right|=0}\Leftrightarrow \heva{
				& x=2 \\
				& x=0\vee x=-2.}$\\
		Vậy không có giá trị $x$ làm cho $|x-2|+\left| x^2+2x\right|=0$, do đó $|x-2|+\left| x^2+2x\right|\ne 0$ đúng với mọi $x\in \mathbb{R}$. Vậy tập xác định của hàm số là $\mathscr{D}=\mathbb{R}$.}
\end{ex}
\begin{ex}%[Phan Anh]%[0D2K1-2]
	Tìm tập xác định $\mathscr{D}$ của hàm số $y=\dfrac{2x-1}{\sqrt{x|x-4|}}$.
	\choice
	{$\mathscr{D}=\mathbb{R}\setminus\left\{0;4\right\}$}
	{$\mathscr{D}=\left(0;+\infty \right)$}
	{$\mathscr{D}=\left[0;+\infty \right)\setminus\left\{4\right\}$}
	{\True $\mathscr{D}=\left(0;+\infty \right)\setminus\left\{4\right\}$}
	\loigiai{
		Hàm số xác định khi $x|x-4|>0\Leftrightarrow \heva{
				& \left| x-4\right|\ne 0 \\
				& x>0}\Leftrightarrow \heva{
				& x\ne 4 \\
				& x>0.}$\\
		Vậy tập xác định của hàm số là $\mathscr{D}=\left(0;+\infty \right)\setminus\left\{4\right\}$.}
\end{ex}
\begin{ex}%[Phan Anh]%[0D2K1-2]
	Tìm tập xác định $\mathscr{D}$ của hàm số $y=\dfrac{\sqrt{5-3\left| x\right|}}{x^2+4x+3}$.
	\choice
	{\True $\mathscr{D}=\left[-\dfrac{5}{3};\dfrac{5}{3}\right]\setminus\left\{-1\right\}$}
	{$\mathscr{D}=\mathbb{R}$}
	{$\mathscr{D}=\left(-\dfrac{5}{3};\dfrac{5}{3}\right)\setminus\left\{-1\right\}$}
	{$\mathscr{D}=\left[-\dfrac{5}{3};\dfrac{5}{3}\right]$}
	\loigiai{
		Hàm số xác định khi $\heva{
				& 5-3\left| x\right|\ge 0 \\
				& x^2+4x+3\ne 0}\Leftrightarrow \heva{
				& \left| x\right|\le \dfrac{5}{3} \\
				& x\ne-1 \\
				& x\ne-3}\Leftrightarrow \heva{
				&-\dfrac{5}{3}\le x\le \dfrac{5}{3} \\
				& x\ne-1 \\
				& x\ne-3}\Leftrightarrow \heva{
				&-\dfrac{5}{3}\le x\le \dfrac{5}{3} \\
				& x\ne-1.}$\\
		Vậy tập xác định của hàm số là $\mathscr{D}=\left[-\dfrac{5}{3};\dfrac{5}{3}\right]\setminus\left\{-1\right\}$.}
\end{ex}
\begin{ex}%[Phan Anh]%[0D2K1-2]
	Tìm tập xác định $\mathscr{D}$ của hàm số $f(x)=\left\{\begin{array}{*{35}{l}}
			\dfrac{1}{2-x} & ;x\ge 1 \\
			\sqrt{2-x}     & ;x<1.
		\end{array}\right.$
	\choice
	{$\mathscr{D}=\mathbb{R}$}
	{$\mathscr{D}=\left(2;+\infty \right)$}
	{$\mathscr{D}=\left(-\infty;2\right)$}
	{\True $\mathscr{D}=\mathbb{R}\setminus\left\{2\right\}$}
	\loigiai{
		Hàm số xác định khi $\hoac{
				& \heva{
					& x\ge 1 \\
					& 2-x\ne 0} \\
				& \heva{
					& x<1 \\
					& 2-x\ge 0}}\Leftrightarrow \hoac{
				& \heva{
					& x\ge 1 \\
					& x\ne 2} \\
				& \heva{
					& x<1 \\
					& x\le 2}}\Leftrightarrow \hoac{
				& \heva{
					& x\ge 1 \\
					& x\ne 2} \\
				& x<1.}$\\
		Vậy xác định của hàm số là $\mathscr{D}=\mathbb{R}\setminus\left\{2\right\}$.}
\end{ex}
\begin{ex}%[Phan Anh]%[0D2K1-2]
	Tìm tập xác định $\mathscr{D}$ của hàm số $f(x)=\left\{\begin{array}{*{35}{l}}
			\dfrac{1}{x} & ;x\ge 1 \\
			\sqrt{x+1}   & ;x<1.
		\end{array}\right.$
	\choice
	{$\mathscr{D}=\left\{-1\right\}$}
	{$\mathscr{D}=\mathbb{R}$}
	{\True $\mathscr{D}=\left[-1;+\infty \right)$}
	{$\mathscr{D}=\left[-1;1\right)$}
	\loigiai{
	Hàm số xác định khi $\hoac{
			& \heva{
				& x\ge 1 \\
				& x\ne 0} \\
			& \heva{
				& x<1 \\
				& x+1\ge 0}}\Leftrightarrow \hoac{
			& x\ge 1 \\
			& \heva{
				& x<1 \\
				& x\ge-1.}}$\\
	Vậy xác định của hàm số là $\mathscr{D}=\left[-1;+\infty \right)$.}
\end{ex}
\begin{ex}%[Phan Anh]%[0D2K1-2]
	Tìm tất cả các giá trị thực của tham số $m$ để hàm số $y=\sqrt{x-m+1}+\dfrac{2x}{\sqrt{-x+2m}}$ xác định trên khoảng $(-1;3)$.
	\choice
	{\True Không có giá trị $m$ thỏa mãn}
	{$m\ge 2$}
	{$m\ge 3$}
	{$m\ge 1$}
	\loigiai{
	Hàm số xác định khi $\heva{
			& x-m+1\ge 0 \\
			&-x+2m>0}\Leftrightarrow \heva{
			& x\ge m-1 \\
			& x<2m.}$\\
	Tập xác định của hàm số là $\mathscr{D}=\left[m-1;2m\right)$ với điều kiện $m-1<2m\Leftrightarrow m>-1$.\\
	Hàm số đã cho xác định trên $\left(-1;3\right)$ khi và chỉ khi $\left(-1;3\right)\subset \left[m-1;2m\right)$\\
	$\Leftrightarrow m-1\le-1<3\le 2m\Leftrightarrow \heva{
			& m\le 0 \\
			& m\ge \dfrac{3}{2}.}$\\
	Vậy không có giá trị $m$ thỏa bài toán.}
\end{ex}
\begin{ex}%[Phan Anh]%[0D2K1-2]
	Tìm tất cả các giá trị thực của tham số $m$ để hàm số $y=\dfrac{x+2m+2}{x-m}$ xác định trên $\left(-1;0\right)$.
	\choice
	{$\hoac{
				& m>0 \\
				& m<-1}$}
	{$m\le-1$}
	{\True $\hoac{
				& m\ge 0 \\
				& m\le-1}$}
	{$m\ge 0$}
	\loigiai{
		Hàm số xác định khi $x-m\ne 0\Leftrightarrow x\ne m$.
		Tập xác định của hàm số là $\mathscr{D}=\mathbb{R}\setminus\left\{m\right\}$.\\
		Hàm số xác định trên $\left(-1;0\right)$ khi và chỉ khi $m\notin \left(-1;0\right)\Leftrightarrow \hoac{
				& m\ge 0 \\
				& m\le-1.}$}
\end{ex}
\begin{ex}%[Phan Anh]%[0D2K1-2]
	Tìm tất cả các giá trị thực của tham số $m$ để hàm số $y=\dfrac{mx}{\sqrt{x-m+2}-1}$ xác định trên $(0;1)$.
	\choice
	{$m\in \left(-\infty;\dfrac{3}{2}\right]\cup \left\{2\right\}$}
	{$m\in \left(-\infty;-1\right]\cup \left\{2\right\}$}
	{$m\in \left(-\infty;1\right]\cup \left\{3\right\}$}
			{\True $m\in \left(-\infty;1\right]\cup \left\{2\right\}$}
				\loigiai{
				Hàm số xác định khi $\heva{
					& x-m+2\ge 0 \\
					& \sqrt{x-m+2}-1\ne 0}\Leftrightarrow \heva{
					& x\ge m-2 \\
					& x\ne m-1.}$
				\\ Tập xác định của hàm số là $\mathscr{D}=\left[m-2;+\infty \right)\setminus\left\{m-1\right\}$.\\
			Hàm số xác định trên $\left(0;1\right)$ khi và chỉ khi $\left(0;1\right)\subset \left[m-2;+\infty \right)\setminus\left\{m-1\right\}$\\
		$\Leftrightarrow \hoac{
				& m-2\le 0<1\le m-1 \\
				& m-1\le 0}\Leftrightarrow \hoac{
				& \heva{
					& m\le 2 \\
					& m\ge 2} \\
				& m\le 1}\Leftrightarrow \hoac{
				& m=2 \\
				& m\le 1.}$}
\end{ex}
\begin{ex}%[Phan Anh]%[0D2K1-2]
	Tìm tất cả các giá trị thực của tham số $m$ để hàm số $y=\sqrt{x-m}+\sqrt{2x-m-1}$ xác định trên $(0;+\infty)$.
	\choice
	{$m\le 0$}
	{$m\ge 1$}
	{$m\le 1$}
	{\True $m\le-1$}
	\loigiai{
		Hàm số xác định khi $\heva{
				& x-m\ge 0 \\
				& 2x-m-1\ge 0}\Leftrightarrow \heva{
				& x\ge m \\
				& x\ge \dfrac{m+1}{2}}\,(*)$.
		\begin{itemize}
			\item Nếu $m\ge \dfrac{m+1}{2}\Leftrightarrow m\ge 1$ thì $\left(*\right)\Leftrightarrow x\ge m$.\\
			      Tập xác định của hàm số là $\mathscr{D}=\left[m;+\infty \right)$.
			      Khi đó, hàm số xác định trên $\left(0;+\infty \right)$ khi và chỉ khi $\left(0;+\infty \right)\subset \left[m;+\infty \right)\Leftrightarrow m\le 0$
			      $\Rightarrow $ Không thỏa mãn điều kiện $m\ge 1$.
			\item Nếu $m\le \dfrac{m+1}{2}\Leftrightarrow m\le 1$ thì $\left(*\right)\Leftrightarrow x\ge \dfrac{m+1}{2}$.\\
			      Tập xác định của hàm số là $\mathscr{D}=\left[\dfrac{m+1}{2};+\infty \right)$.
			      Khi đó, hàm số xác định trên $\left(0;+\infty \right)$
			      khi và chỉ khi $\left(0;+\infty \right)\subset \left[\dfrac{m+1}{2};+\infty \right)\Leftrightarrow \dfrac{m+1}{2}\le 0\Leftrightarrow m\le-1$.\\
			      $\Rightarrow $ Thỏa mãn điều kiện $m\le 1$.
		\end{itemize}
		Vậy $m\le-1$ thỏa yêu cầu bài toán.}
\end{ex}
\begin{ex}%[Phan Anh]%[0D2K1-2]
	Tìm tất cả các giá trị thực của tham số $m$ để hàm số $y=\dfrac{2x+1}{\sqrt{x^2-6x+m-2}}$ xác định trên $\mathbb{R}$.
	\choice
	{$m\ge 11$}
	{\True $m>11$}
	{$m<11$}
	{$m\le 11$}
	\loigiai{
		Hàm số xác định khi $x^2-6x+m-2>0\Leftrightarrow {\left(x-3\right)}^2+m-11>0$.\\
		Hàm số xác định với $\forall x\in \mathbb{R}\Leftrightarrow (x-3)^2+m-11>0$ đúng với mọi $x\in \mathbb{R}$
		$\Leftrightarrow m-11>0\Leftrightarrow m>11$.}
\end{ex}
\begin{ex}%[Phan Anh]%[0D2B1-3]
	Cho hàm số $f(x)=4-3x$. Khẳng định nào sau đây đúng?
	\choice
	{Hàm số đồng biến trên $\left(-\infty;\dfrac{4}{3}\right)$}
	{\True Hàm số nghịch biến trên $\left(\dfrac{4}{3};+\infty \right)$}
	{Hàm số đồng biến trên $\mathbb{R}$}
	{Hàm số đồng biến trên $\left(\dfrac{3}{4};+\infty \right)$}
	\loigiai{
		TXĐ: $\mathscr{D}=\mathbb{R}$. \\Với mọi $x_1,x_2\in \mathbb{R}$ và $x_1<x_2$, ta có
		$f\left(x_1\right)-f\left(x_2\right)=\left(4-3x_1\right)-\left(4-3x_2\right)=-3\left(x_1-x_2\right)>0.$\\
		Suy ra $f\left(x_1\right)>f\left(x_2\right)$. Do đó, hàm số nghịch biến trên $\mathbb{R}$.\\
		Mà $\left(\dfrac{4}{3};+\infty \right)\subset \mathbb{R}$ nên hàm số cũng nghịch biến trên $\left(\dfrac{4}{3};+\infty \right)$.}
\end{ex}
% \begin{ex}%[Phan Anh]%[0D2B1-3]
% 	Xét tính đồng biến, nghịch biến của hàm số $f(x)=x^2-4x+5$ trên khoảng $\left(-\infty;2\right)$ và trên khoảng $\left(2;+\infty \right)$. Khẳng định nào sau đây đúng?
% 	\choice
% 	{\True Hàm số nghịch biến trên $\left(-\infty;2\right)$, đồng biến trên $\left(2;+\infty \right)$}
% 	{Hàm số đồng biến trên $\left(-\infty;2\right)$, nghịch biến trên $\left(2;+\infty \right)$}
% 	{Hàm số nghịch biến trên các khoảng $\left(-\infty;2\right)$ và $\left(2;+\infty \right)$}
% 	{Hàm số đồng biến trên các khoảng $\left(-\infty;2\right)$ và $\left(2;+\infty \right)$}
% 	\loigiai{
% 		Ta có $f\left(x_1\right)-f\left(x_2\right)=\left(x_1^2-4x_1+5\right)-\left(x_2^2-4x_2+5\right)$
% 		$=\left(x_1^2-x_2^2\right)-4\left(x_1-x_2\right)=\left(x_1-x_2\right)\left(x_1+x_2-4\right)$.
% 		Với mọi $x_1, x_2\in \left(-\infty;2\right)$ và $x_1<x_2$. Ta có $\heva{
% 			& x_1<2 \\ 
% 			& x_2<2 \\}\Rightarrow x_1+x_2<4$.\\
% 		Suy ra $\dfrac{f\left(x_1\right)-f\left(x_2\right)}{x_1-x_2}=\dfrac{\left(x_1-x_2\right)\left(x_1+x_2-4\right)}{x_1-x_2}=x_1+x_2-4<0$.\\
% 		Vậy hàm số nghịch biến trên $\left(-\infty;2\right)$.\\
% 		Với mọi $x_1, x_2\in \left(2;+\infty \right)$ và $x_1<x_2$. Ta có $\heva{
% 			& x_1>2 \\ 
% 			& x_2>2 \\}\Rightarrow x_1+x_2>4$.\\
% 		Suy ra $\dfrac{f\left(x_1\right)-f\left(x_2\right)}{x_1-x_2}=\dfrac{\left(x_1-x_2\right)\left(x_1+x_2-4\right)}{x_1-x_2}=x_1+x_2-4>0$.\\
% 		Vậy hàm số đồng biến trên $\left(2;+\infty \right)$.}
% \end{ex}
\begin{ex}%[Phan Anh]%[0D2B1-3]
	Xét sự biến thiên của hàm số $f(x)=\dfrac{3}{x}$ trên khoảng $(0;+\infty)$. Khẳng định nào sau đây đúng?
	\choice
	{Hàm số đồng biến trên khoảng $\left(0;+\infty \right)$}
	{\True Hàm số nghịch biến trên khoảng $\left(0;+\infty \right)$}
	{Hàm số vừa đồng biến, vừa nghịch biến trên khoảng $\left(0;+\infty \right)$}
	{Hàm số không đồng biến, cũng không nghịch biến trên khoảng $\left(0;+\infty \right)$}
	\loigiai{
		Ta có $f\left(x_1\right)-f\left(x_2\right)=\dfrac{3}{x_1}-\dfrac{3}{x_2}=\dfrac{3\left(x_2-x_1\right)}{x_1x_2}=-\dfrac{3\left(x_1-x_2\right)}{x_1x_2}.$\\
		Với mọi $x_1, x_2\in \left(0;+\infty \right)$ và $x_1<x_2$. Ta có $\heva{
				& x_1>0 \\
				& x_2>0 \\}\Rightarrow x_1\cdot x_2>0$.\\
		Suy ra $\dfrac{f\left(x_1\right)-f\left(x_2\right)}{x_1-x_2}=-\dfrac{3}{x_1x_2}<0\Rightarrow f(x)$ nghịch biến trên $\left(0;+\infty \right)$.}
\end{ex}
\begin{ex}%[Phan Anh]%[0D2B1-3]
	Xét sự biến thiên của hàm số $f(x)=x+\dfrac{1}{x}$ trên khoảng $\left(1;+\infty \right)$. Khẳng định nào sau đây đúng?
	\choice
	{\True Hàm số đồng biến trên khoảng $\left(1;+\infty \right)$}
	{Hàm số nghịch biến trên khoảng $\left(1;+\infty \right)$}
	{Hàm số vừa đồng biến, vừa nghịch biến trên khoảng $\left(1;+\infty \right)$}
	{Hàm số không đồng biến, cũng không nghịch biến trên khoảng $\left(1;+\infty \right)$}
	\loigiai{
		Ta có
		$f\left(x_1\right)-f\left(x_2\right)=\left(x_1+\dfrac{1}{x_1}\right)-\left(x_2+\dfrac{1}{x_2}\right)=\left(x_1-x_2\right)+\left(\dfrac{1}{x_1}-\dfrac{1}{x_2}\right)=\left(x_1-x_2\right)\left(1-\dfrac{1}{x_1x_2}\right).$\\
		Với mọi $x_1, x_2\in \left(1;+\infty \right)$ và $x_1<x_2$. Ta có $\heva{
				& x_1>1 \\
				& x_2>1 \\}\Rightarrow x_1\cdot x_2>1\Rightarrow \dfrac{1}{x_1\cdot x_2}<1.$\\
		Suy ra $\dfrac{f\left(x_1\right)-f\left(x_2\right)}{x_1-x_2}=1-\dfrac{1}{x_1x_2}>0\Rightarrow f(x)$ đồng biến trên $\left(1;+\infty \right)$.}
\end{ex}
\begin{ex}%[Phan Anh]%[0D2B1-3]
	Xét tính đồng biến, nghịch biến của hàm số $f(x)=\dfrac{x-3}{x+5}$ trên khoảng $\left(-\infty;-5\right)$ và trên khoảng $\left(-5;+\infty \right)$. Khẳng định nào sau đây đúng?
	\choice
	{Hàm số nghịch biến trên $\left(-\infty;-5\right)$, đồng biến trên $\left(-5;+\infty \right)$}
	{Hàm số đồng biến trên $\left(-\infty;-5\right)$, nghịch biến trên $\left(-5;+\infty \right)$}
	{Hàm số nghịch biến trên các khoảng $\left(-\infty;-5\right)$ và $\left(-5;+\infty \right)$}
	{\True Hàm số đồng biến trên các khoảng $\left(-\infty;-5\right)$ và $\left(-5;+\infty \right)$}
	\loigiai{
		Ta có
		\begin{eqnarray*}
			f\left(x_1\right)-f\left(x_2\right)&=&\left(\dfrac{x_1-3}{x_1+5}\right)-\left(\dfrac{x_2-3}{x_2+5}\right)\\
			&=&\dfrac{\left(x_1-3\right)\left(x_2+5\right)-\left(x_2-3\right)\left(x_1+5\right)}{\left(x_1+5\right)\left(x_2+5\right)}\\
			&=&\dfrac{8\left(x_1-x_2\right)}{\left(x_1+5\right)\left(x_2+5\right)}.
		\end{eqnarray*}
		Với mọi $x_1, x_2\in \left(-\infty;-5\right)$ và $x_1<x_2$. Ta có $\heva{& x_1<-5 \\& x_2<-5}\Leftrightarrow \heva{&x_1+5<0 \\& x_2+5<0.}$\\
		Suy ra $\dfrac{f\left(x_1\right)-f\left(x_2\right)}{x_1-x_2}=\dfrac{8}{\left(x_1+5\right)\left(x_2+5\right)}>0\Rightarrow f(x)$ đồng biến trên $\left(-\infty;-5\right)$.\\
		Với mọi $x_1, x_2\in \left(-5;+\infty \right)$ và $x_1<x_2$. Ta có $\heva{
				& x_1>-5 \\
				& x_2>-5 \\}\Leftrightarrow \heva{
				& x_1+5>0 \\
				& x_2+5>0 \\}$.\\
		Suy ra $\dfrac{f\left(x_1\right)-f\left(x_2\right)}{x_1-x_2}=\dfrac{8}{\left(x_1+5\right)\left(x_2+5\right)}>0\Rightarrow f(x)$ đồng biến trên $\left(-5;+\infty \right)$.}
\end{ex}

\begin{ex}%[Phan Anh]%[0D2K1-3]
	Cho hàm số $f(x)=\sqrt{2x-7}$. Khẳng định nào sau đây đúng?
	\choice
	{Hàm số nghịch biến trên $\left(\dfrac{7}{2};+\infty \right)$}
	{\True Hàm số đồng biến trên $\left(\dfrac{7}{2};+\infty \right)$}
	{Hàm số đồng biến trên $\mathbb{R}$}
	{Hàm số nghịch biến trên $\mathbb{R}$}
	\loigiai{
	Tập xác định là $\mathscr{D}=\left[\dfrac{7}{2};+\infty \right)$ nên ta loại đáp án C và D.\\
	Xét $f\left(x_1\right)-f\left(x_2\right)=\sqrt{2x_1-7}-\sqrt{2x_2-7}=\dfrac{2\left(x_1-x_2\right)}{\sqrt{2x_1-7}+\sqrt{2x_2-7}}.$\\
	Với mọi $x_1, x_2\in \left(\dfrac{7}{2};+\infty \right)$ và $x_1<x_2$, ta có $\dfrac{f\left(x_1\right)-f\left(x_2\right)}{x_1-x_2}=\dfrac{2}{\sqrt{2x_1-7}+\sqrt{2x_2-7}}>0.$\\
	Vậy hàm số đồng biến trên $\left(\dfrac{7}{2};+\infty \right)$.}
\end{ex}
\begin{ex}%[Phan Anh]%[0D2K1-3]
	Có bao nhiêu giá trị nguyên của tham số $m$ thuộc đoạn $\left[-3;3\right]$ để hàm số $f(x)=\left(m+1\right)x+m-2$ đồng biến trên $\mathbb{R}$?
	\choice
	{$7$}
	{$5$}
	{\True $4$}
	{$3$}
	\loigiai{
		Tập xác định $\mathscr{D}=\mathbb{R}.$\\
		Với mọi $x_1,x_2\in\mathscr{D}$ và $x_1<x_2$. \\Ta có
		$f\left(x_1\right)-f\left(x_2\right)=\left[\left(m+1\right)x_1+m-2\right]-\left[\left(m+1\right)x_2+m-2\right]=\left(m+1\right)\left(x_1-x_2\right).$\\
		Suy ra $\dfrac{f\left(x_1\right)-f\left(x_2\right)}{x_1-x_2}=m+1$.\\
		Để hàm số đồng biến trên $\mathbb{R}$ khi và chỉ khi
		$m+1>0\Leftrightarrow m>-1\xrightarrow{m\in \left[-3;3\right]}{m\in \mathbb{Z}}\Rightarrow m\in \left\{0;1;2;3\right\}$.\\
		Vậy có 4 giá trị nguyên của $m$ thỏa mãn.}
\end{ex}
% \begin{ex}%[Phan Anh]%[0D2K1-3]
% 	Tìm tất cả các giá trị thực của tham số $m$ để hàm số $y=-x^2+\left(m-1\right)x+2$ nghịch biến trên khoảng $\left(1;2\right)$.
% 	\choice
% 	{$m<5$}
% 	{$m>5$}
% 	{\True $m<3$}
% 	{$m>3$}
% 	\loigiai{
% 		Với mọi $x_1\ne x_2$, ta có\\
% 		$\dfrac{f\left(x_1\right)-f\left(x_2\right)}{x_1-x_2}=\dfrac{\left[-x_1^2+\left(m-1\right)x_1+2\right]-\left[-x_2^2+\left(m-1\right)x_2+2\right]}{x_1-x_2}=-\left(x_1+x_2\right)+m-1.$\\
% 		Để hàm số nghịch biến trên $\left(1;2\right)\Leftrightarrow-\left(x_1+x_2\right)+m-1<0$, với mọi $x_1,x_2\in \left(1;2\right)$\\
% 		$\Leftrightarrow m<\left(x_1+x_2\right)+1$, với mọi $x_1,x_2\in \left(1;2\right)$
% 		$\Leftrightarrow m<\left(1+1\right)+1=3$.}
% \end{ex}
\begin{ex}%[Phan Anh]%[0D2K1-3]
	\immini{Cho hàm số $y=f(x)$ có tập xác định là $\left[-3;3\right]$ và đồ thị của nó được biểu diễn bởi hình bên. Khẳng định nào sau đây là đúng?
		\choice
		{\True Hàm số đồng biến trên khoảng $\left(-3;-1\right)$ và $\left(1;3\right)$}
		{Hàm số đồng biến trên khoảng $\left(-3;-1\right)$và $\left(1;4\right)$}
		{Hàm số đồng biến trên khoảng $\left(-3;3\right)$}
		{Hàm số nghịch biến trên khoảng $\left(-1;0\right)$}}
	{\begin{tikzpicture}[>=stealth,scale=0.7]
			\draw[->](-4,0)--(4,0)node[above]{$x$};
			\draw[->](0,-2)--(0,5)node[right]{$y$};
			\draw (-3,-1)--(-1,1)--(0,1)node[above left]{$1$}--(3,4);
			\draw[dashed](-3,0)node[above]{$-3$}--(-3,-1)--(0,-1)node[right]{$-1$};
			\draw[dashed](-1,0)node[below]{$-1$}--(-1,1);
			\draw[dashed](3,0)node[below]{$3$}--(3,4)--(0,4)node[left]{$4$};
			\fill (-3,0)circle(1.2pt) (-3,-1)circle(1.2pt) (0,-1)circle(1.2pt) (-1,0)circle(1.2pt) (-1,1)circle(1.2pt) (0,1)circle(1.2pt) (3,0)circle(1.2pt) (3,4)circle(1.2pt) (0,4)circle(1.2pt) (0,0)node[above right]{$O$}circle(1.2pt);
		\end{tikzpicture}}
	\loigiai{
		Trên khoảng $\left(-3;-1\right)$ và $\left(1;3\right)$ đồ thị hàm số đi lên từ trái sang phải\\
		$\Rightarrow $ Hàm số đồng biến trên khoảng $\left(-3;-1\right)$ và $\left(1;3\right).$}
\end{ex}
\begin{ex}%[Phan Anh]%[0D2K1-3]
	\immini{Cho đồ thị hàm số $y=x^3$ như hình bên. Khẳng định nào sau đây \textbf{sai}?
		\choice
		{Hàm số đồng biến trên khoảng $\left(-\infty;0\right)$}
		{Hàm số đồng biến trên khoảng $\left(0;+\infty \right)$}
		{Hàm số đồng biến trên khoảng $\left(-\infty;+\infty \right)$}
		{\True Hàm số đồng biến tại gốc tọa độ $O$}}
	{\begin{tikzpicture}[>=stealth,scale=0.6]
			\draw[->](-2,0)--(2,0)node[above]{$x$};
			\draw[->](0,-3)--(0,3)node[right]{$y$};
			\draw[smooth,samples=100,domain=-1.4:1.4]plot(\x,{(\x)^3});
			\fill (0,0)node[above left]{$O$}circle(1.2pt);
		\end{tikzpicture}}
	\loigiai{Dựa vào đồ thị, ta thấy hàm số đồng biến trên toàn miền xác định. Nhưng không thể đồng biến chỉ tại đúng một điểm.}
\end{ex}
\Closesolutionfile{ans}
\Closesolutionfile{ansbook}
% \setcounter{section}{0}
\section{HÀM SỐ VÀ ĐỒ THỊ}
\subsection{TÓM TẮT LÝ THUYẾT}
\subsubsection{Khái niệm hàm số. Tập xác định và tập giá trị của hàm số}
\begin{itemize}
	\item [\faSunO] \textbf{Định nghĩa:} Giả sử $x$ và $y$ là hai đại lượng biến thiên và $x$ nhận giá trị thuộc tập số $\mathscr{D}$.	Nếu với \textbf{mỗi giá trị $x$ thuộc $\mathscr{D}$}, ta xác định được \textbf{một và chỉ một giá trị tương ứng $y$} thuộc tập hợp số thực $\mathbb{R}$ thì ta có một hàm số.
	\begin{khung}
		\begin{itemize}
			\item Ta gọi $x$ là biến số và $y$ là hàm số của $x$.
			\item Tập hợp $\mathscr{D}$ được gọi là tập xác định của hàm số.
			\item Tập hợp $T$ gồm tất cả các giá trị $y$ (tương ứng với $x$ thuộc $\mathscr{D}$) gọi là tập giá trị của hàm số.
		\end{itemize}
	\end{khung}
	\item [\faSunO] \textbf{Cách cho một hàm số:} Một hàm số có thể được cho bởi một công thức hoặc nhiều công thức; có thể cho bằng mô tả; cho bằng bảng hoặc cho bằng biểu đồ.
\end{itemize}
% \begin{vd}
% 	Bản tin dự báo thời tiết cho biết nhiệt độ ở một số thời điểm trong ngày 01/05/2021 tại thành phố Hồ Chí Minh được ghi lại với biểu đồ bên dưới\\
% 	\centerline{\begin{tikzpicture}[font=\footnotesize, x=1.2cm,line join=round, line cap=round, >=stealth,scale=0.7,color=cyan]
% 		\draw[->] (.5,0)--(.5,5.5)node[left]{nhiệt độ};
% 		\draw[->] (.5,0)--(9,0) node[below]{giờ};
% 		\foreach \x/\y in {0/24,1/26,2/28,3/30,4/32,5/34} {
% 			\draw (.5,\x) node[left]{$\y$};}
% 		\foreach \x/\y in {1/1,2/4,3/7,4/10,5/13,6/16,7/19,8/22}\draw (\x,0.1)--(\x,-0.1) node [below] {\footnotesize $\y$};
% 		\foreach \x in {1,2,...,5} { \draw[orange!30] (.5,\x)--(9,\x); }
% 		\draw [line width=1pt,blue] (1,2)node[above]{\scriptsize $28$}--(2,1.5)node[above]{\scriptsize $27$}--(3,2)node[above left]{\scriptsize $28$}--(4,4)node[above]{\scriptsize $32$}--(5,3.5)node[above]{\scriptsize $31$}--(6,2.5)node[above]{\scriptsize $29$}--(7,2)node[above]{\scriptsize $28$}--(8,1.5)node[above]{\scriptsize $27$};
% 	\end{tikzpicture}}\\
% Rõ ràng với mỗi mốc giờ xác định trong ngày, ta có tương ứng duy nhất 1 số đo nhiệt độ được dự báo nên  có xem đây là 1 hàm số với
% 	\begin{itemize}
% 		\item  Tập xác định $D=\{1;4;7;10;13;16;19;22\}$
% 		\item  Tập giá trị $T=\{28; 27; 32;31;29\}$.
% 	\end{itemize}
% \end{vd}
% \begin{vd}
% 	Xét công thức $y=2x+1$. Ta đã biết đây là một hàm số bậc nhất với
% 	\begin{itemize}
% 		\item  Tập xác định $D=\mathbb{R}$
% 		\item  Tập giá trị $T=\mathbb{R}$.
% 	\end{itemize}
% \end{vd}
\subsubsection{Đồ thị hàm số}
\begin{itemize}
	\item [\faSunO] \textbf{Định nghĩa:} Cho hàm số $y=f(x)$ có tập xác định $\mathscr{D}$. Trên mặt phẳng toạ độ $Oxy$, đồ thị $(C)$ của hàm số là tập hợp tất cả các điểm $M(x;y)$ với $x \in \mathscr{D}$ và $y=f(x)$. Vậy $(C)=\{M(x;f(x)) \mid x \in \mathscr{D}\}$.
	\item [\faSunO] \textbf{Lưu ý:} Điểm $M\left(x_{M};y_{M}\right)$ thuộc đồ thị hàm số $y=f(x)$ khi và chỉ khi $x_{M} \in \mathscr{D}$ và $y_{M}=f\left(x_{M}\right)$.
\end{itemize}
\subsubsection{Sự đồng biến, hàm số nghịch biến của hàm số}
\begin{itemize}
	\item [\faSunO] \textbf{Khái niệm:} Với hàm số $y=f(x)$ xác định trên khoảng $(a;b)$, ta nói
	\begin{itemize}
		\item [\iconCH] Hàm số đồng biến trên khoảng $(a;b)$ nếu
		\boxmini{$\forall x_{1}, x_{2} \in(a ; b), x_{1}<x_{2} \Rightarrow f\left(x_{1}\right)<f\left(x_{2}\right)$}
		\item [\iconCH] Hàm số nghịch biến trên khoảng $(a;b)$ nếu
		\boxmini{$\forall x_{1}, x_{2} \in(a ; b), x_{1}<x_{2} \Rightarrow f\left(x_{1}\right)>f\left(x_{2}\right)$}
	\end{itemize}
	\item [\faSunO] \textbf{Lưu ý:} Khi vẽ bảng biến thiên, xét từ trái sang phải, ta dùng mũi tên đi xuống để minh họa khoảng nghịch biến và mũi tên đi lên để minh họa khoảng đồng biến. 
\end{itemize}

\subsection{RÈN LUYỆN KĨ NĂNG GIẢI TOÁN}
\begin{dang}{Tính giá trị của hàm số tại một điểm}
	Cho hàm số $y=f(x)$ có tập xác định $\mathscr{D}$ và $x_0 \in \mathscr{D}$.
	\begin{itemize}
		\item[\faPencilSquareO] Tính giá trị hàm số tại $x_0$: Ta chỉ việc thay $x_0$ vào biểu thức $y=f(x)$, tìm được $y_0$.
		\item[\faPencilSquareO] Nếu $f(x)$ là hàm cho bởi nhiều biểu thức thì ta thay $x_0$ vào biểu thức mà miền xác định của nó chứa $x_0$.
	\end{itemize}
\end{dang}

\begin{vd}
	Cho hai hàm số $f(x)=x^2-2x$ và $g(x)=1-x$. Tính $f(1)$; $g(-2)$; $f(1)+g(-2)$.
	\loigiai{}
\end{vd}

\begin{vd}
	Cho hàm số $f(x)=\heva{&3x-2 &\text{với } &x\ge 1 \\&1-2x^2 &\text{với } &x<1}$. Tính $f(1), f(2), f(0), f(-3)$.
		\loigiai{}
\end{vd}



\begin{vd}
	Cho hàm số $y=2x^3-3(m-1)x+2$, với $m$ là tham số. 
	\begin{tasks}
		\task Tìm $m$ để đồ thị hàm số đi qua điểm $M(1;2)$.
		\task Tìm $m$ để đồ thị hàm số đi qua điểm $N(-3;1)$.
	\end{tasks}
	\loigiai{}
\end{vd}

\begin{vd}
\immini{ Cho hàm số $y=f(x)$ và hàm số $y=g(x)$ có đồ thị như hình bên.
	\begin{tasks}
		\task Trong các điểm $A(2;2)$ $B(4;2)$, $C(3;3)$ điểm nào thuộc đồ thị $f(x)$? điểm nào thuộc đồ thị $g(x)$?
		\task Tính giá trị $f(1)+g(2)$.
		\task Tìm điểm trên đồ thị $f(x)$ có tung độ bằng $3$.
	\end{tasks}
}{
	\begin{tikzpicture}[>=stealth,scale=0.7]
	\draw[->] (-1.2,0)--(0,0)%
	node[below left]{$O$}--(7.5,0) node[below]{$x$};
	\draw[->] (0,-0.8) --(0,5) node[right]{$y$};
	\foreach \x in {1,2,3,4,5,6,7}{
		\draw (\x,0) node[below]{$\x$};%Ox
	}
	\foreach \y in {1,2,3,4}{
		\draw (0,\y) node[left]{$\y$};%Oy
	}
	\draw[violet,dashed,line width=0.2pt] (0,0) grid (7,4.5);
	\draw [blue, line width=1.5pt] (0,0)--(4,2)--(7,2);
	\draw [red, line width=1.5pt, domain=0.5:3.5, samples=100] %
	plot (\x, {(\x)^2 -4*(\x)+6});
	\draw (3.1,3.4)node[right]{\footnotesize$y=f(x)$} (5,1.5)node[right]{\footnotesize $y=g(x)$};
	
	\end{tikzpicture}}
	\loigiai{}
\end{vd}

\begin{dang}{Tìm tập xác định, tập giá trị của hàm số}
	\begin{itemize}
		\item [\faSunO] \textbf{Tập xác định:} Ta tìm tập hợp tất cả các giá trị của $x$ để hàm số đã cho có nghĩa. Cần lưu ý hai vấn đề sau:
		\begin{listEX}[2]
			\item [\ding{172}] $\dfrac{A}{B}$ có nghĩa khi $B \ne 0$.
			\item [\ding{173}] $\sqrt{B}$ có nghĩa khi $B \ge 0$.
		\end{listEX}
		\item [\faSunO] \textbf{Tập giá trị:} Với $x$ thuộc miền xác định $\mathscr{D}$, ta có thể căn cứ vào bảng biến thiên hoặc đồ thị để tìm miền giá trị (\textit{nhìn khoảng "dao động" của} $y$.).
	\end{itemize}

\end{dang}

\begin{vd}%[0D2B1-1]
	Sau khi đun nóng băng phiến lên đến gần $90^{\circ}C$, người ta để nguội, quan sát, ghi nhận nhiệt độ và trạng thái của băng phiến sau mẫu phút như bảng sau
	\begin{center}
	\textit{Nhiệt độ và trạng thái của băng phiến khi để nguội}
		\begin{tabular}{|l|c|c|c|c|c|c|c|c|c|c|c|}
			\hline 
			\textbf{Thời gian nguội (phút)} & $0$ & $1$ & $2$ & $3$ & $4$ & $5$ & $6$ & $7$ & $8$ & $9$ & $10$ \\ 
			\hline 
			\textbf{Nhiệt độ ($^{\circ}C$)} & $86$ & $84$ & $82$ & $81$ & $80$ & $80$ & $80$  &$80$ & $79$ & $77$ & $75$ \\
			\hline
			\textbf{Trạng thái} & \multicolumn{4}{|c|}{lỏng} &\multicolumn{4}{|c|}{lỏng và rắng} & \multicolumn{3}{|c|}{rắn}\\
			\hline
		\end{tabular}
	\end{center}
\begin{tasks}(1)
	\task Tại sao từ bảng trên, có thể nói nhiệt độ của băng phiến là một hàm số theo thời gian (nung nóng)? Tìm tập xác định và tập giá trị của hàm số trên.
	\task Sau khi để nguội $3$ phút, nhiệt độ băng phiến là bao nhiêu?
	\task Băng phiến chuyển hoàn toàn sang trạng thái rắn sau bao nhiêu phút?
\end{tasks}
	\loigiai{
		\begin{enumerate}[a)]
			\item Bảng giá trị cho thấy nhiệt độ (kí hiệu là $y$) là một hàm số theo thời gian (kí hiệu là $x$) vì khi cho $x$ một giá trị bất kì, ta luôn tìm được duy nhất một giá trị của $y$. Do vậy bảng này xác định một hàm số biểu thị nhiệt độ của băng phiến theo thời gian.\\
			Từ bảng giá trị của hàm số, ta có tập xác định $\mathscr{D}=\{0 ; 1 ; 2 ; 3 ; 4 ; 5 ; 6 ; 7 ; 8 ; 9 ; 10\}$ và tập giá trị $T=\{75 ; 77 ; 79 ; 80 ; 81 ; 82 ; 84 ; 86\}$.
			\item Sau khi để nguội $3$ phút, nhiệt độ băng phiến là $81^{\circ}C$.
			\item Băng phiến chuyển hoàn toàn sang trạng thái rắn sau $8$ phút (lúc đó nhiệt độ băng phiến là $79^{\circ}$).
		\end{enumerate}
	}
\end{vd}

\begin{vd}
	\immini{Cho hàm số $y=f(x)$ có tập xác định là $\mathscr{D}$ và đồ thị là đường liền nét được vẽ trên miền $\mathscr{D}$ như hình bên
	\begin{tasks}(1)
		\task Xác định tập xác định $\mathscr{D}$.
		\task Tìm tập giá trị của hàm số trên miền $\mathscr{D}$.
		\task Tìm các điểm thuộc đồ thị và có tung độ bằng $3$.
	\end{tasks}}{
	\begin{tikzpicture}[smooth,samples=300,scale=0.8,>=stealth,font=\footnotesize]
		\draw[line width=0.1pt,gray!80] (-4,-2) grid (5,4);
		\draw[->] (-4.2,0)--(5.7,0) node[below]{$x$};
		\draw[->] (0,-2.2)--(0,4.5) node[right]{$y$};
		\draw (0,0) node[above right]{$O$};
		\draw[line width=1pt,domain=-1:3,magenta] plot(\x,{(\x-1)^2-1});
		\draw[line width=1pt,magenta] (-3,-1)--(-1,3) (3,3)--(5,1);
		\draw[fill=blue] (-3,-1) circle(2.5pt) (-1,3) circle(2.5pt) (3,3) circle(2.5pt) (5,1) circle(2.5pt) (1,-1) circle(2.5pt);
		\foreach \x in {-4,-3,-2,-1,1,2,3,4,5}\draw (\x,0.1)--(\x,-0.1) node [below] {\footnotesize $\x$};
		\foreach \y in {-2,-1,1,2,3}\draw (0.1,\y)--(-0.1,\y) node [left] {\footnotesize $\y$};
\end{tikzpicture}}
	\loigiai{}
\end{vd}

\begin{vd}
	Tìm tập xác định của các hàm số sau đây:
	\begin{listEX}[2]
		\item $y=x^4+x^2-2$.
		\item $y=\dfrac{x+2}{x-2}$.
		\item $y=\dfrac{x^2+2}{4-x}$.
		\item $y=\dfrac{1}{-x^2+3x}$
	\end{listEX}
	\loigiai{}
\end{vd}
\begin{vd}
	Tìm tập xác định của các hàm số sau đây:
	\begin{listEX}[2]
		\item $y=\sqrt{x-2}$.
		\item $y=\dfrac{2x-1}{\sqrt{x+2}}$.
		\item $y=x+\dfrac{1}{\sqrt{3-x}}$.
		\item $y=\sqrt{2+x}+\sqrt{x-2}$.
	\end{listEX}
	\loigiai{}
\end{vd}

\begin{vd}
	Tìm tập xác định của các hàm số sau:
	\begin{listEX}[2]
		\item $f(x)=\heva{&2x+1 &\text{ nếu }& x \le 0 \\& x^2 &\text{ nếu }& x > 0}$.
		\item $f(x)=\heva{&\dfrac{1}{x-1}&\text{ nếu }& x \le 2 \\& x^2 &\text{ nếu }& x > 2}$.
	\end{listEX}
\loigiai{
	\begin{enumerate}[a)]
		\item Ta xét hai trường hợp
		\begin{itemize}
			\item [$\bullet$] Khi $x\le 0$ thì $f(x)=2x+1$. Hàm này luôn xác định với mọi $x \in \mathbb{R}$ nên sẽ xác định với mọi $x \le 0$.
			\item [$\bullet$] Khi $x>0$ thì $f(x)=x^2$. Hàm này luôn xác định với mọi $x\in \mathbb{R}$ nên sẽ xác định với mọi $x >0$.
		\end{itemize}
	Kết hợp hai trường hợp, ta được tập xác định của hàm số là $\mathscr{D}=\mathbb{R}$.
		\item Ta xét hai trường hợp
		\begin{itemize}
			\item [$\bullet$] Khi $x\le 2$ thì $f(x)=\dfrac{1}{x-1}$. Hàm này  xác định khi và chỉ khi $x-1 \ne 0 \Leftrightarrow x \ne 1$.
			\item [$\bullet$] Khi $x>2$ thì $f(x)=x^2$. Hàm này luôn xác định với mọi $x\in \mathbb{R}$ nên sẽ xác định với mọi $x >2$.
		\end{itemize}
		Kết hợp hai trường hợp, ta được tập xác định của hàm số là $\mathscr{D}=\mathbb{R}\backslash \{1\}$.
\end{enumerate}}
\end{vd}


\begin{dang}{Tìm khoảng đồng biến, khoảng nghịch biến của hàm số}
\begin{itemize}
	\item[\faPencilSquareO] Nếu đề bài cho bảng biến thiên hoặc đồ thị: Xét từ trái sang phải thì
	\begin{itemize}
		\item [$\bullet$] Khoảng nào có mũi tên đi xuống (đồ thị đổ xuống) thì khoảng đó hàm số nghịch biến.
		\item [$\bullet$] Khoảng nào có mũi tên đi lên (đồ thị đi lên) thì khoảng đó hàm số đồng biến.
	\end{itemize}
	\item[\faPencilSquareO] Nếu đề bài yêu cầu xét tính đồng biến, nghịch biến của hàm số $y=f(x)$ trên khoảng xác định $(a;b)$:  Ta lấy $x_1, x_2$ tùy ý thuộc $(a;b)$, với $x_1<x_2$ và tính $f(x_1)-f(x_2)$, nếu
	\begin{itemize}
		\item [$\bullet$] $f(x_1)-f(x_2)<0$ thì hàm số $y=f(x)$ đồng biến trên khoảng $(a;b)$.
		\item [$\bullet$] $f(x_1)-f(x_2)>0$ thì hàm số $y=f(x)$ nghịch biến trên khoảng $(a;b)$.
	\end{itemize}
	\item[\faPencilSquareO] Trong nhiều trường hợp, để tìm được khoảng đồng biến và nghịch biến của hàm số, ta có thể lập bảng biến thiên của hàm số đó trên miền xác định.
\end{itemize}
\end{dang}
\begin{vd}
	\immini{Cho hàm số $y=f(x)$ có tập xác định là $\mathscr{D}$ và đồ thị là đường liền nét được vẽ trên miền $\mathscr{D}$ như hình bên. Tìm các khoảng đồng biến và nghịch biến của hàm số trên miền $\mathscr{D}$.
	}{
		\begin{tikzpicture}[smooth,samples=300,scale=0.6,>=stealth,font=\footnotesize]
			\draw[line width=0.1pt,gray!80] (-4,-2) grid (5,4);
			\draw[->] (-4.2,0)--(5.7,0) node[below]{$x$};
			\draw[->] (0,-2.2)--(0,4.5) node[right]{$y$};
			\draw (0,0) node[above right]{$O$};
			\draw[line width=1pt,domain=-1:3,magenta] plot(\x,{(\x-1)^2-1});
			\draw[line width=1pt,magenta] (-3,-1)--(-1,3) (3,3)--(5,1);
			\draw[fill=blue] (-3,-1) circle(2.5pt) (-1,3) circle(2.5pt) (3,3) circle(2.5pt) (5,1) circle(2.5pt) (1,-1) circle(2.5pt);
			\foreach \x in {-4,-3,-2,-1,1,2,3,4,5}\draw (\x,0.1)--(\x,-0.1) node [below] {\footnotesize $\x$};
			\foreach \y in {-2,-1,1,2,3}\draw (0.1,\y)--(-0.1,\y) node [left] {\footnotesize $\y$};
	\end{tikzpicture}}
	\loigiai{}
\end{vd}
\begin{vd}
	Cho hàm số $y=f(x)=-2x^2-7$. Xét tính đồng biến và nghịch biến của hàm số trên các khoảng $(-4;0)$; $(3;10)$.
		\loigiai{}
\end{vd}


\begin{vd}
	Xét tính đồng biến và nghịch biến của hàm số $y=f(x)=x^2+10x+9$ trên $(-5;+\infty)$.
		\loigiai{}
\end{vd}

\begin{vd}
	Xét tính đồng biến và nghịch biến của hàm số $y=f(x)=\dfrac{x}{x-7}$ trên các khoảng $(-\infty;7)$;  $(7;+\infty)$.
		\loigiai{}
\end{vd}

\begin{dang}{Vẽ đồ thị hàm số cho bởi nhiều biểu thức}
	
\end{dang}
\begin{vd}
	Tìm tập xác định và vẽ đồ thị các hàm số sau:
	\begin{tasks}(2)
		\task $f(x)=\heva{& 2x & \text{ với }& x \ge 0 \\ & -x & \text{ với }& x<0.}$
		\task $f(x)=\heva{& -x^2 & \text{ với }& x\le 1\\ & 1 & \text{ với }& x>1.}$
		\task $f(x)= \big|x\big|$.
		\task $f(x)= \big|x+2\big|$.
	\end{tasks}
	\loigiai{}
\end{vd}


\begin{dang}{Viết công thức hàm số cho một số bài toán thực tế}
\end{dang}

\begin{vd}%[0D2T1-1]
	Theo quyết định số 2019/QĐ-BĐVN ngày 01/11/2018 của Tổng công ty Bưu điện Việt Nam, giá cước dịch vụ Bưu chính phổ cập đối với dịch vụ thư cơ bản và bưu thiếp trong nước có khối lượng đến $250$g như trong bảng sau
	\immini{\begin{enumerate}[a)]
			\item Số tiền dịch vụ thư cơ bản phải trả $y$ (đồng) có là hàm số của khối lượng thư cơ bản $x$ (g) hay không? Nếu đúng, hãy xác định những công thức tính $y$.
			\item Tính số tiền phải trả khi bạn Dương gửi thư có khối lượng $150$g, $200$g.
	\end{enumerate} }
	{\vspace{0.6 cm}
		\begin{tabular}{|c|c|}
			\hline
			\textcolor{blue}{Khối lượng đến $250$ g} & \textcolor{blue}{Mức cước (đồng)}\\
			\hline
			Đến $20$ g & $4000$\\
			\hline
			Trên $20$ g đến $100$ g & $6000$ \\
			\hline
			Trên $100$ g đến $250$ g & $8000$\\
			\hline
	\end{tabular}}
	\loigiai{
		\begin{enumerate}[a)]
			\item Số tiền dịch vụ thư cơ bản phải trả $y$ là hàm số của $x$ vì mỗi giá trị của $x$ (chính là khối lượng của thư) có đúng một giá trị của $y$ (mức cước hay số tiền phải trả) tương ứng.\\
			Quan sát bảng ta thấy:
			\begin{itemize}
				\item[$\bullet$] Nếu khối lượng thư đến $20$ g hay $0<x \le 20$ thì mức cước phải trả là $4000$ đồng hay $y= 4000$.
				\item[$\bullet$] Nếu khối lượng thư trên $20$ g hay $100$ g hay $20 < x \le 100$ g thì mức cước là $6000$ đồng hay $y=6000$.
				\item[$\bullet$] Nếu khối lượng thư trên $100$ g đến $250$ g hay $100 < x \le 250$ thì mức cước là $8000$ đồng hay $y=8000$.
			\end{itemize}
			Vậy ta có công thức xác định $y$ như sau:
			$y=\heva{&4000&\text{nếu}& 0<x\le 20\\&6000&\text{nếu}& 20 <x\le 100\\&8000&\text{nếu}& 100<x\le 250.}$
			\item Vì $100 < 150 <250$ và $100 <200 <250$ nên bức thư có khối lượng $150$ g thì cần trả cước là $8000$ đồng và bức thư có khối lượng $200$ g cũng cần trả cước là $8000 $ đồng. \\
			Vậy tổng số tiền phải trả khi bạn Dương gửi thư có khối lượng $150$ g và $200$ g là \\ \centerline{$8000 + 8000 = 16 000$ (đồng).}
		\end{enumerate}
	}
\end{vd}


\begin{vd}
	Nhiệt độ ở mặt đất đo được khoảng $30^{\circ} \mathrm{C}$. Biết rằng cứ lên $1 \mathrm{~km}$ thì nhiệt độ giảm đi $5^{\circ}$.
	\begin{tasks}(1)
		\task Hãy lập hàm số $T$ theo $h$, trong đó $T$ tính bằng độ $\left({ }^{\circ}\right)$ và $h$ tính bằng ki-lô-mét $(\mathrm{km})$.
		\task Hãy tính nhiệt độ khi ở độ cao $3 \mathrm{~km}$ so với mặt đất.
	\end{tasks}
	\loigiai{
		\begin{enumerate}[a)]
			\item Hàm số $T$ theo $h$ là $T=30-5h$.
			\item Thay $h=3$ vào công thức $T=30-5h$, ta được $
			T=30-5 \cdot 3=15$.\\
			Vậy khi lên độ cao $3 \mathrm{~km}$ thì nhiệt độ tại đó là $15^{\circ}$
	\end{enumerate}}
\end{vd}

\begin{vd}%[0D2T1-1]
	Một công ty viễn thông A cung cấp dịch vụ truyền hình cáp với mức phí ban đầu là 300000 đồng và mỗi tháng phải đóng 150000 đồng. Công ty viễn thông B cũng cung cấp dịch vụ truyền hình cáp nhưng không tính phí ban đầu và mỗi tháng khách hàng sẽ phải đóng 200000 đồng.
	\begin{tasks}(1)
		\task Gọi $T$ (đồng) là số tiền khách hàng phải trả cho mỗi công ty viễn thông trong $t$ (tháng) sử dụng dịch vụ truyền hình cáp. Khi đó hãy lập hàm số $T$ theo $t$ đối với mỗi công ty.
		\task Tính số tiền khách hàng phải trả sau khi sử dụng dịch vụ truyền hình cáp trong 5 tháng đối với mỗi công ty.
		\task Khách hàng cần sử dụng dịch vụ truyền hình cáp trên mấy tháng thì đăng kí bên công ty viễn thông A sẽ tiết kiệm chi phí hơn?
	\end{tasks}
	\loigiai{
		\begin{enumerate}[a)]
			\item Hàm số $T$ theo $t$ đối với công ty A là $T=150000t+300000$.\\
			Hàm số $T$ theo $t$ đối với công ty B là $T=200000t$.
			\item Thay $t=5$ lần lượt vào hai công thức trên, ta được
			\begin{itemize}
				\item [$\bullet$] Số tiền phải trả trong 5 tháng khi sử dụng dịch vụ truyền hình cáp của công ty A là $150000.5+300000=1050000$ đồng.
				\item [$\bullet$] Số tiền phải trả trong 5 tháng khi sử dụng dịch vụ truyền hình cáp của công ty B là $200000.5=1000000$ đồng.
			\end{itemize}
			\item Để dịch vụ truyền hình cáp của công ty A lợi hơn dịch vụ truyền hình cáp của công ty B thì:
			$$
			150000t+300000<200000t \Leftrightarrow 300000<50000t \Leftrightarrow t>6
			$$
			Vậy nếu sử dụng từ 7 tháng trở lên thì sử dụng dịch vụ truyền hình cáp bên công ty A sẽ có lợi hơn.
	\end{enumerate}}
\end{vd}
\subsection{BÀI TẬP TỰ LUYỆN}

\begin{bt}%[0D2T1-1]
	Trong kinh tế thị trường, lượng cầu và lượng cung là hai khái niệm quan trọng. Lượng cầu chỉ khả năng về số lượng sản phẩm cần mua của bên mua (người tiêu dùng), tuỳ theo đơn giá bán sản phẩm; còn lượng cung chỉ khả năng cung cấp số lượng sản phẩm nảy cho thị trường của bên bán (nhà sản xuất) cũng phụ thuộc vào đơn giá bán sản phẩm.\\
	Người ta khảo sát nhu cầu của thị trường đối với sản phẩm $\mathrm{A}$ theo đơn giá của sản phẩm này và thu được bảng sau:
	\begin{center}
		\begin{tabular}{|c|c|c|c|c|c|}
			\hline Đơn giá sản phẩm $A$ (đơn vị: nghìn đồng) & 10 & 20 & 40 & 70 & 90 \\
			\hline Lượng cầu (nhu cầu về số sản phẩm) & 338 & 288 & 200 & 98 & 50 \\
			\hline
		\end{tabular}
	\end{center}
	\begin{enumerate}
		\item Hãy cho biết tại sao bảng giá trị trên xác định một hàm số? Hãy tìm tập xác định và tập giá trị của hàm số đó (gọi là hàm cầu).
		\item Giả sử lượng cung của sản phẩm $A$ tuân theo công thức $y=f(x)=\dfrac{x^{2}}{50}$, trong đó $x$ là đơn giá sản phẩm $A$ và $y$ là lượng cung ứng với đơn giá này. Hãy điền các giá trị của hàm số $f(x)$ (gọi là hàm cung) vào bảng sau
		\begin{center}
			\begin{tabular}{|c|l|l|l|l|l|}
				\hline Đơn giá sản phẩm $A$ (đơn vị: nghìn đồng) & 10 & 20 & 40 & 70 & 90 \\
				\hline Lượng cung (khả năng cung cấp về số sản phẩm) & & & & & \\
				\hline
			\end{tabular}
		\end{center}
		\item Ta nói thị trường của một sản phẩm là cần bằng khi lượng cung và lượng cầu bằng nhau. Hãy tìm đơn giá $x$ của sản phẩm $A$ khi thị trường cân bằng.
	\end{enumerate}
	\loigiai{
		\begin{enumerate}
			\item Có thể thấy với mỗi mức đơn giá, đều có duy nhất một giá trị về lượng cầu. Do vậy, bảng giá trị đã cho ở đề bài xác định một hàm số.\\
			Hàm số có tập xác định $\mathscr{D}=\{10;20;40;70;90\}$ và tập giá trị $\mathscr{T}=\{338;288;200;98;50\}$.\\
			\item \hfill
			\begin{center}
				\begin{tabular}{|c|c|c|c|c|c|}
					\hline Đơn giá sản phẩm $A$ (đơn vị nghìn đồng) & 10 & 20 & 40 & 70 & 90 \\
					\hline Lượng cung (khả năng cung cấp về số sản phẩm) & 2 & 8 & 32 & 98 & 162 \\
					\hline
				\end{tabular}
			\end{center}
			\item Dựa vào hai bảng giá trị của lượng cung và lượng cầu, ta tìm được giá trị $x=70$ thì lượng cung và lượng cầu đều bằng $98$.\\
			Vậy thị trường của sản phẩm $A$ cân bằng khi đơn giá của sản phẩm $\mathrm{A}$ này là $70\,000$(đồng).
		\end{enumerate}
	}
\end{bt}

\begin{bt}
	\immini{Cho hàm số $y=f(x)$ có tập xác định là $\mathscr{D}$ và đồ thị là đường liền nét được vẽ trên miền $\mathscr{D}$ như hình bên. 
		\begin{tasks}(1)
			\task Trong các điểm $A(2;2)$, $B(0;1)$, $C(4;2)$, $D(-3;-1)$, điểm nào thuộc $(C)$? điểm nào không thuộc $(C)$?
			\task Tìm tập xác định $\mathscr{D}$ và tập giá trị $\mathscr{T}$ của hàm số  $y=f(x)$. 
			\task Tìm các khoảng đồng biến và nghịch biến của hàm số trên miền $\mathscr{D}$.
		\end{tasks}
		
	}{
		\begin{tikzpicture}[smooth,samples=300,scale=0.7,>=stealth,font=\footnotesize]
			\draw[line width=0.1pt,gray!80] (-4,-2) grid (5,4);
			\draw[->] (-4.2,0)--(5.7,0) node[below]{$x$};
			\draw[->] (0,-2.2)--(0,4.5) node[right]{$y$};
			\draw (0,0) node[above right]{$O$};
			\draw[line width=1pt,magenta] (-3,-1)--(1,0)-- (3,3)--(5,1);
			\draw[fill=blue] (-3,-1) circle(2.5pt) (1,0) circle(2.5pt) (3,3) circle(2.5pt) (5,1) circle(2.5pt);
			\foreach \x in {-4,-3,-2,-1,1,2,3,4,5}\draw (\x,0.1)--(\x,-0.1) node [below] {\footnotesize $\x$};
			\foreach \y in {-2,-1,1,2,3}\draw (0.1,\y)--(-0.1,\y) node [left] {\footnotesize $\y$};
	\end{tikzpicture}}
	\loigiai{}
\end{bt}

\begin{bt}
	Cho hai hàm số $f(x)=x^2-2x$ và $g(x)=1-\sqrt{x}$. Tính giá trị $\dfrac{f(-1)}{g(4)}$.
		\loigiai{}
\end{bt}

\begin{bt}
	Cho hàm số $f(x)=4-\sqrt[3]{x}$.
	\begin{listEX}[3]
		\item Tính $f(-8)$.
		\item Tính $f(a^3)$.
		\item Tìm $a>0$ thỏa $f(a^6)=0$
	\end{listEX}
	\loigiai{}
\end{bt}

\begin{bt}
	Cho hàm số $f(x)=\heva{&x^2-2x-1 &\text{với } x\le 0 \\&\dfrac{x+1}{x^2+x+1} &\text{với } x > 0}$. Tính giá trị của hàm số đó tại $x=1; x=0; x=-2$.
		\loigiai{}
\end{bt}

\begin{bt}%[0D2B1-2]
	Tìm tập xác định của mỗi hàm số sau
	\begin{tasks}(2)
		\task $y=-x^2$.
		\task $y = \sqrt{2-3x}$.
		\task $y = \dfrac{4}{x+1}$.
		\task $y = \heva{& 1 &\text{ nếu }& x \in \mathbb{Q} \\ & 0 &\text{ nếu }& x \in \mathbb{R} \setminus \mathbb{Q}.}$
	\end{tasks}
	\loigiai{
		\begin{enumerate}[a)]
			\item Hàm số $y = -x^2$ có tập xác định $\mathscr{D} = \mathbb{R}$.
			\item Biểu thức $\sqrt{2-3x}$ có nghĩa khi $2 - 3x \geq 0 \Leftrightarrow x \leq \dfrac{2}{3}$. \\
			Vậy tập xác định $\mathscr{D} = \left(-\infty;\dfrac{2}{3}\right]$.
			\item Biểu thức $\dfrac{4}{x+1}$ có nghĩa khi và chỉ khi $x \ne - 1$. \\
			Vậy tập xác định $\mathscr{D} = \mathbb{R} \setminus \{-1 \}$.
			\item Tập xác định $\mathscr{D} = \mathbb{R}$.
		\end{enumerate}
	}
\end{bt}

\begin{bt}%[0D2B1-2]%
	Tìm tập xác định của các hàm số sau
	\begin{tasks}(2)
		\task $y=2-4x$.
		\task $y=\dfrac{x-3}{5-2x}$.
		\task $y=\dfrac{x}{x^2-3x+2}$.
		\task $y=\dfrac{2x+1}{(x-2) \left( x^2-4x+3 \right)}$.
	\end{tasks}
	\loigiai{
		\begin{enumEX}{1}
			\item Ta có $\mathscr{D}=\mathbb{R}$.
			\item Hàm số xác định khi $5-2x \neq 0 \Leftrightarrow x \neq \dfrac{5}{2}$.\\
			Tập xác định $\mathscr{D}=\mathbb{R} \setminus \left\{ \dfrac{5}{2} \right\}$.
			\item Hàm số xác định khi $x^2-3x+2 \neq 0 \Leftrightarrow \heva{&x \neq 1 \\ & x \neq 2}$.\\
			Tập xác định $\mathscr{D}=\mathbb{R} \setminus \{1;2\}$.
			\item Hàm số xác định khi $(x-2) \left( x^2-4x+3 \right) \neq 0 \Leftrightarrow \heva{& x-2\neq 0\\& x^2-3x+2\neq 0} \Leftrightarrow \heva{& x \neq 1\\ & x \neq 2 \\ & x \neq 3}$.\\
			Tập xác định $\mathscr{D}=\mathbb{R} \setminus \{1;2;3\}$.
		\end{enumEX}
	}
\end{bt}

\begin{bt}%[0D2B1-2]%
	Tìm tập xác định của các hàm số
	\begin{tasks}(2)
		\task $ y= \dfrac{\sqrt{ 4-2x}}{x^2-6x+5}$.
		\task $ y= \sqrt{ \dfrac{x^2}{x-1}}$.
	\end{tasks}
	\loigiai{
		\begin{enumerate}[a)]
			\item Hàm số xác định $\Leftrightarrow \heva{& 4-2x \geq 0\\&x^2-6x+5 \neq 0} \Leftrightarrow \heva{& x\leq 2\\& x \neq 1\\&x \neq 5} \Leftrightarrow \heva{& x\leq 2\\&x\neq 1.}$\\
			Vậy tập xác định của hàm số là $\mathscr {D} = \left(-\infty ;2\right] \backslash \{1\}$.
			\item Hàm số xác định $\Leftrightarrow \heva{& x\neq 1\\& \dfrac{x^2}{x-1}\geq 0} \Leftrightarrow \hoac{&x=0\\&x >1.}$\\
			Vậy tập xác định của hàm số là $\mathscr {D} = \{0\} \cup (1;+\infty)$.
		\end{enumerate}
	}
\end{bt}

\begin{bt}%[0D2B1-2]
	Tìm tập xác định các hàm số sau
	\begin{tasks}(2)
		\task $f(x)=\dfrac{4x-1}{\sqrt{2x-5}}$.
		\task $f(x)=\heva{& \dfrac{1}{x-3} & \text{ với }& x\ge 0\\ & 1 & \text{ với }& x<0.}$
	\end{tasks}
	\loigiai{
		\begin{enumerate}[a)]
			\item Hàm số xác định khi và chỉ khi $2x-5>0\Leftrightarrow x>\dfrac{5}{2}$.\\
			Tập xác định $\mathscr{D}=\left(\dfrac{5}{2};+\infty\right)$.
			\item Với $x\ge 0$ thì $f(x)=\dfrac{1}{x-3}$, khi đó $f(x)$ xác định khi $x-3\ne 0\Leftrightarrow x\ne 3$.\\
			Với $x<0$, $f(x)=1$ luôn xác định và nhận giá trị bằng $1$.\\
			Vậy tập xác định $\mathscr{D}=\mathbb{R}\setminus \{3\}$.
		\end{enumerate}	
	}
\end{bt}

\begin{bt}%[0D2B1-3]
	Xét sự biến thiên của hàm số sau trên khoảng $(1;+\infty)$.
	\begin{listEX}[2]
		\item $y=\dfrac{3}{x-1}$.
		\item $y=x+\dfrac{1}{x}$.
	\end{listEX}
	\loigiai{
		\begin{enumerate}[a)]
			\item Với mọi $x_1,\,x_2\in(1;+\infty),\,\,x_1\ne{x_2}$ ta có
			\[f(x_2)-f(x_1)=\dfrac{3}{x_2-1}-\dfrac{3}{x_1-1}=\dfrac{3(x_1-x_2)}{(x_2-1)(x_1-1)}.\]
			Suy ra $\dfrac{f(x_2)-f(x_1)}{x_2-x_1}=-\dfrac{3}{(x_2-1)(x_1-1)}$.\\
			Vì $x_1> 1,\,\,x_2> 1\Rightarrow\dfrac{f(x_2)-f(x_1)}{x_2-x_1}< 0$ nên hàm số $ y=\dfrac{3}{x-1}$ nghịch biến trên khoảng $(1;+\infty)$.
			\item Với mọi $x_1,\,x_2\in(1;+\infty),\,x_1\ne{x_2}$ ta có
			\[f(x_2)-f(x_1)=\left(x_2+\dfrac{1}{x_2}\right)-\left(x_1+\dfrac{1}{x_1}\right)=(x_2-x_1)\left(1-\dfrac{1}{x_1x_2}\right).\]
			Suy ra $\dfrac{f(x_2)-f(x_1)}{x_2-x_1}=1-\dfrac{1}{x_1x_2}$.\\
			Vì $x_1> 1$, $x_2> 1\Rightarrow\dfrac{f(x_2)-f(x_1)}{x_2-x_1}> 0$.\\
			Vậy hàm số $ y=x+\dfrac{1}{x}$ đồng biến trên khoảng $(1;+\infty)$.
		\end{enumerate}
	}
\end{bt}


\begin{bt}%[0D2K1-3]
	Tìm khoảng đồng biến, nghịch biến của các hàm số sau
		\begin{listEX}[3]
			\item  $f(x)=1-3x$.
			\item  $f(x)=\dfrac{1}{x-3}$.
			\item   $f(x)=|2x-1|$.
		\end{listEX}
	\loigiai{
		\begin{enumerate}[a)]
			\item Hàm số có tập xác định $\mathscr{D}=\mathbb{R}$.\\
			Lấy $x_1$, $x_2$ là hai số tuỳ ý thỏa $x_1<x_2$ thì 
			$$f(x_1)-f(x_2)=-3(x_1-x_2)>0$$
			nên hàm số đã chi nghịch biến trên $\mathbb{R}$.
			\item Hàm số $f(x)=\dfrac{1}{x-3}$ xác định khi $x-3\ne 0\Leftrightarrow x\ne 3$\\
			Suy ra tập xác định $\mathscr{D}=\mathbb{R}\setminus \{3\}$.\\
			Lấy $x_1$, $x_2$ là hai số tuỳ ý cùng thuộc mỗi khoảng $(-\infty;3)$, $(3;+\infty)$ sao cho $x_1<x_2$, ta có
			$$f(x_1)-f(x_2)=\dfrac{1}{x_1-3}-\dfrac{1}{x_2-3}=\dfrac{x_2-x_1}{(x_1-3)(x_2-3)}.$$
			Do $x_1<x_2$ nên $x_2-x_1>0$.\\
			Mặt khác, khi lấy $x_1$ và $x_2$ cùng nhỏ hơn $3$ hoặc cùng lớn hơn $3$, ta đều có $x_1-3$ và $x_2-3$ luôn cùng dấu nên $(x_1-3)(x_2-3)<0$.\\
			Suy ra $f(x_1)-f(x_2)>0\Rightarrow f(x_1)>f(x_2)$.\\
			Vậy hàm số $f(x)$ luôn nghịch biến trên các khoảng $(-\infty;3)$ và $(3;+\infty)$.
			\item Hàm số $f(x)=|2x-1|$ còn được viết lại như sau
			$$f(x)=|2x-1|=\heva{& 2x-1 & \text{ với } 2x-1\ge 0\\ & -(2x-1)& \text{với } 2x-1<0}=\heva{& 2x-1 & \text{với }x\ge \dfrac{1}{2}\\ & -2x+1 & \text{với }x<\dfrac{1}{2}.}$$
			Xét hàm số $g(x)=2x-1$. Hàm số này xác định trên $\mathbb{R}$.\\
			Lấy hai số $x_1$, $x_2$ tuỳ ý sao cho $x_1<x_2$, ta có 
			$$x_1<x_2\Rightarrow 2x_1<2x_2\Rightarrow 2x_1-1<2x_2-1\Rightarrow g(x_1)<g(x_2).$$
			Suy ra $g(x)$ đồng biến trên $\mathbb{R}$ nên $f(x)$ đồng biến trên $\left(\dfrac{1}{2};+\infty\right)$.\\
			Xét hàm số $h(x)=-2x+1$. Hàm số này xác định trên $\mathbb{R}$.\\
			Lấy hai số $x_1$, $x_2$ tuỳ ý sao cho $x_1<x_2$, ta có 
			$$ x_1<x_2\Rightarrow -2x_1>-2x_2\Rightarrow -2x_1+1>-2x_2+1\Rightarrow h(x_1)>h(x_2).$$
			Suy ra $h(x)$ nghịch biến trên $\mathbb{R}$ nên $f(x)$ nghịch biến trên $\left(-\infty;\dfrac{1}{2}\right)$.\\
			Vậy hàm số $f(x)=|2x-1|$ đồng biến trên khoảng $\left(\dfrac{1}{2};+\infty\right)$ và nghịch biến trên khoảng $\left(-\infty;\dfrac{1}{2}\right)$.
			
		\end{enumerate}
	}
\end{bt}

\begin{bt}%[0D2B3-3]
	Vẽ đồ thị các hàm số sau
	\begin{tasks}(2)
		\task $f(x)=\heva{& x^2 & \text{ với }& x\le 2\\ & x+2 &\text{với }& x>2.}$
		\task $f(x)=|x+3|-2$.
	\end{tasks}
	\loigiai{
		\begin{enumerate}[a)]
			\item Ta vẽ đồ thị $g(x)=x^2$ và giữ phần đồ thị ứng với $x\le 2$, vẽ đồ thị $h(x)=x+2$ và giữ phần đồ thị ứng với $x>2$. Ta được đồ thị như hình vẽ
			\begin{center}
				\begin{tikzpicture}[scale=0.7, font=\footnotesize, line join=round, line cap=round,>=stealth]
					\def \xmin{-3};
					\def \xmax{6};
					\def \ymin{-1};
					\def \ymax{9};
					\draw[->] (\xmin, 0.) -- (\xmax,0.) node[anchor=north] {$x$};
					\draw[->] (0.,\ymin) -- (0.,\ymax) node[anchor=west] {$y$};
					\clip(\xmin-0.1,\ymin-0.1) rectangle (\xmax+0.1,\ymax+0.1);
					\draw[smooth] plot[domain=-3:2] (\x,{(\x)^2});
					\draw[dashed] plot[domain=2:3] (\x,{(\x)^2});
					\draw[smooth] plot[domain=2:6] (\x,{\x+2});
					\draw[dashed] plot[domain=-3:2] (\x,{\x+2});
					\foreach \i in {-2,-1,1,2,3,4,5}
					\draw[fill=black] (\i,0) circle(1pt) node[below]{$\i$};
					\foreach \j in {1,2,3,4,5,6,7,8}
					\draw[fill=black] (0,\j) circle(1pt) node[left]{$\j$};
					\draw[fill=black] (0,0) circle(1pt) node[below left]{$O$} (4,4) node[above]{$y=f(x)$} ;
					
				\end{tikzpicture}
			\end{center}
			\item Ta có $f(x)=|x+3|-2=\heva{& x+3-2 & \text{với } x+3\ge 0\\ & -(x+3)-2& \text{với }x+3<0}=\heva{& x+1& \text{với }x\ge -3\\ & -x-5& \text{với } x<-3.}$\\
			Ta vẽ đồ thị $g(x)=x+1$ và giữ lại phần đồ thị ứng với $x\ge -3$, đồng thời vẽ đồ thị $h(x)=-x-5$ và giữ lại phần đồ thị ứng với $x<-3$. Ta được đồ thị của $f(x)$ như sau
			\begin{center}
				\begin{tikzpicture}[scale=0.7, font=\footnotesize, line join=round, line cap=round,>=stealth]
					\def \xmin{-6};
					\def \xmax{2};
					\def \ymin{-5};
					\def \ymax{3};
					\draw[->] (\xmin, 0.) -- (\xmax,0.) node[anchor=north] {$x$};
					\draw[->] (0.,\ymin) -- (0.,\ymax) node[anchor=west] {$y$};
					\clip(\xmin-0.1,\ymin-0.1) rectangle (\xmax+0.1,\ymax+0.1);
					\draw[smooth] plot[domain=-3:2] (\x,{\x+1});
					\draw[dashed] plot[domain=-6:-3] (\x,{\x+1});
					\draw[smooth] plot[domain=-6:-3] (\x,{-\x-5});
					\draw[dashed] plot[domain=-3:2] (\x,{-\x-5});
					\foreach \i in {-5,-4,-3,-2,-1,1}
					\draw[fill=black] (\i,0) circle(1pt) node[below]{$\i$};
					\foreach \j in {1,2,-1,-2,-3,-4,-5}
					\draw[fill=black] (0,\j) circle(1pt) node[left]{$\j$};
					\draw[fill=black] (0,0) circle(1pt) node[below left]{$O$} (1,1) node[below]{$y=f(x)$} (-3,-2) circle(1pt);
					
				\end{tikzpicture}
			\end{center}
		\end{enumerate}
	}
\end{bt}

\begin{bt}%[0D2T1-1]
	Một lớp muốn thuê một chiếc xe khách cho chuyến tham quan với tổng đoạn đường cần di chuyển trong khoảng từ $550$ km đến $600$ km, có hai công ty được tiếp cận để tham khảo giá. Công ty A có giá khởi đầu là $3{,}75$ triệu đồng cộng thêm $5000$ đồng cho mỗi km chạy xe. Công ty B có giá khởi đầu là $2{,}5$ triệu đồng cộng thêm $7500$ đồng cho mỗi km chạy xe. Lớp đó nên chọn công ty nào để chi phí là thấp nhất?
	\loigiai{
		Hàm số biểu thị giá tiền theo km của công ty A là $y_A= 3750+5x$ (đồng).\\
		Hàm số biểu thị giá tiền theo km của công ty B là $y_B=2500 + 7{,}5x$ (đồng).\\
		Với $550 \le x \le 600$.\\
		Ta có $y_A -y_B = 1250 -2{,}5x$.\\
		Do $550 \le x \le 600 \Leftrightarrow -250 \le 1250 -2{,}5x \le -120$ nên $y_A -y_B <0$.\\
		Vậy chi phí thuê xe công ty A thấp hơn công ty B.
	}

\end{bt}

\begin{bt}
	Một người đang dự định đi mua xe máy mà muốn chọn 1 trong hai loại xe sau:
	\begin{itemize}
		\item [$\bullet$] \textbf{Loại 1:} Có giá 27000000 (đồng) và trung bình số ki-lô-mét đi được mỗi lít xăng là 58 km/lít xăng
		\item [$\bullet$] \textbf{Loại 2:} Có giá 30000000 (đồng) và trung bình số ki-lô-mét đi được mỗi lít xăng là 62,5 km/lít xăng.
	\end{itemize}
	Biết rằng giá trung bình của 1 lít xăng là 18000 (đồng). Người ta dự tính mua xe máy để sử dụng khoảng 8 năm, mỗi năm người đó ước chừng đi khoảng 7250 km.
	\begin{tasks}(1)
		\task Gọi $s$ (đồng) là chi phí từng năm theo thời gian $t$ (năm) của mỗi loại xe (bao gồm tiền mua xe và tiền xăng). Lập hàm số của $s$ theo $t$.
		\task Nên chọn loại xe nào để tiết kiệm hơn? Tại sao?
	\end{tasks}
	\loigiai{
		\begin{enumerate}[a)]
			\item Đối với xe loại 1, mổi năm xe tiêu thụ hết
			$7250: 58=125$ (lít).
			Suy ra mỗi năm xe loại 1 tiêu thụ hết
			$125 \times  18000=2250000$ (đồng).\\
			Hàm số của $s$ theo $t$ đối với xe loại 1 là
			$$
			s=27000000+2250000 . t
			$$
			Đối với xe loại 2, mỗi năm xe tiêu thụ hết
			$
			7250: 62,5=116$ (lít). Suy ra mỗi năm xe loại 2 tiêu thụ hết
			$116\times  18000=2088000$ (đồng).\\
			Hàm số của s theo t đối với xe loại 2 là
			$$
			s=30000000+2088000.t
			$$
			\item Trong thời gian sử dụng 8 năm $(t=8)$, xe loại 1 tiêu thụ hết
			$$
			s=27000000+2250000\times 8=45000000 \text { (đồng). }
			$$
			Trong thời gian sử dụng 8 năm $(t=8)$, xe loại 2 tiêu thụ hết
			$$
			s=30000000+2088000\times 8=46704000 \text { (đồng). }
			$$
			Vậy nên chọn xe loại 1 để tiết kiệm hơn
		\end{enumerate}
	}
\end{bt}

\begin{bt}
	Bảng giá cước của một hãng Taxi như sau:
\begin{center}
	\begin{tikzpicture}[font=\small , thick, >=latex]
		\def\drong{5.5}
		\def\fonta{ \fontsize{10pt}{0pt}\selectfont } %bảng giá cước
		\def\fontb{ \fontsize{8pt}{0pt}\selectfont }%% Chữ màu vàng
		\def\fontc{ \fontsize{11pt}{0pt}\selectfont }%% Số đỏ
		\def\fontd{ \fontsize{5pt}{0pt}\selectfont }%% Số nhỏ
		
		\path (0,0)coordinate(O)
		($(O)+(-\drong,0)$)coordinate(NW)
		($(O)+(\drong,0)$)coordinate(NE)
		;
		\filldraw  [rounded corners=0.5cm, green!70!blue!70!black] (NW) rectangle ($(NE)+(0,-4.5)$) ; %% nền xanh lá
		\filldraw [yellow!40!green] ($(NW)+(0,-0.8)$) rectangle ($(NE)+(0,-3.4)$) ; % nền xanh lá nhạt
		\filldraw [white] ($(NW)+(0.2,-1)$) rectangle ($(NE)+(-0.2,-3.4)$) ; % Nền trắng
		\draw [line width=1pt, green!70!blue!90!black]
		($(NW)+(0.2,-1)$) -- ($(NE)+(-0.2,-1)$) 
		($(NW)+(0.2,-1.9)$) -- ($(NE)+(-0.2,-1.9)$) 
		($(NW)+(0.2,-2.7)$) -- ($(NE)+(-0.2,-2.7)$) 
		($(NW)+(0.2,-3.4)$) -- ($(NE)+(-0.2,-3.4)$) 
		($(NW)+(0.2,-1)$) -- ($(NW)+(0.2,-3.4)$) 
		($(NE)+(-0.2,-1)$) -- ($(NE)+(-0.2,-3.4)$) 
		($(NE)+(-3.5,-1)$) -- ($(NE)+(-3.5,-3.4)$)
		($(O)+(-1.7,-1)$) -- ($(O)+(-1.7,-2.7)$)
		;
		
		\draw ($(O)+(0,-0.4)$)node[ white]{\fonta\bfseries \sffamily Bảng Giá Cước
			\fontb\bfseries\textcolor{yellow!90!black}{ - Taxi Fare Quote
		}};
		
		\path ($(O)+(-0.65*\drong,-1.3)$)node[red!90!black, xscale=0.8]{\fontb \bfseries GIÁ MỞ CỬA}
		($(O)+(0.65*\drong,-1.3)$)node[red!90!black, xscale=0.8]{\fontb \bfseries TỪ KM THỨ 31}
		($(O)+(0,-1.3)$)node[red!90!black, xscale=0.8]{\fontb \bfseries GIÁ KM TIẾP THEO}
		($(O)+(-0.65*\drong,-1.7)$)node[black, xscale=0.7]{\fontd\bfseries First 0.7km }
		($(O)+(0.65*\drong,-1.7)$)node[black, xscale=0.7]{\fontd\bfseries From 31st km }
		($(O)+(0,-1.7)$)node[black, xscale=0.7]{\fontd\bfseries Each additional 0.8 km up to 30th km}
		
		
		($(O)+(0.65*\drong,-2.9)$)node[red!90!black, xscale=0.7]{\fontd\bfseries\sffamily GIÁ TIỀN ĐÃ BAO GỒM 10\% THUẾ VAT }
		($(O)+(0.65*\drong,-3.2)$)node[black, xscale=0.9]{\fontd\bfseries\sffamily (10\% VAT  INCLUDED) }
		($(O)+(-0.3*\drong,-3)$)node[red!90!black, xscale=0.7]{\fontb\bfseries\sffamily Phí thời gian chờ \textcolor{black}{(Each 5 minutes of wait time: VNĐ 3000)}  }
		
		;
		
		
		\path ($(O)+(-0.65*\drong,-2.3)$)node[red!90!black, yscale=1.3]{\fontc \bfseries \sffamily 11.000Đ/ 0.7Km}
		($(O)+(0.65*\drong,-2.3)$)node[red!90!black, yscale=1.3]{\fontc \bfseries \sffamily 12.500Đ/ 1Km}
		($(O)+(0,-2.3)$)node[red!90!black, yscale=1.3]{\fontc \bfseries \sffamily 15.800Đ/ 1Km} ;	
		
		\path ($(O)+(0,-3.6)$)node[yellow!90!black]{\fontd\sffamily\bfseries QUÝ KHÁCH VUI LÒNG THANH TOÁN PHÍ CẦU ĐƯỜNG, PHÀ VÀ BẾN BẢI (NẾU CÓ) }
		($(O)+(0,-3.85)$)node[white]{\fontd\sffamily\bfseries All tolls, road \& bridge use charge or parking fee shall be surcharged (if any) }
		($(O)+(0,-4.1)$)node[yellow!90!black]{\fontd\sffamily\bfseries TAXI MAI LINH CAM KẾT TÍNH GIÁ CƯỚC THEO ĐỒNG HỒ TÍNH TIỀN }
		($(O)+(0,-4.3)$)node[white]{\fontd\sffamily\bfseries Metter - bassed Fare }
		;
		
		
	\end{tikzpicture}
\end{center}
	\begin{tasks}(1)
		\task Gọi $y$ (đồng) là số tiền khách hàng phải trả sau khi đi $x$ (km). Lập hàm số của $y$ theo $x$ (giả sử rằng không có phát sinh chi phí khác).
		\task Một hành khách thuê taxi đi quãng đường 40 km phải trả số tiền là bao nhiêu?
	\end{tasks}
	\loigiai{
	\begin{enumerate}[a)]
		\item Nếu quãng đường khách hàng đi không quá $0,7$ km, ta có hàm số là $y=11000$.\\
		Nếu quãng đường khách hàng đi trên $0,7$ km đến $30$ km, ta có hàm số là $y=11000+(x-0,7) \cdot 15800=15800x-60$.\\
		Nếu quãng đường khách hàng đi trên $30$ km, ta có hàm số là $y=11000+(30-0,7) \cdot 15800+(x-30) \cdot 12500=12500x+98940
		$.\\
		Vậy ta có hàm số là 
		$$y=\heva{& 11000 &\text{ nếu }&  0<x \le 0,7 \\& 15800x-60 &\text{ nếu }&  0,7<x \le 30\\& 12500x+98940 &\text{ nếu }&  x>30}$$
		\item Thay $x=40$ vào công thức $y=12500xx+98940$ (vì $40 \mathrm{~km}>30 \mathrm{~km}$ ), ta được
		$
		y=12500.40+98940=598940
		$.\\
		Vậy hành khách phải trả số tiền là 598940 đồng.
\end{enumerate}}
\end{bt}
% \subsection{BÀI TẬP TRẮC NGHIỆM}
\Opensolutionfile{ans}[ans/ans0D6-B1]

\begin{ex}%[0D2Y1-1]
	Điểm nào sau đây thuộc đồ thị hàm số $y = 2x^2 + x - 3$?
	\choice
	{\True $\left(0;-3\right)$}
	{$\left(-2;1\right)$}
	{$\left(-1;0\right)$}
	{$\left(3;-7\right)$}
	\loigiai{Thử tọa độ các điểm vào hàm số ta được điểm $\left(0;-3\right)$ thuộc đồ thị hàm số.
	}
\end{ex}


\begin{ex}%[0D2Y1]
	Điểm nào sau đây thuộc đồ thị hàm số  $y=3x^3-2x+1$?
	\choice
	{$\left( -1;2 \right)$}
	{$\left( 1;1 \right)$}
	{$\left( 0;0 \right)$}
	{\True  $\left( 1;2 \right)$}
	\loigiai
	{\begin{itemize}
			\item Với $ x = -1 \Rightarrow y = 0 $ nên $ (-1;2) $ không thuộc đồ thị hàm số $y=3x^3-2x+1$.
			\item Với $ x = 1 \Rightarrow y = 2 $ nên $ (1;2) $ thuộc đồ thị hàm số $y=3x^3-2x+1$.
		\end{itemize}
	}
\end{ex}

\begin{ex}%[0D2B1]
	Tìm tập xác định $\mathcal{D}$ của hàm số $y=\dfrac{x-1}{x-2}$.
	\choice
	{\True $\mathcal{D}=\mathbb{R}\setminus\{2\}$}
	{$\mathcal{D}=\mathbb{R}\setminus\{1\}$}
	{$\mathcal{D}=\mathbb{R}$}
	{$\mathcal{D}=\mathbb{R}\setminus\{1;2\}$}
	\loigiai{
		\begin{itemize}
			\item [$\bullet$] Điều kiện xác định $x-2 \ne 0 \Leftrightarrow x \ne 2$.
			\item [$\bullet$] Suy ra, tập xác định là $\mathcal{D}=\mathbb{R}\setminus\{2\}$.
		\end{itemize}
	}
\end{ex}

\begin{ex}%[0D2B1]
	Tìm tập xác định $\mathcal{D}$ của hàm số $y=\dfrac{x-2}{x^2-2x+2}$.
	\choice
	{$\mathcal{D}=\mathbb{R}\setminus\{1\}$}
	{$\mathcal{D}=\mathbb{R}\setminus\{2\}$}
	{\True $\mathcal{D}=\mathbb{R}$}
	{$\mathcal{D}=\mathbb{R}\setminus\{1;2\}$}
	\loigiai{
		\begin{itemize}
			\item [$\bullet$] Điều kiện xác định $x^2-2x+2 \ne 0$ (luôn đúng).
			\item [$\bullet$] Suy ra, tập xác định là $\mathcal{D}=\mathbb{R}$.
		\end{itemize}
	}
\end{ex}

\begin{ex}%[0D2B1]
	Tìm tập xác định $\mathcal{D}$ của hàm số $y=\sqrt{x-2}$.
	\choice
	{$\mathcal{D}=\mathbb{R}\setminus\{2\}$}
	{$\mathcal{D}=(2;+\infty)$}
	{$\mathcal{D}=(-\infty;2)$}
	{\True $\mathcal{D}=[2;+\infty)$}
	\loigiai{
		\begin{itemize}
			\item [$\bullet$] Điều kiện xác định $x-2 \ge 0 \Leftrightarrow x \ge 2$.
			\item [$\bullet$] Suy ra, tập xác định là $\mathcal{D}=[2;+\infty)$.
		\end{itemize}
	}
\end{ex}

\begin{ex}%[0D2B1]
	Tìm tập xác định của hàm số $y=\dfrac{2x+3}{x^2-x}$.
	\choice{$\mathbb{R}\setminus\{1\}$}
	{$\mathbb{R}$}
	{$\mathbb{R}\setminus\{0\}$}
	{\True $\mathbb{R}\setminus\{0,1\}$}
	\loigiai{
			\begin{itemize}
				\item [$\bullet$] Điều kiện xác định $x^2-x \ne 0 \Leftrightarrow x \ne 0$ và $x \ne 1$.
				\item [$\bullet$] Suy ra, tập xác định là $\mathcal{D}=\mathbb{R}\setminus\{0,1\}$.
			\end{itemize}
	}
\end{ex}

\begin{ex}%[0D2B1-3]%
	\immini{Cho hàm số $y=f(x)$ có tập xác định là $[-3;3]$ và đồ thị của nó được biểu diễn bởi hình bên. Khẳng định nào sau đây đúng?
	\choice
	{\True Hàm số đồng biến trên khoảng $(-3;-1)$}
	{Hàm số đồng biến trên khoảng $(-3;3)$}
	{Hàm số đồng biến trên khoảng $(-3;0)$}
	{Hàm số nghịch biến trên khoảng $(-1;2)$}}{
	\begin{tikzpicture}[>=stealth,x=1.0 cm,y=1.0 cm, scale=0.75]
		\draw[->] (-3.5,0)--(4,0) node[below left] {$x$};
		\draw[->] (0,-2)--(0,4.5) node[below right] {$y$};
		\foreach \x in {,-3,-2,-1,1,2,3,,}
		\draw[shift={(\x,0)},color=black] (0pt,2pt) -- (0pt,-2pt) node[below] {\footnotesize $\x$};
		\foreach \y in {,-1,1,2,3,4,}
		\draw[shift={(0,\y)},color=black] (2pt,0pt) -- (-2pt,0pt) node[left] {\footnotesize $\y$};
		\draw[color=black] (0pt,-10pt) node[right] {\footnotesize $0$};
		\clip(-4,-2.5) rectangle (4,4.5);
		\draw[dashed] (0,-1)--(-3,-1)--(-3,0);
		\draw[dashed] (-1,0)--(-1,1)--(0,1);
		\draw[dashed] (0,4)--(3,4)--(3,0);
		\draw[magenta,thick] (-3,-1)--(-1,1)--(0,1)--(3,4);
\end{tikzpicture}}
	\loigiai{
		Trên khoảng $(-3;-1)$ và $(1;3)$, đồ thị hàm số đi lên, do đó hàm số đồng biến trên $(-3;-1)$ và $(1;3)$.
	}
\end{ex}

\begin{ex}%[0D2B1-3]%
	Khẳng định nào sau đây về hàm số $y=x^2$ là khẳng định đúng?
	\choice
	{Hàm số nghịch biến trên $\mathbb{R}$}
	{Hàm số đồng biến trên $\mathbb{R}$}
	{Hàm số nghịch biến trên $[0;+\infty)$}
	{\True Hàm số đồng biến trên $[0;+\infty)$}
	\loigiai{
		Hàm số $y=x^2$ có hệ số $a=1>0$ nên hàm đồng biến trên $\left(-\dfrac{b}{2a};+\infty \right) $ hay  $[0;+\infty)$ (vì hàm số chỉ bằng $0$ tại một điểm là $x=0$).
	}
\end{ex}

\begin{ex}
	\immini{Cho hàm số $y=f(x)$ liên tục trên $\mathbb{R}$ và có đồ thị như hình bên. Khẳng định nào sau đây đúng?
	\choice
		{Hàm số đồng biến trên khoảng $(1;3)$}
		{Hàm số nghịch biến trên khoảng $(6;+\infty)$}
		{Hàm số đồng biến trên khoảng $(-\infty;3)$}
		{\True Hàm số nghịch biến trên khoảng $(3;6)$}
	}{
		\begin{tikzpicture}[>=stealth,line cap=round,line join=round,x=1cm,y=1cm,scale=0.7]
			\draw[->] (-1,0)--(4,0) node[below] {$x$};
			\draw[->] (0,-2.2)--(0,1.3) node[right] {$y$};
			\node[below left](0,0){$O$};
			\draw[magenta,thick,samples=1000,domain=-0.1:3.5] plot(\x,{(\x-1)^3-2*(\x-1)^2-(\x)+1.5});
			\draw[dashed] (0.78,0.6) -- (0.78,0) node[below] {$2$};
			\draw[dashed] (2.55,-2.13) -- (2.55,0) node[above] {$7$};
	\end{tikzpicture}}
	\loigiai{
		Dựa vào đồ thị thấy hàm số nghịch biến trên khoảng $(2;7)$, do đó hàm số nghịch biến trên khoảng $(3;6)$.
	}
\end{ex}

\begin{ex}%[0D2Y1]
	Tập xác định của hàm số $y=\dfrac{x^2+\sqrt{3-x}}{x-2}$ là
	\choice
	{$(-\infty;3)\backslash \{2\}$}
	{$(2;3]$}
	{\True $(-\infty;3]\backslash \{2\}$}
	{$(-\infty;3]$}
	\loigiai{
		\begin{itemize}
			\item [$\bullet$] Điều kiện xác định $\heva{& 3-x \ge 0\\&x-2 \ne 0} \Leftrightarrow \heva{& x \le 3\\&x \ne 2}$.
			\item [$\bullet$] Suy ra, tập xác định là $\mathcal{D}=(-\infty;3]\backslash \{2\}$.
		\end{itemize}
	}
\end{ex}

\begin{ex}%[0D2B1]
	Tìm tập xác định của hàm số $y=\sqrt{3+x}+\sqrt{6-x}$.
	\choice
	{\True $[-3;6]$}
	{$(-3;6)$}
	{$(-\infty;-3)\cup (6;+\infty)$}
	{$\mathbb{R}\backslash(-3;6)$}
	\loigiai{
		\begin{itemize}
			\item [$\bullet$] Điều kiện xác định $\heva{& 3+x \ge 0\\& 6-x \ge 0} \Leftrightarrow -3\le x\le 6$.
			\item [$\bullet$] Suy ra, tập xác định là $\mathscr{D}=[-3;6]$.
		\end{itemize}
	}
\end{ex}

\begin{ex}%[0D2B1-2]
	Tập xác định của hàm số $y=\dfrac{x+2}{\sqrt{x-1}}+\sqrt{3-x}$ là
	\choice
	{$[1;3]$}
	{\True $(1;3]$}
	{$(-\infty;3]$}
	{$(1;+\infty)$}
	\loigiai{
		$$\dfrac{x+2}{\sqrt{x-1}}+\sqrt{3-x}\text{ có nghĩa }\Leftrightarrow \heva{&x-1>0\\&3-x\geq 0}\Leftrightarrow 1<x\leq 3.$$
		Vậy tập xác định của hàm số đã cho là $(1;3]$.
	}
\end{ex}

\begin{ex}
	\immini{Cho hàm số $y=f(x)$ liên tục trên $\mathbb{R}$ và có đồ thị như hình bên. Tính giá trị biểu thức $P=2f(1)+f(4)-f(3)$
		\haicot
		{$P=1$}
		{$P=0$}
		{\True $P=2$}
		{$P=4$}
	}{\hspace{1cm}
		\begin{tikzpicture}[smooth,samples=300,scale=0.6,>=stealth]
		\draw[black!30!] (-2,-1.5) grid (4,3);
		\draw[->] (-2,0)--(5,0) node[below]{$x$};
		\draw[->] (0,-1.5)--(0,3) node[right]{$y$};
		\foreach \x in {1,2,3,4}{
			\draw (\x,0) node[below]{$\x$};%Ox
		}
		\foreach \y in {1}{
			\draw (0,\y) node[left]{$\y$};%Oy
		}
		\draw (0,0) node[below left]{$O$};
		\draw[domain=-1.8:1,thick] plot(\x,{-(\x)^2+2});
		\draw [magenta,thick](1,1)--(4,1) node[above]{\small $y=f(x)$};
		\draw[fill=black] (0,2) circle(1.5pt) (1,1) circle(1pt);
		\end{tikzpicture}}
		\loigiai{
		Quan sát đồ thị, ta có các kết quả $f(1)=1$, $f(3)=1$ và $f(4)=1$ nên
		$$P=2f(1)+f(4)-f(3)=2+1-1=2.$$
	}
\end{ex}
\begin{ex}%[0D2Y1-1]
	Cho hàm số $y=f(x)=\heva{&\sqrt{x+4}&&\text{ khi } x>1\\&x^2+1&&\text{ khi } -1\leq x\leq 1\\&2x-1&&\text{ khi } x<-1}$. Giá trị $f(0)$ bằng
	\choice
	{$-2$}
	{$2$}
	{$-1$}
	{\True $1$}
	\loigiai{
		Ta có $f(0)=0^2+1=1$.
	}
\end{ex}

\begin{ex}%[0D2Y1-1]
	Cho hàm số $y=\heva{&2x+1 &\text{khi}\ x\le 2\\ &x^2-3 &\text{khi}\ x>2}$. Trong các điểm sau đây, điểm nào thuộc đồ thị hàm số?
	\choice
	{\True $(0;1)$}
	{$(0;-3)$}
	{$(3;7)$}
	{$(-3;6)$}
	\loigiai{
		Điểm $(0;1)$ thuộc đồ thị hàm số.
	}
\end{ex}

\begin{ex}
	\immini{Cho đồ thị hàm số $y=f(x)$ trên miền $[-3;5]$ như hình bên. Trong các điểm sau, điểm nào thuộc đồ thị hàm số đã cho?
	\haicot
	{$A(4;1)$}
	{$B(1;1)$ }
	{\True $C(3;3)$}
	{$D(0;2)$ }}{
	\begin{tikzpicture}[smooth,samples=300,scale=0.6,>=stealth,font=\footnotesize]
		\draw[line width=0.1pt,gray!80] (-4,-2) grid (5,4);
		\draw[->] (-4.2,0)--(5.7,0) node[below]{$x$};
		\draw[->] (0,-2.2)--(0,4.5) node[right]{$y$};
		\draw (0,0) node[above right]{$O$};
		\draw[line width=1pt,domain=-1:3] plot(\x,{(\x-1)^2-1});
		\draw[line width=1pt] (-3,-1)--(-1,3) (3,3)--(5,1);
		\draw[fill=magenta] (-3,-1) circle(2.5pt) (-1,3) circle(2.5pt) (3,3) circle(2.5pt) (5,1) circle(2.5pt) (1,-1) circle(2.5pt);
		\foreach \x in {-4,-3,-2,-1,1,2,3,4,5}\draw (\x,0.1)--(\x,-0.1) node [below] {\footnotesize $\x$};
		\foreach \y in {-2,-1,1,2,3}\draw (0.1,\y)--(-0.1,\y) node [left] {\footnotesize $\y$};
\end{tikzpicture}}
\loigiai{}
\end{ex}

\begin{ex}
	\immini{Cho đồ thị hàm số $y=f(x)$ trên miền $\mathscr{D}=[-3;5]$ như hình bên. Tập giá trị của hàm số này trên miền $\mathscr{D}$ là
		\haicot
		{$[-3;5]$}
		{$[-2;5]$ }
		{$[-3;3]$}
		{\True $[-2;2]$}}{
		\begin{tikzpicture}[smooth,samples=300,scale=0.6,>=stealth,font=\footnotesize]
			\draw[line width=0.1pt,gray!80] (-4,-2) grid (5,4);
			\draw[->] (-4.2,0)--(5.7,0) node[below]{$x$};
			\draw[->] (0,-2.2)--(0,4.5) node[right]{$y$};
			\draw (0,0) node[above right]{$O$};
			\draw[line width=1pt,domain=0:3] plot(\x,{-(\x-1)^2+2});
			\draw[line width=1pt] (-3,2)--(0,1) (3,-2)--(5,1);
			\draw[fill=magenta] (-3,2) circle(2.5pt) (0,1) circle(2.5pt) (3,-2) circle(2.5pt) (5,1) circle(2.5pt) ;
			\foreach \x in {-4,-3,-2,-1,1,2,3,4,5}\draw (\x,0.1)--(\x,-0.1) node [below] {\footnotesize $\x$};
			\foreach \y in {-2,-1,1,2,3}\draw (0.1,\y)--(-0.1,\y) node [left] {\footnotesize $\y$};
	\end{tikzpicture}}
	\loigiai{}
\end{ex}


\begin{ex}%[0D2B1]
	Cho hàm số $y=\dfrac{x+1}{x-1}$. Tìm tọa độ điểm thuộc đồ thị của hàm số có tung độ bằng $-2$.
	\choice        
	{$(0;-2)$}
	{\True $\left(\dfrac{1}{3};-2\right)$}
	{$(-2;-2)$}
	{$(-1;-2)$}
	\loigiai{
		Thay $y=-2$ vào phương trình hàm số $y=\dfrac{x+1}{x-1}$ ta được $x=\dfrac{1}{3}$.
	}
\end{ex}
\begin{ex}%[Phan Anh]%[0D2B1-1]
	Điểm nào sau đây thuộc đồ thị hàm số $y=\dfrac{1}{x-1}$?
	\choice
	{\True $M_1(2;1)$}
	{$M_2(1;1)$}
	{$M_3(2;0)$}
	{$M_4(0;-2)$}
	\loigiai{
		Xét điểm $M_1$, thay $x=2$ và $y=1$
		vào hàm số $y=\dfrac{1}{x-1}$ ta được $1=\dfrac{1}{2-1}$ ta thấy đúng nên nhận $M_1$.}
\end{ex}
\begin{ex}%[Phan Anh]%[0D2B1-1]
	Điểm nào sau đây \textbf{không} thuộc đồ thị hàm số $y=\dfrac{\sqrt{x^2-4x+4}}{x}$?
	\choice
	{$A\left(2;0\right)$}
	{$B\left(3;\dfrac{1}{3}\right)$}
	{\True $C\left(1;-1\right)$}
	{$D\left(-1;-3\right)$}
	\loigiai{Thay từng đáp án vào hàm số $y=\dfrac{\sqrt{x^2-4x+4}}{x}$.
		\begin{itemize}
			\item Với $x=2$ và $y=0$, ta được $0=\dfrac{\sqrt{2^2-4.2+4}}{2}$ (đúng).
			\item Với $x=3$ và $y=\dfrac{1}{3}$, ta được $\dfrac{1}{3}=\dfrac{\sqrt{3^2-4\cdot3+4}}{3}$ (đúng).
			\item Với thay $x=1$ và $y=-1$, ta được $-1=\dfrac{\sqrt{1^2-4\cdot1+4}}{1}\Leftrightarrow-1=1$ (sai).
		\end{itemize}}
\end{ex}
\begin{ex}%[Phan Anh]%[0D2B1-1]
	Cho hàm số $y=f(x)=|-5x|$. Khẳng định nào sau đây là \textbf{sai}?
	\choice
	{$f(-1)=5$}
	{$f(2)=10$}
	{$f(-2)=10$}
	{\True $f\left(\dfrac{1}{5}\right)=-1$}
	\loigiai{Ta có
		\begin{itemize}
			\item $f(-1)=|-5\cdot(-1)|=|5|=5$.
			\item $f(2)=|-5\cdot2|=|-10|=10$.
			\item $f(-2)=|-5\cdot(-2)|=|10|=10$.
			\item $f\left(\dfrac{1}{5}\right)=\left|-5\cdot\dfrac{1}{5}\right|=|-1|=1$
		\end{itemize}
		Cách khác: Vì hàm đã cho là hàm trị tuyệt đối nên không âm. Do đó $f\left(\dfrac{1}{5}\right)=-1$ là sai.}
\end{ex}
\begin{ex}%[Phan Anh]%[0D2B1-1]
	Cho hàm số $f(x)=\left\{\begin{array}{*{35}{l}}
			\dfrac{2}{x-1} & , x\in(-\infty;0) \\
			\sqrt{x+1}     & , x\in[0;2]       \\
			x^2-1          & , x\in(2;5]
		\end{array}\right.$. Tính giá trị của $f(4)$.
	\choice
	{$f(4)=\dfrac{2}{3}$}
	{\True $f(4)=15$}
	{$f(4)=\sqrt{5}$}
	{Không tính được}
	\loigiai{Do $4\in(2;5]$ nên $f(4)=4^2-1=15$.}
\end{ex}
\begin{ex}%[Phan Anh]%[0D2B1-1]
	Cho hàm số $f(x)=\left\{\begin{array}{*{35}{l}}
			\dfrac{2\sqrt{x+2}-3}{x-1} & , x\ge 2 \\
			x^2 +1                     & , x<2
		\end{array}\right.$. Tính $P=f(2)+f(-2)$.
	\choice
	{$P=\dfrac{8}{3}$}
	{$P=4$}
	{\True $P=6$}
	{$P=\dfrac{5}{3}$}
	\loigiai{\begin{itemize}
			\item Khi $x\ge 2$ thì $f(2)=\dfrac{2\sqrt{2+2}-3}{2-1}=1$.
			\item Khi $x<2$ thì $f(-2)=(-2)^2+1=5$.
		\end{itemize}
		Vậy $f(2)+f(-2)=6$.}
\end{ex}
\begin{ex}%[Phan Anh]%[0D2B1-2]
	Tìm tập xác định $\mathscr{D}$ của hàm số $y=\dfrac{3x-1}{2x-2}$.
	\choice
	{$\mathscr{D}=\mathbb{R}$}
	{$\mathscr{D}=(1;+\infty)$}
	{\True $\mathscr{D}=\mathbb{R}\setminus\{1\}$}
	{$\mathscr{D}=[1;+\infty)$}
	\loigiai{
		Hàm số xác định khi $2x-2\ne0\Leftrightarrow x\ne1$.\\
		Vậy tập xác định của hàm số là $\mathscr{D}=\mathbb{R}\setminus\{1\}$.}
\end{ex}
\begin{ex}%[Phan Anh]%[0D2B1-2]
	Tìm tập xác định $\mathscr{D}$ của hàm số $y=\dfrac{2x-1}{(2x+1)(x-3)}$.
	\choice
	{$\mathscr{D}=(3;+\infty)$}
	{\True $\mathscr{D}=\mathbb{R}\setminus\left\{-\dfrac{1}{2};3\right\}$}
	{$\mathscr{D}=\left(-\dfrac{1}{2};+\infty\right)$}
	{$\mathscr{D}=\mathbb{R}$}
	\loigiai{
		Hàm số xác định khi $\heva{
				& 2x+1\ne 0 \\
				& x-3\ne 0}\Leftrightarrow \heva{
				& x\ne-\dfrac{1}{2} \\
				& x\ne 3.}$\\
		Vậy tập xác định của hàm số là $ \mathscr{D}=\mathbb{R}\setminus\left\{-\dfrac{1}{2};3\right\}$}
\end{ex}
\begin{ex}%[Phan Anh]%[0D2B1-2]
	Tìm tập xác định $\mathscr{D}$ của hàm số $y=\dfrac{x^2+1}{x^2+3x-4}$.
	\choice
	{$\mathscr{D}=\{1;-4\}$}
	{\True $\mathscr{D}=\mathbb{R}\setminus\{1;-4\}$}
	{$\mathscr{D}=\mathbb{R}\setminus\{1;4\}$}
	{$\mathscr{D}=\mathbb{R}$}
	\loigiai{
		Hàm số xác định khi $x^2+3x-4\ne 0\Leftrightarrow \heva{
				& x\ne 1 \\
				& x\ne-4}.$\\
		Vậy tập xác định của hàm số là $\mathscr{D}=\mathbb{R}\setminus\{1;-4\}$.}
\end{ex}
\begin{ex}%[Phan Anh]%[0D2B1-2]
	Tìm tập xác định $\mathscr{D}$ của hàm số $y=\dfrac{x+1}{(x+1)(x^2+3x+4)}$.
	\choice
	{$\mathscr{D}=\mathbb{R}\setminus\left\{1\right\}$}
	{$\mathscr{D}=\left\{-1\right\}$}
	{\True $\mathscr{D}=\mathbb{R}\setminus\left\{-1\right\}$}
	{$\mathscr{D}=\mathbb{R}$}
	\loigiai{
		Hàm số xác định khi $\heva{
				& x+1\ne 0 \\
				& x^2+3x+4\ne 0}\Leftrightarrow x\ne-1$.\\
		Vậy tập xác định của hàm số là $\mathscr{D}=\mathbb{R}\setminus\left\{-1\right\}$.}
\end{ex}
\begin{ex}%[Phan Anh]%[0D2B1-2]
	Tìm tập xác định $\mathscr{D}$ của hàm số $y=\dfrac{2x+1}{x^3-3x+2}$.
	\choice
	{$\mathscr{D}=\mathbb{R}\setminus\left\{1;2\right\}$}
	{\True $\mathscr{D}=\mathbb{R}\setminus\left\{-2;1\right\}$}
	{$\mathscr{D}=\mathbb{R}\setminus\left\{-2\right\}$}
	{$\mathscr{D}=\mathbb{R}$}
	\loigiai{
		Hàm số xác định khi
		\begin{align*}
			                & x^3-3x+2\ne 0
			\Leftrightarrow (x-1)(x^2+x-2)\ne 0                         \\
			\Leftrightarrow & \,\heva{                       & x-1\ne 0 \\
			                & x^2+x-2\ne 0}
			\Leftrightarrow \heva{
			                & x\ne 1                                    \\ & \heva{ & x\ne 1 \\
			                & x\ne-2}}\Leftrightarrow \heva{
			                & x\ne                                      \\
			                & x\ne-2.}
		\end{align*}
		Vậy tập xác định của hàm số là $\mathscr{D}=\mathbb{R}\setminus\left\{-2;1\right\}$.
	}
\end{ex}
\begin{ex}%[Phan Anh]%[0D2B1-2]
	Tìm tập xác định $\mathscr{D}$ của hàm số $y=\sqrt{x+2}-\sqrt{x+3}$.
	\choice
	{$\mathscr{D}=[-3;+\infty)$}
	{\True $\mathscr{D}=\left[-2;+\infty \right)$}
	{$\mathscr{D}=\mathbb{R}$}
	{$\mathscr{D}=\left[2;+\infty \right)$}
	\loigiai{
	Hàm số xác định khi $\heva{
			& x+2\ge 0 \\
			& x+3\ge 0 \\}\Leftrightarrow \heva{
			& x\ge-2 \\
			& x\ge-3 \\}\Leftrightarrow x\ge-2$.\\
	Vậy tập xác định của hàm số là $\mathscr{D}=\left[-2;+\infty \right)$.}
\end{ex}
\begin{ex}%[Phan Anh]%[0D2B1-2]
	Tìm tập xác định $\mathscr{D}$ của hàm số $y=\sqrt{6-3x}-\sqrt{x-1}$.
	\choice
	{$\mathscr{D}=\left(1;2\right)$}
	{\True $\mathscr{D}=\left[1;2\right]$}
	{$\mathscr{D}=\left[1;3\right]$}
	{$\mathscr{D}=\left[-1;2\right]$}
	\loigiai{
		Hàm số xác định khi $\heva{
				& 6-3x\ge 0 \\
				& x-1\ge 0}\Leftrightarrow \heva{
				& x\le 2 \\
				& x\ge 1}\Leftrightarrow 1\le x\le 2$.\\
		Vậy tập xác định của hàm số là $\mathscr{D}=\left[1;2\right]$.}
\end{ex}
\begin{ex}%[Phan Anh]%[0D2B1-2]
	Tìm tập xác định $\mathscr{D}$ của hàm số $y=\dfrac{\sqrt{3x-2}+6x}{\sqrt{4-3x}}$.
	\choice
	{\True $\mathscr{D}=\left[\dfrac{2}{3};\dfrac{4}{3}\right)$}
	{$\mathscr{D}=\left[\dfrac{3}{2};\dfrac{4}{3}\right)$}
	{$\mathscr{D}=\left[\dfrac{2}{3};\dfrac{3}{4}\right)$}
	{$\mathscr{D}=\left(-\infty;\dfrac{4}{3}\right)$}
	\loigiai{
	Hàm số xác định khi $\heva{
			& 3x-2\ge 0 \\
			& 4-3x>0}\Leftrightarrow \heva{
			& x\ge \dfrac{2}{3} \\
			& x<\dfrac{4}{3}}\Leftrightarrow \dfrac{2}{3}\le x<\dfrac{4}{3}$.\\
	Vậy tập xác định của hàm số là $\mathscr{D}=\left[\dfrac{2}{3};\dfrac{4}{3}\right)$.}
\end{ex}
\begin{ex}%[Phan Anh]%[0D2B1-2]
	Tìm tập xác định $\mathscr{D}$ của hàm số $y=\dfrac{x+4}{\sqrt{x^2-16}}$.
	\choice
	{$\mathscr{D}=\left(-\infty;-2\right)\cup \left(2;+\infty \right)$}
	{$\mathscr{D}=\mathbb{R}$}
	{\True $\mathscr{D}=\left(-\infty;-4\right)\cup \left(4;+\infty \right)$}
	{$\mathscr{D}=\left(-4;4\right)$}
	\loigiai{Hàm số xác định khi $x^2-16>0\Leftrightarrow x^2>16\Leftrightarrow \hoac{
				& x>4 \\
				& x<-4}$.\\
		Vậy tập xác định của hàm số là $\mathscr{D}=\left(-\infty;-4\right)\cup \left(4;+\infty \right)$.}
\end{ex}
\begin{ex}%[Phan Anh]%[0D2B1-2]
	Tìm tập xác định $\mathscr{D}$ của hàm số $y=\sqrt{x^2-2x+1}+\sqrt{x-3}$.
	\choice
	{$\mathscr{D}=(-\infty;3]$}
	{$\mathscr{D}=[1;3]$}
	{\True $\mathscr{D}=[3;+\infty)$}
	{$\mathscr{D}=(3;+\infty)$}
	\loigiai{
	Hàm số xác định khi $\heva{
			& x^2-2x+1\ge 0 \\
			& x-3\ge 0}\Leftrightarrow \heva{
			& {\left(x-1\right)}^2\ge 0 \\
			& x-3\ge 0}\Leftrightarrow \heva{
			& x\in \mathbb{R} \\
			& x\ge 3}\Leftrightarrow x\ge 3$.\\
	Vậy tập xác định của hàm số là $\mathscr{D}=\left[3;+\infty \right)$.}
\end{ex}
\begin{ex}%[Phan Anh]%[0D2B1-2]
	Tìm tập xác định $\mathscr{D}$ của hàm số $y=\dfrac{\sqrt{2-x}+\sqrt{x+2}}{x}$.
	\choice
	{$\mathscr{D}=[-2;2]$}
	{$\mathscr{D}=(-2;2)\setminus\left\{0\right\}$}
	{\True $\mathscr{D}=[-2;2]\setminus\left\{0\right\}$}
	{$\mathscr{D}=\mathbb{R}$}
	\loigiai{
		Hàm số xác định khi $\heva{
				& 2-x\ge 0 \\
				& x+2\ge 0 \\
				& x\ne 0}\Leftrightarrow \heva{
				& x\le 2 \\
				& x\ge-2 \\
				& x\ne 0.}$\\
		Vậy tập xác định của hàm số là $\mathscr{D}=\left[-2;2\right]\setminus\left\{0\right\}$.}
\end{ex}
\begin{ex}%[Phan Anh]%[0D2B1-2]
	Tìm tập xác định $\mathscr{D}$ của hàm số $y=\dfrac{\sqrt{x+1}}{x^2-x-6}$.
	\choice
	{$\mathscr{D}=\left\{3\right\}$}
	{\True $\mathscr{D}=\left[-1;+\infty \right)\setminus\left\{3\right\}$}
	{$\mathscr{D}=\mathbb{R}$}
	{$\mathscr{D}=\left[-1;+\infty \right)$}
	\loigiai{
	Hàm số xác định khi $\heva{
			& x+1\ge 0 \\
			& x^2-x-6\ne 0}\Leftrightarrow \heva{
			& x\ge-1 \\
			& x\ne 3 \\
			& x\ne-2}\Leftrightarrow \heva{
			& x\ge-1 \\
			& x\ne 3.}$\\
	Vậy tập xác định của hàm số là $\mathscr{D}=[-1;+\infty)\setminus\left\{3\right\}$.}
\end{ex}
\begin{ex}%[Phan Anh]%[0D2B1-2]
	Tìm tập xác định $\mathscr{D}$ của hàm số $y=\sqrt{6-x}+\dfrac{2x+1}{1+\sqrt{x-1}}$.
	\choice
	{$\mathscr{D}=(1;+\infty)$}
	{\True $\mathscr{D}=[1;6]$}
	{$\mathscr{D}=\mathbb{R}$}
	{$\mathscr{D}=(1;6)$}
	\loigiai{
	Hàm số xác định khi $\heva{
			& 6-x\ge 0 \\
			& x-1\ge 0 \\
			& 1+\sqrt{x-1}\ne 0\left(\text{luôn đúng} \right)}\Leftrightarrow \heva{
			& x\le 6 \\
			& x\ge 1}\Leftrightarrow 1\le x\le 6$.\\
	Vậy tập xác định của hàm số là $\mathscr{D}=[1;6]$.}
\end{ex}
\begin{ex}%[Phan Anh]%[0D2B1-2]
	Tìm tập xác định $\mathscr{D}$ của hàm số $y=\dfrac{x+1}{(x-3)\sqrt{2x-1}}$.
	\choice
	{$\mathscr{D}=\mathbb{R}$}
	{$\mathscr{D}=\left(-\dfrac{1}{2};+\infty \right)\setminus\left\{3\right\}$}
	{$\mathscr{D}=\left[\dfrac{1}{2};+\infty \right)\setminus\left\{3\right\}$}
	{\True $\mathscr{D}=\left(\dfrac{1}{2};+\infty \right)\setminus\left\{3\right\}$}
	\loigiai{
		Hàm số xác định khi $\heva{
				& x-3\ne 0 \\
				& 2x-1>0}\Leftrightarrow \heva{
				& x\ne 3 \\
				& x>\dfrac{1}{2}.}$\\
		Vậy tập xác định của hàm số là $\mathscr{D}=\left(\dfrac{1}{2};+\infty \right)\setminus\left\{3\right\}$.}
\end{ex}
\begin{ex}%[Phan Anh]%[0D2B1-2]
	Tìm tập xác định $\mathscr{D}$ của hàm số $y=\dfrac{\sqrt{x+2}}{x\sqrt{x^2-4x+4}}$.
	\choice
	{\True $\mathscr{D}=[-2;+\infty)\setminus\left\{0;2\right\}$}
	{$\mathscr{D}=\mathbb{R}$}
	{$\mathscr{D}=[-2;+\infty)$}
	{$\mathscr{D}=(-2;+\infty)\setminus\left\{0;2\right\}$}
	\loigiai{
	Hàm số xác định khi $\heva{
			& x+2\ge 0 \\
			& x\ne 0 \\
			& x^2-4x+4>0}\Leftrightarrow \heva{
			& x+2\ge 0 \\
			& x\ne 0 \\
			& (x-2)^2>0}\Leftrightarrow \heva{
			& x\ge-2 \\
			& x\ne 0 \\
			& x\ne 2.}$\\
	Vậy tập xác định của hàm số là $\mathscr{D}=\left[-2;+\infty \right)\setminus\left\{0;2\right\}$.}
\end{ex}
%Câu 21
\begin{ex}%[Phan Anh]%[0D2B1-2]
	Tìm tập xác định $\mathscr{D}$ của hàm số $y=\dfrac{x}{x-\sqrt{x}-6}$.
	\choice
	{$\mathscr{D}=\left[0;+\infty \right)\setminus\left\{3\right\}$}
	{\True $\mathscr{D}=\left[0;+\infty \right)\setminus\left\{9\right\}$}
	{$\mathscr{D}=\left[0;+\infty \right)\setminus\left\{\sqrt{3}\right\}$}
	{$\mathscr{D}=\mathbb{R}\setminus\left\{9\right\}$}
	\loigiai{
	Hàm số xác định khi $\heva{
			& x\ge 0 \\
			& x-\sqrt{x}-6\ne 0}\Leftrightarrow \heva{
			& x\ge 0 \\
			& \sqrt{x}\ne 3}\Leftrightarrow \heva{
			& x\ge 0 \\
			& x\ne 9.}$\\
	Vậy tập xác định của hàm số là $\mathscr{D}=\left[0;+\infty \right)\setminus\left\{9\right\}$.}
\end{ex}
\begin{ex}%[Phan Anh]%[0D2B1-2]
	Tìm tập xác định $\mathscr{D}$ của hàm số $y=\dfrac{\sqrt[3]{x-1}}{x^2+x+1}$.
	\choice
	{$\mathscr{D}=\left(1;+\infty \right)$}
	{$\mathscr{D}=\left\{1\right\}$}
	{\True $\mathscr{D}=\mathbb{R}$}
	{$\mathscr{D}=\left(-1;+\infty \right)$}
	\loigiai{
		Hàm số xác định khi $x^2+x+1\ne 0$ luôn đúng với mọi $x\in \mathbb{R}$.\\
		Vậy tập xác định của hàm số là $\mathscr{D}=\mathbb{R}$.}
\end{ex}
\begin{ex}%[Phan Anh]%[0D2B1-2]
	Tìm tập xác định $\mathscr{D}$ của hàm số $y=\dfrac{\sqrt{x-1}+\sqrt{4-x}}{\left(x-2\right)\left(x-3\right)}$.
	\choice
	{$\mathscr{D}=\left[1;4\right]$}
	{$\mathscr{D}=\left(1;4\right)\setminus\left\{2;3\right\}$}
	{\True $\mathscr{D}=\left[1;4\right]\setminus\left\{2;3\right\}$}
	{$\mathscr{D}=\left(-\infty;1\right]\cup \left[4;+\infty \right)$}
	\loigiai{
		Hàm số xác định khi $\heva{
				& x-1\ge 0 \\
				& 4-x\ge 0 \\
				& x-2\ne 0 \\
				& x-3\ne 0}\Leftrightarrow \heva{
				& x\ge 1 \\
				& x\le 4 \\
				& x\ne 2 \\
				& x\ne 3}\Leftrightarrow \heva{
				& 1\le x\le 4 \\
				& x\ne 2 \\
				& x\ne 3.}$\\
		Vậy tập xác định của hàm số là $\mathscr{D}=\left[1;4\right]\setminus\left\{2;3\right\}$.}
\end{ex}
\begin{ex}%[Phan Anh]%[0D2B1-2]
	Tìm tập xác định $\mathscr{D}$ của hàm số $y=\sqrt{\sqrt{x^2+2x+2}-(x+1)}$.
	\choice
	{$\mathscr{D}=\left(-\infty;-1\right)$}
	{$\mathscr{D}=\left[-1;+\infty \right)$}
	{$\mathscr{D}=\mathbb{R}\setminus\left\{-1\right\}$}
	{\True $\mathscr{D}=\mathbb{R}$}
	\loigiai{
		Hàm số xác định khi $\begin{aligned}[t]
				                & \sqrt{x^2+2x+2}-(x+1)\ge 0\Leftrightarrow \sqrt{(x+1)^2+1}\ge x+1 \\
				\Leftrightarrow & \, \hoac{
				                & \heva{
				                & x+1<0                                                             \\
				                & (x+1)^2+1\ge 0}                                                   \\
				                & \heva{
				                & x+1\ge 0                                                          \\
				                & (x+1)^2+1\ge(x+1)^2}}\Leftrightarrow \hoac{
				                & x+1<0                                                             \\
				                & x+1\ge 0}\Leftrightarrow x\in \mathbb{R}.
			\end{aligned}$\\
		Vậy tập xác định của hàm số là $\mathscr{D}=\mathbb{R}$.}
\end{ex}
\begin{ex}%[Phan Anh]%[0D2B1-2]
	Tìm tập xác định $\mathscr{D}$ của hàm số $y=\dfrac{2018}{\sqrt[3]{x^2-3x+2}-\sqrt[3]{x^2-7}}$.
	\choice
	{\True $\mathscr{D}=\mathbb{R}\setminus\left\{3\right\}$}
	{$\mathscr{D}=\mathbb{R}$}
	{$\mathscr{D}=\left(-\infty;1\right)\cup \left(2;+\infty \right)$}
	{$\mathscr{D}=\mathbb{R}\setminus\left\{0\right\}$}
	\loigiai{
		Hàm số xác định khi $\begin{aligned}[t]
				                & \sqrt[3]{x^2-3x+2}-\sqrt[3]{x^2-7}\ne 0\Leftrightarrow \sqrt[3]{x^2-3x+2}\ne \sqrt[3]{x^2-7} \\
				\Leftrightarrow & \,x^2-3x+2\ne x^2-7\Leftrightarrow 9\ne 3x\Leftrightarrow x\ne 3.
			\end{aligned}$\\
		Vậy tập xác định của hàm số là $\mathscr{D}=\mathbb{R}\setminus\left\{3\right\}$.}
\end{ex}
\begin{ex}%[Phan Anh]%[0D2K1-2]
	Tìm tập xác định $\mathscr{D}$ của hàm số $y=\dfrac{|x|}{|x-2|+\left|x^2+2x\right|}$.
	\choice
	{\True $\mathscr{D}=\mathbb{R}$}
	{$\mathscr{D}=\mathbb{R}\setminus\left\{-2;0\right\}$}
	{$\mathscr{D}=\mathbb{R}\setminus\left\{-2;0;2\right\}$}
	{$\mathscr{D}=\left(2;+\infty \right)$}
	\loigiai{
		Hàm số xác định khi $|x-2|+\left|x^2+2x\right|\ne0$.\\
		Xét phương trình $|x-2|+\left|x^2+2x\right|=0\Leftrightarrow \heva{
				& |x-2|=0 \\
				& \left|x^2+2x\right|=0}\Leftrightarrow \heva{
				& x=2 \\
				& x=0\vee x=-2.}$\\
		Vậy không có giá trị $x$ làm cho $|x-2|+\left| x^2+2x\right|=0$, do đó $|x-2|+\left| x^2+2x\right|\ne 0$ đúng với mọi $x\in \mathbb{R}$. Vậy tập xác định của hàm số là $\mathscr{D}=\mathbb{R}$.}
\end{ex}
\begin{ex}%[Phan Anh]%[0D2K1-2]
	Tìm tập xác định $\mathscr{D}$ của hàm số $y=\dfrac{2x-1}{\sqrt{x|x-4|}}$.
	\choice
	{$\mathscr{D}=\mathbb{R}\setminus\left\{0;4\right\}$}
	{$\mathscr{D}=\left(0;+\infty \right)$}
	{$\mathscr{D}=\left[0;+\infty \right)\setminus\left\{4\right\}$}
	{\True $\mathscr{D}=\left(0;+\infty \right)\setminus\left\{4\right\}$}
	\loigiai{
		Hàm số xác định khi $x|x-4|>0\Leftrightarrow \heva{
				& \left| x-4\right|\ne 0 \\
				& x>0}\Leftrightarrow \heva{
				& x\ne 4 \\
				& x>0.}$\\
		Vậy tập xác định của hàm số là $\mathscr{D}=\left(0;+\infty \right)\setminus\left\{4\right\}$.}
\end{ex}
\begin{ex}%[Phan Anh]%[0D2K1-2]
	Tìm tập xác định $\mathscr{D}$ của hàm số $y=\dfrac{\sqrt{5-3\left| x\right|}}{x^2+4x+3}$.
	\choice
	{\True $\mathscr{D}=\left[-\dfrac{5}{3};\dfrac{5}{3}\right]\setminus\left\{-1\right\}$}
	{$\mathscr{D}=\mathbb{R}$}
	{$\mathscr{D}=\left(-\dfrac{5}{3};\dfrac{5}{3}\right)\setminus\left\{-1\right\}$}
	{$\mathscr{D}=\left[-\dfrac{5}{3};\dfrac{5}{3}\right]$}
	\loigiai{
		Hàm số xác định khi $\heva{
				& 5-3\left| x\right|\ge 0 \\
				& x^2+4x+3\ne 0}\Leftrightarrow \heva{
				& \left| x\right|\le \dfrac{5}{3} \\
				& x\ne-1 \\
				& x\ne-3}\Leftrightarrow \heva{
				&-\dfrac{5}{3}\le x\le \dfrac{5}{3} \\
				& x\ne-1 \\
				& x\ne-3}\Leftrightarrow \heva{
				&-\dfrac{5}{3}\le x\le \dfrac{5}{3} \\
				& x\ne-1.}$\\
		Vậy tập xác định của hàm số là $\mathscr{D}=\left[-\dfrac{5}{3};\dfrac{5}{3}\right]\setminus\left\{-1\right\}$.}
\end{ex}
\begin{ex}%[Phan Anh]%[0D2K1-2]
	Tìm tập xác định $\mathscr{D}$ của hàm số $f(x)=\left\{\begin{array}{*{35}{l}}
			\dfrac{1}{2-x} & ;x\ge 1 \\
			\sqrt{2-x}     & ;x<1.
		\end{array}\right.$
	\choice
	{$\mathscr{D}=\mathbb{R}$}
	{$\mathscr{D}=\left(2;+\infty \right)$}
	{$\mathscr{D}=\left(-\infty;2\right)$}
	{\True $\mathscr{D}=\mathbb{R}\setminus\left\{2\right\}$}
	\loigiai{
		Hàm số xác định khi $\hoac{
				& \heva{
					& x\ge 1 \\
					& 2-x\ne 0} \\
				& \heva{
					& x<1 \\
					& 2-x\ge 0}}\Leftrightarrow \hoac{
				& \heva{
					& x\ge 1 \\
					& x\ne 2} \\
				& \heva{
					& x<1 \\
					& x\le 2}}\Leftrightarrow \hoac{
				& \heva{
					& x\ge 1 \\
					& x\ne 2} \\
				& x<1.}$\\
		Vậy xác định của hàm số là $\mathscr{D}=\mathbb{R}\setminus\left\{2\right\}$.}
\end{ex}
\begin{ex}%[Phan Anh]%[0D2K1-2]
	Tìm tập xác định $\mathscr{D}$ của hàm số $f(x)=\left\{\begin{array}{*{35}{l}}
			\dfrac{1}{x} & ;x\ge 1 \\
			\sqrt{x+1}   & ;x<1.
		\end{array}\right.$
	\choice
	{$\mathscr{D}=\left\{-1\right\}$}
	{$\mathscr{D}=\mathbb{R}$}
	{\True $\mathscr{D}=\left[-1;+\infty \right)$}
	{$\mathscr{D}=\left[-1;1\right)$}
	\loigiai{
	Hàm số xác định khi $\hoac{
			& \heva{
				& x\ge 1 \\
				& x\ne 0} \\
			& \heva{
				& x<1 \\
				& x+1\ge 0}}\Leftrightarrow \hoac{
			& x\ge 1 \\
			& \heva{
				& x<1 \\
				& x\ge-1.}}$\\
	Vậy xác định của hàm số là $\mathscr{D}=\left[-1;+\infty \right)$.}
\end{ex}
\begin{ex}%[Phan Anh]%[0D2K1-2]
	Tìm tất cả các giá trị thực của tham số $m$ để hàm số $y=\sqrt{x-m+1}+\dfrac{2x}{\sqrt{-x+2m}}$ xác định trên khoảng $(-1;3)$.
	\choice
	{\True Không có giá trị $m$ thỏa mãn}
	{$m\ge 2$}
	{$m\ge 3$}
	{$m\ge 1$}
	\loigiai{
	Hàm số xác định khi $\heva{
			& x-m+1\ge 0 \\
			&-x+2m>0}\Leftrightarrow \heva{
			& x\ge m-1 \\
			& x<2m.}$\\
	Tập xác định của hàm số là $\mathscr{D}=\left[m-1;2m\right)$ với điều kiện $m-1<2m\Leftrightarrow m>-1$.\\
	Hàm số đã cho xác định trên $\left(-1;3\right)$ khi và chỉ khi $\left(-1;3\right)\subset \left[m-1;2m\right)$\\
	$\Leftrightarrow m-1\le-1<3\le 2m\Leftrightarrow \heva{
			& m\le 0 \\
			& m\ge \dfrac{3}{2}.}$\\
	Vậy không có giá trị $m$ thỏa bài toán.}
\end{ex}
\begin{ex}%[Phan Anh]%[0D2K1-2]
	Tìm tất cả các giá trị thực của tham số $m$ để hàm số $y=\dfrac{x+2m+2}{x-m}$ xác định trên $\left(-1;0\right)$.
	\choice
	{$\hoac{
				& m>0 \\
				& m<-1}$}
	{$m\le-1$}
	{\True $\hoac{
				& m\ge 0 \\
				& m\le-1}$}
	{$m\ge 0$}
	\loigiai{
		Hàm số xác định khi $x-m\ne 0\Leftrightarrow x\ne m$.
		Tập xác định của hàm số là $\mathscr{D}=\mathbb{R}\setminus\left\{m\right\}$.\\
		Hàm số xác định trên $\left(-1;0\right)$ khi và chỉ khi $m\notin \left(-1;0\right)\Leftrightarrow \hoac{
				& m\ge 0 \\
				& m\le-1.}$}
\end{ex}
\begin{ex}%[Phan Anh]%[0D2K1-2]
	Tìm tất cả các giá trị thực của tham số $m$ để hàm số $y=\dfrac{mx}{\sqrt{x-m+2}-1}$ xác định trên $(0;1)$.
	\choice
	{$m\in \left(-\infty;\dfrac{3}{2}\right]\cup \left\{2\right\}$}
	{$m\in \left(-\infty;-1\right]\cup \left\{2\right\}$}
	{$m\in \left(-\infty;1\right]\cup \left\{3\right\}$}
			{\True $m\in \left(-\infty;1\right]\cup \left\{2\right\}$}
				\loigiai{
				Hàm số xác định khi $\heva{
					& x-m+2\ge 0 \\
					& \sqrt{x-m+2}-1\ne 0}\Leftrightarrow \heva{
					& x\ge m-2 \\
					& x\ne m-1.}$
				\\ Tập xác định của hàm số là $\mathscr{D}=\left[m-2;+\infty \right)\setminus\left\{m-1\right\}$.\\
			Hàm số xác định trên $\left(0;1\right)$ khi và chỉ khi $\left(0;1\right)\subset \left[m-2;+\infty \right)\setminus\left\{m-1\right\}$\\
		$\Leftrightarrow \hoac{
				& m-2\le 0<1\le m-1 \\
				& m-1\le 0}\Leftrightarrow \hoac{
				& \heva{
					& m\le 2 \\
					& m\ge 2} \\
				& m\le 1}\Leftrightarrow \hoac{
				& m=2 \\
				& m\le 1.}$}
\end{ex}
\begin{ex}%[Phan Anh]%[0D2K1-2]
	Tìm tất cả các giá trị thực của tham số $m$ để hàm số $y=\sqrt{x-m}+\sqrt{2x-m-1}$ xác định trên $(0;+\infty)$.
	\choice
	{$m\le 0$}
	{$m\ge 1$}
	{$m\le 1$}
	{\True $m\le-1$}
	\loigiai{
		Hàm số xác định khi $\heva{
				& x-m\ge 0 \\
				& 2x-m-1\ge 0}\Leftrightarrow \heva{
				& x\ge m \\
				& x\ge \dfrac{m+1}{2}}\,(*)$.
		\begin{itemize}
			\item Nếu $m\ge \dfrac{m+1}{2}\Leftrightarrow m\ge 1$ thì $\left(*\right)\Leftrightarrow x\ge m$.\\
			      Tập xác định của hàm số là $\mathscr{D}=\left[m;+\infty \right)$.
			      Khi đó, hàm số xác định trên $\left(0;+\infty \right)$ khi và chỉ khi $\left(0;+\infty \right)\subset \left[m;+\infty \right)\Leftrightarrow m\le 0$
			      $\Rightarrow $ Không thỏa mãn điều kiện $m\ge 1$.
			\item Nếu $m\le \dfrac{m+1}{2}\Leftrightarrow m\le 1$ thì $\left(*\right)\Leftrightarrow x\ge \dfrac{m+1}{2}$.\\
			      Tập xác định của hàm số là $\mathscr{D}=\left[\dfrac{m+1}{2};+\infty \right)$.
			      Khi đó, hàm số xác định trên $\left(0;+\infty \right)$
			      khi và chỉ khi $\left(0;+\infty \right)\subset \left[\dfrac{m+1}{2};+\infty \right)\Leftrightarrow \dfrac{m+1}{2}\le 0\Leftrightarrow m\le-1$.\\
			      $\Rightarrow $ Thỏa mãn điều kiện $m\le 1$.
		\end{itemize}
		Vậy $m\le-1$ thỏa yêu cầu bài toán.}
\end{ex}
\begin{ex}%[Phan Anh]%[0D2K1-2]
	Tìm tất cả các giá trị thực của tham số $m$ để hàm số $y=\dfrac{2x+1}{\sqrt{x^2-6x+m-2}}$ xác định trên $\mathbb{R}$.
	\choice
	{$m\ge 11$}
	{\True $m>11$}
	{$m<11$}
	{$m\le 11$}
	\loigiai{
		Hàm số xác định khi $x^2-6x+m-2>0\Leftrightarrow {\left(x-3\right)}^2+m-11>0$.\\
		Hàm số xác định với $\forall x\in \mathbb{R}\Leftrightarrow (x-3)^2+m-11>0$ đúng với mọi $x\in \mathbb{R}$
		$\Leftrightarrow m-11>0\Leftrightarrow m>11$.}
\end{ex}
\begin{ex}%[Phan Anh]%[0D2B1-3]
	Cho hàm số $f(x)=4-3x$. Khẳng định nào sau đây đúng?
	\choice
	{Hàm số đồng biến trên $\left(-\infty;\dfrac{4}{3}\right)$}
	{\True Hàm số nghịch biến trên $\left(\dfrac{4}{3};+\infty \right)$}
	{Hàm số đồng biến trên $\mathbb{R}$}
	{Hàm số đồng biến trên $\left(\dfrac{3}{4};+\infty \right)$}
	\loigiai{
		TXĐ: $\mathscr{D}=\mathbb{R}$. \\Với mọi $x_1,x_2\in \mathbb{R}$ và $x_1<x_2$, ta có
		$f\left(x_1\right)-f\left(x_2\right)=\left(4-3x_1\right)-\left(4-3x_2\right)=-3\left(x_1-x_2\right)>0.$\\
		Suy ra $f\left(x_1\right)>f\left(x_2\right)$. Do đó, hàm số nghịch biến trên $\mathbb{R}$.\\
		Mà $\left(\dfrac{4}{3};+\infty \right)\subset \mathbb{R}$ nên hàm số cũng nghịch biến trên $\left(\dfrac{4}{3};+\infty \right)$.}
\end{ex}
% \begin{ex}%[Phan Anh]%[0D2B1-3]
% 	Xét tính đồng biến, nghịch biến của hàm số $f(x)=x^2-4x+5$ trên khoảng $\left(-\infty;2\right)$ và trên khoảng $\left(2;+\infty \right)$. Khẳng định nào sau đây đúng?
% 	\choice
% 	{\True Hàm số nghịch biến trên $\left(-\infty;2\right)$, đồng biến trên $\left(2;+\infty \right)$}
% 	{Hàm số đồng biến trên $\left(-\infty;2\right)$, nghịch biến trên $\left(2;+\infty \right)$}
% 	{Hàm số nghịch biến trên các khoảng $\left(-\infty;2\right)$ và $\left(2;+\infty \right)$}
% 	{Hàm số đồng biến trên các khoảng $\left(-\infty;2\right)$ và $\left(2;+\infty \right)$}
% 	\loigiai{
% 		Ta có $f\left(x_1\right)-f\left(x_2\right)=\left(x_1^2-4x_1+5\right)-\left(x_2^2-4x_2+5\right)$
% 		$=\left(x_1^2-x_2^2\right)-4\left(x_1-x_2\right)=\left(x_1-x_2\right)\left(x_1+x_2-4\right)$.
% 		Với mọi $x_1, x_2\in \left(-\infty;2\right)$ và $x_1<x_2$. Ta có $\heva{
% 			& x_1<2 \\ 
% 			& x_2<2 \\}\Rightarrow x_1+x_2<4$.\\
% 		Suy ra $\dfrac{f\left(x_1\right)-f\left(x_2\right)}{x_1-x_2}=\dfrac{\left(x_1-x_2\right)\left(x_1+x_2-4\right)}{x_1-x_2}=x_1+x_2-4<0$.\\
% 		Vậy hàm số nghịch biến trên $\left(-\infty;2\right)$.\\
% 		Với mọi $x_1, x_2\in \left(2;+\infty \right)$ và $x_1<x_2$. Ta có $\heva{
% 			& x_1>2 \\ 
% 			& x_2>2 \\}\Rightarrow x_1+x_2>4$.\\
% 		Suy ra $\dfrac{f\left(x_1\right)-f\left(x_2\right)}{x_1-x_2}=\dfrac{\left(x_1-x_2\right)\left(x_1+x_2-4\right)}{x_1-x_2}=x_1+x_2-4>0$.\\
% 		Vậy hàm số đồng biến trên $\left(2;+\infty \right)$.}
% \end{ex}
\begin{ex}%[Phan Anh]%[0D2B1-3]
	Xét sự biến thiên của hàm số $f(x)=\dfrac{3}{x}$ trên khoảng $(0;+\infty)$. Khẳng định nào sau đây đúng?
	\choice
	{Hàm số đồng biến trên khoảng $\left(0;+\infty \right)$}
	{\True Hàm số nghịch biến trên khoảng $\left(0;+\infty \right)$}
	{Hàm số vừa đồng biến, vừa nghịch biến trên khoảng $\left(0;+\infty \right)$}
	{Hàm số không đồng biến, cũng không nghịch biến trên khoảng $\left(0;+\infty \right)$}
	\loigiai{
		Ta có $f\left(x_1\right)-f\left(x_2\right)=\dfrac{3}{x_1}-\dfrac{3}{x_2}=\dfrac{3\left(x_2-x_1\right)}{x_1x_2}=-\dfrac{3\left(x_1-x_2\right)}{x_1x_2}.$\\
		Với mọi $x_1, x_2\in \left(0;+\infty \right)$ và $x_1<x_2$. Ta có $\heva{
				& x_1>0 \\
				& x_2>0 \\}\Rightarrow x_1\cdot x_2>0$.\\
		Suy ra $\dfrac{f\left(x_1\right)-f\left(x_2\right)}{x_1-x_2}=-\dfrac{3}{x_1x_2}<0\Rightarrow f(x)$ nghịch biến trên $\left(0;+\infty \right)$.}
\end{ex}
\begin{ex}%[Phan Anh]%[0D2B1-3]
	Xét sự biến thiên của hàm số $f(x)=x+\dfrac{1}{x}$ trên khoảng $\left(1;+\infty \right)$. Khẳng định nào sau đây đúng?
	\choice
	{\True Hàm số đồng biến trên khoảng $\left(1;+\infty \right)$}
	{Hàm số nghịch biến trên khoảng $\left(1;+\infty \right)$}
	{Hàm số vừa đồng biến, vừa nghịch biến trên khoảng $\left(1;+\infty \right)$}
	{Hàm số không đồng biến, cũng không nghịch biến trên khoảng $\left(1;+\infty \right)$}
	\loigiai{
		Ta có
		$f\left(x_1\right)-f\left(x_2\right)=\left(x_1+\dfrac{1}{x_1}\right)-\left(x_2+\dfrac{1}{x_2}\right)=\left(x_1-x_2\right)+\left(\dfrac{1}{x_1}-\dfrac{1}{x_2}\right)=\left(x_1-x_2\right)\left(1-\dfrac{1}{x_1x_2}\right).$\\
		Với mọi $x_1, x_2\in \left(1;+\infty \right)$ và $x_1<x_2$. Ta có $\heva{
				& x_1>1 \\
				& x_2>1 \\}\Rightarrow x_1\cdot x_2>1\Rightarrow \dfrac{1}{x_1\cdot x_2}<1.$\\
		Suy ra $\dfrac{f\left(x_1\right)-f\left(x_2\right)}{x_1-x_2}=1-\dfrac{1}{x_1x_2}>0\Rightarrow f(x)$ đồng biến trên $\left(1;+\infty \right)$.}
\end{ex}
\begin{ex}%[Phan Anh]%[0D2B1-3]
	Xét tính đồng biến, nghịch biến của hàm số $f(x)=\dfrac{x-3}{x+5}$ trên khoảng $\left(-\infty;-5\right)$ và trên khoảng $\left(-5;+\infty \right)$. Khẳng định nào sau đây đúng?
	\choice
	{Hàm số nghịch biến trên $\left(-\infty;-5\right)$, đồng biến trên $\left(-5;+\infty \right)$}
	{Hàm số đồng biến trên $\left(-\infty;-5\right)$, nghịch biến trên $\left(-5;+\infty \right)$}
	{Hàm số nghịch biến trên các khoảng $\left(-\infty;-5\right)$ và $\left(-5;+\infty \right)$}
	{\True Hàm số đồng biến trên các khoảng $\left(-\infty;-5\right)$ và $\left(-5;+\infty \right)$}
	\loigiai{
		Ta có
		\begin{eqnarray*}
			f\left(x_1\right)-f\left(x_2\right)&=&\left(\dfrac{x_1-3}{x_1+5}\right)-\left(\dfrac{x_2-3}{x_2+5}\right)\\
			&=&\dfrac{\left(x_1-3\right)\left(x_2+5\right)-\left(x_2-3\right)\left(x_1+5\right)}{\left(x_1+5\right)\left(x_2+5\right)}\\
			&=&\dfrac{8\left(x_1-x_2\right)}{\left(x_1+5\right)\left(x_2+5\right)}.
		\end{eqnarray*}
		Với mọi $x_1, x_2\in \left(-\infty;-5\right)$ và $x_1<x_2$. Ta có $\heva{& x_1<-5 \\& x_2<-5}\Leftrightarrow \heva{&x_1+5<0 \\& x_2+5<0.}$\\
		Suy ra $\dfrac{f\left(x_1\right)-f\left(x_2\right)}{x_1-x_2}=\dfrac{8}{\left(x_1+5\right)\left(x_2+5\right)}>0\Rightarrow f(x)$ đồng biến trên $\left(-\infty;-5\right)$.\\
		Với mọi $x_1, x_2\in \left(-5;+\infty \right)$ và $x_1<x_2$. Ta có $\heva{
				& x_1>-5 \\
				& x_2>-5 \\}\Leftrightarrow \heva{
				& x_1+5>0 \\
				& x_2+5>0 \\}$.\\
		Suy ra $\dfrac{f\left(x_1\right)-f\left(x_2\right)}{x_1-x_2}=\dfrac{8}{\left(x_1+5\right)\left(x_2+5\right)}>0\Rightarrow f(x)$ đồng biến trên $\left(-5;+\infty \right)$.}
\end{ex}

\begin{ex}%[Phan Anh]%[0D2K1-3]
	Cho hàm số $f(x)=\sqrt{2x-7}$. Khẳng định nào sau đây đúng?
	\choice
	{Hàm số nghịch biến trên $\left(\dfrac{7}{2};+\infty \right)$}
	{\True Hàm số đồng biến trên $\left(\dfrac{7}{2};+\infty \right)$}
	{Hàm số đồng biến trên $\mathbb{R}$}
	{Hàm số nghịch biến trên $\mathbb{R}$}
	\loigiai{
	Tập xác định là $\mathscr{D}=\left[\dfrac{7}{2};+\infty \right)$ nên ta loại đáp án C và D.\\
	Xét $f\left(x_1\right)-f\left(x_2\right)=\sqrt{2x_1-7}-\sqrt{2x_2-7}=\dfrac{2\left(x_1-x_2\right)}{\sqrt{2x_1-7}+\sqrt{2x_2-7}}.$\\
	Với mọi $x_1, x_2\in \left(\dfrac{7}{2};+\infty \right)$ và $x_1<x_2$, ta có $\dfrac{f\left(x_1\right)-f\left(x_2\right)}{x_1-x_2}=\dfrac{2}{\sqrt{2x_1-7}+\sqrt{2x_2-7}}>0.$\\
	Vậy hàm số đồng biến trên $\left(\dfrac{7}{2};+\infty \right)$.}
\end{ex}
\begin{ex}%[Phan Anh]%[0D2K1-3]
	Có bao nhiêu giá trị nguyên của tham số $m$ thuộc đoạn $\left[-3;3\right]$ để hàm số $f(x)=\left(m+1\right)x+m-2$ đồng biến trên $\mathbb{R}$?
	\choice
	{$7$}
	{$5$}
	{\True $4$}
	{$3$}
	\loigiai{
		Tập xác định $\mathscr{D}=\mathbb{R}.$\\
		Với mọi $x_1,x_2\in\mathscr{D}$ và $x_1<x_2$. \\Ta có
		$f\left(x_1\right)-f\left(x_2\right)=\left[\left(m+1\right)x_1+m-2\right]-\left[\left(m+1\right)x_2+m-2\right]=\left(m+1\right)\left(x_1-x_2\right).$\\
		Suy ra $\dfrac{f\left(x_1\right)-f\left(x_2\right)}{x_1-x_2}=m+1$.\\
		Để hàm số đồng biến trên $\mathbb{R}$ khi và chỉ khi
		$m+1>0\Leftrightarrow m>-1\xrightarrow{m\in \left[-3;3\right]}{m\in \mathbb{Z}}\Rightarrow m\in \left\{0;1;2;3\right\}$.\\
		Vậy có 4 giá trị nguyên của $m$ thỏa mãn.}
\end{ex}
% \begin{ex}%[Phan Anh]%[0D2K1-3]
% 	Tìm tất cả các giá trị thực của tham số $m$ để hàm số $y=-x^2+\left(m-1\right)x+2$ nghịch biến trên khoảng $\left(1;2\right)$.
% 	\choice
% 	{$m<5$}
% 	{$m>5$}
% 	{\True $m<3$}
% 	{$m>3$}
% 	\loigiai{
% 		Với mọi $x_1\ne x_2$, ta có\\
% 		$\dfrac{f\left(x_1\right)-f\left(x_2\right)}{x_1-x_2}=\dfrac{\left[-x_1^2+\left(m-1\right)x_1+2\right]-\left[-x_2^2+\left(m-1\right)x_2+2\right]}{x_1-x_2}=-\left(x_1+x_2\right)+m-1.$\\
% 		Để hàm số nghịch biến trên $\left(1;2\right)\Leftrightarrow-\left(x_1+x_2\right)+m-1<0$, với mọi $x_1,x_2\in \left(1;2\right)$\\
% 		$\Leftrightarrow m<\left(x_1+x_2\right)+1$, với mọi $x_1,x_2\in \left(1;2\right)$
% 		$\Leftrightarrow m<\left(1+1\right)+1=3$.}
% \end{ex}
\begin{ex}%[Phan Anh]%[0D2K1-3]
	\immini{Cho hàm số $y=f(x)$ có tập xác định là $\left[-3;3\right]$ và đồ thị của nó được biểu diễn bởi hình bên. Khẳng định nào sau đây là đúng?
		\choice
		{\True Hàm số đồng biến trên khoảng $\left(-3;-1\right)$ và $\left(1;3\right)$}
		{Hàm số đồng biến trên khoảng $\left(-3;-1\right)$và $\left(1;4\right)$}
		{Hàm số đồng biến trên khoảng $\left(-3;3\right)$}
		{Hàm số nghịch biến trên khoảng $\left(-1;0\right)$}}
	{\begin{tikzpicture}[>=stealth,scale=0.7]
			\draw[->](-4,0)--(4,0)node[above]{$x$};
			\draw[->](0,-2)--(0,5)node[right]{$y$};
			\draw (-3,-1)--(-1,1)--(0,1)node[above left]{$1$}--(3,4);
			\draw[dashed](-3,0)node[above]{$-3$}--(-3,-1)--(0,-1)node[right]{$-1$};
			\draw[dashed](-1,0)node[below]{$-1$}--(-1,1);
			\draw[dashed](3,0)node[below]{$3$}--(3,4)--(0,4)node[left]{$4$};
			\fill (-3,0)circle(1.2pt) (-3,-1)circle(1.2pt) (0,-1)circle(1.2pt) (-1,0)circle(1.2pt) (-1,1)circle(1.2pt) (0,1)circle(1.2pt) (3,0)circle(1.2pt) (3,4)circle(1.2pt) (0,4)circle(1.2pt) (0,0)node[above right]{$O$}circle(1.2pt);
		\end{tikzpicture}}
	\loigiai{
		Trên khoảng $\left(-3;-1\right)$ và $\left(1;3\right)$ đồ thị hàm số đi lên từ trái sang phải\\
		$\Rightarrow $ Hàm số đồng biến trên khoảng $\left(-3;-1\right)$ và $\left(1;3\right).$}
\end{ex}
\begin{ex}%[Phan Anh]%[0D2K1-3]
	\immini{Cho đồ thị hàm số $y=x^3$ như hình bên. Khẳng định nào sau đây \textbf{sai}?
		\choice
		{Hàm số đồng biến trên khoảng $\left(-\infty;0\right)$}
		{Hàm số đồng biến trên khoảng $\left(0;+\infty \right)$}
		{Hàm số đồng biến trên khoảng $\left(-\infty;+\infty \right)$}
		{\True Hàm số đồng biến tại gốc tọa độ $O$}}
	{\begin{tikzpicture}[>=stealth,scale=0.6]
			\draw[->](-2,0)--(2,0)node[above]{$x$};
			\draw[->](0,-3)--(0,3)node[right]{$y$};
			\draw[smooth,samples=100,domain=-1.4:1.4]plot(\x,{(\x)^3});
			\fill (0,0)node[above left]{$O$}circle(1.2pt);
		\end{tikzpicture}}
	\loigiai{Dựa vào đồ thị, ta thấy hàm số đồng biến trên toàn miền xác định. Nhưng không thể đồng biến chỉ tại đúng một điểm.}
\end{ex}
\begin{ex}
	\immini{Cho hàm số $y=f(x)$ có đồ thị là một đường liền nét trên đoạn $[-2;4]$ (hình bên). Xét trên $[-2;4]$, có bao nhiêu giá trị của $x$ để $y=1$?
		\haicot
		{$4$}
		{$5$}
		{\True vô số}
		{$1$}
	}{\hspace{1cm}
		\begin{tikzpicture}[smooth,samples=300,scale=0.6,>=stealth]
			\draw[black!30!] (-2,-1.5) grid (4,3);
			\draw[->] (-2,0)--(4.5,0) node[below]{$x$};
			\draw[->] (0,-1.5)--(0,3) node[right]{$y$};
			\foreach \x in {-2,-1,1,2,3,4}{
				\draw (\x,0) node[below]{$\x$};%Ox
			}
			\foreach \y in {1}{
				\draw (0,\y) node[left]{$\y$};%Oy
			}
			\draw (0,0) node[above right]{$O$};
			\draw[domain=-1.8:1,thick] plot(\x,{-(\x)^2+2});
			\draw [thick](1,1)--(4,1) node[above]{\small $y=f(x)$};
			\draw[fill=black] (0,2) circle(1.5pt) (1,1) circle(1pt);
	\end{tikzpicture}}
	\loigiai{
		Quan sát đồ thị, ta có các kết quả $f(1)=1$, $f(3)=1$ và $f(4)=1$ nên
		$$P=2f(1)+f(4)-f(3)=2+1-1=2.$$
	}
\end{ex}

\begin{ex} Trong các công thức dưới đây, công thức nào được xem là công thức của một hàm số $y$ theo biến $x$?
	\choice
	{$3x^2-y^2=0$ }
	{\True $3x^2-y+1=0$}
	{$y^2=x$}
	{$(y-x)(y+x)=1$}
	\loigiai{
	}
\end{ex}
\begin{ex}%[0D2B1-4]
	Trong các đường biểu diễn dưới đây, đường nào \textbf{không} phải là đồ thị của một hàm số?
	\choice
	{\begin{tikzpicture}[smooth,samples=300,scale=0.4,>=stealth]
		\draw[->] (-0.8,0)--(3.7,0) node[below]{$x$};
		\draw[->] (0,-1.5)--(0,2.5) node[right]{$y$};
		\draw (0,0) node[above left]{$O$};
		\draw[domain=0.3:3.7,thick] plot(\x,{(\x)^2-4*(\x)+3});
		\end{tikzpicture}}
	{\begin{tikzpicture}[smooth,samples=300,scale=0.4,>=stealth]
		\draw[->] (-2,0)--(4,0) node[below]{$x$};
		\draw[->] (0,-1.5)--(0,2.5) node[right]{$y$};
		\draw (0,0) node[below right]{$O$};
		\draw[magenta,thick] (-1,-1)--(0,2)--(1,0)--(3,2);
		\end{tikzpicture}}
	{\True \begin{tikzpicture}[smooth,samples=300,scale=0.4,>=stealth]
		\draw[->] (-2,0)--(2,0) node[below]{$x$};
		\draw[->] (0,-1.5)--(0,2.5) node[right]{$y$};
		\draw (0,0) node[above right]{$O$};
		\draw[magenta,thick] (90:1) arc (90:270:1);
		\draw[domain=0:1.7,thick] plot(\x,{(\x)^2-1});
		\end{tikzpicture} }
	{\begin{tikzpicture}[smooth,samples=300,scale=0.4,>=stealth]
		\draw[->] (-3,0)--(1,0) node[below]{$x$};
		\draw[->] (0,-1.5)--(0,2.5) node[right]{$y$};
		\draw (0,0) node[above left]{$O$};
		\draw[domain=-2.8:0.8,thick] plot(\x,{-(\x+3)^2+4*(\x+3)-2});
		\end{tikzpicture}}
	\loigiai{
	}
\end{ex}


\begin{ex}%[0D2B1-4]
	Trong các đường biểu diễn dưới đây, đường nào \textbf{không} phải là đồ thị của một hàm số?
	
	\choice
	{\begin{tikzpicture}[smooth,samples=300,scale=0.5,>=stealth]
			\draw[->] (-0.8,0)--(3.7,0) node[below]{$x$};
			\draw[->] (0,-1.5)--(0,2.5) node[right]{$y$};
			\draw (0,0) node[above left]{$O$};
			\draw[domain=0.3:3.7,thick] plot(\x,{(\x)^2-4*(\x)+3});
	\end{tikzpicture}}
	{\True \begin{tikzpicture}[smooth,samples=300,scale=0.5,>=stealth]
			\draw[->] (-2,0)--(4,0) node[below]{$x$};
			\draw[->] (0,-1.5)--(0,2.5) node[right]{$y$};
			\draw (0,0) node[below right]{$O$};
			\draw[magenta,thick] (-1,-1)--(-1,2)--(1,0)--(3,2);
	\end{tikzpicture}}
	{ \begin{tikzpicture}[smooth,samples=300,scale=0.5,>=stealth]
			\draw[->] (-2,0)--(2,0) node[below]{$x$};
			\draw[->] (0,-1.5)--(0,2.5) node[right]{$y$};
			\draw (0,0) node[below right]{$O$};
			\draw[magenta,thick] (0:1) arc (0:180:1) (1,0)--(2,2);
		\end{tikzpicture} }
	{\begin{tikzpicture}[>=stealth,scale=0.5]
			\draw[->] (-3,0)-- (0,0) node [below right]{$O$}--(6,0) node[below]{$x$};
			\draw[->] (0,-1.5)--(0,2) node [right]{$y$};
			\draw [line width = 1.2pt, domain = -3:5, samples=150] plot (\x,{cos(\x*180/pi)});
			\end{tikzpicture}}
	\loigiai{
		}
\end{ex}

\begin{ex}
	Bảng giá cước gọi quốc tế của công ty viễn thông A được cho bởi bảng sau:
	\begin{center}
		\begin{tikzpicture}[xscale=7,yscale=0.8,font=\footnotesize]
			\begin{scope}[shift={(-.5,.5)}]
				\fill[orange!15] (0,-1) rectangle (1,-5);
				\fill[cyan!30] (0,0) rectangle (2,-1);
				\draw [line width=0.7pt,gray](0,0) grid (2,-5)
					;
			\end{scope}
			\path
			(0,0) node{\text{\textbf{Thời gian gọi (phút)}}}  
			(-0.4,-1) node[right]{Không quá 8 phút}
			(-0.4,-2) node[right]{Từ phút thứ 9 đến phút thứ 15}    
			(-0.4,-3) node[right]{Từ phút thứ 16 đến phút thứ 25}
			(-0.4,-4) node[right]{Từ phút 26 trở đi}    
					
			(1,0) node{\text{\textbf{Giá cước điện thoại (đồng/phút)}}}    
			(1,-1) node[right]{6500}
			(1,-2) node[right]{6000}    							
			(1,-3) node[right]{5500}
			(1,-4) node[right]{5000}    
				;
		\end{tikzpicture}
	\end{center}
	Ông An thực hiện cuộc gọi quốc tế 12 phút. Số tiền cước ông An phải trả là
	\choice
	{72 000 đồng  }
	{\True 76 000 đồng }
	{70 000 đồng }
	{90 000 đồng}
	\loigiai{
		\begin{itemize}
			\item [$\bullet$] Có thể thiệt lập biểu thức tính giá cước từ phút thứ 9 đến 15 là $6000t + 4000$. Thay $t =12$, ta được số tiền là 76 000 đồng.
			\item [$\bullet$] Hoặc tính nhanh: $ 8 \times 6500 + 4 \times 6000 = 76 000$ đồng.
		\end{itemize}
	}
\end{ex}
\begin{ex}%[0D2B1]
	Tìm tất cả các giá trị của $m$ để hàm số $y=\dfrac{x\sqrt{5}}{x^2-2x+m}$ có tập xác định là $\mathbb{R}$.
	\choice
	{\True $m>1$}
	{$m=1$}
	{$m<1$}
	{$m<0$}
	\loigiai{
	Hàm số có tập xác định là $\mathbb{R}$ khi và chỉ khi $x^2-2x+m =0$ vô nghiệm $\Leftrightarrow \Delta'=1-m<0 \Leftrightarrow m>1$.
	}
\end{ex}

\begin{ex}%[0D2G1-2]%
	Tìm các giá trị thực của tham số $m$ để hàm số $y=\dfrac{x+m+2}{x-m}$ xác định trên $(-1;2)$.
	\choice
	{$\left\{\begin{aligned}
			&m\leq -1\\
			&m\geq 2\\
		\end{aligned}\right. $}
	{$\left[\begin{aligned}
			&m<-1\\
			&m>2\\
		\end{aligned}\right. $ }
	{$-1<m<2$}
	{\True $\left[\begin{aligned}
			&m\leq -1\\
			&m\geq 2\\
		\end{aligned}\right. $ }
	\loigiai{
		Hàm số $y=\dfrac{x+m+2}{x-m}$ xác định khi $x\ne m$.\\
		Để hàm số $y=\dfrac{x+m+2}{x-m}$ xác định trên $(-1;2)$ khi và chỉ khi $\left[\begin{aligned}
			&m\leq -1\\
			&m\geq 2.\\
		\end{aligned}\right. $}
\end{ex}

\Closesolutionfile{ans}

%--------Lời giải chi tiết
\FULLWIDTH \hienLG \anDA % ẩn đáp án
% % % % --
% % % % \setcounter{deso}{0}
\chap{LỜI GIẢI CHI TIẾT} \setcounter{dang}{0} \setcounter{section}{0}

%%Chương 1
% \setlistsEX{column-sep=-25pt,after-skip=-10pt,after-item-skip=0ex}
\section{Mệnh đề}
\subsection{Tóm tắt lý thuyết}
\begin{tomtat}
\subsubsection{Mệnh đề}
\begin{boxdn}{}
	\textit{Mệnh đề toán học} (gọi tắt là \textit{mệnh đề}) là một khẳng định về một sự kiện toán học \textbf{hoặc đúng hoặc sai}, \textbf{không thể vừa đúng vừa sai}.	
	\begin{itemize}
		\item Mệnh đề thường được kí hiệu bằng các chữ cái in hoa. Ví dụ: Q: \lq\lq  6 chia hết cho 3\rq\rq.
	\end{itemize}
\end{boxdn}

\begin{note}
	\begin{itemize}
		\item Các câu hỏi, câu cảm thán, câu mệnh lệnh không phải là mệnh đề.
		\item Một câu chưa xác định được đúng hay sai nhưng chắc chắn nó chỉ đúng hoặc sai (không thể vừa đúng vừa sai) cũng là một mệnh đề. Ví dụ: \lq\lq  $2^{2023^2+2023+1}+1$ là số nguyên tố\rq\rq\ là một mệnh đề.
		\item Trong thực tế, có những mệnh đề mà tính đúng sai của nó luôn gắn với một thời gian và địa điểm cụ thể: đúng ở thời gian hoặc địa điểm này nhưng sai ở thời gian hoặc địa điểm khác. Nhưng ở bất kì thời gian, địa điểm nào cũng luôn có giá trị chân lí hoặc đúng hoặc sai. Ví dụ: Số 1 là số tự nhiên nhỏ nhất. (Trong một số chương trình, tập số tự nhiên không bao gồm số 0. Tìm hiểu thêm ở topic: \lq\lq  Natural Number\rq\rq\ trên Wikipedia) 
	\end{itemize}
\end{note}
\subsubsection{Mệnh đề chứa biến}
\begin{boxdn}{}
	Những khẳng định mà tính đúng, sai của chúng phụ thuộc vào giá trị của biến gọi là \textit{mệnh đề chứa biến}.
\end{boxdn}
Ví dụ: Cho $P(x): x>x^2$ với $x$ là số thực. Ta chưa khẳng định được tính đúng sai của câu này, do đó nó chưa phải là mệnh đề.\\
Tuy nhiên, khi thay $x$ bởi những giá trị cụ thể thì ta được một mệnh đề, chẳng hạn, $P(2)$ là mệnh đề sai, $P\left(\dfrac{1}{2}\right)$ là mệnh đề đúng.

\subsubsection{Mệnh đề phủ định}

\begin{boxdn}{}
	Cho mệnh đề $P$. Mệnh đề \lq\lq  Không phải $P$\rq\rq\ được gọi là mệnh đề phủ định của $P$ và kí hiệu là $\overline{P}$.
	\begin{itemize}
		\item Mệnh đề $P$ và mệnh đề phủ định $\overline{P}$ là hai khẳng định trái ngược nhau. Nếu $P$ đúng thì $\overline{P}$ sai, nếu $P$ sai thì $\overline{P}$ đúng.
		\item Mệnh đề phủ định của $P$ có thể diễn đạt theo nhiều cách khác nhau. Chẳng hạn, xét mệnh đề $P$: \lq\lq  $2$ là số chẵn\rq\rq. Khi đó, mệnh đề phủ định của $P$ có thể phát biểu là $\overline{P}$: \lq\lq  $2$ không phải là số chẵn\rq\rq\ hoặc \lq\lq  $2$ là số lẻ\rq\rq.
	\end{itemize} 
\end{boxdn}

\subsubsection{Mệnh đề kéo theo và mệnh đề đảo}

\begin{boxdn}{}
	Cho hai mệnh đề $P$ và $Q$. Mệnh đề \lq\lq  Nếu $P$ thì $Q$\rq\rq\ được gọi là mệnh đề kéo theo.
	\begin{itemize}
		\item Kí hiệu là $P\Rightarrow Q.$
		\item Mệnh đề kéo theo chỉ sai khi $P$ đúng $Q$ sai.
		\item $P\Rightarrow Q$ còn được phát biểu là \lq\lq  $P$ kéo theo $Q$\rq\rq, \lq\lq  $P$ suy ra $Q$\rq\rq\ hay \lq\lq  Vì $P$ nên $Q$\rq\rq.
	\end{itemize}
\end{boxdn}

\begin{note}
	Trong toán học, định lí là một mệnh đề đúng, thường có dạng $P\Rightarrow Q$.
	Khi đó ta nói 
	\begin{itemize}
		\item $P$ là giả thiết, $Q$ là kết luận của định lí.
		\item $P$ là $\underline{\textit{điều kiện đủ}}$ để có $Q$, còn $Q$ là $\underline{\textit{điều kiện cần}}$ để có $P$.
	\end{itemize}
\end{note}

% \begin{note}
% 	Trong logic toán học, khi xét giá trị chân lí của mệnh đề $P\Rightarrow Q$ người ta không quan tâm đến mối quan hệ về nội dung của hai mệnh đề $P$, $Q$. Không phân biệt trường hợp $P$ có phải là nguyên nhân để có $Q$ hay không mà chỉ quan tâm đến tính đúng, sai của chúng.
	
% 	Ví dụ: \lq\lq  Nếu mặt trời quay quanh trái đất thì Việt Nam nằm ở châu Âu\rq\rq\ là một mệnh đề đúng. Vì ở đây hai mệnh đề $P$: \lq\lq  Mặt trời quay xung quanh trái đất\rq\rq\ và $Q$: \lq\lq  Việt Nam nằm ở châu Âu\rq\rq\ đều là mệnh đề sai.
% 	(Tìm hiểu thêm ở topic \lq\lq  Mệnh đề toán học\rq\rq trên Wikipedia)
% \end{note}

\begin{boxdn}{}
	Cho mệnh đề kéo theo $P\Rightarrow Q$. Mệnh đề $Q\Rightarrow P$ được gọi là mệnh đề đảo của mệnh đề $P\Rightarrow Q$.
\end{boxdn}

\begin{note}
	Mệnh đề đảo của một mệnh đề đúng không nhất thiết là một mệnh đề đúng.
\end{note}

\subsubsection{Mệnh đề tương đương}

\begin{boxdn}{}
	Cho hai mệnh đề $P$ và $Q$. Mệnh đề có dạng \lq\lq  $P$ nếu và chỉ nếu $Q$\rq\rq\ được gọi là mệnh đề tương đương.
	\begin{itemize}
		\item Kí hiệu là $P \Leftrightarrow Q$.
		\item Mệnh đề $P \Leftrightarrow Q$ đúng khi cả hai mệnh đề $P\Rightarrow Q$ và $Q \Rightarrow P$ cùng đúng hoặc cùng sai. \\
		(Hay $P \Leftrightarrow Q$ đúng khi cả hai mệnh đề $P$ và $Q$ cùng đúng hoặc cùng sai).
		\item $P\Leftrightarrow Q$ còn được phát biểu là \lq\lq  $P$ khi và chỉ khi $Q$\rq\rq, \lq\lq  $P$ tương đương với $Q$\rq\rq, hay \lq\lq  $P$ là điều kiện cần và đủ để có $Q$\rq\rq.
	\end{itemize}
\end{boxdn}

% \begin{note}
% 	Trong logic học, hai mệnh đề $P$, $Q$ tương đương với nhau hoàn toàn không có nghĩa là nội dung của chúng như nhau, mà nó chỉ nói lên rằng chúng có cùng giá trị chân lí (cùng đúng hoặc cùng sai).\\
% 	Ví dụ: \lq\lq  Hình vuông có một góc tù khi và chỉ khi 100 là số nguyên tố\rq\rq\ là một mệnh đề đúng.
% \end{note}

\subsubsection{Mệnh đề có chứa kí hiệu $\forall$ và $\exists$}

\begin{itemize}
	\item Kí hiệu $\forall$ (với mọi): \lq\lq $ \forall x \in X, P(x)$\rq\rq\ hoặc \lq\lq $ \forall x \in X : P(x)$\rq\rq.
	\item Kí hiệu $\exists$ (tồn tại): \lq\lq $ \exists x \in X, P(x)$\rq\rq\ hoặc \lq\lq $ \exists x \in X : P(x)$\rq\rq.
\end{itemize}

\begin{note}\hfil
	\begin{itemize}
		\item Phủ định của mệnh đề \lq\lq $ \forall x \in X, P(x)$\rq\rq\ là mệnh đề \lq\lq $ \exists x\in X, \overline{P(x)}$\rq\rq.
		\item Phủ định của mệnh đề \lq\lq $ \exists x\in X, P(x)$\rq\rq\ là mệnh đề  \lq\lq $ \forall x\in X, \overline{P(x)}$\rq\rq.
	\end{itemize}
\end{note}

\end{tomtat}


\subsection{Các dạng toán}

\begin{dang}{Xác định mệnh đề và xét tính đúng - sai của mệnh đề}
\end{dang}

\subsubsection{Ví dụ minh hoạ}

\begin{vd}%[Thành Đức Trung]%[0D1Y1-1]
	Phát biểu nào sau đây là một mệnh đề toán học?
	\begin{enumerate}
		\item Hà Nội là Thủ đô của Việt Nam.
		\item Số $\pi$ là một số hữu tỉ.
		\item $x=1$ có phải là nghiệm của phương trình $x^2-1=0$ không?
		\item Phương trình $3x^2-5x+2=0$ có nghiệm nguyên.
		\item $5<7-3$.
		\item Đây là cách xử lí khôn ngoan!
	\end{enumerate}
	\loigiai
	{
		\begin{enumerate}
			\item Phát biểu \lq\lq  Hà Nội là Thủ đô của Việt Nam\rq\rq\ là mệnh đề nhưng không phải là mệnh đề toán học.
			\item Phát biểu \lq\lq  Số $\pi$ là một số hữu tỉ\rq\rq\ là một mệnh đề toán học.
			\item Phát biểu \lq\lq  $x=1$ có phải là nghiệm của phương trình $x^2-1=0$ không?\rq\rq\ là một câu hỏi nên không phải là một mệnh đề toán học.
			\item Phát biểu \lq\lq  Phương trình $3x^2-5x+2=0$ có nghiệm nguyên\rq\rq\ là một mệnh đề toán học.
			\item Phát biểu \lq\lq  $5<7-3$\rq\rq\ là một mệnh đề toán học.
			\item Phát biểu \lq\lq  Đây là cách xử lí khôn ngoan!\rq\rq\ là một câu cảm thán nên không phải là một mệnh đề toán học.
		\end{enumerate}
	}
\end{vd}

\begin{vd}%[Thành Đức Trung]%[0D1Y1-2]
	Trong các mệnh đề toán học sau đây, mệnh đề nào là một khẳng định đúng? Mệnh đề nào là một khẳng định sai?
	\begin{enumerate}
		\item $P\colon$\lq\lq  Tổng hai góc đối của một tứ giác nội tiếp bằng $180^{\circ}$\rq\rq.
		\item $Q\colon$\lq\lq  $7$ là số chính phương\rq\rq.
		\item $R\colon$\lq\lq  $1$ là số nguyên tố\rq\rq.
	\end{enumerate}
	\loigiai
	{
		Mệnh đề $P$ là mệnh đề đúng. \\
		Mệnh đề $Q$ và $R$ là mệnh đề sai. \\
	}
\end{vd}

% \begin{vd}%[Thành Đức Trung]%[0D1Y1-2]
% 	Thay dấu \lq\lq ?\rq\rq\ bằng dấu \lq\lq  x\rq\rq\ vào ô thích hợp trong bảng sau
% 	\begin{center}
% 		\begin{tabular}{|>{\centering\arraybackslash}m{5.5cm}|>{\centering\arraybackslash}m{2cm}|>{\centering\arraybackslash}m{2cm}|>{\centering\arraybackslash}m{2cm}|}
% 			\hline
% 			Câu & Không phải MĐ & MĐ đúng & MĐ sai \\
% 			\hline
% 		\end{tabular}
% 		\begin{tabular}{|>{\raggedright\arraybackslash}m{5.5cm}|>{\centering\arraybackslash}m{2cm}|>{\centering\arraybackslash}m{2cm}|>{\centering\arraybackslash}m{2cm}|}
% 			$13$ là số nguyên tố. & ? & ? & ? \\
% 			\hline
% 			Tổng độ dài hai cạnh bất kì của một tam giác nhỏ hơn độ dài cạnh còn lại. & ? & ? & ? \\
% 			\hline
% 			Bạn đã làm bài tập chưa? & ? & ? & ? \\
% 			\hline
% 			Thời tiết hôm nay thật đẹp! & ? & ? & ? \\
% 			\hline
% 			$9>2$. & ? & ? & ? \\
% 			\hline
% 			$27$ chia hết cho $5$. & ? & ? & ? \\
% 			\hline
% 			$2+3=6$. & ? & ? & ? \\
% 			\hline
% 			$36$ là số chính phương. & ? & ? & ? \\
% 			\hline
% 			Chó có khôn hơn lợn không? & ? & ? & ? \\
% 			\hline
% 		\end{tabular}
% 	\end{center}
% 	\loigiai
% 	{
% 		\begin{center}
% 			\begin{tabular}{|>{\centering\arraybackslash}m{5.5cm}|>{\centering\arraybackslash}m{2cm}|>{\centering\arraybackslash}m{2cm}|>{\centering\arraybackslash}m{2cm}|}
% 				\hline
% 				Câu & Không phải mệnh đề & Mệnh đề đúng & Mệnh đề sai \\
% 				\hline
% 			\end{tabular}
% 			\begin{tabular}{|>{\raggedright\arraybackslash}m{5.5cm}|>{\centering\arraybackslash}m{2cm}|>{\centering\arraybackslash}m{2cm}|>{\centering\arraybackslash}m{2cm}|}
% 				$13$ là số nguyên tố. &  & x &  \\
% 				\hline
% 				Tổng độ dài hai cạnh bất kì của một tam giác nhỏ hơn độ dài cạnh còn lại. &  &  & x \\
% 				\hline
% 				Bạn đã làm bài tập chưa? & x &  &  \\
% 				\hline
% 				Thời tiết hôm nay thật đẹp! & x &  &  \\
% 				\hline
% 				$9>2$. &  & x &  \\
% 				\hline
% 				$27$ chia hết cho $5$. &  &  & x \\
% 				\hline
% 				$2+3=6$. &  &  & x \\
% 				\hline
% 				$36$ là số chính phương. &  & x &  \\
% 				\hline
% 				Chó có khôn hơn lợn không? & x &  &  \\
% 				\hline
% 			\end{tabular}
% 		\end{center}
% 	}
% \end{vd}

\subsubsection{Bài tập tự luận}
\begin{bt}%[Thành Đức Trung]%[0D1Y1-1]
	Trong các phát biểu sau, phát biểu nào là mệnh đề toán học?
	\begin{enumerate}
		\item Tích hai số thực trái dấu là một số thực âm.
		\item Mọi số tự nhiên đều là số dương.
		\item Có sự sống ngoài Trái Đất.
		\item Ngày $1$ tháng $5$ là ngày Quốc tế Lao động.
	\end{enumerate}
	\loigiai
	{
		\begin{itemize}
			\item Phát biểu \lq\lq  Tích hai số thực trái dấu là một số thực âm\rq\rq\ là mệnh đề toán học.
			\item Phát biểu \lq\lq  Mọi số tự nhiên đều là số dương\rq\rq\ là mệnh đề toán học.
			\item Phát biểu \lq\lq  Có sự sống ngoài Trái Đất\rq\rq\ là mệnh đề nhưng không là mệnh đề toán học.
			\item Phát biểu \lq\lq  Ngày $1$ tháng $5$ là ngày Quốc tế Lao động\rq\rq\ là mệnh đề nhưng không là mệnh đề toán học.
		\end{itemize}
	}
\end{bt}
\begin{bt}%[Thành Đức Trung]%[0D1Y1-2]
	Xét tính đúng sai của mỗi mệnh đề sau
	\begin{listEX}[2]
		\item $\pi<\dfrac{10}{3}$.
		\item Phương trình $3x+7=0$ có nghiệm.
		\item Tồn tại số cộng với chính nó bằng $0$.
		\item $2022$ là hợp số.
	\end{listEX}
	\loigiai
	{
		\begin{enumerate}
			\item Mệnh đề \lq\lq  $\pi<\dfrac{10}{3}$\rq\rq\ là mệnh đề đúng.
			\item Mệnh đề \lq\lq  Phương trình $3x+7=0$ có nghiệm\rq\rq\ là mệnh đề đúng vì $3x+7=0 \Leftrightarrow x=-\dfrac{7}{3}$.
			\item Mệnh đề \lq\lq  Tồn tại số cộng với chính nó bằng $0$\rq\rq\ là mệnh đề đúng vì $0+0=0$.
			\item Mệnh đề \lq\lq  $2022$ là hợp số\rq\rq\ là mệnh đề đúng vì $2022$ có ít nhất $3$ ước là $1$; $2$ và $2022$.
		\end{enumerate}
	}
\end{bt}

\begin{bt}%[Thành Đức Trung]%[0D1Y1-2]
	Xét tính đúng sai của mỗi mệnh đề sau
	\begin{listEX}[2]
		\item $1993$ chia hết cho $3$.
		\item $\sqrt{12}$ là một số hữu tỉ.
		\item $9$ là một số chính phương.
		\item $|-1997|\leqslant0$.
	\end{listEX}
	\loigiai
	{
		\begin{enumerate}
			\item Mệnh đề \lq\lq  $1993$ chia hết cho $3$\rq\rq\ là mệnh đề sai vì $1993$ chia $3$ dư $1$.
			\item Mệnh đề \lq\lq  $\sqrt{12}$ là một số hữu tỉ\rq\rq\ là mệnh đề sai vì $\sqrt{12}$ là một số vô tỉ.
			\item Mệnh đề \lq\lq  $9$ là một số chính phương\rq\rq\ là mệnh đề đúng vì $\sqrt{9}=3$.
			\item Mệnh đề \lq\lq  $|-1997|\leqslant0$\rq\rq\ là mệnh đề sai vì $|-1997|=1997>0$.
		\end{enumerate}
	}
\end{bt}

\begin{bt}%[Thành Đức Trung]%[0D1Y1-2]
	Xét tính đúng sai của mỗi mệnh đề sau
	\begin{listEX}[3]
		\item $\sqrt{3}+\sqrt{2}=\dfrac{1}{\sqrt{3}-\sqrt{2}}$.
		\item $\left(\sqrt{2}-\sqrt{18}\right)^2\geqslant8$.
		\item $\left(\sqrt{3}+\sqrt{12}\right)^2$ là một số hữu tỉ.
		\item! $x=2$ là một nghiệm của phương trình $\dfrac{x^2-4}{x-2}=0$.
	\end{listEX}
	\loigiai
	{
		\begin{enumerate}
			\item Mệnh đề \lq\lq  $\sqrt{3}+\sqrt{2}=\dfrac{1}{\sqrt{3}-\sqrt{2}}$\rq\rq\ là mệnh đề đúng.
			\item Mệnh đề \lq\lq  $\left(\sqrt{2}-\sqrt{18}\right)^2\geqslant8$\rq\rq\ là mệnh đề đúng vì $\left(\sqrt{2}-\sqrt{18}\right)^2=8$.
			\item Mệnh đề \lq\lq  $\left(\sqrt{3}+\sqrt{12}\right)^2$ là một số hữu tỉ\rq\rq\ là mệnh đề đúng vì $\left(\sqrt{3}+\sqrt{12}\right)^2=27$.
			\item Mệnh đề \lq\lq  $x=2$ là một nghiệm của phương trình $\dfrac{x^2-4}{x-2}=0$\rq\rq\ là mệnh đề sai vì $x=2$ vi phạm điều kiện xác định của phương trình.
		\end{enumerate}
	}
\end{bt}

\begin{bt}%[Thành Đức Trung]%[0D1Y1-2]
	Thay dấu \lq\lq ?\rq\rq\ bằng dấu \lq\lq  x\rq\rq\ vào ô thích hợp trong bảng sau
	\begin{center}
		\begin{tabular}{|>{\centering\arraybackslash}m{5.5cm}|>{\centering\arraybackslash}m{2cm}|>{\centering\arraybackslash}m{2cm}|>{\centering\arraybackslash}m{2cm}|}
			\hline
			Câu & Không phải mệnh đề & Mệnh đề đúng & Mệnh đề sai \\
			\hline
		\end{tabular}
		\begin{tabular}{|>{\raggedright\arraybackslash}m{5.5cm}|>{\centering\arraybackslash}m{2cm}|>{\centering\arraybackslash}m{2cm}|>{\centering\arraybackslash}m{2cm}|}
			Hãy đi nhanh lên! & ? & ? & ? \\
			\hline
			$5+7+4=15$. & ? & ? & ? \\
			\hline
			Phương trình $x^2-3x+2=0$ có nghiệm. & ? & ? & ? \\
			\hline
			$2^{10}-1$ chia hết cho $11$. & ? & ? & ? \\
			\hline
			Có vô số số nguyên tố. & ? & ? & ? \\
			\hline
			Bây giờ là mấy giờ? & ? & ? & ? \\
			\hline
			$\sqrt{5}$ là số vô tỉ. & ? & ? & ? \\
			\hline
		\end{tabular}
	\end{center}
	\loigiai
	{
		\begin{center}
			\begin{tabular}{|>{\centering\arraybackslash}m{5.5cm}|>{\centering\arraybackslash}m{2cm}|>{\centering\arraybackslash}m{2cm}|>{\centering\arraybackslash}m{2cm}|}
				\hline
				Câu & Không phải mệnh đề & Mệnh đề đúng & Mệnh đề sai \\
				\hline
			\end{tabular}
			\begin{tabular}{|>{\raggedright\arraybackslash}m{5.5cm}|>{\centering\arraybackslash}m{2cm}|>{\centering\arraybackslash}m{2cm}|>{\centering\arraybackslash}m{2cm}|}
				Hãy đi nhanh lên! & x &  &  \\
				\hline
				$5+7+4=15$. &  &  & x \\
				\hline
				Phương trình $x^2-3x+2=0$ có nghiệm. &  & x &  \\
				\hline
				$2^{10}-1$ chia hết cho $11$. &  & x &  \\
				\hline
				Có vô số số nguyên tố. &  & x &  \\
				\hline
				Bây giờ là mấy giờ? & x &  &  \\
				\hline
				$\sqrt{5}$ là số vô tỉ. &  & x &  \\
				\hline
			\end{tabular}
		\end{center}
	}
\end{bt}

\begin{dang}{Mệnh đề phủ định, mệnh đề đảo, mệnh đề kéo theo, tương đương}
\end{dang}

\subsubsection{Ví dụ minh hoạ}

\begin{vd}
	Phát biểu mệnh đề phủ định của các mệnh đề sau và cho biết tính đúng sai của mệnh đề phủ định đó.
	\begin{enumerate}
		\item $P\colon$\lq\lq  $\sqrt{5}$ là số hữu tỉ\rq\rq.
		\item $Q\colon $\lq\lq  Tổng ba góc trong một tam giác bằng $180^\circ$\rq\rq.
		\item $R\colon$\lq\lq  $25$ là một số chính phương\rq\rq.
		\item $T\colon $\lq\lq  Hình vuông không phải là hình bình hành\rq\rq.
	\end{enumerate}
	\loigiai{
		\begin{enumerate}
			\item Mệnh đề phủ định của mệnh đề $P$ là $\overline{P}\colon$\lq\lq  $\sqrt{5}$ không phải là số hữu tỉ\rq\rq.\\
			Đây là một mệnh đề đúng vì $\sqrt{5}$ không thể biểu diễn dưới dạng $\dfrac{a}{b}$ với $a$, $b\in \mathbb{Z}$.
			\item Mệnh đề phủ định của mệnh đề $Q$ là $\overline{Q}\colon $\lq\lq  Tổng ba góc trong tam giác không bằng $180^\circ$.\\
			Đây là một mệnh đề sai.
			\item Mệnh đề phủ định của mệnh đề $R$ là $\overrightarrow{R}\colon $\lq\lq  $25$ không phải là một số chính phương\rq\rq.\\
			Đây là một mệnh đề sai.
			\item Mệnh đề phủ định của mệnh đề $T$ là $\overline{T}\colon$\lq\lq  Hình vuông là hình bình hành\rq\rq.\\
			Đây là một mệnh đề đúng.
		\end{enumerate}
	}
\end{vd}

\begin{vd}
	Cho tam giác $ABC$. Xét hai mệnh đề $P\colon $\lq\lq  tam giác $ABC$ vuông\rq\rq\text{} và $Q\colon $\lq\lq  $AB^2+AC^2=BC^2$\rq\rq. Phát biểu và cho biết mệnh đề sau đúng hay sai.
	\begin{enumEX}{2}
		\item $P\Rightarrow Q$.
		\item $Q\Rightarrow P$.
	\end{enumEX}
	\loigiai{
		\begin{enumerate}
			\item Mệnh đề $P\Rightarrow Q$ là \lq\lq  Nếu tam giác $ABC$ vuông thì $AB^2+AC^2=BC^2$.\\
			Mệnh đề $P\Rightarrow Q$ sai vì chưa chắc tam giác $ABC$ đã vuông tại $A$.
			\item Mệnh đề $Q\Rightarrow P$ là \lq\lq  Nếu tam giác $ABC$ có $AB^2+AC^2=BC^2$ thì tam giác vuông\rq\rq.\\
			Mệnh đề $Q\Rightarrow P$ đúng (theo định lí Py-ta-go).
		\end{enumerate}
	}
\end{vd}

\begin{vd}
	Cho $\triangle ABC$ có hai đường trung tuyến $BM$, $CN$. Lập mệnh đề $P\Rightarrow Q$ và mệnh đề đảo của nó, rồi xét tính đúng sai của chúng khi
	\begin{enumerate}
		\item $P\colon$\lq\lq  Góc $A$ tù\rq\rq\text{} và $Q\colon $\lq\lq  Cạnh $BC$ lớn nhất\rq\rq.
		\item $P\colon$\lq\lq  $BM=CN$\rq\rq\text{} và $Q\colon $\lq\lq  tam giác $ABC$ cân\rq\rq.
	\end{enumerate}
	\loigiai{
		\begin{enumerate}
			\item $P\colon$\lq\lq  Góc $A$ tù\rq\rq\text{} và $Q\colon $\lq\lq  Cạnh $BC$ lớn nhất\rq\rq.
			\begin{itemize}
				\item Mệnh đề $P\Rightarrow Q$ là \lq\lq  Nếu góc $A$ tù thì cạnh $BC$ lớn nhất\rq\rq. Đây là mệnh đề đúng.
				\item Mệnh đề $Q\Rightarrow P$ là \lq\lq  Nếu cạnh $BC$ lớn nhất thì $A$ là góc tù\rq\rq. Đây là mệnh đề sai ($A$ vẫn có thể là góc nhọn hoặc góc vuông).
			\end{itemize}
			\item $P\colon$\lq\lq  $BM=CN$\rq\rq\text{} và $Q\colon $\lq\lq  tam giác $ABC$ cân\rq\rq.
			\begin{itemize}
				\item Mệnh đề $P\Rightarrow
				Q$ là \lq\lq  Nếu $BM=CN$ thì tam giác $ABC$ cân\rq\rq. Đây là một mệnh đề đúng.
				\item Mệnh đề $Q\Rightarrow P$ là \lq\lq  Nếu tam giác $ABC$ cân thì $BM=CN$\rq\rq. Đây là một mệnh đề sai vì chưa chắc tam giác $ABC$ đã cân tại $A$.
			\end{itemize}
		\end{enumerate}
	}
\end{vd}

\begin{vd}
	Cho định lí \lq\lq  Nếu $MA\perp MB$ thì $M$ thuộc đường tròn đường kính $AB$\rq\rq. Hãy xác định giả thiết của định lí, kết luận của định lí và dùng thuật ngữ \lq\lq  điều kiện cần\rq\rq, \lq\lq  điều kiện đủ\rq\rq\text{} để phát biểu lại định lí.
	\loigiai{
		Giả thiết của định lí là $MA\perp MB$.\\
		Kết luật của định lí là $M$ thuộc đường  tròn đường kính $AB$.
		\begin{itemize}
			\item Điều kiện cần để $MA\perp MB$ là $M$  thuộc đường tròn đường kính $AB$.
			\item Điều kiện đủ để $M$ thuộc đường tròn đường kính $AB$ là $MA\perp MB$.
		\end{itemize}
	}
\end{vd}

\begin{vd}
	Phát biểu mệnh đề $P\Leftrightarrow Q$ và cho biết tính đúng sai của nó.
	\begin{enumerate}
		\item $P\colon $\lq\lq  Tứ giác $ABCD$ là hình vuông\rq\rq \text{ và }$Q\colon$ \lq\lq  Tứ giác $ABCD$ là hình thoi có $AC=BD$\rq\rq.
		\item $P\colon $\lq\lq  Điểm $M$ nằm trên phân giác của góc $xOy$\rq\rq \text{} và $Q\colon $\lq\lq  Điểm $M$ cách đều hai cạnh $Ox$, $Oy$\rq\rq.
		\item $P\colon$\lq\lq  Tam giác $ABC$ đều\rq\rq \text{} và $Q\colon$\lq\lq  Tam giác $ABC$ có ba đường cao bằng nhau\rq\rq.
	\end{enumerate}
	\loigiai{
		\begin{enumerate}
			\item Mệnh đề tương đương $P\Leftrightarrow Q$ là \lq\lq  Tứ giác $ABCD$ là hình vuông khi và chỉ khi tứ giác $ABCD$ là hình thoi có $AC=BD$\rq\rq.\\
			Mệnh đề $P\Leftrightarrow Q$ đúng vì mệnh đề $P\Rightarrow Q$ và mệnh đề $Q\Rightarrow P$ là hai mệnh đề đúng.
			\item Mệnh đề tương đương $P\Leftrightarrow Q$ là \lq\lq  Điểm $M$ nằm trên phân giác của góc $xOy$ khi và chỉ khi điểm $M$ cách đều hai cạnh $Ox$, $Oy$\rq\rq.\\
			Mệnh đề $P\Leftrightarrow Q$ đúng vì mệnh đề $P\Rightarrow Q$ và $Q\Rightarrow P$ là hai mệnh đề đúng.
			\item Mệnh đề tương đương $P\Leftrightarrow Q$ là \lq\lq  Tam giác $ABC$ đều khi và chỉ khi ba đường cao bằng nhau\rq\rq.\\
			Mệnh đề $P\Leftrightarrow Q$ đúng vì hai mệnh đề $P\Rightarrow Q$ và $Q\Rightarrow P$ là hai mệnh đề đúng. 
		\end{enumerate}
	}
\end{vd}

\subsubsection{Bài tập tự luận}

\begin{bt}
	Phát biểu mệnh đề phủ định của các mệnh đề sau
	\begin{enumerate}
		\item $A\colon$\lq\lq  $2022$ chia hết cho $7$\rq\rq.
		\item $B\colon$\lq\lq  Tích của ba số tự nhiên liên tiếp chia hết cho $6$\rq\rq.
		\item $C\colon $\lq\lq  Phương trình $x^2+x+1=0$ vô nghiệm\rq\rq.
	\end{enumerate}
	\loigiai{
		\begin{enumerate}
			\item Mệnh đề phủ định của mệnh đề $A$ là $\overline{A}\colon$\lq\lq  $2022$ không chia hết cho $7$\rq\rq.
			\item Mệnh đề phủ định của mệnh đề $B$ là $\overline{B}\colon$\lq\lq  Tích của ba số tự nhiên liên tiếp không chia hết cho $6$\rq\rq.
			\item Mệnh đề phủ định của mệnh đề $C$ là $\overline{C}\colon$\lq\lq  Phương trình $x^2-x+1=0$ có nghiệm\rq\rq.
		\end{enumerate}
	}
\end{bt}

\begin{bt}
	Hãy lập mệnh đề phủ định của các mệnh đề sau đây và cho biết các mệnh đề phủ định đó đúng hay sai?
	\begin{enumerate}
		\item $A\colon$\lq\lq  $735$ là số nguyên tố\rq\rq.
		\item $B\colon$\lq\lq  Phương trình $x^2+9x-2011=0$ vô nghiệm\rq\rq.
		\item $C\colon$\lq\lq  Đường tròn có một tâm đối xứng\rq\rq.
		\item $D\colon$\lq\lq  Hai đường thẳng song song không có điểm chung\rq\rq.
	\end{enumerate}
	\loigiai{
		\begin{enumerate}
			\item Phủ định của mệnh đề $A$ là $\overline{A}\colon$\lq\lq  Số $735$ không phải là số nguyên tố\rq\rq. Đây là mệnh đề đúng vì $735\,\vdots\,5$.
			\item Phủ định của mệnh đề $B$ là $\overline{B}\colon$\lq\lq  Phương trình $x^2+9x-2022=0$ có nghiệm\rq\rq. Đây là mệnh đề đúng vì $a=1$ và $c=-2022$ trái dấu.
			\item Phủ định của mệnh đề $C$ là $\overline{C}\colon$\lq\lq  Không phải đường tròn có một tâm đối xứng\rq\rq. Đây là một mệnh đề sai.
			\item Phủ định của mệnh đề $D$ là $\overline{D}\colon$\lq\lq  Hai đường thẳng song song có điểm chung\rq\rq. Đây là mệnh đề sai.
		\end{enumerate}
	}
\end{bt}

\begin{bt}
	Phát biểu mệnh đề đảo của mệnh đề sau và xét tính đúng sai của mệnh đề đảo.
	\begin{enumerate}
		\item Nếu một số chia hết cho $6$ thì số đó chia hết cho $3$.
		\item Nếu một số là số tự nhiên lẻ thì nó là số nguyên tố.
		\item Nếu $\dfrac{AB}{MN}=\dfrac{AC}{MP}$ thì $\triangle ABC\backsim \triangle MNP$.
	\end{enumerate}
	\loigiai{
		\begin{enumerate}
			\item Nếu một số chia hết cho $3$ thì số đó chia hết cho $6$. Đây là mệnh đề sai.
			\item Nếu một số là số nguyên tố thì nó là số lẻ. Đây là mệnh đề sai vì $2$ là số nguyên tố chẵn.
			\item Nếu $\triangle ABC\backsim \triangle MNP$ thì $\dfrac{AB}{MN}=\dfrac{AC}{MP}$. Đây là mệnh đề đúng.
		\end{enumerate}
	}
\end{bt}

\begin{bt}
	Phát biểu mệnh đề đảo của mệnh đề sau và cho biết tính đúng sai của mệnh đề đảo.
	\begin{enumerate}
		\item Nếu hai tam giác bằng nhau thì chúng có diện tích bằng nhau.
		\item Nếu tứ giác $ABCD$ là hình bình hành thì nó có hai cạnh đối song song và bằng nhau.
	\end{enumerate}
	\loigiai{
		\begin{enumerate}
			\item Nếu hai tam giác có diện tích bằng nhau thì nó bằng nhau.\\
			Đây là một mệnh đề sai.
			\item Nếu tứ giác $ABCD$ có hai cạnh đối song song và bằng nhau thì nó là hình bình hành.\\
			Đây là mệnh đề đúng. 
		\end{enumerate}
	}
\end{bt}

\begin{bt}
	Hãy xác định giả thiết, kết luận đồng thời dùng thuật ngữ \lq\lq  điều kiện đủ\rq\rq,\text{} để phát biểu các định lí sau
	\begin{enumerate}
		\item Nếu $a$ và $b$ là hai số hữu tỉ thì tổng $a+b$ cũng là số hữu tỉ.
		\item Nếu một số tự nhiên $n$ có tổng các chữ số chia hết cho $9$ thì nó chia hết cho $9$.
	\end{enumerate}
	\loigiai{
		\begin{enumerate}
			\item Giả thiết của định lí là \lq\lq  $a$ và $b$ là hai số hữu tỉ\rq\rq.\\
			Kết luận của định lí là \lq\lq  tổng $a+b$ là số hữu tỉ\rq\rq.\\
			Phát biểu định lí dưới dạng điều kiện đủ \lq\lq  Điều kiện đủ để tổng $a+b$ là số hữu tỉ là cả hai số $a$ và $b$ đều là số hữu tỉ\rq\rq.
			\item Giả thiết của định lí là \lq\lq  Một số tự nhiên $n$ có tổng các chữ số chia hết cho $9$\rq\rq.\\
			Kết luận của định lí là \lq\lq  $n$ chia hết cho $9$\rq\rq.\\
			Phát biểu định lí dưới dạng điều kiện đủ \lq\lq  Điều kiện đủ để $n$ chia hết cho $9$ là tổng các chữ số của $n$ chia hết cho $9$\rq\rq.
		\end{enumerate}
	}
\end{bt}

\begin{bt}
	Cho định lí \lq\lq  Cho số tự nhiên $n$, nếu $n^5$ chia hết cho $5$ thì $n$ chia hết cho $5$\rq\rq. Định lí này được viết dưới dạng $P\Rightarrow Q$.
	\begin{enumerate}
		\item Hãy xác định các mệnh đề $P$ và $Q$.
		\item Phát biểu định lí trên bằng cách dùng thuật ngữ \lq\lq  điều kiện cần\rq\rq.
		\item Phát biểu định lí trên bằng cách dùng thuật ngữ \lq\lq  điều kiện đủ\rq\rq.
		Hãy phát biểu định lí đảo (nếu có) của định lí trên rồi dùng các thuật ngữ \lq\lq  điều kiện cần và điều kiện đủ\rq\rq\text{} phát biểu gộp cả hai định lí thuận và đảo.
	\end{enumerate}
	\loigiai{
		\begin{enumerate}
			\item $P\colon $\lq\lq  $n$ là số tự nhiên và $n^5$ chia hết cho $5$\rq\rq, $Q\colon $\lq\lq  $n$ chia hết cho $5$\rq\rq.
			\item Với $n$ là số tự nhiên, $n$ chia hết cho $5$ là điều kiện cần để $n^5$ chia hết cho $5$.
			\item Với $n$ là số tự nhiên, $n^5$ chia hết cho $5$ là điều kiện đủ để $n$ chia hết cho $5$.
			\item 
			\begin{itemize}
				\item Định lí đảo \lq\lq  Cho số tự nhiên $n$, nếu $n$ chia hết cho $5$ thì $n^5$ chia hết cho $5$\rq\rq.\\
				\item Phát biểu gộp cả hai định lí \lq\lq  Điều kiện cần và đủ để $n$ chia hết cho $5$ là $n^5$ chia hết cho $5$\rq\rq.
			\end{itemize} 
		\end{enumerate}
	}
\end{bt}

\begin{bt}
	Cho tam giác ABC với trung tuyến $AM$. Xét hai mệnh đề\\
	$P\colon $\lq\lq  Tam giác $ABC$ vuông tại $A$\rq\rq.
	$Q\colon $\lq\lq  Trung tuyến $AM$ bằng một nửa cạnh $BC$\rq\rq
	\begin{enumerate}
		\item Hãy phát biểu mệnh đề $P\Rightarrow Q$. Mệnh đề này đúng hay sai?
		\item Hãy phát biểu mệnh đề $Q\Rightarrow P$. Mệnh đề này đúng hay sai?
		\item Phát biểu mệnh đề $P\Leftrightarrow Q$ và cho biết mệnh đề đó đúng hay sai?
	\end{enumerate}
	\loigiai{
		\begin{enumerate}
			\item Mệnh đề $P\Rightarrow Q$ là \lq\lq  Nếu tam giác $ABC$ vuông tại $A$ thì trung tuyến $AM$ bằng một nửa cạnh $BC$\rq\rq.\\
			Đây là mệnh đề đúng.
			\item Mệnh đề $Q\Rightarrow P$ là \lq\lq  Nếu trung tuyến $AM$ bằng một nửa cạnh $BC$ thì tam giác $ABC$ vuông tại $A$\rq\rq.\\
			Đây là mệnh đề đúng.
			\item Mệnh đề $P\Leftrightarrow Q$ là \lq\lq  Tam giác $ABC$ vuông tại $A$ khi và chỉ khi trung tuyến $AM$ bằng một nửa cạnh $BC$\rq\rq.\\
			Mệnh đề tương đương $P\Leftrightarrow Q$ đúng vì $P\Rightarrow Q$ và $Q\Rightarrow P$ là hai mệnh đề đúng.
		\end{enumerate}
	}
\end{bt}

\begin{bt}
	Phát biểu mệnh đề $P\Rightarrow Q$ và phát biểu mệnh đề đảo, xét tính đúng sai của nó.
	\begin{enumerate}
		\item $P\colon $\lq\lq  Tứ giác $ABCD$ là hình chữ nhật\rq\rq \text{} và $Q\colon $\lq\lq  Tứ giác $ABCD$ có $AC$ và $BD$ cắt nhau tại trung điểm của mỗi đường\rq\rq.
		\item $P\colon$\lq\lq  Hình thang $ABCD$ nội tiếp một đường tròn \rq\rq \text{} và $Q\colon$\lq\lq  Hình thang $ABCD$ cân\rq\rq.
	\end{enumerate}
	\loigiai{
		\begin{enumerate}
			\item Mệnh đề đảo của mệnh đề $P\Rightarrow Q$ là $Q\Rightarrow P\colon $\lq\lq  Nếu tứ giác $ABCD$ có $AC$ và $BD$ cắt nhau tại trung điểm của mỗi đường thì nó là hình chữ nhật\rq\rq.\\
			Đây là một mệnh đề sai vì tứ giác có hai đường chéo cắt nhau tại trung điểm của mỗi đường thì nó chỉ là hình bình hành, chưa đủ điều kiện để là hình chữ nhật.
			\item Mệnh đề đảo của mệnh đề $P\Rightarrow Q$ là $Q\Rightarrow P\colon $\lq\lq  Nếu $ABCD$ là hình thang cân thì $ABCD$ nội tiếp một đường tròn\rq\rq.\\
			Đây là một mệnh đề đúng vì hình thang cân có tổng hai góc đối bằng $180^\circ$.
		\end{enumerate}
	}
\end{bt}

\begin{bt}
	Hãy phát biểu mệnh đề $P\Leftrightarrow Q$ và cho biết mệnh đề đó đúng hay sai nếu biết
	\begin{enumerate}
		\item $P\colon $\lq\lq  $a$ và $b$ cùng chia hết cho $c$\rq\rq\text{} và $Q\colon $\lq\lq  $a+b$ chia hết cho $c$\rq\rq.
		\item $P\colon $\lq\lq  $a$ chia hết cho $3$\rq\rq\text{} và $Q\colon $\lq\lq  $a$ chia hết cho $9$\rq\rq.
		\item $P\colon $\lq\lq  $ABCD$ là hình chữ nhật\rq\rq\text{} và $Q\colon $\lq\lq  Tứ giác $ABCD$ có ba góc vuông\rq\rq.
	\end{enumerate}
	\loigiai{
		\begin{enumerate}
			\item Mệnh đề $P\Leftrightarrow Q\colon $\lq\lq  $a$ và $b$ cùng chia hết cho $c$ nếu và chỉ nếu $a+b$ chia hết cho $c$\rq\rq.\\
			Đây là mệnh đề sai vì mệnh đề $P\Rightarrow Q$ đúng nhưng mệnh đề $Q\Rightarrow P$ là sai.
			\item Mệnh đề $P\Leftrightarrow Q\colon $\lq\lq  $a$ chia hết cho $3$ nếu và chỉ nếu $a$ chia hết cho $9$\rq\rq.\\
			Đây là mệnh đề sai vì mệnh đề $P\Rightarrow Q$ là mệnh đề đúng còn mệnh đề $Q\Rightarrow P$ là mệnh đề sai.
			\item Mệnh đề $P\Leftrightarrow Q\colon $\lq\lq  $ABCD$ là hình chữ nhật khi và chỉ khi nó có ba góc vuông\rq\rq.\\
			Đây là một mệnh đề đúng vì mệnh đề $P\Rightarrow Q$ và $Q\Rightarrow P$ là hai mệnh đề đúng.
		\end{enumerate}
	}
\end{bt}

\begin{dang}{Mệnh đề chứa biến- mệnh đề chứa kí hiệu $\forall$ và $\exists$}
	% Kí hiệu $\forall$ đọc là \lq \lq với mọi\rq \rq.\\
	% Kí hiệu $\exists$ đọc là \lq \lq có một\rq \rq \,(tồn tại một) hay \lq \lq có ít nhất một\rq \rq\,(tồn tại ít nhất một).\\
	% Mối quan hệ giữa $\exists$ và $\forall$.\\
	% Cho mệnh đề \lq \lq $P(x),\, x \in X$\rq \rq.\\
	% Phủ định của mệnh đề \lq \lq $ \forall x \in X,\;P(x)$\rq \rq \;là mệnh đề \lq \lq $\exists x \in X,\;\overline{P(x)}$\rq \rq.\\
	% Phủ định của mệnh đề \lq \lq $ \exists x \in X,\;P(x)$\rq \rq \;là mệnh đề \lq \lq $ \forall x \in X,\;\overline{P(x)}$\rq \rq.
\end{dang}

\subsubsection{Ví dụ minh hoạ}

\begin{vd}%[Nguyễn Cường- BG Toán 10]%[0D1Y1-1]
	Xét câu \lq \lq $n$ là số chẵn\rq \rq. (với $n$ là số nguyên) \\
	Ta chưa khẳng định được tính đúng sai của câu này. Tuy nhiên, với mỗi giá trị của $n$ thuộc tập số nguyên, câu này cho ta một mệnh đề.
	Chẳng hạn,
	\begin{itemize}
		\item Với $n=1$ ta được mệnh đề \lq \lq $1$ là số chẵn\rq \rq\, (đây là mệnh đề sai).
		\item Với $n=2$ ta được mệnh đề \lq \lq $2$ là số chẵn\rq \rq\, (đây là mệnh đề đúng).
	\end{itemize}
	Ta nói rằng câu \lq \lq $n$ là số chẵn\rq \rq\, là một mệnh đề chứa biến.	
\end{vd}
%%==========Ví dụ 2
\begin{vd}%[Nguyễn Cường- BG Toán 10]%[0D1Y1-2]
	Xét câu \lq\lq $x>1$\rq\rq. Hãy tìm hai giá trị thực của $x$, ta nhận được một mệnh đề đúng và một mệnh đề sai.
	\loigiai{\begin{enumerate}
			\item Cho $x=2$ ta được mệnh đề đúng.
			\item Cho $x=0$ ta được mệnh đề sai.
		\end{enumerate}
	}
\end{vd}
%%==========Ví dụ 3
\begin{vd}%[Nguyễn Cường- BG Toán 10]%[0D1Y1-1]
	Trong các câu sau, câu nào là mệnh đề chứa biến?
	\begin{enumerate}
		\item $18$ chia hết cho $9$;
		\item $3n$ chia hết cho $9$.
	\end{enumerate}
	\loigiai{
		\begin{enumerate}
			\item Câu \lq\lq $18$ chia hết cho $9$\rq\rq\,là mệnh đề nhưng không phải là mệnh đề chứa biến.
			\item Câu \lq\lq $3n$ chia hết cho $9$\rq\rq\,là mệnh đề chứa biến, kí hiệu là $P(n)\colon$\lq\lq $3n$ chia hết cho $9$\rq\rq.
		\end{enumerate}
	}
\end{vd}
%%==========Ví dụ 4
\begin{vd}%[Nguyễn Cường- BG Toán 10]%[0D1Y1-5]
	Cho mệnh đề $P\colon$\lq\lq  $\forall x \in \mathbb{N}: x-2>0$\rq\rq. Tìm mệnh đề phủ định của mệnh đề $P$. Xét tính đúng sai của mệnh đề $\overline{P}$.
	\loigiai{
		Ta có $\overline{P}\colon$\lq\lq  $\exists x \in \mathbb{N}: x-2\leq 0$\rq\rq.\\
		Đây là mệnh đề đúng, vì với $x=0$ thì $x-2=-2<0$.
	}
\end{vd}
%%==========Ví dụ 5
\begin{vd}%[Nguyễn Cường- BG Toán 10]%[0D1Y1-5]
	Viết mệnh đề phủ định của mệnh đề sau và xác định tính đúng sai của nó.\break
	$P\colon$ \lq\lq $\exists x\in\mathbb{R}, x^2+1=0$\rq\rq.
	\loigiai{
		Mệnh đề $P$ có thể phát biểu là \lq\lq  Tồn tại một số thực mà bình phương của nó cộng với $1$ bằng $0$\rq\rq.\\
		Phủ định của mệnh đề $P$ là \lq\lq  Không tồn tại một số thực mà bình phương của nó cộng với $1$ bằng $0$\rq\rq.\\
		Tức là \lq\lq  Mọi số thực mà bình phương của nó cộng với $1$ khác $0$\rq\rq.\\
		Ta có thể viết mệnh đề phủ định của $P$ là $\overline{P}\colon$\lq\lq $\forall x\in\mathbb{R}, x^2+1\ne 0$\rq\rq. Mệnh đề phủ định này đúng.
	}
\end{vd}

\subsubsection{Bài tập tự luận}

%%==========Bài 1
% \begin{bt}%[Nguyễn Cường- BG Toán 10]%[0D1Y1-2]
% 	Cho câu \lq\lq $x>5$\rq\rq. Hãy tìm hai giá trị thực của $x$ để từ câu đã cho, ta nhận được một mệnh đề đúng và một mệnh đề sai.
% 	\loigiai{
% 		\begin{enumerate}
% 			\item Cho $x=7$ ta được mệnh đề đúng.
% 			\item Cho $x=5$ ta được mệnh đề sai.
% 		\end{enumerate}
% 	}
% \end{bt}
%%==========Bài 2
\begin{bt}%[Nguyễn Cường- BG Toán 10]%[0D1Y1-5]
	Sử dụng kí hiệu \lq\lq $\forall$\rq\rq \,để viết mỗi mệnh đề sau và xét xem mệnh đề đó là đúng hay sai, giải thích vì sao.
	\begin{enumerate}
		\item $P\colon$\lq\lq  Với mọi số thực $x, x^2+1>0$\rq\rq.
		\item $Q\colon$\lq\lq  Với mọi số tự nhiên $n, n^2+n$ chia hết cho $6$\rq\rq.
	\end{enumerate}
	\loigiai{
		\begin{enumerate}
			\item $P\colon$\lq\lq  Với mọi số thực $x, x^2+1>0$\rq\rq.\\
			Mệnh đề được viết là $P \colon \lq\lq \forall x \in \mathbb{R}, x^2+1>0$\rq\rq.\\
			Xét một số thực $x$ tùy ý, ta phải chứng tỏ rằng $x^2+1>0$.\\
			Thật vậy, ta có $x^2+1 \geq 1>0$.\\
			Vậy mệnh đề $P$ là mệnh đề đúng.
			\item $Q\colon$\lq\lq  Với mọi số tự nhiên $n, n^2+n$ chia hết cho $6$\rq\rq.\\
			Mệnh đề được viết là $Q\colon\lq\lq  \forall n \in \mathbb{N},\left(n^2+n\right) \,\vdots\, 6$\rq\rq.\\
			Để chứng minh mệnh đề $Q$ là sai, ta cần chỉ ra một giá trị cụ thể của $n$ để nhận được mệnh đề sai.\\
			Thật vậy, chọn $n=1$, ta thấy $n^2+n=2$ không chia hết cho $6$.\\
			Vậy mệnh đề $Q$ là mệnh đề sai.
		\end{enumerate}
	}
\end{bt}
%%==========Bài 3
\begin{bt}%[Nguyễn Cường- BG Toán 10]%[0D1Y1-5]
	Sử dụng kí hiệu \lq\lq $\exists$\rq\rq\, để viết mỗi mệnh đề sau và xét xem mệnh đề đó là đúng hay sai, giải thích vì sao.
	\begin{enumerate}
		\item $M\colon$\lq\lq  Tồn tại số thực $x$ sao cho $x^3=-8$\rq\rq.
		\item $N\colon$\lq\lq  Tồn tại số nguyên $x$ sao cho $2x+1=0$\rq\rq.
	\end{enumerate}
	\loigiai{
		\begin{enumerate}
			\item $M\colon$\lq\lq  Tồn tại số thực $x$ sao cho $x^3=-8$\rq\rq.\\
			Mệnh đề được viết là $M\colon\lq\lq \exists x \in \mathbb{R}, x^3=-8$\rq\rq.
			Để chứng tỏ mệnh đề $M$ là đúng, ta cần chỉ ra một giá trị cụ thể của $x$ để nhận được mệnh đề đúng.\\
			Thật vậy, chọn $x=-2$, ta thấy $(-2)^3=-8$.\\
			Vậy mệnh đề $M$ là mệnh đề đúng.\\
			Mệnh đề $N\colon\lq\lq \exists x \in \mathbb{Z}, 2x+1=0$\rq\rq.
			\item $N\colon$\lq\lq  Tồn tại số nguyên $x$ sao cho $2x+1=0$\rq\rq.\\
			Để chứng minh mệnh đề $N$ là sai, ta phải chứng tỏ rằng với số nguyên $x$ tùy ý thì $2x+1 \neq 0$.\\
			Thật vậy, xét một số nguyên $x$ tùy ý, ta có $2x+1 \neq 0$.\\
			Vì thế mệnh đề $N$ là mệnh đề sai.
		\end{enumerate}
	}
\end{bt}
%%==========Bài 4
\begin{bt}%[Nguyễn Cường- BG Toán 10]%[0D1B1-5]
	Bạn An nói \lq\lq  Mọi số thực đều có bình phương là một số không âm\rq\rq.
	Bạn Bình phủ định lại câu nói của bạn An \lq\lq  Có một số thực mà bình phương của nó là một số âm\rq\rq.
	\begin{enumerate}
		\item Sử dụng kí hiệu \lq\lq $\forall$\rq\rq\,để viết mệnh đề của bạn An.
		\item Sử dụng kí hiệu \lq\lq $\exists$\rq\rq\,để viết mệnh đề của bạn Bình.
	\end{enumerate}
	\loigiai{
		\begin{enumerate}
			\item \lq\lq $\forall x\in\mathbb{R}, x^2\ge 0$\rq\rq.
			\item \lq\lq $\exists x\in\mathbb{R}, x^2< 0$\rq\rq.
		\end{enumerate}	
	}
\end{bt}
%%==========Bài 5
\begin{bt}%[Nguyễn Cường- BG Toán 10]%[0D1B1-5]
	Lập mệnh đề phủ định của mỗi mệnh đề sau
	\begin{enumerate}
		\item $\forall x \in \mathbb{R},|x| \geq x$.
		\item $\exists x \in \mathbb{R}, x^2+1=0$.
	\end{enumerate}
	\loigiai{
		\begin{enumerate}
			\item Phủ định của mệnh đề \lq\lq $\forall x \in \mathbb{R},|x| \geq x$\rq\rq\,là mệnh đề \lq\lq $\exists x \in \mathbb{R},|x|<x$\rq\rq.
			\item Phủ định của mệnh đề \lq\lq $\exists x \in \mathbb{R}, x^2+1=0$\rq\rq\,là mệnh đề \lq\lq $\forall x \in \mathbb{R}, x^2+1 \neq 0$\rq\rq.
		\end{enumerate}
	}
\end{bt}
%%==========Bài 6
% \begin{bt}%[Nguyễn Cường- BG Toán 10]%[0D1B1-5]
% 	Phát biểu mệnh đề phủ định của mỗi mệnh đề sau
% 	\begin{enumerate}
% 		\item Tồn tại số nguyên chia hết cho $3$.
% 		\item Mọi số thập phân đều viết được dưới dạng phân số.
% 	\end{enumerate}
% 	\loigiai{
% 		\begin{enumerate}
% 			\item Mọi số nguyên đều không chia hết cho $3$.
% 			\item Tồn tại số thập phân không viết được dưới dạng phân số.
% 		\end{enumerate}	
% 	}
% \end{bt}
%%==========Bài 7
% \begin{bt}%[Nguyễn Cường- BG Toán 10]%[0D1B1-5]
% 	Phát biểu các mệnh đề sau
% 	\begin{enumerate}
% 		\item $\forall x \in \mathbb{R}, x^2 \geq 0$.
% 		\item $\exists x \in \mathbb{R}, \dfrac{1}{x}>x$.
% 	\end{enumerate}
% 	\loigiai{
% 		\begin{enumerate}
% 			\item Mọi số thực đều không âm.
% 			\item Tồn tại số thực sao cho nghịch đảo của số đó lớn hơn chính số đó.
% 		\end{enumerate}	
% 	}
% \end{bt}
%%==========Bài 8
\begin{bt}%[Nguyễn Cường- BG Toán 10]%[0D1B1-5]
	Lập mệnh đề phủ định của mỗi mệnh đề sau và xét tính đúng sai của mỗi mệnh đề phủ định đó
	\begin{enumerate}
		\item $\forall x \in \mathbb{R}, x^2 \neq 2x-2$.
		\item $\forall x \in \mathbb{R}, x^2 \leq 2x-1$.
		\item $\exists x \in \mathbb{R}, x+\dfrac{1}{x} \geq 2$.
		\item $\exists x \in \mathbb{R}, x^2-x+1<0$.
	\end{enumerate}
	\loigiai{
		\begin{enumerate}
			\item $\exists x \in \mathbb{R}, x^2=2x-2$.\\
			Mệnh đề này sai vì phương trình $x^2-2x+2=0$ vô nghiệm trên tập số thực.
			\item $\exists x \in \mathbb{R}, x^2 > 2x-1$.\\
			Mệnh đề này đúng vì với $x=2$ thì $2^2>2\cdot 2-1$.
			\item $\forall x \in \mathbb{R}, x+\dfrac{1}{x}<2$.\\
			Mệnh đề này sai vì với $x=1$ thì $1+\dfrac{1}{1}=2$.
			\item $\forall x \in \mathbb{R}, x^2-x+1\ge 0$.\\
			Mệnh đề này đúng vì $x^2-x+1=\left(x-\dfrac{1}{2}\right)^2+\dfrac{3}{4}> 0$ với mọi $x\in\mathbb{R}$.
		\end{enumerate}	
	}
\end{bt}
%%==========Bài 9
\begin{bt}%[Nguyễn Cường- BG Toán 10]%[0D1B1-5]
	Trong tiết học môn Toán, Nam phát biểu: \lq\lq  Mọi số thực đều có bình phương khác $1$\rq\rq. Mai phát biểu: \lq\lq  Có một số thực mà bình phương của nó bằng $1$\rq\rq.
	\begin{enumerate}
		\item Hãy cho biết bạn nào phát biểu đúng.
		\item Dùng kí hiệu $\forall$, $\exists$ để viết lại các phát biểu của Nam và Mai dưới dạng mệnh đề.
	\end{enumerate}
	\loigiai{
		\begin{enumerate}
			\item Bạn Mai phát biểu là đúng vì có số $1$ bình phương lên bằng $1$.
			\item Nam phát biểu \lq\lq $\forall x\in \mathbb{R}, x^2\ne 1$\rq\rq.\\
			Mai phát biểu \lq\lq $\exists x\in \mathbb{R}, x^2=1$\rq\rq.\\
		\end{enumerate}	
	}
\end{bt}
%%==========Bài 10
\begin{bt}%[Nguyễn Cường- BG Toán 10]%[0D1B1-5]
	Phát biểu bằng lời mệnh đề sau và cho biết mệnh đề đó đúng hay sai.
	$$
	\forall x \in \mathbb{R}, x^2+1 \leq 0
	$$
	\loigiai{
		Mọi số thực bình phương lên và cộng cho một luôn không dương.\\
		Đây là một mệnh đề sai vì $0^2+1=1>0$.	
	}
\end{bt}

\subsection{BÀI TẬP TRẮC NGHIỆM ÔN TẬP CUỐI BÀI}

% \Opensolutionfile{ansbook}[ans/ansbook-0D1-1-TN]
\Opensolutionfile{ans}[ans/ans-0D1-1-TN]

\begin{ex}%[Lương Như Quỳnh]%[0D1Y1-1]
	Phát biểu nào dưới đây là mệnh đề?
	\choice
	{\True $2+3=9$}
	{Phong cảnh đẹp quá!}
	{$5-x=7$}
	{Bây giờ là mấy giờ?}
	\loigiai{
		\lq\lq $2+3=9$\rq\rq\ là mệnh đề sai.\\
		\lq\lq  Phong cảnh đẹp quá!\rq\rq\ không là mệnh đề vì đây là câu cảm thán.\\
		\lq\lq $5-x=7$\rq\rq\ là mệnh đề chứa biến.\\
		\lq\lq  Bây giờ là mấy giờ?\rq\rq\ không là mệnh đề vì đây là câu nghi vấn.
	}
\end{ex}
\begin{ex}%[Lương Như Quỳnh]%[0D1B1-1]
	Các câu sau đây, câu nào {\bf không} là mệnh đề?
	\choice
	{Phương trình $ x^2-x+1=0$ vô nghiệm}
	{\True $x+y>1$}
	{$12$ không là số nguyên tố}
	{Hai phương trình $ x^2-4x+3=0$ và $ 2x^2-\sqrt{x+3}=0$ có nghiệm chung}
	\loigiai{
		\lq\lq  Phương trình $ x^2-x+1=0$ vô nghiệm\rq\rq\ là mệnh đề sai.\\
		\lq\lq  $12$ không là số nguyên tố\rq\rq\ là mệnh đề đúng.\\
		\lq\lq  Hai phương trình $ x^2-4x+3=0$ và $ 2x^2-\sqrt{x+3}=0$ có nghiệm chung\rq\rq\ là mệnh đề đúng.\\
		\lq\lq  $x+y>1$\rq\rq\ là mệnh đề chứa biến.}
\end{ex}
\begin{ex}%[Lương Như Quỳnh]%[0D1B1-4]
	Trong các câu sau, câu nào là mệnh đề \textbf{đúng}?
	\choice
	{Nếu $a\ge b$ thì $a^2\ge b^2$}
	{\True Nếu $a$ chia hết cho $9$ thì $a$ chia hết cho $3$}
	{Nếu bạn tự tin thì bạn thành công}
	{Nếu một tam giác có một góc bằng $60^\circ $ thì tam giác đó đều}
	\loigiai
	{
		\begin{itemize}
			\item Mệnh đề \lq\lq  Nếu $a\ge b$ thì $a^2\ge b^2$\rq\rq\ là một mệnh đề sai vì $b\le a < 0$ thì $a^2\le b^2$ .
			\item Mệnh đề \lq\lq  Nếu $a$ chia hết cho $9$ thì $a$ chia hết cho $3$\rq\rq\ là mệnh đề đúng.\\
			Vì $a$ $\vdots$ $9\Rightarrow \heva{&a=9n, n\in \mathbb{Z}\\&9\hspace{0.15cm}\vdots\hspace{0.15cm} 3}\Rightarrow a$ $\vdots$  $3$.
			\item \lq\lq  Nếu bạn tự tin thì bạn thành công\rq\rq\ chưa là mệnh đề vì chưa khẳng định được tính đúng, sai.
			\item Mệnh đề \lq\lq  Nếu một tam giác có một góc bằng $60^\circ $ thì tam giác đó đều\rq\rq\ là mệnh đề sai vì chưa đủ điều kiện để khẳng định một tam giác là đều.
		\end{itemize}
	}
\end{ex}
\begin{ex}%[Lương Như Quỳnh]%[0D1Y1-2]
	Mệnh đề nào sau đây là \textbf{sai}?
	\choice
	{Phương trình $ x^2+bx+c=0$ có nghiệm $\Leftrightarrow b^2-4c\geqslant 0$}
	{\True $\heva{
			&a>b\\
			&b>c} \Leftrightarrow a>c$}
	{$\Delta ABC$ vuông tại $A\Leftrightarrow \widehat{B}+\widehat{C}=90^\circ$}
	{ $ n^2$ chẵn $\Leftrightarrow n$ chẵn}
	\loigiai{
		Xét mệnh đề $\heva{
			&a>b\\
			&b>c\\
		} \Leftrightarrow a>c$, ta có
		\begin{itemize}
			\item $\heva{
				&a>b\\
				&b>c\\
			} \Rightarrow a>c$ đúng.
			\item $ a>c\Rightarrow \heva{
				&a>b\\
				&b>c.\\
			} $ sai. Chẳng hạn $ a=5$; $c=3$; $b=1$ thì $5>3\Rightarrow \heva{
				&5>1\\
				&1>3} $ vô lý.
		\end{itemize} 
	}
\end{ex}
\begin{ex}%[Lương Như Quỳnh]%[0D1Y1-5]
	Trong các mệnh đề sau, mệnh đề nào \textbf{sai}?
	\choice
	{$\exists x\in \mathbb{R},\,x^2-3x+2=0$}
	{$\forall x\in \mathbb{R},\,x^2+1>0$}
	{\True $\exists x\in \mathbb{R},\,x^2<0$}
	{$\forall x\in \mathbb{R},\,|x+1|\ge 0$}
	\loigiai{
		Mệnh đề \lq\lq $\exists x\in \mathbb{R},x^2<0$\rq\rq\, sai, vì $ x^2\ge 0,\,\forall x\in \mathbb{R}$.
	}
\end{ex}

\begin{ex}%[Lương Như Quỳnh]%[0D1B1-4]
	Trong các mệnh đề sau, mệnh đề nào có mệnh đề đảo \textbf{đúng}?
	\choice
	{Nếu số nguyên $n$ có chữ số tận cùng là $5$ thì số nguyên $n$ chia hết cho $5$}
	{\True Nếu tứ giác $ABCD$ có hai đường chéo cắt nhau tại trung điểm mỗi đường thì tứ giác $ABCD$ là hình bình hành}
	{Nếu tứ giác $ABCD$ là hình chữ nhật thì tứ giác $ABCD$ có hai đường chéo bằng nhau}
	{Nếu tứ giác $ABCD$ là hình thoi thì tứ giác $ABCD$ có hai đường chéo vuông góc với nhau}
	\loigiai
	{
		\begin{itemize}
			\item Mệnh đề đảo của mệnh đề \lq\lq  Nếu số nguyên $n$ có chữ số tận cùng là $5$ thì số nguyên $n$ chia hết cho $5$\rq\rq\, là \lq\lq  Nếu số nguyên $n$ chia hết cho $5$ thì số nguyên $n$ có chữ số tận cùng là $5$ \rq\rq. Mệnh đề này sai vì số nguyên $n$ cũng có thể có chữ số tận cùng là $0$.
			\item Mệnh đề đảo của mệnh đề \lq\lq  Nếu tứ giác $ABCD$ có hai đường chéo cắt nhau tại trung điểm mỗi đường thì tứ giác $ABCD$ là hình bình hành\rq\rq\, là \lq\lq  Nếu tứ giác $ABCD$ là hình bình hành thì tứ giác $ABCD$ có hai đường chéo cắt nhau tại trung điểm mỗi đường\rq\rq. Mệnh đề này đúng.
			\item Mệnh đề đảo của mệnh đề \lq\lq  Nếu tứ giác $ABCD$ là hình chữ nhật thì tứ giác $ABCD$ có hai đường chéo bằng nhau\rq\rq\, là \lq\lq  Nếu tứ giác $ABCD$ có hai đường chéo bằng nhau thì tứ giác $ABCD$ là hình chữ nhất\rq\rq. Mệnh đề này sai vì hình thang cân cũng có hai đường chéo bằng nhau, nhưng không là hình chữ nhật.
			\item Mệnh đề đảo của mệnh đề \lq\lq  Nếu tứ giác $ABCD$ là hình thoi thì tứ giác $ABCD$ có hai đường chéo vuông góc\rq\rq\, là \lq\lq  Nếu tứ giác $ABCD$ có hai đường chéo vuông góc thì tứ giác $ABCD$ là hình thoi\rq\rq. Mệnh đề này sai.
		\end{itemize}
	}
\end{ex}
\begin{ex}%[Lương Như Quỳnh]%[0D1B1-3]
	Trong các mệnh đề sau, mệnh đề nào có mệnh đề đảo là \textbf{sai}?
	\choice
	{Nếu tam giác $ABC$ cân thì tam giác có hai cạnh bằng nhau}
	{Nếu $ a$ chia hết cho $6$ thì $ a$ chia hết cho $2$ và $3$}
	{\True Nếu $ABCD$ là hình bình hành thì $AB$ song song với $CD$}
	{Nếu tứ giác có hai đường chéo vuông góc thì tứ giác đó là hình thoi}
	\loigiai{
		Mệnh đề đảo của mệnh đề \lq\lq  Nếu $ABCD$ là hình bình hành thì $AB$ song song với $CD$\rq\rq\, là \lq\lq  Nếu tứ giác $ABCD$ có $AB$ song song với $CD$ thì $ABCD$ là hình bình hành \rq\rq. Mệnh đề này sai vì tứ giác $ABCD$ có thể là hình thang có hai đáy là $AB$ và $CD$. }
\end{ex}
\begin{ex}%[Lương Như Quỳnh]%[0D1B1-5]
	Cho mệnh đề $P(x)\colon$ \lq\lq $\forall x\in \mathbb{R},\ x^2+x+1>0$\rq\rq. Mệnh đề phủ định của mệnh đề $P(x)$ là
	\choice
	{\lq\lq $\forall x\in \mathbb{R},\ x^2+x+1<0$\rq\rq}
	{\lq\lq $\forall x\in \mathbb{R},\ x^2+x+1\leqslant 0$\rq\rq}
	{\True \lq\lq $\exists x\in \mathbb{R},\ x^2+x+1\leqslant 0$\rq\rq}
	{\lq\lq $x\in \mathbb{R},\ x^2+x+1>0$\rq\rq}
	\loigiai{
		Phủ định của mệnh đề $P(x)$ là $\overline{P(x)}\colon$ \lq\lq $\exists x\in \mathbb{R},\ x^2+x+1\leqslant 0$\rq\rq.}
\end{ex}
\begin{ex}%[Lương Như Quỳnh]%[0D1Y1-3]
	Cho mệnh đề $P\colon$ \lq\lq $\exists x\in \mathbb{R},\, x<\dfrac{1}{x}$\rq\rq. Xác định mệnh đề phủ định của mệnh đề $P$.
	\choice
	{$\overline{P}\colon$ \lq\lq  $\exists x\in \mathbb{R},\, x\ge \dfrac{1}{x}$\rq\rq}
	{$\overline{P}\colon$ \lq\lq  $\forall x\in \mathbb{R},\, x> \dfrac{1}{x}$\rq\rq}
	{\True 	$\overline{P}\colon$ \lq\lq  $\forall x\in \mathbb{R},\, x\ge \dfrac{1}{x}$\rq\rq}
	{$\overline{P}\colon$ \lq\lq  $\exists x\in \mathbb{R},\, x> \dfrac{1}{x}$\rq\rq}
	\loigiai{
		Phủ định của mệnh đề $P\colon$ \lq\lq $\exists x\in \mathbb{R},\, x<\dfrac{1}{x}$\rq\rq\, là mệnh đề $\overline{P}\colon$ \lq\lq  $\forall x\in \mathbb{R},\, x\ge \dfrac{1}{x}$\rq\rq.}
\end{ex}
\begin{ex}%[Lương Như Quỳnh]%[0D1B1-2]
	Cách phát biểu nào sau đây \textbf{không} thể dùng để phát biểu mệnh đề $A \Rightarrow B$?
	\choice
	{Nếu $A$ thì $B$}
	{$A$ kéo theo $B$}
	{$A$ là điều kiện đủ để có $B$}
	{\True $A$ là điều kiện cần để có $B$}
	\loigiai{
		$A$ là điều kiện cần để có $B$ dùng để phát biểu mệnh đề $B \Rightarrow A$.
	}
\end{ex}
\begin{ex}%[Lương Như Quỳnh]%[0D1B1-5]
	Trong các mệnh đề sau đây, mệnh đề nào đúng?
	\choice
	{\True Với mọi số thực $x$, nếu $x <-2$ thì $x^2> 4$}
	{Với mọi số thực $x$, nếu $x^2< 4$ thì $x <-2$}
	{Với mọi số thực $x$, nếu $x <-2$ thì $x^2< 4$}
	{Với mọi số thực $x$, nếu $x^2> 4$ thì $x >-2$}
	\loigiai{
		Mệnh đề \lq\lq  Với mọi số thực $x$, nếu $x^2< 4$ thì $x <-2$\rq\rq\ sai. Chẳng hạn $x=1\Rightarrow{x^2}=1 < 4$ nhưng $1 >-2$.\\
		Mệnh đề \lq\lq  Với mọi số thực $x$, nếu $x <-2$ thì $x^2< 4$\rq\rq\  sai. Chẳng hạn $x=-3 <-2$ nhưng $x^2=9 > 4$.\\
		Mệnh đề \lq\lq  Với mọi số thực $x$, nếu $x^2> 4$ thì $x >-2$\rq\rq\ sai. Chẳng hạn $x=-3\Rightarrow{x^2}=9 > 4$ nhưng $-3 <-2$.}
\end{ex}
\begin{ex}%[Lương Như Quỳnh]%[0D1B1-2]
	Biết $A$ là mệnh đề sai và $B$ là mệnh đề đúng. Mệnh đề nào sau đây đúng?
	\choice
	{$B\Rightarrow A$}
	{$B\Leftrightarrow A$}
	{$\overline{A}\Leftrightarrow \overline{B}$}
	{\True $B\Rightarrow \overline{A}$}
	\loigiai{
		Ta có $\overline{A}$ và $B$ đúng nên $B\Rightarrow \overline{A}$ là mệnh đề đúng.
	}
\end{ex}
\begin{ex}%[Lương Như Quỳnh]%[0D1B1-4]
	Cho $P\Leftrightarrow Q$ là mệnh đề đúng. Khẳng định nào sau đây là \textbf{sai}?
	\choice
	{$\overline{P}\Leftrightarrow Q$ sai}
	{$\overline{P}\Leftrightarrow\overline{Q}$ đúng}
	{$\overline{Q}\Leftrightarrow P$ sai}
	{\True $\overline{P}\Leftrightarrow \overline{Q}$ sai}
	\loigiai
	{
		Ta có $P\Leftrightarrow Q$ đúng nên $P\Rightarrow Q$ đúng và $Q\Rightarrow P$ đúng.\\
		Do đó $\overline P\Rightarrow\overline Q $ đúng và $\overline Q\Rightarrow\overline P $ đúng.\\
		Vậy $\overline P\Leftrightarrow\overline Q $ đúng.
	}
\end{ex}
\begin{ex}%[Lương Như Quỳnh]%[0D1B1-2]
	Cho $A$, $B$, $C$ là ba mệnh đề đúng. Mệnh đề nào sau đây là đúng?
	\choice
	{$A\Rightarrow (B\Rightarrow \overline{C})$}
	{$C\Rightarrow \overline{A}$}
	{$B\Rightarrow (\overline{A\Rightarrow C})$}
	{\True $C\Rightarrow (A\Rightarrow B)$}
	\loigiai{
		Ta có $A$, $B$, $C$ là ba mệnh đề đúng nên 
		\begin{itemize}
			\item $ B\Rightarrow \overline{C} $ sai và $A\Rightarrow (B\Rightarrow \overline{C})$ sai.
			\item $ \overline{A} $ sai và $C\Rightarrow \overline{A}$ sai.
			\item $ \overline{A}\Rightarrow C $ đúng và $B\Rightarrow (\overline{A\Rightarrow C})$ sai.
			\item $ A\Rightarrow B $ đúng và $C\Rightarrow (A\Rightarrow B)$ đúng.
	\end{itemize}}
\end{ex}
\begin{ex}%[Lương Như Quỳnh]%[0D1B1-2]
	Trong các mệnh đề nào sau đây mệnh đề nào \textbf{sai}?
	\choice
	{Hai tam giác bằng nhau khi và chỉ khi chúng đồng dạng và có một góc bằng nhau}
	{Một tứ giác là hình chữ nhật khi và chỉ khi chúng có $3$ góc vuông}
	{\True Một tam giác là vuông khi và chỉ khi nó có một góc bằng tổng hai góc còn lại}
	{Một tam giác là đều khi và chỉ khi chúng có hai đường trung tuyến bằng nhau và có một góc bằng $60^{\circ}$}
	\loigiai{
		Mệnh đề \lq\lq  Một tam giác là vuông khi và chỉ khi nó có một góc bằng tổng hai góc còn lại\rq\rq sai. Chẳng hạn tam giác có $A=60^\circ$, $B=70^\circ$, $C=50^\circ$ nhưng tam giác $ABC$ không là tam giác vuông.}
\end{ex}
\begin{ex}%[Lương Như Quỳnh]%[0D1B1-2]
	Trong các mệnh đề sau, mệnh đề nào là mệnh đề đúng?
	\choice
	{Tổng của hai số tự nhiên là một số chẵn khi và chỉ khi cả hai số đều là số chẵn}
	{Tích của hai số tự nhiên là một số chẵn khi và chỉ khi cả hai số đều là số chẵn}
	{Tổng của hai số tự nhiên là một số lẻ khi và chỉ khi cả hai số đều là số lẻ}
	{\True Tích của hai số tự nhiên là một số lẻ khi và chỉ khi cả hai số đều là số lẻ}
	\loigiai{
		Mệnh đề \lq\lq  Tổng của hai số tự nhiên là một số chẵn khi và chỉ khi cả hai số đều là số chẵn\rq\rq\, sai. Ví dụ: $3+5=8$ là số chẵn nhưng $3$ và $5$ là hai số lẻ.\\
		Mệnh đề \lq\lq  Tích của hai số tự nhiên là một số chẵn khi và chỉ khi cả hai số đều là số chẵn\rq\rq\, sai. Ví dụ: $2\cdot 3=6$ là số chẵn nhưng $3$ là số lẻ.\\
		Mệnh đề \lq\lq  Tổng của hai số tự nhiên là một số lẻ khi và chỉ khi cả hai số đều là số lẻ\rq\rq\, sai. Ví dụ: $1+3=4$ là số chẵn nhưng $1$, $3$ là hai số lẻ.}
\end{ex}
\begin{ex}%[Lương Như Quỳnh]%[0D1Y1-2]
	Cho mệnh đề chứa biến $P(x)\colon$ \lq\lq $x>x^3$\rq\rq. Trong các khẳng định sau, khẳng định nào đúng?
	\choice
	{\True $P(1)$ là mệnh đề sai}
	{$P(1)$ là mệnh đề đúng}
	{$P(1)$ là mệnh đề vừa đúng vừa sai}
	{$P(1)$ không phải là mệnh đề}
	\loigiai{
		Mệnh đề $P(1)\colon\lq\lq  1>1^3$\rq\rq\, sai.}
\end{ex}
\begin{ex}%[Lương Như Quỳnh]%[0D1Y1-2]
	Xét mệnh đề chứa biến $P(x)\colon\lq\lq  x\in\mathbb{R},\ x^2-2x\geqslant 0$\rq\rq. Tìm một giá trị của biến để được mệnh đề đúng.
	\choice
	{$ x=\dfrac{1}{4}$}
	{\True $ x=3$}
	{$ x=1$}
	{$ x=0{,}5$}
	\loigiai{
		\begin{itemize}
			\item Với $x=\dfrac14$ ta có $P\left(\dfrac14\right)\colon\lq\lq \left(\dfrac14\right)^2-2\cdot\dfrac14\geqslant0$\rq\rq\ là mệnh đề sai.
			\item Với $x=3$ ta có $P\left(3\right)\colon\lq\lq 3^2-2\cdot3\geqslant0$\rq\rq\ là mệnh đề đúng.
			\item Với $x=1$ ta có $P\left(1\right)\colon\lq\lq 1^2-2\cdot1\geqslant0$\rq\rq\ là mệnh đề sai.
			\item Với $x=0{,}5$ ta có $P\left(0{,}5\right)\colon\lq\lq 0{,}5^2-2\cdot0{,}5\geqslant0$\rq\rq\ là mệnh đề sai.
		\end{itemize}
	}
\end{ex}

\begin{ex}%[Lương Như Quỳnh]%[0D1B1-5]
	Mệnh đề nào dưới đây {\bf sai}?
	\choice
	{$x\left(1-2x\right)\le\dfrac{1}{8},\, \forall x$}
	{\True $x^2+2+\dfrac{1}{x^2+2}>\dfrac{5}{2},\, \forall x$}
	{$\dfrac{x^2-x+1}{x^2+x+1}\ge\dfrac{1}{3},\, \forall x$}
	{$\dfrac{x}{x^2+1}\le\dfrac{1}{2},\, \forall x$}
	\loigiai{
		Ta có
		\begin{itemize}
			\item $x\left(1-2x\right)\le\dfrac{1}{8}\Leftrightarrow 2\left(x-\dfrac{1}{4}\right)^2\ge 0$ (đúng).
			\item $\dfrac{x^2-x+1}{x^2+x+1}\ge \dfrac{1}{3}\Leftrightarrow \dfrac{3(x^2-x+1)-(x^2+x+1)}{x^2+x+1}\ge 0\Leftrightarrow \dfrac{2(x-1)^2}{x^2+x+1}\ge 0$ (đúng).
			\item $\dfrac{x}{x^2+1}\le \dfrac{1}{2}\Leftrightarrow (x-1)^2\ge 0$ (đúng).
			\item Với $x=0$ dễ thấy $0^2+2+\dfrac{1}{0^2+2}>\dfrac{5}{2}$ sai.
		\end{itemize}
	}
\end{ex}

\begin{ex}%[Lương Như Quỳnh]%[0D1K1-5]
	Mệnh đề nào sau đây {\bf sai}?
	\choice
	{$\forall x \in \mathbb{R},\,3x^2-4x+4>0$}
	{\True $\exists x \in \mathbb{R},\,(x-1)^2+(x+1)^2=0$}
	{$\exists x \in \mathbb{Q},\,x<\dfrac{1}{x}$}
	{$\exists n \in \mathbb{N},\,(1+2+3+ \cdots +n)\ \vdots\ 11$}
	\loigiai{
		\begin{itemize}
			\item Mệnh đề \lq\lq  $\forall x \in \mathbb{R},\,3x^2-4x+4>0$\rq\rq\, đúng vì $3x^2-4x+4=2x^2+(x-2)^2>0,\ \forall x\in\mathbb{R}$.
			\item Mệnh đề \lq\lq  $\exists x \in \mathbb{Q},\,x<\dfrac{1}{x}$\rq\rq\, đúng vì với $x=\dfrac{1}{2}$ thì $x<\dfrac{1}{x}$.
			\item Mệnh đề \lq\lq  $\exists n \in \mathbb{N},\,(1+2+3+ \cdots +n)\ \vdots\ 11$\rq\rq\, đúng vì với $n=10$ thì $1+2+\cdots +10=5\cdot 11\ \vdots\ 11$.
			\item Mệnh đề \lq\lq  $\exists x \in \mathbb{R},\,(x-1)^2+(x+1)^2=0$\rq\rq\, sai vì $(x-1)^2+(x+1)^2>0,\ \forall x\in\mathbb{R}$.
		\end{itemize}
	}
\end{ex}

\Closesolutionfile{ans}
% \Closesolutionfile{ansbook}
% \indapan{10}{ans/ans-0D1-1-TN}
% %\Opensolutionfile{ansbook}[ans/ansbook-BG10-2022-2]
%\Opensolutionfile{ans}[ans/ans-BG10-2022-2]
\setcounter{section}{1}
\section{Tập hợp và các phép toán trên tập hợp}
\subsection{LÝ THUYẾT}
\subsubsection{Tập hợp}
\begin{tcolorbox}
	Có thể mô tả một tập hợp bằng một trong hai cách sau:\\
	\textbf{Cách 1}. Liệt kê các phần tử của tập hợp;\\
	\textbf{Cách 2}. Chỉ ra tính chất đặc trưng cho các phần tử của tập hợp.
\end{tcolorbox}
$a \in S$: phần tử $a$ thuộc tập hợp $S$.
$a \notin S$: phần tử $a$ không thuộc tập hợp $S$.
\begin{note}
	\textbf{Chú ý:}\\
	$\bullet$ Số phần tử của tập hợp $S$ được kí hiệu là $n(S)$.\\
	$\bullet$ Tập hợp không chứa phần tử nào được gọi là tập rỗng, kí hiệu là $\varnothing$.
\end{note}
\subsubsection{Tập hợp con}
\begin{tcolorbox}
	$T \subset S \Leftrightarrow \forall x, (x \in T \Rightarrow x \in S).$ 
\end{tcolorbox} \noindent
\begin{note}
	$\bullet$ Quy ước tập rỗng là tập con của mọi tập hợp.
\end{note}

\begin{tcolorbox}
	\immini {$\bullet$ Người ta thường minh hoạ một tập hợp bằng một hình phẳng được bao quanh bởi một đường kín, gọi là biểu đồ Ven (H.1.2).}
	{\begin{tikzpicture}[>=stealth,line join=round,line cap=round,font=\footnotesize,scale=.7]
			\draw (1,1) circle (2 and 1);
			\coordinate[label=center:$X$] (X)at(-.5,1.3);
			\coordinate[label=center:$ H.1.2$] (X)at(1,-.5);
	\end{tikzpicture}}
	\immini {$\bullet$ Minh hoạ $T$ là một tập con của $S$ như Hình $1.3$.}
	{\begin{tikzpicture}[>=stealth,line join=round,line cap=round,font=\footnotesize,scale=.7]
			\coordinate[label=center:$$] (I)at(.4,1);
			\coordinate[label=center:$$] (M)at(1,1);
			\draw (.4,1) circle (3 and 2);
			\coordinate[label=center:$S$] (X)at(-2,1.5);
			\draw (1,1) circle (2 and 1);
			\coordinate[label=center:$T$] (X)at(1,.5);
			\coordinate[label=center:$ H.1.3$] (X)at(1,-1.5);
			\fill[cyan,opacity=.2] (I) ellipse (3 cm and 2 cm)--cycle;
			\fill[violet,opacity=.4] (M) ellipse (2 cm and 1 cm)--cycle;
	\end{tikzpicture}}
\end{tcolorbox}
\subsubsection{Hai tập hợp bằng nhau}
\begin{tcolorbox}
	$S=T \Leftrightarrow \heva{& S \subset T \\ & T \subset S} \Leftrightarrow \forall x,\ (x\in S \Leftrightarrow x \in T)$
\end{tcolorbox}

\subsubsection{Mối quan hệ giữa các tập hợp số}
$\bullet$ Tập hợp các số tự nhiên $\mathbb{N}=\{0 ; 1 ; 2 ; 3 ; \ldots\}$.\\
$\bullet$ Tập hợp các số nguyên $\mathbb{Z}$ gồm các số tự nhiên và các số nguyên âm:
$\mathbb{Z}=\{\ldots ;-2 ;-1 ; 0 ; 1 ; 2 ; \ldots\}$.
$\bullet$ Tập hợp các số hữu tỉ $\mathbb{Q}$ gồm các số viết được dưới dạng phân số $\dfrac{a}{b}$, với $a, b \in \mathbb{Z}, b \neq 0$. Số hữu tỉ còn được biểu diễn dưới dạng số thập phân hữu hạn hoặc vô hạn tuần hoàn.\\
$\bullet$ Tập hợp các số thực $\mathbb{R}$ gồm các số hữu tỉ và các số vô tỉ. Số vô tỉ là các số thập phân vô hạn không tuần hoàn.
\begin{tcolorbox}
	\immini { Mối quan hệ giữa các tập hợp số:  $\mathbb{N} \subset \mathbb{Z} \subset \mathbb{Q} \subset \mathbb{R}$.}
	{\begin{tikzpicture}[>=stealth,line join=round,line cap=round,font=\footnotesize,scale=.7]
			\path (1.3,1)coordinate[label=center:$$](M) (1.5,.7)coordinate[label=center:$$](N) (1.6,.5)coordinate[label=center:$$](P) (1.7,.3)coordinate[label=center:$$](Q)
			(1.5,-1.5)coordinate[label=center:$H.1.5$](H);
			\draw (1.3,1) circle (3 and 2);
			\draw (1.5,.7) circle (2 and 1.5);
			\draw (1.6,.5) circle (1.5 and 1);
			\draw (1.7,.3) circle (1 and .5);
			\fill[red,opacity=.2] (M) ellipse (3 cm and 2 cm)--cycle;
			\fill[cyan,opacity=.3] (N) ellipse (2 cm and 1.5 cm)--cycle;
			\fill[violet,opacity=.4] (P) ellipse (1.5 cm and 1 cm)--cycle;
			\fill[blue,opacity=.7] (Q) ellipse (1 cm and .5 cm)--cycle;
			\coordinate[label=center:$\mathbb{R}$] (X)at(-1,2);
			\coordinate[label=center:$\mathbb{Q}$] (Q)at(0.2,1.5);
			\coordinate[label=center:$\mathbb{Z}$] (Z)at(1,1.2);
			\coordinate[label=center:$\mathbb{N}$] (N)at(2,.5);
	\end{tikzpicture}}
\end{tcolorbox}
\subsubsection{Các tập con thường dùng của $\mathbb{R}$}
\begin{tcolorbox}
	Một số tập con thường dùng của tập số thực $\mathbb{R}$.\\
	$\bullet$ Khoảng\\
	\immini {$(a ; b)=\{x \in \mathbb{R} \mid a<x<b\}$} {\begin{tikzpicture}[>=stealth,line join=round,line cap=round,font=\footnotesize,scale=1]
			\draw[->] (0,0)--(6,0);
			\path[pattern=north east lines,pattern color=blue] (0,-3pt)rectangle(2,3pt);
			\path[pattern=north east lines,pattern color=blue] (6,-3pt)rectangle(4,3pt);
			\path 
			(2,-0.1)coordinate[label=below:$a$](a) 
			(4,-0.1)coordinate[label=below:$b$](b)
			(2,0)coordinate[label=center:$($](()
			(4,0)coordinate[label=center:$)$](s);
	\end{tikzpicture}}
	\immini {$(a ;+\infty)=\{x \in \mathbb{R} \mid x>a\}$}
	{\begin{tikzpicture}[>=stealth,line join=round,line cap=round,font=\footnotesize,scale=1]
			\draw[->] (0,0)--(6,0);
			\path[pattern=north east lines,pattern color=blue] (0,-3pt)rectangle(2,3pt);
			%\path[pattern=north east lines,pattern color=blue] (6,-3pt)rectangle(4,3pt);
			\path 
			(2,-0.1)coordinate[label=below:$a$](a) 
			%(4,0)coordinate[label=below:$a$](b)
			(2,0)coordinate[label=center:$($](()
			%(4,0)coordinate[label=center:$)$](s)
			;
	\end{tikzpicture}}
	\immini {$(-\infty ; b)=\left\{x \in \mathbb{R} \mid x<b\right\}$}
	{\begin{tikzpicture}[>=stealth,line join=round,line cap=round,font=\footnotesize,scale=1]
			\draw[->] (0,0)--(6,0);
			%\path[pattern=north east lines,pattern color=blue] (0,-3pt)rectangle(2,3pt);
			\path[pattern=north east lines,pattern color=blue] (6,-3pt)rectangle(4,3pt);
			\path 
			%(2,0)coordinate[label=below:$a$](a) 
			(4,-0.1)coordinate[label=below:$b$](b)
			%(2,0)coordinate[label=center:$($](()
			(4,0)coordinate[label=center:$)$](s)
			;
	\end{tikzpicture}}
	\immini {$(-\infty ;+\infty)$}
	{\begin{tikzpicture}[>=stealth,line join=round,line cap=round,font=\footnotesize,scale=1]
			\draw[->] (0,0)--(6,0);
			%\path[pattern=north east lines,pattern color=blue] (0,-3pt)rectangle(2,3pt);
			%\path[pattern=north east lines,pattern color=blue] (6,-3pt)rectangle(4,3pt);
			\path 
			(3,-0.1)coordinate[label=below:$O$](a) 
			%(4,0)coordinate[label=below:$a$](b)
			(3,0)coordinate[label=center:$|$](()
			%(4,0)coordinate[label=center:$)$](s)
			;
	\end{tikzpicture}}
	$\bullet$ Đoạn\\
	\immini {$[a ; b]=\{x \in \mathbb{R} \mid a \leq x \leq b\}$}
	{\begin{tikzpicture}[>=stealth,line join=round,line cap=round,font=\footnotesize,scale=1]
			\draw[->] (0,0)--(6,0);
			\path[pattern=north east lines,pattern color=blue] (0,-3pt)rectangle(2,3pt);
			\path[pattern=north east lines,pattern color=blue] (6,-3pt)rectangle(4,3pt);
			\path 
			(2,-0.1)coordinate[label=below:$a$](a) 
			(4,-0.1)coordinate[label=below:$b$](b)
			(2,0)coordinate[label=center:$[$]()
			(4,0)coordinate[label=center:$\mathrm{]}$](s);
	\end{tikzpicture}}
	$\bullet$ Nửa khoảng\\
	\immini {$[a ; b)=\{x \in \mathbb{R} \mid a \leq x<b\}$}
	{\begin{tikzpicture}[>=stealth,line join=round,line cap=round,font=\footnotesize,scale=1]
			\draw[->] (0,0)--(6,0);
			\path[pattern=north east lines,pattern color=blue] (0,-3pt)rectangle(2,3pt);
			\path[pattern=north east lines,pattern color=blue] (6,-3pt)rectangle(4,3pt);
			\path 
			(2,-0.1)coordinate[label=below:$a$](a) 
			(4,-0.1)coordinate[label=below:$b$](b)
			(2,0)coordinate[label=center:$[$]()
			(4,0)coordinate[label=center:$\mathrm{)}$](s);
	\end{tikzpicture}}
	\immini {$(a ; b]=\{x \in \mathbb{R} \mid a<x \leq b\}$}
	{\begin{tikzpicture}[>=stealth,line join=round,line cap=round,font=\footnotesize,scale=1]
			\draw[->] (0,0)--(6,0);
			\path[pattern=north east lines,pattern color=blue] (0,-3pt)rectangle(2,3pt);
			\path[pattern=north east lines,pattern color=blue] (6,-3pt)rectangle(4,3pt);
			\path 
			(2,-0.1)coordinate[label=below:$a$](a) 
			(4,-0.1)coordinate[label=below:$b$](b)
			(2,0)coordinate[label=center:$($]()
			(4,0)coordinate[label=center:$\mathrm{]}$](s);
	\end{tikzpicture}}
	\immini {$[a ;+\infty)=\{x \in \mathbb{R} \mid x \geq a\}$}
	{\begin{tikzpicture}[>=stealth,line join=round,line cap=round,font=\footnotesize,scale=1]
			\draw[->] (0,0)--(6,0);
			\path[pattern=north east lines,pattern color=blue] (0,-3pt)rectangle(2,3pt);
			%\path[pattern=north east lines,pattern color=blue] (6,-3pt)rectangle(4,3pt);
			\path 
			(2,-0.1)coordinate[label=below:$a$](a) 
			%(4,0)coordinate[label=below:$a$](b)
			(2,0)coordinate[label=center:$[$](()
			%(4,0)coordinate[label=center:$)$](s)
			;
	\end{tikzpicture}}
	\immini {$(-\infty ; b]=\{x \in \mathbb{R} \mid x \leq b\}$}
	{\begin{tikzpicture}[>=stealth,line join=round,line cap=round,font=\footnotesize,scale=1]
			\draw[->] (0,0)--(6,0);
			%\path[pattern=north east lines,pattern color=blue] (0,-3pt)rectangle(2,3pt);
			\path[pattern=north east lines,pattern color=blue] (6,-3pt)rectangle(4,3pt);
			\path 
			%(2,0)coordinate[label=below:$a$](a) 
			(4,-0.1)coordinate[label=below:$b$](b)
			%(2,0)coordinate[label=center:$($](()
			(4,0)coordinate[label=center:$\mathrm{]}$](s)
			;
	\end{tikzpicture}}
\end{tcolorbox}

\subsubsection{Giao của hai tập hợp}
\begin{tcolorbox}
	\immini{Tập hợp gồm các phần tử thuộc cả hai tập hợp $S$ và $T$ gọi là giao của hai tập hợp $S$ và $T$, kí hiệu là $S \cap T$.\\
		$S \cap T=\{x \mid x \in S$ và $x \in T\}$.}
	{\begin{tikzpicture}[>=stealth,line join=round,line cap=round,font=\footnotesize,scale=1]
			\begin{scope}
				\clip (0.5,0) ellipse (1.5 and 1 );
				\fill[pattern = north east lines]	(2.5,0) ellipse (2 and 1.5 );
			\end{scope}
			\draw (0.5,0) ellipse (1.5 and 1 );
			\draw (2.5,0) ellipse (2 and 1.5 );
			\coordinate[label=center:$S\cap T$] (M)at(1.1,.3);
	\end{tikzpicture}}
\end{tcolorbox}
\subsubsection{Hợp của hai tập hợp}
\begin{tcolorbox}
	\immini {Tập hợp gồm các phần tử thuộc tập hợp $S$ hoặc thuộc tập hợp $T$ gọi là hợp của hai tập hợp $S$ và $T$. Kí hiệu là $S \cup T$.
		\[S \cup T=\{x \mid x \in S \text { hoặc } x \in T\}.\]}
	{\begin{tikzpicture}[>=stealth,line join=round,line cap=round,font=\footnotesize,scale=.7]
			\coordinate[label=center:$$] (I)at(0,1);
			\coordinate[label=center:$$] (M)at(1,1);
			\draw[fill,pattern = north east lines] (0,1) circle (1.5 and 1);
			\draw[fill,pattern = north east lines] (1,1) circle (1.5 and 1);
			\coordinate[label=center:$S\cup T$] (X)at(.6,-.5);
			\coordinate[label=center:$S$] (X)at(-1,1);
			\coordinate[label=center:$T$] (X)at(2,1);
	\end{tikzpicture}}
\end{tcolorbox}
\subsubsection{Hiệu của hai tập hợp}
\begin{tcolorbox}
	\immini {$\bullet$ Hiệu của hai tập hợp $S$ và $T$ là tập hợp gồm các phần tử thuộc S nhưng không thuộc $T$, kí hiệu là $S \backslash T$.
		\[S \backslash T=\{x \mid x \in S \text{ và } x \notin T\}\].}
	{\begin{tikzpicture}[>=stealth,line join=round,line cap=round,font=\footnotesize,scale=.8]
			\draw (0.5,0) ellipse (1.5 and 1 );
			\draw (2.5,0) ellipse (1.5 and 1 );
			\fill[pattern = north east lines] (0.5,0) ellipse (1.5 and 1 );
			\fill[fill=white] (2.5,0) ellipse (1.5 and 1 );
			\path 
			(0,.5)coordinate[label=center:$S$](S) (2,.5)coordinate[label=center:$T$](T)
			(1.7,1.5)coordinate[label=center:$S\backslash T$](T) ;
			
	\end{tikzpicture}}
	\immini {$\bullet$ Nếu $T \subset S$ thì $S \backslash T$ được gọi là phần bù của $T$ trong $S$, kí hiệu là $C_{s} T$.}
	{	\begin{tikzpicture}[>=stealth,line join=round,line cap=round,font=\footnotesize,scale=.8]
			\draw (0.5,0) ellipse (1.5 and 1 );
			\draw (.8,0) ellipse (1 and .5 );
			\fill[pattern = north east lines] (0.5,0) ellipse (1.5 and 1 );
			\fill[fill=white] (.8,0) ellipse (1 and .5 );
			\path 
			(-.3,.5)coordinate[label=center:$S$](S) (1,0)coordinate[label=center:$T$](T)
			(1,1.3)coordinate[label=center:$C_S T$](T) ;
			
	\end{tikzpicture}}
\end{tcolorbox}
\subsection{CÁC DẠNG BÀI TẬP}
\begin{dang}{Xác định tập hợp}
	Được mô tả theo 2 cách:
	\begin{enumEX}{1}
		\item  Liệt kê tất cả các phần tử của tập hợp.
		\item  Nêu tính chất đặc trưng.
	\end{enumEX}
\end{dang}
\subsubsection{Ví dụ minh hoạ}
\begin{vd}%[BG10-2022]%[Đỗ Văn Dự]%[0D1Y2-1]
	Cho $D=\{n \in \mathbb{N} \mid n$ là số nguyên tố, $5<n<20\}$.
	\begin{enumEX}{1}
		\item Dùng kí hiệu $\in, \notin$ để viết câu trả lời cho câu hỏi sau: Trong các số $5$; $12$; $17$; $18$, số nào thuộc tập $D$, số nào không thuộc tập $D$?
		\item Viết tập hợp $D$ bằng cách liệt kê các phần tử. Tập hợp $D$ có bao nhiêu phần tử?
	\end{enumEX}
	\loigiai{
		\begin{enumEX}{1}
			\item $5 \notin D$; $12 \notin D$; $17 \in D$; $18 \notin D$.
			\item $D=\{7 ; 11 ; 13 ; 17 ; 19\}$. Tập hợp $D$ có $5$ phần tử.
		\end{enumEX}	
	}
\end{vd}

\begin{vd}%[BG10-2022]%[Đỗ Văn Dự]%[0D1B2-1] 
	Viết mỗi tập hợp sau bằng cách liệt kê các phần tử.
	\begin{enumEX}{2}
		\item $A=\left\{\left. x\in \mathbb{R}\right|\left(2x-x^2\right)\left(3x-2\right)=0\right\}$.
		\item $B=\left\{\left. x\in \mathbb{Z}\right|2x^3-3x^2-5x=0\right\}$.
		\item $C=\left\{\left. x\in \mathbb{Z}\right|2x^2-75x-77=0\right\}$.
		\item $D=\left\{\left. x\in \mathbb{R}\right|(x^2-x-2)(x^2-9)=0\right\}$.
	\end{enumEX}
	\loigiai{
		\begin{enumEX}{1}
			\item Ta giải phương trình\\
			 $\left(2x-x^2\right)\left(2x^2-3x-2\right)=0\Leftrightarrow \hoac{
				& 2x-x^2=0 \\ 
				& 2x^2-3x-2=0}\Leftrightarrow \hoac{
				& x=0\vee x=2 \\ 
				& x=-\dfrac{1}{2}\vee x=2}$.\\
			Do $x\in \mathbb{R}$ nên $A=\left\{-\dfrac{1}{2};0;2\right\}$.
			\item Ta giải phương trình $2x^3-3x^2-5x=0\Leftrightarrow x\left(2x^2-3x-5\right)=0\Leftrightarrow \hoac{&x=0\\&x=-1\\&x=\dfrac{5}{3}}$.\\
			Do $x\in \mathbb{Z}$ nên $B=\left\{0;-1\right\}$.
			\item Ta giải phương trình $2x^2-75x-77=0\Leftrightarrow \hoac{&x=-1\\&x=\dfrac{77}{2}}$.\\
			Do $x\in \mathbb{Z}$ nên $C=\left\{-1\right\}$.
	\end{enumEX}}
\end{vd}

\begin{vd}%[BG10-2022]%[Đỗ Văn Dự]%[0D1B2-1] 
	Viết mỗi tập hợp sau bằng cách liệt kê các phần tử.
	\begin{enumEX}{1}
		\item $A=\left\{\left. n\in {\mathbb{N}}^{*}\right|3<n^2<30\right\}$.
		\item $B=\left\{\left. n\in \mathbb{Z}\right|\left| n\right|<3\right\}$.
		\item $C=\left\{\left. x\right|x=3k\right.$ với $k\in \mathbb{Z}$ và $\left.-4<x<12\right\}$.
		\item $D=\left\{\left. n^2+3\right|n \in \mathbb{N} \text{ và } n<5\right\}$.
	\end{enumEX}
	\loigiai{
		\begin{enumEX}{1}
			\item Với $3<n^2<30$ và $n\in {\mathbb{N}}^{*}$ nên chọn $n=2;3;4;5$.\\
			Vậy $A=\left\{2;3;4;5\right\}$.
			\item  Vì $x<\left| 3\right|\Leftrightarrow-3<x<3$.\\
			Do $x\in \mathbb{Z}$ nên $B=\left\{-2;-1;0;1;2\right\}$.
			\item Ta có $-4<x<12\Leftrightarrow-4<3k<12\Leftrightarrow-\dfrac{4}{3}<k<4$.\\
			Do $k\in \mathbb{Z}$ nên ta chọn $k=\left\{-10;1;2;3\right\}$ suy ra $x=3k=\left\{-3;0;3;6;9\right\}$.\\
			Vậy $C=\left\{-3;0;3;6;9\right\}$.
			\item Vì $n \in \mathbb{N} \text{ và } n<5$ nên chọn  $n=0,1,2;3;4$.\\
			Vậy $A=\left\{3;4;12;19\right\}$.
		\end{enumEX}
	}
\end{vd}

\begin{vd}%[BG10-2022]%[Đỗ Văn Dự]%[0D1K2-1]
	Viết mỗi tập hợp sau bằng cách nêu tính chất đặc trưng.
	\begin{enumEX}{2}
		\item $A=\left\{\dfrac{2}{3};\dfrac{3}{8};\dfrac{4}{15};\dfrac{5}{24};\dfrac{6}{35}\right\}$.
		\item $B=\left\{0;3;8;15;24;35\right\}$.
		\item $C=\left\{-4;1;6;11;16\right\}$.
		\item $D=\left\{1;-2;7\right\}$.
	\end{enumEX}
	\loigiai{
		\begin{enumEX}{2}
			\item $A=\left\{\left. \dfrac{n}{n^2-1}\right|n\in \mathbb{N},2\le n\le 6\right\}$.
			\item $B=\left\{\left. n^2-1\right|n\in \mathbb{N},1\le n\le 6\right\}$.
			\item $C=\left\{\left. n\in \mathbb{N}\right|\right.\left. 5n-4\right\}$.
			\item $D=\left\{\left. x\in \mathbb{R}\right|\left(x-1\right)\left(x+2\right)\left(x-7\right)=0\right\}$.
		\end{enumEX}
	}
\end{vd}
\subsubsection{Bài tập tự luận}
\begin{bt}%[Huỳnh Quy]%[0D1B2-1]
	Liệt kê các phần tử của các tập hợp sau:
	\begin{enumerate}
		\item $A=\left\lbrace n\in \mathbb{N} \mid n<5\right\rbrace$.
		\item $B$ là tập hợp các số tự nhiên lớn hơn $0$ và nhỏ hơn $5$.
		\item $C=\left\lbrace x\in \mathbb{R}\mid (x-1)(x+2)=0\right\rbrace$.
	\end{enumerate}
	\loigiai{
		\begin{enumerate}
			\item $A=\left\lbrace 0;1;2;3;4\right\rbrace$.
			\item $B=\left\lbrace 1;2;3;4\right\rbrace$.
			\item Ta có $(x-1)(x+2)=0 \Leftrightarrow \hoac{&x=1\\&x=-2.}$\\
			Mà $x\in \mathbb{R}$ nên
			$C=\left\lbrace -2;1\right\rbrace$.
		\end{enumerate}
	}
\end{bt}
\begin{bt}%[Huỳnh Quy]%[0D1B2-1]
	Viết các tập hợp sau bằng phương pháp liệt kê:
	\begin{enumerate}
		\item $A=\left\lbrace  x\in \mathbb{Q}\mid (x^2-2x+1)(x^2-5)\right\rbrace=0$.
		\item $B=\left\lbrace x \in \mathbb{N}\mid 5<x^2<40\right\rbrace$.
		\item $C=\left\lbrace x\in \mathbb{Z}\mid x^2<9\right\rbrace$.
		\item $D=\left\lbrace x\in \mathbb{R}\mid \left|2x+1\right|=5\right\rbrace$.
	\end{enumerate}
	\loigiai{
		\begin{enumerate}
			\item Ta có $x\in A\Leftrightarrow\hoac{&x^2-2x+1=0\\&x^2-5=0}\Leftrightarrow\hoac{&x=1\in\mathbb{Q}\\&x=\pm\sqrt{5}\not\in \mathbb{Q}.}$\\
			Vậy $A=\left\lbrace 1\right\rbrace$.
			\item $B=\left\lbrace 3;4;5;6\right\rbrace$.
			\item $C=\left\lbrace -2;-1;0;1;2\right\rbrace$.
			\item Ta có $\left|2x+1\right|=5\Leftrightarrow \hoac{& 2x+1=5 \\ & 2x+1 =-5} \Leftrightarrow \hoac{&x=2\\&x=-3.}$\\
			Vậy $D=\left\lbrace 2;-3\right\rbrace$.
		\end{enumerate}
	}
\end{bt}
\begin{bt}%[Huỳnh Quy]%[0D1B2-1]
	Viết các tập hợp sau bằng cách chỉ ra tính chất đặc trưng cho các phần tử của tập hợp đó.
	\begin{enumerate}
		\item $A=\left\lbrace 0;4;8;12;16;\ldots ;52\right\rbrace$.
		\item $B=\left\lbrace 3;6;9;12;15;\ldots ;51\right\rbrace$.
		\item $C=\left\lbrace 2;5;8;11;14;\ldots ;62\right\rbrace$.
	\end{enumerate}
	\loigiai{
		\begin{enumerate}
			\item $A=\left\lbrace x\in \mathbb{N}\mid 0\le x\le 52 \text{ và } x\;\vdots \;4\right\rbrace$.
			\item $B=\left\lbrace x\in \mathbb{N}\mid 3\le x\le 51 \text{ và } x\;\vdots \;3\right\rbrace$.
			\item $C=\left\lbrace  x\in \mathbb{N}\mid 2\le x\le 62 \text{ và } (x-2)\;\vdots \;3\right\rbrace$.
		\end{enumerate}
	}
\end{bt}
\begin{bt}%[Huỳnh Quy]%[0D1K2-1]
	Viết các tập hợp sau bằng cách chỉ ra tính chất đặc trưng cho các phần tử của tập hợp đó.
	\begin{enumerate}
		\item $A=\left\lbrace 2;3;5;7;11;13;17\right\rbrace$.
		\item $B=\left\lbrace -2;4;-8;16;-32;64\right\rbrace$.
	\end{enumerate}
	\loigiai{
		\begin{enumerate}
			\item $A=\left\lbrace x\in \mathbb{N}\mid x\le 17 \text{ và }x \text{ là số nguyên tố} \right\rbrace$.
			\item $B=\left\lbrace x=(-2)^n\mid n\in \mathbb{N}, 1\le n\le 6 \right\rbrace$.
		\end{enumerate}
	}
\end{bt}

\begin{bt}%[Huỳnh Quy]%[0D1B2-1]
	Tìm một tính chất đặc trưng xác định các phần tử của mỗi tập hợp sau
	\begin{align*}
		A&=\{1; 2; 3; 4; 5; 6; 7; 8; 9\} \\
		B&=\{0; 7; 14; 21; 28\}
	\end{align*}
	\loigiai{
		\begin{align*}
			A&=\{x\in \mathbb{N^*} \mid x\leq 9\} \\
			B&=\{x\in \mathbb{N} \mid x \;\vdots \;7 \text{ và } x\leq 28\}
		\end{align*}
	}
\end{bt}

\begin{dang}{Tập hợp con, xác định tập hợp con}
	Cho tập hợp $A$ gồm $n$ phần tử.
	\begin{enumEX}{1}
		\item  Khi liệt kê tất cả các tập con của $A$, ta liệt kê đầy đủ theo thứ tự:\\		
		\centerline{ $\varnothing$; tập $1$ phần tử; tập $2$ phần tử; tập $3$ phần tử;...; $A$.}
		\item  Số tập con của $A$ là $2^n$.
		\item  Số tập con gồm $k$ phần tử của $A$ là $\mathrm{C}_n^k$.
	\end{enumEX}
\end{dang}
\subsubsection{Ví dụ minh hoạ}
\begin{vd}%[BG10-2022]%[Đỗ Văn Dự]%[0D1Y2-2]
	Cho tập hợp $S=\{2 ; 3 ; 5\}$. Những tập hợp nào sau đây là tập con của $S$?
	$$S_{1}=\{3\};S_{2}=\{0 ; 2\}; S_{3}=\{3 ; 5\}$$.
	\loigiai{
		Các tập hợp $S_{1}=\{3\}$, $S_{3}=\{3 ; 5\}$ là những tập con của $S$.\\
		Tập $S_{2}=\{0 ; 2\}$ không là tập con của $S$.
	}
\end{vd}

\begin{vd}%[BG10-2022]%[Đỗ Văn Dự]%[0D1B2-2]
	Cho tập hợp $A=\left\{2;3;4\right\}$ và $B=\left\{2;3;4;5;6\right\}$.
	\begin{enumEX}{1}
		\item Xác định tất cả tập con có hai phần tử của $A$.
		\item Xác định tất cả tập con có ít hơn hai phần tử của $A$.
		\item Tập $A$ có tất cả bao nhiêu tập con.
		\item Xác định tất cả các tập $X$ thỏa $A \subset X \subset B$.
	\end{enumEX}
		\loigiai{
	\begin{enumEX}{1}		
		\item 	Các tập hợp $S_1=\{2;3\}$, $S_2=\{2;4\}$, $S_3=\{3;4\}$  là những tập con của $A$.\\
			Tập $S_{2}=\{0 ; 2\}$ không là tập con của $S$.
		\item 	Các tập hợp $\varnothing$, $\{2\}$, $\{3\}$, $\{4\}$  là những tập con ít hơn $2$ phần tử của $A$.\\
		\item 	Tập $A$ có tất cả $8$ tập con.
		\item 	 Tất cả các tập $X$ thỏa $A \subset X \subset B$ là $\{2;3;4\}$, $\{2;3;4,5\}$, $\{2;3;4,5,6\}$.	
	\end{enumEX}
		}	
\end{vd}
\subsubsection{Bài tập tự luận}
\begin{bt}%[Huỳnh Quy]%[0D1B2-2]
	Tìm tất cả các tập con của tập $A=\{a,1,2\}$.
	\loigiai{Tập $A$ có $2^3=8$ tập con.
		\begin{itemize}
			\item 0 phần tử: $ \varnothing $.
			\item 1 phần tử: $\{a\}$, $\{1\}$, $\{2\}$.
			\item 2 phần tử: $\{a, 1\}$, $\{a,2\}$, $\{1,2\}$.
			\item 3 phần tử: $\{a,1,2\}$.
	\end{itemize}}
\end{bt}
\begin{bt}%[Huỳnh Quy]%[0D1B2-2]
	Tìm tất cả các tập con có 2 phần tử của tập $A=\{1,2,3,4,5,6\}$.
	\loigiai{$\{1,2\}$,$\{1,3\}$, $\{1,4\}$, $\{1,5\}$, $\{1,6\}$, $\{2,3\}$, $\{2,4\}$, $\{2,5\}$, $\{2,6\}$, $\{3,4\}$, $\{3,5\}$, $\{3,6\}$, $\{4,5\}$, $\{4,6\}$, $\{5,6\}$.}
\end{bt}
\begin{bt}%[Huỳnh Quy]%[0D1B2-2]
	Xác định tập hợp $X$ biết $\{1,2\} \subset X \subset \{1,2,5\}$.
	\loigiai{Ta có
		\begin{itemize}
			\item  Vì $\{1,2\} \subset X$ nên tập hợp $X$ có chứa các phần tử $1,2$.
			\item  Vì $X \subset \{1,2,5\}$ nên các phần tử của tập hợp $X$ có thể là $1,2,5$.
		\end{itemize}
		Khi đó tập hợp $X$ có thể là $\{1,2\}, \{1,2,5\}$.}
\end{bt}
\begin{bt}%[Huỳnh Quy]%[0D1B2-2]
	Xác định tập hợp $X$ biết $\{a,1\} \subset X \subset \{a,b,1,2\}$.
	\loigiai{Ta có
		\begin{itemize}
			\item  Vì $\{a,1\} \subset X$ nên tập hợp $X$ có chứa 2 phần tử là $a,1$.
			\item  Vì $X \subset \{a,b,1,2\}$ nên các phần tử của tập hợp $X$ có thể là $a,b,1,2$.
		\end{itemize}
		Suy ra, tập hợp $X$ có 2 phần tử, 3 phần tử hoặc 4 phần tử.\\
		Khi đó, tập hợp $X$ có thể là $\{a,1\}, \{a,1,2\}, \{a,b,1\}, \{a,b,2\}, \{a,b,1,2\}$.\\
		\underline{\textbf{Cách khác:}} $X=\{a;1\} \cup X'$ với $X' \subset \{b;2\}$. \\
		Vì có $4$ tập hợp $X'$ nên có $4$ tập hợp $X$ thỏa yêu cầu bài toán.}
\end{bt}
\begin{bt}%[Huỳnh Quy]%[0D1K2-2]
	Cho tập hợp $A=\{1;2;3;4;5;6\}$. Tìm tất cả các tập con có $3$ phần tử của tập hợp $A$ sao cho tổng các phần tử này là một số lẻ.
	\loigiai{
		Để tổng của ba số nguyên là một số lẻ thì trong ba số chỉ có một số lẻ hoặc cả ba số đều lẻ. Nói cách khác tập con này của $A$ phải có một số lẻ hoặc ba số lẻ.\\
		Chỉ có một tập con gồm ba số lẻ của $A$ là $\{1;3;5\}$. Các tập con gồm ba số của $A$ trong đó có một số lẻ là: \\
		$\{1;2;4\}$; $\{1;2;6\}$; $\{1;4;6\}$;$\{3;2;4\}$; $\{3;2;6\}$; $\{3;4;6\}$; $\{5;2;4\}$; $\{5;2;6\}$; $\{5;4;6\}$.\\
		\textit{\underline{Nhận xét:} Tổng các số nguyên là một số lẻ khi số số lẻ là số lẻ.}
	}
\end{bt}
\begin{bt}%[Huỳnh Quy]%[0D1K2-2]
	Cho $A=\{n\in\mathbb{N}\mid n \text{ là ước của }2\}$; $B=\{x\in\mathbb{R}\mid (x^2-1)(x-2)(x-4)=0 \}$. Tìm tất cả các tập hợp $X$ sao cho $A\subset X\subset B$.
	\loigiai{ 
		Liệt kê các phần tử của tập hợp $A$ và $B$ ta được : \\
		$A=\{1;2\}$; $B=\{-1;1;2;4\}$.\\
		Muốn tìm tập $X$ thỏa điều kiện $A\subset X\subset B$ đầu tiên ta lấy $X=A$, sau đó ghép thêm các phần tử thuộc $B$ mà không thuộc $A$. Với cách thực hiện như trên, ta có các tập hợp $X$ thỏa mãn yêu cầu bài toán là: $X=A=\{1;2 \}$, rồi ghép thêm vào một phần tử ta được: $\{-1;1;2 \}$;$\{4;1;2 \}$\\
		Ghép thêm vào $A$ hai trong bốn phần tử còn lại của $B$ ta được : $X=B=\{-1;1;2;4\}$
	}
\end{bt}
\begin{dang}{Các phép toán trên tập hợp}
\end{dang}
\subsubsection{Ví dụ minh hoạ}
\begin{vd}%[BG10-2022]%[Đỗ Văn Dự]%[0D1B2-2]
	Cho hai tập hợp:
	$C=\{n \in \mathbb{N} \mid n$ là bội chung của 2 và $3 ; n<30\}$;
	$D=\{n \in \mathbb{N} \mid n$ là bội của $6 ; n<30\}$.
	Chứng minh rằng $C=D$.
	\loigiai{
		Ta có: $C=\{0 ; 6 ; 12 ; 18 ; 24\}$.\\
		$D=\{0 ; 6 ; 12 ; 18 ; 24\}$.\\
		Vậy $C=D$.
	}
\end{vd}

\begin{vd}%[BG10-2022]%[Đỗ Văn Dự]%[0D1Y4-1]
	Viết các tập hợp sau dưới dạng các khoảng, đoạn, nửa khoảng trong $\mathbb{R}$ rồi biểu diễn trên trục số: $C=\{x \in \mathbb{R} \mid 2 \leq x \leq 7\}$; $D=\{x \in \mathbb{R} \mid x<2\}$.
	\loigiai{
		\immini{$C=[2;7]$}{\begin{tikzpicture}[scale=1,>=stealth, font=\footnotesize, line join=round, line cap=round]
				\draw [-stealth] (0,0)--(9,0);
				\path (2,0) node{$[$} (2,-0.1)node[below]{$2$}
				(7,0) node{$]$} (7,-0.1)node[below]{$7$};
				\foreach \x in{0,0.1,...,2} \draw (\x,0)--++(45:.07) (\x,0)--++(-135:.07);
				\foreach \x in{7.1,7.2,...,8.8} \draw (\x,0)--++(45:.07) (\x,0)--++(-135:.07);
		\end{tikzpicture}}
		\immini{$C=(-\infty;2)$}{\begin{tikzpicture}[scale=1,>=stealth, font=\footnotesize, line join=round, line cap=round]
				\draw [-stealth] (0,0)--(9,0);
				\path (2,0) node{$)$} (2,-0.1)node[below]{$2$};
				\foreach \x in{2,2.1,...,8.9} \draw (\x,0)--++(45:.07) (\x,0)--++(-135:.07);
		\end{tikzpicture}}	
		
	}
\end{vd}

\begin{vd}%[BG10-2022]%[Đỗ Văn Dự]%[0D1B4-1]
	\begin{enumerate}
		\item Cho hai tập hợp $C=\{4 ; 7 ; 27\}$ và $D=\{2 ; 4 ; 9 ; 27 ; 36\}$. Hãy xác định tập hợp $C \cap D$.
		\item Cho hai tập hợp $E=[1 ;+\infty)$ và $F=(-\infty ; 3]$. Hãy xác định tập hợp $E \cap F$.
	\end{enumerate}
	\loigiai{
		\begin{enumEX}{1}
			\item Giao của hai tập hợp $C$ và $D$ là $C \cap D=\{4 ; 27\}$.
			\item Giao của hai tập hợp $E$ và $F$ là $E \cap F=[1 ; 3]$.\\
			\begin{tikzpicture}[scale=1,>=stealth, font=\footnotesize, line join=round, line cap=round]
				\draw [-stealth] (-1,0)--(5,0);
				\path (1,0) node{$[$} (1,-0.1)node[below]{$1$} (-1.5,0)node[left]{$[1 ;+\infty)$};
				\foreach \x in{-0.9,-0.8,...,0.9} \draw (\x,0)--++(45:.08) (\x,0)--++(-135:.08);
			\end{tikzpicture}\\
			\begin{tikzpicture}[scale=1,>=stealth, font=\footnotesize, line join=round, line cap=round]
				\draw [-stealth] (-1,0)--(5,0);
				\path (3,0) node{$]$} (3,-0.1)node[below]{$3$} (-1.5,0)node[left]{$(-\infty ; 3]$};
				\foreach \x in{3.1,3.2,...,4.9} \draw (\x,0)--++(-45:.08) (\x,0)--++(135:.08);
			\end{tikzpicture}\\
			\begin{tikzpicture}[scale=1,>=stealth, font=\footnotesize, line join=round, line cap=round]
				\draw [-stealth] (-1,0)--(5,0);
				\path (3,0) node{$]$} (3,-0.1)node[below]{$3$} 
				(1,0) node{$[$} (1,-0.1)node[below]{$1$} (-1.5,0)node[left]{$[1 ;+\infty)\cap(-\infty ; 3]=[1;3]$}
				;
				\foreach \x in{3.1,3.2,...,4.9} \draw (\x,0)--++(-45:.08) (\x,0)--++(135:.08);
				\foreach \x in{-0.9,-0.8,...,0.9} \draw (\x,0)--++(45:.08) (\x,0)--++(-135:.08);
			\end{tikzpicture}
		\end{enumEX}
	}
\end{vd}
\begin{vd}%[BG10-2022]%[Đỗ Văn Dự]%[0D1B3-1]
	Cho hai tập hợp: $C=\{2 ; 3 ; 4 ; 7\}$; $D=\{-1 ; 2 ; 3 ; 4 ; 6\}$. Hãy xác định tập hợp $C \cup D$.
	\loigiai{
		Hợp của hai tập hợp $C$ và $D$ là $C \cup D=\{-1 ; 2 ; 3 ; 4 ; 6 ; 7\}$.
	}
\end{vd}

\begin{vd}%[BG10-2022]%[Đỗ Văn Dự]%[0D1B3-2]
	Cho các tập hợp: $D=\{-2 ; 3 ; 5 ; 6\}$; $E=\{x \mid x$ là số nguyên tố nhỏ hơn 10$\}$; $X=\{x \mid x$ là số nguyên dương nhỏ hơn 10$\}$.
	\begin{enumEX}{1}
		\item Tìm $D \backslash E$ và $E \backslash D$.
		\item $E$ có là tập con của $X$ không? Hãy tìm phần bù của $E$ trong $X$ (nếu có).
	\end{enumEX}
	\loigiai{
		\begin{enumEX}{1}
			\item Ta có: $E=\{2 ; 3 ; 5 ; 7\}$.\\
			Do đó, $D \backslash E=\{-2 ; 6\}$; $E \backslash D=\{2 ; 7\}$.
			\item Ta có: $X=\{1 ; 2 ; 3 ; 4 ; 5 ; 6 ; 7 ; 8 ; 9\}$. Vậy $E$ là tập con của $X$.\\
			Phần bù của $E$ trong $X$ là $X \backslash E=C_{X} E=\{1 ; 4 ; 6 ; 8 ; 9\}$.
		\end{enumEX}	
	}
\end{vd}

\begin{vd}%[BG10-2022]%[Đỗ Văn Dự]%[0D1B3-1]
	Cho hai tập hợp $A=\left\{0;1;2;3;4\right\}$ và $B=\left\{2;3;4;5;6\right\}$.
	\begin{enumEX}{1}
		\item Tìm các tập hợp $A\cup B, A\cap B, A\backslash B, B\backslash A$.
		\item  Tìm các tập $\left(A\backslash B\right)\cup \left(B\backslash A\right), \left(A\backslash B\right)\cap \left(B\backslash A\right)$.
	\end{enumEX}
	\loigiai{
	\begin{enumEX}{1}
			\item Ta có $A\backslash B=\left\{0;1\right\}$, $B\backslash A=\left\{5;6\right\}$, $A\cup B=\left\{0;1;2;3;4;5;6\right\}$, $A\cap B=\left\{2;3;4\right\}$.
			\item Ta có $\left(A\backslash B\right)\cup \left(B\backslash A\right)=\left\{0;1;5;6\right\}$, $\left(A\backslash B\right)\cap \left(B\backslash A\right)=\varnothing $.
	\end{enumEX}
	}
\end{vd}
\subsubsection{Bài tập tự luận}
\begin{bt}%[Huỳnh Quy]%[0D1B3-2]
	Cho hai tập hợp $A=\{ 1;2;3;4;5\}$ và $B=\{ 0;2;4\}$. Xác định $A\cap B$, $A\cup B$.
	\loigiai{
		Ta có $A\cap B=\{2;4\}$ và $A\cup B=\{0;1;2;3;4;5\}$.
	}
\end{bt}
\begin{bt}%[Huỳnh Quy]%[0D1B3-1]
	Cho hai tập hợp $A=\{1;2;3;5;7\}$ và  $B=\{n\in \mathbb{N} |\, n \text{ là ước số của } 12\}$. Tìm $A\cap B$  và $A\cup B$.
	\loigiai{
		Ta có: $B=\{1;2;3;4;6;12\}$.
		Vậy: $A\cap B=\{ 1;2;3\}$ và $A\cup B=\{ 1;2;3;4;5;6;7;12\}$.
	}
\end{bt}
\begin{bt}%[Huỳnh Quy]%[0D1B3-2]
	Cho hai tập hợp $A$ và $B$. Tìm $A\cap B, A\cup B$ biết
	\begin{enumerate}
		\item $A=\{x\mid x\ \text{là ước nguyên dương của 12} \}$ và 	$B=\{x\mid x\ \text{là ước nguyên dương của 18} \}$.
		\item $A=\{x\mid x\ \text{là ước nguyên dương của 27}\}$ và $B=\{x\mid x\ \text{là ước nguyên dương của 15} \}$.
	\end{enumerate}
	\loigiai{
		\begin{enumerate}
			\item $A=\{1;2;4;6;12\}$, $B=\{1;2;3;6;9;18\}$ $\Rightarrow \begin{cases} A\cap B=\{1;2;6\}\\
				A\cup B=\{1;2;3;4;6;9;12;18\} \end{cases}$
			\item $A=\{1;3;9;27\}$, $B=\{1;3;5;15\}$$\Rightarrow \begin{cases} A\cap B=\{1;3\}\\A\cup B=\{1;3;5;9;15;27\}\end{cases}$
		\end{enumerate}
	}
\end{bt}
\begin{bt}%[Huỳnh Quy]%[0D1B3-1]
	Cho $A$ là tập hợp học sinh lớp $12$ của trường Buôn Ma Thuột và $B$ là tập hợp học sinh của trường Buôn Ma Thuột dự kiến sẽ lựa chọn thi khối $A$ vào các trường đại học. Hãy mô tả các học sinh thuộc tập hợp sau
	\begin{enumEX}{2}
		\item $A\cap B$.
		\item $A\cup B$.
	\end{enumEX}
	\loigiai{
		\begin{enumerate}
			\item $A\cap B$ là tập hợp các học sinh lớp 12 thi khối $A$ của trường Buôn Ma Thuột.
			\item $A\cup B$ là tập hợp các học sinh hoặc lớp 12 hoặc học sinh chọn thi khối A của trường Buôn Ma Thuột. 
		\end{enumerate}
	}
\end{bt}
\begin{bt}%[Huỳnh Quy]%[0D1K3-1]
	Cho tập hợp $B=\{ x\in \mathbb{Z}|\, -4< x \le 4 \}$ và $C=\{ x\in \mathbb{Z}|\, x\le a\}$.
	Tìm số nguyên $a$ để tập hợp $B\cap C=\varnothing $.
	\loigiai{
		Ta có $B=\{-3;-2;-1;0;1;2;3;4\}$, $C=\{\ldots,a-1,a\}$.\\
		Để $B\cap C=\varnothing $ thì  $a\le -4$.
	}
\end{bt}
\begin{bt}%[Huỳnh Quy]%[0D1B3-1]
	Xác định tập hợp $A\cap B$ biết 
	$$A=\{x\in\mathbb{N}|\, x \text { là bội của }3 \}, \,\, B=\{x\in\mathbb{N}|\, x\text { là bội của }7\}.$$
	\loigiai{
		Ta có $A\cap B=\{x\in\mathbb{N}|\, x\text { là bội của }3 \text{ và bội của }7 \}= \{x\in\mathbb{N}|\, x\text {  là bội của  }21\} $.
	}
\end{bt}
\begin{bt}%[Huỳnh Quy]%[0D1B3-2]
	Cho $A$ là tập hợp các số tự nhiên chẵn không lớn hơn $10$, 
	$B=\left\{n\in \mathbb{N}|n\le 6\right\}$ và 
	$C=\left\{n\in \mathbb{N}|4\le n\le 10\right\}$. 
	Hãy tìm $A\cap (B\cup C)$.
	\loigiai{
		Ta có $A=\{0;2;4;6;8;10\}$; $B=\{0;1;2;3;4;5;6\}$ và $C=\{4;5;6;7;8;9;10\}$\\
		$B\cup C=\{0; 1; 2; 3; 4; 5; 6; 7; 8; 9; 10\}$ nên $A\cap (B\cup C)=\{0;2; 4; 6; 8; 10\}$.\\
		\underline{\textbf{Cách khác:}} Vì $B \cup C = \{n \in \mathbb{N} | n \ge 10\}$ nên $A \subset (B \cup C)$.\\
		Do đó $A \cap (B \cup C) = A = \{0;2;4;6;8;10\}$.
	}
\end{bt}
\begin{bt}%[Huỳnh Quy]%[0D1B3-1]
	Cho các tập hợp $A=\{x\in\mathbb{N}\mid x<8\}$ và $B=\{x\in\mathbb{Z}\mid  -3\leq x\leq 5\}$. Tìm $A\cap B$; $A\cup B$.
	\loigiai{
		Ta có $A=\{0;1;2;3;4;5;6;7\}$; $B=\{-3;-2;-1;0;1;2;3;4;5\}$.\\
		Vậy $A\cap B=\{0;1;2;3;4;5\}$ và $A\cup B=\{-3;-2;-1;0;1;2;3;4;5;6;7\}$.
	}
\end{bt}
\begin{bt}%[Huỳnh Quy]%[0D1K3-1]
	Cho các tập hợp $A=\{x \in \mathbb{Z}\big| |x-1|<4\}$, $B=\{x \in \mathbb{Z}\big| |x-1|>2\}$. Tìm $A \cap B$.
	\loigiai{
		Ta có $|x-1|<4 \Leftrightarrow -4<x-1<4 \Leftrightarrow -3<x<5$, $A=\{-2;-1;0;1;2;3;4\}$. \\
		Lại có $|x-1|>2 \Leftrightarrow x<-1\vee x>3$, $B=\{\ldots;-3;-2;4;5;6;\ldots\}$ nên $A \cap B=\{-2;4\}$.
	}
\end{bt}
\begin{bt}%[Huỳnh Quy]%[0D1G3-1]
	Cho các tập hợp $A=\{x\in\mathbb{Z}\,|\, 2m-1<x<2m+3\}$, $B=\{x\in\mathbb{Z}\,\big|\, |x|<2\}$. Tìm $m$ để $A\cap B=\varnothing$.
	\loigiai{
		Ta có $B=\{x\in\mathbb{Z}\,|\, -2<x<2\}=\{-1;0;1\}$ và $A=\{2m,\ldots,2m+2\}$.\\
		$A\cap B=\varnothing \Leftrightarrow \hoac{& 2m+2 \le -2 \\ & 2m \ge 2} \Leftrightarrow \hoac{& m \le -2 \\ & m \ge 1.}$
	}
\end{bt}
\begin{bt}%[Huỳnh Quy]%[0D1B4-1]
	Cho $A=\left[-2;4\right],B=\left(2;+\infty\right),C=(-\infty;3)$. Xác định các tập hợp sau đây và biểu diễn chúng trên trục số.
	\begin{enumerate}
		\item $A\cap B, B\cap C$.
		\item $\mathbb{R}\cap A,\mathbb{R}\cap B$.
	\end{enumerate}
	\loigiai{
		\begin{tikzpicture}[scale=1]
			\draw[->,thick](-4,0) --(6,0);
		\end{tikzpicture}\\
		%
		\begin{tikzpicture}[scale=1]
			\draw[->,thick](-4,0) --(6,0) node at (-2,0.7){$-2$} node at (4,0.7){$4$} node at (-2,0) {$\Big [$} node at (4,0) {$\Big ]$}node at (6.7,0){$A$};
			\foreach \b in {-3.8,-3.6,...,-2.2} {\draw[ thick](\b-0.1,-0.2)--(\b+0.1,0.2);}
			\foreach \b in {4.2,4.4,...,5.8} {\draw[ thick](\b-0.1,-0.2)--(\b+0.1,0.2);}
		\end{tikzpicture}\\
		%
		\begin{tikzpicture}[scale=1]
			\draw[->,thick](-4,0) --(6,0) node at (2,0.7){$2$} node at (2,0) {$\Big ($}node at (6.7,0){$B$};
			\foreach \b in {-3.8,-3.6,...,1.8} {\draw[ thick](\b-0.1,-0.2)--(\b+0.1,0.2);}
		\end{tikzpicture}\\
		%
		\begin{tikzpicture}[scale=1]
			\draw[->,thick](-4,0) --(6,0) node at (3,0.7){$3$} node at (3,0) {$\Big )$}node at (6.7,0){$C$};
			\foreach \b in {3.2,3.4,...,5.8} {\draw[ thick](\b-0.1,-0.2)--(\b+0.1,0.2);}
		\end{tikzpicture}\\
		%
		\begin{enumerate}
			\item $A\cap B=\left(2;4\right], B\cap C=\left(2;3\right)$.
			\item $\mathbb{R}\cap A=\left[-2;4 \right],\mathbb{R}\cap B=\left(2;+\infty \right)$.
		\end{enumerate}
	}
\end{bt}
\begin{bt}%[Huỳnh Quy]%[0D1B4-2]
	Cho hai tập hợp $A=\lbrace x\in \mathbb{R}\vert  x \leq 2 \rbrace$, $B=\lbrace x\in \mathbb{R}\vert -2< x \rbrace$. Tìm $A\setminus B, B\setminus A$.
	\loigiai{
		\begin{tikzpicture}[scale=1]
			\draw[->,thick](-4,0) --(6,0) node at (2,0.7){$2$} node at (2,0) {$\Big ]$}node at (6.7,0){$A$};
			\foreach \b in {2.2,2.4,...,5.8} {\draw[ thick](\b-0.1,-0.2)--(\b+0.1,0.2);}
		\end{tikzpicture}\\
		%
		\begin{tikzpicture}[scale=1]
			\draw[->,thick](-4,0) --(6,0) node at (-2,0.7){$-2$} node at (-2,0) {$\Big ($} node at (6.7,0){$B$};
			\foreach \b in {-3.8,-3.6,...,-2.2} {\draw[ thick](\b-0.1,-0.2)--(\b+0.1,0.2);}
		\end{tikzpicture}\\
		$\Rightarrow A\setminus B=\left(-\infty;-2 \right], B\setminus A=\left(2;+\infty \right)$.}
\end{bt}
\begin{bt}%[Huỳnh Quy]%[0D1B4-2] 
	Cho $A=\left[-2;4\right],B=\left(2;+\infty\right),C=(-\infty;3)$. Xác định các tập hợp sau đây và biểu diễn chúng trên trục số.
	\begin{enumerate}
		\item $A\setminus B, B\setminus A$.
		\item $\mathbb{R}\setminus A,\mathbb{R}\setminus B, \mathbb{R}\setminus  C$.
	\end{enumerate}
	\loigiai{
		\begin{tikzpicture}[scale=1]
			\draw[->,thick](-4,0) --(6,0);
		\end{tikzpicture}\\
		%
		\begin{tikzpicture}[scale=1]
			\draw[->,thick](-4,0) --(6,0) node at (-2,0.7){$-2$} node at (4,0.7){$4$} node at (-2,0) {$\Big [$} node at (4,0) {$\Big ]$}node at (6.7,0){$A$};
			\foreach \b in {-3.8,-3.6,...,-2.2} {\draw[ thick](\b-0.1,-0.2)--(\b+0.1,0.2);}
			\foreach \b in {4.2,4.4,...,5.8} {\draw[ thick](\b-0.1,-0.2)--(\b+0.1,0.2);}
		\end{tikzpicture}\\
		%
		\begin{tikzpicture}[scale=1]
			\draw[->,thick](-4,0) --(6,0) node at (2,0.7){$2$} node at (2,0) {$\Big ($}node at (6.7,0){$B$};
			\foreach \b in {-3.8,-3.6,...,1.8} {\draw[ thick](\b-0.1,-0.2)--(\b+0.1,0.2);}
		\end{tikzpicture}\\
		%
		\begin{tikzpicture}[scale=1]
			\draw[->,thick](-4,0) --(6,0) node at (3,0.7){$3$} node at (3,0) {$\Big )$}node at (6.7,0){$C$};
			\foreach \b in {3.2,3.4,...,5.8} {\draw[ thick](\b-0.1,-0.2)--(\b+0.1,0.2);}
		\end{tikzpicture}
		%
		\begin{enumerate}
			\item $A\setminus B=\left[-2;2\right], B\setminus A=\left(4;+\infty\right)$.
			\item $\mathbb{R}\setminus A=\left(\infty;-2\right)\cup\left(4;+\infty \right),\mathbb{R}\setminus B=\left(-\infty;2\right], \mathbb{R}\setminus  C=\left[3;+\infty \right)$.
		\end{enumerate}
	}
\end{bt}
\begin{dang}{Ứng dụng thực tế các phép toán tập hợp}
\end{dang}
\subsubsection{Ví dụ minh hoạ}
\begin{vd}%[BG10-2022]%[Đỗ Văn Dự]%[0D1Y3-3]
	Cho $A$ là tập hợp các học sinh giỏi Toán của trường THPT X và $B$ là tập hợp học sinh giỏi Văn của trường này. Hãy mô tả các học sinh thuộc tập hợp sau\begin{enumEX}{3} 
			\item $A\cup B$.
			\item $A\cap B$.
			\item $A\setminus B$.
			\item $B\setminus A$.
			\item $\left(A\cup B\right)\setminus \left(A\cap B\right)$.
	\end{enumEX}
	\loigiai{
		\immini{
			\begin{enumerate}
				\item $A\cup B$ là tập hợp các học sinh giỏi Toán hoặc giỏi Văn của trường.
				\item $A\cap B$ là tập hợp các học sinh giỏi cả hai môn Toán và Văn của trường.
				\item $A\setminus B$ là tập hợp các học sinh chỉ giỏi Toán, không giỏi Văn.
				\item $B\setminus A$ là tập hợp các học sinh chỉ giỏi Văn, không giỏi Toán.		
				\item $\left(A\cup B\right)\setminus \left(A\cap B\right)$ là tập hợp các học sinh chỉ giỏi Toán hoặc giỏi Văn của trường.	
			\end{enumerate}
		}
		{\begin{tikzpicture}[scale=1, line join=round, line cap=round]	
				\coordinate (A) at (-1.2,0);
				\coordinate (B) at (0.8,0);
				\begin{scope}
					\clip[rotate=20] (A) ellipse (2cm and 1cm); %cat theo elip A
					\fill[rotate=30,pattern=north west lines] (B) ellipse (2cm and 1cm); %to elip B nhung bi cat mat theo phan giao voi elip A
				\end{scope}
				\draw[rotate=20] (A) ellipse (2cm and 1cm) node[fill=white,above]{$A\setminus B$}; %ve lai elip A
				\draw[rotate=30] (B) ellipse (2cm and 1cm) node[fill=white,above,right]{$B\setminus A$}; %ve lai elip B
				\path (-0.2,0.2) node[below]{$A\cap B$};
		\end{tikzpicture}}
	}
\end{vd}
\begin{vd}%[BG10-2022]%[Đỗ Văn Dự]%[0D1B3-3]
	Trong kì thi học sinh giỏi cấp trường, lớp 10C1 có $45$ học sinh trong đó có  $17$ bạn đạt học sinh giỏi Văn, $25$ bạn đạt học sinh giỏi Toán và $13$ bạn học sinh không đạt học sinh giỏi. Tìm số học sinh giỏi cả Văn và Toán của lớp 10C1.
	\loigiai{
		\immini{
			\begin{itemize}			
				\item Gọi $A$, $B$ theo thứ tự là tập hợp các học sinh giỏi Văn và giỏi Toán của lớp. 
				Theo đề ta có $n(A)=17$, $n(B)=25$, $n(A \cup B)= 45-13=32$.
				\item Số học sinh giỏi cả Văn và Toán là $$n(A \cap B )=n(A) + n(B) - n(A \cup B)=25+17-32=10.$$
			\end{itemize}
		}
		{	\begin{tikzpicture}[scale=.8]
				\def\radius{2cm}	
				\coordinate (ceni);
				\coordinate[xshift=.9*\radius] (cenii);
				
				\draw (ceni) circle (.9*\radius);
				\draw (cenii) circle (\radius);
				\draw  ([xshift=-25pt,yshift=15pt]current bounding box.north west) 
				rectangle ([xshift=25pt,yshift=-15]current bounding box.south east);
				
				\node[xshift=-.3*\radius] at (ceni) {$n(B)=15$};
				\node[xshift=.3*\radius] at (cenii) {$n(A)=35$};
				\node[xshift=.4*\radius] at (ceni) {$5$};
	\end{tikzpicture}}}
\end{vd}
\begin{vd}%[BG10-2022]%[Đỗ Văn Dự]%[0D1B3-3]
	Một lớp học có $ 50 $ học sinh trong đó có $ 30 $ em biết chơi bóng chuyền, $25$ em biết chơi bóng đá, $ 10 $ em biết chơi cả bóng đá và bóng chuyền. Hỏi có bao nhiêu em không biết chơi môn nào trong hai môn ở trên?
	\loigiai{
		Gọi tập $A$ là tập hợp các học sinh biết chơi bóng chuyền.
		\\Tập $B$ là tập hợp các học sinh biết chơi bóng đá.
		\\Khi đó số học sinh biết chơi ít nhất một trong hai môn bóng chuyền hoặc bóng đá là 
		$$n(A \cup B)=n(A)+n(B)-n(A \cap B)=30+25-10=45.$$
		Vậy số học sinh không biết chơi môn nào là $50-45=5$. 		
	}	
\end{vd}
\begin{vd}%[BG10-2022]%[Đỗ Văn Dự]%[0D1B3-3]
	Trong số $45$ cán bộ được triệu tập để chuẩn bị công tác cho một cuộc hội nghị quốc tế có $25$ cán bộ phiên dịch tiếng Anh, $15$ cán bộ phiên dịch tiếng Pháp, trong đó có $10$ cán bộ vừa phiên dịch được tiếng Anh, vừa phiên dịch được tiếng Pháp. Hỏi
	\begin{enumerate}
		\item Nhóm có bao nhiêu cán bộ được cấp thẻ đỏ, biết rằng muốn được cấp thẻ đỏ cán bộ đó phải phiên dịch được tiếng Anh hoặc phiên dịch được tiếng Pháp.
		\item Nhóm có bao nhiêu cán bộ không phiên dịch được tiếng Anh và không phiên dịch được tiếng Pháp.
	\end{enumerate}
	\loigiai{
		Gọi $A$, $B$ theo thứ tự là tập hợp các cán bộ phiên dịch tiếng Anh và tập hợp các các bộ phiên dịch tiếng Pháp. 
		Theo đề ta có $n(A)=25$, $n(B)=15$, $n(A \cap B)= 10$.
		\begin{enumerate}				
			\item Tập hợp các cán bộ được cấp thẻ đỏ là $A\cup B$.\\
			Vậy số cán bộ được cấp thẻ đỏ là $n(A \cup B)=n(A)+n(B)-n(A \cap B)=25+15-10=30$.
			\item Tập hợp các cán bộ của nhóm không phiên dịch được tiếng Anh và tiếng Pháp chính là số cán bộ không được cấp thẻ đỏ.\\
			Vậy số cán bộ đó là $45-30=15$.
	\end{enumerate}}
\end{vd}
\begin{vd}%[BG10-2022]%[Đỗ Văn Dự]%[0D1B3-3]
	Lớp $10A$ có $15$ bạn thích môn Văn, $20$ bạn thích môn Toán. Trong số các bạn thích văn hoặc toán có $8$ bạn thích cả $2$ môn. Trong lớp vẫn còn $10$ bạn không thích môn nào trong $2$ môn Văn và Toán. Hỏi lớp $10A$ có bao nhiêu học sinh?
	\loigiai{Ta sử dụng sơ đồ Ven.		
		\begin{center}
			\begin{tikzpicture}
				\draw[] (0.7,0) circle ( 2.5cm);
				\draw[] (0,0) circle ( 1.5cm);
				\draw (1.5,0) circle ( 1.5 cm);
				\draw (-0.5,0) node {$7$};
				\draw (2,0) node {$12$};
				\draw (0.5,0) node {$8$};
				\draw (0.5,1.8) node {$10$};
			\end{tikzpicture}
		\end{center}
		\begin{itemize}
			\item Hình tròn lớn ngoài cùng thể hiện số học sinh cả lớp.\\
			Như vậy, ta có:\\
			\item	Số bạn chỉ thích Văn là $ 15-8=7$(bạn).\\
			\item	Số bạn chỉ thích Toán là $20-8=12$(bạn).\\
			\item	Số học sinh cả lớp là tổng các phần không giao nhau: $7+8+12+10=37$.
		\end{itemize}
	}
\end{vd}

\begin{vd}%[BG10-2022]%[Đỗ Văn Dự]%[0D1B3-3]
	Một lớp có $40$ học sinh, mỗi học sinh đều đăng ký chơi ít nhất $1$ trong $2$ môn thể thao là bóng đá hoặc cầu lông. Có $30$ học sinh có đăng ký môn bóng đá, $25$ học sinh có đăng ký môn cầu lông. Hỏi có bao nhiêu em đăng ký cả $2$ môn.
	\loigiai{
		Gọi $A$ là tập hợp các học sinh đăng kí chơi bóng đá, $B$ là tập học sinh đăng kí chơi cầu lông thì $A\cap B$ là tập hợp các học sinh đăng kí chơi cả hai môn.\\	
		Vậy số học sinh đăng kí chơi cả hai môn là 
		$n(A \cap B)=n(A)+n(B)-n(A \cup B)=30+25-40=15$. }
\end{vd}

\begin{vd}%[BG10-2022]%[Đỗ Văn Dự]%[0D1B3-3]
	Ở xứ sở của thần Thoại ngoài các vị thần thì còn có các sinh vật gồm $27$ con người, $311$ con yêu quái một mắt, $205$ con yêu quái tóc rắn và yêu quái vừa một mắt vừa tóc rắn. Tìm số yêu quái vừa một mắt vừa tóc rắn biết có tổng số sinh vật là 500 con.
	\loigiai{
		\begin{itemize}
			\item Số sinh vật không phải con người là $500-27=473$ (con).
			\item Gọi $A$, $B$ lần lượt là tập hợp yêu quái một mắt và yêu quái tóc rắn. Khi đó $n(A)=311$,  $n(B)=205$, $n(A \cup B)=473$.
			\item Vậy số yêu quái vừa một mắt vừa tóc rắn là $|A \cap B| =311+205-473=43$.
		\end{itemize}
	}
\end{vd}

\begin{vd}%[BG10-2022]%[Đỗ Văn Dự]%[0D1B3-3]
	Mỗi học sinh của lớp $10A$ đều chơi bóng đá hoặc bóng chuyền. Biết rằng có $25$ bạn chơi bóng đá, $20$ bạn chơi bóng chuyền và $10$ bạn chơi cả $2$ môn thể thao. Hỏi lớp $10A$ có bao nhiêu học sinh.
	\loigiai{
		Gọi $A$ là tập hợp các học sinh chơi bóng đá, $B$ là tập các học sinh chơi bóng chuyền. Do đó $A\cap B$ là tập các học sinh chơi cả hai môn.\\
		Theo đề $n(A)=25$, $n(B)=20$, $|A \cap B| =10$.\\
		Vậy số học sinh cả lớp là $|A \cup B| =n(A)+n(B)-n(A \cap B)=25+20-10=35$.}
\end{vd}

\begin{vd}%[BG10-2022]%[Đỗ Văn Dự]%[0D1B3-3]
	Lớp 10A có $45$ học sinh, có $15$ học sinh giỏi và $20$ học sinh xếp hạnh kiểm tốt, trong đó có $10$ bạn vừa học giỏi vừa xếp hạnh kiểm tốt. Các học sinh được học sinh giỏi hoặc hạnh kiểm tốt đều được khen thưởng. Số học sinh được khen thưởng của lớp 10A là là bao nhiêu?
	\loigiai{
		Gọi $A$ là tập hợp các học sinh giỏi, 
		$B$ là tập hợp các học sinh xếp hạnh kiểm tốt.\\
		Khi đó số học sinh được khen thưởng là $n(A \cup B)$.\\
		Vậy số học sinh được khen thưởng là 
		$n(A \cup B)=n(A)+n(B)-n(A \cap B)=15+20-10=25.$		
	}
\end{vd}

\begin{vd}%BT6%[Nguyễn Thành Tuấn]%[0D1K3-3]
	Kết quả thi học kì một của một trường THPT có $48$ thí sinh giỏi môn Toán, $37$ thí sinh giỏi môn Vật Lí,$42$ thí sinh giỏi môn Văn. Biết rằng có $75$ thí sinh giỏi môn Toán hoặc môn Vật lí, $76$ thí sinh giỏi  môn Toán hoặc môn Văn, $66$ thí sinh giỏi môn Vật lí hoặc môn Văn và có $4$ thí sinh giỏi cả ba môn. Hỏi 
	\begin{enumerate}
		\item có bao nhiêu học sinh chỉ giỏi 1 môn.
		\item có bao nhiêu học sinh chỉ giỏi 2 môn.
		\item có bao nhiêu học sinh giỏi ít nhất 1 môn.
	\end{enumerate}
	\loigiai{		
		Gọi $A$, $B$, $C$ theo thứ tự là tập hợp các học sinh giỏi Toán, giỏi Lí và giỏi Văn. Theo đề ta có
		\begin{itemize}
			\item Số học sinh giỏi Toán và Lí là $$n(A \cap B)=n(A)+n(B)-n(A \cup B)=48+37-75=10.$$	
			\item Số học sinh giỏi Toán và Văn là $$n(A \cap C)=n(A)+n(C)-n(A\cup C)=48+42 - 76 = 14 .$$	
			\item Số học sinh giỏi Lí và Văn là $$n(B \cap C)=n(B)+n(C)-n(B\cup C)= 42+37-66=13 .$$	
			\item Số học sinh chỉ giỏi môn Toán $48-10-14+4=28 $.	
			\item Số học sinh chỉ giỏi môn Lí $37-10-13+4=18 $.	
			\item Số học sinh chỉ giỏi môn Văn $42-13-14+4=19 $.
		\end{itemize}
		\begin{center}
			\begin{tikzpicture}[scale=0.6,>=stealth, font=\footnotesize, line join=round, line cap=round]	
				\def\firstcircle{(0,0) circle (2.1cm)}
				\def\secondcircle{(45:2.5cm) circle (1.95cm)}
				\def\thirdcircle{(-15:1.7cm) circle (1.4cm)}
				\colorlet{circle edge}{black!50}
				\colorlet{circle area}{black!20}
				\tikzset{filled/.style={fill=circle area, draw=circle edge, thick},
					outline/.style={draw=circle edge, thick}}
				\draw \firstcircle;
				\draw \secondcircle;
				\draw \thirdcircle;
				\begin{scope}
					\clip \firstcircle;
					\fill[filled] \secondcircle;
				\end{scope}
				\begin{scope}
					\clip \firstcircle;
					\fill[filled] \thirdcircle;
				\end{scope}
				\begin{scope}
					\clip \secondcircle;
					\fill[filled] \thirdcircle;
				\end{scope}  
				\node at (-1,0){48};  
				%\node at (0.5,1){3}; 
				%	\node at (1.2,0.3){1};  
				%	\node at (2.3,0.3){1};      
				%	\node at (1,-0.5){2};   
				\node at (2,-1.2){42};     
				\node at (2.5,2.5){37}; 
				\draw[outline] \firstcircle
				\secondcircle  \thirdcircle;    
				\node at (-2,4) {\textbf{Toán}};
				\node at (2,6) {\textbf{Toán và Lý}};
				\node at (5,5) {\textbf{Lý}};
				\node at (7,1) {\textbf{Văn và Lý}};   
				\node at (4.6,-3) {\textbf{Văn}};
				\node at (-2,-3) {\textbf{Toán và Văn}};
				\draw[dashed] (-2,3.5) -- (-1,1.2) 
				(2,5.5) -- (0.7,0.3) 
				(2,5.5) -- (0.2,0.9)
				(5,4.5) -- (3,2.8)
				(5,0.9) -- (1.2,0.5)
				(5,0.9) -- (2.4,0.2)
				(4,-2.9) -- (2.1,-1.4)
				(-2,-2.5) -- (1.2,0.3)
				(-2,-2.5) -- (0.9,-0.9)	;
		\end{tikzpicture}	
		\end{center}
		
		\begin{enumerate}	
			\item Số học sinh chỉ giỏi đúng 1 môn là $28+18+19=65$. 
			\item Số học sinh chỉ giỏi đúng 2 môn là $10+14+13-4\cdot2=25 $. 
			\item Số học sinh giỏi ít nhất một môn là $65+25+4=94$.
		\end{enumerate}
	}
\end{vd}
\begin{vd}%[BG10-2022]%[Đỗ Văn Dự]%[0D1K3-3]
	Một nhóm học sinh giỏi các bộ môn: Anh, Toán, Văn. Có $18$ em giỏi Văn, $10$ em giỏi Anh, $12$ em giỏi Toán, $3$ em giỏi Văn và Toán, $4$ em giỏi Toán và Anh, $5$ em giỏi Văn và Anh, $2$ em giỏi cả ba môn. Hỏi nhóm đó có bao nhiêu em?
	\loigiai{
		Gọi $A$ là tập hợp những học sinh giỏi Anh,
		$T$ là tập hợp những học sinh giỏi toán,
		$V$ là tập hợp những học sinh giỏi Văn.
		Ta có\\
		$\bullet$ $n\left( V \right)=18,\; n\left( A \right)=10$,\; $n\left( T \right)=12$,\; 
		$n(V\cap T)=3,\;n(T\cap A)=4,\;n(V\cap A)=5$,\; $n(A\cap B\cap C)=2$.\\
		$\bullet$  $\begin{aligned}[t] n(V\cup A\cup T)&=n\left( V \right)+n\left( A \right)+n\left( T \right)-\left[ n(V\cap A)+n(A\cap T)+n(T\cap V) \right]+n\left( V\cap A\cap T \right)\\ &=
		18+10+12-\left[ 3+4+5 \right]+2=30 \end{aligned}$.\\
		Vậy nhóm đó có 30 em.}
\end{vd}

\begin{vd}%[BG10-2022]%[Đỗ Văn Dự]%[0D1K3-3]
	Trong số $42$ học sinh của lớp 10A có $13$ bạn được xếp loại học lực giỏi, $22$ bạn được xếp loại hạnh kiểm tốt, trong đó $7$ bạn vừa học lực giỏi, vừa có hạnh kiểm tốt. Hỏi lớp 10A có bao nhiêu bạn được khen thưởng? Biết rằng muốn được khen thưởng thì bạn đó phải có học lực giỏi hoặc có hạnh kiểm tốt.
	\loigiai{
		Gọi tập hợp các học sinh học lực giỏi là $G$, tập hợp các bạn học sinh hạnh kiểm tốt là $T$. Khi đó tập hợp các bạn học sinh vừa có học lực giỏi là, vừa có hạnh kiểm tốt là $G\cap T$, tập hợp các bạn học sinh đạt học lực giỏi hoặc hạnh kiểm tốt là $G\cup T$. Ta có \\
		$n(G)=13$, $n(T)=22$, $n(G\cap T)=7$.\\
		$n(G\cup T)=n(G)+n(T)-n(G\cap T)=13+22-7=28.$}
\end{vd}

\begin{vd}%[BG10-2022]%[Đỗ Văn Dự]%[0D1K3-3]
	Một nhóm học sinh giỏi các bộ môn: Anh, Toán, Văn. Có $ 18 $ em giỏi Văn, $ 10 $ em giỏi Anh, $ 12 $ em giỏi Toán, $ 3 $ em giỏi Văn và Toán, $ 4 $ em giỏi Toán và Anh, $ 5 $ em giỏi Văn và Anh, $ 2 $ em giỏi cả ba môn. Hỏi nhóm đó có bao nhiêu em?
	\loigiai{
		\begin{center}
			\begin{tikzpicture}[scale=.8]
				\def\radius{2cm}	
				\coordinate (ceni);
				\coordinate[xshift=.8*\radius, yshift=.2*\radius] (cenii);
				\coordinate[xshift=.6*\radius,yshift=-.8*\radius] (ceniii);
				\draw (ceni) circle (.9*\radius);
				\draw (cenii) circle (\radius);
				\draw (ceniii) circle (1.11*\radius);
				
				\node[xshift=-.3*\radius, yshift=.05*\radius] at (ceni) {$8$(V)};
				\node[xshift=.35*\radius, yshift=.25*\radius] at (ceni) {$3$(VT)};
				\node[xshift=.1*\radius, yshift=-.55*\radius] at (ceni) {$5$(VA)};
				\node[xshift=.4*\radius, yshift=-.15*\radius] at (ceni) {{\footnotesize $2$(TVA)}};
				\node[xshift=.2*\radius, yshift=-.55*\radius] at (cenii) {$4$(TA)};
				\node[xshift=1.1*\radius,yshift=.3*\radius] at (ceni) {$12$(T)};
				\node[yshift=-.4*\radius] at (ceniii) {$10$(A)};
			\end{tikzpicture}
		\end{center}
		Ký hiệu $ A $ là tập hợp những học sinh giỏi Anh,\\
		$ T $ là tập hợp những học sinh giỏi Toán,\\
		$ V $ là tập hợp những học sinh giỏi Văn.\\
		$\bullet$ $n(V)=18,\; n(A)=10,\; n(T)=12$,\\
		$\bullet$ $n(T \cap V)=3,\; n(T \cap A)=4,\; n(V \cap A)=5, n(A \cap B \cap C)=2$.\\
		Số học sinh của nhóm là
		\begin{eqnarray*}
			n(V \cup A \cup T)&=&n(V)+n(A)+n(T)-n(V \cap A)-n(T \cap A)-n(T \cap V)+n(A \cap B \cap C)\\
			&=&18+10+12-(3+4+5)+2=30.
		\end{eqnarray*}
		Vậy nhóm đó có $ 30 $ em.
	}
\end{vd}

\begin{vd}%[BG10-2022]%[Đỗ Văn Dự]%[0D1K3-3]
	Có $44$ học sinh giỏi, mỗi em giỏi ít nhất một môn. Có $22$ em giỏi Văn, $25$ em giỏi Toán, $20$ em giỏi Anh. Có $8$ em giỏi đúng hai môn Văn, Toán; Có $7$ em giỏi đúng hai môn Toán, Anh; Có $6$ em giỏi đúng hai môn Anh, Văn. Hỏi có bao nhiêu em giỏi cả ba môn Văn, Toán, Anh?
	\loigiai{
		Ta có \\
		$n\left( V \right)=22,\;n\left( T \right)=25$,\; $n\left( A \right)=20$\\
		$n((V\cap T)\setminus A)=8,\;n((T\cap A)\setminus V)=7,\;n((V\cap A)\setminus T)=6,\; n(V\cup T\cup A)=44$.\\
		$n(V\cup A\cup T)=n\left( V \right)+n\left( A \right)+n\left( T \right)-n(V\cap A)-n(A\cap T)-n(T\cap V)+n\left( V\cap A\cap T \right)$\\
		$  \Leftrightarrow  44=22+20+25-6-7-8+4n\left( V\cap A\cap T \right)$.\\
		$\Rightarrow n\left( V\cap A\cap T \right)=1$.
	}
\end{vd}
\begin{vd}%[BG10-2022]%[Đỗ Văn Dự]%[0D1B3-3]
	Để thành lập đội tuyển học sinh giỏi khối $ 10 $, nhà trường tổ chức thi chọn các môn Toán, Văn, Anh trên tổng số $ 111 $ học sinh. Kết quả có: $ 70 $ học sinh giỏi Toán, $ 65 $ học sinh giỏi Văn, $ 62 $ học sinh giỏi Anh. Trong đó có $ 49 $ học sinh giỏi cả hai môn Văn và Toán, $ 32 $ học sinh giỏi cả hai môn Toán và Anh, $ 34 $ học sinh giỏi cả hai môn Văn và Anh. Xác định số học sinh giỏi cả ba môn Văn, Toán, Anh. Biết rằng có $ 6 $ học sinh không đạt yêu cầu cả ba môn.
	\loigiai{
		\immini
		{Có $ 111-6=105 $ học sinh thi đạt ít nhất $ 1 $ môn.\\
			Gọi $ A $ là số học sinh giỏi môn Toán và Tiếng Anh nhưng không giỏi Văn.\\
			Gọi $ B $ là số học sinh giỏi môn Toán và Văn nhưng không giỏi Tiếng Anh.\\
			Gọi $ C $ là số học sinh giỏi môn Văn và Tiếng Anh nhưng không giỏi Toán.\\
			Gọi $ D $ là số học sinh giỏi cả ba môn. Ta có hệ:\\
			$\begin{cases}
				B+D=49\\
				A+D=32\\
				C+D=34\\
				70+65+62-(A+B+C+2D)=105
			\end{cases}\\
			\Rightarrow 92=32-D+49-D+34-D+2D\\
			\Rightarrow D=23$.\\
			Vậy có $ 23 $ học sinh giỏi cả ba môn.
		}
		{
			\begin{tikzpicture}[scale=.8]
				\def\radius{2cm}	
				\coordinate (ceni);
				\coordinate[xshift=.8*\radius, yshift=.2*\radius] (cenii);
				\coordinate[xshift=.6*\radius,yshift=-.8*\radius] (ceniii);
				\draw (ceni) circle (.9*\radius);
				\draw (cenii) circle (\radius);
				\draw (ceniii) circle (\radius);
				
				\node[xshift=.3*\radius, yshift=.25*\radius] at (ceni) {A};
				\node[xshift=.1*\radius, yshift=-.5*\radius] at (ceni) {B};
				\node[xshift=.45*\radius, yshift=-.25*\radius] at (ceni) {D};
				\node[xshift=.1*\radius, yshift=-.55*\radius] at (cenii) {C};
		\end{tikzpicture}}
	}
\end{vd}
\subsubsection{Bài tập tự luận}
\begin{bt}%[Huỳnh Quy]%[0D1B3-3]
	Mỗi học sinh của lớp $10A$ đều chơi bóng đá hoặc bóng chuyền. Biết rằng có $25$ bạn chơi bóng đá, $20$ bạn chơi bóng chuyền và $10$ bạn chơi cả $2$ môn thể thao. Hỏi lớp $10A$ có bao nhiêu học sinh.
	\loigiai{
		Ngoài sơ đồ Ven ta có thể dùng công thức số phần tử. Gọi $A$ là tập hợp các học sinh chơi bóng đá, $B$ là tập các học sinh chơi bóng chuyền. Do đó $A\cap B$ là tập các học sinh chơi cả hai môn. Ta có
		$$n(A)=25, n(B)=20, n(A \cap B) =10.$$
		Số học sinh cả lớp là số phần tử của tập $A \cup B$ nên $n(A \cup B) = 25+20-10=35$ (học sinh).
	}
\end{bt}
\begin{bt}%[Huỳnh Quy]%[0D1B3-3]
	Lớp $10{{B}_{1}}$ có $7$ học sinh giỏi Toán, $5$ học sinh giỏi Lý, $6$ học sinh giỏi Hóa, $3$ học sinh giỏi cả Toán và Lý, $4$ học sinh giỏi cả Toán và Hóa, $2$ học sinh giỏi cả Lý và Hóa, $1$ học sinh giỏi cả $3$ môn Toán, Lý, Hóa. Tính số học sinh giỏi ít nhất một môn (Toán, Lý, Hóa) của lớp $10{{B}_{1}}$.
	\loigiai{
		Ta dùng biểu đồ Ven để giải:
		\begin{center}
			\begin{tikzpicture}
				\def\firstcircle{(0,0) circle (1.5cm)}
				\def\secondcircle{(45:2.5cm) circle (2cm)}
				\def\thirdcircle{(-15:1.7cm) circle (1.4cm)}
				\colorlet{circle edge}{black!50}
				\colorlet{circle area}{black!20}
				\tikzset{filled/.style={fill=circle area, draw=circle edge, thick},
					outline/.style={draw=circle edge, thick}}
				\draw \firstcircle;
				\draw \secondcircle;
				\draw \thirdcircle;
				\begin{scope}
					\clip \firstcircle;
					\fill[filled] \secondcircle;
				\end{scope}
				\begin{scope}
					\clip \firstcircle;
					\fill[filled] \thirdcircle;
				\end{scope}
				\begin{scope}
					\clip \secondcircle;
					\fill[filled] \thirdcircle;
				\end{scope}  
				\node at (-1,0){1};  
				\node at (0.5,1){2}; 
				\node at (1,0.3){1};  
				\node at (2,0.3){1};      
				\node at (1,-0.5){3};   
				\node at (2,-1){1};     
				\node at (3,2.5){1}; 
				\draw[outline] \firstcircle
				\secondcircle  \thirdcircle;    
				\node at (-2,4) {\textbf{Toán}};
				\node at (2,6) {\textbf{Giỏi Toán + Lý}};
				\node at (5,5) {\textbf{Lý}};
				\node at (7,1) {\textbf{Giỏi Hóa + Lý}};   
				\node at (5,-3) {\textbf{Hóa}};
				\node at (-2,-3) {\textbf{Giỏi Toán + Hóa}};
				\draw[->] (-2,3.5) -- (-1,1.2) 
				(2,5.5) -- (0.7,0.3) 
				(2,5.5) -- (0.2,0.9)
				(5,4.5) -- (3.6,2.5)
				(5,0.9) -- (1.2,0.5)
				(5,0.9) -- (2.4,0.2)
				(4,-2.9) -- (2,-1.8)
				(-2,-2.5) -- (1.2,0)
				(-2,-2.5) -- (0.9,-0.9)
				;
			\end{tikzpicture}
		\end{center}
		Nhìn vào biểu đồ, số học sinh giỏi ít nhất $1$ trong $3$ môn là: $1+2+1+3+1+1+1=10$}
\end{bt}
\begin{bt}%[Huỳnh Quy]%[0D1B3-3]
	Lớp $10{{\text{A}}_{1}}$ có $7$ học sinh giỏi Toán, $5$ học sinh giỏi Lý, $6$ học sinh giỏi Hóa, $3$ học sinh giỏi cả Toán và Lý, $4$ học sinh giỏi cả Toán và Hóa, $2$ học sinh giỏi cả Lý và Hóa, $1$ học sinh giỏi cả $3$ môn Toán, Lý, Hóa. Tính số học sinh giỏi đúng hai môn học của lớp $10{{\text{A}}_{1}}$.
	\loigiai{
		\begin{center}
			\begin{tikzpicture}
				\def\firstcircle{(0,0) circle (1.5cm)}
				\def\secondcircle{(45:2.5cm) circle (2cm)}
				\def\thirdcircle{(-15:1.7cm) circle (1.4cm)}
				\colorlet{circle edge}{black!50}
				\colorlet{circle area}{black!20}
				\tikzset{filled/.style={fill=circle area, draw=circle edge, thick},
					outline/.style={draw=circle edge, thick}}
				\draw \firstcircle;
				\draw \secondcircle;
				\draw \thirdcircle;
				\begin{scope}
					\clip \firstcircle;
					\fill[filled] \secondcircle;
				\end{scope}
				\begin{scope}
					\clip \firstcircle;
					\fill[filled] \thirdcircle;
				\end{scope}
				\begin{scope}
					\clip \secondcircle;
					\fill[filled] \thirdcircle;
				\end{scope}  
				\node at (-1,0){1};  
				\node at (0.5,1){2}; 
				\node at (1,0.3){1};  
				\node at (2,0.3){1};      
				\node at (1,-0.5){3};   
				\node at (2,-1){1};     
				\node at (3,2.5){1}; 
				\draw[outline] \firstcircle
				\secondcircle  \thirdcircle;    
				\node at (-2,4) {\textbf{Toán}};
				\node at (2,6) {\textbf{Giỏi Toán + Lý}};
				\node at (5,5) {\textbf{Lý}};
				\node at (7,1) {\textbf{Giỏi Hóa + Lý}};   
				\node at (5,-3) {\textbf{Hóa}};
				\node at (-2,-3) {\textbf{Giỏi Toán + Hóa}};
				\draw[->] (-2,3.5) -- (-1,1.2) 
				(2,5.5) -- (0.7,0.3) 
				(2,5.5) -- (0.2,0.9)
				(5,4.5) -- (3.6,2.5)
				(5,0.9) -- (1.2,0.5)
				(5,0.9) -- (2.4,0.2)
				(4,-2.9) -- (2,-1.8)
				(-2,-2.5) -- (1.2,0)
				(-2,-2.5) -- (0.9,-0.9)
				;
			\end{tikzpicture}
		\end{center}
		Dựa vào biểu đồ ven trên, ta có số học sinh giỏi đúng hai môn học là $2+1+3=6$}
\end{bt}

\subsection{Bài tập trắc nghiệm}
\Opensolutionfile{ans}[ans/ans-0D1-2-TN]
\begin{ex}%[Bài giảng Toán 10 - 2022]%[Nhật Thiện]%[0D1Y3-1]
	Cho hai tập hợp $X=\{1;3;5;8\}$, $Y=\{3;5;7;9\}$. Tập hợp $X\cup Y$ bằng tập hợp nào sau đây?
	\choice
	{$\{1;3;5\}$}
	{$\{3;5\}$}
	{$\{1;7;9\}$}
	{\True $\{1;3;5;7;8;9\}$}
	\loigiai{
		Ta có $X\cup Y=\{1;3;5;7;8;9\}$.
	}
\end{ex}
\begin{ex}%[Bài giảng Toán 10 - 2022]%[Nhật Thiện]%[0D1Y3-2]
	Cho hai tập hợp $A=\{1;2;3;4;5\}$ và $B=\{0;2;4;6;8\}$. Tìm $A\setminus B$.
	\choice
	{$A\setminus B=\{2;4\}$}
	{\True $A\setminus B=\{1;3;5\}$}
	{$A\setminus B=\{0;1;3;5\}$}
	{$A\setminus B=\{0;6;8\}$}
	\loigiai{
		Ta có $A\setminus B=\{1;3;5\}$.
	}
\end{ex}
\begin{ex}%[Bài giảng Toán 10 - 2022]%[Nhật Thiện]%[0D1B3-1]
	Cho hai tập hợp $ A=\left\lbrace x\in \mathbb{R}\,\mid\,(2x-x^2)(x-1)=0 \right\rbrace$, $ B=\left\lbrace n \in \mathbb{N}\,\mid\,0<n^2<10 \right\rbrace$. Chọn mệnh đề đúng?
	\choice
	{\True $A \cap B =\left\lbrace 1;2 \right\rbrace $}
	{$A \cap B =\left\lbrace 2 \right\rbrace $}
	{ $A \cap B =\left\lbrace 0;1;2;3 \right\rbrace $}
	{$A \cap B =\left\lbrace 0;3 \right\rbrace $}
	\loigiai{
		Ta có
		\begin{itemize}
			\item $ (2x-x^2)(x-1)=0\Leftrightarrow \hoac{& x=0 \\ &x=1\\&x=2 }\Rightarrow A=\left\lbrace 0;1;2 \right\rbrace  $.
			\item  $ B= \left\lbrace 1;2;3 \right\rbrace$.
		\end{itemize}
		Suy ra $ A\cap B= \left\lbrace 1;2 \right\rbrace$.}
\end{ex}
\begin{ex}%[Bài giảng Toán 10 - 2022]%[Nhật Thiện]%[0D1B3-1]
	Cho các tập hợp $A=\left\{ x\in \mathbb{N} \ | \ (4-x^2)(x^2-5x+4)=0 \right\}; B = \left\{ x\in \mathbb{Z} \ | \ x \text{ là ước của } 4 \right\}$. Tập hợp $A \cap B$ là
	\choice
	{$\{-2,1,2,4\}$}
	{\True $\{1,2,4\}$}
	{$\{2,4\}$}
	{$\{-4,-2,-1,1,2,4\}$}
	\loigiai{
		Ta có $(4-x^2)(x^2-5x+4)=0 \Leftrightarrow \hoac{&4-x^2=0\\&x^2-5x+4=0} \Leftrightarrow \hoac{&x=2\\&x=-2\\&x=1\\&x=4.}$\\
		Vì $x\in \mathbb{N}$ nên $x \in \{1,2,4\}$.\\
		Do đó $A=\{1,2,4\} \quad (1)$.\\
		Ta có các ước của $4$ là $\pm 1, \pm 2, \pm 4$.\\
		Do đó $B=\{-4,-2,-1,1,2,4\} \quad (2)$.\\
		Từ $(1), (2)$ ta có $A \cap B = \{1,2,4\}$.
	}
\end{ex}
\begin{ex}%[Bài giảng Toán 10 - 2022]%[Nhật Thiện]%[0D1Y4-2]
	Cho hai tập hợp $A = \left(-5;7\right)$ và $B = \left(1;+\infty\right)$. Tìm $A\setminus B$.
	\choice
	{\True $A\setminus B = \left(-5;1\right]$}
	{$A\setminus B = \left(-5;1\right)$}
	{$A\setminus B = \left[7;+\infty\right)$}
	{$A\setminus B = \left(7;+\infty\right)$}
	\loigiai{
		Ta có $A\setminus B = \left(-5;1\right]$.
	}
\end{ex}
\begin{ex}%[Bài giảng Toán 10 - 2022]%[Nhật Thiện]%[0D1Y4-2]
	Cho hai tập hợp $A=\left[ -2;4\right)$ và $B=\left(0;+\infty\right)$. Tìm khẳng định đúng.
	\choice
	{$A\cup B=\left(4;+\infty\right)$}
	{\True $A\cap B=\left(0;4\right)$}
	{$B\setminus A=\left[ -2;+\infty\right)$}
	{$A\setminus B=\left[ -2;0\right)$}
	\loigiai{
		$A\cup B=[-2;+\infty) \rightarrow $ loại.\\
		$A\cap B = (0;4) \rightarrow$ chọn.\\
		$B\setminus A = [4;+\infty) \rightarrow$ loại.\\
		$A\setminus B = [-2;0] \rightarrow$ loại.
	}
\end{ex}
\begin{ex}%[Bài giảng Toán 10 - 2022]%[Nhật Thiện]%[0D1B3-2]
	Cho $A$ là tập hợp các hình thoi, $B$ là tập hợp các hình chữ nhật và $C$ là tập hợp các hình vuông. Khi đó
	\choice
	{\True $A\cap B=C $}
	{$A\setminus B=C$}
	{$B\setminus A=C$}
	{$A\cup B=C$}
	\loigiai{
		Ta có hình thoi có hai cạnh kề vuông góc khi và chỉ khi nó là hình vuông.\\
		Hình chữ nhật có hai cạnh kề bằng nhau khi và chỉ khi nó là hình vuông.
	}
\end{ex}
\begin{ex}%[Bài giảng Toán 10 - 2022]%[Nhật Thiện]%[0D1B3-2]
	Cho hai tập hợp $M=\{1;2;3;5\}$ và $N=\{2;6;-1\}$. Xét các khẳng định
	\begin{enumEX}[(I)]{3}
		\item $M\cap N=\{2\}$
		\item $N\setminus M=\{1;3;5\}$
		\item $M\cup N=\{1;2;3;5;6;-1\}.$
	\end{enumEX}
	Có bao nhiêu khẳng định đúng trong ba khẳng định nêu trên?
	\choice
	{$0$}
	{$3$}
	{$1$}
	{\True $2$}
	\loigiai{
		Ta có $M\cap N=\{2\}$; $N\setminus M=\{6;-1\}$ và $M\cup N=\{1;2;3;5;6;-1\}$.\\
		Vậy có $2$ khẳng định đúng là (I) và (III).
	}
\end{ex}
\begin{ex}%[Bài giảng Toán 10 - 2022]%[Nhật Thiện]%[0D1B3-2]
	Cho hai tập hợp $A=\{2;4;6;8\}$ và $B$ là tập hợp các số tự nhiên nhỏ hơn $10$. Phần bù của $A$ trong $B$ là
	\choice
	{\True $\{0;1;3;5;7;9\}$}
	{$[0;10) \setminus \{2;4;6;8\}$}
	{$\varnothing$}
	{$\{1;3;5;7;9\}$}
	\loigiai{
		Vì $B$ là tập hợp các số tự nhiên nhỏ hơn $10$ nên $B=\{0;1;2;3;4;5;6;7;8;9\}$.\\
		Khi đó $\mathrm{C}_B A=\{0;1;3;5;7;9\}$.
	}
\end{ex}
\begin{ex}%[Bài giảng Toán 10 - 2022]%[Nhật Thiện]%[0D1B4-2]
	Cho hai tập hợp $C_{\mathbb{R}}A = (0;+\infty)$ và $C_{\mathbb{R}}B =(-\infty;-5)\cup  (-2;+\infty)$. Xác định tập $A \cup B$.
	\choice
	{$A \cap B = (-2;0)$}
	{$A \cap B = (-5;-2)$}
	{$A \cap B = (-5;0]$}
	{\True $A \cap B = [-5;-2]$}
	\loigiai{
		Ta có $C_{\mathbb{R}}A \cup C_{\mathbb{R}}B = C_{\mathbb{R}} (A \cap B) = (-\infty;-5)\cup  (-2;+\infty)$, suy ra $A \cap B = [-5;-2]$.
	}
\end{ex}

\begin{ex}%[Bài giảng Toán 10 - 2022]%[Nhật Thiện]%[0D1B4-1]
	Hình vẽ nào dưới đây biểu diễn cho tập hợp $[-2;1]\cap (0;1)$?
	\choice
	{\begin{tikzpicture}[scale=.5,line join=round]
			\draw[->](-3,0)->(5,0); %Ve truc so
			\IntervalLR{-3}{-1}
			\IntervalGRF{}{}{(}{0}
			\IntervalLR{3}{5}
			\IntervalGRF{]}{1}{}{}
			\def\skipInterval{0.5cm}%khoang cach dat nhan
	\end{tikzpicture}}
	{\begin{tikzpicture}[scale=.5,line join=round]
			\draw[->](-3,0)->(5,0); %Ve truc so
			\IntervalLR{-3}{-1}
			\IntervalGRF{}{}{[}{-2}
			\IntervalLR{3}{5}
			\IntervalGRF{)}{1}{}{}
			\def\skipInterval{0.5cm}%khoang cach dat nhan
	\end{tikzpicture}}
	{\True \begin{tikzpicture}[scale=.5,line join=round]
			\draw[->](-3,0)->(5,0); %Ve truc so
			\IntervalLR{-3}{-1}
			\IntervalGRF{}{}{(}{0}
			\IntervalLR{3}{5}
			\IntervalGRF{)}{1}{}{}
			\def\skipInterval{0.5cm}%khoang cach dat nhan
	\end{tikzpicture}}
	{\begin{tikzpicture}[scale=.5,line join=round]
			\draw[->](-3,0)->(5,0); %Ve truc so
			\IntervalLR{-3}{-1}
			\IntervalGRF{}{}{[}{0}
			\IntervalLR{3}{5}
			\IntervalGRF{]}{1}{}{}
			\def\skipInterval{0.5cm}%khoang cach dat nhan
	\end{tikzpicture}}
	\loigiai{
		Ta có $[-2;1]\cap (0;1)=(0;1)$.}
\end{ex}
\begin{ex}%[Bài giảng Toán 10 - 2022]%[Nhật Thiện]%[0D1K3-2]
	Cho hai tập $A=\left\{x\in \mathbb{Z}\left| \dfrac{x+5}{x+1}\in \mathbb{Z}\right. \right\}$ và $B=\{x\in \mathbb{N}\mid x^2-4x+3=0\}$. Có bao nhiêu tập hợp $X$ thỏa mãn $B\subset X \subset A$?
	\choice
	{$64$}
	{\True $16$}
	{$8$}
	{$32$}
	\loigiai{
		Ta có $\dfrac{x+5}{x+1}=1+\dfrac{4}{x+1}$.\\
		Vì $x\in\mathbb{Z}$ và $\dfrac{x+5}{x+1}\in \mathbb{Z}$ nên $\dfrac{4}{x+1}\in \mathbb{Z}\Leftrightarrow x+1\in \{1;2;4;-1;-2;-4\} \Leftrightarrow x\in\{0;1;3;-2;-3;-5\}$.\\
		Do đó $A=\{-5;-3;-2;0;1;3\}$.\\
		Vì $x^2-4x+3=0\Leftrightarrow \hoac{&x=1\\&x=3}$ nên $B=\{1;3\}$.\\
		Ta có $B\subset X \subset A\Leftrightarrow \{1;3\}\subset X \subset \{-5;-3;-2;0;1;3\}$.\\
		Suy ra số tập $X$ đúng bằng số tập con của tập $\{-5;-3;-2;0\}$.\\
		Vậy số tập $X$ là $2^4=16$.
	}
\end{ex}
\begin{ex}%[Bài giảng Toán 10 - 2022]%[Nhật Thiện]%[0D1G3-2]
	Cho tập hợp $X=\{3;-4;5\}$ có hai tập con $A$ và $B$ (số phần tử của tập $B$ ít hơn số phần tử của tập $A$). Có bao nhiêu cặp $(A;B)$ mà $\{3;-4\}\cup (A\setminus B)=X$?
	\choice
	{$12$}
	{$10$}
	{\True $11$}
	{$15$}
	\loigiai{
		Từ giả thiết $\{3;-4\}\cup (A\setminus B)=X\Rightarrow 5\in (A\setminus B)\Rightarrow \heva{&5\in A\\&5\notin B.}$\\
		Vì số phần tử của tập $B$ ít hơn số phần tử của tập $A$ nên tập $B$ có không quá $2$ phần tử.\\
		Các khả năng có thể xảy ra và thỏa mãn là
		\begin{itemize}
			\item TH1: $A=\{3;-4;5\}$ và $B$ bằng một trong các tập sau $\varnothing$, $\{3\}$, $\{-4\}$, $\{3;-4\}$.
			\item TH2: $A=\{-4;5\}$ và $B$ bằng một trong các tập sau $\varnothing$, $\{3\}$, $\{-4\}$.
			\item TH3: $A=\{3;5\}$ và $B$ bằng một trong các tập sau $\varnothing$, $\{3\}$, $\{-4\}$.
			\item TH4: $A={5}$ và $B=\varnothing$.
		\end{itemize}
		Vậy tất cả có $11$ kết quả thỏa mãn.
	}
\end{ex}
\begin{ex}%[Bài giảng Toán 10 - 2022]%[Nhật Thiện]%[0D1K4-2]
	Tìm điều kiện của tham số $m$ để $A\cap B$ là một khoảng, biết $A ( m ; m + 2 )$, $B ( 4 ; 7 )$.
	\choice
	{$4 \leq m<7$}
	{\True $2<m<7$}
	{$2 \leq m<7$}
	{$2< m < 4$}
	\loigiai{
		Để $A \cap B = \varnothing$ thì $\hoac {&m+2 \leq 4 \\ & m \geq 7 } \Leftrightarrow \hoac{&m \leq 2\\&m\geq 7.}$ \\
		Do đó, để $A\cap B$ là một khoảng thì $2<m<7$.}
\end{ex}
\begin{ex}%[Bài giảng Toán 10 - 2022]%[Nhật Thiện]%[0D1K4-2]
	Cho hai tập hợp $A=(m-1;5)$ và $B=(3;+\infty)$. Tìm tất cả các giá trị thực của tham số $m$ để $A\setminus B=\varnothing$.
	\choice
	{$4\leq m\leq 6$}
	{$m=4$}
	{\True $m\geq 4$}
	{$4\leq m<6$}
	\loigiai
	{Ta có $A\setminus B=\varnothing\Leftrightarrow 3\leq m-1\Leftrightarrow m\geq 4$.
	}
\end{ex}
\begin{ex}%[Bài giảng Toán 10 - 2022]%[Nhật Thiện]%[0D1G4-1]
	Cho hai tập hợp $A=[-5;8)$ và $B=[-m;m+2]$. Tìm tất cả các giá trị của $m$ để $A \cap B \ne \varnothing$.
	\choice
	{$m\in(-8;6)$}
	{$m\in[-7;+\infty)$}
	{$m\in(-8;+\infty)$}
	{\True $m\in(-1;+\infty)$}
	\loigiai{
		$A \cap B \ne \varnothing \Leftrightarrow \heva{&-m < m + 2\\&-m < 8\\&m+2 \ge -5}\Leftrightarrow m > -1$.
	}
\end{ex}
\begin{ex}%[Nguyễn Vương Hiển]%[0D1Y2-1]
	Tập hợp $A=\left\{x\in \mathbb{R}\big| x^2-6x+8=0\right\}$ có bao nhiêu phần tử?
	\choice
	{$0$}
	{$1$}
	{\True $2$}
	{vô số}
	\loigiai{
		$x^2-6x+8=0\Leftrightarrow \hoac{&x=2\\&x=4.}$\\
		Vậy tập hợp $A$ có $2$ phần tử.
	}
\end{ex}
\begin{ex}%[Nguyễn Vương Hiển]%[0D1B2-1]
	Tập hợp $A=\left\{x\in \mathbb{Z}^{+}\big|x^2-x=0\right\}$ có bao nhiêu phần tử?
	\choice
	{\True $1$}
	{$2$}
	{$0$}
	{$3$}
	\loigiai{
	$x^2-x=0\Leftrightarrow \hoac{&x=0\\&x=1.}$\\
	Vì $x\in\mathbb{Z}^{+}$ nên $A=\{1\}$.
	Vậy tập hợp $A$ có $1$ phần tử.
	}
\end{ex}
\begin{ex}%[Nguyễn Vương Hiển]%[0D1Y2-1]
	Hãy viết tập hợp $A=\left\{x\in \mathbb{R}\big| x^2-6x+8=0\right\}$ dưới dạng liệt kê các phần tử.
	\choice
	{\True $A=\left\{2;4\right\}$}
	{$A=\left\{6;8\right\}$}
	{$A=\left\{-2;2\right\}$}
	{$A=\left(2; 4\right)$}
	\loigiai{
		Ta có $x^2-6x+8=0\Leftrightarrow \hoac{&x=2\\&x=4}$ nên $A=\left\{2; 4\right\}$.	
	}
\end{ex}
\begin{ex}%[Nguyễn Vương Hiển]%[0D1Y2-1]
	Trong các mệnh đề sau, mệnh đề nào đúng?
	\choice{\True ``$x\in [-4;1)\Leftrightarrow -4 \le x <1$''}
	{``$x\in [-4;1)\Leftrightarrow -4 < x \le 1 $''}
	{``$x\in [-4;1)\Leftrightarrow -4 \le x \le 1$''}
	{``$x\in [-4;1)\Leftrightarrow -4 < x <1$''}
	\loigiai{ Ta có ``$x\in [-4;1)\Leftrightarrow -4 \le x <1$''.
	}
\end{ex}
\begin{ex}%[Nguyễn Vương Hiển]%[0D1Y2-1]
	Số tập con của tập hợp $X=\{x\in\mathbb{Z}\ |\ 2x^2-5x+2=0\}$ là?
	\choice
	{$1$}
	{$3$}
	{\True $2$}
	{$4$}
	\loigiai{
		Ta có $2x^2-5x+2=0\Leftrightarrow \hoac{&x=2\\ &x=\dfrac{1}{2}}$, mà $x\in\mathbb{Z}$ nên $x=2$.\\ Vậy $X=\{2\}$ nên có $2$ tập con.
	}
\end{ex}
\begin{ex}%[Nguyễn Vương Hiển]%[0D1Y2-1]
	Tập hợp $A=\{1;2;3;4;5;6\}$ được viết dưới dạng chỉ ra tính chất đặc trưng cho các phần tử của nó là
	\choice
	{$A=\left\{n\in \mathbb{N}\big| 1<n\le 6\right\}$}
	{$A=\left\{n\in \mathbb{N}\big| n\le 6\right\}$}
	{\True $A=\left\{n\in \mathbb{N}\big| 0<n\le 6\right\}$}
	{$A=\left\{n\in \mathbb{N}\big| 0<n<6\right\}$}
	\loigiai{
		Ta có $A=\{1;2;3;4;5;6\}=\left\{n\in \mathbb{N}\big| 0<n\le6\right\}$.
	}
\end{ex}
\begin{ex}%[Nguyễn Vương Hiển]%[0D1Y3-1]
	Cho hai tập hợp $X=\left\{ 7, 2, 8, 4, 9, 12 \right\}$ và $Y=\left\{ 1, 3, 7, 4 \right\}$. Tìm tập hợp $X\cap Y$.
	\choice
	{$\left\{ 1, 2, 3, 4, 8, 9, 7, 12 \right\}$}
	{$\left\{ 2, 8, 9, 12 \right\}$}
	{\True $\left\{ 4, 7 \right\}$}
	{$\left\{ 1, 3 \right\}$}
	\loigiai{
		Ta có	$X\cap Y=\{4,7\}$.
	}
\end{ex}
\begin{ex}%[Nguyễn Vương Hiển]%[0D1Y3-1]
	Cho hai tập hợp $X=\left\{ 2, 4, 6, 9 \right\}$ và $Y=\left\{ 1, 2, 3, 4 \right\}$. Tìm tập hợp $X \cup Y$.
	\choice
	{$\left\{1, 3 \right\}$	}
	{$\left\{6, 9 \right\}$}
	{\True $\left\{1, 2, 3, 4, 6, 9 \right\}$}
	{$\left\{2, 4 \right\}$}
	\loigiai{
		Ta có $X \cup Y=\left\{1, 2, 3, 4, 6, 9 \right\}$.
	}
\end{ex}
\begin{ex}%[Nguyễn Vương Hiển]%[0D1Y3-2]
	Cho hai tập hợp $X=\left\{0, 1, 2, 3, 4\right\}$ và $Y=\left\{ 2, 3, 4, 5, 6 \right\}$. Tìm tập hợp $X\setminus Y$.
	\choice
	{$\left\{ 0 \right\}$}
	{\True $\left\{ 0, 1 \right\}$}
	{$\left\{ 1, 2 \right\}$}
	{$\left\{ 1, 5 \right\}$}
	\loigiai{
		Ta có $X\setminus Y=\left\{ 0, 1 \right\}$.
	}
\end{ex}
\begin{ex}%[Nguyễn Vương Hiển]%[0D1Y3-2]
	Cho hai tập hợp $A = \left\{0, 2, 4, 6, 8\right\}$ và $B = \left\{0, 2, 4\right\}$. Tìm tập hợp $C_{A}B$.
	\choice
	{$\left\{0, 2, 4, 6\right\}$}
	{$\left\{0, 2, 4, 8\right\}$}
	{$\left\{2, 4\right\}$}
	{\True $\left\{6, 8\right\}$}
	\loigiai{
		Ta có $B\subset A$ và $A\setminus B=\{6,8\}\Rightarrow C_{A}B=\{6, 8\}$.
	}
\end{ex}
\begin{ex}%[Nguyễn Vương Hiển]%[0D1B4-1]
	Cho $A=(-\infty;-2], B=[3;+\infty)$ và $C=(0;4)$. Khi đó tập $(A\cup B)\cap C$ là
	\choice
	{$(-\infty;-2)\cup[3;+\infty)$}
	{$(-\infty;-2]\cup (3;+\infty)$}
	{\True $[3;4)$}
	{$[3;4]$}
	\loigiai{
		$(A\cup B)=(-\infty;-2]\cup [3;+\infty)$.\\
		Vậy $(A\cup B)\cap C =[3;4)$.
	}
\end{ex}
\begin{ex}%[Nguyễn Vương Hiển]%[0D1B4-1]
	Cho hai tập hợp $A=(-3;4]$ và $B=(-\sqrt 2;+\infty)$. Tập hợp $A\cap B$ là
	\choice
	{\True $(-\sqrt 2;4]$}
	{$(-3;+\infty)$}
	{$(-3;-\sqrt 2]$}
	{$(4;+\infty)$}
	\loigiai{
		Ta có $A\cap B=(-\sqrt 2;4]$.
	}
\end{ex}
\begin{ex}%[Nguyễn Vương Hiển]%[0D1K4-1]
	Cho hai tập hợp $A=\left\{ x\in \mathbb{R}\big|x+2\geq 0 \right\}$ và $B=\left\{ x\in \mathbb{R}\big|5-x\geq 0 \right\}$. Tìm tập hợp $A\cap B$. 
	\choice
	{\True $\left[ -2;5 \right]$}
	{$\left[ -2;6 \right]$}
	{$\left[ -5;2 \right]$}
	{$\left( -2;+\infty  \right)$}
	\loigiai{
		Ta có $\heva{& A=\left\{ x\in \mathbb{R}\big|x+2\geq 0 \right\}=[-2;+\infty)\\&B=\left\{ x\in \mathbb{R}\big|5-x\geq 0 \right\}=(-\infty;5].}$\\
		Khi đó $A\cap B=[-2;+\infty)\cap (-\infty;5]=[-2;5]$.
	}
\end{ex}
\begin{ex}%[Nguyễn Vương Hiển]%[0D1K4-1]
	Cho các tập hợp $M = [1; 4]$, $N = (2; 6)$ và $P = (1; 2)$. Tìm tập hợp $(M \cap N) \cap P$.
	\choice
	{$[0; 4]$}
	{$[5; + \infty  )$}
	{$(- \infty  ; 1)$}
	{\True $\varnothing $}
	\loigiai{
		Ta có $M\cap N=(2;4]\Rightarrow M\cap N\cap P=(2;4]\cap (1;2)=\varnothing$.
	}
\end{ex}
\begin{ex}%[Nguyễn Vương Hiển]%[0D1G3-3]
	Lớp $10A$ có $10$ học sinh giỏi Văn, $15$ học sinh giỏi Sử, $5$ học sinh giỏi cả $2$ môn Văn, Sử và $2$ học sinh không giỏi môn nào. Hỏi lớp $10A$ có bao nhiêu học sinh?
	\choice
	{$20$ }
	{\True $22$}
	{$25$}
	{$28$}
	\loigiai{
		Số học sinh giỏi một môn Văn: $10-5=5$(học sinh).\\
		Số học sinh giỏi một môn Sử: $15-5=10$(học sinh).\\
		Số học sinh lớp $10A$: $2+5+10+5=22$(học sinh).
	}
\end{ex}
\begin{ex}%[Nguyễn Vương Hiển]%[0D1G3-3]
	Để phục vụ cho công việc tiêm vắc-xin phòng chống Covid-19, Sở y tế đã huy động $30$ cán bộ đo huyết áp, $25$ cán bộ tiêm vắc-xin. Trong đó có $12$ cán bộ làm được cả $2$ công việc đo huyết áp và tiêm vắc-xin. Hỏi Sở y tế đã huy động tất cả bao nhiêu cán bộ cho công việc tiêm vắc-xin phòng chống Covid-19?
	\choice{$42$}
	{$31$}
	{$55$}
	{\True $43$}
	\loigiai{
		Số cán bộ được huy động là: $30+25-12=43$ cán bộ.
	}
\end{ex}

\Closesolutionfile{ans}
% % \indapan{10}{ans/ans-0D1-2-TN}
% \Closesolutionfile{ansbook}
% \section*{Đề kiểm tra Chương 1}
\subsection*{Đề số 1}
\setcounter{ex}{0}\setcounter{bt}{0}
\Opensolutionfile{ans}[ans/ans-KT-101]
\noindent\textbf{I. PHẦN TRẮC NGHIỆM}
%[Phan Quốc Trí, Bai Giảng T10(2022)]%
\begin{ex}%[Phan Quốc Trí, Bai Giảng T10(2022)]%[0D1Y1-1]
	Trong các câu sau, câu nào không phải là mệnh đề?
	\choice
	{Hình bình hành có bốn cạnh bằng nhau}
	{\True Chúc bạn may mắn}
	{Số $8$ là số chính phương}
	{Cà Mau là tên một tỉnh của nước Việt Nam}
	\loigiai{
		Mệnh đề là một câu khẳng định \textbf{đúng} hoặc một câu khẳng định \textbf{sai}.
	}
\end{ex}
\begin{ex}%[Phan Quốc Trí, Bai Giảng T10(2022)]%[0D1Y1-1]
	Trong các câu sau, câu nào là mệnh đề?
	\choice
	{$x^2+x=2$}
	{Hôm nay trời đẹp quá!}
	{$2n+1$ chia hết cho 3}
	{\True Số 15 là một số nguyên tố}
	\loigiai{
\begin{itemize}
	\item $x^2+x=2$  là một khẳng định, nhưng không là mệnh đề. 
	\item  Hôm nay trời đẹp quá! không phải là mệnh đề.
	\item Câu $2n+1$ chia hết cho 3 là một khẳng định, nhưng không là mệnh đề. 
	\item  Số 15 là một số nguyên tố là một mệnh đề sai.
\end{itemize}
	}
\end{ex}
\begin{ex}%[Phan Quốc Trí, Bai Giảng T10(2022)]%[0D1Y1-2]
	Trong các mệnh đề sau, mệnh đề nào là mệnh đề \textbf{sai}?
	\choice
	{\True $\sqrt{3}$ là số nguyên}
	{$6$ chia hết cho $2$}
	{$5$ chia hết cho $5$}
	{$30$ là một số chẵn}
	\loigiai{
		Trong các mệnh đề đã cho, mệnh đề sai là \lq\lq$\sqrt{3}$ là số nguyên\rq\rq.
	}
\end{ex}
\begin{ex}%[Phan Quốc Trí, Bai Giảng T10(2022)]%[0D1Y1-2]
	Cho mệnh đề chứa biến $P(x): 3x+5\le x^2$ với $x$ là số thực. Mệnh đề nào sau đây là đúng?
	\choice
	{$P(3)$}
	{$P(4)$}
	{$P(1)$}
	{\True $P(5)$}
	\loigiai{
		$P(3)= 3.3+5\leqslant 3^2 \Leftrightarrow 14\leqslant 9$ là mệnh đề sai.\\
		$P(4) =3.4+5\leqslant 4^2 \Leftrightarrow 17\leqslant 16$ là mệnh đề sai.\\
		$P(1)=3.1+5\leqslant 1^2 \Leftrightarrow 8\leqslant 1$ là mệnh đề sai.\\
		$P(5) =3.5+5\leqslant 5^2 \Leftrightarrow 20\leqslant 25$ là mệnh đề đúng.}
\end{ex}
\begin{ex}%[Phan Quốc Trí, Bai Giảng T10(2022)]%[0D1Y1-3]
	Cho mệnh đề $P\colon$ \lq\lq  $9$ là số chia hết cho $3$\rq\rq. Mệnh đề phủ định của mệnh đề $P$ là
	\choice
	{$\overline{P}\colon$\lq\lq  $9$ là ước của $3$\rq\rq}
	{$\overline{P}\colon$\lq\lq  $9$ là bội của $3$\rq\rq}
	{\True $\overline{P}\colon$ \lq\lq  $9$ là số không chia hết cho $3$\rq\rq}
	{$\overline{P}\colon$\lq\lq  $9$ là số lớn hơn $3$\rq\rq}
	\loigiai{Mệnh đề $P\colon$\lq\lq  $9$ là số chia hết cho $3$\rq\rq      có mệnh đề phủ định là $\overline{P}\colon$ \lq\lq  $9$ là số không chia hết cho $3$\rq\rq.}
\end{ex}
\begin{ex}%[Phan Quốc Trí, Bai Giảng T10(2022)]%[0D1Y1-3]
	Cho mệnh đề \lq\lq$\forall x\in \mathbb{R},\, x^2+1>0$\rq \rq. Mệnh đề phủ định của mệnh đề đã cho là
	\choice
	{\lq\lq$\forall x\in \mathbb{R},\, x^2+1\leq 0$\rq \rq}
	{\lq\lq$\forall x\in \mathbb{R},\, x^2+1<0$\rq \rq}
	{\True \lq\lq$\exists x\in \mathbb{R},\, x^2+1\leq 0$\rq \rq}
	{\lq\lq$\exists x\in \mathbb{R},\, x^2+1>0$\rq \rq}
	\loigiai{
		Mệnh đề phủ định của mệnh đề \lq\lq$\forall x\in \mathbb{R},\, x^2+1>0$\rq \rq\, là \lq\lq$\exists x\in \mathbb{R},\, x^2+1\leq 0$\rq \rq.}
\end{ex}
\begin{ex}%[Phan Quốc Trí, Bai Giảng T10(2022)]%[0D1Y1-4]
	Cho mệnh đề $P\colon$ ``Tam giác $ABC$ cân tại $A$'', mệnh đề $Q\colon$ ``$AB=AC$''. Phát biểu mệnh đề ``$P$ kéo theo $Q$'' là
	\choice
	{Nếu $AB=AC$ thì tam giác $ABC$ cân tại $A$}
	{\True Nếu tam giác $ABC$ cân tại $A$ thì $AB=AC$}
	{Tam giác $ABC$ cân tại $B$ là điều kiện cần và đủ để $AB=AC$}
	{Tam giác $ABC$ cân tại $A$ khi và chỉ khi $AB=AC$}
	\loigiai{
		Mệnh đề ``$P$ kéo theo $Q$'' là ``Nếu tam giác $ABC$ cân tại $A$ thì $AB=AC$''.
	}
\end{ex}
\begin{ex}%[Phan Quốc Trí, Bai Giảng T10(2022)]%[0D1Y1-4]
	Cho mệnh đề $P$: \lq\lq Nếu tam giác có hai đường trung tuyến bằng nhau thì đó là tam giác cân\rq\rq. Mệnh đề nào sau đây là mệnh đề đảo của $P$?
	\choice
	{Tam giác có hai đường trung tuyến bằng nhau thì nó là tam giác cân}
	{\True Nếu tam giác $ABC$ cân thì tam giác đó có hai đường trung tuyến bằng nhau }
	{Tam giác là tam giác cân khi và chỉ khi nó có hai đường trung tuyến bằng nhau}
	{Tam giác là tam giác cân nếu nó có hai đường trung tuyến bằng nhau}
	\loigiai{
Mệnh đề đảo là \lq \lq Nếu tam giác $ABC$ cân thì tam giác đó có hai đường trung tuyến bằng nhau.\rq \rq		
	}
\end{ex}
\begin{ex}%[Phan Quốc Trí, Bai Giảng T10(2022)]%[0D1Y1-5]
	Mệnh đề "Bình phương mọi số thực đều không âm" mô tả mệnh đề nào dưới đây?
	\choice
	{"$\forall n\in\mathbb{N}: n^2\geq 0$"}
	{"$\exists x\in\mathbb{R}:x^2\geq 0$"}
	{\True "$\forall x\in\mathbb{R}: x^2\geq 0$"}
	{"$\forall x\in\mathbb{R}:x^2>0$"}
	\loigiai{
	Mệnh đề trên được viết lại dạng: \lq \lq $\forall x\in\mathbb{R}: x^2\geq 0$\rq \rq	
	}
\end{ex}
\begin{ex}%[Phan Quốc Trí, Bai Giảng T10(2022)]%[0D1B1-5]
	Mệnh đề nào sau đây là đúng?
	\choice
	{$\forall n\in\mathbb{N}\colon n^2>n$}
	{$\forall x\in\mathbb{R}\colon x^2<2$}
	{$\forall x\in\mathbb{Z}\colon 2x>1$}
	{\True $\exists x\in\mathbb{R}\colon x^2>x$}
	\loigiai
	{
		\begin{itemize}
			\item Mệnh đề ``$\forall n\in\mathbb{N}\colon n^2>n$'' sai vì với $n=1$ thì $n^2=1=n$.
			\item Mệnh đề ``$\forall x\in\mathbb{R}\colon x^2<2$'' sai vì với $x=2$ thì $x^2=4>2$.
			\item Mệnh đề ``$\forall x\in\mathbb{Z}\colon 2x>1$'' sai vì với $x=0$ thì $2x=0<1$.
			\item Mệnh đề ``$\exists x\in\mathbb{R}\colon x^2>x$'' đúng vì với $x=2$ thì $x^2=4$ nên $x^2>x$.
		\end{itemize}
	}
\end{ex}
\begin{ex}%[Phan Quốc Trí, Bai Giảng T10(2022)]%[0D1Y2-1]
	Hãy liệt kê các phần tử của tập hợp $X = \left\{ x \in \mathbb{Z} | 2x^2-5x+3=0 \right\}$.
	\choice
	{$X = \left\{1;\dfrac{3}{2} \right\}$}
	{\True $X = \{ 1 \}$}
	{$X = \left\{ \dfrac{3}{2} \right\}$}
	{$X = \varnothing$}
	\loigiai{
		Ta có $2x^2-5x+3=0\Leftrightarrow \hoac{&x=1\in \mathbb{Z}\\&x=\dfrac{3}{2}\notin \mathbb{Z}}$.\\
		Vậy $X = \{ 1 \}$.		
	}
\end{ex}
\begin{ex}%[Phan Quốc Trí, Bai Giảng T10(2022)]%[0D1B2-1]
	Viết tập hợp $A=\lbrace x \in \mathbb{Z}|x^2<17 \rbrace $ theo cách liệt kê các phần tử, ta được tập hợp nào sau đây?
	\choice{\True $ \lbrace -4;-3;-2;-1;0;1;2;3;4 \rbrace $}
	{$ \lbrace 1;2;3;4 \rbrace $}
	{$\lbrace 0;1;2;3;4  \rbrace $}
	{$ \lbrace -4; -3;-2;-1 \rbrace $}
	\loigiai{Ta có $x^2<17\Leftrightarrow \left|x\right|<\sqrt{17}\Leftrightarrow -\sqrt{17}<x<\sqrt{17}$.\\
		Vì $x\in \mathbb{Z}$ nên $A=\lbrace -4;-3;-2;-1;0;1;2;3;4 \rbrace $.
	}
\end{ex}

\begin{ex}%[Phan Quốc Trí, Bai Giảng T10(2022)]%[0D1Y2-1]
	Cho tập hợp $A=\left\{ x \in \mathbb{N}| x^2+2x-3=0 \right\}$. Mệnh đề nào sau đây là đúng?
	\choice
	{\True $-3 \notin A$}
	{$A=\left\{1;-3\right\}$}
	{$1 \notin A$}
	{$A=\left\{1;3\right\}$}
	\loigiai{
		Ta có $x^2+2x-3=0 \Leftrightarrow \hoac{&x=1\\&x=-3}$. Do $x \in \mathbb{N}$ nên $A=\left\{1\right\}$. Do đó $-3 \notin A$.
	}
\end{ex}
\begin{ex}%[Phan Quốc Trí, Bai Giảng T10(2022)]%[0D1Y2-2]
	Cho tập $A=\{a;b;5\}$. Số tập con của tập $A$ là
	\choice
	{$5$}
	{\True $8$}
	{$7$}
	{$4$}
	\loigiai{Tập con của $A$ là $\varnothing, \{a\}, \{b\},\{5\}, \{a;b\}, \{a;5\}, \{b;5\}, \{a;b;5\} $. Vậy số tập con của $A$ là $8$.
	}
\end{ex}
\begin{ex}%[Phan Quốc Trí, Bai Giảng T10(2022)]%[0D1Y2-2]
	Có bao nhiêu tập $ X $ thỏa mãn $ \{a;b\} \subset X \subset \{1;2;a;b\}$?
	\choice
	{$3$}
	{$2$}
	{\True $4$}
	{$5$}
	\loigiai{
		Các tập $ X $ thỏa mãn là $ \{a;b\} $, $ \{1;a;b\} $, $ \{2;a;b\} $, $ \{1;2;a;b\} $.
	}
\end{ex}
\begin{ex}%[Phan Quốc Trí, Bai Giảng T10(2022)]%[0D1B4-1]
	Cho tập hợp $A=\{x\in\mathbb{R}|-3<x\le 3\}$. Mệnh đề nào dưới đây đúng?
	\choice
	{$A=\{-2;-1;0;1;2;3\}$}
	{\True $A=(-3;3]$}
	{$A=[-3;3]$}
	{$A=[-3;3)$}
	\loigiai{
		Từ giả thiết, có $A=(-3;3]$.
	}
\end{ex}
\begin{ex}%[Phan Quốc Trí, Bai Giảng T10(2022)]%[0D1B2-2]
	Cho tập hợp $A = \left\{x \in \mathbb{N}\big| x^2 + 8x + 15 = 0\right\}$. Khẳng định nào sau đây đúng?
	\choice
	{$A = \left\{-3;-5\right\}$}
	{\True $A =\varnothing$}
	{$A = \left\{\varnothing\right\}$}
	{$A = \left\{0\right\}$}
	\loigiai{
		Phương trình $x^2+8x+15=0$ có hai nghiệm $x_1=-3$, $x_2=-5$. Tuy nhiên $x_1, x_2\notin \mathbb{N}$. Vậy $A=\varnothing$.
	}
\end{ex}
\begin{ex}%[Phan Quốc Trí, Bai Giảng T10(2022)]%[0D1B2-2]
	Gọi $A$ là tập hợp tất cả các hình bình hành và $B$ là tập hợp tất cả các hình chữ nhật. Trong các kết luận sau, kết luận nào đúng?
	\choice
	{$A \subset B$}
	{\True $B \subset A$}
	{$A=B$}
	{$A \cap B=\varnothing$}
	\loigiai{
		Ta có hình chữ nhật là hình bình hành có một góc vuông nên $B \subset A$.
	}
\end{ex}

\begin{ex}%[Phan Quốc Trí, Bai Giảng T10(2022)]%[0D1Y2-2]
	Khẳng định nào sau đây là đúng?
	\choice
	{$ \mathbb{R}\subset\mathbb{Q} $}
	{$ \mathbb{Z}\subset\mathbb{N} $}
	{$ \mathbb{Q}\subset\mathbb{Z} $}
	{\True $ \mathbb{N}\subset\mathbb{R} $}
	\loigiai{
		Ta có $ \mathbb{N}\subset\mathbb{Z}\subset\mathbb{Q}\subset\mathbb{R} $.
	}
\end{ex}
\begin{ex}%[Phan Quốc Trí, Bai Giảng T10(2022)]%[0D1Y3-1]
	Cho hai tập hợp $X=\left\{1;3;5;8\right\}$ và $Y=\left\{3;5;7;9\right\}$. Tập hợp $X\cup Y$ bằng
	\choice
	{$\left\{1;7;9\right\}$}
	{$\left\{3;5\right\}$}
	{$\left\{1;3;5\right\}$}
	{\True $\left\{1;3;5;7;8;9\right\}$}
	\loigiai{
		Ta có $X\cup Y=\left\{1;3;5;7;8;9\right\}$.
	}
\end{ex}
\begin{ex}%[Phan Quốc Trí, Bai Giảng T10(2022)]%[0D1Y3-1]
	Cho $A=\{2;3;6;7\}, B=\{3;6;8\}$. Tập hợp $A\cap B$ bằng
	\choice
	{$\{3;6;8\}$}
	{$\{2;3;6;7;8\}$}
	{\True $\{3;6\}$}
	{$\{2;7\}$}
	\loigiai{
		Ta có $A\cap B=\{3;6\}.$
	}
\end{ex}
\begin{ex}%[Phan Quốc Trí, Bai Giảng T10(2022)]%[0D1Y3-2]
	Cho hai tập hợp $A=\{2;4;6;9\}$, $B=\{1;2;3;4\}$. Tập $A \setminus B$ bằng tập hợp nào sau đây?
	\choice
	{$\{2;4\}$}
	{$\{1;3\}$}
	{\True $\{6;9\}$}
	{$\{6;9;1;3\}$}
	\loigiai{
		Ta có: $A \setminus B=\{6;9\}$.}
\end{ex}

\begin{ex}%[Phan Quốc Trí, Bai Giảng T10(2022)]%[0D1Y3-2] 
	Cho tập $ X=\left\lbrace 0;1;2;3;4;5 \right\rbrace$ và tập $ A=\left\lbrace 0;2;4 \right\rbrace  $. Tìm phần bù của $ A $ trong $ X $.	
	\choice
	{$ \varnothing $}
	{$ \left\lbrace 2;4 \right\rbrace $ }
	{$ \left\lbrace 0;1;3 \right\rbrace  $}
	{\True $ \left\lbrace 1;3;5 \right\rbrace  $}
	\loigiai
	{
		Ta có phần bù của $ A $ trong $ X $ bằng tập $ X\setminus A=\left\lbrace 1;3;5 \right\rbrace  $.	
	}
\end{ex}
\begin{ex}%[Phan Quốc Trí, Bai Giảng T10(2022)]%[0D1B3-1]
	Cho hai tập hợp $A=(-3;3)$ và $B=(0;+\infty)$. Tìm  $A \cup B$.
	\choice
	{\True $ A \cup B = (-3;+\infty) $}
	{$ A \cup B = [-3;+\infty) $}
	{$ A \cup B = [-3;0) $}
	{$ A \cup B = (0;3) $}
	\loigiai{
		\immini{Tập hợp $A=(-3;3)$ có biểu diễn là}{ 
			\begin{tikzpicture}[scale=1, font=\footnotesize, line join = round, line cap = round, >=stealth]
				\draw[thick,->] (-2,0)node[below=6pt]{$-\infty$} -- (0,0) node[scale=1.5]{\bf ( } node[below=6pt]{$-3$} -- (3,0) node[scale=1.5]{\bf )} node[below=6pt]{$3$} -- (5,0)node[below=6pt]{$+\infty$};
				\foreach \i in {1,...,10} 
				\draw ($(0,0)-(.2*\i,0)$) node[scale=.6]{/};
				\draw ($(0,0)$) node[scale=.6]{/};
				\foreach \i in {1,...,9} 
				\draw ($(3,0)+(.2*\i,0)$) node[scale=.6]{/};
				\draw ($(3,0)$) node[scale=.6]{/};
		\end{tikzpicture}}
		\immini{Tập hợp $B=(0;+\infty)$ có biểu diễn là}{
			\begin{tikzpicture}[scale=1, font=\footnotesize, line join = round, line cap = round, >=stealth]
				\draw[thick,->] (-2,0)node[below=6pt]{$-\infty$}--(1,0) node[scale=1.5]{\bf (} node[below=6pt]{$0$}-- (5,0)node[below=6pt]{$+\infty$};
				\foreach \i in {1,...,15} 
				\draw ($(1,0)-(.2*\i,0)$) node[scale=.6,rotate=60]{/};
				\draw ($(1,0)$) node[scale=.6,rotate=60]{/};
				%	\foreach \i in {1,...,5} 
				%	\draw ($(4,0)+(.2*\i,0)$) node[scale=.6,rotate=60]{/};
		\end{tikzpicture}}
		\noindent Do đó $A \cup B=(-3;+\infty)$.
	}
\end{ex}
\begin{ex}%[Phan Quốc Trí, Bai Giảng T10(2022)]%[0D1B3-1]
	Cho tập hợp $X=(-\infty;2] \cap (-6;+\infty)$. Khẳng định nào sau đây là đúng?
	\choice
	{\True $X=(-6;2]$}
	{$(-6;+\infty)$}
	{$X=(-\infty;+\infty)$}
	{$X=(-\infty;2]$}
	\loigiai{
		Ta có: $X=(-\infty;2] \cap (-6;+\infty)=(-6;2]$.}
\end{ex}

\begin{ex}%[Phan Quốc Trí, Bai Giảng T10(2022)]%[0D1B3-2]
	Cho tập hợp $ A=[-2;3] $ và $ B=(1;5] $. Khi đó $ A\setminus B $ là
	\choice
	{$(-2;1]$}
	{$(-2;-1)$}
	{$[-2;1) $}
	{\True $[-2;1]$}
	\loigiai{
		Ta có $ A\setminus B =[-2;3]\setminus(1;5]= [-2;1]$. 
	}
\end{ex}

 \begin{ex}%[Phan Quốc Trí, Bai Giảng T10(2022)]%[0D1B3-2]
 	Cho tập hợp $A=\left\{x \in \mathbb{R} |  0 \le x+2 <5 \right\}$. Tập hợp $C_{\mathbb{R}}A$ bằng
 	\choice
 	{$(-\infty;-2)$}
 	{$(-\infty;-2] \cup (3;+\infty)$}
 	{\True $(-\infty;-2) \cup [3;+\infty)$}
 	{$[3;+\infty)$}
 	\loigiai{
 		Ta có: $C_{\mathbb{R}}A=(-\infty;-2) \cup [3;+\infty)$.}
 \end{ex}
\begin{ex}%[Phan Quốc Trí, Bai Giảng T10(2022)]%[0D1B3-3]
	Một lớp học có $25$ học sinh giỏi môn Toán, $23$ học sinh giỏi môn Lý, $14$ học sinh giỏi cả môn Toán và Lý và có $6$ học sinh không giỏi môn nào cả. Hỏi lớp đó có bao nhiêu học sinh?
	\choice
	{$26$}
	{$54$}
	{$68$}
	{\True $40$}
	\loigiai
	{
		Vì có $25$ học sinh giỏi môn Toán và $14$ học sinh giỏi cả môn Toán và Lý nên có $25-14=11$ học sinh chỉ giỏi môn Toán mà không giỏi môn Lý. \\
		Vì có $23$ học sinh giỏi môn Lý và $14$ học sinh giỏi cả môn Toán và Lý nên có $23-14=9$ học sinh chỉ giỏi môn Lý mà không giỏi môn Toán. \\
		Vậy lớp đó có $11+9+14+6=40$ học sinh.
	}
\end{ex}
\begin{ex}%[Phan Quốc Trí, Bai Giảng T10(2022)]%[0D1B3-3]
	Mỗi học sinh của lớp $10A$ đều chơi bóng đá hoặc bóng chuyền. Biết rằng có $25$ bạn chơi bóng đá, $20$ bạn chơi bóng chuyền và $10$ bạn chơi cả $2$ môn thể thao. Hỏi lớp $10A$ có bao nhiêu học sinh.
	\choice
	{$30$}
	{$55$}
	{$45$}
	{\True $35$}
	\loigiai{
		Ngoài sơ đồ Ven ta có thể dùng công thức số phần tử. Gọi $A$ là tập hợp các học sinh chơi bóng đá, $B$ là tập các học sinh chơi bóng chuyền. Do đó $A\cap B$ là tập các học sinh chơi cả hai môn. Ta có
		$$|A|=25, |B|=20, |A \cap B| =10.$$
		Số học sinh cả lớp là số phần tử của tập $A \cup B$. Theo công thức ta có $|A \cup B| = 25+20-10=35$ (học sinh).
	}
\end{ex}
\begin{ex}%[Phan Quốc Trí, Bai Giảng T10(2022)]%[0D1B3-1] 
	Cho các tập hợp $M=[-3;6]$ và $N=(-\infty; -2)\cup (3;+\infty)$. Khi đó $M\cap N$ là
	\choice
	{$(-\infty;-2)\cup (3;6)$}
	{$(-\infty;-2)\cup [3;+\infty)$}
	{\True $[-3;-2)\cup (3;6]$}
	{$(-3;-2)\cup (3;6)$}
	\loigiai{
		Biểu diễn trục số:  
		\begin{center}
			\begin{tikzpicture}
				\draw[->](-4,0)->(7,0);
				\IntervalLR{-4}{-3}
				\IntervalGR{}{}{\big[}{-3}
				\IntervalLR{6}{6.9}
				\IntervalGR{\big]}{6}{}{}
				\IntervalLR{-2}{3}
				\def\skipInterval{0.5cm}
				\IntervalGLF{\big)}{-2}{\big(}{3}
			\end{tikzpicture}
		\end{center}
		Từ hình vẽ, ta có
		Khi đó: $M\cap N= [-3;-2)\cup (3;6]$.}
\end{ex}

\begin{ex}%[Phan Quốc Trí, Bai Giảng T10(2022)]%%[Trần Ngọc Minh]%[301-320-Huỳnh Thanh Tiến]%[0D1B4-1]
	Tập hợp $(1;2)\cap\mathbb{N}$ là tập hợp nào sau đây?
	\choice
	{$\{1;2\}$}
	{$\{1\}$}
	{\True $\varnothing$}
	{$\{2\}$}
	\loigiai{
		Ta có $\mathbb{N}=\{0,1,2,\ldots\}$.
		Do đó 	$(1;2)\cap\mathbb{N}=\varnothing$.
	}
\end{ex}	
\begin{ex}%[Phan Quốc Trí, Bai Giảng T10(2022)]%[0D1B4-1]
	Cho $A=(-5;1]$, $B=[3;+\infty)$, $C=(-\infty;-2)$. Khẳng định nào sau đây đúng?
	\choice
	{$A\cap C=[-5;-2]$}
	{$A\cup B=(-5;+\infty)$}
	{$B\cup C=(-\infty;+\infty)$}
	{$\True B\cap C= \varnothing$}
	\loigiai{
		\begin{center}
			\begin{tikzpicture}
				\draw[->](-1,0)->(5,0);
				\IntervalLR{-1}{3}
				\def\skipInterval{0.5cm}
				\IntervalGRF{}{}{\big[}{3}
				\IntervalLR{4}{4.8}
				\def\skipInterval{0.5cm}
			\end{tikzpicture}\\
			\begin{tikzpicture}
				\draw[->](-1,0)->(5,0);
				\IntervalLR{-1}{1/2}
				\def\skipInterval{0.5cm}
				\IntervalLR{2}{4.9}
				\def\skipInterval{0.5cm}
				\IntervalGRF{\big)}{-2}{}{}
			\end{tikzpicture}
		\end{center}
		Từ biểu diễn tập nghiệm của $B$ và $C$ ta thấy $B\cap C= \varnothing$.
	}
\end{ex}
\begin{ex}%[Phan Quốc Trí, Bai Giảng T10(2022)]%%[0-HK2-2021, Trường THPT Trần Phú, Hải Phòng, năm học 2017 - 2018]%[Trần Quang Thạnh]%[0D1Y4-2]
	Cho tập hợp $A=[-2;3]$ và $B=(-2;5]$. Khi đó $A\setminus B$ là
	\choice
	{$[-2;5] $}
	{$(-2;-1) $}
	{$(3;5) $}
	{\True $\left\{-2\right\} $}
	\loigiai{
		Ta có $A\setminus B=\left\{-2\right\}$.
	}
\end{ex}
\begin{ex}%[Phan Quốc Trí, Bai Giảng T10(2022)]%[0D1B4-2]
	Cho các tập $A=\left\{x\in\mathbb{R}\mid x\ge -1\right\}$; $B=\left\{x\in\mathbb{R}\mid x<3\right\}$. Tập hợp $\mathbb{R}\setminus\left(A\cap B\right)$ là
	\choice
	{$[-1;3)$}
	{$(-\infty;-1]\cup(3;+\infty)$}
	{\True $(-\infty;-1)\cup[3;+\infty)$}
	{$(-1;3]$}
	\loigiai{
		Ta có $A\cap B=[-1;3)$, suy ra $\mathbb{R}\setminus\left(A\cap B\right)=(-\infty;-1)\cup[3;+\infty)$.
	}
\end{ex}
\begin{ex}%[Phan Quốc Trí, Bai Giảng T10(2022)]%[0D1K2-2]
	Tìm tất cả các giá trị của $m$ để đoạn $[m;m+3]$ là tập con của nửa khoảng $(-2;9]$.
	\choice
	{$-2\le m\le 6$}
	{$-2\le m<6$}
	{\True $-2<m\le 6$}
	{$-2<m<6$}
	\loigiai{
		Đoạn $[m;m+3]$ là tập con của nửa khoảng $(-2;9]$ khi và chỉ khi $\heva{& -2<m \\& m+3\le 9}\Leftrightarrow \heva{& -2<m \\& m\le 6}\Leftrightarrow -2<m\le 6.$
	}
\end{ex}



\noindent\textbf{II. PHẦN TỰ LUẬN}
\begin{bt}%[Phan Quốc Trí, Bai Giảng T10(2022)]%[0D1B4-1] 
	Cho $A=\left \{x \in \mathbb{R} \Big|  x \geq 3\right \}$ và $B=(-2 ; 7]$.  Tìm các tập hợp $A \cap B$, $A \cup B$.
	\loigiai{
		Ta có $A=\left \{x \in \mathbb{R} \Big|  x \geq 3\right \} = [3;+\infty )$. \\
		$A \cap B =[3;+\infty ) \cap (-2 ; 7] =    [3;7 ]$. \\
		$A \cup B = [3;+\infty ) \cup (-2 ; 7] = (-2;+\infty)$. 
	}
\end{bt}
\begin{bt}%[Phan Quốc Trí, Bai Giảng T10(2022)]%[0D1B3-2]
	Cho hai tập hợp $A=\left\{ 0;2 \right\}$ và $B=\left\{ 0;1;2;3;4 \right\}$. Tìm tất cả các tập hợp $X$ thỏa mãn $A\cup X=B$.
	\loigiai { 
		Vì $A\cup X=B$ nên $X$ chắc chắn có chứa các phần tử $1;3;4$\\
		Các tập $X$ có thể là $\left\{ 1;3;4 \right\},\,\left\{ 1;3;4;0 \right\},\,\left\{ 1;3;4;2 \right\},\,\left\{ 1;3;4;0;2 \right\}$}
\end{bt}
\begin{bt}%[Phan Quốc Trí, Bai Giảng T10(2022)]%[0D1K4-1]
	Cho hai tập hợp $A=(2m-1;m+3)$, $B=(-4;5)$. Tìm $m$ để
 $A\cap B=\varnothing $.
	\loigiai{
		Điều kiện: $2m-1<m+3 \Leftrightarrow m<4$.\\
	 Ta có  $A\cap B=\varnothing $ khi và chỉ khi $\left[
			\begin{aligned}
				&m+3\leqslant -4\\
				&2m-1\geqslant 5
			\end{aligned}\right. \Leftrightarrow \left[
			\begin{aligned}
				&m\leqslant -7\\
				&m\geqslant 3.
			\end{aligned}\right. $\\
			Đối chiếu điều kiện, ta được $m\leqslant -7$ hoặc $3\leqslant m<4$ thỏa yêu cầu bài toán.
	}
\end{bt}
\begin{bt}%[Phan Quốc Trí, Bai Giảng T10(2022)]%[0D1B3-3]
Lớp 10A có $45$ học sinh, trong đó có $18$ học sinh tham gia cuộc thi vẽ đồ họa trên máy tính, $24$ học sinh tham gia cuộc thi tin học văn phòng cấp trường và $9$ học sinh không tham gia cả hai cuộc thi này. Hỏi lớp 10A có bao nhiêu học sinh tham gia đồng thời cả hai cuộc thi?	
	\loigiai{
\immini{
Gọi $A$ là tập hợp các học sinh tham gia cuộc thi vẽ đồ họa trên máy tính. Suy ra $n(A)= 18$.\\
$B$ là tập hợp các học sinh tham gia  cuộc thi tin học văn phòng cấp trường. Suy ra $n(B)= 24$.\\
Ta có $A \cap B$ là tập hợp các học sinh tham gia đồng thời cả hai cuộc thi.\\
 $A \cup B$ là tập hợp các học sinh tham gia cuộc thi vẽ đồ họa trên máy tính hoặc tham gia  cuộc thi tin học văn phòng cấp trường. 
}{
	\begin{tikzpicture}
		\draw[] (0,0) circle ( 1.0cm);
		\draw (1.0,0) circle ( 1.0 cm);
		\draw (-0.5,0) node {$18$};
		\draw (1.5,0) node {$24$};
		\draw (-1.3,-0.75) node {$9$};
		\draw (1.5,1.5) node {$45$};
		\node[rectangle,
		draw = lightgray,
		minimum width = 4cm, 
		minimum height = 2.5cm] (r) at (0.5,0) {};
	\end{tikzpicture}
}
$$n \left(A \cup B \right)= 45-9=36.$$	
$$n \left( A \cap B \right) = n(A)+ n(B)-n(A \cup B)=18+24-36= 6.$$
Vậy có $6$  học sinh tham gia đồng thời cả hai cuộc thi.
	}
\end{bt}
\Closesolutionfile{ans}

\newpage
\begin{indapan}{10}
{ans/ans-KT-101}
\end{indapan}

\section*{Đề kiểm tra Chương 1}
\subsection*{Đề số 2}
\setcounter{ex}{0}\setcounter{bt}{0}
\Opensolutionfile{ans}[ans/ans-KT-102]
\noindent\textbf{I. PHẦN TRẮC NGHIỆM}
%[Thi thử, Sở GD và ĐT - Điện Biên, 2018]%[Dương BùiĐức, 12EX10]%[2D2Y3-2]
\begin{ex}%[Phan Quốc Trí, Bai Giảng T10(2022)]%[0D1Y1-1]
	Trong các câu sau, câu nào không phải là mệnh đề?
	\choice
	{Hình bình hành có bốn cạnh bằng nhau}
	{\True Chúc bạn may mắn}
	{Số $8$ là số chính phương}
	{Cà Mau là tên một tỉnh của nước Việt Nam}
	\loigiai{
		Mệnh đề là một câu khẳng định \textbf{đúng} hoặc một câu khẳng định \textbf{sai}.
	}
\end{ex}
\begin{ex}%[Phan Quốc Trí, Bai Giảng T10(2022)]%[0D1Y1-1]
	Trong các câu sau, câu nào là mệnh đề?
	\choice
	{$x^2+x=2$}
	{Hôm nay trời đẹp quá!}
	{$2n+1$ chia hết cho 3}
	{\True Số 15 là một số nguyên tố}
	\loigiai{
		\begin{itemize}
			\item $x^2+x=2$  là một khẳng định, nhưng không là mệnh đề. 
			\item  Hôm nay trời đẹp quá! không phải là mệnh đề.
			\item Câu $2n+1$ chia hết cho 3 là một khẳng định, nhưng không là mệnh đề. 
			\item  Số 15 là một số nguyên tố là một mệnh đề sai.
		\end{itemize}
	}
\end{ex}
\begin{ex}%[Phan Quốc Trí, Bai Giảng T10(2022)]%[0D1Y1-2]
	Trong các mệnh đề sau, mệnh đề nào là mệnh đề \textbf{sai}?
	\choice
	{\True $\sqrt{3}$ là số nguyên}
	{$6$ chia hết cho $2$}
	{$5$ chia hết cho $5$}
	{$30$ là một số chẵn}
	\loigiai{
		Trong các mệnh đề đã cho, mệnh đề sai là \lq\lq$\sqrt{3}$ là số nguyên\rq\rq.
	}
\end{ex}
\begin{ex}%[Phan Quốc Trí, Bai Giảng T10(2022)]%[0D1Y1-2]
	Cho mệnh đề chứa biến $P(x): 3x+5\le x^2$ với $x$ là số thực. Mệnh đề nào sau đây là đúng?
	\choice
	{$P(3)$}
	{$P(4)$}
	{$P(1)$}
	{\True $P(5)$}
	\loigiai{
		$P(3)= 3.3+5\leqslant 3^2 \Leftrightarrow 14\leqslant 9$ là mệnh đề sai.\\
		$P(4) =3.4+5\leqslant 4^2 \Leftrightarrow 17\leqslant 16$ là mệnh đề sai.\\
		$P(1)=3.1+5\leqslant 1^2 \Leftrightarrow 8\leqslant 1$ là mệnh đề sai.\\
		$P(5) =3.5+5\leqslant 5^2 \Leftrightarrow 20\leqslant 25$ là mệnh đề đúng.}
\end{ex}
\begin{ex}%[Phan Quốc Trí, Bai Giảng T10(2022)]%[0D1Y1-3]
	Cho mệnh đề $P\colon$ \lq\lq  $9$ là số chia hết cho $3$\rq\rq. Mệnh đề phủ định của mệnh đề $P$ là
	\choice
	{$\overline{P}\colon$\lq\lq  $9$ là ước của $3$\rq\rq}
	{$\overline{P}\colon$\lq\lq  $9$ là bội của $3$\rq\rq}
	{\True $\overline{P}\colon$ \lq\lq  $9$ là số không chia hết cho $3$\rq\rq}
	{$\overline{P}\colon$\lq\lq  $9$ là số lớn hơn $3$\rq\rq}
	\loigiai{Mệnh đề $P\colon$\lq\lq  $9$ là số chia hết cho $3$\rq\rq      có mệnh đề phủ định là $\overline{P}\colon$ \lq\lq  $9$ là số không chia hết cho $3$\rq\rq.}
\end{ex}
\begin{ex}%[Phan Quốc Trí, Bai Giảng T10(2022)]%[0D1Y1-3]
	Cho mệnh đề \lq\lq$\forall x\in \mathbb{R},\, x^2+1>0$\rq \rq. Mệnh đề phủ định của mệnh đề đã cho là
	\choice
	{\lq\lq$\forall x\in \mathbb{R},\, x^2+1\leq 0$\rq \rq}
	{\lq\lq$\forall x\in \mathbb{R},\, x^2+1<0$\rq \rq}
	{\True \lq\lq$\exists x\in \mathbb{R},\, x^2+1\leq 0$\rq \rq}
	{\lq\lq$\exists x\in \mathbb{R},\, x^2+1>0$\rq \rq}
	\loigiai{
		Mệnh đề phủ định của mệnh đề \lq\lq$\forall x\in \mathbb{R},\, x^2+1>0$\rq \rq\, là \lq\lq$\exists x\in \mathbb{R},\, x^2+1\leq 0$\rq \rq.}
\end{ex}
\begin{ex}%[Phan Quốc Trí, Bai Giảng T10(2022)]%[0D1Y1-4]
	Cho mệnh đề $P\colon$ ``Tam giác $ABC$ cân tại $A$'', mệnh đề $Q\colon$ ``$AB=AC$''. Phát biểu mệnh đề ``$P$ kéo theo $Q$'' là
	\choice
	{Nếu $AB=AC$ thì tam giác $ABC$ cân tại $A$}
	{\True Nếu tam giác $ABC$ cân tại $A$ thì $AB=AC$}
	{Tam giác $ABC$ cân tại $B$ là điều kiện cần và đủ để $AB=AC$}
	{Tam giác $ABC$ cân tại $A$ khi và chỉ khi $AB=AC$}
	\loigiai{
		Mệnh đề ``$P$ kéo theo $Q$'' là ``Nếu tam giác $ABC$ cân tại $A$ thì $AB=AC$''.
	}
\end{ex}
\begin{ex}%[Phan Quốc Trí, Bai Giảng T10(2022)]%[0D1Y1-4]
	Cho mệnh đề $P$: \lq\lq Nếu tam giác có hai đường trung tuyến bằng nhau thì đó là tam giác cân\rq\rq. Mệnh đề nào sau đây là mệnh đề đảo của $P$?
	\choice
	{Tam giác có hai đường trung tuyến bằng nhau thì nó là tam giác cân}
	{\True Nếu tam giác $ABC$ cân thì tam giác đó có hai đường trung tuyến bằng nhau }
	{Tam giác là tam giác cân khi và chỉ khi nó có hai đường trung tuyến bằng nhau}
	{Tam giác là tam giác cân nếu nó có hai đường trung tuyến bằng nhau}
	\loigiai{
		Mệnh đề đảo là \lq \lq Nếu tam giác $ABC$ cân thì tam giác đó có hai đường trung tuyến bằng nhau.\rq \rq		
	}
\end{ex}
\begin{ex}%[Phan Quốc Trí, Bai Giảng T10(2022)]%[0D1Y1-5]
	Mệnh đề "Bình phương mọi số thực đều không âm" mô tả mệnh đề nào dưới đây?
	\choice
	{"$\forall n\in\mathbb{N}: n^2\geq 0$"}
	{"$\exists x\in\mathbb{R}:x^2\geq 0$"}
	{\True "$\forall x\in\mathbb{R}: x^2\geq 0$"}
	{"$\forall x\in\mathbb{R}:x^2>0$"}
	\loigiai{
		Mệnh đề trên được viết lại dạng: \lq \lq $\forall x\in\mathbb{R}: x^2\geq 0$\rq \rq	
	}
\end{ex}
\begin{ex}%[Phan Quốc Trí, Bai Giảng T10(2022)]%[0D1B1-5]
	Mệnh đề nào sau đây là đúng?
	\choice
	{$\forall n\in\mathbb{N}\colon n^2>n$}
	{$\forall x\in\mathbb{R}\colon x^2<2$}
	{$\forall x\in\mathbb{Z}\colon 2x>1$}
	{\True $\exists x\in\mathbb{R}\colon x^2>x$}
	\loigiai
	{
		\begin{itemize}
			\item Mệnh đề ``$\forall n\in\mathbb{N}\colon n^2>n$'' sai vì với $n=1$ thì $n^2=1=n$.
			\item Mệnh đề ``$\forall x\in\mathbb{R}\colon x^2<2$'' sai vì với $x=2$ thì $x^2=4>2$.
			\item Mệnh đề ``$\forall x\in\mathbb{Z}\colon 2x>1$'' sai vì với $x=0$ thì $2x=0<1$.
			\item Mệnh đề ``$\exists x\in\mathbb{R}\colon x^2>x$'' đúng vì với $x=2$ thì $x^2=4$ nên $x^2>x$.
		\end{itemize}
	}
\end{ex}
\begin{ex}%[Phan Quốc Trí, Bai Giảng T10(2022)]%[0D1Y2-1]
	Hãy liệt kê các phần tử của tập hợp $X = \left\{ x \in \mathbb{Z} | 2x^2-5x+3=0 \right\}$.
	\choice
	{$X = \left\{1;\dfrac{3}{2} \right\}$}
	{\True $X = \{ 1 \}$}
	{$X = \left\{ \dfrac{3}{2} \right\}$}
	{$X = \varnothing$}
	\loigiai{
		Ta có $2x^2-5x+3=0\Leftrightarrow \hoac{&x=1\in \mathbb{Z}\\&x=\dfrac{3}{2}\notin \mathbb{Z}}$.\\
		Vậy $X = \{ 1 \}$.		
	}
\end{ex}
\begin{ex}%[Phan Quốc Trí, Bai Giảng T10(2022)]%[0D1B2-1]
	Viết tập hợp $A=\lbrace x \in \mathbb{Z}|x^2<17 \rbrace $ theo cách liệt kê các phần tử, ta được tập hợp nào sau đây?
	\choice{\True $ \lbrace -4;-3;-2;-1;0;1;2;3;4 \rbrace $}
	{$ \lbrace 1;2;3;4 \rbrace $}
	{$\lbrace 0;1;2;3;4  \rbrace $}
	{$ \lbrace -4; -3;-2;-1 \rbrace $}
	\loigiai{Ta có $x^2<17\Leftrightarrow \left|x\right|<\sqrt{17}\Leftrightarrow -\sqrt{17}<x<\sqrt{17}$.\\
		Vì $x\in \mathbb{Z}$ nên $A=\lbrace -4;-3;-2;-1;0;1;2;3;4 \rbrace $.
	}
\end{ex}

\begin{ex}%[Phan Quốc Trí, Bai Giảng T10(2022)]%[0D1Y2-1]
	Cho tập hợp $A=\left\{ x \in \mathbb{N}| x^2+2x-3=0 \right\}$. Mệnh đề nào sau đây là đúng?
	\choice
	{\True $-3 \notin A$}
	{$A=\left\{1;-3\right\}$}
	{$1 \notin A$}
	{$A=\left\{1;3\right\}$}
	\loigiai{
		Ta có $x^2+2x-3=0 \Leftrightarrow \hoac{&x=1\\&x=-3}$. Do $x \in \mathbb{N}$ nên $A=\left\{1\right\}$. Do đó $-3 \notin A$.
	}
\end{ex}
\begin{ex}%[Phan Quốc Trí, Bai Giảng T10(2022)]%[0D1Y2-2]
	Cho tập $A=\{a;b;5\}$. Số tập con của tập $A$ là
	\choice
	{$5$}
	{\True $8$}
	{$7$}
	{$4$}
	\loigiai{Tập con của $A$ là $\varnothing, \{a\}, \{b\},\{5\}, \{a;b\}, \{a;5\}, \{b;5\}, \{a;b;5\} $. Vậy số tập con của $A$ là $8$.
	}
\end{ex}
\begin{ex}%[Phan Quốc Trí, Bai Giảng T10(2022)]%[0D1Y2-2]
	Có bao nhiêu tập $ X $ thỏa mãn $ \{a;b\} \subset X \subset \{1;2;a;b\}$?
	\choice
	{$3$}
	{$2$}
	{\True $4$}
	{$5$}
	\loigiai{
		Các tập $ X $ thỏa mãn là $ \{a;b\} $, $ \{1;a;b\} $, $ \{2;a;b\} $, $ \{1;2;a;b\} $.
	}
\end{ex}
\begin{ex}%[Phan Quốc Trí, Bai Giảng T10(2022)]%[0D1B4-1]
	Cho tập hợp $A=\{x\in\mathbb{R}|-3<x\le 3\}$. Mệnh đề nào dưới đây đúng?
	\choice
	{$A=\{-2;-1;0;1;2;3\}$}
	{\True $A=(-3;3]$}
	{$A=[-3;3]$}
	{$A=[-3;3)$}
	\loigiai{
		Từ giả thiết, có $A=(-3;3]$.
	}
\end{ex}
\begin{ex}%[Phan Quốc Trí, Bai Giảng T10(2022)]%[0D1B2-2]
	Cho tập hợp $A = \left\{x \in \mathbb{N}\big| x^2 + 8x + 15 = 0\right\}$. Khẳng định nào sau đây đúng?
	\choice
	{$A = \left\{-3;-5\right\}$}
	{\True $A =\varnothing$}
	{$A = \left\{\varnothing\right\}$}
	{$A = \left\{0\right\}$}
	\loigiai{
		Phương trình $x^2+8x+15=0$ có hai nghiệm $x_1=-3$, $x_2=-5$. Tuy nhiên $x_1, x_2\notin \mathbb{N}$. Vậy $A=\varnothing$.
	}
\end{ex}
\begin{ex}%[Phan Quốc Trí, Bai Giảng T10(2022)]%[0D1B2-2]
	Gọi $A$ là tập hợp tất cả các hình bình hành và $B$ là tập hợp tất cả các hình chữ nhật. Trong các kết luận sau, kết luận nào đúng?
	\choice
	{$A \subset B$}
	{\True $B \subset A$}
	{$A=B$}
	{$A \cap B=\varnothing$}
	\loigiai{
		Ta có hình chữ nhật là hình bình hành có một góc vuông nên $B \subset A$.
	}
\end{ex}

\begin{ex}%[Phan Quốc Trí, Bai Giảng T10(2022)]%[0D1Y2-2]
	Khẳng định nào sau đây là đúng?
	\choice
	{$ \mathbb{R}\subset\mathbb{Q} $}
	{$ \mathbb{Z}\subset\mathbb{N} $}
	{$ \mathbb{Q}\subset\mathbb{Z} $}
	{\True $ \mathbb{N}\subset\mathbb{R} $}
	\loigiai{
		Ta có $ \mathbb{N}\subset\mathbb{Z}\subset\mathbb{Q}\subset\mathbb{R} $.
	}
\end{ex}
\begin{ex}%[Phan Quốc Trí, Bai Giảng T10(2022)]%[0D1Y3-1]
	Cho hai tập hợp $X=\left\{1;3;5;8\right\}$ và $Y=\left\{3;5;7;9\right\}$. Tập hợp $X\cup Y$ bằng
	\choice
	{$\left\{1;7;9\right\}$}
	{$\left\{3;5\right\}$}
	{$\left\{1;3;5\right\}$}
	{\True $\left\{1;3;5;7;8;9\right\}$}
	\loigiai{
		Ta có $X\cup Y=\left\{1;3;5;7;8;9\right\}$.
	}
\end{ex}
\begin{ex}%[Phan Quốc Trí, Bai Giảng T10(2022)]%[0D1Y3-1]
	Cho $A=\{2;3;6;7\}, B=\{3;6;8\}$. Tập hợp $A\cap B$ bằng
	\choice
	{$\{3;6;8\}$}
	{$\{2;3;6;7;8\}$}
	{\True $\{3;6\}$}
	{$\{2;7\}$}
	\loigiai{
		Ta có $A\cap B=\{3;6\}.$
	}
\end{ex}
\begin{ex}%[Phan Quốc Trí, Bai Giảng T10(2022)]%[0D1Y3-2]
	Cho hai tập hợp $A=\{2;4;6;9\}$, $B=\{1;2;3;4\}$. Tập $A \setminus B$ bằng tập hợp nào sau đây?
	\choice
	{$\{2;4\}$}
	{$\{1;3\}$}
	{\True $\{6;9\}$}
	{$\{6;9;1;3\}$}
	\loigiai{
		Ta có: $A \setminus B=\{6;9\}$.}
\end{ex}

\begin{ex}%[Phan Quốc Trí, Bai Giảng T10(2022)]%[0D1Y3-2] 
	Cho tập $ X=\left\lbrace 0;1;2;3;4;5 \right\rbrace$ và tập $ A=\left\lbrace 0;2;4 \right\rbrace  $. Tìm phần bù của $ A $ trong $ X $.	
	\choice
	{$ \varnothing $}
	{$ \left\lbrace 2;4 \right\rbrace $ }
	{$ \left\lbrace 0;1;3 \right\rbrace  $}
	{\True $ \left\lbrace 1;3;5 \right\rbrace  $}
	\loigiai
	{
		Ta có phần bù của $ A $ trong $ X $ bằng tập $ X\setminus A=\left\lbrace 1;3;5 \right\rbrace  $.	
	}
\end{ex}
\begin{ex}%[Phan Quốc Trí, Bai Giảng T10(2022)]%[0D1B3-1]
	Cho hai tập hợp $A=(-3;3)$ và $B=(0;+\infty)$. Tìm  $A \cup B$.
	\choice
	{\True $ A \cup B = (-3;+\infty) $}
	{$ A \cup B = [-3;+\infty) $}
	{$ A \cup B = [-3;0) $}
	{$ A \cup B = (0;3) $}
	\loigiai{
		\immini{Tập hợp $A=(-3;3)$ có biểu diễn là}{ 
			\begin{tikzpicture}[scale=1, font=\footnotesize, line join = round, line cap = round, >=stealth]
				\draw[thick,->] (-2,0)node[below=6pt]{$-\infty$} -- (0,0) node[scale=1.5]{\bf ( } node[below=6pt]{$-3$} -- (3,0) node[scale=1.5]{\bf )} node[below=6pt]{$3$} -- (5,0)node[below=6pt]{$+\infty$};
				\foreach \i in {1,...,10} 
				\draw ($(0,0)-(.2*\i,0)$) node[scale=.6]{/};
				\draw ($(0,0)$) node[scale=.6]{/};
				\foreach \i in {1,...,9} 
				\draw ($(3,0)+(.2*\i,0)$) node[scale=.6]{/};
				\draw ($(3,0)$) node[scale=.6]{/};
		\end{tikzpicture}}
		\immini{Tập hợp $B=(0;+\infty)$ có biểu diễn là}{
			\begin{tikzpicture}[scale=1, font=\footnotesize, line join = round, line cap = round, >=stealth]
				\draw[thick,->] (-2,0)node[below=6pt]{$-\infty$}--(1,0) node[scale=1.5]{\bf (} node[below=6pt]{$0$}-- (5,0)node[below=6pt]{$+\infty$};
				\foreach \i in {1,...,15} 
				\draw ($(1,0)-(.2*\i,0)$) node[scale=.6,rotate=60]{/};
				\draw ($(1,0)$) node[scale=.6,rotate=60]{/};
				%	\foreach \i in {1,...,5} 
				%	\draw ($(4,0)+(.2*\i,0)$) node[scale=.6,rotate=60]{/};
		\end{tikzpicture}}
		\noindent Do đó $A \cup B=(-3;+\infty)$.
	}
\end{ex}
\begin{ex}%[Phan Quốc Trí, Bai Giảng T10(2022)]%[0D1B3-1]
	Cho tập hợp $X=(-\infty;2] \cap (-6;+\infty)$. Khẳng định nào sau đây là đúng?
	\choice
	{\True $X=(-6;2]$}
	{$(-6;+\infty)$}
	{$X=(-\infty;+\infty)$}
	{$X=(-\infty;2]$}
	\loigiai{
		Ta có: $X=(-\infty;2] \cap (-6;+\infty)=(-6;2]$.}
\end{ex}

\begin{ex}%[Phan Quốc Trí, Bai Giảng T10(2022)]%[0D1B3-2]
	Cho tập hợp $ A=[-2;3] $ và $ B=(1;5] $. Khi đó $ A\setminus B $ là
	\choice
	{$(-2;1]$}
	{$(-2;-1)$}
	{$[-2;1) $}
	{\True $[-2;1]$}
	\loigiai{
		Ta có $ A\setminus B =[-2;3]\setminus(1;5]= [-2;1]$. 
	}
\end{ex}

\begin{ex}%[Phan Quốc Trí, Bai Giảng T10(2022)]%[0D1B3-2]
	Cho tập hợp $A=\left\{x \in \mathbb{R} |  0 \le x+2 <5 \right\}$. Tập hợp $C_{\mathbb{R}}A$ bằng
	\choice
	{$(-\infty;-2)$}
	{$(-\infty;-2] \cup (3;+\infty)$}
	{\True $(-\infty;-2) \cup [3;+\infty)$}
	{$[3;+\infty)$}
	\loigiai{
		Ta có: $C_{\mathbb{R}}A=(-\infty;-2) \cup [3;+\infty)$.}
\end{ex}
\begin{ex}%[Phan Quốc Trí, Bai Giảng T10(2022)]%[0D1B3-3]
	Một lớp học có $25$ học sinh giỏi môn Toán, $23$ học sinh giỏi môn Lý, $14$ học sinh giỏi cả môn Toán và Lý và có $6$ học sinh không giỏi môn nào cả. Hỏi lớp đó có bao nhiêu học sinh?
	\choice
	{$26$}
	{$54$}
	{$68$}
	{\True $40$}
	\loigiai
	{
		Vì có $25$ học sinh giỏi môn Toán và $14$ học sinh giỏi cả môn Toán và Lý nên có $25-14=11$ học sinh chỉ giỏi môn Toán mà không giỏi môn Lý. \\
		Vì có $23$ học sinh giỏi môn Lý và $14$ học sinh giỏi cả môn Toán và Lý nên có $23-14=9$ học sinh chỉ giỏi môn Lý mà không giỏi môn Toán. \\
		Vậy lớp đó có $11+9+14+6=40$ học sinh.
	}
\end{ex}
\begin{ex}%[Phan Quốc Trí, Bai Giảng T10(2022)]%[0D1B3-3]
	Mỗi học sinh của lớp $10A$ đều chơi bóng đá hoặc bóng chuyền. Biết rằng có $25$ bạn chơi bóng đá, $20$ bạn chơi bóng chuyền và $10$ bạn chơi cả $2$ môn thể thao. Hỏi lớp $10A$ có bao nhiêu học sinh.
	\choice
	{$30$}
	{$55$}
	{$45$}
	{\True $35$}
	\loigiai{
		Ngoài sơ đồ Ven ta có thể dùng công thức số phần tử. Gọi $A$ là tập hợp các học sinh chơi bóng đá, $B$ là tập các học sinh chơi bóng chuyền. Do đó $A\cap B$ là tập các học sinh chơi cả hai môn. Ta có
		$$|A|=25, |B|=20, |A \cap B| =10.$$
		Số học sinh cả lớp là số phần tử của tập $A \cup B$. Theo công thức ta có $|A \cup B| = 25+20-10=35$ (học sinh).
	}
\end{ex}
\begin{ex}%[Phan Quốc Trí, Bai Giảng T10(2022)]%[0D1B3-1] 
	Cho các tập hợp $M=[-3;6]$ và $N=(-\infty; -2)\cup (3;+\infty)$. Khi đó $M\cap N$ là
	\choice
	{$(-\infty;-2)\cup (3;6)$}
	{$(-\infty;-2)\cup [3;+\infty)$}
	{\True $[-3;-2)\cup (3;6]$}
	{$(-3;-2)\cup (3;6)$}
	\loigiai{
		Biểu diễn trục số:  
		\begin{center}
			\begin{tikzpicture}
				\draw[->](-4,0)->(7,0);
				\IntervalLR{-4}{-3}
				\IntervalGR{}{}{\big[}{-3}
				\IntervalLR{6}{6.9}
				\IntervalGR{\big]}{6}{}{}
				\IntervalLR{-2}{3}
				\def\skipInterval{0.5cm}
				\IntervalGLF{\big)}{-2}{\big(}{3}
			\end{tikzpicture}
		\end{center}
		Từ hình vẽ, ta có
		Khi đó: $M\cap N= [-3;-2)\cup (3;6]$.}
\end{ex}


\begin{ex}%[Phan Quốc Trí, Bai Giảng T10(2022)]%%[Trần Ngọc Minh]%[301-320-Huỳnh Thanh Tiến]%[0D1B4-1]
	Tập hợp $(1;2)\cap\mathbb{N}$ là tập hợp nào sau đây?
	\choice
	{$\{1;2\}$}
	{$\{1\}$}
	{\True $\varnothing$}
	{$\{2\}$}
	\loigiai{
		Ta có $\mathbb{N}=\{0,1,2,\ldots\}$.
		Do đó 	$(1;2)\cap\mathbb{N}=\varnothing$.
	}
\end{ex}	
\begin{ex}%[Phan Quốc Trí, Bai Giảng T10(2022)]%[0D1B4-1]
	Cho $A=(-5;1]$, $B=[3;+\infty)$, $C=(-\infty;-2)$. Khẳng định nào sau đây đúng?
	\choice
	{$A\cap C=[-5;-2]$}
	{$A\cup B=(-5;+\infty)$}
	{$B\cup C=(-\infty;+\infty)$}
	{$\True B\cap C= \varnothing$}
	\loigiai{
		\begin{center}
			\begin{tikzpicture}
				\draw[->](-1,0)->(5,0);
				\IntervalLR{-1}{3}
				\def\skipInterval{0.5cm}
				\IntervalGRF{}{}{\big[}{3}
				\IntervalLR{4}{4.8}
				\def\skipInterval{0.5cm}
			\end{tikzpicture}\\
			\begin{tikzpicture}
				\draw[->](-1,0)->(5,0);
				\IntervalLR{-1}{1/2}
				\def\skipInterval{0.5cm}
				\IntervalLR{2}{4.9}
				\def\skipInterval{0.5cm}
				\IntervalGRF{\big)}{-2}{}{}
			\end{tikzpicture}
		\end{center}
		Từ biểu diễn tập nghiệm của $B$ và $C$ ta thấy $B\cap C= \varnothing$.
	}
\end{ex}
\begin{ex}%[Phan Quốc Trí, Bai Giảng T10(2022)]%%[0-HK2-2021, Trường THPT Trần Phú, Hải Phòng, năm học 2017 - 2018]%[Trần Quang Thạnh]%[0D1Y4-2]
	Cho tập hợp $A=[-2;3]$ và $B=(-2;5]$. Khi đó $A\setminus B$ là
	\choice
	{$[-2;5] $}
	{$(-2;-1) $}
	{$(3;5) $}
	{\True $\left\{-2\right\} $}
	\loigiai{
		Ta có $A\setminus B=\left\{-2\right\}$.
	}
\end{ex}
\begin{ex}%[Phan Quốc Trí, Bai Giảng T10(2022)]%[0D1B4-2]
	Cho các tập $A=\left\{x\in\mathbb{R}\mid x\ge -1\right\}$; $B=\left\{x\in\mathbb{R}\mid x<3\right\}$. Tập hợp $\mathbb{R}\setminus\left(A\cap B\right)$ là
	\choice
	{$[-1;3)$}
	{$(-\infty;-1]\cup(3;+\infty)$}
	{\True $(-\infty;-1)\cup[3;+\infty)$}
	{$(-1;3]$}
	\loigiai{
		Ta có $A\cap B=[-1;3)$, suy ra $\mathbb{R}\setminus\left(A\cap B\right)=(-\infty;-1)\cup[3;+\infty)$.
	}
\end{ex}
\begin{ex}%[Phan Quốc Trí, Bai Giảng T10(2022)]%[0D1B2-2]
	Tìm tất cả các giá trị của $m$ để đoạn $[m;m+3]$ là tập con của nửa khoảng $(-2;9]$.
	\choice
	{$-2\le m\le 6$}
	{$-2\le m<6$}
	{\True $-2<m\le 6$}
	{$-2<m<6$}
	\loigiai{
		Đoạn $[m;m+3]$ là tập con của nửa khoảng $(-2;9]$ khi và chỉ khi $\heva{& -2<m \\& m+3\le 9}\Leftrightarrow \heva{& -2<m \\& m\le 6}\Leftrightarrow -2<m\le 6.$
	}
\end{ex}



\noindent\textbf{II. PHẦN TỰ LUẬN}
\begin{bt}%[Phan Quốc Trí, Bai Giảng T10(2022)]%[0D1B4-1] 
	Cho $A=\left \{x \in \mathbb{R} \Big|  x \geq 3\right \} ; B=(-2 ; 7]$.  Tìm các tập hợp $A \cap B, A \cup B$.
	\loigiai{
		Ta có $A=\left \{x \in \mathbb{R} \Big|  x \geq 3\right \} = [3;+\infty )$. \\
		$A \cap B =[3;+\infty ) \cap (-2 ; 7] =    [3;7 ]$. \\
		$A \cup B = [3;+\infty ) \cup (-2 ; 7] = (-2;+\infty)$. 
	}
\end{bt}
\begin{bt}%[Phan Quốc Trí, Bai Giảng T10(2022)]%[0D1B3-2]
	Cho hai tập hợp $A=\left\{ 0;2 \right\}$ và $B=\left\{ 0;1;2;3;4 \right\}$. Tìm tất cả các tập hợp $X$ thỏa mãn $A\cup X=B$.
	\loigiai { 
		Vì $A\cup X=B$ nên $X$ chắc chắn có chứa các phần tử $1;3;4$\\
		Các tập $X$ có thể là $\left\{ 1;3;4 \right\},\,\left\{ 1;3;4;0 \right\},\,\left\{ 1;3;4;2 \right\},\,\left\{ 1;3;4;0;2 \right\}$}
\end{bt}
\begin{bt}%[Phan Quốc Trí, Bai Giảng T10(2022)]%[0D1B4-1]
	Cho hai tập hợp $A=(2m-1;m+3)$, $B=(-4;5)$. Tìm $m$ để
	$A\cap B=\varnothing $.
	\loigiai{
		Điều kiện: $2m-1<m+3 \Leftrightarrow m<4$.
		Để $A\cap B=\varnothing $ khi và chỉ khi $\left[
		\begin{aligned}
			&m+3\leqslant -4\\
			&2m-1\geqslant 5
		\end{aligned}\right. \Leftrightarrow \left[
		\begin{aligned}
			&m\leqslant -7\\
			&m\geqslant 3
		\end{aligned}\right. $.\\
		Đối chiếu điều kiện, ta được $m\leqslant -7$ hoặc $3\leqslant m<4$ thỏa yêu cầu bài toán.
	}
\end{bt}
\begin{bt}%[Phan Quốc Trí, Bai Giảng T10(2022)]%[0D1B3-3]
	Lớp 10A có $45$ học sinh, trong đó có $18$ học sinh tham gia cuộc thi vẽ đồ họa trên máy tính, $24$ học sinh tham gia cuộc thi tin học văn phòng cấp trường và $9$ học sinh không tham gia cả hai cuộc thi này. Hỏi lớp 10A có bao nhiêu học sinh tham gia đồng thời cả hai cuộc thi?	
	\loigiai{
		\immini{
			Gọi $A$ là tập hợp các học sinh tham gia cuộc thi vẽ đồ họa trên máy tính. Suy ra $n(A)= 18$.\\
			$B$ là tập hợp các học sinh tham gia  cuộc thi tin học văn phòng cấp trường. Suy ra $n(B)= 24$.\\
			Ta có $A \cap B$ là tập hợp các học sinh tham gia đồng thời cả hai cuộc thi.\\
			$A \cup B$ là tập hợp các học sinh tham gia cuộc thi vẽ đồ họa trên máy tính hoặc tham gia  cuộc thi tin học văn phòng cấp trường. 
		}{
			\begin{tikzpicture}
				\draw[] (0,0) circle ( 1.0cm);
				\draw (1.0,0) circle ( 1.0 cm);
				\draw (-0.5,0) node {$18$};
				\draw (1.5,0) node {$24$};
				\draw (-1.3,-0.75) node {$9$};
				\draw (1.5,1.5) node {$45$};
				\node[rectangle,
				draw = lightgray,
				minimum width = 4cm, 
				minimum height = 2.5cm] (r) at (0.5,0) {};
			\end{tikzpicture}
		}
		$$n \left(A \cup B \right)= 45-9=36.$$	
		$$n \left( A \cap B \right) = n(A)+ n(B)-n(A \cup B)=18+24-36= 6.$$
		Vậy có $6$  học sinh tham gia đồng thời cả hai cuộc thi.
	}
\end{bt}

\Closesolutionfile{ans}

\newpage
\begin{indapan}{10}
	{ans/ans-KT-102}
\end{indapan}




\section*{Đề kiểm tra Chương 1}
\subsection*{Đề số 3}
\setcounter{ex}{0}\setcounter{bt}{0}
\Opensolutionfile{ans}[ans/ans-KT-103]
\noindent\textbf{I. PHẦN TRẮC NGHIỆM}
%[Thi thử, Sở GD và ĐT - Điện Biên, 2018]%[Dương BùiĐức, 12EX10]%[2D2Y3-2]
\begin{ex}%[Phan Quốc Trí, Bai Giảng T10(2022)]%[0D1Y1-1]
	Câu nào trong các câu sau \textbf{không phải} là mệnh đề?
	\choice
	{\True $\pi$ có phải là một số vô tỷ không?}
	{$2+2=5$}
	{$\sqrt{2}$ là một số hữu tỷ}
	{$\dfrac{4}{2}=2$}
	\loigiai{
		\lq\lq $\pi$ có phải là một số vô tỷ không?\rq\rq\, là câu hỏi, nên không phải là mệnh đề.
	}
\end{ex}
\begin{ex}%[Phan Quốc Trí, Bai Giảng T10(2022)]%[0D1Y1-1]
	Phát biểu nào sau đây là một mệnh đề?
	\choice
	{Mùa thu Hà Nội đep quá!}
	{\True Hà Nội là thủ đô của Việt Nam}
	{Bạn có đi học không?}
	{Đề thi môn Toán khó quá!}
	\loigiai{
		Hà Nội là thủ đô của Việt Nam là một khẳng định đúng.
	}
\end{ex}
\begin{ex}%[Phan Quốc Trí, Bai Giảng T10(2022)]%[0D1Y1-2]
	Trong các mệnh đề sau, mệnh đề nào \textbf{sai}?
	\choice
	{$-2 \in \mathbb{Q}$}
	{$\sqrt{2} \in \mathbb{R}$}
	{\True $\dfrac{1}{2} \in \mathbb{Z}$}
	{$2 \in \mathbb{N}$}
	\loigiai{
		Mệnh đề sai là  $\dfrac{1}{2} \in \mathbb{Z}$.
	}
\end{ex}
\begin{ex}%[Phan Quốc Trí, Bai Giảng T10(2022)]%[0D1Y1-2]
	Cho mệnh đề chứa biến $P(n) \colon$ "$n^2-1$ chia hết cho $4$" với $n$ là số nguyên. Khẳng định nào sau đây đúng?
	\choice
	{$P(5)$ đúng và $P(2)$ đúng}
	{\True $P(5)$ đúng và $P(2)$ sai}
	{$P(5)$ sai và $P(2)$ sai}
	{$P(5)$ sai và $P(2)$ đúng}
	\loigiai{
		Ta có:\\
		$P(5) \colon$ "$5^2-1$ chia hết cho $4$" tức là "$24$ chia hết cho $4$", nên $P(5)$ đúng.\\
		$P(2) \colon$ "$2^2-1$ chia hết cho $4$" tức là "$3$ chia hết cho $4$", nên $P(2)$ sai.\\
		Vậy $P(5)$ đúng và $P(2)$ đúng.}
\end{ex}
\begin{ex}%[Phan Quốc Trí, Bai Giảng T10(2022)]%[0D1Y1-3]
	Phủ định của mệnh đề: ``$\pi$ là số vô tỷ'' là
	\choice
	{$\pi$ là số nguyên}
	{$\pi$ là số dương}
	{$\pi$ là số thực}
	{\True $\pi$ không phải là số vô tỷ}	
	\loigiai{Mệnh đề phủ định là ``$\pi$ không phải là số vô tỷ''.}
\end{ex}
\begin{ex}%[Phan Quốc Trí, Bai Giảng T10(2022)]%[0D1Y1-3]
	Mệnh đề phủ định của mệnh đề \lq \lq $\exists x\in \mathbb{N}\colon x^2-4=0$\rq \rq \ là
	\choice
	{\lq \lq $\forall x\in \mathbb{N}\colon x^2-4=0$\rq \rq }
	{\lq \lq $\forall x\in \mathbb{N}\colon x^2-4>0$\rq \rq }
	{\lq \lq $\exists x\in \mathbb{N}\colon x^2-4\ne 0$\rq \rq }
	{\True \lq \lq $\forall x\in \mathbb{N}\colon x^2-4\ne 0$\rq \rq}
	\loigiai{
		Ta có mệnh đề phủ định của mệnh đề \lq\lq $\exists x \in X, P(x)$\rq\rq\ là \lq\lq $\forall x \in X, \overline{P(x)}$\rq\rq.\\
		Nên mệnh đề phủ định của mệnh đề \lq \lq $\exists x\in \mathbb{N}\colon x^2-4=0$\rq \rq \ là \lq \lq $\forall x\in \mathbb{N}\colon x^2-4\ne 0$\rq \rq.
	}
\end{ex}
\begin{ex}%[Phan Quốc Trí, Bai Giảng T10(2022)]%[0D1Y1-4]
	Cho mệnh đề $P\colon$ ``$b^2\ge 4ac$'', mệnh đề $Q\colon$ ``Phương trình $ax^2 +bx+c=0$ vô nghiệm'' (với $a,b,c$ là số thực và $a\ne 0$). Phát biểu mệnh đề ``$P$ kéo theo $Q$'' là
	\choice
	{Nếu Phương trình $ax^2 +bx+c=0$ vô nghiệm thì $b^2\ge 4ac$ (với $a,b,c$ là số thực và $a\ne 0$)}
	{\True Nếu $b^2\ge 4ac$ thì phương trình $ax^2 +bx+c=0$ vô nghiệm (với $a,b,c$ là số thực và $a\ne 0$) }
	{$b^2\ge 4ac$ là điều kiện cần và đủ để phương trình $ax^2 +bx+c=0$ vô nghiệm (với $a,b,c$ là số thực và $a\ne 0$)}
	{$b^2\ge 4ac$ khi và chỉ khi phương trình $ax^2 +bx+c=0$ vô nghiệm (với $a,b,c$ là số thực và $a\ne 0$)}
	\loigiai{
		Mệnh đề ``$P$ kéo theo $Q$'' là ``Nếu $b^2\ge 4ac$ thì phương trình $ax^2 +bx+c=0$ vô nghiệm (với $a,b,c$ là số thực và $a\ne 0$)''.
	}
\end{ex}
\begin{ex}%[Phan Quốc Trí, Bai Giảng T10(2022)]%[0D1Y1-4]
	Cho mệnh đề $P$: \lq\lq Nếu tam giác $ABC$ có hai góc bằng $60^{\circ}$ thì đó là tam giác$ABC$ đều\rq\rq. Mệnh đề nào sau đây là mệnh đề đảo của $P$?
	\choice
	{Tam giác $ABC$ có hai góc bằng $60^{\circ}$ thì đó là tam giác$ABC$ đều}
	{\True Nếu tam giác $ABC$ đều thì tam giác đó có hai góc bằng $60^{\circ}$}
	{Tam giác $ABC$ đều khi và chỉ khi tam giác đó có hai góc bằng $60^{\circ}$}
	{Tam giác$ABC$ đều nếu nó có hai góc bằng $60^{\circ}$}
	\loigiai{
		Mệnh đề đảo là \lq \lq Nếu tam giác $ABC$ đều thì tam giác đó có hai góc bằng $60^{\circ}$.\rq \rq		
	}
\end{ex}
\begin{ex}%[Phan Quốc Trí, Bai Giảng T10(2022)]%[0D1Y1-5]
	Mệnh đề "Có ít nhất một số tự nhiên khác 0" mô tả mệnh đề nào dưới đây?
	\choice{"$\forall n\in\mathbb{N}: n\neq 0$"}
	{"$\exists x\in\mathbb{N}:x=0$"}
	{"$\exists x\in\mathbb{Z}: x\neq 0$"}
	{\True "$\exists x\in\mathbb{N}:x\neq 0$"}
	\loigiai{
		Mệnh đề được viết lại là: "$\exists x\in\mathbb{N}:x\neq 0$".}
\end{ex}
\begin{ex}%[Phan Quốc Trí, Bai Giảng T10(2022)]%[0D1Y1-5]
	Mệnh đề nào sau đây đúng?
	\choice{$\forall n\in\mathbb{N}: n > 0$}
	{\True $\exists m\in\mathbb{Z}:2m=m$}
	{$\forall x\in\mathbb{R}:x^2> 0$}
	{$\exists k\in\mathbb{Q}:k^2=2$}
	\loigiai{
		Mệnh đề $\forall n\in\mathbb{N}: n > 0$ là mệnh đề sai, ví dụ $n=0$.\\
		Mệnh đề $\exists m\in\mathbb{Z}:2m=m$ là mệnh đề đúng, vì tồn tại $m=0$ thỏa mãn.\\
		Mệnh đề $\forall x\in\mathbb{R}:x^2> 0$ là mệnh đề sai, ví dụ $x=0$.\\
		Mệnh đề $\exists k\in\mathbb{Q}:k^2=2$ là mệnh đề sai.}
\end{ex}
\begin{ex}%[Phan Quốc Trí, Bai Giảng T10(2022)]%[0D1Y2-1]
	Hãy liệt kê các phần tử của tập $X=\left\{ x \in \mathbb{Q}| (x^2-x-6)(x^2-5)=0 \right\}$.
	\choice
	{$X=\left\{ \sqrt{5};3 \right\}$}
	{\True $X=\left\{ -2;3 \right\}$}
	{$X=\left\{ -\sqrt{5};-2;\sqrt{5}; 3 \right\}$}
	{$X=\left\{ -\sqrt{5};\sqrt{5}  \right\}$}
	\loigiai{
		Ta có 
		\begin{itemize}
			\item $x^2-x-6 = 0 \Leftrightarrow \hoac{&x=3 \in \mathbb{Q} \\&x=-2 \in \mathbb{Q}. }$
			\item $x^2-5 =0 \Leftrightarrow \hoac{&x= \sqrt{5} \notin \mathbb{Q} \\& x= - \sqrt{5} \notin \mathbb{Q}.}$
		\end{itemize}
		Do đó $X=\left\{ -2;3 \right\}$.
	}
\end{ex}
\begin{ex}%[Phan Quốc Trí, Bai Giảng T10(2022)]%[0D1Y2-1]
	Hãy liệt kê các phần tử của tập hợp $X=\left\{x \in \mathbb{N} \ |  \ x \leq 3\right\}$
	\choice
	{$X=\left[0;3\right]$}
	{\True $X=\left\{0;1;2;3\right\}$}
	{$X=\left\{1;2;3\right\}$}
	{$X=\left\{0  \longrightarrow 3\right\}$}
	\loigiai{ 
		Liệt kê các phần tử của tập hợp $X=\left\{x \in \mathbb{N} \ | \ x \leq 3\right\}=\left\{0;1;2;3\right\}$.
	}
\end{ex}

\begin{ex}%[Phan Quốc Trí, Bai Giảng T10(2022)]%[0D1Y2-1]
	Cho tập hợp $A=\{x\in \mathbb{R}|x^2-2x+5=0\}$. Mệnh đề nào sau đây đúng?
	\choice
	{\True $A=\varnothing$}
	{$A=0$}
	{$A=\{-1\}$}
	{$A=\{0\}$}
	\loigiai{
		Vì phương trình $x^2-2x+5=0$ vô ngiệm nên $A=\varnothing$.	
	}	
\end{ex}
\begin{ex}%[Phan Quốc Trí, Bai Giảng T10(2022)]%[0D1Y2-2]
	Cho tập hợp $A=\{a;b;c;d\}$. Số tập hợp con của $A$ có hai phần tử là
	\choice
	{\True $6$}
	{$7$}
	{$8$}
	{$5$}
	\loigiai{
		Sô tập con có hai phần tử của tập $A$ là $6$ đó là các tập $\{a;b\}$, $\{a;c\}$, $\{a;d\}$, $\{b;c\}$, $\{b;d\}$, $\{c;d\}$.}
\end{ex}
\begin{ex}%[Phan Quốc Trí, Bai Giảng T10(2022)]%[0D1Y2-2]
	Có bao nhiêu tập $A$ để $\{m; n\} \subset A \subset \{m; n; x; y\}$? 
	\choice
	{$2$}
	{$3$}
	{$1$}
	{\True $4$}
	\loigiai{
		Các tập $A$ thỏa mãn là $\{m; n\}$, $\{m; n; x\}$, $\{m; n; y\}$ và $\{m; n; x; y\}$. 
	}
\end{ex}
\begin{ex}%[Phan Quốc Trí, Bai Giảng T10(2022)]%[0D1B4-1]
	Cho tập hợp $X = \left\{x \in \mathbb{R}\big|- 2\le x \le 5\right\}$. Khẳng định nào sau đây đúng?
	\choice{$X = \left(-2;5\right)$}
	{$X = \left\{-2;5\right\}$}
	{$X = \left[-2;5\right)$}
	{\True $X = \left[-2;5\right]$}
	\loigiai{
		$\left\{x \in \mathbb{R}\big|- 2\le x \le 5\right\}= \left[-2;5\right]$.
	}
\end{ex}
\begin{ex}%[Phan Quốc Trí, Bai Giảng T10(2022)]%[0D1B2-2]
	Cho tập hợp $A = \left\{x \in \mathbb{N}\big| x^2 + 3x  = 0\right\}$. Khẳng định nào sau đây đúng?
	\choice
	{$A = \left\{-3;0\right\}$}
	{$A =\varnothing$}
	{$A = \left\{\varnothing\right\}$}
	{\True $ A = \left\{0\right\}$}
	\loigiai{
		Phương trình $x^2+3x=0$ có hai nghiệm $x_1=-3$, $x_2=0$. Tuy nhiên chỉ $x_1=0 \in \mathbb{N}$. Vậy $A=\left\{0\right\} $.
	}
\end{ex}
\begin{ex}%[Phan Quốc Trí, Bai Giảng T10(2022)]%[0D1B2-2]
	Gọi $A$ là tập hợp tất cả các tam giác cân và $B$ là tập hợp tất cả các tam giác đều. Trong các kết luận sau, kết luận nào đúng?
	\choice
	{$A \subset B$}
	{\True $B \subset A$}
	{$A=B$}
	{$A \cap B=\varnothing$}
	\loigiai{
		Ta có tam giác đều là một tam giác cân và có một góc $60^{\circ}$ nên $B \subset A$.
	}
\end{ex}

\begin{ex}%[Phan Quốc Trí, Bai Giảng T10(2022)]%[0D1B2-2]
	Cho $\mathbb{N}$, $\mathbb{Z}$, $\mathbb{Q}$, $\mathbb{R}$ là các tập hợp số. Mệnh đề nào sau đây \textbf{sai}?
	\choice
	{$\mathbb{Q} \subset \mathbb{R}$}
	{$\mathbb{N} \subset \mathbb{Z} \subset \mathbb{Q} \subset \mathbb{R}$}
	{$\mathbb{N} \subset \mathbb{Z} \subset \mathbb{Q}$}
	{\True $\mathbb{R} \subset \mathbb{Z}$}
	\loigiai{
		Theo mối quan hệ giữa các tập hợp số, ta có $\mathbb{N} \subset \mathbb{Z} \subset \mathbb{Q} \subset \mathbb{R}$.
	}
\end{ex}
\begin{ex}%[Phan Quốc Trí, Bai Giảng T10(2022)]%[0D1Y3-1]
	Cho tập hợp $M=\{1;2;3\}$ và $N=\{1;a;b\}$. Tìm $M\cup N$.
	\choice
	{\True $M\cup N=\{1;2;3;a;b\}$}
	{$M\cup N=\{2;3;a;b\}$}
	{$M\cup N=\{1\}$}
	{$M\cup N=\{2;3\}$}
	\loigiai{
		$M\cup N=\{1;2;3;a;b\}$.
	}
\end{ex}
\begin{ex}%[Phan Quốc Trí, Bai Giảng T10(2022)]%[0D1Y3-1]
	Cho hai tập hợp $A=\{a;c;d;e\}$ và $B=\{c;d;f;1;2\}$. Khi đó $A\cap B$ là	
	\choice
	{$\{1;2;f\}$}
	{$\{c;d;a;e;f;1;2\}$}
	{\True $\{c;d;\}$}
	{$\{a;e\}$}
	\loigiai{
		Ta có $A\cap B=\{c;d\}$.	
	}
\end{ex}
\begin{ex}%[Phan Quốc Trí, Bai Giảng T10(2022)]%[0D1Y3-2]
	Cho hai tập hợp $A= \left\{0, 1, 2,3,4,5,7\right\}$  và $B= \left\{2,3,4,5,6\right\}$. Tập hợp $A \setminus B$ bằng 
	\choice
	{$ \left\{0,1,2,7\right\}$}
	{$\left\{0,7\right\}$}
	{\True  $\left\{0,1,7\right\}$}
	{$\left\{0,1,6,7\right\}$}
	\loigiai{
		Ta có \[A \setminus B=\left\{0, 1, 7\right\}.\]
	}
\end{ex}

\begin{ex}%[Phan Quốc Trí, Bai Giảng T10(2022)]%[0D1Y3-2]
	Cho hai tập hợp $M=\{0;2;4;6;8;10\}$ và $N=\{0;1;2;3;4;5;6;7;8;9;10\}$. Hãy tìm phần bù của $M$ trong tập hợp $N$.
	\choice
	{$C_NM=N $}
	{$C_NM=M $}
	{\True $C_NM=\{1;3;5;7;9\} $}
	{$C_NM=\{0;2;6;8\} $}
	\loigiai{
		Vì $C_NM$ là tập hợp tất cả các phần tử của $N$ nhưng không là phần tử của $M$, do đó $C_NM= \{1;3;5;7;9\} $.
	}
\end{ex}
\begin{ex}%[Phan Quốc Trí, Bai Giảng T10(2022)]%[0D1B4-1]
	Cho $A=(-\infty;5]$ và $B=(0;+\infty)$. Tập hợp $A\cap B$ là
	\choice
	{\True $(0;5]$}
	{$[0;5)$}
	{$(0;5)$}
	{$(-\infty;+\infty)$}
	\loigiai
	{
		\begin{center}
			\begin{tikzpicture}[scale=1, font=\footnotesize, >=stealth]
				\draw[->] (0,0)--(9,0);
				\draw (3,0) node{$\Big($} + (90:.5) node{$0$} (6,0) node{$\Big]$} + (90:.5) node{$5$};
				\fill[pattern=north east lines] (0,-3pt) rectangle (3,3pt) (6,-3pt) rectangle (9,3pt);
				\begin{scope}[on background layer]\path[white]node{MDD-138};\end{scope}
			\end{tikzpicture}
		\end{center}
		Ta có $A\cap B = (-\infty;5]\cap (0;+\infty) = (0;5]$.
	}
\end{ex}
\begin{ex}%[Phan Quốc Trí, Bai Giảng T10(2022)]%[0D1B4-1]
	Cho tập hợp $A=[-2;5)$ và $B=[0;+\infty)$. Tìm $A\cup B$.
	\choice
	{$A\cup B=[0;5)$}
	{$A\cup B=[-2;0)$}
	{\True $A\cup B=[-2;+\infty)$}
	{$A\cup B[5;+\infty)$}
	\loigiai{
		Ta có 	$A\cup B=[-2;+\infty)$.
	}
\end{ex}

\begin{ex}%[Phan Quốc Trí, Bai Giảng T10(2022)]%[0D1B4-2]
	Tập hợp $\left( -2;3 \right)\setminus \left[1;6 \right]$ bằng tập hợp nào sau đây?
	\choice
	{$\left(-2;1\right]$}
	{$\left(-3;-2\right)$}
	{$\left(-2;6\right)$}
	{\True $\left(-2;1\right)$}
	\loigiai{Ta có $\left(-2;3\right)\setminus \left[1;6\right]=\left(-2;1\right)$.
	}
\end{ex}

\begin{ex}%[Phan Quốc Trí, Bai Giảng T10(2022)]%[0D1B4-2]
	Cho tập hợp $A=\left[ -2;\,\sqrt{5} \right)$. Tập hợp $C_{\mathbb{R}}A$ bằng
	\choice
	{$\left( -\infty ;\,-2 \right]\cup \left[ \sqrt{5};\,+\infty \right)$}
	{\True $\left( -\infty ;\,-2 \right)\cup \left[ \sqrt{5};\,+\infty \right)$}
	{$\left( -\infty ;\,-2 \right]\cup \left( \sqrt{5};\,+\infty \right)$}
	{$\left( -\infty ;\,-2 \right)\cup \left( \sqrt{5};\,+\infty \right)$}
	\loigiai{
		Ta có $C_{\mathbb{R}}A=\Bbb{R}\setminus A=\left( -\infty ;\,-2 \right)\cup \left[ \sqrt{5};\,+\infty \right).$
	}
\end{ex}
\begin{ex}%[Phan Quốc Trí, Bai Giảng T10(2022)]%[0D1B3-3]
	Một lớp có $40$ học sinh, trong đó có $24$ học sinh giỏi Toán, $18$ học sinh giỏi Văn và $10$ học
	sinh không giỏi môn nào trong hai môn Toán và Văn. Hỏi lớp đó có bao nhiêu học sinh giỏi cả hai môn
	Toán và Văn?
	\choice
	{\True $12$ học sinh}
	{$8$ học sinh}
	{$10$ học sinh}
	{$14$ học sinh}
	\loigiai{
		Số học sinh giỏi ít nhất một trong hai môn Toán và Văn là $40-10=30$.\\
		Do có $24$ học sinh giỏi Toán, $18$ học sinh giỏi Văn nên số học sinh giỏi cả hai môn là
		\[24+18-30=12.\]
	}	
\end{ex}
\begin{ex}%[Phan Quốc Trí, Bai Giảng T10(2022)]%[0D1B3-3]
	Mỗi học sinh lớp 10A phải học ít nhất một trong hai môn ngoại ngữ tiếng Anh hoặc tiếng Nhật. Biết lớp 10A có $51$ bạn học sinh trong đó có $31$ bạn học tiếng Anh và $27$ bạn học tiếng Nhật. Hỏi lớp 10A có bao nhiêu bạn học cả tiếng Anh và tiếng Nhật?
	\choice
	{\True $7$}
	{$9$}
	{$5$}
	{$12$}
	\loigiai{
		Số học sinh học cả tiếng Anh và tiếng Nhật của lớp 10A là $31+27-51=7$ bạn.
	}
\end{ex}
\begin{ex}%[Phan Quốc Trí, Bai Giảng T10(2022)]%[0D1B4-1]
	Cho hai tập hợp $A=(-4;5) \cup (7;9)$ và $B=(2;8)$. Tìm $A\cap B$ ta được
	\choice
	{$A \cap B=(7;8)$}
	{$A \cap B=(2;5)$}
	{\True $A \cap B=(2;5) \cup (7;8)$}
	{$A \cap B=[2;5] \cup [7;8]$}
	\loigiai{
		Ta có $A\cap B = \left[(-4;5) \cup (7;9)\right] \cap (2;8)=\left[(-4;5) \cap (2;8)\right] \cup \left[(7;9) \cap (2;8)\right]=(2;5)\cup (7;8)$.	
	}
\end{ex}

\begin{ex}%[Phan Quốc Trí, Bai Giảng T10(2022)]%[0D1B4-1]
	Cho hai tập $A=(-2; 4] \cap \mathbb{Z}$, $B=[-5; 7] \cap \mathbb{N}^{*}$. Số phần tử của tập hợp $A \cup B$ là
	\choice
	{\True $9$}
	{$13$}
	{$10$}
	{$8$}
	\loigiai{
		$A=(-2; 4] \cap \mathbb{Z}=\{-1;0;1;2;3;4\}$, $B=[-5; 7] \cap \mathbb{N}^{*}=\{1;2;3;4;5;6;7\}$.\\
		Ta có $A \cup B=\{-1;0;1;2;3;4;5;6;7\}$. Vậy tập hợp $A \cup B$ có $9$ phần tử.
	}
\end{ex}
\begin{ex}%[Phan Quốc Trí, Bai Giảng T10(2022)]%[0D1B4-1]
	Cho các tập hợp $A=\left[-2;2\right]$, $B=\left(1;5\right]$ và $ C=\left[0;3\right)$. Khi đó tập $\left(A \setminus B\right)\cap C$ là
	\choice
	{\True $\left[0;1\right]$}
	{$\left\{0;1\right\}$}
	{$\left[0;1\right)$}
	{$\left(0;1\right]$}
	\loigiai{
		Ta có $ A\setminus B=\left[-2;1\right]$$\Rightarrow\left(A\setminus B\right)\cap C=\left[0;1\right]$.
	}
\end{ex}
\begin{ex}%[Phan Quốc Trí, Bai Giảng T10(2022)]%[0D1Y4-2]
	Cho tập hợp $A=(-2;1]$ và $B=[-2;8)$. Khi đó $A\setminus B$ bằng
	\choice
	{$\left\{ 2 \right\}$}
	{$(1;8) $}
	{$\left\{ 8 \right\}$}
	{\True $\varnothing$}
	\loigiai{
		Ta có $A\setminus B=\varnothing$.
	}
\end{ex}
\begin{ex}%[Phan Quốc Trí, Bai Giảng T10(2022)]%[0D1B4-2]
	Cho $A=\left\{x\in \mathbb{R}|x+1\ge 0\right\}$, $B=\left\{x\in \mathbb{R}|4-x\ge 0\right\}$. Khi đó $A\backslash B$ là 
	\choice
	{$\left[-1; 4\right]$}
	{$\left[4;+\infty\right)$}
	{\True $\left(4;+\infty\right)$}
	{$\left(-\infty;-1\right)$}
	\loigiai{
		$A=\left\{x\in \mathbb{R}|x+1\ge 0\right\}=\left[-1;+\infty\right)$; $B=\left\{x\in \mathbb{R}|4-x\ge 0\right\}=\left(-\infty; 4\right]$.\\
		Nên $A\backslash B=\left(4;+\infty\right)$.
	}
\end{ex}
\begin{ex}%[Phan Quốc Trí, Bai Giảng T10(2022)]%[0D1K4-2]
	Cho tập hợp $A=[m;m+1]$, $B=[1;3]$. Tập hợp tất cả các giá trị của $m$ để $A\subset B$ là
	\choice
	{$m\leq 1$ hoặc $m\geq 2$}
	{\True $1\leq m\leq 2$}
	{$1<m<2$}
	{$0\leq m\leq 2$}
	\loigiai{Để $A\subset B$ thì $\heva{&m\geq 1\\&m+1\leq 3}\Leftrightarrow 1\leq m\leq 2$.}
\end{ex}
\noindent\textbf{II. PHẦN TỰ LUẬN}
\begin{bt}%[Phan Quốc Trí, Bai Giảng T10(2022)]%[0D1B4-1] 
	Cho $A=\left \{x \in \mathbb{R} \Big|  -1<x \le 7 \right \} ; B=(-\infty ; 2]$.  Tìm các tập hợp $A \cap B, A \cup B$.
	\loigiai{
		Ta có $A = (-1;7]$, $A \cap B = (-1;2]$ và 
		$A \cup B = (-\infty ; 7] $. 
	}
\end{bt}
\begin{bt}%[Phan Quốc Trí, Bai Giảng T10(2022)]%[0D1B3-2]
	Cho hai tập hợp $A=\left\{ a;b \right\}$ và $B=\left\{ a;b;c;2;3;4 \right\}$. Tìm tất cả các tập hợp $X$ thỏa mãn $A\cup X=B$.
	\loigiai { 
		Vì $A\cup X=B$ nên $X$ chắc chắn có chứa các phần tử $c;2;3;4$\\
		Các tập $X$ có thể là $\left\{ c;2;3;4 \right\},\,\left\{ c;2;3;4;a \right\},\,\left\{ c;2;3;4;b \right\},\,\left\{ c;2;3;4; a;b \right\}$.
	}
\end{bt}
\begin{bt}%[Phan Quốc Trí, Bai Giảng T10(2022)]%[0D1K4-1]
	Cho hai tập khác rỗng $A=(m-1;4]$; $B=(-2;2m+2)$, $m\in\mathbb{R}$. Tìm tất cả các giá trị của $m$ để $A\cap B\neq\varnothing$.
	\loigiai{
		Điều kiện: $\heva{&m-1<4\\&2m+2>-2} \Leftrightarrow -2<m<5$.\\
		Ta có	$A\cap B \ne \varnothing \Leftrightarrow 2m+2> m-1\Leftrightarrow m>-3$.\\
		Kết hợp với điều kiện ta có  $A\cap B \ne \varnothing\Leftrightarrow -3 < m < 5$.
	}
\end{bt}

\begin{bt}%[Phan Quốc Trí, Bai Giảng T10(2022)]%[0D1B3-3]
	Trong số $40$ học sinh của lớp 10A có $12$ bạn được xếp loại học lực giỏi, $19$ bạn được xếp loại hạnh kiểm tốt, trong đó có $8$ bạn vừa có học lực giỏi, vừa có hạnh kiểm tốt. Để được khen thưởng thì bạn đó phải có học lực giỏi hoặc có hạnh kiểm tốt. Hỏi có bao nhiêu bạn \textbf{không} được khen thưởng?
	\loigiai{
		Gọi $A$ là tập hợp các học sinh được xếp loại học lực giỏi, $B$ là tập hợp các học sinh được xếp loại hạnh kiểm tốt.\\
		Ta có 
		\begin{itemize}
			\item $A\cup B$ là tập hợp các học sinh được xếp loại học lực giỏi hoặc hạnh kiểm tốt.
			\item $A\cap B$ là tập hợp các học sinh được xếp loại học lực giỏi và hạnh kiểm tốt.
			\item Theo đề  $$n(A)=12, n(B)=19, n(A\cap B)=8.$$
			\item Do đó $$n(A\cup B=n(A)+n(B)-n(A \cap B))=23.$$
		\end{itemize}
		Vậy lớp 10A có $23$ học sinh được khen thưởng và $17$ học sinh \textbf{không} được khen thưởng.}
\end{bt}

\Closesolutionfile{ans}

\newpage
\begin{indapan}{10}
	{ans/ans-KT-103}
\end{indapan}

%%Chương 2
% \section{Bất phương trình bậc nhất hai ẩn}
\setcounter{dang}{0}
\subsection{Tóm tắt lý thuyết}
\subsubsection{Bất phương trình bậc nhất hai ẩn}
Bất phương trình bậc nhất hai ẩn $x$, $y$ có dạng tổng quát là 
\begin{center}
\fbox{$ax+by\le c \text{ (hoặc } ax+by<c; ax+by\ge c; ax+by>c),$}
\end{center}
trong đó $a$, $b$, $c$ là những số thực, $a$ và $b$ không đồng thời bằng $0$, $x$ và $y$ là các ẩn số.

\subsubsection{Biểu diễn tập nghiệm của bất phương trình bậc nhất hai ẩn}
Cũng như bất phương trình bậc nhất một ẩn, các bất phương trình bậc nhất hai ẩn có vô số nghiệm và để mô tả tập nghiệm của chúng, ta sử dụng phương pháp biểu diễn hình học.\\ 
Trong mặt phẳng tọa độ $Oxy$, tập hợp các điểm có tọa độ là nghiệm của bất phương trình được gọi là \textbf{miền nghiệm} của nó.\\
Quy tắc thực hành biểu diễn miền nghiệm của bất phương trình $ax+by \le c$ như sau (tương tự cho bất phương trình $ax+by\ge c$)
\begin{itemize}
	\item{\bf Bước 1:} Trên mặt phẳng tọa độ $Oxy$, vẽ đường thẳng $\Delta \colon ax+by=c$.
	\item{\bf Bước 2:} Lấy một điểm $M_0\left(x_0;y_0\right)$ không thuộc $\Delta$ (ta thường lấy gốc tọa độ $O$).
	\item{\bf Bước 3:} Tính $ax_0+by_0$ và so sánh $ax_0+by_0$ với $c$.
	\item{\bf Bước 4:} Kết luận,
	\begin{itemize}
		\item Nếu $ax_0+by_0<c$ thì nửa mặt phẳng bờ $\Delta$ chứa $M_0$ là miền nghiệm của $ax_0+by_0\leq c$.
		\item Nếu $ax_0+by_0>c$ thì nửa mặt phẳng bờ $\Delta$ không chứa $M_0$ là miền nghiệm của $ax_0+by_0\le c$.
	\end{itemize}
\end{itemize}

\begin{note}
	Miền nghiệm của bất phương trình $ax_0+by_0\leq c$ bỏ đi đường thẳng $ax+by=c$ là miền nghiệm của bất phương trình $ax_0+by_0<c$.
\end{note}
\subsection{Các dạng toán}
\begin{dang}{Bất phương trình bậc nhất hai ẩn và bài toán liên quan}
\end{dang}
\viduminhhoa
\begin{vd}%[Dự án Tex hóa đề thi lớp 10-11-Nhóm Word-T-Begin]%[Lê Vũ Hải-Phản biện: Trần Quốc Tráng]%[0D4Y4-1]%
	Cho bất phương trình: $2x-y<0$ . Trong các cặp số $(-1;2)$, $\left(2;0\right)$, $(0;1)$, $\left(3;-2\right)$, $(-1;-2)$, cặp nào là nghiệm của bất phương trình, cặp nào không phải là nghiệm của bất phương trình?
	\loigiai{
		Bằng cách thử trực tiếp, các cặp $(-1;2)$, $(0;1)$ là nghiệm, các cặp còn lại không phải là nghiệm của bất phương trình.
	}
\end{vd}
\begin{vd}%[Dự án Chuyển tex 10-11, Cao Thành Thái]%[0D4B4-1]
	Biểu diễn hình học tập nghiệm của bất phương trình $2x+y\le 3$.
	\loigiai
	{
		\immini{
			Vẽ đường thẳng $\Delta\colon 2x+y=3$.\\
			Lấy gốc tọa độ $O(0;0)$, ta thấy $O\notin \Delta $ và có $2 \cdot 0+0<3$ nên nửa mặt phẳng bờ $\Delta$ chứa gốc tọa độ $O$ là miền nghiệm của bất phương trình đã cho (miền không bị tô đậm trong hình vẽ).
		}
		{
			\begin{tikzpicture}[line join=round, line cap=round, >=stealth,font=\footnotesize, scale=0.8]
				\draw[->](-2,0)--(3,0) node[below right] {$x$};
				\draw[->](0,-1)--(0,4) node[right] {$y$};
				\clip (-2,-1) rectangle (3,4);
				\node (0,0) [below left]{$ O $};
				\foreach \x in {-1,...,2}
				\draw[shift={(\x,0)},color=black] (0pt,2pt) -- (0pt,-2pt);
				\foreach \y in {1,...,3}
				\draw[shift={(0,\y)},color=black] (2pt,0pt) -- (-2pt,0pt);
				\draw[samples=100,smooth,domain=-2:3] plot(\x,{-2*(\x)+3});
				\draw [pattern=north west lines] (-1,5)--(3,5)--(3,-1) -- (2,-1)--cycle;
				\draw[fill=black] (0,3) circle(1pt) node[left]{$3$};
				\draw[fill=black] (1.5,0)circle(1pt) node[below left]{\tiny $\dfrac{3}{2}$};
			\end{tikzpicture}
		}
	}
\end{vd}
\begin{vd}%[Mai Hà Lan]%[0D4B4-1]
	\begin{enumerate}
		\item Biểu diễn hình học tập nghiệm của bất phương trình $-2x + 3y > 0$.
		\item Cho hai điểm $A(2;1)$ và $B(3; 3)$, hỏi hai điểm này cùng phía hay khác phía đối với bờ $(d)$.
	\end{enumerate}
	\loigiai{
		\immini{
			\begin{enumerate}
				\item Vẽ đường thẳng $d: -2x + 3y = 0$.\\
				Thay tọa độ điểm $M(1;0)$ vào vế trái phương trình đường thẳng $(d)$, ta được: $-2 < 0$.\\
				Vậy miền nghiệm của bất phương trình là nửa mặt phẳng không chứa điểm $M$. (Trên hình là nửa mặt phẳng không bị gạch bỏ).
				\item Thế tọa độ điểm $A$ vào vế trái của phương trình đường thẳng $(d)$ ta được $-2 \cdot 2 + 3 \cdot 1 = -1 < 0$.\hfill $(1)$\\
				Thế tọa độ điểm $B$ vào vế trái của phương trình đường thẳng $(d)$ ta được $-2 \cdot 3 + 3 \cdot 3 = 3 > 0$. \hfill $(2)$ \\
				Từ $(1)$ và $(2)$ suy ra hai điểm nằm ở hai phía đối với bời $(d)$.
			\end{enumerate}
		}{
			\begin{tikzpicture}
				%---------------------- Vẽ hệ trục tọa độ
				\draw[->] (-2.25,0)--(4.25,0) node[below right] {$x$};
				\draw[->] (0,-0.755)--(0,2.25) node[right] {$y$};
				\node (0,0) [below right]{$ O $};
				%----------------------- Vẽ đoạn chắn trên trục
				\foreach \x in {-2,-1,1,2,3,4}
				\draw[shift={(\x,0)},color=black] (0pt,2pt) -- (0pt,-2pt);
				%\node at (3.8,0.5) {$4$};
				\foreach \y in {1,2}
				\draw[shift={(0,\y)},color=black] (2pt,0pt) -- (-2pt,0pt);
				%\node at (-0.5,-1.8) {$-2$};
				
				%--------------------- Vẽ hàm
				\draw [thick, domain=-1:3, samples=100] plot (\x, {(2/3) * \x});
				\node at (2,1.75) {$(d)$};
				
				%---------------------- Điểm M
				\fill (1,0) circle (2pt) node[below right]{$M(1;0)$};
				
				%----------------------Vẽ miền nghiệm
				\tkzDefPoints{-1/-.66/A, 4/-.66/B, 4/2/C, 3/2/D}
				\tkzDrawPolygon[ pattern=north east lines,opacity=.3](A,B,C,D)
			\end{tikzpicture}
	}}
\end{vd}
\begin{vd}%[Mai Hà Lan]%[0D4G4-1]
	\begin{enumerate}
		\item Biểu diễn hình học tập nghiệm của bất phương trình $x + y -3 < 0$.
		\item Tìm điều kiện của $m$ và $n$ để mọi điểm thuộc đường thẳng $(d')$: $ (m^2-2)x - y + m + n= 0 $ đều là nghiệm của bất phương trình trên.
	\end{enumerate}
	\loigiai{
		\immini{
			\begin{itemize}
				\item[a)] Vẽ đường thẳng $d: x + y = 3 $.\\
				Thay tọa độ điểm $O(0;0)$ vào vế trái phương trình đường thẳng $(d)$, ta được: $0< 3$.\\
				Vậy miền nghiệm của bất phương trình là nửa mặt phẳng chứa điểm $O$. (Trên hình là nửa mặt phẳng không bị gạch bỏ).
				\item[b)] Để mọi điểm thuộc đường thẳng  $(d')$ đều là nghiệm của bất phương trình thì điều kiện cần là $(d')$ phải song song với $(d)$. Ta có $d:  y = -x + 3$ và $d': y = (m^2-2)x +  m + n$. Để $(d)$ song song $(d')$ thì $\heva{& m^2 - 2 = -1 \\& m + n \ne 3} \Leftrightarrow \hoac{&\heva{&m=1\\&n\ne 2}\\&\heva{&m=-1\\&n\ne 4}}$\\
				Với $\heva{&m=1\\&n\ne 2}$ thì ta được $d': y = - x + n + 1$. Để thỏa yêu cầu bài toán thì điều kiện đủ là đường thẳng $(d')$ là đồ thị của đường thẳng $(d)$ khi $(d)$ tịnh tiến xuống dưới theo trục $Oy$. Tức $n + 1 < 3 \Leftrightarrow n < 2$.
			\end{itemize}
		}{
			\begin{tikzpicture}[scale=.8]
				%---------------------- Vẽ hệ trục tọa độ
				\draw[->] (-2.25,0)--(4.25,0) node[below right] {$x$};
				\draw[->] (0,-1.25)--(0,4.25) node[right] {$y$};
				\node (0,0) [below left]{$ O $};
				%----------------------- Vẽ đoạn chắn trên trục
				\foreach \x in {-2,-1,1,2,3,4}
				\draw[shift={(\x,0)},color=black] (0pt,2pt) -- (0pt,-2pt);
				\node at (2.75,-0.4) {$3$};
				\foreach \y in {-1,1,2,3,4}
				\draw[shift={(0,\y)},color=black] (2pt,0pt) -- (-2pt,0pt);
				\node at (-0.35,2.75) {$3$};
				
				%--------------------- Vẽ hàm
				\draw [thick, domain=-1:4, samples=100] plot (\x, {3-\x});
				\node at (1.2,1.2) {$(d)$};
				
				%----------------------Vẽ miền nghiệm
				\tkzDefPoints{-1/4/A, 4/4/B, 4/-1/C}
				\tkzDrawPolygon[pattern=north east lines,opacity=.3](A,B,C)
			\end{tikzpicture}
	}}
\end{vd}
\baitaptl
\begin{bt}%[Dự án Chuyển tex 10-11, Cao Thành Thái]%[0D4B4-1]%
	Biểu diễn hình học tập nghiệm của bất phương trình $2x+y\le 3$.
	\loigiai
	{
		\immini{
			Vẽ đường thẳng $\Delta\colon 2x+y=3$.\\
			Lấy gốc tọa độ $O(0;0)$, ta thấy $O\notin \Delta $ và có $2 \cdot 0+0<3$ nên nửa mặt phẳng bờ $\Delta$ chứa gốc tọa độ $O$ là miền nghiệm của bất phương trình đã cho (miền không bị tô đậm trong hình vẽ).
		}
		{
			\begin{tikzpicture}[line join=round, line cap=round, >=stealth,font=\footnotesize, scale=0.8]
				\draw[->](-2,0)--(3,0) node[below right] {$x$};
				\draw[->](0,-1)--(0,4) node[right] {$y$};
				\clip (-2,-1) rectangle (3,4);
				\node (0,0) [below left]{$ O $};
				\foreach \x in {-1,...,2}
				\draw[shift={(\x,0)},color=black] (0pt,2pt) -- (0pt,-2pt);
				\foreach \y in {1,...,3}
				\draw[shift={(0,\y)},color=black] (2pt,0pt) -- (-2pt,0pt);
				\draw[samples=100,smooth,domain=-2:3] plot(\x,{-2*(\x)+3});
				\draw [pattern=north west lines] (-1,5)--(3,5)--(3,-1) -- (2,-1)--cycle;
				\draw[fill=black] (0,3) circle(1pt) node[left]{$3$};
				\draw[fill=black] (1.5,0)circle(1pt) node[below left]{\tiny $\dfrac{3}{2}$};
			\end{tikzpicture}
		}
	}
\end{bt}
\begin{bt}%[0D4B4-1]%
	Biểu diễn hình học tập nghiệm của bất phương trình bậc nhất hai ẩn $2x - 4y < 8$.
	\loigiai{
		\immini{
			Vẽ đường thẳng $d: 2x - 4y =8$.\\
			Thay tọa độ điểm $O(0;0)$ vào vế trái phương trình đường thẳng $(d)$, ta được: $0 < 8$.\\
			Vậy miền nghiệm của bất phương trình là nửa mặt phẳng chứa điểm $O$. (Trên hình là nửa mặt phẳng không bị gạch bỏ).
		}{
			\begin{tikzpicture}[scale=.7]
				%----------------- Vẽ hệ trục tọa độ
				\draw[->] (-2.25,0)--(8.25,0) node[below right] {$x$};
				\draw[->] (0,-3.25)--(0,1.25) node[right] {$y$};
				\node (0,0) [below left] {$ O $};
				%----------------- Vẽ đoạn chắn trên trục
				\foreach \x in {-2,-1,1,2,3,4,5,6,7,8}
				\draw[shift={(\x,0)},color=black] (0pt,2pt) -- (0pt,-2pt);
				\node at (3.8,0.5) {$4$};
				\foreach \y in {-3,-2,-1,1}
				\draw[shift={(0,\y)},color=black] (2pt,0pt) -- (-2pt,0pt);
				\node at (-0.5,-1.8) {$-2$};
				
				%------------- Vẽ hàm
				\draw [thick, domain=-2:6, samples=100] plot (\x, {(1/2)*\x - 2});
				\node at (4.5,.75) {$(d)$};
				
				%---------------- Vẽ miền nghiệm
				\tkzDefPoints{6/1/A, -2/-3/B, 8/-3/C, 8/1/D}
				\tkzDrawPolygon[ pattern=north east lines,opacity=.3](A,B,C,D)
			\end{tikzpicture}
	}}
\end{bt}
\begin{bt}%[Mai Hà Lan]%[0D4B4]
	Biểu diễn hình học tập nghiệm của bất phương trình bậc nhất hai ẩn $3x - y \le 0$.
	\loigiai{
		\immini{
			Vẽ đường thẳng $d: 3x - y = 0 $.\\
			Thay tọa độ điểm $M(0;2)$ vào vế trái phương trình đường thẳng $(d)$, ta được: $-2 < 0$.\\
			Vậy miền nghiệm của bất phương trình là nửa mặt phẳng không chứa điểm $M$, kể cả bờ $(d)$. (Trên hình là nửa mặt phẳng không bị gạch bỏ).
		}{
			\begin{tikzpicture}
				%---------------------- Vẽ hệ trục tọa độ
				\draw[->] (-2.25,0)--(2.25,0) node[below right] {$x$};
				\draw[->] (0,-1.25)--(0,3.25) node[right] {$y$};
				\node (0,0) [above right]{$ O $};
				%----------------------- Vẽ đoạn chắn trên trục
				\foreach \x in {-2,-1,1}
				\draw[shift={(\x,0)},color=black] (0pt,2pt) -- (0pt,-2pt);
				%\node at (3.8,0.5) {$4$};
				\foreach \y in {-1,1,2,3}
				\draw[shift={(0,\y)},color=black] (2pt,0pt) -- (-2pt,0pt);
				%\node at (-0.5,-1.8) {$-2$};
				
				%--------------------- Vẽ hàm
				\draw [thick, domain=-.33:1, samples=100] plot (\x, {3*\x});
				\node at (.5,2.5) {$(d)$};
				
				%---------------------- Điểm M
				\fill (0,2) circle (2pt) node[left]{$M(0;2)$};
				
				%----------------------Vẽ miền nghiệm
				\tkzDefPoints{-.33/-.99/A, 2/-1/B, 2/3/C, 1/3/D}
				\tkzDrawPolygon[ pattern=north east lines,opacity=.3](A,B,C,D)
			\end{tikzpicture}
	}}
\end{bt}
\begin{bt}%[Mai Hà Lan]%[0D4K4]
	a) Biểu diễn hình học tập nghiệm của bất phương trình bậc nhất hai ẩn $\dfrac{x}{3} + \dfrac{y}{6} < 1$.\\
	b) Tìm điểm $A$ thuộc miền nghiệm của bất phương trình trên. Biết rằng điểm $A$ là giao điểm của parabol $(P)$ có dạng $y = x^2 - 5x +4$ và trục hoành. 
	\loigiai{
		\immini{
			\begin{itemize}
				\item[a)] $\dfrac{x}{3} + \dfrac{y}{6} < 1 \Leftrightarrow 2x + y  <6$\\
				Vẽ đường thẳng $d: 2x + y = 6$.\\
				Thay tọa độ điểm $O(0;0)$ vào vế trái phương trình đường thẳng $(d)$, ta được: $0 < 6$.\\
				Vậy miền nghiệm của bất phương trình là nửa mặt phẳng chứa điểm $O$. (Trên hình là nửa mặt phẳng không bị gạch bỏ).
				\item[b)] Điểm $A$ nằm trên parabol $(P)$ có dạng $y = x^2 - 5x +4$ và trục hoành nên hoành độ của $A$ là nghiệm của phương trình $x^2 - 5x + 4 = 0 \Leftrightarrow \hoac{& x = 1\\ & x = 4.}$ \\
				Suy ra ta được hai điểm $(1;0)$ và $(4;0)$. Lần lượt thế tọa độ từng điểm vào vế trái của phương trình đường thẳng $(d)$, do $A$ thuộc miền nghiệm của bất phương trình đã cho nên ta được $A$ có tọa độ là $(1;0)$.
			\end{itemize}
		}{
			\begin{tikzpicture}[scale=.6]
				%---------------------- Vẽ hệ trục tọa độ
				\draw[->] (-2.25,0)--(6.25,0) node[below right] {$x$};
				\draw[->] (0,-3.25)--(0,8.25) node[right] {$y$};
				\node (0,0) [below left]{$ O $};
				%----------------------- Vẽ đoạn chắn trên trục
				\foreach \x in {-2,-1,1,2,3,4,5,6}
				\draw[shift={(\x,0)},color=black] (0pt,2pt) -- (0pt,-2pt);
				\node at (2.75,-0.4) {$3$};
				\node at (.75,-0.4) {$1$};
				\node at (4.25,-0.4) {$4$};
				\foreach \y in {-3,-2,-1,1,2,3,4,5,6,7,8}
				\draw[shift={(0,\y)},color=black] (2pt,0pt) -- (-2pt,0pt);
				\node at (0.5,6.25) {$6$};
				
				%--------------------- Vẽ hàm
				\draw [thick, domain=-1:4.5, samples=100] plot (\x, {-2*\x + 6});
				\node at (3.5,-2.75) {$(d)$};
				
				%---------------------- Vẽ (P)
				\draw [thick, domain=-.7:5.7, samples=100] plot (\x, {(\x)^2 - 5*\x + 4});
				\node at (1,-2) {$(P)$};
				
				%----------------------Vẽ miền nghiệm
				\tkzDefPoints{-1/8/A, 6/8/B, 6/-3/C, 4.5/-3/D}
				\tkzDrawPolygon[pattern=north east lines,opacity=.3](A,B,C,D)
			\end{tikzpicture}
	}}
\end{bt}
\begin{bt}%[Lê Xuân Dũng]%[0D4B4]%[0D4K4]
	Cho bất phương trình $2x+y-1 \le 0$.
	
	a) Biểu diễn miền nghiệm của bất phương trình đã cho trong mặt phẳng tọa độ $Oxy.$
	
	b) Tìm tất cả giá trị  tham số $m$ để điểm $M(m,1)$ nằm trong miền nghiệm của bất phương trình đã và biểu diễn tập
	hợp $M$ tìm được trong cùng hệ trục tọa độ $Oxy$ ở câu a).
	\loigiai{
		\immini{a) Đường thẳng $(d){:} \, 2x+y-1=0$ có đồ thị như hình vẽ bên.
			Ta có $2.0+0-1 <0.$ Do đó, miền nghiệm là đường thẳng $(d)$ và miền không gạch chéo như hình vẽ bên (Miền chứa gốc tọa độ).
			
			b) Để $M$ là một nghiệm thì $2m+1-1 \le 0\Leftrightarrow m\le 0.$ 
			Vì $M$ nằm trên đường thẳng $(\Delta): y = 1.$ Do đó, tập hợp tất cả điểm $M$
			là nghiệm của bất phương trình trình đã cho là tia $At$ như hình vẽ.
		}{\begin{tikzpicture}[scale=0.7,thick,>=stealth']
				\draw[->] (-1.3,0) -- (3.3,0)node[above]{$x$};
				\foreach \x in {}
				\draw[shift={(\x,0)},color=black] (0pt,2pt) -- (0pt,-2pt) node[below] {\footnotesize $\x$};
				\draw[->,color=black] (0,-2) -- (0,3.3)node[right]{$y$};
				\foreach \y in {}
				\draw[shift={(0,\y)},color=black] (2pt,0pt) -- (-2pt,0pt) node[left] {\footnotesize $\y$};
				\node[below left] at (0,0) {$O$};
				\node[below left] at (0,1) {$A$};
				\node[below] at (1,-1.3) {$d$};
				\node[above right] at (0,1) {$1$};
				\node[below] at (0.5,0) {$\frac{1}{2}$};
				\clip(-1.3,-2) rectangle (3,3);
				\node[below] at (-1.2,1) {$t$};
				%	\fill[pattern=north east lines] (-1,2.66667) -- (-1,4) -- (4,4) -- (4,-0.6667) -- (3,0) -- cycle;
				%\draw[line width=1.2pt,smooth,samples=100,domain=-1:4] plot(\x,{2-0.666667*(\x)});
				\fill[pattern=north east lines,pattern color=blue!30] (-1,3)--(0,1)--(0.5,0)--(1.5,-2)--(3,-2)--(3,0)--(3,3)-- cycle;
				\draw[line width=1.2pt,smooth,samples=100,domain=-3:3] plot(\x,{1-2*(\x)});
				\draw[line width=1.2pt][-] (0,1)--(-1.3,1);
		\end{tikzpicture}}
	}
\end{bt}

\begin{bt}%[Lê Xuân Dũng]%[0D4B4]%[0D4K4]
	Cho bất phương trình $x-2y+4m > 0.$
	
	a) Tùy theo giá trị tham số $m,$ hãy biểu diễn tập nghiệm của bất phương trình đã cho
	trong hệ trục tọa độ $Oxy.$
	
	b) Gọi $A,B$ lần lượt  là giao của đường thẳng $x-2y+4m=0$ với trục hoành và trục tung. 
	Tìm tất cả các giá trị của tham số $m$ để tập nghiệm của bất phương trình đã cho chứa điểm $C(2;1)$ 
	sao cho diện tích tam giác $ABC$ bằng $4.$
	\loigiai{
		\immini{a) Xét đường thẳng  $(d_m){:} \, x-2y+4m=0$ có đồ thị như hình vẽ bên.
			Ta có $0-2.0+4m = 4m.$ Do đó, với mọi $m \ne 0$ miền nghiệm luôn chứa gốc tọa độ.
			Nếu $m=0$ thì miền nghiệm chứa điểm $(1;0).$ Vậy với mọi $m$ miền nghiệm
			là miền không gạch chéo như hình vẽ bên.
			
			b) Để $C$ là một nghiệm của bất phương trình đã cho thì $2-2+4m > 0\Leftrightarrow m> 0.$ 
			Khi đó, $OC \parallel (d_m),$ suy ra $S_{\Delta ABC}=S_{\Delta ABO} = 4m^2.$
			Theo giả thiết, ta có $4m^2 = 4 \Leftrightarrow m=1.$
		}{\begin{tikzpicture}[scale=0.6,thick,>=stealth']
				\draw[->] (-5.0,0) -- (4.3,0)node[above]{$x$};
				\foreach \x in {1,2}
				\draw[shift={(\x,0)},color=black] (0pt,2pt) -- (0pt,-2pt) node[below] {\footnotesize $\x$};
				\draw[->,color=black] (0,-1) -- (0,3.3)node[right]{$y$};
				\foreach \y in {}
				\draw[shift={(0,\y)},color=black] (2pt,0pt) -- (-2pt,0pt) node[left] {\footnotesize $\y$};
				\node[below right] at (0,0) {$O$};
				\node[below ] at (-4,0) {$A$};
				\node[above left] at (-4,0) {$-4m$};
				\node[right=0.3] at (0,2) {$B$};
				\node[above left] at (0,2) {$2m$};
				\node[above] at (2,1) {$C$}; 
				\node at (-0.3,1) {$1$}; 
				\draw[fill]  (2,1) circle (1.5pt) (-4,0) circle (1.5pt) (0,2) circle (1.5pt) (2,1) circle (1.5pt);
				\draw  (-4,0)--(2,1)--(0,2);
				\draw[dashed] (2,1)--(2,0) (2,1)--(0,1);
				\clip(-5.0,-2) rectangle (4.3,3.0);	
				\fill[pattern=north east lines,pattern color=blue!30] (-5,3)--(-5,0)--(-5,-0.5)--(-4,0)--(0,2)--(2,3)-- cycle;
				\draw[line width=1.2pt,smooth,samples=100,domain=-5.0:4.3] plot(\x,{2+0.5*(\x)});
				\draw[line width=1.2pt,smooth,samples=100,domain=-2.0:4.3] plot(\x,{0.5*(\x)});
		\end{tikzpicture}}
	}
\end{bt}
\begin{dang}{Bài toán thực tế liên quan}
	
\end{dang}
\viduminhhoa
\begin{vd}%[Nguyện Ngô]%[0D4B4]
	Hà mang $95000$ đồng ra chợ mua hoa cúc và hoa hồng. Một bông hoa cúc có giá $4000$ đồng, một bông hoa hồng có giá $7000$ đồng. Viết bất phương trình bậc nhất hai ẩn cho số tiền mà Hà phải chi để mua $x$ bông hoa cúc và $y$ bông hoa hồng.  
	\loigiai{
		Ta có $x, y\in\mathbb{N}^*$.\\
		Giá của $x$ bông hoa cúc là $4000x$ đồng, giá của $y$ bông hoa hồng là $7000y$ đồng.\\
		Vì số tiền Hà mang đi là $95000$ đồng nên ta có bất phương trình 
		\[4000x+7000y\le 95000\Leftrightarrow 4x+7y\le 95.\] 
	}
\end{vd}

\begin{vd}%[Nguyện Ngô]%[0D4K4]
	Mỗi ngày Nga đều dành không quá $30$ phút để đọc cả $2$ cuốn sách A, B. Nga đọc được $3$ trang sách A trong $2$ phút, đọc được $2$ trang sách B trong $1$ phút. Gọi $x$, $y$ lần lượt là số phút đọc sách A và số phút đọc sách B. Tìm điều kiện của $x$ và $y$ để Nga đọc được ít nhất $35$ trang sách trong một ngày.
	\loigiai{
		Gọi $x$, $y$ lần lượt là số phút đọc sách A và số phút đọc sách B trong một ngày, $x, y>0$. Tổng số phút đọc sách không quá $30$ phút nên $x+y\le 30$.\\
		Số trang sách A đọc được sau $x$ phút là $\dfrac{3x}{2}$.
		Số trang sách B đọc được sau $y$ phút là $2y$.\\
		Nga đọc được ít nhất $35$ trang sách trong một ngày khi và chỉ khi $\dfrac{3x}{2}+2y\ge 35$.\\
		Vậy $x,y$ cần thỏa mãn các điều kiện $\heva{&x,y>0\\&x+y\le30\\&\dfrac{3x}{2}+2y\ge35.}$
	}
\end{vd}

\begin{vd}%[Nguyện Ngô]%[0D4K4]
	Một cửa hàng bán hai loại trà sữa, trong đó $4$ cốc loại $1$ có giá $100000$ đồng, $1$ cốc loại $2$ có giá $30000$ đồng. Muốn có lãi theo dự tính thì mỗi ngày cửa hàng phải bán được ít nhất $5$ triệu đồng tiền hàng. Hỏi số cốc trà sữa bán được trong một ngày trong những trường hợp nào thì cửa hàng có lãi như dự tính?
	\loigiai{
		Gọi $x$, $y$ lần lượt là số cốc trà sữa loại $1$, loại $2$ bán được ($x, y\in\mathbb{N}$).\\
		Tổng số tiền bán trà sữa là $25x+30y$ nghìn đồng.\\
		Cửa hàng có lãi như dự tính trong trường hợp số tiền bán trà sữa thu được trong một ngày không nhỏ hơn $5$ triệu đồng, tức là 
		\[25x+30y\ge 5000.\quad\quad (1)\]
		\immini{
			Miền nghiệm của bất phương trình (1) được xác định như sau\\
			+/ Vẽ đường thẳng $d\colon 25x+30y=5000$.\\
			+/ Chọn gốc tọa độ $O(0;0)$ và tính $25\cdot0+30\cdot0<500$.\\
			Do đó miền nghiệm của bất phương trình (1) là nửa mặt phẳng bờ $d$, không chứa gốc tọa độ $O$, lấy cả đường thẳng $d$.\\
			Gọi $A$, $B$ lần lượt là giao điểm của $d$ và $Ox$, $Oy$. Khi đó, nếu bán được $x$ cốc trà sữa loại $1$ và $y$ cốc trà sữa loại $2$ mà điểm $(x;y)$ nằm ở góc phần tư thứ nhất đồng thời nằm ngoài miền tam giác $OAB$ (có thể nằm trên cạnh $AB$) (phần gạch chéo) thì cửa hàng sẽ có lãi như dự tính.
		}
		{
			\begin{tikzpicture}[scale=.7,>=stealth]
				\draw[->] (0,0) -- (5.3,0)node[below]{$x$};
				\draw[->,color=black] (0,0) -- (0,5.3)node[left]{$y$};
				\node[below left] at (0,0){$O$};
				\node[left] at (0,3){$\dfrac{5000}{3}$};
				\node[above right] at (0,3){B};
				\node[below] at (4,0){$200$};
				\node[above] at (4,0){A};
				\clip(0,0) rectangle (5.3,5.3);
				\fill[pattern=north east lines](4,0)-- (5.3,0) -- (5.3,5.3) -- (0,5.3)--(0,3)-- cycle;
				\draw[line width=1.2pt,smooth,samples=100,domain=0:4] plot(\x,{-0.75 *(\x) +3});
			\end{tikzpicture}
		}
	}
\end{vd}
\baitaptl
\begin{bt}%[Nguyện Ngô]%[0D4B4]
	Giá sách của Hoa có thể chứa được khối lượng sách tối đa là $4$ kg. Hoa xếp cả hai loại sách (loại $1$ và loại $2$) vào giá. Sách loại $1$ có khối lượng $100$ gam mỗi cuốn và sách loại $2$ có khối lượng $200$ gam mỗi cuốn. Viết bất phương trình bậc nhất hai ẩn cho khối lượng của $x$ cuốn loại $1$ và $y$ cuốn loại $2$ có thể được xếp lên giá sách.
	\loigiai{
		Ta có $4$ kg $=4000$ gam.\\
		Khối lượng của $x$ cuốn sách loại $1$ là $100x$ gam.
		Khối lượng của $y$ cuốn sách loại $2$ là $200y$ gam.\\
		Hoa xếp cả hai loại sách nên $x, y\in\mathbb{N}^*$.
		Vì giá sách của Hoa có thể chứa được khối lượng sách tối đa là $4$ kg nên số cuốn sách ($x$ cuốn loại $1$ và $y$ cuốn loại $2$) có thể được xếp lên giá sách thỏa mãn bất phương trình 
		$100x+200y\le 4000\Leftrightarrow x+2y\le 40$.
	}
\end{bt}

\begin{bt}%[Nguyện Ngô]%[0D4B4]
	Công ty viễn thông Mobifone tính phí $1$ nghìn đồng mỗi phút gọi nội mạng, $2$ nghìn đồng mỗi phút gọi ngoại mạng. Mỗi tháng Minh gọi điện thoại hết từ $200$ đến $300$ nghìn đồng. Viết bất phương trình bậc nhất hai ẩn mô tả cho số tiền điện thoại trả cho ($x$) phút gọi nội mạng và ($y$) phút gọi ngoại mạng trong một tháng.
	\loigiai{
		Số tiền điện thoại trả cho $x$ phút gọi nội mạng là $x$ nghìn đồng.\\
		Số tiền điện thoại trả cho $y$ phút gọi nội mạng là $2y$ nghìn đồng.\\
		Mỗi tháng Minh gọi điện thoại hết từ $200$ đến $300$ nghìn đồng nên ta có 
		\[200\le x+2y\le 300.\]
	}
\end{bt}

\begin{bt}%[Nguyện Ngô]%[0D4K4]
	Bạn An giải $10$ bài Toán trong $20$ phút thì đúng được $80\%$ số bài Toán, giải $12$ bài Lý trong $15$ phút thì đúng được $\dfrac{3}{4}$ số bài Lý. Viết bất phương trình bậc nhất hai ẩn cho thời gian giải $x$ bài Toán đúng và $y$ bài Lý đúng, biết thời gian giải ít hơn $150$ phút.   
	\loigiai{
		Sau $20$ phút An làm đúng được $10\cdot 80\%=8$ bài Toán.\\
		Suy ra thời gian An giải đúng $x$ bài Toán là $\dfrac{20x}{8}=\dfrac{5x}{2}$ phút.\\
		Sau $15$ phút An làm đúng được $12\cdot \dfrac{3}{4}=9$ bài Lý.\\
		Suy ra thời gian An giải đúng $y$ bài Lý là $\dfrac{15y}{9}=\dfrac{5y}{3}$ phút.\\
		Vì thời gian giải ít hơn $150$ phút nên ta có 
		\[\dfrac{5x}{2}+\dfrac{5x}{3}<150\Leftrightarrow 3x+2y<180.\] 
	}
\end{bt}

\begin{bt}%[Nguyện Ngô]%[0D4K4]
	Một gian hàng trưng bày bàn và ghế rộng $100$ m$^2$. Diện tích để kê một chiếc ghế là $1$ m$^2$, một chiếc bàn là $2$ m$^2$ và diện tích mặt sàn dành cho lưu thông tối thiểu là $24$ m$^2$. Gọi $x$ là số chiếc ghế, $y$ là số chiếc bàn được kê, hãy viết bất phương trình bậc nhất hai ẩn $x$, $y$ cho phần mặt sàn để kê bàn và ghế và chỉ ra hai nghiệm của bất phương trình.
	\loigiai{
		Diện tích kê $x$ chiếc ghế là $x$ m$^2$, ($x\in\mathbb{N^*}$).\\
		Diện tích kê $y$ chiếc ghế là $2y$ m$^2$, ($y\in\mathbb{N^*}$).\\
		Diện tích mặt sàn tối đa có thể kê bàn, ghế là $100-24=76$ m$^2$.\\
		Do đó ta có bất phương trình $x+2y\le 76$.\\
		Cho $x=26$, ta có $26+2y\le 76\Leftrightarrow y\le 25$.\\ 
		Lần lượt chọn $y=23$, $y=24$ ta được hai nghiệm của bất phương trình là $(26;23)$ và $(26;24)$.
	}
\end{bt}

\begin{bt}%[Nguyện Ngô]%[0D4K4]
	Một rạp chiếu phim $2$D phục vụ khán giả một bộ phim mới với $2$ loại vé khác nhau. Vé loại $1$ (từ thứ $2$ đến thứ $5$) giá $80000$ đồng/vé, vé loại $2$ (từ thứ $6$ đến chủ nhật và ngày lễ) giá $100000$ đồng/vé. Để không phải bù lỗ thì số tiền vé thu được ở rạp chiếu phim này phải đạt tối thiểu $150$ triệu đồng. Hỏi số lượng vé bán được trong những trường hợp nào thì rạp chiếu phim phải bù lỗ?
	\loigiai{
		Gọi $x$, $y$ lần lượt là số vé loại $1$, loại $2$ bán được ($x, y\in\mathbb{N}$).\\
		Tổng số tiền bán vé là $80x+100y$ nghìn đồng.\\
		Rạp chiếu phim phải bù lỗ trong trường hợp số tiền bán vé nhỏ hơn $150$ triệu đồng, tức là 
		\[80x+100y<150000\Leftrightarrow 4x+5y<7500.\quad\quad (1)\]
		\immini{
			Miền nghiệm của bất phương trình (1) được xác định như sau\\
			+/ Vẽ đường thẳng $d\colon 4x+5y=7500$.\\
			+/ Chọn gốc tọa độ $O(0;0)$ và tính $4\cdot0+5\cdot0<7500$.\\
			Do đó miền nghiệm của bất phương trình (1) là nửa mặt phẳng bờ $d$, chứa gốc tọa độ $O$, không kể đường thẳng $d$.\\
			Gọi $A$, $B$ lần lượt là giao điểm của $d$ và $Ox$, $Oy$. Khi đó, nếu bán được $x$ vé loại $1$ và $y$ vé loại $2$ mà điểm $(x;y)$ nằm trong miền tam giác $OAB$ không kể cạnh $AB$ thì rạp chiếu phim sẽ phải bù lỗ.
		}
		{
			\begin{tikzpicture}[scale=.7,>=stealth]
				\draw[->] (0,0) -- (5.3,0)node[below]{$x$};
				\draw[->,color=black] (0,0) -- (0,5.3)node[left]{$y$};
				\node[below left] at (0,0){$O$};
				\node[left] at (0,3){$1500$};
				\node[above right] at (0,3){B};
				\node[below] at (4,0){$1875$};
				\node[above] at (4,0){A};
				\clip(0,0) rectangle (5.3,5.3);
				\fill[pattern=north east lines](4,0)-- (5.3,0) -- (5.3,5.3) -- (0,5.3)--(0,3)-- cycle;
				\draw[line width=1.2pt,smooth,samples=100,domain=0:4] plot(\x,{-0.75 *(\x) +3});
			\end{tikzpicture}
		}	
	}
\end{bt}


\begin{bt}%[Nguyện Ngô]%[0D4K4]
	Một bác nông dân cần trồng lúa và khoai trên diện tích đất $6$ ha, với lượng phân bón dự trữ là $100$ kg và sử dụng tối đa $120$ ngày công. Để trồng $1$ ha lúa cần sử dụng $20$ kg phân bón, $10$ ngày công với lợi nhuận là $30$ triệu đồng; để trồng $1$ ha khoai cần sử dụng $10$ kg phân bón, $30$ ngày công với lợi nhuận là $60$ triệu đồng. Biết bác nông dân đã trồng $x$ (ha) lúa và $y$ (ha) khoai. Tìm giá trị của $x$ để bác nông dân đạt được lợi nhuận cao nhất.
	\loigiai{
		Theo bài toán, ta có:\\
		$ \heva{& x+y=6\\&20x+10y\leq 100\\&10x+30y\leq 120\\&
			T=30x+60y \longrightarrow Max}$ 
		$\Leftrightarrow \heva{& y=6-x\\&x\leq 4\\ &x\geq 3\\& T=24x+360 \longrightarrow Max}$
		$\Leftrightarrow \heva{&y=6-x\\&3\leq x\leq 4\\& T=24x+360 \longrightarrow Max.}$\\
		Vì $T=24x+360$ là hàm số bậc nhất và có hệ số $a=24>0$ nên $T$ đạt GTLN tại $x=4$.\\
		Vậy $x=4$ là giá trị cần tìm.
	}
\end{bt}
\subsection{Câu hỏi trắc nghiệm}
% \Opensolutionfile{ansbook}[ans/ansbook-BPTbacnhathaian]
\Opensolutionfile{ans}[ans/ans-BPTbacnhathaian]
\begin{ex}%[0D4Y4-1]%
	Trong các bất phương trình sau đây, đâu là bất phương trình bậc nhất hai ẩn
	\choice
	{$2x^2-3x \geq 1$}
	{\True $2x+y\leq 1$}
	{$3x+1\leq 0$}
	{$3x+y=1$}
	\loigiai{
		Theo định nghĩa $2x+y\leq 1$ là bất phương trình bậc nhất hai ẩn.
	}
\end{ex}
\begin{ex}%[Dự án Tex Khối 10-11 W-T-Begin lần 4]%[Biên soạn: Tuan Nguyen, Phản biện: Phan Văn Thành]%[0D4B4-1]%Câu 1
	Cho bất phương trình $2x+3y-6 \leq 0\quad (1)$. Chọn khẳng định đúng trong các khẳng định sau.
	\choice
	{Bất phương trình $(1)$ chỉ có một nghiệm duy nhất}
	{Bất phương trình $(1)$ vô nghiệm}
	{\True Bất phương trình $(1)$ luôn có vô số nghiệm}
	{Bất phương trình $(1)$ có tập nghiệm là $\mathbb{R}$}
	\loigiai{
		Trên mặt phẳng tọa độ, đường thẳng $(d) \colon 2x+3y-6=0$ chia mặt phẳng thành hai nửa mặt phẳng.\\
		Chọn điểm $O(0;0)$ không thuộc đường thẳng đó. Ta thấy $(x;y)=(0;0)$ là nghiệm của bất phương trình đã cho.\\ Vậy miền nghiệm của bất phương trình là nửa mặt phẳng bờ $(d)$ chứa điểm $O(0;0)$ kể cả $(d)$.\\
		Vậy bất phương trình $(1)$ luôn có vô số nghiệm.}
\end{ex}
\begin{ex}%[Dự án Tex Khối 10-11 W-T-Begin lần 4]%[Biên soạn: Tuan Nguyen, Phản biện: Phan Văn Thành]%[0D4B4-1]%Câu 5
	Trong các cặp số sau đây, cặp nào \textbf{không} là nghiệm của bất phương trình $x-4y+1 \geq 0$?
	\choice
	{$(-1;0)$}
	{$(-2;-1)$}
	{\True $(-1;3)$}
	{$(0;0)$}
	\loigiai{
		Ta có $(-1)-4\cdot 3+1\ge 0$ là mệnh đề sai nên cặp số $(-1;3)$ không là nghiệm của của bất phương trình trên.
	}
\end{ex}
\begin{ex}%[Dự án Tex hóa đề thi lớp 10-11-Nhóm Word-T-Begin]%[Nguyễn Trung Kiên. Phản biện: Trần Nhân Kiêt]%[0D4Y4-1]%
	Miền nghiệm của bất phương trình $4(x-1)+5(y-3)>2x-9$ là nửa mặt phẳng chứa điểm nào?
	\choice
	{$(0;0)$}
	{$(1;1)$}
	{$(-1;1)$}
	{\True $(2;5)$}
	\loigiai{
		Ta có $4(x-1)+5(y-3)>2x-9\Leftrightarrow 4x-4+5y-15>2x-9\Leftrightarrow 2x+5y-10>0$.\\
		Dễ thấy tại điểm $(2;5)$ ta có $2\cdot 2+5\cdot 5-10>0$ (đúng).}
\end{ex}
\begin{ex}%[0D4B4-1]%
	Điểm nào sau đây thuộc miền nghiệm của bất phương trình $x+y-2>0$?
	\choice
	{\True $(2;1)$}
	{$(0;0)$ }
	{$(1;0)$ }
	{$(0;1)$ }
	\loigiai{
		\immini{
			Tập hợp các điểm biểu diễn nghiệm của bất phương trình $x+y-2>0$  là nửa mặt phẳng bờ là đường thẳng $y=x+2$  và không chứa gốc tọa độ.
			Từ đó ta có điểm $(2;1)$  thuộc miền nghiệm của bất phương trình.
		}
		{\begin{tikzpicture}[>=stealth, scale=0.7]
				\draw[->,line width = 1pt] (-2.5,0)--(0,0) node[below right]{$O$}--(4,0) node[below]{$x$};
				\draw[->,line width = 1pt] (0,-2.5) --(0,3.5) node[right]{$y$};
				\foreach \x in {-2,-1,1,2,3}{
					\draw (\x,0) node[below]{$\x$} circle (1pt);
					\draw (0,\x) node[left]{$\x$} circle (1pt);
				}
				\draw [pattern = horizontal lines, thick, domain=-1:4.0, samples=100] plot (\x, {-(\x)+2}) node[right]{$d$};
				\draw[pattern = horizontal lines,opacity=.3, line width = 1.2pt,draw=none] plot[domain=-1:4.0] (\x, {-(\x)+2})--(-2,-2)--(-2,3)--cycle;
				\clip (-2.5,-2.5) rectangle (4.0,3.5);
				\draw (2,1) node[right]{$M(2;1)$} circle(2pt);
				\draw[dashed] (2,0)--(2,1)--(0,1);
			\end{tikzpicture}
		}
	}
\end{ex}
\begin{ex}%[0D4Y4-1]%
	Điểm $A(-1;3)$ thuộc miền của bất phương trình
	\choice
	{$x+3y<0$}
	{$3x-y>0$}
	{\True  $-3x+2y-4>0$}
	{$2x-y+4>0$}
	\loigiai{
		Thay tọa độ $A(-1;3)$ vào các bất phương trình:
		\begin{itemize}
			\item[•] Với bất phương trình $x+3y<0$, ta có $(-1)+3\cdot 3<0$ sai.
			\item[•] Với bất phương trình $3x-y>0$, ta có $3\cdot (-1)-3>0$ sai.
			\item[•] Với bất phương trình $-3x+2y-4>0$, ta có $-3\cdot (-1)+2\cdot 3-4>0$ đúng.
			\item[•] Với bất phương trình $2x-y+4>0$, ta có $2\cdot (-1)-3+4>0$ sai.
		\end{itemize}
		Vậy $A(-1;3)$ thuộc miền nghiệm bất phương trình $-3x+2y-4>0$.
	}
\end{ex}
\begin{ex}%[Nguyễn Trung Hiếu]%[781-810 Phạm Quốc Toàn]%[0D4K4-1]%
	Tìm tất cả các số thực $a$ sao cho miền nghiệm của bất phương trình $x\le a$ chứa điểm $M(-1;0)$.
	\choice
	{$a>-1$}
	{\True $a \ge -1$}
	{$a>0$}
	{$a\ge 0$}
	\loigiai{Để $M(-1;0)$ thuộc miền nghiệm của bất phương trình $x\le a$ thì $a \geq -1$.
	}
\end{ex}
\begin{ex}%[0-GHK2-2021, THPT Nguyễn Trường Tộ, 2020-2021]%[Vô Văn Tự]%[0D4B4-1]%
	Cho đường thẳng $d\colon 7x-9y+2=0$ chia mặt phẳng toạ độ làm hai nửa  mặt phẳng, trong đó miền nghiệm của bất phương trình $7x-9y+2>0$ là nửa mặt phẳng
	\choice
	{có bờ là đường thẳng $d$ và không chứa điểm $O(0;0)$}
	{\True không có bờ $d$ và chứa điểm $O(0;0)$}
	{có bờ là đường thẳng $d$ và chứa điểm $O(0;0)$}
	{không chứa bờ $d$ và không chứa điểm $O(0;0)$}
	\loigiai{
		Ta có toạ độ điểm $O(0;0)$ thoả mãn bất phương trình $7x-9y+2>0$ nên miền nghiệm của bất phương trình $7x-9y+2>0$ là nửa mặt phẳng không có bờ $d$ và chứa điểm $O(0;0)$.
	}
\end{ex}
\begin{ex}%[Word: Nguyễn Văn Mến, LaTeX: Nguyễn Tài Tuệ, PB: Nguyễn Tấn Linh]%[0D4B4-1]%
	\immini{ Phần gạch chéo trong hình vẽ dưới đây (không bao gồm đường thẳng d) là miền nghiệm cuả bất phương trình bậc nhất hai ẩn nào sau đây?
		\choice
		{$2x-y<0$}
		{$x-2y<2$}
		{\True $2y-x<-2$}
		{$2x-y>1$}}{\begin{tikzpicture}[line join=round, line cap=round,>=stealth]
			\tikzset{label style/.style={font=\footnotesize}}
			\begin{scope}
				\clip (-2.5,-3) rectangle (3,2);
				\fill[pattern=north east lines] (-4.5,-3.25)--(7,-3.25)--(7,2.5)--cycle;
				\draw (6,2)--(-4,-3) node [pos=0.45, above, sloped] {};
			\end{scope}
			\draw[->] (-2.5,0)--(3,0) node[below]{$x$};
			\draw[->] (0,-3)--(0,2) node[left]{$y$};
			\draw (0,0) node[below left]{$O$};
			\foreach \x in {2}
			\draw[thin] (\x,1pt)--(\x,-1pt) node [below] {$\x$};
			\foreach \y in {-1}
			\draw[thin] (1pt,\y)--(-1pt,\y) node [left] {$\y$};
	\end{tikzpicture}}
	\loigiai{
		%Fb tác giả: Nguyễn Văn Mến\\
		Đường thẳng d đi qua hai điểm $A(0;-1)$ và $B(2;0)$ nên có phương trình là $y=\dfrac{1}{2}x-1$.\\
		Lại có điểm $O(0;0)$ không thuộc vào miền nghiệm nên $y<\dfrac{1}{2}x-1$ (vì $0<\dfrac{1}{2} \cdot 0-1$ \textbf{không đúng}).\\
		Hay $2y<x-2 \Leftrightarrow 2y-x<-2$.}
\end{ex}
\begin{ex}%[0D4K4-1]%
	\immini{ Bất phương trình nào sau đây có miền nghiệm (phần không gạch sọc) như hình vẽ bên?
		\choice
		{\True $2x-y+1<0$}
		{$x-y+1<0$ }
		{$2x-3y+1<0$ }
		{$2x-y-1<0$ }
	}
	{
		\begin{tikzpicture}[>=stealth, scale=0.7]
			\draw[->,line width = 1pt] (-3,0)--(0,0) node[below right]{$O$}--(4,0) node[below]{$x$};
			\draw[->,line width = 1pt] (0,-2) --(0,3.5) node[right]{$y$};
			\foreach \x in {-2,-1,1,2,3}{
				\draw (\x,0) node[below]{$\x$} circle (1pt);
				\draw (0,\x) node[left]{$\x$} circle (1pt);
			}
			\draw [pattern = north west lines, thick, domain=-1.5:1, samples=100] plot (\x, {2*(\x)+1}) node[right]{$d$};
			\draw[pattern = north east lines,opacity=.3, line width = 1.2pt,draw=none] plot[domain=-1.5:1] (\x, {2*(\x)+1})--(3,3)--(3,-2)--cycle;
		\end{tikzpicture}
	}
	\loigiai{Tập hợp các điểm biểu diễn nghiệm của bất phương trình $2x-y+1<0$ là nửa mặt phẳng bờ là đường thẳng $y=2x+1$ và không chứa gốc tọa độ.
		Từ đó ta có điểm $(2;1)$ thuộc miền nghiệm của bất phương trình.
	}
\end{ex}
\begin{ex}%[0D4B4-1]%
	Miền nghiệm của bất phương trình $x+y \leq 2$ là phần không bị gạch sọc của hình vẽ nào trong các hình sau?
	\choice
	{
		\begin{tikzpicture}[scale=1, font=\footnotesize, line join=round, line cap=round, >=stealth]
			\def\xmin{-1}\def\xmax{3.0}\def\ymin{-1}\def\ymax{3.0}
			\draw[->] (\xmin-0.2,0)--(\xmax+0.2,0) node[below] {\footnotesize $x$};
			\draw[->] (0,\ymin-0.2)--(0,\ymax+0.2) node[right] {$y$};
			\draw (0,0) node [below left] {$O$};
			\foreach \x in {-1,1,2,3}\draw (\x,0.1)--(\x,-0.1) node [below] {\footnotesize $\x$};
			\foreach \y in {-1,1,2,3}\draw (0.1,\y)--(-0.1,\y) node [left] {\footnotesize $\y$};
			\clip (\xmin,\ymin) rectangle (\xmax,\ymax);
			\draw[pattern = north west lines,smooth,samples=200,domain=\xmin:\xmax] plot (\x,{-1*(\x)+2});
			\draw[pattern = north east lines,opacity=.3, line width = 1.2pt,draw=none] plot[domain=\xmin:\xmax] (\x, {-1*(\x)+2})--(-1,-1)--(-1,3)--cycle;
		\end{tikzpicture}
	}
	{
		\begin{tikzpicture}[scale=1, font=\footnotesize, line join=round, line cap=round, >=stealth]
			\def\xmin{-3.0}\def\xmax{1}\def\ymin{-1}\def\ymax{3.0}
			\draw[->] (\xmin-0.2,0)--(\xmax+0.2,0) node[below] {\footnotesize $x$};
			\draw[->] (0,\ymin-0.2)--(0,\ymax+0.2) node[right] {$y$};
			\draw (0,0) node [below left] {$O$};
			\foreach \x in {-3,-2,-1,1}\draw (\x,0.1)--(\x,-0.1) node [below] {\footnotesize $\x$};
			\foreach \y in {-1,1,2,3}\draw (0.1,\y)--(-0.1,\y) node [left] {\footnotesize $\y$};
			\clip (\xmin,\ymin) rectangle (\xmax,\ymax);
			\draw[pattern = north west lines,smooth,samples=200,domain=\xmin:\xmax] plot (\x,{1*(\x)+2});
			\draw[pattern = north east lines,opacity=.3, line width = 1.2pt,draw=none] plot[domain=\xmin:\xmax] (\x, {1*(\x)+2})--(1,3)--(1,-1)--cycle;
		\end{tikzpicture}
	}
	{\True
		\begin{tikzpicture}[scale=1, font=\footnotesize, line join=round, line cap=round, >=stealth]
			\def\xmin{-1}\def\xmax{3.0}\def\ymin{-1}\def\ymax{3.0}
			\draw[->] (\xmin-0.2,0)--(\xmax+0.2,0) node[below] {\footnotesize $x$};
			\draw[->] (0,\ymin-0.2)--(0,\ymax+0.2) node[right] {$y$};
			\draw (0,0) node [below left] {$O$};
			\foreach \x in {-1,1,2,3}\draw (\x,0.1)--(\x,-0.1) node [below] {\footnotesize $\x$};
			\foreach \y in {-1,1,2,3}\draw (0.1,\y)--(-0.1,\y) node [left] {\footnotesize $\y$};
			\clip (\xmin,\ymin) rectangle (\xmax,\ymax);
			\draw[pattern = north west lines,smooth,samples=200,domain=\xmin:\xmax] plot (\x,{-1*(\x)+2});
			\draw[pattern = north east lines,opacity=.3, line width = 1.2pt,draw=none] plot[domain=\xmin:\xmax] (\x, {-1*(\x)+2})--(3,3)--(-1,3)--cycle;
		\end{tikzpicture}
	}
	{
		\begin{tikzpicture}[scale=1, font=\footnotesize, line join=round, line cap=round, >=stealth]
			\def\xmin{-3.0}\def\xmax{1}\def\ymin{-1}\def\ymax{3.0}
			\draw[->] (\xmin-0.2,0)--(\xmax+0.2,0) node[below] {\footnotesize $x$};
			\draw[->] (0,\ymin-0.2)--(0,\ymax+0.2) node[right] {$y$};
			\draw (0,0) node [below left] {$O$};
			\foreach \x in {-3,-2,-1,1}\draw (\x,0.1)--(\x,-0.1) node [below] {\footnotesize $\x$};
			\foreach \y in {-1,1,2,3}\draw (0.1,\y)--(-0.1,\y) node [left] {\footnotesize $\y$};
			\clip (\xmin,\ymin) rectangle (\xmax,\ymax);
			\draw[pattern = north west lines,smooth,samples=200,domain=\xmin:\xmax] plot (\x,{1*(\x)+2});
			\draw[pattern = north east lines,opacity=.3, line width = 1.2pt,draw=none] plot[domain=\xmin:\xmax] (\x, {1*(\x)+2})--(-3,3)--(-3,-1)--cycle;
		\end{tikzpicture}
	}
	\loigiai{
		\immini{
			Biểu diễn miền nghiệm trên mặt phẳng $Oxy$:\\
			- Vẽ đường thẳng $d: x+y=2$.\\
			- Lấy điểm $O(0;0)$ thay tọa độ vào ta có $0+0 \leq 2$ đúng.\\
			Vậy miền nghiệm bất phương trình là nửa mặt phẳng chứa điểm $O(0;0)$ và có bờ là đường thẳng $d$, kể cả đường thẳng $d$.
		}{
			\begin{tikzpicture}[scale=1, font=\footnotesize, line join=round, line cap=round, >=stealth]
				\def\xmin{-1}\def\xmax{3.0}\def\ymin{-1}\def\ymax{3.0}
				\draw[->] (\xmin-0.2,0)--(\xmax+0.2,0) node[below] {\footnotesize $x$};
				\draw[->] (0,\ymin-0.2)--(0,\ymax+0.2) node[right] {$y$};
				\draw (0,0) node [below left] {$O$};
				\foreach \x in {-1,1,2,3}\draw (\x,0.1)--(\x,-0.1) node [below] {\footnotesize $\x$};
				\foreach \y in {-1,1,2,3}\draw (0.1,\y)--(-0.1,\y) node [left] {\footnotesize $\y$};
				\clip (\xmin,\ymin) rectangle (\xmax,\ymax);
				\draw[pattern = north west lines,smooth,samples=200,domain=\xmin:\xmax] plot (\x,{-1*(\x)+2});
				\draw[pattern = north east lines,opacity=.3, line width = 1.2pt,draw=none] plot[domain=\xmin:\xmax] (\x, {-1*(\x)+2})--(3,3)--(-1,3)--cycle;
			\end{tikzpicture}
		}
	}
\end{ex}
\begin{ex}%[0D4K4-1]%Câu 24%[Dự án Tex hóa đề thi lớp 10-11-Nhóm Word-T-Begin-Lần 4]%[Lê Quốc Dũng\& Phản biện: Thanh Phong]%
	Cho bất phương trình $2x+3y-2<0$. Miền nghiệm của bất phương trình là
	\choice
	{\True nửa mặt phẳng chứa điểm $O$ có bờ là đường thẳng $2x+3y-2=0$ (không kể bờ)}
	{nửa mặt phẳng chứa điểm $O$ có bờ là đường thẳng $2x+3y-2=0$ (kể cả bờ)}
	{nửa mặt phẳng không chứa điểm $O$ có bờ là đường thẳng $2x+3y-2=0$ (không kể bờ)}
	{nửa mặt phẳng không chứa điểm $O$ có bờ là đường thẳng $2x+3y-2=0$ (kể cả bờ)}
	\loigiai{
		\immini{Vẽ đường thẳng $2x+3y-2=0$.\\
			Xét điểm $O(0;0)$ không thuộc đường thẳng $2x+3y-2=0$.\\
			Ta có $P=2 \cdot 0+3 \cdot 0-2<0$.\\
			Vậy nửa mặt phẳng chứa điểm $O$ có bờ là đường thẳng $2x+3y-2=0$ (không kể bờ) là miền nghiệm của bất phương trình.}{
			\begin{tikzpicture}[scale=0.8, font=\footnotesize, line join=round, line cap=round, >=stealth]
				\clip(-2,-1) rectangle (3,2);
				\draw[line width=0.8pt,dash pattern=on 3pt off 3pt,fill=black,pattern=north east lines,pattern color=black](-4.08,3.38)--(-4.08,-7.81)--(4.74,-7.81)--(4.74,-2.49)--(-4.08,3.38);
				\draw [->,line width=0.4pt] (-2,0) -- (3,0);
				\draw [->,line width=0.4pt] (0,-1) -- (0,2);
				\begin{scriptsize}
					\draw (-0.3,1.8) node {$y$};
					\draw (2.95,-0.2) node {$x$};
					\draw (-0.3,-0.3) node {$O$};
					\draw [fill=black] (1,0) circle (1pt);
					\draw (1,0.25 ) node {$1$};
					\draw (0,1) node[right] {$2x+3y-2<0$};
				\end{scriptsize}
	\end{tikzpicture}}}
\end{ex}
\begin{ex} %[Trần Ngọc Lam]%[1181-1200 Trần Chiến; PB: Nguyễn Tài Tuệ]%[0D4K4-1]%
	Miền nghiệm của bất phương trình $x-2y+5<0$ là
	\choice
	{ \True Nửa mặt phẳng không chứa gốc tọa độ, bờ là đường thẳng $y=\dfrac{1}{2}x+\dfrac{5}{2}$ (không bao gồm đường thẳng)}
	{ Nửa mặt phẳng chứa gốc tọa độ, bờ là đường thẳng $y=\dfrac{1}{2}x+\dfrac{5}{2}$ (không bao gồm đường thẳng)}
	{ Nửa mặt phẳng không chứa gốc tọa độ, bờ là đường thẳng $y=\dfrac{1}{2}x+\dfrac{5}{2}$ (bao gồm đường thẳng)}
	{ Nửa mặt phẳng chứa gốc tọa độ, bờ là đường thẳng $y=\dfrac{1}{2}x+\dfrac{5}{2}$ (không bao gồm đường thẳng)}
	\loigiai{
		\immini{
			Thay tọa độ điểm $ O(0;0) $ vào phương trình đường thẳng ta thấy không thỏa mãn.\\
			Do đó miền nghiệm là nửa mặt phẳng không chứa gốc tọa độ bờ là đường thẳng $ y=\dfrac{1}{2}x+\dfrac{5}{2} $( không bao gồm đường thẳng, như hình vẽ).}{
			\begin{tikzpicture}[thick,>=stealth,scale=0.7]
				\draw[->] (-4,0) -- (3,0) node[below]{\small $x$};
				\draw[->] (0,-2) -- (0,3) node[right]{\small $y$};
				\foreach \x in {1}
				\draw[shift={(\x,0)}] (0pt,2pt) -- (0pt,-2pt) node[right] {\footnotesize $\x$};
				\draw (0pt,-10pt) node[right] {\footnotesize $O$};
				\fill[black] (0,0) circle(2pt);
				\clip(-4,-2) rectangle (3,3);
				\draw[very thick, blue, smooth, domain=-4:3]
				plot(\x,{(\x+5)/2});
				\fill[pattern=north west lines] (-4,0.5)--(-4,3)--(1,3)--cycle;
			\end{tikzpicture}
		}
		
	}
\end{ex}
\begin{ex}%[Đỗ Vũ Minh Thắng]%[751-780 Lê Quốc Bảo]%[0D4K4-1]%
	Cặp điểm nào sau đây thuộc miền nghiệm của bất phương trình $3(x+\sqrt{2}y-\sqrt{3})>8(\sqrt{3}x+2y-\sqrt{2})$?
	\choice{$A(2;-2)$ và $B(2;2)$}
	{\True $C(-\sqrt{3};-\sqrt{2})$ và $D(\sqrt{2};-1-\sqrt{5})$}
	{$E(\sqrt{2};\sqrt{2})$ và $F(\sqrt{5}; 1)$}
	{$G(-\sqrt{2};2+\sqrt{3})$ và $H(1;4)$}
	\loigiai{Ta có $3(x+\sqrt{2}y-\sqrt{3})>8(\sqrt{3}x+2y-\sqrt{2}) \Leftrightarrow \left(3-8\sqrt{3}\right) x + \left(3\sqrt{2}-16\right) y -3\sqrt{3}+8\sqrt{2} >0$.\\
		Thay điểm $C(-\sqrt{3};-\sqrt{2})$ vào bất phương trình trên, ta có
		$$\left(3-8\sqrt{3}\right) \cdot (-\sqrt{3}) + \left(3\sqrt{2}-16\right) \cdot (-\sqrt{2}) -3\sqrt{3}+8\sqrt{2} = 18 - 6\sqrt{3} + 24\sqrt{2} >0 \text{ (đúng).}$$
		Thay điểm $D(\sqrt{2};-1-\sqrt{5})$ vào bất phương trình trên, ta có
		$$\left(3-8\sqrt{3}\right) \cdot (\sqrt{2}) + \left(3\sqrt{2}-16\right) \cdot (-1-\sqrt{5}) -3\sqrt{3}+8\sqrt{2}>0 \text{ (đúng).}$$
		Nên cặp điểm $C$, $D$ thuộc miền nghiệm của bất phương trình trên.
	}
\end{ex}
\begin{ex}%[0D4K4-1]%
	Giao miền nghiệm của ba bất phương trình $y\geq 0; 3x-2y\geq -6; 3x+4y\leq 12$ tạo thành một tam giác có diện tích bằng
	\choice
	{$18$}
	{\True $9$}
	{$6$}
	{$12$}
	\loigiai{
		\immini{
			Vẽ các đường thẳng $d_1: y=0; d_2: 3x-2y=6; d_3: 3x+4y=12$.\\
			- Lấy điểm $O(0;0)$ thế vào vế trái $d_2$ ta được $3\cdot 0-2\cdot 0 \geq -6$ đúng. Vậy miền nghiệm bất phương trình $3x-2y\geq -6$ chứa $O$ có bờ là $d_2$.\\
			- Lấy điểm $O(0;0)$ thế vào vế trái $d_3$ ta được $3\cdot 0+4\cdot 0 \leq 12$ đúng. Vậy miền nghiệm bất phương trình $3x+4y\leq 12$ chứa $O$ có bờ là $d_3$.\\
			Gọi $A, B, C$ là ba đỉnh của tam giác. Ta có $A(-2;0);B(0;3),C(4;0)$.\\
			Ta có $BO=3;AC=6$. Diện tích tam giác $ABC$ là
			\[S=\dfrac{1}{2}BO\cdot AC = \dfrac{1}{2}\cdot 3\cdot 6 = 9. \]
		}{
			\begin{tikzpicture}[scale=1, font=\footnotesize, line join=round, line cap=round, >=stealth]
				\def\xmin{-3.0}\def\xmax{5.0}\def\ymin{-1.0}\def\ymax{4.0}
				\draw[->] (\xmin-0.2,0)--(\xmax+0.2,0) node[below] {\footnotesize $x$};
				\draw[->] (0,\ymin-0.2)--(0,\ymax+0.2) node[right] {$y$};
				\draw (0,0) node [below left] {$O$};
				\foreach \x in {-3,-2,-1,1,2,3,4,5}\draw (\x,0.1)--(\x,-0.1) node [below] {\footnotesize $\x$};
				\foreach \y in {-1,1,2,3,4}\draw (0.1,\y)--(-0.1,\y) node [left] {\footnotesize $\y$};
				\clip (\xmin,\ymin) rectangle (\xmax,\ymax);
				\draw[pattern = north west lines,smooth,samples=200,domain=\xmin:\xmax] plot (\x,{1.5*(\x)+3});
				\draw[pattern = north east lines,opacity=.3, line width = 1.2pt,draw=none] plot[domain=\xmin:\xmax] (\x, {1.5*(\x)+3})--(-3,4)--(-3,-1)--cycle;
				\draw[pattern = north west lines,smooth,samples=200,domain=\xmin:\xmax] plot (\x,{-0.75*(\x)+3});
				\draw[pattern = north east lines,opacity=.3, line width = 1.2pt,draw=none] plot[domain=\xmin:\xmax] (\x, {-0.75*(\x)+3})--(5,-1)--(5,4)--cycle;
				\draw[pattern = north west lines,smooth,samples=200,domain=\xmin:\xmax] plot (\x,{0*(\x)});
				\draw[pattern = north east lines,opacity=.3, line width = 1.2pt,draw=none] plot[domain=\xmin:\xmax] (\x, {0*(\x)})--(5,-1)--(-3,-1)--cycle;
			\end{tikzpicture}
		}
	}
\end{ex}
\begin{ex}%[0D4K4-1]%
	Giao miền nghiệm của ba bất phương trình $x+4y\geq 8; -x+2y\leq 4; x+y\leq 5$ tạo thành một tam giác có chu vi bằng
	\choice
	{\True $\sqrt{17}+\sqrt{5}+2\sqrt{2}$}
	{$\sqrt{17}+\sqrt{5}+\sqrt{2}$}
	{$\sqrt{17}+2\sqrt{5}+\sqrt{2}$}
	{$\sqrt{17}+2\sqrt{5}+2\sqrt{2}$}
	\loigiai{
		\immini{
			Vẽ các đường thẳng $d_1: x+4y=8; d_2: -x+2y=4; d_3: x+y=5$.\\
			- Lấy điểm $O(0;0)$ thế vào vế trái $d_1$ ta được $3\cdot 0+4\cdot 0 \geq 8$ sai. Vậy miền nghiệm bất phương trình $x+4y\geq 8$ không chứa $O$ có bờ là $d_1$.\\
			- Lấy điểm $O(0;0)$ thế vào vế trái $d_2$ ta được $-0+2\cdot 0 \leq 4$ đúng. Vậy miền nghiệm bất phương trình $3-x+2y\leq 4$ chứa $O$ có bờ là $d_2$.\\
			- Lấy điểm $O(0;0)$ thế vào vế trái $d_3$ ta được $ 0+ 0 \leq 5$ đúng. Vậy miền nghiệm bất phương trình $x+y\leq 5$ chứa $O$ có bờ là $d_3$.\\
			Gọi $A, B, C$ là ba đỉnh của tam giác. Ta có $A(0;2);B(4;1),C(2;3)$.\\
			Ta có:\\
			$AB=\sqrt{(4-0)^2+(1-2)^2}=\sqrt{17}$. \\
			$AC=\sqrt{(2-0)^2+(3-2)^2}=\sqrt{5}$.\\
			$BC=\sqrt{(2-4)^2+(3-1)^2}=2\sqrt{2}$.\\
			Chu vi tam giác $ABC$ là
			\[2P = \sqrt{17}+\sqrt{5}+2\sqrt{2} . \]
		}{
			\begin{tikzpicture}[scale=1, font=\footnotesize, line join=round, line cap=round, >=stealth]
				\def\xmin{-1}\def\xmax{6.0}\def\ymin{-1}\def\ymax{6.0}
				\draw[->] (\xmin-0.2,0)--(\xmax+0.2,0) node[below] {\footnotesize $x$};
				\draw[->] (0,\ymin-0.2)--(0,\ymax+0.2) node[right] {$y$};
				\draw (0,0) node [below left] {$O$};
				\foreach \x in {-1,1,2,3,4,5,6}\draw (\x,0.1)--(\x,-0.1) node [below] {\footnotesize $\x$};
				\foreach \y in {-1,1,2,3,4,5,6}\draw (0.1,\y)--(-0.1,\y) node [left] {\footnotesize $\y$};
				\clip (\xmin,\ymin) rectangle (\xmax,\ymax);
				\draw[pattern = north west lines,smooth,samples=200,domain=\xmin:\xmax] plot (\x,{-1*(\x)+5});
				\draw[pattern = north east lines,opacity=.3, line width = 1.2pt,draw=none] plot[domain=\xmin:\xmax] (\x, {-1*(\x)+5})--(6,6)--(-1,6)--cycle;
				\draw[pattern = north west lines,smooth,samples=200,domain=\xmin:\xmax] plot (\x,{0.5*(\x)+2});
				\draw[pattern = north east lines,opacity=.3, line width = 1.2pt,draw=none] plot[domain=\xmin:\xmax] (\x, {0.5*(\x)+2})--(6,6)--(-1,6)--cycle;
				\draw[pattern = north west lines,smooth,samples=200,domain=\xmin:\xmax] plot (\x,{-0.25*(\x)+2});
				\draw[pattern = north east lines,opacity=.3, line width = 1.2pt,draw=none] plot[domain=\xmin:\xmax] (\x, {-0.25*(\x)+2})--(6,-1)--(-1,-1)--cycle;
				\tkzDefPoint(0,2){A}
				\tkzDefPoint(4,1){B}
				\tkzDefPoint(2,3){C}
				\tkzDrawPoints[color=blue](A,B,C)
				\tkzLabelPoints[right](A)
				\tkzLabelPoints[right](B)
				\tkzLabelPoints[above](C)
				\draw[dashed](2,0)--(2,3)--(0,3);
				\draw[dashed](4,0)--(4,1)--(0,1);
			\end{tikzpicture}
		}
	}
\end{ex}
\begin{ex}%[Đỗ Vũ Minh Thắng]%[751-780 Lê Quốc Bảo]%[0D4K4-1]%
	Tìm tất cả các giá trị thực của tham số $m$ để bất phương trình $3x+my-7 \geq 0$ có miền nghiệm chứa điểm $A(\sqrt{2};1)$.
	\choice{$m\in [3\sqrt{2}-7; +\infty )$}
	{$m\in (-\infty ;3\sqrt{2}-7)$}
	{$m\in (-\infty ;7-3\sqrt{3})$}
	{\True $m\in [7-3\sqrt{2}; +\infty )$}
	\loigiai{Vì điểm $A(\sqrt{2};1)$ thuộc miền nghiệm của bất phương trình đã cho, nên
		$$3 \cdot \sqrt{2} + m \cdot 1 - 7 \geq 0 \Leftrightarrow m \geq 7- 3 \cdot \sqrt{2}.$$
	}
\end{ex}
\begin{ex}%[Đỗ Vũ Minh Thắng]%[781-810 Phạm Quốc Toàn]%[0D4K4-1]%
	%781
	Cho bất phương trình $mx+\sqrt{2}y-1<0$ với $m$ là tham số thực. Điểm nào dưới đây luôn luôn \textbf{không} thuộc miền nghiệm của bất phương trình đã cho?
	\choice
	{$E(m;m^2)$}
	{$F(2m^2;m)$}
	{\True $G(0;1+m^2)$}
	{$H(0;-1-m^2)$}
	\loigiai{Điểm $E (m; m^2)$ không thỏa mãn vì $m^2+\sqrt{2} m^2 -1<0 \Leftrightarrow - \dfrac{1}{\sqrt{1 + \sqrt{2}}} < m <  \dfrac{1}{\sqrt{1 + \sqrt{2}}}$. \\
		Điểm $F(2m^2;m)$ không thỏa mãn vì $2 m^3 + \sqrt{2} m-1<0$ (bất phương trình này luôn có nghiệm). \\
		Điểm $H(0;-1-m^2)$ không thỏa mãn vì $m.0 +\sqrt{2} (-1 - m^2) -1<0 \Leftrightarrow \sqrt{2} m^2 > -1 - \sqrt{2}$ (thỏa mãn với mọi $m$). \\
		Với điểm $G(0;1+m^2)$, ta có $mx+\sqrt{2}y-1=m.0 +\sqrt{2} (1 +m^2) - 1 = \sqrt{2} m^2 + (\sqrt{2} - 1) > 0$ với mọi $m \in \mathbb{R}$. Vậy điểm $G$ không thuộc miền nghiệm của bất phương trình đã cho.
	}
\end{ex}
\begin{ex}%[Nguyễn Phúc Đức]%[781-810 Phạm Quốc Toàn]%[0D4K4-1]%
	%782
	Với giá trị nào của $m$ thì điểm $A(1-m;m)$ {\bf không thuộc} miền nghiệm của bất phương trình $2x-3(y-x)>4$.
	\choice
	{$0\leq m \leq 1$}
	{$m<\dfrac{1}{8}$}
	{$\dfrac{1}{8}\leq m\leq 1$}
	{\True $m \geq \dfrac{1}{8}$}
	\loigiai{ $A(1-m;m)$ không thuộc miền nghiệm của bất phương trình $2x-3(y-x)>4$ khi tọa độ của nó không thỏa mãn bất phương trình, tức là $2(1-m)-3(m+m-1) \leq 4$ hay $m \geq \dfrac{1}{8}$.}
\end{ex}
\begin{ex}%[Bài thi mẫu khảo sát, ĐHQG TP Hồ Chí Minh, 2019]%[Ngụy Như Thái, 12EX4]%[0D4K4-3]%
	Một bác nông dân cần trồng lúa và khoai trên diện tích đất $6$ ha, với lượng phân bón dự trữ là $100$ kg và sử dụng tối đa $120$ ngày công. Để trồng $1$ ha lúa cần sử dụng $20$ kg phân bón, $10$ ngày công với lợi nhuận là $30$ triệu đồng; để trồng $1$ ha khoai cần sử dụng $10$ kg phân bón, $30$ ngày công với lợi nhuận là $60$ triệu đồng. Để đạt lợi nhuận cao nhất, bác nông dân đã trồng $x$ (ha) lúa và $y$ (ha) khoai. Giá trị của $x$ là
	\choice
	{$2$}
	{$3$}
	{\True  $4$}
	{$5$}
	\loigiai{
		Theo bài toán, ta có:\\
		$ \heva{& x+y=6\\&20x+10y\leq 100\\&10x+30y\leq 120\\&
			T=30x+60y \longrightarrow Max}$
		$\Leftrightarrow \heva{& y=6-x\\&x\leq 4\\ &x\geq 3\\& T=24x+360 \longrightarrow Max}$
		$\Leftrightarrow \heva{&y=6-x\\&3\leq x\leq 4\\& T=24x+360 \longrightarrow Max.}$\\
		Vì $T=24x+360$ là hàm số bậc nhất và có hệ số $a=24>0$ nên nó đạt GTLN tại $x=4$.\\
		Vậy $x=4$ là giá trị cần tìm.
	}
\end{ex}

\begin{ex}%[BG Toán 10-2022]%[Trần Nhân Kiệt]%[0D4T4-3]
	Một người thợ mộc tốn $6$ giờ để làm một cái bàn và $4$ giờ để làm một cái ghế. Gọi $x$, $y$ lần lượt là số bàn và số ghế mà người thợ mộc sản xuất trong một tuần. Viết bất phương trình biểu thị mối liên hệ giữa $x$ và $y$ biết trong một tuần người thợ mộc có thể làm tối đa $50$ giờ.
	\choice
	{\True $3x+2y\le 25$}
	{$3x+2y> 25$}
	{$3x+2y\ge 25$}
	{$3x+2y< 25$}
	\loigiai{
		Trong một tuần, số giờ làm ra $x$ cái bàn là $6x$ và số giờ làm ra $y$ cái ghế là $4y$.\\
		Vì trong một tuần người thợ mộc làm tối đa $50$ giờ nên 
		$$6x+4y\le 50\Leftrightarrow 3x+2y\le 25.$$
	}
\end{ex}
\begin{ex}%[BG Toán 10-2022]%[Trần Nhân Kiệt]%[0D4T4-3]
	Một gian hàng trưng bày bàn và ghế rộng $60$ m$^2$. Diện tích để kê một chiếc ghế là $0{,}6$ m$^2$, một chiếc bàn là $1{,}3$ m$^2$. Gọi $x$ là số chiếc ghế, $y$ là số chiếc bàn được kê. Viết bất phương trình bậc nhất hai ẩn $x$, $y$ cho phần mặt sàn để kê bàn và ghế, biết diện tích mặt sàn dành cho lưu thông tối thiểu là $10$ m$^2$.
	\choice
	{$0{,}6x+1{,}3y\ge 50$}
	{\True $0{,}6x+1{,}3y\le 50$}
	{$1{,}3x+0{,}6y\le 50$}
	{$1{,}3x+0{,}6y\ge 50$}
	\loigiai{
		Diện tích để kê $x$ chiếc ghế và $y$ chiếc bàn là $0{,}6x+1{,}3y$.\\
		Vì diện tích mặt sàn dành cho lưu thông tối thiểu là $10$ m$^2$ nên diện tích để kê $x$ chiếc ghế và $y$ chiếc bàn tối đa là $50$ m$^2$.\\
		Do đó $0{,}6x+1{,}3y\le 50$.
	}
\end{ex}
\begin{ex}%[BG Toán 10-2022]%[Trần Nhân Kiệt]%[0D4T4-3]
	Bạn Nam đang sưu tầm các đồng tiền vàng và bạc để vào một các túi, trọng lượng tối đa mà túi chứa được là $20$ gam. Mỗi đồng xu vàng nặng khoảng $14$ gam, mỗi đồng xu bạc nặng khoảng $7$ gam. Bất phương trình nào sau đây mô tả số đồng tiền vàng ($x$) và số đồng tiền bạc ($y$) có thể được chứa trong túi?
	\choice
	{$7x+14y\le 20$}
	{$7x+14y>20$}
	{\True $14x+7y\le 20$}
	{$14x+7y>20$}
	\loigiai{
		Khối lượng của $x$ đồng tiền vàng là $14x$ gam.\\
		Khối lượng của $y$ đồng tiền bạc là $7y$ gam.\\	
		Số đồng tiền vàng và bạc có thể chứa trong túi khi $14x+7y\le 20$.
	}
\end{ex}
\begin{ex}%[BG Toán 10-2022]%[Trần Nhân Kiệt]%[0D4T4-3]
	Trong $1$ lạng ($100$ g) thịt bò chứa khoảng $26$ g protein và $1$ lạng cá rô phi chứa khoảng $20$ g protein. Trung bình trong một ngày, một người đàn ông cần tối thiểu $52$ g protein. Gọi $x$, $y$ lần lượt là số lạng thịt bò và số lạng cá rô phi mà một người đàn ông nên ăn trong một ngày. Viết bất phương trình bậc nhất hai ẩn $x$, $y$ để biểu diễn lượng protein cần thiết cho một người đàn ông trong một ngày.
	\choice
	{$26x+20y\le 52$}
	{$26x+20y< 52$}
	{\True $13x+10y\ge 26$}
	{$13x+10y> 26$}
	\loigiai{
		Trong $x$ lạng thịt bò chứa $26x$ g protein.\\
		Trong $y$ lạng cá rô phi chứa $20y$ g protein.\\
		Do đó lượng protein cần thiết trong một ngày của một người đàn ông là 
		$$26x+20y\ge 52\Leftrightarrow 13x+10y\ge 26.$$
	}
\end{ex}
\begin{ex}%[BG Toán 10-2022]%[Trần Nhân Kiệt]%[0D4T4-3]
	Công ty viễn thông Viettel có gói cước Hi School tính phí là $1190$ đồng mỗi phút gọi nội mạng và $1390$ đồng mỗi phút gọi ngoại mạng. Một bạn học sinh đăng kí gói cước trên và sử dụng $x$ phút gọi nội mạng, $y$ phút gọi ngoại mạng trong một tháng. Viết bất phương trình bậc nhất hai ẩn $x$, $y$ để mô tả số tiền bạn đó phải trả trong một tháng ít hơn $100$ nghìn đồng.
	\choice
	{$119x+139y\ge 10000$}
	{$139x+119y< 10000$}
	{$119x+139y\le 10000$}
	{\True $119x+139y< 10000$}
	\loigiai{
		Trong một tháng, số tiền gọi nội mạng là $1190 x$ đồng và số tiền gọi ngoại mạng là $1390y$ đồng.\\
		Tổng số tiền trong một tháng bạn học sinh phải trả là $1190x+1390y$.\\
		Để số tiền trong một tháng phải trả ít hơn $100$ nghìn đồng thì 
		$$1190x+1390y< 100000\Leftrightarrow 119y+139y< 10000.$$
		
	}
\end{ex}
\begin{ex}%[BG Toán 10-2022]%[Trần Nhân Kiệt]%[0D4T4-3]
	Nhân ngày Quốc tế Thiếu nhi $1-6$, một rạp chiếu phim phục vụ các khán giả một bộ phim hoạt hình. Vé được bán ra có hai loại: loại $1$ dành cho trẻ từ $6-13$ tuổi, giá vé là $50000$ đồng/vé và loại $2$ dành cho người trên $13$ tuổi, giá vé là $80000$ đồng/vé. Gọi $x$ là số vé loại $1$ và $y$ là số vé loại $2$ bán được. Viết bất phương trình bậc nhất hai ẩn $x$, $y$ để biểu diễn điều kiện sao cho số tiền bán vé thu được tối thiểu $10$ triệu đồng.
	\choice
	{$5x+8y\ge 100$}
	{$5x+8y> 1000$}
	{$8x+5y\ge 1000$}
	{\True $5x+8y\ge 1000$}
	\loigiai{
		Số tiền thu được từ $x$ vé loại $1$ là $50000x$ và số tiền thu được từ $y$ vé loại $2$ là $80000y$.\\
		Tổng số tiền bán vé thu được là $50000x+80000y$.\\
		Để số tiền bán vé thu được tối thiểu $10$ triệu đồng thì 
		$$50000x+80000y\ge 10000000\Leftrightarrow 5x+8y\ge 1000.$$
	}
\end{ex}

\begin{ex}%[BG Toán 10-2022]%[Trần Nhân Kiệt]%[0D4T4-3]
	Ngoài giờ học, bạn Nam làm thêm việc phụ bán cơm được $15$ nghìn đồng/một giờ và phụ bán tạp hóa được $10$ nghìn đồng/một giờ. Gọi $x$, $y$ lần lượt là số giờ phụ bán cơm và phụ bán tạp hóa trong mỗi tuần. Viết bất phương trình bậc nhất hai ẩn $x$ và $y$ sao cho Nam kiếm thêm tiền mỗi tuần được ít nhất là $900$ nghìn đồng.
	\choice
	{$3x+2y\le 180$}
	{$3x+2y> 180$}
	{\True $3x+2y\ge 180$}
	{$3x+2y< 180$}
	\loigiai{
		Số tiền từ việc phụ bán cơm là $15x$ nghìn đồng và số tiền từ việc phụ bán tạp hóa là $10y$ nghìn đồng.\\
		Số tiền Nam kiếm được mỗi tuần là $15x+10y$.\\
		Để số tiền Nam kiếm được mỗi tuần ít nhất là $900$ nghìn đồng thì 
		$$15x+10y\ge 900\Leftrightarrow 3x+2y\ge 180.$$
	}
\end{ex}


\begin{ex}%[BG Toán 10-2022]%[Trần Nhân Kiệt]%[0D4T4-3]
	Anh A muốn thuê một chiếc ô tô (có người lái) trong một tuần. Giá thuê xe như sau: từ thứ hai đến thứ sáu phí cố định là $900$ nghìn đồng/ngày và phí tính theo quãng đường di chuyển là $10$ nghìn đồng/km còn thứ bảy và chủ nhật thì phí cố định là $1200$ nghìn đồng/ngày và phí tính theo quãng đường di chuyển là $15$ nghìn đồng/km. Gọi $x$, $y$ lần lượt là số km mà anh A đi trong các ngày từ thứ hai đến thứ sáu và trong hai ngày cuối tuần. Viết bất phương trình biểu thị mối liên hệ giữa $x$ và $y$ sao cho tổng số tiền anh A phải trả không quá $20$ triệu đồng.
	\choice
	{$10x+15y\le 20000$}
	{$2x+3y\ge 2720$}
	{$10x+15y\ge 20000$}
	{\True $2x+3y\le 2720$}
	\loigiai{
		Số tiền thuê xe của anh A từ thứ hai đến thứ sáu là $900\cdot 5+10x$ nghìn đồng và hai ngày thứ bảy, chủ nhật là $1200\cdot 2+15y$ nghìn đồng.\\
		Để số tiền anh A phải trả không quá $20$ triệu đồng thì 
		$$(900\cdot 5+10x)+(1200\cdot 2+15y)\le 20000\Leftrightarrow 2x+3y\le 2720.$$
	}
\end{ex}

\begin{ex}%[BG Toán 10-2022]%[Trần Nhân Kiệt]%[0D4T4-3]
	Một cửa hàng làm kệ sách và bàn làm việc. Mỗi kệ sách cần $4$ giờ hoàn thiện. Mỗi bàn làm việc cần $3$ giờ hoàn thiện. Mỗi tháng cửa hàng có tối đa $240$ giờ làm việc. Hãy biểu diễn đồ thị mô tả số giờ làm việc trong mỗi tháng của cửa hàng theo số kệ sách hoàn thiện ($x$) và số bàn hoàn thiện ($y$).
	\choice
	{\centering \begin{tikzpicture}[scale=.7,font=\footnotesize, line join=round, line cap=round, >=stealth]
			\draw[->] (0,0) -- (5.3,0)node[below]{$x$};
			\draw[->,color=black] (0,0) -- (0,5.3)node[left]{$y$};
			\node[above left] at (0,0){$O$};
			\node[below] at (4,0){$80$};
			\node[left] at (0,3){$60$};
			\clip(0,0) rectangle (5.3,5.3);
			\fill[pattern=north east lines](4,0)-- (5.3,0) -- (5.3,5.3) -- (0,5.3)--(0,3)-- cycle;
			\draw[line width=1.2pt,smooth,samples=100,domain=0:4] plot(\x,{-0.75 *(\x) +3});
	\end{tikzpicture}}
	{\centering \begin{tikzpicture}[scale=.7,font=\footnotesize, line join=round, line cap=round, >=stealth]
			\draw[->] (0,0) -- (5.3,0)node[below]{$x$};
			\draw[->,color=black] (0,0) -- (0,5.3)node[left]{$y$};
			\node[above left] at (0,0){$O$};
			\node[below] at (4,0){$80$};
			\node[left] at (0,3){$60$};
			\clip(0,0) rectangle (5.3,5.3);
			\fill[pattern=north east lines](0,0)-- (4,0) -- (0,3) -- cycle;
			\draw[line width=1.2pt,smooth,samples=100,domain=0:4] plot(\x,{-0.75 *(\x) +3});
	\end{tikzpicture}}
	{\True \centering \begin{tikzpicture}[scale=.7,font=\footnotesize, line join=round, line cap=round, >=stealth]
			\draw[->] (0,0) -- (5.3,0)node[below]{$x$};
			\draw[->,color=black] (0,0) -- (0,5.3)node[left]{$y$};
			\node[above left] at (0,0){$O$};
			\node[below] at (3,0){$60$};
			\node[left] at (0,4){$80$};
			\clip(0,0) rectangle (5.3,5.3);
			\fill[pattern=north east lines](3,0)-- (5.3,0) -- (5.3,5.3) -- (0,5.3)--(0,4)-- cycle;
			\draw[line width=1.2pt,smooth,samples=100,domain=0:3] plot(\x,{-1.3333 *(\x) +4});
	\end{tikzpicture}}
	{\centering \begin{tikzpicture}[scale=.7,font=\footnotesize, line join=round, line cap=round, >=stealth]
			\draw[->] (0,0) -- (5.3,0)node[below]{$x$};
			\draw[->,color=black] (0,0) -- (0,5.3)node[left]{$y$};
			\node[above left] at (0,0){$O$};
			\node[below] at (3,0){$60$};
			\node[left] at (0,4){$80$};
			\clip(0,0) rectangle (5.3,5.3);
			\fill[pattern=north east lines](0,0)-- (4,0) -- (0,3)-- cycle;
			\draw[line width=1.2pt,smooth,samples=100,domain=0:3] plot(\x,{-1.3333 *(\x) +4});
	\end{tikzpicture}}
	\loigiai{
		\immini{Ta có bất phương trình $4x+3y \le 240$ mô tả số giờ làm việc trong mỗi tháng của cửa hàng. Biểu diễn nghiệm của bất phương trình như sau
		}
		{
			\centering \begin{tikzpicture}[scale=.7,font=\footnotesize, line join=round, line cap=round, >=stealth]
				\draw[->] (0,0) -- (5.3,0)node[below]{$x$};
				\draw[->,color=black] (0,0) -- (0,5.3)node[left]{$y$};
				\node[above left] at (0,0){$O$};
				\node[below] at (3,0){$60$};
				\node[left] at (0,4){$80$};
				\clip(0,0) rectangle (5.3,5.3);
				\fill[pattern=north east lines](3,0)-- (5.3,0) -- (5.3,5.3) -- (0,5.3)--(0,4)-- cycle;
				\draw[line width=1.2pt,smooth,samples=100,domain=0:3] plot(\x,{-1.3333 *(\x) +4});
			\end{tikzpicture}
		}
	}
\end{ex}


\begin{ex}%[BG Toán 10-2022]%[Trần Nhân Kiệt]%[0D4T4-3]
	Một gia đình cần $x$ kg thịt bò và $y$ kg thịt lợn trong một ngày, giá tiền $1$ kg thịt bò là $200$ nghìn đồng, $1$ kg thịt lợn là $60$ nghìn đồng. Biểu diễn đồ thị mô tả chi phí gia đình đó mua thịt bò và thịt lợn mỗi ngày để số tiền bỏ ra trong một ngày không quá $300$ nghìn đồng.
	\choice
	{\True \begin{tikzpicture}[scale=.7,font=\footnotesize, line join=round, line cap=round, >=stealth]
			\tikzset{label style/.style={font=\footnotesize}}
			\begin{scope}
				\clip (0,0) rectangle (4,6);
				\fill[pattern=north east lines] (-1,8.33)--(5,8.33)--(5,-11.67)--cycle;
				\draw (-0.3,6)--(1.5,0) node [pos=0.45, above, sloped] {};
			\end{scope}
			\draw[->] (0,0)--(4,0) node[below]{$x$};
			\draw[->] (0,0)--(0,6) node[left]{$y$};
			\draw (0,0) node[below left]{$O$};
			\draw (1.5,0) node[below]{$1{,}5$};
			\foreach \y in {5}
			\draw[thin] (1pt,\y)--(-1pt,\y) node [left] {$\y$};
	\end{tikzpicture}}
	{\begin{tikzpicture}[scale=.7,font=\footnotesize, line join=round, line cap=round, >=stealth]
			\tikzset{label style/.style={font=\footnotesize}}
			\begin{scope}
				\clip (0,0) rectangle (4,6);
				\fill[pattern=north east lines] (-1,8.33)--(-1,-11.67)--(5,-11.67)--cycle;
				\draw (-0.3,6)--(1.5,0) node [pos=0.45, above, sloped] {};
			\end{scope}
			\draw[->] (0,0)--(4,0) node[below]{$x$};
			\draw[->] (0,0)--(0,6) node[left]{$y$};
			\draw (0,0) node[below left]{$O$};
			\draw (1.5,0) node[below]{$1{,}5$};
			\foreach \y in {5}
			\draw[thin] (1pt,\y)--(-1pt,\y) node [left] {$\y$};
	\end{tikzpicture}}
	{\begin{tikzpicture}[scale=.7,font=\footnotesize, line join=round, line cap=round, >=stealth]
			\tikzset{label style/.style={font=\footnotesize}}
			\begin{scope}
				\clip (0,0) rectangle (4,6);
				\fill[pattern=north east lines] (-1,8.33)--(5,8.33)--(5,-11.67)--cycle;
				\draw (-0.3,6)--(1.5,0) node [pos=0.45, above, sloped] {};
			\end{scope}
			\draw[->] (0,0)--(4,0) node[below]{$x$};
			\draw[->] (0,0)--(0,6) node[left]{$y$};
			\draw (0,0) node[below left]{$O$};
			\draw (1.5,0) node[below]{$1$};
			\foreach \y in {5}
			\draw[thin] (1pt,\y)--(-1pt,\y) node [left] {$\y$};
	\end{tikzpicture}}
	{\begin{tikzpicture}[scale=.7,font=\footnotesize, line join=round, line cap=round, >=stealth]
			\tikzset{label style/.style={font=\footnotesize}}
			\begin{scope}
				\clip (0,0) rectangle (4,6);
				\fill[pattern=north east lines] (-1,8.33)--(-1,-11.67)--(5,-11.67)--cycle;
				\draw (-0.3,6)--(1.5,0) node [pos=0.45, above, sloped] {};
			\end{scope}
			\draw[->] (0,0)--(4,0) node[below]{$x$};
			\draw[->] (0,0)--(0,6) node[left]{$y$};
			\draw (0,0) node[below left]{$O$};
			\draw (1.5,0) node[below]{$1$};
			\foreach \y in {5}
			\draw[thin] (1pt,\y)--(-1pt,\y) node [left] {$\y$};
	\end{tikzpicture}}
	\loigiai{
		Số tiền mua thịt bò là $200x$ và số tiền mua thịt lợn là $60y$.\\
		Tổng số tiền trong một ngày mua thịt lợn và thịt bò là $200x+60y$.\\
		Để chi phí mua thịt bò và thịt lợn mỗi ngày không quá $300$ nghìn đồng thì 
		$$200x+60y\le 300\Leftrightarrow 10x+3y\le 15.$$
		Khi đó biểu diễn đồ thị mô tả chi phí là
		\begin{center}
			\begin{tikzpicture}[scale=.7,font=\footnotesize, line join=round, line cap=round, >=stealth]
				\tikzset{label style/.style={font=\footnotesize}}
				\begin{scope}
					\clip (0,0) rectangle (4,6);
					\fill[pattern=north east lines] (-1,8.33)--(5,8.33)--(5,-11.67)--cycle;
					\draw (-0.3,6)--(1.5,0) node [pos=0.45, above, sloped] {};
				\end{scope}
				\draw[->] (0,0)--(4,0) node[below]{$x$};
				\draw[->] (0,0)--(0,6) node[left]{$y$};
				\draw (0,0) node[below left]{$O$};
				\draw (1.5,0) node[below]{$1{,}5$};
				\foreach \y in {5}
				\draw[thin] (1pt,\y)--(-1pt,\y) node [left] {$\y$};
			\end{tikzpicture}
		\end{center}
	}
\end{ex}
\Closesolutionfile{ans}
% \Closesolutionfile{ansbook}
% \indapan{10}{ans/ans-BPTbacnhathaian}
% \subsection{Lời giải câu hỏi trắc nghiệm}
% \input{ans/ansbook-BPTbacnhathaian}
% \section{Hệ bất phương trình bậc nhất hai ẩn}
\setcounter{dang}{0}
\subsection{Tóm tắt lý thuyết}
\subsubsection{Khái niệm hệ bất phương trình bậc nhất hai ẩn}
\begin{boxdn}
	\textbf{\textit{Hệ bất phương trình bậc nhất hai ẩn}} là hệ gồm hai hay nhiều bất phương trình bậc nhất hai ẩn $x$, $y$. Mỗi nghiệm chung của tất cả các bất phương trình đó được gọi là một nghiệm của hệ bất phương trình đã cho.\\
	Trên mặt phẳng toạ độ $Oxy$, tập hợp các điểm $(x_0;y_0)$ có tọa độ là nghiệm của hệ bất phương trình bậc nhất hai ẩn được gọi là \textbf{\textit{miền nghiệm}} của hệ bất phương trình đó.
\end{boxdn}
\subsubsection{Biểu diễn miền nghiệm của hệ bất phương trình bậc nhất hai ẩn}
\begin{boxdn}
	Để \textbf{\textit{biểu diễn miền nghiệm}} của hệ bất phương trình bậc nhất hai ẩn trên mặt phẳng toạ độ $Oxy$, ta thực hiện như sau:
	\begin{itemize}
	\item Trên cùng mặt phẳng tọa độ, biểu diễn miền nghiệm của mỗi bất phương trình của hệ.
	\item Phần giao của các miền nghiệm là miền nghiệm của hệ bất phương trình.
	\end{itemize}
\end{boxdn}
\begin{note}
	Miền mặt phẳng tọa độ bao gồm một đa giác lồi và phần nằm bên trong đa giác đó được gọi là một miền đa giác.
\end{note}
\subsubsection{Tìm giá trị lớn nhất và giá trị nhỏ nhất của biểu thức $\mathbf{F=ax+by}$ trên một miền đa giác}
Hệ bất phương trình giúp ta mô tả được nhiều bài toán thực tế để tìm ra cách giải quyết tối ưu. Chúng thường được đưa về bài toán tìm giá trị lớn nhất (GTLN) hoặc giá trị nhỏ nhất (GTNN) của biểu thức $F=ax+by$ trên một miền đa giác.\\
Người ta chứng minh được $F$ đạt giá trị lớn nhất hoặc nhỏ nhất tại một trong các đỉnh của đa giác.
\subsection{Các dạng toán}
\begin{dang}{Biểu diễn hình học của tập nghiệm}
	
\end{dang}

\viduminhhoa
\begin{vd}%[0D4B4]
	Biểu diễn hình học tập nghiệm của hệ bất phương trình bậc nhất hai ẩn sau
	$$\left\{\begin{aligned}
		x+y &> 1\\
		x-y &<2 \\
	\end{aligned}\right.$$
	\loigiai
	{\immini{
			Vẽ các đường thẳng
			\begin{gather*}
				d_1: x+y=1,
				d_2: x-y=2.
			\end{gather*}
			Vì điểm $M(0,2)$ có tọa độ thỏa mãn các bất phương trình trong hệ nên ta tô đậm các nửa mặt phẳng bờ $d_1,d_2$ không chứa $M$. 
			
			Miền không bị tô đậm trong hình vẽ và không chứa các tia giới hạn miền là miền nghiệm của hệ đã cho.
		}{
			\begin{tikzpicture}[line cap=round, line join=round, font=\footnotesize, >=stealth, scale=1]
				\tikzset{label style/.style={font=\footnotesize}}
				\draw[color=white,fill=black!10] (-2,-2) rectangle (4,3);
				\draw[fill=white,color=white] (-2,3) -- (1.5,-.5) -- (4,2) -- (4,3) -- cycle;
				\draw[->] (0,-2) -- (0,3) node[left]{\scriptsize $y$};
				\draw[->] (-2,0) -- (4,0) node[above]{\scriptsize $x$};
				\draw (0.15,0.1) node[below left]{\scriptsize $O$};
				\draw (-2,3) -- (3,-2) node[right]{\scriptsize $d_1$};
				\draw (0,-2) -- (4,2) node[right]{\scriptsize $d_2$};
				\draw[dashed] (0,-0.5)node[left]{\scriptsize $-\dfrac{1}{2}$} -- (1.5,-.5) node[right]{\scriptsize $I$} -- (1.5,0) node[above]{\scriptsize $\dfrac{3}{2}$};
				\draw (0,2) node[right]{\scriptsize $M$} (0,2) node[left]{\scriptsize $2$} (-0.05,2) -- (0.05,2);
			\end{tikzpicture}
		}
	}
\end{vd}
\begin{vd}%[0D4B4]
	Biểu diễn hình học tập nghiệm của hệ bất phương trình bậc nhất hai ẩn sau
	$$\left\{\begin{aligned}
		x+y &< 2\\
		x-y &>1 \\
		y &>-1
	\end{aligned}\right.$$
	\loigiai
	{\immini{
			Vẽ các đường thẳng
			\begin{gather*}
				d_1: x+y=2,\\
				d_2: x-y=1,\\
				d_3: y=-1.
			\end{gather*}
			Vì điểm $M\biggl(\dfrac{3}{2},0\biggr)$ có tọa độ thỏa mãn các bất phương trình trong hệ nên ta tô đậm các nửa mặt phẳng bờ $d_1,d_2,d_3$ không chứa $M$. Miền không bị tô đậm trong hình vẽ, không bao gồm các đoạn giới hạn miền là miền nghiệm của hệ đã cho.
		}{
			\begin{tikzpicture}[line cap=round, line join=round, font=\footnotesize, >=stealth, scale=1]
				\tikzset{label style/.style={font=\footnotesize}}
				\draw[color=white,fill=black!10] (-2,-2) rectangle (5,3);
				\path (-1,3) coordinate (A1);
				\path (4,-2) coordinate (B1);
				\path (-1,-2) coordinate (A2);
				\path (4,3) coordinate (B2);
				\path (-2,-1) coordinate (A3);
				\path (5,-1) coordinate (B3);
				\path (intersection of A1--B1 and A2--B2) coordinate (A);
				\path (intersection of A1--B1 and A3--B3) coordinate (B);
				\path (intersection of A2--B2 and A3--B3) coordinate (C);
				\draw[fill=white] (A) -- (B) -- (C) -- cycle;
				\draw[->] (0,-2) -- (0,3) node[right]{\scriptsize $y$};
				\draw[->] (-2,0) -- (5,0) node[above]{\scriptsize $x$};
				\draw (0,0) node[above left]{\scriptsize $O$};
				\draw (A1) -- (B1) node[right]{\scriptsize $d_1$} (A2) -- (B2)node[right]{\scriptsize $d_2$} (A3) -- (B3)node[right]{\scriptsize $d_3$};
				\draw[dashed] (B)node[below]{\scriptsize $B$} -- (B |- 0,0) node[above]{\scriptsize $3$}
				(A-| 0,0)node[left]{\scriptsize $\dfrac{3}{2}$} -- (A)node[above]{\scriptsize $A$} -- (A|- 0,0)node[below]{\scriptsize $\dfrac{3}{2}$};
				\draw (1.5,0)node[above right]{\scriptsize $M$} (C) node[below right]{\scriptsize $C$} (C) node[above left]{\scriptsize $-1$};
			\end{tikzpicture}
		}
	}
\end{vd}
\begin{vd}%[0D4B4]
	Biểu diễn hình học tập nghiệm của hệ bất phương trình bậc nhất hai ẩn sau
	$$\left\{\begin{aligned}
		2x+5y &> 2\\
		x-3y &\geq 1 \\
		x+y &<3
	\end{aligned}\right.$$
	\loigiai
	{\immini{
			Vẽ các đường thẳng
			\begin{gather*}
				d_1: 2x+5y=2,\\
				d_2: x-3y=1,\\
				d_3: x+y=3.
			\end{gather*}
			Vì điểm $M(2,0)$ có tọa độ thỏa mãn các bất phương trình trong hệ nên ta tô đậm các nửa mặt phẳng bờ $d_1,d_2,d_3$ không chứa $M$. 
			
			Miền không bị tô đậm trong hình vẽ có chứa đoạn $AC$ và không chứa các điểm $A,C$, không chứa các đoạn $AB,BC$ là miền nghiệm của hệ đã cho.
		}{
			\begin{tikzpicture}[line cap=round, line join=round, font=\footnotesize, >=stealth, scale=1]
				\tikzset{label style/.style={font=\footnotesize}}
				\draw[color=white,fill=black!10] (-1,-2) rectangle (6,3.2);
				\path (-1,0.8) coordinate (A1);
				\path (6,-2) coordinate (B1);
				\path (-1,-2/3) coordinate (A2);
				\path (6,5/3) coordinate (B2);
				\path (-0.2,3.2) coordinate (A3);
				\path (5,-2) coordinate (B3);
				\path (intersection of A1--B1 and A2--B2) coordinate (A);
				\path (intersection of A1--B1 and A3--B3) coordinate (B);
				\path (intersection of A2--B2 and A3--B3) coordinate (C);
				\draw[fill=white] (A) -- (B) -- (C) -- cycle;
				\draw[->] (0,-2) -- (0,3.2) node[right]{\scriptsize $y$};
				\draw[->] (-1,0) -- (6,0) node[above]{\scriptsize $x$};
				\draw (0.15,0.1) node[below left]{\scriptsize $O$};
				\draw (A1) -- (B1) node[right]{\scriptsize $d_1$} (A2) -- (B2)node[right]{\scriptsize $d_2$} (A3) -- (B3)node[right]{\scriptsize $d_3$};
				\draw[dashed] (B -| 0,0) node[left]{\scriptsize $-\dfrac{4}{3}$}--(B)node[below]{\scriptsize $B$} -- (B |- 0,0) node[above]{\scriptsize $\dfrac{13}{3}$}
				(C-| 0,0) node[left]{\scriptsize $\dfrac{1}{2}$} -- (C)node[above]{\scriptsize $C$} -- (C|- 0,0)node[below]{\scriptsize $\dfrac{5}{2}$};
				\draw (2,0)node[above]{\scriptsize $2$} (2,0.05)  -- (2,-0.05) node[below]{\scriptsize $M$} (A) node[below]{\scriptsize $A$} (A) node[above]{\scriptsize $1$};	
			\end{tikzpicture}
		}
	}
\end{vd}
\begin{vd}%[0D4B4]
	Biểu diễn hình học tập nghiệm của hệ bất phương trình bậc nhất hai ẩn sau
	$$\left\{\begin{aligned}
		2x+y &\geq 2\\
		x-2y &\leq 1 \\
		y &\leq 2\\
		x &\leq 3
	\end{aligned}\right.$$
	\loigiai
	{\immini{
			Vẽ các đường thẳng
			\begin{gather*}
				d_1: 2x+y=2,\\
				d_2: x-2y=1,\\
				d_3: y=2,
				d_4: x=3.
			\end{gather*}
			Vì điểm $M(2,1)$ có tọa độ thỏa mãn các bất phương trình trong hệ nên ta tô đậm các nửa mặt phẳng bờ $d_1,d_2,d_3,d_4$ không chứa $M$. Miền không bị tô đậm trong hình vẽ là miền nghiệm của hệ đã cho bao gồm các đoạn thẳng xác định miền.
		}{\begin{tikzpicture}[>=stealth]
				\draw[color=white,fill=black!10] (-2,-2) rectangle (5,3);
				\path (-1/2,3) coordinate (A1);
				\path (2,-2) coordinate (B1);
				\path (-2,-3/2) coordinate (A2);
				\path (5,2) coordinate (B2);
				\path (-2,2) coordinate (A3);
				\path (5,2) coordinate (B3);
				\path (3,-2) coordinate (A4);
				\path (3,3) coordinate (B4);
				\path (intersection of A1--B1 and A2--B2) coordinate (A);
				\path (intersection of A1--B1 and A3--B3) coordinate (B);
				\path (intersection of A3--B3 and A4--B4) coordinate (C);
				\path (intersection of A2--B2 and A4--B4) coordinate (D);
				\draw[fill=white] (A) -- (B) -- (C) -- (D) -- cycle;
				\draw[->] (0,-2) -- (0,3) node[right]{\scriptsize $y$};
				\draw[->] (-2,0) -- (5,0) node[above]{\scriptsize $x$};
				\draw (0.15,0.1) node[below left]{\scriptsize $O$};
				\draw (A1) -- (B1) node[right]{\scriptsize $d_1$} (A2)node[left]{\scriptsize $d_2$} -- (B2) (A3)node[left]{\scriptsize $d_3$} -- (B3) (A4) -- (B4) node[right]{\scriptsize $d_4$};
				\draw[dashed] (A) node[above]{\scriptsize $1$} (A) node[below]{\scriptsize $A$} (B) node[below left]{\scriptsize $2$} (B) node[above right]{\scriptsize $B$} (C) node[above right]{\scriptsize $C$} (C|- 0,0) node[below right]{\scriptsize $3$} (D) node[below right]{\scriptsize $D$} -- (D-| 0,0) node[left]{\scriptsize $1$} (2,1) node[above]{\scriptsize $M$} -- (2,0)node[below]{\scriptsize $2$};
			\end{tikzpicture}
		}
	}
\end{vd}

\baitaptl
\begin{bt}%[0D4B4]
	Biểu diễn hình học tập nghiệm của hệ bất phương trình bậc nhất hai ẩn sau
	$$\left\{\begin{aligned}
		x+2y &\geq 1\\
		3x-y & \leq 2 \\
	\end{aligned}\right.$$
	\loigiai{
		\immini{
			Vẽ hai đường thẳng
			\begin{gather*}
				d_1:x+2y=1,\\
				d_2:3x-y=2
			\end{gather*}
			trên cùng một hệ trục tọa độ $Oxy$. Dễ dàng kiểm tra được điểm $O$ thuộc miền nghiệm của cả hai bất phương trình nên ta có miền nghiệm của hệ bất phương trình là miền không bị tô đậm bao gồm cả bờ.
		}{
			\begin{tikzpicture}[line cap=round, line join=round, font=\footnotesize, >=stealth, scale=1]
				\tikzset{label style/.style={font=\footnotesize}}
				\fill[black!10] (-1,-2) rectangle (3,1);
				\fill[white] (-1,1)--(5/7,1/7)--(0,-2)--(-1,-2)--cycle;
				\draw[->] (-1.2,0)--(3,0) node[below]{$x$};
				\draw[->] (0,-2)--(0,1) node[left]{$y$};
				\draw (-1,1)--(3,-1) node[below]{$d_1$}
				(0,-2)--(1,1) node[right]{$d_2$};
				\draw[fill=black](0,0) circle (1pt) node[below left]{$O$}
				(0,2/3) circle (1pt) node[right]{$\frac{2}{3}$};
				\foreach\x in {-1,1,2} \draw[fill=black] (\x,0) circle (1pt) node[below]{$\x$};
				\foreach\y in {-2,-1} \draw[fill=black] (0,\y) circle (1pt) node[left]{$\y$};
			\end{tikzpicture}	
			
		}
	}
\end{bt}

\begin{bt}%[0D4B4]
	Biểu diễn hình học tập nghiệm của hệ bất phương trình bậc nhất hai ẩn sau
	$$\left\{\begin{aligned}
		x-2y &< 1\\
		x+3y &<-2 \\
		-x+y &<2
	\end{aligned}\right.$$
	\loigiai{
		\immini{
			Vẽ các đường thẳng
			\begin{gather*}
				d_1:x-2y=1\\ d_2:x+2y=-2\\ d_3:-x+y=2
			\end{gather*}
			trên cùng mặt phẳng tọa độ $Oxy$. Ta kiểm tra được điểm $M(-2;-1)$ thuộc miền nghiệm của hệ bất phương trình nên ta có miền nghiệm của hệ bất phương trình là miền trong tam giác $ABC$ không kể các cạch.
		}{
			\begin{tikzpicture}[line cap=round, line join=round, font=\scriptsize, >=stealth, scale=1]
				\tikzset{label style/.style={font=\scriptsize}}
				\fill[black!10] (-6,-4) rectangle (2,1);
				\path (-2,0) coordinate (A)
				(-5,-3) coordinate (B)
				(-1/5,-3/5) coordinate (C)
				(-2,-1) coordinate (M);
				\fill[white] (A)--(B)--(C)--cycle;
				\draw[->] (-6,0)--(2,0) node[below]{$x$};
				\draw[->] (0,-4)--(0,1) node[left]{$y$};
				\draw[fill=black] (0,0) circle (1pt) node[above right]{$O$}
				(-5,0) circle (1pt) node[above]{$-5$}
				(-2,0) circle (1pt) node[above]{$-2$}
				(0,-3) circle (1pt) node[right]{$-3$}
				(0,-1) circle (1pt) node[right]{$-1$};
				\draw[smooth] plot[domain=-6:2] (\x,{ \x/2-1/2 }) node[above]{$d_1$}
				plot[domain=-5:2] (\x,{ -\x/3-2/3 }) node[below]{$d_2$}
				plot[domain=-6:-1] (\x,{ \x+2}) node[left]{$d_3$};
				\foreach\x/\y in {A/-45,B/-60,C/135,M/-145} \draw[fill=black] (\x) circle (1pt)+(\y:0.3) node{$\x$};
				\draw[dashed,fill=black] (-1/5,0) circle (1pt) node[above]{$\frac{-1}{5}$}-- (C) (C)--(0,-3/5) circle (1pt) node[right]{$\frac{-3}{5}$};
				\draw[dashed] (-5,0)--(B)--(0,-3)
				(A)--(M)--(0,-1);
			\end{tikzpicture}	
			
		} 
	}
\end{bt}

\begin{bt}%[0D4B4]
	Biểu diễn hình học tập nghiệm của hệ bất phương trình bậc nhất hai ẩn sau
	$$\left\{\begin{aligned}
		3x+y & \leq 5\\
		x+y & \leq 4 \\
		x &\geq 0\\
		y &\geq 0
	\end{aligned}\right.$$
	\loigiai{
		\immini{
			Vẽ các đường thẳng $d_1:3x+y=5$ và $d_2:x+y=4$ lên cùng hệ trục tọa độ. Ta thấy điểm $M(1;1)$ thỏa mãn tất cả các bất phương trình của hệ, do đó tập nghiệm của hệ bất phương trình đã cho là miền trong tứ giác $ABCO$ kể cả các cạnh.
		}{
			\begin{tikzpicture}[line cap=round, line join=round, font=\scriptsize, >=stealth, scale=1,y=0.7cm]
				\tikzset{label style/.style={font=\footnotesize}}
				\fill[black!10] (-1,-1) rectangle (5,6);
				\path (0,4) coordinate (A)
				(0.5,3.5) coordinate (B)
				(5/3,0) coordinate (C)
				(0,0) coordinate (O)
				(1,1) coordinate (M);
				\fill[white] (A)--(B)--(C)--(O)--cycle;
				\draw[->] (-1,0)--(5,0) node[below]{$x$};
				\draw[->] (0,-1)--(0,6) node[right]{$y$};
				\draw[fill=black]
				(0.5,0) circle (1pt) node[below]{$\frac{1}{2}$}
				(1,0) circle (1pt) node[below]{$1$}
				(5/3,0) circle (1pt) node[above right]{$\frac{5}{3}$}
				(0,3.5) circle (1pt) node[left]{$\frac{7}{2}$}
				(0,1) circle (1pt) node[left]{$1$}
				(0,4) circle (1pt) node[left]{$4$};
				\draw[smooth] plot[domain=-1:5] (\x,{ 4-\x }) node[above]{$d_2$}
				plot[domain=-0.33:2] (\x,{ 5-3*\x }) node[right]{$d_1$};
				\foreach\x/\y in {A/30,B/-135,C/-135,M/90,O/-135} \draw[fill=black] (\x) circle (1pt)+(\y:0.3) node{$\x$};
				\draw[dashed] (0,3.5)--(B)--(0.5,0)
				(1,0)--(M)--(0,1);
			\end{tikzpicture}	
			
		}
	}
\end{bt}

\begin{dang}{Tìm cực trị của biểu thức $F=ax+by$ trên một miền đa giác}
	\begin{enumerate}
		\item Bài toán: \\
		Tìm giá trị lớn nhất, giá trị nhỏ nhất của biểu thức $F=ax+by$ ($a$, $b$ là hai số đã cho không đồng thời bằng $0$) với $x$, $ y$ thỏa mãn hệ bất phương trình bậc nhất hai ẩn (có miền nghiệm là miền đa giác $A_1 A_2 \ldots A_i A_{i+1} \ldots A_n$).
		\item Người ta chứng minh được: Biểu thức $F=ax+by$  có giá trị nhỏ nhất, giá trị lớn nhất tại một trong các đỉnh của đa giác $A_1 A_2 \ldots A_i A_{i+1} \ldots A_n$.
		\item Phương pháp: 
		\begin{itemize}
			\item Bước 1. Tìm miền đa giác $A_1 A_2 \ldots A_i A_{i+1} \ldots A_n$ là miền nghiệm của hệ bất phương trình.
			\item Bước 2. Tìm tọa độ các đỉnh $A_1$, $A_2$, $\ldots$, $A_n$.
			\item Bước 3. Tính $F\left(x_i ; y_i\right)$ trong đó $A_i\left(x_i;y_i\right)$ với $i=1,2,\ldots,\ n$.
			\item Bước 4. Kết luận\\
			Giá trị lớn nhất $M=\max \limits_{i=1,2,\ldots n} F\left(x_i; y_i\right)$.\\
			Giá trị nhỏ nhất $m=\min\limits_{i=1,2,\ldots n} F\left(x_i; y_i\right)$.
		\end{itemize}
	\end{enumerate}
\end{dang}
\viduminhhoa
\begin{vd}%[BG10-2022]%[Toanvo]%[0D4G4-4]
	Cho cặp số $\left(x;y\right)$  là nghiệm của hệ  $\heva{&3x-y \geq -1\\&2x+y \leq 6\\&x+3y>3}$. Tìm giá trị lớn nhất và nhỏ nhất của biểu thức $f\left(x;y\right)=2x-3y+1$.
	\loigiai{
		\begin{itemize}
			\item Trước hết ta biểu diễn miền nghiệm của hệ (*):
			\begin{itemize}
				\item  Vẽ các đường thẳng $d_1 \colon 3x-y=-1$; $d_2 \colon 2x+y=6$; $d_3 \colon x+3y=3$.
				\item  Điểm $M(1;1)$ có tọa độ thỏa mãn tất cả các bất phương trình trong hệ nên ta tô đậm các nửa mặt phẳng bờ $d_1;d_2;d_3$ không chứa điểm $M$. Miền không bị tô đậm là hình tam giác $ABC$, tính cả ba cạnh $AB,BC,CA$ trong hình vẽ dưới là miền nghiệm của hệ bất phương trình đã cho.
			\end{itemize}
			\begin{center}
				\begin{tikzpicture}
					\draw[->,line width=0.8pt] (-1,0)--(6.5,0) node[right]{$x$};
					\draw[->,line width=0.8pt] (0,-1)--(0,5) node[above left]{$y$};
					
					\clip (-1,-1) rectangle (6.5,5);
					
					
					\begin{scope}
						\clip (-2,-5)--(-2,7)--(2,7)--(-2,-5);
						\foreach \x in {-1,-.85,...,20}{
							\pgfmathsetmacro{\y}{-2*\x+8}
							\draw[gray] (\x,\y)--++(150:10);
						}			
					\end{scope}
					
					\begin{scope}
						\clip (-2,10)--(6,10)--(6,-6)--(-2,10);
						\foreach \x in {-1,-.9,...,20}{
							\pgfmathsetmacro{\y}{-2*\x+1}
							\draw[gray] (\x,\y)--++(1500:10);
						}			
					\end{scope}
					
					\pgfmathsetmacro{\hsgg}{-1/3}
					\begin{scope}
						\clip (-3,2)--(-3,-3)--(12,-3)--(-3,2);
						\foreach \x in {-1,-.95,...,20}{
							\pgfmathsetmacro{\y}{-4*\x-3}
							\draw[gray] (\x,\y)--++(10:10);
						}			
					\end{scope}
					
					
					
					\draw[thick,smooth,domain=-1:6.5,blue] plot (\x,{3*\x+1});
					\draw[thick,smooth,domain=-1:6.5,red] plot (\x,{-2*\x+6});
					\draw[thick,smooth,domain=-1:6.5,green] plot (\x,{\hsgg*\x+1});
					
					\path (2.5,4.5)--(3,4.5) node[fill=white,pos=.5,sloped]{$d_{1} \colon 3x-y=-1$};
					\path (1,4)--(3,0) node[fill=white,pos=.5,sloped,above]{$d_{2} \colon 2x+y=6$};
					\path (1,-1)--(2,-1) node[fill=white,pos=.5,sloped,above]{$d_{3} \colon x+3y=3$};
					
					\foreach \i in {-1,1,2,3,4}{
						\fill (0,\i) circle (1.3pt);
					}
					\foreach \i in {-1,0,1,2,3,4,5,6}{
						\fill (\i,0) circle (1.3pt);
					}
					\fill (1,1) circle (1.3pt) (1,4) circle (1.3pt);
					
					\path 
					(0,1) node[below left,fill=white]{$A$}
					(1,1) node[above]{$M$}
					(3,0) node[above right,fill=white]{$C$}
					(1.1,4) node[right,fill=white]{$B$}
					(0,0) node[below left,fill=white]{$O$};
				\end{tikzpicture}
			\end{center}
			\item Tìm tọa độ các điểm $A,B,C$:
			\begin{itemize}
				\item $A=d_1 \cap d_3$ nên tọa độ thỏa mãn $\heva{	&3x-y=-1\\
					&x+3y=3} \Leftrightarrow \heva{&x=0\\
					&y=1}$. Vậy $A(0;1)$.
				\item  $B=d_1 \cap d_2$ nên tọa độ thỏa mãn $\heva{&3x-y=-1\\
					&2x+y=6} \Leftrightarrow \heva{&x=1\\
					&y=4} $. Vậy $B(1;4)$.
				\item $C=d_2 \cap d_3$ nên tọa độ thỏa mãn $\heva{&2x+y=6\\
					&x+3y=3}\Leftrightarrow \heva{&x=3\\
					&y=0} $. Vậy $C(3;0)$.
			\end{itemize}
		\end{itemize}
		Tính giá trị của $f(x;y)=2x-3y+1$ tại tất cả các đỉnh của tam giác $ABC$: 
		\begin{center}
			\begin{tabular}{|p{5cm}||p{2cm}|p{2cm}|p{2cm}|}
				\hline
				$(x;y)$ & $A(0;1)$ & $B(1;4)$ & $C(3;0)$ \\
				\hline
				$f(x;y)=2x-3y+1$  & $-2$  & $-9$ & $7$ \\
				\hline
			\end{tabular}
		\end{center}
		Suy ra $\min f(x;y)=f(1;4)=-9$ và $\max f(x;y)=f(3;0)=7$.\\
		
	}
\end{vd}
\begin{vd}%[BG10-2022]%[Toanvo]%[0D4G4-4]
	Quảng cáo sản phẩm trên truyền hình là một hoạt động quan trọng trong kinh doanh của các doanh nghiệp.\\
	Theo Thông báo số $10/2019$, giá quảng cáo trên VTV1 là $30$ triệu đồng cho $15$ giây/$1$ lần quảng cáo vào khoảng $20$h$30$; là $6$ triệu đồng cho $15$ giây/$1$ lần quảng cáo vào khung giờ $16$h$00-17$h$00$.\\
	Một công ty dự định chi không quá $900$ triệu đồng để quảng cáo trên VTV1 với yêu cầu quảng cáo về số lần phát như sau: ít nhất $10$ lần quảng cáo vào khoảng
	$20$h$30$ và không quá $50$ lần quảng cáo vào khung giờ $16$h$00-17$h$00$. 
	\loigiai{
		Gọi $x, \,y$ lần lượt là số lần phát quảng cáo vào khoảng $20$h$30$  và vào khung giờ $16$h$00-17$h$00$. Theo giả thiết, ta có: $x \in \mathbb{N},\, y \in \mathbb{N},\, x \geq 10,0 \leq y \leq 50$.\\
		Tổng số lần phát quảng cáo là $T=x+y$.\\
		Số tiền công ty cần chi là $30 x+6 y$ (triệu đồng).\\
		Do công ty dự định chi không quá $900$ triệu đồng nên $30 x+6 y \leq 900$ hay $5 x+y \leq 150$.\\
		Ta có hệ bất phương trình: $\heva{5 x+y \leq 150 \\ x \geq 10 \\ 0 \leq y \leq 50.} $ \hfill(I)	
		Bài toán đưa về tìm $x, y$ là nghiệm của hệ bất phương trình (I) sao cho $T=x+y$ có giá trị lớn nhất.\\
		Trước hết, ta xác định miền nghiệm của hệ bất phương trình (I).\\
		Miền nghiệm của hệ bất phương trình (I) là miền tứ giác $A B C D$ với $A(30 ; 0), B(20 ; 50)$, $C(10 ; 50), D(10 ; 0)$ (Hình vẽ).
		\immini{
			Người ta chứng minh được: Biểu thức $T=x+y$ đạt được giá trị lớn nhất tại một trong các đỉnh của tứ giác $A B C D$.
			Tính giá trị của biểu thức $T=x+y$ tại cặp số $(x ; y)$ là toạ độ các đỉnh của tứ giác $A B C D$ rồi so sánh các giá trị đó. Ta được $T$ đạt giá trị lớn nhất khi $x=20, y=50$ ứng với tọa độ đỉnh $B$.\\
			Vậy để phát được số lần quảng cáo nhiều nhất thì số lần phát quảng cáo vào khoảng $20$h$30$  và vào khung giờ $16$h$00-17$h$00$  lần lượt là $20$ và $50$ lần.
		}{\begin{tikzpicture}[scale=0.07,line join=round, line cap=round,>=stealth,thick]
				\tikzset{label style/.style={font=\footnotesize}}
				\begin{scope}
					
					\clip (-20,-20) rectangle (60,60);
					\fill[pattern=crosshatch dots] (-21,255)--(61,255)--(61,-155)--cycle;
					\fill[pattern=crosshatch dots] (10,-20)--(-20,-20)--(-20,60)--(10,60)--cycle;
					\fill[pattern=crosshatch dots] (-20,0)--(-20,-20)--(60,-20)--(60,0)--cycle;
					\fill[pattern=crosshatch dots] (-20,50)--(-20,60)--(60,60)--(60,50)--cycle;
					\draw (18,60)--(34,-20) ;
					\draw (10,-20)--(10,60) ;
					\draw (-20,50)--(60,50) ;
				\end{scope}
				\draw[->] (-20,0)--(60,0) node[below]{$x$};
				\draw[->] (0,-20)--(0,60) node[left]{$y$};
				\draw (0,0) node[below left]{$O$};
				\path (10,50) coordinate(C) (10,0) coordinate(D) (30,0) coordinate(A) (20,50) coordinate(B);
				\foreach \i in {A,B,C,D} \fill (\i) circle(20pt) ($(\i)+(55:5)$)node{$\i$};
				\foreach \x in {10,20,30}
				\draw[thin] (\x,1pt)--(\x,-1pt) node [below right] {$\x$};
				\foreach \y in {50}
				\draw[thin] (1pt,\y)--(-1pt,\y) node [below left] {$\y$};
		\end{tikzpicture}}
	}
\end{vd}
\begin{vd}%[BG10-2022]%[Toanvo]%[0D4G4-4]
	Một hộ nông dân dự định trồng đậu và cà trên diện tích $8$ ha. Nếu trồng đậu thì cần $20$ công và thu $3$ triệu đồng trên diện tích mỗi ha, nếu trồng cà thì cần $30$ công và thu $4$ triệu đồng trên diện tích mỗi ha. Hỏi cần trồng mỗi loại cây trên với diện tích là bao nhiêu để thu về được nhiều tiền nhất, biết rằng tổng số công không quá $180$.
	\loigiai{
		Gọi diện tích để trồng đậu là $x$ (ha); diện tích để trồng cà là  $y$ (ha). ( điều kiện: $0 \leq x,y \leq 8$ ).\\
		Tổng số diện tích sử dụng là $x+y$.\\
		Tổng số công cần sử dụng là $20x+30y$. \\
		Ta có hệ bất phương trình
		$$\heva{&0\leq x\leq8\\&0\leq y\leq8\\&x+y\leq8\\&20x+30y\leq180}\Leftrightarrow\heva{&0\leq x\leq8\\&0\leq y\leq8\\&x+y\leq8\\&2x+3y\leq18.}$$
		Vẽ các đường thẳng thẳng $(d_1) \colon x+y=8,\ (d_2) \colon 2x+3y=18,\ (d_3) \colon x=8,\ (d_4) \colon y=8$ ta được miền
		nghiệm của hệ bất phương trình là phần tô đậm như hình vẽ
		\begin{center}
			\begin{tikzpicture}[line join = round, line cap = round, >=stealth, font=\footnotesize, scale=.5]
				\tikzset{label style/.style={font=\footnotesize}}
				\def \xmin{-6}
				\def \xmax{15}
				\def \ymin{-5}
				\def \ymax{14}
				\tkzDefPoints{0/6/A, 6/2/B, 8/0/C, 0/0/D, 0/8/E, 9/0/F, 8/8/G}
				\draw[gray!20](\xmin,\ymin) grid (\xmax,\ymax);
				\draw [->](\xmin,0)--(0,0)node[below left]{$O$}--(\xmax,0)node[above]{$x$};
				\draw [->](0,\ymin)--(0,\ymax)node[right]{$y$};
				\foreach \x in {-5,5,10} \draw[shift={(\x,0)}] (0pt,2pt)--(0pt,-2pt) node[below]{$\x$};
				\foreach \y in {5,10} \draw [shift={(0,\y)}] (2pt,0pt)--(-2pt,0pt) node[left]{$\y$};
				\tkzDrawLines[add=0.5 and 0.5](E,C A,F C,G E,G)
				\tkzDrawPoints[fill=black](A,B,C,D)
				\tkzLabelPoints[right](A)
				\tkzLabelPoints[above](B)
				\tkzLabelPoints[below left](C)
				\tkzLabelPoints[below right](D)
				\tkzFillPolygon[color=cyan,fill opacity=.5](A,B,C,D)
				\tkzLabelLine[pos=1.7,above right](B,E){$(d_1)$}
				\tkzLabelLine[pos=1.8,above right](B,A){$(d_2)$}
				\tkzLabelLine[pos=1.5,right](C,G){$(d_3)$}
				\tkzLabelLine[pos=1.5,above](E,G){$(d_4)$}
			\end{tikzpicture}
		\end{center}
		Ta có 
		{\allowdisplaybreaks
			\begin{eqnarray*}
				&&A(0;6)=(d_2) \cap Oy,B(6;2)=(d_1) \cap (d_2)\\
				&&C(8;0)=(d_1) \cap Ox,D \equiv O(0;0).
			\end{eqnarray*}
		}
		Số tiền thu về là $f(x;y)=3x+4y$ (triệu đồng).
		\begin{center}
			\renewcommand\arraystretch{1.6}
			\renewcommand{\tabcolsep}{6mm}
			\begin{tabular}{|c|c|c|c|c|}
				\hline 
				$M(x;y)$& $A$ & $B$ & $C$ & $D$ \\ 
				\hline 
				$f(x,y)=3x+4y$& $24$ & $26$ & $24$ & $0$ \\ 
				\hline 
			\end{tabular} 
		\end{center}
		Do đó $f(x,y)$ đạt giá trị lớn nhất tại $B(6;2)$.\\
		Vậy để thu được nhiều tiền nhất thì cần trồng 6 ha đậu và 2 ha cà.
	}	
\end{vd}
\baitaptl
\begin{bt}%[BG10-2022]%[Toanvo]%[0D4G4-4]
	Một gia đình cần ít nhất $900$ đơn vị protein và $400$ đơn vị lipit trong thức ăn mỗi ngày. Mỗi kg thịt bò chứa $800$ đơn vị protein và $200$ đơn vị lipit. Mỗi kg thịt lợn chứa $600$ đơn vị protein và $400$ đơn vị lipit. Biết rằng mỗi ngày gia đình này chỉ mua tối đa $1{,}5$ kg thịt bò và $1$ kg thịt lợn, giá tiền $1$ kg thịt bò là $200$ nghìn đồng, $1$ kg thịt lợn là $100$ nghìn đồng. Hỏi gia đình đó phải mua bao nhiêu kg thịt mỗi loại để số tiền bỏ ra là ít nhất.
	\loigiai{
		Gọi số kg thịt bò cần mua là $x$ (kg); số kg thịt lợn cần mua là $y$ (kg). Điều kiện: $0 \leq x \leq 1{,}5, 0 \leq y \leq 1$.\\
		Khi đó số đơn vị protein là $800x+600y$.\\
		Số đơn vị lipit là $200x+400y$.\\
		Ta có hệ bất phương trình $$\heva{&0\leq x\leq 1{,}5\\&0\leq y\leq 1\\&800x+600y\geq 900\\&200x+400y\geq200}\Leftrightarrow\heva{&0\leq x\leq 1{,}5\\&0\leq y\leq 1\\&8x+6y\geq9\\&x+2y\geq2.}$$	
		Vẽ các đường thẳng $(d_1)\colon x=1{,}5$, $(d_2)\colon y=1$, $(d_3)\colon 8x+6y=9$, $(d_4)\colon x+2y=2$. Ta được miền nghiệm của hệ bất phương trình là phần tô đậm trong hình vẽ.
		\begin{center}
			\begin{tikzpicture}[line join = round, line cap = round, >=stealth, font=\footnotesize, scale=1.2]
				\tikzset{label style/.style={font=\footnotesize}}
				\def \xmin{-3.5}
				\def \xmax{5.5}
				\def \ymin{-1.5}
				\def \ymax{6}
				\tkzDefPoints{0.375/1/A, 1.5/1/B, 1.5/0.25/C, 0.6/0.7/D}
				\draw[gray!20](\xmin,\ymin) grid (\xmax,\ymax);
				\draw [->](\xmin,0)--(0,0)node[below left]{$O$}--(\xmax,0)node[above]{$x$};
				\draw [->](0,\ymin)--(0,\ymax)node[right]{$y$};
				\foreach \x in {-3,-2,-1,1,2,3,4,5} \draw[shift={(\x,0)}] (0pt,2pt)--(0pt,-2pt) node[below]{$\x$};
				\foreach \y in {-1,2,3,4,5} \draw [shift={(0,\y)}] (2pt,0pt)--(-2pt,0pt) node[left]{$\y$};
				\draw [shift={(0,1)}] (2pt,0pt)--(-2pt,0pt) node[below left]{$1$};
				\tkzDrawLine[add=6 and 2](B,C)
				%\tkzDrawLine[add=0.5 and 0.5](B,C A,B A,D D,C)
				\tkzDrawLine[add=3 and 3](A,B)
				\tkzDrawLine[add=12 and 6](A,D)
				\tkzDrawLines[add=4 and 2](D,C)
				\tkzDrawPoints[fill=black](A,B,C,D)
				\tkzLabelPoints[above right](A,B,C)
				\tkzLabelPoints[below](D)
				\tkzFillPolygon[color=cyan,fill opacity=.5](A,B,C,D)
				\tkzLabelLine[pos=7,right](C,B){$(d_1)$}
				\tkzLabelLine[pos=3.5,above right](A,B){$(d_2)$}
				\tkzLabelLine[pos=12,above right](D,A){$(d_3)$}
				\tkzLabelLine[pos=5,above right](C,D){$(d_4)$}
			\end{tikzpicture}
		\end{center}
		Ta có 
		{\allowdisplaybreaks
			\begin{eqnarray*}
				&&A\left(\dfrac{3}{8};1\right)=(d_3) \cap (d_2), B(1{,}5;1)=(d_1) \cap (d_2),\\
				&&C(1{,}5;0{,}25)=(d_1) \cap (d_4), D\left(\dfrac{3}{5};\dfrac{7}{10}\right)=(d_3) \cap (d_4).
			\end{eqnarray*}
		}
		Số tiền bỏ ra là $f(x,y)=200x+100y$ (nghìn đồng).
		\begin{center}
			\renewcommand\arraystretch{1.6}
			\renewcommand{\tabcolsep}{6mm}
			\begin{tabular}{|c|c|c|c|c|}
				\hline 
				$M(x;y)$& $A$ & $B$ & $C$ & $D$ \\ 
				\hline 
				$f(x,y)=200x+100y$& $175$ & $400$ & $325$ & $190$ \\ 
				\hline 
			\end{tabular} 
		\end{center}
		Do đó $f(x,y)$ đạt giá trị nhỏ nhất tại $A\left(\dfrac{3}{8};1\right)$.\\
		Vậy để số tiền bỏ ra nhỏ nhất thì cần mua $\dfrac{3}{8}$ kg thịt bò và $1$ kg thịt lợn.
	}
\end{bt}
\begin{bt}%[BG10-2022]%[Toanvo]%[0D4G4-4]
	Người ta định dùng hai loại nguyên liệu để chiết xuất ít nhất $120$ kg hóa chất $A$ và $9$ kg hóa chất $B$. Từ mỗi tấn nguyên liệu loại I giá $4$ triệu đồng có thể chiết xuất được $20$ kg chất $A$ và $0{,}6$ kg chất $B$. Từ mỗi tấn nguyên liệu loại II giá $3$ triệu đồng có thể chiết xuất được $10$ kg chất $A$ và $1{,}5$ kg chất $B$. Hỏi phải dùng bao nhiêu tấn nguyên liệu mỗi loại để chi phí mua nguyên liệu là ít nhất. Biết rằng cơ sở cung cấp nguyên liệu chỉ có thể cung cấp không quá $10$ tấn nguyên liệu loại I và không quá $9$ tấn nguyên liệu loại II. 
	\loigiai{
		Gọi số tấn nguyên liệu loại I cần sử dụng là $x$ (tấn); số tấn nguyên liệu loại II cần sử dụng là  $y$ (tấn).\\
		Điều kiện: $0 \leq x \leq 10$, $0 \leq y \leq 9$.\\
		Khi đó số kg chất $A$ thu được là $20x+10y$, số kg chất $B$ thu được là $0{,}6x+1{,}5y$.\\
		Ta có hệ bất phương trình $$\heva{&0\leq x\leq 10\\&0\leq y\leq 9\\&20x+10y\geq120\\&0{,}6x+1{,}5y\geq9}\Leftrightarrow \heva{&0\leq x \leq10\\&0\leq y \leq 9\\&2x+y\geq 12\\&2x+5y\geq30.}$$
		Vẽ các đường thẳng $(d_1) \colon x=10$, $(d_2) \colon y=9$, $(d_3) \colon 2x+y=12$, $(d_4) \colon 2x+5y=30$. \\
		Ta có miền nghiệm của hệ bất phương trình là phần tô màu như hình vẽ:
		\begin{center}
			\begin{tikzpicture}[line join = round, line cap = round, >=stealth, font=\footnotesize, scale=.5]
				\tikzset{label style/.style={font=\footnotesize}}
				\def \xmin{-8}
				\def \xmax{17}
				\def \ymin{-5}
				\def \ymax{15}
				\tkzDefPoints{1.5/9/A, 10/9/B, 10/2/C, 3.75/4.5/D}
				\draw[gray!20](\xmin,\ymin) grid (\xmax,\ymax);
				\draw [->](\xmin,0)--(0,0)node[below left]{$O$}--(\xmax,0)node[above]{$x$};
				\draw [->](0,\ymin)--(0,\ymax)node[right]{$y$};
				\foreach \x in {-5,5,15} \draw[shift={(\x,0)}] (0pt,2pt)--(0pt,-2pt) node[below]{$\x$};
				\draw[shift={(10,0)}] (0pt,2pt)--(0pt,-2pt) node[below left]{$10$};
				\foreach \y in {5,10} \draw [shift={(0,\y)}] (2pt,0pt)--(-2pt,0pt) node[left]{$\y$};
				\tkzDrawLine[add=.8 and .8](C,B)
				\tkzDrawLine[add=.8 and .5](A,B)
				\tkzDrawLine[add=2 and 1.2](D,A)
				\tkzDrawLines[add=1 and 1.5](C,D)
				\tkzDrawPoints[fill=black](A,B,C,D)
				\tkzLabelPoints[above right](A,B,C)
				\tkzLabelPoints[below](D)
				\tkzFillPolygon[color=cyan,fill opacity=.5](A,B,C,D)
				\tkzLabelLine[pos=1.7,right](C,B){$(d_1)$}
				\tkzLabelLine[pos=1.3,above right](A,B){$(d_2)$}
				\tkzLabelLine[pos=2.2,below left](D,A){$(d_3)$}
				\tkzLabelLine[pos=2.4,below left](C,D){$(d_4)$}
			\end{tikzpicture}
		\end{center}
		Ta có 
		{\allowdisplaybreaks
			\begin{eqnarray*}
				&&(d_2) \cap (d_3)=A\left(\dfrac{3}{2};9\right),\ (d_2) \cap (d_1)=B(10;9),\\
				&&(d_1) \cap (d_4)=C(10;2),\ (d_4) \cap (d_3)=D\left(\dfrac{15}{4};\dfrac{9}{2}\right).
			\end{eqnarray*}
		}
		Chi phí mua nguyên liệu cần bỏ ra là $f(x,y)=4x+3y$ (triệu đồng).
		\begin{center}
			\renewcommand\arraystretch{1.6}
			\renewcommand{\tabcolsep}{6mm}
			\begin{tabular}{|c|c|c|c|c|}
				\hline 
				$M(x;y)$& $A$ & $B$ & $C$ & $D$ \\ 
				\hline 
				$f(x,y)=4x+3y$& $3$ & $67$ & $46$ & $28{,}5$ \\
				\hline 
			\end{tabular} 
		\end{center}
		Do đó $f(x,y)$ đạt giá trị nhỏ nhất tại $D\left(\dfrac{15}{4};\dfrac{9}{2}\right)$.\\
		Vậy để chi phí nguyên liệu là ít nhất ta cần sử dụng $\dfrac{15}{4}=3{,}75$ tấn nguyên liệu loại I và $\dfrac{9}{2}=4{,}5$ tấn nguyên liệu loại II.
	}
\end{bt}
\begin{bt}%[BG10-2022]%[Toanvo]%[0D4G4-4]
	Có ba nhóm máy $A$, $B$, $C$ dùng để sản xuất ra hai loại sản phẩm I và II. Để sản xuất một đơn vị sản phẩm mỗi loại phải lần lượt dùng các máy thuộc các nhóm khác nhau. Số máy trong một nhóm và số máy của từng nhóm cần thiết để sản xuất ra một đơn vị sản phẩm thuộc mỗi loại  được cho trong bảng sau:
	\begin{center}
		\begin{tabular}{|c|c|p{2.5cm}|p{2.5cm}|}
			\hline 
			\multirow{2}{*}{Nhóm}& \multirow{2}{*}{Số máy trong mỗi nhóm}  & \multicolumn{2}{p{5cm}|}{Số máy trong từng nhóm để sản xuất ra một đơn vị sản phẩm} \\ 
			\cline{3-4} 
			&  & Loại I & Loại II  \\ 
			\hline 
			A & 10 & 2 & 2 \\ 
			\hline 
			B & 4 & 0 & 2 \\ 
			\hline 
			C & 12 & 2 & 4 \\ 
			\hline 
		\end{tabular} 
	\end{center}
	Một đơn vị sản phẩm I lãi ba nghìn đồng, một đơn vị sản phẩm loại II lãi năm nghìn đồng. Hãy lập phương án để việc sản xuất hai loại sản phẩm trên có lãi cao nhất. 
	\loigiai{
		Gọi số sản phẩm loại I cần sản xuất là $x$; số sản phẩm loại II cần sản xuất là $y$. Điều kiện: $x,y \geq 0$.\\
		Số máy nhóm $A$ cần sử dụng là $2x+2y$.\\
		Số máy nhóm $B$ cần sử dụng là $2y$.\\
		Số máy nhóm $C$ cần sử dụng là $2x+4y$.\\
		Ta có hệ bất phương trình $$\heva{&x\geq0\\&y\geq0\\&2x+2y\leq10\\&2y\leq4\\&x+2y\leq6}\Leftrightarrow\heva{&x\geq0\\&0\leq y \leq 2\\&x+y\leq5\\&x+2y\leq6.}$$
		Vẽ các đường thẳng $(d_1) \colon y=2$, $(d_2) \colon x+y=5$, $(d_3) \colon x+2y=6$. Ta có miền nghiệm của bất phương trình là phần tô màu như hình vẽ:
		\begin{center}
			\begin{tikzpicture}[line join = round, line cap = round, >=stealth, font=\footnotesize, scale=1.2]
				\tikzset{label style/.style={font=\footnotesize}}
				\def \xmin{-2.5}
				\def \xmax{8}
				\def \ymin{-2}
				\def \ymax{6.5}
				\tkzDefPoints{0/2/A, 2/2/B, 4/1/C, 5/0/D, 0/0/E}
				\draw[gray!20](\xmin,\ymin) grid (\xmax,\ymax);
				\draw [->](\xmin,0)--(0,0)node[below left]{$O$}--(\xmax,0)node[above]{$x$};
				\draw [->](0,\ymin)--(0,\ymax)node[right]{$y$};
				\foreach \x in {-2,-1,1,2,3,4,5,6,7} \draw[shift={(\x,0)}] (0pt,2pt)--(0pt,-2pt) node[below]{$\x$};
				\foreach \y in {-1,1,3,4,5,6} \draw [shift={(0,\y)}] (2pt,0pt)--(-2pt,0pt) node[left]{$\y$};
				\draw [shift={(0,2)}] (2pt,0pt)--(-2pt,0pt) node[below left]{$2$};
				\tkzDrawLine[add=1 and 2.7](A,B)
				%\tkzDrawLine[add=0.5 and 0.5](B,C A,B A,D D,C)
				\tkzDrawLine[add=1.5 and 5](D,C)
				\tkzDrawLine[add=2 and 2](C,B)
				%\tkzDrawLines[add=4 and 2](D,C)
				\tkzDrawPoints[fill=black](A,B,C,D,E)
				\tkzLabelPoints[above right](A,B,C,D)
				%\tkzLabelPoints[below](D)
				\tkzLabelPoints[below right](E)
				\tkzFillPolygon[color=cyan,fill opacity=.5](A,B,C,D,E)
				\tkzLabelLine[pos=3.6,above](A,B){$(d_1)$}
				\tkzLabelLine[pos=6,below left](D,C){$(d_2)$}
				\tkzLabelLine[pos=3,below left](C,B){$(d_3)$}
				%\tkzLabelLine[pos=5,above right](C,D){$(d_4)$}
			\end{tikzpicture}
		\end{center} 
		Ta có 
		{\allowdisplaybreaks
			\begin{eqnarray*}
				&&(d_1) \cap Oy=A(0;2), (d_1) \cap (d_3)=B(2;2), (d_2)\cap (d_3)=C(4;1)\\
				&&(d_2) \cap Ox=D(5;0), E \equiv O=(0;0). 
			\end{eqnarray*}
		}
		Lãi suất thu được là $f(x,y)=3x+5y$ (nghìn đồng).
		\begin{center}
			\renewcommand\arraystretch{1.6}
			\renewcommand{\tabcolsep}{6mm}
			\begin{tabular}{|c|c|c|c|c|c|}
				\hline 
				$M(x;y)$& $A$ & $B$ & $C$ & $D$ & $E$ \\ 
				\hline 
				$f(x,y)=3x+5y$& $10$ & $16$ & $17$ & $15$ & $0$ \\
				\hline 
			\end{tabular} 
		\end{center}
		Do đó $f(x,y)$ đạt giá trị lớn nhất tại $C(4;1)$.\\
		Vậy phương án sản xuất 4 sản phẩm loại I và $1$ sản phẩm loại II sẽ cho lãi cao nhất.
	}
\end{bt}
\begin{bt}%[BG10-2022]%[Toanvo]%[0D4G4-4]
	Một nhà khoa học nghiên cứu về tác động phối hợp của vitamin $A$ và vitamin $B$ đối với cơ thể con người. Kết quả như sau:
	\begin{enumerate}
		\item  Một người có thể tiếp nhận được mỗi ngày không quá $600$ đơn vị vitamin $A$ và không quá $500$ đơn vị vitamin $B$.
		\item Một người mỗi ngày cần từ $400$ đến $1000$ đơn vị vitamin cả $A$ lẫn $B$.
		\item  Do tác động phối hợp của hai loại vitamin, mỗi ngày số đơn vị vitamin $B$ phải nhiều hơn $\dfrac{1}{2}$ số đơn vị vitamin $A$ nhưng không nhiều hơn ba lần số đơn vị vitamin $A$. Biết giá một đơn vị vitamin $A$ là $9$ đồng và giá một đơn vị vitamin $B$ là $7{,}5$ đồng.
	\end{enumerate}
	Tìm phương án dùng vitamin $A$ và vitamin $B$ thỏa mãn các điều kiện trên sao cho số tiền phải trả ít nhất.
	\loigiai{
		Gọi số đơn vị vitamin $A$ cần dùng là $x$; số đơn vị vitamin $B$ cần dùng là $y$.\\
		Điều kiện: $0 \leq x \leq 600$, $0 \leq y \leq 500$.\\
		Tổng số đơn vị vitamin $A$ và vitamin $B$ cần dùng là $x+y$.\\
		Ta có hệ bất phương trình $\heva{&0\leq x\leq 600\\&0\leq y\leq 500\\&400\leq x+y\leq 1000\\&\dfrac{x}{2}\leq y\leq 3x.}$\\
		Vẽ các đường thẳng $(d_1) \colon x=600$, $(d_2) \colon y=500$, $(d_3) \colon x+y=400$, $(d_4) \colon x+y=1000$, \\
		$(d_5) \colon \dfrac{x}{2}-y=0$, $(d_6) \colon y=3x$.\\
		Ta có miền nghiệm của hệ bất phương trình như hình vẽ:
		\begin{center}
			\begin{tikzpicture}[line join = round, line cap = round, >=stealth, font=\footnotesize, scale=1.2]
				\tikzset{label style/.style={font=\footnotesize}}
				\def \xmin{-3.5}
				\def \xmax{5.5}
				\def \ymin{-1}
				\def \ymax{6}
				\tkzDefPoints{0.833333/2.5/A, 2.5/2.5/B, 3/2/C, 3/1.5/D, 1.33333/0.66666/E, 0.5/1.5/F}
				\draw[gray!20](\xmin,\ymin) grid (\xmax,\ymax);
				\draw [->](\xmin,0)--(0,0)node[above left]{$O$}--(\xmax,0)node[above]{$x$};
				\draw [->](0,\ymin)--(0,\ymax)node[right]{$y$};
				\draw[shift={(-3,0)}] (0pt,2pt)--(0pt,-2pt) node[below]{$-600$};
				\draw[shift={(-2,0)}] (0pt,2pt)--(0pt,-2pt) node[below]{$-400$};
				\draw[shift={(-1,0)}] (0pt,2pt)--(0pt,-2pt) node[below]{$-200$};
				\draw[shift={(1,0)}] (0pt,2pt)--(0pt,-2pt) node[below]{$200$};
				\draw[shift={(2,0)}] (0pt,2pt)--(0pt,-2pt) node[below]{$400$};
				\draw[shift={(3,0)}] (0pt,2pt)--(0pt,-2pt) node[below right]{$600$};
				\draw[shift={(4,0)}] (0pt,2pt)--(0pt,-2pt) node[below]{$800$};
				\draw[shift={(5,0)}] (0pt,2pt)--(0pt,-2pt) node[below]{$1000$};
				\draw[shift={(0,1)}] (2pt,0pt)--(-2pt,0pt) node[left]{$200$};
				\draw[shift={(0,2)}] (2pt,0pt)--(-2pt,0pt) node[left]{$400$};
				\draw[shift={(0,3)}] (2pt,0pt)--(-2pt,0pt) node[left]{$600$};
				\draw[shift={(0,4)}] (2pt,0pt)--(-2pt,0pt) node[left]{$800$};
				\draw[shift={(0,5)}] (2pt,0pt)--(-2pt,0pt) node[left]{$1000$};
				\tkzDrawLine[add=4 and 7](D,C)
				\tkzDrawLine[add=1.5 and 2](B,A)
				\tkzDrawLine[add=1.6 and 4](E,F)
				\tkzDrawLines[add=4.5 and 6](C,B)
				\tkzDrawLines[add=1.7 and 1](E,D)
				\tkzDrawLines[add=2.2 and 3](F,A)
				\tkzDrawPoints[fill=black](A,B,C,D,E,F)
				\tkzLabelPoints[above right](A,B,C)
				\tkzLabelPoints[below right](D)
				\tkzLabelPoints[below](E)
				\tkzLabelPoints[left](F)
				\tkzFillPolygon[color=cyan,fill opacity=.5](A,B,C,D,E,F)
				\tkzLabelLine[pos=7.5,right](D,C){$(d_1)$}
				\tkzLabelLine[pos=3,above](B,A){$(d_2)$}
				\tkzLabelLine[pos=5,above right](E,F){$(d_3)$}
				\tkzLabelLine[pos=7,above](C,B){$(d_4)$}
				\tkzLabelLine[pos=2,below](E,D){$(d_5)$}
				\tkzLabelLine[pos=4,left](F,A){$(d_6)$}
			\end{tikzpicture}
		\end{center} 
		Ta có 
		{\allowdisplaybreaks
			\begin{eqnarray*}
				&&(d_1) \cap (d_6)=A\left(\dfrac{500}{3};500\right), (d_4) \cap (d_2)=B(500;500), (d_1) \cap (d_4)=C(600;400),\\
				&&(d_1) \cap (d_5)=D(600;300), (d_3) \cap (d_5)=E\left(\dfrac{800}{3};\dfrac{400}{3}\right), (d_3) \cap (d_6)=F(100;300).
			\end{eqnarray*}
		}
		Số tiền phải trả là $f(x,y)=9x+7{,}5y$ (nghìn đồng).
		\begin{center}
			\renewcommand\arraystretch{1.6}
			\renewcommand{\tabcolsep}{6mm}
			\begin{tabular}{|c|c|c|c|c|c|c|}
				\hline 
				$M(x;y)$& $A$ & $B$ & $C$ & $D$ & $E$ & $F$ \\ 
				\hline 
				$f(x,y)=9x+7{,}5y$& $5250$ & $8250$ & $8400$ & $7650$ & $3400$ & $3150$ \\
				\hline 
			\end{tabular} 
		\end{center}
		Do đó $f(x,y)=9x+7{,}5y$ đạt giá trị nhỏ nhất tại $F(100;300)$.\\
		Vậy phương án dùng mỗi ngày $100$ đơn vị vitamin $A$ và $300$ vitamin $B$ thì số tiền phải trả là ít nhất.
	}
\end{bt}
\subsection{Bài tập trắc nghiệm}
\Opensolutionfile{ans}[ans/ans-0D2-bieu-dien-hinh-hoc-tap-nghiem]
\begin{ex}%[0D4Y4-2]
	Điểm nào sau đây thuộc miền nghiệm của hệ bất phương trình $\heva{&2x+7y-3>0\\&x-2y\ge 0}$?
	\choice
	{$P(-1;-5)$}
	{$O(0;0)$}
	{$M(3;-1)$}
	{\True $N(2;0)$}
	\loigiai{
		Thay lần lượt tọa độ các điểm vào hệ bất phương trình, ta thấy tọa độ điểm $N$ thỏa mãn.\\
		Vậy điểm $N(2;0)$ thuộc miền nghiệm của hệ bất phương trình.
	}
\end{ex}
\begin{ex}%[0D4Y4-4]
	Miền nghiệm của hệ bất phương trình $\heva{&2x-5y-1>0\\&2x+y+5>0\\&x+y+1<0}$ chứa điểm nào trong các điểm sau?
	\choice
	{$(0;0)$}
	{$(1;0)$}
	{\True $(0;-2)$}
	{$(0;2)$}
	\loigiai{Thay điểm $(0;-2)$ vào hệ bất phương trình, ta có 
		$\heva{&2 \cdot 0 - 5 \cdot (-2)-1=9>0\\&2 \cdot 0 + (-2) +5=3>0\\&0+(-2)+1=-1<0}$ (đúng).
	}
\end{ex}

\begin{ex}%[0D4Y4-4]
	Miền nghiệm của hệ bất phương trình $\heva{& x-y\ge 3 \\ & 2x+y<4}$ chứa điểm nào trong các điểm sau?
	\choice
	{\True $(1;-3)$}
	{$(-2;1)$}
	{$(3;-2)$}
	{$(4;1)$}
	\loigiai{Thay điểm $(1;-3)$ vào hệ bất phương trình, ta có
		$\heva{& 1-(-3) = 4 \ge 3 \\ & 2 \cdot 1 + (-3) = -2 <4}$ (đúng).
	}
\end{ex}

\begin{ex}%[0D4B4-4]
	Miền nghiệm của hệ bất phương trình $\heva{& 2x-y>0\\ & x+y\ge -1 \\ & x-y<-2}$ \textbf{không} chứa điểm nào trong các điểm sau?
	\choice
	{$(5;8)$}
	{$(6;9)$}
	{$(4;7)$}
	{\True $(3,4)$}
	\loigiai{Thay điểm $(3;4)$ vào hệ bất phương trình, ta có
		$\heva{& 2 \cdot 3 - 4 = 2>0 \\ & 3+4=7 \ge -1 \\ & 3-4 = -1<-2}$ (sai).
	}
\end{ex}

\begin{ex}%[0D4B4-4]
	Điểm nào sau đây thuộc miền nghiệm của hệ bất phương trình $\heva{& 2x+3y-1>0\\& 5x-y+4<0.}$
	\choice
	{$(0;0)$}
	{$(-2;0)$}
	{$(-1;-4)$}
	{\True $(-3;4)$}
	\loigiai{
		Thay tọa độ từng điểm vào mỗi hệ bất phương trình.
		\begin{itemize}
			\item Với điểm $(0;0)$ ta được $2\cdot 0+3 \cdot 0-1=-1<0$ (sai) nên không thỏa mãn bất phương trình đầu.
			\item Với điểm $(-2;0)$ ta được $2 \cdot (-2)+3\cdot 0-1=-5<0$ (sai) nên không thỏa mãn bất phương trình đầu.
			\item Với điểm $(-1;-4)$ ta được $2 \cdot (-1)+3 \cdot (-4)-1=-15<0$ (sai) nên không thỏa mãn bất phương trình đầu.
			\item Với điểm $(-3;4)$ ta được $\heva{&2 \cdot(-3)+3\cdot 4-1=5>0 \\ &5 \cdot (-3)-4+4=-15<0}$ (đúng) thỏa mãn cả hai bất phương trình của hệ.
		\end{itemize}
	}
\end{ex}

\begin{ex}%[0D4B4-4]
	Cho hệ bất phương trình $\heva{&y\geq0\\&3x+2y-6<0}$ có miền nghiệm $S$ và bốn điểm $O(0;0)$, $A(2;3)$, $B(-1;1)$, $C(-1;3)$. Trong các điểm đã cho, có bao nhiêu điểm thuộc $S$?
	\choice
	{$1$}
	{$2$}
	{\True $3$}
	{$4$}
	\loigiai{Thay điểm $O(0;0)$ vào hệ bất phương trình, ta có 
		$\heva{&0\geq0\\&3\cdot 0+2 \cdot 0-6=-6<0}$ (đúng).\\
		Thay điểm $A(2;3)$ vào hệ bất phương trình, ta có 
		$\heva{&3\geq 0\\&3\cdot 2+2 \cdot 3-6=6<0}$ (sai).\\
		Thay điểm $B(-1;1)$ vào hệ bất phương trình, ta có 
		$\heva{&1\geq 0\\&3\cdot (-1)+2 \cdot 1-6=-7<0}$ (đúng).\\
		Thay điểm $C(-1;3)$ vào hệ bất phương trình, ta có 
		$\heva{&3\geq 0\\&3\cdot (-1)+2 \cdot 3-6=-3<0}$ (đúng).
	}
\end{ex}

\begin{ex}%[0D4B4-4]
	Xét hệ bất phương trình $\heva{& x+y\le 2\\ & x-2y\ge -1 \\ & y\ge 1}$ và bốn điểm $A(1;1)$, $B(2;1)$, $C(0;1)$, $D(-2;0)$. Trong các điểm trên, có bao nhiêu điểm thuộc miền nghiệm của hệ bất phương trình đã cho?
	\choice
	{\True $1$}
	{$2$}
	{$3$}
	{$4$}
	\loigiai{Thay điểm $A(1;1)$ vào hệ bất phương trình, ta có 
		$\heva{& 1+1 =2 \le 2\\ & 1-2 \cdot 1 =-1 \ge -1 \\ & 1\ge 1}$ (đúng).\\
		Thay điểm $B(2;1)$ vào hệ bất phương trình, ta có 
		$\heva{& 2+1 =3 \le 2\\ & 2-2 \cdot 1 =0 \ge -1 \\ & 1\ge 1}$ (sai).\\
		Thay điểm $C(0;1)$ vào hệ bất phương trình, ta có 
		$\heva{& 0+1 =1 \le 2\\ & 0-2 \cdot 1 =-2 \ge -1 \\ & 1\ge 1}$ (sai).\\
		Thay điểm $D(-2;0)$ vào hệ bất phương trình, ta có 
		$\heva{& -2+0 =-2 \le 2\\ & -2-2 \cdot 0 =-2 \ge -1 \\ & 0\ge 1}$ (sai).\\
	}
\end{ex}

\begin{ex}%[0D4B4-4]
	Cặp số $(x;y)$ nào sau đây là một nghiệm của hệ bất phương trình $\heva{&2x+3y-1>0\\&5x-y+4\leq0}$?
	\choice
	{\True $(0;4)$}
	{$(0;0)$}
	{$(-2;-4)$}
	{$(-3;-4)$}
	\loigiai{Thay cặp số $(0;4)$ vào hệ bất phương trình đã cho, ta có
		$\heva{&2 \cdot 0 + 3 \cdot 4-1=11>0\\&5 \cdot 0-4+4 = 0 \leq 0}$
		(đúng).
	} 
\end{ex}

\begin{ex}%[0D4B4-4]
	Trong các cặp số $(x;y)$ sau, cặp số nào \textbf{không là} nghiệm của hệ bất phương trình $$\heva{&x-y-2\leq0\\&3x-2y+2>0}?$$
	\choice
	{$(x;y)=(0;0)$}
	{$(x;y)=(1;1)$}
	{\True $(x;y)=(-1;1)$}
	{$(x;y)=(-1;-1)$}
	\loigiai{Thay cặp số $(-1;1)$ vào hệ bất phương trình đã cho, ta có
		$\heva{&-1-1-2=-4 \leq 0\\&3 \cdot (-1)-2 \cdot 1+2=-3>0}$ (sai).
	} 
\end{ex}

\begin{ex}%[0D4B4-4]
	Cặp số $(x;y)=(0;0)$ \textbf{không} là nghiệm của hệ bất phương trình nào trong các hệ bất phương trình sau?
	\def\dotEX{}	
	\choice
	{$\heva{& 2x-y<1 \\& x\ge 0 \\& y\le 1}$}
	{\True $\heva{& 2x+y<1 \\& x\ge 0 \\& y<0}$}
	{$\heva{& 2x-y<1 \\& x\ge 0 \\& y\ge 0}$}
	{$\heva{& 2x+y<1 \\& x\le 0 \\& y<1}$}
	\loigiai{Thay cặp số $(0;0)$ vào hệ bất phương trình $\heva{& 2x+y<1 \\& x\ge 0 \\& y<0}$, ta có
		$\heva{& 2 \cdot 0+0=0<1 \\& 0 \ge 0 \\& 0<0}$ (sai).
	} 
\end{ex}

\begin{ex}%[0D4B4-4]
	Điểm nào sau đây thuộc miền nghiệm của hệ bất phương trình $\heva{&5x+3y-19\leq 0 \\ &12x-5y-13\geq 0}$?
	\choice
	{\True $N(1+\sqrt{2};\sqrt{2})$}
	{$N(1+\sqrt{2};2+\sqrt{2})$}
	{$N(1;3+\sqrt{2})$}
	{$N(5+\sqrt{2};\sqrt{2})$}
	\loigiai{Thay điểm $N(1+\sqrt{2};\sqrt{2})$ vào hệ bất phương trình đã cho, ta có
		$$\heva{&5 \cdot \left(1+\sqrt{2} \right)+3 \cdot \sqrt{2}-19 = -14+8\sqrt{2} \leq 0 \\ &12 \cdot \left(1+\sqrt{2} \right)-5 \sqrt{2} -13 = -1+ 7 \sqrt{2}\geq 0} \text{ (đúng).}$$
	} 
\end{ex}

\begin{ex}%[0D4B4-4]
	Cặp số $(x;y)=(-1;3)$ là nghiệm của hệ bất phương trình nào trong các hệ bất phương trình sau?	
	\def\dotEX{}
	\choice
	{$\heva{& x-y\le 2\\& 3x+2y\ge 2\\& y\le 0\\& x<0}$}
	{$\heva{& x-y\le 2\\& 3x+y\ge 2\\& y\le 0\\& x<0}$}
	{$\heva{& x-y\le 2\\& 3x+y\ge 2\\& y\ge 0\\& x<0}$}
	{\True $\heva{& x-y\le 2\\& 3x+2y\ge 2\\& y\ge 0\\& x<0}
		$}
	\loigiai{Thay cặp số $(-1;3)$ vào hệ bất phương trình $\heva{& x-y\le 2\\& 3x+2y\ge 2\\& y\ge 0\\& x<0}$, ta có
		$\heva{& -1-3 =-4 \le 2\\& 3 \cdot (-1)+2 \cdot 3 = 3 \ge 2\\& 3 \ge 0\\& -1<0}$ (đúng).
	} 
\end{ex}

\begin{ex}%[0D4B4-4]
	Hệ bất phương trình $\heva{& y\le x+1 \\& y+x>3}$ nhận cặp số $(x;y)$ nào sau đây làm nghiệm của nó?
	\choice
	{$(x;y)=(2;1)$}
	{\True $(x;y)=(2;3)$}
	{$(x;y)=(3;0)$}
	{$(x;y)=(1;3)$}
	\loigiai{Thay cặp số $(2;3)$ vào hệ bất phương trình đã cho, ta có $\heva{&  3 \le 2+1=3 \\& 3+2=5>3}$ (đúng).
	} 
\end{ex}

\begin{ex}%[0D4K4-2]
	Cho hệ $\heva{&2x+3y<5\\&x+\dfrac{3}{2}y<5.}$  Gọi $S_1$ là tập nghiệm của bất phương trình $2x+3y<5$, $S_2$ là tập nghiệm của bất phương trình $x+\dfrac{3}{2}y<5$ và $S$ là tập nghiệm của hệ thì
	\choice
	{\True $S \subset S_2$}
	{$S_2\subset S_1 $}
	{$S_2\subset S $}
	{$S=S_1 \cup S_2$}
	\loigiai{Ta có $x+\dfrac{3}{2}y<5 \Leftrightarrow 2x + 3y < 10$. Do vậy với mọi cặp $(x_0, y_0)$ thỏa mãn bất phương trình $2x + 3y < 5$ đều thỏa mãn bất phương trình $2x + 3y < 10$.\\
		Vậy tập nghiệm của hệ ban đầu sẽ là tập con của bất phương trình $x+\dfrac{3}{2}y<5$. 
	}
\end{ex}

\begin{ex}%[0D4K4-2]
	Cho hệ phương trình $\heva{3x+y&>4 \hspace{0.5cm}(1)\\x+\dfrac{1}{3}y&>4\hspace{0.5cm}(2).}$ Gọi $S_1$ là tập nghiệm của bất phương trình $(1)$, $S_2$ là nghiệm của bất phương trình $(2)$ và $S$ là tập nghiệm của hệ bất phương trình đã cho. Khẳng định nào sau đay là đúng?
	\choice
	{$S_1\subset S_2$}
	{\True $S_2\subset S_1$}
	{$S_2\cup S=S_1$}
	{$S_1\subset S$}
	\loigiai{Lần lượt biểu diễn tập nghiệm của bất phương trình $(1)$, bất phương trình $(2)$ và hệ bất phương trình ta có các hình vẽ sau:\\
		\begin{tabular}{ccc}
			\begin{tikzpicture}[scale=0.5, font=\footnotesize, line join=round, line cap=round, >=stealth,xscale=1.5]
				\draw[->] (-1.2,0) -- (4.9,0)node[above]{$x$};
				\foreach \x in {-1,1,2,3,4}
				\draw[shift={(\x,0)},color=black] (0pt,2pt) -- (0pt,-2pt) node[below] {\footnotesize $\x$};
				\draw[->,color=black] (0,-1.2) -- (0,13.2)node[right]{$y$};
				\foreach \y in {1,2,3,4,5,6,7,8,9,10,11,12}
				\draw[shift={(0,\y)},color=black] (2pt,0pt) -- (-2pt,0pt) node[right] {\footnotesize $\y$};
				\node[above left] at (0,0){$O$};
				\clip(-2,-1) rectangle (5,13);
				\fill[pattern=north west lines] (1.666666,-1) -- (-1,-1) -- (-1,7) -- cycle;
				\draw[line width=1.2pt,smooth,samples=100,domain=-1:3] plot(\x,{4-3*(\x)});
			\end{tikzpicture}&\begin{tikzpicture}[scale=0.5, font=\footnotesize, line join=round, line cap=round, >=stealth,xscale=1.5]
				\draw[->] (-1.2,0) -- (4.9,0)node[above]{$x$};
				\foreach \x in {-1,1,2,3,4}
				\draw[shift={(\x,0)},color=black] (0pt,2pt) -- (0pt,-2pt) node[below] {\footnotesize $\x$};
				\draw[->,color=black] (0,-1.2) -- (0,13.2)node[right]{$y$};
				\foreach \y in {1,2,3,4,5,6,7,8,9,10,11,12}
				\draw[shift={(0,\y)},color=black] (2pt,0pt) -- (-2pt,0pt) node[right] {\footnotesize $\y$};
				\node[above left] at (0,0){$O$};
				\clip(-2,-1) rectangle (5,13);
				\fill[pattern=north east lines] (4.33333,-1) -- (-1,-1) -- (-1,13) -- (-0.333333,13) -- cycle;
				\draw[line width=1.2pt,smooth,samples=100,domain=-1:13] plot(\x,{12-3*(\x)});
			\end{tikzpicture}&\begin{tikzpicture}[scale=0.5, font=\footnotesize, line join=round, line cap=round, >=stealth,xscale=1.5]
				\draw[->] (-1.2,0) -- (4.9,0)node[above]{$x$};
				\foreach \x in {-1,1,2,3,4}
				\draw[shift={(\x,0)},color=black] (0pt,2pt) -- (0pt,-2pt) node[below] {\footnotesize $\x$};
				\draw[->,color=black] (0,-1.2) -- (0,13.2)node[right]{$y$};
				\foreach \y in {1,2,3,4,5,6,7,8,9,10,11,12}
				\draw[shift={(0,\y)},color=black] (2pt,0pt) -- (-2pt,0pt) node[right] {\footnotesize $\y$};
				\node[above left] at (0,0){$O$};
				\clip(-2,-1) rectangle (5,13);
				\fill[pattern=north east lines] (4.33333,-1) -- (-1,-1) -- (-1,13) -- (-0.333333,13) -- cycle;
				\fill[pattern=north west lines] (1.666666,-1) -- (-1,-1) -- (-1,7) -- cycle;
				\draw[line width=1.2pt,smooth,samples=100,domain=-1:13] plot(\x,{12-3*(\x)});
				\draw[line width=1.2pt,smooth,samples=100,domain=-1:3] plot(\x,{4-3*(\x)});
			\end{tikzpicture}\\
			$S_1$&$S_2$&$S$	
		\end{tabular}\\
		Dựa vào miền nghiệm của mỗi bất phương trình và của hệ ta thấy $S_2\subset S_1$ và $S=S_2$.}
\end{ex}

\begin{ex}%[0D4K4-2]
	Tìm số thực $a$ sao cho miền nghiệm của hệ bất phương trình $\heva{&x\ge 0\\&y\geq0\\&ax-3y\geq-12}$ là một tam giác có diện tích bằng $6$.
	\choice
	{\True $a=-4$} 
	{$a=4$}
	{$a=6$}
	{$a=12$}
	\loigiai
	{\immini{
			Xét $ax-3y=-12$.\\
			Với $x=0\Rightarrow y=4$.\\ Với $y=0\Rightarrow x=-\dfrac{12}{a}$.\\
			Do $x\ge 0$ suy ra $-\dfrac{12}{a}\ge 0$ hay $a<0$.\\
			Dựa vào hình bên ta có:\\
			$S$=$\dfrac{1}{2}\cdot 4\cdot \dfrac{-12}{a}=6$\\
			$\Rightarrow$ $a= -4$.\\
		}{
			\begin{tikzpicture}[scale=0.8, font=\footnotesize, line join=round, line cap=round, >=stealth]
				\draw[->] (-1,0) -- (4.5,0)node[above]{$x$};
				\foreach \x in {1,2,3,4}
				\draw[shift={(\x,0)},color=black] (0pt,2pt) -- (0pt,-2pt) node[below] {\footnotesize $\x$};
				\draw[->,color=black] (0,-1) -- (0,4.4)node[right]{$y$};
				\foreach \y in {1,2,3,4}
				\draw[shift={(0,\y)},color=black] (2pt,0pt) -- (-2pt,0pt) node[left]{\footnotesize $\y$};
				\node[below left] at (0,0){$O$};
				\node[below left] at (4,1.5){$\dfrac{-12}{a}$};
				\clip(-1,-1) rectangle (4.3,4.3);
				\fill[pattern=north east lines] (0,4) -- (3.25,0) -- (0,0) -- cycle;
				\draw[line width=1.2pt,smooth,samples=100,domain=-1:4] plot(\x,{4-1.25*(\x)});
				%\fill[pattern=north east lines] (-1,0) -- (-1,-1) -- (4,-1) -- (4,0)-- cycle;
			\end{tikzpicture}
		}
	}
\end{ex}

\begin{ex}%[0D4K4-2]
	Tính diện tích $S$ của miền nghiệm hệ bất phương trình $\heva{& y+x\le 3 \\ & y-x\le 3 \\ & y\ge -1}$.
	\choice
	{$S=8$}
	{$S=25$}
	{\True $S=16$}
	{$S=12$}
	\loigiai{
		\immini{
			Miền nghiệm là miền tam giác như hình vẽ.\\
			Diện tích $S=\dfrac{1}{2}.8.4=16$
		}
		{
			\begin{tikzpicture}[scale=0.5, font=\footnotesize, line join=round, line cap=round, >=stealth]
				\def\xmin{-5} \def\xmax{5}
				\def\ymin{-2} \def\ymax{4}
				\clip(\xmin,\ymin) rectangle (\xmax,\ymax);
				\tkzDefPoints{\xmax/\ymax/A1,\xmin/\ymax/A2,\xmin/\ymin/A3,\xmax/\ymin/A4}
				\fill[pattern=north east lines,pattern color=black!60] (A1)--(A2)--(A3)--(A4)--cycle;
				\tkzDefPoints{-4/-1/M,0/3/I,4/-1/N}
				\fill[color=white] (M)--(I)--(N)--cycle;
				\draw[domain=-5:5] plot(\x,{3-(\x)}) plot(\x,{3+(\x)}) plot(\x,{-1});
				\begin{scriptsize}
					\draw[->](\xmin,0)--(\xmax,0); \draw(\xmax-0.1,0) node[below]{$x$};
					\draw[->](0,\ymin)--(0,\ymax); \draw(0,\ymax-0.2) node[right]{$y$};
					\foreach \x in {-4,4}
					\draw (\x,0.05) -- ++(0,-0.1) node [above] {$\x$};
					\draw node [above right]{$O$} (0,3) 
					node[left]{$3$} (0,-1) 
					node[below left]{$-1$};
					\draw[dashed] (-4,0)--(M) (4,0)--(N);
				\end{scriptsize}
			\end{tikzpicture}
		}
	}
\end{ex}

\begin{ex}%[0D4K4-2]
	Tính diện tích $S$ của miền nghiệm của hệ bất phương trình $\heva{& x\ge -3 \\ & y+x\le 8 \\ & y-x\ge -2}$.
	\choice
	{$S=48$}
	{\True $S=64$}
	{$S=81$}
	{$S=49$}
	\loigiai{
		\immini{
			Miền nghiệm là miền tam giác như hình vẽ.\\
			Diện tích $S=\dfrac{1}{2}.16.8=64$.
		}
		{
			\begin{tikzpicture}[scale=0.3, font=\footnotesize, line join=round, line cap=round, >=stealth]
				\def\xmin{-6} \def\xmax{9}
				\def\ymin{-6} \def\ymax{12}
				\clip(\xmin,\ymin) rectangle (\xmax,\ymax);
				\tkzDefPoints{\xmax/\ymax/A1,\xmin/\ymax/A2,\xmin/\ymin/A3,\xmax/\ymin/A4}
				\fill[pattern=north east lines,pattern color=black!60] (A1)--(A2)--(A3)--(A4)--cycle;
				\tkzDefPoints{-3/-5/M,-3/11/N,5/3/I}
				\fill[color=white] (M)--(N)--(I)--cycle;
				\draw[domain=-5:5] plot(\x,{8-(\x)}) plot(\x,{(\x)-2});
				\draw (-3,\ymin)--(-3,\ymax);
				\begin{scriptsize}
					\draw[->](\xmin,0)--(\xmax,0); \draw(\xmax-0.1,0) node[below]{$x$};
					\draw[->](0,\ymin)--(0,\ymax); \draw(0,\ymax-0.2) node[right]{$y$};
					\foreach \x in {-3,5}
					\draw (\x,0.05) -- ++(0,-0.1) node [above right] {$\x$};
					\foreach \x in {-5,3,11}
					\draw (0.05,\x) -- ++(-0.1,0) node [above left] {$\x$};
					\draw node [above right]{$O$};
					\draw[dashed] (M)--(0,-5) (N)--(0,11) (0,3)--(I)--(5,0);
				\end{scriptsize}
			\end{tikzpicture}
		}
	}
\end{ex}

\begin{ex}%[0D4K4-2]
	Tính chu vi $P$ của miền nghiệm hệ bất phương trình $\heva{& x\ge -3 \\ & x\le 6 \\& y\le 5 \\ & y\ge -6}$.
	\choice
	{$P=38$}
	{$P=36$}
	{$P=42$}
	{\True $P=40$}
	\loigiai{
		\immini{
			Miền nghiệm là miền hình chữ nhật như hình vẽ.\\
			Chu vi $P=2(11+9)=40$.
		}
		{
			\begin{tikzpicture}[scale=0.3, font=\footnotesize, line join=round, line cap=round, >=stealth]
				\def\xmin{-5} \def\xmax{8}
				\def\ymin{-8} \def\ymax{7}
				\clip(\xmin,\ymin) rectangle (\xmax,\ymax);
				\tkzDefPoints{\xmax/\ymax/A1,\xmin/\ymax/A2,\xmin/\ymin/A3,\xmax/\ymin/A4}
				\fill[pattern=north east lines,pattern color=black!60] (A1)--(A2)--(A3)--(A4)--cycle;
				\tkzDefPoints{-3/-6/M,-3/5/N,6/5/P,6/-6/Q}
				\fill[color=white] (M)--(N)--(P)--(Q)--cycle;
				\draw (-3,\ymin)--(-3,\ymax) (6,\ymin)--(6,\ymax) (\xmin,5)--(\xmax,5) (\xmin,-6)--(\xmax,-6);
				\begin{scriptsize}
					\draw[->](\xmin,0)--(\xmax,0); \draw(\xmax-0.1,0) node[below]{$x$};
					\draw[->](0,\ymin)--(0,\ymax); \draw(0,\ymax-0.2) node[right]{$y$};
					\foreach \x in {-3,6}
					\draw (\x,0.05) -- ++(0,-0.1) node [above right] {$\x$};
					\foreach \x in {-6,5}
					\draw (0.05,\x) -- ++(-0.1,0) node [above left] {$\x$};
					\draw node [above right]{$O$};
				\end{scriptsize}
			\end{tikzpicture}
		}	
	}
\end{ex}

\begin{ex}%[0D4K4-2]
	Tìm giá trị của số thực $a$ sao cho miền nghiệm của hệ bất phương trình $\heva{& x\le a\\& x\ge 0\\& y\ge 0\\& y\le 2}$ có diện tích bằng $6$.
	\choice
	{$a=-3$}
	{$a=8$}
	{\True $a=3$}
	{$a=-8$}
	\loigiai{
		\immini{
			Từ giả thiết suy ra $a>0$.\\
			Diện tích $S=2a=6$. Do đó $a=3$.
		}
		{
			\begin{tikzpicture}[scale=0.6, font=\footnotesize, line join=round, line cap=round, >=stealth]
				\def\xmin{-1} \def\xmax{4}
				\def\ymin{-1} \def\ymax{3}
				\clip(\xmin,\ymin) rectangle (\xmax,\ymax);
				\tkzDefPoints{\xmax/\ymax/A1,\xmin/\ymax/A2,
					\xmin/\ymin/A3,\xmax/\ymin/A4}
				\fill[pattern=north east lines,pattern color=black!60] (A1)--(A2)--(A3)--(A4)--cycle;
				\tkzDefPoints{0/0/O,0/2/A,3/2/B,3/0/C}
				\fill[color=white] (O)--(A)--(B)--(C)--cycle;
				\draw (\xmin,2)--(\xmax,2) (3,\ymin)--(3,\ymax);
				\tkzDrawPoints(A,B,C)
				\begin{scriptsize}
					\draw[->](\xmin,0)--(\xmax,0); \draw(\xmax-0.1,0) node[below]{$x$};
					\draw[->](0,\ymin)--(0,\ymax); \draw(0,\ymax-0.2) node[right]{$y$};
					\draw node [below left]{$O$} 
					(A) node [below left]{$2$} 
					(C) node [below left]{$a$};
				\end{scriptsize}
			\end{tikzpicture}
		}	
	}
\end{ex}

\begin{ex}%[0D4K4-2]
	Tìm giá trị của số thực $a$ sao cho miền nghiệm của hệ bất phương trình $\heva{& x-y\ge a\\& x\le 0\\& y\ge 0}$ là một tam giác có diện tích bằng $2$.
	\choice
	{$a=2$}
	{\True $a=-2$}
	{$a=\sqrt{2}$}
	{$a=-\sqrt{2}$}
	\loigiai{
		\immini{
			Do $a\le 0$, $y\ge 0$ suy ra $x-y\le 0$
			suy ra $a<0$.\\
			Diện tích $S=\dfrac{1}{2} a^2 =2$. Do đó $a=-2$.
		}
		{
			\begin{tikzpicture}[scale=0.6, font=\footnotesize, line join=round, line cap=round, >=stealth]
				\def\xmin{-3} \def\xmax{1}
				\def\ymin{-1} \def\ymax{3}
				\clip(\xmin,\ymin) rectangle (\xmax,\ymax);
				\tkzDefPoints{\xmax/\ymax/A1,\xmin/\ymax/A2,
					\xmin/\ymin/A3,\xmax/\ymin/A4}
				\fill[pattern=north east lines,pattern color=black!60] (A1)--(A2)--(A3)--(A4)--cycle;
				\tkzDefPoints{0/0/O,0/2/A,-2/0/B}
				\fill[color=white] (O)--(A)--(B)--cycle;
				\draw[domain=-3:1] plot(\x,{(\x)+2});
				\tkzDrawPoints(A,B)
				\begin{scriptsize}
					\draw[->](\xmin,0)--(\xmax,0); \draw(\xmax-0.1,0) node[below]{$x$};
					\draw[->](0,\ymin)--(0,\ymax); \draw(0,\ymax-0.2) node[right]{$y$};
					\draw node [below left]{$O$} 
					(A) node [left]{$-a$} 
					(B) node [above]{$a$};
				\end{scriptsize}
			\end{tikzpicture}
		}	
	}
\end{ex}

\begin{ex}%[0D4K4-2]
	Tìm giá trị của số thực $m$ sao cho miền nghiệm của hệ bất phương trình $\heva{& x+my\le 2\\& x\ge 0\\& y\ge 0}$ là một tam giác có diện tích bằng $4$.
	\choice
	{$m=2$}
	{$m=4$}
	{$m=\dfrac{1}{4}$}
	{\True $m=\dfrac{1}{2}$}
	\loigiai{
		\immini{
			
			Diện tích $S=\dfrac{1}{2}\cdot 2\cdot \dfrac{2}{m}=4$.\\ Do đó $m=\dfrac{1}{2}$.
		}
		{
			\begin{tikzpicture}[scale=0.6, font=\footnotesize, line join=round, line cap=round, >=stealth]
				\def\xmin{-1} \def\xmax{3}
				\def\ymin{-1} \def\ymax{5}
				\clip(\xmin,\ymin) rectangle (\xmax,\ymax);
				\tkzDefPoints{\xmax/\ymax/A1,\xmin/\ymax/A2,
					\xmin/\ymin/A3,\xmax/\ymin/A4}
				\fill[pattern=north east lines,pattern color=black!60] (A1)--(A2)--(A3)--(A4)--cycle;
				\tkzDefPoints{0/0/O,0/4/A,2/0/B}
				\fill[color=white] (O)--(A)--(B)--cycle;
				\draw[domain=-1:3] plot(\x,{4-2*(\x)});
				\tkzDrawPoints(A,B)
				\begin{scriptsize}
					\draw[->](\xmin,0)--(\xmax,0); \draw(\xmax-0.1,0) node[below]{$x$};
					\draw[->](0,\ymin)--(0,\ymax); \draw(0,\ymax-0.2) node[right]{$y$};
					\draw node [below left]{$O$} 
					(A) node [left]{$\dfrac{2}{m}$} 
					(B) node [above]{$2$};
				\end{scriptsize}
			\end{tikzpicture}
		}	
	}
\end{ex}

\begin{ex}%[0D4K4-2]
	Tìm giá trị của số thực $m$ sao cho miền nghiệm của hệ bất phương trình $\heva{& x\ge 0\\& x\le 2\\& y\le -1\\& y\ge m}$ có chu vi bằng $8$.
	\choice
	{\True $m=-3$}
	{$m=2$}
	{$m=3$}
	{$m=-2$}
	\loigiai{
		\immini{
			Từ giả thiết suy ra $m<-1$ hay $-1-m>0$.\\
			Chu vi $P=2(-1-m+2)=8$.\\ Do đó $m=-3$.
		}
		{
			\begin{tikzpicture}[scale=0.6, font=\footnotesize, line join=round, line cap=round, >=stealth]
				\def\xmin{-1} \def\xmax{3}
				\def\ymin{-4} \def\ymax{1}
				\clip(\xmin,\ymin) rectangle (\xmax,\ymax);
				\tkzDefPoints{\xmax/\ymax/A1,\xmin/\ymax/A2,
					\xmin/\ymin/A3,\xmax/\ymin/A4}
				\fill[pattern=north east lines,pattern color=black!60] (A1)--(A2)--(A3)--(A4)--cycle;
				\tkzDefPoints{0/-1/A,0/-3/B,2/-3/C,2/-1/D}
				\fill[color=white] (A)--(B)--(C)--(D)--cycle;
				\draw (\xmin,-1)--(\xmax,-1) (\xmin,-3)--(\xmax,-3) (2,\ymin)--(2,\ymax);
				\tkzDrawPoints(A,B,C,D)
				\begin{scriptsize}
					\draw[->](\xmin,0)--(\xmax,0); \draw(\xmax-0.1,0) node[below]{$x$};
					\draw[->](0,\ymin)--(0,\ymax); \draw(0,\ymax-0.2) node[right]{$y$};
					\draw node [below left]{$O$} 
					(A) node [below left]{$-1$} 
					(0,-3) node [below left]{$m$} 				
					(2,0) node [above left]{$2$};
				\end{scriptsize}
			\end{tikzpicture}
		}	
	}
\end{ex}

\begin{ex}%[0D4K4-2]
	Tìm giá trị của số thực dương $m$ sao cho miền nghiệm của hệ bất phương trình $\heva{& x\ge 0\\& y\ge 0\\& 2x+3y\le 12\\& mx+y\ge 2}$ có diện tích bằng $8$.
	\choice
	{$m=2$}
	{$m=3$}
	{$m=\dfrac{1}{3}$}
	{\True $m=\dfrac{1}{2}$}
	\loigiai{
		\immini{
			Diện tích cần tìm $S_{ABCD}=S_{OBC}-S_{OAD}$.\\
			Do đó $S_{OAD}=S_{OBC}-S_{ABCD}=12-8=4=\dfrac{1}{2}.2.\dfrac{2}{m}$.\\
			Suy ra $m=\dfrac{1}{2}$.
		}
		{
			\begin{tikzpicture}[scale=0.6, font=\footnotesize, line join=round, line cap=round, >=stealth]
				\def\xmin{-2} \def\xmax{8}
				\def\ymin{-2} \def\ymax{5}
				\clip(\xmin,\ymin) rectangle (\xmax,\ymax);
				\tkzDefPoints{\xmax/\ymax/A1,\xmin/\ymax/A2,
					\xmin/\ymin/A3,\xmax/\ymin/A4}
				\fill[pattern=north east lines,pattern color=black!60] (A1)--(A2)--(A3)--(A4)--cycle;
				\tkzDefPoints{0/2/A,0/4/B,6/0/C,4/0/D}
				\fill[color=white] (A)--(B)--(C)--(D)--cycle;
				\draw[domain=-2:8] plot(\x,{4-2*(\x)/3}) plot(\x,{2-(\x)/2});
				\tkzDrawPoints(A,B,C,D)
				\begin{scriptsize}
					\draw[->](\xmin,0)--(\xmax,0); \draw(\xmax-0.1,0) node[below]{$x$};
					\draw[->](0,\ymin)--(0,\ymax); \draw(0,\ymax-0.2) node[right]{$y$};
					\draw node [below left]{$O$} 
					(A) node [below left]{$A(0;2)$} 
					(B) node [below left]{$B(0;4)$}
					(C) node [above right]{$C(6;0)$} 
					(D) node [below left]{$D(\dfrac{2}{m};0)$};;
				\end{scriptsize}
			\end{tikzpicture}
		}	
	}
\end{ex}

\begin{ex}%[0D4K4-3]
	Ngoài giờ học, bạn Nam làm thêm việc phụ bán cơm được $15$ nghìn đồng/một giờ và phụ bán tạp hóa được $10$ nghìn đồng/một giờ. Nam không thể làm thêm việc nhiều hơn $15$ giờ mỗi tuần. Gọi $x$, $y$ lần lượt là số giờ phụ bán cơm và phụ bán tạp hóa. Hệ bất phương trình nào sau đây xác định số giờ để làm mỗi việc nếu Nam muốn kiếm được ít nhất $100$ nghìn đồng mỗi tuần?
	\def\dotEX{}
	\choice
	{$\heva{& x+y\ge 15\\& 15x+10y\ge 100.}$}
	{$\heva{& x+y\le 15\\& 15x+10y>100.}$}
	{\True $\heva{& x+y\le 15\\& 15x+10y\ge 100.}$}
	{$\heva{& x+y>15\\& 15x+10y<100.}$}
	\loigiai{
		Gọi $x$, $y$ lần lượt là số giờ phụ bán cơm và phụ bán tạp hóa, tổng số giờ này không được nhiều hơn $15$ giờ nên $x+y\le 15$.\\
		Số tiền kiếm được sau $x$ giờ phục vụ cơm là $15x$.\\
		Số tiền kiếm được sau $y$ giờ bán tạp hóa là $10y$.\\
		Để Nam kiếm được ít nhất 100 nghìn đồng mỗi tuần thì $15x+10y\ge 100.$\\
		Vậy ta có hệ $\heva{& x+y\le 15\\& 15x+10y\ge 100.}$
	}
\end{ex}

\begin{ex}%[0D4K4-3]
	Để trở thành một thành viên của ban nhạc thì một sinh viên phải đạt điểm trung bình các môn học ít nhất là $7{,}0$ và phải có tối thiểu $5$ lần thực hành sau giờ học. Gọi $x$ là điểm trung bình các môn học và $y$ là số lần thực hành sau giờ học, hãy chọn hệ bất phương trình thể hiện tốt nhất tình huống này.
	\def\dotEX{}
	\choice
	{\True $\heva{& x\ge 7 \\ & y\ge 5.}$}
	{$\heva{& x\le 7 \\ & y\le 5.}$}
	{$\heva{& x<7 \\ & y<5.}$}
	{$\heva{& x>7 \\ & y>5.}$}
	\loigiai{
		Theo đề điểm trung bình các môn học ít nhất là $7,0$, tức là $x\ge 7$.\\
		Học sinh phải có tối thiểu $5$ lần thực hành sau giờ học, tức là $y\ge 5.$	\\
		Vậy ta có hệ $\heva{& x\ge 7 \\ & y\ge 5.}$
	}
\end{ex}

\begin{ex}%[0D4K4-4]
	Cho hệ bất phương trình $\heva{&x-y>0\\&2x+7y<0}$ có tập nghiệm $S$. Chọn khẳng định đúng trong các khẳng định sau.
	\choice
	{\True $(1;-1)\in S$}
	{$(1;-\dfrac{1}{2})\notin S$}
	{$(4;-1)\in S$}
	{$(-\dfrac{1}{2};-\dfrac{2}{7})\in S$}
	\loigiai{Bằng cách thay từng cặp giá trị vào hệ bất phương trình ta thấy chỉ có $(1;-1)$ và $(1;-\dfrac{1}{2})$ thỏa mãn. Vậy $(1;-1)\in S$ là đúng.}
\end{ex}

\begin{ex}%[0D4K4-4]
	Điểm $A\left (0;\dfrac{5}{3}\right )$ luôn thuộc miền nghiệm của bất phương trình nào trong các bất phương trình dưới đây (với $m$ là tham số thực)?
	\choice
	{\True $(m^2-4)x+3y-5\leq 0$}
	{$(m^2-4)x+3y-5> 0$}
	{$(m^2-4)x+3y-5< 0$}
	{$(m^2-4)x+3y+7\leq 0$}
	\loigiai{Thay điểm $A\left (0;\dfrac{5}{3}\right )$ vào bất phương trình $(m^2-4)x+3y-5\leq 0$, ta có
		$$(m^2-4) \cdot 0+3 \cdot \dfrac{5}{3} -5 = 0 \leq 0 \text{ (đúng).}$$
		Thay điểm $A$ vào lần lượt các bất phương trình ở các phương án còn lại, ta thấy không thỏa mãn.
	}
\end{ex}

\begin{ex}%[0D4K4-4]
	Hình vẽ dưới đây là biểu diễn hình học tập nghiệm của hệ bất phương trình nào? (với miền nghiệm là miền \textbf{không} gạch sọc và chứa bờ)
	\begin{center}
		\begin{tikzpicture}[scale=1, font=\footnotesize, line join=round, line cap=round, >=stealth]
			\clip(-4,-2.5) rectangle (4,4);
			\def\xmin{-4} \def\xmax{4}
			\def\ymin{-4} \def\ymax{4}
			\tkzDefPoints{\xmax/\ymax/A,\xmin/\ymax/B,\xmin/\ymin/C,\xmax/\ymin/D}
			%Nhập hệ số a,b,c đt d1: x+y-1=0-----------Đường thẳng MN
			\def\hsa{3} \def\hsb{4} \def\hsc{-8}
			\ifnum\hsb=0{
				\tkzDefPoint(-\hsc/ \hsa,\ymin){M};
				\tkzDefPoint(-\hsc/ \hsa,\ymax){N};}
			\else{ 
				\tkzDefPoint(\xmin,-\hsa*\xmin/\hsb-\hsc/\hsb){M};
				\tkzDefPoint(\xmax,-\hsa*\xmax/\hsb-\hsc/\hsb){N};}\fi
			%Nhập hệ số a,b,c đt d2: 2x-3y-1=0---------Đường thẳng PQ
			\def\hsa{5} \def\hsb{-12} \def\hsc{-3}
			\ifnum\hsb=0{
				\tkzDefPoint(-\hsc/ \hsa,\ymin){P};
				\tkzDefPoint(-\hsc/ \hsa,\ymax){Q};}
			\else{ 
				\tkzDefPoint(\xmin,-\hsa*\xmin/\hsb-\hsc/\hsb){P};
				\tkzDefPoint(\xmax,-\hsa*\xmax/\hsb-\hsc/\hsb){Q};}\fi
			\tkzInterLL(M,N)(P,Q) \tkzGetPoint{I}
			\tkzInterLL(M,N)(A,B) \tkzGetPoint{M'}
			\fill[pattern=north east lines,pattern color=blue!30] (C) rectangle (A);
			%\fill[white] (P)--(I)--(N)--(A)--(B)--cycle;
			\fill[white] (Q)--(I)--(M)--(A)--cycle;
			\tkzDrawSegment(M,N)
			\tkzDrawSegment(P,Q)
			\draw[->](\xmin,0)--(\xmax,0); \draw(\xmax-0.1,0) node[below]{$x$};
			\draw[->](0,\ymin)--(0,\ymax); \draw(0,\ymax-0.2) node[right]{$y$};
			\begin{scriptsize}
				\foreach \x in {-3,-2,-1,1,2,3} {
					\draw (\x,0.05) -- ++(0,-0.1) node [below] {$\x$};
					\draw (0.05,\x) -- ++(-0.1,0) node [left] {$\x$}; }
				\draw node [above left]{$O$};
				\draw (2.67,0) node [below] {$\dfrac{8}{3}$};
				\draw (0,-0.25) node [below left] {$-0,25$};
			\end{scriptsize}
			\draw [dashed] (0,1)--(3,1)--(3,0);
		\end{tikzpicture}
	\end{center}
	\def\dotEX{}
	\choice
	{\True $\heva{&3x+4y-8 \geq 0 \\ &5x-12y-3\leq 0.}$}
	{$\heva{&3x+4y-8 \leq 0 \\ &5x-12y-3\leq 0.}$}
	{$\heva{&3x+4y-8 \geq 0 \\ &5x-12y-3 \geq 0.}$}
	{$\heva{&3x+4y-3 \geq 0 \\ &5x-12y-8 \leq 0.}$}
	\loigiai{Xét các điểm $A(0; 2)$, $B (3; 1)$ và $C \left(0; - \dfrac{1}{4}\right)$ thuộc bờ. \\
		Điểm $B (3; 1)$ không thỏa mãn bất phương trình $3x+4y-8 \leq 0$ nên loại $\heva{&3x+4y-8 \leq 0 \\ &5x-12y-3\leq 0.}$ \\ 
		Điểm $A (0; 2)$ không thỏa mãn bất phương trình $5x - 12 y - 3 \geq 0$ nên loại $\heva{&3x+4y-8 \geq 0 \\ &5x-12y-3 \geq 0.}$\\
		Điểm $C \left(0; - \dfrac{1}{4}\right)$ không thỏa mãn bất phương trình $3x + 4y - 3 \geq 0$ nên loại $\heva{&3x+4y-3 \geq 0 \\ &5x-12y-8 \leq 0.}$
	}
\end{ex}

\begin{ex}%[0D4K4-4]
	\immini{Phần mặt phẳng không bị gạch, kể cả phần biên của nó trên đường thẳng $y=0$ trong hình vẽ bên là miền nghiệm của hệ bất phương trình nào?
		\def\dotEX{}
		\choice
		{$ \heva{&y\leq0\\ &2x+y>1.}$}
		{\True $ \heva{&x+y<2\\&y\geq0.}$}
		{$\heva{&2x-2y>6\\&2x+y\geq1.}$}
		{$ \heva{&y\leq0\\&x+y<1.}$}
	}{\begin{tikzpicture}[scale=1, font=\footnotesize, line join=round, line cap=round, >=stealth]
			\draw[->] (-1,0) -- (3.5,0)node[above]{$x$};
			\foreach \x in {1,2,3}
			\draw[shift={(\x,0)},color=black] (0pt,2pt) -- (0pt,-2pt) node[below] {\footnotesize $\x$};
			\draw[->,color=black] (0,-1) -- (0,3.5)node[right]{$y$};
			\foreach \y in {1,2,3}
			\draw[shift={(0,\y)},color=black] (2pt,0pt) -- (-2pt,0pt) node[left] {\footnotesize $\y$};
			\node[below left] at (0,0) {$O$};
			\clip(-1,-1) rectangle (3.5,3.5);
			\fill[pattern=north east lines] (-1,3) -- (3,3) -- (3,-1) -- cycle;
			\draw[line width=1.2pt,smooth,samples=100,domain=-1:4] plot(\x,{2-(\x)});
			\fill[pattern=north east lines] (-1,0)--(-1,-1) --(3,-1)--(3,0)-- cycle;
		\end{tikzpicture}
	}
	\loigiai{Phần không bị gạch nằm phía trên trục hoành nên nó là miền nghiệm của bất phương trình $y \geq 0$ $(1)$. \\ 
		Điểm $A (0; 1)$ thỏa mãn bất phương trình $x + y < 2$ nên miền không bị gạch chính là miền nghiệm của bất phương trình $x +y < 2$ $(2)$. \\
		Từ $(1)$ và $(2)$ suy ra phần mặt phẳng không bị gạch, kể cả phần biên của nó trên đường thẳng $y=0$ trong hình vẽ bên là miền nghiệm của hệ bất phương trình $ \heva{&x+y<2\\&y\geq0.}$
	}
\end{ex}

\begin{ex}%[0D4K4-4]
	\immini{Phần mặt phẳng không bị gạch, kể cả phần biên của nó trên đường thẳng $d$ trong hình vẽ bên là miền nghiệm của hệ bất phương trình nào?
		\def\dotEX{}
		\choice
		{$ \heva{&2x+3y\leq6\\ &2x+y>2.}$}
		{$ \heva{&x-2y<1\\&3x+2y\leq3.}$}
		{\True $\heva{&2x+3y<6\\&2x+y\leq2.}$}
		{$ \heva{&2x-3y\leq6\\&2x+y<1.}$}
	}{\begin{tikzpicture}[scale=1, font=\footnotesize, line join=round, line cap=round, >=stealth]
			\draw[->] (-1,0) -- (4.3,0)node[above]{$x$};
			\foreach \x in {1,2,3,4}
			\draw[shift={(\x,0)},color=black] (0pt,2pt) -- (0pt,-2pt) node[below] {\footnotesize $\x$};
			\draw[->,color=black] (0,-1) -- (0,4)node[right]{$y$};
			\foreach \y in {1,2,3}
			\draw[shift={(0,\y)},color=black] (2pt,0pt) -- (-2pt,0pt) node[left] {\footnotesize $\y$};
			\node[below left] at (0,0) {$O$};
			\node[below left] at (1,-0.5) {$d$};
			\node at (0.5,0.6) {$d$};
			\clip(-1,-1) rectangle (4,3.5);
			\fill[pattern=north east lines] (-1,2.66667) -- (-1,4) -- (4,4) -- (4,-0.6667) -- (3,0) -- cycle;
			\draw[line width=1.2pt,smooth,samples=100,domain=-1:4] plot(\x,{2-0.666667*(\x)});
			\fill[pattern=north east lines] (-0,75,3.5)--(0,2) --(0.5,1)--(1,0)--(1.5,-1)--(4,-1)--(4,0)--(4,4)-- cycle;
			\draw[line width=1.2pt,smooth,samples=100,domain=-1:4] plot(\x,{2-2*(\x)});
		\end{tikzpicture}
	}	
	\loigiai{Đường thẳng $d$ có phương trình $2x + y = 2$.\\ 
		Đường thẳng $\Delta$ đi qua hai điểm $(3; 0)$ và $(0; 2)$ có phương trình là $2x + 3y = 6$. \\
		Tại điểm $A (0; 1)$, ta có $2. 0 + 1 = 1 < 2$, suy ra điểm $A$ thuộc miền nghiệm của bất phương trình $2x + y < 2$ $(1)$. Tương tự, ta cũng kiểm tra được rằng điểm $A$ cũng thuộc miền nghiệm của bất phương trình $2x + 3y < 6$ $(2)$. \\
		Vậy phần mặt phẳng không bị gạch, kể cả phần biên của nó trên đường thẳng $d$ trong hình vẽ bên là miền nghiệm của hệ bất phương trình $\heva{&2x+3y<6\\&2x+y\leq2.}$
	}
\end{ex}

\begin{ex}%[0D4K4-4]
	\immini{Cho hệ bất phương trình $\heva{& x \geq -2\\& y \geq -2\\ & x+y<2.}$ Biết rằng  $A$, $B$, $C$ là giao điểm của hai trong ba đường thẳng $x=-2$, $y=-2$, $x+y=2$ \textit{(được cho như hình vẽ)}. Khẳng định nào dưới đây là đúng?}
	{
		\begin{tikzpicture}[scale=0.5, font=\footnotesize, line join=round, line cap=round, >=stealth]
			\clip(-3.5,-4) rectangle (5,5);
			\fill[pattern=north east lines] (-5,-4) -- (-5,5) -- (5,5) -- (5,-4) -- cycle;
			\fill[white] (-2,-2) -- (-2,4) -- (4,-2) -- cycle;
			\draw[violet,line width=2pt,samples=100] (-2,-2) -- (-2,4) -- (4,-2)--(-2,-2);
			\draw[->] (-3.5,0) -- (5,0); \draw (4.7,0)node[above]{$x$};
			\foreach \x in {-2,2,4}
			\draw[shift={(\x,0)},color=black] (0pt,2pt) -- (0pt,-2pt) node[below left] {\footnotesize $\x$};
			\draw[->,color=black] (0,-4.2) -- (0,5); \draw (0,4.7) node[left]{$y$};
			\foreach \y in {-2,2,4}
			\draw[shift={(0,\y)},color=black] (2pt,0pt) -- (-2pt,0pt) node[above right] {\footnotesize $\y$};
			\node[above left] at (0,0){$O$};
			\node[below left] at (-2,4){$A$};
			\node[above right] at (4,-2){$B$};
			\node[below left] at (-2,-2){$C$};
			\draw[dashed] (4,0)--(4,-2);
			\draw[dashed] (-2,4)--(0,4);
			\draw[line width=1.2pt,smooth,samples=100,domain=-3:4] (-2,-4)--(-2,5);
			\draw[line width=1.2pt,smooth,samples=100,domain=-5:5] plot(\x,{-2-0*(\x)});
			\draw[line width=1.2pt,smooth,samples=100,domain=-3:5] plot(\x,{2-(\x)});
		\end{tikzpicture}
	}
	\choice
	{Miền nghiệm của hệ bất phương trình là miền tam giác $ABC$ bao gồm cả các cạnh $AB$, $AC$, $BC$ }
	{\True Miền nghiệm của hệ bất phương trình là miền tam giác $ABC$ bao gồm các cạnh  $AC$, $BC$ ngoại trừ điểm $A$, điểm $B$}
	{Miền nghiệm của hệ bất phương trình là miền tam giác $ABC$ bao gồm các cạnh $AB$, $AC$, $BC$ ngoại trừ điểm $A$, điểm $B$, điểm $C$}
	{Miền nghiệm của hệ bất phương trình là miền tam giác $ABC$ bao gồm các cạnh  $AB$, $BC$ ngoại trừ điểm $A$, điểm $C$}
	
	\loigiai{Ta thấy điểm $O$, điểm $C$, cạnh $AC$ (ngoại trừ điểm $A$), cạnh $BC$ (ngoại trừ điểm $B$) thuộc miền nghiệm của cả ba bất phương trình. Đường thẳng $x+y=2$ (chứa cạnh $AB$) không thuộc miền nghiệm của bất phương trình $x+y<2$. Vậy \textit{miền nghiệm của hệ bất phương trình trên là miền tam giác $ABC$ bao gồm các cạnh  $AC$, $BC$ ngoại trừ điểm $A$, điểm $B$.}
	}
\end{ex}

\begin{ex}%[0D4K4-4]
	\immini{Miền không bị gạch chéo (kể cả đường thẳng $d_1$ và $d_2$) là miền nghiệm của hệ bất phương trình nào sau đây? 
		\def\dotEX{}
		\choice
		{$\heva{x-y&\leq -2\\-2x-y&\geq -2.}$}
		{$\heva{x+y&\leq 2\\-2x-y&\geq -2.}$}
		{\True $\heva{x+y&\geq -2\\-2x+y&\geq -2.}$}
		{$\heva{-x-y&\leq -2\\2x-y&\geq -2.}$}}
	{\begin{tikzpicture}[scale=0.7, font=\footnotesize, line join=round, line cap=round, >=stealth]
			\clip(-3,-4) rectangle (3,3);
			\fill[pattern=north west lines] (-3,-4) rectangle (3,3);
			\fill[white] (0,-2) -- (-3,1) -- (-3,3) -- (2.5,3)  -- cycle;
			\draw[->] (-3,0) -- (3,0)node[above left]{$x$};
			\foreach \x in {-2,-1,1,2,3}
			\draw[shift={(\x,0)},color=black] (0pt,2pt) -- (0pt,-2pt) node[below] {\footnotesize $\x$};
			\draw[->,color=black] (0,-4) -- (0,3)node[below right]{$y$};
			\foreach \y in {-3,-2,-1,1,2}
			\draw[shift={(0,\y)},color=black] (2pt,0pt) -- (-2pt,0pt) node[left] {\footnotesize $\y$};
			\node[below left] at (0,0){$O$};
			\node[left] at (2,2){$d_2$};
			\node[right] at (-3,1){$d_1$};
			\draw[line width=1.2pt,smooth,samples=100,domain=-3:3] plot(\x,{-2-(\x)});
			\draw[line width=1.2pt,smooth,samples=100,domain=-3:3] plot(\x,{2*(\x)-2});
	\end{tikzpicture}}
	\loigiai{Đường thẳng $d_1$ đi qua hai điểm $(-2; 0)$ và $(0; -2)$ nên có phương trình là $x+y=-2$. Đường thẳng $d_2$ đi qua hai điểm $(1; 0)$ và $(0; -2)$ nên có phương trình là $-2x+y=-2$. \\
		Điểm $O (0; 0)$ thỏa mãn hệ bất phương trình $\heva{x+y&\geq -2\\-2x+y&\geq -2}$ nên phần không bị gạch chính là miền nghiệm của hệ bất phương trình trên. 
	}
\end{ex}

\begin{ex}%[0D4K4-4]
	\immini{
		Miền tam giác không bị gạch kể cả $3$ cạnh của nó trong hình bên là miền nghiệm của hệ bất phương trình nào?
		\def\dotEX{}
		\choice
		{$\heva{&y\geq0\\&5x-4y\leq10\\&5x+4y\leq10.}$}
		{$\heva{&x\geq0\\&4x-5y\leq10\\&5x+4y\leq10.}$}
		{\True $\heva{&x\geq0\\&5x-4y\leq10\\&4x+5y\leq10.}$}
		{$\heva{&x>0\\&5x-4y\leq10\\&4x+5y\leq10.}$}
	}{\begin{tikzpicture}[scale=1, font=\footnotesize, line join=round, line cap=round, >=stealth]
			\draw[->] (-1,0) -- (4.3,0)node[above]{$x$};
			\foreach \x in {1,2,3,4}
			\draw[shift={(\x,0)},color=black] (0pt,2pt) -- (0pt,-2pt) node[below] {\footnotesize $\x$};
			\draw[->,color=black] (0,-3) -- (0,4.3)node[right]{$y$};
			\foreach \y in {-2,-1,1,2,3,4}
			\draw[shift={(0,\y)},color=black] (2pt,0pt) -- (-2pt,0pt) node[left] {\footnotesize $\y$};
			\node[below left] at (0,0) {$O$};
			\clip(-1,-3) rectangle (4,4);
			\fill[pattern=north east lines] (-0.4,-3) -- (4,-3) -- (4,2.5)-- cycle;
			\draw[line width=1.2pt,smooth,samples=100,domain=-1:4] plot(\x,{-2.5+1.25*(\x)});
			\fill[pattern=north east lines] (-1,2.8)--(-1,4) --(4,4)--(4,-1.2)-- cycle;
			\draw[line width=1.2pt,smooth,samples=100,domain=-1:4] plot(\x,{2-0.8*(\x)});
			\fill[pattern=north east lines] (0,4)--(-1,4) --(-1,-3)--(0,-3)-- cycle;
		\end{tikzpicture}
	}
	\loigiai{Miền không bị gạch nằm bên phải trục tung nên là miền nghiệm của bất phương trình $x \geq 0$. \\
		Gọi $A (x_0; y_0)$ là một đỉnh của tam giác (điểm $A$ không nằm trên trục $Oy$). Dựa vào hình vẽ ta thấy $x_0 > 2$, $y_0 > 0$. Từ đó suy ra $5 x_0 + 4 y_0 > 10$. Vậy điểm $A$ không thuộc miền nghiệm của bất phương trình $5x + 4y \leq 10$. \\
		Vậy miền tam giác không bị gạch kể cả ba cạnh của nó trong hình bên là miền nghiệm của hệ bất phương trình $\heva{&x\geq0\\&5x-4y\leq10\\&4x+5y\leq10.}$ 
	}
\end{ex}

\begin{ex}%[0D4K4-4]
	\immini{Miền tam giác $ABC$ kể cả ba cạnh là miền nghiệm của hệ bất phương trình nào trong bốn hệ sau?
		\def\dotEX{}
		\choice
		{$\heva{y&\geq 0\\2x-3y&\geq 6\\x+y&\leq 3.} $}
		{\True $\heva{x&\geq 0\\-2x+3y&\geq -6\\x+y&\leq 3.} $}
		{$\heva{x&\geq 0\\-2x+3y&\leq -6\\x+y&\leq 3.} $}
		{$\heva{y&\geq 0\\2x-3y&\leq -6\\x+y&\leq 3.} $}
	}
	{\begin{tikzpicture}[scale=0.7, font=\footnotesize, line join=round, line cap=round, >=stealth,xscale=1]
			\fill[pattern=north east lines] (-2,-3) -- (-2,4) -- (4,4) -- (4,-3) -- cycle;
			\fill[white] (0,-2) -- (3,0) -- (0,3) -- cycle;
			\draw[violet,line width=2pt,samples=100] (0,-2) -- (3,0) -- (0,3)--(0,-2);
			\draw[->] (-2,0) -- (4.3,0)node[above]{$x$};
			\foreach \x in {3}
			\draw[shift={(\x,0)},color=black] (0pt,2pt) -- (0pt,-2pt) node[below] {\footnotesize $\x$};
			\draw[->,color=black] (0,-3) -- (0,4.3)node[right]{$y$};
			\foreach \y in {-2,3}
			\draw[shift={(0,\y)},color=black] (2pt,0pt) -- (-2pt,0pt) node[right] {\footnotesize $\y$};
			\node[above left] at (0,0){$O$};
			\node[below left] at (0,3){$A$};
			\node[above] at (3,0){$B$};
			\node[above left] at (0,-2){$C$};
			\clip(-2,-3) rectangle (4,4);
			\draw[line width=1.2pt,smooth,samples=100,domain=-3:4] plot(\x,{3-(\x)});
			\draw[line width=1.2pt,smooth,samples=100,domain=-3:4] plot(\x,{2/3*(\x)-2});
	\end{tikzpicture}}
	\loigiai{Ta thấy cạnh $AC$ thuộc đường thẳng $x=0$ và miền tam giác $ABC$ thuộc miền nghiệm của bất phương trình $x\geq 0$.\\
		Cạnh $AB$ thuộc đường thẳng $x+y=3$ và miền tam giác $ABC$ thuộc miền nghiệm của bất phương trình $x+y\leq 3$.\\
		Cạnh $BC$ thuộc đường thẳng $-2x+3y=-6$ và miền tam giác $ABC$ thuộc miền nghiệm của bất phương trình $-2x+3y\geq -6$.\\
		Vậy miền tam giác $ABC$ kể cả ba cạnh là miền nghiệm của hệ bất phương trình:  $\heva{x&\geq 0\\-2x+3y&\geq -6\\x+y&\leq 3.} $}
\end{ex}

\begin{ex}%[0D4K4-4]
	Phần mặt phẳng không bị gạch, kể cả phần biên của nó nằm trên đường thẳng $d$ trong hình vẽ nào sau đây là miền nghiệm của hệ bất phương trình $\heva{&y<0\\&2x+y\leq4.}$
	\choice
	{\begin{tikzpicture}[scale=.7, font=\footnotesize, line join=round, line cap=round, >=stealth]
			\draw[->] (-1,0) -- (4.3,0)node[above]{$x$};
			\foreach \x in {1,2,3,4}
			\draw[shift={(\x,0)},color=black] (0pt,2pt) -- (0pt,-2pt) node[below] {\footnotesize $\x$};
			\draw[->,color=black] (0,-1) -- (0,4.3)node[right]{$y$};
			\foreach \y in {1,2,3,4}
			\draw[shift={(0,\y)},color=black] (2pt,0pt) -- (-2pt,0pt) node[left]{\footnotesize $\y$};
			\node[below left] at (0,0){$O$};
			\node[below left] at (0.5,-0.2){$d$};
			\clip(-1,-1) rectangle (4.3,4.3);
			\fill[pattern=north east lines] (-0.25,-1) -- (-1,-1) -- (-1,4.3) -- (2.4,4.3) -- cycle;
			\draw[line width=1.2pt,smooth,samples=100,domain=-1:4] plot(\x,{-0.5+2*(\x)});
			\fill[pattern=north east lines] (-1,0) -- (-1,4.3) -- (4,4.3) -- (4,0)-- cycle;
	\end{tikzpicture}}
	{\True \begin{tikzpicture}[scale=.7, font=\footnotesize, line join=round, line cap=round, >=stealth]
			\draw[->] (-1,0) -- (4.3,0)node[above]{$x$};
			\foreach \x in {1,2,3,4}
			\draw[shift={(\x,0)},color=black] (0pt,2pt) -- (0pt,-2pt) node[below] {\footnotesize $\x$};
			\draw[->,color=black] (0,-1) -- (0,4.3)node[right]{$y$};
			\foreach \y in {1,2,3,4}
			\draw[shift={(0,\y)},color=black] (2pt,0pt) -- (-2pt,0pt) node[left]{\footnotesize $\y$};
			\node[below left] at (0,0){$O$};
			\node[below] at (2,-0.5){$d$};
			\clip(-1,-1) rectangle (4,4.3);
			\fill[pattern=north east lines] (-0.15,4.3) -- (4,4.3) -- (4,-1) -- (2.5,-1)-- cycle;
			\draw[line width=1.2pt,smooth,samples=100,domain=-1:4] plot(\x,{4-2*(\x)});
			\fill[pattern=north east lines] (-1,0) -- (-1,4.3) -- (4,4.3) -- (4,0)-- cycle;
	\end{tikzpicture}}
	{\begin{tikzpicture}[scale=.7, font=\footnotesize, line join=round, line cap=round, >=stealth]
			\draw[->] (-1,0) -- (4.3,0)node[above]{$x$};
			\foreach \x in {1,2,3,4}
			\draw[shift={(\x,0)},color=black] (0pt,2pt) -- (0pt,-2pt) node[below] {\footnotesize $\x$};
			\draw[->,color=black] (0,-1) -- (0,4.3)node[right]{$y$};
			\foreach \y in {1,2,3,4}
			\draw[shift={(0,\y)},color=black] (2pt,0pt) -- (-2pt,0pt) node[left]{\footnotesize $\y$};
			\node[below left] at (0,0){$O$};
			\node[below left] at (1,2){$d$};
			\clip(-1,-1) rectangle (4,4.3);
			\fill[pattern=north east lines] (-0.15,4.3) -- (4,4.3) -- (4,-1) -- (2.5,-1)-- cycle;
			\draw[line width=1.2pt,smooth,samples=100,domain=-1:4] plot(\x,{4-2*(\x)});
			\fill[pattern=north east lines] (-1,0) -- (-1,-1) -- (4,-1) -- (4,0)-- cycle;
	\end{tikzpicture}}
	{\begin{tikzpicture}[scale=.7, font=\footnotesize, line join=round, line cap=round, >=stealth]
			\draw[->] (-1,0) -- (4.3,0)node[above]{$x$};
			\foreach \x in {1,2,3,4}
			\draw[shift={(\x,0)},color=black] (0pt,2pt) -- (0pt,-2pt) node[below] {\footnotesize $\x$};
			\draw[->,color=black] (0,-1) -- (0,4.3)node[right]{$y$};
			\foreach \y in {1,2,3,4}
			\draw[shift={(0,\y)},color=black] (2pt,0pt) -- (-2pt,0pt) node[left]{\footnotesize $\y$};
			\node[below left] at (0,0){$O$};
			\node[below left] at (1.2,1){$d$};
			\clip(-1,-1) rectangle (4.3,4.3);
			\fill[pattern=north east lines] (-0.25,-1) -- (-1,-1) -- (-1,4.3) -- (2.4,4.3) -- cycle;
			\draw[line width=1.2pt,smooth,samples=100,domain=-1:4] plot(\x,{-0.5+2*(\x)});
			\fill[pattern=north east lines] (-1,0) -- (-1,-1) -- (4,-1) -- (4,0)-- cycle;
	\end{tikzpicture}}
	\loigiai{Miền nghiệm của bất phương trình $y<0$ nằm bên dưới trục hoành $(1)$. \\
		Đường thẳng đi qua hai điểm $(0; 4)$ và $(2; 0)$ có phương trình là $2x + y =4$ (đây chính là đường thẳng $d$) $(2)$. \\
		Từ $(1)$ và $(2)$ suy ra hình B biểu diễn miền nghiệm của hệ bất phương trình đã cho. 
	}
\end{ex}

\begin{ex}%[0D4K4-4]
	\immini{Hệ bất phương trình nào sau đây có miền nghiệm là phần mặt phẳng không bị gạch có hai bờ là hai đường thẳng $a$ và $b$ như hình bên?
		\def\dotEX{}		
		\choice
		{\True $\heva{& 2x+y\le 2\\& 2x+y\ge -2.}$}
		{$\heva{& 2x+y\le -2\\& 2x+y\ge 2.}$}
		{$\heva{& 2x-y\le 2\\& 2x-y\ge -2.}$}
		{$\heva{& 2x-y\le -2\\& 2x-y\ge 2.}$}
	}
	{
		\begin{tikzpicture}[scale=0.6, font=\footnotesize, line join=round, line cap=round, >=stealth]
			\def\xmin{-3} \def\xmax{3}
			\def\ymin{-3} \def\ymax{3}
			\clip(\xmin,\ymin) rectangle (\xmax,\ymax);
			\tkzDefPoints{\xmax/\ymax/A1,\xmin/\ymax/A2,\xmin/\ymin/A3,\xmax/\ymin/A4}
			\tkzDefPoints{-2.5/3/M,.5/-3/N, -.5/3/P,2.5/-3/Q}
			\fill[pattern=north east lines,pattern color=blue!60] (M)--(A2)--(A3)--(N)--cycle (P)--(Q)--(A4)--(A1)--cycle;
			\draw (M)--(N) (P)--(Q);
			\begin{scriptsize}
				\draw[->](\xmin,0)--(\xmax,0); \draw(\xmax-0.1,0) node[below]{$x$};
				\draw[->](0,\ymin)--(0,\ymax); \draw(0,\ymax-0.2) node[right]{$y$};
				\foreach \x in {-1,1}
				\draw (\x,0.05) -- ++(0,-0.1) node [above] {$\x$};
				\foreach \x in {-2,2}
				\draw (0.05,\x) -- ++(-0.1,0) node [left] {$\x$};
				\draw node [above right]{$O$}
				(1,-2.5) node [below left]{$a$}
				(1.8,-2.5) node [below right]{$b$};
			\end{scriptsize}
		\end{tikzpicture}
	}
	\loigiai{Đường thẳng $a$ đi qua hai điểm $(-1; 0)$ và $(0; -2)$ nên có phương trình là $2x + y =-2$. \\
		Đường thẳng $b$ đi qua hai điểm $(1; 0)$ và $(0; 2)$ nên có phương trình là $2x + y =2$. \\
		Điểm $O(0; 0)$ thuộc miền nghiệm của bất phương trình $2x +y \leq 2$ và $2x + y \geq -2$. \\
		Vậy miền không bị gạch là miền nghiệm của hệ bất phương trình $\heva{& 2x+y\le 2\\& 2x+y\ge -2.}$ 
	}
\end{ex}

\begin{ex}%[0D4K4-4]
	\immini{Hệ bất phương trình nào sau đây có miền nghiệm là phần mặt phẳng không bị gạch như hình bên (kể cả các điểm nằm trên hai đường thẳng $a$, $b$ và không thuộc miền bị gạch)?
		\def\dotEX{}
		\choice
		{$\heva{& 2x+y\le 2\\& -2x+y\ge 2.}$}
		{$\heva{& 2x+y\ge 2\\& -2x+y\ge 2.}$}
		{$\heva{& 2x+y\ge 2\\& -2x+y\ge -2.}$}
		{\True $\heva{& 2x+y\le 2\\& -2x+y\le 2.}$}
	}
	{
		\begin{tikzpicture}[scale=0.6, font=\footnotesize, line join=round, line cap=round, >=stealth]
			\def\xmin{-3} \def\xmax{3}
			\def\ymin{-2} \def\ymax{4}
			\clip(\xmin,\ymin) rectangle (\xmax,\ymax);
			\tkzDefPoints{\xmax/\ymax/A1,\xmin/\ymax/A2,\xmin/\ymin/A3,\xmax/\ymin/A4}
			\tkzDefPoints{-2.5/-3/M,2.5/-3/N,0/2/I}
			\fill[pattern=north east lines,pattern color=blue!60]   (M)--(A3)--(A2)--(A1)--(A4)--(N)--(I)--cycle;
			\draw[domain=-3:3] plot(\x,{2-2*(\x)}) plot(\x,{2+2*(\x)});
			\begin{scriptsize}
				\draw[->](\xmin,0)--(\xmax,0); \draw(\xmax-0.1,0) node[below]{$x$};
				\draw[->](0,\ymin)--(0,\ymax); \draw(0,\ymax-0.2) node[right]{$y$};
				\foreach \x in {-1,1}
				\draw (\x,0.05) -- ++(0,-0.1) node [below] {$\x$};
				\foreach \x in {-1,2}
				\draw (0.05,\x) -- ++(-0.1,0) node [left] {$\x$};
				\draw node [above right]{$O$}
				(-1.6,-1.7) node {$a$}
				(1.6,-1.7) node {$b$};
			\end{scriptsize}
		\end{tikzpicture}
	}
	\loigiai{Đường thẳng $a$ đi qua hai điểm $(-1; 0)$ và $(0; 2)$ nên có phương trình là $-2x + y =2$. \\
		Đường thẳng $b$ đi qua hai điểm $(1; 0)$ và $(0; 2)$ nên có phương trình là $2x + y =2$. \\
		Điểm $O(0; 0)$ thuộc miền nghiệm của bất phương trình $-2x +y \leq 2$ và $2x + y \leq 2$. \\
		Vậy miền không bị gạch là miền nghiệm của hệ bất phương trình $\heva{& 2x+y\le 2\\& -2x+y\le 2.}$}
\end{ex}

\begin{ex}%[0D4K4-4]
	\immini{Hệ bất phương trình nào sau đây có miền nghiệm là phần mặt phẳng không bị gạch như hình bên (kể cả các điểm nằm trên các đường thẳng $a$, $b$, $c$ và không thuộc miền bị gạch)?
		\def\dotEX{}
		\choice
		{$\heva{& 2x+y\ge 1\\& x-2y\ge 2\\& x\le 0.}$}
		{$\heva{& 2x-y\le 1\\& x-2y\ge 2\\& y\le 0.}$}
		{\True $\heva{& 2x+y\le 1\\& x-2y\ge 2\\& x\ge 0.}$}
		{$\heva{& 2x-y\le 1\\& x-2y\ge 2\\& x\ge 0.}$}
	}
	{
		\begin{tikzpicture}[scale=0.8, font=\footnotesize, line join=round, line cap=round, >=stealth]
			\def\xmin{-2} \def\xmax{3}
			\def\ymin{-3} \def\ymax{2}
			\clip(\xmin,\ymin) rectangle (\xmax,\ymax);
			\tkzDefPoints{\xmax/\ymax/A1,\xmin/\ymax/A2,\xmin/\ymin/A3,\xmax/\ymin/A4}
			\tkzDefPoints{0/-3/M,0/-1/N,2/-3/P}
			\tkzDefPoint(4/5,-3/5){I}
			\fill[pattern=north east lines,pattern color=blue!60]   (M)--(N)--(I)--(P)--(A4)--(A1)--(A2)--(A3)--cycle;
			\draw[domain=-3:3] plot(\x,{1-2*(\x)}) plot(\x,{(\x)/2-1});		
			\begin{scriptsize}
				\draw[->](\xmin,0)--(\xmax,0); \draw(\xmax-0.1,0) node[below]{$x$};
				\draw[->](0,\ymin)--(0,\ymax); \draw(0,\ymax-0.2) node[right]{$y$};
				\foreach \x in {-1,1,2}
				\draw (\x,0.05) -- ++(0,-0.1) node [below] {$\x$};
				\foreach \x in {-2,-1,1}
				\draw (0.05,\x) -- ++(-0.1,0) node [left] {$\x$};
				\draw node [below left]{$O$}
				(0.2,-2.5) node {$a$}
				(1.5,-2.5) node {$b$}
				(0.5,-1) node {$c$};
			\end{scriptsize}
		\end{tikzpicture}
	}
	\loigiai{Đường thẳng $a$ có phương trình $x=0$. \\
		Đường thẳng $b$ đi qua hai điểm $\left(\dfrac{1}{2}; 0 \right)$ và $(0; 1)$ nên có phương trình là $2x + y =1$. \\
		Đường thẳng $c$ đi qua hai điểm $\left(2; 0 \right)$ và $(0; -1)$ nên có phương trình là $x - 2y =2$. \\
		Điểm $(0; -2)$ thuộc miền nghiệm của bất phương trình $x \geq 0$, $2x +y \leq 1$ và $x - 2y \geq 2$. \\
		Vậy miền không bị gạch là miền nghiệm của hệ bất phương trình $\heva{& 2x+y\le 1\\ & x-2y\ge 2\\ & x\ge 0.}$
	}
\end{ex}

\begin{ex}%[0D4K4-4]
	Tìm tất cả các số thực $a$, $b$ sao cho miền nghiệm của hệ bất phương trình $\heva{& x\ge a\\& y<b}$ chứa điểm $M(-1;1)$.
	\choice
	{$a\ge -1$; $b\le 1$}
	{$a<-1$; $b\ge 1$}
	{\True $a\le -1$; $b>1$}
	{$a\le -1$; $b<1$}
	\loigiai{Để $M(-1;1)$ thuộc miền nghiệm của bất phương trình $\heva{& x \ge a \\& y<b}$ thì $a \leq -1$ và $b > 1$. 
	}
\end{ex}

\begin{ex}%[0D4K4-4]
	Tìm tất cả các giá trị của $m$ để đường thẳng $y=m$ có điểm chung với miền nghiệm của hệ bất phương trình $\heva{&x\geq -2\\&y\geq -2\\& x+y\leq 2.}$ 
	\choice
	{$m\geq -2$}
	{$m\leq 4$}
	{\True $-2\leq m\leq 4$}
	{ $-2<m<4$}
	\loigiai{Miền nghiệm của hệ bất phương trình đã cho là miền tam giác $ABC$ và các cạnh (như hình vẽ).\\
		\begin{center}
			\begin{tikzpicture}[scale=1, font=\footnotesize, line join=round, line cap=round, >=stealth]
				\fill[pattern=north east lines] (-5,-4) -- (-5,5) -- (5,5) -- (5,-4) -- cycle;
				\fill[white] (-2,-2) -- (-2,4) -- (4,-2) -- cycle;
				\draw[violet,line width=2pt,samples=100] (-2,-2) -- (-2,4) -- (4,-2)--(-2,-2);
				\draw[->] (-5.2,0) -- (5.3,0)node[above]{$x$};
				\foreach \x in {-2,2,4}
				\draw[shift={(\x,0)},color=black] (0pt,2pt) -- (0pt,-2pt) node[below left] {\footnotesize $\x$};
				\draw[->,color=black] (0,-4.2) -- (0,5.3)node[right]{$y$};
				\foreach \y in {-2,2,4}
				\draw[shift={(0,\y)},color=black] (2pt,0pt) -- (-2pt,0pt) node[above right] {\footnotesize $\y$};
				\draw[color=red] (-5,1) -- (5,1); %node[above ]{$y=m$};
				\node[above right] at (-2, 1){$y=m$};
				\node[above left] at (0,0){$O$};
				\node[below left] at (-2,4){$A$};
				\node[above right] at (4,-2){$B$};
				\node[below left] at (-2,-2){$C$};
				\clip(-5,-4) rectangle (5,5);
				%	\draw[dashed] (4,0)--(4,-2);
				\draw[dashed] (-2,4)--(0,4);
				\draw[line width=1.2pt,smooth,samples=100,domain=-3:4] (-2,-4)--(-2,5);
				\draw[line width=1.2pt,smooth,samples=100,domain=-5:5] plot(\x,{-2-0*(\x)});
				\draw[line width=1.2pt,smooth,samples=100,domain=-3:5] plot(\x,{2-(\x)});
			\end{tikzpicture}
		\end{center}
		Ba đỉnh của tam giác là $A(-2;4)$, $B(4;-2)$ và $C(-2;-2)$.\\ 
		Ta thấy điểm thấp nhất và cao nhất của miền nghiệm lần lượt có tung độ là $y=-2$ và $y=4$. Mặt khác $y=m$ là đường thẳng song song với trục $Ox$.\\
		Vậy để đường thẳng $y=m$ có điểm chung với miền nghiệm thì $-2\leq m\leq 4$.}
\end{ex}

\begin{ex}%[0D4K4-4]
	Cho hệ bất phương trình $\heva{&(a-2)x+(a-4)y\geq 2 \\ &(a+1)x+(3a+2)y\leq -1}$ với $a\in \mathbb{R}$, $a\neq 0$ và $a\neq \dfrac{1}{2}$. Điểm nào sau đây luôn thuộc miền nghiệm của hệ bất phương trình đã cho?
	\choice
	{$M\left (\dfrac{-3}{2a-1};\dfrac{7}{2a-1}\right )$}
	{$N\left (\dfrac{-7}{2a-1};\dfrac{-3}{2a-1}\right )$}
	{\True $P\left (\dfrac{7}{2a-1};\dfrac{-3}{2a-1}\right )$}
	{$P\left (\dfrac{7}{2a-1};\dfrac{3}{2a-1}\right )$}
	\loigiai{
		Dễ dàng nhận thấy rằng	nếu có một điểm luôn thuộc miền nghiệm của bất phương trình đã cho thì điểm đó phải là nghiệm của hệ $\heva{&(a-2)x+(a-4)y= 2 \\ &(a+1)x+(3a+2)y= -1} \Leftrightarrow \heva{&x=\dfrac{7}{2a-1} \\ & y=\dfrac{-3}{2a-1}.}$
	}
\end{ex}

\begin{ex}%[0D4K4-4]
	Miền nghiệm của hệ bất phương trình $\heva{&\sqrt{2}x+\sqrt{3}y-1\leq 0 \\ &\sqrt{3}x-\sqrt{2}y+1\geq 0 \\ &y\geq -4}$ là 
	\choice
	{\True tam giác vuông kể cả các điểm nằm trên ba cạnh của tam giác}
	{tam giác đều kể cả các điểm nằm trên ba cạnh của tam giác}
	{tam giác tù kể cả các điểm nằm trên ba cạnh của tam giác}
	{tam giác cân (không vuông) kể cả các điểm nằm trên ba cạnh của tam giác}
	\loigiai{
		\begin{center}
			\begin{tikzpicture}[scale=1, font=\footnotesize, line join=round, line cap=round, >=stealth]
				\clip(-5,-5) rectangle (7,1.5);
				\fill[pattern=north east lines,pattern color=blue!30] (-5,-5) rectangle (7,1.5);
				\fill[white] (-0.0636,0.6293)--(5.606,-4)--(-3.8433,-4)--cycle;
				\draw[->](-5,0)--(7,0); \draw(7-0.1,0) node[below]{$x$};
				\draw[->](0,-5)--(0,1.5); \draw(0,1.5-0.2) node[right]{$y$};
				\foreach \x in {-4,-3,-2,-1,1,2,3,4,5,6} {
					\draw (\x,0.05) -- ++(0,-0.1) node [below] {$\x$};
					\draw (0.05,\x) -- ++(-0.1,0) node [left] {$\x$}; }
				\draw node [below left]{$O$};
				\draw [domain=-4:7] plot(\x,{(-0.8165*\x)+0.5774});
				\draw plot(\x,{(1.2247*\x)+0.7071});
				\draw [domain=-5:7] plot(\x,{-4});
			\end{tikzpicture}	
		\end{center}
		Hai đường thẳng $\sqrt{2}x+\sqrt{3}y-1= 0$ và $\sqrt{3}x-\sqrt{2}y+1=0$ vuông góc với nhau nên miền nghiệm là tam giác vuông, kể cả các điểm nằm trên ba cạnh của tam giác.
	}
\end{ex}

\begin{ex}%[0D4G4-4]
	Miền nghiệm của bất phương trình $\vert x+y \vert +\vert x-y \vert \leq 4$ là
	\choice
	{một hình vuông (không kể biên)}
	{một hình chữ nhật (không phải là hình vuông và không kể biên)}
	{một hình chữ nhật (không phải là hình vuông và kể cả biên)}
	{\True một hình vuông (kể cả biên)}
	\loigiai{
		Để phá dấu giá trị tuyệt đối, ta xét dấu của $x+y$ và $x-y$, có $4$ trường hợp sau đây
		\immini{$(1)$ $\heva{&x+y\geq 0 \\&x-y\geq 0 \\&2x\leq 4 }$; $(2)$ $\heva{&x+y> 0 \\&x-y< 0 \\&2y\leq 4 }$; \\
			$(3)$ $\heva{&x+y< 0 \\&x-y> 0 \\&-2y\leq 4 }$ và $(4)$ $\heva{&x+y< 0 \\&x-y< 0 \\&-2x\leq 4 .}$ 
		}{
			\begin{tikzpicture}[scale=0.8, font=\footnotesize, line join=round, line cap=round, >=stealth]
				\clip(-3,-3) rectangle (3,3);
				\fill[pattern=north east lines,pattern color=blue!30] (-3,-3) rectangle (3,3);
				\fill[white] (-2,-2)--(-2,2)--(2,2)--(2,-2)--cycle;
				\draw[->](-3,0)--(3,0); \draw(3-0.1,0) node[below]{$x$};
				\draw[->](0,-3)--(0,3); \draw(0,3-0.2) node[right]{$y$};
				\begin{scriptsize}
					\foreach \x in {-2,-1,1,2} {
						\draw (\x,0.05) -- ++(0,-0.1) node [below] {$\x$};
						\draw (0.05,\x) -- ++(-0.1,0) node [left] {$\x$}; }
					\draw node [below left]{$O$};
					\draw (2,2) node [below right] {$A$};
					\draw (2,-2) node [above right] {$B$};
					\draw (-2,-2) node [below right] {$C$};
					\draw (-2,2) node [above right] {$D$};
				\end{scriptsize}
				
				\draw plot(\x,{\x)});
				\draw plot(\x,{-\x)});
				\draw plot(\x,{2});
				\draw plot(\x,{-2});
				\draw (-2,3)--(-2,-3)(2,3)--(2,-3);
			\end{tikzpicture}	
		}	
		
		Giải bốn hệ bất phương trình này rồi kết hợp lại ta được miền nghiệm của bất phương trình đã cho là hình vuông $ABCD$ với $A(2;2)$, $B(2;-2)$; $C(-2;-2)$; $D(-2;2)$ kể cả biên.
	}
\end{ex}
\begin{ex}%[0D4K4-2]
	Tìm giá trị lớn nhất $M$ của biểu thức $z=3x+2y$ biết rằng $x$, $y$ thỏa mãn hệ bất phương trình $\heva{&x\ge 0, \ y\ge 0 \\& x+2y\le 4 \\& x-y\le 1.}$
	\choice
	{\True $M=8$}
	{$M=10$}
	{$M=6$}
	{$M=9$}
	\loigiai{
		\immini{
			Miền nghiệm là tứ giác như hình vẽ.\\
			$z$ lớn nhất là $8$ tại đỉnh $(2;1)$.
		}
		{
			\begin{tikzpicture}[scale=0.7, font=\footnotesize, line join=round, line cap=round, >=stealth]
				\def\xmin{-1} \def\xmax{5}
				\def\ymin{-1.5} \def\ymax{3}
				\clip(\xmin,\ymin) rectangle (\xmax,\ymax);
				\tkzDefPoints{0/0/O,1/0/M,2/1/N,0/2/P}
				\tkzDrawPoints(O,M,N,P)
				\fill[pattern=north east lines,pattern color=blue!60] (O)--(M)--(N)--(P)--cycle;
				\draw[domain=-1:5] plot(\x,{2-(\x)/2}) plot(\x,{(\x)-1});
				\begin{scriptsize}
					\draw[->](\xmin,0)--(\xmax,0); \draw(\xmax-0.1,0) node[below]{$x$};
					\draw[->](0,\ymin)--(0,\ymax); \draw(0,\ymax-0.2) node[right]{$y$};
					\foreach \x in {1,2,4}
					\draw (\x,0.05) -- ++(0,-0.1) node [below] {$\x$};
					\foreach \x in {1,2}
					\draw (0.05,\x) -- ++(-0.1,0) node [left] {$\x$};
					\draw node [below left]{$O$};
				\end{scriptsize}
			\end{tikzpicture}
		}	
	}
\end{ex}

\begin{ex}%[0D4K4-2]
	Tìm giá trị lớn nhất của biểu thức $F(x;y)=x-y-1$ với $x$, $y$ thỏa mãn hệ $\heva{&x-2y+2\geq0\\&3x+8y-24\leq0\\&x\geq0,\ y\geq0.}$
	\choice
	{$ 5 $}
	{$ 6 $}
	{\True $ 7 $}
	{$ 8 $}
	\loigiai{
		\immini{
			Dễ thấy rằng: miền nghiệm của hệ đã cho là hình tứ giác OABC trên hình vẽ (Kể cả biên), trong đó các đỉnh của tứ giác có tọa độ: $O (0;0)$, $A (0;1)$, $B \left(\dfrac{16}{7};\dfrac{15}{7} \right)$, $C (8;0)$. 
		}
		{\begin{tikzpicture}[scale=1, font=\footnotesize, line join=round, line cap=round, >=stealth]
				\draw[->] (-3,0) -- (8.3,0)node[above]{$x$};
				\foreach \x in {-3,-2,-1,1,2,3,4,5,6,7,8}
				\draw[shift={(\x,0)},color=black] (0pt,2pt) -- (0pt,-2pt) node[below] {\footnotesize $\x$};
				\draw[->,color=black] (0,-1) -- (0,4.3)node[right]{$y$};
				\foreach \y in {-1,1,2,3,4}
				\draw[shift={(0,\y)},color=black] (2pt,0pt) -- (-2pt,0pt) node[left] {\footnotesize $\y$};
				\clip(-3,-1) rectangle (8.3,4);
				\fill[pattern=north east lines] (-2,3.75) -- (-3,4) -- (8,4) -- (8,0)-- cycle;
				\draw[line width=1.2pt,smooth,samples=100,domain=-3:9] plot(\x,{3-0.375*(\x)});
				\fill[pattern=north east lines] (-3,4)--(-3,-0.5)--(6,4)--(8,4)-- cycle;
				\draw[line width=1.2pt,smooth,samples=100,domain=-3:9] plot(\x,{1+0.5*(\x)});
				\fill[pattern=north east lines] (0,4)--(-3,4) --(-3,-3)--(0,-3)-- cycle;
				\fill[pattern=north east lines] (-3,0)--(-3,-3) --(8,-3)--(8,0)-- cycle;
				\color{red}{\node[below] at (2.285,2) {$B$};
					\node[below right] at (0,1) {$A$};
					\node[below left] at (8,0) {$C$};
					\node[above right] at (0,0) {$O$};
				}
			\end{tikzpicture}
		}
		\noindent Ta biết rằng giá trị lớn nhất của biểu thức $F(x;y)$ sẽ đạt được tại các đỉnh của tứ giác, do đó ta tính giá trị của $F(x;y)$ tại các đỉnh này.
		$F(0;0)=-1$, $F(0;1)=-2$, $F\left(\dfrac{16}{7};\dfrac{15}{7} \right)=-\dfrac{6}{7}$, $F(8;0)=7$. \\
		Vậy giá trị lớn nhất của biểu thức thỏa mãn hệ là $F(8;0)=7$.
	}
\end{ex}

\begin{ex}%[0D4K4-2]
	Tìm giá trị lớn nhất $a$ và giá trị nhỏ nhất $b$ của $F(x;y)=3x+9y$ với $(x;y)$ là nghiệm của hệ bất phương trình $\heva{&x-y+1 \leq 0\\&2x-y+4 \geq 0\\&x+y+1 \geq 0\\&2x+y-4 \leq 0.}$
	\choice
	{$a=21,b=1$}
	{$a=21,b=-3$}
	{$a=36,b=1$}
	{\True $a=36,b=-3$}
	\loigiai{ Ta đã biết giá trị lớn nhất và giá trị nhỏ nhất của biểu thức $F(x;y)$ đạt được tại các điểm $(1;2)$, $(0;4)$, $(-1;0)$, $\left(\dfrac{-5}{3};\dfrac{2}{3} \right)$ theo trên hình vẽ minh họa. Thử lại ta thấy giá trị lớn nhất $a=36$ tại $(x;y)=(0;4)$, giá trị nhỏ nhất $b=-3$ tại $(x;y)=(-1;0)$.\\
		\begin{center}
			\begin{tikzpicture}[scale=1, font=\footnotesize, line join=round, line cap=round, >=stealth]
				\draw[->] (-3.5,0) -- (3.3,0)node[above]{$x$};
				\foreach \x in {-3,-2,-1,1,2,3}
				\draw[shift={(\x,0)},color=black] (0pt,2pt) -- (0pt,-2pt) node[below] {\footnotesize $\x$};
				\draw[->,color=black] (0,-1.3) -- (0,4.3)node[right]{$y$};
				\foreach \y in {-1,1,2,3,4}
				\draw[shift={(0,\y)},color=black] (2pt,0pt) -- (-2pt,0pt) node[left] {\footnotesize $\y$};
				\clip(-3,-1) rectangle (3.3,4.3);
				\fill[pattern=north east lines] (-2.5,-1) -- (-3.3,-1) -- (-3.3,4.3) -- (0.15,4.3)-- cycle;
				\draw[line width=1.2pt,smooth,samples=100,domain=-3:9] plot(\x,{1+(\x)});
				\fill[pattern=north east lines] (-0.15,4.3)--(3.3,4.3)--(3.3,-1)--(2.5,-1)-- cycle;
				\draw[line width=1.2pt,smooth,samples=100,domain=-3:9] plot(\x,{4+2*(\x)});
				\fill[pattern=north east lines] (3.3,4.3)--(3.3,-1) --(-2,-1)-- cycle;
				\draw[line width=1.2pt,smooth,samples=100,domain=-3:9] plot(\x,{-1-(\x)});
				\fill[pattern=north east lines] (0,-1)--(-3,-1) --(-3,2)-- cycle;
				\draw[line width=1.2pt,smooth,samples=100,domain=-3:9] plot(\x,{4-2*(\x)});
			\end{tikzpicture}
		\end{center}
		\begin{center}
			Hình vẽ minh họa
		\end{center}
	}
\end{ex}

\begin{ex}%[0D4K4-2]
	Cho hệ bất phương trình $\heva{&0\leq x\leq 5 \\ &0\leq y \leq 10 \\ &5x+3y\geq 15 \\ &-x+y\geq 2}$ và biểu thức $P(x;y)=2x-2y+3$ với $(x;y)$ thuộc miền nghiệm của hệ bất phương trình đã cho. Tìm giá trị nhỏ nhất của $P$.
	\choice
	{\True $-17$}
	{$-34$}
	{$-7$}
	{$-14$}
	\loigiai{
		\immini{
			Miền nghiệm của hệ bất phương trình cho ở giả thiết bài toán được biểu diễn như hình trên, với miền nghiệm là hình ngũ giác màu trắng, kể cả biên. $P$ chỉ có thể đạt giá trị nhỏ nhất tại các đỉnh của ngũ giác, các đỉnh đó có tọa độ lần lượt là $A(0;10)$, $B(5;10)$, $C(5;7)$, $D\left (\dfrac{9}{8}; \dfrac{25}{8}\right )$, $E(0;5)$. Thay tọa độ các đỉnh vào $P$ ta tìm được giá trị nhỏ nhất của $P$ bằng $-17$ tại $x=0$, $y=10$.
		}{
			\begin{tikzpicture}[scale=0.6, font=\footnotesize, line join=round, line cap=round, >=stealth]
				\clip(-3,-1) rectangle (6,11);
				\def\xmin{-3} \def\xmax{6}
				\def\ymin{-1} \def\ymax{11}
				\tkzDefPoints{\xmax/\ymax/A,\xmin/\ymax/B,\xmin/\ymin/C,\xmax/\ymin/D}
				%Nhập hệ số a,b,c đt d1: x+y-1=0-----------Đường thẳng MN
				\def\hsa{5} \def\hsb{3} \def\hsc{-15}
				\ifnum\hsb=0{
					\tkzDefPoint(-\hsc/ \hsa,\ymin){M};
					\tkzDefPoint(-\hsc/ \hsa,\ymax){N};}
				\else{ 
					\tkzDefPoint(\xmin,-\hsa*\xmin/\hsb-\hsc/\hsb){M};
					\tkzDefPoint(\xmax,-\hsa*\xmax/\hsb-\hsc/\hsb){N};}\fi
				%Nhập hệ số a,b,c đt d2: 2x-3y-1=0---------Đường thẳng PQ
				\def\hsa{-1} \def\hsb{1} \def\hsc{-2}
				\ifnum\hsb=0{
					\tkzDefPoint(-\hsc/ \hsa,\ymin){P};
					\tkzDefPoint(-\hsc/ \hsa,\ymax){Q};}
				\else{ 
					\tkzDefPoint(\xmin,-\hsa*\xmin/\hsb-\hsc/\hsb){P};
					\tkzDefPoint(\xmax,-\hsa*\xmax/\hsb-\hsc/\hsb){Q};}\fi
				%Nhập hệ số a,b,c đt d1: x+y-1=0-----------Đường thẳng RS
				\def\hsa{1} \def\hsb{0} \def\hsc{-5}
				\ifnum\hsb=0{
					\tkzDefPoint(-\hsc/ \hsa,\ymin){R};
					\tkzDefPoint(-\hsc/ \hsa,\ymax){S};}
				\else{ 
					\tkzDefPoint(\xmin,-\hsa*\xmin/\hsb-\hsc/\hsb){R};
					\tkzDefPoint(\xmax,-\hsa*\xmax/\hsb-\hsc/\hsb){S};}\fi
				%Nhập hệ số a,b,c đt d1: x+y-1=0-----------Đường thẳng UV
				\def\hsa{0} \def\hsb{1} \def\hsc{-10}
				\ifnum\hsb=0{
					\tkzDefPoint(-\hsc/ \hsa,\ymin){U};
					\tkzDefPoint(-\hsc/ \hsa,\ymax){V};}
				\else{ 
					\tkzDefPoint(\xmin,-\hsa*\xmin/\hsb-\hsc/\hsb){U};
					\tkzDefPoint(\xmax,-\hsa*\xmax/\hsb-\hsc/\hsb){V};}\fi
				\tkzInterLL(M,N)(P,Q) \tkzGetPoint{I}
				\tkzInterLL(P,Q)(R,S) \tkzGetPoint{K}
				\fill[pattern=north east lines,pattern color=blue!30] (C) rectangle (A);
				\fill[white] (I)--(K)--(5,10)--(0,10)--(0,5)--cycle;
				\tkzDrawSegment(M,N)
				\tkzDrawSegment(P,Q)
				\tkzDrawSegment(R,S)
				\tkzDrawSegment(U,V)
				\draw[->](\xmin,0)--(\xmax,0); \draw(\xmax-0.1,0) node[below]{$x$};
				\draw[->](0,\ymin)--(0,\ymax); \draw(0,\ymax-0.2) node[right]{$y$};
				\begin{scriptsize}
					\foreach \x in {-3,-2,-1,1,2,3,4,5,6,7,8,9,10} {
						\draw (\x,0.05) -- ++(0,-0.1) node [below] {$\x$};
						\draw (0.05,\x) -- ++(-0.1,0) node [left] {$\x$}; }
					\draw node [below left]{$O$};
				\end{scriptsize}
				
			\end{tikzpicture}
		}	
	}
\end{ex}

\begin{ex}%[0D4K4-2]
	Tìm giá trị nhỏ nhất của biểu thức $F=y-x$ trên miền xác định bởi hệ $\heva{y-2x&\leq 2\\2y-x&\geq 4\\x+y&\leq 5.}$
	\choice
	{\True $\min F=1$ khi $x=2$, $y=3$}
	{$\min F=2$ khi $x=0$, $y=2$}
	{$\min F=3$ khi $x=1$, $y=4$}
	{$\min F=-1$ khi $x=2$, $y=1$}
	\loigiai{Biểu diễn miền nghiệm của bất phương trình đã cho ta được miền nghiệm là tam giác $ABC$ với tọa độ các đỉnh là: $A(0;2)$, $B(2;3)$, $C(1;4)$.
		\begin{center}
			\begin{tikzpicture}[scale=1, font=\footnotesize, line join=round, line cap=round, >=stealth]
				\fill[pattern=north east lines] (-5,-2) -- (-5,6) -- (6,6) -- (6,-2) -- cycle;
				\fill[white] (0,2)--(1,4) -- (2,3) -- cycle;
				\draw[->] (-5.2,0) -- (6.3,0)node[above]{$x$};
				\foreach \x in {-4,-1,4,5}
				\draw[shift={(\x,0)},color=black] (0pt,2pt) -- (0pt,-2pt) node[below] {\footnotesize $\x$};
				\draw[->,color=black] (0,-2.1) -- (0,6.3)node[right]{$y$};
				\foreach \y in {2,5}
				\draw[shift={(0,\y)},color=black] (2pt,0pt) -- (-2pt,0pt) node[above right] {\footnotesize $\y$};
				\node[above right] at (0,0){$O$};
				\node[below right] at (0,2){$A$};
				\node[below] at (2,3){$B$};
				\node[left] at (1,4){$C$};
				\clip(-5,-2) rectangle (6,6);
				\draw[line width=1.2pt,smooth,samples=100,domain=-2:6] plot(\x,{2+2*(\x)});
				\draw[line width=1.2pt,smooth,samples=100,domain=-5:6] plot(\x,{0.5*(\x)+2});
				\draw[line width=1.2pt,smooth,samples=100,domain=-2:6] plot(\x,{5-(\x)});
			\end{tikzpicture}
		\end{center}
		Tính giá trị của biểu thức $F=y-x$ tại tọa độ các đỉnh ta có:\\
		Tại $A(0;2)$: $F=y-x=2$.\\
		Tại $B(2;3)$: $F=y-x=1$.\\
		Tại $C(1;4)$: $F=y-x=3$.\\
		Vậy $\min F=1$ khi $x=2$, $y=3$.
	}
\end{ex}

\begin{ex}%[0D4K4-2]
	Tìm giá trị nhỏ nhất $T$ của biểu thức $z=5x+7y$ biết rằng $x$, $y$ là các số không âm thỏa mãn hệ bất phương trình $\heva{&2x+3y\ge 6 \\ & 3x-y\le 15 \\& -x+y\le 4 \\& 2x+5y\le 27.}$
	\choice
	{$T=12$}
	{\True $T=14$}
	{$T=28$}
	{$T=18$}
	\loigiai{
		\immini{
			Miền nghiệm là miền lục giác có tọa độ các đỉnh lần lượt là:\\ $(0;2)$, $(0;4)$, $(1;5)$, $(6;3)$, $(5;0)$, $(3;0)$.\\
			Giá trị nhỏ nhất $T=14$ đạt tại đỉnh $(0;2)$.
		}
		{
			\begin{tikzpicture}[scale=0.5, font=\footnotesize, line join=round, line cap=round, >=stealth]
				\def\xmin{-2} \def\xmax{7}
				\def\ymin{-1.5} \def\ymax{6}
				\clip(\xmin,\ymin) rectangle (\xmax,\ymax);
				\tkzDefPoints{\xmax/\ymax/A1,\xmin/\ymax/A2,\xmin/\ymin/A3,\xmax/\ymin/A4}
				\fill[pattern=north east lines,pattern color=blue!60] (A1)--(A2)--(A3)--(A4)--cycle;
				\tkzDefPoints{0/2/A,0/4/B,1/5/C,6/3/D,5/0/E,3/0/F}
				\tkzDrawPoints(A,B,C,D,E,F)
				\fill[color=white] (A)--(B)--(C)--(D)--(E)--(F)--cycle;
				\draw[domain=-2:7] plot(\x,{2-2*(\x)/3}) plot(\x,{3*(\x)-15}) plot(\x,{(\x)+4}) plot(\x,{27/5-2*(\x)/5});
				\begin{scriptsize}
					\draw[->](\xmin,0)--(\xmax,0); \draw(\xmax-0.1,0) node[below]{$x$};
					\draw[->](0,\ymin)--(0,\ymax); \draw(0,\ymax-0.2) node[right]{$y$};
					\foreach \x in {2,4,6}
					\draw (\x,0.05) -- ++(0,-0.1) node [below] {$\x$};
					\foreach \x in {2,4}
					\draw (0.05,\x) -- ++(-0.1,0) node [left] {$\x$};
					\draw node [below left]{$O$};
				\end{scriptsize}
			\end{tikzpicture}
		}	
	}
\end{ex}

\begin{ex}%[0D4K4-2]
	Tìm các cặp số $(x;y)$ thỏa mãn hệ bất phương trình $\heva{&0\le y\leq 2\\&y\leq x\\&x+y\leq 5\\&x\leq 4}$ sao cho biểu thức $S=2x+y$ đạt giá trị lớn nhất.
	\choice
	{$(x;y)=(4;0)$}
	{\True $(x;y)=(4;1)$}
	{$(x;y)=(3;2)$}
	{$(x;y)=(2;2)$}
	\loigiai{Miền nghiệm của hệ bất phương trình đã cho là ngũ giác $OABCD$ (bao gồm các điểm trên các cạnh)
		\begin{center}
			\begin{tikzpicture}[scale=1, font=\footnotesize, line join=round, line cap=round, >=stealth]
				\fill[pattern=north east lines] (-2,-2) -- (-2,6) -- (6,6) -- (6,-2) -- cycle;
				\fill[white] (0,0) -- (2,2) -- (3,2) --(4,1)--(4,0)-- cycle;
				\draw[->] (-2.1,0) -- (6.3,0)node[above]{$x$};
				\foreach \x in {2,3,4,5}
				\draw[shift={(\x,0)},color=black] (0pt,2pt) -- (0pt,-2pt) node[below left] {\footnotesize $\x$};
				\draw[->,color=black] (0,-2.1) -- (0,6.3)node[right]{$y$};
				\foreach \y in {2,5}
				\draw[shift={(0,\y)},color=black] (2pt,0pt) -- (-2pt,0pt) node[above right] {\footnotesize $\y$};
				\node[above left] at (0,0){$O$};
				\node[below] at (2,2){$A$};
				\node[below] at (3,2){$B$};
				\node[above right] at (4,1){$C$};
				\node[above left] at (4,0){$D$};
				\clip(-2,-2) rectangle (6,6);
				\draw[line width=1.2pt,smooth,samples=100,domain=-2:6] plot(\x,{2-0*(\x)});
				\draw[line width=1.2pt,smooth,samples=100,domain=-2:6] plot(\x,{(\x)});
				\draw[line width=1.2pt,smooth,samples=100,domain=-2:6] plot(\x,{5-(\x)});
				\draw[line width=1.2pt,smooth,samples=100,domain=-2:6] (4,-2)--(4,6);
			\end{tikzpicture}
		\end{center}	
		Tọa độ các đỉnh là: $O(0;0)$, $A(2;2)$, $B(3;2)$, $C(4;1)$, $D(4;0)$.\\
		Lần lượt tính giá trị của biểu thức tại các cặp số là tọa độ các đỉnh, suy ra biểu thức $S=2x+y$ đạt giá trị lớn nhất với cặp số $(4;1)$ ứng với tọa độ đỉnh $C$.
	}
\end{ex}

\begin{ex}%[0D4K4-3]
	Khẩu phần dinh dưỡng hàng ngày cho người ăn kiêng cần cung cấp ít nhất $300$ calo, $36$ đơn vị vitamin $A$ và $90$ đơn vị vitamin $C$. Một tách thức uống $X$ có giá $5$ nghìn đồng và cung cấp $60$ calo, $12$ đơn vị vitamin $A$ và $10$ đơn vị vitamin $C$. Một tách thức uống $Y$ có giá $6$ nghìn đồng và cung cấp $60$ calo, $6$ đơn vị vitamin $A$ và $30$ đơn vị vitamin $C$. Mỗi ngày nên uống bao nhiêu tách mỗi loại để có được chi phí tối ưu và vẫn đáp ứng được yêu cầu dinh dưỡng hàng ngày?
	\choice
	{$1$ tách loại $X$, $4$ tách loại $Y$}
	{\True $3$ tách loại $X$, $2$ tách loại $Y$}
	{$2$ tách loại $X$, $3$ tách loại $Y$}
	{$4$ tách loại $X$, $1$ tách loại $Y$}
	\loigiai{
		\immini{
			Ta có hệ: $\heva{& 60x+60y\ge 300 \\ & 12x+6y\ge 36 \\ & 10x+30y \ge 90 \\& x\ge 0 \\ & y\ge 0}$\\
			Giá $C=5x+6y$.\\
			Miền nghiệm như hình vẽ. Các đỉnh là:\\
			$M(0;6)$, $N(1;4)$, $P(3;2)$, $Q(9;0)$.\\
			$C$ nhỏ nhất tại đỉnh $P(3;2)$.\\
			Vậy nên uống $3$ tách loại $X$ và $2$ tách loại $Y$.
		}
		{
			\begin{tikzpicture}[scale=0.5, font=\footnotesize, line join=round, line cap=round, >=stealth]
				\def\xmin{-2} \def\xmax{12}
				\def\ymin{-2} \def\ymax{8}
				\clip(\xmin,\ymin) rectangle (\xmax,\ymax);
				\tkzDefPoints{\xmax/\ymax/A1,\xmin/\ymax/A2,\xmin/\ymin/A3,\xmax/\ymin/A4}
				\tkzDefPoints{0/6/M,1/4/N,3/2/P,9/0/Q}
				\tkzDrawPoints(M,N,P,Q) \tkzLabelPoints[above right](M,N,P,Q)
				
				\fill[pattern=north east lines,pattern color=black!60] (A2)--(A3)--(A4)--(\xmax,0)--(Q)--(P)--(N)--(M)--(0,\ymax)--cycle;
				\draw[domain=-2:12] plot(\x,{5-(\x)}) plot(\x,{6-2*(\x)}) plot(\x,{3-(\x)/3});
				\begin{scriptsize}
					\draw[->](\xmin,0)--(\xmax,0); \draw(\xmax-0.1,0) node[below]{$x$};
					\draw[->](0,\ymin)--(0,\ymax); \draw(0,\ymax-0.2) node[right]{$y$};
					\foreach \x in {2,4,6,8,10}
					\draw (\x,0.05) -- ++(0,-0.1) node [below] {$\x$};
					\foreach \x in {2,4,6,8}
					\draw (0.05,\x) -- ++(-0.1,0) node [left] {$\x$};
					\draw node [below left]{$O$};
				\end{scriptsize}
			\end{tikzpicture}
		}	
	}
\end{ex}

\begin{ex}%[0D4K4-3]
	Một gia đình cần ít nhất $900$ đơn vị prô-tê-in và $400$ đơn vị li-pít trong thức ăn mỗi ngày. Mỗi kí-lô-gam thịt bò chứa $800$ đơn vị prô-tê-in và $200$ đơn vị li-pít. Mỗi kí-lô-gam thịt lợn chứa $600$ đơn vị prô-tê-in và $400$ đơn vị li-pít. Biết rằng gia đình này chỉ mua tối đa $1,6$ kg thịt bò và $1,1$ kg thịt lợn; giá tiền $1$ kg thịt bò là $45000$ đồng, $1$ kg thịt lợn là $35000$ đồng. Hỏi gia đình đó phải mua bao nhiêu kí-lô-gam thịt mỗi loại để số tiền bỏ ra là ít nhất?
	\choice
	{$0{,}3$ kg thịt bò và $1{,}1$ kg thịt lợn}
	{\True $0{,}6$ kg thịt bò và $0{,}7$ kg thịt lợn}
	{$1{,}6$ kg thịt bò và $1{,}1$ kg thịt lợn}
	{$0{,}6$ kg thịt lợn và $0{,}7$ kg thịt bò}
	\loigiai{Gọi $x$ và $y$ lần lượt là số kí-lô-gam thịt bò và thịt lợn mà gia đình đó mua mỗi ngày ($0\leq x\leq 1{,}6;0\leq y\leq 1{,}1$).\\
		Khi đó chi phí để mua số thịt trên là: $F=45000x+35000y$ đồng.\\
		Trong $x$ kg thịt bò chứa $800x$ đơn vị prô-tê-in và $200x$ đơn vị li-pít.\\
		Trong $y$ kg thịt lợn chứa $600x$ đơn vị prô-tê-in và $400y$ đơn vị li-pít.\\
		Suy ra, số đơn vị prô-tê-in và số đơn li-pít lần lượt là $800x+600y$ đơn vị và $200x+400y$ đơn vị.
		Do gia đình này cần ít nhất $900$ đơn vị prô-tê-in và $400$ đơn vị li-pít trong thức ăn mỗi ngày nên ta có hệ bất phương trình sau: 
		$$\heva{&800x+600y\geq 900\\&200x+400y\geq 400\\&0\leq x\leq 1{,}6\\&0\leq y\leq 1{,}1}\Leftrightarrow \heva{&8x+6y\geq 9\\&x+2y\geq 2\\&0\leq x\leq 1{,}6\\&0\leq y\leq 1{,}1.}$$	
		\immini{Bài toán trở thành: Tìm GTNN của $F=45000x+35000y$ với $x$, $y$ thỏa hệ trên.\\
			Giải hệ bất phương trình trên, ta có miền nghiệm là tứ giác $ABCD$ (hình bên) với tọa độ các đỉnh là: $A(1{,}6;1{,}1)$, $B(1{,}6;0{,}2)$, $C(0{,}6;0{,}7)$, $D(0{,}3;1{,}1)$.\\
			Khi đó: \\Tại $A(1{,}6;1{,}1)$: $F=110500$\\
			Tại $B(1{,}6;0{,}2)$: $F=79000$\\
			Tại $C(0{,}6;0{,}7)$: $F=51500$\\
			Tại $D(0{,}3;1{,}1)$: $F=52000$.
		}
		{\begin{tikzpicture}[scale=1, font=\footnotesize, line join=round, line cap=round, >=stealth]
				\fill[pattern=north east lines] (-1,-1) -- (-1,2) -- (3,2) -- (3,-1) -- cycle;
				\fill[white] (1.6,1.1) -- (1.6,0.2) -- (0.6,0.7) --(0.3,1.1)-- cycle;
				\draw[->] (-1.3,0) -- (3.3,0)node[above]{$x$};
				\foreach \x in {1,2}
				\draw[shift={(\x,0)},color=black] (0pt,2pt) -- (0pt,-2pt) node[below left] {\footnotesize $\x$};
				\draw[->,color=black] (0,-1.2) -- (0,2.3)node[right]{$y$};
				\foreach \y in {1}
				\draw[shift={(0,\y)},color=black] (2pt,0pt) -- (-2pt,0pt) node[above right] {\footnotesize $\y$};
				\node[above left] at (0,0){$O$};
				\node[above left] at (1.6,1.1){$A$};
				\node[above right] at (1.6,0.2){$B$};
				\node[below left] at (0.6,0.7){$C$};
				\node[above right] at (0.3,1.1){$D$};
				\clip(-1,-1) rectangle (3,2);
				\draw[smooth,samples=100,domain=-1:3] plot(\x,{1.5-1.333333*(\x)});
				\draw[smooth,samples=100,domain=-1:3] plot(\x,{1-0.5*(\x)});
				\draw[,smooth,samples=100,domain=-1:3] plot(\x,{1.1-0*(\x)});
				\draw[smooth,samples=100,domain=-1:3] (1.6,-1)--(1.6,3);
			\end{tikzpicture}
		}
		Suy ra, $F$ nhỏ nhất khi $(x;y)=(0{,}6;0{,}7)$. Do đó gia đình này cần mua $0{,}6$ kg thịt bò và $0{,}7$ kg thịt lợn.
	}
\end{ex}

\begin{ex}%[0D4K4-4]
	Một cửa hàng làm kệ sách và bàn làm việc. Mỗi kệ sách cần $5$ giờ chế biến gỗ và $4$ giờ hoàn thiện. Mỗi bàn làm việc cần $10$ giờ chế biến gỗ và $3$ giờ hoàn thiện. Mỗi tháng cửa hàng có $600$ giờ lao động để chế biến gỗ và $240$ giờ để hoàn thiện. Lợi nhuận của mỗi kệ sách là $400$ nghìn đồng và mỗi bàn là $750$ nghìn đồng. Có bao nhiêu sản phẩm mỗi loại cần được làm mỗi tháng để thu được lợi nhuận tối đa?
	\choice
	{$24000$}
	{$45000$}
	{\True $45600$}
	{$46000$}
	\loigiai{
		\immini{Ta có hệ: $\heva{& 5x+10y\le 600 \\ & 4x+3y\le 240 \\ & x\ge 0 \\ & y\ge 0}$\\
			Lợi nhuận: $P=400x+750y$.\\
			Miền nghiệm của hệ là miền tứ giác $OABC$ với:\\
			$A(0;60)$, $B(24;48)$, $C(60;0)$.\\
			Lợi nhuận tối đa $P_{max}=P(B)=45600$.
		}
		{
			\begin{tikzpicture}[scale=0.05, font=\footnotesize, line join=round, line cap=round, >=stealth]
				\def\xmin{-10} \def\xmax{140}
				\def\ymin{-10} \def\ymax{100}
				\clip(\xmin,\ymin) rectangle (\xmax,\ymax);
				\tkzDefPoints{\xmax/\ymax/A1,\xmin/\ymax/A2,\xmin/\ymin/A3,\xmax/\ymin/A4}
				\fill[pattern=north east lines,pattern color=black!60] (A1)--(A2)--(A3)--(A4)--cycle;
				\tkzDefPoints{0/0/O,0/60/A,24/48/B,60/0/C}
				\fill[color=white] (O)--(A)--(B)--(C)--cycle;
				\tkzDrawPoints(A,B,C)
				\draw[domain=-10:140] plot(\x,{60-(\x)/2}) plot(\x,{80-4*(\x)/3});
				\begin{scriptsize}
					\draw[->](\xmin,0)--(\xmax,0); \draw(\xmax-4,0) node[below]{$x$};
					\draw[->](0,\ymin)--(0,\ymax); \draw(0,\ymax-4) node[right]{$y$};
					\draw node [below left]{$O$}
					(A) node [above right]{$A$}
					(B) node [below left]{$B$}
					(C) node [above right]{$C$};
				\end{scriptsize}
			\end{tikzpicture}
		}
	}
\end{ex}

\begin{ex}%[0D4G4-4]
	Cho hệ bất phương trình $\heva{&\vert x-1 \vert \leq 2 \\ &\vert y+1 \vert \leq 3}$ và biểu thức $P(x;y)=3x+2y-5$ với $(x;y)$ thuộc miền nghiệm của hệ bất phương trình đã cho. Tìm giá trị lớn nhất của $P$.
	\choice
	{$16$}
	{$-16$}
	{\True $8$}
	{$-8$}
	\loigiai{
		$$\heva{&\vert x-1 \vert \leq 2 \\ &\vert y+1 \vert \leq 3} 
		\Leftrightarrow
		\heva{&-1\leq x\leq 3 \\ &-4\leq y\leq 2.}$$
		Miền nghiệm là hình chữ nhật $ABCD$ với $A(3;2)$, $B(3;-4)$, $C(-1;-4)$ và $D(-1;2)$. Giá trị lớn nhất của $P$ đạt được tại đỉnh $A(3;2)$ và $P(3;2)=8$.	
	}
\end{ex}

\begin{ex}%[0D4K4-3]
	Người ta dự định dùng hai loại nguyên liệu để chiết xuất ít nhất $140$ kg chất A và $9$ kg chất B. Từ mỗi tấn nguyên liệu loại I giá $4$ triệu đồng, có thể chiết xuất được $20$ kg chất A và $0{,}6$ kg chất B. Từ mỗi tấn nguyên liệu loại II giá $3$ triệu đồng có thể chiết xuất được $10$ kg chất A và $1{,}5$ kg chất B. Hỏi phải dùng bao nhiêu tấn nguyên liệu mỗi loại để chi phí mua nguyên liệu là ít nhất, biết rằng cơ sở cung cấp nguyên liệu chỉ có thể cung cấp không quá $10$ tấn nguyên liệu loại I và không quá $9$ tấn nguyên liệu loại II?
	\choice
	{$2{,}5$ tấn loại I và $9$ tấn loại II}
	{$10$ tấn loại I và $9$ tấn loại II}
	{$10$ tấn loại I và $2$ tấn loại II}
	{\True $5$ tấn loại I và $4$ tấn loại II}
	\loigiai{
		Gọi $x$, $y$ lần lượt là số tấn nguyên liệu loại I và loại II phải dùng.\\
		Từ bài toán ta đưa được hệ bất phương trình:
		$\heva{&0 \leq x \leq 10\\&0 \leq y \leq 9\\&2x+y\geq 14\\&2x+5y\geq 30} (*)$
		\immini{
			Tổng chi phí là $F(x;y)=4x+3y$\\
			Ta tìm $x$, $y$ thỏa mãn hệ $(*)$ sao cho $F(x;y)$ nhỏ nhất.\\
			Ta biết giá trị nhỏ nhất đạt tại các điểm $A(5;4)$, $B(10;2)$, $C(10;9)$, $D(3;9)$.\\
			Thử lại thấy $F(5;4)$=32 là giá trị nhỏ nhất.\\
			Vậy cần 5 tấn nguyên liệu loại I và 4 tấn nguyên liệu loại II.
		}{
			\begin{tikzpicture}[scale=0.4, font=\footnotesize, line join=round, line cap=round, >=stealth]
				\draw[->] (-1,0) -- (15.7,0)node[above]{$x$};
				\foreach \x in {,,,4,,,7,,,10,,,,,15}
				\draw[shift={(\x,0)},color=black] (0pt,2pt) -- (0pt,-2pt) node[below left] {\footnotesize $\x$};
				\draw[->,color=black] (0,-1) -- (0,14.7)node[right]{$y$};
				\foreach \y in {,2,,,,6,,,9,,,,,14}
				\draw[shift={(0,\y)},color=black] (2pt,0pt) -- (-2pt,0pt) node[above left] {\footnotesize $\y$};
				\node[below left] at (0,0) {$O$};
				\clip(-1,-1) rectangle (15.3,14.3);
				\fill[pattern=north east lines] (-0.15,14.3) -- (-1,14.3) -- (-1,-1) -- (7.5,-1)-- cycle;
				\draw[smooth,samples=100,domain=-1:9] plot(\x,{14-2*(\x)});
				\fill[pattern=north east lines] (-1,6.4)--(-1,-1)--(15.3,-1)--(15.3,-0.12)-- cycle;
				\draw[smooth,samples=100,domain=-1:15.3] plot(\x,{6-0.4*(\x)});
				\fill[pattern=north east lines] (-1,9)--(-1,14.3) --(15.3,14.3)--(15.3,9)-- cycle;
				\draw[smooth,samples=100,domain=-1:15.3] plot(\x,{9+0*(\x)});
				\fill[pattern=north east lines] (10,14.3)--(15.3,14.3) --(15.3,-1)--(10,-1)-- cycle;
				\draw[smooth,samples=100](10,-1)--(10,15.3);
				\fill[pattern=north east lines] (0,4)--(-3,4) --(-3,-3)--(0,-3)-- cycle;
				\draw[smooth,samples=100,domain=-3:9] plot(\x,{0*(\x)});
				\node[above left] at (5.5,4.5) {$A$};
				\node[left] at (10,3) {$B$};
				\node[below left] at (10,9) {$C$};
				\node[below right] at (2.75,9) {$D$};	
			\end{tikzpicture}
			
		}
	}
\end{ex}
\begin{ex}%[0D4K4-4]
	Giá trị nhỏ nhất $F_{\text {min }}$ của biểu thức $F(x ; y)=y-x$ trên miền xác định bởi hệ $\heva{&y-2 x \leq 2 \\ &2 y-x \geq 4  \\ &x+y \leq 5}$ là
	\choice
	{\True $F_{\min }=1$}
	{$F_{\min }=2$}
	{$F_{\min }=3$}
	{$F_{\min }=4$}
	\loigiai{
		Ta có $\heva{&y-2 x \leq 2 \\ &2 y-x \geq 4  \\ &x+y \leq 5} \Leftrightarrow \heva{&y-2 x -2\leq 0 \\ &2 y-x-4 \geq 0  \\ &x+y-5 \leq 0.}\quad(*)$\\
		Trong mặt phẳng tọa độ $ Oxy $, vẽ các đường thẳng $d_1\colon y-2x-2=0 $, $ d_2\colon 2y-x-4=0 $, $ d_3\colon x+y-5=0 $. Khi đó miền nghiệm của hệ bất phương trình $(*)$ là phần mặt phẳng (tam giác $ABC$ kể cả biên) như hình vẽ. Xét các đỉnh của miền khép kín tạo bởi hệ $(*)$ là $A(0 ; 2)$, $B(2 ; 3)$, $C(1 ; 4)$. 
		\begin{center}
			\begin{tikzpicture}[line cap=round,line join=round,>=stealth,x=1.0cm,y=1.0cm,scale=.7]
				\clip(-2,-1) rectangle (6,6);
				\draw[->] (-2,0) -- (6,0) node[above left]{$x$};
				\draw[->] (0,-1) -- (0,6)node[below right]{$y$};
				\foreach \x in{1,2}\draw[fill=black] (\x,0) circle (.5pt) node[below]{\footnotesize $\x$};
				\foreach \y in{2,3,4,5}\draw[fill=black] (0,\y)circle (.5pt) node[left]{\footnotesize $\y$};
				\draw[smooth,samples=100,domain=-2:6] plot(\x,{(--2--2*\x)/1});
				\draw[smooth,samples=100,domain=-2:6] plot(\x,{(--4--1*\x)/2});
				\draw[smooth,samples=100,domain=-2:6] plot(\x,{(--5-1*\x)/1});
				\fill[pattern=north east lines] (-2,-1) -- (-2,6) -- (6,6) -- (6,-1) -- cycle;
				\draw [fill=white] (0,2) -- (1,4) -- (2,3) -- cycle;
				\draw [fill=black] (0,0)node[above left]{$O$}circle(.5pt) (0,2)node[below right]{$A$}circle(.5pt) (1,4)node[right]{$B$}circle(.5pt) (2,3)node[above]{$C$}circle(1.2pt);
				\draw [fill=black] (1.7,5.5)node[right]{$d_1$}circle(.5pt) (4,4.8)node[right]{$d_2$}circle(.5pt) (4.5,.5)node[above]{$d_3$}circle(.5pt);
				\draw [dashed] (0,4) -- (1,4) -- (1,0) (0,3) -- (2,3) -- (2,0);
			\end{tikzpicture}
		\end{center}
		Ta có $\heva{
			&F(0 ; 2)=2 \\
			&F(2 ; 3)=1 \\
			&F(1 ; 4)=3
		}\longrightarrow F_{\min }=1$.
	}
\end{ex}
\begin{ex}%[0D4K4-3]
	Một nhà máy sản xuất, sử dụng ba loại máy đặc chủng để sản xuất sản phẩm $A$ và sản phẩm $B$ trong một chu trình sản xuất. Đề sản xuất một tấn sản phẩm $A$ lãi $4$ triệu đồng người ta sử dụng máy $I$ trong 1 giờ, máy II trong $2$ giờ và máy III trong $3$ giờ. Để sản xuất ra một tấn sản phẩm $B$ lãi được $3$ triệu đồng người ta sử dụng máy I trong $6$ giờ, máy II trong $3$ giờ và máy III trong $2$ giờ. Biết rằng máy $I$ chỉ hoạt động không quá $36$ giờ, máy hai hoạt động không quá $23$ giờ và máy III hoạt động không quá $27$ giờ. Hãy lập kế hoạch sản xuất cho nhà máy để tiền lãi được nhiều nhất. 
	\choice
	{Sản xuất $9$ tấn sản phẩm $A$ và không sản xuất sản phẩm $B$}
	{\True Sản xuất $7$ tấn sản phẩm $A$ và $3$ tấn sản phẩm $B$}
	{Sản xuất $\dfrac{45}{8}$ tấn sản phẩm $A$ và $\dfrac{81}{16}$ tấn sản phẩm $B$}
	{Sản xuất $6$ tấn sản phẩm $B$ và không sản xuất sản phẩm $A$}
	\loigiai{
		Gọi $x \geq 0$, $y \geq 0$ (tấn) là sản lượng cần sản xuất của sản phẩm $A$ và sản phẩm $B$. Ta có:\\
		$x+6 y$ là thời gian hoạt động của máy I,\\
		$2 x+3 y$ là thời gian hoạt động của máy II,\\
		$3 x+2 y$ là thời gian hoạt động của máy III.\\
		Số tiền lãi của nhà máy: $f(x;y)=4 x+3 y$ (triệu đồng).\\
		Bài toán trở thành: Tìm $(x;y)$ thỏa mãn $\heva{&x+6 y \leq 36 \\ &2 x+3 y \leq 23 \\ &3 x+2 y \leq 27 \\ & x \ge 0,\ y \ge 0}$ để $f(x;y)=4 x+3 y $ đạt giá trị lớn nhất.\\
		Miền nghiệm của hệ trên là ngũ giác $OABCD$ (kể cả bờ).
		\begin{center}
			\begin{tikzpicture}[line cap=round,line join=round,>=stealth,x=1.0cm,y=1.0cm,scale=.7]
				\clip(-1,-1.5) rectangle (10,7);
				\draw[->] (-1,0) -- (10,0) node[above left]{$x$};
				\draw[->] (0,-1.5) -- (0,7)node[below right]{$y$};
				\draw[smooth,samples=100,domain=-1:10] plot(\x,{(--36-1*\x)/6});
				\draw[smooth,samples=100,domain=-1:10] plot(\x,{(--23-2*\x)/3});
				\draw[smooth,samples=100,domain=-1:10] plot(\x,{(--27-3*\x)/2});
				\fill[pattern=north east lines] (-1,-1.5) -- (-1,7) -- (10,7) -- (10,-1.5) -- cycle;
				\draw [fill=white] (0,0)--(0,6)--(3.33,5.44) -- (7,3) -- (9,0)-- cycle;
				\fill (0,0)node[above left]{$O$}circle(1.2pt) (0,6)node[above left]{$A$}circle(1.2pt) (3.33,5.44)node[above right]{$B$}circle(1.2pt) (7,3)node[above right]{$C$}circle(1.2pt) (9,0)node[below left]{$D$}circle(1.2pt) ;
				% \draw[fill=black] (5.63,0) node[below]{$\dfrac{45}{8}$} (0,5.06)node[left]{$\dfrac{81}{16}$};
				% \draw[dashed] (5.63,0) |- (5.63,5.06) |- (0,5.06);
			\end{tikzpicture}
		\end{center}
		Thay toạ độ các điểm $O(0;0)$, $A\left(0;6\right)$, $B\left(\dfrac{10}{3};\dfrac{49}{9}\right)$, $C(7;3)$, $D(9;0)$ vào $f(x;y)$ ta được 
		$f(0;0)=0$, $f\left(\dfrac{10}{3};\dfrac{49}{9}\right)=\dfrac{89}{3}$, $f(7;3)=37$, $f(9,0)=36$.\\
		Suy ra $f(x;y)$ lớn nhất khi $(x;y)=\left(7;3\right)$. \\
		Như vậy để tiền lãi được nhiều nhất thì sản xuất $7$ tấn sản phẩm $A$ và $3$ tấn sản phẩm $B$.}
\end{ex}
\begin{ex}%[0D4K4-1]
	Biểu thức $F=y-x$ đạt giá trị nhỏ nhất với điều kiện 
	$\heva{
		&-2x+y\le -2 \\
		&x-2y\le 2 \\
		&x+y\le 5 \\
		&x\ge 0}$ tại điểm $S(x;y)$ có toạ độ là
	\choice
	{\True $(4;1)$}
	{$(3;1)$}
	{$(2;1)$}
	{$(1;1)$}
	\loigiai{
		Biểu diễn miền nghiệm của hệ bất phương trình $\heva{
			&-2x+y\le -2 \\
			&x-2y\le 2 \\
			&x+y\le 5 \\
			&x\ge 0}$ trên hệ trục tọa độ như dưới đây:
		\begin{center}
			\begin{tikzpicture}[scale=1, font=\footnotesize, line join=round, line cap=round, >=stealth]
				%=========== Vẽ lưới và hệ trục tọa độ ===========%
				\def \xmin{-1.5} \def \xmax{5.5}
				\def \ymin{-1.5} \def \ymax{4.5} 
				\def \f(#1){2*(#1)-2}
				\def \g(#1){((#1)-2)/2}
				\def \h(#1){(5-(#1))}
				\begin{scope}
					\clip (\xmin,\ymin) rectangle (\xmax,\ymax);
					\draw [] plot [domain=\xmin:\xmax](\x,{\f(\x)}) node [below right=0.5mm]{$f(x)$};
					\draw [] plot [domain=\xmin:\xmax](\x,{\g(\x)});
					\draw [] plot [domain=\xmin:\xmax](\x,{\h(\x)});
				\end{scope}
				\fill[pattern=north east lines] (\xmin,\ymin) -- (\xmax,\ymin) -- (\xmax,\ymax) -- (\xmin,\ymax) -- cycle;
				\draw [fill=white] (0.67,-0.67) -- (2.33,2.66) -- (4,1) -- cycle;
				\draw[-stealth,blue] (\xmin,0)--(\xmax,0) node [below]{$x$};
				\draw[-stealth,blue] (0,\ymin)--(0,\ymax) node[left]{$y$};
				\draw [blue,fill](0,0) circle(1pt) node[below left]{$O$};
				\draw [blue] (0,0.2)-|(0.2,0);
				\draw [fill](0.67,-0.67) circle(1pt) node[above left] {$A$};
				\draw [fill](2.33,2.66) circle(1pt) node[above] {$B$};
				\draw [fill](4,1) circle(1pt) node[above] {$C$};
			\end{tikzpicture}
		\end{center}
		Nhận thấy biết thức $F=y-x$ chỉ đạt giá trị nhỏ nhất tại các điểm $A$, $B$ hoặc $C$.\\
		Chỉ $C(4;1)$ có tọa độ nguyên nên thỏa mãn.
		Vậy $\min F=-3$ khi $x=4$, $y=1$.
	}
\end{ex}

\begin{ex}%[0D4G4-2]
	Giá trị lớn nhất của biểu thức $F(x; y)=x+2y$, với điều kiện $\heva{
		&0\le y\le 4 \\
		&x\ge 0 \\
		&x-y-1\le 0 \\
		&x+2y-10\le 0}$ là
	\choice
	{$6$}
	{$8$}
	{\True $10$}
	{$12$}
	\loigiai{
		Vẽ các đường thẳng $d_1 \colon y=4$;
		$d_2 \colon x-y-1=0$; $d_3 \colon x+2y-10=0$;
		$Ox \colon y=0$; $Oy \colon x=0$.
		\begin{center}
			\begin{tikzpicture}[scale=0.8, font=\footnotesize, line join=round, line cap=round, >=stealth]
				%=========== Vẽ lưới và hệ trục tọa độ ===========%
				\def \xmin{-1.5} \def \xmax{5}
				\def \ymin{-1.5} \def \ymax{4.5} 
				\fill [pattern=north east lines] (\xmin,\ymin) rectangle (\xmax,\ymax);
				\fill [white] (0,0)--(1,0)--(4,3)--(2,4)--(0,4)--cycle;
				\draw[-stealth,blue] (\xmin,0)--(\xmax,0) node [below]{$x$};
				\draw[-stealth,blue] (0,\ymin)--(0,\ymax) node[left]{$y$};
				\draw [fill](0,0) circle(1pt) node[above right]{$O$};
				\draw [blue] (0,0.2)-|(0.2,0);
				\def \f(#1){4}
				\def \g(#1){(#1)-1}
				\def \h(#1){(10-(#1))/2}
				\begin{scope}
					\clip (\xmin,\ymin) rectangle (\xmax,\ymax);
					\draw [] plot [domain=\xmin:\xmax](\x,{\f(\x)}) node [below right=0.5mm]{$f(x)$};
					\draw [] plot [domain=\xmin:\xmax](\x,{\g(\x)});
					\draw [] plot [domain=\xmin:\xmax](\x,{\h(\x)});
				\end{scope}
				\draw (0,4) circle(1pt) node[below right] {$A$};
				\draw (1,0) circle(1pt) node[above] {$B$};
				\draw (4,3) circle(1pt) node[left] {$C$};
				\draw (2,4) circle(1pt) node[below] {$D$};
			\end{tikzpicture}
		\end{center}
		Các đường thẳng trên đôi một cắt nhau tại $A(0;4)$, $O(0;0)$, $B(1;0)$, $C(4;3)$, $D(2;4)$.\\
		Vì điểm $M_0(1;1)$ có toạ độ thoả mãn tất cả các bất phương trình trong hệ nên ta tô đậm các nửa mặt phẳng bờ $d_1$, $d_2$, $d_3$, $Ox$, $Oy$ không chứa điểm $M_0$.\\
		Miền không bị tô đậm là đa giác $OADCB$ kể cả các cạnh (hình bên) là miền nghiệm của hệ bất phương trình đã cho.\\
		Kí hiệu $F(A)=F\left(x_A;y_A\right)=x_A+2y_A$, ta có 
		$F(A)=8$, $F(O)=0$, $F(B)=1$, $F(C)=10$; $F(D)=10$; $0<1<8<10$.\\
		Giá trị lớn nhất cần tìm là $10$.
	}
\end{ex}
\Closesolutionfile{ans}


%%Chương 4
\def\tenchude{GTLG CỦA MỘT GÓC TỪ $\mathbf{0^\circ}$ ĐẾN $\mathbf{180^\circ}$}
\setcounter{section}{0}
\section{GTLG CỦA MỘT GÓC TỪ $\mathbf{0^\circ}$ ĐẾN $\mathbf{180^\circ}$}
\subsection{Tóm tắt lí thuyết}
\subsubsection{Khái niệm}
	\immini
	{
Điểm $M(x_0;y_0)$ nằm trên nửa đường tròn đơn vị sao cho $\widehat{xOM}=\alpha$. Khi đó
\begin{itemize}
	\item $\sin\alpha=y_0$;
	\item $\cos\alpha=x_0$;
	\item $\tan\alpha=\dfrac{\sin\alpha}{\cos\alpha}$ với $(\alpha\ne 90^\circ)$;
	\item $\cot\alpha=\dfrac{\cos\alpha}{\sin\alpha}$ với ($\alpha\ne 0^\circ,180^\circ$).
\end{itemize}
}
{
\begin{tikzpicture}[scale=1.5, font=\footnotesize,line join=round, line cap=round, >=stealth, x=2cm,y=2cm]
	\def\x{0.5}
	\def\a{60}
	\pgfmathsetmacro{\y}{\x*tan(\a)}
	\draw[->] (-1.2,0)--(1.2,0) node [below]{$x$};
	\draw[->] (0,-0.2)--(0,1.2) node [left]{$y$};
	\node at (0,0) [below left]{$O$};
	\draw (1,0) arc (0:180:2cm);
	\fill (1,0) node[below]{$1$} circle(1pt);
	\fill (-1,0) node[below]{$-1$} circle(1pt);
	\fill (0,1) node[above left]{$1$} circle(1pt);
	%	\fill (\x,0) node[below]{$x_0$} circle(1pt);
	\fill (\x,0) node[below]{$x_0$} circle(1pt);
	%	\fill (\a:1) node[above right]{$M$} circle(1pt);
	\fill (\a:1) node[above]{$M$} circle(1pt);
	\fill (0,\y) node[left]{$y_0$} circle(1pt);
	\draw[dashed] (\x,0)|-(0,\y);
	\draw (0,0)--(\a:1);
	\draw (0.3,0) arc (0:\a:0.6cm);
	\node at (0,0)[shift={(25:0.2)}]{$\alpha$};
\end{tikzpicture}
}
\subsubsection{Dấu của giá trị lượng giác.}
\begin{center}
	\renewcommand\arraystretch{1.2} %tăng độ rộng
	% \renewcommand{\tabcolsep}{6mm} %tăng chiều dài
	\begin{tabular}{|c|l l r|l r c|}
		\hline
		Góc $\alpha$ & $0^\circ$ && \multicolumn{2}{c}{$90^\circ$} && $180^\circ$\\
		\hline
		$\sin\alpha$ && $+$ &&& $+$ &\\
		\hline
		$\cos\alpha$ && $+$ &&& $-$ &\\
		\hline
		$\tan\alpha$ && $+$ &&& $-$ &\\
		\hline
		$\cot\alpha$ && $+$ &&& $-$ &\\
		\hline
	\end{tabular}
\end{center}
\subsubsection{Bảng giá trị lượng giác của một số góc đặc biệt cần nhớ}
\begin{center}
	\renewcommand{\arraystretch}{2}%
	\begin{tabular}{|c|c|c|c|c|c|c|c|c|c|}
		\hline
		$\alpha$ & $0^\circ$ & $30^\circ$ & $45^\circ$ & $60^\circ$ & $90^\circ$ & $120^\circ$ & $135^\circ$ & $150^\circ$& $180^\circ$ \\
		\hline
		$\sin\alpha$& $0$ & $\dfrac{1}{2}$ & $\dfrac{\sqrt{2}}{2}$  &  $\dfrac{\sqrt{3}}{2}$ & $1$ & $\dfrac{\sqrt{3}}{2}$ & $\dfrac{\sqrt{2}}{2}$ & $\dfrac{1}{2}$ & $0$ \\
		\hline
		$\cos\alpha$& $1$ & $\dfrac{\sqrt{3}}{2}$ & $\dfrac{\sqrt{2}}{2}$  &  $\dfrac{1}{2}$& $0$& $\dfrac{-1}{2}$&$\dfrac{-\sqrt{2}}{2}$&$\dfrac{-\sqrt{3}}{2}$&$-1$ \\
		\hline
		$\tan\alpha$& $0$ & $\dfrac{\sqrt{3}}{3}$ & $1$  &  $\sqrt{3}$& $\parallel$& $-\sqrt{3}$&$-1$&$\dfrac{-\sqrt{3}}{3}$& $0$\\
		\hline
		$\cot\alpha$& $\parallel$ & $\sqrt{3}$ & $1$  &  $\dfrac{\sqrt{3}}{3}$& $0$& $\dfrac{-\sqrt{3}}{3}$&$-1$&$-\sqrt{3}$& $\parallel$\\
		\hline
	\end{tabular}
\end{center}

\subsubsection{Tính chất}
\begin{multicols}{2}
\begin{enumerate}
	\item Giá trị lượng giá của hai góc phụ nhau
	\begin{itemize}
		\item $\sin(90^\circ-\alpha)=\cos\alpha$.
		\item $\cos(90^\circ-\alpha)=\sin\alpha$.
		\item $\tan(90^\circ-\alpha)=\cot\alpha$.
		\item $\cot(90^\circ-\alpha)=\tan\alpha$.
	\end{itemize}
	\item Giá trị lượng giác của hai góc bù nhau
	\begin{itemize}
		\item $\sin(180^\circ-\alpha)=\sin\alpha$.
		\item $\cos(180^\circ-\alpha)=-\cos\alpha$.
		\item $\tan(180^\circ-\alpha)=-\tan\alpha$.
		\item $\cot(180^\circ-\alpha)=-\cot\alpha$.
	\end{itemize}
\end{enumerate}
\end{multicols}
\begin{enumerate}
\item[c)] Hệ thức cơ bản
	\begin{itemize}
		\item $\sin^2\alpha+\cos^2\alpha=1$.
		\item $1+\tan^2\alpha=\dfrac{1}{\cos^2\alpha}$ với $(\alpha\ne 90^\circ)$.
		\item $1+\cot^2\alpha=\dfrac{1}{\sin^2\alpha}$ với $(0^\circ<\alpha< 180^\circ)$.
		\item $\tan\alpha\cdot\cot\alpha=1$ với $(0^\circ<\alpha< 180^\circ, \alpha\ne 90^\circ)$.
	\end{itemize}
\end{enumerate}
\subsection{Các dạng toán}
%\setcounter{subsection}{1}% Reset lại số đếm subsection
\begin{dang}{Tính giá trị biểu thức lượng giác. Chứng minh đẳng thức lượng giác}
	Áp dụng các công thức lượng giác
\end{dang}
\subsubsection{Ví dụ minh hoạ}
\begin{vd}%[0H2Y1]
	Tính giá trị biểu thức sau
	\begin{tasks}(2)
		\task $A= 2\cos 30^\circ+3\sin 120^\circ$.
		\task $B=a\cos60^{\circ}+2a\tan45^{\circ}-3a\sin30^{\circ}$.
	\end{tasks}
	\loigiai{
		\begin{enumerate}[a)]
			\item
			\item Ta có $B=\dfrac{1}{2}a+2a-\dfrac{1}{2}.3a=a$.
		\end{enumerate}
		
	}
\end{vd}
\begin{vd}%[0H2Y1]
	Cho $x=30^{\circ}$. Tính $A=\sin (2x)-3\cos x$.
	\loigiai{
		$A=\sin 2.(30^{\circ})-3\cos30^{\circ}=\sin60^{\circ}-3\cos30^{\circ}=\dfrac{\sqrt{3}}{2}-3\dfrac{\sqrt{3}}{2}=-\sqrt{3}$.
	}
\end{vd}

\begin{vd}
	Biết $\sin15^\circ=\dfrac{\sqrt{3}-1}{2\sqrt{2}}$. Tính giá trị biểu thức $P=\sin165^\circ+\cos75^\circ$.
\end{vd}

\begin{vd}
Không dùng máy tính, tính giá trị của các biểu thức sau
\begin{enumerate}
	\item $A=\sin 45^\circ\cot 135^\circ+\cos 60^\circ\cdot\sin 150^\circ-\cos 30^\circ\cdot\sin 120^\circ$.
	\item $B=\tan 135^\circ+\cot 60^\circ\cot 30^\circ-\tan 60^\circ\tan 150^\circ$.
	\item $C=2\sin 60^\circ\tan 150^\circ-\cos 180^\circ\cdot \cot 45^\circ$.
\end{enumerate}	
\loigiai{
\begin{enumerate}
	\item Ta có $\sin 45^\circ=-\cos 135^\circ=\dfrac{\sqrt{2}}{2}$, $\cos 60^\circ=\sin 150^\circ=\dfrac{1}{2}$ và $\cos 30^\circ=\sin 120^\circ=\dfrac{\sqrt{3}}{2}$.\\
	Từ đó suy ra
	$A=\dfrac{\sqrt{2}}{2}\cdot\left(\dfrac{-\sqrt{2}}{2}\right)+\dfrac{1}{2}\cdot\dfrac{1}{2}-\dfrac{\sqrt{3}}{2}\cdot\dfrac{\sqrt{3}}{2}=\dfrac{-1}{2}+\dfrac{1}{4}-\dfrac{3}{4}=-1$.
	\item Do $\tan 135^\circ=-1$, $\cot 60^\circ=\dfrac{\sqrt{3}}{3}$, $\cot 30^\circ=\tan 60^\circ=\sqrt{3}$ và $\tan 150^\circ=\dfrac{-\sqrt{3}}{3}$ nên
	$$B=-1+\dfrac{\sqrt{3}}{3}\cdot \sqrt{3}-\sqrt{3}\cdot\left(\dfrac{-\sqrt{3}}{3}\right)=1.$$
	\item Ta có $\sin 60^\circ=\dfrac{\sqrt{3}}{2}$, $\tan 150^\circ=\dfrac{-\sqrt{3}}{3}$, $\cos 180^\circ=-1$ và $\cot 45^\circ=1$.\\
	Suy ra $C=2\cdot\dfrac{\sqrt{3}}{2}\cdot\left(\dfrac{-\sqrt{3}}{3}\right)-(-1)\cdot 1=0$.
\end{enumerate}	
\textbf{Chú ý.} Nếu để ý đến mối liên hệ giữa các góc có trong biểu thức, như các góc bù nhau, các góc phụ nhau, thì ta có thể giải bài toán theo cách sau
\begin{enumerate}
	\item Do $135^\circ=180^\circ-45^\circ$, $150^\circ=180^\circ-30^\circ$, $120^\circ=180^\circ-60^\circ$ nên 
	\allowdisplaybreaks
	\begin{eqnarray*}
		A&=&\sin 45^\circ\cdot(-\cos 45^\circ)+\cos 60^\circ\cdot\sin 30^\circ-\cos 30^\circ\cdot\sin 60^\circ\\
		&=&\dfrac{\sqrt{2}}{2}\cdot\left(\dfrac{-\sqrt{2}}{2}\right)+\dfrac{1}{2}\cdot\dfrac{1}{2}-\dfrac{\sqrt{3}}{2}\cdot\dfrac{\sqrt{3}}{2}=-\dfrac{1}{2}+\dfrac{1}{4}-\dfrac{3}{4}=-1.
	\end{eqnarray*}
\item Do $135^\circ=180^\circ-45^\circ$, $60^\circ=90^\circ-30^\circ$, $150^\circ=180^\circ-30^\circ$ nên
$$B=-1+1-\tan 60^\circ\cdot(-\tan 30^\circ)=1.$$
\item Do $150^\circ=180^\circ-30^\circ$ nên
\allowdisplaybreaks
\begin{eqnarray*}
	C&=&2\sin 60^\circ\cdot(-\tan 30^\circ)-\cos 180^\circ\cdot\cot 45^\circ\\
	&=&2\cdot\dfrac{\sqrt{3}}{2}\cdot\left(\dfrac{-\sqrt{3}}{3}\right)-(-1)\cdot 1=0.
\end{eqnarray*}
\end{enumerate}
}
\end{vd}

\baitaptl
\begin{bt}
Tính giá trị của các biểu thức
\begin{enumerate}
	\item $A= \sin 45^\circ+2 \sin 60^\circ+\tan 120^\circ+\cos 135^\circ$;
	\item $B= \tan 45^\circ \cdot \cot 135^\circ-\sin 30^\circ \cdot \cos 120^\circ-\sin 60^\circ \cdot \cos 150^\circ$;
	\item $C=\cos^2 5^\circ+\cos^2 25^\circ+\cos^2 45^\circ+\cos^2 65^\circ+\cos^2 85^\circ$;
	\item $D= \dfrac{12}{1+\tan^273^\circ} -4\tan 75^\circ\cdot\cot 105^\circ+12\sin^2 107^\circ-2 \tan 40^\circ \cdot \cos 60^\circ \cdot \tan 50^\circ$;
	\item $E=4 \tan 32^\circ \cdot \cos 60^\circ \cdot \cot 148^\circ+\dfrac{5 \cot^2 108^\circ}{1+\tan^2 18^\circ}+5\sin^272^\circ$.
\end{enumerate}	
\loigiai{
\begin{enumerate}
	\item 
	\allowdisplaybreaks
	\begin{eqnarray*}
		A&=& \sin 45^\circ+2 \sin 60^\circ+\tan 120^\circ+\cos 135^\circ\\
		&=&\dfrac{\sqrt{2}}{2}+2\cdot\dfrac{\sqrt{3}}{2}-\sqrt{3}-\dfrac{\sqrt{2}}{2}\\
		&=&\sqrt{3}-\sqrt{3}=0.
	\end{eqnarray*}
	\item 
	\allowdisplaybreaks
	\begin{eqnarray*}
		B&=& \tan 45^\circ \cdot \cot 135^\circ-\sin 30^\circ \cdot \cos 120^\circ-\sin 60^\circ \cdot \cos 150^\circ\\
		&=&1\cdot (-1)-\dfrac{1}{2}\cdot\left(\dfrac{-1}{2}\right)-\dfrac{\sqrt{3}}{2}\cdot \left(\dfrac{-\sqrt{3}}{2}\right)\\
		&=&-1+\dfrac{1}{4}+\dfrac{3}{4}=0.
	\end{eqnarray*}
	\item Do $5^\circ=90^\circ-85^\circ$, $25^\circ=90^\circ-65^\circ$ nên 
	\allowdisplaybreaks
	\begin{eqnarray*}
		C&=&\cos^25^\circ+\cos^2 25^\circ+\cos^2 45^\circ+\cos^2 65^\circ+\cos^2 85^\circ\\
		&=&\sin^285^\circ+\cos^285^\circ+\sin^225^\circ+\cos^225^\circ+\cos^245^\circ\\
		&=&1+1+\left(\dfrac{\sqrt{2}}{2}\right)^2=2+\dfrac{1}{2}=\dfrac{5}{2}.
	\end{eqnarray*}
	\item 
	\allowdisplaybreaks
	\begin{eqnarray*}
		D&=& \dfrac{12}{1+\tan^273^\circ} -4\tan 75^\circ\cdot\cot 105^\circ+12\sin^2 107^\circ-2 \tan 40^\circ \cdot \cos 60^\circ \cdot \tan 50^\circ\\
		&=&12\cos^273^\circ-4\tan75^\circ\cdot\cot(180^\circ-75^\circ)+12\sin^2(180^\circ-73^\circ)-2\tan(90^\circ-50)\cos60^\circ\tan 50^\circ\\
		&=&12\cos^273^\circ+4\tan75^\circ\cdot\cot75^\circ+12\sin^273^\circ-2\cot 50^\circ\cdot \tan 50^\circ\cdot \cos 60^\circ\\
		&=&12+4-1=15.
	\end{eqnarray*}
	\item Ta có do $148^\circ=180^\circ-32^\circ$, $108^\circ=180^\circ-72^\circ$ và $18^\circ=90^\circ-72^\circ$ nên
	\allowdisplaybreaks
	\begin{eqnarray*}
		E&=&4 \tan 32^\circ \cdot \cos 60^\circ \cdot \cot 148^\circ+\dfrac{5 \cot^2 108^\circ}{1+\tan^2 18^\circ}+5\sin^272^\circ\\
		&=&-4 \tan 32^\circ \cdot \cos 60^\circ \cdot \cot 32^\circ+5\cot^2108^\circ\cdot\cos^218^\circ+5\sin^272^\circ\\
		&=&-4\cdot\dfrac{1}{2}+5\cot^2108^\circ\cdot\sin^272^\circ+5\sin^272^\circ\\
		&=&-2+5\sin^272^\circ\cdot\left(1+\cot^2108^\circ\right)\\
		&=&-2+5\sin^272^\circ\cdot \dfrac{1}{\sin^2 108^\circ}\\
		&=&-2+5=3.
	\end{eqnarray*}
\end{enumerate}	
}
\end{bt}
\begin{bt}%[0H2K1]
	Tính giá trị các biểu thức sau:
	\begin{tasks}(1)
		\task $A=\sin^2 10^{\circ}+\sin^2 20^{\circ}+\dots+\sin^2 170^{\circ}+\sin^2 180^{\circ}$.
		\task $B=\tan 10^{\circ}.\tan 20^{\circ}\dots\tan 80^{\circ}$.
		\task $C=\cot 20^{\circ}+\cot 40^{\circ}+\dots +\cot 140^{\circ}+\cot160^{\circ}$.
	\end{tasks}
	\loigiai{
		\begin{enumerate}[a)]
			\item Ta có $\sin 10^{\circ}=\sin170^{\circ},\ \sin20^{\circ}=\sin160^{\circ},\dots$, suy ra $C= 2\bigl(\sin^2 10^{\circ}+\sin^2 20^{\circ}+\dots+\sin^2 80^{\circ}\bigr)+\sin^2 90^{\circ}$. Mặt khác ta có $\sin 80^{\circ}=\cos 10^{\circ},\ \sin 70^{\circ}=\cos 20^{\circ},\dots$, có 4 cặp như vậy nên ta tính được $A=5$.
			\item $\tan 10^{\circ}=\cot 80^{\circ}$, $\tan 20^{\circ}=\cot 70^{\circ}$, $\tan 30^{\circ}=\cot 60^{\circ}$, $\tan 40^{\circ}=\cot 50^{\circ}$. Do đó, ta tính được $B=1$.
			\item $\cot20^{\circ}=-\cot160^{\circ},\ \cot40^{\circ}=-\cot140^{\circ},\dots$ nên ta tính được $C=0$.
		\end{enumerate}
	}
\end{bt}
\begin{bt}
	Chứng minh rằng
	\begin{enumerate}
		\item $\sin^4\alpha+\cos^4\alpha=1-2\sin^2\alpha\cdot\cos^2\alpha$;
		\item $\sin^6\alpha+\cos^6\alpha=1-3\sin^2\alpha\cdot\cos^2\alpha$;
		\item $\sqrt{\sin^4\alpha+6\cos^2\alpha+3}+\sqrt{\cos^4\alpha+4\sin^2\alpha}=4$.
	\end{enumerate}
\loigiai{
	\begin{enumerate}
	\item Ta có
	\allowdisplaybreaks
	\begin{eqnarray*}
		\sin^4\alpha+\cos^4\alpha&=&(\sin^2\alpha)^2+(\cos^2\alpha)^2\\
		&=&(\sin^2\alpha)^2+(\cos^2\alpha)^2+2\sin^2\alpha\cdot\cos^2\alpha-2\sin^2\alpha\cdot\cos^2\alpha\\
		&=&\left(\sin^2\alpha+\cos^2\alpha\right)^2-2\sin^2\alpha\cdot\cos^2\alpha\\
		&=&1-2\sin^2\alpha\cdot\cos^2\alpha.
	\end{eqnarray*}
		\item Ta có
		\allowdisplaybreaks
	\begin{eqnarray*}
		\sin^6\alpha+\cos^6\alpha&=&(\sin^2\alpha)^3+(\cos^2\alpha)^3\\
		&=&\left(\sin^2\alpha+\cos^2\alpha\right)\cdot\left(\sin^4\alpha-\sin^2\alpha\cdot\cos^2\alpha+\cos^4\alpha\right)\\
		&=&\left(\sin^2\alpha+\cos^2\alpha\right)^2-3\sin^2\alpha\cdot\cos^2\alpha\\
		&=&1-3\sin^2\alpha\cdot\cos^2\alpha.
	\end{eqnarray*}
	\item \allowdisplaybreaks
	\begin{eqnarray*}
		&&\sqrt{\sin^4\alpha+6\cos^2\alpha+3}+\sqrt{\cos^4\alpha+4\sin^2\alpha}\\
		&=&\sqrt{\sin^4\alpha+6(1-\sin^2\alpha)+3}+\sqrt{\cos^4\alpha+4(1-\cos^2\alpha)}\\
		&=&\sqrt{\sin^4\alpha-6\sin^2\alpha+9}+\sqrt{\cos^4\alpha-4\cos^2\alpha+4}\\
		&=&\sqrt{(3-\sin^2\alpha)^2}+\sqrt{(2-\cos^2\alpha)}\\
		&=&3-\sin^2\alpha+2-\cos^2\alpha=5-(\sin^2\alpha + \cos^2 \alpha)=4.
	\end{eqnarray*}
\end{enumerate}
}
\end{bt}

\begin{bt}%[0H2B1]
	Cho $A, B, C$ là các góc của tam giác. Chứng minh các đẳng thức sau:
	\begin{tasks}(2)
		\task $\sin\left(A+B\right)=\sin C.$
		\task $\cos\left(A+B\right)+\cos C=0.$
		\task $\sin\dfrac{A+B}{2}=\cos\dfrac{C}{2}.$
		\task $\tan\left(A-B+C\right)=-\tan2B.$
	\end{tasks}
	\loigiai{ Do $A, B, C$ là các góc của tam giác nên ta có $A+B+C=180^{\circ}$.
		\begin{enumerate}[a)]
			\item Ta có $A+B+C=180^{\circ}\Leftrightarrow A+B=180^{\circ}-C.$\\
			Từ đó suy ra $\sin\left(A+B\right)=\sin \left(180^{\circ}-C\right)=\sin C.$
			\item Ta có $A+B+C=180^{\circ}\Leftrightarrow A+B=180^{\circ}-C.$\\
			Từ đó suy ra $\cos\left(A+B\right)=\cos \left(180^{\circ}-C\right)=-\cos C \Rightarrow \cos\left(A+B\right)+\cos C=0.$
			\item Ta có $A+B+C=180^{\circ}\Leftrightarrow \dfrac{A+B}{2}=\dfrac{180^{\circ}-C}{2}=90^{\circ}-\dfrac{C}{2}.$\\
			Từ đó suy ra $\sin\dfrac{A+B}{2}=\sin\left(90^{\circ}-\dfrac{C}{2}\right)= \cos\dfrac{C}{2}.$
			\item Ta có $\tan\left(A-B+C\right)=\tan\left(A+B+C-2B\right)=\tan\left(180^{\circ}-2B\right)=-\tan2B.$
		\end{enumerate}
	}
	\end{bt}
\begin{dang}{Tìm các GTLG khi biết một GTLG của góc}
Áp dụng tính chất về dấu của GTLG của một góc và các công thức lượng giác cơ bản.
\end{dang}
\viduminhhoa
\begin{vd}%[0H2B1-2]%[Nguyễn Tiến]%Ví dụ 1.
	\text{}
	\begin{enumerate}
		\item Cho $\sin\alpha=\dfrac{1}{3}$ với $90^\circ<\alpha<180^\circ$. Tính $\cos\alpha$ và $\tan\alpha$.
		\item Cho $\cos\alpha=-\dfrac{2}{3}$ và $\sin\alpha>0$. Tính $\sin\alpha$ và $\cot\alpha$.
		\item Cho $\tan\alpha=-2\sqrt{2}$, tính giá trị lượng giác còn lại.
	\end{enumerate}
	\loigiai{
		\begin{enumerate}
			\item Vì $90^\circ<\alpha<180^\circ$ nên $\cos\alpha<0$, mặt khác $\sin^2\alpha+\cos^2\alpha=1$ suy ra
			$$\cos\alpha=-\sqrt{1-\sin^2\alpha}=-\sqrt{1-\dfrac{1}{9}}=-\dfrac{2\sqrt{2}}{3}.$$
			Do đó $\tan\alpha=\dfrac{\sin\alpha}{\cos\alpha}=\dfrac{\dfrac{1}{3}}{-\dfrac{2\sqrt{2}}{3}}=-\dfrac{1}{2\sqrt{2}}$.
			\item Vì $\sin^2\alpha+\cos^2\alpha=1$ và $\sin\alpha>0$, nên $\sin\alpha=\sqrt{1-\cos^2\alpha}=\sqrt{1-\dfrac{4}{9}}=\dfrac{\sqrt{5}}{3}$.\\
			Ta có $\cot\alpha=\dfrac{\cos\alpha}{\sin\alpha}=\dfrac{-\dfrac{2}{3}}{\dfrac{\sqrt{5}}{3}}=-\dfrac{2}{\sqrt{5}}$.
			\item Vì $\tan\alpha=-2\sqrt{2}<0\Rightarrow\cos\alpha<0$.\\
			Ta có $\tan^2\alpha+1=\dfrac{1}{\cos^2\alpha}$, suy ra $\cos\alpha=-\sqrt{\dfrac{1}{\tan^2+1}}=-\sqrt{\dfrac{1}{8+1}}=-\dfrac{1}{3}$.\\
			Do đó $\tan\alpha=\dfrac{\sin\alpha}{\cos\alpha}\Rightarrow\sin\alpha=\tan\alpha\cdot\cos\alpha=-2\sqrt{2}\cdot\left(-\dfrac{1}{3}\right)=\dfrac{2\sqrt{2}}{3}$.\\
			$\Rightarrow\cot\alpha=\dfrac{\cos\alpha}{\sin\alpha}=\dfrac{-\dfrac{1}{3}}{\dfrac{2\sqrt{2}}{3}}=-\dfrac{1}{2\sqrt{2}}$.
		\end{enumerate}
	}
\end{vd}
\begin{vd}%[0H2B1-3]%[Nguyễn Tiến]%Ví dụ 2.
	\begin{enumerate}
		\item Cho $\cos\alpha=\dfrac{3}{4}$ với $0^\circ<\alpha<90^\circ$. Tính $A=\dfrac{\tan\alpha+3\cot\alpha}{\tan\alpha+\cot\alpha}$.
		\item Cho $\tan\alpha=\sqrt{2}$. Tính $B=\dfrac{\sin\alpha-\cos\alpha}{\sin^3\alpha+3\cos^3\alpha+2\sin\alpha}$.
	\end{enumerate}
	\loigiai{
		\begin{enumerate}
			\item Ta có $A=\dfrac{\tan\alpha+3\dfrac{1}{\tan\alpha}}{\tan\alpha+\dfrac{1}{\tan\alpha}}=\dfrac{\tan^2\alpha+3}{\tan^2\alpha+1}=\dfrac{\dfrac{1}{\cos^2\alpha}+2}{\dfrac{1}{\cos^2\alpha}}=1+2\cos^2\alpha$.\\
			Suy ra $A=1+2\cdot\dfrac{9}{16}=\dfrac{17}{8}$.
			\item Ta có $B=\dfrac{\dfrac{\sin\alpha}{\cos^3\alpha}-\dfrac{\cos\alpha}{\cos^3\alpha}}{\dfrac{\sin^3\alpha}{\cos^3\alpha}+\dfrac{3\cos^3\alpha}{\cos^3\alpha}+\dfrac{2\sin\alpha}{\cos^3\alpha}}=\dfrac{\tan\alpha\left(\tan^2\alpha+1\right)-\left(\tan^2\alpha+1\right)}{\tan^3\alpha+3+2\tan\alpha\left(\tan^2\alpha+1\right)}$.\\
			Suy ra $B=\dfrac{\sqrt{2}(2+1)-(2+1)}{2\sqrt{2}+3+2\sqrt{2}(2+1)}=\dfrac{3(\sqrt{2}-1)}{3+8\sqrt{2}}$.
		\end{enumerate}
	}
\end{vd}
\baitaptl
\begin{bt}
Cho góc $\alpha$, $0^\circ<\alpha<180^\circ$ thỏa mãn $\cos\alpha=\dfrac{-1}{3}$.
\begin{enumerate}
	\item Tính $\tan\alpha$.
	\item Tính giá trị của biểu thức $P=\tan\alpha+2\cot\alpha$.
\end{enumerate}	
\loigiai{
\begin{enumerate}
	\item Do $\cos\alpha=\dfrac{-1}{3}<0$ nên $\alpha$ là góc tù và $\tan\alpha=-\sqrt{\dfrac{1}{\cos^2\alpha}-1}=-2\sqrt{2}$.
	\item Do $\tan\alpha\cot\alpha=1$ và $\tan\alpha=-2\sqrt{2}$ nên $\cot\alpha=\dfrac{-\sqrt{2}}{4}$ và bởi vậy $$P=-2\sqrt{2}+2\cdot\left(\dfrac{-\sqrt{2}}{4}\right)=\dfrac{-5\sqrt{2}}{4}.$$
\end{enumerate}
\textbf{Nhận xét.} Khi tính $\tan\alpha$ từ $\cos\alpha$ nhờ đẳng thức $1+\tan^2\alpha=\dfrac{1}{\cos^2\alpha}$ sai lầm thường gặp của học sinh là mặc định coi $\tan\alpha=\sqrt{\dfrac{1}{\cos^2\alpha}-1}$ mà quên mất $\tan\alpha<0$ khi $\alpha$ là góc tù.
}
\end{bt}
\begin{bt}
Cho góc $\alpha$ thỏa mãn $0^\circ<\alpha<180^\circ$ và $\tan\alpha=2$. Tính giá trị của các biểu thức sau
\begin{enumerate}
	\item $G=2\sin\alpha+\cos\alpha$;
	\item $H=\dfrac{2\sin\alpha+\cos\alpha}{\sin\alpha-\cos\alpha}$.
\end{enumerate}	
\loigiai{
	\begin{enumerate}
		\item Do $\alpha$ thỏa mãn $0^\circ<\alpha<180^\circ$ và $\tan\alpha=2$ nên $\sin\alpha>0$ và $\cos\alpha>0$.\\
		Ta có $\cos\alpha=\sqrt{\dfrac{1}{1+\tan^2\alpha}}=\sqrt{\dfrac{1}{1+4}}=\dfrac{\sqrt{5}}{5}$.\\
		Từ đó $\sin\alpha=\tan\alpha\cdot\cos\alpha=\dfrac{2\sqrt{5}}{5}$.\\
		Vậy $G=2\sin\alpha+\cos\alpha=\dfrac{4\sqrt{5}}{5}+\dfrac{\sqrt{5}}{5}=\sqrt{5}$.
		\item Ta có $H=\dfrac{2\sin\alpha+\cos\alpha}{\sin\alpha-\cos\alpha}=\dfrac{2\tan\alpha+1}{\tan\alpha-1}=5$.
	\end{enumerate}
}
\end{bt}

\begin{bt}
Cho góc $\alpha$ với $90^\circ<\alpha<180^\circ$ thỏa mãn $\sin\alpha=\dfrac{3}{4}$. Tính giá trị của biểu thức $F=\dfrac{\tan\alpha+2\cot\alpha}{\tan\alpha+\cot\alpha}$.
\loigiai{
Do $\alpha\in (90^\circ;180^\circ)$ nên $\cos\alpha<0$.\\
Ta có $\cos\alpha=-\sqrt{1-\sin^2\alpha}=-\sqrt{1-\left(\dfrac{3}{4}\right)^2}=\dfrac{-\sqrt{7}}{4}$.\\
Suy ra $\tan\alpha=\dfrac{\sin\alpha}{\cos\alpha}=\dfrac{-3\sqrt{7}}{7}$ và $\cot\alpha=\dfrac{1}{\tan\alpha}=\dfrac{-\sqrt{7}}{3}$.\\
Vậy $F=\dfrac{\tan\alpha+2\cot\alpha}{\tan\alpha+\cot\alpha}=\dfrac{23}{16}$.
}	
\end{bt}

\begin{bt}
Cho góc $\alpha$ thỏa mãn $0^\circ<\alpha<180^\circ$ và $\tan\alpha=\sqrt{2}$. Tính giá trị của các biểu thức sau $$K=\dfrac{\sin^3\alpha+\sin\alpha\cdot\cos^2\alpha+2\sin^2\alpha\cdot\cos\alpha-4\cos^3\alpha}{\sin\alpha-\cos\alpha}.$$
\loigiai{
Ta có \allowdisplaybreaks
\begin{eqnarray*}
	K&=&\dfrac{\sin^3\alpha+\sin\alpha\cdot\cos^2\alpha+2\sin^2\alpha\cdot\cos\alpha-4\cos^3\alpha}{\sin\alpha-\cos\alpha}\\
	&=&\dfrac{\cos^3\alpha\left(\tan^3\alpha+\tan\alpha+2\tan^2\alpha-4\right)}{\cos^3\alpha\left(\tan\alpha\cdot(1+\tan^2\alpha)-(1+\tan^2\alpha)\right)}\\
	&=&\dfrac{\tan^3\alpha+\tan\alpha+2\tan^2\alpha-4}{(\tan\alpha-1)(1+\tan^2\alpha)}\\
	&=&\dfrac{2\sqrt{2}+\sqrt{2}+2\cdot 2-4}{(\sqrt{2}-1)(1+2)}\\
	&=&\dfrac{\sqrt{2}}{\sqrt{2}-1}=2+\sqrt{2}.
\end{eqnarray*}
}
\end{bt}
\subsection{Câu hỏi trắc nghiệm}
\Opensolutionfile{ans}[ans/ans-0D3-5-TN]
\begin{ex}%[0H2Y1-2]%[Nguyễn Tiến]%Câu 1.
	Giá trị của $\cos 60^\circ+\sin 30^\circ$ bằng bao nhiêu?
	\choice
	{$\dfrac{\sqrt{3}}{2}$}
	{$\sqrt{3}$}
	{$\dfrac{\sqrt{3}}{3}$}
	{\True $1$}
	\loigiai{
		Ta có $\cos 60^\circ+\sin 30^\circ=\dfrac{1}{2}+\dfrac{1}{2}=1$.
	}
\end{ex}
\begin{ex}%[0H2Y1-2]%[Nguyễn Tiến]%Câu 2.
	Giá trị của $\tan 30^\circ+\cot 30^\circ$ bằng bao nhiêu?
	\choice
	{\True $\dfrac{4}{\sqrt{3}}$}
	{$\dfrac{1+\sqrt{3}}{3}$}
	{$\dfrac{2}{\sqrt{3}}$}
	{$2$}
	\loigiai{
		Ta có $\tan 30^\circ+\cot 30^\circ=\dfrac{\sqrt{3}}{3}+\sqrt{3}=\dfrac{4\sqrt{3}}{3}$.
	}
\end{ex}
\begin{ex}%[0H2Y1-2]%[Nguyễn Tiến]%Câu 3.
	Trong các đẳng thức sau đây, đẳng thức nào \textbf{sai}?
	\choice
	{$\sin 0^\circ+\cos 0^\circ=1$}
	{$\sin 90^\circ+\cos 90^\circ=1$}
	{$\sin 180^\circ+\cos 180^\circ=-1$}
	{\True $\sin 60^\circ+\cos 60^\circ=1$}
	\loigiai{
		Ta có $\sin 60^\circ=\dfrac{\sqrt{3}}{2}$, $\cos 60^\circ=\dfrac{1}{2}$ nên đẳng thức sai là ``$\sin 60^\circ+\cos 60^\circ=1$''.
	}
\end{ex}
\begin{ex}%[0H2Y1-2]%[Nguyễn Tiến]%Câu 4.
	Trong các khẳng định sau, khẳng định nào \textbf{sai}?
	\choice
	{$\cos 60^\circ=\sin 30^\circ$}
	{\True $\cos 60^\circ=\sin 120^\circ$}
	{$\cos 30^\circ=\sin 120^\circ$}
	{$\sin 60^\circ=-\cos 120^\circ$}
	\loigiai{
		Ta có cặp góc $60^\circ$, $120^\circ$ bù nhau nên khẳng định sai là ``$\cos 60^\circ=\sin 120^\circ$''.
	}
\end{ex}
\begin{ex}%[0H2Y1-2]%[Nguyễn Tiến]%Câu 5.
	Đẳng thức nào sau đây \textbf{sai}?
	\choice
	{$\sin 45^\circ+\sin 45^\circ=\sqrt{2}$}
	{$\sin 30^\circ+\cos 60^\circ=1$}
	{$\sin 60^\circ+\cos 150^\circ=0$}
	{\True $\sin 120^\circ+\cos 30^\circ=0$}
	\loigiai{
		Ta có $\sin 120^\circ=\cos 30^\circ=\dfrac{\sqrt{3}}{2}$ nên đẳng thức sai là ``$\sin 120^\circ+\cos 30^\circ=0$''.
	}
\end{ex}
\begin{ex}%[0H2Y1-2]%[Nguyễn Tiến]%Câu 6.
	Giá trị $\cos 45^\circ+\sin 45^\circ$ bằng bao nhiêu?
	\choice
	{$1$}
	{\True $\sqrt{2}$}
	{$\sqrt{3}$}
	{$0$}
	\loigiai{
		Ta có $\cos 45^\circ=\sin 45^\circ=\dfrac{\sqrt{2}}{2}$ nên $\cos 45^\circ+\sin 45^\circ=\sqrt{2}$.
	}
\end{ex}
\begin{ex}%[0H2Y1-2]%[Nguyễn Tiến]%Câu 7.
	Trong các đẳng thức sau, đẳng thức nào \textbf{đúng}?
	\choice
	{$\sin\left( 180^\circ-\alpha\right)=-\cos\alpha$}
	{$\sin\left(180^\circ-\alpha\right)=-\sin\alpha$}
	{\True $\sin\left(180^\circ-\alpha\right)=\sin\alpha$}
	{$\sin\left(180^\circ-\alpha\right)=\cos\alpha$}
	\loigiai{
		Theo tính chất của cặp góc bù nhau thì ``$\sin\left(180^\circ-\alpha\right)=\sin\alpha$''.
	}
\end{ex}
\begin{ex}%[0H2Y1-2]%[Nguyễn Tiến]%Câu 8.
	Trong các đẳng thức sau, đẳng thức nào \textbf{sai}?
	\choice
	{\True $\sin 0^\circ+\cos 0^\circ=0$}
	{$\sin 90^\circ+\cos 90^\circ=1$}
	{$\sin 180^\circ+\cos 180^\circ=-1$}
	{$\sin 60^\circ+\cos 60^\circ=\dfrac{\sqrt{3}+1}{2}$}
	\loigiai{
		Ta có $\sin 0^\circ=0$, $\cos 0^\circ=1$ nên đẳng thức sai là ``$\sin 0^\circ+\cos 0^\circ=0$''.
	}
\end{ex}
\begin{ex}%[0H2Y1-2]%[Nguyễn Tiến]%Câu 9.
	Cho $\alpha$ là góc tù. Điều khẳng định nào sau đây là \textbf{đúng}?
	\choice
	{$\sin\alpha<0$}
	{$\cos\alpha>0$}
	{\True $\tan\alpha<0$}
	{$\cot\alpha>0$}
	\loigiai{
		Góc tù có điểm biểu diễn thuộc góc phần tư thứ II, suy ra $\tan\alpha<0$.
	}
\end{ex}
\begin{ex}%[0H2B1-2]%[Nguyễn Tiến]%Câu 10.
	Giá trị của $E=\sin 36^\circ\cos 6^\circ-\sin 126^\circ\cos 84^\circ$ là
	\choice
	{\True $\dfrac{1}{2}$}
	{$\dfrac{\sqrt{3}}{2}$}
	{$1$}
	{$-1$}
	\loigiai{
		Ta có
		\allowdisplaybreaks
		\begin{eqnarray*}
			E&= & \sin 36^\circ\cos 6^\circ-\sin\left(90^\circ+36^\circ\right)\cos\left(90^\circ-6^\circ\right)\\
			&= & \sin 36^\circ\cos 6^\circ-\cos 36^\circ\sin 6^\circ=\sin 30^\circ=\dfrac{1}{2}.
		\end{eqnarray*}
	}
\end{ex}
\begin{ex}%[0H2B1-2]%[Nguyễn Tiến]%Câu 11.
	Giá trị của biểu thức $A=\sin^2 51^\circ+\sin^2 55^\circ+\sin^2 39^\circ+\sin^2 35^\circ$ là
	\choice
	{$3$}
	{$4$}
	{$1$}
	{\True $2$}
	\loigiai{
		Ta có
		\allowdisplaybreaks
		\begin{eqnarray*}
			A&= & \left(\sin^2 51^\circ+\sin^2 39^\circ\right)+\left(\sin^2 55^\circ+\sin^2 35^\circ\right)\\
			&= & \left(\sin^2 51^\circ+\cos^2 51^\circ\right)+\left(\sin^2 55^\circ+\cos^2 55^\circ\right)=2.
		\end{eqnarray*}
	}
\end{ex}
\begin{ex}%[0H2K1-2]%[Nguyễn Tiến]%Câu 12.
	Giá trị của biểu thức $A=\tan 1^\circ\tan 2^\circ\tan 3^\circ\cdots\tan 88^\circ\tan 89^\circ$ là
	\choice
	{$0$}
	{$2$}
	{$3$}
	{\True $1$}
	\loigiai{
		Ta có $A=\left(\tan 1^\circ\cdot\tan 89^\circ\right)\cdot\left(\tan 2^\circ\cdot\tan 88^\circ\right)\cdots\left(\tan 44^\circ\cdot\tan 46^\circ\right)\cdot\tan 45^\circ=1$.
	}
\end{ex}
\begin{ex}%[0H2K1-2]%[Nguyễn Tiến]%Câu 13.
	Tổng $\sin^2 2^\circ+\sin^2 4^\circ+\sin^2 6^\circ+\cdots +\sin^2 84^\circ+\sin^2 86^\circ+\sin^2 88^\circ$ bằng
	\choice
	{$21$}
	{$23$}
	{\True $22$}
	{$24$}
	\loigiai{
		Ta có
		\allowdisplaybreaks
		\begin{eqnarray*}
			S&= & \sin^2 2^\circ+\sin^2 4^\circ+\sin^2 6^\circ+\cdots +\sin^2 84^\circ+\sin^2 86^\circ+\sin^2 88^\circ\\
			&= & \left(\sin^2 2^\circ+\sin^2 88^\circ\right)+\left(\sin^2 4^\circ+\sin^2 86^\circ\right)+\cdots +\left(\sin^2 44^\circ+\sin^2 46^\circ\right)\\
			&= & \left(\sin^2 2^\circ+\cos^2 2^\circ\right)+\left(\sin^2 4^\circ+\cos^2 4^\circ\right)+\cdots +\left(\sin^2 44^\circ+\cos^2 44^\circ\right)=22.
		\end{eqnarray*}
	}
\end{ex}
\begin{ex}%[0H2K1-2]%[Nguyễn Tiến]%Câu 14.
	Giá trị của $A=\tan 5^\circ\cdot\tan 10^\circ\cdot\tan 15^\circ\cdots\tan 80^\circ\cdot\tan 85^\circ$ là
	\choice
	{$2$}
	{\True $1$}
	{$0$}
	{$-1$}
	\loigiai{
		Ta có
		\allowdisplaybreaks
		\begin{eqnarray*}
			A&= & \left(\tan 5^\circ\cdot\tan 85^\circ\right)\cdot\left(\tan 10^\circ\cdot\tan 80^\circ\right)\cdots\left(\tan 40^\circ\tan 50^\circ\right)\cdot\tan 45^\circ\\
			&= & \left(\tan 5^\circ\cdot\cot 5^\circ\right)\cdot\left(\tan 10^\circ\cdot\cot 10^\circ\right)\cdots\left(\tan 40^\circ\cot 40^\circ\right)\cdot\tan 45^\circ =1.
		\end{eqnarray*}
	}
\end{ex}
\begin{ex}%[0H2B1-2]%[Nguyễn Tiến]%Câu 15.
	Giá trị của $B=\cos^2 73^\circ+\cos^2 87^\circ+\cos^2 3^\circ+\cos^2 17^\circ$ là
	\choice
	{$\sqrt{2}$}
	{\True $2$}
	{$-2$}
	{$1$}
	\loigiai{
		Ta có 
		\allowdisplaybreaks
		\begin{eqnarray*}
			B&= & \left(\cos^2 73^\circ+\cos^2 17^\circ\right)+\left(\cos^2 87^\circ+\cos^2 3^\circ\right)\\
			&= & \left(\cos^2 73^\circ+\sin^2 73^\circ\right)+\left(\cos^2 87^\circ+\sin^2 87^\circ\right)=2.
		\end{eqnarray*}
	}
\end{ex}
\begin{ex}%Câu 1.%[Nguyễn Chiến Thắng - TLDH7]%[0H2K1-2]
	Cho $\cos x=\dfrac 12$. Tính biểu thức $P=3\sin^2x+4\cos^2x$ 
	\choice
	{\True $\dfrac{13}{4}$}
	{$\dfrac{7}{4}$}
	{$\dfrac{11}{4}$}
	{$\dfrac{15}{4}$}
	\loigiai{
		Ta có $P=3\sin^2x+4\cos^2x=3\left(\sin^2x+\cos^2x\right)+\cos^2x=3+\left(\dfrac 12\right)^2=\dfrac{13}4$.}
\end{ex}
\begin{ex}%Câu 2.%[Nguyễn Chiến Thắng - TLDH7]%[0H2K1-2]
	Biết $\cos\alpha=\dfrac 13$. Giá trị đúng của biểu thức $P=\sin^2\alpha+3\cos^2\alpha$ là 
	\choice
	{$\dfrac{1}{3}$}
	{$\dfrac{10}{9}$}
	{\True $\dfrac{11}{9}$}
	{$\dfrac{4}{3}$}
	\loigiai{
		Ta có:	$\cos\alpha=\dfrac 13\Rightarrow P=\sin^2\alpha+3cos^2\alpha=\left(\sin^2\alpha+cos^2\alpha\right)+2cos^2\alpha=1+2cos^2\alpha=\dfrac{11}9$.}
\end{ex}
\begin{ex}%Câu 3.%[Nguyễn Chiến Thắng - TLDH7]%[0H2K1-2]
	Cho biết $\tan\alpha=\dfrac{1}{2}$. Tính $\cot\alpha$. 
	\choice
	{\True $\cot\alpha=2$}
	{$\cot\alpha=\sqrt{2}$}
	{$\cot\alpha=\dfrac{1}{4}$}
	{$\cot\alpha=\dfrac{1}{2}$}
	\loigiai{
		Ta có	$\tan\alpha\cdot\cot\alpha=1\Rightarrow\cot\alpha=\dfrac{1}{\tan\alpha}=2$.}
\end{ex}
\begin{ex}%Câu 4.%[Nguyễn Chiến Thắng - TLDH7]%[0H2K1-2]
	Cho biết $\cos\alpha=-\dfrac{2}{3}$ và $0<\alpha<\dfrac{\pi}{2}$. Tính $\tan\alpha$?
	\choice
	{$\dfrac{5}{4}$}
	{$-\dfrac{5}{2}$}
	{$\dfrac{\sqrt{5}}{2}$}
	{\True $-\dfrac{\sqrt{5}}{2}$}
	\loigiai{
		Do $0<\alpha<\dfrac{\pi}{2}\Rightarrow\tan\alpha<0$. \\
		Ta có: $1+\tan^2\alpha=\dfrac 1{\cos^2\alpha}\Leftrightarrow\tan^2\alpha=\dfrac 54\Rightarrow\tan\alpha=-\dfrac{\sqrt 5}2$.}
\end{ex}
\begin{ex}%Câu 5.%[Nguyễn Chiến Thắng - TLDH7]%[0H2K1-2]
	Cho $\alpha$ là góc tù và $\sin\alpha=\dfrac{5}{13}$. Giá trị của biểu thức $3\sin\alpha+2\cos\alpha$ là
	\choice
	{$3$}
	{\True $-\dfrac{9}{13}$}
	{$-3$}
	{$\dfrac{9}{13}$}
	\loigiai{
		Ta có $\cos^2\alpha=1-\sin^2\alpha=\dfrac{144}{169}\Rightarrow\cos\alpha=\pm\dfrac{12}{13}$.\\
		Do $\alpha$ là góc tù nên $\cos\alpha<0$, từ đó $\cos\alpha=-\dfrac{12}{13}$.\\
		Như vậy $3\sin\alpha+2\cos\alpha=3\cdot\dfrac{5}{13}+2\left(-\dfrac{12}{13}\right)=-\dfrac{9}{13}$.}
\end{ex}
\begin{ex}%Câu 6.%[Nguyễn Chiến Thắng - TLDH7]%[0H2K1-2]
	Cho biết $\sin\alpha+\cos\alpha=a$. Giá trị của $\sin\alpha\cdot\cos\alpha$ bằng bao nhiêu?
	\choice
	{$\sin\alpha\cdot\cos\alpha=a^2$}
	{$\sin\alpha\cdot\cos\alpha=2a$}
	{$\sin\alpha\cdot\cos\alpha=\dfrac{1-a^2}{2}$}
	{\True $\sin\alpha\cdot\cos\alpha=\dfrac{a^2-1}{2}$}
	\loigiai{
		$a^2=\left(\sin\alpha+\cos\alpha\right)^2=1+2\sin\alpha\cos\alpha\Rightarrow\sin\alpha\cos\alpha=\dfrac{a^2-1}{2}$.}
\end{ex}
\begin{ex}%Câu 7.%[Nguyễn Chiến Thắng - TLDH7]%[0H2K1-2]
	Cho biết $\cos\alpha=-\dfrac{2}{3}$. Tính giá trị của biểu thức $E=\dfrac{\cot\alpha+3\tan\alpha}{2\cot\alpha+\tan\alpha}$?
	\choice
	{$-\dfrac{19}{13}$}
	{\True $\dfrac{19}{13}$}
	{$\dfrac{25}{13}$}
	{$-\dfrac{25}{13}$}
	\loigiai{
		Ta có	$E=\dfrac{\cot\alpha+3\tan\alpha}{2\cot\alpha+\tan\alpha}=\dfrac{1+3\tan^2\alpha}{2+\tan^2\alpha}=\dfrac{3\left(\tan^2\alpha+1\right)-2}{1+\left(1+\tan^2\alpha\right)}=\dfrac{\dfrac 3{\cos^2\alpha}-2}{\dfrac 1{\cos^2\alpha}+1}=\dfrac{3-2\cos^2\alpha}{1+\cos^2\alpha}=\dfrac{19}{13}$.}
\end{ex}
\begin{ex}%Câu 8.%[Nguyễn Chiến Thắng - TLDH7]%[0H2K1-2]
	Cho biết $\cot\alpha=5$. Tính giá trị của $E=2\cos^2\alpha+5\sin\alpha\cos\alpha+1$?
	\choice
	{$\dfrac{10}{26}$}
	{$\dfrac{100}{26}$}
	{$\dfrac{50}{26}$}
	{\True $\dfrac{101}{26}$}
	\loigiai{
		$E=\sin^2\alpha\left(2\cot^2\alpha+5\cot\alpha+\dfrac{1}{\sin^2\alpha}\right)=\dfrac{1}{1+\cot^2\alpha}\left(3\cot^2\alpha+5\cot\alpha+1\right)=\dfrac{101}{26}$.}
\end{ex}
\begin{ex}%Câu 9.%[Nguyễn Chiến Thắng - TLDH7]%[0H2K1-2]
	Cho $\cot\alpha=\dfrac{1}{3}$. Giá trị của biểu thức $A=\dfrac{3\sin\alpha+4\cos\alpha}{2\sin\alpha-5\cos\alpha}$ là 
	\choice
	{$-\dfrac{15}{13}$}
	{$-13$}
	{$\dfrac{15}{13}$}
	{\True $13$}
	\loigiai{
		Ta có	$A=\dfrac{3\sin\alpha+4\sin\alpha\cdot\cot\alpha}{2\sin\alpha-5\sin\alpha\cdot\cot\alpha}=\dfrac{3+4\cot\alpha}{2-5\cot\alpha}=13$.}
\end{ex}
\begin{ex}%Câu 10.%[Nguyễn Chiến Thắng - TLDH7]%[0H2K1-2]
	Cho biết $\cos\alpha=-\dfrac{2}{3}$. Giá trị của biểu thức $E=\dfrac{\cot\alpha-3\tan\alpha}{2\cot\alpha-\tan\alpha}$ bằng bao nhiêu?
	\choice
	{$-\dfrac{25}{3}$}
	{$-\dfrac{11}{13}$}
	{\True $-\dfrac{11}{3}$}
	{$-\dfrac{25}{13}$}
	\loigiai{
		Ta có	$E=\dfrac{\cot\alpha-3\tan\alpha}{2\cot\alpha-\tan\alpha}=\dfrac{1-3\tan^2\alpha}{2-\tan^2\alpha}=\dfrac{4-3\left(\tan^2\alpha+1\right)}{3-\left(1+\tan^2\alpha\right)}=\dfrac{4-\dfrac{3}{\cos^2\alpha}}{3-\dfrac{1}{\cos^2\alpha}}=\dfrac{4\cos^2\alpha-3}{3\cos^2\alpha-1}=-\dfrac{11}{3}$.}
\end{ex}
\begin{ex}%Câu 11.%[Nguyễn Chiến Thắng - TLDH7]%[0H2K1-2]
	Biết $\sin a+\cos a=\sqrt{2}$. Hỏi giá trị của $\sin^4a+\cos^4a$ bằng bao nhiêu?
	\choice
	{$\dfrac{3}{2}$}
	{\True $\dfrac{1}{2}$}
	{$-1$}
	{$0$}
	\loigiai{
		Ta có: $\sin a+\cos a=\sqrt{2}\Rightarrow 2=\left(\sin a+\cos a\right)^2\Rightarrow\sin a\cdot\cos a=\dfrac{1}{2}$.\\
		$\sin^4a+\cos^4a=\left(\sin^2a+\cos^2a\right)-2\sin^2a\cos^2a=1-2\left(\dfrac{1}{2}\right)^2=\dfrac{1}{2}$.}
\end{ex}
\begin{ex}%Câu 12.%[Nguyễn Chiến Thắng - TLDH7]%[0H2K1-2]
	Cho $\tan\alpha+\cot\alpha=m$. Tìm $m$ để $\tan^2\alpha+\cot^2\alpha=7$. 
	\choice
	{$m=9$}
	{$m=3$}
	{$m=-3$}
	{\True $m=\pm 3$}
	\loigiai{
		Ta có	$7=\tan^2\alpha+\cot^2\alpha=\left(\tan\alpha+\cot\alpha\right)^2-2\Rightarrow m^2=9\Leftrightarrow m=\pm 3$.}
\end{ex}
\begin{ex}%Câu 13.%[Nguyễn Chiến Thắng - TLDH7]%[0H2K1-2]
	Cho biết $3\cos\alpha-\sin\alpha=1$, $0^{\circ}<\alpha<90^{\circ}$ Giá trị của $\tan\alpha$ bằng
	\choice
	{\True $\tan\alpha=\dfrac{4}{3}$}
	{$\tan\alpha=\dfrac{3}{4}$}
	{$\tan\alpha=\dfrac{4}{5}$}
	{$\tan\alpha=\dfrac{5}{4}$}
	\loigiai{
		Ta có $3\cos\alpha-\sin\alpha=1\Leftrightarrow 3\cos\alpha=\sin\alpha+1\to 9\cos^2\alpha=\left(\sin\alpha+1\right)^2$ \\
		$ \Leftrightarrow 9\cos^2\alpha=\sin^2\alpha+2\sin\alpha+1\Leftrightarrow 9\left(1-\sin^2\alpha\right)=\sin^2\alpha+2\sin\alpha+1 $ \\
		$ \Leftrightarrow 10\sin^2\alpha+2\sin\alpha-8=0\Leftrightarrow\hoac{&\sin\alpha=-1\\&\sin\alpha=\dfrac{4}{5}} $.
		\begin{itemize}
			\item 	$\sin\alpha=-1 $: không thỏa mãn vì $0^{\circ}<\alpha<90^{\circ}$.
			\item 	$\sin\alpha=\dfrac{4}{5}\Rightarrow\cos\alpha=\dfrac{3}{5}\Rightarrow\tan\alpha=\dfrac{\sin\alpha}{\cos\alpha}=\dfrac{4}{3}$.
		\end{itemize}
	}
\end{ex}
\begin{ex}%Câu 14.%[Nguyễn Chiến Thắng - TLDH7]%[0H2K1-2]
	Cho biết $2\cos\alpha+\sqrt{2}\sin\alpha=2$, $0^{\circ}<\alpha<90^{\circ}$. Tính giá trị của $\cot\alpha$. 
	\choice
	{$\cot\alpha=\dfrac{\sqrt{5}}{4}$}
	{$\cot\alpha=\dfrac{\sqrt{3}}{4}$}
	{\True $\cot\alpha=\dfrac{\sqrt{2}}{4}$}
	{$\cot\alpha=\dfrac{\sqrt{2}}{2}$}
	\loigiai{
		Ta có $2\cos\alpha+\sqrt{2}\sin\alpha=2\Leftrightarrow\sqrt{2}\sin\alpha=2-2\cos\alpha\to 2\sin^2\alpha=\left(2-2\cos\alpha\right)^2$.\\
		$\begin{aligned}&\Leftrightarrow 2\sin^2\alpha=4-8\cos\alpha+4\cos^2\alpha\Leftrightarrow 2\left(1-\cos^2\alpha\right)=4-8\cos\alpha+4\cos^2\alpha\\&\Leftrightarrow 6\cos^2\alpha-8\cos\alpha+2=0\Leftrightarrow\hoac{&\cos\alpha=1\\&\cos\alpha=\dfrac{1}{3}}.\end{aligned}$ 
		\begin{itemize}
			\item 	$\cos\alpha=1$: không thỏa mãn vì $0^{\circ}<\alpha<90^{\circ}$.
			\item 	$\cos\alpha=\dfrac{1}{3}\Rightarrow\sin\alpha=\dfrac{2\sqrt{2}}{3}\Rightarrow\cot\alpha=\dfrac{\cos\alpha}{\sin\alpha}=\dfrac{\sqrt{2}}{4}$.
		\end{itemize}
	}
\end{ex}
\begin{ex}%Câu 15.%[Nguyễn Chiến Thắng - TLDH7]%[0H2G1-2]
	Cho biết $\cos\alpha+\sin\alpha=\dfrac{1}{3}$. Giá trị của $P=\sqrt{\tan^2\alpha+\cot^2\alpha}$ bằng bao nhiêu?
	\choice
	{$P=\dfrac{5}{4}$}
	{\True $P=\dfrac{7}{4}$}
	{$P=\dfrac{9}{4}$}
	{$P=\dfrac{11}{4}$}
	\loigiai{
		Ta có $\cos\alpha+\sin\alpha=\dfrac{1}{3}\to\left(\cos\alpha+\sin\alpha\right)^2=\dfrac{1}{9}\Leftrightarrow 1+2\sin\alpha\cos\alpha=\dfrac{1}{9}\Leftrightarrow\sin\alpha\cos\alpha=-\dfrac{4}{9}$.\\
		Ta có $P=\sqrt{\tan^2\alpha+\cot^2\alpha}=\sqrt{\left(\tan\alpha+\cot\alpha\right)^2-2\tan\alpha\cot\alpha}=\sqrt{\left(\dfrac{\sin\alpha}{\cos\alpha}+\dfrac{\cos\alpha}{\sin\alpha}\right)^2-2}$.\\
		$=\sqrt{\left(\dfrac{\sin^2\alpha+\cos^2\alpha}{\sin\alpha\cos\alpha}\right)^2-2}=\sqrt{\left(\dfrac{1}{\sin\alpha\cos\alpha}\right)^2-2}=\sqrt{\left(-\dfrac{9}{4}\right)^2-2}=\dfrac{7}{4}$.}
\end{ex}
\begin{ex}%Câu 16.%[Nguyễn Chiến Thắng - TLDH7]%[0H2G1-2]
	Cho biết $\sin\alpha-\cos\alpha=\dfrac{1}{\sqrt{5}}$. Giá trị của $P=\sqrt{\sin^4\alpha+\cos^4\alpha}$ bằng bao nhiêu?
	\choice
	{$P=\dfrac{\sqrt{15}}{5}$}
	{\True $P=\dfrac{\sqrt{17}}{5}$}
	{$P=\dfrac{\sqrt{19}}{5}$}
	{$P=\dfrac{\sqrt{21}}{5}$}
	\loigiai{
		Ta có $\sin\alpha-\cos\alpha=\dfrac{1}{\sqrt{5}}\to\left(\sin\alpha-\cos\alpha\right)^2=\dfrac{1}{5}\Leftrightarrow 1-2\sin\alpha\cos\alpha=\dfrac{1}{5}\Leftrightarrow\sin\alpha\cos\alpha=\dfrac{2}{5}$.\\
		$P=\sqrt{\sin^4\alpha+\cos^4\alpha}=\sqrt{\left(\sin^2\alpha+\cos^2\alpha\right)^2-2\sin^2\alpha\cos^2\alpha} =\sqrt{1-2\left(\sin\alpha cos\alpha\right)^2}=\dfrac{\sqrt{17}}{5}$.}
\end{ex}
\Closesolutionfile{ans}

\def\tenchude{HỆ THỨC LƯỢNG TRONG TAM GIÁC}
\setcounter{section}{1}
\setcounter{dang}{0}
\section{HỆ THỨC LƯỢNG TRONG TAM GIÁC}
\subsection{Tóm tắt lý thuyết}
\subsubsection{Định lý cosin}
\immini{Cho tam giác $ ABC$ có $ BC=a$, $ AC=b$ và $ AB=c$.
	\begin{listEX}
		\item[$\bullet$] $ a^2=b^2+c^2-2bc\cdot \cos A \Rightarrow \cos A=$ \dotfill 
		\item[$\bullet$] $ b^2=c^2+a^2-2ca\cdot \cos B \Rightarrow \cos B=$ \dotfill 
		\item[$\bullet$] $ c^2=a^2+b^2-2ab\cdot \cos C \Rightarrow \cos A=$\dotfill 
	\end{listEX}
}
{\begin{tikzpicture}[scale=0.7,font=\footnotesize,line join=round, line cap=round,>=stealth]
		\tkzDefPoints{0/0/B,1/2/A,4/0/C}
		\tkzDrawPoints[fill=black](A,B,C)
		\tkzDefMidPoint(A,B) \tkzGetPoint{c}
		\tkzDefMidPoint(C,B) \tkzGetPoint{a}
		\tkzDefMidPoint(A,C) \tkzGetPoint{b}
		\tkzDrawPolygon(A,B,C)
		\tkzLabelPoints[above](A) 
		\tkzLabelPoints[below](B,C,a)
		\tkzLabelPoints[left](c)
		\tkzLabelPoints[right](b)
\end{tikzpicture}}
\subsubsection{Định lý sin}
\immini{
	Cho tam giác $ ABC$ có $ BC=a,AC=b$, $ AB=c$ và 
	$ R$ là bán kính đường tròn ngoại tiếp. 
	Ta có 
	$$ \dfrac{a}{\sin A}=\dfrac{b}{\sin B}=\dfrac{c}{\sin C}=2R$$
	\begin{note}
		Ghi nhớ: Tỉ lệ "cạnh chia sin góc đối" thì bằng nhau.
	\end{note}
}
{
	\begin{tikzpicture}[scale=0.6,font=\footnotesize,line join=round, line cap=round,>=stealth]
		\tkzDefPoints{0/0/B,1/3/A,4/0/C}
		\tkzCircumCenter(A,B,C) \tkzGetPoint{I}
		\tkzDrawPoints[fill=black](A,B,C,I)
		\tkzDrawCircle(I,A)
		\tkzDefMidPoint(A,B) \tkzGetPoint{c}
		\tkzDefMidPoint(C,B) \tkzGetPoint{a}
		\tkzDefMidPoint(A,C) \tkzGetPoint{b}
		\tkzDefMidPoint(a,c) \tkzGetPoint{R}
		\tkzDrawPolygon(A,B,C)
		\tkzDrawSegments(I,A I,B I,C)
		\tkzLabelPoints[above](A) 
		\tkzLabelPoints[below](B,C,a,I,R)
		\tkzLabelPoints[left](c)
		\tkzLabelPoints[right](b)
\end{tikzpicture}} 

\subsubsection{Công thức tính diện tích tam giác}
Gọi $S$ là diện tích tam giác $ABC$. Ta có
\begin{itemize}
	\item $S=\dfrac{1}{2}a\cdot h_a=\dfrac{1}{2}b\cdot h_b=\dfrac{1}{2}c\cdot h_c$,
	\item $S=\dfrac{1}{2}bc\sin A=\dfrac{1}{2}ca\sin B=\dfrac{1}{2}ab\sin C$,
	\item $S=\dfrac{abc}{4R}$, $S=p\cdot r$, (đọc thêm)
	\item $S=\sqrt{p(p-a)(p-b)(p-c)}$.
\end{itemize}
Trong đó:
\begin{itemize}
	\item [$\bullet$] $ h_a$, $h_b$, $h_c$ là độ dài đường cao lần lượt tương ứng với các cạnh $ BC$, $CA$, $AB$.
	\item [$\bullet$] $ R$ là bán kính đường tròn ngoại tiếp tam giác.
	\item [$\bullet$] $ r$ là bán kính đường tròn nội tiếp tam giác.
	\item [$\bullet$] $ p=\dfrac{a+b+c}{2}$ là nửa chu vi tam giác.
\end{itemize}

\subsection{Các dạng toán}
\begin{dang}{Áp dụng định lý cos}
	\textbf{Nhận dạng định lý:}
	\begin{itemize}
		\item [$\bullet$] Cho tam giác biết trước độ dài hai cạnh và số đo của một góc.
		\item [$\bullet$] Cho tam giác biết trước độ dài ba cạnh.
	\end{itemize}
\end{dang}
\viduminhhoa
\begin{vd}
	Cho tam giác $ABC$ có $b=5, c=7$ và $\cos A=\dfrac{3}{5}$. Tính cạnh $a$ và cosin các góc còn lại của tam giác đó.
	\loigiai{Ta có:
		\begin{align*}
			&a^2=b^2+c^2-2bc\cos A=25+49-2.5.7.\dfrac{3}{5}=32\Rightarrow a=\sqrt{32}=4\sqrt{2} \\
			&\cos B=\dfrac{c^2+a^2-b^2}{2ca}=\dfrac{32+49-25}{56\sqrt{2}}=\dfrac{\sqrt{2}}{2}\\
			&\cos C=\dfrac{a^2+b^2-c^2}{2ab}=\dfrac{32+25-49}{40\sqrt{2}}=\dfrac{8}{40\sqrt{2}}=\dfrac{\sqrt{2}}{10}.
		\end{align*}
	}
\end{vd}
\begin{vd}
	Cho tam giác $ABC$ có $AC = 10 \textrm{cm}, BC = 16 \textrm{cm}$ và $C=120^\circ$, tính độ dài cạnh $AB$.
	\loigiai{Áp dụng định lý hàm số cosin ta có $AB^2=CA^2+CB^2-2CA.CB\cos C$ ta suy ra $AB=\sqrt{516}\, \textrm{cm}$
	}
\end{vd}
\begin{note}
	Cho tam giác $ ABC$ có $ m_a$, $ m_b$, $ m_c$ lần lượt là các trung tuyến kẻ từ $ A$, $ B$, $ C$. 
	Ta có
	\immini{
		\begin{listEX}
			\item [$\bullet$] $ m_a^2=\dfrac{b^2+c^2}{2}-\dfrac{a^2}{4}$.
			\item [$\bullet$] $ m_b^2=\dfrac{a^2+c^2}{2}-\dfrac{b^2}{4}$.
			\item [$\bullet$] $ m_{c}^2=\dfrac{a^2+b^2}{2}-\dfrac{c^2}{4}$.
		\end{listEX}
	}
	{
		\begin{tikzpicture}[scale=0.7,font=\footnotesize,line join=round, line cap=round,>=stealth]
			\tkzDefPoints{0/0/B,1/3/A,6/0/C}
			\tkzDefMidPoint(A,B) \tkzGetPoint{c}
			\tkzDefMidPoint(C,B) \tkzGetPoint{a}
			\tkzDefMidPoint(A,C) \tkzGetPoint{b}
			\coordinate (m_a) at ($ (A)!0.4!(a)$ );
			\coordinate (m_b) at ($ (B)!0.4!(b)$ );
			\coordinate (m_c) at ($ (C)!0.4!(c)$ );
			\tkzDrawPoints[fill=black](A,B,C,a,b,c)
			\tkzDrawPolygon(A,B,C)
			\tkzDrawSegments(a,A b,B c,C)
			\tkzLabelPoints[above](A) 
			\tkzLabelPoints[below](B,C,a,m_b,m_c)
			\tkzLabelPoints[left](c,m_a)
			\tkzLabelPoints[above right](b)
	\end{tikzpicture}}
\end{note}
\begin{vd}
	Cho tam giác $ABC$ có $AB=4~\mathrm{cm}$, $AC=3 ~\mathrm{cm}$ và $BC=6 ~\mathrm{cm}$. Tính
	độ dài trung tuyến kẻ từ $C$ của tam giác $ABC$. 
	\loigiai{	\immini{Độ dài trung tuyến kẻ từ $C$ của tam giác $ABC$ là
			\\
			$ m_{c}^2=\dfrac{a^2+b^2}{2}-\dfrac{c^2}{4}= \dfrac{6^2 + 3^2}{2} - \dfrac{4^2}{4} =\dfrac{37}{2} \Rightarrow m_c = \dfrac{\sqrt{74}}{2}$.	
		}
		{
			\begin{tikzpicture}[scale=0.7,font=\footnotesize,line join=round, line cap=round,>=stealth]
				\tkzDefPoints{0/0/B,1/3/A,6/0/C}
				\tkzDefMidPoint(A,B) \tkzGetPoint{c}
				\tkzDefMidPoint(C,B) \tkzGetPoint{a}
				\tkzDefMidPoint(A,C) \tkzGetPoint{b}
				\coordinate (m_a) at ($ (A)!0.3!(a)$ );
				\coordinate (m_b) at ($ (B)!0.3!(b)$ );
				\coordinate (m_c) at ($ (C)!0.3!(c)$ );
				\tkzDrawPoints[fill=black](A,B,C,a,b,c)
				\tkzDrawPolygon(A,B,C)
				\tkzDrawSegments(a,A b,B c,C)
				\tkzLabelPoints[above](A) 
				\tkzLabelPoints[below](B,C,a,m_b,m_c)
				\tkzLabelPoints[left](c,m_a)
				\tkzLabelPoints[above right](b)
\end{tikzpicture}}}\end{vd}
\begin{vd}
	Cho tam giác $ABC$ có $BC=3, CA=4$ và $AB=6$. Tính cosin của góc có số đo lớn nhất của tam giác đã cho.
	\loigiai{Do $AB>AC>BC$ nên $C>B>A$.\\
		Áp dụng định lý hàm số cosin ta có $\cos C=-\dfrac{11}{24}.$
	}
\end{vd}
\begin{vd}
	\immini
	{
		Hai chiếc tàu thủy cùng xuất phát từ một vị trí $ A$, đi thẳng theo hai hướng tạo với nhau góc $ 60^\circ$. Tàu $ B$ chạy với tốc độ $ 20$ hải lí một giờ. Tàu $ C$ chạy với tốc độ $ 15$ hải lí một giờ. Hỏi sau hai giờ, hai tàu cách nhau bao nhiêu hải lí?
	}
	{
		\begin{tikzpicture}[line join=round, line cap=round,>=stealth,scale=1]
			\foreach \x in {1,2,3}
			\foreach \y in {0,1,2,3,4,5}
			\draw(0.25+\y,\x-0.25)--(\y,\x-0.25) ;
			\foreach \x in {1,2,3}
			\foreach \y in {0,1,2,3,4,5}
			\draw(-0.25+\y,\x-0.75)--(\y,\x-0.75) ;
			\tkzDefPoints{0/0/A,4/0/B,1.5/2.5/C}
			\tkzDefMidPoint(A,B) \tkzGetPoint{40}
			\tkzDefMidPoint(A,C) \tkzGetPoint{30} 
			\tkzDrawPoints[fill=black](A,B,C)
			\tkzDrawSegments(A,B A,C)
			\fill plot [smooth cycle] coordinates{(3.75,0)(4.25,0)(4.3,0.2)(4.15,0.2)(4.15,0.3)(4.05,0.3)(4.05,0.4)(3.95,0.4)(3.95,0.3)(3.85,0.3)(3.85,0.2)(3.7,0.2)} ;
			\fill plot [smooth cycle] coordinates{(1.25,2.5)(1.75,2.5)(1.8,2.7)(1.65,2.7)(1.65,2.8)(1.55,2.8)(1.55,2.9)(1.45,2.9)(1.45,2.8)(1.35,2.8)(1.35,2.7)(1.2,2.7)} ;
			\tkzLabelPoints[below](A,B,C,40)
			\tkzLabelPoints[left](30)
		\end{tikzpicture}
	}
	\loigiai{
		Sau $ 2$ giờ tàu $ B$ đi được $ 40$ hải lí, tàu $ C$ đi được $ 30$ hải lí. \\
		Vậy tam giác $ ABC$ có $ AB=40$, $AC=30$ và $ \widehat{A}=60^\circ$. \\
		Áp dụng định lí cosine vào tam giác $ ABC$, ta có
		$$ a^2=b^2+c^2-2bc\cdot\cos A=30^2+40^2-2\cdot 30\cdot 40\cdot \cos 60^\circ=1300\Rightarrow a\simeq 36.$$ 
		Vậy sau $ 2$ giờ, hai tàu cách nhau khoảng $ 36$ hải lí.}
\end{vd}
\begin{vd}%[0H2B3-2]
	Tam giác $ABC$ có $AB= c$; $BC=a$; $CA= b.$ Các cạnh $a$, $b$, $c$ liên hệ với nhau bởi đẳng thức $b(b^2 -a^2) = c(a^2 -c^2)$. Tính số đo góc $\widehat{BAC}$.
	\loigiai{
		Theo định lý hàm côsin, ta có $$\cos \widehat{BAC}= \dfrac{AB^2 + AC^2 - BC^2}{2 \cdot AB\cdot AC}= \dfrac{c^2 + b^2 -a^2}{2bc}.$$
		Mà 
		\begin{eqnarray*}
			& & b(b^2 -a^2) = c(a^2 -c^2) \\ 
			& \Leftrightarrow & b^3 -a^2 b = a^2 c -c^3 \\
			& \Leftrightarrow &-a^2 (b+c) + (b+c)(b^2 + c^2 -bc)=0\\
			& \Leftrightarrow & b^2 + c^2 - a^2 =bc.
		\end{eqnarray*} 
		Khi đó $\cos \widehat{BAC} = \dfrac{bc}{2bc}= \dfrac{1}{2}$.\\
		Vậy $\widehat{BAC}= 60^\circ$.
	}
\end{vd}
\baitaptl
\begin{bt}
	Cho tam giác $ABC$ có $\widehat{A} = 60^\circ$, $AB=6$, $AC=8$. Tính $BC$.
	\loigiai{Áp dụng định lý cosine trong tam giác $ABC$ ta có $BC^2=AB^2 +AC^2 -2AB\cdot AC \cdot \cos A=6^2 +8^2 -2\cdot6\cdot 8 \cos 60^\circ = 52 \Rightarrow BC=2\sqrt{13}$.}
\end{bt}
\begin{bt}
	Cho tam giác $ABC$ có các cạnh $BC=6$, $CA=4\sqrt{2}$, $AB=2$. Tính $\cos A$ và góc $\widehat{A}$.
	\loigiai{Áp dụng hệ quả của định lý cosine ta có\\
		$\cos A = \dfrac{AB^2+AC^2 -BC^2}{2AB \cdot AC}= \dfrac{2^2 + \left(4\sqrt{2}\right)^2 -6^2}{2\cdot 2\cdot 4\sqrt{2}} =0 \Leftrightarrow \widehat{A}=90^\circ $.}
\end{bt}
\begin{bt}
	Cho tam giác $ABC$ có $AB=6$ cm; $AC =5$ cm và $\widehat{ACB}=60^\circ$. Tính $BC$.
	\loigiai{ Áp dụng định lý cosine trong tam giác ta có\\ $AB^2 =AC^2 +BC^2 -2AC\cdot BC \cdot \cos \widehat{ACB}$ $\Rightarrow 6^2 =5^2 +BC^2 -2\cdot 5 \cdot BC \cos 60^\circ$\\ $\Leftrightarrow BC^2 -5BC-11=0 \Leftrightarrow BC =\dfrac{5+ \sqrt{69}}{2}$.}
\end{bt}
\begin{bt}
	Tam giác $ABC$ có $b= 6$, $c= 8$ và $m_a= 5$. Tính $a$, $\widehat{A}$.
	\loigiai{Áp dụng công thức đường trung tuyến trong tam giác ta có 
		\\
		$m_a^2 = \dfrac{b^2+c^2}{2} -\dfrac{a^2}{4} \Leftrightarrow 5^2 = \dfrac{6^2+8^2}{2} - \dfrac{a^2}{4} \Leftrightarrow a=10$.
	}
\end{bt}
\begin{bt}%[0H2K3]
	Cho tam giác $ABC$, gọi $l_a$ là độ dài đường phân giác trong kẻ từ đỉnh $A$ của tam giác $ABC$. Chứng minh rằng $l_a=\dfrac{bc\sin A}{(b+c)\sin\frac{A}{2}}$.
	\loigiai{
		\immini{
			Gọi $D$ là chân đường phân giác trong kẻ từ đỉnh $A$ của tam giác $ABC$. Ta có $l_a=AD$. Ta có
			$$\begin{aligned}
				&S_{ABC}=S_{ABD}+S_{ACD}\\
				\Leftrightarrow &\dfrac{1}{2}AB.AC.\sin A=\dfrac{1}{2}AB.AD.\sin \frac{A}{2}+\dfrac{1}{2}AC.AD.\sin \frac{A}{2}\\
				\Leftrightarrow & cb\sin A=l_a(c+b)\sin\frac{A}{2}\\
				\Leftrightarrow &l_a=\dfrac{bc\sin A}{(b+c)\sin\frac{A}{2}}
			\end{aligned}$$
		}{
			% \begin{tikzpicture}
			% 	\clip (-3,-3) rectangle (4,2);
			% 	\tkzDefPoints{0/1/A,-1/-2/B,3/-2/C}
			% 	\draw (A)node[above]{$A$}--(B)node[below]{$B$}--(C)node[below]{$C$}--(A);
			% 	\tkzInCenter(A,B,C) \tkzGetPoint{I}
			% 	\tkzInterLL(B,C)(A,I) \tkzGetPoint{D}
			% 	\tkzDrawBisector(B,A,C)
			% 	\tkzLabelPoints[below](D)
			% 	\tkzMarkAngle[size=0.5](D,A,C)
			% 	\tkzMarkAngle[size=0.6](B,A,D)
			% \end{tikzpicture}
		}
	}
\end{bt}
\begin{bt}
	Hai lực $\overrightarrow{f_1}$ và $\overrightarrow{f_2}$ cho trước cùng tác dụng lên một vật và tạo thành góc nhọn $\left(\overrightarrow{f_1},\overrightarrow{f_2}\right)=\alpha$. Hãy lập công thức tính cường độ của hợp lực $\overrightarrow{s}$.
	\loigiai{
		\centerline{
			\begin{tikzpicture}[>=stealth,scale=1]
				\tkzDefPoints{0/0/A,1/2/B}
				\tkzDefShiftPoint[A](0:4){D}
				\tkzDefShiftPoint[B](0:4){C}
				\tkzDrawSegments[->](A,B A,D A,C)
				\tkzDrawSegments(B,C C,D)
				\tkzLabelPoints[left](B,A)
				\tkzLabelPoints[right](C,D)
				\tkzMarkAngles[mkpos=.2,size=.3](D,A,B A,B,C)
				\tkzLabelAngle[pos=.6](D,A,B){\footnotesize $\alpha$}
				\tkzLabelLine[pos=.5,left](A,B){\footnotesize $\overrightarrow{f_1}$}
				\tkzLabelLine[pos=.5,below](A,D){\footnotesize $\overrightarrow{f_2}$}
				\tkzLabelLine[pos=.5,above left](A,C){\footnotesize $\overrightarrow{s}$}
		\end{tikzpicture}}\\
		Đặt $\overrightarrow{AB}=\overrightarrow{f_1}, \overrightarrow{AD}=\overrightarrow{f_2}$ và vẽ hình bình hành $ABCD$.\\
		Khi đó $\overrightarrow{AC}=\overrightarrow{AB}+\overrightarrow{AD}=\overrightarrow{f_1}+\overrightarrow{f_2}=\overrightarrow{s}$.\\
		Vậy $\left|\overrightarrow{s}\right|=\left|\overrightarrow{AC}\right|=\left|\overrightarrow{f_1}+\overrightarrow{f_2}\right|$.\\
		Theo định lí côsin đối với tam giác $ABC$, ta có:\\
		$AC^2=AB^2+BC^2-2.AB.BC.\cos B$, hay $\left|\overrightarrow{s}\right|^2=\left|\overrightarrow{f_1}\right|^2+\left|\overrightarrow{f_2}\right|^2-2\left|\overrightarrow{f_1}\right|.\left|\overrightarrow{f_2}\right|.\cos(180^\circ-\alpha)$.\\
		Do đó: $\left|\overrightarrow{s}\right|=\sqrt{\left|\overrightarrow{f_1}\right|^2+\left|\overrightarrow{f_2}\right|^2+2\left|\overrightarrow{f_1}\right|.\left|\overrightarrow{f_2}\right|.\cos\alpha}$.
	}
\end{bt}
\begin{dang}{Áp dụng định lý sin}
\textbf{Nhận dạng định lý:}
\begin{itemize}
	\item [$\bullet$] Cho tam giác biết trước độ dài hai cạnh và số đo của một góc.
	\item [$\bullet$] Cho tam giác biết trước độ dài một cạnh và số đo của hai góc.
	\item [$\bullet$] Cho tam giác biết trước độ dài một cạnh, số đo góc đối diện và bán kính đường tròn ngoại tiếp tam giác.
\end{itemize}
\end{dang}
\viduminhhoa
\begin{vd}%[Phạm Tuấn]%[0H2B3-1] 
	Cho tam giác $ABC$ có $\widehat{A}= 120^\circ$ và $BC= 10 \mathrm{~cm}$. Tính bán kính đường tròn ngoại tiếp tam giác $ABC$.
	\loigiai{
		Áp dụng định lí sin ta có $R = \dfrac{BC}{2\sin A} = \dfrac{10}{2\sin 120^\circ} = \dfrac{10\sqrt{3}}{3} \mathrm{~cm}$. 
	}
\end{vd}


\begin{vd}%[Phạm Tuấn]%[0H2B3-1] 
	\immini{
		Cho tam giác $ABC$ có $\widehat{A}= 40^\circ$, $\widehat{B}=55^\circ$ và $AB = 100$. Tính độ dài cạnh $BC$ (làm tròn kết quả đến hàng phần mười).
	}
	{
		\begin{tikzpicture}[scale=1, font=\footnotesize, line join=round, line cap=round,>=stealth]
			\path
			(0,0) coordinate (A)
			(4,0) coordinate (B)
			;
			\coordinate (A') at ($(A)+(40:3)$); 
			\coordinate (B') at ($(B)+(-55:3)$); 
			\coordinate (C) at (intersection of A--A' and B--B');
			\draw (A)--(B)--(C)--(A);
			
			\tkzMarkAngle[arc=ll,size=0.5,mark=0](B,A,C)
			\tkzLabelAngle[pos=1.1](B,A,C) {$40^\circ$}
			\tkzMarkAngle[arc=l,size=0.5,mark=0](C,B,A)
			\tkzLabelAngle[pos=0.8](C,B,A) {$55^\circ$}
			\foreach \x/\g in {A/-120,B/-60,C/90} 
			\fill[black] (\x) circle (1pt)+(\g:3mm) node {$\x$};
		\end{tikzpicture}
	}
	\loigiai{
		Ta có $\widehat{C}= 180^\circ -  \widehat{A} -\widehat{B} = 180^\circ - 40^\circ-55^\circ  =85^\circ$. \\
		Áp dụng định lí sin ta có 
		\[
		\dfrac{AB}{\sin C} = \dfrac{BC}{\sin A} \Rightarrow BC = \dfrac{AB\sin A}{\sin C}  = \dfrac{100\sin 40^{\circ}}{\sin 85^{\circ}} \approx 64{,}5. 
		\]
	}
\end{vd}


\begin{vd}%[Phạm Tuấn]%[0H2B3-1] 
	Cho tam giác $ABC$ có $\dfrac{AB}{2} = \dfrac{BC}{3}$ và $\widehat{A} = 45^\circ$. Tính các góc $B$, $C$  của tam giác đó (làm tròn kết quả đến hàng phần mười).
	\loigiai{
		Áp dụng định lí sin ta có
		\[
		\dfrac{AB}{\sin C}  = \dfrac{BC}{\sin A} \Rightarrow \sin C = \dfrac{AB\sin A}{BC} = \dfrac{2\sin 45^\circ}{3} \Rightarrow \widehat{C} \approx 
		28{,}1^\circ.\]
		Khi đó $\widehat{B}= 180^\circ - \widehat{A}- \widehat{C}= 180^\circ - 45^\circ -28{,}1^\circ=106{,}9^\circ $.
	}
\end{vd}


\begin{vd}%[Phạm Tuấn]%[0H2B3-1] 
	Cho tam giác $ABC$ có $\widehat{A}= 30^\circ$, $\widehat{B}=50^\circ$ và bán kính đường tròn ngoại tiếp bằng $10 \mathrm{~cm}$. Tính độ dài các cạnh của tam giác $ABC$ (làm tròn đến hàng phần mười).
	\loigiai{
		Ta có $\widehat{C}= 180^\circ - \widehat{A} - \widehat{B} =  180^\circ  -30^\circ - 50^\circ= 100^\circ$.  \\
		Áp dụng định lí sin
		\begin{align*}
			&  AB = 2R\sin C = 2\cdot 10 \cdot \sin 100^\circ \approx 19{,}7 \mathrm{~cm}; \\
			& BC = 2R\sin A = 2\cdot 10 \cdot \sin 30^\circ  = 10 \mathrm{~cm};  \\
			&AC = 2R\sin B = 2\cdot 10 \cdot \sin 50^\circ \approx 15{,}3 \mathrm{~cm}.
		\end{align*}
	}
\end{vd}


\begin{vd}%[Phạm Tuấn]%[0H2B3-1] 
	Cho tam giác $ABC$. Chứng minh rằng  $\sin^2 A = \sin B \sin C$ khi và chỉ khi $a^2 = bc$. 
	\loigiai{
		Theo định lí sin ta có  $\sin A = \dfrac{a}{2R}$;   $\sin B = \dfrac{b}{2R} $; $\sin C = \dfrac{c}{2R}$. \\
		Do đó 
		\[
		\sin^2 A = \sin B \sin C  \Leftrightarrow \left (\dfrac{a}{2R}\right )^2 =  \dfrac{b}{2R} \cdot  \dfrac{c}{2R} \Leftrightarrow a^2 = bc. 
		\]
	}
\end{vd}

\begin{vd}%[Phạm Tuấn]%[0H2B3-1] 
	Cho tam giác $ABC$.  Biết $AB= 5 \mathrm{~cm}$,  $BC= 6 \mathrm{~cm}$ và $2\sin A = \sin B + \sin C$. Tính độ dài cạnh $AC$. 
	\loigiai{
		Theo định lí sin ta có  $\sin A = \dfrac{BC}{2R}$;   $\sin B = \dfrac{AC}{2R} $; $\sin C = \dfrac{AB}{2R}$. \\
		Do đó 
		\[
		2\sin  A = \sin B +  \sin C  \Leftrightarrow \dfrac{2BC}{2R} =  \dfrac{AC}{2R} +  \dfrac{AB}{2R} \Leftrightarrow 2BC =AC+AB. 
		\]
		Suy ra $AC= 2BC-AB=  12 -5 =7    \mathrm{~cm}$. 
	}
\end{vd}

\baitaptl

\begin{bt}%[Phạm Tuấn]%[0H2B3-1] 
	Cho tam giác $ABC$ có $\widehat{B}= 70^\circ$ và $AC= 15 \mathrm{~cm}$. Tính bán kính đường tròn ngoại tiếp tam giác $ABC$ (làm tròn kết quả đến hàng phần mười).
	\loigiai{
		Áp dụng định lí sin ta có $R = \dfrac{AC}{2\sin B} = \dfrac{15}{2\sin 70^\circ} \approx 8 \mathrm{~cm}$. 
	}
\end{bt}


\begin{bt}%[Phạm Tuấn]%[0H2B3-1] 
	Cho tam giác $ABC$ có $\widehat{B}= 30^\circ$, $\widehat{C}=65^\circ$ và $BC = 50$. Tính độ dài cạnh $AB$ (làm tròn kết quả đến hàng phần mười).
	\loigiai{
		Ta có $\widehat{A}= 180^\circ -  \widehat{B} -\widehat{C} = 180^\circ - 30^\circ-65^\circ  =75^\circ$. \\
		Áp dụng định lí sin ta có 
		\[
		\dfrac{AB}{\sin C} = \dfrac{BC}{\sin A} \Rightarrow AB = \dfrac{BC\sin C}{\sin A}  = \dfrac{50\sin 65^{\circ}}{\sin 75^{\circ}} \approx 46{,}9. 
		\]
	}
\end{bt}

\begin{bt}%[Phạm Tuấn]%[0H2B3-1] 
	Cho tam giác $ABC$ có $\dfrac{BC}{3} = \dfrac{AC}{5}$ và $\widehat{A} = 30^\circ$. Tính các góc $B$, $C$  của tam giác đó (làm tròn kết quả đến hàng phần mười).
	\loigiai{
		Áp dụng định lí sin ta có
		\[
		\dfrac{AC}{\sin B}  = \dfrac{BC}{\sin A} \Rightarrow \sin B = \dfrac{AC\sin A}{BC} = \dfrac{5\sin 30^\circ}{3} \Rightarrow B \approx 
		56{,}4^\circ.\]
		Khi đó $\widehat{C}= 180^\circ - \widehat{A}- \widehat{B}= 180^\circ - 30^\circ -56{,}4^\circ=93{,}6^\circ $.
	}
\end{bt}

\begin{bt}%[Phạm Tuấn]%[0H2B3-2] 
	Cho tam giác $ABC$ thỏa mãn $a \sin B = c \sin A$.  Chứng minh rằng tam giác $ABC$ cân.
	\loigiai{
		Từ giả thiết suy ra $\dfrac{a}{\sin A} = \dfrac{c}{\sin B}$.  \quad (1)\\
		Áp dụng định lí sin ta có  $\dfrac{a}{\sin A} = \dfrac{b}{\sin B}$. \quad (2) \\
		Từ (1) và (2) suy ra $\dfrac{c}{\sin B} = \dfrac{b}{\sin B} \Rightarrow b=c$. \\
		Vậy tam giác $ABC$ cân. 
	}
\end{bt}

\begin{bt}%[Phạm Tuấn]%[0H2B3-2] 
	Cho tam giác $ABC$ thỏa mãn $\sin^2A = \sin^2B + \sin^2 C$.  Chứng minh rằng tam giác $ABC$ vuông. 
	\loigiai{
		Từ định lí sin suy ra $\sin A = \dfrac{a}{2R}$, $\sin B = \dfrac{b}{2R}$, $\sin C = \dfrac{c}{2R}$. \\
		Khi đó 
		\[
		\sin^2A = \sin^2B + \sin^2 C \Leftrightarrow \left (\dfrac{a}{2R}\right )^2 = \left (\dfrac{b}{2R}\right )^2 +\left (\dfrac{c}{2R}\right )^2 \Leftrightarrow a^2=b^2+c^2.
		\]
		Vậy tam giác $ABC$ vuông tại $A$. 
	}
\end{bt}

\begin{bt}%[Phạm Tuấn]%[0H2K3-2] 
	\immini{
		Cho tam giác $ABC$. Gọi $D$ là điểm thuộc miền trong tam giác $ABC$ sao cho $\widehat{BAD} = \widehat{CBD} = \widehat{ACD} = \varphi$.  Chứng minh rằng 
		\[
		\sin^3 \varphi = \sin (A- \varphi)  \sin (B- \varphi)   \sin (C- \varphi) . 
		\]
	}
	{
		\begin{tikzpicture}[scale=1, font=\footnotesize, line join=round, line cap=round,>=stealth]
			\path
			(2,4) coordinate (A)
			(1,1) coordinate (B)
			(5,1) coordinate (C)
			(2.4,1.76) coordinate (D)
			;
			\draw (A)--(B)--(C)--(A)--(D)  (B)--(D)--(C);
			
			\tkzMarkAngle[arc=l, size=0.7cm,mark=0](B,A,D)
			\tkzLabelAngle[pos=1.1](B,A,D) {$\varphi$}
			\tkzMarkAngle[arc=l, size=0.7cm,mark=0](C,B,D)
			\tkzLabelAngle[pos=1.1](C,B,D) {$\varphi$}
			\tkzMarkAngle[arc=l, size=0.7cm,mark=0](A,C,D)
			\tkzLabelAngle[pos=1.1](A,C,D) {$\varphi$}
			\foreach \x/\g in {A/90,B/-120,C/-60,D/50} 
			\fill[black] (\x) circle (1pt)+(\g:3mm) node {$\x$};
		\end{tikzpicture}
	}
	\loigiai{
		Áp dụng định lí sin cho các tam giác $ABD$, $BCD$ và $ACD$ ta nhận được
		\[
		\heva{&\dfrac{BD}{\sin \varphi} = \dfrac{AD}{\sin (B- \varphi)}\\&  \dfrac{CD}{\sin \varphi} = \dfrac{BD}{\sin (C- \varphi)}\\&\dfrac{AD}{\sin \varphi} = \dfrac{CD}{\sin (A- \varphi)}} \Rightarrow \dfrac{BD}{\sin \varphi} \cdot \dfrac{CD}{\sin \varphi}  \cdot \dfrac{AD}{\sin \varphi} = \dfrac{AD}{\sin (B- \varphi)} \cdot \dfrac{BD}{\sin (C- \varphi)} \cdot \dfrac{CD}{\sin (A- \varphi)}.
		\]
		Rút gọn,  ta suy ra $\sin^3 \varphi = \sin (A- \varphi)  \sin (B- \varphi)   \sin (C- \varphi)$. 
	}
\end{bt}
\begin{dang}{Giải tam giác và ứng dụng}
	Giải tam giác là bài toán tìm độ dài tất cả các cạnh và độ lớn tất cả các góc của tam giác.
\end{dang}
\viduminhhoa
\begin{vd}%[Phạm Tuấn]%[0H2B3-1] 
	\immini{
		Cho tam giác $A B C$ có $BC=40\mathrm{~cm}$, $\widehat{B}=30^{\circ}, \widehat{C}=45^{\circ}$. Tính góc $\widehat{A}$ và độ dài các cạnh $A B$, $A C$  của tam giác đó (làm tròn kết quả đến hàng phần mười).
	}
	{
		\begin{tikzpicture}[scale=1, font=\footnotesize, line join=round, line cap=round,>=stealth]
			\path
			(0,0) coordinate (B)
			(4,0) coordinate (C)
			;
			\coordinate (B') at ($(B)+(30:3)$) ; 
			\coordinate (C') at ($(C)+(-45:3)$) ; 
			\coordinate (A) at (intersection of B--B' and C--C');
			\draw (A)--(B)--(C)--(A) ;
			
			\tkzMarkAngle[arc=ll,size=0.6,mark=0](C,B,A)
			\tkzLabelAngle[pos=.9](C,B,A) {$30^\circ$}
			\tkzMarkAngle[arc=l, size=0.6,mark=0](A,C,B)
			\tkzLabelAngle[pos=0.9](A,C,B) {$45^\circ$}
			\foreach \x/\g in {A/90,B/-120,C/-60} 
			\fill[black] (\x) circle (1pt)+(\g:3mm) node {$\x$};
		\end{tikzpicture}
	}
	\loigiai{
		Ta  có $\widehat{A} = 180^\circ - (\widehat{B}+\widehat{C}) = 180^\circ - (30^{\circ}+45^{\circ}) = 105^{\circ}$. \\
		Áp dụng định lí sin ta có 
		\begin{align*}
			&\dfrac{AB}{\sin C} = \dfrac{BC}{\sin A} \Rightarrow AB = \dfrac{BC\sin C}{\sin A}  = \dfrac{40\sin 45^{\circ}}{105^{\circ}} \approx 29{,}3 \mathrm{~(cm)}; \\
			&\dfrac{AC}{\sin B} = \dfrac{BC}{\sin A} \Rightarrow AC = \dfrac{BC\sin B}{\sin A}  = \dfrac{40\sin 30^{\circ}}{105^{\circ}} \approx 20{,}7 \mathrm{~(cm)}. 
		\end{align*}
	}
\end{vd}

\begin{vd}%[Phạm Tuấn]%[0H2B3-1] 
	Cho tam giác $ABC$ có $AB=25$, $AC=20$, $\widehat{A}=120^{\circ}$. Tính cạnh $BC$ và các góc $B$, $C$  của tam giác đó.  
	\loigiai{
		Áp dụng định lí côsin ta có 
		\[ BC^2=AB^2+AC^2-2AB \cdot AC \cos A =25^2+20^2-2 \cdot 25 \cdot 20 \cos 120^\circ  =  1525 \Rightarrow BC=5\sqrt{61} \approx 39.\]
		Áp dụng định lí sin ta có 
		\begin{align*}
			\dfrac{AC}{\sin B} = \dfrac{BC}{\sin A} \Rightarrow \sin B = \dfrac{AC\sin A}{BC}  = \dfrac{20 \sin 120^{\circ}}{5\sqrt{61}} \Rightarrow B \approx 26{,}3^\circ.  
		\end{align*}
		Khi đó $\widehat{C}= 180^\circ - \widehat{A} - \widehat{B} =  180^\circ  -120^\circ - 26{,}3^\circ= 33{,}7^\circ$.
	}
\end{vd}

\begin{vd}%[Phạm Tuấn]%[0H2B3-4]
	\immini{
		Để đo chiều rộng $AB$ của một khúc sông, người ta chọn điểm $C$.  Sau đó,  đo khoảng cách $BC$, các góc $B$ và $C$. Biết rằng $BC = 200$ m, $\widehat{B} = 107^\circ$,  $\widehat{C} = 28^\circ$. Tìm chiều rộng $AB$ của khúc sông đó (làm tròn đến chữ số thập phân thứ nhất).
	}
	{
		\begin{tikzpicture}[scale=1]
			\coordinate (X) at (1,1);
			\foreach \i in {0,...,6}
			\foreach \j  in {0,...,2}  
			\draw[blue] ($(X)+({\i+0.5},{0.6*(\j)+0.4})$)--($(X)+({\i+0.7},{0.6*(\j)+0.4})$);
			\coordinate [label=above left:$A$](A) at (3,3);
			\coordinate [label=below left:$B$](B) at (3,1);
			\coordinate [label=right:$C$](C) at (5,0.5);
			\draw (X)--(8,1) (8,3)--(1,3) (A)--(B)--(C)--(A);
			\foreach \i in {A,B,C} \draw[fill=black] (\i) circle(1.2pt);
		\end{tikzpicture}
	}
	\loigiai{
		Ta có $\widehat{A}= 180^\circ -  \widehat{B} -\widehat{C} =180^\circ- 107^\circ - 28^\circ=55^\circ$. \\
		Áp dụng định lí sin ta có 
		\[
		\dfrac{AB}{\sin C} = \dfrac{BC}{\sin A} \Rightarrow AB = \dfrac{BC\sin C}{\sin A}  = \dfrac{200\sin 28^{\circ}}{\sin 55^{\circ}} \approx 113{,}6\mathrm{~m}. 
		\]
	}
\end{vd}

\begin{vd}%[Phạm Tuấn]%[0H2B3-4]
	\immini{
		Để đo chiều cao $CH$ của một  tháp truyền hình, người ta chọn hai điểm quan sát $A$, $B$ trên mặt đất (hình vẽ).  Biết $\widehat{CAH} =51^\circ$, $\widehat{CBH} =66^\circ$ và $AB=75 \mathrm{~m}$, tính chiều cao của tháp.
	}
	{
		\begin{tikzpicture}[scale=1, font=\footnotesize, line join=round, line cap=round,>=stealth]
			\path
			(1,0) coordinate (A)
			(2,0) coordinate (B)
			(4,3.5) coordinate (C)
			(4,0) coordinate (H)
			(3.75,0) coordinate (U)
			(4.25,0) coordinate (V)
			;
			\draw[thick] (U)--(C)--(V); 
			\draw (B)--(C)--(A)--(V) ;
			\draw[dashed] (C)--(H) ;  
			\draw  
			($(C)!{1}!(U)$)--($(C)!{0.9}!(V)$)
			--($(C)!{0.8}!(U)$)--($(C)!{0.7}!(V)$)
			--($(C)!{0.6}!(U)$)--($(C)!{0.5}!(V)$)
			--($(C)!{0.4}!(U)$)--($(C)!{0.3}!(V)$)
			--($(C)!{0.2}!(U)$)--($(C)!{0.1}!(V)$)
			;
			
			\tkzMarkAngle[arc=l, size=0.6,mark=0](H,A,C)
			\tkzLabelAngle[pos=1](H,A,C) {$51^{\circ}$}
			\tkzMarkAngle[arc=ll, size=0.6,mark=0](H,B,C)
			\tkzLabelAngle[pos=1](H,B,C) {$66^{\circ}$}
			\foreach \x/\g in {A/-90,B/-90,C/90,H/-90} 
			\fill[black] (\x) circle (1pt)+(\g:3mm) node {$\x$};
		\end{tikzpicture}
	}
	
	\loigiai{
		Ta có $\widehat{ACB}=\widehat{CBH} - \widehat{CAH}=  66^\circ -51^\circ= 15^\circ$. \\
		Áp dụng định lí sin ta có
		\[
		\dfrac{AB}{\sin \widehat{ACB}} = \dfrac{BC}{\sin \widehat{CAH} } \Rightarrow BC = \dfrac{AB\sin \widehat{CAH}}{\sin \widehat{ACB}}  = \dfrac{75\sin 51^\circ}{\sin 15^\circ}.
		\]
		Suy ra $CH =BC\sin \widehat{CBH} = \dfrac{75 \sin 51^\circ \sin 66^\circ}{\sin 15^\circ} \approx 205{,}7\mathrm{~m}.$
	}
\end{vd}

\begin{vd}%[Phạm Tuấn]%[0H2B3-4]
	\immini{
		Trên ngọn đồi có một cái tháp cao $120 \mathrm{~m}$. Đỉnh tháp $B$ và chân tháp $C$ nhìn điểm $A$ ở chân đồi dưới các góc tương ứng bằng $35^{\circ}$ và $60^{\circ}$ so với phương thẳng đứng. Xác định chiều cao $H A$ của ngọn đồi. (Làm tròn đến phần mười)
	}
	{
		\begin{tikzpicture}[scale=1, font=\footnotesize, line join=round, line cap=round, >=stealth]
			\path 
			(0,0) coordinate (X)
			(1.1,2) coordinate (Y)
			(2,2) coordinate (Z)
			(5,0) coordinate (A)
			(1.5,5) coordinate (B)
			(1.5,2) coordinate (C)
			(1.5,0) coordinate (K)
			(1.3,2) coordinate (U)
			(1.8,2) coordinate (V) 
			(5,2) coordinate (H);
			\draw  
			($(B)!{1}!(U)$)--($(B)!{0.9}!(V)$)
			--($(B)!{0.8}!(U)$)--($(B)!{0.7}!(V)$)
			--($(B)!{0.6}!(U)$)--($(B)!{0.5}!(V)$)
			--($(B)!{0.4}!(U)$)--($(B)!{0.3}!(V)$)
			--($(B)!{0.2}!(U)$)--($(B)!{0.1}!(V)$)
			;
			\draw[thick]  (X)--(A)--(Z)--(Y)--(X);
			\draw (A)--(B)  (U)--(B)--(V) (Y)--($(Y)!{1.2}!(H)$)  (X)--($(X)!{1.2}!(A)$);
			\draw[->] (A)--(H) ; 
			\draw ($(A)!0.5!(H)$) node[right]{$h$};
			\draw[dashed](B)--(K) (A)--(C);
			
			\tkzMarkAngle[arc=l, size=0.6,mark=0](K,C,A)
			\tkzLabelAngle[pos=0.9](K,C,A) {$60^{\circ}$}
			\tkzMarkAngle[arc=ll, size=0.7,mark=0](K,B,A)
			\tkzLabelAngle[pos=1.1](V,B,A) {$35^{\circ}$}
			\foreach \x/\g in{A/-90,B/90,C/-130,H/90}
			\fill[black](\x) ($(\x)+(\g:3mm)$)node{$\x$}; 
		\end{tikzpicture}
	}
	\loigiai{
		Ta có $\widehat{BAC} = 60^\circ - 35^\circ =25^\circ $;  $\widehat{ACH} = 90^\circ - 60^\circ =30^\circ $\\
		Áp dụng định lí sin ta có
		\[
		\dfrac{AC}{\sin \widehat{ABC}} = \dfrac{BC}{\sin \widehat{BAC}} \Rightarrow AC =  \dfrac{BC\sin \widehat{ABC}}{\sin \widehat{BAC}}
		= \dfrac{120\sin 35^\circ}{\sin 25^\circ}.
		\]
		Suy ra $AH =AC \sin \widehat{ACH} = \dfrac{120 \sin 35^\circ \sin 30^\circ }{\sin 25^\circ} \approx 81{,}4 \mathrm{~m}$.
	}
\end{vd}
\baitaptl

\begin{bt}%[Phạm Tuấn]%[0H2B3-1]
	Cho tam giác $ABC$ có $AB=8$, $BC=10$, $AC=15$.  Tính $\widehat{A} + 2\widehat{C}$ (làm tròn kết quả đến hàng phần mười).
	\loigiai{
		Áp dụng định lí côsin ta có
		\begin{align*}
			&\cos A = \dfrac{AB^2+AC^2-BC^2}{2 \cdot AB \cdot AC} = \dfrac{8^2+15^2-10^2}{2 \cdot 8 \cdot 15 } = \dfrac{63}{80} \Rightarrow \widehat{A} \approx  38{,}04^\circ. \\
			&\cos C = \dfrac{AC^2+BC^2-AB^2}{2 \cdot AC \cdot BC} = \dfrac{15^2+10^2-8^2}{2 \cdot 15 \cdot 10 } = \dfrac{87}{100} \Rightarrow \widehat{C} \approx  29{,}54^\circ.
		\end{align*}
		Suy ra $\widehat{A} + 2\widehat{C} \approx 97{,}1^\circ$. 
	}
\end{bt}

\begin{bt}%[Phạm Tuấn]%[0H2B3-1]
	Cho tam giác $A B C$ có $AB=15\mathrm{~cm}$, $AC=21\mathrm{~cm}$, $\widehat{A}=30^{\circ}$. Tính cạnh $BC$ và các góc $B$, $C$  của tam giác đó (làm tròn kết quả đến hàng phần mười).
	\loigiai{
		Áp dụng định lí côsin ta có 
		\[ BC^2=AB^2+AC^2-2AB \cdot AC \cos A =15^2+21^2-2 \cdot 15 \cdot 21 \cos 30^\circ   \Rightarrow BC \approx 11 \mathrm{~cm}.\]
		Áp dụng định lí sin ta có 
		\begin{align*}
			\dfrac{AC}{\sin B} = \dfrac{BC}{\sin A} \Rightarrow \sin B = \dfrac{AC\sin A}{BC}  = \dfrac{21 \sin 30^{\circ}}{11} \Rightarrow B \approx 72{,}7^\circ.
		\end{align*}
		Khi đó $\widehat{C}= 180^\circ - \widehat{A}- \widehat{B}= 180^\circ - 30^\circ -72{,}7^\circ =77{,}3^\circ $.
	}
\end{bt}

\begin{bt}%[Phạm Tuấn]%[0H2K3-1] 
	\immini{
		Cho tam giác $A B C$ có $AB=15$, $AC=12$, $\widehat{A}=60^{\circ}$. $M$ là điểm thuộc cạnh $AB$  sao cho $AM=2BM$. Tính cạnh $CM$,  góc $\widehat{BCM}$ và bán kính đường tròn ngoại tiếp tam giác  $BCM$ (làm tròn kết quả đến hàng phần mười).
	}
	{
		\begin{tikzpicture}[scale=1, font=\footnotesize, line join=round, line cap=round,>=stealth]
			\path
			(0,0) coordinate (A)
			(4,0) coordinate (B)
			;
			\coordinate (M) at ($(A)!{2/3}!(B)$) ;
			\coordinate (C) at ($(A)+(60:3)$) ; 
			\draw (A)--(B)--(C)--(A) (C)--(M) ;
			
			\tkzMarkAngle[arc=l, size=0.6,mark=0](B,A,C)
			\tkzLabelAngle[pos=1](B,A,C) {$60^{\circ}$}
			\foreach \x/\g in {A/-120,B/-60,C/90,M/-90} 
			\fill[black] (\x) circle (1pt)+(\g:3mm) node {$\x$};
		\end{tikzpicture}
	}
	\loigiai{
		Ta có $AM=2BM \Rightarrow BM = \dfrac{1}{3}AB=5$ và $AM=\dfrac{2}{3}AB=10$. \\
		Áp dụng định lí côsin ta có 
		\begin{align*}
			&CM^2=AM^2+AC^2-2AM \cdot AC \cos A =10^2+12^2-2 \cdot 10 \cdot 12 \cos 60^\circ  =  124 \Rightarrow CM=\sqrt{124} \approx 11{,}1; \\
			&BC^2=AB^2+AC^2-2AB \cdot AC \cos A =15^2+12^2-2 \cdot 15 \cdot 12 \cos 60^\circ  =  189\Rightarrow BC=\sqrt{189}.
		\end{align*}
		Áp dụng định lí côsin ta có 
		\begin{align*}
			&BM^2=CM^2+CB^2-2CM \cdot CB \cos \widehat{BCM}  \\
			\Leftrightarrow~ & \cos \widehat{BCM} = \dfrac{CM^2+CB^2- BM^2}{2CM \cdot CB}  \\
			\Leftrightarrow~ & \cos \widehat{BCM} = \dfrac{124+189- 5^2}{2\sqrt{124} \cdot \sqrt{189}}  \\
			\Rightarrow~& \widehat{BCM} \approx 19{,}8^\circ. 
		\end{align*}
		Áp đụng định lí sin,  ta nhận được bán kính đường tròn ngoại tiếp tam giác $BCM$ là
		\[
		R = \dfrac{BM}{2\sin \widehat{BCM}} \approx 7{,}4. 
		\]
	}
\end{bt}

\begin{bt}%[Phạm Tuấn]%[0H2B3-4]
	\immini{
		Để đo chiều rộng $AB$ của một khúc sông, người ta chọn điểm $C$,  đo khoảng cách $BC$, các góc $B$ và $C$. Biết rằng $BC = 250$ m, $\widehat{B} = 104^\circ$,  $\widehat{C} = 31^\circ$. Tìm chiều rộng $AB$ của khúc sông đó (làm tròn đến chữ số hàng đơn vị).
	}
	{
		\begin{tikzpicture}[scale=1, font=\footnotesize, line join=round, line cap=round,>=stealth]
			\coordinate (X) at (1,1);
			\foreach \i in {0,...,6}
			\foreach \j  in {0,...,2}  
			\draw[blue] ($(X)+({\i+0.5},{0.6*(\j)+0.4})$)--($(X)+({\i+0.7},{0.6*(\j)+0.4})$);
			\coordinate [label=above:$A$](A) at (3,3);
			\coordinate [label=below:$B$](B) at (3,1);
			\coordinate [label=right:$C$](C) at (5,0.5);
			\draw (X)--(8,1) (8,3)--(1,3) (A)--(B)--(C)--(A);
			\foreach \i in {A,B,C} \draw[fill=black] (\i) circle(1.2pt);
		\end{tikzpicture}
	}
	\loigiai{
		Ta có $\widehat{A}= 180^\circ -  \widehat{B} -\widehat{C} =180^\circ- 104^\circ - 31^\circ=45^\circ$. \\
		Áp dụng định lí sin ta có 
		\[
		\dfrac{AB}{\sin C} = \dfrac{BC}{\sin A} \Rightarrow AB = \dfrac{BC\sin C}{\sin A}  = \dfrac{250\sin 31^{\circ}}{\sin 45^{\circ}} \approx 182 \mathrm{~m}. 
		\]
	}
\end{bt}

\begin{bt}%[Phạm Tuấn]%[0H2B3-4]
	\immini{
		Để đo chiều cao $CH$ của một  tháp truyền hình, người ta chọn hai điểm quan sát $A$, $B$ trên mặt đất (hình vẽ).  Biết $\widehat{CAH} =54^\circ$, $\widehat{CBH} =68^\circ$ và $AB=80 \mathrm{~m}$, tính chiều cao của tháp (Làm tròn đến hàng đơn vị).
	}
	{
		\begin{tikzpicture}[scale=1, font=\footnotesize, line join=round, line cap=round,>=stealth]
			\path
			(1,0) coordinate (A)
			(2,0) coordinate (B)
			(4,3.5) coordinate (C)
			(4,0) coordinate (H)
			(3.75,0) coordinate (U)
			(4.25,0) coordinate (V)
			;
			\draw[thick] (U)--(C)--(V); 
			\draw (B)--(C)--(A)--(V) ;
			\draw[dashed] (C)--(H) ;  
			\draw  
			($(C)!{1}!(U)$)--($(C)!{0.9}!(V)$)
			--($(C)!{0.8}!(U)$)--($(C)!{0.7}!(V)$)
			--($(C)!{0.6}!(U)$)--($(C)!{0.5}!(V)$)
			--($(C)!{0.4}!(U)$)--($(C)!{0.3}!(V)$)
			--($(C)!{0.2}!(U)$)--($(C)!{0.1}!(V)$)
			;
			
			\tkzMarkAngle[arc=l, size=0.6,mark=0](H,A,C)
			\tkzLabelAngle[pos=1](H,A,C) {$54^{\circ}$}
			\tkzMarkAngle[arc=ll, size=0.6,mark=0](H,B,C)
			\tkzLabelAngle[pos=1](H,B,C) {$68^{\circ}$}
			\foreach \x/\g in {A/-90,B/-90,C/90,H/-90} 
			\fill[black] (\x) circle (1pt)+(\g:3mm) node {$\x$};
		\end{tikzpicture}
	}
	
	\loigiai{
		Ta có $\widehat{ACB}=\widehat{CBH} - \widehat{CAH}=  68^\circ -54^\circ= 14^\circ$. \\
		Áp dụng định lí sin ta có
		\[
		\dfrac{AB}{\sin \widehat{ACB}} = \dfrac{BC}{\sin \widehat{CAH} } \Rightarrow BC = \dfrac{AB\sin \widehat{CAH}}{\sin \widehat{ACB}}  = \dfrac{80\sin 54^\circ}{\sin 14^\circ}.
		\]
		Suy ra $CH =BC\sin \widehat{CBH} = \dfrac{80 \sin 54^\circ \sin 68^\circ}{\sin 14^\circ} \approx 248 \mathrm{~m}.$
	}
\end{bt}
\begin{dang}{Bài tập tổng hợp}
	
\end{dang}
\viduminhhoa
\begin{vd}%[Chim Khuyên]%[0H2B3-1] 
	Cho tam giác $ABC$ có $\widehat{A}= 60^\circ$ và $AB= 8 \mathrm{~cm}$, $AC= 5 \mathrm{~cm}$. 
	\begin{enumerate}
		\item Tính diện tích của tam giác $ABC$.
		\item Tính độ dài đường cao hạ từ đỉnh $A$ của tam giác $ABC$.
		\item Tính bán kính đường tròn nội tiếp tam giác $ABC$.
	\end{enumerate}
	\loigiai{
		\begin{center}
			\begin{tikzpicture}[scale=1, font=\footnotesize, line join=round, line cap=round,>=stealth]
				\path
				(0,0) coordinate (B)
				(5,0) coordinate (C)
				(3,2) coordinate (A)
				($(B)!(A)!(C)$) coordinate (H)
				;
				\draw (A)--(B)--(C)--(A)--(H);
				\draw pic["$60^{\circ}$", draw=black, angle eccentricity=1.5, angle radius=0.5cm]{angle=B--A--C} ;
				\foreach \x/\g in {A/90,B/-120,C/-60,H/-60} 
				\fill[black] (\x) circle (1pt)+(\g:3mm) node {$\x$};
			\end{tikzpicture}
		\end{center}
		\begin{enumerate}
			\item Áp dụng công thức tính diện tích tam giác ta có\\
			$S_{\triangle ABC}=\dfrac{1}{2}AB\cdot AC \cdot \sin 60^{\circ} = \dfrac{1}{2} \cdot 8 \cdot 5 \cdot \dfrac{\sqrt{3}}{2} = 10\sqrt{3} \mathrm{~cm}^2$. 
			\item Áp dụng định lí cosin trong tam giác $ABC$ ta có \\
			$BC^2= AB^2+AC^2-2 AB\cdot AC \cdot \cos 60^{\circ}= 8^2+5^2-2 \cdot 8\cdot 5 \cdot \dfrac{1}{2} =49 \Rightarrow BC=7$.\\
			$S_{\triangle ABC}=\dfrac{1}{2} AH \cdot BC \Rightarrow AH=\dfrac{2S_{\triangle ABC}}{BC}=\dfrac{2\cdot 10 \sqrt{3}}{7}=\dfrac{20\sqrt{3}}{7} $.
			\item $S_{\triangle  ABC}=pr \Rightarrow r=\dfrac{S_{\triangle  ABC}}{p}=\dfrac{2S_{\triangle  ABC}}{AB+BC+AC}=\dfrac{2 \cdot 10\sqrt{3}}{5+8+7}=\sqrt{3}$.
		\end{enumerate}
	}
\end{vd}
\begin{vd}%[Chim Khuyên]%[0H2B3-1] 
	Cho hình bình hành $ABCD$ có $AB=6, BC=8$ và $\widehat{ABC}=60^{\circ}$. Tính diện tích hình bình hành $ABCD$.
	\loigiai{
		\immini{Ta có \\
			$S_{ABCD}=2S_{\triangle ABC}=2\cdot \dfrac{1}{2} BA \cdot BC \cdot \cos \widehat{ABC} $\\
			$= 6\cdot 8 \cdot \cos 60^{\circ} =24$}{	\begin{tikzpicture}[line cap=round,line join=round, >=stealth,scale=.7]
				\tkzDefPoints{0/0/O,-2/1/A} 
				\coordinate (D) at ($(A)+(6,0)$);
				\tkzDefPointBy[homothety = center O ratio -1](A) \tkzGetPoint{C}   
				\tkzDefPointBy[homothety = center O ratio -1](D) \tkzGetPoint{B}  
				\tkzDrawSegments(A,B B,C C,D D,A) 
				\tkzLabelPoints[left](A,B) 
				\tkzLabelPoints[right](C,D)  
		\end{tikzpicture}}
	}
\end{vd}

\begin{vd}%[Chim Khuyên]%[0H2K3-1] 
	Cho tam giác $ABC$ có $\widehat{A}=120^\circ$, $\widehat{B}=30^\circ$, diện tích tam giác $ABC$ bằng $9\sqrt{3}$. Tính các cạnh của tam giác $ABC$.
	\loigiai
	{
		\immini
		{
			Ta có $\widehat{C}=180^\circ-(\widehat{A}+\widehat{B})=30^\circ$.\\
			Khi đó
			\allowdisplaybreaks
			\begin{eqnarray*}
				&&\heva{& \dfrac{BC}{\sin 120^\circ}=\dfrac{AC}{\sin 30^\circ}=\dfrac{AB}{\sin 30^\circ}\\ & S_{\triangle ABC}=\dfrac{1}{2}\cdot BC\cdot AC\cdot \sin 30^\circ=9\sqrt{3}}\\ &\Leftrightarrow&\heva{& BC=\sqrt{3}AC \\ &BC\cdot AC=36\sqrt{3}\\ &AC=AB} \Leftrightarrow \heva{& BC=6\sqrt{3}  \\ & AC=6\\ &AB=6.}
			\end{eqnarray*}
			Vậy $BC=6\sqrt{3}$, $AC=6$, $AB=6$.
		}
		{
			\begin{tikzpicture}[>=stealth,line join=round,line cap=round,line width=0.6pt,font=\footnotesize,scale=1]
				\coordinate[label=below left:$A$](A) at (0,0);
				\coordinate[label=below right:$B$](B) at (3,0);
				\coordinate [label=above right:$C$](C) at ($(A)!1!120:(B)$); % Quay hướng 60 độ và vị tự tỉ số 1 điểm B tâm A thành điểm B1
				\draw (A)--(B)--(C)--cycle;
				\foreach \x in {C,A,B}\draw[->] pic[draw,blue,angle radius=3mm] {angle = B--A--C};
				\foreach \x in {A,B,C} \fill (\x) circle (1.5pt) ;
				\draw (100:0.7)node[below right]{$120^\circ$};
			\end{tikzpicture}
		}
	}
\end{vd}
\begin{vd}%[Chim Khuyên]%[0H2K3-1]
	Cho tam giác $ABC$ có $AB=2$, $AC=2\sqrt{7}$ và $BC=4$.
	\begin{enumerate}
		\item Tính góc $B$ và diện tích tam giác $ABC$.
		\item Tính độ dài đường phân giác trong của góc $B$ của tam giác $ABC$.
	\end{enumerate}
	\loigiai
	{
		\immini
		{
			\begin{enumerate}
				\item Ta có $\cos B=\dfrac{AB^2+BC^2-AC^2}{2\cdot AB \cdot BC}= \dfrac{4+16-28}{2\cdot 2 \cdot 4}=\dfrac{-1}{2} \Rightarrow \widehat{B}= 120^\circ$.\\
				Và $S_{\triangle ABC}=\dfrac{1}{2}AB\cdot BC\cdot \sin 120^\circ=2\sqrt{3}$.
				\item Gọi $D$ là chân đường phân giác trong của góc $B$.\\
				Ta có
				\allowdisplaybreaks
				\begin{eqnarray*}
					&&S_{\triangle ABC} = S_{\triangle ABD} + S_{\triangle BCD}\\ &\Leftrightarrow& 2\sqrt{3}=\dfrac{1}{2} AB \cdot BD \cdot \sin \widehat{ABD} +\dfrac{1}{2} CB \cdot BD \cdot \sin \widehat{CBD}\\
					&\Leftrightarrow& 2\sqrt{3}=\dfrac{1}{2} \cdot 2 \cdot BD \cdot \sin 60^\circ +\dfrac{1}{2}\cdot 4 \cdot BD \cdot \sin 60^\circ\\
					&\Leftrightarrow & 2\sqrt{3} = \dfrac{3\sqrt{3}}{2}BD\Leftrightarrow BD= \dfrac{4}{3}.
				\end{eqnarray*}
			\end{enumerate}
		}
		{
			\begin{tikzpicture}[>=stealth,line join=round,line cap=round,line width=0.6pt,font=\footnotesize]
				\coordinate[label=below left:$B$](B) at (0,0);
				\coordinate[label=below right:$C$](C) at (4,0);
				\coordinate [label=above right:$A$](A) at ($(B)!0.5!120:(C)$); 
				\coordinate [label=above  right:$D$](D) at ($(B)!0.333!60:(C)$);
				\draw (A)--(B)--(C)--cycle (B)--(D);
				\foreach \x in {A,B,C,D} \fill (\x) circle (1.5pt) ;
				\draw pic[draw,blue,angle radius=4mm] {angle = D--B--A} ($($(B)!4mm!(A)$)!.5!($(B)!4mm!(D)$)$)node[rotate=180]{\scriptsize $|$}; 
				\draw pic[draw,blue,angle radius=4mm] {angle = C--B--D} ($($(B)!4mm!(C)$)!.5!($(B)!4mm!(D)$)$)node[rotate=120]{\scriptsize $|$}; 
				
			\end{tikzpicture}
		}
	}
\end{vd}

\baitaptl

\begin{bt}%[Chim Khuyên]%[0H2B2-8]
	Cho tam giác với ba cạnh $ a=13,b=14,c=15$. Tính diện tích của tam giác và độ dài đường cao $ h_c$.
	\loigiai{
		Ta có $ S=\sqrt{p(p-a)(p-b)(p-c)}=84$
		Lại có 
		$ S=\dfrac{1}{2}h_c \cdot 15 \Rightarrow h_c=11\dfrac{1}{5}. $	
	}
	
\end{bt}

\begin{bt}%[Chim Khuyên]%[0H2K3-1]
	Cho tam giác $ABC$ có $AB=10$, $BC=6$ và góc $\widehat{B}=120^\circ$.
	\begin{enumerate}
		\item Tính $AC$ và diện tích tam giác $ABC$.
		\item Tính đường cao $AH$ và bán kính đường tròn nội tiếp tam giác $ABC$.
		\item Tính độ dài đường phân giác trong $BD$ của tam giác $ABC$.
	\end{enumerate}
	
	\loigiai{
		\begin{enumerate}
			\item Ta có $AC= \sqrt{AB^2+BC^2-2AB\cdot BC \cdot \cos B}=14$
			và $S_{\triangle ABC}=\dfrac{1}{2}\cdot AB \cdot BC \cdot \sin B=15\sqrt{3}$.
			\item Ta có 
			\immini
			{
				$AH=\dfrac{2S_{\triangle ABC}}{BC}=\dfrac{2\cdot 15\sqrt{3}}{14}$\\ và $r=\dfrac{S_{\triangle ABC}}{p}=\dfrac{15\sqrt{3}}{15}=\sqrt{3}$ với $p=\dfrac{6+10+14}{2}=15$.
			}
			{
				\begin{tikzpicture}[>=stealth,line join=round,line cap=round,line width=0.6pt,font=\footnotesize]
					\coordinate[label=below left:$B$](B) at (0,0);
					\coordinate[label=below right:$C$](C) at (3,0);
					\coordinate [label=above right:$A$](A) at ($(B)!1.2!120:(C)$); 
					\coordinate [label=above  right:$D$](D) at ($(B)!0.545!60:(C)$);
					\draw (A)--(B)--(C)--cycle (B)--(D);
					\foreach \x in {A,B,C,D} \fill (\x) circle (1.5pt) ;
					\draw pic[draw,blue,angle radius=4mm] {angle = D--B--A} ($($(B)!4mm!(A)$)!.5!($(B)!4mm!(D)$)$)node[rotate=180]{\scriptsize $|$}; 
					\draw pic[draw,blue,angle radius=4mm] {angle = C--B--D} ($($(B)!4mm!(C)$)!.5!($(B)!4mm!(D)$)$)node[rotate=120]{\scriptsize $|$}; 
					
				\end{tikzpicture}
			}
			\item  Ta có
			\allowdisplaybreaks
			\begin{eqnarray*}
				&&S_{\triangle ABC} = S_{\triangle ABD} + S_{\triangle BCD}\\ &\Leftrightarrow& 15\sqrt{3}=\dfrac{1}{2} AB \cdot BD \cdot \sin \widehat{ABD} +\dfrac{1}{2} CB \cdot BD \cdot \sin \widehat{CBD}\\
				&\Leftrightarrow& 15\sqrt{3}=\dfrac{1}{2}\cdot 10 \cdot BD \cdot \sin 60^\circ +\dfrac{1}{2} \cdot 6 \cdot BD \cdot \sin 60^\circ\\
				&\Leftrightarrow&15\sqrt{3}=4\sqrt{3}\cdot BD\Leftrightarrow BD= \dfrac{15}{4}.
			\end{eqnarray*}
	\end{enumerate}	}
\end{bt}

\begin{bt}%[ Chim Khuyên]%[0H2K3-1]
	Cho tam giác $ABC$ có $AB=2$, $AC=3$ và $\widehat{BAC}=120^\circ$. Tính độ dài $BC$, diện tích tam giác $ABC$, độ dài đường phân giác trong $AD$ của tam giác $ABC$.
	\loigiai
	{
		\immini
		{
			Ta có 
			\begin{itemize}
				\item[$\bullet$] $BC= \sqrt{AB^2+AC^2-2AB\cdot AC \cdot \cos A}=\sqrt{19}$\\ và $S_{\triangle ABC}=\dfrac{1}{2}AB\cdot AC \sin A=\dfrac{3\sqrt{3}}{2}$.
				\item  [$\bullet$] Và 
				\allowdisplaybreaks
				\begin{eqnarray*}
					&&S_{\triangle ABC} = S_{\triangle BAD} + S_{\triangle DAC}\\ &\Leftrightarrow& \dfrac{ 3\sqrt{3}}{2}=\dfrac{1}{2} AB \cdot AD \cdot \sin \widehat{BAD} +\dfrac{1}{2} AC \cdot AD \cdot \sin \widehat{DAC}\\
					&\Leftrightarrow& \dfrac{ 3\sqrt{3}}{2}=\dfrac{1}{2} \cdot 2 \cdot AD \cdot \sin 60^\circ +\dfrac{1}{2}\cdot 3 \cdot AD \cdot \sin 60^\circ\\
					&\Leftrightarrow & \dfrac{ 3\sqrt{3}}{2} = \dfrac{5\sqrt{3}}{4}AD\Leftrightarrow AD= \dfrac{6}{5}.
				\end{eqnarray*}
			\end{itemize}
			
		}
		{
			\begin{tikzpicture}[>=stealth,line join=round,line cap=round,line width=0.6pt,font=\footnotesize,scale=1.3]
				\coordinate[label=below left:$A$](A) at (0,0);
				\coordinate[label=below right:$C$](C) at (3,0);
				\coordinate [label=above right:$B$](B) at ($(A)!0.5!120:(C)$); 
				\coordinate [label=above  right:$D$](D) at ($(A)!0.333!60:(C)$);
				\draw (A)--(B)--(C)--cycle (A)--(D);
				\foreach \x in {A,B,C,D} \fill (\x) circle (1.5pt) ;
				\draw pic[draw,blue,angle radius=4mm] {angle = D--A--B} ($($(A)!3mm!(B)$)!.5!($(A)!3mm!(D)$)$)node[rotate=180]{\scriptsize $|$}; 
				\draw pic[draw,blue,angle radius=4mm] {angle = C--A--D} ($($(A)!3mm!(C)$)!.5!($(A)!3mm!(D)$)$)node[rotate=120]{\scriptsize $|$}; 
				
			\end{tikzpicture}
		}
	}
\end{bt}

\begin{bt}%[Chim Khuyên]%[0H2K3-2]
	Cho tam giác $ABC$ có $AB=c$, $BC=a$, $AC=b$. Gọi $h_a$, $h_b$, $h_c$ lần lượt là các đường cao tương ứng xuất phát từ các đỉnh $A$, $B$, $C$ và $r$ là bán kính đường tròn nội tiếp tam giác $ABC$. Chứng minh $\dfrac{1}{h_a}+\dfrac{1}{h_b}+\dfrac{1}{h_c}=\dfrac{1}{r}$.
	\loigiai{
		Ta có $S=\dfrac{1}{2}ah_a=\dfrac{1}{2}bh_b=\dfrac{1}{2}ch_c \Rightarrow \dfrac{1}{h_a}=\dfrac{a}{2S},\dfrac{1}{h_b}=\dfrac{b}{2S}, \dfrac{1}{h_c}=\dfrac{c}{2S}$ và $S=pr \Rightarrow \dfrac{1}{r}=\dfrac{p}{S}$.
		{\allowdisplaybreaks
			\begin{eqnarray*}
				VT=\dfrac{1}{h_a}+\dfrac{1}{h_b}+\dfrac{1}{h_c}&=&\dfrac{a}{2S}+\dfrac{b}{2S}+\dfrac{c}{2S}\\
				&=&\dfrac{a+b+c}{2S}=\dfrac{2p}{2S}\\
				&=&\dfrac{p}{S}=\dfrac{1}{r}.
		\end{eqnarray*}}
	}
\end{bt}

\begin{bt}%[Chim Khuyên]%[0H2K2-2]
	Cho tam giác $ABC$ không vuông ở $A$, chứng minh $S=\dfrac{1}{4}\left(b^2+c^2-a^2\right)\tan A$.
	\loigiai{
		Ta có 
		{\allowdisplaybreaks
			\begin{eqnarray*}
				S&=&\dfrac{1}{2}bc\sin A\\
				&=&\dfrac{1}{2}bc\cos A \cdot \dfrac{\sin A}{\cos A}\\
				&=&\dfrac{1}{2}bc\cdot \dfrac{b^2+c^2-a^2}{2bc} \cdot \tan A\\
				&=&\dfrac{1}{4}\left(b^2+c^2-a^2\right) \cdot \tan A.	
		\end{eqnarray*}}	
	}
\end{bt}
\subsection{Câu hỏi trắc nghiệm}
	\Opensolutionfile{ansbook}[ans/ansbook-0D3-6-TN]
	\Opensolutionfile{ans}[ans/ans-0D3-6-TN]
	\begin{ex}%[0H2Y3-1]
		Tam giác $ ABC$ có $ AB=5$, $BC=7$, $CA=8$. Số đo góc $ \widehat{A}$ bằng
		\choice
		{$ 90^\circ $}
		{$ 45^\circ $}
		{\True $ 60^\circ $}
		{$ 30^\circ $}
		\loigiai
		{Theo định lí hàm cosine, ta có $ \cos{A}=\dfrac{AB^2+AC^2-BC^2}{2AB\cdot AC}=\dfrac{5^2+8^2-7^2}{2\cdot 5\cdot 8}=\dfrac{1}{2}$.\\
			Do đó, $ \widehat{A}=60^\circ $.}
	\end{ex}
	\begin{ex}%[0H2Y3-1]
		Tam giác $ ABC$ có $ AB=\sqrt{2}$, $AC=\sqrt{3}$ và $ \widehat{C}=45^\circ $. Tính độ dài cạnh $ BC$.
		\choice
		{$ BC=\sqrt{5}$}
		{\True $ BC=\dfrac{\sqrt{6}+\sqrt{2}}{2}$}
		{$ BC=\sqrt{6}$}
		{$ BC=\dfrac{\sqrt{6}-\sqrt{2}}{2}$}
		\loigiai
		{Theo định lí hàm cosine, ta có\\
			$ AB^2=AC^2+BC^2-2\cdot AC\cdot BC\cdot \cos \widehat{C}\Rightarrow {(\sqrt{2} )}^2={(\sqrt{3} )}^2+BC^2-2\cdot \sqrt{3}\cdot BC\cdot \cos 45^\circ $ \\
			$ \Rightarrow BC=\dfrac{\sqrt{6}+\sqrt{2}}{2}$.}
	\end{ex}
	\begin{ex}%[0H2Y3-1]
		Tam giác $ ABC$ có $ AB=2$, $AC=1$ và $ \widehat{A}=60^\circ $. Tính độ dài cạnh $ BC$.
		\choice
		{$ BC=\sqrt{2}$}
		{\True $ BC=\sqrt{3}$}
		{$ BC=1$}
		{$ BC=2$}
		\loigiai
		{Theo định lí hàm cosine, ta có\\
			$ BC^2=AB^2+AC^2-2AB\cdot AC\cdot \cos{A}=2^2+1^2-2\cdot 2\cdot 1\cdot \cos 60^\circ =3$\\
			$\Rightarrow BC=\sqrt{3}$.}
	\end{ex}
	\begin{ex}%[0H2B3-1]
		Tam giác $ ABC$ có $ AB=3$, $AC=6$, $\widehat{BAC}=60^\circ $. Tính độ dài đường cao $ h_a$ của tam giác.
		\choice
		{$ h_a=3\sqrt{3}$}
		{$ h_a=\sqrt{3}$}
		{$ h_a=\dfrac{3}{2}$}
		{\True $ h_a=3$}
		\loigiai
		{Áp dụng định lý hàm số cosine, ta có
			$ BC^2=AB^2+AC^2-2AB\cdot AC\cos A=27\Rightarrow BC=3\sqrt{3}$.\\
			Ta có $ S_{\Delta ABC}=\dfrac{1}{2}\cdot AB\cdot AC\cdot \sin{A}=\dfrac{1}{2}\cdot 3\cdot 6\cdot \sin 60^\circ=\dfrac{9\sqrt{3}}{2}$.\\
			Lại có $ S_{\Delta ABC}=\dfrac{1}{2}\cdot BC\cdot h_a\Rightarrow h_a=\dfrac{2S}{BC}=3$.}
	\end{ex}
	\begin{ex}%[0H2B3-1]
		Tam giác $ ABC$ có $ AB=\dfrac{\sqrt{6}-\sqrt{2}}{2}$, $BC=\sqrt{3}$, $CA=\sqrt{2}$. Gọi $ D$ là chân đường phân giác trong góc $ \widehat{A}$. Khi đó góc $ \widehat{ADB}$ bằng
		\choice
		{$ 90^\circ $}
		{$ 45^\circ $}
		{$ 60^\circ $}
		{\True $ 75^\circ $}
		\loigiai{
			\immini
			{
				Theo định lí hàm cosine, ta có\\
				$ \begin{aligned}
					& \cos \widehat{BAC}=\dfrac{AB^2+AC^2-BC^2}{2\cdot AB\cdot AC}=-\dfrac{1}{2}. \\
					\Rightarrow& \widehat{BAC}=120^\circ \Rightarrow \widehat{BAD}=60^\circ. \\
				\end{aligned}$ \\
				$ \begin{aligned}
					& \cos \widehat{ABC}=\dfrac{AB^2+BC^2-AC^2}{2\cdot AB\cdot BC}=\dfrac{\sqrt{2}}{2}.\\
					\Rightarrow &\widehat{ABC}=45^\circ.
				\end{aligned}$ \\
				Trong $ \Delta ABD$ có $ \widehat{BAD}=60^\circ$, $\widehat{ABD}=45^\circ. \\
				\Rightarrow \widehat{ADB}=75^\circ $.
			}
			{
				\begin{tikzpicture}[scale=1, font=\footnotesize, line join = round, line cap = round,>=stealth]
					\tkzDefPoints{-2/0/B,0/3/A,5/0/C}
					\clip (-2.2,-0.5) rectangle (5.2,3.5);
					\tkzInCenter(A,B,C) \tkzGetPoint{I}
					\tkzInterLL(A,I)(C,B) \tkzGetPoint{D}
					\tkzDrawPoints[fill=black](A,B,C,D)
					\tkzDrawPolygon(A,B,C)
					\tkzDrawSegments(D,A)
					\tkzLabelPoints[above](A)
					\tkzLabelPoints[below](D,B,C)
					\tkzMarkAngles[size=0.5cm,arc=l,mark=||](B,A,D D,A,C)
				\end{tikzpicture}
			}
		}
	\end{ex}
	\begin{ex}%[0H2B3-1]
		Tam giác $ ABC$ có $ AB=4$, $BC=6$, $AC=2\sqrt{7}$. Điểm $ M$ thuộc đoạn $ BC$ sao cho $ MC=2MB$. Tính độ dài cạnh $ AM$.
		\choice
		{$ AM=4\sqrt{2}$}
		{$ AM=3\sqrt{2}$}
		{\True $ AM=2\sqrt{3}$}
		{$ AM=3$}
		\loigiai{
			\immini
			{
				Theo định lí hàm cosine, ta có\\
				$ \cos B=\dfrac{AB^2+BC^2-AC^2}{2\cdot AB\cdot BC}=\dfrac{4^2+6^2-(2\sqrt{7} )^2}{2\cdot 4\cdot 6}=\dfrac{1}{2}$.\\
				Do $ MC=2MB\Rightarrow BM=\dfrac{1}{3}BC=2$.\\
				Theo định lí hàm cosine, ta có\\
				$ \begin{aligned}
					AM^2&=AB^2+BM^2-2\cdot AB\cdot BM\cdot \cos B \\
					& =4^2+2^2-2\cdot 4\cdot 2\cdot \dfrac{1}{2}=12.
				\end{aligned}\\
				\Rightarrow AM=2\sqrt{3}$.
			}
			{
				\begin{tikzpicture}[scale=0.9, font=\footnotesize, line join = round, line cap = round,>=stealth]
					\tkzDefPoints{-2/0/B,0/3/A,5/0/C}
					\coordinate (M) at ($(B)!0. 33!(C)$);
					\tkzDrawPoints[fill=black](A,B,C,M)
					\tkzDrawPolygon(A,B,C)
					\tkzDrawSegments(M,A)
					\tkzLabelPoints[above](A)
					\tkzLabelPoints[below](M,B,C)
				\end{tikzpicture}
		}}
	\end{ex}
	\begin{ex}%[0H2B3-1]
		Cho hình thoi $ ABCD$ cạnh bằng $ 1$ cm và có $ \widehat{BAD}=60^\circ $. Tính độ dài cạnh $ AC$.
		\choice
		{$ AC=2$}
		{\True $ AC=\sqrt{3}$}
		{$ AC=2\sqrt{3}$}
		{$ AC=\sqrt{2}$}
		\loigiai{
			\immini{
				Do $ ABCD$ là hình thoi, có $ \widehat{BAD}=60^\circ \Rightarrow \widehat{ABC}=120^\circ $.\\
				Theo định lí hàm cosine, ta có\\
				$ \begin{aligned}
					AC^2&=AB^2+BC^2-2\cdot AB\cdot BC\cdot \cos \widehat{ABC} \\
					& =1^2+1^2-2\cdot 1\cdot 1\cdot \cos 120^\circ\\
					& =3.\\
					\Rightarrow AC=\sqrt{3}\cdot \\
				\end{aligned}$}
			{
				\begin{tikzpicture}[scale=1, font=\footnotesize, line join = round, line cap = round,>=stealth]
					\tkzDefPoints{-2/0/A,0/3/B}
					\tkzDefPointBy[rotation = center A angle -60](B) \tkzGetPoint{D}
					\coordinate (C) at ($(B)+(D)-(A)$);
					\tkzDrawPoints[fill=black](A,B,C,D)
					\tkzDrawPolygon(A,B,C,D)
					\tkzDrawSegments(C,A D,B)
					\tkzLabelPoints[above](B,C)
					\tkzLabelPoints[below](A,D)
					\tkzMarkAngles[size=0.5cm,arc=l](D,A,B)
			\end{tikzpicture}}
		}
	\end{ex}
	\begin{ex}%[0H2B3-4]
		Khoảng cách từ $A$ đến $B$ không thể đo trực tiếp được vì phải qua một đầm lầy. Người ta xác định được một điểm $C$ mà từ đó có thể nhìn được $A$ và $B$ dưới một góc $78^\circ 24'$. Biết $CA=250$ m, $CB=120$ m. Khoảng cách $AB$ bằng bao nhiêu?
		\choice
		{$266$ m}
		{\True $255$ m}
		{$166$ m}
		{$298$ m}
		\loigiai{
			\immini
			{
				Áp dụng định lí cosine cho $\triangle ABC$, ta có
				$
				\begin{aligned}
					AB^2& =CA^2+CB^2-2CA\cdot CB\cdot \cos C\\
					& =250^2+120^2-2\cdot 250\cdot 120\cdot \cos 78^\circ 24'\\
					& \approx 64835
				\end{aligned}$.\\
				$\Rightarrow AB\approx 255$ (m).
			}
			{
				\begin{tikzpicture}[scale=1, font=\footnotesize, line join=round, line cap=round,>=stealth]
					\tkzInit[xmin=-1,xmax=7,ymin=-1,ymax=3.5]
					\tkzClip
					\tkzDefPoints{0/0/A,6/0/B}
					\tkzDefPointBy[rotation= center A angle 28](B) \tkzGetPoint{a}
					\tkzDefPointBy[rotation= center B angle -74](A) \tkzGetPoint{b}
					\tkzInterLL(A,a)(B,b) \tkzGetPoint{C}
					\tkzDrawPoints[fill=black](A,B,C)
					\tkzDrawSegments(A,B B,C C,A)
					\tkzLabelPoints[above](C) \tkzLabelPoints[above left](A)
					\tkzLabelPoints[above right](B)
					\draw[pattern = north west lines] (2,-0.0) parabola bend (3,-0.75) (4,-0.0)--cycle;
					\tkzLabelSegment[above left](A,C){$250$ m}
					\tkzLabelSegment[above right](C,B){$120$ m}
					\tkzLabelAngle[pos=.8](A,C,B){$78^{\circ}24'$}
					\tkzMarkAngle[size=.5](A,C,B)
				\end{tikzpicture}
			}	
		}
	\end{ex}
	\begin{ex} Cho tam giác $ABC$ có $BC = 2\sqrt{3}$, $AB= \sqrt{6}- \sqrt{2}$, $AC= 2\sqrt{2}$. $AD$ là tia phân giác của góc $\widehat{BAD}$. Tính góc $\widehat{BAD}$.
		\choice
		{\True $60^\circ$}
		{ $90^\circ$}
		{$45^\circ$}
		{$75^\circ$}
		\loigiai{
			Áp dụng hệ quả định lý cosine trong tam giác $ABC$, ta có:
			$\begin{aligned}
				\cos A &=\dfrac{A B^2+A C^2-B C^2}{2\cdot A B\cdot A C}\\ &=\dfrac{(\sqrt 6-\sqrt 2)^2+(2\sqrt 2)^2-(2\sqrt 3)^2}{2\cdot(\sqrt 6-\sqrt 2)\cdot(2\sqrt 2)}\\ &=\dfrac{8-4\sqrt 3+8-12}{2\cdot(\sqrt 6-\sqrt 2)\cdot(2\sqrt 2)}\\ &=\dfrac{4-4\sqrt 3}{-8+8\sqrt 3}=\dfrac{-1}2
			\end{aligned}$
		}	
	\end{ex}
	\begin{ex}
		\immini{Một ô tô muốn đi từ địa điểm H đến địa điểm G, nhưng giữa H và G là một ngọn núi cao nên ô tô phải đi thành 2 đoạn từ H lên K (ô tô leo dốc lên núi) và từ K đến G (ô tô xuống núi). Các đoạn đường tạo thành tam giác $HKG$ với $HK = 15$ km, $KG = 20$ km và $\widehat{HKG}=120^\circ$. Giả sử cứ chạy $1$ km, ô tô tiêu thụ hết $0{,}3$ lít xăng. Giá thành xăng hiện nay là $13050$ đồng một lít xăng. Hỏi ô tô đi từ H đến G hết bao nhiêu tiền xăng?}
		{\begin{tikzpicture}[scale=1, font=\footnotesize, line join=round, line cap=round, >=stealth]
				\coordinate [label =left: $H$ ] (H) at (0,0);
				\coordinate [label =above: $K$ ] (K) at (1.5,3);
				\coordinate [label =right: $G$ ] (G) at (5,0);
				\draw (H)--(K)--(G)--cycle;	
				\draw[fill] (2,-0.2)--(2.1,0.7)--(2.2,1)--(2.4,0.5)--(2.6,1.3)--(3,-0.2)--cycle ;
				\foreach \diem in {H,K,G}\fill (\diem)circle(1.5pt);
			\end{tikzpicture}
		}
		\choice
		{\True $137025$ đồng}
		{ $107025$ đồng}
		{$12278$ đồng}
		{$137000$ đồng}	
		\loigiai{Tổng quãng đường mà ô tô phải đi là$ S = HK + KG = 15 + 20 = 35$ km.\\	
			Ô tô đi hết quãng đường tiêu thụ hết số lít xăng là 	$35 \cdot 0{,}3 = 10{,}5$ lít.\\	
			Ô tô đi từ H đến G hết số tiền xăng là
			$10{,}5 \cdot 13050 = 137025$ đồng.}
	\end{ex}	
	\begin{ex}%[Phạm Tuấn]%[0H2B3-1]
		Cho tam giác $ABC$ có góc $\widehat{B}=45^{\circ}$, $AC= 28$, $BC=25$. Tính số đo góc $A$  của tam giác (làm tròn kết quả đến hàng phần mười).
		\choice
		{$39{,}1^\circ$}
		{$40{,}2^\circ$}
		{\True $39{,}2^\circ$}
		{$40^\circ$}
		\loigiai{
			Áp dụng định lí sin ta có 
			\[ \dfrac{AC}{\sin B}=\dfrac{BC}{\sin A} \Rightarrow \sin A=\dfrac{BC\sin B}{AC}=\dfrac{25\sin 45^\circ}{28} =\dfrac{25\sqrt{2}}{56} \Rightarrow \widehat{A} \approx 39{,}2^\circ.\]
		}
	\end{ex}
	
	
	\begin{ex}%[Phạm Tuấn]%[0H2B3-1]
		Cho tam giác $ABC$ có góc $\widehat{B}=30^{\circ}, \widehat{C}=75^{\circ}$, $AB=20$. Độ dài cạnh $AC$ là
		\choice
		{$20(\sqrt{6}-\sqrt{2})$}
		{\True $10(\sqrt{6}-\sqrt{2})$}
		{$10(\sqrt{6}-1)$}
		{$5(\sqrt{6}+\sqrt{2})$}
		\loigiai{
			Áp dụng định lí sin ta có 
			\[ \dfrac{AC}{\sin B}=\dfrac{AB}{\sin C} \Rightarrow AC=\dfrac{AB\cdot\sin B}{\sin C}=\dfrac{20\cdot \sin 30^\circ}{\sin 75^\circ} =10(\sqrt{6}-\sqrt{2}).\]
		}
	\end{ex}
	
	\begin{ex}%[Phạm Tuấn]%[0H2B3-1] 
		Cho tam giác $ABC$ có $\widehat{B}= 30^\circ$, $\widehat{C}=45^\circ$ và $BC = 30\mathrm{~cm}$. Tính độ dài cạnh $AB$ (làm tròn kết quả đến hàng phần mười).
		\choice
		{$15 (\sqrt{3}+1) \mathrm{~cm}$}
		{$15 (\sqrt{3}-1) \mathrm{~cm}$}
		{$30 (2\sqrt{3}-1) \mathrm{~cm}$}
		{\True $30 (\sqrt{3}-1) \mathrm{~cm}$}
		\loigiai{
			Ta có $\widehat{A}= 180^\circ -  \widehat{B} -\widehat{C} = 180^\circ - 30^\circ-45^\circ  =105^\circ$. \\
			Áp dụng định lí sin ta có 
			\[
			\dfrac{AB}{\sin C} = \dfrac{BC}{\sin A} \Rightarrow AB = \dfrac{BC\sin C}{\sin A}  = \dfrac{30\sin 45^{\circ}}{\sin 105^{\circ}} = 30\sqrt{3}-30 \mathrm{~cm}.
			\]
		}
	\end{ex}
	
	
	\begin{ex}%[Phạm Tuấn]%[0H2B3-1]
		Cho tam giác $ABC$ có $BC= 11$, $\widehat{A} = 30^\circ$. Độ dài  cạnh $AB$ lớn nhất bằng bao nhiêu?
		\choice
		{$11\sqrt{3}$}
		{$\dfrac{22\sqrt{3}}{2}$}
		{\True $22$}
		{$11 (\sqrt{3}+1)$}
		\loigiai{
			Áp dụng định lí sin ta có 
			\begin{align*}
				\dfrac{AB}{\sin C} = \dfrac{BC}{\sin A} \Rightarrow AB = \dfrac{BC\sin C}{\sin A} = \dfrac{11 \sin C}{\sin 30^\circ}   \leq 22.
			\end{align*}
			Đẳng thức xảy ra khi $\widehat{C} = 90^\circ$. \\
			Vậy độ dài cạnh $AB$ lớn nhất bằng $22$.
		}
	\end{ex}
	
	\begin{ex}%[Phạm Tuấn]%[0H2B3-1] 
		Cho tam giác $ABC$ có $\widehat{C}= 30^\circ$ và $AB= 30 \mathrm{~cm}$. Tính bán kính đường tròn ngoại tiếp tam giác $ABC$.
		\choice
		{$30\sqrt{3} \mathrm{~cm}$}
		{$15 \sqrt{3}\mathrm{~cm}$}
		{\True $30 \mathrm{~cm}$}
		{$15 \mathrm{~cm}$}
		\loigiai{
			Áp dụng định lí sin ta có $R = \dfrac{AB}{2\sin C} = \dfrac{30}{2\sin 30^\circ} = 30 \mathrm{~cm}$. 
		}
	\end{ex}
	
	\begin{ex}%[Phạm Tuấn]%[0H2B3-1]
		Cho tam giác $MNK$ có $MN = a$, $MK=3a$, $\widehat{M} = 120^\circ$.  Tính bán kính đường tròn ngoại tiếp $R$ của tam giác $MNK$.
		\choice
		{\True $\dfrac{a\sqrt{39}}{3}$}
		{$\dfrac{a\sqrt{21}}{3}$}
		{$\dfrac{a\sqrt{33}}{3}$}
		{$\dfrac{a\sqrt{42}}{3}$}
		\loigiai{
			Áp dụng định lí côsin ta có
			\[
			NK^2= MN^2+MK^2-2MN \cdot MK \cos M = a^2+9a^2 -2 \cdot a \cdot 3a \cos 120^\circ = 13a^2 \Rightarrow NK=a\sqrt{13}.
			\]
			Áp dụng định lí sin ta có $R= \dfrac{NK}{2\sin M} = \dfrac{a\sqrt{13}}{2\sin 120^\circ} = \dfrac{a\sqrt{39}}{3}$.
		}
	\end{ex}
	
	
	\begin{ex}%[Phạm Tuấn]%[0H2B3-1] 
		\immini{
			Để đo bán kính của một chiếc đĩa cổ chỉ còn lại một phần, các nhà khảo cổ chọn $3$ điểm trên chiếc đĩa (hình vẽ). Biết $\widehat{A}=33^\circ$, $BC=15{,}3 \mathrm{~cm}$, tính bán kính của chiếc đĩa (làm tròn kết quả đến hàng phần mười).
			\choice
			{\True $13{,}8 \mathrm{cm}$}
			{$12{,}6 \mathrm{cm}$}
			{$12{,}9 \mathrm{cm}$}
			{$13{,}1 \mathrm{cm}$}
		}
		{
			\begin{tikzpicture}[scale=1, font=\footnotesize, line join=round, line cap=round,>=stealth]
				\coordinate (O) at (0,0); 
				\coordinate (A) at ($(O)+(131.4:3)$) ; 
				\coordinate (B) at ($(O)+(101.6:3)$) ; 
				\coordinate (C) at ($(O)+(41.7:3)$) ; 
				\coordinate (D) at ($(O)+(140:3)$) ; 
				\coordinate (E) at ($(O)+(90:1)$) ; 
				\coordinate (F) at ($(O)+(18:3)$) ; 
				\draw
				(D) 
				.. controls ++(0:0.5) and ++(150:0.5) .. (E)
				.. controls ++(180:-0.5) and ++(160: 0.5) .. (F)
				;
				\draw (A)--(B)--(C)--(A)  ; 
				\draw  (F) arc (18:140:3);
				\foreach \x/\g in {A/140,B/90,C/60} 
				\fill[black] (\x) circle (1.1pt)+(\g:3mm) node {$\x$};
			\end{tikzpicture}
		}
		\loigiai{
			Áp dụng định lí sin suy ra bán kính của chiếc đĩa là
			\[
			R = \dfrac{BC}{2\sin A} = \dfrac{15,3}{2\sin 33^\circ} \approx 13{,}8 \mathrm{~(cm)}. 
			\]
		}
	\end{ex}
	
	
	\begin{ex}%[Phạm Tuấn]%[0H2B3-2]
		Cho tam giác $ABC$ có $b^2= a^2+c^2+ac$. Khẳng định nào sau đây đúng?
		\choice
		{$\sin^2A = \sin^2B+ \sin^2C + \sin B\sin C$}
		{\True $\sin^2B = \sin^2A+ \sin^2C + \sin A\sin C$}
		{$\widehat{A} = 120^\circ$}
		{$\widehat{A} = 60^\circ$}
		\loigiai{
			Ta có
			\begin{align*}
				&b^2= a^2+c^2+ac \\
				\Leftrightarrow~& (2R\sin B)^2 =  (2R\sin A)^2+(2R\sin C)^2 + (2R\sin A) \cdot (2R\sin C )\\
				\Leftrightarrow~& \sin^2B = \sin^2A+ \sin^2C + \sin A\sin C. 
			\end{align*}
		}
	\end{ex}
	
	\begin{ex}%[Phạm Tuấn]%[0H2B3-2]
		Cho tam giác $ABC$. Khẳng định nào sau đây đúng?
		\choice
		{$\cot A = \dfrac{b^2+c^2-a^2}{2bc}$}
		{$\cot A = \dfrac{b^2+c^2-a^2}{abc}$}
		{$\cot A = \dfrac{R(b^2+c^2-a^2)}{2abc}$}
		{\True $\cot A = \dfrac{R(b^2+c^2-a^2)}{abc}$}
		\loigiai{
			Ta có
			\[
			\cot A = \dfrac{\cos A}{\sin A} = \dfrac{b^2+c^2 - a^2}{2bc \cdot \dfrac{a}{2R}} = \dfrac{R(b^2+c^2-a^2)}{abc}.
			\]
		}
	\end{ex}
	
	\begin{ex}%[Phạm Tuấn]%[0H2B3-1] 
		\immini{
			Cho tam giác $ABCD$ nội tiếp đường tròn tâm $O$.  Biết $\widehat{ACB} = 32^\circ$, $\widehat{ADC} = 75^\circ$ và $BC=8{,}8 \mathrm{~cm}$. Tính bán kính đường tròn đường tròn $(O)$. (Làm tròn kết quả đến hàng phần mười)
			\choice
			{$7{,}8 \mathrm{~cm}$}
			{$7{,}5 \mathrm{~cm}$}
			{$6{,}6 \mathrm{~cm}$}
			{\True $6{,}5 \mathrm{~cm}$}
		}
		{
			\begin{tikzpicture}[scale=1, font=\footnotesize, line join=round, line cap=round,>=stealth]
				\coordinate (O) at (1,1); 
				\coordinate (A) at ($(O)+(133:2.5)$) ; 
				\coordinate (B) at ($(O)+(198:2.5)$) ; 
				\coordinate (C) at ($(O)+(282:2.5)$) ; 
				\coordinate (D) at ($(O)+(0:2.5)$) ; 
				\draw (O) circle[radius=2.5cm];
				\draw (A)--(B)--(C)--(D)--(A)--(C) (B)--(D)  ; 
				\foreach \x/\g in {A/130,B/200,C/-90,D/0,O/-30} 
				\fill[black] (\x) circle (1.1pt)+(\g:3mm) node {$\x$};
			\end{tikzpicture}
		}
		\loigiai{
			Tứ giác $ABCD$ nội tiếp suy ra $\widehat{ADB} = \widehat{ACB} = 32^\circ $ $\Rightarrow \widehat{BCD} = \widehat{ADC} - \widehat{ADB} = 43^\circ$. \\
			Khi đó, bán kính đường tròn tâm $O$ là 
			\[
			R = \dfrac{BC}{2\sin \widehat{BDC}} = \dfrac{8{,}8}{2\sin 43^\circ} \approx 6{,}5 \mathrm{~(cm)}. 
			\]
		}
	\end{ex}
	\begin{ex}%[Phạm Tuấn]%[0H2B3-1]
		Cho tam giác $ABC$ có $AB=12$, $BC=15$, $AC=18$.  Tính $\widehat{A} + 2\widehat{C}$ (làm tròn kết quả đến hàng phần mười).
		\choice
		{$129{,}3^\circ$}
		{$142{,}7^\circ$}
		{$118{,}4^\circ$}
		{\True $138{,}6^\circ$}
		\loigiai{
			Áp dụng định lí côsin ta có
			\begin{align*}
				&\cos A = \dfrac{AB^2+AC^2-BC^2}{2 \cdot AB \cdot AC} = \dfrac{12^2+18^2-15^2}{2 \cdot 12 \cdot 18 } = \dfrac{9}{16} \Rightarrow \widehat{A} \approx  55{,}77^\circ. \\
				&\cos C = \dfrac{AC^2+BC^2-AB^2}{2 \cdot AC \cdot BC} = \dfrac{18^2+15^2-12^2}{2 \cdot 18 \cdot 15 } = \dfrac{3}{4} \Rightarrow \widehat{C} \approx  41{,}4^\circ.
			\end{align*}
			Suy ra $\widehat{A} + 2\widehat{C} \approx 138{,}6^\circ$. 
		}
	\end{ex}
	
	\begin{ex}%[Phạm Tuấn]%[0H2B3-1]
		Cho tam giác $ABC$ có góc $\widehat{A}=60^{\circ}$, $\widehat{B}=45^{\circ}$, $AB=25$. Độ dài cạnh $BC$ gần với giá trị nào nhất dưới đây?
		\choice
		{$22$}
		{\True $22{,}5$}
		{$24{,}5$}
		{$21{,}5$}
		\loigiai{
			Ta có $\widehat{C}=180^\circ - \widehat{A}-\widehat{B} =  180^\circ - 60^{\circ}-45^{\circ} = 75^{\circ}$. \\
			Áp dụng định lí sin ta có 
			\[ \dfrac{BC}{\sin A}=\dfrac{AB}{\sin C} \Rightarrow BC=\dfrac{AB\cdot\sin A}{\sin C}=\dfrac{25\cdot \sin 60^\circ}{\sin 75^\circ} \approx 22{,}4.\]
		}
	\end{ex}
	
	\begin{ex}%[Phạm Tuấn]%[0H2B3-1]
		Cho tam giác $ABC$ có $AB=8$, $AC = 11$, $\widehat{A}=30^\circ$.  Số đo góc $B$ gần với giá trị nào nhất dưới đây?
		\choice
		{$50{,}5^\circ$}
		{$45{,}8^\circ$}
		{$65{,}3^\circ$}
		{\True $55{,}2^\circ$}
		\loigiai{
			Áp dụng định lí côsin ta có 
			\[ BC^2=AB^2+AC^2-2AB \cdot AC \cos A =8^2+11^2-2 \cdot 8 \cdot 11 \cos 30^\circ \Rightarrow BC \approx 6{,}7.\]
			Áp dụng định lí sin ta có 
			\begin{align*}
				\dfrac{AC}{\sin B} = \dfrac{BC}{\sin A} \Rightarrow \sin B = \dfrac{AC\sin A}{BC}  = \dfrac{11 \sin 30^{\circ}}{6{,}7} \Rightarrow \widehat{B} \approx 55{,}2^\circ.  
			\end{align*}
		}
	\end{ex}

	\begin{ex}%[Phạm Tuấn]%[0H2B3-4] 
		\immini{
			Để đo bán kính của một chiếc đĩa cổ chỉ còn lại một phần, các nhà khảo cổ chọn ba điểm trên chiếc đĩa (hình vẽ).  Biết $AB=7{,}1 \mathrm{~cm}$, 
			$BC=16{,}3 \mathrm{~cm}$, $AC=19{,}6 \mathrm{~cm}$,  tính bán kính của chiếc đĩa (làm tròn kết quả đến hàng phần mười).
			\choice
			{$11{,}1 \mathrm{cm}$}
			{$9{,}8 \mathrm{cm}$}
			{\True $10{,}3 \mathrm{cm}$}
			{$10{,}1 \mathrm{cm}$}
		}
		{
			\begin{tikzpicture}[scale=1, font=\footnotesize, line join=round, line cap=round,>=stealth]
				\coordinate (O) at (0,0); 
				\coordinate (A) at ($(O)+(131.4:3)$) ; 
				\coordinate (B) at ($(O)+(101.6:3)$) ; 
				\coordinate (C) at ($(O)+(41.7:3)$) ; 
				\coordinate (D) at ($(O)+(140:3)$) ; 
				\coordinate (E) at ($(O)+(90:1)$) ; 
				\coordinate (F) at ($(O)+(18:3)$) ; 
				\draw
				(D) 
				.. controls ++(0:0.5) and ++(150:0.5) .. (E)
				.. controls ++(180:-0.5) and ++(160: 0.5) .. (F)
				;
				\draw (A)--(B)--(C)--(A)  ; 
				\draw  (F) arc (18:140:3);
				\foreach \x/\g in {A/140,B/90,C/60} 
				\fill[black] (\x) circle (1.1pt)+(\g:3mm) node {$\x$};
			\end{tikzpicture}
		}
		\loigiai{
			Áp dụng định lí côsin ta có
			\begin{align*}
				&BC^2=AB^2+AC^2-2AB \cdot AC \cos A  \\
				\Leftrightarrow~ & \cos A = \dfrac{AB^2+AC^2- BC^2}{2AB \cdot AC}  \\
				\Leftrightarrow~ & \cos A = \dfrac{7{,}1^2+19{,}6^2-16{,}3^2}{2 \cdot 7{,}1 \cdot 19{,}6}\\
				\Rightarrow~& \widehat{A} \approx 52{,}6427^\circ. 
			\end{align*}
			Áp dụng định lí sin suy ra bán kính của chiếc đĩa là
			\[
			R = \dfrac{BC}{2\sin A} = \dfrac{16{,}3}{2\sin 52{,}6427^\circ} \approx 10{,}3 \mathrm{~(cm)}. 
			\]
		}
	\end{ex}
	
	
	\begin{ex}%[Phạm Tuấn]%[0H2B3-4] 
		\immini{
			Để đo khoảng cách từ $A$ đến $B$ ngang qua một đầm lầy, người ta chọn điểm  $C$, sau đó khoảng cách từ $A$ đến $C$ và các góc $A$, $C$. Tính khoảng cách từ $A$ đến $B$ biết $AC=115\mathrm{~m}$, $\widehat{A}=98^\circ$, $\widehat{C}=52^\circ$.
			\choice
			{$188{,}1 \mathrm{~m}$}
			{$190{,}7 \mathrm{~m}$}
			{\True $181{,}2 \mathrm{~m}$}
			{$193{,}6 \mathrm{~m}$}
		}
		{
			\begin{tikzpicture}[scale=1, font=\footnotesize, line join=round, line cap=round,>=stealth]
				\path
				(2,2) coordinate (A)
				(7,2) coordinate (B)
				(1.6,4.5) coordinate (C)
				(2.5,2) coordinate (D)
				(3.5,3.1) coordinate (E)
				(5.5,2.9) coordinate (F)
				(6.4,1.5) coordinate (G)
				(5.2,0.7) coordinate (H)
				(3.5,0.7) coordinate (I)
				(2.6,1.3) coordinate (J)
				;
				\draw[fill=gray!40] 
				(D) 
				.. controls ++(65:0.1) and ++(200: 1) .. (E)
				.. controls ++(200:-0.5) and ++(170: 0.3) .. (F)
				.. controls ++(170:-0.5) and ++(100: 0.3) .. (G)
				.. controls ++(100:-0.3) and ++(30: 0.3) .. (H)
				.. controls ++(30:-0.3) and ++(150: -0.3) .. (I)
				.. controls ++(150:0.3) and ++(130: -0.3) .. (J)
				.. controls ++(130:0.3) and ++(65: -0.1) .. (D)
				;
				\draw[dashed] (A)--(B)--(C) ;
				\draw (A)--(C)   ;
				\foreach \x/\g in {A/-120,B/-60,C/90} 
				\fill[black] (\x) circle (1pt)+(\g:3mm) node {$\x$};
			\end{tikzpicture}
		}
		\loigiai{
			Ta có $\widehat{B} = 180^\circ -  \widehat{A} -\widehat{C}  = 30^\circ $. \\
			Áp dụng định lí sin ta có 
			\begin{align*}
				\dfrac{AB}{\sin C} = \dfrac{AC}{\sin B} \Rightarrow AB = \dfrac{AC\sin C}{\sin B} = \dfrac{115 \sin 52^\circ}{\sin 30^\circ} \approx 181{,}2 \mathrm{~(m).}
			\end{align*}
		}
	\end{ex}
	
	\begin{ex}%[Phạm Tuấn]%[0H2K3-1] 
		\immini{
			Cho tam giác $A B C$ có $AB=8$, $AC=10$, $\widehat{A}=75^{\circ}$. $M$ là điểm thuộc cạnh $BC$  sao cho $CM=2BM$. Bán kính đường tròn ngoại tiếp tam giác  $ABM$  gần nhất với giá trị nào dưới đây?
			\choice
			{$3{,}8$}
			{\True $4{,}1$}
			{$3{,}6$}
			{$3{,}5$}
		}
		{
			\begin{tikzpicture}[scale=1, font=\footnotesize, line join=round, line cap=round,>=stealth]
				\path
				(0,0) coordinate (B)
				(4,0) coordinate (C)
				;
				\coordinate (M) at ($(B)!{1/3}!(C)$) ;
				\coordinate (A) at ($(B)+(66:3.5)$) ; 
				\draw (A)--(B)--(C)--(A) (A)--(M) ;
				\foreach \x/\g in {A/90,B/-120,C/-60,M/-90} 
				\fill[black] (\x) circle (1pt)+(\g:3mm) node {$\x$};
			\end{tikzpicture}
		}
		\loigiai{
			Áp dụng định lí côsin ta có 
			\begin{align*}
				&BC^2=AB^2+AC^2-2AB \cdot AC \cos A =8^2+10^2-2 \cdot 8 \cdot 10 \cos 75^\circ  \Rightarrow BC\approx  11{,}072; \\
				& \cos B = \dfrac{AB^2+BC^2-AC^2}{2AB \cdot BC} \approx  0{,}4888 \Rightarrow \widehat{B} \approx  60{,}4^\circ.
			\end{align*}
			Ta có $CM=2BM \Rightarrow BM = \dfrac{1}{3}BC =  3{,}69$. \\
			Áp dụng định lí côsin ta có 
			\begin{align*}
				AM^2=AB^2+BM^2-2AB \cdot BM \cos B =8^2+3{,}69^2-2 \cdot 8 \cdot 3{,}69 \cdot  0{,}4888 \Rightarrow AM \approx 6{,}983.
			\end{align*}
			Áp đụng định lí sin,  suy ra bán kính đường tròn ngoại tiếp tam giác $ABM$ là
			\[
			R = \dfrac{AM}{2\sin B} = \dfrac{6{,}983}{2\sin 60{,}4^\circ} \approx 4. 
			\]
		}
	\end{ex}
	
	\begin{ex}%[Phạm Tuấn]%[0H2B3-4] 
		\immini{
			Tàu $A$ rời cảng vào lúc 6h00 và chuyển động với vận tốc $30\mathrm{~km/h}$.  Tàu $B$ rời cảng vào lúc 6h30. Vào lúc 9h30 tàu $B$ gặp tàu $A$ tại điểm $C$ (hình vẽ). Giả sử hai tàu chuyển động thẳng và có vận tốc không đổi trong suốt quá trình di chuyển, tính vận tốc tàu $B$ (kết quả làm tròn đến hàng phần mười).
			\choice
			{$42{,}5\mathrm{~km/h}$}
			{\True $44{,}8\mathrm{~km/h}$}
			{$41{,}7\mathrm{~km/h}$}
			{$45{,}4\mathrm{~km/h}$}
		}
		{
			\begin{tikzpicture}[scale=1, font=\footnotesize, line join=round, line cap=round,>=stealth]
				\path
				(0.5,2) coordinate (A)
				(0,0) coordinate (B)
				(4,2) coordinate (C)
				;
				\draw (A)--(B)--(C)--(A) ;
				
				\tkzMarkAngle[arc=ll, size=0.5,mark=0](C,B,A)
				\tkzLabelAngle[pos=0.9](C,B,A) {$55^{\circ}$}
				\tkzMarkAngle[arc=ll, size=0.5,mark=0](B,A,C)
				\tkzLabelAngle[pos=0.9](B,A,C) {$102^{\circ}$}
				\foreach \x/\g in {A/120,B/-90,C/60} 
				\fill[black] (\x) circle (1pt)+(\g:3mm) node {$\x$};
			\end{tikzpicture}
		}
		\loigiai{
			Khoảng cách từ $A$ đến $C$ là $30 \cdot 2{,}5 =75 \mathrm{~km}$. \\
			Áp dụng định lí sin ta có 
			\begin{align*}
				\dfrac{AC}{\sin B} = \dfrac{BC}{\sin A} \Rightarrow BC = \dfrac{AC\sin A}{\sin B}  = \dfrac{75 \sin 102^{\circ}}{\sin 55^{\circ}}.
			\end{align*}
			Suy ra vận tốc của tàu $B$ là  $v= \dfrac{BC}{2} = \dfrac{75 \sin 102^{\circ}}{2\sin 55^{\circ}}  \approx 44{,}8\mathrm{~km/h}$.
		}
	\end{ex}

	\begin{ex}%[0H2Y3-1]
		Chọn công thức đúng trong các đáp án sau
		\choice
		{$S=\dfrac{1}{2}bc\sin B$}
		{\True $S=\dfrac{1}{2}bc\sin A$}
		{$S=\dfrac{1}{2}ab\sin B$}
		{$S=\dfrac{1}{2}ac\sin A$}
		\loigiai{
			Công thức đúng là $S=\dfrac{1}{2}bc\sin A$.
		}
	\end{ex}
	\begin{ex}%[0H2B3-2]
		Cho $\triangle ABC$ với các cạnh $AB=c$, $AC=b$, $BC=a$. Gọi $R$, $r$, $S$ lần lượt là bán kính đường tròn ngoại tiếp, nội tiếp và diện tích của tam giác $ABC$. Trong các phát biểu sau, phát biểu nào \textbf{sai}?
		\choice
		{$S=\dfrac{abc}{4R}$}
		{\True $R=\dfrac{a}{\sin A}$}
		{$S=\dfrac{1}{2}ab\sin C$}
		{$a^2+b^2-c^2=2ab\cos C$}
		\loigiai{Theo định lý Sin trong tam giác, ta có $\dfrac{a}{\sin A}=2R$. Nên mệnh đề \textbf{sai} là ``$R=\dfrac{a}{\sin A}$''.}
	\end{ex}
	
	\begin{ex}%[0H2Y3-1]
		Cho tam giác $ABC$ có $AB=4$, $AC=3$, $\widehat{BAC}=30^\circ$. Khi đó diện tích tam giác $ABC$ bằng
		\choice
		{\True $3$}
		{$4\sqrt{3}$}
		{$6\sqrt{3}$}
		{$6$}
		\loigiai{
			Ta có $S_{ABC}=\dfrac{1}{2}AB\cdot AC\cdot\sin\widehat{BAC}=\dfrac{4\cdot 3\cdot\sin30^\circ}{2}=3$.
		}
	\end{ex}
	\begin{ex}%[0H2B3-1]
		Tìm chu vi tam giác $ABC$, biết $AB=6$ và $2\sin A=3\sin B=4\sin C$.
		\choice
		{\True $26$}
		{$13$}
		{$5\sqrt{26}$}
		{$10\sqrt{6}$}
		\loigiai{
			Từ $2\sin A=3\sin B=4\sin C$ suy ra $2BC=3AC=4AB$.\\
			Mà $AB=6$ nên $AC=8$, $BC=12$. Chu vi tam giác bằng $26$.
		}
	\end{ex}
	\begin{ex}%[0H2B3-1]
		Cho tam giác $ABC$ có $a=13$ m, $b= 14$ m, $c=15$ m. Tính diện tích $S$ của tam giác $ABC$.
		\choice
		{\True $S= 84$ m$^2$}
		{$S= 90$ m$^2$}
		{$S= 76$ m$^2$}
		{$S= 80$ m$^2$}
		\loigiai{
			Ta có $p=\dfrac{a+b+c}{2} =21$ và $S=\sqrt{p(p-a)(p-b)(p-c)}=\sqrt{21(21-13)(21-14)(21-15)} =84$ m$^2$.	
		}
	\end{ex}
	\begin{ex}%[0H2B3-1]
		Cho tam giác $ABC$. Biết $AB=3$, $AC=4$, $BC>5$ và diện tích tam giác $ABC$ bằng $3\sqrt{3}$. Số đo góc $\widehat{BAC}$ bằng
		\choice
		{\True $120^{\circ}$}
		{$60^{\circ}$}
		{$135^{\circ}$}
		{$45^{\circ}$}
		\loigiai{
			Ta có $S_{\triangle ABC}=\dfrac{1}{2}\cdot AB \cdot AC \cdot \sin{\widehat{BAC}}$, suy ra 
			\[\sin{\widehat{BAC}}=\dfrac{2S_{\triangle ABC}}{AB \cdot AC}=\dfrac{2\cdot 3\sqrt{3}}{3 \cdot 4}=\dfrac{\sqrt{3}}{2} \Rightarrow \hoac{&\widehat{BAC}=60^\circ\\&\widehat{BAC}=120^\circ}.\]
			Mặt khác, ta có $\cos{\widehat{BAC}}=\dfrac{AB^2+AC^2-BC^2}{2\cdot AB \cdot AC}<\dfrac{9+16-25}{2 \cdot 3 \cdot 4}=0$.\\
			Vậy $\widehat{BAC}=120^{\circ}$.
		}
	\end{ex}
	\begin{ex}%[0H2K3-1]
		Cho tam giác $ABC$ có $AB=2$, $AC=3$, $BC=4$. Khi đó độ dài đường cao của tam giác $ABC$ kẻ từ $A$ bằng
		\choice
		{$\dfrac{3\sqrt{15}}{2}$}
		{$\dfrac{3\sqrt{15}}{4}$}
		{\True $\dfrac{3\sqrt{15}}{8}$}
		{$3\sqrt{15}$}
		\loigiai{
			Ta có nữa chu vi $p=\dfrac{2+3+4}{2}=\dfrac{9}{2}$.\\
			Suy ra $S_{ABC}=\sqrt{p(p-AB)(p-AC)(p-BC)}=\sqrt{\dfrac{9}{2}\left(\dfrac{9}{2}-2\right)\left(\dfrac{9}{2}-3\right)\left(\dfrac{9}{2}-4\right)}=\dfrac{3\sqrt{15}}{4}$.\\
			Suy ra độ dài đường cao kẻ từ $A$ bằng $\dfrac{2S_{ABC}}{BC}=\dfrac{2\cdot\dfrac{3\sqrt{15}}{4}}{4}=\dfrac{3\sqrt{15}}{8}$.
		}
	\end{ex}
	\begin{ex}%[0H2B3-1]
		Cho tam giác $ABC$ có $AB=9$cm, $AC=12$cm và $BC=15$cm. Khi đó đường trung tuyến $BM$ của tam giác $ABC$ có độ dài là 
		\choice
		{$117$cm}
		{$18{,}82$cm}
		{$10{,}82$cm}
		{\True $7{,}5$cm}
		\loigiai{
			Ta có $m_a^2= \dfrac{2(b^2+c^2)-a^2}{4}=\dfrac{2(12^2+9^2)-15^2}{4}= \dfrac{225}{4} \Rightarrow m_a= 7{,}5$. 		
		}
	\end{ex}
	\begin{ex}%[0H2K3-1]
		Tam giác $ABC$ có các trung tuyến $m_a=10$, $m_b=8$ và $m_c=6$. Tính diện tích $S$ của tam giác $ABC$.
		\choice
		{\True $S=32$}
		{$S=24$}
		{$S=48$}
		{$S=64$}
		\loigiai{
			\immini{
				Gọi $M,N,P$ lần lượt là trung điểm $BC,CA,AB$, $G$ là trọng tâm tam giác $ABC$.\\
				Theo bài ra ta có $AM=10,BN=8,CP=6$.\\
				Lấy $Q$ đối xứng với $G$ qua $M$ thì $BGCQ$ là hình bình hành và ta có $BQ=CG=\dfrac{2CP}{3}=4$, $QG=2GM=\dfrac{2AM}{3}=\dfrac{20}{3}$.\\
				Mà $BG=\dfrac{2BN}{3}=\dfrac{16}{3}$ nên $QG^2=BG^2+BQ^2$ hay $\triangle BGQ$ vuông tại $B$.\\
				Suy ra $S_{BGQ}=\dfrac{BG\cdot BQ}{2}=\dfrac{32}{3}$.\\
				Mà $S_{BGQ}=S_{BGC}=\dfrac{1}{3}S_{ABC}\Rightarrow S_{ABC}=32$.
			}{
				\begin{tikzpicture}[scale=1, font=\footnotesize, line join=round, line cap=round,>=stealth]
					\tkzInit[xmin=-0.5, xmax=5.5, ymin=-0.5, ymax=6.5]
					\tkzClip
					\tkzDefPoints{4/1.5/C,2/0/Q}
					\tkzDefPointBy[rotation=center C angle -90](Q)\tkzGetPoint{g}
					\tkzDefPointBy[homothety=center C ratio 3/4](g)\tkzGetPoint{G}
					\tkzDefPointBy[symmetry=center G](Q)\tkzGetPoint{A}
					\tkzDefMidPoint(G,Q)\tkzGetPoint{M}
					\tkzDefPointBy[symmetry=center M](C)\tkzGetPoint{B}
					\tkzDefMidPoint(A,B)\tkzGetPoint{P}
					\tkzDefMidPoint(A,C)\tkzGetPoint{N}
					\tkzDrawPoints[fill=black](C,G,Q,B,A,P,M,N)
					\tkzDrawSegments(A,B B,C C,A A,Q B,N C,P B,Q C,Q)
					\tkzLabelPoints[above](A)
					\tkzLabelPoints[below](B,Q,C)
					\tkzLabelPoints[right](G,N)
					\tkzLabelPoints[below right](M)
					\tkzLabelPoints[left](P)
				\end{tikzpicture}
			}
		}
	\end{ex}
	\begin{ex}%[0H2K3-4]
		Cho tam giác $ABC$ có chu vi bằng $26$ cm và $\dfrac{\sin A}{2} = \dfrac{\sin B}{6} = \dfrac{\sin C}{5}$. Tính diện tích của tam giác $ABC$. 
		\choice
		{$ 2\sqrt{23} $ (cm$^2$)}
		{$ 6\sqrt{13} $ (cm$^2$)}
		{\True $ 3\sqrt{39} $ (cm$^2$)}
		{$ 5\sqrt{21} $ (cm$^2$)}
		\loigiai{
			Ta có $\dfrac{\sin A}{2} = \dfrac{\sin B}{6} = \dfrac{\sin C}{5} \Leftrightarrow \heva{&\sin B = 3\sin A\\&\sin C = \dfrac{5}{2}\sin A.}$\\
			Mặt khác theo định lí $\sin$ trong tam giác $ABC$ ta có
			$\dfrac{a}{\sin A} = \dfrac{B}{\sin B} = \dfrac{c}{\sin C} \Leftrightarrow \heva{&b = 3a\\&c = \dfrac{5}{2}a.}$\\
			Mà $a + b + c = 26 \Leftrightarrow a + 3a + \dfrac{5}{2}a = 26 \Leftrightarrow \dfrac{13a}{2} = 26 \Leftrightarrow a = 4 \Rightarrow b = 12$ và $c = 10$.\\
			Vậy diện tích tam giác $ABC$ là 
			$$S_{\triangle ABC} = \sqrt{13(13 - 4)(13 - 12)(13 - 10)} = 3\sqrt{39} \,\, (\textrm{cm}^2).$$
		}
	\end{ex}
	\begin{ex}%[0H2B3-1]
		Cho tam giác $ABC$ vuông tại $C$ và $BC=6$, $CA=8$. Tính bán kính đường tròn nội tiếp của tam giác $ABC$.
		\choice
		{\True $2$}
		{$2\sqrt{2}$}
		{$\sqrt{2}$}
		{$4$}
		\loigiai{
			Vì tam giác $ABC$ vuông tại $C$ nên $AB=\sqrt{AC^2+BC^2}=10$ và $S_{ABC}=\dfrac{1}{2}\cdot AC\cdot BC=24$.\\
			Mặt khác $p=\dfrac{6+8+10}{2}=12$, $S_{ABC}=p\cdot r\Rightarrow r=\dfrac{S_{ABC}}{p}=\dfrac{24}{12}=2$.
		}
	\end{ex}
	
	\begin{ex}%[0H2T3-4]
		Từ vị trí $A$ người ta quan sát một cây cao (Hình vẽ). Biết $AH=4$ m, $HB=20$ m, $\widehat{BAC}=45^{\circ}$. Chiều cao của cây gần nhất với giá trị nào sau đây?
		\begin{center}
			\usetikzlibrary{decorations.pathmorphing,shapes}
			\tikzset{
				treetop/.style = {
					decoration={random steps, segment length=0.4mm},
					decorate
				},
				trunk/.style = {
					decoration={random steps, segment length=2mm, amplitude=0.2mm},
					decorate
				}
			}
			\tikzset{
				man/.pic={%
					\fill [rounded corners=1.5] (0,0.4) -- (0,0.4) -- (0.4,0.5) -- (0.4,0.4) --
					(0.325,0.4) -- (0.325,0.7) -- (0.3,0.7) -- (0.3,0) -- (0.225,0) --
					(0.225,0.4) -- (0.175,0.4) -- (0.175,0) -- (0.1,0) -- (0.1,0.7) --
					(0.075,0.7) -- (0.075,0.4) -- cycle;
					\fill (0.2,0.9) circle (0.1);
					\coordinate (-head) at (0.2,1);
					\coordinate (-foot) at (0.2,0);
				}
			}
			\begin{tikzpicture}
				\path 
				(-5,-2.09) coordinate (A)
				(0,1.5) coordinate (C)
				(0.1,-3) coordinate (T)
				(-5,-3) coordinate (H)
				(0,-3) coordinate (B)		
				;
				%\pic[red] at (-6.3,-3) (myman) {man};
				\foreach \w/\f in {0.3/30,0.2/50,0.1/70} {
					\fill [brown!\f!black, trunk] (0,0) ++(-\w/2,0) rectangle +(\w,-3);
				}
				\foreach \n/\f in {1.4/40,1.2/50,1/60,0.8/70,0.6/80,0.4/90} {
					\fill [green!\f!black, treetop] ellipse (\n/1.5 and \n);
				}
				\draw (H)--(T) (A)--(H) (A)--(C) (A)--(B);
				\pic[draw,"$45^{\circ}$", angle eccentricity=1.4,angle radius=0.8cm]{angle=B--A--C};
				\pic[draw,"$ $", angle eccentricity=1.4,angle radius=0.7cm]{angle=B--A--C};
				\draw pic[draw, angle radius=2mm]{right angle=B--H--A};
				\path (H)--(B) node[below,midway,sloped]{$20$ m};
				\foreach \x/\g in {A/180,B/-90,C/90,H/-90} \fill[black] (\x)+(\g:.3) node {$\x$};
			\end{tikzpicture}
		\end{center}
		\choice
		{$14$ m}
		{$15$ m}
		{\True $17$ m}
		{$16$ m}
		\loigiai{\immini{Ta có $AB= \sqrt{AH^2 + BH^2} = \sqrt{4^2+20^2} = 4 \sqrt{26}$.\\
				$\tan \widehat{HAB} = \dfrac{HB}{HA} = \dfrac{20}{4} = 5 \Rightarrow \widehat{HAB} \approx 78{,}69^{\circ}$.\\
				Do $AH \parallel BC$ nên $ \widehat{ABC} = \widehat{HAB} \approx 78{,}69^{\circ}$.\\
				$\widehat{ACB} = 180^{\circ} - 45^{\circ} - \widehat{ABC} \approx 56{,}31^{\circ}$.\\
				Áp dụng định lí hàm số $\sin$ trong tam giác $ABC$ ta có
				$$ \dfrac{BC}{\sin 45^{\circ}} = \dfrac{AB}{\sin 56{,}31^{\circ}} = \dfrac{4 \sqrt{26}}{\sin 56{,}31^{\circ}} \Rightarrow BC \approx  17{,}33.$$	}
			{\begin{tikzpicture}[scale=0.8, font=\footnotesize,line join = round, line cap = round, >=stealth]
					\tkzDefPoints{-5/-2.09/A, 0/1.5/C, 0.1/-3/T, -5/-3/H,0/-3/B}
					\tkzDrawSegments(H,B A,H A,C A,B B,C)
					\tkzMarkAngles[size=0.6,arc=ll](B,A,C)
					\tkzMarkRightAngles(B,H,A)
					\tkzLabelAngles[pos=1.1](B,A,C){$45^{\circ}$}
					\tkzLabelPoints[above](C)
					\tkzLabelPoints[above](A)
					\tkzLabelPoints[below](B)
					\tkzLabelPoints[below](H)
			\end{tikzpicture}}
		}
	\end{ex}
	
	\begin{ex}%[0H2G3-1]
		Một miếng giấy hình tam giác $ABC$ diện tích $S$ có $I$ là trung điểm $BC$ và $O$ là trung điểm của $AI$. Cắt miếng giấy theo một đường thẳng qua $O$, đường thẳng này đi qua $M$, $N$ lần lượt trên các cạnh $AB$, $AC$. Khi đó diện tích miếng giấy chứa điểm $A$ có diện tích thuộc đoạn $\left[mS; nS\right]$. Tính $T = \dfrac{1}{m} + \dfrac{1}{n}$.
		\choice
		{$T = \dfrac{7}{12}$}
		{$T = 12$}
		{\True $T = 7$}
		{$T=\dfrac{12}{7}$}
		\loigiai
		{\immini
			{Ta có $\dfrac{S_{\triangle AMN}}{S_{\triangle ABC}} = \dfrac{AM}{AB}\cdot\dfrac{AN}{AC}$\\
				Dễ thấy $S_{\triangle ABI} = S_{\triangle ACI} = \dfrac{1}{2}\cdot S_{\triangle ABC}$.\\
				Mặt khác   
				\begin{eqnarray*}
					&{  }&\dfrac{S_{\triangle AMO}}{S_{\triangle ABI}} = \dfrac{AO}{AI}\cdot\dfrac{AM}{AB}\\
					&=& \dfrac{1}{2}\cdot\dfrac{AM}{AB}\Rightarrow \dfrac{2\cdot S_{\triangle AMO}}{S_{\triangle ABC}} = \dfrac{1}{2}\cdot\dfrac{AM}{AB}\quad (1)
				\end{eqnarray*}	
			}
			{\begin{tikzpicture}[scale=1, font=\footnotesize, line join=round, line cap=round, >=stealth]
					\clip(-1,-1) rectangle (6.5,4.5);
					\tkzDefPoints{0/0/B,1/4/A, 6/0/C}
					\tkzDefMidPoint(B,C) \tkzGetPoint{I}
					\tkzDefMidPoint(A,I) \tkzGetPoint{O}
					\tkzDefBarycentricPoint(A=2,B=3)
					\tkzGetPoint{M}
					\tkzInterLL(M,O)(A,C)\tkzGetPoint{N}
					\tkzInterLL(B,O)(A,C)\tkzGetPoint{N'}
					\tkzInterLL(C,O)(A,B)\tkzGetPoint{M'}
					\tkzDefLine[parallel = through I](B,N') \tkzGetPoint{c}
					\tkzInterLL(I,c)(A,C)\tkzGetPoint{I'}
					\tkzLabelPoints[above](A)
					\tkzLabelPoints[left](M,M')
					\tkzLabelPoints[above right](N,N',I')
					\tkzLabelPoints[below left](I)
					\tkzLabelPoints[below](B,C)
					\tkzDrawPoints[fill=black](A,B,C,M,N,O,I,M',N',I')
					\tkzDrawSegments(A,B B,C C,A A,I M,N)
					\tkzDrawSegments [dashed](B,N' C,M' I,I')
					\draw ($(O)+(-0.1,-0.35)$) node {$O$};
				\end{tikzpicture}
			}\noindent
			Tương tự  $\dfrac{2\cdot S_{\triangle ANO}}{S_{\triangle ABC}} = \dfrac{1}{2}\cdot\dfrac{AN}{AC}\quad (2)$.	
			Từ $(1)$ và $(2)$ suy ra 
			$$\dfrac{2\cdot S_{\triangle AMN}}{S_{\triangle ABC}} = \dfrac{1}{2}\cdot\left(\dfrac{AM}{AB} + \dfrac{AN}{AC}\right)\Leftrightarrow \dfrac{S_{\triangle AMN}}{S_{\triangle ABC}} = \dfrac{1}{4}\cdot\left(\dfrac{AM}{AB} + \dfrac{AN}{AC}\right)$$
			Theo bất đẳng thức Côsi suy ra 
			$$\dfrac{AM}{AB} + \dfrac{AN}{AC}\geq 2\sqrt{\dfrac{AM}{AB}\cdot \dfrac{AN}{AC}}\Leftrightarrow \left(\dfrac{AM}{AB} + \dfrac{AN}{AC}\right)^2\geq 4\cdot \dfrac{AM}{AB}\cdot \dfrac{AN}{AC}$$
			Đặt $t = \dfrac{S_{\triangle AMN}}{S_{\triangle ABC}}$ điều kiện $t > 0$. Khi đó ta có  $16t^2\geq 4t\Leftrightarrow t\geq\dfrac{1}{4}$ suy ra $S_{\triangle AMN}\geq \dfrac{1}{4}\cdot S_{ABC}$.\\
			Khi $M\equiv B$ suy ra $N\equiv N'$ khi đó $S_{\triangle AMN} =  S_{\triangle ABN'}$.\\
			Mà $S_{\triangle ABN'} = S_{\triangle ABO}+ S_{\triangle AON'}$.\\
			Dễ thấy $S_{\triangle ABO} = \dfrac{1}{2}\cdot S_{\triangle ABI} = \dfrac{1}{4}\cdot S_{\triangle ABC}$.\\
			Mặt khác từ $I$ kẻ $II'\parallel BN'$, khi đó $AN' = N'I' = I'C$ nên
			$$\dfrac{S_{\triangle AON'}}{S_{\triangle AIC}} = \dfrac{AO}{AI}\cdot\dfrac{AN'}{AC} = \dfrac{1}{2}\cdot\dfrac{1}{3} = \dfrac{1}{6}\Rrightarrow S_{\triangle AON'} = \dfrac{1}{6}\cdot S_{\triangle AIC}$$
			Do đó $S_{\triangle AON'} = \dfrac{1}{12}\cdot S_{\triangle ABC}$ nên $S_{\triangle ABN'} = \dfrac{1}{4}\cdot S_{\triangle ABC} + \dfrac{1}{12}\cdot S_{\triangle ABC} = \dfrac{1}{3}\cdot S_{\triangle ABC}$.\\
			Khi $N\equiv C$ suy ra $M\equiv M'$ khi đó $S_{\triangle AMN} =  S_{\triangle ACM'}$.\\
			Chứng minh tương tự, ta có  $S_{\triangle ACM'} = \dfrac{1}{3}\cdot S_{\triangle ABC}$.\\
			Do đó khi $MN$ đi thay đổi qua $O$ suy ra 
			$$\dfrac{1}{4}\cdot S_{\triangle ABC}\leq  S_{\triangle AMN}\leq  \dfrac{1}{3}\cdot S_{\triangle ABC}\Leftrightarrow \dfrac{1}{4}\cdot S\leq S_{\triangle AMN} \leq  \dfrac{1}{3}\cdot S$$
			Do đó $m = \dfrac{1}{4}$ và $n = \dfrac{1}{3}$ nên $T = \dfrac{1}{m} + \dfrac{1}{n} = 4 + 3 = 7$.
		}
	\end{ex}
	\Closesolutionfile{ans}
	\Closesolutionfile{ansbook}
	% % \indapan{10}{ans/ans-0D3-6-TN}
	
%%Chương 3
% \chap{HÀM SỐ VÀ ĐỒ THỊ}
\setcounter{section}{0}
\section{Hàm số}

\subsection{Tóm tắt lý thuyết}
\subsubsection{Hàm số và tập xác định của hàm số}
\begin{dn}{}
	Giả sử $x$ và $y$ là hai đại lượng biến thiên và $x$ nhận giá trị thuộc tập số $\mathscr{D}$.\\
	Nếu với mỗi giá trị của $x$ thuộc tập $\mathscr{D}$, ta xác định được một và chỉ một giá trị tương ứng $y$ thuộc tập số thực $\mathbb{R}$ thì ta có một \textbf{hàm số}. \\
	Ta gọi $x$ là \textbf{biến số} và $y$ là \textbf{hàm số} của $x$.\\
	Tập hợp $\mathscr{D}$ được gọi là \textbf{tập xác định} của hàm số.\\
	Tập hợp $T$ gồm tất cả các giá trị $y$ (tương ứng với $x$ thuộc $\mathscr{D}$) được gọi là \textbf{tập giá trị} của hàm số.
\end{dn}

\subsubsection{Cách cho hàm số}

\begin{listEX}[3]
	\item Cho bằng bảng
	\item Cho bằng biểu đồ
	\item Cho bằng công thức
\end{listEX}

\begin{note}
	Khi cho hàm số bằng công thức mà không chỉ rõ tập xác định của nó thì ta quy ước \textbf{tập xác định} của hàm số $y=f(x)$ là tập hợp tất cả các số thực $x$ để biểu thức $f(x)$ \textbf{có nghĩa}.
\end{note}

\subsubsection{Đồ thị của hàm số}
\begin{dn}{}
	Cho hàm số $y=f(x)$ có tập xác định $\mathscr{D}$. Trên mặt phẳng tọa độ $Oxy$, \textbf{đồ thị} $(C)$ của hàm số là tập hợp tất cả các điểm $M(x;y)$ với $x\in\mathscr{D}$ và $y=f(x)$.\\
	Vậy $(C)=\{M(x;f(x))\mid x\in\mathscr{D}\}$.
\end{dn}

Ta thường gặp trường hợp đồ thị của hàm số $y=f(x)$ là một đường (đường thẳng, đường cong,...). Khi đó, ta nói $y=f(x)$ là \textbf{phương trình} của đường đó.

\subsubsection{Sự biến thiên của hàm số}

\begin{dn}{}
	Hàm số $y=f(x)$ gọi là \textbf{đồng biến (tăng)} trên khoảng $(a;b)$ nếu
	$$\forall x_1, x_2\in (a;b), x_1 < x_2 \Rightarrow f(x_1) < f(x_2).$$
	Hàm số $y=f(x)$ gọi là \textbf{nghịch biến (giảm)} trên khoảng $(a;b)$ nếu
	$$\forall x_1, x_2\in (a;b), x_1 < x_2 \Rightarrow f(x_1) > f(x_2).$$
\end{dn}

%\begin{note}
%	\textbf{Xét chiều biến thiên của một hàm số} là tìm các khoảng đồng biến và các khoảng nghịch biến của hàm số đó.
%\end{note} 

\begin{note}
	Khi hàm số đồng biến trên $(a;b)$ thì đồ thị của nó có dạng đi lên từ trái sang phải.\\
	Khi hàm số nghịch biến trên $(a;b)$ thì đồ thị của nó có dạng đi xuống từ trái sang phải.
\end{note}
\begin{center}
	\begin{tikzpicture}[line cap=round, line join=round, scale=.6]
		\begin{axis}[
				legend pos=outer north east,
				xlabel = $x$,
				ylabel = $y$,
				axis lines = middle
			]
			\addplot [
				domain=-2:2,
				samples=100,
				%color=blue,
			]
			{-x^2+1.5};
			% \addlegendentry{$f(x)$}
		\end{axis}
	\end{tikzpicture}
	\begin{tikzpicture}[line join = round, line cap = round,>=stealth,font=\footnotesize,scale=1]
		\tkzTabInit[nocadre=false,lgt=1.2,espcl=2.5,deltacl=0.6]
		{$x$ /0.6, $f(x)$ /2}
		{$-\infty$,$0$,$+\infty$}
		\tkzTabVar{-/$ $,+/$ $,-/$ $}
	\end{tikzpicture}
\end{center}
\subsection{Các dạng toán}

\begin{dang}{Tập xác định, tập giá trị của hàm số}
\end{dang}

\viduminhhoa

\begin{vd}%[Thành Đức Trung]%[0D2B1-2]
	Tìm tập xác định của các hàm số sau
	\begin{enumerate}
		\begin{minipage}{0.5\linewidth}
			\item $y=\dfrac{\sqrt[3]{x^2-1}}{x^2+2x+3}$.
		\end{minipage} \begin{minipage}{0.5\linewidth}
			\item $y=\dfrac{x}{x-\sqrt{x}-6}$.
		\end{minipage}
		\begin{minipage}{0.5\linewidth}
			\item $y=\sqrt{x+2}-\sqrt{x+3}$.
		\end{minipage} \begin{minipage}{0.5\linewidth}
			\item $y=\heva{ & \dfrac{1}{x} & & \text{khi} \ x\geqslant1 \\ & \sqrt{1-x} & & \text{khi} \ x<0.}$
		\end{minipage}
	\end{enumerate}
	\loigiai{
		\begin{enumerate}
			\item Điều kiện xác định $x^2+2x+3\neq0$ đúng với mọi $x$. \\
			      Vậy tập xác định của hàm số là $\mathscr{D}=\mathbb{R}$.
			\item Điều kiện xác định $\heva{ & x\geqslant0 \\ & x-\sqrt x-6\neq0} \Leftrightarrow \heva{ & x\geqslant0 \\ &\sqrt{x}\neq-2 \\ &\sqrt{x}\neq3} \Leftrightarrow \heva{ & x\geqslant0 \\ & x \neq9.}$\\
			      Vậy tập xác định của hàm số là $\mathscr{D}=[0;+\infty)\setminus\{9\}$.
			\item Điều kiện xác định $\heva{ & x+2\geqslant0 \\ & x+3\geqslant0} \Leftrightarrow \heva{ & x\geqslant-2 \\ & x\geqslant-3} \Leftrightarrow x\geqslant-2$. \\
			      Vậy tập xác định của hàm số là $\mathscr{D}=[-2;+\infty)$.
			\item Khi $x\geqslant1$ thì hàm số là $y=\dfrac{1}{x}$ luôn xác định với $x\geqslant1$. \\
			      Khi $x<0$ thì hàm số là $y=\sqrt{1-x}$ luôn xác định với $x<0$. \\
			      Vậy tập xác định của hàm số là $\mathscr{D}=(-\infty;0)\cup[1;+\infty)$.
		\end{enumerate}
	}
\end{vd}

\begin{vd}%[Thành Đức Trung]%[0D1Y1-2]
	Cho bảng giá trị tương ứng của hai đại lượng $x$ và $y$. Đại lượng $y$ có là hàm số của đại lượng $x$ không? Nếu có, hãy tìm tập xác định và tập giá trị của hàm số đó.
	\begin{enumerate}
		\item
		      \begin{tabular}{|c|c|c|c|c|c|c|c|c|c|}
			      \hline
			      $x$ & $-5$ & $-3$ & $-1$ & $0$ & $1$ & $2$ & $5$ & $8$  & $9$  \\
			      \hline
			      $y$ & $-6$ & $-8$ & $-4$ & $1$ & $3$ & $2$ & $3$ & $12$ & $15$ \\
			      \hline
		      \end{tabular}
		\item
		      \begin{tabular}{|c|c|c|c|c|c|c|c|c|c|}
			      \hline
			      $x$ & $-10$ & $-8$  & $-4$ & $2$ & $3$ & $6$  & $7$  & $6$  & $13$ \\
			      \hline
			      $y$ & $-16$ & $-14$ & $-2$ & $4$ & $5$ & $20$ & $18$ & $24$ & $25$ \\
			      \hline
		      \end{tabular}
	\end{enumerate}
	\loigiai
	{
		\begin{enumerate}
			\item Đại lượng $y$ có là hàm số của đại lượng $x$ vì mỗi giá trị của $x$ có duy nhất một giá trị $y$ tương ứng. \\
			      Tập xác định là $\{-5;-3;-1;0;1;2;5;8;9\}$. \\
			      Tập giá trị là $\{-8;-6;-4;1;2;3;12;15\}$.
			\item Đại lượng $y$ không là hàm số của đại lượng $x$ vì với $x=6$ có hai giá trị $y=20$ và $y=24$.
		\end{enumerate}
	}
\end{vd}

\begin{vd}%[Thành Đức Trung]%[0D2B1-2]
	Tìm tất cả các giá trị thực của tham số $m$ để hàm số $y=\dfrac{x+2m+2}{x-m}$ xác định trên $(-1;0)$.
	\loigiai
	{
		Điều kiện $x-m\neq0 \Leftrightarrow x\neq m$. \\
		Hàm số xác định trên $(-1;0)$ khi và chỉ khi $m\notin(-1;0) \Leftrightarrow \hoac{ & m\leqslant-1 \\ & m\geqslant0.}$
	}
\end{vd}

\begin{vd}%[Thành Đức Trung]%[0D2K1-2]
	Tìm tất cả các giá trị thực của tham số $m$ để hàm số $ y=\sqrt{x-m+1}+\dfrac{2x}{\sqrt{-x+2m}}$ xác định trên khoảng $(-1;3)$.
	\loigiai
	{
	Điều kiện $\heva{ & x-m+1\geqslant0 \\ & -x+2m>0} \Leftrightarrow \heva{ & x\geqslant m-1 \\ & x<2m.}$ \\
	Ta cần $m-1<2m \Leftrightarrow m>-1$. \\
	Suy ra tập xác định là $\mathscr{D}=[m-1;2m)$. \\
	Hàm số xác định trên $(-1;3)$ khi $(-1;3)\subset\mathscr{D} \Leftrightarrow \Leftrightarrow \heva{ & m-1\leqslant-1 \\ & 2m\geqslant3} \Leftrightarrow \heva{ & m\leqslant0 \\ & m\geqslant\dfrac{3}{2}}$: vô nghiệm. \\
	Vậy không có giá trị $m$ thỏa mãn.
	}
\end{vd}

\begin{vd}%[Thành Đức Trung]%[0D2B1-1]
	Cho hàm số $f(x)=\heva{ & x-\sqrt{x^2+m^2} & & \text{khi} \ x<1 \\ & 2x & & \text{khi} \ x\geqslant1}$ với $m$ là tham số. Biết đồ thị hàm số cắt trục tung tại điểm có tung độ bằng $-3$. Tính giá trị biểu thức $P=f(-4)+f(1)$.
	\loigiai
	{
		Vì đồ thị hàm số cắt trục tung tại điểm có tung độ bằng $-3$ nên
		\[f(0)=-3 \Leftrightarrow -\sqrt{m^2}=-3 \Leftrightarrow m^2=9.\]
		Ta có $P=-4-\sqrt{\left(-4\right)^2+9}+2\cdot1=-7$.
	}
\end{vd}

\baitaptl

\begin{bt}%[Thành Đức Trung]%[0D2B1-2]
	Tìm tập xác định của các hàm số sau
	\begin{enumerate}[\indent a)]
		\begin{minipage}{0.5\linewidth}
			\item $y=-x^2$.
		\end{minipage} \begin{minipage}{0.5\linewidth}
			\item $y=\sqrt{2-3x}$.
		\end{minipage}
		\begin{minipage}{0.5\linewidth}
			\item $y=\dfrac{4}{x+1}$.
		\end{minipage} \begin{minipage}{0.5\linewidth}
			\item $y=\heva{ & 1 & & \text{nếu} \ x\in\mathbb{Q} \\ & 0 & & \text{nếu} \ x\in\mathbb{R}\setminus\mathbb{Q}.}$
		\end{minipage}
	\end{enumerate}
	\loigiai
	{
		\begin{enumerate}[\indent a)]
			\item Tập xác định của hàm số là $\mathscr{D}=\mathbb{R}$.
			\item Điều kiện xác định $2-3x\geqslant0 \Leftrightarrow x\leqslant\dfrac{2}{3}$. \\
			      Vậy tập xác định của hàm số là $\mathscr{D}=\left(-\infty;\dfrac{2}{3}\right]$.
			\item Điều kiện xác định $x+1\neq0 \Leftrightarrow x\neq-1$. \\
			      Vậy tập xác định của hàm số là $\mathscr{D}=\mathbb{R}\setminus\{-1\}$.
			\item Khi $x\in\mathbb{Q}$ thì hàm số là $y=1$ luôn xác định với $x\in\mathbb{Q}$. \\
			      Khi $x\in\mathbb{R}\setminus\mathbb{Q}$ thì hàm số là $y=0$ luôn xác định với $x\in\mathbb{R}\setminus\mathbb{Q}$. \\
			      Vậy tập xác định của hàm số là $\mathscr{D}=\mathbb{R}$.
		\end{enumerate}
	}
\end{bt}

\begin{bt}%[Thành Đức Trung]%[0D1Y1-2]
	Theo quyết định số 2019/QĐ-BĐVN ngày 01/11/2018 của Tổng công ty Bưu điện Việt Nam, giá cước dịch vụ Bưu chính phổ cập đối với dịch vụ thư cơ bản và bưu thiếp trong nước có khối lượng đến 250g như trong bảng sau
	\immini
	{
		\begin{enumerate}
			\item Số tiền dịch vụ thư cơ bản phải trả $y$ (đồng) có là hàm số của khối lượng thư cơ bản $x$ (g) hay không? Nếu đúng, hãy xác định những công thức tính $y$.
			\item Tính số tiền phải trả khi bạn Dương gửi thư có khối lượng $150$ g, $200$ g.
		\end{enumerate}
	}
	{
		\begin{tabular}{|l|c|}
			\hline
			Khối lượng đến $250$ g   & Mức cước (đồng) \\
			\hline
			Đến $20$ g               & $4 000$         \\
			\hline
			Trên $20$ g đến $100$ g  & $6 000$         \\
			\hline
			Trên $100$ g đến $250$ g & $8 000$         \\
			\hline
		\end{tabular}
	}
	\loigiai
	{
		\begin{enumerate}
			\item Đại lượng $y$ có là hàm số của đại lượng $x$ vì mỗi giá trị của $x$ có duy nhất một giá trị $y$ tương ứng. \\
			      Ta có $y=\heva{ & 4000 & & \text{nếu} \ x\leqslant20 \\ & 6000 & & \text{nếu} \ 20<x\leqslant100 \\ & 8000 & & \text{nếu} \ 100<x\leqslant250.}$ \\
			\item Số tiền bạn Dương phải trả khi gửi thư có khối lượng $150$ g là $8000$ đồng. \\
			      Số tiền bạn Dương phải trả khi gửi thư có khối lượng $200$ g là $8000$ đồng.
		\end{enumerate}
	}
\end{bt}

\begin{bt}%[Thành Đức Trung]%[0D2K1-2]
	Cho hàm số $y=\sqrt{2x-3m+4}+\dfrac{x}{x+m-1}$ với $m$ là tham số. Tìm $m$ để hàm số có tập xác định là $[0;+\infty)$.
	\loigiai
	{
	Điều kiện xác định $\heva{ & 2x-3m+4\geqslant0 \\ & x+m-1\neq0} \Leftrightarrow \heva{ & x\geqslant\dfrac{3m-4}{2} \\ & x\neq1-m.}$ \\
	Với $1-m\geqslant\dfrac{3m-4}{2} \Leftrightarrow m\leqslant\dfrac{6}{5}$, khi đó tập xác định của hàm số là $\mathscr{D}=\left[\dfrac{3m-4}2;+\infty\right)\setminus\{1-m\}$. \\
	Do đó $m\leqslant\dfrac{6}{5}$ không thỏa mãn yêu cầu bài toán. \\
	Với $m>\dfrac{6}{5}$ khi đó tập xác định của hàm số là $\mathscr{D}=\left[\dfrac{3m-4}2;+\infty\right)$.\\
	Do đó để hàm số có tập xác định là $[0;+\infty) \Leftrightarrow \dfrac{3m-4}{2}=0 \Leftrightarrow m=\dfrac{4}{3}$ (thỏa mãn). \\
	Vậy $m=\dfrac{4}{3}$ là giá trị cần tìm.
	}
\end{bt}

\begin{bt}%[Thành Đức Trung]%[0D2K1-2]
	Tìm tất cả các giá trị thực của tham số $m$ để hàm số $y=\dfrac{mx}{\sqrt{x-m+2}-1}$ xác định trên $(0;1)$.
	\loigiai
	{
	Điều kiện xác định $\heva{ & x-m+2\geqslant0 \\ & \sqrt{x-m+2}-1\neq0} \Leftrightarrow \heva{ & x\geqslant m-2 \\ & x\neq m-1.}$ \\
	Suy ra tập xác định $\mathscr{D}=[m-2;+\infty)\setminus\{m-1\}$. \\
	Hàm số xác định trên $(0;1)$ khi $(0;1)\subset\mathscr{D} \Leftrightarrow \heva{ & m-2\leqslant0 \\ & \hoac{ & m-1\leqslant0 \\ & m-1\geqslant1}} \Leftrightarrow \heva{ & m\leqslant2 \\ & \hoac{ & m\leqslant1 \\ & m\geqslant2}} \Leftrightarrow \hoac{ & m\leqslant1 \\ & m=2.}$
	}
\end{bt}

\begin{bt}%[Thành Đức Trung]%[0D2B1-1]
	Cho hàm số $f(x)=\heva{ & 2x+m & & \text{khi} \ x<3 \\ & x^2+4 & & \text{khi} \ x\geqslant3}$ với $m$ là tham số. Biết đồ thị hàm số cắt trục tung tại điểm có tung độ bằng $4$. Tính giá trị biểu thức $T=f(0)+f(10)$.
	\loigiai
	{
		Vì đồ thị hàm số cắt trục tung tại điểm có tung độ bằng $4$ nên $f(0)=4 \Leftrightarrow m=4$. \\
		Ta có $T=4+10^2+4=108$.
	}
\end{bt}


\begin{dang}{Tính đồng biến nghịch biến của hàm số}

\end{dang}
\viduminhhoa
\begin{vd}%[Bùi Mạnh Tiến]%[0D2B1-3]
	Xét tính đồng biến nghịch biến của hàm số
	\begin{enumerate}
		\item $y=f(x)=x^2-3x+2$ trên khoảng $\left(-\infty;
			      \dfrac{3}{2}\right)$;
		\item $y=g(x)=\dfrac{x-1}{x+1}$ trên khoảng $(-1;+\infty)$;
		\item $y=h(x)=\sqrt{4-3x}$ trên khoảng $\left(-\infty;\dfrac{4}{3}\right)$.
		\item $y=t(x)=|x-2|$ trên các khoảng $(-\infty;2)$ và $(2;+\infty)$.
	\end{enumerate}
	\loigiai
	{
		\begin{enumerate}
			\item Với mọi $x_1$, $x_2\in \left(-\infty;
				      \dfrac{3}{2}\right)$ và $x_1\neq x_2$ ta có $x_1<\dfrac{3}{2}$ và $x_2<\dfrac{3}{2}\Rightarrow x_1+x_2<3$. Khi đó
			      \begin{eqnarray*}
				      P&=&\dfrac{f(x_2)-f(x_1)}{x_2-x_1}\\
				      &=&\dfrac{x_2^2-3x_2+2-(x_1^2-3x_1+2)}{x_2-x_1}\\
				      &=&\dfrac{(x_2-x_1)(x_2+x_1-3)}{x_2-x_1}\\
				      &=&x_1+x_2-3<0.
			      \end{eqnarray*}
			      Do đó $y=f(x)$ là hàm số nghịch biến trên $\left(-\infty;\dfrac{3}{2}\right)$.
			\item Với mọi $x_1$, $x_2\in (-1;+\infty)$ và $x_1\neq x_2$ ta có $x_1>-1$, $x_2>-1\Rightarrow x_1+1>0$, $x_2+1>0$. Khi đó
			      \begin{eqnarray*}
				      P&=&\dfrac{g(x_2)-g(x_1)}{x_2-x_1}\\
				      &=&\dfrac{\dfrac{x_2-1}{x_2+1}-\dfrac{x_1-1}{x_1+1}}{x_2-x_1}\\
				      &=&\dfrac{\dfrac{(x_2-1)(x_1+1)-(x_1-1)(x_2+1)}{(x_2+1)(x_1+1)}}{x_2-x_1}\\
				      &=&\dfrac{2(x_2-x_1)}{(x_2+1)(x_1+1)(x_2-x_1)}\\
				      &=&\dfrac{2}{(x_2+1)(x_1+1)}>0.
			      \end{eqnarray*}
			      Vậy $y=g(x)$ là hàm đồng biến trên $(-1;+\infty)$.
			\item Với mọi $x_1$, $x_2\in \left(-\infty;\dfrac{4}{3}\right)$ và $x_1\neq x_2$ ta có
			      \begin{eqnarray*}
				      P&=&\dfrac{h(x_2)-h(x_1)}{x_2-x_1}\\
				      &=&\dfrac{\sqrt{4-3x_2}-\sqrt{4-3x_1}}{x_2-x_1}=\dfrac{4-3x_2-(4-3x_1)}{(x_2-x_1)(\sqrt{4-3x_2}+\sqrt{4-3x_1})}\\
				      &=&-\dfrac{3}{\sqrt{4-3x_2}+\sqrt{4-3x_1}}<0.
			      \end{eqnarray*}
			      Vậy hàm số đã cho nghịch biến trên khoảng $\left(-\infty;\dfrac{4}{3}\right)$.
			\item Xét biểu thức $P=\dfrac{t(x_2)-t(x_1)}{x_2-x_1}=\dfrac{|x_2-2|-|x_1-2|}{x_2-x_1}$.
			      \begin{itemize}
				      \item Với mọi $x_1$, $x_2\in (-\infty;2)$ và $x_1\neq x_2$ thì $x_1<2$, $x_2<2$ nên $|x_1-2|=2-x_1$ và $|x_2-2|=2-x_2$, do đó
				            \begin{align*}
					            P=\dfrac{2-x_2-(2-x_1)}{x_2-x_1}=\dfrac{x_1-x_2}{x_2-x_1}=-1<0.
				            \end{align*}
				      \item Với mọi $x_1$, $x_2\in (2;+\infty)$ và $x_1\neq x_2$ thì $x_1>2$, $x_2>2$ nên $|x_1-2|=x_1-2$ và $|x_2-2|=x_2-2$, do đó
				            \begin{align*}
					            P=\dfrac{x_2-2-(x_1-2)}{x_2-x_1}=\dfrac{x_2-x_1}{x_2-x_1}=1>0.
				            \end{align*}
			      \end{itemize}
			      Vậy hàm số $y=t(x)$ đồng biến trên khoảng $(2;+\infty)$ và nghịch biến trên khoảng $(-\infty;2)$.
		\end{enumerate}
	}
\end{vd}

\begin{vd}%[Bùi Mạnh Tiến]%[0D2B1-3]
	Tìm tất cả các giá trị của tham số $m$ để hàm số $y=f(x)=(1-3m)x+2m-2$ đồng biến trên tập xác định.
	\loigiai{
		Tập xác định: $\mathscr{D}=\mathbb{R}$.\\
		Gọi $x_1$, $x_2$ là hai giá trị phân biệt tùy ý thuộc $\mathbb{R}$, ta có
		\begin{align*}
			\dfrac{f(x_2)-f(x_1)}{x_2-x_1}=\dfrac{\left[(1-3m)x_2+2m-2\right]-\left[(1-3m)x_1+2m-2\right]}{x_2-x_1}=\dfrac{(1-3m)(x_2-x_1)}{x_2-x_1}=1-3m.
		\end{align*}
		Hàm số đồng biến trên $\mathbb{R}$ khi và chỉ khi $1-3m>0\Leftrightarrow m<\dfrac{1}{3}$.\\
		Vậy $m<\dfrac{1}{3}$.
	}
\end{vd}

\begin{vd}%[Bùi Mạnh Tiến]%[0D2B1-3]
	Cho hàm số $y=f(x)$ có đồ thị như hình vẽ bên dưới
	\begin{center}
		\begin{tikzpicture}[line join = round, line cap = round,>=stealth,font=\footnotesize,scale=1]
			\def\xmin{-2};
			\def\xmax{2};
			\def\ymin{-0.5};
			\def\ymax{4};
			\clip (\xmin,\ymin) rectangle (\xmax+0.15,\ymax+0.15);
			\draw[->] (\xmin,0) -- (\xmax,0) node[below] {$x$};
			\draw[->] (0,\ymin) -- (0,0) node[below left] {$O$} -- (0,\ymax) node[left] {$y$};
			\draw[smooth,samples=100] plot[domain=-1.7:1.7] (\x,{(\x)^4-2*(\x)^2+1});
			\fill (-1,0) circle (1.2pt) node[below] {$-1$};
			\fill (1,0) circle (1.2pt) node[below] {$1$};
		\end{tikzpicture}
	\end{center}
	Xác định các khoảng đồng biến và nghịch biến của hàm số.
	\loigiai{
		Từ đồ thị trên ta thấy
		\begin{itemize}
			\item hàm số đồng biến trên các khoảng $(-1;0)$ và $(1;+\infty)$.
			\item hàm số nghịch biến trên các khoảng $(-\infty;-1)$ và $(0;1)$.
		\end{itemize}
	}
\end{vd}

% \begin{vd}%[Bùi Mạnh Tiến]%[0D2B1-3]
% 	Tìm $m$ để hàm số $y=mx-\sqrt{2-m}$ đồng biến trên $\mathbb{R}$?
% 	\loigiai{
% 		Tập xác định $\mathscr{D}=\mathbb{R}$.\\
% 		Ta chỉ xét với $ 2-m\ge 0\Leftrightarrow m\le 2$. \quad (1)\\
% 		Với mọi $x_1$, $x_2\in \mathbb{R}$, $x_1\neq x_2$. Xét biểu thức
% 		\begin{align*}
% 			P=\dfrac{f(x_2)-f(x_1)}{x_2-x_1}=\dfrac{mx_2-\sqrt{2-m}-\left(mx_1-\sqrt{2-m}\right)}{x_2-x_1}=\dfrac{m(x_2-x_1)}{x_2-x_1}=m.
% 		\end{align*}
% 		Hàm số đồng biến trên $\mathbb{R}$ khi $m>0$. \quad (2)\\
% 		Từ $(1)$ và $(2)$ suy ra $0<m\leq 2$.
% 	}
% \end{vd}
%
%\begin{vd}
%	
%\end{vd}

\baitaptl

\begin{bt}%[Bùi Mạnh Tiến]%[0D2B1-3]
	Xét tính đồng biến nghịch biến của hàm số
	\begin{enumerate}
		\item $y=f(x)=\dfrac{-x+2}{x-1}$ trên khoảng $(-\infty;1)$ và $(1;+\infty)$.
		\item $y=g(x)=\dfrac{x-2}{2x-3}$ trên khoảng $\left(-\infty;\dfrac{3}{2}\right)$ và $\left(\dfrac{3}{2};+\infty\right)$.
	\end{enumerate}
	\loigiai
	{
		\begin{enumerate}
			\item Xét biểu thức $P=\dfrac{f(x_2)-f(x_1)}{x_2-x_1}$.
			      \begin{itemize}
				      \item Với mọi $x_1$, $x_2\in (-\infty;1)$ và $x_1\neq x_2$ thì $x_1<1$, $x_2<1$ do đó $(x_1-1)(x_2-1)>0$. Khi đó
				            \begin{align*}
					            P=\dfrac{\dfrac{-x_2+2}{x_2-1}-\dfrac{-x_1+2}{x_1-1}}{x_2-x_1}=\dfrac{-1}{(x_2-1)(x_1-1)}<0.
				            \end{align*}
				      \item Với mọi $x_1$, $x_2\in (1;+\infty)$ và $x_1\neq x_2$ thì $x_1>1$, $x_2>1$ do đó $(x_1-1)(x_2-1)>0$. Khi đó
				            \begin{align*}
					            P=\dfrac{\dfrac{-x_2+2}{x_2-1}-\dfrac{-x_1+2}{x_1-1}}{x_2-x_1}=\dfrac{-1}{(x_2-1)(x_1-1)}<0.
				            \end{align*}
			      \end{itemize}
			      Vậy hàm số $y=f(x)$ là hàm nghịch biến trên các khoảng $(-\infty;1)$ và $(1;+\infty)$.
			\item Xét biểu thức $P=\dfrac{g(x_2)-g(x_1)}{x_2-x_1}$.
			      \begin{itemize}
				      \item Với mọi $x_1$, $x_2\in \left(-\infty;\dfrac{3}{2}\right)$ và $x_1\neq x_2$ thì $x_1<\dfrac{3}{2}$, $x_2<\dfrac{3}{2}$ do đó $\left(x_1-\dfrac{3}{2}\right)\left(x_2-\dfrac{3}{2}\right)>0$. Khi đó
				            \begin{align*}
					            P=\dfrac{\dfrac{x_2-2}{2x_2-3}-\dfrac{x_1-2}{2x_1-3}}{x_2-x_1}=\dfrac{1}{(2x_2-3)(2x_1-3)}>0.
				            \end{align*}
				      \item Với mọi $x_1$, $x_2\in \left(\dfrac{3}{2};+\infty\right)$ và $x_1\neq x_2$ thì $x_1>\dfrac{3}{2}$, $x_2>\dfrac{3}{2}$ do đó $\left(x_1-\dfrac{3}{2}\right)\left(x_2-\dfrac{3}{2}\right)>0$. Khi đó
				            \begin{align*}
					            P=\dfrac{\dfrac{x_2-2}{2x_2-3}-\dfrac{x_1-2}{2x_1-3}}{x_2-x_1}=\dfrac{1}{(2x_2-3)(2x_1-3)}>0.
				            \end{align*}
			      \end{itemize}
		\end{enumerate}
	}
\end{bt}

\begin{bt}%[Bùi Mạnh Tiến]%[0D2B1-3]
	Dùng định nghĩa xét sự đồng biến nghịch biến của hàm số $y=f(x)=x^2+2x+2$ trên các khoảng $(-\infty;-1)$, $(-1;+\infty)$.
	\loigiai{\\
		Xét biểu thức
		\begin{align*}
			P=\dfrac{f(x_1)-f(x_2)}{x_1-x_2}=\dfrac{(x_1^2+2x_1+2)-(x_2^2+2x_2+2)}{x_1-x_2}=x_1+x_2+2.
		\end{align*}
		\begin{itemize}
			\item Trường hợp $x_1, x_2$ phân biệt cùng thuộc $(-\infty;-1)\Rightarrow x_1<-1$, $x_2<-1$ thì $x_1+x_2+2<0\Leftrightarrow P<0$, suy ra hàm số nghịch biến trên $(-\infty;-1)$.
			\item Trường hợp $x_1, x_2$ phân biệt cùng thuộc $(-1;+\infty)\Rightarrow x_1>-1$, $x_2>-1$ thì $P=x_1+x_2+2>0$, suy ra hàm số đồng biến trên $(-1;+\infty)$.
		\end{itemize}
	}
\end{bt}

% \begin{bt}%[Bùi Mạnh Tiến]%[0D2K1-3]
% 	Dùng định nghĩa xét sự đồng biến nghịch biến của hàm số $y=f(x)=\left|\sqrt{2-x}+1\right|$ trên khoảng $(-\infty;2)$
% 	\loigiai{\\
% 		Gọi $x_1, x_2$ là hai giá trị tùy ý thuộc $(-\infty;2)$, $x_1\neq x_2\Rightarrow 2-x_1>0$, $2-x_2>0$. Xét biểu thức
% 		\begin{eqnarray*}
% 			\dfrac{f(x_1)-f(x_2)}{x_1-x_2}&=&\dfrac{\left|\sqrt{2-x_1}+1\right|-\left|\sqrt{2-x_2}+1\right|}{x_1-x_2}\\
% 			&=&\dfrac{\sqrt{2-x_1}-\sqrt{2-x_2}}{x_1-x_2}\\
% 			&=&\dfrac{(2-x_1)-(2-x_2)}{(x_1-x_2)\left(\sqrt{2-x_1}+\sqrt{2-x_2}\right)}\\
% 			&=&\dfrac{-1}{\sqrt{2-x_1}+\sqrt{2-x_2}}<0.
% 		\end{eqnarray*}
% 		Vậy hàm số đã cho luôn nghịch biến trên khoảng $(-\infty;2)$.
% 	}
% \end{bt}

% \begin{bt}%[Bùi Mạnh Tiến]%[0D2K1-3]
% 	Dùng định nghĩa xét tính đơn điệu của hàm số $y=\dfrac{x}{x^2+1}$ trên các khoảng $(0;1)$, $(1;+\infty)$.
% 	\loigiai{
% 		Xét biểu thức
% 		\begin{eqnarray*}
% 			P&=&\dfrac{f(x_1)-f(x_2)}{x_1-x_2}\\
% 			&=&\dfrac{\dfrac{x_1}{x_1^2+1}-\dfrac{x_2}{x_2^2+1}}{x_1-x_2}\\
% 			&=&\dfrac{x_1(x_2^2+1)-x_2(x_1^2+1)}{(x_1-x_2)(x_1^2+1)(x_2^2+1)}\\
% 			&=&\dfrac{x_1x_2(x_2-x_1)-(x_2-x_1)}{(x_1-x_2)(x_1^2+1)(x_2^2+1)}\\
% 			&=&\dfrac{(1-x_1x_2)(x_1-x_2)}{(x_1-x_2)(x_1^2+1)(x_2^2+1)}\\
% 			&=&\dfrac{1-x_1x_2}{(x_1^2+1)(x_2^2+1)}.
% 		\end{eqnarray*}
% 		\begin{itemize}
% 			\item Trường hợp $x_1, x_2\in (0;1)$ suy ra $0<x_1$ ,$ x_2<1\Rightarrow P=1-x_1x_2>0$, từ đó ta có $$\dfrac{f(x_1)-f(x_2)}{x_1-x_2}>0.$$ \\
% 			      Vậy hàm số đã cho đồng biến trên khoảng $(0;1)$.
% 			\item Trường hợp $x_1, x_2\in (1;+\infty)$ suy ra $x_1$, $x_2>1\Rightarrow P=1-x_1x_2<0$.\\
% 			      Vậy hàm số đã cho nghịch biến trên khoảng $(1;+\infty)$.
% 		\end{itemize}
% 	}
% \end{bt}

\begin{bt}%[Bùi Mạnh Tiến]%[0D2K1-3]
	Tìm tất cả các giá trị của tham số $m$ để hàm số $y=f(x)=(2m-3)x+5-m$ nghịch biến trên tập xác định.
	\loigiai{
		Tập xác định: $\mathscr{D}=\mathbb{R}$.\\
		Gọi $x_1$, $x_2$ là hai giá trị phân biệt tùy ý thuộc $\mathbb{R}$, ta có
		\begin{align*}
			\dfrac{f(x_2)-f(x_1)}{x_2-x_1}=\dfrac{\left[(2m-3)x_2+5-m\right]-\left[(2m-3)x_1+5-m\right]}{x_2-x_1}=\dfrac{(2m-3)(x_2-x_1)}{x_2-x_1}=2m-3.
		\end{align*}
		Hàm số nghịch biến trên $\mathbb{R}$ khi và chỉ khi $2m-3<0\Leftrightarrow m<\dfrac{3}{2}$.\\
		Vậy $m<\dfrac{3}{2}$.
	}
\end{bt}

\begin{bt}%[Bùi Mạnh Tiến]%[0D2B1-3]
	Cho hàm số $y=f(x)$ có đồ thị như hình vẽ bên dưới
	\begin{center}
		\begin{tikzpicture}[line join = round, line cap = round,>=stealth,font=\footnotesize,scale=1]
			\def\xmin{-3.5};
			\def\xmax{3.5};
			\def\ymin{-1.5};
			\def\ymax{4.5};
			\clip (\xmin,\ymin) rectangle (\xmax+0.15,\ymax+0.15);
			\draw[->] (\xmin,0) -- (\xmax,0) node[below] {$x$};
			\draw[->] (0,\ymin) -- (0,0) node[below left] {$O$} -- (0,\ymax) node[left] {$y$};
			\draw (-3,-1) -- (-1,1) -- (1,0) -- (3,4);
			\draw[dashed] (-3,0) -- (-3,-1) -- (0,-1) (-1,0) -- (-1,1) -- (0,1) (3,0) -- (3,4) -- (0,4);
			\fill (-1,0) circle (1.2pt) node[below] {$-1$};
			\fill (-2,0) circle (1.2pt) node[below] {$-2$};
			\fill (-3,0) circle (1.2pt) node[above] {$-3$};
			\fill (0,-1) circle (1.2pt) node[right] {$-1$};
			\fill (0,1) circle (1.2pt) node[right] {$1$};
			\fill (0,4) circle (1.2pt) node[left] {$4$};
			\fill (1,0) circle (1.2pt) node[below] {$1$};
			\fill (3,0) circle (1.2pt) node[below] {$3$};
		\end{tikzpicture}
	\end{center}
	Hãy xác định các khoảng đồng biến và nghịch biến của hàm số trên $(-3;3)$.
	\loigiai
	{
		Dựa vào đồ thị ta thấy hàm số đã cho
		\begin{itemize}
			\item Đồng biến trên các khoảng $(-3,-1)$ và $(1;3)$;
			\item Nghịch biến trên các khoảng $(-1;1)$.
		\end{itemize}
	}
\end{bt}

\begin{bt}%[Bùi Mạnh Tiến]%[0D2K1-3]
	Cho hàm số $y=f(x)$ có bảng biến thiên như hình vẽ
	\begin{center}
		\begin{tikzpicture}[line join = round, line cap = round,>=stealth,font=\footnotesize,scale=1]
			\tkzTabInit[nocadre=false,lgt=1.2,espcl=2.5,deltacl=0.6]
			{$x$ /0.6, $f(x)$ /2}
			{$-\infty$,$0$,$4$,$+\infty$}
			\tkzTabVar{-/$-\infty$,+/$2$,-/$-32$,+/$+\infty$}
		\end{tikzpicture}
	\end{center}
	Chứng minh rằng hàm số $y=g(x)=5x-f(x)$ nghịch biến trên khoảng $(0;4)$.
	\loigiai
	{
		Từ bảng biến thiên ta thấy hàm số $y=f(x)$ nghịch biến trên $(0;4)$ nên với mỗi $x_1$, $x_2\in (0;4)$ và $x_1\neq x_2$ ta có
		\begin{align*}
			\dfrac{f(x_2)-f(x_1)}{x_2-x_1}<0.
		\end{align*}
		Với mỗi $x_1$, $x_2\in (0;4)$ và $x_1\neq x_2$ ta có
		\begin{align*}
			P=\dfrac{g(x_2)-g(x_1)}{x_2-x_1}=\dfrac{5x_2-f(x_2)-(5x_1-f(x_1))}{x_2-x_1}=5-\dfrac{f(x_2)-f(x_1)}{x_2-x_1}>5>0.
		\end{align*}
		Do đó $y=g(x)$ là hàm nghịch biến trên khoảng $(0;4)$.
	}
\end{bt}




\begin{dang}{Bài toán thực tế về hàm số}

\end{dang}
\viduminhhoa
%%==========Ví dụ 1
\begin{vd}%[Nguyễn Cường- BG Toán 10]%[0D2T2-5]
	Một cửa hàng bán sách online sẽ tính chi phí tiền vận chuyển sách khi mua sách như sau
	\begin{itemize}
		\item Số tiền mua sách không quá $2000000$ đồng thì phí vận chuyển là $50000$ đồng.
		\item Số tiền mua sách nhiều hơn $2000000$ đồng thì miễn phí vận chuyển.
	\end{itemize}
	\begin{enumerate}
		\item Viết công thức tính tổng chi phí $C$ mua sách của cửa hàng.
		\item Tính $C(1000000)$, $C(2000000)$, $C(2500000)$.
	\end{enumerate}
	\loigiai{
		\begin{enumerate}
			\item Viết công thức tính tổng chi phí $C$ mua sách của cửa hàng.\\
			      Gọi $x$ là số tiền mua sách. Ta có
			      $$C(x)=\heva{&x+50000&\text{nếu }x\le 2000000\\
					      &x&\text{nếu }x> 2000000.}$$
			\item Tính $C(1000000)$, $C(2000000)$, $C(2500000)$.\\
			      Ta có $C(1000000)=1000000$, $C(2000000)=2000000+50000=2050000$, $C(2500000)=2500000=2500000$.
		\end{enumerate}
	}
\end{vd}
%%==========Ví dụ 2
\begin{vd}%[Nguyễn Cường- BG Toán 10]%[0D2T2-5]
	Tốc độ quy định trên đường cao tốc là $50$ km/h đến $120$ km/h. Giả sử mức phạt tiền là $1000000$ đồng với mỗi km/h nếu tài xế chạy vượt quá tốc độ quy định hoặc dưới tốc độ quy định.
	\begin{enumerate}
		\item Hoàn thành hàm số $F(x)$ về quy định tiền phạt, với $x$ là tốc độ xe chạy.
		\item Tính $F(45)$, $F(60)$, $F(100)$, $F(125)$ và cho biết ý nghĩa của mỗi giá trị này.
	\end{enumerate}
	\loigiai{
		\begin{enumerate}
			\item Hoàn thành hàm số $F(x)$ về quy định tiền phạt, với $x$ km/h là tốc độ xe chạy.
			      $$F(x)=\heva{&1000000(x-120)&\text{nếu }x>120\\
					      &0&\text{nếu }50\le x\le 120\\
					      &1000000(50-x)&\text{nếu }x<50.}$$
			\item Tính $F(45)$, $F(60)$, $F(100)$, $F(125)$ và cho biết ý nghĩa của mỗi giá trị này.
			      \begin{itemize}
				      \item $F(45)=1000000(50-45)=5000000$ đồng, mức phạt $5000000$ đồng do xe chạy không đúng quy định tốc độ tối thiếu.
				      \item $F(60)=F(100)=0$ đồng, do xe đi trong giới hạn vận tốc cho phép.
				      \item $F(125)=1000000(125-120)=5000000$ đồng, mức phạt $5000000$ đồng do xe chạy không đúng quy định tốc độ tối đa.
			      \end{itemize}
		\end{enumerate}
	}
\end{vd}

%%==========Ví dụ 3
\begin{vd}%[Nguyễn Cường- BG Toán 10]%[0D2T2-5]
	Một khách sạn tại Đà Lạt cho thuê phòng với giá tiền $750000$ đồng một ngày đêm cho hai ngày đêm đầu tiên và $500000$ cho mỗi ngày đêm tiếp theo. Tổng số tiền $T$ cần phải trả là một hàm số của số ngày $x$ mà khách ở tại khách sạn.
	\begin{enumerate}
		\item Viết công thức của hàm số $T(x)$.
		\item Tính $T(2)$, $T(4)$, $T(6)$ và cho biết ý nghĩa của mỗi giá trị này.
	\end{enumerate}
	\loigiai{
		\begin{enumerate}
			\item Viết công thức của hàm số $T(x)$.
			      $$T(x)=\heva{&750000x&\text{nếu }1\le x\le 2\\
					      &1500000+500000(x-2)&\text{nếu }x\ge 3.}$$
			\item Tính $T(2)$, $T(4)$, $T(6)$ và cho biết ý nghĩa của mỗi giá trị này.
			      \begin{itemize}
				      \item $T(2)=750000\cdot 2=1500000$ đồng, khách thuê hai ngày đêm nên chi phí là $1500000$ đồng.
				      \item $T(4)=750000\cdot 2+500000(4-2)=2500000$ đồng, khách thuê bốn ngày đêm nên chi phí là $2500000$ đồng.
				      \item $T(6)=750000\cdot 2+500000(6-2)=3500000$ đồng, khách thuê sáu ngày đêm nên chi phí là $3500000$ đồng.
			      \end{itemize}
		\end{enumerate}
	}
\end{vd}
\baitaptl
%%==========Bài 1
\begin{bt}%[Nguyễn Cường- BG Toán 10]%[0D2T2-5]
	Một cửa hàng đồng loạt giảm giá các sản phẩm. Trong đó có chương trình nếu mua một gói kẹo thứ hai trở đi sẽ được giảm $10\%$ so với giá ban đầu là $50000$ đồng.
	\begin{enumerate}
		\item Nếu gọi số gói kẹo đã mua là $x$, số tiền phải trả là $y$. Hãy biểu diễn $y$ theo $x$.
		\item Bạn Thư muốn mua $10$ gói kẹo thì hết bao nhiêu tiền.
	\end{enumerate}

	\loigiai{
		\begin{enumerate}
			\item  Số tiền $y$ theo biến $x$ là $y = 90\%(x - 1)\cdot 50000 + 50000$.\\
			      Vậy $y = 45000x + 5000$.
			\item  Số tiền bạn Thư phải trả cho $10$ gói kẹo là
			      $y = 45000\cdot 10+5000= 455000$ đồng.
		\end{enumerate}
	}
\end{bt}

%%==========Bài 2
\begin{bt}%[Nguyễn Cường- BG Toán 10]%[0D2T2-5]
	Một cửa tiệm sách có một chính sách như sau: Nếu khách hàng đăng kí làm hội viên của cửa hàng thì mỗi năm phải đóng $50\;000$ đồng chi phí và được mướn sách với giá $5\;000$ đồng/cuốn, còn nếu khách hàng không phải hội viên thì sẽ mướn sách với giá $10\;000$ đồng/cuốn.
	Gọi $s$ (đồng) là tổng số tiền mỗi khách hàng phải trả trong một năm và $t$ là số cuốn sách mà khách hàng mướn.
	\begin{enumerate}
		\item Lập hàm số của s theo t đối với khách hàng là hội viên và đối với khách hàng không phải là hội viên.
		\item Trung là một hội viên của cửa hàng sách. Năm ngoái Trung đã trả cho cửa hàng sách tổng cộng $90\;000$ đồng. Hỏi nếu Trung không phải hội viên thì số tiền Trung phải trả là bao nhiêu?
	\end{enumerate}

	\loigiai{
		\begin{enumerate}
			\item Hàm số của $s$ theo $t$ đối với khách hàng là hội viên là $s=50\;000 + 5\;000t$.\\
			      Hàm số của $s$ theo $t$ đối với khách hàng không phải là hội viên là $s=10\;000t$.
			\item Thay $s=90\;000$ vào $s=50\;000 + 5\;000t$, ta được
			      \[50\;000 + 5\;000t = 90\;000 \Rightarrow t = 8.\]
			      Thay $t=8$ vào $s = 10\;000 t$, ta được
			      \[s = 10\;000 \cdot 8 = 80\;000.\]
			      Vậy nếu Trung không phải hội viên thì số tiền Trung phải trả là $80\;000$ đồng.
		\end{enumerate}

	}
\end{bt}

%%==========Bài 3
\begin{bt}%[Nguyễn Cường- BG Toán 10]%[0D2T2-5]
	Một người thuê nhà với giá $5000000$ đồng/tháng và người đó phải trả tiền dịch vụ giới thiệu là $1000000$ đồng (tiền dịch vụ chỉ trả $1$ lần). Gọi $x$ (tháng) là khoảng thời gian người đó thuê nhà, $y$ (đồng) là số tiền người đó phải tốn khi thuê nhà trong $x$ tháng.
	\begin{enumerate}
		\item Tìm một hệ thức liên hệ giữa $y$ và $x$.
		\item Tính số tiền người đó phải tốn sau khi ở $6$ tháng, $1$ năm.
	\end{enumerate}
	\loigiai{
		\begin{enumerate}
			\item Hệ thức liên hệ giữa $y$ và $x$ là $y = 5000000\cdot x + 1000000$.
			\item Số tiền mà người thuê nhà phải trả khi thuê nhà trong
			      \begin{itemize}
				      \item $6$ tháng: $y = 5000000\cdot 6 + 1000000 = 31000000$ (đồng).
				      \item $1$ năm: $y = 5000000\cdot 12 + 1000000 = 61000000$ (đồng).
			      \end{itemize}
		\end{enumerate}
	}
\end{bt}

%%==========Bài 4
\begin{bt}%[Nguyễn Cường- BG Toán 10]%[0D2T2-5]
	Một người vay ngân hàng $30~000~000$ (ba mươi triệu) đồng với lãi suất ngân hàng là $5 \%$ một năm và theo thể thức lãi đơn (tiền lãi không gộp vào chung với vốn).
	\begin{enumerate}
		\item Hãy thiết lập hàm số thể hiện mối liên hệ giữa tổng số tiền nợ $T$ (VNĐ) và số nợ (năm).
		\item Hãy cho biết sau 4 năm, người đó nợ ngân hàng tất cả bao nhiêu tiền?
	\end{enumerate}
	\loigiai{
		\begin{enumerate}
			\item Một người vay ngân hàng $30~000~000$ đồng với lãi suất $5\%$ một năm theo thể thức lãi đơn.
			      \begin{itemize}
				      \item Sau một năm người này nợ thêm: $30~000~000\cdot 5\%=1~500~000$ (đồng).
				      \item Sau $n$ năm người này nợ thêm: $1~500~000\cdot n$ (đồng).
			      \end{itemize}
			      Khi đó tổng số tiền người đó nợ sau $n$ năm là
			      \[1~500~000n+30~000~000\ (\text{đồng}). \]
			      Hàm số thể hiện mối liên hệ giữa tổng số tiền nợ $T$ (đồng) và số nợ $n$ (năm) là
			      \[T=1~500~000n+30~000~000. \]
			\item Thay $n=4$ vào công thức $T=1~500~000n+30~000~000$, ta được
			      \[T=1~500~000\cdot 4+30~000~000=36~000~000\ (\text{đồng}). \]
			      Vậy sau 4 năm người đó nợ $36~000~000$ đồng.
		\end{enumerate}
	}
\end{bt}


%%==========Bài 5
\begin{bt}%[Nguyễn Cường- BG Toán 10]%[0D2T2-5]
	Khách sạn A tại Đà Lạt có mức phí cho mỗi phòng được tính như sau: Mỗi phòng có giá là $300\ 000$ đồng/đêm, với thuế giá trị gia tăng là $8\%$. Do số lượng khách đến Đà Lạt vào dịp Tết tăng nhanh, khách sạn quyết định phụ thu thêm phí dịch vụ là $50\ 000$ đồng cho mỗi phòng và phí này chỉ thu một lần cố định.
	\begin{listEX}[1]
		\item Gọi $x$ là số đêm bạn An ở tại khách sạn A, $y$ là số tiền bạn An phải trả. Hãy viết biểu thức biểu diễn $y$ theo $x$.
		\item Biết bạn An phải trả tổng cộng $1\ 346\ 000$ đồng, hãy tính số đêm mà bạn An ở tại khách sạn A.
	\end{listEX}
	\dapso{a) $y=324\ 000x+50\ 000$, b) $4$ đêm.}
	\loigiai
	{
		\begin{enumerate}
			\item Số tiền bạn An phải trả là $y=300\ 000\cdot (100\%+8\%)x+50\ 000=324\ 000x+50\ 000$ đồng.
			\item Do bạn An phải trả tổng cộng $1\ 346\ 000$ nên $324\ 000x+50\ 000=1\ 346\ 000\Leftrightarrow x=4$.\\
			      Vậy An ở lại $4$ đêm.
		\end{enumerate}
	}
\end{bt}

%%==========Bài 6
\begin{bt}%[Nguyễn Cường- BG Toán 10]%[0D2T2-5]
	Một cửa hàng bán lại bánh $A$ như sau: nếu mua không quá $3$ hộp thì giá $35$ nghìn đồng mỗi hộp, nếu mua nhiều hơn $3$ hộp thì bắt đầu từ hộp thứ tư trở đi giá mỗi hộp sẽ giảm đi $20\%$ giá ban đầu.
	\begin{enumerate}
		\item Viết công thức tính $y$ (số tiền mua bánh) theo $x$ (số hộp bánh mua trong trường hợp nhiều hơn $3$ hộp).
		\item Lan và Hồng đều mua loại bánh $A$ với số hộp nhiều hơn $3$. Hỏi mỗi bạn mua bao nhiêu hộp biết rằng số hộp bánh Lan mua gấp đôi số hộp Hồng mua, đồng thời số tiền mua bánh của Lan nhiều hơn Hồng $140$ nghìn đồng.
	\end{enumerate}
	\loigiai
	{
		\begin{enumerate}
			\item Giá tiền mỗi hộp bánh khi giảm $20\%$ là $80\%\cdot 35\,000=28\,000$ đồng.\\
			      Giá tiền $3$ hộp bánh là $3\cdot 35\,000=105\,000$ đồng.\\
			      Công thức tính $y$ theo $x$ là \[y=28\,000(x-3)+105\,000\Leftrightarrow y=28000x-21000.\]
			\item Gọi $x$ (hộp) là số hộp bánh Hồng mua ($x>3$).\\
			      $2x$ (hộp) là số hộp bánh Lan mua.\\
			      Theo giải thiết, ta có
			      \allowdisplaybreaks
			      \begin{eqnarray*}
				      \left(28000\cdot 2x-21000\right)-\left(28000\cdot x-21000\right)=140000&\Leftrightarrow& 56000x-28000x=140000\\&\Leftrightarrow&28000x=140000\\
				      &\Leftrightarrow&x=5.
			      \end{eqnarray*}
			      Vậy số hộp bánh Hồng mua là $5$ hộp và số hộp bánh Lan mua là $10$ hộp.
		\end{enumerate}
	}
\end{bt}

%%==========Bài 7
\begin{bt}%[Nguyễn Cường- BG Toán 10]%[0D2T2-5]
	Một tiệm bánh có chương trình giảm $5\%$ trên tổng hóa đơn khi mua hàng chỉ trong ngày $09/01/2021$, bạn My mua $5$ hộp bánh bông lan cùng loại trong ngày $09/01/2021$, số tiền bạn phải trả là $37\,\, 250$ đồng. Ngày $12/01/2021$, bạn Uyên mua $6$ hộp bánh bông lan cùng loại với bạn My đã mua thì trả số tiền là $470\,\,000$ đồng. Biết số tiền phải trả (khi chưa có chương trình khuyến mãi) và số hộp bánh bông lan liên hệ bằng công thức: $y = ax + b$, $y$ (đồng) là số tiền phải trả và $x$ là số hộp bánh bông lan cùng loại.
	\begin{enumerate}
		\item Viết hàm số biểu diễn $y$ theo $x$.
		\item Hỏi vào ngày $12/01/2021$, bạn Nhân mua bao nhiêu hộp bánh bông lan cùng loại với bạn My? Biết số tiền Nhân trả là $320\,\,000$ đồng.
	\end{enumerate}
	\loigiai
	{
		\begin{enumerate}
			\item Bạn My mua $5$ hộp bánh bông lan cùng loại trong ngày $09/01/2021$, khi đó có chương trình khuyến mãi $5\%$ hóa đơn, số tiền bạn phải trả là $375\,\,250$ đồng nên ta có: $95\%(5a + b) = 375\,\,250$ hay $4{,}75a + 0{,}95b = 375\,\,250$. $\hfill (1)$\\
			      Ngày $12/01/2021$, bạn Uyên mua $6$ hộp bánh bông lan cùng loại với bạn Uyên thì trả số tiền là $470\,\,000$ đồng nên ta có: $6a + b = 470\,\,000$. $\hfill (2)$\\
			      Từ $(1)$ và $(2)$, ta có hệ phương trình
			      $$ \heva{&4{,}75a + 0{,}95b = 375\,\,250\\&6a + b = 470\,\,000} \Leftrightarrow \heva{&a = 75\,\,000\\&b = 20\,\,000.} $$
			      $\Rightarrow y = 75\,\,000x + 20\,\,000$.
			\item Bạn Nhân mua bánh vào ngày $12/01/2021$ nên không có chương trình khuyến mãi.\\
			      Vì bạn Nhân đã mua bánh hết $320\,\,000$ đồng nên $y = 320\,\,000$. Thay $y = 320\,\,000$ vào $y = 75\,\,000x + 20\,\,000$, ta được
			      $$ 320\,\,000 = 75\,\,000x + 20\,\,000 \Leftrightarrow x = 4. $$
			      Vậy bạn Nhân đã mua $4$ hộp.
		\end{enumerate}
	}
\end{bt}

%%==========Bài 8
\begin{bt}%[Nguyễn Cường- BG Toán 10]%[0D2T2-5]
	Nồng độ cồn trong máu $(BAC)$ được định nghĩa là phần trăm rượu (rượu ethyl hoặc ethanol) trong máu của một người. $BAC$ là $ 0{,}05 \%$ có nghĩa là có $0{,}05$ gam rượu trong $100 \mathrm{ml}$ máu. Càng uống nhiều rượu bia thì nồng độ cồn trong máu càng cao và càng nguy hiểm khi tham gia giao thông. Nồng độ $BAC$ trong máu của một người được thể hiện qua đồ thị sau:
	\begin{center}
		\begin{tikzpicture}[>=stealth, line join=round, line cap=round, font=\footnotesize, scale=1,yscale=.75]
			\draw[->] (-1,0)--(0,0)node[below left]{$O$}--(7,0)node[below]{$t$ (giờ)};
			\draw[->] (0,-1)--(0,5)node[left]{$BAC$ ($\%$)};
			\draw[dashed](3,0)node[below]{$1$}--(3,2)--(0,2)node[left]{$0{,}068$};
			\draw (0,3)node[left]{$0{,}076$}--(6,1);
			\fill[black](3,2)circle (1pt);
		\end{tikzpicture}
	\end{center}
	\begin{listEX}[1]
		\item	Viết công thức biểu thị mối quan hệ giữa nồng độ cồn trong máu $(BAC)$ sau $t$ giờ sử dụng.
		\item Theo nghị định 100/2019/ND-CP về xử phạt vi phạm hành chính, các mức phạt (đối với xe máy). Hỏi sau $3$ giờ, nếu người này tham gia giao thông thì sẽ bị xử phạt ở mức độ nào?
	\end{listEX}
	\begin{center}
		\begin{tabular}{|l|c|}
			\hline
			Mức $1$: Nồng độ cồn chưa vượt quá $50$mg/$100$ml máu        & Phạt tiền từ $02-03$ triệu đồng \\&(tước bằng từ $10-12$ tháng) \\
			\hline
			Mức $2$: Nồng độ cồn  vượt quá $50$mg đến $80$mg/$100$ml máu & Phạt tiền từ $04-05$ triệu đồng \\&(tước bằng từ $16-18$ tháng)\\
			\hline
			Mức $3$: Nồng độ cồn vượt quá $80$mg/$100$ml máu             & Phạt tiền từ $06-08$ triệu đồng \\&(tước bằng từ $22-24$ tháng) \\
			\hline
		\end{tabular}
	\end{center}
	\loigiai{
		\begin{enumerate}
			\item Dựa vào đồ thị ta gọi công thức biểu thị mối liên hệ giữa nồng độ cồn trong máu $(BAC)$ sau $t$ giờ sử dụng có công thức $BAC=at+b$.\\
			      Từ đồ thị ta có hàm số đi qua các điểm $(0;0{,}076)$ và $(1;0{,}068)$ nên ta được
			      $\heva{& BAC=0{,}076 \\ & a=-\dfrac{1}{125}.}$\\
			      Công thức biểu thị mối quan hệ giữa nồng độ cồn trong máu $(BAC)$ sau $t$ giờ sử dụng là $BAC=-\dfrac{1}{125}t+0{,}076$.
			\item Nồng độ cồn trong máu sau $3$ giờ là $BAC=-\dfrac{1}{125}\cdot 3+0{,}076=0{,}052$.\\
			      Do nồng độ cồn trong máu sau $3$ giờ là $0{,}052$mg/$100$ml máu nằm ở mức $2$ nên người này bị phạt tiền từ $04-05$ triệu đồng  và tước bằng từ $16-18$ tháng.
		\end{enumerate}
	}
\end{bt}

%%==========Bài 9
\begin{bt}%[Nguyễn Cường- BG Toán 10]%[0D2T2-5]
	Một cửa hàng cho thuê sách cũ có quy định: Nếu khách hàng là hội viên của cửa hàng thì phải đóng phí $70000$ đồng/năm và được thuê sách với giá $6000$ đồng/quyển, còn nếu khách hàng không là hội viên phải thuê sách với giá $10000$ đồng/quyển. Gọi $y$ (đồng) là tổng số tiền khách hàng phải trả trong một năm và $x$ là số quyển sách thuê trong một năm.
	\begin{listEX}
		\item Lập hàm số của $y$ theo $x$ với khách hàng là hội viên và với khách hàng không là hội viên của cửa hàng.
		\item Anh Nam là một hội viên của cửa hàng, năm vừa rồi anh Nam trả cho cửa hàng tổng cộng
		$322000$ đồng. Hỏi nếu anh Nam không là hội viên của cửa hàng thì năm vừa rồi anh phải trả
		cho cửa hàng bao nhiêu tiền?
	\end{listEX}
	\loigiai
	{
		\begin{enumerate}
			\item Đối với khách hàng hội viên ta có $y=70000+6000x$.\\
			      Đối với khách hàng không hội viên ta có $y=10000x$.
			\item Thế $y=322000$ vào $y=70000+6000x$, ta có $320000=70000+6000x \Leftrightarrow x=42$.\\
			      Thế $x=42$ vào $y=10000x$, ta có $y=420000$.\\
			      Vậy năm vừa rồi nếu không là hội viên anh Nam phải trả $420000$ đồng.
		\end{enumerate}
	}
\end{bt}

%%==========Bài 10
\begin{bt}%[Nguyễn Cường- BG Toán 10]%[0D2T2-5]
	Bạn Bình muốn mua một đôi giày thể thao mới. Hiện tại bạn đang có sẵn một số tiền nhưng không đủ để mua. Vì vậy bạn lên kế hoạch tiết kiệm tiền từ ngày 01/02/2020 đến ngày
	31/03/2020. Tháng Tư, Bình rủ An đến cửa hàng để mua giày. Sau khi mua giày xong, Bình mua hai thêm hai ly trà sữa với giá $30000$ đồng một ly thì Bình còn dư lại $60000$ đồng. Gọi $y$ (đồng) là số tiền bạn Bình có sẵn, $x$ (đồng) là số tiền bạn để dành mỗi ngày từ 01/02/2020 đến 31/03/2020.
	\begin{enumerate}
		\item Lập hàm số $y$ theo $x$ biết giá đôi giày bạn mua là $680000$ đồng.
		\item Biết số tiền bạn Bình có sẵn do ông bà lì xì Tết là $200000$ đồng. Hỏi để có tiền mua giày thì mỗi ngày Bình phải tiết kiệm bao nhiêu tiền?
	\end{enumerate}
	\loigiai{
		\begin{enumerate}
			\item Từ ngày 01/02/2020 đến 31/03/2020 có $60$ ngày, nên ta có\\
			      $y+60x=680000+2\cdot30000+60000\Rightarrow y=800000-60x$.
			\item Ta có $y=200000$ (đồng), mà $x=\dfrac{800000-y}{60}$ $\Rightarrow x=\dfrac{800000-20000}{60}=10000$ (đồng).\\
			      Vậy mỗi ngày Bình phải tiết kiệm $10000$ đồng để có tiền mua giày.
		\end{enumerate}
	}
\end{bt}


\subsection{Câu hỏi trắc nghiệm}

\Opensolutionfile{ansbook}[ans/ansbook-0D1-1-TN]
\Opensolutionfile{ans}[ans/ans-0D1-1-TN]
\begin{ex}%[Phan Anh]%[0D2B1-1]
	Điểm nào sau đây thuộc đồ thị hàm số $y=\dfrac{1}{x-1}$?
	\choice
	{\True $M_1(2;1)$}
	{$M_2(1;1)$}
	{$M_3(2;0)$}
	{$M_4(0;-2)$}
	\loigiai{
		Xét điểm $M_1$, thay $x=2$ và $y=1$
		vào hàm số $y=\dfrac{1}{x-1}$ ta được $1=\dfrac{1}{2-1}$ ta thấy đúng nên nhận $M_1$.}
\end{ex}
\begin{ex}%[Phan Anh]%[0D2B1-1]
	Điểm nào sau đây \textbf{không} thuộc đồ thị hàm số $y=\dfrac{\sqrt{x^2-4x+4}}{x}$?
	\choice
	{$A\left(2;0\right)$}
	{$B\left(3;\dfrac{1}{3}\right)$}
	{\True $C\left(1;-1\right)$}
	{$D\left(-1;-3\right)$}
	\loigiai{Thay từng đáp án vào hàm số $y=\dfrac{\sqrt{x^2-4x+4}}{x}$.
		\begin{itemize}
			\item Với $x=2$ và $y=0$, ta được $0=\dfrac{\sqrt{2^2-4.2+4}}{2}$ (đúng).
			\item Với $x=3$ và $y=\dfrac{1}{3}$, ta được $\dfrac{1}{3}=\dfrac{\sqrt{3^2-4\cdot3+4}}{3}$ (đúng).
			\item Với thay $x=1$ và $y=-1$, ta được $-1=\dfrac{\sqrt{1^2-4\cdot1+4}}{1}\Leftrightarrow-1=1$ (sai).
		\end{itemize}}
\end{ex}
\begin{ex}%[Phan Anh]%[0D2B1-1]
	Cho hàm số $y=f(x)=|-5x|$. Khẳng định nào sau đây là \textbf{sai}?
	\choice
	{$f(-1)=5$}
	{$f(2)=10$}
	{$f(-2)=10$}
	{\True $f\left(\dfrac{1}{5}\right)=-1$}
	\loigiai{Ta có
		\begin{itemize}
			\item $f(-1)=|-5\cdot(-1)|=|5|=5$.
			\item $f(2)=|-5\cdot2|=|-10|=10$.
			\item $f(-2)=|-5\cdot(-2)|=|10|=10$.
			\item $f\left(\dfrac{1}{5}\right)=\left|-5\cdot\dfrac{1}{5}\right|=|-1|=1$
		\end{itemize}
		Cách khác: Vì hàm đã cho là hàm trị tuyệt đối nên không âm. Do đó $f\left(\dfrac{1}{5}\right)=-1$ là sai.}
\end{ex}
\begin{ex}%[Phan Anh]%[0D2B1-1]
	Cho hàm số $f(x)=\left\{\begin{array}{*{35}{l}}
			\dfrac{2}{x-1} & , x\in(-\infty;0) \\
			\sqrt{x+1}     & , x\in[0;2]       \\
			x^2-1          & , x\in(2;5]
		\end{array}\right.$. Tính giá trị của $f(4)$.
	\choice
	{$f(4)=\dfrac{2}{3}$}
	{\True $f(4)=15$}
	{$f(4)=\sqrt{5}$}
	{Không tính được}
	\loigiai{Do $4\in(2;5]$ nên $f(4)=4^2-1=15$.}
\end{ex}
\begin{ex}%[Phan Anh]%[0D2B1-1]
	Cho hàm số $f(x)=\left\{\begin{array}{*{35}{l}}
			\dfrac{2\sqrt{x+2}-3}{x-1} & , x\ge 2 \\
			x^2 +1                     & , x<2
		\end{array}\right.$. Tính $P=f(2)+f(-2)$.
	\choice
	{$P=\dfrac{8}{3}$}
	{$P=4$}
	{\True $P=6$}
	{$P=\dfrac{5}{3}$}
	\loigiai{\begin{itemize}
			\item Khi $x\ge 2$ thì $f(2)=\dfrac{2\sqrt{2+2}-3}{2-1}=1$.
			\item Khi $x<2$ thì $f(-2)=(-2)^2+1=5$.
		\end{itemize}
		Vậy $f(2)+f(-2)=6$.}
\end{ex}
\begin{ex}%[Phan Anh]%[0D2B1-2]
	Tìm tập xác định $\mathscr{D}$ của hàm số $y=\dfrac{3x-1}{2x-2}$.
	\choice
	{$\mathscr{D}=\mathbb{R}$}
	{$\mathscr{D}=(1;+\infty)$}
	{\True $\mathscr{D}=\mathbb{R}\setminus\{1\}$}
	{$\mathscr{D}=[1;+\infty)$}
	\loigiai{
		Hàm số xác định khi $2x-2\ne0\Leftrightarrow x\ne1$.\\
		Vậy tập xác định của hàm số là $\mathscr{D}=\mathbb{R}\setminus\{1\}$.}
\end{ex}
\begin{ex}%[Phan Anh]%[0D2B1-2]
	Tìm tập xác định $\mathscr{D}$ của hàm số $y=\dfrac{2x-1}{(2x+1)(x-3)}$.
	\choice
	{$\mathscr{D}=(3;+\infty)$}
	{\True $\mathscr{D}=\mathbb{R}\setminus\left\{-\dfrac{1}{2};3\right\}$}
	{$\mathscr{D}=\left(-\dfrac{1}{2};+\infty\right)$}
	{$\mathscr{D}=\mathbb{R}$}
	\loigiai{
		Hàm số xác định khi $\heva{
				& 2x+1\ne 0 \\
				& x-3\ne 0}\Leftrightarrow \heva{
				& x\ne-\dfrac{1}{2} \\
				& x\ne 3.}$\\
		Vậy tập xác định của hàm số là $ \mathscr{D}=\mathbb{R}\setminus\left\{-\dfrac{1}{2};3\right\}$}
\end{ex}
\begin{ex}%[Phan Anh]%[0D2B1-2]
	Tìm tập xác định $\mathscr{D}$ của hàm số $y=\dfrac{x^2+1}{x^2+3x-4}$.
	\choice
	{$\mathscr{D}=\{1;-4\}$}
	{\True $\mathscr{D}=\mathbb{R}\setminus\{1;-4\}$}
	{$\mathscr{D}=\mathbb{R}\setminus\{1;4\}$}
	{$\mathscr{D}=\mathbb{R}$}
	\loigiai{
		Hàm số xác định khi $x^2+3x-4\ne 0\Leftrightarrow \heva{
				& x\ne 1 \\
				& x\ne-4}.$\\
		Vậy tập xác định của hàm số là $\mathscr{D}=\mathbb{R}\setminus\{1;-4\}$.}
\end{ex}
\begin{ex}%[Phan Anh]%[0D2B1-2]
	Tìm tập xác định $\mathscr{D}$ của hàm số $y=\dfrac{x+1}{(x+1)(x^2+3x+4)}$.
	\choice
	{$\mathscr{D}=\mathbb{R}\setminus\left\{1\right\}$}
	{$\mathscr{D}=\left\{-1\right\}$}
	{\True $\mathscr{D}=\mathbb{R}\setminus\left\{-1\right\}$}
	{$\mathscr{D}=\mathbb{R}$}
	\loigiai{
		Hàm số xác định khi $\heva{
				& x+1\ne 0 \\
				& x^2+3x+4\ne 0}\Leftrightarrow x\ne-1$.\\
		Vậy tập xác định của hàm số là $\mathscr{D}=\mathbb{R}\setminus\left\{-1\right\}$.}
\end{ex}
\begin{ex}%[Phan Anh]%[0D2B1-2]
	Tìm tập xác định $\mathscr{D}$ của hàm số $y=\dfrac{2x+1}{x^3-3x+2}$.
	\choice
	{$\mathscr{D}=\mathbb{R}\setminus\left\{1;2\right\}$}
	{\True $\mathscr{D}=\mathbb{R}\setminus\left\{-2;1\right\}$}
	{$\mathscr{D}=\mathbb{R}\setminus\left\{-2\right\}$}
	{$\mathscr{D}=\mathbb{R}$}
	\loigiai{
		Hàm số xác định khi
		\begin{align*}
			                & x^3-3x+2\ne 0
			\Leftrightarrow (x-1)(x^2+x-2)\ne 0                         \\
			\Leftrightarrow & \,\heva{                       & x-1\ne 0 \\
			                & x^2+x-2\ne 0}
			\Leftrightarrow \heva{
			                & x\ne 1                                    \\ & \heva{ & x\ne 1 \\
			                & x\ne-2}}\Leftrightarrow \heva{
			                & x\ne                                      \\
			                & x\ne-2.}
		\end{align*}
		Vậy tập xác định của hàm số là $\mathscr{D}=\mathbb{R}\setminus\left\{-2;1\right\}$.
	}
\end{ex}
\begin{ex}%[Phan Anh]%[0D2B1-2]
	Tìm tập xác định $\mathscr{D}$ của hàm số $y=\sqrt{x+2}-\sqrt{x+3}$.
	\choice
	{$\mathscr{D}=[-3;+\infty)$}
	{\True $\mathscr{D}=\left[-2;+\infty \right)$}
	{$\mathscr{D}=\mathbb{R}$}
	{$\mathscr{D}=\left[2;+\infty \right)$}
	\loigiai{
	Hàm số xác định khi $\heva{
			& x+2\ge 0 \\
			& x+3\ge 0 \\}\Leftrightarrow \heva{
			& x\ge-2 \\
			& x\ge-3 \\}\Leftrightarrow x\ge-2$.\\
	Vậy tập xác định của hàm số là $\mathscr{D}=\left[-2;+\infty \right)$.}
\end{ex}
\begin{ex}%[Phan Anh]%[0D2B1-2]
	Tìm tập xác định $\mathscr{D}$ của hàm số $y=\sqrt{6-3x}-\sqrt{x-1}$.
	\choice
	{$\mathscr{D}=\left(1;2\right)$}
	{\True $\mathscr{D}=\left[1;2\right]$}
	{$\mathscr{D}=\left[1;3\right]$}
	{$\mathscr{D}=\left[-1;2\right]$}
	\loigiai{
		Hàm số xác định khi $\heva{
				& 6-3x\ge 0 \\
				& x-1\ge 0}\Leftrightarrow \heva{
				& x\le 2 \\
				& x\ge 1}\Leftrightarrow 1\le x\le 2$.\\
		Vậy tập xác định của hàm số là $\mathscr{D}=\left[1;2\right]$.}
\end{ex}
\begin{ex}%[Phan Anh]%[0D2B1-2]
	Tìm tập xác định $\mathscr{D}$ của hàm số $y=\dfrac{\sqrt{3x-2}+6x}{\sqrt{4-3x}}$.
	\choice
	{\True $\mathscr{D}=\left[\dfrac{2}{3};\dfrac{4}{3}\right)$}
	{$\mathscr{D}=\left[\dfrac{3}{2};\dfrac{4}{3}\right)$}
	{$\mathscr{D}=\left[\dfrac{2}{3};\dfrac{3}{4}\right)$}
	{$\mathscr{D}=\left(-\infty;\dfrac{4}{3}\right)$}
	\loigiai{
	Hàm số xác định khi $\heva{
			& 3x-2\ge 0 \\
			& 4-3x>0}\Leftrightarrow \heva{
			& x\ge \dfrac{2}{3} \\
			& x<\dfrac{4}{3}}\Leftrightarrow \dfrac{2}{3}\le x<\dfrac{4}{3}$.\\
	Vậy tập xác định của hàm số là $\mathscr{D}=\left[\dfrac{2}{3};\dfrac{4}{3}\right)$.}
\end{ex}
\begin{ex}%[Phan Anh]%[0D2B1-2]
	Tìm tập xác định $\mathscr{D}$ của hàm số $y=\dfrac{x+4}{\sqrt{x^2-16}}$.
	\choice
	{$\mathscr{D}=\left(-\infty;-2\right)\cup \left(2;+\infty \right)$}
	{$\mathscr{D}=\mathbb{R}$}
	{\True $\mathscr{D}=\left(-\infty;-4\right)\cup \left(4;+\infty \right)$}
	{$\mathscr{D}=\left(-4;4\right)$}
	\loigiai{Hàm số xác định khi $x^2-16>0\Leftrightarrow x^2>16\Leftrightarrow \hoac{
				& x>4 \\
				& x<-4}$.\\
		Vậy tập xác định của hàm số là $\mathscr{D}=\left(-\infty;-4\right)\cup \left(4;+\infty \right)$.}
\end{ex}
\begin{ex}%[Phan Anh]%[0D2B1-2]
	Tìm tập xác định $\mathscr{D}$ của hàm số $y=\sqrt{x^2-2x+1}+\sqrt{x-3}$.
	\choice
	{$\mathscr{D}=(-\infty;3]$}
	{$\mathscr{D}=[1;3]$}
	{\True $\mathscr{D}=[3;+\infty)$}
	{$\mathscr{D}=(3;+\infty)$}
	\loigiai{
	Hàm số xác định khi $\heva{
			& x^2-2x+1\ge 0 \\
			& x-3\ge 0}\Leftrightarrow \heva{
			& {\left(x-1\right)}^2\ge 0 \\
			& x-3\ge 0}\Leftrightarrow \heva{
			& x\in \mathbb{R} \\
			& x\ge 3}\Leftrightarrow x\ge 3$.\\
	Vậy tập xác định của hàm số là $\mathscr{D}=\left[3;+\infty \right)$.}
\end{ex}
\begin{ex}%[Phan Anh]%[0D2B1-2]
	Tìm tập xác định $\mathscr{D}$ của hàm số $y=\dfrac{\sqrt{2-x}+\sqrt{x+2}}{x}$.
	\choice
	{$\mathscr{D}=[-2;2]$}
	{$\mathscr{D}=(-2;2)\setminus\left\{0\right\}$}
	{\True $\mathscr{D}=[-2;2]\setminus\left\{0\right\}$}
	{$\mathscr{D}=\mathbb{R}$}
	\loigiai{
		Hàm số xác định khi $\heva{
				& 2-x\ge 0 \\
				& x+2\ge 0 \\
				& x\ne 0}\Leftrightarrow \heva{
				& x\le 2 \\
				& x\ge-2 \\
				& x\ne 0.}$\\
		Vậy tập xác định của hàm số là $\mathscr{D}=\left[-2;2\right]\setminus\left\{0\right\}$.}
\end{ex}
\begin{ex}%[Phan Anh]%[0D2B1-2]
	Tìm tập xác định $\mathscr{D}$ của hàm số $y=\dfrac{\sqrt{x+1}}{x^2-x-6}$.
	\choice
	{$\mathscr{D}=\left\{3\right\}$}
	{\True $\mathscr{D}=\left[-1;+\infty \right)\setminus\left\{3\right\}$}
	{$\mathscr{D}=\mathbb{R}$}
	{$\mathscr{D}=\left[-1;+\infty \right)$}
	\loigiai{
	Hàm số xác định khi $\heva{
			& x+1\ge 0 \\
			& x^2-x-6\ne 0}\Leftrightarrow \heva{
			& x\ge-1 \\
			& x\ne 3 \\
			& x\ne-2}\Leftrightarrow \heva{
			& x\ge-1 \\
			& x\ne 3.}$\\
	Vậy tập xác định của hàm số là $\mathscr{D}=[-1;+\infty)\setminus\left\{3\right\}$.}
\end{ex}
\begin{ex}%[Phan Anh]%[0D2B1-2]
	Tìm tập xác định $\mathscr{D}$ của hàm số $y=\sqrt{6-x}+\dfrac{2x+1}{1+\sqrt{x-1}}$.
	\choice
	{$\mathscr{D}=(1;+\infty)$}
	{\True $\mathscr{D}=[1;6]$}
	{$\mathscr{D}=\mathbb{R}$}
	{$\mathscr{D}=(1;6)$}
	\loigiai{
	Hàm số xác định khi $\heva{
			& 6-x\ge 0 \\
			& x-1\ge 0 \\
			& 1+\sqrt{x-1}\ne 0\left(\text{luôn đúng} \right)}\Leftrightarrow \heva{
			& x\le 6 \\
			& x\ge 1}\Leftrightarrow 1\le x\le 6$.\\
	Vậy tập xác định của hàm số là $\mathscr{D}=[1;6]$.}
\end{ex}
\begin{ex}%[Phan Anh]%[0D2B1-2]
	Tìm tập xác định $\mathscr{D}$ của hàm số $y=\dfrac{x+1}{(x-3)\sqrt{2x-1}}$.
	\choice
	{$\mathscr{D}=\mathbb{R}$}
	{$\mathscr{D}=\left(-\dfrac{1}{2};+\infty \right)\setminus\left\{3\right\}$}
	{$\mathscr{D}=\left[\dfrac{1}{2};+\infty \right)\setminus\left\{3\right\}$}
	{\True $\mathscr{D}=\left(\dfrac{1}{2};+\infty \right)\setminus\left\{3\right\}$}
	\loigiai{
		Hàm số xác định khi $\heva{
				& x-3\ne 0 \\
				& 2x-1>0}\Leftrightarrow \heva{
				& x\ne 3 \\
				& x>\dfrac{1}{2}.}$\\
		Vậy tập xác định của hàm số là $\mathscr{D}=\left(\dfrac{1}{2};+\infty \right)\setminus\left\{3\right\}$.}
\end{ex}
\begin{ex}%[Phan Anh]%[0D2B1-2]
	Tìm tập xác định $\mathscr{D}$ của hàm số $y=\dfrac{\sqrt{x+2}}{x\sqrt{x^2-4x+4}}$.
	\choice
	{\True $\mathscr{D}=[-2;+\infty)\setminus\left\{0;2\right\}$}
	{$\mathscr{D}=\mathbb{R}$}
	{$\mathscr{D}=[-2;+\infty)$}
	{$\mathscr{D}=(-2;+\infty)\setminus\left\{0;2\right\}$}
	\loigiai{
	Hàm số xác định khi $\heva{
			& x+2\ge 0 \\
			& x\ne 0 \\
			& x^2-4x+4>0}\Leftrightarrow \heva{
			& x+2\ge 0 \\
			& x\ne 0 \\
			& (x-2)^2>0}\Leftrightarrow \heva{
			& x\ge-2 \\
			& x\ne 0 \\
			& x\ne 2.}$\\
	Vậy tập xác định của hàm số là $\mathscr{D}=\left[-2;+\infty \right)\setminus\left\{0;2\right\}$.}
\end{ex}
%Câu 21
\begin{ex}%[Phan Anh]%[0D2B1-2]
	Tìm tập xác định $\mathscr{D}$ của hàm số $y=\dfrac{x}{x-\sqrt{x}-6}$.
	\choice
	{$\mathscr{D}=\left[0;+\infty \right)\setminus\left\{3\right\}$}
	{\True $\mathscr{D}=\left[0;+\infty \right)\setminus\left\{9\right\}$}
	{$\mathscr{D}=\left[0;+\infty \right)\setminus\left\{\sqrt{3}\right\}$}
	{$\mathscr{D}=\mathbb{R}\setminus\left\{9\right\}$}
	\loigiai{
	Hàm số xác định khi $\heva{
			& x\ge 0 \\
			& x-\sqrt{x}-6\ne 0}\Leftrightarrow \heva{
			& x\ge 0 \\
			& \sqrt{x}\ne 3}\Leftrightarrow \heva{
			& x\ge 0 \\
			& x\ne 9.}$\\
	Vậy tập xác định của hàm số là $\mathscr{D}=\left[0;+\infty \right)\setminus\left\{9\right\}$.}
\end{ex}
\begin{ex}%[Phan Anh]%[0D2B1-2]
	Tìm tập xác định $\mathscr{D}$ của hàm số $y=\dfrac{\sqrt[3]{x-1}}{x^2+x+1}$.
	\choice
	{$\mathscr{D}=\left(1;+\infty \right)$}
	{$\mathscr{D}=\left\{1\right\}$}
	{\True $\mathscr{D}=\mathbb{R}$}
	{$\mathscr{D}=\left(-1;+\infty \right)$}
	\loigiai{
		Hàm số xác định khi $x^2+x+1\ne 0$ luôn đúng với mọi $x\in \mathbb{R}$.\\
		Vậy tập xác định của hàm số là $\mathscr{D}=\mathbb{R}$.}
\end{ex}
\begin{ex}%[Phan Anh]%[0D2B1-2]
	Tìm tập xác định $\mathscr{D}$ của hàm số $y=\dfrac{\sqrt{x-1}+\sqrt{4-x}}{\left(x-2\right)\left(x-3\right)}$.
	\choice
	{$\mathscr{D}=\left[1;4\right]$}
	{$\mathscr{D}=\left(1;4\right)\setminus\left\{2;3\right\}$}
	{\True $\mathscr{D}=\left[1;4\right]\setminus\left\{2;3\right\}$}
	{$\mathscr{D}=\left(-\infty;1\right]\cup \left[4;+\infty \right)$}
	\loigiai{
		Hàm số xác định khi $\heva{
				& x-1\ge 0 \\
				& 4-x\ge 0 \\
				& x-2\ne 0 \\
				& x-3\ne 0}\Leftrightarrow \heva{
				& x\ge 1 \\
				& x\le 4 \\
				& x\ne 2 \\
				& x\ne 3}\Leftrightarrow \heva{
				& 1\le x\le 4 \\
				& x\ne 2 \\
				& x\ne 3.}$\\
		Vậy tập xác định của hàm số là $\mathscr{D}=\left[1;4\right]\setminus\left\{2;3\right\}$.}
\end{ex}
\begin{ex}%[Phan Anh]%[0D2B1-2]
	Tìm tập xác định $\mathscr{D}$ của hàm số $y=\sqrt{\sqrt{x^2+2x+2}-(x+1)}$.
	\choice
	{$\mathscr{D}=\left(-\infty;-1\right)$}
	{$\mathscr{D}=\left[-1;+\infty \right)$}
	{$\mathscr{D}=\mathbb{R}\setminus\left\{-1\right\}$}
	{\True $\mathscr{D}=\mathbb{R}$}
	\loigiai{
		Hàm số xác định khi $\begin{aligned}[t]
				                & \sqrt{x^2+2x+2}-(x+1)\ge 0\Leftrightarrow \sqrt{(x+1)^2+1}\ge x+1 \\
				\Leftrightarrow & \, \hoac{
				                & \heva{
				                & x+1<0                                                             \\
				                & (x+1)^2+1\ge 0}                                                   \\
				                & \heva{
				                & x+1\ge 0                                                          \\
				                & (x+1)^2+1\ge(x+1)^2}}\Leftrightarrow \hoac{
				                & x+1<0                                                             \\
				                & x+1\ge 0}\Leftrightarrow x\in \mathbb{R}.
			\end{aligned}$\\
		Vậy tập xác định của hàm số là $\mathscr{D}=\mathbb{R}$.}
\end{ex}
\begin{ex}%[Phan Anh]%[0D2B1-2]
	Tìm tập xác định $\mathscr{D}$ của hàm số $y=\dfrac{2018}{\sqrt[3]{x^2-3x+2}-\sqrt[3]{x^2-7}}$.
	\choice
	{\True $\mathscr{D}=\mathbb{R}\setminus\left\{3\right\}$}
	{$\mathscr{D}=\mathbb{R}$}
	{$\mathscr{D}=\left(-\infty;1\right)\cup \left(2;+\infty \right)$}
	{$\mathscr{D}=\mathbb{R}\setminus\left\{0\right\}$}
	\loigiai{
		Hàm số xác định khi $\begin{aligned}[t]
				                & \sqrt[3]{x^2-3x+2}-\sqrt[3]{x^2-7}\ne 0\Leftrightarrow \sqrt[3]{x^2-3x+2}\ne \sqrt[3]{x^2-7} \\
				\Leftrightarrow & \,x^2-3x+2\ne x^2-7\Leftrightarrow 9\ne 3x\Leftrightarrow x\ne 3.
			\end{aligned}$\\
		Vậy tập xác định của hàm số là $\mathscr{D}=\mathbb{R}\setminus\left\{3\right\}$.}
\end{ex}
\begin{ex}%[Phan Anh]%[0D2K1-2]
	Tìm tập xác định $\mathscr{D}$ của hàm số $y=\dfrac{|x|}{|x-2|+\left|x^2+2x\right|}$.
	\choice
	{\True $\mathscr{D}=\mathbb{R}$}
	{$\mathscr{D}=\mathbb{R}\setminus\left\{-2;0\right\}$}
	{$\mathscr{D}=\mathbb{R}\setminus\left\{-2;0;2\right\}$}
	{$\mathscr{D}=\left(2;+\infty \right)$}
	\loigiai{
		Hàm số xác định khi $|x-2|+\left|x^2+2x\right|\ne0$.\\
		Xét phương trình $|x-2|+\left|x^2+2x\right|=0\Leftrightarrow \heva{
				& |x-2|=0 \\
				& \left|x^2+2x\right|=0}\Leftrightarrow \heva{
				& x=2 \\
				& x=0\vee x=-2.}$\\
		Vậy không có giá trị $x$ làm cho $|x-2|+\left| x^2+2x\right|=0$, do đó $|x-2|+\left| x^2+2x\right|\ne 0$ đúng với mọi $x\in \mathbb{R}$. Vậy tập xác định của hàm số là $\mathscr{D}=\mathbb{R}$.}
\end{ex}
\begin{ex}%[Phan Anh]%[0D2K1-2]
	Tìm tập xác định $\mathscr{D}$ của hàm số $y=\dfrac{2x-1}{\sqrt{x|x-4|}}$.
	\choice
	{$\mathscr{D}=\mathbb{R}\setminus\left\{0;4\right\}$}
	{$\mathscr{D}=\left(0;+\infty \right)$}
	{$\mathscr{D}=\left[0;+\infty \right)\setminus\left\{4\right\}$}
	{\True $\mathscr{D}=\left(0;+\infty \right)\setminus\left\{4\right\}$}
	\loigiai{
		Hàm số xác định khi $x|x-4|>0\Leftrightarrow \heva{
				& \left| x-4\right|\ne 0 \\
				& x>0}\Leftrightarrow \heva{
				& x\ne 4 \\
				& x>0.}$\\
		Vậy tập xác định của hàm số là $\mathscr{D}=\left(0;+\infty \right)\setminus\left\{4\right\}$.}
\end{ex}
\begin{ex}%[Phan Anh]%[0D2K1-2]
	Tìm tập xác định $\mathscr{D}$ của hàm số $y=\dfrac{\sqrt{5-3\left| x\right|}}{x^2+4x+3}$.
	\choice
	{\True $\mathscr{D}=\left[-\dfrac{5}{3};\dfrac{5}{3}\right]\setminus\left\{-1\right\}$}
	{$\mathscr{D}=\mathbb{R}$}
	{$\mathscr{D}=\left(-\dfrac{5}{3};\dfrac{5}{3}\right)\setminus\left\{-1\right\}$}
	{$\mathscr{D}=\left[-\dfrac{5}{3};\dfrac{5}{3}\right]$}
	\loigiai{
		Hàm số xác định khi $\heva{
				& 5-3\left| x\right|\ge 0 \\
				& x^2+4x+3\ne 0}\Leftrightarrow \heva{
				& \left| x\right|\le \dfrac{5}{3} \\
				& x\ne-1 \\
				& x\ne-3}\Leftrightarrow \heva{
				&-\dfrac{5}{3}\le x\le \dfrac{5}{3} \\
				& x\ne-1 \\
				& x\ne-3}\Leftrightarrow \heva{
				&-\dfrac{5}{3}\le x\le \dfrac{5}{3} \\
				& x\ne-1.}$\\
		Vậy tập xác định của hàm số là $\mathscr{D}=\left[-\dfrac{5}{3};\dfrac{5}{3}\right]\setminus\left\{-1\right\}$.}
\end{ex}
\begin{ex}%[Phan Anh]%[0D2K1-2]
	Tìm tập xác định $\mathscr{D}$ của hàm số $f(x)=\left\{\begin{array}{*{35}{l}}
			\dfrac{1}{2-x} & ;x\ge 1 \\
			\sqrt{2-x}     & ;x<1.
		\end{array}\right.$
	\choice
	{$\mathscr{D}=\mathbb{R}$}
	{$\mathscr{D}=\left(2;+\infty \right)$}
	{$\mathscr{D}=\left(-\infty;2\right)$}
	{\True $\mathscr{D}=\mathbb{R}\setminus\left\{2\right\}$}
	\loigiai{
		Hàm số xác định khi $\hoac{
				& \heva{
					& x\ge 1 \\
					& 2-x\ne 0} \\
				& \heva{
					& x<1 \\
					& 2-x\ge 0}}\Leftrightarrow \hoac{
				& \heva{
					& x\ge 1 \\
					& x\ne 2} \\
				& \heva{
					& x<1 \\
					& x\le 2}}\Leftrightarrow \hoac{
				& \heva{
					& x\ge 1 \\
					& x\ne 2} \\
				& x<1.}$\\
		Vậy xác định của hàm số là $\mathscr{D}=\mathbb{R}\setminus\left\{2\right\}$.}
\end{ex}
\begin{ex}%[Phan Anh]%[0D2K1-2]
	Tìm tập xác định $\mathscr{D}$ của hàm số $f(x)=\left\{\begin{array}{*{35}{l}}
			\dfrac{1}{x} & ;x\ge 1 \\
			\sqrt{x+1}   & ;x<1.
		\end{array}\right.$
	\choice
	{$\mathscr{D}=\left\{-1\right\}$}
	{$\mathscr{D}=\mathbb{R}$}
	{\True $\mathscr{D}=\left[-1;+\infty \right)$}
	{$\mathscr{D}=\left[-1;1\right)$}
	\loigiai{
	Hàm số xác định khi $\hoac{
			& \heva{
				& x\ge 1 \\
				& x\ne 0} \\
			& \heva{
				& x<1 \\
				& x+1\ge 0}}\Leftrightarrow \hoac{
			& x\ge 1 \\
			& \heva{
				& x<1 \\
				& x\ge-1.}}$\\
	Vậy xác định của hàm số là $\mathscr{D}=\left[-1;+\infty \right)$.}
\end{ex}
\begin{ex}%[Phan Anh]%[0D2K1-2]
	Tìm tất cả các giá trị thực của tham số $m$ để hàm số $y=\sqrt{x-m+1}+\dfrac{2x}{\sqrt{-x+2m}}$ xác định trên khoảng $(-1;3)$.
	\choice
	{\True Không có giá trị $m$ thỏa mãn}
	{$m\ge 2$}
	{$m\ge 3$}
	{$m\ge 1$}
	\loigiai{
	Hàm số xác định khi $\heva{
			& x-m+1\ge 0 \\
			&-x+2m>0}\Leftrightarrow \heva{
			& x\ge m-1 \\
			& x<2m.}$\\
	Tập xác định của hàm số là $\mathscr{D}=\left[m-1;2m\right)$ với điều kiện $m-1<2m\Leftrightarrow m>-1$.\\
	Hàm số đã cho xác định trên $\left(-1;3\right)$ khi và chỉ khi $\left(-1;3\right)\subset \left[m-1;2m\right)$\\
	$\Leftrightarrow m-1\le-1<3\le 2m\Leftrightarrow \heva{
			& m\le 0 \\
			& m\ge \dfrac{3}{2}.}$\\
	Vậy không có giá trị $m$ thỏa bài toán.}
\end{ex}
\begin{ex}%[Phan Anh]%[0D2K1-2]
	Tìm tất cả các giá trị thực của tham số $m$ để hàm số $y=\dfrac{x+2m+2}{x-m}$ xác định trên $\left(-1;0\right)$.
	\choice
	{$\hoac{
				& m>0 \\
				& m<-1}$}
	{$m\le-1$}
	{\True $\hoac{
				& m\ge 0 \\
				& m\le-1}$}
	{$m\ge 0$}
	\loigiai{
		Hàm số xác định khi $x-m\ne 0\Leftrightarrow x\ne m$.
		Tập xác định của hàm số là $\mathscr{D}=\mathbb{R}\setminus\left\{m\right\}$.\\
		Hàm số xác định trên $\left(-1;0\right)$ khi và chỉ khi $m\notin \left(-1;0\right)\Leftrightarrow \hoac{
				& m\ge 0 \\
				& m\le-1.}$}
\end{ex}
\begin{ex}%[Phan Anh]%[0D2K1-2]
	Tìm tất cả các giá trị thực của tham số $m$ để hàm số $y=\dfrac{mx}{\sqrt{x-m+2}-1}$ xác định trên $(0;1)$.
	\choice
	{$m\in \left(-\infty;\dfrac{3}{2}\right]\cup \left\{2\right\}$}
	{$m\in \left(-\infty;-1\right]\cup \left\{2\right\}$}
	{$m\in \left(-\infty;1\right]\cup \left\{3\right\}$}
			{\True $m\in \left(-\infty;1\right]\cup \left\{2\right\}$}
				\loigiai{
				Hàm số xác định khi $\heva{
					& x-m+2\ge 0 \\
					& \sqrt{x-m+2}-1\ne 0}\Leftrightarrow \heva{
					& x\ge m-2 \\
					& x\ne m-1.}$
				\\ Tập xác định của hàm số là $\mathscr{D}=\left[m-2;+\infty \right)\setminus\left\{m-1\right\}$.\\
			Hàm số xác định trên $\left(0;1\right)$ khi và chỉ khi $\left(0;1\right)\subset \left[m-2;+\infty \right)\setminus\left\{m-1\right\}$\\
		$\Leftrightarrow \hoac{
				& m-2\le 0<1\le m-1 \\
				& m-1\le 0}\Leftrightarrow \hoac{
				& \heva{
					& m\le 2 \\
					& m\ge 2} \\
				& m\le 1}\Leftrightarrow \hoac{
				& m=2 \\
				& m\le 1.}$}
\end{ex}
\begin{ex}%[Phan Anh]%[0D2K1-2]
	Tìm tất cả các giá trị thực của tham số $m$ để hàm số $y=\sqrt{x-m}+\sqrt{2x-m-1}$ xác định trên $(0;+\infty)$.
	\choice
	{$m\le 0$}
	{$m\ge 1$}
	{$m\le 1$}
	{\True $m\le-1$}
	\loigiai{
		Hàm số xác định khi $\heva{
				& x-m\ge 0 \\
				& 2x-m-1\ge 0}\Leftrightarrow \heva{
				& x\ge m \\
				& x\ge \dfrac{m+1}{2}}\,(*)$.
		\begin{itemize}
			\item Nếu $m\ge \dfrac{m+1}{2}\Leftrightarrow m\ge 1$ thì $\left(*\right)\Leftrightarrow x\ge m$.\\
			      Tập xác định của hàm số là $\mathscr{D}=\left[m;+\infty \right)$.
			      Khi đó, hàm số xác định trên $\left(0;+\infty \right)$ khi và chỉ khi $\left(0;+\infty \right)\subset \left[m;+\infty \right)\Leftrightarrow m\le 0$
			      $\Rightarrow $ Không thỏa mãn điều kiện $m\ge 1$.
			\item Nếu $m\le \dfrac{m+1}{2}\Leftrightarrow m\le 1$ thì $\left(*\right)\Leftrightarrow x\ge \dfrac{m+1}{2}$.\\
			      Tập xác định của hàm số là $\mathscr{D}=\left[\dfrac{m+1}{2};+\infty \right)$.
			      Khi đó, hàm số xác định trên $\left(0;+\infty \right)$
			      khi và chỉ khi $\left(0;+\infty \right)\subset \left[\dfrac{m+1}{2};+\infty \right)\Leftrightarrow \dfrac{m+1}{2}\le 0\Leftrightarrow m\le-1$.\\
			      $\Rightarrow $ Thỏa mãn điều kiện $m\le 1$.
		\end{itemize}
		Vậy $m\le-1$ thỏa yêu cầu bài toán.}
\end{ex}
\begin{ex}%[Phan Anh]%[0D2K1-2]
	Tìm tất cả các giá trị thực của tham số $m$ để hàm số $y=\dfrac{2x+1}{\sqrt{x^2-6x+m-2}}$ xác định trên $\mathbb{R}$.
	\choice
	{$m\ge 11$}
	{\True $m>11$}
	{$m<11$}
	{$m\le 11$}
	\loigiai{
		Hàm số xác định khi $x^2-6x+m-2>0\Leftrightarrow {\left(x-3\right)}^2+m-11>0$.\\
		Hàm số xác định với $\forall x\in \mathbb{R}\Leftrightarrow (x-3)^2+m-11>0$ đúng với mọi $x\in \mathbb{R}$
		$\Leftrightarrow m-11>0\Leftrightarrow m>11$.}
\end{ex}
\begin{ex}%[Phan Anh]%[0D2B1-3]
	Cho hàm số $f(x)=4-3x$. Khẳng định nào sau đây đúng?
	\choice
	{Hàm số đồng biến trên $\left(-\infty;\dfrac{4}{3}\right)$}
	{\True Hàm số nghịch biến trên $\left(\dfrac{4}{3};+\infty \right)$}
	{Hàm số đồng biến trên $\mathbb{R}$}
	{Hàm số đồng biến trên $\left(\dfrac{3}{4};+\infty \right)$}
	\loigiai{
		TXĐ: $\mathscr{D}=\mathbb{R}$. \\Với mọi $x_1,x_2\in \mathbb{R}$ và $x_1<x_2$, ta có
		$f\left(x_1\right)-f\left(x_2\right)=\left(4-3x_1\right)-\left(4-3x_2\right)=-3\left(x_1-x_2\right)>0.$\\
		Suy ra $f\left(x_1\right)>f\left(x_2\right)$. Do đó, hàm số nghịch biến trên $\mathbb{R}$.\\
		Mà $\left(\dfrac{4}{3};+\infty \right)\subset \mathbb{R}$ nên hàm số cũng nghịch biến trên $\left(\dfrac{4}{3};+\infty \right)$.}
\end{ex}
% \begin{ex}%[Phan Anh]%[0D2B1-3]
% 	Xét tính đồng biến, nghịch biến của hàm số $f(x)=x^2-4x+5$ trên khoảng $\left(-\infty;2\right)$ và trên khoảng $\left(2;+\infty \right)$. Khẳng định nào sau đây đúng?
% 	\choice
% 	{\True Hàm số nghịch biến trên $\left(-\infty;2\right)$, đồng biến trên $\left(2;+\infty \right)$}
% 	{Hàm số đồng biến trên $\left(-\infty;2\right)$, nghịch biến trên $\left(2;+\infty \right)$}
% 	{Hàm số nghịch biến trên các khoảng $\left(-\infty;2\right)$ và $\left(2;+\infty \right)$}
% 	{Hàm số đồng biến trên các khoảng $\left(-\infty;2\right)$ và $\left(2;+\infty \right)$}
% 	\loigiai{
% 		Ta có $f\left(x_1\right)-f\left(x_2\right)=\left(x_1^2-4x_1+5\right)-\left(x_2^2-4x_2+5\right)$
% 		$=\left(x_1^2-x_2^2\right)-4\left(x_1-x_2\right)=\left(x_1-x_2\right)\left(x_1+x_2-4\right)$.
% 		Với mọi $x_1, x_2\in \left(-\infty;2\right)$ và $x_1<x_2$. Ta có $\heva{
% 			& x_1<2 \\ 
% 			& x_2<2 \\}\Rightarrow x_1+x_2<4$.\\
% 		Suy ra $\dfrac{f\left(x_1\right)-f\left(x_2\right)}{x_1-x_2}=\dfrac{\left(x_1-x_2\right)\left(x_1+x_2-4\right)}{x_1-x_2}=x_1+x_2-4<0$.\\
% 		Vậy hàm số nghịch biến trên $\left(-\infty;2\right)$.\\
% 		Với mọi $x_1, x_2\in \left(2;+\infty \right)$ và $x_1<x_2$. Ta có $\heva{
% 			& x_1>2 \\ 
% 			& x_2>2 \\}\Rightarrow x_1+x_2>4$.\\
% 		Suy ra $\dfrac{f\left(x_1\right)-f\left(x_2\right)}{x_1-x_2}=\dfrac{\left(x_1-x_2\right)\left(x_1+x_2-4\right)}{x_1-x_2}=x_1+x_2-4>0$.\\
% 		Vậy hàm số đồng biến trên $\left(2;+\infty \right)$.}
% \end{ex}
\begin{ex}%[Phan Anh]%[0D2B1-3]
	Xét sự biến thiên của hàm số $f(x)=\dfrac{3}{x}$ trên khoảng $(0;+\infty)$. Khẳng định nào sau đây đúng?
	\choice
	{Hàm số đồng biến trên khoảng $\left(0;+\infty \right)$}
	{\True Hàm số nghịch biến trên khoảng $\left(0;+\infty \right)$}
	{Hàm số vừa đồng biến, vừa nghịch biến trên khoảng $\left(0;+\infty \right)$}
	{Hàm số không đồng biến, cũng không nghịch biến trên khoảng $\left(0;+\infty \right)$}
	\loigiai{
		Ta có $f\left(x_1\right)-f\left(x_2\right)=\dfrac{3}{x_1}-\dfrac{3}{x_2}=\dfrac{3\left(x_2-x_1\right)}{x_1x_2}=-\dfrac{3\left(x_1-x_2\right)}{x_1x_2}.$\\
		Với mọi $x_1, x_2\in \left(0;+\infty \right)$ và $x_1<x_2$. Ta có $\heva{
				& x_1>0 \\
				& x_2>0 \\}\Rightarrow x_1\cdot x_2>0$.\\
		Suy ra $\dfrac{f\left(x_1\right)-f\left(x_2\right)}{x_1-x_2}=-\dfrac{3}{x_1x_2}<0\Rightarrow f(x)$ nghịch biến trên $\left(0;+\infty \right)$.}
\end{ex}
\begin{ex}%[Phan Anh]%[0D2B1-3]
	Xét sự biến thiên của hàm số $f(x)=x+\dfrac{1}{x}$ trên khoảng $\left(1;+\infty \right)$. Khẳng định nào sau đây đúng?
	\choice
	{\True Hàm số đồng biến trên khoảng $\left(1;+\infty \right)$}
	{Hàm số nghịch biến trên khoảng $\left(1;+\infty \right)$}
	{Hàm số vừa đồng biến, vừa nghịch biến trên khoảng $\left(1;+\infty \right)$}
	{Hàm số không đồng biến, cũng không nghịch biến trên khoảng $\left(1;+\infty \right)$}
	\loigiai{
		Ta có
		$f\left(x_1\right)-f\left(x_2\right)=\left(x_1+\dfrac{1}{x_1}\right)-\left(x_2+\dfrac{1}{x_2}\right)=\left(x_1-x_2\right)+\left(\dfrac{1}{x_1}-\dfrac{1}{x_2}\right)=\left(x_1-x_2\right)\left(1-\dfrac{1}{x_1x_2}\right).$\\
		Với mọi $x_1, x_2\in \left(1;+\infty \right)$ và $x_1<x_2$. Ta có $\heva{
				& x_1>1 \\
				& x_2>1 \\}\Rightarrow x_1\cdot x_2>1\Rightarrow \dfrac{1}{x_1\cdot x_2}<1.$\\
		Suy ra $\dfrac{f\left(x_1\right)-f\left(x_2\right)}{x_1-x_2}=1-\dfrac{1}{x_1x_2}>0\Rightarrow f(x)$ đồng biến trên $\left(1;+\infty \right)$.}
\end{ex}
\begin{ex}%[Phan Anh]%[0D2B1-3]
	Xét tính đồng biến, nghịch biến của hàm số $f(x)=\dfrac{x-3}{x+5}$ trên khoảng $\left(-\infty;-5\right)$ và trên khoảng $\left(-5;+\infty \right)$. Khẳng định nào sau đây đúng?
	\choice
	{Hàm số nghịch biến trên $\left(-\infty;-5\right)$, đồng biến trên $\left(-5;+\infty \right)$}
	{Hàm số đồng biến trên $\left(-\infty;-5\right)$, nghịch biến trên $\left(-5;+\infty \right)$}
	{Hàm số nghịch biến trên các khoảng $\left(-\infty;-5\right)$ và $\left(-5;+\infty \right)$}
	{\True Hàm số đồng biến trên các khoảng $\left(-\infty;-5\right)$ và $\left(-5;+\infty \right)$}
	\loigiai{
		Ta có
		\begin{eqnarray*}
			f\left(x_1\right)-f\left(x_2\right)&=&\left(\dfrac{x_1-3}{x_1+5}\right)-\left(\dfrac{x_2-3}{x_2+5}\right)\\
			&=&\dfrac{\left(x_1-3\right)\left(x_2+5\right)-\left(x_2-3\right)\left(x_1+5\right)}{\left(x_1+5\right)\left(x_2+5\right)}\\
			&=&\dfrac{8\left(x_1-x_2\right)}{\left(x_1+5\right)\left(x_2+5\right)}.
		\end{eqnarray*}
		Với mọi $x_1, x_2\in \left(-\infty;-5\right)$ và $x_1<x_2$. Ta có $\heva{& x_1<-5 \\& x_2<-5}\Leftrightarrow \heva{&x_1+5<0 \\& x_2+5<0.}$\\
		Suy ra $\dfrac{f\left(x_1\right)-f\left(x_2\right)}{x_1-x_2}=\dfrac{8}{\left(x_1+5\right)\left(x_2+5\right)}>0\Rightarrow f(x)$ đồng biến trên $\left(-\infty;-5\right)$.\\
		Với mọi $x_1, x_2\in \left(-5;+\infty \right)$ và $x_1<x_2$. Ta có $\heva{
				& x_1>-5 \\
				& x_2>-5 \\}\Leftrightarrow \heva{
				& x_1+5>0 \\
				& x_2+5>0 \\}$.\\
		Suy ra $\dfrac{f\left(x_1\right)-f\left(x_2\right)}{x_1-x_2}=\dfrac{8}{\left(x_1+5\right)\left(x_2+5\right)}>0\Rightarrow f(x)$ đồng biến trên $\left(-5;+\infty \right)$.}
\end{ex}

\begin{ex}%[Phan Anh]%[0D2K1-3]
	Cho hàm số $f(x)=\sqrt{2x-7}$. Khẳng định nào sau đây đúng?
	\choice
	{Hàm số nghịch biến trên $\left(\dfrac{7}{2};+\infty \right)$}
	{\True Hàm số đồng biến trên $\left(\dfrac{7}{2};+\infty \right)$}
	{Hàm số đồng biến trên $\mathbb{R}$}
	{Hàm số nghịch biến trên $\mathbb{R}$}
	\loigiai{
	Tập xác định là $\mathscr{D}=\left[\dfrac{7}{2};+\infty \right)$ nên ta loại đáp án C và D.\\
	Xét $f\left(x_1\right)-f\left(x_2\right)=\sqrt{2x_1-7}-\sqrt{2x_2-7}=\dfrac{2\left(x_1-x_2\right)}{\sqrt{2x_1-7}+\sqrt{2x_2-7}}.$\\
	Với mọi $x_1, x_2\in \left(\dfrac{7}{2};+\infty \right)$ và $x_1<x_2$, ta có $\dfrac{f\left(x_1\right)-f\left(x_2\right)}{x_1-x_2}=\dfrac{2}{\sqrt{2x_1-7}+\sqrt{2x_2-7}}>0.$\\
	Vậy hàm số đồng biến trên $\left(\dfrac{7}{2};+\infty \right)$.}
\end{ex}
\begin{ex}%[Phan Anh]%[0D2K1-3]
	Có bao nhiêu giá trị nguyên của tham số $m$ thuộc đoạn $\left[-3;3\right]$ để hàm số $f(x)=\left(m+1\right)x+m-2$ đồng biến trên $\mathbb{R}$?
	\choice
	{$7$}
	{$5$}
	{\True $4$}
	{$3$}
	\loigiai{
		Tập xác định $\mathscr{D}=\mathbb{R}.$\\
		Với mọi $x_1,x_2\in\mathscr{D}$ và $x_1<x_2$. \\Ta có
		$f\left(x_1\right)-f\left(x_2\right)=\left[\left(m+1\right)x_1+m-2\right]-\left[\left(m+1\right)x_2+m-2\right]=\left(m+1\right)\left(x_1-x_2\right).$\\
		Suy ra $\dfrac{f\left(x_1\right)-f\left(x_2\right)}{x_1-x_2}=m+1$.\\
		Để hàm số đồng biến trên $\mathbb{R}$ khi và chỉ khi
		$m+1>0\Leftrightarrow m>-1\xrightarrow{m\in \left[-3;3\right]}{m\in \mathbb{Z}}\Rightarrow m\in \left\{0;1;2;3\right\}$.\\
		Vậy có 4 giá trị nguyên của $m$ thỏa mãn.}
\end{ex}
% \begin{ex}%[Phan Anh]%[0D2K1-3]
% 	Tìm tất cả các giá trị thực của tham số $m$ để hàm số $y=-x^2+\left(m-1\right)x+2$ nghịch biến trên khoảng $\left(1;2\right)$.
% 	\choice
% 	{$m<5$}
% 	{$m>5$}
% 	{\True $m<3$}
% 	{$m>3$}
% 	\loigiai{
% 		Với mọi $x_1\ne x_2$, ta có\\
% 		$\dfrac{f\left(x_1\right)-f\left(x_2\right)}{x_1-x_2}=\dfrac{\left[-x_1^2+\left(m-1\right)x_1+2\right]-\left[-x_2^2+\left(m-1\right)x_2+2\right]}{x_1-x_2}=-\left(x_1+x_2\right)+m-1.$\\
% 		Để hàm số nghịch biến trên $\left(1;2\right)\Leftrightarrow-\left(x_1+x_2\right)+m-1<0$, với mọi $x_1,x_2\in \left(1;2\right)$\\
% 		$\Leftrightarrow m<\left(x_1+x_2\right)+1$, với mọi $x_1,x_2\in \left(1;2\right)$
% 		$\Leftrightarrow m<\left(1+1\right)+1=3$.}
% \end{ex}
\begin{ex}%[Phan Anh]%[0D2K1-3]
	\immini{Cho hàm số $y=f(x)$ có tập xác định là $\left[-3;3\right]$ và đồ thị của nó được biểu diễn bởi hình bên. Khẳng định nào sau đây là đúng?
		\choice
		{\True Hàm số đồng biến trên khoảng $\left(-3;-1\right)$ và $\left(1;3\right)$}
		{Hàm số đồng biến trên khoảng $\left(-3;-1\right)$và $\left(1;4\right)$}
		{Hàm số đồng biến trên khoảng $\left(-3;3\right)$}
		{Hàm số nghịch biến trên khoảng $\left(-1;0\right)$}}
	{\begin{tikzpicture}[>=stealth,scale=0.7]
			\draw[->](-4,0)--(4,0)node[above]{$x$};
			\draw[->](0,-2)--(0,5)node[right]{$y$};
			\draw (-3,-1)--(-1,1)--(0,1)node[above left]{$1$}--(3,4);
			\draw[dashed](-3,0)node[above]{$-3$}--(-3,-1)--(0,-1)node[right]{$-1$};
			\draw[dashed](-1,0)node[below]{$-1$}--(-1,1);
			\draw[dashed](3,0)node[below]{$3$}--(3,4)--(0,4)node[left]{$4$};
			\fill (-3,0)circle(1.2pt) (-3,-1)circle(1.2pt) (0,-1)circle(1.2pt) (-1,0)circle(1.2pt) (-1,1)circle(1.2pt) (0,1)circle(1.2pt) (3,0)circle(1.2pt) (3,4)circle(1.2pt) (0,4)circle(1.2pt) (0,0)node[above right]{$O$}circle(1.2pt);
		\end{tikzpicture}}
	\loigiai{
		Trên khoảng $\left(-3;-1\right)$ và $\left(1;3\right)$ đồ thị hàm số đi lên từ trái sang phải\\
		$\Rightarrow $ Hàm số đồng biến trên khoảng $\left(-3;-1\right)$ và $\left(1;3\right).$}
\end{ex}
\begin{ex}%[Phan Anh]%[0D2K1-3]
	\immini{Cho đồ thị hàm số $y=x^3$ như hình bên. Khẳng định nào sau đây \textbf{sai}?
		\choice
		{Hàm số đồng biến trên khoảng $\left(-\infty;0\right)$}
		{Hàm số đồng biến trên khoảng $\left(0;+\infty \right)$}
		{Hàm số đồng biến trên khoảng $\left(-\infty;+\infty \right)$}
		{\True Hàm số đồng biến tại gốc tọa độ $O$}}
	{\begin{tikzpicture}[>=stealth,scale=0.6]
			\draw[->](-2,0)--(2,0)node[above]{$x$};
			\draw[->](0,-3)--(0,3)node[right]{$y$};
			\draw[smooth,samples=100,domain=-1.4:1.4]plot(\x,{(\x)^3});
			\fill (0,0)node[above left]{$O$}circle(1.2pt);
		\end{tikzpicture}}
	\loigiai{Dựa vào đồ thị, ta thấy hàm số đồng biến trên toàn miền xác định. Nhưng không thể đồng biến chỉ tại đúng một điểm.}
\end{ex}
\Closesolutionfile{ans}
\Closesolutionfile{ansbook}
% \setcounter{section}{0}
\section{HÀM SỐ VÀ ĐỒ THỊ}
\subsection{TÓM TẮT LÝ THUYẾT}
\subsubsection{Khái niệm hàm số. Tập xác định và tập giá trị của hàm số}
\begin{itemize}
	\item [\faSunO] \textbf{Định nghĩa:} Giả sử $x$ và $y$ là hai đại lượng biến thiên và $x$ nhận giá trị thuộc tập số $\mathscr{D}$.	Nếu với \textbf{mỗi giá trị $x$ thuộc $\mathscr{D}$}, ta xác định được \textbf{một và chỉ một giá trị tương ứng $y$} thuộc tập hợp số thực $\mathbb{R}$ thì ta có một hàm số.
	\begin{khung}
		\begin{itemize}
			\item Ta gọi $x$ là biến số và $y$ là hàm số của $x$.
			\item Tập hợp $\mathscr{D}$ được gọi là tập xác định của hàm số.
			\item Tập hợp $T$ gồm tất cả các giá trị $y$ (tương ứng với $x$ thuộc $\mathscr{D}$) gọi là tập giá trị của hàm số.
		\end{itemize}
	\end{khung}
	\item [\faSunO] \textbf{Cách cho một hàm số:} Một hàm số có thể được cho bởi một công thức hoặc nhiều công thức; có thể cho bằng mô tả; cho bằng bảng hoặc cho bằng biểu đồ.
\end{itemize}
% \begin{vd}
% 	Bản tin dự báo thời tiết cho biết nhiệt độ ở một số thời điểm trong ngày 01/05/2021 tại thành phố Hồ Chí Minh được ghi lại với biểu đồ bên dưới\\
% 	\centerline{\begin{tikzpicture}[font=\footnotesize, x=1.2cm,line join=round, line cap=round, >=stealth,scale=0.7,color=cyan]
% 		\draw[->] (.5,0)--(.5,5.5)node[left]{nhiệt độ};
% 		\draw[->] (.5,0)--(9,0) node[below]{giờ};
% 		\foreach \x/\y in {0/24,1/26,2/28,3/30,4/32,5/34} {
% 			\draw (.5,\x) node[left]{$\y$};}
% 		\foreach \x/\y in {1/1,2/4,3/7,4/10,5/13,6/16,7/19,8/22}\draw (\x,0.1)--(\x,-0.1) node [below] {\footnotesize $\y$};
% 		\foreach \x in {1,2,...,5} { \draw[orange!30] (.5,\x)--(9,\x); }
% 		\draw [line width=1pt,blue] (1,2)node[above]{\scriptsize $28$}--(2,1.5)node[above]{\scriptsize $27$}--(3,2)node[above left]{\scriptsize $28$}--(4,4)node[above]{\scriptsize $32$}--(5,3.5)node[above]{\scriptsize $31$}--(6,2.5)node[above]{\scriptsize $29$}--(7,2)node[above]{\scriptsize $28$}--(8,1.5)node[above]{\scriptsize $27$};
% 	\end{tikzpicture}}\\
% Rõ ràng với mỗi mốc giờ xác định trong ngày, ta có tương ứng duy nhất 1 số đo nhiệt độ được dự báo nên  có xem đây là 1 hàm số với
% 	\begin{itemize}
% 		\item  Tập xác định $D=\{1;4;7;10;13;16;19;22\}$
% 		\item  Tập giá trị $T=\{28; 27; 32;31;29\}$.
% 	\end{itemize}
% \end{vd}
% \begin{vd}
% 	Xét công thức $y=2x+1$. Ta đã biết đây là một hàm số bậc nhất với
% 	\begin{itemize}
% 		\item  Tập xác định $D=\mathbb{R}$
% 		\item  Tập giá trị $T=\mathbb{R}$.
% 	\end{itemize}
% \end{vd}
\subsubsection{Đồ thị hàm số}
\begin{itemize}
	\item [\faSunO] \textbf{Định nghĩa:} Cho hàm số $y=f(x)$ có tập xác định $\mathscr{D}$. Trên mặt phẳng toạ độ $Oxy$, đồ thị $(C)$ của hàm số là tập hợp tất cả các điểm $M(x;y)$ với $x \in \mathscr{D}$ và $y=f(x)$. Vậy $(C)=\{M(x;f(x)) \mid x \in \mathscr{D}\}$.
	\item [\faSunO] \textbf{Lưu ý:} Điểm $M\left(x_{M};y_{M}\right)$ thuộc đồ thị hàm số $y=f(x)$ khi và chỉ khi $x_{M} \in \mathscr{D}$ và $y_{M}=f\left(x_{M}\right)$.
\end{itemize}
\subsubsection{Sự đồng biến, hàm số nghịch biến của hàm số}
\begin{itemize}
	\item [\faSunO] \textbf{Khái niệm:} Với hàm số $y=f(x)$ xác định trên khoảng $(a;b)$, ta nói
	\begin{itemize}
		\item [\iconCH] Hàm số đồng biến trên khoảng $(a;b)$ nếu
		\boxmini{$\forall x_{1}, x_{2} \in(a ; b), x_{1}<x_{2} \Rightarrow f\left(x_{1}\right)<f\left(x_{2}\right)$}
		\item [\iconCH] Hàm số nghịch biến trên khoảng $(a;b)$ nếu
		\boxmini{$\forall x_{1}, x_{2} \in(a ; b), x_{1}<x_{2} \Rightarrow f\left(x_{1}\right)>f\left(x_{2}\right)$}
	\end{itemize}
	\item [\faSunO] \textbf{Lưu ý:} Khi vẽ bảng biến thiên, xét từ trái sang phải, ta dùng mũi tên đi xuống để minh họa khoảng nghịch biến và mũi tên đi lên để minh họa khoảng đồng biến. 
\end{itemize}

\subsection{RÈN LUYỆN KĨ NĂNG GIẢI TOÁN}
\begin{dang}{Tính giá trị của hàm số tại một điểm}
	Cho hàm số $y=f(x)$ có tập xác định $\mathscr{D}$ và $x_0 \in \mathscr{D}$.
	\begin{itemize}
		\item[\faPencilSquareO] Tính giá trị hàm số tại $x_0$: Ta chỉ việc thay $x_0$ vào biểu thức $y=f(x)$, tìm được $y_0$.
		\item[\faPencilSquareO] Nếu $f(x)$ là hàm cho bởi nhiều biểu thức thì ta thay $x_0$ vào biểu thức mà miền xác định của nó chứa $x_0$.
	\end{itemize}
\end{dang}

\begin{vd}
	Cho hai hàm số $f(x)=x^2-2x$ và $g(x)=1-x$. Tính $f(1)$; $g(-2)$; $f(1)+g(-2)$.
	\loigiai{}
\end{vd}

\begin{vd}
	Cho hàm số $f(x)=\heva{&3x-2 &\text{với } &x\ge 1 \\&1-2x^2 &\text{với } &x<1}$. Tính $f(1), f(2), f(0), f(-3)$.
		\loigiai{}
\end{vd}



\begin{vd}
	Cho hàm số $y=2x^3-3(m-1)x+2$, với $m$ là tham số. 
	\begin{tasks}
		\task Tìm $m$ để đồ thị hàm số đi qua điểm $M(1;2)$.
		\task Tìm $m$ để đồ thị hàm số đi qua điểm $N(-3;1)$.
	\end{tasks}
	\loigiai{}
\end{vd}

\begin{vd}
\immini{ Cho hàm số $y=f(x)$ và hàm số $y=g(x)$ có đồ thị như hình bên.
	\begin{tasks}
		\task Trong các điểm $A(2;2)$ $B(4;2)$, $C(3;3)$ điểm nào thuộc đồ thị $f(x)$? điểm nào thuộc đồ thị $g(x)$?
		\task Tính giá trị $f(1)+g(2)$.
		\task Tìm điểm trên đồ thị $f(x)$ có tung độ bằng $3$.
	\end{tasks}
}{
	\begin{tikzpicture}[>=stealth,scale=0.7]
	\draw[->] (-1.2,0)--(0,0)%
	node[below left]{$O$}--(7.5,0) node[below]{$x$};
	\draw[->] (0,-0.8) --(0,5) node[right]{$y$};
	\foreach \x in {1,2,3,4,5,6,7}{
		\draw (\x,0) node[below]{$\x$};%Ox
	}
	\foreach \y in {1,2,3,4}{
		\draw (0,\y) node[left]{$\y$};%Oy
	}
	\draw[violet,dashed,line width=0.2pt] (0,0) grid (7,4.5);
	\draw [blue, line width=1.5pt] (0,0)--(4,2)--(7,2);
	\draw [red, line width=1.5pt, domain=0.5:3.5, samples=100] %
	plot (\x, {(\x)^2 -4*(\x)+6});
	\draw (3.1,3.4)node[right]{\footnotesize$y=f(x)$} (5,1.5)node[right]{\footnotesize $y=g(x)$};
	
	\end{tikzpicture}}
	\loigiai{}
\end{vd}

\begin{dang}{Tìm tập xác định, tập giá trị của hàm số}
	\begin{itemize}
		\item [\faSunO] \textbf{Tập xác định:} Ta tìm tập hợp tất cả các giá trị của $x$ để hàm số đã cho có nghĩa. Cần lưu ý hai vấn đề sau:
		\begin{listEX}[2]
			\item [\ding{172}] $\dfrac{A}{B}$ có nghĩa khi $B \ne 0$.
			\item [\ding{173}] $\sqrt{B}$ có nghĩa khi $B \ge 0$.
		\end{listEX}
		\item [\faSunO] \textbf{Tập giá trị:} Với $x$ thuộc miền xác định $\mathscr{D}$, ta có thể căn cứ vào bảng biến thiên hoặc đồ thị để tìm miền giá trị (\textit{nhìn khoảng "dao động" của} $y$.).
	\end{itemize}

\end{dang}

\begin{vd}%[0D2B1-1]
	Sau khi đun nóng băng phiến lên đến gần $90^{\circ}C$, người ta để nguội, quan sát, ghi nhận nhiệt độ và trạng thái của băng phiến sau mẫu phút như bảng sau
	\begin{center}
	\textit{Nhiệt độ và trạng thái của băng phiến khi để nguội}
		\begin{tabular}{|l|c|c|c|c|c|c|c|c|c|c|c|}
			\hline 
			\textbf{Thời gian nguội (phút)} & $0$ & $1$ & $2$ & $3$ & $4$ & $5$ & $6$ & $7$ & $8$ & $9$ & $10$ \\ 
			\hline 
			\textbf{Nhiệt độ ($^{\circ}C$)} & $86$ & $84$ & $82$ & $81$ & $80$ & $80$ & $80$  &$80$ & $79$ & $77$ & $75$ \\
			\hline
			\textbf{Trạng thái} & \multicolumn{4}{|c|}{lỏng} &\multicolumn{4}{|c|}{lỏng và rắng} & \multicolumn{3}{|c|}{rắn}\\
			\hline
		\end{tabular}
	\end{center}
\begin{tasks}(1)
	\task Tại sao từ bảng trên, có thể nói nhiệt độ của băng phiến là một hàm số theo thời gian (nung nóng)? Tìm tập xác định và tập giá trị của hàm số trên.
	\task Sau khi để nguội $3$ phút, nhiệt độ băng phiến là bao nhiêu?
	\task Băng phiến chuyển hoàn toàn sang trạng thái rắn sau bao nhiêu phút?
\end{tasks}
	\loigiai{
		\begin{enumerate}[a)]
			\item Bảng giá trị cho thấy nhiệt độ (kí hiệu là $y$) là một hàm số theo thời gian (kí hiệu là $x$) vì khi cho $x$ một giá trị bất kì, ta luôn tìm được duy nhất một giá trị của $y$. Do vậy bảng này xác định một hàm số biểu thị nhiệt độ của băng phiến theo thời gian.\\
			Từ bảng giá trị của hàm số, ta có tập xác định $\mathscr{D}=\{0 ; 1 ; 2 ; 3 ; 4 ; 5 ; 6 ; 7 ; 8 ; 9 ; 10\}$ và tập giá trị $T=\{75 ; 77 ; 79 ; 80 ; 81 ; 82 ; 84 ; 86\}$.
			\item Sau khi để nguội $3$ phút, nhiệt độ băng phiến là $81^{\circ}C$.
			\item Băng phiến chuyển hoàn toàn sang trạng thái rắn sau $8$ phút (lúc đó nhiệt độ băng phiến là $79^{\circ}$).
		\end{enumerate}
	}
\end{vd}

\begin{vd}
	\immini{Cho hàm số $y=f(x)$ có tập xác định là $\mathscr{D}$ và đồ thị là đường liền nét được vẽ trên miền $\mathscr{D}$ như hình bên
	\begin{tasks}(1)
		\task Xác định tập xác định $\mathscr{D}$.
		\task Tìm tập giá trị của hàm số trên miền $\mathscr{D}$.
		\task Tìm các điểm thuộc đồ thị và có tung độ bằng $3$.
	\end{tasks}}{
	\begin{tikzpicture}[smooth,samples=300,scale=0.8,>=stealth,font=\footnotesize]
		\draw[line width=0.1pt,gray!80] (-4,-2) grid (5,4);
		\draw[->] (-4.2,0)--(5.7,0) node[below]{$x$};
		\draw[->] (0,-2.2)--(0,4.5) node[right]{$y$};
		\draw (0,0) node[above right]{$O$};
		\draw[line width=1pt,domain=-1:3,magenta] plot(\x,{(\x-1)^2-1});
		\draw[line width=1pt,magenta] (-3,-1)--(-1,3) (3,3)--(5,1);
		\draw[fill=blue] (-3,-1) circle(2.5pt) (-1,3) circle(2.5pt) (3,3) circle(2.5pt) (5,1) circle(2.5pt) (1,-1) circle(2.5pt);
		\foreach \x in {-4,-3,-2,-1,1,2,3,4,5}\draw (\x,0.1)--(\x,-0.1) node [below] {\footnotesize $\x$};
		\foreach \y in {-2,-1,1,2,3}\draw (0.1,\y)--(-0.1,\y) node [left] {\footnotesize $\y$};
\end{tikzpicture}}
	\loigiai{}
\end{vd}

\begin{vd}
	Tìm tập xác định của các hàm số sau đây:
	\begin{listEX}[2]
		\item $y=x^4+x^2-2$.
		\item $y=\dfrac{x+2}{x-2}$.
		\item $y=\dfrac{x^2+2}{4-x}$.
		\item $y=\dfrac{1}{-x^2+3x}$
	\end{listEX}
	\loigiai{}
\end{vd}
\begin{vd}
	Tìm tập xác định của các hàm số sau đây:
	\begin{listEX}[2]
		\item $y=\sqrt{x-2}$.
		\item $y=\dfrac{2x-1}{\sqrt{x+2}}$.
		\item $y=x+\dfrac{1}{\sqrt{3-x}}$.
		\item $y=\sqrt{2+x}+\sqrt{x-2}$.
	\end{listEX}
	\loigiai{}
\end{vd}

\begin{vd}
	Tìm tập xác định của các hàm số sau:
	\begin{listEX}[2]
		\item $f(x)=\heva{&2x+1 &\text{ nếu }& x \le 0 \\& x^2 &\text{ nếu }& x > 0}$.
		\item $f(x)=\heva{&\dfrac{1}{x-1}&\text{ nếu }& x \le 2 \\& x^2 &\text{ nếu }& x > 2}$.
	\end{listEX}
\loigiai{
	\begin{enumerate}[a)]
		\item Ta xét hai trường hợp
		\begin{itemize}
			\item [$\bullet$] Khi $x\le 0$ thì $f(x)=2x+1$. Hàm này luôn xác định với mọi $x \in \mathbb{R}$ nên sẽ xác định với mọi $x \le 0$.
			\item [$\bullet$] Khi $x>0$ thì $f(x)=x^2$. Hàm này luôn xác định với mọi $x\in \mathbb{R}$ nên sẽ xác định với mọi $x >0$.
		\end{itemize}
	Kết hợp hai trường hợp, ta được tập xác định của hàm số là $\mathscr{D}=\mathbb{R}$.
		\item Ta xét hai trường hợp
		\begin{itemize}
			\item [$\bullet$] Khi $x\le 2$ thì $f(x)=\dfrac{1}{x-1}$. Hàm này  xác định khi và chỉ khi $x-1 \ne 0 \Leftrightarrow x \ne 1$.
			\item [$\bullet$] Khi $x>2$ thì $f(x)=x^2$. Hàm này luôn xác định với mọi $x\in \mathbb{R}$ nên sẽ xác định với mọi $x >2$.
		\end{itemize}
		Kết hợp hai trường hợp, ta được tập xác định của hàm số là $\mathscr{D}=\mathbb{R}\backslash \{1\}$.
\end{enumerate}}
\end{vd}


\begin{dang}{Tìm khoảng đồng biến, khoảng nghịch biến của hàm số}
\begin{itemize}
	\item[\faPencilSquareO] Nếu đề bài cho bảng biến thiên hoặc đồ thị: Xét từ trái sang phải thì
	\begin{itemize}
		\item [$\bullet$] Khoảng nào có mũi tên đi xuống (đồ thị đổ xuống) thì khoảng đó hàm số nghịch biến.
		\item [$\bullet$] Khoảng nào có mũi tên đi lên (đồ thị đi lên) thì khoảng đó hàm số đồng biến.
	\end{itemize}
	\item[\faPencilSquareO] Nếu đề bài yêu cầu xét tính đồng biến, nghịch biến của hàm số $y=f(x)$ trên khoảng xác định $(a;b)$:  Ta lấy $x_1, x_2$ tùy ý thuộc $(a;b)$, với $x_1<x_2$ và tính $f(x_1)-f(x_2)$, nếu
	\begin{itemize}
		\item [$\bullet$] $f(x_1)-f(x_2)<0$ thì hàm số $y=f(x)$ đồng biến trên khoảng $(a;b)$.
		\item [$\bullet$] $f(x_1)-f(x_2)>0$ thì hàm số $y=f(x)$ nghịch biến trên khoảng $(a;b)$.
	\end{itemize}
	\item[\faPencilSquareO] Trong nhiều trường hợp, để tìm được khoảng đồng biến và nghịch biến của hàm số, ta có thể lập bảng biến thiên của hàm số đó trên miền xác định.
\end{itemize}
\end{dang}
\begin{vd}
	\immini{Cho hàm số $y=f(x)$ có tập xác định là $\mathscr{D}$ và đồ thị là đường liền nét được vẽ trên miền $\mathscr{D}$ như hình bên. Tìm các khoảng đồng biến và nghịch biến của hàm số trên miền $\mathscr{D}$.
	}{
		\begin{tikzpicture}[smooth,samples=300,scale=0.6,>=stealth,font=\footnotesize]
			\draw[line width=0.1pt,gray!80] (-4,-2) grid (5,4);
			\draw[->] (-4.2,0)--(5.7,0) node[below]{$x$};
			\draw[->] (0,-2.2)--(0,4.5) node[right]{$y$};
			\draw (0,0) node[above right]{$O$};
			\draw[line width=1pt,domain=-1:3,magenta] plot(\x,{(\x-1)^2-1});
			\draw[line width=1pt,magenta] (-3,-1)--(-1,3) (3,3)--(5,1);
			\draw[fill=blue] (-3,-1) circle(2.5pt) (-1,3) circle(2.5pt) (3,3) circle(2.5pt) (5,1) circle(2.5pt) (1,-1) circle(2.5pt);
			\foreach \x in {-4,-3,-2,-1,1,2,3,4,5}\draw (\x,0.1)--(\x,-0.1) node [below] {\footnotesize $\x$};
			\foreach \y in {-2,-1,1,2,3}\draw (0.1,\y)--(-0.1,\y) node [left] {\footnotesize $\y$};
	\end{tikzpicture}}
	\loigiai{}
\end{vd}
\begin{vd}
	Cho hàm số $y=f(x)=-2x^2-7$. Xét tính đồng biến và nghịch biến của hàm số trên các khoảng $(-4;0)$; $(3;10)$.
		\loigiai{}
\end{vd}


\begin{vd}
	Xét tính đồng biến và nghịch biến của hàm số $y=f(x)=x^2+10x+9$ trên $(-5;+\infty)$.
		\loigiai{}
\end{vd}

\begin{vd}
	Xét tính đồng biến và nghịch biến của hàm số $y=f(x)=\dfrac{x}{x-7}$ trên các khoảng $(-\infty;7)$;  $(7;+\infty)$.
		\loigiai{}
\end{vd}

\begin{dang}{Vẽ đồ thị hàm số cho bởi nhiều biểu thức}
	
\end{dang}
\begin{vd}
	Tìm tập xác định và vẽ đồ thị các hàm số sau:
	\begin{tasks}(2)
		\task $f(x)=\heva{& 2x & \text{ với }& x \ge 0 \\ & -x & \text{ với }& x<0.}$
		\task $f(x)=\heva{& -x^2 & \text{ với }& x\le 1\\ & 1 & \text{ với }& x>1.}$
		\task $f(x)= \big|x\big|$.
		\task $f(x)= \big|x+2\big|$.
	\end{tasks}
	\loigiai{}
\end{vd}


\begin{dang}{Viết công thức hàm số cho một số bài toán thực tế}
\end{dang}

\begin{vd}%[0D2T1-1]
	Theo quyết định số 2019/QĐ-BĐVN ngày 01/11/2018 của Tổng công ty Bưu điện Việt Nam, giá cước dịch vụ Bưu chính phổ cập đối với dịch vụ thư cơ bản và bưu thiếp trong nước có khối lượng đến $250$g như trong bảng sau
	\immini{\begin{enumerate}[a)]
			\item Số tiền dịch vụ thư cơ bản phải trả $y$ (đồng) có là hàm số của khối lượng thư cơ bản $x$ (g) hay không? Nếu đúng, hãy xác định những công thức tính $y$.
			\item Tính số tiền phải trả khi bạn Dương gửi thư có khối lượng $150$g, $200$g.
	\end{enumerate} }
	{\vspace{0.6 cm}
		\begin{tabular}{|c|c|}
			\hline
			\textcolor{blue}{Khối lượng đến $250$ g} & \textcolor{blue}{Mức cước (đồng)}\\
			\hline
			Đến $20$ g & $4000$\\
			\hline
			Trên $20$ g đến $100$ g & $6000$ \\
			\hline
			Trên $100$ g đến $250$ g & $8000$\\
			\hline
	\end{tabular}}
	\loigiai{
		\begin{enumerate}[a)]
			\item Số tiền dịch vụ thư cơ bản phải trả $y$ là hàm số của $x$ vì mỗi giá trị của $x$ (chính là khối lượng của thư) có đúng một giá trị của $y$ (mức cước hay số tiền phải trả) tương ứng.\\
			Quan sát bảng ta thấy:
			\begin{itemize}
				\item[$\bullet$] Nếu khối lượng thư đến $20$ g hay $0<x \le 20$ thì mức cước phải trả là $4000$ đồng hay $y= 4000$.
				\item[$\bullet$] Nếu khối lượng thư trên $20$ g hay $100$ g hay $20 < x \le 100$ g thì mức cước là $6000$ đồng hay $y=6000$.
				\item[$\bullet$] Nếu khối lượng thư trên $100$ g đến $250$ g hay $100 < x \le 250$ thì mức cước là $8000$ đồng hay $y=8000$.
			\end{itemize}
			Vậy ta có công thức xác định $y$ như sau:
			$y=\heva{&4000&\text{nếu}& 0<x\le 20\\&6000&\text{nếu}& 20 <x\le 100\\&8000&\text{nếu}& 100<x\le 250.}$
			\item Vì $100 < 150 <250$ và $100 <200 <250$ nên bức thư có khối lượng $150$ g thì cần trả cước là $8000$ đồng và bức thư có khối lượng $200$ g cũng cần trả cước là $8000 $ đồng. \\
			Vậy tổng số tiền phải trả khi bạn Dương gửi thư có khối lượng $150$ g và $200$ g là \\ \centerline{$8000 + 8000 = 16 000$ (đồng).}
		\end{enumerate}
	}
\end{vd}


\begin{vd}
	Nhiệt độ ở mặt đất đo được khoảng $30^{\circ} \mathrm{C}$. Biết rằng cứ lên $1 \mathrm{~km}$ thì nhiệt độ giảm đi $5^{\circ}$.
	\begin{tasks}(1)
		\task Hãy lập hàm số $T$ theo $h$, trong đó $T$ tính bằng độ $\left({ }^{\circ}\right)$ và $h$ tính bằng ki-lô-mét $(\mathrm{km})$.
		\task Hãy tính nhiệt độ khi ở độ cao $3 \mathrm{~km}$ so với mặt đất.
	\end{tasks}
	\loigiai{
		\begin{enumerate}[a)]
			\item Hàm số $T$ theo $h$ là $T=30-5h$.
			\item Thay $h=3$ vào công thức $T=30-5h$, ta được $
			T=30-5 \cdot 3=15$.\\
			Vậy khi lên độ cao $3 \mathrm{~km}$ thì nhiệt độ tại đó là $15^{\circ}$
	\end{enumerate}}
\end{vd}

\begin{vd}%[0D2T1-1]
	Một công ty viễn thông A cung cấp dịch vụ truyền hình cáp với mức phí ban đầu là 300000 đồng và mỗi tháng phải đóng 150000 đồng. Công ty viễn thông B cũng cung cấp dịch vụ truyền hình cáp nhưng không tính phí ban đầu và mỗi tháng khách hàng sẽ phải đóng 200000 đồng.
	\begin{tasks}(1)
		\task Gọi $T$ (đồng) là số tiền khách hàng phải trả cho mỗi công ty viễn thông trong $t$ (tháng) sử dụng dịch vụ truyền hình cáp. Khi đó hãy lập hàm số $T$ theo $t$ đối với mỗi công ty.
		\task Tính số tiền khách hàng phải trả sau khi sử dụng dịch vụ truyền hình cáp trong 5 tháng đối với mỗi công ty.
		\task Khách hàng cần sử dụng dịch vụ truyền hình cáp trên mấy tháng thì đăng kí bên công ty viễn thông A sẽ tiết kiệm chi phí hơn?
	\end{tasks}
	\loigiai{
		\begin{enumerate}[a)]
			\item Hàm số $T$ theo $t$ đối với công ty A là $T=150000t+300000$.\\
			Hàm số $T$ theo $t$ đối với công ty B là $T=200000t$.
			\item Thay $t=5$ lần lượt vào hai công thức trên, ta được
			\begin{itemize}
				\item [$\bullet$] Số tiền phải trả trong 5 tháng khi sử dụng dịch vụ truyền hình cáp của công ty A là $150000.5+300000=1050000$ đồng.
				\item [$\bullet$] Số tiền phải trả trong 5 tháng khi sử dụng dịch vụ truyền hình cáp của công ty B là $200000.5=1000000$ đồng.
			\end{itemize}
			\item Để dịch vụ truyền hình cáp của công ty A lợi hơn dịch vụ truyền hình cáp của công ty B thì:
			$$
			150000t+300000<200000t \Leftrightarrow 300000<50000t \Leftrightarrow t>6
			$$
			Vậy nếu sử dụng từ 7 tháng trở lên thì sử dụng dịch vụ truyền hình cáp bên công ty A sẽ có lợi hơn.
	\end{enumerate}}
\end{vd}
\subsection{BÀI TẬP TỰ LUYỆN}

\begin{bt}%[0D2T1-1]
	Trong kinh tế thị trường, lượng cầu và lượng cung là hai khái niệm quan trọng. Lượng cầu chỉ khả năng về số lượng sản phẩm cần mua của bên mua (người tiêu dùng), tuỳ theo đơn giá bán sản phẩm; còn lượng cung chỉ khả năng cung cấp số lượng sản phẩm nảy cho thị trường của bên bán (nhà sản xuất) cũng phụ thuộc vào đơn giá bán sản phẩm.\\
	Người ta khảo sát nhu cầu của thị trường đối với sản phẩm $\mathrm{A}$ theo đơn giá của sản phẩm này và thu được bảng sau:
	\begin{center}
		\begin{tabular}{|c|c|c|c|c|c|}
			\hline Đơn giá sản phẩm $A$ (đơn vị: nghìn đồng) & 10 & 20 & 40 & 70 & 90 \\
			\hline Lượng cầu (nhu cầu về số sản phẩm) & 338 & 288 & 200 & 98 & 50 \\
			\hline
		\end{tabular}
	\end{center}
	\begin{enumerate}
		\item Hãy cho biết tại sao bảng giá trị trên xác định một hàm số? Hãy tìm tập xác định và tập giá trị của hàm số đó (gọi là hàm cầu).
		\item Giả sử lượng cung của sản phẩm $A$ tuân theo công thức $y=f(x)=\dfrac{x^{2}}{50}$, trong đó $x$ là đơn giá sản phẩm $A$ và $y$ là lượng cung ứng với đơn giá này. Hãy điền các giá trị của hàm số $f(x)$ (gọi là hàm cung) vào bảng sau
		\begin{center}
			\begin{tabular}{|c|l|l|l|l|l|}
				\hline Đơn giá sản phẩm $A$ (đơn vị: nghìn đồng) & 10 & 20 & 40 & 70 & 90 \\
				\hline Lượng cung (khả năng cung cấp về số sản phẩm) & & & & & \\
				\hline
			\end{tabular}
		\end{center}
		\item Ta nói thị trường của một sản phẩm là cần bằng khi lượng cung và lượng cầu bằng nhau. Hãy tìm đơn giá $x$ của sản phẩm $A$ khi thị trường cân bằng.
	\end{enumerate}
	\loigiai{
		\begin{enumerate}
			\item Có thể thấy với mỗi mức đơn giá, đều có duy nhất một giá trị về lượng cầu. Do vậy, bảng giá trị đã cho ở đề bài xác định một hàm số.\\
			Hàm số có tập xác định $\mathscr{D}=\{10;20;40;70;90\}$ và tập giá trị $\mathscr{T}=\{338;288;200;98;50\}$.\\
			\item \hfill
			\begin{center}
				\begin{tabular}{|c|c|c|c|c|c|}
					\hline Đơn giá sản phẩm $A$ (đơn vị nghìn đồng) & 10 & 20 & 40 & 70 & 90 \\
					\hline Lượng cung (khả năng cung cấp về số sản phẩm) & 2 & 8 & 32 & 98 & 162 \\
					\hline
				\end{tabular}
			\end{center}
			\item Dựa vào hai bảng giá trị của lượng cung và lượng cầu, ta tìm được giá trị $x=70$ thì lượng cung và lượng cầu đều bằng $98$.\\
			Vậy thị trường của sản phẩm $A$ cân bằng khi đơn giá của sản phẩm $\mathrm{A}$ này là $70\,000$(đồng).
		\end{enumerate}
	}
\end{bt}

\begin{bt}
	\immini{Cho hàm số $y=f(x)$ có tập xác định là $\mathscr{D}$ và đồ thị là đường liền nét được vẽ trên miền $\mathscr{D}$ như hình bên. 
		\begin{tasks}(1)
			\task Trong các điểm $A(2;2)$, $B(0;1)$, $C(4;2)$, $D(-3;-1)$, điểm nào thuộc $(C)$? điểm nào không thuộc $(C)$?
			\task Tìm tập xác định $\mathscr{D}$ và tập giá trị $\mathscr{T}$ của hàm số  $y=f(x)$. 
			\task Tìm các khoảng đồng biến và nghịch biến của hàm số trên miền $\mathscr{D}$.
		\end{tasks}
		
	}{
		\begin{tikzpicture}[smooth,samples=300,scale=0.7,>=stealth,font=\footnotesize]
			\draw[line width=0.1pt,gray!80] (-4,-2) grid (5,4);
			\draw[->] (-4.2,0)--(5.7,0) node[below]{$x$};
			\draw[->] (0,-2.2)--(0,4.5) node[right]{$y$};
			\draw (0,0) node[above right]{$O$};
			\draw[line width=1pt,magenta] (-3,-1)--(1,0)-- (3,3)--(5,1);
			\draw[fill=blue] (-3,-1) circle(2.5pt) (1,0) circle(2.5pt) (3,3) circle(2.5pt) (5,1) circle(2.5pt);
			\foreach \x in {-4,-3,-2,-1,1,2,3,4,5}\draw (\x,0.1)--(\x,-0.1) node [below] {\footnotesize $\x$};
			\foreach \y in {-2,-1,1,2,3}\draw (0.1,\y)--(-0.1,\y) node [left] {\footnotesize $\y$};
	\end{tikzpicture}}
	\loigiai{}
\end{bt}

\begin{bt}
	Cho hai hàm số $f(x)=x^2-2x$ và $g(x)=1-\sqrt{x}$. Tính giá trị $\dfrac{f(-1)}{g(4)}$.
		\loigiai{}
\end{bt}

\begin{bt}
	Cho hàm số $f(x)=4-\sqrt[3]{x}$.
	\begin{listEX}[3]
		\item Tính $f(-8)$.
		\item Tính $f(a^3)$.
		\item Tìm $a>0$ thỏa $f(a^6)=0$
	\end{listEX}
	\loigiai{}
\end{bt}

\begin{bt}
	Cho hàm số $f(x)=\heva{&x^2-2x-1 &\text{với } x\le 0 \\&\dfrac{x+1}{x^2+x+1} &\text{với } x > 0}$. Tính giá trị của hàm số đó tại $x=1; x=0; x=-2$.
		\loigiai{}
\end{bt}

\begin{bt}%[0D2B1-2]
	Tìm tập xác định của mỗi hàm số sau
	\begin{tasks}(2)
		\task $y=-x^2$.
		\task $y = \sqrt{2-3x}$.
		\task $y = \dfrac{4}{x+1}$.
		\task $y = \heva{& 1 &\text{ nếu }& x \in \mathbb{Q} \\ & 0 &\text{ nếu }& x \in \mathbb{R} \setminus \mathbb{Q}.}$
	\end{tasks}
	\loigiai{
		\begin{enumerate}[a)]
			\item Hàm số $y = -x^2$ có tập xác định $\mathscr{D} = \mathbb{R}$.
			\item Biểu thức $\sqrt{2-3x}$ có nghĩa khi $2 - 3x \geq 0 \Leftrightarrow x \leq \dfrac{2}{3}$. \\
			Vậy tập xác định $\mathscr{D} = \left(-\infty;\dfrac{2}{3}\right]$.
			\item Biểu thức $\dfrac{4}{x+1}$ có nghĩa khi và chỉ khi $x \ne - 1$. \\
			Vậy tập xác định $\mathscr{D} = \mathbb{R} \setminus \{-1 \}$.
			\item Tập xác định $\mathscr{D} = \mathbb{R}$.
		\end{enumerate}
	}
\end{bt}

\begin{bt}%[0D2B1-2]%
	Tìm tập xác định của các hàm số sau
	\begin{tasks}(2)
		\task $y=2-4x$.
		\task $y=\dfrac{x-3}{5-2x}$.
		\task $y=\dfrac{x}{x^2-3x+2}$.
		\task $y=\dfrac{2x+1}{(x-2) \left( x^2-4x+3 \right)}$.
	\end{tasks}
	\loigiai{
		\begin{enumEX}{1}
			\item Ta có $\mathscr{D}=\mathbb{R}$.
			\item Hàm số xác định khi $5-2x \neq 0 \Leftrightarrow x \neq \dfrac{5}{2}$.\\
			Tập xác định $\mathscr{D}=\mathbb{R} \setminus \left\{ \dfrac{5}{2} \right\}$.
			\item Hàm số xác định khi $x^2-3x+2 \neq 0 \Leftrightarrow \heva{&x \neq 1 \\ & x \neq 2}$.\\
			Tập xác định $\mathscr{D}=\mathbb{R} \setminus \{1;2\}$.
			\item Hàm số xác định khi $(x-2) \left( x^2-4x+3 \right) \neq 0 \Leftrightarrow \heva{& x-2\neq 0\\& x^2-3x+2\neq 0} \Leftrightarrow \heva{& x \neq 1\\ & x \neq 2 \\ & x \neq 3}$.\\
			Tập xác định $\mathscr{D}=\mathbb{R} \setminus \{1;2;3\}$.
		\end{enumEX}
	}
\end{bt}

\begin{bt}%[0D2B1-2]%
	Tìm tập xác định của các hàm số
	\begin{tasks}(2)
		\task $ y= \dfrac{\sqrt{ 4-2x}}{x^2-6x+5}$.
		\task $ y= \sqrt{ \dfrac{x^2}{x-1}}$.
	\end{tasks}
	\loigiai{
		\begin{enumerate}[a)]
			\item Hàm số xác định $\Leftrightarrow \heva{& 4-2x \geq 0\\&x^2-6x+5 \neq 0} \Leftrightarrow \heva{& x\leq 2\\& x \neq 1\\&x \neq 5} \Leftrightarrow \heva{& x\leq 2\\&x\neq 1.}$\\
			Vậy tập xác định của hàm số là $\mathscr {D} = \left(-\infty ;2\right] \backslash \{1\}$.
			\item Hàm số xác định $\Leftrightarrow \heva{& x\neq 1\\& \dfrac{x^2}{x-1}\geq 0} \Leftrightarrow \hoac{&x=0\\&x >1.}$\\
			Vậy tập xác định của hàm số là $\mathscr {D} = \{0\} \cup (1;+\infty)$.
		\end{enumerate}
	}
\end{bt}

\begin{bt}%[0D2B1-2]
	Tìm tập xác định các hàm số sau
	\begin{tasks}(2)
		\task $f(x)=\dfrac{4x-1}{\sqrt{2x-5}}$.
		\task $f(x)=\heva{& \dfrac{1}{x-3} & \text{ với }& x\ge 0\\ & 1 & \text{ với }& x<0.}$
	\end{tasks}
	\loigiai{
		\begin{enumerate}[a)]
			\item Hàm số xác định khi và chỉ khi $2x-5>0\Leftrightarrow x>\dfrac{5}{2}$.\\
			Tập xác định $\mathscr{D}=\left(\dfrac{5}{2};+\infty\right)$.
			\item Với $x\ge 0$ thì $f(x)=\dfrac{1}{x-3}$, khi đó $f(x)$ xác định khi $x-3\ne 0\Leftrightarrow x\ne 3$.\\
			Với $x<0$, $f(x)=1$ luôn xác định và nhận giá trị bằng $1$.\\
			Vậy tập xác định $\mathscr{D}=\mathbb{R}\setminus \{3\}$.
		\end{enumerate}	
	}
\end{bt}

\begin{bt}%[0D2B1-3]
	Xét sự biến thiên của hàm số sau trên khoảng $(1;+\infty)$.
	\begin{listEX}[2]
		\item $y=\dfrac{3}{x-1}$.
		\item $y=x+\dfrac{1}{x}$.
	\end{listEX}
	\loigiai{
		\begin{enumerate}[a)]
			\item Với mọi $x_1,\,x_2\in(1;+\infty),\,\,x_1\ne{x_2}$ ta có
			\[f(x_2)-f(x_1)=\dfrac{3}{x_2-1}-\dfrac{3}{x_1-1}=\dfrac{3(x_1-x_2)}{(x_2-1)(x_1-1)}.\]
			Suy ra $\dfrac{f(x_2)-f(x_1)}{x_2-x_1}=-\dfrac{3}{(x_2-1)(x_1-1)}$.\\
			Vì $x_1> 1,\,\,x_2> 1\Rightarrow\dfrac{f(x_2)-f(x_1)}{x_2-x_1}< 0$ nên hàm số $ y=\dfrac{3}{x-1}$ nghịch biến trên khoảng $(1;+\infty)$.
			\item Với mọi $x_1,\,x_2\in(1;+\infty),\,x_1\ne{x_2}$ ta có
			\[f(x_2)-f(x_1)=\left(x_2+\dfrac{1}{x_2}\right)-\left(x_1+\dfrac{1}{x_1}\right)=(x_2-x_1)\left(1-\dfrac{1}{x_1x_2}\right).\]
			Suy ra $\dfrac{f(x_2)-f(x_1)}{x_2-x_1}=1-\dfrac{1}{x_1x_2}$.\\
			Vì $x_1> 1$, $x_2> 1\Rightarrow\dfrac{f(x_2)-f(x_1)}{x_2-x_1}> 0$.\\
			Vậy hàm số $ y=x+\dfrac{1}{x}$ đồng biến trên khoảng $(1;+\infty)$.
		\end{enumerate}
	}
\end{bt}


\begin{bt}%[0D2K1-3]
	Tìm khoảng đồng biến, nghịch biến của các hàm số sau
		\begin{listEX}[3]
			\item  $f(x)=1-3x$.
			\item  $f(x)=\dfrac{1}{x-3}$.
			\item   $f(x)=|2x-1|$.
		\end{listEX}
	\loigiai{
		\begin{enumerate}[a)]
			\item Hàm số có tập xác định $\mathscr{D}=\mathbb{R}$.\\
			Lấy $x_1$, $x_2$ là hai số tuỳ ý thỏa $x_1<x_2$ thì 
			$$f(x_1)-f(x_2)=-3(x_1-x_2)>0$$
			nên hàm số đã chi nghịch biến trên $\mathbb{R}$.
			\item Hàm số $f(x)=\dfrac{1}{x-3}$ xác định khi $x-3\ne 0\Leftrightarrow x\ne 3$\\
			Suy ra tập xác định $\mathscr{D}=\mathbb{R}\setminus \{3\}$.\\
			Lấy $x_1$, $x_2$ là hai số tuỳ ý cùng thuộc mỗi khoảng $(-\infty;3)$, $(3;+\infty)$ sao cho $x_1<x_2$, ta có
			$$f(x_1)-f(x_2)=\dfrac{1}{x_1-3}-\dfrac{1}{x_2-3}=\dfrac{x_2-x_1}{(x_1-3)(x_2-3)}.$$
			Do $x_1<x_2$ nên $x_2-x_1>0$.\\
			Mặt khác, khi lấy $x_1$ và $x_2$ cùng nhỏ hơn $3$ hoặc cùng lớn hơn $3$, ta đều có $x_1-3$ và $x_2-3$ luôn cùng dấu nên $(x_1-3)(x_2-3)<0$.\\
			Suy ra $f(x_1)-f(x_2)>0\Rightarrow f(x_1)>f(x_2)$.\\
			Vậy hàm số $f(x)$ luôn nghịch biến trên các khoảng $(-\infty;3)$ và $(3;+\infty)$.
			\item Hàm số $f(x)=|2x-1|$ còn được viết lại như sau
			$$f(x)=|2x-1|=\heva{& 2x-1 & \text{ với } 2x-1\ge 0\\ & -(2x-1)& \text{với } 2x-1<0}=\heva{& 2x-1 & \text{với }x\ge \dfrac{1}{2}\\ & -2x+1 & \text{với }x<\dfrac{1}{2}.}$$
			Xét hàm số $g(x)=2x-1$. Hàm số này xác định trên $\mathbb{R}$.\\
			Lấy hai số $x_1$, $x_2$ tuỳ ý sao cho $x_1<x_2$, ta có 
			$$x_1<x_2\Rightarrow 2x_1<2x_2\Rightarrow 2x_1-1<2x_2-1\Rightarrow g(x_1)<g(x_2).$$
			Suy ra $g(x)$ đồng biến trên $\mathbb{R}$ nên $f(x)$ đồng biến trên $\left(\dfrac{1}{2};+\infty\right)$.\\
			Xét hàm số $h(x)=-2x+1$. Hàm số này xác định trên $\mathbb{R}$.\\
			Lấy hai số $x_1$, $x_2$ tuỳ ý sao cho $x_1<x_2$, ta có 
			$$ x_1<x_2\Rightarrow -2x_1>-2x_2\Rightarrow -2x_1+1>-2x_2+1\Rightarrow h(x_1)>h(x_2).$$
			Suy ra $h(x)$ nghịch biến trên $\mathbb{R}$ nên $f(x)$ nghịch biến trên $\left(-\infty;\dfrac{1}{2}\right)$.\\
			Vậy hàm số $f(x)=|2x-1|$ đồng biến trên khoảng $\left(\dfrac{1}{2};+\infty\right)$ và nghịch biến trên khoảng $\left(-\infty;\dfrac{1}{2}\right)$.
			
		\end{enumerate}
	}
\end{bt}

\begin{bt}%[0D2B3-3]
	Vẽ đồ thị các hàm số sau
	\begin{tasks}(2)
		\task $f(x)=\heva{& x^2 & \text{ với }& x\le 2\\ & x+2 &\text{với }& x>2.}$
		\task $f(x)=|x+3|-2$.
	\end{tasks}
	\loigiai{
		\begin{enumerate}[a)]
			\item Ta vẽ đồ thị $g(x)=x^2$ và giữ phần đồ thị ứng với $x\le 2$, vẽ đồ thị $h(x)=x+2$ và giữ phần đồ thị ứng với $x>2$. Ta được đồ thị như hình vẽ
			\begin{center}
				\begin{tikzpicture}[scale=0.7, font=\footnotesize, line join=round, line cap=round,>=stealth]
					\def \xmin{-3};
					\def \xmax{6};
					\def \ymin{-1};
					\def \ymax{9};
					\draw[->] (\xmin, 0.) -- (\xmax,0.) node[anchor=north] {$x$};
					\draw[->] (0.,\ymin) -- (0.,\ymax) node[anchor=west] {$y$};
					\clip(\xmin-0.1,\ymin-0.1) rectangle (\xmax+0.1,\ymax+0.1);
					\draw[smooth] plot[domain=-3:2] (\x,{(\x)^2});
					\draw[dashed] plot[domain=2:3] (\x,{(\x)^2});
					\draw[smooth] plot[domain=2:6] (\x,{\x+2});
					\draw[dashed] plot[domain=-3:2] (\x,{\x+2});
					\foreach \i in {-2,-1,1,2,3,4,5}
					\draw[fill=black] (\i,0) circle(1pt) node[below]{$\i$};
					\foreach \j in {1,2,3,4,5,6,7,8}
					\draw[fill=black] (0,\j) circle(1pt) node[left]{$\j$};
					\draw[fill=black] (0,0) circle(1pt) node[below left]{$O$} (4,4) node[above]{$y=f(x)$} ;
					
				\end{tikzpicture}
			\end{center}
			\item Ta có $f(x)=|x+3|-2=\heva{& x+3-2 & \text{với } x+3\ge 0\\ & -(x+3)-2& \text{với }x+3<0}=\heva{& x+1& \text{với }x\ge -3\\ & -x-5& \text{với } x<-3.}$\\
			Ta vẽ đồ thị $g(x)=x+1$ và giữ lại phần đồ thị ứng với $x\ge -3$, đồng thời vẽ đồ thị $h(x)=-x-5$ và giữ lại phần đồ thị ứng với $x<-3$. Ta được đồ thị của $f(x)$ như sau
			\begin{center}
				\begin{tikzpicture}[scale=0.7, font=\footnotesize, line join=round, line cap=round,>=stealth]
					\def \xmin{-6};
					\def \xmax{2};
					\def \ymin{-5};
					\def \ymax{3};
					\draw[->] (\xmin, 0.) -- (\xmax,0.) node[anchor=north] {$x$};
					\draw[->] (0.,\ymin) -- (0.,\ymax) node[anchor=west] {$y$};
					\clip(\xmin-0.1,\ymin-0.1) rectangle (\xmax+0.1,\ymax+0.1);
					\draw[smooth] plot[domain=-3:2] (\x,{\x+1});
					\draw[dashed] plot[domain=-6:-3] (\x,{\x+1});
					\draw[smooth] plot[domain=-6:-3] (\x,{-\x-5});
					\draw[dashed] plot[domain=-3:2] (\x,{-\x-5});
					\foreach \i in {-5,-4,-3,-2,-1,1}
					\draw[fill=black] (\i,0) circle(1pt) node[below]{$\i$};
					\foreach \j in {1,2,-1,-2,-3,-4,-5}
					\draw[fill=black] (0,\j) circle(1pt) node[left]{$\j$};
					\draw[fill=black] (0,0) circle(1pt) node[below left]{$O$} (1,1) node[below]{$y=f(x)$} (-3,-2) circle(1pt);
					
				\end{tikzpicture}
			\end{center}
		\end{enumerate}
	}
\end{bt}

\begin{bt}%[0D2T1-1]
	Một lớp muốn thuê một chiếc xe khách cho chuyến tham quan với tổng đoạn đường cần di chuyển trong khoảng từ $550$ km đến $600$ km, có hai công ty được tiếp cận để tham khảo giá. Công ty A có giá khởi đầu là $3{,}75$ triệu đồng cộng thêm $5000$ đồng cho mỗi km chạy xe. Công ty B có giá khởi đầu là $2{,}5$ triệu đồng cộng thêm $7500$ đồng cho mỗi km chạy xe. Lớp đó nên chọn công ty nào để chi phí là thấp nhất?
	\loigiai{
		Hàm số biểu thị giá tiền theo km của công ty A là $y_A= 3750+5x$ (đồng).\\
		Hàm số biểu thị giá tiền theo km của công ty B là $y_B=2500 + 7{,}5x$ (đồng).\\
		Với $550 \le x \le 600$.\\
		Ta có $y_A -y_B = 1250 -2{,}5x$.\\
		Do $550 \le x \le 600 \Leftrightarrow -250 \le 1250 -2{,}5x \le -120$ nên $y_A -y_B <0$.\\
		Vậy chi phí thuê xe công ty A thấp hơn công ty B.
	}

\end{bt}

\begin{bt}
	Một người đang dự định đi mua xe máy mà muốn chọn 1 trong hai loại xe sau:
	\begin{itemize}
		\item [$\bullet$] \textbf{Loại 1:} Có giá 27000000 (đồng) và trung bình số ki-lô-mét đi được mỗi lít xăng là 58 km/lít xăng
		\item [$\bullet$] \textbf{Loại 2:} Có giá 30000000 (đồng) và trung bình số ki-lô-mét đi được mỗi lít xăng là 62,5 km/lít xăng.
	\end{itemize}
	Biết rằng giá trung bình của 1 lít xăng là 18000 (đồng). Người ta dự tính mua xe máy để sử dụng khoảng 8 năm, mỗi năm người đó ước chừng đi khoảng 7250 km.
	\begin{tasks}(1)
		\task Gọi $s$ (đồng) là chi phí từng năm theo thời gian $t$ (năm) của mỗi loại xe (bao gồm tiền mua xe và tiền xăng). Lập hàm số của $s$ theo $t$.
		\task Nên chọn loại xe nào để tiết kiệm hơn? Tại sao?
	\end{tasks}
	\loigiai{
		\begin{enumerate}[a)]
			\item Đối với xe loại 1, mổi năm xe tiêu thụ hết
			$7250: 58=125$ (lít).
			Suy ra mỗi năm xe loại 1 tiêu thụ hết
			$125 \times  18000=2250000$ (đồng).\\
			Hàm số của $s$ theo $t$ đối với xe loại 1 là
			$$
			s=27000000+2250000 . t
			$$
			Đối với xe loại 2, mỗi năm xe tiêu thụ hết
			$
			7250: 62,5=116$ (lít). Suy ra mỗi năm xe loại 2 tiêu thụ hết
			$116\times  18000=2088000$ (đồng).\\
			Hàm số của s theo t đối với xe loại 2 là
			$$
			s=30000000+2088000.t
			$$
			\item Trong thời gian sử dụng 8 năm $(t=8)$, xe loại 1 tiêu thụ hết
			$$
			s=27000000+2250000\times 8=45000000 \text { (đồng). }
			$$
			Trong thời gian sử dụng 8 năm $(t=8)$, xe loại 2 tiêu thụ hết
			$$
			s=30000000+2088000\times 8=46704000 \text { (đồng). }
			$$
			Vậy nên chọn xe loại 1 để tiết kiệm hơn
		\end{enumerate}
	}
\end{bt}

\begin{bt}
	Bảng giá cước của một hãng Taxi như sau:
\begin{center}
	\begin{tikzpicture}[font=\small , thick, >=latex]
		\def\drong{5.5}
		\def\fonta{ \fontsize{10pt}{0pt}\selectfont } %bảng giá cước
		\def\fontb{ \fontsize{8pt}{0pt}\selectfont }%% Chữ màu vàng
		\def\fontc{ \fontsize{11pt}{0pt}\selectfont }%% Số đỏ
		\def\fontd{ \fontsize{5pt}{0pt}\selectfont }%% Số nhỏ
		
		\path (0,0)coordinate(O)
		($(O)+(-\drong,0)$)coordinate(NW)
		($(O)+(\drong,0)$)coordinate(NE)
		;
		\filldraw  [rounded corners=0.5cm, green!70!blue!70!black] (NW) rectangle ($(NE)+(0,-4.5)$) ; %% nền xanh lá
		\filldraw [yellow!40!green] ($(NW)+(0,-0.8)$) rectangle ($(NE)+(0,-3.4)$) ; % nền xanh lá nhạt
		\filldraw [white] ($(NW)+(0.2,-1)$) rectangle ($(NE)+(-0.2,-3.4)$) ; % Nền trắng
		\draw [line width=1pt, green!70!blue!90!black]
		($(NW)+(0.2,-1)$) -- ($(NE)+(-0.2,-1)$) 
		($(NW)+(0.2,-1.9)$) -- ($(NE)+(-0.2,-1.9)$) 
		($(NW)+(0.2,-2.7)$) -- ($(NE)+(-0.2,-2.7)$) 
		($(NW)+(0.2,-3.4)$) -- ($(NE)+(-0.2,-3.4)$) 
		($(NW)+(0.2,-1)$) -- ($(NW)+(0.2,-3.4)$) 
		($(NE)+(-0.2,-1)$) -- ($(NE)+(-0.2,-3.4)$) 
		($(NE)+(-3.5,-1)$) -- ($(NE)+(-3.5,-3.4)$)
		($(O)+(-1.7,-1)$) -- ($(O)+(-1.7,-2.7)$)
		;
		
		\draw ($(O)+(0,-0.4)$)node[ white]{\fonta\bfseries \sffamily Bảng Giá Cước
			\fontb\bfseries\textcolor{yellow!90!black}{ - Taxi Fare Quote
		}};
		
		\path ($(O)+(-0.65*\drong,-1.3)$)node[red!90!black, xscale=0.8]{\fontb \bfseries GIÁ MỞ CỬA}
		($(O)+(0.65*\drong,-1.3)$)node[red!90!black, xscale=0.8]{\fontb \bfseries TỪ KM THỨ 31}
		($(O)+(0,-1.3)$)node[red!90!black, xscale=0.8]{\fontb \bfseries GIÁ KM TIẾP THEO}
		($(O)+(-0.65*\drong,-1.7)$)node[black, xscale=0.7]{\fontd\bfseries First 0.7km }
		($(O)+(0.65*\drong,-1.7)$)node[black, xscale=0.7]{\fontd\bfseries From 31st km }
		($(O)+(0,-1.7)$)node[black, xscale=0.7]{\fontd\bfseries Each additional 0.8 km up to 30th km}
		
		
		($(O)+(0.65*\drong,-2.9)$)node[red!90!black, xscale=0.7]{\fontd\bfseries\sffamily GIÁ TIỀN ĐÃ BAO GỒM 10\% THUẾ VAT }
		($(O)+(0.65*\drong,-3.2)$)node[black, xscale=0.9]{\fontd\bfseries\sffamily (10\% VAT  INCLUDED) }
		($(O)+(-0.3*\drong,-3)$)node[red!90!black, xscale=0.7]{\fontb\bfseries\sffamily Phí thời gian chờ \textcolor{black}{(Each 5 minutes of wait time: VNĐ 3000)}  }
		
		;
		
		
		\path ($(O)+(-0.65*\drong,-2.3)$)node[red!90!black, yscale=1.3]{\fontc \bfseries \sffamily 11.000Đ/ 0.7Km}
		($(O)+(0.65*\drong,-2.3)$)node[red!90!black, yscale=1.3]{\fontc \bfseries \sffamily 12.500Đ/ 1Km}
		($(O)+(0,-2.3)$)node[red!90!black, yscale=1.3]{\fontc \bfseries \sffamily 15.800Đ/ 1Km} ;	
		
		\path ($(O)+(0,-3.6)$)node[yellow!90!black]{\fontd\sffamily\bfseries QUÝ KHÁCH VUI LÒNG THANH TOÁN PHÍ CẦU ĐƯỜNG, PHÀ VÀ BẾN BẢI (NẾU CÓ) }
		($(O)+(0,-3.85)$)node[white]{\fontd\sffamily\bfseries All tolls, road \& bridge use charge or parking fee shall be surcharged (if any) }
		($(O)+(0,-4.1)$)node[yellow!90!black]{\fontd\sffamily\bfseries TAXI MAI LINH CAM KẾT TÍNH GIÁ CƯỚC THEO ĐỒNG HỒ TÍNH TIỀN }
		($(O)+(0,-4.3)$)node[white]{\fontd\sffamily\bfseries Metter - bassed Fare }
		;
		
		
	\end{tikzpicture}
\end{center}
	\begin{tasks}(1)
		\task Gọi $y$ (đồng) là số tiền khách hàng phải trả sau khi đi $x$ (km). Lập hàm số của $y$ theo $x$ (giả sử rằng không có phát sinh chi phí khác).
		\task Một hành khách thuê taxi đi quãng đường 40 km phải trả số tiền là bao nhiêu?
	\end{tasks}
	\loigiai{
	\begin{enumerate}[a)]
		\item Nếu quãng đường khách hàng đi không quá $0,7$ km, ta có hàm số là $y=11000$.\\
		Nếu quãng đường khách hàng đi trên $0,7$ km đến $30$ km, ta có hàm số là $y=11000+(x-0,7) \cdot 15800=15800x-60$.\\
		Nếu quãng đường khách hàng đi trên $30$ km, ta có hàm số là $y=11000+(30-0,7) \cdot 15800+(x-30) \cdot 12500=12500x+98940
		$.\\
		Vậy ta có hàm số là 
		$$y=\heva{& 11000 &\text{ nếu }&  0<x \le 0,7 \\& 15800x-60 &\text{ nếu }&  0,7<x \le 30\\& 12500x+98940 &\text{ nếu }&  x>30}$$
		\item Thay $x=40$ vào công thức $y=12500xx+98940$ (vì $40 \mathrm{~km}>30 \mathrm{~km}$ ), ta được
		$
		y=12500.40+98940=598940
		$.\\
		Vậy hành khách phải trả số tiền là 598940 đồng.
\end{enumerate}}
\end{bt}
% \subsection{BÀI TẬP TRẮC NGHIỆM}
\Opensolutionfile{ans}[ans/ans0D6-B1]

\begin{ex}%[0D2Y1-1]
	Điểm nào sau đây thuộc đồ thị hàm số $y = 2x^2 + x - 3$?
	\choice
	{\True $\left(0;-3\right)$}
	{$\left(-2;1\right)$}
	{$\left(-1;0\right)$}
	{$\left(3;-7\right)$}
	\loigiai{Thử tọa độ các điểm vào hàm số ta được điểm $\left(0;-3\right)$ thuộc đồ thị hàm số.
	}
\end{ex}


\begin{ex}%[0D2Y1]
	Điểm nào sau đây thuộc đồ thị hàm số  $y=3x^3-2x+1$?
	\choice
	{$\left( -1;2 \right)$}
	{$\left( 1;1 \right)$}
	{$\left( 0;0 \right)$}
	{\True  $\left( 1;2 \right)$}
	\loigiai
	{\begin{itemize}
			\item Với $ x = -1 \Rightarrow y = 0 $ nên $ (-1;2) $ không thuộc đồ thị hàm số $y=3x^3-2x+1$.
			\item Với $ x = 1 \Rightarrow y = 2 $ nên $ (1;2) $ thuộc đồ thị hàm số $y=3x^3-2x+1$.
		\end{itemize}
	}
\end{ex}

\begin{ex}%[0D2B1]
	Tìm tập xác định $\mathcal{D}$ của hàm số $y=\dfrac{x-1}{x-2}$.
	\choice
	{\True $\mathcal{D}=\mathbb{R}\setminus\{2\}$}
	{$\mathcal{D}=\mathbb{R}\setminus\{1\}$}
	{$\mathcal{D}=\mathbb{R}$}
	{$\mathcal{D}=\mathbb{R}\setminus\{1;2\}$}
	\loigiai{
		\begin{itemize}
			\item [$\bullet$] Điều kiện xác định $x-2 \ne 0 \Leftrightarrow x \ne 2$.
			\item [$\bullet$] Suy ra, tập xác định là $\mathcal{D}=\mathbb{R}\setminus\{2\}$.
		\end{itemize}
	}
\end{ex}

\begin{ex}%[0D2B1]
	Tìm tập xác định $\mathcal{D}$ của hàm số $y=\dfrac{x-2}{x^2-2x+2}$.
	\choice
	{$\mathcal{D}=\mathbb{R}\setminus\{1\}$}
	{$\mathcal{D}=\mathbb{R}\setminus\{2\}$}
	{\True $\mathcal{D}=\mathbb{R}$}
	{$\mathcal{D}=\mathbb{R}\setminus\{1;2\}$}
	\loigiai{
		\begin{itemize}
			\item [$\bullet$] Điều kiện xác định $x^2-2x+2 \ne 0$ (luôn đúng).
			\item [$\bullet$] Suy ra, tập xác định là $\mathcal{D}=\mathbb{R}$.
		\end{itemize}
	}
\end{ex}

\begin{ex}%[0D2B1]
	Tìm tập xác định $\mathcal{D}$ của hàm số $y=\sqrt{x-2}$.
	\choice
	{$\mathcal{D}=\mathbb{R}\setminus\{2\}$}
	{$\mathcal{D}=(2;+\infty)$}
	{$\mathcal{D}=(-\infty;2)$}
	{\True $\mathcal{D}=[2;+\infty)$}
	\loigiai{
		\begin{itemize}
			\item [$\bullet$] Điều kiện xác định $x-2 \ge 0 \Leftrightarrow x \ge 2$.
			\item [$\bullet$] Suy ra, tập xác định là $\mathcal{D}=[2;+\infty)$.
		\end{itemize}
	}
\end{ex}

\begin{ex}%[0D2B1]
	Tìm tập xác định của hàm số $y=\dfrac{2x+3}{x^2-x}$.
	\choice{$\mathbb{R}\setminus\{1\}$}
	{$\mathbb{R}$}
	{$\mathbb{R}\setminus\{0\}$}
	{\True $\mathbb{R}\setminus\{0,1\}$}
	\loigiai{
			\begin{itemize}
				\item [$\bullet$] Điều kiện xác định $x^2-x \ne 0 \Leftrightarrow x \ne 0$ và $x \ne 1$.
				\item [$\bullet$] Suy ra, tập xác định là $\mathcal{D}=\mathbb{R}\setminus\{0,1\}$.
			\end{itemize}
	}
\end{ex}

\begin{ex}%[0D2B1-3]%
	\immini{Cho hàm số $y=f(x)$ có tập xác định là $[-3;3]$ và đồ thị của nó được biểu diễn bởi hình bên. Khẳng định nào sau đây đúng?
	\choice
	{\True Hàm số đồng biến trên khoảng $(-3;-1)$}
	{Hàm số đồng biến trên khoảng $(-3;3)$}
	{Hàm số đồng biến trên khoảng $(-3;0)$}
	{Hàm số nghịch biến trên khoảng $(-1;2)$}}{
	\begin{tikzpicture}[>=stealth,x=1.0 cm,y=1.0 cm, scale=0.75]
		\draw[->] (-3.5,0)--(4,0) node[below left] {$x$};
		\draw[->] (0,-2)--(0,4.5) node[below right] {$y$};
		\foreach \x in {,-3,-2,-1,1,2,3,,}
		\draw[shift={(\x,0)},color=black] (0pt,2pt) -- (0pt,-2pt) node[below] {\footnotesize $\x$};
		\foreach \y in {,-1,1,2,3,4,}
		\draw[shift={(0,\y)},color=black] (2pt,0pt) -- (-2pt,0pt) node[left] {\footnotesize $\y$};
		\draw[color=black] (0pt,-10pt) node[right] {\footnotesize $0$};
		\clip(-4,-2.5) rectangle (4,4.5);
		\draw[dashed] (0,-1)--(-3,-1)--(-3,0);
		\draw[dashed] (-1,0)--(-1,1)--(0,1);
		\draw[dashed] (0,4)--(3,4)--(3,0);
		\draw[magenta,thick] (-3,-1)--(-1,1)--(0,1)--(3,4);
\end{tikzpicture}}
	\loigiai{
		Trên khoảng $(-3;-1)$ và $(1;3)$, đồ thị hàm số đi lên, do đó hàm số đồng biến trên $(-3;-1)$ và $(1;3)$.
	}
\end{ex}

\begin{ex}%[0D2B1-3]%
	Khẳng định nào sau đây về hàm số $y=x^2$ là khẳng định đúng?
	\choice
	{Hàm số nghịch biến trên $\mathbb{R}$}
	{Hàm số đồng biến trên $\mathbb{R}$}
	{Hàm số nghịch biến trên $[0;+\infty)$}
	{\True Hàm số đồng biến trên $[0;+\infty)$}
	\loigiai{
		Hàm số $y=x^2$ có hệ số $a=1>0$ nên hàm đồng biến trên $\left(-\dfrac{b}{2a};+\infty \right) $ hay  $[0;+\infty)$ (vì hàm số chỉ bằng $0$ tại một điểm là $x=0$).
	}
\end{ex}

\begin{ex}
	\immini{Cho hàm số $y=f(x)$ liên tục trên $\mathbb{R}$ và có đồ thị như hình bên. Khẳng định nào sau đây đúng?
	\choice
		{Hàm số đồng biến trên khoảng $(1;3)$}
		{Hàm số nghịch biến trên khoảng $(6;+\infty)$}
		{Hàm số đồng biến trên khoảng $(-\infty;3)$}
		{\True Hàm số nghịch biến trên khoảng $(3;6)$}
	}{
		\begin{tikzpicture}[>=stealth,line cap=round,line join=round,x=1cm,y=1cm,scale=0.7]
			\draw[->] (-1,0)--(4,0) node[below] {$x$};
			\draw[->] (0,-2.2)--(0,1.3) node[right] {$y$};
			\node[below left](0,0){$O$};
			\draw[magenta,thick,samples=1000,domain=-0.1:3.5] plot(\x,{(\x-1)^3-2*(\x-1)^2-(\x)+1.5});
			\draw[dashed] (0.78,0.6) -- (0.78,0) node[below] {$2$};
			\draw[dashed] (2.55,-2.13) -- (2.55,0) node[above] {$7$};
	\end{tikzpicture}}
	\loigiai{
		Dựa vào đồ thị thấy hàm số nghịch biến trên khoảng $(2;7)$, do đó hàm số nghịch biến trên khoảng $(3;6)$.
	}
\end{ex}

\begin{ex}%[0D2Y1]
	Tập xác định của hàm số $y=\dfrac{x^2+\sqrt{3-x}}{x-2}$ là
	\choice
	{$(-\infty;3)\backslash \{2\}$}
	{$(2;3]$}
	{\True $(-\infty;3]\backslash \{2\}$}
	{$(-\infty;3]$}
	\loigiai{
		\begin{itemize}
			\item [$\bullet$] Điều kiện xác định $\heva{& 3-x \ge 0\\&x-2 \ne 0} \Leftrightarrow \heva{& x \le 3\\&x \ne 2}$.
			\item [$\bullet$] Suy ra, tập xác định là $\mathcal{D}=(-\infty;3]\backslash \{2\}$.
		\end{itemize}
	}
\end{ex}

\begin{ex}%[0D2B1]
	Tìm tập xác định của hàm số $y=\sqrt{3+x}+\sqrt{6-x}$.
	\choice
	{\True $[-3;6]$}
	{$(-3;6)$}
	{$(-\infty;-3)\cup (6;+\infty)$}
	{$\mathbb{R}\backslash(-3;6)$}
	\loigiai{
		\begin{itemize}
			\item [$\bullet$] Điều kiện xác định $\heva{& 3+x \ge 0\\& 6-x \ge 0} \Leftrightarrow -3\le x\le 6$.
			\item [$\bullet$] Suy ra, tập xác định là $\mathscr{D}=[-3;6]$.
		\end{itemize}
	}
\end{ex}

\begin{ex}%[0D2B1-2]
	Tập xác định của hàm số $y=\dfrac{x+2}{\sqrt{x-1}}+\sqrt{3-x}$ là
	\choice
	{$[1;3]$}
	{\True $(1;3]$}
	{$(-\infty;3]$}
	{$(1;+\infty)$}
	\loigiai{
		$$\dfrac{x+2}{\sqrt{x-1}}+\sqrt{3-x}\text{ có nghĩa }\Leftrightarrow \heva{&x-1>0\\&3-x\geq 0}\Leftrightarrow 1<x\leq 3.$$
		Vậy tập xác định của hàm số đã cho là $(1;3]$.
	}
\end{ex}

\begin{ex}
	\immini{Cho hàm số $y=f(x)$ liên tục trên $\mathbb{R}$ và có đồ thị như hình bên. Tính giá trị biểu thức $P=2f(1)+f(4)-f(3)$
		\haicot
		{$P=1$}
		{$P=0$}
		{\True $P=2$}
		{$P=4$}
	}{\hspace{1cm}
		\begin{tikzpicture}[smooth,samples=300,scale=0.6,>=stealth]
		\draw[black!30!] (-2,-1.5) grid (4,3);
		\draw[->] (-2,0)--(5,0) node[below]{$x$};
		\draw[->] (0,-1.5)--(0,3) node[right]{$y$};
		\foreach \x in {1,2,3,4}{
			\draw (\x,0) node[below]{$\x$};%Ox
		}
		\foreach \y in {1}{
			\draw (0,\y) node[left]{$\y$};%Oy
		}
		\draw (0,0) node[below left]{$O$};
		\draw[domain=-1.8:1,thick] plot(\x,{-(\x)^2+2});
		\draw [magenta,thick](1,1)--(4,1) node[above]{\small $y=f(x)$};
		\draw[fill=black] (0,2) circle(1.5pt) (1,1) circle(1pt);
		\end{tikzpicture}}
		\loigiai{
		Quan sát đồ thị, ta có các kết quả $f(1)=1$, $f(3)=1$ và $f(4)=1$ nên
		$$P=2f(1)+f(4)-f(3)=2+1-1=2.$$
	}
\end{ex}
\begin{ex}%[0D2Y1-1]
	Cho hàm số $y=f(x)=\heva{&\sqrt{x+4}&&\text{ khi } x>1\\&x^2+1&&\text{ khi } -1\leq x\leq 1\\&2x-1&&\text{ khi } x<-1}$. Giá trị $f(0)$ bằng
	\choice
	{$-2$}
	{$2$}
	{$-1$}
	{\True $1$}
	\loigiai{
		Ta có $f(0)=0^2+1=1$.
	}
\end{ex}

\begin{ex}%[0D2Y1-1]
	Cho hàm số $y=\heva{&2x+1 &\text{khi}\ x\le 2\\ &x^2-3 &\text{khi}\ x>2}$. Trong các điểm sau đây, điểm nào thuộc đồ thị hàm số?
	\choice
	{\True $(0;1)$}
	{$(0;-3)$}
	{$(3;7)$}
	{$(-3;6)$}
	\loigiai{
		Điểm $(0;1)$ thuộc đồ thị hàm số.
	}
\end{ex}

\begin{ex}
	\immini{Cho đồ thị hàm số $y=f(x)$ trên miền $[-3;5]$ như hình bên. Trong các điểm sau, điểm nào thuộc đồ thị hàm số đã cho?
	\haicot
	{$A(4;1)$}
	{$B(1;1)$ }
	{\True $C(3;3)$}
	{$D(0;2)$ }}{
	\begin{tikzpicture}[smooth,samples=300,scale=0.6,>=stealth,font=\footnotesize]
		\draw[line width=0.1pt,gray!80] (-4,-2) grid (5,4);
		\draw[->] (-4.2,0)--(5.7,0) node[below]{$x$};
		\draw[->] (0,-2.2)--(0,4.5) node[right]{$y$};
		\draw (0,0) node[above right]{$O$};
		\draw[line width=1pt,domain=-1:3] plot(\x,{(\x-1)^2-1});
		\draw[line width=1pt] (-3,-1)--(-1,3) (3,3)--(5,1);
		\draw[fill=magenta] (-3,-1) circle(2.5pt) (-1,3) circle(2.5pt) (3,3) circle(2.5pt) (5,1) circle(2.5pt) (1,-1) circle(2.5pt);
		\foreach \x in {-4,-3,-2,-1,1,2,3,4,5}\draw (\x,0.1)--(\x,-0.1) node [below] {\footnotesize $\x$};
		\foreach \y in {-2,-1,1,2,3}\draw (0.1,\y)--(-0.1,\y) node [left] {\footnotesize $\y$};
\end{tikzpicture}}
\loigiai{}
\end{ex}

\begin{ex}
	\immini{Cho đồ thị hàm số $y=f(x)$ trên miền $\mathscr{D}=[-3;5]$ như hình bên. Tập giá trị của hàm số này trên miền $\mathscr{D}$ là
		\haicot
		{$[-3;5]$}
		{$[-2;5]$ }
		{$[-3;3]$}
		{\True $[-2;2]$}}{
		\begin{tikzpicture}[smooth,samples=300,scale=0.6,>=stealth,font=\footnotesize]
			\draw[line width=0.1pt,gray!80] (-4,-2) grid (5,4);
			\draw[->] (-4.2,0)--(5.7,0) node[below]{$x$};
			\draw[->] (0,-2.2)--(0,4.5) node[right]{$y$};
			\draw (0,0) node[above right]{$O$};
			\draw[line width=1pt,domain=0:3] plot(\x,{-(\x-1)^2+2});
			\draw[line width=1pt] (-3,2)--(0,1) (3,-2)--(5,1);
			\draw[fill=magenta] (-3,2) circle(2.5pt) (0,1) circle(2.5pt) (3,-2) circle(2.5pt) (5,1) circle(2.5pt) ;
			\foreach \x in {-4,-3,-2,-1,1,2,3,4,5}\draw (\x,0.1)--(\x,-0.1) node [below] {\footnotesize $\x$};
			\foreach \y in {-2,-1,1,2,3}\draw (0.1,\y)--(-0.1,\y) node [left] {\footnotesize $\y$};
	\end{tikzpicture}}
	\loigiai{}
\end{ex}


\begin{ex}%[0D2B1]
	Cho hàm số $y=\dfrac{x+1}{x-1}$. Tìm tọa độ điểm thuộc đồ thị của hàm số có tung độ bằng $-2$.
	\choice        
	{$(0;-2)$}
	{\True $\left(\dfrac{1}{3};-2\right)$}
	{$(-2;-2)$}
	{$(-1;-2)$}
	\loigiai{
		Thay $y=-2$ vào phương trình hàm số $y=\dfrac{x+1}{x-1}$ ta được $x=\dfrac{1}{3}$.
	}
\end{ex}
\begin{ex}%[Phan Anh]%[0D2B1-1]
	Điểm nào sau đây thuộc đồ thị hàm số $y=\dfrac{1}{x-1}$?
	\choice
	{\True $M_1(2;1)$}
	{$M_2(1;1)$}
	{$M_3(2;0)$}
	{$M_4(0;-2)$}
	\loigiai{
		Xét điểm $M_1$, thay $x=2$ và $y=1$
		vào hàm số $y=\dfrac{1}{x-1}$ ta được $1=\dfrac{1}{2-1}$ ta thấy đúng nên nhận $M_1$.}
\end{ex}
\begin{ex}%[Phan Anh]%[0D2B1-1]
	Điểm nào sau đây \textbf{không} thuộc đồ thị hàm số $y=\dfrac{\sqrt{x^2-4x+4}}{x}$?
	\choice
	{$A\left(2;0\right)$}
	{$B\left(3;\dfrac{1}{3}\right)$}
	{\True $C\left(1;-1\right)$}
	{$D\left(-1;-3\right)$}
	\loigiai{Thay từng đáp án vào hàm số $y=\dfrac{\sqrt{x^2-4x+4}}{x}$.
		\begin{itemize}
			\item Với $x=2$ và $y=0$, ta được $0=\dfrac{\sqrt{2^2-4.2+4}}{2}$ (đúng).
			\item Với $x=3$ và $y=\dfrac{1}{3}$, ta được $\dfrac{1}{3}=\dfrac{\sqrt{3^2-4\cdot3+4}}{3}$ (đúng).
			\item Với thay $x=1$ và $y=-1$, ta được $-1=\dfrac{\sqrt{1^2-4\cdot1+4}}{1}\Leftrightarrow-1=1$ (sai).
		\end{itemize}}
\end{ex}
\begin{ex}%[Phan Anh]%[0D2B1-1]
	Cho hàm số $y=f(x)=|-5x|$. Khẳng định nào sau đây là \textbf{sai}?
	\choice
	{$f(-1)=5$}
	{$f(2)=10$}
	{$f(-2)=10$}
	{\True $f\left(\dfrac{1}{5}\right)=-1$}
	\loigiai{Ta có
		\begin{itemize}
			\item $f(-1)=|-5\cdot(-1)|=|5|=5$.
			\item $f(2)=|-5\cdot2|=|-10|=10$.
			\item $f(-2)=|-5\cdot(-2)|=|10|=10$.
			\item $f\left(\dfrac{1}{5}\right)=\left|-5\cdot\dfrac{1}{5}\right|=|-1|=1$
		\end{itemize}
		Cách khác: Vì hàm đã cho là hàm trị tuyệt đối nên không âm. Do đó $f\left(\dfrac{1}{5}\right)=-1$ là sai.}
\end{ex}
\begin{ex}%[Phan Anh]%[0D2B1-1]
	Cho hàm số $f(x)=\left\{\begin{array}{*{35}{l}}
			\dfrac{2}{x-1} & , x\in(-\infty;0) \\
			\sqrt{x+1}     & , x\in[0;2]       \\
			x^2-1          & , x\in(2;5]
		\end{array}\right.$. Tính giá trị của $f(4)$.
	\choice
	{$f(4)=\dfrac{2}{3}$}
	{\True $f(4)=15$}
	{$f(4)=\sqrt{5}$}
	{Không tính được}
	\loigiai{Do $4\in(2;5]$ nên $f(4)=4^2-1=15$.}
\end{ex}
\begin{ex}%[Phan Anh]%[0D2B1-1]
	Cho hàm số $f(x)=\left\{\begin{array}{*{35}{l}}
			\dfrac{2\sqrt{x+2}-3}{x-1} & , x\ge 2 \\
			x^2 +1                     & , x<2
		\end{array}\right.$. Tính $P=f(2)+f(-2)$.
	\choice
	{$P=\dfrac{8}{3}$}
	{$P=4$}
	{\True $P=6$}
	{$P=\dfrac{5}{3}$}
	\loigiai{\begin{itemize}
			\item Khi $x\ge 2$ thì $f(2)=\dfrac{2\sqrt{2+2}-3}{2-1}=1$.
			\item Khi $x<2$ thì $f(-2)=(-2)^2+1=5$.
		\end{itemize}
		Vậy $f(2)+f(-2)=6$.}
\end{ex}
\begin{ex}%[Phan Anh]%[0D2B1-2]
	Tìm tập xác định $\mathscr{D}$ của hàm số $y=\dfrac{3x-1}{2x-2}$.
	\choice
	{$\mathscr{D}=\mathbb{R}$}
	{$\mathscr{D}=(1;+\infty)$}
	{\True $\mathscr{D}=\mathbb{R}\setminus\{1\}$}
	{$\mathscr{D}=[1;+\infty)$}
	\loigiai{
		Hàm số xác định khi $2x-2\ne0\Leftrightarrow x\ne1$.\\
		Vậy tập xác định của hàm số là $\mathscr{D}=\mathbb{R}\setminus\{1\}$.}
\end{ex}
\begin{ex}%[Phan Anh]%[0D2B1-2]
	Tìm tập xác định $\mathscr{D}$ của hàm số $y=\dfrac{2x-1}{(2x+1)(x-3)}$.
	\choice
	{$\mathscr{D}=(3;+\infty)$}
	{\True $\mathscr{D}=\mathbb{R}\setminus\left\{-\dfrac{1}{2};3\right\}$}
	{$\mathscr{D}=\left(-\dfrac{1}{2};+\infty\right)$}
	{$\mathscr{D}=\mathbb{R}$}
	\loigiai{
		Hàm số xác định khi $\heva{
				& 2x+1\ne 0 \\
				& x-3\ne 0}\Leftrightarrow \heva{
				& x\ne-\dfrac{1}{2} \\
				& x\ne 3.}$\\
		Vậy tập xác định của hàm số là $ \mathscr{D}=\mathbb{R}\setminus\left\{-\dfrac{1}{2};3\right\}$}
\end{ex}
\begin{ex}%[Phan Anh]%[0D2B1-2]
	Tìm tập xác định $\mathscr{D}$ của hàm số $y=\dfrac{x^2+1}{x^2+3x-4}$.
	\choice
	{$\mathscr{D}=\{1;-4\}$}
	{\True $\mathscr{D}=\mathbb{R}\setminus\{1;-4\}$}
	{$\mathscr{D}=\mathbb{R}\setminus\{1;4\}$}
	{$\mathscr{D}=\mathbb{R}$}
	\loigiai{
		Hàm số xác định khi $x^2+3x-4\ne 0\Leftrightarrow \heva{
				& x\ne 1 \\
				& x\ne-4}.$\\
		Vậy tập xác định của hàm số là $\mathscr{D}=\mathbb{R}\setminus\{1;-4\}$.}
\end{ex}
\begin{ex}%[Phan Anh]%[0D2B1-2]
	Tìm tập xác định $\mathscr{D}$ của hàm số $y=\dfrac{x+1}{(x+1)(x^2+3x+4)}$.
	\choice
	{$\mathscr{D}=\mathbb{R}\setminus\left\{1\right\}$}
	{$\mathscr{D}=\left\{-1\right\}$}
	{\True $\mathscr{D}=\mathbb{R}\setminus\left\{-1\right\}$}
	{$\mathscr{D}=\mathbb{R}$}
	\loigiai{
		Hàm số xác định khi $\heva{
				& x+1\ne 0 \\
				& x^2+3x+4\ne 0}\Leftrightarrow x\ne-1$.\\
		Vậy tập xác định của hàm số là $\mathscr{D}=\mathbb{R}\setminus\left\{-1\right\}$.}
\end{ex}
\begin{ex}%[Phan Anh]%[0D2B1-2]
	Tìm tập xác định $\mathscr{D}$ của hàm số $y=\dfrac{2x+1}{x^3-3x+2}$.
	\choice
	{$\mathscr{D}=\mathbb{R}\setminus\left\{1;2\right\}$}
	{\True $\mathscr{D}=\mathbb{R}\setminus\left\{-2;1\right\}$}
	{$\mathscr{D}=\mathbb{R}\setminus\left\{-2\right\}$}
	{$\mathscr{D}=\mathbb{R}$}
	\loigiai{
		Hàm số xác định khi
		\begin{align*}
			                & x^3-3x+2\ne 0
			\Leftrightarrow (x-1)(x^2+x-2)\ne 0                         \\
			\Leftrightarrow & \,\heva{                       & x-1\ne 0 \\
			                & x^2+x-2\ne 0}
			\Leftrightarrow \heva{
			                & x\ne 1                                    \\ & \heva{ & x\ne 1 \\
			                & x\ne-2}}\Leftrightarrow \heva{
			                & x\ne                                      \\
			                & x\ne-2.}
		\end{align*}
		Vậy tập xác định của hàm số là $\mathscr{D}=\mathbb{R}\setminus\left\{-2;1\right\}$.
	}
\end{ex}
\begin{ex}%[Phan Anh]%[0D2B1-2]
	Tìm tập xác định $\mathscr{D}$ của hàm số $y=\sqrt{x+2}-\sqrt{x+3}$.
	\choice
	{$\mathscr{D}=[-3;+\infty)$}
	{\True $\mathscr{D}=\left[-2;+\infty \right)$}
	{$\mathscr{D}=\mathbb{R}$}
	{$\mathscr{D}=\left[2;+\infty \right)$}
	\loigiai{
	Hàm số xác định khi $\heva{
			& x+2\ge 0 \\
			& x+3\ge 0 \\}\Leftrightarrow \heva{
			& x\ge-2 \\
			& x\ge-3 \\}\Leftrightarrow x\ge-2$.\\
	Vậy tập xác định của hàm số là $\mathscr{D}=\left[-2;+\infty \right)$.}
\end{ex}
\begin{ex}%[Phan Anh]%[0D2B1-2]
	Tìm tập xác định $\mathscr{D}$ của hàm số $y=\sqrt{6-3x}-\sqrt{x-1}$.
	\choice
	{$\mathscr{D}=\left(1;2\right)$}
	{\True $\mathscr{D}=\left[1;2\right]$}
	{$\mathscr{D}=\left[1;3\right]$}
	{$\mathscr{D}=\left[-1;2\right]$}
	\loigiai{
		Hàm số xác định khi $\heva{
				& 6-3x\ge 0 \\
				& x-1\ge 0}\Leftrightarrow \heva{
				& x\le 2 \\
				& x\ge 1}\Leftrightarrow 1\le x\le 2$.\\
		Vậy tập xác định của hàm số là $\mathscr{D}=\left[1;2\right]$.}
\end{ex}
\begin{ex}%[Phan Anh]%[0D2B1-2]
	Tìm tập xác định $\mathscr{D}$ của hàm số $y=\dfrac{\sqrt{3x-2}+6x}{\sqrt{4-3x}}$.
	\choice
	{\True $\mathscr{D}=\left[\dfrac{2}{3};\dfrac{4}{3}\right)$}
	{$\mathscr{D}=\left[\dfrac{3}{2};\dfrac{4}{3}\right)$}
	{$\mathscr{D}=\left[\dfrac{2}{3};\dfrac{3}{4}\right)$}
	{$\mathscr{D}=\left(-\infty;\dfrac{4}{3}\right)$}
	\loigiai{
	Hàm số xác định khi $\heva{
			& 3x-2\ge 0 \\
			& 4-3x>0}\Leftrightarrow \heva{
			& x\ge \dfrac{2}{3} \\
			& x<\dfrac{4}{3}}\Leftrightarrow \dfrac{2}{3}\le x<\dfrac{4}{3}$.\\
	Vậy tập xác định của hàm số là $\mathscr{D}=\left[\dfrac{2}{3};\dfrac{4}{3}\right)$.}
\end{ex}
\begin{ex}%[Phan Anh]%[0D2B1-2]
	Tìm tập xác định $\mathscr{D}$ của hàm số $y=\dfrac{x+4}{\sqrt{x^2-16}}$.
	\choice
	{$\mathscr{D}=\left(-\infty;-2\right)\cup \left(2;+\infty \right)$}
	{$\mathscr{D}=\mathbb{R}$}
	{\True $\mathscr{D}=\left(-\infty;-4\right)\cup \left(4;+\infty \right)$}
	{$\mathscr{D}=\left(-4;4\right)$}
	\loigiai{Hàm số xác định khi $x^2-16>0\Leftrightarrow x^2>16\Leftrightarrow \hoac{
				& x>4 \\
				& x<-4}$.\\
		Vậy tập xác định của hàm số là $\mathscr{D}=\left(-\infty;-4\right)\cup \left(4;+\infty \right)$.}
\end{ex}
\begin{ex}%[Phan Anh]%[0D2B1-2]
	Tìm tập xác định $\mathscr{D}$ của hàm số $y=\sqrt{x^2-2x+1}+\sqrt{x-3}$.
	\choice
	{$\mathscr{D}=(-\infty;3]$}
	{$\mathscr{D}=[1;3]$}
	{\True $\mathscr{D}=[3;+\infty)$}
	{$\mathscr{D}=(3;+\infty)$}
	\loigiai{
	Hàm số xác định khi $\heva{
			& x^2-2x+1\ge 0 \\
			& x-3\ge 0}\Leftrightarrow \heva{
			& {\left(x-1\right)}^2\ge 0 \\
			& x-3\ge 0}\Leftrightarrow \heva{
			& x\in \mathbb{R} \\
			& x\ge 3}\Leftrightarrow x\ge 3$.\\
	Vậy tập xác định của hàm số là $\mathscr{D}=\left[3;+\infty \right)$.}
\end{ex}
\begin{ex}%[Phan Anh]%[0D2B1-2]
	Tìm tập xác định $\mathscr{D}$ của hàm số $y=\dfrac{\sqrt{2-x}+\sqrt{x+2}}{x}$.
	\choice
	{$\mathscr{D}=[-2;2]$}
	{$\mathscr{D}=(-2;2)\setminus\left\{0\right\}$}
	{\True $\mathscr{D}=[-2;2]\setminus\left\{0\right\}$}
	{$\mathscr{D}=\mathbb{R}$}
	\loigiai{
		Hàm số xác định khi $\heva{
				& 2-x\ge 0 \\
				& x+2\ge 0 \\
				& x\ne 0}\Leftrightarrow \heva{
				& x\le 2 \\
				& x\ge-2 \\
				& x\ne 0.}$\\
		Vậy tập xác định của hàm số là $\mathscr{D}=\left[-2;2\right]\setminus\left\{0\right\}$.}
\end{ex}
\begin{ex}%[Phan Anh]%[0D2B1-2]
	Tìm tập xác định $\mathscr{D}$ của hàm số $y=\dfrac{\sqrt{x+1}}{x^2-x-6}$.
	\choice
	{$\mathscr{D}=\left\{3\right\}$}
	{\True $\mathscr{D}=\left[-1;+\infty \right)\setminus\left\{3\right\}$}
	{$\mathscr{D}=\mathbb{R}$}
	{$\mathscr{D}=\left[-1;+\infty \right)$}
	\loigiai{
	Hàm số xác định khi $\heva{
			& x+1\ge 0 \\
			& x^2-x-6\ne 0}\Leftrightarrow \heva{
			& x\ge-1 \\
			& x\ne 3 \\
			& x\ne-2}\Leftrightarrow \heva{
			& x\ge-1 \\
			& x\ne 3.}$\\
	Vậy tập xác định của hàm số là $\mathscr{D}=[-1;+\infty)\setminus\left\{3\right\}$.}
\end{ex}
\begin{ex}%[Phan Anh]%[0D2B1-2]
	Tìm tập xác định $\mathscr{D}$ của hàm số $y=\sqrt{6-x}+\dfrac{2x+1}{1+\sqrt{x-1}}$.
	\choice
	{$\mathscr{D}=(1;+\infty)$}
	{\True $\mathscr{D}=[1;6]$}
	{$\mathscr{D}=\mathbb{R}$}
	{$\mathscr{D}=(1;6)$}
	\loigiai{
	Hàm số xác định khi $\heva{
			& 6-x\ge 0 \\
			& x-1\ge 0 \\
			& 1+\sqrt{x-1}\ne 0\left(\text{luôn đúng} \right)}\Leftrightarrow \heva{
			& x\le 6 \\
			& x\ge 1}\Leftrightarrow 1\le x\le 6$.\\
	Vậy tập xác định của hàm số là $\mathscr{D}=[1;6]$.}
\end{ex}
\begin{ex}%[Phan Anh]%[0D2B1-2]
	Tìm tập xác định $\mathscr{D}$ của hàm số $y=\dfrac{x+1}{(x-3)\sqrt{2x-1}}$.
	\choice
	{$\mathscr{D}=\mathbb{R}$}
	{$\mathscr{D}=\left(-\dfrac{1}{2};+\infty \right)\setminus\left\{3\right\}$}
	{$\mathscr{D}=\left[\dfrac{1}{2};+\infty \right)\setminus\left\{3\right\}$}
	{\True $\mathscr{D}=\left(\dfrac{1}{2};+\infty \right)\setminus\left\{3\right\}$}
	\loigiai{
		Hàm số xác định khi $\heva{
				& x-3\ne 0 \\
				& 2x-1>0}\Leftrightarrow \heva{
				& x\ne 3 \\
				& x>\dfrac{1}{2}.}$\\
		Vậy tập xác định của hàm số là $\mathscr{D}=\left(\dfrac{1}{2};+\infty \right)\setminus\left\{3\right\}$.}
\end{ex}
\begin{ex}%[Phan Anh]%[0D2B1-2]
	Tìm tập xác định $\mathscr{D}$ của hàm số $y=\dfrac{\sqrt{x+2}}{x\sqrt{x^2-4x+4}}$.
	\choice
	{\True $\mathscr{D}=[-2;+\infty)\setminus\left\{0;2\right\}$}
	{$\mathscr{D}=\mathbb{R}$}
	{$\mathscr{D}=[-2;+\infty)$}
	{$\mathscr{D}=(-2;+\infty)\setminus\left\{0;2\right\}$}
	\loigiai{
	Hàm số xác định khi $\heva{
			& x+2\ge 0 \\
			& x\ne 0 \\
			& x^2-4x+4>0}\Leftrightarrow \heva{
			& x+2\ge 0 \\
			& x\ne 0 \\
			& (x-2)^2>0}\Leftrightarrow \heva{
			& x\ge-2 \\
			& x\ne 0 \\
			& x\ne 2.}$\\
	Vậy tập xác định của hàm số là $\mathscr{D}=\left[-2;+\infty \right)\setminus\left\{0;2\right\}$.}
\end{ex}
%Câu 21
\begin{ex}%[Phan Anh]%[0D2B1-2]
	Tìm tập xác định $\mathscr{D}$ của hàm số $y=\dfrac{x}{x-\sqrt{x}-6}$.
	\choice
	{$\mathscr{D}=\left[0;+\infty \right)\setminus\left\{3\right\}$}
	{\True $\mathscr{D}=\left[0;+\infty \right)\setminus\left\{9\right\}$}
	{$\mathscr{D}=\left[0;+\infty \right)\setminus\left\{\sqrt{3}\right\}$}
	{$\mathscr{D}=\mathbb{R}\setminus\left\{9\right\}$}
	\loigiai{
	Hàm số xác định khi $\heva{
			& x\ge 0 \\
			& x-\sqrt{x}-6\ne 0}\Leftrightarrow \heva{
			& x\ge 0 \\
			& \sqrt{x}\ne 3}\Leftrightarrow \heva{
			& x\ge 0 \\
			& x\ne 9.}$\\
	Vậy tập xác định của hàm số là $\mathscr{D}=\left[0;+\infty \right)\setminus\left\{9\right\}$.}
\end{ex}
\begin{ex}%[Phan Anh]%[0D2B1-2]
	Tìm tập xác định $\mathscr{D}$ của hàm số $y=\dfrac{\sqrt[3]{x-1}}{x^2+x+1}$.
	\choice
	{$\mathscr{D}=\left(1;+\infty \right)$}
	{$\mathscr{D}=\left\{1\right\}$}
	{\True $\mathscr{D}=\mathbb{R}$}
	{$\mathscr{D}=\left(-1;+\infty \right)$}
	\loigiai{
		Hàm số xác định khi $x^2+x+1\ne 0$ luôn đúng với mọi $x\in \mathbb{R}$.\\
		Vậy tập xác định của hàm số là $\mathscr{D}=\mathbb{R}$.}
\end{ex}
\begin{ex}%[Phan Anh]%[0D2B1-2]
	Tìm tập xác định $\mathscr{D}$ của hàm số $y=\dfrac{\sqrt{x-1}+\sqrt{4-x}}{\left(x-2\right)\left(x-3\right)}$.
	\choice
	{$\mathscr{D}=\left[1;4\right]$}
	{$\mathscr{D}=\left(1;4\right)\setminus\left\{2;3\right\}$}
	{\True $\mathscr{D}=\left[1;4\right]\setminus\left\{2;3\right\}$}
	{$\mathscr{D}=\left(-\infty;1\right]\cup \left[4;+\infty \right)$}
	\loigiai{
		Hàm số xác định khi $\heva{
				& x-1\ge 0 \\
				& 4-x\ge 0 \\
				& x-2\ne 0 \\
				& x-3\ne 0}\Leftrightarrow \heva{
				& x\ge 1 \\
				& x\le 4 \\
				& x\ne 2 \\
				& x\ne 3}\Leftrightarrow \heva{
				& 1\le x\le 4 \\
				& x\ne 2 \\
				& x\ne 3.}$\\
		Vậy tập xác định của hàm số là $\mathscr{D}=\left[1;4\right]\setminus\left\{2;3\right\}$.}
\end{ex}
\begin{ex}%[Phan Anh]%[0D2B1-2]
	Tìm tập xác định $\mathscr{D}$ của hàm số $y=\sqrt{\sqrt{x^2+2x+2}-(x+1)}$.
	\choice
	{$\mathscr{D}=\left(-\infty;-1\right)$}
	{$\mathscr{D}=\left[-1;+\infty \right)$}
	{$\mathscr{D}=\mathbb{R}\setminus\left\{-1\right\}$}
	{\True $\mathscr{D}=\mathbb{R}$}
	\loigiai{
		Hàm số xác định khi $\begin{aligned}[t]
				                & \sqrt{x^2+2x+2}-(x+1)\ge 0\Leftrightarrow \sqrt{(x+1)^2+1}\ge x+1 \\
				\Leftrightarrow & \, \hoac{
				                & \heva{
				                & x+1<0                                                             \\
				                & (x+1)^2+1\ge 0}                                                   \\
				                & \heva{
				                & x+1\ge 0                                                          \\
				                & (x+1)^2+1\ge(x+1)^2}}\Leftrightarrow \hoac{
				                & x+1<0                                                             \\
				                & x+1\ge 0}\Leftrightarrow x\in \mathbb{R}.
			\end{aligned}$\\
		Vậy tập xác định của hàm số là $\mathscr{D}=\mathbb{R}$.}
\end{ex}
\begin{ex}%[Phan Anh]%[0D2B1-2]
	Tìm tập xác định $\mathscr{D}$ của hàm số $y=\dfrac{2018}{\sqrt[3]{x^2-3x+2}-\sqrt[3]{x^2-7}}$.
	\choice
	{\True $\mathscr{D}=\mathbb{R}\setminus\left\{3\right\}$}
	{$\mathscr{D}=\mathbb{R}$}
	{$\mathscr{D}=\left(-\infty;1\right)\cup \left(2;+\infty \right)$}
	{$\mathscr{D}=\mathbb{R}\setminus\left\{0\right\}$}
	\loigiai{
		Hàm số xác định khi $\begin{aligned}[t]
				                & \sqrt[3]{x^2-3x+2}-\sqrt[3]{x^2-7}\ne 0\Leftrightarrow \sqrt[3]{x^2-3x+2}\ne \sqrt[3]{x^2-7} \\
				\Leftrightarrow & \,x^2-3x+2\ne x^2-7\Leftrightarrow 9\ne 3x\Leftrightarrow x\ne 3.
			\end{aligned}$\\
		Vậy tập xác định của hàm số là $\mathscr{D}=\mathbb{R}\setminus\left\{3\right\}$.}
\end{ex}
\begin{ex}%[Phan Anh]%[0D2K1-2]
	Tìm tập xác định $\mathscr{D}$ của hàm số $y=\dfrac{|x|}{|x-2|+\left|x^2+2x\right|}$.
	\choice
	{\True $\mathscr{D}=\mathbb{R}$}
	{$\mathscr{D}=\mathbb{R}\setminus\left\{-2;0\right\}$}
	{$\mathscr{D}=\mathbb{R}\setminus\left\{-2;0;2\right\}$}
	{$\mathscr{D}=\left(2;+\infty \right)$}
	\loigiai{
		Hàm số xác định khi $|x-2|+\left|x^2+2x\right|\ne0$.\\
		Xét phương trình $|x-2|+\left|x^2+2x\right|=0\Leftrightarrow \heva{
				& |x-2|=0 \\
				& \left|x^2+2x\right|=0}\Leftrightarrow \heva{
				& x=2 \\
				& x=0\vee x=-2.}$\\
		Vậy không có giá trị $x$ làm cho $|x-2|+\left| x^2+2x\right|=0$, do đó $|x-2|+\left| x^2+2x\right|\ne 0$ đúng với mọi $x\in \mathbb{R}$. Vậy tập xác định của hàm số là $\mathscr{D}=\mathbb{R}$.}
\end{ex}
\begin{ex}%[Phan Anh]%[0D2K1-2]
	Tìm tập xác định $\mathscr{D}$ của hàm số $y=\dfrac{2x-1}{\sqrt{x|x-4|}}$.
	\choice
	{$\mathscr{D}=\mathbb{R}\setminus\left\{0;4\right\}$}
	{$\mathscr{D}=\left(0;+\infty \right)$}
	{$\mathscr{D}=\left[0;+\infty \right)\setminus\left\{4\right\}$}
	{\True $\mathscr{D}=\left(0;+\infty \right)\setminus\left\{4\right\}$}
	\loigiai{
		Hàm số xác định khi $x|x-4|>0\Leftrightarrow \heva{
				& \left| x-4\right|\ne 0 \\
				& x>0}\Leftrightarrow \heva{
				& x\ne 4 \\
				& x>0.}$\\
		Vậy tập xác định của hàm số là $\mathscr{D}=\left(0;+\infty \right)\setminus\left\{4\right\}$.}
\end{ex}
\begin{ex}%[Phan Anh]%[0D2K1-2]
	Tìm tập xác định $\mathscr{D}$ của hàm số $y=\dfrac{\sqrt{5-3\left| x\right|}}{x^2+4x+3}$.
	\choice
	{\True $\mathscr{D}=\left[-\dfrac{5}{3};\dfrac{5}{3}\right]\setminus\left\{-1\right\}$}
	{$\mathscr{D}=\mathbb{R}$}
	{$\mathscr{D}=\left(-\dfrac{5}{3};\dfrac{5}{3}\right)\setminus\left\{-1\right\}$}
	{$\mathscr{D}=\left[-\dfrac{5}{3};\dfrac{5}{3}\right]$}
	\loigiai{
		Hàm số xác định khi $\heva{
				& 5-3\left| x\right|\ge 0 \\
				& x^2+4x+3\ne 0}\Leftrightarrow \heva{
				& \left| x\right|\le \dfrac{5}{3} \\
				& x\ne-1 \\
				& x\ne-3}\Leftrightarrow \heva{
				&-\dfrac{5}{3}\le x\le \dfrac{5}{3} \\
				& x\ne-1 \\
				& x\ne-3}\Leftrightarrow \heva{
				&-\dfrac{5}{3}\le x\le \dfrac{5}{3} \\
				& x\ne-1.}$\\
		Vậy tập xác định của hàm số là $\mathscr{D}=\left[-\dfrac{5}{3};\dfrac{5}{3}\right]\setminus\left\{-1\right\}$.}
\end{ex}
\begin{ex}%[Phan Anh]%[0D2K1-2]
	Tìm tập xác định $\mathscr{D}$ của hàm số $f(x)=\left\{\begin{array}{*{35}{l}}
			\dfrac{1}{2-x} & ;x\ge 1 \\
			\sqrt{2-x}     & ;x<1.
		\end{array}\right.$
	\choice
	{$\mathscr{D}=\mathbb{R}$}
	{$\mathscr{D}=\left(2;+\infty \right)$}
	{$\mathscr{D}=\left(-\infty;2\right)$}
	{\True $\mathscr{D}=\mathbb{R}\setminus\left\{2\right\}$}
	\loigiai{
		Hàm số xác định khi $\hoac{
				& \heva{
					& x\ge 1 \\
					& 2-x\ne 0} \\
				& \heva{
					& x<1 \\
					& 2-x\ge 0}}\Leftrightarrow \hoac{
				& \heva{
					& x\ge 1 \\
					& x\ne 2} \\
				& \heva{
					& x<1 \\
					& x\le 2}}\Leftrightarrow \hoac{
				& \heva{
					& x\ge 1 \\
					& x\ne 2} \\
				& x<1.}$\\
		Vậy xác định của hàm số là $\mathscr{D}=\mathbb{R}\setminus\left\{2\right\}$.}
\end{ex}
\begin{ex}%[Phan Anh]%[0D2K1-2]
	Tìm tập xác định $\mathscr{D}$ của hàm số $f(x)=\left\{\begin{array}{*{35}{l}}
			\dfrac{1}{x} & ;x\ge 1 \\
			\sqrt{x+1}   & ;x<1.
		\end{array}\right.$
	\choice
	{$\mathscr{D}=\left\{-1\right\}$}
	{$\mathscr{D}=\mathbb{R}$}
	{\True $\mathscr{D}=\left[-1;+\infty \right)$}
	{$\mathscr{D}=\left[-1;1\right)$}
	\loigiai{
	Hàm số xác định khi $\hoac{
			& \heva{
				& x\ge 1 \\
				& x\ne 0} \\
			& \heva{
				& x<1 \\
				& x+1\ge 0}}\Leftrightarrow \hoac{
			& x\ge 1 \\
			& \heva{
				& x<1 \\
				& x\ge-1.}}$\\
	Vậy xác định của hàm số là $\mathscr{D}=\left[-1;+\infty \right)$.}
\end{ex}
\begin{ex}%[Phan Anh]%[0D2K1-2]
	Tìm tất cả các giá trị thực của tham số $m$ để hàm số $y=\sqrt{x-m+1}+\dfrac{2x}{\sqrt{-x+2m}}$ xác định trên khoảng $(-1;3)$.
	\choice
	{\True Không có giá trị $m$ thỏa mãn}
	{$m\ge 2$}
	{$m\ge 3$}
	{$m\ge 1$}
	\loigiai{
	Hàm số xác định khi $\heva{
			& x-m+1\ge 0 \\
			&-x+2m>0}\Leftrightarrow \heva{
			& x\ge m-1 \\
			& x<2m.}$\\
	Tập xác định của hàm số là $\mathscr{D}=\left[m-1;2m\right)$ với điều kiện $m-1<2m\Leftrightarrow m>-1$.\\
	Hàm số đã cho xác định trên $\left(-1;3\right)$ khi và chỉ khi $\left(-1;3\right)\subset \left[m-1;2m\right)$\\
	$\Leftrightarrow m-1\le-1<3\le 2m\Leftrightarrow \heva{
			& m\le 0 \\
			& m\ge \dfrac{3}{2}.}$\\
	Vậy không có giá trị $m$ thỏa bài toán.}
\end{ex}
\begin{ex}%[Phan Anh]%[0D2K1-2]
	Tìm tất cả các giá trị thực của tham số $m$ để hàm số $y=\dfrac{x+2m+2}{x-m}$ xác định trên $\left(-1;0\right)$.
	\choice
	{$\hoac{
				& m>0 \\
				& m<-1}$}
	{$m\le-1$}
	{\True $\hoac{
				& m\ge 0 \\
				& m\le-1}$}
	{$m\ge 0$}
	\loigiai{
		Hàm số xác định khi $x-m\ne 0\Leftrightarrow x\ne m$.
		Tập xác định của hàm số là $\mathscr{D}=\mathbb{R}\setminus\left\{m\right\}$.\\
		Hàm số xác định trên $\left(-1;0\right)$ khi và chỉ khi $m\notin \left(-1;0\right)\Leftrightarrow \hoac{
				& m\ge 0 \\
				& m\le-1.}$}
\end{ex}
\begin{ex}%[Phan Anh]%[0D2K1-2]
	Tìm tất cả các giá trị thực của tham số $m$ để hàm số $y=\dfrac{mx}{\sqrt{x-m+2}-1}$ xác định trên $(0;1)$.
	\choice
	{$m\in \left(-\infty;\dfrac{3}{2}\right]\cup \left\{2\right\}$}
	{$m\in \left(-\infty;-1\right]\cup \left\{2\right\}$}
	{$m\in \left(-\infty;1\right]\cup \left\{3\right\}$}
			{\True $m\in \left(-\infty;1\right]\cup \left\{2\right\}$}
				\loigiai{
				Hàm số xác định khi $\heva{
					& x-m+2\ge 0 \\
					& \sqrt{x-m+2}-1\ne 0}\Leftrightarrow \heva{
					& x\ge m-2 \\
					& x\ne m-1.}$
				\\ Tập xác định của hàm số là $\mathscr{D}=\left[m-2;+\infty \right)\setminus\left\{m-1\right\}$.\\
			Hàm số xác định trên $\left(0;1\right)$ khi và chỉ khi $\left(0;1\right)\subset \left[m-2;+\infty \right)\setminus\left\{m-1\right\}$\\
		$\Leftrightarrow \hoac{
				& m-2\le 0<1\le m-1 \\
				& m-1\le 0}\Leftrightarrow \hoac{
				& \heva{
					& m\le 2 \\
					& m\ge 2} \\
				& m\le 1}\Leftrightarrow \hoac{
				& m=2 \\
				& m\le 1.}$}
\end{ex}
\begin{ex}%[Phan Anh]%[0D2K1-2]
	Tìm tất cả các giá trị thực của tham số $m$ để hàm số $y=\sqrt{x-m}+\sqrt{2x-m-1}$ xác định trên $(0;+\infty)$.
	\choice
	{$m\le 0$}
	{$m\ge 1$}
	{$m\le 1$}
	{\True $m\le-1$}
	\loigiai{
		Hàm số xác định khi $\heva{
				& x-m\ge 0 \\
				& 2x-m-1\ge 0}\Leftrightarrow \heva{
				& x\ge m \\
				& x\ge \dfrac{m+1}{2}}\,(*)$.
		\begin{itemize}
			\item Nếu $m\ge \dfrac{m+1}{2}\Leftrightarrow m\ge 1$ thì $\left(*\right)\Leftrightarrow x\ge m$.\\
			      Tập xác định của hàm số là $\mathscr{D}=\left[m;+\infty \right)$.
			      Khi đó, hàm số xác định trên $\left(0;+\infty \right)$ khi và chỉ khi $\left(0;+\infty \right)\subset \left[m;+\infty \right)\Leftrightarrow m\le 0$
			      $\Rightarrow $ Không thỏa mãn điều kiện $m\ge 1$.
			\item Nếu $m\le \dfrac{m+1}{2}\Leftrightarrow m\le 1$ thì $\left(*\right)\Leftrightarrow x\ge \dfrac{m+1}{2}$.\\
			      Tập xác định của hàm số là $\mathscr{D}=\left[\dfrac{m+1}{2};+\infty \right)$.
			      Khi đó, hàm số xác định trên $\left(0;+\infty \right)$
			      khi và chỉ khi $\left(0;+\infty \right)\subset \left[\dfrac{m+1}{2};+\infty \right)\Leftrightarrow \dfrac{m+1}{2}\le 0\Leftrightarrow m\le-1$.\\
			      $\Rightarrow $ Thỏa mãn điều kiện $m\le 1$.
		\end{itemize}
		Vậy $m\le-1$ thỏa yêu cầu bài toán.}
\end{ex}
\begin{ex}%[Phan Anh]%[0D2K1-2]
	Tìm tất cả các giá trị thực của tham số $m$ để hàm số $y=\dfrac{2x+1}{\sqrt{x^2-6x+m-2}}$ xác định trên $\mathbb{R}$.
	\choice
	{$m\ge 11$}
	{\True $m>11$}
	{$m<11$}
	{$m\le 11$}
	\loigiai{
		Hàm số xác định khi $x^2-6x+m-2>0\Leftrightarrow {\left(x-3\right)}^2+m-11>0$.\\
		Hàm số xác định với $\forall x\in \mathbb{R}\Leftrightarrow (x-3)^2+m-11>0$ đúng với mọi $x\in \mathbb{R}$
		$\Leftrightarrow m-11>0\Leftrightarrow m>11$.}
\end{ex}
\begin{ex}%[Phan Anh]%[0D2B1-3]
	Cho hàm số $f(x)=4-3x$. Khẳng định nào sau đây đúng?
	\choice
	{Hàm số đồng biến trên $\left(-\infty;\dfrac{4}{3}\right)$}
	{\True Hàm số nghịch biến trên $\left(\dfrac{4}{3};+\infty \right)$}
	{Hàm số đồng biến trên $\mathbb{R}$}
	{Hàm số đồng biến trên $\left(\dfrac{3}{4};+\infty \right)$}
	\loigiai{
		TXĐ: $\mathscr{D}=\mathbb{R}$. \\Với mọi $x_1,x_2\in \mathbb{R}$ và $x_1<x_2$, ta có
		$f\left(x_1\right)-f\left(x_2\right)=\left(4-3x_1\right)-\left(4-3x_2\right)=-3\left(x_1-x_2\right)>0.$\\
		Suy ra $f\left(x_1\right)>f\left(x_2\right)$. Do đó, hàm số nghịch biến trên $\mathbb{R}$.\\
		Mà $\left(\dfrac{4}{3};+\infty \right)\subset \mathbb{R}$ nên hàm số cũng nghịch biến trên $\left(\dfrac{4}{3};+\infty \right)$.}
\end{ex}
% \begin{ex}%[Phan Anh]%[0D2B1-3]
% 	Xét tính đồng biến, nghịch biến của hàm số $f(x)=x^2-4x+5$ trên khoảng $\left(-\infty;2\right)$ và trên khoảng $\left(2;+\infty \right)$. Khẳng định nào sau đây đúng?
% 	\choice
% 	{\True Hàm số nghịch biến trên $\left(-\infty;2\right)$, đồng biến trên $\left(2;+\infty \right)$}
% 	{Hàm số đồng biến trên $\left(-\infty;2\right)$, nghịch biến trên $\left(2;+\infty \right)$}
% 	{Hàm số nghịch biến trên các khoảng $\left(-\infty;2\right)$ và $\left(2;+\infty \right)$}
% 	{Hàm số đồng biến trên các khoảng $\left(-\infty;2\right)$ và $\left(2;+\infty \right)$}
% 	\loigiai{
% 		Ta có $f\left(x_1\right)-f\left(x_2\right)=\left(x_1^2-4x_1+5\right)-\left(x_2^2-4x_2+5\right)$
% 		$=\left(x_1^2-x_2^2\right)-4\left(x_1-x_2\right)=\left(x_1-x_2\right)\left(x_1+x_2-4\right)$.
% 		Với mọi $x_1, x_2\in \left(-\infty;2\right)$ và $x_1<x_2$. Ta có $\heva{
% 			& x_1<2 \\ 
% 			& x_2<2 \\}\Rightarrow x_1+x_2<4$.\\
% 		Suy ra $\dfrac{f\left(x_1\right)-f\left(x_2\right)}{x_1-x_2}=\dfrac{\left(x_1-x_2\right)\left(x_1+x_2-4\right)}{x_1-x_2}=x_1+x_2-4<0$.\\
% 		Vậy hàm số nghịch biến trên $\left(-\infty;2\right)$.\\
% 		Với mọi $x_1, x_2\in \left(2;+\infty \right)$ và $x_1<x_2$. Ta có $\heva{
% 			& x_1>2 \\ 
% 			& x_2>2 \\}\Rightarrow x_1+x_2>4$.\\
% 		Suy ra $\dfrac{f\left(x_1\right)-f\left(x_2\right)}{x_1-x_2}=\dfrac{\left(x_1-x_2\right)\left(x_1+x_2-4\right)}{x_1-x_2}=x_1+x_2-4>0$.\\
% 		Vậy hàm số đồng biến trên $\left(2;+\infty \right)$.}
% \end{ex}
\begin{ex}%[Phan Anh]%[0D2B1-3]
	Xét sự biến thiên của hàm số $f(x)=\dfrac{3}{x}$ trên khoảng $(0;+\infty)$. Khẳng định nào sau đây đúng?
	\choice
	{Hàm số đồng biến trên khoảng $\left(0;+\infty \right)$}
	{\True Hàm số nghịch biến trên khoảng $\left(0;+\infty \right)$}
	{Hàm số vừa đồng biến, vừa nghịch biến trên khoảng $\left(0;+\infty \right)$}
	{Hàm số không đồng biến, cũng không nghịch biến trên khoảng $\left(0;+\infty \right)$}
	\loigiai{
		Ta có $f\left(x_1\right)-f\left(x_2\right)=\dfrac{3}{x_1}-\dfrac{3}{x_2}=\dfrac{3\left(x_2-x_1\right)}{x_1x_2}=-\dfrac{3\left(x_1-x_2\right)}{x_1x_2}.$\\
		Với mọi $x_1, x_2\in \left(0;+\infty \right)$ và $x_1<x_2$. Ta có $\heva{
				& x_1>0 \\
				& x_2>0 \\}\Rightarrow x_1\cdot x_2>0$.\\
		Suy ra $\dfrac{f\left(x_1\right)-f\left(x_2\right)}{x_1-x_2}=-\dfrac{3}{x_1x_2}<0\Rightarrow f(x)$ nghịch biến trên $\left(0;+\infty \right)$.}
\end{ex}
\begin{ex}%[Phan Anh]%[0D2B1-3]
	Xét sự biến thiên của hàm số $f(x)=x+\dfrac{1}{x}$ trên khoảng $\left(1;+\infty \right)$. Khẳng định nào sau đây đúng?
	\choice
	{\True Hàm số đồng biến trên khoảng $\left(1;+\infty \right)$}
	{Hàm số nghịch biến trên khoảng $\left(1;+\infty \right)$}
	{Hàm số vừa đồng biến, vừa nghịch biến trên khoảng $\left(1;+\infty \right)$}
	{Hàm số không đồng biến, cũng không nghịch biến trên khoảng $\left(1;+\infty \right)$}
	\loigiai{
		Ta có
		$f\left(x_1\right)-f\left(x_2\right)=\left(x_1+\dfrac{1}{x_1}\right)-\left(x_2+\dfrac{1}{x_2}\right)=\left(x_1-x_2\right)+\left(\dfrac{1}{x_1}-\dfrac{1}{x_2}\right)=\left(x_1-x_2\right)\left(1-\dfrac{1}{x_1x_2}\right).$\\
		Với mọi $x_1, x_2\in \left(1;+\infty \right)$ và $x_1<x_2$. Ta có $\heva{
				& x_1>1 \\
				& x_2>1 \\}\Rightarrow x_1\cdot x_2>1\Rightarrow \dfrac{1}{x_1\cdot x_2}<1.$\\
		Suy ra $\dfrac{f\left(x_1\right)-f\left(x_2\right)}{x_1-x_2}=1-\dfrac{1}{x_1x_2}>0\Rightarrow f(x)$ đồng biến trên $\left(1;+\infty \right)$.}
\end{ex}
\begin{ex}%[Phan Anh]%[0D2B1-3]
	Xét tính đồng biến, nghịch biến của hàm số $f(x)=\dfrac{x-3}{x+5}$ trên khoảng $\left(-\infty;-5\right)$ và trên khoảng $\left(-5;+\infty \right)$. Khẳng định nào sau đây đúng?
	\choice
	{Hàm số nghịch biến trên $\left(-\infty;-5\right)$, đồng biến trên $\left(-5;+\infty \right)$}
	{Hàm số đồng biến trên $\left(-\infty;-5\right)$, nghịch biến trên $\left(-5;+\infty \right)$}
	{Hàm số nghịch biến trên các khoảng $\left(-\infty;-5\right)$ và $\left(-5;+\infty \right)$}
	{\True Hàm số đồng biến trên các khoảng $\left(-\infty;-5\right)$ và $\left(-5;+\infty \right)$}
	\loigiai{
		Ta có
		\begin{eqnarray*}
			f\left(x_1\right)-f\left(x_2\right)&=&\left(\dfrac{x_1-3}{x_1+5}\right)-\left(\dfrac{x_2-3}{x_2+5}\right)\\
			&=&\dfrac{\left(x_1-3\right)\left(x_2+5\right)-\left(x_2-3\right)\left(x_1+5\right)}{\left(x_1+5\right)\left(x_2+5\right)}\\
			&=&\dfrac{8\left(x_1-x_2\right)}{\left(x_1+5\right)\left(x_2+5\right)}.
		\end{eqnarray*}
		Với mọi $x_1, x_2\in \left(-\infty;-5\right)$ và $x_1<x_2$. Ta có $\heva{& x_1<-5 \\& x_2<-5}\Leftrightarrow \heva{&x_1+5<0 \\& x_2+5<0.}$\\
		Suy ra $\dfrac{f\left(x_1\right)-f\left(x_2\right)}{x_1-x_2}=\dfrac{8}{\left(x_1+5\right)\left(x_2+5\right)}>0\Rightarrow f(x)$ đồng biến trên $\left(-\infty;-5\right)$.\\
		Với mọi $x_1, x_2\in \left(-5;+\infty \right)$ và $x_1<x_2$. Ta có $\heva{
				& x_1>-5 \\
				& x_2>-5 \\}\Leftrightarrow \heva{
				& x_1+5>0 \\
				& x_2+5>0 \\}$.\\
		Suy ra $\dfrac{f\left(x_1\right)-f\left(x_2\right)}{x_1-x_2}=\dfrac{8}{\left(x_1+5\right)\left(x_2+5\right)}>0\Rightarrow f(x)$ đồng biến trên $\left(-5;+\infty \right)$.}
\end{ex}

\begin{ex}%[Phan Anh]%[0D2K1-3]
	Cho hàm số $f(x)=\sqrt{2x-7}$. Khẳng định nào sau đây đúng?
	\choice
	{Hàm số nghịch biến trên $\left(\dfrac{7}{2};+\infty \right)$}
	{\True Hàm số đồng biến trên $\left(\dfrac{7}{2};+\infty \right)$}
	{Hàm số đồng biến trên $\mathbb{R}$}
	{Hàm số nghịch biến trên $\mathbb{R}$}
	\loigiai{
	Tập xác định là $\mathscr{D}=\left[\dfrac{7}{2};+\infty \right)$ nên ta loại đáp án C và D.\\
	Xét $f\left(x_1\right)-f\left(x_2\right)=\sqrt{2x_1-7}-\sqrt{2x_2-7}=\dfrac{2\left(x_1-x_2\right)}{\sqrt{2x_1-7}+\sqrt{2x_2-7}}.$\\
	Với mọi $x_1, x_2\in \left(\dfrac{7}{2};+\infty \right)$ và $x_1<x_2$, ta có $\dfrac{f\left(x_1\right)-f\left(x_2\right)}{x_1-x_2}=\dfrac{2}{\sqrt{2x_1-7}+\sqrt{2x_2-7}}>0.$\\
	Vậy hàm số đồng biến trên $\left(\dfrac{7}{2};+\infty \right)$.}
\end{ex}
\begin{ex}%[Phan Anh]%[0D2K1-3]
	Có bao nhiêu giá trị nguyên của tham số $m$ thuộc đoạn $\left[-3;3\right]$ để hàm số $f(x)=\left(m+1\right)x+m-2$ đồng biến trên $\mathbb{R}$?
	\choice
	{$7$}
	{$5$}
	{\True $4$}
	{$3$}
	\loigiai{
		Tập xác định $\mathscr{D}=\mathbb{R}.$\\
		Với mọi $x_1,x_2\in\mathscr{D}$ và $x_1<x_2$. \\Ta có
		$f\left(x_1\right)-f\left(x_2\right)=\left[\left(m+1\right)x_1+m-2\right]-\left[\left(m+1\right)x_2+m-2\right]=\left(m+1\right)\left(x_1-x_2\right).$\\
		Suy ra $\dfrac{f\left(x_1\right)-f\left(x_2\right)}{x_1-x_2}=m+1$.\\
		Để hàm số đồng biến trên $\mathbb{R}$ khi và chỉ khi
		$m+1>0\Leftrightarrow m>-1\xrightarrow{m\in \left[-3;3\right]}{m\in \mathbb{Z}}\Rightarrow m\in \left\{0;1;2;3\right\}$.\\
		Vậy có 4 giá trị nguyên của $m$ thỏa mãn.}
\end{ex}
% \begin{ex}%[Phan Anh]%[0D2K1-3]
% 	Tìm tất cả các giá trị thực của tham số $m$ để hàm số $y=-x^2+\left(m-1\right)x+2$ nghịch biến trên khoảng $\left(1;2\right)$.
% 	\choice
% 	{$m<5$}
% 	{$m>5$}
% 	{\True $m<3$}
% 	{$m>3$}
% 	\loigiai{
% 		Với mọi $x_1\ne x_2$, ta có\\
% 		$\dfrac{f\left(x_1\right)-f\left(x_2\right)}{x_1-x_2}=\dfrac{\left[-x_1^2+\left(m-1\right)x_1+2\right]-\left[-x_2^2+\left(m-1\right)x_2+2\right]}{x_1-x_2}=-\left(x_1+x_2\right)+m-1.$\\
% 		Để hàm số nghịch biến trên $\left(1;2\right)\Leftrightarrow-\left(x_1+x_2\right)+m-1<0$, với mọi $x_1,x_2\in \left(1;2\right)$\\
% 		$\Leftrightarrow m<\left(x_1+x_2\right)+1$, với mọi $x_1,x_2\in \left(1;2\right)$
% 		$\Leftrightarrow m<\left(1+1\right)+1=3$.}
% \end{ex}
\begin{ex}%[Phan Anh]%[0D2K1-3]
	\immini{Cho hàm số $y=f(x)$ có tập xác định là $\left[-3;3\right]$ và đồ thị của nó được biểu diễn bởi hình bên. Khẳng định nào sau đây là đúng?
		\choice
		{\True Hàm số đồng biến trên khoảng $\left(-3;-1\right)$ và $\left(1;3\right)$}
		{Hàm số đồng biến trên khoảng $\left(-3;-1\right)$và $\left(1;4\right)$}
		{Hàm số đồng biến trên khoảng $\left(-3;3\right)$}
		{Hàm số nghịch biến trên khoảng $\left(-1;0\right)$}}
	{\begin{tikzpicture}[>=stealth,scale=0.7]
			\draw[->](-4,0)--(4,0)node[above]{$x$};
			\draw[->](0,-2)--(0,5)node[right]{$y$};
			\draw (-3,-1)--(-1,1)--(0,1)node[above left]{$1$}--(3,4);
			\draw[dashed](-3,0)node[above]{$-3$}--(-3,-1)--(0,-1)node[right]{$-1$};
			\draw[dashed](-1,0)node[below]{$-1$}--(-1,1);
			\draw[dashed](3,0)node[below]{$3$}--(3,4)--(0,4)node[left]{$4$};
			\fill (-3,0)circle(1.2pt) (-3,-1)circle(1.2pt) (0,-1)circle(1.2pt) (-1,0)circle(1.2pt) (-1,1)circle(1.2pt) (0,1)circle(1.2pt) (3,0)circle(1.2pt) (3,4)circle(1.2pt) (0,4)circle(1.2pt) (0,0)node[above right]{$O$}circle(1.2pt);
		\end{tikzpicture}}
	\loigiai{
		Trên khoảng $\left(-3;-1\right)$ và $\left(1;3\right)$ đồ thị hàm số đi lên từ trái sang phải\\
		$\Rightarrow $ Hàm số đồng biến trên khoảng $\left(-3;-1\right)$ và $\left(1;3\right).$}
\end{ex}
\begin{ex}%[Phan Anh]%[0D2K1-3]
	\immini{Cho đồ thị hàm số $y=x^3$ như hình bên. Khẳng định nào sau đây \textbf{sai}?
		\choice
		{Hàm số đồng biến trên khoảng $\left(-\infty;0\right)$}
		{Hàm số đồng biến trên khoảng $\left(0;+\infty \right)$}
		{Hàm số đồng biến trên khoảng $\left(-\infty;+\infty \right)$}
		{\True Hàm số đồng biến tại gốc tọa độ $O$}}
	{\begin{tikzpicture}[>=stealth,scale=0.6]
			\draw[->](-2,0)--(2,0)node[above]{$x$};
			\draw[->](0,-3)--(0,3)node[right]{$y$};
			\draw[smooth,samples=100,domain=-1.4:1.4]plot(\x,{(\x)^3});
			\fill (0,0)node[above left]{$O$}circle(1.2pt);
		\end{tikzpicture}}
	\loigiai{Dựa vào đồ thị, ta thấy hàm số đồng biến trên toàn miền xác định. Nhưng không thể đồng biến chỉ tại đúng một điểm.}
\end{ex}
\begin{ex}
	\immini{Cho hàm số $y=f(x)$ có đồ thị là một đường liền nét trên đoạn $[-2;4]$ (hình bên). Xét trên $[-2;4]$, có bao nhiêu giá trị của $x$ để $y=1$?
		\haicot
		{$4$}
		{$5$}
		{\True vô số}
		{$1$}
	}{\hspace{1cm}
		\begin{tikzpicture}[smooth,samples=300,scale=0.6,>=stealth]
			\draw[black!30!] (-2,-1.5) grid (4,3);
			\draw[->] (-2,0)--(4.5,0) node[below]{$x$};
			\draw[->] (0,-1.5)--(0,3) node[right]{$y$};
			\foreach \x in {-2,-1,1,2,3,4}{
				\draw (\x,0) node[below]{$\x$};%Ox
			}
			\foreach \y in {1}{
				\draw (0,\y) node[left]{$\y$};%Oy
			}
			\draw (0,0) node[above right]{$O$};
			\draw[domain=-1.8:1,thick] plot(\x,{-(\x)^2+2});
			\draw [thick](1,1)--(4,1) node[above]{\small $y=f(x)$};
			\draw[fill=black] (0,2) circle(1.5pt) (1,1) circle(1pt);
	\end{tikzpicture}}
	\loigiai{
		Quan sát đồ thị, ta có các kết quả $f(1)=1$, $f(3)=1$ và $f(4)=1$ nên
		$$P=2f(1)+f(4)-f(3)=2+1-1=2.$$
	}
\end{ex}

\begin{ex} Trong các công thức dưới đây, công thức nào được xem là công thức của một hàm số $y$ theo biến $x$?
	\choice
	{$3x^2-y^2=0$ }
	{\True $3x^2-y+1=0$}
	{$y^2=x$}
	{$(y-x)(y+x)=1$}
	\loigiai{
	}
\end{ex}
\begin{ex}%[0D2B1-4]
	Trong các đường biểu diễn dưới đây, đường nào \textbf{không} phải là đồ thị của một hàm số?
	\choice
	{\begin{tikzpicture}[smooth,samples=300,scale=0.4,>=stealth]
		\draw[->] (-0.8,0)--(3.7,0) node[below]{$x$};
		\draw[->] (0,-1.5)--(0,2.5) node[right]{$y$};
		\draw (0,0) node[above left]{$O$};
		\draw[domain=0.3:3.7,thick] plot(\x,{(\x)^2-4*(\x)+3});
		\end{tikzpicture}}
	{\begin{tikzpicture}[smooth,samples=300,scale=0.4,>=stealth]
		\draw[->] (-2,0)--(4,0) node[below]{$x$};
		\draw[->] (0,-1.5)--(0,2.5) node[right]{$y$};
		\draw (0,0) node[below right]{$O$};
		\draw[magenta,thick] (-1,-1)--(0,2)--(1,0)--(3,2);
		\end{tikzpicture}}
	{\True \begin{tikzpicture}[smooth,samples=300,scale=0.4,>=stealth]
		\draw[->] (-2,0)--(2,0) node[below]{$x$};
		\draw[->] (0,-1.5)--(0,2.5) node[right]{$y$};
		\draw (0,0) node[above right]{$O$};
		\draw[magenta,thick] (90:1) arc (90:270:1);
		\draw[domain=0:1.7,thick] plot(\x,{(\x)^2-1});
		\end{tikzpicture} }
	{\begin{tikzpicture}[smooth,samples=300,scale=0.4,>=stealth]
		\draw[->] (-3,0)--(1,0) node[below]{$x$};
		\draw[->] (0,-1.5)--(0,2.5) node[right]{$y$};
		\draw (0,0) node[above left]{$O$};
		\draw[domain=-2.8:0.8,thick] plot(\x,{-(\x+3)^2+4*(\x+3)-2});
		\end{tikzpicture}}
	\loigiai{
	}
\end{ex}


\begin{ex}%[0D2B1-4]
	Trong các đường biểu diễn dưới đây, đường nào \textbf{không} phải là đồ thị của một hàm số?
	
	\choice
	{\begin{tikzpicture}[smooth,samples=300,scale=0.5,>=stealth]
			\draw[->] (-0.8,0)--(3.7,0) node[below]{$x$};
			\draw[->] (0,-1.5)--(0,2.5) node[right]{$y$};
			\draw (0,0) node[above left]{$O$};
			\draw[domain=0.3:3.7,thick] plot(\x,{(\x)^2-4*(\x)+3});
	\end{tikzpicture}}
	{\True \begin{tikzpicture}[smooth,samples=300,scale=0.5,>=stealth]
			\draw[->] (-2,0)--(4,0) node[below]{$x$};
			\draw[->] (0,-1.5)--(0,2.5) node[right]{$y$};
			\draw (0,0) node[below right]{$O$};
			\draw[magenta,thick] (-1,-1)--(-1,2)--(1,0)--(3,2);
	\end{tikzpicture}}
	{ \begin{tikzpicture}[smooth,samples=300,scale=0.5,>=stealth]
			\draw[->] (-2,0)--(2,0) node[below]{$x$};
			\draw[->] (0,-1.5)--(0,2.5) node[right]{$y$};
			\draw (0,0) node[below right]{$O$};
			\draw[magenta,thick] (0:1) arc (0:180:1) (1,0)--(2,2);
		\end{tikzpicture} }
	{\begin{tikzpicture}[>=stealth,scale=0.5]
			\draw[->] (-3,0)-- (0,0) node [below right]{$O$}--(6,0) node[below]{$x$};
			\draw[->] (0,-1.5)--(0,2) node [right]{$y$};
			\draw [line width = 1.2pt, domain = -3:5, samples=150] plot (\x,{cos(\x*180/pi)});
			\end{tikzpicture}}
	\loigiai{
		}
\end{ex}

\begin{ex}
	Bảng giá cước gọi quốc tế của công ty viễn thông A được cho bởi bảng sau:
	\begin{center}
		\begin{tikzpicture}[xscale=7,yscale=0.8,font=\footnotesize]
			\begin{scope}[shift={(-.5,.5)}]
				\fill[orange!15] (0,-1) rectangle (1,-5);
				\fill[cyan!30] (0,0) rectangle (2,-1);
				\draw [line width=0.7pt,gray](0,0) grid (2,-5)
					;
			\end{scope}
			\path
			(0,0) node{\text{\textbf{Thời gian gọi (phút)}}}  
			(-0.4,-1) node[right]{Không quá 8 phút}
			(-0.4,-2) node[right]{Từ phút thứ 9 đến phút thứ 15}    
			(-0.4,-3) node[right]{Từ phút thứ 16 đến phút thứ 25}
			(-0.4,-4) node[right]{Từ phút 26 trở đi}    
					
			(1,0) node{\text{\textbf{Giá cước điện thoại (đồng/phút)}}}    
			(1,-1) node[right]{6500}
			(1,-2) node[right]{6000}    							
			(1,-3) node[right]{5500}
			(1,-4) node[right]{5000}    
				;
		\end{tikzpicture}
	\end{center}
	Ông An thực hiện cuộc gọi quốc tế 12 phút. Số tiền cước ông An phải trả là
	\choice
	{72 000 đồng  }
	{\True 76 000 đồng }
	{70 000 đồng }
	{90 000 đồng}
	\loigiai{
		\begin{itemize}
			\item [$\bullet$] Có thể thiệt lập biểu thức tính giá cước từ phút thứ 9 đến 15 là $6000t + 4000$. Thay $t =12$, ta được số tiền là 76 000 đồng.
			\item [$\bullet$] Hoặc tính nhanh: $ 8 \times 6500 + 4 \times 6000 = 76 000$ đồng.
		\end{itemize}
	}
\end{ex}
\begin{ex}%[0D2B1]
	Tìm tất cả các giá trị của $m$ để hàm số $y=\dfrac{x\sqrt{5}}{x^2-2x+m}$ có tập xác định là $\mathbb{R}$.
	\choice
	{\True $m>1$}
	{$m=1$}
	{$m<1$}
	{$m<0$}
	\loigiai{
	Hàm số có tập xác định là $\mathbb{R}$ khi và chỉ khi $x^2-2x+m =0$ vô nghiệm $\Leftrightarrow \Delta'=1-m<0 \Leftrightarrow m>1$.
	}
\end{ex}

\begin{ex}%[0D2G1-2]%
	Tìm các giá trị thực của tham số $m$ để hàm số $y=\dfrac{x+m+2}{x-m}$ xác định trên $(-1;2)$.
	\choice
	{$\left\{\begin{aligned}
			&m\leq -1\\
			&m\geq 2\\
		\end{aligned}\right. $}
	{$\left[\begin{aligned}
			&m<-1\\
			&m>2\\
		\end{aligned}\right. $ }
	{$-1<m<2$}
	{\True $\left[\begin{aligned}
			&m\leq -1\\
			&m\geq 2\\
		\end{aligned}\right. $ }
	\loigiai{
		Hàm số $y=\dfrac{x+m+2}{x-m}$ xác định khi $x\ne m$.\\
		Để hàm số $y=\dfrac{x+m+2}{x-m}$ xác định trên $(-1;2)$ khi và chỉ khi $\left[\begin{aligned}
			&m\leq -1\\
			&m\geq 2.\\
		\end{aligned}\right. $}
\end{ex}

\Closesolutionfile{ans}

% ---------Mục lục chính
\FULLWIDTH
\tableofcontents %lệnh in mục lục chính
\begin{center}
\includegraphics[width=5cm]{QRcode/10D4X12.png}
\end{center}
\end{document}