% Câu 14
\setcounter {section} {13}
\setcounter{ex}{0}
\section{Thể tích khối chóp}
\subsection{Kiến thức cần nhớ}
\begin{khung}
\subsubsection{Thể tích khối chóp}
			$$V=\dfrac{1}{3}\cdot B\cdot h.$$			
\immini{Trong đó:
	\begin{itemize}
		\item $B$ là diện tích đa giác đáy.
		\item $h$ là chiều cao khối chóp.
\end{itemize}}{\begin{tikzpicture}[scale=0.8, font=\footnotesize, line join=round, line cap=round,>=stealth,declare function={a=4;h=4;}]
\path 	(0:0) coordinate (A)
++(0:a) coordinate (B)
++(-120:a/2) coordinate (C)
($(C)!.5!(B)$) coordinate (M)
($(A)!2/3!(M)$) coordinate (H)
($(H)+(90:h)$) coordinate (S);
\draw[thick] (S)--(A)--(C)--(B)--cycle (S)--(C) ;
\draw[dashed,thick] (A)--(B) (S)--(H);
\foreach \x / \goc in {A/180,B/0,C/-90,H/-90,S/90}
\fill (\x) circle (1.5pt)
($(\x)+(\goc:3mm)$) node {$\x$};
\path pic[draw,angle radius=5pt]{right angle= M--H--S};
\end{tikzpicture}}
\subsubsection{Diện tích đa giác}
\begin{minipage}[tl]{10cm}
	\begin{itemize}
		\item Diện tích tam giác vuông: $S_{\triangle ABC}=\dfrac{1}{2} AB\cdot AC$.\\
		\\
		\item  Diện tích tam giác đều: $S_{\triangle ABC}=\dfrac{AB^2\sqrt{3}}{4}$.\\
		\\
		\item Diện tích hình chữ nhật: $S_{ABCD}=\dfrac{1}{2}AB\cdot AD.$
	\end{itemize}
\end{minipage}
\hspace*{2cm}
\begin{minipage}[tr]{4cm}
% Tam giác vuông
\begin{tikzpicture}[scale=0.6, font=\footnotesize, line join=round, line cap=round,>=stealth,declare function={r=3;h=3;}]
	\path (0:0) coordinate (A)
	(0:r) coordinate (B)
	(90:h) coordinate (C);
	\draw[thick] (B)--(A)--(C)--cycle;
	\path pic[draw,angle radius=5pt]{right angle= B--A--C};
	\foreach \x/\g in {A/180,B/0,C/90}
	\fill[black] 	(\x) circle (1pt)
	($(\g:4mm)+(\x)$) node {$\x$};	
\end{tikzpicture}
% Tam giác đều
\begin{tikzpicture}[scale=0.6, font=\footnotesize, line join=round, line cap=round,>=stealth,declare function={r=3;}]
	\path (0:0) coordinate (B)
	(0:r) coordinate (C)
	(60:r) coordinate (A)
	($(B)!.5!(C)$) coordinate (H);
	\draw [thick](B)--(A)--(C)--cycle (A)--(H);
	\path pic[draw,angle radius=5pt]{right angle= C--H--A};
	\foreach \x/\g in {A/90,B/180,C/0,H/-90}
	\fill[black] 	(\x) circle (1pt)
	($(\g:4mm)+(\x)$) node {$\x$};	
	\foreach \x/\y in {B/H,H/C}{
		\path (\x)--(\y) node[midway,sloped]{\tikz{\draw [shift={(-0.65pt,0)}](-90:1pt)--(90:1pt) [shift={(0.65pt,0)}](-90:1pt)--(90:1pt);}};}
\end{tikzpicture}
% Hình chữ nhật
\begin{tikzpicture}[scale=0.6, font=\footnotesize, line join=round, line cap=round,>=stealth,declare function={r=3;a=4;}]
	\path (0,0) coordinate (A)
	(a,0) coordinate (B)
	(a,r) coordinate (C)
	(0,r) coordinate (D);
	\draw[thick] (A)--(B)--(C)--(D)--cycle;
	\foreach \t/\g in {A/-90,B/-90,C/90,D/90}{
		\path (\t) node[shift={(\g:7pt)}]{$ \t $};}
	\path pic[draw,angle radius=5pt]{right angle= B--A--D};
\end{tikzpicture}
\end{minipage}\\
\begin{minipage}[tl]{10cm}
	\begin{itemize}
		\item Diện tích hình thoi: $S_{ABCD}=\dfrac{1}{2}AC\cdot BD.$\\
		\\
		\item  Diện tích lục giác đều: $S_{ABCDEF}=6\cdot S_{\triangle OAB}$.
	\end{itemize}
\end{minipage}
\hspace*{2cm}
\begin{minipage}[t]{4cm}
	% Hình thoi
	\begin{tikzpicture}[scale=0.6, font=\footnotesize, line join=round, line cap=round,>=stealth,declare function={r=3;}]
		\path (0:0) coordinate (A)
		(-30:r) coordinate (B)
		(30:r) coordinate (D)
		($(B)+(D)-(A)$) coordinate (C);
		\path (intersection of A--C and B--D) coordinate (I);
		\draw [thick](A)--(B)--(C)--(D)--cycle (A)--(C) (B)--(D);
		\foreach \t/\g in {A/180,B/-90,C/0,D/90}{
			\path (\t) node[shift={(\g:7pt)}]{$ \t $};}
		\path pic[draw,angle radius=5pt]{right angle= C--I--D};
	\end{tikzpicture}
	% Hình lục giác đều
	\begin{tikzpicture}[scale=0.5, font=\footnotesize, line join=round, line cap=round,>=stealth,declare function={r=3;}]
		\path (0,0) coordinate (A)
		(r,0) coordinate (B)
		($(B)!1!120:(A)$) coordinate (C)
		($(C)!1!120:(B)$) coordinate (D)
		($(D)!1!120:(C)$) coordinate (E)
		($(E)!1!120:(D)$) coordinate (F);
		\path (intersection of A--D and B--E) coordinate (O);
		\draw[thick] (A)--(B)--(C)--(D)--(E)--(F)--cycle (A)--(D) (B)--(E) (C)--(F);
		\foreach \t/\g in {A/90,B/90,C/0,D/-90,E/-90,F/180,O/-90}{
			\path (\t) node[shift={(\g:8pt)},font=\scriptsize]{$ \t $};
		}
	\end{tikzpicture}
\end{minipage}
\end{khung}
\subsection{Bài tập mẫu}
\Opensolutionfile{ans}[ans/ANS-DANG-14]
\begin{khung}
	\begin{vd}%[Nguyễn Văn Hiệp]%[2H1B3-2]
	[Đề minh họa BGD 2022-2023]
			\immini{Cho khối chóp $S.ABC$ có đáy là tam giác vuông cân tại $A$, $AB=2$, $SA$ vuông góc với đáy và $SA=3$ (Tham khảo hình vẽ). Thể tích khối chóp đã cho bằng
			\choice
			{$12$}
			{\True $2$}
			{$6$}
			{$4$}
			}{\begin{tikzpicture}[scale=0.8, font=\footnotesize, line join=round, line cap=round,>=stealth,declare function={a=3;h=3;}]
					\path 	(0:0) coordinate (A)
					++(0:a) coordinate (C)
					++(-160:1.4*a) coordinate (B)
					($(A)+(90:h)$) coordinate (S);
					\draw [thick] (B)--(C)--(S)--cycle;
					\draw[thick,dashed] 	(S)--(A)--(C) (A)--(B);
					\foreach \x /\g in {A/180,C/0,B/-90,S/90}
					\fill[black] (\x) circle (1pt)
					($(\x)+(\g:3mm)$) node {$\x$};
					\foreach \x/\y/\z in {C/A/S,B/A/C}{
						\path pic[draw,angle radius=5pt]{right angle= \x--\y--\z};}
			\end{tikzpicture}}
			\loigiai{
			Thể tích khối chóp là $V=\dfrac{1}{3}\cdot S_{\triangle ABC}\cdot SA=\dfrac{1}{3}\cdot \dfrac{1}{2}\cdot AB \cdot AC\cdot SA=\dfrac{1}{6}\cdot 2\cdot 2\cdot 3=2$.
			}
	\end{vd}
\end{khung}
\subsection{Bài tập tương tự và phát triển}
\begin{ex}%[Nguyễn Văn Hiệp]%[2H1B3-2] % Câu 1.
	Cho hình chóp $S.ABC$ có đáy $ABC$ là tam giác vuông tại $B$. Cạnh bên $SA$ vuông góc với mặt phẳng đáy, $SA=AB=2a$, $BC=3a$. Thể tích của khối chóp $S.ABC$ là
	\choice
	{$a^3$}
	{$3a^3$}
	{$4a^3$}
	{\True $2a^3$}
	\loigiai{
	\begin{minipage}[c]{10cm}
		$V=\dfrac {1}{3}\cdot \dfrac {1}{2}\cdot AB\cdot BC\cdot SA=2a^3$.
	\end{minipage}
	\hspace*{1cm}\begin{minipage}[l]{4cm}
		\begin{tikzpicture}[scale=0.8, font=\footnotesize, line join=round, line cap=round,>=stealth,declare function={a=3;h=3;}]
				\path 	(0:0) coordinate (A)
				++(0:a) coordinate (C)
				++(-150:3*a/4) coordinate (B)
				($(A)+(90:h)$) coordinate (S);
				\draw[thick] (A)--(B)--(C)--(S)--cycle (S)--(B);
				\draw[thick,dashed] 	(A)--(C);
				\foreach \x /\g in {A/180,C/0,B/-135,S/90}
				\fill[black] (\x) circle (1pt)
				($(\x)+(\g:3mm)$) node {$\x$};
				\foreach \x/\y/\z in {C/B/A,C/A/S}{
					\path pic[draw,angle radius=5pt]{right angle= \x--\y--\z};}
			\end{tikzpicture}
		\end{minipage}
			}
\end{ex}
\begin{ex}%[Nguyễn Văn Hiệp]%[2H1B3-2] % Câu 2.
	Cho khối chóp $S.ABCD$ có đường cao $SA$ và đáy $ABCD$ là hình thoi. Thể tích khối chóp đã cho được tính theo công thức nào sau đây?
	\choice
	{$\dfrac {1}{3} \,SA\cdot AB^2$}
	{$\dfrac {1}{3} \,SA\cdot AC\cdot BD$}
	{\True $\dfrac {1}{6} \,SA\cdot AC\cdot BD$}
	{$\dfrac {1}{2}\, SA\cdot AB^2 $}
		\loigiai{\immini{Diện tích đáy của hình chóp là $S_{ABCD}=\dfrac {1}{2}AC\cdot BD$.\\
				Ta có thể tích khối chóp $S.ABCD$ là\\
				$V=\dfrac {1}{3}\cdot S_{ABCD}\cdot SA=\dfrac {1}{3}\cdot \dfrac {1}{2}\cdot AC\cdot BD\cdot SA=\dfrac{1}{6}\,SA\cdot AC\cdot BD$.}{\begin{tikzpicture}[scale=0.8, font=\footnotesize, line join=round, line cap=round,>=stealth,declare function={a=3;h=3;}]
					\path 	(0:0) coordinate (A)
					++(0:a) coordinate (B)
					++(-130:a/2) coordinate (C)
					($(A)+(C)-(B)$) coordinate (D)
					($(A)+(90:h)$) coordinate (S)
					(intersection of A--C and B--D) coordinate (O);
					\draw[thick,dashed] 	(B)--(A)--(S) (A)--(C) (B)--(D)	(D)--(A);
					\draw[thick] (D)-- (C)--(B)--(S)--cycle (C)--(S);
					\foreach \x/\g in {A/160,D/-135,C/-45,B/20,S/90,O/-90}
					\fill[black] 	(\x) circle (1pt)
					($(\g:3mm)+(\x)$) node {$\x$};	
					\foreach \x/\y/\z in {B/A/S,C/O/B}{
						\path pic[draw,angle radius=5pt]{right angle= \x--\y--\z};}
			\end{tikzpicture}}}
\end{ex}
	\begin{ex}%[Nguyễn Văn Hiệp]%[2H1B3-2] % Câu 3}
	Cho hình chóp $S.ABC$ có $SA$ vuông góc với mặt phẳng $(ABC)$. Biết $SA=2a$ và tam giác $ABC$ vuông tại $A$ có $AB=3a$, $AC=4a$. Tính thể tích khối chóp $S.ABC$ theo $a$.
	\choice
	{$6a^3$}
	{$8a^3$}
	{\True $4a^3$}
	{$12a^3$}
	\loigiai{\immini{Ta có $S_{\triangle ABC}=\dfrac {1}{2}\cdot 3a\cdot 4a=6a^2$.\\
			$V_{SABC}=\dfrac {1}{3}\cdot SA\cdot S_{\triangle ABC}=\dfrac {1}{3}\cdot 2a\cdot 6a^2=4a^3$.}{\begin{tikzpicture}[scale=0.8, font=\footnotesize, line join=round, line cap=round,>=stealth,declare function={a=4;h=2.5;}]
				\path 	(0:0) coordinate (A)
				++(0:a) coordinate (C)
				++(-150:3*a/4) coordinate (B)
				($(A)+(90:h)$) coordinate (S);
				\draw[thick] (A)--(B)--(C)--(S)--cycle (S)--(B);
				\draw[thick,dashed] 	(A)--(C);
				\foreach \x /\g in {A/180,C/0,B/-135,S/90}
				\fill[black] (\x) circle (1pt)
				($(\x)+(\g:3mm)$) node {$\x$};
				\foreach \x/\y/\z in {B/A/C,C/A/S}{
					\path pic[draw,angle radius=5pt]{right angle= \x--\y--\z};}
		\end{tikzpicture}}}
\end{ex}
\begin{ex}%[Nguyễn Văn Hiệp]%[2H1B3-2] % Câu 4.
	Cho hình chóp $S.ABC$ có đáy $ABC$ là tam giác đều cạnh $a$, cạnh bên $SA$ vuông góc với đáy, $SA=a$ . Thể tích của khối chóp $S.ABC$ bằng
	\choice
	{\True $\dfrac {a^3\sqrt {3}}{12}$}
	{$\dfrac {a^3\sqrt {3}}{4}$}
	{$\dfrac {a^3\sqrt {3}}{6}$}
	{$\dfrac {a^3}{4}$}
	\loigiai{\immini{Thể tích của khối chóp $S.ABC$ bằng\\
			$V=\dfrac {1}{3}\cdot S_{\triangle ABC}\cdot SA=\dfrac {1}{3}\cdot \dfrac {a^2 \sqrt{3}}{4} \cdot a=\dfrac {a^3\sqrt {3}}{12}$.}{	\begin{tikzpicture}[scale=0.7, font=\footnotesize, line join=round, line cap=round,>=stealth,declare function={a=4;h=3;}]
				\path 	(0:0) coordinate (A)
				++(0:a) coordinate (C)
				++(-130:5*a/6) coordinate (B)
				($(A)+(90:h)$) coordinate (S);
				\draw[thick] (A)--(B)--(C)--(S)--cycle (S)--(B);
				\draw[thick,dashed] 	(A)--(C);
				\foreach \x /\g in {A/180,C/0,B/-90,S/90}
				\fill[black] (\x) circle (1pt)
				($(\x)+(\g:3mm)$) node {$\x$};
				\foreach \x/\y/\z in {C/A/S}{
					\path pic[draw,angle radius=5pt]{right angle= \x--\y--\z};}
		\end{tikzpicture}}}
\end{ex}
\begin{ex}%[Nguyễn Văn Hiệp]%[2H1B3-2] % Câu 5.
	Cho hình chóp $S.ABCD$ có đáy $ABCD$ là hình vuông cạnh $a$. Biết $SA\perp (ABCD)$ và $SA=a\sqrt {3}$. Thể tích của khối chóp $S.ABCD$ là
	\choice
	{$ a^3\sqrt {3}$}
	{$\dfrac {a^3\sqrt {3}}{12}$}
	{\True $\dfrac {a^3\sqrt {3}}{3}$}
	{$\dfrac {a^3}{4}$}
	\loigiai{\immini{Ta có $h=SA=a\sqrt {3}$; $B=S_{ABCD}=a^2$.\\
			Thể tích khối chóp là $V=\dfrac {1}{3}Bh=\dfrac {a^3\sqrt {3}}{3}$.}{\begin{tikzpicture}[scale=0.6, font=\footnotesize, line join=round, line cap=round,>=stealth,declare function={a=4;h=4;}]
				\path 	(0:0) coordinate (A)
				++(0:a) coordinate (B)
				++(-130:3*a/4) coordinate (C)
				($(A)+(C)-(B)$) coordinate (D)
				($(A)+(90:h)$) coordinate (S)
				(intersection of A--C and B--D) coordinate (O);
				\draw[thick,dashed] 	(B)--(A)--(S) (A)--(C) (B)--(D)	(D)--(A);
				\draw[thick] (D)-- (C)--(B)--(S)--cycle (C)--(S);
				\foreach \t/\g in {A/160,D/-135,C/-45,B/20,S/90,O/-100}{
					\draw[fill=white] (\t) circle (1pt) node[shift={(\g:7pt)},font=\scriptsize]{$ \t $};
					\foreach \x/\y/\z in {B/A/S}{
						\path pic[draw,angle radius=5pt]{right angle= \x--\y--\z};}
				}
		\end{tikzpicture}}}
\end{ex}
%Thay C6 vì trùng đề C5
\begin{ex}%[Nguyễn Văn Hiệp]%[2H1B3-2] % Câu 6.
	Cho khối chóp $S.ABC$ có chiều cao bằng $3$, đáy $ABC$ có diện tích bằng $10$. Thể tích của khối chóp $S.ABC$ bằng
	\choice
	{$2$}
	{$15$}
	{\True $10$}
	{$30$}
	\loigiai{
		Thể tích khối khóp $S.ABC$ là $V=\dfrac {1}{3}Bh=\dfrac {1}{3}10\cdot 3=10$.}
\end{ex}
\begin{ex}%[Nguyễn Văn Hiệp]%[2H1B3-2] % Câu 7.
	Cho hình chóp $S.ABC$ có đáy $ABC$ là tam giác vuông tại $B$, $AB=a$, $BC=2a$, $SA\perp (ABC)$, $SA=3a$. Thể tích của khối chóp $S.ABC$ bằng
	\choice
	{\True $a^3$}
	{$\dfrac {1}{3}a^3$}
	{$3a^3$}
	{$\dfrac {1}{6}a^3$}
	\loigiai{\immini{Thể tích khối chóp là\\
			$V_{S.ABC}=\dfrac {1}{3}S_{\triangle ABC}\cdot SA=\dfrac {1}{3}\cdot \dfrac {1}{2}BA\cdot BC\cdot SA=\dfrac {1}{6} a\cdot 2a\cdot 3a=a^3$.}{\begin{tikzpicture}[scale=0.8, font=\footnotesize, line join=round, line cap=round,>=stealth,declare function={a=3;h=4;}]
				\path 	(0:0) coordinate (A)
				++(0:a) coordinate (C)
				++(-150:3*a/4) coordinate (B)
				($(A)+(90:h)$) coordinate (S);
				\draw[thick] (A)--(B)--(C)--(S)--cycle (S)--(B);
				\draw[thick,dashed] 	(A)--(C);
				\foreach \x /\g in {A/180,C/0,B/-135,S/90}
				\fill[black] (\x) circle (1pt)
				($(\x)+(\g:3mm)$) node {$\x$};
				\foreach \x/\y/\z in {C/B/A,C/A/S}{
					\path pic[draw,angle radius=5pt]{right angle= \x--\y--\z};}
		\end{tikzpicture}}}
\end{ex}
\begin{ex}%[Nguyễn Văn Hiệp]%[2H1B3-2] % Câu 8.
	Cho hình chóp $S.ABC$ có các cạnh $SA,SB,SC$ đôi một vuông góc với nhau. Biết $SA=3$, $SB=4$, $SC=5$, thể tích khối chóp $S.ABC$ bằng
	\choice
	{$60$}
	{\True $10$}
	{$20$}
	{$30$}
	\loigiai{\immini{Dễ thấy $SA\perp (SBC)$\\
			nên $V_{S.ABC}=\dfrac {1}{3}\cdot S_{\triangle SBC}\cdot SA=\dfrac {1}{3}\cdot \dfrac {1}{2}\cdot 4\cdot 5\cdot 3=10$.}{\begin{tikzpicture}[scale=0.8, font=\footnotesize, line join=round, line cap=round,>=stealth,declare function={a=3;h=3;}]
				\path 	(0:0) coordinate (S)
				++(0:a) coordinate (C)
				++(-160:1.4*a) coordinate (B)
				($(S)+(90:h)$) coordinate (A);
				\draw [thick] (B)--(C)--(A)--cycle;
				\draw[thick,dashed] 	(A)--(S)--(B) (S)--(C);
				\foreach \x /\g in {S/180,C/0,B/-90,A/90}
				\fill[black] (\x) circle (1pt)
				($(\x)+(\g:3mm)$) node {$\x$};
				\foreach \x/\y/\z in {C/S/A,B/S/C}{
					\path pic[draw,angle radius=5pt]{right angle= \x--\y--\z};}
		\end{tikzpicture}}}
\end{ex}
\begin{ex}%[Nguyễn Văn Hiệp]%[2H1B3-2] % Câu 9.
	Cho hình chóp $S.ABC$ có đáy $ABC$ là tam giác vuông cân tại $A$, $AC=2a$. Cạnh bên $SA$ vuông góc với mặt phẳng đáy và $SA=a$. Thể tích khối chóp $S.ABC$ bằng
	\choice
	{$4a^3$}
	{\True $\dfrac {2a^3}{3}$}
	{$2a^3$}
	{$\dfrac {4a^3}{3}$}
	\loigiai{\immini{
			Diện tích đáy $S_{ABC}=\dfrac {1}{2}AB\cdot AC=\dfrac {1}{2}\cdot 2a\cdot 2a=2a^2$.\\
			Thể tích khối chóp là $V=\dfrac {1}{3}\cdot SA\cdot S_{\triangle ABC}=\dfrac {1}{3}\cdot a\cdot 2a^2=\dfrac {2a^3}{3}$.
	}{\begin{tikzpicture}[scale=0.8, font=\footnotesize, line join=round, line cap=round,>=stealth,declare function={a=5;h=3;}]
		\path 	(0:0) coordinate (A)
		++(0:a) coordinate (C)
		++(-130:3*a/4) coordinate (B)
		($(A)+(90:h)$) coordinate (S);
		\draw[thick] (A)--(B)--(C)--(S)--cycle (S)--(B);
		\draw[thick,dashed] 	(A)--(C);
		\foreach \x /\g in {A/180,C/0,B/-90,S/90}
		\fill[black] (\x) circle (1pt)
		($(\x)+(\g:3mm)$) node {$\x$};
		\foreach \x/\y/\z in {B/A/C,C/A/S}{
		\path pic[draw,angle radius=5pt]{right angle= \x--\y--\z};}
		\path (A)--(S) node[midway,left]{$a$};
		\path (A)--(B) node[midway,left]{$2a$};
		\path (A)--(C) node[midway,above]{$2a$};
	\end{tikzpicture}}}
\end{ex}
\begin{ex}%[Nguyễn Văn Hiệp]%[2H1B3-2] % Câu 10.
	Cho hình chóp $S.ABCD$ có đáy là hình vuông cạnh $a,\,\ SA \perp (ABCD)$ và $SA=a$. Thể tích khối chóp $S.ABCD$ bằng
	\choice
	{\True $\dfrac {1}{3}a^3$}
	{$2a^3$}
	{$3a^3$}
	{$a^3$}
	\loigiai{
		Ta có $V_{S.ABCD}=\dfrac {1}{3}\cdot SA\cdot S_{ABCD}=\dfrac {1}{3}\cdot a\cdot a^2=\dfrac {1}{3}a^3$.
	}
\end{ex}
%Thay C11 vì trùng C13 
\begin{ex}%[Nguyễn Văn Hiệp]%[2H1B3-2] % Câu 11.
	Cho hình chóp $S.ABCD$ có đáy $ABCD$ là hình chữ nhật, $AB=3a$ và $AD=4a$. Cạnh bên $SA$ vuông góc với mặt phẳng $( ABCD )$ và $SA=a\sqrt {2}$. Thể tích của khối chóp $S.ABCD$ bằng
	\choice
	{\True $4\sqrt {2}{a^3}$}
	{$12\sqrt {2}{a^3}$}
	{$\dfrac{4\sqrt {2}{a^3}}{3}$}
	{$\dfrac{2\sqrt {2}{a^3}}{3}$}
		\loigiai{\immini{Diện tích đáy hình chữ nhật là \\
				$S_{ABCD}=AB\cdot AD=3a\cdot 4a =12a^2$.\\
			Thể tích của khối chóp $S.ABCD$ là\\
			 $V=\dfrac{1}{3}S_{ABCD}\cdot SA=\dfrac{1}{3}\cdot 12a^2\cdot a\sqrt{2}=4\sqrt{2}a^3$.}{\begin{tikzpicture}[scale=0.6, font=\footnotesize, line join=round, line cap=round,>=stealth,declare function={a=6;h=4;}]
			 	\path 	(0:0) coordinate (A)
			 	++(0:a) coordinate (D)
			 	++(-120:3*a/5) coordinate (C)
			 	($(A)+(C)-(D)$) coordinate (B)
			 	($(A)+(90:h)$) coordinate (S);
			 	\draw[thick,dashed] 	(B)--(A)--(S) (D)--(A);
			 	\draw[thick] (D)-- (C)--(B)--(S)--cycle (C)--(S);
			 	\foreach \t/\g in {A/180,D/0,C/-90,B/-90,S/90}{
			 		\draw[fill=white] (\t) circle (1pt) node[shift={(\g:7pt)},font=\scriptsize]{$ \t $};
			 		\foreach \x/\y/\z in {C/B/A,S/A/D}{
			 			\path pic[draw,angle radius=5pt]{right angle= \x--\y--\z};}}
		 		\path (A)--(D) node[midway,above,sloped]{$4a$};
		 		\path (A)--(B) node[midway,right]{$3a$};
		 \end{tikzpicture}}
			}
\end{ex}
\begin{ex}%[Nguyễn Văn Hiệp]%[2H1B3-2] % Câu 12
	Cho hình chóp $S.ABC$ có $SA$ vuông góc với đáy. Tam giác $ABC$ vuông cân tại $B$, biết $SA=AC=2a$. Tính thể tích khối chóp $S.ABC$.
	\choice
	{$\dfrac {2\sqrt {2}}{3} a^3$}
	{$\dfrac {4}{3} a^3$}
	{\True $\dfrac {2}{3} a^3$}
	{$\dfrac {1}{3} a^3$}
	\loigiai{\immini{Ta có $AB=BC=\dfrac {AC}{\sqrt {2}}=\dfrac {2a}{\sqrt {2}}=a\sqrt {2}$.\\
			Thể tích khối chóp $S.ABC$ là\\
			$V=\dfrac {1}{3}S_{\triangle ABC}\cdot SA=\dfrac {1}{3}\cdot \dfrac {1}{2}AB^2\cdot SA=\dfrac {1}{6}\cdot \left(a\sqrt {2} \right)^2\cdot 2a=\dfrac {2}{3}a^3$.}{\begin{tikzpicture}[scale=0.8, font=\footnotesize, line join=round, line cap=round,>=stealth,declare function={a=3;h=3;}]
				\path 	(0:0) coordinate (A)
				++(0:a) coordinate (C)
				++(-130:3*a/4) coordinate (B)
				($(A)+(90:h)$) coordinate (S);
				\draw[thick] (A)--(B)--(C)--(S)--cycle (S)--(B);
				\draw[thick,dashed] 	(A)--(C);
				\foreach \x /\g in {A/180,C/0,B/-90,S/90}
				\fill[black] (\x) circle (1pt)
				($(\x)+(\g:3mm)$) node {$\x$};
				\foreach \x/\y/\z in {C/B/A,C/A/S}{\path pic[draw,angle radius=5pt]{right angle= \x--\y--\z};}
		\end{tikzpicture}}}
\end{ex}
\begin{ex}%[Nguyễn Văn Hiệp]%[2H1B3-2] % Câu 13.
	Cho hình chóp $S.ABCD$ có đáy $ABCD$ là hình vuông cạnh $a$, $SA=3a$ và $SA$ vuông góc với mặt phẳng đáy. Tính thể tích khối chóp $S.ABCD$.
	\choice
	{$3a^3$}
	{$\dfrac {a^3}{3}$}
	{$9a^3$}
	{\True $ a^3$}
	\loigiai{\immini{Ta có $S_{ABCD}=a^2$, đường cao $SA=3a$.\\
			Vậy thể tích khối chóp $S.ABCD$ là\\
			 $V=\dfrac {1}{3}S_{ABCD}\cdot SA=\dfrac {1}{3}\cdot a^2\cdot 3a=a^3$.}{
			\begin{tikzpicture}[scale=0.6, font=\footnotesize, line join=round, line cap=round,>=stealth,declare function={a=4;h=4;}]
				\path 	(0:0) coordinate (A)
				++(0:a) coordinate (B)
				++(-130:3*a/4) coordinate (C)
				($(A)+(C)-(B)$) coordinate (D)
				($(A)+(90:h)$) coordinate (S)
				(intersection of A--C and B--D) coordinate (O);
				\draw[thick,dashed] 	(B)--(A)--(S) (A)--(C) (B)--(D)	(D)--(A);
				\draw[thick] (D)-- (C)--(B)--(S)--cycle (C)--(S);
				\foreach \t/\g in {A/160,D/-135,C/-45,B/20,S/90,O/-100}{
					\draw[fill=white] (\t) circle (1pt) node[shift={(\g:7pt)},font=\scriptsize]{$ \t $};
					\foreach \x/\y/\z in {B/A/S,S/A/D}{
						\path pic[draw,angle radius=5pt]{right angle= \x--\y--\z};}}
		\end{tikzpicture}}
	}
\end{ex}
\begin{ex}%[Nguyễn Văn Hiệp]%[2H1B3-2] % Câu 14.
	Cho khối chóp $S.ABC$ có đáy $ABC$ là tam giác vuông tại $A$, $AC=3$, $AB=4$, $SA$ vuông góc với mặt phẳng $(ABC)$ và $SA=3$. Thể tích của khối chóp đã cho bằng
	\choice
	{$12$}
	{\True $6$}
	{$18$}
	{$20$}
	\loigiai{
		\begin{itemize}
		\item Diện tích tam giác $ABC$ là $S_{\triangle ABC}=\dfrac {1}{2}AB\cdot AC=\dfrac {1}{2}\cdot 4\cdot 3=6$.
		\item Thể tích khối chóp $S.ABC$ là $V=\dfrac {1}{3}S_{\triangle ABC}\cdot SA=\dfrac {1}{3}\cdot 6\cdot 3=6$.
		\end{itemize}
		}
\end{ex}
%Thay C15 vì trùng C13
\begin{ex}%[Nguyễn Văn Hiệp]%[2H1B3-2] % Câu 15.
	Cho khối chóp $S.ABC$ có $SA$ vuông góc với đáy, $SA=4$, $AB=6$, $BC=10$ và $CA=8$. Thể tích $V$ của khối chóp $S.ABC$
	\choice
	{\True $V=32$}
	{$V=192$}
	{$V=40$}
	{$V=24$}
	\loigiai{\immini{Ta có $10^2=8^2+6^2$ hay $BC^2=CA^2+AB^2$.\\
			Suy ra $\triangle ABC$ vuông tại $A$.\\
			Diện tích $\triangle ABC$ là $S_{\triangle ABC}=\dfrac{1}{2}\cdot AB\cdot AC=\dfrac{1}{2}\cdot 6\cdot 8=24.$\\
			Thể tích của khối chóp $S.ABC$:\\
			$V=\dfrac{1}{3}\cdot S_{ABC}\cdot SA=\dfrac{1}{3}\cdot 24\cdot 4=32$.}{\begin{tikzpicture}[scale=0.8, font=\footnotesize, line join=round, line cap=round,>=stealth,declare function={a=4;h=2.5;}]
				\path 	(0:0) coordinate (A)
				++(0:a) coordinate (C)
				++(-150:3*a/4) coordinate (B)
				($(A)+(90:h)$) coordinate (S);
				\draw[thick] (A)--(B)--(C)--(S)--cycle (S)--(B);
				\draw[thick,dashed] 	(A)--(C);
				\foreach \x /\g in {A/180,C/0,B/-135,S/90}
				\fill[black] (\x) circle (1pt)
				($(\x)+(\g:3mm)$) node {$\x$};
				\foreach \x/\y/\z in {B/A/C,C/A/S}{
					\path pic[draw,angle radius=5pt]{right angle= \x--\y--\z};}
		\end{tikzpicture}}
	
	}
\end{ex}
\begin{ex}%[Nguyễn Văn Hiệp]%[2H1B3-2] % Câu 16.
	Cho hình chóp lục giác đều có cạnh đáy bằng $1$ chiều cao bằng $4$. Thể tích của khối chóp đã cho bằng
	\choice
	{\True $2\sqrt {3}$}
	{$6\sqrt {3}$}
	{$\dfrac{2\sqrt {3}}{3}$}
	{$\dfrac{\sqrt {3}}{3}$}
	\loigiai{
		Giả sử đáy là hình lục giác đều ${ABCDEF}$ tâm $O$.\\
		Ta có diện tích của hình lục giác đều là $S=6\cdot S_{\triangle OAB}=6\cdot \dfrac{1^{2} \sqrt{3}}{4}=\dfrac{3\sqrt {3}}{2}$.\\
		Vậy thể tích của khối chóp là $V=\dfrac{1}{3}\cdot S\cdot h=\dfrac{1}{3}\cdot \dfrac{3\sqrt {3}}{2}\cdot 4=2\sqrt {3}$.
		}
\end{ex}
\begin{ex}%[Nguyễn Văn Hiệp]%[2H1B3-2] % Câu 17.
	Cho khối chóp tam giác đều có cạnh đáy bằng $2$ và chiều cao $h=12$. Thể tích của khối chóp đã cho bằng
	\choice
	{$24\sqrt {3}$}
	{$12\sqrt {3}$}
	{$6\sqrt {3}$}
	{\True $4\sqrt {3}$}
	\loigiai{
		Ta có thể tích của khối chóp tam giác đều là $V=\dfrac {1}{3}\cdot \dfrac {2^2\sqrt {3}}{4}\cdot 12=4\sqrt {3}$.}
\end{ex}
\begin{ex}%[Nguyễn Văn Hiệp]%[2H1B3-2] % Câu 18.
	Cho hình chóp tứ giác đều $S.ABCD$ có đáy là hình vuông cạnh bằng $ a\sqrt {3}$ và các cạnh bên bằng $a\sqrt {2}$. Thể tích khối chóp $S.ABCD$ bằng
	\choice
	{\True $\dfrac {a^3\sqrt {2}}{2}$}
	{$\dfrac {a^3\sqrt {2}}{6}$}
	{$\dfrac {a^3\sqrt {3}}{3}$}
	{$\dfrac {a^3\sqrt {6}}{6}$}
	\loigiai{\immini{Gọi $O$ là tâm của hình vuông $ABCD$.\\
			Vì $S.ABCD$ là hình chóp đều nên $SO\perp (ABCD)$.\\
			Ta có $S_{ABCD}=\left( a\sqrt {3} \right)^2=3a^2$; $OB=\dfrac {1}{2}BD=\dfrac {a\sqrt {6}}{2}$;\\ $SO=\sqrt {SB^2-OB^2}=\sqrt {2a^2-\dfrac {6a^2}{4}}=\dfrac {a\sqrt {2}}{2}$.\\
			Thể tích khối chóp đều $S.ABCD$ là\\
			$V=\dfrac {1}{3}S_{ABCD}\cdot SO=\dfrac {1}{3}\cdot 3a^2\cdot \dfrac {a\sqrt {2}}{2}=\dfrac {a^3\sqrt {2}}{2}$.}{\begin{tikzpicture}[scale=0.8, font=\footnotesize, line join=round, line cap=round,>=stealth,declare function={a=4;h=4.5;}]
				\path 	(0:0) coordinate (A)
				++(0:a) coordinate (D)
				++(-130:3*a/4) coordinate (C)
				($(A)+(C)-(D)$) coordinate (B)
				(intersection of A--C and B--D) coordinate (O)
				($(O)+(90:h)$) coordinate (S)
				($(S)!.5!(B)$) coordinate (M);
				\draw[dashed,thick] 	(B)--(A)--(D)	(A)--(S)
				(A)--(C)	(B)--(D)	(S)--(O)	;
				\draw[thick] 			(B)--(C)--(D)
				(B)--(S)	(C)--(S)	(D)--(S);
				\foreach \x/\g in {A/135,B/-135,C/-45,D/45,S/90,O/-90}
				\fill[black] 	(\x) circle (1.5pt)
				($(\g:3mm)+(\x)$) node {$\x$};	
				\draw pic[draw,angle radius=5pt]{right angle=D--O--S};
				\path (B)--(S) node[midway,sloped,above,xslant=0.1]{$a\sqrt{2}$};
				\path (B)--(C) node[midway,sloped,below]{$a\sqrt{3}$};
		\end{tikzpicture}}}
\end{ex}
\begin{ex}%[Nguyễn Văn Hiệp]%[2H1B3-2] % Câu 19.
	Cho khối chóp tứ giác đều có cạnh đáy bằng $a$, có chiều cao bằng $\dfrac {a\sqrt {2}}{2}$. Thể tích của khối chóp đã cho bằng
	\choice
	{$\dfrac {a^3\sqrt {2}}{3}$}
	{\True $\dfrac {a^3\sqrt {2}}{6}$}
	{$\dfrac {\sqrt {2}a^3}{2}$}
	{$\dfrac {\sqrt {3}a^3}{4}$}
	\loigiai{
		$ h=\dfrac {a\sqrt {2}}{2};S=a^2\Rightarrow V=\dfrac {1}{3}h\cdot S=\dfrac {1}{3}\cdot \dfrac {a\sqrt {2}}{2}\cdot a^2=\dfrac {a^3\sqrt {2}}{6}$.\\
	}
\end{ex}
\begin{ex}%[Nguyễn Văn Hiệp]%[2H1B3-2] % Câu 20.
	Thể tích khối tứ diện đều có cạnh bằng $3$ là
	\choice
	{$\dfrac {4\sqrt {2}}{9}$}
	{$2\sqrt {2}$}
	{\True $\dfrac {9\sqrt {2}}{4}$}
	{$\sqrt {2}$}
	\loigiai{\immini{Cho $ABCD$ là tứ diện đều.\\
			Gọi $F$ là trung điểm $CD$, $G$ là tâm của tam giác đều $BCD$, ta có $AG\perp (BCD)$; $BF=\sqrt {BC^2-CF^2}=\dfrac {3\sqrt {3}}{2}$.\\
			Xét tam giác $ABG$ vuông tại $G$:\\
			$AB=3$, $BG=\dfrac {2}{3}BF=\dfrac {2}{3}\cdot \dfrac {3\sqrt {3}}{2}=\sqrt {3}$,\\
			$AG=\sqrt {AB^2-BG^2}=\sqrt {3^2-\left( \sqrt {3} \right)^2}=\sqrt {6}$.\\
			Ta có $S_{\triangle BCD}=\dfrac {1}{2}BF\cdot CD=\dfrac {3^2\sqrt {3}}{4}=\dfrac {9\sqrt {3}}{4}$.\\
			Vậy $V_{ABCD}=\dfrac {1}{3}AG\cdot S_{\triangle BCD}=\dfrac {1}{3}\cdot \sqrt {6}\cdot \dfrac {9\sqrt {3}}{4}=\dfrac {9\sqrt {2}}{4}$.}{\begin{tikzpicture}[scale=0.8,declare function={r=3;}]
				\path (160:{r} and {r*0.35}) coordinate(B)
				(260:{r} and {r*0.5}) coordinate (C)
				(20:{r} and {r*0.35})coordinate (D)
				(90:{r*1.25}) coordinate (A)
				($(C)!.5!(D)$) coordinate (F)
				($(B)!2/3!0:(F)$) coordinate (G);
				\draw[thick,dash pattern=on 2pt off 2 pt] (B)--(D) (B)--(G)--(F) (A)--(G);
				\draw[thick] (A)--(B)--(C)--(D)--cycle (A)--(C);
				\foreach \x in {A,B,C,D,F,G} \draw[fill=white](\x) circle (1pt);
				\foreach \t/\g in {A/90,B/180,C/-90,D/0,F/-30,G/-90}{
					\path (\t) node[shift={(\g:7pt)},font=\scriptsize]{$ \t $};}
				\foreach \x/\y/\z in {F/G/A,B/F/C}{
					\path pic[draw,angle radius=5pt]{right angle= \x--\y--\z};
				}
		\end{tikzpicture}}
		}
\end{ex}
\Closesolutionfile{ans}
%======================
\subsection{Bảng đáp án}
\inputansbox{8}{ans/ANS-DANG-14}


