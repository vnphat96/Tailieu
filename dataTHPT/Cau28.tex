\setcounter {section} {27}
\setcounter{ex}{0}
\section{Lôgarit}
\subsection{Kiến thức cần nhớ}
\begin{khung}
	\begin{itemize}
		\item Cho hai số dương $a$, $b$ với $a\neq 1$. Số $\alpha$ thỏa mãn đẳng thức $a^\alpha =b$ được gọi là lôgarit cơ số $a$ của $b$ và kí hiệu là $\log_a b$. Ta viết \boxed{\alpha =\log_a b\Leftrightarrow a^\alpha =b}
	\end{itemize} 
	\begin{enumerate}
		\item
		\begin{itemize}
			\item $\log_a a =1$, $\log_a 1=0$
			\item $a^{\log_a b}=b$, $\log_a (a^\alpha)=\alpha$
		\end{itemize}
		\item Lôgarit của một tích: Cho $3$ số dương $a$, $b_1$, $b_2$ với $a\neq 1$, ta có:\\
		$\bullet$ $\log_a (b_1 b_2)=\log_a b_1 +\log_a b_2$
		\item Lôgarit của một thương: Cho $3$ số dương $a$, $b_1$, $b_2$ với $a\neq 1$, ta có:
		\begin{itemize}
			\item $\log_a \dfrac{b_1}{b_2} =\log_a b_1 -\log_a b_2$
			\item  Đặc biệt: với $a$, $b>0$, $a\neq 1$ $\log_a \dfrac{1}{a}=-\log_a b$
		\end{itemize}
		\item Lôgarit của lũy thừa: Cho $a$, $b>0$, $a\neq 1$, với mọi $\alpha$ ta có:
		\begin{itemize}
			\item $\log_a b^\alpha=\alpha\log_a b$
		\end{itemize}
		\item Công thức đổi cơ số: Cho $3$ số dương $a$, $b$, $c$ với $a\neq 1$, $c\neq 1$, ta có:
		\begin{itemize}
			\item $\log_a b=\dfrac{\log_c b}{\log_c a}$
		\end{itemize}
		\item Lôgarit thập phân và Lôgarit tự nhiên
		\begin{itemize}
			\item Lôgarit thập phân là lôgarit cơ số $10$\\
			Viết: $\log_{10}b=\log b=\lg b$
			\item  Lôgarit tự nhiên và lôgarit cơ số $e$\\
			Viết: $\log_e b=\ln b$ với $e \approx 2{,}71828\ldots$
		\end{itemize}
		
	\end{enumerate}   
\end{khung}
\subsection{Bài tập mẫu}
\Opensolutionfile{ans}[ans/ANS-DANG-28]
\begin{khung}
	\begin{vd}[Đề tham khảo BGD 2022-2023]%[2D2Y3-2]
		Với $a$ là số thực dương tùy ý, $\ln (3 a)-\ln (2 a)$ bằng	
		\choice
		{$\ln a$}
		{$\ln \dfrac{2}{3}$}
		{$\ln \left(6 a^{2}\right)$}
		{$\ln \dfrac{3}{2}$}
		\loigiai{
			Ta có $\ln (3 a)-\ln (2 a)=\ln \left(\dfrac{3a}{2a}\right)=\ln \dfrac{3}{2}$.
		}
	\end{vd}
\end{khung}

%%==========Câu 1
\begin{ex}%[Thống Trần]%[2D2Y3-2]
	Với $a$ là số nguyên dương tùy ý, $\log_{\tfrac{1}{2}}a^3$ bằng
	\choice
	{ $-3\log_2 a$ }
	{\True $3-\log_2 a$ }
	{ $\dfrac{3}{2}\log_2 a$ }
	{ $3\log_2 a$ }
	\loigiai
	{
		Ta có $\log_{\tfrac{1}{2}}a^3=3\log_{2^{-1}}a=-3\log_2 a$.
	}
\end{ex}


%%==========Câu 2
\begin{ex}%[Thống Trần]%[2D2Y3-2]
	Với $a$ là số thực dương tùy ý, $\log_3 \sqrt{a}$ bằng
	\choice
	{\True $\dfrac{1}{2}\log_3 a$ }
	{ $\dfrac{1}{2}+\log_3 a$ }
	{ $2\log_3 a$ }
	{ $-\dfrac{1}{2}\log_3 a$ }
	\loigiai
	{
		Ta có $\log_3 \sqrt{a}=\log_3 a^{\tfrac{1}{2}}=\dfrac{1}{2}\log_3 a$.
	}
\end{ex}


%%==========Câu 3
\begin{ex}%[Thống Trần]%[2D2Y3-2]
	Với $a$ là số thực dương tùy ý, $\log_3\left(\dfrac{3}{a}\right)$ bằng
	\choice
	{\True $1-\log_3 a$ }
	{ $3-\log_3 a$ }
	{ $\dfrac{1}{\log_3 a}$ }
	{ $1+\log_3 a$ }
	\loigiai
	{
		Ta có $\log_3\left(\dfrac{3}{a}\right)=\log_3 3-\log_3 a=1-\log_3 a$.	
	}
\end{ex}


%%==========Câu 4
\begin{ex}%[Thống Trần]%[2D2Y3-2]
	Với $a$ là số thực dương khác $1$, giá trị $\log_{a}\left(a^3\sqrt[4]{a}\right)$ bằng
	\choice
	{ $\dfrac{3}{4}$ }
	{ $7$ }
	{ $12$ }
	{\True $\dfrac{13}{4}$ }
	\loigiai
	{
		Ta có $\log_{a}\left(a^3\sqrt[4]{a}\right)=\log_a \left(a^3\cdot a^{\tfrac{1}{4}}\right)=\log_a a^{3+\tfrac{1}{4}}=\log_a a^{\tfrac{13}{4}}=\dfrac{13}{4}\log_a a=\dfrac{13}{4}$.	
	}
\end{ex}


%%==========Câu 5
\begin{ex}%[Thống Trần]%[2D2Y3-2]
	Với mọi $a$, $b$, $x$ là các số thực dương thoả mãn $\log_{2}x=5\log_{2}a+3\log_{2}b$. Mệnh đề nào dưới đây đúng
	\choice
	{$x=5a+3b$}
	{$x=a^5+b^3$}
	{\True $x=a^5b^3$}
	{$x=3a+5b$}
	\loigiai{
		Ta có	$\log_{2}x=5\log_{2}a+3\log_{2}b=\log_{2}a^5+\log_{2}b^3=\log_{2}{\left(a^5b^3\right)}\Rightarrow x=a^5b^3$.
	}
\end{ex}



%%==========Câu 6
\begin{ex}%[Thống Trần]%[2D2Y3-2]
	Cho $0<a\ne 1$. Giá trị của biểu thức $A=\log_{a}{\left(a^3\sqrt{a^7}\right)}$ là
	\choice
	{$3$}
	{$\dfrac{7}{2}$}
	{\True $\dfrac{13}{2}$}
	{$\dfrac{5}{3}$}
	\loigiai{
		Ta có	$P=\log_{a}{\left(a^3\sqrt{a^7}\right)}=\log_{a}{\left(a^3a^{\frac{7}{2}}\right)}=\log_{a}{a^{\frac{13}{2}}}=\dfrac{13}{2}$.
	}
\end{ex}



%%==========Câu 7
\begin{ex}%[Thống Trần]%[2D2Y3-2]
	Cho hai số thực dương $x$, $y>1$ thoả mãn $y=x\sqrt{x}$. Giá trị của $\log_{x}{(x^2y)}$ bằng
	\choice
	{$\dfrac{5}{2}$}
	{$\dfrac{8}{3}$}
	{$3$}
	{\True $\dfrac{7}{2}$}
	\loigiai{
		Ta có	$\log_{x}{(x^2y)}=\log_{x}{x^2}+\log_{x}y=2\log_{x}x+\log_{x}{x\sqrt{x}}=2+\log_{x}{x^{\frac{3}{2}}}=2+\dfrac{3}{2}=\dfrac{7}{2}$.
	}
\end{ex}



%%==========Câu 8
\begin{ex}%[Thống Trần]%[2D2Y3-2]
	Với $a$ là số thực dương tuỳ ý, $\log_{2}{\left(\dfrac{a^2}{4}\right)}$ bằng
	\choice
	{$2(1-\log_{2}a)$}
	{$2\log_{2}a-1$}
	{\True $2(\log_{2}a-1)$}
	{$2(\log_{2}a+1)$}
	\loigiai{
		Ta có $\log_{2}{\left(\dfrac{a^2}{4}\right)}=2\log_{2}a-2=2(\log_{2}a-1)$.
	}
\end{ex}



%%==========Câu 9
\begin{ex}%[Thống Trần]%[2D2Y3-2]
	Cho $a$ là số thực dương $a \ne 1$ và $\log_{\sqrt[3]{a}}{a^3}$. Mệnh đề nào sau đây đúng?
	\choice
	{$P=1$}
	{\True $P=9$}
	{$P=\dfrac{1}{3}$}
	{$P=3$}
	\loigiai{
		Ta có	$\log_{\sqrt[3]{a}}{a^3}=\log_{a^\frac{1}{3}}{a^3}=9$.
	}
\end{ex}



%%==========Câu 10
\begin{ex}%[Thống Trần]%[2D2Y3-2]
	Với $a$, $b$ là hai số thực dương tuỳ ý, $\log_{3}{(a^3\sqrt{b})}$ bằng
	\choice
	{\True $3\log_{3}a+\dfrac{1}{2}\log_{3}b$}
	{$3\log_{3}a+2\log_{3}b$}
	{$\dfrac{3}{2}\log_{3}{(ab)}$}
	{$\dfrac{3}{2}\log_{3}{(a+b)}$}
	\loigiai{
		Ta có $\log_{3}{(a^3\sqrt{b})}=\log_{3}{a^3}+\log_{3}{\sqrt{b}}=3\log_{3}a+\dfrac{1}{2}\log_{3}b$.
	}
\end{ex}


%%==========Câu 11
\begin{ex}%[Thống Trần]%[2D2Y3-2]
	Xét tất cả các số thực dương $a$ và $b$ thỏa mãn $\log_3 a=\log_{27}\left(a^2\sqrt{b}\right)$. Mệnh đề nào dưới đây \textbf{đúng}?
	\choice 
	{ $a=b$}
	{ \True $a^2=b$}
	{ $a=b^2$}
	{ ${{a}^3}=b$
	}
	\loigiai{
		Ta có\\ $\begin{aligned}
		& 	\log_3 a=\log_{27}\left(a^2\sqrt{b}\right) \Leftrightarrow \log_3 a=\dfrac{1}{3}\log_3\left(a^2\sqrt{b}\right)\Leftrightarrow 3\log_3 a=\log_3\left(a^2\sqrt{b}\right)\\
		\Leftrightarrow& \log_3 a^3=\log_3\left(a^2\sqrt{b}\right)\Leftrightarrow a=\sqrt{b}\Leftrightarrow a^2=b.
		\end{aligned}$
		
} \end{ex}


%%==========Câu 12
\begin{ex}%[Thống Trần]%[2D2Y3-2]	
	Với $a$ là số thực dương tùy ý, $\log_2 a^2$ bằng
	\choice 
	{ $\dfrac{1}{2}\log_2 a$}
	{ $2+\log_2 a$}
	{ \True $2\log_2 a$}
	{ $\dfrac{1}{2}+\log_2 a$}
	\loigiai{
		Vì $a$ là số thực dương tùy ý nên $\log_2 a^2=2\log_2 a$.
	} 
\end{ex}



%%==========Câu 13
\begin{ex}%[Thống Trần]%[2D2Y3-2]
	Với $a$ là số thực dương tùy ý, $\log_{\tfrac{1}{3}}(9a^2)$ bằng
	\choice 
	{ \True $-2-2\log_3 a$}
	{ $-2-2\log_{\tfrac{1}{3}} a$} 
	{ $2+2\log_{\tfrac{1}{3}} a$}
	{ $2+2\log_3 a$}
	\loigiai{
		Ta có $\log_{\tfrac{1}{3}}(9a^2)=-\log_3 (9a^2)=-\left(\log_3 9+\log_3 a^2\right)=-(2+2\log_3 a)=-2-2\log_3 a$.
	} 
\end{ex}



%%==========Câu 14
\begin{ex}%[Thống Trần]%[2D2Y3-2] 
	Cho $0<a\ne 1$. Giá trị của biểu thức $P=\log_a\left(a\cdot \sqrt[3]{a^2}\right)$ là
	\choice 
	{ $3$}
	{\True $\dfrac{5}{3}$}
	{ $\dfrac{5}{2}$}
	{ $\dfrac{4}{3}$}
	\loigiai{
		Ta có $P=\log_a\left(a\cdot \sqrt[3]{a^2}\right)=\log_a a+\log_a \sqrt[3]{a^2}=1+\log_a a^{\tfrac{2}{3}}=1+\dfrac{2}{3}=\dfrac{5}{3}$.
	} 
\end{ex}



%%==========Câu 15
\begin{ex}%[Thống Trần]%[2D2Y3-2] 
	Cho $a$ là số thực dương tùy ý. Giá trị của $\log_2(4a^2)$ bằng
	\choice 
	{ $4+\dfrac{1}{2}\log_2 a$}
	{\True $2( \log_2 a+1 )$}
	{ $2+\log_2 a$}
	{ $8\log_2 a$}
	\loigiai{
		Ta có $\log_2(4a^2)=\log_2 4+\log_2 a^2=2+2\log_2 a=2(1+\log_2 a)$.
	} 
\end{ex}



%%==========Câu 16
\begin{ex}%[Thống Trần]%[2D2Y3-2] 
	Cho $a$ là số thực dương tùy ý, $\ln \left(\dfrac {\mathrm{e}}{a^2}\right)$ bằng
	\choice 
	{ $1+2\ln a$}
	{\True $1-2\ln a$}
	{ $1+\ln ( 2a )$}
	{ $1-\ln ( 2a )$}
	\loigiai{
		Với $a>0$, ta có: $\ln \left(\dfrac {\mathrm{e}}{a^2}\right) =\ln \mathrm{e}-\ln a^2=1-2\ln a$.} 
\end{ex}



%%==========Câu 17
\begin{ex}%[Thống Trần]%[2D2Y3-2] 
	Cho $a$ và $b$ là hai số thực dương thỏa mãn $\sqrt{a}\cdot b^3=27$. Giá trị của $\log_{3} a+6\log_{3}b$ bằng
	\choice 
	{ $3$}
	{\True $6$}
	{ $9$}
	{ $1$}
	\loigiai{ Ta có $\sqrt{a}\cdot b^3=27\Leftrightarrow \sqrt{a}=\left(  \dfrac{3}{b} \right)^3 \Rightarrow \log _{3}\sqrt{a}=\log_3\left(\dfrac{3}{b}\right)^3$\\
	$\Rightarrow \dfrac{1}{2}\log_3 a=3( 1-\log_3 b)\Rightarrow \log_3 a+3\log_3 b=6$		
	} 
\end{ex}



%%==========Câu 18
\begin{ex}%[Thống Trần]%[2D2Y3-2] 
	Giá trị của $\log_{2}( 4\sqrt{2} )$ bằng
	\choice 
	{\True $\dfrac{5}{2}$}
	{ $4$}
	{ $3$}
	{ $\dfrac{3}{2}$}
	\loigiai{
		Ta có $\log_{2}( 4\sqrt{2} )=\log_2\left(2^2\cdot 2^{\tfrac{1}{2}}\right)=\log_2 2^{\tfrac{5}{2}}=\dfrac{5}{2}$.
	} 
\end{ex}


%%==========Câu 19
\begin{ex}%[Thống Trần]%[2D2Y3-2] 
	Với $a$ là số thực dương khác $1$, $\log_{a^2}\left(a\sqrt{a}\right)$ bằng
	\choice 
	{ $\dfrac{1}{4}$}
	{ \True $\dfrac{3}{4}$}
	{ $3$}
	{ $\dfrac{3}{2}$}
	\loigiai{
		Ta có $\log_{a^2}\left(a\sqrt{a}\right)=\log_{a^2}a^{\tfrac{3}{2}}=\dfrac{3}{2}\cdot \dfrac{1}{2}=\dfrac{3}{4}\cdot$} 
\end{ex}


%%==========Câu 20
\begin{ex}%[Thống Trần]%[2D2Y3-2] 
	Cho $a,\,\,b$ là các số thực dương và $a$ khác $1$, thỏa mãn $\log_{a^3}\left(\dfrac{a^5}{\sqrt[4]{b}}\right)=2$. Giá trị của biểu thức ${{\log }_{a}}b$ bằng
	\choice 
	{ $\dfrac{1}{4}$}
	{ $-\dfrac{1}{4}$}
	{ $4$ }
	{\True $-4$}
	\loigiai{
		Xét $\log_{a^3}\left(\dfrac{a^5}{\sqrt[4]{b}}\right)=2 \Leftrightarrow \log_{a^3} a^5-\log_{a^3} b^{\tfrac{1}{4}}=2\Leftrightarrow \dfrac{5}{3}-\dfrac{1}{12}\log_{a} b=2\Leftrightarrow \log_{a} b=-4$.
	} 
\end{ex}
\Closesolutionfile{ans}

\subsection{Bảng đáp án}
\inputansbox{8}{ans/ANS-DANG-28}	