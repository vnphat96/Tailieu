%\setcounter{section}{36}
\setcounter{ex}{0}
\section{Điểm đối xứng, hình chiếu của 1 điểm}
\subsection{Kiến thức cần nhớ}
\begin{khung}
	\subsubsection{Nhóm bài toán liên quan đến hình chiếu, điểm đối xứng của điểm lên trục, lên mặt phẳng tọa độ}
	\begin{enumerate}
	\item \textbf{\underline{Hình chiếu}}: \lq\lq \textbf{Thiếu cái nào, cho cái đó bằng 0}\rq\rq. Nghĩa là hình chiếu của $M(a;b;c)$ lên:
	\begin{listEX}[3]
		\item[$\bullet$] $Ox$ là $M_1(a;0;0)$.
		\item[$\bullet$] $Oy$ là $M_2(0;b;0)$.
		\item[$\bullet$] $Oz$ là $M_3(0;0;c)$.
		\item[$\bullet$] $(Oxy)$ là $M_4(a;b;0)$.
		\item[$\bullet$] $(Oxz)$ là $M_5(a;0;c)$.
		\item[$\bullet$] $(Oyz)$ là $M_6(0;b;c)$.
	\end{listEX}
	\item \textbf{\underline{Đối xứng}}: \lq\lq \textbf{Thiếu cái nào, đổi dấu cái đó}\rq\rq. Nghĩa là điểm đối xứng của $N(a;b;c)$ qua:
	\begin{listEX}[3]
		\item[$\bullet$] $Ox$ là $N_1(a;-b;-c)$.
		\item[$\bullet$] $Oy$ là $N_2(-a;b;-c)$.
		\item[$\bullet$] $Oz$ là $N_3(-a;-b;c)$.
		\item[$\bullet$] $(Oxy)$ là $N_4(a;b;-c)$.
		\item[$\bullet$] $(Oxz)$ là $N_5(a;-b;c)$.
		\item[$\bullet$] $(Oyz)$ là $N_6(-a;b;c)$.
	\end{listEX}
	\item \textbf{\underline{Khoảng cách}}: Để tìm khoảng cách từ điểm $M$ đến trục (hoặc mặt phẳng tọa độ), ta tìm \textbf{hình chiếu} $H$ của điểm $M$ lên trục (hoặc mặt phẳng tọa độ), từ đó suy ra \textbf{khoảng cách cần tìm là} $d = MH$.
\end{enumerate}
\subsubsection{Điểm thuộc, không thuộc mặt phẳng}
	\begin{enumerate}
	\item Phương trình tổng quát của mặt phẳng: $\fbox{$(\alpha): A x+B y+C z+D=0$}$\\
	(với $A^{2}+B^{2}+C^{2} \neq 0 ;(\alpha)$ có VTPT là $\vec{n}=(A ; B ; C)$)
	\item 
	Điểm $M(x_0;y_0;z_0)\in (P)\colon Ax+By+Cz+D=0\Leftrightarrow Ax_0+By_0+Cz_0+D=0$.
	\item Điểm $M(x_0;y_0;z_0)\notin (P)\colon Ax+By+Cz+D=0\Leftrightarrow Ax_0+By_0+Cz_0+D\ne 0$.
\end{enumerate}
\end{khung}

\subsection{Bài tập mẫu}
\Opensolutionfile{ans}[ans/ANS-DANG-37]
\begin{khung}
%	\setcounter{vd}{36}
		\begin{vd}[Đề minh họa BGD năm 2022 - 2023]%[Nguyễn Văn Hồng-Dự án 50 CĐ-PTMH 2023]%[2H3B2-4]
		Trong không gian $O x y z$, cho điểm $A(1; 2; 3)$. Điểm đối xứng với $A$ qua mặt phẳng $(O x z)$ có tọa độ là
		\choice
		{\True $(1;-2; 3)$}
		{$(1; 2;-3)$}
		{$(-1;-2;-3)$}
		{$(-1; 2; 3)$}
			\loigiai{
			Gọi $H$ là hình chiếu của $A$ trên $(Oxz)$, suy ra $H(1;0;3)$.\\
			Điểm $A'$ đối xứng với $A$ qua mặt phẳng $(Oxz)$ thì $AA'$ nhận $H$ làm trung điểm.\\
			 Vậy $A'(1;-2;3)$.}
	\end{vd}

\end{khung}
\subsection{Bài tập tương tự và phát triển}
\setcounter{ex}{0}
\begin{ex}%[Nguyễn Văn Hồng-Dự án 50 CĐ-PTMH 2023]%[2H3Y1-1]
	Cho điểm $A(3;-1;1)$. Hình chiếu vuông góc của $A$ trên $(Oyz)$ là điểm
	\choice{$M(3;0;0)$}{\True$N(0;-1;1)$}{$P(0;-1;0)$}{$Q(0;0;1)$}
	\loigiai{
		Hình chiếu vuông góc của $A$ trên $(Oyz)$ là điểm $N(0;-1;1)$.
		
	}
\end{ex}
%VD2
\begin{ex}%[Nguyễn Văn Hồng-Dự án 50 CĐ-PTMH 2023]%[2H3Y1-1]
	Trong KG $Oxyz$, tìm tọa độ điểm $H$ là hình chiếu của $M(1;2;-4)$ lên $(Oxy)$.
	\choice{$H(1;2;-4)$}{$H(0;2;-4)$}{$H(1;0;-4)$}{\True$H(1;2;0)$}
	\loigiai{
		Hình chiếu vuông góc của $M$ trên $(Oxy)$ là điểm $H(1;2;0)$.
	}
\end{ex}
%VD3
\begin{ex}%[Nguyễn Văn Hồng-Dự án 50 CĐ-PTMH 2023]%[2H3Y1-1]
	Hình chiếu vuông góc của $A(3;-1;1)$ trên $(Oxz)$ là $A'(x;y;z)$. Khi đó $x-y-z$ bằng
	\choice{$-4$}{\True$2$}{$4$}{$3$}
	\loigiai{
		Hình chiếu vuông góc của $A(3;-1;1)$ trên $(Oxz)$ là $A'(3;0;1)$. Vậy $x - y -z = 2$.
	}
\end{ex}
%VD4
\begin{ex}%[Nguyễn Văn Hồng-Dự án 50 CĐ-PTMH 2023]%[2H3Y1-1]
	Trong không gian với hệ trục tọa độ  $Oxyz$, tìm tọa độ điểm $H$ là hình chiếu của $M(4;5;6)$ lên trục $Ox$.
	\choice{$H(0;5;6)$}{\True$H(4;0;0)$}{$H(0;0;6)$}{$H(4;5;0)$}
	\loigiai{
		Tọa độ điểm $H$ là hình chiếu của $M(4;5;6)$ lên trục $Ox$, $H(4;0;0)$.
	}
\end{ex}
%VD5
\begin{ex}%[Nguyễn Văn Hồng-Dự án 50 CĐ-PTMH 2023]%[2H3Y1-1]
	Trong không gian với hệ trục tọa độ  $Oxyz$, tìm tọa độ điểm $H$ là hình chiếu của $M(1;-1;2)$ lên trục $Oy$.
	\choice{\True$H(0;-1;0)$}{$H(1;0;0)$}{$H(0;0;2)$}{$H(0;1;0)$}
	\loigiai{
		Tọa độ hình chiếu của $M(1;-1;2)$ lên trục $Oy$ là điểm $H(0;-1;0)$.
	}
\end{ex}

%VD7
\begin{ex}%[Nguyễn Văn Hồng-Dự án 50 CĐ-PTMH 2023]%[2H3B1-1]
	Tìm tọa độ $M'$ là điểm đối xứng của điểm $M(1;2;3)$ qua gốc tọa độ $O$.
	\choice{$M'(-1;2;3)$}{$M'(-1;-2;3)$}{\True$M'(-1;-2;-3)$}{$M'(1;2;-3)$}
	\loigiai{
		Tọa độ điểm đối xứng của $M(1;2;3)$ qua $O$
		là điểm $M'(-1;-2;-3)$.
	}
\end{ex}
%VD8
\begin{ex}%[Nguyễn Văn Hồng-Dự án 50 CĐ-PTMH 2023]%[2H3B1-1]
	Tìm $M'$ là điểm đối xứng của $M(1;-2;0)$ qua điểm $A(2;1;-1)$.
	\choice{$M'(1;3;-1)$}{$M'(3;-3;1)$}{$M'(0;-5;1)$}{\True$M'(3;4;-2)$}
	\loigiai{
		$M'$ là điểm đối xứng của $M$ qua điểm $A \Leftrightarrow A$ là trung điểm $MM'$.\\
		Khi đó, tọa độ $M'$ thỏa hệ $\heva{&x_{M'}=2x_A-x_M\\&y_{M'} = 2y_A-y_M\\&z_{M'} = 2z_{A} - z_M}\Leftrightarrow \heva{&x_{M'}= 2\cdot 2 -1 = 3\\&y_{M'}= 2\cdot 1 + 2 =4\\&z_{M'} = 2\cdot (-1) - 0 = -2.}$
		\\Vậy $M'(3;4;-2)$.
	}
\end{ex}
%VD9
\begin{ex}%[Nguyễn Văn Hồng-Dự án 50 CĐ-PTMH 2023]%[2H3Y1-1]
	Tìm tọa độ $M'$ là điểm đối xứng của điểm $M(3;2;1)$ qua trục $Ox$.
	\choice{\True$M'(3;-2;-1)$}{$M'(-3;2;1)$}{$M'(-3;-2;-1)$}{$M'(3;-2;1)$}
	\loigiai{
		Tọa độ điểm đối xứng của $M(3;2;1)$ qua trục $Ox$
		là điểm $M'(3;-2;-1)$.
	}
\end{ex}
\begin{ex}%[Nguyễn Văn Hồng-Dự án 50 CĐ-PTMH 2023]%[2H3B1-1]
	Trong KG $Oxyz$, điểm $N$ đối xứng với điểm $M(3;-1;2)$ qua trục $Oy$ là
	\choice
	{$N(-3;1;-2)$}
	{$N(3;1;-2)$}
	{\True $N(-3;-1;-2)$}
	{$N(3;-1;-2)$}
	\loigiai{
 Tọa độ điểm đối xứng với điểm $M(3;-1;2)$ qua trục $Oy$ là $N(-3;-1;-2)$.
	}
\end{ex}
%VD10
\begin{ex}%[Nguyễn Văn Hồng-Dự án 50 CĐ-PTMH 2023]%[2H3Y1-1]
	Tìm tọa độ $M'$ là điểm đối xứng của điểm $M(2;3;4)$ qua trục $Oz$.
	\choice{$M'(2;-3;-4)$}{$M'(-2;3;4)$}{\True$M'(-2;-3;4)$}{$M'(2;-3;4)$}
	\loigiai{
		Tọa độ điểm đối xứng của $M(2;3;4)$ qua trục $Oz$
		là điểm $M'(-2;-3;4)$.
	}
\end{ex}
%VD11
\begin{ex}%[Nguyễn Văn Hồng-Dự án 50 CĐ-PTMH 2023]%[2H3Y1-1]
	Tìm tọa độ $M'$ là điểm đối xứng của điểm $M(1;2;5)$ qua $(Oxy)$.
	\choice{$M'(-1;-2;5)$}{$M'(1;2;0)$}{\True$M'(1;-2;5)$}{$M'(1;2;-5)$}
	\loigiai{
		Tọa độ điểm đối xứng của $M(1;2;5)$ qua  $(Oxy)$
		là điểm $M'(1;2;-5)$.
	}
\end{ex}
%VD12
\begin{ex}%[Nguyễn Văn Hồng-Dự án 50 CĐ-PTMH 2023]%[2H3B1-1]
	Trong KG $Oxyz$, điểm đối xứng với điểm $A(-2;7;5)$ qua mặt phẳng $(Oxz)$ là điểm $B$ có tọa độ là
	\choice
	{$B(2;7;-5)$}
	{\True$B(-2;-7;5)$}
	{$B(-2;7;-5)$}
	{$B(2;-7;-5)$}
	\loigiai{
	Điểm đối xứng với điểm $A(-2;7;5)$ qua mặt phẳng $(Oxz)$ là điểm $B(-2;-7;5)$.
	}
\end{ex}
\begin{ex}%[Nguyễn Văn Hồng-Dự án 50 CĐ-PTMH 2023]%[2H3Y1-1]
	Tìm tọa độ $M'$ là điểm đối xứng của điểm $M(1;-2;3)$ qua  $(Oyz)$.
	\choice{\True$M'(-1;-2;3)$}{$M'(1;2;-3)$}{$M'(-1;2;-3)$}{$M'(0;-2;3)$}
	\loigiai{
		Tọa độ điểm đối xứng của $M(1;-2;3)$ qua  $(Oyz)$
		là điểm $M'(-1;-2;3)$.
	}
\end{ex}


%VD15
\begin{ex}%[Nguyễn Văn Hồng-Dự án 50 CĐ-PTMH 2023]%[2H3B1-1]
	Tính khoảng cách $\mathrm{d}$ từ điểm $M(1;-2;-3)$ đến $(Oxz)$.
	\choice{$\mathrm{d}=1$}{\True$\mathrm{d}=2$}{$\mathrm{d}=3$}{$\mathrm{d}=4$}
	\loigiai{
		Hình chiếu của $M$ lên  $(Oxz)$ là $M'(1;0;-3)$.\\
		Khi đó khoảng cách từ điểm $M$ đến trục hoành $Ox$ là độ dài đoạn thẳng $$MM' = \sqrt{(1-1)^2+(0+2)^2+(-3+3)^2} = 2.$$ Vậy $\mathrm{d} = 2.$
	}
\end{ex}
%VD16
\begin{ex}%[Nguyễn Văn Hồng-Dự án 50 CĐ-PTMH 2023]%[2H3B1-1]
	Trong KG $Oxyz$, hãy tính khoảng cách từ điểm $M(-3;2;4)$ đến trục $Oy$.
	\choice{$d=2$}{$d=3$}{$d=4$}{\True$d=5$}
	\loigiai{
		Hình chiếu của $M$ lên trục $Oy$ là $M'(0;2;0)$.\\
		Khi đó khoảng cách từ điểm $M$ đến trục $Ox$ là độ dài đoạn thẳng $$MM' = \sqrt{(0+3)^2+(2-2)^2+(0-4)^2} = 5.$$
	}
\end{ex}
\begin{ex}%[Nguyễn Văn Hồng-Dự án 50 CĐ-PTMH 2023]%[2H3Y2-4]
	Cho mặt phẳng $(P)\colon x-2y+z=5$. Điểm nào dưới đây thuộc $(P)$?
	\choice
	{$Q(2;-1;5)$}
	{$P(0;0;-5)$}
	{$N(-5;0;0)$}
	{\True $M(1;1;6)$}
	\loigiai{
		Thế tọa độ của $M(1;1;6)$ vào phương trình của $(P)$ ta được: $1-2\cdot 1+6=5$.\\
		Vậy điểm $M(1;1;6)\in(P)$.
	}
\end{ex}
\begin{ex}%[Nguyễn Văn Hồng-Dự án 50 CĐ-PTMH 2023]%[2H3B2-4]
	Tìm $m$ để điểm $M(m;1;6)$ thuộc mặt phẳng $(P)\colon x-2y+z-5=0$.
	\choice
	{\True $m=1$}
	{$m=-1$}
	{$m=3$}
	{$m=2$}
	\loigiai{
		Điểm $M(m;1;5)\in(P)\Leftrightarrow m-2\cdot 1+6-5=0\Leftrightarrow m=1$.
	}
\end{ex}


\begin{ex}%[Nguyễn Văn Hồng-Dự án 50 CĐ-PTMH 2023]%[2H3Y2-4]
	Trong không gian  $Oxyz$, cho mặt phẳng $(\alpha):x+y+z-6=0$. Điểm nào dưới đây \textbf{không} thuộc $(\alpha)$?
	\choice
	{$N(2;2;2)$}
	{$Q(3;3;0)$}
	{$P(1;2;3)$}
	{\True $M(1;-1;1)$}
	\loigiai{
	
		Thay tọa độ của $M(1;-1;1)$ vào phương trình mặt phẳng $(\alpha)$ ta được $$1-1+1-6=-5\neq 0\Rightarrow M\notin (\alpha).$$
	}
\end{ex}

\begin{ex}%[Nguyễn Văn Hồng-Dự án 50 CĐ-PTMH 2023]%[2H3Y2-3]
	Trong KG $Oxyz$, mặt phẳng nào dưới đây đi qua điểm $M(1;-2;1)$?
	\choice
	{\True $\left(P_1\right)\colon x+y+z=0$}	
	{$\left(P_2\right)\colon x+y+z-1=0$}
	{$\left(P_3\right)\colon x-2y+z=0$}
	{$\left(P_4\right)\colon x+2y+z-1=0$}
	\loigiai{
	
		Thay tọa độ điểm $M(1;-2;1)$ vào từng phương trình mặt phẳng để kiểm tra.\\
		Ta thấy $1+(-2)+1=0$ nên $M\in\left(P_1\right)$.}
\end{ex}

\begin{ex}%[Nguyễn Văn Hồng-Dự án 50 CĐ-PTMH 2023]%[2H3Y2-4]
	Trong KG $Oxyz$, điểm nào dưới đây thuộc mặt phẳng $(Oxy)?$
	\choice
	{$Q(3;-1;3)$}
	{$N(3;-1;2)$}
	{\True $M(2;2;0)$}
	{$P(0;0;-2)$}
	\loigiai{
	
		Điểm thuộc mặt phẳng $(Oxy)$ có cao độ bằng $0$.\\
		Vậy $M\in (Oxy)$.}
\end{ex}
\Closesolutionfile{ans}
%======================
\subsection{Bảng đáp án}
\inputansbox{8}{ans/ANS-DANG-37}
