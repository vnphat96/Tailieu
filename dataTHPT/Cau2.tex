%Dạng 1
\setcounter {section} {1}
\setcounter{ex}{0}
\section{Hàm số logarit}
\subsection{Kiến thức cần nhớ}
\begin{khung}
	\subsubsection{Hàm số logarit}
		Với $a$ là số thực dương khác $1$.
		\begin{itemize}
			\item Hàm số logarit cho bởi công thức: $y=\log_a{x}$.
			\item Tập xác định: $\mathscr{D}=(0;+\infty)$.
			\item Với hàm số $y=\log_a{u(x)}$ thì điều kiện xác định là $u(x)>0$.
		\end{itemize}
	\subsubsection{Đạo hàm của hàm số logarit}
		\begin{itemize}
			\item Với $y=\ln x$ thì $y'=\dfrac{1}{x}$.
			\item Với $y=\log_ax$ thì $y'=\dfrac{1}{x\ln a}$.
			\item Với $y=\ln u(x)$ thì $y'=\dfrac{u'(x)}{u(x)}$.
			\item Hàm số hợp $y=\log_a\left[u(x)\right]$ thì $y'=\dfrac{u'(x)}{u(x)\ln a}$.
			\item Với $y=\log_a\left|u(x)\right|$ thì $y'=\dfrac{u'(x)}{u(x)\ln a}$.
		\end{itemize}
			
	\subsubsection{Sự biến thiên của hàm số logarit}
		\begin{itemize}
			\item Với $a>1$ thì hàm số $y=\log_ax$ đồng biến trên $(0;+\infty)$.
			\item Với $0<a<1$ thì hàm số $y=\log_ax$ nghịch biến trên $(0;+\infty)$.
		\end{itemize}
	
	\subsubsection{Đồ thị của hàm số logarit}
		\begin{minipage}[b]{0.5\textwidth}
			\centerline{$a>1$}
			\centerline{
				\begin{tikzpicture}[scale=1, font=\footnotesize, line join=round, line cap=round, >=stealth]
					\def\f(#1){ln(#1)}%Thay đổi hàm số tại đây
					\def\xmin{0.1}%Giá trị nhỏ nhất trên Ox
					\def\xmax{3}%Giá trị lớn nhất trên Ox
					\def\ymin{-2.5}%Giá trị nhỏ nhất trên Oy
					\def\ymax{1.5}%Giá trị lớn nhất trên Oy
					%	\draw[gray!30,opacity=.5] (\xmin,\ymin) grid (\xmax,\ymax);
					\draw[->] (\xmin-0.2,0)--(\xmax+0.2,0) node[below] {\footnotesize $x$};
					\draw[->] (0,\ymin-0.2)--(0,\ymax+0.2) node[right] {\footnotesize $y$};
					\draw (0,0) node [below left] {\footnotesize $O$};
					\foreach \x in {1,2,3}\draw (\x,0.1)--(\x,-0.1) node [below] {\footnotesize $\x$};
					\foreach \y in {-2,-1,1}\draw (0.1,\y)--(-0.1,\y) node [left] {\footnotesize $\y$};
					\clip (\xmin,\ymin) rectangle (\xmax,\ymax);
					\draw[smooth,samples=200,domain=\xmin:\xmax] plot (\x,{\f(\x)});
			\end{tikzpicture}}
		\end{minipage}
		\begin{minipage}[b]{0.5\textwidth}
			\centerline{$0<a<1$}
			\centerline{
				\begin{tikzpicture}[scale=1, font=\footnotesize, line join=round, line cap=round, >=stealth]
					\def\f(#1){-ln(#1)}%Thay đổi hàm số tại đây
					\def\xmin{0.1}%Giá trị nhỏ nhất trên Ox
					\def\xmax{3}%Giá trị lớn nhất trên Ox
					\def\ymin{-1.5}%Giá trị nhỏ nhất trên Oy
					\def\ymax{2.5}%Giá trị lớn nhất trên Oy
					%	\draw[gray!30,opacity=.5] (\xmin,\ymin) grid (\xmax,\ymax);
					\draw[->] (\xmin-0.2,0)--(\xmax+0.2,0) node[below] {\footnotesize $x$};
					\draw[->] (0,\ymin-0.2)--(0,\ymax+0.2) node[right] {\footnotesize $y$};
					\draw (0,0) node [below left] {\footnotesize $O$};
					\foreach \x in {1,2,3}\draw (\x,0.1)--(\x,-0.1) node [below] {\footnotesize $\x$};
					\foreach \y in {-1,1,2}\draw (0.1,\y)--(-0.1,\y) node [left] {\footnotesize $\y$};
					\clip (\xmin,\ymin) rectangle (\xmax,\ymax);
					\draw[smooth,samples=200,domain=\xmin:\xmax] plot (\x,{\f(\x)});
			\end{tikzpicture}}
		\end{minipage}
\end{khung}
\subsection{Bài tập mẫu}
\Opensolutionfile{ans}[ans/ANS-DANG-2]
\begin{khung}
	\begin{vd}[Đề minh họa BGD 2022-2023]%[Nguyễn Giang, ĐMH]%[2D2Y4-2]
		Trên khoảng $(0; +\infty)$, đạo hàm của hàm số $y=\log_3 x$ là
		\choice
		{$y'=\dfrac{1}{x}$}
		{\True $y'=\dfrac{1}{x\ln 3}$}
		{$y'=\dfrac{\ln 2}x$}
		{$y'=-\dfrac{1}{x\ln 3}$}
		\loigiai{
			Đạo hàm của hàm số $y=\log_3 x$ là $y'=\dfrac{1}{x\ln 3}$.
		}
	\end{vd}
\end{khung}
\subsection{Bài tập tương tự và phát triển}
\begin{ex}%[Nguyễn Giang]%[2D2Y4-2]
	Tính đạo hàm của hàm số $y=\log_3(3x+1)$.
	\choice
	{\True $y'=\dfrac{3}{(3x+1)\ln {3}}$}
	{$y'=\dfrac{1}{(3x+1)\ln {3}}$}
	{$y'=\dfrac{3}{3x+1}$}
	{$y'=\dfrac{1}{3x+1}$}
	\loigiai{
		Ta có $y=\log_3(3x+1)\Rightarrow y'=\dfrac{3}{(3x+1)\ln {3}}$.
	}
\end{ex}

\begin{ex}%[2D2Y4-2]
	Đạo hàm của hàm số $y=\log_3\left(1-2x\right)$ là
	\choice
	{$y'=\dfrac{2}{(1-2x)\ln {3}}$}
	{$y'=\dfrac{1}{(1-2x)\ln {3}}$}
	{\True $y'=\dfrac{-2}{(1-2x)\ln {3}}$}
	{$y'=\dfrac{-2\ln {3}}{1-2x}$}
	\loigiai{
		Với $x\in\left(-\infty; \dfrac{1}{2}\right)$, ta có $y'=\dfrac{(1-2x)'}{(1-2x)\ln 3}=\dfrac{-2}{(1-2x)\ln {3}}$.
	}
\end{ex}

\begin{ex}%[Nguyễn Giang]%[2D2Y4-2]
	Đạo hàm của hàm số $y=\log_3(2-x)$ là
	\choice
	{\True $y'=\dfrac{1}{(x-2)\ln 3}$}
	{$y'=\dfrac{\ln 3}{2-x}$}
	{$y'=\dfrac{1}{(2-x)\ln 3}$}
	{$y'=\dfrac{\ln 3}{x-2}$}
	\loigiai{
		Ta có $y'=\dfrac{(2-x)'}{(2-x)\ln 3}=\dfrac{1}{(x-2)\ln 3}$.
	}
\end{ex}

\begin{ex}%[Nguyễn Giang]%[2D2Y4-2]
	Tính đạo hàm của hàm số $y=\log_3(2x+1)$.
	\choice
	{$y'=\dfrac{1}{2x+1}$}
	{\True $y'=\dfrac{2}{(2x+1)\ln 3}$}
	{$y'=(2x+1)\cdot \ln 3$}
	{$y'=\dfrac{1}{(2x+1)\ln 3}$}
	\loigiai{
		Đạo hàm của hàm số $y=\log_3(2x+1)$ là $y'=\dfrac{2}{(2x+1)\ln 3}$.
	}
\end{ex}

\begin{ex}%[Nguyễn Giang]%[2D2Y4-2]
	Tính đạo hàm của hàm số $y=\log_3(3x+2)$.
	\choice
	{$y'=\dfrac{1}{(3x+2)}$}
	{$y'=\dfrac{3}{(3x+2)}$}
	{\True $y'=\dfrac{3}{(3x+2)\ln 3}$}
	{$y'=\dfrac{1}{(3x+2)\ln 3}$}
	\loigiai{
		Ta có $y'=\dfrac{3}{(3x+2)\ln 3}$.
	}
\end{ex}

\begin{ex}%[Nguyễn Giang]%[2D2Y4-2]
	Đạo hàm của hàm số $y=\ln \left(x^2+x+1\right)$ là hàm số nào sau đây?
	\choice
	{$y'=\dfrac{-(2x+1)}{x^2+x+1}$}
	{$y'=\dfrac{-1}{x^2+x+1}$}
	{\True $y'=\dfrac{2x+1}{x^2+x+1}$}
	{$y'=\dfrac{1}{x^2+x+1}$}
	\loigiai{
		Ta có $y'=\dfrac{\left(x^2+x+1\right)^\prime}{x^2+x+1}=\dfrac{2x+1}{x^2+x+1}$.}
\end{ex}

\begin{ex}%[Nguyễn Giang]%[2D2Y4-2]
	Đạo hàm của hàm số $y=\,x\,+\ln ^2x\,$ là hàm số nào dưới đây?
	\choice
	{$y'=1+2x\ln x$}
	{$y'=1+2\ln x$}
	{$y'=1+\dfrac{2}{x\ln x}$}
	{\True $y'=1+\dfrac{2\ln x}{x}$}
	\loigiai{
		Ta có $y'=1+2\ln x\cdot \left(\ln x\right)^\prime=1+\dfrac{2\ln x}{x}$.}
\end{ex}

\begin{ex}%[Nguyễn Giang]%[2D2Y4-2]
	Tính đạo hàm của hàm số $y=\log x$.
	\choice
	{$y'=\dfrac{x}{\ln 10}$}
	{$y'=\dfrac{\ln 10}{x}$}
	{\True $y'=\dfrac{1}{x\ln 10}$}
	{$y'=\dfrac{1}{x}$}
	\loigiai{
		Ta có $y'=(\log x)^\prime=\dfrac{1}{x\ln 10}$.
	}
\end{ex}

\begin{ex}%[Nguyễn Giang]%[2D2Y4-2]
	Đạo hàm của hàm số $y=\log (1-x)$ bằng
	\choice
	{$\dfrac{1}{x-1}$}
	{\True $\dfrac{1}{(x-1)\ln 10}$}
	{$\dfrac{1}{1-x}$}
	{$\dfrac{1}{(1-x)\ln 10}$}
	\loigiai{
		Ta có $y'=\left[\log (1-x)\right]^\prime=\dfrac{(1-x)^\prime}{(1-x)\ln 10}=\dfrac{-1}{(1-x)\ln 10}=\dfrac{1}{(x-1)\ln 10}$.
	}
\end{ex}

\begin{ex}%[Nguyễn Giang]%[2D2Y4-2]
	Tính đạo hàm của hàm số $f(x)=\ln |x|$.
	\choice
	{$f'(x)=\dfrac{1}{|x|}$}
	{\True $f'(x)=\dfrac{1}{x}$}
	{$f'(x)=-\dfrac{1}{x}$}
	{$f'(x)=-\dfrac{1}{|x|}$}
	\loigiai{
		Ta có $f'(x)=\dfrac{1}{x}$, $\forall x\ne 0$.
	}
\end{ex}

\begin{ex}%[Nguyễn Giang]%[2D2Y4-2]
	Đạo hàm của hàm số $y=\log_3(2x-3)$ tại điểm $x=2$ bằng
	\choice
	{\True $\dfrac{2}{\ln 3}$}
	{$\dfrac{1}{2\ln 3}$}
	{$2\ln 3$}
	{$1$}
	\loigiai{
		Ta có $y'=\dfrac{(2x-3)'}{(2x-3)\ln 3}=\dfrac{2}{(2x-3)\ln 3}\Rightarrow y'(2)=\dfrac{2}{\ln 3}$.
	}
\end{ex}

\begin{ex}%[Nguyễn Giang]%[2D2Y4-2]
	Đạo hàm của hàm số $y=\log_3\left(4x+1\right)$ là
	\choice
	{$y'=\dfrac{4\ln 3}{4x+1}$}
	{$y'=\dfrac{1}{\left(4x+1\right)\ln 3}$}
	{\True $y'=\dfrac{4}{\left(4x+1\right)\ln 3}$}
	{$y'=\dfrac{\ln 3}{4x+1}$}
	\loigiai{
		Với $x >-\dfrac{1}{4}$, ta có $y'=\dfrac{(4x+1)'}{\left(4x+1\right)\ln 3}=\dfrac{4}{\left(4x+1\right)\ln 3}$.
	}
\end{ex}

\begin{ex}%[Nguyễn Giang]%[2D2Y4-2]
	Tính đạo hàm của hàm số $y=\ln \left(3x^2+1\right)$.
	\choice
	{\True $y'=\dfrac{6x}{3x^2+1}$}
	{$y'=\dfrac{6x+1}{3x^2+1}$}
	{$y'=\dfrac{1}{3x^2+1}$}
	{$y'=\dfrac{3x}{3x^2+1}$}
	\loigiai{
		Ta có $y'=\dfrac{6x}{3x^2+1}$.
	}
\end{ex}

\begin{ex}%[Nguyễn Giang]%[2D2Y4-2]
	Tính đạo hàm của hàm số $y=\log_5\left(x^2+2\right)$.
	\choice
	{$y'=\dfrac{2x}{x^2+2}$}
	{\True $y'=\dfrac{2x}{\left(x^2+2\right)\ln 5}$}
	{$y'=\dfrac{2x\ln 5}{x^2+2}$}
	{$y'=\dfrac{1}{\left(x^2+2\right)\ln 5}$}
	\loigiai{
		Ta có $y'=\left[\log_5\left(x^2+2\right)\right]^\prime=\dfrac{\left(x^2+2\right)^\prime}{\left(x^2+2\right)\ln 5}=\dfrac{2x}{\left(x^2+2\right)\ln 5}$\\
		Vậy $y'=\dfrac{2x}{\left(x^2+2\right)\ln 5}$.
	}
\end{ex}

\begin{ex}%[Nguyễn Giang]%[2D2Y4-2]
	Đạo hàm của hàm số $y=\log_2\left(x-1\right)$ trên tập xác định là
	\choice
	{\True $\dfrac{1}{\left(x-1\right)\ln 2}$}
	{$\dfrac{\ln 2}{x-1}$}
	{$\dfrac{1}{\left(1-x\right)\ln 2}$}
	{$\dfrac{\ln 2}{1-x}$}
	\loigiai{
		Ta có $y'=\dfrac{1}{\left(x-1\right)\ln 2}$.
	}
\end{ex}

\begin{ex}%[Nguyễn Giang]%[2D2Y4-2]
	Tính đạo hàm của hàm số $y=\log_8\left(6x-5\right)$.
	\choice
	{\True $y'=\dfrac{2}{\left(6x-5\right)\ln 2}$}
	{$y'=\dfrac{1}{\left(6x-5\right)\ln 8}$}
	{$y'=\dfrac{6}{6x-5}$}
	{$y'=\dfrac{6}{\left(6x-5\right)\ln 4}$}
	\loigiai{
		Tập xác định của hàm số $\mathscr{D}=\left(\dfrac{5}{6};+\infty\right)$.\\
		Khi đó ta có $y'=\left[\log_8(6x-5)\right]^\prime=\dfrac{(6x-5)^\prime}{(6x-5)\ln 8}=\dfrac{6}{(6x-5)\cdot 3\cdot \ln 2}=\dfrac{2}{(6x-5)\ln 2}$.
	}
\end{ex}

\begin{ex}%[Nguyễn Giang]%[2D2Y4-2]
	Tìm đạo hàm của hàm số $y=\log_2(1-x)$.
	\choice
	{$y'=\dfrac{1}{\log_2(1-x)}$}
	{$y'=\dfrac{1}{1-x}$}
	{$y'=\dfrac{\ln 2}{1-x}$}
	{\True $y'=\dfrac{1}{(x-1)\ln 2}$}
	\loigiai{
		Ta có $y'=\dfrac{(1-x)^\prime}{(1-x).\ln 2}=\dfrac{1}{(x-1)\ln 2}$.
	}
\end{ex}

\begin{ex}%[Nguyễn Giang]%[2D2Y4-2]
	Tính đạo hàm của hàm số $y=\ln (\sin x)$.
	\choice
	{$y'=\tan x$}
	{$y'=-\tan x$}
	{\True $y'=\cot x$}
	{$y'=-\cot x$}
	\loigiai{
		Ta có $y'=\left(\ln (\sin x)\right)^\prime=\dfrac{(\sin x)^\prime}{\sin x}=\dfrac{\cos x}{\sin x}=\cot x$.\\
	}
\end{ex}

\begin{ex}%[Nguyễn Giang]%[2D2Y4-2]
	Đạo hàm của hàm số $y=\ln x+x^2$ là
	\choice
	{$y'=\dfrac{1}{x}+\dfrac{x^3}{3}$}
	{$y'=\dfrac{1}{x}+x$}
	{\True $y'=\dfrac{1}{x}+2x$}
	{$y'=\dfrac{1}{x}-2x$}
	\loigiai{
		Ta có $y'=\dfrac{1}{x}+2x$.
	}
\end{ex}

\begin{ex}%[Nguyễn Giang]%[2D2Y4-2]
	Đạo hàm của hàm số $y=\log_8\left(x^2-3x-4\right)$ là
	\choice
	{$y'=\dfrac{2x-3}{\left(x^2-3x-4\right)\ln 2}$}
	{$y'=\dfrac{2x-3}{x^2-3x-4}$}
	{$y'=\dfrac{1}{\left(x^2-3x-4\right)\ln 8}$}
	{\True $y'=\dfrac{2x-3}{\left(x^2-3x-4\right)\ln 8}$}
	\loigiai{
		Ta có $y'=\dfrac{\left(x^2-3x-4\right)'}{\left(x^2-3x-4\right)\ln 8}=\dfrac{2x-3}{\left(x^2-3x-4\right)\ln 8}$.
	}
\end{ex}
\Closesolutionfile{ans}
%======================
\subsection{Bảng đáp án}
\inputansbox{8}{ans/ANS-DANG-2}

