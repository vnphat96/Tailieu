%Dạng 1
\setcounter {section} {23}
\setcounter{ex}{0}
\section{Tích phân}
\subsection{Kiến thức cần nhớ}
\begin{khung}
	\subsubsection{Định nghĩa tích phân}
	Cho hàm số $y=f(x)$ liên tục và xác định trên đoạn $[a;b]$. Giả sử $F(x)$ là một nguyên hàm của $f(x)$ trên đoạn $[a;b]$. \\ Hiệu số $F(b)-F(a)$ được gọi là tích phân từ $a$ đến $b$ của hàm số $f(x)$. Kí hiệu là $\displaystyle\int\limits_a^b f(x)\mathrm{\,d}x$.\\
	Vậy $\displaystyle\int\limits_a^b f(x)\mathrm{\,d}x= F(x)\bigg|_a^b=F(b)-F(a)$.\\ 
	\subsubsection{Tính chất tích phân xác định} Tính chất của tích phân xác định.
	\begin{itemize}
		\item $\displaystyle\int\limits_a^b f(x)\mathrm{\,d}x=\displaystyle\int\limits_a^c f(x)\mathrm{\,d}x+\displaystyle\int\limits_c^b f(x)\mathrm{\,d}x$ với $a<c<b$.
		\item $k\displaystyle\int\limits_a^b f(x)\mathrm{\,d}x=\displaystyle\int\limits_a^b kf(x)\mathrm{\,d}x$ với $(k\neq 0)$.
		\item $\displaystyle\int\limits_a^b f(x)\mathrm{\,d}x=-\displaystyle\int\limits_b^a f(x)\mathrm{\,d}x$.
		\item $\displaystyle\int\limits_a^b\left(f(x) \pm g(x)\right)\mathrm{\,d}x=\displaystyle\int\limits_a^b f(x)\mathrm{\,d}x \pm \displaystyle\int\limits_a^b g(x)\mathrm{\,d}x$.
		\item $\displaystyle\int\limits_a^b f(x)\mathrm{\,d}x=\displaystyle\int\limits_a^b f(t)\mathrm{\,d}t=\displaystyle\int\limits_a^b f(z) \mathrm{\,d}z$.
		\item $\displaystyle\int\limits_a^b f'(x)\mathrm{\,d}x= f(x)\bigg|_a^b=f(b)-f(a)$.
	\end{itemize}
\end{khung}
\subsection{Bài tập mẫu}
\Opensolutionfile{ans}[ans/ANS-DANG-24]
\begin{khung}
	\begin{vd}[Đề minh họa BGD 2022-2023]%[Thien Tran Xuan]%[2D3B2-1]
		Nếu $\displaystyle\int_{0}^{2} f(x) \mathrm{d} x = 4$ thì $\displaystyle\int_{0}^{2}\left[\dfrac{1}{2} f(x)-2\right] \mathrm{d} x$ bằng
		\choice
		{$0$}
		{$6$}
		{$8$}
		{\True $-2$}
		\loigiai{
		\begin{phantich}
			\muccon{Dạng toán:}
			là dạng toán sử dụng tính chất để tính tích phân xác định của hàm số.\\
			\muccon{Hướng giải:}
			\begin{enumerate}[\bf B1.]
				\item Dựa trên giả thiết $I = \displaystyle\int_{0}^{2}\left[\dfrac{1}{2} f(x)-2\right] \mathrm{d} x$, ta tính tích phân $\dfrac{1}{2}\displaystyle\int_{0}^{2} f(x) \mathrm{d}x - 2\displaystyle\int_{0}^{2} \mathrm{d}x$.
				\item Ta có $I = \dfrac{1}{2}\displaystyle\int_{0}^{2} f(x) \mathrm{d}x - 2\displaystyle\int_{0}^{2} \mathrm{d}x$.
			\end{enumerate}
		\end{phantich}
		Ta có $\displaystyle\int_{0}^{2}\left[\dfrac{1}{2} f(x)-2\right] \mathrm{d} x = \dfrac{1}{2}\displaystyle\int_{0}^{2} f(x) \mathrm{d}x - 2\displaystyle\int_{0}^{2} \mathrm{d}x = \dfrac{1}{2} \cdot 4 - 2 \left( x \bigg|_0^2\right) = -2$.
		}
	\end{vd}
\end{khung}
\subsection{Bài tập tương tự và phát triển}
%%==========Câu 1
\begin{ex}%[Thien Tran Xuan]%[2D3B2-1]
	Cho $\displaystyle\int\limits_1^2 f(x)\mathrm{\,d}x = 3$ và $\displaystyle\int\limits_1^2 \left[3f(x) - g(x)\right]\mathrm{\,d}x = 10$, khi đó $\displaystyle\int\limits_1^2 g(x)\mathrm{\,d}x$ bằng
	\choice
	{\True $-1$}
	{$-4$}
	{$17$}
	{$1$}
	\loigiai{
	Ta có 
	\allowdisplaybreaks
	\begin{eqnarray*}
			\displaystyle\int\limits_1^2 \left[3f(x)-g(x)\right]\mathrm{\,d}x = 10&\Leftrightarrow&3\displaystyle\int\limits_1^2 f(x)\mathrm{\,d}x - \displaystyle\int\limits_1^2 g(x)\mathrm{\,d}x = 10\\
			&\Leftrightarrow&3\cdot 3 - \displaystyle\int\limits_1^2 g(x)\mathrm{\,d}x = 10\\
		&\Leftrightarrow&\displaystyle\int\limits_1^2 g(x)\mathrm{\,d}x = -1.
	\end{eqnarray*}
	}
\end{ex}
%%==========Câu 2
\begin{ex}%[Thien Tran Xuan]%[2D3Y2-3]
	Trong các công thức sau đây, công thức nào đúng?
	\choice
	{$\displaystyle\int\limits_a^b u\mathrm{\,d}v = uv \bigg|_a^b - \displaystyle\int\limits_b^a v\mathrm{\,d}u$}
	{\True $\displaystyle\int\limits_a^b u\mathrm{\,d}v = uv \bigg|_a^b - \displaystyle\int\limits_a^b v\mathrm{\,d}u$}
	{$\displaystyle\int\limits_a^b u\mathrm{\,d}v = uv \bigg|_a^b + \displaystyle\int\limits_a^b v\mathrm{\,d}u$}
	{$\displaystyle\int\limits_a^b u\mathrm{\,d}v = uv  - \displaystyle\int\limits_b^a v\mathrm{\,d}u$}
	\loigiai{
	Công thức tính tích phân từng phần $\displaystyle\int\limits_a^b u\mathrm{\,d}v = uv \bigg|_a^b - \displaystyle\int\limits_a^b v\mathrm{\,d}u$.
	}
\end{ex}
%%==========Câu 3
\begin{ex}%[Thien Tran Xuan]%[2D3B2-1]
	Cho hàm số $f(x)$ có đạo hàm trên đoạn $[1; 3]$, $f(3) = 5$ và $\displaystyle\int\limits_1^3 f'(x)\mathrm{\,d}x = 6$. Khi đó $f(1)$ bằng
	\choice
	{$10$}
	{$11$}
	{$1$}
	{\True $-1$}
	\loigiai{
	Ta có $\displaystyle\int\limits_1^3 f'(x) \mathrm{\,d}x = 6 \Leftrightarrow f(x) \bigg|_1^3 = 6 \Leftrightarrow f(3) - f(1) = 6 \Leftrightarrow 5 - f(1) = 6 \Leftrightarrow f(1) = -1$.\\
	Vậy $f(1) = -1$.
	}
\end{ex}
%%==========Câu 4
\begin{ex}%[Thien Tran Xuan]%[2D3B2-1]
	Biết $F(x)$ là một nguyên hàm của $f(x)$ trên đoạn $[a; b]$ và $\displaystyle\int\limits_a^b f(x)\mathrm{\,d}x = 1$; $F(b) = 2$. Tính $F(a)$.
	\choice
	{\True $1$}
	{$3$}
	{$-1$}
	{$2$}
	\loigiai{
	Ta có $1 = \displaystyle\int\limits_a^b f(x)\mathrm{\,d}x = F(b) - F(a) = 2 - F(a)$. Suy ra $F(a) = 2 - 1 = 1$.
	}
\end{ex}
%%==========Câu 5
\begin{ex}%[Thien Tran Xuan]%[2D3Y2-1]
	Cho $\displaystyle\int\limits_{-3}^2 f(x)\mathrm{\,d}x = -7$. Tính $\displaystyle\int\limits_{-3}^2 3\cdot f(x)\mathrm{\,d}x$?
	\choice
	{$4$}
	{$21$}
	{\True $-21$}
	{$-4$}
	\loigiai{
	Ta có $\displaystyle\int\limits_{-3}^2 3f(x)\mathrm{\,d}x = 3 \displaystyle\int\limits_{-3}^2 f(x)\mathrm{\,d}x = 3 \cdot (-7) = -21$.
	}
\end{ex}
%%==========Câu 6
\begin{ex}%[Thien Tran Xuan]%[2D3B2-1]
	Nếu $\displaystyle\int\limits_1^4 f(x)\mathrm{\,d}x = 9$ và $\displaystyle\int\limits_3^4 f(x)\mathrm{\,d}x = -1$ thì $\displaystyle\int\limits_1^3 f(x)\mathrm{\,d}x$ bằng
	\choice
	{$-8$}
	{$-10$}
	{$8$}
	{\True $10$}
	\loigiai{
	Ta có 
	\allowdisplaybreaks
	\begin{eqnarray*}
		\displaystyle\int\limits_1^4 f(x)\mathrm{\,d}x = \displaystyle\int\limits_1^3 f(x)\mathrm{\,d}x + \displaystyle\int\limits_3^4 f(x)\mathrm{\,d}x
		&\Leftrightarrow&\displaystyle\int\limits_1^3 f(x)\mathrm{\,d}x = \displaystyle\int\limits_1^4 f(x)\mathrm{\,d}x - \displaystyle\int\limits_3^4 f(x)\mathrm{\,d}x\\
		&\Leftrightarrow&\displaystyle\int\limits_1^3 f(x)\mathrm{\,d}x = 9 - (-1) = 10.
	\end{eqnarray*}
	}
\end{ex}
%%==========Câu 7
\begin{ex}%[Thien Tran Xuan]%[2D3B2-1]
	Nếu $F'(x) = \dfrac{1}{2x+1}$ và $F(1) = 1$ thì giá trị của $F(2)$ bằng
	\choice
	{$1 + \dfrac{1}{2} \ln 5$}
	{\True $1 + \dfrac{1}{2} \ln \dfrac{5}{3}$}
	{$1 + \ln \dfrac{5}{3}$}
	{$1 + \ln 5$}
	\loigiai{
	Ta có $\displaystyle\int\limits_1^2 F'(x)\mathrm{\,d}x = F(x) \bigg|_1^2 = F(2) - F(1)$.\\
	Mặt khác $\displaystyle\int\limits_1^2 F'(x)\mathrm{\,d}x = \displaystyle\int\limits_1^2 \dfrac{1}{2x+1}\mathrm{\,d}x = \dfrac{1}{2} \ln |2x + 1| \bigg|_1^2 = \dfrac{1}{2}\ln 5 - \dfrac{1}{2}\ln 3 = \dfrac{1}{2}\ln \dfrac{5}{3}$.\\
	Suy ra $F(2) - F(1) = \dfrac{1}{2}\ln \dfrac{5}{3}$.\\
	Do đó $F(2) = \dfrac{1}{2}\ln \dfrac{5}{3} + F(1) = \dfrac{1}{2}\ln \dfrac{5}{3} + 1$.
	}
\end{ex}
%%==========Câu 8
\begin{ex}%[Thien Tran Xuan]%[2D3Y2-1]
	Cho hàm số $f$, $g$ liên tục trên $K$ và $a$, $b$, $c$ thuộc $K$. Công thức nào sau đây \textbf{sai}?
	\choice
	{$\displaystyle\int\limits_a^b{kf(x)\mathrm{\,d}x=}k\int\limits_a^b{f(x)\mathrm{\,d}x}$}
	{$\displaystyle\int\limits_a^b{f(x)\mathrm{\,d}x+\int\limits_b^c{f(x)\mathrm{\,d}x}=}\int\limits_a^c{f(x)\mathrm{\,d}x}$ với $a < b < c$}
	{$\displaystyle\int\limits_a^b [f(x) + g(x)]\mathrm{\,d}x = \displaystyle\int\limits_a^b f(x)\mathrm{\,d}x + \displaystyle\int\limits_a^b g(x)\mathrm{\,d}x$}
	{\True $\displaystyle\int\limits_a^b f(x)\mathrm{\,d}x = \displaystyle\int\limits_b^a f(x)\mathrm{\,d}x$}
	\loigiai{
	Ta có $\displaystyle\int\limits_a^b f(x)\mathrm{\,d}x = -\displaystyle\int\limits_b^a f(x)\mathrm{\,d}x$ nên $\displaystyle\int\limits_a^b f(x)\mathrm{\,d}x = \displaystyle\int\limits_b^a f(x)\mathrm{\,d}x$ sai.
	}
\end{ex}
%%==========Câu 9
\begin{ex}%[Thien Tran Xuan]%[2D3B2-1]
	Cho $F(x)$ là một nguyên hàm của hàm số $f(x)$. Khi đó hiệu số $F(1) - F(2)$ bằng
	\choice
	{$\displaystyle\int\limits_1^2 [-F(x)]\mathrm{\,d}x$}
	{$\displaystyle\int\limits_1^2 f(x)\mathrm{\,d}x$}
	{\True $\displaystyle\int\limits_1^2 [-f(x)]\mathrm{\,d}x$}
	{$\displaystyle\int\limits_2^1 F(x)\mathrm{\,d}x$}
	\loigiai{
	Ta có $F(1) - F(2) = F(x) \bigg|_2^1 = \displaystyle\int\limits_2^1 f(x)\mathrm{\,d}x = \displaystyle\int\limits_1^2 [-f(x)]\mathrm{\,d}x$.
	}
\end{ex}
%%==========Câu 10
\begin{ex}%[Thien Tran Xuan]%[2D3Y2-1]
	Với mọi hàm số $f(x)$ liên tục trên $\mathbb{R}$, ta có
	\choice
	{$\displaystyle\int\limits_0^3{f(x)\mathrm{\,d}x}=\int\limits_3^0{f(x)\mathrm{\,d}x}$}
	{$\displaystyle\int\limits_0^3{f(x)\mathrm{\,d}x}=\int\limits_{-3}^0{f(x)\mathrm{\,d}x}$}
	{$\displaystyle\int\limits_0^3{f(x)\mathrm{\,d}x}=-\int\limits_3^0{f(x)\mathrm{\,d}x}$}
	{\True $\displaystyle\int\limits_0^3{f(x)\mathrm{\,d}x}=-\int\limits_{-3}^0{f(x)\mathrm{\,d}x}$}
	\loigiai{
	Áp dụng công thức $\displaystyle\int\limits_a^b f(x)\mathrm{\,d}x = -\displaystyle\int\limits_b^a f(x)\mathrm{\,d}x$, với $a < b$.\\
	Ta có $\displaystyle\int\limits_0^3 f(x)\mathrm{\,d}x = -\displaystyle\int\limits_3^0 f(x)\mathrm{\,d}x$.
	}
\end{ex}
%%==========Câu 11
\begin{ex}%[Thien Tran Xuan]%[2D3B2-1]
	Nếu $\displaystyle\int\limits_{-1}^3 f(x)\mathrm{\,d}x = 2$ và $\displaystyle\int\limits_{-1}^3 g(x)\mathrm{\,d}x = -1$ thì $\displaystyle\int\limits_{-1}^3 [ f(x) - g(x)]\mathrm{\,d}x$ bằng
	\choice
	{$-3$}
	{$-1$}
	{\True $3$}
	{$4$}
	\loigiai{
	Ta có $\displaystyle\int\limits_{-1}^3 [f(x) - g(x)]\mathrm{\,d}x = \displaystyle\int\limits_{-1}^3 f(x)\mathrm{\,d}x - \displaystyle\int\limits_{-1}^3 g(x)\mathrm{\,d}x = 2 - (-1) = 3$.
	}
\end{ex}
%%==========Câu 12
\begin{ex}%[Thien Tran Xuan]%[2D3B2-1]
	Nếu $\displaystyle\int\limits_{-1}^0 f(x) \mathrm{\,d}x = -3$ và $\displaystyle\int\limits_0^1 f(x)\mathrm{\,d}x = -1$ thì $\displaystyle\int\limits_{-1}^1 f(x)\mathrm{\,d}x$ bằng
	\choice
	{$3$}
	{$-2$}
	{$2$}
	{\True $-4$}
	\loigiai{
	Ta có $\displaystyle\int\limits_{-1}^1 f(x) \mathrm{\,d}x = \displaystyle\int\limits_{-1}^0 f(x) \mathrm{\,d}x + \displaystyle\int\limits_0^1 f(x)\mathrm{\,d}x = -3 - 1 = -4$.
	}
\end{ex}
%%==========Câu 13
\begin{ex}%[Thien Tran Xuan]%[2D3B2-1]
	Cho $\displaystyle\int\limits_1^2 f(x)\mathrm{\,d}x = 1$ và $\displaystyle\int\limits_2^3 f(x)\mathrm{\,d}x = -2$. Giá trị của $\displaystyle\int\limits_1^3 f(x)\mathrm{\,d}x$ bằng
	\choice
	{$3$}
	{$1$}
	{$-3$}
	{\True $-1$}
	\loigiai{
	Ta có $\displaystyle\int\limits_1^3 f(x)\mathrm{\,d}x = \displaystyle\int\limits_1^2 f(x)\mathrm{\,d}x + \displaystyle\int\limits_2^3 f(x)\mathrm{\,d}x = -1$.
	}
\end{ex}
%%==========Câu 14
\begin{ex}%[Thien Tran Xuan]%[2D3B2-1]
	Cho $f(x)$ là một hàm số liên tục trên $\mathbb{R}$ và $F(x)$ là một nguyên hàm của hàm số $f(x)$ thoả $\displaystyle\int\limits_1^2 f(x)\mathrm{\,d}x = 5$; $F(2) = 11$. Khi đó $F(1)$ bằng
	\choice
	{$16$}
	{$4$}
	{\True $6$}
	{$7$}
	\loigiai{
	Ta có $\displaystyle\int\limits_1^2 f(x)\mathrm{\,d}x = 5 \Leftrightarrow F(2) - F(1) = 5 \Leftrightarrow 11 - F(1) = 5 \Leftrightarrow F(1) = 6$.
	}
\end{ex}
%%==========Câu 15
\begin{ex}%[Thien Tran Xuan]%[2D3B2-1]
	Nếu $\displaystyle\int\limits_{-1}^2 f(x)\mathrm{\,d}x = 2$ và $\displaystyle\int\limits_2^5 f(x)\mathrm{\,d}x = -3$ thi $\displaystyle\int\limits_{-1}^5 f(x)\mathrm{\,d}x$ bằng
	\choice
	{$-6$}
	{\True $-1$}
	{$-5$}
	{$5$}
	\loigiai{
	Ta có $\displaystyle\int\limits_{-1}^5 f(x)\mathrm{\,d}x = \displaystyle\int\limits_{-1}^2 f(x)\mathrm{\,d}x + \displaystyle\int\limits_2^5 f(x)\mathrm{\,d}x = 2 + (-3) = -1$.
	}
\end{ex}
%%==========Câu 16
\begin{ex}%[Thien Tran Xuan]%[2D3B2-1]
	Nếu $\displaystyle\int\limits_1^2 f(x)\mathrm{\,d}x = 5$ và $\displaystyle\int\limits_2^3 f(x)\mathrm{\,d}x = -2$ thì $\displaystyle\int\limits_1^3 f(x)\mathrm{\,d}x$ bằng
	\choice
	{$-7$}
	{\True $3$}
	{$7$}
	{$-10$}
	\loigiai{
	Áp dụng công thức $\displaystyle\int\limits_a^c f(x)\mathrm{\,d}x + \displaystyle\int\limits_c^b f(x)\mathrm{\,d}x = \displaystyle\int\limits_a^b f(x)\mathrm{\,d}x$ với $a < c < b$, ta có
	$$\displaystyle\int\limits_1^3 f(x)\mathrm{\,d}x = \displaystyle\int\limits_1^2 f(x)\mathrm{\,d}x + \displaystyle\int\limits_2^3 f(x)\mathrm{\,d}x = 5 + (-2) = 3.$$
	}
\end{ex}
%%==========Câu 17
\begin{ex}%[Thien Tran Xuan]%[2D3B2-1]
	Giả sử $\displaystyle\int\limits_0^9 f(x)\mathrm{\,d}x = 37$ và $\displaystyle\int\limits_9^0 g(x)\mathrm{\,d}x = 16$. Khi đó, $I = \displaystyle\int\limits_0^9 \left[2f(x) + 3g(x)\right]\mathrm{\,d}x$ bằng
	\choice
	{$I=143$}
	{$I=58$}
	{$I=122$}
	{\True $I=26$}
	\loigiai{
	Ta có $I = \displaystyle\int\limits_0^9 [2f(x) + 3g(x)]\mathrm{\,d}x = 2\displaystyle\int\limits_0^9 f(x)\mathrm{\,d}x - 3\displaystyle\int\limits_9^0 g(x)\mathrm{\,d}x = 26$.
	}
\end{ex}
%%==========Câu 18
\begin{ex}%[Thien Tran Xuan]%[2D3B2-1]
	Cho $\displaystyle\int\limits_2^5 f(x)\mathrm{\,d}x = 10$. Khi đó $\displaystyle\int\limits_2^5 \left[2 - 4f(x)\right]\mathrm{\,d}x$ bằng
	\choice
	{$36$}
	{$-36$}
	{$34$}
	{\True $-34$}
	\loigiai{
	Ta có $\displaystyle\int\limits_2^5 \left[2 - 4f(x)\right]\mathrm{\,d}x = 2\displaystyle\int\limits_2^5 \mathrm{\,d}x - 4\displaystyle\int\limits_2^5 f(x)\mathrm{\,d}x = \left(2x \bigg|_2^5\right) - 4 \cdot 10 = 2(5 - 2) - 40 = -34$.
	}
\end{ex}
%%==========Câu 19
\begin{ex}%[Thien Tran Xuan]%[2D3B2-1]
	Nếu $\displaystyle\int\limits_1^0 f(x)\mathrm{\,d}x = 3$ và $\displaystyle\int\limits_0^1 g(x) \mathrm{\,d}x = -4$ thì $\displaystyle\int\limits_0^1 \left[f(x) - 2g(x)\right] \mathrm{\,d}x$ bằng bao nhiêu?
	\choice
	{ $11$}
	{\True $5$}
	{ $-1$}
	{ $7$}
	\loigiai{
		Ta có $\displaystyle\int\limits_0^1 \left[f(x) - 2g(x)\right]\mathrm{\,d}x = - \displaystyle\int\limits_1^0 f(x)\mathrm{\,d}x - 2\displaystyle\int\limits_0^1 g(x)\mathrm{\,d}x = -3 - 2 \cdot (-4) = 5$.
	}
\end{ex}
%%==========Câu 20
\begin{ex}%[Thien Tran Xuan]%[2D3Y2-1]
	Cho hàm số $y = f(x)$ liên tục trên khoảng $K$ và $a$, $b$, $c \in K$. Mệnh đề nào sau đây \textbf{sai}?
	\choice
	{\True $\displaystyle\int\limits_a^b f(x)\mathrm{\,d}x + \displaystyle\int\limits_c^b f(x)\mathrm{\,d}x = \displaystyle\int\limits_a^c f(x)\mathrm{\,d}x$}
	{$\displaystyle\int\limits_a^b f(x)\mathrm{\,d}x = \displaystyle\int\limits_a^b f(t)\mathrm{\,d}t$}
	{$\displaystyle\int\limits_a^b f(x)\mathrm{\,d}x = -\displaystyle\int\limits_b^a f(x)\mathrm{\,d}x$}
	{$\displaystyle\int\limits_a^a f(x)\mathrm{\,d}x = 0$}
	\loigiai{
	Mệnh đề sai là $\displaystyle\int\limits_a^b f(x)\mathrm{\,d}x + \displaystyle\int\limits_c^b f(x)\mathrm{\,d}x = \displaystyle\int\limits_a^c f(x)\mathrm{\,d}x$.
	}
\end{ex}
%%==========Câu 21
\begin{ex}%[Thien Tran Xuan]%[2D3Y2-1]
	Cho hàm số $f(x)$ liên tục trên $[a; b]$ và $F(x)$ là một nguyên hàm của $f(x)$. Tìm khẳng định \textbf{sai}.
	\choice
	{\True $\displaystyle\int\limits_a^b f(x)\mathrm{\,d}x = F(a) - F(b)$}
	{$\displaystyle\int\limits_a^a f(x)\mathrm{\,d}x = 0$}
	{$\displaystyle\int\limits_a^b f(x)\mathrm{\,d}x = -\displaystyle\int\limits_b^a f(x)\mathrm{\,d}x$}
	{$\displaystyle\int\limits_a^b f(x)\mathrm{\,d}x = F(b) - F(a)$}
	\loigiai{
	Khẳng định $\displaystyle\int\limits_a^b f(x)\mathrm{\,d}x = F(a) - F(b)$ sai.
	}
\end{ex}
%%==========Câu 22
\begin{ex}%[Thien Tran Xuan]%[2D3B2-1]
	Cho hàm số phức $f(x)$ và $g(x)$ liên tục trên đoạn $[1; 7]$ sao cho $\displaystyle\int\limits_1^7{f(x)\mathrm{\,d}x}=2$ và $\displaystyle\int\limits_1^7{g(x)\mathrm{\,d}x}=-3$. Giá trị của $\displaystyle\int\limits_1^7{[ f(x)-g(x) ]\mathrm{\,d}x}$ bằng
	\choice
	{$-1$}
	{$-5$}
	{\True $5$}
	{$6$}
	\loigiai{
	Ta có $\displaystyle\int\limits_1^7 [f(x) - g(x)]\mathrm{\,d}x = \displaystyle\int\limits_1^7 f(x)\mathrm{\,d}x - \displaystyle\int\limits_1^7 g(x)\mathrm{\,d}x = 2 - (-3) = 5$.
	}
\end{ex}
%%==========Câu 23
\begin{ex}%[Thien Tran Xuan]%[2D3Y2-1]
	Cho các số thực $a$, $b$ ($a < b$). Nếu hàm số $y = f(x)$ có đạo hàm là hàm số liên tục trên $\mathbb{R}$ thì
	\choice
	{ $\displaystyle\int\limits_a^b f(x)\mathrm{\,d}x = f'(a) - f'(b)$}
	{\True $\displaystyle\int\limits_a^b f'(x)\mathrm{\,d}x = f(b) - f(a)$}
	{ $\displaystyle\int\limits_a^b f(x)\mathrm{\,d}x = f'(b) - f'(a)$}
	{ $\displaystyle\int\limits_a^b f'(x)\mathrm{\,d}x = f(a) - f(b)$}
	\loigiai{
	Ta có $\displaystyle\int\limits_a^b f'(x)\mathrm{\,d} x = f(x) \bigg|_a^b = f(b) - f(a)$.
	}
\end{ex}
%%==========Câu 24
\begin{ex}%[Thien Tran Xuan]%[2D3B2-1]
	Cho hàm số $y = f(x)$ xác định và liên tục trên $R$, có $f(8) = 20$; $f(4) = 12$. Tính tích phân $I = \displaystyle\int\limits_4^8 f'(x)\mathrm{\,d}x$.
	\choice
	{$I=16$}
	{$I=4$}
	{$I=32$}
	{\True $I=8$}
	\loigiai{
	Ta có $I = \displaystyle\int\limits_4^8 f'(x)\mathrm{\,d}x = f(x) \bigg|_4^8 = f(8) - f(4) = 8$.
	}
\end{ex}
%%==========Câu 25
\begin{ex}%[Thien Tran Xuan]%[2D3B2-1]
	Cho hàm số $y = f(x)$ thoả mãn điều kiện $f(1) = 12$, $f'(x)$ liên tục trên $\mathbb{R}$ và $\displaystyle\int\limits_1^4 f'(x)\mathrm{\,d}x = 17$. Khi đó $f(4)$ bằng
	\choice
	{$5$}
	{\True $29$}
	{$19$}
	{$9$}
	\loigiai{
	Ta có $\displaystyle\int\limits_1^4 f'(x)\mathrm{\,d}x = 17 \Leftrightarrow f(x) \bigg|_1^4 = 17 \Leftrightarrow f(4) - f(1) = 17 \Leftrightarrow f(4) = 29$.
	}
\end{ex}
%%==========Câu 26
\begin{ex}%[Thien Tran Xuan]%[2D3Y2-1]
	Cho hàm số $y = f(x)$ liên tục trên khoảng $K$ và $a$, $b$, $c \in K$. Mệnh đề nào sau đây \textbf{sai}?
	\choice
	{\True $\displaystyle\int\limits_a^b f(x)\mathrm{\,d}x + \displaystyle\int\limits_c^b f(x)\mathrm{\,d}x = \displaystyle\int\limits_a^c f(x)\mathrm{\,d}x$}
	{$\displaystyle\int\limits_a^b f(x)\mathrm{\,d}x = \displaystyle\int\limits_a^b f(t)\mathrm{\,d}t$}
	{$\displaystyle\int\limits_a^b f(x)\mathrm{\,d}x = -\displaystyle\int\limits_b^a f(x)\mathrm{\,d}x$}
	{$\displaystyle\int\limits_a^a f(x)\mathrm{\,d}x = 0$}
	\loigiai{
	Mệnh đề sai là $\displaystyle\int\limits_a^b f(x)\mathrm{\,d}x + \displaystyle\int\limits_c^b f(x)\mathrm{\,d}x = \displaystyle\int\limits_a^c f(x)\mathrm{\,d}x$.
	}
\end{ex}
%%==========Câu 27
\begin{ex}%[Thien Tran Xuan]%[2D3B2-1]
	Cho $\displaystyle\int\limits_1^5 h(x)\mathrm{\,d}x = 4$ và $\displaystyle\int\limits_1^7 h(x)\mathrm{\,d}x = 10$, khi đó $\displaystyle\int\limits_5^7 h(x)\mathrm{\,d}x$ bằng
	\choice
	{\True $6$}
	{$5$}
	{$7$}
	{$2$}
	\loigiai{
	Ta có 
	\allowdisplaybreaks
	\begin{eqnarray*}
		\displaystyle\int\limits_1^5h(x)\mathrm{\,d}x + \displaystyle\int\limits_5^7h(x)\mathrm{\,d}x = \displaystyle\int\limits_1^7 h(x)\mathrm{\,d}x
		&\Leftrightarrow&\displaystyle\int\limits_5^7 h(x)\mathrm{\,d}x = \displaystyle\int\limits_1^7 h(x)\mathrm{\,d}x - \displaystyle\int\limits_1^5h(x)\mathrm{\,d}x\\
		&\Leftrightarrow&\displaystyle\int\limits_5^7 h(x)\mathrm{\,d}x = 10 - 4 = 6.
	\end{eqnarray*}
	}
\end{ex}
%%==========Câu 28
\begin{ex}%[Thien Tran Xuan]%[2D3B2-1]
	Cho $\displaystyle\int\limits_{-2}^2 f(x)\mathrm{\,d}x = 1$, $\displaystyle\int\limits_{-2}^4 f(t)\mathrm{\,d}t = -4$. Tính $I = \displaystyle\int\limits_2^4 f(y)\mathrm{\,d}y$.
	\choice
	{$I = -3$}
	{\True $I = -5$}
	{$I = 5$}
	{$I = 3$}
	\loigiai{
		Ta có $\displaystyle\int\limits_{-2}^4 f(t)\mathrm{\,d}t = \displaystyle\int\limits_{-2}^4 f(x)\mathrm{\,d}x = \displaystyle\int\limits_{-2}^2 f(x)\mathrm{\,d}x + \displaystyle\int\limits_2^4 f(x)\mathrm{\,d}x = \displaystyle\int\limits_{-2}^2 f(x)\mathrm{\,d}x + \displaystyle\int\limits_2^4 f(y)\mathrm{\,d}y$.\\
		Suy ra $\displaystyle\int\limits_2^4 f(y)\mathrm{\,d}y = \displaystyle\int\limits_{-2}^4 f(t)\mathrm{\,d}t - \displaystyle\int\limits_{-2}^2 f(x)\mathrm{\,d}x = -4 - 1 = -5$.
	}
\end{ex}
%%==========Câu 29
\begin{ex}%[Thien Tran Xuan]%[2D3B2-1]
	Cho hàm số $f(x)$ có đạo hàm $f'(x)$ liên tục trên $[a; b]$, $f(b) = 5$ và $\displaystyle\int\limits_a^b f'(x)\mathrm{\,d}x = 1$, khi đó $f(a)$ bằng
	\choice
	{\True $4$}
	{$6$}
	{$-4$}
	{$-6$}
	\loigiai{
	Ta có $1 = \displaystyle\int\limits_a^b f'(x)\mathrm{\,d}x = f(x) \bigg|_a^b = f(b) - f(a) = 5 - f(a) \Rightarrow f(a) = 4$.
	}
\end{ex}
%%==========Câu 30
\begin{ex}%[Thien Tran Xuan]%[2D3Y2-1]
	Mệnh đề nào sau đây đúng?
	\choice
	{\True $\displaystyle\int\limits_a^b f(x)\mathrm{\,d}x + \displaystyle\int\limits_b^a f(x)\mathrm{\,d}x = 0$}	
	{$\displaystyle\int\limits_a^b f(x)\mathrm{\,d}x = F(a) - F(b)$ ($F(x)$ là một nguyên hàm của $f(x)$)}
	{$\displaystyle\int\limits_{-a}^a f(x)\mathrm{\,d}x = 0$}
	{$\displaystyle\int\limits_a^b f(x)\mathrm{\,d}x + \displaystyle\int\limits_a^c f(x)\mathrm{\,d}x = \displaystyle\int\limits_b^c f(x)\mathrm{\,d}x$}
	\loigiai{Ta có $\displaystyle\int\limits_a^b f(x)\mathrm{\,d}x + \displaystyle\int\limits_b^a f(x)\mathrm{\,d}x = \displaystyle\int\limits_a^a f(x)\mathrm{\,d}x = 0$ là đúng.\\
	Do $\displaystyle\int\limits_a^b f(x)\mathrm{\,d}x + \displaystyle\int\limits_a^c f(x)\mathrm{\,d}x = \displaystyle\int\limits_b^c f(x)\mathrm{\,d}x \Leftrightarrow \displaystyle\int\limits_a^c f(x)\mathrm{\,d}x = \displaystyle\int\limits_b^c f(x)\mathrm{\,d}x - \int\limits_a^b f(x)\mathrm{\,d}x$ là sai.\\
	Vì $\displaystyle\int\limits_a^b f(x)\mathrm{\,d}x = F(a) - F(b)$ là sai vì $\displaystyle\int\limits_a^b f(x)\mathrm{\,d}x = F(b) - F(a)$.\\
	Vì $\displaystyle\int\limits_{-a}^a f(x)\mathrm{\,d}x = 0$ là sai vì $\displaystyle\int\limits_{-a}^a f(x)\mathrm{\,d}x = \left(x \bigg| _{-a}^a \right) = 2a$.
	}
\end{ex}
%%==========Câu 31
\begin{ex}%[Thien Tran Xuan]%[2D3B2-1]
	Cho $\displaystyle\int\limits_{-1}^2 f(x)\mathrm{\,d}x = 2$ và $\displaystyle\int\limits_{-1}^2 g(x)\mathrm{\,d}x = -1$. Tính $I = \displaystyle\int\limits_{-1}^2 [x + 2f(x) + 3g(x)]\mathrm{\,d}x$ bằng
	\choice
	{\True $I = \dfrac{5}{2}$}
	{$I = \dfrac{7}{2}$}
	{$I = \dfrac{17}{2}$}
	{$I = \dfrac{11}{2}$}
	\loigiai{
	Ta có 
	$$I = \displaystyle\int\limits_{-1}^2 [x + 2f(x) + 3g(x)]\mathrm{\,d}x = \displaystyle\int\limits_{-1}^2{x\mathrm{\,d}x} + 2\displaystyle\int\limits_{-1}^2 f(x)\mathrm{\,d}x + 3\displaystyle\int\limits_{-1}^2 g(x)\mathrm{\,d}x = \left(\dfrac{x^2}{2} \bigg|_{-1}^2\right) + 2 \cdot 2 + 3 \cdot ( -1 ) = \dfrac{5}{2}.$$
	}
\end{ex}
%%==========Câu 32
\begin{ex}%[Thien Tran Xuan]%[2D3B2-1]
	Cho $\displaystyle\int\limits_{-1}^2 f(x)\mathrm{\,d}x = 2$ và $\int\limits_{-1}^2 g(x)\mathrm{\,d}x = -1$. Tính $I = \int\limits_{-1}^2 [x + 2f(x) + 3g(x)]\mathrm{\,d}x$ bằng
	\choice
	{$I = \dfrac{17}{2}$}
	{\True $I = \dfrac{5}{2}$}
	{$I = \dfrac{11}{2}$}
	{$I = \dfrac{7}{2}$}
	\loigiai{
	Ta có $I = \left(\dfrac{x^2}{2} \bigg|_{-1}^2\right) + 2\displaystyle\int\limits_{-1}^2 f(x)\mathrm{\,d}x + 3\displaystyle\int\limits_{-1}^2 g(x)\mathrm{\,d}x = \dfrac{3}{2} + 4 - 3 = \dfrac{5}{2}$.
	}
\end{ex}
%%==========Câu 33
\begin{ex}%[Thien Tran Xuan]%[2D3B2-1]
	Tính $I = \displaystyle\int\limits_0^1 \mathrm{e}^{3x} \mathrm{\,d}x$.
	\choice
	{ $I = \mathrm{e}^3 + \dfrac{1}{2}$}
	{ $I = \mathrm{e}^3 - 1$}
	{ $I = \mathrm{e} - 1$}
	{ $\dfrac{\mathrm{e}^3 - 1}{3}$}
	\loigiai{
	Ta có $I = \displaystyle\int\limits_0^1 \mathrm{e}^{3x} \mathrm{\,d}x = \left(\dfrac{1}{3}\mathrm{e}^{3x} \bigg|_0^1\right) = \dfrac{\mathrm{e}^{3x} - 1}{3}$.
	}
\end{ex}
%%==========Câu 34
\begin{ex}%[2D3Y2-1]
	Nếu $\displaystyle\int\limits_1^2 f(x)\mathrm{\,d}x = 5$ thì $\displaystyle\int\limits_2^1 \pi f(x)\mathrm{\,d}x$ bằng
	\choice
	{ $5\pi $}
	{ $\dfrac{\pi}{5}$}
	{\True $-5\pi $}
	{ $\dfrac{-\pi }{5}$}
	\loigiai{
	Ta có $\displaystyle\int\limits_2^1 \pi f(x)\mathrm{\,d}x = -\pi \displaystyle\int\limits_1^2 f(x)\mathrm{\,d}x = -5\pi $.
	}
\end{ex}
%%==========Câu 35
\begin{ex}%[Thien Tran Xuan]%[2D3Y2-1]
	Cho $\displaystyle\int\limits_1^2 f(x)\mathrm{\,d}x = -3$ và $\displaystyle\int\limits_2^3 f(x)\mathrm{\,d}x = 4$, khi đó tích phân $\displaystyle\int\limits_1^3 f(x)\mathrm{\,d}x$ bằng
	\choice
	{$7$}
	{\True $1$}
	{$12$}
	{$-12$}
	\loigiai{
	Ta có $\displaystyle\int\limits_1^3 f(x)\mathrm{\,d}x = \displaystyle\int\limits_1^2 f(x)\mathrm{\,d}x + \displaystyle\int\limits_2^3 f(x)\mathrm{\,d}x = -3 + 4 = 1$.
	}
\end{ex}
%%==========Câu 36
\begin{ex}%[Thien Tran Xuan]%[2D3B2-1]
	Nếu $\displaystyle\int\limits_0^1 f(x)\mathrm{\,d}x = 2$ và $\displaystyle\int\limits_0^1 g(x)\mathrm{\,d}x = 3$ thì $\displaystyle\int\limits_0^1 \left[f(x) + g(x)\right]\mathrm{\,d}x$ bằng
	\choice
	{ $2$}
	{ $6$}
	{\True $5$}
	{ $3$}
	\loigiai{
	Ta có $\displaystyle\int\limits_0^1 \left[f(x) + g(x)\right]\mathrm{\,d}x = \displaystyle\int\limits_0^1 f(x)\mathrm{\,d}x + \displaystyle\int\limits_0^1 g(x)\mathrm{\,d}x = 2 + 3 = 5 $.
	}
\end{ex}
%%==========Câu 37
\begin{ex}%[Thien Tran Xuan]%[2D3B2-1]
	Cho hàm số $f(x)$ có đạo hàm $f'(x)$ liên tục trên $[a; b]$, $f(b) = 5$ và $\displaystyle\int\limits_a^b f'(x)\mathrm{\,d}x = 3\sqrt{5}$. Tính $f(a)$.
	\choice
	{\True $f(a) = \sqrt {5}\left(\sqrt{5} - 3\right)$}
	{$f(a) = 3\sqrt{5}$}
	{$f(a) = \sqrt{5}\left(3 - \sqrt{5}\right)$}
	{$f(a) = \sqrt{3}\left(\sqrt{5} - 3\right)$}
	\loigiai{
	Ta có $\displaystyle\int\limits_a^b f'(x)\mathrm{\,d}x = \left(f(x) \bigg|_a^b\right) = f(b) - f(a) = 3\sqrt {5}$.\\
	Suy ra $f(a) = f(b) - 3\sqrt{5} = 5 - 3\sqrt{5} = \sqrt{5}\left(\sqrt{5} - 3\right)$.
	}
\end{ex}
%%==========Câu 38
\begin{ex}%[Thien Tran Xuan]%[2D3B2-1]
	Tính tích phân $I = \displaystyle\int\limits_0^2 (2x + 1)\mathrm{\,d}x$.
	\choice
	{$I = 2$}
	{$I = 4$}
	{$I = 5$}
	{\True $I = 6$}
	\loigiai{ 
	Ta có $I = \displaystyle\int\limits_0^2 (2x + 1)\mathrm{\,d}x = \left((x^2 + x) \bigg|_0^2\right) = 4 + 2 = 6$.
	}
\end{ex}
%%==========Câu 39
\begin{ex}%[Thien Tran Xuan]%[2D3B2-1]
	Tích phân $\displaystyle\int\limits_1^2 3^{x - 1}\mathrm{\,d}x$ bằng
	\choice
	{\True $\dfrac{2}{\ln 3}$}
	{$2\ln 3$}
	{$\dfrac{3}{2}$}
	{$2$}
	\loigiai{
	Ta có $\displaystyle\int\limits_1^2 3^{x - 1}\mathrm{\,d}x = \displaystyle\int\limits_1^2 3^{x-1}\mathrm{\,d}(x - 1) = \left(\dfrac{3^{x-1}}{\ln 3} \bigg|_1^2\right) = \dfrac{2}{\ln 3}$.
	}
\end{ex}


\Closesolutionfile{ans}
%======================
\subsection{Bảng đáp án}
\inputansbox{8}{ans/ANS-DANG-24}

