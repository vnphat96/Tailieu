%Dạng 1
\setcounter{ex}{0}
\section{Cấp số cộng, cấp số nhân}
\subsection{Kiến thức cần nhớ}
\begin{khung}
	\subsubsection{Nhận dạng cấp số cộng, cấp số nhân}
	\begin{itemize}
		\item Nếu $u_{n+1}=u_n+d$, với $d$ là hằng số $\Rightarrow \left(u_n \right)$ là cấp số cộng.
		\item Nếu $u_{n+1}=u_n\cdot q$, với $q$ là hằng số $\Rightarrow \left(u_n \right)$ là cấp số nhân.
	\end{itemize}
	\subsubsection{Số hạng tổng quát, số hạng thứ $n$}
	\begin{itemize}
		\item Nếu $\left(u_n\right)$ là cấp số cộng thì số hạng tổng quát của $\left(u_n\right)$ là $u_n=u_1+\left(n-1 \right)\cdot d$.
		\item Nếu $\left(u_n \right)$ là cấp số nhân, thì số hạng tổng quát của $\left(u_n\right)$ là $u_n=u_1\cdot q^{n-1}$.
	\end{itemize}
	\subsubsection{Công sai, công bội}
	\begin{itemize}
		\item Cấp số cộng $\left(u_n \right)$ có công sai là $d=u_{n+1}-u_n$.
		\item Cấp số nhân $\left(u_n\right)$ có công bội là $q=\dfrac{u_{n+1}}{u_n}$.
	\end{itemize}
	\subsubsection{Tổng $n$ số hạng đầu của cấp số cộng, cấp số nhân}
	\begin{itemize}
		\item Tổng $n$ số hạng đầu tiên $S_n$ của cấp số cộng $\left(u_n \right)$ được xác định bởi công thức
		\begin{center}
			\begin{tcolorbox}
				\begin{center}
					$S_n=u_1+u_2+\ldots+u_n=\dfrac{n}{2}\left( u_1+u_n \right)=\dfrac{n}{2}\left[ 2u_1+\left( n-1 \right)d \right].$
				\end{center}
			\end{tcolorbox}
			
		\end{center}
		\item Tổng $n$ số hạng đầu tiên $S_n$ của cấp số nhân $\left(u_n \right)$ được xác định bởi công thức 
		\begin{center}
			\begin{tcolorbox}
			\begin{center}
				$S_n=u_1+u_2+\ldots+u_n=u_1\cdot\dfrac{1-q^n}{1-q}.$
			\end{center}
		\end{tcolorbox}
		\end{center}
	\end{itemize}
\end{khung}
\begin{khung}
	\subsubsection{Điều kiện tạo thành cấp số cộng, cấp số nhân}
	\begin{itemize}
		\item Ba số $a, b, c$ theo thứ tự tạo thành một cấp số cộng $\Leftrightarrow a+c=2b$.
		\item Ba số $a, b, c$ theo thứ tự tạo thành một cấp số nhân $\Leftrightarrow a\cdot c=b^2$.
	\end{itemize}
\end{khung}
\subsection{Bài tập mẫu}
\Opensolutionfile{ans}[ans/ANS-DANG-1]
\begin{khung}
%	\setcounter{vd}{4}
	\begin{vd}[Đề minh họa BGD 2020-2021]%[ĐMH-LVĐ]%[1D3B4-3]
		Cho cấp số nhân $\left(u_n\right)$ với $u_1=2$ và công bội $q=\dfrac{1}{2}$. Giá trị $u_3$ bằng
		\choice
		{$3$}	
		{\True $\dfrac{1}{2}$}
		{$\dfrac{1}{4}$}
		{$\dfrac{7}{2}$}
		\loigiai{
				Ta có: $u_n=u_1\cdot q^{n-1}\Rightarrow u_3=u_1\cdot q^2\Leftrightarrow u_3=2\cdot \left(\dfrac{1}{2}\right)^2=\dfrac{1}{2}$.
		}
	\end{vd}
\end{khung}
\subsection{Bài tập tương tự và phát triển}
\begin{ex}%[1D3B4-1]%Câu 1%
	Trong các dãy số sau, dãy số nào là một cấp số nhân?
	\choice
		{\True $1;2;4;8;16;32;\ldots$}
		{$2;4;6;8;16;32;\ldots$}
		{ $-2;-3;-4;-5;-6;-7;\ldots$}
		{$1;2;3;4;5;6;\ldots$}
	\loigiai{
		Nhận thấy $\dfrac{{{u}_2}}{{{u}_1}}\ne \dfrac{{{u}_3}}{{{u}_2}}$ nên các dãy số ở đáp án B, C và D không phải là cấp số nhân.\\
		Riêng đối với dãy $1;2;4;8;16;32;\ldots.$ở đáp án A thỏa mãn: $u_{n+1}=2\cdot u_n,\forall n\in \mathbb{N}^*$.\\
		Vậy dãy số $1;2;4;8;16;32;\ldots$ là cấp số nhân với $u_1=1$ và công bội $q=2$.}
\end{ex}
\begin{ex}%[1D3Y4-3]%Câu 2
	Cho cấp số nhân $\left(u_n \right)$ với $u_1=3$ và $u_2=6$. Công bội của cấp số nhân đã cho bằng
	\choice
		{\True $2$}
		{$9$}
		{$3$}
		{$\dfrac{1}{2}$}
	\loigiai{
		Gọi $q$ là công bội của cấp số nhân đã cho, ta có $q=\dfrac{{{u}_2}}{{{u}_1}}=2$.}
\end{ex}
\begin{ex}%[1D3Y4-3]%Câu 3
	Cho cấp số nhân $\left(u_n \right)$ có $u_1=2$ và công bội $q=2$. Tính $u_3$?
	\choice
		{ $u_3=18$}
		{$u_3=6$}
		{\True $u_3=8$}
		{$u_3=4$}
	\loigiai{
		Ta có $u_3=u_1\cdot q^2=2\cdot 2^2=8$.}
\end{ex}
\begin{ex}%[1D3B4-3]%Câu 4
	Cho cấp số nhân $\left(u_n \right)$ thỏa mãn $u_1=3;u_5=48$. Công bội của cấp số nhân bằng
	\choice
		{$2$}
		{\True $\pm 2$ }
		{$16$}
		{$-2$}
	\loigiai{
		Gọi $q$ là công bội của cấp số nhân $\left(u_n \right)$.\\
		Với ${u_1}=3;{{u}_5}=48$ suy ra $ \heva{&u_1=3\\&u_1\cdot q^4=48}\Leftrightarrow \heva{&u_1=3 \\ & q^4=16}\Leftrightarrow \heva{& u_1=3 \\& q=\pm 2.}$\\
		Vậy công bội của cấp số nhân $\left(u_n \right)$ là $q=\pm 2$.}
\end{ex}
\begin{ex}%[1D3Y3-2]%Câu 5
	Cho cấp số cộng $\left(u_n \right)$ có $u_1=-2$ và công sai $d=3$. Số hạng tổng quát $u_n$ của cấp số cộng là
	\choice
		{$u_n=-3n+2$ }
		{$u_n=3n-2$}
		{\True $u_n=3n-5$}
		{$u_n=-2n+3$}
	\loigiai{
		Ta có: $u_n=u_1+\left( n-1 \right)d=-2+\left( n-1 \right)\cdot 3=3n-5$.\\
		Vậy số hạng tổng quát $u_n$ của cấp số cộng là $u_n=3n-5$.}
\end{ex}
\begin{ex}%[1D3Y3-2]%Câu 6
	Cho cấp số cộng $\left(u_n\right)$ với số hạng đầu $u_1$ và công sai $d$. Tìm số hạng tổng quát của $\left(u_n\right)$?
	\choice
		{\True  $u_n=u_1+\left( n-1 \right)d,\,n\ge 2$}
		{$u_n=u_1\cdot d^n$}
		{ $u_n=u_1\cdot d^{n-1}$}
		{$u_n=u_1+nd$}
	\loigiai{
		Số hạng tổng quát của $\left(u_n\right)$ là: $u_n=u_1+\left(n-1\right)d,n\ge 2$.}
\end{ex}
\begin{ex}%[1D3B3-3]%Câu 7
	Cho cấp số cộng $\left(u_n\right)$ có $u_1=-3$, $u_6=27$. Tính công sai $d$?
	\choice
		{ $d=5$}
		{$d=8$}
		{\True $d=6$}
		{$d=7$}
	\loigiai{
		Ta có $u_6=u_1+5d\Leftrightarrow27=-3+5d\Leftrightarrow d=6$.}
\end{ex}
\begin{ex}%[1D3Y3-3]%Câu 8
	Cho cấp số cộng $\left(u_n\right)$ với $u_1=3$ và công sai $d=4$. Số hạng thứ hai của cấp số cộng đã cho là
	\choice
		{$12$}
		{$10$}
		{\True $7$}
		{$-1$}
	\loigiai{
		Ta có số hạng thứ hai là $u_2=u_1+d=3+4=7$.}
\end{ex}
\begin{ex}%[1D3Y3-3]%Câu 9
	Cho cấp số cộng $\left(u_n\right)$ có $u_2=1$ và $u_3=3$. Giá trị của $u_4$ bằng
	\choice
		{ $6$}
		{$9$}
		{$4$}
		{\True $5$}
	\loigiai{
		Công sai $d=u_3-u_2=2$.\\ Vậy $u_4=u_3+d=3+2=5$.}
\end{ex}
\begin{ex}%[1D3Y4-3]%Câu 10
	Cho cấp số nhân $\left(u_n\right)$ với $u_1=2$ và $u_2=-6$. Công bội của cấp số nhân đã cho bằng
	\choice
		{$3$}
		{\True $-3$}
		{$-\dfrac{1}{3}$}
		{$\dfrac{1}{3}$}
	\loigiai{
		Công bội của cấp số nhân đã cho là $q=\dfrac{u_2}{u_1}=\dfrac{-6}{2}=-3$.}
\end{ex}
\begin{ex}%[1D3B3-3]%Câu 11
	Cho cấp số cộng $\left(u_n\right)$ có $u_1=123$, $u_3-u_{15}=84$. Số hạng $u_{17}$ bằng
	\choice
		{\True  $11$}
		{$12$}
		{$132$ }
		{$235$}
	\loigiai{
		Giả sử cấp số cộng $\left(u_n\right)$ có công sai $d$.\\
		Theo giả thiết ta có: $u_3-u_{15}=84 \Leftrightarrow u_1+2d-u_1-14d=84 \Leftrightarrow -12d=84 \Leftrightarrow d=-7$.\\
		Vậy $u_{17}=u_1+16d =123+16\cdot \left( -7 \right) =11$.}
\end{ex}
\begin{ex}%[1D3B3-3]%Câu 12
	Cho cấp số cộng với số hạng đầu $u_1=-3$, số hạng cuối $u_n=487$ và công sai $d=5$. Hỏi cấp số cộng có bao nhiêu số hạng?
	\choice
		{$69$}
		{$79$}
		{$89$}
		{\True $99$}
	\loigiai{
		Ta có: công thức cấp số cộng $u_n=u_1+\left( n-1 \right)d\Rightarrow n=\dfrac{u_n-u_1}{d}+1=99$.}
\end{ex}
\begin{ex}%[1D3B4-3]%Câu 13
	Cho cấp số nhân $\left(u_n\right)$ với $u_2=2$ và $u_4=18$. Công bội của cấp số nhân đã cho bằng
	\choice
		{$16$}
		{\True $\pm 3$}
		{$\dfrac{1}{9}$}
		{$9$}
	\loigiai{
		Ta có: $u_n=u_1\cdot q^{n-1}$ nên $\heva{&u_2=u_1\cdot q=2 \\ &u_4=u_1\cdot q^3=18}\Rightarrow q^2=9\Leftrightarrow q=\pm 3$.	}
\end{ex}
\begin{ex}%[1D3B4-3]%Câu 14
	Cho cấp số nhân $\left(u_n\right)$ với $u_1=3$ và $u_4=-24$. Công bội của cấp số nhân đã cho bằng
	\choice
		{$2$}
		{\True $-2$}
		{$-8$}
		{$-\dfrac{4}{3}$}
	\loigiai{
		Ta có: $u_4=u_1\cdot q^3\Leftrightarrow q^3=\dfrac{u_4}{u_1}\Leftrightarrow q^3=\dfrac{-24}{3}\Leftrightarrow q^3=-8\Leftrightarrow q=-2$.}
\end{ex}
\begin{ex}%[1D3Y4-3]%Câu 15
	Cho cấp số nhân $\left(u_n\right)$ có số hạng đầu $u_1=2$ và công bội $q=3$. Giá trị của $u_6$ bằng
	\choice
		{$729$}
		{$1458$}
		{$243$}
		{\True $486$}
	\loigiai{
		Ta có $u_6=u_1\cdot q^5=2\cdot 3^5=486$.
			}
\end{ex}
\begin{ex}%[1D3Y4-3]%Câu 16
	Cho cấp số nhân $\left(u_n\right)$ có số hạng đầu $u_1=2$ và công bội $q=3$. Số hạng thứ 5 bằng
	\choice
		{$48$}
		{$486$}
		{\True $162$}
		{$96$}
	\loigiai{
		Số hạng tổng quát $u_n=u_1\cdot q^{n-1}$ suy ra $u_5=u_1\cdot q^4=2\cdot 3^4=162$.}
\end{ex}
\begin{ex}%[1D3Y4-3]%Câu 17
	Cho cấp số nhân $\left(u_n\right)$ có số hạng đầu $u_1=2$ và $u_6=486$. Công bội $q$ bằng
	\choice
		{\True  $q=3$}
		{$q=5$}
		{$q=\dfrac{3}{2}$}
		{$q=\dfrac{2}{3}$}
	\loigiai{
		Theo đề ra ta có: $\heva{&u_1=2\\& u_6=486}\Leftrightarrow \heva{&u_1=2 \\ & 486=u_1\cdot q^5}\Rightarrow q^5=243=3^5 \Rightarrow q=3$.}
\end{ex}
\begin{ex}%[1D3Y4-3]%Câu 18
	Cho cấp số nhân $\left(u_n \right)$, biết $u_2=1; u_3=5$. Công bội $q$ của cấp số nhân đã cho bằng
	\choice
		{\True $5$}
		{$\pm 4$}
		{$4$}
		{$21$}
	\loigiai{
		Theo công thức tổng quát của cấp số nhân $u_3=u_2\cdot q \Leftrightarrow 5=1\cdot q \Leftrightarrow q=5$.}
\end{ex}
\begin{ex}%[1D3Y4-3]%Câu 19
	Cho cấp số nhân $\left(u_n\right)$ có $u_1=3$, công bội $q=2$. Ta có $u_5$ bằng
	\choice
		{$11$}
		{\True $48$}
		{$9$}
		{$24$}
	\loigiai{
		Cấp số nhân $\left(u_n\right)$ với $u_1=3$, công bội $q=2$ có số hạng tổng quát là $u_n=3\cdot 2^{n-1},n\in \mathbb{N},n\ge 2$.\\
		Do đó $u_5=3 \cdot 2^4=48$.}
\end{ex}
\begin{ex}%[1D3Y4-3]%Câu 20
	Cho cấp số nhân $\left(u_n\right)$, với $u_1=-9,u_4=\dfrac{1}{3}$. Công bội của cấp số nhân đã cho bằng
	\choice
		{\True $-\dfrac{1}{3}$}
		{$-3$}
		{$3$}
		{$\dfrac{1}{3}$}
	\loigiai{
		Nếu $\left(u_n\right)$ là cấp số nhân với công bội $q$ thì $u_n=u_1\cdot q^{n-1},\forall n \in \mathbb{N},n\ge 2$.\\
		Theo đề ta có $u_4=u_1\cdot q^3\Leftrightarrow \dfrac{1}{3}=-9\cdot q^3\Rightarrow q=-\dfrac{1}{3}$.}
\end{ex}

\Closesolutionfile{ans}
%======================
\subsection{Bảng đáp án}
\inputansbox{8}{ans/ANS-DANG-1}


