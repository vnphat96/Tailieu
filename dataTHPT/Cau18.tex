%Dạng 1
\setcounter {section} {17}
\setcounter{ex}{0}
\section{Phương trình đường thẳng}
\subsection{Kiến thức cần nhớ}
\begin{khung}
	\subsubsection{Phương trình đường thẳng}
	\begin{itemize}
		\item Đường thẳng $d$ đi qua điểm $M(x_0;y_0;z_0)$ và có véc-tơ chỉ phương (VTCP) $\vec{u}_d=(a_1;a_2;a_3)$ có phương trình tham số $\heva{&x=x_0+a_1t\\&y=y_0+a_2t\\&z=z_0+a_3t},\, ( t\in \mathbb{R})$.
		\item Điểm $M$ thuộc đường thẳng $d\Leftrightarrow M(x_0+at_1;y_0+at_2;z_0+at_3)$.
		\item Nếu $a_1\cdot a_2\cdot a_3 \ne 0$ thì $\dfrac{x-x_0}{a_1}=\dfrac{y-y0}{a_2}=\dfrac{z-z_0}{a_3}$ được gọi là phương trình chính tắc của $d$. 
	\end{itemize}
\end{khung}
\subsection{Bài tập mẫu}
\Opensolutionfile{ans}[ans/ANS-DANG-18]
\begin{khung}
	\begin{vd}%[Phạm Hoài, PT-ĐMH2023]%[2H3Y3-3]
		Trong không gian $O x y z$, cho đường thẳng $d: \dfrac{x-1}{2}=\dfrac{y-2}{-1}=\dfrac{z+3}{-2}$. Điểm nào dưới đây thuộc $d$?
		\choice
		{$P(1; 2; 3)$}
		{\True $Q(1; 2;-3)$}
		{$N(2; 1; 2)$}
		{$M(2;-1;-2)$}
		\loigiai{
		Thay toạ độ điểm $Q(1;2;-3)$ vào phương trình $d$ ta có\\ $\dfrac{1-1}{2}=\dfrac{2-2}{-1}=\dfrac{-3+3}{-2}$.
		Suy ra $Q\in d$.}
	\end{vd}
\end{khung}
\subsection{Bài tập tương tự và phát triển}
\begin{ex}%[Phạm Hoài, PT-ĐMH2023]%[2H3Y3-3]%Câu 1
	Cho đường thẳng $\Delta\colon\dfrac{x-2}{2}=\dfrac{y-3}{3}=\dfrac{z}{1}$. Khi đó $\Delta $ đi qua điểm $ M$ có tọa độ
	\choice
	{\True $\left(2;\,3;\,0\right)$}
	{$\left(0;\,0;\,1\right)$}
	{$\left(1;\,-1;\,2\right)$}
	{$\left(0;\,2;\,-1\right)$}
	\loigiai{
		Dễ thấy $\Delta $ đi qua điểm $ M\left(2;\,3;\,0\right)$.}
\end{ex}

\begin{ex}%[Phạm Hoài, PT-ĐMH2023]%[2H3Y3-3]%Câu 2
	Trong không gian với hệ tọa độ $ Oxyz$, cho đường thẳng $ d\colon \dfrac{x-1}{1}=\dfrac{y}{-2}=\dfrac{z-1}{2}$. 
	Điểm nào dưới đây không thuộc $ d?$
	\choice
	{$ N\left(1;0;1\right)$}
	{$ F\left(3;-4;5\right)$}
	{\True $ M\left(0;2;1\right)$}
	{$ E\left(2;-2;3\right)$}
	\loigiai{
		Thay tọa độ điểm $ E\left(2;-2;3\right)$ vào $ d$ ta được $\dfrac{2-1}{1}=\dfrac{-2}{-2}=\dfrac{3-1}{2}$ thỏa mãn nên loại $ E\left(2;-2;3\right)$.\\
		Thay tọa độ điểm $ N\left(1;0;1\right)$ vào $ d$ ta được $\dfrac{1-1}{1}=\dfrac{0}{-2}=\dfrac{1-1}{2} $ thỏa mãn nên loại $ N\left(1;0;1\right)$.\\
		Thay tọa độ điểm $ F\left(3;-4;5\right)$ vào $ d$ ta được $\dfrac{3-1}{1}=\dfrac{-4}{-2}=\dfrac{5-1}{2} $ thỏa mãn nên loại $ F\left(3;-4;5\right)$.\\
		Thay tọa độ điểm $ M\left(0;2;1\right)$ vào $ d$ ta được $\dfrac{0-1}{1}=\dfrac{2}{-2}=\dfrac{1-1}{2} $ không thỏa mãn nên chọn $ M\left(0;2;1\right)$.}
\end{ex}

\begin{ex}%[Phạm Hoài, PT-ĐMH2023]%[2H3Y3-3]%Câu 3
	Trong không gian $Oxyz$, cho đường thẳng $d\colon \dfrac{x+1}{2}=\dfrac{y-1}{3}=\dfrac{z+3}{-2}$. Điểm nào dưới đây thuộc $d$?
	\choice
	{$ N(2;3;-2)$}
	{$ M(1;1;-3)$}
	{$ Q(3;2;-2)$}
	{\True $ P(-1;1;-3)$}
	\loigiai{
		Đường thẳng $d$ đi qua điểm $ M(x_0;y_0;z_0)$ nhận véc-tơ chỉ phương $\overrightarrow{u}=(u_1;u_2;u_3)$ có phương trình\\
		$\dfrac{x-x_0}{u_1}=\dfrac{y-y_0}{u_2}=\dfrac{z-z_0}{u_3}$.\\
		Nên theo đề bài thì đường thẳng $d$ đi qua điểm $ P(-1;1;-3)$.}
\end{ex}

\begin{ex}%[Phạm Hoài, PT-ĐMH2023]%[2H3Y3-3]%Câu 4
	Trong không gian $Oxyz$ , điểm nào dưới đây thuộc đường thằng $d\colon \dfrac{x+2}{1}=\dfrac{y-1}{1}=\dfrac{z+2}{2}$ ?
	\choice
	{$P\left(1;1;2\right)$}
	{$N\left(2;-1;2\right)$}
	{\True $Q\left(-2;1;-2\right)$}
	{$M\left(-2;-2;1\right)$}
	\loigiai{
		Đường thằng $d\colon \dfrac{x+2}{1}=\dfrac{y-1}{1}=\dfrac{z+2}{2}$ đi qua điểm $\left(-2;1;-2\right)$.}
\end{ex}

\begin{ex}%[Phạm Hoài, PT-ĐMH2023]%[2H3Y3-3]%Câu 5
	Trong không gian $ Oxyz$, đường thẳng $ d\colon \dfrac{x-1}{1}=\dfrac{y-2}{-3}=\dfrac{z+3}{5}$ không đi qua điểm nào dưới đây?
	\choice
	{$ N(0;5;-8)$}
	{$ Q(1;2;-3)$}
	{$ M(2;-1;2)$}
	{\True $P(0;2;-8)$}
	\loigiai{
		Lần lượt thay tọa độ các điểm vào phương trình đường thẳng 
		\begin{itemize}
			\item $\dfrac{0-1}{1}=\dfrac{5-2}{-3}=\dfrac{-8+3}{5}=-1$. Suy ra $N\in d$.
			\item $\dfrac{1-1}{1}=\dfrac{2-2}{-3}=\dfrac{-3+3}{5}=0$. Suy ra  $Q\in d$.
			\item $\dfrac{2-1}{1}=\dfrac{-1-2}{-3}=\dfrac{2+3}{5}=1$. Suy ra $M\in d$.
			\item $\dfrac{0-1}{1}=\dfrac{2-2}{-3}\ne\dfrac{-8+3}{5}$. Suy ra $P\notin d$.
		\end{itemize}
		
		}
\end{ex}

\begin{ex}%[Phạm Hoài, PT-ĐMH2023]%[2H3Y3-3]%Câu 6
	Trong không gian với hệ trục toạ độ $Oxyz$ cho đường thẳng $\Delta $ có phương trình $\Delta \colon \dfrac{x-1}{2}=\dfrac{y-2}{-3}=\dfrac{z+3}{4}$. Đường thẳng $\Delta $ đi qua điểm $M$ nào bên dưới?
	\choice
	{$M\left(5;-4;7\right)$}
	{\True $M\left(-5;11;-15\right)$}
	{$M\left(-5;7;-12\right)$}
	{$M\left(5;4;-7\right)$}
	\loigiai{
		Thay tọa độ các điểm $ M$ trong bốn đáp án A, B, C, D vào phương trình đường thẳng $\Delta $ ta thấy điểm $M\left(-5;11;-15\right)$ thỏa mãn. Vậy đường thẳng $\Delta $ đi qua điểm $M\left(-5;11;-15\right)$.}
\end{ex}

\begin{ex}%[Phạm Hoài, PT-ĐMH2023]%[2H3Y3-3]%Câu 7
	Trong không gian $Oxyz$ , điểm nào dưới đây thuộc đường thẳng $d\colon\dfrac{x+2}{1}=\dfrac{y-1}{1}=\dfrac{z+2}{2}$.
	\choice
	{\True $Q\left(-2;1;-2\right)$}
	{$M\left(-2;-2;1\right)$}
	{$P\left(1;1;2\right)$}
	{$N\left(2;-1;2\right)$}
	\loigiai{
		Đường thằng $d\colon \dfrac{x+2}{1}=\dfrac{y-1}{1}=\dfrac{z+2}{2}$ đi qua điểm $\left(-2;1;-2\right)$ .}
\end{ex}

\begin{ex}%[Phạm Hoài, PT-ĐMH2023]%[2H3Y3-3]%Câu 8
	Trong không gian $ Oxyz$, đường thẳng $d\colon \heva{& x=2+3t\\&y=-1-4t\\ & z=5t}$ đi qua điểm nào sau đây?
	\choice
	{$M\left(5;5;5\right)$}
	{$M\left(3;-4;5\right)$}
	{\True $M\left(2;-1;0\right)$}
	{$M\left(8;,9;10\right)$}
	\loigiai{
		Thay $ t=0$ vào phương trình đường thẳng $d$ ta được $\heva{
			& x=2\\ 
			& y=-1\\ 
			& z=0.}$\\
		Do đó điểm $ M\left(2;-1;0\right)$ thuộc $d$ .}
\end{ex}

\begin{ex}%[Phạm Hoài, PT-ĐMH2023]%[2H3Y3-3]%Câu 9
	Trong không gian $Oxyz$ , điểm nào sau đây không nằm trên đường thẳng $\Delta \colon \dfrac{x-1}{2}=\dfrac{y+1}{1}=\dfrac{z}{3}$?
	\choice
	{$ M\left(1;-1;0\right)$}
	{$ N\left(3;0;3\right)$}
	{$ P\left(-3;-3;-6\right)$}
	{\True $ Q\left(5;1;5\right)$}
	\loigiai{
		Thay tọa độ $ M$ vào phương trình đường thẳng $\Delta $ ta được $ 0=0=0$ (đúng). Suy ra $M\in\Delta $.\\
		Thay tọa độ $ N$ vào phương trình đường thẳng $\Delta $ ta được $ 1=1=1$ (đúng). Suy ra $ N\in\Delta $.\\
		Thay tọa độ $ P$ vào phương trình đường thẳng $\Delta $ ta được $-2=-2=-2$ (đúng). Suy ra $P\in\Delta $.\\
		Thay tọa độ $ Q$ vào phương trình đường thẳng $\Delta $ ta được $ 2=2=\dfrac{5}{3}$ (Vô lý)\\
		$\Rightarrow Q\notin\Delta $.}
\end{ex}

\begin{ex}%[Phạm Hoài, PT-ĐMH2023]%[2H3Y3-3]%Câu 10
	Trong không gian $Oxyz$ , cho đường thẳng $(d) \colon \heva{& x=1-t\\ 
		& y=2t\\ 
		& z=2+t}$. Điểm nào dưới đây thuộc $(d)$?
	\choice
	{$ M\left(1;2;2\right)$}
	{\True $ N\left(0;2;3\right)$}
	{$ P\left(-1;4;2\right)$}
	{$ Q\left(-1;2;1\right)$}
	\loigiai{
		Thay lần lượt tọa độ các điểm vào phương trình đường thẳng $(d)$, ta thấy chỉ có đáp án $ N\left(0;2;3\right)$ thỏa mãn.}
\end{ex}

\begin{ex}%[Phạm Hoài, PT-ĐMH2023]%[2H3Y3-3]%Câu 11
	Trong không gian $Oxyz$, điểm nào dưới đây thuộc đường thẳng $ d\colon\dfrac{x-2}{-3}=\dfrac{y+1}{2}=\dfrac{z+3}{1}$?
	\choice
	{$\left(3;-2 ; 1\right)$}
	{\True $\left(2 ;-1 ;-3\right)$}
	{$\left(-2 ; 1 ; 3\right)$}
	{$\left(-3 ; 2 ; 1\right)$}
	\loigiai{
		Do đường thẳng $\Delta \colon \dfrac{x-x_0}{a}=\dfrac{y-y_0}{b}=\dfrac{z-z_0}{c}$ luôn đi qua điểm $ M\left(x_0;y_0;z_0\right)$ nên đường thẳng đã cho đi qua điểm $A\left(2;-1;-3\right)$ , các điểm còn lại không thuộc $ d$.}
\end{ex}

\begin{ex}%[Phạm Hoài, PT-ĐMH2023]%[2H3Y3-3]%Câu 12
	Trong không gian $ Oxyz$, điểm nào sau đây không thuộc đường thẳng $ d\colon \heva{
		& x=1+2t\\ 
		& y=3-4t\\ 
		& z=6-5t}$?
	\choice
	{\True $ P\left(-1;-3;-6\right)$}
	{$ Q\left(-1;7;11\right)$}
	{$ M\left(1;3;6\right)$}
	{$ N\left(3;-1;1\right)$}
	\loigiai{
		Điểm $ M,\, N$ và $ Q$ thuộc đường thẳng $ d$. Do đó loại $ Q\left(-1;7;11\right)$, $ M\left(1;3;6\right)$,  $ N\left(3;-1;1\right)$.
		Điểm $ P\left(-1;-3;-6\right)\notin d\Rightarrow $ chọn $ P\left(-1;-3;-6\right)$.}
\end{ex}

\begin{ex}%[Phạm Hoài, PT-ĐMH2023]%[2H3Y3-3]%Câu 13
	Trong không gian $ Oxyz$, điểm nào sau đây thuộc đường thẳng $ d\colon\dfrac{x+1}{-1}=\dfrac{y-2}{3}=\dfrac{z-1}{3}$
	\choice
	{$M(1;2;1)$}
	{$Q(1;-2;-1)$}
	{$N(-1;3;2)$}
	{\True $P(-1;2;1)$}
	\loigiai{
		Theo phương trình đường thẳng, đường thẳng $ d$ đi qua điểm $ P(-1;2;1)$.}
\end{ex}

\begin{ex}%[Phạm Hoài, PT-ĐMH2023]%[2H3Y3-3]%Câu 14
	Trong không gian $ Oxyz$, đường thẳng $d\colon \heva{
		& x=1+2t\\ 
		& y=2-3t\\ 
		& z=3-t} , \, t\in\mathbb{R}$ không đi qua điểm nào dưới đây?
	\choice
	{$ Q\left(1\,;2\,;3\right)$}
	{$M\left(3\,;-1\,;2\right)$}
	{\True $ P\left(2\,;-2\,;3\right)$}
	{$ N\left(-1\,;5\,;4\right)$}
	\loigiai{
		Thay tọa độ $4$ điểm $ M$, $N$, $P$, $Q$ vào phương trình đường thẳng $ d$ ta có:\\
		\begin{itemize}
			\item $\heva{
				& 1=1+2t\\ 
				& 2=2-3t\\ 
				& 3=3-t}\Rightarrow t=0\Rightarrow Q\in d$.
			\item $\heva{
				& 3=1+2t\\ 
				&-1=2-3t\\ 
				& 2=3-t}\Rightarrow t=1\Rightarrow M\in d$.
			\item $\heva{
				& 2=1+2t\\ 
				&-2=2-3t\\ 
				& 3=3-t}\Rightarrow \heva{
				& t=\dfrac{1}{2}\\ 
				& t=\dfrac{4}{3}\\ 
				& t=0}\Rightarrow $ hệ vô nghiệm $\Rightarrow P\notin d$ .\\
			\item $\heva{
				&-1=1+2t\\ 
				& 5=2-3t\\ 
				& 4=3-t}\Rightarrow t=-1\Rightarrow N\in d$
		\end{itemize}
	}
\end{ex}

\begin{ex}%[Phạm Hoài, PT-ĐMH2023]%Câu 15
	Trong không gian $ Oxyz$, điểm nào dưới đây thuộc đường thẳng $\dfrac{x-1}{2}=\dfrac{y+1}{-1}=\dfrac{z-2}{3}$?
	\choice
	{$ P\left(2;-1;3\right)$}
	{$ M\left(-1;1;-2\right)$}
	{\True $ N\left(1;-1;2\right)$}
	{$ Q\left(-2;1;-3\right)$}
	\loigiai{
		Xét điểm $ N\left(1;-1;2\right)$ ta có $\dfrac{1-1}{2}=\dfrac{-1+1}{-1}=\dfrac{2-2}{3}$ nên điểm $ N\left(1;-1;-2\right)$ thuộc đường thẳng đã cho.}
\end{ex}


\begin{ex}%[Phạm Hoài, PT-ĐMH2023]%[2H3Y3-3]%Câu 18
	Trong không gian $Oxyz$, điểm nào dưới đây thuộc đường thẳng $d\colon\heva{
		& x=1-t\\ 
		& y=5+t\\ 
		& z=2+3t}$ ?
	\choice
	{$P\left(1;2;5\right)$}
	{\True $N\left(1;5;2\right)$}
	{$Q\left(-1;1;3\right)$}
	{$M\left(1;1;3\right)$}
	\loigiai{
		\begin{itemize}
			\item \textbf{Cách 1.}
			Nếu $d$ qua $M\left(x_0;y_0;z_0\right)$ , có véctơ chỉ phương $\overrightarrow{u}\left(a;b;c\right)$ thì phương trình đường thẳng $d$ là\\
			 $\heva{
				& x=x_0+at\\ 
				& y=y_0+bt\\ 
				& z=z_0+ct}$ , ta chọn đáp án $N\left(1;5;2\right)$.
			\item \textbf{Cách 2.} Thay tọa độ các điểm vào phương trình đường thẳng $d$ , ta có\\
			$\heva{
				& 1=1-t\\ 
				& 2=5+t\\ 
				& 5=2+3t}\Leftrightarrow \heva{
				& t=0\\ 
				& t=-3\\ 
				& t=1}$ (Vô lý). \\
			Loại đáp án $P\left(1;2;5\right)$.\\
			Thay tọa độ điểm $N$ vào phương trình đường thẳng $d$ , ta có $\heva{
				& 1=1-t\\ 
				& 5=5+t\\ 
				& 2=2+3t}\Leftrightarrow t=0$. Nhận đáp án $N\left(1;5;2\right)$.
	\end{itemize}}
\end{ex}

\begin{ex}%[Phạm Hoài, PT-ĐMH2023]%Câu 16
	Trong không gian với hệ trục tọa độ $Oxyz$, đường thẳng $ d\colon \dfrac{x-1}{3}=\dfrac{y+2}{-4}=\dfrac{z-3}{-5}$ đi qua điểm
	\choice
	{$\left(-1;2;-3\right)$}
	{\True $\left(1;-2;3\right)$}
	{$\left(-3;4;5\right)$}
	{$\left(3;-4;-5\right)$}
	\loigiai{
	}
\end{ex}

\begin{ex}%[Phạm Hoài, PT-ĐMH2023]%Câu 17
	Trong không gian $Oxyz$, điểm nào dưới đây thuộc đường thẳng $d\colon \heva{&x=1-t\\&y=5+t\\&z=2+3t}$?
	\choice
	{$P(1;2;5)$}
	{\True $N(1;5;2)$ }
	{$M(1;1;3)$}
	{$Q(-1;1;3)$}
	\loigiai{
		Với $t=0\Rightarrow \heva{&x=1\\&y=5\\&z=2}\Rightarrow N(1;5;2)\in d$.}
\end{ex}

\begin{ex}%[Phạm Hoài, PT-ĐMH2023]%[2H3Y3-3]%Câu 18
	Trong không gian $Oxyz$, cho đường thẳng $d\colon \dfrac{x-3}{-1}=\dfrac{y-2}{3}=\dfrac{z+1}{-2}$. Điểm nào sau đây không thuộc $d$?
	\choice
	{$ P\left(3;2;-1\right)$}
	{\True $Q\left(-3;-2;1\right)$}
	{$ M\left(4;-1;1\right)$}
	{$ N\left(2;5;-3\right)$}
	\loigiai{
		Thay lần lượt các tọa độ trên vào phương trình đường thẳng $ d$, ta thấy $Q\left(-3;-2;1\right)$ là điểm không thuộc $ d$.}
\end{ex}

\begin{ex}%[Phạm Hoài, PT-ĐMH2023]%[2H3Y3-3]%Câu 19
	Trong không gian $ Oxyz$, đường thẳng $ d\colon \dfrac{x-1}{2}=\dfrac{y}{1}=\dfrac{z}{3}$ đi qua điểm nào dưới đây?
	\choice
	{$\left(3;2;3\right)$}
	{$\left(2;1;3\right)$}
	{$\left(3;1;2\right)$}
	{\True $\left(3;1;3\right)$}
	\loigiai{
		Thay $\heva{& x=3\\ 
			& y=1\\ 
			& z=3}$ vào phương trình đường thẳng $ d$ ta được $\dfrac{2}{2}=\dfrac{1}{1}=\dfrac{3}{3}$ (đúng).}
\end{ex}

\begin{ex}%[Phạm Hoài, PT-ĐMH2023]%[2H3Y3-3]%Câu 20
	Đường thẳng $ d\colon \dfrac{x-1}{2}=\dfrac{y-2}{1}=\dfrac{z+1}{-2}$ không đi qua điểm nào sau đây?
	\choice
	{$ M\left(1;2;-1\right)$}
	{\True $ M\left(1;2;1\right)$}
	{$ M\left(-1;1;1\right)$}
	{$ M\left(5;4;-5\right)$}
	\loigiai{
		
		Thay tọa độ các điểm vào phương trình đường thẳng $d$
		
		\begin{itemize}
			\item $\dfrac{1-1}{2}=\dfrac{2-2}{1}=\dfrac{-1+1}{-2}=0$. Suy ra $ M\in d$.
			\item  $\dfrac{1-1}{2}=\dfrac{2-2}{1}\ne\dfrac{1+1}{-2}$. Suy ra  $M\notin d$.
			\item $\dfrac{-1-1}{2}=\dfrac{1-2}{1}=\dfrac{1+1}{-2}$. Suy ra $M\in d$.
			\item $\dfrac{5-1}{2}=\dfrac{4-2}{1}=\dfrac{-5+1}{-2}$. Suy ra $ M\in d$.
		\end{itemize}}
	\end{ex}
	
	\begin{ex}%[Phạm Hoài, PT-ĐMH2023]%[2H3Y3-3]%Câu 21
		Trong không gian $ Oxyz$, cho đường thẳng $ d\colon \dfrac{x+2}{1}=\dfrac{y-3}{2}=\dfrac{z}{-5}$. Đường thẳng $ d$ không đi qua điểm nào sau đây?
		\choice
		{\True $ P\left(3;1;5\right)$}
		{$ Q\left(0;7;-10\right)$}
		{$ M\left(-2;3;0\right)$}
		{$ N\left(-1;5;-5\right)$}
		\loigiai{
			Đường thẳng có phương trình chính tắc $\dfrac{x-x_0}{a}=\dfrac{y-y_0}{b}=\dfrac{z-z_0}{c}$ đi qua điểm $ M\left(x_0;y_0;z_0\right)$, suy ra đường thẳng đã cho đi qua điểm $ M\left(-2;3;0\right)$.\\
			Thay tọa độ điểm $ N\left(-1;5;-5\right)$ vào phương trình đường thẳng $ d$ ta có\\ $\dfrac{-1+2}{1}=\dfrac{5-3}{2}=\dfrac{-5}{-5}=1$. \\
			Suy ra điểm $ N$ thuộc đường thẳng $ d$.\\
			Thay tọa độ điểm $ P\left(3;1;5\right)$ vào phương trình đường thẳng $ d$ ta có\\ $\dfrac{3+2}{1}\ne\dfrac{1-3}{2}$.\\
			 Suy ra điểm $ P$ không thuộc đường thẳng $ d$.}
	\end{ex}
	
	\begin{ex}%[Phạm Hoài, PT-ĐMH2023]%[2H3Y3-3]%Câu 22
		Trong không gian $Oxyz,$ điểm nào dưới đây thuộc đường thẳng\\
		$d\colon \heva{	& x=-2+t\\ 
			& y=2-2t\\ 
			& z=3+}$ ?
		\choice
		{$Q(0;1;4)$}
		{$M(3;2;-2)$}
		{$N(1;1;2)$}
		{\True $M(3;3;-6)$}
		\loigiai{
			Thay tọa độ của lần lượt các điểm đã cho vào phương trình đường thẳng $ d$ chỉ thấy tọa độ của điểm $ M$ thỏa mãn.}
	\end{ex}
	
	\begin{ex}%[Phạm Hoài, PT-ĐMH2023]%Câu 23
		Trong không gian với hệ tọa độ $ Oxyz$, cho đường thẳng $ d\colon \dfrac{x-1}{1}=\dfrac{y}{-2}=\dfrac{z-1}{2}$. Điểm nào dưới đây không thuộc $ d?$
		\choice
		{$ N(1;0;1)$}
		{$ F(3;-4;5)$}
		{\True $ M(0;2;1)$}
		{$ E(2;-2;3)$}
		\loigiai{
			Thay tọa độ các điểm $ E,N,F$ vào phương trình đường thẳng $ d$ ta thấy thỏa mãn nên $ E,N,F$ thuộc $ d.$\\
			Thay tọa độ các điểm $ M$vào phương trình đường thẳng $ d$ ta thấy không thỏa mãn nên $ M$ không thuộc $ d.$}
	\end{ex}
	
	\begin{ex}%[Phạm Hoài, PT-ĐMH2023]%[2H3Y3-3]%Câu 24
		Trong không gian tọa độ $Oxyz$, cho đường thẳng $d\colon \dfrac{x-1}{2}=\dfrac{y+1}{-1}=\dfrac{z+2}{-2}$. Điểm nào dưới đây KHÔNG thuộc đường thẳng $d$?
		\choice
		{$N\left(1;\,-1;\,-2\right)$}
		{$P\left(-1;\,0;\,0\right)$}
		{\True $Q\left(-3;\,1;\,-2\right)$}
		{$M\left(3;\,-2;\,-4\right)$}
		\loigiai{
			Thay lần lượt tọa độ các điểm vào phương trình đường thẳng $d$ ta thấy tọa độ của $Q$ không thỏa mãn phương trình $\left(\dfrac{-3-1}{2}=\dfrac{1+1}{-1}\ne\dfrac{-2+2}{-2}\right)$. \\Vậy điểm $Q$ không thuộc đường thẳng $d$ .}
	\end{ex}
	
	\begin{ex}%[Phạm Hoài, PT-ĐMH2023]%[2H3Y3-3]%Câu 25
		Trong không gian $ Oxyz$, điểm nào sau đây không thuộc đường thẳng
		$ d\colon \dfrac{x+2}{2}=\dfrac{y-1}{-3}=\dfrac{z-4}{2}$?
		\choice
		{\True $Q\left(2;-5;4\right)$}
		{$N\left(0;-2;6\right)$}
		{$P\left(4;-8;10\right)$}
		{$M\left(-2;1;4\right)$}
	\loigiai{
		Vì khi thay điểm $Q(2;-5;4)$ vào phương trình đường thẳng $d$ ta được\\
		$\dfrac{2+2}{2}=\dfrac{-5-1}{-3}=\dfrac{4-4}{2}$ (sai).\\
		}
\end{ex}
\Closesolutionfile{ans}
%======================
\subsection{Bảng đáp án}
\inputansbox{8}{ans/ANS-DANG-18}


