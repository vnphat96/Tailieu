%Dạng 1
\setcounter{ex}{0}
\section{Nguyên hàm}
\subsection{Kiến thức cần nhớ}
\begin{khung}
	\subsubsection{Định nghĩa}
	\begin{itemize}
		\item $F(x)$ là một nguyên hàm của $f(x)$ trên $K$ nếu $F'(x)=f(x),\, \forall x \in K$.\\
		Họ nguyên hàm của $f(x)$ trên $K$ là
		$\displaystyle\int f(x)\mathrm{\,d}x =F(x)+C$
			\end{itemize}
			\subsubsection{Tính chất}
			\begin{itemize}
		\item $\displaystyle\int f'(x) \mathrm{\,d}x = f(x)+C$.
		\item $\displaystyle\int k f(x) \mathrm{\,d}x = k \displaystyle\int f(x) \mathrm{\,d}x,\, \forall k \ne 0$. 
		\item $\displaystyle\int \left[f(x) \pm g(x)\right] \mathrm{\,d}x =  \displaystyle\int f(x) \mathrm{\,d}x \pm \displaystyle\int g(x) \mathrm{\,d}x $.   
			\end{itemize}
			\subsubsection{Một số công thức nguyên hàm cơ bản}
			\begin{itemize}
		\item $\displaystyle\int 0 \mathrm{\,d}x = C$,
		\item $\displaystyle\int 1 \mathrm{\,d}x = x+C$,
		\item $\displaystyle\int x^{n}\mathrm{\,d}x = \dfrac{x^{n+1}}{n+1}+C \longrightarrow \displaystyle\int \left(ax+b\right)^{n}\mathrm{\,d}x = \dfrac {1}{a} \cdot \dfrac{\left(ax+b\right)^{n+1}}{n+1}+C$,
		\item $\displaystyle\int \dfrac{1}{x}\mathrm{\,d}x = \ln \left|x\right|+C \longrightarrow \displaystyle\int \dfrac{1}{ax+b}\mathrm{\,d}x = \dfrac {1}{a} \ln \left|ax+b\right|+C$,
		\item $\displaystyle\int \dfrac{1}{x^2}\mathrm{\,d}x =-\dfrac{1}{x}+C \longrightarrow \displaystyle\int \dfrac{1}{\left(ax+b\right)^2}\mathrm{\,d}x = -\dfrac {1}{a} \cdot \dfrac{1}{ax+b}+C$,
		\item $\displaystyle\int \sin {x} \mathrm{\,d}x = -\cos {x}+C \longrightarrow \displaystyle\int \sin {\left(ax+b\right)}\mathrm{\,d}x = -\dfrac {1}{a} \cos {\left(ax+b\right)}+C$,
		\item $\displaystyle\int \cos {x} \mathrm{\,d}x = \sin {x}+C \longrightarrow \displaystyle\int \cos {\left(ax+b\right)}\mathrm{\,d}x = \dfrac {1}{a} \sin {\left(ax+b\right)}+C$,
		\item $\displaystyle\int \dfrac{1}{\cos^2 {x}} \mathrm{\,d}x = \tan {x}+C \longrightarrow \displaystyle\int \dfrac{1}{\cos^2 {\left(ax+b\right)}}\mathrm{\,d}x = \dfrac {1}{a} \tan {\left(ax+b\right)}+C$,
		\item $\displaystyle\int \dfrac{1}{\sin^2 {x}} \mathrm{\,d}x = -\cot {x}+C \longrightarrow \displaystyle\int \dfrac{1}{\sin^2 {\left(ax+b\right)}}\mathrm{\,d}x =- \dfrac {1}{a} \cot {\left(ax+b\right)}+C$,
		\item $\displaystyle\int \mathrm{e}^x \mathrm{\,d}x = \mathrm{e}^x+C \longrightarrow \displaystyle\int \mathrm{e}^{ax+b} = \dfrac {1}{a} \mathrm{e}^{ax+b}+C$,
		\item $\displaystyle\int a^x \mathrm{\,d}x = \dfrac{a^x}{\ln {a}}+C \longrightarrow \displaystyle\int a^{mx+n} \mathrm{\,d}x = \dfrac {1}{m} \cdot \dfrac{a^{mx+n}}{\ln {a}}+C$.
		\end{itemize}
\end{khung}
\subsection{Bài tập mẫu}
\Opensolutionfile{ans}[ans/ANS-DANG-1]
\begin{khung}
	\begin{vd}[Đề minh họa BGD 2022-2023]%[2D3Y1-1]
		Cho hàm số $f(x) = \cos {x} +x$. Khẳng định nào dưới đây đúng?
		\choice
		{$\displaystyle\int f(x) \mathrm{\,d}x = -\sin {x}+ x^2+ C$}	
		{$\displaystyle\int f(x) \mathrm{\,d}x = \sin {x}+ x^2+ C$}
		{$\displaystyle\int f(x) \mathrm{\,d}x = -\sin {x}+ \dfrac{x^2}{2} + C$}
		{\True $\displaystyle\int f(x) \mathrm{\,d}x = \sin {x}+ \dfrac{x^2}{2} + C$}
		\loigiai{
		 Ta có $\displaystyle\int f(x) \mathrm{\,d}x = \displaystyle\int \left(\cos {x} + x\right)  \mathrm{\,d}x = \sin {x}+ \dfrac{x^2}{2} + C$.
		}
	\end{vd}
\end{khung}
\subsection{Bài tập tương tự và phát triển}
\begin{ex}%[2D3Y1-1]
	Họ nguyên hàm của hàm $f(x) = \sin {x}$ là 
	\choice
	{$\displaystyle\int f(x) \mathrm{\,d}x = \cos {x} + C$}	
	{$\displaystyle\int f(x) \mathrm{\,d}x = -\sin {x} + C$}
	{\True $\displaystyle\int f(x) \mathrm{\,d}x = -\cos {x} + C$}
	{$\displaystyle\int f(x) \mathrm{\,d}x = \sin {x} + C$}
\loigiai{
	Ta có $\displaystyle\int f(x) \mathrm{\,d}x = \displaystyle\int \sin {x} \mathrm{\,d}x = -\cos {x} + C$.
}
\end{ex}
\begin{ex}%[2D3Y1-1]
	Họ nguyên hàm của hàm $f(x) = 4^{x}$ là 
	\choice
	{$\displaystyle\int f(x) \mathrm{\,d}x = 4^{x} \ln {4} + C$}	
	{$\displaystyle\int f(x) \mathrm{\,d}x = 4^{x+1} + C$}
	{ $\displaystyle\int f(x) \mathrm{\,d}x = \dfrac {4^{x+1}}{x+1} + C$}
	{\True $\displaystyle\int f(x) \mathrm{\,d}x = \dfrac {4^{x}}{\ln {4}} + C$}
	\loigiai{
	Ta có $\displaystyle\int f(x) \mathrm{\,d}x = \displaystyle\int 4^{x} \mathrm{\,d}x = \dfrac {4^{x}}{\ln {4}} + C$.
	}
\end{ex}
\begin{ex}%[2D3Y1-1]
	Họ nguyên hàm của hàm $f(x) = \cos {2x}$ là 
	\choice
	{$\displaystyle\int f(x) \mathrm{\,d}x = 2\sin {2x} + C$}	
	{\True $\displaystyle\int f(x) \mathrm{\,d}x = \dfrac{1}{2} \sin{2x} + C$}
	{$\displaystyle\int f(x) \mathrm{\,d}x = \dfrac {1}{2} \cos {2x} + C$}
	{$\displaystyle\int f(x) \mathrm{\,d}x = -\dfrac{1}{2} \sin {2x} + C$}
	\loigiai{
		Ta có $\displaystyle\int f(x) \mathrm{\,d}x = \displaystyle\int \cos{2x} \mathrm{\,d}x = \dfrac {1}{2} \sin {2x} + C$.
	}
\end{ex}
\begin{ex}%[2D3Y1-1]
Họ nguyên hàm của hàm $f(x) = 2x +3$ là 
\choice
{$\displaystyle\int f(x) \mathrm{\,d}x = 2x^2+3x + C$}	
{$\displaystyle\int f(x) \mathrm{\,d}x = x^2 + C$}
{\True $\displaystyle\int f(x) \mathrm{\,d}x = x^2+3x + C$}
{$\displaystyle\int f(x) \mathrm{\,d}x = 2x^2 + C$}
\loigiai{
	Ta có $\displaystyle\int f(x) \mathrm{\,d}x = \displaystyle\int \left(2x+3\right) \mathrm{\,d}x = x^2+3x + C$.
}
\end{ex}
\begin{ex}%[2D3Y1-1]
Họ nguyên hàm của hàm $f(x) = \mathrm{e}^{2x} -2x$ là 
\choice
{$\displaystyle\int f(x) \mathrm{\,d}x = \dfrac {1}{2x+1} \mathrm {e}^{2x} - x^2 + C$}	
{\True $\displaystyle\int f(x) \mathrm{\,d}x = \dfrac {1}{2} \mathrm {e}^{2x} - x^2 + C$}
{$\displaystyle\int f(x) \mathrm{\,d}x = \mathrm {e}^{2x} - x^2 + C$}
{$\displaystyle\int f(x) \mathrm{\,d}x = 2 \mathrm {e}^{2x} - 2 + C$}
\loigiai{
	Ta có $\displaystyle\int f(x) \mathrm{\,d}x = \displaystyle\int \left(\mathrm {e}^{2x} -2x\right) \mathrm{\,d}x = \dfrac{1}{2} \mathrm{e}^{2x} - x^2 + C$.
}
\end{ex}
\begin{ex}%[2D3Y1-1]
Họ nguyên hàm của hàm $f(x) = \dfrac {1}{2x+1} $ là 
\choice
{$\displaystyle\int f(x) \mathrm{\,d}x = -\dfrac {1}{2} \ln \left|2x+1\right| + C$}	
{$\displaystyle\int f(x) \mathrm{\,d}x = -\ln \left|2x+1\right| + C$}
{\True $\displaystyle\int f(x) \mathrm{\,d}x = \dfrac {1}{2} \ln \left|2x+1\right| + C$}
{$\displaystyle\int f(x) \mathrm{\,d}x = \ln \left|2x+1\right| + C$}
\loigiai{
Ta có 	$\displaystyle\int f(x) \mathrm{\,d}x = \displaystyle\int \dfrac {1}{2x+1} \mathrm{\,d}x = \dfrac{1}{2} \ln \left|2x+1\right| + C$.
}
\end{ex}
\begin{ex}%[2D3Y1-1]
	Trong các khẳng định sau, khẳng định nào \textbf{sai} ?
	\choice
{$\displaystyle\int 0 \mathrm{\,d}x = C$}	
{$\displaystyle\int \mathrm{e}^x \mathrm{\,d}x = \mathrm {e}^x + C$}
{$\displaystyle\int \mathrm{\,d}x = x + C$}
{\True $\displaystyle\int x^{n} \mathrm{\,d}x = \dfrac{x^{n+1}}{n+1}+ C$}
\loigiai{
Ta có 	$\displaystyle\int x^{n} \mathrm{\,d}x = \dfrac{x^{n+1}}{n+1}+ C$ chỉ đúng khi $n \ne -1$.
}
\end{ex}
\begin{ex}%[2D3Y1-1]
	Họ nguyên hàm của hàm $f(x) = \sin {\left(2x+1\right)}$ là 
	\choice
	{\True $\displaystyle\int f(x) \mathrm{\,d}x = -\dfrac{1}{2} \cos {\left(2x+1\right)} + C$}	
	{$\displaystyle\int f(x) \mathrm{\,d}x = 2 \cos {\left(2x+1\right)} + C$}
	{$\displaystyle\int f(x) \mathrm{\,d}x = -2\cos {\left(2x+1\right)} + C$}
	{$\displaystyle\int f(x) \mathrm{\,d}x = \dfrac{1}{2} \cos {\left(2x+1\right)} + C$}
	\loigiai{
		Ta có $\displaystyle\int f(x) \mathrm{\,d}x = \displaystyle\int \sin {\left(2x+1\right)} \mathrm{\,d}x = -\dfrac {1}{2} \cos {\left(2x+1\right)} + C$.
	}
\end{ex}
\begin{ex}%[2D3Y1-1]
		Họ nguyên hàm của hàm $f(x) = -4\sin {2x} +2\cos {x}- \mathrm {e}^x$ là 
	\choice
	{$\displaystyle\int f(x) \mathrm{\,d}x = 4\cos {2x} -2\sin {x}- \mathrm {e}^x + C$}	
	{\True $\displaystyle\int f(x) \mathrm{\,d}x = 2\cos {2x} +2\sin {x}- \mathrm {e}^x + C$}
	{$\displaystyle\int f(x) \mathrm{\,d}x = -8\cos {2x} +2\sin {x}- \mathrm {e}^x + C$}
	{$\displaystyle\int f(x) \mathrm{\,d}x = 8\cos {2x} +2\sin {x}- \mathrm {e}^x + C$}
	\loigiai{
		Ta có $\displaystyle\int f(x) \mathrm{\,d}x = \displaystyle\int \left(-4\sin {2x} +2\cos {x}- \mathrm {e}^x\right) \mathrm{\,d}x = 2\cos {2x} +2\sin {x}- \mathrm {e}^x + C$.
	}
\end{ex}
\begin{ex}%[2D3Y1-1]
		Trong các khẳng định sau, khẳng định nào \textbf{đúng}?
	\choice
	{$\displaystyle\int \dfrac{1}{x} \mathrm{\,d}x = \ln {x} + C$}	
	{$\displaystyle\int \dfrac{1}{\sin^2 {x}} \mathrm{\,d}x = \cot {x} + C$}
	{$\displaystyle\int \cos {x} \mathrm{\,d}x = -\sin {x} + C$}
	{\True $\displaystyle\int \left(2^{x}+ \mathrm{e}^x\right) \mathrm{\,d}x = \dfrac{2^{x}}{\ln {2}}+ \mathrm{e}^x + C$}
	\loigiai{
		Ta có 
		\begin{itemize}
			\item 	$\displaystyle\int \dfrac{1}{x} \mathrm{\,d}x = \ln {\left|x\right|} + C$
			\item $\displaystyle\int \dfrac{1}{\sin^2 {x}} \mathrm{\,d}x = -\cot {x} + C$\\
			\item $\displaystyle\int \cos {x} \mathrm{\,d}x = \sin {x} + C$
			\item $\displaystyle\int \left(2^{x}+ \mathrm{e}^x\right) \mathrm{\,d}x = \dfrac{2^{x}}{\ln {2}}+ \mathrm{e}^x + C$
		\end{itemize}
		}
\end{ex}
\begin{ex}%[2D3Y1-1]
	Họ nguyên hàm của hàm $f(x) = x^3 -3x^2 +5$ là 
	\choice
	{\True $\displaystyle\int f(x) \mathrm{\,d}x = \dfrac {x^4}{4} -x^3 + 5x + C$}	
	{$\displaystyle\int f(x) \mathrm{\,d}x = x^4 -x^3+5x+C$}
	{$\displaystyle\int f(x) \mathrm{\,d}x = 3x^2-6x + C$}
	{$\displaystyle\int f(x) \mathrm{\,d}x = x^4 - \dfrac{1}{3} x^3 +5x + C$}
	\loigiai{
		Ta có $\displaystyle\int f(x) \mathrm{\,d}x = \displaystyle\int \left(x^3-3x^2+5\right) \mathrm{\,d}x = \dfrac {x^4}{4} -x^3 + 5x + C$.
	}
\end{ex}
\begin{ex}%[2D3Y1-1]
	Trong các khẳng định sau, khẳng định nào \textbf{sai}?
	\choice
	{$\displaystyle\int x^{\mathrm{e}} \mathrm{\,d}x =\dfrac{x^{\mathrm{e}+1}}{\mathrm{e}+1} + C$}	
	{$\displaystyle\int x^2 \mathrm{\,d}x = \dfrac{1}{3} x^3 + C$}
	{\True $\displaystyle\int \mathrm {e}^{x} \mathrm{\,d}x = \dfrac{\mathrm{e}^{x+1}}{x+1} + C$}
	{$\displaystyle\int x^7 \mathrm{\,d}x = \dfrac{1}{8} x^8 + C$}
	\loigiai{
	Ta có $\displaystyle\int \mathrm{e}^x \mathrm{\,d}x = \mathrm{e}^x + C$.
	}
\end{ex}
\begin{ex}%[2D3Y1-1]
	$F(x)= \sin {2x}$ là nguyên hàm của hàm số nào dưới đây?
\choice
{$f(x)= \cos {2x}$}	
{\True $f(x)=2\cos {2x} $}
{$f(x)=-2 \cos {2x}$}
{$f(x) = -\dfrac{1}{2} \cos {2x}$}
\loigiai{
	Ta có $F'(x) = \left(\sin {2x}\right)' = 2\cos {2x}$, nên $F(x)=\sin {2x}$ là nguyên hàm của $f(x)=2\cos {2x}$.
}
\end{ex}
\begin{ex}%[2D3Y1-1]
	Họ nguyên hàm của hàm $f(x) = 2x + \dfrac {1}{x} $ là 
	\choice
	{$\displaystyle\int f(x) \mathrm{\,d}x = 4x^2 -\dfrac {1}{x^2}  + C$}	
	{$\displaystyle\int f(x) \mathrm{\,d}x = x^2 -\dfrac {1}{x^2}  + C$}
	{$\displaystyle\int f(x) \mathrm{\,d}x = 4x^2 + \ln \left|x\right| + C$}
	{\True $\displaystyle\int f(x) \mathrm{\,d}x = x^2 + \ln \left|x\right| + C$}
	\loigiai{
		$\displaystyle\int f(x) \mathrm{\,d}x = \displaystyle\int \left(2x + \dfrac {1}{x} \right)\mathrm{\,d}x = x^2 + \ln \left|x\right| + C$.
	}
\end{ex}
\begin{ex}%[2D3Y1-1]
	Họ nguyên hàm của hàm $f(x) = x^3 - \dfrac {2}{x} + \sqrt{x}$ là 
	\choice
	{$\displaystyle\int f(x) \mathrm{\,d}x = \dfrac {1}{4} x^4 -2\ln \left|x\right| -\dfrac{2}{3} \sqrt{x^3} + C$}	
	{$\displaystyle\int f(x) \mathrm{\,d}x = \dfrac {1}{4} x^4 + 2\ln \left|x\right| + \dfrac{2}{3} \sqrt{x^3} + C$}
	{\True $\displaystyle\int f(x) \mathrm{\,d}x = \dfrac {1}{4} x^4 -2\ln \left|x\right| + \dfrac{2}{3} \sqrt{x^3} + C$}
	{$\displaystyle\int f(x) \mathrm{\,d}x = \dfrac {1}{4} x^4 + 2\ln \left|x\right| -\dfrac{2}{3} \sqrt{x^3} + C$}
	\loigiai{
		Ta có $\displaystyle\int f(x) \mathrm{\,d}x = \displaystyle\int \left(x^3 - \dfrac {2}{x} + \sqrt{x} \right)\mathrm{\,d}x = \dfrac {1}{4} x^4 -2\ln \left|x\right| + \dfrac{2}{3} \sqrt{x^3} + C$.
	}
\end{ex}
\begin{ex}%[2D3Y1-1]
	Họ nguyên hàm của hàm $f(x) = \sin {3x}+ \cos {4x}$ là 
	\choice
	{$\displaystyle\int f(x) \mathrm{\,d}x = -\dfrac{1}{3}\cos {x}+ \dfrac{1}{4}\sin {x} + C$}	
	{\True $\displaystyle\int f(x) \mathrm{\,d}x = -\dfrac{1}{3}\cos {3x}+ \dfrac{1}{4}\sin {4x} + C$}
	{$\displaystyle\int f(x) \mathrm{\,d}x = 3\cos {3x} - 4\sin {4x} + C$}
	{$\displaystyle\int f(x) \mathrm{\,d}x = \dfrac{1}{3}\cos {3x} - \dfrac{1}{4}\sin {4x} + C$}
	\loigiai{
	Ta có 	$\displaystyle\int f(x) \mathrm{\,d}x = \displaystyle\int \left(\sin {3x}+ \cos {4x}\right) \mathrm{\,d}x = -\dfrac{1}{3}\cos {3x}+ \dfrac{1}{4}\sin {4x} + C$.
	}
\end{ex}
\begin{ex}%[2D3Y1-1]
	Họ nguyên hàm của hàm $f(x) = \cos \left(2x + \dfrac{\pi}{6}\right)$ là 
	\choice
	{$\displaystyle\int f(x) \mathrm{\,d}x = \dfrac{1}{6}\sin \left(2x + \dfrac{\pi}{6}\right) + C$}	
	{\True $\displaystyle\int f(x) \mathrm{\,d}x = \dfrac{1}{2}\sin \left(2x + \dfrac{\pi}{6}\right) + C$}
	{$\displaystyle\int f(x) \mathrm{\,d}x = \sin \left(2x + \dfrac{\pi}{6}\right) + C$}
	{$\displaystyle\int f(x) \mathrm{\,d}x = -\dfrac{1}{2}\sin \left(2x + \dfrac{\pi}{6}\right) + C$}
	\loigiai{
	Ta có 	$\displaystyle\int f(x) \mathrm{\,d}x = \displaystyle\int \cos \left(2x + \dfrac{\pi}{6}\right) \mathrm{\,d}x = \dfrac{1}{2}\sin \left(2x + \dfrac{\pi}{6}\right) + C$.
	}
\end{ex}
\begin{ex}%[2D3Y1-1]
	Họ nguyên hàm của hàm $f(x) = 2x^2+x+1$ là 
	\choice
	{\True $\displaystyle\int f(x) \mathrm{\,d}x = \dfrac{2x^3}{3}+ \dfrac{x^2}{2} + x+C$}	
	{$\displaystyle\int f(x) \mathrm{\,d}x = 4x+1 + C$}
	{$\displaystyle\int f(x) \mathrm{\,d}x = \dfrac{2x^3}{3}+ x^2 + x + C$}
	{$\displaystyle\int f(x) \mathrm{\,d}x = \dfrac{2x^3}{3} + \dfrac{x^2}{2} + x$}
	\loigiai{
		Ta có $\displaystyle\int f(x) \mathrm{\,d}x = \displaystyle\int  \left(2x^2+x+1\right) \mathrm{\,d}x = \dfrac{2x^3}{3}+ \dfrac{x^2}{2} + x+C$.
	}
\end{ex}
\begin{ex}%[2D3Y1-1]
	Họ nguyên hàm của hàm $f(x) = 7^{x}$ là 
	\choice
	{\True $\displaystyle\int f(x) \mathrm{\,d}x = \dfrac{7^x}{\ln {7}} +C$}	
	{$\displaystyle\int f(x) \mathrm{\,d}x = \dfrac{7^{x+1}}{x+1}+C$}
	{$\displaystyle\int f(x) \mathrm{\,d}x = 7^x \ln {7} + C$}
	{$\displaystyle\int f(x) \mathrm{\,d}x = 7^{x+1}+C$}
	\loigiai{
		Ta có $\displaystyle\int f(x) \mathrm{\,d}x = \displaystyle\int  7^x \mathrm{\,d}x = \dfrac{7^x}{\ln {7}} +C$.
	}
\end{ex}
\begin{ex}%[2D3Y1-1]
	Họ nguyên hàm của hàm $f(x) = \dfrac {1}{1+x} $ là 
	\choice
	{$\displaystyle\int f(x) \mathrm{\,d}x = -\dfrac {1}{{\left(1+x\right)}^2} + C$}	
	{$\displaystyle\int f(x) \mathrm{\,d}x = \ln \left(1+x\right) + C$}
	{$\displaystyle\int f(x) \mathrm{\,d}x = \log \left|1+x\right| + C$}
	{\True $\displaystyle\int f(x) \mathrm{\,d}x = \ln \left|1+x\right| + C$}
	\loigiai{
		Ta có $\displaystyle\int f(x) \mathrm{\,d}x = \displaystyle\int \dfrac {1}{1+x} \mathrm{\,d}x = \ln \left|1+x\right| + C$.
	}
\end{ex}
\Closesolutionfile{ans}
%======================
\subsection{Bảng đáp án}
\inputansbox{8}{ans/ANS-DANG-1}
