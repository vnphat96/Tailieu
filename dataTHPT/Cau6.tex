\setcounter {section} {5}
\setcounter{ex}{0}
\section{Phương trình mặt phẳng}
\subsection{Kiến thức cần nhớ}
\begin{khung}
	\subsubsection{Phương trình mặt phẳng}
	\begin{itemize}
		\item Trong không gian, véc-tơ $\overrightarrow{n}$ khác $\overrightarrow{0}$ là véc-tơ pháp tuyến của mặt phẳng $(P)$ nếu giá của nó vuông góc với mặt phẳng $(P)$. Hơn nữa với $k \ne 0$ ta cũng có $k\overrightarrow{n}$ cũng là một véc-tơ pháp tuyến của $(P)$.
		\item Trong không gian $Oxyz$. Đường thẳng $(d)$ đi qua điểm $A(x_0;y_0;z_0)$ và nhận $\overrightarrow{n}=(a;b;c)$ làm véc-tơ pháp tuyến có phương trình tổng quát là 
		\begin{eqnarray*}
			a(x-x_0)+b(y-y_0)+c(z-z_0)=0.
		\end{eqnarray*}
		\item Trong không gian $Oxyz$. Phương trình $$ax+by+cz+d=0$$ (với $a$, $b$, $c$ không đồng thời bằng $0$) là phương trình của một đường thẳng nào đó có véc-tơ pháp tuyến là $\overrightarrow{n}=(a;b;c)$.
	\end{itemize}
\subsubsection{Phương trình đường thẳng}
\begin{itemize}
	\item Trong không gian, véc-tơ $\overrightarrow{u}$ khác $\overrightarrow{0}$ là véc-tơ chỉ phương của đường thẳng $d$ nếu giá của nó song song với đường thẳng $d$. Hơn nữa với $k \ne 0$ ta cũng có $k\overrightarrow{n}$ cũng là một véc-tơ chỉ phương của  đường thẳng $d$.
	\item Trong không gian $Oxyz$ Đường thẳng $d$ đi qua điểm $A(x_0;y_0;z_0)$ và nhận $\overrightarrow{u}=(a;b;c)$ làm véc-tơ chỉ phương có phương trình là
	\begin{eqnarray*}
	&&\text{Phương trình tham số} \ d \colon	\heva{&x=x_0+at \\ &y=y_0+bt \\ &z=z_0+ct} \\
	&&\text{Phương trình chính tắc} \ d \colon \dfrac{x-x_0}{a}=\dfrac{y-y_0}{b}=\dfrac{z-z_0}{c} \ (\text{với $abc \ne 0$}).
	\end{eqnarray*} 
\end{itemize}
	\end{khung}

\subsection{Bài tập mẫu}
\Opensolutionfile{ans}[ans/ANS-DANG-6]
\begin{khung}
\begin{vd}[Đề minh họa BGD 2022-2023]%[Nguyễn Ngọc Nguyên]% [2H3Y2-2]
	Trong không gian $Oxyz$, mặt phẳng $(P)\colon x+y+z+1=0$ có một véc-tơ pháp tuyến là
	\choice
	{$\overrightarrow{n_1}=(-1;1;1)$}
	{$\overrightarrow{n_4}=(1;1;-1)$}
	{\True $\overrightarrow{n_3}=(1;1;1)$}
	{$\overrightarrow{n_2}=(1;-1;1)$}
	\loigiai{
		Mặt phẳng $(P)\colon x+y+z+1=0$ có một véc-tơ pháp tuyến là $\overrightarrow{n}=(1;1;1)$.
	}
\end{vd}
\end{khung}

\subsection{Bài tập tương tự và phát triển}
\begin{ex}%[Nguyễn Ngọc Nguyên]% [2H3Y2-2]
	Trong không gian $Oxyz$. Mặt phẳng $(Oxy)$ có một véc-tơ pháp tuyến là
	\choice
	{$\overrightarrow{i}=(1;0;0)$}
	{$\overrightarrow{j}=(0;1;0)$}
	{\True $\overrightarrow{k}=(0;0;1)$}
	{$\overrightarrow{t}=(1;1;1)$}
	\loigiai{
		Ta có $Oz \perp (Oxy)$ do đó véc-tơ $\overrightarrow{k}=(0;0;1)$ là véc-tơ pháp tuyến của mặt phẳng $(Oxy)$.
	}
\end{ex}
\begin{ex}%[Nguyễn Ngọc Nguyên]% [2H3Y2-2]
	Trong không gian $Oxyz$, cho mặt phẳng $(P) \colon 4x-2y+z-1=0$. Véc-tơ nào dưới đây là một véc-tơ pháp tuyến của $(P)$?
	\choice
	{$\overrightarrow{n_1}=(4;-2;-1)$}
	{$\overrightarrow{n_4}=(4;2;1)$}
	{$\overrightarrow{n_3}=(4;-2;0)$}
	{\True $\overrightarrow{n_2}(4;-2;1)$}
	\loigiai{
Mặt phẳng $(P) \colon 4x-2y+z-1=0$ suy ra $(P)$ có một véc-tơ pháp tuyến là $\overrightarrow{n_2}=(4;-2;1)$. 		
	}
\end{ex}

\begin{ex}%[Nguyễn Ngọc Nguyên]% [2H3Y2-2]
	Trong không gian với hệ tọa độ $Oxyz$, cho mặt phẳng $(\alpha) \colon 2x-y+3z-1=0$. Véc-tơ nào sau đây là véc-tơ pháp tuyến của mặt phẳng $(\alpha)$?
	\choice
	{$\overrightarrow{n}=(2;1;3)$}
	{\True $\overrightarrow{n}=(-4;2;-6)$}
	{$\overrightarrow{n}=(2;1;-3)$}
	{$\overrightarrow{n}=(-2;1;3)$}
	\loigiai{
Mặt phẳng $(\alpha)$ có véc-tơ pháp tuyến $\overrightarrow{n}=(2;-1;3)$ nên $(\alpha)$ cũng nhận véc-tơ $-2\overrightarrow{n}=(-4;2;-6)$ làm véc-tơ pháp tuyến.
}
\end{ex}

\begin{ex}%[Nguyễn Ngọc Nguyên]% [2H3Y2-2]
		Trong hệ trục tọa độ $Oxyz$, cho mặt phẳng $(P)$ có phương trình $3x-z+1=0$. Véc-tơ pháp tuyến của mặt phẳng $(P)$ có tọa độ là
	\choice
	{$(-3;1;1)$}
	{\True $(3;0;-1)$}
	{$(3;-1;1)$}
	{$(3;-1;0)$}
	\loigiai{
Mặt phẳng $(P)$ có một véc-tơ pháp tuyến là $\overrightarrow{n}=(3;0;-1)$.		
	}
\end{ex}


\begin{ex}%[Nguyễn Ngọc Nguyên]% [2H3Y2-2]
	Trong không gian tọa độ $Oxyz$, mặt phẳng $(Q) \colon x-2y+5z+2023=0$ có một véc-tơ pháp tuyến là
	\choice
	{$\overrightarrow{n_2}=(3;6;15)$}
	{$\overrightarrow{n_3}=(-1;2;5)$}
	{\True $\overrightarrow{n_1}=(-2;4;-10)$}
	{$\overrightarrow{n_4}=(-2;4;10)$}
	\loigiai{
Mặt phẳng $(Q)$ có một véc-tơ pháp tuyến là $\overrightarrow{n}=(1;-2;5)$ nên $-2\overrightarrow{n}=(-2;4;-10)$ cũng là một véc-tơ pháp tuyến của $(Q)$.		
	}
\end{ex}

\begin{ex}%[Nguyễn Ngọc Nguyên]% [2H3Y2-2]
Trong không gian $Oxyz$, mặt phẳng $(P) \colon 2x+y+3z-1=0$ có một véc-tơ pháp tuyến là
	\choice
	{$\overrightarrow{n_3}=(2;1;3)$}
	{$\overrightarrow{n_2}=(-1;3;2)$}
	{$\overrightarrow{n_4}=(1;3;2)$}
	{$\overrightarrow{n_1}=(3;1;2)$}
	\loigiai{
	Mặt phẳng $(P)$ có một véc-tơ pháp tuyến là $\overrightarrow{n}=(2;1;3)$.
	}
\end{ex}

\begin{ex}%[Nguyễn Ngọc Nguyên]% [2H3Y2-2]
	Trong không gian với hệ trục tọa độ $Oxyz$, véc-tơ nào sau đây không phải véc-tơ pháp tuyến của mặt phẳng $(P) \colon x+3y-5z+2=0$?
	\choice
	{$\overrightarrow{n}=(2;6;-10)$}
	{\True $\overrightarrow{n}=(-2;-6;-10)$}
	{$\overrightarrow{n}=(-3;-9;15)$}
	{$\overrightarrow{n}=(-1;-3;-5)$}
	\loigiai{
Mặt phẳng $(P)$ có một véc-tơ pháp tuyến là $\overrightarrow{n_1}=(1;;3;-5)$. \\
Ta có $\overrightarrow{n}=(-2;-6;-10)$ không cùng phương với $\overrightarrow{n_1}$ vì $\dfrac{1}{-2}=\dfrac{3}{-6} \ne \dfrac{-5}{-10}$. \\
Do đó $\overrightarrow{n}=(-2;-6;-10)$ không là véc-tơ pháp tuyến của $(P)$.		
	}
\end{ex}
 \begin{ex}%[Nguyễn Ngọc Nguyên]%[2H3Y2-2]
 		Trong không gian với hệ trục tọa độ $Oxyz$, phương trình mặt phẳng nào sau đây nhận véc-tơ $\overrightarrow{n}=(2;1;-1)$ làm véc-tơ pháp tuyến?
 	\choice
 	{\True $2x+y-z-1=0$}
 	{$2x+y+z-1=0$}
 	{$4x+2y-z-1=0$}
 	{$-2x-y-z+1=0$}
 	\loigiai{
 Mặt phẳng $(P) \colon 2x+y-z-1=0$ có một véc-tơ pháp tuyến là $\overrightarrow{n}=(2;1;-1)$.		
 	}
 \end{ex}

\begin{ex}%[Nguyễn Ngọc Nguyên]% [2H3Y2-2]
	Trong không gian $Oxyz$, cho mặt phẳng $(P) \colon \dfrac{x}{3}+\dfrac{y}{2}+\dfrac{z}{1}=1$. Véc-tơ nào dưới đây là véc-tơ pháp tuyến của mặt phẳng $(P)$?
	\choice
	{$\overrightarrow{n}=(3;2;6)$}
	{\True $\overrightarrow{n}=(2;3;6)$}
	{$\overrightarrow{n}=(3;2;1)$}
	{$\overrightarrow{n}=(3;-2;-2)$}
	\loigiai{
Ta có $(P) \colon \dfrac{x}{3}+\dfrac{y}{2}+\dfrac{z}{1}=1 \Leftrightarrow 2x+3y+6z-6=0$ có một véc-tơ pháp tuyến là $\overrightarrow{n}=(2;3;6)$.		
	}
\end{ex}

\begin{ex}%[Nguyễn Ngọc Nguyên]% [2H3B2-2]
	Trong không gian $Oxyz$, cho các điểm $A(-1;1;3)$, $B(2;1;0)$ và $C(4;-1;5)$. Một véc-tơ pháp tuyến của mặt phẳng $(ABC)$ có tọa độ là
	\choice
	{\True $(2;7;2)$}
	{$(-2;7;-2)$}
	{$(16;1;6)$}
	{$(16;-1;6)$}
	\loigiai{
		Ta có $\overrightarrow{AB}=(3;0;-3)$; $\overrightarrow{AC}=(5;-2;2)$. \\
		Mặt phẳng $(ABC)$ có cặp véc-tơ chỉ phương là $\overrightarrow{AB}$ và $\overrightarrow{AC}$ nên có một véc-tơ pháp tuyến là $\overrightarrow{n}=-\dfrac{1}{3} \left[ \overrightarrow{AB}; \overrightarrow{AC}\right] =(2;7;2)$.
	}
\end{ex}

\begin{ex}%[Nguyễn Ngọc Nguyên]%[2H3Y3-1]
		Trong không gian $Oxyz$, véc-tơ nào dưới đây là véc-tơ chỉ phương của đường thẳng $d \colon \dfrac{x-1}{2}=\dfrac{y+1}{1}=\dfrac{z-2}{-1}$.
	\choice
	{$\overrightarrow{u_2}=(1;1;2)$}
	{$\overrightarrow{u_3}=(1;-1;2)$}
	{\True $\overrightarrow{u_4}=(2;1;-1)$}
	{$\overrightarrow{u_1}=(2;1;1)$}
	\loigiai{
Đường thẳng $d$ có một véc-tơ chỉ phương là $\overrightarrow{u}=(2;1;-1)$.		
	}
\end{ex}

\begin{ex}%[Nguyễn Ngọc Nguyên]%[2H3Y3-1]
	Trong không gian với hệ tọa độ $Oxyz$, véc-tơ nào sau đây là một véc-tơ chỉ phương của đường thẳng $d \colon \dfrac{x}{2}=\dfrac{y+1}{-3}=\dfrac{z}{1}$?
	\choice
	{$\overrightarrow{u}=(1;-3;2)$}
	{$\overrightarrow{u}=(2;3;1)$}
	{$\overrightarrow{u}=(2;-6;1)$}
	{\True $\overrightarrow{u}=(4;-6;2)$}
	\loigiai{
Đường thẳng $d$ có một véc-tơ chỉ phương là $\overrightarrow{t}=(2;-3;1)$. Do đó $\overrightarrow{u}=2\overrightarrow{t}=(4;-6;2)$ cũng là một véc-tơ chỉ phương của $d$.		
	}
\end{ex}

\begin{ex}%[Nguyễn Ngọc Nguyên]%[2H3Y3-1]
	Trong không gian $Oxyz$, cho đường thẳng $d \colon \heva{&x=1-2t \\ &y=1+t \\ &z=t+2}$ ($t\in \mathbb{R}$). Tìm tọa độ một véc-tơ chỉ phương của đường thẳng $d$.
	\choice
	{\True $(-2;1;1)$}
	{$(1;1;1)$}
	{$(2;-1;-2)$}
	{$(-2;1;2)$}
	\loigiai{
Đường thẳng $d$ có một véc-tơ chỉ phương là $\overrightarrow{t}=(-2;1;1)$.		
	}
\end{ex}

\begin{ex}%[Nguyễn Ngọc Nguyên]%[2H3Y3-1]
	Trong không gian $Oxyz$, cho đường thẳng $\Delta \colon \dfrac{x+1}{-3}=\dfrac{y-2}{2}=\dfrac{z+1}{1}$. Tìm tọa độ một véc-tơ chỉ phương của $\Delta$.
	\choice
	{$(-1;2;-1)$}
	{$(1;-2;1)$}
	{\True $(3;-2;-1)$}
	{$(-3;2;0)$}
	\loigiai{
Đường thẳng $\Delta$ có một véc-tơ chỉ phương là $\overrightarrow{u}=(-3;2;1)$ nên $\overrightarrow{v}=-\overrightarrow{u}=(3;-2;-1)$ cũng là véc-tơ chỉ phương của $\Delta$.		
	}
\end{ex}

\begin{ex}%[Nguyễn Ngọc Nguyên]%[2H3Y3-1]
	Trong không gian $Oxyz$, cho đường thẳng $d \colon \heva{&x=2 \\ &y=3+4t \\ &z=5-t}$ ($t \in \mathbb{R}$). Véc-tơ nào dưới đây là một véc-tơ chỉ phương của đường thẳng $d$?
	\choice
	{$\overrightarrow{u_2}=(2;3;5)$}
	{\True $\overrightarrow{u_3}=(0;4;-1)$}
	{$\overrightarrow{u_1}=(2;4;-1)$}
	{$\overrightarrow{u_4}=(2;-4;-1)$}
	\loigiai{
		Đường thẳng $d \colon \heva{&x=2 \\ &y=3+4t \\ &z=5-t}$ có $\overrightarrow{u_3}=(0;4;-1)$ là một véc-tơ chỉ phương.
	}
\end{ex}

\begin{ex}%[Nguyễn Ngọc Nguyên]%[2H3Y3-1]
	Trong không gian $Oxyz$, cho đường thẳng $(\Delta)$ có phương trình $\dfrac{x-1}{2}=\dfrac{y+2}{3}=\dfrac{z+1}{-1}$. Véc-tơ chỉ phương của đường thẳng là
	\choice
	{\True $\overrightarrow{u}=(2;3;-1)$}
	{$\overrightarrow{u}=(2;3;1)$}
	{$\overrightarrow{u}=(-2;3;-1)$}
	{$\overrightarrow{u}=(-2;-3;-1)$}
	\loigiai{
Đường thẳng $(\Delta)$ có một véc-tơ chỉ phương là $\overrightarrow{u}=(2;3;-1)$.
	}
\end{ex}

\begin{ex}%[Nguyễn Ngọc Nguyên]%[2H3Y3-1]
	Trong không gian $Oxyz$, đường thẳng $d \colon \dfrac{x-1}{3}=\dfrac{y-5}{2}=\dfrac{z+2}{-5}$ có một véc-tơ chỉ phương là
	\choice
	{$\overrightarrow{u}=(1;5;-2)$}
	{\True $\overrightarrow{u}=(3;2;-5)$}
	{$\overrightarrow{u}=(-3;2;-5)$}
	{$\overrightarrow{u}=(2;3;-5)$}
	\loigiai{
Đường thẳng $d$ có một véc-tơ chỉ phương là $\overrightarrow{u}=(3;2;-5)$.	
	}
\end{ex}
\begin{ex}%[Nguyễn Ngọc Nguyên]%[2H3Y3-1]
	Trong không gian $Oxyz$, cho đường thẳng $d \colon \dfrac{x+2}{1}=\dfrac{y-1}{-3}=\dfrac{z+1}{2}$. Véc-tơ nào sau đây là một véc-tơ chỉ phương của đường thẳng $d$?
	\choice
	{$\overrightarrow{u_1}=(-2;1;-1)$}
	{\True $\overrightarrow{u_1}=(1;-3;2)$}
	{$\overrightarrow{u_3}=(-1;-3;2)$}
	{$\overrightarrow{u_4}=(1;3;-2)$}
	\loigiai{
Đường thẳng $d$ có một véc-tơ chỉ phương là $\overrightarrow{u_1}=(1;-3;2)$.		
	}
\end{ex}

\begin{ex}%[Nguyễn Ngọc Nguyên]%[2H3Y3-1]
	Trong không gian $Oxyz$, đường thẳng $d$ song song với đường thẳng $\Delta \colon \dfrac{x+2}{1}=\dfrac{y+1}{-2}=\dfrac{z-3}{1}$ có véc-tơ chỉ phương là
	\choice
	{\True $\overrightarrow{u}=(1;-2;1)$}
	{$\overrightarrow{u}=(-1;-3;4)$}
	{$\overrightarrow{u}=(-2;-1;3)$}
	{$\overrightarrow{u}=(0;-2;3)$}
	\loigiai{
Ta có $\Delta$ có một véc-tơ chỉ phương là $\overrightarrow{u}=(1;-2;1)$.\\
Vì $d$ song song với $\Delta$ nên cũng có véc-tơ chỉ phương là $\overrightarrow{u}=(1;-2;1)$.
	}
\end{ex}

\begin{ex}%[Nguyễn Ngọc Nguyên]%[2H3Y3-1]
	Trong không gian $Oxyz$, cho đường thẳng $d \colon \dfrac{x-1}{5}=\dfrac{y-2}{-8}=\dfrac{z+3}{7}$. Véc-tơ nào sau đây là một véc-tơ chỉ phương của $d$?
	\choice
	{$\overrightarrow{u_3}=(1;2;-3)$}
	{$\overrightarrow{u_4}=(7;-8;5)$}
	{$\overrightarrow{u_1}=(-1;-2;3)$}
	{\True $\overrightarrow{u_3}=(5;-8;7)$}
	\loigiai{
Đường thẳng $d$ có một véc-tơ chỉ phương là $\overrightarrow{u_3}=(5;-8;7)$.	
	}
\end{ex}

\Closesolutionfile{ans}
%======================
\subsection{Bảng đáp án}
\inputansbox{8}{ans/ANS-DANG-6}