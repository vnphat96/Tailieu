\setcounter {section} {7}
\setcounter{ex}{0}
\section{Tính chất tích phân}
\subsection{Kiến thức cần nhớ}
\begin{khung}
	\subsubsection{Định nghĩa}
	Cho hàm số $ f(x) $ liên tục trên đoạn $ [a;b] $. Giả sử $ F(x) $ là một nguyên hàm của hàm số $ f(x) $ trên đoạn $ [a;b] $, hiệu số $ F(b)-F(a) $ được gọi là tích phân từ $ a $ đến $ b $ (hay tích phân xác định trên đoạn $ [a;b] $) của hàm số $ f(x) $.\\
	Kí hiệu $ \displaystyle\int\limits_{a}^{b} f(x)\mathrm{d}x $.\\
\subsubsection{Các tính chất}
	\begin{itemize}
		\item $ \displaystyle\int\limits_{a}^{a} f(x)\mathrm{d}x=0 $.
		\item $ \displaystyle\int\limits_{a}^{b} f(x)\mathrm{d}x=\displaystyle\int\limits_{a}^{b}f(t)\mathrm{d}t=\displaystyle\int\limits_{a}^{b} f(u)\mathrm{d} u=\ldots $
		\item $ \displaystyle\int\limits_{a}^{b} kf(x)\mathrm{d}x=k\displaystyle\int\limits_{a}^{b}f(x)\mathrm{d}x $ (với $ k $ là hằng số).
		\item $ \displaystyle\int\limits_{a}^{b}f(x)\mathrm{d}x=-\displaystyle\int\limits_{b}^{a}f(x)\mathrm{d}x $.
		\item $ \displaystyle\int\limits_{a}^{b} \left[f(x)\pm g(x)\right]\mathrm{d}x =\displaystyle\int\limits_{a}^{b} f(x)\mathrm{d}x \pm \displaystyle\int\limits_{a}^{b}g(x)\mathrm{d}x $.
		\item $ \displaystyle\int\limits_{a}^{b}f(x)\mathrm{d}x =\displaystyle\int\limits_{a}^{c}f(x)\mathrm{d}x +\displaystyle\int\limits_{c}^{b}f(x)\mathrm{d}x $ (với $ a<b<c $).
		
	\end{itemize}
\end{khung}
\subsection{Bài tập mẫu}
\Opensolutionfile{ans}[ans/ANS-DANG-8]
\begin{khung}%[Nguyễn Thái Hoàng]%[2D3B2-1]
	\begin{vd}
		[Đề minh họa BGD 2022-2023]Nếu $ \displaystyle\int\limits_{-1}^{4} f(x)\mathrm{d}x=2 $ và $ \displaystyle\int\limits_{-1}^{4} g(x)\mathrm{d}x=3 $ thì $ \displaystyle\int\limits_{-1}^{4} \left[f(x)+g(x)\right]\mathrm{d}x $ bằng
		\choice
		{\True $ 5 $}
		{$ 6 $}
		{$ 1 $}
		{$ 7 $}
		\loigiai{
			Ta có $ \displaystyle\int\limits_{-1}^{4} \left[f(x)+g(x)\right]\mathrm{d}x =\displaystyle\int\limits_{-1}^{4}f(x)\mathrm{d}x +\displaystyle\int\limits_{-1}^{4} g(x)\mathrm{d}x=2+3=5$.
		}
	\end{vd}
\end{khung}

\subsection{Bài tập tương tự và phát triển}
\begin{ex}%[Nguyễn Thái Hoàng]%[2D3B2-1]
Biết $ \displaystyle \int\limits_{0}^{1}f(x)\mathrm{d}x =\dfrac{1}{3} $ và $ \displaystyle \int\limits_{0}^{1}g(x)\mathrm{d}x =\dfrac{4}{3} $. Khi đó $ \displaystyle \int\limits_{0}^{1}\left[f(x)-g(x)\right]\mathrm{d}x $ bằng
\choice
{$ \dfrac{5}{3} $}
{\True $ -1 $}
{$ 1 $}
{$ -\dfrac{5}{3} $}
\loigiai{
Ta có $ \displaystyle \int\limits_{0}^{1}\left[f(x)-g(x)\right]\mathrm{d}x = \displaystyle \int\limits_{0}^{1} f(x)\mathrm{d}x-\displaystyle \int\limits_{0}^{1}g(x)\mathrm{d}x=\dfrac{1}{3}-\dfrac{4}{3}=-1 $.
}
\end{ex}

\begin{ex}%[Nguyễn Thái Hoàng]%[2D3B2-1]
Cho $ I=\displaystyle \int\limits_{1}^{5} f(x)\mathrm{d}x=4$, $ J=\displaystyle \int\limits_{1}^{5}g(x)\mathrm{d}x=3 $. Khi đó $ K=\displaystyle \int\limits_{1}^{5}\left[4f(x)-3g(x)\right]\mathrm{d}x $ bằng
\choice
{$ 4 $}
{$ 2 $}
{\True $ 7 $}
{$ 8 $}
\loigiai{
Ta có $ K=\displaystyle \int\limits_{1}^{5}\left[4f(x)-g(x)\right]\mathrm{d}x=4\displaystyle \int\limits_{1}^{5}f(x)\mathrm{d}x-3\displaystyle \int\limits_{1}^{5}g(x)\mathrm{d}x=16-9=7 $.
}
\end{ex}

\begin{ex}%[Nguyễn Thái Hoàng]%[2D3B2-1]
Cho hàm số $ y=f(x) $ liên tục trên $ [a;b] $, nếu $ \displaystyle \int\limits_{a}^{d}f(x)\mathrm{d}x=5 $ và $ \displaystyle \int\limits_{b}^{d}f(x)\mathrm{d}x=2 $ với $ a<d<b $ thì $ \displaystyle \int\limits_{a}^{b}f(x)\mathrm{d}x $ bằng
\choice
{$ 10 $}
{\True$ 3 $}
{$ 7 $}
{$ \dfrac{5}{2} $}
\loigiai{
\begin{center}
$ \displaystyle \int\limits_{a}^{b}f(x)\mathrm{d}x= \displaystyle \int\limits_{a}^{d}f(x)\mathrm{d}x +\displaystyle \int\limits_{d}^{b}f(x)\mathrm{d}x = \displaystyle \int\limits_{a}^{d}f(x)\mathrm{d}x -\displaystyle \int\limits_{b}^{d}f(x)\mathrm{d}x =5-2 =3 $.
\end{center}
}
\end{ex}

\begin{ex}%[Nguyễn Thái Hoàng]%[2D3B2-1]
Cho hàm số $ f(x) $, $ g(x) $ liên tục trên $ K $ và $ a $, $ b $, $ c $ thuộc $ K $. Công thức nào sau đây sai?
\choice
{$ \displaystyle \int\limits_{a}^{b}\left[f(x)+g(x)\right]\mathrm{d}x =\displaystyle \int\limits_{a}^{b}f(x)\mathrm{d}x+\displaystyle \int\limits_{a}^{b}g(x)\mathrm{d}x $}
{$ \displaystyle \int\limits_{a}^{b}kf(x)\mathrm{d}x=k\displaystyle \int\limits_{a}^{b}f(x)\mathrm{d}x $}
{\True$ \displaystyle \int\limits_{a}^{b}f(x)\mathrm{d}x=\displaystyle \int\limits_{b}^{a}f(x)\mathrm{d}x $}
{$ \displaystyle \int\limits_{a}^{b}f(x)\mathrm{d}x+\displaystyle \int\limits_{b}^{c}f(x)\mathrm{d}x=\displaystyle \int\limits_{a}^{c}f(x)\mathrm{d}x $}
\loigiai{

}
\end{ex}

\begin{ex}%[Nguyễn Thái Hoàng]%[2D3B2-1]
Cho $ \displaystyle \int\limits_{1}^{0}f(x)\mathrm{d}x=3 $ và $ \displaystyle \int\limits_{0}^{1}g(x)\mathrm{d}x=-4 $. Giá trị của $ \displaystyle \int\limits_{0}^{1}\left[f(x)-2g(x)\right]\mathrm{d}x $ bằng
\choice
{$ 11 $}
{$ 7 $}
{$ -1 $}
{\True $ 5 $}
\loigiai{
Ta có $ \displaystyle \int\limits_{0}^{1}\left[f(x)-2g(x)\right]\mathrm{d}x=\displaystyle \int\limits_{0}^{1}f(x)\mathrm{d}x-2\displaystyle \int\limits_{0}^{1}g(x)\mathrm{d}x=-\displaystyle \int\limits_{1}^{0}f(x)\mathrm{d}x-2\displaystyle \int\limits_{0}^{1}g(x)\mathrm{d}x=-3-2\cdot (-4)=5 $.
}
\end{ex}

\begin{ex}%[Nguyễn Thái Hoàng]%[2D3B2-1]
Cho hàm số $ y=f(x) $ thỏa mãn $ \displaystyle \int\limits_{1}^{2}f(x)\mathrm{d}x=-3 $ và $ \displaystyle \int\limits_{2}^{3}f(x)\mathrm{d}x=4 $. Khi đó $ \displaystyle \int\limits_{1}^{3}f(x)\mathrm{d}x $ bằng
\choice
{$ 12 $}
{$ -12 $}
{\True $ 1 $}
{$ 7 $}
\loigiai{
Ta có $ \displaystyle \int\limits_{1}^{3}f(x)\mathrm{d}x=\displaystyle \int\limits_{1}^{2}f(x)\mathrm{d}x+\displaystyle \int\limits_{2}^{3}f(x)\mathrm{d}x=-3+4=1 $.
}
\end{ex}

\begin{ex}%[Nguyễn Thái Hoàng]%[2D3B2-1]
Biết $ \displaystyle \int\limits_{0}^{1}f(x)\mathrm{d}x=3 $, khi đó $ \displaystyle \int\limits_{0}^{1}\left[4x-3f(x)\right]\mathrm{d}x $ bằng
\choice
{$ -9 $}
{\True $ -7 $}
{$ -5 $}
{$ 11 $}
\loigiai{
Ta có $ \displaystyle \int\limits_{0}^{1}\left[4x-3f(x)\right]\mathrm{d}x=\displaystyle \int\limits_{0}^{1}4x\mathrm{d}x-3\displaystyle \int\limits_{0}^{1}f(x)\mathrm{d}x=2x^{2}\bigg|_{0}^{1}-3\cdot 3=2-9=-7 $.
}
\end{ex}

\begin{ex}%[Nguyễn Thái Hoàng]%[2D3B2-1]
Biết $ \displaystyle \int\limits_{0}^{1}f(x)\mathrm{d}x=2 $, $ \displaystyle \int\limits_{0}^{1}g(x)\mathrm{d}x=-4 $. Khi đó $ \displaystyle \int\limits_{0}^{1}\left[f(x)+2g(x)\right]\mathrm{d}x $ bằng
\choice
{\True $ -6 $}
{$ 6 $}
{$ -2 $}
{$ 2 $}
\loigiai{
Ta có $ \displaystyle \int\limits_{0}^{1}\left[f(x)+2g(x)\right]\mathrm{d}x=\displaystyle \int\limits_{0}^{1}f(x)\mathrm{d}x+2\displaystyle \int\limits_{0}^{1}g(x)\mathrm{d}x=2+2\cdot(-4)=-6 $.
}
\end{ex}

\begin{ex}%[Nguyễn Thái Hoàng]%[2D3B2-1]
Cho hai số thực $a$; $b$ tùy ý, $F(x)$ là một nguyên hàm của hàm số $f(x)$ trên tập $\mathbb{R}$. Mệnh đề nào sau đây đúng?
\choice
{\True $\displaystyle\int\limits_a^b{f(x)\mathrm{d}x}=F(b)-F(a)$}
{$\displaystyle\int\limits_a^b{f(x)\mathrm{d}x}=F(a)-F(b)$}
{$\displaystyle\int\limits_a^b{f(x)\mathrm{d}x}=F(b)+F(a)$}
{$\displaystyle\int\limits_a^b{f(x)\mathrm{d}x}=f(b)-f(a)$}
\loigiai{
$\displaystyle\int\limits_a^b{f(x)\mathrm{d}x}=F(b)-F(a)$.
}
\end{ex}

\begin{ex}%[Nguyễn Thái Hoàng]%[2D3B2-1]
Cho $\displaystyle\int\limits_{-1}^2f(x)\mathrm{d}x=2$ và $\displaystyle\int\limits_{-1}^2g(x)\mathrm{d}x=-1$. Giá trị của $\displaystyle\int\limits_{-1}^2\left[2f(x)+3g(x)\right]\mathrm{d}x$ bằng
\choice
{$ -7 $}
{\True $ 1 $}
{$ 5 $}
{$ 7 $}
\loigiai{
Áp dụng tính chất của tích phân ta có
\begin{center}
	$\displaystyle\int\limits_{-1}^2\left[2f(x)+3g(x)\right]\mathrm{d}x=2\displaystyle\int\limits_{-1}^2f(x)dx+3\displaystyle\int\limits_{-1}^2g(x)\mathrm{d}x=4-3=1$.
\end{center}
}
\end{ex}

\begin{ex}%[Nguyễn Thái Hoàng]%[2D3B2-1]
Nếu $\displaystyle\int\limits_{-1}^3f(x)\mathrm{d}x=2$ và $\displaystyle\int\limits_{-1}^3g(x)\mathrm{d}x=-1$ thì $\displaystyle\int\limits_{-1}^3\left[f(x)-g(x)\right]\mathrm{d}x$ bằng
\choice
{\True $ 3 $}
{$ 4 $}
{$ -3 $}
{$ -1 $}
\loigiai{
$\displaystyle\int\limits_{-1}^3\left[f(x)-g(x)\right]\mathrm{d}x=\displaystyle\int\limits_{-1}^3f(x)\mathrm{d}x-\displaystyle\int\limits_{-1}^3g(x)\mathrm{d}x$ $=2-\left(-1\right)=3$.
}
\end{ex}

\begin{ex}%[Nguyễn Thái Hoàng]%[2D3B2-1]
Cho $\displaystyle\int\limits_{0}^{1}f(x)\mathrm{d}x=2$ và $\displaystyle\int\limits_0^1g(x)\mathrm{d}x=5$, khi đó $\displaystyle\int\limits_0^1\left[3f(x)-2g(x)\right]\mathrm{d}x$ bằng:
\choice
{$11$}
{\True $-4$}
{$16$}
{$-3$}
\loigiai{
\begin{center}
	$\displaystyle\int\limits_0^1\left[3f(x)-2g(x)\right]\mathrm{d}x=3\displaystyle\int\limits_0^1f(x)\mathrm{\,d}x-2\displaystyle\int\limits_0^1g(x)\mathrm{d}x=3\cdot2-2\cdot5=-4$.
\end{center}
}
\end{ex}

\begin{ex}%[Nguyễn Thái Hoàng]%[2D3B2-1]
Nếu $\displaystyle\int\limits_0^2f(x)\mathrm{d}x=3$ và $\displaystyle\int\limits_0^2g(x)\mathrm{d}x=-2$ thì $\displaystyle\int\limits_0^2\left[f(x)-g(x)\right]\mathrm{d}x$ bằng
\choice
{\True $5$}
{$1$}
{$-1$}
{$-5$}
\loigiai{
$\displaystyle\int\limits_0^2\left[f(x)-g(x)\right]\mathrm{d}x=\displaystyle\int\limits_0^2f(x)dx-\displaystyle\int\limits_0^2g(x)\mathrm{d}x=3+2=5$.
}
\end{ex}

\begin{ex}%[Nguyễn Thái Hoàng]%[2D3B2-1]
Cho hàm số $f(x)$ liên tục trên $\mathbb{R}$ và có $\displaystyle\int\limits_0^2f(x)\mathrm{d}x=9;\displaystyle\int\limits_2^4f(x)\mathrm{d}x=4$. Tính $I=\displaystyle\int\limits_0^4f(x)\mathrm{d}x$.
\choice
{$I=5$}
{$I=\dfrac{9}{4}$}
{$I=36$}
{\True $I=13$}
\loigiai{
Ta có $\displaystyle\int\limits_0^4f(x)\mathrm{d}x=\displaystyle\int\limits_0^2f(x)\mathrm{d}x+\displaystyle\int\limits_2^4f(x)\mathrm{d}x=9+4=13$.
}
\end{ex}

\begin{ex}%[Nguyễn Thái Hoàng]%[2D3B2-1]
Cho $\displaystyle\int\limits_{-2}^3f(x)\mathrm{d}x=-4$ và $\displaystyle\int\limits_1^3f(x)\mathrm{d}x=2$. Khi đó $\displaystyle\int\limits_{-2}^1f(x)\mathrm{d}x$ bằng
\choice
{$-2$}
{$6$}
{$-8$}
{\True $-6$}
\loigiai{
Ta có $\displaystyle\int\limits_{-2}^3f(x)\mathrm{d}x=\displaystyle\int\limits_{-2}^1f(x)\mathrm{d}x+\displaystyle\int\limits_1^3f(x)\mathrm{d}x$.\\
Vậy $\displaystyle\int\limits_{-2}^1f(x)\mathrm{d}x=\displaystyle\int\limits_{-2}^3f(x)\mathrm{\,d}x-\displaystyle\int\limits_1^3f(x)\mathrm{d}x=-4-2=-6$.
}
\end{ex}

\begin{ex}%[Nguyễn Thái Hoàng]%%[2D3Y1-1]
Cho $f(x)$, $g(x)$ là các hàm số liên tục trên $\mathbb{R}$. Trong các mệnh đề sau, mệnh đề nào sai?
\choice
{\True $\displaystyle\int f(x)g(x)\mathrm{d}x=\displaystyle\int{f(x)\mathrm{d}x\cdot\displaystyle\int{g(x)\mathrm{d}x}}$}
{$\displaystyle\int{\left[f(x)+g(x)\right]\mathrm{d}x=\displaystyle\int{f(x)\mathrm{d}x+\displaystyle\int{g(x)\mathrm{d}x}}}$}
{$\displaystyle\int{\left[f(x)-g(x)\right]\mathrm{d}x=\displaystyle\int{f(x)\mathrm{d}x-\displaystyle\int{g(x)\mathrm{d}x}}}$}
{$\displaystyle\int{2f(x)}\mathrm{d}x=2\displaystyle\int{f(x)\mathrm{d}}x$}
\loigiai{
Mệnh đề sai là $\displaystyle\int f(x)g(x)\mathrm{d}x=\displaystyle\int{f(x)\mathrm{d}x\cdot\displaystyle\int{g(x)\mathrm{d}x}}$.
}
\end{ex}

\begin{ex}%[Nguyễn Thái Hoàng]%[2D3B2-1]
Cho hàm số $f(x)$ có đạo hàm liên tục trên đoạn $[2;4]$ và thỏa mãn $f(2)=2$, $f(4)=2020$. Tính $I=\displaystyle\int\limits_1^2f'(2x)\mathrm{d}x$.
\choice
{\True $I=1009$}
{$I=2018$}
{$I=2022$}
{$I=1011$}
\loigiai{
Ta có $I=\displaystyle\int\limits_1^2f'\left(2x\right)\mathrm{d}x=\dfrac{1}{2}f\left(2x\right)\bigg|_1^2=\dfrac{1}{2}\left[f(4)-f(2)\right]=\dfrac{1}{2}\left(2020-2\right)=1009$.
}
\end{ex}

\begin{ex}%[Nguyễn Thái Hoàng]%[2D3B2-1]
Cho $\displaystyle\int\limits_1^3f(x)\mathrm{d}x=18$. Khi đó $\displaystyle\int\limits_1^3\left[5-2f(x)\right]\mathrm{d}x$ bằng
\choice
{\True $-26$}
{$16$}
{$-31$}
{$-46$}
\loigiai{
$\displaystyle\int\limits_1^3\left[5-2f(x)\right]\mathrm{d}x=\displaystyle\int\limits_1^3 5\mathrm{d}x-2\displaystyle\int\limits_1^3 f(x)\mathrm{d}x=5x\bigg|_1^3-2\cdot18=10-36=-26$.
}
\end{ex}

\begin{ex}%[Nguyễn Thái Hoàng]%[2D3B2-1]
Cho $\displaystyle\int\limits_0^2 f(x)\mathrm{d}x=3$ và $\displaystyle\int\limits_0^2 g(x)\mathrm{d}x=7$, khi đó $\displaystyle\int\limits_0^2 \left[f(x)+3g(x)\right] \mathrm{d}x$ bằng
\choice
{$10$}
{$16$}
{$-18$}
{\True $24$}
\loigiai{
Ta có $\displaystyle\int\limits_0^2\left[f(x)+3g(x)\right]\mathrm{d}x=\displaystyle\int\limits_0^2 f(x)dx+3\displaystyle\int\limits_0^2 g(x)\mathrm{d}x=24$.
}
\end{ex}

\begin{ex}%[Nguyễn Thái Hoàng]%[2D3B2-1]
Cho $\displaystyle\int\limits_1^5f(x)\mathrm{d}x=5$, $\displaystyle\int\limits_4^5f(u)\mathrm{d}u=2$ và $\displaystyle\int\limits_1^4g(x)\mathrm{d}x=3$. Tính $I=\displaystyle\int\limits_1^4\left[f(x)+g(x)\right]\mathrm{d}x$ .
\choice
{$I=5$}
{$I=10$}
{$I=3$}
{\True $I=6$}
\loigiai{
Ta có $\displaystyle\int\limits_4^5f(u)\mathrm{d}u=2\Rightarrow\displaystyle\int\limits_4^5f(x)\mathrm{d}x=2$.\\ $\displaystyle\int\limits_1^5f(x)\mathrm{d}x=\displaystyle\int\limits_1^4f(x)\mathrm{d}x+\displaystyle\int\limits_4^5f(x)\mathrm{d}x\Rightarrow\displaystyle\int\limits_1^4f(x)\mathrm{d}x=\displaystyle\int\limits_1^5f(x)\mathrm{d}x-\displaystyle\int\limits_4^5f(x)\mathrm{d}x=5-2=3$.\\
$I=\displaystyle\int\limits_1^4\left[f(x)+g(x)\right]\mathrm{d}x=\displaystyle\int\limits_1^4f(x)\mathrm{d}x+\displaystyle\int\limits_1^4g(x)\mathrm{d}x=3+3=6$.
}
\end{ex}
\Closesolutionfile{ans}
%======================
\subsection{Bảng đáp án}
\inputansbox{8}{ans/ANS-DANG-8}