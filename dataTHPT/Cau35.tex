%Dạng 1
\setcounter{ex}{0}
\section{Phép đếm}
\subsection{Kiến thức cần nhớ}
\begin{khung}
	\begin{enumerate}[\color{violet}\faCubes]
		\item \textbf{Phương pháp chung của bài toán tìm tập hợp điểm biểu diễn số phức là}
		\begin{itemize}
			\item  Gọi $M(x;y)$ là điểm biểu diễn số phức $z=x+yi$.
			\item  Thay vào điều kiện đề bài, ta được một phương trình biểu diễn theo hai biến $x$ và $y$.\\
		
				Chú ý các công thức
				\begin{listEX}[2]
					\item [\color{violet} \faCaretRight] $\overline{z}=x-yi$.
					\item [\color{violet} \faCaretRight] $|z|=\sqrt{x^2+y^2}$.
					\item [\color{violet} \faCaretRight] $z\cdot \overline{z}=x^2+y^2$.
					\item [\color{violet} \faCaretRight] $z^2=x^2-y^2+2xyi$.
				\end{listEX}
		
			\item  Tùy thuộc vào phương trình thu được, ta kết luận tập hợp điểm chạy trên "đối tượng hình" tương ứng.\\
			Một số dạng thường gặp
			\begin{itemize}
				\item [\color{violet} \faCaretRight] Dạng $Ax+By+C=0$ : tập hợp điểm là đường thẳng.
				\item [\color{violet} \faCaretRight] Dạng $(x-x_0)^2+(y-y_0)^2=R^2$: tập hợp điểm là đường tròn có tâm $I(x_0;y_0)$ và bán kính $R$.
				\item [\color{violet} \faCaretRight] Dạng $(x-x_0)^2+(y-y_0)^2\leq R^2$: tập hợp điểm là hình tròn có tâm $I(x_0;y_0)$ và bán kính $R$.
				\item [\color{violet} \faCaretRight]	Dạng $x^2+y^2-2ax-2by+c=0, (a^2+b^2-c>0)$ : tập hợp điểm là đường tròn có tâm $I(a;b)$ và bán kính $R=\sqrt{a^2+b^2-c}$.
				\item [\color{violet} \faCaretRight]	Dạng $\dfrac{x^2}{a^2}+\dfrac{y^2}{b^2}=1$: tập hợp điểm là đường elip.
			\item [\color{violet} \faCaretRight]	Dạng $\dfrac{x^2}{a^2}-\dfrac{y^2}{b^2}=1$: tập hợp điểm là đường hyperbol.
			\item [\color{violet} \faCaretRight]	Dạng $y=ax^2+bx+c \quad (a\neq 0)$: tập hợp điểm là đường parabol.
					\end{itemize}
		\end{itemize}
	
	\end{enumerate}
	\end{khung}
\subsection{Bài tập mẫu}
\Opensolutionfile{ans}[ans/ANS-DANG-1]
\begin{khung}
	\begin{vd}[Đề minh họa BGD 2022-2023]%[Lê Thị Thanh Tuyền]%[2D4B1-2]
		
			Trên mặt phẳng tọa độ, biết tập hợp điểm biểu diễn các số phức $z$ thỏa mãn $|z+2 i|=1$ là một đường tròn. Tâm của đường tròn đó có tọa độ là
			\choice
			{$(0; 2)$}
			{$(-2; 0)$}
			{\True$(0;-2)$}
			{$(2; 0)$}
	\loigiai{
	Gọi $z=x+yi \quad (x,y\in\mathbb{R})$.\\
	Ta có $|z+2i|=1\Leftrightarrow |x+(y+2)i|\Leftrightarrow x^2+\left(y+2\right)^2=1$.\\
	Vậy tập hợp điểm $M$ biểu diễn số phức $z$ là đường tròn có tâm $I\left(0;-2\right)$, bán kính $R=1$.
}
	\end{vd}
\end{khung}
\subsection{Bài tập tương tự và phát triển}
\begin{ex}%[Lê Thị Thanh Tuyền]%[2D4B1-2]
	Cho số phức $z$ thoả $\left|z-1+i\right|=4$. Phát biểu nào sau đây là đúng?
	\choice
	{\True Tập hợp điểm biểu diễn số phức $z$ là đường tròn có tâm $I(1;-1)$ và bán kính bằng $R=4$}
	{Tập hợp điểm biểu diễn số phức $z$ là một đường tròn có tâm $I(-1;1)$ và bán kính  $R=2$}
	{Tập hợp điểm biểu diễn số phức $z$ là một đường parabol}
	{Tập hợp điểm biểu diễn số phức $z$ là một đường thẳng }
\loigiai{
	Gọi $z=x+yi \quad (x,y\in\mathbb{R})$.\\
Ta có $|z-1+i|=4\Leftrightarrow |x-1+(y+1)i|\Leftrightarrow \left(x-1\right)^2+\left(y+1\right)^2=16$.\\
Vậy tập hợp điểm $M$ biểu diễn số phức $z$ là đường tròn có tâm $I\left(1;-1\right)$, bán kính $R=4$.
}
	\end{ex}
\begin{ex}%[Lê Thị Thanh Tuyền]%[2D4B1-2]
Biết tập hợp điểm biểu diễn số phức $z$ thoả $\left|iz-1+2i\right|=4$ là một đường tròn . Tìm toạ độ tâm $I$ của đường tròn đó.
\choice
{$I(1;2)$}
{$I(-1;-2)$}
{\True$I(-2;-1)$}
{$I(2;1)$}
\loigiai{
	Gọi $z=x+yi \quad (x,y\in\mathbb{R})$.\\
	Ta có 
	\begin{eqnarray*}
\left|iz-1+2i\right|=4&\Leftrightarrow& \left|i(x+yi)-1+2i\right|=4.\\
&\Leftrightarrow& \left|-y-1+(x+2)i\right|=4\\
&\Leftrightarrow& \left(x+2\right)^2+\left(y+1\right)^2=16.
	\end{eqnarray*}

	Vậy tập hợp điểm biểu diễn số phức $z$ là đường tròn có tâm $I(-2;-1)$ và bán kính $R=4$.
}
\end{ex}
\begin{ex}%[Lê Thị Thanh Tuyền]%[2D4B1-2]
Tập hợp điểm biểu diễn số phức $z$ thoả $\left|z+4-4i\right|\leq  2$ là 
	\choice
{Hình tròn tâm $I(4;-4)$ và bán kính $R=4$}
{\True Hình tròn tâm $I(-4;4)$ và bán kính $R=2$}
{Đường tròn tâm $I(4;-4)$ và bán kính $R=4$}
{Đường tròn tâm $I(-4;4)$ và bán kính $R=2$}
\loigiai{
		Gọi $z=x+yi \quad (x,y\in\mathbb{R})$.\\
		Ta có $\left|z+4-4i\right|\leq  2 \Leftrightarrow \left(x+4\right)^2+\left(y-4\right)^2\leq 4$. 
		Vậy tập hợp điểm biểu diễn số phức $z$ là hình tròn tâm $I(-4;4)$ và bán kính $R=2$.
}
	
\end{ex}

\begin{ex}%[Lê Thị Thanh Tuyền]%[2D4B1-2]
Gọi $\left(H\right)$ là hình gồm tập hợp điểm $M$ biểu diễn số phức $z$ thoả $|z+3|^2+|z-3|^2=50$. Tính diện tích hình $(H)$.
\choice
{$S=8\pi$}
{\True $S=16\pi$}
{$S=15\pi$}
{$S=20\pi$}

\loigiai{
Gọi $z=x+yi \quad (x,y\in\mathbb{R})$.\\
Ta có  
\begin{eqnarray*}
|z+3|^2+|z-3|^2=50&\Leftrightarrow&\left(x+3\right)^2+y^2+(x-3)^2+y^2=50.\\
&\Leftrightarrow& 2x^2+2y^2+18=50.\\
&\Leftrightarrow& x^2+y^2=16.
\end{eqnarray*}
 Vậy tập hợp điểm biểu diễn $M$ là đường tròn tâm $O$ và bán kính $R=4$.\\
 Do đó diện tích hình $(H)$ là $S=\pi\cdot R^2=16\pi$.
}	
\end{ex}

\begin{ex}%[Lê Thị Thanh Tuyền]%[2D4B1-2]
Biết tập hợp điểm biểu biễn số phức $z$ thoả $|z-2i|-|2z+1|=0$ là một đường tròn. Tính chu vi $C$ của đường tròn đó.
\choice
{\True$C=\dfrac{2\sqrt{17}\pi}{3}$}
{$C=\dfrac{\sqrt{17}\pi}{3}$}
{$C=\dfrac{17\pi}{9}$}
{$C=\dfrac{4\sqrt{17}\pi}{3}$}	
\loigiai{
	Gọi $z=x+yi \quad (x,y\in\mathbb{R})$.\\
	Ta có  
	\begin{eqnarray*}
		|z-2i|-|2z+1|=0&\Leftrightarrow& |x+(y-2)i|=|2x+1+2yi|\\
		&\Leftrightarrow& x^2+\left(y-2\right)^2=\left(2x+1\right)^2+\left(2y\right)^2.\\
		&\Leftrightarrow& 3x^2+3y^2+4x+4y-3=0.\\
		&\Leftrightarrow& x^2+y^2+\dfrac{4}{3}x+\dfrac{4}{3}y-1=0.
	\end{eqnarray*}
Vậy tập hợp điểm biểu diễn số phức $z$ là đường tròn có tâm $I\left(-\dfrac{2}{3};-\dfrac{2}{3}\right)$ và bán kính $R=\dfrac{\sqrt{17}}{3}$.
	\\
	Suy ra chu vi của đường tròn là $C=2\pi\cdot R=\dfrac{2\sqrt{17}\pi}{3}$.
}
\end{ex}
\begin{ex}%[Lê Thị Thanh Tuyền]%[2D4B1-2]
Tập hợp các điểm biểu diễn số phức $z$ thỏa mãn $|(1-i)z-4+2i|=2$ là một đường tròn. Tìm tọa độ tâm $I$ và tính bán kính $R$ của đường tròn đó.
\choice
{$I(-3;-1), R=2$}
{\True $I(3;1),R=\sqrt{2}$}
{$I(3;1), R=2$}
{$I(-3;-1), R=\sqrt{2}$}
	\loigiai
{
	Chia hai vế của đẳng thức $|(1-i)z-4+2i|=2$ cho $|1-i|$ ta được $|z-(3+i)|=\sqrt{2}$.\\
Gọi $z=x+yi \quad (x,y\in\mathbb{R})$.\\
Ta có  	$|z-(3+i)|=\sqrt{2}\Leftrightarrow (x-3)^2+(y-1)^2=2$.\\
	 Suy ra tập hợp các điểm biểu diễn số phức $z$ là đường tròn tâm $I(3;1)$, bán kính $R=\sqrt{2}$.
}
\end{ex}
\begin{ex}%[Lê Thị Thanh Tuyền]%[2D4B1-2]
Biết tập hợp điểm biểu diễn số phức $z$ thoả $\left|\overline{z}+2-i\right|=4$ là một đường tròn có tâm $I$ và bán kính $R$. Khẳng định nào sau đây là đúng?
\choice
{\True$I(-2;-1), R=4$}
{$I(2;-1), R=4$}
{$I(2;-1), R=2$}
{$I(-2;-1), R=2$}
\loigiai{
	Gọi $z=x+yi \quad (x,y\in\mathbb{R})$.\\
Ta có $\left|\overline{z}+2-i\right|=4\Leftrightarrow \left|x+2-(y+1)i\right|= 4\Leftrightarrow (x+2)^2 +(y+1)^2=16$.\\
Vậy tập hợp điểm biểu diễn số phức $z$ là đường tròn có tâm $I(-2;-1)$ và bán kính $R=4$.
}	
	
\end{ex}

\begin{ex}%[Lê Thị Thanh Tuyền]%[2D4B1-2]
	Cho số phức $z$ thoả mãn   $\left(z+1\right)\left(\overline{z}-2i\right)$ là một số thuần ảo. Biết tập hợp điểm biểu diễn số phức $z$ là một đường tròn. Tìm bán kính $R$ của đường tròn đó.
\choice
{\True$R=\dfrac{\sqrt{5}}{2}$}
{$R=\dfrac{\sqrt{5}}{4}$}
{$R=\sqrt{5}$}
{$R=\dfrac{5}{4}$}
	\loigiai{
	Gọi $z=x+yi \quad (x,y\in\mathbb{R})$.\\
Ta có $\left(z+1\right)\left(\overline{z}-2i\right)= \left((x+1)+yi\right)	\left(x-(y+2)i\right)=x^2+x+y^2+2y-\left(2x+y+2\right)i$.\\
Theo đề bài ta có $x^2+x+y^2+2y=0\Leftrightarrow \left(x+\dfrac{1}{2}\right)^2+\left(y+1\right)^2=\dfrac{5}{4}$.\\
Vậy tập hợp điểm biểu diễn số phức $z$ là đường tròn có tâm $I\left(-\dfrac{1}{2};-1\right)$, bán kính $R=\dfrac{\sqrt{5}}{2}$.

}	
	
\end{ex}
\begin{ex}%[Lê Thị Thanh Tuyền]%[2D4B2-4]
	Biết tập hợp điểm biểu diễn số phức $z$ thỏa  $|z - i| = |2 - 3i - z|$ là một đường thẳng. Tính khoảng cách $\mathrm{d}$ từ gốc toạ độ $O$ đến đường thẳng đó.
	\choice
	{$\mathrm{d}=3$}
	{$\mathrm{d}=\sqrt{3}$}
	{\True$\mathrm{d}=\dfrac{3\sqrt{5}}{5}$}
	{$\mathrm{d}=\dfrac{5}{5}$}
	
	\loigiai{
		Đặt $z = x + yi$ ($x,y \in \mathbb{R}$). \\
		Khi đó, ta có
		\begin{eqnarray*}
			& & |z - i| = |2 - 3i - z| \\
			& \Leftrightarrow & |x + yi - i| = |2 - 3i - x - yi| \\
			& \Leftrightarrow & |x + (y - 1)i| = |(2 - x) + (-3 - y)i| \\
			& \Leftrightarrow & x^2 + (y - 1)^2 = (2 - x)^2 + (-3 - y)^2 \\
			& \Leftrightarrow & x - 2y - 3 = 0. 
		\end{eqnarray*}
		Vậy tập hợp điểm biểu diễn số phức $z$ là đường thẳng $\Delta$ có phương trình $x-2y-3=0$.\\
		Suy ra khoảng cách từ $O$ đến $\Delta$ là $\mathrm{d}=\dfrac{3\sqrt{5}}{5}$.
	}
	
\end{ex}
\begin{ex}%[Lê Thị Thanh Tuyền]%[2D4B2-4]
	Gọi $(H)$ là hình gồm tập hợp các điểm biểu diễn số phức $z$ thoả $1\le \left|z-1\right|\le 2$. Tính diện tích hình $(H)$.
	\choice
	{$S_{(H)}=2\pi$}
	{$S_{(H)}=5\pi$}
	{\True $S_{(H)}=3\pi$}
	{$S_{(H)}=\pi$}
	\loigiai{
		Giả sử $z=x+yi\quad (x,y\in\mathbb{R}$).\\
		 Ta có $1\le \left|z-1\right|\le 2\Leftrightarrow 1\le (x-1)^2+y^2\le 4$.\\ Suy ra điểm biểu diễn số phức $z$ nằm trong hình vành khuyên giới hạn bởi $2$ hình tròn đồng tâm $I(1;0)$ có bán kính lần lượt là $R_1=1$ và $R_2=2$. \\Từ đó suy ra diện tích hình $(H)$ là $S=\pi R_2^2-\pi R_1^2=3\pi$.
	}
\end{ex}

\begin{ex}%[Lê Thị Thanh Tuyền]%[2D4B2-4]
	Tập hợp điểm biểu diễn số phức $z$ thỏa mãn $\vert z+2\vert +\vert z-2\vert =8$ là
	\choice
	{Một đoạn thẳng}
	{Một đường hyperbol}
	{\True Một đường elip}
	{Một đường parabol}
	
	\loigiai{
		Đặt $z = x + yi$ ($x,y \in \mathbb{R}$). \\
		Ta có $\vert z+2\vert +\vert z-2\vert =8 \Leftrightarrow \sqrt{x^2+(y+2)^2}+\sqrt{x^2+(y-2)^2}=8$.\\
		Gọi $M(x;y)$, $F_1(-2;0)$, $F_2(2;0)$ suy ra $MF_1+MF_2=8$.\\
		Suy ra điểm $M$ nằm trên elip $(E)$ có $2a=8 \Leftrightarrow a=4$, ta có $F_1F_2=2c\Leftrightarrow 4=2c \Leftrightarrow c=2$.\\
		Ta có $b^2=a^2-c^2=16-4=12$. Vậy tập hợp các điểm $M$ là elip $(E): \dfrac{x^2}{16}+\dfrac{y^2}{12}=1$}	.
\end{ex}
\begin{ex}%[Lê Thị Thanh Tuyền]%[2D4B2-4]
	Cho số phức $z$ thoả $(z-i)(2+i)$ là một số thuần ảo. Tập hợp các điểm biểu diễn số phức $z$ là
	\choice
	{\True Đường thẳng có phương trình  $2x-y+1=0$}
	{Đường tròn có tâm $I(1;-1)$ và bán kính $R=4$}
	{Đường thẳng có phương trình $x+2y-2=0$}
	{Đường tròn có tâm $I(2;-1)$ và bán kính $R=2$}
	\loigiai{
		Gọi $z=x+yi$, $x,y \in \mathbb{R}$.\\
		$(z-i)(2+i)=(x+yi-i)(2+i)=(2x-y+1)+(x+2y-2)i$.\\
		Để $(z-i)(2+i)$ là một số thuần ảo thì $2x-y+1=0$ hay tập hợp các điểm trên mặt phẳng tọa độ biểu diễn số phức $z$ là đường thẳng $2x-y+1=0$. 	
	}
\end{ex}
\begin{ex}%[Lê Thị Thanh Tuyền]%[2D4B2-4]
	Biết tập hợp các điểm biểu diễn các số phức $z$ thỏa mãn điều kiện $\left|\dfrac{z}{z-1}\right|= 3$ là một đường tròn. Tìm toạ độ tâm $I$ của đường tròn đó.
	\choice
	{\True$I\left(\dfrac{9}{8};0\right)$}
	{$I\left(\dfrac{9}{4};0\right)$}
	{$I\left(-\dfrac{9}{8};0\right)$}
	{$I\left(-\dfrac{9}{4};0\right)$}
	\loigiai{
		Giả sử $z=x+yi\, (x,y \in \mathbb{R})$.\\
		Ta có
		\allowdisplaybreaks
		\begin{eqnarray*}
			&&\left|\dfrac{z}{z-1}\right|= 3\\ 
			& \Leftrightarrow & |z|= 3|z-1|\\
			& \Leftrightarrow & x^2+y^2= 9(x-1)^2+9y^2\\
			& \Leftrightarrow &8x^2+8y^2-18x+9=0\\
			& \Leftrightarrow & x^2+y^2-\dfrac{9}{4}x+\dfrac{9}{8}=0.
		\end{eqnarray*}
		Vậy tập hợp điểm biểu diễn số phức $z$ là một đường tròn có tâm $I\left(\dfrac{9}{8};0\right)$.
	}	
\end{ex}
\begin{ex}%[Lê Thị Thanh Tuyền]%[2D4B1-2]
Cho số phức $z$ thoả $|z|=2$. Biết tập hợp điểm biểu diễn số phức $w$ thoả mãn \\$w=3+i-(3-4i)z$ là một đường tròn. Tìm bán kính $R$ của đường tròn đó.
\choice
{\True$10$}
{$2\sqrt{5}$}
{$5\sqrt{2}$}
{$5\sqrt{5}$}

\loigiai{
	Gọi $w=x+yi \quad (x,y\in\mathbb{R})$.\\
	Ta có $w=3+i-(3-4i)z\Leftrightarrow w-3-i=(-3+4i)z$.\\
	Suy ra $\left|w-3-i\right|=\left|(-3+4i)z\right|=10\Leftrightarrow (x-3)^2+(y-1)^2=100$.\\
	Vậy tập hợp điểm biểu diễn số phức $w$ là một đường tròn có tâm $I(3;1)$ và bán kính $R=10$.
}
	
\end{ex}

\begin{ex}%[Lê Thị Thanh Tuyền]%[2D4B1-2]
Cho số phức $z$ thoả $|z|=\sqrt{5}$. Biết tập hợp điểm biểu diễn số phức $w$ thoả mãn $w=(1+2i)z+i$ là một đường tròn. Tìm bán kính $r$ của đường tròn đó.
\choice
{$r=2\sqrt{5}$}
{$r=\sqrt{5}$}
{$r=10$}
{\True$r=5$}

\loigiai{
	Gọi $w=x+yi \quad (x,y\in\mathbb{R})$.\\
	Ta có $w=(1+2i)z+i\Leftrightarrow w-i=(1+2i)z$.\\
	Suy ra $\left|w-i\right|=\left|(1+2i)z\right|=5\Leftrightarrow x^2+(y-1)^2=25$.\\
	Vậy tập hợp điểm biểu diễn số phức $w$ là một đường tròn có tâm $I(0;1)$ và bán kính $r=5$.
}
\end{ex}
\begin{ex}%[Lê Thị Thanh Tuyền]%[2D4B1-2]
Xét các số phức $z$ thoả mãn $\left|z-2i+1\right|=4$. Biết rằng tập hợp các điểm biểu diễn số phức $w=(12-5i)z+3i$ là một đường tròn. Tìm tâm $I$ của đường tròn đó.
\choice
{$I(-1;2)$}
{\True$I(-2;32)$}
{$I(2;-32)$}
{$I(1;-5)$}
\loigiai{
	
Ta có $w=(12-5i)z+3i\Rightarrow z=\dfrac{w-3i}{12-5i}$.\\
$\left|z-2i+1\right|=4\Leftrightarrow \left|\dfrac{w-3i}{12-5i}-2i+1\right|=4\Leftrightarrow \left|w+2-32i\right|=4\left|12-5i\right|=52 \quad (1)$.\\
Gọi $w=x+yi \quad (x,y\in\mathbb{R})$.\\
$(1)\Rightarrow \left|x+2+(y-32)i\right|=52\Leftrightarrow (x+2)^2+(y-32)^2=52^2$.\\
Vậy tập hợp điểm biểu diễn số phức $w$ là đường tròn có tâm $I(-2;32)$.
}
\end{ex}
\begin{ex}%[Lê Thị Thanh Tuyền]%[2D4B1-2]
Cho số phức $z$ thoả $|z|=3$. Biết tập hợp điểm biểu diễn số phức $w=\overline{z}+i$ là một đường tròn. Tìm toạ độ tâm của đường tròn đó.
\choice
{\True $I(0;1)$}
{$I(0;-1)$}
{$I(1;0)$}
{$I(-1;0)$}
\loigiai{
Gọi $w=x+yi \quad (x,y\in\mathbb{R})$.\\
$|z|=3\Rightarrow |\overline{z}|=3\Leftrightarrow |w-i|=3\Leftrightarrow x^2+(y-1)^2=9$.\\
Vậy tập hợp điểm biểu diễn số phức $w$ là một đường tròn có tâm $I(0;1)$.
}	
\end{ex}
\begin{ex}%[Lê Thị Thanh Tuyền]%[2D4B1-2]

Cho hai số phức $z, w$ thoả $|z|=10$ và $\overline{z}=(3+4i)\overline{w}$. Tập hợp điểm biểu diễn số phức $w$ là 
\choice
{Đường tròn tâm $I(2;0)$ và bán kính $R=2$}
{\True Đường tròn tâm $O(0;0)$ và bán kính $R=2$}
{Đường tròn tâm $I(0;2)$ và bán kính $R=2$}
{Đường tròn tâm $O(0;0)$ và bán kính $R=1$}
\loigiai{
Gọi $w=x+yi \quad (x,y\in\mathbb{R})$.\\
Ta có $\overline{z}=(3+4i)\overline{w}\Leftrightarrow |\overline{z}|=|(3+4i)\overline{w}|=5|\overline{w}|\Leftrightarrow |z|=5|w|\Leftrightarrow |w|=2\Leftrightarrow x^2+y^2=4$.\\
Vậy tập hợp điểm biểu diễn số phức $w$ là đường tròn có tâm $O(0;0)$ và bán kính $R=2$.
}
\end{ex}

\begin{ex}%[Lê Thị Thanh Tuyền]%[2D4B2-4]
Xét các số phức $z$ thỏa mãn $|z-1|=|\overline{z}-i|$. Quỹ tích các điểm biểu diễn của số phức $w=\left(3-4i\right)z+i$ là
\choice
{\True Một đường thẳng}
{Một đường tròn}
{Một đường parabol}
{Một đường elip}
\loigiai
{
	Đặt $z=x+yi\Rightarrow \overline{z}=x-yi$.\\
	Ta có \\
	$|z-1|=\sqrt{\left(x-1\right)^2+y^2}$ và
	$|\overline{z} -i|=\sqrt{x^2+\left(y+1\right)^2}$.\\
	Theo giả thiết thì $\sqrt{\left(x-1\right)^2+y^2}=\sqrt{x^2+\left(y+1\right)^2}\Leftrightarrow y=-x$.\\
	Xét số phức $w=\left(3-4i\right)\left(x+yi\right)+i=\left(3x+4y\right)+\left(3y-4x+1\right)i=-x+(-7x+1)i$.\\
	Gọi $M'\left(x';y'\right)$ là điểm biểu diễn số phức $w$.\\
	$\Rightarrow \heva{& x'=-x \\ & y'=-7x+1}\Leftrightarrow y'=7x'+1\Leftrightarrow 7x'-y'+1=0$.\\
	Vậy tập hợp điểm biểu diễn số phức $w$ là đường thẳng có phương trình $7x-y+1=0$.
}	
\end{ex}
\begin{ex}%[Lê Thị Thanh Tuyền]%[2D4B2-4]
	Xét số phức $z$ thỏa mãn $|z|=\sqrt{2}$. Trên mặt phẳng tọa độ $Oxy$, tập hợp điểm biểu diễn các số phức $w=\dfrac{4+iz}{1+z}$ là một đường tròn có bán kính bằng
	\choice
	{\True $\sqrt{34}$}
	{$26$}
	{$34$}
	{$\sqrt{26}$}
	\loigiai{
		\begin{eqnarray*}
			&&w=\dfrac{4+iz}{1+z} \Leftrightarrow (1+z)w=4+iz \Leftrightarrow z(w-i)=4-w \\
			&\Leftrightarrow& |z| \cdot |w-i|=|4-w| \Leftrightarrow \sqrt{2} \cdot |w-i|=|4-w|.\quad(*)
		\end{eqnarray*}		
		Gọi $w=x+yi$, $\left(x,y \in \mathbb{R}\right)$ khi đó thay vào $(*)$ ta có:
		\begin{eqnarray*}
			&&\sqrt{2} \cdot |x+yi-i|=|4-x-yi| \Leftrightarrow 2[x^2+(y-1)^2]=(x-4)^2+y^2 \\
			&\Leftrightarrow& x^2+y^2+8x-4y-14=0 \Leftrightarrow (x+4)^2+(y-2)^2=34.
		\end{eqnarray*}
		Vậy tập hợp điểm biểu diễn các số phức $w=\dfrac{4+iz}{1+z}$ là một đường tròn có bán kính bằng $\sqrt{34}$.}
	
\end{ex}

\Closesolutionfile{ans}
%======================
\subsection{Bảng đáp án}
\inputansbox{8}{ans/ANS-DANG-1}

