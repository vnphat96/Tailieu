%Dạng 11
\setcounter{ex}{0}
\section{Góc giữa hai mặt phẳng}
\subsection{Kiến thức cần nhớ}
\begin{khung}
		\faCheckSquareO \, Trong không gian $Oxyz$, cho hai mặt phẳng $(P)\colon a_{1}x+b_{1}y+c_{1}z+d_{1}=0$ và \linebreak $(Q)\colon a_{2}x+b_{2}y+c_{2}z+d_{2}=0$. Khi đó
		\[\cos\left((P), (Q)\right)=\left|\cos\left(\overrightarrow{n}_{(P)}, \overrightarrow{n}_{(Q)}\right)\right|=\dfrac{\left|a_{1}\cdot a_{2}+b_{1}\cdot b_{2}+c_{1}\cdot c_{2}\right|}{\sqrt{a_{1}^{2}+b_{1}^{2}+c_{1}^{2}}\cdot\sqrt{a_{2}^{2}+b_{2}^{2}+c_{2}^{2}}}.\]
		\faCheckSquareO \, Lưu ý: $0^\circ\le \left((P), (Q)\right)\le 90^\circ$.
\end{khung}

\subsection{Bài tập mẫu}
\setcounter{vd}{10}
\Opensolutionfile{ans}[ans/ANS-DANG-1]
\begin{khung}
	\begin{vd}[Đề minh họa BGD 2022-2023]%[1D2Y2-1]
		Trong không gian $Oxyz$, góc giữa hai mặt phẳng $(Oxy)$ và $(Oyz)$ bằng
		\choice
		{$30^{\circ}$}
		{$45^{\circ}$}
		{$60^{\circ}$}
		{\True $90^{\circ}$}
		\loigiai{ 
			Mặt phẳng $(Oxy)$ và $(Oyz)$ vuông góc với nhau nên góc giữa chúng bằng $90^\circ$.
		}
	\end{vd}
\end{khung}

\subsection{Bài tập tương tự và phát triển}
\begin{ex}%[2H3Y2-5]
	Trong không gian $Oxyz$, cho hai mặt phẳng $(P)\colon x-2y-z+1=0$ và \linebreak $(Q)\colon x+y+2z+7=0$. Tính góc giữa hai mặt phẳng đó.
	\choice
	{$30^{\circ}$}
	{\True $60^{\circ}$}
	{$45^{\circ}$}
	{$120^{\circ}$}
	\loigiai{
		$\overrightarrow{n}_{P}=(1;-2;-1)$ là một véctơ pháp tuyến của $(P)$.\\
		$\overrightarrow{n}_{Q}=(1;1;2)$ là một véctơ pháp tuyến của $(Q)$.\\
		Gọi $\alpha$ là góc giữa hai mặt phẳng $(P)$ và $(Q)$, ta có
		\[\cos \alpha=\dfrac{\left|\overrightarrow{n}_{P}\cdot \overrightarrow{n}_{Q}\right|}{\left|\overrightarrow{n}_{P}\right|\cdot\left|\overrightarrow{n}_{Q}\right|}=\dfrac{|1-2-2|}{\sqrt{6}\cdot\sqrt{6}}=\dfrac{1}{2} \Rightarrow \alpha=60^{\circ}.\]
	}
\end{ex}

%%==========Câu 2
\begin{ex}%[2H3Y2-5]
	Trong không gian $Oxyz$, cho mặt phẳng $(P)\colon x-z-3=0$. Tính góc giữa $(P)$ và mặt phẳng $(Oxy)$.
	\choice
	{\True $45^{\circ}$}
	{$90^{\circ}$}
	{$30^{\circ}$}
	{$60^{\circ}$}
	\loigiai{
		Mặt phẳng $(P)$ có véc-tơ pháp tuyến $\overrightarrow{n}_{1}=(1; 0;-1)$, mặt phẳng $(Oxy)$ có véc-tơ pháp tuyến $\overrightarrow{n}_{2}=(0; 0; 1)$.\\
		Gọi $\alpha$ là góc giữa mặt phẳng $(P)$ và mặt phẳng $(Oxy)$, ta có $\cos\alpha=\dfrac{1}{\sqrt{2}} \Rightarrow \alpha=45^{\circ}$.
	}
\end{ex}

%%==========Câu 3
\begin{ex}%[2H3Y2-5]
	Trong không gian $Oxyz$, biết hình chiếu của $O$ lên mặt phẳng $(P)$ là $H(2;-1;-2)$. Số đo góc giữa mặt phẳng $(P)$ với mặt phẳng $(Q)\colon x-y-5=0$ là
	\choice
	{$90^{\circ}$}
	{$60^{\circ}$}
	{\True $45^{\circ}$}
	{$30^{\circ}$}
	\loigiai{
		Gọi $\alpha$ là góc giữa mặt phẳng $(P)$ với mặt phẳng $(Q)$.\\
		Ta có $\overrightarrow{OH}=(2;-1;-2)$ là một véctơ pháp tuyến của mặt phẳng $(P)$, $\overrightarrow{n}=(1;-1;0)$ là một véctơ pháp tuyến của mặt phẳng $(Q)$.\\
		Khi đó
		$\cos\alpha=\left|\cos(\overrightarrow{OH},\overrightarrow{n})\right|=\dfrac{\left|\overrightarrow{OH}\cdot \overrightarrow{n}\right|}{\left|\overrightarrow{OH}\right|\cdot\left|\overrightarrow{n}\right|}=\dfrac{1}{\sqrt{2}} \Rightarrow \alpha=45^{\circ}$.
	}
\end{ex}

%%==========Câu 4
\begin{ex}%[2H3Y2-5]
	Trong không gian $Oxyz$, cho hai mặt phẳng $(P)\colon x-y-6=0$ và $(Q)$. Biết rằng điểm $H(2;-1;-2)$ là hình chiếu vuông góc của gốc tọa độ $O(0;0;0)$ xuống mặt phẳng $(Q)$. Số đo góc giữa mặt phẳng $(P)$ và mặt phẳng $(Q)$ bằng
	\choice
	{\True $45^{\circ}$}
	{$60^{\circ}$}
	{$30^{\circ}$}
	{$90^{\circ}$}
	\loigiai{
		Mặt phẳng $(P)$ có một véc-tơ pháp tuyến $\overrightarrow{n}=(1;-1;0)$.\\
		Vì $H(2;-1;-2)$ là hình chiếu vuông góc của gốc tọa độ $O(0;0;0)$ xuống mặt phẳng $(Q)$ nên $(Q)$ nhận $\overrightarrow{OH}=(2;-1;-2)$ là vectơ pháp tuyến.\\
		Ta có 
		$\begin{aligned}[t]
			\cos \left((P),(Q)\right)&=\left|\cos\left(\overrightarrow{n},\overrightarrow{OH}\right)\right|=\dfrac{\left|\overrightarrow{n}\cdot \overrightarrow{OH}\right|}{\left|\overrightarrow{n}\right|\cdot\left|\overrightarrow{OH}\right|}\\
			& =\dfrac{\left|1\cdot 2+(-1)\cdot(-1)+0\cdot(-2)\right|}{\sqrt{1^2+(-1)^2+0^2}\sqrt{2^2+(-1)^2+(-2)^2}}\\&=\dfrac{3}{3\sqrt{2}}=\dfrac{\sqrt{2}}{2}.
		\end{aligned}$\\
		Vậy góc giữa mặt phẳng $(P)$ và mặt phẳng $(Q)$ bằng $45^{\circ}$.
	}
\end{ex}
%%==========Câu 5
\begin{ex}%[2H3Y2-5]
	Trong không gian $Oxyz$, góc giữa mặt phẳng $(\alpha)\colon\sqrt{2}x+y+z-5=0$ và mặt phẳng $(Oxy)$ là
	\choice
	{$90^{\circ}$}
	{$30^{\circ}$}
	{$45^{\circ}$}
	{\True $60^{\circ}$}
	\loigiai{
		Ta có vec-tơ pháp tuyến của $(\alpha)$ và $(Oxy)$ lần lượt là $\overrightarrow{n}=\left(\sqrt{2};1;1\right)$ và $\overrightarrow{k}=(0;0;1)$.\\
		Gọi $\varphi$ là góc giữa mặt phẳng
		$(\alpha)$ và $(Oxy)$, khi đó $\cos \varphi=\dfrac{\left|\overrightarrow{n}\cdot\overrightarrow{k}\right|}{\left|\overrightarrow{n}\right|\cdot\left|\overrightarrow{k}\right|}=\dfrac{1}{2} \Rightarrow \varphi=60^{\circ}$.
	}
\end{ex}

%%==========Câu 6
\begin{ex}%[2H3Y2-5]
	Trong không gian $Oxyz$, cho điểm $H(2;-1;-2)$ là hình chiếu vuông góc của gốc tọa độ $O$ xuống mặt phẳng $(P)$, số đo góc giữa mặt $(P)$ và mặt phẳng $(Q)\colon x-y-11=0$ bằng bao nhiêu?
	\choice
	{$90^{\circ}$}
	{$60^{\circ}$}
	{\True $45^{\circ}$}
	{$30^{\circ}$}
	\loigiai{
		Vì $H(2;-1;-2)$ là hình chiếu vuông góc của $O$ xuống mặt $(P)$ nên $OH\perp (P)$.\\
		Do đó $(P)$ có vec-tơ pháp tuyến là $\overrightarrow{n}_{(P)}=(2;-1;-2)$.\\
		$(Q)$ có vectơ pháp tuyến là $\overrightarrow{n}_{(Q)}=(1;-1; 0)$.
		\[\cos ((P),(Q))=\left|\cos \left(\overrightarrow{n}_{(P)}, \overrightarrow{n}_{(Q)}\right)\right|=\dfrac{\left|\overrightarrow{n}_{(P)}\cdot \overrightarrow{n}_{(P)}\right|}{\left|\overrightarrow{n}_{(P)}\right|\cdot\left|\overrightarrow{n}_{(P)}\right|}=\dfrac{\left|2\cdot 1-1\cdot(-1)-2\cdot 0\right|}{\sqrt{4+1+4}\cdot\sqrt{1+1+0}}=\dfrac{\sqrt{2}}{2}.\]
		Suy ra $\left((P),(Q)\right)=45^{\circ}$.
	}
\end{ex}

%%==========Câu 7
\begin{ex}%[2H3Y2-5]
	Trong không gian $Oxyz$, cho hai mặt phẳng $(P)\colon x-2y-z+2=0$ và $(Q)\colon 2x-y+z+1=0$. Góc giữa $(P)$ và $(Q)$ là
	\choice
	{$120^{\circ}$}
	{$90^{\circ}$}
	{$30^{\circ}$}
	{\True $60^{\circ}$}
	\loigiai{
		$(P)\colon x-2y-z+2=0$ có véc-tơ pháp tuyến là $\overrightarrow{n}_{1}=(1;-2;-1)$.\\
		$(Q)\colon 2x-y+z+1=0$ có véc-tơ pháp tuyến là $\overrightarrow{n}_{2}=(2;-1; 1)$.\\
		Áp dụng công thức
		\[\cos ((P),(Q))=\dfrac{\left|\overrightarrow{n}_{1}\cdot \overrightarrow{n}_{2}\right|}{\left|\overrightarrow{n}_{1}\right|\cdot\left|\overrightarrow{n}_{2}\right|}=\dfrac{|1\cdot 2+(-2)\cdot(-1)+(-1)\cdot 1|}{\sqrt{1^2+(-2)^2+(-1)^2}\sqrt{2^2+(-1)^2+1^2}}=\dfrac{1}{2}.\]
		Suy ra góc giữa $(P)$ và $(Q)$ là $60^{\circ}$.
	}
\end{ex}

%%==========Câu 8
\begin{ex}%[2H3B2-5]
	Trong không gian $Oxyz$, cho hai mặt phẳng $(P)\colon x+(m+1)y-2z+m=0$ và $(Q)\colon 2x-y+3=0$, với $m$ là tham số thực. Để $(P)$ vuông góc với $(Q)$ thì giá trị của $m$ bằng bao nhiêu?
	\choice
	{$m=3$}
	{$m=-1$}
	{$m=-5$}
	{\True $m=1$}
	\loigiai{
		Mặt phẳng $(P)$ có véc-tơ pháp tuyến $\overrightarrow{n}_{1}=(1; m+1;-2)$, mặt phẳng $(Q)$ có véc-tơ pháp tuyến $\overrightarrow{n}_{2}=(2;-1;0)$.\\
		Để $(P)\perp(Q) \Leftrightarrow \overrightarrow{n}_{1}\perp \overrightarrow{n}_{2} \Leftrightarrow \overrightarrow{n}_{1}\cdot \overrightarrow{n}_{2}=0 \Leftrightarrow 2-m-1=0 \Leftrightarrow m=1$.
	}
\end{ex}

%%==========Câu 9
\begin{ex}%[2H3B2-5]
	Trong không gian $Oxyz$, cho mặt phẳng $(P)$ đi qua các điểm $A(-2;0;0)$, $B(0;3;0)$, $C(0;0;-3)$. Mặt phẳng $(P)$ vuông góc với mặt phẳng nào trong các mặt phẳng sau?
	\choice
	{$3x-2y+2z+6=0$}
	{$x-2y-z-3=0$}
	{\True $2x+2y-z-1=0$}
	{$x+y+z+1=0$}
	\loigiai{
		Mặt phẳng $(P)$ cắt các trục tọa độ $Ox$, $Oy$, $Oz$ lần lượt tại các điểm $A(-2;0;0)$, $B(0;3;0)$, $C(0;0;-3)$.\\
		Áp dụng phương trình của mặt phẳng theo đoạn chắn, ta có phương trình của mặt phẳng $(P)$ là \[\dfrac{x}{-2}+\dfrac{y}{3}+\dfrac{z}{-3}=1 \Leftrightarrow 3x-2y+2z+6=0.\]
		Mặt phẳng $(P)$ có vectơ pháp tuyến $\overrightarrow{n}_{(P)}=(3;-2;2)$.
		\begin{itemize}
			\item Mặt phẳng $3x-2y+2z+6=0$ trùng với mặt phẳng $(P)$ nên loại $3x-2y+2z+6=0$.
			\item Mặt phẳng $x-2y-z-3=0$ có véc-tơ pháp tuyến $\overrightarrow{n}_1=(1;-2;-1)$. Ta có $\overrightarrow{n}_{(P)}\cdot \overrightarrow{n}_1=5 \neq 0$ nên loại $x-2y-z-3=0$.
			\item Mặt phẳng $2x+2y-z-1=0$ có véc-tơ pháp tuyến $\overrightarrow{n}_2=(2;2;-1)$. Ta có $\overrightarrow{n}_{(P)}\cdot \overrightarrow{n}_2=0$ nên chọn $2x+2y-z-1=0$.
			\item Mặt phẳng $x+y+z+1=0$ có véc-tơ pháp tuyến $\overrightarrow{n}_3=(1;1;1)$. Ta có $\overrightarrow{n}_{(P)}\cdot \overrightarrow{n}_3=3 \neq 0$ nên loại $x+y+z+1=0$.
		\end{itemize}
	}
\end{ex}

%%==========Câu 10
\begin{ex}%[2H3B2-5]
	Trong không gian $Oxyz$, cho mặt phẳng $x-my+z-1=0\;(m \in \mathbb{R})$, mặt phẳng $(Q)$ chứa trục $Ox$ và qua điểm $A(1;-3;1)$. Tìm số thực $m$ để hai mặt phẳng $(P)$, $(Q)$ vuông góc.
	\choice
	{$m=-\dfrac{1}{3}$}
	{$m=\dfrac{1}{3}$}
	{\True $m=3$}
	{$m=-3$}
	\loigiai{
		Ta có $\overrightarrow{OA}=(1;-3;1), \vec{i}=(1;0;0)$.\\
		Mặt phẳng $(Q)$ qua điểm $A(1;-3;1)$ và chứa trục $Ox$ suy ra $(Q)$ có véc-tơ pháp tuyến\\ $\overrightarrow{n}_{Q}=\left[\overrightarrow{OA}, \vec{i}\right]=(0;1;3)$.\\
		Mặt phẳng $(P)$ có véc-tơ pháp tuyến $\overrightarrow{n}_{P}=(1;-m;1)$.
		$$(P)\perp(Q) \Leftrightarrow \overrightarrow{n}_{P}\cdot \overrightarrow{n}_{Q}=0 \Leftrightarrow 0\cdot 1+1\cdot (-m)+1\cdot 3=0 \Leftrightarrow m=3.$$
	}
\end{ex}

\begin{ex}%[2H3B2-5]
	Trong không gian $Oxyz$, cho mặt phẳng $(P)\colon x+2y-2z+3=0$, mặt phẳng $(Q)\colon x-3y+5z-2=0$. Cô-sin của góc giữa hai mặt phẳng $(P)$, $(Q)$ là
	\choice
	{$-\dfrac{5}{7}$}
	{$-\dfrac{\sqrt{35}}{7}$}
	{$\dfrac{5}{7}$}
	{\True $\dfrac{\sqrt{35}}{7}$}
	\loigiai{
		Ta có véc-tơ pháp tuyến của mặt phẳng $(P)$ là $\overrightarrow{n}_{(P)}=(1;2;-2)$, véc-tơ pháp tuyến của mặt phẳng $(Q)$ là $\overrightarrow{n}_{Q}=(1;-3;5)$.\\
		Gọi $\alpha$ là góc giữa hai mặt phẳng $(P)$, $(Q)$ ta có
		\[\cos \alpha=\dfrac{\left|\overrightarrow{n}_{(P)}\cdot \overrightarrow{n}_{(Q)}\right|}{\left|\overrightarrow{n}_{(P)}\right|\cdot\left|\overrightarrow{n}_{(Q)}\right|}=\dfrac{\left|1\cdot 1+2\cdot(-3)-2\cdot 5\right|}{\sqrt{1^2+2^2+(-2)^2}\cdot\sqrt{1^2+(-3)^2+5^2}}=\dfrac{15}{3\sqrt{35}}=\dfrac{\sqrt{35}}{7}.\]
	}
\end{ex}

\begin{ex}%[2H3B2-5]
	Trong không gian $Oxyz$, gọi $\alpha$ là góc giữa hai mặt phẳng $(P)\colon x-\sqrt{3}y+2z+1=0$ và mặt phẳng $(Oxy)$. Khẳng định nào sau đây đúng?
	\choice
	{$\alpha=30^{\circ}$}
	{$\alpha=60^{\circ}$}
	{$\alpha=90^{\circ}$}
	{\True $\alpha=45^{\circ}$}
	\loigiai{
		Mặt phẳng $(P)$ có một véc-tơ pháp tuyến là $\overrightarrow{n}_{(P)}=\left(1;-\sqrt{3};2\right)$.\\
		Mặt phẳng $(Oxy)\colon z=0$ có một véc-tơ pháp tuyến là $\overrightarrow{n}=(0;0;1)$.\\
		Ta có $\cos\alpha=\dfrac{\left|\overrightarrow{n}_{(P)}\cdot \overrightarrow{n}\right|}{\left|\overrightarrow{n}_{(P)}\right|\cdot\left|\overrightarrow{n}\right|}=\dfrac{1}{\sqrt{2}} \Rightarrow \alpha=45^{\circ}$.
	}
\end{ex}

\begin{ex}%[2H3B2-5]
	Trong hệ tọa độ $Oxyz$, cho hai mặt phẳng $(P)\colon\dfrac{x-2}{3}+\dfrac{y-1}{2}+\dfrac{z-4}{-6}=1$ và $(Q)\colon x+2y+3z+7=0$. Tính $\tan$ góc tạo bởi hai mặt phẳng đã cho.
	\choice
	{$\dfrac{3}{\sqrt{19}}$}
	{$\dfrac{3}{5\sqrt{19}}$}
	{$\dfrac{5}{3\sqrt{19}}$}
	{\True $\dfrac{3\sqrt{19}}{5}$}
	\loigiai{
		Ta có $(P)\colon\dfrac{x-2}{3}+\dfrac{y-1}{2}+\dfrac{z-4}{-6}=1 \Leftrightarrow (P)\colon 2x+3y-z-9=0$ suy ra $(P)$ có một véc-tơ pháp tuyến là $\overrightarrow{n}_{(P)}=(2;3;-1)$.\\
		$(Q)\colon x+2y+3z+7=0 \Rightarrow (Q)$ có một véc-tơ pháp tuyến là $\overrightarrow{n}_{(Q)}=(1;2;3)$.\\
		Gọi $\alpha\,\,\left(0^\circ \leq \alpha \leq 90^\circ\right)$ là góc giữa hai mặt phẳng $(P)$ và $(Q)$.
		Ta có
		\[\cos \alpha=\dfrac{\left|\overrightarrow{n}_{(P)}\cdot \overrightarrow{n}_{(Q)}\right|}{\left|\overrightarrow{n}_{(P)}\right|\cdot\left|\overrightarrow{n}_{(Q)}\right|}=\dfrac{\left|2\cdot 1+3\cdot 2+(-1)\cdot 3\right|}{\sqrt{2^2+3^2+(-1)^2}\cdot\sqrt{1^2+2^2+3^2}}=\dfrac{5}{14}.\]
		Từ đó ta ta tìm được
		\[\tan^2 \alpha=\dfrac{1}{\cos^2 \alpha}-1=\dfrac{171}{25} \Rightarrow \tan \alpha=\dfrac{3\sqrt{19}}{5}.\]
	}
\end{ex}

\begin{ex}%[2H3B2-5]
	Trong không gian $Oxyz$, cho mặt phẳng $(P)\colon x-2y+2z-5=0$. Xét mặt phẳng
	$(Q)\colon x+(2m-1)z+7=0$, với $m$ là tham số thực. Tìm tất cả giá trị của $m$ để $(P)$ tạo với $(Q)$ góc $\dfrac{\pi}{4}$.
	\choice
	{\True $\hoac{&m=1\\&m=4}$}
	{$\hoac{&m=2\\&m=-2\sqrt{2}}$}
	{$\hoac{&m=2\\&m=4}$}
	{$\hoac{&m=4\\&m=\sqrt{2}}$}
	\loigiai{
		Mặt phẳng $(P)$, $(Q)$ có véc-tơ pháp tuyến lần lượt là $\overrightarrow{n}_{(P)}=(1;-2;2)$, $\overrightarrow{n}_{(Q)}=(1;0;2m-1)$. Vì $(P)$ tạo với $(Q)$ góc $\dfrac{\pi}{4}$ nên
		\begin{eqnarray*}
			\cos\dfrac{\pi}{4}=\left|\cos \left(\overrightarrow{n}_{(P)}, \overrightarrow{n}_{(Q)}\right)\right|&\Leftrightarrow&\dfrac{1}{\sqrt{2}}=\dfrac{|1+2(2m-1)|}{3\cdot\sqrt{1+(2m-1)^2}}\\
			&\Leftrightarrow &2(4m-1)^2=9\left(4m^2-4m+2\right)\\
			&\Leftrightarrow &4m^2-20 m+16=0\\
			&\Leftrightarrow &\hoac{&m=1\\&m=4.}
		\end{eqnarray*}
	}
\end{ex}

\begin{ex}%[2H3B2-7]
	Trong không gian $Oxyz$, cho hai mặt phẳng $(\alpha)\colon x+y+z-1=0$ và \linebreak $(\beta)\colon 2x-y+mz-m+1=0$, với $m$ là tham số thực. Giá trị của $m$ để $(\alpha)\perp(\beta)$ là
	\choice
	{$1$}
	{$-4$}
	{\True $-1$}
	{$0$}
	\loigiai{
		Mặt phẳng $(\alpha)$ có véc-tơ pháp tuyến $\overrightarrow{n}_(\alpha)=(1;1;1)$ và $(\beta)$ có véc-tơ pháp tuyến $\overrightarrow{n}_(\beta)=(2;-1;m)$.\\
		Ta có $(\alpha)\perp(\beta) \Leftrightarrow \overrightarrow{n}_(\alpha)\perp \overrightarrow{n}_(\beta) \Leftrightarrow \overrightarrow{n}_(\alpha)\cdot \overrightarrow{n}_(\beta)=0 \Leftrightarrow 1+m=0 \Leftrightarrow m=-1$.
	}
\end{ex}

\begin{ex}%[2H3B2-5]
	Trong không gian $Oxyz$, cho mặt phẳng $(\alpha)\colon 2x-y+z-3=0$ và $(\beta)\colon 3x-4y+5z=0$. Góc tạo bởi hai mặt phẳng $(\alpha)$ và $(\beta)$ bằng
	\choice
	{$90^{\circ}$}
	{\True $30^{\circ}$}
	{$60^{\circ}$}
	{$45^{\circ}$}
	\loigiai{
		Gọi $\varphi$ là góc tạo bởi hai mặt phẳng $(\alpha)$ và $(\beta)$.\\
		Ta có $\overrightarrow{n}_{(\alpha)}=(2;-1;1), \overrightarrow{n}_{(\beta)}=(3;-4;5)$. Suy ra
		\[\cos \varphi=\dfrac{\left|\overrightarrow{n}_{(\alpha)}\cdot \overrightarrow{n}_{(\beta)}\right|}{\left|\overrightarrow{n}_{(\alpha)}\right|\cdot\left|\overrightarrow{n}_{(\beta)}\right|}=\dfrac{|6+4+5|}{\sqrt{2^2+(-1)^2+1^2}\cdot\sqrt{3^2+(-4)^2+5^2}}=\dfrac{\sqrt{3}}{2} \Rightarrow \varphi=30^{\circ}.\]	
	}
\end{ex}

\begin{ex}%[2H3B2-5]
	Trong không gian $Oxyz$, cho hai mặt phẳng $(P)\colon x-y-6=0$ và $(Q)$. Biết rằng điểm $H(2;-1;-2)$ là hình chiếu vuông góc của gốc tọa độ $O(0;0;0)$ xuống mặt phẳng $(Q)$. Số đo góc giữa mặt phẳng $(P)$ và mặt phẳng $(Q)$ bằng
	\choice
	{$60^{\circ}$}
	{\True $45^{\circ}$}
	{$30^{\circ}$}
	{$90^{\circ}$}
	\loigiai{
		Ta có $(P)\colon x-y-6=0 \Rightarrow \overrightarrow{n}_{(P)}=(1;-1;0)$.\\
		Theo giả thiết điểm $H(2;-1;-2)$ là hình chiếu vuông góc của gốc tọa độ $O(0;0;0)$ xuống mặt phẳng $(Q)$ nên $\overrightarrow{n}_{(Q)}=\overrightarrow{OH}=(2;-1;-2)$. Do đó \[\cos\left(\widehat{(P),(Q)}\right)=\dfrac{\left|\overrightarrow{n}_{(P)}\cdot  \overrightarrow{n}_{(Q)}\right|}{\left|\overrightarrow{n}_{(P)}\right|\cdot\left|\overrightarrow{n}_{(Q)}\right|}=\dfrac{|1.2+(-1)(-1)+0\cdot(-2)|}{\sqrt{1^2+(-1)^2+0^2}\cdot\sqrt{2^2+(-1)^2+(-2)^2}}=\dfrac{\sqrt{2}}{2}.\]
		Suy ra $\left(\widehat{(P),(Q)}\right)=45^{\circ}$.
	}
\end{ex}

\begin{ex}%[2H3B2-5]
	Trong không gian $Oxyz$, cho hai mặt phẳng $(P)$ và $(Q)$ lần lượt có hai véc-tơ pháp tuyến là $\overrightarrow{n}_{(P)}$ và $\overrightarrow{n}_{(Q)}$. Biết góc giữa hai véc-tơ $\overrightarrow{n}_{(P)}$ và $\overrightarrow{n}_{(Q)}$ bằng $30^{\circ}$. Góc giữa hai măt phẳng $(P)$ và $(Q)$ bằng
	\choice
	{\True $30^{\circ}$}
	{$45^{\circ}$}
	{$60^{\circ}$}
	{$90^{\circ}$}
	\loigiai{
		Ta có $\left(\widehat{\overrightarrow{n}_{(P)}, \overrightarrow{n}_{(Q)}}\right)=30^{\circ} \Rightarrow \left(\widehat{(P),(Q)}\right)=30^{\circ}$.
	}
\end{ex}

\begin{ex}%[2H3B2-5]
	Trong không gian $Oxyz$, cho hai mặt phẳng $(P)$ và $(Q)$ lần lượt có hai véc-tơ pháp tuyến là $\overrightarrow{n}_{(P)}$ và $\overrightarrow{n}_{(Q)}$. Biết góc giữa hai véc-tơ $\overrightarrow{n}_{(P)}$ và $\overrightarrow{n}_{(Q)}$ bằng $120^{\circ}$. Góc giữa hai mặt phẳng $(P)$ và $(Q)$ bằng
	\choice
	{$30^{\circ}$}
	{$45^{\circ}$}
	{\True $60^{\circ}$}
	{$90^{\circ}$}
	\loigiai{
		Ta có $\left(\widehat{\overrightarrow{n}_{(P)}, \overrightarrow{n}_{(Q)}}\right)=120^{\circ} \Rightarrow \left(\widehat{(P) ,(Q)}\right)=180^{\circ}-120^{\circ}=60^{\circ}$.	
	}
\end{ex}

\begin{ex}%[2H3B2-5]
	Trong không gian $Oxyz$, cho hai mặt phẳng $(P)$ và $(Q)$ lần lượt có hai véc-tơ pháp tuyến là $\overrightarrow{n}_{(P)}$ và $\overrightarrow{n}_{(Q)}$. Biết cô-sin góc giữa hai véc-tơ $\overrightarrow{n}_{(P)}$ và $\overrightarrow{n}_{(Q)}$ bằng $\dfrac{1}{2}$. Góc giữa hai mặt phẳng $(P)$ và $(Q)$ bằng
	\choice
	{$30^{\circ}$}
	{$45^{\circ}$}
	{\True $60^{\circ}$}
	{$90^{\circ}$}
	\loigiai{
		Ta có $\cos \left(\widehat{(P),(Q)}\right)=\left|\cos \left(\widehat{\overrightarrow{n}_{(P)}, \overrightarrow{n}_{(Q)}}\right)\right|=\left|\dfrac{1}{2}\right|=\dfrac{1}{2} \Rightarrow \left(\widehat{(P),(Q)}\right)=60^{\circ}$.
	}
\end{ex}

\Closesolutionfile{ans}
%======================
\subsection{Bảng đáp án}
\inputansbox{8}{ans/ANS-DANG-1}