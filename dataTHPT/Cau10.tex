%Câu 10
\setcounter {section} {9}
\setcounter{ex}{0}
\section{Phương trình mặt cầu}
\subsection{Kiến thức cần nhớ}
\begin{khung}
	\begin{itemize}
		\item Mặt cầu $(S)\colon (x-a)^2+(y-b)^2+(z-c)^2=R^2$ có tâm $I(a;b;c)$, bán kính $R$.
		\item Mặt cầu $(S)\colon x^2+y^2+z^2-2ax-2by-2cz+d=0$ (với $a^2+b^2+c^2-d>0$) có tâm $I(a;b;c)$, bán kính $R=\sqrt{a^2+b^2+c^2-d}$.
	\end{itemize}
\end{khung}
\subsection{Bài tập mẫu}
\Opensolutionfile{ans}[ans/ANS-DANG-10]
\begin{khung}
	\begin{vd}[Đề minh họa BGD 2022-2023]%[Nguyễn Tiến, PT ĐMH-2023]%[2H3Y1-3]
		Trong không gian $Oxyz$, cho mặt cầu $(S) \colon x^2+y^2+z^2-2x-4y-6z+1=0$. Tâm của $(S)$ có tọa độ là
		\choice
		{$(-1;-2;-3)$}
		{$(2;4;6)$}
		{$(-2;-4;-6)$}
		{\True $(1;2;3)$}
		\loigiai{
			Gọi $I(a;b;c)$ là tâm của mặt cầu $(S)$, ta có $\heva{&-2a=-2\\&-2b=-4\\&-2c=-6}
			\Leftrightarrow \heva{&a=1\\&b=2\\&c=3.}$\\
			Do đó tâm của $(S)$ có tọa độ là $I(1;2;3)$.
		}
	\end{vd}
\end{khung}
\subsection{Bài tập tương tự và phát triển}
\begin{ex}%[Nguyễn Tiến, PT ĐMH-2023]%[2H3Y1-3]
	Trong không gian $Oxyz$, tâm của mặt cầu $(S)\colon x^2+y^2+z^2-4x+2y+6z-2=0$ là
	\choice
	{$B(-2;1;3)$}
	{\True $D(2;-1;-3)$}
	{$A(-4;2;6)$}
	{$C(4;-2;-6)$}
	\loigiai{
		Mặt cầu $(S)$ có tâm $D(2;-1;-3)$.
	}
\end{ex}
\begin{ex}%[Nguyễn Tiến, PT ĐMH-2023]%[2H3Y1-3]
	Trong không gian $Oxyz$, cho mặt cầu $(S)\colon x^2+y^2+z^2-2x+6y-8z-10=0$. Bán kính $R$ của mặt cầu $(S)$ là
	\choice
	{$R=36$}
	{$R=\sqrt{6}$}
	{$R=\sqrt{114}$}
	{\True $R=6$}
	\loigiai{
		Mặt cầu $(S)$ có tâm $I(1;-3;4)$ và bán kính $R=\sqrt{1^2+(-3)^2+4^2-(-10)}=6$.
	}
\end{ex}
\begin{ex}%[Nguyễn Tiến, PT ĐMH-2023]%[2H3Y1-3]
	Trong không gian $Oxyz$, mặt cầu $(S)\colon (x+4)^2+(y-5)^2+(z+6)^2=9$ có tâm và bán kính lần lượt là
	\choice
	{$I(4;-5;6)$, $R=81$}
	{$I(-4;5;-6)$, $R=81$}
	{$I(4;-5;6)$, $R=3$}
	{\True $I(-4;5;-6)$, $R=3$}
	\loigiai{
		Tọa độ tâm và bán kính mặt cầu lần lượt là $I(-4;5;-6)$, $R=3$.
	}
\end{ex}
\begin{ex}%[Nguyễn Tiến, PT ĐMH-2023]%[2H3Y1-3]
	Trong không gian $Oxyz$, cho mặt cầu $(S)\colon (x-1)^2+(y+3)^2+(z-4)^2=4$. Tìm tọa độ tâm $I$ và bán kính $R$ của mặt cầu $(S)$ là
	\choice
	{$I(1;-3;4)$, $R=4$}
	{$I(-1;3;-4)$, $R=4$}
	{$I(-1;3;-4)$, $R=2$}
	{\True $I(1;-3;4)$, $R=2$}
	\loigiai{
		Ta có tâm $I(1;-3;4)$, $R=2$.
	}
\end{ex}
\begin{ex}%[Nguyễn Tiến, PT ĐMH-2023]%[2H3Y1-3]
	Trong không gian $Oxyz$, mặt cầu $(S)\colon x^2+y^2+z^2-2x+4y-2z-10=0$ có bán kính bằng
	\choice
	{$6$}
	{$3$}
	{\True $4$}
	{$5$}
	\loigiai{
		Bán kính của mặt cầu là $R=\sqrt{1^2+(-2)^2+1^2-(-10)}=4$.
	}
\end{ex}
\begin{ex}%[Nguyễn Tiến, PT ĐMH-2023]%[2H3Y1-3]
	Trong không gian $Oxyz$, cho mặt cầu $(S)\colon (x-2)^2+(y+4)^2+(z-1)^2=9$. Tâm của $(S)$ có tọa độ là
	\choice
	{$(2;4;1)$}
	{$(-2;-4;-1)$}
	{$(-2;4;-1)$}
	{\True $(2;-4;1)$}
	\loigiai{
		Mặt cầu $(S)$ có tâm $I(2;-4;1)$.
	}
\end{ex}
\begin{ex}%[Nguyễn Tiến, PT ĐMH-2023]%[2H3Y1-3]
	Trong không gian $Oxyz$, cho mặt cầu $(S)\colon x^2+y^2+z^2-4x+6y-8z-3=0$. Tâm của $(S)$ có tọa độ là
	\choice
	{$(4;-6;8)$}
	{\True $(2;-3;4)$}
	{$(-4;6;-8)$}
	{$(-2;3;-4)$}
	\loigiai{
		Mặt cầu $(S)$ có tâm $I(2;-3;4)$.
	}
\end{ex}
\begin{ex}%[Nguyễn Tiến, PT ĐMH-2023]%[2H3Y1-3]
	Trong không gian $Oxyz$, mặt cầu $(S)\colon (x+1)^2+(y-2)^2+(z+3)^2=4$ có tâm và bán kính lần lượt là
	\choice
	{\True $I(-1;2;-3)$, $R=2$}
	{$I(1;-2;3)$, $R=2$}
	{$I(1;-2;3)$, $R=4$}
	{$I(-1;2;-3)$, $R=4$}
	\loigiai{
		Tọa độ tâm và bán kính lần lượt là $I(-1;2;-3)$, $R=2$.
	}
\end{ex}
\begin{ex}%[Nguyễn Tiến, PT ĐMH-2023]%[2H3Y1-3]
	Trong không gian $Oxyz$, cho mặt cầu $(S)\colon (x-1)^2+(y-2)^2+z^2=16$. Tâm của $(S)$ có tọa độ là
	\choice
	{$(1;-2;0)$}
	{$(-1;2;0)$}
	{$(-1;-2;0)$}
	{\True $(1;2;0)$}
	\loigiai{
		Tâm của mặt cầu $(S)$ có tọa độ là $(1;2;0)$.
	}
\end{ex}
\begin{ex}%[Nguyễn Tiến, PT ĐMH-2023]%[2H3Y1-3]
	Trong không gian $Oxyz$, cho mặt cầu $(S)\colon x^2+y^2+z^2-2x-4y-6z-2=0$. Tọa độ tâm $I$ của mặt cầu $(S)$ là
	\choice
	{$I(2;4;6)$}
	{$I(-2;-4;-6)$}
	{\True $I(1;2;3)$}
	{$I(-1;-2;-3)$}
	\loigiai{
		Tọa độ tâm của mặt cầu $(S)$ là $I(1;2;3)$.
	}
\end{ex}
\begin{ex}%[Nguyễn Tiến, PT ĐMH-2023]%[2H3Y1-3]
	Trong không gian $Oxyz$, cho mặt cầu $(S)\colon (x+3)^2+(y+1)^2+(z-1)^2=2$. Xác định tọa độ tâm của mặt cầu $(S)$.
	\choice
	{\True $(-3;-1;1)$}
	{$(3;-1;1)$}
	{$(-3;1;-1)$}
	{$(3;1;-1)$}
	\loigiai{
		Mặt cầu $(S)$ có tâm là $I(-3;-1;1)$.
	}
\end{ex}
\begin{ex}%[Nguyễn Tiến, PT ĐMH-2023]%[2H3Y1-3]
	Trong không gian $Oxyz$, cho mặt cầu $(S)\colon (x-1)^2+(y+2)^2+(z-3)^2=9$. Tâm $I$ và bán kính $R$ của mặt cầu là
	\choice
	{$I(1;2;3)$, $R=3$}
	{$I(-1;2;-3)$, $R=3$}
	{\True $I(1;-2;3)$, $R=3$}
	{$I(1;2;-3)$, $R=3$}
	\loigiai{
		Mặt cầu $(S)$ có tâm $I(1;-2;3)$, bán kính $R=3$.
	}
\end{ex}
\begin{ex}%[Nguyễn Tiến, PT ĐMH-2023]%[2H3Y1-3]
	Trong không gian $Oxyz$, cho mặt cầu $(S)\colon x^2+y^2+z^2-6x+4y-8z+4=0$. Tìm tọa độ tâm $I$ và tính bán kính $R$ của mặt cầu $(S)$.
	\choice
	{$I(-3;2;-4)$, $R=5$}
	{\True $I(3;-2;4)$, $R=5$}
	{$I(-3;2;-4)$, $R=25$}
	{$I(3;-2;4)$, $R=25$}
	\loigiai{
		Mặt cầu $(S)$ có tâm $I(3;-2;4)$, bán kính $R=\sqrt{3^2+(-2)^2+4^2-4}=5$.
	}
\end{ex}
\begin{ex}%[Nguyễn Tiến, PT ĐMH-2023]%[2H3Y1-3]
	Trong không gian $Oxyz$, tâm của mặt cầu $(S)\colon x^2+y^2+z^2+2x-4y+6z-1=0$ có tọa độ là
	\choice
	{$(1;-2;3)$}
	{$(2;-4;6)$}
	{$(-2;4;-6)$}
	{\True $(-1;2;-3)$}
	\loigiai{
		Tâm của mặt cầu $(S)$ là $I(-1;2;-3)$.
	}
\end{ex}
\begin{ex}%[Nguyễn Tiến, PT ĐMH-2023]%[2H3Y1-3]
	Trong không gian $Oxyz$, cho mặt cầu $(S)\colon x^2+y^2+z^2-2x+4y+1=0$. Tâm của mặt cầu $(S)$ có tọa độ là
	\choice
	{$(-1;2;0)$}
	{$(2;-1;0)$}
	{\True $(1;-2;0)$}
	{$(-2;1;0)$}
	\loigiai{
		Tọa độ tâm của mặt cầu $(S)$ đã cho là $I(1;-2;0)$.
	}
\end{ex}
\begin{ex}%[Nguyễn Tiến, PT ĐMH-2023]%[2H3Y1-3]
	Trong không gian $Oxyz$, cho mặt cầu $(S)$ có phương trình $(S)\colon x^2+y^2+z^2+4x-4y+8z=0$. Tìm tọa độ tâm $I$ và bán kính $R$.
	\choice
	{$I(2;-2;4)$, $R=2\sqrt{6}$}
	{$I(-2;2;-4)$, $R=24$}
	{$I(2;-2;4)$, $R=24$}
	{\True $I(-2;2;-4)$, $R=2\sqrt{6}$}
	\loigiai{
		Mặt cầu $(S)$ có tâm $I(-2;2;-4)$ và bán kính $R=\sqrt{(-2)^2+2^2+(-4)^2}=2\sqrt{6}$.
	}
\end{ex}
\begin{ex}%[Nguyễn Tiến, PT ĐMH-2023]%[2H3B1-3]
	Trong không gian $Oxyz$, cho mặt cầu $(S)\colon x^2+y^2+z^2-2x+4y-6z-12=0$, gọi $I(a;b;c)$ là tâm của mặt cầu $(S)$. Tính $T=a+b-c$.
	\choice
	{$2$}
	{\True $-4$}
	{$4$}
	{$5$}
	\loigiai{
		Mặt cầu $(S)$ có tâm $I(1;-2;3)$; suy ra $a=1$, $b=-2$, $c=3$.\\
		Vậy $T=a+b-c=-4$.
	}
\end{ex}
\begin{ex}%[Nguyễn Tiến, PT ĐMH-2023]%[2H3Y1-3]
	Trong không gian $Oxyz$, tâm $I$ của mặt cầu $(S)\colon x^2+y^2+z^2-8x-2y+1=0$ có tọa độ là
	\choice
	{\True $I(4;1;0)$}
	{$I(4;-1;0)$}
	{$(-4;1;0)$}
	{$(-4;-1;0)$}
	\loigiai{
		Tọa độ tâm $I$ của mặt cầu $(S)$ là $I(4;1;0)$.
	}
\end{ex}
\begin{ex}%[Nguyễn Tiến, PT ĐMH-2023]%[2H3Y1-3]
	Trong không gian $Oxyz$, cho mặt cầu $(S)\colon (x-1)^2+(y+2)^2+(z+3)^2=16$. Tọa độ tâm $I$ của $(S)$ là
	\choice
	{$I(-1;-2;-3)$}
	{$I(-1;2;3)$}
	{\True $I(1;-2;-3)$}
	{$I(1;-2;3)$}
	\loigiai{
		Tọa độ tâm $I$ là $I(1;-2;-3)$.
	}
\end{ex}
\begin{ex}%[Nguyễn Tiến, PT ĐMH-2023]%[2H3Y1-3]
	Trong không gian $Oxyz$, cho mặt cầu $(S)\colon x^2+y^2+z^2-2x-4y+6z+10=0$ có bán kính $R$ bằng
	\choice
	{$R=1$}
	{\True $R=2$}
	{$R=3$}
	{$R=4$}
	\loigiai{
		Ta có $a=1$, $b=2$, $c=-3$, $d=10$.\\
		Suy ra $R=\sqrt{a^2+b^2+c^2-d}=\sqrt{1^2+2^2+(-3)^2-10}=2$.
	}
\end{ex}
\Closesolutionfile{ans}
%======================
\subsection{Bảng đáp án}
\inputansbox{8}{ans/ANS-DANG-10}

