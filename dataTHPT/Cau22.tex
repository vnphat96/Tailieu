%Câu 22
\setcounter{section}{21}
\setcounter{ex}{0}
\section{Phép đếm - Hoán vị - Chỉnh hợp - Tổ hợp}
\subsection{Kiến thức cần nhớ}
\begin{khung}
\subsubsection{Quy tắc cộng}
Một công việc được hoàn thành bởi một trong hai hành động. Nếu hành động này có $m$ cách thực hiện, hành động kia có $n$ các thực hiện không trùng với bất kì cách nào của hành động thứ nhất thì công việc đó có $m+n$ cách thực hiện.
\subsubsection{Quy tắc nhân}
Một công việc được hoàn thành bởi hai hành động liên tiếp. Nếu có $m$ cách thực hiện hành động thứ nhất và ứng với mỗi cách đó có $n$ cách thực hiện cho hành động thứ hai thì có $m\cdot n$ cách hoàn thành công việc.
\subsubsection{Hoán vị}
\begin{itemize}
	\item Cho tập hợp $A$ gồm $n$ phần tử $(n\geq 1)$. Mỗi kết quả của sự sắp xếp thứ tự $n$ phần tử của tập hợp $A$ được gọi là một hoán vị của $n$ phần tử đó.
	\item Số các hoán vị của $n$ phần tử là $\mathrm{P}_n=n(n-1)\cdots 2\cdot 1=n!$.
\end{itemize}
\subsubsection{Chỉnh hợp}
\begin{itemize}
	\item Cho tập hợp $A$ gồm $n$ phần tử $(n\geq 1)$.
	Kết quả của việc lấy $k$ phần tử khác nhau từ $n$ phần tử của tập hợp $A$ và sắp xếp chúng theo một thứ tự nào đó được gọi là một chỉnh hợp chập $k$ của $n$ phần tử đã cho.
	\item Số chỉnh hợp chập $k$ của $n$ phần tử là $\mathrm{A}^k_n=\dfrac{n!}{(n-k)!}$ với $1\leq k\leq n$.
\end{itemize}
\subsubsection{Tổ hợp}
\begin{itemize}
	\item Cho tập hợp $A$ gồm $n$ phần tử $(n\geq 1)$. Mỗi tập con gồm $k$ phần tử của $A$ được gọi là một tổ hợp chập $k$ của $n$ phần tử đã cho.
	\item Số các tổ hợp chập $k$ của $n$ phần tử là $\mathrm{C}^k_n=\dfrac{n!}{k!(n-k)!}$ với $0\leq k\leq n$.
	\item Một số tính chất của các số $\mathrm{C}^k_n$:
	\begin{enumerate}[i)]
		\item $\mathrm{C}^k_n=\mathrm{C}^{n-k}_n$ với $0\leq k\leq n$.
		\item $\mathrm{C}^{k-1}_{n-1}+\mathrm{C}^k_{n-1}=\mathrm{C}^k_n$ với $1\leq k< n$.
		\end{enumerate}
\end{itemize}
\end{khung}
\subsection{Bài tập mẫu}
\Opensolutionfile{ans}[ans/ANS-CAU-22]
\begin{khung}
	\begin{vd}[Đề minh họa BGD 2022-2023]%[1D2Y2-1]
		Cho tập hợp $A$ có $15$ phần tử. Số tập con gồm hai phần tử của $A$ bằng
		\choice
		{$225$}	
		{$30$}
		{$210$}
		{\True $105$}
		\loigiai{
			Số tập con gồm hai phần tử của $A$ có $15$ phần tử là số tổ hợp chập $2$ của $15$ phần tử.\\
			Do đó, số tập con gồm hai phần tử của $A$ là  $\mathrm{C}^2_{15}=105$.
		}
	\end{vd}
\end{khung}
\subsection{Bài tập tương tự và phát triển}
%câu 1
\begin{ex} %[Võ Lê Hồng Sơn] %[1D2Y2-1]
	Với $k$, $n$ là hai số nguyên dương tùy ý thỏa mãn $k\leq n$, mệnh đề nào dưới đây \textbf{sai}?
	\choice
	{$\mathrm{C}^k_n+\mathrm{C}^{k+1}_n=\mathrm{C}^{k+1}_{n+1}$}
	{\True $\mathrm{C}^k_n=\dfrac{\mathrm{P}_n}{k!}$}
	{$\mathrm{C}^k_n=\dfrac{\mathrm{A}^k_n}{k!}$}
	{$\mathrm{C}^k_n=\mathrm{C}^{n-k}_n$}
	\loigiai{$\mathrm{C}^k_n=\dfrac{n!}{k!\cdot (n-k)!}=\dfrac{1}{k!}\cdot \mathrm{A}^k_n\Rightarrow \mathrm{A}^k_n=k!\cdot \mathrm{C}^k_n$.
	}
\end{ex}
%câu 2
\begin{ex}%[Võ Lê Hồng Sơn] %[1D2Y2-1]
	Cho $n$ là số tự nhiên lớn hơn $2$. Số các chỉnh hợp chập $2$ của $n$ phần tử là
	\choice
	{\True $n(n-1)$}
	{$2n$}
	{$\dfrac{n(n-1)}{2!}$}
	{$2!\cdot n(n-1)$}
	\loigiai{Ta có $\mathrm{A}^2_n=\dfrac{n!}{(n-2)!}=n(n-1)$.
	}
\end{ex}
%câu 3
\begin{ex}%[Võ Lê Hồng Sơn] %[1D2Y2-1]
	Cho tập $A=\left\{1;2;3;4;5;6\right\}$, có bao nhiêu tập con gồm $3$ phần tử của tập hợp $A$?
	\choice
	{$\mathrm{A}^3_6$}
	{$\mathrm{P}_6$}
	{$\mathrm{P}_3$}
	{\True $\mathrm{C}^3_6$}
	\loigiai{Số tập con có $3$ phần tử của tập hợp $A$ gồm $6$ phần tử là số tổ hợp chập $3$ của $6$ phần tử. Do đó, số tập con cần tìm là $\mathrm{C}^3_6$.
	}
\end{ex}
%câu 4
\begin{ex}%[Võ Lê Hồng Sơn] %[1D2Y2-1]
	Cho $n,\,k\in \mathbb{N}^*$ và $n\geq k$. Tìm công thức đúng.
	\choice
	{$\mathrm{C}^k_n=\dfrac{n!}{(n-k)!}$}
	{$\mathrm{A}^k_n=\dfrac{n!}{(n-k)! k!}$}
	{\True $\mathrm{A}^k_n=\dfrac{n!}{(n-k)!}$}
	{$\mathrm{C}^k_n=\dfrac{n!}{(n-k)! (k+1)!}$}
	\loigiai{Công thức đúng là $\mathrm{A}^k_n=\dfrac{n!}{(n-k)!}$.
	}
\end{ex}
%câu 5
\begin{ex}%[Võ Lê Hồng Sơn] %[1D2Y2-1]
	Số tập con có hai phần tử của tập hợp gồm $10$ phần tử là
	\choice
	{$20$}
	{$90$}
	{$100$}
	{\True $45$}
	\loigiai{Số tập hợp con có hai phần tử của tập hợp gồm $10$ phần tử là $\mathrm{C}^2_{10}=45$.
	}
\end{ex}
%câu 6
\begin{ex}%[Võ Lê Hồng Sơn] %[1D2Y2-1]
Cho tập hợp $X$ gồm $10$ phần tử. Số các hoán vị của $10$ phần tử của tập hợp $X$ là 	
	\choice
	{$10^{10}$}
	{$10^2$}
	{$2^{10}$}
	{\True $10!$}
	\loigiai{Số các hoán vị của $10$ phần tử là $10!$.
	}
\end{ex}
%câu 7
\begin{ex}%[Võ Lê Hồng Sơn] %[1D2Y2-1]
	Cho tập $A=\left\{1;2;3;\ldots;9;10\right\}$. Một tổ hợp chập $2$ của $10$ phần tử của $A$ là
	\choice
	{\True $\left\{1;2\right\}$}
	{$2!$}
	{$\mathrm{A}^2_{10}$}
	{$\mathrm{C}^2_{10}$}
	\loigiai{Mỗi tập hợp con gồm $2$ phần tử của $A$ được gọi là một tổ hợp chập $2$ của $10$ phần tử của tập $A$. Mà $\left\{1;2\right\}$ là một tập con gồm $2$ phần tử của $A$ hay một tổ hợp chập $2$ của $10$ phần tử của $A$.
	}
\end{ex}
%câu 8
\begin{ex}%[Võ Lê Hồng Sơn] %[1D2Y2-1]
	Công thức nào dưới đây đúng?
	\choice
	{\True $\mathrm{A}^k_n=\dfrac{n!}{(n-k)!}$}
	{$\mathrm{A}^k_n=\dfrac{(n-k)!}{k!}$}
	{$\mathrm{A}^k_n=\dfrac{n!}{k!}$}
	{$\mathrm{A}^k_n=\dfrac{n!}{k!(n-k)!}$}
	\loigiai{Công thức tính số chỉnh hợp chập $k$ của $n$ là $\mathrm{A}^k_n=\dfrac{n!}{(n-k)!}$.
	}
\end{ex}
%câu 9
\begin{ex}%[Võ Lê Hồng Sơn] %[1D2Y2-1]
	Với $n$ là số nguyên dương, công thức nào dưới đây đúng?
	\choice
	{\True $\mathrm{P}_n=n!$}
	{$\mathrm{P}_n=n-1$}
	{$\mathrm{P}_n=(n-1)!$}
	{$\mathrm{P}_n=n$}
	\loigiai{Ta có $\mathrm{P}_n$ là kí hiệu số các hoán vị của $n$ phần tử. Do đó $\mathrm{P}_n=n!$.
}
\end{ex}
%câu 10
\begin{ex}%[Võ Lê Hồng Sơn] %[1D2Y1-1]
	Lớp 12A có $43$ học sinh và lớp 12B có $30$ học sinh. Chọn ngẫu nhiên $1$ học sinh từ lớp 12A và 12B. Hỏi có bao nhiêu cách chọn?
	\choice
	{$43$}
	{$30$}
	{$1290$}
	{\True $73$}
	\loigiai{Tổng số học sinh của hai lớp là $43+30=73$. Do đó số cách chọn là $73$.
	}
\end{ex}
%câu 11
\begin{ex}%[Võ Lê Hồng Sơn] %[1D2Y1-2]
	Một học sinh cần mua một cây bút mực và một cây bút chì. Các cây bút mực có $8$ màu khác nhau và các cây bút chì cũng có $8$ màu khác nhau. Như vậy, học sinh đó có bao nhiêu cách chọn?
	\choice
	{$16$}
	{$2$}
	{$3$}
	{\True $64$}
	\loigiai{Mua một cây bút mực có $8$ cách.\\
		Mua một cây bút chì có $8$ cách.\\
		Theo quy tắc nhân, để mua được $1$ cây bút mực và $1$ cây bút chì ta có $8\cdot 8=64$ cách.
	}
\end{ex}
%câu 12
\begin{ex}%[Võ Lê Hồng Sơn] %[1D2Y1-1]
	Một học sinh cần mua một cây bút để viết bài. Bút mực có $8$ loại khác nhau, bút chì có $8$ loại khác nhau. Như vậy, học sinh đó có bao nhiêu cách chọn?
	\choice
	{\True $16$}
{$2$}
{$3$}
{$64$}
	\loigiai{Công việc mua bút có $2$ phương án thực hiện độc lập nhau.\\
		Phương án $1$: Mua một cây bút mực có $8$ cách.\\
		Phương án $2$: Mua một cây bút chì có $8$ cách.\\
		Theo quy tắc cộng, ta có $8+8=16$ cách mua một cây bút chì để viết bài
	}
\end{ex}
%câu 13
\begin{ex}%[Võ Lê Hồng Sơn] %[1D2Y2-1]
	Có bao nhiêu cách chọn ra $3$ học sinh từ một nhóm có $7$ học sinh nam và $3$ học sinh nữ?
	\choice
	{$\mathrm{C}^3_3$}
	{\True $\mathrm{C}^3_{10}$}
	{$\mathrm{A}^3_{10}$}
	{$\mathrm{P}_3$}
	\loigiai{Số cách chọn ra $3$ học sinh từ $10$ học sinh là $\mathrm{C}^3_{10}$.
	}
\end{ex}
%câu 14
\begin{ex}%[Võ Lê Hồng Sơn] %[1D2Y1-2]
	Từ thành phố $A$ có 10 con đường đến thành phố $B$, từ thành phố $B$ có $7$ con đường đến thành phố $C$. Từ $A$ đến $C$ phải qua $B$, hỏi có bao nhiêu cách đi từ $A$ đến $C$ mà chỉ đi qua $B$ đúng một lần?
	\choice
	{$10$}
	{$7$}
	{$17$}
	{\True $70$}
	\loigiai{Để đi từ $A$ đến $C$ ta thực hiện hai hành động liên tiếp.\\
		Hành động $1$: Đi từ $A$ đến $B$ có $10$ cách.\\
		Hành động $2$: Đi từ $B$ đến $C$ có $7$ cách.\\
		Theo quy tắc nhân, đi từ $A$ đến $C$ có $10\cdot 7=70$ cách.
	}
\end{ex}
%câu 15
\begin{ex}%[Võ Lê Hồng Sơn] %[1D2Y1-2]
	Một người vào cửa hàng ăn, người đó chọn thực đơn gồm $1$ món ăn trong $5$ món, $1$ loại quả trong $5$ loại, $1$ loại nước uống trọng $3$ loại. Hỏi có bao nhiêu cách lập thực đơn?
	\choice
	{$73$}
	{\True $75$}
	{$85$}
	{$95$}
	\loigiai{Lập thực đơn gồm $3$ hành động liên tiếp.\\
		Hành động $1$: Chọn $1$ món ăn có $5$ cách.\\
		Hành động $2$: Chọn $1$ loại quả có $5$ cách.\\
		Hành động $3$: Chọn $1$ loại nước uống có $3$ cách.\\
		Theo quy tắc nhân, có $5\cdot 5\cdot 3=75$ cách lập thực đơn.
	}
\end{ex}
%câu 16
\begin{ex}%[Võ Lê Hồng Sơn] %[1D2Y2-1]
	Một tổ có $4$ học sinh nam và $6$ học sinh nữ. Hỏi có bao nhiêu cách chọn ra $3$ học sinh trong đó có $2$ học sinh nam?
	\choice
	{$\mathrm{C}^2_4+ \mathrm{C}^1_6$}
	{\True $\mathrm{C}^2_4\cdot \mathrm{C}^1_6$}
	{$\mathrm{A}^2_4\cdot \mathrm{A}^1_6$}
	{$\mathrm{A}^2_4+ \mathrm{A}^1_6$}
	\loigiai{Để chọn $3$ học sinh trong đó có $2$ học sinh nam thì ta thực hiện $2$ hành động liên tiếp.\\
		Hành động $1$: Chọn $2$ học sinh nam có $\mathrm{C}^2_4$ cách.\\
		Hành động $2$: Chọn $1$ học sinh nữ có $\mathrm{C}^1_6$ cách.\\
		Theo quy tắc nhân, ta có $\mathrm{C}^2_4\cdot \mathrm{C}^1_6$ cách chọn thỏa mãn yêu cầu.
	}
\end{ex}
%câu 17
\begin{ex}%[Võ Lê Hồng Sơn] %[1D2Y2-1]
	Trong một trận chung kết bóng đá phải phân định thắng thua bằng đá luân lưu $11$ mét. Huấn luyện viên của mỗi đội cần trình với trọng tài một danh sách sắp thứ tự $5$ cầu thủ trong $11$ cầu thủ để đá luân lưu $5$ quả sút luân lưu. Hỏi huấn luyện viên của mỗi đội sẽ có bao nhiêu cách lập danh sách thứ tự đá luân lưu?
	\choice
	{$\mathrm{C}^5_{11}$}
	{\True $\mathrm{A}^5_{11}$}
	{$5!$}
	{$11!$}
	\loigiai{Mỗi cách chọn và sắp thứ tự $5$ cầu thủ trong $11$ cầu thủ để đá luân lưu là một chỉnh hợp chập $5$ của $11$ phần tử. Vậy số cách lập danh sách thứ tự đá luân lưu là $\mathrm{A}^5_{11}$.
	}
\end{ex}
%câu 18
\begin{ex}%[Võ Lê Hồng Sơn] %[1D2Y2-1]
	Có $15$ học sinh giỏi gồm $6$ học sinh khối $12$, $5$ học sinh khối $11$ và $4$ học sinh khối $10$. Hỏi có bao nhiêu cách chọn ra $6$ học sinh sao cho mỗi khối có đúng $2$ học sinh? 
	\choice
	{\True $\mathrm{C}^2_6\cdot \mathrm{C}^2_5\cdot \mathrm{C}^2_4$}
	{$\mathrm{A}^2_6\cdot \mathrm{A}^2_5\cdot \mathrm{A}^2_4$}
	{$\mathrm{C}^2_6+ \mathrm{C}^2_5+ \mathrm{C}^2_4$}
	{$\mathrm{A}^2_6+ \mathrm{A}^2_5+ \mathrm{A}^2_4$}
	\loigiai{Chọn $2$ học sinh khối $12$ có $\mathrm{C}^2_6$ cách.\\
		Chọn $2$ học sinh khối $11$ có $\mathrm{C}^2_5$ cách.\\
		Chọn $2$ học sinh khối $10$ có $\mathrm{C}^2_4$ cách.\\
		Theo quy tắc nhân, có $\mathrm{C}^2_6\cdot \mathrm{C}^2_5\cdot \mathrm{C}^2_4$ cách chọn thỏa mãn yêu cầu.
	}
\end{ex}
%câu 19
\begin{ex}%[Võ Lê Hồng Sơn] %[1D2Y2-1]
	Một câu lạc bộ có $30$ thành viên. Có bao nhiêu cách chọn một ban quản lí gồm $1$ chủ tịch, $1$ phó chủ tịch và $1$ thư kí?
	\choice
	{\True $\mathrm{A}^3_{30}$}
	{$\mathrm{C}^3_{30}$}
	{$30!$}
	{$3!$}
	\loigiai{Mỗi cách chọn ra $3$ người và sắp xếp vào $3$ vị trí là một chỉnh hợp chập $3$ của $30$ thành viên. Vậy số cách chọn thỏa mãn yêu cầu là $\mathrm{A}^3_{30}$.
	}
\end{ex}
%câu 20
\begin{ex}%[Võ Lê Hồng Sơn] %[1D2Y2-1]
	Một hộp chứa $10$ quả cầu phân biệt. Số cách lấy ra cùng lúc $3$ quả cầu từ hộp đó là
	\choice
	{$720$}
	{$10^3$}
	{\True $120$}
	{$3^{10}$}
	\loigiai{Số cách chọn cùng lúc $3$ quả cầu từ một hộp chứa $10$ quả cầu phân biệt là $\mathrm{C}^3_{30}$.
	}
\end{ex}
%câu 21
\begin{ex}%[Võ Lê Hồng Sơn] %[1D2Y2-1]
	Giả sử ta dùng $6$ màu để tô cho $4$ nước khác nhau trên bản đồ và không có màu nào được dùng $2$ lần. Số cách để chọn ra những màu cần dùng và tô lên bản đồ là
	\choice
	{\True $\mathrm{A}^4_6$}
	{$10$}
	{$\mathrm{C}^4_6$}
	{$6^4$}
	\loigiai{Số cách chọn $4$ màu từ $6$ màu và tô vào $4$ nước trên bản đồ (không có màu nào được dùng $2$ lần) là $\mathrm{A}^4_6$.
	}
\end{ex}
%câu 22
\begin{ex}%[Võ Lê Hồng Sơn] %[1D2Y2-1]
	Có bao nhiêu cách phân công $3$ bạn từ một tổ có $9$ bạn để làm trực nhật?
	\choice
	{$9^3$}
	{$3^9$}
	{$\mathrm{A}^3_9$}
	{\True $\mathrm{C}^3_9$}
	\loigiai{Mỗi cách phân công ba bạn từ một tổ có $9$ bạn để làm trực nhật là một tổ hợp chập $3$ của $9$. Nên số cách phân công là $\mathrm{C}^3_9$.
	}
\end{ex}
%câu 23
\begin{ex}%[Võ Lê Hồng Sơn] %[1D2Y2-3]
	Cho $20$ điểm phân biệt nằm trên một đường tròn. Hỏi có bao nhiêu tam giác được tạo thành từ các điểm này?
	\choice
	{$8000$}
	{\True $1140$}
	{$6480$}
	{$600$}
	\loigiai{Chọn $3$ điểm từ $20$ điểm ta được một tam giác nên số tam giác được tạo thành từ $20$ điểm đã cho là $\mathrm{C}^3_{20}$.
	}
\end{ex}
%câu 24
\begin{ex}%[Võ Lê Hồng Sơn] %[1D2Y2-1]
	Trong một bình đựng $4$ viên bi đỏ và $3$ viên bi xanh. Lấy ngẫu nhiên đồng thời $2$ viên. Có bao nhiêu cách lấy?
	\choice
	{$18$}
	{\True $21$}
	{$42$}
	{$10$}
	\loigiai{Số cách lấy $2$ viên bi từ $7$ viên bi là $\mathrm{C}^2_7=21$.
	}
\end{ex}
%câu 25
\begin{ex}%[Võ Lê Hồng Sơn] %[1D2Y2-1]
	Số cách sắp xếp $6$ học sinh nữ và $4$ học sinh nam thành một hàng dọc là
	\choice
	{$6!+4!$}
	{$6!\cdot 4!$}
	{$\mathrm{C}^4_{10}\cdot \mathrm{C}^6_{10}$}
	{\True $10!$}
	\loigiai{Số cách sắp xếp $6$ học sinh nữ và $4$ học sinh nam thành một hàng dọc là $10!$.
	}
\end{ex}
%câu 26
\begin{ex}%[Võ Lê Hồng Sơn] %[1D2Y2-1]
	Cho tập hợp $X=\left\{1;2;3;4;5;6;7\right\}$. Từ tập hợp $X$, hỏi có thể lập được bao nhiêu số tự nhiên gồm ba chữ số đôi một khác nhau?
	\choice
	{$35$}
	{\True $210$}
	{$840$}
	{$5040$}
	\loigiai{Số cách lập số tự nhiên gồm ba chữ số khác nhau từ tập hợp $X$ là $\mathrm{A}^3_7=210$.
	}
\end{ex}
%câu 27
\begin{ex}%[Võ Lê Hồng Sơn] %[1D2Y2-1]
	Có bao nhiêu số tự nhiên có hai chữ số khác nhau mà các chữ số được lấy từ tập hợp $X=\left\{1;2;3;4;5\right\}$?
	\choice
	{$5^2$}
	{$2^5$}
	{\True $\mathrm{A}^2_5$}
	{$\mathrm{C}^2_5$}
	\loigiai{Từ $5$ chữ số của tập $X$, ta lấy $2$ chữ số bất kì rồi sắp xếp vị trí được một số có hai chữ số. Như vậy có $\mathrm{A}^2_5$ số được tạo thành.
	}
\end{ex}
%câu 28
\begin{ex}%[Võ Lê Hồng Sơn] %[1D2Y2-1]
	Giả sử $9$ vận động viên tham gia một cuộc thi bơi lội. Nếu không kể trường hợp có hai vận động viên về đích cùng một lúc thì có bao nhiêu kết quả có thể xảy ra đối với các vị trí thứ nhất, thứ nhì và thứ ba?
	\choice
	{$84$}
	{$729$}
	{\True $504$}
	{$3^9$}
	\loigiai{Số kết quả có thể xảy ra đối với các vị trí thứ nhất, thứ nhì và thứ ba là $\mathrm{A}^3_9=504$ kết quả.
	}
\end{ex}
%câu 29
\begin{ex}%[Võ Lê Hồng Sơn] %[1D2Y2-1]
	Có bao nhiêu tập con gồm $3$ phần từ của tập hợp $X=\left\{1;2;3;4;7;8;9\right\}$?
	\choice
	{\True $\mathrm{C}^3_7$}
	{$\mathrm{A}^3_9$}
	{$\mathrm{A}^3_7$}
	{$\mathrm{C}^3_9$}
	\loigiai{Số tập con gồm $3$ phần tử của tập hợp $X=\left\{1;2;3;4;7;8;9\right\}$ là số tổ hợp chập $3$ của $7$ phần tử. Vậy có $\mathrm{C}^3_7$ tập hợp.
	}
\end{ex}
%câu 30
\begin{ex}%[Võ Lê Hồng Sơn] %[1D2Y2-1]
Một hộp có $8$ bi xanh, $5$ bi đỏ và $4$ bi vàng. Có bao nhiêu cách chọn ra $3$ viên bi sao cho có đúng $1$ bi đỏ?
	\choice
	{$\mathrm{C}^1_5\cdot \mathrm{C}^1_8\cdot \mathrm{C}^1_4$}
	{$\mathrm{A}^1_5\cdot \mathrm{A}^2_{12}$}
	{\True $\mathrm{C}^1_5\cdot \mathrm{C}^2_{12}$}
	{$\mathrm{A}^1_5\cdot \mathrm{A}^1_8\cdot \mathrm{A}^1_4$}
	\loigiai{Chọn $1$ bi đỏ có $\mathrm{C}^1_5$ cách.\\
		Chọn $2$ bi còn lại có $\mathrm{C}^2_{12}$ cách.\\
		Theo quy tắc nhân, ta có $\mathrm{C}^1_5\cdot \mathrm{C}^2_{12}$ cách chọn thỏa mãn yêu cầu.
	}
\end{ex}
\Closesolutionfile{ans}
%======================
\subsection{Bảng đáp án}
\inputansbox{8}{ans/ANS-CAU-22}

