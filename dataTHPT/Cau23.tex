\section{Nguyên hàm}
\subsection{Kiến thức cần nhớ}
\begin{khung}
\subsubsection{Định nghĩa nguyên hàm}
Cho hàm số $f(x)$ xác định trên $\mathbb{K}$ . Hàm số $F(x)$ được gọi là nguyên hàm của hàm số $f(x)$ trên $\mathbf{K}$ nếu \quad\fbox{$F'(x)=f(x),~ \forall x \in \mathbb{R}.$} \\
Nếu $F(x)$ là một nguyên hàm của hàm số $f(x)$ trên $\mathbb{K}$ thì mọi nguyên hàm của $f(x)$ trên $K$ đều có dạng $F(x)+C$ với $C$ là hằng số.
\subsubsection{Tính chất của nguyên hàm}
$\bullet \displaystyle \int f'(x)\mathrm{\,d}x=f(x)+C$.\\
$\bullet \displaystyle\int k f(x)\mathrm{\,d}x=k\displaystyle\int f(x)\mathrm{\,d}x$ với $k\neq 0$.\\
$\bullet \displaystyle\int[f(x)\pm g(x)]\mathrm{\,d}x=\displaystyle\int f(x)\mathrm{\,d}x\pm\displaystyle\int g(x)\mathrm{\,d}x$.
\subsubsection{Bảng nguyên hàm của một số hàm số thường gặp}
\begin{longtable}{|p{6.5cm}|p{9.0cm}|}
	\hline
	\centerline{\textbf{Nguyên hàm cơ bản}}&\centerline{\textbf{Nguyên hàm mở rộng}}\\
	\hline
	\endhead
	$\bullet$	$\displaystyle\int 0\mathrm{\,d}x=C$ &\\
	$\bullet$	$\displaystyle\int  \mathrm{\,d}x = x + C$ &\\
	$\bullet$  $\displaystyle\int x^\alpha \mathrm{\,d}x = \dfrac{x^{\alpha + 1}}{\alpha + 1} +  C,\left(  \alpha \neq -1\right)$& $\bullet$ $\displaystyle\int \left(ax + b \right) ^\alpha \mathrm{\,d}x = \dfrac{1}{a}\dfrac{\left(ax + b \right) ^{\alpha + 1}}{\alpha + 1} +  C, ~ \left(  \alpha \neq -1\right)$\\		
	$\bullet$ $ \displaystyle\int \dfrac{1}{x} \mathrm{\,d}x = \ln\left| x\right| +  C$ &$\bullet$	$ \displaystyle\int \dfrac{1}{ax + b} \mathrm{\,d}x = \dfrac{1}{a} \cdot \ln\left| ax  +b\right| +  C$\\	
	$\bullet$  $ \displaystyle\int \alpha^x \mathrm{\,d}x = \dfrac{\alpha^x}{\ln \alpha}  + C,~ (0< \alpha \neq 1)$ & $\bullet$ $\displaystyle\int \alpha^{ax+b}=\dfrac{1}{a}\cdot \dfrac{\alpha^{ax+b}}{\ln \alpha}+C$\\	
	$\bullet$ $\displaystyle\int \mathrm{e}^x \mathrm{\,d}x = \mathrm{e}^x  + C$& $\bullet$ $\displaystyle\int \mathrm{e}^{\left(ax + b\right)}  \mathrm{\,d}x = \dfrac{1}{a} \cdot \mathrm{e}^{\left(ax + b \right)}  + C$\\
	$\bullet$  $\displaystyle\int \sin x \mathrm{\,d}x = -\cos x + C$ & $\bullet$ $\displaystyle\int \sin (ax + b) \mathrm{\,d}x = -\dfrac{1}{a}\cos (ax + b) +C, ~(a \neq 0)$\\
	$\bullet$  $\displaystyle\int \cos x \mathrm{\,d}x = \sin x + C$&$\bullet$  $\displaystyle\int \cos (ax + b) \mathrm{\,d}x = \dfrac{1}{a}\sin (ax + b) + C ,~ (a \neq 0)$\\
	$\bullet$ $\displaystyle\int \dfrac{1}{\cos^{2}x} \mathrm{\,d}x = \tan x + C$&  $\bullet$ $\displaystyle\int \dfrac{1}{\cos^{2}(ax + b)} \mathrm{\,d}x = \dfrac{1}{a}\tan (ax + b) + C,~ (a \neq 0)$\\
	$\bullet$  $\displaystyle\int \dfrac{1}{\sin^{2}x} \mathrm{\,d}x = -\cot x + C$&$\bullet$ $\displaystyle\int \dfrac{1}{\sin^{2}(ax + b)} \mathrm{\,d}x = -\dfrac{1}{a}\cot (ax + b) + C,~ (a \neq 0)$
	\\ 
 %	&$\bullet$ $\displaystyle\int \dfrac{1}{\left(ax + b \right) ^2} \mathrm{\,d}x = -\dfrac{1}{a}\cdot \dfrac{1}{ax +b} + C$ \\
	\hline
\end{longtable}
\end{khung}
\subsection{Bài tập mẫu}
\Opensolutionfile{ans}[ans/ANS-DANG-23]
\begin{khung}
	\begin{vd}%[2D3Y1-1]%[Hoàng Thanh Phương]%[Đề minh họa BGD 2022-2023]
			Cho $\displaystyle\displaystyle\int\dfrac{1}{x}\mathrm{\,d}x=F(x)+C$. Khẳng định nào dưới đây \textbf{đúng} ?
		\choice
		{$F'(x)=\dfrac{2}{x^2}$}
		{$F'(x)=\ln x$}
		{\True $F'(x)=\dfrac{1}{x}$}
		{$F'(x)=-\dfrac{1}{x^2}$}
		\loigiai{
			Ta có $F'(x)=\left(\displaystyle\displaystyle\int\dfrac{1}{x}\mathrm{\,d}x\right)'=\dfrac{1}{x}$.
		}
	\end{vd}
\end{khung}
\subsection{Bài tập tương tự và phát triển}
\begin{ex}%[2D3Y1-1]%[Hoàng Thanh Phương]%Câu 1
	Hàm số $ f(x)=\cos\left(4x+7\right)$ có một nguyên hàm là
	\choice
	{\True $\dfrac{1}{4}\sin\left(4x+7\right)-3$}
	{$-\dfrac{1}{4}\sin\left(4x+7\right)+3$}
	{$\sin\left(4x+7\right)-1$}
	{$-\sin\left(4x+7\right)+x$}
	\loigiai{
		Ta có  $\displaystyle\int \cos\left(4x+7\right)\mathrm{\,d}x=\dfrac{1}{4}\sin\left(4x+7\right)+C$ nên hàm số $ f(x)=\cos\left(4x+7\right)$ có một nguyên hàm là $\dfrac{1}{4}\sin\left(4x+7\right)-3$.}
\end{ex}

\begin{ex}%[2D3Y1-1]%[Hoàng Thanh Phương]%Câu 2
	Họ nguyên hàm của hàm số $ f(x)=\dfrac{1}{x^2}-x^2-\dfrac{1}{3}$ là:
	\choice
	{\True $\dfrac{-x^3}{3}-\dfrac{1}{x}-\dfrac{x}{3}+C$}
	{$\dfrac{-2}{x^2}-2x+C$}
	{$-\dfrac{x^4+x^2+3}{3x}+C$}
	{$\dfrac{-x^4+x^2+3}{3x}+C$}
	\loigiai{
		Ta có $\displaystyle\int \left(\dfrac{1}{x^2}-x^2-\dfrac{1}{3}\right)\mathrm{\,d}x=-\dfrac{1}{x}-\dfrac{x^3}{3}-\dfrac{x}{3}+C$.}
\end{ex}

\begin{ex}%[2D3Y1-1]%[Hoàng Thanh Phương]%Câu 3
	Tìm $\displaystyle\int \sin 5x\mathrm{\,d}x$
	\choice
	{\True $\displaystyle\int \sin 5x\mathrm{\,d}x=\dfrac{1}{5}\cos 5x+C$}
	{$\displaystyle\int \sin 5x\mathrm{\,d}x=-\dfrac{1}{5}\cos 5x+C$}
	{$\displaystyle\int \sin 5x\mathrm{\,d}x=-\cos 5x+C$}
	{$\displaystyle\int \sin 5x\mathrm{\,d}x=-5\cos 5x+C$}
	\loigiai{
		Ta có $\displaystyle\int \sin 5x\mathrm{\,d}x=-\dfrac{1}{5}\cos 5x+C.$}
\end{ex}

\begin{ex}%[2D3Y1-1]%[Hoàng Thanh Phương]%Câu 4
	Tìm họ nguyên hàm $F(x)$ của hàm số $ f(x)=\cos\left(2x+3\right)$.
	\choice
	{\True $F(x)=\dfrac{1}{2}\sin\left(2x+3\right)+C$}
	{$F(x)=-\dfrac{1}{2}\sin\left(2x+3\right)+C$}
	{$F(x)=\sin\left(2x+3\right)+C$}
	{$F(x)=-\sin\left(2x+3\right)+C$}
	\loigiai{
	$F(x)=\displaystyle\int{\cos\left(2x+3\right)dx}=\dfrac{1}{2}\sin\left(2x+3\right)+C$.}
\end{ex}

\begin{ex}%[2D3Y1-1]%[Hoàng Thanh Phương]%Câu 5
	Nguyên hàm của hàm số $f(x)=3^{2x+1}$ là:
	\choice
	{\True $\dfrac{1}{2\ln 3}{3^{2x+1}}+C$}
	{$\dfrac{1}{\ln 3}{3^{2x+1}}+C$}
	{$\dfrac{1}{2}{3^{2x+1}}+C$}
	{$\dfrac{1}{2}{3^{2x+1}}\ln 3+C$}
	\loigiai{
		Áp dụng công thức: $\displaystyle\int{a^{mx+n}\mathrm{\,d}x=\dfrac{a^{mx+n}}{m\ln a}}+C$.\\
		Ta có: $\displaystyle\int{3^{2x+1}\mathrm{\,d}x=\dfrac{3^{2x+1}}{2\ln 3}}+C$.}
\end{ex}

\begin{ex}%[2D3Y1-1]%[Hoàng Thanh Phương]%Câu 6
	Nguyên hàm của hàm số $ f(x)=3x^2+\mathrm{e}^x+1$ là
	\choice
	{\True $F(x)=x^3+\mathrm{e}^x+x+C$}
	{$F(x)=x^3+\mathrm{e}^x+1+C$}
	{$F(x)=2x^3+\mathrm{e}^x+x+C$}
	{$F(x)=6x+\mathrm{e}^x+C$}
	\loigiai{
		Ta có $\displaystyle\int f(x)\mathrm{\,d}x=\displaystyle\int \left(3x^2+\mathrm{e}^x+1\right)\mathrm{\,d}x=x^3+\mathrm{e}^x+x+C$.}
\end{ex}

\begin{ex}%[2D3Y1-1]%[Hoàng Thanh Phương]%Câu 7
	Công thức nguyên hàm nào sau đây không đúng?
	\choice
	{\True $\displaystyle\int\dfrac{1}{x}\mathrm{\,d}x=\ln x+C$}
	{$\displaystyle\int\dfrac{1}{\cos^2 x}\mathrm{\,d}x=\tan x+C$}
	{$\displaystyle\int x^\alpha\mathrm{\,d}x=\dfrac{x^{\alpha+1}}{\alpha+1}+C\,\,\left(\alpha\ne-1\right)$}
	{$\displaystyle\int a^x\mathrm{\,d}x=\dfrac{a^x}{\ln a}+C\,\left(0 < a\ne 1\right)$}
	\loigiai{
		Ta có: $\displaystyle\int \dfrac{1}{x}\mathrm{\,d}x=\ln\left| x\right|+C\Rightarrow\displaystyle\int \dfrac{1}{x}\mathrm{\,d}x=\ln x+C$ sai.}
\end{ex}

\begin{ex}%[2D3Y1-1]%[Hoàng Thanh Phương]%Câu 8
	Họ các nguyên hàm của hàm số $ y=\mathrm{e}^{-3x+1}$ là
	\choice
	{$-3\mathrm{e}^{-3x+1}+C$}
	{$\dfrac{1}{3}\mathrm{e}^{-3x+1}+C$}
	{\True $-\dfrac{1}{3}\mathrm{e}^{-3x+1}+C$}
	{$ 3\mathrm{e}^{-3x+1}+C$}
	\loigiai{
		Ta có $\displaystyle\int{\mathrm{e}^{-3x+1}\mathrm{\,d}x}=-\dfrac{1}{3}{\mathrm{e}^{-3x+1}}+C$.}
\end{ex}

\begin{ex}%[2D3Y1-1]%[Hoàng Thanh Phương]%Câu 9
	Họ tất cả nguyên hàm của hàm số $ f(x)=\cos x+6x$ là
	\choice
	{$-\sin x+C$}
	{$-\sin x+3x^2+C$}
	{\True $\sin x+3x^2+C$}
	{$\cos x+6x^2+C$}
	\loigiai{
		$\displaystyle\int{f(x)\mathrm{\,d}x}=\sin x+3x^2+C$.}
\end{ex}

\begin{ex}%[2D3Y1-1]%[Hoàng Thanh Phương]%Câu 10
	Họ nguyên hàm của hàm số $ f(x)=\sin 2x+\cos x$ là
	\choice
	{$\cos^2x-\sin x+C$}
	{\True $\sin^2x+\sin x+C$}
	{$\cos 2x-\sin x+C$}
	{$-\cos 2x+\sin x+C$}
	\loigiai{
		Ta có: 
		\begin{eqnarray*}
			\displaystyle\int\left(\sin 2x+\cos x\right)\mathrm{\,d}x&=&-\dfrac{1}{2}\cos 2x+\sin x+C' \\
			&=&-\dfrac{1}{2}\left(1-2\sin^2x\right)+\sin x+C' \\
			&=&\sin ^2x+\sin x+C \quad \left(C=C'-\dfrac{1}{2}\right).
		\end{eqnarray*}}
\end{ex}

\begin{ex}%[2D3Y1-1]%[Hoàng Thanh Phương]%Câu 11
	Họ nguyên hàm của hàm số $f(x)=\mathrm{e}^{-x}-1$ là
	\choice
	{$-\mathrm{e}^x-x+C$}
	{$\mathrm{e}^{-x}-x+C$}
	{$\mathrm{e}^x+x+C$}
	{\True $-\mathrm{e}^{-x}-x+C$}
	\loigiai{Ta có $\displaystyle\int f(x)\mathrm{\,d}x=-\mathrm{e}^{-x}-x+C$.
		}
\end{ex}

\begin{ex}%[2D3Y1-1]%[Hoàng Thanh Phương]%Câu 12
	Khẳng định nào đây đúng?
	\choice
	{$\displaystyle\int\sin x\mathrm{\,d}x=-\sin x+C$}
	{$\displaystyle\int\sin x\mathrm{\,d}x=\dfrac{1}{2}\sin^2x+C$}
	{$\displaystyle\int\sin x\mathrm{\,d}x=\cos x+C$}
	{\True $\displaystyle\int\sin x\mathrm{\,d}x=-\cos x+C$}
	\loigiai{
		Ta có $\displaystyle\int{\sin x\mathrm{\,d}x}=-\cos x+C$.}
\end{ex}

\begin{ex}%[2D3Y1-1]%[Hoàng Thanh Phương]%Câu 13
	Họ tất cả các nguyên hàm của hàm số $ f(x)=2^x+4x$ là
	\choice
	{$2^x\ln 2+2x^2+C$}
	{\True $\dfrac{2^x}{\ln 2}+2x^2+C$}
	{$2^x\ln 2+C$}
	{$\dfrac{2^x}{\ln 2}+C$}
	\loigiai{
	Ta có: $\displaystyle\int f(x)\mathrm{\,d}x=\displaystyle\int\left(2^x+4x\right)\mathrm{\,d}x=\displaystyle\int2^x\mathrm{\,d}x+\displaystyle\int 4x\mathrm{\,d}x=\dfrac{2^x}{\ln 2}+2x^2+C$.}
\end{ex}

\begin{ex}%[2D3Y1-1]%[Hoàng Thanh Phương]%Câu 14
	Nguyên hàm của hàm số $ f(x)=2x^3-9$ là:
	\choice
	{$\dfrac{1}{4}{x^4}+C$}
	{$ 4x^3-9x+C$}
	{\True $\dfrac{1}{2}{x^4}-9x+C$}
	{$ 4x^4-9x+C$}
	\loigiai{
		Ta có $\displaystyle\int f(x)\mathrm{\,d}x=\displaystyle\int\left(2x^3-9\right)\mathrm{\,d}x=2\dfrac{x^4}{4}-9x+C=\dfrac{x^4}{2}-9x+C$.}
\end{ex}

\begin{ex}%[2D3Y1-1]%[Hoàng Thanh Phương]%Câu 15
	Mệnh đề nào sau đây đúng?
	\choice
	{$\displaystyle\int\sin x\mathrm{\,d}x=\cos x+C$}
	{\True $\displaystyle\int\cos x\mathrm{\,d}x=\sin x+C$}
	{$\displaystyle\int a^x\mathrm{\,d}x=a^x+C~\left(0< a\ne 1\right)$}
	{$\displaystyle\int\dfrac{1}{x}\mathrm{\,d}x=-\dfrac{1}{x^2}+C~\left(x\ne 0\right)$}
	\loigiai{
	Ta có $\displaystyle\int\cos x\mathrm{\,d}x=\sin x+C$.}
\end{ex}

\begin{ex}%[2D3Y1-1]%[Hoàng Thanh Phương]%Câu 16
	Khẳng định nào sau đây là \textbf{sai}?
	\choice
	{Mọi hàm số $ f(x)$ liên tục trên đoạn $\left[a\,;\,b\right]$ đều có nguyên hàm trên đoạn $\left[a;b\right]$}
	{$\displaystyle\int e^x\mathrm{\,d}x=e^x+C$ ($C$ là hằng số)}
	{$\displaystyle\int \dfrac{1}{x}\mathrm{\,d}x=\ln\left| x\right|+C$ ($ C$ là hằng số) với $ x\ne 0$}
	{\True $\displaystyle\int x^\alpha\mathrm{\,d}x=\dfrac{x^{\alpha+1}}{\alpha+1}+C$ ($ C$ là hằng số, $\alpha$ là hằng số)}
	\loigiai{
		$\displaystyle\int x^\alpha\mathrm{\,d}x=\dfrac{x^{\alpha+1}}{\alpha+1}+C$ ($C$ là hằng số, $\alpha $ là hằng số và $\alpha\ne-1$).}
\end{ex}

\begin{ex}%[2D3Y1-1]%[Hoàng Thanh Phương]%Câu 17
	Họ nguyên hàm của hàm số $ f(x)=\dfrac{1}{x^2}-x^2-\dfrac{1}{3}$ là
	\choice
	{$-\dfrac{x^4+x^2+3}{3x}+C$}
	{\True $\dfrac{-x^3}{3}-\dfrac{1}{x}-\dfrac{x}{3}+C$}
	{$\dfrac{-x^4+x^2+3}{3x}+C$}
	{$\dfrac{-2}{x^2}-2x+C$}
	\loigiai{
		Ta có $\displaystyle\int\left(\dfrac{1}{x^2}-x^2-\dfrac{1}{3}\right)\mathrm{\,d}x=\displaystyle\int\left(x^{-2}-x^2-\dfrac{1}{3}\right)\mathrm{\,d}x=-\dfrac{1}{x}-\dfrac{x^3}{3}-\dfrac{x}{3}+C$.}
\end{ex}

\begin{ex}%[2D3Y1-1]%[Hoàng Thanh Phương]%Câu 18
	Nếu hàm số $ y=\sin x$ là một nguyên hàm của hàm số $ y=f(x)$ thì
	\choice
	{$f(x)=-\sin x$}
	{$f(x)=-\cos x$}
	{$f(x)=\sin x$}
	{\True $f(x)=\cos x$}
	\loigiai{
		Ta có $ f(x)=\left(\sin x\right)^\prime=\cos x$.}
\end{ex}

\begin{ex}%[2D3Y1-1]%[Hoàng Thanh Phương]%Câu 19
	Nguyên hàm của hàm số $f(x)=\sqrt[3]{x}$ là
	\choice
	{\True $F(x)=\dfrac{3x\sqrt[3]{x}}{4}+C$}
	{$F(x)=\dfrac{4x}{3\sqrt[3]{x}}+C$}
	{$F(x)=\dfrac{4x}{3\sqrt[3]{x^2}}+C$}
	{$F(x)=\dfrac{3\sqrt[3]{x^2}}{4}+C$}
	\loigiai{
		$\displaystyle\int{\sqrt[3]{x}dx}=\displaystyle\int {x^{\tfrac{1}{3}}dx}=\dfrac{x^{\tfrac{4}{3}}}{\dfrac{4}{3}}+C=\dfrac{3}{4}\sqrt[3]{x^4}+C=\dfrac{3}{4}x\sqrt[3]{x}+C$.}
\end{ex}

\begin{ex}%[2D3Y1-1]%[Hoàng Thanh Phương]%Câu 20
	Họ nguyên hàm của hàm số $ f(x)=\dfrac{1}{x-1}$ là
	\choice
	{\True $\ln\left|x-1\right|+C$}
	{$-\dfrac{1}{\left(x-1\right)^2}+C$}
	{$ 2\ln\left|x-1\right|+C$}
	{$\ln\left(x-1\right)+C$}
\loigiai{
	Có $\displaystyle\int\dfrac{1}{x-1}\mathrm{\,d}x=\ln\left|x-1\right|+C$ .\\
	Vậy họ nguyên hàm của hàm số $f(x)=\dfrac{1}{x-1}$ là $\ln\left|x-1\right|+C$.}
\end{ex}
\Closesolutionfile{ans}
%======================
\subsection{Bảng đáp án}
\inputansbox{8}{ans/ANS-DANG-23}


