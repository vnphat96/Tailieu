\setcounter{section}{45}
\setcounter{ex}{0}
\section{Phương trình mặt phẳng và khoảng cách}
\subsection{Kiến thức cần nhớ}
\begin{khung}
	Cho mặt phẳng $(\alpha)\colon Ax+By+Cz+D=0$ và điểm $M\left(x_0;y_0;z_0\right)$. Khoảng cách từ điểm $M$ đến mặt phẳng $(\alpha)$ là $$\mathrm{d}(M,(\alpha))=\dfrac{\left|Ax_0+By_0+Cz_0+D\right|}{\sqrt{A^2+B^2+c^2}}.$$
\end{khung}
\subsection{Bài tập mẫu}
\begin{khung}
\begin{vd}%[ĐỀ MH BGD 2022-2023]%[2H3K3-1]
	Trong KG $Oxyz$, cho điểm $A(0;1;2)$ và đường thẳng $d\colon\dfrac{x-2}{2}=\dfrac{y-1}{2}=\dfrac{z-1}{-3}$. Gọi $(P)$ là mặt phẳng đi qua $A$ chứa $d$. Khoảng cách từ điểm $M(5;-1;3)$ đến $(P)$ bằng 
	\choice
	{$5$}
	{$\dfrac{1}{3}$}
	{\True $1$}
	{$\dfrac{11}{3}$}
	\loigiai{
		Chọn $B(2;1;1)\in(d)\Rightarrow \overrightarrow{AB}=(2;0;-1)$.\\
		Đường thẳng $d$ có VTCP là $\overrightarrow{u}_d=(2;2;-3)$.\\
		Vì $(P)$ đi qua $A$ và chứa $d$ nên $\overrightarrow{n}_{(P)}=\left[\overrightarrow{AB},\overrightarrow{u}_d\right]=(2;4;4)\Rightarrow$ chọn lại $\overrightarrow{n}_{(P)}=(1;2;2)$.\\
		Phương trình mặt phẳng $(P)$ là 
		\begin{eqnarray*}
			&&1(x-0)+2(y-1)+2(z-2)=0.\\
			&\Leftrightarrow&x+2y+2z-6=0.
		\end{eqnarray*}
		Khi đó $\mathrm{d}(M,(P))=\dfrac{\left|5+2\cdot(-1)+2\cdot3-6\right|}{\sqrt{1^2+2^2+2^2}}=1$.
	}
\end{vd}
\end{khung}
\subsection{Bài tập tương tự và phát triển}
%Cau 46.1
\Opensolutionfile{ans}[ans/ANS-CAU-46]
\begin{ex}%[Vô Văn Tự, PTĐMH-2023]%[2H3G3-7]
	Trong không gian $O x y z$, cho hai mặt cầu $(S)\colon x^2+y^2+(z-1)^2=25$ và $\left(S'\right)\colon (x-1)^2+(y-2)^2+(z-3)^2=1$. Mặt phẳng $(P)$ tiếp xúc $\left(S'\right)$ và cắt $(S)$ theo giao tuyến là một đường tròn có chu vi bằng $6 \pi$. Khoảng cách từ $O$ đến mặt phẳng $(P)$ bằng
	\choice
	{$\dfrac{17}{7}$}
	{$\dfrac{19}{2}$}
	{$\dfrac{8}{9}$}
	{\True $\dfrac{14}{3}$}
	\loigiai{
	\begin{center}
		\begin{tikzpicture}[scale=1, font=\footnotesize, line join=round, line cap=round, >=stealth]
			\path
			(0,0) coordinate (K)
			(2,0) coordinate (H)
			(0,1.5) coordinate (I')
			(2,4) coordinate (I)
			(-3,-1) coordinate (a)
			(3,-1) coordinate (b)
			(4.5,1) coordinate (c)
			($(a)+(c)-(b)$) coordinate (d)
			($(a)!0.5!(d)$) coordinate (e)
			(intersection of K--I' and c--d) coordinate (m)
			(intersection of H--I and c--d) coordinate (n)
			;
			\draw 
			(m)--(d)--(a)--(b)--(c)--(n)
			(H)--(I)--(I')--(K)--(I)
			;
			\draw[dashed] 
			(n)--(m)
			;
			\foreach \p/\g in {K/-90, I'/120, I/90, H/-90}
			\draw[fill=black] (\p) circle (1pt) node[shift=(\g:3mm)] {$\p$};
			\pic[draw,angle radius=9pt]{right angle=I--H--K};
			\pic[draw,angle radius=9pt]{right angle=I'--K--e};
			\pic["$P$",draw,angle radius=6mm]{angle=b--a--d};
		\end{tikzpicture}
	\end{center}
	\begin{itemize}
		\item Mặt cầu $(S)$ có tâm $I(0;0;1)$ và bán kính $R=5$.
		\item Mặt cầu $\left(S'\right)$ có tâm $I'(1;2;3)$ và bán kính $R'=1$.
		\item Gọi $H, K$ lần lượt là hình chiếu của $I$ và $I'$ lên mặt phẳng $(P)$.
		\item Mặt phẳng $(P)$ cắt $(S)$ theo giao tuyến là một đường tròn có chu vi bằng $6 \pi$.\\
		$\Rightarrow$ Bán kính đường tròn giao tuyến $r=3 \Rightarrow \mathrm{d}(I,(P))=I H=\sqrt{R^2-r^2}=4$.
		\item Mặt phẳng $(P)$ tiếp xúc $\left(S'\right) \Leftrightarrow \mathrm{d}\left(I',(P)\right)=I'K=R'=1$.
		\item Có $I'=3$. Ta thấy $I I'+I' K=3+1=4=I H$.
		\item Mà $I I'+I' K \geq I K \geq I H$, dấu \lq\lq=\rq\rq~ xảy ra $\Leftrightarrow\heva{&H \equiv K \\ &I' \in I H}$
		$\Rightarrow I I' \perp(P) \Rightarrow \overrightarrow{I I'}=(1;2;2)$ là một véc-tơ pháp tuyến của mặt phẳng $(P)$.\\
		$\Rightarrow$ Phương trình $(P)$ có dạng $x+2 y+2 z+m=0$.\\
		Ta có $\heva{&\mathrm{d}(I,(P))=4 \\ &\mathrm{d}\left(I',(P)\right)=1} \Leftrightarrow\heva{&\dfrac{|2+m|}{3}=4 \\ &\dfrac{|11+m|}{3}=1} \Leftrightarrow\heva{&\hoac{&m=10 \\ &m=-14}\\ &\hoac{&m=-8 \\ &m=-14}} \Leftrightarrow m=-14$.\\
		Vậy phương trình $(P)\colon x+2 y+2 z-14=0 \Rightarrow \mathrm{d}(O,(P))=\dfrac{14}{3}$.
	\end{itemize}        	
	}
\end{ex}
%Cau 46.2
\begin{ex}%[Vô Văn Tự, PTĐMH-2023]%[2H3G2-7]
	Trong không gian với hệ tọa độ $O x y z$ cho mặt cầu $(S)$ có phương trình \break $x^2+y^2+z^2-2(a+4 b) x+2(a-b+c) y+2(b-c) z+d=0$ có tâm $I$ nằm trên mặt phẳng $(\alpha)$ cố định. Biết rằng $4 a+b-2 c=4$. Tìm khoảng cách từ điểm $D(1;2 ;-2)$ đến mặt phẳng $(\alpha)$.
	\choice
	{$\dfrac{9}{\sqrt{15}}$}
	{$\dfrac{1}{\sqrt{314}}$}
	{$\dfrac{15}{\sqrt{23}}$}
	{\True $\dfrac{1}{\sqrt{915}}$}
	\loigiai{
		Mặt cầu $(S)$ có tâm $I(a+4 b ;-a+b-c ;-b+c)$.\\
		Giả sử mặt phẳng $(\alpha)$ có phương trình $A x+B y+C z+D=0$.\\
		Vì $I \in(\alpha)$ nên ta có
		$$
		\begin{aligned}
			& A(a+4 b)+B(-a+b-c)+C(-b+c)+D=0 \\
			& \Leftrightarrow(A-B) a+(4 A+B-C) b+(-B+C) c=-D.\quad(1)
		\end{aligned}
		$$
		Theo bài ra ta có $4 a+b-2 c=4$.\quad(2)\\
		Đồng nhất (1) và (2) ta có hệ phương trình
		$$
		\heva{&A-B=4\\&4A+B-C=1\\&-B+C=-2\\&D=-4} \Leftrightarrow \heva{&A=-\dfrac{1}{4} \\&B=-\dfrac{17}{4}\\&C=-\dfrac{25}{4}\\&D=-4.}
		$$
		Suy ra $(\alpha)$ có phương trình $x+17 y+25 z+16=0$.\\
		Vậy khảng cách từ điểm $D(1;2 ;-2)$ đến $(\alpha)$ bằng
		$$
		\mathrm{d}(D,(\alpha))=\dfrac{|1+17 \cdot 2+25 \cdot(-2)+16|}{\sqrt{1^2+17^2+25^2}}=\dfrac{1}{\sqrt{915}} .
		$$
	}
\end{ex}
%Câu 46.3.
\begin{ex}%[Vô Văn Tự, PTĐMH-2023]%[2H3G2-6]
	Cho mặt phẳng $(\alpha)$ đi qua hai điểm $M(4;0;0)$, $N(0;0;3)$ sao cho mặt phẳng $(\alpha)$ tạo với mặt phẳng $O y z$ một góc bằng $60^{\circ}$. Tính khoảng cách từ gốc tọa độ đến mặt phẳng $(\alpha)$.
	\choice
	{$\dfrac{2}{\sqrt{3}}$}
	{\True $2$}
	{$1$}
	{$\dfrac{3}{2}$}
	\loigiai{
		\begin{center}
			\begin{tikzpicture}[scale=1, font=\footnotesize, line join=round, line cap=round, >=stealth]
				\path
				(0,0) coordinate (O)
				(1.5,-2) coordinate (M)
				(5,0) coordinate (P)
				(0,3) coordinate (N)
				($(N)!.35!(P)$) coordinate (K)
				($(M)!.35!(K)$) coordinate (H)
				;
				\draw 
				(N)--(M)--(P)--(N)--(O)--(M)--(K)
				;
				\draw[dashed] 
				(P)--(O) (K)--(O)--(H)
				;
				\foreach \p/\g in {O/180, M/-120, N/90, P/0, K/60, H/-10}
				\draw[fill=black] (\p) circle (1pt) node[shift=(\g:3mm)] {$\p$};
				\pic[draw,angle radius=7pt]{right angle=O--K--N};
				\pic[draw,angle radius=7pt]{right angle=O--H--M};
			\end{tikzpicture}
		\end{center}
		Giả sử mặt phẳng $(\alpha)$ cắt trục $O y$ tại $P$ ($P$ không trùng với $\mathrm{O})$.\\
		Khi đó ta có $O.MNP$ là tam diện vuông tại $O$.\\
		Kẻ $O K \perp N P \Rightarrow M K \perp N P$.\\
		Ta có $\widehat{O K M}$ là góc nhọn trong tam giác vuông $OMK$ nên góc $\widehat{OKM}$ là góc giữa $(\alpha)$ và mặt phẳng $(O y z)$.\\
		Khi đó $\widehat{ O M H}=30^{\circ}$.\\
		Kẻ $O H \perp M K \Rightarrow O H \perp(M N P) \Rightarrow \mathrm{d}(O ,(\alpha))=O H$.\\
		Mà trong tam giác vuông $\mathrm{OMH}$ ta có $O H=O M \cdot \sin 30^{\circ}=2$.
	}
\end{ex}
%Câu 46.4.
\begin{ex}%[Vô Văn Tự, PTĐMH-2023]%[2H3G2-6]
	Trong không gian với hệ tọa độ $O x y z$, cho mặt phẳng $(P)\colon 2 x-2 y+z+3=0$ và điểm $A(1 ;-2;3)$. Gọi $M(a;b;c) \in(P)$ sao cho $A M=4$. Tính $a+b+c$.
	\choice
	{\True $\dfrac{2}{3}$}
	{$2$}
	{$\dfrac{8}{3}$}
	{$12$}
	\loigiai{
		Ta có $\mathrm{d}(A,(P))=\dfrac{|2+4+3+3|}{\sqrt{2^2+(-2)^2+1}}=4=A M \Rightarrow M$ là hình chiếu vuông góc của $A$ lên $(P)$.\\
		 $\Rightarrow M \in(P)$ và $\overrightarrow{A M}=(a-1;b+2;c-3)$ cùng phương với véc-tơ pháp tuyến của $(P)$ là $\vv{n}=(2 ;-2;1)$.\\
		$\Rightarrow$ Tọa độ của $M$ là nghiệm của hệ $$\heva{&2 a-2 b+c+3=0 \\ &\dfrac{a-1}{2}=\dfrac{b+2}{-2}=\dfrac{c-3}{1}} \Leftrightarrow\heva{&a=-\dfrac{5}{3} \\ &b=\dfrac{2}{3} \\& c=\dfrac{5}{3}.}$$ Vậy $a+b+c=\dfrac{2}{3}$.
	}
\end{ex}
%Câu 46.5.
\begin{ex}%[Vô Văn Tự, PTĐMH-2023]%[2H3G2-6]
	Trong không gian $O x y z$, cho điểm $M(1;2;3)$. Mặt phẳng $(P)\colon x+A y+B z+C=0$ chứa trục $O z$ và cách điểm $M$ một khoảng lớn nhất, khi đó tổng $A+B+C$ bằng
	\choice
	{$-3$}
	{$3$}
	{\True $2$}
	{$6$}
	\loigiai{
		Vì $(P)$ chứa trục $O z$ nên luôn có $\mathrm{d}(M,(P)) \leq \mathrm{d}(M,O z)$.\\
		Suy ra $\mathrm{d}(M,(P))$ đạt giá trị lớn nhất bằng $\mathrm{d}(M,O z)=M H$, với $H$ là hình chiếu của $M$ trên trục $O z$.\\
		Dễ có $H(0;0;3)$. Vậy $(P)$ đi qua $H(0;0;3)$, có véc tơ pháp tuyến $\overrightarrow{M H}=(-1 ;-2;0)$.\\
		$(P)\colon -x-2 y=0 \Leftrightarrow x+2 y=0 \Rightarrow A=2$; $B=C=0 \Rightarrow A+B+C=2$.
	}
\end{ex}
%Câu 46.6.
\begin{ex}%[Vô Văn Tự, PTĐMH-2023]%[2H3G2-6]
	Trong không gian với hệ tọa độ $O x y z$, cho mặt phẳng $(P)$ đi qua điểm $M(2;3;5)$ cắt các tia $O x$, $O y$, $O z$ lần lượt tại ba điểm $A$, $B$, $C$ sao cho $O A$, $O B$, $O C$ theo thứ tự lập thành cấp số nhân có công bội bằng $3$. Khoảng cách từ $O$ đến mặt phẳng $(P)$ bằng
	\choice
	{\True $\dfrac{32}{\sqrt{91}}$}
	{$\dfrac{18}{\sqrt{91}}$}
	{$\dfrac{16}{\sqrt{91}}$}
	{$\dfrac{24}{\sqrt{91}}$}
	\loigiai{
		Vì $(P)$ cắt các tia $O x$, $O y$, $O z$ lần lượt ờ $A$, $B$, $C$ nên ta gọi tọa độ các điểm là
		$A(a;0;0)$, $B(0;b;0)$, $C(0;0;c)$ với $a$, $b$, $c>0$.\\
		Khi đó phương trình mặt phẳng $(P)\colon \dfrac{x}{a}+\dfrac{y}{b}+\dfrac{z}{c}=1$.\\
		Vì $M(2;3;5) \in(P) \Rightarrow \dfrac{2}{a}+\dfrac{3}{b}+\dfrac{5}{c}=1$.\\
		Vì độ dài các đoạn $OA$, $OB$, $OC$ lập thành cấp số nhân với công bội bằng 3 $\Rightarrow\heva{&b=3 a \\ &c=3 b=9a.}$\\
		$\Rightarrow \dfrac{2}{a}+\dfrac{3}{3 a}+\dfrac{5}{9 a}=1 \Leftrightarrow a=\dfrac{32}{9} \Rightarrow\heva{&b=\dfrac{32}{3} \\& c=32.}$\\
		Khi đó ta có phương trình mặt phẳng $(P)\colon \dfrac{x}{\frac{32}{9}}+\dfrac{y}{\frac{32}{3}}+\dfrac{z}{32}=1
	\Leftrightarrow 9 x+3 y+z-32=0$.\\
		Do đó $\mathrm{d}(O ,(P))=\dfrac{|-32|}{\sqrt{9^2+3^2+1^2}}=\dfrac{32}{\sqrt{91}}$.\\
		\textbf{Bình luận:}\\
		\textit{Bài này có thể dùng cách khác như sau:}\\
		Khoảng cách từ $O$ đến $(A B C)\colon \dfrac{1}{h^2}=\dfrac{1}{a^2}+\dfrac{1}{(3 a)^2}+\dfrac{1}{(9 a)^2} \Rightarrow h=\dfrac{9 a}{\sqrt{91}}$.\\
		Mà $a=\dfrac{32}{9}$ (theo trên). Từ đó tìm được $h=\dfrac{32}{\sqrt{91}}$.
	}
\end{ex}
%Câu 46.7.
\begin{ex}%[Vô Văn Tự, PTĐMH-2023]%[2H3G2-6]
	Trong không gian với hệ tọa độ $O x y$, cho mặt phẳng $(P)\colon 2 y-z+3=0$ và điểm $A(2;0;0)$. Mặt phẳng $(\alpha)$ đi qua $A$, vuông góc với $(P)$, cách gốc tọa độ $O$ một khoảng bằng $\dfrac{4}{3}$ và cắt các tia $Oy$, $Oz$ lần lượt tại các điểm $B$, $C$ khác $O$. Thể tích khối tứ diện $OABC$ bẳng
	\choice
	{$8$}
	{$16$}
	{\True $\dfrac{8}{3}$}
	{$\dfrac{16}{3}$}
	\loigiai{
		Giả sử $B(0;b;0)$ và $C(0;0;c)$, với $b$, $c>0$.\\
		Khi đó phương trình mặt phẳng $(\alpha)$ là $\dfrac{x}{2}+\dfrac{y}{b}+\dfrac{z}{c}=1$.\\
		Vì $(\alpha) \perp(P)$ nên $\dfrac{2}{b}-\dfrac{1}{c}=0 \Leftrightarrow \dfrac{1}{c}=2 \cdot \dfrac{1}{b}$.\\
		Mặt khác $\begin{aligned}[t]\mathrm{d}(O,(\alpha))=\dfrac{4}{3} &\Leftrightarrow \dfrac{1}{\sqrt{\left(\dfrac{1}{2}\right)^2+\left(\dfrac{1}{b}\right)^2+\left(\dfrac{1}{c}\right)^2}}=\dfrac{4}{3} \\
			&\Leftrightarrow \dfrac{5}{b^2}=\dfrac{5}{16} \\
			&\Leftrightarrow b^2=16 \\
			&\Leftrightarrow b=4 \Rightarrow c=2.\end{aligned}$\\
		Vậy $V_{OABC}=\dfrac{1}{6} \cdot O A \cdot O B \cdot O C=\dfrac{8}{3}$.
	}
\end{ex}
%Câu 46.8.
\begin{ex}%[Vô Văn Tự, PTĐMH-2023]%[2H3G2-6]
	Trong KG $Oxyz$, cho hình lăng trụ $ABC.A'B' C'$ có đáy $ABC$ là tam giác vuông cân tại $A$ và có độ dài các cạnh $B C=4$ và thể tích khối lăng trụ bằng $8$. Biết phương trình mặt phẳng $(ABC)\colon x+y-z-2=0$. Hãy viết phương trình mặt phẳng $\left(A' B' C'\right)$ biết nó cắt trục $O x$ tại điểm có hoành độ dương.
	\choice
	{$x+y-z-2+2 \sqrt{3}=0$}
	{$2 x+2 y-z-4-\sqrt{3}=0$}
	{\True $x+y-z-2-2 \sqrt{3}=0$}
	{$x+y-z-4+\sqrt{3}=0$}
	\loigiai{
		Xét $\triangle ABC$ vuông cân tại $A$ có $B C=4 \Rightarrow A B=A C=2 \sqrt{2}$.\\ $\Rightarrow S_{\triangle A B C}=\dfrac{A B \cdot A C}{2}=\dfrac{2 \sqrt{2} \cdot 2 \sqrt{2}}{2}=4$.\\
		Ta có $V_{ABC.A'B'C'}=S_{\triangle ABC} \cdot \mathrm{d}\left((ABC) ,\left(A'B'C'\right)\right) \Rightarrow \mathrm{d}\left((ABC) ,\left(A'B'C'\right)\right)=\dfrac{V_{ABC \cdot B' C}}{S_{\triangle A B C}}=\dfrac{8}{4}=2$.\\
		Mặt phẳng $\left(A'B'C'\right) \parallel(ABC) \Rightarrow\left(A'B'C'\right)\colon x+y-z+m=0$ vì $\left(A' B' C'\right) \parallel (ABC)$ và cắt trục $Ox$ tại điểm có hoành độ dương nên ta có điều kiện $\heva{&m \neq-2 \\&m<0.}$\\
		Lấy một điểm $M$ bất kì nằm trên mặt phẳng đáy $(A B C)$, suy ra $M(0;0 ;-2)$.\\
		Vậy $\mathrm{d}\left((ABC),\left(A'B'C'\right)\right)=\mathrm{d}\left(M ,\left(A'B'C'\right)\right)=\dfrac{|2+m|}{\sqrt{1+1+1}}=2 \Leftrightarrow\hoac{&m=-2+2 \sqrt{3}~\text{(loại)} \\& m=-2-2 \sqrt{3}.~\text{(nhận)}}$\\
		Vậy $\left(A' B' C'\right)\colon x+y-z-2-2 \sqrt{3}=0$.
	}
\end{ex}
%Câu 46.9.
\begin{ex}%[Vô Văn Tự, PTĐMH-2023]%[2H3G2-6]
	Trong không gian với hệ tọa độ $O x y z$, mặt phẳng $(\alpha)$ đi qua điểm $M(1;2;1)$ và cắt các tia $O x, O y, O z$ lần lượt tại $A, B, C$ sao cho độ dài $O A, O B, O C$ theo thứ tự tạo thành một cấp số nhân có công bội bằng 2. Tính khoảng cách từ gốc tọa độ $O$ tới mặt phẳng $(\alpha)$.
	\choice
	{$\dfrac{\sqrt{21}}{21}$}
	{\True $\dfrac{3 \sqrt{21}}{7}$}
	{$9 \sqrt{21}$}
	{$\dfrac{4}{\sqrt{21}}$}
	\loigiai{
	\begin{center}
		\begin{tikzpicture}[scale=1, font=\footnotesize, line join=round, line cap=round, >=stealth]
			\path
			(0,0) coordinate (O)
			(-2,-2) coordinate (x)
			(4,-1) coordinate (y)
			(0,5) coordinate (z)
			($(O)!.3!(x)$) coordinate (A)
			($(O)!.7!(y)$) coordinate (B)
			($(O)!.8!(z)$) coordinate (C)
			(1,1) coordinate (M)
			;
			\draw 
			(A)--(B)--(C)--cycle (A)--(x) (B)--(y) (C)--(z)
			;
			\draw[dashed] 
			(B)--(O)--(A) (O)--(C)
			;
			\foreach \p/\g in {O/135, A/-80, B/-80, C/170, M/45}
			\draw[fill=black] (\p) circle (1pt) node[shift=(\g:3mm)] {$\p$};
			\foreach \p/\g in {x/180, y/80, z/0, M/45}
			\draw[fill=black] (\p) node[shift=(\g:3mm)] {$\p$};
		\end{tikzpicture}
	\end{center}
		Đặt $O A=a~(a>0)$. Khi đó $O B=2 a, O C=4 a$.\\
		Áp dụng phương trình mặt phẳng theo đoạn chắn ta có mặt phẳng $(\alpha)$ có phương trình $$\dfrac{x}{a}+\dfrac{y}{2 a}+\dfrac{z}{4 a}=1.$$
		Do $M(1;2;1) \in(\alpha)$ nên $\dfrac{1}{a}+\dfrac{2}{2 a}+\dfrac{1}{4 a}=1 \Leftrightarrow \dfrac{9}{4 a}=1 \Leftrightarrow a=\dfrac{9}{4}$ (thỏa mãn $a>0$ ).\\ Phương trình tổng quát của mặt phẳng $(\alpha)$ là $4 x+2 y+z-9=0$.\\
		Suy ra $\mathrm{d}(O ,(\alpha))=\dfrac{|4\cdot0+2\cdot0+0-9|}{\sqrt{4^2+2^2+1^2}}=\dfrac{3 \sqrt{21}}{7}$.
	}
\end{ex}
%Câu 46.10.
\begin{ex}%[Vô Văn Tự, PTĐMH-2023]%[2H3G3-7]
	Trong KG $Oxyz$, cho mặt cầu $(S)\colon x^2+(y-3)^2+(z-6)^2=45$ và $M(1;4;5)$. Ba đường thẳng thay đổi $d_1$, $d_2$, $d_3$ nhưng luôn đôi một vuông góc tại $O$ cắt mặt cầu tại điểm thứ hai lần lượt là $A$, $B$, $C$. Khoảng cách lớn nhất từ $M$ đến mặt phẳng $(A B C)$ là
	\choice
	{$\sqrt{5}$}
	{$4$}
	{\True $\sqrt{6}$}
	{$3$}
	\loigiai{
		Mặt cầu $(S)$ có tâm $I(0;3;6)$, bán kính $R=3 \sqrt{5}$.\\
		Tứ diện $O A B C$ vuông đỉnh $O$ và nội tiếp mặt cầu $(S)$ nên gọi $O'$, $A'$, $B'$, $C'$ lần lượt là các điểm đối xứng với $O$, $A$, $B$, $C$ qua tâm $I$.\\
		Suy ra $OBA'C.AC'O'B'$ là hình hộp chữ nhật nội tiếp mặt cầu $(S)$ và đường chéo $OO'$ của hình hộp cắt mặt chéo tam giác $ABC$ tại trọng tâm $H$ của tam giác $ABC$ và $\overrightarrow{OH}=\dfrac{1}{3} \overrightarrow{O O'}=\dfrac{2}{3} \overrightarrow{O I} \Rightarrow H(0;2;4)$.\\
		Mặt phẳng $(A B C)$ thay đổi, luôn đi qua $H(0;2;4)$ nên $\mathrm{d}(M,(A B C)) \leq M H=\sqrt{6}$.\\
		$\mathrm{d}(M,(A B C))=\sqrt{6}$ khi mặt phẳng $(A B C)$ vuông góc với $M H$.
	}
\end{ex}
%Câu 46.11.
\begin{ex}%[Vô Văn Tự, PTĐMH-2023]%[2H3G2-6]
	Trong không gian $O x y z$, cho ba điểm $A(a ; 0 ; 0)$, $B(0 ; b ; 0)$, $C(0 ; 0 ; c)$ với $a$, $b$, $c$ là các số thực dương thay đồi tùy ý sao cho $a^2+b^2+c^2=1$. Khoảng cách từ $O$ đến mặt phẳng $(A B C)$ lớn nhất là
	\choice
	{$1$}
	{\True $\dfrac{1}{3}$}
	{$3$}
	{$\dfrac{1}{\sqrt{3}}$}
	\loigiai{
	Do $a$, $b$, $c>0$ nên phương trình mặt phẳng $(A B C)\colon  \dfrac{x}{a}+\dfrac{y}{b}+\dfrac{z}{c}=1$.\\
	Do đó $\mathrm{d}(O,(A B C))=\dfrac{1}{\sqrt{\dfrac{1}{a^2}+\dfrac{1}{b^2}+\dfrac{1}{c^2}}}$.\\
	Theo BĐT Côsi ta có $\left(a^2+b^2+c^2\right)\left(\dfrac{1}{a^2}+\dfrac{1}{b^2}+\dfrac{1}{c^2}\right) \geq 9 \Rightarrow \dfrac{1}{a^2}+\dfrac{1}{b^2}+\dfrac{1}{c^2} \geq 9$.\\
	Do đó $\mathrm{d}(O,(A B C)) \leq \dfrac{1}{3}$. Dấu \lq\lq=\rq\rq~ xảy ra khi $a=b=c=\dfrac{1}{\sqrt{3}}$.\\
	\textbf{*Chú ý:} Đề bài không cần $a$, $b$, $c$ là các số thực dương mà có thể tùy ý thì Lời giải tương tự.
}
\end{ex}
%Câu 46.12.
\begin{ex}%[Vô Văn Tự, PTĐMH-2023]%[2H3G2-6]%[2H3G2-6]
	Trong không gian $O x y z$, cho các điểm $A(a;0;0)$, $B(0;b;0)$, $C(0;0;2)$, trong đó $a$, $b$ là các số hữu tỷ dương và mặt phẳng $(P)$ có phương trình $2 x-2 y+1=0$. Biết rằng mặt phẳng $(ABC)$ vuông góc với mặt phẳng $(P)$ và khoảng cách từ $O$ đến mặt phẳng $(ABC)$ bằng $\dfrac{2}{\sqrt{33}}$. Giá trị $a \cdot b$ bằng
	\choice
	{\True $\dfrac{1}{4}$}
	{$4$}
	{$1$}
	{$0$}
	\loigiai{
		Vì $A \in O x$, $B \in O y$, $C \in O z$, $a>0$, $b>0$ nên phương trình mặt phẳng $(A B C)$ có dạng $\dfrac{x}{a}+\dfrac{y}{b}+\dfrac{z}{2}=1$.\\
		$\mathrm{d}(O,(A B C))=\dfrac{2}{\sqrt{33}} \Leftrightarrow \dfrac{|-1|}{\sqrt{\left(\dfrac{1}{a}\right)^2+\left(\dfrac{1}{b}\right)^2+\left(\dfrac{1}{2}\right)^2}}=\dfrac{2}{\sqrt{33}} \Leftrightarrow \dfrac{1}{a^2}+\dfrac{1}{b^2}=8
		$.\\
		Véc-tơ pháp tuyến của mặt phẳng $(A B C)$ và mặt phẳng $(P)$ lần lượt là $\overrightarrow{n}_1=\left(\dfrac{1}{a};\dfrac{1}{b};\dfrac{1}{2}\right)$ và $\overrightarrow{n}_2=(2 ;-2;0)$.\\
		Vì $(A B C) \perp (P) \Rightarrow \overrightarrow{n_1} \cdot \overrightarrow{n}_2=0 \Leftrightarrow \dfrac{2}{a}-\dfrac{2}{b}=0$. (2)\\
		Từ (1), (2) và điều kiện $a>0, b>0 \Rightarrow\heva{&a=\dfrac{1}{2} \\& b=\dfrac{1}{2}.}$\\
		Vậy $a \cdot b=\dfrac{1}{4}$.
	}
\end{ex}
%Câu 46.13.
\begin{ex}%[Vô Văn Tự, PTĐMH-2023]%[2H3G2-6]
	Trong không gian với hệ tọa độ $O x y z$, cho mặt phẳng $(P)\colon 3 x+4 y+5 z+1=0$ và ba điểm $A(2;5 ;-3)$, $B(-2;1;1)$, $C(2;0;1)$. Tìm điểm $D(a;b;c)$, $(b>0)$ là điểm nằm trên $(P)$ sao cho có vô số mặt phẳng $(Q)$ đi qua hai điểm $C$, $D$ và thỏa mãn khoảng cách từ điểm $A$ đến mặt phẳng $(Q)$ gấp $3$ lần khoảng cách từ $B$ đến $(Q)$. Tính $T=abc$.
	\choice
	{$-16$}
	{$0$}
	{\True $16$}
	{$12$}
	\loigiai{
		\begin{itemize}
			\item Trường hợp 1: $A$, $B$ cùng phía với $(Q)$.\\
			Gọi $M(x;y;z)$ thỏa $\overrightarrow{A M}=3 \overrightarrow{B M}$.\\
		Suy ra $\heva{&x-2=3(x+2) \\&y-5=3(y-1) \\&z+3=3(z-1)} \Leftrightarrow\heva{&x=-4 \\&y=-1 \\&z=3} \Rightarrow M(-4 ;-1;3)$.\\
		Đường thẳng $M C$ qua $C(2;0;1)$ và có VTCP $\vv{u}=\overrightarrow{MC}=(6;1 ;-2)$.\\
		Phương trình $M C\colon\heva{&x=2+6 t \\&y=t \\&z=1-2 t.}$\\
		Gọi $D(2+6 t;t;1-2 t) \in M C$.\\
		$D \in(P) \Rightarrow 3(2+6 t)+4t+5(1-2 t)+1=0 \Leftrightarrow t=-1 \Rightarrow D(-4 ;-1;3)$ (loại).
		\item Trường hợp 2: $A$, $B$ khác phía với $(Q)$.\\
		Gọi $M(x;y;z)$ thỏa $\overrightarrow{A M}=-3 \overrightarrow{B M}$.\\
		Suy ra $\heva{&x-2=-3(x+2) \\&y-5=-3(y-1) \\&z+3=-3(z-1} \Leftrightarrow\heva{&x=-1 \\&y=2 \\&z=0} \Rightarrow M(-1;2;0)$.\\
		Đường thẳng $M C$ qua $C(2;0;1)$ và có VTCP $\vv{u}=\overrightarrow{M C}=(3 ;-2;1)$.\\
		Phương trình $M C\colon \heva{&x=2+3 t \\&y=-2 t \\&z=1+t.}$\\
		Gọi $D(2+3 t ;-2 t;1+t) \in M C$.\\
		$D \in(P) \Rightarrow 3(2+3 t)+4(-2t)+5(1+t)+1=0 \Leftrightarrow t=-2 \Rightarrow D(-4;4 ;-1)$ (thỏa).
	\end{itemize}
		Suy ra $\heva{&a=-4 \\&b=4 \\&c=-1} \Rightarrow a b c=16$.
	}
\end{ex}
%Câu 46.14.
\begin{ex}%[Vô Văn Tự, PTĐMH-2023]%[2H3G3-8]
	Cho mặt phẳng $(\alpha)$ đi qua hai điểm $M(4;0;0)$ và $N(0;0;3)$ sao cho mặt phẳng $(\alpha)$ tạo với mặt phẳng $(O y z)$ một góc bẳng $60^{\circ}$. Tính khoảng cách từ điểm gốc tọa độ đến mặt phẳng $(\alpha)$.
	\choice
	{$1$}
	{$\dfrac{3}{2}$}
	{$\dfrac{2}{\sqrt{3}}$}
	{\True $2$}
	\loigiai{
		Giả sử mặt phẳng $(\alpha)$ cắt $O y$ tại $P(0;b;0)$.\\
		Phương trình mặt phẳng $(\alpha)\colon \dfrac{x}{4}+\dfrac{y}{b}+\dfrac{z}{3}=1$.\\
		Mặt phẳng $(\alpha)$ có một véc-tơ pháp tuyến là $\overrightarrow{n}_1=\left(\dfrac{1}{4};\dfrac{1}{b};\dfrac{1}{3}\right)$.\\
		Mặt phẳng $(O y z)$ có một véc-tơ pháp tuyến là $\overrightarrow{n_2}=(1;0;0)$.\\
		Mặt phẳng $(\alpha)$ tạo với mặt phẳng $(O y z)$ một góc bằng $60^{\circ}$ nên ta có
		$$
		\left|\cos \left(\overrightarrow{n}_1, \overrightarrow{n}_2\right)\right|=\dfrac{1}{2} \text { hay } \dfrac{\dfrac{1}{4}}{\sqrt{\dfrac{1}{16}+\dfrac{1}{b^2}+\dfrac{1}{9}}}=\dfrac{1}{2} \Rightarrow \dfrac{|3 b|}{\sqrt{9 b^2+144+16 b^2}}=\dfrac{1}{2}.
		$$
		Suy ra $11 b^2=144 \Rightarrow b= \pm \dfrac{12}{\sqrt{11}}$.\\
		Phương trình mặt phẳng $(\alpha)$ $$ 3 \sqrt{11} x+11 y+4 \sqrt{11} z-12 \sqrt{11}=0\Leftrightarrow -3 \sqrt{11} x+11 y-4 \sqrt{11} z+12 \sqrt{11}=0.$$
		Khoảng cách từ điểm gốc tọa độ đến mặt phẳng $(\alpha)$ là
		$\mathrm{d}=\dfrac{12 \sqrt{11}}{\sqrt{99+121+176}}=2.$
	}
\end{ex}
%Câu 46.15.
\begin{ex}%[Vô Văn Tự, PTĐMH-2023]%[2H3G3-8]
	Trong không gian $O x y z$ cho mặt cầu $(S)\colon(x-1)^2+(y-2)^2+(z-3)^2=9$ và mặt phẳng $(P)\colon  2 x-2 y+z+3=0$. Gọi $M(a;b;c)$ là điểm trên mặt cầu sao cho khoảng cách từ $M$ đến $(P)$ lớn nhất. Khi đó
	\choice
	{\True $a+b+c=7$}
	{$a+b+c=5$}
	{$a+b+c=6$}
	{$a+b+c=8$}
	\loigiai{
	\begin{center}
		\begin{tikzpicture}[scale=0.7, font=\footnotesize, line join=round, line cap=round, >=stealth]
			\path 
			(0,2.12)coordinate(I)
			(0,0)coordinate(A)
			(0:2.12 and .3)coordinate(K)
			(-4.5,-1.5)coordinate(P)
			(3,-1.5)coordinate(Q)
			(-3.5,1)coordinate(S)
			(180:2.12 and .3)coordinate(N)
			(90:2.12 and 5.12)coordinate(M)
			;
			\draw (I)+(-45:3)arc(-45:225:3)arc(-180:0:2.12 and .3);
			\draw[densely dashed] (I)+(-135:3)arc(-135:-45:3)arc(0:180:2.12 and .3)
			(I)--(A)--(K)
			(-2.83,1)--(2.83,1)(N)--(A) (I)--(M);
			\draw (-2.83,1)--(S)--(P)--(Q)--(4,1)--(2.83,1);
			\foreach \p/\g in {I/0, M/90}\draw[fill=black](\p)circle(1pt)+(\g:.2)node{$\p$};
			\pic["$P$",draw,angle radius=6mm]{angle=Q--P--S};
		\end{tikzpicture}
	\end{center}
		Mặt $(S)$ cầu có tâm $I(1;2;3)$, $R=3$.\\
		$\mathrm{d}(I,(P))=\dfrac{|2\cdot1-2\cdot2+3+3|}{\sqrt{2^2+(-2)^2+1^2}}=\dfrac{4}{3}<R$ mặt phẳng cắt mặt cầu theo một đường tròn.\\
		Gọi $M(a;b;c)$ là điểm trên mặt cầu sao cho khoảng cách từ $M$ đến $(P)$ lớn nhất.\\
		Khi đó $M$ thuộc đường thẳng $\Delta$ vuông đi qua $M$ và vuông góc với $(P)\Rightarrow 
		\Delta\colon \heva{&x=1+2 t \\&y=2-2 t\\&z=3+t.}$\\
		Thay vào mặt cầu $(S)$ ta được $(2 t)^2+(-2 t)^2+(t)^2=9 \Rightarrow 9 t^2=9 \Rightarrow t= \pm 1$.\\
		Với $t=1 \Rightarrow M(3;0;4) \Rightarrow \mathrm{d}(M ,(P))=\dfrac{|2\cdot3-2\cdot0+4+3|}{\sqrt{2^2+(-2)^2+1^2}}=\dfrac{10}{3}$.\\
		Với $t=-1 \Rightarrow M(-1;4;2) \Rightarrow \mathrm{d}(M ,(P))=\dfrac{|2 \cdot(-1)-2 \cdot 4+2+3|}{\sqrt{2^2+(-2)^2+1^2}}=\dfrac{1}{3}$.\\
		Vậy $M(3;0;4) \Rightarrow a+b+c=7$.
	}
\end{ex}
%Câu 46.16.
\begin{ex}%[Vô Văn Tự, PTĐMH-2023]%[2H3G3-8]
	Trong không gian $O x y z$, cho điểm $A(10;2;1)$ và đường thẳng $d\colon \dfrac{x-1}{2}=\dfrac{y}{1}=\dfrac{z-1}{3}$. Gọi $(P)$ là mặt phẳng đi qua điểm $A$, song song với đường thẳng $d$ sao cho khoảng cách giữa $d$ và $(P)$ lớn nhất. Khoảng cách từ điểm $M(-1;2;3)$ đến mặt phẳng $(P)$ bằng
	\choice
	{$\dfrac{2 \sqrt{13}}{13}$}
	{$\dfrac{76 \sqrt{790}}{790}$}
	{\True $\dfrac{533}{\sqrt{2765}}$}
	{$\dfrac{97 \sqrt{3}}{15}$}
	\loigiai{
	\begin{center}
		\begin{tikzpicture}[scale=1, font=\footnotesize, line join=round, line cap=round, >=stealth]
			\path
			(0,0) coordinate (O)
			(1.5,2) coordinate (x)
			(5,-0.2) coordinate (y)
			(2.5,1) coordinate (A)
			(3.3,1.2) coordinate (H)
			(3,3) coordinate (K)
			(1.5,3) coordinate (m)
			(4.8,3) coordinate (d)
			;
			\draw 
			(x)--(O)--(y) (A)--(K)--(H) (m)--(d)
			;
			\foreach \p/\g in {A/-135, H/0, K/90}
			\draw[fill=black] (\p) circle (1pt) node[shift=(\g:3mm)] {$\p$};
			\foreach \p/\g in {d/90}
			\draw[fill=black] (\p) node[shift=(\g:3mm)] {$\p$};
			\pic[draw,angle radius=9pt]{right angle=A--H--K};
			\pic[draw,angle radius=9pt]{right angle=m--K--A};
			\pic["$P$",draw,angle radius=6mm]{angle=y--O--x};
		\end{tikzpicture}
	\end{center}
		Gọi $K$ là hình chiếu vuông góc của $A$ lên đường thẳng $d$ và $H$ là hình chiếu vuông góc của $K$ lên mặt phẳng $(P)$.\\
		Vì $K \in d$ nên ta đặt $K(1+2 t;t;1+3 t)$ $\Rightarrow \overrightarrow{A K}=(2 t-9;t-2;3 t)$.\\
		Véc-tơ chỉ phương của đường thẳng $d$ là $\overrightarrow{u}_d=(2;1;3)$.\\
		Vì $A K \perp d \Rightarrow \overrightarrow{A K} \cdot \overrightarrow{u}_d=0 \Leftrightarrow 2\cdot(2 t-9)+1\cdot(t-2)+3 \cdot 3 t=0 \Leftrightarrow 14 t-20=0 \Leftrightarrow t=\dfrac{10}{7}$.\\
		Vậy $\overrightarrow{A K}=\left(-\dfrac{43}{7} ;-\dfrac{4}{7};\dfrac{30}{7}\right)$.\\
		Khoảng cách giữa $d$ và $(P)$ là độ dài đoạn thẳng
		$H K$ mà $H K \leq A K$ nên khoảng cách giữa $d$ và $(P)$ lớn nhất bằng $A K$ khi $H$ 	trùng $A$.\\
		Khi đó $A K \perp(P)$ nên $(P)$ có véc-tơ pháp tuyến $\overrightarrow{A K}=\left(-\dfrac{43}{7} ;-\dfrac{4}{7};\dfrac{30}{7}\right)$,
		ta chọn véc-tơ pháp tuyến của $(P)$ là $\overrightarrow{n}=(43;4 ;-30)$.\\
		Phương trình mặt phẳng $(P)$ là $43(x-10)+4(y-2)-30(z-1)=0 \Leftrightarrow 43 x+4 y-30 z-408=0$.\\
		Vậy ta có $\mathrm{d}(M ,(P))=\dfrac{533}{\sqrt{2765}}$.
	}
\end{ex}
%Câu 46.17.
\begin{ex}%[Vô Văn Tự, PTĐMH-2023]%[2H3G3-8]
	Tìm $m$ để khoảng cách từ điểm $A\left(\dfrac{1}{2};1;4\right)$ đến đường thẳng \break $(d)\colon \heva{&x=1-2 m+m t \\&y=-2+2 m+(1-m) t \\&z=1+t}$ đạt giá trị lớn nhất.
	\choice
	{$m=\dfrac{1}{3}$}
	{$m=16$}
	{$m=\dfrac{2}{3}$}
	{\True $m=\dfrac{4}{3}$}
	\loigiai{
		Với $t=2 \Rightarrow\heva{&x=1 \\&y=0 \\&z=3} \Rightarrow B(1;0;3)$ là điểm cố định mà đường thẳng $d$ luôn đi qua.\\
		Gọi $H$ là hình chiếu của $A$ trên $d$.\\
		Ta có $\mathrm{d}(A, d)=A H \leq A B \Rightarrow \mathrm{d}(A, d)$ đạt GTLN bằng $A B$ khi $A B \perp d \Leftrightarrow \overrightarrow{A B} \cdot \overrightarrow{u_d}=0$.\quad (1)\\
		Trong đó $\overrightarrow{A B}=\left(\dfrac{1}{2} ;-1 ;-1\right)$, $\overrightarrow{u}_d=(m;1-m;1)$ là véc-tơ chỉ phương của đường thẳng $d$,\\
		Do đó $(1) \Leftrightarrow \dfrac{1}{2} \cdot m-1 \cdot(1-m)-1=0 \Leftrightarrow m=\dfrac{4}{3}$.
	}
\end{ex}
%Câu 46.18.
\begin{ex}%[Vô Văn Tự, PTĐMH-2023]%[2H3G3-8]
	Trong KG $Oxyz$, cho ba điểm $A(-1;0;1)$, $B(3;2;1)$, $C(5;3;7)$. Gọi $M(a;b;c)$ là điểm thỏa mãn $M A=M B$ và $M B+M C$ đạt giá trị nhỏ nhất. Tính $P=a+b+c$
	\choice
	{$P=0$}
	{$P=2$}
	{\True $P=5$}
	{$P=4$}
	\loigiai{
		Gọi $I$ là trung điểm của $A B$, suy ra $I(1;1;1)$; $\overrightarrow{A B}=(4;2;0)$.\\
		Phương trình mặt phẳng trung trực của $A B$ là $(\alpha)\colon 2 x+y-3=0$.\\
		Vì $(2\cdot3+1\cdot2-3) \cdot(2\cdot5+1\cdot3-3)=50>0$ nên $B$, $C$ nằm về một phía so với $(\alpha)$, suy ra $A$, $C$ nằm về hai phía so với $(\alpha)$.\\
		Điểm $M$ thỏa mãn $M A=M B$ khi $M \in(\alpha)$. Khi đó $M B+M C=M A+M C \geq A C$.\\
		$M B+M C$ nhỏ nhất bằng $A C$ khi $M=A C \cap(\alpha)$.\\
		PTĐT $A C\colon \heva{&x=-1+2 t \\&y=t \\&z=1+2 t.}$\\
		Do đó tọa độ điểm $M$ là nghiệm của hệ phương trình
		$$
		\heva{&x=-1+2t\\&y=t\\&z=1+2t\\&2x+y-3=0} \Leftrightarrow \heva{
			&t=1 \\	&x=1 \\	&y=1 \\	&z=3.}$$
		Do đó $M(1;1;3)$, $a+b+c=5$.
	}
\end{ex}
%Câu 46.19.
\begin{ex}%[Vô Văn Tự, PTĐMH-2023]%[2H3G3-8]
	Trong KG $Oxyz$, cho điểm $A(1;2 ;-3)$ và mặt phẳng $(P)\colon 2 x+2 y-z+9=0$. Đường thẳng $d$ đi qua $A$ và vuông góc với mặt phẳng $(Q)\colon 3 x+4 y-4 z+5=0$, cắt mặt phẳng $(P)$ tại $B$. Điểm $M$ nằm trong mặt phẳng $(P)$ sao cho $M$ luôn nhìn $A B$ dưới góc vuông. Tính độ dài lớn nhất của $M B$.
	\choice
	{$M B=\dfrac{\sqrt{5}}{2}$}
	{\True $M B=\sqrt{5}$}
	{$M B=\sqrt{41}$}
	{$M B=\dfrac{\sqrt{41}}{2}$}
	\loigiai{
		Đường thẳng qua $A$ và vuông góc với mặt phẳng $(Q)$ có phương trình là $\heva{&x=1+3 t \\&y=2+4 t \\&z=-3-4 t.}$\\
		Điểm $B$ là giao của đường thẳng đó và $(P)$ nên $B(1+3 t;2+4 t ;-3-4 t)$ thỏa $$2(1+3 t)+2(2+4 t)+3+4 t+9=0 \Leftrightarrow t=-1 \Rightarrow B(-2 ;-2;1) \Rightarrow A B=\sqrt{41}.$$
		Tam giác $A B M$ vuông tại $M$ nên $B M^2=A B^2-A M^2$ và $A B$ cố định nên $B M$ lớn nhất khi $A M$ bé nhất, suy ra $M$ là hình chiếu vuông của $A$ lên $(P)$.\\
		$ \Rightarrow \min A M=\mathrm{d}(A ,(P))=\dfrac{|2\cdot1+2\cdot 2+3+9|}{\sqrt{2^2+2^2+(-1)^2}}=6$. \\
		Lúc đó $\max B M=\sqrt{A B^2-d^2(A ,(P))}=\sqrt{41-36}=\sqrt{5}$.
	}
\end{ex}
%Câu 46.20.
\begin{ex}%[Vô Văn Tự, PTĐMH-2023]%[2H3G3-8]
	Trong không gian với hệ tọa độ $O x y z$, cho mặt cầu $(S)\colon (x-1)^2+(y-2)^2+(z-3)^2=4$. Xét đường thẳng $d\colon \heva{&x=1+t \\ &y=-m t \\&z=(m-1)t}$ với $m$ là tham số thực. Giả sử $(P)$ và $\left(P'\right)$ là hai mặt phẳng chứa $d$, tiếp xúc với $(S)$ lần lượt tại $T$ và $T'$. Khi $m$ thay đổi, tính giá trị nhỏ nhất của độ dài đoạn thẳng $T T'$
	\choice
	{$\dfrac{4 \sqrt{13}}{5}$}
	{\True $2 \sqrt{2}$}
	{$2$}
	{$\dfrac{2 \sqrt{11}}{3}$}
	\loigiai{
		Mặt cầu $(S)$ có tâm $I(1;2;3)$, bán kính $R=2$. Mặt phẳng $\left(I T T'\right)$ cắt $d$ tại điểm $M$. Gọi $H$ là giao điểm của $T T'$ và $M I$.\\
		Vì $\triangle T I H \backsim \triangle M T T$ nên ta có
		$$
		\dfrac{T H}{T M}=\dfrac{T I}{M I} \Rightarrow T H=\dfrac{T M \cdot T I}{M I}=\dfrac{R \sqrt{M I^2-R^2}}{M I}=R \sqrt{1-\dfrac{R^2}{M^2}}
		.$$
		Do $T T'=2 T H$ nên $T T_{\min }' \Leftrightarrow T H_{\min } \Leftrightarrow M M_{\min }$.\\
		Nhận xét rằng với $\heva{&x=1+t \\&y=-m t \\&z=(m-1) t}$ ta có $x+y+z=1+t-m t+(m-1) t=1$
		nên khi $m$ thay đổi ta luôn có $(d) \subset(P)\colon x+y+z-1=0$ cố định. Vì thế
		$$
		M I_{\min }=\mathrm{d}(I,(P))=\dfrac{|1+2+3-1|}{\sqrt{1^2+1^2+1^2}}=\dfrac{5}{\sqrt{3}}.
		$$
		Từ đó ta có $T T_{\min }'=2 T H_{\min }=2 R \sqrt{1-\dfrac{R^2}{M I_{\min }^2}}=2.2 \sqrt{1-\dfrac{2^2}{\left(\dfrac{5}{\sqrt{3}}\right)^2}}=\dfrac{4 \sqrt{13}}{5}$.\\
		Ta kiểm tra điều kiện đủ của bài toán, tức là chứng minh rằng hình chiếu vuông góc của $I$ lên $(P)$ thuộc vào đường thẳng $d$.\\
		Gọi $d'$ là đường thẳng qua $I$ và vuông góc với $(P)$ ta có  $d'\colon \heva{&x=1+t \\&y=2+t \\&z=3+t.}$\\
		Gọi $M$ là hình chiếu vuông góc của $I$ lên $(P)$ ta có $M=d' \cap(P)$ suy ra
		$$(1+t)+(2+t)+(3+t)-1=0 \Leftrightarrow t=\dfrac{-5}{3} \Rightarrow M\left(\dfrac{-2}{3};\dfrac{1}{3};\dfrac{4}{3}\right).$$
		Hệ phương trình $\heva{&\dfrac{-2}{3}=1+t \\&\dfrac{1}{3}=-m t \\&\dfrac{4}{3}=(m-1) t} \Leftrightarrow \heva{&t=\dfrac{-5}{3} \\& m=\dfrac{1}{5}.}$\\
		Vậy với $m=\dfrac{1}{5}$ thì độ dài của $T T'$ nhỏ nhất.
	}
\end{ex}
\Closesolutionfile{ans}
%======================
\subsection{Bảng đáp án}
\inputansbox{8}{ans/ANS-CAU-46}


