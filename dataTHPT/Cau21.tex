%Dạng 21
%\setcounter{section}{20}
\setcounter{ex}{0}
\section{Phương trình và bất phương trình logarit}
\subsection{Kiến thức cần nhớ}
\begin{khung}
	\subsubsection{Phương trình logarit}
	\begin{itemize}
		\item $\log_a x=b$ $\Leftrightarrow$ $x=a^b$.
		\item $\log_a f(x)=\log_a g(x)$ $\Leftrightarrow$ $\heva{& g(x)>0\\ & f(x)=g(x).}$
	\end{itemize}
	\subsubsection{Bất phương trình logarit}
	\begin{enumerate}%[{\bfseries $\star$ TH 1:}]
		\item Nếu $a>1$ thì
		\begin{itemize}
			\item $\log_a x>b$ $\Leftrightarrow$ $x>a^b$.
			\item $\log_a x<b$ $\Leftrightarrow$ $0<x<a^b$.
			\item $\log_a f(x)>\log_a g(x)$ $\Leftrightarrow$ $f(x)>g(x)>0$.
		\end{itemize}
		\item Nếu $0<a<1$ thì
		\begin{itemize}
			\item $\log_a x>b$ $\Leftrightarrow$ $0<x<a^b$.
			\item $\log_a x<b$ $\Leftrightarrow$ $x>a^b$.
			\item $\log_a f(x)>\log_a g(x)$ $\Leftrightarrow$ $g(x)>f(x)>0$.
		\end{itemize}
	\end{enumerate}
\end{khung}
\subsection{Bài tập mẫu}
\setcounter{vd}{20}
\Opensolutionfile{ans}[ans/ANS-DANG-21]
\begin{khung}
	\begin{vd}[Đề minh họa BGD 2022-2023]%[2D2Y6-1]
		Tập nghiệm của bất phương trình $\log \left(x-2\right)>0$ là
		\choice
		{$\left(2; 3\right)$}
		{$\left(-\infty; 3\right)$}
		{\True $\left(3;+\infty\right)$}
		{$\left(12;+\infty\right)$}
		\loigiai{
			Điều kiện: $x>2$.\\
			Ta có $\log \left( x-2 \right)>0$ $\Leftrightarrow$ $x-2>10^0$ $\Leftrightarrow$ $x-2>1$ $\Leftrightarrow$ $x>3$.\\
			Vậy tập nghiệm của bất phương trình là $\left( 3;+\infty \right)$.
		}
	\end{vd}
\end{khung}
\subsection{Bài tập tương tự và phát triển}
\begin{ex}%[2D2Y5-1]
	Nghiệm của phương trình $\log_2\left(x-1\right)=4$ là
	\choice
	{$x=15$}
	{$x=9$}
	{\True $x=17$}
	{$x=2$}
	\loigiai{
		Ta có $\log_2\left(x-1\right)=4$ $\Leftrightarrow$ $x-1=2^4$ $\Leftrightarrow$ $x-1=16$ $\Leftrightarrow$ $x=17$.\\
		Vậy phương trình đã cho có nghiệm là $x=17$.
	}
\end{ex}
\begin{ex}%[2D2Y5-1]
	Nghiệm nhỏ nhất của phương trình $\log_5\left(x^2-3x+5\right)=1$ là
	\choice
	{$1$}
	{$3$}
	{\True $0$}
	{$-3$}
	\loigiai{
		Điều kiện: $x\in\mathbb R$ vì $x^2-3x+5>0$, $\forall x\in\mathbb{R}$.\\
		$\log_5\left(x^2-3x+5\right)=1$ $\Leftrightarrow$ $x^2-3x+5=5$ $\Leftrightarrow$ $x^2-3x=0$ $\Leftrightarrow$ $\hoac{
			& x=3\\ 
			& x=0.
		}$\\
		Vậy nghiệm nhỏ nhất của phương trình $\log_5\left(x^2-3x+5\right)=1$ là $0$.
	}
\end{ex}
\begin{ex}%[2D2Y5-1]
	Tìm nghiệm của phương trình $\log_{64}\left(x+1\right)=\dfrac{1}{2}$.
	\choice
	{\True $7$}
	{$-\dfrac{1}{2}$}
	{$-1$}
	{$4$}
	\loigiai{
		Điều kiện: $x>-1$.\\
		Ta có $\log_{64}\left(x+1\right)=\dfrac{1}{2}$ $\Leftrightarrow$ $x+1=8$ $\Leftrightarrow$ $x=7$ (thỏa điều kiện).
	}
\end{ex}
\begin{ex}%[2D2Y5-1]
	Tìm nghiệm của phương trình $\log_{25}\left(x+1\right)=\dfrac{1}{2}$.
	\choice
	{$x=\dfrac{23}{2}$}
	{$x=-6$}
	{$x=6$}
	{\True $x=4$}
	\loigiai{
		Điều kiện: $x>-1$.\\
		Phương trình $\log_{25}\left(x+1\right)=\dfrac{1}{2}$ $\Leftrightarrow$ $x+1=5$ $\Leftrightarrow$ $x=4$.
	}
\end{ex}
\begin{ex}%[2D2B5-1]
	Nghiệm của phương trình $\log_5\left(2x-1\right)^3=6$ là
	\choice
	{$10$}
	{$12$}
	{\True $13$}
	{$14$}
	\loigiai{
		Điều kiện: $2x-1>0$ $\Leftrightarrow$ $x>\dfrac{1}{2}$.\\
		$\log_5\left(2x-1\right)^3=6$ $\Leftrightarrow$ $3\log_5\left(2x-1\right)=6$ $\Leftrightarrow$ $2x-1=5^2$ $\Leftrightarrow$ $x=13$.
	}
\end{ex}
\begin{ex}%[2D2Y5-1]
	Tìm các nghiệm của phương trình $\log_3\left(2x-3\right)=2$.
	\choice
	{$x=\dfrac{9}{2}$}
	{\True $x=6$}
	{$x=5$}
	{$x=\dfrac{11}{2}$}
	\loigiai{
		$\log_3\left(2x-3\right)=2$ $\Leftrightarrow$ $2x-3=9$ $\Leftrightarrow$ $x=6$.
	}
\end{ex}
\begin{ex}%[2D2Y5-1]
	Số nghiệm của phương trình $\log_2\left(x^2-2x\right)=2$ là
	\choice
	{$1$}
	{\True $2$}
	{$4$}
	{$3$}
	\loigiai{
		$\log_2\left(x^2-2x\right)=2$ $\Leftrightarrow$ $x^2-2x=4$ $\Leftrightarrow$ $x^2-2x-4=0$ $\Leftrightarrow$ $\hoac{
			& x=1-\sqrt{5}\\ 
			& x=1+\sqrt{5}.
		}$
	}
\end{ex}
\begin{ex}%[2D2B6-1]
	Tập nghiệm của bất phương trình $\log_2\left(x^2+3x\right)\le 2$ là
	\choice
	{\True $\left[ -4;-3\right)\cup\left(0;1\right]$}
	{$\left(0;1\right]$}
	{$\left(-\infty ;-3\right)\cup\left(0;+\infty\right)$}
	{$\left(0;\dfrac{1}{2}\right]$}
	\loigiai{
		Ta có
		\begin{eqnarray*}\allowdisplaybreaks
			&&\log_2\left(x^2+3x\right)\le 2\\
			&\Leftrightarrow&0< x^2+3x\le 2^2\\
			&\Leftrightarrow&\heva{
				& x^2+3x>0\\ 
				& x^2+3x-4\le 0
			}\\
			&\Leftrightarrow&\heva{
				&\hoac{
					& x<-3\\ 
					& x>0
				}\\ 
				& -4\le x\le 1
			}\\
			&\Leftrightarrow&\hoac{
				& -4\le x<-3\\ 
				& 0<x\le 1.
			}
		\end{eqnarray*}
		Vậy tập nghiệm của bất phương trình là $\left[ -4;-3\right)\cup\left(0;1\right]$.
	}
\end{ex}
\begin{ex}%[2D2Y5-1]
	Tập nghiệm $S$ của phương trình $\log_3\left(2x+3\right)=1$.
	\choice
	{$S=\left\{1\right\}$}
	{$S=\left\{3\right\}$}
	{$S=\left\{-1\right\}$}
	{\True $S=\left\{0\right\}$}
	\loigiai{
		Điều kiện: $2x+3>0$ $\Leftrightarrow$ $x>-\dfrac{3}{2}$.\\
		$\log_3\left(2x+3\right)=1$ $\Leftrightarrow$ $2x+3=3$ $\Leftrightarrow$ $x=0$.\\
		Vậy $S=\left\{0\right\}$.
	}
\end{ex}
\begin{ex}%[2D2B6-2]
	Tập nghiệm của bất phương trình $\log_5\left(2x-1\right)<\log_5\left(x+2\right)$ là
	\choice
	{$S=\left(-\infty ;3\right)$}
	{\True $S=\left(\dfrac{1}{2};3\right)$}
	{$S=\left(-2;3\right)$}
	{$S=\left(3;+\infty\right)$}
	\loigiai{
		Điều kiện: $\heva{
			& 2x-1>0\\ 
			& x+2>0
		}$ $\Leftrightarrow$ $x>\dfrac{1}{2}$.\\
		Ta có: $\log_5\left(2x-1\right)<\log_5\left(x+2\right)$ $\Leftrightarrow$ $2x-1<x+2$ $\Leftrightarrow$ $x<3$.\\
		Kết hợp điều kiện, tập nghiệm của bất phương trình đã cho là $\dfrac{1}{2}<x<3$.
	}
\end{ex}
\begin{ex}%[2D2Y6-1]
	Tập nghiệm của bất phương trình $\log x\ge 2$ là
	\choice
	{$\left(10;+\infty\right)$}
	{$\left(0;+\infty\right)$}
	{\True $\left[ 100;+\infty\right)$}
	{$\left(-\infty ;10\right)$}
	\loigiai{
		Điều kiện $x>0$.\\
		Bất phương trình $\log x\ge 2$ $\Leftrightarrow$ $x\ge 100$.\\
		Vậy tập nghiệm của bất phương trình đã cho là $\left[ 100;+\infty\right)$.
	}
\end{ex}
\begin{ex}%[2D2Y5-1]
	Phương trình $\log_2\left(3x+1\right)=-4$ có tập nghiệm là
	\choice
	{$\varnothing$}
	{\True $\left\{-\dfrac{5}{16}\right\}$}
	{$\left\{\dfrac{17}{48}\right\}$}
	{$\left\{5\right\}$}
	\loigiai{
		Ta có $\log_2\left(3x+1\right)=-4$ $\Leftrightarrow$ $3x+1=\dfrac{1}{16}$ $\Leftrightarrow$ $3x=-\dfrac{15}{16}$ $\Leftrightarrow$ $x=-\dfrac{5}{16}$.
	}
\end{ex}
\begin{ex}%[2D2Y6-1]
	Tập hợp nghiệm của bất phương trình $\log_2x\ge 3$ là
	\choice
	{$\left[ 6;+\infty\right)$}
	{$\left[ 9;+\infty\right)$}
	{$\left(8;+\infty\right)$}
	{\True $\left[ 8;+\infty\right)$}
	\loigiai{
		Ta có: $\log_2x\ge 3$ $\Leftrightarrow$ $x\ge 2^3$ $\Leftrightarrow$ $x\ge 8$.\\
		Do đó, tập hợp nghiệm của bất phương trình $\log_2x\ge 3$ là $\left[ 8;+\infty\right)$.
	}
\end{ex}
\begin{ex}%[2D2B5-2]
	Tập nghiệm của phương trình $\log_2\left(x^2-1\right)=\log_2\left(2x-1\right)$ là
	\choice
	{\True $\left\{2\right\}$}
	{$\varnothing$}
	{$\left\{0;1;2\right\}$}
	{$\left\{0;2\right\}$}
	\loigiai{
		Điều kiện: $x>1$.\\
		$\log_2\left(x^2-1\right)=\log_2\left(2x-1\right)$ $\Leftrightarrow$ $x^2-1=2x-1$ $\Leftrightarrow$ $x^2-2x=0$ $\Leftrightarrow$ $\hoac{
			& x=0&&\left(\text{loại}\right)\\ 
			& x=2&&\left(\text{nhận}\right).
		}$
	}
\end{ex}
\begin{ex}%[2D2B5-2]
	Nghiệm của phương trình $\log_2\left(x+3\right)+\log_2\left(x-1\right)=\log_25$ là
	\choice
	{\True $x=2$}
	{$x=1$}
	{$x=4$}
	{$x=3$}
	\loigiai{
		Điều kiện: $\heva{
			& x+3>0\\ 
			& x-1>0
		}$ $\Leftrightarrow$ $x>1$.\\
		Ta có
		\begin{eqnarray*}\allowdisplaybreaks
			&&\log_2\left(x+3\right)+\log_2\left(x-1\right)=\log_25\\
			&\Leftrightarrow&\log_2\left[\left(x+3\right)\left(x-1\right)\right]=\log_2 5\\
			&\Leftrightarrow&\heva{
				& x>1\\ 
				&\left(x+3\right)\left(x-1\right)=5 
			}\\
			&\Leftrightarrow&\heva{&x>1\\ &x^2+2x-8=0}\\
			&\Leftrightarrow&x=2.
		\end{eqnarray*}
		Vậy phương trình có nghiệm $x=2$.
	}
\end{ex}
\begin{ex}%[2D2Y6-2]
	Tập nghiệm $S$ của bất phương trình $\log_{\frac{1}{2}}\left(x^2-6x+5\right)+\log_2\left(x-1\right)>0$ là
	\choice
	{\True $\left(5;6\right)$}
	{$\left[ 5;6\right)$}
	{$\left(1;6\right)$}
	{$\left(1;+\infty\right)$}
	\loigiai{
		Điều kiện: $\heva{
			& x^2-6x+5>0\\ 
			& x-1>0
		}$ $\Leftrightarrow$ $x>5$.\\
		Với điều kiện trên, bất phương trình $\log_{\frac{1}{2}}\left(x^2-6x+5\right)+\log_2\left(x-1\right)>0$ tương đương với
		\begin{eqnarray*}\allowdisplaybreaks
			&&-\log_2\left(x^2-6x+5\right)+\log_2\left(x-1\right)>0\\
			&\Leftrightarrow&\log_2\left(x^2-6x+5\right)<\log_2\left(x-1\right)\\
			&\Leftrightarrow&x^2-6x+5<x-1\\
			&\Leftrightarrow&x^2-7x+6<0\\
			&\Leftrightarrow&1<x<6.
		\end{eqnarray*}
		Kết hợp với điều kiện ta được tập nghiệm của bất phương trình là: $S=\left(5;6\right)$.
	}
\end{ex}
\begin{ex}%[2D2B6-2]
	Số nghiệm nguyên của bất phương trình $\log_{\frac{1}{2}}\left(x-3\right)\ge \log_{\frac{1}{2}}4$ là
	\choice
	{$1$}
	{\True $4$}
	{$2$}
	{$3$}
	\loigiai{
		Bất phương trình $\log_{\frac{1}{2}}\left(x-3\right)\ge \log_{\frac{1}{2}}4$ $\Leftrightarrow$ $\heva{
			& x-3\le 4\\ 
			& x-3>0
		}$ $\Leftrightarrow$ $\heva{
			& x\le 7\\ 
			& x>3
		}$ $\Leftrightarrow$ $3<x\le 7$.\\
		Vì $\heva{
			& x\in\mathbb{Z}\\ 
			& 3<x\le 7
		}$ nên ta chọn $x\in\left\{4;5;6;7\right\}$.\\
		Vậy bất phương trình đã cho có tất cả $4$ nghiệm nguyên.
	}
\end{ex}
\begin{ex}%[2D2B6-2]
	Nghiệm của phương trình $\log_3\left(x-2\right)+\log_{\frac{1}{3}}\left(x-4\right)=1$ là
	\choice
	{$x=3$}
	{$x=6$}
	{$x=4$}
	{\True $x=5$}
	\loigiai{
		Phương trình đã cho tương đương với
		\begin{eqnarray*}\allowdisplaybreaks
			&&\heva{
				& x>4\\ 
				& \log_3\left(x-2\right)-\log_3\left(x-4\right)=1
			}\\
			&\Leftrightarrow& \heva{
				& x>4\\ 
				& \log_3\dfrac{x-2}{x-4}=1
			}\\
			&\Leftrightarrow&\heva{
				& x>4\\ 
				&\dfrac{x-2}{x-4}=3
			}\\
			&\Leftrightarrow&\heva{
				& x>4\\ 
				& x-2=3x-12
			}\\
			&\Leftrightarrow&x=5.   
		\end{eqnarray*}
	}
\end{ex}
\begin{ex}%[2D2B6-2]
	Tập nghiệm của bất phương trình $\log_{\frac{\pi}{4}}\left(x+1\right)>\log_{\frac{\pi}{4}}\left(2x-5\right)$ là
	\choice
	{$\left(-1;6\right)$}
	{$\left(\dfrac{5}{2};6\right)$}
	{$\left(-\infty ;6\right)$}
	{\True $\left(6;+\infty\right)$}
	\loigiai{
		Ta có $\log_{\frac{\pi}{4}}\left(x+1\right)>\log_{\frac{\pi}{4}}\left(2x-5\right)$ $\Leftrightarrow$ $\heva{
			& x+1>0\\ 
			& 2x-5>0\\ 
			& x+1<2x-5
		}$ $\Leftrightarrow$ $x>6$.
	}
\end{ex}
\begin{ex}%[2D2B5-2]
	Tập nghiệm của phương trình $\log_2\left(x-1\right)=\log_4\left(2 x\right)$ là
	\choice
	{$\left\{2\pm\sqrt{3}\right\}$}
	{\True $\left\{2+\sqrt{3}\right\}$}
	{$\left\{\dfrac{3}{2}\right\}$}
	{$\left\{2-\sqrt{3}\right\}$}
	\loigiai{
		Điều kiện: $x>1$.\\
		Ta có
		\begin{eqnarray*}\allowdisplaybreaks
			&&\log_2\left(x-1\right)=\log_4\left(2x\right)\\
			&\Leftrightarrow& 2\log_2\left(x-1\right)=\log_2 2x\\
			&\Leftrightarrow& (x-1)^2=2x\\
			&\Leftrightarrow& x^2-4x-1=0\\
			&\Leftrightarrow& \hoac{
				& x=2+\sqrt{3}\\ 
				& x=2-\sqrt{3}.
			}
		\end{eqnarray*}
		Đối chiếu với điều kiện ta được: $x=2+\sqrt{3}$.
	}
\end{ex}
\Closesolutionfile{ans}
%======================
\subsection{Bảng đáp án}
\inputansbox{8}{ans/ANS-DANG-21}

