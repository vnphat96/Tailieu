%Dạng 1
\setcounter{ex}{0}
\section{Xác suất}
\subsection{Kiến thức cần nhớ}
\begin{khung}
	\subsubsection{Định nghĩa xác suất}
	Xác suất của biến cố $A$ được tính bởi công thức
\begin{center}
	$\mathrm{P}(A)=\dfrac{n(A)}{n(\Omega)}$.
\end{center}
Trong đó 
\begin{itemize}
	\item $n(A)$ là số kết quả thuận lợi của biến cố $A$;
	\item $n(\Omega)$ là số kết quả có thể xảy ra của phép thử.
\end{itemize}
	\subsubsection{Tính chất}
	\begin{itemize}
		\item Giả sử $A$ và $B$ là các biến cố liên quan đến một phép thử có một số hữu hạn kết quả đồng khả năng xuất hiện. Khi đó, ta có
		\begin{enumerate}
			\item $\mathrm{P}(\varnothing)=0$, $\mathrm{P}(\Omega)=1$.
			\item $0\le \mathrm{P}(A)\le 1$, với mọi biến cố $A$.
			\item Nếu $A$ và $B$ xung khắc, thì
			$\mathrm{P}(A\cup B)=\mathrm{P}(A)+\mathrm{P}(B)$
			 \textit{(công thức cộng xác suất).}
		\end{enumerate}
	\item Các biến cố $A$ và $B$ là xung khắc nếu và chỉ nếu chúng không khi nào cùng xảy ra.
	\item Với mọi biến cố $A$, ta có 
	$$\mathrm{P}\left(\overline{A}\right)=1-\mathrm{P}(A).$$
	\item Với hai biến cố bất kỳ, ta có mối quan hệ sau (công thức nhân xác suất):
	$$A \text{ và } B \text{ là hai biến cố độc lập}\Leftrightarrow \mathrm{P}(A\cdot B)=\mathrm{P}(A)\cdot \mathrm{P}(B).$$
	
	\end{itemize}
\end{khung}

\subsection{Bài tập mẫu}
\Opensolutionfile{ans}[ans/ANS-DANG-1]
\begin{khung}
\begin{vd}[Đề tham khảo BGD 2022-2023] %[1D2K5-2]%
	Một hộp chứa $15$ quả cầu gồm $6$ quả màu đỏ được đánh số từ $1$ đến $6$ và $9$ quả màu xanh được đánh số từ $1$ đến $9$. Lấy ngẫu nhiên hai quả từ hộp đó, xác suất để lấy được hai quả khác màu đồng thời tổng hai số ghi trên chúng là số chẵn bằng
	\choice
	{$\dfrac{9}{35}$}
	{$\dfrac{18}{35}$}
	{$\dfrac{4}{35}$}
	{$\dfrac{1}{7}$}
	\loigiai{Gọi $A$ là biến cố lấy được hai quả khác màu đồng thời tổng hai số ghi trên chúng là số chẵn. \\Khi đó, $n(\Omega)=\mathrm{C}_{15}^2=105$. 
		\\Do tổng hai số của chúng là số chẵn nên ta có các trường hợp sau
		\begin{itemize}
			\item Cả hai quả cầu đều ghi số lẻ. Số cách lấy là $3\cdot 5=15$.
			\item Cả hai quả cầu đều ghi số chẵn. Số cách lấy là $3\cdot 4=12$.
		\end{itemize}
		Số phần tử của biến cố $A$ là $n(A)=15+12=27$.
		\\Vậy xác suất xuất hiện của biến cố $A$ là $\mathrm P(A)= \dfrac{27}{105} =\dfrac{9}{35}$. }
\end{vd}
\end{khung}
\subsection{Bài tập tương tự và phát triển}
%%==========Câu 1
\begin{ex} %[1D2K5-2]%
	Cho một hộp chứa $9$ viên bi được đánh số từ $1$ đến $9$. Chọn ngẫu nhiên 3 viên bi rồi cộng các số trên 3 viên đó với nhau. Xác suất để số thu được là số lẻ bằng
	\choice
	{$\dfrac{3}{4}$}
	{$\dfrac{11}{21}$}
	{$\dfrac{1}{2}$}
	{\True $\dfrac{10}{21}$}
	\loigiai{
		Không gian mẫu: $n(\Omega)=\mathrm{C}_9^3=84$.\\
		Gọi $A$: \lq\lq  Thu được 3 viên có tổng số ghi trên bi là số lẻ\rq\rq .\\
		Để biến cố $A$ xảy ra có 2 khả năng
		\begin{itemize}
			\item Khả năng $1$: Cả $3$ viên đều có số lẻ. Số cách lấy: $\mathrm{C}_5^3$.
			\item 	Khả năng $2$: Lấy được $2$ viên chẵn, $1$ viên lẻ. Số cách lấy: $\mathrm{C}_4^2\cdot \mathrm{C}_5^1$.
		\end{itemize}
		Vậy $n(A)=\mathrm{C}_5^3+\mathrm{C}_4^2\cdot \mathrm{C}_5^1=40\Rightarrow \mathrm P(A)=\dfrac{10}{21}$.}
\end{ex}

%%==========Câu 2
\begin{ex}%[1D2K5-2]%
	Một hộp chứa 6 bi vàng, 5 bi đỏ và 4 bi xanh. Lấy ngẫu nhiên 8 bi trong hộp. Xác suất để trong 8 bi lấy ra có số bi vàng và số bi đỏ khác nhau là
	\choice
	{$\dfrac{344}{429}$}
	{$\dfrac{526}{1001}$}
	{$\dfrac{95}{429}$}
	{\True $\dfrac{334}{429}$}
	\loigiai{
		Ta có $n(\Omega)=\mathrm{C}_{15}^8=6435$.\\
		Gọi $A$ là biến cố: \lq\lq  Trong $8$ bi lấy ra có số bi vàng và số bi đỏ khác nhau\rq\rq.\\
		Khi đó $\overline{A}$ là biến cố: \lq\lq  Trong $8$ bi lấy ra có số bi vàng và số bi đỏ bằng nhau\rq\rq.
		\begin{itemize}
		\item TH1: \lq\lq  $2$ bi vàng, $2$ bi đỏ, $4$ bi xanh\rq\rq có $\mathrm{C}_6^2\cdot \mathrm{C}_5^2\cdot \mathrm{C}_4^4=150$.
		\item TH2: \lq\lq  $3$ bi vàng, $3$ bi đỏ, $2$ bi xanh\rq\rq có $\mathrm{C}_6^3\cdot \mathrm{C}_5^3\cdot \mathrm{C}_4^2=1200$.
		\item TH3: \lq\lq  $4$ bi vàng, $4$ bi đỏ\rq\rq có $\mathrm{C}_6^4\cdot \mathrm{C}_5^4=75$.
	\end{itemize}
		Ta có $n(\overline{A})=150+1200+75=1425$.\\
		Suy ra $\mathrm P(\overline{A})= \dfrac{1425}{6435}=\dfrac{95}{429}\Rightarrow \mathrm P(A)= 1-\mathrm P(\overline{A}) =\dfrac{334}{429}$.}
\end{ex}

%%==========Câu 3
\begin{ex}%[1D2K5-2]%
	Có $3$ chiếc hộp. Mỗi hộp chứa $4$ tấm thẻ được đánh số từ $1$ đến $4$. Lấy ngẫu nhiên từ mỗi hộp một thẻ. Tính xác suất để $3$ thẻ được lấy ra đều mang số chẵn.
	\choice
	{$\dfrac{2}{3}$}
	{$\dfrac{3}{32}$}
	{$\dfrac{1}{2}$}
	{\True $\dfrac{1}{8}$}
	\loigiai{
		Số phần tử của không gian mẫu: $n(\Omega)=\mathrm{C}_4^1\cdot \mathrm{C}_4^1\cdot \mathrm{C}_4^1=64$.\\
		Gọi $A$ là biến cố \lq\lq $3$ thẻ mang số chẵn\rq\rq. Suy ra  $n(A)=\mathrm{C}_2^1\cdot \mathrm{C}_2^1\cdot \mathrm{C}_2^1=8$.\\
		Xác suất cho biến cố $A$: $\mathrm P(A)=\dfrac{8}{64}=\dfrac{1}{8}$.}
\end{ex}

%%==========Câu 4
\begin{ex}%[1D2K5-2]%
	Gọi $S$ là tập hợp các số tự nhiên có $3$ chữ số được lập từ tập $A=\left\{0;1;2;3;\ldots;9\right\}$. Chọn ngẫu nhiên một số từ tập $S$, tính xác suất để chọn được số tự nhiên có tích các chữ số bằng $30$.
	\choice
	{\True $\dfrac{1}{75}$}
	{$\dfrac{4}{3\cdot 10^3}$}
	{$\dfrac{1}{50}$}
	{$\dfrac{1}{108}$}
	\loigiai{
		Gọi số tự nhiên có $3$ chữ số được lập từ tập $A=\left\{0;1;2;3;\ldots;9\right\}$ là $\overline{abc}\, (a\ne 0)$ khi đó số phần tử của tập $S$ là: $9\cdot 10\cdot 10=900\Rightarrow$ số phần tử của không gian mẫu là: $n(\Omega)=\mathrm{C}_{900}^1=900$.\\
		Bộ $3$ chữ số có tích bằng $30$ là $(1;5;6);(2;5;3)$.\\
		Từ $2$ bộ $3$ chữ số trên lập được $2\cdot 3!=12$ số tự nhiên có $3$ chữ số mà tích các chữ số bằng $30$.\\
		Khi đó gọi $B$ là biến cố \lq\lq  chọn được số tự nhiên có tích các chữ số bằng $30$\rq\rq thì $n(B)=12$.\\
		$\Rightarrow \mathrm P(B)=\dfrac{12}{900}=\dfrac{1}{75}$.}
\end{ex}

%%==========Câu 5
\begin{ex}%[1D2K5-2]%
	Một hộp chứa $11$ viên bi được đánh số từ $1$ đến $11$. Chọn ngẫu nhiên $6$ viên bi từ hộp. Tính xác suất để tổng các số trên các viên bi là một số lẻ?
	\choice
	{$\dfrac{103}{231}$}
	{$\dfrac{215}{462}$}
	{\True $\dfrac{118}{231}$}
	{$\dfrac{115}{231}$}
	\loigiai{
		Phép thử là chọn $6$ viên bi từ $11$ viên bi nên số phần tử của không gian mẫu là: $n(\Omega)=\mathrm{C}_{11}^6$.\\
		Gọi biến cố $A$:\lq\lq  Tổng các số trên các viên bi là một số lẻ\rq\rq.
		\begin{itemize}
			\item 	TH1: Chọn được 5 viên bi đánh số lẻ và 1 viên bi số chẵn: có $\mathrm{C}_6^5\mathrm{C}_5^1$ cách chọn.
			\item 	TH2: Chọn được 3 viên bi đánh số lẻ và 2 viên bi số chẵn: có $\mathrm{C}_6^3\mathrm{C}_5^2$ cách chọn.
			\item TH3: Chọn được 1 viên bi đánh số lẻ và 5 viên bi số chẵn: có $\mathrm{C}_6^1\mathrm{C}_5^5$ cách chọn.\\
			$n(A)=\mathrm{C}_6^5\mathrm{C}_5^1+\mathrm{C}_6^3\mathrm{C}_5^2+\mathrm{C}_6^1\mathrm{C}_5^5$.
		\end{itemize}
	Vậy	xác suất của biến cố $A$ là $\mathrm {P}(A)=\dfrac{n(A)}{n(\Omega)} =\dfrac{118}{231}$.}
\end{ex}

%%==========Câu 6
\begin{ex}%[1D2K5-2]%
	Có $30$ tấm thẻ đánh số từ $1$ đến $30$. Chọn ngẫu nhiên $10$ tấm thẻ. Tính xác suất để có $5$ tấm thẻ mang số lẻ, $5$ tấm thẻ mang số chẵn, trong đó có đúng $1$ thẻ mang số chia hết cho $10$.
	\choice
	{\True $\dfrac{99}{667}$}
	{$0$, $1$}
	{$\dfrac{48}{105}$}
	{$0$, $17$}
	\loigiai{
		Số phần tử không gian mẫu: $n(\Omega)=\mathrm{C}_{30}^{10}$.\\
		Gọi biến cố A: \lq\lq  Lấy được $5$ tấm thẻ mang số lẻ, $5$ tấm thẻ mang số chẵn, trong đó có đúng $1$ thẻ mang số chia hết cho $10$\rq\rq.\\
		Lấy $5$ tấm thẻ mang số lẻ trong $15$ tấm thẻ mang số lẻ có: $\mathrm{C}_{15}^5$ cách.\\
		Lấy 1 tấm thẻ mang số chia hết cho $10$ trong 3 tấm thẻ mang số chia hết cho $10$ có: $\mathrm{C}_3^1$ cách.\\
		Lấy $4$ tấm thẻ mang số chẵn và không chia hết cho $10$ trong $12$ tấm thẻ mang số chẵn và không chia hết cho $10$ có: $\mathrm{C}_{12}^4$.\\
		Do đó: $n(A)=\mathrm{C}_{15}^5\mathrm{C}_3^1\mathrm{C}_{12}^4=4459455$.\\
		Vậy $\mathrm {P}(A)= \dfrac{n(A)}{n(\Omega)}=\dfrac{4459455}{\mathrm{C}_{30}^{10}}=\dfrac{99}{667}$.}
\end{ex}
%%==========Câu 7
\begin{ex}%[1D2K5-2]%
	Chọn ngẫu nhiên một số tự nhiên có $4$ chữ số. Tính xác suất để số được chọn không vượt quá $2023$, đồng thời nó chia hết cho $5$.
	\choice
	{\True $\dfrac{41}{1800}$}
	{$\dfrac{99}{750}$}
	{$\dfrac{48}{1800}$}
	{$\dfrac{17}{105}$}
\loigiai{
	Số các số tự nhiên có $4$ chữ số là $9\cdot 10\cdot 10\cdot 10=9000$ (số).\\
	Suy ra số phần tử của không gian mẫu là $n(\Omega)=9000$.\\
	Gọi $A$ là biến cố \lq\lq  Số được chọn không vượt quá $2023$ và chia hết cho $5$\rq\rq.\\
	Số có bốn chữ số nhỏ nhất chia hết cho $5$ là $1000$, số có bốn chữ số lớn nhất không vượt quá $2023$ chia hết cho $5$ là $2020$.\\
	Suy ra số phần tử của biến cố $A$ là $n(A)=(2020-1000)\colon 5+1=205$.\\
	Xác suất của biến cố $A$ là $\mathrm{P}(A)=\dfrac{n(A)}{n(\Omega)}=\dfrac{205}{9000}=\dfrac{41}{1800}$.}
\end{ex}
%%==========Câu 8
\begin{ex}%[1D2K5-2]%
	Cho tập hợp $A=\left\{0,1,2,3,4,5\right\}$. Gọi $S$ là tập hợp các số gồm có $3$ chữ số khác nhau được lập từ các chữ số của tập $A$. Chọn ngẫu nhiên một số từ $S$. Tính xác suất để số được chọn có chữ số cuối gấp đôi chữ số đầu.
	\choice
	{$\dfrac{1}{5}$}
	{$\dfrac{23}{25}$}
	{\True $\dfrac{2}{25}$}
	{$\dfrac{4}{5}$}
	\loigiai{
		Gọi số tự nhiên cần lập gồm $3$ chữ số khác nhau là $\overline{abc}$.\\
		Chọn chữ số $a\colon 5$ cách.\\
		Chọn chữ số $b$, $c\colon \mathrm{A}_5^2$.\\
		Suy ra $ n(\Omega)=5\cdot \mathrm{A}_5^2$ cách.\\
		Gọi $A$: \lq\lq  là số được chọn gồm ba chữ số có chữ số cuối gấp đôi chữ số đầu\rq\rq. \\
		Vì chữ số cuối gấp đôi chữ số đầu nên $c=2a(a\ne 0)$.\\
		Chọn $a=1\Rightarrow c=2\Rightarrow$ chọn $b$ có $4$ cách.\\
		Chọn $a=2\Rightarrow c=4\Rightarrow$ chọn $b$ có $4$ cách.\\
		$\Rightarrow$ Số cách lập số thoả yêu cầu bài toán là $4+4=8$ cách.\\
		Suy ra số phần tử của biến cố $A$ là  $n(A)=8$.	Vậy xác suất của biến cố $A$ là $\mathrm {P}(A)=\dfrac{n(A)}{n(\Omega)}=\dfrac{2}{25}$.}
\end{ex}

%%==========Câu 9
\begin{ex}%[1D2K5-2]%
	Có $6$ học sinh lớp $11$ và $3$ học sinh lớp $12$. Tính xác suất để trong các cách sắp xếp ngẫu nhiên $9$ học sinh đó vào một dãy có $9$ chiếc ghế sao cho không có hai học sinh lớp $12$ nào ngồi cạnh nhau.
	\choice
	{$\dfrac{5}{72}$}
	{$\dfrac{7}{72}$}
	{\True $\dfrac{5}{12}$}
	{$\dfrac{1}{1728}$}
	\loigiai{
		Số phần tử của không gian mẫu là $n(\Omega)=9!$.\\
		Gọi $A$ là biến cố: \lq\lq  xếp $9$ học sinh vào một dãy sao cho không có hai học sinh lớp $12$ nào ngồi cạnh nhau\rq\rq.\\
		Số cách xếp $6$ học sinh lớp $11$ là $6!$. Khi đó có $7$ vách ngăn tạo ra, ta chọn $3$ trong $7$ vách ngăn đó để xếp $3$ học sinh lớp 12 là $\mathrm{A}_7^3$.\\
		Xác suất cần tính là $\mathrm {P}(A)=\dfrac{6!\cdot \mathrm{A}_7^3}{9!}=\dfrac{5}{12}$.}
\end{ex}

%%==========Câu 10
\begin{ex}%[1D2K5-2]%
	Chọn ngẫu nhiên hai số khác nhau từ $27$ số nguyên dương đầu tiên. Xác suất để chọn được hai số có tổng là một số chẵn bằng
	\choice
	{\True $\dfrac{13}{27}$}
	{$\dfrac{14}{27}$}
	{$\dfrac{1}{2}$}
	{$\dfrac{365}{729}$}
	\loigiai{
		Gọi $\Omega$ là không gian mẫu của phép thử.\\
		Số cách chọn hai số khác nhau từ $27$ số nguyên dương là $\mathrm{C}_{27}^2$ $\Rightarrow n(\Omega)=\mathrm{C}_{27}^2=351$.\\
		Gọi $A$ là biến cố: \lq\lq  Chọn được hai số có tổng là một số chẵn\rq\rq.\\
		Có 2 trường hợp: Chọn được 2 số lẻ trong 14 số lẻ dương đầu tiên hoặc chọn được hai số chẵn trong 13 số chẵn dương đầu tiên.\\
		Suy ra số phần tử của biến cố $A$ là  $n(A)=\mathrm{C}_{13}^2+\mathrm{C}_{14}^2=169$. Vậy xác suất của biến cố $A$ là $\mathrm {P}(A)=\dfrac{n(A)}{n(\Omega)}=\dfrac{13}{27}$.}
\end{ex}

%%==========Câu 11
\begin{ex}%[1D2K5-2]%
	Gọi $S$ là tập các số tự nhiên có bốn chữ số khác nhau được tạo từ tập $E=\{1; 2; 3; 4; 5\}$. Chọn ngẫu nhiên một số từ tập $S$. Tính xác suất để số được chọn là một số chẵn?
	\choice
	{$\dfrac{3}{4}$}
	{\True $\dfrac{2}{5}$}
	{$\dfrac{3}{5}$}
	{$\dfrac{1}{2}$}
	\loigiai{
		Gọi $\Omega$ là không gian mẫu của phép thử.\\
		Số tự nhiên có bốn chữ số khác nhau được tạo từ tập $E$ là $\mathrm{A}_5^4$ $\Rightarrow n(\Omega)=\mathrm{A}_5^4=120$.\\
		Gọi $A$ là biến cố: \lq\lq  Chọn một số chẵn\rq\rq.\\
		Suy ra số phần tử của biến cố $A$ là $n(A)=2\cdot \mathrm{A}_4^3=48$. \\
		Vậy xác suất của biến cố $A$ là $\mathrm {P}(A)=\dfrac{n(A)}{n(\Omega)}=\dfrac{2}{5}$.}
\end{ex}

%%==========Câu 12
\begin{ex}%[1D2K5-2]%
	Một hộp đựng $11$ viên bi được đánh số từ $1$ đến $11$. Lấy ngẫu nhiên $4$ viên bi, rồi cộng các số trên các bi lại với nhau. Xác suất để kết quả thu được là $1$ số lẻ bằng
	\choice
	{$\dfrac{31}{32}$}
	{$\dfrac{11}{32}$}
	{\True $\dfrac{16}{33}$}
	{$\dfrac{21}{32}$}
	\loigiai{
		Gọi $\Omega$ là không gian mẫu của phép thử.\\
		Số cách lấy ngẫu nhiên $4$ viên bi từ hộp là $\mathrm{C}_{11}^4$ $\Rightarrow n(\Omega)=\mathrm{C}_{11}^4=330$.\\
		Gọi $A$ là biến cố: \lq\lq  Kết quả thu được là $1$ số lẻ\rq\rq.
		\begin{itemize}
			\item 	Trường hợp 1: Chọn được $1$ số lẻ và $3$ số chẵn $\Rightarrow$ có $\mathrm{C}_6^1\cdot \mathrm{C}_5^3=60$ cách.
			\item Trường hợp 2: chọn được 3 số lẻ và 1 số chẵn $\Rightarrow$ có $\mathrm{C}_6^3\cdot \mathrm{C}_5^1=100$ cách.
		\end{itemize}
	Suy ra số phần tử của biến cố $A$ là $ n(A)=160$.\\ Vậy xác suất của biến cố $A$ là $\mathrm {P}(A)=\dfrac{n(A)}{n(\Omega)}=\dfrac{16}{33}$.}
\end{ex}

%%==========Câu 13
\begin{ex}%[1D2K5-2]%
	Cho $14$ tấm thẻ đánh số từ $1$ đến $14$. Chọn ngẫu nhiên $3$ thẻ. Xác suất để tích $3$ số ghi trên $3$ tấm thẻ này chia hết cho $3$ bằng
	\choice
	{$\dfrac{30}{91}$}
	{\True $\dfrac{61}{91}$}
	{$\dfrac{31}{91}$}
	{$\dfrac{12}{17}$}
	\loigiai{
		Gọi $\Omega$ là không gian mẫu của phép thử.\\
		Số cách lấy ngẫu nhiên $3$ tấm thẻ là $\mathrm{C}_{14}^3$ $\Rightarrow n(\Omega)=\mathrm{C}_{14}^3=364$.\\
		Gọi $A$ là biến cố: \lq\lq  Tích $3$ số ghi trên $3$ tấm thẻ này chia hết cho $3$\rq\rq.\\
		Gọi $B$ là tập chứa các số chia hết cho 3 $\Rightarrow B=\left\{3;6;9;12\right\}$.\\
		Gọi $C$ là tập chứa các số không chia hết cho 3 $\Rightarrow n(C)=10$.
		\begin{itemize}
		\item Trường hợp 1: Chọn được 1 số trong $B$ và 2 số trong $C \Rightarrow$ có $\mathrm{C}_4^1\cdot \mathrm{C}_{10}^2=180$ cách.
		\item Trường hợp 2: Chọn được 2 số trong $B$ và 1 số trong $C$ $\Rightarrow$ có $\mathrm{C}_4^2\cdot \mathrm{C}_{10}^1=60$ cách.
		\item Trường hợp 3: Chọn được $3$ số trong $B \Rightarrow$ có $\mathrm{C}_4^3=4$ cách.
		\end{itemize}
		Suy ra số phần tử của biến cố $A$ là $ n(A)=244$.\\ Vậy xác suất của biến cố $A$ là $\mathrm {P}(A)=\dfrac{n(A)}{n(\Omega)}=\dfrac{61}{91}$.}
\end{ex}

%%==========Câu 14
\begin{ex}%[1D2K5-2]%
	Gọi $S$ là tất cả các số tự nhiên gồm hai chữ số khác nhau lập từ các chữ số $0, 1, 2, 3, 4, 5, 6$. Chọn ngẫu nhiên hai số từ tập $S$. Tính xác suất để tích hai số được chọn là số chẵn.
	\choice
	{$\dfrac{1}{6}$}
	{$\dfrac{2}{5}$}
	{\True $\dfrac{5}{6}$}
	{$\dfrac{3}{4}$}
	\loigiai{
		Gọi $\Omega$ là không gian mẫu của phép thử.\\
		Số tự nhiên có $2$ chữ số khác nhau lập từ $0,1,2,3,4,5,6$ là 
		$$6\cdot 6=36\Rightarrow n(\Omega)=\mathrm{C}_{36}^2=630.$$
		Gọi $A$ là biến cố: \lq\lq  Tích hai số được chọn là số chẵn\rq\rq.\\
		Số tự nhiên chẵn có $2$ chữ số khác nhau lập từ $0;1;2;3;4;5;6$ là $\overline{ab}$
		\begin{itemize}
			\item Với $b=0 \Rightarrow$ có 6 số.
			\item Với $b\ne 0 \Rightarrow$ có số $3\cdot 5=15$ số.
		\end{itemize}
		Có $21$ số tự nhiên là số chẵn suy ra có $15$ số tự nhiên là số lẻ.\\
		Gọi $A$ là biến cố \lq\lq  tích hai số được chọn là số chẵn\rq\rq.
		\begin{itemize}
			\item Trường hợp 1: Chọn được 1 số lẻ và 1 số chẵn $\Rightarrow$ có $\mathrm{C}_{21}^1\cdot \mathrm{C}_{15}^1=315$ cách.
			\item Chọn được 2 số chẵn $\Rightarrow$ có $\mathrm{C}_{21}^2=210$ cách.
		\end{itemize}
		Suy ra số phần tử của biến cố $A$ là $ n(A)=525$.\\ Vậy xác suất của biến cố $A$ là $\mathrm {P}(A)=\dfrac{n(A)}{n(\Omega)}=\dfrac{5}{6}$.}
	
\end{ex}

%%==========Câu 15
\begin{ex}%[1D2K5-2]%
	Gọi $A$ là tập hợp các số có ba chữ số khác nhau được lập từ các chữ số $1, 2, 3, 4, 5$. Chọn ngẫu nhiên ba số từ tập hợp $A$, xác suất để trong ba số được chọn có đúng một số có mặt chữ số $4$ bằng
	\choice
	{\True $\dfrac{2484}{8555}$}
	{$\dfrac{5}{17}$}
	{$\dfrac{2518}{8555}$}
	{$\dfrac{4}{17}$}
	\loigiai{
		Gọi $\Omega$ là không gian mẫu của phép thử.\\
		Số tự nhiên có ba chữ số khác nhau được lập từ các chữ số $1,2,3,4,5$ là 
		$$\mathrm{A}_5^3=60 \Rightarrow n(\Omega)=\mathrm{C}_{60}^3=34220.$$
		Gọi $B$ là biến cố: \lq\lq  Trong ba số được chọn có đúng một số có mặt chữ số $4$\rq\rq.\\
		Số tự nhiên có mặt chữ số $4$ là $\overline{abc}$.\\
		Đưa số $4$ vào $3$ vị trí trên có $3$ cách chọn.
		$2$ số còn lại có $\mathrm{A}_4^2=12$.\\
		Số các số tự nhiên có mặt chữ số $4$ là $3\cdot 12=36$\\
		Suy ra số phần tử của biến cố $B$ là $ n(B)=\mathrm{C}_{36}^1\cdot \mathrm{C}_{24}^2=9936$.\\ 
		Vậy xác suất của biến cố $B$ là $\mathrm {P}(B)=\dfrac{n(B)}{n(\Omega)}=\dfrac{2484}{8555}$.}
	\end{ex}

%%==========Câu 16
\begin{ex}%[1D2K5-2]%
	Gọi $S$ là tập hợp tất cả các số tự nhiên có $3$ chữ số được lập từ tập $X=\left\{0;1;2;3;4;5;6;7\right\}$. Rút ngẫu nhiên một số từ $S$. Tính xác suất để rút được số mà trong số đó, chữ số đứng sau luôn lớn hơn hoặc bằng chữ số đứng trước.
	\choice
	{$\dfrac{3}{32}$}
	{$\dfrac{2}{7}$}
	{\True $\dfrac{3}{16}$}
	{$\dfrac{11}{64}$}
	\loigiai{
		Số các số tự nhiên có $3$ chữ số được lập từ $X$ là: $7\cdot 8\cdot 8=448$ (số).\\
		$\Rightarrow n(\Omega)=448$.\\
		Gọi $A$ là biến cố \lq\lq  rút được số mà trong số đó chữ số đứng sau luôn lớn hơn hoặc bằng chữ số đứng trước\rq\rq.
		\begin{itemize}
			\item TH1: Số rút được có dạng $\overline{aaa}$, có 7 số như thế nên có 7 cách rút.
			\item 	TH2: Số rút được có dạng $\overline{aab}$, với $a<b$, có $\mathrm{C}_7^2=21$ số như thế nên có $21$ cách rút.
			\item TH3: Số rút được có dạng $\overline{abb}$, với $a<b$, có $\mathrm{C}_7^2=21$ số như thế nên có $21$ cách rút.
			\item 	TH4: Số rút được có dạng $\overline{abc}$, với $a<b<c$, có $\mathrm{C}_7^3=35$ số như thế nên có $35$ cách rút.
		\end{itemize}
		Vậy có $7+21+21+35=84$ (số) $\Rightarrow n(A)=84$. Do đó $\mathrm{P}(A)=\dfrac{84}{448}=\dfrac{3}{16}$.}
\end{ex}

%%==========Câu 17
\begin{ex}%[1D2K5-2]%
	Một hộp đựng $50$ chiếc thẻ được đánh số từ $1$ đến $50$. Chọn ngẫu nhiên từ một hộp hai thẻ. Tính xác suất để hiệu bình phương số ghi trên hai thẻ lấy được là số chia hết cho $3$.
	\choice
	{$\dfrac{409}{1225}$}
	{\True $\dfrac{681}{1225}$}
	{$\dfrac{8}{25}$}
	{$\dfrac{801}{1225}$}
	\loigiai{
		Mỗi lần lấy ra hai thẻ từ hộp có $50$ cái thẻ là một tổ hợp chập $2$ của $50$.\\
		Số cách lấy bằng $\mathrm{C}_{50}^2$.
		Suy ra $n(\Omega)=\mathrm{C}_{50}^2=1225$.\\
		Gọi $A$ là biến cố: \lq\lq  hiệu bình phương số ghi trên hai thẻ lấy được là số chia hết cho $3$\rq\rq.\\
		Trong $50$ chiếc thẻ được đánh số từ $1$ đến $50$ có $16$ cái thẻ đánh số chia hết cho $3$, $17$ cái thẻ đánh số chia cho $3$ dư $1$ và $17$ cái thẻ đánh số chia cho $3$ dư $2$.\\
		Để hiệu bình phương số ghi trên hai thẻ lấy được là số chia hết cho $3$ xảy ra $4$ trường hợp.
		\begin{itemize}
			\item TH1: Hai thẻ có số chia hết cho $3$ có $\mathrm{C}_{16}^2$.
			\item TH2: Hai thẻ có số chia cho $3$ dư $1$ có $\mathrm{C}_{17}^2$.
			\item TH3: Hai thẻ có số chia cho $3$ dư $2$ có $\mathrm{C}_{17}^2$.
			\item 	TH4: Có $1$ thẻ có số chia cho $3$ dư $1$ và một thẻ có số chia cho $3$ dư $2$ có $\mathrm{C}_{17}^1\cdot \mathrm{C}_{17}^1$.
		\end{itemize}
		Vậy số cách lấy ra hai thẻ thỏa mãn là $\mathrm{C}_{16}^2+\mathrm{C}_{17}^2+\mathrm{C}_{17}^2+\mathrm{C}_{17}^1\cdot \mathrm{C}_{17}^1=681$.\\
		Xác suất để hiệu bình phương số ghi trên hai thẻ lấy được là số chia hết cho 3 là $\mathrm{P}(A)=\dfrac{681}{1225}$.}
\end{ex}

%%==========Câu 18
\begin{ex}%[1D2K5-2]%
	Một hộp gồm $30$ quả cầu được đánh số từ $1$ đến $30$. Chọn ngẫu nhiên $3$ quả cầu từ hộp đó. Xác suất để lấy được $3$ quả cầu có đúng $1$ quả cầu ghi số lẻ và tích $3$ số ghi trên ba quả cầu là một số chia hết cho $8$ bằng
	\choice
	{\True $\dfrac{33}{116}$}
	{$\dfrac{21}{58}$}
	{$\dfrac{45}{116}$}
	{$\dfrac{6}{29}$}
	\loigiai{
		Ta có số phần tử không gian mẫu là số cách chọn $3$ quả cầu từ hộp nên $n(\Omega)=\mathrm{C}_{30}^3$.\\
		Gọi $A$ là biến cố \lq\lq  lấy được $3$ quả cầu có đúng $1$ quả cầu ghi số lẻ và tích $3$ số ghi trên ba quả cầu là một số chia hết cho $8$\rq\rq.\\
		Chọn $1$ số lẻ từ $30$ số: có $15$ cách.\\
		Đặt $A_1=\left\{4; 8; 12; 16; 20; 24; 28\right\}$, $A_2=\left\{2; 6; 10; 14; 18; 22; 16; 30\right\}$ trong đó $A_1$ gồm những số chia hết cho $4$ và $A_2$ là những số chẵn không chia hết cho $4$.\\
		Do quả cầu đầu tiên mang số lẻ nên để chọn $3$ quả cầu thỏa mãn yêu cầu bài toán thì tích $2$ số trên $2$ quả cầu còn lại phải là số chia hết cho $8$.\\
		Trường hợp 1: $2$ số từ tập $A_1$ có $\mathrm{C}_7^2=21$ cách.\\
		Trường hợp 2: $1$ số từ tập $A_1$ và số còn lại là số tùy ý từ $A_2$ có $7\cdot 8=56$ cách.\\
		Vậy $n(A)=15(21+56)=1155$ nên $\mathrm{P}(A)=\dfrac{n(A)}{n(\Omega)}=\dfrac{1155}{\mathrm{C}_{30}^3}=\dfrac{33}{116}$.}
\end{ex}
%%==========Câu 19
\begin{ex}%[1D2K5-2]%
	Gọi $S$ là tập hợp các số tự nhiên gồm $4$ chữ số đôi một khác nhau được lập nên từ các chữ số $0;$ $1;$ $2;$ $3;$ $4;$ $5$. Chọn ngẫu nhiên một số từ tập $S$. Xác suất để số được chọn có chứa ít nhất một trong hai chữ số $1$ hoặc $2$ bằng
	\choice
	{$\dfrac{1}{3}$}
	{$\dfrac{1}{15}$}
	{$\dfrac{3}{50}$}
	{\True $\dfrac{47}{50}$}
	\loigiai{
		Ta có số phần tử của tập $S$ là: $5\cdot A_5^3=300$.\\
		$\Rightarrow$ Số phần tử của không gian mẫu là $n(\Omega)=300$.\\
		Gọi $A$ là biến cố: \lq\lq  Số được chọn từ tập $S$ có chứa ít nhất một trong hai chữ số $1$ hoặc $2$\rq\rq.\\
		$\Rightarrow$ $\overline{A}$ là biến cố: \lq\lq  Số được chọn từ tập $S$ không có mặt cả hai chữ số $1$ và $2$\rq\rq.\\
		Số các số tự nhiên gồm $4$ chữ số đôi một khác nhau được lập nên từ các chữ số $0;$ $3;$ $4;$ $5$ là: $3\cdot 3!=18$. Do đó $n(\overline{A})=18$.\\
		Vậy $\mathrm{P}(A)=1-P(\overline{A})=1-\dfrac{n(\overline{A})}{n(\Omega)}=1-\dfrac{18}{300}=\dfrac{47}{50}$.}
\end{ex}

%%==========Câu 20
\begin{ex}%[1D2K5-2]%
	Có $6$ học sinh nam và $3$ học sinh nữ được xếp chỗ ngồi ngẫu nhiên vào một dãy gồm $9$ ghế. Xác suất để mỗi học sinh nữ được xếp ngồi xen giữa hai học sinh nam là
	\choice
	{\True $11{,}9\%$}
	{$58{,}33\%$}
	{$60{,}71\%$}
	{$6{,}94\%$}
	\loigiai{
		Số cách xếp chỗ ngồi cho $9$ học sinh vào $9$ ghế là: $n(\Omega)=9!$.\\
		Gọi $A$ là biến cố: \lq\lq  Mỗi học sinh nữ được xếp ngồi xen giữa hai học sinh nam\rq\rq.\\
		Xếp thứ tự $6$ học sinh nam có $6!$ cách.\\
		Xếp thứ tự $3$ học sinh nữ vào giữa các học sinh nam có $\mathrm{A}_5^3$ cách
		$\Rightarrow n(A)=6!\cdot A_5^3$.\\
		Xác suất biến cố $A$ là: $\mathrm{P}(A)=\dfrac{n(A)}{n(\Omega)}=\dfrac{6!A_5^3}{9!}=\dfrac{5}{42}=11{,}9\%$.}
\end{ex}

\Closesolutionfile{ans}
%======================
\subsection{Bảng đáp án}
\inputansbox{8}{ans/ANS-DANG-1}

