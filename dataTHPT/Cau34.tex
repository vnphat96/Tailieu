\setcounter{section}{33}
\setcounter{ex}{0}
\section{Phương trình mũ}
\subsection{Kiến thức cần nhớ}
\begin{khung}
	Sử dụng kiến thức $a^x=b\Leftrightarrow x=\log_{a}b$ với $a,b>0$,$a\ne 1$.
\end{khung}
\subsection{Bài tập mẫu}
\Opensolutionfile{ans}[ans/ANS-DANG-34]
\begin{khung}
	\begin{vd}[Đề minh họa BGD 2022-2023]%[Triết Nguyễn Bạch Nha]%[2D2B5-3]
		Tích tất cả các nghiệm của phương trình $\ln ^2{x}+2\ln x-3=0$ bằng
		\choice
		{\True $\dfrac{1}{\mathrm{e^3}}$}
		{$-2$}
		{$-3$}
		{$\dfrac{1}{\mathrm{e^2}}$}
		\loigiai{
			Điều kiện $x>0$. Đặt $\ln x=t$. Phương trình đã cho trở thành $$t^2+2t-3=0\Leftrightarrow (t-1)(t+3)=0\Leftrightarrow\hoac{&t=1 \\&t=-3}.$$
			Ta có $x_1x_2=\mathrm{e}^{t_1}\mathrm{e}^{t_2}=\mathrm{e}^{t_1t_2}=\mathrm{e}^{-3}=\dfrac{1}{\mathrm{e}^3}$.
		}
	\end{vd}
\end{khung}
\subsection{Bài tập tương tự và phát triển}
\begin{ex}%[Triết Nguyễn Bạch Nha]%[2D2B5-3]
	Khi đặt $3^x=t$ thì phương trình $9^{x+1}-3^{x+1}-30=0$ trở thành
	\choice
	{\True $3t^2-t-10=0$}
	{$2t^2-t-1=0$}
	{$9t^2-3t-10=0$}
	{$t^2-t-10=0$}
	\loigiai{
		Ta có $$9^{x+1}-3^{x+1}-30=0\Leftrightarrow{9\cdot9^x}-3\cdot3^x-30=0\Leftrightarrow{3\cdot9^x}-3^x-10=0.$$
		Đặt $3^x=t$ thì phương trình $9^{x+1}-3^{x+1}-30=0$ trở thành $3t^2-t-10=0$.
	}
\end{ex}
\begin{ex}%[Triết Nguyễn Bạch Nha]%[2D2B5-3]
	Xét bất phương trình $5^{2x}-3\cdot5^{x+2}+32<0$. Nếu đặt $t=5^x$ thì bất phương trình trở thành bất phương trình nào sau đây?
	\choice
	{$t^2-16t+32<0$}
	{$t^2-6t+32<0$}
	{\True $t^2-75t+32<0$}
	{$t^2-3t+32<0$}
	\loigiai{
		$5^{2x}-3\cdot5^{x+2}+32<0\Leftrightarrow{5^{2x}}-3\cdot5^2\cdot 5^x+32<0\Leftrightarrow{5^{2x}}-75\cdot5^x+32<0$.\\
		Nếu đặt $t=5^x>0$ thì bất phương trình trở thành bất phương trình $t^2-75t+32<0$.
	}
\end{ex}
\begin{ex}%[Triết Nguyễn Bạch Nha]%[2D2B5-3]
	Phương trình $4^x-3\cdot2^x+2=0$ có nghiệm thuộc khoảng
	\choice
	{$\left(-1;0\right)$}
	{$\left(3;6\right)$}
	{\True $\left(\dfrac{1}{2};2\right)$}
	{$\left(2;4\right)$}
	\loigiai{
		$4^x-3\cdot2^x+2=0\Leftrightarrow\hoac{
			&2^x=1\\ 
			&2^x=2}\Leftrightarrow\hoac{
			&x=0\\ 
			&x=1\in\left(\dfrac{1}{2};2\right).}$
	}
\end{ex}
\begin{ex}%[Triết Nguyễn Bạch Nha]%[2D2B5-3]
	Phương trình $4^x-3\cdot2^x+2=0$ có nghiệm thuộc khoảng
	\choice
	{$\left(3;6\right)$}
	{\True $\left(\dfrac{1}{2};2\right)$}
	{$\left(2;4\right)$}
	{$\left(-1;0\right)$}
	\loigiai{
		$4^x-3.2^x+2=0\Leftrightarrow\hoac{
			&{2^x}=1\\ 
			&{2^x}=2\\}\Leftrightarrow\hoac{
			& x=0\\ 
			& x=1\in\left(\dfrac{1}{2};2\right)\\}$.
	}
\end{ex}
\begin{ex}%[Triết Nguyễn Bạch Nha]%[2D2B6-3]
	Xét bất phương trình $5^{2x}-3\cdot5^{x+2}+32<0$. Nếu đặt $t=5^x$ thì bất phương trình trở thành bất phương trình nào sau đây?
	\choice
	{$t^2-6t+32<0$}
	{\True $t^2-75t+32<0$}
	{$t^2-3t+32<0$}
	{$t^2-16t+32<0$}
	\loigiai{
		$5^{2x}-3\cdot5^{x+2}+32<0\Leftrightarrow{5^{2x}}-3\cdot5^2\cdot5^x+32<0\Leftrightarrow{5^{2x}}-75\cdot5^x+32<0$.\\
		Nếu đặt $t=5^x>0$ thì bất phương trình trở thành bất phương trình $t^2-75t+32<0$.
	}
\end{ex}
\begin{ex}%[Triết Nguyễn Bạch Nha]%[2D2B6-3]
	Tập nghiệm $S$ của bất phương trình $9^{x+\frac{1}{2}}-10\cdot3^x+3\le 0$.
	\choice
	{$S=(-\infty;-1]\cup[1;+\infty)$}
	{$S=\left\{-1;1\right\}$}
	{$S=\left(-1;1\right)$}
	{\True $S=\left[-1;1\right]$}
	\loigiai{
		Ta có $9^{x+\frac{1}{2}}-10\cdot3^x+3\le 0\Leftrightarrow 3\cdot(3^x)^2-10\cdot3^x+3\le 0\Leftrightarrow\dfrac{1}{3}\le{3^x}\le 3\Leftrightarrow-1\le x\le 1$.
	}
\end{ex}
\begin{ex}%[Triết Nguyễn Bạch Nha]%[2D2B5-3]
	Tìm tổng các nghiệm của phương trình $2^{2x+1}-5\cdot2^x+2=0$.
	\choice
	{$2$}
	{\True $0$}
	{$\dfrac{5}{2}$}
	{$1$}
	\loigiai{
		Ta có $2\cdot\left(2^x\right)^2-5\cdot\left(2^x\right)+2=0\Leftrightarrow\hoac{
			&2^x=2\\ 
			&2^x=\dfrac{1}{2}}\Leftrightarrow\hoac{
			& x=1\\ 
			& x=-1.}$
	}
\end{ex}
\begin{ex}%[2D2B5-3]
	Khi đặt $3^x=t$ thì phương trình $9^{x+1}-3^{x+1}-30=0$ trở thành
	\choice
	{$t^2-t-10=0$}
	{$2t^2-t-1=0$}
	{\True $3t^2-t-10=0$}
	{$9t^2-3t-10=0$}
	\loigiai{
		Ta có $9^{x+1}-3^{x+1}-30=0\Leftrightarrow 9\cdot\left(3^x\right)^2-3\cdot3^x-30=0$.\\
		Do đó khi đặt $ t=3^x$ ta có phương trình $\Leftrightarrow 9t^2-3t-30=0\Leftrightarrow 3t^2-t-10=0$.
	}
\end{ex}
\begin{ex}%[Triết Nguyễn Bạch Nha]%[2D2K6-3]
	Bất phương trình $9^x-4\cdot3^x+3\le 0$ có bao nhiêu nghiệm nguyên?
	\choice
	{$1$}
	{\True $2$}
	{$3$}
	{$0$}
	\loigiai{
		Ta có $9^x-4\cdot3^x+3\le 0\Leftrightarrow{\left(3^x\right)^2}-4\cdot3^x+3\le 0\Leftrightarrow 1\le{3^x}\le 3\Leftrightarrow 0\le x\le 1$.\\
		Với $x\in\mathbb{Z}\Rightarrow x\in\left\{ 0;1\right\}$.
	}
\end{ex}
\begin{ex}%[Triết Nguyễn Bạch Nha]%[2D2B6-3]
	Tập nghiệm của bất phương trình $4^x-6\cdot2^x+8<0$ là
	\choice
	{$\left(0;2\right)$}
	{$\left(-\infty;1\right)\cup\left(2;+\infty\right)$}
	{\True $\left(1;2\right)$}
	{$\left(2;4\right)$}
	\loigiai{
		Ta có $4^x-6\cdot2^x+8<0\Leftrightarrow 2<2^x<4\Leftrightarrow 1<x<2$.\\
		Vậy tập nghiệm của bất phương trình là $S=\left(1;2\right)$.
	}
\end{ex}
\begin{ex}%[Triết Nguyễn Bạch Nha]%[2D2B5-3]
	Tổng các nghiệm của phương trình $3^x-8\cdot3^{\tfrac{x}{2}}+15=0$ bằng
	\choice
	{\True $2\left(1+\log_35\right)$}
	{$4\log_53$}
	{$3\log_35$}
	{$2+\log_35$}
	\loigiai{
		Phương trình $3^x-8\cdot3^{\frac{x}{2}}+15=0$ $\Leftrightarrow\hoac{
			&{3^{\tfrac{x}{2}}}=5\\ 
			&{3^{\tfrac{x}{2}}}=3}\Leftrightarrow\hoac{
			&\dfrac{x}{2}=\log_35\\ 
			&\dfrac{x}{2}=1}\Leftrightarrow\hoac{
			&x=2\log_35\\ 
			&x=2.}$\\
		Suy ra tổng các nghiệm của phương trình là $2\left(1+\log_35\right)$.
	}
\end{ex}
\begin{ex}%[Triết Nguyễn Bạch Nha]%[2D2K5-5]
	Phương trình $2^{\sin^2x}+3^{\cos^2x}=4\cdot3^{\sin^2x}$ có bao nhiêu nghiệm thuộc $\left[-2017;2017\right]$.
	\choice
	{\True $1285$}
	{$4035$}
	{$1284$}
	{$4034$}
	\loigiai{
		Ta có $2^{\sin^2x}+3^{\cos^2}x=4.3^{\sin^2x}\Leftrightarrow{2^{\sin^2x}}+3^{1-\sin^2x}=4\cdot3^{\sin^2x}$.\\
		Đặt $\sin^2x=t$ với $t\in\left[0;1\right]$, ta có phương trình
		$$2^t+\dfrac{3}{3^t}=4\cdot3^t\Leftrightarrow{\left(\dfrac{2}{3}\right)^t}+3\cdot\left(\dfrac{1}{9}\right)^t=4.$$ 
		Vì hàm số $f(t)=\left(\dfrac{2}{3}\right)^t+3\cdot\left(\dfrac{1}{9}\right)^t$ nghịch biến với $t\in\left[0;1\right]$\\
		nên phương trình có nghiệm duy nhất $t=0$. \\
		Do đó $\sin x=0\Leftrightarrow x=k\pi$, $k\in\mathbb{Z}$.\\
		Vì $ x\in\left[-2017;2017\right]$ nên ta có $-2017\le k\pi\le\ 2017\Leftrightarrow\dfrac{-2017}{\pi}\le k\le\dfrac{2017}{\pi}$ nên có $1285$ giá trị nguyên của $k$ thỏa mãn. Vậy có $1285$ nghiệm.
	}
\end{ex}
\begin{ex}%[Triết Nguyễn Bạch Nha]%[2D2B6-3]
	Tập nghiệm của bất phương trình $4^x-3\cdot2^{x+1}+5\le 0$ là
	\choice
	{$\left[-\infty;{\log_2}5\right)$}
	{$\left[-1;{\log_2}5\right]$}
	{$\left[\log_25;+\infty\right)$}
	{\True $\left[0;{\log_2}5\right]$}
	\loigiai{
		Ta có $4^x-3\cdot2^{x+1}+5\le 0\Leftrightarrow{2^{2x}}-6\cdot2^x+5\le 0\Leftrightarrow 1\le{2^x}\le 5\Leftrightarrow 0\le x\le{\log_2}5$\\
		Vậy tập nghiệm của bất phương trình là $\left[0;{\log_2}5\right]$.
	}
\end{ex}
\begin{ex}%[Triết Nguyễn Bạch Nha]%[2D2B5-3]
	Tìm tập nghiệm $ S$ của phương trình $\mathrm{e}^{6x}-3\mathrm{e}^{3x}+2=0$.
	\choice
	{$S=\left\{1;\dfrac{\ln 2}{3}\right\}$}
	{$S=\left\{0;\ln 2\right\}$}
	{$S=\left\{1;\ln 2\right\}$}
	{\True $S=\left\{0;\dfrac{\ln 2}{3}\right\}$}
	\loigiai{
		Ta có $\mathrm{e}^{6x}-3\mathrm{e}^{3x}+2=0\Leftrightarrow\hoac{
			&{\mathrm{e}^{3x}}=1\\ 
			&{\mathrm{e}^{3x}}=2}\Leftrightarrow\hoac{
			&3x=0\\ 
			&3x=\ln 2}\Leftrightarrow\hoac{
			&x=0\\ 
			&x=\dfrac{\ln 2}{3}.}$\\
		Vậy phương trình có tập nghiệm là $S=\left\{ 0;\dfrac{\ln 2}{3}\right\}$.
	}
\end{ex}
\begin{ex}%[Triết Nguyễn Bạch Nha]%[2D2B6-3]
	Tập nghiệm của bất phương trình $\log_3^2x-2\log_3x^2+3<0$ là
	\choice
	{$\left(-\infty ;3\right)\cup\left(27;+\infty\right)$}
	{\True $\left(3;27\right)$}
	{$\left(0;3\right)\cup\left(27;+\infty\right)$}
	{$\left[3;27\right]$}
	\loigiai{
		Điều kiện $x>0.$\\
		Khi đó $\log^2_3x-2\log_3x^2+3<0\Leftrightarrow{\log^2}_3x-4\log_3x+3<0$\\
		Đặt $t=\log_3x$. Bất phương trình đã cho trở thành
		$$t^2-4t+3<0\Leftrightarrow 1<t<3\Leftrightarrow 1<\log_3x<3\Leftrightarrow 3<x<27.$$
		Kết hợp với điều kiện, nghiệm của bất phương trình là $S=$ $\left(3;27\right)$.
	}
\end{ex}
\begin{ex}%[Triết Nguyễn Bạch Nha]%[2D2B5-3]
	Tổng tất cả các nghiệm của phương trình $\log_2\left(10\cdot\left(\sqrt{2019}\right)^x-2019^x\right)=4$ bằng
	\choice
	{$\log_{2019}10$}
	{$2\log_{2019}10$}
	{$\log_{2019}16$}
	{\True $2\log_{2019}16$}
	\loigiai{
		Giải phương trình $\log_2\left(10\cdot\left(\sqrt{2019}\right)^x-2019^x\right)=4$\\
		Đặt $ t=\left(\sqrt{2019}\right)^x>0$.\\
		Phương trình đã cho trở thành $$ 10t-t^2=2^4\Leftrightarrow (t-2)(t-8)=0\Leftrightarrow\hoac{
			&t=2\\ 
			&t=8\\}\Leftrightarrow\hoac{
			&{2019^{\tfrac{x}{2}}}=2\\ 
			&{2019^{\tfrac{x}{2}}}=8}\Leftrightarrow\hoac{
			&x=2\log_{2019}2\\ 
			&x=2\log_{2019}8.}$$
		Tổng hai nghiệm là $2\log_{2019}2+2\log_{2019}8=2\log_{2019}16$.
	}
\end{ex}
\begin{ex}%[Triết Nguyễn Bạch Nha]%[2D2K6-3]
	Phương trình $\log_2^2x-8\sqrt{\log_2\left(8x\right)}-12=0$ có tất cả bao nhiêu nghiệm?
	\choice
	{\True $1$}
	{$3$}
	{$2$}
	{$0$}
	\loigiai{
		Điều kiện $x\ge\dfrac{1}{8}$.\\
		Phương trình tương đương với $\log_2^2x-8\sqrt{3+\log_2x}-12=0$\\
		Đặt $\sqrt{3+\log_2x}=t;t\ge 0$ ta có phương trình trở thành $$\left(t^2-3\right)^2-8t-12=0\Leftrightarrow{t^4}-6t^2-8t-3=0\Leftrightarrow{\left(t+1\right)^3}\cdot\left(t-3\right)=0 \Leftrightarrow\hoac{
			&t=-1\\ 
			&t=3.}$$
		So với điều kiện suy ra $ t=3\Rightarrow{\log_2}x=6\Leftrightarrow x=64$. \\
		Vậy phương trình có duy nhất $1$ nghiệm.
	}
\end{ex}
\begin{ex}%[Triết Nguyễn Bạch Nha]%[2D2K5-3]
	Cho $x>1$ và thỏa mãn $\log_3\left(\log_{27}x\right)=\log_{27}\left(\log_3x\right)$. Khi đó giá trị $\log_3x$ bằng
	\choice
	{$\dfrac{1}{3}$}
	{$3$}
	{\True $3\sqrt{3}$}
	{$27$}
	\loigiai{
		Phương trình đã cho tương đương $$\log_3\left(\log_{3^3}x\right)=\log_{3^3}\left(\log_3x\right)\Leftrightarrow\log_3\left(\dfrac{1}{3}{\log_3}x\right)=\dfrac{1}{3}{\log_3}\left(\log_3x\right)$$
		Đặt $\log_3x=t>0$ (vì $x>1$).\\
		Phương trình trở thành $$\log_3\left(\dfrac{1}{3}t\right)=\log_3\sqrt[3]{t}\Leftrightarrow\dfrac{1}{3}t=\sqrt[3]{t}\Leftrightarrow t^3=27t \Leftrightarrow t^3-27t=0\Leftrightarrow\hoac{
			&t=0\\
			&t=3\sqrt{3}\\
			&t=-3\sqrt{3}.}$$\\
		Vì $t>0$ nên ta chọn $ t=3\sqrt{3}$. Vậy $\log_3x=3\sqrt{3}$.
	}
\end{ex}
\begin{ex}%[Triết Nguyễn Bạch Nha]%[2D2B5-3]
	Biết phương trình $\log_2^2x-2\log_2\left(2x\right)-1=0$ có hai nghiệm $x_1$; ${x_2}$. Tính $x_1x_2$.
	\choice
	{$x_1x_2=\dfrac{1}{2}$}
	{$x_1x_2=-3$}
	{\True $x_1x_2=4$}
	{$x_1x_2=\dfrac{1}{8}$}
	\loigiai{
		Điều kiện $ x>0$.\\
		Với mọi $ x>0$ ta có $\begin{aligned}[t]
			\log_2^2x-2\log_2\left(2x\right)-1=0&\Leftrightarrow\log_2^2x-2\left(1+\log_2x\right)-1=0\\
			&\Leftrightarrow\log_2^2x-2\log_2x-3=0 \\
			&\Leftrightarrow\hoac{
				&{\log_2}x=-1\\ 
				&{\log_2}x=3\\} \\
			&\Leftrightarrow\hoac{
				&x=2^{-1}=\dfrac{1}{2}\\ 
				&x=2^3=8.}
		\end{aligned}$\\
		Vậy phương trình có $2$ nghiệm $x_1=\dfrac{1}{2}$; ${x_2}=8$.\\
		Do đó $x_1x_2=\dfrac{1}{2}\cdot8=4$.
	}
\end{ex}
\begin{ex}%[Triết Nguyễn Bạch Nha]%[2D2K6-3]
	Tìm tập nghiệm $S$ của bất phương trình: $\log_2^2\left(x-1\right)-4\log_2\left(x-1\right)+3\ge 0$
	\choice
	{$S=[{3};9]$}
	{\True $S=(1;3]\cup[{9};+\infty)$}
	{$S=(-\infty ;1]\cup[{3};+\infty)$}
	{$S=(-\infty ;3]\cup[{9};+\infty)$}
	\loigiai{
		Ta có $\log_2^2\left(x-1\right)-4\log_2\left(x-1\right)+3\ge 0$ điều kiện $x>1.$\\
		Đặt $t=\log_2(x-1)$, bất phương trình trở thành\\
		$$t^2-4t+3\ge 0\Leftrightarrow\hoac{
			&t\ge 3\\ 
			&t\le 1\\}\Leftrightarrow\hoac{
			&{\log_2}\left(x-1\right)\ge 3\\ 
			&{\log_2}\left(x-1\right)\le 1}\Leftrightarrow\hoac{
			&x-1\ge 8\\ 
			&x-1\le 2}\Leftrightarrow\hoac{
			&x\ge 9\\ 
			&x\le 3.}$$
		Kết hợp điều kiện ta có tập nghiệm của bất phương trình là $S=(1;3]\cup[{9};+\infty)$.
	}
\end{ex}          
\Closesolutionfile{ans}
%======================
\subsection{Bảng đáp án}
\inputansbox{8}{ans/ANS-DANG-34}

