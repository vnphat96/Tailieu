\section{Ứng dụng tích phân tính thể tích vật thể tròn xoay}
\subsection{Kiến thức cần nhớ}
\begin{khung}
	\begin{itemize}
			\item Thể tích khối tròn xoay được sinh ra khi quay hình phẳng giới hạn bởi các đường $y=f(x)$, trục hoành và hai đường thẳng $x=a$, $x=b$ quanh trục $Ox$
			\begin{center}
				\begin{tikzpicture}[line join=round, line cap = round, >=stealth, scale=1,font=\footnotesize,transform shape]
					\begin{scope}[scale=0.7]
						\draw[->] (-.5,0)--(5,0) node[above]{$x$};
						\draw[->] (0,-2.5)--(0,2.5) node[above left]{$y$};
						\draw[pattern = north west lines] 
						(1,0)--(1,2) .. controls (3,1) .. (4,1.5)--(4,0)--(1,0)
						;
						\draw (1,-2) .. controls (3,-1) .. (4,-1.5);
						\draw[dashed]
						(1,0)--(1,-2) (4,0)--(4,-1.5) 
						(1.4,0) arc(0:360:0.4 and 2) (4.3,0) arc(0:360:0.3 and 1.5)
						;
						\foreach \x/\y/\z/\g in {0/0/O/135,1/0/a/-45,4/0/b/-45}
						\fill[black] (\x,\y) circle(1pt) ($(\x,\y)+(\g:2mm)$) node{$\z$};
						\draw[->] (4.5,-0.3) arc(-90:45:0.1 and 0.3);
						\draw (4.5,0.3) arc(90:270:0.1 and .3);
						\draw (3,1.7) node{$y=g(x)$};
					\end{scope}
					\draw (6.5,0) node{$\heva{&(C)\colon y=f(x)\\ &(Ox)\colon y=0\\ &x=a\\ &x=b}$};
					\draw (9.5,0) node[rectangle]{\fbox{$V=\pi\displaystyle\int\limits_{a}^{b}\left[ f(x) \right]^2\mathrm{\,d}x$}};
				\end{tikzpicture}
			\end{center}
			\item Thể tích khối tròn xoay được sinh ra khi quay hình phẳng giới hạn bởi các đường $y=f(x)$, $y=g(x)$ (cùng nằm một phía so với $Ox$) và hai đường thẳng $x=a$, $x=b$ quanh trục $Ox$:	
			\begin{center}
				\begin{tikzpicture}[line join=round, line cap = round, >=stealth, scale=1,font=\footnotesize,transform shape]
					\draw[->] (-.5,0)--(4.25,0) node[above]{$x$};
					\draw[->] (0,-.5)--(0,3) node[left]{$y$};
					\draw[pattern = north west lines]
					(.5,2.2) .. controls +(40:.7) and +(135:.8) .. (3,1.75)
					--(3,.7) .. controls +(200:.5) and +(-20:.5) .. (.5,.5)
					--(.5,2.2)
					;
					\draw
					(.5,.5) -- (.25,.6)
					(.5,2.2) .. controls +(-140:.2) and +(90:.0) .. (.25,1.85)
					(3,.7) -- (3.25,.8) node[right]{$g(x)$}
					(3,1.75) .. controls +(-45:.2) and +(135:.1) .. (3.25,1.5) node[right]{$f(x)$}
					(3,0)--(3,.7) (.5,0)--(.5,.5)
					;
					\draw[->] (3.5,-0.3) arc(-90:45:0.1 and 0.3);
					\draw (3.5,0.3) arc(90:270:0.1 and .3);
					\draw (7,1.5) node{\fbox{$V=\pi\displaystyle\int\limits_{a}^{b}\left| f^2(x) - g^2(x) \right|\mathrm{\,d}x$}};
					\foreach \x/\y/\z/\g in {0/0/O/135,0.5/0/a/-90,3/0/b/-90}
					\fill[black] (\x,\y) circle(1pt) ($(\x,\y)+(\g:3mm)$) node{$\z$};
				\end{tikzpicture}
			\end{center}
		\end{itemize}
\end{khung}
\subsection{Bài tập mẫu}
\Opensolutionfile{ans}[ans/ANS-DANG-29]
\begin{khung}
\setcounter{vd}{28}
\begin{vd}[Đề tham khảo BGD 2022-2023]%[2D3B3-3]
	Thể tích khối tròn xoay thu được khi quay hình phẳng giới hạn bởi hai đường $y=-x^2+2x$ và $y=0$ quanh trục $Ox$ bằng
	\choice
	{$\dfrac{16}{15}$}
	{$\dfrac{16 \pi}{9}$}
	{$\dfrac{16}{9}$}
	{\True $\dfrac{16 \pi}{15}$}
	\loigiai{
		Hoành độ giao điểm của đồ thị và trục $Ox$ là nghiệm của phương trình $$-x^2+2x=0 \Leftrightarrow \hoac{&x=0\\&x=2.}$$
		Khi đó $V=\pi \displaystyle\int\limits_0^2 f^2(x)\mathrm{\,d}x=\pi \displaystyle\int\limits_0^2 \left(x^4-4x^3+4x^2\right)\mathrm{\,d}x=\pi\left(\dfrac{x^5}{5}-x^4+\dfrac{4}{3}x^3\right)\Bigg|_0^2=\dfrac{16\pi}{15}$.
	}
\end{vd}
\end{khung}


\subsection{Bài tập tương tự và phát triển}
\begin{ex}%[Lê Quân, ĐMH-LVĐ]%[2D3B3-3]
Cho hình phẳng $\mathscr{D}$ giới hạn bởi đường cong $y=\mathrm{e}^x$, trục hoành và các đường thẳng $x=0$, $x=1$. Khối tròn xoay tạo thành khi quay $\mathscr{D}$ quanh trục hoành có thể tích $V$ bằng bao nhiêu?
\choice
{$V=\dfrac{\pi\left(\mathrm{e}^2+1\right)}{2}$}
{\True $V=\dfrac{\pi\left(\mathrm{e}^2-1\right)}{2}$}
{$\dfrac{\pi{\mathrm{e}^2}}{2}$}
{$V=\dfrac{\mathrm{e}^2-1}{2}$}
\loigiai{
Thể tích khối tròn xoay cần tính là $V=\pi\displaystyle\int\limits_0^1\left(\mathrm{e}^x\right)^2\mathrm{d}x=\pi\left.\left(\dfrac{\mathrm{e}^{2x}}{2}\right)\right|_0^1=\dfrac{\pi\left(\mathrm{e}^2-1\right)}{2}$.}
\end{ex}

\begin{ex}%[Lê Quân, ĐMH-LVĐ]%[2D3Y3-3]
Cho hình phẳng $\mathscr{D}$ giới hạn bởi các đường $y=\sqrt{2019x+2020}$, trục $Ox$ và hai đường thẳng $x=0$; $x=1$. Gọi  $V$ là thể tích của khối tròn xoay được tạo thành khi quay $\mathscr{D}$ quanh trục $Ox$. Khẳng định nào sau đây đúng?
\choice
{$V=\displaystyle\int\limits_0^1\left(2019x+2020\right){\mathrm{d}}x$}
{\True $V=\pi\displaystyle\int\limits_0^1\left(2019x+2020\right){\mathrm{d}}x$}
{$V=\displaystyle\int\limits_0^1\sqrt{2019x+2020}\,\mathrm{d}x$}
{$V=\pi\displaystyle\int\limits_0^1\sqrt{2019x+2020}\,\mathrm{d}x$}
\loigiai{
Thể tích $V$ của khối tròn xoay được tạo thành khi quay $\mathscr{D}$ quanh trục $Ox$ được tính bởi công thức\\
 \[V=\pi\displaystyle\int\limits_0^1\left(\sqrt{2019x+2020}\right)^2\mathrm{d}x=\pi\displaystyle\int\limits_0^1\left(2019x+2020\right)\,\mathrm{d}x.\]}
\end{ex}

\begin{ex}%[Lê Quân, ĐMH-LVĐ]%[2D3Y3-3]
Cho hình phẳng $\mathscr{H}$ giới hạn bởi đồ thị hàm số $y=x\cdot\ln x$, trục hoành và hai đường thẳng $x=1$; $x=2$. Thể tích vật thể tròn xoay sinh bới $\mathscr{H}$ khi nó quay quanh trục hoành có thể tích $V$ được xác định bởi
\choice
{$V=\pi\displaystyle\int\limits_1^2\left(x\cdot\ln x\right){\mathrm{d}}x$}
{$V=\displaystyle\int\limits_1^2\left(x\cdot\ln x\right){\mathrm{d}}x$}
{$V=\displaystyle\int\limits_1^2\left(x\cdot\ln x\right)^2\mathrm{d}x$}
{\True $V=\pi\displaystyle\int\limits_1^2\left(x\cdot\ln x\right)^2\mathrm{d}x$}
\loigiai{
Thể tích vật thể tròn xoay sinh bởi $(H):\heva{&
y=x\cdot\ln x\\
&y=0\\
&x=1;x=2}$ khi  quay quanh trục hoành được tính bởi công thức  $V=\pi\displaystyle\int\limits_1^2\left(x\cdot\ln x\right)^2\mathrm{d}x$.}
\end{ex}

\begin{ex}%[Lê Quân, ĐMH-LVĐ]%[2D3Y3-3]
Gọi $\mathscr{D}$ là hình phẳng giới hạn bởi các đường $y=\dfrac{x}{4}$, $y=0$, $x=1$, $x=4$. Tính thể tích vật thể tròn xoay tạo thành khi quay hình $\mathscr{D}$ quanh trục $Ox$.
\choice
{$\dfrac{15\pi}{8}$}
{\True $\dfrac{21\pi}{16}$}
{$\dfrac{21}{16}$}
{$\dfrac{15}{16}$}
\loigiai{
Thể tích vật thể tròn xoay tạo thành khi quay hình $\mathscr{D}$ quanh trục $Ox$ là
\[V=\pi\displaystyle\int\limits_1^4\left(\dfrac{x}{4}\right)^2\mathrm{\,d}x =\left.\dfrac{\pi{x^3}}{48}\right|_1^4 =\dfrac{21\pi}{16}.\]}
\end{ex}

\begin{ex}%[Lê Quân, ĐMH-LVĐ]%[2D3Y3-3]
Cho hình phẳng $\mathscr{H}$ giới hạn bởi các đường $y=x^2+3$,$y=0$, $x=1$, $x=3$. Gọi $V$ là thể tích của khối tròn xoay được tạo thành khi quay $\mathscr{H}$ xung quanh trục $Ox$. Mệnh đề nào sau đây đúng?
\choice
{$V=\displaystyle\int\limits_1^3\left(x^2+3\right){\mathop{\rm d}\nolimits}x$}
{$V=\pi\displaystyle\int\limits_1^3\left(x^2+3\right){\mathrm{d}}x$}
{\True $V=\pi\displaystyle\int\limits_1^3\left(x^2+3\right)^2\mathrm{d}x$}
{$V=\displaystyle\int\limits_1^3\left(x^2+3\right)^2\mathrm{d}x$}
\loigiai{
Thể tích của khối tròn xoay được tạo thành khi quay $\mathscr{H}$ xung quanh trục $Ox$ là
\[V=\pi\displaystyle\int\limits_1^3\left(x^2+3\right)^2\mathrm{\,d}x.\]}
\end{ex}

\begin{ex}%[Lê Quân, ĐMH-LVĐ]%[2D3B3-3]
Cho hình phẳng $\mathscr{D}$ được giới hạn bởi các đường $f(x)=\sqrt{2x+1}, Ox, x=0, x=1$. Gọi  $V$ là thể tích của khối tròn xoay tạo thành khi quay $\mathscr{D}$ xung quanh trục $Ox$. Khẳng định nào sau đây đúng? 
\choice
{$V=\pi\displaystyle\int\limits_0^1\sqrt{2x+1}\mathrm{\,d}x$}
{$V=\displaystyle\int\limits_0^1\left(2x+1\right)\mathrm{\,d}x$}
{\True $V=\pi\displaystyle\int\limits_0^1\left(2x+1\right)\mathrm{\,d}x$}
{$V=\displaystyle\int\limits_0^1\sqrt{2x+1}\mathrm{\,d}x$}
\loigiai{
Ta có $V=\pi\displaystyle\int\limits_0^1\left(\sqrt{2x+1}\right)^2\mathrm{dx}=\pi\displaystyle\int\limits_0^1\left(2x+1\right)\mathrm{\,d}x$.}
\end{ex}

\begin{ex}%[Lê Quân, ĐMH-LVĐ]%[2D3Y3-3]
Thể tích khối tròn xoay tạo thành khi quay hình phẳng giới hạn bởi các đường $y=x\cdot\mathrm{e}^x$, $y=0$, $x=0$, $x=1$ xung quanh trục $Ox$ là
\choice
{$V=\pi\displaystyle\int\limits_0^1 x\mathrm{e}^{x}\mathrm{\, d}x$}
{\True $V=\pi\displaystyle\int\limits_0^1x^2\mathrm{e}^{2x}\mathrm{\, d}x$}
{$V=\pi\displaystyle\int\limits_0^1x^2\mathrm{e}^x\mathrm{\, d}x$}
{$V=\displaystyle\int\limits_0^1x^2\mathrm{e}^{2x}\mathrm{\, d}x$}
\loigiai{
Ta có $V=\pi\displaystyle\int\limits_0^1 \left(x\mathrm{e}^{x}\right)^2 \mathrm{\, d}x =\pi\displaystyle\int\limits_0^1  x^2\mathrm{e}^{2x}\mathrm{\, d}x.$}
\end{ex}

\begin{ex}%[Lê Quân, ĐMH-LVĐ]%[2D3B3-3]
Thể tích khối tròn xoay sinh ra khi quay quanh trục hoành hình phẳng giới hạn bởi đồ thị hàm số $y=\mathrm{e}^{\frac{x}{2}}$, trục hoành, trục tung và đường thẳng $x=2$ bằng
\choice
{$\pi{\mathrm{e}^2}$}
{$\mathrm{e}^2-1$}
{\True $\pi\left(\mathrm{e}^2-1\right)$}
{$\pi\left(\mathrm{e}-1\right)$}
\loigiai{ 
Ta có $V=\pi\displaystyle\int\limits_0^2\mathrm{e}^x \mathrm{\,d}x=\pi \cdot \mathrm{e}^x\big|_0^2=\pi\left(\mathrm{e}^2-1\right)$.}
\end{ex}

\begin{ex}%[Lê Quân, ĐMH-LVĐ]%[2D3Y3-3]
Cho hình phẳng $\mathscr{H}$ được giới hạn bởi các đường $x=0$, $x=\pi $, $y=0$ và $y=-\cos x$. Gọi  $V$ là thể tích của khối tròn xoay tạo thành khi quay $\mathscr{H}$ xung quanh trục $Ox$. Khẳng định nào sau đây đúng?
\choice
{\True $V=\pi\displaystyle\int\limits_0^\pi \cos^2x\mathrm{\,d}x$}
{$V=\pi\left|\displaystyle\int\limits_0^\pi\left(-\cos x\right)\mathrm{\,d}x\right|$}
{$V=\pi\displaystyle\int\limits_0^\pi\left|\cos x\right|\mathrm{\,d}x$}
{$V=\displaystyle\int\limits_0^\pi\cos^2x\mathrm{\,d}x$}
\loigiai{ 
Ta có thể tích $V$ của khối tròn xoay tạo thành khi quay $\mathscr{H}$ xung quanh trục $Ox$ được tính theo công thức 
\[V=\pi\displaystyle\int\limits_0^\pi\left(-\cos x\right)^2\mathrm{\,d}x=\pi\displaystyle\int\limits_0^\pi\cos^2x\mathrm{\,d}x.\]}
\end{ex}

\begin{ex}%[Lê Quân, ĐMH-LVĐ]%[2D3Y3-3]
Cho hình phẳng $\mathscr{H}$ giới hạn bởi các đường $y=x^3-x+1$ , $y=0$, $x=0$, $x=2$. Gọi $V$ là thể tích khối tròn xoay được tạo thành khi quay $\mathscr{H}$ xung quanh trục $Ox$. Mệnh đề nào sau đây đúng?
\choice
{$V=\pi\displaystyle\int\limits_0^2\left(x^3-x^2+1\right)\mathrm{\,d}x$}
{$V=\pi\displaystyle\int\limits_0^2\left(x^3-x+1\right)\mathrm{\,d}x$}
{$V=\displaystyle\int\limits_0^2\left(x^3-x+1\right)^2\mathrm{\,d}x$}
{\True $V=\pi\displaystyle\int\limits_0^2\left(x^3-x+1\right)^2\mathrm{\,d}x$}
\loigiai{ 
Thể tích khối tròn xoay được tạo thành khi quay $\mathscr{H}$ xung quanh trục $Ox$ là
\[V=\pi\displaystyle\int\limits_0^2\left(x^3-x+1\right)^2\mathrm{d}x.\]}
\end{ex}

\begin{ex}%[Lê Quân, ĐMH-LVĐ]%[2D3B3-3]
Cho hình phẳng $\mathscr{H}$ giới hạn bởi đồ thị hàm số $y=\dfrac{1}{x}$ và các đường thẳng $y=0$, $x=1$, $x=4$. Thể tích $V$ của khối tròn xoay sinh ra khi cho hình phẳng $\mathscr{H}$ quay quanh trục $Ox$ bằng
\choice
{$\dfrac{3}{4}$}
{$2\ln 2$}
{$2\pi\ln 2$}
{\True $\dfrac{3\pi}{4}$}
\loigiai{ 
Thể tích $V$ của khối tròn xoay sinh ra khi cho hình phẳng $\mathscr{H}$ quay quanh trục $Ox$ là
\[V=\pi\displaystyle\int\limits_1^4\left(\dfrac{1}{x}\right)^2\mathrm{\,d}x =\left.\pi\left(-\dfrac{1}{x}\right)\right|_1^4 =\pi\left(-\dfrac{1}{4}+1\right)=\dfrac{3\pi}{4}.\]}
\end{ex}

\begin{ex}%[Lê Quân, ĐMH-LVĐ]%[2D3Y3-3]
Cho hình phẳng $\mathscr{D}$ được giới hạn bởi các đường $x=0$, $x=1$, $y=0$ và $y=\sqrt{2x+1}$. Gọi $V$ là thể tích của khối tròn xoay tạo thành khi quay $\mathscr{D}$ xung quanh trục $Ox$. Mệnh đề nào sau đây đúng?
\choice
{\True $V=\pi\displaystyle\int\limits_0^1\left(2x+1\right)\mathrm{\,d}x$}
{$V=\displaystyle\int\limits_0^1\left(2x+1\right)\mathrm{\,d}x$}
{$V=\displaystyle\int\limits_0^1\sqrt{2x+1}\mathrm{\,d}x$}
{$V=\pi\displaystyle\int\limits_0^1\sqrt{2x+1}\mathrm{\,d}x$}
\loigiai{ 
Ta có $V=\pi\displaystyle\int\limits_0^1\left(\sqrt{2x+1}\right)^2\mathrm{d}x$ $=\pi\displaystyle\int\limits_0^1\left(2x+1\right){\mathrm{d}}x$ .}
\end{ex}

 

 

\begin{ex}%[Lê Quân, ĐMH-LVĐ]%[2D3B3-3]
Cho hình phẳng $\mathscr{H}$ được giới hạn bởi đồ thị hàm số $y=\sqrt x $ và các đường thẳng $x=0$; $x=1$ và trục hoành. Tính thể tích $V$ của khối tròn xoay sinh bởi hình $\mathscr{H}$ quay xung quanh trục $Ox$.
\choice
{\True $\dfrac{\pi}{2}$}
{$\sqrt\pi $}
{$\dfrac{\pi}{3}$}
{$\pi $}
\loigiai{ 
Thể tích $V$ của khối tròn xoay cần tìm là $V=\pi\displaystyle\int\limits_0^1x{\mathrm{d}}x$ $=\left.\dfrac{\pi{x^2}}{2}\right|_0^1=\dfrac{\pi}{2}$ (đvtt).}
\end{ex}

\begin{ex}%[Lê Quân, ĐMH-LVĐ]%[2D3B3-3]
Goi $\mathscr{H}$ là hình phẳng giới hạn bởi đồ thị hàm số $y=\mathrm{e}^x$, trục $Ox$ và hai đường thẳng $x=0,$ $x=1$. Thể tích của khối tròn xoay tạo thành khi quay $\mathscr{H}$ xung quanh trục $Ox$ bằng
\choice
{$\dfrac{\pi}{2}\left(\mathrm{e}^2+1\right)$}
{$\pi\left(\mathrm{e}^2-1\right)$}
{\True $\dfrac{\pi}{2}\left(\mathrm{e}^2-1\right)$}
{$\pi\left(\mathrm{e}^2+1\right)$}
\loigiai{ 
Thể tích khối tròn xoay $V=\pi\displaystyle\int\limits_0^1\mathrm{e}^{2x}\mathrm{\,d}x=\left.\dfrac{\pi}{2}{\mathrm{e}^{2x}}\right|_0^1=\dfrac{\pi}{2}\left(\mathrm{e}^2-1\right).$}
\end{ex}

 
\begin{ex}%[Lê Quân, ĐMH-LVĐ]%[2D3B3-3]
Cho hình phẳng $\mathscr{D}$ giới hạn bởi đồ thị hàm số $y=\sin x$, trục hoành và hai đường thẳng $x=0$; $x=\pi$. Thể tích khối tròn xoay thu được khi quay $\mathscr{D}$ quanh trục $Ox$ bằng
\choice
{\True $\dfrac{\pi^2}{2}$}
{$\dfrac{\pi^2}{4}$}
{$\dfrac{\pi}{4}$}
{$\dfrac{\pi}{2}$}
\loigiai{ 
Thể tích khối tròn xoay thu được khi quay $\mathscr{D}$ quanh trục $Ox$ là
\[V=\pi\displaystyle\int_0^\pi\left(\sin x\right)^2 \mathrm{\,d}x=\pi\displaystyle\int_0^\pi\dfrac{1-\cos 2x}{2}\mathrm{\,d}x=\dfrac{\pi}{2}\left(x-\dfrac{1}{2}\sin 2x\right)\Bigg|_0^\pi=\dfrac{\pi^2}{2}.\]}
\end{ex}

\begin{ex}%[Lê Quân, ĐMH-LVĐ]%[2D3B3-3]
Tính thể tích khối tròn xoay được tạo thành khi quay hình phẳng giới hạn bởi đồ thị hàm số $y=3x-x^2$ và trục hoành, quanh trục hoành.
\choice
{$\dfrac{41\pi}{7}$}
{$\dfrac{8\pi}{7}$}
{\True $\dfrac{81\pi}{10}$}
{$\dfrac{85\pi}{10}$}
\loigiai{ 
Ta có $3x-x^2=0\Leftrightarrow \hoac{&x=0	\\&x=3.}$\\
Thể tích khối tròn xoay cần tìm là
\[V=\pi\displaystyle\int\limits_0^3\left(3x-x^2\right)^2\mathrm{\,d}x =\pi\displaystyle\int\limits_0^3\left(9x^2-6x^3+x^4\right)\mathrm{\,d}x=\pi\left.\left(3x^3-\dfrac{3x^4}{2}+\dfrac{x^5}{5}\right)\right|_0^3=\dfrac{81\pi}{10}.\]}
\end{ex}

 
 

 
 


 

\begin{ex}%[Lê Quân, ĐMH-LVĐ]%[2D3B3-3]
Tính thể tích vật thể tròn xoay tạo thành khi cho hình phẳng giới hạn bởi các đường parabol $y=x^2$, trục hoành và đường thẳng $x=1$ quay xung quanh trục $Ox$.
\choice
{$\dfrac{1}{5}$}
{$\dfrac{1}{3}$}
{\True $\dfrac{\pi}{5}$}
{$\dfrac{\pi}{3}$}
\loigiai{ 
Phương trình hoành độ giao điểm  $x^2=0\Leftrightarrow x=0$.\\
Thể tích vật thể tròn xoay cần tìm là \[V=\pi\displaystyle\int\limits_0^1\left(x^2\right)^2\mathrm{\,d}x=\pi\displaystyle\int\limits_0^1x^4\mathrm{\,d}x=\pi\left.\dfrac{x^5}{5}\right|_0^1=\dfrac{\pi}{5}.\]}
\end{ex}

\begin{ex}%[Lê Quân, ĐMH-LVĐ]%[2D3B3-3]
Cho hình phẳng $\mathscr{D}$ giới hạn bởi đường cong $y=\sqrt{2+\cos x},$ trục hoành và các đường thẳng $x=0,x=\dfrac{\pi}{2}$. Khối tròn xoay tạo thành khi cho $\mathscr{D}$ quay quanh trục hoành có thể tích $V$ bằng bao nhiêu?
\choice
{$V=(\pi-1)\pi $}
{\True $V=(\pi+1)\pi $}
{$V=\pi-1$}
{$V=\pi+1$}
\loigiai{
Ta có 
\[V=\pi\displaystyle\int\limits_0^{\frac{\pi}{2}}\left(\sqrt{2+\cos x}\right)^2\mathrm{\,d}x = \pi\left(2x+\sin x\right)\Bigg|_0^{\tfrac{\pi}{2}}=\pi(\pi+1).\]}
\end{ex}

 

\begin{ex}%[Lê Quân, ĐMH-LVĐ]%[2D3B3-3]
Kí hiệu $\mathscr{H}$ là hình phẳng giới hạn bởi đồ thị hàm số $y=2x-x^2$ và $y=0$. Tính thể tích vật thể tròn xoay được sinh ra bởi hình phẳng $\mathscr{H}$ khi  quay quanh trục $Ox$.
\choice
{$\dfrac{19\pi}{15}$}
{$\dfrac{17\pi}{15}$}
{$\dfrac{18\pi}{15}$}
{\True $\dfrac{16\pi}{15}$}
\loigiai{ 
Xét phương trình $2x-x^2=0 \Leftrightarrow\hoac{&x=0\\ &x=2.}$\\
Thể tích của vật thể cần tính là \[V=\pi\displaystyle\int\limits_0^2\left(2x-x^2\right)^2\mathrm{\,d}x=\dfrac{16\pi}{15}.\]}
\end{ex}
\begin{ex}%[Lê Quân, ĐMH-LVĐ]%[2D3B3-3]
Cho hình phẳng $\mathscr{H}$ giới hạn bởi các đường $y=\cos x$, $y=0$, $x=0$, $x=\dfrac{\pi}{4}$. Thể tích của khối tròn xoay được tạo thành khi quay $\mathscr{H}$ xung quanh trục $Ox$ bằng
\choice
{$\dfrac{\pi+2}{8}$}
{\True $\dfrac{\pi (\pi+2)}{8}$}
{$\dfrac{\pi ^2+1}{4}$}
{$\dfrac{\pi (\pi+2)}{4}$}
\loigiai{ 
Ta có
\[V=\pi\displaystyle\int\limits_0^{\tfrac{\pi}{4}}\cos^2x\mathrm{\,d}x=\dfrac{\pi}{2}\displaystyle\int\limits_0^{\tfrac{\pi}{4}}(1+\cos 2x)\mathrm{\,d}x=\dfrac{\pi}{2}\left(x+\dfrac{\sin 2x}{2}\right)\Bigg|_0^{\tfrac{\pi}{4}}=\dfrac{\pi}{2}\left(\dfrac{\pi}{4}+\dfrac{1}{2}\right)=\dfrac{\pi (\pi+2)}{8}.\]
}
\end{ex}

\Closesolutionfile{ans}
%======================
\subsection{Bảng đáp án}
\inputansbox{8}{ans/ANS-DANG-29}