%Dạng 1
\setcounter{ex}{0}
\section{Phương trình với hệ số phức}
\subsection{Kiến thức cần nhớ}
\begin{khung}
	\subsubsection{Cách giải}
	\begin{itemize}
		\item Xét phương trình bậc hai $az^2+bz+c=0$, với $z\in\mathbb{C}$; $a,b,c\in\mathbb{R}$ và $a\ne 0$.\\
		Xét biệt thức $\Delta=b^2-4ac$.
		\begin{itemize}
		\item Nếu $\Delta\ne 0$ thì phương trình có hai nghiệm phân biệt $z_1=\dfrac{-b+\delta}{2a}$ và $z_2=\dfrac{-b-\delta}{2a}$, trong đó $\delta $ là một căn bậc hai của $\Delta $.
		\item Nếu $\Delta=0$ thì phương trình có nghiệm kép $z_1=z_2=-\dfrac{b}{2a}$.
		\end{itemize}
	\end{itemize}
	\subsubsection{Đặc biệt}
	\begin{itemize}
		\item Khi $\Delta $ là số thực dương thì phương trình có hai nghiệm $z_1=\dfrac{-b+\sqrt{\Delta}}{2a}$ và $z_2=\dfrac{-b-\sqrt{\Delta}}{2a}$.
		\item Khi $\Delta $ là số thực âm thì phương trình có hai nghiệm $z_1=\dfrac{-b+i\sqrt{-\Delta}}{2a}$ và $z_2=\dfrac{-b-i\sqrt{-\Delta}}{2a}$.\\
	\end{itemize}
	\subsubsection{Nhận xét}
	\begin{itemize}
		\item Trên tập hợp số phức, mọi phương trình bậc 2 đều có 2 nghiệm (không nhất thiết phân biệt).
		\item Định lý Vi-et: Phương tình bậc hai $a{z^2}+bz+c$, với $z\in\mathbb{C}$; $a,b,c\in\mathbb{R}$ và $a\ne 0$ có 2 nghiệm phức $z_1$ và $z_2$ thì: 
		$$\heva{
			&{z_1}+z_2=\dfrac{-b}{a}\\ 
			&{z_1}{z_2}=\dfrac{c}{a}.\\ 
		}$$
	\end{itemize}
\end{khung}
\subsection{Bài tập mẫu}
\Opensolutionfile{ans}[ans/ANS-DANG-45]
\begin{khung}
	\begin{vd}%[2D4K4-2]
		Trên tập hợp số phức, xét phương trình $z^{2}-2(m+1) z+m^{2}=0$ ($m$ là tham số thực). Có bao nhiêu giá trị của $m$ để phương trình đó có hai nghiệm phân biệt $z_{1}, z_{2}$ thỏa mãn $\left|z_{1}\right|+\left|z_{2}\right|=2$ ?
		\choice
		{$1$}
		{$4$}
		{$2$}
		{$3$}
		\loigiai{
	Ta có $\Delta'=2m+2$
	\begin{itemize}
	\item \textbf{Trường hợp 1}: $\Delta'<0\Leftrightarrow m<-1$.\\
	Phương trình có hai nghiệm phức, khi đó $\left|z_1\right|=\left|z_2\right|=\sqrt{\dfrac{c}{a}}=\sqrt{m^2}$.\\
	Suy ra $2\sqrt{m^2}=2\Leftrightarrow\hoac{&m=1\,(\text{nhận})\\&m=-1 \,(\text{loại}).}$
	\item \textbf{Trường hợp 2}: $\Delta'>0\Leftrightarrow m>-1$.\\
	Vì $a\cdot c=m^2\ge 0$ nên phương trình có hai nghiệm phân biệt $z_1\cdot z_2\ge 0$ hoặc $z_1\cdot z_2\le 0$.\\
	Suy ra 
	$$\left|z_1\right|+\left|z_2\right|=2\Leftrightarrow\left|z_1+z_2\right|=2\Leftrightarrow\left|2m+2\right|=2\Leftrightarrow\hoac{&m=-2\,(\text{loại})\\&m=0\,(\text{nhận}).}$$
	Vậy có $2$ giá trị của $m$ thỏa yêu cầu bài toán.
	\end{itemize}	
	}
	\end{vd}
\end{khung}
\subsection{Bài tập tương tự và phát triển}
\begin{ex}%Câu 1%[2D4K4-2]
	Gọi $z_1$, $z_2$ là các nghiệm phức của phương trình $z^2+4z+5=0$. Đặt $w=\left(1+z_1\right)^{100}+\left(1+z_2\right)^{100}$. Khi đó
	\choice
	{$w=-2^{51}i$}
	{$w=2^{51}i$}
	{$w=2^{51}$}
	{\True $w=-2^{51}$}
	\loigiai{
		Ta có $z^2+4z+5=0\Leftrightarrow z=-2\pm i$.\\
		$\left(1+z_1\right)^{100}=\left(1-2+i\right)^{100}=\left[\left(-1+i\right)^2\right]^{50}=\left(-2i\right)^{50}=2^{50}\cdot{\left(-1\right)^{25}}=-2^{50}$.\\
		$\left(1+z_2\right)^{100}=\left(1-2-i\right)^{100}=\left(1+i\right)^{100}=\left(2i\right)^{50}=-2^{50}$.\\
		$w=\left(1+z_1\right)^{100}+\left(1+z_2\right)^{100}=-2^{50}-2^{50}=-2^{51}$.
	}
\end{ex}

\begin{ex}%Câu 2%[2D4K4-2]
	Gọi $z_1$, $z_2$ là hai nghiệm phức của phương trình $z^2+z+1=0$. Giá trị của $P=\left|z_1^{2019}+z_2^{2019}\right|$ là
	\choice
	{\True $P=2$}
	{$P=3$}
	{$P=2\sqrt{3}$}
	{$P=4038$}
	\loigiai{
		Phương trình $z^2+z+1=0$ có hai nghiệm phức là
		 $\heva{&z_1=-\dfrac{1}{2}-\dfrac{\sqrt{3}}{2}i=-\dfrac{1}{2}\left(1+\sqrt{3}i\right)\\ &z_2=-\dfrac{1}{2}+\dfrac{\sqrt{3}}{2}i=-\dfrac{1}{2}\left(1-\sqrt{3}i\right).}$\\
		Ta có:
		$\heva{&z_1=-\dfrac{1}{2}+\dfrac{\sqrt{3}}{2}i=\cos\dfrac{2\pi}{3}+i\,\sin\dfrac{2\pi}{3}\\
		&z_2=-\dfrac{1}{2}-\dfrac{\sqrt{3}}{2}i=\cos\dfrac{4\pi}{3}+i\,\sin\dfrac{4\pi}{3}.}$\\
		Áp dụng công thức Moive ta được\\
		$z_1^{2019}=\cos\left(2019\cdot\dfrac{2\pi}{3}\right)+i\sin\left(2019\cdot\dfrac{2\pi}{3}\right)=\cos\left(1346\pi\right)+i\sin\left(1346\pi\right)=\cos 0+i\sin 0=1$\\ $z_2^{2019}=\cos\left(2019\cdot\dfrac{4\pi}{3}\right)+i\sin\left(2019\cdot\dfrac{4\pi}{3}\right)=\cos\left(2692\pi\right)+i\sin\left(2692\pi\right)=\cos 0+i\sin 0=1$.\\
		Vậy $P=\left|z_1^{2019}+z_2^{2019}\right|=\left| 1+1\right|=2$.
	}
\end{ex}

\begin{ex}%Câu 3%[2D4K4-2]
	Có bao nhiêu số nguyên $a$ để phương trình $z^2-\left(a-3\right)z+a^2+a=0$ có hai nghiệm phức $z_1$, $z_2$ thỏa mãn $\left|z_1+z_2\right|=\left|z_1-z_2\right|$ ?
	\choice
	{$3$}
	{$1$}
	{\True $4$}
	{$2$}
	\loigiai{
		\begin{itemize}
		\item Trường hợp 1: Hai nghiệm là hai số phức $z_1$ và $z_2$ có phần ảo khác không.\\
		Để phương trình bậc hai với hệ số thực có hai nghiệm phức có phần ảo khác không khi
		\allowdisplaybreaks
		\begin{eqnarray*}
		\Delta=\left(a-3\right)^2-4\left(a^2+a\right)<0&\Leftrightarrow& -3a^2-10a+9<0\\&\Leftrightarrow& a\in\left(-\infty ;\dfrac{-2\sqrt{13}-5}{3}\right)\cup\left(\dfrac{2\sqrt{13}-5}{3};+\infty\right).
		\end{eqnarray*}
		Giả sử $z_1=\dfrac{-b-i\sqrt{\left|\Delta\right|}}{2}$; $z_2=\dfrac{-b+i\sqrt{\left|\Delta\right|}}{2}$.\\
		Ta có
		\allowdisplaybreaks
		\begin{eqnarray*}
		 \left|z_1+z_2\right|=\left|z_1-z_2\right|&\Leftrightarrow&\left| a-3\right|=\sqrt{\left|-3a^2-10a+9\right|}\\
		&\Leftrightarrow&{\left(a-3\right)^2}=\left|-3a^2-10a+9\right|\\
		&\Leftrightarrow&\left[\begin{aligned}
			& a=-9\\ 
			& a=\pm 1\\ 
			& a=0.
		\end{aligned}\right.
	\end{eqnarray*}
	 So với điều kiện ta nhận được $a=-9$; $a=1$.
		\item Trường hợp 2: Hai nghiệm là hai số thực $z_1$ và $z_2$ .\\
		$\left|z_1+z_2\right|=\left|z_1-z_2\right|\Leftrightarrow{S^2}=S^2-4P\Leftrightarrow P=0\Leftrightarrow\left[\begin{aligned}
			& a=0\\ 
			& a=-1.
		\end{aligned}\right.$\\
	 Thử lại thỏa mãn.
	\end{itemize}
	}
\end{ex}

\begin{ex}%Câu 4%[2D4K4-2]
	Gọi $z$ là một nghiệm của phương trình $z^2-z+1=0$. Giá trị của biểu thức $M=z^{2019}+z^{2018}+\dfrac{1}{z^{2019}}+\dfrac{1}{z^{2018}}+5$ bằng
	\choice
	{$5$}
	{\True $2$}
	{$7$}
	{$-1$}
	\loigiai{
		Nhận xét $z=-1$ không là nghiệm phương trình nên
		$$z^2-z+1=0\Leftrightarrow\left(z^2-z+1\right)\left(z+1\right)=0\Leftrightarrow{z^3}=-1.$$
		Do đó $$M=\left(z^3\right)^{673}+z^2\cdot\left(z^3\right)^{672}+\dfrac{1}{\left(z^3\right)^{673}}+\dfrac{z}{z^3\left(z^3\right)^{672}}+5=-1+z^2-1-z+5=z^2-z+1+2=2.$$
	}
\end{ex}

\begin{ex}%Câu 5%[2D4K4-2]
	Gọi $z_1$, $z_2$ là các nghiệm phức của phương trình $z^2-4z+5=0$. Giá trị của $(z_1-1)^{2018}+(z_2-1)^{2018}$ bằng
	\choice
	{$2^{1009}i$}
	{\True $0$}
	{$2^{2018}$}
	{$-2^{1010}i$}
	\loigiai{
		Ta có $z^2-4z+5=0\Leftrightarrow\left[\begin{aligned}
			& z=2+i=z_1\\ 
			& z=2-i=z_2.\\ 
		\end{aligned}\right.$\\
	   Suy ra
	   \allowdisplaybreaks
	   \begin{eqnarray*}
		\left(z_1-1\right)^{2018}+\left(z_2-1\right)^{2018} =\left(1+i\right)^{2018}+\left(1-i\right)^{2018} &=&\left(1+2i+i^2\right)^{1009}+\left(1-2i+i^2\right)^{1009} \\
		&=&\left(2i\right)^{1009}+\left(-2i\right)^{1009}\\ &=&\left(2i\right)^{1009}-\left(2i\right)^{1009}\\
		&=&0.
		\end{eqnarray*}
	}
\end{ex}

\begin{ex}%Câu 6%[2D4K4-2]
	Gọi $z_1$, $z_2$ là hai nghiệm phức của phương trình $z^2+4z+5=0$, trong đó $z_2$ là nghiệm phức có phần ảo dương. Môđun của số phức $w=z_1-2z_2$ bằng
	\choice
	{$\sqrt{5}$}
	{$3$}
	{$2$}
	{\True $\sqrt{13}$}
	\loigiai{
		Ta có $z^2+4z+5=0$ $\Leftrightarrow\left[\begin{aligned}
			& z=-2-i\\ 
			& z=-2+i.\\ 
		\end{aligned}\right.$\\
		Do $z_2$ là nghiệm phức có phần ảo dương nên ta có $z_1=-2-i$, $z_2=-2+i$.\\
		Suy ra $w=z_1-2z_2=-2-i-2\left(-2+i\right)=2-3i$.\\
		Vậy $\left| w\right|=\left| 2-3i\right|=\sqrt{2^2+\left(-3\right)^2}=\sqrt{13}$.
	}
\end{ex}

\begin{ex}%Câu 7%[2D4K4-2]
	Gọi $S$ là tổng tất cả các số thực $m$ để phương trình $z^2-2z+1-m=0$ có nghiệm phức $z$ thỏa mãn $\left| z\right|=2$. Tính $S$.
	\choice
	{$-3$}
	{\True $7$}
	{$6$}
	{$10$}
	\loigiai{
		Phương trình đã cho tương đương $\left(z-1\right)^2=m$.
		\begin{itemize}
		\item Với $m\ge 0$, phương trình có các nghiệm $z=1\pm\sqrt{m}$.\\
		Khi đó $\left| 1\pm\sqrt{m}\right|=2\Leftrightarrow\left[\begin{aligned}
			& 1+\sqrt{m}=2\\ 
			& 1+\sqrt{m}=-2\\ 
			& 1-\sqrt{m}=2\\ 
			& 1-\sqrt{m}=-2\\ 
		\end{aligned}\right.\Leftrightarrow\left[\begin{aligned}
			& m=1\\ 
			& m=9.\\ 
		\end{aligned}\right.$
		\item Với $m<0$, phương trình có nghiệm $z=1\pm i\sqrt{-m}$.
	\end{itemize}
		Khi đó $\left| 1\pm i\sqrt{-m}\right|=2\Leftrightarrow 1-m=4\Leftrightarrow m=-3$.\\
		Từ đó suy ra $S=9+1+\left(-3\right)=7$.
	}
\end{ex}

\begin{ex}%Câu 8%[2D4K4-2]
	Cho $m$ là số thực, biết phương trình $z^2-mz+13=0$ có hai nghiệm phức $z_1,\,\,z_2$; trong đó có một nghiệm có phần ảo là 2. Tính $\left|z_1\right|^2+\left|z_2\right|^2$.
	\choice
	{\True 26}
	{$2\sqrt{13}$}
	{13}
	{$\sqrt{13}$}
	\loigiai{
		Gọi $z_1=a+2i$. Theo giả thiết ta có
		$z_1$ là nghiệm của phương trình $z^2-mz+13=0$
		\allowdisplaybreaks
		\begin{eqnarray*}
		\left(a+2i\right)^2-m\left(a+2i\right)+13=0
		&\Leftrightarrow&\left(a^2-ma+9\right)+\left(4a-2m\right)i=0\\
	    &\Leftrightarrow&\heva{&a^2-ma+9=0\\&4a-2m=0}\\
	    &\Leftrightarrow&\heva{&-a^2+9=0\\&m=2a}\\
	    &\Leftrightarrow&\hoac{&\heva{&a=3\\&m=6}\\&\heva{&a=-3\\&m=-6}}
\end{eqnarray*}
		Nên có hai cặp số $z_1$, $z_2$ thỏa mãn là $\left\{\begin{aligned}
			&{z_1}=3+2i\\ 
			&{z_2}=3-2i\\ 
		\end{aligned}\right.$ hoặc $\left\{\begin{aligned}
			&{z_1}=-3+2i\\ 
			&{z_2}=-3-2i.\\ 
		\end{aligned}\right.$\\
		Đối với mỗi cặp số $z_1$, $z_2$ trên đều có $\left|z_1\right|^2+\left|z_2\right|^2=26$.
	}
\end{ex}

\begin{ex}%Câu 9%[2D4K4-2]
	Gọi $z_1$, $z_2$ là hai nghiệm của phương trình $\dfrac{\left| z\right|^4}{z^2}+\overline{z}=4$. Khi đó $\left|z_1+z_2\right|$ bằng
	\choice
	{$2$}
	{$4$}
	{$8$}
	{\True $1$}
	\loigiai{
		Ta có\\
		$\dfrac{\left| z\right|^4}{z^2}+\overline{z}=-4\Leftrightarrow{\left(\dfrac{\left| z\right|^2}{z}\right)^2}+\overline{z}=-4\Leftrightarrow{\left(\dfrac{z.\overline{z}}{z}\right)^2}+\overline{z}=4\Leftrightarrow{\overline{z}^2}+\overline{z}+4=0\Leftrightarrow\left[\begin{aligned}
			&\overline{z}=-\dfrac{1}{2}+\dfrac{\sqrt{15}}{2}i\\ 
			&\overline{z}=-\dfrac{1}{2}-\dfrac{\sqrt{15}}{2}i\\ 
		\end{aligned}\right.$\\
		Suy ra $\left[\begin{aligned}
			& z=-\dfrac{1}{2}-\dfrac{\sqrt{15}}{2}i\\ 
			& z=-\dfrac{1}{2}+\dfrac{\sqrt{15}}{2}i.\\ 
		\end{aligned}\right.$\\
	 Vậy $\left|z_1+z_2\right|=\left|-\dfrac{1}{2}-\dfrac{\sqrt{15}}{2}i+-\dfrac{1}{2}+\dfrac{\sqrt{15}}{2}i\right|=\left|-1\right|=1$.
	}
\end{ex}

\begin{ex}%Câu 10%[2D4K4-2]
	Cho số phức $z$ thỏa mãn $z^2-2z+3=0$. Tính $\left| w\right|$ biết $w=z^{2018}-z^{2017}+z^{2016}+3z^{2015}+3z^2-z+9$.
	\choice
	{$9\sqrt{3}$}
	{$\sqrt{3}$}
	{\True $5\sqrt{3}$}
	{$2018\sqrt{3}$}
	\loigiai{
		Ta có $z^2-2z+3=0$ $\Leftrightarrow $ $\left[\begin{aligned}
			& z=1-\sqrt{2}\cdot i\\ 
			& z=1+\sqrt{2}\cdot i.
		\end{aligned}\right.$\\
		Theo giả thiết $$w=z^{2018}-z^{2017}+z^{2016}+3z^{2015}+3z^2-z+9=z^{2015}\cdot\left(z^3-z^2+z+3\right)+3z^2-z+9.$$
		\begin{itemize}
		\item Với $z_1=1-\sqrt{2}i$ $\Rightarrow $ $w_1=5-5\sqrt{2}i$ $\Rightarrow $ $\left|w_1\right|=5\sqrt{3}$.
		\item Với $z_2=1+\sqrt{2}i$ $\Rightarrow $ $w_2=5=5\sqrt{2}i$ $\Rightarrow $ $\left|w_2\right|=5\sqrt{3}$.
		\end{itemize}
		Vậy $\left| w\right|=5\sqrt{3}$.
	}
\end{ex}

\begin{ex}%Câu 11%[2D4K4-2]
	Cho hai số phức không thuần thực $z_1$ và $z_2$ thoả mãn phương trình $z^2-2az+3=0$, với $a$ là số thực. Giá trị của biểu thức $T=\left|z_1\right|+\left|z_2\right|$ bằng
	\choice
	{$6$}
	{\True $2\sqrt{3}$}
	{$2\sqrt{5}$}
	{$2\sqrt{2}$}
	\loigiai{
		Hai số phức không thuần thực $z_1$ và $z_2$ thoả mãn phương trình $z^2-2az+3=0$, chứng tỏ $\Delta'=a^2-3<0$. \\
		Các nghiệm của phương trình là $z_1=a+\sqrt{3-a^2}\cdot i$, $z_2=a-\sqrt{3-a^2}\cdot i$\\
		Suy ra $\left|z_1\right|=\left|z_2\right|=\sqrt{a^2+3-a^2}=\sqrt{3}$.\\
		Vậy $T=\left|z_1\right|+\left|z_2\right|=2\sqrt{3}$.
	}
\end{ex}

\begin{ex}%Câu 12%[2D4K4-2]
	Gọi $z_1$, $z_2$, $z_3$, $z_4$ là bốn nghiệm phân biệt của phương trình $z^4+z^2+1=0$ trên tập số phức. Tính giá trị của biểu thức $P=\left|z_1\right|^2+\left|z_2\right|^2+\left|z_3\right|^2+\left|z_4\right|^2$.
	\choice
	{\True $4$}
	{$8$}
	{$6$}
	{$2$}
	\loigiai{
		Ta có
		\allowdisplaybreaks
		\begin{eqnarray*}
		 z^4+z^2+1=0\Leftrightarrow{\left(z^2+1\right)^2}-z^2=0 &\Leftrightarrow&\left(z^2+z+1\right)\cdot \left(z^2-z+1\right)=0\\
		&\Leftrightarrow&\hoac{
			&{z^2}+z+1=0\\ 
			&{z^2}-z+1=0\\ 
		}\\
	    &\Leftrightarrow&\hoac{
			&{\left(z+\dfrac{1}{2}\right)^2}=\dfrac{3}{4}{i^2}\\ 
			&{\left(z-\dfrac{1}{2}\right)^2}=\dfrac{3}{4}{i^2}\\ 
		}\\
	    &\Leftrightarrow&\hoac{
			&{z_{1,2}}=\dfrac{-1+\sqrt{3}i}{2}\\ 
			&{z_{3,4}}=\dfrac{-1+\sqrt{3}i}{2}.
		}
        \end{eqnarray*}	
	Suy ra $\left|z_1\right|=\left|z_2\right|=\left|z_3\right|=\left|z_4\right|=1$.\\
		Vậy $P=\left|z_1\right|^2+\left|z_2\right|^2+\left|z_3\right|^2+\left|z_4\right|^2=4$.
	}
\end{ex}

\begin{ex}%Câu 13%[2D4K4-2]
	Phương trình $z^2+az+b=0$ có một nghiệm là $2-3i$. Khi đó $3a+b$ bằng
	\choice
	{$0$}
	{$3$}
	{$2$}
	{\True $1$}
	\loigiai{
		Vì $2-3i$ là nghiệm của phương trình $z^2-az+b=0$ nên ta có
		$$\left(2-3i\right)^2+a\left(2-3i\right)+b=0 \Leftrightarrow-5-12i+2a-3ai+b=0\Leftrightarrow\heva{
			& 2a+b=5\\ 
			&-3a=12\\ 
		}\Leftrightarrow\heva{
			& b=13\\ 
			& a=-4.\\ 
		}$$
		Vậy $3a+b=1$.
	}
\end{ex}

\begin{ex}%Câu 14%[2D4K4-2]
	Bốn nghiệm của phương trình $z^4-1=0$ được biểu diễn bởi bốn điểm $A$, $B$, $C$, $D$ trên mặt phẳng tọa độ $Oxy$. Tính diện tích tứ giác tạo thành từ bốn điểm trên.
	\choice
	{$4$}
	{\True $2$}
	{$1$}
	{$\dfrac{1}{4}$}
	\loigiai{
		Ta có $z^4-1=0\Leftrightarrow\left(z^2-1\right)\left(z^2+1\right)=0\Leftrightarrow\hoac{
			&{z_1}=1\\ 
			&{z_2}=-1\\ 
			&{z_3}=i\\ 
			&{z_4}=-i.\\ 
		}$\\
		Giả sử $A$, $B$, $C$, $D$ lần lượt là bốn điểm biểu diễn các số phức $z_1,z_2,z_3,z_4$, khi đó ta có
		$$A\left(1;0\right), B\left(-1;0\right), C\left(0;1\right),D\left(0;-1\right).$$
		Tứ giác $ACBD$ là hình vuông có cạnh bằng $\sqrt{2}.$\\
		Vậy $S_{ACBD}=2.$
	}
\end{ex}

\begin{ex}%Câu 15%[2D4K4-2]
	Cho phương trình $z^4-2z^3+6z^2-8z+9=0$ có bốn nghiệm phức phân biệt là $z_1$, $z_2$, $z_3$, $z_4$. Tính giá trị của biểu thức $T=\left(z_1^2+4\right)\left(z_2^2+4\right)\left(z_3^2+4\right)\left(z_4^2+4\right)$.
	\choice
	{$T=0$}
	{$T=2i$}
	{\True $T=1$}
	{$T=-2i$}
	\loigiai{
		Đặt $f(z)=z^4-2z^3+6z^2-8z+9\Rightarrow f(z)=0$ .\\
		Ta có $z^2+4=z^2-4i^2=\left(z+2i\right)\left(z-2i\right)$\\
		Suy ra $$ T=\left[\left(z_1+2i\right)\left(z_2+2i\right)\left(z_3+2i\right)\left(z_4+2i\right)\right]\cdot \left[\left(z_1-2i\right)\left(z_2-2i\right)\left(z_3-2i\right)\left(z_4-2i\right)\right]=\left[f\left(-2i\right)\cdot f\left(2i\right)\right]^4=1.$$
	}
\end{ex}

\begin{ex}%Câu 16%[2D4K4-2]
	Cho $a$, $b$, $c$ là các số thực sao cho phương trình $z^3+a{z^2}+bz+c=0$ có ba nghiệm phức lần lượt là $z_1=w+3i$; $z_2=w+9i$; $z_3=2w-4$, trong đó $w$ là một số phức nào đó. Tính giá trị của $P=\left| a+b+c\right|$.
	\choice
	{\True $P=136$}
	{$P=84$}
	{$P=36$}
	{$P=208$}
	\loigiai{
		Đặt $w=x+yi$, với $x,y\in\mathbb{R}$.\\
		Ta có 
		\allowdisplaybreaks
		\begin{eqnarray*}
		z_1+z_2+z_3=-a\Rightarrow 4w+4+12i=-a&\Leftrightarrow&\left(4x+4+a\right)+\left(12+4y\right)i=0\\
		&\Leftrightarrow&\heva{
			& 4x+4+a=0\\ 
			& 12+4y=0\\ 
		}\\
	    &\Leftrightarrow&\heva{
			& 4x+4=-a\\ 
			& y=-3.\\ 
		}
	    \end{eqnarray*}
		Từ đó $w=x-3i$ $\Rightarrow{z_1}=x$; $z_2=x+6i$; $z_3=2x-4-6i$.\\
		Vì phương trình bậc ba $z^3+a{z^2}+bz+c=0$ có một nghiệm thực nên hai nghiệm phức còn lại phải là hai số phức liên hợp, suy ra $x=2x-4\Leftrightarrow x=4$.\\
		Như vậy $z_1=4$; $z_2=4+6i$; $z_3=4-6i$.\\
		Do đó
		$$\heva{
			&{z_1}+z_2+z_3=-a\\ 
			&{z_1}{z_2}+z_2z_3+z_3z_1=-b\\ 
			&{z_1}{z_2}{z_3}=c\\ 
		}
	    \Leftrightarrow\heva{
			& 12=-a\\ 
			& 84=b\\ 
			& 208=-c\\ 
		}
	    \Leftrightarrow\heva{
			& a=-12\\ 
			& b=84\\ 
			& c=-208.\\ 
		}$$
		Vậy $P=\left| a+b+c\right|=\left|-12+84+\left(-208\right)\right|=136$.
	}
\end{ex}

\begin{ex}%Câu 17%[2D4K4-2]
	Gọi $S$ là tập hợp các số phức $z$ thỏa mãn điều kiện $z^4=\left| z\right|$. Số phần tử của $z$ là
	\choice
	{\True $5$}
	{$4$}
	{$7$}
	{$6$}
	\loigiai{
		Ta có $z^4=\left| z\right|$ $\Leftrightarrow{\left| z\right|^4}=\left| z\right|$ $\Leftrightarrow\left| z\right|\left(\left| z\right|^3-1\right)=0$ $\Leftrightarrow\hoac{
			&\left| z\right|=0\\ 
			&\left| z\right|=1.\\ 
		}$
	    \begin{itemize}
	   \item $\left| z\right|=0$ $\Leftrightarrow z=0$.
		\item $\left| z\right|=1$ $\Leftrightarrow{z^4}=1$ $\Leftrightarrow\left(z^2-1\right)\left(z^2+1\right)=0$ $\Leftrightarrow\hoac{
			& z=-1\\ 
			& z=1\\ 
			& z=i\\ 
			& z=-i.\\ 
		}$
	    \end{itemize}
		Suy ra $S$ có $5$ phần tử.
	}
\end{ex}

\begin{ex}%Câu 18%[2D4K4-2]
	Gọi $S$ là tập hợp các số thực $m$ để phương trình $z^2+3z+m^2-2m=0$ có một nghiệm phức $z_0$ với $\left|z_0\right|=2$. Tổng tất cả các phần tử trong $S$ là
	\choice
	{\True $4$}
	{$6$}
	{$2$}
	{$3$}
	\loigiai{
		\begin{itemize}
		\item \textbf{Trường hợp 1:} $z_0$ là số thực. Ta có
		$$\left|z_0\right|=2\Leftrightarrow\hoac{
			&{z_0}=2\\ 
			&{z_0}=-2\\ 
		}\Rightarrow\hoac{
			&{m^2}-2m+10=0\\ 
			&{m^2}-2m-2=0\\ 
		}\Leftrightarrow m=1\pm\sqrt{3}.$$
		\item\textbf{ Trường hợp 2:} $z_0$ không phải là số thực $\Leftrightarrow\Delta=9-4\left(m^2-2m\right)<0\Leftrightarrow{m^2}-2m>\dfrac{9}{4}$ (1)\\
		Vì phương trình $z^2+3z+m^2-2m=0\,(*)$ có các hệ số thực và $z_0$ là nghiệm của $(*)$ nên $\bar{z}_0$ cũng là nghiệm của $(*)$.\\
		Theo Viet ta có 
		$$z_0\cdot\overline{z}_0=m^2-2m\Leftrightarrow 4=\left|z_0\right|^2=m^2-2m
		\Leftrightarrow{m^2}-2m-4=0\Leftrightarrow m=1\pm\sqrt{5}$$
		\end{itemize}
		Vậy tổng các phần tử của $S$ bằng $4$.
	}
\end{ex}

\begin{ex}%Câu 19%[2D4K4-2]
	Gọi $z_1$, $z_2$ là hai nghiệm phức của phương trình $z^2-2z+2=0$. Tập hợp các điểm biểu diễn của số phức $w$ thỏa mãn $\left| w-z_1\right|=\left| w-z_2\right|$ là đường thẳng có phương trình
	\choice
	{$x+y=0$}
	{\True $y=0$}
	{$x-y=0$}
	{$x=0$}
	\loigiai{
		Ta có: $z^2-2z+2=0\Leftrightarrow\hoac{
			& z=1-i\\ 
			& z=1+i.\\ 
		}$\\
		Không mất tính tổng quát ta giả sử $z_1=1-i$ và $z_2=1+i$.\\
		Gọi $w=x+yi$. Ta có
		\allowdisplaybreaks
		\begin{eqnarray*}
		\left| w-z_1\right|=\left| w-z_2\right|&\Leftrightarrow&\left|\left(x-1\right)+\left(y+1\right)i\right|=\left|\left(x-1\right)+\left(y-1\right)i\right|\\
		&\Leftrightarrow&{\left(x-1\right)^2}+\left(y+1\right)^2=\left(x-1\right)^2+\left(y-1\right)^2\\
		&\Leftrightarrow&{x^2}+y^2-2x+2y+2=x^2+y^2-2x-2y+2\\
		&\Leftrightarrow& 4y=0\\
		&\Leftrightarrow& y=0.
		\end{eqnarray*}
		Vậy tập hợp các điểm biểu diễn số phức $w$ thỏa yêu cầu bài toán là đường thẳng $y=0$.
	}
\end{ex}

\begin{ex}%Câu 20%[2D4K4-2]
	Trên tập hợp số phức, cho phương trình $z^2+bz+c=0$ với $b,c\in\mathbb{R}$. Biết rằng hai nghiệm của phương trình có dạng $w+3$ và $2w-15i+9$ với $w$ là một số phức. Tính $S=b^2-2c$.
	\choice
	{\True $S=-32$}
	{$S=1608$}
	{$S=1144$}
	{$S=-64$}
\loigiai{
	Từ đề bài suy ra $\heva{
		&{\left(w+3\right)^2}+b\left(w+3\right)+c=0\\ 
		&{\left(2w-15i+9\right)^2}+b\left(2w-15i+9\right)+c=0\\ 
	}$
     $\Leftrightarrow\heva{
		&\left(2w-15i+9\right)\left(w+3\right)=c\\ 
		& 2w-15i+9+w+3=-b.\\ 
	}$\\
	Giả sử $w=x+yi$, $\left(x,y\in\mathbb{R}\right)$.\\
	Khi đó $w+3=x+3+yi$, $2w-15i+9=2x+9+\left(2y-15\right)i$.\\
	Theo đề ta có $\heva{
		&\left(2w-15i+9\right)\left(w+3\right)=c\\ 
		& 2w-15i+9+w+3=-b.\\ 
	}$ 
       $\Leftrightarrow\heva{
		&\left(2x+9+\left(2y-15\right)i\right)\left(x+3+yi\right)=c\\ 
		&\left(2x+9+\left(2y-15\right)i\right)+\left(x+3+yi\right)=-b.\\ 
	}$\\
	Vì $b,c\in\mathbb{R}$ nên $\heva{
		&\left(x+3\right)\left(2y-15\right)+y\left(2x+9\right)=0\\ 
		& 2y-15+y=0\\ 
	}$
     $\Leftrightarrow\heva{
		& x=-6\\ 
		& y=5.\\ 
	}$\\
	Suy ra $w=-6+5i$, do đó $\heva{
		&\left(2w-15i+9\right)\left(w+3\right)=c\\ 
		& 2w-15i+9+w+3=-b\\ 
	}$
     $\Leftrightarrow\heva{
		& c=34\\ 
		& b=-6.\\ 
	}$\\
	Suy ra $S=b^2-2c=-32$.
}
\end{ex}
\Closesolutionfile{ans}
%======================
\subsection{Bảng đáp án}
\inputansbox{8}{ans/ANS-DANG-45}