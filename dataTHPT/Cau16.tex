%Dạng 16
\setcounter {section} {15}
\setcounter{ex}{0}
\section{Số phức và các phép toán}
\subsection{Kiến thức cần nhớ}
\begin{khung}
	\subsubsection{Phần thực, phần ảo của số phức, số phức liên hợp}	
	\begin{enumerate}[\textbullet]
		\item Số phức có dạng $z=a+b i\left(a, b \in \mathbb{R}, i^2=-1\right)$. Phần thực của $z$ là $a$, phần ảo của $z$ là $b$ và $i$ được gọi là đơn vị ảo.
		\item Số phức liên hợp của $z$ là $\bar{z}=\overline{a+b i}=a-b i$.
		\begin{listEX}[2]
			\item[$\oplus$] $ z\cdot \bar{z}=a^2+b^2$
			\item[$\oplus$] $\overline{z_1 \pm z_2}=\bar{z}_1 \pm \bar{z}_2$
			\item[$\oplus$] $\overline{z_1 \cdot z_2}=\bar{z}_1 \cdot \bar{z}_2$
			\item[$\oplus$]  $\overline{\left(\dfrac{z_1}{z_2}\right)}=\dfrac{\bar{z}_1}{\bar{z}_2}$
			\item[$\oplus$] Tổng và tích của $z$ và $\bar{z}$ luôn là một số thực.
		\end{listEX}
		
		\item Lưu ý: $i^{4n}=1; i^{4 n+1}=i; i^{4n+2}=-1; i^{4n+3}=-i;$ với $n \in \mathbb{N}$.
	\end{enumerate}
	\subsubsection{Hai số phức bằng nhau}
	Cho hai số phức $z_1=a_1+b_1 i, z_2=a_2+b_2 i\left(a_1, a_2, b_2, b_2 \in \mathbb{R}\right)$. \\
	Khi đó: $$z_1=z_2 \Leftrightarrow\heva{&a_1=a_2 \\& b_1=b_2}$$
	\subsubsection{Biểu diễn hình học của số phức, môđun của số phức}
	\begin{center}
		\begin{tikzpicture}[scale=.6,>=stealth, font=\footnotesize, line join=round, line cap=round]
			\draw[->] (-2,0)--(6,0) node [below]{$x$};
			\draw[->] (0,-5)--(0,5) node [left]{$y$};
			\node at (0,0) [below left]{$O$};
			\clip (-2,-5) rectangle (6,6);
			\draw[fill=black,dashed] 
			(3,0) node[below left] {$a$} circle(2pt)
			(3,3) node[above right] {$M(a;b)$} circle(2pt)
			(0,3) node[left] {$b$} circle(2pt)
			(0,-3) node[left] {$-b$} circle(2pt)
			(3,-3) node[below right] {$M'(a;-b)$} circle(2pt)
			;
			\draw [dashed] (0,3)--(3,3)--(3,-3)--(0,-3);
			\draw [thick] (0,0)--(3,3) (0,0)--(3,-3);
		\end{tikzpicture}
	\end{center}
	\begin{enumerate}[\textbullet]
		\item \textbf{Biễu diễn hình học của số phức}
		\begin{itemize}
			\item Số phức $z=a+b i(a, b \in \mathbb{R})$ được biểu diễn bởi điểm $M(a; b)$ trong mặt phẳng tọa độ.
			\item $z$ và $\bar{z}$ được biểu diễn bởi hai điểm đối xứng nhau qua trục $O x$.
		\end{itemize}
		\item \textbf{Mô đun của số phức}
		\begin{itemize}
			\item Mô đun của số phức $z$ là $|z|=|\overrightarrow{O M}|=\sqrt{a^2+b^2}$.
			\item Ta có: $|z|=\sqrt{z\cdot \bar{z}};|z|=|\bar{z}|$.
		\end{itemize}
	\end{enumerate}
\end{khung}
\subsection{Bài tập mẫu}
\Opensolutionfile{ans}[ans/ANS-DANG-16]
\begin{khung}
	\begin{vd}[Đề minh họa BGD 2022-2023]%[Mui Doan, ĐMH-2023]%[2D4Y1-1]
		Phần ảo của số phức $z=2-3 i$ là
		\choice
		{\True $-3$}
		{$-2$}
		{$ 2 $}
		{$ 3 $}
	\loigiai{
	Phần ảo của số phức $z=2-3 i$ là $-3$.
	}	
	\end{vd}
\end{khung}
\subsection{Bài tập tương tự và phát triển}

%== Câu 1==
\begin{ex}%[Mui Doan, PT-ĐMH2023]%[2D4Y1-1]
	Phần thực và phần ảo của số phức $z=(1+2i)i$ lần lượt là
	\choice
	{$1$ và $2$}
	{\True $-2$ và $1$}
	{$1$ và $-2$}
	{$2$ và $1$}
	\loigiai{
		Ta có $z=(1+2i)i=i+2i^2=-2+i$.\\
		Vậy phần thực và phần ảo của $z$ lần lượt là $-2$ và $1$.
	}
\end{ex}
%== Câu 2==
\begin{ex}%[Mui Doan, PT-ĐMH2023]%[2D4Y1-1]
	Cho hai số phức $z_1=5-3i$, $z_2=-1+2i$. Tổng phần thực, phần ảo của tổng hai số phức đã cho là
	\choice
	{\True $S=3$}
	{$S=7$}
	{$S=4$}
	{$S=5$}
	\loigiai{
		Ta có $z_1+z_2=(5-3i)+(-1+2i)=4-i$.\\
		Vậy tổng phần thực và phần ảo của tổng hai số phức đã cho là $S=3$.
	}
\end{ex}
%== Câu 3==
\begin{ex}%[Mui Doan, PT-ĐMH2023]%[2D4Y1-1]
	Phần ảo của số phức $z=4-5i$ là
	\choice
	{$-5i$}
	{\True $-5$}
	{$5$}
	{$4$}
	\loigiai{
		Phần ảo của số phức $z=4-5i$ là $-5$.
	}
\end{ex}
%== Câu 4==
\begin{ex}%[Mui Doan, PT-ĐMH2023]%[2D4Y1-1]
	Số phức liên hợp của số phức $1-2i$ là
	\choice
	{$-1+2i$}
	{$-1-2i$}
	{\True $1+2i$}
	{$-2+i$}
	\loigiai{
		Theo định nghĩa số phức liên hợp của số phức $z=a+bi$, $a$, $b\in \mathbb{R}$ là số phức $z=a-bi$, $a$, $b\in \mathbb{R}$.
	}
\end{ex}
%== Câu 5==
\begin{ex}%[Mui Doan, PT-ĐMH2023]%[2D4Y1-1]
	Cho số phức $z=2-3i$. Số phức liên hợp $z$ là
	\choice
	{$\overline{z}=-2-3i$}
	{$\overline{z}=3-2i$}
	{$\overline{z}=-2+3i$}
	{\True $\overline{z}=2+3i$}
	\loigiai{
		Ta có $z=a+bi\Rightarrow \overline{z}=a-bi$. Do đó $z=2-3i\Rightarrow \overline{z}=2+3i$.\\
			}
\end{ex}
%== Câu 6==
\begin{ex}%[Mui Doan, PT-ĐMH2023]%[2D4Y1-1]
	Số phức $z$ thỏa mãn $z=5-8i$ có phần ảo là
	\choice
	{$8$}
	{$-8i$}
	{\True $-8$}
	{$5$}
	\loigiai{
		Số phức $z=5-8i$ có phần ảo là $-8$.
	}
\end{ex}
%== Câu 7==
\begin{ex}%[Mui Doan, PT-ĐMH2023]%[2D4Y1-1]
	Mô-đun của số phức $z=4-2i$ bằng
	\choice
	{\True $2\sqrt{5}$}
	{$4$}
	{$2\sqrt{3}$}
	{$3\sqrt{2}$}
	\loigiai{
	Ta có $|z|=\sqrt{4^2+(-2)^2}=2\sqrt{5}$.
	}
\end{ex}
%== Câu 8==
\begin{ex}%[Mui Doan, PT-ĐMH2023]%[2D4Y1-1]
	Cho $z=3+2i$. Tìm mô-đun của $z$.
	\choice
	{$|z|=5$}
	{$|z|=13$}
	{\True $|z|=\sqrt{13}$}
	{$|z|=\sqrt{5}$}
	\loigiai{
		Ta có $|z|=\left| 3+2i\right|=\sqrt{9+4}=\sqrt{13}$.
	}
\end{ex}
%== Câu 9==
\begin{ex}%[Mui Doan, PT-ĐMH2023]%[2D4Y1-1]
	Số phức liên hợp của số phức $z=6-8i$ là
	\choice
	{$-6+8i$}
	{\True $6+8i$}
	{$-6-8i$}
	{$8-6i$}
	\loigiai{
		Ta có số phức $z=a+bi$ sẽ có số phức liên hợp là $\overline{z}=a-bi$.\\
		Do đó số phức liên hợp của $z=6-8i$ là $\overline{z}=6+8i$.
	}
\end{ex}
%== Câu 10==
\begin{ex}%[Mui Doan, PT-ĐMH2023]%[2D4Y1-1]
	Số phức liên hợp của số phức $z$ có phần thực bằng $4$, phần ảo bằng $5$ là
	\choice
	{$\overline{z}=4+5i$}
	{\True $\overline{z}=4-5i$}
	{$\overline{z}=5-4i$}
	{$\overline{z}=5+4i$}
	\loigiai{
		Ta có số phức $z$ có phần thực bằng $4$, phần ảo bằng $5$ là $z=4+5i$.\\
		Do đó $\overline{z}=\overline{4+5i}=4-5i$.
	}
\end{ex}
%== Câu 11==
\begin{ex}%[Mui Doan, PT-ĐMH2023]%[2D4Y1-1]
	Cho số phức $z=3-4i$. Mệnh đề nào dưới đây \textbf{sai}?
	\choice
	{\True Số phức liên hợp của $z$ là $3-4i$}
	{Phần thực và phần ảo của $z$ lần lượt là $3$ và $-4$}
	{Biểu diễn số phức $z$ lên mặt phẳng tọa độ là điểm $M(3;-4)$}
	{Mô-đun của số phức $z$ bằng $5$}
	\loigiai{
		Số phức liên hợp của $z=3-4i$ là $\overline{z}=3+4i$. \\
		Mệnh đề \lq\lq  Số phức liên hợp của $z$ là $3-4i$\rq\rq\, sai.
	}
\end{ex}
%== Câu 12==
\begin{ex}%[Mui Doan, PT-ĐMH2023]%[2D4Y1-1]
	Số phức nào dưới đây là số thuần ảo
	\choice
	{\True $z=-2i$}
	{$z=2+2i$}
	{$z=-1+\sqrt{2}i$}
	{$z=-2$}
	\loigiai{
		Số phức thuần ảo là $z=-2i$.
	}
\end{ex}
%== Câu 13==
\begin{ex}%[Mui Doan, PT-ĐMH2023]%[2D4Y1-1]
	Phần ảo của số phức $z=2+3i$ là
	\choice
	{$2i$}
	{$3i$}
	{$2$}
	{\True $3$}
	\loigiai{
		Ta có $z=2+3i$. Phần ảo của số phức $z$ là $3$.
		}
\end{ex}
%== Câu 14==
\begin{ex}%[Mui Doan, PT-ĐMH2023]%[2D4Y1-1]
	Số phức liên hợp của số phức $z=2-3i$ là
	\choice
	{\True $\overline{z}=2+3i$}
	{$\overline{z}=-2+3i$}
	{$z=3+2i$}
	{$\overline{z}=3-2i$}
	\loigiai{
		Số phức liên hợp của số phức $z=2-3i$ là $\overline{z}=2+3i$.
	}
\end{ex}
%== Câu 15==
\begin{ex}%[Mui Doan, PT-ĐMH2023]%[2D4Y1-1]
	Mô-đun của số phức $z=7-5i$ bằng
	\choice
	{$2\sqrt{6}$}
	{$74$}
	{$24$}
	{\True $\sqrt{74}$}
	\loigiai{
		Ta có $|z|=\sqrt{7^2+5^2}=\sqrt{74}$.
	}
\end{ex}
%== Câu 16==
\begin{ex}%[Mui Doan, PT-ĐMH2023]%[2D4Y1-1]
	Cho số phức $z=3+4i$. Tính $|z|$.
	\choice
	{$|z|=\sqrt{5}$}
	{$|z|=13$}
	{$|z|=\sqrt{13}$}
	{\True $|z|=5$}
	\loigiai{
		$z=3+4i\Rightarrow |z|=\sqrt{3^2+4^2}=5$.
	}
\end{ex}
%== Câu 17==
\begin{ex}%[Mui Doan, PT-ĐMH2023]%[2D4Y1-1]
	Cho số phức $z=3-4i$. Mô-đun của $z$ bằng
	\choice
	{$12$}
	{\True $5$}
	{$7$}
	{$1$}
	\loigiai{
		Mô-đun của $z$ bằng $|z|=\sqrt{3^2+(-4)^2}=5$.
	}
\end{ex}
%== Câu 18==
\begin{ex}%[Mui Doan, PT-ĐMH2023]%[2D4Y1-1]
	Cho số phức $z$ thỏa mãn $z=5-8i$ có phần ảo là
	\choice
	{\True $-8$}
	{$8$}
	{$5$}
	{$-8i$}
	\loigiai{
		Ta có $z=5-8i$ có phần ảo bằng $-8$.
	}
\end{ex}
%== Câu 19==
\begin{ex}%[Mui Doan, PT-ĐMH2023]%[2D4Y1-1]
	Mô-đun của số phức $2+3i$ bằng
	\choice
	{\True $\sqrt{13}$}
	{$13$}
	{$\sqrt{5}$}
	{$5$}
	\loigiai{
		Ta có $\left| 2+3i\right|=\sqrt{2^2+3^2}=\sqrt{13}$.
	}
\end{ex}
%== Câu 20==
\begin{ex}%[Mui Doan, PT-ĐMH2023]%[2D4Y1-1]
	Mô-đun của số phức $z=3+4i$ là
	\choice
	{\True $5$}
	{$7$}
	{$3$}
	{$4$}
	\loigiai{
		$|z|=\sqrt{3^2+4^2}=5$.
	}
\end{ex}
%== Câu 21==
\begin{ex}%[Mui Doan, PT-ĐMH2023]%[2D4Y1-1]
	Gọi $a$, $b$ lần lượt là phần thực và phần ảo của số phức $z=-3+2i$. Giá trị $a-b$ bằng
	\choice
	{$1$}
	{$5$}
	{\True $-5$}
	{$-1$}
	\loigiai{
		Số phức $z=-3+2i$, có phần thực $a=-3$, phần ảo $b=2$. Do đó $a-b=-3-2=-5$.
	}
\end{ex}
%== Câu 22==
\begin{ex}%[Mui Doan, PT-ĐMH2023]%[2D4Y1-1]
	Số phức có phần thực bằng $3$ và phần ảo bằng $2$ là
	\choice
	{$3-2i$}
	{\True $3+2i$}
	{$2+3i$}
	{$2-3i$}
	\loigiai{
		Số phức có phần thực bằng $3$ và phần ảo bằng $2$ là $z=3+2i$.
	}
\end{ex}
%== Câu 24==
\begin{ex}%[Mui Doan, PT-ĐMH2023]%[2D4Y1-1]
	Với $x$, $y$ là các số thực thì số phức $z=x-1+(y+2)i$ là số ảo khi và chỉ khi
	\choice
	{$y=-2$}
	{$x=1$, $y\ne -2$}
	{$y=-2$, $x\ne 1$}
	{\True $x=1$}
	\loigiai{
		Ta có $z=x-1+(y+2)i$ là số thuần ảo $\Leftrightarrow$ $x-1=0$ $\Leftrightarrow$ $x=1$.
	}
\end{ex}
%== Câu 25==
\begin{ex}%[Mui Doan, PT-ĐMH2023]%[2D4Y1-1]
	Cho số phức $z=-2+i$. Tìm phần thực và phần ảo của số phức $\overline{z}$.
	\choice
	{Phần thực bằng $2$ và phần ảo bằng $1$}
	{Phần thực bằng $2$ và phần ảo bằng $i$}
	{Phần thực bằng $-2$ và phần ảo bằng $-i$}
	{\True Phần thực bằng $-2$ và phần ảo bằng $-1$}
	\loigiai{
	Phần thực bằng $-2$ và phần ảo bằng $-1$.	
	}
\end{ex}
%== Câu 26==
\begin{ex}%[Mui Doan, PT-ĐMH2023]%[2D4Y1-1]
	Mô-đun của số phức $z=5-4i$ bằng
	\choice
	{$1$}
	{$41$}
	{\True $\sqrt{41}$}
	{$3$}
	\loigiai{
		$|z|=\sqrt{5^2+(-4)^2}=\sqrt{41}$.
	}
\end{ex}
%== Câu 27==
\begin{ex}%[Mui Doan, PT-ĐMH2023]%[2D4Y1-1]
	Cho số phức $z=2+\sqrt{3}i$. Mô-đun của $z$ bằng
	\choice
	{$5$}
	{$\sqrt{5}$}
	{\True $\sqrt{7}$}
	{$7$}
	\loigiai{
		Ta có $|z|=\sqrt{2^2+3}=\sqrt{7}$.
	}
\end{ex}
%== Câu 28==
\begin{ex}%[Mui Doan, PT-ĐMH2023]%[2D4Y1-1]
	Phần thực của số phức $(2-i)(1+2i)$ là
	\choice
	{$5$}
	{$3$}
	{\True $4$}
	{$0$}
	\loigiai{
		Ta có $(2-i)(1+2i)=2+4i-i-2i^2=4+3i$.
	}
\end{ex}
%== Câu 29==
\begin{ex}%[Mui Doan, PT-ĐMH2023]%[2D4Y1-1]
	Số phức liên hợp của số phức $z=1-2i$ là
	\choice
	{$-1+2i$}
	{$-1-2i$}
	{$2-i$}
	{\True $1+2i$}
	\loigiai{
		Số phức liên hợp của số phức $z=1-2i$ là $\overline{z}=1+2i$.
	}
\end{ex}
%== Câu 30==
\begin{ex}%[Mui Doan, PT-ĐMH2023]%[2D4Y1-1]
	Mô-đun của số phức $\mathrm{z}=5-2i$ bằng
	\choice
	{$29$}
	{\True $\sqrt{29}$}
	{$3$}
	{$7$}
	\loigiai{
		Ta có $|z|=\left| 5-2i\right|=\sqrt{5^2+(-2)^2}=\sqrt{29}$.
	}
\end{ex}

\Closesolutionfile{ans}
%======================
\subsection{Bảng đáp án}
\inputansbox{8}{ans/ANS-DANG-16}


