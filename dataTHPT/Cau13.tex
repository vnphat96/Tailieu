\section{Tính thể tích khối lăng trụ đứng}
\subsection{Kiến Thức Cần Nhớ}
\Opensolutionfile{ans}[ans/ANS-DANG-13]
\begin{khung}
	\begin{enumerate}
		\item Thể tích khối lăng trụ $V=B\cdot h$ với $B\colon$diện tích đáy, $h\colon$chiều cao.
		\item Các hệ thức lượng trong tam giác vuông.
		\immini
		{Cho tam giác $ABC$ vuông tại $A$, đường cao $AH$, trung tuyến $AM$. Khi đó
			\begin{itemize}
				\item $BC^2=AB^2+AC^2$.
				\item $BH\cdot BC =AB^2$; $CH\cdot CB =CA^2$.
				\item $AB\cdot AC=AH\cdot BC$; $AM=\dfrac{1}{2}BC$.
				\item $CH\cdot BH=AH^2$.
				\item $\dfrac{1}{AH^2}=\dfrac{1}{AB^2}+\dfrac{1}{AC^2}$.
				\item $\sin \widehat{ABC}=\dfrac{AC}{BC} $; $\cos\widehat{ABC}=\dfrac{AB}{BC}$; $\tan \widehat{ABC}=\dfrac{AC}{AB}$; $\cot \widehat{ABC}=\dfrac{AB}{AC}$.
			\end{itemize}
		}
		{\begin{tikzpicture}[scale=1, font=\footnotesize, line join=round, line cap=round, >=stealth]
				\def\a{3}
				\def\b{4}
				\pgfmathsetmacro{\m}{sqrt{((\a)^2+(\b)^2)}}
				\pgfmathsetmacro{\n}{(\b)^2/\m}
				\path
				(0:0) coordinate (A)
				--++(0:\b) coordinate (B)
				(0:0) coordinate (A)
				--++(90:\a) coordinate (C)
				;
				\coordinate (M) at ($(B)!0.5!(C)$);
				\coordinate (H) at ($(B)!\n/\m!(C)$);
				\draw(A)--(B)--(C)--cycle(A)--(M)(A)--(H);
				\draw pic[draw,angle radius=2mm] {right angle = C--A--B};
				\draw pic[draw,angle radius=2mm]{right angle = C--H--A};
				\foreach \diem/\goc in {A/-90,B/-90,C/90,M/30,H/30} \fill[black](\diem) circle (1pt) ($(\diem)+(\goc:3.5mm)$) node{$\diem$};
		\end{tikzpicture}}
		\item Đường chéo của hình vuông cạnh $a$ có độ dài bằng $a\sqrt{2}$.
		\item Đường cao của tam giác đều cạnh $a$ có độ dài bằng $\dfrac{a\sqrt{3}}{2}$.
		\item Diện tích tam giác bất kỳ
		\begin{itemize}
			\item $S_{\triangle ABC}=\dfrac{1}{2}\cdot a\cdot h_a=\dfrac{1}{2}\cdot b\cdot h_b=\dfrac{1}{2}\cdot c\cdot h_c$, trong đó $h_a, h_b, h_c$ lần lượt là đường cao hạ từ các đỉnh $A$, $B$, $C$ của tam giác $ABC$; $BC=a$, $AC=b$, $AB=c$.
			\item $S_{\triangle ABC} =\dfrac{1}{2}\cdot b\cdot c\cdot \sin A=\dfrac{1}{2}\cdot a\cdot c\cdot \sin B =\dfrac{1}{2}\cdot a\cdot b\cdot \sin C$.
			\item $S_{\triangle ABC}=\dfrac{abc}{4R}$, trong đó $R$ là bán kính đường tròn ngoại tiếp $\triangle ABC$.
			\item $S_{\triangle ABC}=p\cdot r$, trong đó $r$ là bán kính đường tròn nội tiếp $\triangle ABC$.
			\item $S_{\triangle ABC} =\sqrt{p(p-a)(p-b)(p-c)}$, trong đó $p=\dfrac{a+b+c}{2}$.
		\end{itemize}
		\item Trường hợp đặc biệt
		\begin{itemize}
			\item Diện tích tam giác $\triangle ABC$ vuông tại $A$ là $S=\dfrac{1}{2}\cdot AB \cdot AC$.
			\item Diện tích của tam giác đều cạnh $a$ là $S=\dfrac{a^2\sqrt{3}}{4}$.
		\end{itemize}
		\item Diện tích hình chữ nhật $S=a\cdot b$, trong đó $a$, $b$ lần lượt là chiều dài và chiều rộng của hình chữ nhật.
		\item Diện tích hình vuông cạnh $a$ là $S=a^2$.
		\item Diện tích hình thoi $S=\dfrac{1}{2}\cdot AC\cdot BD$, trong đó $AC$ và $BD$ là hai đường chéo.
		\item Diện tích hình thang $S=\dfrac{(\text{đáy lớn + đáy bé})\cdot h}{2}$, trong đó $h$ là chiều cao của hình thang.
		\item Diện tích hình bình hành $ABCD$ là $S=AH\cdot CD$, trong đó $AH$ là chiều cao của tam giác $ABD$.
		\item Định lí hàm số $\sin$: $\dfrac{a}{\sin A}=\dfrac{b}{\sin B}=\dfrac{c}{\sin C}=2R$.
		\item Định lí hàm số côsin
			\begin{itemize}
				\item $a^2=b^2+c^2-2bc\cdot\cos A$.
				\item $b^2=a^2+c^2-2ac\cdot \cos B$.
				\item $c^2=a^2+b^2-2ab\cdot \cos C$.
			\end{itemize}
		\item Công thức đường trung tuyến
			\begin{multicols}{3}
			\begin{itemize}
				\item $m_a^2=\dfrac{b^2+c^2}{2}-\dfrac{a^2}{4}$.
				\item $m_b^2=\dfrac{a^2+c^2}{2}-\dfrac{b^2}{4}$.
				\item $m_c^2=\dfrac{a^2+b^2}{2}-\dfrac{c^2}{4}$.
			\end{itemize}
		\end{multicols}	
	\end{enumerate}
\end{khung}
\subsection{Bài tập mẫu}
\begin{khung}
	\begin{vd}[Đề tham khảo 2023]%[2H1Y3-2]%
		Cho khối lăng trụ có diện tích đáy $B=3$ và chiều cao $h=4$. Thể tích của khối lăng trụ đã cho bằng
		\choice
		{ $4$}
		{ \True $12$}
		{ $8$}
		{ $6$}
		\loigiai{
			Thể tích của khối lăng trụ đã cho là $V=B\cdot h=3\cdot 4=12$.}
	\end{vd}
\end{khung}
\subsection{Bài tập tương tự và phát triển}
\begin{ex}%[2H1Y3-2]% ::Cau 1::
	Thể tích của khối hộp chữ nhật có độ dài các cạnh lần lượt là $a$, $2a$, $3a$ bằng
	\choice
	{ $3a^3$}
	{ $2a^3$}
	{ \True$6a^3$}
	{ $\dfrac{2a^3}{3}$}
	\loigiai{
		Thể tích của khối hộp chữ nhật là $V=a\cdot 2a\cdot 3a=6a^3$.}
\end{ex}

\begin{ex}%[2H1B3-4]% ::Cau 2::
	Gọi $h$, $S$, $V$ lần lượt là chiều cao, diện tích đáy và thể tích của hình lăng trụ. Chiều cao khối lăng trụ là
	\choice
	{ \True$\dfrac{V}{S}$}
	{ $\dfrac{S}{V}$}
	{ $\dfrac{3V}{S}$}
	{ $\dfrac{1}{3}SV$}
	\loigiai{
		Ta có $V=S\cdot h\Rightarrow h=\dfrac{V}{S}$.}
\end{ex}

\begin{ex}%[2H1B3-4]% ::Cau 3::
	Cho lăng trụ tứ giác $ABCD.A'B'C'D'$ có đáy là hình vuông cạnh $a$ và có thể tích bằng $3a^3$. Tính chiều cao $h$ của lăng trụ đã cho.
	\choice
	{ $h=a$}
	{ \True$h=3a$}
	{ $h=9a$}
	{ $h=\dfrac{a}{3}$}
	\loigiai{
		Gọi $S$ là diện tích đáy và $h$ là chiều cao của khối lăng trụ $ABCD.A'B'C'D'$.\\
		Khi đó, thể tích của khối lăng trụ $ABCD.A'B'C'D'$ là $V=S\cdot h\Rightarrow h=\dfrac{V}{S}=\dfrac{3a^3}{a^2}\Rightarrow h=3a$.}
\end{ex}

\begin{ex}%[2H1Y3-2]% ::Cau 4::
	Một khối lăng trụ có chiều cao bằng $2a$ và diện tích đáy bằng $2a^2$. Thể tích khối lăng trụ đã cho bằng
	\choice
	{ $V=\dfrac{2a^3}{3}$}
	{ \True$V=4a^3$}
	{ $V=\dfrac{4a^3}{3}$}
	{ $V=\dfrac{4a^2}{3}$}
	\loigiai{
		Thể tích khối lăng trụ đã cho bằng $V=2a\cdot 2a^2=4a^3$.}
\end{ex}

\begin{ex}%[2H1Y3-2]% ::Cau 5::
	Thể tích khối lăng trụ có diện tích đáy bằng $4$ và chiều cao bằng $3$ là
	\choice
	{ $4$}
	{ $48$}
	{ $16$}
	{ \True$12$}
	\loigiai{
		Thể tích của khối lăng trụ có chiều cao $h$ và diện tích đáy bằng $B$ là $V=B\cdot h=4\cdot 3=12$.}
\end{ex}

\begin{ex}%[2H1Y3-2]% ::Cau 6::
	Khối lăng trụ có diện tích đáy bằng $24$ $\mathrm{cm}^2$, chiều cao bằng $3$ $\mathrm{cm}$ thì có thể tích bằng
	\choice
	{ $8$ $\mathrm{cm}^3$}
	{ \True $72$ $\mathrm{cm}^3$}
	{ $126$ $\mathrm{cm}^3$}
	{ $24$ $\mathrm{cm}^3$}
	\loigiai{
		Áp dụng công thức tính thể tích khối lăng trụ $V=B\cdot h$ $=24\cdot 3=72$ $\mathrm{cm}^3$ .}
\end{ex}

\begin{ex}%[2H1Y3-2]% ::Cau 7::
	Cho khối lăng trụ có diện tích đáy bằng $4a^2$ và khoảng cách giữa hai đáy bằng $a$. Thể tích của khối lăng trụ đã cho bằng
	\choice
	{ \True $4a^3$}
	{ $\dfrac{1}{3}{a^3}$}
	{ $3a^3$}
	{ $a^3$}
	\loigiai{
		Áp dụng công thức tính thể tích khối lăng trụ $V=B\cdot h$ $=4a^2\cdot a=4a^3$.}
\end{ex}

\begin{ex}%[2H1Y3-2]% ::Cau 8::
	Khối lăng trụ có chiều cao bằng $h$, diện tích đáy bằng $B$ có thể tích là
	\choice
	{ $V=\dfrac{1}{3}B\cdot h$}
	{ $V=\dfrac{1}{2}B\cdot h$}
	{ $V=\dfrac{1}{6}B\cdot h$}
	{ \True $V=B\cdot h$}
	\loigiai{
		Thể tích khối lăng trụ là $V=B\cdot h$.}
\end{ex}

\begin{ex}%[2H1B3-1]% ::Cau 9::
	Biết rằng thể tích của một khối lập phương bằng $8$. Tính tổng diện tích các mặt của hình lập phương đó?
	\choice
	{ $36$}
	{ $27$}
	{ $16$}
	{ \True $24$}
	\loigiai{
		Gọi $a$ là độ dài một cạnh của khối lập phương $\left( a>0 \right)$.\\
		Theo giả thiết ta có $a^3=8\Leftrightarrow a=2$.\\
		Diện tích một mặt của khối lập phương cạnh bằng $2$ là $2^2=4$.\\
		Tổng diện tích các mặt của khối lập phương là $6\cdot 4=24$.}
\end{ex}

\begin{ex}%[2H1Y3-2]% ::Cau 10::
	Khối lăng trụ có diện tích đáy bằng $4$ $\mathrm{cm}^2$, chiều cao bằng $2$ $\mathrm{cm}$ có thể tích bằng
	\choice
	{ \True $8$ $ \mathrm{cm}^3$}
	{ $\dfrac{8}{3}$ $ \mathrm{cm}^3$}
	{ $4$ $ \mathrm{cm}^3$}
	{ $6$ $ \mathrm{cm}^3$}
	\loigiai{
		Thể tích khối lăng trụ là $V=B\cdot h=4\cdot2=8$ $ \mathrm{cm}^3$.}
\end{ex}

\begin{ex}%[2H1Y3-2]% ::Cau 11::
	Thể tích của khối lăng trụ có diện tích đáy $a^2$ và chiều cao $a$ là
	\choice
	{ \True $a^3$}
	{ $\dfrac{a^3}{3}$}
	{ $3a^3$}
	{ $2a^3$}
	\loigiai{
		Thể tích của khối lăng trụ là $V=a^2\cdot a=a^3.$}
\end{ex}

\begin{ex}%[2H1Y3-2]% ::Cau 12::
	Nếu một khối lăng trụ đứng có diện tích đáy bằng $B$ và cạnh bên bằng $h$ thì có thể tích là
	\choice
	{ \True $Bh$}
	{ $\dfrac{1}{3}Bh$}
	{ $\dfrac{1}{2}Bh$}
	{ $3Bh$}
	\loigiai{
		Thể tích của khối lăng trụ là $V=Bh.$}
\end{ex}

\begin{ex}%[2H1Y3-2]% ::Cau 13::
	Cho hình lăng trụ $ABC.A'B'C'$ có diện tích đáy là $15$ và chiều cao của lăng trụ là $10$. Thể tích khối lăng trụ $ABC.A'B'C'$ là
	\choice
	{ $200$}
	{ \True $150$}
	{ $100$}
	{ $50$}
	\loigiai{
		Áp dụng công thức tính thể tích khối lăng trụ $V=B\cdot h=10\cdot 15=150$.}
\end{ex}

\begin{ex}%[2H1B3-2]% ::Cau 14::
	Nếu cạnh của hình lập phương tăng lên gấp $2$ lần thì thể tích của hình lập phương đó sẽ tăng lên bao nhiêu lần?
	\choice
	{ $9$}
	{ \True $8$}
	{ $6$}
	{ $4$}
	\loigiai{
		Ta có thể tích của hình lập phương cạnh $a$ là $a^3$.\\
		Do đó khi tăng cạnh hình lập phương lên $2$ lần thì thể tích là $(2a)^3=8a^3$.}
\end{ex}

\begin{ex}%[2H1B3-2]% ::Cau 15::
	Thể tích của khối lăng trụ đứng tam giác đều có tất cả các cạnh bằng $a$ bằng
	\choice
	{$\dfrac{a^3\sqrt{2}}{3}$}
	{$\dfrac{a^3}{3}$}
	{\True $\dfrac{a^3\sqrt{3}}{4}$}
	{$\dfrac{a^3\sqrt{3}}{6}$}
	\loigiai{
		Diện tích đáy là $\dfrac{a^2\sqrt{3}}{4}$\\
		$\Rightarrow$ Thể tích của khối lăng trụ cần tìm là $V=a\cdot \dfrac{a^2\sqrt{3}}{4}=\dfrac{a^3\sqrt{3}}{4}$.}
\end{ex}

\begin{ex}%[2H1B3-2]% ::Cau 16::
	Tính thể tích của khối lăng trụ tam giác đều có cạnh đáy bằng $2a$ và cạnh bên bằng $a$.
	\choice
	{$\dfrac{a^3\sqrt{3}}{4}$}
	{$a^3$}
	{$\dfrac{a^3\sqrt{3}}{3}$}
	{\True $a^3\sqrt{3}$}
	\loigiai{
		Vì lăng trụ đứng nên đường cao bằng $a$.\\
		Vì đáy là tam giác đều nên diện tích đáy là $S=\dfrac{(2a)^2 \cdot \sqrt{3}}{4}=a^2\sqrt{3}$.\\
		Thể tích của khối lăng trụ cần tìm là $V=S\cdot h=a^2\sqrt{3} \cdot a=a^3\sqrt{3}$.
	}
\end{ex}

\begin{ex}%[2H1B3-2]::Cau 17::
	Tính thể tích $V$ của khối lập phương $ABCD \cdot A'B'C'D'$ biết $AC'=2 a \sqrt{3}$.
	\choice
	{$V=a^3$}
	{$V=24\sqrt{3} a^3$}
	{\True $V=8 a^3$}
	{$V=3\sqrt{3} a^3$}
	\loigiai{
		\immini
		{
			Gọi độ dài cạnh hình lập phương bằng $x$. Suy ra $AC'=x\sqrt{3}$.\\
			Ta có $x\sqrt{3}=2a\sqrt{3}\Leftrightarrow x=2a$.\\
			Thể tích khối lập phương là $V=x^3=8a^3$.
		}
		{
			\begin{tikzpicture}[scale=.7, font=\footnotesize, line join=round, line cap=round, >=stealth]
				\def\a{3.5}
				\def \b{3}
				\coordinate (B) at (0,0);
				\coordinate (A) at (1,1.2);
				\coordinate (C) at (\a,0);
				\coordinate (D) at ($(C)-(B)+(A)$);
				\coordinate (A') at ($(A)+(90:\b)$);
				\coordinate (B') at ($(B)-(A)+(A')$);
				\coordinate (C') at ($(C)-(A)+(A')$);
				\coordinate (D') at ($(D)-(A)+(A')$);
				\draw (B')--(B)--(C)--(D)--(D')--(A')--(B')--(C')--(D') (C)--(C');
				\draw[dashed] (A)--(D) (A')--(A)--(B);
				\foreach \diem/\goc in {A/160,B/-90,C/-90,D/0,B'/180,C'/-25,D'/90,A'/90} \fill[black](\diem) circle (1pt) ($(\diem)+(\goc:3mm)$) node{$\diem$};
			\end{tikzpicture}
		}
	}
\end{ex}

\begin{ex}%[2H1Y3-2]%::Cau 18::
	Cho lăng trụ đứng $ABC.A'B'C'$ có đáy $ABC$ là tam giác vuông tại $A$. Biết $AB=a$, $AC=2a$, $AA'=3a$. Thể tích khối lăng trụ đã cho bằng
	\choice
	{\True $3a^3$}
	{$6a^3$}
	{$a^3$}
	{$3a^2$}
	\loigiai{
		\immini
		{
			Diện tích tam giác $ABC$ là $S_{ABC}=\dfrac{1}{2}\cdot AB\cdot AC=\dfrac{1}{2}\cdot a\cdot 2a=a^2$.\\
			Thể tích của khối lăng trụ đã cho là $V=AA'\cdot S_{ABC}=3a\cdot a^2=3a^3$.}
		{\begin{tikzpicture}[scale=.8, font=\footnotesize, line join=round, line cap=round, >=stealth]
				\def\a{3.5}
				\def \b{3}
				\coordinate (A) at (0,0);
				\coordinate (B) at (\a,0);
				\coordinate (C) at (2.5,-1);
				\coordinate (A') at ($(A)+(-90:\b)$);
				\coordinate (B') at ($(B)+(-90:\b)$);
				\coordinate (C') at ($(C)+(-90:\b)$);
				\draw(A)--(B)--(C)--cycle(A)--(A')--(C')--(B')--(B)(C)--(C');
				\draw[dashed](A')--(B');
				\foreach \diem/\goc in {A/160,B/0,C/0,A'/-100,B'/-60,C'/-90} \fill[black](\diem) circle (1pt) ($(\diem)+(\goc:3mm)$) node{$\diem$};
		\end{tikzpicture}}
	}
\end{ex}

\begin{ex}%[2H1Y3-2]%::Cau 19::
	Thể tích của khối hộp chữ nhật $ABCD.A'B'C'D'$ có các cạnh $AB=4$, $AD=5$ và $AA'=6$ là
	\choice
	{$V=200$}
	{$V=100$}
	{\True $V=120$}
	{$V=130$}
	\loigiai{
		\immini
		{
			Thể tích của khối hộp chữ nhật là $V=AB\cdot AD\cdot AA'=4\cdot 5\cdot 6=120$.}
		{\begin{tikzpicture}[scale=.7, font=\footnotesize, line join=round, line cap=round, >=stealth]
				\def\a{3.5}
				\def \b{3}
				\coordinate (B) at (0,0);
				\coordinate (A) at (1,1.2);
				\coordinate (C) at (\a,0);
				\coordinate (D) at ($(C)-(B)+(A)$);
				\coordinate (A') at ($(A)+(90:\b)$);
				\coordinate (B') at ($(B)-(A)+(A')$);
				\coordinate (C') at ($(C)-(A)+(A')$);
				\coordinate (D') at ($(D)-(A)+(A')$);
				\draw (B')--(B)--(C)--(D)--(D')--(A')--(B')--(C')--(D') (C)--(C');
				\draw[dashed] (A)--(D) (A')--(A)--(B);
				\foreach \diem/\goc in {A/160,B/-90,C/-90,D/0,B'/180,C'/-25,D'/90,A'/90} \fill[black](\diem) circle (1pt) ($(\diem)+(\goc:3mm)$) node{$\diem$};
		\end{tikzpicture}}
	}
\end{ex}

\begin{ex}%[2H1Y3-2]%::Cau 20::
	Cho khối lăng trụ đứng $ABCD.A'B'C'D'$ có đáy $ABCD$ là hình thoi có độ dài hai đường chéo $AC=2a$ và $BD=a$, cạnh bên $AA'=3a$. Thể tích $V$ của khối lăng trụ đã cho là
	\choice
	{$V=6a^3$}
	{$V=12a^3$}
	{\True $V=3a^3$}
	{$V=2a^3$}
	\loigiai{
		\immini
		{
			Diện tích đáy hình thoi là $S_{ABCD}=\dfrac{1}{2}AC\cdot BD=\dfrac{1}{2}a\cdot 2a=a^2$.\\
			Thể tích khối lăng trụ đã cho là $V=S_{ABCD}\cdot AA'=a^2\cdot 3a=3a^3$.}
		{\begin{tikzpicture}[scale=.7, font=\footnotesize, line join=round, line cap=round, >=stealth]
				\def\a{3.5}
				\def \b{3}
				\coordinate (B) at (0,0);
				\coordinate (A) at (1,1.2);
				\coordinate (C) at (\a,0);
				\coordinate (D) at ($(C)-(B)+(A)$);
				\coordinate (A') at ($(A)+(90:\b)$);
				\coordinate (B') at ($(B)-(A)+(A')$);
				\coordinate (C') at ($(C)-(A)+(A')$);
				\coordinate (D') at ($(D)-(A)+(A')$);
				\draw (B')--(B)--(C)--(D)--(D')--(A')--(B')--(C')--(D') (C)--(C');
				\draw[dashed] (A)--(D) (A')--(A)--(B);
				\foreach \diem/\goc in {A/160,B/-90,C/-90,D/0,B'/180,C'/-25,D'/90,A'/90} \fill[black](\diem) circle (1pt) ($(\diem)+(\goc:3mm)$) node{$\diem$};
		\end{tikzpicture}}
	}
\end{ex}
\Closesolutionfile{ans}
\subsection{Bảng đáp án}
\inputansbox{8}{ans/ANS-DANG-13}