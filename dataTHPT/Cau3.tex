%Dạng 1
\setcounter{ex}{0}
\section{Đạo hàm hàm lũy thừa - Hàm mũ - logarit}
\subsection{Kiến thức cần nhớ}
\begin{khung}
	\begin{itemize}
		\item $\left(x^\alpha\right)'=\alpha x^{\alpha -1} \Rightarrow \left(u^\alpha\right)'=\alpha\cdot u^{\alpha -1}\cdot u'$.
		\item $(a^u)'=a^u \ln a \cdot u'\Rightarrow (a^x)'=a^x \cdot \ln a $.
		\item $(\mathrm{e}^u)'=u'\cdot \mathrm{e}^u \Rightarrow (\mathrm{e}^x)'=\mathrm{e}^x$.
		\item $( \log_a u)'=\dfrac{u'}{u\ln a} \Rightarrow (\log_a x)'=\dfrac{1}{x \ln a}$.
		\item $\left(\ln x\right)'=\dfrac{1}{x} \Rightarrow (\ln u)'=\dfrac{u'}{u}$.
	\end{itemize}

\end{khung}
\subsection{Bài tập mẫu}
\Opensolutionfile{ans}[ans/ANS-DANG-1]
\begin{khung}
	\begin{vd}[Đề Minh họa BGD 2022-2023]%[2D2Y2-2]
			Trên khoảng $(0; +\infty)$, đạo hàm của hàm số $y=x^\pi$ là
			\choice
			{\True $y'=\pi x^{\pi-1}$}
			{$y'=x^{\pi-1}$}
			{$y'=\dfrac1\pi x^{\pi-1}$}
			{$y'=\pi x^\pi$}
			\loigiai{
				Đạo hàm của hàm số $y=x^\pi$ là $y'=\pi x^{\pi-1}$.
			}
	\end{vd}
\end{khung}
\subsection{Bài tập tương tự và phát triển}
\begin{ex}%[2D2Y2-2]
	Trên khoảng $(0; +\infty)$, đạo hàm của hàm số $y=x^\mathrm{e}$ là
	\choice
	{\True $y'=\mathrm{e} \cdot x^{\mathrm{e}-1}$}
	{$y'=x^{\mathrm{e}-1}$}
	{$y'=\dfrac{1}{\mathrm{e}} x^{\mathrm{e}-1}$}
	{$y'=\mathrm{e} \cdot x^\mathrm{e}$}
	\loigiai{
		Đạo hàm của hàm số $y=x^\mathrm{e}$ là $y'=\mathrm{e} \cdot x^{\mathrm{e}-1}$.
	}
\end{ex}
\begin{ex}%[2D2Y2-2]
	Tính đạo hàm của hàm số $y=\left(x^2+x\right)^\alpha$ với $\alpha$ là hằng số.
	\choice
	{$2\alpha\left(x^2+x\right)^{\alpha-1}$}
	{$\alpha\left(x^2+x\right)^{\alpha+1}\left(2x+1\right)$}
	{\True $\alpha\left(x^2+x\right)^{\alpha-1}\left(2x+1\right)$}
	{$\alpha\left(x^2+x\right)^{\alpha-1}$}
	\loigiai{
		Ta có $y'=\alpha\left(x^2+x\right)^{\alpha-1}\cdot\left(x^2+x\right)'=\alpha\left(x^2+x\right)^{\alpha-1}\left(2x+1\right)$.
	}
\end{ex}
\begin{ex}%[2D2Y2-2]
	Đạo hàm của hàm số $y=\sqrt[3]{x^2\cdot\sqrt{x^3}}$ trên khoảng $\left(0;+\infty\right)$ là
	\choice
	{\True $\dfrac{7}{6}\sqrt[6]{x}$}
	{$\dfrac{4}{3}\sqrt[3]{x}$}
	{$7\sqrt[6]{x}$}
	{$\sqrt[9]{x}$}
	\loigiai{
		Ta có $y=x^{\frac{7}{6}}$. Do đó $y'=\dfrac{7}{6}\cdot x^{\frac{1}{6}}=\dfrac{7}{6}\cdot\sqrt[6]{x}$.
	}
\end{ex}
\begin{ex}%[2D2Y2-2]
	Cho hàm số $y=x^\pi$. Giá trị của $y''\left(1\right)$ bằng
	\choice
	{$\ln ^2\pi$}
	{$\pi\ln \pi$}
	{$0$}
	{\True $\pi\left(\pi-1\right)$}
	\loigiai{
		Đạo hàm cấp một: $y'=\pi\cdot x^{\pi-1}$.\\
		Đạo hàm cấp hai: $y''=\pi\cdot\left(\pi-1\right)\cdot x^{\pi-2}$. Do đó $y''\left(1\right)=\pi\left(\pi-1\right)$.
	}
\end{ex}

\begin{ex}%[2D2Y4-2]
	Tính đạo hàm của hàm số $ y={{2019}^x}$.
	\choice
	{ $y'={{2019}^{x-1}}$}
	{ $y'={{2019}^x}$}
	{\True $y'={{2019}^x}\cdot \ln 2019$}
	{ $y'=x{{\cdot 2019}^{x-1}}$}
	\loigiai{
		Áp dụng công thức ${{( {a^x} )}^{\prime }}={a^x}\cdot \ln a$\\
		Ta có $ y={{2019}^x}\Rightarrow y'={{2019}^x}\cdot \ln 2019$.}
\end{ex}
\begin{ex}%[2D2Y4-2]
	Đạo hàm của hàm số $ y={5^x}$ là
	\choice
	{ ${5^x}\cdot \ln x$}
	{ $ x{{\cdot 5}^{x-1}}$}
	{\True ${5^x}\cdot \ln 5$}
	{ ${5^x}$}
	\loigiai{
		Ta có $( {5^x} )'={5^x}\cdot \ln 5$.}
\end{ex}

\begin{ex}%[2D2Y4-2]
	Đạo hàm của hàm số $ f( x )={\left( \dfrac{1}{2} \right)^x}$ là
	\choice
	{\True $f'( x )=-{{\left( \dfrac{1}{2} \right)}^x}\cdot \ln 2 $}
	{ $f'( x )={{\left( \dfrac{1}{2} \right)}^x}\cdot \log 2 $}
	{ $f'( x )=-{{\left( \dfrac{1}{2} \right)}^x}\cdot \log 2 $}
	{ $f'( x )={{\left( \dfrac{1}{2} \right)}^x}\cdot \ln 2 $}
	\loigiai{
		Ta có $f'( x )={{\left( \dfrac{1}{2} \right)}^x}\cdot \ln \dfrac{1}{2}=-{{\left( \dfrac{1}{2} \right)}^x}\cdot \ln 2$.}
\end{ex}
\begin{ex}%[2D2Y4-2]
	Đạo hàm của hàm số $y=2^{2x^2 + x}$ là
	\choice
	{$2^{2x^2 + x}\cdot\ln 2$}
	{\True $(4x + 1)\cdot 2^{2x^2 + x}\cdot\ln 2$}
	{ $\left(2x^2 + x\right) 2^{2x^2 + x}\ln 2$}
	{$(4x + 1)\ln \left(2x^2 + x\right)$}
	\loigiai{
		$y'=\left(2x^2 + x\right)'\cdot 2^{2x^2 + x}\cdot\ln 2=(4x + 1)\cdot 2^{2x^2 + x}\cdot\ln 2$.
	}
\end{ex}
\begin{ex}%[2D2Y4-2]
	Đạo hàm của hàm số $ y={\mathrm{e}^{2x-3}}$ là
	\choice
	{ $y'=2{\mathrm{e}^{2x}}$}
	{ $y'={\mathrm{e}^{2x-3}}$}
	{ $y'=( 2x-3 ){\mathrm{e}^{2x-3}}$}
	{\True $y'=2{\mathrm{e}^{2x-3}}$}
	\loigiai{
		Ta có $y'={{( {\mathrm{e}^{2x-3}} )}^{\prime }}$ $\Rightarrow y'=2{\mathrm{e}^{2x-3}}$.}
\end{ex}
\begin{ex}%[2D2Y4-2]
	Đạo hàm của hàm số $y=\mathrm{e}^{x^2 + x}$ là
	\choice
	{$\left(x^2 + x\right)\cdot\mathrm{e}^{2x + 1}$}
	{$(2x + 1)\cdot\mathrm{e}^{2x + 1}$}
	{\True $(2x + 1)\cdot\mathrm{e}^{x^2 + x}$}
	{$(2x + 1)\cdot\mathrm{e}^x$}
	\loigiai{
		Học sinh ghi công thức $\left(\mathrm{e}^{u}\right)'=u'\cdot \mathrm{e}^u$\\
		$\Rightarrow y'=\left(x^2 + x\right)'\cdot \mathrm{e}^{x^2 + x} = (2x + 1)\cdot \mathrm{e}^{x^2 + x}$.
	}
\end{ex}
\begin{ex}%[2D2Y4-2]
	Tính đạo hàm của hàm số $y=\mathrm{e}^{\cos 2x}$.
	\choice
	{$y'=\sin 2x\cdot\mathrm{e}^{\cos 2x}$}
	{$y'=\dfrac{1}{2}\sin 2x\cdot\mathrm{e}^{\cos 2x}$}
	{ $y'=2\cos 2x\cdot\mathrm{e}^{\sin 2x}$}
	{\True $y'= - 2\sin 2x\cdot\mathrm{e}^{\cos 2x}$}
	\loigiai{
		Ta có $(\cos u)'= -u'\sin u$.\\
		$\Rightarrow y'=(\cos2x)'\cdot\mathrm{e}^{\cos 2 x}= -2 \sin 2 x \cdot \mathrm{e}^{\cos 2 x} $.
	}
\end{ex}

\begin{ex}%[2D2Y4-2]
	Hàm số $ f( x )={2^{{x^2}+3x+1}}$ có đạo hàm là
	\choice
	{ $f'( x )=\dfrac{2x+3}{{2^{{x^2}+3x+1}}\ln 2}$}
	{\True $f'( x )={2^{{x^2}+3x+1}}( 2x+3 )\ln 2$}
	{ $f'( x )=\dfrac{2x+3}{{2^{{x^2}+3x+1}}}$}
	{ $f'( x )={2^{{x^2}+3x+1}}( 2x+3 )$}
	\loigiai{
		${{( {2^{{x^2}+3x+1}} )}^{\prime }}={2^{{x^2}+3x+1}}\cdot {{( {x^2}+3x+1 )}^{\prime }}\cdot \ln 2$$={2^{{x^2}+3x+1}}\cdot ( 2x+3 )\cdot \ln 2$.}
\end{ex}
\begin{ex}%[2D2Y4-2]
	Đạo hàm của hàm số $ y={3^x}+1$ là
	\choice
	{\True $y'={3^x}\ln 3$}
	{ $y'={3^x}$}
	{ $y'=\dfrac{{3^x}}{\ln 3}$}
	{ $y'=x{3^{x-1}}$}
	\loigiai{
		Đạo hàm của hàm số $ y={3^x}+1$ là $y'={3^x} \cdot \ln 3$.}
\end{ex}

\begin{ex}%[2D2Y4-2]
	Đạo hàm của hàm số $y=5^{\sin x}$ là
	\choice
	{\True $5^{\sin x} \cdot \ln 5 \cdot \cos x$}
	{$5^{\sin x}\cdot\cos x$}
	{ $5^{\sin x - 1}\cdot\sin x$}
	{$5^{\sin x}\cdot\ln 5$}
	\loigiai{
	Ta có	$\left(a^u\right)'= u'\cdot a^u\cdot \ln u$ và $(\sin x)'=\cos x $\\
		$\Rightarrow y'= 5^{\sin x}\cdot\ln 5\cdot \cos x$.
	}
\end{ex}
%6

%7
\begin{ex}%[2D2B4-2]
	Đạo hàm của hàm số $y=\left(x^2 - 2x + 2\right) \mathrm{e}^x$ là
	\choice
	{$\left(x^2 + 2\right) \mathrm{e}^x$}
	{\True $x^2\mathrm{e}^x$}
	{$(2x - 2) \mathrm{e}^x$}
	{$ - 2x \mathrm{e}^x$}
	\loigiai{
		$(u\cdot v)'= u'\cdot v + v'\cdot u$\\
		$\Rightarrow y'=\left(x^2 - 2x + 2\right)'\mathrm{e}^x + \left(\mathrm{e}^x\right)'\cdot \left(x^2 - 2x + 2\right) = (2x-2)\cdot \mathrm{e}^{x} + \mathrm{e}^{x}\cdot \left(x^2-2x+2\right)= x^2\mathrm{e}^{x}$.
	}
\end{ex}

\begin{ex}%[2D2Y4-2]
	Đạo hàm của hàm số $y=\log_2\left(2x+1\right)$ là
	\choice
	{$\dfrac{2}{\left(2x+1\right)\cdot\ln x}$}
	{\True $\dfrac{2}{\left(2x+1\right)\cdot\ln 2}$}
	{$\dfrac{2\cdot\ln 2}{x+1}$}
	{$\dfrac{2}{\left(x+1\right)\cdot\ln 2}$}
	\loigiai{
		Áp dụng công thức $\left(\log_au\right)'=\dfrac{u'}{u\cdot\ln a}$.\\
		Ta có $y'=\dfrac{\left(2x+1\right)'}{\left(2x+1\right)\cdot\ln 2}=\dfrac{2}{\left(2x+1\right)\cdot\ln 2}$.
	}
\end{ex}

\begin{ex}%[2D2Y4-2]
	Đạo hàm của hàm số $y=\log_2\left(x^2+1\right)$ là
	\choice
	{\True $\dfrac{2x}{\left(x^2+1\right)\cdot\ln 2}$}
	{$\dfrac{1}{x^2+1}$}
	{$\dfrac{1}{\left(x^2+1\right)\cdot\ln 2}$}
	{$\dfrac{2x}{x^2+1}$}
	\loigiai{
		Áp dụng công thức $\left(\log_au\right)'=\dfrac{u'}{u\cdot\ln a}$.\\ 
		Ta có $y'=\dfrac{\left(x^2+1\right)'}{\left(x^2+1\right)\cdot\ln 2}=\dfrac{2x}{\left(x^2+1\right)\cdot\ln 2}$.		
	}
\end{ex}

\begin{ex}%[2D2Y4-2]
	Đạo hàm của hàm số $y=\log\left(x^2-x\right)$ là
	\choice
	{$\dfrac{1}{\left(x^2-x\right)\cdot\ln 10}$}
	{$\dfrac{2x-1}{x^2-x}$}
	{$\True \dfrac{2x-1}{\left(x^2-x\right)\cdot\ln 10}$}
	{$\dfrac{2x-1}{\left(x^2-1\right)\cdot\log\mathrm{e}}$}
	\loigiai{
		Áp dụng công thức $\left(\log_au\right)'=\dfrac{u'}{u\cdot\ln a}$.\\ 
		Ta có $y'=\dfrac{\left(x^2-x\right)'}{\left(x^2-x\right)\cdot\ln 10}=\dfrac{2x-1}{\left(x^2-x\right)\cdot\ln 10}$.		
	}
\end{ex}

\begin{ex}%[2D2Y4-2]
	Đạo hàm của hàm số $y=\log\left(\mathrm{e}^x+2\right)$ là
	\choice
	{\True $\dfrac{\mathrm{e}^x}{\left(\mathrm{e}^x+2\right)\cdot\ln 10}$}
	{$\dfrac{\mathrm{e}^x}{\mathrm{e}^x+2}$}
	{$\dfrac{1}{\left(\mathrm{e}^x+2\right)\cdot\ln 10}$}
	{$\dfrac{1}{\mathrm{e}^x+2}$}
	\loigiai{
		Áp dụng công thức $\left(\log_au\right)'=\dfrac{u'}{u\cdot\ln a}$.\\ 
		Ta có $y'=\dfrac{\left(\mathrm{e}^x+2\right)'}{\left(\mathrm{e}^x+2\right)\cdot\ln 10}=\dfrac{\mathrm{e}^x}{\left(\mathrm{e}^x+2\right)\cdot\ln 10}$.		
	}
\end{ex}

\begin{ex}%[2D2Y4-2]
	Đạo hàm của hàm số $y=\mathrm{e}^x-\ln \left(3x\right)$ là
	\choice
	{$\mathrm{e}^x-\dfrac{1}{3x}$}
	{\True $\mathrm{e}^x-\dfrac{1}{x}$}
	{$\mathrm{e}^x-\dfrac{3}{x}$}
	{$\mathrm{e}^x+\dfrac{1}{x}$}
	\loigiai{
		Ta có $y'=\mathrm{e}^x-\dfrac{\left(3x\right)'}{3x}=\mathrm{e}^x-\dfrac{1}{x}$.	
	}
\end{ex}
\Closesolutionfile{ans}
%======================
\subsection{Bảng đáp án}
\inputansbox{10}{ans/ANS-DANG-1}

