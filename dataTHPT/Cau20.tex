%Dạng 1
\setcounter{section}{19}
\setcounter{ex}{0}
\section{Đường tiệm cận}
\subsection{Kiến thức cần nhớ}
\begin{khung}
	\subsubsection{Đường tiệm cận ngang}
		Cho hàm số $y=f(x)$ xác định trên một khoảng vô hạn (là khoảng dạng $(a;+\infty)$, $(-\infty;a)$ hoặc $(-\infty;+\infty)$). Đường thẳng $y=y_0$ là đường \textbf{tiệm cận ngang} (hay tiệm cận ngang) của đồ thị $y=f(x)$ nếu ít nhất một trong các điều kiện sau được thỏa mãn
		\begin{equation*}
			\lim\limits_{x\rightarrow+\infty}f(x)=y_0, \lim\limits_{x\rightarrow-\infty}f(x)=y_0.
		\end{equation*}
	Như vậy, để tìm tìm cận ngang của đồ thị hàm số ta chỉ cần tính giới hạn của hàm số đó tại vô cực.
	\subsubsection{Đường tiệm cận đứng}
		Đường thẳng $x=x_0$ được gọi là đường \textbf{tiệm cận đứng} (hay tiệm cận đứng) của đồ thị hàm số $y=f(x)$ nếu ít nhất một trong các điều kiện sau được thỏa mãn
		\begin{equation*}
			\lim\limits_{x\rightarrow x_0^+}f(x)=+\infty,\lim\limits_{x\rightarrow x_0^-}f(x)=-\infty,\lim\limits_{x\rightarrow x_0^+}f(x)=-\infty,\lim\limits_{x\rightarrow x_0^-}f(x)=+\infty.
		\end{equation*}
\end{khung}
\subsection{Bài tập mẫu}
\Opensolutionfile{ans}[ans/ANS-DANG-20]
\begin{khung}
	\begin{vd}[Đề minh họa BGD 2022-2023]%[2D1B4-3]%Phan Trung Hiếu
		Tiệm cận ngang của đồ thị hàm số $y=\dfrac{2x+1}{3x-1}$ là đường thẳng có phương trình
		\choice
		{$y=\dfrac{1}{3}$}	
		{$y=-\dfrac{2}{3}$}
		{$y=-\dfrac{1}{3}$}
		{\True $y=\dfrac{2}{3}$}
		\loigiai{
			Ta có: $\lim\limits_{x\rightarrow\pm\infty} y =\lim\limits_{x\rightarrow\pm\infty}\dfrac{2x+1}{3x-1} = \dfrac{2}{3}$.\\
			Suy ra $y=\dfrac{2}{3}$ là đường tiệm cận ngang của đồ thị hàm số.
		}
	\end{vd}
\end{khung}
\subsection{Bài tập tương tự và phát triển}
%Câu1
\begin{ex}%[2D1B4-3]%Phan Trung Hiếu
	Đồ thị của hàm số nào trong các hàm số sau đây có tiệm cận ngang?
	\choice
	{\True $y=\dfrac{2-x}{x}$}
	{$y=x^3-x^2+x-3$}
	{$y=x^4-x^2-2$}
	{$y=\dfrac{3x^2-1}{x+1}$}
	\loigiai{
		Xét đáp án $y=x^4-x^2-2$ có $\lim\limits_{x\to\pm+\infty}=\pm\infty$ nên đồ thị hàm số không có tiệm cận ngang.\\
		Xét đáp án $y=\dfrac{3x^2-1}{x+1}$ có $\lim\limits_{x\to\pm\infty}=\pm\infty$ nên đồ thị hàm số không có tiệm cận ngang.\\
		Xét đáp án $y=\dfrac{2-x}{x}$ có tiệm cận ngang là đường thẳng $y=-1$.\\
		Xét đáp án $y=x^3-x^2+x-3$ có $\lim\limits_{x\to\pm\infty}=\pm\infty$ nên đồ thị hàm số không có tiệm cận ngang.
	}
\end{ex}
%Câu2
\begin{ex}%[2D1B4-3]%Phan Trung Hiếu
	Đường thẳng nào dưới đây là tiệm cận ngang của đồ thị hàm số $y=10+\dfrac{1}{x-10}$?
	\choice
	{$y=-10$}
	{$x=-10$}
	{\True $y=10$}
	{$x=10$}
	\loigiai{
		Ta có $\lim\limits_{x\to\pm\infty}y=\lim\limits_{x\to\pm\infty}\left(10+\dfrac{1}{x-10}\right)=10\Rightarrow y=10$ là đường tiệm cận ngang của đồ thị hàm số.
	}
\end{ex}
%Câu3
\begin{ex}%[2D1B4-3]%Phan Trung Hiếu
	Đồ thị hàm số $y=\dfrac{x-2}{x+1}$ có đường tiệm cận đứng là
	\choice
	{$x=1$}
	{\True $x=-1$}
	{$y=1$}
	{$y=-1$}
	\loigiai{
		Ta có $\lim\limits_{x\to(-1)^\pm}(x-2)=-3$ và $\lim\limits_{x\to(-1)^\pm}(x+1)=0$ nên \\
		$\lim\limits_{x\to(-1)^+}y=\lim\limits_{x\to(-1)^+}\dfrac{x-2}{x+1}=-\infty$ và $\lim\limits_{x\to(-1)^-}y=\lim\limits_{x\to(-1)^-}\dfrac{x-2}{x+1}=+\infty$.\\
		Do đó đồ thị hàm số $y=\dfrac{x-2}{x+1}$ có đường tiệm cận đứng là $x=-1$.
	}
\end{ex}
%Câu4
\begin{ex}%[2D1B4-3]%Phan Trung Hiếu
	Tìm tiệm cận ngang của đồ thị hàm số $y=\dfrac{2x+1}{x-1}$.
	\choice
	{$x=-1$}
	{$y=-1$}
	{$x=1$}
	{\True $y=2$}
	\loigiai{
		Vì $\lim\limits_{x\to\pm\infty}y=\lim\limits_{x\to\pm\infty}\dfrac{2x+1}{x-1}=\lim\limits_{x\to\pm\infty}\dfrac{2+\frac{1}{x}}{1-\frac{1}{x}}=\dfrac{2+0}{1-0}=2$.\\
		Do đó, tiệm cận ngang của đồ thị hàm số $y=\dfrac{2x+1}{x-1}$ là $y=2$.
	}
\end{ex}
%Câu5
\begin{ex}%[2D1B4-3]%Phan Trung Hiếu
	 Số đường tiệm cận của đồ thị hàm số $y=\dfrac{x-1}{x+1}$ là
	\choice
	{$2$}
	{$1$}
	{$3$}
	{\True $0$}
	\loigiai{
		Ta có $\lim\limits_{x\to\pm\infty}=\lim\limits_{x\to\pm\infty}\dfrac{x-1}{x+1}=1$, suy ra $y=1$ là tiệm cận ngang của đồ thị hàm số đã cho.\\
		$\lim\limits_{x\to(-1)^-}\dfrac{x-1}{x+1}=+\infty$, $\lim\limits_{x\to(-1)^+}\dfrac{x-1}{x+1}=-\infty$, suy ra $x=-1$ là tiệm cận đứng của đồ thị hàm số đã cho.\\
		Vậy đồ thị hàm số đã cho có hai đường tiệm cận.
	}
\end{ex}
%Câu6
\begin{ex}%[2D1B4-3]%Phan Trung Hiếu
	Đường thẳng $x=1$ là tiệm cận đứng của đồ thị hàm số nào sau đây?
	\choice
	{$y=\dfrac{2x^2+3x+2}{2-x}$}
	{$y=\dfrac{2x-2}{x+2}$}
	{$y=\dfrac{1+x^2}{1+x}$}
	{\True $y=\dfrac{1+x}{1-x}$}
	\loigiai{
		Đồ thị hàm số $y=\dfrac{1+x}{1-x}$ có tiệm cận đứng là đường thẳng $x=1$.
	}
\end{ex}
%Câu7
\begin{ex}%[2D1B4-3]%Phan Trung Hiếu
	Đường thẳng $y=-1$ là tiệm cận ngang của đồ thị hàm số nào sau đây?
	\choice
	{\True $y=\dfrac{1+x}{1-x}$}
	{$y=\dfrac{2x^2+3x+2}{2-x}$}
	{$y=\dfrac{2x-2}{x+2}$}
	{$y=\dfrac{1+x^2}{1+x}$}
	\loigiai{
		Ta có $\lim\limits_{x\to+\infty}\dfrac{1+x}{1-x}=-1$. Suy ra $y=-1$ là TCN của đồ thị hàm số $y=\dfrac{1+x}{1-x}$.
	}
\end{ex}
%Câu8
\begin{ex}%[2D1B4-3]%Phan Trung Hiếu
	Tiệm cận đứng của đồ thị hàm số $y=\dfrac{2x+1}{2x-1}$ là
	\choice
	{$x=1$}
	{\True $x=\dfrac{1}{2}$}
	{$y=\dfrac{1}{2}$}
	{$y=1$}
	\loigiai{
		Ta có $\lim\limits_{x\to(\frac{1}{2})^+}\dfrac{2x+1}{2x-1}=+\infty$ và $\lim\limits_{x\to(\frac{1}{2})^-}\dfrac{2x+1}{2x-1}=-\infty$, do đó đường thẳng $x=1$ là đường tiệm cận đứng của đồ thị hàm số.
	}
\end{ex}
%Câu9
\begin{ex}%[2D1B4-3]%Phan Trung Hiếu
	Đồ thị hàm số $y=\dfrac{x+3}{x-2}$ có các đường tiệm cận đứng và tiệm cận ngang lần lượt là
	\choice
	{\True $x=2$ và $y=1$}
	{$x=1$ và $y=2$}
	{$x=2$ và $y=-3$}
	{$x=-2$ và $y=1$}
	\loigiai{
		Ta có $\lim\limits_{x\to\pm\infty}y=\lim\limits_{x\to\pm\infty}\dfrac{x+3}{x-2}=\lim\limits_{x\to\pm\infty}\dfrac{1+\dfrac{3}{x}}{1-\dfrac{2}{x}}=1$ nên $y=1$ là đường tiệm cận ngang của đồ thị hàm số.\\
		Và $\lim\limits_{x\to2^+}\dfrac{x+3}{x-2}=+\infty$ và $\lim\limits_{x\to2^-}\dfrac{x+3}{x-2}=-\infty$ nên đồ thị nhận đường thẳng $x=2$ là tiệm cận đứng.
	}
\end{ex}
%Câu10
\begin{ex}%[2D1B4-3]%Phan Trung Hiếu
	 Cho hàm số $y=\dfrac{x-2}{x-1}$. Đường tiệm cận đứng của đồ thị hàm số là
	\choice
	{$x=2$}
	{$y=2$}
	{\True $x=1$}
	{$y=1$}
	\loigiai{
		Ta có $\lim\limits_{x\to1^+}\dfrac{x-2}{x-1}=-\infty$ và $\lim\limits_{x\to1^-}\dfrac{x-2}{x-1}=+\infty$, do đó đường thẳng $x=1$ là đường tiệm cận đứng của đồ thị hàm số.
	}
\end{ex}
%Câu11
\begin{ex}%[2D1B4-3]%Phan Trung Hiếu
	Đường tiệm ngang của đồ thị hàm số $y=\dfrac{2x-6}{x-2}$ là
	\choice
	{$y-3=0$}
	{$x-2=0$}
	{$x-3=0$}
	{\True $y-2=0$}
	\loigiai{
		Ta có $\lim\limits_{x\to\pm\infty}y=\lim\limits_{x\to\pm\infty}\dfrac{2x-6}{x-2}=2$. Suy ra đường thẳng $y=2$ là tiệm cận ngang của đồ thị hàm số.
	}
\end{ex}
%Câu12
\begin{ex}%[2D1B4-3]%Phan Trung Hiếu
	Đường tiệm cận đứng và tiệm cận ngang của đồ thị hàm số $y=\dfrac{1-2x}{-x+2}$ là
	\choice
	{$x=-2$, $y=-2$}
	{$x=-2$, $y=-2$}
	{$x=-2$, $y=-2$}
	{\True $x=2$, $y=2$}
	\loigiai{
		Đồ thị của hàm số $y=\dfrac{ax+b}{cx+d}$ có đường tiệm cận ngang $y=\dfrac{a}{c}$ và tiệm cận đứng $x=-\dfrac{d}{c}$.\\
		Vậy đồ thị của hàm số $y=\dfrac{1-2x}{-x+2}$ có đường tiệm cận ngang $y=2$ và tiệm cận đứng $x=2$.
	}
\end{ex}
%Câu13
\begin{ex}%[2D1B4-3]%Phan Trung Hiếu
	Đồ thị hàm số $y=\dfrac{x-1}{x^2+1}$ có tất cả bao nhiêu đường tiệm cận (nếu chỉ tính TCĐ và TCN)?
	\choice
	{$3$}
	{$2$}
	{$0$}
	{\True $1$}
	\loigiai{
		Ta có $\lim\limits_{x\to\pm\infty}\dfrac{x-1}{x^2+1}=0$ nên đường tiệm cận ngang là $y=0$.\\
		Đồ thị hàm số không có tiệm cận đứng.
	}
\end{ex}
%Câu14
\begin{ex}%[2D1B4-3]%Phan Trung Hiếu
	Tiệm cận đứng của đồ thị hàm số $y=\dfrac{x-2}{x+3}$ là
	\choice
	{\True $x=-3$}
	{$y=1$}
	{$x=1$}
	{$y=-1$}
	\loigiai{
		Ta có $\lim\limits_{x\to(-3)^+}\dfrac{x-2}{x+3}=+\infty$ và $\lim\limits_{x\to(-3)^-}\dfrac{x-2}{x+3}=-\infty$, do đó đường thẳng $x=-3$ là đường tiệm cận đứng của đồ thị hàm số.
	}
\end{ex}
%Câu15
\begin{ex}%[2D1B4-3]%Phan Trung Hiếu
	Đồ thị của hàm số nào sau đầy \textbf{không} có tiệm cận ngang?
	\choice
	{$y=\dfrac{1}{2x^2+x}$}
	{$y=\textrm{e}^x$}
	{\True $y=2x^2+x$}
	{$y=\dfrac{2x+1}{x+2}$}
	\loigiai{
		Vì $\lim\limits_{x\to\pm\infty}y=+\infty$ nên đồ thị hàm số $y=2x^2+x$ không có tiệm cận ngang.
	}
\end{ex}
%Câu16
\begin{ex}%[2D1B4-3]%Phan Trung Hiếu
	Tiệm cận ngang của đồ thị hàm số $y=\dfrac{2-x}{x+1}$ là
	\choice
	{\True $y=-1$}
	{$y=2$}
	{$x=-1$}
	{$x=2$}
	\loigiai{
		Ta có $\lim\limits_{x\to+\infty}\dfrac{2-x}{x+1}=-1$ và $\lim\limits_{x\to-\infty}\dfrac{2-x}{x+1}=-1$.\\
		Suy ra $y=-1$ là tiệm cận ngang của đồ thị.
	}
\end{ex}
%Câu17
\begin{ex}%[2D1B4-3]%Phan Trung Hiếu
	Đường thẳng nào dưới đây là tiệm cận ngang của đồ thị hàm số $y=\dfrac{1-4x}{2x-1}$?
	\choice
	{$y=\dfrac{1}{2}$}
	{\True $y=-2$}
	{$y=2$}
	{$y=4$}
	\loigiai{
		Vì $\lim\limits_{x\to\pm\infty}y=\lim\limits_{x\to\pm\infty}\dfrac{\frac{1}{x}-4}{2-\frac{1}{x}}=-2$. Suy ra $y=-2$ là tiệm cận ngang của hàm số.
	}
\end{ex}
%Câu18
\begin{ex}%[2D1B4-3]%Phan Trung Hiếu
	Đồ thị hàm số $y=\dfrac{2x+1}{x+1}$ có tiệm cận đứng là
	\choice
	{$y=2$}
	{\True $x=-1$}
	{$y=-1$}
	{$x=1$}
	\loigiai{
		Hàm số $y=\dfrac{2x+1}{x+1}$ có tập xác định: $\mathscr{D}=\mathbb{R}\setminus\{-1\}$.\\
		Ta có $\lim\limits_{x\to(-1)^+}\dfrac{2x+1}{x+1}=-\infty$ và $\lim\limits_{x\to(-1)^-}\dfrac{2x+1}{x+1}=+\infty$ nên $x=-1$ là tiệm cận đứng của đồ thị hàm số $y=\dfrac{2x+1}{x+1}$.
	}
\end{ex}
%Câu19
\begin{ex}%[2D1B4-3]%Phan Trung Hiếu
	Đồ thị hàm số nào sau đây có đường tiệm cận đứng là $x=1$?
	\choice
	{$y=\dfrac{x-1}{x}$}
	{$y=\dfrac{2x}{1+x^2}$}
	{\True $y=\dfrac{2x}{1-x}$}
	{$y=\dfrac{x-1}{x+1}$}
	\loigiai{
		\begin{itemize}
			\item Đồ thị hàm số $y=\dfrac{x-1}{x}$ có tiệm cận đứng $x=0$ loại đáp án A.
			\item Hàm số $y=\dfrac{2x}{1+x^2}$ xác định với $\forall x\in \mathbb{R}$ suy ra đồ thị không có tiệm cận đứng.
			\item Đồ thị hàm số $y=\dfrac{x-1}{x+1}$ có tiệm cận đứng $x=-1$ loại đáp án D.
			\item Đồ thị hàm số $y=\dfrac{2x}{1-x}$ có tiệm cận ngang $y=-2$ và tiệm cận đứng $x=1$ (thỏa mãn).
		\end{itemize}
	}
\end{ex}
%Câu20
\begin{ex}%[2D1B4-3]%Phan Trung Hiếu
	Đường thẳng nào dưới đây là tiệm cận đứng của đồ thị hàm số $y=\dfrac{2-x}{x+3}$?
	\choice
	{$x=2$}
	{\True $x=-3$}
	{$y=-1$}
	{$y=-3$}
	\loigiai{
		TXĐ: $\mathscr{D}=\mathbb{R}\setminus\{-3\}$.\\
		Ta có $\lim\limits_{x\to(-3)^-}y=\lim\limits_{x\to(-3)^-}\dfrac{2-x}{x+3}=-\infty$, $\lim\limits_{x\to(-3)^+}y=\lim\limits_{x\to(-3)^+}\dfrac{2-x}{x+3}=+\infty$.\\
		Vậy đồ thị hàm số đã cho có tiệm cận đứng $x=-3$.
	}
\end{ex}

\Closesolutionfile{ans}
%======================
\subsection{Bảng đáp án}
\inputansbox{8}{ans/ANS-DANG-20}


