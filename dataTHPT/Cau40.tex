%Dạng 1
\setcounter{section}{39}
\setcounter{ex}{0}
\section{Tích phân hàm ẩn}
\subsection{Kiến thức cần nhớ}
\begin{khung}
    \subsubsection{Định nghĩa}
    Nếu hàm số $f(x)$ liên tục trên $[a;b]$ và $F(x)$ là một nguyên hàm của $f(x)$ trên $[a;b]$ thì
    \[\displaystyle\int\limits_{a}^{b} f(x) \mathrm{\,d}x=F(x)\bigg|_a^b=F(b)-F(a).\quad(*) \]
Tên gọi
\begin{itemize}
    \item $\displaystyle\int\limits_{a}^{b} f(x) \mathrm{\,d}x$ đọc là \lq\lq Tích phân từ $a$ đến $b$ của $f(x)\mathrm{\,d}x$\rq\rq.
    \item $a$ và $b$ gọi là hai cận của tích phân, trong đó $a$ là cận dưới và $b$ là cận trên.
    \item $(*)$ gọi là công thức \textbf{Newton-Leibnitz}.
\end{itemize}
\subsubsection{Tính chất}
\begin{enumerate}
    \item $\displaystyle\int\limits_a^b[m\cdot f(x)\pm n\cdot g(x)]\mathrm{\,d}x=m\displaystyle\int\limits_a^bf(x)\mathrm{\,d}x\pm n\displaystyle\int\limits_a^bf(x)\mathrm{\,d}x$
    \item $\displaystyle\int\limits_a^bf(x)\mathrm{\,d}x=\displaystyle\int\limits_a^cf(x)\mathrm{\,d}x+\displaystyle\int\limits_c^b(x)\mathrm{\,d}x,\forall c\in[a ; b]$.
    \item $\displaystyle\int\limits_{a}^bf(x)\mathrm{\,d}x=-\displaystyle\int\limits_b^af(x)\mathrm{\,d}x$; $\displaystyle\int\limits_a^af(x)\mathrm{\,d}x=0$. 
    \item $\displaystyle\int\limits_{a}^bf(x)\mathrm{\,d}x=\displaystyle\int\limits_a^b f(t)\mathrm{\,d}t=\displaystyle\int\limits_a^b f(u)\mathrm{\,d}u= \ldots$
\end{enumerate}
\subsubsection{Phương pháp đổi biến số}
Cho hàm số $f(x)$ liên tục trên đoạn $[a;b]$. Giả sử hàm số $x= \phi (t)$ có đạo hàm liên tục trên đoạn $[\alpha; \beta]$ sao cho $\phi (\alpha)=a$, $\phi (\beta)=b$ và $a \le \phi (t) \le b$ với mọi $t \in [\alpha; \beta]$. Khi đó
$$\displaystyle\int\limits_a^b f(x)\mathrm{\,d}x= \displaystyle\int\limits_{\alpha}^{\beta} f\left(\phi (t)\right) \phi'(t)\mathrm{\,d}t.$$
\subsubsection{Phương pháp tích phân từng phần}
Nếu $u=u(x)$ và $v=v(x)$ là hai hàm số có đạo hàm liên tục trên đoạn $[a;b]$ thì
\begin{eqnarray*}
    &&\displaystyle\int\limits_a^b u(x)\cdot v'(x)\mathrm{\,d}x= \left[u(x)\cdot v(x)\right]\Big|_a^b- \displaystyle\int\limits_a^b u'(x)\cdot v(x)\mathrm{\,d}x\\
    \text{hay} && \displaystyle\int\limits_a^b u\mathrm{\,d}v= \left(u\cdot v\right)\Big|_a^b- \displaystyle\int\limits_a^b v\mathrm{\,d}u.
\end{eqnarray*}
\end{khung}
\subsection{Bài tập mẫu}
\Opensolutionfile{ans}[ans/ANS-DANG-40]
\begin{khung}
    \begin{vd}%[Trường Sơn Hồng]%[2D3G2-4]
        [Đề minh họa 2023 - BGD\&ĐT] Cho hàm số $f(x)$ liên tục trên $\mathbb{R}$. Gọi $F(x), G(x)$ là hai nguyên hàm của $f(x)$ trên $\mathbb{R}$ thỏa mãn $F(4)+G(4)=4$ và $F(0)+G(0)=1$. Khi đó $\displaystyle\int\limits_0^2 f(2 x)\mathrm{\,d}x$ bằng
    \choice
    {$3$}
    {\True $\dfrac{3}{4}$}
    {$6$}
    {$\dfrac{3}{2}$}
    \loigiai{
    Đặt $u= 2x \Rightarrow \mathrm{d}u = 2 \mathrm{d}x$ hay $\mathrm{d}x = \dfrac{1}{2} \mathrm{d}u$.\\
    Khi $x=0$ thì $u(0)=0$, $x=2$ thì $u(2)=4$.\\
    Khi đó 
    \begin{eqnarray*}
        \displaystyle\int\limits_{0}^{2}f(2x) \mathrm{d}x &=& \dfrac{1}{2}\displaystyle\int\limits_{0}^{4}f(u) \mathrm{d}u= \dfrac{1}{2}\displaystyle\int\limits_{0}^{4}f(x) \mathrm{d}x= \dfrac{1}{4}\left[\displaystyle\int\limits_{0}^{4}f(x) \mathrm{d}x+\displaystyle\int\limits_{0}^{4}f(x) \mathrm{d}x\right]\\
        &=& \dfrac{1}{4} \left[\left( F(4)- F(0) \right) +\left( G(4)- G(0) \right)\right]\\
        &=& \dfrac{1}{4} \left[\left( F(4)+ G(4) \right) -\left( F(0)+ G(0) \right)\right]= \dfrac{3}{4}.
    \end{eqnarray*}
    }
    \end{vd}
\end{khung}
\subsection{Bài tập tương tự và phát triển}

\begin{ex}%[2D3G2-4]%[Trường Sơn Hồng]
    Cho hàm số $f(x)$ liên tục trên $\mathbb{R}$. Gọi $F(x)$, $G(x)$ là hai nguyên hàm của $f(x)$ trên $\mathbb{R}$ thỏa mãn $F(-25)+G(-25)=6$ và $F(-33)+G(-33)=9$. Tính $\displaystyle\int\limits_{-5}^{-4}f(8 x + 7) \mathrm{d}x$.
    \choice
    {$15$}
    {\True $-\dfrac{3}{16}$}
    {$-3$}
    {$\dfrac{3}{8}$}
\loigiai{
Đặt $u= 8 x + 7 \Rightarrow \mathrm{d}u = 8 \mathrm{d}x$, hay $\mathrm{d}x = \dfrac{1}{8} \mathrm{d}u$.\\
Khi $x=-5$ thì $u(-5)=-33$, $x=-4$ thì $u(-4)=-25$.\\
Khi đó
\begin{eqnarray*}
    \displaystyle\int\limits_{-5}^{-4}f(8 x + 7) \mathrm{d}x &=& \dfrac{1}{8}\displaystyle\int\limits_{-33}^{-25}f(u) \mathrm{d}u= \dfrac{1}{8}\displaystyle\int\limits_{-33}^{-25}f(x) \mathrm{d}x= \dfrac{1}{16}\left[\displaystyle\int\limits_{-33}^{-25}f(x) \mathrm{d}x+\displaystyle\int\limits_{-33}^{-25}f(x) \mathrm{d}x\right]\\
    &=& \dfrac{1}{16} \left[\left( F(-25)- F(-33) \right) +\left( G(-25)- G(-33) \right)\right]\\
    &=& \dfrac{1}{16} \left[\left( F(-25)+ G(-25) \right) -\left( F(-33)+ G(-33) \right)\right]= - \dfrac{3}{16}.
\end{eqnarray*}
}
\end{ex}

\begin{ex}%[2D3G2-4]%[Trường Sơn Hồng]
Cho hàm số $f(x)$ liên tục trên $\mathbb{R}$.
Gọi $F(x)$, $G(x)$ là hai nguyên hàm của $f(x)$ trên $\mathbb{R}$
thỏa mãn $F(22)+G(22)=-1$ và $F(-33)+G(-33)=0$.
Tính $\displaystyle\int\limits_{-6}^{5}f(5 x - 3) \mathrm{d}x$.
\choice
    {$11$}
    {$-11$}
    {$\dfrac{1}{10}$}
    {\True $- \dfrac{1}{10}$}
\loigiai{
Đặt $u= 5 x - 3 \Rightarrow \mathrm{d}u = 5 \mathrm{d}x$,
hay $\mathrm{d}x = \dfrac{1}{5} \mathrm{d}u$.
\\
Khi $x=-6$ thì $u(-6)=-33$.
Khi $x=5$ thì $u(5)=22$.
Khi đó
\[
\begin{aligned}
\displaystyle\int\limits_{-6}^{5}f(5 x - 3) \mathrm{d}x
&\ =\dfrac{1}{5}
\displaystyle\int\limits_{-33}^{22}f(u) \mathrm{d}u
= \dfrac{1}{5}
\displaystyle\int\limits_{-33}^{22}f(x) \mathrm{d}x
= \dfrac{1}{10}
 \left[
 \displaystyle\int\limits_{-33}^{22}f(x) \mathrm{d}x
 +
 \displaystyle\int\limits_{-33}^{22}f(x) \mathrm{d}x
 \right]
\\
&\ = \dfrac{1}{10} \left[
    \left( F(22)- F(-33) \right) +
    \left( G(22)- G(-33) \right)
    \right]
\\
&\ = \dfrac{1}{10} \left[
    \left( F(22)+ G(22) \right) -
    \left( F(-33)+ G(-33) \right)
    \right]
    = - \dfrac{1}{10}
    \text{.}
\end{aligned}
\]
}
\end{ex}
\begin{ex}%[2D3G2-4]%[Trường Sơn Hồng]
Cho hàm số $f(x)$ liên tục trên $\mathbb{R}$.
Gọi $F(x)$, $G(x)$ là hai nguyên hàm của $f(x)$ trên $\mathbb{R}$
thỏa mãn $F(52)+G(52)=0$ và $F(-46)+G(-46)=-8$.
Tính $\displaystyle\int\limits_{-7}^{7}f(7 x + 3) \mathrm{d}x$.
\choice
    {\True $\dfrac{4}{7}$}    
    {$8$}
    {$\dfrac{1}{14}$}
    {$-\dfrac{4}{7}$}
\loigiai{
Đặt $u= 7 x + 3 \Rightarrow \mathrm{d}u = 7 \mathrm{d}x$,
hay $\mathrm{d}x = \dfrac{1}{7} \mathrm{d}u$.
\\
Khi $x=-7$ thì $u(-7)=-46$.
Khi $x=7$ thì $u(7)=52$.
Khi đó
\[
\begin{aligned}
\displaystyle\int\limits_{-7}^{7}f(7 x + 3) \mathrm{d}x
&\ =\dfrac{1}{7}
\displaystyle\int\limits_{-46}^{52}f(u) \mathrm{d}u
= \dfrac{1}{7}
\displaystyle\int\limits_{-46}^{52}f(x) \mathrm{d}x
= \dfrac{1}{14}
 \left[
 \displaystyle\int\limits_{-46}^{52}f(x) \mathrm{d}x
 +
 \displaystyle\int\limits_{-46}^{52}f(x) \mathrm{d}x
 \right]
\\
&\ = \dfrac{1}{14} \left[
    \left( F(52)- F(-46) \right) +
    \left( G(52)- G(-46) \right)
    \right]
\\
&\ = \dfrac{1}{14} \left[
    \left( F(52)+ G(52) \right) -
    \left( F(-46)+ G(-46) \right)
    \right]
    = \dfrac{4}{7}
    \text{.}
\end{aligned}
\]
}
\end{ex}
\begin{ex}%[2D3G2-4]%[Trường Sơn Hồng]
Cho hàm số $f(x)$ liên tục trên $\mathbb{R}$.
Gọi $F(x)$, $G(x)$ là hai nguyên hàm của $f(x)$ trên $\mathbb{R}$
thỏa mãn $F(132)+G(132)=3$ và $F(-28)+G(-28)=8$.
Tính $\displaystyle\int\limits_{-3}^{17}f(8 x - 4) \mathrm{d}x$.
\choice
    {$5$}
    {$-5$}
    {\True $\dfrac{5}{16}$}  
    {$-\dfrac{5}{16}$}
\loigiai{
Đặt $u= 8 x - 4 \Rightarrow \mathrm{d}u = 8 \mathrm{d}x$,
hay $\mathrm{d}x = \dfrac{1}{8} \mathrm{d}u$.
\\
Khi $x=-3$ thì $u(-3)=-28$.
Khi $x=17$ thì $u(17)=132$.
Khi đó
\[
\begin{aligned}
\displaystyle\int\limits_{-3}^{17}f(8 x - 4) \mathrm{d}x
&\ =\dfrac{1}{8}
\displaystyle\int\limits_{-28}^{132}f(u) \mathrm{d}u
= \dfrac{1}{8}
\displaystyle\int\limits_{-28}^{132}f(x) \mathrm{d}x
= \dfrac{1}{16}
 \left[
 \displaystyle\int\limits_{-28}^{132}f(x) \mathrm{d}x
 +
 \displaystyle\int\limits_{-28}^{132}f(x) \mathrm{d}x
 \right]
\\
&\ = \dfrac{1}{16} \left[
    \left( F(132)- F(-28) \right) +
    \left( G(132)- G(-28) \right)
    \right]
\\
&\ = \dfrac{1}{16} \left[
    \left( F(132)+ G(132) \right) -
    \left( F(-28)+ G(-28) \right)
    \right]
    = - \dfrac{5}{16}
    \text{.}
\end{aligned}
\]
}
\end{ex}
\begin{ex}%[2D3G2-4]%[Trường Sơn Hồng]
Cho hàm số $f(x)$ liên tục trên $\mathbb{R}$.
Gọi $F(x)$, $G(x)$ là hai nguyên hàm của $f(x)$ trên $\mathbb{R}$
thỏa mãn $F(147)+G(147)=2$ và $F(28)+G(28)=-3$.
Tính $\displaystyle\int\limits_{4}^{21}f(7 x) \mathrm{d}x$.
\choice
    {$5$}
    {\True $\dfrac{5}{14}$} 
    {$-5$}
    {$-\dfrac{5}{14}$}
\loigiai{
Đặt $u= 7 x \Rightarrow \mathrm{d}u = 7 \mathrm{d}x$,
hay $\mathrm{d}x = \dfrac{1}{7} \mathrm{d}u$.
\\
Khi $x=4$ thì $u(4)=28$.
Khi $x=21$ thì $u(21)=147$.
Khi đó
\[
\begin{aligned}
\displaystyle\int\limits_{4}^{21}f(7 x) \mathrm{d}x
&\ =\dfrac{1}{7}
\displaystyle\int\limits_{28}^{147}f(u) \mathrm{d}u
= \dfrac{1}{7}
\displaystyle\int\limits_{28}^{147}f(x) \mathrm{d}x
= \dfrac{1}{14}
 \left[
 \displaystyle\int\limits_{28}^{147}f(x) \mathrm{d}x
 +
 \displaystyle\int\limits_{28}^{147}f(x) \mathrm{d}x
 \right]
\\
&\ = \dfrac{1}{14} \left[
    \left( F(147)- F(28) \right) +
    \left( G(147)- G(28) \right)
    \right]
\\
&\ = \dfrac{1}{14} \left[
    \left( F(147)+ G(147) \right) -
    \left( F(28)+ G(28) \right)
    \right]
    = \dfrac{5}{14}
    \text{.}
\end{aligned}
\]
}
\end{ex}
\begin{ex}%[2D3G2-4]%[Trường Sơn Hồng]
Cho hàm số $f(x)$ liên tục trên $\mathbb{R}$.
Gọi $F(x)$, $G(x)$ là hai nguyên hàm của $f(x)$ trên $\mathbb{R}$
thỏa mãn $F(90)+G(90)=4$ và $F(27)+G(27)=9$.
Tính $\displaystyle\int\limits_{3}^{12}f(7 x + 6) \mathrm{d}x$.
\choice
    {$5$}
    {$\dfrac{5}{14}$} 
    {$-5$}
    {\True $-\dfrac{5}{14}$}
\loigiai{
Đặt $u= 7 x + 6 \Rightarrow \mathrm{d}u = 7 \mathrm{d}x$,
hay $\mathrm{d}x = \dfrac{1}{7} \mathrm{d}u$.
\\
Khi $x=3$ thì $u(3)=27$.
Khi $x=12$ thì $u(12)=90$.
Khi đó
\[
\begin{aligned}
\displaystyle\int\limits_{3}^{12}f(7 x + 6) \mathrm{d}x
&\ =\dfrac{1}{7}
\displaystyle\int\limits_{27}^{90}f(u) \mathrm{d}u
= \dfrac{1}{7}
\displaystyle\int\limits_{27}^{90}f(x) \mathrm{d}x
= \dfrac{1}{14}
 \left[
 \displaystyle\int\limits_{27}^{90}f(x) \mathrm{d}x
 +
 \displaystyle\int\limits_{27}^{90}f(x) \mathrm{d}x
 \right]
\\
&\ = \dfrac{1}{14} \left[
    \left( F(90)- F(27) \right) +
    \left( G(90)- G(27) \right)
    \right]
\\
&\ = \dfrac{1}{14} \left[
    \left( F(90)+ G(90) \right) -
    \left( F(27)+ G(27) \right)
    \right]
    = - \dfrac{5}{14}
    \text{.}
\end{aligned}
\]
}
\end{ex}
\begin{ex}%[2D3G2-4]%[Trường Sơn Hồng]
Cho hàm số $f(x)$ liên tục trên $\mathbb{R}$.
Gọi $F(x)$, $G(x)$ là hai nguyên hàm của $f(x)$ trên $\mathbb{R}$
thỏa mãn $F(20)+G(20)=-4$ và $F(8)+G(8)=8$.
Tính $\displaystyle\int\limits_{1}^{7}f(2 x + 6) \mathrm{d}x$.
\choice
    {$3$}
    {$4$} 
    {\True $-3$}
    {$-4$}
\loigiai{
Đặt $u= 2 x + 6 \Rightarrow \mathrm{d}u = 2 \mathrm{d}x$,
hay $\mathrm{d}x = \dfrac{1}{2} \mathrm{d}u$.
\\
Khi $x=1$ thì $u(1)=8$.
Khi $x=7$ thì $u(7)=20$.
Khi đó
\[
\begin{aligned}
\displaystyle\int\limits_{1}^{7}f(2 x + 6) \mathrm{d}x
&\ =\dfrac{1}{2}
\displaystyle\int\limits_{8}^{20}f(u) \mathrm{d}u
= \dfrac{1}{2}
\displaystyle\int\limits_{8}^{20}f(x) \mathrm{d}x
= \dfrac{1}{4}
 \left[
 \displaystyle\int\limits_{8}^{20}f(x) \mathrm{d}x
 +
 \displaystyle\int\limits_{8}^{20}f(x) \mathrm{d}x
 \right]
\\
&\ = \dfrac{1}{4} \left[
    \left( F(20)- F(8) \right) +
    \left( G(20)- G(8) \right)
    \right]
\\
&\ = \dfrac{1}{4} \left[
    \left( F(20)+ G(20) \right) -
    \left( F(8)+ G(8) \right)
    \right]
    = -3
    \text{.}
\end{aligned}
\]
}
\end{ex}
\begin{ex}%[2D3G2-4]%[Trường Sơn Hồng]
Cho hàm số $f(x)$ liên tục trên $\mathbb{R}$.
Gọi $F(x)$, $G(x)$ là hai nguyên hàm của $f(x)$ trên $\mathbb{R}$
thỏa mãn $F(30)+G(30)=5$ và $F(2)+G(2)=-3$.
Tính $\displaystyle\int\limits_{-2}^{12}f(2 x + 6) \mathrm{d}x$.
\choice
    {\True $2$}
    {$8$} 
    {$-2$}
    {$-8$}
\loigiai{
Đặt $u= 2 x + 6 \Rightarrow \mathrm{d}u = 2 \mathrm{d}x$,
hay $\mathrm{d}x = \dfrac{1}{2} \mathrm{d}u$.
\\
Khi $x=-2$ thì $u(-2)=2$.
Khi $x=12$ thì $u(12)=30$.
Khi đó
\[
\begin{aligned}
\displaystyle\int\limits_{-2}^{12}f(2 x + 6) \mathrm{d}x
&\ =\dfrac{1}{2}
\displaystyle\int\limits_{2}^{30}f(u) \mathrm{d}u
= \dfrac{1}{2}
\displaystyle\int\limits_{2}^{30}f(x) \mathrm{d}x
= \dfrac{1}{4}
 \left[
 \displaystyle\int\limits_{2}^{30}f(x) \mathrm{d}x
 +
 \displaystyle\int\limits_{2}^{30}f(x) \mathrm{d}x
 \right]
\\
&\ = \dfrac{1}{4} \left[
    \left( F(30)- F(2) \right) +
    \left( G(30)- G(2) \right)
    \right]
\\
&\ = \dfrac{1}{4} \left[
    \left( F(30)+ G(30) \right) -
    \left( F(2)+ G(2) \right)
    \right]
    = 2
    \text{.}
\end{aligned}
\]
}
\end{ex}
\begin{ex}%[2D3G2-4]%[Trường Sơn Hồng]
Cho hàm số $f(x)$ liên tục trên $\mathbb{R}$.
Gọi $F(x)$, $G(x)$ là hai nguyên hàm của $f(x)$ trên $\mathbb{R}$
thỏa mãn $F(50)+G(50)=-2$ và $F(-58)+G(-58)=-7$.
Tính $\displaystyle\int\limits_{-6}^{6}f(9 x - 4) \mathrm{d}x$.
\choice
    {$-\dfrac{5}{9}$} 
    {$-\dfrac{5}{18}$}
    {\True $\dfrac{5}{18}$}
    {$\dfrac{5}{9}$}
\loigiai{
Đặt $u= 9 x - 4 \Rightarrow \mathrm{d}u = 9 \mathrm{d}x$,
hay $\mathrm{d}x = \dfrac{1}{9} \mathrm{d}u$.
\\
Khi $x=-6$ thì $u(-6)=-58$.
Khi $x=6$ thì $u(6)=50$.
Khi đó
\[
\begin{aligned}
\displaystyle\int\limits_{-6}^{6}f(9 x - 4) \mathrm{d}x
&\ =\dfrac{1}{9}
\displaystyle\int\limits_{-58}^{50}f(u) \mathrm{d}u
= \dfrac{1}{9}
\displaystyle\int\limits_{-58}^{50}f(x) \mathrm{d}x
= \dfrac{1}{18}
 \left[
 \displaystyle\int\limits_{-58}^{50}f(x) \mathrm{d}x
 +
 \displaystyle\int\limits_{-58}^{50}f(x) \mathrm{d}x
 \right]
\\
&\ = \dfrac{1}{18} \left[
    \left( F(50)- F(-58) \right) +
    \left( G(50)- G(-58) \right)
    \right]
\\
&\ = \dfrac{1}{18} \left[
    \left( F(50)+ G(50) \right) -
    \left( F(-58)+ G(-58) \right)
    \right]
    = \dfrac{5}{18}
    \text{.}
\end{aligned}
\]
}
\end{ex}
\begin{ex}%[2D3G2-4]%[Trường Sơn Hồng]
Cho hàm số $f(x)$ liên tục trên $\mathbb{R}$.
Gọi $F(x)$, $G(x)$ là hai nguyên hàm của $f(x)$ trên $\mathbb{R}$
thỏa mãn $F(32)+G(32)=-8$ và $F(-44)+G(-44)=-3$.
Tính $\displaystyle\int\limits_{-9}^{10}f(4 x - 8) \mathrm{d}x$.
\choice
    {$\dfrac{5}{4}$} 
    {\True $-\dfrac{5}{8}$}
    {$\dfrac{5}{8}$}
    {$-\dfrac{5}{4}$}
\loigiai{
Đặt $u= 4 x - 8 \Rightarrow \mathrm{d}u = 4 \mathrm{d}x$,
hay $\mathrm{d}x = \dfrac{1}{4} \mathrm{d}u$.
\\
Khi $x=-9$ thì $u(-9)=-44$.
Khi $x=10$ thì $u(10)=32$.
Khi đó
\[
\begin{aligned}
\displaystyle\int\limits_{-9}^{10}f(4 x - 8) \mathrm{d}x
&\ =\dfrac{1}{4}
\displaystyle\int\limits_{-44}^{32}f(u) \mathrm{d}u
= \dfrac{1}{4}
\displaystyle\int\limits_{-44}^{32}f(x) \mathrm{d}x
= \dfrac{1}{8}
 \left[
 \displaystyle\int\limits_{-44}^{32}f(x) \mathrm{d}x
 +
 \displaystyle\int\limits_{-44}^{32}f(x) \mathrm{d}x
 \right]
\\
&\ = \dfrac{1}{8} \left[
    \left( F(32)- F(-44) \right) +
    \left( G(32)- G(-44) \right)
    \right]
\\
&\ = \dfrac{1}{8} \left[
    \left( F(32)+ G(32) \right) -
    \left( F(-44)+ G(-44) \right)
    \right]
    = - \dfrac{5}{8}
    \text{.}
\end{aligned}
\]
}
\end{ex}

\begin{ex}%[2D3G2-4]%[Trường Sơn Hồng]
    Cho hàm số $f\left( x \right)$ liên tục trên $\mathbb{R}$ thỏa mãn $\displaystyle\int\limits_0^7 f\left( x \right)\mathrm{\,d}x=10,\,\displaystyle\int\limits_0^3 f\left( x \right)\mathrm{\,d}x=6$. Tính $I=\displaystyle\int\limits_{-2}^3 f\left( \left| 3-2x \right| \right)\mathrm{\,d}x$.
\choice
{$16$}
{$3$}
{$15$}
{\True $8$}
\loigiai{
Vì $\displaystyle\int\limits_3^7 f\left( x \right)\mathrm{\,d}x =\displaystyle\int\limits_0^7 f\left( x \right)\mathrm{\,d}x -\displaystyle\int\limits_0^3 f\left( x \right)\mathrm{\,d}x=4$.\\
Đặt $t=3-2x\Rightarrow \mathrm{\,d}t=-2\mathrm{\,d}x$. Khi $x=-2\Rightarrow t=7,\,x=3\Rightarrow t=-3$.
\begin{eqnarray*}
    \Rightarrow I&=&\displaystyle\int\limits_7^{-3} -\dfrac{1}{2}f\left( \left| t \right| \right)\mathrm{\,d}t=\dfrac{1}{2}\displaystyle\int\limits_{-3}^7 f\left( \left| t \right| \right)\mathrm{\,d}t\\
    &=&\dfrac{1}{2}\left[ \displaystyle\int\limits_{-3}^3 f\left( \left| t \right| \right)\mathrm{\,d}t+\displaystyle\int\limits_3^7 f\left( \left| t \right| \right)\mathrm{\,d}t \right]\\
    &=&\dfrac{1}{2}\left[ 2\displaystyle\int\limits_0^3 f\left( t \right)\mathrm{\,d}t+\displaystyle\int\limits_3^7 f\left( t \right)\mathrm{\,d}t \right]\\
    &=&\dfrac{1}{2}\left( 2\cdot 6+4 \right)=8.
\end{eqnarray*}
}
\end{ex}

\begin{ex}%[2D3G2-4]%[Trường Sơn Hồng]
    Cho hàm số $f\left( x \right)$ liên tục trên $\mathbb{R}$. Biết $\displaystyle\int\limits_0^{\tfrac{\pi }{2}} \sin2x\cdot f\left( \cos^2x \right)\mathrm{\,d}x=1$, khi đó $I=\displaystyle\int\limits_0^1 \left[ 2f\left( 1-x \right)-3x^2+5 \right]\mathrm{\,d}x$ bằng
\choice
{\True $6$}
{$4$}
{$5$}
{$3$}
\loigiai{
Ta có $\displaystyle\int\limits_0^{\tfrac{\pi }{2}} \sin2x\cdot f\left( \cos^2x \right)\mathrm{\,d}x=1 \quad\left( * \right)$\\
Đặt $t=\cos^2x \Rightarrow \mathrm{\,d}t=-2\cos x\cdot \sin x\mathrm{\,d}x=-\sin2x\mathrm{\,d}x$. Đổi cận $\heva{& x=0\Rightarrow t=1 \\ & x=\dfrac{\pi }{2}\Rightarrow t=0.}$\\ 
Từ $\left( * \right)$ suy ra $-\displaystyle\int\limits_1^0 f\left( t \right)\mathrm{\,d}t=1\Leftrightarrow \displaystyle\int\limits_0^1 f\left( t \right)\mathrm{\,d}t=1$.\\
Xét tích phân $\displaystyle\displaystyle\int\limits_0^1{f\left( 1-x \right)}{d}x$.\\
Đặt $t=1-x\Rightarrow \mathrm{\,d}t=-{d}x$. Đổi cận $\heva{& x=0\Rightarrow t=1 \\ & x=1\Rightarrow t=0.}$\\ 
Suy ra $\displaystyle\int\limits_0^1 f\left( 1-x \right)\mathrm{\,d}x=-\displaystyle\int\limits_1^0 f\left( t \right)\mathrm{\,d}t=\displaystyle\int\limits_0^1 f\left( t \right)\mathrm{\,d}t=1$.\\
Do đó
\begin{eqnarray*}
    I&=&\displaystyle\int\limits_0^1 \left[ 2f\left( 1-x \right)-3x^2+5 \right]\mathrm{\,d}x=2\displaystyle\int\limits_0^1 f\left( 1-x \right)\mathrm{\,d}x+\displaystyle\int\limits_0^1 \left( -3x^2+5 \right)\mathrm{\,d}x\\
    &=&2\displaystyle\int\limits_0^1 f\left( 1-x \right)\mathrm{\,d}x+4 =2\cdot 1+4=6.
\end{eqnarray*}
}
\end{ex}

\begin{ex}%[2D3G2-4]%[Trường Sơn Hồng]
    Cho hàm số $f\left( x \right)$ thỏa mãn $f\left( 2 \right)=25$ và $f'\left( x \right)=4x\sqrt{f\left( x \right)}$ với mọi $x\in \mathbb{R}$. Khi đó $\displaystyle\int\limits_2^3 f\left( x \right)\mathrm{\,d}x$ bằng
\choice
{$\dfrac{1073}{15}$}
{$\dfrac{458}{15}$}
{\True $\dfrac{838}{15}$}
{$\dfrac{1016}{15}$}
\loigiai{
Ta có $f'\left( x \right)=4x\sqrt{f\left( x \right)}\ge 0,\,\forall x\in \left[ 2;3 \right]$. Suy ra hàm số $f(x)$ đồng biến trên $\left[ 2;3 \right]$.\\ 
Suy ra $f\left( x \right)\ge f\left( 2 \right)=25\Rightarrow f\left( x \right)>0,\,\forall x\in \left[ 2;3 \right]$.\\ 
Do đó, ta có $f'\left( x \right)=4x\sqrt{f\left( x \right)}\Rightarrow \dfrac{f'\left( x \right)}{\sqrt{f\left( x \right)}}=4x,\,\forall x\in \left[ 2;3 \right] \quad \left( 1 \right)$.\\
Lấy nguyên hàm hai vế của đẳng thức $\left( 1 \right)$ ta có $\displaystyle\displaystyle\int \dfrac{f'\left( x \right)}{\sqrt{f\left( x \right)}}\mathrm{\,d}x=\displaystyle\int 4x\mathrm{\,d}x$.\\
Đặt $t=\sqrt{f\left( x \right)}\Rightarrow t^2=f\left( x \right)\Rightarrow 2t\mathrm{\,d}t=f'\left( x \right)\mathrm{\,d}x$.\\
Ta có 
\begin{eqnarray*}
    &&\displaystyle\int \dfrac{2t}{t}\mathrm{\,d}t=2x^2+C \Rightarrow 2\sqrt{f\left( x \right)}=2x^2+C.\\
    &\Rightarrow& 2\sqrt{f\left( 2 \right)}=2\cdot 2^2+C \Rightarrow 2\cdot\sqrt{25}=8+C\Rightarrow C=2.\\
    &\Rightarrow& \sqrt{f\left( x \right)}=x^2+1\Rightarrow f\left( x \right)=\left( x^2+1 \right)^2.
\end{eqnarray*}
Suy ra $\displaystyle\int\limits_2^3 f\left( x \right)\mathrm{\,d}x=\displaystyle\int\limits_2^3 \left( x^2+1 \right)^2\mathrm{\,d}x=\dfrac{838}{15}$.
}
\end{ex}

\begin{ex}%[2D3G2-4]%[Trường Sơn Hồng]
    Cho hàm số $f\left( x \right)$ thỏa mãn $\displaystyle\int\limits_0^1 \left( x+1 \right)f'\left( x \right)\mathrm{\,d}x=10$ và $2f\left( 1 \right)-f\left( 0 \right)=2$. Tính $I=\displaystyle\int\limits_0^1 f\left( x \right)\mathrm{\,d}x$.
\choice
{ $I=1$}
{ $I=8$}
{ $I=-12$}
{\True $I=-8$}
\loigiai{
Gọi $f\left( x \right)=ax+b$, $\left( a\ne 0 \right) \Rightarrow f'\left( x \right)=a$.\\
Theo giả thiết ta có
\begin{itemize}
    \item $\displaystyle\int\limits_0^1 \left( x+1 \right) f'\left( x \right)\mathrm{\,d}x=10$
    $$\Leftrightarrow a\displaystyle\int\limits_0^1 \left( x+1 \right)\mathrm{\,d}x=10 \Leftrightarrow \displaystyle\int\limits_0^1 \left( x+1 \right)\mathrm{\,d}x=\dfrac{10}{a}\Leftrightarrow \dfrac{3}{2}=\dfrac{10}{a}\Rightarrow a=\dfrac{20}{3}.$$
    \item $2f\left( 1 \right)-f\left( 0 \right)=2 \Leftrightarrow 2\cdot\left( \dfrac{20}{3}+b \right)-b=2 \Leftrightarrow b=-\dfrac{34}{3}$.
\end{itemize}
Do đó, $f\left( x \right)=\dfrac{20}{3}x-\dfrac{34}{3}$.\\
Vậy $I=\displaystyle\int\limits_0^1 f\left( x \right)\mathrm{\,d}x =\displaystyle\int_0^1 \left( \dfrac{20}{3}x-\dfrac{34}{3} \right)\mathrm{\,d}x=-8$.
}
\end{ex}

\begin{ex}%[2D3G2-4]%[Trường Sơn Hồng]
    Cho hàm số $y=f\left( x \right)$ liên tục trên $\mathbb{R}$ và thỏa mãn $f\left( -x \right)+2018f\left( x \right)=2x\sin x$. Tính $I=\displaystyle\int\limits_{-\tfrac{\pi }{2}}^{\tfrac{\pi }{2}} f\left( x \right)\mathrm{\,d}x$.
\choice
{$\dfrac{2}{2018}$}
{$\dfrac{2}{1009}$}
{\True $\dfrac{4}{2019}$}
{$\dfrac{2}{2019}$}
\loigiai{
Ta có 
\begin{eqnarray*}
    &&\displaystyle\int\limits_{-\tfrac{\pi}{2}}^{\tfrac{\pi}{2}} \left( f\left( -x \right)+2018f\left( x \right) \right)\mathrm{\,d}x=\displaystyle\int\limits_{-\tfrac{\pi }{2}}^{\tfrac{\pi }{2}} 2x\sin x\mathrm{\,d}x\\
    &\Leftrightarrow& \displaystyle\int\limits_{-\tfrac{\pi }{2}}^{\tfrac{\pi }{2}} f\left( -x \right)\mathrm{\,d}x +2018\displaystyle\int\limits_{-\tfrac{\pi }{2}}^{\tfrac{\pi }{2}} f\left( x \right)\mathrm{\,d}x=\displaystyle\int\limits_{-\tfrac{\pi }{2}}^{\tfrac{\pi }{2}} 2x\sin x\mathrm{\,d}x\\
    &\Leftrightarrow& 2019\displaystyle\int\limits_{-\tfrac{\pi }{2}}^{\tfrac{\pi }{2}} f\left( x \right)\mathrm{\,d}x=\displaystyle\int\limits_{-\tfrac{\pi }{2}}^{\tfrac{\pi }{2}} 2x\sin x\mathrm{\,d}x \quad \left( 1 \right).
\end{eqnarray*}
Xét $P=\displaystyle\int\limits_{-\tfrac{\pi }{2}}^{\tfrac{\pi }{2}} 2x\sin x\mathrm{\,d}x$. Đặt $\heva{& u=2x \\ & \mathrm{\,d}v=\sin x\mathrm{\,d}x} \Rightarrow \heva{& \mathrm{\,d}u=2\mathrm{\,d}x \\ & v=-\cos x.}$\\
$$P=2x\cdot\left( -\cos x \right)\Big| _{-\tfrac{\pi }{2}}^{\tfrac{\pi }{2}}+2\sin x \Big| _{-\tfrac{\pi }{2}}^{\tfrac{\pi }{2}}=4.$$
Từ $\left( 1 \right)$ suy ra $I=\displaystyle\int\limits_{-\tfrac{\pi }{2}}^{\tfrac{\pi }{2}} f\left( x \right)\mathrm{\,d}x=\dfrac{4}{2019}$.
}
\end{ex}

\begin{ex}%[2D3G2-4]%[Trường Sơn Hồng]
    Biết $\displaystyle\int\limits_0^{\pi } f\left( \sin x \right)\mathrm{\,d}x=1$. Tính $\displaystyle\int\limits_0^{\pi } xf\left( \sin x \right)\mathrm{\,d}x$.
\choice
{\True $\dfrac{\pi }{2}$}
{$0$}
{$\pi $}
{$\dfrac{1}{2}$}
\loigiai{
Ta chứng minh rằng nếu hàm số $f\left( x \right)$ liên tục trên $\left[ a;\,b \right]$ mà $f\left( a+b-x \right)=f\left( x \right)$ thì
$$\displaystyle\int\limits_a^b xf\left( x \right)\mathrm{\,d}x=\dfrac{a+b}{2}\displaystyle\int\limits_a^b f\left( x \right)\mathrm{\,d}x.$$
Thật vậy, đặt $u=a+b-x\Rightarrow \mathrm{\,d}u=-\mathrm{\,d}x$.\\
Ta có 
\begin{eqnarray*}
    \displaystyle\int\limits_a^b xf\left( x \right)\mathrm{\,d}x&=&\left( a+b \right)\displaystyle\int\limits_a^b f\left( x \right)\mathrm{\,d}x-\displaystyle\int\limits_a^b \left( a+b-x \right)f\left( a+b-x \right)\mathrm{\,d}x\\
    &=&\left( a+b \right)\displaystyle\int\limits_a^b f\left( x \right)\mathrm{\,d}x+\displaystyle\int\limits_{u\left( a \right)}^{u\left( b \right)} uf\left( u \right)\mathrm{\,d}u\\
    &=&\left( a+b \right)\displaystyle\int\limits_a^b f\left( x \right)\mathrm{\,d}x+\displaystyle\int\limits_b^a xf\left( x \right)\mathrm{\,d}x\\
    &=&\left( a+b \right)\displaystyle\int\limits_a^b f\left( x \right)\mathrm{\,d}x-\displaystyle\int\limits_a^b xf\left( x \right)\mathrm{\,d}x.
\end{eqnarray*}
Do đó $\displaystyle\int\limits_a^b xf\left( x \right)\mathrm{\,d}x=\dfrac{a+b}{2}\displaystyle\int\limits_a^b f\left( x \right)\mathrm{\,d}x$.\\
Vậy $\displaystyle\int\limits_0^{\pi } xf\left( \sin x \right)\mathrm{\,d}x=\dfrac{\pi }{2}\displaystyle\int\limits_0^{\pi } f\left( \sin x \right)\mathrm{\,d}x=\dfrac{\pi }{2}$.
}
\end{ex}

\begin{ex}%[2D3G2-4]%[Trường Sơn Hồng]
    Cho hàm số $f(x)$ có $f(2)=0$ và $f'(x)=\dfrac{x+7}{\sqrt{2x-3}},\,\forall x\in \left( \dfrac{3}{2};+\infty \right)$. Biết rằng $\displaystyle\int\limits_4^7 f\left( \dfrac{x}{2} \right)\mathrm{\,d}x=\dfrac{a}{b},\,\left( a,b\in \mathbb{Z},b>0 \right)$, $\dfrac{a}{b}$ là phân số tối giản. Khi đó $a+b$ bằng
\choice
{$250$}
{\True $251$}
{$133$}
{$221$}
\loigiai{
Xét nguyên hàm $J =\displaystyle\int \dfrac{x+7}{\sqrt{2x-3}}\mathrm{\,d}x$.\\
Đặt $u=\sqrt{2x-3}\Rightarrow u^2=2x-3\Rightarrow u\mathrm{\,d}u=\mathrm{\,d}x$. Suy ra $x=\dfrac{u^2+3}{2}$.\\
Do đó 
\begin{eqnarray*}
    J &=& \displaystyle\int \left( \dfrac{u^2+3}{2}+7 \right)\mathrm{\,d}u=\displaystyle\int \left( \dfrac{u^2}{2}+\dfrac{17}{2} \right)\mathrm{\,d}u\\
    &=& \dfrac{u^3}{6}+\dfrac{17}{2}u+C = \dfrac{\left( \sqrt{2x-3} \right)^3}{6}+\dfrac{17}{2}\sqrt{2x-3}+C.\\
\end{eqnarray*}
Ta có $f(2)=0\Leftrightarrow \dfrac{1}{6}+\dfrac{17}{2}+C=0\Leftrightarrow C=\dfrac{-26}{3}$.\\
Suy ra 
\begin{eqnarray*}
    \displaystyle\int\limits_4^7 f\left( \dfrac{x}{2} \right)\mathrm{\,d}x &=& \displaystyle\int\limits_4^7 \left[ \dfrac{\left( \sqrt{x-3} \right)^3}{6}+\dfrac{17}{2}\sqrt{x-3}+\dfrac{-26}{3} \right]\mathrm{\,d}x\\
    &=& \displaystyle\int\limits_4^7 \left[ \dfrac{\left( x-3 \right)^{\tfrac{3}{2}}}{6}+\dfrac{17}{2}{\left( x-3 \right)^{\tfrac{1}{2}}}+\dfrac{-26}{3} \right]\mathrm{\,d}x\\
    &=& \left. \left[ \dfrac{1}{15} \left( x-3 \right)^{\tfrac{5}{2}}+\dfrac{17}{3}\left( x-3 \right)^{\tfrac{3}{2}}-\dfrac{26}{3}x \right] \right|_4^7\\
    &=& \dfrac{-66}{5}-\dfrac{-434}{15}=\dfrac{236}{15}.
\end{eqnarray*}
Suy ra $a=236;b=15 \Rightarrow a+b=251$.
}
\end{ex}

\begin{ex}%[2D3G2-4]%[Trường Sơn Hồng]
    Cho hàm số $f\left( x \right)$ liên tục trên $\mathbb{R}$ và $f\left( 2 \right)=16$, $\displaystyle\int\limits_0^2 f\left( x \right)\mathrm{\,d}x=4$. Tính $I=\displaystyle\int\limits_0^1 xf'\left( 2x \right)\mathrm{\,d}x$.
\choice
{$20$}
{$13$}
{\True $7$}
{$12$}
\loigiai{
Ta có 
\begin{eqnarray*}
    I&=&\displaystyle\int\limits_0^1 xf'\left( 2x \right)\mathrm{\,d}x=\dfrac{1}{2}\displaystyle\int\limits_0^1 x\mathrm{\,d}\left( f\left( 2x \right) \right)\\
    &=&\left. \dfrac{1}{2}xf\left( 2x \right) \right|_0^1-\dfrac{1}{2}\displaystyle\int\limits_0^1 f\left( 2x \right)\mathrm{\,d}x\\
    &=& \dfrac{1}{2}\left( f\left( 2 \right)-0 \right)-\dfrac{1}{4}\displaystyle\int\limits_0^1 f\left( 2x \right)\mathrm{\,d}\left( 2x \right)\\
    &=& 8-\dfrac{1}{4}\displaystyle\int\limits_0^2 f\left( u \right)\mathrm{\,d}\left( u \right) \text{ với }  u=2x.\\
    &=& 8-1 = 7.
\end{eqnarray*}
}
\end{ex}

\begin{ex}%[2D3G2-4]%[Trường Sơn Hồng]
    Cho hàm số $f\left( x \right)$ có đạo hàm và liên tục trên $\left[ 0;1 \right]$, thỏa mãn $\displaystyle\int\limits_1^2 f\left( x-1 \right)\mathrm{\,d}x=3$ và $f\left( 1 \right)=4$. Tích phân $\displaystyle\int\limits_0^1 x^3f'\left( x^2 \right)\mathrm{\,d}x$.
\choice
{$1$}
{$-1$}
{$-\dfrac{1}{2}$}
{\True $\dfrac{1}{2}$}
\loigiai{
Đặt $t=x^2 \Rightarrow \mathrm{\,d}t=2x\mathrm{\,d}x$. Đổi cận $\heva{& x=0\Rightarrow t=0 \\ & x=1\Rightarrow t=1.}$\\
Ta có 
\begin{eqnarray*}
    \displaystyle\int\limits_0^1 x^3 f'\left( x^2 \right)\mathrm{\,d}x&=&\dfrac{1}{2}\displaystyle\int\limits_0^1 tf'\left( t \right)\mathrm{\,d}t=\dfrac{1}{2}\displaystyle\int\limits_0^1 t\mathrm{\,d}\left[ f\left( t \right) \right]\\
    &=&\dfrac{1}{2}\left[ tf\left( t \right) \Big|_0^1-\displaystyle\int\limits_0^1 f\left( t \right)\mathrm{\,d}t \right]\\
    &=&\dfrac{1}{2}\left[ f\left( 1 \right)-\displaystyle\int\limits_0^1 f\left( x \right)\mathrm{\,d}x \right].
\end{eqnarray*}
Lại có $\displaystyle\int\limits_1^2 f\left( x-1 \right)\mathrm{\,d}x=\displaystyle\int\limits_1^2 f\left( x-1 \right)\mathrm{\,d}\left( x-1 \right)$ 
$$\xrightarrow{u=x-1}\displaystyle\int\limits_1^2 f\left( x-1 \right)\mathrm{\,d}x=\displaystyle\int\limits_0^1 f\left( u \right)\mathrm{\,d}u=\displaystyle\int\limits_0^1 f\left( x \right)\mathrm{\,d}x=3.$$
Suy ra $\displaystyle\int\limits_0^1 x^3 f'\left( x^2 \right)\mathrm{\,d}x=\dfrac{1}{2}\left( 4-3 \right)=\dfrac{1}{2}$.
}
\end{ex}

\begin{ex}%[2D3G2-4]%[Trường Sơn Hồng]
    Cho hàm số $f\left( x \right)$ có $f\left( 1 \right)=1$ và $f'\left( x \right)=-\dfrac{\ln x}{x^2},\,\forall x>0$. Khi đó $I=\displaystyle\int\limits_1^{\mathrm{e}} f\left( x \right)\mathrm{\,d}x$ bằng
\choice
{$I=-\dfrac{3}{2}$}
{\True $I=\dfrac{3}{2}$}
{$I=\dfrac{2}{\mathrm{e}}-1$}
{$I=1-\dfrac{2}{\mathrm{e}}$}
\loigiai{
Ta có
\begin{eqnarray*}
    f\left( x \right)&=&\displaystyle\int f'\left( x \right)\mathrm{\,d}x=-\displaystyle\int \dfrac{\ln x}{x^2}\mathrm{\,d}x=\displaystyle\int \ln x\mathrm{\,d}\left( \dfrac{1}{x} \right)\\
    &=&\dfrac{\ln x}{x}-\displaystyle\int \dfrac{1}{x}\mathrm{\,d}\left( \ln x \right)=\dfrac{\ln x}{x}-\displaystyle\int \dfrac{1}{x^2}\mathrm{\,d}x\\
    &=&\dfrac{\ln x}{x}+\dfrac{1}{x}+C.
\end{eqnarray*}
Do $f\left( 1 \right)=1$ nên $1=1+C\Leftrightarrow C=0$. Suy ra $f\left( x \right)=\dfrac{\ln x+1}{x}$.\\
Do đó 
\begin{eqnarray*}
    I&=&\displaystyle\int\limits_1^{\mathrm{e}} f\left( x \right)\mathrm{\,d}x=\displaystyle\int\limits_1^{\mathrm{e}} \left( \dfrac{\ln x+1}{x} \right)\mathrm{\,d}x\\
    &=&\displaystyle\int\limits_1^{\mathrm{e}} \left( \ln x+1 \right)\mathrm{\,d}\left( \ln x+1 \right)=\left. \dfrac{\left( \ln x+1 \right)^2}{2} \right|_1^{\mathrm{e}}=2-\dfrac{1}{2}=\dfrac{3}{2}.
\end{eqnarray*}
}
\end{ex}
\Closesolutionfile{ans}
\subsection{Bảng đáp án}
\inputansbox{8}{ans/ANS-DANG-40}

