%Dạng 1
\setcounter {section} {11}
\setcounter{ex}{0}
\section{Các phép toán cơ bản của số phúc}
\subsection{Kiến thức cần nhớ}
\begin{khung}
	\subsubsection{Định nghĩa}
\begin{enumerate}
	\item Một số phức là một biểu thức dạng $z=a+bi$ với  $a, b\in \mathbb{R}$ và $i^2=-1$,
	$i$  được gọi là đơn vị ảo, $a$ được gọi là phần thực và $b$ được gọi là phần ảo của số phức $z=a+bi$.
	\item Tập hợp các số phức được kí hiệu là $\mathbb{C}$,  
	$\mathbb{C}=\left\{a+bi|a, b\in \mathbb{R};i^2=-1\right\}$.
	\item Chú ý
	\begin{itemize}
		\item Khi phần ảo $b=0\Leftrightarrow z=a$ là số thực.
		\item Khi phần thực $a=0\Leftrightarrow z=bi\Leftrightarrow z$ là số thuần ảo.
		\item Số $0=0+0i$ vừa là số thực, vừa là số ảo.		
	\end{itemize}
	\item Hai số phức bằng nhau $a+bi=c+di \Leftrightarrow \heva{
		&a=c\\
		&b=d}$ với $a, b, c, d\in \mathbb{R}.$
	\item Hai số phức $z_1=a+bi; z_2=-a-bi$ được gọi là hai số phức đối nhau.
\end{enumerate}
\subsubsection{Số phức liên hợp}
Số phức liên hợp của $z=a+bi$ với $a,b\in \mathbb{R}$ là $a-bi$ và được kí hiệu bởi $\overline{z}$. Rõ ràng $\overline{\overline{z}}=z$
\subsubsection{Biễu diễn hình học}
Trong mặt phẳng phức $Oxy$ ($Ox$ là trục thực, $Oy$ là trục ảo), số phức  $z=a+bi$ với  $a, b\in \mathbb{R}$ được biểu diễn bằng điểm $M(a;b)$.
\subsubsection{Mô-đun của số phức}
Mô-đun của số phức $z=a+bi (a, b\in \mathbb{R})$ là  $|z|=\sqrt{a^2+b^2}$ .
\subsubsection{Các phép toán trên tập số phức}
Cho hai số phức $z=a+bi$; $z'=a'+b'i  $ với $a, b, a', b'\in \mathbb{R}$và số $k\in \mathbb{R}$.
\begin{enumerate}
	\item Tổng hai số phức: $z+z'=a+a'+(b+b')i$.
	\item Hiệu hai số phức: $z+z'=a-a'+(b-b')i$.
	\item Nhân hai số phức:  $z\cdot z'=(a+bi)(a'+b'i)=(a\cdot a'-b\cdot b')+(a\cdot b'+a'\cdot b)i$.
	\item Chia 2 số phức:    
	\begin{itemize}
		\item Số phức nghịch đảo: $\dfrac{1}{z}=\dfrac{1}{{| z|}^2}\overline{z}$.
		\item Nếu $z\ne 0$ thì $\dfrac{z'}{z}=\dfrac{z'\cdot \overline{z}}{|z|^2}$, nghĩa là nếu muốn chia số phức $z'$ cho số phức $z\ne 0$ thì ta nhân cả tử và mẫu của thương $\dfrac{z'}{z}$ cho $\overline{z}$.
	\end{itemize}
\end{enumerate}
\end{khung}
\subsection{Bài tập mẫu}
\Opensolutionfile{ans}[ans/ANS-DANG-12]
\begin{khung}
\begin{vd}[ĐỀ MINH HỌA BGD 2022-2023]%[2D4B2-2]
	Cho số phức $z=2+9i$, phần thực của số phức $z^2$ bằng
	\choice
	{\True $-77$}
	{$4$}
	{$36$}
	{$85$}
	\loigiai{
		Ta có $z^2=(2+9i)^2=4+36i+81i^2=-77+36i$ nên phần thực của số phức $z^2$ bằng $-77$.
	}
\end{vd}
\end{khung}
\subsection{Bài tập tương tự và phát triển}
\begin{ex}%[Nguyen Tuan, phát triển 50 dạng toán đề MH 2023]%[2D4B2-2]%Câu 1
	Số phức liên hợp của $z=(2+4i)+(1-3i)$ là
	\choice
	{$\overline{z}=-3-i$}
	{$\overline{z}=1+3i$}
	{$\overline{z}=3+i$}
	{\True $\overline{z}=3-i$}
	\loigiai{
		Ta có $z=(2+4i)+(1-3i)=3+i\Rightarrow\overline{z}=3-i$.}
\end{ex}

\begin{ex}%[Nguyen Tuan, phát triển 50 dạng toán đề MH 2023]%[2D4B3-2]%Câu 2
	Tìm phần ảo của số phức $\overline{z}$, biết $z=\dfrac{(1+i)3i}{1-i}$.
	\choice
	{\True $0$}
	{$-1$}
	{$3$}
	{$-3$}
	\loigiai{
		Ta có $z=\dfrac{(1+i)3i}{1-i}=\dfrac{(1+i)^2\cdot 3i}{1-i^2}=\dfrac{2i\cdot 3i}{2}=-3$.\\
		Suy ra $\overline{z}=-3$.\\ 
		Vậy phần ảo của số phức $\overline{z}$ là $0$.}
\end{ex}

\begin{ex}%[Nguyen Tuan, phát triển 50 dạng toán đề MH 2023]%[2D4Y1-1]%Câu 3
	Số phức liên hợp của số phức $3-2i$ là
	\choice
	{\True $3+2i$}
	{$-3-2i$}
	{$-2+3i$}
	{$-3+2i$}
	\loigiai{
		Số phức liên hợp của số phức $3-2i$ là $3+2i$.}
\end{ex}

\begin{ex}%[Nguyen Tuan, phát triển 50 dạng toán đề MH 2023]%[2D4B2-2]%Câu 4
	Cho các số phức $z_1=2+3i$, $z_2=4+5i$. Số phức liên hợp của số phức $w=2(z_2-z_1)$ là
	\choice
	{$\overline{w}=8-15i$}
	{$\overline{w}=4+4i$}
	{\True $\overline{w}=4-4i$}
	{$\overline{w}=8+15i$}
	\loigiai{
		Ta có $w=2(z_2-z_1)=2\left[(4+5i)-(2+3i)\right]=4+4i\Rightarrow\overline{w}=4-4i$.}
\end{ex}

\begin{ex}%[Nguyen Tuan, phát triển 50 dạng toán đề MH 2023]%[2D4B2-2]%Câu 5
	Cho số phức $z=4-2i$. Phần ảo của số phức $3-4z$ là
	\choice
	{$-4$}
	{$-8$}
	{$-2$}
	{\True $8$}
	\loigiai{
	Ta có $3-4z=3-4(4-2i)=-13+8i$.\\
	Vậy phần ảo của $3-4z$ bằng $8$.
}
\end{ex}

\begin{ex}%[Nguyen Tuan, phát triển 50 dạng toán đề MH 2023]%[2D4B2-2]%Câu 6
	Cho số phức $z=2-3i$. Tìm mô-đun của số phức $w=2z+(1+i)\overline{z}$.
	\choice
	{\True $| w|=\sqrt{10}$}
	{$| w|=4$}
	{$| w|=\sqrt{15}$}
	{$| w|=\sqrt{2}$}
	\loigiai{

		Ta có $w=(2-3i)+(1+i)(2+3i)=3-i$ $\Rightarrow | w|=\sqrt{3^2+(-1)^2}=\sqrt{10}$.}
\end{ex}

\begin{ex}%[Nguyen Tuan, phát triển 50 dạng toán đề MH 2023]%[2D4B2-2]%Câu 7
	Phần ảo của số phức $z=(2-3i)^2-(1+i)^2$ là
	\choice
	{$-10$}
	{$-10i$}
	{$-14i$}
	{\True $-14$}
	\loigiai{
		Ta có $z=(2-3i)^2-(1+i)^2=-5-14i$.\\
		Suy ra, phần ảo của số phức $z$ là $-14$.}
\end{ex}

\begin{ex}%[Nguyen Tuan, phát triển 50 dạng toán đề MH 2023]%[2D4B2-2]%Câu 8
	Cho số phức $z=-2+xi, (x \in \mathbb{R})$ có mô-đun bằng
	\choice
	{$\sqrt{x^2+2}$}
	{\True $\sqrt{x^2+4}$}
	{$|x|+2$}
	{$|2x|$}
	\loigiai{
		Ta có $|z|=|-2+xi|=\sqrt{(-2)^2+x^2}=\sqrt{x^2+4}$.}
\end{ex}

\begin{ex}%[Nguyen Tuan, phát triển 50 dạng toán đề MH 2023]%[2D4B2-2]%Câu 9
	Cho hai số phức $z_1=1+2i$ và $z_2=2-3i$. Phần ảo của số phức $w=3z_1-2z_2$ là
	\choice
	{$1$}
	{$11$}
	{\True $12$}
	{$12i$}
	\loigiai{
		Ta có $w=3z_1-2z_2=3(1+2i)-2(2-3i)=-1+12i$.\\
		Vậy phần ảo của số phức $w$ là $12$.}
\end{ex}

\begin{ex}%[Nguyen Tuan, phát triển 50 dạng toán đề MH 2023]%[2D4B2-2]%Câu 10
	Cho số phức $z$ thỏa mãn $z(2-i)+13i=1$. Tính mô-đun của số phức $z$.
	\choice
	{$|z|=\dfrac{5\sqrt{34}}{3}$}
	{$|z|=34$}
	{\True $|z|=\sqrt{34}$}
	{$|z|=\dfrac{\sqrt{34}}{3}$}
	\loigiai{
		Ta có $z(2-i)+13i=1\Leftrightarrow z=\dfrac{1-13i}{2-i}=\dfrac{(1-13i)(2+i)}{5}=\dfrac{15-25i}{5}=3-5i$.\\
		Suy ra $|z|=\sqrt{3^2+(-5)^2}=\sqrt{34}$.}
\end{ex}

\begin{ex}%[Nguyen Tuan, phát triển 50 dạng toán đề MH 2023]%[2D4B2-2]%Câu 11
	Cho số phức $z=(1+i)^2(1+2i)$. Số phức $z$ có phần ảo là
	\choice
	{$-4$}
	{$2i$}
	{$4$}
	{\True $2$}
	\loigiai{
		Ta có $z=(1+i)^2(1+2i)=2i(1+2i)=-4+2i$.\\
		Do đó phần ảo của $z$ là $2$.}
\end{ex}

\begin{ex}%[Nguyen Tuan, phát triển 50 dạng toán đề MH 2023]%[2D4Y1-1]%Câu 12
	Tìm số phức liên hợp của số phức $z=3+2i$.
	\choice
	{\True $\overline{z}=3-2i$}
	{$\overline{z}=-3-2i$}
	{$\overline{z}=2-3i$}
	{$\overline{z}=-2-3i$}
	\loigiai{
Ta có 
		$\overline{z}=3-2i$.}
\end{ex}

\begin{ex}%[Nguyen Tuan, phát triển 50 dạng toán đề MH 2023]%[2D4Y1-1]%Câu 13
	Số phức $z=4-3i$ có mô-đun bằng
	\choice
	{$2\sqrt{2}$}
	{$25$}
	{\True $5$}
	{$8$}
	\loigiai{
		Ta có $|z|=\sqrt{4^2+(-3)^2}=5$.}
\end{ex}

\begin{ex}%[Nguyen Tuan, phát triển 50 dạng toán đề MH 2023]%[2D4B2-2]%Câu 14
	Cho số phức $z=5-4i$. Số phức $z-2$ có
	\choice
	{Phần thực bằng $5$ và phần ảo bằng $-4$}
	{\True Phần thực bằng $3$ và phần ảo bằng $-4$}
	{Phần thực bằng $-4$ và phần ảo bằng $3$}
	{Phần thực bằng $3$ và phần ảo bằng $-4i$}
	\loigiai{
		Với $z=5-4i$ ta có $z-2=5-4i-2=3-4i$ có phần thực là $3$ và phần ảo là $-4$.}
\end{ex}

\begin{ex}%[Nguyen Tuan, phát triển 50 dạng toán đề MH 2023]%[2D4B2-2]%Câu 15
	Cho số phức $z_1=2+3i$ và $z_2=1-2i$. Số phức liên hợp của số phức $w=z_1+z_2$ là
	\choice
	{$\overline{w}=3+i$}
	{\True $\overline{w}=3-i$}
	{$\overline{w}=3-2i$}
	{$\overline{w}=1-4i$}
	\loigiai{
		Vì $z_1=2+3i$ và $z_2=1-2i$ nên $w=z_1+z_2$ $=2+3i+1-2i=3+i$.\\
		Suy ra $\overline{w}=3-i$.}
\end{ex}

\begin{ex}%[Nguyen Tuan, phát triển 50 dạng toán đề MH 2023]%[2D4B2-2]%Câu 16
	Cho các số phức $z_1=2-3i$, $z_2=1+4i$. Tìm số phức liên hợp với số phức $z_1z_2$.
	\choice
	{$-14-5i$}
	{$-10-5i$}
	{$-10+5i$}
	{\True $14-5i$}
	\loigiai{
		Ta có $z_1z_2=(2-3i)(1+4i)=14+5i$.\\ 
		Vậy $\overline{z_1z_2}=14-5i$.}
\end{ex}

\begin{ex}%[Nguyen Tuan, phát triển 50 dạng toán đề MH 2023]%[2D4B2-2]%Câu 17
	Cho hai số phức $z_1=1+i$ và $z_2=2-3i$. Tính mô-đun của số phức $z_1+z_2$.
	\choice
	{$|z_1+z_2|=5$}
	{\True $|z_1+z_2|=\sqrt{13}$}
	{$|z_1+z_2|=1$}
	{$|z_1+z_2|=\sqrt{5}$}
	\loigiai{
		Ta có $z_1+z_2=(1+i)+(2-3i)=3-2i$.\\
		Vậy $|z_1+z_2|=|3-2i|=\sqrt{3^2+(-2)^2}=\sqrt{13}$.}
\end{ex}

\begin{ex}%[Nguyen Tuan, phát triển 50 dạng toán đề MH 2023]%[2D4B2-2]%Câu 18
	Tìm số phức liên hợp của số phức $z=(-3-4i)(2+i)+1-3i$.
	\choice
	{\True $\overline{z}=-1+14i$}
	{$\overline{z}=1-14i$}
	{$\overline{z}=1+14i$}
	{$\overline{z}=-1-14i$}
	\loigiai{
		Ta có $z=(-3-4i)(2+i)+1-3i=-1-14i$.\\
		Vậy số phức liên hợp của $z$ là $\overline{z}=-1+14i$.}
\end{ex}

\begin{ex}%[Nguyen Tuan, phát triển 50 dạng toán đề MH 2023]%[2D4B2-2]%Câu 19
	Cho hai số phức $z_1=1+i$ và $z_2=2-3i$. Tính mô-đun của $z_1+z_2$.
	\choice
	{$|z_1+z_2|=1$}
	{$|z_1+z_2|=\sqrt{5}$}
	{\True $|z_1+z_2|=\sqrt{13}$}
	{$|z_1+z_2|=5$}
	\loigiai{
		Ta có $z_1+z_2=3-2i$.\\
		Vậy $|z_1+z_2|=\sqrt{3^2+2^2}=\sqrt{13}$.}
\end{ex}

\begin{ex}%[Nguyen Tuan, phát triển 50 dạng toán đề MH 2023]%[2D4B2-2]%Câu 20
	Cho hai số phức $z_1=3+i$ và $z_2=2-4i$. Mô-đun của số phức $z_1z_2$ bằng
	\choice
	{$10$}
	{\True $10\sqrt{2}$}
	{$-10$}
	{$20$}
\loigiai{
	Ta có $z_1z_2=(3+i)(2-4i)=10-10i$.\\ 
	Vậy $|z_1z_2|=\sqrt{10^2+10^2}=10\sqrt{2}$.}
\end{ex}
\Closesolutionfile{ans}
%======================
\subsection{Bảng đáp án}
\inputansbox{8}{ans/ANS-DANG-12}


