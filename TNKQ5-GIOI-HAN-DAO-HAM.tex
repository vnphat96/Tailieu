\begin{dang}
    {GIỚI HẠN VÀ ĐẠO HÀM}
\end{dang}
%Câu 1
\begin{ex}
Tìm $I= \lim \limits_{x\to +\infty } \,\left(\dfrac{2x-3}{1-x}\right)$.
\choice
{\True $I=-2$}
{$I=2$}
{$I=-3$}
{$I=3$}
\loigiai{
$I= \lim \limits_{x\to +\infty } \,\left(\dfrac{2x-3}{1-x}\right)= \lim \limits_{x\to +\infty } \,\dfrac{x\left(2-\dfrac{3}{x}\right)}{x\left(\dfrac{1}{x}-1\right)}= \lim \limits_{x\to +\infty } \,\dfrac{2-\dfrac{3}{x}}{\dfrac{1}{x}-1}=-2$
}
\end{ex}
%Câu 2
\begin{ex}
Tính đạo hàm của hàm số $y=\sin 2025x$.
\choice
{$y'=\cos 2025x$}
{$y'=\sin 2025x$}
{\True $y'=2025 \cdot \cos 2025x$}
{$y'=2025\cos x$}
\loigiai{
$y=\sin 2025x\Rightarrow y'=(2025x)'\cos 2025x=2025\cos 2025x$
}
\end{ex}
%Câu 3
\begin{ex}
Giá trị của $ \lim \limits_{x\to 1} \left(2x^2-3x+1\right)$ bằng
\choice
{$1$}
{\True $0$}
{$+\infty $}
{$2$}
\loigiai{
Ta có: $ \lim \limits_{x\to 1} \left(2x^2-3x+1\right)=2 \cdot 1^2-3 \cdot 1+1=0$
}
\end{ex}
%Câu 4
\begin{ex}
Tính đạo hàm của hàm số $f(x)=\dfrac{2x}{x-1}$ tại điểm $x=-1$.
\choice
{\True $f'(-1)=-\dfrac{1}{2}$}
{$f'(-1)=1$}
{$f'(-1)=-2$}
{$f'(-1)=0$}
\loigiai{
Tập xác định $\mathscr{D}=\mathbb{R}\setminus \{1\}$\\
Ta có $f'(x)=-\dfrac{2}{{{(x-1)}^2}}$, $f'(-1)=-\dfrac{2}{{{\left(-1-1\right)}^2}}=-\dfrac{1}{2}$
}
\end{ex}
%Câu 5
\begin{ex}
Một đoàn tàu chuyển động thẳng khởi hành từ một nhà ga. Quãng đường $s$ đi được của đoàn tàu là một hàm số của thời gian $t$ , hàm số đó là $s=6t^2-t^3$. Thời điểm $t$ mà tại đó vận tốc $v\left(m/s\right)$ của đoàn tàu đạt giá trị lớn nhất là
\choice
{$t=4(s)$}
{$t=3(s)$}
{\True $t=2(s)$}
{$t=6(s)$}
\loigiai{
Ta có: $v(t)=s'(t)=12t-3t^2$.\\
Do đó vận tốc của đoàn tàu đạt giá trị lớn nhất tại thời điểm $t=-\dfrac{12}{2 \cdot (-3)}=2(s)$
}
\end{ex}
%Câu 6
\begin{ex}
Tìm giới hạn sau $\lim \dfrac{2n^3+3n+1}{n^3+2n^2+1}$.
\choice
{$3$}
{\True $2$}
{$-3$}
{$1$}
\loigiai{
\\
$\lim \dfrac{2n^3+3n+1}{n^3+2n^2+1}=\lim \dfrac{2+\dfrac{3}{n^2}+\dfrac{1}{n^3}}{1+\dfrac{2}{n}+\dfrac{1}{n^3}}=\dfrac{2}{1}=2$
}
\end{ex}
%Câu 7
\begin{ex}
Giới hạn $ \lim \limits_{x\to +\infty } \,\dfrac{2-x}{1-x}$ bằng
\choice
{$-2$}
{$2$}
{\True $1$}
{$-1$}
\loigiai{
Ta có $ \lim \limits_{x\to +\infty } \,\dfrac{2-x}{1-x}= \lim \limits_{x\to +\infty } \,\dfrac{\dfrac{2}{x}-1}{\dfrac{1}{x}-1}=1$
}
\end{ex}
%Câu 8
\begin{ex}
Đạo hàm của hàm số $y=3^x$ là
\choice
{\True $y'=3^x \cdot \ln 3$}
{$y'=\dfrac{3^x}{\ln 3}$}
{$y'=3^x$}
{$y'=x{{ \cdot 3}^{x-1}}$}
\loigiai{
$y'=3^x \cdot \ln 3$
}
\end{ex}
%Câu 9
\begin{ex}
Biết $ \lim \limits_{x\to 2} \,\dfrac{x^2+bx+c}{x-2}=5(b,c\in \mathbb{R})$. Kết quả đúng của biểu thức $T=b+c$
\choice
{\True $T=-5$}
{$T=-3$}
{$T=1$}
{$T=-6$}
\loigiai{
Có: $ \lim \limits_{x\to 2} \,\dfrac{x^2+bx+c}{x-2}=5$ nên $x=2$ phải là nghiệm của phương trình $x^2+bx+c=0$\\
Suy ra: $2^2+b \cdot 2+c=0\Leftrightarrow 2b+c=-4\Leftrightarrow c=-4-2b$\\
$\begin{aligned}
&  \lim \limits_{x\to 2} \,\dfrac{x^2+bx+c}{x-2}= \lim \limits_{x\to 2} \,\dfrac{x^2+bx-2b-4}{x-2}= \lim \limits_{x\to 2} \,\dfrac{(x-2)(x+2)+b(x-2)}{x-2} \\& = \lim \limits_{x\to 2} \,\dfrac{(x-2)(x+b+2)}{x-2}= \lim \limits_{x\to 2} \,(x+b+2)=b+4 \end{aligned}$\\
Yêu cầu bài toán $b+4=5\Leftrightarrow b=1$. Suy ra $c=-4-2b=-6$\\
Vậy $T=b+c=-5$
}
\end{ex}
%Câu 10
\begin{ex}
Một vật chuyển động thẳng không đều xác định bởi phương trình $s(t)=3t^2-4t+9$, trong đó $s$ tính bằng mét và $t$ tính bằng giây. Vận tốc nhỏ nhất vật đạt được trong khoảng thời gian từ $t=1$ đến $t=3$ bằng
\choice
{4 $m/s$}
{1 $m/s$}
{3 $m/s$}
{\True 2 $m/s$}
\loigiai{
Vận tốc của vật theo thời gian là $v(t)=s'(t)=6t-4$.\\
Ta có $1\le t\le 3\Leftrightarrow 6\le 6t\le 18\Leftrightarrow 2\le 6t-4\le 14\Leftrightarrow 2\le v(t)\le 14$.\\
Vậy vận tốc nhỏ nhất vật đạt được trong khoảng thời gian từ $t=1$ đến $t=3$ bằng 2 $m/s$
}
\end{ex}
%Câu 11
\begin{ex}
Giá trị của $ \lim \limits_{n\to +\infty } \,\dfrac{2n-3}{n+1}$ bằng
\choice
{\True $2$}
{$-3$}
{$1$}
{$+\infty $}
\loigiai{
Ta có $ \lim \limits_{n\to +\infty } \,\dfrac{2n-3}{n+1}= \lim \limits_{x\to +\infty } \,\dfrac{2-\dfrac{3}{n}}{1+\dfrac{1}{n}}=\dfrac{2}{1}=2$
}
\end{ex}
