\documentclass[10pt,a4paper,onecolumn,titlepage,twoside,openany]{book}
% \usepackage[utf8]{vietnam}
%\usepackage{fouriernc}
\usepackage{tasks}
\usepackage{xcolor} 
%%%%%%%%%%%%% KIỂU MÀU VÀ ĐỘ RỘNG NOTE
\def\kieumau{N} %Y: Màu; N: đen-trắng
% \def\kieumau{Y} %Y: Màu; N: đen-trắng
\def\leftnote{5} %Độ rộng cột Note
%%%%%%%%%%%%% ĐN CƠ BẢN
\input{cautrucDT/color\kieumau} %MÀU
%=====================================
% Khai báo nhóm Tex (cơ bản)
%=====================================
\usepackage{amsmath,amssymb,mathrsfs,maybemath,xlop,polynom,slashbox}
\usepackage{yhmath} %\let\widering\relax %cần khi sd với font fouriernc

\usepackage{enumerate}
\usepackage{tikz} 
\usepackage{tkz-euclide}
%\usepackage{ex_tkz-euclide}
%\usetkzobj{all}
\usepackage{tikz-3dplot}
\usepackage{tkz-tab}
\usepackage{pifont} %kí hiệu đặc biệt
% \usepackage{xcolor}
%\usepackage{bbding}
%\usepackage{array}
\usepackage{tasks}
% \usepackage{casiovn}
%==========
\usetikzlibrary{math,through,calc,intersections,angles,quotes,shapes,shapes.geometric,arrows,patterns,snakes,matrix,chains,arrows.meta,decorations.shapes,decorations.fractals,decorations.markings,shadows}
\usetikzlibrary{positioning,decorations.text,decorations.pathmorphing}% Để uốn cong văn bản 
\usetikzlibrary{shadings,fadings} %ĐỔ BÓNG
\usepackage{pgfplots}
\usepackage{pgfornament}
\usepgfplotslibrary{fillbetween}
\pgfplotsset{compat=1.9}
\usepackage[hidelinks,unicode]{hyperref}
\usepackage{currfile}
\usepackage[outline]{contour} %viền
\usepackage{fontawesome} % Gói kí hiệu
\usepackage{lipsum} %Lấy text
\usepackage{tabularx}
%%---------
%\usepackage{setspace}
%\usepackage{scrextend}
\usepackage{varwidth}
%===========Bảng
\usepackage{longtable,multirow,makecell}
\usepackage{diagbox}
\renewcommand{\tabcolsep}{3mm}
\newcolumntype{C}[1]{>{\centering\arraybackslash}p{#1}}
\newcolumntype{L}[1]{>{\raggedright\arraybackslash}p{#1}}
%-----------Trang vb


%%%%%%%%%%%%% Các thông số trang tài liệu
\def\tren{1.5}\def\duoi{1.5}\def\trai{1.25}\def\phai{0.75} %cách lề
\def\topset{0.75} %kc giữa đáy header và vùng vb
\def\botset{0.75} %kc giữa đỉnh footer và vùng vb
%\usepackage{ifoddpage}
\pgfmathsetmacro{\mepphai}{\phai+\leftnote} 
%\usepackage[top=\tren cm, bottom=\duoi cm, left=\trai cm, right=\mepphai cm] {geometry}
%%%%%%%%%%%%%
%---------------------------------Các thông số trang tài liệu
\pgfmathsetmacro{\so}{\leftnote - 0.5} 
\usepackage[top=\tren cm, bottom=\duoi cm, left=\trai cm, right=\mepphai cm,
marginparwidth=\so cm, marginparsep=5mm,
%,headsep=6mm
%,footskip=10mm
] {geometry}
%-------------------------------------
\usepackage{marginnote}
\setlength{\marginparwidth}{\so cm}
\renewcommand*{\marginfont}{\small}
%--------------Gói trắc nghiệm EX-TEST
% \usepackage[dethi]{ex_test}
\usepackage[loigiai]{ex_test} 
% \usepackage[solcolor]{ex_test}
%----Lời giải, Hiền thị tên EX; Dấu kết thúc
\font\damEX=ugqb8v at 11pt
\def\loigiaiEX{\color{\mauLG}\damEX\strut\faCommenting\ Lời giải.}
%lời giải EXS
\def\loigiaiEXS{\loigiaiEX{\fontsize{8}{16}\selectfont\color{\maucham}\dotfill}}
%--
\renewcommand{\nameex}{\damEX\color{\mauEX} CÂU}
\newtheorem{EX}{\nameex} %MÔI TRƯỜNG PHỤ CHO TÁCH CÂU
\def\mauVuong{cyan}
\def\qedEX{\color{\mauVuong}\ensuremath{\square}}
%--------------Cài đặt lại dòng kẻ \dotline
\renewcommand{\dotlineEX}[1]{
	\def\numlinedot{#1}
	\par
	\foreach \dotline in{1,...,\numlinedot}
	{
		\noindent
		\fontsize{8}{16}\selectfont
		\color{\maucham}\dotfill
		\par
	}
}
% sd cho \dongcham
\newcommand{\dotlineEXS}[1]{
	\def\numlinedot{#1}
	\foreach \dotline in{1,...,\numlinedot}
	{
		\noindent
		\fontsize{8}{16}\selectfont
		\color{\maucham}\dotfill
		\par
	}
}
%---------- Khai báo viết tắt, in đáp án
\newcommand{\hoac}[1]{ %hệ hoặc
	\left[\begin{aligned}#1\end{aligned}\right.}
\newcommand{\heva}[1]{ %hệ và
	\left\{\begin{aligned}#1\end{aligned}\right.}
%--In đáp án
\newcommand{\indapan}[2]{
	\addcontentsline{toc}{subsection}{\sf Bảng đáp án} % đưa MT vào mục lục
	\begin{center}
		\begin{tikzpicture}%
			\node[thick,scale=1,fill=\mauEX!2,draw=\maufoot,minimum width=3.5cm,minimum height=0.1cm,rounded corners=2mm]{\fontfamily{qag}\fontsize{11}{11}\selectfont\bfseries\color{\mauEX} BẢNG ĐÁP ÁN};
		\end{tikzpicture}%
	\end{center}
	\inputansbox{#1}{#2}
}
%----------
\usepackage{esvect}
\def\vec{\vv} %vecto
\def\overrightarrow{\vv}
%Lệnh song song
\DeclareSymbolFont{symbolsC}{U}{txsyc}{m}{n}
\DeclareMathSymbol{\varparallel}{\mathrel}{symbolsC}{9}
\DeclareMathSymbol{\parallel}{\mathrel}{symbolsC}{9}
%--------------------------
% HEADER AND FOOTER STYLING
%--------------------------
%--------------------------
\newcommand{\myfancyhead}{% trên và chấm trái
		\boldmath
\begin{tikzpicture}[remember picture,overlay,>=stealth]
		\path ([yshift=-\tren cm+0.5*\topset cm]current page.north west) coordinate (AA)
		++(\paperwidth,0)coordinate (BB); 
\checkoddpage\ifoddpage %nếu trang lẻ
		%-----đường kẻ
		\draw[\maufoot, line width=2pt] 
		([xshift=\trai cm]AA) --([xshift=-\phai cm]BB);
		%-----bên phải
		\node[text=\maufoot, anchor=south east,inner sep=0pt] at ([xshift=-\phai cm,yshift=4pt]BB){
			\fontfamily{qag}\fontsize{8.5pt}{12pt}\selectfont 
			{\color{\mauSO}\faMapMarker}\,\, \diachi\,\,{\color{\mauSO}\faMapMarker}
		};
		%-----bên trái
		\node[text=\maufoot, anchor=south west,inner sep=0pt] at ([xshift=\trai cm,yshift=4pt]AA){
			\fontfamily{qag}\fontsize{10pt}{10pt}\selectfont\bfseries\faEdit\, \tenchuyende
		};
		%----- Kẻ đứng
		\draw[\maufoot] ([xshift=-\mepphai cm+2.5mm]BB)--([yshift=\duoi cm-0.75*\botset cm,xshift=-\mepphai cm+2.5mm]current page.south east);
		%--		
		\path ([yshift=-\tren cm+0.5*\topset cm-0.5cm,xshift=-\phai cm-0.5*\leftnote cm+2.5mm]current page.north east) coordinate (DDD); 
		\begin{scope}
			\clip ([yshift=-\tren cm+0.5*\topset cm-1pt,xshift=-\phai cm]current page.north east) rectangle ([yshift=\duoi cm-0.5*\botset cm,xshift=-\mepphai cm+5mm]current page.south east);% cắt chấm
			\node[inner sep =0pt,scale=1,anchor=north] at ([yshift=0cm,xshift=0pt]DDD) {
				\parbox{\leftnote cm}{\centering
					\def\maucham{\maufoot}\dotlineEX{60}
				}
			};
			%--note dưới
			\node[inner sep =6pt, text=white,scale=1,anchor=north,fill=\maufoot] (noteduoi) at ([yshift=2.5mm]DDD) {
				\parbox{\leftnote cm-5mm-12pt}{ \fontsize{11}{1}\fontfamily{qag}\selectfont\bfseries\centering
					QUICK NOTE
				}
			};
			\draw[\maufoot, line width=0.4pt] ([yshift=-2pt]noteduoi.south west)--([yshift=-2pt]noteduoi.south east);
		\end{scope}
\else %chẵn
		%-----đường kẻ
		\draw[\maufoot, line width=2pt] 
		([xshift=\phai cm]AA) --([xshift=-\trai cm]BB);
		%-----bên trái
		\node[text=\maufoot, anchor=south west,inner sep=0pt] at ([xshift=\phai cm,yshift=4pt]AA){
			\fontfamily{qag}\fontsize{8.5pt}{12pt}\selectfont 
			{\color{\mauSO}\faMapMarker}\,\, \diachi\,\,{\color{\mauSO}\faMapMarker}
		};
		%-----bên phải
		\node[text=\maufoot, anchor=south east,inner sep=0pt] at ([xshift=-\trai cm,yshift=4pt]BB){
			\fontfamily{qag}\fontsize{10pt}{10pt}\selectfont\bfseries\faEdit\, \tenchuyende
		};
		%----- Kẻ đứng
		\draw[\maufoot] ([xshift=\mepphai cm-2.5mm]AA)--([yshift=\duoi cm-0.75*\botset cm,xshift=\mepphai cm-2.5mm]current page.south west);
		%--		
		\path ([yshift=-\tren cm+0.5*\topset cm-0.5cm,xshift=\phai cm+0.5*\leftnote cm-2.5mm]current page.north west) coordinate (DDD); 
		\begin{scope}
			\clip ([yshift=-\tren cm+0.5*\topset cm-1pt,xshift=\phai cm]current page.north west) rectangle ([yshift=\duoi cm-0.5*\botset cm,xshift=\mepphai cm-5mm]current page.south west);% cắt chấm
			\node[inner sep =0pt,scale=1,anchor=north] at ([yshift=0cm,xshift=0pt]DDD) {
				\parbox{\leftnote cm}{\centering
					\def\maucham{\maufoot}\dotlineEX{60}
				}
			};
			%--note dưới
			\node[inner sep =6pt, text=white,scale=1,anchor=north,fill=\maufoot] (noteduoi) at ([yshift=2.5mm]DDD) {
				\parbox{\leftnote cm-5mm-12pt}{ \fontsize{11}{1}\fontfamily{qag}\selectfont\bfseries\centering
					QUICK NOTE
				}
			};
			\draw[\maufoot, line width=0.4pt] ([yshift=-2pt]noteduoi.south west)--([yshift=-2pt]noteduoi.south east);
		\end{scope}
\fi
\end{tikzpicture}%
}
% trên mục lục
\newcommand{\headmucluc}{%
	\boldmath
\begin{tikzpicture}[remember picture,overlay,>=stealth]
	\path ([yshift=-\tren cm+0.5*\topset cm]current page.north west) coordinate (AA)
	++(\paperwidth,0)coordinate (BB); 
\checkoddpage\ifoddpage %nếu trang lẻ
	%-----đường kẻ
	\draw[\maufoot, line width=2pt] 
	([xshift=\trai cm]AA) --([xshift=-\phai cm]BB);
	%-----bên phải
	\node[text=\maufoot, anchor=south east,inner sep=0pt] at ([xshift=-\phai cm,yshift=4pt]BB){
		\fontfamily{qag}\fontsize{8.5pt}{12pt}\selectfont 
		{\color{\mauSO}\faMapMarker}\,\, \diachi\,\,{\color{\mauSO}\faMapMarker}
	};
	%-----bên trái
	\node[text=\maufoot, anchor=south west,inner sep=0pt] at ([xshift=\trai cm,yshift=4pt]AA){
		\fontfamily{qag}\fontsize{12pt}{12pt}\selectfont\bfseries\faEdit\, \tenchuyende
	};
\else %chẵn
	%-----đường kẻ
	\draw[\maufoot, line width=2pt] 
	([xshift=\phai cm]AA) --([xshift=-\trai cm]BB);
	%-----bên trái
	\node[text=\maufoot, anchor=south west,inner sep=0pt] at ([xshift=\phai cm,yshift=4pt]AA){
		\fontfamily{qag}\fontsize{8.5pt}{12pt}\selectfont 
		{\color{\mauSO}\faMapMarker}\,\, \diachi\,\,{\color{\mauSO}\faMapMarker}
	};
	%-----bên phải
	\node[text=\maufoot, anchor=south east,inner sep=0pt] at ([xshift=-\trai cm,yshift=4pt]BB){
		\fontfamily{qag}\fontsize{12pt}{12pt}\selectfont\bfseries\faEdit\, \tenchuyende
	};
\fi
\end{tikzpicture}%
}
%===========================
\newcommand{\myfancyfoot}{% dưới
	\begin{tikzpicture}[remember picture,overlay]
	\path ([yshift=\duoi cm-0.75*\botset cm]current page.south west) coordinate (AA)
	++(\paperwidth,0)coordinate (BB); 
	\checkoddpage\ifoddpage %nếu trang lẻ
		%---kẻ
		\draw[\maufoot, line width=2pt] ([xshift=2*\trai cm+4pt]AA)--([xshift=-\phai cm+3pt]BB);
		%-----bên trái
		\fill[fill=\maufoot, rounded corners=2mm] ([xshift=2*\trai cm,yshift=0.25 cm]AA) rectangle +(-3*\trai cm,-0.5cm);
		%-----trang
		\node[anchor=west,text=white,inner sep=0pt,xshift=-0.75cm] at ([xshift=2*\trai cm]AA) {\fontfamily{put}\bfseries\thepage};
		%-----tên tg
		\node[anchor=west,text=\maufoot,inner sep=0pt,fill=white] at ([xshift=2*\trai cm]AA){\fontfamily{qag}\fontsize{9pt}{1pt}\selectfont\bfseries \,\,\, \tentacgia \,\,\, };
	\else %chẵn
		%---kẻ
		\draw[\maufoot, line width=2pt] ([xshift=-2*\trai cm+4pt]BB)--([xshift=\phai cm-3pt]AA);
		%-----bên trái
		\fill[fill=\maufoot, rounded corners=2mm] ([xshift=-2*\trai cm,yshift=0.25 cm]BB) rectangle +(3*\trai cm,-0.5cm);
		%-----trang
		\node[anchor=east,text=white,inner sep=0pt,xshift=0.75cm] at ([xshift=-2*\trai cm]BB) {\fontfamily{put}\bfseries\thepage};
		%-----tên tg
		\node[anchor=east,text=\maufoot,inner sep=0pt,fill=white] at ([xshift=-2*\trai cm]BB){\fontfamily{qag}\fontsize{9pt}{1pt}\selectfont\bfseries \,\,\, \tentacgia \,\,\,};
	\fi
	\end{tikzpicture}%
}
%======================Head chapter theo note, fullwidth
%----------------------
\usepackage{changepage}
\strictpagecheck
\usepackage{lastpage}
\usepackage{fancyhdr,lastpage}
\pagestyle{fancy}
\fancyhf{}
\fancypagestyle{plain}{
	\fancyhead[LO,RE]{\headmucluc}
	\fancyfoot[LO,RE]{\myfancyfoot}
}
\fancyhead[LO,RE]{\myfancyhead}
\fancyfoot[LO,RE]{\myfancyfoot}
\renewcommand{\footrulewidth}{0pt}
\renewcommand{\headrulewidth}{0pt}
%--------------4.2
\usepackage[most]{tcolorbox}
\colorlet{tcbcol@back}{tcbcolback}
\colorlet{tcbcol@frame}{tcbcolframe}
%---------------------------------------------------------------
% ĐỊNH NGHĨA SECTION. SUBSECTION, SUBSUBSECTION ... THEO Ý RIÊNG
%---------------------------------------------------------------
\usepackage[explicit]{titlesec} % để gọi #1
\usepackage{titledot} % gói lệnh chứa cả titlesec và titletoc
%=====================================
\setcounter{secnumdepth}{4} %độ sâu
\renewcommand{\thechapter}{\Roman{chapter}}
\renewcommand{\thesection}{\arabic{section}}
\renewcommand{\thesubsection}{\Alph{subsection}}
\renewcommand{\thesubsubsection}{\arabic{subsubsection}}
%--------------Tròn
\newcommand{\tron}[1]{
	\begin{tikzpicture}[baseline=(A.base)]%
		\node[circle,draw=\mauSUBSEC,line width=0.5pt,fill=white,inner sep=2pt,outer sep=1pt] (A) {\color{white} #1};
		\node[circle,draw=none,fill=\mauSUBSEC,inner sep=1pt,outer sep=1pt] (A) {\color{white} #1};
	\end{tikzpicture}%
}
%================= Đn chương
\font\fontchap=ugqb8v at 21pt
\titlespacing{\chapter}{0cm}{0cm}{0.5cm}[0cm] %1: , 2: Trên, 3: dưới
\titleformat{\chapter}[display]
{\fontsize{20pt}{20pt}\fontfamily{qag}\selectfont\bfseries\color{\mauCHUONG}} %định dạng chung
{\fontsize{16pt}{20pt}\selectfont\chaptername\, \thechapter.} %đánh số
{1mm}
{\fontchap\centering\MakeUppercase{#1}}
[\vspace{0cm}]
%============================Mục lục - Chapter*
\titleformat{name=\chapter,numberless}[display]
{\fontsize{14pt}{16pt}\fontfamily{qag}\selectfont\bfseries\color{\mauCHUONG}} %định dạng chung
{}
{-1em}
{%
	\begin{tikzpicture}
		%-----Nội dung
		\node[inner sep=0pt,right] (ndchuong) at (0,0){\fontchap \MakeUppercase{#1}};
%		%-----Đường kẻ ngang
%		\begin{scope}
%			\clip (0,-0.75) rectangle +(\textwidth,1.5);
%			\draw[\mauchuong,line width=2pt] (ndchuong.south east)++(10pt,8pt) --++(\linewidth,0);
%		\end{scope}
	\end{tikzpicture}
}
[
\vspace{-3mm}
%\thispagestyle{empty}
]
%--------Đn Section---------------------------
%\titlespacing*{\section}{0cm}{0cm}{0cm}[0cm]
\titleformat
{\section}
{\color{\mauSEC}\fontfamily{qag}\fontsize{16pt}{1pt}\selectfont\bfseries\centering}
{Bài\,\thesection.}
{3mm}
{\MakeUppercase{#1}}
[]
%-------Đn subsection---------------------------
\titlespacing{\subsection}{0cm}{0cm}{0cm}[0cm]
\titleformat{\subsection}
{\normalfont\fontsize{15pt}{20pt}\fontfamily{put}\selectfont\bfseries\color{\mausubsec}}
{\thesubsection.}
{3mm}
{\MakeUppercase{#1}}
[]
%----------ĐN subsubsection-----------------------
\titlespacing{\subsubsection}{0pt}{0mm}{0mm}[0cm]
\titleformat{\subsubsection}
{\fontsize{13pt}{18pt}\fontfamily{put}\selectfont\bfseries\color{\mausubsubsec}}
{\thesubsubsection.}
{3mm}
{#1}
[]
%----------ĐN paragraph-----------------------
\titlespacing{\paragraph}{0pt}{0mm}{0mm}[0cm]
\titleformat{\paragraph}
{\fontsize{11.5pt}{17pt}\fontfamily{put}\selectfont\bfseries\color{\mausubsubsec}}
{\theparagraph.}
{3mm}
{#1}
[]
%============================
\def\itemKN{\color{\mauitemKN}\faCheckSquareO}
\def\itemCI{\color{\mauitemCI}\faCheckCircleO}
%%======= Thiết lập labelitem, labelenumerate
%\renewcommand{\labelitemi}{\color{red}\faCheckSquareO}
\renewcommand{\labelitemi}{\color{\mauitem}\faCheckCircleO}
\renewcommand{\labelitemii}{\color{\mauitem}\bf ---}
\renewcommand{\labelitemiii}{\color{\mauitem}\bf +}
\renewcommand{\labelenumi}{\alph{enumi})}
%\renewcommand{\labelenumii}{\color{blue}\bf\arabic{enumi}.\arabic{enumii}}
%============================
%============================
% Canh chỉnh mục lục chính
\setcounter{secnumdepth}{4} %Độ sâu đánh số
\setcounter{tocdepth}{2} %Độ sâu mục lục
\contentsmargin{0cm}
%~~~~~~~~~~~~~~~~~~~~~
\renewcommand*\l@part[2]{%
	\ifnum \c@tocdepth >-2\relax
	\addpenalty{-\@highpenalty}%
	\addvspace{10pt \@plus\p@}%
	\setlength\@tempdima{3em}%
	\begingroup
	\hypersetup{linkcolor=violet}
	\tikz[remember picture, overlay]{
		\fill[\mauPHAN] (0,0) rectangle +(\textwidth,1);
		\draw (0,0.5) node[right=5pt]
		{\color{white}\fontsize{16pt}{1pt}\fontfamily{qag}\selectfont\bfseries  {\scshape Phần} #1};
	}
	\par\smallskip
	\penalty\@highpenalty
	\endgroup
	\fi
}
%------------------------
%\titlecontents{part}[0pc]
%{\addvspace{10pt}%
%	\color{red!70!black}\fontsize{18pt}{1pt}\fontfamily{qag}\selectfont\bfseries 
%}%
%{}
%{}
%{}%
%~~~~~~~~~~~~~~~~~~~~~
\titlecontents{chapter}[6.5pc] % nd cách trái
{\addvspace{5pt}%
	\color{\mauCHUONG}\fontsize{13pt}{16pt}\fontfamily{put}\selectfont\bfseries
}%
{\contentslabel[\chaptertitlename\,\thecontentslabel.]{6.5pc}} %nhãn
{}
{\hfill\bfseries\thecontentspage
}%
[\vspace*{5pt}]
%~~~~~~~~~~~~~~~~~~~~~
\titlecontents{section}[10pc]
{\addvspace{0pt}\bfseries\color{\mauSEC}}
{\fontsize{12.5pt}{15pt}\selectfont\sffamily\contentslabel[{Bài\,\thecontentslabel.}]{3.5pc}}
{}
{\hfill
	\thecontentspage
}
[]
%~~~~~~~~~~~~~~~~~~~~~
\titlecontents{subsection}[10pc]
{\addvspace{0pt}\color{\mauSUBSEC}}
{\fontsize{12pt}{15pt}\selectfont\sffamily\contentslabel[\tron{\thecontentslabel}]{2.8pc}}
{}
{{\tiny\dotfill}\thecontentspage}
[]
%~~~~~~~~~~~~~~~~~~~~~
%--------------------------
% ĐỊNH NGHĨA CÁC MÔI TRƯỜNG 
%--------------------------
\listenumerate{dn,dl,tc,nx,ex}%xuống dòng khi liệt kê
\theoremstyle{plain} %
\theoremheaderfont{\scshape} %đầu
\theorembodyfont{\normalfont} % thân
\theoremseparator {.} % Ngăn cách
\newtheorem{dn}{\color{\maudn}\faBolt\, Định nghĩa}[section]
%===================================ĐNghĩa
\theoremstyle{plain} %
\theoremheaderfont{\fontfamily{put}\bfseries} %đầu
\theorembodyfont{\normalfont} % thân
\theoremseparator {.} % Ngăn cách
\newtheorem{vd}{\color{\mauVD}\damEX
	%\faToggleOn\ 
	%\faUnlink\ 
	VÍ DỤ}%[section]
\newtheorem{bt}{\color{\mauBT}\damEX BÀI}
%===================================
\theoremstyle{plain} %
\theoremheaderfont{\scshape} %đầu
\theorembodyfont{\slshape} % thân 
\theoremseparator {.} % Ngăn cách 
\newtheorem{dl}{\color{\maudl}\faBolt\, Định lí}[section]
\newtheorem{tc}{\color{\mauhq!70!black}\faBolt\, Tính chất}[section]
\newtheorem{hq}{\color{\mauhq!70!black}\faBolt\, Hệ quả}[section]
%====================================
\theoremstyle{nonumberplain} %ko đánh số, ko xuống dòng
\theoremheaderfont{\scshape} %đầu
\theorembodyfont{\normalfont} %phần thân
\theoremseparator {.} %ngăn cách
\newtheorem{nx}{\color{\mauhq!70!black}\faBolt\, Nhận xét}
\newtheorem{tomtat}{\!\!\!\!\!\!\!\!}
%====================================Hộp
%--------------------Chú ý
\newenvironment{note}
{\begin{tcolorbox}
		[enhanced jigsaw,breakable,pad at break*=1mm,
		opacityback=0,boxrule=0pt,frame hidden,
		left=8mm, right=0pt, bottom=0pt, top=0pt,
		before skip=1mm,
		after skip=1mm,
		underlay unbroken and first={
			\draw ([xshift=0.3cm,yshift=-0.32cm]interior.north west) node[\mauly]{\large\bfseries \faExclamationTriangle};
		},
		fontupper=\it,
		]}
	{\end{tcolorbox}}
%\let\mynote\note
%\renewcommand{\note}{\mynote{\bfseries\color{\mauly}Lưu ý:}} 
%---------------Dạng toán
\newcounter{dang}\setcounter{dang}{0}
\renewcommand{\thedang}{\arabic{dang}}
%---Dạng 1
\newtcolorbox{dang}[1]{
	fonttitle=\fontfamily{qag}\bfseries,%fontupper=\itshape,
	colframe=\maudang,colback=yellow!20,coltitle=white,
	sharp corners, breakable, halign title=center,%adjusted title=center, %canh giữa DẠNG
	before skip=2mm,after skip=3mm,
	left=2mm,right=2mm,top=2mm,bottom=2mm,
	boxrule=1pt,
	title={\faFolderOpen\ Dạng~\stepcounter{dang}\thedang.\ #1}
		\addcontentsline{toc}{subsection}{\it\sffamily \faFolderOpen\ Dạng~\thedang.#1}
		\setcounter{subsubsection}{0}
		\setcounter{vd}{0}
		\setcounter{ex}{0}
		\setcounter{bt}{0}
}
%====================
\setlength{\parindent}{0pt} %không thụt đầu dòng
%--Name
\newcounter{deso}
\font\dam=ugqb8v at 13pt
\font\damTT=ugqb8v at 18pt
%================đn notenam
\def\notename{
		\begin{tikzpicture}[remember picture,overlay,>=stealth]
		\checkoddpage\ifoddpage %nếu trang lẻ
		%--tiêu đề phải
		\path ([yshift=-\tren cm+0.5*\topset cm-0.5cm,xshift=-\phai cm-0.5*\leftnote cm+2.5mm]current page.north east) coordinate (DD); 
		%--
		\fill[white] ([yshift=-\tren cm+0.5*\topset cm-5pt,xshift=\trai cm-2pt]current page.north east) rectangle ([yshift=\duoi cm-0.5*\botset cm-12cm,xshift=-\mepphai cm+3mm]current page.north east);
		\node[inner sep =0pt,anchor=north] (thanhta) at ([yshift=-2mm]DD) {
			\includegraphics[width=4.5cm]{logo/logo.jpg}		
		};	
		\else
		%--tiêu đề phải
		\path ([yshift=-\tren cm+0.5*\topset cm-0.5cm,xshift=\phai cm+0.5*\leftnote cm-2.5mm]current page.north west) coordinate (DD); 
		%--
		\fill[white] ([yshift=-\tren cm+0.5*\topset cm-5pt,xshift=\phai cm-2pt]current page.north west) rectangle ([yshift=\duoi cm-0.5*\botset cm-12cm,xshift=\mepphai cm-3mm]current page.north west);
		\node[inner sep =0pt,anchor=north] (thanhta) at ([yshift=-2mm]DD) {
			\includegraphics[width=4.5cm]{logo/logo.jpg}		
		};
		\fi
		%\draw (thanhta.south) node[below=0pt,xscale=0.8]{\small\normalfont\color{\mauname} Sưu tầm \& Biên tập};
		%---note
		\node[inner sep =6pt, text=black,scale=1,anchor=north,fill=\maufoot!3,draw=\maufoot] (bon) at ([yshift=-1cm]thanhta.south) {
			\parbox{\leftnote cm-5mm-12pt}{ \fontsize{10}{15}\selectfont\normalfont
				\vspace*{2pt}
				\chamngon%
			}
		};
		\draw[\maufoot, line width=5pt] (bon.north west)--(bon.north east);
		\draw[\maufoot!50] ([yshift=9pt,line width=0.4pt]bon.north east)--([yshift=9pt]bon.north west)
		node[fill=white,inner sep=2pt,anchor=south west,yshift=-2pt,xshift=-2pt]{\bfseries\color{\maufoot}ĐIỂM:}
		;
		%--note dưới
		\node[inner sep =6pt, text=white,scale=1,anchor=north,fill=\maufoot] (noteduoi) at ([yshift=-0.25cm]bon.south) {
			\parbox{\leftnote cm-5mm-12pt}{ \fontsize{11}{1}\selectfont\bfseries\centering
				QUICK NOTE
			}
		};
		\draw[\maufoot, line width=0.4pt] ([yshift=-2pt]noteduoi.south west)--([yshift=-2pt]noteduoi.south east);
	\end{tikzpicture}
}
%===================note và nonote
%FULL WIDTH
\def\FULLWIDTH{
	\newpage
	\fancyhead[LO,RE]{\headmucluc}
	\def\notename{}
	\newgeometry{top=\tren cm, bottom=\duoi cm, left=\trai cm, right=\phai cm}
}
\def\NOTE{
	\newpage
	\fancyhead[LO,RE]{\myfancyhead}
	\def\notename{
		\begin{tikzpicture}[remember picture,overlay,>=stealth]
			\checkoddpage\ifoddpage %nếu trang lẻ
			%--tiêu đề phải
			\path ([yshift=-\tren cm+0.5*\topset cm-0.5cm,xshift=-\phai cm-0.5*\leftnote cm+2.5mm]current page.north east) coordinate (DD); 
			%--
			\fill[white] ([yshift=-\tren cm+0.5*\topset cm-5pt,xshift=\trai cm-2pt]current page.north east) rectangle ([yshift=\duoi cm-0.5*\botset cm-12cm,xshift=-\mepphai cm+3mm]current page.north east);
			\node[inner sep =0pt,anchor=north] (thanhta) at ([yshift=-2mm]DD) {
				\includegraphics[width=4.5cm]{logo/logo.jpg}		
			};	
			\else
			%--tiêu đề phải
			\path ([yshift=-\tren cm+0.5*\topset cm-0.5cm,xshift=\phai cm+0.5*\leftnote cm-2.5mm]current page.north west) coordinate (DD); 
			%--
			\fill[white] ([yshift=-\tren cm+0.5*\topset cm-5pt,xshift=\phai cm-2pt]current page.north west) rectangle ([yshift=\duoi cm-0.5*\botset cm-12cm,xshift=\mepphai cm-3mm]current page.north west);
			\node[inner sep =0pt,anchor=north] (thanhta) at ([yshift=-2mm]DD) {
				\includegraphics[width=4.5cm]{logo/logo.jpg}		
			};
			\fi
			%\draw (thanhta.south) node[below=0pt,xscale=0.8]{\small\normalfont\color{\mauname} Sưu tầm \& Biên tập};
			%---note
			\node[inner sep =6pt, text=black,scale=1,anchor=north,fill=\maufoot!3,draw=\maufoot] (bon) at ([yshift=-1cm]thanhta.south) {
				\parbox{\leftnote cm-5mm-12pt}{ \fontsize{10}{15}\selectfont\normalfont
					\vspace*{2pt}
					\chamngon%
				}
			};
			\draw[\maufoot, line width=5pt] (bon.north west)--(bon.north east);
			\draw[\maufoot!50] ([yshift=9pt,line width=0.4pt]bon.north east)--([yshift=9pt]bon.north west)
			node[fill=white,inner sep=2pt,anchor=south west,yshift=-2pt,xshift=-2pt]{\bfseries\color{\maufoot}ĐIỂM:}
			;
			%--note dưới
			\node[inner sep =6pt, text=white,scale=1,anchor=north,fill=\maufoot] (noteduoi) at ([yshift=-0.25cm]bon.south) {
				\parbox{\leftnote cm-5mm-12pt}{ \fontsize{11}{1}\selectfont\bfseries\centering
					QUICK NOTE
				}
			};
			\draw[\maufoot, line width=0.4pt] ([yshift=-2pt]noteduoi.south west)--([yshift=-2pt]noteduoi.south east);
		\end{tikzpicture}
	}
	\newgeometry{top=\tren cm, bottom=\duoi cm, left=\trai cm, right=\mepphai cm}
}
%===================đn name
\newcommand{\name}[4]{
%	\NOTE
%	\newpage
	\setcounter{ex}{0}\setcounter{bt}{0}%\setcounter{EX}{0}
	\boldmath\fontfamily{qag}\selectfont\color{\mauname}
\hoten \dotfill {\fontsize{10}{11}\selectfont \ngaylamde}
	\begin{tcolorbox}[boxrule=0.7pt,arc=0mm,breakable,colframe=\mauSO,colback=\mauname!2,before skip=2mm,after skip=2mm]\color{\mauname}
	\begin{center}
		%%---
		{\damTT \MakeUppercase{#1}}\\[1pt]
		{\dam \MakeUppercase{#2 --- Đề} \stepcounter{deso}\thedeso}\\[1pt]
		% {\dam \MakeUppercase{#2}}\\[1pt]
		{\dam\color{\mauSO} \MakeUppercase{#3}}\\[1pt]
		{\fontsize{10}{10}\selectfont \textit{#4}}%\\[-1mm]
	\end{center}
	\end{tcolorbox}
	%%--- Phần note đầu đề
	\notename
\vspace*{0.5cm}
	\addcontentsline{toc}{section}{\hspace*{-4.2cm}\sf Đề \thedeso: #2 --- #3} % đưa MT vào mục lục
}
%--Sang trang 
\BeforeBeginEnvironment{name}{
	\ifnum\the\value{deso}>0
	\newpage
	\fi
}
%%---Đánh số trang
%\AtEndEnvironment{name}{
%	\ifnum\the\value{deso}=1
%	\pagenumbering{arabic}%đánh số trang dạng 1,2,...
%	\fi
%}
%---------
\def\chap#1{
	\begin{center}
		\fontchap\color{\mauCHUONG} #1
	\end{center}
	\addcontentsline{toc}{chapter}{\hspace*{-2.75cm}#1}
}
%---Hiện bảng ĐA
\newcommand{\hienDA}{
	\renewcommand{\indapan}[2]{
		\addcontentsline{toc}{subsection}{\hspace*{-4.2cm}\sf Bảng đáp án} % đưa MT vào mục lục
		%		\begin{center}
		\par\vspace*{5mm}
		\begin{tikzpicture}%
			\draw (0,0)++(0.5*\textwidth,0) node[thick,scale=1,fill=\mauEX!2,draw=\maufoot,minimum width=3.5cm,minimum height=0.1cm,rounded corners=2mm] {\damEX\color{\mauname} BẢNG ĐÁP ÁN};
		\end{tikzpicture}%
		%		\end{center}
		\vspace*{-2mm}
		\inputansbox{##1}{##2}
	}
}
%---Ẩn bảng ĐA
\newcommand{\anDA}{
	\renewcommand{\indapan}[2]{}
}
%---Dòng chấm từng câu
\newcommand{\dongchamEX}[1]{
%	\hideansEX{ex}
	\anLG
	\AfterEndEnvironment{ex}{%
		\foreach \cauEX/\dongEX in {#1}{
			\ifnum\dongEX=0
			\else
			\ifnum\the\value{ex}=\cauEX
			\par\noindent\loigiaiEXS\par
			\dotlineEX{\dongEX}
			\fi
			\fi
		}
	}
}
%---Dòng chấm nhiều câu
\newcommand{\dongchamEXS}[2]{
%	\hideansEX{ex}
	\anLG
	\AfterEndEnvironment{ex}{%
		\foreach \cauEX in {#1}{
			\ifnum#2=0
			\else
			\ifnum\the\value{ex}=\cauEX
			\par\noindent\loigiaiEXS\par
			\dotlineEXS{#2}
			\fi
			\fi
		}
	}
}
%---Dòng chấm từng câu theo đề
\newcommand{\DEdongchamEX}[2]{
%	\hideansEX{ex}
	\anLG
	\AfterEndEnvironment{ex}{%
		\foreach \cauEX/\dongEX in {#2}{
			\ifnum\dongEX=0
			\else
			\ifnum\the\value{deso}=#1
			\ifnum\the\value{ex}=\cauEX
			\par\noindent\loigiaiEXS\par
			\dotlineEX{\dongEX}
			\fi
			\fi
			\fi
		}
	}
}
%---Dòng chấm nhiều câu theo đề
\newcommand{\DEdongchamEXS}[3]{
%	\hideansEX{ex}
	\anLG
	\AfterEndEnvironment{ex}{%
		\foreach \cauEX in {#2}{
			\ifnum#3=0
			\else
			\ifnum\the\value{deso}=#1
			\ifnum\the\value{ex}=\cauEX
			\par\noindent\loigiaiEXS\par
			\dotlineEX{#3}
			\fi
			\fi
			\fi
		}
	}
}
%---Ẩn LG
\newcommand{\anLG}{
	\renewcommand{\loigiai}[1]{	}%
	% \chooseNSA
	\renewcommand{\TrueTF}{\FalseTF}
	\renewcommand{\TrueEX}{\FalseEX}
	\renewcommand{\writekeyTFone}{\gdef\TrueX{}\gdef\FalseX{}}
	\renewcommand{\writekeyTF}{&&}
}
%---Hiện LG
\newcommand{\hienLG}{
	%Xuất hiện chữ Lời giải trong môi trường onlysolution
	\renewcommand{\loigiai}[1]{%
		\begin{onlysolution}%
			##1
		\end{onlysolution}%
	}%
	%---
	\def\loigiaiEXS{}
	%\choiceTF
	\renewcommand{\writekeyTFone}{\gdef\TrueX{}\gdef\FalseX{\tickF}}
	\renewcommand{\writekeyTF}{%
		&\centering\leavevmode\TrueX%
		&\parbox[t]{\linewidth}{\centering\leavevmode\FalseX}%
			\gdef\TrueX{}\gdef\FalseX{\tickF}%
	}
	\def\kindSA{ShowSAKeyColor}
	\showanswers
	% \SAOPTN{kindSA=oly}
	\renewcommand{\dotlineEXS}[1]{}	
}
%=======Đn các phương án
\def\khoanhtrondapan{
	\renewcommand*\circled[1]{\tikz[baseline=(char.base)]{
			\node[shape=circle,draw=\mauDA,inner sep=1pt] (char) {##1};}}
	\renewcommand{\TrueEX}{\stepcounter{dapan}
		{\squareEX{\textbf{\damEX\color{\mauDA}\Alph{dapan}}}} \ignorespaces}
	\renewcommand{\FalseEX}{\stepcounter{dapan}
		{\circled{\textbf{\damEX\color{\mauDA}\Alph{dapan}}}} \ignorespaces}
	%---Chọn đáp án
	\renewcommand{\circEX}[2][fill=\mauTrue!3,draw=\mauTrue]{%
	\tikz[baseline=(char.base)]{\node[shape=circle,inner sep=1pt,##1] (char) {\color{red}##2};}}
}
%----
\def\khongkhoanhtrondapan{
	\renewcommand{\TrueEX}{\stepcounter{dapan}
		{\squareEX{\textbf{\damEX\color{\mauDA}\Alph{dapan}}}} \ignorespaces}
	\renewcommand{\FalseEX}{\stepcounter{dapan}
		{\textbf{\damEX\color{\mauDA}\Alph{dapan}.}} \ignorespaces}
	%---Chọn đáp án
	\renewcommand{\circEX}[2][fill=\mauTrue!3,draw=\mauTrue]{%
	\tikz[baseline=(char.base)]{\node[shape=circle,inner sep=1pt,##1] (char) {\color{red}##2};}}
}
%=============ĐN HIỆN CÂU EX CẦN THIẾT if
\newcommand{\hienEXS}[2]{
	%\foreach \bdem in {#1,...,#2}{%11-30
	\def\biendau{#1}\def\biencuoi{#2}%
	\pgfmathsetmacro{\sodau}{\fpeval{round(\biendau-1,0)}}
	\pgfmathsetmacro{\socuoi}{\fpeval{round(\biencuoi+1,0)}}
	\setcounter{EX}{#1-1}
	\RenewEnviron{ex}{
		\stepcounter{ex}%
		\ifnum\value{ex}<\socuoi
		\ifnum\value{ex}>\sodau
		\par%
		\begin{EX}
			\BODY% 
		\end{EX}
		\fi\fi
	}
	%
	\AtEndEnvironment{name}{\setcounter{EX}{#1-1}}
	%
	\AtEndEnvironment{EX}{
		\ifnum\the\value{numTrue}=1
		\scantokens{\begin{EXsol}A\end{EXsol}}
		\fi
		\ifnum\the\value{numTrue}=2
		\scantokens{\begin{EXsol}B\end{EXsol}}
		\fi
		\ifnum\the\value{numTrue}=3
		\scantokens{\begin{EXsol}C\end{EXsol}}
		\fi
		\ifnum\the\value{numTrue}=4
		\scantokens{\begin{EXsol}D\end{EXsol}}
		\fi
		\setcounter{numTrue}{0}
	}
}
%%%%%%%%%%%%%%%%%%%%%%%

%============================ Khung
\newenvironment{khung}
{\begin{tcolorbox}[
		enhanced,breakable,
		colback=yellow!10,
		colframe=blue,
		boxrule=0.5pt,
		%		drop fuzzy shadow=gray,
		left=5pt,right=5pt,top=5pt,bottom=5pt,
		arc=0mm
		]}
	{\end{tcolorbox}}
%-----------------------------Mục con = subsub
\newcounter{muccon}
\newcommand{\muccon}[1]{%
	\stepcounter{muccon}
	{%\setcounter{bt}{0}\setcounter{vd}{0}\setcounter{ex}{0}
		%\fontsize{13pt}{15pt}\selectfont
		%		\color{violet!70!black}\sffamily
		\bfseries\sffamily\bfseries\hspace*{0mm}\themuccon.\  
		#1}
}
%----------------------------------------------------

% Hộp định nghĩa
\newenvironment{boxdn}
{\begin{tcolorbox}
		[enhanced jigsaw,breakable,pad at break*=1mm,
		colback=cyan!2,
		%standard jigsaw, 
		opacityback=0, %ko nền
		boxrule=0pt,frame hidden, left=0.7cm, right=0pt, bottom=2pt, top=0pt,
		borderline west={1mm}{0.5cm}{cyan},
		overlay={
			\fill[fill=cyan!20,draw=none] ([xshift=0.6cm]interior.north west) rectangle (interior.south east)
			;
		}
		\setcounter{muccon}{0}
		]}%0mm lề trái
	{\end{tcolorbox}}
%===============================================
\theoremstyle{nonumberbreak} % ko đánh số
\theoremheaderfont{\sffamily\bfseries} %tên
\theorembodyfont{\normalfont} %thân
\theoremsymbol{\ensuremath{_\blacksquare}} %Dấu kết thúc là ô vuông đen.
\theoremseparator {:} % Dấu ngăn cách
\newtheorem{myphantich}{\color{violet}%\faServer\ 
	\faFileText\ PHÂN TÍCH}
%===============================================
\newenvironment{phantich}{\begin{boxdn}\begin{myphantich}}{\end{myphantich}\end{boxdn}}
%-------------- Khung (Trong main này ko sd)
\newtcolorbox[auto counter]{khung4}[1]{enhanced, breakable,
	before skip=1mm,after skip=1mm,
	left=1mm,right=1mm,top=2mm,bottom=1mm,
	colframe=myblue,colback=cyan!0,colbacktitle=cyan!6,coltitle=myblue,colupper=black,sharp corners,
	,boxrule=0.4mm,
	coltext=mauE,
	attach boxed title to top center=
	{yshift=-0.1mm-\tcboxedtitleheight/2,yshifttext=2mm-\tcboxedtitleheight/2},
	varwidth boxed title*=-3cm,
	boxed title style={boxrule=0.3mm,
		frame code={ \path[tcb fill frame] ([xshift=-4mm]frame.west)
			-- (frame.north west) -- (frame.north east) -- ([xshift=4mm]frame.east)
			-- (frame.south east) -- (frame.south west) -- cycle; },
		interior code={ \path[tcb fill interior] ([xshift=-2mm]interior.west)
			-- (interior.north west) -- (interior.north east)
			-- ([xshift=2mm]interior.east) -- (interior.south east) -- (interior.south west)
			-- cycle;} 
	},
	fonttitle=\fontsize{10}{0}
	\bfseries,
	fontupper=\fontsize{10}{0},
	title={#1}
}

\newcommand{\boxmini}[1]{
	\vspace*{-2mm}
	\begin{center}
		\begin{tikzpicture}[outline/.style={draw=##1,thick,fill=##1!3},outline/.default=myblue]
			\node [outline,
			sharp corners] at (0,0) {\fontfamily{qag} \selectfont\bfseries\color{\mauEX} #1};
		\end{tikzpicture}
	\end{center}
	\vspace*{0mm}
}

%-----------------------
\newcommand{\inden}[1]{
	{\fontsize{11pt}{9pt}\sffamily \selectfont\bfseries\color{\maudn} #1}
}
\newcommand{\indam}[1]{
	{\fontsize{11.5pt}{9pt}\sffamily \selectfont\bfseries\color{\maudn} #1}
}
\newcommand{\indamm}[1]{
	{\fontsize{11.5pt}{9pt}\sffamily \selectfont\bfseries\color{\maudl} #1}
}
\newcommand{\ind}[1]{
	{\fontsize{11.5pt}{9pt}\sffamily \selectfont\bfseries\color{\maucham} #1}
}
%%%===Các biểu tượng===
\def\iconGN{{\color{magenta}\faPencilSquareO}}
\def\iconNS{{\color{gray}\faStar}}
\def\iconQS{{\color{magenta}\faFolderOpen}}
\def\iconMT{{\color{magenta!80!black}\faSunO}}
\def\iconX{{\color{red}\faClose}}
\def\iconCH{{\color{myblue}\faCheckCircle}}
\def\iconVD{\faCubes}
\def\iconCV{{\color{myblue}\faCubes}}

\newcommand{\dongcham}[1]{
	\def\sod{#1}
	\pgfmathsetmacro{\sodong}{2*\sod -1} 
	\columnsep=10pt
	\vspace*{-3.5mm}
	\begin{multicols}{2}
		\foreach \dotline in{1,...,\sodong}
		{\noindent\color{gray}{\dotfill}\\[1mm]
		}\noindent\color{gray}{\dotfill}\\[-4mm]
	\end{multicols}
}


\def\TNTF{
    {\bfseries Phần II. Trong mỗi ý a), b), c) và d) ở mỗi câu, học sinh chọn đúng hoặc sai.}
}
\def\TN{
    {\bfseries Phần I. Mỗi câu hỏi học sinh chọn một trong bốn phương án A, B, C, D.}
}
\def\TNSA{
    {\bfseries Phần III. Học sinh điền kết quả vào ô trống.}
}
\def\BTTL{
    \begin{center}
        \fcolorbox{black}{white}{{\bfseries BÀI TẬP TỰ LUẬN TRẢ LỜI NGẮN}}
    \end{center}
}
\def\BTTF{
    \begin{center}
        \fcolorbox{black}{white}{{\bfseries BÀI TẬP TRẮC NGHIỆM ĐÚNG SAI}}
    \end{center}
%    \TNTF
}
\def\BTTN{
    \begin{center}
        \fcolorbox{black}{white}{{\bfseries BÀI TẬP TRẮC NGHIỆM 4 PHƯƠNG ÁN}}
    \end{center}
}
\def\TL{
    {\bfseries Phần II. Câu hỏi tự luận.}
}
 %Khai báo cơ bản
\usepackage{tkz-euclide,circuitikz}
%%%%%%%%%%%%% ĐIỀU KHIỂN LỜI GIẢI ,DÒNG CHẤM, ĐÁP SỐ
%------------Dòng chấm bằng chiều dài LG (bật)
% \dotlinefull{ex}\dotlinefull{vd}\dotlinefull{bt}
%------------Thay Loi giải bằng n dòng kẻ (bật)
% \dotlineans{2}{ex}
%------------Ẩn lời giải
%\hideansEX{ex}
%------------Dòng chấm tùy ý (ko cần \loigiai{})
%---Nhiều câu cùng dòng chấm (tách dụng lên mọi đề)
% \dongchamEXS{13,...,19}{3}
% \dongchamEXS{20,21}{8}
%\dongchamEXS{41,...,50}{10}
%---Nhiều câu cùng dòng chấm (tách dụng lên MỘT ĐỀ đc chọn)
%\DEdongchamEXS{3}{1,...,20}{2} %{3} là đề thứ 3
%---Dòng chấm từng câu, tác dụng lên mọi đề
%\dongchamEX{1/3,2/5,3/7} % câu / số dòng chấm của câu đó
%---Dòng chấm từng câu, tác dụng lên 1 đê
%\DEdongchamEX{3}{1/3,2/5,3/7} % câu / số dòng chấm của câu đó, {3} là đề số 3 
\renewcommand{\dongcham}[1]{}
%------------Ẩn đáp số (bật), đáp án
\exitdapso %ẩn đs
%\renewcommand{\indapan}[2]{} %ẩn đáp án
%%%%%%%%%%%%% khung NAME
\def\hoten{Gọi tôi là:}
\def\ngaylamde{Ngày làm đề: ...../...../........} %để {} nếu ko muốn
\def\tenchude{MẪU SỐ LIỆU GHÉP NHÓM}
\def\tendethi{ÔN TẬP KIỂM TRA CHƯƠNG I}
\def\tentruong{LỚP TOÁN THẦY PHÁT}
\def\thoigian{Thời gian: 90 phút - Không kể thời gian phát đề}
%%%%%%%%%%%%% Nội dung head & foot
% \def\diachi{ }
\def\diachi{VNPmath - 0962940819}
\def\tenchuyende{\tenchude}
\def\tentacgia{GV.VŨ NGỌC PHÁT}
\def\chamngon{\lq\lq  It's not how much time you have, it's how you use it.\rq\rq
}
%%%%%%%%%%%%% Đn lại A.B.C.D
\khoanhtrondapan
% \khongkhoanhtrondapan
%%%%%%%%%%%%%
\renewcommand{\arraystretch}{1}

\newenvironment{boxdl}
{\begin{tcolorbox}
		[enhanced jigsaw,breakable,pad at break*=1mm,
		boxrule=0pt,frame hidden, left=2mm, right=0pt, bottom=1.5pt, top=1.5pt,
		before skip=2mm,
		after skip=2mm,
		%		borderline west={1mm}{0cm}{green!70!black}
		overlay={\draw[double,line width=1.5pt,\maudl] ([xshift=3pt]interior.north west)--([xshift=3pt]interior.south west);}
		]}%0mm lề trái
	{\end{tcolorbox}}
%%===================================================

\OPTN{kindTF=t,dapanTF=a,boldTF=1,phatbieu=Mệnh đề,viettat=1}
% \SAOPTN{kindSA=oly,ketquaSA=KQ:,widthSA=4,heightSA=0.9,dapanSA=a}
%=================BẮT ĐẦU TÀI LIỆU===================

\begin{document}
\renewcommand{\chaptername}{Chương}
\pagenumbering{arabic}%đánh số trang dạng 1,2,...
%====================================================
%==================BẮT ĐẦU TÀI LIỆU==================
%\hienEXS{41}{50} %chỉ hiện câu từ 41 đến 50 của đề
% --------Đề bài
% \NOTE \anLG \anDA 
% \FULLWIDTH \anLG \anDA
% \notename
% %Chương I
%%Bài 1. GTLG
% 
\section{Giá trị lượng giác của một góc lượng giác}
\subsection{Tóm tắt lý thuyết}
\begin{tomtat}
	\subsubsection{Khái niệm góc lượng giác và số đo của góc lượng giác}
	Trong mặt phẳng, cho hai tia $Ou$, $Ov$. Xét tia $Om$ cùng nằm trong mặt phẳng này. Nếu tia $Om$ quay quanh điểm $O$, theo một chiều nhất định từ $Ou$ đến $Ov$, thì ta nói nó quét một góc lượng giác với tia đầu  $Ou$, tia cuối $Ov$ và kí hiệu là ($Ou$, $Ov$).\\
	Mỗi góc lượng giác gốc $O$ được xác định bởi tia đầu $Ou$, tia cuối $Ov$ và số đo của nó.
	\begin{center}
		\begin{minipage}[H]{0.3\textwidth}
			\begin{tikzpicture}[scale=.7]	
				\draw (0,0) -- (4,0)node[below] {$u$};
				\draw[red] (0,0) -- (45:4)node[below right] {$v$};
				\draw[dashed,green!50!black] (0,0) -- (20:4)node[below right] {$m$};
				\draw[-stealth,red] (0:1) arc (0:45:1);
				\draw[-stealth] (2.75,1) arc (0:45:1);
				\path (30:3.5) node[below=-2pt]{$+$};
				\path (0:0) node[below left]{$O$};
			\end{tikzpicture}
		\end{minipage}
		\begin{minipage}[H]{0.3\textwidth}
			\begin{tikzpicture} [scale=.7]	
				\draw (0,0) -- (4,0)node[below] {$u$};
				\draw[red] (0,0) -- (45:4)node[below right] {$v$};
				\draw[dashed,green!50!black] (0,0) -- (75:3.5)node[below right] {$m$};
				\draw[red,-stealth,smooth,samples=100] plot[domain =0:2.25*pi]({.5*(1.1)^(\x) *cos(\x r)},{.5*(1.1)^(\x) *sin(\x r)});
				\draw[-stealth] (0.5,2) arc (75:110:2);
				\path (90:2.5) node[below=-2pt]{$+$};
				\path (0:0) node[below left]{$O$};
			\end{tikzpicture}
		\end{minipage}
		\begin{minipage}[H]{0.3\textwidth}
			\begin{tikzpicture}[scale=.7]		
				\draw (0,0) -- (4,0)node[below] {$u$};
				\draw[red] (0,0) -- (45:4)node[below right] {$v$};
				\draw[dashed,green!50!black] (0,0) -- (120:3)node[above right] {$m$};
				\draw[-stealth,red] (0:.8) arc (0:-315:.8);
				\draw[-stealth] (-1,1.7) arc (120:60:1);
				\path (105:2.4) node[below=-2pt]{$-$};
				\path (0:0) node[below left]{$O$};
			\end{tikzpicture}
		\end{minipage}
	\end{center}
	\subsubsection{Hệ thức Chasles}
	\immini{Hệ thức Chasles: Với ba tia $Ou$, $Ov$, $Ow$ bất kì, ta có 	
	$$
		\text{sđ}(Ou, Ov)+\text{sđ}(Ov,Ow)=\text{sđ}(Ou,Ow)+k 360^{\circ}(k \in \mathbb{Z}). 
		$$}
	{\begin{tikzpicture}[scale=0.77, font=\footnotesize, line join=round, line cap=round, >=stealth]		
			\draw (0,0) -- (4,0)node[below] {$u$};
			\draw[red] (0,0) -- (75:3.5)node[below right] {$w$};
			\draw[green!50!black] (0,0) -- (30:4)node[below right] {$v$};
			\draw[-stealth,red] (0:1) arc (0:-285:1);
			\draw[-stealth] (0:.9) arc (0:30:.9);
			\draw[-stealth] (.86,.5) arc (30:75:1);
			\path (0:0) node[below left]{$O$};
	\end{tikzpicture}}
	% Nhận xét. Từ hệ thức Chasles, ta suy ra:
	% Với ba tia tuỳ ý $Ox$, $Ou$, $Ov$ ta có
	% $$
	% \text{sđ}(Ou, Ov)=\text{sđ}(Ox, Ov)-\text{sđ}(Ox,Ou)+k360^{\circ}(k \in \mathbb{Z}). 
	% $$
	% Hệ thức này đóng vai trò quan trọng trong việc tính toán số đo của góc lượng giác.
	\subsubsection{Đơn vị đo góc và cung tròn}
	\textbf{Đơn vị độ}: Góc $1^{\circ}$ bằng $\dfrac{1}{180}$ góc bẹt.\\
	Đơn vị độ được chia thành những đơn vị nhỏ hơn: $1^{\circ}=60'; 1'=60"$.\\
	% Đối với các góc lượng giác, khi mà số vòng quay trong chuyển động tương ứng từ tia đầu đến tia cuối là khá lớn thì số đo của chúng tính bằng độ sẽ trở nên cồng kềnh. Do đó, trong khoa học và kĩ thuật, bên cạnh việc đo bằng độ, người ta còn sử dụng đơn vị đo góc bằng rađian.\\
	\immini{\textbf{Đơn vị rađian}: Cho đường tròn $(O)$ tâm $O$, bán kính $R$ và một cung $AB$ trên $(O)$.
		Ta nói cung tròn $AB$ có số đo bằng 1 rađian nếu độ dài của nó đúng bằng bán kính $R$.
		Khi đó ta cũng nói rằng góc $AOB$ có số đo bằng 1 rađian và viết: $\overset\frown{AOB}=1$ rad.}
	{\begin{tikzpicture}[scale=0.77, font=\footnotesize, line join=round, line cap=round, >=stealth]		
			\draw[green!50!black] (30:2) arc (30:-270:2);
			\draw[red] (30:2) arc (30:90:2);
			\draw (90:2)node[above]{$B$}--(0:0)--(30:2)node[above right]{$A$};
			\path (0:0) node[below left]{$O$};
			\draw(60:2) node[above right]{1 rad};
	\end{tikzpicture}}
	\textbf{Quan hệ giữa độ và rađian:}
	$$
	1 \text{ góc bẹt }=180^\circ = 1 \mathrm{rad} \Leftrightarrow  1^\circ=\dfrac{\pi}{180} \mathrm{rad} \quad \text { và }\quad 1\,  \mathrm{rad}=\left(\dfrac{180}{\pi}\right)^\circ.
	$$
	\begin{note}
		Khi viết số đo của một góc theo đơn vị rađian, người ta thường không viết chữ rad sau số đo. Chẳng hạn góc $\dfrac{\pi}{2}$ được hiểu là góc $\dfrac{\pi}{2}$ rad.
	\end{note}
	\begin{note}
		Dưới đây là bảng tương ứng giữa số đo bằng độ và số đo bằng rađian của các góc đặc biệt trong phạm vi từ $0^{\circ}$ đến $180^{\circ}$.
	\end{note}
	\begin{center}
		\renewcommand{\arraystretch}{2}
		\begin{tabular}{|l|c|c|c|c|c|c|c|c|c|}
			\hline Độ & $0^{\circ}$ & $30^{\circ}$ & $45^{\circ}$ & $60^{\circ}$ & $90^{\circ}$ & $120^{\circ}$ & $135^{\circ}$ & $150^{\circ}$ & $180^{\circ}$ \\
			\hline Rađian & 0 & $\dfrac{\pi}{6}$ & $\dfrac{\pi}{4}$ & $\dfrac{\pi}{3}$ & $\dfrac{\pi}{2}$ & $\dfrac{2 \pi}{3}$ & $\dfrac{3 \pi}{4}$ & $\dfrac{5 \pi}{6}$ & $\pi$ \\
			\hline
		\end{tabular}
	\end{center}
	\subsubsection{Độ dài cung tròn}
	Một cung của đường tròn bán kính $R$ và có số đo $\alpha$ rad thì có độ dài $l=R \alpha$.
	\subsubsection{Đường tròn lượng giác}
	\immini{\begin{itemize}
			\item Đường tròn lượng giác là đường tròn có tâm tại gốc toạ độ, bán kính bằng $1$, được định hướng và lấy điểm $A(1 ; 0)$ làm điểm gốc của đường tròn.
			\item Điểm trên đường tròn lượng giác biểu diễn góc lượng giác có số đo $\alpha$ là điểm $M$ trên đường tròn lượng giác sao cho sđ$(OA, OM)=\alpha$.
	\end{itemize}
	\begin{note}
		Góc $\alpha$ và $\beta$ có chung điểm biểu diễn khi \fbox{$\alpha - \beta = k2\pi$} (chẵn lần $\pi$)
		\end{note}}
	{
		\begin{tikzpicture}[line join = round, line cap = round, >=stealth, font=\footnotesize, scale=0.6]
			\tikzset{label style/.style={font=\footnotesize}}
			\path (0,0) coordinate (O)
			(3,0) coordinate (A)
			(0,3) coordinate (B)
			(0,-3) coordinate (B')
			(-3,0) coordinate (A')
			(0:0)++(150:3) coordinate (M)
			($(O)!(M)!(A')$) coordinate (H)
			($(O)!(M)!(B)$) coordinate (K)
			;
			\draw[->] (-4,0) -- (4,0) node[above,blue]{$x$};
			\draw[->] (0,-4) -- (0.,4) node[left,blue]{$y$};
			\draw[orange] (O) circle (3cm);
			\draw[rotate=0,->,green!50!black] (0.5,0) arc (0:150:0.5cm);
			\draw (0.35,0.25) node[above,blue] {$\alpha$};
			\draw[dashed] (H)--(M)--(K);
			\draw[green!50!black] (M)--(O);
			\foreach \p/\r in {A/-45,M/150,H/-90,O/-150,A'/-135,B'/-45,B/45,K/0}
			\fill (\p) circle (1pt) node[shift={(\r:3mm)},blue]{$\p$};
		\end{tikzpicture}
	}
	\subsubsection{Các giá trị lượng giác của góc lượng giác}
	\immini{Gọi $M(x;y)$ là điểm biểu diễn của góc lượng giác $\alpha$ trên đường tròn lượng giác. Khi đó, ta có:
		\begin{itemize}
			\item $\cos\alpha=x.$
			\item $\sin\alpha=y.$
			\item $\tan\alpha=\dfrac{\sin\alpha}{\cos\alpha}=\dfrac{y}{x} ~(x\neq0).$
		\item $\cot\alpha=\dfrac{\cos\alpha}{\sin\alpha}=\dfrac{x}{y} ~(y\neq0).$
	\end{itemize}}
	{\begin{tikzpicture}[line join = round, line cap = round, >=stealth, font=\footnotesize, scale=0.6]
			\tikzset{label style/.style={font=\footnotesize}}
			\path (0,0) coordinate (O)
			(3,0) coordinate (A)
			(0,3) coordinate (B)
			(0,-3) coordinate (B')
			(-3,0) coordinate (A')
			(0:0)++(40:3) coordinate (M)
			($(O)!(M)!(A')$) coordinate (H)
			($(O)!(M)!(B)$) coordinate (K)
			;
			\draw[->] (-4,0) -- (4,0) node[above,blue]{$x$};
			\draw[->] (0,-4) -- (0.,4) node[left,blue]{$y$};
			\draw[orange] (O) circle (3cm);
			\draw[rotate=0,->,green!50!black] (0.7,0) arc (0:40:0.7cm);
			\draw (1,0) node[above,blue] {$\alpha$};
			\draw[dashed] (H)--(M)--(K);
			\draw[green!50!black] (M)--(O);
			\draw[blue,fill=black] (0,2) node[left]{$\sin\alpha$}(2,0) circle(1pt) node[below]{$\cos\alpha$}(3,2.3) node{$M(x;y)$};
			\foreach \p/\r in {A/-45,O/-135,A'/-135,B'/-45,B/45}
			\fill (\p) circle (1pt) node[shift={(\r:3mm)},blue]{$\p$};
	\end{tikzpicture}}
	\begin{note}
		a) Ta còn gọi trục tung là trục sin, trục hoành là trục côsin.\\
		b) Từ định nghĩa ta suy ra:
		\begin{itemize}
			\item $\sin\alpha$, $\cos\alpha$ xác định với mọi giá trị của $\alpha$ và ta có:
			$$-1\leq \sin\alpha\leq 1; \quad -1\leq \cos\alpha\leq 1; \quad \sin(\alpha+k2\pi)=\sin\alpha;\quad \cos(\alpha+k2\pi)=\cos\alpha\,\, (k\in\mathbb{Z}).$$
			\item $\tan\alpha$ xác định khi $\alpha\neq\dfrac{\pi}{2}+k\pi\,\,  (k\in\mathbb{Z})$.
			\item $\cot\alpha$ xác định khi $\alpha\neq k\pi\,\,  (k\in\mathbb{Z})$.
			\item Dấu của các giá trị lượng giác của một góc lượng giác phụ thuộc vào vị trí điểm biểu diễn $M$ trên đường tròn lượng giác.
		\end{itemize}
	\end{note}
	\begin{minipage}[h]{0.6\textwidth}
		\begin{tabular}{c|c|c|c|c|}
			\cline{2-5}
			& \multicolumn{4}{c|}{Góc phần tư} \\ \hline
			\multicolumn{1}{|c|}{Giá trị lượng giác} & I     & II     & III     & IV    \\ \hline
			\multicolumn{1}{|c|}{$\sin \alpha$}     &   $+$    &  $ +$      &    $-$    &   $-$  \\ \hline
			\multicolumn{1}{|c|}{$\cos \alpha$}     &   $+$    &  $ -$      &    $-$    &   $+$  \\ \hline
			\multicolumn{1}{|c|}{$\tan \alpha$}     &   $+$    &  $ -$      &    $+$    &   $-$  \\ \hline
			\multicolumn{1}{|c|}{$\cot \alpha$}     &   $+$    &  $ -$      &    $+$    &   $-$  \\ \hline
		\end{tabular}
	\end{minipage}
	\begin{minipage}[h]{0.6\textwidth}
		\begin{tikzpicture}[line join = round, line cap = round, >=stealth, font=\footnotesize, scale=0.6]
			\tikzset{label style/.style={font=\footnotesize}}
			\path (0,0) coordinate (O)
			(3,0) coordinate (A)
			(0,3) coordinate (B)
			(0,-3) coordinate (B')
			(-3,0) coordinate (A')
			(0:0)++(-60:3) coordinate (M)
			($(O)!(M)!(A')$) coordinate (H)
			($(O)!(M)!(B)$) coordinate (K)
			;
			\draw[->] (-4,0) -- (4,0) node[above,blue]{$x$};
			\draw[->] (0,-4) -- (0.,4) node[left,blue]{$y$};
			\draw[orange] (O) circle (3cm);
			\draw[rotate=0,->,green!50!black] (0.5,0) arc (0:-60:0.5cm);
			\draw (0.75,-0.35) node[blue] {$\alpha$};
			\draw[dashed] (H)--(M)--(K);
			\draw[green!50!black] (M)--(O);
			\draw[blue] (2.5,2.5) node{$I$}(-2.5,2.5) node{$II$}(-2.5,-2.5) node{$III$}(2.5,-2.5) node{$IV$};
			\foreach \p/\r in {A/-45,M/-60,H/90,O/-150,A'/-135,B'/-45,B/45,K/180}
			\fill (\p) circle (1pt) node[shift={(\r:3mm)},blue]{$\p$};
		\end{tikzpicture}
	\end{minipage}
	% \subsubsection{Giá trị lượng giác của các góc đặc biệt}
	% \begin{center}
	% 	\renewcommand{\arraystretch}{2}
	% 	\begin{tabular}{|c|c|c|c|c|c|}
	% 		\hline
	% 		\multirow{2}{*}{Góc $\alpha$} & $0$              & $\dfrac{\pi}{6}$  & $\dfrac{\pi}{4}$  & $\dfrac{\pi}{3}$  & $\dfrac{\pi}{2}$              \\ \cline{2-6} 
	% 		& $0^\circ$              & $30^\circ$  & $45^\circ$  & $60^\circ$  & $90^\circ$             \\ \hline
	% 		$\sin\alpha$                  & $0$             & $\dfrac{1}{2}$ & $\dfrac{\sqrt{2}}{2}$ & $\dfrac{\sqrt{3}}{2}$ & 1              \\ \hline
	% 		$\cos\alpha$                  & $1$             & $\dfrac{\sqrt{3}}{2}$ & $\dfrac{\sqrt{2}}{2}$ & $\dfrac{1}{2}$ & 0              \\ \hline
	% 		$\tan\alpha$                 & $0$             & $\dfrac{1}{\sqrt{3}}$ & 1 & $\sqrt{3}$ & Không xác định \\ \hline
	% 		$\cot\alpha$                  & Không xác định & $\sqrt{3}$ & 1 & $\dfrac{1}{\sqrt{3}}$ & 0              \\ \hline
	% 	\end{tabular}
	% \end{center}
	\subsubsection{Các công thức lượng giác cơ bản}
	Đối với các giá trị lượng giác, ta có các hệ thức cơ bản sau
	\begin{enumEX}[$\bullet$]{2}
		\item $\sin^2 \alpha  + \cos^2 \alpha =1$
		\item $ 1+ \tan^2 \alpha= \dfrac{1}{\cos^2 \alpha}$ $\left(\alpha \neq \dfrac{\pi}{2}+k\pi , k\in \mathbb{Z}\right)$
		\item $ 1+ \cot^2 \alpha= \dfrac{1}{\sin^2 \alpha}$ $\left(\alpha \neq k\pi , k\in \mathbb{Z}\right)$
		\item $\tan \alpha \cdot \cot \alpha =1 $ $\left(\alpha \neq \dfrac{k\pi}{2}, k\in \mathbb{Z}\right)$
	\end{enumEX}
	\newpage
	\subsubsection{Giá trị lượng giác của các góc có liên quan đặc biệt}
	\begin{enumerate}
		\item Góc đối nhau ($\alpha$ và $-\alpha$)
		\immini{\begin{itemize}
				\item $\cos (-\alpha)=\cos \alpha$
				\item $\sin (-\alpha) =-\sin \alpha$
				\item $\tan (-\alpha) =-\tan \alpha$
				\item $\cot (-\alpha) =-\cot \alpha$
		\end{itemize}}
		{\vspace*{-1cm}\begin{tikzpicture}[line join = round, line cap = round, >=stealth, font=\footnotesize, scale=0.5]
				\tikzset{label style/.style={font=\footnotesize}}
				\path (0,0) coordinate (O)
				(3,0) coordinate (A)
				(0:0)++(120:3) coordinate (M)
				(0:0)++(-120:3) coordinate (N)
				(0,4) coordinate (C)
				(0,-4) coordinate (D)
				($(O)!(M)!(C)$) coordinate (E)
				($(O)!(N)!(D)$) coordinate (F)
				;
				\draw[->] (-4,0) -- (4,0) node[above,blue]{$x$};
				\draw[->] (0,-4) -- (0.,4) node[left,blue]{$y$};
				\draw[orange] (O) circle (3cm);
				\draw[rotate=0,->,red] (0.5,0) arc (0:120:0.5cm);
				\draw[rotate=0,->,green!50!black] (0.6,0) arc (0:-120:0.6cm);
				\draw (0,0) node[above right=2pt,blue] {$\alpha$} (0,-0) node[below right=2pt,blue]{$-\alpha$};
				\draw[dashed] (E)--(M)--(N)--(F);
				\draw[green!50!black] (M)--(O);
				\draw[red] (N)--(O);
				\foreach \p/\r in {A/-45,M/120,N/-120,O/-150}
				\fill (\p) circle (1pt) node[shift={(\r:3mm)},blue]{$\p$};
		\end{tikzpicture}}
		\item Góc bù nhau ($\alpha$ và $\pi-\alpha$)
		\immini{\begin{itemize}
				\item $\sin (\pi -\alpha)=\sin \alpha$
				\item $\cos (\pi -\alpha) =-\cos \alpha$
				\item $\tan (\pi -\alpha) =-\tan \alpha$
				\item $\cot (\pi -\alpha) =-\cot \alpha$
		\end{itemize}}
		{\vspace*{-0.5cm}\begin{tikzpicture}[line join = round, line cap = round, >=stealth, font=\footnotesize, scale=0.5]
				\tikzset{label style/.style={font=\footnotesize}}
				\path (0,0) coordinate (O)
				(3,0) coordinate (A)
				(0:0)++(30:3) coordinate (M)
				(0:0)++(150:3) coordinate (N)
				(4,0) coordinate (C)
				(-4,0) coordinate (D)
				($(O)!(M)!(C)$) coordinate (E)
				($(O)!(N)!(D)$) coordinate (F)
				;
				\draw[->] (-4,0) -- (4,0) node[above,blue]{$x$};
				\draw[->] (0,-4) -- (0.,4) node[left,blue]{$y$};
				\draw[orange] (O) circle (3cm);
				\draw[rotate=0,->,red] (0.5,0) arc (0:150:0.5cm);
				\draw[rotate=0,->,green!50!black] (1.6,0) arc (0:30:1.6cm);
				\draw (2,0) node[above,blue] {$\alpha$} (0.3,1.5) node[below,blue]{$\pi-\alpha$};
				\draw[dashed] (E)--(M)--(N)--(F);
				\draw[red] (O)--(N);
				\draw[green!50!black] (O)--(M);
				\foreach \p/\r in {A/-45,M/30,N/150,O/-130}
				\fill (\p) circle (1pt) node[shift={(\r:3mm)},blue]{$\p$};
		\end{tikzpicture}}
		\item Góc phụ nhau ($\alpha$ và $\dfrac{\pi}{2}-\alpha$)
		\immini{\begin{itemize}
				\item $\sin \left( \dfrac{\pi}{2}-\alpha\right)=\cos \alpha$
				\item $\cos \left( \dfrac{\pi}{2}-\alpha\right)=\sin \alpha$
				\item $\tan \left( \dfrac{\pi}{2}-\alpha\right)=\cot \alpha$
				\item $\cot \left( \dfrac{\pi}{2}-\alpha\right)=\tan \alpha$
		\end{itemize}}
		{\vspace*{-0.5cm}\begin{tikzpicture}[line join = round, line cap = round, >=stealth, font=\footnotesize, scale=0.5]
				\tikzset{label style/.style={font=\footnotesize}}
				\path (0,0) coordinate (O)
				(3,0) coordinate (A)
				(0:0)++(20:3) coordinate (M)
				(0:0)++(70:3) coordinate (N)
				(0,4) coordinate (C)
				(4,0) coordinate (D)
				($(O)!(M)!(C)$) coordinate (E)
				($(O)!(N)!(D)$) coordinate (F)
				(2.82,0) coordinate (G)
				(0,2.82) coordinate (H)
				;
				\draw[->] (-4,0) -- (4,0) node[above,blue]{$x$};
				\draw[->] (0,-4) -- (0.,4) node[left,blue]{$y$};
				\draw[orange] (O) circle (3cm);
				\draw[rotate=0,->,red] (0.7,0) arc (0:70:0.7cm);
				\draw[rotate=0,->,green!50!black] (1.6,0) arc (0:20:1.6cm);
				\draw (2,0) node[above,blue] {$\alpha$} (1,-.2) node[below,blue]{$\frac{\pi}{2}-\alpha$};
				\draw[dashed] (E)--(M) (F)--(N) (G)--(M) (H)--(N);
				\draw[dashed] (-3,-3)--(3,3);
				\draw[->] (0.8,-.5)--(0.5,0.45);
				\draw[red] (O)--(N);
				\draw[green!50!black] (O)--(M);
				\foreach \p/\r in {A/-45,M/20,N/70,O/-220}
				\fill (\p) circle (1pt) node[shift={(\r:3mm)},blue]{$\p$};
		\end{tikzpicture}}
		\item Góc hơn kém $\pi$ ($\alpha$ và $\pi+\alpha$)
		\immini{\begin{itemize}
				\item $\sin (\pi +\alpha)=-\sin \alpha$
				\item $\cos (\pi +\alpha)=-\cos \alpha$
				\item $\tan (\pi +\alpha)=\tan \alpha$
				\item $\cot (\pi +\alpha)=\cot \alpha$
		\end{itemize}}
		{\vspace*{-0.5cm}\begin{tikzpicture}[line join = round, line cap = round, >=stealth, font=\footnotesize, scale=0.5]
				\tikzset{label style/.style={font=\footnotesize}}
				\path (0,0) coordinate (O)
				(3,0) coordinate (A)
				(0:0)++(60:3) coordinate (M)
				(0:0)++(240:3) coordinate (N)
				;
				\draw[->] (-4,0) -- (4,0) node[above,blue]{$x$};
				\draw[->] (0,-4) -- (0.,4) node[left,blue]{$y$};
				\draw[orange] (O) circle (3cm);
				\draw[rotate=0,->,red] (1.7,0) arc (0:240:1.7cm);
				\draw[rotate=0,->,green!50!black] (0.6,0) arc (0:60:0.6cm);
				\draw (1,0) node[above,blue] {$\alpha$};
				\draw (-1.2,1) node[below,blue,rotate=60]{$\pi+\alpha$};
				\draw[red] (O)--(N);
				\draw[green!50!black] (O)--(M);
				\foreach \p/\r in {A/-45,M/60,N/240,O/-150}
				\fill (\p) circle (1pt) node[shift={(\r:3mm)},blue]{$\p$};
		\end{tikzpicture}}
	\end{enumerate}
\end{tomtat}

% \foreach \i in {1,2,...,7} {\input{data/11KNTT/data1/1K1-2-\i.tex}}
%%Bài 2. CTLG
% %\chapter{Hàm số  lượng giác và phương trình lượng giác}
\setcounter{section}{1}
\section{Công thức lượng giác}
\subsection{Tóm tắt lý thuyết}
\begin{tomtat}
% 	\begin{center}
% 		\begin{tikzpicture}[scale = 2.5]
% 			\path (0,0) coordinate (O) (1.5,0) coordinate (x) (0,1.5) coordinate (y);
% 			\draw[thick,->] (-1.5,0)--(x);
% 			\draw[thick,->] (0,-1.5)--(y);
% 			\draw (O) circle (1);
% 			\path ($(O)+(55:1)$) coordinate (M) 
% 			($(O)+(30:1)$) coordinate (N);
% 			\path ($(O)!(M)!(x)$) coordinate (x_M)
% 			($(O)!(M)!(y)$) coordinate (y_M)
% 			($(O)!(N)!(x)$) coordinate (x_N)
% 			($(O)!(N)!(y)$) coordinate (y_N);
% 			\draw[dashed] (x_M)--(M)--(y_M) (x_N)--(N)--(y_N);
% 			\foreach \x/\g in {O/-135,x/-90,y/180,x_M/-90,x_N/-90,y_M/180,y_N/180}
% 			\fill ($(\g:1mm)+(\x)$) node {$\x$};
% 			\fill 	(M) circle (0.5pt)
% 			($(15:4mm)+(M)$) node {$M\left(x_M,y_M\right)$};
% 			\fill (N) circle (0.5pt)
% 			($(15:4mm)+(N)$) node {$N\left(x_N,y_N\right)$};
% 	\draw (M)--(O)--(N);		
% 	\draw pic[draw,,angle radius=6mm,->,red]{angle=x--O--M};
% 	\fill[red] (45:3mm) node {$\alpha$};
% 	\draw pic[red,draw,,angle radius=10mm,->]{angle=x--O--N};
% 	\fill[red] (15:5mm) node {$\beta$};
% 		\end{tikzpicture}
% 	\end{center}
% Trong mặt phẳng $Oxy$ cho hai điểm $M,N$ trên đường tròn lượng giác.\\ Đặt $\alpha = \text{sđ} (Ox,OM), \beta = \text{sđ} (Ox,ON)$, ta có $M(\cos \alpha,\sin \alpha)$ và $N(\cos \beta, \sin \beta)$. Khi đó ta tính được $\overrightarrow{OM}.\overrightarrow{ON}$ bằng hai cách
% \begin{align*}
% 	\overrightarrow{OM}.\overrightarrow{ON}&=\left|\overrightarrow{OM}\right|.\left|\overrightarrow{ON}\right|.\cos \left(\overrightarrow{OM},\overrightarrow{ON}\right) = \cos (\alpha-\beta),\\
% 	\overrightarrow{OM}.\overrightarrow{ON} &= x_Mx_N+y_My_N= \cos \alpha \cos\beta +\sin\alpha\sin\beta.
% \end{align*}
% Từ đó dẫn tới công thức
% \begin{align*}
% 	\cos (\alpha-\beta) = \cos \alpha \cos \beta + \sin \alpha\sin \beta \tag{$\star$}
% \end{align*}
% Tất cả các công thức trong bài học được xây dựng dựa trên công thức $(\star)$.\\
% Trong suốt bài học, khi không nói gì thêm, chỉ xét các góc lượng giác mà trong đó giá trị lượng giác được để cập có nghĩa. 
	\subsubsection{Công thức cộng}
	\begin{khung4}{Công thức cộng}
	\begin{tasks}[style=itemize](2)
		\task $\cos (a-b) = \cos a \cos b + \sin a\sin b$.
		\task $\cos (a+b) = \cos a \cos b - \sin a\sin b$.
		\task $\sin (a-b) = \sin a \cos b - \sin b \cos a$.
		\task $\sin (a+b) = \sin a \cos b + \sin b \cos a$.
		\task $\tan (a-b) = \dfrac{\tan a - \tan b}{1+\tan a \tan b}$.
		\task $\tan (a+b) = \dfrac{\tan a + \tan b}{1-\tan a \tan b}$.
	\end{tasks}
	\end{khung4}
	\subsubsection{Công thức nhân đôi}
	Công thức nhân đôi được xây dựng bằng cách thay $b=a$ trong công thức cộng.
	\begin{khung4}{Công thức nhân đôi}
		\begin{tasks}[style=itemize]
			\task $\sin 2a = 2\sin a \cos a$.
			\task $\cos 2a = \cos^2a-\sin^2a = 2\cos^2a-1 = 1-2\sin^2a$.
			\task $\tan 2a = \dfrac{2\tan a}{1-\tan^2a}$.
		\end{tasks}
		\end{khung4}
\begin{note}
	Từ công thức nhân đôi, ta có công thức hạ bậc:
\end{note}
	\begin{khung4}{Công thức hạ bậc}
		\begin{tasks}[style=itemize](3)
			\task $\sin^2a= \dfrac{1-\cos 2a}{2}$.
		\task $\cos^2a = \dfrac{1+\cos 2a}{2}$.
		\task $\tan^2a=\dfrac{1-\cos2a}{1+\cos 2a}$.
		\end{tasks}
		\end{khung4}

\begin{note}
	Áp dụng công thức cộng cho $3a = a +2a$, ta có công thức nhân ba:
\end{note}
	\begin{khung4}{Công thức nhân ba}
		\begin{tasks}[style=itemize](2)
			\task $\sin3a= 3\sin a -4\sin^3a$.
		\task $\cos3a= 4\cos^3a-3\cos a$.
		\task $\tan3a = \dfrac{3\tan a - \tan^3 a}{1-3\tan^2a}$.
		\end{tasks}
		\end{khung4}

	\subsubsection{Công thức biến đổi tích thành tổng}
	\begin{khung4}{Công thức tích thành tổng}
		\begin{tasks}[style=itemize]
			\task $\cos a \cos b = \dfrac{1}{2}\left[\cos (a-b) + \cos (a+b)\right]$.
		\task $\sin a \sin b = \dfrac{1}{2}\left[\cos (a-b)-\cos(a+b)\right]$.
		\task $\sin a \cos b = \dfrac{1}{2}\left[\sin (a-b)+\sin (a+b)\right]$.
		\end{tasks}
		\end{khung4}
	\subsubsection{Công thức biến đổi tổng thành tích}
	Công thức biến đổi tổng thành tích được xây dựng bằng cách $a=\dfrac{a+b}{2}, b = \dfrac{a-b}{2}$ trong công thức biến đổi tích thành tổng.
	\begin{khung4}{Công thức tổng thành tích}
		\begin{tasks}[style=itemize](2)
			\task $\cos a+ \cos b = 2\cos\dfrac{a+b}{2}\cos \dfrac{a-b}{2}$.
		\task $\cos a- \cos b = -2\sin\dfrac{a+b}{2}\sin \dfrac{a-b}{2}$.
		\task $\sin a+ \sin b = 2\sin\dfrac{a+b}{2}\cos \dfrac{a-b}{2}$.
		\task $\sin a -\sin b = 2\cos\dfrac{a+b}{2}\sin \dfrac{a-b}{2}$.
		\end{tasks}
		\end{khung4}
\end{tomtat}


% \foreach \i in {1,2,...,5} {\input{data/11KNTT/data1/1K1-2-\i.tex}}
%%Bài 3. HSLG
% \setcounter{section}{2}
\section{Hàm số lượng giác}
\subsection{Tóm tắt lý thuyết}
\begin{tomtat}
	% \subsubsection{Định nghĩa hàm số lượng giác}
	% \begin{dn}
	% 	\begin{itemize}
	% 		\item Hàm số sin $y=\sin x$ có tập xác định là $\mathbb{R}$.
	% 		\item Hàm số cos $y=\cos x$ có tập xác định là $\mathbb{R}$.
	% 		\item Hàm số tan  $y=\tan x$ có tập xác định là $\mathbb{R} \setminus\left\{\dfrac{\pi}{2}+k \pi \Big| k \in \mathbb{Z} \right\}$.
	% 		\item  Hàm số cot $y=\cot x$ có tập xác định là $\mathbb{R} \setminus \left\{k \pi \Big| k \in \mathbb{Z} \right\}$.
	% 	\end{itemize}
	% \end{dn}
	\subsubsection{Hàm số chẵn, hàm số lẻ}
	\begin{dn}
	Cho hàm số $y=f(x)$ có tập xác định là $\mathscr{D}$.
	\begin{itemize}
		\item Hàm số $f(x)$ được gọi là \textbf{hàm số chẵn} nếu $\forall x \in \mathscr{D}$ thì $-x \in \mathscr{D}$ và $f(-x)=f(x)$. Đồ thị của một hàm số chẵn nhận trục tung là trục đối xứng.
		\item Hàm số $f(x)$ được gọi là \textbf{hàm số lẻ} nếu $\forall x \in \mathscr{D}$ thì $-x \in \mathscr{D}$ và $f(-x)=-f(x)$. Đồ thị của một hàm số lẻ nhận gốc toạ độ là tâm đối xứng.
	\end{itemize}
	\end{dn}
	\subsubsection{Hàm số tuần hoàn}
	\begin{dn}
		Hàm số $y=f(x)$ có tập xác định $\mathscr{D}$ được gọi là \textbf{hàm số tuần hoàn} nếu tồn tại số $T \neq 0$ sao cho với mọi $x \in \mathscr{D}$ ta có:
		\begin{enumerate}[i)]
			\item $x+T \in \mathscr{D}$ và $x-T \in \mathscr{D}$;
			\item $f(x+T)=f(x)$.
		\end{enumerate}
		Số $T$ dương nhỏ nhất thỏa mãn các điều kiện trên (nếu có) được gọi là \textbf{chu kì} của hàm số tuần hoàn đó.
	\end{dn}
	\begin{nx}
		\
		\begin{itemize}
			\item  Các hàm số $y=\sin x$ và $y=\cos x$ tuần hoàn với chu kì $2 \pi$. Các hàm số $y=\tan x$ và $y=\cot x$ tuần hoàn với chu kì $\pi$.
		\end{itemize}
	\end{nx}
	\begin{note} 
		Tổng quát, người ta chứng minh được các hàm số $y=A \sin \omega x$ và $y=A \cos \omega x$ $(\omega>0)$ là những hàm số tuần hoàn với chu kì \fbox{$T=\dfrac{2 \pi}{\omega}$}.
	\end{note}
	\subsubsection{Đồ thị và tính chất của hàm số $y=\sin x$}
	\begin{tc}
		Hàm số $y=\sin x$:
		\begin{itemize}
			\item   Có tập xác định là $\mathbb{R}$ và tập giá trị là $[-1 ; 1]$;
			\item   Là hàm số lẻ và tuần hoàn với chu kì $2 \pi$;
			\item    Đồng biến trên mỗi khoảng $\left(-\dfrac{\pi}{2}+k 2 \pi ; \dfrac{\pi}{2}+k 2 \pi\right)$ và nghịch biến trên mỗi khoảng \\
			$\left(\dfrac{\pi}{2}+k 2 \pi ; \dfrac{3 \pi}{2}+k 2 \pi\right)$, $k \in \mathbb{Z}$;
			\item    Có đồ thị đối xứng qua gốc toạ độ và gọi là một \textbf{đường hình sin}.
		\end{itemize}
	\begin{center}
		\begin{tikzpicture}[>=stealth,scale=0.7,transform shape] 
			\path
			({-2.5*pi},0) coordinate (X1)
			({-2*pi},0) coordinate (X2)
			({-1.5*pi},0) coordinate (X3)
			({-pi},0) coordinate (X4)
			({-0.5*pi},0) coordinate (X5)
			(0,0) coordinate (O)
			({0.5*pi},0) coordinate (X6)
			({pi},0) coordinate (X7)
			({1.5*pi},0) coordinate (X8)
			({2*pi},0) coordinate (X9)
			({2.5*pi},0) coordinate (X10)
			({-pi},-2) coordinate (A)
			({pi},-2) coordinate (B)
			;
			\draw[->] (-9.5,0) -- (9.5,0) node[below] {\small $x$};
			\draw[->] (0,-1.5) -- (0,1.8) node[right] {\small $y$};
			\draw [dotted] (X3)--({-1.5*pi},1)--({2.5*pi},1)--({2.5*pi},0)  ({0.5*pi},1)--({0.5*pi},0)
			(X1)--({-2.5*pi},-1)--({1.5*pi},-1)--({1.5*pi},0)  ({-0.5*pi},-1)--({-0.5*pi},0)
			({-pi},0) -- (A) ({pi},0) -- (B);
			\foreach \x/\g/\z in {X1/90/-\tfrac{5\pi}{2},X2/140/-2\pi,X3/-90/-\tfrac{3\pi}{2},X4/-135/-\pi,X5/90/-\tfrac{\pi}{2},X6/-90/\tfrac{\pi}{2},X7/60/\pi,X8/90/\tfrac{3\pi}{2},X9/-40/2\pi,X10/-90/\tfrac{5\pi}{2}} 
			\fill[black] (\x) circle(1pt) +(\g:5mm) node {$\z$};
			\draw [<->] ({-pi},-1.7)--({pi},-1.7) ; 
			\draw (0,0) node[below right]{$O$} (0,-1.7) node[below]{$T=2\pi$}
			(0,1) node[above right]{$1$} (0,-1) node[below right]{$-1$};
			\clip (-9.5,-1.4) rectangle (9.5,1.6) ;
			\draw[thick,samples=100,domain=-9.3:9.3] plot(\x,{sin((\x)*180/pi)});
			
		\end{tikzpicture}
	\end{center}
	\end{tc}
	\subsubsection{Đồ thị và tính chất của hàm số $y=\cos x$}
	\begin{tc}
		Hàm số $y=\cos x$:
		\begin{itemize}
			\item    Có tập xác định là $\mathbb{R}$ và tập giá trị là $[-1 ; 1]$;
			\item    Là hàm số chẵn và tuần hoàn với chu kì $2 \pi$;
			\item    Đồng biến trên mỗi khoảng $(-\pi+k 2 \pi ; k 2 \pi)$ và nghịch biến trên mỗi khoảng $(k 2 \pi ; \pi+k 2 \pi), k \in \mathbb{Z}$;
			\item    Có đồ thị là một đường hình sin đối xứng qua trục tung.
		\end{itemize}
	\begin{center}
		\begin{tikzpicture}[>=stealth,scale=0.77,transform shape] 
			\path
			({-2.5*pi},0) coordinate (X1)
			({-2*pi},0) coordinate (X2)
			({-1.5*pi},0) coordinate (X3)
			({-pi},0) coordinate (X4)
			({-0.5*pi},0) coordinate (X5)
			(0,0) coordinate (O)
			({0.5*pi},0) coordinate (X6)
			({pi},0) coordinate (X7)
			({1.5*pi},0) coordinate (X8)
			({2*pi},0) coordinate (X9)
			({2.5*pi},0) coordinate (X10)
			({-pi},-2) coordinate (A)
			({pi},-2) coordinate (B)
			;
			\draw[->] (-9.5,0) -- (9.5,0) node[below] {\small $x$};
			\draw[->] (0,-1.5) -- (0,1.8) node[right] {\small $y$};
			\draw [dotted] (X2)--({-2*pi},1)--({2*pi},1)--({2*pi},0) (X4)--({-pi},-1)--({pi},-1)--({pi},0) 
			({-pi},0) -- (A) ({pi},0) -- (B);
			\foreach \x/\g/\z in {X1/125/-\tfrac{5\pi}{2},X2/-90/-2\pi,X3/-120/-\tfrac{3\pi}{2},X4/90/-\pi,X5/110/-\tfrac{\pi}{2},X6/-120/\tfrac{\pi}{2},X7/90/\pi,X8/120/\tfrac{3\pi}{2},X9/-90/2\pi,X10/-120/\tfrac{5\pi}{2}} 
			\fill[black] (\x) circle(1pt) +(\g:5mm) node {$\z$};
			\draw [<->] ({-pi},-1.7)--({pi},-1.7) ; 
			\draw (0,0) node[below right]{$O$} (0,-1.7) node[below]{$T=2\pi$}
			(0,1) node[above right]{$1$} (0,-1) node[below right]{$-1$}
			;
			\clip (-9.5,-1.4) rectangle (9.5,1.6) ;
			\draw[thick,samples=100,domain=-9.3:9.3] plot(\x,{cos((\x)*180/pi)});
			
		\end{tikzpicture}
	\end{center}
	\end{tc}
	\subsubsection{Đồ thị và tính chất của hàm số $y=\tan x$}
	\begin{tc}
		Hàm số $y=\tan x$:
		\begin{itemize}
			\item    Có tập xác định là $\mathbb{R} \setminus\left\{\dfrac{\pi}{2}+k \pi \Big| k \in \mathbb{Z} \right\}$ và tập giá trị là $\mathbb{R}$;
			\item    Là hàm số lẻ và tuần hoàn với chu kì $\pi$;
			\item    Đồng biến trên mỗi khoảng $\left(-\dfrac{\pi}{2}+k \pi ; \dfrac{\pi}{2}+k \pi\right)$, $k \in \mathbb{Z}$;
			\item    Có đồ thị đối xứng qua gốc toạ độ.
		\end{itemize}
	\begin{center}
		\begin{tikzpicture}[>=stealth,scale=0.77,transform shape] 
			\path
			({-1.5*pi},0) coordinate (X1)
			({-pi},0) coordinate (X2)
			({-0.5*pi},0) coordinate (X3)
			({0.5*pi},0) coordinate (X4)
			({pi},0) coordinate (X5)
			({1.5*pi},0) coordinate (X6)
			;
			\draw[->] (-6.5,0) -- (6.5,0) node[below] {\small $x$};
			\draw[->] (0,-3.5) -- (0,3.5) node[right] {\small $y$};
			\draw [dashed] ({-3*pi/2},3.5)--({-3*pi/2},-3.5) ({-pi/2},3.5)--({-pi/2},-3.5) ({pi/2},3.5)--({pi/2},-3.5) ({3*pi/2},3.5)--({3*pi/2},-3.5) ;
			\foreach \x/\g/\z in {X1/-150/-\tfrac{3\pi}{2},X2/-40/-\pi,X3/-40/-\tfrac{\pi}{2},X4/-40/\tfrac{\pi}{2},X5/-400/\pi,X6/-40/\tfrac{3\pi}{2}} 
			\fill[black] (\x) circle(1pt) +(\g:5mm) node {$\z$};
			\draw (0,0) node[below right]{$O$};
			\clip (-6.5,-3.5) rectangle (6.5,3.5) ;
			\draw[thick,samples=100,domain={-pi/2+0.2}:{pi/2-0.2}] plot(\x,{tan((\x)*180/pi)});
			\draw[thick,samples=100,domain={pi/2+0.2}:{3*pi/2-0.2}] plot(\x,{tan((\x)*180/pi)});
			\draw[thick,samples=100,domain={-3*pi/2+0.2}:{-pi/2-0.2}] plot(\x,{tan((\x)*180/pi)});
		\end{tikzpicture}
	\end{center}
	\end{tc}
	\subsubsection{Đồ thị và tính chất của hàm số $y=\cot x$}
	\begin{tc}
		Hàm số $y=\cot x$:
		\begin{itemize}
			\item    Có tập xác định là $\mathbb{R} \setminus\{k \pi \mid k \in \mathbb{Z} \}$ và tập giá trị là $\mathbb{R}$; 
			\item    Là hàm số lẻ và tuần hoàn với chu kì $\pi$;
			\item    Nghịch biến trên mỗi khoảng $(k \pi ; \pi+k \pi), k \in \mathbb{Z}$;
			\item    Có đồ thị đối xứng qua gốc toạ độ.
		\end{itemize}
	\begin{center}
		\begin{tikzpicture}[>=stealth,scale=0.77,transform shape] 
			\path
			({-2*pi},0) coordinate (X1)
			({-1.5*pi},0) coordinate (X2)
			({-pi},0) coordinate (X3)
			({-0.5*pi},0) coordinate (X4)
			({0.5*pi},0) coordinate (X5)
			({pi},0) coordinate (X6)
			({1.5*pi},0) coordinate (X7)
			({2*pi},0) coordinate (X8)
			;
			\draw[->] (-7.5,0) -- (7.5,0) node[below] {\small $x$};
			\draw[->] (0,-3.5) -- (0,3.5) node[left] {\small $y$};
			\draw [dashed] ({-2*pi},3.5)--({-2*pi},-3.5) ({-pi},3.5)--({-pi},-3.5) ({pi},3.5)--({pi},-3.5) ({2*pi},3.5)--({2*pi},-3.5) ;
			\foreach \x/\g/\z in {X1/-150/-2\pi,X2/-130/-\tfrac{3\pi}{2},X3/-140/-\pi,X4/-100/-\tfrac{\pi}{2},X5/-100/\tfrac{\pi}{2},X6/-120/\pi,X7/-100/\tfrac{3\pi}{2}, X8/-120/2\pi}
			\fill[black] (\x) circle(1pt) +(\g:5mm) node {$\z$};
			\draw (0,0) node[below left]{$O$};
			\clip (-6.5,-3.5) rectangle (6.5,3.5) ;
			\draw[thick,samples=100,domain={0.2}:{pi-0.2}] plot(\x,{cot((\x)*180/pi)});
			\draw[thick,samples=100,domain={pi+0.2}:{2*pi-0.2}] plot(\x,{cot((\x)*180/pi)});
			\draw[thick,samples=100,domain={-0.2}:{-pi+0.2}] plot(\x,{cot((\x)*180/pi)});
			\draw[thick,samples=100,domain={-pi-0.2}:{-2*pi+0.2}] plot(\x,{cot((\x)*180/pi)});
		\end{tikzpicture}
	\end{center}
	\end{tc}
\end{tomtat}
\subsection{Các dạng toán thường gặp}

% \foreach \i in {1,2,...,5} {\input{data/11KNTT/data1/1K1-3-\i.tex}}
%%Bài 4. PTLG
% 
\setcounter{section}{3}
\section{Phương trình lượng giác cơ bản}
\subsection{Tóm tắt lý thuyết}
\begin{tomtat}
\subsubsection{Phương trình $\sin x=m$}
\begin{itemize}
	\item Với $|m|>1$ thì phương trình $\sin x=m$ vô nghiệm.
	\item Với $|m|\leq 1$, sẽ tồn tại duy nhất $\alpha \in \left[-\dfrac{\pi}{2}; \dfrac{\pi}{2}\right]$ thỏa mãn $\sin\alpha=m$. Khi đó
	\begin{center}
		$\sin x=m\Leftrightarrow\sin x=\sin\alpha\Leftrightarrow\hoac{&x=\alpha+k2\pi\\&x=\pi-\alpha+k2\pi}$ ($k\in \mathbb{Z}$).
	\end{center}
\item Nếu số đo của góc $\alpha$ được đo bằng đơn vị độ thì
\begin{center}
	$\sin x=\sin\alpha^\circ\Leftrightarrow\hoac{&x=\alpha^\circ+k360^\circ\\&x=180^\circ-\alpha^\circ+k360^\circ}$ ($k\in\mathbb{Z}$).
\end{center}
\item Tổng quát,
\begin{center}
	$\sin f(x)=\sin g(x)\Leftrightarrow\hoac{&f(x)=g(x)+k2\pi\\&f(x)=\pi - g(x)+k2\pi}$ ($k\in\mathbb{Z}$).
\end{center}
\item Một số trường hợp đặt biệt:
\begin{enumEX}[\faCheckCircleO]{1}
	\item $\sin x=0\Leftrightarrow x=k\pi$, $k\in\mathbb{Z}$.
	\item $\sin x=1\Leftrightarrow x=\dfrac{\pi}{2}+k2\pi$, $k\in\mathbb{Z}$.
	\item $\sin x=-1\Leftrightarrow x=-\dfrac{\pi}{2}+k2\pi$, $k\in \mathbb{Z}$.
\end{enumEX}
\end{itemize}
	\subsubsection{Phương trình $\cos x=m$}
\begin{itemize}
	\item Với $|m|>1$ thì phương trình $\cos x=m$ vô nghiệm.
	\item Với $|m|\leq 1$, sẽ tồn tại duy nhất $\alpha\in\left[0; \pi\right]$ thỏa mãn $\cos x=m$. Khi đó
	\begin{center}
		$\cos x=m\Leftrightarrow\cos x=\cos \alpha\Leftrightarrow\hoac{&x=\alpha+k2\pi\\&x=-\alpha+k2\pi}$ ($k\in\mathbb{Z}$).
	\end{center}
\item Nếu số đo của góc $\alpha$ được đo bằng đơn vị độ thì
\begin{center}
	$\cos x=\cos\alpha\Leftrightarrow\hoac{&x=\alpha^\circ+k360^\circ\\&x=-\alpha^\circ+k360^\circ}$ ($k\in\mathbb{Z}$).
\end{center}
\item Tổng quát,
\begin{center}
	$\cos f(x)=\cos g(x)\Leftrightarrow\hoac{&f(x)=g(x)+k2\pi\\&f(x)=-g(x)+k2\pi}$ ($k\in \mathbb{Z}$)
\end{center}
\item Một số trường hợp đặc biệt:
\begin{enumEX}[\faCheckCircleO]{1}
\item $\cos x=0\Leftrightarrow x=\dfrac{\pi}{2}+k\pi$, $k\in\mathbb{Z}$.
\item $\cos x=1\Leftrightarrow x=k2\pi$, $k\in\mathbb{Z}$.
\item $\cos x=-1\Leftrightarrow x=\pi+k2\pi$, $k\in\mathbb{Z}$.
\end{enumEX}
\end{itemize}
\subsubsection{Phương trình $\tan x=m$}
\begin{itemize}
	\item Với mọi $m\in\mathbb{R}$, tồn tại duy nhất $\alpha\in\left(-\dfrac{\pi}{2};\dfrac{\pi}{2}\right)$ thỏa mãn $\tan \alpha=m$. Khi đó
	\begin{center}
		$\tan x=m\Leftrightarrow\tan x=\tan \alpha\Leftrightarrow x=\alpha+k\pi$ ($k\in\mathbb{Z}$).
	\end{center}
\item Nếu số đo của góc $\alpha$ được đo bằng đơn vị độ thì
\begin{center}
	$\tan x=\tan\alpha^\circ\Leftrightarrow x=\alpha^\circ+k180^\circ$, $k\in \mathbb{Z}$
\end{center}
\item Tổng quát,
\begin{center}
	$\tan f(x)=\tan g(x)\Leftrightarrow f(x)=g(x)+k\pi$, $k\in\mathbb{Z}$.
\end{center}
\end{itemize}
\subsubsection{Phương trình $\cot x=m$}
\begin{itemize}
	\item Với mọi $m\in\mathbb{R}$, tồn tại duy nhất $\alpha\in\left(0;\pi\right)$ thỏa mãn $\cot\alpha=m$. Khi đó
	\begin{center}
		$\cot x=m\Leftrightarrow\cot x=\cot\alpha\Leftrightarrow x=\alpha+k\pi$ $k\in\mathbb{Z}$.
	\end{center}
\item Nếu số đo của góc $\alpha$ được đo bằng đơn vị độ thì
\begin{center}
	$\cot x=\cot\alpha^\circ\Leftrightarrow x=\alpha^\circ+k180^\circ$, $k\in\mathbb{Z}$.
\end{center}
\item Tổng quát,
\begin{center}
	$\cot f(x)=\cot g(x)\Leftrightarrow f(x)=g(x)+k\pi$, $k\in\mathbb{Z}$.
\end{center}
\end{itemize}
\end{tomtat}


% \foreach \i in {2,3,4,5,6,7,11} {\input{data/11KNTT/data1/1K1-4-\i.tex}}
%%Ôn tập chương I
% \newpage
% \section*{LÝ THUYẾT GÓC LƯỢNG GIÁC - GIÁ TRỊ - HÀM SỐ LƯỢNG GIÁC}
\boldmath
\subsection{GTLG Góc lượng giác}
\begin{itemize}
	\item \textbf{Đổi đơn vị đo}: \fbox{1 vòng = $360^\circ = 2\pi\ rad$}, \fbox{$180^\circ = \pi rad$}
	      \begin{center}
		      \renewcommand{\arraystretch}{2}
		      \begin{tabular}{|l|c|c|c|c|c|c|c|c|c|}
			      \hline Độ     & $0^{\circ}$ & $30^{\circ}$     & $45^{\circ}$     & $60^{\circ}$     & $90^{\circ}$     & $120^{\circ}$      & $135^{\circ}$      & $150^{\circ}$      & $180^{\circ}$ \\
			      \hline Rađian & 0           & $\dfrac{\pi}{6}$ & $\dfrac{\pi}{4}$ & $\dfrac{\pi}{3}$ & $\dfrac{\pi}{2}$ & $\dfrac{2 \pi}{3}$ & $\dfrac{3 \pi}{4}$ & $\dfrac{5 \pi}{6}$ & $\pi$         \\
			      \hline
		      \end{tabular}
	      \end{center}
	      \immini{
	\item \textbf{Độ dài cung tròn} bán kính $R$ số đo $\alpha$ rad là \fbox{$l=R\alpha$}.
	\item \textbf{Điểm biểu diễn góc lượng giác} $\alpha$ lên đường tròn lượng giác là $M$. Khi đó $M$ cũng biểu diễn các góc lượng giác $\alpha+k2\pi$.\\
	      Góc $\alpha$ và $\beta$ có chung điểm biểu diễn khi \fbox{$\alpha - \beta = k2\pi$} (chẵn lần $\pi$)}
	      {
	      \begin{tikzpicture}[line join = round, line cap = round, >=stealth, font=\footnotesize, scale=0.4]
		      \tikzset{label style/.style={font=\footnotesize}}
		      \path (0,0) coordinate (O)
		      (3,0) coordinate (A)
		      (0,3) coordinate (B)
		      (0,-3) coordinate (B')
		      (-3,0) coordinate (A')
		      (0:0)++(150:3) coordinate (M)
		      ($(O)!(M)!(A')$) coordinate (H)
		      ($(O)!(M)!(B)$) coordinate (K)
		      ;
		      \draw[->] (-4,0) -- (4,0) node[above,blue]{$x$};
		      \draw[->] (0,-4) -- (0.,4) node[left,blue]{$y$};
		      \draw[orange] (O) circle (3cm);
		      \draw[rotate=0,->,green!50!black] (0.5,0) arc (0:150:0.5cm);
		      \draw (0.35,0.25) node[above,blue] {$\alpha$};
		      \draw[dashed] (H)--(M)--(K);
		      \draw[green!50!black] (M)--(O);
		      \foreach \p/\r in {A/-45,M/150,O/-150,A'/-135,B'/-45,B/45}
		      \fill (\p) circle (1pt) node[shift={(\r:3mm)},blue]{$\p$};
	      \end{tikzpicture}
	      }
\end{itemize}
\begin{multicols}{2}
	\begin{khung4}{Định nghĩa GTLG}
		\immini{\begin{itemize}
				\item $\cos \alpha =x$
				\item $\sin \alpha = y$
				\item $\tan \alpha =\dfrac{\sin\alpha}{\cos\alpha}=\dfrac{y}{x}$
				\item $\cot\alpha=\dfrac{\cos\alpha}{\sin\alpha}=\dfrac{x}{y}$
			\end{itemize}}
		{\begin{tikzpicture}[line join = round, line cap = round, >=stealth, font=\footnotesize, scale=0.4]
				\tikzset{label style/.style={font=\footnotesize}}
				\path (0,0) coordinate (O)
				(3,0) coordinate (A)
				(0,3) coordinate (B)
				(0,-3) coordinate (B')
				(-3,0) coordinate (A')
				(0:0)++(40:3) coordinate (M)
				($(O)!(M)!(A')$) coordinate (H)
				($(O)!(M)!(B)$) coordinate (K)
				;
				\draw[->] (-4,0) -- (4,0) node[above,blue]{$x$};
				\draw[->] (0,-4) -- (0.,4) node[left,blue]{$y$};
				\draw[orange] (O) circle (3cm);
				\draw[rotate=0,->,green!50!black] (0.7,0) arc (0:40:0.7cm);
				\draw (1,0) node[above,blue] {$\alpha$};
				\draw[dashed] (H)--(M)--(K);
				\draw[green!50!black] (M)--(O);
				\draw[blue,fill=black] (0,2) node[left]{$\sin\alpha$}(2,0) circle(1pt) node[below]{$\cos\alpha$}(3,2.3) node{$M(x;y)$};
				\foreach \p/\r in {A/-45,O/-135,A'/-135,B'/-45,B/45}
				\fill (\p) circle (1pt) node[shift={(\r:3mm)},blue]{$\p$};
			\end{tikzpicture}}
	\end{khung4}
	\begin{khung4}{Các công thức lượng giác cơ bản}
		\begin{itemize}
			\item $\sin^2 \alpha  + \cos^2 \alpha =1$
			\item $ 1+ \tan^2 \alpha= \dfrac{1}{\cos^2 \alpha}$ $\left(\alpha \neq \dfrac{\pi}{2}+k\pi , k\in \mathbb{Z}\right)$
			\item $ 1+ \cot^2 \alpha= \dfrac{1}{\sin^2 \alpha}$ $\left(\alpha \neq k\pi , k\in \mathbb{Z}\right)$
			\item $\tan \alpha \cdot \cot \alpha =1 $ $\left(\alpha \neq \dfrac{k\pi}{2}, k\in \mathbb{Z}\right)$
		\end{itemize}
	\end{khung4}
\end{multicols}
\textit{\underline{Chú ý}: $\tan\alpha$ xác định khi $\alpha\neq\dfrac{\pi}{2}+k\pi\,\,  (k\in\mathbb{Z})$ và $\cot\alpha$ xác định khi $\alpha\neq k\pi\,\,  (k\in\mathbb{Z})$.}

\begin{multicols}{2}
	\begin{khung4}{$\cos$ đối}
		\immini{
			\begin{itemize}
				\item $\cos (-\alpha)=\cos \alpha$
				\item $\sin (-\alpha) =-\sin \alpha$
				\item $\tan (-\alpha) =-\tan \alpha$
				\item $\cot (-\alpha) =-\cot \alpha$
			\end{itemize}
		}{\begin{tikzpicture}[line join = round, line cap = round, >=stealth, font=\footnotesize, scale=0.4]
				\tikzset{label style/.style={font=\footnotesize}}
				\path (0,0) coordinate (O)
				(3,0) coordinate (A)
				(0:0)++(120:3) coordinate (M)
				(0:0)++(-120:3) coordinate (N)
				(0,4) coordinate (C)
				(0,-4) coordinate (D)
				($(O)!(M)!(C)$) coordinate (E)
				($(O)!(N)!(D)$) coordinate (F)
				;
				\draw[->] (-4,0) -- (4,0) node[above,blue]{$x$};
				\draw[->] (0,-4) -- (0.,4) node[left,blue]{$y$};
				\draw[orange] (O) circle (3cm);
				\draw[rotate=0,->,red] (0.5,0) arc (0:120:0.5cm);
				\draw[rotate=0,->,green!50!black] (0.6,0) arc (0:-120:0.6cm);
				\draw (0,0) node[above right=2pt,blue] {$\alpha$} (0,-0) node[below right=2pt,blue]{$-\alpha$};
				\draw[dashed] (E)--(M)--(N)--(F);
				\draw[green!50!black] (M)--(O);
				\draw[red] (N)--(O);
				\foreach \p/\r in {A/-45,M/120,N/-120,O/-150}
				\fill (\p) circle (1pt) node[shift={(\r:3mm)},blue]{$\p$};
			\end{tikzpicture}}
	\end{khung4}
	\begin{khung4}{phụ chéo}
		\immini{\begin{itemize}
				\item $\sin \left( \dfrac{\pi}{2}-\alpha\right)=\cos \alpha$
				\item $\cos \left( \dfrac{\pi}{2}-\alpha\right)=\sin \alpha$
				\item $\tan \left( \dfrac{\pi}{2}-\alpha\right)=\cot \alpha$
				\item $\cot \left( \dfrac{\pi}{2}-\alpha\right)=\tan \alpha$
			\end{itemize}}
		{\begin{tikzpicture}[line join = round, line cap = round, >=stealth, font=\footnotesize, scale=0.4]
				\tikzset{label style/.style={font=\footnotesize}}
				\path (0,0) coordinate (O)
				(3,0) coordinate (A)
				(0:0)++(20:3) coordinate (M)
				(0:0)++(70:3) coordinate (N)
				(0,4) coordinate (C)
				(4,0) coordinate (D)
				($(O)!(M)!(C)$) coordinate (E)
				($(O)!(N)!(D)$) coordinate (F)
				(2.82,0) coordinate (G)
				(0,2.82) coordinate (H)
				;
				\draw[->] (-4,0) -- (4,0) node[above,blue]{$x$};
				\draw[->] (0,-4) -- (0.,4) node[left,blue]{$y$};
				\draw[orange] (O) circle (3cm);
				\draw[rotate=0,->,red] (0.7,0) arc (0:70:0.7cm);
				\draw[rotate=0,->,green!50!black] (1.6,0) arc (0:20:1.6cm);
				\draw (2,0) node[above,blue] {$\alpha$} (-2,3.5) node[blue]{$\frac{\pi}{2}-\alpha$};
				\draw[dashed] (E)--(M) (F)--(N) (G)--(M) (H)--(N);
				\draw[dashed] (-3,-3)--(3,3);
				\draw[->] (-2,3)--(0.5,2);
				\draw[red] (O)--(N);
				\draw[green!50!black] (O)--(M);
				\foreach \p/\r in {A/-45,M/20,N/70,O/-220}
				\fill (\p) circle (1pt) node[shift={(\r:3mm)},blue]{$\p$};
			\end{tikzpicture}}
	\end{khung4}
	\begin{khung4}{$\sin$ bù}
		\immini{\begin{itemize}
				\item $\sin (\pi -\alpha)=\sin \alpha$
				\item $\cos (\pi -\alpha) =-\cos \alpha$
				\item $\tan (\pi -\alpha) =-\tan \alpha$
				\item $\cot (\pi -\alpha) =-\cot \alpha$
			\end{itemize}}
		{\begin{tikzpicture}[line join = round, line cap = round, >=stealth, font=\footnotesize, scale=0.4]
				\tikzset{label style/.style={font=\footnotesize}}
				\path (0,0) coordinate (O)
				(3,0) coordinate (A)
				(0:0)++(30:3) coordinate (M)
				(0:0)++(150:3) coordinate (N)
				(4,0) coordinate (C)
				(-4,0) coordinate (D)
				($(O)!(M)!(C)$) coordinate (E)
				($(O)!(N)!(D)$) coordinate (F)
				;
				\draw[->] (-4,0) -- (4,0) node[above,blue]{$x$};
				\draw[->] (0,-4) -- (0.,4) node[left,blue]{$y$};
				\draw[orange] (O) circle (3cm);
				\draw[rotate=0,->,red] (0.5,0) arc (0:150:0.5cm);
				\draw[rotate=0,->,green!50!black] (1.6,0) arc (0:30:1.6cm);
				\draw (2,0) node[above,blue] {$\alpha$} (0.3,1.5) node[below,blue]{$\pi-\alpha$};
				\draw[dashed] (E)--(M)--(N)--(F);
				\draw[red] (O)--(N);
				\draw[green!50!black] (O)--(M);
				\foreach \p/\r in {A/-45,M/30,N/150,O/-130}
				\fill (\p) circle (1pt) node[shift={(\r:3mm)},blue]{$\p$};
			\end{tikzpicture}}
	\end{khung4}
	\begin{khung4}{$\pm \pi\ \tan, \cot$}
		\immini{\begin{itemize}
				\item $\sin (\pi +\alpha)=-\sin \alpha$
				\item $\cos (\pi +\alpha)=-\cos \alpha$
				\item $\tan (\pi +\alpha)=\tan \alpha$
				\item $\cot (\pi +\alpha)=\cot \alpha$
			\end{itemize}}
		{\begin{tikzpicture}[line join = round, line cap = round, >=stealth, font=\footnotesize, scale=0.4]
				\tikzset{label style/.style={font=\footnotesize}}
				\path (0,0) coordinate (O)
				(3,0) coordinate (A)
				(0:0)++(60:3) coordinate (M)
				(0:0)++(240:3) coordinate (N)
				;
				\draw[->] (-4,0) -- (4,0) node[above,blue]{$x$};
				\draw[->] (0,-4) -- (0.,4) node[left,blue]{$y$};
				\draw[orange] (O) circle (3cm);
				\draw[rotate=0,->,red] (1.7,0) arc (0:240:1.7cm);
				\draw[rotate=0,->,green!50!black] (0.6,0) arc (0:60:0.6cm);
				\draw (1,0) node[above,blue] {$\alpha$};
				\draw (-1.2,1) node[below,blue,rotate=60]{$\pi+\alpha$};
				\draw[red] (O)--(N);
				\draw[green!50!black] (O)--(M);
				\foreach \p/\r in {A/-45,M/60,N/240,O/-150}
				\fill (\p) circle (1pt) node[shift={(\r:3mm)},blue]{$\p$};
			\end{tikzpicture}}
	\end{khung4}
\end{multicols}
\newpage

\subsection{Công thức lượng giác}
\subsubsection{Công thức cộng}
\begin{khung4}{Công thức cộng}
	\begin{multicols}{2}
	\begin{itemize}
		\item $\cos (a-b) = \cos a \cos b + \sin a\sin b$.
		\item $\cos (a+b) = \cos a \cos b - \sin a\sin b$.
		\item $\sin (a-b) = \sin a \cos b - \sin b \cos a$.
		\item $\sin (a+b) = \sin a \cos b + \sin b \cos a$.
		\item $\tan (a-b) = \dfrac{\tan a - \tan b}{1+\tan a \tan b}$.
		\item $\tan (a+b) = \dfrac{\tan a + \tan b}{1-\tan a \tan b}$.
	\end{itemize}
\end{multicols}
\end{khung4}
\begin{khung4}{Trường hợp đặc biệt}
	\begin{itemize}
		\item $\sin x + \cos x = \sqrt{2}\sin\left(x+\dfrac{\pi}{4}\right) = \sqrt{2}\cos \left(x-\dfrac{\pi}{4}\right)$.
		\item $\sqrt{3}\sin x + \cos x = 2\sin \left(x+\dfrac{\pi}{6}\right) = 2\cos \left(x-\dfrac{\pi}{3}\right)$.
		\item $\sin x + \sqrt{3}\cos x = 2\sin \left(x+ \dfrac{\pi}{3}\right) = 2\cos \left(x-\dfrac{\pi}{6}\right)$.
	\end{itemize}
\end{khung4}
\subsubsection{Công thức nhân đôi}
\begin{multicols}{2}
	\begin{khung4}{Công thức nhân đôi}
		\begin{itemize}
			\item $\sin 2a = 2\sin a \cos a$.
			\item $\cos 2a = \cos^2a-\sin^2a = 2\cos^2a-1 = 1-2\sin^2a$.
			\item $\tan 2a = \dfrac{2\tan a}{1-\tan^2a}$.
		\end{itemize}
	\end{khung4}
	\begin{khung4}{Công thức hạ bậc}
		\begin{itemize}
			\item $\sin^2a= \dfrac{1-\cos 2a}{2}$.
			\item $\cos^2a = \dfrac{1+\cos 2a}{2}$.
			\item $\tan^2a=\dfrac{1-\cos2a}{1+\cos 2a}$.
		\end{itemize}
	\end{khung4}
\end{multicols}

\begin{note}
	Áp dụng công thức cộng cho $3a = a +2a$, ta có công thức nhân ba:
\end{note}
\begin{khung4}{Công thức nhân ba}
	\begin{multicols}{2}
	\begin{itemize}
		\item $\sin3a= 3\sin a -4\sin^3a$.		
		\item $\cos3a= 4\cos^3a-3\cos a$.
		\item $\tan3a = \dfrac{3\tan a - \tan^3 a}{1-3\tan^2a}$.
	\end{itemize}
\end{multicols}
\end{khung4}

\subsubsection{Công thức biến đổi tích thành tổng}
\begin{khung4}{Công thức tích thành tổng}
	\begin{multicols}{2}
	\begin{itemize}
		\item $\cos a \cos b = \dfrac{1}{2}\left[\cos (a-b) + \cos (a+b)\right]$.
		\item $\sin a \sin b = \dfrac{1}{2}\left[\cos (a-b)-\cos(a+b)\right]$.
		\item $\sin a \cos b = \dfrac{1}{2}\left[\sin (a-b)+\sin (a+b)\right]$.
	\end{itemize}
	\end{multicols}
\end{khung4}
\subsubsection{Công thức biến đổi tổng thành tích}
Công thức biến đổi tổng thành tích được xây dựng bằng cách $a=\dfrac{a+b}{2}, b = \dfrac{a-b}{2}$ trong công thức biến đổi tích thành tổng.
\begin{khung4}{Công thức tổng thành tích}
	\begin{multicols}{2}
	\begin{itemize}
		\item $\cos a+ \cos b = 2\cos\dfrac{a+b}{2}\cos \dfrac{a-b}{2}$.
		\item $\cos a- \cos b = -2\sin\dfrac{a+b}{2}\sin \dfrac{a-b}{2}$.
		\item $\sin a+ \sin b = 2\sin\dfrac{a+b}{2}\cos \dfrac{a-b}{2}$.
		\item $\sin a -\sin b = 2\cos\dfrac{a+b}{2}\sin \dfrac{a-b}{2}$.
	\end{itemize}
\end{multicols}
\end{khung4}
\newpage

\subsection{Hàm số lượng giác}
\begin{khung4}{Hàm số chẵn, hàm số lẻ}
	\begin{multicols}{2}
		\begin{itemize}
			\item Hàm số $f(x)$ được gọi là \textbf{hàm số chẵn} nếu $\forall x \in \mathscr{D}$ thì $-x \in \mathscr{D}$ và $f(-x)=f(x)$. Đồ thị của một \textbf{hàm số chẵn} nhận \textbf{trục tung} là trục đối xứng.
			\item Hàm số $f(x)$ được gọi là \textbf{hàm số lẻ} nếu $\forall x \in \mathscr{D}$ thì $-x \in \mathscr{D}$ và $f(-x)=-f(x)$. Đồ thị của một \textbf{hàm số lẻ} nhận \textbf{gốc toạ độ} là tâm đối xứng.
		\end{itemize}
	\end{multicols}
	Các hàm số $y=\sin x$, $y=\tan x$, $y=\cot x$ là hàm số \textit{lẻ}, hàm số $y=\cos x$ là hàm số \textit{chẵn}.
\end{khung4}
\begin{khung4}{Hàm số tuần hoàn}
	\begin{dn}
		Hàm số $y=f(x)$ có tập xác định $\mathscr{D}$ được gọi là \textbf{hàm số tuần hoàn} nếu tồn tại số $T \neq 0$ sao cho với mọi $x \in \mathscr{D}$ ta có:
		\begin{itemize}
			\item $x+T \in \mathscr{D}$ và $x-T \in \mathscr{D}$;
			\item $f(x+T)=f(x)$.
		\end{itemize}
		Số $T$ dương nhỏ nhất thỏa mãn các điều kiện trên (nếu có) được gọi là \textbf{chu kì} của hàm số tuần hoàn đó.
	\end{dn}
	Các hàm số $y=A \sin \omega x$ và $y=A \cos \omega x$ $(\omega>0)$ là những hàm số tuần hoàn với chu kì $T=\dfrac{2 \pi}{\omega}$.\\
	Các hàm số $y=A \tan \omega x$ và $y=A \cot \omega x$ $(\omega>0)$ là những hàm số tuần hoàn với chu kì $T=\dfrac{\pi}{\omega}$.
\end{khung4}
\begin{center}
	\begin{tikzpicture}[line join = round, line cap = round, >=stealth, font=\small, thick, scale=.7]
		\path (0,0) coordinate (O)
		(3,0) coordinate (A)
		(0,3) coordinate (B)
		(0,-3) coordinate (B')
		(-3,0) coordinate (A');
		\draw[->] (-4,0) -- (4,0) node[above,blue]{$\cos$};
		\draw[->] (0,-4) -- (0.,3.5) node[left,blue]{$\sin$};
		\draw[orange] (O) circle (3cm);
		\draw[->,green!50!black] (40:3.5) arc (40:50:3.5cm) node[midway,above right] {(+)};
		\draw[->,green!50!black](5,0)node[align=center,rotate=90,below]{$\sin x$ đồng biến\\ trên $\left(-\dfrac{\pi}{2}+k2\pi;\dfrac{\pi}{2}+k2\pi\right)$}
		(-30:5) arc (-30:30:5cm) ;
		\draw[->,red!50!black](0,5)node[align=center,above]{$\cos x$ nghịch biến\\ trên $\left(k2\pi;\pi+k2\pi\right)$}
		(60:5) arc (60:120:5cm) ;
		\draw[->,red!50!black](-5,0)node[align=center,rotate=-90,below]{$\sin x$ nghịch biến\\ trên $\left(\dfrac{\pi}{2}+k2\pi;\dfrac{3\pi}{2}+k2\pi\right)$}
		(150:5) arc (150:210:5cm) ;

		\draw[->,green!50!black](0,-5)node[align=center,below]{$\cos x$ đồng biến\\ trên $\left(-\pi+k2\pi;k2\pi\right)$}
		(-120:5) arc (-120:-60:5cm) ;

		\foreach \p/\r in {A/-45,O/-150,A'/-135,B'/-45,B/45}
		\fill (\p) circle (1pt) node[shift={(\r:3mm)},blue]{$\p$};
	\end{tikzpicture}
\end{center}

\subsection{Phương trình lượng giác}
\begin{khung4}{Phương trình $\sin x = a$.}
	\begin{enumerate}[\faCheckSquareO]
		\item Trường hợp $a>1$ hoặc $a<-1$ phương trình vô nghiệm.
		\item Trường hợp $a \in \{-1;0;1\}$.
		      \begin{multicols}{3}
			      \begin{tikzpicture}[smooth,samples=300,scale=0.8,>=stealth]
				      \draw[->] (-1.5,0)--(1.5,0) node[below]{\footnotesize$cos$};
				      \draw[->] (0,-1.5)--(0,1.8) node[right]{\footnotesize$sin$};
				      \draw (0,0) node[below left]{\footnotesize$O$};
				      \tkzDefPoints{0/0/I}
				      \draw[orange] (I) circle(1cm);
				      \draw[fill,blue] (0,1) circle(2.5pt) node[above left] {$B$};
				      \node[below] at (0,-1.5) {\fbox{$\sin x=1 \Leftrightarrow x=\tfrac{\pi}{2}+k2\pi$}};
			      \end{tikzpicture}
			      \hspace{-0.4cm}
			      \begin{tikzpicture}[smooth,samples=300,scale=0.8,>=stealth]
				      \draw[->] (-1.5,0)--(1.5,0) node[below]{\footnotesize$cos$};
				      \draw[->] (0,-1.5)--(0,1.8) node[right]{\footnotesize$sin$};
				      \draw (0,0) node[below left]{\footnotesize$O$};
				      \tkzDefPoints{0/0/I}
				      \draw[orange] (I) circle(1cm);
				      \draw[fill,blue] (0,-1) circle(2.5pt)node[below left] {$B'$};
				      \node[below] at (0,-1.5) {\fbox{$\sin x=-1 \Leftrightarrow x=-\tfrac{\pi}{2}+k2\pi$}};
			      \end{tikzpicture}
			      \hspace{-0.4cm}
			      \begin{tikzpicture}[smooth,samples=300,scale=0.8,>=stealth]
				      \draw[->] (-1.5,0)--(1.5,0) node[below]{\footnotesize$cos$};
				      \draw[->] (0,-1.5)--(0,1.8) node[right]{\footnotesize$sin$};
				      \draw (0,0) node[below left]{\footnotesize$O$};
				      \tkzDefPoints{0/0/I}
				      \draw[orange] (I) circle(1cm);
				      \draw[fill,blue] (1,0) circle(2.5pt) (-1,0) circle(2.5pt) node[above right] at (1,0) {$A$};
				      \node[above left,blue] at (-1,0) {$A'$};
				      \node[below] at (0,-1.5) {\fbox{$\sin x=0 \Leftrightarrow x=k\pi$}};
			      \end{tikzpicture}
		      \end{multicols}
		\item Trường hợp $a \in \left\{\pm \dfrac{1}{2};\pm \dfrac{\sqrt{2}}{2};\pm \dfrac{\sqrt{3}}{2}\right\}$ hoặc $a \in (-1;1)$. Ta bấm máy \shiftk \sink để đổi tìm góc $\alpha$ hoặc $\beta^\circ$.
			      \immini{
				      \begin{listEX}[1]
					      \item [\ding{172}] Công thức theo đơn vị rad:
					      $\sin x = \sin \alpha \Leftrightarrow \hoac{&x=\alpha+k2\pi\\&x=\pi-\alpha+k2\pi}, \,k \in \mathbb{Z}$
					      \item [\ding{173}] Công thức theo đơn vị độ: $\sin x = \sin \beta^\circ \Leftrightarrow \hoac{&x=\beta^\circ+k360^\circ\\&x=180^\circ-\beta^\circ+k360^\circ}, \,k \in \mathbb{Z}$
				      \end{listEX}
			      }{\begin{tikzpicture}[smooth,samples=300,scale=1,>=stealth]
					      \draw[->] (-1.5,0)--(1.5,0);
					      \draw[->] (0,-1.3)--(0,1.5) node[right]{\footnotesize$sin$};
					      \draw (0,0) node[below left]{\footnotesize$O$};
					      \tkzDefPoints{0/0/I}
					      \draw[orange] (I) circle(1cm);
					      \coordinate (M) at ($(I)+(50:1cm)$);
					      \coordinate (N) at ($(I)+(130:1cm)$);
					      \tkzDrawPoints[size=3,fill=blue](M,N)
					      \tkzDrawSegments(I,M I,N)
					      \tkzDrawSegments[dashed](M,N)
					      \tkzLabelPoints[right,blue](M)
					      \tkzLabelPoints[left,blue](N)
					      \node[below right] at (0,0.7) {$a$};
				      \end{tikzpicture}
			      }
	\end{enumerate}
\end{khung4}
\begin{khung4}{Phương trình $\cos x = a$.}
\begin{enumerate}[\faCheckSquareO]
	\item Trường hợp $a>1$ hoặc $a<-1$ phương trình vô nghiệm.
	\item Trường hợp $a \in \{-1;0;1\}$.
	      \begin{multicols}{3}
		      \begin{tikzpicture}[smooth,samples=300,scale=0.8,>=stealth]
			      \draw[->] (-1.5,0)--(1.8,0) node[below]{\footnotesize$cos$};
			      \draw[->] (0,-1.5)--(0,1.8) node[right]{\footnotesize$sin$};
			      \draw (0,0) node[below left]{\footnotesize$O$};
			      \tkzDefPoints{0/0/I}
			      \draw[orange] (I) circle(1cm);
			      \draw[fill,blue] (1,0) circle(2.5pt)node[above right]  {$A$};
			      \node[below] at (0,-1.5) {\fbox{$\cos x=1 \Leftrightarrow x=k2\pi$}};
		      \end{tikzpicture}
		      \begin{tikzpicture}[smooth,samples=300,scale=0.8,>=stealth]
			      \draw[->] (-1.5,0)--(1.5,0) node[below]{\footnotesize$cos$};
			      \draw[->] (0,-1.8)--(0,1.5) node[right]{\footnotesize$sin$};
			      \draw (0,0) node[below left]{\footnotesize$O$};
			      \tkzDefPoints{0/0/I}
			      \draw[orange] (I) circle(1cm);
			      \draw[fill,blue] (-1,0) circle(2.5pt)node[below left] {$A'$};
			      \node[below] at (0,-1.5) {\fbox{$\cos x=-1 \Leftrightarrow x=\pi+k2\pi$}};
		      \end{tikzpicture}
		      \begin{tikzpicture}[smooth,samples=300,scale=0.8,>=stealth]
			      \draw[->] (-1.5,0)--(1.8,0) node[below]{\footnotesize$cos$};
			      \draw[->] (0,-1.5)--(0,1.5);
			      \draw (0,0) node[below left]{\footnotesize$O$};
			      \tkzDefPoints{0/0/I}
			      \draw[orange] (I) circle(1cm);
			      \draw[fill,blue] (0,1) circle(2.5pt) (0,-1) circle(2.5pt) node[above right] at (0,1) {$B$};
			      \node[below left,blue] at (0,-1) {$B'$};
			      \node[below] at (0,-1.5) {\fbox{$\cos x=0 \Leftrightarrow x=\frac{\pi}{2}+k\pi$}};
		      \end{tikzpicture}
	      \end{multicols}
	\item Trường hợp $a \in \left\{\pm \dfrac{1}{2};\pm \dfrac{\sqrt{2}}{2};\pm \dfrac{\sqrt{3}}{2}\right\}$ hoặc $a \in (-1;1)$. Ta bấm máy \shiftk \cosk để tìm góc $\alpha$ hoặc $\beta^\circ$ tương ứng.
	      \immini{
		      \begin{listEX}[1]
			      \item [\ding{172}] Công thức theo đơn vị rad:
			      $\cos x = \cos \alpha \Leftrightarrow \hoac{&x=\alpha+k2\pi\\&x=-\alpha+k2\pi}, \,k \in \mathbb{Z}$
			      \item [\ding{173}] Công thức theo đơn vị độ: $\cos x = \cos \beta^\circ \Leftrightarrow \hoac{&x=\beta^\circ+k360^\circ\\&x=-\beta^\circ+k360^\circ}, \,k \in \mathbb{Z}$
		      \end{listEX}
	      }{\begin{tikzpicture}[smooth,samples=300,scale=1,>=stealth]
			      \draw[->] (-1.5,0)--(1.7,0)node[above]{\footnotesize$cos$};
			      \draw[->] (0,-1.3)--(0,1.5);
			      \draw (0,0) node[below left]{\footnotesize$O$};
			      \tkzDefPoints{0/0/I}
			      \draw[orange] (I) circle(1cm);
			      \coordinate (M) at ($(I)+(50:1cm)$);
			      \coordinate (N) at ($(I)+(-50:1cm)$);
			      \tkzDrawPoints[size=3,fill=blue](M,N)
			      \tkzDrawSegments(I,M I,N)
			      \tkzDrawSegments[dashed](M,N)
			      \tkzLabelPoints[above,blue](M)
			      \tkzLabelPoints[below,blue](N)
			      \node[below right] at (0.7,0) {$a$};
		      \end{tikzpicture}}
\end{enumerate}
\end{khung4}
\begin{khung4}{Phương trình $\tan x = a$ và $\cot x = b$.}
\begin{enumerate}[\faCheckSquareO]
	\item Trường hợp $a \in \left\{0;\pm \dfrac{\sqrt{3}}{3};\pm 1; \pm \sqrt{3}\right\}$ hoặc $a$ bất kì. Ta bấm máy \shiftk \tank để tìm góc $\alpha$ hoặc $\beta^\circ$ tương ứng.
		      \immini{
			      \begin{listEX}[1]
				      \item [\ding{172}] Công thức theo đơn vị rad:
				      $$\tan x = \tan \alpha \Leftrightarrow x=\alpha+k\pi, \,k \in \mathbb{Z}$$
				      \item [\ding{173}] Công thức theo đơn vị độ:
				      $$\tan x = \tan \beta^\circ \Leftrightarrow x=\beta^\circ +k180^\circ, \,k \in \mathbb{Z}$$
			      \end{listEX}
		      }{\begin{tikzpicture}[smooth,samples=300,scale=1,>=stealth]
				      \draw[->] (-1.5,0)--(1.5,0);
				      \draw[->] (0,-1.3)--(0,1.5);
				      \draw (0,0) node[below right]{\footnotesize$O$};
				      \tkzDefPoints{0/0/I, 1/0.9/A}
				      \draw[orange] (I) circle(1cm);
				      \draw[->] (1,-1.3)--(1,1.5)node[right]{\footnotesize$tan$};
				      \tkzInterLC[R](I,A)(I,1cm)\tkzGetPoints{M}{N}
				      \tkzDrawPoints[size=3,fill=blue](I,M,N,A)
				      \tkzDrawSegments(A,N)
				      \tkzLabelPoints[below,font=\footnotesize,blue](M)
				      \tkzLabelPoints[above,font=\footnotesize,blue](N)
				      \node[right] at (1,0.9) {$a$};
			      \end{tikzpicture}}
\end{enumerate}

\textbf{$\bigstar$ Phương trình $\cot x = b$.}
	$b \in \left\{\pm \dfrac{\sqrt{3}}{3};\pm 1; \pm \sqrt{3}\right\}$ hoặc $b$ bất kì. Ta bấm máy \shiftk \tank \fbox{$\tfrac{1}{b}$} để tìm góc $\alpha$ hoặc $\beta^\circ$ tương ứng. Riêng $b=0$ thì $\alpha=\dfrac{\pi}{2}$. Công thức nghiệm tương tự phương trình $\tan x =a$
\end{khung4}
% 
% \section{BÀI TẬP ÔN TẬP CHƯƠNG I}
\Opensolutionfile{ans}[ans/ans-1K1-4-OTC]
\begin{ex}%[Câu 1]%[1K1Y1-1]
	Đổi $225^\circ$ sang rađian.
	\choice
	{$\dfrac{4\pi}{5}$}
	{$\dfrac{6\pi}{5}$}
	{$\dfrac{3\pi}{7}$}
	{\True $\dfrac{5\pi}{4}$}
	%<MyLT2>
	\loigiai{
		Ta có $225^\circ = \dfrac{225}{180}\pi=\dfrac{5\pi}{4}$ (rađian).
	}
\end{ex}
\begin{ex}%[Câu 2]%[1K1Y1-3]
	Một đường tròn có bán kính $R=10$ cm. Độ dài cung $40^\circ$ trên đường tròn gần bằng
	\choice
	{$11$ cm}
	{$13$ cm}
	{\True $7$ cm}
	{$9$ cm}
	\loigiai{
		Ta có $40^\circ = 40\cdot \dfrac{\pi}{180} =\dfrac{2\pi}{9}$ rađian.\\
		Độ dài cung $l=\dfrac{2\pi}{9}\cdot 10=\dfrac{20\pi}{9}\approx 7$ cm.
	}
\end{ex}
\begin{ex}%[Câu 3]%[1K1B1-4]
	Bánh xe của người đi xe đạp quay được $2$ vòng trong $6$ giây. Hỏi trong $1$ giây, bánh xe quay được bao nhiêu độ?
	\choice
	{$60^\circ$}
	{$72^\circ$}
	{$240^\circ$}
	{\True $120^\circ$}
	\loigiai{
		Trong $6$ giây, bánh xe quay được $2\cdot 360^\circ=720^\circ$.\\
		Trong $1$ giây, bánh xe quay được $720^\circ\colon 6=120^\circ$.
	}
\end{ex}
\begin{ex}%[Câu 4]%[1K1Y1-6]
	Cho góc $\alpha$ thỏa mãn $90^\circ <\alpha <180^\circ$. Khẳng định nào sau đây đúng?
	\choice
	{$\cos\alpha>0$}
	{\True $\sin\alpha>0$}
	{$\tan\alpha>0$}
	{$\cot\alpha>0$}
	\loigiai
	{Vì $90^\circ <\alpha <180^\circ$ nên $\sin\alpha>0$, $\cos\alpha<0$, $\tan\alpha<0$ và $\cot\alpha<0$.}
\end{ex}
\begin{ex}%[Câu 5]%[1K1B1-6]
	Cho $\sin \alpha =\dfrac{1}{3}$ và $\dfrac{\pi}{2}<\alpha<\pi$. Khi đó $\cos \alpha$ có giá trị là
	\choice
	{$\cos \alpha =-\dfrac{2}{3}$}
	{$\cos \alpha =\dfrac{2\sqrt{2}}{3}$}
	{$\cos \alpha =\dfrac{8}{9}$}
	{\True $\cos \alpha =-\dfrac{2\sqrt{2}}{3}$}
	\loigiai{
		Ta có $\cos^2 \alpha =1-\sin^2 \alpha =1-\left(\dfrac{1}{3}\right)^2=\dfrac{8}{9}$.\\
		Vì $\dfrac{\pi}{2}<\alpha<\pi$ nên $\cos \alpha <0$.\\
		Do đó $\cos \alpha =-\dfrac{2\sqrt{2}}{3}$.
	}
\end{ex}
\begin{ex}%[Câu 6]%[1K1B1-7]
	Cho $A$, $B$, $C$ là ba góc của tam giác $ABC$. Trong các khẳng định sau, khẳng định nào \textbf{sai}?
	\choice
	{$\sin (B+C)=\sin A$}
	{$\cos (B+C)=-\cos A$}
	{\True $\tan (B+C)=\tan A$}
	{$\cot (B+C)=-\cot A$}
	\loigiai{
		Ta có $B+C=180^\circ-A$.\\Suy ra $\tan(B+C)=\tan (180^\circ-A)=-\tan A$.
	}
\end{ex}
\begin{ex}%[Câu 7]%[1K1B1-8]
	Tính giá trị biểu thức $P=\cos ^2\dfrac{\pi}{8}+\cos ^2\dfrac{{3\pi}}{8}+\cos ^2\dfrac{{5\pi}}{8}+\cos ^2\dfrac{{7\pi}}{8}.$
	\choice
	{$P=-1$}
	{$P=0$}
	{$P=1$}
	{\True $P=2$}
	\loigiai{Ta có $\cos ^2\dfrac{7\pi}{8}=\cos ^2\dfrac{\pi}{8}$ và $\cos ^2\dfrac{5\pi}{8}=\cos ^2\dfrac{3\pi}{8}$
		$\Rightarrow P=2\left({{\cos}^2\dfrac{\pi}{8}+{\cos}^2\dfrac{{3\pi}}{8}}\right)$.\\
		Vì $\dfrac{\pi}{8}+\dfrac{{3\pi}}{8}=\dfrac{\pi}{2}\Rightarrow \cos \dfrac{\pi}{8}=\sin \dfrac{{3\pi}}{8}\Rightarrow \cos ^2\dfrac{\pi}{8}=\sin ^2\dfrac{{3\pi}}{8}.$\\
		Do đó $ P=2 \left({{\sin}^2\dfrac{{3\pi}}{8}+{\cos}^2\dfrac{{3\pi}}{8}}\right)=2\cdot1=2.$
	}
\end{ex}
\begin{ex}%[Câu 8]%[1K1B1-8]
	Cho $\sin a + \cos a = - \dfrac 54$, khi đó giá trị của $\sin a \cos a$ bằng
	\choice
	{$1$}
	{$\dfrac{5}{4}$}
	{$\dfrac{3}{16}$}
	{\True $\dfrac{9}{32}$}
	\loigiai{
		$\sin a\cos a = \dfrac{(\sin a + \cos a)^2 -1}{2} = \dfrac{9}{32}$.
	}
\end{ex}
\begin{ex}%[Câu 9]%[1K1Y1-8]
	Cho $\tan x=\dfrac{1}{2}$. Tính $\tan \left(x+\dfrac{\pi}{4}\right)$.
	\choice
	{$2$}
	{$\dfrac{3}{2}$}
	{$6$}
	{\True $3$}
	\loigiai
	{
		Ta có $\tan \left(x+\dfrac{\pi}{4}\right)=\dfrac{\tan x+\tan \dfrac{\pi}{4}}{1-\tan x\cdot \tan \dfrac{\pi}{4}}=\dfrac{\dfrac{1}{2}+1}{1-\dfrac{1}{2}} = 3$.
	}
\end{ex}
\begin{ex}%[Câu 10]%[1K1Y1-3]
	Biểu diễn các góc lượng giác $\alpha=-\dfrac{5\pi}{6}$, $\beta=\dfrac{\pi}{3}$, $\gamma=\dfrac{25\pi}{3}$, $\delta=\dfrac{17\pi}{6}$ trên đường tròn lượng giác. Các góc nào có điểm biểu diễn trùng nhau?
	\choice
	{\True $\beta$ và $\gamma$}
	{$\alpha$, $\beta$, $\gamma$}
	{$\beta$, $\gamma$, $\delta$}
	{$\alpha$ và $\beta$}
	\loigiai{
		Ta có $\beta+8\pi=\dfrac{\pi}{3}+8\pi=\dfrac{25\pi}{3}=\gamma$.\\
		Do đó, $\beta$ và $\gamma$ có điểm biểu diễn trùng nhau trên đường tròn lượng giác.
	}
\end{ex}
\begin{ex}%[Câu 11]%[1K1Y1-7]
	Trong các khẳng định sau, khẳng định  nào là \textbf{sai}?
	\choice
	{$\sin(\pi-\alpha)=\sin\alpha$}
	{\True $\cos(\pi-\alpha)=\cos \alpha$}
	{$\sin(\pi+\alpha)=-\sin\alpha$}
	{$\cos(\pi+\alpha)=-\cos \alpha$}
	\loigiai{
		Ta có $\cos(\pi-\alpha)=-\cos \alpha$ nên $\cos(\pi-\alpha)=\cos \alpha$ là khẳng định \textbf{sai}.
	}
\end{ex}
\begin{ex}%[Câu 12]%[1K1Y1-2]
	Góc lượng giác nào tương ứng với chuyển động quay $3\dfrac{1}{5}$ vòng ngược chiều kim đồng hồ?
	\choice
	{$\dfrac{16 \pi}{5}$}
	{$\left(\dfrac{16}{5}\right)^\circ$}
	{\True $1152^\circ$}
	{$1152 \pi$}
	\loigiai{
		Chuyển động quay ngược chiều kim đồng hồ là quay theo chiều dương; góc tương ứng là
		$$3\dfrac{1}{5}\cdot 2\pi=\dfrac{32\pi}{5}, \text{ tương ứng với } 1152^\circ.$$
	}
\end{ex}

\begin{ex}%[Câu 13]%[1K1Y2-1]
	Trong các khẳng định sau, khẳng định nào \textbf{sai}?
	\choice
	{\True $\cos (a-b)=\cos a\cos b-\sin a\sin b$}
	{$\sin (a-b)=\sin a\cos b-\cos a\sin b$}
	{$\cos (a+b)=\cos a\cos b-\sin a\sin b$}
	{$\sin (a+b)=\sin a\cos b+\cos a\sin b$}
	\loigiai{
		Ta có $\cos (a-b)=\cos a\cos b+\sin a\sin b$ nên $\cos (a-b)=\cos a\cos b-\sin a\sin b$ là khẳng định \textbf{sai}.
	}
\end{ex}
\begin{ex}%[Câu 14]%[1K1Y1-7]
	Trong trường hợp nào dưới đây $\cos \alpha=\cos \beta$ và $\sin \alpha=-\sin \beta$?
	\choice
	{\True $\beta=-\alpha$}
	{$\beta=\pi-\alpha$}
	{$\beta=\pi+\alpha$}
	{$\beta=\dfrac{\pi}{2}+\alpha$}
	\loigiai{
		Trong trường hợp hai cung đối nhau thì các giá trị $\cos$ của chúng bằng nhau, các giá trị $\sin$ của chúng đối nhau.
	}
\end{ex}

\begin{ex}%[Câu 15]%[1K1B2-2]
	Nếu $\cos a=\dfrac{1}{4}$ thì $\cos 2 a$ bằng
	\choice
	{$\dfrac{7}{8}$}
	{\True $-\dfrac{7}{8}$}
	{$\dfrac{15}{16}$}
	{$-\dfrac{15}{16}$}
	\loigiai{
		Ta có  $\cos 2 a =2 \cos^2 a-1=2 \cdot \left( \dfrac{1}{4} \right)^2-1=-\dfrac{7}{8}$.
	}
\end{ex}
\begin{ex}%[Câu 16]%[1K1K2-2]
	Nếu $\tan (a+b)=3, \tan (a-b)=-3$ thì $\tan 2 a$ bằng
	\choice
	{\True $0$}
	{$\dfrac{3}{5}$}
	{$1$}
	{$-\dfrac{3}{4}$}
	\loigiai{
		Ta có $\tan (a+b)=3 \Leftrightarrow \tan a+ \tan b= 3- 3 \tan a \tan b$ \quad (1)\\
		và 
		$\tan (a-b)=-3 \Leftrightarrow \tan a- \tan b= -3- 3 \tan a \tan b$. \quad (2) \\
		Lấy vế trừ vế của (1) và (2) ta được $2\tan b=6\Leftrightarrow \tan b =3$.\\
		Thay $\tan b =3$ vào (1) ta được $\tan a= 0$.\\
		Khi đó $\tan 2 a = \dfrac{2 \tan a}{1- \tan^2 a}=0$.
	}
\end{ex}

\begin{ex}%[Câu 17]%[1K1K2-3]
	Nếu $\cos a=\dfrac{3}{5}$ và $\cos b=-\dfrac{4}{5}$ thì $\cos (a+b) \cos (a-b)$ bằng
	\choice
	{\True $0$}
	{$2$}
	{$4$}
	{$5$}
	\loigiai{
		Do  $\cos a=\dfrac{3}{5}$ và $\cos b=-\dfrac{4}{5}$ nên  $\cos 2a=-\dfrac{7}{25}$ và $\cos 2b=\dfrac{7}{25}$.\\
		Ta có $2 \cos (a+b) \cos (a-b)= \cos 2a +\cos 2b=-\dfrac{7}{25}+ \dfrac{7}{25}=0$.\\
		Do đó $\cos (a+b) \cos (a-b)=0$.
	}
\end{ex}




\begin{ex}%[Câu 18]%[1K1B2-3]
	Rút gọn biểu thức $M=\cos(a+b)\cos(a-b)-\sin (a+b)\sin(a-b)$, ta được
	\choice
	{$M=\sin 4a$}
	{$M=1-2\cos^2a$}
	{\True $M=1-2\sin^2a$}
	{$M=\cos 4a$}
	\loigiai{
		Ta có
		\allowdisplaybreaks
		\begin{eqnarray*}
			M&=&\cos(a+b)\cos(a-b)-\sin (a+b)\sin(a-b)\\
			&=&\dfrac{1}{2}\left(\cos2a+\cos 2b\right)+\dfrac{1}{2}\left(\cos2a-\cos 2b\right)\\
			&=&\cos 2a\\
			&=&1-2\sin^2a.
		\end{eqnarray*}
	}
\end{ex}
\begin{ex}%[Câu 19]%[1K1B2-2]
	Nếu $\sin x +\cos x = \dfrac{1}{2}$ thì $\sin 2x$ bằng
	\choice
	{$\dfrac{3}{4}$}
	{$\dfrac{3}{8}$}
	{$\dfrac{\sqrt{2}}{2}$}
	{\True $-\dfrac{3}{4}$}
	\loigiai{
		Ta có $\sin 2x=\left( {\sin x+\cos x} \right)^2-\left( {\sin^2 x+\cos ^2x} \right)=\left( {\dfrac{1}{2}} \right)^2-1=-\dfrac{3}{4}$.}
\end{ex}
\begin{ex}%[Câu 20]%[1K1Y2-3]
	Mệnh đề nào dưới đây đúng?
	\choice
	{\True $\cos3x\cdot\cos5x=\dfrac{1}{2}(\cos8x+\cos2x)$}
	{$\cos3x\cdot\cos5x=\dfrac{1}{2}(\cos8x-\cos2x)$}
	{$\cos3x\cdot\cos5x=\dfrac{1}{2}(\cos2x-\cos8x)$}
	{$\cos3x\cdot\cos5x=\dfrac{1}{2}(\sin8x+\sin2x)$}
	\loigiai{Ta có $\cos3x\cdot\cos5x=\dfrac{1}{2}[\cos(3x+5x)+\cos(3x-5x)]=\dfrac{1}{2}(\cos8x+\cos2x)$.}
\end{ex}
\begin{ex}%[Câu 21]%[1K1B2-4]
	Giả sử $3\sin ^4x-\cos ^4x=\dfrac{1}{2}$ thì $\sin ^4x+3\cos ^4x$ có giá trị bằng
	\choice
	{$2$}
	{\True $1$}
	{$4$}
	{$3$}
	\loigiai{
		\begin{eqnarray*}
			3\sin ^4x-\cos ^4x=\dfrac{1}{2}\Leftrightarrow 6\sin ^4x-2\cos ^4x=1&\Leftrightarrow& 6\sin ^4x-2\left(1-\sin^2\alpha\right)^2=1\\
			&\Leftrightarrow& 4\sin ^4x-4\sin ^2\alpha-3=0\\
			&\Leftrightarrow& \left(2\sin^2\alpha+3\right)\left(2\sin^2\alpha-1\right)=0\\
			&\Rightarrow&{\sin ^2}\alpha=\dfrac{1}{2}.
		\end{eqnarray*}
		Ta có $\sin ^4x+3\cos ^4x$ $=\sin ^4\alpha+3\left(1-\sin^2\alpha\right)^2$ $=\dfrac{1}{4}+3\left(1-\dfrac{1}{2}\right)^2=1$.}
\end{ex}
\begin{ex}%[Câu 22]%[1K1B3-2]
	Hàm số $y=\sin x$ đồng biến trên khoảng
	\choice
	{$(0 ; \pi)$}
	{$\left(-\dfrac{3 \pi}{2} ;-\dfrac{\pi}{2}\right)$}
	{\True $\left(-\dfrac{ \pi}{2} ;\dfrac{\pi}{2}\right)$}
	{$(-\pi ; 0)$}
	\loigiai{ 
		Do hàm số $y=\sin x$ đồng biến trên mỗi khoảng $\left( -\dfrac{\pi}{2}+k2 \pi;\dfrac{\pi}{2}+k2 \pi \right) $ nên ứng với $k=0$, ta có hàm số $y=\sin x$ đồng biến trên khoảng $\left(-\dfrac{ \pi}{2} ;\dfrac{\pi}{2}\right)$.
	}
\end{ex}

\begin{ex}%[Câu 23]%[1K1B3-2]
	Hàm số nghịch biến trên khoảng $(\pi ; 2 \pi)$ là
	\choice
	{$y=\sin x$}
	{$y=\cos x$}
	{$y=\tan x$}
	{\True $y=\cot x$}
	\loigiai{
		Do hàm số $y=\cot x$ nghịch biến trên mỗi khoảng $\left( k \pi; \pi +k \pi \right) $ nên  ứng với $k=1$, ta có hàm số $y=\cot x$ nghịch biến trên khoảng $(\pi ; 2 \pi)$.
	}
\end{ex}
\begin{ex}%[Câu 24]%[1K1B3-1]
	Tập xác định của hàm số $y=\dfrac{\cos x}{\sin x-1}$ là
	\choice
	{$\mathbb{R}\setminus \left\{k2\pi| k\in\mathbb{Z}\right\}$}
	{\True $\mathbb{R}\setminus \left\{\dfrac{\pi}{2}+k2\pi| k\in\mathbb{Z}\right\}$}
	{$\mathbb{R}\setminus \left\{\dfrac{\pi}{2}+k\pi| k\in\mathbb{Z}\right\}$}
	{$\mathbb{R}\setminus \left\{k\pi| k\in\mathbb{Z}\right\}$}
	\loigiai{
		Hàm số xác định khi và chỉ khi $\sin x-1\ne 0\Leftrightarrow\sin x\ne 1\Leftrightarrow x\ne \dfrac{\pi}{2}+k2\pi$ với $k\in \mathbb{Z}$.\\
		Vậy tập xác định của hàm số là $\mathbb{R}\setminus \left\{\dfrac{\pi}{2}+k2\pi| k\in\mathbb{Z}\right\}$.
	}
\end{ex}
\begin{ex}%[Câu 25]%[1K1Y3-1]
	Khẳng định nào sau đây là \textbf{sai}?
	\choice
	{Hàm số $y=\cos x$ có tập xác định là $\mathbb{R}$}
	{Hàm số $y=\cos x$ có tập giá trị là $[-1;1]$}
	{\True Hàm số $y=\cos x$ là hàm số lẻ}
	{Hàm số $y=\cos x$ tuần hoàn với chu kì $2\pi$}
	\loigiai{
		Hàm số $y=\cos x$ là hàm số chẵn.
	}
\end{ex}
\begin{ex}%[Câu 26]%[1K1Y3-4]
	Trong các hàm số sau đây, hàm số nào là hàm tuần hoàn?
	\choice
	{$y=\tan x+x$}
	{$y=x^2+1$}
	{\True $y=\cot x$}
	{$y=\dfrac{\sin x}{x}$}
	\loigiai{
		Hàm số $y=\cot x$ là hàm số tuần hoàn với chu kỳ $T=\pi$.
	}
\end{ex}
\begin{ex}%[Câu 27]%[1K1Y3-3]
	Khẳng định nào sau đây đúng?
	\choice
	{Hàm số $y=\sin x$ là hàm số chẵn}
	{\True Hàm số $y=\cos x$ là hàm số chẵn}
	{Hàm số $y=\tan x$ là hàm số chẵn}
	{Hàm số $y=\cot x$ là hàm số chẵn}
	\loigiai{
		Hàm số $y=\cos x$ là hàm số chẵn;  các hàm số còn lại là hàm số lẻ.
	}
\end{ex}
\begin{ex}%[Câu 28]%[1K1B3-3]
	Khẳng định nào sau đây là đúng?
	\choice
	{Hàm số $y=\cos x$ là hàm số lẻ}
	{\True Hàm số $y=\tan 2x- \sin x$ là hàm số lẻ}
	{Hàm số $y=\sin x$ là hàm số chẵn}
	{Hàm số $y=\tan x \cdot \sin x$ là hàm số lẻ}
	\loigiai{
		Xét hàm số $y=f\left( {x} \right)=\tan 2x-\sin x$.
		\\Hàm số xác định khi $\cos 2x \ne 0 \Leftrightarrow x \ne \dfrac{\pi}{4}+k\dfrac{\pi}{2}$, $\left( {k \in \mathbb{Z}} \right)$.
		\\Tập xác định $\mathscr{D}=\mathbb{R} \setminus \left\{ {\dfrac{\pi}{4}+k\dfrac{\pi}{2},k \in \mathbb{Z}} \right\}$.
		\\ Với mọi $x \in \mathscr{D}$ thì $-x \in \mathscr{D}$ và $f\left( {-x} \right)=\tan \left( {-2x} \right)- \sin \left( {-x} \right)=-\tan 2x + \sin x=-f\left( {x} \right)$.\\ Do đó hàm số $y=\tan 2x-\sin x$ là hàm số lẻ.}
\end{ex}
\begin{ex}%[Câu 29]%[1K1B3-1]
	Tập xác định của hàm số $y=\dfrac{\cot x}{\cos x-1}$ là
	\choice
	{$\mathbb{R}\setminus\left\{\dfrac{k\pi}{2}, k\in\mathbb{Z}\right\}$}
	{$\mathbb{R}\setminus\left\{\dfrac{k}{2}+k\pi, k\in\mathbb{Z}\right\}$}
	{\True $\mathbb{R}\setminus\left\{k\pi, k\in\mathbb{Z}\right\}$}
	{$\mathbb{R}\setminus\left\{k2\pi, k\in\mathbb{Z}\right\}$}
	\loigiai{
		Hàm số xác định khi và chỉ khi $\heva{&\sin x\ne 0\\ &\cos x\ne 1}\Leftrightarrow\heva{&x\ne k\pi\\ &x\ne l2\pi}\ (k,l\in\mathbb{Z})\Leftrightarrow x\ne k\pi, k\in\mathbb{Z}$.\\
		Vậy, tập xác định của hàm số $y=\dfrac{\cot x}{\cos x-1}$ là $\mathbb{R}\setminus\left\{k\pi, k\in\mathbb{Z}\right\}$.
	}
\end{ex}
\begin{ex}%[Câu 30]%[1K1K3-2]
	Cho đồ thị hàm số $y=\sin x$ như hình vẽ sau
	\begin{center}
		
		\begin{tikzpicture}[>=stealth,scale=0.7]
			\draw [->] (-11,0)--(0,0)
			node[below right]{$O$}--(11,0)node[below]{$x$}; % Hệ trục tọa độ
			\draw[->] (0,-1.5) --(0,2) node[left]{$y$};
			\draw[dashed] (-5*pi/2,0)node[above]{$-\tfrac{5\pi}{2}$}--(-5*pi/2,-1)--(3*pi/2,-1)--(3*pi/2,0)node[above]{$\tfrac{3\pi}{2}$};
			\draw[dashed] (-3*pi/2,0)node[below]{$-\tfrac{3\pi}{2}$} --(-3*pi/2,1)--(5*pi/2,1)--(5*pi/2,0)node[below]{$\tfrac{5\pi}{2}$};
			\draw[dashed] (-pi/2,0)node[above]{$-\tfrac{\pi}{2}$}--(-pi/2,-1);
			\draw[dashed] (pi/2,0)node[below]{$\tfrac{\pi}{2}$}--(pi/2,1);
			\draw(-2.9*pi,0) node[below left]{$-3\pi$}(-1.9*pi,0) node[below]{$-2\pi$}(-1.1*pi,0) node[below]{$-\pi$}(3*pi,0) node[below]{$3\pi$}(2*pi,0) node[below]{$2\pi$}(pi,0) node[above]{$\pi$}(0,-1.6)node[below]{$2\pi$}(0,1)node[above right]{$1$};
			\draw[dashed] (-pi,0)--(-pi,-1.6)(pi,0)--(pi,-1.6);
			\draw[<->](-pi,-1.6)--(pi,-1.6);
			\draw [domain=-3.5*pi:3.4*pi,samples=100] plot (\x, {sin(\x r)});
		\end{tikzpicture}
	\end{center}
	Mệnh đề nào dưới đây \textbf{sai}?
	\choice
	{Hàm số $y=\sin x$ tăng trên khoảng $\left(-\dfrac{\pi}{2};\dfrac{\pi}{2}\right)$}
	{Hàm số $y=\sin x$ giảm trên khoảng $\left(\dfrac{\pi}{2};\dfrac{3\pi}{2}\right)$}
	{Hàm số $y=\sin x$ giảm trên khoảng $\left(-\dfrac{3\pi}{2};-\pi \right)$}
	{\True Hàm số $y=\sin x$ tăng trên khoảng $\left(0;\pi \right)$}
	\loigiai{
		\begin{itemize}
			\item Hàm số $y=\sin x$ tăng trên $\left(0;\dfrac{\pi}{2}\right)$ và giảm trên $\left(\dfrac{\pi}{2};\pi \right)$.
			\item Vậy trên khoảng $\left(0;\pi \right)$, hàm số $y=\sin x$ vừa tăng vừa giảm nên khẳng định hàm số $y=\sin x$ tăng trên khoảng $\left(0;\pi \right)$ là khẳng định \textbf{sai}.
	\end{itemize}}
\end{ex}
\begin{ex}%[Câu 31]%[1K1B3-2]
	Chọn khẳng định đúng trong các khẳng định sau
	\choice
	{Hàm số $y = \tan x$ tuần hoàn với chu kì $2\pi$}
	{Hàm số $y = \cos x$ tuần hoàn với chu kì $\pi$}
	{\True Hàm số $y = \sin x$ đồng biến trên khoảng $\left(0; \dfrac{\pi}{2}\right)$}
	{Hàm số $y = \cot x$ nghịch biến trên $\mathbb{R}$}
	\loigiai{Ta xét $y = \sin x$ suy ra  $y'  = \cos x$. Dễ thấy $\cos x > 0\ ,\  \forall x\in \left(0; \dfrac{\pi}{2}\right)$. Do đó hàm số $y = \sin x$ đồng biến trên khoảng $\left(0; \dfrac{\pi}{2}\right)$.
	}
\end{ex}
\begin{ex}%[Câu 32]%[1K1K3-6]
	Đồ thị của hàm số $y=\sin x$ và $y=\cos x$ cắt nhau tại bao nhiêu điểm có hoành độ thuộc đoạn $\left[-2\pi;\dfrac{5\pi}{2}\right]$?
	\choice
	{\True $5$}
	{$6$}
	{$4$}
	{$7$}
	\loigiai{
		Xét phương trình hoành độ giao điểm của hai đồ thị hàm số $\sin x=\cos x$.\\
		Nếu $\cos x=0$ thì $\sin x=0$ nên vô lý.\\
		Do đó, $\cos x\ne 0$. Ta có
		\allowdisplaybreaks
		\begin{eqnarray*}
			\sin x=\cos x&\Leftrightarrow&\tan x=1\\
			&\Leftrightarrow&x=\dfrac{\pi}{4}+k\pi ,\quad \left(k\in\mathbb{Z}\right).
		\end{eqnarray*}
		Ta lại có
		\allowdisplaybreaks
		\begin{eqnarray*}
			-2\pi \le x\le \dfrac{5\pi}{2}&\Leftrightarrow& -2\pi \le \dfrac{\pi}{4}+k\pi\le \dfrac{5\pi}{2}\\
			&\Leftrightarrow& -2 \le \dfrac{1}{4}+k\le \dfrac{5}{2}\\
			&\Leftrightarrow& \dfrac{-9}{4} \le k\le \dfrac{9}{4}.
		\end{eqnarray*}
		Do $k\in\mathbb{Z}$ nên $k\in\left\{-2;-1;0;1;2\right\}$.\\
		Vậy hai đồ thị hàm số cắt nhau tại $5$ điểm có hoành độ thuộc đoạn $\left[-2\pi;\dfrac{5\pi}{2}\right]$.
	}
\end{ex}
\begin{ex}%[Câu 33]%[1K1B3-5]
	Tìm tập giá trị của hàm số $y=2\cos3x +1$.
	\choice
	{$[-3;1]$}
	{$[-3;-1]$}
	{\True $[-1;3]$}
	{$[1;3]$}
	\loigiai{
		$\forall x\in \mathbb{R}$ ta có
		\begin{eqnarray*}
			&& -1\leq\cos3x\leq1 \\
			&\Leftrightarrow& -2\leq2\cos3x\leq2 \\
			&\Leftrightarrow& -1\leq2\cos3x+1\leq3.
		\end{eqnarray*}
	}
\end{ex}
\begin{ex}%[Câu 34]%[1K1B3-6]
	Đường cong trong hình bên là đồ thị trên đoạn $\left[-\pi ;\pi\right]$ của một hàm số trong bốn hàm số được liệt kê ở bốn phương án $\textbf{A, B, C, D}$ dưới đây. Hỏi đó là hàm số nào?
	\begin{center}
		\definecolor{x}{rgb}{0.75,0.75,0.75}
		\begin{tikzpicture}[scale=1, line join=round, line cap=round,>=stealth]
			\draw[->] (-4,0.) -- (4,0.)node [above] { $x$};
			\draw[shift={(-3.14,0)}] node[below left] {\footnotesize $-\pi$};
			\draw[shift={(-1.57,0)}] node[above left] {\footnotesize $-\dfrac{\pi}{2}$};
			\draw[shift={(1.57,0)}] node[below left] {\footnotesize $\dfrac{\pi}{2}$};
			\draw[shift={(3.14,0)}] node[below left] {\footnotesize $\pi$};
			\draw[->] (0.,-1.3) -- (0.,1.5)node [right] { $y$};
			\draw (0,1) node[above left] {\footnotesize $1$};
			\draw (0,-1) node[above left] {\footnotesize $-1$};
			\draw (0pt,-10pt) node[right] {\footnotesize $O$};
			\clip(-4.2,-1.3) rectangle (4.2,1.5);
			\draw[line width=1.2pt,smooth,samples=100,domain=-3.14:3.14] plot(\x,{sin(((\x))*180/pi)});
			\draw [dashed] (-1.57,0)--(-1.57,-1)--(0,-1)(1.57,0)--(1.57,1)--(0,1);
		\end{tikzpicture}
	\end{center}
	\choice
	{\True $y=\sin x$}
	{$y=\cos x$}
	{$y=\tan x$}
	{$y=\cot x$}
	\loigiai{
		Đồ thị hàm số đi qua các điểm $(0;0),(\pi;0), \left(\dfrac{\pi}{2};1\right)$ và nhận $O$ làm tâm đối xứng.
	}
\end{ex}
\begin{ex}%[Câu 35]%[1K1Y4-3]
	Phương trình $\cot x=-1$ có nghiệm là
	\choice
	{\True $-\dfrac{\pi}{4}+k \pi(k \in \mathbb{Z})$}
	{$\dfrac{\pi}{4}+k \pi(k \in \mathbb{Z})$}
	{$\dfrac{\pi}{4}+k 2 \pi(k \in \mathbb{Z})$}
	{$-\dfrac{\pi}{4}+k 2 \pi(k \in \mathbb{Z})$}
	\loigiai{
		Ta có $\cot x=-1 \Leftrightarrow \cot x=\cot \left(-\dfrac{\pi}{4}\right) \Leftrightarrow x =-\dfrac{\pi}{4}+k \pi(k \in \mathbb{Z}) $.
	}
\end{ex}
\begin{ex}%[Câu 36]%[1K1Y4-3]
	Trong các phép biến đổi sau, phép biến đổi nào \textbf{sai}?
	\choice
	{$\sin x=1\Leftrightarrow x=\dfrac{\pi}{2}+k2\pi,(k\in \mathbb{Z})$}
	{$\tan x=1\Leftrightarrow x=\dfrac{\pi}{4}+k\pi,(k\in \mathbb{Z})$}
	{$\cos x=\dfrac{1}{2}\Leftrightarrow \hoac{
			& x=\dfrac{\pi}{3}+k2\pi,(k\in \mathbb{Z}) \\
			& x=-\dfrac{\pi}{3}+k2\pi,(k\in \mathbb{Z})}$}
	{\True $\sin x=0\Leftrightarrow x=k2\pi,(k\in \mathbb{Z})$}
	\loigiai{
		Ta có $\sin x=0\Leftrightarrow x=k\pi,(k\in \mathbb{Z})$, nên đáp án $\sin x=0\Leftrightarrow x=k2\pi,(k\in \mathbb{Z})$ sai.}
\end{ex}
\begin{ex}%[Câu 37]%[1K1B4-5]
	Nghiệm của phương trình $\sin x\cdot \cos x=\dfrac{1}{2}$ là
	\choice
	{$x=k2\pi$; $k\in \mathbb{Z}$}
	{$x=\dfrac{k\pi}{4}$; $k\in \mathbb{Z}$}
	{\True $x=\dfrac{\pi}{4}+k\pi$; $k\in \mathbb{Z}$}
	{$x=k\pi$; $k\in \mathbb{Z}$}
	\loigiai{
		Ta có $\sin x\cdot \cos x=\dfrac{1}{2}\Leftrightarrow \sin 2x=1\Leftrightarrow 2x=\dfrac{\pi}{2}+k2\pi\Leftrightarrow x=\dfrac{\pi}{4}+k\pi$ với $k\in\mathbb{Z}$.
	}
\end{ex}
\begin{ex}%[Câu 38]%[1K1Y4-3]
	Họ nghiệm của phương trình $\sin2x=1$ là
	\choice
	{$x=\dfrac{\pi}{2}+k\pi,\,k\in\mathbb{Z}$}
	{$x=\dfrac{\pi}{2}+k2\pi,\,k\in\mathbb{Z}$}
	{\True $x=\dfrac{\pi}{4}+k\pi,\,k\in\mathbb{Z}$}
	{$x=\dfrac{\pi}{4}+\dfrac{k\pi}{2},\,k\in\mathbb{Z}$}
	\loigiai{
		Ta có $\sin2x=1\Leftrightarrow 2x=\dfrac{\pi}{2}+k2\pi\Leftrightarrow x=\dfrac{\pi}{4}+k\pi,\, k\in\mathbb{Z}$.
	}
\end{ex}
\begin{ex}%[Câu 39]%[1K1B4-5]
	Phương trình $\sin 2x \cos x = \sin 7x \cos 4x$ có các họ nghiệm là
	\choice
	{$x=\dfrac{k2\pi}{5};x=\dfrac{\pi}{12}+\dfrac{k\pi}{6} (k \in \Bbb{Z})$}
	{$x=\dfrac{k\pi}{5};x=\dfrac{\pi}{12}+\dfrac{k\pi}{3} (k \in \Bbb{Z})$}
	{\True $x=\dfrac{k\pi}{5};x=\dfrac{\pi}{12}+\dfrac{k\pi}{6} (k \in \Bbb{Z})$}
	{$x=\dfrac{k2\pi}{5};x=\dfrac{\pi}{12}+\dfrac{k\pi}{3} (k \in \Bbb{Z})$}
	\loigiai{
		Ta có \begin{eqnarray*}
			\sin 2x \cos x = \sin 7x \cos 4x &\Leftrightarrow & \dfrac{1}{2}(\sin 3x+\sin x)=\dfrac{1}{2}(\sin 11x+\sin 3x)\\
			&\Leftrightarrow & \sin 11x = \sin x\\
			&\Leftrightarrow & \hoac{&x=\dfrac{k\pi}{5}\\&x=\dfrac{\pi}{12}+\dfrac{k\pi}{3} }(k \in \Bbb{Z}).
		\end{eqnarray*}
	}
\end{ex}
\begin{ex}%[Câu 40]%[1K1K4-3]
	Số nghiệm của phương trình $\cos x=0$ trên đoạn $[0 ; 10 \pi]$ là
	\choice
	{$5$}
	{$9$}
	{\True $10$}
	{$11$}
	\loigiai{
		Ta có $\cos x=0 \Leftrightarrow x =\dfrac{\pi}{2}+ k \pi (k \in \mathbb{Z})$.\\
		Do $0 \leq x \leq 10 \pi \Leftrightarrow 0 \leq \dfrac{\pi}{2}+ k \pi \leq 10 \Leftrightarrow -\dfrac{1}{2} \leq k \leq \dfrac{19}{2}\Leftrightarrow 0\leq k \leq 9( k \in  \mathbb{Z} )$.\\
		Do đó phương trình $\cos x=0$ có $10$ nghiệm.
	}
\end{ex}

\begin{ex}%[Câu 41]%[1K1B4-3]
	Số nghiệm của phương trình $\sin x=0$ trên đoạn $[0 ; 10 \pi]$ là
	\choice
	{$10$}
	{$6$}
	{$5$}
	{\True $11$}
	\loigiai{
		Ta có $\sin x=0 \Leftrightarrow x = k \pi (k \in \mathbb{Z})$.\\
		Do $0 \leq x \leq 10 \pi \Leftrightarrow 0 \leq k\leq 10$.\\
		Do đó phương trình $\sin x=0$ có $11$ nghiệm.
	}
\end{ex}

\begin{ex}%[Câu 42]%[1K1B4-3]
	Số nghiệm của phương trình $\sin \left(x+\dfrac{\pi}{4}\right)=\dfrac{\sqrt{2}}{2}$ trên đoạn $[0; \pi]$ là
	\choice
	{$4$}
	{$1$}
	{\True $2$}
	{$3$}
	\loigiai{
		Ta có $\sin \left(x+\dfrac{\pi}{4}\right)=\dfrac{\sqrt{2}}{2} \Leftrightarrow \sin \left(x+\dfrac{\pi}{4}\right)=\sin \left( \dfrac{\pi}{4}\right) \Leftrightarrow  \hoac{&x= k 2 \pi\\&x=\dfrac{\pi}{2}+ k 2 \pi} (k \in \mathbb{Z})$.\\
		Do $x \in [0 ; \pi]$ nên $x=0$ hoặc $x=\dfrac{\pi}{2}$.
	}
\end{ex}
\begin{ex}%[Câu 43]%[1K1B4-5]
	Phương trình $ \sin{2x}+3\cos x=0 $ có bao nhiêu nghiệm trong khoảng $ (0;\pi)$?
	\choice
	{$ 0 $}
	{\True $ 1 $}
	{$ 2 $}
	{$ 3 $}
	\loigiai
	{
		Ta có $ \sin{2x}+3\cos x=0 \Leftrightarrow \hoac{& \cos x=0 \\ &\sin x=-\dfrac{3}{2}}\Leftrightarrow \cos x=0 \Leftrightarrow x= \dfrac{\pi}{2}+k\pi$. Do $ x \in (0;\pi) $ nên có một nghiệm là $ x=\dfrac{\pi}{2}$.
	}
\end{ex}

\begin{ex}%[Câu 44]%[1K1B1-9]
	Một bánh xe có $72$ răng. Số đo góc mà bánh xe đã quay được khi di chuyển $10$ răng là
	\choice
	{$40^\circ	$}
	{\True $50^\circ$}
	{$60^\circ$}
	{$30^\circ$}
	\loigiai{
		1 bánh răng tương ứng với $\dfrac{360^\circ}{72}=5^\circ$$\Rightarrow 10$ bánh răng là $50^\circ$.}
\end{ex}

\begin{ex}%[Câu 45]%[1K1K1-9]
	\immini{Người ta muốn làm một cánh diều hình quạt có bán kính là $a$, độ dài cung tròn là $b$ và có chu vi là $80$ cm (như hình vẽ). Khi diện tích cánh diều đạt giá trị lớn nhất, tổng $a+b$ bằng
		\choice
		{$50$ cm}
		{$40$ cm}
		{$70$ cm}
		{\True $60$ cm}}{
		\begin{tikzpicture}
			\draw (0,3) arc (150:210:3);
			\coordinate [label=below:$A$] (A) at (0,0);
			\coordinate [label=right:$O$] (O) at (30:3);
			\coordinate [label=above:$C$] (C) at (0,3);
			\foreach \point in {O,A,C} \fill[black] (\point) circle (1pt);
			\draw (O)--(C) (O)--(A);
		\end{tikzpicture}}
	\loigiai{
		Gọi $\varphi$ (rad) là số đo cung của hình quạt. Khi đó $\varphi =\dfrac{b}{a}$.\\
		Chu vi cánh diều bằng $b+2a=80$.\\
		Diện tích cánh diều bằng $S=\dfrac{\varphi a^2}{2}=\dfrac{ab}{2}=\dfrac{1}{4}(b \cdot 2a) \le \dfrac{1}{4} \cdot \left(\dfrac{b+2a}{2}\right)^2=400$.\\
		Dấu bằng xảy ra khi và chỉ khi $\heva{&b=2a \\& b+2a=80}\Leftrightarrow \heva{&b=40 \\& a=20.}$\\
		Do vậy $a+b=60$ cm.}
\end{ex}
\begin{ex}%[Câu 46]%[1K1K1-9]
	\immini{
		Khi một tia sáng truyền từ không khí vào mặt nước thì một phần tia sáng bị phản xạ trên bề mặt, phần còn lại bị khúc xạ như hình bên. Góc tới $i$ liên hệ với góc khúc xạ $r$ bởi Định luật khúc xạ ánh sáng
		$$\dfrac{\sin i}{\sin r}=\dfrac{n_2}{n_1}.$$
		Ở đây, $n_1$ và $n_2$ tương ứng với chiết suất của môi trường $1$ (không khí) và môi trường $2$ (nước). Cho biết góc tới $i=50^\circ$ và  chiết suất của không khí bằng $1$ còn chiết suất của nước là $1{,}33$. Khi đó  góc khúc xạ gần với kết quả nào sau đây.
		\choice
		{\True$35{,}17^\circ$}
		{$55{,}47^\circ$}
		{$31{,}42^\circ$}
		{$12{,}35^\circ$}
	}
	{
		\begin{tikzpicture}[>=stealth,line join=round,line cap=round,font=\footnotesize,scale=.7]
			\path
			(0,0)coordinate(I)++(90:3)coordinate(N)++(-90:6)coordinate(N')
			(I)++(0:3)coordinate(B)++(180:6)coordinate(A)
			(I)++(30:3)coordinate(S')
			(I)++(150:3)coordinate(S)
			(I)++(-55:3)coordinate(R)
			;
			\fill[cyan!20!](-3,-3)rectangle(3,0)
			;
			\draw (A)--(B)
			;
			\draw[dashed](N)--(N')
			;
			\draw[->,midway](S)--(I)
			;
			\draw[->](I)--(S')
			;
			\draw[->](I)--(R)
			;
			\foreach \p/\r in {N/180,N'/180,S/160,S'/90,R/0,I/-135}
			\fill (\p) node[shift={(\r:3mm)}]{$\p$}
			;
			\draw pic[angle radius=3mm,draw=red,fill=green!50,angle eccentricity=1.5] {angle = N--I--S}
			;
			\draw pic[angle radius=4mm,draw=orange,fill=orange!50,angle eccentricity=1.5] {angle = S'--I--N}
			;
			\draw pic[angle radius=4mm,draw=blue,fill=blue!50,angle eccentricity=1.5] {angle = N'--I--R}
			;
			\draw (-2.5,.5)circle(7pt)node{$1$}
			(-2.5,-.5)circle(8pt)node{$2$}
			;	
		\end{tikzpicture}
	}
	\loigiai{
		Ta có $\dfrac{\sin i}{\sin r}=\dfrac{n_2}{n_1}\Leftrightarrow \dfrac{\sin 50^\circ}{\sin r}=\dfrac{1{,}33}{1}\Leftrightarrow \sin r=\dfrac{\sin 50^\circ}{1{,}33}\Rightarrow r\approx 35{,}17^\circ$.
	}
\end{ex}
\begin{ex}%[Câu 47]%[1K1G2-4]
	Giả sử $a, b, c$ lần lượt là ba cạnh đối diện với ba góc $A, B, C$ của tam giác $ABC$ thỏa điều kiện $2\cos\dfrac{B}{2}\cos\dfrac{C}{2}=\dfrac{1}{2}+\dfrac{b+c}{a}\sin\dfrac{A}{2}$. Tính góc $A$ của tam giác $ABC$.
	\choice
	{$30^\circ$}
	{$45^\circ$}
	{\True $60^\circ$}
	{$90^\circ$}
	\loigiai
	{\noindent Đặt $2\cos\dfrac{B}{2}\cos\dfrac{C}{2}=\dfrac{1}{2}+\dfrac{b+c}{a}\sin\dfrac{A}{2}\;(\star)$. Ta có
		\begin{align*}
			(\star)&\Leftrightarrow  2\cos\dfrac{B}{2}\cos\dfrac{C}{2}=\dfrac{1}{2}+\dfrac{\sin B+\sin C}{\sin A}\sin\dfrac{A}{2}\\
			&\Leftrightarrow  \cos\dfrac{B+C}{2}+\cos\dfrac{B-C}{2}=\dfrac{1}{2}+\dfrac{2\sin\dfrac{B+C}{2}\cos\dfrac{B-C}{2}}{2\sin\dfrac{A}{2}\cos\dfrac{A}{2}}\sin\dfrac{A}{2}\\
			&\Leftrightarrow  \sin\dfrac{A}{2}+\cos\dfrac{B-C}{2}=\dfrac{1}{2}+\cos\dfrac{B-C}{2}\;\left(\text{vì}\;\sin\dfrac{A}{2}>0, \cos\dfrac{A}{2}=\sin\dfrac{B+C}{2}\right)\\
			&\Leftrightarrow  \sin\dfrac{A}{2}=\dfrac{1}{2}\Leftrightarrow  A=\dfrac{\pi}{3}.
		\end{align*}
	}
\end{ex}
\begin{ex}%[Câu 48]%[1K1G4-5]
	Phương trình $2\sqrt{3}\sin\left(x-\dfrac{\pi}{8}\right)\cos\left(x-\dfrac{\pi}{8}\right)+2\cos^2\left(x-\dfrac{\pi}{8}\right) = \sqrt{3}+1$ có nghiệm là
	\choice
	{\True $x=\dfrac{5\pi}{24}+k\pi$, $x=\dfrac{3\pi}{8}+k\pi$ với $k\in\mathbb{Z}$}
	{$x=\dfrac{5\pi}{12}+k\pi$, $x=\dfrac{3\pi}{4}+k\pi$ với $k\in\mathbb{Z}$}
	{$x=\dfrac{5\pi}{4}+k\pi$, $x=\dfrac{5\pi}{16}+k\pi$ với $k\in\mathbb{Z}$}
	{$x=\dfrac{5\pi}{8}+k\pi$, $x=\dfrac{7\pi}{24}+k\pi$ với $k\in\mathbb{Z}$}
	\loigiai
	{
		Ta có
		\allowdisplaybreaks
		\begin{eqnarray*}
			&& 2\sqrt{3}\sin\left(x-\dfrac{\pi}{8}\right)\cos\left(x-\dfrac{\pi}{8}\right)+2\cos^2\left(x-\dfrac{\pi}{8}\right) = \sqrt{3}+1\\
			&\Leftrightarrow & \sqrt{3}\sin\left(2x-\dfrac{\pi}{4}\right)+1+\cos\left(2x-\dfrac{\pi}{4}\right) = \sqrt{3}+1\\
			&\Leftrightarrow & \sqrt{3}\sin\left(2x-\dfrac{\pi}{4}\right)+\cos\left(2x-\dfrac{\pi}{4}\right) = \sqrt{3}\\
			&\Leftrightarrow & \dfrac{\sqrt{3}}{2}\sin\left(2x-\dfrac{\pi}{4}\right)+\dfrac{1}{2}\cos\left(2x-\dfrac{\pi}{4}\right) = \dfrac{\sqrt{3}}{2}\\
			&\Leftrightarrow & \sin\left(2x-\dfrac{\pi}{12}\right) = \dfrac{\sqrt{3}}{2}\\
			&\Leftrightarrow & \left[\begin{aligned}&2x-\dfrac{\pi}{12}=\dfrac{\pi}{3}+k2\pi,k\in\mathbb{Z} \\&2x-\dfrac{\pi}{12}=\dfrac{2\pi}{3}+k2\pi,k\in\mathbb{Z}\end{aligned}\right.\\
			&\Leftrightarrow & \left[\begin{aligned}&x=\dfrac{5\pi}{24}+k\pi,k\in\mathbb{Z} \\&x=\dfrac{3\pi}{8}+k\pi,k\in\mathbb{Z}.\end{aligned}\right.
		\end{eqnarray*}
		Vậy phương trình đã cho có nghiệm $x=\dfrac{5\pi}{24}+k\pi$, $x=\dfrac{3\pi}{8}+k\pi$ với $k\in\mathbb{Z}$.
	}
\end{ex}
\begin{ex}%[Câu 49]%[1K1G4-5]
	Nghiệm dương nhỏ nhất của phương trình $\sin x+\sin 2x=\cos x+2\cos^2 x$ là
	\choice{$\dfrac{\pi}{6}$}{$\dfrac{\pi}{3}$}{$2\dfrac{\pi}{3}$}{\True$\dfrac{\pi}{4}$}
	\loigiai{\begin{eqnarray*}
			& &\sin x+\sin 2x=\cos x+2\cos^2 x\\
			&\Leftrightarrow & \sin x + 2\sin x\cos x = \cos x\left(2\cos x + 1\right)\\
			&\Leftrightarrow & \sin x\left(2\cos x + 1\right) = \cos x\left(2\cos x + 1\right)\\
			&\Leftrightarrow & \hoac{&\cos x = - \dfrac{1}{2}\\&\sin x = \cos x}\\
			&\Leftrightarrow & \hoac{&x = \pm \dfrac{2\pi}{3} + k2\pi\\&x = \dfrac{\pi}{4} + k\pi} \quad \left(k \in \mathbb{Z}\right).
		\end{eqnarray*}
		Khi đó nghiệm dương nhỏ nhất của phương trình là $x = \dfrac{\pi}{4}$.}
\end{ex}
\begin{ex}%[Câu 50]%[1K1G4-5]
	Số nghiệm của phương trình $ \dfrac{2\sin x-1}{2\sin^2x+\sin x-1}=2 $ trong khoảng $ \left(\dfrac{\pi}{2}; \dfrac{7\pi}{2}\right) $ là
	\choice
	{$ 5 $}
	{$ 2 $}
	{$ 4 $}
	{\True $ 3 $}
	\loigiai{
		Điều kiện $ 2\sin^2 x+\sin x-1\neq 0\Leftrightarrow\heva{& \sin x\neq -1\\ & \sin x\neq \dfrac{1}{2}} $.\\
		Khi đó phương trình đã cho tương đương với \begin{eqnarray*}
			& &2\sin x-1=4\sin^2 x+2\sin x-2\\
			& \Leftrightarrow & 4\sin^2 x=1\\
			&\Leftrightarrow &\hoac{& \sin x=\dfrac{1}{2}\ (\text{không thỏa mãn điều kiện})\\
				&\sin x=-\dfrac{1}{2}\ (\text{thỏa mãn điều kiện})}\\
			&\Leftrightarrow & \hoac{& x=-\dfrac{\pi}{6}+k2\pi\\ & x=\dfrac{7\pi}{6}+k2\pi},\ k\in\mathbb{Z}.
		\end{eqnarray*}
		\begin{itemize}
			\item Trường hợp $ x=-\dfrac{\pi}{6}+k2\pi $. Khi đó, $\begin{aligned}[t]
				x\in \left(\dfrac{\pi}{2}; \dfrac{7\pi}{2}\right)&\Leftrightarrow \dfrac{\pi}{2}< -\dfrac{\pi}{6}+k2\pi<\dfrac{7\pi}{2}\\
				&\Leftrightarrow \dfrac{2\pi}{3}<k2\pi<\dfrac{11\pi}{3}\\
				&\Leftrightarrow \dfrac{1}{3}<k<\dfrac{11}{6}\\
				&\Leftrightarrow k=1 \ (\text{vì}\; k\in\mathbb{Z}).
			\end{aligned} $
			\item Trường hợp $ x=\dfrac{7\pi}{6}+k2\pi $. Khi đó, $\begin{aligned}[t]
				x\in \left(\dfrac{\pi}{2}; \dfrac{7\pi}{2}\right)&\Leftrightarrow \dfrac{\pi}{2}< \dfrac{7\pi}{6}+k2\pi<\dfrac{7\pi}{2}\\
				&\Leftrightarrow -\dfrac{\pi}{3}<k2\pi<\dfrac{7\pi}{3}\\
				&\Leftrightarrow -\dfrac{1}{6}<k<\dfrac{7}{6}\\
				&\Leftrightarrow k\in\{0; 1\} \ (\text{vì}\; k\in\mathbb{Z}).
			\end{aligned} $
		\end{itemize}
		Vậy phương trình đã cho có tất cả 3 nghiệm thuộc khoảng $ \left(\dfrac{\pi}{2}; \dfrac{7\pi}{2}\right) $.
	}
\end{ex}
\Closesolutionfile{ans}
% \setcounter{deso}{0}
\begin{name}
	{\tenchude}
	{ĐỀ ÔN TẬP CHƯƠNG I}
	{LỚP TOÁN THẦY PHÁT}
	{\thoigian}
\end{name}
\TN
%Câu 1
\begin{ex}
	Cho góc lượng giác $\alpha $. Mệnh đề nào sau đây đúng?
	\choice
	{$\sin \left(-\alpha\right)=\sin \alpha $}
	{$\cos \left(-\alpha\right)=-\cos \alpha $}
	{$\tan \left(-\alpha\right)=\tan \alpha $}
	{\True $\cot \left(-\alpha\right)=-\cot \alpha $}
	\loigiai{
		Dựa vào tính chất của hai góc đối nhau nên $\cot \left(-\alpha\right)=-\cot \alpha $
	}
\end{ex}
%Câu 2
\begin{ex}
	Giá trị $\cos 75^\circ $ là :
	\choice
	{$\dfrac{\sqrt{6}+\sqrt{2}}{4}$}
	{$\dfrac{\sqrt{6}-\sqrt{2}}{2}$}
	{\True $\dfrac{\sqrt{6}-\sqrt{2}}{4}$}
	{$\dfrac{\sqrt{6}+\sqrt{2}}{2}$}
	\loigiai{
		Ta có $\cos 75^\circ =\cos \left(30^\circ+45^\circ \right)=\cos 30^\circ \cos 45^\circ -\sin 30^\circ \sin 45^\circ=\dfrac{\sqrt{6}-\sqrt{2}}{4}$
	}
\end{ex}
%Câu 3
\begin{ex}
	Cho $\sin \alpha =\dfrac{5}{13}$ với $\dfrac{\pi }{2}<\alpha <\pi $. Mệnh đề nào sau đây đúng?
	\choice
	{$\cos \alpha =\dfrac{12}{13}$}
	{$\cos \alpha =\dfrac{8}{13}$}
	{$\cos \alpha =-\dfrac{8}{13}$}
	{\True $\cos \alpha =-\dfrac{12}{13}$}
	\loigiai{
	Ta có $\cos \alpha =\pm \sqrt{1-{{\sin }^2}\alpha }=\pm \dfrac{12}{13}$. Do $\dfrac{\pi }{2}<\alpha <\pi $ nên $\cos \alpha =-\dfrac{12}{13}$
	}
\end{ex}
%Câu 4
\begin{ex}
	Cho các góc $\alpha $, $\beta $ thỏa mãn $\alpha ,\beta \in \left(\dfrac{\pi }{2};\pi\right)$ và $\sin \alpha =\dfrac{1}{3}$, $\cos \beta =-\dfrac{2}{3}$. Tính $\sin \left(\alpha +\beta\right)$.
	\choice
	{\True $\sin \left(\alpha +\beta\right)=-\dfrac{2+2\sqrt{10}}{9}$}
	{$\sin \left(\alpha +\beta\right)=\dfrac{2\sqrt{10}-2}{9}$}
	{$\sin \left(\alpha +\beta\right)=\dfrac{\sqrt{5}-4\sqrt{2}}{9}$}
	{$\sin \left(\alpha +\beta\right)=\dfrac{\sqrt{5}+4\sqrt{2}}{9}$}
	\loigiai{
	Do $\alpha ,\beta \in \left(\dfrac{\pi }{2};\pi\right)$ nên có: $\heva{& \cos \alpha <0 \\& \sin \beta >0}$.\\
	Ta có $\cos \alpha =-\ \sqrt{1-{{\sin }^2}\alpha }=-\ \sqrt{1-\dfrac{1}{9}}=-\ \dfrac{2\sqrt{2}}{3}$ và $\sin \beta =\sqrt{1-{{\cos }^2}\beta }=\sqrt{1-\dfrac{4}{9}}=\dfrac{\sqrt{5}}{3}$.\\
	Suy ra $\sin \left(\alpha +\beta\right)=\sin \alpha \cdot \cos \beta +\cos \alpha \cdot \sin \beta =\dfrac{1}{3} \cdot \left(-\dfrac{2}{3}\right)+\left(-\dfrac{2\sqrt{2}}{3}\right) \cdot \dfrac{\sqrt{5}}{3}=-\ \dfrac{2+2\sqrt{10}}{9}$.\\
	Vậy $\sin \left(\alpha +\beta\right)=-\ \dfrac{2+2\sqrt{10}}{9}$
	}
\end{ex}
%Câu 5
\begin{ex}
	Biết $\sin \alpha +\text{cos}\alpha =m$. Tính $P=\text{cos}\left(\alpha -\dfrac{\pi }{4}\right)$ theo $m$.
	\choice
	{$P=2m$}
	{$P=\dfrac{m}{2}$}
	{\True $P=\dfrac{m}{\sqrt{2}}$}
	{$P=m\sqrt{2}$}
	\loigiai{
	Ta có $P=\text{cos}\left(\alpha -\dfrac{\pi }{4}\right)=\text{cos}\alpha \cdot \cos \dfrac{\pi }{4}+\sin \alpha \sin \dfrac{\pi }{4}=\dfrac{1}{\sqrt{2}}\text{cos}\alpha +\dfrac{1}{\sqrt{2}}\sin \alpha $\\
	$\Rightarrow P=\dfrac{1}{\sqrt{2}}\left(\sin \alpha +\text{cos}\alpha\right)=\dfrac{m}{\sqrt{2}}$
	}
\end{ex}
%Câu 6
\begin{ex}
	Cho $x=\tan \alpha $. Tính $\sin 2\alpha $ theo $x$.
	\choice
	{$2x\sqrt{1+x^2}$}
	{$\dfrac{1-x^2}{1+x^2}$}
	{$\dfrac{2x}{1-x^2}$}
	{\True $\dfrac{2x}{1+x^2}$}
	\loigiai{
	Ta có $\sin 2\alpha =2\sin \alpha \cdot \cos\alpha =2\dfrac{\sin \alpha }{\cos\alpha }\cdot \cos^2\alpha =2\tan \alpha \cdot \dfrac{1}{1+{{\tan }^2}\alpha }=\dfrac{2x}{1+x^2}$
	}
\end{ex}
%Câu 7
\begin{ex}
	Tập xác định của hàm số $y=\cot x$ là
	\choice
	{$D=\mathbb{R}$}
	{$D=\mathbb{R}\backslash \left\{ k\dfrac{\pi }{2}\left| k\in \mathbb{Z} \right. \right\}$}
	{$D=\mathbb{R}\backslash \left\{ \pi +k\dfrac{\pi }{2}\left| k\in \mathbb{Z} \right. \right\}$}
	{\True $D=\mathbb{R}\backslash \left\{ k\pi \left| k\in \mathbb{Z} \right. \right\}$}
	\loigiai{
		Điều kiện: $\sin x\ne 0\Leftrightarrow x\ne k\pi \left(k\in \mathbb{Z}\right)$.\\
		Do đó, tập xác định của hàm số $y=\cot x$ là $D=\mathbb{R}\backslash \left\{ k\pi \left| k\in \mathbb{Z} \right. \right\}$
	}
\end{ex}
%Câu 8
\begin{ex}
	Trên khoảng $\left(-\pi ;\pi\right)$, hàm số $y=\sin x$ nghịch biến trên khoảng nào sau đây?
	\choice
	{$\left(-\pi ;0\right)$}
	{$\left(-\dfrac{\pi }{2};\dfrac{\pi }{2}\right)$}
	{$\left(0;\pi\right)$}
	{\True $\left(\dfrac{\pi }{2};\pi\right)$}
	\loigiai{
		Hàm số $y=\sin x$ nghịch biến trong khoảng $\left(\dfrac{\pi }{2};\pi\right)$
	}
\end{ex}
%Câu 9
\begin{ex}
	Hàm số $y={{\sin }^2}2x-{{\cos }^2}2x$ tuần hoàn với chu kỳ bằng
	\choice
	{$2\pi $}
	{$\pi $}
	{\True $\dfrac{\pi }{2}$}
	{$\dfrac{\pi }{4}$}
	\loigiai{
	Ta có $y={{\sin }^2}2x-{{\cos }^2}2x=-\cos 4x$. Vậy hàm số đã cho tuần hoàn với chu kỳ $\dfrac{2\pi }{4}=\dfrac{\pi }{2}$
	}
\end{ex}
%Câu 10
\begin{ex}
	Nghiệm của phương trình $2\sin x+1=0$ là
	\choice
	{$x=\dfrac{\pm \pi }{6}+k2\pi ,k\in \mathbb{Z}$}
	{$x=\dfrac{\pi }{6}+k2\pi ,k\in \mathbb{Z}$}
	{$x=\dfrac{7\pi }{6}+k2\pi ,k\in \mathbb{Z}$}
	{\True $\hoac{& x=\dfrac{-\pi }{6}+k2\pi \\& x=\dfrac{7\pi }{6}+k2\pi},k\in \mathbb{Z}$}
	\loigiai{
		Ta có: $2\sin x+1=0\Leftrightarrow \sin x=\dfrac{-1}{2}\Leftrightarrow \hoac{& x=\dfrac{-\pi }{6}+k2\pi \\& x=\dfrac{7\pi }{6}+k2\pi},k\in \mathbb{Z}$
	}
\end{ex}
%Câu 11
\begin{ex}
	Phương trình nào dưới đây vô nghiệm.
	\choice
	{$\cos x=\dfrac{1}{2}$}
	{\True $\sin x-\cos x=2$}
	{$\sin (5x+1)=1$}
	{$\sin x+\sqrt{3}\cos x=1$}
	\loigiai{
		Chú ý\\
		- $\left| \sin \alpha \right|\le 1,\forall \alpha \in \mathbb{R}$ và $\left| \cos \alpha \right|\le 1,\forall \alpha \in \mathbb{R}$ nên các phương trình ở đáp án A, C có nghiệm.\\
		- Phương trình $a\sin x+b\cos x=c$ có nghiệm khi $a^2+b^2\ge c^2$, ta kiểm tra được phương trình đáp án B vô nghiệm, đáp án D có nghiệm
	}
\end{ex}
%Câu 12
\begin{ex}
	Cho phương trình $2\tan x-3=\dfrac{-2}{\tan x+1}$. Gọi $S$ là tập hợp các nghiệm của phương trình thuộc khoảng $\left(0;\dfrac{\pi }{2}\right)$. Tổng các phần tử của $S$ là
	\choice
	{$0$}
	{$\dfrac{\pi }{3}$}
	{\True $\dfrac{\pi }{4}$}
	{$1$}
	\loigiai{
		Điều kiện : $\cos x\ne 0,\tan x\ne -1$.\\
		Vì $x\in \left(0;\dfrac{\pi }{2}\right)\Rightarrow \tan x>0$.\\
		Phương trình ban đầu tương đương\\
		$\begin{aligned}
				& \Leftrightarrow \left(2\tan x-3\right)\left(\tan x+1\right)=-2\Leftrightarrow 2{{\tan }^2}x-\tan x-3=-2 \\& \Leftrightarrow 2{{\tan }^2}x-\tan x-1=0 \end{aligned}$\\
		$\Leftrightarrow \hoac{& \tan x=1\begin{matrix}\\
					{} & (TM) \\\\
				\end{matrix} \\& \tan x=\dfrac{-1}{2}(L)}$\\
		+ Với $\tan x=1\Leftrightarrow x=\dfrac{\pi }{4}+k\pi ,k\in \mathbb{Z}$. Vì $x\in \left(0;\dfrac{\pi }{2}\right)$ nên $x=\dfrac{\pi }{4}$.\\
		Vậy $S=\left\{ \dfrac{\pi }{4} \right\}$ và tổng các phần tử của $S$ là $\dfrac{\pi }{4}$
	}
\end{ex}
\TNTF
%Câu 13
\begin{ex}
	Xét tính đúng sai của các mệnh đề sau:
	\choiceTF
	{${{\sin }^2}x=\dfrac{1+\sin 2x}{2}$}
	{\True Nếu $\cos \alpha =\dfrac{1}{3}$ thì $\cos 2\alpha =-\dfrac{7}{9}$}
	{\True Nếu $\sin x=\dfrac{3}{4}$ với $x\in \left(0;\dfrac{\pi }{2}\right)$ thì $\sin 2x=\dfrac{3\sqrt{7}}{8}$}
	{\True Cho $\cos \alpha =\dfrac{2}{3}$ với $\alpha \in \left(-\dfrac{\pi }{2};0\right)$ biết $\tan \left(\alpha +\dfrac{\pi }{4}\right)=a+b\sqrt{c}$, $c$ là số nguyên tố $\left(a,b,c\in \mathbb{Z},c\ge 0\right)$ Khi đó $a+b+c=0$}
	\loigiai{
	a) ${{\sin }^2}x=\dfrac{1-\cos 2x}{2}$\\
	b) $\cos 2\alpha =2{{\cos }^2}\alpha -1=2{{\left(\dfrac{1}{3}\right)}^2}-1=\dfrac{-7}{9}$\\
	c) Ta có ${{\cos }^2}x=1-{{\sin }^2}x=1-{{\left(\dfrac{3}{4}\right)}^2}=\dfrac{7}{16}$.\\
	Vì $x\in \left(0;\dfrac{\pi }{2}\right)$ nên $\cos x>0\Rightarrow \cos x=\dfrac{\sqrt{7}}{4}$ suy ra $\sin 2x=2\sin x \cdot \cos x=2\cdot \dfrac{\sqrt{7}}{4}\cdot \dfrac{3}{4}=\dfrac{3\sqrt{7}}{8}$\\
	d) Ta có ${{\tan }^2}\alpha =\dfrac{1}{{{\cos }^2}\alpha }-1=\dfrac{1}{{{\left(\dfrac{2}{3}\right)}^2}}-1=\dfrac{5}{4}$\\
	Vì $\alpha \in \left(-\dfrac{\pi }{2};0\right)$ nên $\tan \alpha <0\Rightarrow \tan \alpha =\dfrac{-\sqrt{5}}{2}$\\
	$\tan \left(\alpha +\dfrac{\pi }{4}\right)=\dfrac{\tan \alpha +\tan \dfrac{\pi }{4}}{1-\tan \alpha \cdot \tan \dfrac{\pi }{4}}=\dfrac{\dfrac{-\sqrt{5}}{2}+1}{1-\left(\dfrac{-\sqrt{5}}{2}\right) \cdot 1}=-9+4\sqrt{5}$\\
	Vậy $a=-9,b=4,c=5$ nên mệnh đề đúng
	}
\end{ex}
%Câu 14
\begin{ex}
	Biết $\cos x=\dfrac{1}{3}$ và $-\dfrac{\pi }{2}<x<0$. Khi đó: Các mệnh đề sau đúng hay sai?
	\choiceTF
	{\True $\sin \left(\dfrac{\pi }{2}-x\right)>0$}
    {$\sin 2x=\dfrac{4\sqrt{2}}{9}$}
	{\True $\cos \left(x+\dfrac{4\pi }{3}\right)=-\dfrac{1+3\sqrt{6}}{6}$}
	{\True $\sin x+\sin 3x=-\dfrac{8\sqrt{2}}{27}$}
	\loigiai{
	a) Ta có $\sin \left(\dfrac{\pi }{2}-x\right)=\cos x=\dfrac{1}{3}>0$\\
	b) Ta có ${{\sin }^2}x=1-{{\cos }^2}x=1-{{\left(\dfrac{1}{3}\right)}^2}=\dfrac{8}{9}\Rightarrow \sin x=\pm \dfrac{2\sqrt{2}}{3}$.\\
	Vì $-\dfrac{\pi }{2}<x<0$ nên $\sin x=-\dfrac{2\sqrt{2}}{3}$.\\
	Áp dụng công thức nhân đôi ta có: $\sin 2x=2\sin x\cos x=2 \cdot \left(-\dfrac{2\sqrt{2}}{3}\right) \cdot \dfrac{1}{3}=-\dfrac{4\sqrt{2}}{9}$\\
	c) $\cos \left(x+\dfrac{4\pi }{3}\right)=\cos x \cdot \cos \dfrac{4\pi }{3}-\sin x \cdot \sin \dfrac{4\pi }{3}=\dfrac{1}{3} \cdot \left(-\dfrac{1}{2}\right)-\left(-\dfrac{2\sqrt{2}}{2}\right) \cdot \left(-\dfrac{\sqrt{3}}{2}\right)=-\dfrac{1+3\sqrt{6}}{6}$\\
	d) Áp dụng công thức ta có:\\
	$\sin x+\sin 3x=2\sin 2x \cdot \cos x=2 \cdot \left(-\dfrac{4\sqrt{2}}{9}\right) \cdot \dfrac{1}{3}=-\dfrac{8\sqrt{2}}{27}$
	}
\end{ex}
%Câu 15
\begin{ex}
	Cho hàm số $f(x)=-2\sin \left(2x-\dfrac{\pi }{2}\right)+2025$. Các mệnh đề sau đúng hay sai?
	\choiceTF
	{\True Hàm số $f(x)$ có tập xác định là $\mathbb{R}$}
	{Hàm số $f(x)$ tuần hoàn với chu kì $T=2\pi $}
	{Hàm số $f(x)$ không chẵn, không lẻ}
	{\True Hàm số $f(x)$ đạt giá trị lớn nhất tại $x=k\pi ,k\in \mathbb{Z}$}
	\loigiai{
		a). Vì tập xác định của hàm $\sin $ là $\mathbb{R}$ nên hàm số $f(x)$ có tập xác định là $\mathbb{R}$.\\
		b). Ta có $-2\sin \left(2x-\dfrac{\pi }{2}\right)+2025=2\sin \left(\dfrac{\pi }{2}-2x\right)+2025=2\cos 2x+2025$.\\
		Do đó $f(x)=2\cos 2x+2025$ nên hàm số $f(x)$ tuần hoàn với chu kì $T=\dfrac{2\pi }{2}=\pi $.\\
		c) Ta có $\forall x\in \mathbb{R},-x\in \mathbb{R}$ và $f(-x)=2\cos (-2x)+2025=2\cos 2x+2025=f(x)$ nên hàm số $f(x)$ là hàm số chẵn.\\
		d) Ta có $-2\le 2\cos 2x\le 2,\forall x\in \mathbb{R}$ hay $2023\le 2\cos 2x+2025\le 2027,\forall x\in \mathbb{R}$.\\
		Do đó $f(x)=2027\Leftrightarrow \cos 2x=1\Leftrightarrow x=k\pi ,k\in \mathbb{Z}$.\\
		Vậy hàm số $f(x)$ đạt giá trị lớn nhất tại $x=k\pi ,k\in \mathbb{Z}$
	}
\end{ex}
%Câu 16
\begin{ex}
	Cho hàm số $f(x)=\dfrac{1}{{{\cos }^2}x}+\dfrac{1}{{{\sin }^2}x}$. Xét tính đúng sai của các mệnh đề sau
	\choiceTF
	{\True Hàm số đã cho là hàm số tuần hoàn}
	{\True Hàm số đã cho là hàm số chẵn}
	{Tập xác định của hàm số là $D=\mathbb{R}\backslash \left\{ \dfrac{\pi }{2}+k\pi ,k\in \mathbb{Z} \right\}$}
	{\True Giá trị nhỏ nhất của hàm số là 4}
	\loigiai{
		a) Hàm số tuần hoàn do hai hàm $y=\operatorname{sinx}$ và $y=\cos x$ cùng tuần hoàn với chu kì $2\pi $.\\
		b) Ta có $f(-x)=\dfrac{1}{{{\cos }^2}(-x)}+\dfrac{1}{{{\sin }^2}(-x)}=\dfrac{1}{{{\left(\operatorname{cosx}\right)}^2}}+\dfrac{1}{{{\left(-\sin x\right)}^2}}=\dfrac{1}{{{\cos }^2}x}+\dfrac{1}{{{\sin }^2}x}=f(x)$.\\
		Do đó hàm số đã cho là hàm số chẵn\\
		c) Hàm số xác định khi $\heva{& \operatorname{sinx}\ne 0 \\& \operatorname{cosx}\ne 0}\Leftrightarrow \sin 2x\ne 0\Leftrightarrow 2x\ne k\pi \Leftrightarrow x\ne \dfrac{k\pi }{2},k\in \mathbb{Z}$\\
		Tập xác định của hàm số là $D=\mathbb{R}\backslash \left\{ \dfrac{k\pi }{2},k\in \mathbb{Z} \right\}$.\\
		d) Khi $x\ne \dfrac{k\pi }{2},k\in \mathbb{Z}$ ta có\\
		$f(x)=\dfrac{1}{{{\cos }^2}x}+\dfrac{1}{{{\sin }^2}x}\ge 2\sqrt{\dfrac{1}{{{\cos }^2}x} \cdot \dfrac{1}{{{\sin }^2}x}}=2\sqrt{\dfrac{4}{{{\sin }^2}2x}}=\dfrac{4}{\left| \sin 2x \right|}\ge \dfrac{4}{1}=4$.\\
		Nên giá trị nhỏ nhất của hàm số là 4
	}
\end{ex}
\TNSA
%Câu 19
\begin{ex}
    Tìm tập giá trị của các hàm số $y=\sqrt{2+\cos x}-5$ là đoạn $[a;b]$. Giá trị $a+b$ (làm tròn đến hàng phần chục) là
    \shortans{-7,3}
    \loigiai{
    Vì $\cos x\ge -1\Leftrightarrow 2+\cos x\ge 1>0,\forall x\in \mathbb{R}$ nên tập xác định của hàm số là $D=\mathbb{R}$.\\
    $\forall x\in \mathbb{R}$, ta có:
    \begin{eqnarray*}
        & & -1\le \cos x\le 1 \\
        & \Leftrightarrow & 1\le 2+\cos x\le 3 \\
        & \Leftrightarrow & 1\le \sqrt{2+\cos x}\le \sqrt{3} \\
        & \Leftrightarrow & -4\le \sqrt{2+\cos x}\,-5\le \sqrt{3}-5
    \end{eqnarray*}
    Vậy tập giá trị của hàm số là $T=\left[-4;\sqrt{3}-5\right]$. Suy ra $a+b \approx -7,3$.
    }    
\end{ex}
%Câu 18
\begin{ex}
	Tổng số giờ ban ngày của ngày thứ $x$ trong một năm không nhuận được tính bởi công thức
	$g(x)=3\sin (0,0172x-1,376)+12$.
	Trong đó $x$ đại diện cho ngày trong năm, $1\le x\le 365$. Ngày $\overline{ab}$ tháng $\overline{cd}$ có số giờ ban ngày dài nhất. Số $\overline{abcd}$ bằng
	\shortans{2006}
	\loigiai{
		Ta có $-1\le \sin (0,0172x-1,376)\le 1$\\
		$-3\le 3\sin (0,0172x-1,376)\le 3$\\
		$9\le 3\sin (0,0172x-1,376)+12\le 15$\\
		Suy ra $9\le g(x)\le 15$\\
		Do đó, số giờ ban ngày dài nhất trong một ngày là 15 giờ.\\
		Ta có phương trình $3\sin (0,0172x-1,376)+12=15$\\
		$\sin (0,0172x-1,376)=1$\\
		$x\approx 171{,}3$\\
		Vậy vào khoảng ngày thứ 171 trong năm (ngày 20 tháng 6) thì số giờ ban ngày dài nhất
	}
\end{ex}
%Câu 19
\begin{ex}
	Hai thành phố có cùng kinh độ. Vĩ tuyến của thành phố A là $10^\circ $ Bắc và vĩ tuyến của thành phố B là $40^\circ $ Bắc. Giả sử bán kính trái đất là 3960 dặm. Tìm khoảng cách giữa hai thành phố (làm tròn đến chữ số hàng đơn vị)
	\shortans{2073}
	\loigiai{
		Khoảng cách từ điểm trên đường xích đạo đến thành phố B ở cùng kinh độ là $3960 \cdot \dfrac{40}{180} \cdot \pi =880\pi $ (dặm)\\
		Khoảng cách từ điểm trên đường xích đạo đến thành phố A ở cùng kinh độ là $3960 \cdot \dfrac{10}{180}\pi =220\pi $ (dặm)\\
		Khoảng cách giữa hai thành phố A và B là $880\pi -220\pi =660\pi \approx 2073$ (dặm)
	}
\end{ex}
%Câu 20
\begin{ex}
	Giả sử vận tốc $v$ (tính bằng lít/ giây) của luồng khí trong một chu kì hô hấp (tức là thời gian từ lúc bắt đầu của một nhịp thở đến khi bắt đầu của nhịp thở tiếp theo) của một người nào đó ở trạng thái nghỉ ngơi được cho bởi công thức $v=0{,}85\sin \dfrac{\pi t}{3}$, trong đó $t$ là thời gian (tính bằng giây). Biết rằng quá trình hít vào xảy ra khi $v>0$ và quá trình thở ra xảy ra khi $v<0$. Trong khoảng thời gian từ 5 đến 10 giây, khoảng thời điểm sau $a$ giây đến trước $b$ giây thì người đó hít vào. Tính $\sqrt{a+b}$ (làm tròn đến hàng phần trăm).
	\shortans{3,87}
	\loigiai{
		+) Vì quá trình hít vào xảy ra khi $v>0$ nên ta có\\
		$0{,}85\sin \dfrac{\pi t}{3}>0\Leftrightarrow \sin \dfrac{\pi t}{3}>0\Leftrightarrow \dfrac{\pi t}{3}\in \left(k2\pi ;\pi +k2\pi\right)(k\in \mathbb{Z})$\\
		$\Leftrightarrow t\in \left(6k;3+6k\right)\,\left(k\in \mathbb{Z}\right)$\\
		+) Vì $t\in [5;10]$ nên $k=1$ suy ra $t\in \left(6;9\right)$.\\
		Trong khoảng thời gian từ 5 đến 10 giây, khoảng thời điểm sau $6$ giây đến trước $9$ giây thì người đó hít vào nên $\sqrt{a+b}=\sqrt{15}\approx 3,87$.
	}
\end{ex}
%Câu 21
\begin{ex}
	Nghiệm phương trình lượng giác $\sqrt{3}\sin x-\cos x=0$ có dạng $x=\dfrac{\pi }{a}+k \cdot b\pi $ ($a$, $b$, $k\in \mathbb{Z}$, $a\ne 0$). Tính $(a+b)^4$.
	\shortans{2401}
	\loigiai{
		Phương trình tương đương\\
		$\dfrac{\sqrt{3}}{2}\sin x-\dfrac{1}{2}\cos x=0$\\
		$\Leftrightarrow \sin x\cos \dfrac{\pi }{6}-\cos x\sin \dfrac{\pi }{6}=0$\\
		$\Leftrightarrow \sin \left(x-\dfrac{\pi }{6}\right)=0$\\
		$\Leftrightarrow x-\dfrac{\pi }{6}=k\pi $ ($k\in \mathbb{Z}$)\\
		$\Leftrightarrow x=\dfrac{\pi }{6}+k\pi $.\\
		Phương trình có nghiệm là: $x=\dfrac{\pi }{6}+k\pi $ ($k\in \mathbb{Z}$).\\
		Suy ra $a=6$; $b=1$. Vậy $(a+b)^4=7^4=2401$.
	}
\end{ex}
%Câu 22
\begin{ex}
	Một vật $M$ được gắn vào đầu lò xo và dao động quanh vị trí cân bằng, toạ độ $x$ (đơn vị: cm) tại thời điểm $t$ (giây) được tính bởi công thức $x=8{,}6\sin \left(8t+\dfrac{\pi }{2}\right)$. Có $n$ thời điểm trong khoảng 2 giây đầu tiên thì $s=4{,}3$ cm. Giá trị $\sqrt[3]{n}$ (làm tròn đến hàng phần trăm)
	\shortans{1,71}
	\loigiai{
	Khi $x=4{,}3$ thì $8{,}6\sin \left(8t+\dfrac{\pi }{2}\right)=4{,}3\Rightarrow \sin \left(8t+\dfrac{\pi }{2}\right)=\dfrac{1}{2}$\\
	$\Leftrightarrow \hoac{&8t+\dfrac{\pi }{2}=\dfrac{\pi }{6}+k2\pi  \\
			&8t+\dfrac{\pi }{2}=\dfrac{5\pi }{6}+l2\pi}(k,l\in \mathbb{Z})
            \Leftrightarrow \hoac{& t=-\dfrac{\pi }{24}+k\dfrac{\pi }{4} \\
            & t=\dfrac{\pi }{24}+l\dfrac{\pi }{4} }(k,l\in \mathbb{Z})$.\\
	Vì $t\in (0;2)$ nên $\heva{& 0<-\dfrac{\pi }{24}+k\dfrac{\pi }{4}<2 \\ & 0<\dfrac{\pi }{24}+l\dfrac{\pi }{4}<2} \Leftrightarrow \heva{& \dfrac{1}{6}<k<\dfrac{8}{\pi }+\dfrac{1}{6}  \\ & -\dfrac{1}{6}<l<\dfrac{8}{\pi }-\dfrac{1}{6}}$\\
	Mà $k,l\in \mathbb{Z}$ nên $k\in \left\{ 1;2 \right\}$; $l\in \left\{ 0;1;2 \right\}$.\\
	Vậy có $5$ thời điểm thỏa mãn đề bài nên $\sqrt[3]{n}\approx 1,71$
	}
\end{ex}

% \begin{name}
	{\tenchude}
	{ĐỀ ÔN TẬP CHƯƠNG I}
	{LỚP TOÁN THẦY PHÁT}
	{\thoigian}
\end{name}
\TN
\setcounter{ex}{0}
\Opensolutionfile{ans}[ans/ans-TN-C1-De1]
\TN
%Câu 1
\begin{ex}
	Rút gọn biểu thức $M=\cos 2x \cdot \cos x+\sin 2x \cdot \sin x$ ta được kết quả là:
	\choice
	{\True $M=\cos x$}
	{$M=\cos 3x$}
	{$M=\sin x$}
	{$M=\sin 3x$}
	\loigiai{
		Ta có: $M=\cos 2x \cdot \cos x+\sin 2x \cdot \sin x=\cos (2x-x)=\cos x$
	}
\end{ex}
%Câu 2
\begin{ex}
	Đẳng thức nào không đúng với mọi $x$?
	\choice
	{$\cos^2 3x=\dfrac{1+\cos 6x}{2}$}
	{$\cos 2x=1-2\sin^2x$}
	{$\sin 2x=2\sin x\cos x$}
	{\True $\sin^2 2x=\dfrac{1+\cos 4x}{2}$}
	\loigiai{
		Ta có $\sin^2 2x=\dfrac{1-\cos 4x}{2}$
	}
\end{ex}
%Câu 3
\begin{ex}
	Góc có số đo $\dfrac{\pi }{24}$ đổi sang độ bằng
	\choice
	{$7^\circ $}
	{\True $7^\circ 3{0}'$}
	{$8^\circ $}
	{$8^\circ 3{0}'$}
	\loigiai{
		Ta có: $\dfrac{\pi }{24}=\dfrac{180^\circ }{24}=7^\circ 30'$
	}
\end{ex}
%Câu 4
\begin{ex}
	Một đường tròn có đường kính là $50$ (cm). Độ dài của cung tròn trên đường tròn có số đo là $\dfrac{\pi }{4}$ bằng (làm tròn đến hàng đơn vị)
	\choice
	{$40$ (cm)}
	{$39$ (cm)}
	{$19$ (cm)}
	{\True $20$ (cm)}
	\loigiai{
		Độ dài của cung tròn $l=\alpha \cdot R=\dfrac{\pi }{4} \cdot 25=\dfrac{25}{4}\pi \approx 20$ (cm)
	}
\end{ex}
%Câu 5
\begin{ex}
	Chọn phát biểu đúng:
	\choice
	{Các hàm số $y=\sin x$, $y=\cos x$, $y=\cot x$ đều là hàm số chẵn}
	{Các hàm số $y=\sin x$, $y=\cos x$, $y=\cot x$ đều là hàm số lẻ}
	{Các hàm số $y=\sin x$, $y=\cot x$, $y=\tan x$ đều là hàm số chẵn}
	{\True Các hàm số $y=\sin x$, $y=\cot x$, $y=\tan x$ đều là hàm số lẻ}
	\loigiai{
		Hàm số $y=\cos x$ là hàm số chẵn, hàm số $y=\sin x$, $y=\cot x$, $y=\tan x$ là các hàm số lẻ
	}
\end{ex}
%Câu 6
\begin{ex}
	Nếu $\sin x+\cos x=\dfrac{1}{2}$ thì $\sin 2x$ bằng
	\choice
	{$\dfrac{3}{4}$}
	{$\dfrac{3}{8}$}
	{$\dfrac{\sqrt{2}}{2}$}
	{\True $\dfrac{-3}{4}$}
	\loigiai{
	Do $\sin x+\cos x=\dfrac{1}{2}\Rightarrow \dfrac{1}{4}={{\left(\sin x+\cos x\right)}^2}={{\left(\sin x\right)}^2}+{{\left(\text{cosx}\right)}^2}+2\sin x \cdot \cos x$\\
	$\Rightarrow \dfrac{1}{4}=1+\sin 2x\Rightarrow \sin 2x =\dfrac{-3}{4}$
	}
\end{ex}
%Câu 7
\begin{ex}
	Một con lắc lò xo sau khi được kéo xuống dưới vị trí cân bằng $4$ cm và thả ra thì nó dao động điều hòa với phương trình: $y=-4\cos 8t$ (cm). Biên độ $A$ cm và chu kỳ $T$ của dao động là
	\choice
	{\True $A=4$ cm, $T=\dfrac{\pi }{4}$}
	{$A=4$ cm, $T=\dfrac{\pi }{2}$}
	{$A=8$ cm, $T=\dfrac{\pi }{4}$}
	{$A=4$ cm, $T=2\pi $}
	\loigiai{
		Biên độ của dao động là: $A=|-4|=4$ (cm).\\
		Chu kỳ của dao động là:$T=\dfrac{2\pi }{|8|}=\dfrac{\pi }{4}$
	}
\end{ex}
%Câu 8
\begin{ex}
	Hãy tìm tập tất cả các giá trị của $m$ để phương trình $\left| \sin x \right|=m$ có nghiệm?
	\choice
	{$-1\le m\le 1$}
	{$-1\le m\le 0$}
	{$-1<m<0$}
	{\True $0\le m\le 1$}
	\loigiai{
		Vì $0<= |\sin x|<=1, \forall x \in \mathbb{R}$ nên phương trình $\left| \sin x \right|=m$ có nghiệm khi và chỉ khi $0\le m\le 1$.
	}
\end{ex}
%Câu 9
\begin{ex}
	Nghiệm của phương trình $2\sin \left(4x-\dfrac{\pi }{3}\right)-1=0$ là:
	\choice
	{$x=\pi +k2\pi ;x=k\dfrac{\pi }{2}\ (k \in \mathbb{Z})$}
	{$x=\dfrac{\pi }{8}+k\dfrac{\pi }{2};x=\dfrac{7\pi }{24}+k\dfrac{\pi }{2}\ (k \in \mathbb{Z})$}
	{$x=k2\pi ;x=\dfrac{\pi }{2}+k2\pi\ (k \in \mathbb{Z})$}
	{$x=k\pi ;x=\pi +k2\pi\ (k \in \mathbb{Z})$}
	\loigiai{
		$2\sin \left(4x-\dfrac{\pi }{3}\right)-1=0\Leftrightarrow \sin \left(4x-\dfrac{\pi }{3}\right)=\dfrac{1}{2}\Leftrightarrow \hoac{& 4x-\dfrac{\pi }{3}=\dfrac{\pi }{6}+k2\pi \\& 4x-\dfrac{\pi }{3}=\pi -\dfrac{\pi }{6}+k2\pi}\Leftrightarrow \hoac{& x=\dfrac{\pi }{8}+k\dfrac{\pi }{2} \\& x=\dfrac{7\pi }{24}+k\dfrac{\pi }{2}}\left(k\in \mathbb{Z}\right)$
	}
\end{ex}
%Câu 10
\begin{ex}
	Biết $\sin \left(\alpha +\dfrac{3\pi }{2}\right)+\cos \left(\alpha +\dfrac{3\pi }{2}\right)=\sqrt{2}$. Tính $\sin \left(\alpha +\pi\right)-2\cos \left(\alpha -\pi\right)$.
	\choice
	{$\dfrac{3}{\sqrt{2}}$}
	{\True $-\dfrac{3}{\sqrt{2}}$}
	{$-\dfrac{1}{\sqrt{2}}$}
	{$\dfrac{1}{\sqrt{2}}$}
	\loigiai{
	Ta có $\sin \left(\alpha +\dfrac{3\pi }{2}\right)=\sin \left(\alpha +2\pi -\dfrac{\pi }{2}\right)=\sin \left(\alpha -\dfrac{\pi }{2}\right)=-\sin \left(\dfrac{\pi }{2}-\alpha\right)=-\cos \alpha $.\\
	$\cos \left(\alpha +\dfrac{3\pi }{2}\right)=\cos \left(\alpha +2\pi -\dfrac{\pi }{2}\right)=\cos \left(\alpha -\dfrac{\pi }{2}\right)=\cos \left(\dfrac{\pi }{2}-\alpha\right)=\sin \alpha $.\\
	Suy ra $\sin \alpha -\cos \alpha =\sqrt{2}\Rightarrow \sin \alpha =\cos \alpha +\sqrt{2}$.\\
	Vì ${{\sin }^2}\alpha +{{\cos }^2}\alpha =1\Rightarrow 2{{\cos }^2}\alpha +2\sqrt{2}\cos \alpha +2=1$\\
	$\Leftrightarrow 2{{\cos }^2}\alpha +2\sqrt{2}\cos \alpha +1=0\Leftrightarrow \cos \alpha =-\dfrac{1}{\sqrt{2}}\Rightarrow \sin \alpha =\dfrac{1}{\sqrt{2}}$.\\
	Do đó $\sin \left(\alpha +\pi\right)-2\cos \left(\alpha -\pi\right)=-\sin \alpha +2\cos \alpha =-\dfrac{3}{\sqrt{2}}$
	}
\end{ex}
%Câu 11
\begin{ex}
	Hằng ngày mực nước của con kênh lên xuống theo thủy triều. Độ sâu $h$(mét) của mực nước trong kênh được tính tại thời điểm $t$ (giờ) trong một ngày bởi công thức $h=3\cos \left(\dfrac{\pi t}{7=8}+\dfrac{\pi }{4}\right)+12$. Mực nước của kênh cao nhất khi:
	\choice
	{$t=13$(giờ)}
	{\True $t=14$(giờ)}
	{$t=15$(giờ)}
	{$t=16$(giờ)}
	\loigiai{
		Mực nước của kênh cao nhất khi $h$ lớn nhất\\
		$\Leftrightarrow \cos \left(\dfrac{\pi t}{8}+\dfrac{\pi }{4}\right)=1\Leftrightarrow \dfrac{\pi t}{8}+\dfrac{\pi }{4}=k2\pi $ với $0<t\le 24$ và $k\in \mathbb{Z}$.\\
		Lần lượt thay các đáp án, ta được đáp án B thỏa mãn.\\
		Vì với $t=14$ thì $\dfrac{\pi t}{8}+\dfrac{\pi }{4}=2\pi $ (đúng với $k=1\in \mathbb{Z}$)
	}
\end{ex}
%Câu 12
\begin{ex}
	Số giờ có ánh sáng mặt trời của một thành phố A ở vĩ độ ${{40}^{\text{o}}}$ bắc trong ngày thứ t của một năm không nhuận được cho bởi hàm số $d(t)=3\sin \left[\dfrac{\pi }{180}(t-80)\right]+12$ với $t\in \mathbb{Z}$ và $0<t\le 365$. Vào ngày nào trong năm thì thành phố A có nhiều giờ có ánh sáng mặt trời nhất?
	\choice
	{\True 170}
	{171}
	{172}
	{173}
	\loigiai{
		Ta có $d(t)=3\sin \left[\dfrac{\pi }{180}(t-80)\right]+12\le 3 \cdot 1+12=15$.\\
		Vậy thành phố A có nhiều giờ có ánh sáng mặt trời nhất khi $\sin \left[\dfrac{\pi }{180}(t-80)\right]=1\Leftrightarrow \dfrac{\pi }{180}(t-80)=\dfrac{\pi }{2}+k2\pi \Leftrightarrow t=170+360k (k\in \mathbb{Z})$.\\
		Vì $0<t\le 365$ nên $0<170+360k\le 365\Leftrightarrow -\dfrac{17}{36}<k\le \dfrac{39}{72}\Rightarrow k=0\Rightarrow t=170$.
	}
\end{ex}

\TNTF
%Câu 13
\begin{ex}
	Cho phương trình $\sin x=a$ (1).
	\choiceTF
	{\True Nếu $a>1$ thì phương trình (1) vô nghiệm}
	{Nếu $a=1$ thì phương trình (1) có nghiệm $\alpha =\dfrac{\pi }{2}+k\pi ,\left(k\in \mathbb{Z}\right)$}
	{\True Nếu $-1\le a\le 1$ thì phương trình (1) có nghiệm $\hoac{& x=\alpha +k2\pi \\ & x=\pi -\alpha +k2\pi} \left(k\in \mathbb{Z}\right)$}
	{Phương trình (1) luôn có hai điểm biểu diễn nghiệm trên đường tròn lượng giác}
	\loigiai{
		Nếu $a=1\Rightarrow \sin \alpha =1\Leftrightarrow \alpha =\dfrac{\pi }{2}+k2\pi ,\left(k\in \mathbb{Z}\right)$
	}
\end{ex}
%Câu 14
\begin{ex}
	Các mệnh đề sau đúng hay sai?
	\choiceTF
	{\True Hàm số $y=\sin \sqrt{x+4}$ có tập xác định là $D=\left[-4;+\infty\right)$}
	{Hàm số $y=\cot \left(\dfrac{\pi }{2}+x\right)$ có tập xác định là $D=\mathbb{R}$}
	{\True Hàm số $y=\sqrt{3-2\cos x}$ có tập xác định là $D=\mathbb{R}$}
	{Hàm số $y=\dfrac{1-3\cos x}{\sin x}$ có tập xác định là $D=\mathbb{R}\backslash \left\{ k\dfrac{\pi }{2},k\in \mathbb{Z} \right\}$}
	\loigiai{
	a) Hàm số xác định khi và chỉ khi $x+4\ge 0\Leftrightarrow x\ge -4$.\\
	Vậy tập xác định của hàm số là $D=\left[-4;+\infty\right)$.\\
	b) Hàm số xác định khi và chỉ khi $\sin \left(x+\dfrac{\pi }{2}\right)\ne 0\Leftrightarrow x+\dfrac{\pi }{2}\ne k\pi \Leftrightarrow x\ne -\dfrac{\pi }{2}+k\pi ;k\in \mathbb{Z}$.\\
	Vậy tập xác định của hàm số là $D=\mathbb{R}\backslash \left\{ -\dfrac{\pi }{2}+k\pi ;k\in \mathbb{Z} \right\}$.\\
	c) Hàm số xác định khi $3-2\cos x\ge 0\Leftrightarrow \cos x\le \dfrac{3}{2}$ (đúng $\forall x\in \mathbb{R}$), vì $-1\le \cos x\le 1,\forall x\in \mathbb{R}$.\\
	Vậy tập xác định của hàm là $D=\mathbb{R}$.\\
	d) Hàm số xác định khi và chỉ khi $\sin x\ne 0\Leftrightarrow x\ne k\pi \left(k\in \mathbb{Z}\right)$.\\
	Vậy tập xác định của hàm số là $D=\mathbb{R}\backslash \left\{ k\pi ,k\in \mathbb{Z} \right\}$
	}
\end{ex}
%Câu 15
\begin{ex}
	Hằng ngày mực nước của con kênh lên xuống theo thủy triều. Độ sâu $h$ (mét) của mực nước trong kênh tính theo thời gian $t$ (giờ) được cho bởi công thức $h(t)=3\cos \left(\dfrac{\pi t}{6}+\dfrac{\pi }{4}\right)+14$.
	\choiceTF
	{Công thức tuần hoàn với chu kì $T=2\pi $}
	{\True Chiều sâu của mực nước thấp nhất là $11 \text{m}$}
	{Chiều sâu của mực nước cao nhất là $14 \text{m}$}
	{\True Thời gian để mực nước cao nhất là $t=9$}
	\loigiai{
		a) Công thức có dạng $y=\cos (ax+b)$ tuần hoàn với chu kì $T=\dfrac{2\pi }{|a|}$ nên chu kì cần tìm là $T=\dfrac{2\pi }{\left| \dfrac{\pi }{6} \right|}=12$.\\
		b) Ta có $\forall t\colon -1\le \cos \left(\dfrac{\pi t}{6}+\dfrac{\pi }{4}\right)\le 1\Leftrightarrow -3\le 3\cos \left(\dfrac{\pi t}{6}+\dfrac{\pi }{4}\right)\le 3\Leftrightarrow 11\le 3\cos \left(\dfrac{\pi t}{6}+\dfrac{\pi }{4}\right)+14\le 17\Leftrightarrow 11\le h\le 17$. Vậy chiều sâu của mực nước thấp nhất là $11 \text{m}$.\\
		c) Ta có $\forall t\colon -1\le \cos \left(\dfrac{\pi t}{6}+\dfrac{\pi }{4}\right)\le 1\Leftrightarrow -3\le 3\cos \left(\dfrac{\pi t}{6}+\dfrac{\pi }{4}\right)\le 3\Leftrightarrow 11\le 3\cos \left(\dfrac{\pi t}{6}+\dfrac{\pi }{4}\right)+14\le 17\Leftrightarrow 11\le h\le 17$. Chiều sâu của mực nước cao nhất là $17 \text{m}$.\\
		d) Ta có $\forall t\colon -1\le \cos \left(\dfrac{\pi t}{6}+\dfrac{\pi }{4}\right)\le 1\Leftrightarrow -3\le 3\cos \left(\dfrac{\pi t}{6}+\dfrac{\pi }{4}\right)\le 3\Leftrightarrow 11\le 3\cos \left(\dfrac{\pi t}{6}+\dfrac{\pi }{4}\right)+14\le 17\Leftrightarrow 11\le h\le 17$. Chiều sâu của mực nước cao nhất là $17 \text{m}$.\\
		Max $h=17\Leftrightarrow \cos \left(\dfrac{\pi t}{6}+\dfrac{\pi }{4}\right)=1\Leftrightarrow \dfrac{\pi t}{6}+\dfrac{\pi }{4}=k2\pi \Leftrightarrow t=-3+12k,k\in \mathbb{Z}$.\\
		Vì thời gian không âm và $k\in \mathbb{Z}$ nên ta chọn $t=1$. Vậy thời gian ngắn nhất $t=-3+12=9$
	}
\end{ex}
%Câu 16
\begin{ex}
	Cho phương trình $\left(2\cos x-1\right)\left(\sin 2x-m\right)=0$ (1).
	\choiceTF
	{\True $x=\dfrac{7\pi }{3}$ là một nghiệm của phương trình $(1)$}
	{Khi $m=2$ thì phương trình $(1)\Leftrightarrow \hoac{& x=\pm\dfrac{\pi }{3}+k2\pi \\& x=\dfrac{\pi }{2}+l2\pi} (k,l \in \mathbb{Z})$}
	{\True Khi $m=1$ thì tập nghiệm của phương trình $(1)$ có tất cả 4 điểm biểu diễn trên đường tròn lượng giác}
	{Chỉ tìm được một giá trị của $m$ để phương trình $(1)$ có đúng hai nghiệm thuộc $\left(-\dfrac{\pi }{4};\dfrac{3\pi }{4}\right]$}
			\loigiai{
			Ta có $\left(2\cos x-1\right)\left(\sin 2x-m\right)=0\Leftrightarrow \hoac{& \cos x=\dfrac{1}{2} \\& \sin 2x=m}\Leftrightarrow \hoac{& x=\dfrac{\pi }{3}+k2\pi \\& x=-\dfrac{\pi }{3}+k2\pi \\& \sin 2x=m}$\\
			a) Thay $x=\dfrac{7\pi }{3}$ phương trình $(1)$ ta thấy thỏa mãn nên $x=\dfrac{7\pi }{3}$ là một nghiệm của phương trình $(1)$.\\
			b) Khi $m=2$ thì phương trình $(1)\Leftrightarrow \hoac{& x=\dfrac{\pi }{3}+k2\pi \\& x=-\dfrac{\pi }{3}+k2\pi} (k \in \mathbb{Z})$\\
			c) Khi $m=1$ phương trình $(1)\Leftrightarrow \hoac{& x=\dfrac{\pi }{3}+k2\pi \\& x=-\dfrac{\pi }{3}+k2\pi \\& \sin 2x=1}\Leftrightarrow \hoac{& x=\dfrac{\pi }{3}+k2\pi \\& x=-\dfrac{\pi }{3}+k2\pi \\& x=\dfrac{\pi }{4}+l\pi}$.\\
			Do đó tập nghiệm của phương trình $(1)$ có tất cả $4$ điểm biểu diễn trên đường tròn lượng giác.\\
			d) Do phương trình $(2)$ có một nghiệm $x=\dfrac{\pi }{3}$ thuộc $\left(-\dfrac{\pi }{4};\dfrac{3\pi }{4}\right]$.\\
			Do đó để phương trình $(1)$ có đúng hai nghiệm thuộc $\left(-\dfrac{\pi }{4};\dfrac{3\pi }{4}\right]$ thì phương trình $\sin 2x=m$ có 1 nghiệm thuộc $\left(-\dfrac{\pi }{4};\dfrac{3\pi }{4}\right]$ khác $\dfrac{\pi }{3}$ (*)\\
			Ta có $x\in \left(-\dfrac{\pi }{4};\dfrac{3\pi }{4}\right]\Rightarrow 2x\in \left(-\dfrac{\pi }{2};\dfrac{3\pi }{2}\right]$ hay $2x\in \left[0;2\pi\right]$\\
			Từ (*) suy ra $m=1$ hoặc $m=-1$\\
	}
\end{ex}

\TNSA
%Câu 17
\begin{ex}
	Cho góc $\alpha $ thỏa mãn $\sin \alpha =\dfrac{1}{5}$. Khi đó giá trị biểu thức $P={{\cos }^2}2x+{{\cos }^2}x$ bằng $\dfrac{a}{b}$. Tính $a+b$. Biết rằng phân số $\dfrac{a}{b}$ là phân số tối giản
	\shortans{1754}
	\loigiai{
		Biến đổi biểu thức $P$ rồi thay giá trị $\sin \alpha =\dfrac{1}{5}$ vào $P$, ta được:\\
		$\begin{aligned}
				& P={{\cos }^2}2x+{{\cos }^2}x \\& \text{ }={{\left(1-2{{\sin }^2}\alpha\right)}^2}+\left(1-{{\sin }^2}\alpha\right)={{\left(1-2 \cdot {{\left(\dfrac{1}{5}\right)}^2}\right)}^2}+\left(1-{{\left(\dfrac{1}{5}\right)}^2}\right)=\dfrac{1129}{625} \end{aligned}$\\
		$\Rightarrow \heva{& a=1129 \\& b=625}\Rightarrow a+b=1754$
	}
\end{ex}
%Câu 18
\begin{ex}
	Số điểm chung của đồ thị hàm số $y=\sin x$ và $y=\cos x$ trên $\left[ -\dfrac{\pi }{2};\dfrac{3\pi }{2} \right]$ là $n$. Giá trị $\sqrt{n}$ (làm tròn đến hàng phần trăm) bằng
	\shortans{1,41}
	\loigiai{
		Số điểm chung của đồ thị hàm số $y=\sin x$ và $y=\cos x$ trên $\left[ -\dfrac{\pi }{2};\dfrac{3\pi }{2} \right]$ bằng số nghiệm phương trình $\sin x = \cos x$ trên $\left[ -\dfrac{\pi }{2};\dfrac{3\pi }{2} \right]$.\\
		Ta có $\sin x = \cos x \Leftrightarrow \sin x - \cos x =0 \Leftrightarrow \sin \left(x-\dfrac{\pi}{4} \right)=0 \Leftrightarrow x-\dfrac{\pi}{4}=k \pi \Leftrightarrow x= \dfrac{\pi}{4} +k\pi \ (k \in \mathbb{Z})$.\\
		$x \in \left[ -\dfrac{\pi }{2};\dfrac{3\pi }{2} \right]$ nên $x \in \left\{ \dfrac{\pi}{4}; \dfrac{5\pi}{4}\right\}$.\\
		Vậy $n=2$ nên $\sqrt{n} \approx 1,41$.
	}
\end{ex}
%Câu 19
\begin{ex}
	Biết có $n$ giá trị nguyên của tham số $m$ để phương trình $\cos x=m$ có nghiệm. Giá trị $\sqrt{n}$ (làm tròn đến hàng phần trăm) bằng
	\shortans{1,73}
	\loigiai{
		$\cos x=m$ có nghiệm $\Leftrightarrow -1\le m\le 1$. Mà $m\in \mathbb{Z}\Rightarrow m\in \left\{ -1;0;1 \right\}$. Vậy $\sqrt{n}\approx 1{,}73$
	}
\end{ex}
%Câu 20
\begin{ex}
	Biết $x=x_0$ là nghiệm duy nhất của phương trình $2\sin \left(x-\dfrac{\pi }{6}\right)+2=0$ trên khoảng $\left(0;2\pi\right)$. Giá trị $x_0$ (làm tròn đến hàng phần trăm) bằng
	\shortans{5,24}
	\loigiai{
		Ta có: $2\sin \left(x-\dfrac{\pi }{6}\right)+2=0\Leftrightarrow \sin \left(x-\dfrac{\pi }{6}\right)=-1\Leftrightarrow x=-\dfrac{\pi }{3}+k2\pi ,k\in \mathbb{Z}$\\
		Do $x\in \left(0;2\pi\right)$ nên $0<-\dfrac{\pi }{3}+k2\pi <2\pi \Leftrightarrow \dfrac{1}{6}<k<\dfrac{7}{6}\Leftrightarrow k=1$.\\
		Vậy phương trình có một nghiệm $x=\dfrac{5\pi }{3}\approx 5{,}24$
	}
\end{ex}
%Câu 21
\begin{ex}
	Gọi $M$ và $m$ lần lượt là giá trị lớn nhất và giá trị nhỏ nhất của hàm số $y=\sin x+\sqrt{3}\cos x+\sqrt{2}$. Tính $M^2m$ (làm tròn đến hàng phần trăm)
	\shortans{6,83}
	\loigiai{
		Ta có $y=\sin x+\sqrt{3}\cos x+\sqrt{2}=2\left(\dfrac{1}{2}\sin x+\dfrac{\sqrt{3}}{2}\cos x\right)+\sqrt{2}=2\sin \left(x+\dfrac{\pi }{3}\right)+\sqrt{2}$.\\
		Suy ra $M=2+\sqrt{2}$, $m=-2+\sqrt{2}$. Nên $M^2m\approx 6{,}83$
	}
\end{ex}
%Câu 22
\begin{ex}
	Mùa xuân ở Hội Lim (tỉnh Bắc Ninh) thường có trò chơi đu. Khi người chơi đu nhún đều, cây đu sẽ đưa người chơi đu dao động qua lại vị trí cân bằng. Nghiên cứu trò chơi này, người ta thấy khoảng cách $h$ (mét) được tính từ vị trí chân người chơi đu đến vị trí cân bằng được biểu diễn bởi hệ thức $h=|d|$ với $d=3\cos \left[\dfrac{\pi }{3}(2t-1)\right]$ ($t\ge 0$ và được tính bằng giây), trong đó ta quy ước $d>0$ khi vị trí cân bằng ở về phía sau lưng người chơi đu và $d<0$ trong trường hợp ngược lại.
	Biết $t_1$, $t_2$ lần lượt là thời điểm đầu tiên người đu ở vị trí phía sau lưng và vị trí phía trước vị trí cân bằng $1{,}5$ mét. Giá trị $t_1+t_2^2$ (làm tròn đến hàng phần trăm) bằng
	\shortans{3,25}
	\loigiai{
	Người chơi cách vị trí cân bằng 1 mét khi $3\cos \left[\dfrac{\pi }{3}(2t-1)\right]=\pm 1{,}5$\\
	$\Leftrightarrow \cos^2\left[\dfrac{\pi }{3}(2t-1)\right]=\dfrac{1}{4}\Leftrightarrow \cos \left[\dfrac{2\pi }{3}(2t-1)\right]=-\dfrac{1}{2}$ $\Leftrightarrow \hoac{& \dfrac{2\pi }{3}(2t-1)=\dfrac{2\pi }{3}+k2\pi \\& \dfrac{2\pi }{3}(2t-1)=-\dfrac{2\pi }{3}+k2\pi} \left(k\in \mathbb{Z}\right)
		\Leftrightarrow \hoac{& t=1+\dfrac{3k}{2} \\& t=\dfrac{3k}{2}}\left(k\in \mathbb{Z}\right)$.\\
	Vì $t>0$ nên $t_1=1$ và $t_2=1{,}5$. Vậy $t_1+t_2^2=3{,}25$
	}
\end{ex}

\Closesolutionfile{ans}

\indapan{10}{ans/ans-SA-C1-De1}
% \section*{BT ÔN TẬP CHƯƠNG 1}
\setcounter{ex}{0}\setcounter{bt}{0}
\Opensolutionfile{ans}[ans/ans1C3-CD-1]
\noindent\textbf{I. PHẦN TRẮC NGHIỆM:}
\begin{ex}%[1T1B2-3]
        Tính tổng $S=\sin^25^{\circ}+\sin^210^{\circ}+\sin^215^{\circ}+ \cdots +\sin^285^{\circ}$.
        \choice
        {$S=\dfrac{19}{2}$}
        {\True $S=\dfrac{17}{2}$}
        {$S=8$}
        {$S=9$}
        \loigiai{
            \begin{align*}
                S&=\sin^25^{\circ}+\sin^210^{\circ}+\sin^215^{\circ}+ \cdots +\sin^285^{\circ}\\
                &=\left(\sin^25^{\circ}+\sin^285^{\circ}\right)+\left(\sin^210^{\circ}+\sin^280^{\circ}\right)+ \cdots +\left(\sin^240^{\circ}+\sin^250^{\circ}\right)+\sin^245^{\circ} \\
                &=8+\dfrac{1}{2}=\dfrac{17}{2}.
        \end{align*}}
    \end{ex}

\begin{ex}%[1C1Y1-2]
Cho góc lượng giác với tia đầu và tia cuối như trong hình. Tên của góc lượng giác là
    \begin{center}
        \begin{tikzpicture}[scale=1, font=\footnotesize, line join=round, line cap=round, >=stealth]
            \begin{axis}[
                axis line style={draw=none},
                axis lines=middle,
                axis equal image,
                enlargelimits,
                xtick=\empty,
                ytick=\empty,
                data cs=polar,
                samples=200,
                thick,
                line cap=round,
                line join=round,
                >=stealth
                ]
                \addplot [smooth, domain=0:390,->] {1+x/5000};
                \addplot [smooth, domain=420:450,->] {1.5+x/5000} node[above,midway]{$+$};
                \addplot [mark=none] (0.475,0) node [below left] {$O$};
                \addplot [mark=none] coordinates {(0,0) (390,3+390/5000)} node[below]{$y$};
                \addplot [mark=none] coordinates {(0,0) (0,3+0/5000)} node[below]{$x$};
                \addplot [mark=none,dashed] coordinates {(0,0) (420,3+420/5000)} node[right]{$m$};
            \end{axis}
        \end{tikzpicture}
    \end{center}
\choice
{\True $(Ox,Oy)$}
{$(Oy,Ox)$}
{$(Om,Oy)$}
{$(Om,Ox)$}
    \loigiai{
        Trong hình, góc lượng giác là $(Ox,Oy)$ với tia đầu $Ox$ và tia cuối $Oy$.
    }
    \end{ex}

\begin{ex}%[1T1B2-2]
        Cho $\tan a=\dfrac{2}{3}$, $5\pi <a<\dfrac{11\pi}{2}$. Khi đó $\cos \left(a+\dfrac{\pi}{3}\right)$ bằng
        \choice
        {$\dfrac{2\sqrt{3}+3}{2\sqrt{13}}$}
        {\True $\dfrac{2\sqrt{3}-3}{2\sqrt{13}}$}
        {$\dfrac{-2\sqrt{3}+3}{2\sqrt{13}}$}
        {$\dfrac{-2\sqrt{3}-3}{2\sqrt{13}}$}
        \loigiai{
            Ta có $\cos^2a=\dfrac{1}{1+\tan^2 a}=\dfrac{1}{1+\dfrac{4}{9}}=\dfrac{9}{13}$.\\
            Vì $5\pi <a<\dfrac{11\pi}{2}$ nên $\cos a<0$ và $\sin a<0$.\\
             Do đó, $\cos a=-\dfrac{3\sqrt{13}}{13}$ và $\sin a=-\dfrac{2\sqrt{13}}{13}$.\\
            Vậy $\cos \left(a+\dfrac{\pi}{3}\right)=\dfrac{1}{2}\cos a-\dfrac{\sqrt{3}}{2}\sin a=\dfrac{-3+2\sqrt{3}}{2\sqrt{13}}$.}
    \end{ex}

\begin{ex}%[Tex hóa SGK CD-CT,T12/22, TVN-006]%[1K1Y1-8]
    Trong các khẳng định sau, khẳng định  nào là \textbf{sai}?
    \choice
    {$\sin(\pi-\alpha)=\sin\alpha$}
    {\True $\cos(\pi-\alpha)=\cos \alpha$}
    {$\sin(\pi+\alpha)=-\sin\alpha$}
    {$\cos(\pi+\alpha)=-\cos \alpha$}
    \loigiai{
        Ta có $\cos(\pi-\alpha)=-\cos \alpha$ nên $\cos(\pi-\alpha)=\cos \alpha$ là khẳng định \textbf{sai}.
    }
\end{ex}

\begin{ex}%[1C1B1-2]
    Cho góc lượng giác gốc $O$ có tia đầu $Ou$, tia cuối $Ov$ và có số đo $\dfrac{2\pi}{3}$. Cho góc lượng giác $(O'u',O'v')$ có tia đầu $O'u'\equiv Ou$, tia cuối $O'v'\equiv Ov$. Viết công thức biểu thị số đo góc lượng giác $(O'u',O'v')$.
\choice
{$(O'u',Ov')=\dfrac{\pi}{3}+k2\pi\ (k\in \mathbb{Z})$}
{$(O'u',Ov')=\dfrac{4\pi}{3}+k2\pi\ (k\in \mathbb{Z})$}
{\True $(O'u',Ov')=\dfrac{2\pi}{3}+k2\pi\ (k\in \mathbb{Z})$}
{$(O'u',Ov')=-\dfrac{\pi}{3}+k2\pi\ (k\in \mathbb{Z})$}
    \loigiai{
        Ta có $(O'u',Ov')=(Ou,Ov)+k2\pi=\dfrac{2\pi}{3}+k2\pi\ (k\in \mathbb{Z})$.
    }
\end{ex}

\begin{ex}%[Tex hóa SGK CD-CT,T12/22, TVN-006]%[1K1B2-3]
    Rút gọn biểu thức $M=\cos(a+b)\cos(a-b)-\sin (a+b)\sin(a-b)$, ta được
    \choice
    {$M=\sin 4a$}
    {$M=1-2\cos^2a$}
    {\True $M=1-2\sin^2a$}
    {$M=\cos 4a$}
    \loigiai{
        Ta có
        \allowdisplaybreaks
        \begin{eqnarray*}
            M&=&\cos(a+b)\cos(a-b)-\sin (a+b)\sin(a-b)\\
            &=&\dfrac{1}{2}\left(\cos2a+\cos 2b\right)+\dfrac{1}{2}\left(\cos2a-\cos 2b\right)\\
            &=&\cos 2a\\
            &=&1-2\sin^2a.
        \end{eqnarray*}
    }
\end{ex}

\begin{ex}%[1T1B5-3]
        Tập nghiệm của phương trình $3\cos\left(3x-\dfrac{\pi}{3}\right)=0$ là
        \choice
        {$\left\{\dfrac{\pi}{2}+k\pi, k \in \mathbb{Z}\right\}$}
        {$\left\{\dfrac{5\pi}{6}+k 2\pi, k \in \mathbb{Z}\right\}$}
        {$\left\{\dfrac{5\pi}{18}+\dfrac{k 2\pi}{3}, k \in \mathbb{Z}\right\}$}
        {\True $\left\{\dfrac{5\pi}{18}+\dfrac{k\pi}{3}, k \in \mathbb{Z}\right\}$}
        \loigiai{
            $3\cos\left(3x-\dfrac{\pi}{3}\right)=0\Leftrightarrow 3x-\dfrac{\pi}{3}=\dfrac{\pi}{2}+k\pi\Leftrightarrow x=\dfrac{5\pi}{18}+\dfrac{k\pi}{3}, k\in\mathbb{Z}$.
            Tập nghiệm phương trình $S=\left\{\dfrac{5\pi}{18}+\dfrac{k\pi}{3}, k\in\mathbb{Z}\right\}$.
        }
    \end{ex}

\begin{ex}%[1T1B6-3]
        Phương trình $\sqrt{3}\sin x+\cos x=1$ tương đương với phương trình nào sau đây?
        \choice
        {$\cos \left( x+\dfrac{\pi}{6}\right) =\dfrac{1}{2}$}
        {$\sin \left( x+\dfrac{\pi}{3}\right) =\dfrac{1}{2}$}
        {\True $\cos \left( x-\dfrac{\pi}{3}\right) =\dfrac{1}{2}$}
        {$\sin \left( x-\dfrac{\pi}{6}\right) =\dfrac{1}{2}$}
        \loigiai{
            Chia hai vế của phương trình cho $2$, ta được
            \begin{eqnarray*}
                &\sqrt{3}\sin x+\cos x=1&\Leftrightarrow\dfrac{\sqrt{3}}{2}\sin x+\dfrac{1}{2}\cos x=\dfrac{1}{2}\\
                &&\Leftrightarrow\sin\dfrac{\pi}{3}\sin x+\cos\dfrac{\pi}{3}\cos x=\dfrac{1}{2}\\
                &&\Leftrightarrow\cos\left( x-\dfrac{\pi}{3}\right) =\dfrac{1}{2}.
            \end{eqnarray*}
        }
    \end{ex}

\begin{ex}%[1T1Y4-1]
        Tìm điều kiện xác định của hàm số  $y=\cot x$.
        \choice
        {$x \neq \dfrac{\pi}{4}+k \pi, k \in \mathbb{Z}$}
        { $x \neq k 2 \pi, k \in \mathbb{Z}$}
        {\True $x \neq k \pi, k \in \mathbb{Z}$}
        {$x\neq \dfrac{\pi}{2}+k \pi, k \in \mathbb{Z}$}
        \loigiai{
            Hàm số $y=\cot x$ xác định khi và chỉ khi $\sin x \ne 0 \Leftrightarrow x\neq k \pi, k \in \mathbb{Z} .$}
    \end{ex}

\begin{ex}%[1T1B4-2]
        Hàm số nào sau đây đồng biến trên khoảng $(0;\pi)$?
        \choice
        {\True $y=x^2$}
        {$y=\cos x$}
        {$y=\sin x$}
        {$y=\tan x$}
        \loigiai{
            Hàm số $y = x^2$ đồng biến khi $x > 0 \Rightarrow$ hàm số đồng biên trên khoảng $\left(0;\pi\right)$.}
    \end{ex}

\begin{ex}%[1C1B1-2]
    Cho góc lượng giác gốc $O$ có tia đầu $Ou$, tia cuối $Ov$ và có số đo $-\dfrac{5\pi}{6}$. Cho góc lượng giác $(O'u',O'v')$ có tia đầu $O'u'\equiv Ou$, tia cuối $O'v'\equiv Ov$. Viết công thức biểu thị số đo góc lượng giác $(O'u',O'v')$.
    \choice
    {$(O'u',Ov')=\dfrac{\pi}{6}+k2\pi\ (k\in \mathbb{Z})$}
    {$(O'u',Ov')=\dfrac{4\pi}{3}+k2\pi\ (k\in \mathbb{Z})$}
    {$(O'u',Ov')=-\dfrac{\pi}{6}+k2\pi\ (k\in \mathbb{Z})$}
    {\True $(O'u',Ov')=-\dfrac{5\pi}{6}+k2\pi\ (k\in \mathbb{Z})$}
    \loigiai{
        Ta có $(O'u',Ov')=(Ou,Ov)+k2\pi=-\dfrac{5\pi}{6}+k2\pi\ (k\in \mathbb{Z})$.
    }
\end{ex}

\begin{ex}%[1T1B4-6]
        Hình bên dưới là đồ thị của hàm số nào dưới đây?
        \begin{center}
            \begin{tikzpicture}[>=stealth,line join=round,line cap=round,font=\footnotesize,scale=0.7]
                \def\a{3.141592654}
                \draw[color=gray,dash pattern=on 1pt off 1pt,xstep=3.14cm,ystep=1.0cm] (-9.424,-3) grid (9.424,3);
                \draw[->] (-9.424,0) -- (9.7,0)node[below]{\scriptsize $x$};
                \draw[->] (0,-3.5) -- (0,3.5) node[left] {\scriptsize $y$};
                \draw (0,0)node[below left]{\scriptsize $O$};
                \clip (-9.58,-3.5)rectangle(9.58,3.5);
                \draw[samples=300,smooth,domain=-9.424:9.424] plot(\x,{-3*cos(\x*180/pi)});
                \path(-2*\a,0)node[shift={(-150:12pt)}]{$-2\pi$}
                (-\a,0)node[shift={(-150:12pt)}]{$-\pi$}
                (\a,0)node[shift={(-135:10pt)}]{$\pi$}
                (2*\a,0)node[shift={(-140:10pt)}]{$2\pi$}
                (0,3)node[shift={(-140:10pt)}]{$3$}
                (0,-3)node[shift={(-140:10pt)}]{$-3$};
                \foreach \x/\y in{-2*\a/0,-\a/0,\a/0,2*\a/0,-3*\a/3,3*\a/3,-2*\a/-3,2*\a/-3,\a/3,-\a/3,0/3,0/-3,0/0}\draw(\x,\y) circle (1pt);
            \end{tikzpicture}
        \end{center}
        \choice
        {\True $y=-3\cos x$}
        {$y=-2-\cos x$}
        {$y=2+|\cos x|$}
        {$y=\cos x-4$}
        \loigiai{
            \begin{itemize}
                \item $y(0)=-3\Rightarrow $ loại $y=\cos x-4$ và $y=2+|\cos x|$.
                \item $y(\pi)=3\Rightarrow $ loại $y=-2-\cos x$.
            \end{itemize}
        }
    \end{ex}

\begin{ex}%[1T1Y4-1]
        Điều kiện xác định của hàm số $y=\cot x$ là
        \choice
        { $x \ne \dfrac{\pi}{8}+k\dfrac{\pi}{2}$}
        {$x \ne \dfrac{\pi}{2}+k\pi$}
        {\True $x \ne k\pi$}
        {$x \ne \dfrac{\pi}{4}+k\pi$}
        \loigiai{
            Hàm số xác định khi và chỉ khi $\sin x \ne 0 \Leftrightarrow x \ne k\pi$, $k \in \mathbb{Z}$.
        }
    \end{ex}

\begin{ex}%[1T1B4-5]
        Cho hàm số $y=\sin^2x-\sin x+2$. Gọi $M,N$ lần lượt là GTLN và GTNN của hàm số đã cho. Khi đó $M+N$ bằng
        \choice
        {$k=-\dfrac{1}{2}$}
        {\True $\dfrac{23}{4}$}
        {$\dfrac{15}{4}$}
        {$6$}
        \loigiai{
            Ta có $y=\sin^2x-\sin x+2=\left(\sin x-\dfrac{1}{2}\right)^2+\dfrac{7}{4}$. \\
            Vì $-1 \leq \sin x \leq 1,\,\forall x \in \mathbb{R}$ nên $-\dfrac{3}{2} \leq \sin x-\dfrac{1}{2} \leq \dfrac{1}{2},\,\forall x \in \mathbb{R}$.\\
            Suy ra $0 \leq \left(\sin x-\dfrac{1}{2}\right)^2 \leq \dfrac{9}{4},\,\forall x \in \mathbb{R}$.\\
            Suy ra $\dfrac{7}{4} \leq \left(\sin x-\dfrac{1}{2}\right)^2+\dfrac{7}{4} \leq 4,\,\forall x \in \mathbb{R}$.\\
            Suy ra $\dfrac{7}{4} \leq y \leq 4,\,\forall x \in \mathbb{R}$.\\
            Vậy $M+N=\dfrac{7}{4}+4=\dfrac{23}{4}$.}
    \end{ex}

\begin{ex}%[Tex hóa SGK CD-CT,T12/22, TVN-006]%[1K1Y3-4]
    Trong các hàm số sau đây, hàm số nào là hàm tuần hoàn?
    \choice
    {$y=\tan x+x$}
    {$y=x^2+1$}
    {\True $y=\cot x$}
    {$y=\dfrac{\sin x}{x}$}
    \loigiai{
        Hàm số $y=\cot x$ là hàm số tuần hoàn với chu kỳ $T=\pi$.
    }
\end{ex}

\begin{ex}%[1T1Y1-1]
        Góc $18^\circ$ có số đo bằng rađian là bao nhiêu?
        \choice
        {$\pi$}
        {$\dfrac{\pi}{360}$}
        {\True $\dfrac{\pi}{10}$}
        {$\dfrac{\pi}{18}$}
        \loigiai{
            Ta có $18^\circ=\dfrac{\pi}{10}$ rad.
        }
    \end{ex}

\begin{ex}%[Tex hóa SGK CD-CT,T12/22, TVN-006]%[1K1Y1-4]
    Biểu diễn các góc lượng giác $\alpha=-\dfrac{5\pi}{6}$, $\beta=\dfrac{\pi}{3}$, $\gamma=\dfrac{25\pi}{3}$, $\delta=\dfrac{17\pi}{6}$ trên đường tròn lượng giác. Các góc nào có điểm biểu diễn trùng nhau?
    \choice
    {\True $\beta$ và $\gamma$}
    {$\alpha$, $\beta$, $\gamma$}
    {$\beta$, $\gamma$, $\delta$}
    {$\alpha$ và $\beta$}
    \loigiai{
        Ta có $\beta+8\pi=\dfrac{\pi}{3}+8\pi=\dfrac{25\pi}{3}=\gamma$.\\
        Do đó, $\beta$ và $\gamma$ có điểm biểu diễn trùng nhau trên đường tròn lượng giác.
    }
\end{ex}

\begin{ex}%[1C1B1-2]
    Cho góc lượng giác $(Ou,Ov)$ có số đo là $\dfrac{3\pi}{4}$, góc lượng giác $(Ou,Ow)$ có số đo là $\dfrac{5\pi}{4}$. Số đo của góc lượng giác $(Ov,Ow)$ là
\choice
{\True $(Ov,Ow)=\dfrac{\pi}{2}+k2\pi\ (k\in \mathbb{Z})$}
{$(Ov,Ow)=2\pi+k2\pi\ (k\in \mathbb{Z})$}
{$(Ov,Ow)=-\dfrac{\pi}{2}+k2\pi\ (k\in \mathbb{Z})$}
{$(Ov,Ow)=-\dfrac{\pi}{6}+k2\pi\ (k\in \mathbb{Z})$}
    \loigiai{
        Theo hệ thức Chasles, ta có
        \begin{eqnarray*}
            (Ov,Ow)&=&(Ou,Ow)-(Ou,Ov)+k2\pi\\
            &=&\dfrac{5\pi}{4}-\dfrac{3\pi}{4}+k2\pi\\
            &=&\dfrac{\pi}{2}+k2\pi\ (k\in \mathbb{Z}).
        \end{eqnarray*}
    }
\end{ex}

\begin{ex}%[1C1B1-2]
    Cho góc lượng giác gốc $O$ có tia đầu $Ou$, tia cuối $Ov$ và có số đo $45^\circ$. Cho góc lượng giác $(O'u',O'v')$ có tia đầu $O'u'\equiv Ou$, tia cuối $O'v'\equiv Ov$. Công thức biểu thị số đo góc lượng giác $(O'u',O'v')$ là
\choice
{$(O'u',Ov')=-45^\circ+k360^\circ\ (k\in \mathbb{Z})$}
{\True $(O'u',Ov')=45^\circ+k360^\circ\ (k\in \mathbb{Z})$}
{$(O'u',Ov')=135^\circ+k360^\circ\ (k\in \mathbb{Z})$}
{$(O'u',Ov')=-135^\circ+k360^\circ\ (k\in \mathbb{Z})$}
    \loigiai{
        Ta có $(O'u',Ov')=(Ou,Ov)+k360^\circ=45^\circ+k360^\circ\ (k\in \mathbb{Z})$.
    }
\end{ex}

\begin{ex}%[1T1B4-5]
        Hàm số $y=3-5\sin x$ có giá trị lớn nhất bằng
        \choice
        {$6$}
        {$2$}
        {\True $8$}
        {$4$}
        \loigiai{
            Ta có
            $$-1\le \sin x\le 1 \Leftrightarrow 5\ge-5\sin x\ge-5\Leftrightarrow  8\ge 3-5\sin x\ge -2\Rightarrow -2\le y\le 8.$$
            Suy ra giá trị lớn nhất của hàm số là $8$, đạt được khi $x=\dfrac{\pi}{2}+k2\pi,k\in\mathbb{Z}$.
        }
    \end{ex}

\begin{ex}%[1T1B2-4]
        Rút gọn biểu thức $M=\sin(\pi-a)+\tan\left(\dfrac{\pi}{2}-a\right)+\sin(-a)+\cot(\pi+a)$ được
        \choice
        {$M=2\cos a$}
        {$M=2\tan a$}
        {\True $M=2\cot a$}
        {$M=0$}
        \loigiai{
            Ta có $M=\sin a+\cot a-\sin a+\cot a=2\cot a$.
        }
    \end{ex}

\begin{ex}%[1T1Y4-6]
        Đồ thị hàm số $y=\cos x$ đi qua điểm nào sau đây?
        \choice
        {$P(-1;\pi)$}
        {$M(\pi;1)$}
        {$Q(3\pi; 1)$}
        {\True $N(0;1)$}
        \loigiai{
            Điểm $N(0;1)$ thuộc đồ thị hàm số.
        }
    \end{ex}

\begin{ex}%[1T1B4-1]
        Tập xác định của hàm số $y=2017\tan^{2018} \left( 2x+\dfrac{\pi}{3}\right)$ là
        \choice
        {\True $\mathscr{D}=\mathbb{R}\setminus\left\lbrace\dfrac{\pi}{12}+k\dfrac{\pi}{2}, k\in\mathbb{Z} \right\rbrace $}
        {$\mathscr{D}=\mathbb{R}\setminus\left\lbrace\dfrac{\pi}{2}+k\dfrac{\pi}{2}, k\in\mathbb{Z} \right\rbrace $}
        {$\mathscr{D}=\mathbb{R}\setminus\left\lbrace\dfrac{\pi}{2}+k\dfrac{\pi}{2}, k\in\mathbb{Z} \right\rbrace $}
        {$\mathscr{D}=\mathbb{R}\setminus\left\lbrace\dfrac{\pi}{2}+k\dfrac{\pi}{2}, k\in\mathbb{Z} \right\rbrace $}
        \loigiai{
            Hàm số xác định khi $2x+\dfrac{\pi}{3}\ne \dfrac{\pi}{2}+k\pi\Leftrightarrow x\ne\dfrac{\pi}{12}+k\dfrac{\pi}{2}, k\in\mathbb{Z}.$
        }
        \end{ex}

\begin{ex}%[1K1Y1-7]
        Tìm khẳng định đúng (với điều kiện các hệ thức đã xác định).
        \choice
        {$\cos \left(\pi -\alpha \right)=\cos \alpha$}
        {\True $\cos \left(-\alpha \right)=\cos \alpha$}
        {$\sin \left(\pi -\alpha \right)=-\sin \alpha$}
        {$\sin \left(-\alpha \right)=\sin \alpha$}
        \loigiai{
            Ta có
            \begin{itemize}
                \item $\sin \left(-\alpha \right)=-\sin \alpha$.
                \item $\cos \left(\pi -\alpha \right)=-\cos \alpha$.
                \item $\cos \left(-\alpha \right)=\cos \alpha$.
                \item $\sin \left(\pi -\alpha \right)=\sin \alpha$.
            \end{itemize}
        }
    \end{ex}


\noindent\textbf{II. PHẦN TỰ LUẬN:}
\begin{ex}%[Cánh Diều]%[1C1B4-3]
    Giải các phương trình
    \begin{multicols}{3}
        \begin{enumerate}[a)]
            \item $\sin x=-\dfrac{1}{2}$;
            \item $\sin x=\dfrac{\sqrt{2}}{2}$;
            \item $\sin3x=\sin2x$;
            \item $\sin x=\cos3x$;
            \item $\cos x=\dfrac{\sqrt{3}}{2}$;
            \item $\cos x=-\dfrac{\sqrt{2}}{2}$;
            \item $\cos x=-\dfrac{1}{2}$;
            \item $\cos3x=\cos\left(x+\dfrac{\pi}{3}\right)$;
            \item $\tan x=\dfrac{1}{\sqrt{3}}$;
            \item $\tan x=-1$;
            \item $\cot2x=-\sqrt{3}$.
        \end{enumerate}
    \end{multicols}
\loigiai{
\begin{enumerate}[a)]
    \item Do $\sin\left(-\dfrac{\pi}{6}\right)=-\dfrac{1}{2}$ nên $$\sin x=\sin\left(-\dfrac{\pi}{6}\right)\Leftrightarrow\hoac{&x=-\dfrac{\pi}{6}+k2\pi\\&x=\pi-\left(-\dfrac{\pi}{6}\right)+k2\pi}\Leftrightarrow\hoac{&x=-\dfrac{\pi}{6}+k2\pi\\&x=\dfrac{7\pi}{6}+k2\pi}\,(k\in\mathbb{Z}).$$
    \item Do $\sin\dfrac{\pi}{4}=\dfrac{\sqrt{2}}{2}$ nên $$\sin x=\sin\dfrac{\pi}{4}\Leftrightarrow\hoac{&x=\dfrac{\pi}{4}+k2\pi\\&x=\pi-\dfrac{\pi}{4}+k2\pi}\Leftrightarrow\hoac{&x=\dfrac{\pi}{4}+k2\pi\\&x=\dfrac{3\pi}{4}+k2\pi}\,(k\in\mathbb{Z}).$$
    \item $\sin3x=\sin2x\Leftrightarrow\hoac{&3x=2x+k2\pi\\&3x=\pi-2x+k2\pi}\Leftrightarrow\hoac{&x=k2\pi\\&x=\dfrac{\pi}{5}+k\dfrac{2\pi}{5}}\,(k\in\mathbb{Z})$.
    \item $\sin x=\cos3x\Leftrightarrow\sin x=\sin\left(\dfrac{\pi}{2}-3x\right)\Leftrightarrow\hoac{&x=\dfrac{\pi}{2}-3x+k2\pi\\&x=\pi-\left(\dfrac{\pi}{2}-3x\right)+k2\pi}\Leftrightarrow\hoac{&x=\dfrac{\pi}{8}+k\dfrac{\pi}{2}\\&x=-\dfrac{\pi}{4}+k\pi}\,(k\in\mathbb{Z})$.
    \item $\cos x=\dfrac{\sqrt{3}}{2}\Leftrightarrow\cos x=\cos\dfrac{\pi}{6}\Leftrightarrow x=\pm\dfrac{\pi}{6}+k2\pi,\,(k\in\mathbb{Z})$;
    \item $\cos x=-\dfrac{\sqrt{2}}{2}\Leftrightarrow \cos x=\cos\dfrac{3\pi}{4}\Leftrightarrow x=\pm\dfrac{3\pi}{4}+k2\pi,\,(k\in\mathbb{Z})$;
    \item $\cos x=-\dfrac{1}{2}\Leftrightarrow\cos x=\cos\dfrac{2\pi}{3}\Leftrightarrow x=\pm\dfrac{2\pi}{3}+k2\pi,\,(k\in\mathbb{Z})$;
    \item $\cos3x=\cos\left(x+\dfrac{\pi}{3}\right)\Leftrightarrow\hoac{&3x=x+\dfrac{\pi}{3}+k2\pi\\&3x=-x-\dfrac{\pi}{3}+k2\pi}\Leftrightarrow\hoac{&x=\dfrac{\pi}{6}+k\pi\\&x=-\dfrac{\pi}{12}+\dfrac{k\pi}{2}},\,(k\in\mathbb{Z})$;
    \item $\tan x=\dfrac{1}{\sqrt{3}}\Leftrightarrow\tan x=\tan\dfrac{\pi}{6}\Leftrightarrow x=\dfrac{\pi}{6}+k\pi,\,(k\in\mathbb{Z})$;
    \item $\tan x=-1\Leftrightarrow\tan x=\tan\left(-\dfrac{\pi}{4}\right)\Leftrightarrow x=-\dfrac{\pi}{4}+k\pi,\,(k\in\mathbb{Z})$;
    \item $\cot2x=-\sqrt{3}\Leftrightarrow\cot2x=\cot\left(-\dfrac{\pi}{6}\right)\Leftrightarrow x=-\dfrac{\pi}{6}+k\pi,\,(k\in\mathbb{Z})$.
\end{enumerate}}
\end{ex}

\begin{ex}%[1C1B4-3]
    Giải phương trình:
    \begin{listEX}[3]
        \item $\sin \left(2x-\dfrac{\pi}{3}\right)=-\dfrac{\sqrt{3}}{2}$;
        \item $\sin \left(3x+\dfrac{\pi}{4}\right)=-\dfrac{1}{2}$;
        \item $\cos \left(\dfrac{x}{2}+\dfrac{\pi}{4}\right) =\dfrac{\sqrt{3}}{2}$;
        \item $2\cos 3x+5=3$;
        \item $3\tan x=-\sqrt{3}$;
        \item $\cot x-3=\sqrt{3}\left(1-\cot x\right)$.
    \end{listEX}
    \loigiai{
        \begin{enumerate}[a)]
            \item Ta có 
            \begin{eqnarray*}
                &&\sin \left(2x-\dfrac{\pi}{3}\right)=-\dfrac{\sqrt{3}}{2}\\
                &\Leftrightarrow& \sin \left(2x-\dfrac{\pi}{3}\right) =\sin \left(-\dfrac{\pi}{3}\right)\\
                &\Leftrightarrow& \hoac{
                    &2x-\dfrac{\pi}{3} =-\dfrac{\pi}{3}+k2\pi\\
                    &2x-\dfrac{\pi}{3} = \pi+\dfrac{\pi}{3}+k2\pi}\\
                &\Leftrightarrow&
                \hoac{&2x=k2\pi\\&2x=\dfrac{5\pi}{3} +k2\pi}\\
                &\Leftrightarrow&
                \hoac{&x=k\pi\\ &x=\dfrac{5\pi}{6}+k\pi} (k\in \mathbb{Z}).
            \end{eqnarray*}         
            \item Ta có 
            \begin{eqnarray*}
                &&\sin \left(3x+\dfrac{\pi}{4}\right)=-\dfrac{1}{2} \\
                &\Leftrightarrow& \sin \left(3x+\dfrac{\pi}{4}\right) =\sin \left(-\dfrac{\pi}{6}\right)\\  
                &\Leftrightarrow& \hoac{
                    &3x+\dfrac{\pi}{4} = -\dfrac{\pi}{6}+k2\pi\\
                    &3x+\dfrac{\pi}{4}=\pi -\left(-\dfrac{\pi}{6}\right)+k2\pi} \\
                &\Leftrightarrow&
                \hoac{
                    &3x= -\dfrac{5\pi}{12}+k2\pi\\
                    &3x=\dfrac{11\pi}{12}+k2\pi} \\
                &\Leftrightarrow&
                \hoac{&x=-\dfrac{5}{36}+\dfrac{k2\pi}{3}\\
                    &x=\dfrac{11\pi}{36}+\dfrac{k2\pi}{3}} (k\in \mathbb{Z}).
            \end{eqnarray*}                     
            \item Ta có 
            \begin{eqnarray*}
                &&\cos \left(\dfrac{x}{2}+\dfrac{\pi}{4}\right) =\dfrac{\sqrt{3}}{2} \\ &\Leftrightarrow& \cos \left(\dfrac{x}{2}+\dfrac{\pi}{4}\right)=\cos \dfrac{\pi}{6}\\
                &\Leftrightarrow& \hoac{
                    &\dfrac{x}{2}+\dfrac{\pi}{4} = \dfrac{\pi}{6}+k2\pi\\
                    &\dfrac{x}{2}+\dfrac{\pi}{4}= -\dfrac{\pi}{6}+k2\pi}\\
                &\Leftrightarrow&
                \hoac{
                    &\dfrac{x}{2}=-\dfrac{\pi}{12}+k2\pi\\
                    &\dfrac{x}{2}=-\dfrac{5\pi}{12}+k2\pi}\\
                &\Leftrightarrow&
                \hoac{
                    &x=-\dfrac{\pi}{6}+k4\pi\\
                    &x=-\dfrac{5\pi}{6}+k4\pi} (k\in \mathbb{Z}).
            \end{eqnarray*} 
            \item Ta có $2\cos 3x+5=3 \Leftrightarrow \cos 3x =-1 \Leftrightarrow 3x=\pi+k2\pi \Leftrightarrow x = \dfrac{\pi}{3}+\dfrac{k2\pi}{3}\,\,(k\in \mathbb{Z})$.
            \item Ta có $3\tan x=-\sqrt{3} \Leftrightarrow \tan x =-\dfrac{\sqrt{3}}{3} \Leftrightarrow 
            \tan x=\tan \left(-\dfrac{\pi}{6}\right) \Leftrightarrow x=-\dfrac{\pi}{6}+k\pi\,\, (k\in \mathbb{Z}).$
            \item Ta có 
            \begin{eqnarray*}
                &&\cot x-3=\sqrt{3}\left(1-\cot x\right)\\
                &\Leftrightarrow& \cot x-3 =\sqrt{3}-\sqrt{3}\cot x\\
                &\Leftrightarrow& (1+\sqrt{3})\cot x=\sqrt{3}(1+\sqrt{3})\\
                &\Leftrightarrow& \cot x=\sqrt{3}\\
                &\Leftrightarrow& \cot x=\cot \dfrac{\pi}{6}\\
                &\Leftrightarrow& x=\dfrac{\pi}{6}+k\pi\,\, (k\in \mathbb{Z}).
            \end{eqnarray*} 
        \end{enumerate}
    }
\end{ex}

\begin{ex}%[1C1B4-3]
    Giải phương trình:
    \begin{listEX}[3]
        \item $\sin \left(2x+\dfrac{\pi}{4}\right)=\sin x$;
        \item $\sin 2x=\cos 3x$;
        \item $\cos^2 2x =\cos^2 \left(x+\dfrac{\pi}{6}\right)$.
    \end{listEX}
    \loigiai{
        \begin{enumerate}[a)]
            \item Ta có 
            \[\sin \left(2x+\dfrac{\pi}{4}\right)=\sin x
            \Leftrightarrow 
            \hoac{&2x+\dfrac{\pi}{4}=x+k2\pi\\&2x+\dfrac{\pi}{4}=\pi-x+k2\pi} 
            \Leftrightarrow \hoac{&x=-\dfrac{\pi}{4}+k2\pi\\&3x=-\dfrac{\pi}{4}+k2\pi} \Leftrightarrow \hoac{&x=-\dfrac{\pi}{4}+k2\pi\\&x=-\dfrac{\pi}{12}+\dfrac{k2\pi}{3}
            },\, (k\in \mathbb{Z}).\]
            \item Ta có 
            \begin{eqnarray*}
                \sin 2x=\cos 3x &\Leftrightarrow& \cos 3x =\cos \left(\dfrac{\pi}{2}-2x\right)\\
                &\Leftrightarrow& \hoac{
                    &3x=\dfrac{\pi}{2}-2x+k2\pi\\
                    &3x=\pi-\left(\dfrac{\pi}{2}-2x\right) +k2\pi}\\
                &\Leftrightarrow&
                \hoac{&5x=\dfrac{\pi}{2}+k2\pi\\&x=\dfrac{\pi}{2}+k2\pi}\\
                &\Leftrightarrow&
                \hoac{&x=\dfrac{\pi}{12}+\dfrac{k2\pi}{5}\\
                    &x=\dfrac{\pi}{2}+k2\pi}\, (k\in \mathbb{Z}).
            \end{eqnarray*} 
            \item Ta có $\cos^2 2x =\cos^2 \left(x+\dfrac{\pi}{6}\right) \Leftrightarrow 
            \hoac{
                &\cos 2x=\cos \left(x+\dfrac{\pi}{6}\right) &(1)\\
                &\cos 2x=-\cos \left(x+\dfrac{\pi}{6}\right). &(2)
            }$\\
            +) $(1) \Leftrightarrow \hoac{
                &2x=x+\dfrac{\pi}{6}+k2\pi\\
                &2x=-\left(x+\dfrac{\pi}{6}\right)+k2\pi
            } \Leftrightarrow
            \hoac{
                &x=\dfrac{\pi}{6}+k2\pi\\
                &3x=-\dfrac{\pi}{6}+k2\pi
            }
            \Leftrightarrow
            \hoac{
                &x=\dfrac{\pi}{6}+k2\pi\\
                &x=-\dfrac{\pi}{18}+\dfrac{k2\pi}{3}
            }(k\in \mathbb{Z})$.\\
            +) $(2) \Leftrightarrow 
            \cos 2x=\cos\left[\pi- \left(x+\dfrac{\pi}{6}\right) \right]
            \Leftrightarrow
            \hoac{
                &2x=\pi- \left(x+\dfrac{\pi}{6}\right)+k2\pi\\
                &2x=-\left[\pi- \left(x+\dfrac{\pi}{6}\right)\right]+k2\pi
            } $
            \[\Leftrightarrow
            \hoac{
                &3x=\dfrac{5\pi}{6}+k2\pi\\
                &x=-\dfrac{5\pi}{6}+k2\pi
            } \Leftrightarrow
            \hoac{
                &x=\dfrac{5\pi}{18}+\dfrac{k2\pi}{3}\\
                &x=-\dfrac{5\pi}{6}+k2\pi
            } \,(k\in\mathbb{Z}).\]
        \end{enumerate}
    }
\end{ex}

\begin{ex}Giải các phương trình sau
    \begin{listEX}[2]
        \item $2\sin x+\sqrt{2}=0$;
        \item $\sin2x-\cos x+2\sin x=1$;
        \item $3\sin ^2 x-5\sin x+2=0$;
        \item $\sqrt{3}\tan^2 x-2\tan x+\sqrt{3}=0$;
        \item $2\cos^2 2x-5\cos 2x+2=0$;
        \item $\sin^2\dfrac{x}{2}+\sin\dfrac{x}{2}-2=0$.
    \end{listEX}
    \loigiai{
        \begin{enumerate}[a)]
            \item $2\sin x+\sqrt{2}=0\Leftrightarrow\sin x=-\dfrac{\sqrt{2}}{2}\Leftrightarrow\sin x=\sin\left(-\dfrac{\pi}{4}\right)\Leftrightarrow\hoac{&x=-\dfrac{\pi}{4}+k2\pi\\&x=\dfrac{5\pi}{4}+k2\pi}\,(k\in\mathbb{Z})$;
            \item $\sin2x-\cos x+2\sin x=1\Leftrightarrow2\sin x\cos x-\cos x+2\sin x-1=0\Leftrightarrow(2\sin x-1)(\cos x+1)=0\\\Leftrightarrow\hoac{&\sin x=\dfrac{1}{2}\\&\cos x=-1}\Leftrightarrow\hoac{&\sin x=\sin\dfrac{\pi}{6}\\&x=(2k+1)\pi}\Leftrightarrow\hoac{&x=\dfrac{\pi}{6}+k2\pi\\&x=\dfrac{5\pi}{6}+k2\pi\\&x=(2k+1)\pi}\,(k\in\mathbb{Z})$;
            \item Đặt $t=\cos x$, $-1\le t\le 1$, phương trình đã cho trở thành $3t^2-5t+2=0$, ta được $t=1$ hoặc $t=\dfrac{2}{3}$.\\
            Với $t=1$ ta có $\cos x=1 \Leftrightarrow x= k2\pi,\, (k\in\mathbb{Z})$.\\
            Với $t=\dfrac{2}{3}$ ta có $\cos x=\dfrac{2}{3}=\cos\alpha\Leftrightarrow x=\pm \alpha +k2\pi,\, (k\in\mathbb{Z})$.\\
            Vậy tập nghiệm của phương trình đã cho là $S=\left\{ k2\pi, \pm \alpha+k2\pi, k\in\mathbb{Z} \right\}$.
            \item Đặt $t=\tan x$, phương trình đã cho trở thành $\sqrt{3} t^2-2t+\sqrt{3}=0$. Phương trình này vô nghiệm. Vậy phương trình đã cho vô nghiệm.
            \item $2\cos^2 2x-5\cos 2x+2=0\Leftrightarrow\hoac{&\cos2x=2\\&\cos2x=\dfrac{1}{2}}\Leftrightarrow\cos2x=\cos\dfrac{\pi}{3}\Leftrightarrow x=\pm\dfrac{\pi}{6}+k\pi,\, (k\in\mathbb{Z})$;
            \item $\sin^2\dfrac{x}{2}+\sin\dfrac{x}{2}-2=0\Leftrightarrow\hoac{&\sin\dfrac{x}{2}=1\\&\sin\dfrac{x}{2}=-2}\Leftrightarrow\dfrac{x}{2}=\dfrac{\pi}{2}+k2\pi\Leftrightarrow x=\pi+k4\pi,\, (k\in\mathbb{Z})$.
        \end{enumerate}
    }
\end{ex}

\begin{ex}%[1D1G1-5]
    Tìm giá trị lớn nhất và giá trị nhỏ nhất của hàm số  $y=2(\sin x+\cos x)+\sin 2 x+3$.
    \loigiai
    {Tập xác định $\mathscr{D}=\mathbb{R}$.\\
        Đặt $t=\sin x+ \cos x=\sqrt{2}\sin \left(x+\dfrac{\pi}{4}\right)$, $t\in \left[-\sqrt{2};\sqrt{2}\right]$.\\
        Ta có $t^2=\left(\sin x+ \cos x\right)^2=1+2\sin x\cos x=1+\sin 2x\Rightarrow \sin 2x =t^2-1$.\\
        Hàm số trở thành $y=g(t)=t^2+2t+2$. \\
        Bảng biến thiên của hàm số $y=g(t)$ trên đoạn $ \left[-\sqrt{2};\sqrt{2}\right]$
        \begin{center}
            \begin{tikzpicture}
                \tkzTabInit[lgt=1,espcl=3,deltacl=1]%nocadre,
                {$t$/1, $g(t)$ /2}
                {$-\sqrt{2}$ , $-1$ ,  $\sqrt{2}$}
                \tkzTabVar {+/$4-2\sqrt{2}$,-/$1$ ,+/ $4+2\sqrt{2}$}
            \end{tikzpicture}
        \end{center}
        Vậy $\max\limits_{x \in \mathbb{R}} y=4+2\sqrt{2}$ và $\min\limits_{x \in \mathbb{R}} y=1$.
    }
\end{ex}

\begin{ex}%[1D1K1-5]
    Tìm giá trị lớn nhất và giá trị nhỏ nhất của hàm số $y=\sqrt{3} \sin x-\cos x+5$.
    \loigiai
    {
        Tập xác định $\mathscr{D}=\mathbb{R}$.\\
        Biến đổi $y=\sqrt{3} \sin x-\cos x+5=2\left(\dfrac{\sqrt{3}}{2}\cdot\sin x-\dfrac{1}{2}\cdot\cos x\right)+5=2\sin\left(x-\dfrac{\pi}{6}\right)+5$.\\
        Với mọi $x\in \mathbb{R}$ ta có
        \allowdisplaybreaks
        \begin{eqnarray*}
            & & -1\leq \sin\left(x-\dfrac{\pi}{6}\right)\leq 1\\
            &\Leftrightarrow& -2\leq 2\sin\left(x-\dfrac{\pi}{6}\right)\leq 2\\
            &\Leftrightarrow&3\leq  2\sin\left(x-\dfrac{\pi}{6}\right)+5\leq 7.
        \end{eqnarray*}
        Vậy $\max\limits_{x \in \mathbb{R}} y=7$ khi $x=\dfrac{2\pi}{3}$ và $\min\limits_{x \in \mathbb{R}} y=3$ khi $x=-\dfrac{\pi}{3}$.
    }
\end{ex}
\Closesolutionfile{ans}

%Chương II
%%Bài 5. DS
% 
\setcounter{section}{4}
\setcounter{dang}{0}
\setcounter{ex}{0}
\setcounter{bt}{0}
\setcounter{vd}{0}
\section{Dãy số}
\subsection{Tóm tắt lý thuyết}
\begin{tomtat}
	\subsubsection{Định nghĩa dãy số} 
	\begin{itemize}
		\item Mỗi hàm $u$ xác định trên tập các số nguyên dương $\mathbb{N^{*}}$ được gọi là một dãy vô hạn (gọi tắt là dãy số), kí hiệu $u=u(n)$.
		\item 	Ta thường viết $u_n$ thay cho $u(n)$ và kí hiệu dãy số $u=u(n)$ bởi $(u_n)$, do đó dãy số $(u_n)$ được viết dưới dạng khai triển $u_1, u_2, u_3, \ldots, u_n, \ldots$\\
		Số $u_1$ gọi là số hạng đầu, $u_n$ gọi là số hạng thứ $n$ và gọi là số hạng tổng quát của dãy số.
		\item Nếu $\forall n \in \mathrm{N^*}, u_n=c$ thì $(u_n)$ được gọi là dãy số không đổi.
		\item Mỗi hàm $u$ xác định trên tập $\mathrm{M}=\left\{1;2;3;\ldots;m\right\}, \forall m \in \mathrm{N^*}$ được gọi là một dãy số hữu hạn.
		\item Dạng khai triển của dãy hữu hạn là $u_1, u_2, u_3, \ldots, u_m$.\\
		Số $u_1$ gọi là số hạng đầu, số $u_m$ gọi là số hạng cuối.
		
	\end{itemize}
	\subsubsection{Các cách cho một dãy số}
	Một dãy số có thể cho bằng:
	\begin{itemize}
		\item Liệt kê các số hạng (chỉ dùng cho các dãy hữu hạn và có ít số hạng);
		\item Công thức của số hạng tổng quát;
		\item Phương pháp mô tả;
		\item Phương pháp truy hồi.
	\end{itemize}
	\subsubsection{Dãy số tăng, dãy số giảm, dãy số bị chặn}
	\begin{itemize}
		\item Dãy số $(u_n)$ được gọi là dãy số tăng nếu ta có $u_{n+1}>u_n, \forall n \in \mathrm{N^*}$.
		\item Dãy số $(u_n)$ được gọi là dãy số giảm nếu ta có $u_{n+1}<u_n, \forall n \in \mathrm{N^*}$.
		\item Dãy số $(u_n)$ được gọi là bị chặn trên nếu tồn tại số $M$ sao cho $u_n \le M, \forall n \in \mathrm{N^*}$.
		\item Dãy số $(u_n)$ được gọi là bị chặn dưới nếu tồn tại số $m$ sao cho $u_n \ge m, \forall n \in \mathrm{N^*}$.
		\item Dãy số $(u_n)$ được gọi là bị chặn nếu nó vừa bị chặn trên vừa bị chặn dưới, tức là  tồn tại các số $m, M$ sao cho $m \le u_n \le M, \forall n \in \mathrm{N^*}$.
	\end{itemize}
\end{tomtat}

\subsection{Các dạng toán thường gặp}
\begin{dang}{Số hạng tổng quát, biểu diễn dãy số}
	Để tìm số hạng tổng quát của một dãy bất kỳ khi biết một vài số hạng đầu của dãy số ta làm như sau
	\begin{itemize}
		\item Phân tích các số hạng sau theo các số hạng đã biết theo một quy luật nào đó.
		\item Dự đoán số hạng tổng quát 
		\item Kiểm tra bằng cách thay lần lượt các giá trị $n\in \mathrm{N^*}$ vào công thức tổng quát (Chứng minh bằng phương pháp quy nạp).
	\end{itemize}
	Để biểu diễn một dãy số khi biết công thức tổng quát ta lần lượt thay $n\in \mathrm{N^*}$ vào công thức tổng quát để tìm các số hạng thứ nhất, thứ hai, $\ldots$
\end{dang}
\subsubsection{Ví dụ minh hoạ}
\begin{vd}[NB]%[DCHT Toán 11 - KNTT -Tên GV]%[1K2Y1-1]
	Xác định số hạng đầu và số hạng tổng quát của dãy số $(u_n)$ các số tự nhiên lẻ $1, 3, 5, 7, \ldots $
	\dapso{$u_n=2n-1$}	
	\loigiai{Dãy $(u_n)$ có số hạng đầu $u_1=1$ và số hạng tổng quát $u_n=2n-1$.}
\end{vd}
\begin{vd}[NB]%[DCHT Toán 11 - KNTT -Tên GV]%[1K2Y1-1]
	Xác định số hạng đầu và số hạng tổng quát của dãy số $(v_n)$ các số nguyên dương chia hết cho $5$: $5,10,15,20,\ldots$
	\dapso{$v_n=5n$}
	\loigiai{Dãy $(v_n)$ có số hạng đầu $v_1=5$ và số hạng tổng quát $v_n=5n$.}
\end{vd}
% \begin{vd}[NB]%[DCHT Toán 11 - KNTT -Tên GV]%[1K2Y1-1]
% 	Viết năm số hạng đầu và số hạng thứ $100$ của dãy số $(u_n)$ có số hạng tổng quát $u_n=3n-2$.
% 	\dapso{$u_{100}=298$}
% 	\loigiai{Năm số hạng đầu của dãy số là $1,4,7,10,13$.\\
% 		Số hạng thứ $100$ của dãy là $u_{100}=3\cdot100-2=298$.}
% \end{vd}
% \begin{vd}[NB]%[DCHT Toán 11 - KNTT -Tên GV]%[1K2Y1-1]
% 	Cho dãy số xác định bằng hệ thức truy hồi: $u_1=1, u_n=3u_{n-1}+2$ với $n\ge 2$. Viết ba số hạng đầu của dãy số này.
% 	\dapso{$u_1=1, u_2=5, u_3=17$}
% 	\loigiai{Ta có $u_1=1, u_2=3u_1+2=5, u_3=3u_2+2=17$.}
% \end{vd}
% \begin{vd}[NB]%[DCHT Toán 11 - KNTT -Tên GV]%[1K2Y1-1]
% 	Dãy số $(u_n)$ cho bởi hệ thức truy hồi: $u_1=1, u_n=n \cdot u_{n-1}$ với $n \ge 2$. Viết năm số hạng đầu của dãy số và dự đoán công thức tổng quát $u_n$.
% 	\dapso{$u_n=n!$}
% 	\loigiai{Năm số hạng đầu của dãy là
% 		$u_1=1, u_2=2\cdot u_1=2, u_3=3\cdot u_2=6, u_4=4 \cdot u_3= 24, u_5=5 \cdot u_4=124$.\\
% 		Số hạng tổng quát\\
% 		Ta có $ u_2=2\cdot 1, u_3=6=3\cdot 2\cdot 1, u_4=24=4\cdot 3\cdot 2\cdot 1, u_5=124= 5\cdot4\cdot3\cdot2\cdot1  $.\\
% 		Vậy số hạng tổng quát $u_n=n!$.}
% \end{vd}
\subsubsection{Bài tập tự luận}
 
\begin{bt}[NB]%[DCHT Toán 11 - KNTT -Tên GV]%[1K2Y1-1]
	Xét dãy số hữu hạn gồm các số tự nhiên lẻ nhỏ hơn 20, sắp xếp theo thứ tự từ bé đến lớn. Liệt kê tất cả các số hạng của dãy số này, tìm số hạng đầu và số hạng cuối của dãy. 
	\dapso{$u_1=1$, $u_{11}=19$}
	\loigiai{Các số hạng của dãy là $1,3,5,7,9,10,11,13,15,17,19$.\\
		Số hạng đầu của dãy là $u_1=1$.\\
		Số hạng cuối của dãy là $u_{11}=19$.}
\end{bt}
\begin{bt}[TH]%[DCHT Toán 11 - KNTT -Tên GV]%[1K2Y1-1]
	Xét dãy số gồm tất cả các số tự nhiên chia cho $5$ dư $1$. Xác định số hạng tổng quát của dãy số.
	\dapso{$u_n=5n+1$}
	\loigiai{Các số tự nhiên chia cho $5$ dư $1$ gồm các số sau:
		$6,11,16,21, \ldots $\\
		Số hạng tổng quát $u_n=5n+1$.}
\end{bt}
% \begin{bt}[NB]%[DCHT Toán 11 - KNTT -Tên GV] %[1K2Y1-1]
% 	Tìm năm số hạng đầu và số hạng thứ $100$ của dãy $(u_n)$ có số hạng tổng quát $u_n= \dfrac{(-1)^n}{n}$.
% 	\dapso{$u_1=-1, \dfrac{1}{2},-\dfrac{1}{3}, \dfrac{1}{4}, -\dfrac{1}{5} $, $u_{100}=\dfrac{1}{100}$}
% 	\loigiai{
% 		Năm số hạng đầu của dãy là $u_1=-1, \dfrac{1}{2},-\dfrac{1}{3}, \dfrac{1}{4}, -\dfrac{1}{5} $.\\
% 		Số hạng thứ $100$ là $u_{100}=\dfrac{1}{100}$.}
% \end{bt}
\begin{bt}[NB]%[DCHT Toán 11 - KNTT -Tên GV]%[1K2Y1-1]
	Viết năm số hạng đầu của dãy số gồm các số nguyên tố theo thứ tự tăng dần.
	\dapso{$2,3,5,7,11$}
	\loigiai {Năm số hạng đầu của dãy số trên là $2,3,5,7,11$.} 
\end{bt}
% \begin{bt}[NB]%[DCHT Toán 11 - KNTT-Tên GV]%[1K2Y1-1]
% 	Viết năm số hạng đầu của dãy $(u_n)$ với số hạng tổng quát là $u_n=n!$.
% 	\dapso{$1,2,6,24,120$}
% 	\loigiai{Năm số hạng đầu của dãy trên là $1,2,6,24,120$.}
% \end{bt}
\subsubsection{Câu hỏi trắc nghiệm}
\Opensolutionfile{ans}[ans/ans-1K2-1-Dang1]
%Cau1
\begin{ex}%[DCHT Toán 11 - KNTT -Tên GV]%[1K2B1-1]
	Cho dãy số có các số hạng đầu là $5,10,15,20,25, \ldots$ Số hạng tổng quát của dãy số này là
	\choice
	{$ u_n=5(n-1) $}
	{\True$ u_n=5n $}
	{$ u_n=5+n $}
	{$ u_n=5n+1 $}
	\loigiai
	{Ta có $5=5\cdot 1, 10=5 \cdot 2, 15 = 5\cdot 3, 20=5 \cdot 4, 25 = 5\cdot 5, \ldots$\\
		Vậy dãy trên có số hạng tổng quát là $u_n=5n$.
	}
\end{ex}
%Cau2
\begin{ex}%[DCHT Toán 11 - KNTT -Tên GV]%[1K2B1-1]
	Cho dãy số $(u_n)$ với $u_n=\dfrac{an^2}{n+1}$, $a$ là hằng số. $u_{n+1}$ là số hạng nào trong các số hạng sau
	\choice
	{\True $u_{n+1}=\dfrac{a(n+1)^2}{n+2} $}
	{$u_{n+1}=\dfrac{a(n+1)^2}{n+1}$}
	{$u_{n+1}=\dfrac{an^2+1}{n+1}$}
	{$u_{n+1}=\dfrac{an^2}{n+2} $}
	\loigiai
	{Ta có $u_{n+1}=\dfrac{a(n+1)^2}{n+1+1}=\dfrac{a(n+1)^2}{n+2}$.
	}
\end{ex}
%cau3
\begin{ex}%[DCHT Toán 11 - KNTT -Tên GV]%[1K2B1-1]
	Cho dãy số có các số hạng đầu là $8,15,22,29,36, \ldots$ Số hạng tổng quát của dãy số này là
	\choice
	{$ u_n=7n+7 $}
	{$ u_n=7n $}
	{\True $ u_n=7n+1 $}
	{$ u_n$ không viết được dưới dạng công thức }
	\loigiai
	{Ta có $8=7\cdot 1+1, 15=7 \cdot 2+1, 22 = 7\cdot 3+1, 29=7 \cdot 4+1, 36 = 7\cdot 5+1, \ldots$\\
		Vậy dãy trên có số hạng tổng quát là $u_n=7n+1$.
	}
\end{ex}
%Cau4
\begin{ex}%[DCHT Toán 11 - KNTT -Tên GV]%[1K2B1-1]
	Cho dãy số có các số hạng đầu là $0,\dfrac{1}{2},\dfrac{2}{3},\dfrac{3}{4},\dfrac{4}{5}, \ldots$ Số hạng tổng quát của dãy số này là
	\choice
	{$ u_n=\dfrac{n+1}{n}$}
	{\True $ u_n=\dfrac{n}{n+1} $}
	{$ u_n=\dfrac{n-1}{n}$}
	{$ u_n=\dfrac{n^2-n}{n+1}$  }
	\loigiai
	{Ta có $0=\dfrac{0}{0+1}, \dfrac{1}{2}=\dfrac{1}{1+1} ,\dfrac{2}{3} = \dfrac{2}{2+1}, \dfrac{3}{4}=\dfrac{3}{3+1}, \dfrac{4}{5} = \dfrac{4}{4+1}, \ldots$\\
		Vậy dãy trên có số hạng tổng quát là $u_n=\dfrac{n}{n+1}$.
	}
\end{ex}
%Cau5
\begin{ex}%[DCHT Toán 11 - KNTT -Tên GV]%[1K2B1-1]
	Cho dãy số $(u_n)$ với $u_1=1, u_{n+1}=u_n-1$. Số hạng tổng quát $u_n$ của dãy số là số hạng nào dưới đây?
	\choice
	{\True $ u_n=2-n$}
	{$ u_n$ không xác định}
	{$ u_n=1-n$}
	{$ u_n=-n$, với mọi $n$ }
	\loigiai
	{Ta có $u_1=1, u_2=0 ,u_3 = -1, u_4=-2,  \ldots$\\
		Dễ dàng dự đoán được số hạng tổng quát là $u_n=2-n$.
	}
\end{ex}
% %cau6
% \begin{ex}%[DCHT Toán 11 - KNTT -Tên GV]%[1K2B1-1]
% 	Cho dãy số $(u_n)$ với $u_n=\dfrac{2n^2-1}{n^2+3}, \forall n \in \mathrm{N*}$. Số hạng đầu tiên của dãy số là 
% 	\choice
% 	{$ u_1=-\dfrac{1}{3}$}
% 	{$ u_1=\dfrac{2}{3}$}
% 	{$ u_1=\dfrac{1}{3}$}
% 	{\True $ u_1=\dfrac{1}{4}$ }
% 	\loigiai
% 	{Ta có $u_1=\dfrac{2\cdot 1^2-1}{1^2+3}=\dfrac{1}{4}$.
% 	}
% \end{ex}
% %cau7
% \begin{ex}%[DCHT Toán 11 - KNTT -Tên GV]%[1K2B1-1]
% 	Cho dãy số $(u_n)$ với $u_1=-1, u_{n+1}=u_n+3$ với $n \ge 1$. Ba số hạng đầu tiên của dãy số lần lượt là 
% 	\choice
% 	{\True $-1, 2, 5$}
% 	{$ 1, 4, 7$}
% 	{$ 4,7,10$}
% 	{$-1,3,7$ }
% 	\loigiai
% 	{Ta có $u_1=-1, u_2=-1+3=2 ,u_3 = 2+3=5$.
% 	}
% \end{ex}
\Closesolutionfile{ans}
\begin{indapan}{10}
	{ans/ans-1K2-1-Dang1}
\end{indapan}

\begin{dang}{Tìm số hạng cụ thể của dãy số}
	Để tìm số hạng cụ thể của dãy số ta làm như sau
	\begin{itemize} 
		\item Với trường hợp dãy số đã cho biết công thức tổng quát của dãy số thì ta chỉ cần thay giá trị tương ứng của số hạng đó vào công thức tổng quát.
		\item  Với trường hợp dãy số cho bởi công thức truy hồi hoặc dưới dạng thì ta phải tìm lần lượt từ những số hạng đầu tiên cho đến số đứng trước số cần tìm trong dãy.
	\end{itemize}
\end{dang}
\subsubsection{Ví dụ minh hoạ}
\begin{vd}[NB]%[1K2Y5-2]
	Cho dãy số $(u_n),$ biết $u_n=(-1 )^n\cdot \dfrac{2^n}{n}$. Tìm số hạng $u_3$.
	\dapso{$u_3=-\dfrac{8}{3}$}
	\choice
	{\True $u_3=-\dfrac{8}{3}$}
	{$u_3=2$}
	{$u_3=-2$}
	{$u_3=\dfrac{8}{3}$}
	\loigiai{
		Ta có
		$$u_3=(-1)^3\cdot \dfrac{2^3}{3}=-\dfrac{8}{3}.$$}
\end{vd}
\begin{vd}[NB]%[DCHT Toán 11 - KNTT -Nguyễn Long]%[1K2Y5-2]
	Cho dãy số $(u_n)$, biết $u_n=\dfrac{2n^2-1}{n^2+3}$. Tìm số hạng $u_5$.
	\dapso{$u_5=\dfrac{7}{4}$}
	\choice
	{$u_5=\dfrac{1}{4}$}
	{\True $u_5=\dfrac{7}{4}$}
	{$u_5=\dfrac{17}{12}$}
	{$u_5=\dfrac{71}{39}$}
	\loigiai{
		Ta có $u_5=\dfrac{2\cdot 5^2-1}{5^2+3}=\dfrac{49}{28}=\dfrac{7}{4}$.}
	
\end{vd}
\begin{vd}[NB]%[1K2Y5-2]
	Cho dãy số $u_n$ bao gồm các số nguyên tố. Tìm số hạng thứ $5$ của dãy số.
	\dapso{$u_5=11$}
	\loigiai{ 
		Ta có
		$u_1=2,u_2=3,u_3=5,u_4=7,u_5=11$. \\
		Vậy số hạng thứ $5$ của dãy số là $11$.
	}
\end{vd}
\begin{vd}[NB]%[1K2Y5-2]
	Cho dãy số $(u_n) $ thỏa mãn $ \heva{& u_1 = 5 \\& u_{n+1} = u_n+n}$. Tìm số hạng thứ $5$ của dãy số.
	\dapso{$u_5=15$}
	\choice
	{$ 11 $}
	{\True $ 15 $}
	{$ 16 $}
	{$ 12 $}
	\loigiai{
		Ta có $ u_2=u_1+1=6$, $ u_3=u_2+2=8$, $ u_4=u_3+3=11$,  $ u_5=u_4+4=15$.
	}
\end{vd}

\begin{vd}[TH]%[VD 5 SGK KNTT]%[1K2B5-2]
	Cho dãy số xác định bằng hệ thức truy hồi
	$$
	u_1=1, u_n=3 u_{n-1}+2 \text { với } n \geq 2
	$$
	Viết ba số hạng đầu của dãy số này.
	\dapso{$u_5=17$}
	\loigiai{
		Ta có: $u_1=1, u_2=3 u_1+2=3 \cdot 1+2=5, u_3=3 u_2+2=3 \cdot 5+2=17$.
	}
\end{vd}

\begin{vd}[VD]%[1K2B5-2]
	Cho dãy số $\left(u_n\right)\colon\heva{&u_1=5 \\ &u_{n+1}=u_n+n}$. Số $20$ là số hạng thứ mấy trong dãy?
	\dapso{số hạng thứ $6$}
	\loigiai{
		Ta có $u_1=5, u_2=6, u_3=8, u_4=11, u_5=16, u_6=20$.\\
		Vậy số $20$ là số hạng thứ $6$.}
\end{vd}

\subsubsection{Bài tập tự luận}
 
\begin{bt}[NB]%[1K2Y5-2]
	Cho dãy số $u_n=\dfrac{1}{\sqrt{n}+1}$. Tìm số hạng $u_4$.	
	\dapso{$u_4=\dfrac{1}{3}$}
	\loigiai{ Ta có
		$u_4=\dfrac{1}{\sqrt{4}}+1=\dfrac{1}{3}.$		
	}
\end{bt}
%%%
\begin{bt}[NB]%[1K2Y5-2]
	Cho dãy số $(u_n)$ có số hạng tổng quát: $u_n=2 n+\sqrt{n^2+4}$. Tìm số hạng thứ $6$ của dãy số.
	\dapso{$u_6=12+2\sqrt{10}$}
	\loigiai{
		Ta có $u_6=12+2 \sqrt{10}$.
	}
\end{bt}
%%%
\begin{bt}[NB]%[1K2Y5-2]
	Cho dãy số $(u_n)$ xác định bởi: $\heva{&u_1=-1 ; u_2=3 \\&u_{n+1}=5 u_n-6 u_{n-1} \forall n \geq 2}.$ Tìm số hạng thứ $7$ của dãy.
	\dapso{$3261$}
	\loigiai{
		Ta có
		$$
		u_3=5 u_2-6 u_1=21 ;~ u_4=5 u_3-6 u_2=87 ;~ u_5=309 ;~ u_6=1023 ;~ u_7=3261
		$$
		Vậy số hạng thứ $7$ của dãy là $3261$.
	}
\end{bt}
%%%%%%%
\begin{bt}[NB]%[1K2Y5-2]
	Viết năm số hạng đầu của dãy số Fibonacci $\left(F_n\right)$ cho bởi hệ thức truy hồi
	$$
	\heva{
		&F_1=1, F_2=1 \\
		&F_n=F_{n-1}+F_{n-2}~(n \geq 3) .
	}
	$$
	\dapso{$F_3=2,~F_4=3,~F_5=5$}
	\loigiai{
		Ta có $F_3=2,~F_4=3,~F_5=5.$
	}
\end{bt}
%%%
\begin{bt}[NB]%[1K2T5-2]
	Người ta nuôi cấy $5$ con vi khuẩn E-coli trong môi trường nhân tạo. Cứ $30$ phút thì vi khuẩn E-coli sẽ nhân đôi 1 lần. Tính số lượng vi khuẩn thu được sau $1,2,3$ lần nhân đôi.
	\dapso{$u_2=10, u_3=20, u_4=40$}
	\loigiai{
		Đặt $u_1=5$, gọi số vi khuẩn sau $n$ lần phân chia là $u_{n+1}$, khi đó ta có dãy số $(u_n)$ thỏa mãn $$u_1=5, \; u_{n+1}=2u_n$$
		Ta có $u_2=10, u_3=20, u_4=40$.
	}	
\end{bt}
%%%%%
\begin{bt}[TH]%[1K2B5-2]
	Cho dãy số $(u_n)$ được xác định bởi $u_n=\dfrac{n^2+3n+7}{n+1}$.
	\begin{listEX}
		\item Viết năm số hạng đầu của dãy.
		\item Dãy số có bao nhiêu số hạng nhận giá trị nguyên.
	\end{listEX}
	\dapso{$u_1=\dfrac{11}{2}$; $u_2=\dfrac{17}{3}$; $u_3=\dfrac{25}{4}$; $u_4=7$; $u_5=\dfrac{47}{6}$. $u_4=7 $}
	\loigiai{		
		\begin{listEX}
			\item Ta có năm số hạng đầu của dãy
			$u_1=\dfrac{1^2+3.1+7}{1+1}=\dfrac{11}{2}$; $u_2=\dfrac{17}{3}$; $u_3=\dfrac{25}{4}$; $u_4=7$; $u_5=\dfrac{47}{6}$.
			\item Ta có: $u_n=n+2+\dfrac{5}{n+1}$, do đó $u_n$ nguyên khi và chỉ khi $ \dfrac{5}{n+1}$ nguyên hay $ n+1 $ là ước của 5. Điều đó xảy ra khi $ n+1=5\Leftrightarrow n=4 $. Vậy dãy số có duy nhất một số hạng nguyên là $u_4=7 $.
		\end{listEX}		
	}
\end{bt}
\begin{bt}[VD]%[1K2K5-2]
	Cho dãy số $\left(x_n\right)$ thỏa mãn điều kiện $x_1=1, x_{n+1}-x_n=\dfrac{1}{n(n+1)}, n=1,2,3, \ldots$. Số hạng $x_{2023}$ bằng
	\dapso{$x_{2023}=\dfrac{4045}{2023}$}
	\loigiai{
		Ta có
		$$
		\begin{aligned}
			x_{n+1}-x_n=\dfrac{1}{n(n+1)}=\dfrac{1}{n}-\dfrac{1}{n+1} & \Leftrightarrow \sum_{k=1}^{n-1}\left(x_{k+1}-x_k\right)=\sum_{k=1}^{n-1}\left(\dfrac{1}{k}-\dfrac{1}{k+1}\right) \\
			& \Leftrightarrow x_n-x_1=1-\dfrac{1}{n} \\
			& \Leftrightarrow x_n=\dfrac{2n-1}{n} .
		\end{aligned}
		$$
	}
\end{bt}
\begin{bt}[VDC]%[1K2G5-2]
	Cho dãy số $\left(u_n\right)$ biết $\heva{&u_1=99 \\&u_{n+1}=u_n-2 n-1, n \geq 1}$. Hỏi số $-861$ là số hạng thứ mấy?
	\dapso{$-861$ là số hạng thứ $31$}
	\loigiai{
		Ta có
		$$
		\begin{aligned}
			&u_n &=& &u_{n-1}-2 n+1 \\
			&u_{n-1} & = & &u_{n-2}-2 n+3 \\
			&\vdots &\vdots&  &\vdots \\
			&u_3 & = & &u_2-2 n+2 n-5 \\
			&u_2 & = & &u_1-2 n+2 n-3
		\end{aligned}
		$$
		Suy ra
		$$
		\begin{aligned}
			& u_n=u_1-2 n \cdot(n-1)+1+3+5+\cdots+(2 n-5)+(2 n-3) \\
			& u_n=99-2 n^2+2 n+\dfrac{n-1}{2}\cdot[2 \cdot 1+(n-2) \cdot 2]=100-n^2
		\end{aligned}
		$$
		Giả sử $u_n=-861 \Rightarrow n^2=961 \Rightarrow n=31$ (vì $n \in \mathbb{N}$).
		Vậy số $-861$ là số hạng thứ $31$ .}
\end{bt}
\subsubsection{Câu hỏi trắc nghiệm}
\Opensolutionfile{ans}[ans/ans-1K2-1-Dang2]
%Câu 1
\begin{ex}%[1K2Y5-2]
	Cho dãy số $({{u}_{n}} )$, biết ${{u}_{n}}=\dfrac{n}{{{3}^{n}}-1}$. Ba số hạng đầu tiên của dãy số đó lần lượt là những số nào dưới đây?
	\choice
	{$\dfrac{1}{2};\dfrac{1}{4};\dfrac{1}{16}$}
	{$\dfrac{1}{2};\dfrac{2}{3};\dfrac{3}{4}$}
	{ \True $\dfrac{1}{2};\dfrac{1}{4};\dfrac{3}{26}$}
	{$\dfrac{1}{2};\dfrac{1}{4};\dfrac{1}{8}$}
	\loigiai {
		Ta có
		${{u}_{1}}=\dfrac{1}{2};\,\,{{u}_{2}}=\dfrac{2}{{{3}^2}-1}=\dfrac{2}{8}=\dfrac{1}{4};\,\,{{u}_{3}}=\dfrac{3}{{{3}^3}-1}=\dfrac{3}{26}$.}
	
\end{ex}
%%%%%%%%%%
%Câu 2
\begin{ex}%[1K2Y5-2]
	Cho dãy số $({{u}_{n}} ),$ biết ${{u}_{n}}={{(-1 )}^{n}}\cdot 2n$. Mệnh đề nào sau đây {\bf sai}?
	\choice
	{${{u}_{3}}=-6$}
	{${{u}_{2}}=4$}
	{ \True ${{u}_{4}}=-8$}
	{${{u}_{1}}=-2$}
	\loigiai {
		Ta có\\
		${{u}_{1}}=-2\cdot 1=-2;\,\,{{u}_{2}}={{(-1 )}^2}\cdot 2\cdot 2=4,\,\,{{u}_{3}}={{(-1 )}^3}\cdot 2\cdot 3=-6;\,\,{{u}_{4}}={{(-1 )}^4}\cdot 2\cdot 4=8$.\\
		\textbf{Nhận xét:} Dễ thấy ${{u}_{n}}>0$ khi $n$ chẵn và ngược lại nên đáp án $u_4=-8$ sai.}
	
\end{ex}
%%%%%%%%%%
%Câu 3
\begin{ex}%[1K2Y5-2]
	Cho dãy số $({{u}_{n}} )$ xác định bởi $\heva{
		& {{u}_{1}}=2 \\
		& {{u}_{n+1}}=\dfrac{1}{3}({{u}_{n}}+1 ) \\
	}.$ Tìm số hạng ${{u}_{4}}$.
	\choice
	{${{u}_{4}}=\dfrac{2}{3}$}
	{${{u}_{4}}=1$}
	{${{u}_{4}}=\dfrac{14}{27}$}
	{ \True ${{u}_{4}}=\dfrac{5}{9}$}
	\loigiai {
		Ta có
		${{u}_{2}}=\dfrac{1}{3}({{u}_{1}}+1 )=\dfrac{1}{3}(2+1 )=1;\,\,{{u}_{3}}=\dfrac{1}{3}({{u}_{2}}+1 )=\dfrac{2}{3};\,\,{{u}_{4}}=\dfrac{1}{3}({{u}_{3}}+1 )=\dfrac{1}{3}\cdot\left(\dfrac{2}{3}+1\right)=\dfrac{5}{9}$. \\
	}
\end{ex}
%%%%%%%%%%
%Câu 4
\begin{ex}%[1K2Y5-2]
	Cho dãy số $({{u}_{n}} )$, biết $\heva{
		& {{u}_{1}}=-1 \\
		& {{u}_{n+1}}={{u}_{n}}+3 \\
	}$ với $n\ge 0$. Ba số hạng đầu tiên của dãy số đó là lần lượt là những số nào dưới đây?
	\choice
	{\True $-1;\,2;\,5$}
	{$-1;3;7$}
	{$1;\,4;\,7$}
	{$4;\,7;\,10$}
	\loigiai {
		Ta có ${{u}_{1}}=-1;\,\,{{u}_{2}}={{u}_{1}}+3=2;\,\,{{u}_{3}}={{u}_{2}}+3=5$. \\
		\textbf{Nhận xét.} (i) Dùng chức năng “lặp” của MTCT để tính:\\
		Nhập vào màn hình: $X=X+3$ \\
		Bấm CALC và cho $X=-1$ (ứng với ${{u}_{1}}=-1)$ \\
		Để tính ${{u}_{n}}$ cần bấm “=” ra kết quả liên tiếp $n-1$ lần. Ví dụ để tính ${{u}_{2}}$ ta bấm “=” ra kết quả lần đầu tiên, bấm “=” ra kết quả thứ hai chính là ${{u}_{3}},\ldots$\\
		(ii) Vì ${{u}_{1}}=-1$ nên loại các đáp án $u_1=1$, $u_1=4$.\\
		Còn lại các đáp án có $u_1=-1$; để biết đáp án nào ta chỉ cần kiểm tra ${{u}_{2}}$ (vì ${{u}_{2}}$ ở hai đáp án là khác nhau): ${{u}_{2}}={{u}_{1}}+3=2$.
	}
	
\end{ex}
%%%%%%%%%%
%Câu 5
\begin{ex}%[1K2B5-2]
	Cho dãy số $({{u}_{n}} ),$ biết ${{u}_{n}}=\dfrac{2n+5}{5n-4}$. Số $\dfrac{7}{12}$ là số hạng thứ mấy của dãy số?
	\choice
	{$9$}
	{$6$}
	{$10$}
	{\True $8$}
	\loigiai {
		Ta có
		$${{u}_{n}}=\dfrac{2n+5}{5n-4}=\dfrac{7}{12}\Leftrightarrow 24n+60=35n-28\Leftrightarrow 11n=88\Leftrightarrow n=8.$$}
	
\end{ex}
%%%%%%%%%%
%Câu 6
\begin{ex}%[1K2B5-2]
	Cho dãy $(u_n)$ xác định bởi $\heva{& u_1=3 \\& u_{n+1}=\dfrac{u_n}{2}+2}$. Mệnh đề nào sau đây {\bf sai}?
	\choice
	{\True ${{u}_{2}}=\dfrac{5}{2}$}
	{${{u}_{4}}=\dfrac{31}{8}$}
	{${{u}_{3}}=\dfrac{15}{4}$}
	{${{u}_{5}}=\dfrac{63}{16}$}
	\loigiai {
		Ta có $\heva{
			& {{u}_{2}}=\dfrac{{{u}_{1}}}{2}+2=\dfrac{3}{2}+2=\dfrac{7}{2};\,\,{{u}_{3}}=\dfrac{{{u}_{2}}}{2}+2=\dfrac{7}{4}+2=\dfrac{15}{4}. \\
			& {{u}_{4}}=\dfrac{{{u}_{3}}}{2}+2=\dfrac{15}{8}+2=\dfrac{31}{8};\,\,{{u}_{5}}=\dfrac{{{u}_{4}}}{2}+2=\dfrac{31}{16}+2=\dfrac{63}{16}. \\
		}$}
\end{ex}
%%%%%%%%%%
%Câu 7
\begin{ex}%[1K2B5-2]
	Cho dãy số $({{u}_{n}} ),$ với ${{u}_{n}}={{\left(\dfrac{n-1}{n+1} \right)}^{2n+3}}$. Tìm số hạng ${{u}_{n+1}}$.
	\choice
	{${{u}_{n+1}}={{\left(\dfrac{n-1}{n+1} \right)}^{2(n-1 )+3}}$}
	{${{u}_{n+1}}={{\left(\dfrac{n-1}{n+1} \right)}^{2(n+1 )+3}}$ }
	{\True ${{u}_{n+1}}={{\left(\dfrac{n}{n+2} \right)}^{2n+5}}$}
	{${{u}_{n+1}}={{\left(\dfrac{n}{n+2} \right)}^{2n+3}}$}
	\loigiai {
		${{u}_{n}}={{\left(\dfrac{n-1}{n+1} \right)}^{2n+3}}\Rightarrow {{u}_{n+1}}={{\left(\dfrac{(n+1 )-1}{(n+1 )+1} \right)}^{2(n+1 )+3}}={{\left(\dfrac{n}{n+2} \right)}^{2n+5}}$.}
	
\end{ex}
%%%%%%%%%%
%Câu 8
\begin{ex}%[1K2K5-2]
	Cho dãy số $({{a}_{n}} ),$ được xác định $\heva{
		& {{a}_{1}}=3 \\
		& {{a}_{n+1}}=\dfrac{1}{2}{{a}_{n}},~n\ge 1 \\
	}$. Mệnh đề nào sau đây {\bf sai}?
	\choice
	{${{a}_{1}}+{{a}_{2}}+{{a}_{3}}+{{a}_{4}}+{{a}_{5}}=\dfrac{93}{16}$}
	{${{a}_{10}}=\dfrac{3}{512}$}
	{ \True ${{a}_{n}}=\dfrac{3}{{{2}^{n}}}$}
	{${{a}_{n+1}}+{{a}_{n}}=\dfrac{9}{{{2}^{n}}}$}
	\loigiai {
		Ta có ${{a}_{1}}=3;\,{{a}_{2}}=\dfrac{{{u}_{1}}}{2};\,\,{{a}_{3}}=\dfrac{{{u}_{2}}}{2}=\dfrac{{{u}_{1}}}{{{2}^2}};\,\,{{a}_{4}}=\dfrac{{{u}_{3}}}{2}=\dfrac{{{u}_{1}}}{{{2}^3}},\ldots \\
		\Rightarrow {{u}_{n}}=\dfrac{{{u}_{1}}}{{{2}^{n-1}}}=\dfrac{3}{{{2}^{n-1}}}$ nên suy ra đáp án ${{a}_{n}}=\dfrac{3}{{{2}^{n}}}$ sai. \\
		Xét đáp án\\
		${{a}_{1}}+{{a}_{2}}+{{a}_{3}}+{{a}_{4}}+{{a}_{5}}=3\left(1+\dfrac{1}{2}+\dfrac{1}{{{2}^2}}+\dfrac{1}{{{2}^3}}+\dfrac{1}{{{2}^4}}\right)=3.\dfrac{1-{{(\dfrac{1}{2} )}^5}}{1-\dfrac{1}{2}}=\dfrac{93}{16}\Rightarrow $  đúng.\\
		Xét đáp án ${{a}_{10}}=\dfrac{3}{{{2}^{9}}}=\dfrac{3}{512}\Rightarrow $  đúng.\\
		Xét đáp án ${{a}_{n+1}}+{{a}_{n}}=\dfrac{3}{{{2}^{n}}}+\dfrac{3}{{{2}^{n-1}}}=\dfrac{3+3\cdot 2}{{{2}^{n}}}=\dfrac{9}{{{2}^{n}}}\Rightarrow $ đúng.}
	
\end{ex}
%%%%%%%%%%
%Câu 9
\begin{ex}%[1K2K5-2]
	Cho dãy số $(u_n)$ biết $\heva{&u_1=1\\&u_2=4\\&u_{n+2}=3u_{n+1}-2u_n}$ với mọi $n \ge 1$. Giá trị $u_{101}-u_{100}$ là 
	\choice
	{$3\cdot 2^{102} $}
	{$3\cdot 2^{101} $}
	{$3\cdot 2^{100} $}
	{\True $ 3\cdot 2^{99}$}
	\loigiai{
		Theo bài  ta có 
		\begin{eqnarray*}
			&u_{n+2}=3u_{n+1}-2u_n\\
			\Leftrightarrow \,& u_{n+2}=u_{n+1}+2(u_{n+1}-u_n)\\
			\Leftrightarrow \,& u_{n+2}-u_{n+1}=2(u_{n+1}-u_n).
		\end{eqnarray*}
		Với $n=99$ ta có 
		\begin{align*}
			u_{101}-u_{100}&=2(u_{100}-u_{99})\\
			&=2\cdot 2 (u_{99}-u_{98})\\
			&= \ldots\\
			&=2^{99}\cdot(u_2-u_1)=3\cdot2^{99}.
		\end{align*}
	}
\end{ex}
%%%%%%%%%%
%Câu 10
\begin{ex}%[1K2G5-2]
	Cho dãy số $\left(u_n\right)$ thoả mãn $u_1=\sqrt{2}$ và $u_{n+1}=\sqrt{2+u_n}$ với mọi $n\geq 1$. Tìm $u_{2023}$.
	\choice
	{$u_{2023}=\sqrt{2}\cos\dfrac{\pi}{2^{2022}}$}
	{\True $u_{2023}=\sqrt{2}\cos\dfrac{\pi}{2^{2024}}$}
	{$u_{2023}=\sqrt{2}\cos\dfrac{\pi}{2^{2023}}$}
	{$u_{2023}=2$}
	\loigiai{Ta chứng minh bằng phương pháp quy nạp số hạng tổng quát của dãy là $u_n=2\cos\dfrac{\pi}{2^{n+1}}$.\\
		Dễ thấy, với $n=1$ ta có $u_1=\sqrt{2}$ (đúng).\\
		Giả sử mệnh đề đúng với $n=k, \forall k\in \mathbb{N}^\ast$ nghĩa là $u_k=2\cos\dfrac{\pi}{2^{k+1}}$ ta phải chứng minh mệnh đề đúng với $n=k+1$ nghĩa là $u_{k+1}=2\cos\dfrac{\pi}{2^{k+2}}$.\\
		Thật vậy, $u_{k+1}=\sqrt{2+u_k}=\sqrt{2+2\cos\dfrac{\pi}{2^{k+1}}}=\sqrt{4\cos^2\dfrac{\pi}{2^{k+2}}}=2\cos\dfrac{\pi}{2^{k+2}}$.\\
		Áp dụng công thức tổng quát trên ta có $u_{2023}=\sqrt{2}\cos\dfrac{\pi}{2^{2024}}$.
	}
\end{ex}
%%%%%%%%%%
\Closesolutionfile{ans}
\begin{indapan}{10}
	{ans/ans-1K2-1-Dang2}
\end{indapan}
\begin{dang}{Xét tính tăng giảm của dãy số}
	\begin{enumerate}
		\item Phương pháp 1. Xét dấu của hiệu số $u_{n+1}-u_n$.
		\begin{enumerate}
			\item Nếu $u_{n+1}-u_n>0, \forall n \in \mathbb{N}^\ast$ thì $(u_n)$ là dãy số tăng.
			\item Nếu $u_{n+1}-u_n<0, \forall n \in \mathbb{N}^\ast$ thì $(u_n)$ là dãy số giảm.
		\end{enumerate}
		\item Phương pháp 2. Nếu $u_n>0, \forall n\in \mathbb{N}^\ast$ thì ta có thể so sánh thương $\dfrac{u_{n+1}}{u_n}$ với $1$.
		\begin{enumerate}
			\item Nếu $\dfrac{u_{n+1}}{u_n}>1$ thì $(u_n)$ là dãy số tăng.
			\item Nếu $\dfrac{u_{n+1}}{u_n}<1$ thì $(u_n)$ là dãy số giảm.
		\end{enumerate}
		Nếu $u_n<0, \forall n\in \mathbb{N}^\ast$ thì ta có thể so sánh thương $\dfrac{u_{n+1}}{u_n}$ với $1$.
		\begin{enumerate}
			\item Nếu $\dfrac{u_{n+1}}{u_n}<1$ thì $(u_n)$ là dãy số tăng.
			\item Nếu $\dfrac{u_{n+1}}{u_n}>1$ thì $(u_n)$ là dãy số giảm.
		\end{enumerate}
		\item Phương pháp 3. Nếu dãy số $(u_n)$ cho bởi hệ thức truy hồi thì thường dùng phương pháp quy nạp để chứng minh $u_{n+1}>u_n, \forall n \in \mathbb{N}^\ast$ (hoặc $u_{n+1}<u_n \forall n \in \mathbb{N}^\ast$).
	\end{enumerate}
\end{dang}
\subsubsection{Ví dụ minh hoạ}
\begin{vd}[NB]%[1K2Y5-3]
	Xét sự tăng giảm của dãy số $(u_n)$ với $u_n=(-1)^n$.
	\dapso{dãy không tăng không giảm}
	\loigiai{
		Ta có:\\ $u_1=(-1)^1=-1,\,
		u_2=(-1)^2=1,\,
		u_3=(-1)^3=-1.$\\
		Vậy $(u_n)$ là dãy không tăng không giảm.
	}
\end{vd}
\begin{vd}[NB]%[1K2Y5-3]
	Xét tính tăng giảm của dãy số sau $(u_n)$ với $u_n=\dfrac{2n+1}{n+1}$.
	\dapso{dãy số tăng}
	\loigiai
	{
		Ta có: $u_n=\dfrac{2n+1}{n+1}=2-\dfrac{1}{n+1}$.\\
		$u_{n+1}-u_n=\left(2-\dfrac{1}{n+1+1}\right)-\left(2-\dfrac{1}{n+1}\right)=\dfrac{1}{n+1}-\dfrac{1}{n+2}>0, \forall n \in \mathbb{N}^\ast$.\\
		Vậy dãy số $(u_n)$ là dãy số tăng.
	}
\end{vd}
\begin{vd}[TH]%[1K2B5-3]
	Xét tính tăng giảm của dãy số $(u_n)$ với $u_n=\sqrt{n}-\sqrt{n+2}$.
	\dapso{dãy số tăng}
	\loigiai{
		Ta có $u_n=\sqrt{n}-\sqrt{n+2}=\dfrac{-2}{\sqrt{n}+\sqrt{n+2}}$.\\
		Xét hiệu\\ 
		$\begin{aligned}
			u_{n+1}-u_n&=\dfrac{-2}{\sqrt{n+1}+\sqrt{n+3}}-\dfrac{-2}{\sqrt{n}+\sqrt{n+2}}\\
			&=\dfrac{2}{\sqrt{n}+\sqrt{n+2}}-\dfrac{2}{\sqrt{n+1}+\sqrt{n+3}}>0, \forall n\in \mathbb{N}^\ast.
		\end{aligned}$\\
		Vậy $(u_n)$ là dãy số tăng.
	}
\end{vd}

\begin{vd}[TH]%[1K2B5-3]
	Xét tính tăng giảm của dãy số $(u_n)$ với $u_n=\dfrac{n}{3^n}$.
	\dapso{dãy số giảm}
	\loigiai{
		Ta có $u_n=\dfrac{n}{3^n}>0, \forall n \in \mathbb{N}^\ast$.\\
		Xét thương $\dfrac{u_{n+1}}{u_n}=\dfrac{n+1}{3^{n+1}}:\dfrac{n}{3^n}=\dfrac{n+1}{3.n}<1, \forall  n \in \mathbb{N}^\ast$.\\
		Vậy $(u_n)$ là dãy số giảm.
	}
\end{vd}
\begin{vd}[VD]%[1K2K5-3]
	Xét tính tăng giảm của dãy số $(u_n)$ với $ \heva{& u_1=2\\
		& u_{n+1}=\dfrac{3u_n+1}{u_n+1}, n\in \mathbb{N}^\ast.}$
	\dapso{dãy số tăng}
	\loigiai{
		Giả sử $u_{n+1}>u_n , \forall n \in \mathbb{N}^\ast. \qquad (*)$\\
		Ta chứng minh $(*)$ bằng phương pháp quy nạp.
		\begin{itemize}
			\item Với $n=1, u_2=\dfrac{3.2+1}{2+1}=\dfrac{6}{3}=\dfrac{7}{3}>u_1=2.$
			\item Giả sử $(*)$ đúng khi $n=k, k\in \mathbb{N}^\ast$, tức là $u_{k+1}>u_k$.\\
			Ta sẽ chứng minh $(*)$ đúng với $n=k+1$, tức là
			$u_{k+2}>u_{k+1}$.\\
			Thật vậy\\ $u_{k+2}-u_{k+1}=\left(3-\dfrac{2}{u_{k+1}+1}\right)-\left(3-\dfrac{2}{u_k+1}\right)=\dfrac{2}{u_k+1}-\dfrac{2}{u_{k+1}+1}.$\\
			Theo giả thiết quy nạp ta có:\\ $u_{k+1}>u_k \Rightarrow u_{k+1}+1>u_k+1 \Rightarrow \dfrac{2}{u_k+1}>\dfrac{2}{u_{k+1}+1}$.\\
			Vậy $u_{k+2}-u_{k+1}>0$.\\
			Do đó, $(*)$ đúng với mọi số nguyên dương $n$.
		\end{itemize}
		Vậy $(u_n)$ là dãy số tăng.}
\end{vd}
\subsubsection{Bài tập tự luận}
 
\begin{bt}[NB]%[1K2Y5-3]
	Xét tính tăng giảm của dãy số $(u_n)$ với $u_n=\dfrac{\sqrt{2}}{3^n}$.
	\dapso{dãy số giảm}
	\loigiai{
		Ta có $u_n>0, \forall n \in \mathbb{N}^\ast$.\\
		Xét thương $$\dfrac{u_{n+1}}{u_n}=\dfrac{\sqrt{2}}{3^{n+1}}: \dfrac{\sqrt{2}}{\sqrt{3^2}}=\dfrac{3^n}{3^{n+1}}=\dfrac{1}{3}<1.$$
		Vậy $\left(u_n\right)$ là dãy số giảm.
	}
\end{bt}
\begin{bt}[NB]%[1K2Y5-3]
	Xét tính tăng giảm của dãy số $\left(u_n\right)$ với $u_n=\dfrac{1}{n(n+1)}$.
	\dapso{dãy số tăng}
	\loigiai{
		Ta có $u_n=\dfrac{1}{n(n+1)}=\dfrac{1}{n}-\dfrac{1}{n+1}$.
		Xét hiệu:
		$$
		\begin{aligned}
			u_{n+1}-u_n & =\left(\dfrac{1}{n}-\dfrac{1}{n+1}\right)-\left(\dfrac{1}{n+1}-\dfrac{1}{n+2}\right) \\
			& =\dfrac{1}{n}-\dfrac{1}{n+2}>0, \forall n \in \mathbb{N}^\ast
		\end{aligned}
		$$
		Vậy  $\left(u_n\right)$ là dãy số tăng.
	}
\end{bt}
\begin{bt}[TH]%[1K2B5-3]
	Xét tính tăng giảm của dãy số $\left(u_n\right)$ với $u_n=n+\cos ^2 n$.
	\dapso{dãy số tăng}
	\loigiai{
		Xét hiệu
		$$
		\begin{aligned}
			u_{n+1}-u_n & =\left(n+1+\cos ^2(n+1)\right)-\left(n+\cos ^2 n\right) \\
			& =1+\cos ^2(n+1)-\cos ^2 n \\
			& =\cos ^2(n+1)+\sin ^2 n>0, \forall n \in \mathbb{N}^\ast .
		\end{aligned}
		$$
		Vậy $\left(u_n\right)$ là dãy số tăng.
	}
\end{bt}
\begin{bt}[TH]%[1K2B5-3]
	Xét tính tăng giảm của dãy số $(u_n)$ với $u_n=\dfrac{1}{n+1}+\dfrac{1}{n+2}+\ldots+\dfrac{1}{2n}$.
	\dapso{dãy số giảm}
	\loigiai{
		Xét hiệu\\
		$\begin{aligned}
			u_{n+1}-u_n&=\left(\dfrac{1}{n+2}+\dfrac{1}{n+3}+\ldots+\dfrac{1}{2(n+1)}\right)-\left(\dfrac{1}{n+1}+\dfrac{1}{n+2}+\ldots+\dfrac{1}{2n}\right)\\
			&=\dfrac{1}{n+2}-\dfrac{1}{2n+1}-\dfrac{1}{2n+2}\\
			&=\dfrac{1}{2n+2}-\dfrac{1}{2n+1}<0, \forall n\in \mathbb{N}^\ast.
		\end{aligned}$\\
		Vậy $(u_n)$ là dãy số giảm.
	}
	
\end{bt}

\begin{bt}[TH]%[1K2B5-3]
	Xét tính tăng giảm của dãy số $\left(u_n\right)$ với $u_n=\dfrac{1}{n+1}+\dfrac{1}{n+2}+\ldots+\dfrac{1}{2 n}$.
	\dapso{dãy số giảm}
	\loigiai{
		Xét hiệu
		$$
		\begin{aligned}
			u_{n+1}-u_n & =\left(\dfrac{1}{n+2}+\dfrac{1}{n+3}+\ldots+\dfrac{1}{2(n+1)}\right)-\left(\dfrac{1}{n+1}+\dfrac{1}{n+2}+\ldots+\dfrac{1}{2 n}\right) \\
			& =\dfrac{1}{n+2}-\dfrac{1}{2 n+1}-\dfrac{1}{2 n+2} \\
			& =\dfrac{1}{2 n+2}-\dfrac{1}{2 n+1}<0, \forall n \in \mathbb{N}^\ast
		\end{aligned}
		$$
		Vậy $\left(u_n\right)$ là dãy số giảm.
	}
\end{bt}
\begin{bt}[VD]%[1K2K5-3]
	Xét tính tăng giảm của dãy số $\left(u_n\right)$ cho bởi
	$$
	\left(u_n\right)\colon\heva{
		&u_1=1 ; u_2=2 \\
		&u_{n+1}=\sqrt{u_n}+\sqrt{u_{n-1}} \forall n \geq 2
	}
	$$
	\dapso{dãy số tăng}
	\loigiai{
		Ta chứng minh dãy $\left(u_n\right)$ là dãy tăng bằng phương pháp quy nạp.\\
		Dễ thấy $u_1<u_2<u_3$.\\
		Giả sử $u_{k-1}<u_k ~\forall k \geq 2$, ta chứng minh $u_{k+1}>u_k$.\\
		Thật vậy ta có $u_{k+1}=\sqrt{u_k}+\sqrt{u_{k-1}}>\sqrt{u_{k-1}}+\sqrt{u_{n-2}}=u_k$.\\ Vậy $\left(u_n\right)$ là dãy tăng.
	}
	
\end{bt}
\begin{bt}[VD]%[1K2K5-3]
	Cho dãy số $\left(u_n\right)$ biết $u_n=\dfrac{b \cdot2 n^2+1}{n^2+3}$ và $b \in \mathbb{R}$. Hãy xác định $b$ để
	\begin{listEX}[2]
		\item $\left(u_n\right)$ là dãy số giảm.
		\item $\left(u_n\right)$ là dãy số tăng.
	\end{listEX}
	\dapso{$b<\dfrac{1}{6}$ dãy số giảm; $b>\dfrac{1}{6}$ dãy số tăng}
	\loigiai{
		Ta có
		$$
		u_n=2 b+\dfrac{1-6 b}{n^2+3}
		$$
		Xét hiệu $$u_{n+1}-u_n=\dfrac{1-6 b}{(n+1)^2+3}-\dfrac{1-6 b}{n^2+3}=(1-6 b) \cdot\left(\dfrac{1}{(n+1)^2+3}-\dfrac{1}{n^2+3}\right)=A_n.$$
		\begin{listEX}
			\item Để $\left(u_n\right)$ là dãy sỗ giảm thì $A_n<0, \forall n \in \mathbb{N}^\ast$.
			$$
			A_n<0 \Leftrightarrow 1-6 b>0 \Leftrightarrow b<\dfrac{1}{6}
			$$
			\item Để $\left(u_n\right)$ là dãy số tăng thì $A_n>0, \forall n \in \mathbb{N}^\ast$.
			$$
			A_n>0 \Leftrightarrow 1-6 b<0 \Leftrightarrow b>\dfrac{1}{6}.
			$$
		\end{listEX}
		
	}
\end{bt}
\begin{bt}[VDC]%[1K2G5-3]
	Xét tính tăng giảm của dãy số $\left(u_n\right)$ với $u_n=\sin n+\cos n$.
	\dapso{dãy khōng tăng, không giảm}
	\loigiai{
		Ta có: $u_n=\sin n+\cos n=\sqrt{2} \sin \left(n+\dfrac{\pi}{4}\right)$.
		Xét hiệu
		$$
		\begin{aligned}
			u_{n+1}-u_n&=\sqrt{2} \sin \left(n+1+\dfrac{\pi}{4}\right)-\sqrt{2} \sin \left(n+\dfrac{\pi}{4}\right) \\
			&=2 \sqrt{2} \cdot \cos \left(2 n+\dfrac{1}{2}+\dfrac{\pi}{4}\right) \cdot \sin \dfrac{1}{2}=A_n . \\
		\end{aligned}
		$$
		Với  $n=1, A_1>0$. Với  $n=100, A_{100}<100$ . \\
		Vậy $\left(u_n\right)$ là dãy khōng tăng, không giảm.
	}
\end{bt}
\subsubsection{Bài tập trắc nghiệm}
\Opensolutionfile{ans}[ans/ans-1K2-1-Dang3]
%Câu 1
\begin{ex}%[1K2Y5-3]
	Cho các dãy số sau. Dãy số nào là dãy số tăng?
	\choice
	{$1;1;1;1;1;1;\ldots $}
	{$1;\dfrac{1}{2};\dfrac{1}{4};\dfrac{1}{8};\dfrac{1}{16};\ldots $}
	{$1;-\dfrac{1}{2};\dfrac{1}{4};-\dfrac{1}{8};\dfrac{1}{16};\ldots $}
	{ \True $1;3;5;7;9;\ldots $}
	\loigiai {
		Xét đáp án $1;1;1;1;1;1;\ldots$ đây là dãy hằng nên không tăng không giảm.\\
		Xét đáp án $1;-\dfrac{1}{2};\dfrac{1}{4};-\dfrac{1}{8};\dfrac{1}{16};\ldots \Rightarrow {{u}_{1}}>{{u}_{2}}<{{u}_{3}}\Rightarrow $ loại.\\
		Xét đáp án $1;3;5;7;9;\ldots \Rightarrow {{u}_{n}}<{{u}_{n+1}},\,\,n\in {{\mathbb{N}}^{*}}\Rightarrow $ chọn.\\
		Xét đáp án $1;\dfrac{1}{2};\dfrac{1}{4};\dfrac{1}{8};\dfrac{1}{16};\ldots \Rightarrow {{u}_{1}}>{{u}_{2}}>{{u}_{3}}\ldots >{{u}_{n}}>\ldots \Rightarrow $ loại.}
	
\end{ex}
%%%%%%%%%%
%Câu 2
\begin{ex}%[1K2Y5-3]
	Với giá trị nào của $a$ thì dãy số $\left(u_n\right)$ với $u_n=\dfrac{a n-1}{n+2}, \forall n \geq 1$ là dãy số tăng?
	\choice
	{$a>2$}
	{$a<-2$}
	{\True $a>-\dfrac{1}{2}$}
	{$a<-\dfrac{1}{2}$}
	\loigiai{
		Ta có $u_n=a-\dfrac{1+2 a}{n+2}$.\\
		$u_{n+1}-u_n=(1+2 a)\left(\dfrac{1}{n+2}-\dfrac{1}{n+3}\right)$.\\
		Suy ra dãy số đã cho tăng khi $a>-\dfrac{1}{2}$.
	}
\end{ex}
%%%%%%%%%%%%
%Câu 3
\begin{ex}%[1K2Y5-3]
	Trong các dãy $\left(u_n\right)$ sau đây dãy nào là dãy số giảm ?
	\choice
	{$u_n=(-1)^n$}
	{$u_n=2^n$}
	{$u_n=3 n+1$}
	{\True $u_n=\dfrac{1}{3^n}$}
	\loigiai{
		Xét dãy số $\left(u_n\right)$ có $u_n=\dfrac{1}{3^n}$, ta thấy $u_n>0, \forall n \in \mathbb{N}^\ast$ và $\dfrac{u_{n+1}}{u_n}=\dfrac{\dfrac{1}{3^{n+1}}}{\dfrac{1}{3^n}}=\dfrac{1}{3}<1$ nên dãy số $\left(u_n\right)$ này là dãy số giảm.
	}
\end{ex}
%%%%%%%%%%
%Câu 4
\begin{ex}%[1K2B5-3]
	Trong các dãy số $({{u}_{n}} )$ cho bởi số hạng tổng quát ${{u}_{n}}$ sau, dãy số nào là dãy số tăng?
	\choice
	{${{u}_{n}}=\dfrac{1}{n}$}
	{${{u}_{n}}=\dfrac{1}{{{2}^{n}}}$}
	{${{u}_{n}}=\dfrac{n+5}{3n+1}$}
	{ \True ${{u}_{n}}=\dfrac{2n-1}{n+1}$ }
	\loigiai {
		Vì ${{2}^{n}};\,n$ là các dãy dương và tăng nên $\dfrac{1}{{{2}^{n}}};\,\,\dfrac{1}{n}$ là các dãy giảm, do đó loại các đáp án ${{u}_{n}}=\dfrac{1}{{{2}^{n}}}$ và ${{u}_{n}}=\dfrac{1}{n}$.\\
		Xét đáp án ${{u}_{n}}=\dfrac{n+5}{3n+1}\Rightarrow \heva{
			& {{u}_{1}}=\dfrac{3}{2} \\
			& {{u}_{2}}=\dfrac{7}{6} \\
		}\Rightarrow {{u}_{1}}>{{u}_{2}}\Rightarrow $ loại.\\
		Xét đáp án ${{u}_{n}}=\dfrac{2n-1}{n+1}=2-\dfrac{3}{n+1}\Rightarrow  {{u}_{n+1}}-{{u}_{n}}=3\left(\dfrac{1}{n+1}-\dfrac{1}{n+2}\right)>0\Rightarrow$ nhận.}
	
\end{ex}
%%%%%%%%%%
%%%%%%%%%%
%Câu 5
\begin{ex}%[1K2B5-3]
	Trong các dãy số $({{u}_{n}} )$ cho bởi số hạng tổng quát ${{u}_{n}}$ sau, dãy số nào là dãy số giảm?
	\choice
	{${{u}_{n}}={{n}^2}$}
	{${{u}_{n}}=\dfrac{3n-1}{n+1}$}
	{${{u}_{n}}=\sqrt{n+2}$}
	{ \True ${{u}_{n}}=\dfrac{1}{{{2}^{n}}}$}
	\loigiai {
		Vì ${{2}^{n}}$ là dãy dương và tăng nên $\dfrac{1}{{{2}^{n}}}$ là dãy giảm. \\
		Xét ${{u}_{n}}=\dfrac{3n-1}{n+1}\Rightarrow \heva{
			& {{u}_{1}}=1 \\
			& {{u}_{2}}=\dfrac{5}{3} \\
		}\Rightarrow {{u}_{1}}<{{u}_{2}},$ loại.\\
		Hoặc
		${{u}_{n+1}}-{{u}_{n}}=\dfrac{3n+2}{n+2}-\dfrac{3n-1}{n+1}=\dfrac{4}{(n+1 )(n+2 )}>0$ nên $({{u}_{n}} )$ là dãy tăng.\\
		Xét ${{u}_{n}}={{n}^2}\Rightarrow {{u}_{n+1}}-{{u}_{n}}={{(n+1 )}^2}-{{n}^2}=2n+1>0,$ loại.\\
		Xét ${{u}_{n}}=\sqrt{n+2}\Rightarrow {{u}_{n+1}}-{{u}_{n}}=\sqrt{n+3}-\sqrt{n+2}=\dfrac{1}{\sqrt{n+3}+\sqrt{n+2}}>0,$ loại.}
	
\end{ex}

%Câu 6

\begin{ex}%[1K2B5-3]
	Trong các dãy số $(u_{n})$ sau, hãy chọn dãy số tăng.
	\choice
	{\True $u_{n}=(-1)^{2n}(5^{n}+1)$, $n\in \mathbb N^*$}
	{$u_{n}=\dfrac{n}{n^{2}+1}$, $n\in \mathbb N^*$}
	{$u_{n}=(-1)^{n+1}\sin \dfrac{\pi}{n}$, $n\in \mathbb N^*$}
	{$u_{n}=\dfrac{1}{\sqrt{n+1}+n}$, $n\in \mathbb N^*$}
	\loigiai
	{
		Xét dãy số $(u_n)$ với $u_{n}=(-1)^{2n}(5^{n}+1)$, ta có
		\[u_{n+1}-u_n = (-1)^{2n+2}(5^{n+1}+1)-(-1)^{2n}(5^{n}+1) = 5^{n+1}+1-5^n-1 = 4\cdot 5^n>0, \forall n\in\mathbb{N}^\ast.\]
		Vậy dãy trên là dãy số tăng.\\
		Xét các dãy số còn lại
		\begin{itemize}
			\item Với $u_{n}=(-1)^{n+1}\sin \dfrac{\pi}{n}$ ta có $u_1=0$, $u_2=-1$ hay $u_1>u_2$. Vậy dãy số này không là dãy số tăng.
			\item Với $u_{n}=\dfrac{1}{\sqrt{n+1}+n}$ ta có $u_1=\sqrt{2}-1$, $u_2=2-\sqrt{3}$ hay $u_1>u_2$. Vậy dãy số này không là dãy số tăng.
			\item Với $u_{n}=\dfrac{n}{n^{2}+1}$ ta có $u_1=\dfrac{1}{2}$, $u_2=\dfrac{2}{5}$ hay $u_1>u_2$. Vậy dãy số này không là dãy số tăng.
		\end{itemize}
	}
\end{ex}
%%%%%%%%%%
%Câu 7
\begin{ex}%[1K2K5-3]
	Trong các dãy số $({{u}_{n}} )$ cho bởi số hạng tổng quát ${{u}_{n}}$ sau, dãy số nào là dãy số giảm?
	\choice
	{${{u}_{n}}=\dfrac{{{n}^2}+1}{n}$}
	{${{u}_{n}}={{(-1 )}^{n}}\cdot ({{2}^{n}}+1 )$}
	{\True ${{u}_{n}}=\sqrt{n}-\sqrt{n-1}\,$}
	{${{u}_{n}}=\sin n$}
	\loigiai {
		Xét ${{u}_{n}}=\sin n\Rightarrow  {{u}_{n+1}}-{{u}_{n}}=2\cos \left(n+\dfrac{1}{2} \right)\sin \dfrac{1}{2}$ có thể dương hoặc âm phụ thuộc $n$ nên đáp án sai. Hoặc dễ thấy $\sin n$ có dấu thay đổi trên ${{\mathbb{N}}^{*}}$ nên dãy $\sin n$ không tăng, không giảm.\\
		Xét ${{u}_{n}}=\dfrac{{{n}^2}+1}{n}=n+\dfrac{1}{n}\Rightarrow  {{u}_{n+1}}-{{u}_{n}}=1+\dfrac{1}{n+1}-\dfrac{1}{n}=\dfrac{{{n}^2}+n-1}{n(n+1 )}>0$ nên dãy đã cho tăng nên đáp án sai.\\
		Xét ${{u}_{n}}=\sqrt{n}-\sqrt{n-1}=\dfrac{1}{\sqrt{n}+\sqrt{n+1}},$ dãy $\sqrt{n}+\sqrt{n-1}>0$ là dãy tăng nên suy ra ${{u}_{n}}$ giảm. \\
		Xét ${{u}_{n}}={{(-1 )}^{n}}({{2}^{n}}+1 )$ là dãy thay dấu nên không tăng không giảm, nên đáp án đúng.\\
		Cách trắc nghiệm\\
		Xét ${{u}_{n}}=\sin n$ có dấu thay đổi trên ${{\mathbb{N}}^{*}}$ nên dãy này không tăng không giảm.\\
		Xét ${{u}_{n}}=\dfrac{{{n}^2}+1}{n}$, ta có $\heva{
			& n=1\to {{u}_{1}}=2 \\
			& n=2\to {{u}_{2}}=\dfrac{5}{2} \\
		}\Rightarrow {{u}_{1}}<{{u}_{2}}\Rightarrow {{u}_{n}}=\dfrac{{{n}^2}+1}{n}$ không giảm.\\
		Xét ${{u}_{n}}=\sqrt{n}-\sqrt{n-1}$, ta có $\heva{
			& n=1\to {{u}_{1}}=1 \\
			& n=2\to {{u}_{2}}=\sqrt{2}-1 \\
		}\Rightarrow {{u}_{1}}>{{u}_{2}}$ nên dự đoán dãy này giảm.\\
		Xét ${{u}_{n}}={{(-1 )}^{n}}({{2}^{n}}+1 )$ là dãy thay dấu nên không tăng không giảm.\\
		Cách CASIO.\\
		Các dãy $\sin n;\,\,{{(-1 )}^{n}}({{2}^{n}}+1 )$ có dấu thay đổi trên ${{\mathbb{N}}^{*}}$ nên các dãy này không tăng không giảm nên loại các đáp án này.\\
		Xét hai đáp án còn lại, ta chỉ cần kiểm tra một đáp án bằng chức năng $TABLE$.\\
		Chẳng hạn kiểm tra đáp án ${{u}_{n}}=\dfrac{{{n}^2}+1}{n}$, ta vào chức năng $TABLE$ nhập $F(X )=\dfrac{X^2+1}{X}$ với thiết lập $\text{Start}=1,\text{ End}=10,\text{ Step}=1$.\\
		Nếu thấy cột $F(X )$ các giá trị tăng thì loại ${{u}_{n}}=\dfrac{{{n}^2}+1}{n}$ nếu ngược lại nếu thấy cột $F(X )$ các giá trị giảm dần thị chọn ${{u}_{n}}=\dfrac{{{n}^2}+1}{n}$.}
	
\end{ex}
%%%%%%%%%%
%Câu 8
\begin{ex}%[1K2K5-3]
	Mệnh đề nào sau đây đúng?
	\choice
	{Dãy số ${{u}_{n}}=\dfrac{1}{n}-2$ là dãy tăng}
	{\True Dãy số ${{u}_{n}}=2n+\cos \dfrac{1}{n}$ là dãy tăng}
	{Dãu số ${{u}_{n}}=\dfrac{n-1}{n+1}$ là dãy giảm}
	{Dãy số ${{u}_{n}}={{(-1 )}^{n}}({{2}^{n}}+1 )$ là dãy giảm}
	\loigiai {
		Xét đáp án ${{u}_{n}}=\dfrac{1}{n}-2\Rightarrow {{u}_{n+1}}-{{u}_{n}}=\dfrac{1}{n+1}-\dfrac{1}{n}<0\Rightarrow $loại.\\
		Xét đáp án ${{u}_{n}}={{(-1 )}^{n}}({{2}^{n}}+1 )$ là dãy có dấu thay đổi nên không giảm nên loại.\\
		Xét đáp án ${{u}_{n}}=\dfrac{n-1}{n+1}=1-\dfrac{2}{n+1}\Rightarrow {{u}_{n+1}}-{{u}_{n}}=2\left(\dfrac{1}{n+1}-\dfrac{1}{n+2}\right)>0\Rightarrow $ loại.\\
		Xét đáp án ${{u}_{n}}=2n+\cos \dfrac{1}{n}\Rightarrow {{u}_{n+1}}-{{u}_{n}}=\left(2-\cos \dfrac{1}{n+1}\right)+\cos \dfrac{1}{n+2}>0$ chọn.}
	
\end{ex}
%%%%%%%%%%
%Câu 9
\begin{ex}%[1K2K5-3]
	Mệnh đề nào sau đây {\bf sai}?
	\haicot
	{Dãy số ${{u}_{n}}=\dfrac{1-n}{\sqrt{n}}$ là dãy giảm}
	{Dãy số ${{u}_{n}}=n+\sin ^2n$ là dãy tăng}
	{\True Dãy số ${{u}_{n}}={{\left(1+\dfrac{1}{n}\right)}^{n}}$ là dãy giảm}
	{Dãy số ${{u}_{n}}=2{{n}^2}-5$ là dãy tăng}
	\loigiai {
		Xét đáp án \\ ${{u}_{n}}=\dfrac{1-n}{\sqrt{n}}=\dfrac{1}{\sqrt{n}}-\sqrt{n}\Rightarrow {{u}_{n+1}}-{{u}_{n}}=\dfrac{1}{\sqrt{n+1}}-\dfrac{1}{\sqrt{n}}+\sqrt{n}-\sqrt{n+1}<0$ nên dãy $({{u}_{n}} )$ là dãy giảm nên đúng.\\
		Xét đáp án ${{u}_{n}}=2{{n}^2}-5$ là dãy tăng vì ${{n}^2}$ là dãy tăng nên đúng. \\
		Hoặc
		${{u}_{n+1}}-{{u}_{n}}=2(2n+1 )>0$ nên $({{u}_{n}} )$ là dãy tăng.\\
		Xét đáp án ${{u}_{n}}={{\left(1+\dfrac{1}{n}\right)}^{n}}={{\left(\dfrac{n+1}{n} \right)}^{n}}>0\Rightarrow\dfrac{{{u}_{n+1}}}{{{u}_{n}}}=\dfrac{n+2}{n+1}\cdot {{\left(\dfrac{n+2}{n}\right)}^{n}}>1\Rightarrow ({{u}_{n}} )$ là dãy tăng nên sai.\\
		Xét đáp án ${{u}_{n}}=n+\sin ^2n\Rightarrow {{u}_{n+1}}-{{u}_{n}}=(1-\sin ^2(n+1 ) )+\sin ^2n>0$.}
	
\end{ex}
%%%%%%%%%%
%Câu 10%VDC 
\begin{ex}%[Nguyễn Long]%[1K2G5-3]
	Cho dãy $(u_n)\colon\heva{&u_1=1\\&u_{n+1}=\dfrac{n}{2(n+1)}u_n+\dfrac{3(n+2)}{2(n+1)}},n \in \mathbb{N^*}$. Nhận xét nào sau đây đúng
	\choice
	{\True Dãy số $(u_n)$ là dãy số tăng}
	{Dãy số $(u_n)$ là dãy số giảm}
	{Dãy số $(u_n)$ là dãy số không tăng, không giảm}
	{Tất cả các đáp án còn lại đều sai}
	\loigiai{ Ta chứng minh quy nạp $u_n<3, \forall n \in N^*$.\\
		Giả sử mđ đúng với $\mathrm{n}=\mathrm{k}$ khi đó có:
		$$
		u_{k+1}=\dfrac{k}{2(k+1)} u_k+\dfrac{3(k+2)}{2(k+1)}<\dfrac{3 k}{2(k+2)}+\dfrac{3(k+2)}{2(k+1)}=3 .
		$$
		Vậy mệnh đề đúng với $\mathrm{n}=\mathrm{k}+1$.
		Từ đó ta có $$u_{n+1}-u_n=\dfrac{\left(3-u_n\right)(n+2)}{n+1}>0.$$
		Vậy dãy $\left(u_n\right)$ tăng }
\end{ex}
\Closesolutionfile{ans}
\begin{indapan}{10}
	{ans/ans-1K2-1-Dang3}
\end{indapan}

\begin{dang}{Xét tính bị chặn của dãy số}
	\begin{itemize}
		\item Để chứng minh dãy số $(u_n)$ bị chặn trên bởi $M$, ta chứng minh $u_n\le M$, $\forall n\in\mathbb{N}^\ast$.
		\item Để chứng minh dãy số $(u_n)$ bị chặn dưới bởi $m$, ta chứng minh $u_n\ge m$, $\forall n\in\mathbb{N}^\ast$.
		\item Để chứng minh dãy số bị chặn ta chứng minh nó bị chặn trên và bị chặn dưới.
		\begin{itemize}
			\item Nếu dãy số $(u_n)$ tăng thì bị chặn dưới bởi $u_1$.
			\item Nếu dãy số $(u_n)$ giảm thì bị chặn trên bởi $u_1$.
		\end{itemize}
	\end{itemize}
\end{dang}
\subsubsection{Ví dụ minh hoạ}

%ví dụ 1
\begin{vd}[NB]%[1K2Y5-4]%[Trương Đăng Khoa]
	Chứng minh rằng dãy số $(u_n)$ với $u_n=\dfrac{3n}{n^2+9}$ bị chặn trên bởi $\dfrac{1}{2}$.
	\dapso{dãy số đã cho bị chặn trên bởi $\dfrac{1}{2}$}
	\loigiai{
		Với mọi $n\ge 1$, ta có $\dfrac{3n}{n^2+9}\le\dfrac{1}{2}\Leftrightarrow n^2+9\le 6n\Leftrightarrow(n-3)^2\le 0$ (đúng).\\
		Vậy dãy số đã cho bị chặn trên bởi $\dfrac{1}{2}$.
	}
\end{vd}

%ví dụ 2
\begin{vd}[NB]%[1K2Y5-4]%[Trương Đăng Khoa]
	Chứng minh rằng dãy số $(u_n)$ xác đinh bởi $u_n=\dfrac{8n+3}{3n+5}$ là một dãy số bị chặn.
	\dapso{dãy số bị chặn}
	\loigiai{
		Ta có $u_n>0$, $\forall n\ge 1$. Suy ra dãy số bị chặn dưới.\\
		Mặt khác $u_n=\dfrac{8n+3}{3n+5}<\dfrac{8n+3}{3n}=\dfrac{8}{3}+\dfrac{1}{n}<\dfrac{8}{3}+1=\dfrac{11}{3}$. Do đó dãy số bị chặn trên bởi $\dfrac{11}{3}$.\\
		Vậy dãy số đã cho bị chặn.
	}
\end{vd}
%ví dụ 4
\begin{vd}[TH]%[1K2B5-4]%[Trương Đăng Khoa]
	Xét tính bị chặn của dãy số $\left(u_n\right)$ với $u_n=\dfrac{3n+1}{n+3}$.
	\dapso{dãy số bị chặn}
	\loigiai{
		Với $n\in \mathbb{N}^\ast$ ta có $u_n=\dfrac{3n+1}{n+3}>0$.\\
		Nên dãy $\left(u_n\right)$ bị chặn dưới bởi $0$.\\
		Mặt khác $u_n=\dfrac{3n+1}{n+3}=\dfrac{3n+9-8}{n+3}=3-\dfrac{8}{n+3}<3$, $\forall n\in\mathbb{N}^\ast$.\\
		Nên dãy $\left(u_n\right)$ bị chặn trên bởi $3$.\\
		Vậy dãy số $\left(u_n\right)$ bị chặn.
	}
\end{vd}
%ví dụ 3
\begin{vd}[VD]%[1K2K5-4]%[Trương Đăng Khoa]
	Cho dãy số $(u_n)$ xác định bởi $u_1=1$ và $u_{n+1}=\dfrac{u_n+2}{u_n+1}$, $\forall n\ge 1$. Chứng minh rằng dãy $(u_n)$ bị chặn trên bởi sô $\dfrac{3}{2}$ và bị chặn dưới bởi số $1$.	
	\loigiai{
		Ta chứng minh $1\le u_n\le\dfrac{3}{2},\forall n\ge 1$ bằng phương pháp quy nạp.
		\begin{itemize}
			\item Với $n=1$ ta có $1\le u_1\le\dfrac{3}{2}$.
			\item Giả sử $1\le u_n\le\dfrac{3}{2}$ với mọi $n=k\ge 1$, tức là $1\le u_k\le\dfrac{3}{2}$. Ta cần chứng minh $1\le u_{k+1}\le\dfrac{3}{2}$.
		\end{itemize}
		Thật vậy 
		$u_{k+1}=1+\dfrac{1}{u_k+1}$.\\
		Vì $u_k+1>0$ nên $u_{k+1}=1+\dfrac{1}{u_k+1}>1$.\\
		Vì $u_k+1\ge 2$ nên $u_{k+1}=1+\dfrac{1}{u_k+1}\le 1+\dfrac{1}{2}=\dfrac{3}{2}$.\\
		Vậy $1\le u_n\le\dfrac{3}{2}$, $\forall n\ge 1$ hay dãy $(u_n)$ bị chặn trên bởi số $\dfrac{3}{2}$ và bị chặn dưới bởi số $1$.
	}
\end{vd}

%ví dụ 5
\begin{vd}[VD]%[1K2K5-4]%[Trương Đăng Khoa]
	Xét tính bị chặn của dãy số $\left(u_n\right)$ với $u_n=\sin n+ \cos n$.
	\dapso{dãy số bị chặn}
	\loigiai{
		Ta có $\begin{aligned}[t]
			&\ \sin n+\cos n \\
			=&\ \sqrt{2}\left(\dfrac{1}{\sqrt{2}}\sin n+\dfrac{1}{\sqrt{2}}\cos n\right)\\
			=&\ \sqrt{2}\left(\sin n\cdot\cos \dfrac{\pi}{4}+\cos n\cdot\sin \dfrac{\pi}{4}\right)\\
			=&\ \sqrt{2}\sin \left(n+\dfrac{\pi}{4}\right).
		\end{aligned}$\\
		Vì $\begin{aligned}[t]
			&\ -1\leq \sqrt{2}\sin \left(n+\dfrac{\pi}{4}\right) \leq 1\\
			\Rightarrow&\ -\sqrt{2}\leq \sqrt{2} \sin \left(n+\dfrac{\pi}{4}\right)\leq \sqrt{2}\\
			\Rightarrow&\ -\sqrt{2}\leq \sin n+\cos n \leq \sqrt{2},\ \forall n\in\mathbb{N}^\ast\\
			\Rightarrow&\ -\sqrt{2}\leq u_n \leq \sqrt{2},\ \forall n\in\mathbb{N}^\ast.
		\end{aligned}$\\
		Vậy dãy số $\left(u_n\right)$ là dãy số bị chặn.}
\end{vd}
\subsubsection{Bài tập tự luận}
 
%Bài 1 
\begin{bt}[TH]%[1K2B5-4]%[Trương Đăng Khoa]
	Xét tính bị chặn của các dãy số sau 
		\begin{listEX}[3]
			\item $u_n=\dfrac{1}{2n^2-1}$.
			\item 
			$u_n=3\cdot\cos\dfrac{n x}{3}$. 
			\item  $u_n=2n^3+1$.
			\item  $u_n=\dfrac{n^2+2n}{n^2+n+1}$.
			\item  $u_n=n+\dfrac{1}{n}$.
		\end{listEX}
	\loigiai{
		\begin{enumerate}
			\item  $u_n=\dfrac{1}{2n^2-1}$.\\
			Ta có $2n^2-1\ge 1\Rightarrow u_n=\dfrac{1}{2n^2-1}\le 1$, $\forall n\ge 1$.\\
			Vậy dãy số bị chặn trên bởi $1$.\\
			\item $u_n=3\cdot\cos\dfrac{n x}{3}$ có $-1\le\cos\dfrac{n x}{3}\le 1\Rightarrow-3\le 3\cdot\cos\dfrac{n x}{3}\le 3$.\\
			Vậy dãy số bị chặn dưới bởi $-3$ và chặn trên bởi $3$.
			\item  $u_n=2n^3+1$ có $2n^3+1\ge 3$, $\forall n\ge 1$.\\
			Vậy dãy số bị chặn dưới bởi $3$.
			\item $u_n=\dfrac{n^2+2n}{n^2+n+1}$ có $u_n=\dfrac{n^2+2n}{n^2+n+1}=1+\dfrac{n-1}{n^2+n+1}\ge 1$, $\forall n\ge 1$.\\
			Vậy dãy số bị chặn dưới bởi $1$.
			\item  $u_n=n+\dfrac{1}{n}$ có $u_n=n+\dfrac{1}{n}\ge 2\sqrt{n\cdot\dfrac{1}{n}}=2$, $\forall n>0$.\\
			Vậy dãy số bị chặn bởi $2$.
		\end{enumerate}
	}
\end{bt}
%Bài 2
\begin{bt}[VD]%[1K2K5-4]%[Trương Đăng Khoa]
	Xét tính bị chặn của dãy số $(u_n)$ với:
	\begin{listEX}[3]	 
		\item $u_{n}=\dfrac{4}{n}-5$.
		\item $u_{n}=\dfrac{n+4}{n+2}$.
		\item $u_{n}=\dfrac{5}{n^2+1}+\dfrac{n+2}{n+1}+\cos n$.
	\end{listEX}
	\loigiai{
		\begin{enumerate} 
			\item $u_{n}=\dfrac{4}{n}-5$.\\
			Ta có $u_n=\dfrac{4}{n}-5 \le \dfrac{4}{1}-5=-1$, $\forall n \in \mathbb{N}^{*}$ suy ra dãy $(u_n)$ bị chặn trên bởi $-1$.\\
			Mặt khác $u_n=\dfrac{4}{n}-5 \ge -5 \,\, \forall n \in \mathbb{N}^{*}$ suy ra dãy $(u_n)$ bị chặn dưới bởi $-5$.\\
			Vậy dãy $(u_n)$ bị chặn.	
			\item $u_{n}=\dfrac{n+4}{n+2}$.\\
			Ta có $u_n=	\dfrac{n+4}{n+2}=1+\dfrac{2}{n+2}> 1$, $\forall n \in \mathbb{N}^{*}$ suy ra dãy $(u_n)$ bị chặn dưới bởi $1$.\\
			Mặt khác $u_n=\dfrac{n+4}{n+2}=1+\dfrac{2}{n+2} \le 1+\dfrac{2}{1+2}=\dfrac{3}{5}$, $\forall n \in \mathbb{N}^{*}$ suy ra dãy $(u_n)$ bị chặn trên bởi $\dfrac{3}{5}$.\\
			Vậy dãy $(u_n)$ bị chặn.
			\item $u_{n}=\dfrac{5}{n^2+1}+\dfrac{n+2}{n+1}+\cos n$.\\
			Ta có $u_n=	\dfrac{5}{n^2+1}+\dfrac{n+2}{n+1}+\cos n=\dfrac{5}{n^2+1}+1+\dfrac{1}{n+1}+\cos n<5$, $\forall n \in \mathbb{N}^{*}$.\\
			Suy ra dãy $(u_n)$ bị chặn trên bởi $5$.\\
			Mặt khác $u_n=	\dfrac{5}{n^2+1}+\dfrac{n+2}{n+1}+\cos n=\dfrac{5}{n^2+1}+1+\dfrac{1}{n+1}+\cos n>0$, $\forall n \in \mathbb{N}^{*}$.\\
			Suy ra dãy $(u_n)$ bị chặn trên bởi $0$.\\
			Vậy dãy $(u_n)$ bị chặn.
		\end{enumerate}			
	}		
\end{bt}
%Bài 3
\begin{bt}[VDC]%[1K2G5-4]%[Trương Đăng Khoa]
	Xét tính bị chặn của dãy số $u_n=\left(1+\dfrac{1}{n}\right)^n$, $n\in N^\ast$.
	\loigiai{
		Ta có $u_n=\left(1+\dfrac{1}{n}\right)^n>0$, $\forall n\in N^\ast$ nên $(u_n)$ bị chặn dưới $(1)$.\\
		Lại có $\begin{aligned}[t]
			u_n&\ =\left(1+\dfrac{1}{n}\right)^n=\displaystyle\sum\limits_{k=0}^n C_n^k\left(\dfrac{1}{n}\right)^k\\
			&\ =\displaystyle\sum\limits_{k=0}^n\left[\dfrac{n!}{k!\cdot(n-k)!\cdot n^k}\right]\\
			&\ =\displaystyle\sum\limits_{k=0}^n\left[\dfrac{1}{k!}\cdot\dfrac{(n-k+1)}{n}\cdot\dfrac{(n-k+2)}{n}\ldots\dfrac{(n-k+k)}{n}\right]\le\displaystyle\sum\limits_{k=0}^n\dfrac{1}{k!},\, n\in \mathbb{N}^\ast
		\end{aligned}$\\
		Mà $\begin{aligned}[t]
			\displaystyle\sum\limits_{k=0}^n\dfrac{1}{k!}&\ \le 1+1+\dfrac{1}{1\cdot2}+\dfrac{1}{2\cdot3}+\dfrac{1}{3\cdot4}+\ldots+\dfrac{1}{(n-1)\cdot n}\\
			&\ =2+\left(1-\dfrac{1}{2}\right)+\left(\dfrac{1}{2}-\dfrac{1}{3}\right)+\ldots+\left(\dfrac{1}{n-1}-\dfrac{1}{n}\right)\\
			&\ 
			=3-\dfrac{1}{n}<3,\, \forall n\in \mathbb{N}^\ast.
		\end{aligned}$\\
		Suy ra $u_n<3$, $\forall n\in \mathbb{N}^\ast$ nên dãy số $(u_n)$ bị chặn trên $(2)$.\\
		Từ $(1)$ và $(2)$  suy ra dãy số $(u_n)$ bị chặn.}
\end{bt}
%Bài 4
\begin{bt}[VD]%[1K2K5-4]%[Trương Đăng Khoa]
	Cho dãy số $(u_n)$ xác định bởi $u_1=0$ và $u_{n+1}=\dfrac{1}{2}u_n+4$, $ \forall n\geq 1$.
	\begin{enumerate}
		\item Chứng minh dãy $(u_n)$ bị chặn trên bởi số $8$.
		\item Chứng minh dãy $(u_n)$ tăng, từ đó suy ra dãy $(u_n)$ bị chặn.
	\end{enumerate}
	\loigiai{
		\begin{enumerate}
			\item Ta chứng minh $u_n\leq 8$ với mọi $n\geq 1$.
			\begin{itemize}
				\item Khi $n=1$, ta có $u_1=0 <8$.
				\item Giả sử $u_n\leq 8$ với $n=k\geq 1$, tức là $u_k\leq 8$.\\
				Ta cần chứng minh $u_{k+1}\leq 8$.\\
				Thật vậy, $u_{k+1}=\dfrac{1}{2}u_k+4\leq \dfrac{1}{2}\cdot 8+4\leq 8$.
			\end{itemize}
			Vậy $u_n\leq 8$ với mọi $n\geq 1$, hay $(u_n)$ bị chặn trên bởi $8$.
			\item Với mọi $n\geq 1$, ta có $u_{n+1}-u_n=4-\dfrac{1}{2}u_n$. Mà $u_n\leq 8$ nên $u_{n+1}-u_n\geq 0$.\\
			Suy ra $u_n$ là dãy số tăng. Do đó $(u_n)$ bị chặn dưới bởi $u_1=0$.\\
			Kết hợp với câu a, ta được dãy số $(u_n)$ bị chặn.
		\end{enumerate}
	}
\end{bt}
%Bài 5
\begin{bt}[VD]%[1K2K5-4]%[Trương Đăng Khoa]
	Trong các dãy số $(u_n)$ sau, dãy số nào bị chặn trên, bị chặn dưới và bị chặn?
	\begin{listEX}[3]
		\item $u_n=n^2+5$.
		\item $u_n=\dfrac{3n+1}{2n+5}$.
		\item $u_n=(-1)^n\cos \dfrac{\pi}{2n}$.
		\item $u_n=\dfrac{n^2+2n}{n^2+n+1}$.
		\item $u_n=\dfrac{n}{\sqrt{n^2+2n}+n}$.
	\end{listEX}
	\loigiai{
		\begin{enumerate}
			\item Dãy số bị chặn dưới bởi $6$, không bị chặn trên.
			\item Dãy $(u_n)$ bị chặn dưới bởi $0$. Vì $u_n<\dfrac{3n+1}{2n}=\dfrac{3}{2}+\dfrac{1}{2n}<\dfrac{3}{2}+1=\dfrac{5}{2}$ nên dãy số bị chặn trên bởi $\dfrac{5}{2}$. Vậy dãy số bị chặn.
			\item Ta có $|u_n|\leq 1$ nên dãy số bị chặn trên bởi 1, bị chặn dưới bởi $-1$.
			\item Dãy số bị chặn dưới bởi $0$. Vì $u_n<\dfrac{n^2+2n}{n^2}=1+\dfrac{2}{n}\leq 3$ nên dãy số bị chặn trên. Vậy dãy số bị chặn.
			\item Ta có $0<u_n\leq 1$ vậy dãy số bị chặn.
		\end{enumerate}
	}
\end{bt}

\subsubsection{Câu hỏi trắc nghiệm}

\Opensolutionfile{ans}[ans/ans-1K2-1-Dang4]
%Câu 1
\begin{ex}%[1K2K5-4]%[Trương Đăng Khoa]
	Cho dãy số $(u_n)$ xác định bởi $u_1=3$ và $u_{n+1}=\dfrac{u_n+1}{2}$, $\forall n\geq 1$. Mệnh đề nào sau đây là đúng?
	\choice
	{\True Dãy số bị chặn}
	{Dãy số bị chặn trên}
	{Dãy số bị chặn dưới}
	{Dãy số không bị chặn}
	\loigiai{Ta chứng minh $u_n>1, \forall n\geq 1$ bằng phương pháp quy nạp.\\
		Suy ra dãy số bị chặn dưới bởi $1$.\\
		Ta có
		$u_{n+1}-u_n=\dfrac{1-u_n}{2}<0$, $\forall n\geq 1$.\\
		Do đó dãy số này là dãy số giảm nên nó bị chặn trên bởi $u_1=3$.\\
		Vậy dãy số đã cho là dãy số bị chặn.		
	}
\end{ex}
%Câu 2
\begin{ex}%[1K2K5-4]%[Trương Đăng Khoa]
	Cho dãy số $(u_n)$ xác định bởi $u_1=\sqrt{2}$ và $u_{n+1}=\sqrt{2+u_n}$, $\forall n\geq 1$. Mệnh đề nào sau đây là đúng?
	\choice
	{Dãy số bị chặn trên}
	{Dãy số bị chặn dưới}
	{\True Dãy số bị chặn}
	{Dãy số không bị chặn}
	\loigiai{Vì $u_n\geq 0$, $\forall n\geq 1$ nên dãy số bị chặn dưới bởi $0$.\\
		Ta chứng minh $u_n\geq 2, \forall n\geq 1$. Suy ra dãy số bị chặn trên bởi $2$.\\
		Vậy dãy số đã cho là dãy số bị chặn.		
	}
\end{ex}
%Câu 3
\begin{ex}%[1K2K5-4]%[Trương Đăng Khoa]
	Xét tính bị chặn của  dãy số $(u_n)$ với $u_n=\dfrac{1}{1\cdot2}+\dfrac{1}{2\cdot3}+\ldots+\dfrac{1}{n\cdot(n+1)}$.
	\choice{Không bị chặn}{Bị chặn trên}{Bị chặn dưới}{\True Bị chặn}
	\loigiai{
		Ta có $u_n=1-\dfrac{1}{2}+\dfrac{1}{2}-\dfrac{1}{3}+\ldots+\dfrac{1}{n}-\dfrac{1}{n+1}=1-\dfrac{1}{n+1}$.\\
		Do đó $0\leq u_n \leq 1$, $\forall n\geq 1$.\\
		Vậy dãy số đã cho bị chặn.
	}
\end{ex}
%Câu 4
\begin{ex}%[1K2G5-4]%[Trương Đăng Khoa]
	Cho dãy số $(u_n)$ với $u_n=\dfrac{1}{1\cdot4}+\dfrac{1}{2\cdot5}+\ldots+\dfrac{1}{n\cdot(n+3)}$. Dãy số $\left(u_n\right)$ bị chặn dưới và chặn trên lần lượt bởi các số $m$ và $M$ nào dưới đây?
	\choice
	{$m=0$, $M=1$}
	{$m=1$, $M=\dfrac{1}{2}$}
	{$m=1$, $M=\dfrac{10}{19}$}
	{\True $m=0$, $M=\dfrac{11}{18}$}
	\loigiai{
		Rõ ràng $u_n>0$, $\forall n\in\mathbb{N}^\ast$ nên $(u_n)$ bị chặn dưới.\\
		Mặt khác $\dfrac{1}{k(k+3)}=\dfrac{1}{3}\left(\dfrac{1}{k}-\dfrac{1}{k+3}\right)$.\\
		Suy ra $\begin{aligned}[t]
			u_n&\  =\dfrac{1}{3}\bigg[\left(1-\dfrac{1}{4}\right)+\left(\dfrac{1}{2}-\dfrac{1}{5}\right)+\left(\dfrac{1}{3}-\dfrac{1}{6}\right)+\left(\dfrac{1}{4}-\dfrac{1}{7}\right)+\\
			&\  \ldots+\left(\dfrac{1}{n-3}-\dfrac{1}{n}\right)+\left(\dfrac{1}{n-2}-\dfrac{1}{n+1}\right)+\left(\dfrac{1}{n-1}-\dfrac{1}{n+2}\right)+\left(\dfrac{1}{n}-\dfrac{1}{n+3}\right)\bigg]\\
			&\ = \dfrac{1}{3}\left(1+\dfrac{1}{2}+\dfrac{1}{3}-\dfrac{1}{n+1}-\dfrac{1}{n+2}-\dfrac{1}{n+3}\right)<\dfrac{11}{18}, \, \forall n\in\mathbb{N}^\ast.
		\end{aligned}$\\
		Do đó $(u_n)$ bị chặn trên.\\
		Vậy $m=0$, $M=\dfrac{11}{18}$.
	}
\end{ex}
%Câu 5
\begin{ex}%[1K2G5-4]%[Trương Đăng Khoa]
	Cho dãy số $(u_n)$ biết $u_n=\dfrac{1\cdot 3\cdot 5\ldots(2n-1)}{2\cdot 4\cdot 6\cdot 2n}$. Dãy số $\left(u_n\right)$ bị chặn dưới và chặn trên lần lượt bởi các số $m$ và $M$. Tính giá trị biểu thức $m+M$?
	\choice{$\dfrac{1}{\sqrt{2}}$}{\True $\dfrac{1}{\sqrt{3}}$}{$\dfrac{1}{\sqrt{5}}$}{$\dfrac{1}{\sqrt{7}}$}
	\loigiai{
		Xét $ \dfrac{2 k-1}{2 k}<\dfrac{2 k-1}{\sqrt{4 k^2-1}}
		=\dfrac{\sqrt{(2 k-1)^2}}{\sqrt{(2 k-1)(2 k+1)}} =\dfrac{\sqrt{2 k-1}}{\sqrt{2 k+1}}$,  $\forall k \ge 1$.\\
		$\Rightarrow u_n<\dfrac{\sqrt{1}}{\sqrt{3}} \cdot\dfrac{\sqrt{3}}{\sqrt{5}} \cdot\dfrac{\sqrt{5}}{\sqrt{7}}\cdot \ldots \cdot \dfrac{\sqrt{2 n-1}}{\sqrt{2 n+1}}=\dfrac{1}{\sqrt{2 n+1}} \le\dfrac{1}{\sqrt{3}}$,  $\forall n \in\mathbb{N}^\ast$.\\
		$\Rightarrow 0<u_n<\dfrac{1}{\sqrt{3}}$, $\forall n \in\mathbb{N}^\ast$.\\
		Vậy $m+M=0+\dfrac{1}{\sqrt{3}}$.
	}
\end{ex}
%Câu 6
\begin{ex}%[1K2G5-4]%[Trương Đăng Khoa]
	Cho dãy số $(u_n)$, với $u_n=\dfrac{1}{2^2}+\dfrac{1}{3^2}+\ldots+\dfrac{1}{n^2}$, $\forall n=2;3;4;\ldots$. Khẳng định nào sau đây là đúng?
	\choice
	{\True Dãy số bị chặn}
	{Dãy số bị chặn trên}
	{Dãy số bị chặn dưới}
	{Dãy số không bị chặn}
	\loigiai{
		Ta có $u_n>0\Rightarrow(u_n)$ bị chặn dưới bởi $0$.\\
		Mặt khác $\dfrac{1}{k^2}<\dfrac{1}{(k-1) k}=\dfrac{1}{k-1}-\dfrac{1}{k}$, ($k\in\mathbb{N}^\ast$, $k\ge 2$) nên suy ra
		\begin{eqnarray*}
			u_n&<&\dfrac{1}{1 \cdot 2}+\dfrac{1}{2 \cdot 3}+\dfrac{1}{3 \cdot 4}+\cdots+\dfrac{1}{n(n+1)}\\
			&=&1-\dfrac{1}{2}+\dfrac{1}{2}-\dfrac{1}{3}+\dfrac{1}{2}-\dfrac{1}{4}+\cdots+\dfrac{1}{n}-\dfrac{1}{n+1}=1-\dfrac{1}{n+1}<1.
		\end{eqnarray*}
		Nên dãy $(u_n)$ bị chặn trên, do đó dãy $(u_n)$ bị chặn.
	}
\end{ex}
%Câu 7
\begin{ex}%[1K2G5-4]%[Trương Đăng Khoa]
	Cho dãy số $\left(u_n\right)$ và đặt $u_n= \displaystyle \sum_{k=1}^{n} a_k$ với $a_k=\dfrac{1}{4k^2-1}$. Mệnh đề nào sau đây là đúng?
	\choice{$0<u_n <1$}
	{$0\leq u_n\leq \dfrac{1}{2}$}
	{\True $0<u_n<\dfrac{1}{2}$}
	{$0\leq u_n\leq 1$}
	\loigiai{
		\begin{itemize}
			\item
			Ta có $a_k=\dfrac{1}{4k^2-1}=\dfrac{1}{(2k+1)(2k-1)}=\dfrac{1}{2}\cdot\dfrac{(2k+1)-(2k-1)}{(2k+1)(2k-1)}=\dfrac{1}{2}\cdot\left(\dfrac{1}{2k-1}-\dfrac{1}{2k+1}\right)$.\\
			\item Mặt khác $u_n=\displaystyle \sum_{k=1}^{n} a_k$.
			Do đó
			\begin{eqnarray*}
				&u_n&=\dfrac{1}{2}\cdot\left(\dfrac{1}{1}-\dfrac{1}{3}\right)+\dfrac{1}{2}\cdot\left(\dfrac{1}{3}-\dfrac{1}{5}\right)+\ldots + \dfrac{1}{2}\cdot \left(\dfrac{1}{2n-1}-\dfrac{1}{2n+1}\right)\\
				&&=\dfrac{1}{2}\left(\dfrac{1}{1}-\dfrac{1}{2n+1}\right)\\
				&&=\dfrac{1}{2}\cdot\dfrac{2n}{2n+1}=\dfrac{n}{2n+1}.
			\end{eqnarray*}
			\item 
			
			Với mọi $n \in \mathbb{N}^\ast$ thì $u_n>0$ nên dãy số $\left(u_n\right)$ bị chặn dưới.\\
			Ta lại có $u_n=\dfrac{1}{2}\cdot\left(1-\dfrac{1}{2n+1}\right)<\dfrac{1}{2}$.\\
			Vậy dãy số bị chặn.
		\end{itemize}
	}
\end{ex}
%Câu 8
\begin{ex}%[1K2G5-4]%[Trương Đăng Khoa]
	Cho dãy số $\left(u_n\right)$ và đặt $u_n= \displaystyle \sum_{k=1}^{n} a_k$ với $a_k=\dfrac{1}{k(k+4)}$.  Dãy số $\left(u_n\right)$ bị chặn dưới và chặn trên lần lượt bởi các số $m$ và $M$ nào sau đây?
	\choice
	{\True $m=0$, $M=\dfrac{25}{48}$}
	{$m=0$, $M=\dfrac{25}{12}$}
	{$m=1$, $M=\dfrac{1}{4}$}
	{$m=1$, $M=\dfrac{1}{2}$}
	\loigiai{
		Ta có $a_k=\dfrac{1}{k(k+4)}=\dfrac{1}{4}\cdot\dfrac{4}{k(k+4)}=\dfrac{1}{4}\cdot\dfrac{k+4-k}{k(k+4)}=\dfrac{1}{4}\cdot\left(\dfrac{1}{k}-\dfrac{1}{k+4}\right)$.\\
		Mặt khác $u_n=\displaystyle \sum_{k=1}^{n} a_k$.
		Do đó
		\begin{eqnarray*}
			&u_n&=\dfrac{1}{4}\cdot\left(\dfrac{1}{1}-\dfrac{1}{5}\right)+\dfrac{1}{4}.\left(\dfrac{1}{2}-\dfrac{1}{6}\right)+\ldots + \dfrac{1}{4}\cdot\left(\dfrac{1}{n}-\dfrac{1}{n+4}\right)\\
			&&=\dfrac{1}{4}\left(\dfrac{1}{1}+\dfrac{1}{2}+\dfrac{1}{3}+\dfrac{1}{4}-\dfrac{1}{n+1}-\dfrac{1}{n+2}-\dfrac{1}{n+3}-\dfrac{1}{n+4}\right)\\
			&&=\dfrac{1}{4}\left(\dfrac{25}{12}-\dfrac{1}{n+1}-\dfrac{1}{n+2}-\dfrac{1}{n+3}-\dfrac{1}{n+4}\right).
		\end{eqnarray*}
		Với mọi $n \in \mathbb{N}^\ast$ thì $u_n>0$ nên dãy số $\left(u_n\right)$ bị chặn dưới.\\
		Ta lại có $u_n=\dfrac{1}{4}\cdot\left(\dfrac{25}{12}-\dfrac{1}{n+1}-\dfrac{1}{n+2}-\dfrac{1}{n+3}-\dfrac{1}{n+4}\right)<\dfrac{1}{4}\cdot\dfrac{25}{12}=\dfrac{25}{48}$.\\
		Vậy $m=0$, $M=\dfrac{25}{48}$.
	}
\end{ex}
%Câu 9
\begin{ex}%[1K2G5-4]%[Trương Đăng Khoa]
	Xét tính bị chặn của dãy số $\left(u_n\right)$ và đặt $u_n=\displaystyle \sum_{k=1}^{n} a_k$ với $a_k=\dfrac{1}{k(k+1)}$.
	\choice{\True Bị chặn}{Bị chặn dưới}{Bị chặn trên}{Không bị chặn.}
	\loigiai{
		Ta có $a_k=\dfrac{1}{k(k+1)}=\dfrac{1}{k}-\dfrac{1}{k+1}$. Do đó\\
		$u_n=\displaystyle \sum_{k=1}^{n}a_k=\left(1-\dfrac{1}{2}\right)+\left(\dfrac{1}{2}-\dfrac{1}{3}\right)+\ldots+\left(\dfrac{1}{n-1}-\dfrac{1}{n}\right)+\left(\dfrac{1}{n}-\dfrac{1}{n+1}\right)=1-\dfrac{1}{n+1}=\dfrac{n}{n+1}$.\\
		Với mọi $n \in \mathbb{N}^*$ thì $u_n>0$ nên dãy số $\left(u_n\right)$ bị chặn dưới.\\
		Ta lại có $u_n=1-\dfrac{n}{n+1}<1$, $\forall n \in \mathbb{N}^\ast$ nên dãy số $\left(u_n\right)$ bị chặn trên.\\
		Vậy dãy số bị chặn.
	}
\end{ex}
%Câu 10
\begin{ex}%[1K2G5-4]%[Trương Đăng Khoa]
	Cho dãy số $(u_n)$, xác định bởi $\heva{&u_1=6\\&u_{n+1}=\sqrt{6+u_n},\, \forall n\in\mathbb{N}^\ast}$. Mệnh đề nào sau đây là đúng?
	\choice{
		$\sqrt{6}<u_n<2\sqrt{3}$	
	}
	{\True $\sqrt{6}\leq u_n\leq 2\sqrt{3}$}
	{$\sqrt{6}<u_n\leq 2\sqrt{3}$	}
	{$\sqrt{6}\geq u_n<2\sqrt{3}$	}
	\loigiai{
		Ta có 
		$\heva{&u_1=6\\
			&u_{n+1}=\sqrt{6+u_n}} \Rightarrow
		\heva{
			&u_1=6\\
			&u_{n+1} \ge 0 } \Rightarrow u_n \ge 0 \Rightarrow\heva{&u_1=6\\
			&u_{n+1}=\sqrt{6+u_n}\ge\sqrt{6}}
		\Rightarrow u_n \ge\sqrt{6}$.\\
		Ta chứng minh quy nạp $\heva{&u_n\le 2\sqrt{3}\\ &u_1\le 2\sqrt{3}\\ &u_k\le 2\sqrt{3}.}$\\
		$\Rightarrow u_{k+1}=\sqrt{6+u_{k+1}} \le\sqrt{6+2 \sqrt{3}}<\sqrt{6+6}=2 \sqrt{3}$.\\
		Vậy $\sqrt{6} \leq  u_n \leq 2\sqrt{3}$.
	}
\end{ex}
\Closesolutionfile{ans}
\begin{indapan}{10}
	{ans/ans-1K2-1-Dang3}
\end{indapan}

\begin{dang}{Toán thực tế về dãy số}
\end{dang}
\subsubsection{Ví dụ minh hoạ}
% \begin{vd}%[1T2B1-5]%[Trương Đăng Khoa]%Ví dụ 1
% 	Một chồng cột gỗ được xếp thành các lớp, hai lớp liên tiếp hơn kém nhau một cột gỗ.
% 	\begin{center}
% 		\begin{tikzpicture}[font=\footnotesize, line join=round, line cap=round, >=stealth,scale=0.8]
% 			\def\r{0.2}
% 			\def\n{25}
% 			\def\g{110}
% 			\fill[teal!50!green](-6*\r,-3*\r)rectangle(3.5*\n*\r,0.5*\n*\r);
% 			\fill[teal!50!green,opacity=0.25](3.5*\n*\r,3*\r)rectangle(-6*\r,2*\n*\r);
% 			\foreach \j in {0,...,12}{
% 				\pgfmathsetmacro{\m}{\n-\j}
% 				\foreach \i in{0,...,\m}{
% 					\fill[left color=orange, right color=teal!30,draw=brown](2*\i*\r,0)++(60:2*\j*\r)++(\g:\r)--++(\g-90:6)arc(\g:-60:\r)--++(\g-270:6)--cycle;
% 					\fill[orange!20!brown!40,draw=teal](2*\i*\r,0)++(60:2*\j*\r)circle(\r);
% 				}
% 			}
% 		\end{tikzpicture}
% 	\end{center}
% 	\begin{enumerate}
% 		\item  Gọi $u_1=25$ là số cột gỗ có ở hàng dưới cùng của chồng cột gỗ, $u_n$ là số cột gỗ có ở hàng thứ $n$ tính từ dưới lên trên. Xét tính tăng, giảm của dãy số này.
% 		\item  Gọi $v_1=14$ là số cột gỗ có ở hàng trên cùng của chồng cột gỗ, $v_n$ là số cột gỗ có ở hàng thứ $n$ tính từ trên xuống dưới. Xét tinh tăng, giảm của dãy số này.
% 	\end{enumerate}
% 	\loigiai{
% 		\begin{enumerate}
% 			\item Ta có $u_n=26-n>u_{n+1}=26-n-1=25-n$.\\
% 			Vậy dãy số $(u_n)$ là dãy số giảm.
% 			\item Ta có $v_n=13+n<v_{n+1}=13+n+1=14+n$.\\
% 			Vậy dãy số $(u_n)$ là dãy số tăng
% 		\end{enumerate}
% 	}
% \end{vd}

\begin{vd}%[1T2B1-5]%[Trương Đăng Khoa]%Ví dụ 2
	Trên lưới ô vuông, mỗi ô cạnh $1$ đơn vị, người ta vẽ $8$ hình vuông và tô màu khác nhau như hình vẽ. Tìm dãy số biểu diễn độ dài cạnh của $8$ hình vuông đó từ nhỏ đến lớn. Có nhận xét gì về dãy số trên?
	\begin{center}
		\begin{tikzpicture}[scale=0.8]
			\def\r{21}
			\def\hv(#1){
				\ifnum #1= 1\else
				\pgfmathsetmacro{\R}{250*rnd}
				\pgfmathsetmacro{\G}{250*rnd}
				\pgfmathsetmacro{\B}{250*rnd}
				\definecolor{mau}{RGB}{\R,\G,\B}
				\fill[mau!30](0,0)rectangle(\r,\r);
				\draw[red,line width=1pt] (0,0) arc(180:90:\r)(0,0)rectangle(\r,\r);
				\pgfmathtruncatemacro{\k}{#1-1}
				\begin{scope}[shift={(45:\r*sqrt(2))},rotate=-90,scale={(sqrt(5)-1)/2}]
					\hv(\k)
					\pgfmathsetmacro{\n}{int((1/(sqrt(5))*(((1+sqrt(5))/2)^(\k)-(1-(sqrt(5))/2)^(\k)+1)}
					\ifnum \k>1
					\path (\r/2,\r/2)node[scale=1.75]{\color{red}$\n$};
					\else
					\fi
				\end{scope}
				\fi
			}
			\begin{scope}[scale=0.35]
				\hv(9)
				\draw[teal](0,0)grid(34,21);
				\path(21/2,21/2)node[scale=2]{21};
			\end{scope}
		\end{tikzpicture}
	\end{center}
	\loigiai{
		\begin{multicols}{4}
			\begin{itemize}
				\item $u_1=1$.
				\item $u_2=1$.
				\item $u_3=2$.
				\item $u_4=3$.
				\item $u_5=5$.
				\item $u_6=8$.
				\item $u_7=13$.
				\item $u_8=21$.
			\end{itemize}
		\end{multicols}
		Ta có dãy số $\left(u_n\right)\colon\heva{& u_1=1\\ &u_2=1\\ &u_n=u_{n-1}-u_{n-2}.}$
	}
\end{vd}

\begin{vd}%[1C2K1-5]%[Trương Đăng Khoa]% Ví dụ 3
	Chị Mai gửi tiền tiết kiệm vào ngân hàng theo thể thức lãi kép như sau. Lần đầu chị gửi $100$ triệu đồng. Sau đó, cứ hết $1$ tháng chị lại gửi thêm vào ngân hàng $6$ triệu đồng. Biết lãi suất của ngân hàng là $0{,}5\%$ một tháng. Gọi $P_n$ (triệu đồng) là số tiền chị có trong ngân hàng sau $n$ tháng.
	\begin{enumerate}
		\item Tính số tiền chị có trong ngân hàng sau $1$ tháng.
		\item Tính số tiền chị có trong ngân hàng sau $3$ tháng.
		\item Dự đoán công thức của $P_n$ tính theo $n$.
	\end{enumerate}
	\loigiai{
		\begin{enumerate}
			\item Số tiền chị có trong ngân hàng sau $1$ tháng là $P_1=+100+100\cdot0{,}5\%+6=100{,}5+6$ (triệu đồng).
			\item Số tiền chị có trong ngân hàng sau 2 tháng là 
			\begin{eqnarray*}
				P_2&=&100{,}5+6+(100{,}5+6)\cdot 0{,}5\%+6\\
				&=&(100{,}5+6)(1+0{,}5\%)+6\\
				&=& 100{,}5(1+0{,}5\%)+6\cdot(1+0{,}5\%)+6\, (\text{triệu đồng}).
			\end{eqnarray*}
			Số tiền chị có trong ngân hàng sau $3$ tháng là
			\begin{eqnarray*}
				P_3&=&(100{,}5+6)(1+0{,}5 \%)+6+[(100{,}5+6)(1+0{,}5 \%)+6] \cdot 0{,}5 \%+6\\
				&=& 100{,}5 \cdot(1+0{,}5 \%)^2+6(1+0{,}5 \%)^2+6 \cdot(1+0{,}5 \%)+6 \text{(triệu đồng)}.
			\end{eqnarray*}
			\item Số tiền chị có trong ngân hàng sau $4$ tháng là
			\begin{eqnarray*}
				P_4&=&(100{,}5+6)(1+0{,}5 \%)^2+6 \cdot(1+0{,}5 \%)+6+\left[(100{,}5+6)(1+0{,}5 \%)^2+6 \cdot(1+0{,}5 \%)+6\right]\cdot 0{,}5 \%+6\\ 
				&=&100{,}5 \cdot(1+0{,}5 \%)^3+6 \cdot(1+0{,}5 \%)^3+6\cdot(1+0{,}5 \%)^2+6 \cdot(1+0{,}5 \%)+6\, (\text{triệu đồng}).
			\end{eqnarray*}
			Số tiền chị có trong ngân hàng sau $n$ tháng là
			$$P_n=100{,}5 \cdot(1+0{,}5 \%)^{n-1}+6\cdot(1+0{,}5 \%)^{n-1}+6\cdot(1+0{,}5 \%)^{n-2}+6 \cdot(1+0{,}5 \%)^{n-3}+\ldots+6$$ với mọi $n \in\mathbb{N}^\ast$.
		\end{enumerate}
	}
\end{vd}

\begin{vd}%[1K2K5-5]%[Trương Đăng Khoa]% Ví dụ 4
	Anh Thanh vừa được tuyển dụng vào một công ty công nghệ, được cam kết lương năm đầu sẽ là $200$ triệu đồng và lương mỗi năm tiếp theo sẽ được tăng thêm $25$ triệu đồng. Gọi $s_n$ (triệu đồng) là lương vào năm thứ $n$ mà anh Thanh làm việc cho công ty đó. Khi đó ta có
	$$s_1=200,\, s_n=s_{n-1}+25\, \text{với}\, n \ge 2.$$
	\begin{enumerate}
		\item Tính lương của anh Thanh vào năm thứ $5$ làm việc cho công ty.
		\item Chứng minh $(s_n)$ là dãy số tăng. Giải thích ý nghĩa thực tế của kết quả này. 
	\end{enumerate}
	\loigiai{
		\begin{enumerate}
			\item Ta có \begin{eqnarray*}
				s_2&=&s_1+25=200+25=225\\
				s_3&=&s_2+25=225+25=250\\
				s_4&=&s_3+25=250+25=275\\
				s_5&=&s_4+25=275+25=300. 
			\end{eqnarray*}
			Vậy lương của anh Thanh vào năm thứ $5$ làm việc cho công ty là $300$ triệu đồng.
			\item  Ta có $s_n=s_{n-1}+25\Leftrightarrow s_n-s_{n-1}=25>0$ với mọi $n\ge 2$, $n\in\mathbb{N}^\ast$.\\
			Tức là $s_n>s_{n-1}$ với mọi $n\ge 2$, $n\in\mathbb{N}^\ast$.\\
			Vậy $(s_n)$ là dãy số tăng.\\
			Điều này có nghĩa là mức lương hàng năm của anh Thanh tăng dần theo thời gian làm việc.
		\end{enumerate}
	}
\end{vd}

\begin{vd}%[1K2K5-5]%[Trương Đăng Khoa]%Ví dụ 5
	Ông An gửi tiết kiệm $100$ triệu đồng kì hạn $1$ tháng với lãi suất $6\%$ một năm theo hình thức tính lãi kép. Số tiền (triệu đồng) của ông An thu được sau $n$ tháng được cho bởi công thứC 
	$$A_n=100\left(1+\dfrac{0{,}06}{12}\right)^n.$$
	\begin{enumerate}
		\item Tìm số tiền ông An nhận được sau tháng thứ nhất, sau tháng thứ hai.
		\item Tìm số tiền ông An nhận được sau $1$ năm.
	\end{enumerate}
	\loigiai{
		\begin{enumerate}
			\item Số tiền ông An nhận được sau tháng thứ nhất là 
			$$A_1=100\left(1+\dfrac{0{,}06}{12}\right)^1=100{,}5\, \text{(triệu đồng)}.$$
			Số tiền ông An nhận được sau tháng thứ hai là 
			$$A_2=100\left(1+\dfrac{0{,}06}{12}\right)^2=101{,}0025\, \text{(triệu đồng)}.$$
			\item  Số tiền ông An nhận được sau $1$ năm ($12$ tháng) là 
			$$A_{12}=100\left(1+\dfrac{0{,}06}{12}\right)^{12} \approx 106{,}17\, \text{(triệu đồng)}.$$
		\end{enumerate}
	}
\end{vd}

\begin{vd}%[1K2G5-5]%[Trương Đăng Khoa]%Ví dụ 6
	Chị Hương vay trả góp một khoản tiền $100$ triệu đồng và đồng ý trả dần $2$ triệu đồng mỗi tháng với lãi suất $0{,}8\%$ số tiền còn lại của mỗi tháng.
	Gọi $A_n$, ($n\in\mathbb{N}$) là số tiền còn nợ (triệu đồng) của chị Hương sau $n$ tháng.
	\begin{enumerate}
		\item Tìm lần lượt $A_0$, $A_1$, $A_2$, $A_3$, $A_4$, $A_5$, $A_6$ đễ tính số tiền còn nợ của chị Hương sau $6$ tháng.
		\item  Dự đoán hệ thức truy hồi đối với dãy số $(A_n)$.
	\end{enumerate}
	\loigiai{
		\begin{enumerate}
			\item  Ta có $A_0=100$ (triệu đồng).
			\begin{itemize}
				\item Tiền lãi chị Hương phải trả sau $1$ tháng là $100\cdot 0{,}8\%=0{,}8$ (triệu đồng).\\
				Do đó, số tiền gốc chị Hương trả được sau $1$ tháng là $2-0{,}8=1{,}2$ (triệu đồng).\\
				Khi đó, số tiền còn nợ của chị Hương sau $1$ tháng là 
				$A_1=100-1{,}2=98{,}8$ (triệu đồng).
				\item  Tiền lãi chị Hương phải trả sau $2$ tháng là $98{,}8\cdot 0{,}8\%=0{,}7904$ (triệu đồng).\\
				Do đó, số tiền gốc chị Hương trả được sau $2$ tháng là $2-0{,}7904=1{,}2096$ (triệu đồng).\\
				Khi đó, số tiền còn nợ của chị Hương sau $2$ tháng là 
				$A_2=98{,}8-1{,}2096=97{,}5904$ (triệu đồng).
				\item Tiền lãi chị Hương phải trả sau $3$ tháng là $97{,}5904\cdot 0{,}8\%=0{,}7807232$ (triệu đồng).\\
				Do đó, số tiền gốc chị Hương trả được sau $3$ tháng là $2-0{,}7807232=1{,}2192768$ (triệu đồng).\\
				Khi đó, số tiền còn nợ của chị Hương sau $3$ tháng là 
				$A_3=97{,}5904-1{,}2192768=96{,}3711232$ (triệu đồng).
				\item Tiền lãi chị Hương phải trả sau $4$ tháng là $96{,}3711232\cdot 0{,}8\%\approx 0{,}77097$ (triệu đồng).\\
				Do đó, số tiền gốc chị Hương trả được sau $4$ tháng là $2-0{,}77097=1{,}22903$ (triệu đồng).\\
				Khi đó, số tiền còn nợ của chị Hương sau $4$ tháng là 
				$A_4=96{,}3711232-1{,}22903=95{,}1420932$ (triệu đồng).
				\item Tiền lãi chị Hương phải trả sau $5$ tháng là $95{,}1420932\cdot 0{,}8\%\approx 0{,}76114$ (triệu đồng).\\
				Do đó, số tiền gốc chị Hương trả được sau $5$ tháng là $2-0{,}76114=1{,}23886$ (triệu đồng).\\
				Khi đó, số tiền còn nợ của chị Hương sau $5$ tháng là $A_5=95{,}1420932-1{,}23886=93{,}9032332$ (triệu đồng).
				\item Tiền lãi chị Hương phải trả sau $6$ tháng là $93{,}9032332\cdot 0{,}8\%\approx 0{,}75123$ (triệu đồng).\\
				Do đó, số tiền gốc chị Hương trả được sau $6$ tháng là $2-0{,}75123=1{,}24877$ (triệu đồng).\\
				Khi đó, số tiền còn nợ của chị Hương sau $6$ tháng là $A_6=93{,}9032332-1{,}24877=92{,}6544632$ (triệu đồng).
			\end{itemize}
			\item Dự đoán hệ thức truy hồi đối với dãy số $(A_n)$ là 
			\[A_0=100,\, A_n=A_{n-1}-\left(2-A_{n-1} \cdot 0{,}8 \%\right)=1{,}008 A_{n-1}-2\]
		\end{enumerate}
	}
\end{vd}

%%Bài 6. CSC
% \def\tenchude{CẤP SỐ CỘNG}
\setcounter{section}{5}
\setcounter{dang}{0}
\setcounter{ex}{0}
\setcounter{bt}{0}
\setcounter{vd}{0}
\section{Cấp số cộng}
\subsection{Tóm tắt lý thuyết}
\begin{tomtat}
	\subsubsection{Định nghĩa}
	Dãy số là cấp số cộng nếu mỗi một số hạng (kể từ số hạng thứ hai) đều bằng tổng của số hạng đứng ngay trước nó với một số không đổi $ d $.\\
	Dãy số $ (u_n) $ là cấp số cộng $ \Leftrightarrow u_{n+1}=u_n+d $, $ \forall n \in \mathbb{N}^* $.\\
	$ d $ là số không đổi, gọi là \textbf{\textit{công sai}} của cấp số cộng.
	\subsubsection{Tính chất}
	Nếu $ (u_n) $ là cấp số cộng thì kể từ số hạng thứ hai (trừ số hạng cuối nếu là cấp số cộng hữu hạn) đều là trung bình cộng của hai số hạng đứng kề nó trong dãy. Tức là $$u_k=\dfrac{u_{k-1}+u_{k+1}}{2}, (\forall k\ge 2, k \in \mathbb{N}^*).$$
	\subsubsection{Số hạng tổng quát}
	Nếu cấp số cộng $ (u_n) $ có số hạng đầu $ u_1 $ và công sai $ d $ thì số hạng tổng quát $ u_n $ được xác định bởi công thức $$u_n=u_1+(n-1)d \text{ với $n\ge 2$}.$$
	\subsubsection{Tổng $ n $ số hạng đầu tiên}
	Cho cấp số cộng $ (u_n) $. Tổng $ n $ số hạng đầu tiên của cấp số cộng kí hiệu là $ S_n=u_1+u_2+\ldots+u_n $.\\
	Khi đó $ S_n $ được tính theo công thức $$ S_n=\dfrac{n(u_1+u_n)}{2}=\dfrac{n}{2}\left[ 2u_1+(n-1)d\right]. $$
\end{tomtat}
\subsection{Các dạng toán thường gặp}
\begin{dang}{Nhận diện cấp số cộng, công sai $ d $, số hạng tổng quát $ u_n $}
	% Dựa theo định nghĩa của cấp số cộng, để nhận diện $ (u_n) $ là cấp số cộng $ \Leftrightarrow u_{n+1}=u_n+d $.\\
	% Khi đó công sai $ d=u_{n+1}-u_{n} $, $ \forall n \in \mathbb{N}^* $.
\end{dang}
\subsubsection{Ví dụ minh hoạ}
\begin{vd}%[NB]%[DCHT Toán 11 - KNTT -Lê Hải Phụng] %[1K2Y6-1]
	Dãy số hữu hạn nào là một cấp số cộng? Vì sao?
	\begin{listEX}[2]
		\item  $-2$, $1$, $4$, $7$, $10$, $13$, $16$.
		\item  $ 1 $, $ -2 $, $ -4 $, $ -6 $, $ -8 $.
	\end{listEX}
	\dapso{ Dãy số 1 là một cấp số cộng, dãy số 2 không là một cấp số cộng.}
	\loigiai{
		\begin{enumerate}
			\item Ta thấy $ u_2=u_1+3 $  do $ 1=(-2)+3 $.\\
			Vì $ u_k=u_{k-1}+d,\ \forall k\geq2$ $\left(\ 1=\left(-2\right)+3;4=1+3;7=4+3;10=7+3;13=10+3;16=13+3\right) $ nên dãy số đã cho là cấp số cộng. 
			\item Ta thấy $ u_2=u_1+(-3) $  do $-2=1+(-3)$.\\
			Vì $ {u_3\neq u}_2+(-3) $ bởi $ \left(\ -4\neq-2+\left(-3\right)\right)\ $ nên dãy số đã cho không là cấp số cộng.
		\end{enumerate}
	}
\end{vd}
\begin{vd}%[TH]%[DCHT Toán 11 - KNTT -Lê Hải Phụng] %[1K2B6-1]
	Trong các dãy số dưới đây, dãy số nào là cấp số cộng? 
	\begin{listEX}[2]
		\item  Dãy số $\left({a_n}\right)$ với ${a_n}=4n-3$;
		\item  Dãy số $\left({c_n}\right)$ với ${c_n}={2018^n}$.
	\end{listEX}
	\dapso{Dãy số 1 là một cấp số cộng, dãy số 2 không là một cấp số cộng.}
	\loigiai{	
		\begin{enumerate}
			\item Ta có $a_{n+1}=4(n+1)-3=4n+1$ nên $a_{n+1}-a_n=(4n+1)-(4n-3)=4$,$\forall n\ge 1.$.\\
			Do đó $(a_n)$ là cấp số cộng.
			\item Ta có $c_{n+1}=2018^{n+1}$ nên $c_{n+1}-c_n=2018^{n+1}-2018^n=2017\cdot 2018^n$ (phụ thuộc vào giá trị của $n$).\\ 
			Suy ra $(c_n)$ không phải là một cấp số cộng.
		\end{enumerate}	
	}
\end{vd}
\begin{vd}%[NB]%[DCHT Toán 11 - KNTT -Lê Hải Phụng] %[1K2Y6-1]
	Cho cấp số cộng $(u_n)$  có công thức số hạng tổng quát $u_n=3n+1$, $n\in\mathbb{N}^\ast$ . Tìm số hạng đầu $u_1$ và công sai $d$?
	\dapso{$u_1=4 $, $d=3$.}
	\loigiai{
		Từ công thức số hạng tổng quát, ta có $ u_1=4 $, $u_2=7$ suy ra $d=u_2-u_1=3$.
	}
\end{vd}

\begin{vd}%[TH]%[DCHT Toán 11 - KNTT -Lê Hải Phụng] %[1K2B6-1]
	Cho cấp số cộng $(u_n)$ với $u_1=3$, $u_2=9$. Công sai của cấp số cộng đã cho bằng bao nhiêu?
	\dapso{$ d=6 $}
	\loigiai{
		Cấp số cộng $(u_n)$ có số hạng tổng quát là $u_n=u_1+(n-1)d$ với $n \ge 2$.\\
		Suy ra $u_2=u_1+d \Leftrightarrow 9=3+d \Leftrightarrow d=6$.\\
		Vậy công sai của cấp số cộng đã cho là $6$.
	}
\end{vd}
\begin{vd}%[VD]%[DCHT Toán 11 - KNTT -Lê Hải Phụng] %[1K2K6-1]
	Tính số hạng đầu $u_1$ và công sai $d$ của một cấp số cộng biết $u_4=10$ và $u_7=19$.
	\dapso{$ u_1=1 $, $ d=3 $.}
	\loigiai{Ta có $ \heva{& u_4=10 \\ & u_7=19} \Leftrightarrow \heva{& u_1+3d=10 \\ & u_1+6d=19} \Leftrightarrow \heva{& u_1=1 \\ & d=3.}$}
\end{vd}

\begin{vd}%[TH]%[Dự án DCHT-11-KNTT]%[Dao-V- Thuy]%[1K2B5-1]
	Xác định số hạng tổng quát của cấp số cộng $(u_n),$ biết $\heva{&u_7=8\\ &d=2.}$
	\dapso{$u_n=2n-6$}
	\loigiai{
		Ta có
		\begin{equation*}
			\heva{&u_7=8\\ &d=2} \Leftrightarrow \heva{&u_1+6d=8\\&d=2} \Leftrightarrow \heva{&u_1=-4\\ &d=2.}
		\end{equation*}
		Vậy công thức tổng quát của cấp số cộng
		\begin{center}
			$u_n=-4+(n-1)2 \Leftrightarrow u_n=2n-6 $ với $n \geq 2.$
		\end{center}	
	}
\end{vd}

% \begin{vd}%[TH]%[Dự án DCHT-11-KNTT]%[Dao-V- Thuy]%[1K2B5-1]
% 	Tìm số hạng đầu và công sai của cấp số cộng $(u_n)$, biết $\heva{&u_1+u_5-u_3=10\\ &u_1+u_6=17.}$
% 	\dapso{$u_1=16$, $d=-3$}
% 	\loigiai{
% 		Ta có
% 		\begin{align*}
% 			\heva{&u_1+u_5-u_3=10\\ &u_1+u_6=17} &\Leftrightarrow \heva{&u_1+u_1+4d-(u_1+2d)=10\\ &u_1+u_1+5d=17}\\ & \Leftrightarrow\heva{&u_1+2d=10 \\ &2u_1+5d=17} \Leftrightarrow \heva{&u_1=16 \\ &d=-3.}
% 		\end{align*}
% 		Vậy $u_1=16$, $d=-3$.
% 	}
% \end{vd}

\begin{vd}%[TH]%[Dự án DCHT-11-KNTT]%[Dao-V- Thuy]%[1K2B5-1]
	Cho cấp số cộng $(u_n)$ với $\heva{&u_1=-9\\ &u_{n-1}=u_n-5}$. Tìm số hạng tổng quát của cấp số cộng $(u_n)$.
	\dapso{$u_n= 5n-14$}
	\loigiai{
		Từ công thức $u_{n-1}=u_n-5 \Leftrightarrow u_n= u_{n-1}+5$, suy ra $d=5$.\\
		Vậy công thức tổng quát của cấp số cộng $(u_n)$ là $u_n=-9 + 5(n-1)= 5n-14$.
	}
\end{vd}

\begin{vd}%[TH]%[Dự án DCHT-11-KNTT]%[Dao-V- Thuy]%[1K2B5-1]
	Cho cấp số cộng $(u_n)$ có $u_{20}=-52$ và $u_{51}=-145$. Hãy tìm số hạng tổng quát của cấp số cộng đó.
	\dapso{$u_n= -3n+8$}
	\loigiai{
		Ta có
		\begin{eqnarray*}
			\heva{&u_{20}=-52 \\&u_{51}=-145} &\Leftrightarrow& \heva{&u_1+19d=-52\\ &u_1+50d=-145} \Leftrightarrow \heva{&u_1=5 \\ &d= -3.}
		\end{eqnarray*}
		Vậy số hạng tổng quát cần tìm là $u_n= u_1+ (n-1)d= 5+(n-1) \cdot (-3)= -3n+8$.
	}
\end{vd}

\begin{vd}%[VD]%[Dự án DCHT-11-KNTT]%[Dao-V- Thuy]%[1K2B5-1]
	Tìm số hạng đầu và công sai của cấp số cộng $(u_n)$, biết
		\begin{listEX}[2]
			\item $\heva{&u_9=5u_2\\ &u_{13}=2u_6+5.}$
			\item $\heva{&u_1-u_3+u_5=10\\ &u_1+u_6=7.}$
		\end{listEX}
	\dapso{$u_1=3$, $d=4$; $u_1=36$, $d=-13$}
	\loigiai
	{
		\begin{enumerate}
			\item Ta có
			\begin{eqnarray*}
				\heva{&u_9=5u_2\\ &u_{13}=2u_6+5} &\Leftrightarrow& \heva{&u_1+8d= 5 \left( u_1+d \right) \\ &u_1+12d = 2 \left( u_1+5d\right) + 5}\\
				&\Leftrightarrow& \heva{&-4u_1+3d=0\\ &-u_1+2d=5} \Leftrightarrow \heva{&u_1=3 \\ &d=4.}
			\end{eqnarray*}
			Vậy $u_1=3$, $d=4$.
			\item Ta có 
			\begin{eqnarray*}
				\heva{&u_1-u_3+u_5=10\\ &u_1+u_6=7} &\Leftrightarrow& \heva{&u_1-\left( u_1+2d\right) + \left( u_1+4d\right) = 10\\ &u_1+ \left( u_1+5d\right) = 7}\\
				&\Leftrightarrow& \heva{&u_1+2d=10\\ &2u_1+5d=7} \Leftrightarrow \heva{&u_1=36 \\ &d=-13.}
			\end{eqnarray*}
			Vậy $u_1=36$, $d=-13$.
		\end{enumerate}
	}
\end{vd}

\begin{vd}%[VD]%[Dự án DCHT-11-KNTT]%[Dao-V- Thuy]%[1K2K5-1]
	Tìm số hạng đầu và công sai của cấp số cộng $(u_n)$, biết
		\begin{listEX}[2]
			\item $\heva{&-u_3+u_7=8\\ &u_2u_7=75.}$
			\item $\heva{&u_5=4u_3\\ &u_2u_6=-11.}$
		\end{listEX}
	\dapso{$\heva{&u_1=3\\ &d=2} \text{ hoặc } \heva{&u_1=-17\\ &d=2}$; $\heva{&u_1=-4\\ &d=3}$ hoặc $\heva{&u_1=4\\ &d=-3}$}
	\loigiai{
		\begin{enumerate}
			\item Ta có
			\begin{eqnarray*}
				\heva{&-u_3+u_7=8\\ &u_2u_7=75} &\Leftrightarrow& \heva{&-\left( u_1+2d\right) + \left( u_1+6d\right) = 8\\ &\left( u_1+d\right) \left( u_1+6d\right) = 75}\\
				&\Leftrightarrow& \heva{&4d=8\\ &u_1^2+7u_1d+6d^2=75}\\
				&\Leftrightarrow& \heva{&d=2\\ &u_1^2+14u_1-51=0}\\
				&\Leftrightarrow& \heva{&u_1=3\\ &d=2} \text{ hoặc } \heva{&u_1=-17\\ &d=2.}
			\end{eqnarray*}
			Vậy $\heva{&u_1=3\\ &d=2} \text{ hoặc } \heva{&u_1=-17\\ &d=2.}$
			\item Ta có 
			\begin{eqnarray*}
				\heva{&u_5=4u_3\\ &u_2u_6=-11} &\Leftrightarrow& \heva{&u_1+4d=4 \left( u_1+2d\right)\\ &\left( u_1+d\right) \left( u_1+5d\right)= -11}\\
				&\Leftrightarrow& \heva{&3u_1+4d=0 &(1)\\ &u_1^2+6du_1+5d^2=-11 &(2)}
			\end{eqnarray*}
			Từ $(1)$ suy ra $3u_1=-4d$. Thay vào $(2)$ ta được
			\begin{eqnarray*}
				9u_1^2+54du_1+45d^2=-99 &\Leftrightarrow& 16d^2 -72d^2+45d^2=-99\\
				&\Leftrightarrow& -11d^2=-99 \Leftrightarrow \hoac{&d=3\\ &d=-3.}
			\end{eqnarray*}
			Với $d=3$, ta có $u_1=-4$.\\
			Với $d=-3$, ta có $u_1=4$.\\
			Vậy $\heva{&u_1=-4\\ &d=3}$ hoặc $\heva{&u_1=4\\ &d=-3.}$
		\end{enumerate}
	}
\end{vd}

\subsubsection{Bài tập tự luận}
 

\begin{bt}%[NB]%[DCHT Toán 11 - KNTT -Lê Hải Phụng] %[1K2Y6-1]
	Trong các dãy số sau, dãy số nào là một cấp số cộng?
	\begin{listEX}[1]
		\item $1$, $-3$, $-7$, $-11$, $-15$, $\ldots$;
		\item $1$, $-2$, $-4$, $-6$, $-8,$ $\ldots$.
		\item $ \dfrac{1}{2} $, $0$, $-\dfrac{1}{2}$, $-1$, $-\dfrac{3}{2}$, $\ldots$
	\end{listEX}
	\dapso{1) và 3) là cấp số cộng.}
	\loigiai{Ta lần lượt đi kiểm tra: $ u_2-u_1=u_3-u_2=u_4-u_3=\ldots $?\\
		Xét từng dãy số thì ta thấy 1) và 3) là cấp số cộng. 
	}
\end{bt}

\begin{bt}%[NB]%[DCHT Toán 11 - KNTT -Lê Hải Phụng] %[1K2Y6-1]
	Trong các dãy số sau, dãy nào là cấp số cộng. Tìm số hạng đầu và công sai của cấp số cộng đó.
	\begin{listEX}[2]
		\item Dãy số $ (u_n) $ với $ u_n=19n-5 $;
		\item Dãy số $ (u_n) $ với $ u_n=n^2+n+1 $. 
	\end{listEX}
	\dapso{Dãy số 1) $ (u_n) $ là một cấp số cộng với số hạng đầu là $ u_1=19\cdot1-5=14 $ và công sai $ d=19 $. Dãy số 2) không là một cấp số cộng.}
	\loigiai{
		\begin{enumerate}
			\item Dãy số $ (u_n) $ với $ u_n=19n-5 $.\\
			Ta có $ u_{n+1}-u_n=19(n+1)-5-(19n-5)=19 $. Vậy $ (u_n) $ là một cấp số cộng với số hạng đầu là $ u_1=19\cdot1-5=14 $ và công sai $ d=19 $.
			\item Dãy số $ (u_n) $ với $ u_n=n^2+n+1 $.\\
			Ta có $ u_{n+1}-u_n=(n+1)^2+(n+1)+1-(n^2+n+1)=2n+2$ phụ thuộc vào $ n $. Vậy $ (u_n) $ không là một cấp số cộng.
	\end{enumerate}}
\end{bt}

\begin{bt}%[TH]%[DCHT Toán 11 - KNTT -Lê Hải Phụng] %[1K2B6-1]
	Cho cấp số cộng $\left(u_n\right)$ với $u_1=3$, $u_2=9$. Công sai của cấp số cộng đã cho bằng bao nhiêu?
	\dapso{Công sai của cấp số cộng đã cho là 6.}
	\loigiai{Cấp số cộng $(u_n)$ có số hạng tổng quát là
		$u_n=u_1+\left(n-1\right)d$ với $n \ge 2$
		(số hạng đầu $u_1$ và công sai $d$)\\
		Suy ra $ u_2=u_1+d\Leftrightarrow9=3+d\Leftrightarrow d=6 $.\\
		Vậy công sai của cấp số cộng đã cho là 6.
	}
\end{bt}


\begin{bt}%[TH]%[Dự án DCHT-11-KNTT]%[Dao-V- Thuy]%[1K2B5-1]
	Xác định công thức tổng quát của cấp số cộng $(u_n)$, biết $\heva{&u_{11}=5\\ &d=-6.}$
	\loigiai{
		Ta có
		\begin{equation*}
			\heva{&u_{11}=5\\ &d=-6} \Leftrightarrow \heva{&u_1+10d=5\\ &d=-6} \Leftrightarrow \heva{&u_1=65\\ &d=-6.}
		\end{equation*}
		Vậy công thức tổng quát của cấp số cộng:
		\begin{center}
			$u_n=65+(n-1).(-6) \Leftrightarrow u_n=-6n+71$  với $n \geq 2.$
		\end{center}	
	}
\end{bt}

\begin{bt}%[TH]%[Dự án DCHT-11-KNTT]%[Dao-V- Thuy]%[1K2B5-1]
	Tìm số hạng đầu và công sai của cấp số cộng $(u_n),$ biết $\heva{&u_2+u_5-u_3=10\\ &u_4+u_6=26.}$
	\loigiai{
		Ta có
		\begin{align*}
			\heva{&u_2+u_5-u_3=10\\ &u_4+u_6=26} 
			&\Leftrightarrow \heva{&u_1+d+u_1+4d-(u_1+2d)=10\\ &u_1+3d+u_1+5d=26}\\ 
			& \Leftrightarrow\heva{&u_1+3d=10 \\ &2u_1+8d=26}
			\Leftrightarrow \heva{&u_1=1 \\&d=3.}
		\end{align*}
		Vậy $u_1=1$, $d=3$.
	}
\end{bt}

\begin{bt}%[TH]%[Dự án DCHT-11-KNTT]%[Dao-V- Thuy]%[1K2B5-1]
	Tìm số hạng đầu và công sai của cấp số cộng, biết
		\begin{listEX}[3]
			\item $\heva{&u_7 = 27\\&u_{15} = 59.}$
			\item $\heva{&u_9 = 5u_2\\&u_{13} = 2u_6 + 5.}$
			\item $\heva{&u_2 + u_4 - u_6 = -7\\&u_8 - u_7 = 2u_4.}$
			\item $\heva{&u_3 - u_7 = -8\\&u_2 \cdot u_7 = 75.}$
			\item $\heva{&u_6 + u_7 = 60\\&u_4^2 + u_{12}^2 = 1170.}$
		\end{listEX}
	\loigiai{
		\begin{enumerate}
			\item Ta có $\heva{&u_7 = 27\\&u_{15} = 59} \Leftrightarrow \heva{&u_1 + 6d = 27\\&u_1 + 14d = 59} \Leftrightarrow \heva{&u_1 = 3\\&d = 4.}$\\
			Vậy số hạng đầu của cấp số cộng là $u_1 = 3$, công sai là $d = 4$.
			\item Ta có $\heva{&u_9 = 5u_2\\&u_{13} = 2u_6 + 5} \Leftrightarrow \heva{&u_1 + 8d = 5u_1 + 5d\\&u_1 + 12d = 2u_1 + 10d + 5} \Leftrightarrow \heva{&4u_1 - 3d = 0\\&-u_1 + 2d = 5} \Leftrightarrow \heva{&u_1 = 3\\&d = 4.}$\\
			Vậy số hạng đầu của cấp số cộng là $u_1 = 3$, công sai là $d = 4$.
			\item Ta có $\heva{&u_2 + u_4 - u_6 = -7\\&u_8 - u_7 = 2u_4} \Leftrightarrow \heva{&u_1 + d + u_1 + 3d - u_1 - 5d = -7\\&u_1 + 7d - u_1 - 6d = 2u_1 + 6d} \Leftrightarrow \heva{&u_1 - d = -7\\&2u_1 + 5d = 0} \Leftrightarrow \heva{&u_1 = -5\\&d = 2.}$\\
			Vậy số hạng đầu của cấp số cộng là $u_1 = -5$, công sai là $d = 2$.
			\item Ta có $\heva{&u_3 - u_7 = -8\\&u_2 \cdot u_7 = 75} \Leftrightarrow \heva{&u_1 + 2d -u_1 - 6d = -8\\&(u_1 + d)(u_1 + 6d) = 75} \Leftrightarrow \heva{&d = 2\\&u_1^2 + 14u_1 - 51 = 0} \Leftrightarrow \heva{&d = 2\\&\hoac{&u_1 = 3\\&u_1 = -17.}}$\\
			Vậy số hạng đầu của cấp số cộng là $u_1 = 3$, công sai là $d = 2$ hoặc $u_1 = -17$, $d = 2$.
			\item Ta có $\heva{&u_6 + u_7 = 60\\&u_4^2 + u_{12}^2 = 1170} \Leftrightarrow \heva{&2u_6 + d = 60&(1)\\&(u_6 - 2d)^2 + (u_6 + 6d)^2 = 1170.&(2)}$\\
			Từ (1), suy ra $d = 60 - 2u_6$, thay vào (2), ta có
			$$(5u_6 - 120)^2 + (360 - 11u_6)^2 = 1170 \Leftrightarrow 146u_6^2 - 9120u_6 + 142830 = 0 \,\, (\text{vô nghiệm}).$$ 
			Vậy không tồn tại cấp số cộng thỏa yêu cầu bài toán.
		\end{enumerate}
	}
\end{bt}
% \begin{bt}%[TH]%[DCHT Toán 11 - KNTT -Lê Hải Phụng] %[1K2B6-1]
% 	Tìm số hạng đầu tiên, công sai của cấp số cộng sau $ \heva{&u_5=19\\&u_9=35.}$
	
% 	\dapso{Số hạng đầu tiên $ u_1=3 $, công sai $ d=4 $.}
% 	\loigiai{Áp dụng công thức $ u_n=u_1+(n-1)d $ ta có $\heva{&u_5=19\\&u_9=35} \Leftrightarrow \heva{&u_1+4d=19\\&u_1+8d=35} \Leftrightarrow \heva{&u_1=3\\&d=4.}$\\
% 	Vậy số hạng đầu tiên $ u_1=3 $, công sai $ d=4 $.}
% \end{bt}

\begin{bt}%[VD]%[1K2K6-1]
	Cho cấp số cộng $ (u_n) $ thỏa mãn $ \heva{&u_2+u_4-u_6=-7\\&u_8+u_7=2u_4} $. Xác định số hạng đầu $ u_1 $ và công sai $ d $ cấp số cộng.        
	
	% \dapso{$ u_1=5 $, $ d=2 $.}
	\loigiai{
		Ta có $ \heva{&u_2+u_4-u_6=-7\\&u_8+u_7=2u_4} \Leftrightarrow \heva{& u_1+d+(u_1+3d)-(u_1+5d)=-7 \\ & u_1+7d-(u_1+6d)=2(u_1+3d)} \Leftrightarrow \heva{& u_1-d=-7 \\ & 2u_1+5d=0} \Leftrightarrow \heva{&u_1=-5\\&d=2.}$}
\end{bt}

\begin{bt}%[VD]%[DCHT Toán 11 - KNTT -Lê Hải Phụng] %[1K2K6-1]
Cho cấp số cộng $ (u_n) $ thỏa mãn $ \heva{&u_2-u_3+u_5=10\\&u_4+u_6=26} $. Xác định số hạng đầu $ u_1 $ và công sai $ d $ cấp số cộng.         
\dapso{$ u_1=1 $, $ d=3 $.}
\loigiai{Ta có $ \heva{&u_2-u_3+u_5=10\\&u_4+u_6=26} \Leftrightarrow \heva{& u_1+d-(u_1+2d)+u_1+4d=10 \\ & u_1+3d+u_1+5d=26} \Leftrightarrow \heva{& u_1+3d=10 \\ & u_1+4d=13} \Leftrightarrow \heva{u_1=1\\d=3.}$}
\end{bt}

\begin{bt}%[VDC]%[DCHT Toán 11 - KNTT -Lê Hải Phụng] %[1K2G6-1]
Tính số hạng đầu $ u_1 $ và công sai $d$ của một cấp số cộng biết $ \heva{&u_1+u_2+u_3=27\\&u_1^2+u_2^2+u_3^2=275} $

\dapso{$ u_1=5 $, $ d=4 $ hoặc $ u_1=13 $, $ d=-4 $.}
\loigiai{Ta có $ \heva{&u_1+u_2+u_3=27\\&u_1^2+u_2^2+u_3^2=275} \Leftrightarrow \heva{&u_2-d+u_2+u_2+d=27\\&(u_2-d)^2+u_2^2+(u_2+d)^2=275}\Leftrightarrow \heva{&u_2=9\\&3u_2^2+2d^2=275.}$\\
Thay $ u_2=9 $ vào $ 3u_2^2+2d^2=275 $ ta được $ d=4 $ hay $ d=-4 $.\\
Vậy $ u_1=5 $, $ d=4 $ hoặc $ u_1=13 $, $ d=-4 $.}
\end{bt}
\subsubsection{Câu hỏi trắc nghiệm}
\Opensolutionfile{ans}[ans/ans-1K2-2-Dang1]

\begin{ex}%[DCHT Toán 11 - KNTT -Lê Hải Phụng] %[1K2Y6-1]
Trong các dãy số sau, dãy số nào là một cấp số cộng?
\choice
{\True $ 1 $; $ -3 $; $ -7 $; $ -11 $; $ -15 $; $ \ldots $}
{$ 1 $; $ -3 $; $ -6 $; $ -9 $; $ -12 $; $ \ldots $}
{$ 1 $; $ -2 $; $ -4 $; $ -6 $; $ -8 $; $ \ldots $}
{$ 1 $; $ -3 $; $ -5 $; $ -7 $; $ -9 $; $ \ldots $}
\loigiai
{
	Ta lần lượt tính khoảng cách $ d $ các phần tử, ta thấy dãy số đáp án A có $ d= -4$.
}
\end{ex}
%Cau2
\begin{ex}%[DCHT Toán 11 - KNTT -Lê Hải Phụng] %[1K2Y6-1]
Dãy số nào sau đây \textbf{không} phải là cấp số cộng?
\choice
{$ -\dfrac{2}{3} $; $ -\dfrac{1}{3} $; $ 0 $; $ \dfrac{1}{3} $; $ \dfrac{2}{3} $; $ 1 $; $ \dfrac{4}{3} $}
{$ 15\sqrt{2} $; $ 12\sqrt{2} $; $ 9\sqrt{2} $; $ 6\sqrt{2} $}
{\True $ \dfrac{4}{5} $; $ 1 $; $ \dfrac{7}{5} $; $ \dfrac{9}{5} $; $ \dfrac{11}{5} $}
{$ \dfrac{1}{\sqrt{3}} $; $ \dfrac{2\sqrt{3}}{3} $; $ \sqrt{3} $; $ \dfrac{4\sqrt{3}}{3} $; $ \dfrac{5}{\sqrt{3}} $}
\loigiai
{
	Ta lần lượt tính khoảng cách $ d $ các phần tử, ta thấy dãy số trừ đáp án C có khoảng cách các phần tử không bằng nhau.
}
\end{ex}
%Cau3
\begin{ex}%[DCHT Toán 11 - KNTT -Lê Hải Phụng] %[1K2Y6-1]
Cho cấp số cộng $ (u_n) $ với $ u_1=2 $ và $ u_2=6 $. Công sai của cấp số cộng đã cho là	
\choice
{\True $ 4 $}
{$ -4 $}
{$ 8 $}
{$ 3 $}
\loigiai
{
	Ta có $ u_2=6 \Leftrightarrow 6=u_1+d \Leftrightarrow d=4 $.
}
\end{ex}
%Cau4
\begin{ex}%[DCHT Toán 11 - KNTT -Lê Hải Phụng] %[1K2Y6-1]
Cho cấp số cộng $ (u_n) $ với $ u_1=-3 $ và $ u_6=27 $. Công sai $ d $ của cấp số cộng đã cho là	
\choice
{$ d=7 $}
{$ d=5 $}
{$ d=8 $}
{\True $ d=6 $}
\loigiai
{
	Ta có $ u_6=27 \Leftrightarrow 27=u_1+5d \Leftrightarrow d=6 $.
}
\end{ex}
%Cau5
\begin{ex}%[DCHT Toán 11 - KNTT -Lê Hải Phụng] %[1K2B6-1]
Cho cấp số cộng $ (u_n) $ với $ u_{17}=33 $ và $ u_{33}=65 $. Công sai của cấp số cộng đã cho là	
\choice
{$ 1 $}
{$ 3 $}
{$ -2 $}
{\True $ 2 $}
\loigiai
{
	Gọi $ u_1 $, $ d $ lần lượt là số hạng đầu và công sai của cấp số cộng $ (u_n) $.\\
	Khi đó, ta có $ u_{17}=u_1+16d $, $ u_{33}=u_1+32d $\\
	Suy ra $ u_{33}-u_{17}=65-33 \Leftrightarrow 16d=32 \Leftrightarrow d=2 $\\
	Vậy công sai bằng $ 2 $.
}
\end{ex}
%Cau6
\begin{ex}%[DCHT Toán 11 - KNTT -Tên GV] %[1K2B6-1]
Cho cấp số cộng có $ u_1=-3 $ và $ d=4 $. Chọn khẳng định đúng trong các khẳng định sau.
\choice
{$ u_5=15 $}
{$ u_4=8 $}
{\True $ u_3=5 $}
{$ u_2=2 $}
\loigiai
{
	Ta có $ u_3=u_1+2d=-3+2\cdot4=5 $.
}
\end{ex}
%Cau7
\begin{ex}%[DCHT Toán 11 - KNTT -Tên GV] %[1K2Y6-1]
Cho cấp số cộng có $ u_1=11 $ và công sai $ d=4 $. Hãy tính $ u_{99} $.
\choice
{$ 401 $}
{\True $ 403 $}
{$ 402 $}
{$ 404 $}
\loigiai
{
	Ta có $ u_{99}=u_1+98d=11+98\cdot4=403 $.
}
\end{ex}
%Cau8
\begin{ex}%[DCHT Toán 11 - KNTT -Tên GV] %[1K2B6-1]
Một cấp số cộng $ (u_n) $ có $ u_{13}=8 $ và $ d=-3 $. Tìm số hạng thứ ba của cấp số cộng $ (u_n) $.
\choice
{$ 50 $}
{$ 28 $}
{\True $ 38 $}
{$ 44 $}
\loigiai
{
	Ta có $ u_{13}=u_1+12d \Leftrightarrow 8=u_1+12\cdot(-3)\Rightarrow u_1=44 \Rightarrow u_{3}=u_1+2d=44-6=38$.
}
\end{ex}
%Cau9
\begin{ex}%[DCHT Toán 11 - KNTT -Tên GV] %[1K2Y6-1]
Cho cấp số cộng $(u_n) $ có số hạng đầu $ u_1=2 $ và công sai $ d=4 $. Hãy tính giá trị $ u_{2019} $ bằng
\choice
{\True $ 8074 $}
{$ 4074 $}
{$ 8078 $}
{$ 4078 $}
\loigiai
{
	Ta có $ u_{2019}=u_1+2018d=2+2018\cdot 4=8074 $.
}
\end{ex}
%Cau10
\begin{ex}%[DCHT Toán 11 - KNTT -Tên GV] %[1K2K6-1]
Cho cấp số cộng $ (u_n) $ có số hạng tổng quát là $ u_n=3n-2 $. Tìm công sai $ d $ của cấp số cộng.
\choice
{\True $ d=3 $}
{$ d=2 $}
{$ d=-2 $}
{$ d=-3 $}
\loigiai
{
	Ta có $ u_{n+1}-u_n=3(n+1)-2-3n+2=3 $. Suy ra công sai $ d=3 $.
}
\end{ex}

\begin{ex}%[Dự án DCHT-11-KNTT]%[Dao-V- Thuy]%[1K2Y5-1]
	Cho cấp số cộng $(u_n)$ có số hạng đầu $u_1$ và công sai $d$. Công thức tìm số hạng tổng quát $u_n$ là 
	\choice
	{\True $u_n=u_1+(n-1)d$}
	{$u_n=u_1+nd$}
	{$u_n=u_1+(n+1)d$}
	{$u_n=nu_1+d$}
	\loigiai{
		Ta có $u_n=u_1+(n-1)d$.
	}
\end{ex}

\begin{ex}%[Dự án DCHT-11-KNTT]%[Dao-V- Thuy]%[1K2Y5-1]
	Cho cấp số cộng $(u_n)$ có $u_1=-3$ và $d=\dfrac{1}{2}$. Khẳng định nào sau đây đúng?
	\choice
	{$u_n=-3+\dfrac{1}{2}(n+1 )$}
	{$u_n=-3+\dfrac{1}{2}n-1$}
	{\True $u_n=-3+\dfrac{1}{2}(n-1)$}
	{$u_n=-3+\dfrac{1}{4}(n-1 )$}
	\loigiai {
		Ta có $\heva{
			&u_1=-3 \\
			& d=\dfrac{1}{2} \\
		}\xrightarrow{CTTQ} u_n=u_1+(n-1 )d=-3+\dfrac{1}{2}(n-1 )$.}
\end{ex}

\begin{ex}%[Dự án DCHT-11-KNTT]%[Dao-V- Thuy]%[1K2Y5-1]
	Cho cấp số cộng $\left(u_n\right)$ xác định bởi $u_n=2n+1$. Xác định số hạng đầu $u_1$ và công sai $d$ của cấp số cộng.
	\choice
	{$u_1=3$, $d=1$}
	{$u_1=1$, $d=1$}
	{\True $u_1=3$, $d=2$}
	{$u_1=1$, $d=2$}
	\loigiai{
		Ta có $u_1=2\cdot 1+1=3$ và $u_2=2\cdot 2+1=5$, nên $d=u_2-u_1=2$.
	}
\end{ex}

\begin{ex}%[Dự án DCHT-11-KNTT]%[Dao-V- Thuy]%[1K2B5-1]
	Cho cấp số cộng $\left(u_n\right)$ có $u_4=-12$, $u_{14}=18$. Tìm số hạng đầu $u_1$ và công sai $d$ của cấp số cộng $\left(u_n\right)$. 
	\choice 
	{$u_1=-20$, $d=-3$}
	{$u_1=-22$, $d=3$ }
	{\True $u_1=-21$, $d=3$}
	{$u_1=-21$, $d=-3$}
	\loigiai{
		Ta có $$\heva{&u_4=u_1+(4-1)d\\&u_{14}=u_1+(14-1)d} \Leftrightarrow \heva{&-12=u_1+3d\\&18=u_1+13d}\Leftrightarrow \heva{&u_1=-12\\&d=3.}$$
	}
\end{ex}

\begin{ex}%[Dự án DCHT-11-KNTT]%[Dao-V- Thuy]%[1K2B5-1]
	Tìm số hạng đầu và công sai của cấp số cộng $(u_n)$ thỏa mãn $\heva{&u_1+u_9=12\\&u_4-3u_2=1.}$
	\choice
	{$u_1=\dfrac{1}{2}$; $d=\dfrac{13}{8}$}
	{$u_1=-1$; $d=\dfrac{13}{8}$}
	{\True $u_1=-\dfrac{1}{2}$; $d=\dfrac{13}{8}$}
	{$u_1=-1$; $d=2$}
	\loigiai{Ta có: $\heva{&u_1+u_9=12\\&u_4-3u_2=1}\Leftrightarrow\heva{&u_1+(u_1+8d)=12\\&(u_1+3d)-3(u_1+d)=1}\Leftrightarrow\heva{&2u_1+8d=12\\&-2u_1=1}\Leftrightarrow\heva{&d=\dfrac{13}{8}\\&u_1=-\dfrac{1}{2}}$}
\end{ex}

\begin{ex}%[Dự án DCHT-11-KNTT]%[Dao-V- Thuy]%[1K2B5-1]
	Cho cấp số cộng $(u_n)$ có $u_4=-12$ và $u_{14} =18$. Khi đó, số hạng đầu tiên $u_1$ và công sai $d$ của cấp số cộng $(u_n)$ lần lượt là
	\choice
	{$u_1=-20$, $d=-3$}
	{$u_1=-22$, $d=3$}
	{\True $u_1=-21$, $d=3$}
	{$u_1=-21$, $d=-3$}
	\loigiai{Ta có: $\heva{&u_4=-12\\&u_{14}=18}\Leftrightarrow\heva{&u_1+3d=-12\\&u_1+13d=18}\Leftrightarrow\heva{&u_1=-21\\&d=3.}$}
\end{ex}

\begin{ex}%[Dự án DCHT-11-KNTT]%[Dao-V- Thuy]%[1K2B5-1]
	Cho cấp số cộng $(u_n )$ có các số hạng đầu lần lượt là $5;\,9;\,13;\,17;\ldots $. Tìm số hạng tổng quát $u_n$ của cấp số cộng.
	\choice
	{$u_n=5n+1$}
	{$u_n=5n-1$}
	{\True $u_n=4n+1$}
	{$u_n=4n-1$}
	\loigiai{
		Cấp số cộng đã cho có $u_1=5$, $ d=u_2-u_1=4 $. Suy ra $u_n=u_1+(n-1 )d=5+4(n-1 )=4n+1$.
		}
\end{ex}

\begin{ex}%[Dự án DCHT-11-KNTT]%[Dao-V- Thuy]%[1K2B5-1]
	Cho cấp số cộng $(u_n)$ có $u_3=15$ và $d=-2$. Tìm $u_n$.
	\choice
	{\True $u_n=-2n+21$}
	{$u_n=-\dfrac{3}{2}n+12$}
	{$u_n=-3n-17$}
	{$u_n=\dfrac{3}{2}{{n}^2}-4$}
	\loigiai {
		Ta có $\heva{ & 15=u_3=u_1+2d \\& d=-2}
		\Leftrightarrow \heva{&u_1=19 \\& d=-2}
		\Rightarrow u_n=u_1+(n-1 )d=-2n+21$.
		}
\end{ex}

\begin{ex}%[Dự án DCHT-11-KNTT]%[Dao-V- Thuy]%[1K2B5-1]
	Trong các dãy số được cho dưới đây, dãy số nào {\bf không} phải là cấp số cộng?
	\choice
	{$u_n=-4n+9$}
	{$u_n=-2n+19$}
	{$u_n=-2n-21$}
	{\True $u_n=-2^n+15$}
	\loigiai {
		Dãy số $u_n=-2^n+15$ không có dạng $an+b$ nên có không phải là cấp số cộng.}
\end{ex}

\begin{ex}%[Dự án DCHT-11-KNTT]%[Dao-V- Thuy]%[1K2B5-1]
	Cho cấp số cộng $(u_n)$ có $u_4=-12$ và $u_{14}=18$. Tìm số hạng đầu tiên $u_1$ và công sai $d$ của cấp số cộng đã cho.
	\choice
	{\True $u_1=-21$; $d=3$}
	{$u_1=-20$; $d=-3$}
	{$u_1=-22$; $d=3$}
	{$u_1=-21$; $d=-3$}
	\loigiai {
		Ta có 
		$\heva{&u_4=-12\\ &u_{14}=18} \Leftrightarrow \heva{
			&u_1+3d=-12\\
			&u_1+13d=18 \\
		}\Leftrightarrow \heva{
			&u_1=-21 \\
			& d=3. \\
		}$}
\end{ex}

\begin{ex}%[Dự án DCHT-11-KNTT]%[Dao-V- Thuy]%[1K2K5-1]
	Cho cấp số cộng $(u_n)$ thoả mãn $\heva{&u_2-u_3+u_5=10\\ &u_3+u_4=17}$. Số hạng đầu tiên và công sai của cấp số cộng đó lần lượt là
	\choice
	{\True $1$ và $3$}
	{$-3$ và $4$}
	{$4$ và $-3$}
	{$-4$ và $-3$}
	\loigiai{
		$\heva{&u_2-u_3+u_5=10\\ &u_3+u_4=17}\Leftrightarrow\heva{&(u_1+d)-(u_1+2d)+(u_1+4d)=10\\&(u_1+2d)+(u_1+3d)=17}\Leftrightarrow\heva{&u_1+3d=10\\&2u_1+5d=17}\Leftrightarrow\heva{&u_1=1\\&d=3.}$}
\end{ex}

\begin{ex}%[Dự án DCHT-11-KNTT]%[Dao-V- Thuy]%[1K2K5-1]
	Cho cấp số cộng $(u_n)$ có công sai $d<0$, $u_{31}+u_{34}=11$ và $(u_{31})^2 + (u_{34})^2=101$. Số hạng tổng quát của $(u_n)$ là
	\choice
	{$u_{n}=86-3n$}
	{$u_{n}=92-3n$}
	{$u_{n}=95-3n$}
	{\True $u_{n}=103-3n$}
	\loigiai{Gọi cấp số cộng $(u_n)$ có công sai $d$.\\
		$(u_{31})^2 + (u_{34})^2=101 \Leftrightarrow \left( {u_{31}+u_{34}}\right)^2-2u_{31}.u_{34}=101$ $\Rightarrow u_{31}.u_{34}=10$.\\
		Do đó, ta có $\heva{&u_{31}+u_{34}=11\\ &u_{31}.u_{34}=10}$ $\Rightarrow \heva{&u_{31}=10 \\ &u_{34}=1}$(vì $d<0$)\\
		$u_{31}+u_{34}=11 \Rightarrow 2u_{31}+3d =11 \Rightarrow d=-3 \,\,\text{và}\,\, u_{1}=100$.\\
		Do đó: $u_{n}=103-3n$.}
\end{ex}
\Closesolutionfile{ans}
% \begin{indapan}{10}
% 	{ans/ans-1K2-2-Dang2}
% \end{indapan}
\begin{dang}{Tổng của $n$ số hạng đầu tiên của một cấp số cộng. Tính chất của cấp số cộng}
	Tổng của $n$ số hạng đầu tiên:	Đặt ${{S}_{n}}={{u}_{1}}+{{u}_{2}}+{{u}_{3}}+\cdots+{{u}_{n}}.$ Khi đó
	\begin{itemize}
		\item [$\bullet$] ${{S}_{n}}=\dfrac{n\left( {{u}_{1}}+{{u}_{n}} \right)}{2}=\dfrac{n\left( {{u}_{2}}+{{u}_{n-1}} \right)}{2}=\dfrac{n\left( {{u}_{3}}+{{u}_{n-2}} \right)}{2}=\cdots$
		\item [$\bullet$] Vì ${{u}_{n}}={{u}_{1}}+\left( n-1 \right)d$ nên công thức trên có thể viết lại là \fbox{${{S}_{n}}=\dfrac{n}{2}\left[2u_1 + \left(n-1\right)d \right]  .$}
	\end{itemize}
	Tính chất của cấp số cộng:
	\begin{itemize}
		\item [\ding{172}] Nếu $a$; $b$; $c$ theo thứ tự lập thành cấp số cộng thì $a+c=2b$.
		\item [\ding{173}] Lưu ý:
		\begin{itemize}
			\item [$\bullet$] Nếu cho ba số liên tiếp của một cấp số cộng, ta có thể xem ba số đó là $$a-d;\quad a; \quad a+d$$
			\item [$\bullet$] Nếu cho bốn số liên tiếp của một cấp số cộng, ta có thể xem ba số đó là $$a-3d;\quad a-d; \quad a+d; \quad a+3d.$$
		\end{itemize}
	\end{itemize}
\end{dang}
\viduminhhoa
\begin{vd}
	Cho một cấp số cộng $(u_n)$ có $u_3 + u_{28} = 100$. Hãy tính tổng của $30$ số hạng đầu tiên của cấp số cộng đó.\dapso{$1500$}
	\loigiai{Ta có $S_{30} = \dfrac{30(u_1 + u_{30})}{2} = \dfrac{30(u_1 + 2d + u_{30} - 2d)}{2} = \dfrac{30(u_3 + u_{28})}{2} = \dfrac{30 \cdot 100}{2} = 1500$.}
\end{vd}\dongcham{7}

\begin{vd}
	Cho một cấp số cộng $(u_n)$ có $S_6 = 18$ và $S_{10} = 110$. Tính $S_{20}$.	\dapso{$ 620 $.}
	\loigiai{
		Giả sử cấp số cộng $(u_n)$ có số hạng đầu là $u_1$ và công sai là $d$.\\
		Ta có $S_6 = 6u_1 + \dfrac{6 \cdot 5}{2}d \Leftrightarrow 6u_1 + 15d = 18$. \quad (1)\\
		$S_{10} = 10u_1 + \dfrac{10 \cdot 9}{2}d \Leftrightarrow 10u_1 + 45d = 110$. \quad (2)\\
		Từ (1) và (2), ta có hệ phương trình $\heva{&6u_1 + 15d = 18\\&10u_1 + 45d = 110} \Leftrightarrow \heva{&u_1 = -7\\&d = 4.}$\\
		Khi đó $S_{20} = 20u_1 + \dfrac{20 \cdot 19}{2}d = 20 \cdot (-7) + 190 \cdot 4 = 620$.
	}
\end{vd}\dongcham{8}


\begin{vd}
	Tìm số hạng đầu và công sai của cấp số cộng, biết
	\begin{tasks}(2)
		\task $\heva{&u_1^2 + u_2^2 + u_3^2 = 155\\&S_3 = 21.}$	\dapso{$u_1 = 9$, $d = -2$ hoặc $u_1 = 5$, $d = 2$.}
		\task $\heva{&S_3 = 12\\&S_5 = 35.}$	\dapso{$u_1 = 1$, $d = 3$.}
	\end{tasks}
	\loigiai{
		\begin{listEX}
			\item $\heva{&u_1^2 + u_2^2 + u_3^2 = 155\\&S_3 = 21} \Leftrightarrow \heva{&u_1^2 + (u_1 + d)^2 + (u_1 + 2d)^2 = 155 &(1)\\&3u_1 + 3d = 21.&(2)}$\\
			Từ (2), ta có $3u_1 + 3d = 21 \Rightarrow d = 7 - u_1$, thay vào (1)
			$$u_1^2 + 7^2 + (14 - u_1)^2 = 155 \Leftrightarrow 2u_1^2 - 28u_1 + 90 = 0 \Leftrightarrow \hoac{&u_1 = 9\\&u_1 = 5.}$$
			Với $u_1 = 9$ thì $d = -2$. Với $u_1 = 5$ thì $d = 2$.\\
			Vậy số hạng đầu của cấp số cộng là $u_1 = 9$, công sai là $d = -2$ hoặc $u_1 = 5$, $d = 2$.
			\item $\heva{&S_3 = 12\\&S_5 = 35} \Leftrightarrow \heva{&3u_1 + 3d = 12\\&5u_1 + 10d = 35} \Leftrightarrow \heva{&u_1 = 1\\&d = 3.}$\\
			Vậy số hạng đầu của cấp số cộng là $u_1 = 1$, công sai là $d = 3$.
	\end{listEX}}
\end{vd}\dongcham{12}

\begin{vd}
	Tìm số hạng tổng quát của cấp số cộng, biết 
	$\heva{&S_4 = 20\\&\dfrac{1}{u_1} + \dfrac{1}{u_2} + \dfrac{1}{u_3} + \dfrac{1}{u_4} = \dfrac{25}{24}}$ và cấp số cộng có công sai là một số nguyên âm.	\dapso{$ u_n=10-2n $.}
	\loigiai{
		$\heva{&S_4 = 20 &(1)\\&\dfrac{1}{u_1} + \dfrac{1}{u_2} + \dfrac{1}{u_3} + \dfrac{1}{u_4} = \dfrac{25}{24}&(2).}$\\
		Từ (1), suy ra $u_1 + u_4 = u_2 + u_3 = 10$ và $u_1 = 5 - \dfrac{3}{2}d$.\\
		Từ (2), ta có 
		\begin{eqnarray*}
			& &\dfrac{u_1 + u_4}{u_1 \cdot u_4} + \dfrac{u_2 + u_3}{u_2 \cdot u_3} = \dfrac{25}{24} \Leftrightarrow \dfrac{10}{u_1(u_1 + 3d)} + \dfrac{10}{(u_1 + d)(u_1 + 2d)} = \dfrac{25}{24}\\
			&\Leftrightarrow & \dfrac{10}{\left(5 - \dfrac{3}{2}d\right)\left(5 + \dfrac{3}{2}d\right)} + \dfrac{10}{\left(5 - \dfrac{1}{2}d\right)\left(5 + \dfrac{1}{2}d\right)} = \dfrac{25}{24} \Leftrightarrow \dfrac{10}{25 - \dfrac{9}{4}d^2} + \dfrac{10}{25 - \dfrac{1}{4}d^2} = \dfrac{25}{24}\\
			&\Leftrightarrow & 10\left(25 - \dfrac{9}{4}d^2 + 25 - \dfrac{1}{4}d^2\right) = \dfrac{25}{24}\left(25 - \dfrac{9}{4}d^2\right)\left(25 - \dfrac{1}{4}d^2\right)\\
			&\Leftrightarrow & \dfrac{75}{128}d^4 - \dfrac{1925}{48}d^2 + \dfrac{3625}{24} = 0 \Leftrightarrow \hoac{&d^2 = \dfrac{580}{9}\\&d^2 = 4} \Leftrightarrow \hoac{&d = \pm \dfrac{2\sqrt{145}}{3}\\&d = \pm 2.}
		\end{eqnarray*}
		Với $d = -2$ thì $u_1 = 8$. Suy ra $u_n=u_1+(n-1)d=10-2n$}
\end{vd}\dongcham{18}

\begin{vd}
	Tính các tổng sau
	\begin{tasks}(2)
		\task $S = 1 + 3 + 5 + \cdots + (2n - 1) + (2n + 1)$.\dapso{$S = (n + 1)^2$}
		\task $S = 100^2 - 99^2 + 98^2 - 97^2 + \cdots + 2^2 - 1^2$.\dapso{$S = 5050$}
	\end{tasks}
	\loigiai{
		\begin{enumEX}{1}
			\item $S = 1 + 3 + 5 + \cdots + (2n - 1) + (2n + 1)$.\\
			Xét cấp số cộng $(u_k)$, $k \in \mathbb{N}^*$ với số hạng đầu là $u_1 = 1$ và công sai là $d = 2$.\\
			Ta có $u_k = u_1 + (k - 1)d \Leftrightarrow 2n + 1 = 1 + 2(k - 1) \Leftrightarrow k = n + 1$.\\
			Vậy $S = \dfrac{k(u_1 + u_k)}{2} = \dfrac{(n + 1)(1 + 2n + 1)}{2} = (n + 1)^2$.
			\item $S = 100^2 - 99^2 + 98^2 - 97^2 + \cdots + 2^2 - 1^2 = 199 + 195 + \cdots + 3$.\\
			Xét cấp số cộng $(u_n)$ có số hạng đầu $u_1 = 199$ và công sai $d = u_2 - u_1 = 195 - 199 = -4$.\\
			Ta có $u_n = u_1 + (n - 1)d \Leftrightarrow 3 = 199 - 4(n - 1) \Leftrightarrow n = 50$.\\
			Khi đó $S = \dfrac{n(u_1 + u_{50})}{2} = \dfrac{50(199 + 3)}{2} = 5050$.
			
		\end{enumEX}
	}
\end{vd}\dongcham{18}

\begin{vd}
	Tìm ba số hạng liên tiếp của một cấp số cộng biết tổng của chúng bằng $27$ và tổng các bình phương của chúng là $293$.\dapso{$4$, $9$, $14$}
	\loigiai{
		Gọi ba số hạng liên tiếp của cấp số cộng là $x - d$, $x$, $x + d$ trong đó $d$ là công sai của cấp số cộng.\\
		Khi đó ta có $x - d + x + x + d = 27 \Leftrightarrow 3x = 27 \Leftrightarrow x = 9$.\\
		Mà $(x - d)^2 + x^2 + (x + d)^2 = 293 \Leftrightarrow (9 - d)^2 + 81 + (9 + d)^2 = 293 \Leftrightarrow 2d^2 -50 = 0 \Leftrightarrow \hoac{&d = 5\\&d = -5.}$\\	
		Với $d = 5$ thì ba số hạng của cấp số cộng là $4$, $9$, $14$.\\
		Với $d = -5$ thì ba số hạng của cấp số cộng là $14$, $9$, $4$.\\
		Vậy ba số hạng liên tiếp của cấp số cộng là $4$, $9$, $14$.
	}
\end{vd}\dongcham{14}

\begin{vd}
	Tìm bốn số hạng liên tiếp của một cấp số cộng, biết tổng của chúng bằng $10$ và tổng bình phương của chúng bằng $30$.\dapso{$1$, $2$, $3$, $4$}
	\loigiai{
		Gọi bốn số hạng liên tiếp của cấp số cộng là $x - 3d$, $x - d$, $x 
		+ d$, $x + 3d$ với $2d$ là công sai của cấp số cộng.\\
		Khi đó ta có $x - 3d + x - d + x + d + x + 3d = 10 \Leftrightarrow 4x = 10 \Leftrightarrow x = \dfrac{5}{2}$.\\
		Mặt khác $$(x - 3d)^2 + (x - d)^2 + (x + d)^2 + (x + 3d)^2 = 30 \Leftrightarrow 4x^2 + 20d^2 = 30 \Leftrightarrow d^2 = \dfrac{1}{4} \Leftrightarrow \hoac{&d = \dfrac{1}{2}\\&d = -\dfrac{1}{2}.}$$
		Với $x = \dfrac{5}{2}$ thì $d = \dfrac{1}{2}$, khi đó bốn số hạng liên tiếp của cấp số cộng là $1$, $2$, $3$, $4$.\\
		Với $x = \dfrac{5}{2}$ thì $d = -\dfrac{1}{2}$, khi đó bốn số hạng liên tiếp của cấp số cộng là $4$, $3$, $2$, $1$.\\
		Vậy bốn số hạng liên tiếp của cấp số cộng là $1$, $2$, $3$, $4$.
	}
\end{vd}\dongcham{14}

\begin{vd}
	Ba góc của một tam giác vuông lập thành một cấp số cộng. Tìm ba góc đó.
	\loigiai{Gọi ba góc của tam giác lần lượt là $A$, $B$, $C$.
		Khi đó ta có $A + B + C = 180^\circ$.\\
		Do ba góc $A$, $B$, $C$ của tam giác theo thứ tự lập thành một cấp số cộng nên $B-A=C-A \Leftrightarrow A + C = 2B$.\\
		Do đó $2B + B = 180^\circ \Rightarrow 3B = 180^\circ \Rightarrow B = 60^\circ$.\\
		Do tam giác $ABC$ vuông nên giả sử $C = 90^\circ$ khi đó công sai $d$ của cấp số cộng là $d = C - B = 30^\circ$.\\
		Vậy góc $A$ của tam giác là $A = 30^\circ$.}
\end{vd}\dongcham{10}

% \begin{vd}
% 	Cho $a$, $b$, $c$ là ba số hạng liên tiếp của một cấp số cộng. Chứng minh rằng
% 	\begin{tasks}(1)
% 		\task $a^2 + 2bc = c^2 + 2ab$.
% 		\task $2(a+b+c)^3 = 9 \left[ a^2(b+c) + b^2(a+c) + c^2(a+b) \right]$.
% 		\task  $b^2 + bc +c^2$, $a^2 + ac + c^2$, $a^2 + ab + b^2$ cũng là một cấp số cộng.
% 	\end{tasks}
% 	\loigiai{
% 		\begin{enumerate}[a)]
% 			\item Vì $a$, $b$, $c$ là ba số liên tiếp của một cấp số cộng nên $a + c = 2b \Rightarrow a = 2b -c$.\\
% 			Do đó
% 			$$a^2 +2bc = (2b-c)^2 + 2bc = 4b^2 - 2bc + c^2 = 2b(2b -c) + c^2 = 2ba + c^2 = c^2 + 2ab.$$
% 			Vậy $a^2 + 2bc = c^2 + 2ab$ (đpcm).
			
% 			\item Vì $a$, $b$, $c$ là ba số liên tiếp của một cấp số cộng nên $a + c = 2b \Rightarrow a = 2b -c$.\\
% 			Do đó
% 			\allowdisplaybreaks
% 			\begin{eqnarray*}
% 				& \mbox{VT}  & = 2(a+b+c)^3 = 2(3b)^3 = 54b^3\\
% 				& \mbox{VP}  & = 9\left[ a^2(b+c) + b^2(a+c) + c^2(a+b) \right] \\
% 				& & = 9\left[ (2b-c)^2(b+c) + b^2(2b-c+c) + c^2(2b-c+b) \right] \\ 
% 				& & = 9\left[ (4b^2 - 4bc+c^2)(b+c) + b^2(2b) + c^2(3b-c) \right] \\
% 				& & = 9\left[ 4b^3 - 4b^2c +bc^2 + 4b^2c - 4bc^2 + c^3 + 2b^3 + 3bc^2 - c^3 \right] \\
% 				& & = 9\cdot (6b^3) = 54b^3 = \mbox{ VT }.
% 			\end{eqnarray*}
% 			Vậy $2(a+b+c)^3 = 9 \left[ a^2(b+c) + b^2(a+c) + c^2(a+b) \right]$ (đpcm).
			
% 			\item Vì ba số $a$, $b$, $c$ theo thứ tự lập thành một cấp số cộng thì $a + c = 2b \Rightarrow a = 2b - c$.\\
% 			Xét
% 			\allowdisplaybreaks
% 			\begin{eqnarray*}
% 				& 2(a^2 + ac + c^2) - (a^2 + ab + b^2) & = a^2 + a(2c-b) + 2c^2 - b^2 \\
% 				& & = (2b-c)^2 + (2b-c)(2c-b) + 2c^2 - b^2 \\
% 				& & = b^2 + bc  +c^2\\
% 				&\Rightarrow (b^2 + bc  +c^2) + (a^2 + ab + b^2) &= 2(a^2 + ac + c^2).
% 			\end{eqnarray*}
% 			Vậy ba số: $b^2 + bc +c^2$, $a^2 + ac + c^2$, $a^2 + ab + b^2$ cũng là một cấp số cộng.
% 		\end{enumerate}
% 	}
% \end{vd}\dongcham{25}

\begin{vd}%[TH]%[Dự án DCHT-11-KNTT]%[Dao-V- Thuy]%[1K2B5-2]
	Xác định $4$ góc của một tứ giác lồi, biết rằng $4$ góc hợp thành cấp số cộng và góc lớn nhất bằng $5$ lần góc nhỏ nhất.
	\dapso{$36^\circ; \, 72^\circ; \, 108^\circ; \, 144^\circ$}
	\loigiai
	{
		Gọi số đo bốn góc cần tìm là $u_1$, $u_2$, $u_3$, $u_4$. Ta có
		\begin{eqnarray*}
			\heva{&u_1+u_2+u_3+u_4=360\\ &u_5=5u_1} \Leftrightarrow \heva{&4u_1+6d=360\\ &4d=4u_1} \Leftrightarrow \heva{&u_1=36\\ &d=36.}
		\end{eqnarray*}
		Vậy số đo bốn góc cần tìm là
		\[
		36^\circ; \, 72^\circ; \, 108^\circ; \, 144^\circ.
		\]
	}
\end{vd}

\subsubsection{Bài tập tự luận}
 

\begin{bt}%[TH]%[Dự án DCHT-11-KNTT]%[Dao-V- Thuy]%[1K2B5-2]
	Giữa các số $10$ và $64$ hãy đặt thêm $17$ số nữa để được một cấp số cộng.
	\dapso{$13; 16; 19; 22; 25; 28; 31; 34; 37; 40; 43; 46; 49; 52; 55; 58; 61$}
	\loigiai{
		Ta có
		\begin{equation*}
			\heva{&u_1=10\\ &u_{19}=64} \Leftrightarrow \heva{&u_1=10\\ &u_1+18d=64} \Leftrightarrow \heva{&u_1=10\\ &d=3.}
		\end{equation*}
		Vậy $17$ số đặt thêm giữa các số $10$ và $64$ để được một cấp số cộng là
		\begin{center}
			13; 16; 19; 22; 25; 28; 31; 34; 37; 40; 43; 46; 49; 52; 55; 58; 61.
		\end{center} 
	}
\end{bt}

\begin{bt}%[TH]%[Dự án DCHT-11-KNTT]%[Dao-V- Thuy]%[1K2B5-2]
	Tổng ba số hạng liên tiếp của một cấp số cộng bằng $2$ và tổng các bình phương của ba số đó bằng $\dfrac{14}{9}$. Xác định ba số đó và tính công sai của cấp số cộng.
	\dapso{$1;\dfrac{2}{3};\dfrac{1}{3}$ ứng với $d=-\dfrac{1}{3}$ hoặc $\dfrac{1}{3};\dfrac{2}{3};1$ ứng với $d=\dfrac{1}{3}$}
	\loigiai{
		Ta có hệ
		\begin{align*}
			&\quad \heva{&u_k+u_{k+1}+u_{k+2}=2\\ &u^2_k+u^2_{k+1}+u^2_{k+2}=\dfrac{14}{9}}
			\Leftrightarrow \heva{&u_k+u_k+d+u_k+2d=2\\ &u^2_k+\left(u_k+d\right)^2 +\left(u_k+2d\right)^2=\dfrac{14}{9}} \\
			&\Leftrightarrow \heva{&3u_k+3d=2\\ &3u^2_k+6u_kd+5d^2=\dfrac{14}{9}}
			\Leftrightarrow \heva{&u_k=1\\ &d=-\dfrac{1}{3}} \text{ hoặc } \heva{&u_k=\dfrac{1}{3}\\ &d=\dfrac{1}{3}.}
		\end{align*}
		Vậy ba số hạng liên tiếp của cấp số cộng thỏa yêu cầu bài toán $1;\dfrac{2}{3};\dfrac{1}{3}$ ứng với $d=-\dfrac{1}{3}$ hoặc $\dfrac{1}{3};\dfrac{2}{3};1$ ứng với $d=\dfrac{1}{3}.$
	}
\end{bt}

\begin{bt}%[TH]%[Dự án DCHT-11-KNTT]%[Dao-V- Thuy]%[1K2B5-2]
	Một cấp số cộng có $7$ số hạng với công sai $d$ dương và số hạng thứ tư bằng $11$. Hãy tìm các số hạng còn lại của cấp số cộng đó, biết hiệu của số hạng thứ ba và số hạng thứ năm bằng $6$.
	\dapso{$u_1=2$; $ u_2=5$; $u_4=11$; $u_6=17$; $u_7=20$}
	\loigiai{
		Gọi số hạng đầu của cấp số cộng là $u_1$, công sai $d$.
		Vì số hạng thứ tư của cấp số cộng bằng $11$ nên ta có $u_4=11$.\\
		Do $d$ dương nên $ u_5>u_3$.\\
		Vì hiệu của số hạng thứ ba và số hạng thứ năm bằng $6$ nên ta có $ u_5-u_3=6$.\\
		Ta có \begin{align*}
			\heva{&u_4=11\\&u_5-u_3=6 }
			\Leftrightarrow \heva{&u_1+3d=11\\&(u_1+4d)-(u_1+2d)=6 }
			\Leftrightarrow \heva{&u_1+3\cdot 3=11\\&d=3 }
			\Leftrightarrow \heva{&u_1=2\\&d=3.}
		\end{align*}
		Vậy các số  hạng còn lại của cấp số cộng là $u_1=2$; $ u_2=5$; $u_4=11$; $u_6=17$; $u_7=20$.
	}
\end{bt}

\begin{bt}%[VD]%[Dự án DCHT-11-KNTT]%[Dao-V- Thuy]%[1K2K5-2]
	Tìm bốn số hạng liên tiếp của một cấp số cộng, biết rằng:
	\begin{enumerate}
		\item Tổng của chúng bằng $10$ và tổng bình phương bằng $70$.
		\item Tổng của chúng bằng $22$ và tổng bình phương bằng $66$.
		\item  Tổng của chúng bằng $36$ và tổng bình phương bằng $504$.
		\item  Chúng có tổng bằng $20$ và tích của chúng bằng $384$.
		\item  Tổng của chúng bằng $ 20$, tổng nghịch đảo của chúng bằng $ \dfrac{25}{24}$ và các số này là những số nguyên.
		\item  Nó là số đo của một tứ giác lồi và góc lớn nhất gấp $5$ lần góc nhỏ nhất.
	\end{enumerate}
	\dapso{$-2$; $ 1$; $ 4$; $7$.} 
	\dapso{không tồn tại bốn số hạng liên tiếp của cấp số cộng thỏa mãn yêu cầu đề bài. $0$; $ 6$; $ 12$; $18$.}
	\dapso{$2$; $ 4$; $ 6$; $8$ hoặc $5-\sqrt{241}$; $ \dfrac{15-\sqrt{241}}{3}$; $ \dfrac{15+\sqrt{241}}{3}$; $5+\sqrt{241}$.} 
	\dapso{$30^\circ$; $70^\circ$; $ 110^\circ$; $150^\circ$.}
	\loigiai{
		\begin{enumerate}
			\item Gọi bốn số hạng liên tiếp của cấp số cộng là $x-3d$; $x-d$; $x+d$, $x+3d$ trong đó $2d$ là công sai.\\
			Theo đề bài ta có 
			\begin{align*}
				&\quad \heva{& (x-3d)+(x-d)+(x+d)+(x+3d)=10\\& (x-3d)^2+(x-d)^2+(x+d)^2+(x+3d)^2=70}
				\Leftrightarrow \heva{& 4x=10 \\& 4x^2+20d^2=70}\\
				&\Leftrightarrow  \heva{&x=\dfrac{5}{2}\\& 4\cdot \left( \dfrac{5}{2}\right) ^2+20d^2=70}
				\Leftrightarrow  \heva{&x=\dfrac{5}{2}\\& d^2=\dfrac{9}{4}}
				\Leftrightarrow  \heva{&x=\dfrac{5}{2}\\& d=\pm \dfrac{3}{2}.}
			\end{align*}
			Vậy bốn số hạng liên tiếp của cấp số cộng là $-2$; $ 1$; $ 4$; $7$.
			\item Gọi bốn số hạng liên tiếp của cấp số cộng là $x-3d$; $x-d$; $x+d$; $x+3d$ trong đó $2d$ là công sai.\\
			Theo đề bài ta có 
			\begin{align*}
				&\quad \heva{& (x-3d)+(x-d)+(x+d)+(x+3d)=22\\& (x-3d)^2+(x-d)^2+(x+d)^2+(x+3d)^2=66}
				\Leftrightarrow \heva{& 4x=22 \\& 4x^2+20d^2=66}\\
				&\Leftrightarrow  \heva{&x=\dfrac{11}{2}\\& 4\cdot \left( \dfrac{11}{2}\right) ^2+20d^2=66}
				\Leftrightarrow  \heva{&x=\dfrac{11}{2}\\& d^2=\dfrac{-11}{4} 
					\ (\text{loại}).}\\
			\end{align*}
			Vậy không tồn tại bốn số hạng liên tiếp của cấp số cộng thỏa mãn yêu cầu đề bài.
			\item Gọi bốn số hạng liên tiếp của cấp số cộng là $x-3d$; $x-d$; $x+d$; $x+3d$ trong đó $2d$ là công sai.\\
			Theo đề bài ta có 
			\begin{align*}
				&\quad \heva{& (x-3d)+(x-d)+(x+d)+(x+3d)=36\\& (x-3d)^2+(x-d)^2+(x+d)^2+(x+3d)^2=504}
				\Leftrightarrow \heva{& 4x=36 \\& 4x^2+20d^2=504}\\
				&\Leftrightarrow  \heva{&x=9\\& 4\cdot 9^2+20d^2=504}
				\Leftrightarrow  \heva{&x=9\\& d^2=9}
				\Leftrightarrow  \heva{&x=9\\& d=\pm 3.}
			\end{align*}
			Vậy  bốn  số hạng liên tiếp của cấp số cộng là $0$; $ 6$; $ 12$; $18$.
			\item Gọi bốn số hạng liên tiếp của cấp số cộng là $x-3d$; $x-d$; $x+d$; $x+3d$ trong đó $2d$ là công sai.\\
			Theo đề bài ta có 
			\begin{align*}
				&\quad \heva{& (x-3d)+(x-d)+(x+d)+(x+3d)=20\\& (x-3d)(x-d)(x+d)(x+3d)=384}
				\Leftrightarrow  \heva{&x=5\\& (x^2-d^2)(x^2-9d^2)=384}\\
				&\Leftrightarrow  \heva{&x=5\\& (25-d^2)(25-9d^2)=384}
				\Leftrightarrow  \heva{&x=5\\& 9d^4-250d^2+241=0}
				\Leftrightarrow  \heva{&x=5\\& \hoac{&d^2=1\\& d^2=\dfrac{241}{9}}}
				\Leftrightarrow  \heva{&x=5\\& \hoac{&d=\pm 1\\& d=\pm \dfrac{\sqrt{241}}{3}.}}
			\end{align*}
			Vậy bốn số hạng liên tiếp của cấp số cộng là $2$; $ 4$; $ 6$; $8$ hoặc $5-\sqrt{241}$; $ \dfrac{15-\sqrt{241}}{3}$; $ \dfrac{15+\sqrt{241}}{3}$; $5+\sqrt{241}$.
			\item Gọi bốn số hạng liên tiếp của cấp số cộng là $x-3d$; $x-d$; $x+d$; $x+3d$ trong đó $2d$ là công sai trong đó $ 2d \in \mathbb{Z}$.\\
			Theo đề bài ta có 
			\begin{align*}
				&\quad \heva{& (x-3d)+(x-d)+(x+d)+(x+3d)=20\\& \dfrac{1}{x-3d}+\dfrac{1}{x-d}+\dfrac{1}{x+d}+\dfrac{1}{x+3d}=\dfrac{25}{24}}
				\Leftrightarrow \heva{& 4x=20 \\& \dfrac{1}{5-3d}+\dfrac{1}{5-d}+\dfrac{1}{5+d}+\dfrac{1}{5+3d}=\dfrac{25}{24}}\\
				&\Leftrightarrow  \heva{&x=5\\& \dfrac{10}{25-9d^2}+\dfrac{10}{25-d^2}=\dfrac{25}{24}}
				\Leftrightarrow  \heva{&x=5\\& 9d^4-250d^2+241=0}\\
				&\Leftrightarrow  \heva{&x=5\\& \hoac{&d^2=1 \\& d^2=\dfrac{241}{9} }}
				\Leftrightarrow  \heva{&x=5\\& \hoac{&d= \pm 1 \ (\text{thỏa mãn})\\& d= \pm \dfrac{\sqrt{241}}{3} \,\,(\text{loại vì} \,  2d \in \mathbb{Z}).}}
			\end{align*}
			Vậy bốn  số hạng  nguyên liên tiếp của cấp số cộng là $2$; $ 4$; $ 6$; $8$.
			\item Gọi bốn số hạng liên tiếp của cấp số cộng xếp theo thứ tự tăng dần  là $x-3d$; $x-d$; $x+d$; $x+3d$ trong đó $2d>0$ là công sai.\\
			Theo đề bài ta có 
			\begin{align*}
				&\quad \heva{& (x-3d)+(x-d)+(x+d)+(x+3d)=360^\circ\\& x+3d=5(x-3d)}\\
				&\Leftrightarrow \heva{& 4x=360^\circ\\& 4x=18d}
				\Leftrightarrow  \heva{&x=90^\circ\\& 4 \cdot 90^\circ=18d}
				\Leftrightarrow  \heva{&x=90^\circ\\& d=20^\circ.}
			\end{align*}
			Vậy bốn  góc của tứ giác lồi lần lượt là  $30^\circ$; $70^\circ$; $ 110^\circ$; $150^\circ$.
		\end{enumerate}
	}
\end{bt}
% \subsubsection{Câu hỏi trắc nghiệm}
% \Opensolutionfile{ans}[ans/ans-1K2-2-Dang3]
% \begin{ex}%[Dự án DCHT-11-KNTT]%[Dao-V- Thuy]%[1K2Y5-2]
% 	Cấp số cộng $(u_n)$ có số hạng đầu $u_1=-5$ và công sai $d=3$. Tính $u_{15}$.
% 	\choice
% 	{$u_{15}=27$}
% 	{\True $u_{15}=37$}
% 	{$u_{15}=47$}
% 	{$u_{15}=57$}
% 	\loigiai{$u_{15}=u_1+14d=-5+14\times 3=37$.}
% \end{ex}

% \begin{ex}%[Dự án DCHT-11-KNTT]%[Dao-V- Thuy]%[1K2Y5-2]
% 	Cho cấp số cộng có các số hạng ban đầu là $1$; $5$; $9$; $13$; $\cdots$. Số hạng thứ $ 6 $ của cấp số cộng này là bao nhiêu?
% 	\choice{\True $ 21$}
% 	{$19 $}
% 	{$ 22$}
% 	{$ 20$}
% 	\loigiai{Ta có $u_1=1$, $d=5-1=4$ nên $u_6=1+5d=1+20=21$.
% 	}
% \end{ex}

% \begin{ex}%[Dự án DCHT-11-KNTT]%[Dao-V- Thuy]%[1K2Y5-2]
% 	Cho cấp số cộng $\left( u_n \right)$ có các số hạng lần lượt là $-4;\,1;\,6;\,x$. Tìm giá trị của $x$.
% 	\choice
% 	{$x=7$}
% 	{$x=10$}
% 	{\True $x=11$}
% 	{$x=12$}
% 	\loigiai{
% 		Dễ thấy $u_1=-4$, $d=5$ nên $u_4=-4+3\cdot 5=11$.
% 	}
% \end{ex}

% \begin{ex}%[Dự án DCHT-11-KNTT]%[Dao-V- Thuy]%[1K2B5-2]
% 	Cho cấp số cộng $(u_n)$ có $u_1=-5$ và $d=3$. Mệnh đề nào sau đây đúng?
% 	\choice
% 	{$u_{15}=34$}
% 	{$u_{15}=45$}
% 	{\True $u_{13}=31$}
% 	{$u_{10}=35$}
% 	\loigiai {
% 		$\heva{
% 			& u_1=-5 \\
% 			& d=3 \\
% 		}\Rightarrow u_n=3n-8\Rightarrow \heva{
% 			& u_{15}=37 \\
% 			& u_{13}=31 \\
% 			& u_{10}=22. \\
% 		}$}
% \end{ex}

% \begin{ex}%[Dự án DCHT-11-KNTT]%[Dao-V- Thuy]%[1K2B5-2]
% 	Cho cấp số cộng có số hạng đầu là $u_1=-\dfrac{1}{2}$, công sai $d=\dfrac{1}{2}$. Trong mỗi bộ gồm năm số hạng dưới đây, bộ năm số nào là các số hạng liên tiếp của dãy này?
% 	\choice
% 	{$-\dfrac{1}{2};\,0;\,1;\,\dfrac{1}{2};\,1$}
% 	{$-\dfrac{1}{2};\,0;\,\dfrac{1}{2};\,0;\,\dfrac{1}{2}$}
% 	{$\dfrac{1}{2};\,1;\,2;\,\dfrac{5}{2};\,\dfrac{7}{2}$}
% 	{\True $1;\,\dfrac{3}{2};\,2;\,\dfrac{5}{2};\,3$}
% 	\loigiai{
% 		Ta có $u_1=-\dfrac{1}{2}$; $u_2=0$; $u_3=\dfrac{1}{2}$, $u_4=1$; $u_5=\dfrac{3}{2}$; $u_6=2$; $u_7=\dfrac{5}{2}$; $u_8=3$.
% 	}
% \end{ex}

% \begin{ex}%[Dự án DCHT-11-KNTT]%[Dao-V- Thuy]%[1K2B5-2]
% 	Cho cấp số cộng $(u_n)$ có $u_7=\dfrac{19}{5}$ và công sai $d=\dfrac{2}{5}$. Tính $u_{10}$.
% 	\choice
% 	{$\dfrac{2}{5}$}
% 	{$\dfrac{19}{5}$}
% 	{\True $5$}
% 	{$\dfrac{27}{5}$}
% 	\loigiai{Ta có: $u_7=u_1+6d\Rightarrow u_1=u_7-6d=\dfrac{19}{5}-6\cdot \dfrac{2}{5}=\dfrac{7}{5}$.\\
% 		Suy ra $u_{10}=u_1+9d=\dfrac{7}{5}+9\cdot \dfrac{2}{5}=5$.}
% \end{ex}

% \begin{ex}%[Dự án DCHT-11-KNTT]%[Dao-V- Thuy]%[1K2B5-2]
% 	Cho cấp số cộng $(u_n)$ có số hạng đầu $u_1=-1$ và công sai $d=-3$. Số hạng thứ $20$ của cấp số cộng này là
% 	\choice
% 	{\True $u_{20}=-58$}
% 	{$u_{20}=60$}
% 	{$u_{20}=-72$}
% 	{$u_{20}=-61$}
% 	\loigiai{Số hạng thứ $20$ là: $u_{20}=u_1+19d=-1+19\cdot (-3)=-58$.}
% \end{ex}

% \begin{ex}%[Dự án DCHT-11-KNTT]%[Dao-V- Thuy]%[1K2B5-2]
% 	Cho cấp số cộng $(u_n)$ có $u_1=-5$ và $d=3$. Số $100$ là số hạng thứ mấy của cấp số cộng?
% 	\choice
% 	{Thứ $15$}
% 	{Thứ $20$}
% 	{Thứ $35$}
% 	{\True Thứ $36$}
% 	\loigiai {
% 		Ta có $\heva{
% 			&u_1=-5 \\
% 			& d=3 \\
% 		}$. Vì $u_n=100 \Rightarrow 100=u_n=u_1+(n-1 )d=3n-8\Leftrightarrow n=36$.
% 	}
% \end{ex}

% \begin{ex}%[Dự án DCHT-11-KNTT]%[Dao-V- Thuy]%[1K2B5-2]
% 	Cho cấp số cộng $(u_n)$ có $u_2=2001$ và $u_5=1995$. Khi đó $u_{1001}$ bằng
% 	\choice
% 	{$u_{1001}=4005$}
% 	{$u_{1001}=4003$}
% 	{\True $u_{1001}=3$}
% 	{$u_{1001}=1$}
% 	\loigiai {
% 		$\heva{
% 			& 2001=u_2=u_1+d \\
% 			& 1995=u_5=u_1+4d \\
% 		}\Leftrightarrow \heva{
% 			&u_1=2003 \\
% 			& d=-2 \\
% 		}\Rightarrow u_{1001}=u_1+1000d=3$.}
% \end{ex}

% \begin{ex}%[Dự án DCHT-11-KNTT]%[Dao-V- Thuy]%[1K2B5-2]
% 	Cho cấp số cộng $(u_n)$ biết $\heva{&u_1+u_3=7\\&u_2+u_4=12}$. Tính $u_{21}$.
% 	\choice
% 	{$u_{21}=1$}
% 	{\True $u_{21}=51$}
% 	{$u_{21}=31$}
% 	{$u_{21}=21$}
% 	\loigiai{
% 		Ta có $\heva{&u_1+u_3=7\\&u_2+u_4=12}\Leftrightarrow\heva{&u_1+u_1+2d=7\\&u_1+d+u_1+3d=12}\Leftrightarrow\heva{&2u_1+2d=7\\&2u_1+4d=12}\Leftrightarrow\heva{&u_1=1\\&d=\dfrac{5}{2}.}$\\
% 		Suy ra $u_{21}=u_1+20d=1+20\cdot\dfrac{5}{2}=1+50=51$.}
% \end{ex}

% \begin{ex}%[Dự án DCHT-11-KNTT]%[Dao-V- Thuy]%[1K2B5-2]
% 	Một cấp số cộng có $7$ số hạng. Biết rằng tổng của số hạng đầu và số hạng cuối bằng $30$, tổng của số hạng thứ ba và số hạng thứ sáu bằng $35$. Tìm số hạng thứ bảy của cấp số cộng đã cho.
% 	\choice
% 	{$u_7=25$}
% 	{\True $u_7=30$}
% 	{$u_7=35$}
% 	{$u_7=40$}
% 	\loigiai{
% 		Theo đề ta có: $\heva{&u_1+u_7=30\\&u_3+u_6=35}\Leftrightarrow \heva{&u_1+(u_1+6d)=30\\&(u_1+2d)+(u_1+5d)=35}\Leftrightarrow \heva{&2u_1+6d=30\\&2u_1+7d=35}\Leftrightarrow \heva{&u_1=0\\&d=5.}$\\
% 		Do đó $u_7=u_1+6d=0+6\cdot 5=30$.}
% \end{ex}

% \begin{ex}%[Dự án DCHT-11-KNTT]%[Dao-V- Thuy]%[1K2G5-2]
% 	Cho dãy số $(u_n)$ có xác định bởi $\heva{&u_1=-2,\\&u_{n+1}=\dfrac{u_n}{1-u_n}} \; (\text{với } n\in\mathbb{N}^*)$ và dãy số $(v_n)$ được xác định bởi $v_n=\dfrac{u_n+1}{u_n}$. Số hạng thứ $2023$ của dãy $(v_n)$ là
% 	\choice
% 	{$ -\dfrac{2023}{3}$}
% 	{$ -\dfrac{4046}{3}$}
% 	{\True$ -\dfrac{4043}{2}$}
% 	{$ -2023$}
% 	\loigiai{
% 		Ta có $v_{n+1}-v_n=\dfrac{u_{n+1}+1}{u_{n+1}}-\dfrac{u_n+1}{u_n}=\dfrac{\dfrac{u_n}{1-u_n}+1}{\dfrac{u_n}{1-u_n}}-\dfrac{u_n+1}{u_n}=\dfrac{1}{u_n}-\dfrac{u_n+1}{u_n}=-1$. Vậy $(v_n)$	là một CSC có công sai $d=-1$.
% 		\\Mặt khác, ta có $v_1=\dfrac{u_1+1}{u_1}=\dfrac{1}{2}$, do đó số hạng tổng quát $v_n=\dfrac{1}{2}+(n-1)(-1)=-n+\dfrac{3}{2}$. \\
% 		Do đó $v_{2023}=-2023+\dfrac{3}{2}=-\dfrac{4043}{2}$.
% 	}
% \end{ex}
% \Closesolutionfile{ans}
% \begin{indapan}{10}
% 	{ans/ans-1K2-2-Dang3}
% \end{indapan}

\begin{dang}{Các bài toán thực tế}
	Các bài toán thực tế về cấp số cộng có thể được giải bằng cách sử dụng công thức của cấp số cộng. Công thức của cấp số cộng là: $ u_n = u_1 + (n-1)d $. Trong đó:
	\begin{itemize}
		\item $ u_n $ là số hạng thứ $ n $ của cấp số cộng.
		\item $ u_1 $ là số hạng đầu tiên của cấp số cộng.
		\item $ d $ là công sai của cấp số cộng.
		\item Một số công thức thường gặp:
		\begin{enumEX}[\faCheckCircleO]{1}
			\item $u_n=\dfrac{u_{n-1}+u_{n+1}}{2}=u_1+(n-1)d$.
			\item $S_n=\dfrac{(u_1+u_n)\cdot n}{2}=\dfrac{2u_1+(n-1)d}{2}\cdot n$.
		\end{enumEX}			
	\end{itemize}
\end{dang}
\subsubsection{Ví dụ minh hoạ}
\begin{vd}%[NB]%[DCHT Toán 11 - KNTT - Nguyễn Hữu Đức] %[1K2B6-6]
	Một người có một khoản tiền gửi ngân hàng với lãi suất 10\% /năm theo hình thức lãi đơn. Nếu sau $ 5 $ năm người đó nhận được tổng số tiền là $ 550 $ triệu đồng thì số tiền gửi ban đầu của người đó là bao nhiêu?
	\dapso{$366{,}67$ triệu đồng.}
	\loigiai{
		Gọi $x$ là số tiền gửi ban đầu của người đó $ (x>0) $.\\
		Sau 5 năm, số tiền nhận được bằng số tiền gốc cộng với lãi suất:
		$$
		x + 0{,}1x \times 5 = 1{,}5x.
		$$
		Theo đề bài, tổng số tiền nhận được sau 5 năm là $550$ triệu đồng, do đó ta có phương trình:
		$$
		1{,}5x = 550.
		$$
		Giải phương trình ta có:
		$$
		x = \frac{550}{1{,}5} \approx 366{,}67.
		$$
		Vậy số tiền gửi ban đầu của người đó là $366{,}67$ triệu đồng.
	}
\end{vd}
\begin{vd}%[DCHT Toán 11 - KNTT - Nguyễn Hữu Đức] %[1K2B6-6]
	Bạn An muốn mua một món quà tặng mẹ nhân ngày mùng $8/3$. Bạn quyết định tiết kiệm từ ngày $1/2/2017$ đến hết ngày $6/3/2017$. Ngày đầu An có $5\,000$ đồng, kể từ ngày thứ hai số tiền An tiết kiệm được ngày sau cao hơn ngày trước mỗi ngày $1\,000$ đồng. Tính số tiền An tiết kiệm được để mua quà tặng mẹ.
	\dapso{$731\,000$ đồng.}
	\loigiai{
		Tính số ngày mà An tiết kiệm được từ ngày $1/2/2017$ đến hết ngày $6/3/2017$:\\
		Số ngày từ ngày $1/2/2017$ đến hết ngày $28/2/2017$ là $28$ ngày.\\
		Số ngày từ ngày $1/3/2017$ đến hết ngày $6/3/2017$ là $6$ ngày.\\
		Vậy An tiết kiệm được $28+6=34$ ngày.\\
		Gọi $u_n$ là số tiền An tiết kiệm được vào ngày thứ $n$ kể từ ngày $1/2/2017$.\\
		Theo đề ta có $u_1=5\,000$ đồng.\\
		Vì ngày sau An tiết kiệm được nhiều hơn ngày trước mỗi ngày $1\,000$ đồng nên $u_n=u_{n-1}+1\,000$, với $n\ge 2$.\\
		Vậy $(u_n)$ là một cấp số cộng với $u_1=5\,000$ và công sai $d=1\,000$.\\
		Tổng số tiền An tiết kiệm được trong $34$ ngày là:
		$$S_{34}=\dfrac{n}{2} \left(2u_1+33d\right)= \dfrac{34}{2} \left(2\cdot 5\,000+33\cdot 1\,000\right)=731\,000.$$
		Vậy số tiền An tiết kiệm được để mua quà tặng mẹ là $731\,000$ đồng.
	}
\end{vd}

\begin{vd}%[TH]%[DCHT Toán 11 - KNTT - Nguyễn Hữu Đức] %[1K2B6-6]
	[Cấp số nhân] Một hội đồng quản trị quyết định tăng lương cho nhân viên hàng năm theo tỷ lệ cố định. Ví dụ, lương của một nhân viên được tăng thêm $ 5 $\% so với năm trước. Hỏi nếu lương của một nhân viên là $ 10 $ triệu đồng/năm vào năm nay, thì lương của nhân viên đó sẽ là bao nhiêu vào năm thứ $ 5 $?
	\dapso{$12{,}1550625 $ triệu đồng/năm.}
	\loigiai{		
		Theo giả thiết, lương của nhân viên được tăng thêm $ 5 $ \% so với năm trước đó.
		\begin{itemize}
			\item Vậy lương của nhân viên vào năm thứ $ 2 $ sẽ là $ 10\cdot(1+0{,}05)=10{,}5 $ triệu đồng/năm.
			\item Tương tự, lương của nhân viên vào năm thứ $ 3 $ sẽ là $ 10{,}5 \cdot(1+0{,}05)=11{,}025 $ triệu đồng/năm.
			\item Lương của nhân viên vào năm thứ $ 4 $ sẽ là $ 11{,}025\cdot (1+0{,}05)=11{,}57625 $ triệu đồng/năm.
			\item Cuối cùng, lương của nhân viên vào năm thứ $ 5 $ sẽ là $ 11{,}57625\cdot(1+0{,}05)=12{,}1550625 $ triệu đồng/năm.
		\end{itemize}
		Vậy lương của nhân viên đó vào năm thứ $ 5 $ sẽ là $ 12{,}1550625 $ triệu đồng/năm.\\
		Chú ý: Lương của nhân viên đó vào năm thứ $ 5 $ sẽ là $ u_5=u_1+4d=10+4\cdot 10\cdot 0{,}05=12 $ triệu đồng chỉ đúng trong trường hợp lương của một nhân viên được tăng thêm $ 5 $\% so với năm đầu tiên.
	}
\end{vd}


\begin{vd}%[TH]%%[DCHT Toán 11 - KNTT - Nguyễn Hữu Đức] %[1K2B6-6]
	Hùng đang tiết kiệm để mua một cây guitar. Trong tuần đầu tiên, anh ta để dành  $ 42 $ đô la, và trong mỗi tuần tiết theo, anh ta đã thêm $ 8 $ đô la vào tài khoản tiết kiệm của mình. Cây guitar Hùng cần mua có giá $ 400 $ đô la. Hỏi vào tuần thứ bao nhiêu thì anh ấy có đủ tiền để mua cây guitar đó?
	\dapso{$ n=46 $.}
	\loigiai{
		Gọi $ n $ là số tuần anh ta đã thêm $ 8 $ đô la vào tài khoản tiết kiệm của mình.\\
		Số tiền anh ta tiết kiệm được sau $ n $ tuần đó là $ S=42+8n $. \\
		Theo bài ra $ S=42+8n\ge 400\Leftrightarrow n\ge 44,75\Rightarrow n=45 $.\\
		Vậy kể cả tuần đầu thì tuần thứ $ 46 $ anh ta có đủ tiền để mua cây guitar đó.
	}
\end{vd}

\begin{vd}%[DCHT Toán 11 - KNTT - Nguyễn Hữu Đức]%[1K2Y6-6]
	[Cấp số nhân] Hàng tháng ông An gửi vào ngân hàng một số tiền như nhau là $5\,000\,000$ đồng (vào ngày đầu mỗi tháng) với lãi suất $0{,}5\%$ một tháng, biết tiền lãi của tháng trước được nhập vào tiền gốc của tháng sau. Hỏi sau $36$ tháng ông An nhận được số tiền vốn và lãi là bao nhiêu? (làm tròn đến hàng đơn vị).
	\dapso{ $ 197\,663\,927 $ đồng.}
	\loigiai{
		Gọi $a$ là số tiền ông An gửi vào hàng tháng, $r$ là lãi suất trên một tháng và $P_n$ là số tiền vốn và lãi ông An nhận được sau $n$ tháng.
		\begin{itemize}
			\item Sau một tháng, ông An có số tiền là $P_1=a+ar=a(1+r)$.
			\item Đầu tháng thứ hai, ông An có số tiền là $P_1+a=a(1+r)+a$.
			\item Sau hai tháng, ông An có số tiền là $P_2=a(1+r)+a+\left[a(1+r)+a\right]r=a\left[(1+r)^2+(1+r)\right]$.
			\item Cuối tháng thứ $36$, ông An có số tiền là
			\begin{align*}
				P_{36}&=a\left[(1+r)^{36}+(1+r)^{35}+\ldots+(1+r)\right]\\
				&=a(1+r)\dfrac{(1+r)^{36}-1}{r}\\
				&=5000000\cdot (1+0{,}005)\cdot\dfrac{(1+0{,}005)^{36}-1}{0{,}005}\\
				&\approx 197\,663\,927 \quad \text{(đồng)}.
			\end{align*}
		\end{itemize}
	}
\end{vd}

\begin{vd}[VDT]%[DCHT Toán 11 - KNTT - Nguyễn Hữu Đức] %[1K2Y6-6]
	Một xưởng có đăng tuyển công nhân với đãi ngộ về lương như sau: Trong quý đầu tiên thì xưởng trả là $ 6 $ triệu đồng/quý và kể từ quý thứ $ 2 $ sẽ tăng lên $ 0{,}5 $ triệu cho $ 1 $ quý. Hỏi với đãi ngộ trên thì sau $ 5 $ năm làm việc tại xưởng, tổng số lương của công nhân đó là bao nhiêu?
	\dapso{ $ 215 $ triệu đồng.}
	\loigiai{
		Gọi $u_n$ (triệu đồng) là số lương của công nhân trong quý thứ $n$.\\
		Theo đề:\\
		Quý đầu: $ u_1 = 6 $ triệu.\\
		Các quý tiếp theo: $ u_{n+1} = u_{n} + 0,5 $ với $\forall n \ge 1$.\\
		Mức lương của công nhân mỗi quý là $ 1 $ số hạng của dãy số $ u_n $. Mặt khác, lương của quý sau hơn lương quý trước là $ 0,5 $ triệu nên dãy số $ u_n $ là một cấp số cộng với công sai $ d = 0{,}5 $.\\
		Ta biết $ 1 $ năm sẽ có $ 4 $ quý nên $ 5 $ năm sẽ có $ 5\cdot 4 = 20 $ quý. Theo yêu cầu của đề bài ta cần tính tổng của $ 20 $ số hạng đầu tiên của cấp số cộng ($ u_n $).\\
		Lương tháng quý $ 20 $ của công nhân: $ u_{20} = 6 + (20 - 1)\cdot 0{,}5 = 15{,}5 $ triệu đồng.\\
		Tổng số lương của công nhân nhận được sau $ 5 $ năm làm việc tại xưởng: $ S_{12}=20\cdot (6+15,5)2=215 $ (triệu đồng).
	}
\end{vd}

% \subsubsection{Bài tập tự luận}
 
%[DCHT Toán 11 - KNTT - Nguyễn Hữu Đức] %[1K2B6-6]
% \begin{bt}%[NB]%[DCHT Toán 11 - KNTT - Nguyễn Hữu Đức] %[1K2B6-6]
% 	Sinh nhật bạn của An vào ngày $ 01 $ tháng năm. An muốn mua một món quà sinh nhật cho bạn nên quyết định bỏ ống heo $ 100 $ đồng vào ngày $ 01 $ tháng $ 01 $ năm $ 2016 $, sau đó cứ liên tục ngày sau hơn ngày trước $ 100 $ đồng. Hỏi đến ngày sinh nhật của bạn, An đã tích lũy được bao nhiêu tiền? (thời gian bỏ ống heo tính từ ngày $ 01 $ tháng $ 01 $ năm $ 2016 $ đến ngày $ 30 $ tháng $ 04 $  năm $ 2016 $).\dapso{$ 738\,100 $ đồng.}
% 	\loigiai{
% 		Từ ngày $1$ tháng $1$ năm $2016$ đến ngày $30$ tháng $4$ năm $2016$ có tổng cộng $31+29+31+30=121$ ngày.\\
% 		Gọi $S$ là số tiền An tích lũy được vào ngày sinh nhật của bạn.\\
% 		Do An bỏ được $100$ đồng vào ngày đầu tiên nên số tiền An tích lũy được vào ngày thứ $n$ là 
% 		$$S= 100 + 100(n-1).$$
% 		Vậy tổng số tiền An tích lũy được là:
% 		$$
% 		S=100 + 200 + \cdots + 12\,100 = \frac{121(100 + 12\,100)}{2} = 738\,100
% 		.$$
% 		Vậy An đã tích lũy được $738\,100$ đồng vào ngày sinh nhật của bạn.}
% \end{bt}

% \begin{bt}%[TH]%[DCHT Toán 11 - KNTT - Nguyễn Hữu Đức] %[1K2Y6-6]
% 	Người ta trồng $ 3\,003 $ cây theo dạng một hình tam giác như sau: hàng thứ nhất trồng $ 1 $ cây, hàng thứ hai trồng $ 2 $ cây, hàng thứ ba trồng $ 3 $ cây... cứ tiếp tục trồng như thế cho đến khi hết số cây. Số hàng cây được trồng là bao nhiêu?
% 	\dapso{$ 77 $ hàng.}
% 	\loigiai{
% 		Tổng số cây trồng được là $1 + 2 + 3 + \cdots + n$, nghĩa là tổng của $n$ số tự nhiên đầu tiên. Ta cần tìm số $n$ để tổng này bằng $3003$.\\
% 		Ta có công thức tổng của $n$ số tự nhiên đầu tiên là:
% 		$$
% 		1 + 2 + 3 + \cdots + n = \frac{n(n+1)}{2}.
% 		$$
% 		Giải phương trình:
% 		$$
% 		\frac{n(n+1)}{2} = 3\,003.
% 		$$
% 		Ta có:
% 		$$
% 		n(n+1) = 6\,006
% 		\Rightarrow n=77.$$
% 		Vậy số hàng cây được trồng là $77$.
% 	}
% \end{bt}
% \begin{bt}%[TH]%[DCHT Toán 11 - KNTT - Nguyễn Hữu Đức] %[1K2B6-6]
% 	Một công ty định mức sản phẩm hàng tháng theo cấp số cộng. Ví dụ, sản lượng hàng tháng của một công ty được tăng thêm $10$ sản phẩm so với tháng trước. Nếu công ty sản xuất được $ 100 $ sản phẩm trong tháng này, hỏi công ty sẽ sản xuất được bao nhiêu sản phẩm trong tháng thứ $ 12 $?
% 	\dapso{$ 210 $ sản phẩm.}
% 	\loigiai{
% 		Công thức cấp số cộng được sử dụng để tính sản lượng hàng tháng của công ty. Nếu công ty sản xuất được $100$ sản phẩm trong tháng này và sản lượng hàng tháng được tăng thêm $10$ sản phẩm so với tháng trước, ta có thể sử dụng công thức sau để tính sản lượng hàng tháng của công ty trong tháng thứ $12$:
% 		$$
% 		a_n = a_1 + (n-1)d.
% 		$$
% 		Trong đó $a_1$ là sản lượng hàng tháng ban đầu, $d$ là công sai và $n$ là số tháng.\\
% 		Với bài toán này, ta có: $a_1 = 100$, $d = 10$, $n = 12$.\\
% 		Sản lượng hàng tháng của công ty trong tháng thứ $12$ là:
% 		$$
% 		a_{12} = a_1 + (n-1)d = 100 + (12-1) \times 10 = 210.
% 		$$
% 		Vậy công ty sẽ sản xuất được $210$ sản phẩm trong tháng thứ $12$.
% 	}
% \end{bt}

\subsubsection{Câu hỏi trắc nghiệm}
\Opensolutionfile{ans}[ans/ans-1K2-2-dang4]
\begin{ex}%[TH]%[DCHT Toán 11 - KNTT - Nguyễn Hữu Đức]%[1K2B6-6]
	Một công ty đang cần tuyển dụng thêm nhân viên. Công ty quyết định tăng số lượng nhân viên hàng tháng theo cấp số cộng. Nếu công ty đã có $ 20 $ nhân viên và quyết định tăng thêm $ 2 $ nhân viên hàng tháng, hỏi sau bao nhiêu tháng công ty sẽ có 50 nhân viên?
	\choice
	{$19$ tháng}
	{\True $16$ tháng}
	{$36$ tháng}
	{$26$ tháng}
	\loigiai{
		Để giải bài toán này, ta có thể sử dụng công thức cấp số cộng:
		$$
		a_n = a_1 + (n-1) \times d.
		$$
		Trong đó $a_1$ là số lượng nhân viên ban đầu, $d$ là số lượng nhân viên tăng hàng tháng và $n$ là số tháng.\\
		Ta cần tìm số tháng $n$ để công ty có được $50$ nhân viên. Thay các giá trị vào công thức cấp số cộng ta có:
		$$
		50 = 20 + (n-1) \times 2.
		$$
		Suy ra:
		$$
		n = \frac{50 - 20}{2} + 1 = 16
		.$$
		Vậy sau $16$ tháng kể từ khi công ty quyết định tăng số lượng nhân viên hàng tháng theo cấp số cộng, công ty sẽ có được $50$ nhân viên.
	}
\end{ex}
\begin{ex}%[VD]%[DCHT Toán 11 - KNTT - Nguyễn Hữu Đức] %[1K2B6-6]
	Một người đang tăng cường luyện tập thể thao hàng ngày. Anh ta quyết định tăng mức độ luyện tập theo cấp số cộng hàng tuần. Nếu anh ta bắt đầu với mức luyện tập $ 30 $ phút mỗi ngày và tăng thêm $ 5 $ phút mỗi ngày, hỏi anh ta sẽ luyện tập được bao lâu để đạt được mức luyện tập $ 60 $ phút mỗi ngày?
	\choice
	{$16$ ngày}
	{\True $6$ ngày}
	{$9$ ngày}
	{$7$ ngày}
	\loigiai{
		Gọi $n$ là số ngày liên tiếp mà người đó tăng mức độ luyện tập. Theo đó, mức độ luyện tập của người đó sau $n$ ngày là:
		$$
		30 + 5n\, \text{(phút)}.
		$$
		Vì để đạt được mức luyện tập $ 60 $ phút mỗi ngày nên:
		$$
		30 + 5n = 60.
		$$
		Từ đó suy ra:
		$$
		n = \frac{60-30}{5} = 6.
		$$
		Vậy người đó cần luyện tập liên tiếp trong $6$ ngày để đạt được mức luyện tập $60$ phút mỗi ngày.	
	}
\end{ex}
\begin{ex}%[VD]%[DCHT Toán 11 - KNTT - Nguyễn Hữu Đức] %[1K2Y6-6]
	Nếu một công ty công nghệ mới thành lập có số lượng người dùng ban đầu là $ 10\,000 $ và mỗi tháng tăng thêm cố định $ 5\,000 $ lượng người dùng, thì sau bao lâu có số lượng người dùng là $ 1 $ triệu. 
	\choice
	{\True $198$ tháng}
	{$197$ tháng}
	{$18$ tháng}
	{$98$ tháng}
	
	\loigiai{
		Ta cần tính số tháng $n$ theo công thức sau:
		$$10\,000 + 5\,000n = 1\,000\,000.$$
		$$\Rightarrow n = \frac{1\,000\,000 - 10\,000}{5\,000} = 198.$$
		Vậy sau khoảng $198$ tháng (khoảng $16$ năm và $6$ tháng), công ty sẽ đạt được $1$ triệu người dùng.
	}
\end{ex}
\begin{ex}%[VDC]%[DCHT Toán 11 - KNTT - Nguyễn Hữu Đức] %[1K2Y6-6]
	Một nhà đầu tư đang đầu tư vào một quỹ đầu tư với mức lợi nhuận cố định hàng năm. Nếu nhà đầu tư đầu tư vào quỹ đầu tư với số tiền ban đầu là $ 20 $ triệu đồng và mức lợi nhuận hàng năm là $ 10 $\%, hỏi số tiền nhà đầu tư sẽ nhận được sau $ 7 $ năm?
	\choice
	{\True $34$ triệu đồng}
	{$14$ triệu đồng}
	{$30$ triệu đồng}
	{$39$ triệu đồng}
	\loigiai{
		Với số tiền ban đầu là $ 20 $ triệu đồng và mức lợi nhuận hàng năm là $ 10 $\%, ta có thể tính được số tiền nhà đầu tư sẽ nhận được sau $ 1 $ năm, sau đó sử dụng cấp số cộng để tính số tiền nhà đầu tư sẽ nhận được sau $ 7 $ năm.
		
		Số tiền nhà đầu tư sẽ nhận được sau $ 1 $ năm là:
		
		$ 20 $ triệu đồng $\times$ $ 10 $\% = $ 2 $ triệu đồng
		
		Số tiền nhà đầu tư sẽ nhận được sau $ 7 $ năm là:
		
		$ 2 $ triệu đồng $\times$ $ 7 $ năm + $ 20 $ triệu đồng = $ 34 $ triệu đồng
		
		Vậy sau $ 7 $ năm, nhà đầu tư sẽ nhận được tổng cộng $ 34 $ triệu đồng.
	}
\end{ex}
\begin{ex}%[VDC]%[DCHT Toán 11 - KNTT - Nguyễn Hữu Đức] %[1K2Y6-6]
	Một công ty sản xuất bánh kẹo tăng sản lượng sản phẩm của mình lên mỗi tháng. Nếu sản lượng ban đầu là $ 1\,000 $ sản phẩm, một sản phẩm lợi nhuận $ 1 $ USD và tăng thêm $ 200 $ sản phẩm mỗi tháng, thì sau bao nhiêu tháng lợi nhuận công ty $ 1 $ triệu đô.
	\choice
	{$8\,000$ tháng}
	{$7\,000$ tháng}
	{$9\,000$ tháng}
	{\True $5\,000$ tháng}
	\loigiai{
		Để tính thời gian công ty đạt được lợi nhuận 1 triệu đô, chúng ta cần biết lợi nhuận của công ty đạt được bao nhiêu sau mỗi tháng.\\
		Giả sử sản lượng ban đầu là $1\,000$ sản phẩm một sản phẩm lợi nhuận $1$ USD và tăng thêm $200$ sản phẩm mỗi tháng. Ta có thể tính được lợi nhuận của công ty sau mỗi tháng như sau:
		\begin{itemize} 
			\item Tháng 1: $1\,000 \times 1 = 1000$ USD.
			\item Tháng 2: $(1\,000 + 200) \times 1 = 1200$ USD.
			\item Tháng 3: $(1\,000 + 2 \times 200) \times 1 = 1\,400$ USD.
			\item Tháng 4: $(1\,000 + 3 \times 200) \times 1 = 1\,600$ USD.
			\item Tháng $n$: $(1\,000 + (n - 1) \times 200) \times 1 = (n - 1) \times 200 + 1\,000$ USD.
		\end{itemize}
		Để tính thời gian để công ty đạt được lợi nhuận $1$ triệu đô, ta giải phương trình sau:
		
		$(n - 1) \times 200 + 1\,000 = 10^6$
		
		$\Rightarrow (n - 1) \times 200 = (10^6 - 1000)$
		
		$\Rightarrow n - 1 = \dfrac{10^6 - 1\,000}{200}$
		
		$\Rightarrow n = \dfrac{10^6 - 1\,000}{200} + 1$
		
		$\Rightarrow n = 5\,001$
		
		Vậy sau $5\,000$ tháng, công ty sẽ đạt được lợi nhuận $1$ triệu đô.
	}
\end{ex}
\begin{ex}%[VDC]%[DCHT Toán 11 - KNTT - Nguyễn Hữu Đức] %[1K2Y6-6]
	Một công ty tăng lương cho nhân viên hàng năm bằng cách thêm một số tiền cố định vào lương của họ. Ví dụ: Nếu lương ban đầu của một nhân viên là $ 10 $ triệu đồng và công ty tăng lương $ 2 $ triệu đồng mỗi năm, thì lương của nhân viên sẽ là bao nhiêu nếu làm cho công ty $ 19 $ năm?
	\choice
	{$ 16 $ triệu đồng}
	{$ 26 $ triệu đồng}
	{$ 28 $ triệu đồng}
	{\True $ 46 $ triệu đồng}
	\loigiai{
		Do tăng lương cho nhân viên hàng năm bằng cách thêm một số tiền cố định nên ta có thể sử dụng công thức tính số hạng thứ $ n $ của cấp số cộng
		$ a_n = a_1 + (n - 1)d $.\\
		Ở bài toán này, ta có:\\
		$ a_1 = 10 $ (triệu đồng) là lương ban đầu của nhân viên.\\
		$ d = 2 $ (triệu đồng) là công sai của cấp số cộng.\\
		$ n = 19  $ là số thứ tự của số hạng.\\
		Ta thay các giá trị này vào công thức trên để tính lương của nhân viên sau $ 19 $ năm:\\
		$ a_{19} = 10 + (19 - 1)2 \Rightarrow$
		$ a_{19} = 46 $ (triệu đồng).\\
		Vậy lương của nhân viên sau $ 19 $ năm làm việc cho công ty là $ 46 $ triệu đồng.
	}
\end{ex}

\begin{ex}%[VDC]%[DCHT Toán 11 - KNTT - Nguyễn Hữu Đức] %[1K2Y6-6]
	Tài sản thường bị khấu hao khiến chúng có tuổi thọ hữu ích giới hạn. Ví dụ, nếu một công ty mua một chiếc xe tải với giá $ 35\,000 $ đô la và nó bị khấu hao với tốc độ không đổi là $ 700 $ đô la mỗi tháng, thì sau bao lâu giá trị của nó còn $ 5\,000 $ đô la.
	\choice
	{$ x = 23 $ tháng}
	{\True  $ x = 43 $ tháng}
	{$ x = 41 $ tháng}
	{$ x = 40 $ tháng}
	\loigiai{
		\textit{Cách 1:} Thời gian để giá trị của chiếc xe tải trên được khấu hao xuống còn $5.000 $ đô la có thể được tính bằng cách sử dụng công thức sau:\\
		Giá trị khởi đầu của chiếc xe tải là $35\,000$
		Giá trị cuối cùng của chiếc xe tải là $5\,000$
		Tốc độ khấu hao tương ứng $700$/tháng\\
		Để tìm ra thời gian cần thiết để giá trị của chiếc xe tải giảm xuống còn $5.000$, ta cần tìm số tháng được khấu hao.\\
		Giả sử số tháng cần khấu hao là $ x $ tháng.\\
		Giá trị của chiếc xe tải sau $ x $ tháng khấu hao được tính bằng:\\ 
		$35\,000 - 700x = 5\,000$.\\
		Giải phương trình trên ta có: $ x \approx 43 $ tháng\\
		Vì vậy, sau $ 43 $ tháng, giá trị của chiếc xe tải sẽ giảm xuống còn $5\,000$.
		Ngoài ra ta có thể giải theo cấp số cộng như sau:\\
		\textit{Cách 2:} Ta có thể sử dụng cộng thức tính số hạng thứ $ n $ của cấp số cộng
		$ a_n = a_{1} + (n - 1)d $
		\begin{itemize}
			\item $ u_1 = 35\,000 $ (đô la) là giá trị ban đầu của xe tải.
			\item $ d = -700 $ (đô la) là công sai của cấp số cộng (âm vì giá trị xe tải giảm).
			\item $ a_n = 5\,000 $ (đô la) là giá trị cuối cùng của xe tải.
		\end{itemize}
		Ta thay các giá trị này vào công thức trên để tính số tháng mà xe tải bị khấu hao đến $ 5\,000 $ đô la:
		$$ 5\,000 = 35\,000 + (n - 1)(-700)\Rightarrow n = 43{,}857.$$
		Vậy sau khoảng $ 43{,}857 $ tháng, tức là khoảng $ 3 $ năm và $ 7 $ tháng, giá trị của xe tải sẽ còn khoảng $ 5\,000 $ đô la.
	}
\end{ex}
\begin{ex}%[VDC]%[DCHT Toán 11 - KNTT - Nguyễn Hữu Đức] %[1K2Y6-6]
	Các thiết bị điện tử như máy tính, điện thoại, hoặc máy ảnh thường bị khấu hao nhanh chóng do sự phát triển của công nghệ mới. Ví dụ, nếu một người mua một máy tính Macbook với giá $ 2\,000 $ đô la và nó bị khấu hao với tốc độ không đổi là $ 100 $ đô la mỗi tháng, thì giá trị của Macbook còn lại $ 1\,000 $ đô la sau bao nhiêu tháng?
	%\dapso{$ 11 $ tháng.}
	\choice
	{$ x = 12 $ tháng}
	{$ x = 43 $ tháng}
	{\True  $ x = 11 $ tháng}
	{$ x = 10 $ tháng}
	\loigiai{
		Để giải bài toán này, ta có thể sử dụng công thức tính số hạng thứ $ n $ của cấp số cộng $ a_n = a + (n - 1)d. $\\
		Ở bài toán này, ta có:\\
		$ a = 2\,000 $ (đô la) là giá trị ban đầu của máy tính Macbook.\\
		$ d = -100 $ (đô la) là công sai của cấp số cộng (âm vì giá trị máy tính giảm).\\
		$ a_n = 1\,000 $ (đô la) là giá trị cuối cùng của máy tính Macbook.\\
		Ta thay các giá trị này vào công thức trên để tính số tháng mà máy tính bị khấu hao đến $ 1\,000 $ đô la:
		$$ 1\,000 = 2\,000 + (n - 1)(-100)\Rightarrow n=11. $$
		Vậy sau $ 11 $ tháng, giá trị của máy tính Macbook sẽ còn $ 1\,000 $ đô la.
	}
\end{ex}
\begin{ex}%[VDC]%[DCHT Toán 11 - KNTT - Nguyễn Hữu Đức] %[1K2Y6-6]
	Ban đầu có 1m$^2$ bèo sinh sôi trên mặt hồ biết tốc độ sinh sôi ngày sau hơn ngày trước $ 0{,}5 $m$^2 $. Biết diện tích mặt hồ nước là $ 120 $m$^2 $ hỏi sau bao lâu bèo phủ đầy mặt hồ?
	%\dapso{$ 238 $ ngày}
	\choice
	{$ x = 120 $ tháng}
	{$ x = 143 $ tháng}
	{\True  $ x = 238 $ tháng}
	{$ x = 130 $ tháng}
	\loigiai{
		Giả sử sau $ x $ ngày, diện tích của bèo phủ đầy mặt hồ là $ S m^2 $.
		
		Theo đề bài, ta biết được rằng:
		\begin{itemize}
			\item Tốc độ sinh sôi của bèo là $ 0{,}5 $m$^2 $/ngày.
			\item Ban đầu, diện tích của bèo là  1 m$^2 $.
			\item Diện tích mặt hồ là  $ 120 $m$^2 $.
		\end{itemize}
		Vậy ta có phương trình sau đây:	$ S = 1 + 0{,}5x. $\\
		Điều kiện để bèo phủ đầy mặt hồ là $ S = 120 $.\\
		$ 1 + 0{,}5x = 120 $ hay	$ 0{,}5x = 119 $ $\Rightarrow x = 238 $ ngày.\\
		Vậy sau $ 238 $ ngày, bèo sẽ phủ đầy mặt hồ.
	}
\end{ex}
\begin{ex}%[VDC]%[DCHT Toán 11 - KNTT - Nguyễn Hữu Đức] %[1K2Y6-6]
	Nhà hát lớn Dạ Cỗ Vĩ Lan ở An Cư có hàng ghế đầu kí hiệu dãy A là $50$ chỗ hàng ghế, sau dãy B là $48$ chỗ và như thế hàng sau ít hơn hàng trước $ 2 $ ghế, biết hàng cuối cùng có $ 10 $ ghế. Tính tổng số dãy ghế và tổng số chỗ ngồi?
	%\dapso{ $21$ dãy và $ 630 $ chỗ.}
	\choice
	{\True  $21$ dãy và $ 630 $ chỗ}
	{$20$ dãy và $ 630 $ chỗ}
	{$11$ dãy và $ 630 $ chỗ}
	{$21$ dãy và $ 930 $ chỗ}
	\loigiai{
		Gọi $n$ là số dãy ghế. Theo đề bài, ta có:
		$$
		\begin{cases}
			S=50 + 48 + \cdots + 10 = \dfrac{50+10}{2}n \\
			S=\dfrac{2.50+(n-1)\cdot (-2)}{2}n
		\end{cases}
		$$
		Từ phương trình đầu tiên, ta có:
		$$
		S = 50 + 48 + \cdots + 10 = \frac{50+10}{2}n = 30n.
		$$
		Từ phương trình thứ hai, ta có:
		$$
		S = \frac{2\cdot 50+(n-1)\cdot(-2)}{2}n = (50 - n + 1)n = (51 - n)n.
		$$
		Do đó, ta có:
		$$
		30n = (51 - n)n
		\Rightarrow n=21.$$
		Vậy $n = 21$ dãy ghế và $ 30\cdot 21=630 $ ghế.
		
	}
\end{ex}
\begin{ex}%[VDC]%[DCHT Toán 11 - KNTT - Nguyễn Hữu Đức] %[1K2Y6-6]
	Người ta trồng  cây theo dạng một hình tam giác như sau: hàng thứ nhất trồng $ 1 $ cây, hàng thứ hai trồng $ 3 $ cây, hàng thứ ba trồng $ 5 $ cây,... cứ tiếp tục trồng như thế cho đến khi hết số cây là $ 6\,561 $. Số hàng cây được trồng là bao nhiêu?
	%\dapso{ $ 81 $ hàng.}
	\choice
	{\True  $ 81 $ hàng}
	{$ 16 $ hàng}
	{$ 100 $ hàng}
	{$ 89 $ hàng}
	\loigiai{
		Để giải bài toán này, ta cần tìm số hàng cây được trồng cho đến khi tổng số cây là $ 2023 $. 
		\begin{itemize}
			\item Hàng thứ nhất trồng $ 1 $ cây. 
			\item Hàng thứ hai trồng $ 3 $ cây ($ 1 $ cây $ + 2 $ cây).
			\item Hàng thứ ba trồng $ 5 $ cây ($ 1 $ cây $ + 2 $ cây $ + 2 $ cây).
			\item ...
		\end{itemize}
		Vậy ta thấy rằng số cây trồng trong hàng thứ $n$ là $(n-1)\cdot 2+1$. \\
		Số cây được trồng trong $n$ hàng đầu tiên là: 
		$$1 + 3 + 5 + ... + (2n-1) = n^2.$$ 
		Để tìm số hàng cây được trồng cho đến khi tổng số cây là $ 6561 $, ta giải phương trình sau:\\ 
		$n^2 = 6\,561.$ 
		Vậy số hàng cây được trồng là $ 81 $ hàng.
	}
\end{ex}
\begin{ex}%[VDC]%[DCHT Toán 11 - KNTT - Nguyễn Hữu Đức] %[1K2Y6-6]
	Người ta thả một $ 1 $ m$^2$ lá bèo vào một hồ nước. Kinh nghiệm cho thấy sau $ x $ giờ, bèo sẽ sinh sôi kín cả mặt hồ $ 500 $ m$^2 $. Biết rằng sau mỗi giờ, lượng lá bèo tăng thêm $ 0{,}5 $ m$^2 $ và tốc độ tăng không đổi tìm $ x $?
	%\dapso{ $999$ giờ.}
	\choice
	{$888$ giờ}
	{$777$ giờ}
	{\True  $999$ giờ}
	{$700$ giờ}
	\loigiai{
		Bài toán này có thể giải bằng cách sử dụng công thức tăng trưởng của bèo. Giả sử lượng lá bèo ban đầu là $ 1 $ m$^2$, sau mỗi giờ lượng lá bèo tăng thêm $ 0{,}5 $ m$^2$. Sau $x$ giờ, lượng lá bèo đã phủ kín mặt hồ $ 500 $ m$^2$. Ta có thể viết phương trình sau:
		$$1 + 0{,}5x = 500.$$
		Giải phương trình ta được:
		$$x = \frac{500-1}{0{,}5} \approx 999.$$
		Vậy sau khoảng $999$ giờ (khoảng 41 ngày), lượng lá bèo sẽ phủ kín mặt hồ $ 500 $ m$^2$.
	}
\end{ex}
\Closesolutionfile{ans}
% \begin{indapan}{10}
% 	{ans/ans-1K2-2-dang6}
% \end{indapan}
%%Bài 7. CSN
% \def\tenchude{CẤP SỐ NHÂN}
\setcounter{section}{6}
\setcounter{dang}{0}
\setcounter{ex}{0}
\setcounter{bt}{0}
\setcounter{vd}{0}
\section{Cấp số nhân}
\subsection{Tóm tắt lý thuyết}
\begin{tomtat}
	\subsubsection{Định nghĩa} 
	Cấp số nhân là một dãy số (hữu hạn hoặc vô hạn) mà trong đó, kể từ số hạng thứ hai, mỗi số hạng đều bằng tích một số đứng ngay trước nó với một số $ q $ không đổi, nghĩa là:
	$$ u_{n}=u_{n-1}\cdot q\,\,\text{với}\,\forall n\in \mathbf{N}{,}\,n\ge 2 $$
	Số $ q $ được gọi là công bội của cấp số nhân
	\subsubsection{Số hạng tổng quát của cấp số nhân}
	Nếu cấp số nhân $ (u_n) $ có số hạng đầu là $ u_1 $ và công bội $ q $ thì số hạng tổng quát $ u_n $ của nó được xác định bởi công thức:
	$$u_n = u_1 \cdot q^{n-1}\,,n\ge 2$$
	\subsubsection{Tổng của $ n $ số hạng đầu tiên của cấp số nhân}
	Giả sử $ (u_n) $ là cấp số nhân có công bội $ q\ne 1 $. Đặt $ S_n=u_1+u_2+\cdots +u_n, $ khi đó
	$$S_n = u_1\cdot\frac{1-q^n}{1-q}.$$
	\begin{note}
		Khi $ q=1 $ thì $ S_n=n\cdot u_1 $.
	\end{note}	
	\begin{itemize}
		\item Công bội của cấp số nhân: $q = \sqrt[n-1]{\frac{u_n}{u_1}}$.
		\item Số hạng đầu tiên của cấp số nhân: $u_1 = \frac{u_n}{q^{n-1}}$.
		\item $ a,b,c $ là ba số hạng liên tiếp cấp số nhân thì $ a\cdot c=b^2 $. 
	\end{itemize}	
\end{tomtat}
\subsection{Các dạng toán thường gặp}
\begin{dang}{Nhận diện cấp số nhân, công bội $ q $}
	Để nhận diện (chứng minh) mỗi dãy số là cấp số nhân, ta làm như sau:\\
	Chứng minh $ u_{n+1}=u_nq $, $ \forall n\in\mathbb{N}^* $ và $ q $ là một số không đổi.\\
	Nếu $ u_n\ne 0 $, $ \forall n\in\mathrm{N}^* $ thì ta lập tỉ số $ \dfrac{u_{n+1}}{u_n}=k $.
	\begin{itemize}
		\item Nếu $ k $ là hằng số thì $ (u_n) $ là cấp số nhân với công bội $ q=k $.
		\item Nếu $ k $ phụ thuộc vào $ n $ thì $ (u_n) $ không phải là cấp số nhân.
	\end{itemize}
	Để chứng minh dãy $ (u_n) $ không phải là một cấp số nhân. Khi đó, ta chỉ cần chỉ ra ba số hạng liên tiếp không tạo thành một cấp số nhân, chẳng hạn $ \dfrac{u_3}{u_2}\ne \dfrac{u_2}{u_1} $.\\
	Để chứng minh ba số $ a,b,c $ theo thứ tự đó lập được một cấp số nhân, thì ta chứng minh $ ac=b^2 $ hoặc $ |b|=\sqrt{ac} $.	
\end{dang}
\subsubsection{Ví dụ minh hoạ}
\begin{vd}%[NB]%[DCHT Toán 11 - KNTT -Tên Huỳnh Thanh Chí]%[1K2Y7-1]
	Dãy số $ 1;1;1;1;\ldots $ có phải là một cấp số nhân hay không?
	\dapso{Dãy số $ 1;1;1;1;\ldots $ là một cấp số nhân.}
	\loigiai{
	Dễ thấy $ \dfrac{u_2}{u_1}=\dfrac{u_3}{u_2}=\ldots=1 $ là một số không đổi.\\
	Do đó dãy số $ 1;1;1;1;\ldots $ là một cấp số nhân.
	}
\end{vd}
\begin{vd}%[TH]%[DCHT Toán 11 - KNTT -Tên Huỳnh Thanh Chí] %[ID6 chương trình mới]
	Dãy số $ u_n=3^n $ có phải là một cấp số nhân không? Nếu có, hãy tìm công bội của cấp số nhân đó.
	\dapso{$ (u_n) $ là cấp số nhân với công bội $ q=3 $.}
	\loigiai{
	Ta có $ \dfrac{u_{n+1}}{u_n}=\dfrac{3^{n+1}}{3^n}=\dfrac{3^n\cdot 3}{3^n}=3 $ là số không đổi nên $ (u_n) $ là cấp số nhân với công bội $ q=3 $.
	}
\end{vd}
\begin{vd}%[TH]%[DCHT Toán 11 - KNTT -Tên Huỳnh Thanh Chí]%[1K2B7-1]
	Dãy số $ \heva{& u_1=3\\ & u_{n+1}=\dfrac{9}{u_n}} $ có phải là một cấp số nhân không? Nếu có, hãy tìm công bội của cấp số nhân đó.
	\dapso{$ (u_n) $ là một cấp số nhân với công bội $ q=1 $.}
	\loigiai{
	Xét dãy số $ \heva{& u_1=3\\ & u_{n+1}=\dfrac{9}{u_n}} $ có
	$ \dfrac{u_{n+1}}{u_n}=\dfrac{9}{u_n}:\dfrac{9}{u_{n-1}}=\dfrac{u_{n-1}}{u_n}\Rightarrow u_{n+1}=u_{n-1},\forall n\ge 2 $.\\
	Do đó ta có $ \heva{& u_1=u_3=u_5=\ldots=u_{2n+1}=\ldots \quad (1)\\ & u_2=u_4=u_6=\ldots=u_{2n}=\ldots \quad (2).} $\\
	Theo đề bài ta có $ u_1=3 \Rightarrow u_2=\dfrac{9}{u_1}=3 $ (3).\\
	Từ $ (1), (2) $ và $ (3) $ suy ra $ u_1=u_2=u_3=u_4=\ldots=u_{2n}=u_{2n+1}=\ldots $.\\
	Do đó $ (u_n) $ là một cấp số nhân với công bội $ q=1 $.
	}
\end{vd}
\begin{vd}%[TH]%[DCHT Toán 11 - KNTT -Tên Huỳnh Thanh Chí]%[1K2B7-1]
	Cho $ (u_n) $ là cấp số nhân có công bội $ q\ne 0,u_1\ne 0 $. Chứng minh rằng dãy số $ (v_n) $ với $ v_n=u_nu_{2n} $ cũng là một cấp số nhân.
	\dapso{$ (v_n) $ là một cấp số nhân với công bội là $ q^3 $.}
	\loigiai{
		Ta có $ \dfrac{v_n}{v_{n-1}}=\dfrac{u_nu_{2n}}{u_{n-1}u_{2(n-1)}}=\dfrac{u_1q^{n-1}\cdot u_1q^{2n-1}}{u_1q^{n-2}\cdot u_1q^{2n-3}}=q^3 $. Do đó $ (v_n) $ là một cấp số nhân với công bội là $ q^3 $.
	}
\end{vd}
\begin{vd}[VDT]%[DCHT Toán 11 - KNTT -Tên Huỳnh Thanh Chí]%[1K2K7-1]
	Cho dãy số $ (u_n) $ được xác định bởi $ \heva{& u_1=2\\ & u_{n+1}=4u_n+9},\forall n\in\mathbb{N}^* $. Chứng minh rằng dãy số $ (v_n) $ xác định bởi $ v_n=u_n+3,\forall n\in\mathbb{N}^* $ là một cấp số nhân. Hãy xác định số hạng đầu và công bội của cấp số nhân đó.
	\dapso{$ (v_n) $ là cấp số nhân với công bội $ q=4 $ và số hạng đầu $ v_1=5 $.}
	\loigiai{
	Ta có $ v_n=u_n+3 $ (1) và $ v_{n+1}=u_{n+1}+3 $ (2).\\
	Theo đề ta có $ u_{n+1}=4u_n+9 \Rightarrow u_{n+1}+3=4(u_n+3) $ (3).\\
	Thay (1) và (2)	vào (3) ta được $ v_{n+1}=4v_n \Rightarrow \dfrac{v_{n+1}}{v_n}=4,\forall n\in\mathbb{N}^* $.\\
	Suy ra $ (v_n) $ là cấp số nhân với công bội $ q=4 $ và số hàng đầu $ v_1=u_1+3=2+3=5 $.
	}
\end{vd}
\subsubsection{Bài tập tự luận}
 
\begin{bt}%[NB]%[DCHT Toán 11 - KNTT -Tên Huỳnh Thanh Chí]%[1K2Y7-1]
	Dãy số $25$; $5$; $1$; $\dfrac{1}{5}$; $\ldots$ có phải là một cấp số nhân không? Nếu có hãy tìm công bội của cấp số nhân đó.
	\dapso{Dãy số $25$; $5$; $1$; $\dfrac{1}{5}$; $\ldots$ là một cấp số nhân với công bội $ q=\dfrac{1}{5} $.}
	\loigiai{
	Ta có $ \dfrac{u_2}{u_1}=\dfrac{u_3}{u_2}=\ldots=\dfrac{1}{5} $ là một số không đổi.\\
	Do đó dãy số $25$; $5$; $1$; $\dfrac{1}{5}$; $\ldots$ là một cấp số nhân với công bội $ q=\dfrac{1}{5} $.
	}
\end{bt}
\begin{bt}%[NB]%[DCHT Toán 11 - KNTT -Tên Huỳnh Thanh Chí]%[1K2B7-1]
	Dãy số $1$; $n$; $n^2$; $n^3$; $ n^4 $; $\ldots$ (với $ n>1 $) có phải là một cấp số nhân không? Nếu có hãy tìm công bội của cấp số nhân đó.
	\dapso{Dãy số $1$; $n$; $n^2$; $n^3$; $ n^4 $; $\ldots$ (với $ n>1 $) là một cấp số nhân với công bội $ q=n $.}
	\loigiai{
		Ta có $ \dfrac{u_2}{u_1}=\dfrac{u_3}{u_2}=\ldots=n $ (với $ n>1 $) là một số không đổi.\\
		Do đó dãy số $1$; $n$; $n^2$; $n^3$; $ n^4 $; $\ldots$ (với $ n>1 $) là một cấp số nhân với công bội $ q=n $.
	}
\end{bt}
\begin{bt}%[TH]%[DCHT Toán 11 - KNTT -Tên Huỳnh Thanh Chí]%[1K2B7-1]
	Cho dãy số $ (u_n) $ được xác định bởi $ \heva{& u_1=2\\ & u_{n+1}=u_n^2} $. Hỏi dãy số $ (u_n) $ có là một cấp số nhân hay không?
	\dapso{Dãy số $ (u_n) $ không là một cấp số nhân.}
	\loigiai{
	Ta có $ u_2=u_1^2=4,u_3=u_2^2=16,u_4=u_3^2=256 $.\\
	Suy ra $ \dfrac{u_2}{u_1}=2 $; $ \dfrac{u_3}{u_2}=4 $ và $ \dfrac{u_4}{u_3}=16 $. Vì $ \dfrac{u_2}{u_1}\ne \dfrac{u_3}{u_2}\ne \dfrac{u_4}{u_3} $ nên $ (u_n) $ không là một cấp số nhân	
	}
\end{bt}
\begin{bt}%[TH]%[DCHT Toán 11 - KNTT -Tên Huỳnh Thanh Chí]%[1K2B7-1]
	Cho dãy số $ (u_n) $, biết $ u_1=2 $ và $ u_{n+1}=\dfrac{1}{3}u_n $. Chứng minh $ (u_n) $ là một cấp số nhân và tìm số hạng $ u_3 $.
	\dapso{}
	\loigiai{
	Ta có $ u_{n+1}=\dfrac{1}{3}u_n\Rightarrow \dfrac{u_{n+1}}{u_n}=\dfrac{1}{3} $ là một số không đổi nên $ (u_n) $ là một cấp số nhân với công bội là $ q=\dfrac{1}{3} $.\\
	Do đó $ u_3=u_2\cdot q=u_1\cdot q^2=2\cdot \dfrac{1}{3^2}=\dfrac{2}{9} $.
	}
\end{bt}
\begin{bt}%[TH]%[DCHT Toán 11 - KNTT -Tên Huỳnh Thanh Chí]%[1K2B7-1]
	Cho $ (u_n) $ là cấp số nhân có công bội $ q\ne 0,u_1\ne 0 $. Chứng minh rằng dãy số $ (v_n) $ với $ v_n=\dfrac{u_nu_{2n+1}}{4} $ cũng là một cấp số nhân.
	\dapso{$ (v_n) $ là một cấp số nhân với công bội là $ q^3 $.}
	\loigiai{
		Ta có $ \dfrac{v_n}{v_{n-1}}=\dfrac{\dfrac{u_nu_{2n+1}}{4}}{\dfrac{u_{n-1}u_{2(n-1)+1}}{4}}=\dfrac{u_1q^{n-1}\cdot u_1q^{2n}}{u_1q^{n-2}\cdot u_1q^{2n-2}}=q^3 $. Do đó $ (v_n) $ là một cấp số nhân với công bội là $ q^3 $.}
\end{bt}
\begin{bt}%[VD]%[DCHT Toán 11 - KNTT -Tên Huỳnh Thanh Chí]%[1K2K7-1]
	Cho dãy số $ (u_n) $ được xác định bởi $ \heva{& u_1=3\\ & u_{n+1}=2u_n-2},\forall n\in\mathbb{N}^* $. Chứng minh rằng dãy số $ (v_n) $ xác định bởi $ v_n=2u_n-4,\forall n\in\mathbb{N}^* $ là một cấp số nhân. Hãy xác định số hạng đầu và công bội của cấp số nhân đó.
	\dapso{$ (v_n) $ là cấp số nhân với công bội $ q=2 $ và số hạng đầu $ v_1=2 $.}
	\loigiai{
	Ta có $ v_n=2u_n-4 $ (1) và $ v_{n+1}=2u_{n+1}-4 $ (2).\\
	Theo đề ta có $ u_{n+1}=2u_n-2 \Rightarrow 2u_{n+1}-4=2(2u_n-4) $ (3).\\
	Thay (1) và (2)	vào (3) ta được $ v_{n+1}=2v_n \Rightarrow \dfrac{v_{n+1}}{v_n}=2,\forall n\in\mathbb{N}^* $.\\
	Suy ra $ (v_n) $ là cấp số nhân với công bội $ q=2 $ và số hàng đầu $ v_1=2u_1-4=2\cdot 3-4=2 $.
	}
\end{bt}
\subsubsection{Câu hỏi trắc nghiệm}
\Opensolutionfile{ans}[ans/ans-1K2-3-dang1]
\begin{ex}%[DCHT Toán 11 - KNTT -Tên Huỳnh Thanh Chí]%[1K2Y7-1]
	Trong các dãy số sau, dãy số nào là một cấp số nhân?
	\choice
	{\True $128$; $-64$; $32$; $-16$; $8$; $\ldots$} 
	{$\sqrt{2}$; $2$; $4$; $4\sqrt{2}$; $\ldots$}
	{$5$; $6$; $7$; $8$; $\ldots$}
	{$15$; $5$; $1$; $\dfrac{1}{5}$; $\ldots$}
	\loigiai{
	Xét phương án $128$; $-64$; $32$; $-16$; $8$; $\ldots$. \\
	Có $ \dfrac{u_2}{u_1}=\dfrac{u_3}{u_2}=\ldots=-\dfrac{1}{2} $ là một số không đổi nên dãy số $128$; $-64$; $32$; $-16$; $8$; $\ldots$ là một cấp số nhân.
	}
\end{ex}
\begin{ex}%[DCHT Toán 11 - KNTT -Tên Huỳnh Thanh Chí]%[1K2Y7-1]
	Dãy số nào sau đây \textbf{không phải} là cấp số nhân?
	\choice
	{$1$; $-1$; $1$; $-1$; $\ldots$}
	{$3$; $3^2$; $3^3$; $3^4$; $\ldots$}
	{$a$; $a^3$; $a^5$; $a^7$; $\ldots$  $(a\not =0)$}
	{\True $\dfrac{1}{\pi}$; $\dfrac{1}{{\pi}^2}$; $\dfrac{1}{{\pi}^4}$; $\dfrac{1}{{\pi}^6}$; $\ldots$}
	\loigiai{
	Xét dãy $\dfrac{1}{\pi}$; $\dfrac{1}{{\pi}^2}$; $\dfrac{1}{{\pi}^4}$; $\dfrac{1}{{\pi}^6}$; $\ldots$ có $ \dfrac{u_2}{u_1}\ne \dfrac{u_3}{u_2} \left(\dfrac{1}{\pi}\ne\dfrac{1}{\pi^2} \right) $.\\
	Do đó dãy $\dfrac{1}{\pi}$; $\dfrac{1}{{\pi}^2}$; $\dfrac{1}{{\pi}^4}$; $\dfrac{1}{{\pi}^6}$; $\ldots$ không là một cấp số nhân.
	}
\end{ex}
\begin{ex}%[DCHT Toán 11 - KNTT -Tên Huỳnh Thanh Chí]%[1K2Y7-1]
	Dãy số $1$; $2$; $4$; $8$; $16$; $32$; $\ldots$ là một cấp số nhân với 
	\choice
	{Công bội là $1$ và số hạng đầu tiên là $2$}
	{\True Công bội là $2$ và số hạng đầu tiên là $1$}
	{Công bội là $2$ và số hạng đầu tiên là $2$}
	{Công bội là $1$ và số hạng đầu tiên là $1$}
	\loigiai{
	Ta có $ q=\dfrac{u_2}{u_1}=\dfrac{u_3}{u_2}=\ldots=2 $. \\
	Vậy dãy số đã cho là một cấp số nhân với công bội là $ q=2 $ và số hạng đầu tiên là $ u_1=1 $.
	}
\end{ex}
\begin{ex}%[DCHT Toán 11 - KNTT -Tên Huỳnh Thanh Chí]%[1K2Y7-1]
	Cho cấp số nhân $(u_n)$ với $u_1=-2$ và công bội $q=-5$. Viết bốn số hạng đầu tiên của cấp số nhân.
	\choice
	{$-2$; $10$; $50$; $-250$}
	{\True $-2$; $10$; $-50$; $250$}
	{$-2$; $-10$; $-50$; $-250$}
	{$-2$; $10$; $50$; $250$}
	\loigiai{
	Vì $ (u_n) $ là một cấp số nhân nên ta có $ u_{n+1}=u_nq $. \\
	Do đó $ u_2=u_1q=(-2)\cdot (-5)=10 $, $ u_3=u_2q=10\cdot (-5)=-50 $, $ u_4=u_3q=(-50)\cdot (-5)=250 $.\\
	Vậy bốn số hạng đầu tiên của cấp số nhân đó là $ -2; 10; -50; 250 $.
	}
\end{ex}
\begin{ex}%[DCHT Toán 11 - KNTT -Tên Huỳnh Thanh Chí]%[1K2B7-1]
	Một cấp số nhân có hai số hạng liên tiếp là $3$ và $12$. Số hạng tiếp theo của cấp số nhân là
	\choice
	{$15$}
	{$21$}
	{$36$}
	{\True $48$}
	\loigiai{
		Một cấp số nhân có hai số hạng liên tiếp là $3$ và $12$, do đó ta có $ q=\dfrac{u_{n+1}}{u_n}=\dfrac{12}{3}=4 $.\\
		Vậy số hạng tiếp theo của cấp số nhân đó là $ u_{n+2}=u_{n+1}q=12\cdot 4=48 $.
	}
\end{ex}
\begin{ex}%[DCHT Toán 11 - KNTT -Tên Huỳnh Thanh Chí]%[1K2B7-1]
	Cho cấp số nhân $(u_n)$ có số hạng tổng quát là $u_n=\dfrac{3}{2}\cdot 5^n$. Khi đó số hạng đầu $u_1$ và công bội $q$ là
	\choice
	{$u_1=\dfrac{3}{2}, q=\dfrac{1}{5}$} 
	{$u_1=\dfrac{3}{2}, q=5$}
	{$u_1=\dfrac{15}{2}, q=\dfrac{1}{5}$}
	{\True $u_1=\dfrac{15}{2}, q=5$}
	\loigiai{	
	Ta có $ u_1=\dfrac{3}{2}\cdot5^1=\dfrac{15}{2}$ và $ u_2=\dfrac{3}{2}\cdot 5^2=\dfrac{75}{2} $.\\
	Vì $ (u_n) $ là một cấp số nhân nên $ q=\dfrac{u_2}{u_1}=\dfrac{75}{2}:\dfrac{15}{2}=5 $.
	}
\end{ex}
\begin{ex}%[DCHT Toán 11 - KNTT -Tên Huỳnh Thanh Chí]%[1K2B7-1]
	Trong các dãy số $(u_n)$ cho bởi số hạng tổng quát $u_n$ sau, dãy số nào là một cấp số nhân?
	\choice
	{\True $u_n=\dfrac{1}{3^{n-2}}$}
	{$u_n=\dfrac{n}{3^n}$}
	{$u_n=(n+2)\cdot 3^n$}
	{$u_n=n^2$}
	\loigiai{
		\begin{itemize}
			\item Với $u_n=\dfrac{1}{3^{n-2}}$, ta có $ q=\dfrac{u_{n+1}}{u_n}=\dfrac{1}{3^{n-3}}:\dfrac{1}{3^{n-2}}=3 $ là một số không đổi.\\
			Vậy dãy số $ (u_n) $ có số hạng tổng quát $u_n=\dfrac{1}{3^{n-2}}$ là một cấp số nhân.
			\item Với $u_n=\dfrac{n}{3^n}$, ta có $ q=\dfrac{u_{n+1}}{u_n}=\dfrac{n+1}{3^{n+1}}:\dfrac{n}{3^n}=\dfrac{n+1}{3n} $ không phải là một số không đổi.\\
			Vậy dãy số $ (u_n) $ có số hạng tổng quát $u_n=\dfrac{n}{3^n}$ không là một cấp số nhân.
			\item Với $u_n=(n+2)\cdot 3^n$, ta có $ q=\dfrac{u_{n+1}}{u_n}=\dfrac{(n+3)\cdot 3^{n+1}}{(n+2)\cdot 3^n}=\dfrac{3(n+3)}{n+2} $ không phải là một số không đổi.\\
			Vậy dãy số $ (u_n) $ có số hạng tổng quát $u_n=(n+2)\cdot 3^n$ không là một cấp số nhân.
			\item Với $u_n=n^2$, ta có $ q=\dfrac{u_{n+1}}{u_n}=\dfrac{(n+1)^2}{n^2}=\left(1+\dfrac{1}{n}\right)^2 $ không là một số không đổi.\\
			Vậy dãy số $ (u_n) $ có số hạng tổng quát $u_n=n^2$ không là một cấp số nhân.
		\end{itemize}
	}
\end{ex}
\begin{ex}%[DCHT Toán 11 - KNTT -Tên Huỳnh Thanh Chí]%[1K2B7-1]
	Trong các dãy số $(u_n)$ cho bởi số hạng tổng quát $u_n$ sau, dãy số nào là một cấp số nhân?
	\choice
	{$u_n=7-3n$}
	{$u_n=7-3^n$}
	{$u_n=\dfrac{7}{3n}$}
	{\True $u_n=7\cdot 3^n$}
	\loigiai{
	\begin{itemize}
		\item Với $u_n=7-3n$, ta có $ q=\dfrac{u_{n+1}}{u_n}=\dfrac{7-3(n+1)}{7-3n}=\dfrac{4-3n}{7-3n} $ không phải là một số không đổi.\\
		Vậy dãy số $ (u_n) $ có số hạng tổng quát $u_n=7-3n$ không là một cấp số nhân.
		\item Với $u_n=7-3^n$, ta có $ q=\dfrac{u_{n+1}}{u_n}=\dfrac{7-3^{n+1}}{7-3^n}=\dfrac{7-3\cdot 3^n}{7-3^n}=1-\dfrac{2\cdot 3^n}{7-3^n} $ không phải là một số không đổi.\\
		Vậy dãy số $ (u_n) $ có số hạng tổng quát $u_n=7-3^n$ không là một cấp số nhân.
		\item Với $u_n=\dfrac{7}{3n}$, ta có $ q=\dfrac{u_{n+1}}{u_n}=\dfrac{7}{3(n+1)}:\dfrac{7}{3n}=\dfrac{n}{n+1} $ không phải là một số không đổi.\\
		Vậy dãy số $ (u_n) $ có số hạng tổng quát $u_n=\dfrac{7}{3n}$ không là một cấp số nhân.
		\item Với $u_n=7\cdot 3^n$, ta có $ q=\dfrac{u_{n+1}}{u_n}=\dfrac{7\cdot 3^{n+1}}{7\cdot 3^n}=3 $ là một số không đổi.\\
		Vậy dãy số $ (u_n) $ có số hạng tổng quát $u_n=7\cdot 3^n$ là một cấp số nhân.
	\end{itemize}
	}
\end{ex}
\begin{ex}%[DCHT Toán 11 - KNTT -Tên Huỳnh Thanh Chí]%[1K2B7-1]
	Mệnh đề nào sau đây \textbf{sai}?
	\choice
	{Dãy số có tất cả các số hạng bằng nhau là một cấp số nhân}
	{Dãy số có tất cả các số hạng bằng nhau là một cấp số cộng}
	{Một cấp số cộng có công sai dương là một dãy số tăng}
	{\True Một cấp số nhân có công bội $q>1$ là một dãy tăng}
	\loigiai{
	\begin{itemize}
		\item Dãy số có tất cả các số hạng bằng nhau là một cấp số nhân là mệnh đề đúng. \\
		Vì xét dãy số $ (u_n) $ là một cấp số nhân.
		Khi đó $ u_{n+1}=u_n\cdot q $ với $ u_n\ne 0,q=1 $ thì $ u_{n+1}=u_n $.
		\item Dãy số có tất cả các số hạng bằng nhau là một cấp số cộng là mệnh đề đúng.\\
		Vì $ u_{n+1}=u_n+d $, với $ d=0 $ thì $ u_{n+1}=u_n $.
		\item Một cấp số cộng có công sai dương là một dãy số tăng là mệnh đề đúng.\\
		Ta xét dãy $ (u_n) $ là một cấp số cộng có công sai $ d>0 $.\\
		Vì $ u_{n+1}=u_n+d \Rightarrow u_{n+1}-u_n=d>0 $. \\
		Do đó dãy $ (u_n) $ là dãy số tăng.
		\item Một cấp số nhân có công bội $q>1$ là một dãy tăng là mệnh đề \textbf{sai}.\\
		Ta xét dãy số $ (u_n) $ là một cấp số nhân có công bội $ q>1 $.\\
		Vì $ u_{n+1}=u_nq $ với $ u_n\ne 0,q>1 $. Khi đó $ u_{n+1}-u_n=u_nq-u_n=u_n(q-1) $.\\
		Nếu $ u_n<0 $ thì $ u_{n+1}-u_n=u_nq-u_n=u_n(q-1) <0 $. \\
		Do đó dãy $ (u_n) $ là dãy số giảm.
	\end{itemize}	
	}
\end{ex}
\begin{ex}%[DCHT Toán 11 - KNTT -Tên Huỳnh Thanh Chí]%[1K2K7-1]
	Cho dãy số $(u_n)$ được xác định bởi $ u_1=2,u_n=2u_{n-1}+3n-1 $. Công thức số hạng tổng quát của dãy số đã cho là biểu thức có dạng $ a2^n+bn+c $, với $ a,b,c\in\mathbb{Z},n\ge 2, n\in\mathbb{N} $. Khi đó tổng $ a+b+c $ có giá trị bằng
	\choice
	{$ -4 $}
	{$ 4 $}
	{\True $ -3 $}
	{$ 3 $}
	\loigiai{
	Ta có $ u_n=2u_{n-1}+3n-1 \Leftrightarrow u_n+3n+5=2\left[u_{n-1}+3(n-1)+5\right] $ với $ n\ge 2, n\in\mathbb{N} $.\\
	Đặt $ v_n=u_n+3n+5 $, ta có $ v_n=2v_{n-1} $ với $ n\ge 2, n\in\mathbb{N} $.\\
	Như vậy $ (v_n) $ là cấp số nhân với công bội $ q=2 $ và $ v_1=10 $.\\
	Do đó $ v_n=10\cdot 2^{n-1}=5\cdot 2^n $.\\
	Suy ra $ u_n+3n+5=5\cdot 2^n $ hay $ u_n=5\cdot 2^n-3n-5 $ với $ n\ge 2, n\in\mathbb{N} $.\\
	Vậy $ a=5,b=-3,c=-5 $, suy ra $ a+b+c=-3 $.
	}
\end{ex}
\Closesolutionfile{ans}
% \begin{indapan}{10}
% 	{ans/ans-1K2-3-dang1}
% \end{indapan}
\begin{dang}{Số hạng tổng quát của cấp số nhân}
	Dựa vào giả thuyết, ta lập một hệ phương trình chứa công bội $ q $ và số hạng đầu $ u_n $. Giải hệ phương trình này tìm được $ u_1 $ và $ q $.\\
	Nếu cấp số nhân $ (u_n) $ có số hạng đầu $ u_1 $ và công bội $ q $ thì số hạng tổng quát $ u_n $ được xác định bởi công thức $$ u_n=u_1\cdot q^{n-1} \text{ với } n\ge 2. $$
\end{dang}
\subsubsection{Ví dụ minh hoạ}
\begin{vd}%[NB]%[DCHT Toán 11 - KNTT -Tên Huỳnh Thanh Chí]%[1K2Y7-2]
	Tìm số hạng tổng quát của dãy số $ 2;4;8;16;32;\ldots $, biết dãy $ (u_n) $ là một cấp số nhân.  
	\dapso{$ u_n=2\cdot 2^{n-1} $.}
	\loigiai{
		Vì dãy số $ (u_n) $ là một cấp số nhân nên $ q=\dfrac{u_2}{u_1}=\dfrac{u_3}{u_2}=\ldots=2 $ và số hạng đầu $ u_1=2 $.\\
		Do đó dãy số $ 2;4;8;16;32;\ldots $ là một cấp số nhân có số hạng tổng quát là $ u_n=u_1q^{n-1}=2\cdot 2^{n-1} $.
	}
\end{vd}
\begin{vd}%[TH]%[DCHT Toán 11 - KNTT -Tên Huỳnh Thanh Chí]%[1K2B7-2]
	Tìm số hạng đầu, công bội và số hạng tổng quát của cấp số nhân, biết $ \heva{& u_1+u_5=51\\ & u_2+u_6=102.} $
	\dapso{$ u_1=3 $, $ q=2 $ và $ u_n=3\cdot 2^{n-1} $.}
	\loigiai{
	Vì $ (u_n) $ là một cấp số nhân nên $ u_n=u_1\cdot q^{n-1} $.\\
	Ta có $ \heva{& u_1+u_5=51\\ & u_2+u_6=102}\Leftrightarrow\heva{& u_1+u_1q^4=51\\ & u_1q+u_1q^5=102}\Leftrightarrow\heva{& u_1(1+q^4)=51 \qquad (1)\\ & u_1q(1+q^4)=102 \qquad (2).} $\\
	Chia từng vế của $ (2) $ cho $ (1) $ ta được $ \dfrac{u_1q(1+q^4)}{u_1(1+q^4)}=\dfrac{102}{51} \Leftrightarrow q=2 $.\\
	Suy ra $ u_1=\dfrac{51}{1+q^4}=\dfrac{51}{17}=3 $, $ u_n=u_1\cdot q^{n-1}=3\cdot 2^{n-1} $.\\
	Vậy $ u_1=3 $, $ q=2 $ và $ u_n=3\cdot 2^{n-1} $.
	}
\end{vd}
\begin{vd}%[TH]%[DCHT Toán 11 - KNTT -Tên Huỳnh Thanh Chí]%[1K2B7-2]
	Tìm số hạng đầu, công bội và số hạng tổng quát của cấp số nhân, biết $ \heva{& u_1+u_6=30\\ & u_2+u_7=120.} $
	\dapso{$ u_1=\dfrac{6}{205} $, $ q=4 $ và $ u_n=\dfrac{6}{205}\cdot 4^{n-1} $.}
	\loigiai{
		Vì $ (u_n) $ là một cấp số nhân nên $ u_n=u_1\cdot q^{n-1} $.\\
		Ta có $ \heva{& u_1+u_6=30\\ & u_2+u_7=120}\Leftrightarrow\heva{& u_1+u_1q^5=30\\ & u_1q+u_1q^6=102}\Leftrightarrow\heva{& u_1(1+q^5)=30 \qquad (1)\\ & u_1q(1+q^5)=120 \qquad (2).} $\\
		Chia từng vế của $ (2) $ cho $ (1) $ ta được $ \dfrac{u_1q(1+q^5)}{u_1(1+q^5)}=\dfrac{120}{30} \Leftrightarrow q=4 $.\\
		Suy ra $ u_1=\dfrac{30}{1+q^5}=\dfrac{30}{1+4^5}=\dfrac{6}{205} $, $ u_n=u_1\cdot q^{n-1}=\dfrac{6}{205}\cdot 4^{n-1} $.\\
		Vậy $ u_1=\dfrac{6}{205} $, $ q=4 $ và $ u_n=\dfrac{6}{205}\cdot 4^{n-1} $.
	}
\end{vd}
\begin{vd}%[TH]%[DCHT Toán 11 - KNTT -Tên Huỳnh Thanh Chí]%[1K2B7-2]
	Tìm số hạng đầu, công bội và số hạng tổng quát của cấp số nhân, biết $ \heva{& u_3=40\\ & u_6=160.} $
	\dapso{$ u_1=\dfrac{40}{9} $, $ q=3 $ và $ u_n=40\cdot 3^{n-3} $.}
	\loigiai{
	Vì $ (u_n) $ là một cấp số nhân nên $ u_n=u_1\cdot q^{n-1} $.\\	
	Ta có $ \heva{& u_3=40\\ & u_6=1080}\Leftrightarrow \heva{& u_1q^2=40 \qquad (1)\\ & u_1q^5=1080\qquad (2).} $\\
	Chia từng vế của $ (2) $ cho $ (1) $ ta được $ \dfrac{u_1q^5}{u_1q^2}=\dfrac{1080}{40} \Leftrightarrow q^3=27 \Leftrightarrow q=3 $.\\
	Suy ra $ u_1=\dfrac{40}{q^2}=\dfrac{40}{3^2}=\dfrac{40}{9} $, $ u_n=u_1\cdot q^{n-1}=\dfrac{40}{9}\cdot 3^{n-1}=40\cdot 3^{n-3} $.\\
	Vậy $ u_1=\dfrac{40}{9} $, $ q=3 $ và $ u_n=40\cdot 3^{n-3} $.
	}
\end{vd}
\begin{vd}[VDT]%[DCHT Toán 11 - KNTT -Tên Huỳnh Thanh Chí]%[1K2K7-2]
	Tìm số hạng đầu, công bội và số hạng tổng quát của cấp số nhân có công bội $ q\in \mathbb{Z},q\ne 0 $, biết $ \heva{& u_2+u_4=10\\ & u_1+u_3+u_5=-21.} $
	\dapso{$ u_1=-1 $, $ q=-2 $ và $ u_n=\dfrac{(-2)^n}{2} $.}
	\loigiai{
		Vì $ (u_n) $ là một cấp số nhân nên $ u_n=u_1\cdot q^{n-1} $ với $ q\in \mathbb{Z},q\ne 0 $.\\
		Ta có $ \heva{& u_2+u_4=10\\ & u_1+u_3+u_5=-21}\Leftrightarrow\heva{& u_1q+u_1q^3=10\\ & u_1q+u_1q^2+u_1q^4=-21}\Leftrightarrow\heva{& u_1(q+q^3)=10 \qquad (1)\\ & u_1(1+q^2+q^4)=-21 \qquad (2).} $\\
		Chia từng vế của $ (2) $ cho $ (1) $ ta được 
		\allowdisplaybreaks
		\begin{eqnarray*}
			 \dfrac{u_1(1+q^2+q^4)}{u_1(q+q^3)}=\dfrac{-21}{10} 
			 &\Leftrightarrow& 10q^4+21q^3+10q^2+21q+10=0 \\
			 &\Leftrightarrow& (q+2)(2q+1)(5q^2-2q+5)=0 \\
			 &\Leftrightarrow& \hoac{& q=-2 \text{ (thỏa mãn)}\\ & q=-\dfrac{1}{2} \text{ (loại)}.}
		\end{eqnarray*}
		Suy ra $ u_1=\dfrac{10}{q+q^3}=-1 $, $ u_n=u_1\cdot q^{n-1}=(-1)\cdot (-2)^{n-1}=-(-2)^{n-1}=-\dfrac{(-2)^n}{-2}=\dfrac{(-2)^n}{2} $.\\
		Vậy $ u_1=-1 $, $ q=-2 $ và $ u_n=\dfrac{(-2)^n}{2} $.
	}
\end{vd}
\subsubsection{Bài tập tự luận}
 
% \begin{bt}%[NB]%[DCHT Toán 11 - KNTT -Tên Huỳnh Thanh Chí]%[1K2Y7-2]
% 	Tìm số hạng thứ $ 100 $ của cấp số nhân $ 8;-4;2;-1;\ldots $
% 	\dapso{$ u_{100}=-\dfrac{1}{2^{96}} $.}
% 	\loigiai{
% 	Cấp số nhân này có số hạng đầu $ u_1=8 $ và công bội $ q=\dfrac{-4}{8}=-\dfrac{1}{2} $.\\
% 	Do đó số hạng tổng quát $ u_n=8\cdot \left(-\dfrac{1}{2}\right)^{n-1} $.\\
% 	Vậy $ u_{100}=8\cdot \left(-\dfrac{1}{2}\right)^{100-1}=8\cdot \left(-\dfrac{1}{2}\right)^{99}=-\dfrac{1}{2^{96}} $.
% 	}
% \end{bt}
\begin{bt}%[NB]%[DCHT Toán 11 - KNTT -Tên Huỳnh Thanh Chí]%[1K2B7-2]
	Tìm số hạng tổng quát của dãy số $ 3;12;48;192;\ldots $, biết dãy $ (u_n) $ là một cấp số nhân.  
	\dapso{$ u_n=3\cdot 4^{n-1} $.}
	\loigiai{
		Vì dãy số $ (u_n) $ là một cấp số nhân nên $ q=\dfrac{u_2}{u_1}=\dfrac{12}{3}=4 $ và số hạng đầu $ u_1=3 $.\\
		Do đó dãy số $ 3;12;48;192;\ldots $ là một cấp số nhân có số hạng tổng quát là $ u_n=u_1q^{n-1}=3\cdot 4^{n-1} $.}
\end{bt}
\begin{bt}%[TH]%[DCHT Toán 11 - KNTT -Tên Huỳnh Thanh Chí]%[1K2B7-2]
	Tìm số hạng tổng quát của cấp số nhân, biết $ \heva{& u_1+u_3=51\\ & u_2+u_4=153.} $
	\dapso{$ u_n=\dfrac{51}{10}\cdot 3^{n-1} $.}
	\loigiai{
		Vì $ (u_n) $ là một cấp số nhân nên $ u_n=u_1\cdot q^{n-1} $.\\
		Ta có $ \heva{& u_1+u_3=51\\ & u_2+u_4=153}\Leftrightarrow\heva{& u_1+u_1q^2=51\\ & u_1q+u_1q^3=153}\Leftrightarrow\heva{& u_1(1+q^2)=51 \qquad (1)\\ & u_1q(1+q^2)=153 \qquad (2).} $\\
		Chia từng vế của $ (2) $ cho $ (1) $ ta được $ \dfrac{u_1q(1+q^2)}{u_1(1+q^2)}=\dfrac{153}{51} \Leftrightarrow q=3 $.\\
		Suy ra $ u_1=\dfrac{51}{1+q^2}=\dfrac{51}{10} $, $ u_n=u_1\cdot q^{n-1}=\dfrac{51}{10}\cdot 3^{n-1} $.\\
		Vậy số hạng tổng quát $ u_n=\dfrac{51}{10}\cdot 3^{n-1} $.
	}
\end{bt}
\begin{bt}%[TH]%[DCHT Toán 11 - KNTT -Tên Huỳnh Thanh Chí]%[1K2B7-2]
	Tìm số hạng đầu, công bội và số hạng tổng quát của cấp số nhân, biết $ \heva{& u_3=15\\ & u_6=120.} $
	\dapso{$ u_1=\dfrac{15}{4} $, $ q=2 $ và $ u_n=15\cdot 3^{n-3} $.}
	\loigiai{
		Vì $ (u_n) $ là một cấp số nhân nên $ u_n=u_1\cdot q^{n-1} $.\\	
		Ta có $ \heva{& u_3=15\\ & u_6=120}\Leftrightarrow \heva{& u_1q^2=15 \qquad (1)\\ & u_1q^5=120\qquad (2).} $\\
		Chia từng vế của $ (2) $ cho $ (1) $ ta được $ \dfrac{u_1q^5}{u_1q^2}=\dfrac{120}{15} \Leftrightarrow q^3=8 \Leftrightarrow q=2 $.\\
		Suy ra $ u_1=\dfrac{15}{q^2}=\dfrac{15}{2^2}=\dfrac{15}{4} $, $ u_n=u_1\cdot q^{n-1}=\dfrac{15}{4}\cdot 2^{n-1}=15\cdot 2^{n-3} $.\\
		Vậy $ u_1=\dfrac{15}{4} $, $ q=2 $ và $ u_n=15\cdot 3^{n-3} $.
	}
\end{bt}
\begin{bt}%[TH]%[DCHT Toán 11 - KNTT -Tên Huỳnh Thanh Chí]%[1K2B7-2]
	Tìm số hạng tổng quát của cấp số nhân, biết $ \heva{& u_4=35\\ & u_8=560.} $
	\dapso{$ u_n=35\cdot 2^{n-4} $ với $ q=2 $ hoặc $ u_n=35\cdot (-2)^{n-4} $ với $ q=-2 $.}
	\loigiai{
		Vì $ (u_n) $ là một cấp số nhân nên $ u_n=u_1\cdot q^{n-1} $.\\	
		Ta có $ \heva{& u_4=35\\ & u_8=560}\Leftrightarrow \heva{& u_1q^3=35 \qquad (1)\\ & u_1q^7=560\qquad (2).} $\\
		Chia từng vế của $ (2) $ cho $ (1) $ ta được $ \dfrac{u_1q^7}{u_1q^3}=\dfrac{560}{35} \Leftrightarrow q^4=16 \Leftrightarrow \hoac{& q=2\\ & q=-2.} $\\
		Với $ q=2 $. Suy ra $ u_1=\dfrac{35}{q^3}=\dfrac{35}{8} $, $ u_n=u_1\cdot q^{n-1}=\dfrac{35}{8}\cdot 2^{n-1}=35\cdot 2^{n-4} $.\\
		Với $ q=-2 $. Suy ra $ u_1=\dfrac{35}{q^3}=-\dfrac{35}{8} $, $ u_n=u_1\cdot q^{n-1}=-\dfrac{35}{8}\cdot (-2)^{n-1}=35\cdot (-2)^{n-4} $.\\
		Vậy $ u_n=35\cdot 2^{n-4} $ với $ q=2 $ hoặc $ u_n=35\cdot (-2)^{n-4} $ với $ q=-2 $.
	}
\end{bt}

\subsubsection{Câu hỏi trắc nghiệm}
\Opensolutionfile{ans}[ans/ans-1K2-3-dang2]
\begin{ex}%[DCHT Toán 11 - KNTT -Tên Huỳnh Thanh Chí]%[1K2Y7-2]
	Cho cấp số nhân $(u_n)$ có số hạng đầu là $u_1\ne 0$ và công bội $q\ne 0$. Số hạng tổng quát của cấp số nhân bằng
	\choice
	{$ u_{n}=u_1+(n-1)q $}
	{\True $ u_{n}=u_1\cdot q^{n-1} $}
	{$ u_{n}=u_1\cdot q^n $}
	{$ u_{n}=u_1\cdot q^{n+1} $}
	\loigiai{
	Số hạng tổng quát của cấp số nhân là $ u_{n}=u_1\cdot q^{n-1} $.
	}
\end{ex}
\begin{ex}%[DCHT Toán 11 - KNTT -Tên Huỳnh Thanh Chí]%[1K2Y7-2]
	Cấp số nhân  $\left(u_{n}\right)$ có $u_{n}=\dfrac{3}{5}\cdot 2^{n}$. Số hạng đầu tiên và công bội $ q $ là
	\choice
	{$u_1=\dfrac{6}{5},q=3$}
	{$u_1=\dfrac{6}{5},q=-2$}
	{\True $u_1=\dfrac{6}{5},q=2$}
	{$u_1=\dfrac{6}{5},q=5$}
	\loigiai{
	Ta có $u_{n}=\dfrac{3}{5}\cdot 2^{n}=\dfrac{6}{5}\cdot 2^{n-1}$, suy ra $ u_1=\dfrac{6}{5} $ và $ q=2 $.
	}
\end{ex}
\begin{ex}%[DCHT Toán 11 - KNTT -Tên Huỳnh Thanh Chí]%[1K2Y7-2]
	Cho cấp số nhân $(u_n)$ có $u_1=-3$ và công bội $q=\dfrac{2}{3}$. Chọn mệnh đề đúng?
	\choice
	{$u_5=-\dfrac{27}{16}$}
	{$u_5=-\dfrac{16}{27}$}
	{\True $u_5=\dfrac{16}{27}$}
	{$u_5=\dfrac{27}{16}$}
	\loigiai{
	Số hạng tổng quát của cấp số nhân là $ u_{n}=u_1\cdot q^{n-1}=3\cdot\left(\dfrac{2}{3}\right)^{n-1} $.\\
	Vậy $ u_5=3\cdot \left(\dfrac{2}{3}\right)^{5-1}=\dfrac{16}{27} $.
	}
\end{ex}
\begin{ex}%[DCHT Toán 11 - KNTT -Tên Huỳnh Thanh Chí]%[1K2Y7-2]
	Dãy số có số hạng tổng quát $u_{n}=\dfrac{1}{\sqrt{3}}^{2n}$ là một cấp số nhân có công bội $ q $ bằng
	\choice
	{$ \dfrac{1}{\sqrt{3}} $}
	{$ \sqrt{3} $}
	{$ \dfrac{1}{9} $}
	{\True $ \dfrac{1}{3} $}
	\loigiai{
		Ta có $u_{n}=\dfrac{1}{\sqrt{3}}^{2n}=\left[\left(\dfrac{1}{\sqrt{3}}\right)^2\right]^n=\left(\dfrac{1}{3}\right)^n=\dfrac{1}{3}\cdot\left(\dfrac{1}{3}\right)^{n-1} $.\\
		Suy ra công bội của cấp số nhân $ q=\dfrac{1}{3} $.
	}
\end{ex}
\begin{ex}%[DCHT Toán 11 - KNTT -Tên Huỳnh Thanh Chí]%[1K2Y7-2]
	Cho cấp số nhân $(u_n)$ có $u_1=1, u_2=-2$. Mệnh đề nào sau đây đúng?
	\choice
	{\True $u_{2024}=-2^{2023}$}
	{$u_{2024}=2^{2023}$}
	{$u_{2024}=-2^{2024}$}
	{$u_{2024}=2^{2024}$}
	\loigiai{
	Số hạng tổng quát của cấp số nhân là $ u_{n}=u_1\cdot q^{n-1}=\left(-2\right)^{n-1} $.\\
	Vậy $ u_{2024}=\left(-2\right)^{2024-1}=(-2)^{2023}=-2^{2023} $.
	}
\end{ex}
\begin{ex}%[DCHT Toán 11 - KNTT -Tên Huỳnh Thanh Chí]%[1K2B7-2]
	Cho cấp số nhân có $\heva{& u_4-u_2=54\\ & u_5-u_3=108}$. Số hạng đầu tiên $u_1$ và công bội $ q $ của cấp số nhân là
	\choice
	{\True $ u_1=9 $ và $ q=2 $ }
	{$ u_1=9 $ và $ q=-2 $}
	{$ u_1=-9 $ và $ q=2 $}
	{$ u_1=-9 $ và $ q=-2 $}
	\loigiai{
	Ta có $\heva{& u_4-u_2=54\\ & u_5-u_3=108}\Leftrightarrow \heva{& u_1q^3-u_1q=54\\ & u_1q^4-u_1q^2=108}\Leftrightarrow\heva{& u_1q(q^2-1)-54 \quad (1)\\ & u_1q^2(q^2-1)=108 \quad (2).} $\\
	Chia từng vế của $ (2) $ cho $ (1) $ ta được $ \dfrac{u_1q^2(q^2-1)}{u_1q(q^2-1)}=\dfrac{108}{54} \Leftrightarrow q=2 $.\\
	Suy ra $ u_1=\dfrac{54}{q^3-q}=\dfrac{54}{2^3-2}=9 $.
	}
\end{ex}
\begin{ex}%[DCHT Toán 11 - KNTT -Tên Huỳnh Thanh Chí]%[1K2B7-2]
	Cho cấp số nhân $\left( u_n\right)$ biết $\heva{& u_1+ u_2+ u_3=31\\& u_1+ u_3=26 }$. Giá trị $u_1$ và $ q $ là
	\choice
	{$ u_1=2; q=5 $ hoặc $u_1=25; q=\dfrac{1}{5}$}
	{$ u_1=5; q=1 $ hoặc $u_1=25; q=\dfrac{1}{5}$}
	{$ u_1=25; q=5 $ hoặc $u_1=1; q=\dfrac{1}{5}$}
	{\True$ u_1=1; q=5 $ hoặc $u_1=25; q=\dfrac{1}{5}$}
	\loigiai{
	Vì $ (u_n) $ là một cấp số nhân nên $ u_n=u_1\cdot q^{n-1} $.\\	
	Ta có $ \heva{& u_1+ u_2+ u_3=31\\& u_1+ u_3=26 }\Leftrightarrow\heva{& u_2=5\\ & u_1+u_3=26}\Leftrightarrow\heva{& u_1q=5 \quad (1)\\ & u_1(1+q^2)=26\quad (2).} $\\
	Chia từng vế của $ (2) $ cho $ (1) $ ta được $ \dfrac{q^2+1}{q}=\dfrac{26}{5} \Leftrightarrow 5q^2-26q+5=0 \Leftrightarrow \hoac{& q=5\\ & q=\dfrac{1}{5}.} $\\
	Với $ q=5 $. Suy ra $ u_1=\dfrac{5}{q}=\dfrac{5}{5}=1 $.\\
	Với $ q=\dfrac{1}{5} $. Suy ra $ u_1=\dfrac{5}{q}=5:\dfrac{1}{5}=25 $.\\
	Vậy $ u_1=1 $ với $ q=5 $ hoặc $ u_1=25 $ với $ q=\dfrac{1}{5} $.
	}
\end{ex}
\begin{ex}%[DCHT Toán 11 - KNTT -Tên Huỳnh Thanh Chí]%[1K2B7-2]
	Số hạng đầu tiên và công bội của cấp số nhân thỏa mãn $\heva{& u_5+u_2=36\\ & u_6-u_4=48}$ (với $ q>0 $) là
	\choice
	{$ u_1=4,q=4 $}
	{$ u_1=2,q=4 $}
	{\True$ u_1=2,q=2 $}
	{$ u_1=4,q=2 $}
	\loigiai{
	Ta có $\heva{& u_5+u_2=36\\ & u_6-u_4=48}\Leftrightarrow\heva{& u_1q^4+u_1q=36\\ & u_1q^5-u_1q^3=48}\Leftrightarrow\heva{& u_1q(q^3+1)=36\quad (1)\\ & u_1q(q^4-q^2)=48\quad (2).} $\\
	 Chia từng vế của $ (2) $ cho $ (1) $ ta được $$ \dfrac{u_1q(q^4-q^2)}{u_1q(q^3+1)}=\dfrac{48}{36}\Leftrightarrow \dfrac{q^4-q^2}{q^3+1}=\dfrac{4}{3}\Leftrightarrow 3q^4-4q^3-3q^2-4=0\hoac{& q=2\\ &q=-1.} $$
	 Từ điều kiện $ q>0 $ suy ra công bội của cấp số nhân là $ q=2 $, do đó $ u_1=\dfrac{36}{q^4+q}=2 $.\\
	 Vậy $ u_1=2 $ và $ q=2 $.
	}
\end{ex}
\begin{ex}%[DCHT Toán 11 - KNTT -Tên Huỳnh Thanh Chí]%[1K2B7-2]
	Cho cấp số nhân $u_2=\dfrac{1}{4},u_5=16$. Công bội và số hạng đầu tiên của cấp số nhân là
	\choice
	{$q=\dfrac{1}{2};u_1=\dfrac{1}{2}$}
	{$q=\dfrac{-1}{2};u_1=\dfrac{-1}{2}$}
	{\True $q=4;u_1=\dfrac{1}{16}$}
	{$q=-4;u_1=\dfrac{-1}{16}$}
	\loigiai{
	Ta có $ u_2=u_1q=\dfrac{1}{4} $ (1) và $ u_5=u_1q^4=16 $ (2).\\
	Lấy $ (2) $ chia cho $ (1) $ vế theo vế ta được $ \dfrac{u_1q^4}{u_1q}=\dfrac{16}{\tfrac{1}{4}} \Leftrightarrow q^3=64 \Leftrightarrow q=4 $.\\
	Suy ra $ u_1=\dfrac{1}{4}:q=\dfrac{1}{4}:4=\dfrac{1}{16} $.\\
	Vậy $ u_1=\dfrac{1}{16},q=4 $.
	}
\end{ex}
\begin{ex}%[DCHT Toán 11 - KNTT -Tên Huỳnh Thanh Chí]%[1K2K7-2]
	Người ta thiết kế một cái tháp gồm $ 11 $ tầng. Diện tích mặt trên của mỗi tầng bằng nửa diện tích mặt trên của tầng ngay bên dưới và diện tích mặt trên của tầng 1 bằng nửa diện tích của đế tháp (có diện tích là $12\ 288$ m$ ^2 $). Diện tích mặt trên cùng (của tầng thứ $ 11 $) có giá trị nào sau đây?
	\choice
	{\True$ 6 $ m$ ^2 $}
	{$ 8 $ m$ ^2 $}
	{$ 10 $ m$ ^2 $}
	{$ 12 $ m$ ^2 $}
	\loigiai{
	Vì diện tích của mặt trên của mỗi tầng bằng nửa diện tích mặt trên của tầng ngay bên dưới và diện tích mặt trên của tầng 1 bằng nửa diện tích của đế tháp.\\
	Do đó diện tích của mỗi tầng tạo nên dãy số và dãy số đó là một cấp số nhân có công bội $ q=\dfrac{1}{2} $.\\
	Vậy số hạng tổng quát của cấp số nhân đó là $ u_n=12\ 288\cdot \left(\dfrac{1}{2}\right)^{n-1} $.\\
	Vì từ đế tháp đến tầng thứ 11 của tháp sẽ có 12 mặt nền, do đó diện tích của mặt của tầng thứ 11 là $ u_{12}=12\ 288\cdot\left(\dfrac{1}{2}\right)^{12-1}=6 $ m$ ^2 $.
	}
\end{ex}
\Closesolutionfile{ans}
% \begin{indapan}{10}
% 	{ans/ans-1K2-3-dang2}
% \end{indapan}
\begin{dang}{Tìm số hạng cụ thể của CSN}
	Ta chuyển các số hạng của CSN về số hạng đầu $u_1$ và công bội $q$. Sử dụng công thức $u_n=u_1\cdot q^{n-1}$. \\
	Chia hai phương trình vế theo vế ta thu được phương trình theo $q$. \\
	Giải tìm $q$ và $u_1$. Từ đó tìm được số hạng cần tìm thỏa ycbt.
\end{dang}
\subsubsection{Ví dụ minh hoạ}
\begin{vd}%[NB]%[1K2Y3-3]
	Cho $u_n$ là CSN thỏa $u_1=2$; $u_4=16$. Tìm số hạng thứ $5$ của CSN.
	\loigiai{
		Do $u_n$ là CSN nên ta có $u_4=u_1\cdot q^3 \Rightarrow q^3=\dfrac{u_4}{u_1}=8 \Rightarrow q=2$. \\
		Vậy $u_5=u_1\cdot q^4=2\cdot 2^4=32$.
	}
\end{vd}
\begin{vd}%[TH]%[1K2B3-3]
	Cho cấp số nhân $(u_n)$ có $\heva{&u_4+u_6=-540 \\ &u_3+u_5=180}$. Tính số hạng đầu $u_1$ và công bội $q$ của cấp Số nhân.
	\loigiai{
		Ta có $\heva{&u_4 + u_6=-540 \\ &u_3+u_5=180}
		\Leftrightarrow \heva{&u_1q^3(1+q^2)=-540 \\ &u_1q^2(1+q^2)=180}
		\Leftrightarrow \heva{&u_1=2 \\ &q=-3.}$ \\
		Vậy $\heva{&u_1=2 \\ &q=-3}$ là số hạng cần tìm.
	}
\end{vd}
\begin{vd}%[TH]%[1K2B3-3]
	Cho cấp số nhân có $u_1=-3$, $q=\dfrac{2}{3}$. Số $\dfrac{-96}{243}$ là số hạng thứ mấy của cấp số nhân?
	\loigiai{
		Giả sử số $\dfrac{-96}{243}$ là số hạng thứ $n$ của cấp số nhân.\\
		Ta có: $u_1\cdot q^{n-1}=\dfrac{-96}{243}\Leftrightarrow(-3)\left(\dfrac{2}{3}\right)^{n-1}=\dfrac{-96}{243}\Leftrightarrow n=6$.\\
		Vậy số $\dfrac{-96}{243}$ là số hạng thứ $6$ của cấp số nhân.}
\end{vd}
\begin{vd}%[TH]%[1K2B3-3]
	Cấp số nhân $\left(u_{n}\right)$ có số hạng tổng quát là $u_n=\dfrac{3}{5} \cdot 2^{n-1}, n \in \mathbb{N}^*$. Số hạng đầu tiên và công bội của cấp số nhân đó là
	\loigiai{
		Ta có $u_{1}=\dfrac{3}{5} \cdot 2^{1-1}=\dfrac{3}{5}$ và $u_{2}=\dfrac{3}{5} \cdot 2^{2-1}=\dfrac{6}{5} \Rightarrow q=\dfrac{u_{2}}{u_{1}}=2$.\\
		Vậy $u_{1}=\dfrac{3}{5}$ và $q=2$.
	}
\end{vd}

\subsubsection{Bài tập tự luận}
 
\begin{bt}%[TH]%[1K2B3-3]
	Cho cấp số nhân $(u_n)$ biết $\heva{&u_4-u_2=25 \\ &u_3-u_1=50.}$
	\begin{enumEX}{1}
		\item Tìm số hạng đầu và công bội của cấp số nhân $(u_n)$.
		\item Tìm số hạng thứ $8$ của cấp số nhân $(u_n)$.
	\end{enumEX}
	\dapso{$\heva{&q=\dfrac{1}{2} \\ &u_1=-200}$, $u_8=-\dfrac{25}{16}$}
	\loigiai{
		\begin{enumerate}
			\item Ta có $\heva{&u_4-u_2=25 \\ &u_3-u_1=50} 
			\Leftrightarrow \heva{&u_1(q^3-q)=25 \\ &u_1(q^2-q)=50} 
			\Rightarrow \heva{&q=\dfrac{1}{2} \\ &u_1=-200.}$
			\item Ta có $u_8=u_1\cdot q^7=-200\cdot \dfrac{1}{2^7}=-\dfrac{25}{16}$.
		\end{enumerate}
	}
\end{bt}
\begin{bt}%[TH]%[1K2B3-3]
	Tìm số hạng thứ $10$ của cấp số nhân $(u_n)$ biết $\heva{&u_4-u_2=72 \\ &u_5-u_3=144.}$	
	\dapso{$u_{10}=6144$}
	\loigiai{
		Ta có $\heva{&u_4-u_2=72 \\ &u_5-u_3=144}\Leftrightarrow \heva{&u_4-u_2=72 \\ &q(u_4-u_2)=144}
		\Rightarrow \heva{&q=2 \\ &u_1(q^3-q)=72} \Leftrightarrow \heva{&q=2 \\ &u_1=12.}$ \\
		Khi đó $u_{10}=u_1\cdot q^9=6144$.
	}
\end{bt}
\begin{bt}%[TH]%[1K2B3-3]
	Cho một cấp số nhân có $5$ số hạng biết $2$ số hạng đầu là số dương, tích số hạng đầu và số hạng thứ $3$ là $1$, tích số hạng thứ $3$ và số hạng cuối là $\dfrac{1}{16}$. Tìm cấp số nhân này.	
	\dapso{$2; 1; \dfrac{1}{2}; \dfrac{1}{4}; \dfrac{1}{8}$}
	\loigiai{
		Gọi $5$ số hạng cần tìm có dạng $\dfrac{x}{q^2}$; $\dfrac{x}{q}$; $x$; $xq$; $xq^2$.\\
		Theo đề ra ta có $\heva{&\dfrac{x}{q^2}\cdot x=1 \\ &x\cdot xq^2=\dfrac{1}{16}} 
		\Leftrightarrow \heva{&x=\dfrac{1}{2} \\ &q=\dfrac{1}{2}}$ (do hai số hạng đầu dương nên $q>0$). \\
		Vậy $5$ số hạng cần tìm là $2; 1; \dfrac{1}{2}; \dfrac{1}{4}; \dfrac{1}{8}$.
	}
\end{bt}
\begin{bt}%[TH]%[1K2B3-3]
	Tìm số hạng đầu và công bội của cấp số nhân $(u_n)$ biết $\heva{&u_2+u_5-u_4=10 \\ &u_3+u_6-u_5=20.}$
	\dapso{$\heva{&q=2 \\ &u_1=1}$}
	\loigiai{
		Ta có $\heva{&u_2+u_5-u_4=10 \\ &u_3+u_6-u_5=20} 
		\Leftrightarrow \heva{&u_1(q+q^4-q^3)=10 \\ &u_1(q^2+q^5-q^4)=20}
		\Leftrightarrow \heva{&q=2 \\ &u_1=1.}$
	}
\end{bt}
\begin{bt}%[TH]%[1K2B3-4]
	Tìm $5$ số lập thành một cấp số nhân có công bội bằng $\dfrac{1}{4}$ số thứ nhất và tổng $2$ số đầu là $\dfrac{5}{4}$.
	\dapso{$1; \dfrac{1}{4}; \dfrac{1}{16}; \dfrac{1}{64}; \dfrac{1}{128}$ hoặc $-5; -\dfrac{5}{4}; -\dfrac{5}{16}; -\dfrac{5}{64}; -\dfrac{1}{128}$}
	\loigiai{
		Theo đề, ta có $\heva{&q=\dfrac{1}{4}u_1 \\ &u_1+u_2=\dfrac{5}{4}}
		\Leftrightarrow \heva{&q=\dfrac{1}{4}u_1 \\ &u_1+u_1\cdot q=\dfrac{5}{4}}
		\Leftrightarrow \heva{&q=\dfrac{1}{4}u_1 \\ &u_1^2+4u_1-5=0}
		\Leftrightarrow \heva{&q=\dfrac{1}{4} \\ &u_1=1}$ hoặc $\heva{&q=-\dfrac{5}{4} \\ &u_1=-5.}$\\
		Vậy có hai CSN là $1; \dfrac{1}{4}; \dfrac{1}{16}; \dfrac{1}{64}; \dfrac{1}{128}$ và $-5; -\dfrac{5}{4}; -\dfrac{5}{16}; -\dfrac{5}{64}; -\dfrac{1}{128}$.
	}
\end{bt}
\begin{bt}%[TH]%[1K2B3-4]
	Tìm $3$ số lập thành một cấp số nhân có tổng là $63$ và tích là $1728$.
	\dapso{$3; 12; 48$}
	\loigiai{
		Gọi ba số cần tìm là $\dfrac{x}{q}; x; xq$. Theo đề ra, ta có $x^3=1728\Rightarrow x=12$. \\
		Mặt khác $\dfrac{x}{q}+x+xq=63\Leftrightarrow 12q+12+\dfrac{12}{q}=63
		\Leftrightarrow 12q^2-51q+12=0 \Leftrightarrow \hoac{&q=4 \\ &q=\dfrac{1}{4}\cdot}$ \\
		Vậy CSN cần tìm là $3; 12; 48$.
	}
\end{bt}
\subsubsection{Câu hỏi trắc nghiệm}
\Opensolutionfile{ans}[ans/ans-1K2-3-dang3]
\begin{ex}%[1K2B3-3]
	Cho cấp số nhân $(u_n)$ có $u_{20}=8u_{17}$. Công bội của cấp số nhân là
	\choice
	{\True $q=2$}
	{$q=-2$}
	{$q=4$}
	{$q=-4$}
	\loigiai{
		Ta có $u_{20}=8u_{17}\Rightarrow u_1\cdot q^{19}=8\cdot u_1\cdot q^{16}\Rightarrow  q=2$.
	}
\end{ex}
\begin{ex}%[1K2B3-3]
	Cho cấp số nhân $\left(u_n\right)$ có $10$ số hạng với công bội $q\neq 0$ và $u_1\neq 0$. Đẳng thức nào sau đây là đúng?
	\choice
	{$u_7=u_4\cdot q^6$}
	{\True $u_7=u_4\cdot q^3$}
	{$u_7=u_4\cdot q^4$}
	{$u_7=u_4\cdot q^5$}
	\loigiai
	{Ta có $u_7=u_1\cdot q^6=\left(u_1\cdot q^3\right)\cdot q^3=u_4\cdot q^3$.
	}
\end{ex}
\begin{ex}%[1K2B3-3]
	Cho cấp số nhân $(u_n)$ có số hạng đầu $u_1=2$ và công bội $q=3$. Giá trị $u_{2019}$ bằng
	\choice
	{$3\cdot2^{2019}$}
	{$2\cdot3^{2019}$}
	{$3\cdot2^{2018}$}
	{\True $2\cdot3^{2018}$}
	\loigiai{
		Áp dụng công thức của số hạng tổng quát $u_n=u_1\cdot q^{n-1}=2\cdot 3^{2018}$.}
\end{ex}
\begin{ex}%[1K2B3-3]
	Cho cấp số nhân $(u_n)$ với công bội $q < 0$ và $u_2=4$, $u_4=9$. Tìm $u_1$.
	\choice
	{$u_1=6$}
	{\True $u_1=-\dfrac{8}{3}$}
	{$u_1=-6$}
	{$u_1=\dfrac{8}{3}$}
	\loigiai{
		Vì $q<0$, $u_2>0$ nên $u_3<0$. Do đó $u_3=-\sqrt{u_2\cdot u_4}=-\sqrt{4\cdot 9}=-6$.\\
		Ta có $u_2^2=u_1\cdot u_3\Rightarrow u_1=\dfrac{u_2^2}{u_3}=\dfrac{4^2}{-6}=-\dfrac{8}{3}$.
	}
\end{ex}
\begin{ex}%[1K2B3-3]
	Cho cấp số nhân $\left(u_{n}\right)$ có $u_{2}=-6, u_{3}=3$. Công bội $q$ của cấp số nhân đã cho bằng
	\choice
	{$2 $}
	{$\dfrac{1}{2}$}
	{\True $-\dfrac{1}{2}$}
	{$-2$}
	\loigiai{
		Công bội của cấp số nhân đã cho là $$q=\dfrac{u_3}{u_2}=-\dfrac{1}{2}.$$}
\end{ex}
\begin{ex}%[1K2Y3-3]
	Cho cấp số nhân có $u_1=-3$, $q=\dfrac{2}{3}$. Tính $u_5$?
	\choice
	{$u_5=\dfrac{27}{16}$}
	{\True $u_5=\dfrac{-16}{27}$}
	{$u_5=\dfrac{-27}{16}$}
	{$u_5=\dfrac{16}{27}$}
	\loigiai{
		Ta có: $u_5=u_1\cdot q^4=(-3)\left(\dfrac{2}{3}\right)^4=-\dfrac{16}{27}$.}
\end{ex}
\begin{ex}%[1K2Y3-3]
	Cho cấp số nhân $(u_n)$ có $u_2=\dfrac{1}{4}$; $u_5=-16$. Tìm $q$ và số hạng đầu tiên của cấp số nhân?
	\choice
	{$q=\dfrac{1}{2};u_1=\dfrac{1}{2}$}
	{$q=-\dfrac{1}{2},u_1=-\dfrac{1}{2}$}
	{\True $q=-4,u_1=\dfrac{1}{16}$}
	{$q=-4,u_1=-\dfrac{1}{16}$}
	\loigiai{
		Ta có $\heva{&u_2=\dfrac{1}{4}\\&u_5=16}\Rightarrow \heva{&u_1\cdot q=\dfrac{1}{4}\\&u_1\cdot q^4=-16}\Rightarrow q^3=-64\Rightarrow q=-4 \Rightarrow u_1=\dfrac{1}{16}$.
	}
\end{ex}
\begin{ex}%[1K2B3-3]
	Cho cấp số nhân $(u_n)$, biết: $u_n=81,u_{n+1}=9$. Lựa chọn đáp án đúng.
	\choice
	{$q=-\dfrac{1}{9}$}
	{\True $q=\dfrac{1}{9}$}
	{$q=9$}
	{$q=-9$}
	\loigiai{
		Ta có  $q=\dfrac{u_{n+1}}{u_n}=\dfrac{9}{81}=\dfrac{1}{9}$.
	}
\end{ex}
\begin{ex}%[1K2Y3-3]
	Cho cấp số nhân $\left( {u_n} \right)$ với $u_1=2$ và công bội $q=3$. Số hạng $u_2$ bằng
	\choice
	{$8$}
	{\True $6$}
	{$12$}
	{$18$}
	\loigiai{
		Ta có $u_2=u_1\cdot q=2\cdot 3=6$.}
\end{ex}
\begin{ex}%[1K2Y3-3]
	Cho cấp số nhân $(u_n)$ với $u_1=2$ và $u_3=8$. Số hạng thứ hai của cấp số nhân đã cho bằng
	\choice
	{$u_2=4$}
	{$u_2=6$}
	{\True $u_2=\pm 4$}
	{$u_2=-4$}
	\loigiai{
		Ta có $u_1 \cdot u_3=u_2^2 \Leftrightarrow u^2_2=16 \Leftrightarrow \hoac{&u_2=4\\&u_2=-4.}$
	}
\end{ex}
\begin{ex}%[1K2B3-3]
	Cho cấp số nhân $(u_n)$ có $u_1=-1; q=\dfrac{-1}{10}$. Số $\dfrac{1}{10^{103}}$ là số hạng thứ bao nhiêu?
	\choice
	{số hạng thứ $103$}
	{số hạng thứ $105$}
	{\True số hạng thứ $104$}
	{Đáp án khác}
	\loigiai{
		Ta có $u_n=u_1\cdot q^{n-1}\Leftrightarrow	\dfrac{1}{10^{103}}=-1\cdot \left(\dfrac{-1}{10}\right)^{n-1} \Leftrightarrow \left(\dfrac{-1}{10}\right)^{n-1}=\left(\dfrac{-1}{10}\right)^{103}\Rightarrow n=104$.
	}
\end{ex}
\begin{ex}%[1K2Y3-3]
	Cho cấp số nhân $\left(u_n\right)$ có các số hạng lần lượt là $3$, $9$, $27$, $81$,\ldots Khi đó $u_n$ bằng
	\choice
	{$3+3^n$}
	{$3^{n-1}$}
	{$3^{n+1}$}
	{\True $3^n$}
	\loigiai
	{Cấp số nhân đã cho có $u_1=3$ và công bội $q=3$ nên $u_n=u_1\cdot q^{n-1}=3\cdot 3^{n-1}=3^n$.
	}
\end{ex}
\begin{ex}%[1K2K3-3]
	Cho cấp số nhân $(u_n)$ có $u_1=3$ và $15u_1-4u_2+u_3$ đạt giá trị nhỏ nhất. Tìm số hạng thứ $13$ của cấp số nhân đã cho.
	\choice
	{\True $u_{13}=12288$}
	{$u_{13}=3072$}
	{$u_{13}=24567$}
	{$u_{13}=49152$}
	\loigiai{
		Gọi $q$ là công bội của cấp số nhân $(u_n)$.\\
		Ta có $15u_1-4u_2+u_3=45-12q+3q^2=3(q-2)^2+33\geq 33$ $\forall q \in \mathbb{R}$.\\
		Suy ra $15u_1-4u_2+u_3$ đạt giá trị nhỏ nhất khi $q=2$.\\
		Khi đó $u_{13}=u_1q^{12}=12288$.}
\end{ex}
\begin{ex}%[1K2B3-3]
	Cho cấp số nhân $(u_n)$ biết $u_1+u_5=51$ và $u_2+u_6=102$. Hỏi số $12288$ là số hạng thứ mấy của cấp số nhân $(u_n)$?
	\choice
	{\True Số hạng thứ $13$}
	{Số hạng thứ $10$}
	{Số hạng thứ $11$}
	{Số hạng thứ $12$}
	\loigiai{
		Gọi $q$ là công bội của cấp số nhân đã cho. Theo đề bài, ta có\\
		\centerline{$\heva{&u_1+u_5=51\\&u_2+u_6=102}\Leftrightarrow\heva{&u_1\left(1+q^4\right)=51\\&u_1q\left(1+q^4\right)=102}\Rightarrow q=2\Rightarrow u_1=3\Rightarrow u_n=3\cdot 2^{n-1.}$}
		Mặt khác $u_n=12288\Leftrightarrow 3\cdot 2^{n-1}=12288\Leftrightarrow 2^{n-1}=2^{12}\Leftrightarrow n=13$.
	}
\end{ex}
% \begin{ex}%[1K2K3-3]
% 	Một tứ giác lồi có số đo các góc lập thành một cấp số nhân. Biết rằng số đo của góc nhỏ nhất bằng $\dfrac{1}{9}$ số đo của góc nhỏ thứ ba. Hãy tính số đo của các góc trong tứ giác đó.
% 	\choice
% 	{$5^{\circ}$, $15^{\circ}$, $45^{\circ}$, $225^{\circ}$}
% 	{\True $9^{\circ}$, $27^{\circ}$, $81^{\circ}$, $243^{\circ}$}
% 	{$7^{\circ}$, $21^{\circ}$, $63^{\circ}$, $269^{\circ}$}
% 	{$8^{\circ}$, $32^{\circ}$, $72^{\circ}$, $248^{\circ}$}
% 	\loigiai{
% 		Gọi các góc của tứ giác là $a$, $aq$, $aq^2$, $aq^3,$ trong đó $q>1$.\\
% 		Theo giả thiết, ta có $a=\dfrac{1}{9}aq^2$ nên $q=3$.\\
% 		Suy ra các góc của tứ giác là $a$, $3a$, $9a$, $27a$.\\
% 		Vì tổng các góc trong tứ giác bằng $360^{\circ}$ nên ta có $a+3a+9a+27a=360^{\circ}\Leftrightarrow a=9^{\circ}$.\\
% 		Vậy số đo các góc trong tứ giác lần lượt là $9^{\circ}$, $27^{\circ}$, $81^{\circ}$, $243^{\circ}$. }
% \end{ex}
\Closesolutionfile{ans}
% \begin{indapan}{10}
% 	{ans/ans-1K2-3-dang3}
% \end{indapan}
\begin{dang}{Tìm điều kiện để một dãy số lập thành CSN}
	Dãy số $a, b, c$ lập thành CSN khi $b^2=a\cdot c$. \\
	Dãy số $a, b, c, d$ lập thành CSN khi $\heva{&b^2=a\cdot c \\ &c^2=b\cdot d.}$
\end{dang}
\subsubsection{Ví dụ minh hoạ}
\begin{vd}%[NB]%[1K2B3-4]
	Cho dãy $3,x,12,y$. Tìm $x,y$ để dãy là CSN.
	\loigiai{
		Dãy là CSN khi $\heva{&x^2=3\cdot 12 \\ &12^2=x\cdot y}\Leftrightarrow 
		\heva{&x=6 \\ &y=24}$ hoặc $\heva{&x=-6 \\ &y=-24.}$
	}
\end{vd}
\begin{vd}%[TH]%[1K2B3-4]
	Cho dãy  $x-1, 2x, 4x+3$. Tìm $x$ để dãy là CSN. 
	\loigiai{
		Dãy là CSN khi $(2x)^2=(x-1)(4x+3) \Leftrightarrow x=-3$.
	} 
\end{vd}
\begin{vd}%[VD]%[1K2K3-4]
	Các số $x+6y$, $5x+2y$, $8x+y$ theo thứ tự đó lập thành một cấp số cộng, đồng thời, các số $x+\dfrac{5}{3}$, $y-1$, $2x-3y$ theo thứ tự đó lập thành một cấp số nhân. Hãy tìm $x$ và $y$.
	\loigiai{
		\begin{itemize}
			\item Ba số $x+6y$, $5x+2y$, $8x+y$ lập thành cấp số cộng nên $(x+6y)+(8x+y)=2(5x+2y)\Leftrightarrow x=3y$.
			\item Ba số $x+\dfrac{5}{3}$, $y-1$, $2x-3y$ lập thành cấp số nhân nên $\left(x+\dfrac{5}{3}\right)(2x-3y)=\left(y-1\right)^2$.
		\end{itemize}
		Thay $x=3y$ vào ta được $8y^2+7y-1=0\Leftrightarrow y=-1$ hoặc $y=\dfrac{1}{8}$.\\
		Với $y=-1$ thì $x=-3$; với $y=\dfrac{1}{8}$ thì $x=\dfrac{3}{8}$.}
\end{vd}
\begin{vd}%[VD]%[1K2K3-4]
	Tìm tất cả các giá trị của tham số $m$ để phương trình sau có ba nghiệm phân biệt lập thành một cấp số nhân $x^3-7x^2+2\left(m^2+6m\right)x-8=0$.
	\loigiai{
		+ \textbf{Điều kiện cần:} \\
		Giả sử phương trình đã cho có ba nghiệm phân biệt $x_1$,$x_2$,$x_3$ lập thành một cấp số nhân.\\
		Theo định lý Vi-ét, ta có $x_1x_2x_3=8$.\\
		Theo tính chất của cấp số nhân, ta có $x_1x_3=x_2^2$. Suy ra  $x_2^3=8\Leftrightarrow x_2=2$.\\
		Với nghiệm $x=2$, ta có $m^2+6m-7=0\Leftrightarrow\hoac{&m=1\\&m=-7.}$ \\
		+ \textbf{Điều kiện đủ:}\\
		Với $m=1$ hoặc $m=-7$ thì $m^2+6m=7$.\\
		Khi đó phương trình ban đầu trở thành $x^3-7x^2+14x-8=0$.\\
		Giải phương trình này, ta được các nghiệm là $1$,$2$,$4$. Hiển nhiên ba nghiệm này lập thành một cấp số nhân với công bội $q=2$.\\
		Vậy $m=1$ và $m=-7$ là các giá trị cần tìm.}
\end{vd}
\begin{vd}%[VD]%[1K2K3-4]
	Các số $x+6y$, $5x+2y$, $8x+y$ theo thứ tự đó lập thành một cấp số cộng; đồng thời các số $x-1$, $y+2$, $x-3y$ theo thứ tự đó lập thành một cấp số nhân. Tính $x^2+y^2$.
	\loigiai{
		Theo giả thiết ta có
		\[
		\heva{&(x+6y)+(8x+y)=2(5x+2y)\\&(x-1)(x-3y)=\left(y+2\right)^2}\Leftrightarrow\heva{&x=3y\\&(3y-1)(3y-3y)=\left(y+2\right)^2}\Leftrightarrow \heva{&x=3y\\&0=\left(y+2\right)^2}\Leftrightarrow \heva{&x=-6\\&y=-2.}
		\]
		Vậy $x^2+y^2=40$.}
\end{vd}
\subsubsection{Bài tập tự luận}
 
\begin{bt}%[TH]%[1K2B3-4]
	Xác định $x$ dương để $2x-3$; $x$; $2x+3$ lập thành cấp số nhân.
	\dapso{$x=\sqrt{3}$}
	\loigiai{
		Ba số $2x-3$; $x$; $2x+3$ lập thành cấp số nhân khi $x^2=(2x-3)(2x+3) \Leftrightarrow x=\pm \sqrt{3}$.\\
		Do $x>0$ nên chọn $x=\sqrt{3}$.
	}
\end{bt}
\begin{bt}%[TH]%[1K2B3-4]
	Cho cấp số nhân $x, 12, y, 192$. Tìm $x$ và $y$.
	\dapso{$\heva{&x=-3 \\ &y=-48}$}	
	\loigiai{
		Bốn số $x, 12, y, 192$ lập thành CSN khi $\heva{&xy=12^2 \\ &y^2=12\cdot 192}
		\Leftrightarrow \heva{&x=3 \\ &y=48}$ hoặc $\heva{&x=-3 \\ &y=-48.}$
	}
\end{bt}
\begin{bt}%[TH]%[1K2B3-4]
	Tìm $x$ để dãy số $1$, $x^2$, $6-x^2$ lập thành cấp số nhân.
	\dapso{$x=\pm \sqrt{2}$}
	\loigiai{
		Ta có $1, x^2, 6-x^2$ lập thành cấp số nhân $\Leftrightarrow x^4=6-x^2 \Leftrightarrow x= \pm \sqrt{2}$.
	}
\end{bt}
\begin{bt}%[TH]%[1K2B3-4]
	Viết $6$ số xen giữa hai số $-2$ và $256$ để được một cấp số nhân có $8$ số hạng. Tìm cấp số nhân này.
	\dapso{$-2; 4; -8; 16; -32; 64; -128; 256$}
	\loigiai{
		Theo đề ra, ta có $\heva{&u_1=-2 \\ &u_8=256} \Leftrightarrow \heva{&u_1=-2 \\ &u_1\cdot q^7=256}
		\Leftrightarrow \heva{&u_1=-2 \\ &q=-2.}$ \\
		Cấp số nhân cần tìm là $-2; 4; -8; 16; -32; 64; -128; 256$.
	}
\end{bt}
\begin{bt}%[VD]%%[1K2K3-4]
	Bốn góc của một tứ giác lồi lập thành một cấp số nhân, góc lớn nhất gấp $8$ lần góc nhỏ nhất. Tìm $4$ góc đó.
	\dapso{$24^\circ; 48^\circ; 96^\circ; 192^\circ$}	
	\loigiai{
		Giả sử $4$ góc của tứ giác là $A\leq B\leq C\leq D$. Suy ra $A+B+C+D=360^\circ$.\\
		Theo đề, ta có $D=8A \Leftrightarrow Aq^3=8A \Leftrightarrow q=2$. Khi đó, ta được
		$$A(1+q+q^2+q^3)=360^\circ \Rightarrow A=24^\circ.$$
		Vậy $4$ góc của tứ giác lần lượt là $24^\circ; 48^\circ; 96^\circ; 192^\circ$.
	}
\end{bt}
\begin{bt}%[VD]%%[1K2K3-4]
	Tìm tất cả các giá trị của tham số $m$ để phương trình sau có ba nghiệm phân biệt lập thành một cấp số nhân $x^3-7mx^2+2(m^2+6 m)x-64=0$.
	\dapso{$m-8$}
	\loigiai{
		+ Điều kiện cần: \\
		Giả sử phương trình đã cho có ba nghiệm phân biệt $x_1; x_2; x_3$ lập thành một cấp số nhân. \\
		Theo định lý Vi-ét, ta có $x_1 \cdot x_2 \cdot x_3=64$. \\
		Theo tính chất của cấp số nhân, ta có $x_1\cdot x_3=x_2^2$. Suy ra ta có $x_2^3=64 \Leftrightarrow x_2=4$. \\
		Thay $x=4$ vào phương trình đã cho ta được 
		$$4^3-7m\cdot 4^2+2(m^2+6m) \cdot 4-64=0
		\Leftrightarrow m^2-8m=0
		\Leftrightarrow \hoac{&m=0 \\ &m=8.}$$
		+ Điều kiện đủ: \\
		Với $m=0$ thay vào phương trình đã cho ta được: $x^3-64=0$ hay $x=4$
		(nghiệm kép-loại). \\
		Với $m=8$ thay vào phương trình đã cho nên ta có phương trình $x^3-56x^2+224x-64=0$. \\
		Phương trình này có $3$ nghiệm phân biệt lập thành cấp số nhân. \\
		Vậy $m=8$ là giá trị cần tìm.
	}
\end{bt}
\subsubsection{Câu hỏi trắc nghiệm}
\Opensolutionfile{ans}[ans/ans-1K2-3-dang4]
% \begin{ex}%[1K2K3-4]
% 	Bốn góc của một tứ giác tạo thành cấp số nhân và góc lớn nhất gấp $27$ lần góc nhỏ nhất. Tổng của góc lớn nhất và góc bé nhất bằng
% 	\choice
% 	{$56^{\circ}$}
% 	{$102^{\circ}$}
% 	{$168^{\circ}$}
% 	{\True $252^{\circ}$}
% 	\loigiai{
% 		Giả sử 4 góc $A$, $B$, $C$, $D$ (với $A<B<C<D$) theo thứ tự đó lập thành cấp số nhân thỏa yêu cầu với công bội $q$.\\
% 		Theo giả thiết ta có
% 		\[\heva{&A+B+C+D=360\\&D=27A}\Leftrightarrow\heva{&A\left(1+q+q^2+q^3\right)=360\\&Aq^3=27A}\Leftrightarrow\heva{&q=3\\&A=9.}\]
% 		Suy ra $D=A\cdot q^3=9\cdot 3^3=243$.\\
% 		Vậy tổng số đo góc lớn nhất và góc bé nhất là $A+D=252^\circ$.
% 	}
% \end{ex}
\begin{ex}%[1K2B3-4]
	Xác định $x$ để $3$ số $2x-1$; $x$; $2x+1$ theo thứ tự lập thành một cấp số nhân:
	\choice
	{$x=\pm\sqrt{3}$}
	{$x=\pm\dfrac{1}{3}$}
	{\True $x=\pm\dfrac{1}{\sqrt{3}}$}
	{Không có giá trị nào của $x$}
	\loigiai{
		Ba số $2x-1$; $x$; $2x+1$ theo thứ tự lập thành cấp số nhân\\
		$\Leftrightarrow (2x-1)(2x+1)=x^2 \Leftrightarrow 3x^2=1 \Leftrightarrow x=\pm\dfrac{1}{\sqrt{3}}$.
	}
\end{ex}
\begin{ex}%[1K2K3-4]
	Cho $4$ số nguyên dương, trong đó $3$ số đầu lập thành cấp số cộng, $3$ số cuối lập thành cấp số nhân. Biết tổng số đầu và cuối là $37$, tổng $2$ số hạng giữa là $36$. Hỏi số lớn nhất thuộc khoảng nào sau đây?
	\choice
	{$\left(26;29\right)$}
	{\True $\left(24;26\right)$}
	{$\left(30;33\right)$}
	{$\left(22;25\right)$}
	\loigiai{
		Giả sử $4$ số đó là $a$, $b$, $c$, $d$ $\left(a,b,c,d\in\mathbb{N}^{\ast}\right)$. \\
		Do $a$, $b$, $c$ lập thành cấp số cộng nên ta có $a+c=2b$ $(1)$.\\
		Do $b$, $c$, $d$ lập thành cấp số nhân nên ta có $b \cdot d=c^2$ $(\ast)$. \\
		Theo giả thiết ta có $\heva{ & a+d=37 & & (2) \\ & b+c=36. & & (3)}$ \\
		Từ $(1)$, $(2)$, $(3)$ ta có $\heva{ & a=-d+37 \\ & b=\dfrac{-d+73}{3} \\ & c=\dfrac{d+35}{3}.}$ \\
		Thay vào $(\ast)$ ta có $\dfrac{-d+73}{3}\cdot d=\left(\dfrac{d+35}{3}\right)^2 \Leftrightarrow 4d^2-149d+1225=0 \Leftrightarrow \hoac{ & d=25 \\ & d=\dfrac{49}{4} & & \text{(loại)}.}$\\
		Với $d=25$, ta có $a=12$, $b=16$, $c=20$. \\
		Vậy số lớn nhất là $25\in\left(24;26\right)$.
	}
\end{ex}
\begin{ex}%[1K2K3-4]
	Ba số $x$, $y$, $z$ theo thứ tự lập thành một cấp số nhân với công bội $q$ khác $1$ đồng thời các số $x$, $2y$, $3z$ theo thứ tự lập thành một cấp số cộng với công sai khác $0$. Tìm giá trị của $q$.
	\choice
	{$q=-\dfrac{1}{3}$}
	{$q=\dfrac{1}{9}$}
	{$q=-3$}
	{\True $q=\dfrac{1}{3}$}
	\loigiai{
		Theo giả thiết ta có
		\[\heva{&y=xq \\ & z=xq^2\\&x+3z=2(2y)}\Rightarrow x+3xq^2=4xq \Rightarrow x\left(3q^2-4q+1\right)=0\Leftrightarrow \hoac{&x=0\\&3q^2-4q+1=0.}\]
		Nếu $x=0\Rightarrow y=z=0\Rightarrow$ công sai của cấp số cộng $x$, $2y$, $3z$ bằng 0 (vô lí).\\
		Nếu $3q^2-4q+1=0\Leftrightarrow\hoac{&q=1\\&q=\dfrac{1}{3}}\Leftrightarrow q=\dfrac{1}{3}$ vì $(q\not=1)$.}
\end{ex}
\begin{ex}%[1D3Y4-4]
	Trong các dãy số $(u_n)$ cho bởi số hạng tổng quát $u_n$ sau, dãy số nào là một cấp số nhân?
	\choice
	{$u_n=\dfrac{1}{3^n}-1$}
	{$u_n=n+\dfrac{1}{3}$}
	{$u_n=n^2-\dfrac{1}{3}$}
	{\True $u_n=\dfrac{1}{3^{n-2}}$}
	\loigiai{
		Từ các đáp án trên, với dãy $(u_n)$ cho bởi $u_n=\dfrac{1}{3^{n-2}}$ là một cấp số nhân, vì
		$$T=\dfrac{u_{n+1}}{u_n}=\dfrac{3^{n-2}}{3^{n-1}}=\dfrac{1}{3} \text{ (không đổi).}$$
	}
\end{ex}
\begin{ex}%[1K2B3-4]
	Trong các mệnh đề dưới đây, mệnh đề nào là \textbf{sai}?
	\choice
	{Dãy số $\left(a_n\right)$, với $a_1=3$ và $a_{n+1}=\sqrt{a_n+6}$, $\forall n\geq 1,$ vừa là cấp số cộng vừa là cấp số nhân}
	{\True Dãy số $\left(d_n\right)$, với $d_1=-3$ và $d_{n+1}=2d_n^2-15$, $\forall n\geq 1,$ vừa là cấp số cộng vừa là cấp số nhân}
	{Dãy số $\left(b_n\right)$, với $b_1=1$ và $b_{n+1}\left(2b_n^2+1\right)=3$, $\forall n\geq 1,$ vừa là cấp số cộng vừa là cấp số nhân}
	{Dãy số $\left(c_n\right)$, với $c_1=2$ và $c_{n+1}=3c_n^2-10$, $\forall n\geq 1,$ vừa là cấp số cộng vừa là cấp số nhân}
	\loigiai{
		Kiểm tra từng phương án ta có
		\begin{itemize}
			\item Ta có $a_2=3$, $a_2=3$, \ldots Bằng phương pháp quy nạp toán học chúng ra chứng minh được rằng $a_n=3$, $\forall n\geq 1$. Do đó $\left(a_n\right)$ là dãy số không đổi. Suy ra nó vừa là cấp số cộng (công sai bằng $0$) vừa là cấp số nhân (công bội bằng $1$).
			\item Tương tự như phương án trên, chúng ta chỉ ra được $b_n=1$, $\forall n\geq 1$. Do đó $\left(b_n\right)$ là dãy số không đổi. Suy ra nó vừa là cấp số cộng (công sai bằng $0$) vừa là cấp số nhân (công bội bằng $1$).
			\item Tương tự như phương án trên, chúng ta chỉ ra được $c_n=2$, $\forall n\geq 1$. Do đó $\left(c_n\right)$ là dãy số không đổi. Suy ra nó vừa là cấp số cộng (công sai bằng $0$) vừa là cấp số nhân (công bội bằng $1$).
			\item Ta có $d_1=-3$, $d_2=3$, $d_3=3$. Ba số hạng này không lập thành cấp số cộng cũng không lập thành cấp số nhân nên dãy số $\left(d_n\right)$ không phải là cấp số cộng và cũng không là cấp số nhân.
	\end{itemize}}
\end{ex}
\begin{ex}%[1K2K3-4]
	Biết rằng tồn tại hai giá trị $m_1$ và $m_2$ để phương trình \[2x^3+2\left(m^2+2m-1\right)x^2-7\left(m^2+2m-2\right)x-54=0\] có ba nghiệm phân biệt lập thành một cấp số nhân. Tính giá trị của biểu thức $P=m_1^3+m_2^3$.
	\choice
	{$P=56$}
	{$P=8$}
	{$P=-8$}
	{\True $P=-56$}
	\loigiai{
		Theo định lý Vi-ét, ta có $x_1\cdot x_2\cdot x_3 = -\dfrac{d}{a}=-\dfrac{-54}{2}=27 \Leftrightarrow x_2^3=27 \Leftrightarrow x_2=3$.\\
		Điều kiện cần để phương trình đã cho có ba nghiệm phân biệt lập thành một cấp số nhân là $x=3$ phải là nghiệm của phương trình đã cho. Suy ra
		\[m^2+2m-8=0\Leftrightarrow \hoac{&m=2 \\&m=-4.}\]
		Vì giả thiết cho biết tồn tại đúng hai giá trị của tham số $m$ nên $m=2$ và $m=-4$ là các giá trị thỏa mãn.\\
		Vậy $P=2^3+\left(-4\right)^3=-56$.
	}
\end{ex}
\begin{ex}%[1K2K3-4]
	Cho bốn số $a$, $b$, $c$, $d$ biết rằng $a$, $b$, $c$ theo thứ tự đó lập thành một cấp số nhân với công bội $q>1$; còn $b$, $c$, $d$ theo thứ tự đó lập thành cấp số cộng. Tìm $q$, biết rằng $a+d=14$ và $b+c=12$.
	\choice
	{$q=\dfrac{20+\sqrt{73}}{24}$}
	{\True $q=\dfrac{19+\sqrt{73}}{24}$}
	{$q=\dfrac{21+\sqrt{73}}{24}$}
	{$q=\dfrac{18+\sqrt{73}}{24}$}
	\loigiai{
		Giả sử $a$, $b$, $c$ lập thành cấp số cộng công bội $q$. Khi đó theo giả thiết ta có
		\[\heva{&b=aq,\;c=aq^2\\&b+d=2c\\&a+d=14 \\&b+c=12}\Rightarrow\heva{&aq+d=2aq^2&\quad(1)\\&a+d=14&(2)\\& a\left(q+q^2\right)=12.&(3)}\]
		\begin{itemize}
			\item Nếu $q=0\Rightarrow b=c=d=0$. (Vô lí!)
			\item Nếu $q=-1\Rightarrow b=-a$; $c=a\Rightarrow b+c=0$. (Vô lí!)
		\end{itemize}
		Vậy $q\not=0$, $q\not=-1$, từ $(2)$ và $(3)$ ta có $d=14-a$ và $a=\dfrac{12}{q+q^2}$. Thay vào $(1)$ ta được
		\[\begin{aligned}\dfrac{12q}{q+q^2}+\dfrac{14q^2+14q-12}{q+q^2}=\dfrac{24q^3}{q+q^2}
			&\Leftrightarrow 12q^3-7q^2-13q+6=0\\
			&\Leftrightarrow(q+1)\left(12q^2-19q+6\right)=0\\
			&\Leftrightarrow \hoac{ & q=-1 \quad \text{(loại)} \\ & q=\dfrac{19+\sqrt{73}}{24} \\ & q=\dfrac{19-\sqrt{73}}{24}.}
		\end{aligned}\]
		Vì $q>1$ nên $q=\dfrac{19+\sqrt{73}}{24}$.
	}
\end{ex}
\begin{ex}%[1K2K3-4]
	Cho dãy số tăng $a$, $b$, $c$ $\left(c\in\mathbb{Z}\right)$ theo thứ tự lập thành cấp số nhân; đồng thời $a$, $b+8$, $c$ theo thứ tự lập thành cấp số cộng và $a$, $b+8$, $c+64$ theo thứ tự lập thành cấp số nhân. Tính giá trị biểu thức $P=a-b+2c$.
	\choice
	{$P=32$}
	{$P=\dfrac{92}{9}$}
	{\True $P=64$}
	{$P=\dfrac{184}{9}$}
	\loigiai{
		Theo giả thiết, ta có hệ phương trình \[\heva{&ac=b^2\\&a+c=2(b+8)\\&a(c+64)=(b+8)^2}\Leftrightarrow\heva{&ac=b^2&\quad(1)\\&a-2b=16-c&(2)\\&ac+64a=\left(b+8\right)^2.&\quad(3)}\]
		Thay $(1)$ vào $(3)$ ta được $b^2+64a=b^2+16b+64\Leftrightarrow 4a-b=4.\qquad(4)$\\
		Kết hợp $(2)$ với $(4)$ ta được \[\heva{&a-2b=16-c\\&4a-b=4}\Leftrightarrow\heva{&a=\dfrac{c-8}{7}\\&b=\dfrac{4c-60}{7}.} \qquad (5)\]
		Thay $(5)$ vào $(1)$ ta được
		\[7(c-8)c=(4c-60)^2\Leftrightarrow 9c^2-424c+3600=0\Leftrightarrow\hoac{&c=36\\&c=\dfrac{100}{9}}\Leftrightarrow c=36.\qquad\left(\text{Vì}\;c\in\mathbb{Z}\right)\]
		Với $c=36$ $\Rightarrow a=4$, $b=12\Rightarrow P=4-12+72=64$.}
\end{ex}
\begin{ex}%[1K2K3-4]
	Cho $3$ số $a$, $b$, $c$ theo thứ tự lập thành cấp số nhân với công bội khác $1$ . Biết cũng theo thứ tự đó chúng lần lượt là số thứ nhất, thứ tư và thứ tám của một cấp số cộng công sai là $d$, $(d \neq 0)$. Tính $\dfrac{a}{d}$.
	\choice
	{$\dfrac{4}{3}$}
	{\True $9$}
	{$\dfrac{4}{9}$}
	{$3$}
	\loigiai{
		Do $a, b, c$ theo thứ tự lần lượt là số thứ nhất, thứ tư và thứ tám của một cấp số cộng công sai là $d,(d \neq 0)$ nên $\left\{\begin{array}{l}b=a+3 d \\ c=a+7 d\end{array}\right.$.\\
		Hơn nữa $a, b, c$ theo thứ tự lập thành cấp số nhân với công bội khác 1 nên $a c=b^{2}$.\\
		Khi đó \begin{eqnarray*}
			a(a+7 d)=(a+3 d)^{2} &\Leftrightarrow& a^{2}+7 a d=a^{2}+6 a d+9 d^{2}\\
			&\Leftrightarrow& 9 d^{2}-a d=0 \Leftrightarrow 9 d=a \Leftrightarrow \dfrac{a}{d}=9.
		\end{eqnarray*}
		Vậy $\dfrac{a}{d}=9$.
	}
\end{ex}
\begin{ex}%[1K2B3-4]
	Cho dãy số $(u_n)$ là một cấp số nhân với $u_n\neq 0$, $n\in\mathbb{N}^*$. Dãy số nào sau đây không phải là cấp số nhân?
	\choice
	{\True $u_1+2$; $u_2+2$; $u_3+2$; $\ldots$}
	{$3u_1$; $3u_2$; $3u_3$; $\ldots$}
	{$\dfrac{1}{u_1}$; $\dfrac{1}{u_2}$; $\dfrac{1}{u_3}$; $\ldots$}
	{$u_1$; $u_3$; $u_5$; $\ldots$}
	\loigiai{
		Giả sử $(u_n)$ là một cấp số nhân với công bội $q$.\\
		Ta có $u_2=u_1 q$, $u_3=u_1 q^2$.\\
		Dễ thấy $\dfrac{u_2+2}{u_1+2}=\dfrac{u_1 q+2}{u_1+2}$ và $\dfrac{u_3+2}{u_2+2}=\dfrac{u_1 q^2+2}{u_1 q+2}$.\\
		Do $\dfrac{u_2+2}{u_1+2}\neq\dfrac{u_3+2}{u_2+2}$ $\Rightarrow$ dãy số $u_1+2$; $u_2+2$; $u_3+2$; $\ldots$ không phải là cấp số nhân.
	}
\end{ex}
\begin{ex}%[1K2B3-4]
	Xác định $x$ để $3$ số $x-2$; $x+1$; $3-x$ theo thứ tự lập thành một cấp số nhân
	\choice
	{$x=\pm 1$}
	{\True Không có giá trị nào của $x$}
	{$x=-3$}
	{$x=2$}
	\loigiai{
		Ba số $x-2$; $x+1$; $3-x$ theo thứ tự lập thành một cấp số nhân\\
		$\Leftrightarrow(x-2)(3-x)=(x+1)^2 \Leftrightarrow 2x^2-3x+7=0$ (Phương trình vô nghiệm).
	}
\end{ex}
\begin{ex}%[1D3Y4-4]
	Trong các dãy số $(u_n)$ cho bởi số hạng tổng quát $u_n$ sau, dãy số nào là một cấp số nhân?
	\choice
	{\True $u_n=7\cdot 3^n$}
	{$u_n=\dfrac{7}{3n}$}
	{$u_n=7-3^n$}
	{$u_n=7-3n$}
	\loigiai{
		Từ các đáp án trên, với dãy $(u_n)$ cho bởi $u_n=7\cdot 3^n$ là một cấp số nhân, vì
		$$T=\dfrac{u_{n+1}}{u_n}=\dfrac{7\cdot 3^{n+1}}{7\cdot 3^n}=3 \text{ (không đổi).}$$
	}
\end{ex}
\begin{ex}%[1K2K3-4]
	Số hạng thứ hai, số hạng đầu và số hạng thứ ba của một cấp số cộng với công sai khác $0$ theo thứ tự đó lập thành một cấp số nhân với công bội $q$. Tìm $q$.
	\choice
	{\True $q=-2$}
	{$q=-\dfrac{3}{2}$}
	{$q=\dfrac{3}{2}$}
	{$q=2$}
	\loigiai{
		Giả sử ba số hạng $a$; $b$; $c$ lập thành cấp số cộng thỏa mãn yêu cầu, khi đó $b$; $a$; $c$ theo thứ tự đó lập thành cấp số nhân công bội $q\neq 1$. Ta có\\
		\[\heva{&a+c=2b\\&a=bq; c=bq^2}\Rightarrow bq+bq^2=2b\Leftrightarrow\hoac{&b=0\\&q^2+q-2=0.}\]
		\begin{itemize}
			\item Nếu $b=0\Rightarrow a=b=c=0$ nên $a$; $b$; $c$ là cấp số cộng công sai $d=0$. (Vô lí!)
			\item Nếu $q^2+q-2=0$ thì $ q=1$ hoặc $q=-2$. Dễ thấy trường hợp $q = 1$ là không thỏa mãn, vì khi đó $a = b = c$. Do đó $q = -2$.
		\end{itemize}
	}
\end{ex}
\begin{ex}%[1K2K3-4]
	Ba số $x$, $y$, $z$ lập thành một cấp số cộng và có tổng bằng $21$. Nếu lần lượt thêm các số $2$, $3$, $9$ vào ba số đó (theo thứ tự của cấp số cộng) thì được ba số lập thành một cấp số nhân. Tính $F=x^2+y^2+z^2$.
	\choice
	{$F=389$ hoặc $F=395$}
	{$F=395$ hoặc $F=179$}
	{$F=441$ hoặc $F=357$}
	{\True $F=389$ hoặc $F=179$}
	\loigiai{
		Theo tính chất của cấp số cộng, ta có $x+z=2y$.\\
		Kết hợp với giả thiết $x+y+z=21$, ta suy ra $3y=21\Leftrightarrow y=7$.\\
		Gọi $d$ là công sai của cấp số cộng thì $x=y-d=7-d$ và $z=y+d=7+d$.\\
		Sau khi thêm các số $2$, $3$, $9$ vào ba số $x$, $y$, $z$ ta được ba số là $x+2$, $y+3$, $z+9$ hay $9-d$, $10$, $16+d$.\\
		Theo tính chất của cấp số nhân, ta có $(9-d)(16+d)=10^2\Leftrightarrow d^2+7d-44=0$.\\
		Giải phương trình ta được $d=-11$ hoặc $d=4$.\\
		Với $d=-11$, cấp số cộng $18$, $7$, $-4$. Lúc này $F=389$.\\
		Với $d=4$, cấp số cộng $3$, $7$, $11$. Lúc này $F=179$.}
\end{ex}
\Closesolutionfile{ans}
% \begin{indapan}{10}
% 	{ans/ans-1K2-3-dang4}
% \end{indapan}
\begin{dang}{Tính tổng của cấp số nhân}
	Phương pháp
	\begin{itemize}
		\item Xác định số hạng đầu $u_1$, công bội $q$.
		\item Áp dụng công thức tính tổng các số hạng của cấp số nhân. 
	\end{itemize}
\end{dang}
\subsubsection{Ví dụ minh hoạ}
\begin{vd}%[NB]%[DCHT Toán 11 - KNTT -Đỗ Chí Tâm] %[1K2Y7-5]
	Tính tổng 10 số hạng đầu tiên của cấp số nhân $(u_n)$, biết $u_1=-3$ và công bội $q=-2$. \dapso{$S_{10}=1023$}
	\loigiai{
		Ta có: $S_{10}=\dfrac{u_1\left(1-q^{10}\right)}{1-q}=1023$.}
\end{vd}

\begin{vd}%[TH]%[DCHT Toán 11 - KNTT -Đỗ Chí Tâm] %[1K2B7-5]
	Tính tổng $8$ số hạng đầu tiên của cấp số nhân $(u_n),$ biết $u_1=3$ và $u_2=6$. \dapso{$S_8=765$}
	\loigiai{
		Ta có: $u_2=u_1.q \Leftrightarrow 6=3.q \Leftrightarrow q=2$\\
		$\hspace*{1.2cm}  S_8=u_1\dfrac{1-q^8}{1-q}=3.\dfrac{1-2^8}{1-2}=765. $
	}
\end{vd}

\begin{vd}[thuộc chương giới hạn]%[TH]%[DCHT Toán 11 - KNTT -Đỗ Chí Tâm] %[1K2B7-5]
	Tính tổng vô hạn $S=1+\dfrac{1}{2}+\dfrac{1}{2^2}+...+\dfrac{1}{2^n}+...$ \dapso{$S=2$}
	\loigiai{
		Đây là tổng của cấp số nhân lùi vô hạn, với $u_1=1, q=\dfrac{1}{2}$. Khi đó
		$$S=\dfrac{u_1}{1-q}=\dfrac{1}{1-\dfrac{1}{2}}=2.$$
	}
\end{vd}

\begin{vd}%[VD]%[DCHT Toán 11 - KNTT -Đỗ Chí Tâm] %[1K2K7-5]
	Tính tổng $200$ số hạng đầu tiên của dãy số $(u_n)$ biết $\heva{&u_1=1\\&u_{n+1}=3u_n}$.
	\loigiai{
		Dễ thấy dãy đã cho là một cấp số nhân với công bội $q=3; u_1=1$.\\
		Từ đó $S_{200}=u_1\dfrac{q^{200}-1}{q-1}$ $=\dfrac{3^{200}-1}{2}$.
	}
\end{vd}

\begin{vd}%[VD]%[DCHT Toán 11 - KNTT -Đỗ Chí Tâm]%[1K2K7-5]
	Một cấp số nhân có số hạng đầu $u_1=3$, công bội $q=2$. Biết $S_n=765$, tìm $n$.
	% \dapso{$n=8$}
	\loigiai{ 
		Áp dụng công thức tính tổng của cấp số nhân ta có
		$S_n=765 \Leftrightarrow \dfrac{u_1( 1-q^n )}{1-q}=765 \Leftrightarrow \dfrac{3.( 1-2^n )}{1-2}=765 \Leftrightarrow 2^n=256=2^8 \Leftrightarrow n=8$.} 
\end{vd}

\subsubsection{Bài tập tự luận}
 
\begin{bt}%[DCHT Toán 11 - KNTT -Đỗ Chí Tâm] %[1K2Y7-5]
	Một cấp số nhân có số hạng đầu $u_1=3$ và công bội $q=2 $. Tính tổng $8$ số hạng đầu của cấp số nhân.
	\dapso{$765$}		
	\loigiai{
		Ta có $S_8=\dfrac{u_1\left(1-q^8\right)}{1-q}=\dfrac{3 \left(1-2^8\right)}{1-2}=765$.	
	}	
\end{bt}

% \begin{bt}%[DCHT Toán 11 - KNTT -Đỗ Chí Tâm]%[1K2Y7-5]
% 	Một cấp số nhân có số hạng đầu $u_1=1$ và công bội $q=3$. Tính $S_{10}.$
% 	\dapso{$29524$}	
% 	\loigiai{
% 		Ta có $S_{10}=\dfrac{u_1\left(1-q^{10}\right)}{1-q}=1.\dfrac{1-3^{10}}{1-3}=29524$.	
% 	}	
% \end{bt}

% \begin{bt}%[DCHT Toán 11 - KNTT -Đỗ Chí Tâm] %[1K2Y7-5]
% 	Một cấp số nhân $(u_n)$ có $u_1=4$ và công bội $q=2$. Tính $S_{20}$.
% 	\dapso{$4194300$}	
% 	\loigiai{
% 		Ta có $S_{20}=\dfrac{u_1\left(1-q^{20}\right)}{1-q}=4194300$.}
% \end{bt}

\begin{bt}[thuộc chương giới hạn]%[DCHT Toán 11 - KNTT -Đỗ Chí Tâm] %[1K2B7-5]	
	Tính tổng $S=1+\dfrac{1}{3}+\dfrac{1}{3^2}+\cdots+\dfrac{1}{3^n}+\cdots$.
	\dapso{$\dfrac{3}{2}$}			
	\loigiai{
		Đây là tổng của một cấp số nhân lùi vô hạn với $u_1=1, q=\dfrac{1}{3}$\\
		Suy ra   $S=\dfrac{u_1}{1-q}=\dfrac{1}{1-\dfrac{1}{3}}=\dfrac{3}{2}.$
	}
\end{bt}

\begin{bt}%[DCHT Toán 11 - KNTT -Đỗ Chí Tâm] %[1K2B7-5]	
	Cho cấp số nhân có $q=-3, S_6=730$. Tính $u_1$.
	\dapso{$4$}		
	\loigiai{
		$S_6=u_1.\dfrac{1-q^6}{1-q}\Rightarrow u_1=S_6\cdot\dfrac{1-q}{1-q^6}=730\cdot \dfrac{1-(-3)}{1-(-3)^6}=4$.
	}
\end{bt}

% \begin{bt}%[DCHT Toán 11 - KNTT -Đỗ Chí Tâm] %[1K2B7-5]
% 	Một cấp số nhân $(u_n)$ có $u_1=-5$, $u_2=10$.Tính tổng của $15$ số hạng đầu của cấp số nhân đó.
% 	\dapso{$-54615$}	
% 	\loigiai{
% 		Công bội của cấp số nhân đã cho là: $q=\dfrac{u_2}{u_1}=\dfrac{10}{-5}=-2$.\\
% 		Tổng của $15$ số hạng đầu của cấp số nhân đó là $S_{15}=-5\cdot \dfrac{1-(-2)^{15}}{1-(-2)}=-54615$.
% 	}
% \end{bt}

% \begin{bt}%[DCHT Toán 11 - KNTT -Đỗ Chí Tâm] %[1K2B7-5]
% 	Một cấp số nhân $(u_n)$ có $u_1=2$, $u_2=-2$.Tính tổng của $9$ số hạng đầu của cấp số nhân đó.
% 	\dapso{$2$}	
% 	\loigiai{
% 		Công bội của cấp số nhân đã cho là: $q=\dfrac{u_2}{u_1}=\dfrac{-2}{2}=-1$.\\
% 		Tổng của $9$ số hạng đầu của cấp số nhân đó là: $S_9=2\cdot \dfrac{1-(-1)^9}{1-(-1)}=2$.
% 	}
% \end{bt}

\begin{bt}%[DCHT Toán 11 - KNTT -Đỗ Chí Tâm] %[1K2K7-5]
	Một cấp số nhân $(u_n)$ có $u_3=8$, $u_5=32$ và công bội $q>0$. Tính tổng của $10$ số hạng đầu tiên của cấp số nhân.
	\dapso{$2046$}	
	\loigiai{
		$\heva{u_3&=8\\u_5&=32}\Leftrightarrow \heva{u_1.q^2&=8\\u_1.q^4&=32}\Rightarrow q^2=\dfrac{32}{8}=4\Rightarrow q=2, u_1=2$\\
		\hspace*{5.2cm}$\Rightarrow S_{10}=u_1.\dfrac{1-q^{10}}{1-q}=2.\dfrac{2-2^{10}}{1-2}=2046$.
	}
\end{bt}

\begin{bt}%[DCHT Toán 11 - KNTT -Đỗ Chí Tâm] %[1K2K7-5]	
	Tính tổng $S=2+6+18+...+13122$.
	\dapso{$19682$}
	\loigiai{
		Xét cấp số nhân có $u_1=2, q=3$. Khi đó $13122=u_1.q^{n-1}\Leftrightarrow 13122=2.3^{n-1}\Leftrightarrow n=9$\\
		Vậy $S=S_9=u_1\dfrac{1-q^9}{1-q}=2.\dfrac{1-3^9}{1-3}=19682$
	}
\end{bt}

\begin{bt}%[DCHT Toán 11 - KNTT -Đỗ Chí Tâm] %[1K2K7-5]	
	Tính tổng $S=1+2+4+8+\cdots+1024$.	
	\dapso{$2047$}
	\loigiai{
		Xét cấp số nhân có $u_1=1, q=2$. Khi đó $1024=u_1.q^{n-1}\Leftrightarrow 1024=1.2^{n-1}\Leftrightarrow n=11$.\\
		Vậy $S=S_{11}=u_1\dfrac{1-q^{11}}{1-q}=1.\dfrac{1-2^{11}}{1-2}=2047$.
	}
\end{bt}

\begin{bt}%[DCHT Toán 11 - KNTT -Đỗ Chí Tâm] %[1K2K7-5]	
	Một cấp số nhân có $u_1=1, q=3$, biết $S_n=3280$. Tìm $n$.	
	\dapso{$8$}
	\loigiai{
		$S_n=u_1\dfrac{1-q^n}{1-q}=1.\dfrac{1-3^n}{1-3}=3280\Rightarrow n=8$.
	}
\end{bt}

% \begin{bt}%[DCHT Toán 11 - KNTT -Đỗ Chí Tâm] %[1K2K7-5]	
% 	Một cấp số nhân $(u_n)$ có $u_4+u_6=-540, u_3+u_5=180$. Tính $S_5$.
% 	\dapso{$122$}
% 	\loigiai{
% 		$\heva{u_4+u_6&=-540\\u_3+u_5&=180}\Leftrightarrow \heva{u_1.q^3+u_1.q^5&=-540\\u_1.q^2+u_1.q^4&=180} \Leftrightarrow \heva{u_1q^3(1+q^2)&=-540\\u_1q^2(1+q^2)&=180}\Leftrightarrow \heva{u_1&=2\\ q&=-3}$.\\
% 		Vậy $S_5=u_1\dfrac{1-q^5}{1-q}=2.\dfrac{1-(-3)^5}{1+3}=122$.
% 	}
% \end{bt}

\begin{bt}%[DCHT Toán 11 - KNTT -Đỗ Chí Tâm] %[1K2K7-5]
	Bốn số hạng liên tiếp của một cấp số nhân, trong đó số hạng thứ hai nhỏ hơn số hạng thứ nhất $35$, còn số hạng thứ ba lớn hơn số hạng thứ tư $560$. Tìm tổng của bốn số hạng trên, biết công bội mang giá trị dương.
	\dapso{$-\dfrac{2975}{3}$}
	\loigiai{
		Theo đề ta có $\heva{u_1-u_2&=35\\u_3-u_4&=560}\Leftrightarrow \heva{&u_1-u_1q=35\\&u_1q^2-u_1q^3=560}\Leftrightarrow \heva{&u_1(1-q)=35\,\,\,\,\,\,\, (1)\\&u_1q^2(1-q)=560\,\,\,\, (2)}$\\
		Thay $(1)$ vào $(2)$ ta được $q^2=16\Leftrightarrow q=\pm 4$.\\
		Với $q=4$ thay vào $(1)$ ta được $u_1=-\dfrac{35}{3}$.	\\
		$S_4=u_1.\dfrac{1-q^4}{1-q}=-\dfrac{2975}{3}$.
	}	
\end{bt}

\begin{bt}[thuộc chương giới hạn]%[DCHT Toán 11 - KNTT -Đỗ Chí Tâm] %[1K2G7-5]
	Tổng của một cấp số nhân lùi vô hạn bằng $ \dfrac{1}{4} $, tổng ba số hạng đầu tiên của cấp số nhân đó bằng $ \dfrac{7}{27} $. Tổng của số hạng đầu và công bội của cấp số nhân đó bằng
	\dapso{$S=0$}	
	\loigiai{Gọi $ u_1 $ và $ q $ với ($ |q|<1 $) lần lượt là số hạng đầu và cộng bội của cấp số nhân lùi vô hạn. Theo giả thiết, ta có $$\begin{cases}
			\dfrac{u_1}{q-1}=\dfrac{1}{4}\\
			u_1+u_1q+u_1q^2=\dfrac{7}{27}
		\end{cases} \Leftrightarrow \begin{cases}
			\dfrac{u_1}{q-1}=\dfrac{1}{4}\\
			u_1 (1-q^3)=\dfrac{7}{27}(1-q)
		\end{cases} \Leftrightarrow \begin{cases}\dfrac{u_1}{1-q}=\dfrac{1}{4}\\ q^3=-\dfrac{1}{27} \end{cases}\Leftrightarrow
		\heva{& u_1=\dfrac{1}{3}\\& q=-\dfrac{1}{3}.}$$ 
		Vậy $ u_1+q=0 $.
	}
\end{bt}

\begin{bt}%[DCHT Toán 11 - KNTT -Đỗ Chí Tâm] %[1K2G7-5]
	Một du khách vào trường đua ngựa đặt cược, lần đầu đặt $20.000$ đồng, mỗi lần sau tiền đặt gấp đôi số tiền lần đặt trước. Người đó thua $10$ lần liên tiếp và thắng ở lần thứ $11$. Hỏi du khách trên thắng hay thua bao nhiêu tiền?
	\dapso{$20000$ đồng}
	\loigiai{
		Số tiền du khách đặt trong mỗi lần (kể từ lần đầu) là một cấp số nhân có $u_1=20.000$ và công bội $q=2$.\\
		Du khách thua trong 10 lần liên tiếp đầu tiên nên tổng số tiền thua là
		$$S_{10}=\dfrac{u_1(1-q^{10})}{1-q}=\dfrac{20000(1-2^{10})}{1-2}=20000(2^{10}-1) \text{(đồng).}$$ 
		Số tiền du khách thắng trong lần thứ 11 là $u_{11}=u_1q^{10}=20000.2^{10}$ (đồng).\\
		Ta có $u_{11}-S_{10}=20000>0$. Vậy du khách thắng $20000$ đồng.
	}
\end{bt}
\subsubsection{Câu hỏi trắc nghiệm}
\Opensolutionfile{ans}[ans/ans-1K2-3-Dang5]
\begin{ex}%[1K2Y7-5]
	Cho cấp số nhân $u_1,u_2,u_3,\ldots,u_n$ với công bội $q$ ($q\neq 0$, $q\neq 1$). Đặt \[S_n=u_1+u_2+u_3+\cdots +u_n.\] Khẳng định nào sau đây là đúng?
	\choice
	{\True $S_n = \dfrac{u_1\left(q^n-1\right)}{q-1}$}
	{$S_n = \dfrac{u_1\left(q^n+1\right)}{q+1}$}
	{$S_n = \dfrac{u_1\left(q^{n-1}-1\right)}{q+1}$}
	{$S_n = \dfrac{u_1\left(q^{n-1}-1\right)}{q-1}$}
	\loigiai{
		Ta có $S_n=u_1+u_2+u_3+\cdots +u_n = u_1\cdot \dfrac{1-q^n}{1-q} = \dfrac{u_1\left(q^n-1\right)}{q-1}$.
	}
\end{ex}
\begin{ex}%[1K2Y7-5]
	Cho cấp số nhân $(u_n)$ có số hạng đầu $u_1=12$ và công sai $q=\dfrac{3}{2}$. Tổng $5$ số hạng đầu của cấp số nhân bằng
	\choice
	{$\dfrac{93}{4}$}
	{$\dfrac{633}{2}$}
	{\True $\dfrac{633}{4}$}
	{$\dfrac{93}{2}$}
	\loigiai{
		Gọi $S_5$ là tổng $5$ số hạng đầu của cấp số nhân đã cho. Khi đó ta có $$S_5=u_1\cdot \dfrac{1-q^5}{1-q}=12\cdot \dfrac{1-\left(\dfrac{3}{2}\right)^5}{1-\dfrac{3}{2}}=\dfrac{633}{4}.$$
	}
\end{ex}
\begin{ex}%[1K2Y7-5]
	Cho cấp số nhân $(u_n)$ có số hạng đầu $u_1=3$, công bội $q=-2$. Tính tổng $10$ số hạng đầu tiên của cấp số nhân $(u_n)$.
	\choice
	{\True $-1023$}
	{$1023$}
	{$513$}
	{$-513$}
	\loigiai{
		Tổng của $10$ số hạng đầu bằng
		$$S_{10}=u_1\cdot\dfrac{q^{10}-1}{q-1}=3\cdot\dfrac{(-2)^{10}-1}{-2-1}=-1023.$$
	}
\end{ex}
\begin{ex}%[1K2Y7-5]
	Cho cấp số nhân $(u_n)$ có $u_2=-2$ và $u_5=54$. Tính tổng $1000$ số hạng đầu tiên của cấp số nhân đã cho.
	\choice
	{$S_{1000}=\dfrac{3^{1000}-1}{2}$}
	{\True $S_{1000}=\dfrac{1-3^{1000}}{6}$}
	{$S_{1000}=\dfrac{3^{1000}-1}{6}$}
	{$S_{1000}=\dfrac{1-3^{1000}}{4}$}
	\loigiai{
		Ta có $u_5=u_2\cdot q^3\Leftrightarrow q^3=\dfrac{u_5}{u_2}=\dfrac{54}{-2}=-27=(-3)^3\Rightarrow q=-3$ và $u_1=\dfrac{u_2}{q}=\dfrac{2}{3}$.\\
		Suy ra $S_{1000}=u_1\cdot\dfrac{1-q^n}{1-q}=\dfrac{2}{3}\cdot\dfrac{1-(-3)^{1000}}{1+3}=\dfrac{1-3^{1000}}{6}$.
	}
\end{ex}
\begin{ex}%[1K2Y7-5]
	Tính tổng tất cả các số hạng của một cấp số nhân, biết số hạng đầu bằng $18$, số hạng thứ hai bằng $54$ và số hạng cuối bằng $39366$.
	\choice
	{$19674$}
	{\True $59040$}
	{$177138$}
	{$~6552$}
	\loigiai{
		$u_1=18,u_2=54 \Rightarrow q=3$.\\
		$u_n=39366 \Leftrightarrow u_1 \cdot q^{n-1}=39366 \Leftrightarrow 18 \cdot 3^{n-1}=39366 \Leftrightarrow 3^{n-1}=3^7 \Leftrightarrow n=8$.\\
		Vậy $S_8=18 \cdot \dfrac{1-3^8}{1-3}=59040$.}
\end{ex}
\begin{ex}%[1K2B7-5]
	Dãy số $\left(u_n\right)$ xác định bởi $\heva{&	u_1=1\\&u_{n + 1}= \dfrac{1}{2}u_n}$ với $n \ge 1$. Tính tổng $S = u_1 + u_2 +\cdots + u_{10}$.
	\choice
	{$S = \dfrac {1023} {2048}$}
	{$S = \dfrac{5}{2}$}
	{\True $\dfrac {1023} {512}$}
	{$S = 2$}
	\loigiai{
		Ta có các số hạng của dãy số $\left(u_n\right)$ là  $1,\dfrac{1}{2},\dfrac{1}{4},\dfrac{1}{8},\dfrac {1}{16},\dfrac{1}{32},\ldots ,\dfrac{1} {2^n}$. Khi đó $\left(u_n\right)$ lập thành một cấp số nhân có $u_1 = 1$ và công bội $q = \dfrac{1}{2}$. \\
		Suy ra $S = u_1 + u_2 +\cdots + u_{10} 
		=1+\dfrac{1}{2} + \dfrac{1}{4} +\cdots + \dfrac{1}{2^9}
		=\dfrac{1\cdot\left[1-\left(\dfrac{1}{2}\right)^{10}\right]}{1 - \dfrac{1}{2}}
		=\dfrac {1023} {512}$.}
\end{ex}
\begin{ex}%[1K2B7-5]
	Cho cấp số nhân $({{u}_{n}} )$ có ${{u}_{1}}=-6$ và $q=-2$. Tổng $n$ số hạng đầu tiên của cấp số nhân đã cho bằng $2046$. Tìm $n$.
	\choice
	{$n=9$}
	{$n=12$}
	{$n=11$}
	{\True $n=10$}
	\loigiai {
		Ta có
		$2046={{S}_{n}}={{u}_{1}}\cdot \dfrac{1-{{q}^{n}}}{1-q}=-6\cdot \dfrac{1-{{(-2 )}^{n}}}{1-(-2 )}=2({{(-2 )}^{n}}-1 )\Rightarrow {{(-2 )}^{n}}=1024\Leftrightarrow n=10$.}
\end{ex}
\begin{ex}%[1K2B7-5]
	Tổng $100$ số hạng đầu của dãy số $\left(u_n\right)$ với $u_n=2 n-1$ là
	\choice
	{$199$}
	{$2^{100}-1$}
	{\True $10000$}
	{$9999$}
	\loigiai{
		Ta có $(u_n)$ là cấp số cộng công sai $d=2$ và $u_1=1$.\\
		Do đó $S_{n}=n\cdot u_1+\dfrac{n(n-1)}{2}\cdot d=100 \cdot 1 +\dfrac{100 \cdot 99 }{2} \cdot 2 = 10000$.
	}
\end{ex}
\begin{ex}%[1K2B7-5]
	Cho dãy số $(u_n)$ với $u_n = \left(\dfrac{1}{2}\right)^n+1, \forall n \in \mathbb{N}^*$. Tính $S_{2019}=u_1+u_2+u_3+ \cdots + u_{2019}$.
	\choice
	{$S_{2019}=2019+\dfrac{1}{2^{2019}}$}
	{$S_{2019}=\dfrac{4039}{2}$}
	{$S_{2019}=\dfrac{6057}{2}$}
	{\True $S_{2019}=2020-\dfrac{1}{2^{2019}}$}
	\loigiai{
		Ta có
		\allowdisplaybreaks
		\begin{eqnarray*}
			S_{2019} &=& u_1+u_2+u_3+ \cdots + u_{2019}\\
			&=& \left(\dfrac{1}{2}+1\right) + \left[\left(\dfrac{1}{2}\right)^2+1\right] + \left[\left(\dfrac{1}{2}\right)^3+1\right] + \cdots + \left[\left(\dfrac{1}{2}\right)^{2019}+1\right]\\
			&=& 2019 + \dfrac{1}{2} + \left(\dfrac{1}{2}\right)^2 + \left(\dfrac{1}{2}\right)^3 + \cdots + \left(\dfrac{1}{2}\right)^{2019}\\
			&=& 2019 + \dfrac{1}{2} \cdot \dfrac{1-\left(\dfrac{1}{2}\right)^{2019}}{1-\dfrac{1}{2}} = 2019 + 1 - \dfrac{1}{2^{2019}}\\
			&=& 2020 - \dfrac{1}{2^{2019}}.
		\end{eqnarray*}
	}
\end{ex}
\begin{ex}%[1K2K7-5]
	Cho $S=11+101+1001+\cdots +\underbrace{1000\ldots 01}_{(n-1)\text{ chữ số 0}}$. Khẳng định nào sau đây là đúng?
	\choice
	{$S=10\left(\dfrac{10^n-1}{9}\right)$}
	{$S=10\left(\dfrac{10^n-1}{9}\right)-n$}
	{\True $S=10\left(\dfrac{10^n-1}{9}\right)+n$}
	{$S=\left(\dfrac{10^n-1}{9}\right)+n$}
	\loigiai{
		Ta có
		\begin{align*}
			S&=(10+1)+(10^2+1)+(10^3+1)+\cdots +(10^n+1)\\
			&=\left(10+10^2+10^3+\cdots +10^n\right)+\underbrace{1+1+1+\cdots+1}_{n\text{ số } 1}\\
			&=10\left(\dfrac{10^n-1}{9}\right)+n.
		\end{align*}
	}
\end{ex}
\begin{ex}%[1K2K7-5]
	Gọi $S=1+11+111+\cdots+\underbrace{111\ldots1}_{(n\text{ số }1)}$ thì $S$ nhận giá trị nào sau đây?
	\choice
	{\True $S=\dfrac{1}{9}\left[10\left(\dfrac{10^n-1}{9}\right)-n \right]$}
	{$S=\dfrac{10^n-1}{81}$}
	{$S=10\left(\dfrac{10^n-1}{81}\right)-n$}
	{$S=10\left(\dfrac{10^n-1}{81} \right)$}
	\loigiai {
		Ta có
		$S=\dfrac{1}{9}(9+99+999+\cdots+\underbrace{99\ldots9}_{\text{n số }9} )=\dfrac{1}{9}\cdot \left[ 10\cdot \dfrac{1-{{10}^{n}}}{1-10}-n \right]$.}
\end{ex}
\begin{ex}%[1K2K7-5]
	Cho dãy số $(u_n)$ thỏa mãn $\heva{&u_1=1\\&u_n=2u_{n-1}+1, n\geq2}$. Tổng $S=u_1+u_2+ \cdots +u_{20}$ là
	\choice
	{$2^{21}-20$}
	{\True $2^{21}-22$}
	{$2^{20}$}
	{$2^{20}-20$}
	\loigiai{
		Dự đoán công thức số hạng tổng quát $u_n=2^n-1$ (Chứng minh bằng phương pháp quy nạp TH).\\
		$S=2^1+2^2+\cdots+2^{20}-20=2\cdot\dfrac{1-2^{20}}{1-2}-20=2^{21}-22$.
	}
\end{ex}
\begin{ex}%[1K2K7-5]
	Biết rằng $S=1+2\cdot 3+{{3\cdot 3}^2}+\cdots+{{11\cdot 3}^{10}}=a+\dfrac{{{21\cdot 3}^{b}}}{4}$. Tính $P=a+\dfrac{b}{4}$.
	\choice
	{\True $P=3$}
	{$P=4$}
	{$P=1$}
	{$P=2$}
	\loigiai {
		Từ giả thiết suy ra $3S=3+{{2\cdot 3}^2}+{{3\cdot 3}^3}+\cdots+{{11\cdot 3}^{11}}$.\\
		Do đó
		{\allowdisplaybreaks
			\begin{eqnarray*}
				-2S&=&S-3S=1+3+{{3}^2}+\cdots+{{3}^{10}}-{{10\cdot 3}^{11}}\\
				&=&\dfrac{1-{{3}^{11}}}{1-3}-{{11\cdot 3}^{11}}=-\dfrac{1}{2}-\dfrac{{{21\cdot 3}^{11}}}{2}\Rightarrow S=\dfrac{1}{4}+\dfrac{21}{4}\cdot{{3}^{11}}.
			\end{eqnarray*}
		}
		Vì $S=\dfrac{1}{4}+\dfrac{{{21\cdot 3}^{11}}}{4}=a+\dfrac{{{21\cdot 3}^{b}}}{4}\Rightarrow a=\dfrac{1}{4},\,\,b=11\Rightarrow P=\dfrac{1}{4}+\dfrac{11}{4}=3$.}
\end{ex}
\Closesolutionfile{ans}
% \begin{indapan}{10}
% 	{ans/ans-1K2-3-Dang5}
% \end{indapan}

\begin{dang}{Kết hợp cấp số cộng và cấp số nhân}
	Nhắc lại tính chất CSC, CSN
	\begin{itemize}
		\item $3$ số $a,b,c$ theo thứ tự lập thành CSC thì $a+c=2b$.
		\item $3$ số $a,b,c$ theo thứ tự lập thành CSN thì $a.c=b^2$.
	\end{itemize}
\end{dang}
\subsubsection{Ví dụ minh hoạ}
\begin{vd}%[TH]%[DCHT Toán 11 - KNTT -Đỗ Chí Tâm] %[1K2K7-6]
	Ba số $x, y, z$ theo thứ tự đó lập thành một CSN với công bội $q (q\ne 1)$, đồng thời các số $x, 2y, 3z$ theo thứ tự đó lập thành một CSC với công sai $d$ . Hãy tìm $q$?
	\loigiai{
		Ta có $x+3z=2.2y \Leftrightarrow x+3xq^2=2.2xq\Leftrightarrow 1+3q^2=4q \Leftrightarrow \hoac{q&=\dfrac{1}{3}\\q&=1 (L)}$
	}
\end{vd}

\begin{vd}%[TH]%[DCHT Toán 11 - KNTT -Đỗ Chí Tâm] %[1K2K7-6]
	Biết rằng $a, b, c$ là ba số hạng liên tiếp của một CSC và $a, c, b$ là ba số hạng liên tiếp của một CSN, đồng thời $a+b+c=30$. Tìm $a,b,c$.
	\loigiai{
		Theo đề ta có  $\heva{&a+c=2b\,\,\,\hspace*{0.8cm}(1)\\&ab=c^2\,\,\,\hspace*{1.2cm}(2)\\&a+b+c=30\,\,\, (3)}$\\
		Từ $(1)$ và $(3)$ ta được $3b=30\Leftrightarrow b=10$\\
		Thay $b=10$ vào $(1), (2)$ ta được $\heva{a+c=20\\10a=c^2}\Leftrightarrow \hoac{&c=10, a=10\,\, (L)\\ &c=-20, a=40 (N)}$\\
		Vậy $a=40, b=10, c=-20$
	}
\end{vd}

\begin{vd}%[VD]%[DCHT Toán 11 - KNTT -Đỗ Chí Tâm] %[1K2K7-6]
	Ba số $x, y, z$ theo thứ tự đó lập thành một CSN. Ba số $x, y-4 , z$ theo thứ tự đó lập thành CSN. Đồng thời các số $x, y-4 , z-9$ theo thứ tự đó lập thành CSC. Tìm $x,y,z$.
	
	\loigiai{
		Theo đề ta có  $\heva{&xz=y^2\,\,\,\hspace*{2.2cm}(1)\\&xz=(y-4)^2\,\,\,\hspace*{1.4cm}(2)\\&x+(z-9)=2(y-4)\,\,\, (3)}$\\
		Từ $(1)$ và $(2)$ ta có $y^2=(y-4)^2\Leftrightarrow y=2$\\
		Thế $y=2$ vào $(1)$ và $(3)$ ta được $\heva{xz=4\\x+z=5}\Rightarrow x=4, z=1$ hoặc $x=1, z=4$\\
		Vậy có 2 bộ $(x,y,z)$ thỏa yêu cầu bài toán là $(1,2,4)$ và $(4,2,1)$
	}
\end{vd}

\begin{vd}%[VD]%[DCHT Toán 11 - KNTT -Đỗ Chí Tâm] %[1K2K7-6]
	Cho $a,b,c$ là ba số hạng liên tiếp của một CSN và $a,b,c-4$ là ba số hạng liên tiếp của một CSC, đồng thời $a,b-1,c-5$ là ba số hạng liên tiếp của một CSN. Tìm $a,b,c$ biết $a,b,c$ là các số nguyên. 
	
	\loigiai{
		Theo đề ta có  $\heva{&ac=b^2\,\,\,\hspace*{1.9cm}(1)\\&a+c-4=2b\,\,\,\hspace*{0.8cm}(2)\\&a(c-5)=(b-1)^2\,\,\,\,(3)}$\\
		Thay $(1)$ vào $(3)$: $b^2-5a=b^2-2b+1\Leftrightarrow b=\dfrac{5a+1}{2}$\\
		Thay vào $(2)$ ta được $a+c-4=5a+1\Leftrightarrow c=4a+5$\\
		Thế $b, c$ theo $a$ vào $(1)$ ta được $9a^2-10a+1=0\Leftrightarrow a=1 \vee a=\dfrac{1}{9} (L)$
		Vậy $a=1, b=3, c=9$
	}
\end{vd}

\begin{vd}%[VDC]%[DCHT Toán 11 - KNTT -Đỗ Chí Tâm]%[1K2G7-6]
	Cho $4$ số nguyên dương, trong đó $3$ số đầu lập thành một CSC, $3$ số hạng sau thành lập CSN.
	Biết rằng tổng của số hạng đầu và số hạng cuối là $37$, tổng của hai số hạng giữa là $36$. Tìm tổng $4$ số đó
	\loigiai{
		Gọi 4 số cần tìm lần lượt là $a, b, c, d$\\
		$a, b, c$ là $3$ số hạng liên tiếp của CSC. Ta có $a+c=2b\,\,\, (1)$\\
		$b,c,d$ là $3$ số hạng liên tiếp của CSN. Ta có $bd=c^2\,\,\, (2)$\\
		Theo giả thuyết ta có $\heva{a+d&=37\,\,\, (3)\\b+c&=36\,\,\, (4)}$\\
		Từ $(4)\Rightarrow b=36-c$ thay vào $(1)$ ta được $a=72-3c$, thay $a$ vào $(3)$ ta được $d=-35+3c$\\
		Thế $b,d$ vào $(2)$ ta được $(36-c)(-35+3c)=c^2\Rightarrow c=20 \vee c=\dfrac{63}{4} (L)$\\
		Vậy $c=20, a=12, b=16, d=95\Rightarrow S=a+b+c+d=143$
	}
\end{vd}

\subsubsection{Bài tập tự luận}
 
\begin{bt}%[DCHT Toán 11 - KNTT -Đỗ Chí Tâm]%[1K2K7-6]
	Biết $x, y, x+4$ theo thứ tự lập thành cấp số cộng và $x+1, y+1, 2y+2$ theo thứ tự lập thành cấp số nhân với $x, y$ là số thực dương. Tính $x+y$.
	\dapso{$4$}
	\loigiai{
		Theo giả thiết ta có:\\
		$\heva{&x+(x+4)=2y\\&(x+1)(2y+2)=(y+1)^2}\Leftrightarrow \heva{&y=x+2\\&(x+1)(2x+6)=(x+3)^2}\Leftrightarrow \heva{&x=1\Rightarrow y=3\\&x=-3\Rightarrow y=-1}$\\
		Do đó $x+y=4$.
	}	
\end{bt}

\begin{bt}%[DCHT Toán 11 - KNTT -Đỗ Chí Tâm] %[1K2K7-6]
	Cho $3$ số $a, b, c$ theo thứ tự tạo thành một cấp số nhân với công bội khác $1$. Biết cũng theo thứ tự đó chúng lần lượt là số hạng thứ nhất, thứ tư và thứ tám của một cấp số cộng với công sai $d\ne 0$. Tính $\dfrac{a}{d}$.
	\dapso{$9$}
	\loigiai{
		$a, b, c$ lần lượt là số hạng thứ nhất, thứ tư, thứ tám của một CSC với công sai $d$\\
		ta có $\heva{b &=a+3d\\c&=a+7d}$.\\
		Mặt khác $a, b, c$ là $3$ số hạng liên tiếp của CSN nên\\ $a.c=b^2\Leftrightarrow a(a+7d)=(a+3d)^2\Leftrightarrow a^2+7ad=a^2+6ad+9d^2\Leftrightarrow 9d^2=ad\Leftrightarrow \dfrac{a}{d}=9$.
	}	
\end{bt}

\begin{bt}%[DCHT Toán 11 - KNTT -Đỗ Chí Tâm] %[1K2K7-6]
	Tìm tích các số dương $a$ và $b$ sao cho $a, a + 2b, 2a + b$ lập thành một cấp số cộng và $(b + 1)^2, ab + 5,(a + 1)^2$ lập thành một cấp số nhân.
	\dapso{$3$}
	\loigiai{
		Theo tính chất CSC ta có $a+(2a+b)=2(a+2b)\,\,\,\, (1)$\\
		Theo tính chất CSN ta có $(b+1)^2.(a+1)^2=(ab+5)^2\,\,\,\, (2)$\\
		Từ $(1)$ ta được $a=3b$, thay vào $(2)$ ta được $(b+1)^2(3b+1)^2=(3b^2+5)^2$\\
		$\Leftrightarrow \hoac{&(b+1)(3b+1)=(3b^2+5)\\& (b+1)(3b+1)=-(3b^2+5) \,\, (\text{Vô nghiêm})}\Leftrightarrow b=1, a=3\Rightarrow ab=3.$	
	}	
\end{bt}

\begin{bt}%[DCHT Toán 11 - KNTT -Đỗ Chí Tâm] %[1K2K7-6]
	$a,b,c\,(a\ne b\ne c)$ là ba số hạng liên tiếp của một cấp số cộng và $b,c,a$ là ba số hạng liên tiếp của một cấp số nhân, đồng thời $a.b.c=125$. Tìm $a,b,c$.	
	\dapso{$5$}
	\loigiai{
		$a,b, c$ là ba số hạng liên tiếp của cấp số cộng, nên có $a+c=2b$.\\
		$b,c,a$ là ba số hạng liên tiếp của một cấp số nhân, nên có $b.a=c^2$.\\
		Ta có hệ $\heva{&a+c=2b\,\,\,\, (1)\\&b.a=c^2\,\,\,\,\,\,\,\,(2) \\&a.b.c=125\,\,\,\, (3)}$\\
		Thay $(2)$ vào $(3)$ ta được $c^3=125\Rightarrow c=5$\\
		Thay $c=5$ vào $(1), (2)$ ta được hệ $\heva{a+5&=2b\\ab&=25}\Leftrightarrow \heva{&a=2b-5\\&2b^2-5b-25=0}\Leftrightarrow \hoac{&b=5\Rightarrow a=5\\&b=-\dfrac{5}{2}\Rightarrow a=-10}$\\
		Vậy $a=-10, b=-\dfrac{5}{2}, c=5$.	
	}	
\end{bt}

\begin{bt}%[DCHT Toán 11 - KNTT -Đỗ Chí Tâm] %[1K2K7-6]
	Một cấp số cộng và một cấp số nhân đều là các dãy tăng các số hạng thứ nhất của hai dãy số đều bằng $3$, các số hạng thứ hai bằng nhau. Tỉ số giữa các số hạng thứ ba của CSN và CSC là $\dfrac{9}{5}$. Tìm tích ba số hạng của cấp số cộng thỏa mãn tính chất trên.
	\dapso{$405$}
	\loigiai{
		Gọi $u_1, u_2, u_3$ là $3$ số hạng liên tiếp của CSC.\\
		Gọi $a_1, a_2, a_3$ là $3$ số hạng liên tiếp của CSN.\\
		Theo đề ta có hệ $\heva{&u_1=a_1=3\\&u_2=a_2\\&a_3=\dfrac{9}{5}u_3}\Leftrightarrow \heva{&u_1=a_1=3\\&3+d=3q\\&5(3q^2)=9(3+2d)}\Rightarrow q=3 \vee q=\dfrac{3}{5}$\\
		Chọn $q=3$ vì dãy tăng, khi đó $d=6$. \\
		Vậy $3$ số hạng của cấp số cộng là $3; 9; 15\Rightarrow 3\cdot9\cdot15=405$	
	}	
\end{bt}

\begin{bt}%[DCHT Toán 11 - KNTT -Đỗ Chí Tâm] %[1K2K7-6]
	Một CSC và CSN đều có số hạng đầu tiên là bằng 5, số hạng thứ hai của CSC lớn hơn số hạng thứ hai của CSN là 10, còn các số hạng thứ 3 của hai cấp số thì bằng nhau. Tìm tổng các số hạng của cấp số cộng biết công bội của cấp số nhân không âm.
	\dapso{$75$}
	\loigiai{
		Gọi $u_1, u_2, u_3$ là $3$ số hạng liên tiếp của CSC với công sai $d$.\\
		Gọi $a_1, a_2, a_3$ là $3$ số hạng liên tiếp của CSN với công bội $q$.\\
		Theo đề bài ta có: $\heva{&u_1=a_1=5\\&u_2-a_2=10\\&u_3=a_3}\Leftrightarrow \heva{&u_1=a_1=5\\&u_1+d-a_1q=10\\&u_1+2d=a_1q^2}\Leftrightarrow \heva{&u_1=a_1=5\\&d=5+5q\\&5+2d=5q^2}\Rightarrow q=3\vee q=-1 (L)$\\
		Với $q=3\Rightarrow d=20$. Vậy CSC là $5;25;45 \Rightarrow S=5+25+45=75$
	}	
\end{bt}

\begin{bt}%[DCHT Toán 11 - KNTT -Đỗ Chí Tâm] %[1K2G7-6]	
	Ba số khác nhau có tổng bằng $114$ có thể coi là ba số hạng liên tiếp của một CSN, hoặc coi là số hạng thứ nhất, thứ tư và thứ hai mươi lăm của một CSC. Tìm ba số đó.
	\dapso{$2; 14; 98$}
	\loigiai{
		Gọi $u_1, u_2, u_3$ là $3$ số hạng liên tiếp của CSN với công bội $q$.\\
		Theo đề $u_1=a_1, u_2=a_4, u_3=a_{25}$ với $a_1,a_4, a_{25}$ là 3 số hạng của CSC với công sai $d$.\\
		Ta có $\heva{&a_4=a_1+3d\\&a_{25}=a_1+24d}\Rightarrow 8a_4-a_{25}=7a_1\Leftrightarrow 8u_2-u_3=7u_1\Leftrightarrow 8u_1q-u_1q^2=7u_1$\\
		\hspace*{4.2cm}$\Leftrightarrow q^2-8q+7=0 \Leftrightarrow q=1 (L) \vee q=7 (N)$\\
		Theo đề ta cũng có $u_1+u_2+u_3=114\Leftrightarrow u_1+u_1q+u_1q^2=114\Rightarrow u_1=2$\\
		Vậy $3$ số cần tìm là $2; 14; 98$.
	}	
\end{bt}

\begin{bt}%[DCHT Toán 11 - KNTT -Đỗ Chí Tâm] %[1K2G7-6]
	Ba số khác nhau có tổng là $217$ có thể coi là các số hạng liên tiếp của một CSN hoặc là các số hạng thứ $2$ thứ $9$ và thứ $44$ của một CSC. Tìm 3 số đó. 
	\dapso{$7; 35; 175$}
	\loigiai{
		Gọi $u_1, u_2, u_3$ là $3$ số hạng liên tiếp của CSN với công bội $q$.\\
		Theo đề $u_1=a_2, u_2=a_9, u_3=a_{44}$ với $a_2,a_9, a_{44}$ là 3 số hạng của CSC với công sai $d$.\\
		Ta có $\heva{&a_9=a_2+7d\\&a_{44}=a_2+42d}\Rightarrow 6a_9-a_{44}=5a_2\Leftrightarrow 6u_2-u_3=5u_1\Leftrightarrow 6u_1q-u_1q^2=5u_1$\\
		\hspace*{4.2cm}$\Leftrightarrow q^2-6q+5=0 \Leftrightarrow q=1 (L) \vee q=5 (N)$\\
		Theo đề ta cũng có $u_1+u_2+u_3=217\Leftrightarrow u_1+u_1q+u_1q^2=217\Rightarrow u_1=7$\\
		Vậy $3$ số cần tìm là $7; 35; 175$.
	}	
\end{bt}

% \begin{bt}%[DCHT Toán 11 - KNTT -Đỗ Chí Tâm] %[1K2K7-6]
% 	$a,b,c$ là ba số hạng liên tiếp của một CSN và $a,b+2,c+9$ là ba số hạng liên tiếp của một CSC, đồng thời $a,b+2,c$ là ba số hạng liên tiếp của một CSN khác. Tìm $a$.
% 	\dapso{$m=\dfrac{-7\pm3\sqrt{5}}{2}$}
% 	\loigiai{
% 		Vì $a,b,c$ là ba số hạng liên tiếp của CSN, ta có $ac=b^2\,\,\,\, (1)$\\
% 		Vì $a,b+2,c+9$ là ba số hạng liên tiếp của CSC, ta có $a+(c+9)=2(b+2)\,\,\,\, (2)$\\
% 		Vì $a,b+2,c$ là ba số hạng liên tiếp của CSN, ta có $a.c=(b+2)^2\,\,\,\, (3)$\\
% 		Thế $(1)$ vào $(3)$, ta được $b^2=(b+2)^2\Leftrightarrow b=-1$\\
% 		Thay $b=-1$ vào $(1), (2)$, ta được $\heva{&ac=1\\&a+c=-7}\Leftrightarrow a=\dfrac{-7+3\sqrt{5}}{2}\vee a=\dfrac{-7-3\sqrt{5}}{2}$
% 	}	
% \end{bt}

% \begin{bt}%[DCHT Toán 11 - KNTT -Đỗ Chí Tâm] %[1K2G7-6]
% 	Một CSC và CSN có cùng các số hạng thứ $m+1$, thứ $n+1$, thứ $p+1$ và $3$ số hạng này là $3$ số dương $a, b, c$. Tính $T=a^{b-c}.b^{c-a}.c^{a-b}$.
% 	\dapso{$m=1$}
% 	\loigiai{
% 		$a=u_1+md=q^m.v_1$\\
% 		$b=u_1+nd=q^n.v_1$\\
% 		$c=u_1+pd=q^p.v_1$\\
% 		Suy ra $T=a^{b-c}.b^{c-a}.c^{a-b}=\left(q^mv_1 \right)^{(n-p)d}. \left(q^nv_1 \right)^{(p-m)d}. \left(q^pv_1 \right)^{(m-n)d}=1$
% 	}	
% \end{bt}

% \begin{bt}%[DCHT Toán 11 - KNTT -Đỗ Chí Tâm] %[1K2G7-6]
% 	Tìm $m$ dương để phương trình $x^3+(5-m)x^2+(6-5m)x-6m=0 \,\,\,(*)$ có $3$ nghiệm phân biệt lập thành cấp số nhân.
% 	\dapso{$m=\sqrt{6}$}
% 	\loigiai{
% 		$(*)\Leftrightarrow (x+2) \left(x^2+(3-m)x-3m \right)=0\Leftrightarrow x=-2 \vee x=-3 \vee x=m$.\\
% 		Để $(*)$ có 3 nghiệm phân biệt thì $m\ne -3$ và $m\ne -2$.\\
% 		Do $3$ nghiệm này lập thành cấp số nhân, ta sắp xếp các nghiệm này theo thứ tự tăng dần được các dãy số sau
% 		\begin{itemize}
% 			\item $-3;-2; m$ lập thành CSN $\Leftrightarrow$ $-3m=(-2)^2\Leftrightarrow m=-\dfrac{4}{3}$
% 			\item $-3; m; -2$ lập thành CSN $\Leftrightarrow -3(-2)=m^2\Leftrightarrow m=\pm \sqrt{6}$
% 			\item $m; -3; -2$ lập thành CSN $\Leftrightarrow m(-2)=(-3)^2\Leftrightarrow m=-\dfrac{9}{2}$
% 		\end{itemize}	
% 		So với điều kiện thì $m=\sqrt{6}$ thỏa yêu cầu bài toán.
% 	}	
% \end{bt}
% \begin{bt}%[DCHT Toán 11 - KNTT -Đỗ Chí Tâm] %[1K2G7-6]
% 	Tìm tham số $m$ để phương trình $x^3-(2m+1)x^2+2mx=0 \,\, (*)$ có $3$ nghiệm phân biệt lập thành một cấp số cộng, biết $m<0$.
% 	\dapso{$m=-\dfrac{1}{2}$}
% 	\loigiai{
% 		$(*)\Leftrightarrow x.\left(x^2-(2m+1)x+2m \right)=0\Leftrightarrow x=0 \vee x=1 \vee x=2m$.\\
% 		Để $(*)$ có 3 nghiệm phân biệt thì $m\ne 0$ và $m\ne \dfrac{1}{2}$.\\
% 		Do $3$ nghiệm này lập thành cấp số cộng, ta sắp xếp các nghiệm này theo thứ tự tăng dần được các dãy số sau
% 		\begin{itemize}
% 			\item $2m;0; 1$ lập thành CSC $\Leftrightarrow$ $2m+1=2.0\Leftrightarrow m=-\dfrac{1}{2}$
% 			\item $0; 2m; 1$ lập thành CSC $\Leftrightarrow 0+1=4m\Leftrightarrow m=\dfrac{1}{4}$
% 			\item $0; 1; 2m$ lập thành CSC $\Leftrightarrow 0+2m=2.1\Leftrightarrow m=1$
% 		\end{itemize}	
% 		Vậy $m=-\dfrac{1}{2}$ là giá trị cần tìm.
% 	}	
% \end{bt}
% \subsubsection{Câu hỏi trắc nghiệm}
% \Opensolutionfile{ans}[ans/ans-1K2-3-Dang6]
% \begin{ex}%[1K2K7-6]
% 	Các số $x+6y$, $5x+2y$, $8x+y$ theo thứ tự đó lập thành một cấp số cộng, đồng thời các số $x-1$, $y+2$, $x-3y$ theo thứ tự đó lập thành một cấp số nhân. Tính $x^2+y^2$.
% 	\choice
% 	{$x^2+y^2=25$}
% 	{\True $x^2+y^2=40$}
% 	{$x^2+y^2=100$}
% 	{$x^2+y^2=10$}
% 	\loigiai{
% 		Theo bài ra, ta có
% 		\[ \heva{& (x+6 y)+(8 x+y)=2(5 x+2 y) \\ & (y+2)^2=(x-1)(x-3y)} \Rightarrow \heva{& x=3y \\ & (y+2)^2=0}\Rightarrow \heva{& x=-6 \\ & y=-2}\Rightarrow x^2+y^2=40. \]
% 	}
% \end{ex}

% \begin{ex}%[1K2K7-6]
% 	Cho hai số dương $ a $ và $ b $ không vượt quá $ 10 $ sao cho $ a-b $; $ 2 $; $ b $ theo thứ tự tạo thành một cấp số cộng và $ a+b $; $ 3a-2b $; $ 5a $ theo thứ tự lập thành một cấp số nhân. Tính giá trị của $ S=a+b $.
% 	\choice
% 	{$ S=8 $}
% 	{$ S=20 $}
% 	{$ S=7 $}
% 	{\True $ S=5 $}
% 	\loigiai{
% 		Vì $ a-b $; $ 2 $; $ b $ theo thứ tự tạo thành một cấp số cộng nên $ 2-(a-b)=b-2 \Leftrightarrow a=4 $.\\
% 		Vì $ a+b $; $ 3a-2b $; $ 5a $ theo thứ tự lập thành một cấp số nhân nên $$ \dfrac{3a-2b}{a+b}=\dfrac{5a}{3a-2b}\Leftrightarrow \dfrac{12-2b}{4+b}=\dfrac{20}{12-2b}\Leftrightarrow \hoac{&b=1\\&b=16 \text{ (loại).}} $$
% 		Vậy $ S=a+b=4+1=5. $
% 	}
% \end{ex}
% \begin{ex}%[1K2K7-6]
% 	Số hạng thứ hai, số hạng đầu và số hạng thứ ba của một cấp số cộng với công sai khác 0 theo thứ tự đó lập thành một cấp số nhân với công bội $q$. Tìm $q$.
% 	\choice
% 	{$q=2$}
% 	{\True $q=-2$}
% 	{$q=\dfrac{3}{2}$}
% 	{$q=-\dfrac{3}{2}$}
% 	\loigiai {
% 		Giả sử ba số hạng $a;b;c$ lập thành cấp số cộng thỏa yêu cầu, khi đó $b;a;c$ theo thứ tự đó lập thành cấp số nhân công bội $q$. Ta có\\
% 		$\heva{
% 			& a+c=2b \\
% 			& a=bq;\,c=b{{q}^2} \\
% 		}\Rightarrow bq+b{{q}^2}=2b\Leftrightarrow \hoac{
% 			& b=0 \\
% 			& {{q}^2}+q-2=0. \\
% 		}$ \\
% 		Nếu $b=0\Rightarrow a=b=c=0$ nên $a;b;c$ là cấp số cộng công sai $d=0$ (vô lí).\\
% 		Nếu ${{q}^2}+q-2=0\Leftrightarrow q=1$ hoặc $q=-2$ Nếu $q=1\Rightarrow a=b=c$ (vô lí), do đó $q=-2$.}
% \end{ex}

% \begin{ex}%[1K2K7-6]
% 	Cho ba số $a$, $b$, $c$ theo thứ tự tạo thành cấp số nhân với công bội khác $1$. Biết cũng theo thứ tự đó chúng lần lượt là số hạng thứ nhất, thứ tư và thứ tám của một cấp số cộng công sai là $s\neq 0$. Tính $\dfrac{a}{s}$.
% 	\choice
% 	{$3$}
% 	{$\dfrac{4}{9}$}
% 	{\True $9$}
% 	{$\dfrac{4}{3}$}
% 	\loigiai{
% 		Rõ ràng $a\neq 0$. Vì $s$ là công sai cấp số cộng nên $a$, $a+3s$, $a+7s$ lập thành cấp số nhân, do đó
% 		$$a(a+7s)=(a+3s)^2 \Leftrightarrow 9s^2-as=0\Leftrightarrow \hoac{& s=0\quad\text{(loại)}\\ & a=9s}\Leftrightarrow \dfrac{a}{s}=9.$$
% 	}
% \end{ex}

% \begin{ex}%[1K2K7-6]
% 	Xét các số thực dương $a, b$ sao cho $-25$, $2a$, $3b$ là cấp số cộng và $2$, $a+2$, $b-3$ là cấp số nhân. Khi đó $a^2 + b^2-3ab$ bằng
% 	\choice
% 	{$76$}
% 	{$89$}
% 	{$31$}
% 	{\True $59$}
% 	\loigiai{
% 		$-25,\,2a,\,3b$ là cấp số cộng $ \Leftrightarrow 2.2a=-25+3b \Leftrightarrow b=\dfrac{1}{3}(4a+25)$.
% 		\begin{eqnarray*}
% 			& 2,a+2,b-3 \text{ là cấp số nhân}& \Leftrightarrow (a+2)^2=2(b-3)\\
% 			& & \Leftrightarrow (a+2)^2=2\left[\dfrac{1}{3}(4a+25)-3\right]\\
% 			& & \Leftrightarrow 3a^2+4a-20=0 \\
% 			& & \Leftrightarrow \hoac{&a=2 \\&a=-\dfrac{10}{3}\,(\text{loại}).}
% 		\end{eqnarray*}
% 		Suy ra $b=11 \Rightarrow a^2+b^2-3ab=59$.
% 	}
% \end{ex}
% \begin{ex}%[1K2K7-6]
% 	Cho ba số $x$; $5$; $2y$ theo thứ tự lập thành cấp số cộng và ba số $x$; $4$; $2y$ theo thứ tự lập thành cấp số nhân thì $|x-2y|$ bằng
% 	\choice
% 	{$10$}
% 	{$8$}
% 	{$9$}
% 	{\True $6$}
% 	\loigiai{
% 		Theo giả thiết ta có
% 		\begin{itemize}
% 			\item Do $x;5;2y$ theo thứ tự lập thành một cấp số cộng nên ta có $5=\dfrac{x+2y}{2}\quad\quad (1)$.
% 			\item Do $x;4;2y$ theo thứ tự lập thành một cấp số nhân nên ta có $x\cdot 2y=4^2\quad\quad (2)$.
% 		\end{itemize}
% 		Từ $(1)$ và $(2)$ ta có $\heva{&x+2y=10\\&x\cdot 2y =16}\Rightarrow \hoac{& x=2,2y=8\\&x=8,2y=2}\Rightarrow \left|x-2y\right|=6$.
% 	}
% \end{ex}

% \begin{ex}%[1K2K7-6]
% 	Cho dãy số tăng $a, b, c\,\,(c\in \mathbb{Z} )$ theo thứ tự lập thành cấp số nhân; đồng thời $a,b+8,c$ theo thứ tự lập thành cấp số cộng và $a, b+8, c+64$ theo thứ tự lập thành cấp số nhân. Tính giá trị biểu thức $P=a-b+2c$.
% 	\choice
% 	{$P=\dfrac{184}{9}$}
% 	{\True $P=64$}
% 	{$P=\dfrac{92}{9}$}
% 	{$P=32$}
% 	\loigiai{
% 		Ta có $\heva{
% 			& ac=b^2 \\
% 			& a+c=2(b+8 ) \\
% 			& a(c+64 )={{(b+8 )}^2} \\
% 		}\Leftrightarrow \heva{
% 			& ac=b^2\quad \quad(1 ) \\
% 			& a-2b=16-c\quad(2 ) \\
% 			& ac+64a=(b+8 )^2\quad(3 )\\
% 		}$.
% 		Thay $(1)$ vào $(3)$ ta được: $$b^2+64a=b^2+16b+64\Leftrightarrow 4a-b=4. \quad
% 		(4 )$$
% 		Kết hợp $(2)$ với $(4)$ ta được: $\heva{
% 			& a-2b=16-c \\
% 			& 4a-b=4 \\
% 		}\Leftrightarrow \heva{
% 			& a=\dfrac{c-8}{7} \\
% 			& b=\dfrac{4c-60}{7}. \\
% 		}\,\,\,\,\,(5 )$ \\
% 		Thay $(5)$ vào $(1)$ ta được:\\
% 		$7(c-8 )c={{(4c-60 )}^2}\Leftrightarrow 9{{c}^2}-424c+3600=0\Leftrightarrow \hoac{
% 			& c=36 \\
% 			& c=\dfrac{100}{9} \\
% 		}\Leftrightarrow c=36\,\,(c\in \mathbb{Z} )$. \\
% 		Với $c=36\Rightarrow a=4,\,\,b=12\Rightarrow P=4-12+72=64$.}
% \end{ex}

% \begin{ex}%[1K2K7-6]
% 	Cho bốn số $a$, $b$, $c$, $d$ theo thứ tự đó tạo thành cấp số nhân với công bội khác $1$. Biết tổng ba số hạng đầu bằng $\dfrac{148}{9}$, đồng thời theo thứ tự đó $a$, $b$, $c$ lần lượt là số hạng thứ nhất, thứ tư và thứ tám của một cấp số cộng. Tính giá trị của biểu thức $T=a-b+c-d$.
% 	\choice
% 	{$T=-\dfrac{101}{27}$}
% 	{$T=\dfrac{100}{27}$}
% 	{\True $T=-\dfrac{100}{27}$}
% 	{$T=\dfrac{101}{27}$}
% 	\loigiai{
% 		Gọi $s$ ($s\neq 0$) là công sai của cấp số cộng. Vì $a$, $b$, $c$ theo thứ tự lần lượt là số hạng thứ nhất, thứ tư và thứ tám của cấp số cộng nên $b=a+3s$ và $c=a+7s$.\\
% 		Mặt khác, $a$, $b$, $c$ theo thứ tự tạo thành cấp số nhân với công bội khác $1$ nên
% 		\[ac=b^2 \Leftrightarrow a(a+7s)=(a+3s)^2 \Leftrightarrow as=9s^2 \Leftrightarrow a=9s \,\,(\text{vì } s\neq 0).\]
% 		Suy ra $b=12s$, $c=16s$.\\
% 		Theo giả thiết
% 		\[a+b+c=\dfrac{148}{9} \Leftrightarrow 9s+12s+16s=\dfrac{148}{9} \Leftrightarrow 37s=\dfrac{148}{9} \Leftrightarrow s=\dfrac{4}{9}.\]
% 		Suy ra $a=4$, $b=\dfrac{16}{3}$, $c=\dfrac{64}{9}$. Từ đó ta tính được $d=4\cdot \left(\dfrac{4}{3}\right)^3 = \dfrac{256}{27}$.\\
% 		Vậy $T=a-b+c-d = 4-\dfrac{16}{3}+\dfrac{64}{9}-\dfrac{256}{27} = -\dfrac{100}{27}$.
% 	}
% \end{ex}
% \begin{ex}%[1K2K7-6]
% 	Cho $x$ và $y$ là các số nguyên thỏa mãn các số $x+6y$ ,$5x+2y$, $8x+y$ theo thứ tự lập thành cấp cộng và các số $x-\dfrac{5}{3}y$, $y-1$, $2x-3y$ theo thứ tự lập thành cấp số nhân. Tính tổng $S=2x+3y$.
% 	\choice
% 	{$9$}
% 	{$6$}
% 	{$-6$}
% 	{\True $-9$}
% 	\loigiai{
% 		Vì các số $x+6y$ ,$5x+2y$, $8x+y$ theo thứ tự lập thành cấp cộng nên ta có
% 		$$ (x+6y)+(8x+y)=2(5x+2y)\Leftrightarrow x=3y. $$
% 		Vì các số $x-\dfrac{5}{3}y$, $y-1$, $2x-3y$ theo thứ tự lập thành cấp số nhân nên ta có
% 		$$ \left(x-\dfrac{5}{3}y\right) (2x-3y)=(y-1)^2.  $$
% 		Thay $x=3y$ vào phương trình trên, ta được
% 		\begin{eqnarray*}
% 			& & \left(3y-\dfrac{5}{3}y\right) (6y-3y)=(y-1)^2\\
% 			&\Leftrightarrow & 4y^2=y^2-2y+1\\
% 			&\Leftrightarrow & \hoac{& y=-1\\& y=\dfrac{1}{3}}.
% 		\end{eqnarray*}
% 		Ta loại trường hợp $y=\dfrac{1}{3}$ vì $y$ là số nguyên. Suy ra $x=3y=3(-1)=-3$. Vậy $$S=2x+3y=2(-3)+3(-1)=-9.$$
% 	}
% \end{ex}
\Closesolutionfile{ans}
% \begin{indapan}{10}
% 	{ans/ans-1K2-3-Dang6}
% \end{indapan}
\begin{dang}{Bài toán thực tế}
	\textit{Bài toán lãi kép:} Một người gửi tiết kiệm vào ngân hàng một số tiền $A$ với lãi suất $r\%$ mỗi kì hạn. Số tiền lãi sẽ được nhập vào vốn ban đầu để tính lãi cho kì hạn tiếp theo. Hỏi sau $n$ kì hạn thì người đó có tất cả bao nhiêu tiền?\\
	\textit{Lời giải:} Gọi $u_n$ là số tiền người đó có sau $n$ kì hạn. Ta có:
	\begin{itemize}
		\item Số tiền người đó có sau kì hạn thứ nhất là: $u_1=A+A\cdot r\%=A\left(1+r\%\right)$.
		\item Số tiền người đó có sau $n$ kì hạn là: $u_n=u_{n-1}+u_{n-1}\cdot r\%=u_n\left(1+r\%\right)$.
	\end{itemize}
	Suy ra dãy số $(u_n)$ là một cấp số nhân với số hạng đầu $u_1=A\left(1+r\%\right)$ và công bội $q=1+r\%$.\\
	Vậy số tiền người đó có sau $n$ kì hạn là: \fbox{$u_n=A\left(1+r\%\right)^n$}.
\end{dang}
\subsubsection{Ví dụ minh hoạ}
\begin{vd}%[VD]%[DCHT Toán 11 - KNTT -Tên Huỳnh Thanh Chí]%[1K2K7-2]
	Trong một lọ nuôi cấy vi khuẩn, ban đầu có $ 5\ 000 $ con vi khuẩn và số lượng vi khuẩn tăng lên thêm $ 8\% $ mỗi giờ. Hỏi sau $ 5 $ giờ thì số lượng vi khuẩn là bao nhiêu?
	\dapso{}
	\loigiai{
	Ta có $ A=5\ 000 $ là số lượng vi khuẩn ban đầu, $ r=8\%=0{,}08 $ là tỉ lệ gia tăng vi khuẩn sau một giờ.
	\begin{itemize}
	\item Tại thời điểm sau $ 1 $ giờ: $ u_1=5000+5000\cdot 0{,}08= 5\ 000\cdot(1{,}08)$.
	\item Tại thời điểm sau $ n $ giờ: $ u_n=u_{n-1}+u_{n-1}\cdot0{,}08=u_{n-1}\cdot (1{,}08)$.
	\end{itemize}
	Do đó ta có thể nhận thấy rằng, số lượng vi khuẩn ở thời gian $ n $ giờ là một cấp số nhân có số hạng đầu $ u_1=5000.1,08 $ và công bội $ q=1{,}08 $.\\
	% Suy ra số hạng tổng quát $ u_n=5\ 000\cdot (1{,}08)^{n} $.\\
	Vậy số lượng vi khuẩn sau $ 5 $ giờ là $ u_5=5\ 000\cdot (1{,}08)^{5}\approx 7346 $ (vi khuẩn).
	}
\end{vd}
\begin{vd}%[VD]%[1K2K3-3]
	Người ta thiết kế một cái tháp gồm $10$ tầng theo cách: Diện tích bề mặt trên của mỗi tầng bằng nửa diện tích bề mặt trên của tầng ngay bên dưới và diện tích bề mặt của tầng 1 bằng nửa diện tích bề mặt đế tháp. Biết diện tích bề mặt đế tháp là $12\, 288$ m$^2$, tính diện tích bề mặt trên cùng của tháp.
	\loigiai{
		Gọi $S$ là diện tích mặt đế và $T_1, T_2, \ldots, T_{10}$ là diện tích bề mặt của tầng 1, tầng 2, \ldots, tầng 10.\\
		Khi đó, ta có
		\allowdisplaybreaks
		\begin{eqnarray*}
			&T_1&=\dfrac{1}{2}\cdot S;\\
			&T_n&=\dfrac{1}{2}\cdot T_{n-1}
		\end{eqnarray*}
		Suy ra $\left(T_n\right)$ là cấp số nhân có số hạng đầu $T_1=\dfrac{1}{2}\cdot 12288=6144$ m$^2$ và công bội $q=\dfrac{1}{2}$.\\
		Vậy diện tích bề mặt trên cùng của tháp là $T_{10}=\dfrac{1}{2^{10}}\cdot 12288=12$ m$^2$.
	}
\end{vd}
\begin{vd}%[TH] %[DCHT Toán 11 - KNTT - Dung Phuong] %[1K2B7-7]
	Dân số trung bình của Việt Nam năm $2020$ là $97{,}6$ triệu người, tỉ lệ tăng dân số là $1{,}14 \% /$năm.
	\begin{flushright}
		\textit{(Nguồn: Niên giám thống kê của Việt Nam năm 2020, NXB Thống kê, 2021)}
	\end{flushright}
	Giả sử tỉ lệ tăng dân số không đổi qua các năm.
	\begin{enumerate}
		\item Sau 1 năm, dân số của Việt Nam sẽ là bao nhiêu triệu người (làm tròn kết quả đến hàng phần mười)?
		\item Viết công thức tính dân số Việt Nam sau $n$ năm kể từ năm $2020$.
	\end{enumerate}
	\dapso{$\approx 98{,}7$ triệu người}
	\loigiai{
		\begin{enumerate}
			\item Sau 1 năm, dân số của Việt Nam sẽ là
			\allowdisplaybreaks
			\begin{eqnarray*}
				u_1&=&97{,}6+97{,}6 \cdot 0{,}0114=97{,}6 \cdot(1+0{,}0114)\\
				&=&97{,}6 \cdot 1{,}0114 \approx 98{,}7 (\text{triệu người}).
			\end{eqnarray*} 
			\item Gọi $u_n$ là dân số của Việt Nam sau $n$ năm.\\
			Do tỉ lệ tăng dân số hàng năm là $1{,}14 \%$ nên ta có
			\allowdisplaybreaks
			\begin{eqnarray*}
				u_n &=&u_{n-1}+u_{n-1} \cdot 0{,}0114=u_{n-1} \cdot(1+0{,}0114) \\
				&=&u_{n-1} \cdot 1{,}0114\ \text{với}\ n \geq 2.
			\end{eqnarray*}
			Do đó, $\left(u_n\right)$ là cấp số nhân có số hạng đầu $u_1=97{,}6\cdot 1{,}0114$, công bội $q=1{,}0114$.\\
			Vậy dân số của Việt Nam sau $n$ năm kể từ năm $2020$ là
			\[u_n=97{,}6 \cdot 1{,}0114 \cdot 1{,}0114^{n-1}=97{,}6 \cdot 1{,}0114^n\ (\text{triệu người}). \]
		\end{enumerate}
	}
\end{vd}

\begin{vd}%[TH] %[DCHT Toán 11 - KNTT - Dung Phuong] %[1K2B7-7]
	Bác Linh gửi vào ngân hàng $100$ triệu đồng tiền tiết kiệm với hình thức lãi kép, kì hạn 1 năm với lãi suất $6 \% /$năm. Viết công thức tính số tiền (cả gốc và lãi) mà bác Linh có được sau $n$ năm (giả sử lãi suất không thay đổi qua các năm).
	\dapso{$100 \cdot 1{,}06^{n-1}$ triệu đồng}
	\loigiai{
		Gọi $u_n$ là số tiền (cả gốc lẫn lãi) mà bác Linh có được sau $n$ năm.\\
		Do lãi suất 1 năm là $6\%$ nên ta có
		\allowdisplaybreaks
		\begin{eqnarray*}
			u_n &=&u_{n-1}+u_{n-1} \cdot 0{,}06=u_{n-1} \cdot(1+0{,}06) \\
			&=&u_{n-1} \cdot 1{,}06\ \text{với}\ n \geq 2.
		\end{eqnarray*}
		Do đó, $\left(u_n\right)$ là cấp số nhân có số hạng đầu $u_1=100\cdot 1{,}06$ (triệu đồng), công bội $q=1{,}06$.\\
		Vậy số tiền mà bác Linh có được sau $n$ năm là
		\[u_n=100 \cdot 1{,}06^{n}\ (\text{triệu đồng}). \]
	}
\end{vd}
\begin{vd}%[VD] %[DCHT Toán 11 - KNTT - Dung Phuong] %[1K2K7-7]
	Một hình vuông có cạnh $1$ đơn vị dài được chia thành chín hình vuông nhỏ hơn và hình vuông ở chính giữa được tô màu xanh như hình. Mỗi hình vuông nhỏ hơn lại được chia thành chín hình vuông con, và mỗi hình vuông con ở chính giữa lại được tô màu xanh. Nếu quá trình này được tiếp tục lặp lại năm lần, thì tổng diện tích các hình vuông được tô màu xanh là bao nhiêu?
	\dapso{$\dfrac{26281}{39366}$}
	\begin{center}
		\begin{tikzpicture}[>=stealth,thick,scale=0.7]
			\def\n{1}
			\def\a{3}
			\pgfmathsetmacro{\m}{int(3^(\n))}
			\def\hv#1{
				\ifnum#1>0
				\fill[blue!50] (-\a/3,-\a/3) rectangle (\a/3,\a/3);
				\pgfmathtruncatemacro{\k}{#1-1}
				\foreach \i in {0,...,3}{\begin{scope}[shift={(90*\i:2)},scale=1/3]\hv{\k}\end{scope}}
				\foreach \i in {0,...,3}{\begin{scope}[shift={(45+90*\i:{4/sqrt(2)})},scale=1/3]\hv{\k}\end{scope}}
				\fi
			}
			\draw(-\a,-\a) rectangle (\a,\a);
			\hv{\n}
			\foreach \i in {0,1,...,\m}{
				\draw[blue!50] 
				({-\a+2*\i *\a/\m},\a)--++(270:2*\a)
				(\a,{-\a+2*\i *\a/\m})--++(180:2*\a)
				;
			}
		\end{tikzpicture}
		\begin{tikzpicture}[>=stealth,thick,scale=0.7]
			\def\n{2}
			\def\a{3}
			\pgfmathsetmacro{\m}{int(3^(\n))}
			\def\hv#1{
				\ifnum#1>0
				\fill[blue!50] (-\a/3,-\a/3) rectangle (\a/3,\a/3);
				\pgfmathtruncatemacro{\k}{#1-1}
				\foreach \i in {0,...,3}{\begin{scope}[shift={(90*\i:2)},scale=1/3]\hv{\k}\end{scope}}
				\foreach \i in {0,...,3}{\begin{scope}[shift={(45+90*\i:{4/sqrt(2)})},scale=1/3]\hv{\k}\end{scope}}
				\fi
			}
			\draw(-\a,-\a) rectangle (\a,\a);
			\hv{\n}
			\foreach \i in {0,1,...,\m}{
				\draw[blue!50] 
				({-\a+2*\i *\a/\m},\a)--++(270:2*\a)
				(\a,{-\a+2*\i *\a/\m})--++(180:2*\a)
				;
			}
		\end{tikzpicture}
		\begin{tikzpicture}[>=stealth,thick,scale=0.7]
			\def\n{4}
			\def\a{3}
			\def\hv#1{
				\ifnum#1>0
				\fill[blue!50] (-\a/3,-\a/3) rectangle (\a/3,\a/3);
				\pgfmathtruncatemacro{\k}{#1-1}
				\foreach \i in {0,...,3}{\begin{scope}[shift={(90*\i:2)},scale=1/3]\hv{\k}\end{scope}}
				\foreach \i in {0,...,3}{\begin{scope}[shift={(45+90*\i:{4/sqrt(2)})},scale=1/3]\hv{\k}\end{scope}}
				\fi
			}
			\draw (-\a,-\a) rectangle (\a,\a);
			\hv{\n}
		\end{tikzpicture}
	\end{center}
	\loigiai{
		Lần phân chia thứ nhất, $1$ hình vuông thành $9$ hình vuông con, diện tích hình vuông tô màu xanh là $u_1=\dfrac{1}{9}$.\\
		Lần phân chia thứ hai, $8$ hình vuông thành $9$ hình vuông con, diện tích hình vuông tô màu xanh tăng thêm là $u_2=\dfrac{1}{9}\left(\dfrac{8}{9}\right)$.\\
		Lần phân chia thứ ba, $8^2$ hình vuông thành $9$ hình vuông con, diện tích hình vuông tô màu xanh tăng thêm là $u_3=\dfrac{1}{9}\left(\dfrac{8}{9}\right)^2$.\\
		Lần phân chia thứ tư, $8^3$ hình vuông thành $9$ hình vuông con, diện tích hình vuông tô màu xanh tăng thêm là $u_4=\dfrac{1}{9}\left(\dfrac{8}{9}\right)^3$.\\
		Lần phân chia thứ năm, $8^4$ hình vuông thành $9$ hình vuông con, diện tích hình vuông tô màu xanh tăng thêm là $u_5=\dfrac{1}{9}\left(\dfrac{8}{9}\right)^4$.\\
		Như vậy diện tích các hình vuông tăng thêm sau mỗi lần chia  tạo thành cấp số nhân có công bội là $q=\dfrac{8}{9}$, số hạng đầu là $u_1=\dfrac{1}{9}$.\\
		Do đó, tổng diện tích hình vuông tô màu xanh sau $5$ lần chia là\\
		\[u_1+u_2+u_3+u_4+u_5=\dfrac{1-q^5}{1-q}\cdot u_1=\dfrac{1-\left(\dfrac{8}{9}\right)^5}{1-\dfrac{8}{9}}\cdot \dfrac{1}{9}=\dfrac{26281}{39366}.\]	
		
	}
\end{vd}

\begin{vd}%[TH] %[DCHT Toán 11 - KNTT - Dung Phuong] %[1K2B7-7]
	Một khay nước có nhiệt độ $23^\circ$ được đặt vào ngăn đá của tủ lạnh. Biết sau mỗi giờ, nhiệt độ của nước giảm $20\%$. Tính nhiệt độ của khay nước đó sau $6$ giờ theo đơn vị độ $C$.
	\dapso{$ \approx 7,5^\circ $ gam}
	\loigiai{
		Nhiệt độ sau mỗi giờ của khay nước theo thứ tự lập thành cấp số nhân với $u_1=23.(1-20\%)$ và $q=(1-20\%)$.\\
		Ta có $u_6=u_1.q^5=23.(1-20\%)^6 \approx 7,5$.\\
		Nhiệt độ của khay nước sau $6$ giờ là $ \approx 6,0^\circ $.
	}
	
\end{vd}
\begin{vd}%[TH] %[DCHT Toán 11 - KNTT - Dung Phuong] %[1K2B7-7]
	Chu kì bán rã của nguyên tố phóng xạ poloni $210$ là $138$ ngày, nghĩa là sau $138$ ngày, khối lượng của nguyên tố đó chi còn một nửa (theo: https://vi.wikipedia.org/wiki/Poloni-210). Tính khối lượng còn lại của $20$ gam poloni $210$ sau:
	\begin{listEX}[2]
		\item [a)]  $690$ ngày;
		\item [b)] $7314$ ngày (khoảng $20$ năm).
	\end{listEX}
	\dapso{$\dfrac{20}{2^{53}}$ gam}
	\loigiai{
		\begin{listEX}[1]
			\item [a)] Ta có $\dfrac{690}{138}=5$ suy ra khối lượng còn lại sau 690 này là $\dfrac{20}{2^5}=0{,}625$ gam;
			\item [b)] Ta có $\dfrac{7314}{138}=53$ suy ra khối lượng còn lại sau 7314 này là $\dfrac{20}{2^{53}}$ gam.
		\end{listEX}
	}
\end{vd}	
\begin{vd}%[TH] %[DCHT Toán 11 - KNTT - Dung Phuong] %[1K2K7-7]
	Tế bào E.Coli trong điều kiện nuôi cấy thích hợp cứ $20$ phút lại phân đôi một lần. Hỏi sau $24$ giờ, tế bào ban đầu sẽ phân chia thành bao nhiếu tế bào?
	\dapso{$2^{72}$}
	\loigiai{
		Lần phân chia thứ nhất, $1$ tế bào thành $2$ tế bào, số tế bào lần $1$ phân chia là $u_1 = 2$.\\
		Lần phân chia thứ hai $2$, số tế bào lần $2$ phân chia là  $u_2=2\cdot 2 = u_1 \cdot 2$.\\
		Lần phân chia thứ $3$ có  $4$ tế bào phân chia, số tế bào lần $3$ phân chia là $u_3=2\cdot u_2$.\\
		Như vậy một tế bào phân đôi sẽ tạo thành cấp số nhân có công bội là $2$, số hạng đầu là $u_1=2$.\\
		Sau $n$ lần phân chia từ một tế bào phân được thành $u_n=2^{n-1}u_1$.\\
		Đổi $24$ giờ $=24 \cdot 60 =  72 \cdot 20$ (phút)  $\Rightarrow 24$ giờ gấp $72$ lần $20$ phút. \\
		Do đó, sau $24$ giờ số tế bào nhận được là $u_{72}=2^{71}\cdot 2 = 2^{72}$ (tế bào).
	}
\end{vd}

\subsubsection{Bài tập tự luận}
 


\begin{bt}%[TH] %[DCHT Toán 11 - KNTT - Dung Phuong] %[1K2B7-7]
	Một quốc gia có dân số năm 2011 là $P$ triệu người. Trong $10$ năm tiếp theo, mỗi năm dân số tăng $a \%$. Chứng minh rằng dân số các năm từ năm 2011 đến năm 2021 của quốc gia đó tạo thành cấp số nhân. Tìm công bội của cấp số nhân này.
	\loigiai{
		Coi ngày điều tra dân số năm 2011 và năm 2021 trùng nhau thì từ năm 2011 đến năm 2021 là 10 năm. Vậy dân số nước ta tính đến năm 2021 là 
		\[u_{10} = P\cdot \left(1+a\%\right)^{10}.\]
		Ta có \[u_{1} = P\cdot \left(1+a\%\right)^{1}.\]
		\[u_{2} = P\cdot \left(1+a\%\right)^{2}.\]
		Và công bội của cáp số nhân này là $\, \dfrac{u_2}{u_1} = q = \dfrac{P\cdot \left(1+a\%\right)^{2}}{P\cdot \left(1+a\%\right)^{1}} = 1+a\%.$
	}
\end{bt}
\begin{bt}%[TH] %[DCHT Toán 11 - KNTT - Dung Phuong] %[1K2B7-7]
	Vào năm 2020, dân số của một quốc gia là khoảng $97$ triệu người và tốc độ tăng trưởng dân số là $0{,}91 \%$. Nếu tốc độ tăng trưởng dân số này được giữ nguyên hằng năm, hãy ước tính dân số của quốc gia đó vào năm 2030.
	\dapso{$106{,}1973784$}
	\loigiai{
		Dân số năm 2021 tăng lên so với năm 2020 là $97 \cdot 0{,}91 \% $ triệu người.\\
		Dân số năm 2021 là 
		\begin{center}
			$97 + 97 \cdot 0{,}91 \% = 97\cdot (1+0{,}91 \%)$ triệu người.
		\end{center}
		Dân số năm 2022 tăng lên so với năm 2021 là $97\cdot (1+0{,}91 \%)\cdot 0{,}91 \% $ triệu người.\\
		Dân số năm 2022 là 
		\begin{center}
			$97\cdot (1+0{,}91 \%) + 97\cdot (1+0{,}91 \%) \cdot 0{,}91 \% = 97\cdot (1+0{,}91 \%)^2$ triệu người.
		\end{center}
		Dân số năm 2023 tăng lên so với năm 2021 là $97\cdot (1+0{,}91 \%)^2\cdot 0{,}91 \% $ triệu người.\\
		Dân số năm 2023 là
		\begin{center}
			$97\cdot (1+0{,}91 \%)^2 + 97\cdot (1+0{,}91 \%)^2\cdot 0{,}91 \% = 97\cdot (1+0{,}91 \%)^3$ triệu người.
		\end{center}
		Tương tự vậy ta có dân số năm 2030 là $97\cdot (1+0{,}91 \%)^{10} = 106{,}1973784$ triệu người.
	}
\end{bt}
\begin{bt}%[TH] %[DCHT Toán 11 - KNTT - Dung Phuong] %[1K2B7-7]
	Một tỉnh có $2$ triệu dân vào năm 2020 với tỉ lệ tăng dân số là $1$ \%/năm. Gọi $u_n$ là số dân của tỉnh đó sau $n$ năm. Giả sử tỉ lệ tăng dân số là không đổi.
	\begin{enumEX}[a)]{1} 
		\item Viết công thức tính số dân của tỉnh đó sau $n$ năm kể từ năm 2020.
		\item Tính số dân của tỉnh đó sau $10$ năm kể từ năm 2020.  
	\end{enumEX}
	\loigiai{
		\begin{enumEX}[a)]{1} 
			\item Với $u_n$ là số dân của tỉnh đó sau $n$ năm. \\
			Ta có $u_1=2 \cdot 1,01$ (triệu dân).\\
			$u_{n+1}=u_n+u_n\cdot 0{,}01 = 1{,}01u_n$. \\
			Do đó, $(u_n)$ là cấp số nhân với số hạng đầu $u_1=2 \cdot 1,01$ và công bội $q=1{,}01$. \\
			Vậy công thức tính số dân của tỉnh đó sau $n$ năm là $u_n=u_1q^{n-1}\Rightarrow u_n=2\cdot 1{,}01^{n}$.
			\item Số dân của tỉnh đó sau $10$ năm kể từ năm 2020 là $u_{10}=2\cdot 1{,}01^10 = 2{,}209$ (triệu dân).
		\end{enumEX}
	}
\end{bt}
\begin{bt}%[TH] %[DCHT Toán 11 - KNTT - Dung Phuong] %[1K2B7-7]
	Giả sử một thành phố có dân số năm 2022 là khoảng $2{,}1$ triệu người và tốc độ gia tăng dân số trung bình mỗi năm là $0{,}75 \%$.
	\begin{listEX}[1]
		\item [a)]  Dự đoán dân số của thành phố đó vào năm $2032$; \dapso{$\approx2262924$ (người)}
		\item [b)]  Nếu tốc độ gia tăng dân số vẫn giữ nguyên như trên thì uớc tính vào năm nào dân số của thành phố đó sẽ tăng gấp đôi so với năm $2022$?  \dapso{$2116$}	\end{listEX}
	
	\loigiai{
		\begin{listEX}
			\item [a)] 	Giả sử dân số năm $2022$ là $u_1=2{,}1\cdot 10^6$ thì dân số năm 
			$2023$ 
			là 
			$u_2=u_1+ 0{,}0075u_1=1{,}0075u_1$.\\
			Tương tự dân số năm $2024$ là $u_3=1{,}0075u_2$.\\
			Do đó dân số của thành phố qua các năm lập thành một cấp số nhân với 
			$u_1=2{,}1\cdot10^6$; $q=1{,}0075$.\\
			Vậy dân số năm $2032$ tương ứng với $u_{11}=u_1\cdot q^{10}=2,1\cdot 
			10^6\cdot1{,}0075^{10}\approx2262924$ (người).
			\item [b)] Giả sử đến năm thứ $n$ thì dân số gấp đôi năm $2022$. \\
			Suy ra 
			$u_n=2u_1 \Leftrightarrow q^{n-1}=2\Leftrightarrow  1{,}0075^{n-1}=2 
			\Leftrightarrow n \approx 93{,}7.$\\
			Vậy $94$ năm sau tức là năm $2116$ thì dân số thành phố sẽ gấp đôi năm $2022$.
	\end{listEX}}
\end{bt}
\begin{bt}%[TH] %[DCHT Toán 11 - KNTT - Dung Phuong] %[1K2B7-7]
	Giả sử anh Tuấn kí hợp đồng lao động trong $10$ năm với điều khoản về tiền lương như sau: Năm thứ nhất, tiền lương của anh Tuấn là $60$ triệu. Kể từ năm thứ hai trở đi, mỗi năm tiền lương của anh Tuấn được tăng lên $8 \%$. Tính tổng số tiền lương anh Tuấn lĩnh được trong $10$ năm đi làm (đơn vị: triệu đồng, làm tròn đến hàng phần nghìn).
	\dapso{$\approx 869{,}194$ triệu người}
	\loigiai{
		Gọi $u_n$ là số tiền lương (triệu đồng) anh Tuấn được lĩnh ở năm làm việc thứ $n$. Ta có: $u_1=60$;
		\[u_n=u_{n-1}+u_{n-1} \cdot 0{,}08=u_{n-1} \cdot(1+0{,}08)=u_{n-1} \cdot 1{,}08. \]
		Do đó, $\left(u_n\right)$ là cấp số nhân có số hạng đầu $u_1=60$, công bội $q=1{,}08$. Áp dụng công thức tính tổng $S_n$, ta có tổng số tiền lương anh Tuấn lĩnh được trong $10$ năm đi làm là
		\[S_{10}=\dfrac{60\cdot\left(1-1{,}08^{10}\right)}{1-1{,}08} \approx 869{,}194\ (\text{triệu người}). \]
	}
\end{bt}
\begin{bt}%[TH] %[DCHT Toán 11 - KNTT - Dung Phuong] %[1K2B7-7]
	Một công ty xây dựng mua một chiếc máy ủi với giá $3$ tỉ đồng. Cứ sau mỗi năm sử dụng, giá trị của chiếc máy ủi này lại giảm $20 \%$ so với giá trị của nó trong năm liền trước đó. Tìm giá trị còn lại của chiếc máy ủi đó sau $5$ năm sử dụng.
	\dapso{$983$ triệu đồng}
	\loigiai{
		Gọi $u_n$ là Giá trị của máy ủi sau $n$ sử dụng. \\
		Dãy số ($u_n$) là một cấp số nhân có $u_1=3.0,8$, $q=0{,}8$.\\
		Số hạng tổng quát của cấp số nhân này là $u_n=3\cdot 0{,}2^{n}$.\\
		Ta có $u_5=3\cdot 0{,}8^5=0{,}98304$.\\
		Tương ứng giá trị của chiếc máy ủi sau $5$ năm xấp xỉ $983$ triệu đồng.
	}
\end{bt}
\begin{bt}%[TH] %[DCHT Toán 11 - KNTT - Dung Phuong] %[1K2B7-7]
	Một gia đình mua một chiếc ô tô giá $800$ triệu đồng. Trung bình sau mỗi năm sử dụng, giá trị còn lại của ô tô giảm đi $4 \%$ (so với năm trước đó).
	\begin{enumEX}[a)]{1} 
		\item Viết công thức tính giá trị của ô tô sau $1$ năm, $2$ năm sử dụng.
		\item Viết công thức tính giá trị của ô tô sau $n$ năm sử dụng.
		\item Sau $10$ năm, giá trị của ô tô ước tính còn bao nhiêu triệu đồng? 
	\end{enumEX} 
	\dapso{$\approx 531{,}87$ triệu đồng}
	\loigiai{
		Gọi $u_n$ là giá trị còn lại của ô tô sau $n$ năm sử dụng. 
		\begin{enumEX}[a)]{1} 
			\item Giá trị của ô tô sau $1$ năm sử dụng là $u_1=800-800\cdot0{,}04=800\cdot0{,}96=768$ triệu đồng.\\
			Giá trị của ô tô sau $2$ năm sử dụng là $u_2=u_1-u_1\cdot0{,}04=u_1\cdot0{,}96=737{,}28$ triệu đồng.
			\item Ta có $u_n=u_{n-1}-u_{n-1}\cdot0{,}04=u_{n-1}\cdot0{,}96$. \\
			Do đó, $(u_n)$ là cấp số nhân với số hạng đầu $u_1=768$ và công bội $q=0{,}96$. \\
			Vậy sau $n$ năm sử dụng, giá trị còn lại của chiếc ô tô là $u_n=u_1q^{n-1}\Rightarrow u_n=768\cdot0{,}96^{n-1}$.
			\item Sau $10$ năm, ước tính giá trị của ô tô còn lại là $u_{10}=768\cdot0{,}96^9\approx 531{,}87$ triệu đồng.
		\end{enumEX} 
	}
\end{bt}
% \begin{bt} [VD] %[DCHT Toán 11 - KNTT - Dung Phuong] %[1K2K7-7]
% 	Ông An vay ngân hàng $1$ tỉ đồng với lãi suất $12\%/$năm. Ông đã trả nợ theo cách: Bắt đầu từ tháng thứ nhất sau khi vay, cuối tháng ông trả ngân hàng số tiền là $a$ (đồng) và đã trả hết nợ sau đúng $2$ năm kể từ ngày vay. Hỏi số tiền mỗi tháng mà ông An phải trả là bao nhiêu đồng (làm tròn kết quả đến hàng nghìn)?
% 	\dapso{$47073500$}
% 	\loigiai{
% 		Do lãi suất là $12\%$/năm tương đương với lãi là $1\%$/tháng.\\
% 		Sau $1$ tháng, ông An còn nợ là: $10^9.(1+1\%)-a=10^9.(1,01)-S_1$.\\
% 		Sau $2$ tháng, ông An còn nợ là: $10^9.(1.01)^2-a.(1.01)-a=10^9(1,01)^2-S_2$.\\
% 		Sau $3$ tháng, ông An còn nợ là: $10^9.(1.01)^3-a(1.01)^2-a(1.01)-a=10^9.(1.01)^3-S_3$.\\
% 		Sau $24$ tháng, ông An còn nợ là: $10^9.(1.01)^{24}-S_{24}=0$.\\
% 		Do đó $S_{24}=10^9.(1.01)^{24}  \Leftrightarrow a.\dfrac{1-(1.01)^{24}}{1-(1.01)}=10^9.(1.01)^{24} \Leftrightarrow a =\dfrac{10^9.(1.01)^{24}.0.01}{(1.01)^{24}-1}\approx 47073472,22$.\\
% 		Vậy mỗi tháng ông An phải trả $47073500$.
% 	}
% \end{bt}
\begin{bt}%[VD] %[DCHT Toán 11 - KNTT - Dung Phuong] %[1K2K7-7]
	\immini
	{
		Một người nhảy bungee (một trò chơi mạo hiểm mà người chơi nhảy từ một nơi có địa thế cao xuống với dây đai an toàn buộc xung quanh người) từ một cây cầu và căng một sợi dây dài $100$ m. Sau mỗi lần rơi xuống, nhờ sự đàn hồi của dây, người nhảy được kéo lên một quãng đường có độ dài bằng $75$\% so với lần rơi trước đó và lại bị rơi xuống đúng bằng quãng đường vừa được kéo lên. Tính tổng quãng đường người đó đi được sau $10$ lần kéo lên và lại rơi xuống. 
	}
	{
		\begin{tikzpicture}[xscale=.5, font=\small, line join=round, line cap=round, >=stealth,yscale=1]
			\def\a{-0.12} % Hệ số a phải khác 0
			\def\b{0.86}
			\def\c{0}
			\def\m{-0.12} % Hệ số a phải khác 0
			\def\n{0.86}
			\def\p{-0.3}
			\clip (-2,-2)rectangle(9,2);
			\fill[red!40] (-2,1.8)--(9,1.8)--(9,1.55)--(-2,1.55)--cycle;%Mặt phẳng của cây cầu
			\fill[green!50] (-2,1.55)--(0,0)--(1.5,-1)--(-1,-2)--(-2,-2)--cycle;
			\fill[green!50] (9,1.55)--(7,0)--(5.5,-1)--(8,-2)--(9,-2)--cycle;
			\fill[blue!40] (-1,-2)--(8,-2)--(5.5,-1)--(1.5,-1)--cycle;
			\draw[color=blue!50,line width=2pt,<->] (2,-.5)--(2,1.5)node[left,midway]{$100$ m};	
			\draw[color=blue!50,line width=2pt,<->] (5,-.5)--(5,1)node[midway,right]{$0{,}75\cdot100$ m};
			\draw(3.5,1.55)--(3.5,-.25)node[rotate=200]{\faChild};
			\draw[color=white,<->] (3.5,1.8)--(3.5,1.55); 
		\end{tikzpicture}
	}
	\dapso{$\approx666{,}2 \text{ m}$}
	\loigiai{
		Gọi $u_n$ là quãng đường người đó được kéo lên ở lần thứ $n$ được kéo lên và lại rơi xuống (đơn vị tính: mét). \\
		Ta có $u_1=0{,}75\cdot100=100\cdot1{,}5=75$ m và $u_n=0{,}75\cdot u_{n-1}$. \\
		Vậy $(u_n)$ là cấp số nhân với số hạng đầu $u_1=75$ và công bội $q=0{,}75$. \\
		Tổng quãng đường người đó đi được sau $10$ lần kéo lên và lại rơi xuống là 
		$$\begin{aligned}
			S&=100+2u_1+2u_2+\cdots+2u_{10}\\
			&=100+2S_{10}
			=100+2\cdot\dfrac{75\left(1-0{,}75^{10}\right)}{1-0{,}75}\\
			&\approx666{,}2 \text{ m}.
		\end{aligned}$$ 
	}
\end{bt}
\begin{bt} [TH] %[DCHT Toán 11 - KNTT - Dung Phuong] %[1K2B7-7]
	Một cái tháp có $11$ tầng. Diện tích của mặt sàn tầng $2$ bằng nửa diện tích của mặt đáy tháp và diện tích của mặt sàn mỗi tầng bằng nửa diện tích của mặt sàn mỗi tầng ngay bên dưới. Biết mặt đáy tháp có diện tích là $12 288m^2$. Tính diện tích của mặt sàn tầng trên cùng của tháp theo đơn vị mét vuông.
	\dapso{$12m^2$}
	\loigiai{ (Lưu ý: Một số nơi xem tầng 1 là tầng trệt. Nên bài toán này giống bài toán tháp 10 tầng ở phần trên)
		Do diện tích của mặt sàn tính từ tầng một lập thành một cấp số nhân với $u_2=\dfrac{1}{2}.12288=6144$ và $q=\dfrac{1}{2}$.\\
		Ta có $\heva{u_2&=6144 \\ q&=\dfrac{1}{2}}  \Leftrightarrow \heva{u_1&=12288 \\ q&=\dfrac{1}{2}}$.\\
		Ta có $u_{11}=u_1.q^{10}=12288.\dfrac{1}{2^{10}}=12m^2$.
		Vậy diện tích của mặt sàn tầng trên cùng là	$12m^2$.
	}
\end{bt}

\begin{bt}%[TH]%[DCHT Toán 11 - KNTT - Dung Phuong]%[1K2B7-7]
	\immini{Cho hình vuông $C_1$ có cạnh bằng $4$. Người ta chia mỗi cạnh hình vuông thành bốn phần bằng nhau và nối các điểm chia một cách thích hợp để có hình vuông $C_2$ . Từ hình vuông $C_2$ lại làm tiếp tục như trên để có hình vuông $C_3$. Cứ tiếp tục quá trình như trên, ta nhận được dãy các hình vuông $C_1, C_2, C_3, \ldots , C_n, \ldots$ Gọi $a_n$ là độ dài cạnh hình vuông $C_n$. Chứng minh rằng dãy số $\left(a_n\right)$ là cấp số nhân.}{
		\begin{tikzpicture}[scale=.8]
			\def\a{2}  %cạnh hình vuông
			\def\t{.7}  % tỷ lệ điểm cho vòng lặp tiếp
			\path 
			(-\a,-\a) coordinate (A1)
			(-\a,\a) coordinate (B1)
			(\a,\a) coordinate (C1)
			(\a,-\a) coordinate (D1);
			\draw (A1)--(B1)--(C1)--(D1)--cycle;
			\foreach \i[count=\j from 2] in {1,...,10}
			\draw
			(barycentric cs:A\i=\t,B\i=1-\t) coordinate (A\j)--
			(barycentric cs:B\i=\t,C\i=1-\t) coordinate (B\j)--
			(barycentric cs:C\i=\t,D\i=1-\t) coordinate (C\j)--
			(barycentric cs:D\i=\t,A\i=1-\t) coordinate (D\j)--cycle
			;	
			% \node at (0,-2.2) [below]{\textit{Hình 4}};
		\end{tikzpicture}	
	}
	\loigiai{
		\immini{Gọi cạnh một hình vuông thứ $n$, $n+1$ lần lượt là $a_n, a_{n+1}$.\\
			Do $MN=\sqrt{MB^2+BN^2}=\sqrt{\left(\dfrac{AB}{4}\right)^2+\left(\dfrac{3AB}{4}\right)^2 }=AB\cdot\dfrac{\sqrt{10}}{4}$.\\
			Nên ta có cạnh hình vuông thứ $n+1$ là:\\ $a_{n+1}=a_n.\dfrac{\sqrt{10}}{4}$.\\
			Vậy dãy số $\left(a_n\right)$ là cấp số nhân.	
		}{
			\begin{tikzpicture}[scale=0.8,>=stealth, font=\footnotesize, line join=round, line cap=round]
				\path
				(0,0) coordinate (A)
				(4,0) coordinate (B)
				(4,4) coordinate (C)
				(0,4) coordinate (D)
				($(A)!0.75!(B)$) coordinate (M)
				($(B)!0.75!(C)$) coordinate (N)
				($(C)!0.75!(D)$) coordinate (P)
				($(D)!0.75!(A)$) coordinate (Q)
				;
				\draw (A)--(B)--(C)--(D)--cycle (M)--(N)--(P)--(Q)--cycle;
				\node at ($(A)!0.5!(B)$)[below]{$a_n$};
				\node at ($(M)!0.5!(N)$)[left]{$a_{n+1}$};
				\foreach \p/\q in {A/180,B/0,C/0,D/180,M/-90,N/0,P/90,Q/180}
				\fill[black] (\p) circle (1.0pt) ($(\p)+(\q:2.5mm)$) node{$\p$};
		\end{tikzpicture}
	}
	}
\end{bt}

\begin{bt}%[TH] %[DCHT Toán 11 - KNTT - Dung Phuong] %[1K2B7-7]
	Một cây đàn organ có tần số âm thanh các phím liên tiếp tạo thành một cấp số nhân. Cho biết tần số phím La trung là $400$ Hz và tần số của phím La cao cao hơn $12$ phím là $800$ Hz (nguồn: https://vi.wikipedia.org/wikiOrgan). Tìm công bội của cấp số nhân nói trên (làm tròn kết quả đến hàng phần nghìn).
	\dapso{$q = \pm \sqrt[12]{2}$}
	\loigiai{
		Theo đề ta có $\heva{&u_1=400\\&u_{13}=800} \Leftrightarrow \heva{&u_1=400\\&u_1q^{12}=800} \Rightarrow q^{12} = 2 \Rightarrow q = \pm \sqrt[12]{2}$.
	}
\end{bt}


\begin{bt}%[VD]%[DCHT Toán 11 - KNTT - Dung Phuong] %[1K2K7-7] 
	Một loại thuốc được dùng mỗi ngày một lần. Lúc đầu nồng độ thuốc trong máu của bệnh nhân tăng nhanh, nhưng mỗi liều kế tiếp có tác dụng ít hơn liều trước đó. Lượng thuốc trong máu ở ngày thứ nhất là $50 \,\mathrm{mg}$, và mỗi ngày sau đó giảm chỉ còn một nửa so với ngày kề trước đó. Tính tổng lượng thuốc (tính bằng $\mathrm{mg}$) trong máu của bệnh nhân sau khi dùng thuốc $10$ ngày liên tiếp.
	\dapso{$99{,}902$ mg.}
	\loigiai{
		Gọi $u_n$ là giá trị của lượng thuốc trong máu của bệnh nhân trong ngày thứ $n$. \\
		Dãy số này là một cấp số nhân có $u_1=50$, $q=\dfrac{1}{2}$.\\
		Tổng của $n$ số hạng đầu tiên của cấp số nhân là $S_n=u_1\dfrac{1-q^n}{1-q}$.\\
		Theo bài toán, ta có $S_{10}=50 \cdot\dfrac{1-\left(\dfrac{1}{2}\right)^{10}}{1-\dfrac{1}{2}} \approx 99{,}902$.\\
		Vậy tổng lượng thuốc trong máu của bệnh nhân sau khi dùng thuốc $10$ ngày liên tiếp là $99{,}902$ mg.	
	}
\end{bt}
\subsubsection{Câu hỏi trắc nghiệm}
\Opensolutionfile{ans}[ans/ans-1K2-2-Dang7]
\begin{ex}%[1K2K7-7]
	\immini{
		Cho hình vuông có cạnh là $1$. Nối các trung điểm của hình vuông trên ta được một hình vuông có diện tích $S_1$, tiếp tục quá trình trên với các hình vuông với diện tích là $S_2$; $S_3$; $\ldots ;S_n;\ldots$. Tính tổng vô hạn $S_1+ S_2+ S_3+\cdots+S_n+\cdots$.
		\choice
		{$2$}
		{$\dfrac{1}{2}$}
		{\True $1$}
		{$\dfrac{3}{2}$}
	}
	{\hspace*{1 cm}
		\begin{tikzpicture}[scale=0.8,line cap=round,line join=round]
			\path
			(0,0) coordinate (A)
			(4,0) coordinate (B)
			(0,4) coordinate (D);			
			\coordinate (C) at ($(B)-(A)+(D)$);
			\coordinate (H) at ($(A)!0.5!(B)$);
			\coordinate (I) at ($(A)!0.5!(D)$);
			\coordinate (J) at ($(D)!0.5!(C)$);
			\coordinate (K) at ($(B)!0.5!(C)$);
			\coordinate (E) at ($(I)!0.5!(H)$);
			\coordinate (F) at ($(H)!0.5!(K)$);
			\coordinate (G) at ($(K)!0.5!(J)$);
			\coordinate (O) at ($(J)!0.5!(I)$);
			\coordinate (M) at ($(E)!0.5!(F)$);
			\coordinate (N) at ($(F)!0.5!(G)$);
			\coordinate (P) at ($(G)!0.5!(O)$);
			\coordinate (Q) at ($(O)!0.5!(E)$);
			\draw (A)--(B)--(C)--(D)--cycle (I)--(H)--(K)--(J)--cycle
			(E)--(F)--(G)--(O)--cycle (M)--(N)--(P)--(Q)--cycle;
			\foreach \p in {A,B,C,D,E,F,G,H,I,J,K,M,N,P,Q,O}
			\fill[black] (\p) circle (1.0pt);			
		\end{tikzpicture}
	}
	\loigiai{
		Ta có $S_1=\dfrac{1}{2}$, $S_2=\dfrac{1}{4}$, $S_3=\dfrac{1}{8},\cdots  S_n=\dfrac{1}{2^n},\ldots$ tạo thành $1$ cấp số nhân với công bội $q=\dfrac{1}{2}<1$. \\
		Vậy $S_1+ S_2+ S_3+\cdots+S_n+\cdots=\dfrac{\dfrac{1}{2}}{1-\dfrac{1}{2}}=1$.
	}
\end{ex}
\begin{ex}%[1K2K7-7]
	Cho $n$ là số nguyên dương và $n$ tam giác $A_1B_1C_1,A_2B_2C_2,\ldots,A_nB_nC_n$, trong đó các điểm lần ${A}_{i+1},{B}_{i+1},{C}_{i+1}$ lượt nằm trên các cạnh $B_iC_i,A_iC_i,A_iB_i(i=1,2,\ldots,n-1)$ sao cho ${A}_{i+1}C_i=3{A}_{i+1}B_i,{B}_{i+1}A_i=3{B}_{i+1}C_i,{C}_{i+1}B_i=3{C}_{i+1}A_i$. Gọi $S$ là tổng tất cả các diện tích của tam giác $A_1B_1C_1,A_2B_2C_2,\ldots,A_nB_nC_n$ biết rằng tam giác $A_1B_1C_1$ có diện tích bằng $\dfrac{9}{16}$. Tìm số nguyên dương sao cho $S=\dfrac{{16}^{29}-7^{29}}{{16}^{29}}$.
	\choice
	{$n=28$}
	{$n=2018$}
	{$n=30$}
	{\True $n=29$}
	\loigiai{
		Gọi $S_i(i=1,2,3,...,n)$ là diện tích của $\Delta A_iB_iC_i$. Ta có $\dfrac{S_{A_1B_2C_2}}{S_{A_1B_1C_1}}=\dfrac{A_1B_2}{A_1C_1}\cdot \dfrac{A_1C_2}{A_1B_1}=\dfrac{1}{4}\cdot \dfrac{3}{4}=\dfrac{3}{16}$. Tương tự, ta có $\dfrac{S_{A_2B_1C_2}}{S_{A_1B_1C_1}}=\dfrac{S_{A_2B_2C_1}}{S_{A_1B_1C_1}}=\dfrac{3}{16}$. Do đó $\dfrac{S_{A_2B_2C_2}}{S_{A_1B_1C_1}}=1-3\cdot \dfrac{3}{16}=\dfrac{7}{16}\Rightarrow S_2=\dfrac{7}{16}S_1$.\\
		Tương tự, ta có ${S}_{i+1}=\dfrac{7}{16}S_i,i=1,2,\ldots,n$.
		Khi đó $$S=S_1\left[1+\dfrac{7}{16}+\cdots+{\left(\dfrac{7}{16}\right)}^{n-1}\right]=\dfrac{9}{16}\cdot \dfrac{1-{\left(\dfrac{7}{16}\right)}^n}{1-\dfrac{7}{16}}=1-{\left(\dfrac{7}{16}\right)}^n.$$
		Theo giả thiết ta có $1-{\left(\dfrac{7}{16}\right)}^n=1-{\left(\dfrac{7}{16}\right)}^{29}\Leftrightarrow n=29$.}
\end{ex}
\begin{ex}%[1K2K7-7]
	Người ta thiết kế một cái tháp gồm $11$ tầng. Diện tích bề mặt trên của mỗi tầng bằng nửa diện của mặt trên tầng ngay bên dưới và diện tích tầng $1$ bằng nửa diện tích của đế tháp. Biết đế tháp có diện tích là $12288\, \mathrm{m}^2$. Tính diện tích mặt trên cùng.
	\choice
	{$12\, \mathrm{m}^2$}
	{\True $6\, \mathrm{m}^2$}
	{$10\, \mathrm{m}^2$}
	{$8\, \mathrm{m}^2$}
	\loigiai{
		Gọi $S_{i}$ là diện tích của tầng thứ $i$ với $i = 1,2,\ldots,11$.\\
		Do giả thiết suy ra $S_{i + 1} = \dfrac{1}{2}S_{i}$ với $i = 1,2,\ldots,10$.\\
		Do đó $\left\{S_{i}\right\}$ là một cấp số nhân với công bội $q = \dfrac{1}{2}$. Do đó  $S_{11} = \dfrac{1}{2^{10}}S_{1} = \dfrac{1}{2^{11}}\cdot 12288 = 6\left(\mathrm{m}^2\right)$.
	}
\end{ex}

\begin{ex}%[1K2K7-7]
	Cho tứ giác $ABCD$ có bốn góc tạo thành cấp số nhân có công bội $ q=2 $. Góc có số đo nhỏ nhất trong bốn góc đó là
	\choice
	{\True $ 24^\circ $}
	{$ 1^\circ $}
	{$ 12^\circ $}
	{$ 30^\circ $}
	\loigiai{
		Gọi số đo bốn góc của tứ giác $ ABCD $ là $ x $, $ 2x $, $ 4x $, $ 8x $.
		\\ Có $ x+2x+4x+8x=360 \Leftrightarrow 15x=360 \Leftrightarrow x=24 $.}
\end{ex}

\begin{ex}%[1K2K7-7]
	Một du khách vào chuồng đua ngựa đặt cược, lần đầu tiên đặt $20000$ đồng, mỗi lần sau tiền đặt gấp đôi lần tiền đặt cược trước. Người đó thua lần $9$ liên tiếp và thắng ở lần thứ $10$. Hỏi du khách đó thắng hay thua bao nhiêu tiền?
	\choice
	{\True Thắng $20000$ đồng}
	{Thua $40000$ đồng}
	{Hòa vốn}
	{Thua $20000$ đồng}
	\loigiai{
		Số tiền đặt cược lần thứ $n$ là $u_n=u_1\cdot 2^{n-1}$ với $u_1=20000$. \\
		Ta có: $u_{10}-\displaystyle\sum_{n=1}^9 u_1\cdot 2^{n-1}=20000\cdot 2^9-\displaystyle\sum_{n=1}^9 20000\cdot 2^{n-1}=20000$. \\
		Vậy du khách thắng $20000$ đồng.
	}
\end{ex}

% \begin{ex}%[1K2K7-7]
% 	Một người gửi tiết kiệm vào ngân hàng với lãi suất $7{,}5$ \%/năm. Biết rằng nếu không rút tiền ra khỏi ngân hàng thì cứ sau mỗi năm số tiền lãi sẽ được nhập vào vốn để tính lãi cho năm tiếp theo. Hỏi sau ít nhất bao nhiêu năm người đó thu được (cả số tiền gửi ban đầu và lãi) gấp đôi số tiền đã gửi, giả định trong khoảng thời gian này lãi suất không thay đổi và người đó không rút tiền ra?
% 	\choice
% 	{$12$ năm}
% 	{$11$ năm}
% 	{\True $10$ năm}
% 	{$9$ năm}
% 	\loigiai{
% 		Áp dụng công thức: $S_n=A(1+r)^n \Rightarrow n=\log_{(1+r)}\left(\dfrac{S_n}{A}\right) \Rightarrow n=\log_{\left(1+7{,}5\%\right)}(2)\approx 9{,}6$.}
% \end{ex}

\begin{ex}%[1K2K7-7]
	Cho tam giác $ ABC $ cân tại $ A $ có cạnh đáy $ BC $,  đường cao $ AH $ và cạnh bên $ AB $ theo thứ tự đó lập thành cấp số nhân công bội $ q $. Giá trị của $ q $ là
	\choice
	{$ q=\dfrac{1}{2}\sqrt{\sqrt{2}+1} $ }
	{$ q=\sqrt{2}+1 $ }
	{$ q=\sqrt{2(\sqrt{2}+1)} $}
	{\True $ q=\dfrac{1}{2}\sqrt{2(\sqrt{2}+1)} $ }
	\loigiai{
		Giả sử $ BC=u_1 $, $ AH=u_1\cdot q $ và $ AB=u_1\cdot q^2 $ với $ u_1> 0, q> 0 $.\\
		Do $ \triangle ABC $ cân tại $ A $ suy ra
		\begin{align*}
			AB^2=AH^2+\dfrac{BC^2}{4}\Leftrightarrow
			& u_1^2\cdot q^4=\dfrac{u_1^2}{4}+u_1^2\cdot q^2\\
			\Leftrightarrow & 4q^4-4q^2-1=0\\
			\Leftrightarrow & q^2=\dfrac{1\pm \sqrt{2}}{2}.
		\end{align*}
		Kết hợp với điều kiện bài toán ta có $ q=\sqrt{\dfrac{1+ \sqrt{2}}{2}}=\dfrac{1}{2}\sqrt{2(\sqrt{2}+1)} $.
	}
\end{ex}
\begin{ex}%[1K2K7-7]
	Giả sử một người đi làm được lĩnh lương khởi điểm là $2.000.000$ đồng/tháng. Cứ $3$ năm người ấy lại được tăng lương một lần với mức tăng bằng $7\%$ của tháng trước đó. Hỏi sau $36$ năm làm việc người ấy lĩnh được tất cả bao nhiêu tiền?
	\choice
	{\True $ 1.287.968.492 $ đồng}
	{$ 10.721.769.110 $ đồng}
	{$ 7{,}068289036\cdot 10^8 $ đồng}
	{$ 429.322.830{,}5 $ đồng}
	\loigiai{
		Ta có $36$ năm tương ứng với $12$ kỳ lương; mỗi kỳ lương có $36$ tháng và kỳ sau tăng $7\%$ so với kỳ trước. Do đó tổng số tiền mỗi kỳ lương là một cấp số nhân với $u_1=36\times 2=72$ (triệu đồng) và công bội $q=1{,}07$.\\
		Vậy tổng số tiền sau $36$ năm là $T=\dfrac{72\cdot \left[(1{,}07)^{12}-1\right]}{1{,}07-1}=1287{,}968492$ (triệu đồng).
	}
\end{ex}

\begin{ex}%[1K2K7-7]
	Từ độ cao $55{,}8$ (mét) của tháp nghiên Pisa nước Italia người ta thả một quả bóng cao su chạm xuống đất. Giả sử mỗi lần chạm đất bóng lại nảy lên độ cao bằng $\dfrac{1}{10}$ độ cao mà bóng đạt trước đó. Tổng độ dài hành trình (mét) của bóng được thả từ lúc ban đầu cho đến khi nó nằm yên trên mặt đất thuộc khoảng nào trong các khoảng sau đây?
	\choice
	{$(69;72)$}
	{$(60;63)$}
	{\True $(67;69)$}
	{$(64;66)$}
	\loigiai{
		Đặt $u_1=55{,}8$ (mét) là quãng đường bóng rơi khi thả xuống, $u_{n+1}=\dfrac{1}{10^{n}} u_1, n\ge 1$ là quãng đường bóng rơi sau lần nảy lên thứ $n$. \\
		Ta có $(u_n)$ là dãy cấp số nhân với $u_1=55{,}8$ và công bội $q=\dfrac{1}{10}$.\\
		Suy ra tổng quãng đường quả bóng rơi xuống là $\displaystyle \lim \limits_{n \rightarrow +\infty} u_1 \cdot \dfrac{1-q^n}{1-q}=\displaystyle \lim \limits_{n \rightarrow +\infty}55{,}8\cdot\dfrac{1-\left( \dfrac{1}{10}\right)^n}{1-\dfrac{1}{10}}=62 $.\\
		Ngoài ra ta còn phải tính tổng quãng đường mà bóng nảy lên. Ta có tổng quãng đường bóng nảy lên bằng tổng quãng đường rơi của bóng trừ đi quãng đường thả rơi xuống.\\
		Vậy tổng quãng đường hành trình của quả bóng là $62+62-55{,}8=68{,}2$ (mét).
	}
\end{ex}

\begin{ex}%[1K2K7-7]
	Một gia đình lập kế hoạch tiết kiệm như sau: Họ lập một sổ tiết kiệm tại một ngân hàng và cứ đầu mỗi tháng họ gửi
	vào sổ tiết kiệm đó $15$ triệu đồng. Giả sử lãi suất tiền gửi không đổi là $0{,}6$ \%/tháng và tiền gửi được tính lãi theo hình thức lãi
	kép. Hỏi sau $3$ năm gia đình đó tiết kiệm được số tiền gần nhất với con số nào dười đây?
	\choice
	{$543240000$ đồng}
	{$589269000$ đồng}
	{$669763000$ đồng}
	{\True $604359000$ đồng}
	\loigiai{
		Gọi $S_0$ triệu đồng là số tiền gia đình đó định kỳ gửi tiết kiệm vào đầu hằng tháng, $r$ là lãi suất tiền gửi hằng tháng. Ta có $S_0=15$ triệu đồng, $r=0{,}6$
		\%/tháng.\\
		Gọi $S_i$, $i=\overline{1,n}$ là số tiền trong sổ tiết kiệm cuối tháng thứ $i$.\\
		Ta có \begin{itemize}
			\item $S_1=S_0+S_0\cdot r=S_0(1+r)$,
			\item  $S_2=\left[ S_0+S_0(1+r)\right]+\left[ S_0+S_0(1+r)\right]r=S_0 (1+r)+S_0(1+r)^2$,
			\item  $\begin{aligned}[t]
				S_3=&\ \left[S_0+S_0(1+r)+S_0(1+r)^2 \right] +\left[S_0+S_0(1+r)+S_0(1+r)^2 \right]r\\
				=&\ S_0(1+r)+S_0(1+r)^2+S_0(1+r)^3,\end{aligned}$,
			\item \ldots
			\item$\begin{aligned}[t]S_n=&\ S_0(1+r)+S_0(1+r)^2+S_0(1+r)^3+\cdots +S_0(1+r)^n\\=&\ S_0\left[ (1+r)+(1+r)^2+(1+r)^3+\cdots+(1+r)^n\right]\\
				=&\  S_0(1+r)\cdot \dfrac{(1+r)^{n}-1}{(1+r)-1}=S_0(1+r)\cdot \dfrac{(1+r)^{n}-1}{r}.
			\end{aligned}$
		\end{itemize}
		Vậy sau $3$ năm, tức cuối tháng thứ $36$ thì gia đình tiết kiệm được số tiền là
		\[S_{36}=15\cdot 10^6(1+0{,}6\cdot 10^{-2})\cdot \dfrac{(1+0{,}6\cdot 10^{-2})^{36}-1}{0{,}6\cdot 10^{-2}}=604358538{,}2 \ \text{đồng}.\]
	}
\end{ex}
\Closesolutionfile{ans}
% \begin{indapan}{10}
% 	{ans/ans-1K2-2-Dang7}
% \end{indapan}

%%Ôn tập chương II
% \setcounter{dang}{0}
\setcounter{ex}{0}
\setcounter{bt}{0}
\setcounter{vd}{0}
\section*{Ôn tập chương 2}
\Opensolutionfile{ans}[ans/ans-1K2-Ontapchuong2]
\begin{ex}%[1K2Y5-2]
	Cho dãy số $\left(u_n\right)$, biết $u_n=\left(-1\right)^n.2n$. Mệnh đề nào sau đây sai?
	\choice
	{$u_1=-2$}
	{$u_2=4$}
	{$u_3=-6$}
	{\True $u_4=-8$}
	\loigiai{
		Thay trực tiếp vào kiểm tra, ta có
		\begin{eqnarray*}
			u_1&=&-2.1=-2\\
			u_2&=&(-1)^2.2.2=4\\
			u_3&=&(-1)^3.2.3=-6\\
			u_4&=&(-1)^4.2.4=8.
		\end{eqnarray*}
	}
\end{ex}
\begin{ex}%[1K2Y5-2]
	Cho dãy số $\left(u_n\right)$, biết $u_n=\left(-1\right)^n.\dfrac{2^n}{n}$. Tìm số hạng $u_3$.
	\choice
	{$u_3=\dfrac{8}{3}$}
	{$u_3=2$}
	{$u_3=-2$}
	{\True $u_3=-\dfrac{8}{3}$}
	\loigiai{
		Thay trực tiếp vào kiểm tra, ta có
		\begin{center}
			$u_3=(-1)^3.\dfrac{2^3}{3}=-\dfrac{8}{3}$.
		\end{center}
	}
\end{ex}
\begin{ex}%[1K2Y5-2]
	Cho dãy số $\left(u_n\right)$, biết $u_n=\dfrac{2n+5}{5n-4}$. Số $\dfrac{7}{12}$ là số hạng thứ mấy của dãy số?
	\choice
	{\True $8$}
	{$6$}
	{$9$}
	{$10$}
\end{ex}
\loigiai{
	Ta có
	\allowdisplaybreaks
	\begin{eqnarray*}
		&&u_n=\dfrac{2n+5}{5n-4}\\
		&\Leftrightarrow&\dfrac{7}{12}=	\dfrac{2n+5}{5n-4}\\
		&\Leftrightarrow&24n+60=35n-28\\
		&\Leftrightarrow&11n=88\\
		&\Leftrightarrow&n=8.
	\end{eqnarray*}
	Vậy số $\dfrac{7}{12}$ là số hạng thứ 8.
}
\begin{ex}%[1K2Y5-2]
	Cho dãy số $\left(u_n\right)$, biết $u_n=2^n$. Tìm số hạng $u_{n+1}$.
	\choice
	{\True $u_{n+1}=2^n.2$}
	{$u_{n+1}=2^n+1$}
	{$u_{n+1}=2\left(n+1\right)$}
	{$u_{n+1}=2^n+2$}
	\loigiai{
		Ta có
	}
	\loigiai{
		Thay $n$ bằng $n+1$ trong công thức $u_n$ ta được
		\allowdisplaybreaks
		\begin{eqnarray*}
			u_{n+1}&=&2^{n+1}\\
			& =&2.2^n.
		\end{eqnarray*}
	}
\end{ex}
\begin{ex}%[1K2B5-2]
	Cho dãy số $\left(u_n\right)$, biết $u_n=5^{n+1}$. Tìm số hạng $u_{n-1}$.
	\choice
	{$u_{n-1}=5^{n-1}$}
	{\True $u_{n-1}=5^{n}$}
	{$u_{n-1}=5.5^{n+1}$}
	{$u_{n-1}=5.5^{n-1}$}
	\loigiai{
		Thay $n$ bằng $n-1$ trong công thức $u_n$ ta được
		\allowdisplaybreaks
		\begin{eqnarray*}
			u_{n-1}& = &5^{n-1+1}\\
			& = &5^n.
		\end{eqnarray*}
	}
\end{ex}
\begin{ex}%[1K2Y5-1]
	Cho dãy số có các số hạng đầu là $-2;0;2;4;6;...$. Số hạng tổng quát của dãy số này là công thức nào dưới đây?
	\choice
	{$u_n=-2n$}
	{$u_n=n-2$}
	{$u_n=-2\left(n+1\right)$}
	{\True $u_n=2n-4$}
	\loigiai{
		Kiểm tra $u_1=-2$ ta loại các đáp án B và C. Tương tự kiểm tra $u_2=0$ ta loại đáp án A.
	}
\end{ex}
\begin{ex}%[1K2B5-1]
	Cho dãy số $\left(u_n\right)$, được xác định $\heva{&u_1=\dfrac{1}{2}\\&u_{n+1}=u_n-2}$. Số hạng tổng quát $u_n$ của dãy số là số hạng nào dưới đây?
	\choice
	{$u_n=\dfrac{1}{2}+2\left(n-1\right)$}
	{\True $u_n=\dfrac{1}{2}-2\left(n-1\right)$}
	{$u_n=\dfrac{1}{2}-2n$}
	{$u_n=\dfrac{1}{2}+2n$}
	\loigiai{
		Ta có
		\begin{center}
			$\heva{&u_1=\dfrac{1}{2}\\&u_{n+1}=u_n-2}\Rightarrow\heva{&u_1=\dfrac{1}{2}\\&u_2=-\dfrac{3}{2}\\&u_3=-\dfrac{7}{2}}$
		\end{center} 
		Ta thấy chỉ có đáp án B đều thoả mãn.
	}
\end{ex}
\begin{ex}%[1K2B5-1]
	Cho dãy số $\left(u_n\right)$, được xác định $\heva{&u_1=-2\\&u_{n+1}=-2-\dfrac{1}{u_n}}$. Số hạng tổng quát $u_n$ của dãy số là số hạng nào dưới đây?
	\choice
	{$u_n=\dfrac{-n+1}{n}$}
	{$u_n=\dfrac{n+1}{n}$}
	{\True $u_n=-\dfrac{n+1}{n}$}
	{$u_n=-\dfrac{n}{n+1}$}
	\loigiai{
		Ta có
		\begin{center}
			$\heva{&u_1=-2\\&u_{n+1}=-2-\dfrac{1}{u_n}}\Rightarrow\heva{&u_1=-2\\&u_2=-\dfrac{3}{2}}$
		\end{center}
		Ta thấy chỉ có đáp án C thoả mãn.
	}
\end{ex}
\begin{ex}%[1K2Y5-2]
	Cho cấp số cộng có số hạng đầu $u_1=-\dfrac{1}{2}$, công sai $d=\dfrac{1}{2}$. Năm số hạng liên tiếp đầu tiên của cấp số này là.
	\choice
	{$-\dfrac{1}{2};0;1;\dfrac{1}{2};1$}
	{$-\dfrac{1}{2};0;\dfrac{1}{2};0;\dfrac{1}{2}$}
	{$\dfrac{1}{2};1;\dfrac{3}{2};2;\dfrac{5}{2}$}
	{\True $-\dfrac{1}{2};0;\dfrac{1}{2};1;\dfrac{3}{2}$}
	\loigiai{
		Ta dùng công thức tổng quát $u_n=u_1+(n-1)d=-\dfrac{1}{2}+(n-1)\dfrac{1}{2}=-1+\dfrac{n}{2}$ để tính các số hạng của một cấp số cộng. Ta có
		\begin{center}
			$u_1=-\dfrac{1}{2},u_2=0,u_3=\dfrac{1}{2},u_4=1,u_5=\dfrac{3}{2}$.
		\end{center}
	}
\end{ex}
\begin{ex}%[1K2B6-3]
	Viết ba số hạng xen giữa các số $2$ và $22$ để được một cấp số cộng có năm số hạng.
	\choice
	{\True 
		$7;12;17$}
	{$6;10;14$}
	{$8;13;18$}
	{$6;12;18$}
	\loigiai{
		Giữa $2$ và $22$ có thêm ba số hạng nữa lập thành cấp số cộng, xem như ta có một cấp số cộng có năm số hạng với $u_1=22;u_5=22$, ta cần tìm $u_2,u_3,u_4$. Ta có
		\begin{eqnarray*}
			&&u_5=u_1+4d\\
			&\Leftrightarrow&d=\dfrac{u_5-u_1}{4}\\&\Leftrightarrow&d=5\\
			&\Rightarrow&\heva{&u_2=7\\&u_3=12\\&u_4=17}.
		\end{eqnarray*}
	}
\end{ex}
\begin{ex}%[1K2K6-3]
	Biết các số $C_n^1;C_n^2;C_n^3$ theo thứ tự lập thành một cấp số cộng với $n>3$. Tìm $n$.
	\choice
	{$n=5$}
	{\True $n=7$}
	{$n=9$}
	{$n=11$}
	\loigiai{
		Ba số $C_n^1;C_n^2;C_n^3$ theo thứ tự $u_1;u_2;u_3$ lập thành một cấp số cộng nên
		\begin{eqnarray*}
			&&u_1+u_3=2u_2\\
			&\Leftrightarrow&C_n^1+C_n^3=2C_n^2\\
			&\Leftrightarrow&n+\dfrac{(n-2)(n-1)n}{6}=2.\dfrac{(n-1)n}{2}\\
			&\Leftrightarrow&1+\dfrac{n^2-3n+2}{6}=n-1\\
			&\Leftrightarrow&n^2-9n+14\\
			&\Leftrightarrow&\hoac{&n=2\\&n=7}	
		\end{eqnarray*}
		Kết hợp với điều kiện $n>3$, do đó $n=7$ thoả mãn yêu cầu bài toán.
	}
\end{ex}
\begin{ex}%[1K2B6-2]
	Cho cấp số cộng $\left(u_n\right)$ có các số hạng đầu lần lượt là $5; 9; 13; 17;...$. Tìm số hạng tổng quát $u_n$ của cấp số cộng.
	\choice
	{$u_n=5n+1$}
	{$u_n=5n-1$}
	{\True $u_n=4n+1$}
	{$u_n=4n-1$}
	\loigiai{
		Các số $5; 9; 13; 17 th;...$ theo thứ tự đó lập thành cấp số cộng $\left(u_n\right)$ nên
		\begin{center}
			$\heva{&u_1=5\\&d=u_2-u_1=4}\Rightarrow u_n=u_1+(n-1)d=5+4(n-1)=4n+1$.
		\end{center}
	}
\end{ex}
\begin{ex}%[1K2K6-1]
	Cho cấp số cộng $\left(u_n\right)$ có $u_1=3$ và $d=\dfrac{1}{2}$. Khẳng định nào sau đây đúng?
	\choice
	{$u_n=-3+\dfrac{1}{2}(n+1)$}
	{$u_n=-3+\dfrac{1}{2}n-1$}
	{\True $u_n=-3+\dfrac{1}{2}(n-1)$}
	{$u_n=-3+\dfrac{1}{4}(n-1)$}
	\loigiai{
		Ta có
		\begin{center}
			$\heva{&u_1=-3\\&d=\dfrac{1}{2}}\Rightarrow u_n=u_1+(n-1)d=-3+\dfrac{1}{2}(n-1)$.
		\end{center}
	}
\end{ex}
\begin{ex}%[1K2K6-1]
	Trong các dãy số được cho dưới đây, dãy số nào là cấp số cộng?
	\choice
	{\True $u_7=7-3n$}
	{$u_7=7-3^n$}
	{$u_7=\dfrac{7}{3n}$}
	{$u_7=7.3^n$}
	\loigiai{
		Dãy $\left(u_n\right)$ là cấp số cộng khi và chỉ khi $u_n=an+b$ với $a,b$ là hằng số.
	}
\end{ex}
\begin{ex}%[1K2B6-3]
	Cho cấp số cộng $\left(u_n\right)$ có $u_1=-5$ và $d=3$. Mệnh đề nào sau đây đúng?
	\choice
	{$u_{15}=34$}
	{$u_{15}=45$}
	{\True $u_{13}=31$}
	{$u_{10}=35$}
	\loigiai{
		Ta có
		\begin{center}
			$\heva{&u_1=-5\\&d=3}\Rightarrow u_n=3n-8\Rightarrow\heva{&u_{15}=37\\&u_{13}=31\\&u_{10}=22}$.
		\end{center}
	}
\end{ex}
\begin{ex}%[1K2B6-3]
	Cho cấp số cộng $\left(u_n\right)$ có $d=-2$ và $S_8=72$. Tìm số hạng đầu tiên $u_1$.
	\choice
	{\True $u_1=16$}
	{$u_1=-16$}
	{$u_1=\dfrac{1}{16}$}
	{$u_1=-\dfrac{1}{16}$}
	\loigiai{
		Ta có $\heva{&d=-2\\&S_8=72}\Leftrightarrow\heva{&d=-2\\&8u_1+\dfrac{8.7}{2}d=72}\Rightarrow 8u_1+28.(-2)=72\Leftrightarrow u_1=16$.
	}
\end{ex}
\begin{ex}%[1K2K6-3]
	Một cấp số cộng có số hạng đầu là $1$, công sai là $4$, tổng của n số hạng đầu là $561$. Khi đó số
	hạng thứ $n$ của cấp số cộng đó là $u_n$ có giá trị là bao nhiêu?
	\choice
	{$u_n=57$}
	{$u_n=61$}
	{\True $u_n=65$}
	{$u_n=69$}
	\loigiai{
		Ta có $\heva{&u_1=1,d=4\\&S_n=561}\Leftrightarrow\heva{&u_=1,d=4\\&nu_1+\dfrac{n(n-1)}{2}d=561}\Rightarrow n+\dfrac{n^2-n}{2}.4=561\Leftrightarrow 2n^2-n-561=0\Leftrightarrow n=17$.\\
		Từ đây suy ra $u_{17}=u_1+16d=1+16.4=65$.
	}
\end{ex}
\begin{ex}%[1K2K6-5]
	Tổng $n$ số hạng đầu tiên của một cấp số cộng là $S_n=\dfrac{3n^2-19n}{4}$ với $n\in\mathbb{N}^*$. Tìm số hạng đầu
	tiên $u_1$ và công sai $d$ của cấp số cộng đã cho.
	\choice
	{$u_1=2,d=-\dfrac{1}{2}$}
	{\True $u_1=-4,d=\dfrac{3}{2}$}
	{$u_1=-\dfrac{3}{2},d=-2$}
	{$u_1=\dfrac{5}{2},d=\dfrac{1}{2}$}
	\loigiai{
		Ta có $\dfrac{3n^2-19n}{4}=\dfrac{3}{4}n^2-\dfrac{19n}{4}=S_n=nu_1+\dfrac{n^2-n}{2}d=\dfrac{d}{2}n^2+\left(u_1-\dfrac{d}{2}\right)n$.\\
		Đồng nhất hai vế của phương trình, ta có $\heva{&\dfrac{d}{2}=\dfrac{3}{4}\\&u_1-\dfrac{d}{2}=-\dfrac{19}{4}}\Leftrightarrow\heva{&u_1=-4\\&d=\dfrac{3}{2}}$.
	}
\end{ex}
\begin{ex}%[1K2K6-3]
	Cho cấp số cộng $\left(u_n\right)$ có $u_2=2001$ và $u_5=1995$. Khi đó $u_{1001}$ bằng.
	\choice
	{$u_{1001}=4005$}
	{$u_{1001}=4003$}
	{\True $u_{1001}=3$}
	{$u_{1001}=1$}
	\loigiai{
		Ta có $\heva{&u_2=2001\\&u_5=1995}\Leftrightarrow\heva{&u_1+d=2001\\&u_1+4d=1995}\Leftrightarrow \heva{&u_1=2003\\&d=-2}\Rightarrow u_{1001}=u_1+1000d=3$.
	}
\end{ex}
\begin{ex}%[1K2B6-1]
	Cho cấp số cộng $\left(u_n\right)$ biết $u_n=-1,u_{n+1}=8$. Tính công sai $d$ của cấp số cộng đó.
	\choice
	{$d=-9$}
	{$d=7$}
	{$d=-7$}
	{\True $d=9$}
	\loigiai{
		Ta có $d=u_{n+1}-u_n=8-(-1)=9$.
	}
\end{ex}
\begin{ex}%[1K2K6-5]
	Cho cấp số cộng $\left(u_n\right)$ thỏa mãn $u_2+u_{23}=60$. Tính tổng $S_24$ của $24$ số hạng đầu tiên của
	cấp số cộng đã cho.
	\choice
	{$S_{24}=60$}
	{$S_{24}=120$}
	{\True $S_{24}=720$}
	{$S_{24}=1440$}
	\loigiai{
		Ta có $u_2+u_{23}=60\Leftrightarrow u_1+d+u_1+22d=60\Leftrightarrow 2u_1+23d=60$.\\
		Khi đó $S_{24}=\dfrac{24}{2}\left(u_1+u_{24}\right)=12\left(u_1+u_1+23d\right)=12.60=720$.
	}
\end{ex}
\begin{ex}%[1K2K6-1]
	Một cấp số cộng có $6$ số hạng. Biết rằng tổng của số hạng đầu và số hạng cuối bằng $17$, tổng
	của số hạng thứ hai và số hạng thứ tư bằng $14$. Tìm công sai $d$ của câp số cộng đã cho.
	\choice
	{$d=2$}
	{\True $d=-3$}
	{$d=4$}
	{$d=5$}
	\loigiai{
		Ta có $\heva{&u_1+u_6=17\\&u_2+u_4=14}\Leftrightarrow\heva{&2u_1+5d=17\\&2u_1+6d=14}\Leftrightarrow\heva{&u_1=16\\&d=-3}$.
	}
\end{ex}
\begin{ex}%[1K2K6-1]
	Cho cấp số cộng $\left(u_n\right)$ thỏa mãn $\heva{&u_7-u_3=8\\&u_2u_7=75}$. Tìm công sai $d$ của cấp số cộng đã cho.
	\choice
	{$d=\dfrac{1}{2}$}
	{$d=\dfrac{1}{3}$}
	{\True $d=2$}
	{$d=3$}
	\loigiai{
		Ta có $\heva{&u_7-u_3=8\\&u_2u_7=75}\Leftrightarrow\heva{&u_1+6d-u_1-2d=8\\&(u_1+d)(u_1+6d)=75}\Leftrightarrow\heva{&d=2\\&(u_1+2)(u_1+12)=75}$.
	}
\end{ex}
\begin{ex}%[1K2K6-3]
	Ba góc của một tam giác vuông tạo thành cấp số cộng. Hai góc nhọn của tam giác có số đo
	(độ) là
	\choice
	{$20^\circ$ và $70^\circ$}
	{$45^\circ$ và $45^\circ$}
	{$20^\circ$ và $45^\circ$}
	{\True $30^\circ$ và $60^\circ$}
	\loigiai{
		Ba góc $A,B,C$ của một tam giác vuông theo thứ tự đó $(A<B<C)$ lập thánh cấp số cộng nên $C=90,C+A=2B$.\\
		Ta có $\heva{&A+B+C=180\\&A+C=2B\\&C=90}\Leftrightarrow\heva{&A=30\\&B=60\\&C+90}$.
	}
\end{ex}
\begin{ex}%[1K2K6-3]
	Một tam giác vuông có chu vi bằng $3$ và độ dài các cạnh lập thành một cấp số cộng. Độ dài các
	cạnh của tam giác đó là
	\choice
	{$\dfrac{1}{2};1;\dfrac{3}{2}$}
	{$\dfrac{1}{3};1;\dfrac{5}{3}$}
	{\True $\dfrac{3}{4};1;\dfrac{5}{4}$}
	{$\dfrac{1}{4};1;\dfrac{7}{4}$}
	\loigiai{
		Ba cạnh $a,b,c,(a<b<c)$ của một tam giác theo thứ tự đó lập thành một cấp số cộng.\\
		Ta có $\heva{&a^2+b^2=c^2\\&a+b+c=3\\&a+c=2b}\Leftrightarrow\heva{&a^2+b^2=c^2\\&3b=3\\&a+c=2b}\Leftrightarrow\heva{&a^2+b^2=c^2\\&b=1\\&a=2-c}$.\\
		Từ đây suy ra $a^2+b^2=c^2\Rightarrow (2-c)^2+1=c^2\Leftrightarrow c=\dfrac{5}{4}\Leftrightarrow\heva{&a=\dfrac{3}{4}\\&b=1\\&c=\dfrac{5}{4}}$.
	}
\end{ex}
\begin{ex}%[1K2K6-6]
	Một rạp hát có $30$ dãy ghế, dãy đầu tiên có $25$ ghế. Mỗi dãy sau có hơn dãy trước $3$ ghế. Hỏi rạp
	hát có tất cả bao nhiêu ghế?
	\choice
	{$1635$}
	{$1792$}
	{\True $2055$}
	{$3125$}
	\loigiai{
		Số ghế của mỗi dãy (bắt đầu từ dãy đầu tiên) theo thứ tự đó lập thành một cấp số cộng có $30$ số hạng có công sai $d=3$ và $u_1=25$.\\
		Tổng số ghế là $S_{30}=30u_1+\dfrac{30.29}{2}d=2055$.
	}
\end{ex}
\begin{ex}%[1K2K6-6]
	Người ta trồng $3003$ cây theo một hình tam giác như sau: hàng thứ nhất trồng $1$ cây, hàng thứ hai trồng $2$ cây, hàng thứ ba trồng $3$ cây,... .Hỏi có tất cả bao nhiêu hàng cây?
	\choice
	{$73$}
	{$75$}
	{\True $77$}
	{$79$}
	\loigiai{
		Số cây mỗi hàng (bắt đầu từ hàng thứ nhất) lập thành một cấp số cộng $(u_n)$ có $u_1=1,d=1$. Giả sử có $n$ hàng cây thì $u_1+u_2+...+u_n=S_n$.\\
		Ta có $S_n=1.n+\dfrac{n(n-1)}{2}.1=3003\Leftrightarrow n=77$.
	}
\end{ex}
\begin{ex}%[1K2G6-6]
	Một chiếc đồng hồ đánh chuông, kể từ thời điểm $0$ (giờ) thì sau mỗi giờ thì số tiếng chuông được đánh đúng bằng số giờ mà đồng hồ chỉ tại thời điểm đánh chuông. Hỏi một ngày đồng hồ đó đánh bao nhiêu tiếng chuông?
	\choice
	{$78$}
	{$156$}
	{\True $300$}
	{$48$}
	\loigiai{
		Kể từ lúc $1$ (giờ) đến $24$ (giờ) số tiếng chuông được đánh lập thành cấp số cộng có $24$ số hạng với $u_1=1$, công sai $d=1$. Vậy số tiếng chuông được đánh trong $1$ ngày là $S_{24}=1.24+\dfrac{24.23}{2}.1=300$.
	}
\end{ex}
\begin{ex}%[1K2G6-6]
	Trên một bàn cờ có nhiều ô vuông, người ta đặt $7$ hạt dẻ vào ô đầu tiên, sau đó đặt tiếp vào ô thứ
	hai số hạt nhiều hơn ô thứ nhất là $5$, tiếp tục đặt vào ô thứ ba số hạt nhiều hơn ô thứ hai là $5$,...
	và cứ thế tiếp tục đến ô thứ $n$. Biết rằng đặt hết số ô trên bàn cờ người ta phải sử dụng $25450$
	hạt. Hỏi bàn cờ đó có bao nhiêu ô vuông?
	\choice
	{$98$}
	{\True $100$}
	{$102$}
	{$104$}
	\loigiai{
		Số hạt dẻ trên mỗi ô (bắt đầu từ ô thứ nhất) theo thứ tự đó lập thành cấp số cộng $(u_n)$ có $u_1=7,d=5$. Gọi $n$ là số ô trên bàn cờ thì $u_1+u_2+...+u_n=S_n$.\\
		Ta có $S_n=25450\Leftrightarrow 7n+\dfrac{n(n-1)}{2}.7=25450\Leftrightarrow n=100$.
	}
\end{ex}
\begin{ex}%[1K2G6-6]
	Một gia đình cần khoan một cái giếng để lấy nước. Họ thuê một đội khoan giếng nước đến để khoan giếng nước. Biết giá của mét khoan đầu tiên là $80.000$ đồng, kể từ mét khoan thứ $2$ giá của mỗi mét khoan tăng thêm $5000$ đồng so với giá của mét khoan trước đó. Biết cần phải khoan sâu xuống $50$ mét mới có nước. Vậy hỏi phải trả bao nhiêu tiền để khoan cái giếng đó?
	\choice
	{$5.250.000$ đồng}
	{\True $10.125.000$ đồng}
	{$4.00.000$ đồng}
	{$4.245.000$ đồng}
	\loigiai{
		Giá tiền khoang mỗi mét (bắt đầu từ mét đầu tiên) lập thành cấp số cộng $(u_n)$ có $u_1=80000,d=5000$. Do cần khoang $50$ mét nên tổng số tiền cần trả là $S_{50}=80000.50+\dfrac{50.49}{2}.5000=10125000$.
	}
\end{ex}
\begin{ex}%[1K2Y7-3]
	Một cấp số nhân có hai số hạng liên tiếp là $16$ và $36$. Số hạng tiếp theo là
	\choice
	{$720$}
	{\True$81$}
	{$64$}
	{$56$}
	\loigiai{
		Ta có cấp số nhân $(u_n)$ có $\heva{&u_n=36\\&u_{n+1}=36}\Rightarrow q=\dfrac{u_{n+1}}{u_n}=\dfrac{9}{4}$. Từ đây suy ra $u_{n+2}=u_{n+1}.q=36.\dfrac{9}{4}=81$.
	}
\end{ex}
\begin{ex}%[1K2B7-3]
	Tìm x để các số $2;8;x;128$ theo thứ tự đó lập thành một cấp số nhân.
	\choice
	{$x=14$}
	{\True 
		$x=32$}
	{$x=64$}
	{$x=68$}
	\loigiai{
		Cấp số nhân$ 2;8;x;128$ theo thứ tự đó sẽ là $u_1,u_2,u_3,u_4$.\\
		Ta có $\heva{&\dfrac{u_2}{u_1}=\dfrac{u_3}{u_2}\\&\dfrac{u_3}{u_2}=\dfrac{u_4}{u_3}}\Leftrightarrow\heva{&\dfrac{8}{2}=\dfrac{x}{8}\\&\dfrac{128}{x}=\dfrac{x}{8}}\Leftrightarrow\heva{&x=32\\&x^2=1024}\Rightarrow x=32$.
	}
\end{ex}
\begin{ex}%[1K2K7-3]
	Tìm tất cả giá trị của $x$ để ba số $2x-11;x;2x+1$ theo thứ tự đó lập thành một cấp số nhân.
	\choice
	{\True $x=\pm\dfrac{1}{\sqrt{3}}$}
	{$x=\pm\dfrac{1}{3}$}
	{$x=\pm\sqrt{3}$}
	{$x=\pm3$}
	\loigiai{
		Cấp số nhân $2x-1;x;2x+1$, suy ra $(2x-1)(2x+1)=x^2\Leftrightarrow x=\pm \dfrac{1}{\sqrt{3}}$.
	}
\end{ex}
\begin{ex}%[1K2K7-3]
	Với giá trị $x,y$ nào dưới đây thì các số hạng lần lượt là $-2;x;-18;y$ theo thứ tự đó lập thành cấp số nhân?
	\choice
	{$\heva{&x=6\\&y=-54}$}
	{$\heva{&x=-10\\&y=-26}$}
	{\True $\heva{&x=-6\\&y=-54}$}
	{$\heva{&x=-6\\&y=54}$}
	\loigiai{
		Cấp số nhân $-2;x;-18;y$, suy ra $\heva{&\dfrac{x}{-2}=\dfrac{-18}{x}\\&\dfrac{-18}{x}=\dfrac{y}{-18}}\Leftrightarrow\heva{&x=\pm6\\& y=\pm 54}$. Vậy $(x,y)=(6;54)$ hoặc $(x;y)=(-6;-54)$.
	}
\end{ex}
\begin{ex}%[1K2K7-3]
	Hai số hạng đầu của của một cấp số nhân là $2x+1$ và $4x^2-1$. Số hạng thứ ba của cấp số nhân là.
	\choice
	{$2x-1$}
	{$2x+1$}
	{\True $8x^3-4x^2-2x+1$}
	{$8x^3+4x^2-2x-1$}
	\loigiai{
		Công bội của cấp số nhân là $q=\dfrac{4x^2-1}{2x+1}=2x-1$. Vậy số hạng thứ ba của cấp số nhân là $(4x^2-1)(2x-1)=8x^3-4x^2-2x+1$.
	}
\end{ex}
\begin{ex}%[1K2B7-1]
	Trong các dãy số $(u_n)$ cho bởi số hạng tổng quát nu sau, dãy số nào là một cấp số nhân
	\choice
	{\True 
		$u_n=\dfrac{1}{3^{n-2}}$}
	{$u_n=\dfrac{1}{3^{n}}-1$}
	{$u_n=n+\dfrac{1}{3}$}
	{$u_n=n^2-\dfrac{1}{3}$}
	\loigiai{
		Dãy $u_n=\dfrac{1}{3^{n-2}}=3\left(\dfrac{1}{3}\right)^{n-1}$ là cấp số nhân có $u_1=3,q=\dfrac{1}{3}$.
	}
\end{ex}
\begin{ex}%[1K2B7-1]
	Một cấp số nhân có $6$ số hạng, số hạng đầu bằng $2$ và số hạng thứ sáu bằng $486$. Tìm công bội $q$ của cấp số nhân đã cho.
	\choice
	{\True $q=3$}
	{$q=-3$}
	{$q=2$}
	{$q=-2$}
	\loigiai{
		Ta có $\heva{&u_1=2\\&u_6=486}\Rightarrow u_6=u_1q^5\Leftrightarrow 486=2.q^5\Leftrightarrow q=3$.
	}
\end{ex}
\begin{ex}%[1K2B7-1]
	Cho cấp số nhân $\left(u_n\right)$ có $u_1=-3$ và $q=\dfrac{2}{3}$ Mệnh đề nào sau đây đúng.
	\choice
	{$u_5=-\dfrac{27}{16}$}
	{\True $u_5=-\dfrac{16}{27}$}
	{$u_5=\dfrac{16}{27}$}
	{$u_5=\dfrac{27}{16}$}
	\loigiai{
		Ta có $\heva{&u_1=-3\\&q=\dfrac{2}{3}}\Rightarrow u_5=u_1.q^4=-3.\left(\dfrac{2}{3}\right)^4=-\dfrac{16}{27}$.
	}
\end{ex}
\begin{ex}%[1K2K7-3]
	Cho cấp số nhân $\left(u_n\right)$ có $u_1=3$ và $q=-2$. Số $192$ là số hạng thứ mấy của cấp số nhân đã cho.
	\choice
	{$5$}
	{$6$}
	{\True $7$}
	{Không là số hạng của cấp số đã cho}
	\loigiai{
		Ta có $u_n=u_1.q^{n-1}\Leftrightarrow 192=3.(-2)^{n-1}\Leftrightarrow n=7$.
	}
\end{ex}
\begin{ex}%[1K2K7-3]
	Một cấp số nhân có công bội bằng $3$ và số hạng đầu bằng $5$. Biết số hạng chính giữa là $32805$. Hỏi cấp số nhân đã cho có bao nhiêu số hạng?
	\choice
	{$18$}
	{\True $17$}
	{$16$}
	{$9$}
	\loigiai{
		Ta có $u_n=u_1.q^{n-1}\Leftrightarrow 32805=3.5^{n-1}\Leftrightarrow n=9$. Vậy $u_9$ là số hạng chính giữa của cấp số nhân, nên cấp số nhân đã cho có $17$ số hạng.
	}
\end{ex}
\begin{ex}%[1K2K7-5]
	Cho cấp số nhân $\left(u_n\right)$ có $u_1=-3$ và $q=-2$. Tính tổng $10$ số hạng đầu tiên của cấp số nhân đã cho.
	\choice
	{$S_{10}=-511$}
	{$S_{10}=-1025$}
	{$S_{10}=1025$}
	{\True $S_{10}=1023$}
	\loigiai{
		Ta có $\heva{&u_1=-3\\&q=-2}\Rightarrow S_{10}=u_1.\dfrac{q^{n}-1}{q-1}=(-3).\dfrac{(-2)^{10}-1}{-2-1}=1023$.
	}
\end{ex}
\begin{ex}%[1K2G7-5]
	Cho cấp số nhân có các số hạng lần lượt là $1;4;16;64;...$. Gọi $S_n$ là tổng của $n$ số hạng đầu tiên của cấp số nhân đó. Mệnh đề nào sau đây đúng?
	\choice
	{$S_n=4^{n-1}$}
	{$S_n=\dfrac{n\left(1+4^{n-1}\right)}{2}$}
	{\True $S_n=\dfrac{4^n-1}{3}$}
	{$S_n=\dfrac{4\left(4^n-1\right)}{3}$}
	\loigiai{
		Ta có $\heva{&u_1=-3\\&q=4}\Rightarrow S_n=u_1.\dfrac{q^{n}-1}{q-1}=\dfrac{4^n-1}{3}$.
	}
\end{ex}
\begin{ex}%[1K2G7-1]
	Số hạng thứ hai, số hạng đầu và số hạng thứ ba của một cấp số cộng với công sai khác $0$ theo thứ tự đó lập thành một cấp số nhân với công bội $q$. Tìm $q$.
	\choice
	{$q=2$}
	{\True $q=-2$}
	{$q=-\dfrac{3}{2}$}
	{$q=\dfrac{3}{2}$}
	\loigiai{
		Giả sử ba số hạng $a;b;c$ lập thành cấp số cộng thỏa yêu cầu, khi đó $b;a;c$ theo thứ tự đó lập thành cấp số nhân công bội $q$. Ta có $\heva{&a+c=2b\\&a=bq\\&c=bq^2}\Rightarrow bq+bq^2=2b\Leftrightarrow\heva{&b=0\\&q^2+q-2=0}$.\\
		Nếu $b=0\Rightarrow a=b=c=0$ nên $a;b;c$ là cấp số cộng công sai $d=0$ (vô lí).\\
		Nếu $q^2+q-2=0\Leftrightarrow\hoac{&q=1\\&q=-2}$. Nếu $q=1\Rightarrow a=b=c$ (vô lí), do đó $q=-2$.
	}
\end{ex}
\begin{ex}%[1K2G7-1]
	Cho bố số $a,b,c,d$ biết rằng $a,b,c$ theo thứ tự đó lập thành một cấp số nhân công bội $q>1$,
	còn $b,c,d$ theo thứ tự đó lập thành cấp số cộng. Tìm $q$ biết rằng $a+d=14$ và $b+c=12$.
	\choice
	{$q=\dfrac{18+\sqrt{73}}{24}$}
	{\True $q=\dfrac{19+\sqrt{73}}{24}$}
	{$q=\dfrac{20+\sqrt{73}}{24}$}
	{$q=\dfrac{21+\sqrt{73}}{24}$}
	\loigiai{
		Giả sử $a,b,c$ lập thành cấp số cộng công bội $q$. Khi đó theo giả thiết ta có\\
		$\heva{&b=aq\\&c=aq^2\\&b+d=2c\\&a+d=14\\&c+d=12}\Rightarrow\heva{&aq+d=aq^2,&(1)\\&a+d=14,&(2)\\&a\left(q+q^2\right)=12,&(3)}$.\\
		Nếu $q=0\Rightarrow b=c=d=0$ (vô lý).\\
		Nếu $q=-1\Rightarrow b=-a=-c\Rightarrow b+c=0$ (vô lý).\\
		Vậy $q\ne 0,q\ne -1$, từ $(2)$ và $(3)$, ta có $d=14-a$ và $a=\dfrac{12}{q+q^2}$, thay vào $(1)$, ta được\\
		$\dfrac{12q}{q+q^2}+\dfrac{14q^2+14q-12}{q+q^2}=\dfrac{24q^3}{q+q^2}\Leftrightarrow 12q^3-7q^2-13q+6=0\Leftrightarrow q=\dfrac{19\pm \sqrt{73}}{24}$.\\
		Mà $q>1$ nên $q=\dfrac{19+\sqrt{73}}{24}$.
	}
\end{ex}
\begin{ex}%[1K2G7-5]
	Gọi $S=1+11+111+\cdots+111\ldots1$ ($n$ số $1$) thì $S$ nhận giá trị nào sau đây?
	\choice
	{$S=\dfrac{10^n-1}{81}$}
	{$S=10\cdot\dfrac{10^n-1}{81}$}
	{$S=10\cdot\dfrac{10^n-1}{81}-1$}
	{\True $S=\dfrac{1}{9}\left[10\cdot\dfrac{10^n-1}{9}-1\right]$}
	\loigiai{
		Ta có $S=\dfrac{1}{9}\left(9+99+999+\cdots+999\ldots9\right)=\dfrac{1}{9}\left(10+100+1000+\cdots+100\ldots0-n\right)=\dfrac{1}{9}\left[10\cdot\dfrac{10^n-1}{9}-1\right]$.
	}
\end{ex}
\begin{ex}%[1K2G7-7]
	Biết rằng $S=1+2\cdot3+3\cdot3^2+\cdots+11.3^{10}=a+\dfrac{21\cdot3^b}{4}$. Tính $P=a+\dfrac{b}{4}$.
	\choice
	{$P=1$}
	{$P=2$}
	{\True $P=3$}
	{$P=4$}
	\loigiai{
		Từ giả thiết suy ra $3S=3+2\cdot3^2+3\cdot3^3+\cdots+11\cdot3^{11}$.\\
		Do đó $-2S=S-3S=1+3+3^2+3^3+\cdots+3^{10}-10.3^{11}=\dfrac{1-3^{11}}{1-3}-11\cdot3^{11}\Rightarrow S=\dfrac{1}{4}+\dfrac{21}{4}\cdot3^{11}$.\\
		Vậy $a=\dfrac{1}{4},b=11$, suy ra $P=3$.
	}
\end{ex}
\begin{ex}%[1K2K7-1]
	Một cấp số nhân có ba số hạng là $a,b,c$ (theo thứ tự đó) trong đó các số hạng đều khác $0$ và công bội $q\ne 0$. Mệnh đề nào sau đây là đúng.
	\choice
	{$\dfrac{1}{a^2}=\dfrac{1}{bc}$}
	{\True $\dfrac{1}{b^2}=\dfrac{1}{ac}$}
	{$\dfrac{1}{c^2}=\dfrac{1}{ba}$}
	{$\dfrac{1}{a}+\dfrac{1}{b}=\dfrac{2}{c}$}
	\loigiai{
		Ta có $ac=b^2\Rightarrow \dfrac{1}{b^2}=\dfrac{1}{ac}$
	}
\end{ex}
\begin{ex}%[1K2K7-3]
	Bốn góc của một tứ giác tạo thành cấp số nhân và góc lớn nhất gấp $27$ lần góc nhỏ nhất. Tổng của góc lớn nhất và góc bé nhất bằng.
	\choice
	{$56^\circ$}
	{$102^\circ$}
	{\True $252^\circ$}
	{$168^\circ$}
	\loigiai{
		Giả sử $4$ góc $A, B, C, D$ (với $A<B<C<D$) theo thứ tự đó lập thành cấp số nhân thỏa yêu cầu với công bội $q$.\\
		Ta có $\heva{&A+B+C+D=360\\&D=27A}\Leftrightarrow\heva{&A\left(1+q+q^2+q^3\right)=360\\&Aq^3=27A}\Leftrightarrow\heva{&q=3\\&A=9\\&D=243}\Rightarrow A+D=252$.
	}
\end{ex}
\begin{ex}%[1K2G7-7]
	Người ta thiết kế một cái tháp gồm $11$ tầng. Diện tích bề mặt trên của mỗi tầng bằng nữa diện tích của mặt trên của tầng ngay bên dưới và diện tích mặt trên của tầng $1$ bằng nửa diện tích của đế tháp (có diện tích là $12288m^2$). Tính diện tích mặt trên cùng.
	\choice
	{\True $6m^2$}
	{$8m^2$}
	{$10m^2$}
	{$12m^2$}
	\loigiai{
		Diện tích bề mặt của mỗi tầng (kể từ $1$) lập thành một cấp số nhân có công bội $q=\dfrac{1}{2}$ và
		$u_1=\dfrac{12288}{2}=6144$. Khi đó diện tích mặt trên cùng là $u_{11}=u_1\cdot q^{10}=6144\cdot\left(\dfrac{1}{2}\right)^{10}=6$.
	}
\end{ex}
\begin{ex}%[1K2G7-7]
	Một du khách vào chuồng đua ngựa đặt cược, lần đầu đặt $20000$ đồng, mỗi lần sau tiền đặt gấp
	đôi lần tiền đặt cọc trước. Người đó thua $9$ lần liên tiếp và thắng ở lần thứ $10$. Hỏi du khác trên thắng hay thua bao nhiêu?
	\choice
	{Hoà vốn}
	{Thua $20000$ đồng}
	{\True Thắng $20000$ đồng}
	{Thua $40000$ đồng}
	\loigiai{
		Số tiền du khác đặt trong mỗi lần (kể từ lần đầu) là một cấp số nhân có $u_1=20000$ và công bội $q=2$. Du khách thua trong $9$ lần đầu tiên nên tổng số tiền thua là $S_9=u_1.\dfrac{q^9-1}{q-1}=20000\cdot\dfrac{2^9-1}{2-1}=10220000$.\\
		Số tiền mà du khách thắng trong lần thứ $10$ là $u_{10}=u_1\cdot q^9=20000\cdot2^9=10240000$.\\
		Ta có $u_{10}-S_9=20000>0$ nên du khách thắng $20000$.
	}
\end{ex}
\Closesolutionfile{ans}
% \begin{indapan}{10}
% 	{ans/ans-1K2-Ontapchuong2}
% \end{indapan}

%Chương III
% \setcounter{chapter}{2}
\setcounter{subsubsection}{0}
\setcounter{ex}{0}
\setcounter{bt}{0}
\chap{Một số yếu tố thống kê và xác suất}
% \section{Các số đặc trưng đo xu thế trung tâm cho mẫu số liệu ghép nhóm}

\subsection{Tóm tắt lý thuyết}
\begin{tomtat}
	\subsubsection{Mẫu số liệu ghép nhóm}
		\begin{enumerate}
			\item \textbf{\textit{Mẫu số liệu ghép nhóm}} là mẫu số liệu cho dưới dạng bảng tần số ghép nhóm.
			\item Mỗi số liệu gồm một số giá trị của mẫu số liệu được ghép nhóm theo một tiêu chí xác định có dạng $\left[a;b\right)$, trong đó $a$ là \textit{đầu mút trái}, $b$ là \textit{đầu mút phải}. Độ dài nhóm là $b-a$.
			\item \textbf{\textit{Tần số tích luỹ}} của một nhóm là số số liệu trong mẫu số liệu có giá trị nhỏ hơn giá trị đầu mút phải của nhóm đó. Tần số tích luỹ của nhóm $1$, nhóm $2$, $\ldots$, nhóm $m$ kí hiệu lần lượt là $cf_1$, $cf_2$, $\ldots$, $cf_m$.
		\end{enumerate}
	\subsubsection{Số trung bình cộng (số trung bình)}
		\begin{enumerate}
			\item Trung điểm $x_i$ của nửa khoảng (tính bằng trung bình cộng của hai đầu mút) ứng với nhóm $i$ là \textit{giá trị đại diện} của nhóm đó.
			\item \textit{Số trung bình cộng} của mẫu số liệu ghép nhóm, kí hiệu $\overline{x}$, được tính theo công thức 
				$$\overline{x} = \dfrac{n_1x_1 + n_2x_2 + \ldots + n_mx_m}{n}.$$
		\end{enumerate}
	\subsubsection{Trung vị}
	Để tính trung vị của mẫu số liệu ghép nhóm, ta làm như sau:
\begin{itemize}
    \item \textbf{Bước 1:} Xác định nhóm chứa trung vị. Giả sử đó là nhóm thứ $p : [a_p; a_{p + 1})$.
    \item \textbf{Bước 2:} Trung vị là 
    $$M_e = a_p + \dfrac{\dfrac{n}{2} - (m_1 + \cdots + m_{p-1})}{m_p} \cdot ( a_{p + 1} - a_p)$$
    trong đó $n$ là cỡ mẫu, $m_p$ là tần số nhóm $p$. Với $p=1$ ta quy ước $m_1 + \cdots + m_{p-1} = 0$.
\end{itemize}
		\begin{note}
			Nhóm chứa trung vị là nhóm đầu tiên có tần số tích luỹ $cf_p=m_1 + \cdots + m_{p}$ lớn hơn hoặc bằng $\dfrac{n}{2}$
		\end{note}
	\subsubsection{Tứ phân vị}Để tính tứ phân vị thứ nhất $Q_1$ của mẫu số liệu ghép nhóm, trước hết ta xác định nhóm chứa $Q_1$, giả sử đó là nhóm thứ  $p:\left[a_p;a_{p+1} \right)$. Khi đó 
	$$Q_1=a_p+\dfrac{\dfrac{n}{4}-\left(m_1+\cdots+m_{p-1}\right)}{m_p}\cdot \left(a_{p+1}-a_p\right).$$
	trong đó, $n$ là cỡ mẫu, $m_p$ là tần số nhóm $p$. Với $p=1$, ta quy ước $m_1+\cdots+m_{p-1}=0$.\\
	Để tính tứ phân vị thứ ba $Q_3$ của mẫu số liệu ghép nhóm, trước hết ta xác định nhóm chứa $Q_3$, giả sử đó là nhóm thứ  $p:\left[a_p;a_{p+1} \right)$. Khi đó 
	$$Q_3=a_p+\dfrac{\dfrac{3n}{4}-\left(m_1+\cdots+m_{p-1}\right)}{m_p}\cdot \left(a_{p+1}-a_p\right).$$
	Trong đó $n$ là cỡ mẫu, $m_p$ là tần số nhóm $p$. Với $p=1$, ta quy ước $m_1+\cdots+m_{p-1}=0$.\\
	Tứ phân vị thứ hai $Q_2$ chính là trung vị $M_e$.\\
\subsubsection{Mốt của mẫu số liệu ghép nhóm}
Để tìm mốt của mẫu số liệu ghép nhóm, ta thực hiện theo các bước sau:
\begin{enumerate}
	\item [Bước 1.] Xác định nhóm có tần số lớn nhất (gọi là nhóm chứa mốt), giả sử là nhóm $j:\left[a_j;a_{j+1} \right)$.
	\item [Bước 2.] Mốt được xác định là $M_o=a_j+\dfrac{m_i-m_{j-1}}{\left(m_i-m_{j-1}\right)+\left(m_i-m_{j+1}\right)}\cdot h$.\\
\end{enumerate}
trong đó, $m_j$ là tần số nhóm $j$ (quy ước $m_0=m_{k+1}=0$) và $h$ là độ dài của nhóm.	

\end{tomtat}
%=================================================
\setcounter{subsubsection}{0}
\setcounter{ex}{0}
\setcounter{bt}{0}
\subsection{Các dạng toán thường gặp}
\begin{dang}{Mẫu số liệu ghép nhóm}
\end{dang}
\subsubsection{Ví dụ minh hoạ}
\begin{vd}%[Cánh Diều]%[1C5Y1-1]
	\immini{
		\textbf{Bảng bên} biểu diễn mẫu số liệu ghép nhóm được cho dưới dạng bảng tần số ghép nhóm. Hãy cho biết 
		\begin{enumerate}
			\item Mẫu số liệu có bao nhiêu số liệu; bao nhiêu nhóm?
			\item Tần số của mỗi nhóm.
		\end{enumerate}
	}{
		\begin{tabular}{|c|c|}
			\hline
			\textbf{Nhóm} & \textbf{Tần số}\\ 
			\hline
			$\left[0;5\right)$ & $11$\\
			\hline
			$\left[5;10\right)$ & $31$\\
			\hline
			$\left[10;15\right)$ & $45$\\
			\hline
			$\left[15;20\right)$ & $21$\\
			\hline
			$\left[20;26\right)$ & $12$\\
			\hline
			& $n = 120$ \\
			\hline
		\end{tabular}
	}
	\loigiai{
		\begin{enumerate}
			\item Mẫu số liệu gồm $120$ số liệu và $5$ nhóm.
			\item Tần số lần lượt của các nhóm $1$, $2$, $3$, $4$, $5$ lần lượt là $11$, $31$, $45$, $21$, $12$.
		\end{enumerate}	
	}
\end{vd}
\begin{vd}%[CTST]%[1T5B1-1]
	Một cửa hàng đã thống kê số ba lô bán được mỗi ngày trong tháng 9 với kết quả cho như sau: \begin{center}
		\begin{tabular}{lllllllllllllll}
			$12$ & $29$ & $12$ & $19$ & $15$ & $21$ & $19$ & $29$ & $28$ & $12$ & $15$ & $25$ & $16$ & $20$ & $29$\\
			$21$ & $12$ & $24$ & $14$ & $10$ & $12$ & $10$ & $23$ & $27$ & $28$ & $18$ & $16$ & $10$ & $20$ & $21$
		\end{tabular}
	\end{center}
	Hãy chia mẫu số liệu trên thành 5 nhóm, lập bảng tần số ghép nhóm, hiệu chỉnh bảng tần số ghép nhóm và xác định giá trị đại diện cho mỗi nhóm.
	\loigiai{
		Khoảng biến thiên của mẫu số liệu trên là $R=29-10=19$.\\
		Độ dài mỗi nhóm $L>\dfrac{R}{k}=\dfrac{19}{5}=3{,}8$.\\
		Ta chọn $L=4$ và chia dữ liệu thành các nhóm $[10; 14)$, $[14; 18)$, $[18; 22)$, $[22; 26)$, $[26; 30)$.\\
		Khi đó ta có bảng tần số ghép nhóm sau
		\begin{center}
			\begin{tabular}{|c|c|c|c|c|c|}
				\hline \textbf{Cân nặng} &{$[10; 14)$} &{$[14; 18)$} &{$[18; 22)$} &{$[22; 26)$} &{$[26; 30)$} \\
				\hline \textbf{Giá trị đại diện} & $12$ & $16$ & $20$ & $24$ & $28$ \\
				\hline \textbf{Số ba lô bán được} & $8$ & $5$ & $8$ & $3$ & $6$ \\
				\hline
			\end{tabular}
		\end{center}
	}
\end{vd}
\begin{vd}%[KNTT]%[Ngọc Hiếu]%[1K3B8-1]
	Bảng thống kê sau cho biết thời gian chạy (phút) của $30$ vận động viên (VĐV) trong một giải chạy Marathon.
	\begin{center}
		\begin{tabular}{|c|c|c|c|c|c|c|c|c|c|c|c|c|}
			\hline
			Thời gian&$129$&$130$&$133$&$134$&$135$&$136$&$138$&$141$&$142$&$143$&$144$&$145$\\
			\hline
			Số VĐV&$1$&$2$&$1$&$1$&$1$&$2$&$3$&$3$&$4$&$5$&$2$&$5$\\
			\hline
		\end{tabular}
	\end{center}
	Hãy chuyển mẫu số liệu trên sang mẫu số liệu ghép nhóm gồm sáu nhóm có độ dài bằng nhau và bằng $3$.
	\loigiai{
		Giá trị nhỏ nhất là $129$, giá trị lớn nhất là $145$ nên khoảng biến thiên là $145-129=16$. Tổng độ dài của sáu nhóm là $18$. Để cho đối xứng, ta chọn đầu mút trái của nhóm đầu tiên là $127{,}5$ và đầu mút phải của nhóm cuối cùng là $145{,}5$ ta được các nhóm là $[127{,}5;130{,}5),\; [130{,5};133{,5}],\ldots , [142{,}5;145{,}5]$. Đếm số giá trị thuộc mỗi nhóm, ta có mẫu số liệu ghép nhóm như sau
		\begin{center}
			\fontsize{9}{1pt}
			{\begin{tabular}{|c|c|c|c|c|c|c|}
					\hline
					Thời gian&$[125{,}5;130{,}5)$&$[130{,}5;133{,}5)$&$[133{,}5;136{,}5)$&$[136{,}5;139{,}5)$&$[139{,}5;142{,}5)$&$[142{,}5;145{,}5)$\\
					\hline
					Số VĐV&$3$&$1$&$4$&$3$&$7$&$12$\\
					\hline
			\end{tabular}}
		\end{center}
	}
\end{vd}
\begin{vd}%[Cánh Diều]%[1C5B1-1]
	Một trường trung học phổ thông chọn $36$ học sinh nam của khối $11$, do chiều cao của các bạn học sinh đó và thu được mẫu số liệu sau (đơn vị: centimét):
	$$
	\begin{array}{llllllllllll}
		160 & 161 & 161 & 162 & 162 & 162 & 163 & 163 & 163 & 164 & 164 & 164 \\
		164 & 165 & 165 & 165 & 165 & 165 & 166 & 166 & 166 & 166 & 167 & 167 \\
		168 & 168 & 168 & 168 & 169 & 169 & 170 & 171 & 171 & 172 & 172 & 174
	\end{array}
	$$
	Lập bảng tần số ghép nhóm bao gồm cả tần số tích luỹ cho mẫu số liệu trên có $5$ nhóm ứng với $5$ nửa khoảng:
	$$
	\left[160;163 \right),\ \left[163;169 \right),\ \left[166;169 \right),\ \left[169;172 \right),\ \left[172;175 \right).
	$$
	\loigiai{
		Bảng tần số ghép nhóm bao gồm cả tần số tích luỹ như sau:
		\begin{center}
			\begin{tabular}{|c|c|c|}
				\hline
				\textbf{Nhóm} & \textbf{Tần số} & \textbf{Tần số tích luỹ}\\ 
				\hline
				$\left[169;163\right)$ & $6$ & $6$\\
				\hline
				$\left[163;166\right)$ & $12$ & $18$\\
				\hline
				$\left[166;169\right)$ & $10$ & $28$\\
				\hline
				$\left[169;172\right)$ & $5$ & $33$\\
				\hline
				$\left[172;175\right)$ & $3$ & $36$\\
				\hline
				& $n = 36$ &\\
				\hline
			\end{tabular}
		\end{center}
	}
\end{vd}
\subsubsection{Bài tập rèn luyện}
\begin{bt}%[KNTT]%[1K3B8-1]
	Trong các mẫu số liệu sau, mẫu nào là mẫu số liệu ghép nhóm? Đọc và giải thích mẫu số liệu ghép nhóm đó.
	\begin{enumerate}
		\item Số tiền mà sinh viên chi cho thanh toán cước điện thoại trong tháng.
		\begin{center}
			\begin{tabular}{|c|c|c|c|c|c|}
				\hline
				Số tiền (nghìn đồng)&$[0;50)$&$[50;100)$&$[100;150)$&$[150;200)$&$[200;250)$\\
				\hline
				Số sinh viên&$5$&$12$&$23$&$17$&$3$\\
				\hline
			\end{tabular}
		\end{center}
		\item Thống kê nhiệt độ tại một điểm trong $40$ ngày, ta có bảng số liệu sau
		\begin{center}
			\begin{tabular}{|c|c|c|c|c|}
				\hline
				Nhiệt độ $(^\circ$ C)&$[19;22)$&$[22;25)$&$[25;28)$&$[28;31)$\\
				\hline
				Số ngày&$7$&$15$&$12$&$6$\\
				\hline
			\end{tabular}
		\end{center}
	\end{enumerate}
	\loigiai{
		Cả hai mẫu số liệu trên đều là mẫu số lớp ghép nhóm.
		\begin{enumerate}
			\item Có năm nhóm là
			\begin{itemize}
				\item Dưới $50$ nghìn đồng có $5$ sinh viên.
				\item Từ $50$ đến dưới $100$ nghìn đồng có $12$ sinh viên.
				\item Từ $100$ đến dưới $150$ nghìn đồng có $23$ sinh viên.
				\item Từ $150$ đến dưới $200$ nghìn đồng có $17$ sinh viên.
				\item Từ $200$ đến dưới $250$ nghìn đồng có $3$ sinh viên.
			\end{itemize}
			\item Có bốn nhóm là
			\begin{itemize}
				\item Từ $19^\circ$ C đến dưới $22^\circ$ C có $7$ ngày.
				\item Từ $22^\circ$ C đến dưới $25^\circ$ C có $15$ ngày.
				\item Từ $25^\circ$ C đến dưới $28^\circ$ C có $12$ ngày.
				\item Từ $128^\circ$ C đến dưới $31^\circ$ C có $6$ ngày.
			\end{itemize}
		\end{enumerate}
	}
\end{bt}	
\begin{bt}%[KNTT]%[1K3B8-1]
	Số sản phẩm một công nhân làm được trong một ngày được cho như sau:
	\begin{center}
		\begin{tabular}{c c c c c c c c c c c c c}
			$18$&$25$&$39$&$12$&$54$&$27$&$46$&$25$&$19$&$8$&$36$&$22$&\\
			$20$&$19$&$17$&$44$&$5$&$18$&$23$&$28$&$25$&$34$&$46$&$27$&$16$
		\end{tabular}
	\end{center}
	Hãy chuyển mẫu số liệu sang dạng ghép nhóm với sáu nhóm có độ dài bằng nhau.
	\loigiai{
		Khoảng biến thiên là $54-5=49$.\\
		Ta chia thành các nhóm sau $[4{,}5;13); [13;21{,}5);[21{,}5;30);\ldots ;[47;55{,}5)$.\\
		Đếm số giá trị của mỗi nhóm, ta có bảng ghép nhóm sau:
		\begin{center}
			\begin{tabular}{|c|c|c|c|c|c|c|}
				\hline
				Số sản phẩm &$[4{,}5;13)$&$[13;21{,}5)$&$[21{,}5;30)$&$[30;38{,}5)$&$[38{,}5;47)$&$[47;55{,}5)$\\
				\hline
				Số công nhân&$3$&$7$&$8$&$2$&$4$&$1$\\
				\hline
			\end{tabular}
		\end{center}
	}
\end{bt}
\begin{bt}%[KNTT]%[1K3B8-1]
	Thời gian ra sân (giờ) của một số cựu cầu thủ ở giải ngoại hạng Anh qua các thời kì được cho như sau:
	\begin{center}
		\begin{tabular}{c c c c c c c c}
			$653$ & $632$ & $609$ & $572$ & $565$ & $535$ & $516$ & $514$ \\
			$508$ & $505$ & $504$ & $504$ & $503$ & $499$ & $496$ & $492$ 
		\end{tabular}
	\end{center}
	Hãy chuyển mẫu số liệu trên sang dạng ghép nhóm với bảy nhóm có độ dài bằng nhau.
	\loigiai{
		Khoảng biến thiên là $653-492=161$.\\
		Ta chia thành các nhóm sau $[492;515); [515;538);[538;561);\ldots; [630;653]$.\\
		Đếm số giá trị của mỗi nhóm, ta có bảng ghép nhóm sau:
		\begin{center}
			\begin{tabular}{|c|c|c|c|c|c|c|c|}
				\hline
				Thời gian &$[492;515)$&$[515;538)$&$[538;561)$&$[561;584)$&$[584;607)$&$[607;630)$&$[630;653]$\\
				\hline
				Số cầu thủ &$9$&$2$&$0$&$2$&$0$&$1$&$2$\\
				\hline
			\end{tabular}
		\end{center}
	}
\end{bt}

%===================================
\setcounter{subsubsection}{0}
\setcounter{ex}{0}
\setcounter{bt}{0}
\begin{dang}{Số trung bình cộng (số trung bình)}
\end{dang}
\subsubsection{Ví dụ minh hoạ}
\begin{vd}%[Cánh Diều]%[1C5Y1-2]
	\immini
	{
		Một nhà thực vật học đo chiều dài của $74$ lá cây (đơn vị: milimét) và thu được bảng tần số như bảng bên. Tính chiều dài trung bình của $74$ lá cây trên theo đơn vị milimét (làm tròn kết quả đến hàng phần trăm).
	}
	{
		\begin{tabular}{|c|c|c|}
			\hline
			\textbf{Nhóm} & \textbf{Giá trị đại diện} & \textbf{Tần số}\\ 
			\hline
			$\left[5{,}45;5{,}85\right)$ & $5{,}65$ & $5$\\
			$\left[5{,}85;6{,}25\right)$ & $6{,}05$ & $9$\\
			$\left[6{,}25;6{,}65\right)$ & $6{,}45$ & $15$\\
			$\left[6{,}65;7{,}05\right)$ & $6{,}85$ & $19$\\
			$\left[7{,}05;7{,}45\right)$ & $7{,}25$ & $16$\\
			$\left[7{,}45;7{,}85\right)$ & $7{,}65$ & $8$\\
			$\left[7{,}85;8{,}25\right)$ & $8{,}05$ & $2$\\
			\hline
			&  & $n = 74$\\
			\hline
		\end{tabular}
	}
	\loigiai{
		Chiều dài trung bình của $74$ lá cây mà nhà thực vật học đo xấp xỉ là 
		\[
		\overline{x} = \dfrac{5\cdot 5{,}65 + 9 \cdot 6{,}05 + 15\cdot 6{,}45 + 19\cdot 6{,}85 + 16 \cdot 7{,}25 + 8\cdot 7{,}65 + 2\cdot 8{,}05}{74} \approx 6{,}80\ (\text{mm}).
		\]
	}
\end{vd}
\begin{vd}%[CTST]%[1T5B1-2]
	Kết quả khảo sát cân nặng của $25$ quả cam ở mỗi lô hàng $A$ và $B$ được cho ở bảng sau:
	\begin{center}
		\begin{tabular}{|c|c|c|c|c|c|}
			\hline \multicolumn{1}{|c|}{Cân nặng $(\mathrm{g})$} &{$[150; 155)$} &{$[155; 160)$} &{$[160; 165)$} &{$[165; 170)$} &{$[170; 175)$} \\
			\hline Số quả cam ở lô hàng $A$ & 2 & 6 & 12 & 4 & 1 \\
			\hline Số quả cam ở lô hàng $B$ & 1 & 3 & 7 & 10 & 4 \\
			\hline
		\end{tabular}
	\end{center}
	\begin{enumerate}
		\item Hãy ước lượng cân nặng trung bình của mỗi quả cam ở lô hàng $A$ và lô hàng $B$.
		\item Nếu so sánh theo số trung bình thì cam ở lô hàng nào nặng hơn?
	\end{enumerate}
	\loigiai{
		Ta có bảng thống kê số lượng cam theo giá trị đại diện:
		\begin{center}
			\begin{tabular}{|c|c|c|c|c|c|}
				\hline \multicolumn{1}{|c|}{Cân nặng $(\mathrm{g})$} &{$152{,}5$} &{$157{,}5$} &{$162{,}5$} &{$167{,}5$} &$172{,}5$\\
				\hline Số quả cam ở lô hàng $A$ & 2 & 6 & 12 & 4 & 1 \\
				\hline Số quả cam ở lô hàng $B$ & 1 & 3 & 7 & 10 & 4 \\
				\hline
			\end{tabular}
		\end{center}
		\begin{enumerate}
			\item Cân nặng trung bình của mỗi quả cam ở lô hàng $A$ xấp xỉ bằng
			\[(2\cdot 152{,}5+6\cdot 157{,}5+12\cdot 162{,}5+4\cdot 167{,}5+1\cdot 172{,}5): 25=161{,}7\ (\mathrm{g}). \]
			Cân nặng trung bình của mỗi quả cam ở lô hàng $B$ xấp xỉ bằng
			\[(1\cdot 152{,}5+3\cdot 157{,}5+7\cdot 162{,}5+10\cdot 167{,}5+4\cdot 172{,}5): 25=165{,}1\ (\mathrm{g}). \]
			\item Nếu so sánh theo số trung bình thì cam ở lô hàng $B$ nặng hơn cam ở lô hàng $A$.
		\end{enumerate}
	}
\end{vd}
\begin{vd}%[KNTT]%[1K3B9-1]
	Tìm cân nặng trung bình của học sinh lớp $11D$ cho trong Bảng $3.5$.
	\begin{center}
		\begin{tabular}{|c|c|c|c|c|c|c|}
			\hline
			Cân nặng	& $\left[40{,}5;45{,}5 \right)$ & $\left[45{,}5;50{,}5 \right)$ & $\left[50{,}5;55{,}5 \right)$ & $\left[55{,}5;60{,}5 \right)$ & $\left[60{,}5;65{,}5 \right)$ & $\left[65{,}5;70{,}5 \right)$ \\
			\hline
			Số học sinh&$10$	& $7$ & $16$ &$4$  & $2$ & $3$ \\
			\hline
		\end{tabular}

		Bảng $3.5$. Cân nặng của học sinh lớp $11D$.	
	\end{center}	
	\loigiai{
		Trong mỗi khoảng cân nặng, giá trị đại diện là trung bình cộng của hai giá trị đầu mút nên ta có bảng sau
		\begin{center}
			\begin{tabular}{|c|c|c|c|c|c|c|}
				\hline
				Cân nặng (kg)	& $43$ & $48$ & $53$ & $58$ & $63$ & $68$ \\
				\hline
				Số học sinh &$10$ & $7$ & $16$ &$4$  & $2$ & $3$ \\
				\hline
			\end{tabular}
		\end{center}	
		Tổng số học sinh là $n=42$. Cân nặng trung bình của học sinh lớp $11D$ là $$\overline{x}=\dfrac{10\cdot 43+7\cdot 48+16\cdot 53+4\cdot 58+2\cdot 63+3\cdot 68}{42}\approx51{,}81\,\mathrm{(kg)}.$$
	}
\end{vd}
\subsubsection{Bài tập rèn luyện}
\begin{bt}%[Cánh diều]%[1C5B1-5]
	Mẫu số liệu dưới đây ghi lại tốc độ của $40$ ô tô khi đi qua một trạm đo tốc độ (đơn vị: km/h)
	\[
	\begin{array}{cccccccccc}
		48{,}5 & 43 & 50 & 55 & 45 & 60 & 53 & 55,5 & 44 & 65 \\
		51 & 62,5 & 41 & 44,5 & 57 & 57 & 68 & 49 & 46{,}5 & 53{,}5 \\
		61 & 49{,}5 & 54 & 62 & 59 & 56 & 47 & 50 & 60 & 61 \\
		49{,}5 & 52{,}5 & 57 & 47 & 60 & 55 & 45 & 47,5 & 48 & 61{,}5
	\end{array}
	\]
	\begin{enumerate}
		\item Lập bảng tần số ghép nhóm cho mẫu số liệu trên có sáu nhóm ứng với sáu nửa khoảng:
		\[
		[40 ; 45),[45 ; 50),[50 ; 55),[55 ; 60),[60 ; 65),[65 ; 70).
		\]
		\item Xác định số trung bình cộng của mẫu số liệu ghép nhóm trên.
	\end{enumerate}
	\loigiai{
		\begin{enumerate}
			\item Ta có bảng tần số ghép nhóm của mẫu số liệu trên như sau:
			\begin{center}
				\begin{tabular}{|c|c|c|c|}
					\hline
					\textbf{Nhóm} & \textbf{Giá trị đại diện} & \textbf{Tần số} & \textbf{Tần số tích luỹ}\\ 
					\hline
					$\left[40;45\right)$ & $42{,}5$ & $4$ & $4$\\
					$\left[45;50\right)$ & $47{,}5$ & $11$ & $15$\\
					$\left[50;55\right)$ & $52{,}5$ & $7$ & $22$\\
					$\left[55;60\right)$ & $57{,}5$ & $8$ & $30$\\
					$\left[60;65\right)$ & $62{,}5$ & $8$ & $38$\\
					$\left[65;70\right)$ & $67{,}5$ & $2$ & $40$\\
					\hline
					&  & $n = 40$ &\\
					\hline
				\end{tabular}
			\end{center}
			\item Trung bình cộng của mẫu số liệu trên là
			\[
			\overline{x} = \dfrac{42{,}5 \cdot 4 + 47{,}5 \cdot 11 + 52{,}5 \cdot 7+ 57{,}5 \cdot 8+ 62{,}5 \cdot 8 + 67{,}5 \cdot 2}{40} = 53{,}875\text{ (km/h)}.
			\]
			\item Ta thấy: Nhóm $2$ ứng với nửa khoảng $\left[45;50\right)$ là nhóm có tần số lớn nhất với $u=45$, $g=5$, $n_2 = 11$. Nhóm $1$ có tần số $n_1 = 4$, nhóm $3$ có tần số $n_3 = 7$.
		\end{enumerate}
	}
\end{bt}
\begin{bt}%[KNTT]%[1K3B9-4]
	Tuổi thọ (năm) của 50 bình ắc quy ô tô được cho như sau:
	\begin{center}
		\begin{tabular}{|c|c|c|c|c|c|c|}
			\hline
			Tuổi thọ (năm)	& $\left[2;2{,}5 \right)$ & $\left[2{,}5;3 \right)$ & $\left[3;3{,}5 \right)$&$\left[3{,}5;4 \right)$&$\left[4;4{,}5 \right)$&$\left[4{,}5;5 \right)$  \\
			\hline
			Tần số &$4$	& $9$ & $14$ &$11$  & $7$&$5$ \\
			\hline
		\end{tabular}
	\end{center}
	Tính tuổi thọ trung bình của $50$ bình ắc quy ô tô này.
	\loigiai{
		Ta có bảng sau
		\begin{center}
			\begin{tabular}{|c|c|c|c|c|c|c|}
				\hline
				Tuổi thọ (năm)	& $2{,}25$ & $2{,}75$ & $3{,}25$&$3{,}75$&$4{,}25$&$4{,}75$  \\
				\hline
				Tần số &$4$	& $9$ & $14$ &$11$  & $7$&$5$\\
				\hline
			\end{tabular}	
		\end{center}
		Tuổi thọ trung bình của 50 bình ắc quy ô tô này là
		$$\overline{x}=\dfrac{2{,}25\cdot 4+2{,}75\cdot 9+3{,}25\cdot 14+3{,}75\cdot 11+4{,}25\cdot 7+4{,}75\cdot 5}{50}=3{,}48 \, \text{(năm)}.$$
	}
\end{bt}
\begin{bt}%[KNTT]%[1K3B9-4]
	\immini{
		Phỏng vấn một số học sinh lớp $11$ về thời gian (giờ) ngủ của một buổi tối, thu được bảng số liệu ở bên. So sánh thời gian ngủ trung bình của các bạn học sinh nam và nữ.
	}
	{
		\begin{tabular}{|c|c|c|}
			\hline
			Thời gian	& Số học sinh nam & Số học sinh nữ\\
			\hline
			$\left[4;5 \right)$	& $6$ & $4$ \\
			\hline
			$\left[5;6 \right)$	& $10$ & $8$ \\
			\hline
			$\left[6;7 \right)$	& $13$ & $10$ \\
			\hline
			$\left[7;8 \right)$	& $9$ & $11$ \\
			\hline
			$\left[8;9 \right)$	& $7$ & $8$ \\
			\hline
		\end{tabular}
	}
	\loigiai{
		Trong mỗi khoảng thời gian, giá trị đại diện là trung bình cộng của giá trị hai đầu mút nên ta có bảng sau:
		\begin{center}
			\begin{tabular}{|c|c|c|}
				\hline
				Thời gian	& Số học sinh nam & Số học sinh nữ\\
				\hline
				$4{,}5$	& $6$ & $4$ \\
				\hline
				$5{,}5$	& $10$ & $8$ \\
				\hline
				$6{,}5$	& $13$ & $10$ \\
				\hline
				$7{,}5$	& $9$ & $11$ \\
				\hline
				$8{,}5$	& $7$ & $8$ \\
				\hline
			\end{tabular}	
		\end{center}
		Tổng số học sinh nam là $n_1=6+10+13+9+7=45$.\\ Thời gian ngủ trung bình của học sinh nam là:
		$$\overline{x_1}=\dfrac{4{,}5\cdot 6+5{,}5\cdot10+6{,}5\cdot13+7{,}5\cdot9+8{,}5\cdot7}{45}=\dfrac{587}{90}\approx 6{,}52\,\, \text{(giờ)}.$$
		Tổng số học sinh nữ là $n_2=4+8+10+11+8=41$. Thời gian ngủ trung bình của học sinh nữ là:
		$$\overline{x_2}=\dfrac{4,5\cdot4+5,5\cdot8+6,5\cdot10+7,5\cdot11+8,5\cdot8}{41}=\dfrac{555}{82}\approx 6{,}77 \,\,\text{(giờ)}.$$
		Vì $\overline{x_2}>\overline{x_1}$ nên thời gian ngủ trung bình của các bạn học sinh nữ lớn hơn thời gian ngủ trung bình của các bạn nam.
	}
\end{bt}
\begin{bt}%[KNTT]%[1K3B9-4]
	Quãng đường (km) từ nhà đến nơi làm việc của 40 công nhân một nhà máy được ghi lại như sau:
	\begin{center}
		\begin{tabular}{cccccccccccccccccccc}
			$5$	& $3$ &$10$ & $20$ & $25$ & $11$ & $13$ & $7$ & $12$ & $31$\\
			$19$ &$10$  &$12$  & $17$ & $18$ & $11$ & $32$ & $17$ &$16$  &$2$ \\
			$7$	& $9$ &$7$ & $8$ & $3$ & $5$ & $12$ & $15$ & $18$ & $3$\\
			$12$ &$14$  &$2$  & $9$ & $6$ & $15$ & $15$ & $7$ &$6$  &$12$
		\end{tabular}
	\end{center}
	\begin{enumerate}
		\item [a)] Ghép nhóm dãy số liệu trên thành các khoảng có độ rộng bằng nhau, khoảng đầu tiên là $\left[0;5\right)$. Tìm giá trị đại diện cho mỗi nhóm.
		\item [b)] Tính số trung bình của mẫu số liệu không ghép nhóm và mẫu số liệu ghép nhóm. Giá trị nào chính xác hơn?
	\end{enumerate}
	\loigiai{
		\begin{enumerate}
			\item [a)] Giá trị nhỏ nhất của mẫu số liệu là $2$, giá trị lớn nhất là $32$, khoảng đầu tiên của mẫu số liệu ghép nhóm là $\left[0;5\right)$ nên ta ghép nhóm mẫu số liệu như sau
			\begin{center}
				\begin{tabular}{|c|c|c|c|c|c|c|c|}
					\hline
					Quãng đường		 & $\left[0;5\right)$ & $\left[5;10\right)$ & $\left[10;15\right)$ & $\left[15;20\right)$ & $\left[20;25\right)$& $\left[25;30\right)$& $\left[30;35\right)$\\
					\hline
					Số công nhân		& $5$ & $11$ & $11$ & $9$ & $1$ & $1$ & $2$ \\
					\hline
				\end{tabular}
			\end{center}
			Trong mỗi khoảng, giá trị đại điện là trung bình cộng của hai giá trị đầu mút nên ta có bảng sau
			\begin{center}
				\begin{tabular}{|c|c|c|c|c|c|c|c|}
					\hline
					Quãng đường		 & $2{,}5$ & $7{,}5$ & $12{,}5$ & $17{,}5$ & $22{,}5$& $27{,}5$& $32{,}5$\\
					\hline
					Số công nhân		& $5$ & $11$ & $11$ & $9$ & $1$ & $1$ & $2$ \\
					\hline
				\end{tabular}
			\end{center}
			\item [b)] Số trung bình của mẫu số liệu không ghép nhóm là
			$$\overline{x}=\dfrac{5+3+10+\cdots +12}{40}=11{,}9.$$
			Số trung bình của mẫu số liệu ghép nhóm là
			$$\overline{x}=\dfrac{5\cdot 2{,}5+11\cdot 7{,}5+11\cdot 12{,}5+9\cdot 17{,}5+1\cdot 22{,}5+1\cdot 27{,}5+2\cdot 32{,}5}{40}=12{,}625.$$
			Số trung bình của mẫu số liệu không ghép nhóm sẽ chính xác hơn số trung bình của mẫu số liệu ghép nhóm vì số trung bình của dữ liệu không ghép nhóm sử dụng chính xác các số liệu, còn số trung bình của dữ liệu ghép nhóm sử dụng giá trị đại diện của mỗi khoảng ghép nhóm.
		\end{enumerate}
	}
\end{bt}
\begin{bt}%[CTST]%[1T5B1-2]
	Anh Văn ghi lại cự li 30 lần ném lao của mình ở bảng sau (đơn vị: mét):
	\begin{center}
		\begin{tabular}{|c|c|c|c|c|c|c|c|c|c|}
			\hline $72{,}1$ & $72{,}9$ & $70{,}2$ & $70{,}9$ & $72{,}2$ & $71{,}5$ & $72{,}5$ & $69{,}3$ & $72{,}3$ & $69{,}7$ \\
			\hline $72{,}3$ & $71{,}5$ & $71{,}2$ & $69{,}8$ & $72{,}3$ & $71{,}1$ & $69{,}5$ & $72{,}2$ & $71{,}9$ & $73{,}1$ \\
			\hline $71{,}6$ & $71{,}3$ & $72{,}2$ & $71{,}8$ & $70{,}8$ & $72{,}2$ & $72{,}2$ & $72{,}9$ & $72{,}7$ & $70{,}7$ \\
			\hline
		\end{tabular}
	\end{center}
	\begin{enumerate}
		\item Tính cự li trung bình của mỗi lần ném.
		\item Tổng hợp lại kết quả ném của anh Văn vào bảng tần số ghép nhóm theo mẫu sau:
		\begin{center}
			\begin{tabular}{|c|c|c|c|c|c|}
				\hline Cự li $(\mathrm{m})$ &{$[69{,}2; 70)$} &{$[70; 70{,}8)$} &{$[70{,}8; 71{,}6)$} &{$[71{,}6; 72{,}4)$} &{$[72{,}4; 73{,}2)$} \\
				\hline Số lần & $?$ & $?$ & $?$ & $?$ & $?$ \\
				\hline
			\end{tabular}
		\end{center}
		\item Hãy ước lượng cự li trung bình mỗi lần ném từ bảng tần số ghép nhóm trên.
		\item Khả năng anh Văn ném được khoảng bao nhiêu mét là cao nhất?
	\end{enumerate}
	\loigiai{
		\begin{enumerate}
			\item Điểm tổng của mỗi đợt gồm 10 lần ném
			\begin{center}
				\begin{tabular}{|c|c|c|c|c|c|c|c|c|c|c|}
					\hline Điểm &Điểm &Điểm &Điểm &Điểm &Điểm &Điểm &Điểm &Điểm &Điểm &Tổng \\
					\hline $72{,}1$ & $72{,}9$ & $70{,}2$ & $70{,}9$ & $72{,}2$ & $71{,}5$ & $72{,}5$ & $69{,}3$ & $72{,}3$ & $69{,}7$ &$713{,}6$\\
					\hline $72{,}3$ & $71{,}5$ & $71{,}2$ & $69{,}8$ & $72{,}3$ & $71{,}1$ & $69{,}5$ & $72{,}2$ & $71{,}9$ & $73{,}1$ &$714{,}9$\\
					\hline $71{,}6$ & $71{,}3$ & $72{,}2$ & $71{,}8$ & $70{,}8$ & $72{,}2$ & $72{,}2$ & $72{,}9$ & $72{,}7$ & $70{,}7$ &$718{,}4$\\
					\hline
				\end{tabular}
			\end{center}
			Cự li trung bình của mỗi lần ném của anh Văn
			\[\overline{x}=\dfrac{713{,}6+714{,}9+718{,}4}{30}\approx71{,}56\ (\mathrm{m}). \]
			\item Bảng tần số ghép nhóm kết quả ném của anh Văn:
			\begin{center}
				\begin{tabular}{|c|c|c|c|c|c|}
					\hline Cự li $(\mathrm{m})$ &{$[69{,}2; 70)$} &{$[70; 70{,}8)$} &{$[70{,}8; 71{,}6)$} &{$[71{,}6; 72{,}4)$} &{$[72{,}4; 73{,}2)$} \\
					\hline Số lần & $4$ & $2$ & $7$ & $12$ & $5$ \\
					\hline
				\end{tabular}
			\end{center}
			\item Bảng tần số ghép nhóm kết quả ném của anh Văn (theo giá trị đại diện):
			\begin{center}
				\begin{tabular}{|c|c|c|c|c|c|}
					\hline Cự li $(\mathrm{m})$ &{$[69{,}2; 70)$} &{$[70; 70{,}8)$} &{$[70{,}8; 71{,}6)$} &{$[71{,}6; 72{,}4)$} &{$[72{,}4; 73{,}2)$} \\
					\hline Giá trị đại diện &$69{,}6$ &$70{,}4$ &$71{,}2$ &$72{,}0$ &$72{,}8$\\
					\hline Số lần & $4$ & $2$ & $7$ & $12$ & $5$ \\
					\hline
				\end{tabular}
			\end{center}
			Cự li trung bình mỗi lần ném của anh Văn qua bảng tần số ghép nhóm
			\[(69{,}6\cdot 4+70{,}4\cdot 2+71{,}2\cdot 7+72\cdot 12+72{,}8\cdot 5):30=71{,}52\ (\mathrm{m}).  \]
			\item Nhóm chứa mốt của mẫu số liệu trên là nhóm $[71{,}6; 72{,}4)$.\\
			Do đó $u_m=71{,}6$; $n_{m-1}=7$; $n_m=12$; $n_{m+1}=5$; $u_{m+1}-u_m=72{,}4-71{,}6=0{,}8$.\\
			Mốt của mẫu số liệu ghép nhóm là
			\[M_0=71{,}6+\dfrac{12-7}{(12-7)+(12-5)} \cdot 0{,}8=\dfrac{101}{14} \approx 71{,}93. \]
			Dựa vào kết quả trên thì khả năng anh Văn ném được cao nhất là khoảng $71{,}93$ mét.
		\end{enumerate}
	}
\end{bt}
\begin{bt}%[CTST]%[1T5B1-2]
	Người ta đếm số xe ô tô đi qua một trạm thu phí mỗi phút trong khoảng thời gian từ $9$ giờ đến $9$ giờ $30$ phút sáng. Kết quả được ghi lại ở bảng sau:
	\begin{center}
		\begin{tabular}{|c|c|c|c|c|c|c|c|c|c|c|c|c|c|c|}
			\hline $15$ & $16$ & $13$ & $21$ & $17$ & $23$ & $15$ & $21$ & $6$ & $11$ & $12$ & $23$ & $19$ & $25$ & $11$ \\
			\hline $25$ & $7$ & $29$ & $10$ & $28$ & $29$ & $24$ & $6$ & $11$ & $23$ & $11$ & $21$ & $9$ & $27$ & $15$ \\
			\hline
		\end{tabular}
	\end{center}
	\begin{enumerate}
		\item Tính số xe trung bình đi qua trạm thu phí trong mỗi phút.
		\item Tổng hợp lại số liệu trên vào bảng tần số ghép nhóm theo mẫu sau:
		\begin{center}
			\begin{tabular}{|c|c|c|c|c|c|}
				\hline Số xe &{$[6; 10]$} &{$[11; 15]$} &{$[16; 20]$} &{$[21; 25]$} &{$[26; 30]$} \\
				\hline Số lần & $?$ & $?$ & $?$ & $?$ & $?$ \\
				\hline
			\end{tabular}
		\end{center}
		\item Hãy ước lượng trung bình số xe đi qua trạm thu phí trong mỗi phút từ bảng tần số ghép nhóm trên.
	\end{enumerate}
	\loigiai{
		\begin{enumerate}
			\item 
%			Bảng tần số
%			\begin{center}
%				\begin{tabular}{|c|c|c|c|c|c|c|c|c|c|c|c|c|c|c|c|c|c|c|c|}
%					\hline Giá trị &$6$ & $7$ & $9$ & $10$ & $11$ & $12$ & $13$ & $15$ & $16$ & $17$ & $19$ & $21$ & $23$ & $24$ & $25$ & $27$ & $28$ & $29$ &\\
%					\hline Tần số &$2$ & $1$ & $1$ & $1$ & $4$ & $1$ & $1$ & $3$ & $1$ & $1$ & $1$ & $3$ & $3$ & $1$ & $2$ & $1$ & $1$ & $2$ &$N=30$\\
%					\hline
%				\end{tabular}
%			\end{center}
			Số xe trung bình đi qua trạm thu phí trong mỗi phút là
			\allowdisplaybreaks
			\begin{eqnarray*}
				\overline{x}&=&\dfrac{6\cdot 2+7+9+10+11\cdot 4+12+13+15\cdot 3}{30}\\
				&&+\dfrac{16+17+19+21\cdot 3+23\cdot 3+24+25\cdot 2+27+28+29\cdot 2}{30}\\
				&\approx& 17{,}43\ (\text{xe}).
			\end{eqnarray*}	
			\item Bảng tần số ghép nhóm
			\begin{center}
				\begin{tabular}{|c|c|c|c|c|c|}
					\hline Số xe &{$[6; 10]$} &{$[11; 15]$} &{$[16; 20]$} &{$[21; 25]$} &{$[26; 30]$} \\
					\hline Số lần & $5$ & $9$ & $3$ & $9$ & $4$ \\
					\hline
				\end{tabular}
			\end{center}
			\item Bảng tần số ghép nhóm (theo giá trị đại diện) được hiệu chỉnh lại như sau
			\begin{center}
				\begin{tabular}{|c|c|c|c|c|c|}
					\hline Số xe &{$[5{,}5; 10{,}5)$} &{$[10{,}5; 15{,}5)$} &{$[15{,}5; 20{,}5)$} &{$[20{,}5; 25{,}5)$} &{$[25{,}5; 30{,}5)$} \\
					\hline Giá trị đại diện &{$8$} &{$13$} &{$18$} &{$23$} &{$28$} \\
					\hline Số lần & $5$ & $9$ & $3$ & $9$ & $4$ \\
					\hline
				\end{tabular}
			\end{center}
			Số xe trung bình đi qua trạm qua bảng tần số ghép nhóm là
			\[\overline{x}=\dfrac{8\cdot 5+13\cdot 9+18\cdot 3+23\cdot 9+28\cdot 4}{30}\approx 17{,}67\ (\text{xe}). \]
		\end{enumerate}
	}
\end{bt}
\begin{bt}%[CTST]%[1T5B1-2]
	Một thư viện thống kê số lượng sách được mượn mỗi ngày trong ba tháng ở bảng sau:
	\begin{center}
		\begin{tabular}{|c|c|c|c|c|c|c|c|}
			\hline Số sách &{$[16; 20]$} &{$[21; 25]$} &{$[26; 30]$} &{$[31; 35]$} &{$[36; 40]$} &{$[41; 45]$} &{$[46; 50]$} \\
			\hline Số ngày & 3 & 6 & 15 & 27 & 22 & 14 & 5 \\
			\hline
		\end{tabular}
	\end{center}
	Hãy ước lượng số trung bình của mẫu số liệu ghép nhóm trên.
	\loigiai{
		Vì số lượng sách được mượn là số nguyên nên ta hiệu chỉnh bảng tần số ghép nhóm (theo giá trị đại diện) như sau
		\begin{center}
			{\footnotesize \begin{tabular}{|c|c|c|c|c|c|c|c|}
					\hline Số sách &{$[15{,}5; 20{,}5)$} &{$[20{,}5; 25{,}5)$} &{$[25{,}5; 30{,}5)$} &{$[30{,}5; 35{,}5)$} &{$[35{,}5; 40{,}5]$} &{$[40{,}5; 45{,}5)$} &{$[45{,}5; 50{,}5)$} \\
					\hline Giá trị đại diện &{$18$} &{$23$} &{$28$} &{$33$} &{$38$} &{$43$} &{$48$} \\
					\hline Số ngày & 3 & 6 & 15 & 27 & 22 & 14 & 5 \\
					\hline
			\end{tabular}}
		\end{center}
		Trung bình số lượng sách được mượn mỗi ngày trong 3 tháng của thư viện là
		\[\overline{x}=\dfrac{18\cdot 3+23\cdot 6+28\cdot 15+33\cdot 27+38\cdot 22+43\cdot 14+48\cdot 5}{92}\approx 34{,}58. \]
	}
\end{bt}
\begin{bt}%[CTST]%[1T5B1-2]
	Kết quả đo chiều cao của $200$ cây keo $3$ năm tuổi ở một nông trường được biểu diễn ở biểu đồ dưới đây.
	\begin{center}
		\begin{tikzpicture}[scale=1,font=\scriptsize]
			\def\hoanh{11.5};
			\def\tung{6.5};
			\def\mau{cyan};
			\foreach \x/\n in{1/20,3/35,5/60,7/55,9/30}{\draw[line width=16mm,\mau] (\x,0)--++(0,{\n/10});
				\draw[dashed] (\x,{\n/10})node[above]{$\n$}--(0,{\n/10}) node[left]{$\n$};}
			\foreach \x/\p in {1/[8{,}5;8{,}8),3/[8{,}8;9{,}1),5/[9{,}1;9{,}4),7/[9{,}4;9{,}7),9/[9{,}7;10{,}0)}{\node[below] at (\x,0){\scriptsize $\p$};}
			\draw[->] (0,0)--(\hoanh,0) node[below]{($m$)};
			\draw[->] (0,0)node[below left]{$O$}--(0,\tung) node[left]{(Số cây)};
			\path (current bounding box.north) node[above]		{\textbf{Chiều cao 200 cây keo 3 năm tuổi}};
		\end{tikzpicture}
	\end{center}
	Hãy ước lượng số trung bình của mẫu số liệu ghép nhóm trên.
	\loigiai{
		Bảng tần số ghép nhóm (theo giá trị đại diện)
		\begin{center}
			\begin{tabular}{|c|c|c|c|c|c|}
				\hline Chiều cao &$[8{,}5; 8{,}8)$ &{$[8{,}8; 9{,}1)$} &{$[9{,}1; 9{,}4)$} &{$[9{,}4; 9{,}7)$} &{$[9{,}7; 10{,}0)$} \\
				\hline Giá trị đại diện &$8{,}65$ &$8{,}95$ &$9{,}25$ &$9{,}55$ &$9{,}85$ \\
				\hline Số cây & $20$ & $35$ & $60$ & $55$ & $30$\\
				\hline
			\end{tabular}
		\end{center}
		Chiều cao trung bình của $200$ cây keo 3 năm tuổi là
		\[\overline{x}=\dfrac{8{,}65\cdot 20+8{,}95\cdot 35+9{,}25\cdot 60+9{,}55\cdot 55+9{,}85\cdot 30}{200}\approx 9{,}31. \]
	}
\end{bt}
\begin{bt}%[CTST]%[1T5K2-2]
	Kiểm tra điện lượng của một số viên pin tiểu do một hãng sản xuất thu được kết quả như sau:
	\begin{center}
		\begin{tabular}{|c|c|c|c|c|c|}
			\hline 
			\begin{tabular}{c}
				\textbf{Điện lượng} \\	\textbf{(nghìn mAh)}
			\end{tabular} 
			& $ \left[ 0{,}9 ; 0{,}95\right)  $ & $ \left[ 0{,}95 ; 1{,}0\right)  $ & $ \left[ 1{,}0 ; 1{,}05\right)  $ &$ \left[ 1{,}05 ; 1{,}1\right)  $  &  $ \left[ 1{,}1 ; 1{,}15\right)  $\\ 
			\hline 
			\textbf{Số viên pin}& $ 10 $ & $ 20 $ & $ 35 $ & $ 15 $ & $ 5 $ \\ 
			\hline 
		\end{tabular} 
	\end{center}
	Hãy ước lượng số trung bình của mẫu số liệu ghép nhóm trên.
	\loigiai{
		Tìm số trung bình của mẫu số liệu ghép nhóm.\\
		Ta có bảng thống kê điện lượng của pin theo giá trị đại diện là:
		\begin{center}
			\begin{tabular}{|c|c|c|c|c|c|}
				\hline 
				\begin{tabular}{c}
					\textbf{Điện lượng} \\	\textbf{(nghìn mAh)}
				\end{tabular} 
				& $ \left[ 0{,}9 ; 0{,}95\right)  $ & $ \left[ 0{,}95 ; 1{,}0\right)  $ & $ \left[ 1{,}0 ; 1{,}05\right)  $ &$ \left[ 1{,}05 ; 1{,}1\right)  $  &  $ \left[ 1{,}1 ; 1{,}15\right)  $\\ 
				\hline 
				\textbf{Giá trị đại diện}& $ 0{,}925 $ & $ 0{,}975 $ & $ 1{,}025 $ & $ 1{,}075 $ & $ 1{,}125 $ \\ 
				\hline
				\textbf{Số viên pin}& $ 10 $ & $ 20 $ & $ 35 $ & $ 15 $ & $ 5 $ \\ 
				\hline 
			\end{tabular} 
		\end{center}
		Số trung bình của mẫu số liệu ghép nhóm theo dõi điện lượng của một số viên pin xấp xỉ bằng $$\dfrac{0{,}925\cdot 10 + 0{,}975\cdot 20 +1{,}025 \cdot 35 +1{,}075 \cdot 15+1{,}125 \cdot 5}{10+20+35+15+5}\approx 1{,}016.$$
	}
\end{bt}

%===================================
\setcounter{subsubsection}{0}
\setcounter{ex}{0}
\setcounter{bt}{0}
\begin{dang}{Trung vị}
\end{dang}
\subsubsection{Ví dụ minh hoạ}
\begin{vd}%[Cánh Diều]%[1C5B1-3]
	\immini
	{
		Sau khi kiểm tra về số học sinh trong $100$ lớp học, người ta chia mẫu số liệu đó thành năm nhóm căn cứ vào số lượng học sinh của mỗi lớp (đơn vị: học sinh) và lập bảng tần số ghép nhóm bao gồm tần số tích luỹ như bảng bên. Tìm trung vị của mẫu số liệu đó.
	}
	{
		\begin{tabular}{|c|c|c|}
			\hline
			\textbf{Nhóm} & \textbf{Tần số} & \textbf{Tần số tích luỹ}\\ 
			\hline
			$\left[36;38\right)$ & $9$ & $9$\\
			$\left[38;40\right)$ & $15$ & $24$\\
			$\left[40;42\right)$ & $25$ & $49$\\
			$\left[42;44\right)$ & $30$ & $79$\\
			$\left[44;46\right)$ & $21$ & $100$\\
			\hline
			& $n = 100$ &\\
			\hline
		\end{tabular}
	}
	\loigiai{
		Số phần tử của mẫu là $n=100$. Ta có $\dfrac{n}{2} = \dfrac{100}{2} = 50$.\\
		Do $cf_3 = 49 < 50 < cf_4 = 79$ nên nhóm $4$ là nhóm đầu tiên có tần số tích luỹ lớn hơn hoặc bằng $50$.\\
		Xét nhóm $4$ là nhóm $\left[42;44\right)$ có $r=42$; $d=2$ và $n_4=30$ và nhóm $3$ là nhóm $\left[40;42\right)$ có $cf_3 = 49$.\\
		Khi đó trung vị của mẫu số liệu là 
		\[
		M_e = 42 + \dfrac{50 - 49}{30} \cdot 2 \approx 42\text{ (học sinh)}.
		\]
	}
\end{vd}
\begin{vd}%[KNTT]%[1K3B9-2]
	Thời gian (phút) truy cập internet mỗi buổi tối của một số học sinh được cho trong bảng sau:
	\begin{center}
		\begin{tabular}{|c|c|c|c|c|c|c|}
			\hline
			Thời gian (phút)	& $\left[9{,}5;12{,}5 \right)$ & $\left[12{,}5;15{,}5 \right)$ & $\left[15{,}5;18{,}5 \right)$ & $\left[18{,}5;21{,}5 \right)$ & $\left[21{,}5;24{,}5 \right)$ \\
			\hline
			Số học sinh&$3$	& $12$ & $15$ &$24$  & $2$  \\
			\hline
		\end{tabular}	
	\end{center}
	Tính trung vị của mẫu số liệu ghép nhóm này.
	\loigiai{
		Cỡ mẫu là $n=3+12+15+24+2=56$.\\
		Gọi $x_1,\,\ldots,\,x_{56}$ là thời gian vào internet của $56$ học sinh và giả sử dãy này đã được sắp xếp theo thứ tự tăng dần. Khi đó, trung vị là $\dfrac{x_{28}+x_{29}}{2}$. Do $2$ giá trị $x_{28},\,x_{29}$ thuộc nhóm $\left[15{,}5;18{,}5 \right)$ nên nhóm này chứa trung vị. Do đó, $p=3$; $a_3=15{,}5$; $m_3=15$; $m_1+m_2=3+12=15$; $a_4-a_3=3$ và ta có $$M_e=15{,}5+\dfrac{\dfrac{56}{2}-15}{15}\cdot 3=18{,}1.$$
	}
\end{vd}
\begin{vd}%[CTST]%[1T5B2-1]
	Kết quả khảo sát cân nặng của $ 25 $ quả bơ ở một lô hàng cho trong bảng sau:
	\begin{center}
		\begin{tabular}{|c|c|c|c|c|c|}
			\hline 
			\textbf{Cân nặng}\textbf{ (g)}	& $ \left[150 ; 155 \right) $ & $ \left[ 155 ; 160\right)  $ & $ \left[160 ; 165\right)  $ & $ \left[ 165 ; 170\right)  $ & $ \left[170 ; 175 \right)  $ \\ 
			\hline 
			\textbf{Số quả bơ}	& $ 1 $ & $ 7 $ & $ 12 $ & $ 3 $ & $ 2 $ \\ 
			\hline 
		\end{tabular} 
	\end{center}
	Hãy tìm trung vị của mẫu số liệu ghép nhóm trên.
	\loigiai{
		Gọi $ x_1; x_2; \ldots ; x_{25} $ là cân nặng của $ 25$ quả bơ xếp theo thứ tự không giảm.\\
		Do $ x_1\in \left[150 ; 155 \right) $; $ x_2, \ldots, x_8 \in \left[ 155 ; 160\right) $; $ x_9, \ldots, x_{20} \in \left[ 160 ; 165\right) $ nên trung vị của mẫu số liệu $ x_1; x_2; \ldots; x_{25} $ là $ x_{13}\in \left[ 160 ; 165\right)$.\\
		Ta xác định được $ n=25 $, $ n_m=12 $, $ C=1+7=8 $, $ u_m=160 $, $ u_{m+1}=165 $.\\
		Vậy trung vị của mẫu số liệu ghép nhóm là $$ M_e=160+\dfrac{\dfrac{25}{2}-8}{12}\cdot(165-160) =161{,}875.$$
	}
\end{vd}
\begin{vd}%[CTST]%[1T5K2-1]
	Trong tuần lễ bảo vệ môi trường, các học sinh khối $ 11 $ tiến hành thu nhặt vỏ chai nhựa để tái chế. Nhà trường thống kê kết quả thu nhặt vỏ chai của học sinh khối $ 11 $ ở bảng sau
	\begin{center}
		\begin{tabular}{|c|c|c|c|c|c|}
			\hline 
			\textbf{Số vỏ chai nhựa}	& $ \left[ 11 ; 15\right]  $ & $ \left[ 16 ; 29\right]  $ & $ \left[21 ; 25 \right]  $ & $ \left[ 26 ; 30\right]  $ & $ \left[31 ; 35 \right]  $ \\ 
			\hline 
			\textbf{Số học sinh}	& $ 53 $ & $ 82 $ & $ 48 $ & $ 39 $ & $ 18 $ \\ 
			\hline 
		\end{tabular} 
	\end{center}
	Hãy tìm trung vị của mẫu số liệu ghép nhóm trên.
	\loigiai{
		Do số vỏ chai là số nguyên nên ta hiệu chỉnh lại như sau:
		\begin{center}
			\begin{tabular}{|c|c|c|c|c|c|}
				\hline 
				\textbf{Số vỏ chai nhựa}	& $ \left[ 10{,}5 ; 15{,}5\right) $ & $ \left[ 15{,}5 ; 20{,}5\right) $ & $ \left[ 20{,}5 ; 25{,}5\right) $ & $ \left[ 25{,}5 ; 30{,}5\right) $ & $ \left[30{,}5 ; 35{,}5 \right)  $ \\ 
				\hline 
				\textbf{Số học sinh}& $ 53 $ & $ 82 $ & $ 48 $ & $ 39 $ & $ 18 $ \\ 
				\hline 
			\end{tabular} 
		\end{center}
		Số học sinh tham gia thu nhặt vỏ chai nhựa là $$ n=53+82+48+39+18=240.$$
		Gọi $ x_1; x_2; \ldots ; x_{240} $ lần lượt là số vỏ chai $ 240 $ học sinh khối $ 11 $ thu nhặt được xếp theo thứ tự không giảm.\\
		Do $ x_1, \ldots, x_{53}\in \left[10{,}5 ; 15{,}5 \right) $; $ x_{54}, \ldots, x_{135}\in \left[ 15{,}5 ; 20{,}5\right)$ nên trung vị của mẫu số liệu $ x_1; x_2; \ldots;x_{240} $ là $$ \dfrac{1}{2}\left( x_{120}+x_{121}\right)\in \left[ 15{,}5 ; 20{,}5\right).$$
		Ta xác định được $ n=240$; $ n_m=82 $; $ C=53 $; $ u_m=15{,}5 $; $ u_{m+1}=20{,}5 $.\\
		Trung vị của mẫu số liệu ghép nhóm là $$ M_e=15{,}5+\dfrac{\dfrac{240}{2}-53}{82}\cdot \left( 20{,}5-15{,}5\right)=\dfrac{803}{41}\approx 19{,}59. $$
	}
\end{vd}
\begin{vd}%[CTST]%[1T5K2-1]
	Trong một hội thao, thời gian chạy $200$m của một nhóm các vận động viên được ghi lại ở bảng sau
	\begin{center}
		\begin{tabular}{|c|c|c|c|c|c|}
			\hline 
			\textbf{Thời gian} \textbf{(giây)}& $ \left[21 ; 21{,}5 \right)  $ & $ \left[ 21{,}5 ; 22\right)  $ & $ \left[ 22 ; 22{,}5\right)  $ & $ \left[ 22{,}5 ; 23\right)  $ & $ \left[ 23 ; 23{,}5\right)  $ \\ 
			\hline 
			\textbf{Số vận động viên} & $ 5 $ & $ 12 $ & $ 32 $ & $ 45 $ & $ 30 $ \\ 
			\hline 
		\end{tabular} 
	\end{center}
	Dựa vào bảng số liệu trên, ban tổ chức muốn chọn ra khoảng $ 50 \% $ số vận động viên chạy nhanh nhất để tiếp tục thi vòng $ 2 $. Ban tổ chức nên chọn các vận động viên có thời gian chạy không quá bao nhiêu giây?
	\loigiai{
		Số vận động viên tham gia là $$n=5+12+32+45+30=124.$$
		Gọi $ x_1; x_2; \ldots ; x_{124} $ lần lượt là thời gian chạy $ 200 $ m của $ 124 $ vận động viên được xếp theo thứ tự không giảm.\\
		Do $ x_1, \ldots, x_5 \in \left[ 21 ; 21{,}5 \right)$, $ x_6, \ldots, x_{17} \in \left[ 21{,}5 ; 22\right) $, $ x_{18}, \ldots, x_{49} \in \left[22 ; 22{,}5\right) $, $ x_{50},\ldots, x_{94} \in \left[ 22{,}5 ; 23\right) $ nên trung vị của mẫu số liệu $ x_1; x_2; \ldots ;x_{124} $ là
		$$\dfrac{1}{2}\cdot \left( x_{62}+x_{63}\right) \in  \left[ 22{,}5 ; 23\right).$$
		Ta xác định được $ n=124 $; $ n_m=45$; $ C=5+12+32=49 $; $ u_m= 22{,}5$; $ u_{m+1}=23$.\\
		Trung vị của mẫu số liệu ghép nhóm là $$M_e=22{,}5 +\dfrac{\dfrac{124}{2}-49}{45}\cdot \left( 23-22{,}5\right)= \dfrac{1019}{45}\approx 22{,}64.$$
		Vậy ban tổ chức nên chọn các vận động viên  có thời gian chạy không quá $ 22{,}64$ (giây) được tiếp tục thi vòng hai.
	}
\end{vd}
\subsubsection{Bài tập rèn luyện}
\begin{bt}%[Cánh diều]%[1C5B1-5]
	\immini
	{
		Bảng bên cho ta bảng tần số ghép nhóm số liệu thống kê chiều cao của $40$ mẫu cây ở một vườn thực vật (đơn vị: centimét). Xác định trung vị của mẫu số liệu ghép nhóm trên.
	}
	{
		\begin{tabular}{|c|c|c|}
			\hline
			\textbf{Nhóm} & \textbf{Tần số} & \textbf{Tần số tích luỹ}\\ 
			\hline
			$\left[30;40\right)$ & $4$ & $4$\\
			$\left[40;50\right)$ & $10$ & $14$\\
			$\left[50;60\right)$ & $14$ & $28$\\
			$\left[60;70\right)$ & $6$ & $34$\\
			$\left[70;80\right)$ & $4$ & $38$\\
			$\left[80;90\right)$ & $2$ & $40$\\
			\hline
			& $n = 40$ &\\
			\hline
		\end{tabular}
	}
	\loigiai{
		Ta có $\dfrac{n}{2} = 20$, mà $14<20<28$ nên nhóm $3$ là nhóm đầu tiên có tần số tích luỹ lớn hơn hoặc bằng $20$. \\
		Xét nhóm $3$ là nhóm $\left[50;60\right)$ có $r=50$, $d=10$, $n_3=14$ và nhóm $2$ có $cf_2 = 14$.\\
		Khi đó, tứ phân vị thứ hai (cũng là trung vị) là
		\[
		M_e = 50 + \dfrac{20 - 14}{14} \cdot 10 = 54{,}3\text{ (cm)}.
		\]
	}
\end{bt}
\begin{bt}%[Cánh diều]%[1C5B1-5]
	Mẫu số liệu sau ghi lại cân nặng của $30$ bạn học sinh (đơn vị: kilôgam)
	\[
	\begin{array}{cccccccccc}
		17 & 40 & 39 & 40{,}5 & 42 & 51 & 41{,}5 & 39 & 41 & 30\\
		40 & 42 & 40{,}5 & 39{,}5 & 41 & 40{,}5 & 37 & 39{,}5 & 40 & 41\\
		38{,}5 & 39{,}5 & 40 & 41 & 39 & 40{,}5 & 40 & 38{,}5 & 39{,}5 & 41{,}5
	\end{array}
	\]
	\begin{enumerate}
		\item Lập bảng tần số ghép nhóm cho mẫu số liệu trên có tám nhóm ứng với tám nửa khoảng:
		\[
		[15 ; 20),[20 ; 25),[25 ; 30),[30 ; 35),[35 ; 40),[40 ; 45),[45 ; 50),[50 ; 55).
		\]
		\item Xác định trung vị của mẫu số liệu ghép nhóm trên.
	\end{enumerate}
	\loigiai{
		\begin{enumerate}
			\item Ta có bảng tần số ghép nhóm của mẫu số liệu trên như sau:
			\begin{center}
				\begin{tabular}{|c|c|c|c|}
					\hline
					\textbf{Nhóm} & \textbf{Giá trị đại diện} & \textbf{Tần số} & \textbf{Tần số tích luỹ}\\ 
					\hline
					$\left[15;20\right)$ & $17{,}5$ & $1$ & $1$\\
					$\left[20;25\right)$ & $22{,}5$ & $0$ & $1$\\
					$\left[25;30\right)$ & $27{,}5$ & $0$ & $1$\\
					$\left[30;35\right)$ & $32{,}5$ & $1$ & $2$\\
					$\left[35;40\right)$ & $37{,}5$ & $10$ & $12$\\
					$\left[40;45\right)$ & $42{,}5$ & $17$ & $29$\\
					$\left[45;50\right)$ & $47{,}5$ & $0$ & $29$\\
					$\left[50;55\right)$ & $52{,}5$ & $1$ & $30$\\
					\hline
					&  & $n = 30$ &\\
					\hline
				\end{tabular}
			\end{center}
			\item Ta có $\dfrac{n}{2} = 15$, mà $12<15<29$ nên nhóm $6$ là nhóm đầu tiên có tần số tích luỹ lớn hơn hoặc bằng $15$. \\
			Xét nhóm $6$ là nhóm $\left[40;45\right)$ có $r=40$, $d=5$, $n_6=17$ và nhóm $5$ có $cf_5 = 12$.\\
			Khi đó, trung vị là
			\[
			M_e = 40 + \dfrac{15-12}{17} \cdot 5 = 40{,}9\text{ (kg)}.
			\]
		\end{enumerate}
	}
\end{bt}
\begin{bt}%[CTST]%[1T5G2-2]
	Cân nặng của một số lợn con mới sinh thuộc hai giống $ A $ và $ B $ được cho ở biểu đồ dưới đây (đơn vị: kg).
	\begin{center}
		\begin{tikzpicture}[>=stealth,line join=round,line cap=round,font=\footnotesize,scale=0.85,line width=1pt]
			\draw[->] (0,0)--(0,5)node[left]{(\text{Số con})};
			\foreach \y in {1,2,3,4}
			\draw[shift={(0,\y)}] (0,0)--(-2pt,0) node[left]{\scriptsize ${\y}0$};
			%	\path (4.5,6) node {\normalsize{\textbf{Cân nặng của một số lợn con mới sinh}}};
			\path (4.5,5.5) node {
				$\begin{array}{c}
					\normalsize{\textbf{Cân nặng của một số}}\\
					\normalsize{\textbf{lợn con mới sinh}}
				\end{array}$
			};
			%% nhãn
			\path (2.5,-1.5) node[rectangle,fill=cyan,draw=none]{};
			\path (3.6,-1.5) node {\text{Giống $ A $}};
			\path (5,-1.5) node[rectangle,fill=orange,draw=none]{};
			\path (6.1,-1.5) node {\text{Giống $ B $}};
			% đường gióng
			\foreach \y in {1,2,3,4}{
				\draw[line width=0.2pt] (0,\y)--(8.4,\y);
			}
			%% cột
			\draw[fill=cyan,draw=none] (0,0)--(0,0.8)--(1,0.8)node[midway,above]{$ 8 $}--(1,0)--cycle;
			\draw[fill=orange,draw=none] (1,0)--(1,1.3)--(2,1.3)node[midway,above]{$ 13 $}--(2,0)--cycle;
			\draw[fill=cyan,draw=none] (2,0)--(2,2.8)--(3,2.8)node[midway,above]{$ 28 $}--(3,0)--cycle;
			\draw[fill=orange,draw=none] (3,0)--(3,1.4)--(4,1.4)node[midway,above]{$ 14 $}--(4,0)--cycle;
			\draw[fill=cyan,draw=none] (4,0)--(4,3.2)--(5,3.2)node[midway,above]{$ 32 $}--(5,0)--cycle;
			\draw[fill=orange,draw=none] (5,0)--(5,2.4)--(6,2.4)node[midway,above]{$ 24 $}--(6,0)--cycle;
			\draw[fill=cyan,draw=none] (6,0)--(6,1.7)--(7,1.7)node[midway,above]{$ 17 $}--(7,0)--cycle;
			\draw[fill=orange,draw=none] (7,0)--(7,1.4)--(8,1.4)node[midway,above]{$ 14 $}--(8,0)--cycle;
			%% miền
			\node [below] at (1,0){$ \left[1{,}0 ; 1{,}1 \right)$};
			\node [below] at (3,0){$ \left[1{,}1 ; 1{,}2 \right)$};
			\node [below] at (5,0){$ \left[1{,}2 ; 1{,}3 \right)$};
			\node [below] at (7,0){$ \left[1{,}3 ; 1{,}4 \right)$};
			\draw[->] (0,0)node [below left=-2pt]{$ O $}--(9,0)node[below]{(\text{kg})};
		\end{tikzpicture}
	\end{center}
	Hãy so sánh cân nặng của lợn con mới sinh giống $ A $ và giống $ B $ theo số trung bình và trung vị.
	\loigiai{
		Bảng tần số ghép nhóm thống kê cân nặng của lợn con mới sinh giống $ A $ và giống $ B $ như sau:
		\begin{center}
			\begin{tabular}{|c|c|c|c|c|}
				\hline 
				\textbf{Cân nặng (kg)}	& $ \left[1{,}0 ; 1{,}1 \right)$  &$ \left[1{,}1 ; 1{,}2 \right)$  &$ \left[1{,}2 ; 1{,}3 \right)$  & $ \left[1{,}3 ; 1{,}4 \right)$ \\ 
				\hline 
				\textbf{Giá trị đại diện (kg)}	& $1{,}05 $ & $ 1{,}15 $ & $ 1{,}25 $ & $ 1{,}35 $ \\ 
				\hline 
				\begin{tabular}{c}
					\textbf{Giống A}
					\\ 
					\textbf{(đơn vị: con)}
				\end{tabular} 	& $ 8 $ & $ 28 $ & $ 32 $ & $ 17 $ \\ 
				\hline 
				\begin{tabular}{c}
					\textbf{Giống B}
					\\ 
					\textbf{(đơn vị: con)}
				\end{tabular} 	& $ 13 $ & $ 14 $ & $ 24 $ & $ 14 $ \\ 
				\hline 
			\end{tabular} 
		\end{center}
		Cân nặng trung bình của lợn con mới sinh giống $ A $ là $$ \dfrac{1{,}05\cdot 8 + 1{,}15 \cdot 28 + 1{,}25 \cdot 32 + 1{,}35 \cdot 17}{8+28+32+17}=\dfrac{2071}{1700}\approx 1{,}218.$$
		Cân nặng trung bình của lợn con mới sinh giống $ B $ là $$ \dfrac{1{,}05\cdot 13 + 1{,}15 \cdot 14 + 1{,}25 \cdot 24 + 1{,}35 \cdot 14}{13+14+24+14}=\dfrac{121}{100} \approx 1{,}21.$$
		Suy ra cân nặng trung bình của lợn con mới sinh giống $ A $  lớn hơn cân nặng trung bình của lợn con mới sinh giống $ B $. \\
		Trung vị của mẫu số liệu ghép nhóm cân nặng của lợn con  giống $ A $ là $$M_e=1{,}2+\dfrac{\dfrac{85}{2}-(8+28)}{32}\cdot (1{,}3-1{,}2)=\dfrac{781}{640}\approx 1{,}22.$$
		Trung vị của mẫu số liệu ghép nhóm cân nặng của lợn con  giống  $ B $ là $$M_e=1{,}2+\dfrac{\dfrac{65}{2}-(13+14)}{24}\cdot (1{,}3-1{,}2)=\dfrac{587}{480}\approx 1{,}22.$$
		Suy ra trung vị của mẫu số liệu ghép nhóm cân nặng của của lợn con giống $ A $ bằng trung vị của mẫu số liệu ghép nhóm cân nặng của của lợn con giống $ B $.
	}
\end{bt}

\setcounter{subsubsection}{0}
\setcounter{ex}{0}
\setcounter{bt}{0}
\begin{dang}{Tứ phân vị}
	
\end{dang}
\subsubsection{Ví dụ minh hoạ}
\begin{vd}
	\immini
	{
		Bảng bên cho biết tần số ghép nhóm số liệu thống kê cân nặng của $40$ học sinh lớp $11A$ trong một trường trung học phổ thông (đơn vị: kilôgam). Xác định tứ phân vị của mẫu số liệu ghép nhóm.
	}
	{
		\begin{tabular}{|c|c|c|}
			\hline
			\textbf{Nhóm} & \textbf{Tần số} & \textbf{Tần số tích luỹ}\\ 
			\hline
			$\left[30;40\right)$ & $2$ & $2$\\
			$\left[40;50\right)$ & $10$ & $12$\\
			$\left[50;60\right)$ & $16$ & $28$\\
			$\left[60;70\right)$ & $8$ & $36$\\
			$\left[70;80\right)$ & $2$ & $38$\\
			$\left[80;90\right)$ & $2$ & $40$\\
			\hline
			& $n = 40$ &\\
			\hline
		\end{tabular}
	}
	\loigiai{
		Số phần tử của mẫu là $n=40$.
		\begin{itemize}
			\item Ta có $\dfrac{n}{4} = 10$, mà $2<10<12$ nên nhóm $2$ là nhóm đầu tiên có tần số tích luỹ lớn hơn hoặc bằng $10$. \\
			Xét nhóm $2$ là nhóm $\left[40;50\right)$ có $s=40$, $h=10$, $n_2=10$ và nhóm $1$ có $cf_1 = 2$.\\
			Khi đó, tứ phân vị thứ nhất là
			\[
			Q_1 = 40 + \dfrac{10-2}{10} \cdot 10 = 48\text{ (kg)}.
			\]
			\item Ta có $\dfrac{n}{2} = 20$, mà $12<20<28$ nên nhóm $3$ là nhóm đầu tiên có tần số tích luỹ lớn hơn hoặc bằng $20$. \\
			Xét nhóm $3$ là nhóm $\left[50;60\right)$ có $r=50$, $d=10$, $n_3=16$ và nhóm $2$ có $cf_2 = 12$.\\
			Khi đó, tứ phân vị thứ hai là
			\[
			Q_2 = 50 + \dfrac{20-12}{16} \cdot 10 = 55\text{ (kg)}.
			\]
			\item Ta có $\dfrac{3n}{4} = 30$, mà $28<30<36$ nên nhóm $4$ là nhóm đầu tiên có tần số tích luỹ lớn hơn hoặc bằng $30$. \\
			Xét nhóm $4$ là nhóm $\left[60;70\right)$ có $t=50$, $l=10$, $n_4=8$ và nhóm $3$ có $cf_3 = 28$.\\
			Khi đó, tứ phân vị thứ ba là
			\[
			Q_3 = 60 + \dfrac{30-28}{8} \cdot 10 = 62{,}5\text{ (kg)}.
			\]
		\end{itemize}
		Vậy tứ phân vị của mẫu số liệu trên là $48$, $55$ và $62{,}5$.
	}
\end{vd}
\subsubsection{Bài tập rèn luyện}
\begin{bt}%[1K3B9-3]
	Thời gian (phút) truy cập internet mỗi buổi tối của một số học sinh được cho trong bảng sau:
	\begin{center}
		\begin{tabular}{|c|c|c|c|c|c|c|}
			\hline
			Thời gian (phút)	& $\left[9{,}5;12{,}5 \right)$ & $\left[12{,}5;15{,}5 \right)$ & $\left[15{,}5;18{,}5 \right)$ & $\left[18{,}5;21{,}5 \right)$ & $\left[21{,}5;24{,}5 \right)$ \\
			\hline
			Số học sinh&$3$	& $12$ & $15$ &$24$  & $2$  \\
			\hline
		\end{tabular}	
	\end{center}
	Tìm tứ phân vị thứ nhất $Q_1$ và tứ phân vị thứ ba $Q_3$ của mẫu số liệu ghép nhóm.
	\loigiai{
		Cỡ mẫu là $n=3+12+15+24+2=56$.\\
		Tứ phân vị thứ nhất $Q_1$ là $\dfrac{x_{14}+x_{15}}{2}$. Do $2$ giá trị $x_{28},\,x_{29}$ thuộc nhóm $\left[12{,}5;15{,}5 \right)$ nên nhóm này chứa $Q_1$. Do đó, $p=2$; $a_2=12{,}5$; $m_2=12$; $m_1=3$; $a_3-a_2=3$ và ta có $$Q_1=12{,}5+\dfrac{\dfrac{56}{4}-3}{12}\cdot 3=15{,}25.$$
		Với tứ phân vị thứ ba $Q_3$ là $\dfrac{x_{42}+x_{43}}{2}$. Do $2$ giá trị $x_{42},\,x_{43}$ thuộc nhóm $\left[18{,}5;21{,}5 \right)$ nên nhóm này chứa $Q_3$. Do đó, $p=4$; $a_4=18{,}5$; $m_4=24$; $m_1+m_2+m_3=3+12+15=30$; $a_5-a_4=3$ và ta có $$Q_3=18{,}5+\dfrac{\dfrac{3\cdot 56}{4}-30}{24}\cdot 3=20.$$
	}
\end{bt}
\begin{bt}%[1K3B9-4]
	Điểm thi môn Toán (thang điểm 100, điểm được làm tròn đến 1) của 60 thí sinh được cho trong bảng sau:
	\begin{center}
		\begin{tabular}{|c|c|c|c|c|c|}
			\hline
			Điểm		& $0-9$ & $10-19$ & $20-29$ & $30-39$ & $40-49$ \\
			\hline
			Số thí sinh	& $1$ & $2$ & $4$ & $6$ & $15$ \\
			\hline
			Điểm	& $50-59$ & $60-69$ & $70-79$ & $80-89$ & $90-99$  \\
			\hline
			Số thí sinh	& $12$ & $10$ & $6$ & $3$ & $1$  \\
			\hline
		\end{tabular}
	\end{center}
	\begin{enumerate}
		\item [a)] Hiệu chỉnh để thu được mẫu số liệu ghép nhóm dạng Bảng $3.2$.
		\item [b)] Tìm các tứ phân vị và giải thích ý nghĩa của chúng.
	\end{enumerate}
	\loigiai{
		\begin{enumerate}
			\item [a)] Bảng số liệu ghép nhóm về điểm thi môn Toán của 60 thí sinh
			\begin{center}
				\begin{tabular}{|c|c|c|c|c|c|}
					\hline
					Điểm		& $\left[0;20\right)$ & $\left[20;40\right)$ & $\left[40;60\right)$ & $\left[60;80\right)$ & $\left[80;100\right)$ \\
					\hline
					Số thí sinh	& $3$ & $10$ & $27$ & $16$ & $4$ \\
					\hline
				\end{tabular}
			\end{center}
			
			\item [b)] Cỡ mẫu $n=60$. Gọi $x_1$, $x_2$,$\ldots$, $x_{60}$ là điểm thi môn Toán của 60 học sinh và giả sử dãy này đã được sắp xếp theo thứ tự tăng dần. Khi đó, trung vị là $\dfrac{x_{30}+x_{31}}{2}$.\\
			Do hai giá trị $x_{30}$, $x_{31}$ thuộc nhóm $\left[40;60\right)$ nên nhóm này chứa trung vị. Do đó, $p=3;a_3=40;m_3=27;m_1+m_2=13;a_4-a_3=20$ và ta có
			$$Q_2=M_e=40+\dfrac{\dfrac{60}{2}-13}{27}\cdot 20\approx 52{,}6.$$
			Tứ phân vị thứ nhất $Q_1=\dfrac{x_{15}+x_{16}}{2}$. Do hai giá trị $x_{15}$, $x_{16}$ thuộc nhóm $\left[40;60\right)$ nên nhóm này chứa $Q_1$. Do đó, $p=3;\,a_3=40;\,m_3=27;\,m_1+m_2=13;\,a_4-a_3=20$ và ta có
			$$Q_1=40+\dfrac{\dfrac{60}{4}-13}{27}\cdot 20\approx 41{,}5.$$
			Tứ phân vị thứ ba $Q_3=\dfrac{x_{45}+x_{46}}{2}$. Do hai giá trị $x_{45}$, $x_{46}$ thuộc nhóm $\left[60;80\right)$ nên nhóm này chứa $Q_3$. Do đó, $p=4;\,a_4=60;\,m_4=16;\,m_1+m_2+m_3=40;\,a_5-a_4=20$ và ta có
			$$Q_3=60+\dfrac{\dfrac{3\cdot 60}{4}-40}{16}\cdot 20\approx 66{,}3.$$
			Khoảng cách từ $Q_1$ đến $Q_2$ là $11{,}1$ còn khoảng cách từ $Q_2$ và $Q_3$ là $13{,}7$. Điều này cho thấy mẫu số liệu tập trung với mật độ cao hơn ở bên trái $Q_2$ và mật độ thấp hơn ở bên phải $Q_2$.
		\end{enumerate}	
	}
\end{bt}
\begin{bt}%[1K3B9-4]
	\immini{
		Phỏng vấn một số học sinh lớp $11$ về thời gian (giờ) ngủ của một buổi tối, thu được bảng số liệu ở bên.
		\begin{enumerate}
			\item [a)] So sánh thời gian ngủ trung bình của các bạn học sinh nam và nữ.
			\item [b)] Hãy cho biết $75\%$ học sinh khối $11$ ngủ ít nhất bao nhiêu giờ?
		\end{enumerate}
	}
	{
		\begin{tabular}{|c|c|c|}
			\hline
			Thời gian	& Số học sinh nam & Số học sinh nữ\\
			\hline
			$\left[4;5 \right)$	& $6$ & $4$ \\
			\hline
			$\left[5;6 \right)$	& $10$ & $8$ \\
			\hline
			$\left[6;7 \right)$	& $13$ & $10$ \\
			\hline
			$\left[7;8 \right)$	& $9$ & $11$ \\
			\hline
			$\left[8;9 \right)$	& $7$ & $8$ \\
			\hline
		\end{tabular}
	}
	\loigiai{
		\begin{enumerate}
			\item [a)] Trong mỗi khoảng thời gian, giá trị đại diện là trung bình cộng của giá trị hai đầu mút nên ta có bảng sau:
			\begin{center}
				\begin{tabular}{|c|c|c|}
					\hline
					Thời gian	& Số học sinh nam & Số học sinh nữ\\
					\hline
					$4{,}5$	& $6$ & $4$ \\
					\hline
					$5{,}5$	& $10$ & $8$ \\
					\hline
					$6{,}5$	& $13$ & $10$ \\
					\hline
					$7{,}5$	& $9$ & $11$ \\
					\hline
					$8{,}5$	& $7$ & $8$ \\
					\hline
				\end{tabular}	
			\end{center}
			Tổng số học sinh nam là $n_1=6+10+13+9+7=45$.\\ Thời gian ngủ trung bình của học sinh nam là:
			$$\overline{x_1}=\dfrac{4{,}5\cdot 6+5{,}5\cdot10+6{,}5\cdot13+7{,}5\cdot9+8{,}5\cdot7}{45}=\dfrac{587}{90}\approx 6{,}52\,\, \text{(giờ)}.$$
			Tổng số học sinh nữ là $n_2=4+8+10+11+8=41$. Thời gian ngủ trung bình của học sinh nữ là:
			$$\overline{x_2}=\dfrac{4,5\cdot4+5,5\cdot8+6,5\cdot10+7,5\cdot11+8,5\cdot8}{41}=\dfrac{555}{82}\approx 6{,}77 \,\,\text{(giờ)}.$$
			Vì $\overline{x_2}>\overline{x_1}$ nên thời gian ngủ trung bình của các bạn học sinh nữ lớn hơn thời gian ngủ trung bình của các bạn nam.
			\item [b)] Tổng số học sinh được điều tra là $n=n_1+n_2=45+41=86$.\\
			Giả sử $x_1;x_2;x_3;\cdot \cdot;x_{86}$ là dãy giá trị được sắp xếp theo thứ tự không giảm.\\
			Ta có bảng sau:
			\begin{center}
				\begin{tabular}{|c|c|c|}
					\hline
					Thời gian	& Số học sinh \\
					\hline
					$\left[4;5 \right)$	& $10$  \\
					\hline
					$\left[5;6 \right)$	& $18$  \\
					\hline
					$\left[6;7 \right)$	& $23$  \\
					\hline
					$\left[7;8 \right)$	& $20$  \\
					\hline
					$\left[8;9 \right)$	& $15$  \\
					\hline
				\end{tabular}
			\end{center}
			Tứ phân vị thứ nhất $Q_1$ là $x_{22}$. Do $x_{22}$ thuộc nhóm $\left[5;6\right)$ nên nhóm này chứa $Q_1$.\\ Do đó, $p=2;\,a_2=5;\,m_2=18;\,m_1=10;\,a_3-a_2=1$ và ta có
			$$Q_1=5+\dfrac{\dfrac{86}{4}-10}{18}\cdot 1=\dfrac{203}{36}\approx 5{,}64 \text{(giờ)}.$$
			Nghĩa là có $25\%$ học sinh khối $11$ ngủ ít hơn $5{,}64$ giờ.\\ Vậy $75\%$ học sinh khối $11$ ngủ ít nhất $5{,}64$ giờ.
		\end{enumerate}
	}
\end{bt}
\setcounter{subsubsection}{0}
\setcounter{ex}{0}
\setcounter{bt}{0}
\begin{dang}{Mốt}
	
\end{dang}
\subsubsection{Ví dụ minh hoạ}
\begin{vd}%[Tex hóa SGK CD, TVN-223]%[1C5B1-5]
	Kết quả kiểm tra môn Toán của lớp $11D$ như sau
	\[
	\begin{array}{cccccccccccccccccccc}
	5 & 6 & 7 & 5 & 6 & 9 & 10 & 8 & 5 & 5 & 4 & 5 & 4 & 5 & 7 & 4 & 5 & 8 & 9 & 10 \\
	5 & 3 & 5 & 6 & 5 & 7 & 5 & 8 & 4 & 9 & 5 & 6 & 5 & 6 & 8 & 8 & 7 & 9 & 7 & 9
	\end{array}
	\]
	\begin{enumerate}
		\item Lập bảng tần số ghép nhóm của mẫu số liệu trên có bốn nhóm ứng với bốn nửa khoảng $\left[3;5\right)$, $\left[5;7\right)$, $\left[7;9\right)$, $\left[9;11\right)$.
		\item Mốt của bảng số liệu ghép nhóm trên là bao nhiêu (làm tròn kết quả đến hàng phần mười)?
	\end{enumerate}
	\loigiai{
		\immini
		{
			\begin{enumerate}
				\item Bảng bên là bảng tần số ghép nhóm cho kết quả kiểm tra môn Toán của lớp $11D$.
				\item Ta thấy: Nhóm $2$ ứng với nửa khoảng $\left[5;7\right)$ là nhóm có tần số lớn nhất với $u=5$, $g=2$, $n_2 = 18$. Nhóm $1$ có tần số $n_1 = 5$, nhóm $3$ có tần số $n_3=10$.\\
				Khi đó, mốt của mẫu số liệu là 
				\[
				M_o = 5 + \left( \dfrac{18- 5}{2\cdot 18 - 5 - 10} \right) \cdot 2 \approx 6{,}2.
				\]
			\end{enumerate}
		}
		{
			\begin{tabular}{|c|c|}
				\hline
				\textbf{Nhóm} & \textbf{Tần số}\\ 
				\hline
				$\left[3;5\right)$ & $5$\\
				\hline
				$\left[5;7\right)$ & $18$\\
				\hline
				$\left[7;9\right)$ & $10$\\
				\hline
				$\left[9;11\right)$ & $7$\\
				\hline
				& $n = 40$ \\
				\hline
			\end{tabular}
		}
	}
\end{vd}
\subsubsection{Bài tập rèn luyện}
\begin{bt}%[Tex hóa SGK CD, TVN-223]%[1C5B1-5]
	Mẫu số liệu dưới đây ghi lại tốc độ của $40$ ô tô khi đi qua một trạm đo tốc độ (đơn vị: km/h):
	\[
	\begin{array}{cccccccccc}
	48{,}5 & 43 & 50 & 55 & 45 & 60 & 53 & 55,5 & 44 & 65 \\
	51 & 62,5 & 41 & 44,5 & 57 & 57 & 68 & 49 & 46{,}5 & 53{,}5 \\
	61 & 49{,}5 & 54 & 62 & 59 & 56 & 47 & 50 & 60 & 61 \\
	49{,}5 & 52{,}5 & 57 & 47 & 60 & 55 & 45 & 47,5 & 48 & 61{,}5
	\end{array}
	\]
	\begin{enumerate}
		\item Lập bảng tần số ghép nhóm cho mẫu số liệu trên có sáu nhóm ứng với sáu nửa khoảng:
		\[
		[40 ; 45),[45 ; 50),[50 ; 55),[55 ; 60),[60 ; 65),[65 ; 70).
		\]
		\item Mốt của mẫu số liệu ghép nhóm trên là bao nhiêu?
	\end{enumerate}
	\loigiai{
		\begin{enumerate}
			\item Ta có bảng tần số ghép nhóm của mẫu số liệu trên như sau:
			\begin{center}
				\begin{tabular}{|c|c|c|c|}
					\hline
					\textbf{Nhóm} & \textbf{Giá trị đại diện} & \textbf{Tần số} & \textbf{Tần số tích luỹ}\\ 
					\hline
					$\left[40;45\right)$ & $42{,}5$ & $4$ & $4$\\
					$\left[45;50\right)$ & $47{,}5$ & $11$ & $15$\\
					$\left[50;55\right)$ & $52{,}5$ & $7$ & $22$\\
					$\left[55;60\right)$ & $57{,}5$ & $8$ & $30$\\
					$\left[60;65\right)$ & $62{,}5$ & $8$ & $38$\\
					$\left[65;70\right)$ & $67{,}5$ & $2$ & $40$\\
					\hline
					&  & $n = 40$ &\\
					\hline
				\end{tabular}
			\end{center}
			
			\item Ta thấy: Nhóm $2$ ứng với nửa khoảng $\left[45;50\right)$ là nhóm có tần số lớn nhất với $u=45$, $g=5$, $n_2 = 11$. Nhóm $1$ có tần số $n_1 = 4$, nhóm $3$ có tần số $n_3 = 7$.\\
			Khi đó, mốt của mẫu số liệu là 
			\[
			M_o = 45 + \left( \dfrac{11 - 4}{2\cdot 11 - 4 - 7} \right) \cdot 5 \approx 48{,}2\text{ (km/h)}.
			\]
		\end{enumerate}
	}
\end{bt}
\begin{bt}%[Tex hóa SGK CD, TVN-223]%[1C5B1-5]
	Mẫu số liệu sau ghi lại cân nặng của $30$ bạn học sinh (đơn vị: kilôgam):
	\[
	\begin{array}{cccccccccc}
	17 & 40 & 39 & 40{,}5 & 42 & 51 & 41{,}5 & 39 & 41 & 30\\
	40 & 42 & 40{,}5 & 39{,}5 & 41 & 40{,}5 & 37 & 39{,}5 & 40 & 41\\
	38{,}5 & 39{,}5 & 40 & 41 & 39 & 40{,}5 & 40 & 38{,}5 & 39{,}5 & 41{,}5
	\end{array}
	\]
	\begin{enumerate}
		\item Lập bảng tần số ghép nhóm cho mẫu số liệu trên có tám nhóm ứng với tám nửa khoảng:
		\[
		[15 ; 20),[20 ; 25),[25 ; 30),[30 ; 35),[35 ; 40),[40 ; 45),[45 ; 50),[50 ; 55).
		\]
		
		\item Mốt của mẫu số liệu ghép nhóm trên là bao nhiêu?
	\end{enumerate}
	\loigiai{
		\begin{enumerate}
			\item Ta có bảng tần số ghép nhóm của mẫu số liệu trên như sau:
			\begin{center}
				\begin{tabular}{|c|c|c|c|}
					\hline
					\textbf{Nhóm} & \textbf{Giá trị đại diện} & \textbf{Tần số} & \textbf{Tần số tích luỹ}\\ 
					\hline
					$\left[15;20\right)$ & $17{,}5$ & $1$ & $1$\\
					$\left[20;25\right)$ & $22{,}5$ & $0$ & $1$\\
					$\left[25;30\right)$ & $27{,}5$ & $0$ & $1$\\
					$\left[30;35\right)$ & $32{,}5$ & $1$ & $2$\\
					$\left[35;40\right)$ & $37{,}5$ & $10$ & $12$\\
					$\left[40;45\right)$ & $42{,}5$ & $17$ & $29$\\
					$\left[45;50\right)$ & $47{,}5$ & $0$ & $29$\\
					$\left[50;55\right)$ & $52{,}5$ & $1$ & $30$\\
					\hline
					&  & $n = 30$ &\\
					\hline
				\end{tabular}
			\end{center}
		
			\item Ta thấy: Nhóm $6$ ứng với nửa khoảng $\left[40;45\right)$ là nhóm có tần số lớn nhất với $u=40$, $g=5$, $n_6 = 17$. Nhóm $5$ có tần số $n_5 = 10$, nhóm $7$ có tần số $n_7 = 0$.\\
			Khi đó, mốt của mẫu số liệu là 
			\[
			M_o = 40 + \left( \dfrac{17 - 10}{2\cdot 17 - 10 - 0} \right) \cdot 5 \approx 41{,}5\text{ (kg)}.
			\]
		\end{enumerate}
	}
\end{bt}
\begin{bt}%[Tex hóa SGK CD, TVN-223]%[1C5B1-5]
	\immini
	{
		Bảng bên cho ta bảng tần số ghép nhóm số liệu thống kê chiều cao của $40$ mẫu cây ở một vườn thực vật (đơn vị: centimét).
		
		
			 Mốt của mẫu số liệu ghép nhóm trên là bao nhiêu?
		
	}
	{
		\begin{tabular}{|c|c|c|}
			\hline
			\textbf{Nhóm} & \textbf{Tần số} & \textbf{Tần số tích luỹ}\\ 
			\hline
			$\left[30;40\right)$ & $4$ & $4$\\
			$\left[40;50\right)$ & $10$ & $14$\\
			$\left[50;60\right)$ & $14$ & $28$\\
			$\left[60;70\right)$ & $6$ & $34$\\
			$\left[70;80\right)$ & $4$ & $38$\\
			$\left[80;90\right)$ & $2$ & $40$\\
			\hline
			& $n = 40$ &\\
			\hline
		\end{tabular}
	}
	\loigiai{
	Ta thấy: Nhóm $3$ ứng với nửa khoảng $\left[50;60\right)$ là nhóm có tần số lớn nhất với $u=50$, $g=10$, $n_3 = 14$. Nhóm $2$ có tần số $n_2 = 10$, nhóm $4$ có tần số $n_4 = 6$.\\
			Khi đó, mốt của mẫu số liệu là 
			\[
			M_o = 50 + \left( \dfrac{14 - 10}{2\cdot 14 - 10 - 6} \right) \cdot 10 \approx 53{,}3\text{ (cm)}.
			\]
	
	}
\end{bt}
\subsection{Bài tập trắc nghiệm}
\Opensolutionfile{ans}[ans/ansOC3]
% \begin{ex}%[1C5Y1-1]
% 	Một cuộc khảo sát đã tiến hành xác định tuổi (theo năm) của $120$ chiếc ô-tô. Kết quả điều tra được cho trong bảng sau
% 	\begin{center}
% 		\begin{tabular}{ |c|c|c|c|c|c|c| }
% 			\hline
% 			Nhóm & $[0;4)$ & $[4;8)$ & $[8;12)$ & $[12;16)$ & $[16;20)$ &  \\
% 			\hline
% 			Tần số & $23$ & $25$ & $27$ & $26$ & $19$ & $n=120$ \\
% 			\hline
% 		\end{tabular}
% 	\end{center}
% 	Mẫu số liệu trên có bao nhiêu nhóm?
% 	\choice
% 	{$10$}
% 	{$11$}
% 	{\True $5$}
% 	{$7$}
% 	\loigiai{
% 		Từ bảng, ta thấy mẫu số liệu trên có $5$ nhóm.
% 	}
% \end{ex}
% \begin{ex}%[1C5Y1-1]
% 	Một cuộc khảo sát đã tiến hành xác định tuổi (theo năm) của $120$ chiếc ô-tô. Kết quả điều tra được cho trong bảng sau
% 	\begin{center}
% 		\begin{tabular}{ |c|c|c|c|c|c|c| }
% 			\hline
% 			Nhóm & $[0;4)$ & $[4;8)$ & $[8;12)$ & $[12;16)$ & $[16;20)$ &  \\
% 			\hline
% 			Tần số & $23$ & $25$ & $27$ & $26$ & $19$ & $n=120$ \\
% 			\hline
% 		\end{tabular}
% 	\end{center}
% 	Nhóm có tần số bằng $19$ là
% 	\choice
% 	{$[0;4)$}
% 	{$[8;12)$}
% 	{$[12;16)$}
% 	{\True $[16;20)$}
% 	\loigiai{
% 		Từ bảng, ta thấy nhóm có tần số bằng $19$ là $[16;20)$.
% 	}
% \end{ex}
\begin{ex}%[1C5Y1-1]
	Một cuộc khảo sát đã tiến hành xác định tuổi (theo năm) của $120$ chiếc ô-tô. Kết quả điều tra được cho trong bảng sau
	\begin{center}
		\begin{tabular}{ |c|c|c|c|c|c|c| }
			\hline
			Nhóm & $[0;4)$ & $[4;8)$ & $[8;12)$ & $[12;16)$ & $[16;20)$ &  \\
			\hline
			Tần số & $23$ & $25$ & $27$ & $26$ & $19$ & $n=120$ \\
			\hline
		\end{tabular}
	\end{center}
	Số ô-tô có độ tuổi dưới $12$ là
	\choice
	{\True $75$}
	{$27$}
	{$48$}
	{$26$}
	\loigiai{
		Từ bảng, ta thấy số ô-tô có độ tuổi dưới $12$ là $23+25+27=75$.
	}
\end{ex}
% \begin{ex}%[1C5Y1-1]
% 	Cho mẫu số liệu ghép nhóm sau
% 	\begin{center}
% 		\begin{tabular}{ |c|c|c|c|c|c|c|c| }
% 			\hline
% 			Thời gian & $[15;20)$ & $[20;25)$ & $[25;30)$ & $[30;35)$ & $[35;40)$ & $[40;45)$ & $[45;50)$ \\
% 			\hline
% 			Số nhân viên & $6$ & $14$ & $25$ & $37$ & $21$ & $13$ & $9$ \\
% 			\hline
% 		\end{tabular}
% 	\end{center}
% 	Tần số của nhóm $[15;20)$ là bao nhiêu?
% 	\choice
% 	{\True $6$}
% 	{$7$}
% 	{$14$}
% 	{$25$}
% 	\loigiai{
% 		Ta thấy tần số của nhóm $[15;20)$ là $6$.
% 	}
% \end{ex}
\begin{ex}%[1C5Y1-1]
	Khảo sát thời gian tập thể dục trong ngày của một số học sinh khối $11$ thu được mẫu số liệu ghép nhóm sau
	\begin{center}
		\begin{tabular}{ |c|c|c|c|c|c| }
			\hline
			Thời gian (phút)& $[0;20)$ & $[20;40)$ & $[40;60)$ & $[60;80)$ & $[80;100)$ \\
			\hline
			Số học sinh & $5$ & $9$ & $12$ & $10$ & $6$ \\
			\hline
		\end{tabular}
	\end{center}
	Giá trị đại diện của nhóm $[20;40)$ là
	\choice
	{$10$}
	{\True $30$}
	{$20$}
	{$40$}
	\loigiai{
		Giá trị đại diện của nhóm $[20;40)$ là $\dfrac{20+40}{2}=30$.
	}
\end{ex}
\begin{ex}%[1C5Y1-1]
	Doanh thu bán hàng trong $20$ ngày được lựa chọn ngẫu nhiên của một cửa hàng được ghi lại ở bảng sau (đơn vị: triệu đồng):
	\begin{center}
		\begin{tabular}{ |c|c|c|c|c|c| }
			\hline
			Doanh thu & $[5;7)$ & $[7;9)$ & $[9;11)$ & $[11;13)$ & $[13;15)$ \\
			\hline
			Số ngày & $2$ & $7$ & $7$ & $3$ & $1$ \\
			\hline
		\end{tabular}
	\end{center}
	Doanh thu bán hàng của cửa hàng trong ngày $A$ là $7$ triệu đồng thì được xếp vào nhóm nào?
	\choice
	{$[5;7)$}
	{\True $[7;9)$}
	{$[9;11)$}
	{$[13;15)$}
	\loigiai{
		Doanh thu bán hàng là $7$ triệu đồng thì được xếp vào nhóm $[7;9)$.
	}
\end{ex}
\begin{ex}%[1C5Y1-1]
	Doanh thu bán hàng trong $20$ ngày được lựa chọn ngẫu nhiên của một cửa hàng được ghi lại ở bảng sau (đơn vị: triệu đồng):
	\begin{center}
		\begin{tabular}{ |c|c|c|c|c|c| }
			\hline
			Doanh thu & $[5;7)$ & $[7;9)$ & $[9;11)$ & $[11;13)$ & $[13;15)$ \\
			\hline
			Số ngày & $2$ & $7$ & $7$ & $3$ & $1$ \\
			\hline
		\end{tabular}
	\end{center}
	Các nhóm có độ dài bằng
	\choice
	{\True $2$}
	{$3$}
	{$4$}
	{$5$}
	\loigiai{
		Các nhóm có độ dài bằng nhau, và bằng $2$.
	}
\end{ex}
\begin{ex}%[1C5B1-1]
	Cho bảng số liệu về khối lượng của $30$ củ khoai tây thu hoạch từ một thửa ruộng như hình bên dưới. Tần suất của lớp $[100;110)$ là bao nhiêu?
	\begin{center}
		\begin{tabular}{ |c|c|c|c|c|c| }
			\hline
			Lớp khối lượng (gam) & $[70;80)$ & $[80;90)$ & $[90;100)$ & $[100;110)$ & $[110;120]$ \\
			\hline
			Tần số & $3$ & $6$ & $12$ & $6$ & $3$ \\
			\hline
		\end{tabular}
	\end{center}
	\choice
	{\True $20\%$}
	{$10\%$}
	{$40\%$}
	{$90\%$}
	\loigiai{
		Tần suất ghép lớp $[100;110)$ là $\dfrac{6}{30}\cdot 100\%=20\%$.
	}
\end{ex}
\begin{ex}%[1C5B1-1]
	Cân nặng của $28$ học sinh nam lớp $11$ được cho ở bảng sau
	\begin{center}
		\begin{tabular}{ |c|c|c|c|c|c|c| }
			\hline
			Cân nặng & $[45;49)$ & $[49;53)$ & $[53;57)$ & $[57;61)$ & $[61;65)$\\
			\hline
			Số học sinh & $4$ & $5$ & $7$ & $7$ & $5$ \\
			\hline
		\end{tabular}
	\end{center}
	Tấn số tích lũy của nhóm $[49;53)$ là bao nhiêu?
	\choice
	{$5$}
	{$4$}
	{\True $9$}
	{$20$}
	\loigiai{
		Tần số tích lũy của nhóm $[49;53)$ là $4+5=9$.
	}
\end{ex}
% \begin{ex}%[1C5K1-1]
% 	Một thư viện thống kê sô người đến đọc sách vào buổi tối trong $30$ ngày của tháng vừa qua như sau
% 	\begin{center}
% 		\begin{tabular}{ cccccccccc }
% 			$26$ & $35$ & $68$ & $84$ & $33$ & $84$ & $62$ & $45$ & $57$ & $46$ \\
% 			$35$ & $29$ & $28$ & $50$ & $26$ & $34$ & $75$ & $74$ & $43$ & $49$ \\
% 			$54$ & $55$ & $83$ & $82$ & $81$ & $54$ & $27$ & $36$ & $41$ & $52$ \\
% 		\end{tabular}
% 	\end{center}
% 	Bạn An lập bảng tần số mẫu số liệu trên như sau
% 	\begin{center}
% 		\begin{tabular}{ |c|c|c|c|c|c|c|c| }
% 			\hline
% 			Nhóm & $[25;35)$ & $[35;45)$ & $[45;55)$ & $[55;65)$ & $[65;75]$ & $[75;85)$ & \\
% 			\hline
% 			Tần số & $7$ & $4$ & $7$ & $3$ & $3$ & $6$ & $n=30$ \\
% 			\hline
% 		\end{tabular}
% 	\end{center}
% 	Bạn Khuê lập bảng tần số mẫu số liệu trên như sau
% 	\begin{center}
% 		\begin{tabular}{ |c|c|c|c|c|c|c|c|c| }
% 			\hline
% 			Nhóm & $[23;31)$ & $[31;39)$ & $[39;47)$ & $[47;55)$ & $[55;63]$ & $[71;79)$ & $[79;87)$ & \\
% 			\hline
% 			Tần số & $5$ & $5$ & $4$ & $5$ & $3$ & $1$ & $2$ & $n=30$ \\
% 			\hline
% 		\end{tabular}
% 	\end{center}
% 	Hỏi bảng tần số của bạn nào đúng?
% 	\choice
% 	{Bảng tần số của bạn An}
% 	{\True Bảng tần số của bạn Khuê}
% 	{Cả hai bạn đều đúng}
% 	{Cả hai bạn đều sai}
% 	\loigiai{
% 		Từ các bảng tần số mẫu số liệu trên, ta thấy bảng tần số của bạn An sai ở hai nhóm $[35;45)$ (tần số đúng bằng $5$) và nhóm $[65;75)$ (tần số đúng bằng $2$).
% 	}
% \end{ex}
% \begin{ex}%[0D5Y3-1]
% 	Cho dãy số liệu thống kê: $21$, $23$, $24$, $25$, $22$, $20$. Số trung bình cộng của các số liệu thống kê đã cho là
% 	\choice
% 	{$23{,}5$}
% 	{$22$}
% 	{\True $22{,}5$}
% 	{$14$}
% 	\loigiai{
% 		Ta có $\overline{x}=\dfrac{21+23+24+25+22+20}{6}=22{,}5$.
% 	}
% \end{ex}
% \begin{ex}%[0D5Y3-1]
% 	Điều tra về số con của $30$ gia đình ở khu vực, kết quả thu được như sau
% 	\begin{center}
% 		\begin{tabular}{|c|c|c|c|c|c|c|}
% 			\hline 
% 			Giá trị (số con) & $0$ &$1$ & $2$ &$3$ & $4$ & Tổng \\ 
% 			\hline 
% 			Tần số & $1$ & $7$ & $15$ & $5$ & $2$ & $N=30$ \\ 
% 			\hline 
% 		\end{tabular}
% 	\end{center}
% 	Tìm số trung bình $\overline{x}$ của mẫu số liệu trên.
% 	\choice
% 	{\True $\overline{x}=2$}
% 	{$\overline{x}=1$}
% 	{$\overline{x}=1{,}5$}
% 	{$\overline{x}=3$}
% 	\loigiai{
% 		Ta có $\overline{x}=\dfrac{0\cdot 1+1\cdot 7+2\cdot 15+3\cdot 5+4\cdot 2}{30}=2$.
% 	}
% \end{ex}
% \begin{ex}%[0D5Y3-1]
% 	Điểm môn Toán của lớp $11A$ được cho trong bảng sau
% 	\begin{center}
% 		\begin{tabular}{|l|c|c|c|c|c|c|c|c|c|c|}
% 			\hline
% 			Điểm&$1$&$2$&$3$&$4$&$5$&$6$&$7$&$8$&$9$&$10$\\
% 			\hline
% 			Tần số&$2$&$1$&$4$&$3$&$9$&$7$&$5$&$5$&$3$&$1$\\
% 			\hline
% 		\end{tabular}
% 	\end{center}
% 	Điểm trung bình của các học sinh lớp $10A$ là bao nhiêu?
% 	\choice
% 	{$5$}
% 	{$5{,}5$}
% 	{$5{,}6$}
% 	{\True $5{,}7$}
% 	\loigiai{
% 		Điểm trung bình lớp $11A$ là
% 		$$\overline{x}= \dfrac{1 \cdot 2 + 2 \cdot 1 + \cdots + 10 \cdot 1}{40}= \dfrac{227}{40} \approx 5{,}7.$$
% 	}
% \end{ex}
% \begin{ex}%[0D5Y3-1]
% 	Kết quả điểm kiểm tra môn Toán của $40$ học sinh lớp $10A$ được trình bày ở bảng sau:
% 	\begin{center}
% 		\begin{tabular}{|c|c|c|c|c|c|c|c|c|}
% 			\hline 
% 			Điểm&$4$  &$5$  &$6$  &$7$  &$8$  &$9$  &$10$  &Cộng  \\ 
% 			\hline 
% 			Tần số&$2$  &$8$  &$7$  &$10$  &$8$  &$3$  &$2$  &$40$  \\ 
% 			\hline 
% 		\end{tabular} 
% 	\end{center}
% 	Tính số trung bình cộng của bảng trên. (làm tròn kết quả đến một chữ số thập phân).
% 	\choice
% 	{\True $6{,}8$}
% 	{$6{,}4$}
% 	{$7{,}0$}
% 	{$6{,}7$}
% 	\loigiai{
% 		Ta có $\overline{x}=\dfrac{4\cdot 2+ 5\cdot 8+ 6\cdot 7+ 7\cdot 10+ 8\cdot 8+ 9\cdot 3+ 10\cdot 2}{40}\approx 6{,}8$.
% 	}
% \end{ex}
\begin{ex}%[0D5B3-1]
	Điểm môn Toán của lớp $10$A được cho như bảng sau
	\begin{center}
		\begin{tabular}{|c|c|c|c|c|c|}
			\hline
			Điểm &$[0;2)$& $[2;4)$& $[4;6)$& $[6;8)$& $[8;10)$\\\hline
			Tần số& $3$& $5$& $12$& $12$& 8\\ \hline
		\end{tabular}
	\end{center}
	Điểm trung bình của các học sinh lớp $10$A là bao nhiêu?
	\choice
	{$5$}
	{\True $5{,}85$}
	{$5{,}65$}
	{$5{,}45$}	
	\loigiai{
		Điểm trung bình $\overline{x}=\dfrac{1\cdot 3+3\cdot 5+5\cdot 12+7\cdot 12+9\cdot 8}{40}=5{,}85$.		
	}
\end{ex}
\begin{ex}%[0D5B3-1]
	Cho bảng phân bố tần số ghép lớp
	\begin{center}
		\begin{tabular}{|l|c|c|c|c|}
			\hline
			{Lớp các giá trị $x$}&{[8; 10)}&{[10; 12)}&{[12; 14]}&{Cộng}\\
			\hline
			{Tần số $n_i$}&{15}&{30}&{55}&{100}\\
			\hline
		\end{tabular}	
	\end{center}
	Số trung bình của các giá trị trong bảng trên là
	\choice
	{$9$}
	{$13$}
	{$11$}
	{\True $11{,}8$}	
	\loigiai{
		Giá trị đại diện của lớp $\left[8; 10\right)$: $c_1=\dfrac{8 + 10}{2}=9$.\\ 
		Giá trị đại diện của lớp $\left[10; 12\right)$: $c_2=\dfrac{10 + 12}{2}=11$.\\ 
		Giá trị đại diện của lớp $\left[12; 14\right)$: $c_3=\dfrac{12 + 14}{2}=13$.\\ 
		Vậy số trung bình cộng $\overline{x}=\dfrac{9\cdot 15 + 11\cdot 30 + 13\cdot 55}{15 + 30 + 55}=\dfrac{59}{5}$.
	}
\end{ex}
\begin{ex}%[0D5B3-1]
	Kết quả khảo sát cân nặng của $25$ quả cam ở lô hàng $A$ được cho như sau
	\begin{center}
		\begin{tabular}{|l|c|c|c|c|c|}
			\hline
			Cân nặng (g) & $[150;155)$ & $[155;160)$ & $[160;165)$ & $[165;170)$ & $[170;175)$ \\
			\hline
			Số quả cam & $2$ & $6$ & $12$ & $4$ & $1$ \\
			\hline
		\end{tabular}	
	\end{center}
	Tính cân nặng trung bình của mỗi quả cam ở lô hàng $A$.
	\choice
	{\True $161{,}7$ (g)}
	{$161{,}7$ (kg)}
	{$155$ (g)}
	{$160$ (kg)}
	\loigiai{
		Ta có giá trị đại diện của các nhóm lần lượt là $152{,}5$; $157{,}5$; $162{,}5$; $167{,}5$; $172{,}5$.\\
		Vậy cân nặng trung bình của mỗi quả cam là
		$$\overline{x}=\dfrac{152{,}5\cdot 2+ 157{,}5\cdot 6+162{,}5\cdot 12+167{,}5\cdot 4+172{,}5\cdot 1}{25}=161{,}7 \text{(g).}$$
	}
\end{ex}
% \begin{ex}%[0D5K3-1]
% 	Ba nhóm học sinh gồm $10$ người, $15$ người, $25$ người. Khối lượng trung bình của mỗi nhóm lần lượt là $50$ kg, $38$ kg, $40$ kg. Khối lượng trung bình của cả ba nhóm học sinh là
% 	\choice
% 	{\True $41{,}4$ kg}
% 	{$42{,}4$ kg}
% 	{$26$ kg}
% 	{$37$ kg}
% 	\loigiai{
% 		Tổng khối lượng nhóm thứ nhất là $50\cdot 10=500$ (kg).\\
% 		Tổng khối lượng nhóm thứ hai là $38\cdot 15=570$ (kg).\\
% 		Tổng khối lượng nhóm thứ ba là $40\cdot 25=1000$ (kg).\\
% 		Tổng khối lượng cả ba nhóm là $500+570+1000=2070$ (kg).\\
% 		Tổng số người cả ba nhóm là $10+15+25=50$ (người).\\
% 		Khối lượng trung bình của cả ba nhóm học sinh là $\dfrac{2070}{50}=41{,}4$ (kg).
% 	}
% \end{ex}
\begin{ex}%[0D5K3-1]
	Sau một kì thi học sinh giỏi Toán, người ta thống kê kết quả (thang điểm $20$) và thu được bảng tần số sau
	\begin{center}
		\begin{tabular}{|l|c|c|c|c|}
			\hline
			Lớp điểm & $[6;10]$ & $[11;15]$ & $[16;20]$ & Cộng\\
			\hline
			Tần số & $22$ & $12$ & $6$ & $40$ \\
			\hline
		\end{tabular}
	\end{center}
	Nếu những học sinh chỉ cần đạt điểm trung bình của bảng điểm trên đều được nhận Giấy Khen của ban tổ chức, thì số học sinh được nhận Giấy Khen là bao nhiêu?
	\choice
	{$11$}
	{\True $18$}
	{$12$}
	{$6$}
	\loigiai{
		Ta lập lại bảng với thêm dòng giá trị đại diện
		\begin{center}
			\begin{tabular}{|l|c|c|c|c|}
				\hline
				Lớp điểm & $[6;10]$ & $[11;15]$ & $[16;20]$ & Cộng\\
				\hline
				Giá trị đại diện & $8$ & $13$ & $18$ & \\
				\hline
				Tần số & $22$ & $12$ & $6$ & $40$ \\
				\hline
			\end{tabular}
		\end{center}
		Điểm trung bình là $\overline{x}=\dfrac{22\cdot 8+12\cdot 13+6\cdot 18}{40}=11$.\\
		Vậy số học sinh được nhận thưởng là $12+6=18$ (học sinh).
	}
\end{ex}
% \begin{ex}%[0D5G3-1]
% 	Cho biết tình hình thu hoạch lúa vụ mùa năm $1980$ của ba hợp tác xã ở địa phương $V$ như sau
% 	\begin{center}
% 		\begin{tabular}{|l|c|c|c|}
% 			\hline
% 			Hợp tác xã & $A$ & $B$ & $C$ \\
% 			\hline
% 			Năng suất lúa (tạ/ha) & $40$ & $38$ & $36$ \\
% 			\hline
% 			Diện tích trồng lúa (ha) & $150$ & $130$ & $120$ \\
% 			\hline
% 		\end{tabular}	
% 	\end{center}
% 	Hãy tính năng suất lúa trung bình của vụ mùa năm $1980$ trong toàn bộ ba hợp tác xã kể trên.
% 	\choice
% 	{\True $38{,}15$ tạ/ha}
% 	{$38{,}05$ tạ/ha}
% 	{$38{,}10$ tạ/ha}
% 	{$38{,}20$ tạ/ha}
% 	\loigiai{
% 		Sản lượng lúa của hợp tác xã $A$ là $40\cdot 150=6000$ (tạ).\\
% 		Sản lượng lúa của hợp tác xã $B$ là $38\cdot 130=4940$ (tạ).\\
% 		Sản lượng lúa của hợp tác xã $C$ là $36\cdot 120=4320$ (tạ).\\
% 		Tổng sản lượng lúa của cả ba hợp tác xã là $6000+4940+4320=15260$ (tạ).\\
% 		Tổng diện tích trồng lúa của cả ba hợp tác xã là $150+130+120=400$ (ha).\\
% 		Vậy năng suất lúa trung bình của cả ba hợp tác xã là $\dfrac{15260}{400}=38{,}15$ tạ/ha.\\
% 		\textbf{\underline{Lưu ý}:} Không thể tính năng suất trung bình bằng cách $\dfrac{40+38+36}{3}=38$ (tạ/ha), vì khi chênh lệch diện tích càng lớn thì số trung bình càng không chính xác.
% 	}
% \end{ex}
% \begin{ex}%[0D5Y3-2]
% 	Tiền lương hàng tháng của $7$ nhân viên trong một công ty du lịch là $650$, $840$, $690$, $2500$, $720$, $670$, $3000$ (đơn vị: nghìn đồng). Tìm số trung vị của các số liệu thống kê đã cho.
% 	\choice
% 	{$690$}
% 	{$2500$}
% 	{\True $720$}
% 	{$670$}
% 	\loigiai{ 
% 		Sắp xếp thứ tự các số liệu thống kê, ta thu được dãy tăng các số liệu như sau: $650$, $670$, $690$, $720$, $840$, $2500$, $3000$ (nghìn đồng).\\
% 		Ta có số các số liệu thống kê là $n=7=2\cdot 3+1$ nên số trung vị là $M_e=x_4=720$.
% 	}
% \end{ex}
% \begin{ex}%[0D5Y3-2]
% 	Điểm học kì một của một học sinh được cho bởi bảng số liệu sau (Đơn vị: điểm)
% 	\begin{center}
% 		\begin{tabular}{|c|c|c|c|c|c|c|c|c|}
% 			\hline
% 			5& 6 &6&7& 7 &8 &8& 8,5&9\\
% 			\hline
% 		\end{tabular}
% 	\end{center}
% 	Số trung vị của bảng nói trên là
% 	\choice
% 	{$6$}
% 	{$9$}
% 	{\True $7$}
% 	{$8$}
% 	\loigiai{ Ta có $N=9$ là số lẻ. Số liệu thứ $\dfrac{N+1}{2} = 5$ là số trung vị.\\
% 		Do đó số trung vị là $M_e = 7$ (điểm).
% 	}
% \end{ex}
% \begin{ex}%[0D5Y3-2]
% 	Điều tra số học sinh giỏi khối $10$ của $15$ trường cấp ba trên địa bản tỉnh $A$, ta được bảng số liệu như sau
% 	\begin{center}
% 		\begin{tabular}{|c|c|c|c|c|c|c|c|c|c|c|c|c|c|c|}
% 			\hline
% 			22& 29 &29&29& 30 &31 &32& 32&33 &34 &34 &35 &35 &35 &36  \\
% 			\hline	
% 		\end{tabular}
% 	\end{center}
% 	Số trung vị của bảng nói trên là
% 	\choice
% 	{\True $8$}
% 	{$9$}
% 	{$6$}
% 	{$7$}
% 	\loigiai{ Ta có $N=15$ là số lẻ $\Rightarrow$ số liệu thứ $\dfrac{15+1}{2}=8$ là số trung vị.\\
% 		Vậy số trung vị là $M_e = 8$.
% 	}
% \end{ex}
% \begin{ex}%[0D5Y3-2]
% 	Cho mẫu số liệu thống kê $\{6;4;4;1;9;10;7\}$. Số liệu trung vị của mẫu số liệu thống kê trên là
% 	\choice
% 	{$1$}
% 	{\True $6$}
% 	{$4$}
% 	{$10$}
% 	\loigiai{
% 		Sắp xếp thành dãy không giảm: $1$, $4$, $4$, $6$, $7$, $9$, $10$.\\
% 		Từ dãy trên ta có số trung vị là số $6$ trong dãy trên.
% 	}
% \end{ex}
% \begin{ex}%[0D5B3-2]
% 	Điểm kiểm tra môn Toán của $10$ học sinh được cho như sau: $6$; $7$; $7$; $6$; $7$; $8$; $8$; $7$; $9$; $9$. Số trung vị của mẫu số liệu trên là	
% 	\choice
% 	{$6$}
% 	{\True $7$}
% 	{$8$}
% 	{$9$}
% 	\loigiai{
% 		Ta sắp xếp số liệu theo thứ tự không giảm như sau: $6$; $6$; $7$; $7$; $7$; $7$; $8$; $8$; $9$; $9$.\\
% 		Dãy số trên có tất cả $10$ giá trị, và $2$ giá trị chính giữa bằng $7$.\\
% 		Vậy số trung vị của mẫu số liệu trên là $\dfrac{7+7}{2}=7$.
% 	}
% \end{ex}
% \begin{ex}%[0D5B3-2]
% 	Một cửa hàng dép da đã thống kê cỡ dép của một số khách hàng nam cho kết quả như sau: $39$; $38$; $39$; $40$; $41$; $41$; $43$; $37$; $38$; $40$; $43$; $41$; $42$; $41$; $42$. Tìm trung vị của mẫu số liệu trên.
% 	\choice
% 	{$37$}
% 	{$39$}
% 	{\True $41$}
% 	{$43$}
% 	\loigiai{
% 		Ta sắp xếp số liệu theo thứ tự không giảm: $37$; $38$; $38$; $39$; $39$; $40$; $40$; $41$; $41$; $41$; $41$; $42$; $42$; $42$; $43$.\\
% 		Vì $n=15$ là số lẻ nên số trung vị là số chính giữa của dãy số liệu.\\
% 		Vậy trung vị là $M_e=41$.
% 	}
% \end{ex}
% \begin{ex}%[0D5B3-2]
% 	Điều tra số học sinh của $30$ lớp học, ta được bảng số liệu như sau
% 	\begin{center}
% 		\begin{tabular}{|c|c|c|c|c|c|c|c|c|c|c|c|c|c|c|}
% 			\hline
% 			35& 39 &39&40& 40 &41 &41& 41&41 &44 &44 &45 &45 &45 &46  \\
% 			\hline
% 			48 &48 &48&48& 49 &49 &49&49 &49 &49 &50 &50 &50 &50 &51  \\
% 			\hline	
% 		\end{tabular}
% 	\end{center}
% 	Số trung vị của bảng nói trên là
% 	\choice
% 	{$46$}
% 	{$49$}
% 	{\True $47$}
% 	{$48$}
% 	\loigiai{ Ta có $N=30$ là số chẵn. Số liệu thứ $15$ và $16$ lần lượt là $46$, $48$ là số trung vị.\\
% 		Vậy số trung vị là $M_e = \dfrac{46+48}{2}=47$ (học sinh).
% 	}
% \end{ex}
% \begin{ex}%[0D5B3-2]
% 	Cho bảng phân bố tần số
% 	\begin{center}
% 		\begin{tabular}{|l|c|c|c|c|c|c|}
% 			\hline
% 			Tuổi & $18$ & $19$ & $20$ & $21$ & $22$ & Cộng \\
% 			\hline
% 			Tần số & $10$ & $50$ & $70$ & $29$ & $10$ & $169$ \\
% 			\hline	
% 		\end{tabular}
% 	\end{center}
% 	Số trung vị của bảng phân bố tần số đã cho là
% 	\choice
% 	{$18$ tuổi}
% 	{\True $20$ tuổi}
% 	{$19$ tuổi}
% 	{$21$ tuổi}
% 	\loigiai{
% 		Sau khi sắp xếp các tuổi trên thành dãy không giảm, do có $169$ số nên số trung vị là số thứ $85$ trong dãy trên.\\
% 		Mà số thứ $85$ trong dãy là $20$. Vậy $M_e=20$.
% 	}
% \end{ex}
% \begin{ex}%[0D5B3-2]
% 	Để khảo sát kết quả thi tuyển sinh môn Toán trong kì thi tuyển sinh Đại học năm vừa qua của trường $A$, người điều tra chọn một mẫu gồm $100$ học sinh tham gia kì thi tuyển sinh đó. Điểm môn Toán (thang điểm $10$) của các học sinh này được cho ở bảng phân bố tần số sau đây
% 	\begin{center}
% 		\begin{tabular}{|l|c|c|c|c|c|c|c|c|c|c|c|c|c|}
% 			\hline
% 			Điểm & $0$ & $1$ & $2$ & $3$ & $4$ & $5$ & $6$ & $7$ & $8$ & $9$ & $10$ &  \\
% 			\hline
% 			Tần số & $1$ & $1$ & $3$ & $5$ & $8$ & $13$ & $19$ & $24$ & $14$ & $10$ & $2$ & $n=100$ \\
% 			\hline	
% 		\end{tabular}
% 	\end{center}
% 	Số trung vị của mẫu số liệu trên.
% 	\choice
% 	{\True $M_e=6{,}5$}
% 	{$M_e=7{,}5$}
% 	{$M_e=5{,}5$}
% 	{$M_e=6$}
% 	\loigiai{
% 		Do kích thước mẫu $n=100$ là một số chẵn nên số trung vị là trung bình cộng của hai giá trị đứng thứ $\dfrac{n}{20}=50$ và $\dfrac{n}{2}+1=51$.\\
% 		Do đó $M_e=\dfrac{6+7}{2}=6{,}5$.
% 	}
% \end{ex}
% \begin{ex}%[0D5B3-2]
% 	Số áo bán được trong một quý ở một cửa hàng bán áo sơ-mi nam được cho trong bảng sau
% 	\begin{center}
% 		\begin{tabular}{|l|c|c|c|c|c|c|c|c|}
% 			\hline
% 			Cỡ số & $36$ & $37$ & $38$ & $39$ & $40$ & $41$ & $42$ & Cộng \\
% 			\hline
% 			Số áo bán được & $13$ & $45$ & $126$ & $110$ & $126$ & $40$ & $5$ & $465$ \\
% 			\hline	
% 		\end{tabular}
% 	\end{center}
% 	Hãy tìm số trung vị của các số liệu thống kê trên.
% 	\choice
% 	{$37$}
% 	{$38$}
% 	{\True $39$}
% 	{$40$}
% 	\loigiai{
% 		Ta sắp xếp dãy số áo bán được theo dãy không giảm
% 		$$36, 36, 36, \ldots, 36, 37, 37, \ldots, 37, 38, 38, \ldots, 38, \ldots, 42, 42.$$
% 		Dãy trên gồm $465$ số nên số trung vị là số thứ $233$.\\
% 		Mà số thứ $233$ là số $39$. Vậy $M_e=233$.
% 	}
% \end{ex}
\begin{ex}
	Cho bảng tần số về cân nặng của 180 người dân trong một xã như sau: (đơn vị: kg)
	\begin{center}
	\begin{tabular}{|c|c|c|}
	\hline
	\textbf{Nhóm} & \textbf{Tần số} & \textbf{Tần số tích luỹ}\\ 
	\hline
	$\left[0;10\right)$ & $6$ & $6$\\
	$\left[10;20\right)$ & $15$ & $21$\\
	$\left[20;30\right)$ & $37$ & $58$\\
	$\left[30;40\right)$ & $48$ & $106$\\
	$\left[40;50\right)$ & $22$ & $128$\\
	$\left[50;60\right)$ & $29$ & $157$\\
	$\left[60;70\right)$ & $23$ & $180$\\

	\hline
	& $n = 180$ &\\
	\hline
\end{tabular}	
	\end{center}
	Tứ phân vị thứ nhất của mẫu số liệu trên là
	\choice
	{$56{,}486$ kg}
	{\True $26{,}486$ kg}
	{$25{,}496$ kg}
	{$36{,}486$ kg}
	\loigiai{
	Số phần tử của mẫu là $n=180$  và $\dfrac{n}{4}=\dfrac{\cdot 180}{4}=45$.\\ Ta có  $21<135<58$ nên nhóm $3$ là nhóm  đầu tiên có   tần số tích luỹ  lớn hơn hoặc bằng $45$.\\
	Xét nhóm $3$ là  nhóm $\left[20;30\right)$ có $s=20$, $h=10$, $n_3=37$, nhóm $2$ là nhóm có $cf_2=21$.\\
	Vậy $Q_1=20+\dfrac{45-21}{37}\cdot 10\approx 26{,}49$ (kg).
}
	\end{ex}

\begin{ex}
	Cho bảng tần số chiều cao của 46 học sinh nam của khối lớp $11$ như sau
	\begin{center}
		\begin{tabular}{|c|c|}
			\hline
			\textbf{Nhóm} & \textbf{Tần số} \\
			\hline
			$\left[155;160\right)$ & $3$ \\
			$\left[160;165\right)$ & $18$ \\
			$\left[165;170\right)$ & $10$ \\
			$\left[170;175\right)$ & $15$ \\
			
			\hline
			& $n = 46$ \\
			\hline
		\end{tabular}	
	\end{center}
	Xác định tứ phân vị thứ nhất của mẫu số liệu trên
\choice
{$161{,}36$}
{$161{,}63$}
{\True $162{,}36$}
{$162{,}63$}
\loigiai{
	Số phần tử của mẫu là $n=46$; $\dfrac{n}{4}=11{,}5$.\\
		Ta có $cf_1=3$, $cf_2=3+18=21$ và $6<11{,}5<21$ nên nhóm $2$ là nhóm đầu tiên có   tần số tích luỹ  lớn hơn hoặc bằng $11{,}5$.\\
		Xét nhóm $2$ là nhóm $\left[160;165\right)$, ta có $s=160$, $h=5$, $n_6=21$; nhóm $1$ có tần số tích luỹ bằng $6$.\\
		Vậy $Q_1=160+\dfrac{11{,}5-3}{18}\cdot 5=162{,}36$.
}
\end{ex}
\begin{ex}
	Cho bảng tần số chiều cao của 46 học sinh nam của khối lớp $11$ như sau
	\begin{center}
		\begin{tabular}{|c|c|}
			\hline
			\textbf{Nhóm} & \textbf{Tần số} \\
			\hline
			$\left[155;160\right)$ & $3$ \\
			$\left[160;165\right)$ & $18$ \\
			$\left[165;170\right)$ & $10$ \\
			$\left[170;175\right)$ & $15$ \\
			
			\hline
			& $n = 46$ \\
			\hline
		\end{tabular}	
	\end{center}
	Xác định tứ phân vị thứ ba của mẫu số liệu trên
	\choice
	{$162{,}36$}
	{$166{,}5$}
	{ $166$}
	{\True $171{,}16$}
	\loigiai{
		Số phần tử của mẫu là $n=46$; $\dfrac{3n}{4}=34{,}5$.\\
		Ta có $cf_3=3+18+10=31$, $cf_4=31+15=46$ và $31<34{,}5<46$ nên nhóm $4$ là nhóm đầu tiên có   tần số tích luỹ  lớn hơn hoặc bằng $34{,}5$.\\
		Xét nhóm $4$ là nhóm $\left[170;175\right)$, ta có $t=170$, $l=5$, $n_4=15$; nhóm $3$ có tần số tích luỹ bằng $31$.\\
		Vậy $Q_3=170+\dfrac{34{,}5-31}{15}\cdot 5=171{,}16$.
	}
\end{ex}
\begin{ex}
	\immini{Cho bảng tần số ghép nhóm số liệu thống kê  chiều cao  của $40$ mẫu cây ở  một vườn thực vật 	(đơn vị: centimét).\\
		Xác định tứ phân vị thứ hai của  số liệu ghép nhóm trên
\choice
{\True $56{,}43$}
{$56{,}34$}
{$46{,}43$}
{$36{,}43$}		

}{\begin{tabular}{|c|c|c|}
	\hline
	\textbf{Nhóm} & \textbf{Tần số} & \textbf{Tần số tích luỹ}\\ 
	\hline
	$\left[30;40\right)$ & $4$ & $4$\\
	$\left[40;50\right)$ & $10$ & $14$\\
	$\left[50;60\right)$ & $14$ & $28$\\
	$\left[60;70\right)$ & $6$ & $34$\\
	$\left[70;80\right)$ & $4$ & $38$\\
	$\left[80; 	90\right)$ & $2$ & $40$\\

	
	\hline
	& $n = 40$ &\\
	\hline
\end{tabular}}
\loigiai{
	Số phần tử của mẫu là $n=46$; $\dfrac{n}{2}=23$.\\
	Ta có $cf_2=14<23<cf_3=28$ nên nhóm $3$ là nhóm đầu tiên có tần số tích lũy lớn hơn hoặc bằng $23$.\\
	Xét nhóm $3$ là nhóm $[50;60)$ có $r=50$, $d=10$, $n_3=14$ và nhóm $2$ là nhóm $\left[40;50\right)$ có $cf_2=14$.\\
	Do đó $Q_2=50+\dfrac{23-14}{14}\cdot 10=56{,}43$.
}
\end{ex}
\begin{ex}
Một bảng xếp hạng đã tính điềm chuẩn hoá cho chỉ số nghiên cứu của một số trường đại học ở
Việt Nam và thu được kết quả sau
\begin{center}
\begin{tabular}{|c|c|c|c|c|c|c|}
	\hline
	\textbf{Điểm} & Dưới $20$  &  $[20;30)$&  $[30;40)$ &  $[40;60)$ &  $[60;80)$ &  $[80;100)$\\
	\hline
	Số điểm & $4$ & $19$&$6$&$2$&$3$&$1$\\

	
	
	\hline

\end{tabular}	
\end{center}
Xác định điểm ngưỡng đề đưa ra danh sách $25$\% trường đại học có chỉ số nghiên cứu tốt nhất Việt Nam.	
\choice
{$25{,}26$}
{\True $35{,}42$}
{$45{,}35$}
{$45{,}42$}	
\loigiai{
Điểm ngưỡng để đưa ra danh sách $25$\% trường đại học có chỉ số nghiên cứu tốt nhất Việt Nam là tứ phân
vị thứ ba.\\
Ta có  $n=35$ và $\dfrac{3n}{4}=26{,}25$.\\
Do $cf_2=4+19=23<26{,}25<cf_3=23+6=29$ nên nhóm $[30;40])$ là nhóm đầu tiên có tần số tích lũy lớn hơn hoặc bằng $23{,}25$.\\
Nhóm $[30;40])$ có $r=30$, $d=10$, $n_3=6$; nhóm $2$ có $cf_2=23$. Do đó
$$Q_3=30+\dfrac{26{,}25-23}{6}\cdot 10\approx 35{,}41.$$ 
Vậy để đưa ra danh sách $25$\% trường đại học có chỉ số nghiên cứu tốt nhất Việt Nam ta lấy các trường có
điểm chuẩn hóa trên $35{,}42$.
}
\end{ex}
% \begin{ex}%1%[1K3B9-3]
% 	Điểm thi môn Toán (thang điểm 100, điểm được làm tròn đến 1) của 60 thí sinh được cho trong bảng sau
% 	\begin{center}
% 		\begin{tabular}{|l|c|c|c|c|c|}
% 			\hline Điểm & $[0-9,5)$ & $[9,5-19,5)$ & $[19,5-29,5)$ & $[29,5-39,5)$ & $[39,5-49,5)$ \\
% 			\hline Số thí sinh & $1 $& $2$ & $4$ & $6$ & $15$ \\
% 			\hline Điểm & $[49,5-59,5)$ & $[59,5-69,5)$ & $[69,5-79,5)$ & $[79,5-89,5)$ & $[89,5-99,5)$ \\
% 			\hline Số thí & $12$ & $10$ & $6$ & $3$ & $1$ \\
% 			\hline
% 		\end{tabular}    
% 	\end{center}
% 	Tìm  tứ phân vị thứ hai của mẫu số liệu.
% 	\choice
% 	{\True $Q_2\approx 51,17$}
% 	{$Q_2\approx 51,67$}
% 	{$Q_2\approx 49,5$}
% 	{$Q_2\approx 41,3$}
% 	\loigiai{Cỡ mẫu là $n=60$.\\
% 		Tứ phân vị thứ nhất $Q_2$ là $\dfrac{x_{30}+x_{31}}{2}$. Do $x_{30}$, $x_{31}$ đều thuộc nhóm $[49,5 ; 59,5)$ nên nhóm này chứa $Q_2$. \\Do đó, $p=6 ; \;a_6=49,5 ;\; m_6=12 ; \;m_1+\ldots+m_5=28, \;a_7-a_6=10$ và ta có
% 		$$
% 		Q_2=49,5+\dfrac{\frac{60}{2}-28}{12}\cdot 10\approx51,17.
% 		$$
% 	}
% \end{ex}
% %2
% \begin{ex}%[1K3B9-3]
% 	Điểm thi môn Toán (thang điểm 100, điểm được làm tròn đến 1) của $60$ thí sinh được cho trong bảng sau
% 	\begin{center}
% 		\begin{tabular}{|l|c|c|c|c|c|}
% 			\hline Điểm & $[0-9,5)$ & $[9,5-19,5)$ & $[19,5-29,5)$ & $[29,5-39,5)$ & $[39,5-49,5)$ \\
% 			\hline Số thí sinh & $1 $& $2$ & $4$ & $6$ & $15$ \\
% 			\hline Điểm & $[49,5-59,5)$ & $[59,5-69,5)$ & $[69,5-79,5)$ & $[79,5-89,5)$ & $[89,5-99,5)$ \\
% 			\hline Số thí & $12$ & $10$ & $6$ & $3$ & $1$ \\
% 			\hline
% 		\end{tabular}    
% 	\end{center}
% 	Tìm  tứ phân vị thứ nhất của mẫu số liệu.
% 	\choice
% 	{\True $Q_1\approx 41,3$}
% 	{$Q_1\approx 51,67$}
% 	{$Q_1\approx 40,83$}
% 	{$Q_1\approx 51,17$}
% 	\loigiai{Cỡ mẫu là $n=60$.\\
% 		Tứ phân vị thứ nhất $Q_1$ là $\dfrac{x_{15}+x_{16}}{2}$. Do $x_{15}$, $x_{16}$ đều thuộc nhóm $[39,5-49,5)$ nên nhóm này chứa $Q_1$. \\Do đó, $p=5 ; \;a_5=39,5 ;\; m_5=15 ; \;m_1+\ldots+m_4=13, \;a_6-a_5=10$ và ta có
% 		$$
% 		Q_1=39,5+\dfrac{\frac{60}{4}-13}{15}\cdot 10\approx 40,83.
% 		$$
% 	}
% \end{ex}
%3
\begin{ex}%[1K3B9-3]
	Phỏng vấn một số học sinh khối 11 vể thời gian (giờ) ngủ của một buổi tối, thu được bảng số liệu như sau.
	\begin{center}
		\begin{tabular}{|l|c|c|c|c|c|}
			\hline Thời gian  (giờ)  &{$[4 ; 5)$}&{$[5 ; 6)$}&{$[6 ; 7)$}&{$[7 ; 8)$}&{$[8 ; 9)$}\\
			\hline Số học sinh & $6$ & $10$ & $13$ & $9$ & $7$ \\
			\hline
		\end{tabular}     
	\end{center}
	Hãy cho biết $75 \%$ học sinh khối 11 ngủ ít nhất bao nhiêu giờ?
	\choice
	{$7,675$}
	{\True $7,53$}
	{$8$}
	{ $7,9$}
	\loigiai{
		Cỡ mẫu là $n=45$.\\
		Gọi $x_1, \ldots, x_{45}$ là mẫu số liệu được sắp xếp theo thứ tự không giảm. Khi đó, trung vị là $x_{23}$. Do đó, tứ phân vị thứ ba $Q_3$ là $x_{34}$. Do $x_{34}$ đều thuộc nhóm $[7;8)$ nên nhóm này chứa $Q_3$. Do đó, $p=4 ; \;a_4=7 ;\; m_4=9 ; \;m_1+m_2+m_3=29 ; \;a_5-a_4=1$ và ta có
		$$
		Q_3=7+\dfrac{\frac{3 \cdot 45}{4}-29}{9}\cdot 1\approx7,53.
		$$ 
		Vậy $75\%$ học sinh khối 11 ngủ ít nhất $7,53$ giờ.
	}
\end{ex}
%4
% \begin{ex}%[1K3B9-3]
% 	Điểm thi môn Toán (thang điểm 100, điểm được làm tròn đến 1) của 60 thí sinh được cho trong bảng sau
% 	\begin{center}
% 		\begin{tabular}{|l|c|c|c|c|c|}
% 			\hline Điểm & $[0-9,5)$ & $[9,5-19,5)$ & $[19,5-29,5)$ & $[29,5-39,5)$ & $[39,5-49,5)$ \\
% 			\hline Số thí sinh & $1 $& $2$ & $4$ & $6$ & $15$ \\
% 			\hline Điểm & $[49,5-59,5)$ & $[59,5-69,5)$ & $[69,5-79,5)$ & $[79,5-89,5)$ & $[89,5-99,5)$ \\
% 			\hline Số thí & $12$ & $10$ & $6$ & $3$ & $1$ \\
% 			\hline
% 		\end{tabular}    
% 	\end{center}
% 	Tìm  tứ phân vị thứ ba của mẫu số liệu.
% 	\choice
% 	{$Q_3=41,3$}
% 	{$Q_3=51,67$}
% 	{$Q_3=45$}
% 	{\True $Q_3=65$}
% 	\loigiai{Cỡ mẫu là $n=60$.\\
% 		Với tứ phân vị thứ ba $Q_3$ là $\dfrac{x_{45}+x_{46}}{2}$. Do $x_{45}$, $x_{46}$ đều thuộc nhóm $[60 ; 70)$ nên nhóm này chứa $Q_3$. Do đó, $p=7 ; \;a_7=60 ;\; m_7=10 ; \;m_1+\ldots+m_6=40 ; \;a_8-a_7=10$ và ta có
% 		$$
% 		Q_3=59,5+\dfrac{\frac{3 \cdot 60}{4}-40}{10}\cdot 10=64,5.
% 		$$
% 	}
% \end{ex}
%5
\begin{ex}%[1K3B9-3]
	Một hãng xe ô tô thống kê lại số lần gặp sự cố về động cơ về động cơ của $100$ chiếc xe cùng loại sau 2 năm sử dụng đầu tiên ở dảng sau
	\begin{center}
		\begin{tabular}{|l|c|c|c|c|c|}
			\hline Số lần gặp sự cố  &{$[0,5;2,5)$}&{$[2,5;4,5)$}&{$[4,5;6,5)$}&{$[6,5 ; 8,5)$}&{$[8,5;10,5)$}\\
			\hline Số xe & $17$ & $33$ & $25$ & $20$ & $5$ \\
			\hline
		\end{tabular}     
	\end{center}
	Tìm tứ phân vị thứ nhất của mẫu số liệu.
	\choice
	{\True $Q_1\approx 4$}
	{$Q_1\approx 2,98$}
	{$Q_1\approx 2,5$}
	{$Q_1\approx 3,5$}
	\loigiai{
		Cỡ mẫu là $n=100$.\\
		Gọi $x_1, \ldots, x_{100}$ là mẫu số liệu được sắp xếp theo thứ tự không giảm. Khi đó, trung vị là $\dfrac{x_{50}+x_{51}}{2}$. 
		Do đó, tứ phân vị thứ nhất $Q_1$ là $\dfrac{x_{25}+x_{26}}{2}$. Do $x_{25}$, $x_{26}$ đều thuộc nhóm $[2,5;4,5)$ nên nhóm này chứa $Q_1$. \\Do đó, $p=2 ; \;a_2=2,5;\; m_2=33 ; \;m_1=17, \;a_3-a_2=2$ và ta có
		$$
		Q_1=2,5+\dfrac{\frac{100}{4}-17}{33}\cdot 2\approx 2,98.
		$$
	}    
\end{ex}
%6
% \begin{ex}%[1K3B9-3]
% 	Một hãng xe ô tô thống kê lại số lần gặp sự cố về động cơ về động cơ của $100$ chiếc xe cùng loại sau 2 năm sử dụng đầu tiên ở dảng sau
% 	\begin{center}
% 		\begin{tabular}{|l|c|c|c|c|c|}
% 			\hline Số lần gặp sự cố  &{$[0,5;2,5)$}&{$[2,5;4,5)$}&{$[4,5;6,5)$}&{$[6,5 ; 8,5)$}&{$[8,5;10,5)$}\\
% 			\hline Số xe & $17$ & $33$ & $25$ & $20$ & $5$ \\
% 			\hline
% 		\end{tabular}     
% 	\end{center}
% 	Tìm   tứ phân vị thứ hai của mẫu số liệu.
% 	\choice
% 	{\True $Q_2=4,5$}
% 	{$Q_2\approx 5,12$}
% 	{$Q_2\approx 4,89$}
% 	{$Q_2\approx 5,2$}
% 	\loigiai{
% 		Cỡ mẫu là $n=100$.\\
% 		Gọi $x_1, \ldots, x_{100}$ là mẫu số liệu được sắp xếp theo thứ tự không giảm. Khi đó, trung vị là $\dfrac{x_{50}+x_{51}}{2}$. Do $x_{50} \in [2,5;4,5)$, $x_{51} \in [4,5;6,5)$  nên tứ phân vị thứ hai của mẫu số liệu ghép nhóm là  $Q_2=4,5$. 
% 	}
% \end{ex}
%7
\begin{ex}%[1K3B9-3]
	Một hãng xe ô tô thống kê lại số lần gặp sự cố về động cơ về động cơ của $100$ chiếc xe cùng loại sau 2 năm sử dụng đầu tiên ở dảng sau
	\begin{center}
		\begin{tabular}{|l|c|c|c|c|c|}
			\hline Số lần gặp sự cố  &{$[0,5;2,5)$}&{$[2,5;4,5)$}&{$[4,5;6,5)$}&{$[6,5 ; 8,5)$}&{$[8,5;10,5)$}\\
			\hline Số xe & $17$ & $33$ & $25$ & $20$ & $5$ \\
			\hline
		\end{tabular}     
	\end{center}
	Tìm   tứ phân vị thứ ba của mẫu số liệu.  
	\choice
	{$Q_3=6,3$}
	{$Q_3=6,8$}
	{$Q_3=7,2$}
	{\True $Q_3=6,5$}
	\loigiai{ Cỡ mẫu là $n=100$.\\
		Với tứ phân vị thứ ba $Q_3$ là $\dfrac{x_{75}+x_{76}}{2}$. Do $x_{75} \in [4,5;6,5)$, $x_{76} \in [6,5 ; 8,5)$  nên tứ phân vị thứ ba của mẫu số liệu ghép nhóm là $Q_3=6,5$. 
		
	}
\end{ex}
%8
% \begin{ex}%[1K3B9-3]
% 	Lương tháng của một số nhân viên văn phòng được ghi lại như sau (đơn vị: triệu đồng)
% 	\begin{center}
% 		\begin{tabular}{|l|c|c|c|c|c|}
% 			\hline Lương tháng (triệu đồng)  &{$[6;8)$}&{$[8;10)$}&{$[10;12)$}&{$[12;14)$}\\
% 			\hline Số nhân viên & $3$ & $6$ & $8$ & $7$  \\
% 			\hline
% 		\end{tabular}     
% 	\end{center}  
% 	Tìm tứ phân vị thứ nhất của mẫu số liệu.
% 	\choice
% 	{\True $Q_1= 9$}
% 	{$Q_1= 8,5$}
% 	{$Q_1= 9,5$}
% 	{$Q_1= 8,2$}
% 	\loigiai{
% 		Cỡ mẫu là $n=24$.\\
% 		Gọi $x_1, \ldots, x_{24}$ là mẫu số liệu được sắp xếp theo thứ tự không giảm. Khi đó, trung vị là $\dfrac{x_{12}+x_{13}}{2}$. 
% 		Do đó, tứ phân vị thứ nhất $Q_1$ là $\dfrac{x_{6}+x_{7}}{2}$. Do $x_{6}$, $x_{7}$ đều thuộc nhóm $[8;10)$ nên nhóm này chứa $Q_1$. \\Do đó, $p=2 ; \;a_2=8;\; m_2=6 ; \;m_1=3, \;a_3-a_2=2$ và ta có
% 		$$
% 		Q_1=8+\dfrac{\frac{24}{4}-3}{6}\cdot 2=9.
% 		$$
% 	}    
% \end{ex}
% %9
% \begin{ex}%[1K3B9-3]
% 	Lương tháng của một số nhân viên văn phòng được ghi lại như sau (đơn vị: triệu đồng)
% 	\begin{center}
% 		\begin{tabular}{|l|c|c|c|c|c|}
% 			\hline Lương tháng (triệu đồng)  &{$[6;8)$}&{$[8;10)$}&{$[10;12)$}&{$[12;14)$}\\
% 			\hline Số nhân viên & $3$ & $6$ & $8$ & $7$  \\
% 			\hline
% 		\end{tabular}     
% 	\end{center}   
% 	Tìm   tứ phân vị thứ hai của mẫu số liệu.
% 	\choice
% 	{\True $Q_2=10,75$}
% 	{$Q_2= 10,5$}
% 	{$Q_2= 11$}
% 	{$Q_2=11,5$}
% 	\loigiai{
% 		Cỡ mẫu là $n=24$.\\
% 		Gọi $x_1, \ldots, x_{24}$ là mẫu số liệu được sắp xếp theo thứ tự không giảm. Khi đó, trung vị là $\dfrac{x_{12}+x_{13}}{2}$. 
% 		Do đó,  tứ phân vị thứ hai $Q_2$ là $\dfrac{x_{12}+x_{13}}{2}$. Do $x_{12}$, $x_{13}$ đều thuộc nhóm $[10;12)$ nên nhóm này chứa $Q_2$. \\Do đó, $p=3 ; \;a_3=10;\; m_3=8 ; \;m_1+m_2=9, \;a_4-a_3=2$ và ta có
% 		$$
% 		Q_2=10+\dfrac{\frac{24}{2}-9}{8}\cdot 2=10,75.
% 		$$
% 	}
% \end{ex}
% %10
% \begin{ex}%[1K3B9-3]
% 	Lương tháng của một số nhân viên văn phòng được ghi lại như sau (đơn vị: triệu đồng)
% 	\begin{center}
% 		\begin{tabular}{|l|c|c|c|c|c|}
% 			\hline Lương tháng (triệu đồng)  &{$[6;8)$}&{$[8;10)$}&{$[10;12)$}&{$[12;14)$}\\
% 			\hline Số nhân viên & $3$ & $6$ & $8$ & $7$  \\
% 			\hline
% 		\end{tabular}     
% 	\end{center}  
% 	Tìm   tứ phân vị thứ ba của mẫu số liệu.
% 	\choice 
% 	{$Q_3\approx 12,5$}
% 	{$Q_3\approx 13,2$}
% 	{$Q_3\approx 13,5$}
% 	{\True $Q_3\approx 12,3$}
% 	\loigiai{
% 		Cỡ mẫu là $n=24$.\\
% 		Gọi $x_1, \ldots, x_{24}$ là mẫu số liệu được sắp xếp theo thứ tự không giảm. Khi đó, trung vị là $\dfrac{x_{12}+x_{13}}{2}$. 
% 		Do đó,  tứ phân vị thứ ba $Q_3$ là $\dfrac{x_{18}+x_{19}}{2}$. Do $x_{18}$, $x_{19}$ đều thuộc nhóm $[12;14)$ nên nhóm này chứa $Q_3$. \\Do đó, $p=4 ; \;a_4=12;\; m_4=7 ; \;m_1+m_2+m_3=17, \;a_4-a_3=2$ và ta có
% 		$$
% 		Q_2=12+\dfrac{\frac{24\cdot 3}{4}-17}{7}\cdot 2\approx 12,3.
% 		$$
% 	}
% \end{ex}
%11
% \begin{ex}%[1K3B9-3]
% 	Số điểm một cầu thủ bóng rổ ghi được trong 20 trận đấu được cho ở bảng sau
% 	\begin{center}
% 		\begin{tabular}{|l|c|c|c|c|c|}
% 			\hline Điểm số  &{$[5,5;10,5)$}&{$[10,5;15,5)$}&{$[15,5;20,5)$}&{$[20,5;25,5)$}\\
% 			\hline Số trận & $3$ & $9$ & $2$ & $6$  \\
% 			\hline
% 		\end{tabular}     
% 	\end{center}  
% 	Tìm   tứ phân vị thứ ba của mẫu số liệu.
% 	\choice 
% 	{$Q_3\approx 23,5$}
% 	{$Q_3\approx 22,2$}
% 	{$Q_3\approx 21,6$}
% 	{\True $Q_3\approx 21,3$}
% 	\loigiai{
% 		Cỡ mẫu là $n=20$.\\
% 		Gọi $x_1, \ldots, x_{20}$ là mẫu số liệu được sắp xếp theo thứ tự không giảm. Khi đó, trung vị là $\dfrac{x_{10}+x_{11}}{2}$. 
% 		Do đó,  tứ phân vị thứ ba $Q_3$ là $\dfrac{x_{15}+x_{16}}{2}$. Do $x_{15}$, $x_{16}$ đều thuộc nhóm $[20,5;25,5)$ nên nhóm này chứa $Q_3$. \\Do đó, $p=4 ; \;a_4=20,5;\; m_4=6 ; \;m_1+m_2+m_3=14, \;a_5-a_4=25,5-20,5=5$ và ta có
% 		$$
% 		Q_2=20,5+\dfrac{\frac{20\cdot 3}{4}-14}{6}\cdot 5 \approx 21,3.
% 		$$
% 	}
% \end{ex}
% %12
% \begin{ex}%[1K3B9-3]
% 	Số điểm một cầu thủ bóng rổ ghi được trong 20 trận đấu được cho ở bảng sau
% 	\begin{center}
% 		\begin{tabular}{|l|c|c|c|c|c|}
% 			\hline Điểm số  &{$[5,5;10,5)$}&{$[10,5;15,5)$}&{$[15,5;20,5)$}&{$[20,5;25,5)$}\\
% 			\hline Số trận & $3$ & $9$ & $2$ & $6$  \\
% 			\hline
% 		\end{tabular}     
% 	\end{center} 
% 	Tìm tứ phân vị thứ nhất của mẫu số liệu.
% 	\choice
% 	{\True $Q_1\approx 11,6$}
% 	{$Q_1\approx 11,3$}
% 	{$Q_1\approx 21,6$}
% 	{$Q_1\approx 21,3$}
% 	\loigiai{
% 		Cỡ mẫu là $n=20$.\\
% 		Gọi $x_1, \ldots, x_{20}$ là mẫu số liệu được sắp xếp theo thứ tự không giảm. Khi đó, trung vị là $\dfrac{x_{10}+x_{11}}{2}$. 
% 		Do đó, tứ phân vị thứ nhất $Q_1$ là $\dfrac{x_{5}+x_{6}}{2}$. Do $x_{5}$, $x_{6}$ đều thuộc nhóm $[10,5;15,5)$ nên nhóm này chứa $Q_1$. \\Do đó, $p=2 ; \;a_2=10,5;\; m_2=9 ; \;m_1=3, \;a_3-a_2=5$ và ta có
% 		$$
% 		Q_1=10,5+\dfrac{\frac{20}{4}-3}{9}\cdot 5\approx 11,6.
% 		$$
% 	}
% \end{ex}
%13
\begin{ex}%[1K3B9-3]
	Số điểm một cầu thủ bóng rổ ghi được trong 20 trận đấu được cho ở bảng sau
	\begin{center}
		\begin{tabular}{|l|c|c|c|c|c|}
			\hline Điểm số  &{$[5,5;10,5)$}&{$[10,5;15,5)$}&{$[15,5;20,5)$}&{$[20,5;25,5)$}\\
			\hline Số trận & $3$ & $9$ & $2$ & $6$  \\
			\hline
		\end{tabular}     
	\end{center} 
	Tìm tứ phân vị thứ nhất của mẫu số liệu.
	\choice
	{\True $Q_1\approx 11,6$}
	{$Q_1\approx 14,4$}
	{$Q_1\approx 15,6$}
	{$Q_1\approx 21,3$}
	\loigiai{
		Cỡ mẫu là $n=20$.\\
		Gọi $x_1, \ldots, x_{20}$ là mẫu số liệu được sắp xếp theo thứ tự không giảm. Khi đó, trung vị là $\dfrac{x_{10}+x_{11}}{2}$. 
		Do đó, tứ phân vị thứ nhất $Q_2$ là $\dfrac{x_{10}+x_{11}}{2}$. Do $x_{10}$, $x_{11}$ đều thuộc nhóm $[10,5;15,5)$ nên nhóm này chứa $Q_2$. \\Do đó, $p=2 ; \;a_2=10,5;\; m_2=9 ; \;m_1=3, \;a_3-a_2=5$ và ta có
		$$
		Q_1=10,5+\dfrac{\frac{20}{2}-3}{9}\cdot 5\approx 14,4.
		$$   
	}
\end{ex}
%14
% \begin{ex}%[1K3B9-3]
% 	Một người thống kê lại thời gian thực hiện các cuộc gọi điện thoại của người đó trong một tuần cho trong bảng sau
% 	\begin{center}
% 		\begin{tabular}{|l|c|c|c|c|c|}
% 			\hline Số bệnh nhân &{$[0;60)$}&{$[60;120)$}&{$[120;180)$}&{$[180;240)$}&{$[240;300)$}\\
% 			\hline Số ngày & $8$ & $10$ & $7$ & $5$ & $2$ \\
% 			\hline
% 		\end{tabular}     
% 	\end{center}
% 	Tìm   tứ phân vị thứ ba của mẫu số liệu.  
% 	\choice
% 	{$Q_3\approx 175,28$}
% 	{$Q_3\approx 150,32 $}
% 	{$Q_3=175$}
% 	{\True $Q_3\approx 171,43$}
% 	\loigiai{ Cỡ mẫu là $n=32$.\\
% 		Gọi $x_1, \ldots, x_{32}$ là mẫu số liệu được sắp xếp theo thứ tự không giảm. Khi đó, trung vị là $\dfrac{x_{16}+x_{17}}{2}$.\\
% 		Do đó, tứ phân vị thứ ba $Q_3$ là $\dfrac{x_{24}+x_{25}}{2}$. Do $x_{24} $, $x_{25} \in [120;180)$   nên nhóm này chứa $Q_3$. \\Do đó, $p= 3; \;a_3=120 ;\; m_3=7 ; \;m_1+m_2=18 ; \;a_4-a_3=60$ và ta có
% 		$$
% 		Q_3=120+\dfrac{\frac{3 \cdot 32}{4}-18}{7}\cdot 60\approx 171,43.
% 		$$
% 	}
% \end{ex}
% %15
% \begin{ex}%[1K3B9-3]
% 	Một người thống kê lại thời gian thực hiện các cuộc gọi điện thoại của người đó trong một tuần cho trong bảng sau
% 	\begin{center}
% 		\begin{tabular}{|l|c|c|c|c|c|}
% 			\hline Số bệnh nhân &{$[0;60)$}&{$[60;120)$}&{$[120;180)$}&{$[180;240)$}&{$[240;300)$}\\
% 			\hline Số ngày & $8$ & $10$ & $7$ & $5$ & $2$ \\
% 			\hline
% 		\end{tabular}     
% 	\end{center}
% 	Tìm   tứ phân vị thứ hai của mẫu số liệu.  
% 	\choice
% 	{$Q_2\approx 80,25$}
% 	{$Q_2\approx 100,32$}
% 	{$Q_2=115$}
% 	{\True $Q_2=108$}
% 	\loigiai{ Cỡ mẫu là $n=32$.\\
% 		Gọi $x_1, \ldots, x_{32}$ là mẫu số liệu được sắp xếp theo thứ tự không giảm. Khi đó, trung vị là $\dfrac{x_{16}+x_{17}}{2}$.\\
% 		Do đó, tứ phân vị thứ hai $Q_2$ là $\dfrac{x_{16}+x_{17}}{2}$. Do $x_{16} $, $x_{17} \in [60;120)$   nên nhóm này chứa $Q_2$. \\Do đó, $p= 2; \;a_2=60 ;\; m_2=10 ; \;m_1=8 ; \;a_3-a_2=60$ và ta có
% 		$$
% 		Q_3=60+\dfrac{\frac{ 32}{2}-8}{10}\cdot 60=108.
% 		$$
% 	}    
% \end{ex}
% \begin{ex}
% 	Một công ty xây dựng khảo sát khách hàng xem họ có nhu cầu mua nhà ở mức giá nào. Kết quả khảo sát được ghi lại ở bảng sau
% 	\begin{center}
% 		\begin{tabular}{|c|c|c|c|c|c|}
% 			\hline \begin{tabular}{c}
% 				\textbf{Mức giá} \\
% 				\textbf{(triệu đồng/$\mathrm{m}^2)$}
% 			\end{tabular} &{$[10; 14)$} &{$[14; 18)$} &{$[18; 22)$} &{$[22; 26)$} &{$[26; 30)$} \\
% 			\hline \textbf{Số khách hàng} & 54 & 78 & 120 & 45 & 12 \\
% 			\hline
% 		\end{tabular}
% 	\end{center}
	
% 	 Công ty nên xây nhà ở mức giá nào để nhiều người có nhu cầu mua nhất?
% 	\choice
% 	{\True $19{,}4$ triệu đồng$/ \mathrm{m}^2$ }
% 	{$20{,}4$ triệu đồng$/ \mathrm{m}^2$ }
% 	{$19{,}6$ triệu đồng$/ \mathrm{m}^2$ }
% 	{$20{,}6$ triệu đồng$/ \mathrm{m}^2$ }	
% 	\loigiai{
% 		\ Nhóm chứa mốt của mẫu số liệu trên là nhóm $[18; 22)$.\\ Do đó $u_m=18$, $n_{m-1}=78$, $n_m=120$, $n_{m+1}=45$, $u_{m+1}-u_m=22-18=4$.\\
% 			Mốt của mẫu số liệu ghép nhóm là
% 			\[M_0=18+\dfrac{120-78}{(120-78)+(120-45)} \cdot 4=\dfrac{758}{39} \approx 19{,}4. \]
% 		 Dựa vào kết quả trên ta có thể dự đoán rằng nếu công ty xây nhà ở mức giá $19{,}4$ triệu đồng$/ \mathrm{m}^2$ thì sẽ có nhiều người có nhu cầu mua nhất.
		
% 	}
% \end{ex}
\begin{ex}%[1T5B1-3]
	Số cuộc gọi điện thoại một nguời thực hiện mỗi ngày trong $30$ ngày được lựa chọn ngẫu nhiên được thống kê trong bảng sau:
\begin{center}
	\begin{tabular}{|c|c|c|c|c|c|}
		\hline Số cuộc gọi &{$[3; 5]$} &{$[6; 8]$} &{$[9; 11]$} &{$[12; 14]$} &{$[15; 17]$} \\
		\hline Số ngày & 5 & 13 & 7 & 3 & 2 \\
		\hline
	\end{tabular}
\end{center}
 Tìm mốt của mẫu số liệu ghép nhóm trên.
 Hãy dự đoán xem khả năng người đó thực hiện bao nhiêu cuộc gọi mỗi ngày là cao nhất.
\choice
{$4$}
{$6$}
{$5$}
{\True $7$}	
\loigiai{
	Do số cuộc gọi là số nguyên nên ta hiệu chỉnh lại như sau:
	\begin{center}
		\begin{tabular}{|c|c|c|c|c|c|}
			\hline Số cuộc gọi &{$[2{,}5; 5{,}5)$} &{$[5{,}5; 8{,}5)$} &{$[8{,}5; 11{,}5)$} &{$[11{,}5; 14{,}5)$} &{$[14{,}5; 17{,}5)$} \\
			\hline Số ngày & 5 & 13 & 7 & 3 & 2 \\
			\hline
		\end{tabular}
	\end{center}	
		 Nhóm chứa mốt của mẫu số liệu trên là nhóm $[5{,}5; 8{,}5)$.\\
		Do đó $u_m=5{,}5$; $n_{m-1}=5$; $n_m=13$; $n_{m+1}=7$; $u_{m+1}-u_m=8{,}5-5{,}5=3$.\\
		Mốt của mẫu số liệu ghép nhóm là
		\[M_0=5{,}5+\dfrac{13-5}{(13-5)+(13-7)} \cdot 3=\dfrac{101}{14} \approx 7{,}2. \]
	 Dựa vào kết quả trên ta có thể dự đoán rằng khả năng người đó thực hiện $7$ cuộc gọi mỗi ngày là cao nhất.
	
}
\end{ex}
\begin{ex}
	Một thư viện thống kê số lượng sách được mượn mỗi ngày trong ba tháng ở bảng sau:
	\begin{center}
		\begin{tabular}{|c|c|c|c|c|c|c|c|}
			\hline Số sách &{$[16; 20]$} &{$[21; 25]$} &{$[26; 30]$} &{$[31; 35]$} &{$[36; 40]$} &{$[41; 45]$} &{$[46; 50]$} \\
			\hline Số ngày & 3 & 6 & 15 & 27 & 22 & 14 & 5 \\
			\hline
		\end{tabular}
	\end{center}
	Hãy ước lượng  mốt của mẫu số liệu ghép nhóm trên.
	\choice
	{$34{,}33$}
	{\True $34{,}03$}
	{$35{,}63$}
	{$34{,}13$}	
	\loigiai{
		Vì số lượng sách được mượn là số nguyên nên ta hiệu chỉnh bảng tần số ghép nhóm (theo giá trị đại diện) như sau
		\begin{center}
			{\footnotesize \begin{tabular}{|c|c|c|c|c|c|c|c|}
					\hline Số sách &{$[15{,}5; 20{,}5)$} &{$[20{,}5; 25{,}5)$} &{$[25{,}5; 30{,}5)$} &{$[30{,}5; 35{,}5)$} &{$[35{,}5; 40{,}5]$} &{$[40{,}5; 45{,}5)$} &{$[45{,}5; 50{,}5)$} \\
					\hline Giá trị đại diện &{$18$} &{$23$} &{$28$} &{$33$} &{$38$} &{$43$} &{$48$} \\
					\hline Số ngày & 3 & 6 & 15 & 27 & 22 & 14 & 5 \\
					\hline
			\end{tabular}}
		\end{center}
		Trung bình số lượng sách được mượn mỗi ngày trong 3 tháng của thư viện là
		\[\overline{x}=\dfrac{18\cdot 3+23\cdot 6+28\cdot 15+33\cdot 27+38\cdot 22+43\cdot 14+48\cdot 5}{92}\approx 34{,}58. \]
		Nhóm chứa mốt của mẫu số liệu trên là nhóm $[30{,}5; 35{,}5)$.\\
		Do đó $u_m=30{,}5$; $n_{m-1}=15$; $n_m=27$; $n_{m+1}=22$; $u_{m+1}-u_m=35{,}5-30{,}5=5$.\\
		Mốt của mẫu số liệu ghép nhóm là
		\[M_0=30{,}5+\dfrac{27-15}{(27-15)+(27-22)} \cdot 5\approx 34{,}03. \]
	}
\end{ex}
\begin{ex}
	Kết quả đo chiều cao của $200$ cây keo $3$ năm tuổi ở một nông trường được biểu diễn ở biểu đồ dưới đây.
	\begin{center}
		\begin{tikzpicture}[scale=1,font=\scriptsize]
		\def\hoanh{11.5};
		\def\tung{6.5};
		\def\mau{cyan};
		\foreach \x/\n in{1/20,3/35,5/60,7/55,9/30}{\draw[line width=16mm,\mau] (\x,0)--++(0,{\n/10});
			\draw[dashed] (\x,{\n/10})node[above]{$\n$}--(0,{\n/10}) node[left]{$\n$};}
		\foreach \x/\p in {1/[8{,}5;8{,}8),3/[8{,}8;9{,}1),5/[9{,}1;9{,}4),7/[9{,}4;9{,}7),9/[9{,}7;10{,}0)}{\node[below] at (\x,0){\scriptsize $\p$};}
		\draw[->] (0,0)--(\hoanh,0) node[below]{($m$)};
		\draw[->] (0,0)node[below left]{$O$}--(0,\tung) node[left]{(Số cây)};
		\path (current bounding box.north) node[above]		{\textbf{Chiều cao 200 cây keo 3 năm tuổi}};
		\end{tikzpicture}
	\end{center}
	Mốt của mẫu số liệu ghép nhóm trên là
	\choice
	{$9{,}35$}
	{$10{,}53$}
	{$10{,}35$}
	{$9{,}53$}	
	\loigiai{
		Bảng tần số ghép nhóm (theo giá trị đại diện)
		\begin{center}
			\begin{tabular}{|c|c|c|c|c|c|}
				\hline Chiều cao &$[8{,}5; 8{,}8)$ &{$[8{,}8; 9{,}1)$} &{$[9{,}1; 9{,}4)$} &{$[9{,}4; 9{,}7)$} &{$[9{,}7; 10{,}0)$} \\
				\hline Giá trị đại diện &$8{,}65$ &$8{,}95$ &$9{,}25$ &$9{,}55$ &$9{,}85$ \\
				\hline Số cây & $20$ & $35$ & $60$ & $55$ & $30$\\
				\hline
			\end{tabular}
		\end{center}
		Chiều cao trung bình của $200$ cây keo 3 năm tuổi là
		\[\overline{x}=\dfrac{8{,}65\cdot 20+8{,}95\cdot 35+9{,}25\cdot 60+9{,}55\cdot 55+9{,}85\cdot 30}{200}\approx 9{,}31. \]
		Nhóm chứa mốt của mẫu số liệu trên là nhóm $[9{,}1; 9{,}4)$.\\
		Do đó $u_m=9{,}1$; $n_{m-1}=35$; $n_m=60$; $n_{m+1}=55$; $u_{m+1}-u_m=9{,}4-9{,}1=0{,}3$.\\
		Mốt của mẫu số liệu ghép nhóm là
		\[M_0=9{,}1+\dfrac{60-35}{(60-35)+(60-55)} \cdot 0{,}3= 9{,}35. \]
	}
\end{ex}
\begin{ex}%[1K3B9-4]
	Bảng số liệu ghép nhóm sau cho biết chiều cao (cm) của $50$ học sinh lớp $11A$.
	\begin{center}
		\begin{tabular}{|c|c|c|c|c|c|c|}
			\hline
			Khoảng chiều cao (cm)	& $\left[145;150 \right)$ & $\left[150;155 \right)$ & $\left[155;160 \right)$ & $\left[160;165 \right)$&$\left[165;170 \right)$  \\
			\hline
			Số học sinh&$7$	& $14$ & $10$ &$10$  & $9$ \\
			\hline
		\end{tabular}
	\end{center}
Mốt của mẫu số liệu ghép nhóm là
	\choice
	{$154{,}20$}
	{\True $153{,}18$}
	{$155{,}12$}
	{$158{,}36$}	
	\loigiai{
		Tần số lớn nhất là $14$ nên nhóm chứa mốt là nhóm $\left[150;155 \right)$. Ta có $j=2$, $a_2=150$, $m_2=14$, $m_1=7$, $m_3=10$, $h=5$. Do đó $$M_o=150+\dfrac{14-7}{\left(14-7\right)+\left(14-10\right)\cdot 5}\approx 153{,}18.$$	
		Số học sinh có chiều cao khoảng $153{,}18$ là nhiều nhất.
	}
\end{ex}
\begin{ex}%[1K3Y9-4]
	Chọn khẳng định \textbf{sai}.
	\choice
	{ Mốt của mẫu số liệu không ghép nhóm là giá trị có khả năng xuất hiện cao nhất khi lấy mẫu}
	{Mốt của mẫu số liệu sau khi ghép nhóm xấp xỉ với mốt của mẫu số liệu không ghép nhóm}
	{\True Một mẫu số liệu ghép nhóm chỉ có một mốt}
	{Một mẫu số liệu ghép nhóm có thể có nhiều nhóm chứa mốt và nhiều mốt}
	\loigiai{
		Mốt của mẫu số liệu không ghép nhóm là giá trị có khả năng xuất hiện cao nhất khi lấy mẫu.\\ Mốt của mẫu số liệu sau khi ghép nhóm xấp xỉ với mốt của mẫu số liệu không ghép nhóm. \\
		Một mẫu số liệu ghép nhóm có thể có nhiều nhóm chứa mốt và nhiều mốt.\\
		Do đó khẳng định sai là: Một mẫu số liệu ghép nhóm chỉ có một mốt.
	}    
\end{ex}
% \begin{ex}%[1K3Y9-4]
% 	Người ta ghi lại tuổi thọ của một số con ong cho kết quả như sau:
% 	\begin{center}
% 		\begin{tabular}{|l|c|c|c|c|c|c|}
% 			\hline Tuồi thọ (ngày) &{$[0;20)$}&{$[20;40)$}&{$[40;60)$}&{$[60;80)$}&{$[80;100)$}\\
% 			\hline Số lượng & $5$ & $12$ & $23$ & $31$ & $29$  \\
% 			\hline
% 		\end{tabular}
% 	\end{center}
% 	Nhóm chứa mốt của mẫu số liệu này là
% 	\choice
% 	{ $[20;40)$}
% 	{$[40;60)$}
% 	{\True $[60;80)$}
% 	{$[80;100)$}
% 	\loigiai{
% 		Nhóm chứa mốt của mẫu số liệu này là $[60;80)$.
% 	}    
% \end{ex}
% %6
% \begin{ex}%[1K3B9-4]
% 	Người ta ghi lại tuổi thọ của một số con ong cho kết quả như sau:
% 	\begin{center}
% 		\begin{tabular}{|l|c|c|c|c|c|c|}
% 			\hline Tuồi thọ (ngày) &{$[0;20)$}&{$[20;40)$}&{$[40;60)$}&{$[60;80)$}&{$[80;100)$}\\
% 			\hline Số lượng & $5$ & $12$ & $23$ & $31$ & $29$  \\
% 			\hline
% 		\end{tabular}
% 	\end{center}
% 	Mốt của mẫu số liệu này là
% 	\choice
% 	{ $M_0=70$}
% 	{$M_0=60$}
% 	{\True $M_0=76$}
% 	{$M_0=31$}
% 	\loigiai{
% 		Tần số lớn nhất là $31$ nên nhóm chứa mốt là nhóm $[60;80)$. \\
% 		Ta có, $j=4, a_4=60, m_4=31$, $m_3=23, m_5=29, h=20$. Do đó
% 		$$
% 		M_0=60+\frac{31-23}{(31-29)+(14-11)}\cdot 20 =76.
% 		$$
% 	}   
% \end{ex}
% %7
% \begin{ex}%[1K3Y9-4]
% 	Doanh thu bán hàng 20 ngày được lựa chọn ngẫu nhiên của một cửa hàng được ghi lại ở bảng sau (đơn vị: triệu đồng)
% 	\begin{center}
% 		\begin{tabular}{|l|c|c|c|c|c|c|}
% 			\hline Doanh thu &{$[5;7)$}&{$[7;9)$}&{$[9;11)$}&{$[11;13)$}&{$[13;15)$}\\
% 			\hline Số ngày & $2$ & $9$ & $7$ & $3$ & $1$ \\
% 			\hline
% 		\end{tabular}
% 	\end{center}
% 	Nhóm chứa mốt của mẫu số liệu này là
% 	\choice
% 	{ $[9;11)$}
% 	{$[5;7)$}
% 	{\True $[7;9)$}
% 	{$[11;13)$}
% 	\loigiai{
% 		Tần số lớn nhất là $9$ nên nhóm chứa mốt là nhóm $[7;9)$. \\
% 	}   
% \end{ex}
% %8
% \begin{ex}%[1K3B9-4]
% 	Doanh thu bán hàng 20 ngày được lựa chọn ngẫu nhiên của một cửa hàng được ghi lại ở bảng sau (đơn vị: triệu đồng)
% 	\begin{center}
% 		\begin{tabular}{|l|c|c|c|c|c|c|}
% 			\hline Doanh thu &{$[5;7)$}&{$[7;9)$}&{$[9;11)$}&{$[11;13)$}&{$[13;15)$}\\
% 			\hline Số ngày & $2$ & $9$ & $7$ & $3$ & $1$ \\
% 			\hline
% 		\end{tabular}
% 	\end{center}
% 	Xác định mốt của mẫu số liệu.
% 	\choice
% 	{ $M_0\approx 8$}
% 	{$M_0\approx 8,5$}
% 	{\True $M_0\approx 8,56$}
% 	{$M_0\approx 9$}
% 	\loigiai{
% 		Tần số lớn nhất là $9$ nên nhóm chứa mốt là nhóm $[7;9)$. \\
% 		Ta có, $j=2, a_2=7, m_2=9$, $m_1=2, m_3=7, h=2$. Do đó
% 		$$
% 		M_0=7+\frac{9-2}{(9-2)+(9-7)}\cdot 2\approx 8,56.
% 		$$
% 	}    
% \end{ex}
% %9
% \begin{ex}%[1K3B9-4]
% 	Điểm kiểm tra môn Toán của lớp 12A được cho trong bảng sau
% 	\begin{center}
% 		\begin{tabular}{|l|c|c|c|c|c|c|c|c|}
% 			\hline Khoảng điểm &{$[6,5;7)$}&{$[7;7,5)$}&{$[7,5;8)$}&{$[8;8,5)$}&{$[8,5;9)$}&{$[9;9,5)$}&{$[9,5;10)$}\\
% 			\hline Tần số & $8$ & $10$ & $16$ & $24$& $13$ & $7$ & $4$ \\
% 			\hline
% 		\end{tabular}
% 	\end{center}
% 	Xác định mốt của mẫu số liệu ghép nhóm này.    
% 	\choice
% 	{ $M_0\approx 8,4$}
% 	{$M_0\approx 8,5$}
% 	{\True $M_0\approx 8,21$}
% 	{$M_0\approx 24$}
% 	\loigiai{
% 		Tần số lớn nhất là $24$ nên nhóm chứa mốt là nhóm $[8;8,5)$. \\
% 		Ta có, $j=4, a_4=8, m_4=24$, $m_3=16, m_5=13, h=0,5$. Do đó
% 		$$
% 		M_0=8+\frac{24-16}{(24-16)+(24-13)}\cdot 0,5\approx 8,21.
% 		$$
% 	}
% \end{ex}
% %10
% \begin{ex}%[1K3Y9-4]
% 	Điểm kiểm tra môn Toán của lớp 12A được cho trong bảng sau
% 	\begin{center}
% 		\begin{tabular}{|l|c|c|c|c|c|c|c|c|}
% 			\hline Khoảng điểm &{$[6,5;7)$}&{$[7;7,5)$}&{$[7,5;8)$}&{$[8;8,5)$}&{$[8,5;9)$}&{$[9;9,5)$}&{$[9,5;10)$}\\
% 			\hline Tần số & $8$ & $10$ & $16$ & $24$& $13$ & $7$ & $4$ \\
% 			\hline
% 		\end{tabular}
% 	\end{center}
% 	Nhóm chứa mốt của mẫu số liệu này là
% 	\choice
% 	{ $[7;7,5)$}
% 	{$[7,5;8)$}
% 	{\True $[8;8,5)$}
% 	{$[8,5;9)$}
% 	\loigiai{
% 		Tần số lớn nhất là $24$ nên nhóm chứa mốt là nhóm $[8,5;9)$. \\
% 	}       
% \end{ex}
% %11
% \begin{ex}%[1K3Y9-4]
% 	Để kiểm tra thời gian sử dụng pin của   chiếc điện thoại mới, bạn A thống kê thời gian sử dụng điện thoại của mình từ lúc sạc đầy cho đến khi hết pin ở bảng sau
% 	\begin{center}
% 		\begin{tabular}{|l|c|c|c|c|c|c|c|c|}
% 			\hline Thời gian sử dụng (giờ) &{$[7;9)$}&{$[9;11)$}&{$[11;13)$}&{$[13;15)$}&{$[15;17)$}\\
% 			\hline Số lần & $2$ & $5$ & $7$ & $6$& $3$  \\
% 			\hline
% 		\end{tabular}
% 	\end{center}
% 	Nhóm chứa mốt của mẫu số liệu này là
% 	\choice
% 	{ $[9;11)$}
% 	{\True $[11;13)$}
% 	{ $[13;15)$}
% 	{$[15;17)$}
% 	\loigiai{
% 		Tần số lớn nhất là $7$ nên nhóm chứa mốt là nhóm $[11;13)$. \\
% 	}        
% \end{ex}
% %12
% \begin{ex}%[1K3B9-4]
% 	Để kiểm tra thời gian sử dụng pin của   chiếc điện thoại mới, bạn A thống kê thời gian sử dụng điện thoại của mình từ lúc sạc đầy cho đến khi hết pin ở bảng sau
% 	\begin{center}
% 		\begin{tabular}{|l|c|c|c|c|c|c|c|c|}
% 			\hline Thời gian sử dụng (giờ) &{$[7;9)$}&{$[9;11)$}&{$[11;13)$}&{$[13;15)$}&{$[15;17)$}\\
% 			\hline Số lần & $2$ & $5$ & $7$ & $6$& $3$  \\
% 			\hline
% 		\end{tabular}
% 	\end{center}   
% 	Xác định mốt của mẫu số liệu ghép nhóm này.    
% 	\choice
% 	{ $M_0\approx 11,67$}
% 	{$M_0\approx 12$}
% 	{\True $M_0\approx 12,33$}
% 	{$M_0\approx 7$}
% 	\loigiai{
% 		Tần số lớn nhất là $7$ nên nhóm chứa mốt là nhóm $[11;13)$. \\
% 		Ta có, $j=3, a_3=11, m_3=7$, $m_2=5, m_4=6, h=2$. Do đó
% 		$$
% 		M_0=11+\frac{7-5}{(7-5)+(7-6)}\cdot 2\approx 12,33.
% 		$$
% 	}
% \end{ex}
% %13
% \begin{ex}%[1K3Y9-4]
% 	Tổng lượng mưa trong tháng 8 đo được tại một trạm quan trắc đặt tại Vũng Tàu từ năm 2002 đến năm 2020 được ghi lại như sau (đơn vị: mm)
% 	\begin{center}
% 		\begin{tabular}{|l|c|c|c|c|c|c|c|c|}
% 			\hline Tổng lượng mưa trong tháng 8 (mm) &{$[120;175)$}&{$[175;230)$}&{$[230;285)$}&{$[285;340)$}\\
% 			\hline Số năm & $10$ & $5$ & $3$ & $1$  \\
% 			\hline
% 		\end{tabular}
% 	\end{center}  
% 	Nhóm chứa mốt của mẫu số liệu này là
% 	\choice
% 	{ $[175;230)$}
% 	{ $[230;285)$}
% 	{\True $[120;175)$}
% 	{$[285;340)$}
% 	\loigiai{
% 		Tần số lớn nhất là $10$ nên nhóm chứa mốt là nhóm $[120;175)$. \\
% 	}        
% \end{ex}
% %14
% \begin{ex}%[1K3B9-4]
% 	Tổng lượng mưa trong tháng 8 đo được tại một trạm quan trắc đặt tại Vũng Tàu từ năm 2002 đến năm 2020 được ghi lại như sau (đơn vị: mm)
% 	\begin{center}
% 		\begin{tabular}{|l|c|c|c|c|c|c|c|c|}
% 			\hline Tổng lượng mưa trong tháng 8 (mm) &{$[120;175)$}&{$[175;230)$}&{$[230;285)$}&{$[285;340)$}\\
% 			\hline Số năm & $10$ & $5$ & $3$ & $1$  \\
% 			\hline
% 		\end{tabular}
% 	\end{center}  
% 	Xác định mốt của mẫu số liệu ghép nhóm này.    
% 	\choice
% 	{ $M_0\approx 172,25$}
% 	{$M_0\approx 146,125$}
% 	{\True $M_0\approx 156,67$}
% 	{$M_0\approx 10$}
% 	\loigiai{
% 		Tần số lớn nhất là $10$ nên nhóm chứa mốt là nhóm $[120;175)$. \\
% 		Ta có, $j=1, a_1=120, m_1=10$, $m_2=5, m_0=0, h=55$. Do đó
% 		$$
% 		M_0=120+\frac{10-0}{(10-0)+(10-5)}\cdot 55\approx 156.67.
% 		$$
% 	}
% \end{ex}
% %15
% \begin{ex}%[1K3B9-4]
% 	Một công ty xây dựng khảo sát khách hàng xem họ có nhu cầu mua nhà ở mức giá nào. Kết quả khảo sát được ghi lại ở bảng sau (đơn vị: triệu đồng/$\mathrm{m}^2$
% 	\begin{center}
% 		\begin{tabular}{|l|c|c|c|c|c|c|c|c|}
% 			\hline Mức giá &{$[10;14)$}&{$[14;18)$}&{$[18;22)$}&{$[22;26)$}&{$[26;30)$}\\
% 			\hline Số khách hàng & $54$ & $78$ & $120$ & $45$& $12$  \\
% 			\hline
% 		\end{tabular}
% 	\end{center}   
% 	Xác định mốt của mẫu số liệu ghép nhóm này.    
% 	\choice
% 	{ $M_0\approx 18$}
% 	{\True $M_0\approx 19,4$}
% 	{ $M_0\approx 20$}
% 	{$M_0\approx 120$}
% 	\loigiai{
% 		Tần số lớn nhất là $120$ nên nhóm chứa mốt là nhóm $[18;22)$. \\
% 		Ta có, $j=3, a_3=18, m_3=120$, $m_2=78, m_4=45, h=4$. Do đó
% 		$$
% 		M_0=18+\frac{120-78}{(120-78)+(120-45)}\cdot 4\approx 19,4.
% 		$$
% 	}
% \end{ex}
% \begin{ex}%[1K3K9-4]
% 	Một công ty xây dựng khảo sát khách hàng xem họ có nhu cầu mua nhà ở mức giá nào. Kết quả khảo sát được ghi lại ở bảng sau (đơn vị: triệu đồng/$\mathrm{m}^2$
% 	\begin{center}
% 		\begin{tabular}{|l|c|c|c|c|c|c|c|c|}
% 			\hline Mức giá &{$[10;14)$}&{$[14;18)$}&{$[18;22)$}&{$[22;26)$}&{$[26;30)$}\\
% 			\hline Số khách hàng & $54$ & $78$ & $120$ & $45$& $12$  \\
% 			\hline
% 		\end{tabular}
% 	\end{center}   
% 	Công ty nên xây nhà ở mức giá nào để nhiều người có nhu cầu mua nhất?    
% 	\choice
% 	{ $ 18$ triệu đồng/$\mathrm{m}^2$}
% 	{\True $19,4$ triệu đồng/$\mathrm{m}^2$}
% 	{ $20$ triệu đồng/$\mathrm{m}^2$}
% 	{$21$ triệu đồng/$\mathrm{m}^2$}
% 	\loigiai{
% 		Tần số lớn nhất là $120$ nên nhóm chứa mốt là nhóm $[18;22)$. \\
% 		Ta có, $j=3, a_3=18, m_3=120$, $m_2=78, m_4=45, h=4$. Do đó
% 		$$
% 		M_0=18+\frac{120-78}{(120-78)+(120-45)}\cdot 4\approx 19,4.
% 		$$
% 		Dựa vào kết quả trên ta dự đoán rằng nếu công ty xây nhà ở mức giá $19,4$ triệu đồng/$\mathrm{m}^2$ thì sẽ có nhiều người có nhu cầu mua nhất.
% 	}
% \end{ex}
% \begin{ex}%[1K3Y9-4]
% 	Số cuộc gọi điện thoại một người thực hiện mỗi ngày trong 30 ngày được lựa chọn ngẫu nhiên được thống kê trong bảng sau
% 	\begin{center}
% 		\begin{tabular}{|l|c|c|c|c|c|c|c|c|}
% 			\hline Số cuộc gọi &{$[2,5;5,5)$}&{$[5,5;8,5)$}&{$[8,5;11,5)$}&{$[11,5;14,5)$}&{$[14,5;17,5)$}\\
% 			\hline Số ngày & $5$ & $13$ & $7$ & $3$& $2$  \\
% 			\hline
% 		\end{tabular}
% 	\end{center}   
% 	Nhóm chứa mốt của mẫu số liệu này là
% 	\choice
% 	{ $[2,5;5,5)$}
% 	{\True $[5,5;8,5)$}
% 	{ $[8,5;11,5)$}
% 	{$[11,5;14,5)$}
% 	\loigiai{
% 		Tần số lớn nhất là $13$ nên nhóm chứa mốt là nhóm $[5,5;8,5)$. \\
% 	}
% \end{ex}
% \begin{ex}%[1K3K9-4]
% 	Số cuộc gọi điện thoại một người thực hiện mỗi ngày trong 30 ngày được lựa chọn ngẫu nhiên được thống kê trong bảng sau
% 	\begin{center}
% 		\begin{tabular}{|l|c|c|c|c|c|c|c|c|}
% 			\hline Số cuộc gọi &{$[2,5;5,5)$}&{$[5,5;8,5)$}&{$[8,5;11,5)$}&{$[11,5;14,5)$}&{$[14,5;17,5)$}\\
% 			\hline Số ngày & $5$ & $13$ & $7$ & $3$& $2$  \\
% 			\hline
% 		\end{tabular}
% 	\end{center}   
% 	Hãy dự đoán xem khả năng người đó thực hiện bao nhiêu cuộc gọi mỗi ngày là cao nhất?   
% 	\choice
% 	{ $5$}
% 	{\True $7$}
% 	{ $6$}
% 	{$8$}
% 	\loigiai{
% 		Tần số lớn nhất là $13$ nên nhóm chứa mốt là nhóm $[5,5;8,5)$. \\
% 		Ta có, $j=2, a_2=5,5, m_2=13$, $m_1=5, m_3=7, h=3$. Do đó
% 		$$
% 		M_0=5,5+\frac{13-5}{(13-5)+(13-7)}\cdot 3\approx 7,2.
% 		$$
% 		Do đó ta có thể dự đoán khả năng người đó thực hiện $7$ cuộc gọi mỗi ngày là cao nhất.
% 	}
% \end{ex}

\Closesolutionfile{ans}

%GK1
% \begin{name}
	{\tenchude}
	{TOÁN 11}
	{LỚP TOÁN THẦY PHÁT}
	{Thời gian: 90 phút - Không kể thời gian phát đề}
\end{name}
\setcounter{ex}{0}\setcounter{bt}{0}
\noindent{\bf\fontfamily{qag}\selectfont\color{violet}A. PHẦN TRẮC NGHIỆM}
\Opensolutionfile{ans}[ans/ans-1-GK1-KNTT-14-NH23-24]
\begin{ex}%[0H3B6-2]% ::Cau 1::
	Với góc $\alpha $ bất kì, đẳng thức nào sau đây là đúng?
	\choice
	{$\cos \left( \pi-\alpha \right)=\cos \alpha $}
	{\True $\cos \left( \pi-\alpha \right)=-\cos \alpha $}
	{$\sin \left( \pi-\alpha \right)=-\sin \alpha $}
	{$\tan \left( \pi-\alpha \right)=\tan \alpha $}
	\loigiai{
		Ta có $\cos \left( \pi-\alpha \right)=-\cos \alpha $, $\sin \left( \pi-\alpha \right)=\sin \alpha $, $\tan \left( \pi-\alpha \right)=-\tan \alpha $.\\
		Do đó ta chọn phương án $\cos \left( \pi-\alpha \right)=-\cos \alpha $.}
\end{ex}
\begin{ex}%[0H3B5-2]% ::Cau 2::
	Biết góc $\alpha $ thỏa mãn $\cos \alpha =\dfrac{2}{3}$. Hỏi $\alpha $ có thể nhận giá trị trong khoảng nào dưới đây?
	\choice
	{$\left( \dfrac{\pi}{2};\dfrac{2\pi}{3} \right)$}
	{$\left( \dfrac{8\pi}{3};\dfrac{17\pi}{6} \right)$}
	{\True $\left( \dfrac{\pi}{4};\dfrac{\pi}{3} \right)$}
	{$\left( -\pi;-\dfrac{2\pi}{3} \right)$}
	\loigiai{
		Vì $\cos \alpha =\dfrac{2}{3}$ nên $\alpha \in \left( -\dfrac{\pi}{2}+k2\pi,\dfrac{\pi}{2}+k2\pi \right)$ với $k\in \mathbb{Z}$.\\
		Với $k=0$ thì $\alpha\in \left(-\dfrac{\pi}{2};\dfrac{\pi}{2}\right)$. Vì $\left( \dfrac{\pi}{4};\dfrac{\pi}{3} \right)\subset \left(-\dfrac{\pi}{2};\dfrac{\pi}{2}\right)$.\\
		Do đó, ta chọn phương án $\left( \dfrac{\pi}{4};\dfrac{\pi}{3} \right)$.}
\end{ex}
\begin{ex}%[0H3Y5-2]% ::Cau 3::
	Cho góc $\alpha $ thỏa $-\dfrac{3\pi}{2}<\alpha <-\pi$. Tìm mệnh đề đúng trong các mệnh đề sau:
	\choice
	{$\cos \alpha >0$}
	{$\cot \alpha >0$}
	{\True $\sin \alpha >0$}
	{$\tan \alpha >0$}
	\loigiai{
		\begin{center}
			\begin{tikzpicture}[scale=1.2,font=\footnotesize,line join=round,line cap=round,>=stealth]
				\path 	
				(0,0) coordinate (O)
				(1,0) coordinate (A)
				(-1,0) coordinate (A')
				(0,1) coordinate (B)
				(0,-1) coordinate (B')
				;		
				\draw[-stealth] (-1.5,0)--(1.5,0)node[below]{$\cos x$};
				\draw[-stealth] (0,-1.5)--(0,1.5)node[left]{$\sin x$};
				\draw (O) node[below left]{$O$}  circle (1);
				\draw (1,1) node{I} (-1,1) node{II} (-1,-1) node{III} (1,-1)node{IV} ;
%				\foreach \x/\g in {A/30,B/45,A'/120,B'/-30}
%				\fill[black] 	(\x) circle (1pt)
%				($(\g:3mm)+(\x)$) node {$\x$};
			\end{tikzpicture}
		\end{center}
		Do $-\dfrac{3\pi}{2}<\alpha <-\pi$ nên điểm $M$ biểu diễn góc lượng giác có số đo $\alpha $ thuộc góc phần tư số II. \\
		Do đó $\sin \alpha >0$, $\cos \alpha <0$, $\tan \alpha <0$, $\cot \alpha <0$.}
\end{ex}
\begin{ex}%[0H3B5-2]% ::Cau 4::
	Cho $\cot \alpha =4\tan \alpha $ và $\alpha \in \left( \dfrac{\pi}{2};\pi \right)$. Khi đó $\sin \alpha $ bằng
	\choice
	{$-\dfrac{\sqrt{5}}{5}$}
	{$\dfrac{1}{2}$}
	{$\dfrac{2\sqrt{5}}{5}$}
	{\True $\dfrac{\sqrt{5}}{5}$}
	\loigiai{
		Ta có 
		\allowdisplaybreaks
		\begin{eqnarray*}
		&& \cot \alpha =4\tan \alpha \\
		&\Leftrightarrow& \dfrac{\cot \alpha }{\tan \alpha }=4\\
		&\Leftrightarrow& \cot^2 \alpha =4\Leftrightarrow 1+\cot^2 \alpha =5\\
		&\Leftrightarrow& \dfrac{1}{{\sin ^2}\alpha }=5\\
		&\Leftrightarrow& {\sin ^2}\alpha =\dfrac{1}{5}\\
		&\Leftrightarrow& \sin \alpha =\pm \dfrac{\sqrt{5}}{5}.
		\end{eqnarray*}
		Vì $\alpha \in \left( \dfrac{\pi}{2};\pi \right)$ nên $\sin \alpha =\dfrac{\sqrt{5}}{5}$.}
\end{ex}
\begin{ex}%[0H3Y6-2]% ::Cau 5::
	Khẳng định nào sau đây là \textbf{sai}?
	\choice
	{$\cos 2a=2\cos^2 a-1$}
	{$\cos 2a=\cos^2 a-{\sin ^2}a$}
	{$\sin 2a=2\sin a\cos a$}
	{\True $\cos 2a=2{\sin ^2}a-1$}
	\loigiai{
		Theo công thức nhân đôi ta có $\cos 2a=1-2{\sin ^2}a$.}
\end{ex}
\begin{ex}%[0H3Y6-2]% ::Cau 6::
	Khẳng định nào sau đây là \textbf{sai}?
	\choice
	{$\cos a+\cos b=2\cos \dfrac{a+b}{2}\cos \dfrac{a-b}{2}$}
	{\True $\cos a-\cos b=2\sin \dfrac{a+b}{2}\sin \dfrac{a-b}{2}$}
	{$\sin a+\sin b=2\sin \dfrac{a+b}{2}\cos \dfrac{a-b}{2}$}
	{$\sin a-\sin b=2\cos \dfrac{a+b}{2}\sin \dfrac{a-b}{2}$}
	\loigiai{
		Theo công thức biến tổng thành tích ta có $\cos a-\cos b=-2\sin \dfrac{a+b}{2}\cdot\sin \dfrac{a-b}{2}$.}
\end{ex}
\begin{ex}%[0H3B5-2]% ::Cau 7::
	Cho $\tan \alpha +\cot \alpha =m$. Tính giá trị của biểu thức $\tan^3 \alpha +\cot^3 \alpha $
	\choice
	{$m^3+3m$}
	{$3m^3+m$}
	{$3m^3-m$}
	{\True $m^3-3m$}
	\loigiai{
		Ta có
		\allowdisplaybreaks
		\begin{eqnarray*}
		\tan^3 \alpha +\cot^3 \alpha &=& \left( \tan \alpha +\cot \alpha \right)\left( \tan^2 \alpha -\tan \alpha \cdot\cot \alpha +\cot^2 \alpha \right)\\
		&=& \left( \tan \alpha +\cot \alpha \right)\left[ \left( \tan \alpha +\cot \alpha \right)^2-3\tan \alpha \cdot\cot \alpha \right]\\
		&=& \left( \tan \alpha +\cot \alpha \right)\cdot\left[ \left( \tan \alpha +\cot \alpha \right)^2-3 \right]\\
		&=& m\left( m^2-3 \right)=m^3-3m.
		\end{eqnarray*}
	}
\end{ex}
\begin{ex}%[1D1Y1-1]% ::Cau 8::
	Tìm tập xác định $\mathscr{D}$ của hàm số $y=\dfrac{2023}{\sin x}$.
	\choice
	{$\mathscr{D}=\mathbb{R}\setminus \left\{\dfrac{\pi}{2}+k\pi\,\middle|\, k\in \mathbb{Z} \right\}$}
	{$\mathscr{D}=\mathbb{R}\setminus \left\{\dfrac{k\pi}{2} \,\middle|\, k\in \mathbb{Z} \right\}$}
	{$\mathscr{D}=\mathbb{R}\setminus \left\{0 \right\}$}
	{\True $D=\mathbb{R}\setminus \left\{k\pi\,\middle|\, k\in \mathbb{Z} \right\}$}
	\loigiai{
		Điều kiện xác định $\sin x\ne 0\Leftrightarrow x\ne k\pi,k\in \mathbb{Z}$.\\
	Vậy $\mathscr{D}=\mathbb{R}\setminus \left\{k\pi\,\middle|\, k\in \mathbb{Z}  \right\}$.}
\end{ex}
\begin{ex}%[1D1Y1-3]% ::Cau 9::
	Cho hàm số $y=\tan x$. Khẳng định sau đây là \textbf{sai}?
	\choice
	{\True Hàm số đã cho là hàm số chẵn}
	{Tập xác định của hàm số đã cho là $\mathbb{R}\setminus \left\{\dfrac{\pi}{2}+k\pi\,\middle|\, k\in \mathbb{Z} \right\}$}
	{Hàm số đã cho đồng biến trên mỗi khoảng $\left( -\dfrac{\pi}{2}+k\pi;\dfrac{\pi}{2}+k\pi \right)$ với $k\in \mathbb{Z}$}
	{Hàm số đã cho tuần hoàn theo chu kì $\pi$}
	\loigiai{
	Hàm số $y=\tan x$ là hàm số lẻ.
	}
\end{ex}
\begin{ex}%[1D1Y1-6]% ::Cau 10::
	Trong các hàm số sau, hàm số nào có đồ thị như hình vẽ bên dưới?
	\begin{center}
		\begin{tikzpicture}[scale=1,font=\footnotesize,line join=round,line cap=round,>=stealth]		
			\draw[-stealth] (-6.7,0)--(6.7,0)node[below]{$x$};
			\draw[-stealth] (0,-1.2)--(0,2)node[left]{$y$};
			\draw (0,0) node[below right]{$O$};
			\draw[thick,smooth,samples=200] plot[domain=-6.6:6.6] (\x,{sin (\x*180/pi)});
			\draw[fill=black]  (-6.28,0) circle (1pt) node [above left] {$-2\pi$};
			\draw[fill=black]  (-4.71,0) circle (1pt) node [below] {$-\dfrac{3\pi}{2}$};
			\draw[fill=black]  (-3.14,0) circle (1pt) node [above] {$-\pi$};
			\draw[fill=black] (-1.57,0) circle (1pt) node [above] {$-\dfrac{\pi}{2}$};
			\draw[fill=black]  (1.57,0) circle (1pt) node [below] {$\dfrac{\pi}{2}$};
			\draw[fill=black]  (3.14,0) circle (1pt) node [above] {$\pi$};
			\draw[fill=black]  (4.71,0) circle (1pt) node [above] {$\dfrac{3\pi}{2}$};
			\draw[fill=black]  (6.28,0) circle (1pt) node [above] {$2\pi$};
			\draw (0,1) node[above right] {$1$} (0,-1) node[below right] {$-1$};
			\draw[dashed] (-6.7,-1)--(6.7,-1) (-6.7,1)--(6.7,1);
			\draw[dashed] (-1.57,0)--(-1.57,-1) (1.57,0)--(1.57,1) (4.71,0)--(4.71,-1) (-4.71,0)--(-4.71,1);
		\end{tikzpicture}
	\end{center}
	\choice
	{\True $y=\sin x$}
	{$y=\cos x$}
	{$y=\tan x$}
	{$y=\cot x$}
	\loigiai{
		Từ hình vẽ ta thấy hàm số có miền giá trị từ $-1$ đến $1$, tuần hoàn với chu kỳ $2\pi$ và nhận gốc tọa độ làm tâm đối xứng nên đây là đồ thị của hàm số $y=\sin x$.}
\end{ex}
\begin{ex}%[1D1B1-4]% ::Cau 11::
	Tìm chu kì $T$ của hàm số $y=2\cos^2 x+2023$.
	\choice
	{$T=3\pi$}
	{$T=2\pi$}
	{\True $T=\pi$}
	{$T=4\pi$}
	\loigiai{
		Ta có $y=2\cos^2 x+2023=\cos 2x+2024$.\\
		Vậy hàm số có chu kì $T=\pi$.}
\end{ex}
\begin{ex}%[1D1K1-1]% ::Cau 12::
	Tập xác định $\mathscr{D}$ của hàm số $y=\sqrt{4+\sin x}-\dfrac{1+x}{\tan^2 \left( x-\dfrac{\pi}{4} \right)-1}+3\tan \left( x+\dfrac{\pi}{4} \right)$ là
	\choice
	{$\mathscr{D}=\mathbb{R}\setminus \left\{\dfrac{\pi}{4}+k\dfrac{\pi}{2},k\in \mathbb{Z} \right\}$}
	{$\mathscr{D}=\mathbb{R}\setminus \left\{k\dfrac{\pi}{2},k\in \mathbb{Z} \right\}$}
	{\True $\mathscr{D}=\mathbb{R}\setminus \left\{k\dfrac{\pi}{4},k\in \mathbb{Z} \right\}$}
	{$\mathscr{D}=\mathbb{R}\setminus \left\{\dfrac{\pi}{4}+k\pi,k\in \mathbb{Z} \right\}$}
	\loigiai{
		Do $-1\le \sin x\le 1$ nên $4+\sin x>0$, $\,\forall x\in \mathbb{R}$.\\
		Hàm số xác định khi và chỉ khi\\
		$\heva{
			& x-\dfrac{\pi}{4}\ne \dfrac{\pi}{2}+m\pi\\
			& \tan^2 \left( x-\dfrac{\pi}{4} \right)\ne 1 \\
			& x+\dfrac{\pi}{4}\ne \dfrac{\pi}{2}+q\pi\\
		}\Leftrightarrow \heva{
			& x\ne \dfrac{3\pi}{4}+m\pi\\
			& x-\dfrac{\pi}{4}\ne \dfrac{\pi}{4}+n\pi\\
			& x-\dfrac{\pi}{4}\ne -\dfrac{\pi}{4}+p\pi\\
			& x\ne \dfrac{\pi}{4}+q\pi\\
		}\Leftrightarrow \heva{
			& x\ne \dfrac{3\pi}{4}+m\pi\\
			& x\ne \dfrac{\pi}{2}+n\pi\\
			& x\ne p\pi\\
			& x\ne \dfrac{\pi}{4}+q\pi\\
		}\Leftrightarrow x\ne k\dfrac{\pi}{4},\,\left( m,n,p,q,k\in \mathbb{Z} \right)$.\\
		Vậy tập xác định $\mathscr{D}=\mathbb{R}\setminus \left\{k\dfrac{\pi}{4},k\in \mathbb{Z} \right\}$.}
\end{ex}
\begin{ex}%[1D1B2-1]% ::Cau 13::
	Giải phương trình $\cos \left( x-\dfrac{\pi}{6} \right)=\dfrac{1}{2}$.
	\choice
	{$\hoac{
			& x=\dfrac{\pi}{3}+k2\pi\\
			& x=k2\pi\\
		}\,\left( k\in \mathbb{Z} \right)$}
	{$\hoac{
			& x=\dfrac{\pi}{2}+k\pi\\
			& x=-\dfrac{\pi}{6}+k\pi\\
		}\left( k\in \mathbb{Z} \right)$}
	{$\hoac{
			& x=\dfrac{\pi}{2}+k2\pi\\
			& x=\dfrac{\pi}{6}+k2\pi\\
		}\left( k\in \mathbb{Z} \right)$}
	{\True $\hoac{
			& x=\dfrac{\pi}{2}+k2\pi\\
			& x=-\dfrac{\pi}{6}+k2\pi\\
		}\left( k\in \mathbb{Z} \right)$}
	\loigiai{
		Ta có $\cos \left( x-\dfrac{\pi}{6} \right)=\dfrac{1}{2}\Leftrightarrow \hoac{
			& x-\dfrac{\pi}{6}=\dfrac{\pi}{3}+k2\pi\\
			& x-\dfrac{\pi}{6}=-\dfrac{\pi}{3}+k2\pi\\
		}\Leftrightarrow \hoac{
			& x=\dfrac{\pi}{2}+k2\pi\\
			& x=-\dfrac{\pi}{6}+k2\pi\\
		}\left( k\in \mathbb{Z} \right)$.}
\end{ex}
\begin{ex}%[1D1B2-1]% ::Cau 14::
	Phương trình $\sin 2x=-\dfrac{1}{2}$ có tập nghiệm là
	\choice
	{$\hoac{
			& x=\dfrac{7\pi}{12}+k\pi\\
			& x=-\dfrac{7\pi}{12}+k\pi\\
		}\left( k\in \mathbb{Z} \right)$}
	{$\hoac{
			& x=\dfrac{\pi}{12}+k2\pi\\
			& x=-\dfrac{7\pi}{12}+k2\pi\\
		}\left( k\in \mathbb{Z} \right)$}
	{$\hoac{
			& x=\dfrac{\pi}{12}+k\pi\\
			& x=\dfrac{7\pi}{12}+k\pi\\
		}\left( k\in \mathbb{Z} \right)$}
	{\True $\hoac{
			& x=-\dfrac{\pi}{12}+k\pi\\
			& x=\dfrac{7\pi}{12}+k\pi\\
		}\left( k\in \mathbb{Z} \right)$}
	\loigiai{
		Ta có $\sin 2x=-\dfrac{1}{2}\Leftrightarrow \hoac{
			& 2x=-\dfrac{\pi}{6}+k2\pi\\
			& 2x=\dfrac{7\pi}{6}+k2\pi\\
		}\Leftrightarrow \hoac{
			& x=-\dfrac{\pi}{12}+k\pi\\
			& x=\dfrac{7\pi}{12}+k\pi\\
		}\left( k\in \mathbb{Z} \right)$.}
\end{ex}
\begin{ex}%[1D1B2-1]% ::Cau 15::
	Phương trình $\sin \left( 2x-\dfrac{\pi}{4} \right)=\sin \left( x+\dfrac{3\pi}{4} \right)$ có tổng các nghiệm thuộc khoảng $\left( 0;\pi \right)$ bằng
	\choice
	{$\dfrac{7\pi}{2}$}
	{\True $\pi$}
	{$\dfrac{3\pi}{2}$}
	{$\dfrac{\pi}{4}$}
	\loigiai{
		Ta có $$\sin \left( 2x-\dfrac{\pi}{4} \right)=\sin \left( x+\dfrac{3\pi}{4} \right)\Leftrightarrow \hoac{
			& 2x-\dfrac{\pi}{4}=x+\dfrac{3\pi}{4}+k2\pi\\
			& 2x-\dfrac{\pi}{4}=\dfrac{\pi}{4}-x+l2\pi}
		\Leftrightarrow \hoac{
			& x=\pi+k2\pi\\
			& x=\dfrac{\pi}{6}+l\dfrac{2\pi}{3}},\left( k,l\in \mathbb{Z} \right).$$
		Họ nghiệm $x=\pi+k2\pi$ không có nghiệm nào thuộc khoảng $\left( 0;\pi \right)$.\\
		Với $x=\dfrac{\pi}{6}+l\dfrac{2\pi}{3}\in \left( 0;\pi \right)\Rightarrow 0<\dfrac{\pi}{6}+l\dfrac{2\pi}{3}<\pi\Leftrightarrow l\in \left\{0;1 \right\}$.\\
		Vậy phương trình có hai nghiệm thuộc khoảng $\left( 0;\pi \right)$ là $x=\dfrac{\pi}{6}$ và $x=\dfrac{5\pi}{6}$. \\
		Từ đó suy ra tổng các nghiệm thuộc khoảng $\left( 0;\pi \right)$ của phương trình này bằng $\pi$.}
\end{ex}
\begin{ex}
	Khai triển $\cos 4 \alpha$ theo $\cos \alpha$ ta được biểu thức $a\cos^4 \alpha +b\cos^2 \alpha +c$. Giá trị biểu thức $a-b+c$ bằng
	\choice
	{$13$}
	{\True $17$}
	{$-11$}
	{$-15$}
	\loigiai{
Ta có $\cos 4 \alpha =2\cos^2 2\alpha -1= 2 \left(2\cos^2 \alpha -1\right)^2 -1 = 8\cos^4 \alpha -8\cos^2 \alpha +1$.
}
\end{ex}
\begin{ex}%[1K1K4-3]%Câu 27
    Cho hai phương trình $\cos 3x-1=0$; $\cos 2x=-\dfrac{1}{2}$. Các giá trị nào dưới đây là nghiệm chung của hai phương trình đã cho?
    \choice
        {$x=\dfrac{\pi }{3}+k2\pi, k\in \mathbb{Z}$}
        {$x=k2\pi, k\in \mathbb{Z}$}
        {$x=\pm \dfrac{\pi }{3}+k2\pi, k\in \mathbb{Z}$}
        {\True $x=\pm \dfrac{2\pi }{3}+k2\pi, k\in \mathbb{Z}$}
    \loigiai{
        Ta có 
        \begin{itemize}
            \item $\cos 3x-1=0\Leftrightarrow \cos 3x=1
            \Leftrightarrow x=k\dfrac{2\pi }{3}, k\in \mathbb{Z}$.
            \item $\cos 2x=-\dfrac{1}{2}
            \Leftrightarrow 2x=\pm \dfrac{2\pi }{3}+k2\pi \Leftrightarrow x=\pm \dfrac{\pi }{3}+k\pi,k\in \mathbb{Z}.$
        \end{itemize}
        Biểu diễn các nghiệm trên đường tròn lượng giác ta có tập các nghiệm của hai phương trình là $x=\pm \dfrac{2\pi }{3}+k\pi, k\in \mathbb{Z}$.}
\end{ex}
\begin{ex}%[1K1B2-3]%Câu 21
    Biểu thức $\dfrac{\sin 10^\circ +\sin 20^\circ}{\cos 10^\circ +\cos 20^\circ}$ bằng $a\tan b$ với $a,b\in \mathbb{N}$ và $b\in [0;180]$. Tính $a+b$. 
    \choice
        {$88$}
        {$69$}
        {$29$}
        {\True $16$}
    \loigiai{
        $\dfrac{\sin 10^\circ+\sin 20^\circ}{\cos 10^\circ+\cos 20^\circ}=\dfrac{2\sin 15^\circ\cos 5^\circ}{2\cos 15^\circ\cos 5^\circ}=\tan 15^\circ$.}
\end{ex}
\begin{ex}%[1D3Y4-3]% ::Cau 16::
	Cho cấp số nhân $(u_n)$ có số hạng đầu $u_1=4$ và công bội $q=2$. Số hạng thứ $10$ của cấp số nhân đó là
	\choice
	{$u_{10}=2^{12}$}
	{\True $u_{10}=2^{11}$}
	{$u_{10}=2^{10}$}
	{$u_{10}=2^9$}
	\loigiai{
	Ta có $u_{10}=u_1\cdot q^{10-1}=4\cdot 2^9=2^2\cdot 2^9=2^{11}$.
	}
\end{ex}
\begin{ex}%[1D3B4-2]% ::Cau 17::
	Cho cấp số nhân $(u_n)$ với $u_1=-3;u_6=96$. Công bội của cấp số nhân đó là
	\choice
	{\True $q=-2$}
	{$q=-3$}
	{$q=2$}
	{$q=3$}
	\loigiai{
	Ta có $u_6=u_1q^5 \Rightarrow q^5=\dfrac{u_6}{u_1}=\dfrac{96}{-3}=-32$, suy ra $q=-2$.}
\end{ex}
\begin{ex}%::Cau 18::
	Công ty muốn ước lượng tỉ lệ các cỡ áo khi may cho học sinh lớp 11 đã đo chiều cao của 36 học sinh nam khối 11 của một trường và thu được mẫu số liệu sau (đơn vị là centimét):
	\begin{center}
		\begin{tabular}{lllllllllllll}
			160 & 161 & 161 & 162 & 162 & 162 & 163 & 163 & 163 & 164 & 164 & 164 & 164 \\ 
			165 & 165 & 165 & 165 & 165 & 166 & 166 & 166 & 166 & 167 & 167 & 168 & 168 \\ 
			168 & 168 & 169 & 169 & 170 & 171 & 171 & 172 & 172 & 174 & & & 
		\end{tabular}
	\end{center}
	Biết rằng học sinh có chiều cao thuộc $[160;167)$ sẽ mua cỡ áo M. Có bao nhiêu học sinh mua cỡ áo M?
	\choice
	{\True $22$}
	{$6$}
	{$15$}
	{$20$}
	\loigiai{
	Bảng tần số ghép nhóm
			\begin{center}
				\begin{tabular}{|c|c|c|c|c|c|}
					\hline Chiều cao $(\mathrm{cm})$ & {$[150 ; 160)$} & {$[160 ; 167)$} & {$[167 ; 170)$} & {$[170 ; 175)$} & {$[175 ; 180)$} \\
					\hline Số học sinh &0& 22 & 8 & 6 & 0 \\
					\hline
				\end{tabular}
			\end{center}
	}
\end{ex}
\begin{ex}%:Cau 19::
	Tìm tứ phân vị thứ nhất và thứ ba (làm tròn đến hàng phần chục) của mẫu số liệu sau 
	\begin{center}
		\begin{tabular}{|c|c|c|c|c|c|}
			\hline Chiều cao $(\mathrm{cm})$ & {$[150 ; 160)$} & {$[160 ; 167)$} & {$[167 ; 170)$} & {$[170 ; 175)$} & {$[175 ; 180)$} \\
			\hline Số học sinh &5& 17 & 8 & 6 & 0 \\
			\hline
		\end{tabular}
	\end{center}
	\choice
	{$Q_1 \approx 159{,}2$, $Q_3 \approx 169{,}8$}
	{\True $Q_1 \approx 161{,}6$, $Q_3 \approx 168{,}9$}
	{$Q_1 \approx 160{,}2$, $Q_3 \approx 170{,}3$}
	{$Q_1 \approx 163{,}6$, $Q_3 \approx 171{,}4$}
	\loigiai{
	\begin{itemize}
		\item $\dfrac{n}{4}=9$, $p=2$, $a_2=160$, $m_1=5$, $m_2=17$, $a_3-a_2=7$. $$Q_1=160 + \dfrac{9-5}{17} \cdot 7 \approx 161{,}6.$$
		\item $\dfrac{3n}{4}=27$, $p=3$, $a_3=167$, $m_1+m_2=22$, $m_3=8$, $a_4-a_3=3$. $$Q_3=167 + \dfrac{27-22}{8} \cdot 3 \approx 168{,}9$$
	\end{itemize}
	}
\end{ex}
\begin{ex}%[1D3Y2-2]% ::Cau 20::
	Cho dãy số $(u_n)$ với $u_n=\dfrac{n+1}{n-2}$. Tính $u_{20}$.
	\choice
	{\True $\dfrac{21}{18}$}
	{$\dfrac{18}{21}$}
	{$\dfrac{11}{8}$}
	{$\dfrac{8}{11}$}
	\loigiai{
		Ta có $u_{20}=\dfrac{20+1}{20-2}=\dfrac{21}{18}$.}
\end{ex}
\begin{ex}%[1D3B4-2]% ::Cau 21::
	Cho cấp số nhân có $u_1=2$ và $u_6=486$. Tìm công bội của cấp số nhân.
	\choice
	{$1$}
	{\True $3$}
	{$2$}
	{$4$}
	\loigiai{
		Ta có $u_6=u_1\cdot q^5\Rightarrow {q^5}=\dfrac{u_6}{u_1}=\dfrac{486}{2}=243=3^5\Rightarrow q=3$.}
\end{ex}
\begin{ex}%[0D5BC-1]% ::Cau 22::
	Cho mẫu số liệu ghép nhóm về thống kê chiều cao (mét) của $35$ cây bạch đàn trong rừng, ta có bảng số liệu sau:
	\begin{center}
		\begin{tabular}{|c|c|c|c|c|}
			\hline
		Khoảng chiều cao (m)	& $[6{,5};7)$ & $[7;7{,5})$ & $[7{,}5;8)$ & $[8;8{,}5)$ \\
			\hline
		Số cây	& $6$ & $15$ & $11$ & $3$ \\
			\hline
		\end{tabular}
	\end{center}
	Tính chiều cao trung bình của $35$ cây bạch đàn trên. (Kết quả làm tròn đến hàng phần nghìn).
	\choice
	{\True $7{,}407$ m}
	{$4{,}707$ m}
	{$7{,}704$ m}
	{$7{,}5$ m}
	\loigiai{
		Ta có giá trị đại diện các nhóm được cho dưới bảng sau:
		\begin{center}
			\begin{tabular}{|c|c|c|c|c|}
				\hline
				Khoảng chiều cao (m)	& $[6{,5};7)$ & $[7;7{,5})$ & $[7{,}5;8)$ & $[8;8{,}5)$ \\
				\hline
				Giá trị đại diện	& $6{,}75$	&$7{,}25$&$7{,}75$	&$8{,}25$	\\
				\hline
				Tần số (Số cây)	& $6$ & $15$ & $11$ & $3$ \\
				\hline
			\end{tabular}
		\end{center}
		Từ đó suy ra chiều cao trung bình của 35 cây bạch đàn là
		$$\overline{x}=\dfrac{6{,}75\cdot 6+7{,}25\cdot 15+7{,}75\cdot 11+8{,}25\cdot 3}{35}=7{,}047 \,\text{m}.$$}
\end{ex}
\begin{ex}%[0D5BC-3]% ::Cau 23::
	Tìm hiểu thời gian hoàn thành một bài tập (đơn vị: phút) của một số học sinh thu được kết quả sau:
	\begin{center}
		\begin{tabular}{|c|c|c|c|c|c|}
			\hline
		Thời gian (phút)	& $[0;4)$ & $[4;8)$ & $[8;12)$ & $[12;16)$ & $[16;20)$ \\
			\hline
		Số học sinh	& $2$ & $4$ & $7$ & $4$ & $3$ \\
			\hline
		\end{tabular}
	\end{center}
	Tứ phân vị thứ ba của mẫu số liệu ghép nhóm này là
	\choice
	{$Q_3=13$}
	{\True $Q_3=14$}
	{$Q_3=15$}
	{$Q_3=12$}
	\loigiai{
		Cỡ mẫu: $n=2+4+7+4+3=20$.\\
		Tứ phân vị thứ ba $Q_3$ là $\dfrac{x_{15}+x_{16}}{2}$. \\
		Do ${x_{15}},{x_{16}}$ đều thuộc nhóm $\left[ 12;16 \right)$ nên nhóm này chứa $Q_3$.\\
		Do đó $p=4$, $a_4=12$, $m_4=4$, $m_1+m_2+m_3=2+4+7=13$, $a_5-a_4=4$. \\
		Ta có $Q_3=12+\dfrac{\dfrac{3\cdot 20}{4}-13}{4}\cdot 4=14$.}
\end{ex}
\begin{ex}%[0D5YC-1]% ::Cau 24::
	Một nhóm 10 học sinh có điểm thi môn toán là: $5$; $6$; $7$; $5$; $8$; $8$; $10$; $9$; $7$; $8$. Tính điểm trung bình của nhóm học sinh trên.
	\choice
	{$8$}
	{\True $7{,}3$}
	{$8{,}3$}
	{$7{,}7$}
	\loigiai{
		Điểm trung bình là $\overline{x}=\dfrac{5\cdot 2+6+7\cdot 2+8\cdot 3+9+10}{10}=7{,}3$.}
\end{ex}
\begin{ex}%[0D5YC-2]% ::Cau 25::
	Thời gian xem ti vi trong tuần (đơn vị: giờ) của một số học sinh thu được kết quả như sau:
	\begin{center}
	\begin{tabular}{|c|c|c|c|c|c|}
		\hline
		Thời gian (giờ)	& $[0;4)$ & $[4;8)$ & $[8;12)$ & $[12;16)$ & $[16;20)$ \\
		\hline
		Số học sinh	& $6$ & $12$ & $4$ & $4$ & $2$ \\
		\hline
	\end{tabular}
	\end{center}
	Giá trị đại diện của nhóm $\left[ 12;16 \right)$ là
	\choice
	{$12$}
	{\True $14$}
	{$10$}
	{$16$}
	\loigiai{
	Giá trị đại diện của nhóm $\left[ 12;16 \right)$ là $\dfrac{12+16}{2}=14$.
	}
\end{ex}
\begin{ex}%[1D3Y2-2]% ::Cau 26::
	Cho dãy số $u_n$ biết với $u_n=\dfrac{1}{n+1}$, ba số hạng đầu tiên của dãy đó là
	\choice
	{\True $\dfrac{1}{2};\dfrac{1}{3};\dfrac{1}{4}$}
	{$1;\dfrac{1}{2};\dfrac{1}{3}$}
	{$\dfrac{1}{2};\dfrac{1}{4};\dfrac{1}{6}$}
	{$1;\dfrac{1}{3};\dfrac{1}{5}$}
	\loigiai{
	Ta có $u_n=\dfrac{1}{n+1}$, khi đó $u_1=\dfrac{1}{1+1}=\dfrac{1}{2}$, $u_2=\dfrac{1}{2+1}=\dfrac{1}{3}$, $u_3=\dfrac{1}{3+1}=\dfrac{1}{4}$.\\
	Ba số hạng đầu tiên của dãy đó là $\dfrac{1}{2};\dfrac{1}{3};\dfrac{1}{4}$.
	}
\end{ex}
\begin{ex}%[1D3B3-6]% ::Cau 27::
	Người ta trồng $465$ cây trong một khu vườn hình tam giác như sau: Hàng thứ nhất có $1$ cây, hàng thứ hai có $2$ cây, hàng thứ ba có $3$ cây. Số hàng cây trong khu vườn là:
	\choice
	{$31$}
	{\True $30$}
	{$29$}
	{$28$}
	\loigiai{
		Cách trồng $465$ cây trong một khu vườn hình tam giác như trên lập thành một cấp số cộng $\left( {u_n} \right)$ với số $u_n$ là số cây ở hàng thứ $n$ và $u_1=1$ và công sai $d=1$.\\
		Tổng số cây trồng được là
		$$S_n=465\Leftrightarrow \dfrac{n\left( n+1 \right)}{2}=465\Leftrightarrow {n^2}+n-930=0\Leftrightarrow \hoac{
			& n=30 \\
			& n=-31\,\left( \text{loại} \right).}$$
		Như vậy số hàng cây trong khu vườn là $30$.}
\end{ex}
\begin{ex}%[1D3B3-5]% ::Cau 28::
	Cho cấp số cộng $\left( u_n \right)$ có $u_{27}+u_2=83$. Khi đó tổng $28$ số hạng đầu tiên của cấp số cộng $\left( u_n \right)$ là
	\choice
	{\True $S_{28}=1162$}
	{$S_{28}=1612$}
	{$S_{28}=2611$}
	{$S_{28}=1261$}
	\loigiai{
		Gọi $d$ và $u_1$ lần lượt là công sai và số hạng đầu của cấp số cộng $\left( {u_n} \right)$\\
		Ta có $S_{28}=\dfrac{28\left( u_1+u_{28} \right)}{2}=\dfrac{28\left( u_2-d+u_{27}+d \right)}{2}=\dfrac{28\left( u_2+u_{27} \right)}{2}=\dfrac{28\cdot 83}{2}=1162$.}
\end{ex}
\begin{ex}%[1D3B3-2]% ::Cau 29::
	Cho cấp số cộng $\left( u_n \right)$ biết $u_{27}=-76$ và $u_{83}=-244$. Khi đó số hạng đầu $u_1$ của cấp số cộng đã cho bằng
	\choice
	{$-3$}
	{$5$}
	{$4$}
	{\True $2$}
	\loigiai{
		Gọi $d$ là công sai của cấp số cộng đã cho.\\
		Áp dụng công thức $u_n=u_1+\left( n-1 \right)d$, ta có 
		$$\heva{
			& u_{27}=-76 \\
			& u_{83}=-244}
		\Leftrightarrow \heva{
			& u_1+26d=-76 \\
			& u_1+82d=-244}
		\Leftrightarrow \heva{
			& u_1=2 \\
			& d=-3.}$$}
\end{ex}
\begin{ex}%[1D3K3-4]% ::Cau 30::
	Cho $a<b<c$ là ba số nguyên. Biết $a$, $b$, $c$ theo thứ tự tạo thành một cấp số cộng và $a$, $c$, $b$ theo thứ tự tạo thành một cấp số nhân. Tìm giá trị nhỏ nhất của $c$.
	\choice
	{$-2$}
	{\True $2$}
	{$-1$}
	{$4$}
	\loigiai{
		Ta có $\heva{& 2b=a+c \\& c^2=ab>0}$. Suy ra $$2c^2=a\left( a+c \right)\Rightarrow 2c^2-ac-a^2=0\Rightarrow \hoac{
			& c=a\,\left( \text{loại} \right) \\
			& c=-\dfrac{a}{2}\Rightarrow b=\dfrac{a}{4}=-\dfrac{c}{2}.}$$
		Suy ra $a$, $b$ trái dấu với $c$ $\Rightarrow \heva{
			& a<0 \\
			& c>0.}$\\
		Do $a$, $b$, $c$ nguyên nên $c$ chia hết cho $2$.\\
		Do đó $c$ nhỏ nhất bằng $2$ khi đó $a=-4$, $b=-1$.}
\end{ex}
\begin{ex}%[Dự án 1 - TLDH - TeamTeXHoa - Lê Quân]%[1K2B5-1] 
	Cho dãy số $(u_n)$ có $u_n=2\cdot 3^n$. Công thức truy hồi của dãy số $(u_n)$ là 
	\choice
	{$\heva{&u_1=6	\\&u_n=6u_{n-1},\, \forall n>1}$}
	{\True $\heva{&u_1=6	\\&u_n=3u_{n-1},\, \forall n>1}$}
	{$\heva{&u_1=3	\\&u_n=3u_{n-1},\, \forall n>1}$}
	{$\heva{&u_1=3	\\&u_n=6u_{n-1},\, \forall n>1}$}
	\loigiai{
		Ta có $u_n=2\cdot 3^n\Rightarrow \heva{&u_1=2\cdot 3^1=6	\\&u_{n+1}=2\cdot 3^{n+1}.}$\\
		$\Rightarrow u_{n+1}=2\cdot3\cdot 3^n =3u_n\Rightarrow u_n=3\cdot u_{n-1}$.\\
		Vậy $\heva{&u_1=6	\\&u_n=3u_{n-1},\, \forall n>1.}$
	}
\end{ex}
\begin{ex}%[Dự án 1 - TLDH - TeamTeXHoa - Sauluoi3105]%[1K2B5-4]
	Cho dãy số $\left(u_n\right)$, với $u_n=\dfrac{1}{1\cdot 4}+\dfrac{1}{2\cdot 5}+\ldots+\dfrac{1}{n(n+3)}, \forall n=1 ; 2 ; 3 \cdots$. Mệnh đề nào sau đây đúng?
	\choice
	{Dãy số $\left(u_n\right)$ bị chặn trên và không bị chặn dưới}
	{Dãy số $\left(u_n\right)$ bị chặn dưới và không bị chặn trên}
	{\True Dãy số $\left(u_n\right)$ bị chặn}
	{Dãy số $\left(u_n\right)$ không bị chặn}
	\loigiai{
		Ta có $u_n>0$ suy ra $\left(u_n\right)$ bị chặn dưới bởi $0$.\\ 
		Mặt khác $\dfrac{1}{k(k+3)}<\dfrac{1}{k(k+1)}=\dfrac{1}{k}-\dfrac{1}{k+1}\left(k \in \mathbb{N}^*\right)$ nên 
		$$u_n<\dfrac{1}{1\cdot 2}+\dfrac{1}{2\cdot 3}+\dfrac{1}{3\cdot 4}+\cdots+\dfrac{1}{n(n+1)}=1-\dfrac{1}{2}+\dfrac{1}{2}-\dfrac{1}{3}+\dfrac{1}{2}-\dfrac{1}{4}+\cdots+\dfrac{1}{n}-\dfrac{1}{n+1}=1-\dfrac{1}{n+1}<1.$$
Suy ra dãy $\left(u_n\right)$ bị chặn trên.\\
Vậy dãy $\left(u_n\right)$ bị chặn.
	}
\end{ex}
\Closesolutionfile{ans}
% \inputans{10}{ans/ans-1-GK1-KNTT-14-NH23-24}
\noindent{\bf\fontfamily{qag}\selectfont\color{violet}B. PHẦN TỰ LUẬN}
\setcounter{bt}{0}
\begin{bt}%[1D1B2-1]% ::Cau 31::
	% \begin{enumEX}{1}
		% \item 
		Cho $\alpha \in (-\dfrac{\pi}{2};0)$ và $\sin \alpha = -\dfrac13$. Tìm $\cos \alpha$, $\tan \alpha$, $\cot \alpha$. 
		% \item Giải phương trình $\cot \left( 2x-40^{\circ} \right)=-\sqrt{3}$.
	% \end{enumEX}
	\loigiai{
		% \begin{enumEX}{1}
		% \item 
		Ta có $\alpha \in (-\dfrac{\pi}{2};0)$ nên $\cos \alpha >0$. Suy ra $$\cos^2 \alpha = 1-\sin^2 \alpha = 1-\left(-\dfrac13\right)=\dfrac89 \Rightarrow \cos \alpha = \dfrac{2\sqrt{2}}{3}.$$ 
		$\tan \alpha = \dfrac{\sin \alpha}{\cos \alpha} = \dfrac{-\dfrac13}{\dfrac{2\sqrt{2}}{3}}=-\dfrac{\sqrt{2}}{4}$.\\
		$\cot \alpha = \dfrac{1}{\tan \alpha} = -2\sqrt{2}$.
		% \item Ta có $\cot \left( 2x-40^{\circ} \right)=-\sqrt{3}\Leftrightarrow 2x-40^{\circ} =-30^{\circ} +k180^{\circ} \Leftrightarrow x=5^{\circ} +k90^{\circ} ,\,\left( k\in \mathbb{Z} \right)$.
	% \end{enumEX}
	}
\end{bt}

\begin{bt}%[1D3B3-3]% ::Cau 34::
	Tìm tổng $15$ số hạng đầu tiên của cấp số cộng $\left( {u_n} \right)$, biết $\heva{& u_1+u_5-u_3=10 \\& u_1+u_6=17.}$
	\loigiai{
	Ta có
		$$\heva{
			& {u_1}+u_5-u_3=10 \\
			& {u_1}+u_6=17}
		\Leftrightarrow \heva{
			& {u_1}+u_1+4d-\left( {u_1}+2d \right)=10 \\
			& {u_1}+u_1+5d=17}
		\Leftrightarrow \heva{
			& {u_1}+2d=10 \\
			& 2u_1+5d=17}
		\Leftrightarrow \heva{
			& u_1=16 \\
			& d=-3.}$$
		Suy ra $S_{15}=\dfrac{15}{2}\left(2u_1+14d\right)=\dfrac{15}{2}[2 \cdot 16 + 14\cdot (-3)]=-150$.}
\end{bt}

\begin{bt}%[1D1K3-8]% ::Cau 36::
	Hàng ngày mực nước của một con kênh lên xuống theo thủy triều. Độ sâu $h$ (mét) của mực nước trong kênh tính theo thời gian $t$ (giờ) $\left( 0\le t\le 24 \right)$ được mô tả bởi công thức $h=A\cos \left( \dfrac{\pi t}{6}+1 \right)+B$, với $A, B$ là các số thực dương cho trước. Biết độ sâu của mực nước lớn nhất là $15$ mét khi thủy triều lên cao và khi thủy triều xuống thấp thì độ sâu của mực nước thấp nhất là $9$ mét. Tính thời điểm độ sâu của mực nước là $13{,}5$ mét (tính chính xác đến $\dfrac{1}{100}$ giờ).
	\loigiai{
		Với mọi $0\le t\le 24$, ta có
		\allowdisplaybreaks
		\begin{eqnarray*}
		&& -1\le \cos \left( \dfrac{\pi t}{6}+1 \right)\le 1 \\
		&\Leftrightarrow& -A+B\le A\cos \left( \dfrac{\pi t}{6}+1 \right)+B\le A+B.
		\end{eqnarray*}
		Độ sâu của mực nước lớn nhất bằng $A+B$ khi $\cos \left( \dfrac{\pi t}{6}+1 \right)=1$ và thấp nhất bằng $-A+B$ khi $\cos \left( \dfrac{\pi t}{6}+1 \right)=-1$.\\
		Ta có hệ $\heva{
			& A+B=15 \\
			& -A+B=9}
		\Leftrightarrow \heva{
			& B=12 \\
			& A=3.}$\\
		Ta được $h=3\cos \left( \dfrac{\pi t}{6}+1 \right)+12$.\\
		Theo đề, ta tìm thời điểm mà độ sâu 
		\allowdisplaybreaks
		\begin{eqnarray*}
		&& h=13{,}5\Leftrightarrow 3\cos \left( \dfrac{\pi t}{6}+1 \right)+12=13{,}5\Leftrightarrow \cos \left( \dfrac{\pi t}{6}+1 \right)=\dfrac{1}{2}\\
		&\Leftrightarrow& \hoac{
			& \dfrac{\pi t}{6}+1=\dfrac{\pi}{3}+k2\pi\\
			& \dfrac{\pi t}{6}+1=-\dfrac{\pi}{3}+k2\pi},\left( k\in \mathbb{Z} \right)
		\Leftrightarrow \heva{
			& t=\left( -1+\dfrac{\pi}{3} \right)\cdot \dfrac{6}{\pi}+12k \\
			& t=\left( -1-\dfrac{\pi}{3} \right)\cdot \dfrac{6}{\pi}+12k},\left( k\in \mathbb{Z} \right).
		\end{eqnarray*}
		Do $0\le t\le 24; k\in \mathbb{Z}$ nên $t=0{,}09$ (giờ); $t=12{,}09$ (giờ); $t=8{,}09$ (giờ); $t=20{,}09$ (giờ).}
\end{bt}

	\begin{bt}%[Trần Ngọc Thành, CTST-BG11]%[1T3T1-6]
		Cho hình vuông $H_0$ cạnh bằng 1 đơn vị độ dài. Chia hình vuông $H_0$ thành chín hình vuông bằng nhau, bỏ đi bốn hình vuông, nhận được hình $H_1$. Tiếp theo, chia mỗi hình vuông của $H_1$ thành chín hình vuông, rồi bỏ đi bốn hình vuông, nhận được hình $H_2$. Tiếp tục quá trình này, ta nhận được một dãy hình $H_n$ $(n=1,2,3,\ldots)$.	
		\\
		\centerline{
			\begin{tikzpicture}[scale=.8]% Muốn vẽ hình Hn thì dùng \hv{n}
				\def\a{2}
				\pgfmathsetmacro\sh{2*\a *sqrt(2)/3}
				\def\hv#1{
					\ifnum#1>0
					\draw[white,fill=white] 
					(-\a/3,\a/3) rectangle (\a/3,\a)
					(-\a/3,-\a/3) rectangle (-\a,\a/3)
					(-\a/3,-\a/3) rectangle (\a/3,-\a)
					(\a/3,-\a/3) rectangle (\a,\a/3)
					;
					\pgfmathtruncatemacro{\k}{#1-1}
					\begin{scope}[scale=1/3]\hv{\k}\end{scope}
					\begin{scope}[shift={(45:\sh)},scale=1/3]\hv{\k}\end{scope}
					\begin{scope}[shift={(135:\sh)},scale=1/3]\hv{\k}\end{scope}
					\begin{scope}[shift={(225:\sh)},scale=1/3]\hv{\k}\end{scope}
					\begin{scope}[shift={(315:\sh)},scale=1/3]\hv{\k}\end{scope}
					\fi
				}
				\begin{scope}
					\fill[gray] (-\a,-\a) rectangle (\a,\a);
					\hv{0}
					\path (0,-\a)node[below]{$H_0$};
				\end{scope}
				\begin{scope}[xshift=4.5cm]
					\fill[gray] (-\a,-\a) rectangle (\a,\a);
					\hv{1}
					\path (0,-\a)node[below]{$H_1$};
				\end{scope}
				\begin{scope}[yshift=-5cm]
					\fill[gray] (-\a,-\a) rectangle (\a,\a);
					\hv{2}
					\path (0,-\a)node[below]{$H_2$};
				\end{scope}
				\begin{scope}[xshift=4.5cm,yshift=-5cm]
					\fill[gray] (-\a,-\a) rectangle (\a,\a);
					\hv{3}
					\path (0,-\a)node[below]{$H_3$};
				\end{scope}
				% \path (current bounding box.south) node[below]{Hình $6$};
			\end{tikzpicture}
		}
		Tính tổng diện tích và tổng chu vi tất cả hình vuông được tô màu trong hình $H_5$.
		\loigiai{
			\begin{enumerate}
				\item Hình vuông $H_1$ có diện tích $S_1=5\cdot \left(\dfrac{1}{3}\right)^2=\dfrac{5}{9}$.\\
				Hình vuông $H_2$ có diện tích $S_2=5^2\cdot \left(\dfrac{1}{3^2}\right)^2=\left(\dfrac{5}{9}\right)^2$.\\
				Hình vuông $H_n$ có diện tích $S_n=5^n\cdot \left(\dfrac{1}{3^n}\right)^2=\left(\dfrac{5}{9}\right)^n$.\\
				
				\item Hình vuông $H_1$ có chu vi $C_1=5\cdot 4\cdot  \dfrac{1}{3}=4\cdot \dfrac{5}{3}$.\\
				Hình vuông $H_2$ có chu vi $C_2=5^2\cdot4\cdot \dfrac{1}{3^2}=4\cdot \left(\dfrac{5}{3}\right)^2$.\\
				Hình vuông $H_n$ có diện tích $C_n=5^n\cdot4\cdot  \dfrac{1}{3^n}=4\cdot \left(\dfrac{5}{3}\right)^n$.\\
				
			\end{enumerate}
			Vậy $S_5=\left(\dfrac{5}{9}\right)^5$ và $C_5=4\cdot \left(\dfrac{5}{3}\right)^5$.
		}
	\end{bt}

% \section*{ÔN TẬP KIỂM TRA GIỮA KÌ 1 - ĐỀ 02}
\setcounter{ex}{0}\setcounter{bt}{0}
\noindent{\bf\fontfamily{qag}\selectfont\color{violet}A. PHẦN TRẮC NGHIỆM}
\Opensolutionfile{ans}[ans/ans-1-GK1-KNTT-De15-NH23-24]

%%==========Câu 1
\begin{ex}%[1K1Y1-5] 
	$\sin\alpha>0$ khi điểm cuối của cung $\alpha$ trên đường tròn lượng giác thuộc các góc phần tư thứ
	\choice
	{I và III}
	{\True I và II}
	{II và IV}
	{I và IV}
	\loigiai{
		$\sin\alpha>0$ khi điểm cuối của cung $\alpha$ trên đường tròn lượng giác các góc phần tư thứ I và II.	
	}
\end{ex}
	\begin{ex}%[1K1Y1-8]
	Trong các khẳng định sau, khẳng định nào sai?
	\choice
	{$\tan\left(\pi-\alpha\right)=-\tan\alpha$}
	{\True $\tan\left(\pi+\alpha\right)=-\tan\alpha$}
	{$\tan\left(-\alpha\right)=-\tan\alpha$}
	{$\tan\left(\dfrac{\pi}{2}-\alpha\right)=\cot\alpha$}
	\loigiai{
		Vì $\tan\left(\pi+\alpha\right)=\tan\alpha$ nên khẳng định sai là $\tan\left(\pi+\alpha\right)=-\tan\alpha$.	
	}
\end{ex}
	\begin{ex}%[1K1Y1-5]
	Khi biểu diễn cung lượng giác $\alpha$ lên đường tròn lượng giác thì điểm cuối của cung $\alpha$ thuộc góc phần tư thứ ba của đường tròn lượng giác. Khẳng định nào sau đây là \textbf{đúng}?
	\choice
	{$\sin\alpha>0$}
	{$\cos\alpha>0$}
	{\True $\tan\alpha>0$}
	{$\cot\alpha<0$}
	\loigiai{
		Vì khi biểu diễn cung lượng giác $\alpha$ lên đường tròn lượng giác thì điểm cuối của cung $\alpha$ thuộc góc phần tư thứ ba của đường tròn lượng giác nên $\tan\alpha>0$.
	}
\end{ex}
\begin{ex}%[1K1B1-6]
	Cho góc $\alpha$ thỏa mãn $\sin\alpha=\dfrac{4}{5}$ và $\dfrac{\pi}{2}<\alpha<\pi$. Tính $\cos\alpha$.
	\choice
	{$\cos\alpha=\dfrac{3}{5}$}
	{\True $\cos\alpha=-\dfrac{3}{5}$}
	{$\cos\alpha=-\dfrac{1}{5}$}
	{$\cos\alpha=\dfrac{1}{5}$}
	\loigiai{
		Ta có $\cos^2\alpha=1-\sin^2\alpha=1-\left(\dfrac{4}{5}\right)^2 \Leftrightarrow \cos^2\alpha =\dfrac{9}{25} \Leftrightarrow \cos\alpha=\pm\dfrac{3}{5}$.\\
		Mà $\dfrac{\pi}{2}<\alpha<\pi$ nên $\cos\alpha<0$. Vậy $\cos\alpha=-\dfrac{3}{5}$.	
	}
\end{ex}
\begin{ex}%[1K1Y2-1]
	Trong các công thức sau, công thức nào đúng?
	\choice
	{\True $\sin\left(a-b\right)=\sin a\cdot\cos b-\sin b\cdot\cos a$}
	{$\cos\left(a-b\right)=\cos a\cdot\cos b-\sin a\cdot\sin b$}
	{$\sin\left(a+b\right)=\sin a\cdot\cos b-\sin b\cdot\cos a$}
	{$\cos\left(a+b\right)=\cos a\cdot\cos b$ $+$ $\sin a\cdot\sin b$}
	\loigiai{
		Công thức cộng $\sin\left(a-b\right)=\sin a\cdot\cos b-\sin b\cdot\cos a$.	
	}
\end{ex}
\begin{ex}%[1K1K2-1]
	Rút gọn biểu thức $\sin\left(a-17^\circ\right)\cos\left(a+13^\circ\right)-\sin\left(a+13^\circ\right)\cos\left(a-17^\circ\right)$, ta được
\choice
{$\sin2a$}
{$\cos2a$}
{\True $-\dfrac{1}{2}$}
{$\dfrac{1}{2}$}
\loigiai{ 
	Ta có $\sin\left(a-17^\circ\right)\cdot\cos\left(a+13^\circ\right)-\sin\left(a+13^\circ\right)\cdot\cos\left(a-17^\circ\right)=\sin\left[\left(a-17^\circ\right)-\left(a+13^\circ\right)\right]$\\
	$=\sin\left(-30^\circ\right)=-\dfrac{1}{2}$.	
	}
\end{ex}
\begin{ex}%[1K1K2-3]
	Với $\alpha$ là số thực bất kỳ, mệnh đề nào sau đây là mệnh đề đúng? 
	\choice
	{$\cos2\alpha+\cos4\alpha=2\cos2\alpha\cdot\cos6\alpha$}
	{$\sin2\alpha+\sin4\alpha=2\sin\alpha\cdot\cos3\alpha$}
	{$\cos2\alpha-\cos4\alpha=-2\sin3\alpha\cdot\sin\alpha$}
	{\True $\sin2\alpha-\sin4\alpha=-2\cos3\alpha\cdot\sin\alpha$}
	\loigiai{
		Ta có\\
		$\cos2\alpha+\cos4\alpha=2\cos\dfrac{2\alpha+4\alpha}{2}\cdot\cos\dfrac{2\alpha-4\alpha}{2}=2\cos3\alpha\cdot\cos\alpha$. Do đó $\cos2\alpha+\cos4\alpha=2\cos2\alpha\cdot\cos6\alpha$ sai.\\
		$\sin2\alpha+\sin4\alpha=2\sin\dfrac{2\alpha+4\alpha}{2}\cdot\cos\dfrac{2\alpha-4\alpha}{2}=2\sin\alpha.\cos\alpha$. Do đó $\sin2\alpha+\sin4\alpha=2\sin\alpha\cdot\cos3\alpha$ sai.\\
		$\cos2\alpha-\cos4\alpha=-2\sin\dfrac{2\alpha+4\alpha}{2}\cdot\sin\dfrac{2\alpha-4\alpha}{2}=2\sin3\alpha\cdot\sin\alpha$. Do đó $\cos2\alpha-\cos4\alpha=-2\sin3\alpha\cdot\sin\alpha$ sai.\\	
		$\sin2\alpha-\sin4\alpha=2\cos\dfrac{2\alpha+4\alpha}{2}\cdot\sin\dfrac{2\alpha-4\alpha}{2}=-2\cos3\alpha\cdot\cos\alpha$ là đáp án đúng.
	}
\end{ex}
\begin{ex}%[1K1B3-1]
	Tập xác định của hàm số $y=\tan x$ là 
	\choice
	{$\mathscr{D}=[-1;1]$}
	{$\mathscr{D}=\mathbb{R}\setminus\left\{k\pi,k\in\mathbb{Z}\right\}$}
	{\True $\mathscr{D}=\mathbb{R}\setminus\left\{\dfrac{\pi}{2}+k\pi,k\in\mathbb{Z}\right\}$}
	{$\mathscr{D}=\mathbb{R}$}
	\loigiai{
		Điều kiện xác định $\cos x\neq 0 \Leftrightarrow x\neq \dfrac{\pi}{2}+k\pi,k\in\mathbb{Z}$.\\
		Vậy tập xác định của hàm số là $\mathscr{D}=\mathbb{R}\setminus\left\{\dfrac{\pi}{2}+k\pi,k\in\mathbb{Z}\right\}$.
	}
\end{ex}
\begin{ex}%[1K1Y3-4]
	Hàm số $y=\sin x$ tuần hoàn với chu kỳ là 
	\choice
	{$\dfrac{\pi}{2}$}
	{$\dfrac{\pi}{3}$}
	{\True $2\pi$}
	{$\pi$}
	\loigiai{
		Hàm số $y=\sin x$ tuần hoàn với chu kỳ là $2\pi$.
	}
\end{ex}
\begin{ex}%[1K1Y3-3]
	Trong các hàm số sau $y=\sin x$, $y=\cos x$, $y=\tan x$, $y=\cot x$ hàm số nào là hàm số chẵn? 
	\choice
	{\True $y=\cos x$}
	{$y=\tan x$}
	{$y=\cot x$}
	{$y=\sin x$}
	\loigiai{
		Hàm số $y=\cos x$ là hàm số chẵn.
	}
\end{ex}
\begin{ex}%[1K1B3-1]
	Tìm tập xác định $\mathscr{D}$ của hàm số $y=\dfrac{3\tan x-5}{1-\sin^2 x}$.
	\choice
	{$\mathscr{D}=\mathbb{R}\setminus\left\{\dfrac{\pi}{2}+k2\pi,k\in\mathbb{Z}\right\}$}
	{\True $\mathscr{D}=\mathbb{R}\setminus\left\{\dfrac{\pi}{2}+k\pi,k\in\mathbb{Z}\right\}$}
	{$\mathscr{D}=\mathbb{R}\setminus\left\{\pi+k\pi,k\in\mathbb{Z}\right\}$}
	{$\mathscr{D}=\mathbb{R}$}
	\loigiai{
	Hàm số xác định khi và chỉ khi $\heva{&1-\sin^2 x\neq 0\\&\cos x\neq 0}$\\
	$\Leftrightarrow \heva{&\sin^2 x\neq 1\\&\cos x\neq 0} \Leftrightarrow \cos x\neq 0 \Leftrightarrow x\neq\dfrac{\pi}{2}+k\pi,k\in\mathbb{Z}$.\\
	Vậy tập xác định của hàm số là $\mathscr{D}=\mathbb{R}\setminus\left\{\dfrac{\pi}{2}+k\pi,k\in\mathbb{Z}\right\}$. 
	}
\end{ex}
\begin{ex}%[1K1K3-5]
	Gọi $M, m$, lần lượt là giá trị lớn nhất, giá trị nhỏ nhất của hàm số $y=\sqrt{3}\sin 2x - \cos 2x -1$. Giá trị của $M+m$ bằng
	\choice
	{$M+m=0$}
	{\True $M+m=-2$}
	{$M+m=1$}
	{$M+m=-1$}
	\loigiai{
		Ta có $y=\sqrt{3}\sin 2x - \cos 2x -1=2\left(\dfrac{\sqrt{3}}{2}\sin 2x -\dfrac{1}{2}\cos 2x\right)-1= 2\sin\left(2x-\dfrac{\pi}{6}\right)-1$\\
		Vì $\forall x\in\mathbb{R}, -1\leq \sin\left(2x-\dfrac{\pi}{6}\right)\leq1$ nên suy ra $-2 -1\leq 2\sin\left(2x-\dfrac{\pi}{6}\right)-1\leq2-1$.\\
		Do đó, $-3\leq y\leq1,\forall x\in\mathbb{R}$.\\
		Do đó giá trị lớn nhất và giá trị nhỏ nhất của hàm số $y=\sqrt{3}\sin 2x - \cos 2x -1$ lần lượt là $M=1,m=-3$.\\
		Khi $\sin\left(2x-\dfrac{\pi}{6}\right)=1 \Leftrightarrow 2x-\dfrac{\pi}{6}=\dfrac{\pi}{2}+k2\pi \Leftrightarrow x=\dfrac{\pi}{3}+k2\pi,k\in\mathbb{Z}$.\\
		$\sin\left(2x-\dfrac{\pi}{6}\right)=-1 \Leftrightarrow 2x-\dfrac{\pi}{6}=-\dfrac{\pi}{2}+k2\pi \Leftrightarrow x=-\dfrac{\pi}{6}+k\pi,k\in\mathbb{Z}$.\\
		Vậy $M+m=1-3=-2$.
	}
\end{ex}
\begin{ex}%[1K1Y4-2]
	Phương trình nào sau đây có nghiệm?
	\choice
	{$\sin2x=2$}
	{$\cos2x=-2$}
	{\True $\sin3x=\dfrac{2}{3}$}
	{$\cos x =\pi$}
	\loigiai{
		Do $\dfrac{2}{3}\in[-1;1]$ nên phương trình $\sin3x=\dfrac{2}{3}$ có nghiệm.
	}
\end{ex}
\begin{ex}%[1K1B4-5]
	Nghiệm của phương trình $\sin x=\dfrac{1}{2}$ là
	\choice
	{$x=\dfrac{\pi}{6}+k\pi$; $x=\dfrac{5\pi}{6}+k\pi,k\in\mathbb{Z}$}
	{$x=\dfrac{\pi}{6}+k\pi$; $x=\dfrac{-\pi}{6}+k\pi,k\in\mathbb{Z}$}
	{\True $x=\dfrac{\pi}{6}+k2\pi$; $x=\dfrac{5\pi}{6}+k2\pi,k\in\mathbb{Z}$}
	{$x=\dfrac{\pi}{6}+k2\pi$; $x=\dfrac{-\pi}{6}+k2\pi,k\in\mathbb{Z}$}
	\loigiai{
		Ta có $\sin x=\dfrac{1}{2} \Leftrightarrow \sin x=\sin\dfrac{\pi}{6} \Leftrightarrow \hoac{&x=\dfrac{\pi}{6}+k2\pi\\&x=\dfrac{5\pi}{6}+k2\pi}, k\in\mathbb{Z}$. 
	}
\end{ex}
\begin{ex}%[1K1K4-5]
	Tổng tất cả các nghiệm của phương trình $\sin\left(x+\dfrac{\pi}{4}\right)+\cos\left(x-\dfrac{3\pi}{4}\right)=0$ thuộc $\left(0;5\pi\right)$ bằng
	\choice
	{\True $10\pi$}
	{$7\pi$}
	{$6\pi$}
	{$9\pi$}
	\loigiai{
		Ta có\\
		$\sin\left(x+\dfrac{\pi}{4}\right)+\cos\left(x-\dfrac{3\pi}{4}\right)=0 \Leftrightarrow \sin x\cdot\cos\dfrac{\pi}{4}+\cos x\cdot\sin\dfrac{\pi}{4}+\cos x\cdot\cos\dfrac{3\pi}{4}+\sin x\cdot\sin\dfrac{3\pi}{4}=0$\\
		$\Leftrightarrow \sqrt{2}\sin x=0 \Leftrightarrow \sin x=0 \Leftrightarrow x=k\pi,k\in\mathbb{Z}$.\\
		Vì $x\in\left(0;5\pi\right) \Rightarrow 0<k\pi<5\pi \Leftrightarrow 0<k<5$.\\
		Vì $k\in\mathbb{Z} \Rightarrow k\in\left\{1;2;3;4\right\} \Rightarrow x\in\left\{\pi;2\pi;3\pi;4\pi\right\}$.\\
		Khi đó, tổng các nghiệm của phương trình là $S=\pi+2\pi+3\pi+4\pi=10\pi$.	
	}
\end{ex}
\begin{ex}%[1K2B5-2]
	Cho dãy số $\left(u_{n}\right)$ có số hạng tổng quát $u_{n}=n^2-3$. Số hạng thứ $10$ của dãy số là
	\choice
	{$7$}
	{\True $97$}
	{$100$}
	{$103$}
	\loigiai{
		Số hạng thứ $10$ của dãy số là $u_{10}=10^2-3=97$.	
	}
\end{ex}
\begin{ex}%[1K2B5-2]
	Cho dãy số $0;2;4;6;\ldots;304$. Hỏi dãy số trên có bao nhiêu số hạng?
	\choice
	{$304$}
	{$152$}
	{\True $153$}
	{$305$}
	\loigiai{
		Số các số hạng của dãy số là $\dfrac{304-0}{2}+1=153$.	
	}
\end{ex}
\begin{ex}%[1K2B6-1]
	Cho cấp số cộng $1;1;1;\ldots$. Công sai của cấp số cộng trên là
	\choice
	{\True $0$}
	{$1$}
	{$-1$}
	{$\varnothing$}
	\loigiai{
		Cấp số cộng có công sai $d=1-1=0$. Đây là dãy số không đổi.	
	}
\end{ex}
\begin{ex}%[1K2B6-3]
	Cho cấp số cộng $\left(u_{n}\right)$ với $u_{1}=-2$ và công sai $d=3$ thì số hạng $u_{5}$ bằng
	\choice
	{$7$}
	{\True $10$}
	{$5$}
	{$6$}
	\loigiai{
		Áp dụng công thức số hạng thứ $n$ của cấp số cộng $\left(u_{n}\right)$ là $u_{n}=u_{1}+\left(n-1\right)\cdot d$.\\
		Khi đó số hạng $u_{5}=u_{1}+\left(5-1\right)\cdot d=-2+4\cdot3=10$. Vậy $u_{5}=10$.	
	}
\end{ex}
\begin{ex}%[1K2G6-6]
	Vào năm $2023$, nhiệt độ trung bình của thành phố $A$ là khoảng $29{,}5^\circ C$. Giả sử do biến đổi khí hậu nên mỗi năm nhiệt độ trung bình của thành phố $A$ đều tăng thêm khoảng $0{,}1^\circ C$. Hãy ước tính kể từ năm nào thì nhiệt độ trung bình của thành phố $A$ đạt từ $35^\circ C$ trở lên.
	\choice
	{$2076$}
	{$2077$}
	{\True $2078$}
	{$2079$}
	\loigiai{
		Theo bài toán, nhiệt độ trung bình ở mỗi năm của thành phố $A$ lập thành cấp số cộng với công sai là $d=0{,}1\left(^\circ C\right)$ và $u_{1}=29{,}5^\circ C$ là nhiệt độ trung bình của thành phố $A$ vào năm $2023$.\\
		Giả sử số hạng thứ $n$ của cấp số cộng có giá trị lớn hơn hoặc bằng $35$.\\
		Tức là, $u_{n}\geq35^\circ C$ hay $u_{1}+\left(n-1\right)\cdot d\geq35 \Leftrightarrow 29{,}5+\left(n-1\right)\cdot0{,}1\geq35 \Leftrightarrow n\geq56$.\\
		Do đó, kể từ số hạng thứ $56$ trở đi thì chúng đều có giá trị lớn hơn hoặc bằng $35$.\\
		Ta có $u_{1}$ là nhiệt độ trung bình của thành phố $A$ vào năm $2023$.\\
		Nên $u_{55}$ là nhiệt độ trung bình của thành phố $A$ vào năm $\left(2023+56-1\right)=2078$.\\
		Vậy kể từ năm $2078$ thì nhiệt độ trung bình của thành phố $A$ đạt từ $35^\circ C$ trở lên. 	
	}
\end{ex}
\begin{ex}%[1K2K7-1]
	Cho cấp số nhân $\left(u_{n}\right)$ có công bội dương và $u_{2}=\dfrac{1}{5}$, $u_{4}=5$. Tính công bội $q$.
	\choice
	{\True $5$}
	{$25$}
	{$\dfrac{1}{5}$}
	{$125$}
	\loigiai{
		Ta có $\heva{&u_{2}=u_{1}\cdot q=\dfrac{1}{5}\\&u_{4}=u_{1}\cdot q^3=5} \Rightarrow \dfrac{u_{4}}{u_{2}}=q^2=25 \Leftrightarrow q=\pm5$.\\
		Mà cấp số nhân $\left(u_{n}\right)$ có công bội dương nên $q=5$.
	}
\end{ex}
\begin{ex}%[1K2B7-4]
	Tìm $x$ để các số $2;8;x;128$ theo thứ tự đó lập thành một cấp số nhân.
	\choice
	{$16$}
	{$64$}
	{$34$}
	{\True $32$}
	\loigiai{
	Các số $2;8;x;128$ theo thứ tự đó lập thành một cấp số nhân khi $x=\sqrt{8\cdot 128}=32$.
	}
\end{ex}
\begin{ex}%[1K2B7-1]
	Cho cấp số nhân $\left(u_{n}\right)$, biết $u_{1}=1$, $u_{4}=64$. Tính công bội $q$ của cấp số nhân.
	\choice
	{$21$}
	{$\pm4$}
	{\True $4$}
	{$2\sqrt{2}$}
	\loigiai{
		Theo công thức tổng quát của cấp số nhân $u_{4}=u_{1}\cdot q^3 \Leftrightarrow 64=1\cdot q^3 \Leftrightarrow q=4$.
	}
\end{ex}
\begin{ex}%[1K2B7-3]
	Cho cấp số nhân $\left(u_{n}\right)$ với $u_{1}=-1$, $q=\dfrac{-1}{10}$. Số $\dfrac{1}{10^{103}}$ là số hạng thứ mấy của $\left(u_{n}\right)$?
	\choice
	{số hạng thứ $103$}
	{\True số hạng thứ $104$}
	{số hạng thứ $105$}
	{Không là số hạng của cấp số đã cho}
	\loigiai{
		Ta có $u_{n}=u_{1}\cdot q^{n-1} \Rightarrow \dfrac{1}{10^{103}}=-1\cdot \left(\dfrac{-1}{10}\right)^{n-1} \Rightarrow n-1=103 \Rightarrow n=104$.
	}
\end{ex}
\begin{ex}%[1K3Y9-5]
	Mỗi nhóm số liệu ghép nhóm là tập hợp gồm
	\choice
	{Các giá trị của số liệu được ghép nhóm theo nhiều tiêu chí xác định}
	{Các giá trị của số liệu được ghép nhóm theo hai tiêu chí xác định}
	{\True Các giá trị của số liệu được ghép nhóm theo một tiêu chí xác định}
	{Các giá trị của số liệu được ghép nhóm theo ba tiêu chí xác định}
	\loigiai{
		Theo định nghĩa số liệu ghép nhóm: Các giá trị của số liệu được ghép nhóm theo một tiêu chí xác định.
	}
\end{ex}
\begin{ex}%[1K3Y8-1]
	Mẫu số liệu sau cho biết phân bố theo độ tuổi của dân số Việt Nam năm $2019$\\
	\begin{center}
		\begin{tabular}{|c|c|c|c|}
		\hline
		Độ tuổi& Dưới $15$ & Từ $15$ đến $65$ & Từ $65$ trở lên\\
		\hline
		Số người& $23 371 882$ & $65 420 451$ & $7 416 651$\\
		\hline
	\end{tabular}
	\end{center}
	Số dân Việt Nam năm $2019$ là
	\choice
	{$73837102$}
	{$72837102$}
	{$95208984$}
	{\True $96208984$}
	\loigiai{
		Số dân Việt Nam năm $2019$ là $23371882+65420451+7416651= 96208984$.	
	}
\end{ex}
\begin{ex}%[1K3Y9-4]
	Khảo sát thời gian tập thể dục của một số học sinh khối $11$ thu được mẫu số liệu ghép nhóm sau
	\begin{center}
		\begin{tabular}{|c|c|c|c|c|c|}
			\hline
			Thời gian $\left(\text{phút}\right)$& $\left[0;20\right)$ & $\left[20;40\right)$ & $\left[40;60\right)$& $\left[60;80\right)$ & $\left[80;100\right)$\\
			\hline
			Số học sinh & $5$ & $9$ & $12$ & $10$ & $6$\\
			\hline
		\end{tabular}
	\end{center}
	Nhóm chứa mốt của mẫu số liệu trên là
	\choice
	{$\left[20;40\right)$}
	{$\left[60;80\right)$}
	{\True $\left[40;60\right)$}
	{$\left[80;100\right)$}
	\loigiai{
		Mốt $M_0$ chứa trong nhóm $\left[40;60\right)$.		
	}
\end{ex}
\begin{ex}%[1K3K9-3]
	Khảo sát thời gian tập thể dục của một số học sinh khối $11$ thu được mẫu số liệu ghép nhóm sau
	\begin{center}
		\begin{tabular}{|c|c|c|c|c|c|}
			\hline
			Thời gian $\left(\text{phút}\right)$& $\left[0;20\right)$ & $\left[20;40\right)$ & $\left[40;60\right)$& $\left[60;80\right)$ & $\left[80;100\right)$\\
			\hline
			Số học sinh & $5$ & $9$ & $12$ & $10$ & $6$\\
			\hline
		\end{tabular}
	\end{center}
	Nhóm chứa tứ phân vị thứ ba của mẫu số liệu trên là
	\choice
	{$\left[20;40\right)$}
	{\True $\left[60;80\right)$}
	{$\left[40;60\right)$}
	{$\left[80;100\right)$}
	\loigiai{
		Ta có $n=42$ nên tứ phân vị thứ ba của mẫu số liệu trên là $Q_3=x_{33}$.\\
		Mà $x_{33}\in\left[60;80\right)$.\\
		Vậy nhóm chứa tứ phân vị thứ ba của mẫu số liệu trên là nhóm $\left[60;80\right)$.		
	}
\end{ex}
\begin{ex}%[1K3B9-1]
	Khi thống kê chiều cao của $40$ bạn lớp $11A$, ta thu được mẫu số liệu ghép nhóm được cho ở bảng sau (đơn vị: centimét).
	\begin{center}
		\begin{tabular}{|c|c|}
		\hline
		Nhóm& Tần số\\
		\hline
		$\left[155;160\right)$&$5$\\
		\hline
		$\left[160;165\right)$&$12$\\
		\hline
		$\left[165;170\right)$& $16$\\
		\hline
		$\left[170;175\right)$&$7$\\
		\hline
		& $n=40$\\
		\hline
	\end{tabular}
	\end{center}
	Số trung bình cộng bằng
	\choice
	{\True $165{,}6$}
	{$156{,}6$}
	{$155{,}6$}
	{$156{,}5$}
	\loigiai{
		Số trung bình cộng là $\bar{x}=\dfrac{5\cdot157{,}5+12\cdot162{,}5+16\cdot167{,}5+7\cdot172{,}5}{40}\approx165{,}6$.		
	}
\end{ex}
\begin{ex}%[1K3K9-3]
	Cho mẫu số liệu ghép nhóm thống kê thời gian sử dụng điện thoại trước khi ngủ (đơn vị: phút) của một người trong $120$ ngày như ở bảng sau. Xác định các số đặc trưng đo xu thế trung tâm cho mẫu số liệu đó (làm tròn các kết quả đến hàng phần mười).
	\begin{center}
		\begin{tabular}{|c|c|}
			\hline
			Nhóm & Tần số\\
			\hline
			$\left[0;4\right)$& $13$\\
			\hline
			$\left[4;8\right)$& $29$\\
			\hline
			$\left[8;12\right)$& $48$\\
			\hline
			$\left[12;16\right)$& $22$\\
			\hline
			$\left[16;20\right)$& $8$\\
			\hline
			& $n=120$\\
			\hline
		\end{tabular}
	\end{center}
	Giá trị các tứ phân vị thứ nhất, thứ hai và thứ ba lần lượt là
	\choice
	{$9{,}5;12;6{,}3$}
	{\True $6{,}3;9{,}5;12$}
	{$9{,}5;6{,}3;12$}
	{$12;6{,}3;9{,}5$}
	\loigiai{
		Bảng tần số ghép nhóm bao gồm cả tần số tích luỹ được cho như ở bảng
	\begin{center}
			\begin{tabular}{|c|c|c|}
			\hline
			Nhóm & Tần số & Tần số tích lũy\\
			\hline
			$\left[0;4\right)$& $13$ & $13$\\
			\hline
			$\left[4;8\right)$& $29$ & $42$\\
			\hline
			$\left[8;12\right)$& $48$ & $90$\\
			\hline
			$\left[12;16\right)$& $22$ & $112$\\
			\hline
			$\left[16;20\right)$& $8$ & $120$\\
			\hline
			& $n=120$ &\\
			\hline
		\end{tabular}
	\end{center}
		Ta có $\dfrac{n}{2}=60$, $\dfrac{n}{4}=30$, $\dfrac{3n}{4}=90$.\\
		Vì $42<60<90$ nên nhóm $3$ là nhóm đầu tiên có tần số tích luỹ lớn hơn hoặc bằng $60$.\\
		Suy ra trung vị là $M_e=8+\left(\dfrac{60-42}{48}\right)\cdot 4=9{,}5$.\\
		Tứ phân vị thứ hai là $Q_2=M_e=9{,}5$.\\
		Do $13<30<42$ nên nhóm $2$ là nhóm đầu tiên có tần số tích luỹ lớn hơn hoặc bằng $30$. Suy ra tứ phân vị thứ nhất là $Q_1=4+\left(\dfrac{30-13}{29}\right)\cdot 4\approx6{,}3$.\\
		Do $42<90\leq 90$ nên nhóm $3$ là nhóm đầu tiên có tần số tích luỹ lớn hơn hoặc bằng $90$. Suy ra tứ phân vị thứ ba là $Q_3=8+\left(\dfrac{90-42}{48}\right)\cdot 4=12$.		
	}
\end{ex}


\Closesolutionfile{ans}
% \inputans{10}{ans/ans-1-GK1-KNTT-De15-NH23-24}
\noindent{\bf\fontfamily{qag}\selectfont\color{violet}B. PHẦN TỰ LUẬN}
\setcounter{bt}{0}

%%==========Bài 1
\begin{bt}%[1K1K3-1]
	\begin{enumEX}{1}
		\item Tìm tập xác định của hàm số $y=\dfrac{\sqrt{1+\cos{2x}}}{1-\left(\sin x-\cos x\right)^2}$.
		\item Cho góc $\alpha \in (-\pi ; -\dfrac{\pi}{2})$ và $\tan \alpha = 3$. Tìm các GTLG của $\alpha$.
	\end{enumEX}
	\loigiai{
	Hàm số xác định khi và chỉ khi $\heva{&1+\cos{2x}\geq0\\&1-\left(\sin x-\cos x\right)^2\neq0} \Leftrightarrow \heva{&\cos{2x}\geq-1\\&\sin{2x}\neq0}$.\\
	$\cos{2x}\geq-1$ thỏa mãn $\forall x\in\mathbb{R}$.\\
	$\sin{2x}\neq0 \Leftrightarrow 2x\neq k\pi,k\in\mathbb{Z} \Leftrightarrow x\neq k\dfrac{\pi}{2},k\in\mathbb{Z}$.\\
	Vậy tập xác định của hàm số là $\mathscr{D}=\mathbb{R}\setminus\left\{k\dfrac{\pi}{2},k\in\mathbb{Z}\right\}$.
	}
\end{bt}

\begin{bt}%[1K2K6-5]
	Cho cấp số cộng $\left(u_n\right)$ có $u_5=-15$, $u_{20}=60$. Tính tổng $10$ số hạng đầu tiên của cấp số cộng đó.
	\loigiai{
		Gọi $u_1$, $d$ lần lượt là số hạng đầu và công sai của cấp số cộng.\\
		Ta có $\heva{&u_5=-15\\&u_{20}=60} \Leftrightarrow \heva{&u_1+4d=-15\\&u_1+19d=60} \Leftrightarrow \heva{&u_1=35\\&d=5}$.\\
		Vậy $S_{10}=\dfrac{10}{2}\cdot\left(2u_1+9d\right)=5\cdot\left[2\cdot\left(-35\right)+9\cdot5\right]=-125$.
	}
\end{bt}
\begin{bt}
	Số giờ có ánh sáng mặt trời của một thành phố A ở vĩ độ $40^\circ$ bắc trong ngày thứ $t$ của một năm không nhuận được cho bởi hàm số $d\left( t \right) = 3\sin \left[ {\dfrac{\pi }{{182}}\left( {t - 80} \right)} \right] + 12$  với $t \in \mathbb{Z}$ và $0<t \le 365$. Hãy cho biết ngày tháng nào có nhiều giờ có ánh sáng mặt trời nhất và ngày tháng nào có ít giờ có ánh sáng mặt trời nhất trong năm (không nhuận)?
\end{bt}
\begin{bt}%[1K2G7-6]
	Tìm $4$ số hạng đầu của một cấp số nhân biết tổng $3$ số hạng đầu bằng $\dfrac{148}{9}$, đồng thời theo thứ tự chúng là số hạng thứ $1$, thứ $4$, thứ $8$ của một cấp số cộng có công sai khác $0$.
	\loigiai{
		Gọi $4$ số hạng đầu của cấp số nhân đã cho là $u_1$, $u_2$, $u_3$, $u_4$; công bội của cấp số nhân là $ q $, công sai của cấp số cộng là $ d $ $\left(d\neq0\right)$.\\
		Tổng $3$ số hạng đầu của cấp số nhân bằng $\dfrac{148}{9}$ nên $u_1+u_2+u_3=\dfrac{148}{9} \Leftrightarrow u_1+u_1\cdot q+u_1\cdot q^2=\dfrac{148}{9}\left(1\right)$.\\
		Do $u_1$, $u_2$, $u_3$ theo thứ tự chúng là số hạng thứ $1$, thứ $4$, thứ $8$ của một cấp số cộng có công sai $d\neq0$ nên\\
		$\heva{&u_1\cdot q=u_1+3d\left(2\right)\\&u_1\cdot q^2=u_1+7d\left(3\right)}$.\\
		Nhân phương trình $\left(2\right)$ với $ 7 $ và nhân phương trình $\left(3\right)$ với $ 3 $, sau đó trừ hai phương trình theo vế ta được $u_1\left(3q^2-7q+4\right)=0\left(4\right)$.\\
		Từ phương trình $\left(1\right)$ ta có $u_1\neq0$. Khi đó $\left(3\right)\Leftrightarrow 3q^2-7q+4=0 \Leftrightarrow \hoac{&q=1\\&q=\dfrac{4}{3}}$.
		\begin{itemize}
			\item[+)] Với $q=1$, thay vào $\left(1\right)$ suy ra $u_1=u_2=u_3=\dfrac{148}{27}$ (loại do $u_1$, $_2$, $u_3$ theo thứ tự chúng là số
			hạng thứ $1$, thứ $4$, thứ $8$ của một cấp số cộng có công sai $d\neq0$).
			\item[+)] Với $q=\dfrac{4}{3}$, thay vào $\left(1\right)$ suy ra $u_1=4$, $u_2=\dfrac{16}{3}$, $u_3=\dfrac{64}{9}$, $u_4=\dfrac{256}{27}$.
		\end{itemize}
			Vậy $4$ số hạng đầu của cấp số nhân là $u_1=4$, $u_2=\dfrac{16}{3}$, $u_3=\dfrac{64}{9}$, $u_4=\dfrac{256}{27}$.
	}
\end{bt}
\begin{bt}%[Dự án TLDH2-Nhóm Latex, Kiều Ngân]%[2D2B5-6]%Câu 7.
	Một người mỗi tháng đều đặn gửi vào ngân hàng một khoản tiền $T$ theo hình thức lãi kép với lãi suất $0{,}6\%$ mỗi tháng. Biết sau $15$ tháng, người đó có số tiền là $100$ triệu đồng. Hỏi số tiền $T$ gần với số tiền nào nhất trong các số sau?
	\loigiai{
		Với số tiền $T$ gửi đều đặn mỗi tháng theo hình thức lãi kép với lãi suất $r\%$ mỗi tháng, ta có\\
		Sau một tháng, số tiền của người đó là $A_1=T(1+r)$ đồng.\\
		Sau hai tháng, số tiền của người đó là $A_2=[T(1+r)+T](1+r)=T\left[(1+r)^2+(1+r)\right]$ đồng.\\
		Sau ba tháng, số tiền của người đó là
		$$A_3=\left\{T\left[(1+r)^2+(1+r)\right]+T\right\}(1+r)=T\left[(1+r)^3+(1+r)^2+(1+r)\right]\text{ đồng}.$$
		\ldots \\
		Sau mười lăm tháng, số tiền của người đó là
		$$A_{15}=T\left[(1+r)^{15}+(1+r)^{14}+\cdots +(1+r)\right]=\dfrac{T}{r}(1+r)\left[(1+r)^{15}-1\right]\text{ đồng}.$$
		Khi đó $T=\dfrac{A_{15}\cdot r}{(1+r)\left[(1+r)^{15}-1\right]}=\dfrac{10^8\cdot 0{,}006}{1{,}006\cdot (1{,}006^{15}-1)}\approx 6.350.000$ đồng.}
\end{bt}

\Closesolutionfile{ans}
% \inputans{10}{ans/ans-0-GK1-CanhDieu-De1-NH23-24}
% \begin{name}
	{\tenchude}
	{TOÁN 11}
	{LỚP TOÁN THẦY PHÁT}
	{Thời gian: 90 phút - Không kể thời gian phát đề}
\end{name}
\setcounter{ex}{0}\setcounter{bt}{0}
\noindent{\bf\fontfamily{qag}\selectfont\color{violet}A. PHẦN TRẮC NGHIỆM}
\Opensolutionfile{ans}[ans/ans-1-GK1-KNTT-De17-NH23-24]
%%==========Câu 1
\begin{ex}%[1K1Y1-1]
Số đo bằng độ của cung lượng giác $\dfrac{\pi}{12}$ là
\choice
{$\left(\dfrac{75}{2}\right)^{\circ}$}
{$45^{\circ}$}
{$-345^{\circ}$}
{\True $15^{\circ}$}
\loigiai{
	Ta có $\dfrac{\pi}{12}=\dfrac{\pi}{12} \cdot\left(\dfrac{180}{\pi}\right)^o=15^{\circ}.$
}
	\end{ex}
%%%Câu 2
\begin{ex}%[1K1Y1-1] 
Đổi $80^{\circ}$ sang radian
	\choice
{\True $\dfrac{4 \pi}{9}$}
{$\dfrac{2 \pi}{9}$}
{$\dfrac{\pi}{9}$}
{$\dfrac{5 \pi}{9}$}
\loigiai{
 Ta có : $80^{\circ}=80 \cdot \dfrac{\pi}{180}=\dfrac{4 \pi}{9}$.
}
\end{ex}
%%%Câu 3
\begin{ex}%[1K1Y1-3] 
Cung tròn bán kính bằng $8$ cm có số đo $3$ rad có độ dài là
	\choice
	{$\dfrac{8}{3}$ cm}
	{$\dfrac{3}{11}$ cm}
	{$11$ cm}
	{\True $24$ cm}
	\loigiai{
		Độ dài cung tròn $l=R\alpha=8\cdot 3=24$ cm.
	}
\end{ex}
	
%%==========Câu 4
\begin{ex}%[1K1Y1-6]
	Cho góc $\alpha$ thỏa mãn $\sin \alpha=\dfrac{4}{5}$ và $\dfrac{\pi}{2}<\alpha<\pi$. Tính $P=\dfrac{1}{1+\tan^2\alpha}$
	\choice
	{$P=-\dfrac{3}{5}$}
	{$P=\dfrac{3}{5}$}
	{$P=-\dfrac{9}{25}$}
	{\True $P=\dfrac{9}{25}$}
	\loigiai{
Ta có $\cos^2 x=1-\sin^2x=1-\left(\dfrac{4}{5}\right)^2=\dfrac{9}{25}$ nên $1+\tan^2x=\dfrac{1}{\cos^2x}=\dfrac{25}{9}$.\\
Do đó $P=\dfrac{1}{1+\tan^2x}=\dfrac{9}{25}$.
}
\end{ex}
%%%%Câu 5
\begin{ex}%[1K1B1-6]
Cho $\sin a=\dfrac{1}{3}$. Tính $P=\dfrac{3 \cot a+2 \tan a}{\cot a+\tan a}$
\choice
{$P=\dfrac{9}{26}$}
{\True $P=\dfrac{26}{9}$}
{$P=-6$}
{$P=6$}
\loigiai{
Ta có 
\[
P=\dfrac{3 \cot a+2 \tan a}{\cot a+\tan a}=\dfrac{3 \cot ^2 a+2}{\cot ^2 a+1}=\dfrac{3\left(\dfrac{1}{\sin ^2 a}-1\right)+2}{\dfrac{1}{\sin ^2 a}-1+1}=\dfrac{3 \dfrac{1}{\sin ^2 a}-1}{\dfrac{1}{\sin ^2 a}}=\dfrac{26}{9}
\]
}
\end{ex}
%%%%Câu 6
\begin{ex}%[1K1Y2-1]
Trong các công thức sau, công thức nào đúng?
\choice
{$\cos (a-b)=\cos a \cdot \cos b-\sin a \cdot \sin b$}
{\True $\cos (a-b)=\cos a \cdot \cos b+\sin a \cdot \sin b$}
{$\sin (a+b)=\sin a \cdot \cos b-\cos a \cdot \sin b$}
{$\sin (a-b)=\sin a \cdot \cos b+\cos a \cdot \sin b$}
\loigiai{
}
\end{ex}
%%%%Câu 7
\begin{ex}%[1T1Y3-2]
Trong các công thức sau, công thức nào \textbf{sai}?
\choice
{$\cos 2 a=\cos ^2 a-\sin ^2 a$}
{\True $\cos 2 a=\cos ^2 a+\sin ^2 a$}
{$\cos 2 a=2 \cos ^2 a-1$}
{$\cos 2 a=1-2 \sin ^2 a$}
\loigiai{

}
\end{ex}
%%%%Câu 8
\begin{ex}%[1K1Y2-2]
Cho $\cos 2 \alpha=\dfrac{1}{2}$. Tính giá trị biểu thức $P=5 \sin ^2 \alpha-4 \cos ^2 \alpha$.
\choice
{$\dfrac{7}{4}$}
{$-\dfrac{5}{8}$}
{\True $-\dfrac{7}{4}$}
{$\dfrac{1}{8}$}
 \loigiai{
  	$$
\begin{aligned}
	P=5 \sin ^2 \alpha-4 \cos ^2 \alpha & =5\left(\frac{1-\cos 2 \alpha}{2}\right)-4\left(\frac{1+\cos 2 \alpha}{2}\right) \\
	& =\frac{1}{2}-\frac{9}{2} \cos 2 \alpha=\frac{1}{2}-\frac{9}{2} \cdot \frac{1}{2}=-\frac{7}{4} .
\end{aligned}
$$
}
\end{ex}
%%%%%câu 9
\begin{ex}%[1K1B2-1]
Cho hai góc $\alpha, \beta$ thỏa mãn $\sin \alpha=\dfrac{5}{13},\left(\dfrac{\pi}{2}<\alpha<\pi\right)$ và $\cos \beta=\dfrac{3}{5},\left(0<\beta<\dfrac{\pi}{2}\right)$. Tính giá trị đúng của $\cos (\alpha-\beta)$.
\choice
{$\dfrac{16}{65}$}
{\True $-\dfrac{16}{65}$}
{$\dfrac{18}{65}$}
{$-\dfrac{18}{65}$}
\loigiai{
Ta có\\
$\sin \alpha=\dfrac{5}{13},\left(\dfrac{\pi}{2}<\alpha<\pi\right)$ nên $\cos \alpha=-\sqrt{1-\sin ^2 \alpha}=-\sqrt{1-\left(\dfrac{5}{13}\right)^2}=-\dfrac{12}{13}$.\\
$\cos \beta=\dfrac{3}{5},\left(0<\beta<\dfrac{\pi}{2}\right)$ nên $\sin \beta=\sqrt{1-\cos ^2 \beta}=\sqrt{1-\left(\dfrac{3}{5}\right)^2}=\dfrac{4}{5}$.\\
Do đó $\cos (\alpha-\beta)=\cos \alpha \cos \beta+\sin \alpha \sin \beta=-\dfrac{12}{13} \cdot \dfrac{3}{5}+\dfrac{5}{13} \cdot \dfrac{4}{5}=-\dfrac{16}{65}$.
}
\end{ex}
%%%%Câu 10
\begin{ex}%[1K1K2-3]
Rút gọn biểu thức: $\dfrac{\sin a+\sin 3 a+\sin 5 a}{\cos a+\cos 3 a+\cos 5 a}$.
\choice
{\True $\tan 3a$}
{$\tan a$}
{$2 \tan 3 a$}
{$\cot 3 a$}
 \loigiai{
 	Ta có
	$$
	\begin{aligned}
		 \dfrac{\sin a+\sin 3 a+\sin 5 a}{\cos a+\cos 3 a+\cos 5 a}&=\dfrac{\sin a+\sin 5 a+\sin 3 a}{\cos a+\cos 5 a+\cos 3 a}=\dfrac{2 \sin 3 a \cdot \cos a+\sin 3 a}{2 \cos 3 a \cdot \cos a+\cos 3 a} \\
		&= \dfrac{\sin 3 a \cdot(2 \cos a+1)}{\cos 3 a \cdot(2 \cos a+1)}=\tan 3 a .
	\end{aligned}
	$$
}
\end{ex}
%%%%Câu 11
\begin{ex}%[1K1Y3-1]
Xét bốn mệnh đề sau:
\begin{enumerate}
	\item [(1)] Hàm số $y=\sin x$ có tập xác định là $\mathbb{R}$.
	\item [(2)] Hàm số $y=\cos x$ có tập xác định là $\mathbb{R}$.
	\item [(3)] Hàm số $y=\tan x$ có tập xác định là $D=\mathbb{R} \backslash\left\{\dfrac{\pi}{2}+k \pi \mid k \in \mathbb{Z}\right\}$.
	\item [(4)] Hàm số $y=\cot x$ có tập xác định là $D=\mathbb{R} \backslash\left\{k \dfrac{\pi}{2} \mid k \in \mathbb{Z}\right\}$.
\end{enumerate}
Số mệnh đề đúng là 
\choice
{\True $3$}
{$2$}
{$1$}
{$4$}
\loigiai{
Các mệnh đề đúng là:
\begin{enumerate}[1)]
	\item  Hàm số $y=\sin x$ có tập xác định là $\mathbb{R}$.
\item (2) Hàm số $y=\cos x$ có tập xác định là $\mathbb{R}$.
\item (3) Hàm số $y=\tan x$ có tập xác định là $D=\mathbb{R} \backslash\left\{\frac{\pi}{2}+k \pi \mid k \in \mathbb{Z}\right\}$.
\end{enumerate}
}
\end{ex}
%%%%Câu 12
\begin{ex}%[1K1Y3-4]
Chu kỳ tuần hoàn của hàm số $y=\tan x$ là
\choice
{$k \pi, (k \in \mathbb{Z})$}
{\True $\pi$}
{$\dfrac{\pi}{3}$}
{$3 \pi$}
\loigiai{
	Hàm số $y=\tan x$ tuần hoàn với chu kỳ là $\pi$.
}
\end{ex}
%%%Câu 13
\begin{ex}%[1K1B3-4]
Tìm chu kì của hàm số $f(x)=\sin \dfrac{x}{2}+2 \cos \dfrac{3 x}{2}$.
\choice
{$5 \pi$}
{$\dfrac{\pi}{2}$}
{\True $4 \pi$}
{$2 \pi$}
\loigiai{
Chu kỳ của $\sin \dfrac{x}{2}$ là $T_1=\dfrac{2 \pi}{\left|\dfrac{1}{2}\right|}=4 \pi$ và chu kỳ của $\cos \dfrac{3 x}{2}$ là $T_2=\dfrac{2 \pi}{\left|\dfrac{3}{2}\right|}=\dfrac{4 \pi}{3}$.\\
Chu kì của hàm ban đầu là bội chung nhỏ nhất của hai chu kì $T_1$ và $T_2$ vừa tìm được ở trên. Do đó chu kì của hàm ban đầu $T=4 \pi$.  
}
\end{ex}
%%%%Câu 14
\begin{ex}%[1K1B3-3]
Xét tính chẵn lẻ của hàm số $y=\dfrac{\sin 2 x}{2 \cos x-3}$ thì $y=f(x)$ là
\choice
{Hàm số chẵn}
{\True Hàm số lẻ}
{Không chẵn không lẻ}
{Vừa chẵn vừa lẻ}
 \loigiai{
 Tập xác định $\mathscr{D}=\mathbb{R}$.\\
 Ta có $\forall x \in \mathscr{D} \Rightarrow-x \in \mathscr{D}$.\\
 $f(-x)=\dfrac{\sin (-2 x)}{2 \cos (-x)-3}=\dfrac{-\sin 2 x}{2 \cos x-3}=-f(x)$.\\
 Vậy hàm số đã cho là hàm số lẻ.}
\end{ex}
%%%%Câu 15
\begin{ex}%[1K1B3-5]
Giá trị nhỏ nhất của hàm số $y=2 \cos ^2 x-\sin 2 x+5$.
\choice
{$\sqrt{2}$}
{$-\sqrt{2}$}
{\True $6-\sqrt{2}$}
{$6+\sqrt{2}$}
\loigiai{
Ta có $y=2 \cos ^2 x-\sin 2 x+5=\cos 2 x-\sin 2 x+6=\sqrt{2} \cos \left(2 x+\dfrac{\pi}{4}\right)+6$.\\
	Do $-\sqrt{2} \leq \sqrt{2} \cos \left(2 x+\dfrac{\pi}{4}\right) \leq \sqrt{2}$ nên $-\sqrt{2}+6 \leq \sqrt{2} \cos \left(2 x+\dfrac{\pi}{4}\right)+6 \leq \sqrt{2}+6$.\\
	Vậy giá trị nhỏ nhất của hàm số $y=2 \cos ^2 x-\sin 2 x+5$ là $6-\sqrt{2}$.
}
\end{ex}
%%%Câu 16
\begin{ex}%[1K1Y4-3]
Cho $x=\dfrac{\pi}{2}+k 2 \pi, k \in \mathbb{Z}$ là nghiệm của phương trình nào sau đây
\choice
{$\sin x=0$}
{\True $\sin x=1$}
{$\sin x=-1$}
{$\cos x=1$}
 \loigiai{
	$$
	\sin x=1 \Leftrightarrow x=\frac{\pi}{2}+k 2 \pi, k \in \mathbb{Z}.
	$$
}
\end{ex}
%%%Câu 17
\begin{ex}%[1K1Y4-3]
Trong các phương trình sau, phương trình nào vô nghiệm?
\choice 
{$\sin x=\dfrac{1}{2}$}
{\True $\sin x=\dfrac{5}{3}$}
{$\tan x=-2023$}
{$\cos x=\dfrac{3}{5}$}
 \loigiai{
Phương trình $\sin x=a$ có nghiệm khi $|a| \leq 1$.
}
\end{ex}
%%%%Vâu 18
\begin{ex}%[1K1Y4-3]
Phương trình $\sin x=m-1$ có nghiệm khi $m$ là
\choice
{$-1 \leq m \leq 1$}
{\True $0 \leq m \leq 2$}
{$m \leq 0$}
{$-1 \leq m \leq 0$}
\loigiai{
Phương trình $\sin x=m-1$ có nghiệm khi $-1 \leq m-1 \leq 1 \Leftrightarrow 0 \leq m \leq 2$
}
\end{ex}
%%%Câu 19
\begin{ex}%[1K1B4-3]
Phương trình $1+2 \sin x \cos x=0$ có nghiệm là
\choice
{$x=\dfrac{\pi}{2}+k 2 \pi$}
{\True $x=-\dfrac{\pi}{4}+k \pi$}
{$x=-\dfrac{\pi}{3}+k 2 \pi$}
{$x=-\dfrac{\pi}{3}+k \pi$}
 \loigiai{
Phương trình: $1+2 \sin x \cos x=0 \Leftrightarrow \sin 2 x=-1 \Leftrightarrow 2 x=-\dfrac{\pi}{2}+k 2 \pi \Leftrightarrow x=-\dfrac{\pi}{4}+k \pi$
}
\end{ex}
%%%Câu 20
\begin{ex}%[1K1Y4-A]
Phương trình $(2 \cos x+1)(\cos 2 x-\sqrt{3})=0$ có nghiệm là
\choice
{$x=\dfrac{\pi}{2}+k 2 \pi$}
{\True $x= \pm \dfrac{2 \pi}{3}+k 2 \pi$}
{$x= \pm \dfrac{\pi}{4}+k 2 \pi$}
{$x= \pm \dfrac{\pi}{6}+k 2 \pi$}
  \loigiai{
Phương trình 
\allowdisplaybreaks
\begin{eqnarray*}
(2 \cos x+1)(\cos 2 x-\sqrt{3})=0 \Leftrightarrow \hoac{&\cos x=-\dfrac{1}{2} \\ &\cos 2 x=\sqrt{3} \quad (\text{Vô Nghiệm})}
\end{eqnarray*}
Với $	\cos x=-\dfrac{1}{2} \Leftrightarrow x= \pm \dfrac{2 \pi}{3}+k 2 \pi
	$}
\end{ex}
%%%Câu 21
\begin{ex}%[1K1B4-A]
\textit{(Bảng số liệu sau dùng cho câu 21-24)}
Quãng đường (km) các cầu thủ (không tính thủ môn) chạy trong một trận bóng đá tại giải ngoại hạng Anh được cho trong bảng thống kê sau:
	\begin{center}
		\begin{tabular}{|c|c|c|c|c|c|}
			\hline Quãng đường & {$[2 ; 4)$} & {$[4 ; 6)$} & {$[6 ; 8)$} & {$[8 ; 10)$} & {$[10 ; 12)$} \\
			\hline Số cầu thủ & 2 & 5 & 6 & 9 & 3 \\
			\hline
		\end{tabular}
	\end{center}
	Tính quãng đường trung bình một cầu thủ chạy trong trận đấu này.
	\choice{$7{,}02$}{\True $7{,}48$}{$5{,}23$}{$8{,}36$}
		\loigiai{
			Tổng số cầu thủ là $n=2+5+6+9+3=25$.\\
			Quãng đường trung bình một cầu thủ chạy trong trận đấu này là
			$$\overline{x}=\dfrac{2\cdot 3+5\cdot 5+6\cdot 7+9\cdot 9+3\cdot 11}{25}=7{,}48~(\mathrm{km}).$$ 
		}
\end{ex}
%%%%Câu 22
\begin{ex}%[1K2Y5-2]
	Tìm trung vị của mẫu số liệu.
	\choice{\True $7{,}83$}{$7{,}48$}{$6{,}23$}{$3{,}56$}
		\loigiai{
			Cỡ mẫu $n = 2 + 5 + 6 + 9 + 3 = 25$.\\
			Gọi $x_1, x_2, \ldots, x_{25}$ là quãng đường chạy của $25$ cầu thủ và giả sử dãy này đã được sắp xếp theo thứ tự không giảm. Khi đó, trung vị là $x_{13}$, mà $x_{13}$ thuộc nhóm $[6; 8)$ nên nhóm này chứa trung vị. Do đó, trung vị là
			$$	M_e=6+\dfrac{\dfrac{25}{2}-(2+5)}{6}\cdot (8-6) \approx 7{,}83.	$$			
		}
\end{ex}
%%%Câu 23
\begin{ex}%[1K2Y5-2]
Tìm $a$ sao cho có $25 \%$ số cầu thủ tham gia trận đấu chạy ít nhất $a~(\mathrm{km})$.
\choice
{\True $9{,}28$}
{$7{,}48$}
{$12{,}23$}
{$13{,}56$}
		\loigiai{
			Số $a$ thỏa mãn có $25 \%$ số cầu thủ tham gia trận đấu chạy ít nhất $a$ (km).\\
			Do đó, $a$ chính là tứ phân vị thứ ba của mẫu số liệu trên.\\
			Cỡ mẫu $n=25$.\\
			Gọi $x_1, x_2 \ldots, x_{25}$ là quãng đường chạy của $25$ cầu thủ và giả sử dãy này đã được sắp xếp theo thứ tự không giảm. Khi đó tứ phân vị thứ ba là $\dfrac{x_{19}+x_{20}}{2}$.\\ Do $x_{19}, x_{20}$ đều thuộc nhóm $[8 ; 10)$ nên nhóm này chứa tứ phân vị thứ ba.\\ Do đó $a=Q_3=8+\dfrac{\dfrac{3.25}{4}-(2+5+6)}{9} \cdot(10-8) \approx 9{,}28$. 
		}
\end{ex}
%%%%Câu 24
\begin{ex}%[1K2B5-3]
Tính mốt của mẫu số liệu thu được.
\choice{$9{,}28$}{$7{,}48$}{\True $8{,}67$}{$13{,}56$}
		\loigiai{
			Tần số lớn nhất là $9$ nên nhóm chứa mốt là $[8; 10)$.\\
			Mốt là $M_o=8+\dfrac{(9-6)}{(9-6)+(9-3)} \cdot 2 \approx 8{,}67$.\\
}
\end{ex}
%%%%%Câu 25
\begin{ex}%[1K2B5-4]
Cho dãy số $\left(u_n\right)$ biết $u_n=\dfrac{4 n+5}{n+1}$. Mệnh đề nào sau đây đúng?
\choice
{Dãy số bị chặn trên}
{Dãy số bị chặn dưới}
{\True Dãy số bị chặn}
{Không bị chặn}
\loigiai{
Ta có $u_n=\dfrac{4 n+5}{n+1}>0, \forall n \in \mathbb{N}^*$. Khi đó
$$
u_n=\dfrac{4 n+5}{n+1}=\dfrac{4(n+1)+1}{n+1}=4+\dfrac{1}{n+1} \leq 4+\dfrac{1}{2}=\dfrac{9}{2} \Rightarrow u_n \leq \dfrac{9}{2}, \forall n \in \mathbb{N}^*
$$
Suy ra $0<u_n \leq \dfrac{9}{2}, \forall n \in \mathbb{N}^*$.\\
Vậy dãy số $\left(u_n\right)$ bị chặn.
}
\end{ex}
%%%%Câu 26
\begin{ex}%[1K2Y6-3]
Cho cấp số cộng có số hạng đầu $u_1=-\dfrac{1}{2}$, công sai $d=\dfrac{1}{2}$. Năm số hạng liên tiếp đầu tiên của cấp số này là
\choice
{$-\dfrac{1}{2}; 0; 1; \dfrac{1}{2}; 1$}
{$-\dfrac{1}{2}; 0; \dfrac{1}{2}; 0; \dfrac{1}{2}$}
{$\dfrac{1}{2}; 1; \dfrac{3}{2}; 2; \dfrac{5}{2}$}
{\True $-\dfrac{1}{2}; 0; \dfrac{1}{2}; 1; \dfrac{3}{2}$}
 \loigiai{
Ta có $u_1=-\dfrac{1}{2}$ và $d=\dfrac{1}{2}$ nên  \[\heva{&u_1=-\dfrac{1}{2} \\ &u_2=u_1+d=0 \\ &u_3-u_2+d=\dfrac{1}{2} \\ &u_4=u_3+d=1 \\ &u_5=u_4+d=\dfrac{3}{2}}.\]
Vậy $5$ số hạng cần tìm là $-\dfrac{1}{2}; 0; \dfrac{1}{2}; 1; \dfrac{3}{2}$.
}
\end{ex}
%%%%Câu 27
\begin{ex}%[1K2Y6-3]
Cho cấp số cộng $\left(u_n\right)$ biết $u_1=2$ và công sai $d=3$. Tính $u_2$ bằng
\choice
{\True $5$}
{$6$}
{$7$}
{$8$}
\loigiai{
	Ta có: $u_2=u_1+d=2+3=5$.
}
\end{ex}
%%%Câu 28
\begin{ex}%[1K2B6-1]
Cho cấp số cộng $\left(u_n\right)$ thỏa mãn $\heva{&u_4=10 \\& u_4+u_6=26}$ có công sai là
\choice
{$d=-3$}
{\True $d=3$}
{$d=5$}
{$d=6$}
\loigiai{
Ta có: $\heva{&u_4=10 \\& u_4+u_6=26} \Leftrightarrow\heva{&u_1+3 d=10 \\& 2 u_1+8 d=26} \Leftrightarrow\heva{&u_1=1 \\& d=3.}$
\\Vậy công sai $d=3$.
}
\end{ex}
%%%%Câu 29
\begin{ex}%[1K2B6-1]
Cho cấp số cộng $\left(u_n\right)$ biết $u_1=1$ và $u_4=10$. Công sai $d$ bằng
\choice
{$-2$}
{$1$}
{\True $3$}
{$-4$}
\loigiai{
Ta có $u_4=u_1+3 d=1+3 d=10 \Leftrightarrow d=3$.
}
\end{ex}
%%%%Câu 30
\begin{ex}%[1K2B6-5]
Tính tổng $K=15+20+25+\ldots+7510$.
\choice
{$5\,634\,750$ }
{$5\,643\,705$}
{$5\,643\,250$}
{\True $5\,643\,750$}
 \loigiai{	Ta thấy các số hạng của tổng $K$ tạo thành một cấp số cộng với số hạng đầu $u_1=15$ và công sai $d=5$.\\
	Giả sử tổng trên có $n$ số hạng thì 
	\[u_n=7510
	\Leftrightarrow u_1+(n-1) d=7\,510 \Leftrightarrow 15+(n-1) 5=7\,510 \Leftrightarrow n=1\,500.\]
Vậy $K=S_{1\,500}=\dfrac{1\,500(15+7\,510)}{2}=5\,643\,750$.}
\end{ex}
%%%%Câu 31
\begin{ex}%[1K2Y7-1]
Cho cấp số nhân $\left(u_n\right)$ biết $u_1=3$ và công bội $q=2$. Tính $u_4$ bằng
\choice
{$9$}
{$16$}
{\True $24$}
{$48$}
 \loigiai{
	Ta có $u_4=u_1 \cdot q^3=3 \cdot 2^3=24$.
}
\end{ex}
%%%%Câu 32
\begin{ex}%[1K2Y7-1]
Cho cấp số nhân $\left(u_n\right)$ biết $u_1=5$ và $u_2=-20$. Công bội $q$ bằng
\choice
{$-2$}
{$4$}
{$3$}
{\True $-4$}
\loigiai{	Ta có $q=\dfrac{u_2}{u_1}=\dfrac{-20}{5}=-4$.
}
\end{ex}
%%%%Câu 33
\begin{ex}%[1K2Y7-3]
Cho cấp số nhân: $1 ; 2 ; 4 ; 8 ; \ldots$ Số hạng thứ năm là
\choice
{$10$}
{\True $16$}
{$12$}
{$32$}
\loigiai{	Ta có công bội $q=\dfrac{u_2}{u_1}=\dfrac{2}{1}=2$.\\
	Suy ra $u_5=u_4 \cdot q=8.2=16$.
}
\end{ex}
%%%%Câu 34
\begin{ex}%[1K2B7-5]
Cho cấp số nhân $\left(u_n\right)$ biết $u_1=5$ và công bội $q=2$. Tính tổng của $10$ số hạng đầu tiên $S_{10}$
\choice
{$4225$}
{$4115$}
{$5225$}
{\True $5115$}
 \loigiai{
Ta có: $S_{10}=u_1 \cdot \dfrac{1-q^{10}}{1-q}=5 \cdot \dfrac{1-2^{10}}{1-2}=5115$.
}
\end{ex}
%%%%Câu 35
\begin{ex}%[1K2B7-1]
Cho cấp số nhân $\left(u_n\right)$ biết $\heva{&u_4+u_6=540 \\& u_1+u_3=20}$. Công bội $q$ bằng
\choice
{$6$}
{$27$}
{$2$}
{\True$3$}
 \loigiai{
	Ta có:
	\allowdisplaybreaks
	\begin{eqnarray*}
\heva{&u_4+u_6=540 \\& u_1+u_3=20} \Leftrightarrow\heva{&u_1 q^3 \cdot\left(1+q^2\right)=540 \\& u_1\left(1+q^2\right)=20} \Leftrightarrow\heva{&q^3=27 \\& u_1\left(1+q^2\right)=20} \Leftrightarrow\heva{&q=3 \\& u_1=2.}
\end{eqnarray*}
}
\end{ex}


\Closesolutionfile{ans}
% \inputans{10}{ans/ans-1-GK1-KNTT-De17-NH23-24}
\noindent{\bf\fontfamily{qag}\selectfont\color{violet}B. PHẦN TỰ LUẬN}
\setcounter{bt}{0}
%%==========Bài 1
 \begin{bt}%[1K1Y1-6]
Cho $\sin \alpha=\dfrac{3}{5}$ và $\dfrac{\pi}{2}<\alpha<\pi$. Tính giá trị của $\cos \alpha$ ?
\loigiai{
 Ta có 
 \[\sin ^2 \alpha+\cos ^2 \alpha=1 \Rightarrow \cos ^2 \alpha=1-\sin ^2 \alpha=1-\dfrac{9}{25}=\dfrac{16}{25} \Leftrightarrow\hoac{&\cos \alpha=\frac{4}{5} \\& \cos \alpha=-\dfrac{4}{5}.}\] 
 Vì $\dfrac{\pi}{2}<\alpha<\pi$ nên $\cos \alpha<0$, do đó $\cos \alpha=-\dfrac{4}{5}$.
} 
 \end{bt}
%%%%%Bài 37
\begin{bt}%[1K2B6-6]
Trong dịp nghỉ lễ 02/9 gia đình anh An cần thuê một xe Taxi để di chuyển từ thủ đô Hà Nội về thăm quê tại TP Bắc Giang. Biết giá của kilômét đầu tiên là $10.000$ đồng, kể từ kilômét thứ 2 giá của mỗi kilômét tăng thêm $500$ đồng so với giá của kilômét trước đó. Biết quãng đường Taxi di chuyển từ thủ đô Hà Nội về TP Bắc Giang là $80$ km. Hỏi gia đình anh An phải trả bao nhiêu tiền cho chuyến Taxi đó?
\loigiai{
Số tiền ở từng kilômet lập thành một cấp số cộng.\\
Số hạng đầu tiên là $u_1=10.000$, công sai $d=500$.\\
Cấp số cộng có $80$ số hạng.\\
Tổng số tiền là $S_{80}=80.10000+\dfrac{79.80 .500}{2}=2.380 .000$ đồng.
}
\end{bt}
%%%%Bài 38
\begin{bt}%[1D1T5-6]
	\immini{
		Một chiếc guồng nước có dạng hình tròn bán kính $2{,}5$ m; trục của nó đặt cách mặt nước $2$ m (hình bên). Khi guồng quay đều, khoảng cách $h$ (m) tính từ một chiếc gầu tại điểm $A$ trên guồng đến mặt nước là $h=|y|$ trong đó $$y=2+2{,}5\sin 2\pi\left(x-\dfrac{1}{4}\right)$$ với $x$ là thời gian quay của guồng ($x\ge 0$), tính bằng phút; ta quy ước rằng $y>0$ khi gầu ở trên mặt nước và $y<0$ khi gầu ở dưới mạt nước
		\begin{enumerate}
			\item Khi nào chiếc gầu ở vị trí cao nhất? Thấp nhất?
			\item Chiếc gầu cách mặt nước $2$ mét lần đầu tiên khi nào?
		\end{enumerate}
	}
	{
		\begin{tikzpicture}[>=stealth,line join=round,line cap=round,font=\footnotesize,scale=1]
			\def\r{2.5}
			\coordinate[label=above:$O$] (O) at (0,0);
			\draw[name path=(C)] (O) circle (\r);
			\coordinate (M) at (0,-2);
			\path 
			(O)++(45:\r)coordinate(A)node[above right]{$A$}
			(-3,-2)coordinate(B)
			(3,-2)coordinate(C)
			($(B)!(A)!(C)$)coordinate(H)
			;
			\draw[dashed] (A)--(H)node[midway,right]{$h$ m}
			(B)--(C)
			;
			\draw[<->](O)--(A)node[midway,left]{$2{,}5$ m};
			\draw[<->](O)--(M)node[midway,right]{$2$ m};
			\draw[->] (A) arc (45:60:2.5);
		\end{tikzpicture}
	}
	\loigiai 
	{
		\begin{enumerate}
			\item Với mọi $x\in \mathbb{R}$, ta có
			\allowdisplaybreaks
			\begin{eqnarray*}
				-1\le \sin 2\pi\left(x-\dfrac{1}{4}\right)\le 1&\Leftrightarrow&-2{,}5\le 2{,}5\sin 2\pi\left(x-\dfrac{1}{4}\right)\le 2{,}5\\
				&\Leftrightarrow&-0{,}5\le 2+2{,5}\sin 2\pi\left(x-\dfrac{1}{4}\right)\le 4{,}5.
			\end{eqnarray*}
			Suy ra, gầu ở vị trí cao nhất khi 
			$$\sin 2\pi\left(x-\dfrac{1}{4}\right)=1\Leftrightarrow 2x\left(x-\dfrac{1}{4}\right)=\dfrac{\pi}{2}+k2\pi\Leftrightarrow x=\dfrac{1}{2}+k, \,(k\in\mathbb{Z}).$$
			Vậy gầu ở vị trí cao nhất tại các thời điểm $\dfrac{1}{2}$, $\dfrac{3}{2}$, $\dfrac{5}{2}$,$\ldots$ phút.\\
			Tương tự, gầu ở vị trí thấp nhất khi 
			$$\sin 2\pi\left(x-\dfrac{1}{4}\right)=-1\Leftrightarrow 2x\left(x-\dfrac{1}{4}\right)=-\dfrac{\pi}{2}+k2\pi\Leftrightarrow xk, \,(k\in\mathbb{Z}).$$
			Vậy gầu ở vị trí thấp nhất tại các thời điểm $0$, $1$, $2$,$\ldots$ phút.
			\item Gầu các mặt nước $2$ m nên 
			\allowdisplaybreaks
			\begin{eqnarray*}
				2+2{,}5\sin 2\pi\left(x-\dfrac{1}{2}\right)=2&\Leftrightarrow&\sin 2\pi\left(x-\dfrac{1}{2}\right)=0\\
				&\Leftrightarrow&2\pi\left(x-\dfrac{1}{4}\right)=k\pi\\
				&\Leftrightarrow&x=\dfrac{1}{4}+\dfrac{k}{2},\,(k\in\mathbb{Z}).
			\end{eqnarray*}
			Vậy chiếc gầu cách mặt nước $2$ m lần đầu tiên tại thời điểm $x=\dfrac{1}{4}$ phút.
		\end{enumerate}		
	}
\end{bt}
%%%%Bài 39
\begin{bt}%[1K2B7-7]
		Từ tờ giấy, cắt một hình tròn bán kính $R$ (cm). Tiếp theo, cắt hai hình tròn bán kính $\dfrac{R}{2}$ rồi chồng lên hình tròn đầu tiên . Tiếp theo, cắt bốn hình tròn bán  kính $\dfrac{R}{4}$ 
		rồi chồng lên các hình trước. Cứ thế tiếp tục mãi. Tính tổng diện tích của các hình tròn.
	\begin{center}
		\begin{tikzpicture}[>=stealth,line join=round,line cap=round,font=\footnotesize,scale=1,declare function={r=1.5;}]
			\begin{scope}
				\draw (0,0)circle(r);
				\path (0,-r) node[below]{$a)$};
			\end{scope}
			\begin{scope}[xshift={3.5cm}]
				\draw(0,0)circle(r);
				\draw (r/2,0)circle(r/2);
				\draw(-r/2,0)circle(r/2);
				\path (0,-r) node[below]{$b)$};
			\end{scope}
			\begin{scope}[xshift={7cm}]
				\draw (0,0)circle(r);
				\draw (r/2,0)circle(r/2);
				\draw (-r/2,0)circle(r/2);
				\foreach \i in {0,1,2,3}
				\draw[shift={(r/2*\i,0)}] (-3*r/4,0)circle(r/4);
				\path (0,-r) node[below]{$c)$};
			\end{scope}
			% \path (current bounding box.south) node[below]{Hình $3$};
		\end{tikzpicture}
	\end{center}
	\loigiai{
		Diện tích của các hình tròn trong các lần cắt là
		\begin{enumerate}
			\item Lần thứ 1: $S_1=\pi R^2$.
			\item  Lần thứ 2: $S_2=2\cdot \pi \left(\dfrac{R}{2}\right)^2= \dfrac{\pi R^2}{2}$.
			\item  Lần thứ 3: $S_2=4\cdot \pi \left(\dfrac{R}{4}\right)^2= \dfrac{\pi R^2}{2^2}$.	
			\item Lần thứ $n$: $S_n= \dfrac{\pi R^2}{2^{n-1}}$.
		\end{enumerate}
		Do đó  diện tích các hình tròn lập thành một cấp số nhân lùi vô hạn có số hạng đầu $S_1=\pi R^2$ và công bội $q=\dfrac{1}{2}$ nên tổng diện tích các hình tròn là 
		\[ S_1+S_2+\cdots=\dfrac{\pi R^2}{1-\dfrac{1}{2}}=2\pi R^2. \]
	}
\end{bt}



%Chương IV
%%Bài 10. ĐT MP
% \setcounter{section}{9} \setcounter{dang}{0}
\section{ĐƯỜNG THẲNG VÀ MẶT PHẲNG TRONG KHÔNG GIAN}
\subsection{KIẾN THỨC CẦN NHỚ}
\subsubsection{CÁC KHÁI NIỆM MỞ ĐẦU}
\begin{enumerate}[\iconMT]
	\item \indam{Mặt phẳng:} Để biểu diễn mặt phẳng, người ta dùng hình bình hành hay một miền góc
	\begin{listEX}[2]
		\item [] \begin{tikzpicture}[scale=0.7, line join=round, line cap=round]
			\tkzDefPoints{0/0/A,4/0/B,5.5/2/C,1.5/2/D}
			\tkzDrawPolygon(A,B,C,D)
			\draw (A)--(B)--(C)--(D)--cycle;
			\tkzMarkAngles[size=0.7cm,arc=l](B,A,D)
			\tkzLabelAngles[pos=0.4,rotate=30](B,A,D){\footnotesize$P$}
			\node[right] at (-0.4,-1) {Kí hiệu $(P)$ hoặc mp$(P)$};
		\end{tikzpicture}
		\item [] \begin{tikzpicture}[scale=0.7, line join=round, line cap=round]
			\tkzDefPoints{0/0/A,4/0/B,1.5/2/D}
			\tkzDrawSegments(A,B A,D)
			\tkzMarkAngles[size=0.7cm,arc=l](B,A,D)
			\tkzLabelAngles[pos=0.4,rotate=30](B,A,D){\footnotesize$\alpha$}
			\node[right] at (-0.4,-1) {Kí hiệu $(\alpha)$ hoặc mp$(\alpha)$};
		\end{tikzpicture}
	\end{listEX}
	\item \indam{Điểm thuộc mặt phẳng:}
	Cho điểm $A$, $B$ và mặt phẳng $(\alpha)$.
		\begin{tcolorbox}[colframe=\maudl,colback=cyan!3!white,boxrule=0.5mm]
	\immini{
			\begin{itemize}
				\item [\ding{172}] Khi $A$ thuộc mặt phẳng $(\alpha)$, ta kí hiệu $A \in (\alpha)$.
				\item [\ding{173}] Khi $B$ không thuộc mặt phẳng $(\alpha)$, ta kí hiệu $B \notin (\alpha)$.
				\begin{note}
					Dấu hiệu nhận biết $A \in (\alpha)$ là điểm $A$ thuộc một đường thẳng nằm trong $(\alpha)$
				\end{note}
			\end{itemize}
			}{
		\begin{tikzpicture}[scale=0.7, font=\footnotesize,>=stealth]
			\path
			%	Vẽ mp
			(0,0) coordinate (M)
			(5,0) coordinate (N)
			(6,2) coordinate (P)
			(1,2) coordinate (Q)
			(2,3) coordinate (B)
			(3,1) coordinate (A)
			(3.5,0) coordinate (A')
			;
			\draw (M)--(N)--(P)--(Q)--cycle (1.5,4)--(A) (A')--(4,-1);
			\draw[dashed] (A)--(A');
			\foreach \x/\g in {B/0,A/60}\draw[fill=black] (\x) circle (.05) +(\g:.5)node{\footnotesize$\x$};
			\draw
			pic["$P$",draw,angle radius=6mm]{angle=N--M--Q};
			\end{tikzpicture}
		}
	\end{tcolorbox}
		\item \indam{Biểu diễn hình không gian lên một mặt phẳng:}
		\begin{tcolorbox}[colframe=\maudl,colback=cyan!3!white,boxrule=0.5mm]
			\begin{itemize}
				\item[\ding{172}] Dùng nét vẽ liền để biểu diễn cho những đường trông thấy và dùng nét đứt đoạn (- - - -) để biểu diễn cho những đường bị che khuất.
				\item[\ding{172}]  Quan hệ thuộc, song song được giữ nguyên, nghĩa là
				\begin{itemize}
					\item Nếu hình thực tế điểm $A$ thuộc đường thẳng $\Delta$ thì hình biểu diễn phải giữ nguyên quan hệ đó.
					\item Nếu hình thực tế hai đường thẳng song song thì hình biểu diễn phải giữ nguyên quan hệ đó.
				\end{itemize}
			\end{itemize}
		\end{tcolorbox}
	\indamm{Hình biểu diễn của các mô hình không gian thường gặp:}\
	\begin{center}
		\begin{tikzpicture}[scale=0.5, line join=round, line cap=round]
			\tkzDefPoints{0/0/B,1.3/-1.6/C,4.5/0/D,1/3.5/A}
			\tkzDrawPolygon(A,B,C,D)
			\tkzDrawSegments(A,C)
			\tkzDrawSegments[dashed](B,D)
			\tkzDrawPoints[fill=black,size=4](A,B,C,D)
			\tkzLabelPoints[above,font=\footnotesize](A)
			\tkzLabelPoints[below,font=\footnotesize](C)
			\tkzLabelPoints[left,font=\footnotesize](B)
			\tkzLabelPoints[right,font=\footnotesize](D)
			\node[below right] at (0,-2.4) {Hình tứ diện};
		\end{tikzpicture}
		\begin{tikzpicture}[scale=0.5, line join=round, line cap=round]
			\tkzDefPoints{0/0/A,-1.3/-1.6/B,2.5/-1.6/C}
			\coordinate (D) at ($(A)+(C)-(B)$);
			\coordinate (S) at ($(A)+(100:3)$);
			\tkzDrawPolygon(S,B,C,D)
			\tkzDrawSegments(S,C)
			\tkzDrawSegments[dashed](A,S A,B A,D)
			\tkzDrawPoints[fill=black,size=4](D,C,A,B,S)
			\tkzLabelPoints[above,font=\footnotesize](S)
			\tkzLabelPoints[below,font=\footnotesize](A,B,C)
			\tkzLabelPoints[right,font=\footnotesize](D)
			\node[below] at (0.6,-2.4) {Hình chóp tứ giác đáy hbh};
		\end{tikzpicture}
		\begin{tikzpicture}[scale=0.5, line join=round, line cap=round]
			\tkzDefPoints{0/0/A,-1.3/-1.1/B,2/-1.1/C}
			\coordinate (D) at ($(A)+(C)-(B)$);
			\coordinate (A') at ($(A)+(0,2.5)$);
			\tkzDefPointsBy[translation=from A to A'](B,C,D){B'}{C'}{D'}
			\tkzDrawPolygon(A',B',B,C,D,D')
			\tkzDrawSegments(B',C' C',D' C,C')
			\tkzDrawSegments[dashed](A,B A,D A,A')
			\tkzDrawPoints[fill=black,size=4](A,B,D,C,A',B',C',D')
			\tkzLabelPoints[above,font=\footnotesize](A',D')
			\tkzLabelPoints[below,font=\footnotesize](A,B,C)
			\tkzLabelPoints[left,font=\footnotesize](B')
			\tkzLabelPoints[right,font=\footnotesize](C',D)
			\node[below] at (0.9,-2.4) {Hình lập phương, hộp chữ nhật};
		\end{tikzpicture}
	\end{center}	
\end{enumerate}
\subsubsection{CÁC TÍNH CHẤT THỪA NHẬN}
Xét trong không gian, ta thừa nhận các tính chất sau:
\begin{enumerate}[\iconMT]
	\item \indam{Tính chất 1:} Có một và chỉ một đường thẳng đi qua hai điểm phân biệt.
	\item \indam{Tính chất 2:} Có một và chỉ một mặt phẳng đi qua ba điểm không thẳng hàng.
	\item \indam{Tính chất 3:} Tồn tại $4$ điểm không cùng thuộc một mặt phẳng.
	\begin{tcolorbox}[colframe=\maudl,colback=cyan!3!white,boxrule=0.5mm]
		Một mặt phẳng hoàn toàn xác định nếu biết ba điểm không thẳng hàng thuộc mặt phẳng đó. Ta kí hiệu mặt phẳng đi qua ba điểm không thẳng hàng $A$, $B$, $C$ là $(A B C)$. Nếu có nhiều điểm cùng thuộc một mặt phẳng thì ta nói những điểm đó đồng phẳng. Nếu \textit{không} có mặt phẳng nào chứa các điểm đó thì ta nói những điểm đó \textit{không đồng phẳng}.
	\end{tcolorbox}
	\item \indam{Tính chất 4:} Nếu một đường thẳng có hai điểm thuộc một mặt phẳng thì tất cả các điểm của đường thẳng đều thuộc mặt phẳng đó.\\
	\begin{tcolorbox}[colframe=\maudl,colback=cyan!3!white,boxrule=0.5mm]
		Cho đường thẳng $d$ và mặt phẳng $(\alpha)$.
		\begin{itemize}
			\item [\ding{172}] Khi $d$ nằm trong $(\alpha)$, ta kí hiệu $d \subset (\alpha)$ hoặc $(P) \supset d$ .\quad (không được viết  $d \in (\alpha)$ nhé!!!)
			\item [\ding{173}] Khi $d$ không nằm trong $(\alpha)$, ta kí hiệu $d \not\subset (\alpha)$.
			\begin{note}
				Dấu hiệu nhận biết $d \subset (\alpha)$  là trên $d$ có hai điểm phân biệt thuộc $(\alpha)$
			\end{note}
		\end{itemize}
	\end{tcolorbox}
	\item \indam{Tính chất 5:} Nếu hai mặt phẳng phân biệt có điểm chung thì các điểm chung của hai mặt phẳng là một đường thẳng đi qua điểm chung đó.
	\immini{
	\begin{tcolorbox}[colframe=\maudl,colback=cyan!3!white,boxrule=0.5mm]
		Đường thẳng chung $d$ (nếu có) của hai mặt phẳng phân biệt $(P)$ và $(Q)$ được gọi là giao tuyến của hai mặt phẳng đó và kí hiệu là $d=(P) \cap(Q)$.
	\end{tcolorbox}}{
\begin{tikzpicture}[declare function={a=2.5;},font=\footnotesize]
	\path 
	(0,0) coordinate (a)
	(0,-a) coordinate (b)
	(b)+(-20:a/1.5) coordinate (c)
	($(a)+(c)-(b)$) coordinate (d)
	(b)+(220:a/1.5) coordinate (e)
	($(a)+(e)-(b)$) coordinate (f)
	($(a)!.35!(b)$) coordinate (M)
	;
	\draw (a)--(b)--(c)--(d)--cycle
	(b)--(e)--(f)--(a)
	;
	\draw pic[draw, angle radius=5mm]{angle=e--f--a};
	\path (f)+(-28:8pt) node{$\alpha$};
	\draw pic[draw, angle radius=5mm]{angle=a--d--c};
	\path (d)+(225:8pt) node{$\beta$};
	\path ($(a)!.7!(b)$)node[right]{$d$};
	\draw[fill=black] (M)node[right]{$M$} circle (1pt);
\end{tikzpicture}}
\item \indam{Tính chất 6:} Trên mỗi mặt phẳng các kết quả đã biết trong hình học phẳng đều đúng.
\end{enumerate}

\subsubsection{CÁCH XÁC ĐỊNH MỘT MẶT PHẲNG}
Ba cách xác định một mặt phẳng
\begin{tcolorbox}[colframe=\maudl,colback=cyan!3!white,boxrule=0.5mm]
	\begin{itemize}
		\item Một mặt phẳng được xác định nếu biết nó đi qua ba điểm $A,B,C$ không thẳng hàng của mặt phẳng, kí hiệu $\left(ABC\right) $.
		\item Một mặt phẳng được xác định nếu biết nó đi qua một đường thẳng $d$ và một điểm $A$ không thuộc $d,$ kí hiệu $\left(A,d\right) $.
		\item Một mặt phẳng được xác định nếu biết nó đi qua hai đường thẳng $a,b$ cắt nhau, kí hiệu $\left(a,b\right) $.
	\end{itemize}
\end{tcolorbox}
\subsubsection{HÌNH CHÓP VÀ HÌNH TỨ DIỆN}
\begin{enumerate}[\iconMT]
	\item \indam{Hình chóp:}
	\begin{itemize}
		\item [\iconCH] \indamm{Định nghĩa:} Cho đa giác $A_1A_2\ldots A_n$ và cho điểm $S$ nằm ngoài mặt phẳng chứa đa giác đó. Nối $S$ với các đỉnh $A_1,A_2,\ldots ,A_n$ ta được $n$ miền đa giác $SA_1A_2,SA_2A_3,\ldots ,SA_{n-1}A_n $.
		Hình gồm $n$ tam giác đó và đa giác $A_1A_2A_3\ldots A_n$ được gọi là hình chóp $S.A_1A_2A_3\ldots A_n $.
		\immini{\item[\iconCH] \indamm{Các tên gọi:}
			\begin{itemize}
				\item Điểm $S$ gọi là đỉnh của hình chóp.
				\item Đa giác $A_1A_2\ldots A_n$ gọi là mặt đáy của hình chóp.
				\item Các đoạn thẳng $A_{1}A_{2},A_{2}A_{3},\ldots,A_{n-1}A_{n}$ gọi là các cạnh đáy của hình chóp.
				\item Các đoạn thẳng $SA_{1},SA_{2},\ldots,SA_{n}$ gọi là các cạnh bên của hình chóp. 
				\item Các miền tam giác $SA_1A_2,SA_2A_3,\ldots ,SA_{n-1}A_n$ gọi là các mặt bên của hình chóp.	 
		\end{itemize}}{
			\begin{tikzpicture}[scale=1.3, line join=round, line cap=round]
				\tkzDefPoints{0/0/A1,0.3/-1/A2,1/-1.6/A3,1.7/-1.5/A4,2.2/-1/A5,2.5/0.3/A6,1/2/S,-1.5/-2/P,-0.7/0.3/Q,3/-2/R}
				\tkzDrawPolygon(S,A1,A2,A3,A4,A5,A6)
				\tkzDrawSegments(S,A2 S,A3 S,A4 S,A5 P,Q P,R)
				\tkzDrawSegments[dashed](A1,A6)
				\tkzDrawPoints[fill=black](S,A1,A2,A3,A4,A5,A6)
				\tkzLabelPoints[above](S)
				\tkzLabelPoints[above right](P)
				\tkzLabelPoint[left](A1){$A_1$}
				\tkzLabelPoint[left](A2){$A_2$}
				\tkzLabelPoint[below](A3){$A_3$}
				\tkzLabelPoint[below](A4){$A_4$}
				\tkzLabelPoint[right](A5){$A_5$}
				\tkzLabelPoint[right](A6){$A_6$}
				\tkzMarkAngle[size=0.6cm,opacity=.4,draw=black,mksize=2](R,P,Q)
		\end{tikzpicture}}
	\end{itemize}
	\item \indam{Hình tứ diện:}
	\begin{itemize}
		\item[\iconCH] \indamm{Định nghĩa:} 
		Cho bốn điểm $A, B, C, D$ không đồng phẳng. Hình gồm bốn tam giác $ABC$, $ACD$, $ABD$, $BCD$ được gọi là hình tứ diện và được kí hiệu là $ABCD$.
		\item[\iconCH] \indamm{Chú ý:} 
	\immini{	\begin{itemize}
			\item Hai cạnh không có đỉnh chung gọi là hai cạnh đối diện, đỉnh không nằm trên một mặt được gọi là đỉnh đối diện với mặt đó.
			\item Hình chóp tam giác còn được gọi là hình tứ diện. 
			\item Hình tứ diện có bốn mặt là những tam giác đều hay có tất cả các cạnh bằng nhau được gọi là hình tứ diện đều. 
		\end{itemize}
	}{
	\begin{tikzpicture}[scale=0.6, line join=round, line cap=round]
		\tkzDefPoints{0/0/B,1.3/-1.6/C,4.5/0/D,1/3.5/A}
		\tkzDrawPolygon(A,B,C,D)
		\tkzDrawSegments(A,C)
		\tkzDrawSegments[dashed](B,D)
		\tkzDrawPoints[fill=black,size=4](A,B,C,D)
		\tkzLabelPoints[above,font=\footnotesize](A)
		\tkzLabelPoints[below,font=\footnotesize](C)
		\tkzLabelPoints[left,font=\footnotesize](B)
		\tkzLabelPoints[right,font=\footnotesize](D)
		\node[below right] at (0,-2.4) {Hình tứ diện};
\end{tikzpicture}}
	\end{itemize}
\end{enumerate}

% \subsection{PHÂN LOẠI, PHƯƠNG PHÁP GIẢI TOÁN}
\begin{dang}{Các quan hệ cơ bản}
	\begin{itemize}
		\item [\ding{172}] Chứng minh điểm $A$ thuộc $(\alpha)$: Ta chứng tỏ điểm $A$ thuộc đường thẳng $\Delta$ nằm trong $\alpha$, nghĩa là
		$$A \in \Delta, \Delta \subset (\alpha) \Rightarrow A \in (\alpha).$$
		\item [\ding{173}] Chứng minh đường thẳng $d$ nằm trong $(\alpha)$: Ta chứng tỏ $d$ có hai điểm phân biệt cùng thuộc $(\alpha)$, nghĩa là
		$$\heva{&A \in (\alpha), B \in (\alpha) \\& A,\,B \in d} \Rightarrow d \subset (\alpha).$$
		\item [\ding{174}] Chứng minh $A$ là điểm chung của hai mặt phẳng $(\alpha)$ và $(\beta)$:Ta thường sử dụng một trong hai cách sau
		$$\heva{&A \in (\alpha) \\& A \in (\beta)} \Rightarrow A \in (\alpha) \cap  (\beta)
		\text{ hoặc } 
		\heva{&d \subset (\alpha) \\& \Delta \subset (\beta)\\& d \cap \Delta = A} \Rightarrow A \in (\alpha) \cap  (\beta).$$
	\end{itemize}
\end{dang}
\begin{vd}
	Cho tam giác $ABC$ và điểm $S$ không thuộc mặt phẳng $\left(ABC\right)$. Lấy $D, E$ là các điểm lần lượt thuộc các cạnh $SA, SB \quad (D, E \text{ khác } S)$.
	\begin{itemize}
		\item [a)] Đường thẳng $DE$ có nằm trong mặt phẳng $\left(SAB\right)$ không?
		\item [b)] Giả sử $DE$ cắt $AB$ tại $F$. Chứng minh rằng $F$ là điểm chung của hai mặt phẳng $\left(SAB\right)$ và $\left(CDE\right)$.
	\end{itemize}
\end{vd}
\begin{vd}
	Cho hình chóp $S.ABCD$, gọi $O$ là giao điểm của $AC$ và $BD$. Lấy $M, N$ lần lượt thuộc các cạnh $SA, SC$.
	\begin{itemize}
		\item[a)] Chứng minh rằng đường thẳng $MN$ nằm trong mặt phẳng $\left(SAC\right)$.
		\item[b)] Chứng minh rằng $O$ là điểm chung của hai mặt phẳng $\left(SAC\right)$ và $\left(SBD\right)$.
	\end{itemize}
\end{vd}
\begin{vd}
	Cho hình tứ diện $ABCD$. Gọi $I$ là trung điểm cạnh $CD$. Gọi $M, N$ lần lượt là trọng tâm của các tam giác $BCD, CDA$.
	\begin{itemize}
		\item[a)] Chứng minh rằng các điểm $M, N$ thuộc mặt phẳng $\left(ABI\right)$.
		\item[b)] Gọi $G$ là giao điểm của $AM$ và $BN$. Chứng minh rằng $\dfrac{GM}{GA}=\dfrac{GN}{GB}=\dfrac{1}{3}$.
		\end{itemize}
\end{vd} 
\begin{dang}{Xác định giao tuyến của hai mặt phẳng}
	Cho hai mặt phẳng $(\alpha)$ và $(\beta)$ cắt nhau. Để xác định giao tuyến của chúng, ta đi tìm hai điểm chung phân biệt. Cụ thể, ta thường gặp một trong ba trường hợp sau:
\begin{itemize}
		\item [\ding{172}] Hai mặt phẳng $(\alpha)$ và $(\beta)$ có sẵn hai điểm chung phân biệt: Khi đó giao tuyến là đường thẳng qua hai điểm chung đó.
		\item [\ding{173}] Hai mặt phẳng  $(\alpha)$ và $(\beta)$ thấy trước một điểm chung $A$:
			\begin{itemize}
				\item [$\bullet$] $A$ là điểm chung thứ nhất hay $A \in (\alpha) \cap (\beta)$.
				\item [$\bullet$] Ta tìm điểm chung thứ 2: Trong $ (\alpha)$ tìm một đường thẳng $d_1$, trong $ (\beta)$ tìm một đường thẳng $d_2$ sao cho chúng có thể cắt nhau (đồng phẳng).			Gọi $B = d_1 \cap d_2$, suy ra $B \in (\alpha) \cap (\beta)$. Vậy $AB=(\alpha) \cap (\beta)$.
		\end{itemize}
		\item [\ding{174}] Hai mặt phẳng  $(\alpha)$ và $(\beta)$ chưa thấy điểm chung: Ta mở rộng mặt phẳng để tìm điểm chung tương tự như cách tìm điểm chung ở mục số \ding{173}.
	\end{itemize}
\end{dang}

\begin{vd}
	\immini{Cho tứ giác $ABCD$ sao cho các cạnh đối không song song với nhau. Lấy một điểm $S$ không thuộc mặt phẳng $\left(ABCD\right)$. Xác định giao tuyến của 
		\begin{itemize}
			\item [a)] Mặt phẳng $\left(SAC\right)$ và mặt phẳng $\left(SBD\right)$. 
			\item [b)] Mặt phẳng $\left(SAB\right)$ và mặt phẳng $\left(SCD\right)$.
			\item [c)] Mặt phẳng $\left(SAD\right)$ và mặt phẳng $\left(SBC\right)$.  
		\end{itemize}
	}{
		\begin{tikzpicture}[scale=0.6, font=\footnotesize, line join=round, line cap=round, >=stealth]
			\tkzDefPoints{0/0/A, 1/-1.5/D, 6/0/B, 4/-2.5/C, 2/3.5/S}
			\tkzLabelPoints[above](S)
			\tkzLabelPoints[above left](A)
			\tkzLabelPoints[below left](D)
			\tkzLabelPoints[right](B,C)
			\tkzDrawPoints[fill=black](A,B,C,D)
			\tkzDrawSegments[dashed](B,A)
			\tkzDrawSegments(S,A S,B S,C S,D A,D D,C C,B)
	\end{tikzpicture}}
	\loigiai{
		\begin{itemize}
			\immini{	
				\item[a)] Gọi $H$ là giao điểm của $AC$ với $BD$. 
				Khi đó 
				$$\heva{&H\in AC\\&H\in BD}\Rightarrow H\in \left(SAC\right)\cap\left(SBD\right)\quad (1).$$
				Dễ thấy 
				$S\in \left(SAC\right)\cap\left(SBD\right)\quad (2).$\\
				Từ $(1)$ và $(2)$ suy ra $SH=\left(SBD\right)\cap\left(SAC\right)$.
				\item[b)] Gọi $K$ là giao điểm của hai đường thẳng $CD$ và $AB$.\\
				Khi đó 
				$\heva{&K\in AB\\&K\in CD}\Rightarrow K\in \left(SAB\right)\cap\left(SCD\right)\quad (3)$.\\
				Dễ thấy 
				$S\in \left(SAB\right)\cap\left(SCD\right)\quad (4)$.\\
				Từ $(3)$ và $(4)$ suy ra $SK=\left(SAB\right)\cap\left(SCD\right)$.}{
				\begin{tikzpicture}[scale=0.6, font=\scriptsize, line join=round, line cap=round, >=stealth]
					\tkzDefPoints{0/0/A, 1/-1.5/D, 6/0/B, 4/-2.5/C, 2/4/S}
					\tkzInterLL(A,C)(B,D)\tkzGetPoint{H}
					\tkzInterLL(A,B)(C,D)\tkzGetPoint{K}
					\tkzInterLL(A,D)(B,C)\tkzGetPoint{L}
					\tkzLabelPoints[above](S)
					\tkzLabelPoints[above left](A)
					\tkzLabelPoints[below left](D)
					\tkzLabelPoints[below](H,K,L)
					\tkzLabelPoints[right](B,C)
					\tkzDrawPoints[fill=black](A,B,C,D,S,H,K,L)
					\tkzDrawSegments[dashed](B,K A,C B,D S,H S,A A,D C,D)
					\tkzDrawSegments(S,C S,B S,D D,K S,K B,C D,L C,L S,L)
			\end{tikzpicture}}
			\item[c)] Gọi $L$ là giao điểm của hai đường thẳng $AD$ và $BC$.\\
			Khi đó $\heva{&L\in AD\\&K\in BC}\Rightarrow L\in \left(SAD\right)\cap\left(SBC\right)\quad (5)$.
			Mặt khác $S\in \left(SAD\right)\cap\left(SBC\right)\quad (6)$.\\
			Từ $(5)$ và $(6)$ suy ra $SL=\left(SAD\right)\cap\left(SBC\right)$.
		\end{itemize}
	}
\end{vd}

\begin{vd}
	\immini{Cho tứ diện $ABCD$. Lấy các điểm $M$ thuộc cạnh $AB$, $N$  thuộc cạnh $AC$ sao cho $MN$ cắt $BC$. Gọi $I$ là điểm bên trong tam giác $BCD$. Tìm giao tuyến của mặt phẳng $\left(MNI\right)$ với các mặt phẳng $\left(ABC\right)$, $\left(BCD\right)$, $\left(ABD\right)$, $\left(ACD\right)$.
	}{
	\begin{tikzpicture}[scale=0.7, font=\footnotesize,>=stealth]
		\path
		%	Vẽ mp
		(0,0) coordinate (B)
		(5,0) coordinate (C)
		(1.5,-1.5) coordinate (D)
		(1,4) coordinate (A)
		(2.3,-0.7) coordinate (I)
		($(A)!0.6!(B)$)coordinate (M)
		($(C)!0.7!(A)$)coordinate (N)
		;
		\draw (B)--(A)--(D)--(C)--(A) (B)--(D);
		\draw[dashed] (M)--(N)--(I)--cycle (B)--(C);
		\foreach \x/\g in {B/180,A/90,C/0,D/-90,M/180,I/30,N/30}\draw[fill=black] (\x) circle (.05) +(\g:.5)node{\footnotesize$\x$};
\end{tikzpicture}}
	\loigiai{
		\begin{itemize}
			\immini{
				\item [a)] Dễ thấy $(MNI) \cap (ABC)=MN$.
				\item [b)] Tìm $(MNI) \cap (BCD)$.
				\begin{itemize}
					\item [$\bullet$] Gọi $H$ là giao điểm của $MN$ và $BC$. Suy ra 
					$$H\in \left(MNI\right)\cap\left(BCD\right)\quad (1).$$
					\item [$\bullet$]Do $I$ là điểm trong $\triangle BCD$ nên 
					$$I\in \left(MNI\right)\cap\left(BCD\right)\quad (2).$$
				\end{itemize}
				Từ $(1)$ và $(2)$ suy ra $IH=\left(MNI\right)\cap\left(BCD\right)$.}{
				\begin{tikzpicture}[scale=0.8, font=\footnotesize, line join=round, line cap=round, >=stealth]
					\tkzDefPoints{0/0/B, 2/-2/C, 6/0/D, 2/4/A, 3/-1/I}
					\tkzDefBarycentricPoint(A=1,B=2)
					\tkzGetPoint{M}
					\tkzDefBarycentricPoint(A=3,C=2)
					\tkzGetPoint{N}
					\tkzInterLL(M,N)(C,B)\tkzGetPoint{H}
					\tkzLabelPoints[left](H)
					\tkzLabelPoints[below left](B,I)
					\tkzLabelPoints[above left](M)
					\tkzLabelPoints[above](A)
					\tkzLabelPoints[below](C)
					\tkzLabelPoints[right](D,N)
					\tkzDrawPoints[fill=black](A,B,C,D,M,N,H,I)
					\tkzDrawSegments[dashed](B,D H,I N,I M,I)
					\tkzDrawSegments(A,C A,B A,D C,D C,H N,H)
			\end{tikzpicture}}
			\immini{\item [c)] Tìm $(MNI) \cap (ABD)$.
				\begin{itemize}
					\item [$\bullet$] Gọi $E=IH\cap BD$. Ta có
					$$\heva{&E\in BD\\&E\in IH}\Rightarrow E\in \left(MNI\right)\cap\left(ABD\right)\quad (3).$$
					\item [$\bullet$] Dễ thấy $M\in \left(ABD\right)\cap\left(MNI\right)\quad (4)$.
				\end{itemize}.\\
				Từ $(3)$ và $(4)$ suy ra $ME=\left(ABD\right)\cap\left(MNI\right)$.
			}{
				\begin{tikzpicture}[scale=0.8, font=\footnotesize, line join=round, line cap=round, >=stealth]
					\tkzDefPoints{0/0/B, 2/-2/C, 6/0/D, 2/4/A, 3/-1/I}
					\tkzDefBarycentricPoint(A=1,B=2)
					\tkzGetPoint{M}
					\tkzDefBarycentricPoint(A=3,C=2)
					\tkzGetPoint{N}
					\tkzInterLL(M,N)(C,B)\tkzGetPoint{H}
					\tkzInterLL(I,H)(B,D)\tkzGetPoint{E}
					\tkzInterLL(I,H)(C,D)\tkzGetPoint{F}
					\tkzLabelPoints[left](H)
					\tkzLabelPoints[below left](B,I)
					\tkzLabelPoints[above left](M)
					\tkzLabelPoints[above](A)
					\tkzLabelPoints[below](C,E)
					\tkzLabelPoints[right](D,N)
					\tkzLabelPoints[below right](F)
					\tkzDrawPoints[fill=black](A,B,C,D,M,N,H,I,E,F)
					\tkzDrawSegments[dashed](B,D H,F M,E M,I N,I)
					\tkzDrawSegments(A,C A,B A,D C,D C,H N,H N,F)
			\end{tikzpicture}}
			\item [d)] Tìm $(MNI) \cap (BCD)$.
			\begin{itemize}
				\item [$\bullet$] Gọi $F=IH\cap CD$. Ta có
				$$\heva{&F\in CD\\&F\in IH}\Rightarrow F\in \left(MNI\right)\cap\left(ACD\right)\quad (5).$$
				\item [$\bullet$] Mặt khác: $N\in AC$ nên $N\in \left(ACD\right)$. Suy ra
				$N\in \left(MNI\right)\cap\left(ACD\right)\quad (6)$.\\
				Từ $(5)$ và $(6)$ suy ra $NF=\left(ACD\right)\cap\left(MNI\right)$.
			\end{itemize}
		\end{itemize}
	}
\end{vd}

\begin{vd}
	Cho tứ diện $ABCD$. Gọi $I$, $J$ lần lượt là trung điểm các cạnh $AD$, $BC$.  
	\begin{itemize}
		\item [a)] Tìm giao tuyến của hai mặt phẳng $\left(IBC\right)$ và mặt phẳng $\left(JAD\right)$. 
		\item [b)] Lấy điểm $M$ thuộc cạnh $AB$, $N$ thuộc cạnh $AC$ sao cho $M$, $N$ không là trung điểm. Tìm giao tuyến của  mặt phẳng $\left(IBC\right)$ và mặt phẳng $\left(DMN\right)$. 
	\end{itemize}
	\loigiai{
		\immini{
			\begin{itemize}
				\item [a)] Do giả thiết $I\in AD$ nên $I\in \left(JAD\right)$. 
				Suy ra 
				$$I\in \left(BCI\right)\cap\left(ADJ\right)\quad (1).$$
				Tương tự, ta có $J\in \left(BCI\right)\cap\left(ADJ\right)\quad (2)$.\\
				Từ $(1)$ và $(2)$ suy ra $IJ=\left(BCI\right)\cap \left(ADJ\right)$.
				\item [b)] Gọi $E=DM\cap BI$. Khi đó 
				$$\heva{&E\in BI\\&E\in DM}\Rightarrow E\in \left(MND\right)\cap\left(IBC\right)\quad (3).$$
				Tương tự, gọi $F=DN\cap CI$ suy ra
				$$F\in \left(BCI\right)\cap\left(MND\right)\quad (4).$$
				Từ $(3)$ và $(4)$ suy ra $EF=\left(BCI\right)\cap \left(MND\right)$.
			\end{itemize}
		}{\begin{tikzpicture}[scale=0.8, font=\footnotesize, line join=round, line cap=round, >=stealth]
				\tkzDefPoints{0/0/B, 2/-2/C, 6/0/D, 2/4/A}
				\tkzDefMidPoint(A,D)\tkzGetPoint{I}
				\tkzDefMidPoint(B,C)\tkzGetPoint{J}
				\tkzDefBarycentricPoint(A=1,B=3)
				\tkzGetPoint{M}
				\tkzDefBarycentricPoint(A=3,C=2)
				\tkzGetPoint{N}
				\tkzInterLL(D,N)(C,I)\tkzGetPoint{F}
				\tkzInterLL(D,M)(B,I)\tkzGetPoint{E}
				\tkzLabelPoints[below left](B,J)
				\tkzLabelPoints[above left](M)
				\tkzLabelPoints[above](A)
				\tkzLabelPoints[below](C,E)
				\tkzLabelPoints[right](D,N)
				\tkzLabelPoints[above right](I,F)
				\tkzDrawPoints[fill=black](A,B,C,D,I,J,M,N,E,F)
				\tkzDrawSegments[dashed](B,D B,I D,J M,D I,J E,F)
				\tkzDrawSegments(A,C A,B A,D C,D B,C C,I A,J M,N N,D)
			\end{tikzpicture}
		}
	}
\end{vd}

\begin{vd}
	Cho tứ diện $ABCD$, $M$ là một điểm bên trong tam giác $ABD$, $N$ là một điểm bên trong tam giác $ACD$. Tìm giao tuyến của các cặp mặt phẳng sau
	\begin{enumEX}[]{2}
		\item $(AMN)$ và $(BCD)$.
		\item $(DMN)$ và $(ABC)$.
	\end{enumEX}
	\loigiai{
		\immini
		{
			\begin{enumEX}[]{1}
				\item Tìm $(AMN)\cap (BCD)$.\\
				Trong $(ABD)$, gọi $E=AM\cap BD$.\\
				Ta có $\heva{& E\in AM\subset (AMN) \\ & E\in BD\subset (BCD)}\Rightarrow E\in (AMN)\cap (BCD)$ $(1)$.\\
				Trong $(ACD)$, gọi $F=AN\cap CD$.\\
				Ta có $\heva{& F\in AN\subset (AMN) \\ & F\in CD\subset (BCD)}\Rightarrow F\in (AMN)\cap (BCD)$ $(2)$.\\
				Từ $(1)$ và $(2)$ suy ra $(AMN)\cap (BCD)=EF$.
				\item Tìm $(DMN)\cap (ABC)$.\\
				Trong $(ABD)$, gọi $P=DM\cap AB$.\\
				Ta có $\heva{& P\in DM\subset (DMN) \\ & P\in AB\subset (ABC)}\Rightarrow P\in (DMN)\cap (ABC)$ $(3)$.\\
				Trong $(ACD)$, gọi $Q=DN\cap AC$.\\
				Ta có $\heva{& Q\in DN\subset (DMN) \\ & Q\in AC\subset (ABC)}\Rightarrow Q\in (DMN)\cap (ABC)$ $(4)$.\\
				Từ $(3)$ và $(4)$ suy ra $(DMN)\cap (ABC)=PQ$.
			\end{enumEX}
		}
		{
			\begin{tikzpicture}
				[scale=1, font=\footnotesize, line join=round, line cap=round, >=stealth]
				\tkzDefPoints{0/0/B,3/-2/C,5/0/D,2/4.5/A}
				\coordinate (E) at ($(B)!1/3!(D)$);
				\coordinate (F) at ($(C)!2/3!(D)$);
				\coordinate (P) at ($(A)!1.3/3!(B)$);
				\coordinate (Q) at ($(A)!3/5!(C)$);
				\tkzInterLL(A,E)(P,D)\tkzGetPoint{M}
				\tkzInterLL(A,F)(Q,D)\tkzGetPoint{N}
				\tkzDrawPolygon(A,B,C,D)
				\tkzDrawSegments(A,C A,F P,Q Q,D)
				\tkzDrawSegments[dashed](B,D A,E E,F P,D)
				\tkzDrawPoints[fill=black](A,B,C,D,Q,P,M,N,E,F)
				\tkzLabelPoints[above](A)
				\tkzLabelPoints[below](C,E)
				\tkzLabelPoints[left](B,P,Q)
				\tkzLabelPoints[right](D,F)
				\tkzLabelPoints[right](M)
				\tkzLabelPoints[right](N)
			\end{tikzpicture}
		}
	}
\end{vd}

\begin{vd}
	\immini{Cho hình chóp $S.ABCD$ đáy là hình bình hành tâm $O$. Gọi $M$, $N$, $P$ lần lượt là trung điểm của cạnh $BC$, $CD$, $SA$. Tìm giao tuyến của 
		\begin{tasks}(1)
			\task $(MNP)$ và $(SAB)$.
			\task $(MNP)$ và $(SBC)$.
			\task $(MNP)$ và $(SAD)$.
			\task $(MNP)$ và $(SCD)$.
	\end{tasks}}{
		\begin{tikzpicture}
			[scale=1, font=\footnotesize, line join=round, line cap=round, >=stealth]
			\tkzDefPoints{0/0/A,-1.3/-1.6/B,2.5/-1.6/C}
			\coordinate (D) at ($(A)+(C)-(B)$);
			\coordinate (S) at ($(A)+(0.5,2.5)$);
			\coordinate (P) at ($(S)!0.5!(A)$);
			\coordinate (M) at ($(B)!0.5!(C)$);
			\coordinate (N) at ($(C)!0.5!(D)$);
			\tkzDrawPoints[fill=black](D,C,A,B,S,M,N,P)
			\tkzDrawPolygon(S,B,C,D)
			\tkzDrawPolygon[dashed](M,N,P)
			\tkzDrawSegments[dashed](A,B A,D A,S)
			\tkzDrawSegments(S,C)
			\tkzLabelPoints[above](S)
			\tkzLabelPoints[below](A,B,C,D,M,N)
			\tkzLabelPoints[above right](P)
	\end{tikzpicture}}
	\loigiai{
		\begin{itemize}
			\immini{\item[a)] Tìm $(MNP)\cap (SAB)$.
				\begin{itemize}
					\item [$\bullet$] Ta có $P\in (MNP)\cap (SAB)$ $(1)$.
					\item [$\bullet$] Gọi $F=MN \cap AB $ thì $\heva{& F\in MN\subset (MNP) \\ & F\in AB\subset (SAB)}$\\
					nên $F\in (MNP)\cap (SAB) \quad (2)$.
				\end{itemize}
				Từ $(1)$ và $(2)$ suy ra $(MNP)\cap (SAB)=PF$.
				\item[b)] Tìm $(MNP)\cap (SBC)$.
				\begin{itemize}
					\item [$\bullet$] Ta có $M\in (MNP)\cap (SBC)$ $(3)$.
					\item [$\bullet$] Gọi $K=PF \cap SB $ thì $\heva{& K\in PF\subset (MNP) \\ & K\in SB\subset (SBC)}$\\
					nên $K\in (MNP)\cap (SBC) \quad (4)$.
				\end{itemize}
				Từ $(3)$ và $(4)$ suy ra $(MNP)\cap (SBC)=MK$.}{
				\begin{tikzpicture}
					[scale=1, font=\footnotesize, line join=round, line cap=round, >=stealth]
					\tkzDefPoints{0/0/A,-1.3/-1.6/B,2.5/-1.6/C}
					\coordinate (D) at ($(A)+(C)-(B)$);
					\coordinate (S) at ($(A)+(0.5,2.5)$);
					\coordinate (P) at ($(S)!0.5!(A)$);
					\coordinate (M) at ($(B)!0.5!(C)$);
					\coordinate (N) at ($(C)!0.5!(D)$);
					\tkzInterLL(A,B)(M,N)\tkzGetPoint{F}
					\tkzInterLL(S,B)(F,P)\tkzGetPoint{K}
					\tkzDrawPoints[fill=black](D,C,A,B,S,M,N,P,F,K)
					\tkzDrawPolygon[dashed](M,N,P)
					\tkzDrawSegments[dashed](A,B A,D A,S K,P F,B B,K B,M)
					\tkzDrawSegments(S,C M,F F,K M,K S,K S,D C,M C,D)
					\tkzLabelPoints[above](S)
					\tkzLabelPoints[left](K)
					\tkzLabelPoints[below](A,B,C,D,M,N,F)
					\tkzLabelPoints[above right](P)
			\end{tikzpicture}	}
			\immini{\item[c)] Tìm $(MNP)\cap (SAD)$.
				\begin{itemize}
					\item [$\bullet$] Ta có $P\in (MNP)\cap (SAD)$ $(5)$.
					\item [$\bullet$] Gọi $E=MN \cap AD $, suy ra 
					$$ E\in (MNP)\cap (SAD) \quad (6).$$
				\end{itemize}
				Từ $(5)$ và $(6)$ suy ra $(MNP)\cap (SAD)=EP$.
				\item[d)] Tìm $(MNP)\cap (SCD)$.
				\begin{itemize}
					\item [$\bullet$] Ta có $N\in (MNP)\cap (SCD)$ $(7)$.
					\item [$\bullet$] Gọi $H=PE \cap SD$, suy ra
					$$H\in (MNP)\cap (SCD) \quad (8).$$
				\end{itemize}
				Từ $(7)$ và $(8)$ suy ra $(MNP)\cap (SCD)=HN$.
			}{
				\begin{tikzpicture}
					[scale=1, font=\footnotesize, line join=round, line cap=round, >=stealth]
					\tkzDefPoints{0/0/A,-1.3/-1.6/B,2.5/-1.6/C}
					\coordinate (D) at ($(A)+(C)-(B)$);
					\coordinate (S) at ($(A)+(0.5,2.5)$);
					\coordinate (P) at ($(S)!0.5!(A)$);
					\coordinate (M) at ($(B)!0.5!(C)$);
					\coordinate (N) at ($(C)!0.5!(D)$);
					\tkzInterLL(A,D)(M,N)\tkzGetPoint{E}
					\tkzInterLL(S,D)(E,P)\tkzGetPoint{H}
					\tkzDrawPoints[fill=black](D,C,A,B,S,M,N,P,E,H)
					\tkzDrawPolygon[dashed](M,N,P)
					\tkzDrawSegments[dashed](A,B A,D A,S H,P E,D D,H D,N)
					\tkzDrawSegments(S,C N,E E,H N,H S,B S,C S,H C,M B,C C,N)
					\tkzLabelPoints[above](S)
					\tkzLabelPoints[above right](H)
					\tkzLabelPoints[below](A,B,C,D,M,N,E)
					\tkzLabelPoints[above right](P)
			\end{tikzpicture}	}
		\end{itemize}
		
	}
\end{vd}

\begin{vd}
	Cho hình chóp $S.ABCD$ có đáy $ABCD$ là hình bình hành. Gọi $M,P$ lần lượt là trung điểm của $SA, BC$. $N$ là điểm trên cạnh $SB$ sao cho $BN=\dfrac{1}{4}BS$. Xác định giao tuyến của mặt phẳng $(MNP)$ với các mặt phẳng
	\begin{tasks}(3)
		\task $(ABCD)$.
		\task $(SAD)$.
		\task $(SCD)$.
	\end{tasks}
	\loigiai{
		\immini{a) Gọi $I$ là giao điểm của $MN$ và $AB$, khi đó ta có $\heva{&I\in MN\\&I\in AB}\Rightarrow \heva{&I\in (MNP)\\&I\in (ABCD)}.$\hfill (1)\\
			Hiển nhiên $\heva{&P\in (MNP)\\&P\in (ABCD)}.$\hfill (2)\\
			Từ (1) và (2) suy ra $PI$ là giao tuyến của các mặt phẳng $(MNP)$ và $(ABCD)$.\\
			b) Gọi $K$ là giao điểm của $IP$ với $AD$, khi đó $\heva{&K\in IP\\&K\in AD}\Rightarrow \heva{&K\in (MNP)\\&K\in (SAD)}.$\hfill (3)\\
			Hiển nhiên $\heva{&M\in (MNP)\\&M\in (ABCD)}.$\hfill (4)\\
			Từ (3) và (4) suy ra $MK$ là giao tuyến của các mặt phẳng $(MNP)$ và $(ABCD)$.\\
			c) Gọi $Q$ là giao điểm của $IP$ và $CD$, $R$ là giao điểm của $MK$ và $SD$. Khi đó ta chứng minh được $QR$ là giao tuyến của các mặt phẳng $(MNP)$ và $(SCD)$.
		}
		{
			\begin{tikzpicture}[scale=1]
				\tkzDefPoints{0/0/A, -2/-2/B, 4/0/D}
				\coordinate (C) at ($(B)+(D)-(A)$);
				\coordinate (S) at ($(A)+(0,5)$);
				\coordinate (M) at ($(A)!0.5!(S)$);
				\coordinate (P) at ($(B)!0.5!(C)$);
				\coordinate (Q) at ($(C)!0.5!(D)$);
				\tkzInterLL(P,Q)(A,B) \tkzGetPoint{I}
				\tkzInterLL(P,Q)(A,D) \tkzGetPoint{K}
				\tkzInterLL(M,I)(S,B) \tkzGetPoint{N}
				\tkzInterLL(M,K)(S,D) \tkzGetPoint{R}
				\tkzDrawSegments(S,N N,I I,P P,N P,C C,Q C,S Q,R Q,K K,R R,S)
				\tkzDrawSegments[dashed](S,A A,I B,P B,N A,K Q,P R,D M,N M,R Q,D)
				\tkzLabelPoints[above](S,R)
				\tkzLabelPoints[below](B,I,P,Q,C,K,D,A)
				\tkzLabelPoints[left](N)
				\tkzLabelPoints[above right](M)
			\end{tikzpicture}
		}
	}
\end{vd}

\begin{dang}{Tìm giao điểm của đường thẳng và mặt phẳng}
	\begin{note}
		Muốn tìm giao điểm của đường thẳng $d$ và mặt phẳng $(P)$ (phân biệt, không song song), ta tìm giao điểm của $d$ với một đường thẳng $a$ nằm trong $(P)$. Xét hai khả năng: 
	\end{note}
	\immini{
		\begin{itemize}
			\item [\ding{172}] Nếu đường thẳng $a$ dễ tìm, nghĩa là có sẵn $a \subset (P)$ và $a$ cắt được $d$. Khi đó
			\begin{itemize}
				\item [$\bullet$] Gọi $M=d \cap a$ thì $\heva{& M \in d\\& M \in a \subset (P)}$.
				\item [$\bullet$] Vậy $M = d \cap (P)$.
			\end{itemize}
			\item [\ding{173}] Nếu đường thẳng $a$ khó tìm, ta thực hiện các bước sau:
			\begin{itemize}
				\item [$\bullet$] Tìm một mặt phẳng $(Q)$ chứa đường thẳng $d$ và dễ tìm giao tuyến với $(P)$;
				\item [$\bullet$] Tìm $(Q) \cap (P)= a$.
				\item [$\bullet$] Tìm $M=d \cap a$, suy ra $M = d \cap (P)$.
			\end{itemize}
		\end{itemize}
	}{\vspace{1cm}
		\begin{tikzpicture}[line cap=round, line join=round,font=\footnotesize,>=stealth, scale=0.9]
			\tikzset{label style/.style={font=\footnotesize}}
			\tkzDefPoints{0/0/a, 4.5/0/b, 4/2/c, 4.2/2.3/b'}
			\tkzDefBarycentricPoint(a=1,c=1,b=-1)\tkzGetPoint{d}
			\tkzDefBarycentricPoint(a=0.75,b=0.25)\tkzGetPoint{a'}
			\tkzDefBarycentricPoint(d=0.75,c=0.25)\tkzGetPoint{d'}
			\tkzDefBarycentricPoint(d'=1,b'=1,a'=-1)\tkzGetPoint{c'}
			\tkzInterLL(a',b')(c,d)\tkzGetPoint{x}
			\tkzDefMidPoint(a',d')\tkzGetPoint{M}
			\tkzDefLine[parallel=through M](a',b')\tkzGetPoint{y}
			\tkzDrawPolygon(a',b',c',d')
			\tkzDrawSegments(a,b b,c d,a c,x d,d')
			\tkzDrawSegments[dashed](x,d')
			\tkzDrawSegments[add=0 and -0.3](M,y)
			\tkzLabelSegments[above, pos=0.7](M,y){$d$}
			\tkzDrawSegments[add=0.2 and -1, dashed](M,y)
			\tkzMarkAngle[size=0.5](c',b',a')
			\tkzLabelAngle[pos=-0.25](a',b',c'){$Q$}
			\tkzMarkAngle[size=0.5](b,a,d)
			\tkzLabelAngle[pos=0.25](b,a,d){$P$}
			\tkzLabelSegments[pos=0.8, right](a',d'){$a$}
			
			\tkzDrawPoints[fill=black](M)
			\tkzLabelPoints[below right](M)
		\end{tikzpicture}
	}
\end{dang}

\begin{vd}
	Cho tứ diện $ABCD$. Gọi $M,N$ lần lượt là trung điểm của $AC$ và $BC$. $K$ là điểm nằm trên $BD$ sao cho $KD<KB$. 
	\begin{tasks}
		\task Tìm giao điểm của $CD$ với mặt phẳng $(MNK)$.
		\task Tìm giao điểm của $AD$ với mặt phẳng $(MNK)$.
	\end{tasks}
	\loigiai{
		\begin{enumerate}[\faPencilSquareO]
			\immini{	\item Tìm giao điểm của $CD$ với mp$(MNK)$.
				\begin{note}
					Dễ thấy trong mặt phẳng $(MNK)$ có đường thẳng $NK$ có thể cắt được đường $CD$. Nên ta giải như sau:
				\end{note}
				\begin{itemize}
					\item [$\bullet$] Do $KD<KC$ nên $K$ không là trung điểm của $BD$, suy ra $NK$ cắt $CD$; 
					\item [$\bullet$] Gọi $I=CD \cap NK$, ta có
					$$\heva{&I \in CD\\& I \in NK, NK \subset (MNK)}\Rightarrow I=CD \cap (MNK).$$
				\end{itemize}
			}{
				\begin{tikzpicture}[line cap=round, line join=round,font=\footnotesize,>=stealth, scale=1.1]
					\tikzset{label style/.style={font=\footnotesize}}
					\tkzDefPoints{0/0/B, 4/0/D, 3/-1.5/C, 1/2.5/A}
					\tkzDefBarycentricPoint(B=0.2,D=0.8)\tkzGetPoint{K}
					\tkzDefMidPoint(A,C)\tkzGetPoint{M}
					\tkzDefMidPoint(B,C)\tkzGetPoint{N}
					\tkzInterLL(N,K)(C,D)\tkzGetPoint{I}
					\tkzInterLL(I,M)(A,D)\tkzGetPoint{H}
					\tkzDrawSegments[dashed](B,D M,K N,I)
					\tkzDrawSegments(A,B B,C C,I I,M A,D A,C M,N)
					
					\tkzDrawPoints[fill=black](A,B,C,D,M,N,K,H,I)
					\tkzLabelPoints[above](A)
					\tkzLabelPoints[left](B,M)
					\tkzLabelPoints[below](C,K)
					\tkzLabelPoints[below right](D)
					\tkzLabelPoints[above right](I,H)
					\tkzLabelPoints[below left](N)
			\end{tikzpicture}}
			\item Tìm giao điểm của $AD$ và $(MNK)$.\\
			Trong mặt phẳng $(ACD)$, gọi $H=AD \cap MI$. Ta có
			$$\heva{&H \in AD\\& H \in MI, MI \subset (MNK)}\Rightarrow H=AD \cap (MNK).$$		
	\end{enumerate}}
\end{vd}

\begin{vd}
	Cho tứ diện $ABCD$. trên cạnh $AC$ và $AD$ lấy hai điểm $M$, $N$ sao cho $AC=3AM$ và $AN=\dfrac{2}{3}AD$. Gọi $O$ là điểm bên trong tam giác $(BCD)$.
	\begin{tasks}(1)
		\task Tìm giao điểm của $BC$ với $(OMN)$.
		\task Tìm giao điểm của $BD$ với $(OMN)$.
	\end{tasks}
	\loigiai{
		
		\begin{enumerate}[a)]
			\immini{
				\item Tìm giao điểm của $BC$ với $(OMN)$.\\
				Xét $BC \subset (BCD)$. Ta đi tìm giao tuyến của $(BCD)$ với $(OMN)$.
				\begin{itemize}
					\item [$\bullet$] Gọi $I= CD \cap MN \Rightarrow I \in (BCD) \cap (OMN) \quad (1)$.\\
					\item [$\bullet$] Mặt khác $O \in (BCD) \cap (OMN) \quad (2)$.
				\end{itemize}
				Từ (1) và (2), suy ra $OI=(BCD) \cap (OMN)$.\\
				Trong $(BCD)$, gọi $P=OI \cap BC$. Ta có 
				$$\heva{&P \in BC\\&P \in OI, OI \subset (OMN)}\Rightarrow P=BC \cap (OMN).$$
			}{\vspace{0.6cm}
				\begin{tikzpicture}[line cap=round, line join=round,font=\footnotesize,>=stealth, scale=1]
					\tikzset{label style/.style={font=\footnotesize}}
					\tkzDefPoints{0/0/B, 4/0/D, 1.6/-1.2/C, 1/3/A}
					\tkzDefBarycentricPoint(A=0.35,D=0.65)\tkzGetPoint{N}
					\tkzDefBarycentricPoint(A=0.6,C=0.4)\tkzGetPoint{M}
					\tkzDefBarycentricPoint(B=0.35,C=0.3,D=0.35)\tkzGetPoint{O}
					\tkzInterLL(M,N)(C,D)\tkzGetPoint{I}
					\tkzInterLL(B,C)(O,I)\tkzGetPoint{P}
					\tkzInterLL(D,B)(O,I)\tkzGetPoint{Q}
					\tkzDrawSegments[dashed](B,D I,O O,M O,N O,P)
					\tkzDrawSegments(A,B B,C C,D A,C M,I A,D I,D P,M)
					\tkzLabelPoints[above](A,Q)
					\tkzLabelPoints[left](B,M)
					\tkzLabelPoints[below](C,O)
					\tkzLabelPoints[below right](D)
					\tkzLabelPoints[above right](N,I)
					\tkzLabelPoints[below left](P)
					\tkzDrawPoints[fill=black](A,B,C,D,M,N,P,Q,I,O)
				\end{tikzpicture}
			}
			\item Tìm giao điểm của $BD$ với $(OMN)$.\\
			Trong $(BCD)$, gọi $Q=OI \cap BD$. Ta có 
			$\heva{&Q \in BD\\&Q \in OI, OI \subset (OMN)}\Rightarrow Q=BD \cap (OMN).$
		\end{enumerate}	
	}
\end{vd}

\begin{vd}
	Cho hình chóp $S.ABCD$ có đáy là hình bình hành. Gọi $M$ là trung điểm của $SC$.
	\begin{itemize}
		\item [a)] Tìm giao điểm $I$ của đường thẳng $AM$ và mặt phẳng $\left(SBD\right)$. Chứng minh $IA=2IM$.
		\item [b)] Tìm giao điểm $E$ của đường thẳng $SD$ và mặt phẳng $\left(ABM\right)$.
		\item [c)] Gọi $N$ là một điểm tuỳ ý trên cạnh $AB$. Tìm giao điểm của đường thẳng $MN$ và mặt phẳng $\left(SBD\right)$.
	\end{itemize}
\end{vd}


\begin{vd}
	Cho tứ giác $ABCD$ và một điểm $S$ không thuộc mặt phẳng $(ABCD)$. Trên đoạn $AB$ lấy một điểm $M$, trên đoạn $SC$ lấy một điểm $N$ ($M,N$ không trùng với các đầu mút).
	\begin{tasks}(1)
		\task Tìm giao điểm của đường thẳng $AN$ với mặt phẳng $(SBD)$.
		\task Tìm giao điểm của đường thẳng $MN$ với mặt phẳng $(SBD)$.
	\end{tasks}
	\loigiai{
		\immini{\begin{enumerate}[a)]
				\item \textbf{Tìm giao điểm của đường thẳng $AN$ với mặt phẳng $(SBD)$.}
				\begin{itemize}
					\item Chọn mặt phẳng phụ $(SAC)\supset AN$. Ta tìm giao tuyến của $(SAC)$ và $(SBD)$.\\
					Trong $(ABCD)$ gọi $P=AC\cap BD$. Suy ra $(SAC)\cap(SBD)=SP$.
					\item Trong $(SAC)$ gọi $I=AN\cap SP$.\\
					$\heva{&I\in AN \\&I\in SP, SP\subset (SBD)}\Rightarrow I=AN\cap (SBD)$.
				\end{itemize}
				\item \textbf{Tìm giao điểm của đường thẳng $MN$ với mặt phẳng $(SBD)$.}
				\begin{itemize}
					\item Chọn mặt phẳng phụ $(SMC)\supset MN$. Ta tìm giao tuyến của $(SMC)$ và $(SBD)$.\\
					Trong $(ABCD)$ gọi $Q=MC\cap BD$. Suy ra $(SMC)\cap(SBD)=SQ$.
					\item Trong $(SMC)$ gọi $J=MN\cap SQ$.\\
					$\heva{&J\in MN \\&J\in SQ, SQ\subset (SBD)}\Rightarrow J=MN\cap (SBD)$.
				\end{itemize}
			\end{enumerate}
		}{
			\begin{tikzpicture}[line cap=round, line join=round,font=\footnotesize,>=stealth, scale=1]
				\tikzset{label style/.style={font=\footnotesize}}
				\tkzDefPoints{0/0/A, 4.5/0/D, 1.2/3/S, 0.9/-1.5/B, 3.5/-1.3/C}
				\tkzDefBarycentricPoint(A=0.3,B=0.7)\tkzGetPoint{M}
				\tkzDefBarycentricPoint(S=0.6,C=0.4)\tkzGetPoint{N}
				\tkzInterLL(A,C)(B,D)\tkzGetPoint{P}
				\tkzInterLL(S,P)(A,N)\tkzGetPoint{I}
				\tkzInterLL(C,M)(B,D)\tkzGetPoint{Q}
				\tkzInterLL(S,Q)(M,N)\tkzGetPoint{J}
				
				\tkzDrawSegments[dashed](S,P S,Q A,D A,C A,N C,M M,N B,D)
				\tkzDrawSegments(S,A S,B S,C S,D S,M A,B B,C C,D)
				
				\tkzLabelPoints[above](S)
				\tkzLabelPoints[left](A,J)
				\tkzLabelPoints[below left](B,M)
				\tkzLabelPoints[below right](C)
				\tkzLabelPoints[right](D)
				\tkzLabelPoints[above right](P,N)
				\tkzLabelPoints[above left](Q)
				\draw (I)+(-0.2,0.15) node{$I$};
				\tkzDrawPoints[fill=black](S,A,B,C,D,M,N,P,Q,I,J)
			\end{tikzpicture}
		} 
	}
\end{vd}

\begin{dang}{Chứng minh ba điểm thẳng hàng}
\begin{note}
	Muốn chứng minh ba điểm $A$, $B$, $C$ thẳng hàng, ta chứng minh ba điểm đó lần lượt thuộc hai mặt phẳng phân biệt $(\alpha)$ và $(\beta)$, nghĩa là chúng cùng nằm trên một đường giao tuyến.
\end{note}
	% \immini{	\begin{itemize}
	% 		\item [$\bullet$]  Ta có $A=a \cap b$, mà $a \subset (\alpha)$, $b \subset (\beta)$ nên
	% 		$$A \in (\alpha) \cap (\beta) \quad (1)$$
	% 		\item [$\bullet$] Tương tự ta cũng tìm xem $B$ và $C$ tương ứng là giao của cặp đường thẳng nào nằm trong $(\alpha)$ và $(\beta)$. Từ đó, suy ra
	% 		$$B \in (\alpha) \cap (\beta) \quad (2)$$
	% 		và 
	% 		$$C \in (\alpha) \cap (\beta) \quad(3)$$
	% 	\end{itemize}
	% }	
	% {\begin{tikzpicture}[scale=0.4, font=\footnotesize,line join=round, line cap=round,>=stealth]
	% 		\tikzset{label style/.style={font=\footnotesize}}
	% 		\tkzDefPoints{0/0/M,7/0/N,10/-2/P,3/-2/Q,6/2/X,3/5/Y, 2.5/3/E, 5.5/-1/F}
	% 		\tkzInterLL(M,N)(Q,X)\tkzGetPoint{T}
	% 		\coordinate (A) at ($(M)!0.15!(Q)$);
	% 		\coordinate (B) at ($(M)!0.45!(Q)$);
	% 		\coordinate (C) at ($(M)!0.84!(Q)$);
	% 		\tkzInterLL(A,F)(X,Q)\tkzGetPoint{I}
	% 		\tkzDrawPoints[size=5,fill=black](A,B,C)
	% 		\tkzLabelPoints[below](A,B,C)
	% 		\tkzLabelSegment(I,F){$b$}
	% 		\tkzLabelSegment(A,E){$a$}
	% 		\tkzLabelAngles[pos=0.6,rotate=30](M,Y,X){$\alpha$}
	% 		\tkzLabelAngles[pos=0.9,rotate=10](N,P,Q){$\beta$}
	% 		\tkzDrawPolygon(M,Q,X,Y)
	% 		\tkzDrawSegments(M,Q Q,P N,P T,N A,E I,F)
	% 		\tkzDrawSegments[dashed](M,T A,I)
	% 		\tkzMarkAngles[size=1cm,arc=l](M,Y,X)
	% 		\tkzMarkAngles[size=1.35cm,arc=l](N,P,Q)
	% 	\end{tikzpicture}\hspace{-2cm}}
	% 	Từ (1), (2), (3) suy ra $A$, $B$, $C$ cùng thuộc đường giao tuyến nên chúng thẳng hàng.
	
\end{dang}
\begin{vd}
	Cho tứ diện $ABCD$ có $G$ là trọng tâm tam giác $BCD$, Gọi $M$, $N$, $P$ lần lượt là trung điểm của $AB$, $BC$, $CD$.
	\begin{tasks}(1)
		\task Tìm giao tuyến của $(AND)$ và $(ABP)$.
		\task Gọi $I=AG\cap MP$, $J=CM\cap AN$. Chứng minh $D$, $I$, $J$ thẳng hàng.
	\end{tasks}
	\loigiai{
		\begin{center}
			\begin{tikzpicture}[scale=1, font=\footnotesize,line join=round, line cap=round,>=stealth]
				\tkzDefPoints{0/0/B,2/-2/C,6/0/D,2/4/A}
				\tkzDefMidPoint(A,B)\tkzGetPoint{M}
				\tkzDefMidPoint(B,C)\tkzGetPoint{N}
				\tkzDefMidPoint(C,D)\tkzGetPoint{P}
				\tkzInterLL(B,P)(D,N)\tkzGetPoint{G}
				\tkzInterLL(A,G)(M,P)\tkzGetPoint{I}
				\tkzInterLL(C,M)(A,N)\tkzGetPoint{J}
				\tkzDrawSegments(A,B B,C C,D D,A A,C A,N C,M A,P)
				\tkzDrawSegments[dashed](B,D B,P D,N M,P D,J A,G D,M)
				\tkzDrawPoints[fill=black](A,B,C,D,M,N,P,I,J,G)
				\tkzLabelPoints[left](M,J)
				\tkzLabelPoints[above right](I)
				\tkzLabelPoints[above](A)
				\tkzLabelPoints[below](D,C,B,N,P,G)
			\end{tikzpicture}
		\end{center}
		\begin{enumerate}[a)]
			\item Tìm giao tuyến của $(AND)$ và $(ABP)$.\\
			$A\in (ABP)\cap (ADN)$.\hfill $(1)$\\
			Ta có $G=BP\cap DN$, có $\heva{&G\in BP,\, BP\subset (ABP)\\&G\in DN,\, DN\subset (ADN)}\Rightarrow G\in (ABP)\cap (ADN)$. \hfill $(2)$\\
			Từ $(1)$ và $(2)$ ta có $AG=(ABP)\cap (ADN)$.
			\item Chứng minh $D$, $I$, $J$ thẳng hàng.\\
			$I=AG\cap MP$, $AG\subset (ADG)$, $MP\subset (DMN)\Rightarrow I\in(ADG)\cap (DMN)$.\hfill $(3)$\\
			$J=CM\cap AN$, $AN\subset (ADG)$, $CM\subset (DMN)\Rightarrow J\in(ADG)\cap (DMN)$.\hfill $(4)$\\
			$D\in (ADG)\cap (DMN)$. \hfill $(5)$\\
			Từ $(3)$, $(4)$, $(5)$ suy ra ba điểm $D$, $I$, $J$ thuộc giao tuyến của hai mặt phẳng $(ADG)$ và $(DMN)$.\\
			Vậy ba điểm $D$, $I$, $J$ thẳng hàng.
		\end{enumerate}	
	}
\end{vd}
\begin{vd}
	Cho hình chóp $S.ABCD$ có đáy là hình bình hành. Gọi $O$ là giao điểm của $AC$ và $BD$ ; $M, N$ lần lượt là trung điểm của $SB, SD$; $P$ thuộc đoạn $SC$ và không là trung điểm của $SC$.
	\begin{itemize}
		\item [a)] Tìm giao điểm $E$ của đường thẳng $SO$ và mặt phẳng $\left(MNP\right)$.
		\item [b)] Tìm giao điểm $Q$ của đường thẳng $SA$ và mặt phẳng $\left(MNP\right)$.
		\item [c)] Gọi $I, J, K$ lần lượt là giao điểm của $QM$ và $AB$, $QP$ và $AC$, $QN$ và $AD$. Chứng minh rằng $I, J, K$ thẳng hàng.
	\end{itemize}
\end{vd}

\begin{dang}{Vận dụng thực tiễn}
\end{dang}
\begin{vd}
	Giải thích tại sao ghế bốn chân có thể bị khập khiễng còn ghế ba chân thì không. 
	\loigiai{
	Dựa vào tính chất được thừa nhận của hình học không gian, có một và chỉ một mặt phẳng đi qua ba điểm không thẳng hàng cho trước. 
	Do đó qua bốn điểm có thể không cùng nằm trên một phẳng phẳng.
	
}
\end{vd}
\begin{vd}
	Giải thích tại sao chân máy ảnh có thể đặt ở hầu hết các loại hình mà vẫn đứng vững.
	\loigiai{
	Dựa vào tính chất được thừa nhận của hình học không gian, có một và chỉ một mặt phẳng đi qua ba điểm không thẳng hàng cho trước. 
	Do đó giá đỡ ba chân của máy ảnh khi đặt trên mặt đất không bị cập kênh. 
	
}
\end{vd}
\begin{vd}
	Hãy giải thích tại sao phần giao nhau giữa 2 vách tường nhà luôn là 1 đường thẳng 
	\loigiai{
Do mỗi vách tường nhà là 1 phần của mặt phẳng nên phần giao nhau là giao tuyến của của 2 mặt phẳng tức là 1 đường thẳng.

}
\end{vd}
\begin{vd}
	Hãy giải thích vì sao khi gấp đôi một tờ giấy thì nếp gấp luôn là 1 đường thẳng 
	\loigiai{
	Do khi gấp đôi tờ giấy thì mỗi phần của tờ giấy trở thành một phần của mặt phẳng khác nhau và nếp gấp là phần chung tức là giao tuyến của 2 mặt phẳng đó.
}
\end{vd}

\subsection{BÀI TẬP TỰ LUYỆN}


\begin{bt}
	Cho tứ diện $ABCD$. Trên $AB$, $AC$ lấy $2$ điểm $M$, $N$ sao cho $MN$ không song song $BC$. Gọi $O$ là một điểm trong tam giác $BCD$.
	\begin{tasks}(1)
		\task Tìm giao tuyến của $(OMN)$ và $(BCD)$.
		\task Tìm giao điểm của $DC$, $BD$ với $(OMN)$.
	%	\task Tìm thiết diện của $(OMN)$ với hình chóp.
	\end{tasks}
	\loigiai{
		\begin{center}
			\begin{tikzpicture}[scale=1, font=\footnotesize, line join=round, line cap=round, >=stealth]
				\tkzDefPoints{0/0/B,1.3/-1.6/C,4.5/0/D,1/3.5/A,2/-0.5/O}
				\coordinate (M) at ($(A)!2/3!(B)$);
				\coordinate (N) at ($(A)!0.4!(C)$);
				\tkzInterLL(M,N)(B,C)\tkzGetPoint{H}
				\tkzInterLL(B,D)(H,O)\tkzGetPoint{I}
				\tkzInterLL(C,D)(H,O)\tkzGetPoint{J}
				\tkzDrawSegments(C,D A,B A,D A,C H,N H,C N,J)
				\tkzDrawSegments[dashed](B,D H,J N,O M,I M,O)
				\tkzDrawPoints[fill=black](A,B,C,D,M,N,H,O,I,J)
				\tkzLabelPoints[above](A)
				\tkzLabelPoints[below](C,O,I,J)
				\tkzLabelPoints[left](B,M,H)
				\tkzLabelPoints[right](D,N)
			\end{tikzpicture}
		\end{center}
		\begin{enumerate}[a)]
			\item Tìm $(OMN)\cap (BCD)$.
			Trong $(ABC)$, gọi $H=MN\cap BC$.\\
			Ta có $\heva{& H\in MN\subset (MNO) \\ & H\in BC\subset (BCD)}\Rightarrow H\in (BCD)\cap (MNO)$ $(1)$.\\
			Mặt khác $O\in (BCD)\cap (MNO)$ $(2)$.\\
			Từ $(1)$ và $(2)$ suy ra $(BCD)\cap (MNO)=HO$.
			\item Tìm $DC\cap (OMN)$ và $BD\cap (OMN)$.\\
			Trong $(BCD)$, gọi $I=BD\cap HO$.\\
			Ta có $\heva{& I\in BD \\ & I\in HO\subset (MNO)}\Rightarrow I=BD\cap (MNO)$.\\
			Trong $(BCD)$, gọi $J=CD\cap HO$.\\
			Ta có $\heva{& J\in CD \\ & J\in HO\subset (MNO)}\Rightarrow J=CD\cap (MNO)$.
		%	\item Tìm thiết diện của $(OMN)$ và hình chóp.\\
		%	Ta có $\heva{& (ABC)\cap (MNO)=MN \\ & (ABD)\cap (MNO)=MI\\&(ACD)\cap (MNO)=NJ\\&(BCD)\cap (MNO)=IJ}.$ Vậy thiết diện cần tìm là tứ giác $MNJI$.
		\end{enumerate}
	}
\end{bt}

\begin{bt}
	Cho hình chóp $S.ABCD$. Gọi $O$ là giao điểm của $AC$ và $BD$. $M$, $N$, $P$ lần lượt là các điểm trên $SA$, $SB$, $SD$.
	\begin{tasks}(1)
		\task Tìm giao điểm $I$ của $SO$ với mặt phẳng $(MNP)$.
		\task Tìm giao điểm $Q$ của $SC$ với mặt phẳng $(MNP)$.
	\end{tasks}
	\loigiai{
		\begin{enumerate}[a)]	
			\item Tìm giao điểm $I$ của $SO$ với mặt phẳng $(MNP)$.
			\immini{Trong mặt phẳng $(SBD)$, gọi $I=SO\cap NP$, có
				$$\heva{&I\in SO\\&I\in NP\subset (MNP)}\Rightarrow I=SO\cap (MNP).$$
				\item Tìm giao điểm $Q$ của $SC$ với mặt phẳng $(MNP)$.\\
				$\bullet$ Chọn mặt phẳng phụ $(SAC)\supset SC$.\\
				$\bullet$ Tìm giao tuyến của $(SAC)$ và $(MNP)$.\\
				Ta có $\heva{&M\in(MNP)\\&M\in SA,\, SA\subset (SAC)}\Rightarrow M\in(MNP)\cap (SAC)$. \hfill $(1)$\\
				Và $\heva{&I\in SP,\, SP\subset(MNP)\\&I\in SO,\, SO\subset (SAC)}\Rightarrow I\in(MNP)\cap (SAC)$. \hfill $(2)$\\
			}
			{\begin{tikzpicture}[scale=1, font=\footnotesize,line join=round, line cap=round,>=stealth]
					\tkzDefPoints{-1/-2/B,-2/0/A,3/0/D,2/-1.5/C,0/4/S}
					\coordinate (M) at ($(A)!0.3!(S)$);
					\coordinate (N) at ($(B)!0.4!(S)$);
					\coordinate (P) at ($(D)!0.6!(S)$);
					\tkzInterLL(A,C)(B,D)\tkzGetPoint{O}
					\tkzInterLL(S,O)(N,P)\tkzGetPoint{I}
					\tkzInterLL(M,I)(S,C)\tkzGetPoint{Q}
					\tkzDrawSegments(A,B B,C C,D S,A S,B S,C S,D M,N N,Q Q,P)
					\tkzDrawSegments[dashed](A,D A,C B,D M,Q N,P M,P S,O)
					\tkzDrawPoints[fill=black](A,B,C,D,S,O,M,N,P,I,Q)
					\tkzLabelPoints[left](B,A,M,N)
					\tkzLabelPoints[right](C,D,P,Q)
					\tkzLabelPoints[above](S)
					\tkzLabelPoints[above left](I)
					\tkzLabelPoints[below](O)
			\end{tikzpicture}}
			Từ $(1)$ và $(2)$ có $(MNP)\cap (SAC)=MI$.\\
			$\bullet$ Trong mặt phẳng $(SAC)$ gọi $Q=SC\cap MI$, có $\heva{&Q\in SC\\&Q\in MI,\, MI\subset(MNP)}\Rightarrow Q=SC\cap (MNP)$.
			
		\end{enumerate}
	}
\end{bt}

\begin{bt}%[Dự án HHKG 11, 2018, TranTony]%[1H2B1-4]
	Cho hình chóp $S.ABCD$ có đáy $ABCD$ là hình thang với $AB$ song song với $CD$. O là giao điểm của hai đường chéo, $M$ thuộc $SB$.
	\begin{tasks}(1)
		\task Xác định giao tuyến của các cặp mặt phẳng: $(SAC)$ và $(SBD)$; $(SAD)$ và $(SBC)$.
		\task Tìm giao điểm $SO\cap (MCD)$; $SA\cap (MCD)$.
	\end{tasks}
	\loigiai{
		\centerline{\begin{tikzpicture}[scale=1,font=
				\footnotesize,line join=round,line cap=round, >=stealth]
				\tkzDefPoints{0/0/A,7/0/B,2/4/S,2/-3/H}
				\tkzDefPointBy[homothety=center A ratio 0.6](H)\tkzGetPoint{D}
				\tkzDefPointBy[homothety=center B ratio 0.6](H)\tkzGetPoint{C}
				\tkzDefPointBy[homothety=center S ratio 0.4](B)\tkzGetPoint{M}
				\tkzInterLL(A,C)(B,D)    \tkzGetPoint{O}
				\tkzInterLL(S,O)(D,M)    \tkzGetPoint{I}
				\tkzInterLL(S,A)(C,I)    \tkzGetPoint{J}
				\tkzDrawSegments[dashed](A,B A,C B,D S,O D,C D,M C,J)
				\tkzDrawSegments(S,A S,B S,D S,H S,C A,H B,H C,M)
				\tkzDrawPoints(A,B,C,S,O,H,D,M,I,J)
				\tkzLabelPoints[left](A,H,D,S,J)
				\tkzLabelPoints[below right](B,C)
				\tkzLabelPoints[right](M,I)
				\tkzLabelPoints[below](O)
		\end{tikzpicture}}\\
		\begin{enumerate}[a)]
			\item Xác định giao tuyến của $(SAC)$ và $(SBD)$.\\
			Ta có $S$ là điểm chung thứ nhất và $O$ là điểm chung thứ hai của hai mặt phẳng $(SAC)$ và $(SBD)$.\\
			Vậy $(SAC)\cap (SBD)=SO$.\\
			Xác định giao tuyến của $(SAD)$ và $(SBC)$.\\
			Ta có $S\in (SAD)\cap (SBC)$. \hfill (1)\\
			Trong mặt phẳng $(ABCD)$ gọi $H=AD\cap BC$, có $\heva{&H\in AD,AD\subset (SAD)\\& H\in BC, BC\subset (SBC)}\Rightarrow H\in (SAD)\cap (SBC)$. \hfill (2)\\
			Từ $(1)$ và $(2)$ suy ra $(SAD)\cap (SBC)=SH$.
			\item Tìm giao điểm $SO\cap (MCD)$; $SA\cap (MCD)$.\\
			Gọi $I=SO\cap DM$ (vì $SO, DM\subset (SBD)$).\\
			Ta có $\heva{&I\in SO\\&I\in DM ,DM\subset (MCD)}\Rightarrow I=SO\cap (MCD)$.\\
			Gọi $J=SA\cap CI$ (vì $SA, CI\subset (SAC)$).\\
			Ta có $\heva{&J\in SA\\& J\in CI, CI\subset (MCD)}\Rightarrow J=SA\cap (MCD)$.
		\end{enumerate}
	}
\end{bt}

\begin{bt}
	Cho hình chóp $S.ABCD$ có đáy $ABCD$ là hình bình hành tâm $O$. Gọi $M$, $N$ lần lượt là trung điểm của $AB$, $SC$.
	\begin{tasks}(2)
		\task Tìm $I=AN\cap (SBD)$.
		\task Tìm $K=MN\cap (SBD)$.
		\task Tính tỉ số $\dfrac{KM}{KN}$.
		\task Chứng minh $B, I, K$ thẳng hàng. Tính $\dfrac{IB}{IK}$.
	\end{tasks}
	\loigiai{
		\begin{center}
			\begin{tikzpicture}[scale=0.8,font=\footnotesize,line join=round,line cap=round, >=stealth]
				\tkzDefPoints{0/0/A,1/-2.7/B,8.6/-2.7/C, 4/4.8/S}
				\tkzDefPointBy[translation=from B to A](C)\tkzGetPoint{D}
				\tkzDefMidPoint(A,B)\tkzGetPoint{M}
				\tkzDefMidPoint(S,C)\tkzGetPoint{N}
				\tkzInterLL(A,C)(B,D)\tkzGetPoint{O}
				\tkzInterLL(S,O)(A,N)\tkzGetPoint{I}
				\tkzInterLL(M,N)(B,I)\tkzGetPoint{K}
				\tkzDrawSegments[dashed](A,D S,O A,C B,D M,N A,N B,I)
				\tkzDrawSegments(A,B B,C C,D S,A S,B S,C S,D B,N)
				\tkzDrawPoints(A,B,C,D,S,M,I,N,O,K)
				\tkzLabelPoints[left](A,B,M,S)
				\tkzLabelPoints[below](O)
				\tkzLabelPoints[above left](I)
				\tkzLabelPoints[below right](K,N)
				\tkzLabelPoints[right](C,D)
				
			\end{tikzpicture}
		\end{center}
		\begin{enumerate}[a)]
			\item Tìm $I=AN\cap (SBD)$.\\
			Trước hết ta tìm giao tuyến của mp$(SAC)$ và mp$(SBD)$. Ta có $S\in (SAC)\cap (SBD)$. \hfill (1)\\
			Có $\heva{&O\in AC, AC\subset (SAC)\\&O\in BD, BD\subset (SBD)}\Rightarrow O\in (SAC)\cap (SBD)$. \hfill (2)\\
			Từ $(1)$ và $(2)$ suy ra $SO=(SAC)\cap (SBD)$.\\
			Gọi $I=SO\cap AN$ (vì $SO, AN\subset (SAC)$). Suy ra $I=AN\cap (SBD)$.
			\item Tìm $K=MN\cap (SBD)$.\\
			Chọn mp$(ABN)$ chứa $MN$. Tìm giao tuyến của mp$(ABN)$ và mp$(SBD)$.\\ Có $\heva{&I\in SO, SO\subset (SBD)\\&I\in AN, AN\subset (ABN)}\Rightarrow I\in (ABN)\cap (SBD)$. \hfill (3)\\
			Có $B\in (ABN)\cap (SBD)$. \hfill (4)\\
			Từ $(3)$ và $(4)$ suy ra $BI=(ABN)\cap (SBD)$; $K=BI\cap MN$. Khi đó $K=MN\cap (SBD)$.
			\item Tính tỉ số $\dfrac{KM}{KN}$.
			\begin{center}
				\begin{tikzpicture}[scale=1,font=
					\footnotesize,line join=round,line cap=round, >=stealth]
					\tkzDefPoints{0/0/A,6/0/N,4/-2/B,2/0/Q}
					\tkzDefMidPoint(B,A)\tkzGetPoint{M}
					\tkzDefMidPoint(Q,N)\tkzGetPoint{I}
					\tkzInterLL(M,N)(B,I)\tkzGetPoint{K}
					\tkzDrawSegments(A,B B,N N,A Q,M M,N B,I)
					\tkzDrawPoints(A,N,I,K,Q)
					\tkzLabelPoints[left](A,M)
					\tkzLabelPoints[below right](K,B)
					\tkzLabelPoints[above](Q,I,N)
					\tkzMarkSegments[mark=||](I,N I,Q Q,A)
					\tkzMarkSegments[mark=|](M,A M,B)
				\end{tikzpicture}
			\end{center}
			Gọi $Q$ là trung điểm của $AI$. Ta có $AQ=QI=IN$ (vì $I$ là trọng tâm tam giác $SAC$). Có $MQ$ là đường trung bình của tam giác $ABI$. Suy ra $MQ\parallel BI$. Ta có $IK$ là đường trung bình tam giác $MNQ$. Vậy $K$ là trung điểm $MN$. Suy ra $\dfrac{KM}{KN}=1$.
			\item Chứng minh $B, I, K$ thẳng hàng. Tính tỉ số $\dfrac{IB}{IK}$.\\
			Theo cách tìm giao tuyến của câu 2 thì ba điểm $B$, $K$, $I$ thẳng hàng.\\
			Trong tam giác $ABI$, có $QM=\dfrac{1}{2}BI\Rightarrow IB=4IK\Leftrightarrow \dfrac{IB}{IK}=4$.
		\end{enumerate}
	}
\end{bt}

\begin{bt}
	Cho hình chóp $S.ABCD$ với đáy $ABCD$ là hình bình hành. Gọi $M$ là điểm bất kỳ thuộc $SB$, $N$ thuộc miền trong tam giác $S\Delta SCD$.
	\begin{tasks}(1)
		\task Tìm giao điểm của $MN$ và mặt phẳng $\left(ABCD\right)$
		\task Tìm $SC\cap \left(AMN\right)$ và $SD\cap \left(AMN\right)$
		\task Tìm $SA\cap \left(CMN\right)$
	\end{tasks}
	\loigiai{
		\begin{center}
			\begin{tikzpicture}[scale=0.7, line join=round, line cap=round,>=stealth]
				\tkzDefPoints{0/0/A,-1.2/-3/B,4.8/-3/C,6/0/D,0/5.2/S}
				\coordinate (M) at ($(S)!0.22!(B)$);
				%\coordinate (K) at ($(S)!0.3!(C)$);
				\coordinate (I) at ($(C)!0.6!(D)$);
				\coordinate (N) at ($(S)!0.4!(I)$);
				\coordinate (Q) at ($(S)!0.3!(A)$);
				\tkzInterLL(B,I)(M,N)\tkzGetPoint{H}
				\tkzInterLL(A,C)(B,I)\tkzGetPoint{O}
				\tkzInterLL(S,O)(C,Q)\tkzGetPoint{E}
				\tkzInterLL(A,E)(S,C)\tkzGetPoint{K}
				\tkzInterLL(K,N)(S,D)\tkzGetPoint{P}
				\tkzDrawPoints[fill=black](S,A,B,C,D,E,M,N,K,I,O,H,P,Q)
				\tkzLabelPoints[above](S)
				\tkzLabelPoints[left](A)
				\tkzLabelPoints[below left](B,M,E)
				\tkzLabelPoints[below right](K,N,P,C,I)
				\tkzLabelPoints[above  right](D,H)
				\tkzLabelPoints[right](Q)
				\tkzLabelPoints[below](O)
				\tkzDrawSegments(S,B S,C S,D B,C C,D S,I C,N N,H C,M H,I K,N N,P)
				\tkzDrawSegments[dashed](S,A A,B A,D A,C B,I M,N A,N S,O C,Q A,K)
			\end{tikzpicture}
		\end{center}
		\begin{enumerate}[a)]
			\item Tìm giao điểm của $MN$ và $\left(ABCD\right)$.\\
			Gọi $I=SN \cap CD$ (vì $SN,\,CD \subset \left(SCD\right)$). Chọn mặt phẳng $\left(SBI\right)$ chứa $MN$. Ta có $B$ và $I$ là hai điểm chung của hai mặt phẳng $\left(SBI\right)$ và $\left(ABCD\right)$. Vậy $\left(SBI\right)\cap \left(ABCD\right)=BI$.\\
			Gọi $H=MN \cap BI$ (vì $MN,\,BI\subset \left(SBI\right)$)
			Ta có $\heva{& H \in MN \\ & H \in BI,\, BI \subset \left(ABCD\right)}$ $\Rightarrow H=MN\cap \left(ABCD\right)$
			\item Tìm $SC\cap \left(MAN\right)$.\\
			Đầu tiên ta tìm giao tuyến của mặt phẳng $\left(SAC\right)$ và $\left(SBI\right)$. Gọi $O=AC \cap BI$ (vì $AC,\,BI \subset \left(ABCD\right)$).\\
			Ta có $S$ và $O$ là hai điểm chung của hai mặt phẳng $\left(SAC\right)$ và $\left(SBI\right)$.\\
			Vậy $SO=\left(SAC\right)\cap \left(SBI\right)$.\\
			Gọi $E=SO \cap MN$ (vì $SO,\,MN \subset \left(SBI\right)$). Chọn mặt phẳng $\left(SAC\right)$ chứa $SC$. Tìm giao tuyến của hai mặt phẳng $\left(SAC\right)$ và $\left(AMN\right)$
			\item
		\end{enumerate}
	}
\end{bt}

\begin{bt}%[Dự án HHKG 11, 2018, Chu Duc Minh]%[1H2K1-5]
	Cho tứ diện $ABCD$	. Gọi $M$ là trung điểm $AB$, $K$ là trọng tâm của tam giác $ACD$. 
	\begin{tasks}(1)
		\task Xác định giao tuyến của $(AKM)$ và $(BCD)$. 
		\task Tìm giao điểm $H$ của $MK$ và mp$(BCD)$. Chứng minh $K$ là trọng tâm của tam giác $ABH$. 
		\task Trên $BC$ lấy điểm $N$. Tìm giao điểm $P, Q$ của $CD$, $AD$ với mp$(MNK)$. 
	\end{tasks}
	\loigiai{
		\begin{center}
			\begin{tikzpicture}[scale=0.6, line join=round, line cap=round]
				\tkzDefPoints{0/0/B,1.3/-1.6/C,4.5/0/D,1/3.5/A}
				\coordinate (M) at ($1/2*(A)+1/2*(B)$);
				\coordinate (N) at ($1/3*(B)+2/3*(C)$);
				\coordinate (K) at ($1/3*(A)+1/3*(C)+1/3*(D)$);
				\coordinate (G) at ($1/2*(C)+1/2*(D)$);
				\tkzInterLL(M,K)(B,G)\tkzGetPoint{H}
				\tkzInterLL(M,N)(A,C)\tkzGetPoint{E}
				\tkzInterLL(E,K)(C,D)\tkzGetPoint{P}	
				\tkzInterLL(E,K)(A,D)\tkzGetPoint{Q}
				\tkzInterLL(M,Q)(N,P)\tkzGetPoint{F}
				\tkzDrawPolygon(A,B,C,D)
				\tkzDrawSegments(A,C A,G H,K H,G E,M E,A E,Q F,N F,M)
				\tkzDrawSegments[dashed](B,D M,K B,G N,P M,Q F,B)
				\tkzDrawPoints[fill=black,size=4](A,B,C,D,K,G,H,E)
				\tkzLabelPoints[above](A)
				\tkzLabelPoints[left](F, E)
				\tkzLabelPoints[above left](B,M)
				\tkzLabelPoints[below left](N)
				\tkzLabelPoints[below right](C,P,G)
				\tkzLabelPoints[right](D,K,Q,H)
			\end{tikzpicture}
		\end{center}
		\begin{enumerate}[a)]
			\item \textbf{Xác định giao tuyến của $(AKM)$ và $(BCD)$.} \\
			Gọi $G = AK \cap CD$  (vì $AK, CD \subset (ACD)$). \\
			Ta có $\heva{&G \in AK, AK \subset (AKM) \\ & G \in CD, CD \subset (BCD)}$\\
			$\Rightarrow G \in (AKM) \cap (BCD). \quad (1)$\\
			$B \in (ABG) \cap (BCD). \quad (2)$\\
			Từ $(1)$ và $(2)$ suy ra $(ABG) \cap (BCD) = BG$. 
			\item \textbf{Tìm giao điểm $H$ của $MK$ và} mp$(BCD)$. \\
			Trong mp$(ABG)$, gọi $H = MK \cap BG$, \\ có $\heva{& H \in MK \\& H \in BG, BG \subset (BCD)}$\\
			$\Rightarrow H = MK \cap (BCD)$.\\
			\textbf{Chứng minh $K$ là trọng tâm của tam giác $ABH$.}
			\begin{center}
				\begin{tikzpicture}[scale=1, line join=round, line cap=round]
					\tkzDefPoints{1/3/A,0/0/B,5/0/H}
					\coordinate (G) at ($1/2*(B)+1/2*(H)$);
					\coordinate (M) at ($1/2*(A)+1/2*(B)$);
					\coordinate (K) at ($1/3*(A)+1/3*(B)+1/3*(H)$);
					\coordinate (L) at ($2*(G)-(K)$);
					\tkzDrawPolygon(A,B,H)
					\tkzDrawSegments(A,L M,H L,B)
					\tkzDrawPoints[fill=black,size=4](A,B,H,G,L,M,K)
					\tkzLabelPoints[above](A)
					\tkzLabelPoints[left](B,M)
					\tkzLabelPoints[above right](K)
					\tkzLabelPoints[below right](H,L)
					\tkzLabelPoints[below left](G)
				\end{tikzpicture}
			\end{center}
			Vì $K$ là trọng tâm của tam giác $ACD$ nên $K$ chia đoạn $AG$ thành ba phần bằng nhau. \\
			Gọi $L$ là điểm đối xứng của $K$ qua $G$ thì $K$ là trung điểm của $AL$. \\ Trong $\triangle ABL$, $MK$ là đường trung bình của tam giác.\\
			Ta có $\triangle BGL = \triangle HGK$(g.c.g) $\Rightarrow BG= HG$.\\ Vậy $K$ là trọng tâm của tam giác $ABH$. 
			\item \textbf{Tìm giao điểm $P, Q$ của $CD, AD$ với} mp$(MNK)$. \\
			Trong mp$(ABC)$ gọi $E = MN \cap AC$. Trong mp$(ACD)$ đường thẳng $EK$ cắt $CD$ và $AD$ lần lượt tại $P, Q$, thì $P$ và $Q$ chính là giao điểm của $CD$ và $AD$ với mp$(MNK)$. 
		\end{enumerate}
	}
\end{bt}

\begin{bt}%[Dự án HHKG 11, 2018, Nguyễn Thành Tiến]%[1H2B1-1]%[1H2K1-3]
	Cho tứ giác $ABCD$ và $ S \not\in \left(ABCD\right)$. Gọi $I,\,J$ là hai điểm trên $AD$ và $SB$, $AD$ cắt $BC$ tại $O$ và $OJ$ cắt $SC$ tại $M$.
	\begin{tasks}(1)
		\task Tìm giao điểm $K=IJ \cap \left(SAC\right)$.
		\task Xác định giao điểm $L=DJ \cap \left(SAC\right)$.
		\task Chứng minh $A,\,K,\,L,\,M$ thẳng hàng.
	\end{tasks}
	\loigiai{
		\begin{center}
			\begin{tikzpicture}[scale=0.8, line join=round, line cap=round,>=stealth]
				\tkzDefPoints{0/0/A, 7/0/B,2/-2.1/D, 4.8/-1.6/C, 2/4/S}
				\coordinate (I) at ($(A)!0.3!(D)$);
				\coordinate (J) at ($(S)!0.33!(B)$);
				\tkzDrawPoints[fill=black](S,A,B,C,D,I,J)
				\tkzLabelPoints[above](S)
				\tkzLabelPoints[left](A)
				\tkzLabelPoints[right](B)
				\tkzLabelPoints[below right](C)
				\tkzLabelPoints[below left](D)
				\tkzLabelPoints[below left](I)
				\tkzLabelPoints[above right](J)
				\tkzDrawSegments(S,A S,B S,C S,D A,D B,C S,I)
				\tkzDrawSegments[dashed](A,C A,B B,D C,D I,J D,J I,B) 		
				\tkzInterLL(A,D)(B,C)\tkzGetPoint{O}
				\tkzInterLL(S,C)(O,J)\tkzGetPoint{M}
				\tkzInterLL (A,C)(B,I)\tkzGetPoint{E}
				\tkzInterLL(A,C)(B,D)\tkzGetPoint{F}
				\tkzInterLL(A,M)(I,J)\tkzGetPoint{K}
				\tkzInterLL(D,J)(S,F)\tkzGetPoint{L}
				\tkzDrawPoints[fill=black](O,E,M,F,K,L)
				\tkzDrawSegments[dashed](S,E A,M)
				\tkzLabelPoints[below](O)
				\tkzLabelPoints[right](M)
				\tkzLabelPoints[below](E)
				\tkzLabelPoints[below](F)
				\tkzLabelPoints[above left](L)
				\tkzDrawSegments(O,D O,C O,J)
			\end{tikzpicture}
		\end{center}	
		\begin{enumerate}[a)]
			\item Tìm giao điểm $K=IJ \cap \left(SAC\right)$.\\
			Chọn mặt phẳng phụ $\left(SIB\right)$ chứa $IJ$.\\
			Tìm giao tuyến của $\left(SIB\right)$ và $\left(SAC\right)$.\\
			có $S \in \left(SBI\right) \cap \left(SAC\right)$\hfill (1)\\
			Trong mặt phẳng $\left(ABCD\right)$ gọi $E=AC \cap BI$, ta có$\colon$\\
			$\heva{& E \in AC, \, AC \subset \left(SAC\right) \\ &E \in BI,\, BI \subset \left(SBI\right)}$ $\Rightarrow E=\left(SAC\right) \cap \left(SBI\right)$ \hfill (2)\\
			Từ $(1)$ và $(2)$ suy ra $SE=\left(SBI\right) \cap \left(SAC\right)$.\\
			Trong mặt phẳng $\left(SIB\right)$, gọi $K=IJ \cap SE$.\\
			Ta có $\heva{& K \in IJ \\ & K \in SE, SE \subset \left(SAC\right)}$ $\Rightarrow K=IJ \cap \left(SAC\right)$
			\item Xác định giao điểm $L=DJ \cap \left(SAC\right)$.\\
			Chọn mặt phẳng phụ $\left(SBD\right)$ chứa $DJ$. Tìm giao tuyến của $\left(SBD\right)$ với $\left(SAC\right)$.\\
			Ta có $S \in \left(SBD\right) \cap \left(SAC\right) $ \hfill (3)\\
			\noindent Trong mặt phẳng $\left(ABCD\right)$ gọi $F=AC \subset BD$. Suy ra $F$ là điểm chung thứ hai của hai mặt phẳng $\left(SBD\right)$ và $\left(SAC\right)$. \hfill (4)\\
			Từ $(3)$ và $(4)$ suy ra $SF=\left(SBD\right)\subset \left(SAC\right)$. Trong mặt phẳng $\left(SBD\right)$ gọi $L=DJ \cap SF $.\\
			Vậy
			$\heva{& L \in DJ \\ & L \in SF,\, SF \subset \left(SAC\right)}$ $\Rightarrow L=DJ \cap \left(SAC\right)$
			\item Chứng minh $A,\, K,\, L, \, M$ thẳng hàng.\\
			Ta có $A \in \left(SAC\right)\cap \left(AJO\right)$ \quad (3)\\
			và $\heva{& K \in IJ,\, IJ \subset \left(AJO\right) \\ & K \in SE,\, SE \subset \left(SAC\right)}$ $\Rightarrow K \in \left(SAC\right)\cap \left(AJO\right)$. \quad (4)\\
			có $\heva{& L \in DJ,\, DJ \subset \left(AJO\right) \\ & L \in SF,\, SF \subset \left(SAC\right)}$ $\Rightarrow L \in \left(SAC\right) \cap \left(AJO\right)$ \quad (5)\\
			có $\heva{& M \in JO,\, JO \subset \left(AJO\right) \\ & M \in SC,\, SC \subset \left(SAC\right)}$ $\Rightarrow M \in \left(SAC\right)\cap \left(AJO\right)$ \quad (6)\\
			Từ (3), (4), (5) và (6) suy ra bốn điểm $A,\,K,\, L,\,M$ cùng thuộc giao tuyến của hai mặt phẳng $\left(SAC\right)$ và $\left(AJO\right)$. Vậy $A,\,K,\,L,\,M$ thẳng hàng.
		\end{enumerate}
	}	
\end{bt}

\begin{bt}
	Cho hình chóp $S.ABCD$ có đáy $ABCD$ và hình bình hành. Gọi $G$ là trọng tâm của tam giác $SAD$, $M$ là trung điểm của $SB$. 
	\begin{tasks}(1)
		\task Tìm giao điểm $N$ của $MG$ và mặt phẳng $(ABCD)$. 
		\task Chứng minh ba điểm $C, D, N$ thẳng hàng và $D$ là trung điểm của $CN$. 
	\end{tasks}
	\loigiai{
		\begin{center}
			\begin{tikzpicture}[scale=1, line join=round, line cap=round]
				\tkzDefPoints{0/0/A,-1.7/-1.6/B,2.5/-1.6/C}
				\coordinate (D) at ($(A)+(C)-(B)$);
				\coordinate (S) at ($(A)+(-0.5,3)$);
				\coordinate (G) at ($1/3*(S)+1/3*(A)+1/3*(D)$);
				\coordinate (M) at ($1/2*(S)+1/2*(B)$);
				\coordinate (E) at ($1/2*(A)+1/2*(D)$);
				\coordinate (F) at ($1/2*(B)+1/2*(M)$);
				\tkzInterLL(B,E)(M,G)\tkzGetPoint{N}
				\tkzInterLL(B,N)(S,D)\tkzGetPoint{X}
				\tkzInterLL(M,N)(S,D)\tkzGetPoint{Y}
				
				\tkzDrawPolygon(S,B,C,D)
				\tkzDrawSegments(S,C S,N Y,N X,N N,D)
				\tkzDrawSegments[dashed](A,S A,B A,D S,E F,E B,X M,Y)
				\tkzDrawPoints[fill=black,size=4](D,C,A,B,S,M,G,N,E,F)
				\tkzLabelPoints[above](S)
				\tkzLabelPoints[left](A,M,F)
				\tkzLabelPoints[below](B,C,E)
				\tkzLabelPoints[right](D,N)
				\tkzLabelPoints[above right](G)
			\end{tikzpicture}
		\end{center}
		\begin{enumerate}[a)]
			\item Trong mặt phẳng chứa $MG$, gọi $N$ là giao điểm của $MG$ và $BE$. 
			Vì $BE$ thuộc mặt phẳng $(ABCD)$, nên $N$ thuộc $(ABCD)$. Vậy $N$ là giao điểm của $MG$ và mặt phẳng $(ABCD)$.
			\item Trong mặt phẳng $(SBN)$, kẻ $EF \parallel MN$ ($F$ thuộc $SB$). \\
			Trong tam giác $SEF$ có $MG \parallel EF$ nên $$\dfrac{SM}{MF} = \dfrac{SG}{GE} = 2 \Rightarrow SM = 2MF \Leftrightarrow BM = 2MF.$$
			Vậy $F$ là trung điểm của $BM$. 
		\end{enumerate}	
		Trong $\triangle  BMN$ có $EF \parallel MN$ nên  $\dfrac{BF}{FM} = \dfrac{BE}{EN} = 1 \Rightarrow BE = EN$. Vậy $E$ là trung điểm của $BN$. 
		
		Dễ dàng chứng minh $\triangle AEB = \triangle DEN$ (c.g.c) $\Rightarrow \widehat{ABE} = \widehat{END}$.\\ Hai góc này bằng nhau theo trường hợp so le trong nên $AB \parallel DN$,  mà $AB \parallel CD$ nên $C, D, N$ thẳng hàng. 
		
		$ED$ là đường trung bình của tam giác $NBC$ suy ra $D$ là trung điểm của $CN$. 
		
	}
\end{bt}

\begin{bt}
	Cho hình chóp $S.ABCD$, đáy $ABCD$ là hình bình hành tâm $O$. Gọi $M$ là trung điểm của $SC$. 
	\begin{tasks}(1)
		\task Xác định giao tuyến của $(ABM)$ và $(SCD)$. 
		\task Gọi $N$ là trung điểm của $BO$. Xác định giao điểm $I$ của $(AMN)$ với $SD$. Chứng minh $\dfrac{SI}{ID} = \dfrac{2}{3}$. 
		%Tìm thiết diện của hình chóp $S.ABCD$ cắt bởi mặt phẳng $(AMN)$. 
	\end{tasks}
	\loigiai{
		\begin{center}
			\begin{tikzpicture}[scale=1, line join=round, line cap=round]
				\tkzDefPoints{0/0/A,-3/-2.4/B,2.4/-2.4/C}
				\coordinate (D) at ($(A)+(C)-(B)$);
				\coordinate (O) at ($(A)!1/2!(C)$);
				\coordinate (S) at ($(A)+(0.5,3)$);
				\coordinate (M) at ($1/2*(S)+1/2*(C)$);
				\coordinate (N) at ($1/2*(B)+1/2*(O)$);
				\coordinate (H) at ($1/2*(S)+1/2*(D)$);
				\tkzInterLL(A,M)(S,O)\tkzGetPoint{K}
				\tkzInterLL(N,K)(S,D)\tkzGetPoint{I}
				\tkzInterLL(A,N)(B,C)\tkzGetPoint{L}
				\coordinate (P) at ($2/3*(I)+1/3*(D)$);
				\tkzFillPolygon[color=gray!50!](A,I,M,L)
				\tkzDrawPolygon(S,B,C,D)
				\tkzDrawSegments(S,C M,H M,I L,M)
				\tkzDrawSegments[dashed](A,S A,B A,D A,C B,D S,O A,M N,I A,I A,L)
				\tkzDrawPoints[fill=black,size=4](D,C,A,B,S,N,O,K,I,P,H,L)
				\tkzLabelPoints[above](S)
				\tkzLabelPoints[left](A)
				\tkzLabelPoints[below](B,C,L,O)
				\tkzLabelPoints[right](D,M)
				\tkzLabelPoints[above right](I,H)
				\tkzLabelPoints[left](K)
			\end{tikzpicture}
		\end{center}
		\begin{enumerate}[a)]
			\item Xác định giao tuyến của $(ABM)$ và $(SCD)$. \\
			Ta có $\heva{& M \in (ABM) \cap (SCD)\\ &AB \parallel CD\\ &AB \subset (ABM), CD \subset (SCD)}\Rightarrow (ABM) \cap (SCD) = MH$ ($MH \parallel AB \parallel CD$.)
			\item Xác định giao điểm $I$ của $(AMN)$  và $SD$
			Ta có $(SAC) \cap (SBD) = SO$. Gọi $K = AM \cap SO$  ($AM, SO \subset (SAC)$). \\
			\textbf{Tìm giao tuyến $(AMN)$ và $(SBD)$.}\\
			Ta có $\heva{&N \in (AMN)\\ &N \in BD, BD \subset (SBD)} \Rightarrow N \in (AMN) \cap (SBD). \quad (1)$\\
			$\heva{&K \in AM, AM \subset (AMN)\\ & K \in BD, BD \subset (SBD)} \Rightarrow K \in (AMN) \cap (SBD). \quad (2)$\\
			Từ $(1)$ và $(2)$ suy ra $(AMN) \cap (SBD) = NK$. 
			$NK$ cắt $SD$ tại điểm $I$, thì $I$ chính là giao điểm của $(AMN)$ và $SD$. \\
			\begin{center}
				\begin{center}
					\begin{tikzpicture}[scale=1, line join=round, line cap=round]
						\tkzDefPoints{0/0/B,1/4/S,5/0/D}
						\coordinate (N) at ($(B)!1/4!(D)$);
						\coordinate (O) at ($(B)!1/2!(D)$);
						\coordinate (I) at ($3/5*(S)+2/5*(D)$);
						\coordinate (P) at ($(I)!1/3!(D)$);
						\coordinate (K) at ($(S)!2/3!(O)$);
						\tkzDrawPolygon(S,B,D)
						\tkzDrawSegments(O,P N,I S,O)
						\tkzDrawPoints[fill=black,size=4](S,B,D,I,N,P,K)
						\tkzLabelPoints[above](S)
						\tkzLabelPoints[below](B,D,O,N)
						\tkzLabelPoints[above right](I,P)
						\tkzLabelPoints[left](K)
					\end{tikzpicture}
				\end{center}
			\end{center}
			Trong mặt phẳng $(SBD)$, từ $O$ dựng $OP \parallel NI (P \in SD)$. \\
			Trong $\triangle DNI$, có $OP \parallel DI$ nên có $\dfrac{DO}{ON} = \dfrac{DP}{PI} = \dfrac{2}{1} = 2 \Rightarrow DP = 2PI. \quad (3)$\\
			Trong $\triangle SOP$ có $KI \parallel OP$ nên có $\dfrac{SK}{KO} = \dfrac{SI}{PI} = \dfrac{2}{1} = 2 \Rightarrow SI = 2PI. \quad (4)$ 
			($K$) là trọng tâm của $\triangle SAC$. 
			Từ $(3)$ và $(4)$ suy ra $\dfrac{IS}{ID} = \dfrac{2}{3}$. 
			
		%	\textbf{Thiết diện của hình chóp bị cắt bởi mặt phẳng $(AMN)$.} \\
		%	Gọi $L$ là giao điểm của $AN$ và $BC$. Kết luận thiết diện là tứ giác $ALMI$. 
		\end{enumerate}
	}
\end{bt}


\begin{bt}
	Cho tứ diện $SABC$. Gọi $I$, $H$ lần lượt là trung điểm của $SA$, $AB$. Trên cạnh $SC$ lấy điểm $K$ sao cho $CK=3SK$.
	\begin{tasks}(1)
		\task Tìm giao điểm $F$ của $BC$ với mặt phẳng $(IHK)$. Tính tỉ số $\dfrac{FB}{FC}$.
		\task Gọi $M$ là trung điểm của đoạn thẳng $IH$. Tìm giao điểm của $KM$ và mặt phẳng $(ABC)$.
	\end{tasks}
	\loigiai{
		\begin{center}
			{\begin{tikzpicture}[scale=1, font=\footnotesize,line join=round, line cap=round,>=stealth]
					\tkzDefPoints{-2/0/A,0/-2.5/B,3/0/C,1/4/S}
					\coordinate (I) at ($(A)!0.5!(S)$);
					\coordinate (H) at ($(A)!0.5!(B)$);
					\coordinate (M) at ($(I)!0.5!(H)$);
					\coordinate (K) at ($(S)!1/4!(C)$);
					\coordinate (D) at ($(S)!0.5!(C)$);
					\coordinate (N) at ($(B)!0.5!(C)$);
					\tkzInterLL(A,C)(K,I)\tkzGetPoint{E}
					\tkzInterLL(E,H)(B,C)\tkzGetPoint{F}
					\tkzInterLL(K,M)(E,H)\tkzGetPoint{J}
					\tkzDrawSegments(H,B B,C S,I S,B S,C E,I K,F E,H I,H M,J)
					\tkzDrawSegments[dashed](A,H A,C A,N H,F H,K A,D M,K E,A A,I I,K)
					\tkzDrawPoints[fill=black](A,B,C,S,I,H,M,K,E,F,D,J,N)
					\tkzLabelPoints[left](A,B,E)
					\tkzLabelPoints[right](C,M,K,F,D,N)
					\tkzLabelPoints[above](S)
					\tkzLabelPoints[above left](I)
					\tkzLabelPoints[below](H,J)
					\tkzMarkSegments[mark=|](S,K K,D)
					\tkzMarkSegments[mark=||](A,H H,B)
					\tkzMarkSegments[mark=|||](A,I S,I)
			\end{tikzpicture}}
		\end{center}
		\begin{enumerate}[a)]
			\item Tìm giao điểm $F$ của $BC$ với mặt phẳng $(IHK)$. Tính tỉ số $\dfrac{FB}{FC}$.\\
			$\bullet$ Ta tìm giao tuyến của $(ABC)$ và $(IHK)$ trước.\\
			Gọi $E=AC\cap KI$ ($AC,\,KI\subset (SAC)$), ta có\\
			$\heva{&E\in AC,\, AC\subset (ABC)\\&E\in KI,\, KI\subset (IHK)}\Rightarrow E\in(ABC)\cap (IHK)$. \hfill $(1)$\\
			$\heva{&H\in (IHK)\\&H\in AB,\, AB\subset (ABC)}\Rightarrow H\in(ABC)\cap (IHK)$. \hfill $(2)$\\
			Từ $(1)$ và $(2)$ suy ra $EH=(ABC)\cap (IHK)$.\\
			$\bullet$ Gọi $F=EH\cap BC$ ($EH,\,BC\subset (ABC)$), có
			$$\heva{&F\in BC\\&F\in EH, \, EH\subset (IHK)}\Rightarrow F=BC\cap (IHK).$$
			Gọi $D$ là trung điểm của $SC$, ta có $IK$ là đường trung bình của $\triangle SAD$.\\
			Trong $\triangle CEK$ có $\dfrac{CA}{AE}=\dfrac{CD}{DK}=2\Rightarrow CA=2CK$.\\
			Trong mặt phẳng $(ABC)$ kẻ $AN\parallel EF$ ($N\in BC$).
			Ta có
			\begin{eqnarray*}
				&& HF\parallel AN\Rightarrow \dfrac{BH}{HA}=\dfrac{BF}{FN}=1\Rightarrow BF=FN.\\
				&& EF\parallel AN\Rightarrow \dfrac{CA}{AE}=\dfrac{CN}{NF}=2\Rightarrow CN=2NF.
			\end{eqnarray*}
			Do đó $\dfrac{FB}{FC}=\dfrac{FB}{FN+NC}=\dfrac{FB}{3FB}=\dfrac{1}{3}$.
			\item Tìm giao điểm của $KM$ và mặt phẳng $(ABC)$.\\
			Ta có $KM\subset (IHK)$. Gọi $J=KM\cap EH$ ($ EH,\, KM\subset (IHK)$).\\
			Ta có $\heva{&J\in KM\\&J\in EH,\, EH\subset (ABC)}\Rightarrow J=KM\cap (ABC)$.
		\end{enumerate}
	}
\end{bt}
% \subsection{BÀI TẬP TRẮC NGHIỆM}

\Opensolutionfile{ans}[ans/1H4.B1]
\setcounter{ex}{0}
\begin{ex}%[1H2Y1-1]
	Cho tứ giác $ABCD$. Có thể xác định được bao nhiêu mặt phẳng chứa tất cả các đỉnh của tứ giác $ABCD$?
	\choice
	{\True $1 $}
	{$3 $}
	{$0 $}
	{$2 $}
	\loigiai{
		$4$ điểm $A,B,C,D$ tạo thành $1$ tứ giác, khi đó $4$ điểm $A,B,C,D$ đã đồng phẳng và tạo thành $1$ mặt phẳng duy nhất là mặt phẳng $\left(ABCD\right)$.}
\end{ex}

\begin{ex}%[1H2Y1-1]%
	Hình chóp tam giác có số cạnh là
	\choice
	{\True $ 6 $}
	{$ 4 $}
	{$ 5 $}
	{$ 3 $}
	\loigiai{
		\immini{Xét hình chóp tam giác $ S.ABC $ có các cạnh là $ SA $, $ SB $, $ SC $, $ AB $, $ BC $ và $ CA $. Vậy hình chóp có số cạnh là $ 6 $.}
		{\begin{tikzpicture}[line join=round,line cap=round,line width=.6pt,font=\footnotesize,scale=0.6]
				\coordinate[label=left:$A$] (A) at (0,0);
				\coordinate[label=below left:$B$] (B) at (1,-1);
				\coordinate[label=right:$C$] (C) at (4,0);
				\coordinate[label=above left:$S$] (S) at (1.2,4);
				\draw (A)--(B)--(C)--(S)--cycle (S)--(B);
				\draw[dashed] (A)--(C);
				\fill (A)circle(2pt) (B)circle(2pt) (C)circle(2pt) (S)circle(2pt);
			\end{tikzpicture}
		}
	}
\end{ex}

\begin{ex}%[1H2B1]
	Hình chóp lục giác có bao nhiêu mặt?
	\choice
	{$10$}
	{$6$}
	{$8$}
	{\True $7$}
	\loigiai{
		Hình chóp có $7$ mặt trong đó có $6$ mặt bên và $1$ mặt đáy.}
\end{ex}


\begin{ex}%[1H2Y1-1]
	Các yếu tố nào sau đây xác định một mặt phẳng duy nhất?
	\choice
	{Một điểm và một đường thẳng}
	{\True Hai đường thẳng cắt nhau}
	{Bốn điểm phân biệt}
	{Ba điểm phân biệt}
	\loigiai{
		\begin{itemize}
			\item Mệnh đề \lq\lq Ba điểm phân biệt\rq\rq\  sai. Trong trường hợp $3$ điểm phân biệt thẳng hàng thì sẽ có vô số mặt phẳng chứa $3 $ điểm thẳng hàng đã cho.
			\item Mệnh đề \lq\lq Một điểm và một đường thẳng\rq\rq\ sai. Trong trường hợp điểm thuộc đường thẳng đã cho, khi đó ta chỉ có $1$ đường thẳng, có vô số mặt phẳng đi qua đường thẳng đó.
			\item  Mệnh đề \lq\lq Bốn điểm phân biệt\rq\rq\ sai. Trong trường hợp $4$ điểm phân biệt thẳng hàng thì có vô số mặt phẳng đi qua $4$ điểm đó hoặc trong trường hợp $4$ điểm mặt phẳng không đồng phẳng thì sẽ không tạo được mặt phẳng nào đi qua cả $4$ điểm.
	\end{itemize}}
\end{ex}

\begin{ex}%[1H2Y1]
	Khẳng định nào sau đây là \textbf{sai}?
	\choice
	{Nếu hai mặt phẳng phân biệt có một điểm chung thì chúng có một đường thẳng chung duy nhất}
	{Nếu hai mặt phẳng có một điểm chung thì chúng có vô số điểm chung khác nữa}
	{Nếu ba điểm phân biệt cùng thuộc hai mặt phẳng phân biệt thì chúng thẳng hàng}
	{\True  Nếu hai mặt phẳng có một điểm chung thì chúng có một đường thẳng chung duy nhất}
	\loigiai{
		Hai mặt phẳng có một điểm chung thì có thể trùng nhau, khi đó chúng có vô số đường thẳng chung.
	}
\end{ex}

\begin{ex}%[1H2Y1-1]
	Cho $5$ điểm $A,B,C,D,E$ trong đó không có $4$ điểm nào đồng phẳng. Hỏi có bao nhiêu mặt phẳng tạo bởi $3$ trong $5$ điểm đã cho?
	\choice
	{\True $10 $}
	{$14 $}
	{$12 $}
	{$8 $}
	\loigiai{
		Với $3$ điểm phân biệt không thẳng hàng, ta luôn tạo được $1$ mặt phẳng xác định.
		Ta có $\mathrm{C}_5^3$ cách chọn $3$ điểm trong $5$ điểm đã cho để tạo được $1$ mặt phẳng xác định. Vậy số mặt phẳng tạo được là $10$.}
\end{ex}

\begin{ex}%[1H2Y1-1]
	Trong các khẳng định sau, khẳng định nào đúng?
	\choice
	{Qua $3$ điểm phân biệt bất kì có duy nhất một mặt phẳng}
	{Qua $4$ điểm phân biệt bất kì có duy nhất một mặt phẳng}
	{Qua $2$ điểm phân biệt có duy nhất một mặt phẳng}
	{\True Qua $3$ điểm không thẳng hàng có duy nhất một mặt phẳng}
	\loigiai{
		\begin{itemize}
			\item Mệnh đề \lq\lq Qua $2$ điểm phân biệt có duy nhất một mặt phẳng\rq\rq\ sai. Vì qua $2 $ điểm phân biệt, tạo được $1$ đường thẳng, khi đó chưa đủ điều kiện để lập một mặt phẳng xác định. Có vô số mặt phẳng đi qua $2$ điểm đã cho.
			\item Mệnh đề \lq\lq Qua $3$ điểm phân biệt bất kì có duy nhất một mặt phẳng\rq\rq\ sai. Vì trong trường hợp $3$ điểm phân biệt thẳng hàng thì chỉ tạo được đường thẳng, khi đó có vô số mặt phẳng đi qua $3$ điểm phân biệt thẳng hàng.
			\item Mệnh đề \lq\lq Qua $4$ điểm phân biệt bất kì có duy nhất một mặt phẳng\rq\rq\ sai. Vì trong trường hợp $4$ điểm phân biệt thẳng hàng thì có vô số mặt phẳng đi qua $4$ điểm đó hoặc trong trường hợp $4$ điểm mặt phẳng không đồng phẳng thì sẽ tạo không tạo được mặt phẳng nào đi qua cả $4$ điểm.
	\end{itemize}}
\end{ex}

\begin{ex}%[Nguyễn Phúc Đức]%[1H2B1]
	Cho các hình vẽ sau: \\
	\begin{tabular}{cccc}
		\begin{tikzpicture}[scale=0.6,line width=1pt,line cap=round,line join=round,>=triangle 45,x=1.0cm,y=1.0cm]
			\clip(-3.48,0.6) rectangle (2.4,5.4);
			\draw (-1,5)-- (-3,2);
			\draw (-1,5)-- (0,1);
			\draw (-1,5)-- (2,2);
			\draw (-3,2)-- (0,1);
			\draw (0,1)-- (2,2);
			\draw [dash pattern=on 2pt off 2pt] (-3,2)-- (2,2);
			\fill [color=black] (-1,5) circle (1.5pt);
			\draw[color=black] (-0.8,5.2) node {$A$};
			\fill [color=black] (-3,2) circle (1.5pt);
			\draw[color=black] (-3.1,2.3) node {$B$};
			\fill [color=black] (0,1) circle (1.5pt);
			\draw[color=black] (-0.3,0.8) node {$C$};
			\fill [color=black] (2,2) circle (1.5pt);
			\draw[color=black] (2.2,2.3) node {$D$};
		\end{tikzpicture} & \begin{tikzpicture}[scale=0.6,line width=1pt,line cap=round,line join=round,>=triangle 45,x=1.0cm,y=1.0cm]
			\clip(-3.52,0.46) rectangle (2.56,5.48);
			\draw (-1,5)-- (-3,1);
			\draw (-1,5)-- (0,2);
			\draw (-1,5)-- (2,1);
			\draw (-3,1)-- (0,2);
			\draw (0,2)-- (2,1);
			\draw (-3,1)-- (2,1);
			\fill [color=black] (-1,5) circle (1.5pt);
			\draw[color=black] (-1.3,5.2) node {$A$};
			\fill [color=black] (-3,1) circle (1.5pt);
			\draw[color=black] (-3.14,1.26) node {$B$};
			\fill [color=black] (0,2) circle (1.5pt);
			\draw[color=black] (-0.04,1.6) node {$C$};
			\fill [color=black] (2,1) circle (1.5pt);
			\draw[color=black] (2.16,1.28) node {$D$};
		\end{tikzpicture} & \begin{tikzpicture}[scale=0.6,line width=1pt,line cap=round,line join=round,>=triangle 45,x=1.0cm,y=1.0cm]
			\clip(-3.42,0.58) rectangle (2.56,5.4);
			\draw (-1,5)-- (-3,1);
			\draw [dash pattern=on 2pt off 2pt] (-1,5)-- (0,2);
			\draw (-1,5)-- (2,1);
			\draw [dash pattern=on 2pt off 2pt] (-3,1)-- (0,2);
			\draw [dash pattern=on 2pt off 2pt] (0,2)-- (2,1);
			\draw (-3,1)-- (2,1);
			\fill [color=black] (-1,5) circle (1.5pt);
			\draw[color=black] (-1.3,5.2) node {$A$};
			\fill [color=black] (-3,1) circle (1.5pt);
			\draw[color=black] (-3.14,1.26) node {$B$};
			\fill [color=black] (0,2) circle (1.5pt);
			\draw[color=black] (-0.04,1.6) node {$C$};
			\fill [color=black] (2,1) circle (1.5pt);
			\draw[color=black] (2.16,1.28) node {$D$};
		\end{tikzpicture}& \begin{tikzpicture}[scale=0.6,line width=1pt,line cap=round,line join=round,>=triangle 45,x=1.0cm,y=1.0cm]
			\clip(-3.48,0.6) rectangle (2.4,5.4);
			\draw (-1,5)-- (-3,2);
			\draw (-1,5)-- (0,1);
			\draw (-1,5)-- (2,2);
			\draw (-3,2)-- (0,1);
			\draw (0,1)-- (2,2);
			\draw (-3,2)-- (2,2);
			\fill [color=black] (-1,5) circle (1.5pt);
			\draw[color=black] (-0.8,5.2) node {$A$};
			\fill [color=black] (-3,2) circle (1.5pt);
			\draw[color=black] (-3.1,2.3) node {$B$};
			\fill [color=black] (0,1) circle (1.5pt);
			\draw[color=black] (-0.3,0.8) node {$C$};
			\fill [color=black] (2,2) circle (1.5pt);
			\draw[color=black] (2.2,2.3) node {$D$};
		\end{tikzpicture} \\ 
		Hình $(1)$& Hình $(2)$ & Hình $(3)$ & Hình $(4)$
	\end{tabular} \\
	Trong các hình trên, những hình nào biểu diễn cho tứ diện?
	\choice{Hình (1) và hình (2)}
	{\True Hình (1), hình (2) và hình (3)}
	{Hình (1) và hình (3)}
	{Hình (1), hình (3) và hình (4)}
\end{ex}

\begin{ex}%[1H2B1-2]
\immini[thm]{Cho tứ diện $ABCD$. Gọi $M, N$ lần lượt là trung điểm của $AC,CD$. Giao tuyến của hai mặt phẳng $\left(MBD\right)$ và $\left(ABN\right)$ là
	\choice
	{\True đường thẳng $BG$ ($G$ là trọng tâm tam giác $ACD$)}
	{đường thẳng $AH$ ($H$ là trực tâm tam giác $ACD$)}
	{đường thẳng $MN $}
	{đường thẳng $AM $}
}{
	\begin{tikzpicture}[scale=0.6, line join=round, line cap=round]
	\tkzDefPoints{0/0/B,1.5/-1.8/C,4/0/D, 2/3/A}
	\tkzDefMidPoint(A,C)\tkzGetPoint{M}
	\tkzDefMidPoint(D,C)\tkzGetPoint{N}
	\tkzDrawPolygon(A,B,C,D)
	\tkzDrawSegments(A,C B,M D,M A,N)
	\tkzDrawSegments[dashed](B,D B,N)
	\tkzDrawPoints[fill=black](M,A,B,C,D,N)
	\tkzLabelPoints[above,font=\footnotesize](A)
	\tkzLabelPoints[below,font=\footnotesize](C,N)
	\tkzLabelPoints[left,font=\footnotesize](B)
	\tkzLabelPoints[right,font=\footnotesize](D)
	\tkzLabelPoints[above left,font=\footnotesize](M)
	\end{tikzpicture}}
	\loigiai{
		\immini{
			\begin{itemize}
				\item $B$ là điểm chung thứ nhất giữa hai mặt phẳng $\left(MBD\right)$ và $\left(ABN\right) $.
				\item Vì $M,N$ lần lượt là trung điểm của $AC,CD$ nên suy ra $AN,DM$ là hai trung tuyến của tam giác $ACD. $\\
				Gọi $G=AN\cap DM$
				$\Rightarrow \left\{\begin{aligned}& G\in AN\subset \left(ABN\right)\Rightarrow G\in \left(ABN\right) \\
				& G\in DM\subset \left(MBD\right)\Rightarrow G\in \left(MBD\right)
				\end{aligned}\right.$\\
				$\Rightarrow G$ là điểm chung thứ hai giữa hai mặt phẳng $\left( {MBD} \right)$ và $\left( {ABN} \right).$
			\end{itemize}
			Vậy $\left(ABN\right)\cap \left(MBD\right)=BG $.
		}
		{
			\begin{tikzpicture}[scale=1, line join=round, line cap=round]
			\tkzDefPoints{0/0/B,1.5/-1.8/C,4/0/D, 2/3/A}
			\tkzDefMidPoint(A,C)\tkzGetPoint{M}
			\tkzDefMidPoint(D,C)\tkzGetPoint{N}
			\tkzCentroid(D,A,C)\tkzGetPoint{G}
			\tkzDrawPolygon(A,B,C,D)
			\tkzDrawSegments(A,C B,M D,M A,N)
			\tkzDrawSegments[dashed](B,D B,G B,N)
			\tkzDrawPoints[fill=black](M,A,B,C,D,N,G)
			\tkzLabelPoints[above](A)
			\tkzLabelPoints[below](C,N)
			\tkzLabelPoints[left](B)
			\tkzLabelPoints[right](D)
			\tkzLabelPoints[above right](M,G)
			\end{tikzpicture}
		}
	}
\end{ex}

\begin{ex}%[1H2B1-2]
	\immini[thm]{Cho $4$ điểm không đồng phẳng $A,B,C,D$.  Gọi $I,K$ lần lượt là trung điểm của $AD$ và $BC$. Giao tuyến của $\left(IBC\right)$ và $\left(KAD\right)$ là
	\choice
	{\True $IK$}
	{$DK $}
	{$AK $}
	{$BC$}}{

\begin{tikzpicture}[scale=0.7, font=\footnotesize,>=stealth]
	\path
	%	Vẽ mp
	(0,0) coordinate (B)
	(5,0) coordinate (C)
	(1.5,-1.5) coordinate (D)
	(1,3.2) coordinate (A)
	($(A)!0.5!(D)$)coordinate (I)
	($(C)!0.5!(B)$)coordinate (K)
	;
	\draw (B)--(A)--(D)--(C)--(A) (B)--(D);
	\draw[dashed] (B)--(C);
	\foreach \x/\g in {B/180,A/90,C/0,D/-90,I/180,K/30}\draw[fill=black] (\x) circle (.05) +(\g:.5)node{\footnotesize$\x$};
\end{tikzpicture}}
	\loigiai{
		\immini{
			Điểm $K$ là trung điểm của $BC$ suy ra $K\in \left(IBC\right)\Rightarrow IK\subset \left(IBC\right) $.\\
			Điểm $I$ là trung điểm của $AD$ suy ra $I\in \left(KAD\right)\Rightarrow IK\subset \left(KAD\right) $.\\
			Vậy giao tuyến của hai mặt phẳng $\left(IBC\right)$ và $\left(KAD\right)$ là $IK$.
		}
		{
			\begin{tikzpicture}[scale=0.7, line join=round, line cap=round]
			\tkzDefPoints{0/0/B,1.5/-1.8/C,4/0/D, 2/2.5/A}
			\tkzDefMidPoint(C,B)\tkzGetPoint{K}
			\tkzDefMidPoint(A,D)\tkzGetPoint{I}
			\tkzDrawPolygon(A,B,C,D)
			\tkzDrawSegments(A,K C,I)
			\tkzDrawSegments[dashed](B,D B,I K,I K,D)
			\tkzDrawPoints[fill=black](K,A,B,C,D,I)
			\tkzLabelPoints[above](A)
			\tkzLabelPoints[above right](I)
			\tkzLabelPoints[below](C)
			\tkzLabelPoints[below left](K)
			\tkzLabelPoints[left](B)
			\tkzLabelPoints[right](D)
			\end{tikzpicture}
		}
	}
\end{ex}

\begin{ex}%[1H2B1-2]
	\immini[thm]{Cho tứ diện $ABCD$. Gọi $G$ là trọng tâm của tam giác $BCD$. Giao tuyến của mặt phẳng $\left(ACD\right)$ và $\left(GAB\right)$ là
	\choice
	{$AH$ ($H$ là hình chiếu của $B$ trên $CD) $}
	{$AM$ ($M$ là trung điểm của $AB) $}
	{$AK$ ($K$ là hình chiếu của $C$ trên $BD) $}
	{\True $AN$ ($N$ là trung điểm của $CD) $}
}{
	\begin{tikzpicture}[scale=0.7, line join=round, line cap=round]
	\tkzDefPoints{0/0/B,0.5/-1.5/C,4/0/D, 1.5/3/A}
	\tkzCentroid(D,B,C)\tkzGetPoint{G}
	\tkzDrawPolygon(A,B,C,D)
	\tkzDrawSegments(A,C)
	\tkzDrawSegments[dashed](B,D A,G B,G)
	\tkzDrawPoints[fill=black](G,A,B,C,D)
	\tkzLabelPoints[above](A)
	\tkzLabelPoints[below](C,G)
	\tkzLabelPoints[left](B)
	\tkzLabelPoints[right](D)
	\end{tikzpicture}}
	\loigiai{
		\immini{
			\begin{itemize}
				\item $A$ là điểm chung thứ nhất giữa hai mặt phẳng $\left(ACD\right)$ và $\left(GAB\right) $.
				\item Ta có $BG\cap CD=N$\\
				$ \Rightarrow \left\{\begin{aligned}& N\in BG\subset \left(ABG\right)\Rightarrow N\in \left(ABG\right) \\
				& N\in CD\subset \left(ACD\right)\Rightarrow N\in \left(ACD\right).
				\end{aligned}\right.$\\
				$\Rightarrow N$ là điểm chung thứ hai giữa hai mặt phẳng $\left(ACD\right)$ và $\left(GAB\right) $.
			\end{itemize}
			Vậy $\left( {ABG} \right) \cap \left( {ACD} \right) = AN.$
		}
		{
			\begin{tikzpicture}[scale=0.8, line join=round, line cap=round]
			\tkzDefPoints{0/0/B,0.5/-1.5/C,4/0/D, 1.5/3/A}
			\tkzDefMidPoint(D,C)\tkzGetPoint{N}
			\tkzCentroid(D,B,C)\tkzGetPoint{G}
			\tkzDrawPolygon(A,B,C,D)
			\tkzDrawSegments(A,C A,N)
			\tkzDrawSegments[dashed](B,D A,G B,N)
			\tkzDrawPoints[fill=black](G,A,B,C,D,N)
			\tkzLabelPoints[above](A)
			\tkzLabelPoints[below](C,G)
			\tkzLabelPoints[left](B)
			\tkzLabelPoints[right](D)
			\end{tikzpicture}
		}
	}
\end{ex}

\begin{ex}%[1H2B1-2]
\immini[thm]{Cho hình chóp $S.ABCD$ có đáy là hình thang $ABCD\left(AD\parallel BC\right) $. Gọi $M$ là trung điểm $CD $. Giao tuyến của hai mặt phẳng $\left(MSB\right)$ và $\left(SAC\right)$ là
	\choice
	{$SJ$ ($J$ là giao điểm của $AM$ và $BD$)}
	{\True $SI$ ($I$ là giao điểm của $AC$ và $BM$)}
	{$SO$ ($O$ là giao điểm của $AC$ và $BD$)}
	{$SP$ ($P$ là giao điểm của $AB$ và $CD$)}
}{
	\begin{tikzpicture}[scale=0.8, line join=round, line cap=round]
	\tkzDefPoints{0/0/A,0.5/-1.5/B,3/-1.5/C,4/0/D, 1/2/S}
	\tkzDefMidPoint(C,D)\tkzGetPoint{M}
	\tkzDrawPolygon(S,A,B,C,D)
	\tkzDrawSegments(S,C S,B S,M)
	\tkzDrawSegments[dashed](A,D A,C B,M)
	\tkzDrawPoints[fill=black](M,A,B,C,D,S)
	\tkzLabelPoints[above](S)
	\tkzLabelPoints[below](B,C)
	\tkzLabelPoints[left](A)
	\tkzLabelPoints[right](D,M)
	\end{tikzpicture}}
	\loigiai{
		\immini{
			\begin{itemize}
				\item  $S$ là điểm chung thứ nhất giữa hai mặt phẳng $\left(MSB\right)$ và $\left(SAC\right) $.\\
				\item  Ta có $\left\{\begin{aligned}& I\in BM\subset \left(SBM\right)\Rightarrow I\in \left(SBM\right) \\
				& I\in \left(AC\right)\in \left(SAC\right)\Rightarrow I\in \left(SAC\right)
				\end{aligned}\right.$\\
				$\Rightarrow I$ là điểm chung thứ hai giữa hai mặt phẳng $\left(SAC \right)$ và $\left(SAC\right).$
			\end{itemize}
			Vậy $\left(MSB\right)\cap \left(SAC\right)=SI $.
		}
		{
			\begin{tikzpicture}[scale=1, line join=round, line cap=round]
			\tkzDefPoints{0/0/A,0.5/-1.5/B,3/-1.5/C,4/0/D, 1/3/S}
			\tkzDefMidPoint(C,D)\tkzGetPoint{M}
			\tkzInterLL(A,C)(B,M)\tkzGetPoint{I}
			\tkzDrawPolygon(S,A,B,C,D)
			\tkzDrawSegments(S,C S,B S,M)
			\tkzDrawSegments[dashed](S,I A,D A,C B,M)
			\tkzDrawPoints[fill=black](M,A,B,C,D,I,S)
			\tkzLabelPoints[above](S)
			\tkzLabelPoints[above right](I)
			\tkzLabelPoints[below](B,C)
			\tkzLabelPoints[left](A)
			\tkzLabelPoints[right](D,M)
			\end{tikzpicture}
		}
	}
\end{ex}

\begin{ex}%[1H2B1-2]
	Cho hình chóp $S.ABCD$ có đáy là hình thang $ABCD\left(AB\parallel CD\right) $. Khẳng định nào sau đây \textbf{sai}?
	\choice
	{Hình chóp $S.ABCD$ có $4$ mặt bên}
	{Giao tuyến của hai mặt phẳng $\left(SAC\right)$ và $\left(SBD\right)$ là $SO$ ($O$ là giao điểm của $AC$ và $BD) $}
	{\True Giao tuyến của hai mặt phẳng $\left(SAB\right)$ và $\left(SAD\right)$ là đường trung bình của $ABCD$
	}
	{Giao tuyến của hai mặt phẳng $\left(SAD\right)$ và $\left(SBC\right)$ là $SI$ ($I$ là giao điểm của $AD$ và $BC) $}
	\loigiai{
		\immini{
			\begin{itemize}
				\item Hình chóp $S.ABCD$ có 4 mặt bên: $\left(SAB\right),\left(SBC\right),\left(SCD\right),\left(SAD\right) $.
				\item là điểm chung thứ nhất của hai mặt phẳng $\left(SAC\right)$ và $\left(SBD\right) $.
				$\left\{\begin{aligned}& O\in AC\subset \left(SAC\right)\Rightarrow O\in \left(SAC\right) \\
				& O\in BD\subset \left(SBD\right)\Rightarrow O\in \left(SBD\right)
				\end{aligned}\right.\Rightarrow O$ là điểm chung thứ hai của hai mặt phẳng $\left(SAC\right)$ và $\left(SBD\right).$\\
				$\Rightarrow \left(SAC\right)\cap \left(SBD\right)=SO $.
				\item Tương tự, ta có $\left(SAD\right)\cap \left(SBC\right)=SI $.
				\item $\left(SAB\right)\cap \left(SAD\right)=SA$ mà $SA$ không phải là đường trung bình của hình thang $ABCD $.
			\end{itemize}
			Vậy \lq\lq Giao tuyến của hai mặt phẳng $\left(SAB\right)$ và $\left(SAD\right)$ là đường trung bình của $ABCD$\rq\rq\ là mệnh đề sai.
		}
		{
			\begin{tikzpicture}[scale=1, line join=round, line cap=round]
			\tkzDefPoints{0/0/A,0.5/-1.5/D,4/0/B,2.4/-1.5/C, 1.5/3/S}
			\tkzInterLL(A,C)(B,D)\tkzGetPoint{O}
			\tkzInterLL(A,D)(B,C)\tkzGetPoint{I}
			\tkzDrawPolygon(S,A,I,B)
			\tkzDrawSegments(S,D S,C S,I)
			\tkzDrawSegments[dashed](S,O A,B D,C A,C B,D)
			\tkzDrawPoints[fill=black](S,A,B,C,D,O,I)
			\tkzLabelPoints[above](S)
			\tkzLabelPoints[below](O,I)
			\tkzLabelPoints[left](A,D)
			\tkzLabelPoints[right](C,B)
			\end{tikzpicture}
		}
	}
\end{ex}

\begin{ex}%[1H2B1-2]
\immini[thm]{Cho hình chóp $S.ABCD$ có đáy $ABCD$ là hình bình hành. Gọi $M, N$ lần lượt là trung điểm $AD$ và $BC$. Giao tuyến của hai mặt phẳng $\left(SMN\right)$ và $\left(SAC\right)$ là
	\choice
	{$SG$ ($G$ là trung điểm $AB$)}
	{$SD$}
	{\True $SO$ ($O$ là tâm hình bình hành $ABCD$)}
	{$SF$ ($F$ là trung điểm $CD$)}
}{
	\begin{tikzpicture}[scale=0.6, line join=round, line cap=round]
	\tkzDefPoints{0/0/A,-1.5/-1.8/B,4/0/D, 0.2/3/S}
	\coordinate (C) at ($(B)+(D)-(A)$);
	\tkzDefMidPoint(A,D)\tkzGetPoint{M}
	\tkzDefMidPoint(B,C)\tkzGetPoint{N}
	\tkzDrawPolygon(S,B,C,D)
	\tkzDrawSegments(S,C S,D S,N)
	\tkzDrawSegments[dashed](S,A A,B A,C M,N A,D S,M)
	\tkzDrawPoints[fill=black](M,A,B,C,D,N,S)
	\tkzLabelPoints[above](S)
	\tkzLabelPoints[above right](M)
	\tkzLabelPoints[below](B,C,N)
	\tkzLabelPoints[left](A)
	\tkzLabelPoints[right](D)
	\end{tikzpicture}}
	\loigiai{
		\immini{
			\begin{itemize}
				\item $S$ là điểm chung thứ nhất giữa hai mặt phẳng $\left(SMN\right)$ và $\left(SAC\right) $.
				\item Gọi $O=AC\cap BD$ là tâm của hình hình hành.
			\end{itemize}
			Trong mặt phẳng $\left(ABCD\right)$, gọi $T=AC\cap MN$\\
			$\Rightarrow \left\{\begin{aligned}& O\in AC\subset \left(SAC\right)\Rightarrow O\in \left(SAC\right) \\
			& O\in MN\subset \left(SMN\right)\Rightarrow O\in \left(SMN\right)
			\end{aligned}\right.$\\
			$\Rightarrow O$ là điểm chung thứ hai giữa hai mặt phẳng $\left(SMN\right)$ và $\left( {SAC} \right).$\\
			Vậy $\left(SMN\right)\cap \left(SAC\right)=SO $.
		}
		{
			\begin{tikzpicture}[scale=1, line join=round, line cap=round]
			\tkzDefPoints{0/0/A,-1.5/-1.8/B,4/0/D, 0.2/3/S}
			\coordinate (C) at ($(B)+(D)-(A)$);
			\tkzDefMidPoint(A,D)\tkzGetPoint{M}
			\tkzDefMidPoint(B,C)\tkzGetPoint{N}
			\tkzInterLL(A,C)(M,N)\tkzGetPoint{O}
			\tkzDrawPolygon(S,B,C,D)
			\tkzDrawSegments(S,C S,D S,N)
			\tkzDrawSegments[dashed](S,A A,B A,C M,N A,D S,O S,M)
			\tkzDrawPoints[fill=black](M,A,B,C,D,N,O,S)
			\tkzLabelPoints[above](S)
			\tkzLabelPoints[above right](M)
			\tkzLabelPoints[below](B,C,N,O)
			\tkzLabelPoints[left](A)
			\tkzLabelPoints[right](D)
			\end{tikzpicture}
		}
	}
\end{ex}

\begin{ex}%[1H2B1-2]
	\immini[thm]{Cho điểm $A$ không nằm trên mặt phẳng $\left(\alpha \right)$ chứa tam giác $BCD. $ Lấy $E,F$ là các điểm lần lượt nằm trên các cạnh $AB, AC$. Khi $EF$ và $BC$ cắt nhau tại $I$ thì $I$ không phải là điểm chung của hai mặt phẳng nào sau đây?
	\choice
	{$\left(BCD\right)$ và $\left(ABC\right) $}
	{\True $\left(BCD\right)$ và $\left(ABD\right) $}
	{$\left(BCD\right)$ và $\left(AEF\right) $}
	{$\left(BCD\right)$ và $\left(DEF\right) $}
}{
	\begin{tikzpicture}[scale=0.7, line join=round, line cap=round]
	\tkzDefPoints{0/0/B,1.5/-1.8/C,4/0/D, 2/3/A}
	\coordinate (E) at ($(A)!0.4!(B)$);
	\coordinate (F) at ($(A)!0.7!(C)$);
	\tkzDrawPolygon(A,B,C)
	\tkzDrawSegments(A,D E,F F,D C,D)
	\tkzDrawSegments[dashed](B,D E,D)
	\tkzDrawPoints[fill=black](E,A,B,C,D,F)
	\tkzLabelPoints[above](A)
	\tkzLabelPoints[below left](C)
	\tkzLabelPoints[left](B,E,F)
	\tkzLabelPoints[right](D)
	\end{tikzpicture}}
	\loigiai{
		\immini{
			Điểm $I$ là giao điểm của $EF$ và $BC$,\\
			mà $\left\{\begin{aligned}& EF\subset \left(DEF\right) \\
			& EF\subset \left(ABC\right) \\
			& EF\subset \left(AEF\right)
			\end{aligned}\right.\Rightarrow \left\{\begin{aligned}& I=\left(BCD\right)\cap \left(DEF\right) \\
			& I=\left(BCD\right)\cap \left(ABC\right) \\
			& I=\left(BCD\right)\cap \left(AEF\right)
			\end{aligned}\right. $.
		}
		{
			\begin{tikzpicture}[scale=0.8, line join=round, line cap=round]
			\tkzDefPoints{0/0/B,1.5/-1.8/C,4/0/D, 2/3/A}
			\coordinate (E) at ($(A)!0.4!(B)$);
			\coordinate (F) at ($(A)!0.7!(C)$);
			\tkzInterLL(E,F)(C,B)\tkzGetPoint{I}
			\tkzInterLL(E,F)(C,D)\tkzGetPoint{K}
			\tkzDrawPolygon(A,B,C)
			\tkzDrawSegments(A,D C,I E,I F,D K,D)
			\tkzDrawSegments[dashed](B,D E,D C,K)
			\tkzDrawPoints[fill=black](E,A,B,C,D,F)
			\tkzLabelPoints[above](A)
			\tkzLabelPoints[below](I)
			\tkzLabelPoints[below left](C)
			\tkzLabelPoints[left](B,E,F)
			\tkzLabelPoints[right](D)
			\end{tikzpicture}
		}
	}
\end{ex}

\begin{ex}%[1H2K1-5]
	\immini[thm]{Cho tứ diện $ABCD$. Gọi $G$ là trọng tâm tam giác $BCD$, $M$ là trung điểm $CD$, $I$ là điểm ở trên đoạn thẳng $AG,BI$ cắt mặt phẳng $\left(ACD\right)$ tại $J $. Khẳng định nào sau đây \textbf{sai}?
	\choice
	{\True $J$ là trung điểm của $AM $}
	{$AM=\left(ACD\right)\cap \left(ABG\right) $}
	{$A,J,M$ thẳng hàng}
	{$DJ=\left(ACD\right)\cap \left(BDJ\right) $}}{
\begin{tikzpicture}[scale=.8, line join=round, line cap=round]
	\tkzDefPoints{0/0/B,1.5/-1.5/C,4/0/D, 1.5/3/A}
	\tkzCentroid(D,B,C)\tkzGetPoint{G}
	\coordinate (I) at ($(A)!0.6!(G)$);
	\tkzDefMidPoint(D,C)\tkzGetPoint{M}
	\tkzInterLL(B,I)(M,A)\tkzGetPoint{J}
	\tkzDrawPolygon(A,B,C,D)
	\tkzDrawSegments(A,C A,M)
	\tkzDrawSegments[dashed](B,D B,M B,J A,G)
	\tkzDrawPoints[fill=black](A,B,C,M,D,G,I,J)
	\tkzLabelPoints[above](A)
	\tkzLabelPoints[above right](J)
	\tkzLabelPoints[below](C,G)
	\tkzLabelPoints[below right](M,I,D)
	\tkzLabelPoints[left](B)
\end{tikzpicture}}
	\loigiai{
		\immini{
			Ta có $A$ là điểm chung thứ nhất giữa hai mặt phẳng $\left(ACD\right)$ và $\left(GAB\right) $. \\
			Do $BG\cap CD=M\Rightarrow \left\{\begin{aligned}& M\in BG\subset \left(ABG\right)\Rightarrow M\in \left(ABG\right) \\
			& M\in CD\subset \left(ACD\right)\Rightarrow M\in \left(ACD\right)
			\end{aligned}\right.$\\
			$\Rightarrow M$ là điểm chung thứ hai giữa hai mặt phẳng $(ABG)$ và $(ACD)$\\
			$\Rightarrow \left(ABG\right)\cap \left(ACD\right)=AM$.\\
			Ta có $\left\{\begin{aligned}& BI\subset \left(ABG\right) \\
			& AM\subset \left(ABM\right) \\
			& \left(ABG\right)\equiv \left(ABM\right)
			\end{aligned}\right.\Rightarrow AM,BI$ đồng phẳng.\\
			$\Rightarrow J=BI\cap AM\Rightarrow A,J,M$ thẳng hàng.
		}
		{
			\begin{tikzpicture}[scale=1, line join=round, line cap=round]
			\tkzDefPoints{0/0/B,1.5/-1.5/C,4/0/D, 1.5/3/A}
			\tkzCentroid(D,B,C)\tkzGetPoint{G}
			\coordinate (I) at ($(A)!0.6!(G)$);
			\tkzDefMidPoint(D,C)\tkzGetPoint{M}
			\tkzInterLL(B,I)(M,A)\tkzGetPoint{J}
			\tkzDrawPolygon(A,B,C,D)
			\tkzDrawSegments(A,C A,M)
			\tkzDrawSegments[dashed](B,D B,M B,J A,G)
			\tkzDrawPoints[fill=black](A,B,C,M,D,G,I,J)
			\tkzLabelPoints[above](A)
			\tkzLabelPoints[above right](J)
			\tkzLabelPoints[below](C,G)
			\tkzLabelPoints[below right](M,I)
			\tkzLabelPoints[left](B)
			\end{tikzpicture}
		}
		Ta có $\left\{\begin{aligned}& DJ\subset \left(ACD\right) \\
		& DJ\subset \left(BDJ\right)
		\end{aligned}\right.\Rightarrow DJ=\left(ACD\right)\cap \left(BDJ\right)$.
		Điểm $I$ di động trên $AG$ nên $J$ có thể không phải là trung điểm của $AM$.
	}
\end{ex}

\begin{ex}%[1H2K1-3]
	\immini[thm]{Cho tứ diện $ABCD$. Gọi $E$ và $F$ lần lượt là trung điểm của $AB$ và $CD$; $G$ là trọng tâm tam giác $BCD$. Giao điểm của đường thẳng $EG$ và mặt phẳng $\left(ACD\right)$ là
	\choice
	{Giao điểm của đường thẳng $EG$ và $CD$}
	{Giao điểm của đường thẳng $EG$ và $AC$}
	{\True Giao điểm của đường thẳng $EG$ và $AF $}
	{Điểm $F $}}{
\begin{tikzpicture}[scale=.8, line join=round, line cap=round]
	\tkzDefPoints{0/0/B,2.6/-1.8/C,4/0/D, 2/3/A}
	\tkzDefMidPoint(A,B)\tkzGetPoint{E}
	\tkzDefMidPoint(D,C)\tkzGetPoint{F}
	\tkzCentroid(D,B,C)\tkzGetPoint{G}
	\tkzDrawPolygon(A,B,C,D)
	\tkzDrawSegments(A,C A,F)
	\tkzDrawSegments[dashed](B,D E,G B,F)
	\tkzDrawPoints[fill=black](A,B,C,D,E,F,G)
	\tkzLabelPoints[above](A)
	\tkzLabelPoints[below](C)
	\tkzLabelPoints[below left](G)
	\tkzLabelPoints[left](B,E)
	\tkzLabelPoints[right](D,F)
\end{tikzpicture}}
	\loigiai{
		\immini{
			Vì $G$ là trọng tâm tam giác $BCD$, $F$ là trung điểm của $CD$\\
			$\Rightarrow G\in \left(ABF\right) $.\\
			Ta có $E$ là trung điểm của $AB$\\
			$\Rightarrow E\in \left(ABF\right) $.\\
			Gọi $M$ là giao điểm của $EG$ và $AF$ mà $AF\subset \left(ACD\right)$ suy ra $M\in \left(ACD\right) $.\\
			Vậy giao điểm của $EG$ và $\left(ACD\right)$ là $M=EG\cap AF $.
		}
		{
			\begin{tikzpicture}[scale=1, line join=round, line cap=round]
			\tkzDefPoints{0/0/B,2.6/-1.8/C,4/0/D, 2/3/A}
			\tkzDefMidPoint(A,B)\tkzGetPoint{E}
			\tkzDefMidPoint(D,C)\tkzGetPoint{F}
			\tkzCentroid(D,B,C)\tkzGetPoint{G}
			\tkzInterLL(E,G)(C,D)\tkzGetPoint{I}
			\tkzInterLL(A,F)(E,G)\tkzGetPoint{M}
			\tkzDrawPolygon(A,B,C,D)
			\tkzDrawSegments(A,C A,M M,I)
			\tkzDrawSegments[dashed](B,D E,I B,F)
			\tkzDrawPoints[fill=black](M,A,B,C,D,E,F,G)
			\tkzLabelPoints[above](A)
			\tkzLabelPoints[below](C,M)
			\tkzLabelPoints[below left](G)
			\tkzLabelPoints[left](B,E)
			\tkzLabelPoints[right](D,F)
			\end{tikzpicture}
		}
	}
\end{ex}

\begin{ex}%[1H2K1-3]
	\immini[thm]{Cho tứ giác $ABCD$ có $AC$ và $BD$ giao nhau tại $O$ và một điểm $S$ không thuộc mặt phẳng $\left(ABCD\right)$. Trên đoạn $SC$ lấy một điểm $M$ không trùng với $S$ và $C$. Giao điểm của đường thẳng $SD$ với mặt phẳng $\left(ABM\right)$ là
	\choice
	{\True Giao điểm của $SD$ và $BK$ (với $K=SO\cap AM$)}
	{Giao điểm của $SD$ và $AB$}
	{Giao điểm của $SD$ và $MK$ (với $K=SO\cap AM$)}
	{Giao điểm của $SD$ và $AM$}}{
\begin{tikzpicture}[scale=.8, line join=round, line cap=round]
	\tkzDefPoints{0/0/A,0.5/-1.7/B,4/0/D, 1.2/3/S,3/-2/C}
	\coordinate (M) at ($(S)!0.45!(C)$);
	\tkzInterLL(A,C)(B,D)\tkzGetPoint{O}
	\tkzDrawPolygon(S,A,B,C,D)
	\tkzDrawSegments(S,C S,B B,M)
	\tkzDrawSegments[dashed](A,D A,C B,D A,M)
	\tkzDrawPoints[fill=black](M,A,B,C,D,O,S)
	\tkzLabelPoints[above](S)
	\tkzLabelPoints[above right](M)
	\tkzLabelPoints[below](B,C,O)
	\tkzLabelPoints[left](A)
\tkzLabelPoints[right](D)
\end{tikzpicture}}
	\loigiai{
		\immini{
			\begin{itemize}
				\item Chọn mặt phẳng phụ $\left(SBD\right)$ chứa $SD$.
				\item Tìm giao tuyến của hai mặt phẳng $\left(SBD\right)$ và $\left(ABM\right)$.
				Ta có $B$ là điểm chung thứ nhất của $\left(SBD\right)$ và $\left(ABM\right)$.
				\item Trong mặt phẳng $\left(ABCD\right)$, gọi $O=AC\cap BD$. Trong mặt phẳng $\left(SAC\right)$, gọi $K=AM\cap SO$. Ta có:
				\begin{itemize}
					\item $K\in SO$ mà $SO \subset (SBD)$ suy ra $K\in (SBD)$.
					\item $K\in AM$ mà $AM \subset (ABM)$ suy ra $K \in (AMB)$.
				\end{itemize}
				Suy ra $K$ là điểm chung thứ hai của $BCD$ và $\left(MNP\right)$.
				Do đó $ (SBD)\cap (ABM)=BK$.
				\item Trong mặt phẳng $(SBD)$, gọi $N=SD\cap BK$. Ta có: $N\in BK$, mà $BK\cap (ABM)$ suy ra $N\cap (ABM)$. Mặt khác $N\in SD$.
			\end{itemize}
			Vậy $N=SD\cap (ABM)$.
		}
		{
			\begin{tikzpicture}[scale=1, line join=round, line cap=round]
			\tkzDefPoints{0/0/A,0.5/-1.7/B,4/0/D, 1.2/3/S,3/-2/C}
			\coordinate (M) at ($(S)!0.45!(C)$);
			\tkzInterLL(A,C)(B,D)\tkzGetPoint{O}
			\tkzInterLL(A,M)(S,O)\tkzGetPoint{K}
			\tkzInterLL(S,D)(B,K)\tkzGetPoint{N}
			\tkzDrawPolygon(S,A,B,C,D)
			\tkzDrawSegments(S,C S,B)
			\tkzDrawSegments[dashed](A,D A,C B,D S,O A,M B,N)
			\tkzDrawPoints[fill=black](M,A,B,C,D,O,K,S,N)
			\tkzLabelPoints[above](S)
			\tkzLabelPoints[above right](M,N)
			\tkzLabelPoints[below](B,C,O)
			\tkzLabelPoints[left](A)
			\tkzLabelPoints[above left](K)
			\tkzLabelPoints[right](D)
			\end{tikzpicture}
		}
	}
\end{ex}


\begin{ex}%[1H2K1-5]
	\immini[thm]{Cho tứ diện $ABCD$. Gọi $M,N$ lần lượt là trung điểm của $AB$ và $CD$. Mặt phẳng $\left(\alpha \right)$ qua $MN$ cắt $AD, BC$ lần lượt tại $P$ và $Q$. Biết $MP$ cắt $NQ$ tại $I $. Ba điểm nào sau đây thẳng hàng?
	\choice
	{\True $I,B,D$}
	{$I,A,C $}
	{$I,C,D$ }
	{$I,A,B$}}{
\begin{tikzpicture}[scale=.8, line join=round, line cap=round]
	\tkzDefPoints{0/0/B,3/-1.8/C,4/0/D, 1.5/3/A}
	\tkzDefMidPoint(A,B)\tkzGetPoint{M}
	\tkzDefMidPoint(D,C)\tkzGetPoint{N}
	\coordinate (P) at ($(A)!0.7!(D)$);
	\tkzInterLL(M,P)(B,D)\tkzGetPoint{I}
	\tkzInterLL(B,C)(I,N)\tkzGetPoint{Q}
	\tkzDrawPolygon(A,B,C)
	\tkzDrawSegments(A,P C,N N,P M,Q)
	\tkzDrawSegments[dashed](B,D M,P N,Q N,D P,D)
	\tkzDrawPoints[fill=black](M,A,B,C,D,N,P,Q)
	\tkzLabelPoints[above](A)
	\tkzLabelPoints[above right](P,D)
	\tkzLabelPoints[below](C,Q)
	\tkzLabelPoints[below right](N)
	\tkzLabelPoints[left](B,M)
\end{tikzpicture}}
	\loigiai{
		\immini{
			Ta có $\left(ABD\right)\cap \left(BCD\right)=BD$.
			Lại có $\left\{\begin{aligned}& I\in MP\subset \left(ABD\right) \\
				& I\in NQ\subset \left(BCD\right)
			\end{aligned}\right.$\\
			$\Rightarrow I$ thuộc giao tuyến của $(ABC)$ và $\left( {BCD} \right)$ \\
			$\Rightarrow I\in BD\Rightarrow I,B,D$ thẳng hàng.
		}
		{
			\begin{tikzpicture}[scale=1, line join=round, line cap=round]
				\tkzDefPoints{0/0/B,3/-1.8/C,4/0/D, 1.5/3/A}
				\tkzDefMidPoint(A,B)\tkzGetPoint{M}
				\tkzDefMidPoint(D,C)\tkzGetPoint{N}
				\coordinate (P) at ($(A)!0.7!(D)$);
				\tkzInterLL(M,P)(B,D)\tkzGetPoint{I}
				\tkzInterLL(B,C)(I,N)\tkzGetPoint{Q}
				\tkzDrawPolygon(A,B,C)
				\tkzDrawSegments(A,P C,N N,P N,I P,I M,Q)
				\tkzDrawSegments[dashed](B,D M,P N,Q N,D P,D I,D M,N)
				\tkzDrawPoints[fill=black](M,A,B,C,D,N,P,Q,I)
				\tkzLabelPoints[above](A)
				\tkzLabelPoints[above right](P,D)
				\tkzLabelPoints[below](C,Q)
				\tkzLabelPoints[below right](N)
				\tkzLabelPoints[left](B,M)
				\tkzLabelPoints[right](I)
			\end{tikzpicture}
		}
	}
\end{ex}
\begin{ex}%[1H2K1-5]
	\immini[thm]{Cho tứ diện $SABC$. Gọi $L,M,N$ lần lượt là các điểm trên các cạnh $SA,SB$ và $AC$ sao cho $LM$ không song song với $AB, LN$ không song song với $SC$. Mặt phẳng $\left(LMN\right)$ cắt các cạnh $AB,BC,SC$ lần lượt tại $K,I,J$. Ba điểm nào sau đây thẳng hàng?
	\choice
	{$M,K,J $}
	{$N,I,J $}
	{$K,I,J $}
	{\True $M,I,J $}}{
\begin{tikzpicture}[scale=.8, line join=round, line cap=round]
	\tkzDefPoints{0/0/A,1.5/-1.5/B,4/0/C, 1.5/3/S}
	\coordinate (L) at ($(S)!0.4!(A)$);
	\coordinate (M) at ($(S)!0.75!(B)$);
	\coordinate (N) at ($(A)!0.8!(C)$);
	\tkzInterLL(M,L)(B,A)\tkzGetPoint{K}
	\tkzInterLL(S,C)(L,N)\tkzGetPoint{J}
	\tkzInterLL(B,C)(L,M)\tkzGetPoint{E}
	\tkzInterLL(B,C)(K,N)\tkzGetPoint{I}
	\tkzInterLL(B,C)(L,N)\tkzGetPoint{H}
	\tkzDrawPolygon(S,A,B)
	\tkzDrawSegments(L,K B,K K,I E,C S,J H,J)
	\tkzDrawSegments[dashed](A,C B,E M,N L,N I,N N,H)
	\tkzDrawPoints[fill=black](S,A,B,C,M,N,L,I,J,K)
	\tkzLabelPoints[above](S,N)
	\tkzLabelPoints[above right](C)
	\tkzLabelPoints[below](K,B,J)
	\tkzLabelPoints[below right](I)
	\tkzLabelPoints[left](A,M,L)
\end{tikzpicture}}
	\loigiai{
		\immini{
			Ta có
			\begin{itemize}
				\item $M\in SB$ suy $M$ là điểm chung của $\left(LMN\right)$ và $\left(SBC\right)$.
				\item $I$ là điểm chung của $\left(LMN\right)$ và $\left(SBC\right)$.
				\item $J$ là điểm chung của $\left(LMN\right)$ và $\left(SBC\right)$.
			\end{itemize}
			Vậy $M,I,J$ thẳng hàng vì cùng thuộc giao tuyến của $\left(LMN\right)$ và $\left(SBC\right)$.
		}
		{
			\begin{tikzpicture}[scale=1, line join=round, line cap=round]
				\tkzDefPoints{0/0/A,1.5/-1.5/B,4/0/C, 1.5/3/S}
				\coordinate (L) at ($(S)!0.4!(A)$);
				\coordinate (M) at ($(S)!0.75!(B)$);
				\coordinate (N) at ($(A)!0.8!(C)$);
				\tkzInterLL(M,L)(B,A)\tkzGetPoint{K}
				\tkzInterLL(S,C)(L,N)\tkzGetPoint{J}
				\tkzInterLL(B,C)(L,M)\tkzGetPoint{E}
				\tkzInterLL(B,C)(K,N)\tkzGetPoint{I}
				\tkzInterLL(B,C)(L,N)\tkzGetPoint{H}
				\tkzDrawPolygon(S,A,B)
				\tkzDrawSegments(L,K B,K K,I E,C S,J H,J)
				\tkzDrawSegments[dashed](A,C B,E M,N L,N I,N N,H)
				\tkzDrawPoints[fill=black](S,A,B,C,M,N,L,I,J,K)
				\tkzLabelPoints[above](S,N)
				\tkzLabelPoints[above right](C)
				\tkzLabelPoints[below](K,B,J)
				\tkzLabelPoints[below right](I)
				\tkzLabelPoints[left](A,M,L)
			\end{tikzpicture}
		}
	}
\end{ex}

\Closesolutionfile{ans}


%%Bài 11. Hai đt song song
% \setcounter{section}{10} \setcounter{dang}{0}
\newpage
\section{HAI ĐƯỜNG THẲNG SONG SONG}
\subsection{KIẾN THỨC CẦN NHỚ}
\subsubsection{VỊ TRÍ TƯƠNG ĐỐI CỦA HAI ĐƯỜNG THẲNG}
Trong không gian, cho hai đường thẳng $a$ và $b$.
\begin{enumerate}[\iconMT]
	\item \indam{Các trường hợp có thể xảy ra:}
	\begin{itemize}
		\item [$\bullet$] Nếu $a$ và $b$ đồng phẳng (cùng thuộc một mặt phẳng) thì chúng có các khả năng: cắt nhau; song song nhau hoặc trùng nhau.
		\item [$\bullet$] Nếu $a$ và $b$ không đồng phẳng (không tồn tại mặt phẳng chưa được cả $a$ và $b$) thì ta nói $a$ và $b$ chéo nhau.
	\end{itemize}
	\begin{tabular}{cccc}
		\begin{tikzpicture}[scale=0.5,font=\small]
			\tkzDefPoints{0/0/A, 5/0/B, 6/3/C}
			\coordinate (D) at ($(A)+(C)-(B)$);
			\tkzDrawPolygon(A,B,C,D)
			\tkzMarkAngle[size=.85](B,A,D)
			\draw (A) node[above right]{$\alpha$};
			\tkzDefPoints{1/2/E, 4.5/0.5/F, 0.8/1.5/G, 5/2.5/H}
			\draw (E)--(F) (G)--(H);
			\tkzInterLL(E,F)(G,H) \tkzGetPoint{M}
			\tkzDrawPoints[size=5,fill=black](M)
			\tkzLabelPoints[above](M)
			\draw (F) node[above]{$a$};
			\draw (H) node[below]{$b$};
		\end{tikzpicture}
		&\begin{tikzpicture}[scale=0.5,font=\small]
			\tkzDefPoints{0/0/A, 5/0/B, 6/3/C}
			\coordinate (D) at ($(A)+(C)-(B)$);
			\tkzDrawPolygon(A,B,C,D)
			\tkzMarkAngle[size=.85](B,A,D)
			\draw (A) node[above right]{$\alpha$};
			\tkzDefPoints{0.8/1.5/G, 5/2.5/H, 1/0.5/I}
			\coordinate (K) at ($(H)+(I)-(G)$);
			\draw (G)--(H) (I)--(K);
			\draw ($(G)!0.8!(H)$) node[above]{$a$};
			\draw ($(I)!0.8!(K)$) node[above]{$b$};
		\end{tikzpicture}
		&\begin{tikzpicture}[scale=0.5,font=\small]
			\tkzDefPoints{0/0/A, 5/0/B, 6/3/C}
			\coordinate (D) at ($(A)+(C)-(B)$);
			\tkzDrawPolygon(A,B,C,D)
			\tkzMarkAngle[size=.85](B,A,D)
			\draw (A) node[above right]{$\alpha$};
			\tkzDefPoints{0.5/0.8/G, 5/2.5/H}
			\draw (G)--(H);
			\draw ($(G)!0.2!(H)$) node[above]{$a$};
			\draw ($(G)!0.2!(H)$) node[below]{$b$};
		\end{tikzpicture}
		&\begin{tikzpicture}[scale=0.5,font=\small]
			\tkzDefPoints{0/0/A, 5/0/B, 6/3/C}
			\coordinate (D) at ($(A)+(C)-(B)$);
			\tkzDrawPolygon(A,B,C,D)
			\tkzMarkAngle[size=.85](B,A,D)
			\draw (A) node[above right]{$\alpha$};
			\tkzDefPoints{0.8/0.5/G, 4.5/1.5/H, 2/3.5/E, 2.5/-0.5/F}
			\tkzInterLL(A,B)(E,F) \tkzGetPoint{K}
			\coordinate (I) at ($(E)!0.4!(F)$);
			\draw (G)--(H) (E)--(I) (K)--(F);
			\draw[dashed] (I)--(K);
			\draw ($(E)!0.1!(F)$) node[above right]{$a$};
			\draw ($(G)!0.9!(H)$) node[above]{$b$};
			\draw [fill=black] (I) circle(1pt);
			\tkzLabelPoints[left](I)
		\end{tikzpicture}\\
		\small * $a$ cắt $b$ & \small * $a$ song song $b$ & \small * $a$ trùng $b$ & \small * $a$ chéo $b$\\
		\small * Kí hiệu $a \cap b = M$ & \small * Kí hiệu $a \parallel b $  & \small * Kí hiệu 	$a \equiv b$ & \small * $a$, $b$ không điểm chung
	\end{tabular}
	\item \indam{Chú ý:}
		Cho hai đường thẳng $a$ và $b$ phân biệt.
		\begin{itemize}
			\item [$\bullet$] Khi kiểm tra hai đường thẳng $a$ và $b$ \textbf{song song} hay \textbf{cắt nhau} thì trước tiên chúng phải đồng phẳng (cùng thuộc một mặt phẳng nào đó);
			\item [$\bullet$] Khi $a$ và $b$ không có điểm chung thì chúng có thể song song hoặc chéo nhau. Vấn đề này các bạn hay bị nhầm lẫn, cần chú ý. 
		\end{itemize}
\end{enumerate}

\subsubsection{CÁC ĐỊNH LÝ VÀ HỆ QUẢ CẦN NHỚ}
\begin{enumerate}[\iconMT]
	\item \indam{Định lý 1:} Trong không gian, qua một điểm không nằm trên đường thẳng cho trước, có một và chỉ một đường thẳng song song với đường thẳng đã cho.
	\item \indam{Định lý 2:} Hai đường thẳng phân biệt cùng song song với đường thẳng thứ ba thì song song với nhau.
\immini{	\item \indam{Định lý 3:} Nếu ba mặt phẳng phân biệt đôi một cắt nhau theo ba giao tuyến phân biệt thì ba giao tuyến đó hoặc đồng quy hoặc đôi một song song với nhau.}{
\begin{tikzpicture}[line cap=round,line join=round,x=1.0cm,y=1.0cm, font=\small]
	\begin{scope}[>=stealth,scale=0.5]
		\tikzset{label style/.style={font=\footnotesize}}
		\tkzDefPoints{0/0/A,0/5/B, 3/4/C,3/-1/D, -3/3/E, -3/-2/F, 0/4/I}
		\coordinate (J) at ($(A)!0.77!(F)$);
		\coordinate (M) at ($(A)!0.5!(D)$);
		\coordinate (c) at ($(I)!0.5!(B)$);
		\coordinate (a) at ($(I)!0.5!(J)$);
		\coordinate (b) at ($(I)!0.5!(M)$);
		\tkzDrawSegments[dashed](A,M A,J A,I)
		\tkzDrawSegments(F,J E,F E,B B,I B,C C,D D,M M,I M,J I,J)
		\tkzMarkAngles[size=0.8cm](F,E,B)
		\tkzLabelAngles[pos=0.5,rotate=30](B,E,F){$\alpha$}
		\tkzMarkAngles[size=0.8cm](B,C,D)
		\tkzLabelAngles[pos=0.5,rotate=320](D,C,B){$\beta$}
		\tkzMarkAngles[size=0.8cm](M,J,I)
		\tkzLabelAngles[pos=0.5,rotate=30](M,J,I){$\gamma$}
		%\tkzMarkAngles[size=0.5cm]
		\tkzLabelPoints[right](b,c)
		\tkzLabelPoints[left](a)
		
	\end{scope}
	\begin{scope}[xshift=5cm,>=stealth,scale=0.5]
		\tikzset{label style/.style={font=\footnotesize}}
		\tkzDefPoints{0/0/A,0/5/B, 3/4/C,3/-1/D, -3/3/E, -3/-2/F, 0/4/I}
		\coordinate (J) at ($(A)!0.6!(F)$);
		\coordinate (P) at ($(B)!0.6!(E)$);
		\coordinate (Q) at ($(B)!0.5!(C)$);
		\tkzInterLL(A,B)(Q,P)    \tkzGetPoint{I}
		\coordinate (M) at ($(A)!0.5!(D)$);
		\coordinate (c) at ($(I)!0.5!(B)$);
		\coordinate (a) at ($(I)!0.5!(J)$);
		\coordinate (b) at ($(I)!0.5!(M)$);
		\tkzDrawSegments[dashed](A,M A,J A,I)
		\tkzDrawSegments(F,J E,F E,B B,I B,C C,D D,M M,Q M,J J,P P,Q)
		\tkzMarkAngles[size=0.8cm](F,E,B)
		\tkzLabelAngles[pos=0.5,rotate=30](B,E,F){$\alpha$}
		\tkzMarkAngles[size=0.8cm](B,C,D)
		\tkzLabelAngles[pos=0.5,rotate=320](D,C,B){$\beta$}
		\tkzMarkAngles[size=0.8cm](M,J,I)
		\tkzLabelAngles[pos=0.5,rotate=30](M,J,P){$\gamma$}
		%\tkzMarkAngles[size=0.5cm]
		\tkzLabelPoints[right](b,c)
		\tkzLabelPoints[left](a)
	\end{scope} 
\end{tikzpicture}}
	\begin{note}
		\textbf{Hệ quả:} Nếu hai mặt phẳng phân biệt lần lượt chứa hai đường thẳng song song thì giao tuyến của chúng (nếu có) cũng song song với hai đường thẳng đó hoặc trùng với một trong hai đường thẳng đó.

	\end{note}
\end{enumerate}

	
% \subsection{PHÂN LOẠI, PHƯƠNG PHÁP GIẢI TOÁN}
\begin{dang} {Xét vị trí tương đối của hai đường thẳng}
	Cho hai đường thẳng $a$ và $b$ phân biệt. Xét vị trí tương đối của $a$ với $b$:
	\begin{itemize}
		\item Nếu $a$ và $b$ không đồng phẳng thì $a$ và $b$ chéo nhau.
		\item Nếu $a$ và $b$ đồng phẳng thì xét số điểm chung của $a$ và $b$. Nếu $a$ và $b$ không có điểm chung thì $a\parallel b$. Nếu $a$ và $b$ có một điểm chung thì $a$ và $b$ cắt nhau.
	\end{itemize}
\end{dang}
\begin{vd}
	Cho hình chóp $S.ABCD$ có đáy $ABCD$ là hình bình hành. Xét vị trí tương đối của các cặp đường thẳng sau
	\begin{listEX}[3]
		\item [a)] $AB$ và $CD$.
		\item [b)] $SA$ và $SC$.
		\item [c)] $SA$ và $BC$.
	\end{listEX}
\end{vd}
\begin{vd}
	Cho tứ diện $ABCD$ có $M, N$ lần lượt là trung điểm của $AB, AC$. Xét vị trí tương đối của các cặp đường thẳng sau
	\begin{listEX}[3]
		\item [a)] $MN$ và $BC$.
		\item [b)] $AN$ và $CD$.
		\item [c)] $MN$ và $CD$.
	\end{listEX}
\end{vd}
\begin{dang}{Chứng minh hai đường thẳng song song}
	\indamm{Phương pháp thường dùng:}
	\begin{itemize}
		\item [\ding{172}] Sử dụng các kết quả của hình học phẳng như:
		      \begin{itemize}
			      \item  Cặp cạnh đối hình bình hành thì song song nhau;...
			      \item  Đường trung bình của tam giác thì song song và bằng nửa cạnh đáy.
		      \end{itemize}
		\item [\ding{173}] Sử dụng tỉ lệ (Định lý Thales)
		      \begin{itemize}
			      \item Nếu $\dfrac{AE}{AB}=\dfrac{AF}{AC}$ thì $EF \parallel BC$.
			      \item Chú ý tỉ lệ trọng tâm:  $AG=\dfrac{2}{3}AM$.
		      \end{itemize}
	\end{itemize}
\end{dang}

\begin{vd}
	Cho tứ diện $ABCD$ có $I$, $J$ lần lượt là trọng tâm của tam giác $ABC$ và $ABD$. Chứng minh rằng $IJ \parallel CD$.
	\loigiai{\immini{Gọi $E$ là trung điểm $AB$. Ta có $\heva{&I\in CE\\&J\in DE}\Rightarrow IJ$ và $CD$ đồng phẳng.\\
			Vì $I$, $J$ lần lượt là trọng tâm của tam giác $ABC$ và $ABD$ nên $$\dfrac{EI}{EC}=\dfrac{EJ}{ED}=\dfrac{1}{3}.$$
			Theo định lí đảo Thales suy ra $IJ\parallel CD$ (đpcm).
		}{	\begin{tikzpicture}[scale=0.6,font=\footnotesize,line join=round,line cap=round,>=stealth]
				\tkzDefPoints{0/0/B, 3/4/A, 5/-3/C, 6/0/D}
				\coordinate (E) at ($(A)!0.5!(B)$);
				\coordinate (I) at ($(E)!1/3!(C)$);
				\coordinate (J) at ($(E)!1/3!(D)$);
				\tkzDrawPolygon(A,B,C,D)
				\tkzDrawSegments(A,C E,C)
				\tkzDrawSegments[dashed](E,D I,J B,D)
				\tkzDrawPoints[fill=black](A,B,C,D,E,I,J)
				\tkzLabelPoints[above](A,J)
				\tkzLabelPoints[right](D,I)
				\tkzLabelPoints[left](B,E)
				\tkzLabelPoints[below](C)
				\tkzMarkSegments[mark=||](E,A E,B)
			\end{tikzpicture}}}
\end{vd}
\begin{vd}
	Cho hình chóp $S.ABCD$ có đáy $ABCD$ là hình thang với $AB$ là đáy lớn và $AB=2CD$. Gọi $M, N$ lần lượt là trung điểm của các cạnh $SA$ và $SB$. Chứng minh rằng $NC\parallel MD$.
\end{vd}
\begin{vd}
	Cho tứ diện $ABCD$. Gọi $I, J$ lần lượt là trung điểm của các cạnh $BC, CD$. Trên cạnh $AC$ lấy điểm $K$. Gọi $M$ là giao điểm của $BK$ và $AI$, $N$ là giao điểm của $DK$ và $AJ$. Chứng minh rằng $MN\parallel BD$.
\end{vd}
\begin{vd}
	Cho tứ diện $ABCD$. Gọi $M$, $N$, $P$, $Q$, $R$, $S$ lần lượt là trung điểm của $AB$, $CD$, $BC$, $AD$, $AC$, $BD$.
	\begin{tasks}(1)
		\task Chứng minh $MPNQ$ là hình bình hành.
		\task Chứng minh ba đoạn thẳng $MN$, $PQ$, $RS$ cắt nhau tại trung điểm $G$ của mỗi đoạn.
	\end{tasks}
	\loigiai{
		\immini{
			\begin{enumerate}[a)]
				\item Vì $MP$ là đường trung bình của $\triangle ABC$ nên
				      $\heva{& MP \parallel AC\\& MP=\dfrac{1}{2}AC.} \qquad (1)$\\
				      Vì $NQ$ là đường trung bình của $\triangle ACD$ nên
				      $\heva{& NQ \parallel AC\\& NQ=\dfrac{1}{2}AC.} \qquad (2)$\\
				      Từ $(1)$ và $(2)$ suy ra $\heva{& MP \parallel NQ\\& MP=NQ.}$\\
				      Do đó, $MPNQ$ là hình bình hành.
				\item 	Do $MPNQ$ là hình bình hành nên $MN$, $PQ$ cắt nhau tại trung điểm $G$ của mỗi đoạn.\\
				      Mặt khá, chứng minh tương tự ta được $PSQR$ là hình bình hành nên $PQ$, $RS$ cắt nhau tại trung điểm $G$ của mỗi đoạn.\\
				      Vậy $MN$, $PQ$, $RS$ cắt nhau tại trung điểm $G$ của mỗi đoạn.
			\end{enumerate}
		}{	\begin{tikzpicture}[scale=0.6,font=\footnotesize,line join=round,line cap=round,>=stealth]
				\tikzset{label style/.style={font=\footnotesize}}
				\tkzDefPoints{0/0/B, 3/6/A, 1.3/-3/C, 8/0/D}
				\coordinate (M) at ($(A)!0.5!(B)$);
				\coordinate (N) at ($(C)!0.5!(D)$);
				\coordinate (P) at ($(B)!0.5!(C)$);
				\coordinate (Q) at ($(A)!0.5!(D)$);
				\coordinate (R) at ($(A)!0.5!(C)$);
				\coordinate (S) at ($(B)!0.5!(D)$);
				\coordinate (G) at ($(M)!0.5!(N)$);
				\tkzDrawPolygon(A,B,C,D)
				\tkzDrawSegments(A,C M,P Q,N P,R R,Q)
				\tkzDrawSegments[dashed](B,D R,S M,N P,Q M,Q P,N P,S S,Q)
				\tkzDrawPoints[fill=black](A,B,C,D,M,N,P,Q,R,S,G)
				\tkzLabelPoints[above](A,G)
				\tkzLabelPoints[right](D)
				\tkzLabelPoints[left](B,R)
				\tkzLabelPoints[below](C)
				\tkzLabelPoints[below right](S,N)
				\tkzLabelPoints[below left](P)
				\tkzLabelPoints[above right](Q)
				\tkzLabelPoints[above left](M)
			\end{tikzpicture}}}
\end{vd}

\begin{vd}
	Cho hình chóp $S.ABCD$ có đáy là hình bình hành. Gọi $M,N,P,Q$ lần lượt là trung điểm $BC,CD,SB,SD$.
	\begin{tasks}(1)
		\task Chứng minh rằng $MN\parallel PQ$.
		\task Gọi $I$ là trọng tâm của tam giác $ABC$, $J$ thuộc $SA$ sao cho $\dfrac{JS}{JA}=\dfrac{1}{2}$. Chứng minh $IJ\parallel SM$.
	\end{tasks}
	\loigiai{
		\begin{enumerate}[a)]
			\immini{	\item Chứng minh $MN\parallel PQ$.\\
			      Ta có: $MN\parallel BD$ ($MN$ là đường trung bình của $\Delta BCD$).\\
			      và $PQ\parallel BD$ ($PQ$ là đường trung bình của $\Delta SBD$).\\
			      Suy ra $MN\parallel PQ$
			\item Chứng minh $IJ\parallel SM$.\\
			      $\dfrac{AJ}{AS}=\dfrac{2}{3}$.\\
			      $ \dfrac{JS}{JA}=\dfrac{1}{2}$.\\
			      $\dfrac{AI}{AM}=\dfrac{2}{3}$ ($I$ là trọng tâm của $\Delta ABC$).\\
			      Suy ra $\dfrac{AJ}{AS}=\dfrac{AI}{AM}$.\\
			      Theo định lí Viet đảo ta có $IJ\parallel SM$.}{\begin{tikzpicture}
				      \tkzDefPoints{-1/4/S,-2/0/B,2/0/C} %Định nghĩa các toạ đô dịnh cơ sở
				      \tkzDefPointBy[rotation = center B angle 60](C)
				      \tkzGetPoint{A1}
				      \tkzDefPointsBy[homothety=center B ratio 0.5](A1){A}
				      \tkzDefPointBy[translation = from B to A](C)
				      \tkzGetPoint{D}
				      \tkzDefMidPoint(S,B)
				      \tkzGetPoint{P}
				      \tkzDefMidPoint(C,B)
				      \tkzGetPoint{M}
				      \tkzDefMidPoint(C,D)
				      \tkzGetPoint{N}
				      \tkzDefMidPoint(S,D)
				      \tkzGetPoint{Q}
				      \tkzDefPointsBy[homothety=center S ratio 0.3333](A){J}
				      \tkzInterLL(A,M)(B,D)
				      \tkzGetPoint{I}
				      \tkzLabelPoints[left](P)
				      \tkzLabelPoints[above](S)
				      \tkzLabelPoints[right](N,Q,J)
				      \tkzLabelPoints[above left](A,B)
				      \tkzLabelPoints[above](I)
				      %\tkzLabelPoints[right](A,B,C,D,N,M,P)
				      \tkzLabelPoints[right](D)
				      \tkzLabelPoints[below](C,M)
				      \tkzDrawSegments[dashed](A,B S,A A,D Q,P A,C D,B I,J A,M M,N)
				      \tkzDrawSegments(S,B S,C S,D S,M B,C C,D)
			      \end{tikzpicture}}
		\end{enumerate}
	}
\end{vd}

\begin{dang}{Xác định giao tuyến $d$ của hai mặt phẳng cắt nhau}
	\indamm{Ta thực hiện một trong hai cách sau đây:}
	\begin{itemize}
		\item [\iconCH] \indamm{Cách 1:}Tìm hai điểm chung phân biệt (đã xét ở bài học trước)
		\item [\iconCH] \indamm{Cách 2:} Tìm 1 điểm chung. Sau đó nếu hai mặt phẳng có cặp đường thẳng song song nhau thì giao tuyến $d$ sẽ đi qua điểm chung và song song (hoặc trùng) với một trong hai đường thẳng đó.
	\end{itemize}
\end{dang}

\begin{vd}
	Cho tứ diện $ABCD$. Trên $AB$, $AC$ lần lượt lấy $M$, $N$ sao cho $\dfrac{AM}{AB}=\dfrac{AN}{AC}$. Tìm giao tuyến của hai mặt phẳng $(DBC)$ và $(DMN)$.
	\loigiai{
		\begin{center}
			\begin{tikzpicture}[>=stealth,line join=round,line cap=round,font=\footnotesize,scale=1]
				\coordinate (A) at (2,4.5);
				\coordinate (B) at (0,0);
				\coordinate (C) at (6,0);
				\coordinate (D) at (3,-2);
				\coordinate (M) at ($(A)!.7!(B)$);
				\coordinate (N) at ($(A)!.7!(C)$);
				\coordinate (X) at ($(C)+(D)-(B)$);
				\coordinate (Y) at ($(D)!.5!(X)$);
				\coordinate (Z) at ($(D)!-0.3!(X)$);
				\draw (A)--(B)--(D)--(C)--(A)--(D)--(M) (D)--(N) (Y)--(Z);
				\draw[dashed] (M)--(N) (B)--(C);
				\foreach \p/\pos in {A/above,B/left,C/right,D/below,M/above left, N/above right}
				\fill (\p) circle(1pt)node[\pos]{\p};
				\draw (Y) node[above] {$x$};
			\end{tikzpicture}
		\end{center}
		Trong tam giác $ABC$, theo giả thiết $\dfrac{AM}{AB}=\dfrac{AN}{AC}$ suy ra $MN\parallel BC$. \\
		Ta có $\left\{\begin{aligned}
				 & D\in (DBC)\cap (DMN)            \\
				 & BC\subset (DBC),MN\subset (DMN) \\
				 & BC\parallel MN
			\end{aligned}\right. $ suy ra \\
		$(DBC)\cap (DMN)=Dx\parallel BC\parallel MN$.
	}
\end{vd}

\begin{vd}
	Cho tứ diện $ABCD$. Gọi $M$, $N$ lần lượt là trung điểm của $AD$ và $BD$; $G$ là trọng tâm tam giác $ABC$. Tìm giao tuyến của hai mặt phẳng $(ABC)$ và $(MNG)$.
	\loigiai{
		\begin{center}
			\begin{tikzpicture}[>=stealth,line join=round,line cap=round,font=\footnotesize,scale=1]
				\coordinate (A) at (4,4);
				\coordinate (B) at (0,0);
				\coordinate (C) at (6,0);
				\coordinate (D) at (5,-2);
				\coordinate (M) at ($(A)!.5!(D)$);
				\coordinate (N) at ($(B)!.5!(D)$);
				\coordinate (X) at ($(A)!.5!(B)$);
				\coordinate (G) at ($(C)!2/3!(X)$);
				\coordinate (P) at ($(C)!2/3!(A)$);
				\coordinate (Q) at ($(C)!2/3!(B)$);
				\draw (A)--(B)--(D)--(C)--(A)--(D) (M)--(N);
				\draw[dashed] (M)--(G)--(N) (P)--(Q) (C)--(X) (B)--(C);
				\foreach \p/\pos in {A/above, B/left, D/below, C/right, M/right, N/below, P/right, G/above, Q/above}
				\fill (\p) circle(1pt) node[\pos]{\p};
			\end{tikzpicture}
		\end{center}
		Do $M$, $N$ lần lượt là trung điểm của $AD$ và $BD$ nên $MN\parallel AB$. \\
		Ta có $\left\{\begin{aligned}
				 & G\in (ABC)\cap (MNG)            \\
				 & AB\subset (ABC),MN\subset (MNG) \\
				 & AB\parallel MN
			\end{aligned}\right. $ suy ra \\
		$(ABC)\cap (MNG)=PQ\parallel AB\parallel MN$, \\
		với $PQ$ qua $G$ và song song với $AB$.
	}
\end{vd}

\begin{vd}
	Cho hình chóp $S.ABCD$ có đáy $ABCD$ là hình bình hành. Điểm $M$ thuộc cạnh $SA$. Điểm $E$, $F$ lần lượt là trung điểm của $AB$ và $BC$.
	\begin{enumEX}{2}
		\item Tìm $(SAB) \cap (SCD)$.
		\item Tìm $(MBC) \cap (SAD)$.
		\item Tìm $(MEF) \cap (SAC)$.
		\item Tìm $AD \cap (MEF)$.
		\item Tìm $SD \cap (MEF)$.
		%	\item Tìm thiết diện của hình chóp cắt bởi $(MEF)$.
	\end{enumEX}
	\loigiai{
		\begin{enumerate}
			\immini{	\item $\heva{& S \in (SAB) \cap (SCD)\\& AB \subset (SAB),~ CD \subset (SCD)\\& AB \parallel CD}$\\
			      $\Rightarrow (SAB) \cap (SCD) =Sx ~\text{với}~Sx \parallel AB \parallel CD$.
			\item $\heva{& M \in (MBC) \cap (SAD)\\& BC \subset (MBC),~ AD \subset (SAD)\\& BC \parallel AD}$\\
			      $\Rightarrow (MBC) \cap (SAD) =My ~\text{với}~My \parallel BC \parallel AD$.
			\item $\heva{& M \in (MEF) \cap (SAC)\\& EF \subset (MEF),~ AC \subset (SAC)\\& EF \parallel AC}$\\
			      $\Rightarrow (MEF) \cap (SAC) =Mz ~\text{với}~Mz \parallel EF \parallel AC$.
			      }{	\begin{tikzpicture}[scale=0.7,font=\footnotesize,line join=round,line cap=round,>=stealth]
				      \tikzset{label style/.style={font=\footnotesize}}
				      \tkzDefPoints{0/0/A, -3/-3/B, 6/0/D}
				      \coordinate (C) at ($(B)+(D)-(A)$);
				      \coordinate (S) at ($(A)+(1,6)$);
				      \coordinate (M) at ($(S)!2/5!(A)$);
				      \coordinate (E) at ($(A)!1/2!(B)$);
				      \coordinate (F) at ($(B)!1/2!(C)$);
				      \tkzDefPointBy[translation=from A to B](S) \tkzGetPoint{x}
				      \coordinate (U) at ($(x)!5/4!(S)$);
				      \tkzDefPointBy[translation=from A to D](M) \tkzGetPoint{y}
				      \coordinate (V) at ($(y)!5/4!(M)$);
				      \tkzInterLL(y,V)(S,B)    \tkzGetPoint{Z}
				      \tkzInterLL(y,V)(S,D)    \tkzGetPoint{W}
				      \tkzDefPointBy[translation=from A to C](M) \tkzGetPoint{z}
				      \coordinate (T) at ($(z)!2/5!(M)$);
				      \tkzInterLL(z,M)(S,C)    \tkzGetPoint{K}
				      \tkzInterLL(E,F)(A,D)    \tkzGetPoint{I}
				      \tkzInterLL(E,F)(S,B)    \tkzGetPoint{J}
				      \tkzInterLL(A,D)(S,B)    \tkzGetPoint{L}
				      \tkzInterLL(I,M)(S,B)    \tkzGetPoint{a}
				      \tkzInterLL(I,M)(S,D)    \tkzGetPoint{N}
				      \tkzDrawPolygon(S,B,C,D)
				      \tkzDrawSegments(S,C x,U y,W V,Z K,T I,L I,J I,a K,N K,F)
				      \tkzDrawSegments[dashed](S,A A,B A,D B,M C,M Z,W M,E E,F M,F A,C M,K A,L E,J a,N)
				      \tkzDrawPoints[fill=black](S,A,B,C,D,E,F,M,I,N,K)
				      \tkzLabelPoints[above](S,x,y)
				      \tkzLabelPoints[right](D,K)
				      \tkzLabelPoints[left](I)
				      \tkzLabelPoints[above left](A,M)
				      \tkzLabelPoints[above right](N)
				      \tkzLabelPoints[above right,,xshift=-1cm,yshift=0.8cm](z)
				      \tkzLabelPoints[below](B,C,F)
				      \tkzLabelPoints[below,xshift=-0.1cm,yshift=-0.2cm](E)
			      \end{tikzpicture}}
			\item Trong $(ABCD)$, gọi $I=EF \cap AD$.\\
			      Mà $EF \subset (MEF)$ nên $AD \cap (MEF)=I$.
			\item Trong $(SAD)$, gọi $N=SD \cap IM$.\\
			      Mà $IM \subset (MEF)$ nên $SD \cap (MEF)=N$.
			      %	\item Thiết diện của hình chóp cắt bởi $(MEF)$ là ngũ giác $MNKFE$.
		\end{enumerate}
	}
\end{vd}

\subsection{BÀI TẬP TỰ LUYỆN}
\begin{bt}
	Cho hình chóp $S.ABCD$ có đáy $ABCD$ là hình bình hành tâm $O$. Gọi $M$, $N$ lần lượt là trung điểm của $SA$, $SD$. Chứng minh
	\begin{listEX}[2]
		\item $MN\parallel AD$ và $MN\parallel BC$;
		\item $MO\parallel SC$ và $NO\parallel SB$.
	\end{listEX}
	\loigiai{
		\begin{center}
			\begin{tikzpicture}[line join = round, line cap = round,>=stealth,font=\footnotesize,scale=1]
				\tkzDefPoints{0/0/A}
				\coordinate (B) at ($(A)+(5,0)$);
				\tkzDefShiftPoint[A](-150:3){D}
				\coordinate (C) at ($(B)+(D)-(A)$);
				\tkzInterLL(A,C)(B,D)    \tkzGetPoint{O}
				\coordinate (S) at ($(A)+(-0.2,4)$);
				\coordinate (M) at ($(S)!0.5!(A)$);
				\coordinate (N) at ($(S)!0.5!(D)$);
				\tkzDrawPolygon(S,B,C,D)
				\tkzDrawSegments(S,C)
				\tkzDrawSegments[dashed](A,S A,B A,D A,C B,D M,N O,M O,N)
				\tkzMarkSegments[mark=||,size=2pt](M,A M,S)
				\tkzMarkSegments[mark=|,size=2pt](N,S N,D)
				\tkzDrawPoints[fill=black](A,B,D,C,O,S,M,N)
				\tkzLabelPoints[above](S)
				\tkzLabelPoints[below](O)
				\tkzLabelPoints[left](A,D,N)
				\tkzLabelPoints[above left](M)
				\tkzLabelPoints[right](B)
				\tkzLabelPoints[below right](C)
			\end{tikzpicture}
		\end{center}
		\begin{enumerate}
			\item Xét tam giác $SAD$ có
			      \begin{itemize}
				      \item $M$ là trung điểm của $SA$ (giả thiết);
				      \item $N$ là trung điểm của $SD$ (giả thiết).
			      \end{itemize}
			      Suy ra $MN$ là đường trung bình của $\triangle SAD$. Do đó $MN\parallel AD$.\\
			      Ta có $\heva{&MN\parallel AD~ (\text{chứng minh trên})\\ &BC\parallel AD ~(ABCD \text{ là hình bình hành})}\Rightarrow MN\parallel BC$.
			\item Xét tam giác $ASC$ có
		\end{enumerate}
		\begin{itemize}
			\item $M$ là trung điểm của $SA$ (giả thiết);
			\item $O$ là trung điểm của $AC$ ($O$ là tâm của hình bình hành $ABCD$).
		\end{itemize}
		Suy ra $OM$ là đường trung bình của $\triangle SAC$. Do đó $MO\parallel SC$.\\
		Tương tự, $NO$ là đường trung bình của $\triangle SDB$ nên $NO\parallel SB$.
	}
\end{bt}

\begin{bt}
	Cho hình chóp $S.ABCD$ có đáy $ABCD$ là hình bình hành tâm $O$. Gọi $M$, $N$ lần lượt là trung điểm của $AB$, $AD$. Gọi $I$, $J$, $G$ lần lượt là trọng tâm của các tam giác $SAB$, $SAD$ và $AOD$. Chứng minh
	\begin{listEX}[2]
		\item $IJ\parallel MN$;
		\item $IJ\parallel BD$ và $GJ\parallel SO$.
	\end{listEX}
	\loigiai{
		\begin{center}
			\begin{tikzpicture}[line join = round, line cap = round,>=stealth,font=\footnotesize,scale=1]
				\tkzDefPoints{0/0/A}
				\coordinate (D) at ($(A)+(5,0)$);
				\tkzDefShiftPoint[A](30:2.5){B}
				\coordinate (C) at ($(B)+(D)-(A)$);
				\tkzInterLL(A,C)(B,D)    \tkzGetPoint{O}
				\coordinate (S) at ($(B)+(0.2,3)$);
				\coordinate (M) at ($(A)!0.5!(B)$);
				\coordinate (N) at ($(A)!0.5!(D)$);
				\tkzDefPointBy[homothety = center S ratio 2/3](M)    \tkzGetPoint{I}
				\tkzDefPointBy[homothety = center S ratio 2/3](N)    \tkzGetPoint{J}
				\tkzDefPointBy[homothety = center O ratio 2/3](N)    \tkzGetPoint{G}
				\tkzDrawPolygon(S,A,D,C)
				\tkzDrawSegments(S,D S,N)
				\tkzDrawSegments[dashed](O,S B,S B,A B,C B,D A,C S,M M,N O,N I,J J,G)
				\tkzMarkSegments[mark=||,pos=.5,size=2pt](N,A N,D)
				\tkzMarkSegments[mark=|,pos=.5,size=2pt](M,A M,B)
				\tkzDrawPoints[fill=black](A,B,D,C,O,S,I,J,G)
				\tkzLabelPoints[above](S)
				\tkzLabelPoints[below](N,B)
				\tkzLabelPoints[left](A,M,I)
				\tkzLabelPoints[above right](J)
				\tkzLabelPoints[right](C,G)
				\tkzLabelPoints[below right](D,O)
			\end{tikzpicture}
		\end{center}
		\begin{enumerate}
			\item Xét tam giác $SMN$ có
			      \begin{itemize}
				      \item $SI=\dfrac{2}{3}SM$ ($I$ là trọng tâm của $\triangle SAB$);
				      \item $SJ=\dfrac{2}{3}SN$ ($J$ là trọng tâm của $\triangle SAD$).
			      \end{itemize}
			      suy ra $IJ\parallel MN$ (định lý Ta-lét đảo).
			\item Vì $MN$ là đường trung bình của $\triangle ABD$ nên $MN\parallel BD$.\\
			      Mà $IJ\parallel MN$ (chứng minh trên) nên $IJ\parallel BD$.\\
			      Xét tam giác $SON$ có
			      \begin{itemize}
				      \item $NG=\dfrac{1}{3}NO$ ($G$ là trọng tâm của $\triangle AOD$);
				      \item $NJ=\dfrac{1}{3}SN$ ($J$ là trọng tâm của $\triangle SAD$).
			      \end{itemize}
			      suy ra $GJ\parallel SO$ (định lý Ta-lét đảo).
		\end{enumerate}
	}
\end{bt}

\begin{bt}
	Cho hình chóp $SABCD$ có đáy $ABCD$ là hình thang đáy lớn $AB$. Gọi $E$,
	$F$ lần lượt là trung điểm của $SA$ và $SB$.
	\begin{listEX}[3]
		\item Chứng minh $EF \parallel CD$.
		\item Tìm $I = AF \cap (SCD)$.
		\item Chứng minh $SI \parallel AB \parallel CD$.
	\end{listEX}
	\loigiai{
		\begin{center}
			\begin{tikzpicture}[scale=1, line join=round, line cap=round]
				\tkzDefPoints{0/0/A,4/0/B,2.5/-1.6/C,1/3/S}
				\tkzDefPointBy[translation=from B to A](C)\tkzGetPoint{D'}
				\tkzDefPointBy[homothety=center C ratio 0.9](D')\tkzGetPoint{D}
				\coordinate (E) at ($(S)!0.5!(A)$);
				\coordinate (F) at ($(S)!0.5!(B)$);
				\tkzDefPointBy[translation=from A to B](S)\tkzGetPoint{x}
				\tkzInterLL(A,F)(S,x)\tkzGetPoint{I}
				\tkzInterLL(S,C)(I,D)\tkzGetPoint{K}

				\tkzDrawPolygon(S,B,C,D)
				\tkzDrawSegments(S,C I,D F,I S,I)
				\tkzDrawSegments[dashed](A,S A,B A,D A,F E,F)
				\tkzDrawPoints[fill=black,size=4](D,C,A,B,S,E,F,I)

				\tkzLabelPoints[above](S,F)
				\tkzLabelPoints[below](A,B,C)
				\tkzLabelPoints[right](D,E)
				\tkzLabelPoints[left](I)
			\end{tikzpicture}
		\end{center}
		\begin{listEX}[]
			\item Ta có $EF$ là đường trung bình của tam giác $SAB$ nên $EF
				\parallel AB$\\
			mà $AB \parallel CD$ (hai đáy của hình thang)\\
			nên $EF \parallel CD$.
			\item Hai mặt phẳng $(SAB)$ và $(SCD)$ có $AB \parallel CD$ nên giao
			tuyến là đường thẳng $Sx \parallel AB \parallel CD$.\\
			Kéo dài $AF$ cắt $Sx$ tại $I$.\\
			Ta thấy $I$ là điểm chung của $AF$ và $(SCD)$.
			\item Theo ý \circled{\textbf{2}}.
		\end{listEX}
	}
\end{bt}

\begin{bt}
	Cho hình chóp $S.ABCD$ có đáy $ABCD$ là hình bình hành. Gọi $M$, $N$ lần lượt là trung điểm của $SA$, $SB$. Gọi $P$ là một điểm trên cạnh $BC$. Tìm giao tuyến của
	\begin{listEX}[3]
		\item $(SBC)$ và $(SAD)$;
		\item $(SAB)$ và $(SCD)$;
		\item $(MNP)$ và $(ABCD)$.
	\end{listEX}
	\loigiai{
		\begin{center}
			\begin{tikzpicture}[line join = round, line cap = round,>=stealth,font=\footnotesize,scale=1]
				\tkzDefPoints{0/0/A}
				\coordinate (D) at ($(A)+(5,0)$);
				\tkzDefShiftPoint[A](30:2.5){B}
				\coordinate (C) at ($(B)+(D)-(A)$);
				\tkzInterLL(A,C)(B,D)    \tkzGetPoint{O}
				\coordinate (S) at ($(A)+(0.5,4)$);
				\coordinate (M) at ($(S)!0.5!(A)$);
				\coordinate (N) at ($(S)!0.5!(B)$);
				\coordinate (P) at ($(B)!0.6!(C)$);
				\coordinate (Q) at ($(A)!0.6!(D)$);
				%
				\coordinate (E) at ($(A)+(S)-(B)$);
				\coordinate (y) at ($(S)+(B)-(A)$);
				\coordinate (F) at ($(E)!0.8!(S)$);
				%
				\draw (0,4)--(4,4) node at (4.2,4){$x$};
				\tkzDrawPolygon(S,A,D,C)
				\tkzDrawPolygon[fill=cyan,dashed](M,N,P)
				\tkzDrawSegments(S,D F,y)
				\tkzDrawSegments[dashed](B,S B,A B,C B,D A,C P,Q)
				\tkzMarkSegments[mark=||,pos=.5,size=2pt](N,S N,B)
				\tkzMarkSegments[mark=|,pos=.5,size=2pt](M,A M,S)
				\tkzDrawPoints[fill=black](A,B,D,C,O,S,M,N,Q,P)
				\tkzLabelPoints[above](S,O,y)
				\tkzLabelPoints[below](Q)
				\tkzLabelPoints[left](M,B)
				\tkzLabelPoints[above right](P,N)
				\tkzLabelPoints[right](C,D)
				\tkzLabelPoints[below left](A)
			\end{tikzpicture}
		\end{center}
		\begin{enumerate}
			\item Ta có
			      \begin{itemize}
				      \item $(SBC)\cap (ABCD)=BC$;
				      \item $(SAD)\cap (ABCD)=AD$;
				      \item $AD\parallel BC$ ($ABCD$ là hình bình hành).
			      \end{itemize}
			      Mà $S$ là điểm chung của 2 mặt phẳng $(SBC)$ và $(SAD)$ nên giao tuyến của 2 mặt phẳng $(SBC)$ và $(SAD)$ là đường thẳng $Sx\parallel BC\parallel AD$.
		\end{enumerate}
		\begin{enumerate}
			\setcounter{enumi}{1}
			\item Giao tuyến của hai mặt phẳng $(SAB)$ và $(SCD)$ là đường thẳng $Sy\parallel AB\parallel CD$.
			\item Vì $MN\parallel AB$ ($MN$ là đường trung bình của $\triangle SAB$) nên qua $P$ kẻ $PQ\parallel AB~(Q\in AD)$. Khi đó giao tuyến của hai mặt phẳng $(MNP)$ và $(ABCD)$ là đường thẳng $PQ$.
		\end{enumerate}
	}
\end{bt}
\begin{bt}
	Cho tứ diện $SABC$. Gọi $E$ và $F$ lần lượt là trung điểm của các cạnh $SB$ và $AB$, $G$ là một điểm trên cạnh $AC$. Tìm giao tuyến của các cặp mặt phẳng sau
	\begin{listEX}[2]
		\item $(SAC)$ và $(EFC)$;
		\item $(SAC)$ và $(EFG)$.
	\end{listEX}
	\loigiai{
		\begin{center}
			\begin{tikzpicture}[line join = round, line cap = round,>=stealth,font=\footnotesize,scale=1]
				\tkzDefPoints{0/0/A}
				\coordinate (C) at ($(A)+(5,0)$);
				\tkzDefShiftPoint[A](-30:3){B}
				\coordinate (S) at ($(A)+(1.5,4)$);
				\coordinate (E) at ($(S)!0.5!(B)$);
				\coordinate (F) at ($(A)!0.5!(B)$);
				\coordinate (G) at ($(A)!0.13!(C)$);
				\coordinate (H) at ($(S)!0.13!(C)$);
				\coordinate (x) at ($(S)+(C)-(A)$);
				\coordinate (D) at ($(A)+(C)-(S)$);
				\coordinate (K) at ($(D)!0.8!(C)$);
				\tkzDrawPolygon(S,A,B,C)
				\tkzDrawSegments(S,B K,x E,F E,C)
				\tkzDrawSegments[dashed](A,C G,H F,C G,E F,G)
				\tkzMarkSegments[mark=||,pos=.5,size=2pt](E,S E,B)
				\tkzMarkSegments[mark=|,pos=.5,size=2pt](F,A F,B)
				\tkzDrawPoints[fill=black](B,C,A,S,E,F,G,H)
				\tkzLabelPoints[above](S)
				\tkzLabelPoints[below](B)
				\tkzLabelPoints[left](A)
				\tkzLabelPoints[right](C,x)
				\tkzLabelPoints[above right](H,E)
				\tkzLabelPoints[below left](F)
				\tkzLabelPoints[above left](G)
			\end{tikzpicture}
		\end{center}
		\begin{enumerate}
			\item Ta có
			      \begin{itemize}
				      \item $(SAC)\cap (SAB)=SA$;
				      \item $(EFC)\cap (SAB)=EF$;
				      \item $SA\parallel EF$ ($EF$ là đường trung bình của $\triangle SAB$).
			      \end{itemize}
			      Do đó giao tuyến của 2 mặt phẳng $(SAC)$ và $(EFC)$ sẽ song song với $SA$ và $EF$.\\
			      Mà $C$ là điểm chung của 2 mặt phẳng $(SAC)$ và $(EFC)$ nên giao tuyến của chúng là đường thẳng $Cx\parallel SA\parallel EF$.

			\item Vì $EF\parallel SA$ ($EF$ là đường trung bình của $\triangle SAB$) nên qua $G$ kẻ $GH\parallel SA~(H\in SC)$. Khi đó giao tuyến của hai mặt phẳng $(SAC)$ và $(EFG)$ là đường thẳng $GH$.
		\end{enumerate}
	}
\end{bt}

\begin{bt}
	Cho hình chóp $S.ABCD$ có đáy $ABCD$ là hình bình hành. Gọi $G$ là trọng tâm tam giác $ABD$, $N$ là trung điểm $SG$. Tìm giao tuyến của hai mặt phẳng $(ABN)$ và $(SCD)$.
	\loigiai{
		\begin{center}
			\begin{tikzpicture}[>=stealth,line join=round,line cap=round,font=\footnotesize,scale=0.9]
				\coordinate (S) at (0,5);
				\coordinate (A) at (-3,-1.5);
				\coordinate (B) at (0,0);
				\coordinate (C) at (5,0);
				\coordinate (D) at ($(A)+(C)-(B)$);
				\coordinate (O) at (intersection of A--C and B--D);
				\coordinate (G) at ($(A)!2/3!(O)$);
				\coordinate (N) at ($(S)!.5!(G)$);
				\coordinate (P) at (intersection of A--N and S--C);
				\coordinate (X) at ($(P)+(D)-(C)$);
				\coordinate (Q) at (intersection of P--X and S--D);
				\draw (S)--(A)--(D)--(C)--(S)--(D) (P)--(Q);
				\draw[dashed] (S)--(B)--(A)--(O) (B)--(D) (S)--(G) (B)--(C) (A)--(P);
				\foreach \p/\pos in {S/above, A/below, D/below, C/right, B/above right, G/below, P/right, Q/left, N/left, O/below}
				\fill (\p) circle(1pt) node[\pos]{\p};
			\end{tikzpicture}
		\end{center}
		Trong mặt phẳng $(SAC)$, gọi $P=AN\cap SC$. Ta có \\
		$P\in AN$ mà $AN\subset (ABN)$ suy ra $P\in (ABN)$. \\
		$P\in SC$ mà $SC\subset (SCD)$ suy ra $P\in (SCD)$. \\
		Do đó $P\in (ABN)\cap (SCD)$. \\
		Ta có $\left\{\begin{aligned}
				 & P\in (ABN)\cap (SCD)            \\
				 & AB\subset (ABN),CD\subset (SCD) \\
				 & AB\parallel CD
			\end{aligned}\right. $ suy ra \\
		$(ABN)\cap (SCD)=PQ\parallel CD\parallel AB$.
	}
\end{bt}

\begin{bt}
	Cho tứ diện $ABCD$. Gọi $M, N$ theo thứ tự là trung điểm của $AB, BC$ và $Q$ là một điểm nằm trên cạnh $AD$ ($QA\neq QD$) và $P$ là giao điểm của $CD$ với mặt phẳng $(MNQ)$. Chứng minh rằng $PQ\parallel MN$ và $PQ\parallel AC$.
	\loigiai{
		\immini{
			Vì $QA\neq QD$ nên gọi $K=QM\cap BD$ suy ra $KN \cap CD=P$.\\
			Theo định lý về giao tuyến ba mặt phẳng\\
			Ta xét ba mặt phẳng $(ABC)$ $(ACD)$ và $(MNQ)$.\\
			Ta có: $\heva{&(ABC)\cap (ACD)=AC\\&(ABC)\cap (MNQ)=MN\\&(ACD)\cap (MNQ)=QP}$.\\
			Vậy $AC \parallel MN $ nên $AC\parallel QP\parallel NM$.}
		{\begin{tikzpicture}%[scale=1]
				%\tkzInit[xmin=-5,ymin=-1.5,xmax=3,ymax=2]
				%\tkzClip
				\tkzDefPoints{-3/0/B,2/0/D,-1/3/A} % Định nghĩa các toạ đô dịnh cơ sở
				\tkzDefPointsBy[homothety=center B ratio 0.6](D){B1}
				\tkzDefPointBy[rotation = center B angle -30](B1)
				\tkzGetPoint{C}
				\tkzDefMidPoint(A,B)
				\tkzGetPoint{M}
				\tkzDefMidPoint(C,B)
				\tkzGetPoint{N}
				\tkzDefPointsBy[homothety=center A ratio 0.3](D){Q}
				\tkzInterLL(M,Q)(B,D)
				\tkzGetPoint{K}
				\tkzInterLL(K,N)(C,D)
				\tkzGetPoint{P}
				\tkzLabelPoints[above](A)
				\tkzLabelPoints[above right](Q)
				\tkzLabelPoints[above left](K,B,M)
				\tkzLabelPoints[right](D,P)
				\tkzLabelPoints[below](C,N)
				\tkzDrawSegments[dashed](B,D M,Q M,B B,N B,K N,P)
				\tkzDrawSegments(A,M A,C A,D A,M Q,P C,D N,C M,K N,K M,N)
			\end{tikzpicture}}
	}
\end{bt}

\begin{bt}
	Cho hình chóp $ S.ABCD $ có đáy $ ABCD $ là hình thang với $ AD $ là đáy lớn và $ AD=2BC $. Gọi $ M $, $ N $, $ P $ lần lượt thuộc các đoạn $ SA $, $ AD $, $ BC $ sao cho $ MA=2MS $, $ NA=2ND $, $ PC=2PB $.
	\begin{itemize}
		\item[a)] Tìm giao tuyến của các cặp mặt phẳng sau: $ (SAD) $ và $ (SBC) $, $ (SAC) $ và $ (SBD) $.
		\item[b)] Xác định giao điểm $ Q $ của $ SB $ với $ (MNP) $.
		\item[c)] Gọi $ K $ là trung điểm của $ SD $. Chứng minh $ CK=(MQK)\cap (SCD) $.
	\end{itemize}
	\loigiai{
		\begin{center}
			\begin{tikzpicture}[line join=round, line cap=round,thick,scale=0.8]
				\tikzset{label style/.style={font=\footnotesize}}
				\coordinate (A) at (0,0);
				\coordinate (B) at (-0.8,-3);
				\coordinate (D) at (10,0);
				\tkzDefPointWith[colinear = at B,K=0.5](A,D)
				\tkzGetPoint{C}
				\coordinate (S) at ($(A)+(-0.5,6)$);
				\coordinate (N) at ($(A)!0.6667!(D)$);
				\coordinate (M) at ($(S)!0.3333!(A)$);
				\coordinate (P) at ($(B)!0.3333!(C)$);

				\tkzInterLL(A,C)(B,D) \tkzGetPoint{O}
				\tkzInterLL(N,P)(B,A) \tkzGetPoint{E}
				\tkzInterLL(E,M)(B,S) \tkzGetPoint{Q}
				\coordinate (K) at ($(S)!0.5!(D)$);
				\coordinate (K') at ($(S)!0.5!(A)$);
				\coordinate (t) at ($(S)+(K)-(K')$);
				\tkzInterLL(K,M)(t,S) \tkzGetPoint{F}
				\tkzDrawSegments(S,B S,C B,C C,D S,D S,t Q,E C,K P,E B,E F,C S,F)
				\tkzDrawSegments[dashed,thin](S,A A,B A,C A,D B,D S,O N,P M,Q M,N F,K Q,K K,K')
				\tkzDrawPoints[fill=black,size=3pt](S,A,B,C,D,M,N,P,O,E,Q,K,F,K')
				\tkzLabelPoints[above](S,N,K,t)
				\tkzLabelPoint[above](F){$ F\equiv F' $}
				\tkzLabelPoints[above left](A)
				\tkzLabelPoints[below](C,P,E)
				\tkzLabelPoints[right](D)
				\tkzLabelPoints[below left](Q)
				\tkzLabelPoints[below right](K',B)
				\tkzLabelPoints[above right](M,O)
			\end{tikzpicture}
		\end{center}
		\begin{listEX}
			\item Vì $ \heva{&S\in (SAD) \cap (SBC)\\&AD\subset (SAD) \text{ và } BC\subset (SBC) \\& AD\parallel BC} $\\ nên $ (SAD)\cap (SBC) = St \parallel AD \parallel BC $.\\
			Gọi $ O=AC\cap BD\Rightarrow \heva{&O\in AC\subset (SAC) \\ &O\in BD\subset (SBD)} $
			suy ra $ SO=(SAC)\cap (SBD) $.
			\item Gọi $ E=NP\cap AB $ và $ Q=EM\cap SB $.
			Vì $ \heva{&Q\in SB \\&Q\in ME\subset (MNP)} $ nên $ Q=SB\cap (MNP) $.
			\item Gọi $ F=MK\cap St $ và $ F'=QC\cap St $. Dựa vào các vị trí các điểm $ Q $, $ C $, $ M $ và $ K $ của giả thiết cho, dễ thấy $ F $ và $ F' $ cùng nằm về một phía so với mặt phẳng $ (SAB) $.\\
			Trong mặt phẳng $ (SF'BC) $, áp dụng định lý Thales (để ý rằng $ SF'\parallel BC $) ta có
			\[ \dfrac{QS}{QB}=\dfrac{BC}{SF'}=\dfrac{1}{2}. \tag {1}\]
			Gọi $ K' $ là trung điểm của $ SA $. suy ra $ \dfrac{MK'}{MS}=\dfrac{1}{2}. $\\
			Trong mặt phẳng $ (SFAD) $, áp dụng định lý Thales (để ý rằng $ SF\parallel KK' $) ta có
			\[ \dfrac{MK'}{MS}=\dfrac{KK'}{SF}=\dfrac{1}{2}. \tag {2}\]
			Từ (1), (2) và $ AD=2BC $ suy ra $ SF=SF' $. Do đó $ F\equiv F' $, suy ra bốn điểm $ Q $, $ C $, $ M $ và $ K $ đồng phẳng.\\
			Vậy $ CK=(MQK)\cap (SCD) $.
		\end{listEX}
	}
\end{bt}

\begin{bt}
	Cho hình chóp $ S.ABCD $ có $ O $ là tâm của hình bình hành $ ABCD $, điểm $ M $ thuộc cạnh $ SA $ sao cho $ SM=2MA $, $ N $ là trung điểm của $ AD $.
	\begin{itemize}
		\item[a)] Tìm giao tuyến của mặt phẳng $ (SAD) $ và $ (MBC) $.
		\item[b)] Tìm giao điểm $ I $ của $ SB $ và $ (CMN) $, giao điểm $ J $ của $ SA $ và $ (ICD) $.
		\item[c)] Chứng minh ba đường thẳng $ ID $, $ JC $, $ SO $ cắt nhau tại $ E $. Tính tỉ số $ \dfrac{SE}{SO} $.
	\end{itemize}
	\loigiai{
		\begin{center}
			\begin{tikzpicture}[line join=round, line cap=round,thick,scale=0.8]
				\tikzset{label style/.style={font=\footnotesize}}
				\coordinate (A) at (0,0);
				\coordinate (B) at (-2.5,-2);
				\coordinate (D) at (8,0);
				\coordinate (C) at ($(B)+(D)-(A)$);
				\coordinate (O) at ($(A)!0.5!(C)$);
				\coordinate (S) at ($(O)+(-1,7)$);
				\coordinate (M) at ($(S)!0.6667!(A)$);
				\coordinate (P) at ($(S)!0.6667!(D)$);
				\coordinate (N) at ($(A)!0.5!(D)$);
				\coordinate (S') at ($(S)+(A)-(D)$);
				\coordinate (t) at ($(S)+(P)-(M)$);
				\tkzInterLL(S,S')(M,N) \tkzGetPoint{F}
				\tkzInterLL(S,B)(C,F) \tkzGetPoint{I}
				\coordinate (I') at ($(I)+(D)-(C)$);
				\tkzInterLL(I,I')(S,A) \tkzGetPoint{J}
				\tkzInterLL(S,O)(I,D) \tkzGetPoint{E}
				\tkzDrawSegments(S,B S,C B,C C,F S,F C,D S,D P,C S,t)
				\tkzDrawSegments[dashed,thin](S,A A,B B,M A,C A,D S,O B,D M,P F,N C,N I,J I,D J,C C,M)
				\tkzDrawPoints[fill=black,size=3pt](S,A,B,C,D,O,M,N,P,F,I,J,E)
				\tkzLabelPoints[above](S,F,N,t)
				\tkzLabelPoints[above right](J,M,E)
				\tkzLabelPoints[left](A)
				\tkzLabelPoints[below left](I)
				\tkzLabelPoints[below](B,C,O)
				\tkzLabelPoints[right](D,P)
			\end{tikzpicture}
		\end{center}
		\begin{listEX}
			\item Vì $ \heva{& M\in (MBC) \cap (SAD) \\ & BC \subset (MBC) \text{ và } AD \subset (SAD) \\& BC\parallel AD}$\\ nên $ (SAD)\cap (MBC)=MP\parallel BC \parallel AD $ (với $ P\in SD $).
			\item Vì $ \heva{& S\in (SAD)\cap (SBC)\\& AD\subset (SAD) \text{ và } BC\subset (SBC) \\ &AD \parallel BC} $ \\ nên $ (SAD) \cap (SBC) =St \parallel AD \parallel BC $.\\
			Gọi $ F=MN\cap St $; $ I=CF\cap SB $.\\
			Vì $ \heva{&I\in SB \\ &I\in CF \subset (CMN)} $ nên $ I=SB\cap (CMN) $.\\
			Qua $ I $ kẻ đường thẳng song song với $ AB $  cắt $ SA $ tại $ J $.\\
			Vì $ \heva{&J\in SA \\ & J\in JI\subset (ICD) (\text{vì } IJ\parallel CD \Rightarrow (IJCD)\equiv (ICD))} $ nên $ J=SA\cap (ICD) $.
			\item Xét $ 3 $ mặt phẳng $ (SAC) $, $ (SBD) $ và $ (CDJI) $, ta có
			$$\heva{& SO=(SAC)\cap (SBD)\\& ID=(SBD)\cap (CDJI)\\& JC=(SAC)\cap (CDJI).}$$
			Do đó ba đường thẳng $ ID $, $ JC $, $ SO $ đồng quy. Gọi điểm đồng quy là $ E $.\\
			Trong mặt phẳng $ (SFAD) $, áp dụng định lý Thales (để ý rằng $ AN\parallel SF $) ta có
			\[ \dfrac{MA}{MS}=\dfrac{AN}{SF}=\dfrac{1}{2}. \]
			Suy ra $ SF=AD=BC $ và $ SFBC $ là hình bình hành.\\
			$ I=SB\cap CF $ nên $ I $ là trung điểm của $ SB $.\\
			$ \triangle SBD $ có $ DI $ và $ SO $ là trung tuyến nên $ E $ là trọng tâm của $ \triangle SBD $.\\
			Vậy $ \dfrac{SE}{SO}=\dfrac{2}{3} $.
		\end{listEX}
	}
\end{bt}
\begin{bt}
	Cho hình chóp $S.ABCD$ có đáy $ABCD$ là hình bình hành. Gọi $M, N, P, Q$ lần lượt là trung điểm của các cạnh $AB, BC, CD, DA$; gọi $I, J, K, L$ lần lượt là trung điểm của các đoạn thẳng $SM, SN, SP, SQ$.
	\begin{itemize}
		\item [a)] Chứng minh rằng bốn điểm $I, J, K, L$ đồng phẳng và tứ giác $IJKL$ là hình bình hành.
		\item [b)] Chứng minh rằng $IK\parallel BC$.
		\item [c)] Xác định giao tuyến của hai mặt phẳng $\left(IJKL\right)$ và $\left(SBC\right)$.
	\end{itemize}
\end{bt}


% \subsection{BÀI TẬP TRẮC NGHIỆM}
\Opensolutionfile{ans}[ans/1H4.B2]
\setcounter{ex}{0}

\begin{ex}%[Huỳnh Văn Quy]%[1H2Y2]
	Hai đường thẳng không có điểm chung thì
	\choice{chéo nhau}
	{song song}
	{cắt nhau}
	{\True chéo nhau hoặc song song}
\end{ex}

\begin{ex}%[Huỳnh Văn Quy]%[1H2Y2]
	Hai đường thẳng phân biệt không song song thì
	\choice{chéo nhau}
	{có điểm chung}
	{\True cắt nhau hoặc chéo nhau}
	{ không có điểm chung}
\end{ex}

\begin{ex}%[TLDH9-Lê Hồng Phi]%[1H2Y2-1]%
	Cho hai đường thẳng phân biệt không có điểm chung cùng nằm trong một mặt phẳng thì hai đường thẳng đó
	\choice
	{trùng nhau}
	{chéo nhau}
	{\True song song}
	{cắt nhau}
	\loigiai{
		Hai đường thẳng phân biệt không có điểm chung cùng nằm trong một mặt phẳng thì hai đường thẳng đó song song.}
\end{ex}

\begin{ex}%[Đề HKI Lớp 11, Lạc Long Quân, Khánh Hòa 2017-2018]%[Lê Hồng Phi - DA 11HK1-18]%[1H2Y2-1]%
	Chọn khẳng định {\bf sai}
	\choice
	{Hai đường thẳng phân biệt cùng song song với đường thẳng thứ ba thì song song với nhau}
	{Nếu hai đường thẳng chéo nhau thì chúng không đồng phẳng}
	{\True Hai đường thẳng song song thì không đồng phẳng và không có điểm chung}
	{Hai đường thẳng cắt nhau thì đồng phẳng và có một điểm chung}
	\loigiai{
		Khẳng định sai là \lq\lq  Hai đường thẳng song song thì không đồng phẳng và không có điểm chung\rq\rq.
	}
\end{ex}

\begin{ex}%[Ân Thọ]%[1H2B2]
	Cho đường thẳng $a$ cắt mặt phẳng $(P)$ tại điểm $A$. Mệnh đề nào sau đây đúng?
	\choice
	{Mọi đường thẳng nằm trong $(P)$ đều chéo với $a$}
	{Mọi đường thẳng nằm trong $(P)$ đều cắt $a$}
	{\True Mọi đường thẳng nằm trong $(P)$ hoặc chéo với $a$, hoặc cắt $a$}
	{Mọi đường thẳng nằm trong $(P)$ đều không cắt $a$}
\end{ex}

\begin{ex}%[Ân Thọ]%[1H2B2]
	% \immini[thm]{
		Cho tứ diện $ABCD$. Gọi $M$ và $N$ là hai điểm phân biệt nằm trên đường thẳng $AB$, $M'$ và $N'$ là hai điểm phân biệt nằm trên đường thẳng $CD$. Các mệnh đề sau đây, mệnh đề nào đúng?	
		\choice
		{Hai đường thẳng $MM'$ và $NN'$ có thể cắt nhau}
		{Hai đường thẳng $MM'$ và $NN'$ có thể song song với nhau}
		{Hai đường thẳng $MM'$ và $NN'$ hoặc cắt nhau hoặc song song với nhau}
		{\True Hai đường thẳng $MM'$ và $NN'$ chéo nhau}
	% }{\begin{tikzpicture}[line cap=round,line join=round,>=stealth,x=0.75cm,y=0.75cm]
	% 	\draw (0.0,-0.0)-- (2.0,-2.0);
	% 	\draw (2.0,-2.0)-- (5.0,0.0);
	% 	\draw (5.0,0.0)-- (3.0,3.0);
	% 	\draw (3.0,3.0)-- (0.0,-0.0);
	% 	\draw (3.0,3.0)-- (2.0,-2.0);
	% 	%\draw [dash pattern=on 2pt off 2pt] (1.5,1.5)-- (3.5,-1.0);
	% 	\draw [dash pattern=on 2pt off 2pt] (0.0,-0.0)-- (5.0,0.0);
	% 	\begin{scriptsize}
	% 	\draw [fill=black] (0.0,-0.0) circle (1pt);
	% 	\draw[color=black] (-0.06,0.2) node {$B$};
	% 	\draw [fill=black] (2.0,-2.0) circle (1pt);
	% 	\draw[color=black] (1.7,-1.95) node {$C$};
	% 	\draw [fill=black] (5.0,0.0) circle (1pt);
	% 	\draw[color=black] (5.07,0.15) node {$D$};
	% 	\draw [fill=black] (3.0,3.0) circle (1pt);
	% 	\draw[color=black] (3.2,3.0) node {$A$};
	% 	\end{scriptsize}
	% 	\end{tikzpicture}
	% }
\end{ex}

\begin{ex}%[Nguyễn Tiến Thùy]%[1H2B2]
	Cho tứ diện $ABCD$, lấy $M, N$ lần lượt là trung điểm của $CD, AB$. Khi đó, xác định vị trí tương đối giữa hai đường thẳng $BC$ và $MN$.
	\choice
	{\True Chéo nhau}
	{Có hai điểm chung}	
	{Song song}	
	{Cắt nhau}
\end{ex}

\begin{ex}%[Nguyễn Tiến Thùy]%[1H2B2]
	Cho tứ diện $MNPQ$. Mệnh đề nào trong các mệnh đề dưới đây là đúng?
	\choice
	{$MN\parallel PQ$}
	{$MN$ cắt $PQ$}
	{$MN$ và $PQ$ đồng phẳng}	
	{\True $MN$ và $PQ$ chéo nhau}
\end{ex}

\begin{ex}%[Nguyễn Tiến Thùy]%[1H2B2]
	Cho hình chóp $S.ABCD$, đáy $ABCD$ là hình bình hành. Điểm $M$ thuộc cạnh $SC$ sao cho $SM=3MC$, $N$ là giao điểm của $SD$ và $(MAB)$. Khi đó tứ giác $ABMN$ là hình gì?
	\choice
	{Tứ giác không có cặp cạnh nào song song}
	{Hình vuông}
	{\True Hình thang}
	{Hình bình hành}
\end{ex}

\begin{ex}%[Nguyễn Tiến Thùy]%[1H2B2]
	Cho hình chóp $S.ABCD$ có đáy $ABCD$ là hình thang $AB\parallel CD$. Gọi $d$ là giao tuyến của hai mặt phẳng $(ASB)$ và $(SCD)$. Khẳng định nào sau đây là đúng?
	\choice
	{\True $d\parallel AB$}
	{$d$ cắt $AB$}
	{$d$ cắt $AD$}	
	{$d$ cắt $CD$}
\end{ex}

\begin{ex}%[Huỳnh Văn Quy]%[1H2B2]
	Cho hình chóp $S.ABCD$. Gọi $G$, $E$ lần lượt là trọng tâm các tam giác $SAD$ và $SCD$. Lấy $M$, $N$ lần lượt là trung điểm $AB$, $BC$ . Khi đó ta có:
	\choice
	{$GE$ và $MN$ trùng nhau}	
	{$GE$ và $MN$ chéo nhau}
	{\True $GE$ và $MN$ song song với nhau}
	{$GE$ cắt $BC$}
	\loigiai{Gọi $I$ là trung điểm của $SD$. Xét tam giác $IAC$ có: $\dfrac{IG}{IA}=\dfrac{IE}{IC}=\dfrac{1}{3}$.
		\immini{Theo định lí đảo của định lí Thalet, ta có $GE||AC$\quad (1).\\
			Mặt khác, ta có $MN$ là đường trung bình của $\triangle ABC$ nên suy ra $MN||AC$\quad (2).\\
			Từ (1) và (2) ta có $MN||GE$}
		{\begin{tikzpicture}[scale=1.5,line cap=round,line join=round,>=stealth,x=1.0cm,y=1.0cm]
			\draw [line width=0.8pt,dash pattern=on 1pt off 1pt] (0.539,2.203)-- (2.201,2.151);
			\draw [line width=0.8pt] (0.539,2.203)-- (0.983,1.253);
			\draw [line width=0.8pt] (0.983,1.253)-- (1.757,1.439);
			\draw [line width=0.8pt] (1.757,1.439)-- (2.201,2.151);
			\draw [line width=0.8pt] (1.241,3.339)-- (0.539,2.203);
			\draw [line width=0.8pt] (1.241,3.339)-- (2.201,2.151);
			\draw [line width=0.8pt] (1.241,3.339)-- (1.757,1.439);
			\draw [line width=0.8pt] (1.241,3.339)-- (0.983,1.253);
			\draw [line width=0.8pt,dash pattern=on 1pt off 1pt] (0.539,2.203)-- (1.757,1.439);
			\draw [line width=0.8pt,dash pattern=on 1pt off 1pt] (0.761,1.728)-- (1.370,1.346);
			\draw [line width=0.8pt] (1.721,2.745)-- (1.757,1.439);
			\draw [line width=0.8pt] (1.979,1.795)-- (1.241,3.339);
			\draw [line width=0.8pt,dash pattern=on 1pt off 1pt] (1.370,2.177)-- (1.241,3.339);
			\draw [line width=0.8pt,dash pattern=on 1pt off 1pt] (1.327,2.565)-- (1.733,2.310);
			\draw [line width=0.8pt,dash pattern=on 1pt off 1pt] (0.539,2.203)-- (1.721,2.745);
			\begin{scriptsize}
			\draw [fill=black] (0.539,2.203) circle (1.0pt);
			\draw[color=black] (0.404,2.239) node {$A$};
			\draw [fill=black] (2.201,2.151) circle (1.0pt);
			\draw[color=black] (2.273,2.281) node {$D$};
			\draw [fill=black] (0.983,1.253) circle (1.0pt);
			\draw[color=black] (0.879,1.114) node {$B$};
			\draw [fill=black] (1.757,1.439) circle (1.0pt);
			\draw[color=black] (1.829,1.403) node {$C$};
			\draw [fill=black] (1.241,3.339) circle (1.0pt);
			\draw[color=black] (1.313,3.468) node {$S$};
			\draw [fill=black] (1.721,2.745) circle (1.0pt);
			\draw[color=black] (1.798,2.869) node {$I$};
			\draw [fill=black] (1.370,1.346) circle (1.0pt);
			\draw[color=black] (1.406,1.238) node {$N$};
			\draw [fill=black] (0.761,1.728) circle (1.0pt);
			\draw[color=black] (0.559,1.671) node {$M$};
			\draw [fill=black] (1.733,2.310) circle (1.0pt);
			\draw[color=black] (1.809,2.435) node {$E$};
			\draw [fill=black] (1.327,2.565) circle (1.0pt);
			\draw[color=black] (1.189,2.663) node {$G$};
			\end{scriptsize}
			\end{tikzpicture}}}
\end{ex}

\begin{ex}%[Ân Thọ]%[1H2B2]
	\immini[thm]{Cho tứ diện $ABCD$ có $P, Q$ lần lượt là trọng tâm tam giác $ABC$ và $BCD$. Xác định giao tuyến của mặt phẳng $(ABQ)$ và mặt phẳng $(CDP)$.
		\choice
		{\True Giao tuyến là đường thẳng đi qua trung điểm hai cạnh $AB$ và $CD$}
		{Giao tuyến là đường thẳng đi qua trung điểm hai cạnh $AB$ và $AD$}
		{Giao tuyến là đường thẳng $PQ$}
		{Giao tuyến là đường thẳng $QA$}}
	{\begin{tikzpicture}[line cap=round,line join=round,>=stealth,x=0.75cm,y=0.75cm]
		\draw (0.0,-0.0)-- (2.0,-2.0);
		\draw (2.0,-2.0)-- (5.0,0.0);
		\draw (5.0,0.0)-- (3.0,3.0);
		\draw (3.0,3.0)-- (0.0,-0.0);
		\draw (3.0,3.0)-- (2.0,-2.0);
		%\draw [dash pattern=on 2pt off 2pt] (1.5,1.5)-- (3.5,-1.0);
		\draw [dash pattern=on 2pt off 2pt] (0.0,-0.0)-- (5.0,0.0);
		\begin{scriptsize}
		\draw [fill=black] (0.0,-0.0) circle (1pt);
		\draw[color=black] (-0.06,0.2) node {$B$};
		\draw [fill=black] (2.0,-2.0) circle (1pt);
		\draw[color=black] (1.7,-1.95) node {$C$};
		\draw [fill=black] (5.0,0.0) circle (1pt);
		\draw[color=black] (5.07,0.15) node {$D$};
		\draw [fill=black] (3.0,3.0) circle (1pt);
		\draw[color=black] (3.2,3.0) node {$A$};
		\end{scriptsize}
		\end{tikzpicture}
	}
\end{ex}

\begin{ex}%[1H2B2-2]
	Cho tứ diện $ABCD.$ Gọi $I$ và $J$ theo thứ tự là trung điểm của $AD$ và $AC,G$ là trọng tâm tam giác $BCD.$ Giao tuyến của hai mặt phẳng $\left(GIJ\right)$ và $\left(BCD\right)$ là đường thẳng
	\choice
	{qua $J$ và song song với $BD$}
	{qua $G$ và song song với $BC$}
	{qua $I$ và song song với $AB$}
	{\True qua $G$ và song song với $CD$}
	\loigiai{
		\immini{
			Ta có $\heva{
				& \left(GIJ\right)\cap \left(BCD\right)=G \\
				& IJ\subset \left(GIJ\right),\ CD\subset \left(BCD\right) \\
				& IJ\parallel CD}$\\
			$\Rightarrow \left(GIJ\right)\cap \left(BCD\right)=Gx\parallel IJ\parallel CD.$
		}
		{
			\begin{tikzpicture}[scale=0.7, line join=round, line cap=round]
			\tkzDefPoints{0/0/C,1.5/-1.8/B,4/0/D,2/3/A}
			\tkzDefMidPoint(A,C)\tkzGetPoint{J}
			\tkzDefMidPoint(D,A)\tkzGetPoint{I}
			\tkzCentroid(D,B,C)\tkzGetPoint{G}
			\tkzDefLine[parallel=through G](D,C)\tkzGetPoint{g}
			\tkzInterLL(g,G)(C,B)\tkzGetPoint{E}
			\tkzInterLL(g,G)(D,B)\tkzGetPoint{F}
			\tkzDrawPolygon(A,C,B,D)
			\tkzDrawSegments(A,B)
			\tkzDrawSegments[dashed](C,D I,J E,F)
			\tkzDrawPoints[fill=black](I,A,B,C,D,J,G)
			\tkzLabelPoints[above](A)
			\tkzLabelPoints[below](B,G)
			\tkzLabelPoints[left](C,J)
			\tkzLabelPoints[right](D,I)
			\end{tikzpicture}
		}
	}
\end{ex}

\begin{ex}%[1H2B2-2]
	Cho hình chóp $S.ABCD$ có đáy là hình thang với các cạnh đáy là $AB$ và $CD.$ Gọi $\left(ACI\right)$ lần lượt là trung điểm của $AD$ và $BC$ và $G$ là trọng tâm của tam giác $SAB.$ Giao tuyến của $\left(SAB\right)$ và $\left(IJG\right)$ là
	\choice
	{đường thẳng qua $S$ và song song với $AB$}
	{\True đường thẳng qua $G$ và song song với $DC$}
	{$SC$}
	{đường thẳng qua $G$ và cắt $BC$}
	\loigiai{
		\immini{
			Ta có: $I,J$ lần lượt là trung điểm của $AD$ và $BC$
			$\Rightarrow IJ$ là đường trung bình của hình thang $ABCD\Rightarrow IJ\parallel AB\parallel CD.$ \\
			Gọi $d=\left(SAB\right)\cap \left(IJG\right)$
			Ta có: $G$ là điểm chung giữa hai mặt phẳng $\left(SAB\right)$ và $\left(IJG\right)$ \\
			Mặt khác: $\heva{
				& \left(SAB\right)\supset AB;\left(IJG\right)\supset IJ \\
				& AB\parallel IJ \\}$ \\
			$\Rightarrow $Giao tuyến $d$ của $\left(SAB\right)$ và $\left(IJG\right)$ là đường thẳng qua $G$ và song song với $AB$ và $IJ.$
		}
		{
			\begin{tikzpicture}[scale=1, line join=round, line cap=round]
			\tkzDefPoints{0/0/A,0.5/-2/D,3.5/-2/C,5/0/B,2/3/S}
			\tkzDefMidPoint(B,C)\tkzGetPoint{J}
			\tkzDefMidPoint(D,A)\tkzGetPoint{I}
			\tkzCentroid(S,B,A)\tkzGetPoint{G}
			\tkzDefLine[parallel=through G](A,B)\tkzGetPoint{g}
			\tkzInterLL(g,G)(S,B)\tkzGetPoint{Q}
			\tkzInterLL(g,G)(S,A)\tkzGetPoint{P}
			\tkzDrawPolygon(S,A,D,C,B)
			\tkzDrawSegments(S,D S,C)
			\tkzDrawSegments[dashed](A,B I,J P,Q I,G J,G)
			\tkzDrawPoints[fill=black](I,A,B,C,D,J,G,S,P,Q)
			\tkzLabelPoints[above](S)
			\tkzLabelPoints[above left](G)
			\tkzLabelPoints[below](C,D)
			\tkzLabelPoints[left](A,I,P)
			\tkzLabelPoints[right](B,J,Q)
			\end{tikzpicture}
		}
	}
\end{ex}

\begin{ex}%[1H2B2-2]
	Cho hình chóp $S.ABCD$ có đáy $ABCD$ là hình bình hành. Gọi $I,J,E,F$ lần lượt là trung điểm $SA,SB,SC,SD.$ Trong các đường thẳng sau, đường thẳng nào không song song với $IJ?$
	\choice
	{$DC$}
	{$AB$}
	{\True $AD$}
	{$EF$}
	\loigiai{
		\immini{
			Ta có $IJ\parallel AB$ (tính chất đường trung bình trong tam giác $SAB$) và $EF\parallel CD$ (tính chất đường trung bình trong tam giác $SCD$).\\
			Mà $CD\parallel AB$ (đáy là hình bình hành)\\ $\Rightarrow CD\parallel AB\parallel EF\parallel IJ.$
		}
		{
			\begin{tikzpicture}[scale=0.7, line join=round, line cap=round]
			\tkzDefPoints{0/0/A,-1.5/-1.8/B,4/0/D, 0.2/3/S}
			\coordinate (C) at ($(B)+(D)-(A)$);
			\tkzDefMidPoint(A,S)\tkzGetPoint{I}
			\tkzDefMidPoint(B,S)\tkzGetPoint{J}
			\tkzDefMidPoint(C,S)\tkzGetPoint{E}
			\tkzDefMidPoint(D,S)\tkzGetPoint{F}
			\tkzDrawPolygon(S,B,C,D)
			\tkzDrawSegments(S,C E,F)
			\tkzDrawSegments[dashed](S,A A,B A,C B,D A,D I,J)
			\tkzDrawPoints[fill=black](S,A,B,C,D,I,J,E,F)
			\tkzLabelPoints[above](S)
			\tkzLabelPoints[above right](F)
			\tkzLabelPoints[below](B,C)
			\tkzLabelPoints[left](A,J)
			\tkzLabelPoints[right](D,I)
			\end{tikzpicture}
		}
	}
\end{ex}

\begin{ex}%[1H2B2-2]
	Cho hình chóp $S.ABCD$ có đáy $ABCD$ là hình bình hành. Gọi $d$ là giao tuyến của hai mặt phẳng $\left(SAD\right)$ và $\left(SBC\right)$. Khẳng định nào sau đây đúng?
	\choice
	{$d$ qua $S$ và song song với $DC$}
	{$d$ qua $S$ và song song với $BD$}
	{\True $d$ qua $S$ và song song với $BC$}
	{$d$ qua $S$ và song song với $AB$}
	\loigiai{
		\immini{
			Ta có $\heva{
				& \left(SAD\right)\cap \left(SBC\right)=S \\
				& AD\subset \left(SAD\right),BC\subset \left(SBC\right) \\
				& AD\parallel BC \\}$\\
			$\Rightarrow \left(SAD\right)\cap \left(SBC\right)=Sx\parallel AD\parallel BC$ (với $d\equiv Sx$).
		}
		{
			\begin{tikzpicture}[scale=0.6, line join=round, line cap=round]
			\tkzDefPoints{0/0/A,-1.5/-1.8/B,4/0/D, 0.2/3/S}
			\coordinate (C) at ($(B)+(D)-(A)$);
			\tkzDefLine[parallel=through S](B,C)\tkzGetPoint{K}
			\tkzDrawLines[add = .2 and .2,color=black](S,K)
			\tkzDrawPolygon(S,B,C,D)
			\tkzDrawSegments(S,C)
			\tkzDrawSegments[dashed](S,A A,B A,D)
			\tkzDrawPoints[fill=black](S,A,B,C,D)
			\tkzLabelPoints[above](S)
			\tkzLabelPoints[below](B,C)
			\tkzLabelPoints[left](A)
			\tkzLabelPoints[right](D)
			\end{tikzpicture}
		}
	}
\end{ex}

\begin{ex}%[1H2B2-1]
	Cho tứ diện $ABCD.$ Gọi $M,N$ là hai điểm phân biệt cùng thuộc đường thẳng $AB$. $P,Q$ là hai điểm phân biệt cùng thuộc đường thẳng $CD.$ Xét vị trí tương đối của hai đường thẳng $MP,NQ.$
	\choice
	{$MP\parallel NQ$}
	{$MP$ cắt $NQ$}
	{$MP$ trùng $NQ$}
	{\True $MP,NQ$ chéo nhau}
	\loigiai{
		\immini{
			Xét mặt phẳng $\left(ABP\right).$ \\
			Ta có: $M,N$ thuộc $AB\Rightarrow M,N$ thuộc mặt phẳng $\left(ABP\right).$ \\
			Mặt khác: $CD\cap \left(ABP\right)=P.$ \\
			Mà: $Q\in CD\Rightarrow Q\notin \left(ABP\right)$\\
			$\Rightarrow M,N,P,Q$ không đồng phẳng.
		}
		{
			\begin{tikzpicture}[scale=0.7, line join=round, line cap=round]
			\tkzDefPoints{0/0/B,1.5/-1.8/C,4/0/D,2/3/A}
			\coordinate (M) at ($(A)!0.3!(B)$);
			\coordinate (N) at ($(A)!0.6!(B)$);
			\coordinate (P) at ($(C)!0.3!(D)$);
			\coordinate (Q) at ($(C)!0.6!(D)$);
			\tkzDrawPolygon(A,B,C,D)
			\tkzDrawSegments(A,P A,C)
			\tkzDrawSegments[dashed](B,D B,P N,Q M,P)
			\tkzDrawPoints[fill=black](M,A,B,C,D,N,P,Q)
			\tkzLabelPoints[above](A)
			\tkzLabelPoints[below](C)
			\tkzLabelPoints[left](B)
			\tkzLabelPoints[right](D)
			\tkzLabelPoints[above left](M,N)
			\tkzLabelPoints[below right](P,Q)
			\end{tikzpicture}
		}
	}
\end{ex}


\begin{ex}%[1H2K2-2]
	Cho hình chóp $S.ABCD$ có đáy $ABCD$ là hình thang với đáy lớn $AB$ đáy nhỏ $CD.$ Gọi $M,N$ lần lượt là trung điểm của $SA$ và $SB.$ Gọi $P$ là giao điểm của $SC$ và $\left(AND\right).$ Gọi $I$ là giao điểm của $AN$ và $DP.$ Hỏi tứ giác $SABI$ là hình gì?
	\choice
	{\True Hình bình hành}
	{Hình thoi}
	{Hình vuông}
	{Hình chữ nhật}
	\loigiai{
		\immini{
			Gọi $E=AD\cap BC, P=NE\cap SC$. Suy ra $P=SC\cap \left(AND\right)$.\\
			Ta có $S$ là điểm chung thứ nhất của hai mặt phẳng $\left(SAB\right)$ và $\left(SCD\right)$; $I=DP\cap AN\Rightarrow I$ là điểm chung thứ hai của hai mặt phẳng $\left(SAB\right)$ và $\left(SCD\right).$\\
			Suy ra $SI=\left(SAB\right)\cap \left(SCD\right)$.\\
			Mà $AB\parallel CD\Rightarrow SI\parallel AB\parallel CD.$
			Vì $MN$ là đường trung bình của tam giác $SAB$ và chứng minh được cũng là đường trung bình của tam giác $SAI$ nên suy ra $SI=AB$.\\
			Vậy $SABI$ là hình bình hành.
		}{
			\begin{tikzpicture}
			\tkzDefPoints{0/0/A,4/0/B,3/-1.5/C,1.5/-1.5/D,1.5/3/S}
			\coordinate (N) at ($(S)!0.5!(A)$);
			\coordinate (M) at ($(S)!0.5!(B)$);
			\coordinate (I) at ($2*(M)-(A)$);
			\tkzInterLL(A,D)(B,C) \tkzGetPoint{E}
			\tkzInterLL(S,C)(D,I) \tkzGetPoint{P}
			\tkzInterLL(S,B)(D,I) \tkzGetPoint{m}
			\draw (S)--(A)--(E)--(B) (S)--(C) (S)--(D)--(I)--(S) (B)--(m) (E)--(P);
			\draw[dashed] (I)--(A)--(B) (C)--(D) (N)--(M)--(P) (S)--(m);
			\tkzDrawPoints(A,B,C,D,E,M,N,P,S,I)
			\tkzLabelPoints[above](S,I,M)
			\tkzLabelPoints[right](B,C,P)
			\tkzLabelPoints[left](A,N)
			\tkzLabelPoints[below left=-0.1](D,E)
			\end{tikzpicture}
		}
	}
\end{ex}

\begin{ex}%[1H2B2-4]%
	Cho hình chóp $ S.ABCD $  có đáy $ABCD$ là hình bình hành. Gọi $ M, N, P, Q $  lần lượt là trung điểm của các cạnh $ SA, SB, SC, SD $. Gọi $I$ là một điểm trên cạnh $B $. Thiết diện của hình chóp với mặt phẳng $(IMN)$ là hình gì?
	\choice
	{Tam giác $MNQ$}
	{Tam giác $MNI$}
	{\True Hình thang $MNIJ$}
	{Hình bình hành $MNIJ$}
	\loigiai{
		\immini{Ta có $ MN\parallel AB $; $ MN=\dfrac{1}{2}AB$ và $PQ \parallel CD; PQ=\dfrac{1}{2}CD$. Từ đó, suy ra $MN=PQ$ và $ MN\parallel PQ$.\\
			Vậy $ MNPQ$  là hình bình hành.\\
			Ta có $\heva{&I\in (IMN)\cap (ABCD)\\&AB\subset (ABCD)\\&MN\subset (IMN)\\&AB\parallel MN}$.
			$\Rightarrow (IMN)\cap (ABCD)=IJ\parallel AB \parallel MN \text{ với } J\in AD$.
			Thiết diện của hình chóp cắt bởi mặt phẳng $ (IMN) $ là hình thang $ MNIJ $. }
		{\begin{tikzpicture}[line join=round,line cap=round,line width=.6pt,font=\footnotesize,scale=0.8]
				\coordinate[label=below left:$B$] (B) at (-0.5,-1);
				\coordinate[label=left:$A$] (A) at (1,.8);
				\coordinate[label=below right:$C$] (C) at (3.5,-1);
				\coordinate[label=above right:$D$] (D) at ($(C)-(B)+(A)$);
				\coordinate[label=above left:$S$] (S) at (1.7,4);
				\draw (S)--(B)--(C)--(D)--(S)--cycle (S)--(C);
				\draw[dashed] (A)--(D) (S)--(A)--(B);
				%\draw ($ (A)!5pt!(D)$)--($(A)!2!($($(A)!5pt!(D)$)!.5!($(A)!5pt!(S)$)$)$)--($(A)!5pt!(S)$);
				\fill (A)circle(1.5pt) (B)circle(1.5pt) (C)circle(1.5pt) (D)circle(1.5pt) (S)circle(1.5pt);
				\coordinate[label=above right:$M$] (M) at ($(S)!0.5!(A)$);
				\coordinate[label=above left:$N$] (N) at ($(S)!0.5!(B)$);
				\coordinate[label=below left:$P$] (P) at ($(S)!0.5!(C)$);
				\coordinate[label=above:$Q$] (Q) at ($(S)!0.5!(D)$);
				\draw(N)--(P)--(Q); \draw[dashed](Q)--(M)--(N);
				\fill(M) circle (1.5pt) (N)circle (1.5pt) (P) circle (1.5pt) (Q) circle (1.5pt);
				\coordinate[label=below:$I$] (I) at ($(B)!0.75!(C)$);
				\coordinate[label=below:$J$] (J) at ($(A)!0.75!(D)$);
				\fill(I) circle (1.5pt) (J) circle (1.5pt) ;
				\draw(N)--(I); \draw[dashed](M)--(J)--(I);
	\end{tikzpicture}}}
\end{ex}

\begin{ex}%[1H2B2-4]%
	Cho hình chóp $S.ABCD$ có đáy $ABCD$ là hình bình hành. Gọi $I$ là trung điểm $SA$. Thiết diện của hình chóp $S.ABCD$ cắt bởi mặt phẳng $(IBC)$ là
	\choice
	{Tam giác $IBC$}
	{Tứ giác $IBCD$}
	{Hình thang $IGBC$ ($G$ là trung điểm $SB$)}
	{\True Hình thang $IBCJ$ ($J$ là trung điểm $SD$)}
	\loigiai{
		\immini{Ta có $\heva{
				& (IBC)\cap (SAD)=I \\
				& BC\subset (IBC), AD\subset (SAD) \\
				& BC\parallel AD}$\\
			Suy ra $(IBC)\cap (SAD)=Ix\parallel BC\parallel AD$.\\
			Trong mặt phẳng $(SAD)$ ta có $Ix\parallel AD$.\\
			Gọi $Ix\cap SD=J\Rightarrow $$IJ\parallel BC$.
			Vậy thiết diện của hình chóp $S.ABCD$ cắt bởi mặt phẳng $(IBC)$ là hình thang $IBCJ$.}
		{\begin{tikzpicture}[line cap=round,line join=round, >=stealth,scale=0.8]
				\tkzDefPoints{0/0/A,-2/-2/B,2/-2/C,4/0/D,-0.2/3/S}
				\tkzDefMidPoint(S,A)    \tkzGetPoint{I}
				\tkzDefMidPoint(S,D)    \tkzGetPoint{J}
				\tkzFillPolygon[green!50,opacity=0.25](B,C,J,I)
				\draw [dashed] (S)--(A)--(D)--(B)--(A)--(C) (B)--(I)--(J);
				\draw (S)--(B)--(C)--(D)--(S)--(C)--(J);
				\tkzDrawPoints[fill=black](A,B,C,D,S,I,J)
				\tkzLabelPoints[right](C,D,J)
				\tkzLabelPoints[left](A,B,I)
				\tkzLabelPoints[above](S)
		\end{tikzpicture}}
	}
\end{ex}

% \centerline{---HẾT---}
\Closesolutionfile{ans}
%%Bài 12. ĐT song song MP
% \setcounter{section}{11}
\setcounter{dang}{0}
\section{ĐƯỜNG THẲNG VÀ MẶT PHẲNG SONG SONG}
\subsection{KIẾN THỨC CẦN NHỚ}
\subsubsection{VỊ TRÍ TƯƠNG ĐỐI CỦA ĐƯỜNG THẲNG VÀ MẶT PHẲNG}
Cho đường thẳng $a$ và mặt phẳng $(P) $. Căn cứ vào số điểm chung của đường thẳng và mặt phẳng, ta có ba trường hợp sau:\\
\begin{tabular}{ccc}
	\begin{tikzpicture}[scale=.6]
		\tkzDefPoints{0/0/A', 4/0/B', 1/2/D', 0/2.5/M}
		\coordinate (C') at ($(B')+(D')-(A')$);
		\tkzDefPointBy[translation = from A' to B'](M) \tkzGetPoint{N}
		\tkzLabelSegment[pos=.3](M,N){ $a$}
		%\tkzDrawSegments[dashed](A,A' A',B' A',C')
		\tkzDrawPolygon(A',B',C',D')
		\tkzDrawSegments(M,N)
		%\tkzLabelPoints[right](B)
		%\tkzLabelPoints[above](z,y)
		\tkzMarkAngles[size=1](B',A',D')
		\tkzLabelAngle[pos=0.6](D',A',B'){\footnotesize $P$ }
		%\tkzDrawPoints(M,N,A)
	\end{tikzpicture}
	&\begin{tikzpicture}[scale=.6]
		\tkzDefPoints{0/0/A', 4/0/B', 1/2/D', 2/1/A, 0/2.5/B}
		\coordinate (C') at ($(B')+(D')-(A')$);
		\tkzLabelSegment[pos=.3](A,B){ $a$}
		\tkzInterLL(A,B)(A',B')\tkzGetPoint{I}
		\tkzInterLL(A,B)(C',B')\tkzGetPoint{J}
		\tkzDrawPolygon(A',B',C',D')
		\tkzDrawSegments(A,B I,J)
		\tkzDrawSegments[dashed](A,I)
		%\tkzLabelPoints[right](B)
		\tkzLabelPoints[above right](A)
		\tkzMarkAngles[size=1](B',A',D')
		\tkzLabelAngle[pos=0.6](D',A',B'){\footnotesize $P$ }
		\tkzDrawPoints(A)
	\end{tikzpicture}
	&\begin{tikzpicture}[scale=.6]
		\tkzDefPoints{0/0/A', 4/0/B', 1/2/D', 1/1/A, 4/1/B}
		\coordinate (C') at ($(B')+(D')-(A')$);
		\tkzLabelSegment[pos=.6](A,B){$a$}
		%\tkzDrawSegments[dashed](A,A' A',B' A',C')
		\tkzDrawPolygon(A',B',C',D')
		\tkzDrawSegments(A,B)
		%\tkzLabelPoints[right](B)
		\tkzLabelPoints[above](A,B)
		\tkzMarkAngles[size=1](B',A',D')
		\tkzLabelAngle[pos=0.6](D',A',B'){\footnotesize $P$ }
	\end{tikzpicture}\\
	$a \parallel (P)$ & $a \cap (P)=A $ & $a \subset (P)$
\end{tabular}

\subsubsection{CÁC ĐỊNH LÝ VÀ HỆ QUẢ CẦN NHỚ}
\begin{enumerate}[\iconMT]
	\immini{\item \indam{Định lý 1:} Nếu đường thẳng $a$ không nằm trong mặt phẳng $(P)$ và song song với một đường thẳng nào đó trong $(P)$ thì $a$ song song với $(P)$, hay
	$$a\not\subset (P) \text{ và } \heva{&a\parallel d \\&d\subset (P)\\}\Rightarrow a\parallel (P)$$}{\hspace{1cm}
	\begin{tikzpicture}[scale=0.57]
		\tkzDefPoints{0/0/A', 4/0/B', 1/2/D', 1/1/A, 4/1/B, 0/2.5/M}
		\coordinate (C') at ($(B')+(D')-(A')$);
		\tkzDefPointBy[translation = from A' to B'](M) \tkzGetPoint{N}
		\tkzLabelSegment[pos=.3](M,N){ $a$}
		\tkzLabelSegment[pos=.3](A,B){ $d$}
		%\tkzDrawSegments[dashed](A,A' A',B' A',C')
		\tkzDrawPolygon(A',B',C',D')
		\tkzDrawSegments(A,B M,N)
		%\tkzLabelPoints[right](B)
		%        \tkzLabelPoints[above](A,B)
		\tkzMarkAngles[size=0.8](B',A',D')
		\tkzLabelAngle[pos=0.5](D',A',B'){\footnotesize $P$ }
	\end{tikzpicture}
}
	\immini{
		\item \indam{Định lý 2:}Cho hai đường thẳng chéo nhau. Có duy nhất một mặt phẳng chứa đường thẳng này và song song với đường thẳng kia.
	}{	\begin{tikzpicture}[>=stealth,scale=0.45, line join=round, line cap=round]
	\tikzset{label style/.style={font=\footnotesize}}
	\tkzDefPoints{0/0/A, 6/0/B, 5/-3/C, -1/-3/D, 1/-1/E, 5/-1/b', 4/-2.5/a, 2/-1/M, 1/-0.5/Q, 1/1/L, 5/1/b}	
	\tkzDrawSegments(A,B B,C C,D D,A E,b' a,Q L,b)		
	\tkzLabelPoints[above](M,b',b,a)	
	\tkzMarkAngles[size=0.8cm](C,D,A)
	\tkzLabelAngles[pos=0.5,rotate=30](C,D,A){\scriptsize$\alpha$}
\end{tikzpicture}}
	\immini{\item \indam{Định lý 3:} Nếu đường thẳng $a$ song song với mặt phẳng $(\alpha)$. Nếu mặt phẳng $(\beta)$ chứa $a$ và cắt $(\alpha)$ theo giao tuyến $b$ thì $b$ song song với $a$.}{
	\begin{tikzpicture}[>=stealth,scale=0.46, line join=round, line cap=round]
		\tikzset{label style/.style={font=\footnotesize}}
		\tkzDefPoints{0/0/A,6/0/B, 4/-2/C,-2/-2/D,1/0/I, 4/0/J, 5/-1/L, 2/-1/M, 2/2/H, -1/2/K, 0.5/1/E, 2.5/1/F, 2/1/a, 3/-1/b}	
		\tkzDrawSegments(A,I J,B B,C C,D D,A M,K L,H H,K E,F)
		\tkzDrawSegments[thick,blue](M,L)
		\tkzDrawSegments[dashed](I,J)
		\tkzLabelPoints[above](a,b)	
		\tkzMarkAngles[size=0.8cm](C,D,A)
		\tkzLabelAngles[pos=0.5,rotate=30](C,D,A){\scriptsize $\alpha$}
		\tkzMarkAngles[size=1cm](M,K,H)
		\tkzLabelAngles[pos=0.6,rotate=340](M,K,H){\scriptsize $\beta$}
\end{tikzpicture}}
\immini{
	\begin{note}
		Nếu hai mặt phẳng phân biệt cùng song song với một đường thẳng thì giao tuyến của chúng (nếu có) cũng song song với đường thẳng đó.
	\end{note}
}{	\begin{tikzpicture}[>=stealth,scale=0.46, line join=round, line cap=round]
		\tikzset{label style/.style={font=\footnotesize}}
		\tkzDefPoints{0/0/A, 4/0/B, 4/5/C, 0/5/D, 2/4/E, 2/0/I, 2/-1/F, 5/4/M, 5/-1/N, 0/2/d', 5/2/d}	
		\tkzDrawSegments(A,D  A,F  F,E  E,D  I,B  B,C  C,D  M,N)
		\tkzDrawSegments[dashed](A,I)
		\tkzLabelPoints[right](d)	
		\tkzLabelPoints[left](d')
		\tkzMarkAngles[size=0.8cm](E,F,A)
		\tkzLabelAngles[pos=0.5,rotate=30](E,F,A){\scriptsize $\alpha$}
		\tkzMarkAngles[size=0.8cm](C,B,A)
		\tkzLabelAngles[pos=0.5,rotate=30](C,B,A){\scriptsize $\beta$}
\end{tikzpicture}}	
\end{enumerate}

		
	

% \subsection{PHÂN LOẠI, PHƯƠNG PHÁP GIẢI TOÁN}
\begin{dang}{Chứng minh đường thẳng song song với mặt phẳng}
	% \begin{enumerate}[\iconMT]
		\indam{Phương pháp giải:} Để chứng minh đường thẳng $a$ song song với mặt phẳng $(P)$, ta cần chứng tỏ các ý sau đây
		\immini{
			\begin{itemize}
				\item [$\bullet$] $a$ không nằm trên $(P)$;
				\item [$\bullet$] $a$ song song với một đường thẳng $b$ nằm trong $(P)$. Suy ra $a\parallel (P)$.
				\item [] Tóm lại \quad \quad \fbox{$\heva{&a\not\subset (P)\\&a\parallel b \\&b\subset (P)\\}\Rightarrow a\parallel (P)$}
			\end{itemize}
		}{\vspace{1cm}
			\begin{tikzpicture}[scale=0.7]
				\tikzset{label style/.style={font=\footnotesize}}
				\tkzDefPoints{0/0/A', 4/0/B', 1/2/D', 1/1/A, 4/1/B, 0/2.5/M}
				\coordinate (C') at ($(B')+(D')-(A')$);
				\tkzDefPointBy[translation = from A' to B'](M) \tkzGetPoint{N}
				\tkzLabelSegment[pos=.3](M,N){ $a$}
				\tkzLabelSegment[pos=.3](A,B){ $d$}
				%\tkzDrawSegments[dashed](A,A' A',B' A',C')
				\tkzDrawPolygon(A',B',C',D')
				\tkzDrawSegments(A,B M,N)
				%\tkzLabelPoints[right](B)
				%        \tkzLabelPoints[above](A,B)
				\tkzMarkAngles[size=0.8](B',A',D')
				\tkzLabelAngle[pos=0.5](D',A',B'){\footnotesize $P$ }
		\end{tikzpicture}}
		% \item \indam{Chú ý:} Việc chứng minh $a \parallel b$, ta thường đi đến việc xét các yếu tố song song đã biết trong hình học phẳng như
		% 	\immini{\begin{listEX}[1]
		% 			\item [\ding{172}] Cặp cạnh đối của hình bình hành;
		% 			\item [\ding{173}] Đường trung bình trong tam giác;
		% 			\item [\ding{174}] Tỉ lệ $\dfrac{AM}{AB}=\dfrac{AN}{AC}\Rightarrow MN \parallel BC$ (hình bên). Đặc biệt cần chú ý tỉ lệ trọng tâm của tam giác.
		% 		\end{listEX}
		% 	}{
		% 		\begin{tikzpicture}[scale=0.8, line join=round, line cap=round]
		% 			\tkzDefPoints{0/0/B,5/0/C,1.5/2/A}
		% 			\coordinate (M) at ($(A)!0.3!(B)$);
		% 			\coordinate (N) at ($(A)!0.3!(C)$);
		% 			\tkzDrawPoints[size=5,fill=black](A,B,C,M,N)
		% 			\tkzDrawSegments(M,N)
		% 			\tkzDrawPolygon(A,B,C)
		% 			\tkzLabelPoints[below,font=\footnotesize](B,C)
		% 			\tkzLabelPoints[above,font=\footnotesize](A)
		% 			\tkzLabelPoints[left,font=\footnotesize](M)
		% 			\tkzLabelPoints[right,font=\footnotesize](N)
		% 	\end{tikzpicture}}
		% \end{enumerate}
\end{dang}

\begin{vd}
Cho tứ diện $ABCD$. Gọi $M$ và $N$ lần lượt là trọng tâm của các tam giác $ACD$ và $BCD$. Chứng minh rằng $MN$ song song với các mặt phẳng $(ABC)$ và $(ABD)$.
	\loigiai{
		\immini
		{Gọi $P$, $Q$ lần lượt là trung điểm của $BC$ và $CD$.\\
			Khi đó, ta có $\dfrac{QM}{MA}=\dfrac{QN}{NB}=\dfrac{1}{3}\Rightarrow MN \parallel  AB$.\\
			Vì $\heva{&MN \not\subset (ABC)\\&AB\subset (ABC)\\&MN\parallel AB}$ nên $MN \parallel (ABC)$.	\\
			Tương tự, ta có $\heva{&MN \not\subset (ABD)\\&AB\subset (ABD)\\&MN\parallel AB}$ nên $MN \parallel (ABD)$.	}
		{\begin{tikzpicture}[scale=1, fill=black]
			\coordinate (A) at (1,4);
			\coordinate (B) at (-1,0);
			\coordinate (D) at (4,-1);
			\coordinate (C) at (1,-2);
			\coordinate (P) at ($(B)!0.5!(C)$);
			\coordinate (Q) at ($(D)!0.5!(C)$);
			\coordinate (M) at ($(A)!0.666!(Q)$);
			\coordinate (N) at ($(B)!0.666!(Q)$);
			\tkzLabelPoints[above](A)
			\tkzLabelPoints[left](B,P)
			\tkzLabelPoints[below=0.1](N)
			\tkzLabelPoints[right](D,M)
			\tkzLabelPoints[below](C,Q)
			\tkzDrawSegments(A,B A,C A,D B,C C,D A,Q)
			\tkzDrawSegments[dashed](B,D B,Q M,N)
			\end{tikzpicture}}
	}
\end{vd}
\begin{vd}
	Cho tứ diện $ABCD$. Gọi $G$ là trọng tâm tam giác $ABD$, điểm $I$ nằm trên cạnh $BC$ sao cho $BI=2IC$. Chứng minh rằng $IG$ song song $\left(ACD\right)$.
\end{vd}
\begin{vd}
	Cho hình chóp $S{.}ABCD$ có đáy $ABCD$ là hình bình hành. Lấy $M$ nằm trên cạnh $AD$ sao cho $AD=3AM$. Gọi $G, N$ lần lượt là trọng tâm của tam giác $SAB$ và $ABC$.
	\begin{itemize}
		\item [a)] Tìm giao tuyến của $\left(SAB\right)$ và $\left(SCD\right)$.
		\item [b)] Chứng minh $MN$ song song $\left(SCD\right)$ và $NG$ song song $\left(SAC\right)$.
	\end{itemize}
\end{vd}
\begin{vd}
	Cho hình chóp $S{.}ABCD$ có đáy $ABCD$ là hình bình hành. Gọi $M$, $N$ lần lượt là trung điểm của các cạnh $AB$ và $CD$.
	\begin{listEX}[1]
		\item Chứng minh $MN$ song song với các mặt phẳng $(SBC)$ và $(SAD)$.
		\item Gọi $E$ là trung điểm của $SA$. Chứng minh $SB$ và $SC$ đều song song với mặt phẳng $(MNE)$.
	\end{listEX}
	\loigiai{
		\immini
		{	\begin{listEX}[1]
				\item Từ giả thiết, ta suy ra $MN\parallel BC$ và $MN\parallel AD$.\\
				Vì $\heva{&MN \not\subset (SBC)\\&BC\subset (SBC)\\&MN\parallel BC}$ nên $MN \parallel (SBC)$.	\\
				Tương tự, ta có  $\heva{&MN \not\subset (SAD)\\&AD\subset (SAD)\\&MN\parallel AD}$ nên $MN \parallel (SAD)$.	
				\item Từ giả thiết, ta có $\dfrac{AE}{AS}=\dfrac{AM}{AB}=\dfrac{1}{2}\Rightarrow ME \parallel SB$.\\
				Vì $\heva{&SB \not\subset (MNE)\\&ME\subset (MNE)\\&ME\parallel SB}$ nên $SB \parallel (MNE)$.	\\
				Tương tự, gọi $O$ là tâm của hình bình hành.\\
				Khi đó $\dfrac{AO}{AC}=\dfrac{AE}{AS}=\dfrac{1}{2}\Rightarrow EO \parallel SC$.\\
				Vì $\heva{&SC \not\subset (MNE)\\&EO\subset (MNE)\\&EO\parallel SC}$ nên $SC \parallel (MNE)$.	
		\end{listEX}	}
		{   \begin{tikzpicture}[scale=1,fill=black]
			\coordinate (S) at (1,5);
			\coordinate (A) at (0,0);
			\coordinate (B) at (-2,-2);
			\coordinate (C) at (3,-2);
			\coordinate (D) at (5,0);
			\coordinate (M) at ($(A)!0.5!(B)$);
			\coordinate (O) at ($(A)!0.5!(C)$);
			\coordinate (N) at ($(C)!0.5!(D)$);
			\coordinate (E) at ($(A)!0.5!(S)$);
			\tkzLabelPoints[above](S)
			\tkzLabelPoints[below](O)
			\tkzLabelPoints[right](C,D,N,E)
			\tkzLabelPoints[left](A,B,M)
			\tkzDrawSegments(S,B S,C S,D B,C C,D)
			\tkzDrawSegments[dashed](S,A  A,B A,D M,N M,E E,O)
			\end{tikzpicture}}
	}
\end{vd}

\begin{vd}
Cho hình chóp $S.ABCD$ có đáy $ABCD$ là hình chữ nhật. Gọi $G$ là trọng tâm tam giác $SAD$ và $E$ là điểm trên cạnh $DC$ sao cho $DC=3DE$, $I$ là trung điểm $AD$.
	\begin{listEX}[1]
		\item Chứng minh $OI$ song song với các mặt phẳng $(SAB)$ và $(SCD)$.
		\item Tìm giao điểm $P$ của $IE$ và $(SBC)$. Chứng minh $GE\parallel (SBC)$.
	\end{listEX}
	\loigiai{
		\immini
		{
			\begin{listEX}[1]
				\item Ta có $\heva{&OI\parallel AB \\& AB\subset(SAB)\\& OI\not\subset (SAB)}\Rightarrow OI \parallel (SAB)$.\\
				Tương tự, $\heva{&OI\parallel CD \\& CD\subset(SCD)\\& OI\not\subset (SCD)}\Rightarrow OI \parallel (SCD)$.
				\item Vì $\dfrac{DI}{DA}=\dfrac{1}{2} \ne \dfrac{1}{3}=\dfrac{DE}{DC}$ nên $IE$ không song song với $ AC$. Trong hình chữ nhật $ABCD$, gọi $P = IE \cap BC$ $\Rightarrow P = IE \cap (SBC)$.\\
				Gọi $K$ là trung điểm của $BC$,  $G'$ là trọng tâm tam giác $SBC$. \\
				Khi đó $\dfrac{SG'}{SK}=\dfrac{SG}{SI}=\dfrac{G'G}{KI}=\dfrac{2}{3}$,suy ra $G'G\parallel KI\parallel CE$ và $\Rightarrow G'G = \dfrac{2}{3}KI=\dfrac{2}{3}CD = CE$. Do dó tứ giác $G'GEC$ là hình bình hành, suy ra $CG'\parallel CE$ $\Rightarrow CG\parallel (SBC)$.
			\end{listEX}
		}
		{\begin{tikzpicture}[scale=1,fill=black]
			\coordinate (A) at (-1,0);
			\coordinate (B) at (-2,-2);
			\coordinate (C) at (3,-2);
			\coordinate (D) at (4,0);
			\coordinate (O) at ($(A)!0.5!(C)$);
			\coordinate (S) at (-0.5,5);
			\coordinate (I) at ($(D)!0.5!(A)$);
			\coordinate (K) at ($(B)!0.5!(C)$);
			\coordinate (G) at ($(S)!0.666!(I)$);
			\coordinate (G') at ($(S)!0.666!(K)$);
			\coordinate (E) at ($(D)!0.33!(C)$);
			%\coordinate (N) at ($(S)!0.5!(D)$);
			\tkzLabelPoints[above](S)
			\tkzLabelPoints[left](A,G,G')
			\tkzLabelPoints[right](D,E)
			\tkzLabelPoints[below=0.1](B,C,O,I,K)
			\tkzDrawSegments(S,B S,C S,D B,C C,D S,K)
			\tkzDrawSegments[dashed](S,A A,B A,D A,C B,D S,I K,I G,E C,G' G,G')
			\end{tikzpicture}}
	}
\end{vd}

\begin{dang}{Tìm giao tuyến của hai mặt phẳng cắt nhau}
	Ngoài các phương pháp đã học ở bài trước, ta có thêm 2 cách nữa là áp dụng định lí 3 ở trên.
\end{dang}

\begin{vd}
\immini{Cho tứ diện $ABCD$ có $G$ là trọng tâm $\triangle ABC$, $M\in CD$ với $MC = 2MD$. 
	\begin{listEX}[1]
		\item Chứng minh $MG$ song song với $(ABD)$.
		\item Tìm giao tuyến của $(ABD)$ với $(BGM)$.
		\end{listEX}
	
	}{
\begin{tikzpicture}[scale=0.7]
\coordinate (B) at (-1,0);
\coordinate (D) at (4,0);
\coordinate (C) at (0,-2);
\coordinate (A) at (1,3);
\coordinate (H) at ($(B)!0.5!(C)$);
\coordinate (G) at ($(A)!0.666!(H)$);
\coordinate (M) at ($(D)!0.333!(C)$);
\tkzLabelPoints[above,font=\footnotesize](A)
\tkzLabelPoints[right,font=\footnotesize](D)
\tkzLabelPoints[left,font=\footnotesize](B,G)
\tkzLabelPoints[below,font=\footnotesize](C,M)
\tkzDrawSegments(A,B A,C C,D D,A A,H A,M B,C)
\tkzDrawSegments[dashed](B,D G,M)
\end{tikzpicture}}
	\loigiai{
		\immini
		{\begin{listEX}[1]
				\item Gọi $N$ là trung điểm của $AB$. Trong tam giác $CDN$, ta có
				$\dfrac{CM}{CD}=\dfrac{CG}{CN}=\dfrac{2}{3}\Rightarrow GM \parallel ND$. Vì $ND \subset (ABD)$, $GM \not\subset (ABD)$ nên $GM\parallel (ABD)$.
				\item Vì $\heva{&GM \parallel (ABD)\\&B \in (ABD) \cap (BGM)}\Rightarrow (ABD) \cap (BGM) = Bx \parallel GM\parallel ND$.
				\end{listEX}
		}
		{\begin{tikzpicture}
			\coordinate (B) at (-1,0);
			\coordinate (D) at (4,0);
			\coordinate (C) at (0,-2);
			\coordinate (A) at (1,4);
			\coordinate (N) at ($(A)!0.5!(B)$);
			\coordinate (H) at ($(B)!0.5!(C)$);
			\coordinate (G) at ($(A)!0.666!(H)$);
			\coordinate (M) at ($(D)!0.333!(C)$);
			\tkzDefPointBy[translation= from G to M](B)
			\tkzGetPoint{x}
			\tkzDefPointBy[translation= from G to M](A)
			\tkzGetPoint{y}
			\tkzLabelPoints[above](A,y)
			\tkzLabelPoints[right](D,M,x)
			\tkzLabelPoints[left](B,N,G)
			\tkzLabelPoints[below](C)
			\tkzDrawSegments(A,B A,C C,D D,A A,H C,N A,M B,C A,y)
			\tkzDrawSegments[dashed](B,D G,M B,M N,D B,x)
			\end{tikzpicture}}
	}
\end{vd}

\begin{vd}
\immini{Cho hình chóp $S.ABCD$ có đáy $ABCD$ là hình bình hành. Gọi $I$, $K$ lần lượt là trung điểm của $BC$ và  $CD$.
	\begin{enumerate}
		\item Tìm giao tuyến của $(SIK)$ và $(SAC)$, $(SIK)$ và $(SBD)$.
		\item Gọi $M$ là trung điểm của $SB$. Chứng minh $SD \parallel (ACM)$.
		\item Tìm giao điểm $F$ của $DM$ và $(SIK)$. Tính tỉ số $\dfrac{MF}{MD}$.
	\end{enumerate}
}{
	\begin{tikzpicture}[scale=0.7, line join=round, line cap=round]
	\tkzDefPoints{0/0/A,-1.7/-1.6/B,2.5/-1.6/C}
	\coordinate (D) at ($(A)+(C)-(B)$);
	\coordinate (S) at ($(A)+(0.7,3)$);
	\tkzDrawPolygon(S,B,C,D)
	\tkzDrawSegments(S,C)
	\tkzDrawSegments[dashed](A,S A,B A,D)
	\tkzDrawPoints[fill=black,size=4](D,C,A,B,S)
	\tkzLabelPoints[above](S)
	\tkzLabelPoints[below](A,B,C)
	\tkzLabelPoints[right](D)
	\end{tikzpicture}}
	\loigiai{
		\begin{center}
			\begin{tikzpicture}[scale=1]
			\tkzDefPoints{0/0/A, -2/-2/B, 3/-2/C}
			\coordinate (D) at ($(A)+(C)-(B)$);
			\coordinate (O) at ($(A)!.5!(C)$);
			\coordinate (H) at ($(B)!.2!(O)$);
			\coordinate (S) at ($(H)+(0,5)$);
			\coordinate (I) at ($(B)!.5!(C)$);
			\coordinate (K) at ($(C)!.5!(D)$);
			\coordinate (M) at ($(S)!.5!(B)$);
			\tkzInterLL(A,C)(I,K)\tkzGetPoint{E}
			\tkzDefPointBy[translation= from B to D](S)
			\tkzGetPoint{x}
			\tkzInterLL(D,M)(S,x)\tkzGetPoint{F}
			\tkzDrawSegments[dashed](M,O S,A A,B A,D A,C B,D I,K S,E A,M D,M)
			\tkzDrawPolygon(S,C,D)
			\tkzDrawSegments(S,B B,C S,I C,K S,K C,M F,x M,F)
			\tkzLabelPoints[above right](A,M)
			\tkzLabelPoints[right](C,D)
			\tkzLabelPoints[above](S,K,x,F)
			\tkzLabelPoints[left](B)
			\tkzLabelPoints[below](O,I,E)
			\tkzDrawPoints[fill=black](A,B,C,D,S,E,O,K,I,M,F)
			\end{tikzpicture}
		\end{center}
		\begin{enumerate}
			\item \begin{itemize}
				\item Ta có $S \in (SIK) \cap (SAC).$\\
				Trong mp$(ABCD)$, gọi $E=IK \cap AC \Rightarrow \heva{&E\in IK \subset (SIK)\\& E\in AC \subset (SAC)} \Rightarrow E \in (SIK) \cap (SAC).$\\
				Suy ra $SE=(SIK) \cap (SAC).$\\
				\item Ta có $\heva{&S \in (SIK) \cap (SBD)\\ &BD \in (SBD), IK \in (SIK), BD \parallel IK}\Rightarrow (SIK) \cap (SBD)=Sx$, (với $Sx \parallel BD \parallel IK).$
			\end{itemize}
			\item Trong mp$(ABCD)$, gọi $O=AC \cap BD$, ta có $SD \parallel MO$. Mà $MO \subset (ACM)$, suy ra $SD \parallel (ACM)$.
			\item \begin{itemize}
				\item Trong mp$(SBD)$, gọi $F=Sx \cap DM \Rightarrow \heva{&S \in DM\\& S\in Sx \subset (SIK)} \Rightarrow F= DM \cap (SIK)$.
				\item Ta có $SF \parallel BD \Rightarrow \dfrac{MF}{MD}=\dfrac{MS}{MB}=1$.
			\end{itemize}
		\end{enumerate}
	}
\end{vd}

\begin{vd}
	Cho hình chóp $S.ABCD$ có đáy là hình thang, đáy lớn $AD$. Gọi $I$ là trung điểm của $SB$. Gọi $(P)$ là mặt phẳng qua $I$, song song với $SD$ và $AC.$ Tìm giao tuyến của $(P)$ với các mặt $(SBD)$ và $(ABCD)$.
	\loigiai{
		\immini{
			\begin{enumerate}[a)]
				\item Ta có: $\heva{
					& I\in (P)\cap (SBD)\\
					& (P) \parallel SD\\
					& SD\subset (SBD)
				}$\\
				$\Rightarrow (P)\cap (SBD)=Ix$ trong đó $Ix\parallel SD$.\\
				Gọi $Ix\cap BD=K \Rightarrow (P)\cap (SBD)=IK.$
				\item Ta có: $\heva{
					& K\in (P)\cap (ABCD)\\
					& (P) \parallel AC\\
					& AC\subset (ABCD)
				}$\\
				$\Rightarrow (P)\cap (SBD)=Ky$ trong đó $Ky\parallel AC$.\\
				Gọi $Ky\cap AD=E, Ky\cap CD=F$\\
				$ \Rightarrow (P)\cap (SBD)=EF.$
			\end{enumerate}
			
		}{
			\begin{tikzpicture}[scale=0.8]
				\tkzInit[xmin=-3,xmax=7,ymin=-3,ymax=5]
				\tkzClip
				\tkzSetUpPoint[fill=black,size=4]
				\tkzDefPoints{0/0/A,1/-2/B,5/-2/C,7/0/D,2/4/S}
				\tkzDefMidPoint(S,B) \tkzGetPoint{I}
				\tkzDefMidPoint(B,D) \tkzGetPoint{K}
				\tkzDefPointWith[linear,K=0.79](D,A)
				\tkzGetPoint{E}
				\tkzDefPointWith[linear,K=0.79](D,C)
				\tkzGetPoint{F}
				\tkzDrawPoints(A,B,C,D,S,I,K,E,F)
				\tkzDrawSegments[dashed](A,D B,D I,K A,C E,F)
				\tkzDrawSegments(A,B B,C C,D S,A S,B S,C S,D)
				\tkzLabelPoint[left](I){\footnotesize $I$}
				\tkzLabelPoint[above](K){\footnotesize $K$}
				\tkzLabelPoint[below](E){\footnotesize $E$}
				\tkzLabelPoint[right](F){\footnotesize $F$}
				\tkzLabelPoint[left](A){\footnotesize $A$}
				\tkzLabelPoint[below](B){\footnotesize $B$}
				\tkzLabelPoint[below](C){\footnotesize $C$}
				\tkzLabelPoint[right](D){\footnotesize $D$}
				\tkzLabelPoint[above](S){\footnotesize $S$}
		\end{tikzpicture}}
	}
\end{vd}

\begin{vd}
\immini{Cho tứ diện $ABCD$. Gọi $M$, $I$ lần lượt là trung điểm của $BC$, $AC$. Mặt phẳng $(P)$ đi qua điểm $M$, song song với $BI$ và $SC$. Xác định trên hình vẽ các giao điểm $H, K, N$ của $(P)$ với các cạnh $AC$, $SA$, $SB$. Tứ giác $MNKH$ là hình gì?
	
}{
	\begin{tikzpicture}[scale=0.6,fill=black]
	\coordinate (A) at (-1,0);
	\coordinate (B) at (0,-2);
	\coordinate (C) at (4,0);
	\coordinate (S) at (1,3);
	\coordinate (M) at ($(B)!0.5!(C)$);
	\coordinate (I) at ($(C)!0.5!(A)$);
	\tkzLabelPoints[right,font=\footnotesize](C)
	\tkzLabelPoints[below,font=\footnotesize](B,M)
	\tkzLabelPoints[above=0.1,font=\footnotesize](S,I)
	\tkzLabelPoints[left,font=\footnotesize](A)
	\tkzDrawSegments(S,A S,B S,C A,B B,C)
	\tkzDrawSegments[dashed](A,C B,I)
	\tkzDrawPoints[size=5,fill=black](A,B,C,S,I,M)
	
	\end{tikzpicture}}
	\loigiai{
		\immini
		{
			Vì $\heva{&(P) \parallel SC\\& M\in (P)\cap (SBC)}\Rightarrow (P) \cap (SBC) = MN \parallel SC$, $N\in SB$ $\qquad (1)$\\
			Tương tự, $\heva{&(P)\parallel BI\\& M\in (P) \cap (ABC)}\Rightarrow (P) \cap (ABC) =MH \parallel BI$, $H\in AC$ $\qquad (2)$
			
			Mặt khác, $\heva{&(P)\parallel (SC)\\& N\in (P) cap (SAC)}\Rightarrow (P) \cap (SAC) = HK \parallel SC$, $K\in SA$ $(3)$
			Từ $(1)$, $(2)$ và $(3)$ ta có thiết diện của $(P)$ với tư diện $ABCD$ là tứ giác $MNKH$.
		}
		{\begin{tikzpicture}[scale=1,fill=black]
			\coordinate (A) at (-1,0);
			\coordinate (B) at (0,-2);
			\coordinate (C) at (4,0);
			\coordinate (S) at (1,4);
			\coordinate (M) at ($(B)!0.5!(C)$);
			\coordinate (I) at ($(C)!0.5!(A)$);
			\coordinate (N) at ($(S)!0.5!(B)$);
			\coordinate (H) at ($(I)!0.5!(C)$);
			\coordinate (Q) at ($(S)!0.5!(A)$);
			\coordinate (K) at ($(S)!0.5!(Q)$);
			\tkzLabelPoints[right](C)
			\tkzLabelPoints[below](B, M)
			\tkzLabelPoints[above=0.1](I,S,H)
			\tkzLabelPoints[left](A,K,N)
			\tkzDrawSegments(S,A S,B S,C M,N K,N A,B B,C)
			\tkzDrawSegments[dashed](A,C M,H H,K B,I)
			\end{tikzpicture}}
	}
\end{vd}

\begin{vd}%[1H2K3-4]
\immini{Cho hình chóp $S.ABCD$. Gọi $M$, $N$ thuộc cạnh $AB$, $CD$. Gọi $(\alpha)$ là mặt phẳng qua $MN$ và song song với $SA$. Tìm giao tuyến của $(\alpha)$ với các mặt của hình chóp.
	
}{
	\begin{tikzpicture}[line join = round, line cap = round,>=stealth,
	font=\footnotesize,scale=.7]
	\tkzDefPoints{0/0/A,1.5/-3/B,4.5/-2.5/C}
	\coordinate (D) at ($(A)+(7,0)$);
	\coordinate (S) at ($(A)+(3,3)$);
	\coordinate (M) at ($(A)!3/5!(B)$);
	\coordinate (N) at ($(C)!2/5!(D)$);
	\tkzDrawSegments(S,C S,B S,D C,D B,C A,B S,A)
	\tkzDrawSegments[dashed](A,D A,C M,N)
	\tkzDrawPoints[fill=black](A,B,D,C,S,M,N)
	\tkzLabelPoints[above](S)
	\tkzLabelPoints[right](D,N)
	\tkzLabelPoints[left](M)
	\tkzLabelPoints[below](B,C)
	\tkzLabelPoints[above left](A)
	\end{tikzpicture}}
	\loigiai{
				\immini{
				Ta có $\heva{&M \in (\alpha) \cap (SAB)\\&SA \parallel (\alpha)\\&SA \subset (SAB)} \Rightarrow (\alpha) \cap (SAB)= MP$, với $MP \parallel SA$.\\
				Trong mặt phẳng $(ABCD)$, gọi $R= MN \cap AC$.\\
				Ta có $\heva{&R \in (\alpha) \cap (SAC)\\&SA \parallel (\alpha)\\&SA \subset (SAC)} \Rightarrow (\alpha) \cap (SAC)= RQ$, với $RQ \parallel SA$.\\
				Ta có $(\alpha) \cap (SCD)= QN$. Vậy thiết diện là tứ giác $MNQP$.
						}{
				\begin{tikzpicture}[line join = round, line cap = round,>=stealth,
				font=\footnotesize,scale=.7]
				\tkzDefPoints{0/0/A,1.5/-3/B,4.5/-2.5/C}
				\coordinate (D) at ($(A)+(7,0)$);
				\coordinate (S) at ($(A)+(3,5)$);
				\coordinate (M) at ($(A)!3/5!(B)$);
				\coordinate (N) at ($(C)!2/5!(D)$);
				\tkzInterLL(A,C)(M,N) \tkzGetPoint{R}
				\coordinate (K') at ($(M)+(S)-(A)$);
				\tkzInterLL(M,K')(S,B) \tkzGetPoint{P}
				\coordinate (H') at ($(S)+(N)-(D)$);
				\tkzInterLL(N,H')(S,C) \tkzGetPoint{Q}
				\tkzDrawSegments(S,C S,B S,D C,D B,C A,B S,A M,P P,Q Q,N)
				\tkzDrawSegments[dashed](A,D A,C M,N R,Q)
				\tkzDrawPoints[fill=black](A,B,D,C,S,M,N,R,P,Q)
				\tkzLabelPoints[above](S)
				\tkzLabelPoints[right](D,N,Q)
				\tkzLabelPoints[left](M,P)
				\tkzLabelPoints[below](B,C,R)
				\tkzLabelPoints[above left](A)
				
				\end{tikzpicture}}
		
	}
\end{vd}

\begin{vd}
	\immini{Cho hình chóp $S.ABCD$ có đáy là hình bình hành, $O$ là giao điểm của $AC$ và $BD$, $M$ là trung điểm của $SA$. 
	\begin{enumerate}
		\item Chứng minh $OM \parallel (SCD)$.
		\item Gọi $(\alpha)$ là mặt phẳng đi qua $M$, đồng thời song song với $SC$ và $AD$. Tìm giao tuyến của mặt phẳng $(\alpha)$ với các mặt của hình chóp $S.ABCD$. Hình tạo bởi các giao tuyến là hình gì?
	\end{enumerate}
}{
	\begin{tikzpicture}[line join = round, line cap = round,>=stealth,
	font=\footnotesize,scale=.9]
	\tkzDefPoints{0/0/A}
	\coordinate (D) at ($(A)+(5,0)$);
	\tkzDefShiftPoint[A](-140:2.7){B}
	\coordinate (C) at ($(B)+(D)-(A)$);
	\tkzInterLL(A,C)(B,D) \tkzGetPoint{O}
	\coordinate (S) at ($(A)+(1,3)$);
	\coordinate (M) at ($(A)!.5!(S)$);
	\tkzDrawPolygon(S,B,C,D)
	\tkzDrawSegments(S,C)
	\tkzDrawSegments[dashed](A,S A,B A,D A,C B,D)
	\tkzDrawPoints[fill=black](A,B,D,C,O,S,M)
	\tkzLabelPoints[above](S)
	\tkzLabelPoints[below](O,C)
	\tkzLabelPoints[left](A,B)
	\tkzLabelPoints[right](D)
	\tkzLabelPoints[right](M)
	\end{tikzpicture}}
	\loigiai{
		\begin{enumerate}
			\item 
			\immini{Ta có $M, O$ là trung điểm của $SA$ và $AC$, suy ra $MO \parallel SC$.\\
				Mà $SC \subset (SCD) \Rightarrow OM \parallel (SCD)$.
				\item Vì $MO \parallel SC \Rightarrow O \in (\alpha)$.\\
				Ta có $\heva{&O \in (\alpha) \cap (ABCD)\\&AD \parallel (\alpha)\\&AD \subset (ABCD)} \Rightarrow (\alpha) \cap (ABCD)= PQ$.\\
				Với $PQ \parallel AD, O \in PQ, Q \in AB, P \in CD$.\\
				Lại có $\heva{&P \in (\alpha) \cap (SCD)\\&SC \parallel (\alpha)\\&SC \subset (SCD)} \Rightarrow (\alpha) \cap (SCD)= PN$, với $PN \parallel SC$.\\
				Có $(\alpha) \cap (SAD)= MN, (\alpha) \cap (SAB)= MQ$.\\
				Nhận thấy $P, Q$ là trung điểm của $CD$ và $AB$. Suy ra $N$ là trung điểm của $SD$.\\
				Suy ra $MN \parallel PQ$. Vậy thiết diện là hình thang $MNPQ$.
				
			}{
				\begin{tikzpicture}[line join = round, line cap = round,>=stealth,
				font=\footnotesize,scale=.7]
				\tkzDefPoints{0/0/A}
				\coordinate (D) at ($(A)+(5,0)$);
				\tkzDefShiftPoint[A](-140:2.7){B}
				\coordinate (C) at ($(B)+(D)-(A)$);
				\tkzInterLL(A,C)(B,D) \tkzGetPoint{O}
				\coordinate (S) at ($(O)+(0,6)$);
				\coordinate (M) at ($(A)!.5!(S)$);
				\coordinate (N) at ($(S)!.5!(D)$);
				\coordinate (Q) at ($(A)!.5!(B)$);
				\coordinate (P) at ($(C)!.5!(D)$);
				\tkzDrawPolygon(S,B,C,D)
				\tkzDrawSegments(S,C N,P)
				\tkzDrawSegments[dashed](A,S A,B A,D A,C B,D Q,M M,N M,O P,Q M,Q)
				\tkzDrawPoints[fill=black](A,B,D,C,O,S,M,N,P,Q)
				\tkzLabelPoints[above](S)
				\tkzLabelPoints[below](O,C)
				\tkzLabelPoints[left](A,B,Q)
				\tkzLabelPoints[right](D,N,P)
				\tkzLabelPoints[above right](M)
				\end{tikzpicture}
				
			}
		\end{enumerate}
	}
\end{vd}
\begin{vd}
	Cho tứ diện $ABCD$ và điểm $M$ thuộc cạnh $AB$. Gọi $\left(\alpha\right)$ là mặt phẳng đi qua $M$, song song với đường thẳng $BC$ và $AD$. Gọi $N, P, Q$ lần lượt là giao điểm của $\left(\alpha\right)$ với các cạnh $AC, CD$ và $DB$ .
	\begin{itemize}
		\item [a)] Chứng minh $MNPQ$ là hình bình hành.
		\item [b)] Trong trường hợp nào thì $MNPQ$ là hình thoi.
	\end{itemize}
\end{vd}

\subsection{BÀI TẬP TỰ LUYỆN}

\begin{bt}
	Cho tứ diện $ABCD$ có $G$ là trọng tâm tam giác $ABD$. Trên đoạn $BC$ lấy điểm $M$ sao cho $MB=2MC$. Chứng minh rằng đường thẳng $MG$ song song với mặt phẳng $(ACD)$.
	\loigiai{
		\begin{center}
			\begin{tikzpicture}
				\tkzDefPoints{0/0/B, 7/0/D, 2/-3/C, 4/5/A}
				\tkzDefMidPoint(A,D) \tkzGetPoint{N}		 
				\coordinate (M) at ($(B)!.666666!(C)$);
				\tkzCentroid(A,B,D) \tkzGetPoint{G}
				\tkzDrawPoints(M,N,G)
				\tkzDrawSegments(A,B B,C C,D D,A C,N A,C)
				\tkzDrawSegments[dashed](B,D M,G B,N)
				\tkzLabelPoints[below left](M)
				\tkzLabelPoints[above](A,G)
				\tkzLabelPoints[left](B)
				\tkzLabelPoints[right](N,D)
				\tkzLabelPoints[below](C)
			\end{tikzpicture}
		\end{center}
		Gọi $N$ là trung điểm của $AD$. Ta có: $\dfrac{BG}{BN}=\dfrac{2}{3}$ (Vì $G$ là trọng tâm tam giác $ABD$).\\
		Theo giả thiết, ta có: $MB=2MC \Rightarrow \dfrac{BM}{BC}=\dfrac{2}{3}$.\\
		Tam giác $BCN$ có $\dfrac{BG}{BN}=\dfrac{BM}{BC}=\dfrac{2}{3}$ $\Rightarrow MG \parallel CN$.\\
		Mà $MG \not\subset (ACD)$, $CN \subset (ACD) \Rightarrow MG \parallel (ACD)$.\\
	}
\end{bt}
\begin{bt}
	Cho hình chóp $S.ABCD$ có đáy $ABCD$ là hình bình hành tâm $O$. Gọi $M$, $N$, $P$ lần lượt là trung điểm của các cạnh $SD$, $CD$, $BC$.
	\begin{enumerate}
		\item Chứng minh đường thẳng $OM$ song song với các mặt phẳng $(SAB)$, $(SBC)$.
		\item Chứng minh đường thẳng $SP$ song song với mặt phẳng $(OMN)$.
	\end{enumerate}
	\loigiai{
		\begin{center}
			\begin{tikzpicture}
				\tkzDefPoints{0/0/D, 3/3/A, 6/0/C, 9/3/B, 2/8/S}
				\tkzDefMidPoint(S,D) \tkzGetPoint{M}
				\tkzDefMidPoint(C,D) \tkzGetPoint{N} 
				\tkzDefMidPoint(B,C) \tkzGetPoint{P}
				\tkzInterLL(A,C)(B,D) \tkzGetPoint{O}
				\tkzInterLL(O,N)(D,P) \tkzGetPoint{I}
				\tkzDrawPoints(M,N,P,O,I)
				\tkzDrawSegments(B,C C,D S,B S,C S,D S,P M,N)
				\tkzDrawSegments[dashed](A,B A,D A,C B,D S,A O,M O,N D,P M,I)
				\tkzLabelPoints[below left](D)
				\tkzLabelPoints[above](S,O)
				\tkzLabelPoints[above right](A)
				\tkzLabelPoints[left](M)
				\tkzLabelPoints[right](B,P)
				\tkzLabelPoints[below](N)
				\tkzLabelPoints[below right](C,I)
			\end{tikzpicture}
		\end{center}
		\begin{enumerate}
			\item Tam giác $SBD$ có $OB=OD$ và $MS=MD$ nên $OM$ là đường trung bình của tam giác $SBD$ $\Rightarrow OM \parallel SB$.\\
			Mà $OM$ không chứa trong các mặt phẳng $(SAB)$ và $(SBC)$ nên $OM \parallel (SAB)$ và $OM \parallel (SBC)$.
			\item Trong mặt phẳng $(ABCD)$, gọi $I$ là giao điểm của $ON$ và $DP$.\\
			Tam giác $BCD$ có $OB=OD$ và $NC=ND$ nên $ON$ là đường trung bình của tam giác $BCD$ $\Rightarrow I$ là trung điểm của $DP$.\\
			Tam giác $SDP$ có $MS=MD$ và $IP=ID$ nên $IM$ là đường trung bình của tam giác $SDP$ $\Rightarrow IM \parallel SP$.\\
			Mà $SP \not\subset (OMN)$, $IM \subset (OMN) \Rightarrow SP \parallel (OMN)$.
		\end{enumerate}
	}
\end{bt}


\begin{bt}
	Cho hình chóp $S.ABCD$ có đáy là hình thang đáy lớn $AB$, với $AB = 2CD$. Gọi $O$ là giao điểm của $AC$ và $BD$, $I$ là trung điểm của $SA$, $G$ là trọng tâm của tam giác $SBC$ và $E$ là một điểm trên cạnh $SD$ sao cho $3SE = 2SD$. Chứng minh:
	\begin{listEX}[3]
		\item $DI\parallel (SBC)$.
		\item $GO\parallel (SCD)$.
		\item $SB\parallel (ACE)$.
	\end{listEX}
	\loigiai{
		\begin{center}
			\begin{tikzpicture}[scale=1,fill=black]
				\coordinate (S) at (0,5);
				\coordinate (A) at (-1,0);
				\coordinate (B) at (5,0);
				\coordinate (D) at (0,-2);
				\coordinate (C) at (3,-2);
				\coordinate (I) at ($(S)!0.5!(A)$);
				\coordinate (N) at ($(S)!0.5!(B)$);
				\coordinate (M) at ($(B)!0.5!(C)$);
				\coordinate (P) at ($(S)!0.5!(C)$);
				\coordinate (E) at ($(S)!0.666!(D)$);
				\coordinate (G) at ($(S)!0.666!(M)$);
				\tkzInterLL(A,C)(B,D) \tkzGetPoint{O}
				\tkzLabelPoints[above](S)
				\tkzLabelPoints[below](C,D,O)
				\tkzLabelPoints[left](A,I,P)
				\tkzLabelPoints[right](B,E,G,M,N)
				\tkzDrawSegments(S,A S,D S,D S,B B,C C,D D,A S,C D,I S,M A,E N,C C,E B,P D,P)
				\tkzDrawSegments[dashed](A,C B,D A,B I,N G,O O,E)
			\end{tikzpicture}
		\end{center}
		\begin{listEX}[1]
			\item Gọi $N$ là trung điểm $SB$, khi đó $IN\parallel AB$ và $IN =\dfrac{1}{2}AB$. Suy ra $IN\parallel CD$, $IN =DC$ suy ra tứ giác $INCD$ là hình bình hành, do đó $ID\parallel NC$. Vậy $ID \parallel (SBC)$.
			\item $GO \parallel (SCD)$\\
			Gọi $P$ là trung điểm của $SC$, khi đó $GO\parallel PD$, suy ra $GO\parallel (SCD)$.
			\item Ta có $EO \parallel SB$, suy ra $SB \parallel (ACE)$.
		\end{listEX}
	}
\end{bt}

\begin{bt}
	Cho tứ diện $ABCD$. Gọi $I,\,J$ lần lượt là trung điểm của $AB$ và $CD$, $M$ là một điểm trên đoạn $IJ$. Gọi $(P)$ là mặt phẳng qua $M$ và song song với $AB$ và $CD$.
	\begin{listEX}
		\item Tìm giao tuyến của mặt phẳng $(P)$ và $(ICD)$.
		\item Xác định giao tuyến của mặt phẳng $(P)$ với các mặt của tứ diện. Hình tạo bởi các giao tuyến là hình gì?
	\end{listEX}
	\loigiai{
		\begin{center}
			\begin{tikzpicture}[line join=round, line cap=round,>=stealth,scale=1,font=\footnotesize]
				\tkzInit[ymin=-2.5,ymax=5,xmin=-2.5,xmax=5.5]
				\tkzClip
				\tkzDefPoints{0/0/B,5/0/D,2/-2/C,1/4/A}
				\tkzDefMidPoint(B,A)\tkzGetPoint{I}
				\tkzDefMidPoint(C,D)\tkzGetPoint{J};
				\coordinate (M) at ($(I)!0.6!(J)$);
				\tkzDefLine[parallel=through M](C,D) \tkzGetPoint{m}
				\tkzInterLL(M,m)(I,D) \tkzGetPoint{F}
				\tkzInterLL(M,m)(I,C) \tkzGetPoint{E}
				\tkzDefLine[parallel=through F](A,B) \tkzGetPoint{f}
				\tkzInterLL(F,f)(B,D) \tkzGetPoint{G}
				\tkzInterLL(F,f)(A,D) \tkzGetPoint{P}
				\tkzDefLine[parallel=through E](A,B) \tkzGetPoint{e}
				\tkzInterLL(E,e)(B,C) \tkzGetPoint{H}
				\tkzInterLL(E,e)(A,C) \tkzGetPoint{Q}
				\tkzDrawPoints[fill=black](A,B,M,C,D,I,J,E,F,G,H,P,Q)
				\tkzLabelPoints(C,J)
				\tkzLabelPoints[above](A,M)
				\tkzLabelPoints[above left](I)	
				\tkzLabelPoints[above right](F)	
				\tkzLabelPoints[below left](H)
				\tkzLabelPoints[below right](G)	
				%	\tkzLabelPoints[above right](M)
				\tkzLabelPoints[right](D,P)
				\tkzLabelPoints[left](B,E,Q)
				\tkzDrawSegments(A,B B,C C,A A,D C,D I,C Q,H P,Q)
				\tkzDrawSegments[dashed](B,D I,J I,D E,F G,H P,G)
			\end{tikzpicture}
		\end{center}
		\begin{listEX}
			\item 
			Gọi $\Delta_1=(P)\cap(ICD)$, ta có\\
			$\heva{&M\in (P)\\&M\in IJ,\,IJ \subset (ICD)}\Rightarrow M \in \Delta_1$.\\
			$\heva{&(P)\parallel CD\\& CD\subset (ICD)\\&(P)\cap (ICD)=\Delta_1}\Rightarrow \Delta_1 \parallel CD$.\\
			Vậy $\Delta_1$ là đường thẳng qua $M$ và song song với $CD$.\\
			Gọi $E=\Delta_1 \cap IC, F=\Delta_1 \cap TD$, ta được $(P)\cap (ICD)=EF$.
			\item 	
			Gọi $\Delta_2=(P)\cap(ABD)$, ta có\\
			$\heva{&F\in (P)\\&F\in ID,\,ID \subset (ABD)}\Rightarrow F \in \Delta_2$.\\
			$\heva{&(P)\parallel AB\\& AB\subset (ABD)\\&(P)\cap (ABD)=\Delta_2}\Rightarrow \Delta_2 \parallel AB$.\\
			Vậy $\Delta_2$ là đường thẳng qua $F$ và song song với $AB$.\\
			Gọi $G=\Delta_2 \cap BD, P=\Delta_2 \cap AD$, ta được $(P)\cap (ICD)=GP$.\\
			Gọi $\Delta_3=(P)\cap(ABC)$, ta có
			$$\heva{&E\in (P)\\&E\in IC,\,IC \subset (ABC)}\Rightarrow E \in \Delta_3.$$
			Ta có $$\heva{&(P)\parallel AB\\& AB\subset (ABC)\\&(P)\cap (ABC)=\Delta_3}\Rightarrow \Delta_3 \parallel AB.$$
			Vậy $\Delta_3$ là đường thẳng qua $E$ và song song với $AB$.\\
			Gọi $H=\Delta_3 \cap BC, Q=\Delta_3 \cap AC$, ta được $(P)\cap (ABC)=HQ$.\\
			Giao tuyến của $(P)$ với các mặt phẳng $(BCD),\,(ABD),\,(ACD),\,(ABC)$ lần lượt là $GH,\,GP,\,PQ,\,QH$. Do đó thiết diện của tứ diện với mặt phẳng $(P)$ là tứ giác $HGPQ$.\\
			Ta có
			$$\heva{&(P)\parallel CD\\& CD\subset (ACD)\\&(P)\cap (ACD)=PQ}\Rightarrow PQ \parallel CD$$
			và
			$$\heva{&(P)\parallel CD\\& CD\subset (BCD)\\&(P)\cap (BCD)=HG}\Rightarrow HG \parallel CD.$$
			Ta có $\heva{&HG\parallel PQ\,(\text{cùng song song với} \,CD)\\&HQ\parallel PG\,(\text{cùng song song với}\, AB )}\Rightarrow $ tứ giác $HGPQ$ là hình bình hành.
		\end{listEX}
	}
\end{bt}
\begin{bt}
	Cho hình chóp $S.ABCD$ có đáy $ABCD$ là hình bình hành tâm $O$. Gọi $K$ và $J$ lần lượt là trọng tâm của các tam giác $ABC$ và $SBC$.
	\begin{listEX}
		\item Chứng minh $KJ \parallel (SAB)$.
		\item Gọi $(P)$ là mặt phẳng chứa $KJ$ và song song với $AD$. Xác định giao tuyến của mặt phẳng $(P)$ với các mặt của hình chóp. Hình tạo bởi các giao tuyến là hình gì?
	\end{listEX}
	\loigiai{
		\begin{center}
			\begin{tikzpicture}[>=stealth, line join=round, line cap = round]
				\tkzInit[ymin=-2.5,ymax=5.5,xmin=-2.5,xmax=5.5]
				\tkzClip
				\tkzDefPoints{0/0/A, 5/0/D, -2/-2/B, 1/5/S}
				\tkzDefPointBy[translation=from A to D](B)\tkzGetPoint{C}
				\tkzInterLL(A,C)(B,D) \tkzGetPoint{O}
				\tkzDefMidPoint(B,C)\tkzGetPoint{H}
				\tkzCentroid(A,B,C)\tkzGetPoint{K}
				\tkzCentroid(S,B,C)\tkzGetPoint{J}
				\tkzDefLine[parallel=through J](C,B) \tkzGetPoint{j}
				\tkzInterLL(J,j)(B,S) \tkzGetPoint{M} \tkzInterLL(J,j)(C,S) \tkzGetPoint{N}
				\tkzDefLine[parallel=through K](C,B) \tkzGetPoint{k}
				\tkzInterLL(K,k)(B,A) \tkzGetPoint{E} \tkzInterLL(K,k)(C,D) \tkzGetPoint{F}
				\tkzDrawPoints[fill=black](A,B,C,D,O,J,K,M,N,E,F,H)
				\tkzDrawSegments(B,C C,D S,B S,C S,D M,N N,F S,H)
				\tkzDrawSegments[dashed](A,B S,A A,C B,D A,D E,F M,E A,H K,J)
				\tkzLabelPoints[left](A,M,E)
				\tkzLabelPoints[below](B,H)
				\tkzLabelPoints[right](C,D,N,F)
				\tkzLabelPoints[above](S,O)
				\tkzLabelPoints[above right](J)
				\tkzLabelPoints[above left](K)
			\end{tikzpicture}
		\end{center}
		\begin{listEX}	
			\item Gọi $H$ là trung điểm $BC$, theo tính chất trọng tâm ta có $\dfrac{HK}{HA}=\dfrac{HJ}{HS}=\dfrac{1}{3}\Rightarrow KJ\parallel SA$ (Định lý Ta-lét đảo).
			Ta có $\heva{&KJ\parallel SA\\&SA\subset (SAB)\\&KJ \not\subset (SAB)}\Rightarrow KJ\parallel (SAB)$.
			\item 
			Gọi $\Delta_1=(P)\cap(ABCD)$, ta có\\
			$\heva{&K\in KJ,\,KJ\subset (P)\\&K\in (ABCD)}\Rightarrow K \in \Delta_1$.\\
			$\heva{&(P)\parallel AD\\& AD\subset (ABCD)\\&(P)\cap (ABCD)=\Delta_1}\Rightarrow \Delta_1 \parallel AD$.\\
			Vậy $\Delta_1$ là đường thẳng qua $K$ và song song với $AD$.\\
			Gọi $E=\Delta_1 \cap AB, F=\Delta_1 \cap CD$, ta được$(P)\cap (ABCD)=EF$.\\
			
			Gọi $\Delta_2=(P)\cap(SBC)$, ta có
			$$\heva{&J\in KJ,\,KJ\subset (P)\\&J\in (SBC)}\Rightarrow K \in \Delta_2.$$
			Và	
			$$\heva{&(P)\parallel AD\parallel BC\\& BC\subset (ABCD)\\&(P)\cap (ABCD)=\Delta_2}\Rightarrow \Delta_2 \parallel BC.$$
			Vậy $\Delta_2$ là đường thẳng qua $J$ và song song với $BC$.\\
			Gọi $M=\Delta_2 \cap SB, N=\Delta_2 \cap SD$, ta được $(P)\cap (SBC)=MN$.\\
			Ta có giao tuyến của $(P)$ với các mặt phẳng $(ABCD),\,(SCD),\,(SBC),\,(SAB)$ lần lượt là $EF,\,FN,\,NM,\,NE$, do đó thiết diện của hình chóp cắt bởi mặt phẳng $(P)$ là tứ giác $MNFE$.
		\end{listEX}
	}
\end{bt}

\begin{bt}
	Cho tứ diện $ABCD$. Lấy điểm $M$ trên cạnh $AB$ sau cho $AM=2MB$. Gọi $G$ là trọng tâm $\triangle BCD$ và $I$ là trung điểm $CD$, $H$ là điểm đối xứng của $G$ qua $I$.
	\begin{enumerate}
		\item Chứng minh $GD \parallel (MCH)$.
		\item Tìm giao điểm $K$ của $MG$ với $(ACD)$. Tính tỉ số $\dfrac{GK}{GM}$.
	\end{enumerate}
	\loigiai{
		\begin{center}
			\begin{tikzpicture}[scale=.8]
				\tkzDefPoints{0/0/B, 4/-2/C, 6/0/D}
				\coordinate (h) at ($(C)!.7!(D)$);   
				\coordinate (H) at ($(B)!.4!(h)$);
				\coordinate (A) at ($(H)+(0,5)$);
				\coordinate (M) at ($(A)!2/3!(B)$);
				\coordinate (I) at ($(C)!.5!(D)$);
				\tkzCentroid(D,B,C)\tkzGetPoint{G}
				\tkzDefPointBy[symmetry=center I](G)\tkzGetPoint{H}
				\tkzInterLL(A,I)(M,H)\tkzGetPoint{h}
				\tkzInterLL(A,I)(M,G)\tkzGetPoint{K}
				\tkzInterLL(A,I)(C,H)\tkzGetPoint{k}
				\tkzInterLL(C,H)(M,G)\tkzGetPoint{i}
				\tkzInterLL(M,H)(C,D)\tkzGetPoint{o}
				\coordinate (E) at ($(A)!.5!(K)$);
				\tkzDrawSegments[dashed](B,D M,I B,I I,H M,h G,D M,i h,k C,o G,E)
				\tkzDrawSegments(A,B A,C A,D B,C o,D A,h M,C C,H h,H k,K i,K D,H)
				\tkzLabelPoints[left](B,M)
				\tkzLabelPoints[right](D,K,E)
				\tkzLabelPoints[below](C,I,G,H)
				\tkzLabelPoints[above](A)
				\tkzDrawPoints[fill=black](A,B,C,D,M,I,G,H,K,E)
			\end{tikzpicture}
		\end{center}
		\begin{enumerate}
			\item Ta có $IC = ID$ và $IG = IH$ nên $GDHC$ là hình bình hành.\\
			Do đó $GD \parallel CH$\\
			mà $CH \subset (MCH)$ nên $GD \parallel (MCH)$.
			\item Trong mp$(ABI)$, gọi $K=AI \cap MG$, ta có $\heva{&K \in AI \subset (ACD)\\&K \in MG}$\\
			$\Rightarrow K=MG \cap (ACD).$\\
			Trong mp$(ABI)$, kẻ $GE \parallel AB$, $(E \in AI)$.\\
			Xét tam giác $ABI$, có $GE \parallel AB$, suy ra $\dfrac{GE}{AB}=\dfrac{IG}{IB}=\dfrac{1}{3} \Rightarrow \dfrac{GE}{AM}=\dfrac{1}{2}$.\\
			Xét tam giác $AKM$, có $GE \parallel AM$, suy ra $\dfrac{KG}{KM}=\dfrac{GE}{AM}=\dfrac{1}{2}\Rightarrow \dfrac{GK}{GM}=1.$
		\end{enumerate}
	}
\end{bt}


\begin{bt}%[Trần Nhân Kiệt]%[1H2G3]
	Cho hình chóp $S.ABCD$ có đáy là hình bình hành tâm $O, M$ là trung điểm của $SA$. Gọi $(P)$ là mặt phẳng qua $O$, song song với $BM$ và $SD.$ Tìm giao tuyến của $(P)$ và $(SAD).$
	\loigiai{
		\immini{
			\begin{itemize}
				\item Tìm giao tuyến của $(P)$ và $(SBD).$\\
				Ta có: $\heva{
					& O\in (P)\cap (SBD)\\
					& (P) \parallel SD\\
					& SD\subset (SBD)
				}$\\
				$\Rightarrow (P)\cap (SBD)=Ox$
				trong đó $Ox\parallel SD$.\\
				Gọi $Ox\cap SB=N \Rightarrow (P)\cap (SBD)=ON.$
				\item Tìm giao tuyến của $(P)$ và $(SAB).$\\
				Ta có: $\heva{
					& N\in (P)\cap (SAB)\\
					& (P) \parallel BM\\
					& BM\subset (SAB)
				}$\\
				$\Rightarrow (P)\cap (SAB)=Ny$
				trong đó $Ny\parallel BM$.\\
				Gọi $Ny\cap SA=E \Rightarrow (P)\cap (SAB)=NE.$
				\item Tìm giao tuyến của $(P)$ và $(SAD).$\\
				Ta có: $\heva{
					& E\in (P)\cap (SAD)\\
					& (P) \parallel SD\\
					& SD\subset (SAD)
				}$\\
				$\Rightarrow (P)\cap (SAD)=Ez$
				trong đó $Ez\parallel SD$.\\
				Gọi $Ez\cap AD=F \Rightarrow (P)\cap (SAD)=EF.$
			\end{itemize}	
		}{\begin{tikzpicture}[scale=0.6]\\
				\tkzInit[xmin=-3,xmax=8.5,ymin=-3,ymax=7]
				\tkzClip
				\tkzSetUpPoint[fill=black,size=4]
				\tkzDefPoints{0/0/A,-2.5/-2/B,5/-2/C,0.5/6/S}
				\tkzDefPointWith[colinear = at A](B,C)
				\tkzGetPoint{D}
				\tkzDefMidPoint(A,C)
				\tkzGetPoint{O}
				\tkzDefMidPoint(S,A)
				\tkzGetPoint{M}
				\tkzDefPointWith[linear,K=0.25](S,A)
				\tkzGetPoint{E}
				\tkzDefMidPoint(S,B)
				\tkzGetPoint{N}
				\tkzDefPointWith[linear,K=0.75](A,D)
				\tkzGetPoint{F}
				\tkzDrawPoints(A,B,C,D,S,M,N,E,F)
				\tkzDrawSegments[dashed](A,D A,B S,A B,D A,C E,F B,M N,E N,O)
				\tkzDrawSegments(B,C C,D S,B S,C S,D)
				\tkzLabelPoint[above right](F){\footnotesize $F$}
				\tkzLabelPoint[right](M){\footnotesize $M$}
				\tkzLabelPoint[left](N){\footnotesize $N$}
				\tkzLabelPoint[left](A){\footnotesize $A$}
				\tkzLabelPoint[below](B){\footnotesize $B$}
				\tkzLabelPoint[below](C){\footnotesize $C$}
				\tkzLabelPoint[right](D){\footnotesize $D$}
				\tkzLabelPoint[right](E){\footnotesize $E$}
				\tkzLabelPoint[above](S){\footnotesize $S$}
				\tkzLabelPoint[above](O){\footnotesize $O$}
		\end{tikzpicture}}
	}
\end{bt}


% \subsection{BÀI TẬP TRẮC NGHIỆM}
\Opensolutionfile{ans}[ans/1H4.B3]
\setcounter{ex}{0}

\begin{ex} Trong không gian cho mặt phẳng $(\alpha)$ và $A$ không thuộc $(\alpha)$. Qua điểm $A$ có thể dựng được bao nhiêu đường thẳng song song với $(\alpha)$?
	\choice
	{Duy nhất}
	{\True Vô số }
	{$2$}
	{$4$}
	\loigiai{}
\end{ex}
\begin{ex}%[1H2B3-1]
	Trong không gian cho đường thẳng $\Delta$ và điểm $O$ không nằm trong $\Delta$. Qua điểm $O$ cho trước, có bao nhiêu mặt phẳng song song với đường thẳng $\Delta$?
	\choice
	{\True Vô số}
	{ $3$}
	{ $1$}
	{ $2$}
	\loigiai{
		Gọi $d$ là đường thẳng qua $O$ và song song với $\Delta$. Khi đó có vô số mặt phẳng chứa $d$ và không chứa $\Delta$. Vậy có vô số mặt phẳng qua $O$ và song song với $\Delta$. }
\end{ex}

\begin{ex}%[1H2B3-1]
	Có bao nhiêu mặt phẳng song song với cả hai đường thẳng chéo nhau?
	\choice
	{\True Vô số}
	{$1 $}
	{$2 $}
	{$3 $}
	\loigiai{}
\end{ex}

\begin{ex}%[1H2B3-1]
	Cho hai đường thẳng phân biệt $a$, $b$ và mặt phẳng $\left(\alpha \right)$. Giả sử $a\parallel \left(\alpha \right), b\subset \left(\alpha \right)$. Khi đó
	\choice
	{$a\parallel b $}
	{$a,b$ chéo nhau}
	{$a,b$ cắt nhau}
	{\True $a\parallel b$ hoặc $a,b$ chéo nhau}
	\loigiai{
		\begin{center}
			\begin{tikzpicture}[scale=1]
				\tkzDefPoints{0/0/A', 4/0/B', 1/2/D', 1/1/A, 4/1/B, 0/2.5/M}
				\coordinate (C') at ($(B')+(D')-(A')$);
				\tkzDefPointBy[translation = from A' to B'](M) \tkzGetPoint{N}
				\tkzLabelSegment[pos=.3](M,N){ $a$}
				\tkzLabelSegment[pos=.3](A,B){ $b$}
				%\tkzDrawSegments[dashed](A,A' A',B' A',C')
				\tkzDrawPolygon(A',B',C',D')
				\tkzDrawSegments(A,B M,N)
				%\tkzLabelPoints[right](B)
				%        \tkzLabelPoints[above](A,B)
				\tkzMarkAngles[size=.7](B',A',D')
				\tkzLabelAngle[pos=0.4](D',A',B'){\footnotesize $\alpha$ }
			\end{tikzpicture}
			\hspace{2cm}
			\begin{tikzpicture}[scale=1]
				\tkzDefPoints{0/0/A', 4/0/B', 1/2/D', 1/1/A, 4/1/B, 0/2.5/M, 1/.5/P, 4/1.75/Q}
				\coordinate (C') at ($(B')+(D')-(A')$);
				\tkzDefPointBy[translation = from A' to B'](M) \tkzGetPoint{N}
				\tkzLabelSegment[pos=.3](M,N){ $a$}
				\tkzLabelSegment[pos=.9](A,B){ $c$}
				\tkzLabelSegment[pos=.1, below](P,Q){ $b$}
				%\tkzDrawSegments[dashed](A,A' A',B' A',C')
				\tkzDrawPolygon(A',B',C',D')
				\tkzDrawSegments(A,B M,N P,Q)
				%\tkzLabelPoints[right](B)
				%        \tkzLabelPoints[above](A,B)
				\tkzMarkAngles[size=.7](B',A',D')
				\tkzLabelAngle[pos=0.4](D',A',B'){\footnotesize $\alpha$ }
			\end{tikzpicture}
		\end{center}
		Vì $a\parallel \left(\alpha \right)$ nên tồn tại đường thẳng $c\subset \left(\alpha \right)$ thỏa mãn $a\parallel c $. Suy ra $b,c$ đồng phẳng và xảy ra các trường hợp sau:
		\begin{itemize}
			\item  Nếu $b$ song song hoặc trùng với $c$ thì $a\parallel b$.
			\item  Nếu $b$ cắt $c$ thì $b$ cắt $\left(\beta \right)\equiv \left({a,c}\right)$ nên $a,b$ không đồng phẳng. Do đó $a,b$ chéo nhau.
		\end{itemize}
	}
\end{ex}

\begin{ex}%[1H2B3-1]
	Cho hai đường thẳng phân biệt $a$, $b$ và mặt phẳng $\left(\alpha \right)$. Giả sử $a\parallel b$ và  $b\parallel \left(\alpha \right)$. Khẳng định nào sau đây là khẳng định đúng?
	\choice
	{$a\parallel \left(\alpha \right) $}
	{$a\subset \left(\alpha \right) $}
	{\True $a\parallel \left(\alpha \right)$ hoặc $a\subset \left(\alpha \right) $}
	{$a$ cắt $\left(\alpha \right) $}
	\loigiai{
	}
\end{ex}

\begin{ex}%[1H2B3-1]
	Cho đường thẳng $a$ nằm trong mặt phẳng $\left(\alpha \right)$ và đường thẳng $b$ không thuộc $\left(\alpha \right)$. Mệnh đề nào sau đây đúng?
	\choice
	{Nếu $b\parallel \left(\alpha \right)$ thì $b\parallel a $}
	{\True Nếu $b\parallel a$ thì $b\parallel \left(\alpha \right) $}
	{Nếu $b$ cắt $\left(\alpha \right)$ và $\left(\beta \right)$ chứa $b$ thì giao tuyến của $\left(\alpha \right)$ và $\left(\beta \right)$ là đường thẳng cắt cả $a$ và $b $.
	}
	{Nếu $b$ cắt $\left(\alpha \right)$ thì $b$ cắt $a $}
	\loigiai{
		\begin{itemize}
			\item  A sai. Nếu $b\parallel \left(\alpha \right)$ thì $b\parallel a$ hoặc $a,b$ chéo nhau.
			\item B sai. Nếu $b$ cắt $\left(\alpha \right)$ thì $b$ cắt $a$ hoặc $a,b$ chéo nhau.
			\item  D sai. Nếu $b$ cắt $\left(\alpha \right)$ và $\left(\beta \right)$ chứa $b$ thì giao tuyến của $\left(\alpha \right)$ và $\left(\beta \right)$ là đường thẳng cắt $a$ hoặc song song với $a$.
		\end{itemize}
	}
\end{ex}


\begin{ex}%[1H2B3-1]
	Cho hai đường thẳng chéo nhau $a$ và $b$. Khẳng định nào sau đây \textbf{sai}?
	\choice
	{\True Có duy nhất một mặt phẳng song song với $a$ và $b $}
	{Có vô số đường thẳng song song với $a$ và cắt $b $}
	{Có duy nhất một mặt phẳng qua $a$ và song song với $b $}
	{Có duy nhất một mặt phẳng qua điểm $M$, song song với $a$ và $b$ (với $M$ là điểm cho trước)}
	\loigiai{
		Có có vô số mặt phẳng song song với 2 đường thẳng chéo nhau.
		}
\end{ex}


\begin{ex}%[1H2B3-1]
	Cho $d\parallel \left(\alpha \right)$, mặt phẳng $\left(\beta \right)$ qua $d$ cắt $\left(\alpha \right)$ theo giao tuyến $d'$. Khẳng định nào sau đây là đúng?
	\choice
	{$d$ cắt $d'$}
	{\True $d\parallel d' $}
	{$d$ và $d'$ chéo nhau}
	{$d\equiv d' $}
	\loigiai{
		Ta có $d'=\left(\alpha \right)\cap \left(\beta \right)$. Do $d$ và $d'$ cùng thuộc $\left(\beta \right)$ nên $d$ cắt $d'$ hoặc $d\parallel d'$.
		Nếu $d$ cắt $d'$. Khi đó, $d$ cắt $\left(\alpha \right)$ (mâu thuẫn với giả thiết).
		Vậy $d\parallel d'$.}
\end{ex}

\begin{ex}%[1H2Y3-2]
	Cho hình chóp tứ giác $S.ABCD$. Gọi $M$ và $N$ lần lượt là trung điểm của $SA$ và $SC$. Khẳng định nào sau đây đúng?
	\choice  
	{\True $MN \parallel (ABCD)$}
	{$MN \parallel (SAB)$} 
	{$MN \parallel (SCD)$}
	{$MN \parallel (SBC)$}
	\loigiai{
		\immini{Xét tam giác $SAC$ có $M, N$ lần lượt là trung điểm của $SA, SC$.\\ Suy ra $MN \parallel AC$ nên $MN \parallel (ABCD)$.}{\begin{tikzpicture}[scale=0.8, line join=round, line cap=round,thick]
				\tikzset{label style/.style={font=\footnotesize}}
				\coordinate (S) at (1,2.5);
				\coordinate (A) at (0,0);
				\coordinate (B) at (1,-0.9);
				\coordinate (D) at (3.5,0);
				\coordinate (C) at (2.5,-1.3);
				\coordinate (M) at ($(S)!0.5!(A)$);
				\coordinate (N) at ($(S)!0.5!(C)$);
				%\draw[name path=MN,dashed] (M)--(N);
				\tkzDrawSegments(S,A S,B S,C S,D A,B B,C C,D)
				\tkzDrawSegments[dashed](A,D M,N A,C)
				\tkzDrawPoints[fill=black,size=2pt](S,A,B,C,D,M,N)
				\tkzLabelPoints[above](S)
				\tkzLabelPoints[left](A,M)
				\tkzLabelPoints[below](B,C)
				\tkzLabelPoints[right](D,N)
		\end{tikzpicture}}
		
		
		
	}
\end{ex}

\begin{ex}%[1H2B3-2]
	Cho hình chóp $S.ABCD$ có đáy $ABCD$ là hình bình hành, $M$ và $N$ là hai điểm trên $SA,SB$ sao cho $\dfrac{SM}{SA}=\dfrac{SN}{SB}=\dfrac{1}{3} $. Vị trí tương đối giữa $MN$ và $\left({ABCD}\right)$ là
	\choice
	{$MN$ và $\left({ABCD}\right)$ chéo nhau}
	{\True $MN$ song song $\left({ABCD}\right)$}
	{$MN$ nằm trong $\left({ABCD}\right) $}
	{$MN$ cắt $\left({ABCD}\right) $}
	\loigiai{
		Theo định lí Talet, ta có $\dfrac{SM}{SA}=\dfrac{SN}{SB}$ suy ra $MN$ song song với $AB. $ \\
		Mà $AB$ nằm trong mặt phẳng $\left({ABCD}\right)$ suy ra $MN \parallel \left({ABCD}\right) $.}
\end{ex}

\begin{ex}%[1H2Y3-3]
	\immini[thm]{Cho hình chóp $S.ABCD$ có đáy $ABCD$ là hình bình hành. Tìm giao tuyến của hai mặt phẳng $(SAD)$ và $(SBC)$.
		\choice
		{Là đường thẳng đi qua đỉnh $S$ và song song với đường thẳng $BD$}
		{Là đường thẳng đi qua đỉnh $S$ và tâm $O$ của đáy}
		{\True Là đường thẳng đi qua đỉnh $S$ và song song với đường thẳng $BC$}
		{Là đường thẳng đi qua đỉnh $S$ và song song với đường thẳng $AB$}
	}
	{\begin{tikzpicture}[scale=0.3,>=stealth]
			\tkzDefPoints{0/0/A, 8/0/B,-3/-3/D, 2/6/S}
			\tkzDefPointBy[translation = from A to B](D)\tkzGetPoint{C}
			\tkzDrawPoints(A,B,C,D,S)
			\tkzLabelPoints[below](A,B,C,D)
			\tkzLabelPoints[above](S)
			\tkzDrawSegments (S,B S,C S,D D,C C,B)
			\tkzDrawSegments[dashed](D,A A,B S,A)
		\end{tikzpicture}
	}
	\loigiai{
		\immini{Do hai mặt phẳng $(SAD)$ và $(SBC)$ có chung điểm $S$ và có hai đường thẳng $AD$, $BC$ song song với nhau nên giao tuyến của hai mặt phẳng $(SAD)$ và $(SBC)$ là đường thẳng đi qua đỉnh $S$ và song song với đường thẳng $BC$.
		}{
			\begin{tikzpicture}[scale=0.3,>=stealth]
				\tkzDefPoints{0/0/A, 8/0/B,-3/-3/D, 2/6/S}
				\tkzDefPointBy[translation = from A to B](D)\tkzGetPoint{C}
				\tkzDefPointBy[translation = from A to D](S)\tkzGetPoint{s}
				\tkzDefPointBy[homothety = center S ratio -0.7](s)\tkzGetPoint{s1}
				\tkzDrawPoints(A,B,C,D,S)
				\tkzLabelPoints[below](A,B,C,D)
				\tkzLabelPoints[above](S)
				\tkzDrawSegments (S,B S,C S,D D,C C,B s,s1)
				\tkzDrawSegments[dashed](D,A A,B S,A)
			\end{tikzpicture}
	}}
\end{ex}

\begin{ex}%[Hà Lê]%[1H2B3]
	Cho tứ diện $ABCD$ có $I$, $J$ lần lượt là trung điểm của $BC$, $BD$. Giao tuyến của mặt phẳng $(AIJ)$ và $(ACD)$ là
	\haicot
	{đường thẳng $d$ đi qua $A$ và song song với $BC$}
	{đường thẳng $d$ đi qua $A$ và song song với $BD$}
	{\True đường thẳng $d$ đi qua $A$ và song song với $CD$}
	{đường thẳng $AB$}
\end{ex}

\begin{ex}%[1H2B3-2]
	Cho tứ diện $ABCD$. Gọi $G$ là trọng tâm của tam giác $ABD,\ Q$ thuộc cạnh $AB$ sao cho $AQ=2QB$, $P$ là trung điểm của $AB $, $M$ là trung điểm của $BD$. Khẳng định nào sau đây đúng?
	\choice
	{$Q \in \left({CDP}\right) $}
	{$QG$ cắt $\left({BCD}\right) $}
	{$MP \parallel \left({BCD}\right) $}
	{\True $GQ \parallel \left({BCD}\right) $}
	\loigiai{
		\immini{
			Vì $G$ là trọng tâm tam giác $ABD$ $\Rightarrow \dfrac{AG}{AM}=\dfrac{2}{3} $.\\
			Điểm $Q\in AB$ sao cho $AQ=2QB\Leftrightarrow \dfrac{AQ}{AB}=\dfrac{2}{3} $. Suy ra $\dfrac{AG}{AM}=\dfrac{AQ}{AB}\xrightarrow{}GQ \parallel BD. $\\
			Mặt khác $BD$ nằm trong mặt phẳng $\left({BCD}\right)$ suy ra $GQ \parallel \left({BCD}\right) $.
		}{
			\begin{tikzpicture}[scale=1]
			\tkzDefPoints{1/3/A, 0/0/B, 2/-2/C, 5/0/D}
			\tkzCentroid(A,B,D)\tkzGetPoint{G}
			\coordinate (M) at ($(B)!.5!(D)$);
			\coordinate (Q) at ($(A)!.67!(B)$);
			\coordinate (P) at ($(A)!.5!(B)$);
			\tkzDrawSegments[dashed](A,M B,D Q,G P,D)
			\tkzDrawPolygon(A,B,C,D)
			\tkzDrawSegments(A,C)
			\tkzLabelPoints[left](B,P,Q)
			\tkzLabelPoints[right](D)
			\tkzLabelPoints[above](A)
			\tkzLabelPoints[below](M,C)
			\tkzLabelPoints[above right](G)
			\tkzDrawPoints(A,B,C,D,M,P,Q,G)
			\end{tikzpicture}
		}
	}
\end{ex}


\begin{ex}%[1H2K3-2]
	Cho hai hình bình hành $ABCD$ và $ABEF$ không cùng nằm trong một mặt phẳng. Gọi $O$, $O_1$ lần lượt là tâm của $ABCD$, $ABEF $; $M$ là trung điểm của $CD. $ Khẳng định nào sau đây \textbf{sai}?
	\choice
	{$OO_1 \parallel \left({BEC}\right) $}
	{$OO_1 \parallel \left({EFM}\right) $}
	{\True $MO_1$ cắt $\left({BEC}\right) $}
	{$OO_1 \parallel \left({AFD}\right) $}
	\loigiai{
		\immini{
			Xét tam giác $ACE$ có $O,O_1$ lần lượt là trung điểm của $AC,AE $.\\
			Suy ra $OO_1$ là đường trung bình trong tam giác $ACE\Rightarrow OO_1 \parallel EC. $\\
			Tương tự, $OO_1$ là đường trung bình của tam giác $BFD$ nên $OO_1 \parallel FD. $\\
			Vậy $OO_1 \parallel \left({BEC}\right), OO_1 \parallel \left({AFD}\right)$ và $OO_1 \parallel \left({EFC}\right)$. Chú ý rằng: $\left({EFC}\right)\equiv \left({EFM}\right) $.
		}{
			\begin{tikzpicture}[scale=1]
				\tkzDefPoints{0/0/A, 4/0/B, 1/2/D, 1.5/-2.5/F}
				\coordinate (C) at ($(D)+(B)-(A)$);
				\coordinate (E) at ($(F)+(B)-(A)$);
				\coordinate (M) at ($(C)!0.5!(D)$);
				\tkzInterLL(A,C)(B,D)\tkzGetPoint{O}
				\tkzInterLL(A,E)(B,F)\tkzGetPoint{O_1}
				\tkzDrawSegments[dashed](B,D A,C A,E B,F O,O_1 A,B B,E B,C)
				\tkzDrawPolygon(D,C,E,F)
				\tkzDrawSegments(A,D A,F)
				\tkzLabelPoints[left](A)
				\tkzLabelPoints[right](B)
				\tkzLabelPoints[below](E,F,O_1)
				\tkzLabelPoints[above](C,D,O,M)
				\tkzDrawPoints(A,B,C,D,O,E,F,O_1,M)
			\end{tikzpicture}
		}
	}
\end{ex}

\begin{ex}%[HK1 THPT Quốc Thái, An Giang 2018]%[Lê Nguyễn Viết Tường-DA 11HK1-18]%[1H2K3-4]%
	Cho hình chóp $S.ABCD$ có đáy là hình bình hành. Thiết diện của hình chóp khi cắt bởi mặt phẳng đi qua trung điểm $M$ của cạnh $AB$ và song song với $BD,SA$ là hình gì?
	\choice
	{Ngũ giác}
	{Hình thang}
	{Tam giác}
	{\True Hình bình hành}
	\loigiai{
		\immini{
			Gọi $(\alpha)$ là mặt phẳng đi qua $M$ và song song với $BD,SA$.
			Ta có
			$BD\parallel (\alpha),BD\subset (ABCD),(\alpha)\cap (ABCD)=Mx$
			$\Rightarrow Mx\parallel BD\Rightarrow Mx$ cắt $AD$ tại $N$ trong $(ABCD)$.
			$SA\parallel (\alpha),SA\subset (SAD),(\alpha)\cap (SAD)=Ny$
			$\Rightarrow Ny\parallel SA\Rightarrow Ny$ cắt $SD$ tại $P$ trong $(SAD)$.
			$SA\parallel (\alpha),SA\subset (SAB),(\alpha)\cap (SAB)=Mt$
			$\Rightarrow Mt\parallel SA\Rightarrow Mt$ cắt $SB$ tại $Q$ trong $(SAB)$.
			Vậy thiết diện là hình bình hành $MNPQ$.
		}{
			\begin{tikzpicture}[line join = round, line cap = round]
				\tkzDefPoints{0/0/D,1.5/1.6/A,5.5/1.6/B,4/0/C,2.5/6/S}
				\tkzDefMidPoint(A,B)\tkzGetPoint{M}
				\tkzDefMidPoint(A,D)\tkzGetPoint{N}
				\tkzDefMidPoint(S,D)\tkzGetPoint{P}
				\tkzDefMidPoint(S,B)\tkzGetPoint{Q}
				\tkzLabelPoints[](M,N)
				\tkzLabelPoints[above](S)
				\tkzLabelPoints[above right](A)
				\tkzLabelPoints[right](B,C,Q)
				\tkzLabelPoints[left](D,P)
				\tkzDrawPoints[fill=black](S,A,B,C,D,M,N,P,Q)
				\tkzDrawSegments[](S,B S,C S,D B,C C,D)
				\tkzDrawSegments[dashed](S,A A,B A,D M,N N,P B,D M,Q Q,P)
			\end{tikzpicture}
		}
	}
\end{ex}
% \centerline{---HẾT---}
\Closesolutionfile{ans}


%%Bài 13. Hai MP song song
% \setcounter{section}{12}
\setcounter{dang}{0}
\section{HAI MẶT PHẲNG SONG SONG}
\subsection{KIẾN THỨC CẦN NHỚ}

\subsubsection{VỊ TRÍ TƯƠNG ĐỐI CỦA HAI MẶT PHẲNG}
Cho hai mặt phẳng $(P)$ và $(Q)$. Các trường hợp có thể xảy ra:
\begin{itemize}
	\item [\iconMT] \indam{Trường hợp 1:} $(P)$ và $(Q)$ trùng nhau.
	\item [\iconMT] \indam{Trường hợp 2:} $(P)$ và $(Q)$ có một điểm chung. Khi đó chúng sẽ có điểm chung khác nữa. Tập hợp tất cả các điểm chung đó gọi là  giao tuyến của hai mặt phẳng $(P)$ và $(Q)$ (\textbf{Hình 1}).
	\item [\iconMT] \indam{Trường hợp 3:} $(P)$ và $(Q)$ không có điểm chung. Khi đó ta nói $(P)$ song song $(Q)$ (\textbf{Hình 2}).
	\begin{itemize}
		\item Kí hiệu $(P)\parallel (Q)$;
		\item Khi $(P)\parallel (Q)$ và $a \subset (P)$ thì $a \parallel (Q)$.
	\end{itemize}
\end{itemize}
\hspace*{1cm}
\begin{tikzpicture}[line join = round, line cap = round,>=stealth,font=\footnotesize,scale=0.7]
\tkzDefPoints{0/0/A,5/0/B,6.5/1.5/C}
\coordinate (D) at ($(A)+(C)-(B)$);
\coordinate (E) at ($(A)!0.5!(B)$);
\coordinate (F) at ($(D)!0.5!(C)$);
\coordinate (M) at ($(E)+(-0.5,2)$);
\coordinate (N) at ($(F)+(-0.5,2)$);
\coordinate (Q) at ($(M)!2!(E)$);
\coordinate (P) at ($(N)!2!(F)$);
\tkzInterLL(M,Q)(C,D)    \tkzGetPoint{G}
\tkzInterLL(N,P)(A,B)    \tkzGetPoint{H}
\draw ($(A)+(0.6,0.2)$) node {$P$};
\draw ($(N)+(-0.1,-0.5)$) node {$Q$};
\tkzDrawSegments(A,B B,C C,F G,D N,F H,P D,A M,N P,Q Q,M E,F)
\tkzDrawSegments[dashed](G,F F,H)
\tkzMarkAngles[size=1cm,arc=l](B,A,D)
\tkzMarkAngles[size=1cm,arc=l](M,N,P)
\draw (3.25,-2.5) node {\textbf{Hình 1.}};
\draw (3.5,1) node[below] {$d$};
\draw (3.25,-3.5) node {$(P)$, $(Q)$ cắt nhau: $(P)\cap (Q)=d$};
\end{tikzpicture}
\hspace*{2cm}
\begin{tikzpicture}[line join = round, line cap = round,>=stealth,font=\footnotesize,scale=0.7]
\begin{scope}[shift={(8,0)}]
\tkzDefPoints{0/0/A,5/0/B,6.5/2/C}
\coordinate (D) at ($(A)+(C)-(B)$);
\draw ($(A)+(0.5,0.3)$) node {$P$};
\tkzDrawSegments(A,B B,C C,D D,A)
\tkzMarkAngles[size=1cm,arc=l](B,A,D)
\end{scope}
\begin{scope}[shift={(8,-3)}]
\tkzDefPoints{0/0/A,5/0/B,6.5/2/C}
\coordinate (D) at ($(A)+(C)-(B)$);
\draw ($(A)+(0.5,0.3)$) node {$Q$};
\draw (2,4)--(5,4.5)node[below]{$a$};
\tkzDrawSegments(A,B B,C C,D D,A)
\tkzMarkAngles[size=1cm,arc=l](B,A,D)
\draw (3.25,-1) node {\textbf{Hình 2.}};
\draw (3.25,-2) node {$(P)$, $(Q)$ không có điểm chung: $(P)\parallel (Q)$};
\end{scope}
\end{tikzpicture}
\subsubsection{CÁC ĐỊNH LÝ CƠ BẢN}
\begin{enumerate}[\iconMT]
	\item \indam{Định lý 1:} \immini{
		Nếu mặt phẳng $(\alpha)$ chứa hai đường thẳng cắt nhau $a$, $b$ và $a$, $b$ cùng song song với mặt phẳng $(\beta)$ thì $(\alpha)$ song song với $(\beta)$.
	}
	{
		\begin{tikzpicture}[line join = round, line cap = round,>=stealth,font=\footnotesize,scale=0.7]
			\begin{scope}[shift={(0,0)}]
				\tkzDefPoints{0/0/A,5/0/B,6.5/1.5/C}
				\coordinate (D) at ($(A)+(C)-(B)$);
				\coordinate (a) at ($(A)+(1.5,0.3)$);
				\coordinate (b) at ($(A)+(4.5,0.3)$);
				\coordinate (c) at ($(A)+(5.5,1.3)$);
				\coordinate (d) at ($(A)+(1.7,1.3)$);
				\tkzInterLL(a,c)(b,d)    \tkzGetPoint{M}
				\draw ($(A)+(0.5,0.2)$) node {$\alpha$};
				\tkzDrawSegments(A,B B,C C,D D,A a,c b,d)
				\tkzMarkAngles[size=0.8cm,arc=l](B,A,D)
				\tkzLabelPoints[above](M,a,b)
				\tkzDrawPoints[fill=black](M)
			\end{scope}
			\begin{scope}[shift={(0,-2)}]
				\tkzDefPoints{0/0/A,5/0/B,6.5/1.5/C}
				\coordinate (D) at ($(A)+(C)-(B)$);
				\draw ($(A)+(0.5,0.2)$) node {$\beta$};
				\tkzDrawSegments(A,B B,C C,D D,A)
				\tkzMarkAngles[size=0.8cm,arc=l](B,A,D)
			\end{scope}
		\end{tikzpicture}
	}

	\begin{note}
		\begin{itemize}
			\item Muốn chứng minh hai mặt phẳng song song, ta phải chứng minh có hai đường thẳng cắt nhau thuộc mặt phẳng này lần lượt song song với mặt phẳng kia.
			\item Muốn chứng minh đường thẳng $a\parallel
			(Q)$, ta chứng minh đường thẳng $a$ nằm trong mặt
			phẳng $(P)$ và $(P)\parallel (Q)$.
		\end{itemize}
	\end{note}
	\item \indam{Định lý 2:} \immini{
		Qua một điểm nằm ngoài một mặt phẳng cho trước có một và chỉ một mặt phẳng song song với mặt phẳng đã cho.
	}
	{
		\begin{tikzpicture}[line join = round, line cap = round,>=stealth,font=\footnotesize,scale=0.7]
			\begin{scope}[shift={(0,0)}]
				\tkzDefPoints{0/0/A,5/0/B,6.5/1.5/C}
				\coordinate (D) at ($(A)+(C)-(B)$);
				\coordinate (a) at ($(A)+(3.5,0.8)$);
				\draw ($(A)+(0.5,0.2)$) node {$\alpha$};
				\draw ($(a)+(0,-0.1)$) node[above] {$A$};
				\tkzDrawSegments(A,B B,C C,D D,A)
				\tkzMarkAngles[size=0.8cm,arc=l](B,A,D)
				\tkzDrawPoints[fill=black](a)
			\end{scope}
			\begin{scope}[shift={(0,-2)}]
				\tkzDefPoints{0/0/A,5/0/B,6.5/1.5/C}
				\coordinate (D) at ($(A)+(C)-(B)$);
				\draw ($(A)+(0.5,0.2)$) node {$\beta$};
				\tkzDrawSegments(A,B B,C C,D D,A)
				\tkzMarkAngles[size=0.8cm,arc=l](B,A,D)
			\end{scope}
		\end{tikzpicture}
	}
\end{enumerate}
\immini{
\iconMT\,\indam{Định lý 3:}Cho hai mặt phẳng song song. Nếu một mặt phẳng cắt mặt phẳng này thì cũng cắt mặt phẳng kia và hai giao tuyến song song với nhau.}
{\begin{tikzpicture}[font=\footnotesize]
	\coordinate (A) at (0,0);
	\coordinate (B) at (3.1,0);
	\coordinate (C) at (4,1.2);
	\coordinate (A1) at (0,1.7);
	\coordinate (M) at (1,3.2);
	\coordinate (N) at (2,-1.5);
	\coordinate (D) at ($(C)-(B)+(A)$);
	\coordinate (B1) at ($(B)-(A)+(A1)$);
	\coordinate (C1) at ($(C)-(B)+(B1)$);
	\coordinate (D1) at ($(C1)-(B1)+(A1)$);
	\coordinate (Q) at ($(D1)-(A1)+(M)$);
	\coordinate (P) at ($(N)-(M)+(Q)$);
	\path[name path=mn] (M)--(N); 
	\path[name path=pq] (P)--(Q);
	\path[name path=ab] (A)--(B); 
	\path[name path=cd] (C)--(D);
	\path[name path=a1b1] (A1)--(B1); 
	\path[name path=c1d1] (C1)--(D1);
	\path[name intersections={of=mn and ab,by=E}];
	\path[name intersections={of=mn and a1b1,by=F}];
	\path[name intersections={of=mn and cd,by=E1}];
	\path[name intersections={of=mn and c1d1,by=F1}];
	\path[name intersections={of=pq and ab,by=G1}];
	\path[name intersections={of=pq and cd,by=G}];
	\path[name intersections={of=pq and a1b1,by=H1}];
	\path[name intersections={of=pq and c1d1,by=H}];
	\draw (E1)--(D)--(A)--(B)--(C)--(G)--(E) (F1)--(D1)--(A1)--(B1)--(C1)--(H)--(F) (H)--(Q)--(M)--(N)--(P)--(G1) (G)--(H1);
	\draw[dashed] (H1)--(H)--(F1) (G1)--(G)--(E1);
	\draw ($(E)!.6!(G)$) node[left]{$a$} ($(F)!.6!(H)$) node[left]{$b$};
	\draw pic[draw,"$\beta$"]{angle=B--A--D} pic[draw,"$\alpha$"]{angle=B1--A1--D1} pic[draw,"$\gamma$", angle eccentricity=0.7]{angle=M--Q--P};
\end{tikzpicture}}
\immini{\iconMT\,\indam{Định lý 4:} (\textit{Định lí Thales}) Ba mặt phẳng đôi một song song chắn trên hai cát tuyến bất kì những đoạn thẳng tương ứng tỉ lệ.}
{\begin{tikzpicture}[line join = round, line cap = round,>=stealth,font=\footnotesize,scale=0.7]
	\tkzDefPoints{0/0/A,5/0/B,6.5/1.5/C,3.5/2/d,4/2/d',1/-4.9/e,4.8/-4.9/e'}
	\coordinate (D) at ($(A)+(C)-(B)$);
	\coordinate (E) at ($(A)+(0,-2.2)$);
	\coordinate (F) at ($(E)+(5,0)$);
	\coordinate (G) at ($(E)+(6.5,1.5)$);
	\coordinate (H) at ($(E)+(G)-(F)$);
	\coordinate (M) at ($(A)+(0,-4.4)$);
	\coordinate (N) at ($(M)+(5,0)$);
	\coordinate (P) at ($(M)+(6.5,1.5)$);
	\coordinate (Q) at ($(M)+(P)-(N)$);
	\draw ($(A)+(0.5,0.2)$) node {$\alpha$};
	\draw ($(E)+(0.5,0.2)$) node {$\beta$};
	\draw ($(M)+(0.5,0.2)$) node {$\gamma$};
	\tkzInterLL(d,e)(A,B)    \tkzGetPoint{m}
	\tkzInterLL(d',e')(A,B)    \tkzGetPoint{n}
	\tkzInterLL(d,e)(E,F)    \tkzGetPoint{p}
	\tkzInterLL(d',e')(E,F)    \tkzGetPoint{q}
	\tkzInterLL(d,e)(M,N)    \tkzGetPoint{r}
	\tkzInterLL(d',e')(M,N)    \tkzGetPoint{s}
	\coordinate (a) at ($(d)!0.2!(e)$);
	\coordinate (b) at ($(d)!0.5!(e)$);
	\coordinate (c) at ($(d)!0.8!(e)$);
	\coordinate (a') at ($(d')!0.15!(e')$);
	\coordinate (b') at ($(d')!0.5!(e')$);
	\coordinate (c') at ($(d')!0.85!(e')$);
	\tkzDrawSegments(A,B B,C C,D D,A E,F F,G G,H H,E M,N N,P P,Q Q,M d,a m,b p,c r,e d',a' n,b' q,c' s,e' a,a' b,b' c,c')
	\draw (a) node[left] {$A$};
	\draw (a') node[right] {$A'$};
	\draw (b) node[left] {$B$};
	\draw (b') node[right] {$B'$};
	\draw (c) node[left] {$C$};
	\draw (c') node[right] {$C'$};
	\tkzDrawSegments[dashed](a,m b,p c,r a',n b',q c',s)
	\tkzMarkAngles[size=0.7cm,arc=l](B,A,D F,E,H N,M,Q)
	\tkzDrawPoints[fill=black](a,a',b,b',c,c')
\end{tikzpicture}}

\subsubsection{HÌNH LĂNG TRỤ VÀ HÌNH HỘP}
\begin{enumerate}[\iconMT]
	\immini{\item \indam{Định nghĩa:}
	Cho hai mặt phẳng $(\alpha)\parallel(\alpha')$. Trong $(\alpha)$ cho đa giác lồi $A_1A_2\ldots A_n$. Qua các điểm $A_1,A_2,\ldots,A_n$ ta dựng các đường song song với nhau và cắt $(\alpha')$ tại $A'_1,A'_2,\ldots,A'_n$.
	
	Hình tạo thành bởi hai đa giác $A_1A_2\ldots A_n$, $A'_1A'_2\ldots A'_n$ cùng với các hình bình hành $A_1A_2A'_2A'_1$, $A_2A_3A'_3A'_2$, \ldots, $A_nA_1A'_1A'_n$ được gọi là \textit{hình lăng trụ} và được ký hiệu bởi $A_1A_2\ldots A_n.A'_1A'_2\ldots A'_n$.
	
	\begin{itemize}
		\item Hai đa giác $A_1A_2\ldots A_n$, $A'_1A'_2\ldots A'_n$ được gọi là hai \textit{mặt đáy} (bằng nhau) của hình lăng trụ.
		\item Các đoạn thẳng $A_1A'_1$, $A_2A'_2$,\ldots, $A_nA'_n$ gọi là các \textit{cạnh bên} của hình lăng trụ.
		\item Các hình bình hành $A_1A_2A'_2A'_1$, $A_2A_3A'_3A'_2$,\ldots, $A_nA_1A'_1A'_n$ gọi là các \textit{mặt bên} của hình lăng trụ.
		\item Các đỉnh của hai đa giác đáy gọi là các \textit{đỉnh} của hình lăng trụ.
	\end{itemize}
}{
	\begin{tikzpicture}[font=\footnotesize]
	\coordinate (A) at (0,0);
	\coordinate (B) at (5.5,0);
	\coordinate (C) at (6.5,2);
	\coordinate (A1) at (0,4.5);
	\coordinate (M1) at (1,7);
	\coordinate (N1) at (2,-1);
	\coordinate (D) at ($(C)-(B)+(A)$);
	\coordinate (B1) at ($(B)-(A)+(A1)$);
	\coordinate (C1) at ($(C)-(B)+(B1)$);
	\coordinate (D1) at ($(C1)-(B1)+(A1)$);
	\coordinate (A11) at ($(M1)!.78!(N1)$);
	\coordinate[shift={(1,-0.3)}] (M2) at (1,7);
	\coordinate[shift={(3,-0.2)}] (M3) at (1,7);
	\coordinate[shift={(3.5,0.5)}] (M4) at (1,7);
	\coordinate[shift={(1.5,0.7)}] (M5) at (1,7);
	\coordinate[shift={(1,-0.3)}] (N2) at (2,-1);
	\coordinate[shift={(3,-0.2)}] (N3) at (2,-1);
	\coordinate[shift={(3.5,0.5)}] (N4) at (2,-1);
	\coordinate[shift={(1.5,0.7)}] (N5) at (2,-1);
	\coordinate[shift={(1,-0.3)}] (B11) at ($(M1)!.78!(N1)$);
	\coordinate[shift={(3,-0.2)}] (C11) at ($(M1)!.78!(N1)$);
	\coordinate[shift={(3.5,0.5)}] (D11) at ($(M1)!.78!(N1)$);
	\coordinate[shift={(1.5,0.7)}] (E11) at ($(M1)!.78!(N1)$);
	\coordinate (A22) at ($(M1)!.22!(N1)$);
	\coordinate[shift={(1,-0.3)}] (B22) at ($(M1)!.22!(N1)$);
	\coordinate[shift={(3,-0.2)}] (C22) at ($(M1)!.22!(N1)$);
	\coordinate[shift={(3.5,0.5)}] (D22) at ($(M1)!.22!(N1)$);
	\coordinate[shift={(1.5,0.7)}] (E22) at ($(M1)!.22!(N1)$);
	\path[name path=k1] (M1)--(N1); 
	\path[name path=k2] (M2)--(N2);
	\path[name path=k3] (M3)--(N3);
	\path[name path=k4] (M4)--(N4);
	\path[name path=k5] (M5)--(N5);
	\path[name path=ab] (A)--(B); 
	\path[name path=cd] (C)--(D);
	\path[name path=a1b1] (A1)--(B1);
	\path[name intersections={of=k1 and ab,by=E1}];
	\path[name intersections={of=k2 and ab,by=E2}];
	\path[name intersections={of=k3 and ab,by=E3}];
	\path[name intersections={of=k4 and ab,by=E4}];
	\path[name intersections={of=k5 and ab,by=E5}];
	\path[name intersections={of=k1 and cd,by=F1}];
	\path[name intersections={of=k4 and cd,by=F4}];
	\path[name intersections={of=k1 and a1b1,by=G1}];
	\path[name intersections={of=k2 and a1b1,by=G2}];
	\path[name intersections={of=k3 and a1b1,by=G3}];
	\path[name intersections={of=k4 and a1b1,by=G4}];
	\path[name intersections={of=k5 and a1b1,by=G5}];
	\draw (M1)--(A22) (M2)--(B22) (M3)--(C22) (M4)--(D22) (M5)--(E22);
	\draw (F1)--(D)--(A)--(B)--(C)--(F4) (A1)--(B1)--(C1)--(D1)--cycle (A11)--(G1) (B11)--(G2) (C11)--(G3) (D11)--(G4) (E1)--(N1) (E2)--(N2) (E3)--(N3) (E4)--(N4) (E5)--(N5) (A11)--(B11)--(C11)--(D11) (A22)--(B22)--(C22)--(D22)--(E22)--cycle;
	\draw[dashed] (A11)--(E1) (B11)--(E2) (C11)--(E3) (D11)--(E4) (E11)--(E5) (F1)--(F4) (A22)--(G1) (B22)--(G2) (C22)--(G3) (D22)--(G4) (E22)--(E11) (A11)--(E11)--(D11);
	\draw (A11) node[left] {$A_1$} (A22) node[left] {$A_1'$} (B11) node[below left] {$A_2$} (C11) node[below left] {$A_3$} (D11) node[right] {$A_4$} (E11) node[above left] {$A_5$} (B22) node[above right] {$A_2'$} (C22) node[below left] {$A_3'$} (D22) node[right] {$A_4'$} (E22) node[above left] {$A_5'$};
	\tkzMarkAngles[size=0.7cm,arc=l](B,A,D B1,A1,D1)
	\tkzLabelAngles[pos=0.5](B,A,D){$\alpha$}
	\tkzLabelAngles[pos=0.5](B1,A1,D1){$\alpha'$}
	\end{tikzpicture}}
	\item \indam{Tính chất:}
		\begin{itemize}
			\item Các cạnh bên của hình lăng trụ thì song song và bằng nhau.
			\item Các mặt bên của hình lăng trụ đều là hình bình hành.
			\item Hai đáy của hình lăng trụ là hai đa giác bằng nhau.
	\end{itemize}
	\begin{tcolorbox}[colframe=cyan,colback=red!3!white,boxrule=0.5mm]
		Hình lăng trụ có đáy là hình bình hành gọi là \textit{hình hộp}. 
		\begin{itemize}
			\item Các mặt của hình hộp là hình bình hành.
			\item Hai mặt phẳng lần lượt chứa hai mặt đối diện của hình hộp thì song song nhau.
		\end{itemize}
	\end{tcolorbox}
	\item \indam{Minh họa vài mô hình thường gặp:}\\
	\begin{tabular}{llll}
		\begin{tikzpicture}[line cap=round,line join=round, scale=.6]%[Hoàng Anh]
			\tkzDefPoints{0/0/A, 1.5/1/B, 3.5/-0.5/C, -0.5/3.5/z}
			\coordinate (D) at ($(A)+(z)$);
			\coordinate (E) at ($(B)+(D)-(A)$);%Vẽ hình bình hành ABED
			\coordinate (F) at ($(C)+(E)-(B)$);
			\tkzDrawPolygon(E,F,D) %Vẽ đa giác EFD
			\tkzDrawSegments(D,A C,F A,C) %Vẽ các đoạn thẳng AD, CF, AC
			\tkzDrawSegments[dashed](A,B B,C B,E)%Vẽ nét đứt các đoạn thẳng AB, BC, BE
		\end{tikzpicture}
		&
		\begin{tikzpicture}[line cap=round,line join=round,  scale=0.5]
			%-------------- Đáy ABCD
			\tkzDefPoints{0/0/A, -0.5/-1/B, 3/0/D, 1/-1/C}
			
			%-------------- Đáy A'B'C'D'
			\tkzDefPointBy[rotation = center A angle 100](D) \tkzGetPoint{A'} %Phép quay tâm A, góc quay 90 độ, biến D thành A'
			\coordinate (B') at ($(B)+(A')-(A)$);
			\coordinate (D') at ($(D)+(A')-(A)$);
			\coordinate (C') at ($(C)+(B')-(B)$);
			%---------------
			\tkzDrawSegments[dashed](A,B A,D A,A')
			\tkzDrawPolygon(A',B',C',D')
			\tkzDrawPolygon(B,C,C',B')
			\tkzDrawSegments(D,D' C,D)
		\end{tikzpicture}
		&
		\begin{tikzpicture}[line cap=round,line join=round,scale=0.8]
			%-------------- Đáy ABCD
			\tkzDefPoints{0/0/A, -0.5/-1/B, 2/0/D}
			\coordinate (C) at ($(B)+(D)-(A)$);
			%-------------- Đáy A'B'C'D'
			\tkzDefPointBy[rotation = center A angle 100](D) \tkzGetPoint{A'} %Phép quay tâm A, góc quay 90 độ, biến D thành A'
			\coordinate (B') at ($(B)+(A')-(A)$);
			\coordinate (D') at ($(D)+(A')-(A)$);
			\coordinate (C') at ($(B')+(D')-(A')$);
			%---------------
			\tkzDrawSegments[dashed](A,B A,D A,A')
			\tkzDrawPolygon(A',B',C',D')
			\tkzDrawPolygon(B,C,C',B')
			\tkzDrawSegments(D,D' C,D)
		\end{tikzpicture}
		&
		\begin{tikzpicture}[line cap=round, line join=round, scale=.45]
			\tkzDefPoints{1/5/A, 3.5/5/B,  6/5.5/C, 5/7.5/D, 2/7.5/E, 0/0/A'}
			\tkzDefPointsBy[translation= from A to A'](B,C,D,E){B',C',D',E'}
			\tkzFillPolygon[white](A,B,B',A')
			\tkzFillPolygon[white](C,B,B',C')
			\tkzDrawPolygon(A,B,C,D,E)
			\tkzDrawSegments[dashed](E,E' D,D' A',E' E',D' D',C')
			\tkzDrawSegments(A,A' B,B' C,C' A',B' B',C')
		\end{tikzpicture}\\
		\small * Lăng trụ tam giác & \small * Lăng trụ tứ giác  & \small * Hình hộp  & \small * Lăng trụ ngũ giác 
	\end{tabular}
\end{enumerate}
% \subsection{PHÂN LOẠI, PHƯƠNG PHÁP GIẢI TOÁN}
\begin{dang}{Chứng minh hai mặt phẳng song song}
\begin{enumerate}[\iconMT]
	\item \indam{Phương pháp:} 
	\immini{Chứng minh trên mặt phẳng này có hai đường thẳng cắt nhau cùng song song với mặt phẳng còn lại.
		$$\heva{&a \text{ cắt } b\\&a \subset (\alpha),\,b\subset (\alpha)\\&a\parallel(\beta),\, b\parallel (\beta)}\quad\Rightarrow (\alpha) \parallel (\beta).$$

}{
\begin{tikzpicture}[scale=0.4]
\tkzDefPoints{0/0/A, 9/0/B, 12/2.5/C}
\coordinate (A1) at ($(A)+(0,3)$);
\coordinate (B1) at ($(B)+(0,3)$);
\coordinate (C1) at ($(C)+(0,3)$);
\coordinate (D) at ($(A)-(B)+(C)$);
\coordinate (D1) at ($(A1)-(B1)+(C1)$);
\tkzDefPoints{2/3.3/a1, 9/5/a2, 3/5/b1, 8/3.3/b2}
\tkzDrawSegments(a1,a2 b1,b2 A,B B,C C,D D,A A1,B1 B1,C1 C1,D1 D1,A1)
\tkzDefPoint[label=above right:$a$](2.5,3.5){a}
\tkzDefPoint[label=right:$b$](4,5){b}
\tkzMarkAngles[size=1.6cm,arc=l](B,A,D)
\tkzMarkAngles[size=1.6cm,arc=l](B1,A1,D1)
\tkzLabelAngles[pos=1,rotate=30](B1,A1,D1){\scriptsize$\alpha$}
\tkzLabelAngles[pos=1,rotate=30](B,A,D){\scriptsize$\beta$}
%	\draw pic[draw,"$\beta$",  angle radius=1cm]{angle=B--A--D} pic[draw,"$\alpha$", angle radius=1cm]{angle=B1--A1--D1};
\end{tikzpicture}}
	\item \indam{Chú ý:} Hai mặt phẳng phân biệt cùng song song với mặt phẳng thứ ba thì song song nhau.
\end{enumerate}
\end{dang}

\begin{vd}
	Cho hình chóp $S.ABCD$ có đáy là hình bình hành tâm $O$. Gọi $M$, $N$, $P$ lần lượt là trung điểm của $SA$, $SD$ và $SB$.
	\begin{enumerate}
		\item Chứng minh rằng $(MNP)\parallel (ABCD)$.
		\item Chứng minh rằng $(OMN)\parallel (SBC)$.
	\end{enumerate}
	\loigiai{
		\begin{center}
			\begin{tikzpicture}[scale=0.8]
				\tkzDefPoints{-2/3/A, 0/0/B, 6/0/C, 3.4/7/S}
				\coordinate (D) at ($(C)-(B)+(A)$);
				\tkzInterLL(A,C)(B,D)\tkzGetPoint{O}
				\tkzDefMidPoint(S,A)\tkzGetPoint{M}
				\tkzDefMidPoint(S,D)\tkzGetPoint{N}
				\tkzDefMidPoint(S,B)\tkzGetPoint{P}
				%\tkzDefPointBy[homothety=center N ratio 0.25](M)\tkzGetPoint{H}
				\tkzDrawSegments(A,B B,C S,A S,B S,C M,P)
				\tkzDrawSegments[dashed](A,D C,D A,C N,P B,D S,D M,N O,M O,N)
				\tkzDrawPoints(A,B,C,D,O,S,M,N,P)
				\tkzLabelPoints[above left](A,M)
				\tkzLabelPoints[below left](B)
				\tkzLabelPoints[below right](C,P)
				\tkzLabelPoints[right](D,N,D)
				\tkzLabelPoints[below](O)
				\tkzLabelPoints[above](S)
			\end{tikzpicture}
		\end{center}
		\begin{enumerate}
			\item Chứng minh $(MNP)\parallel (ABCD)$.
			
			Ta có
			
			$\begin{cases}MN\parallel AD,\text{ (do $MN$ là đường trung bình của $\Delta SAD$)}\\AD\subset (ABCD).\end{cases}$
			Suy ra  $MN\parallel (ABCD).$
			
			Ta lại có
			
			$\begin{cases}NP\parallel AB,\text{ (do $NP$ là đường trung bình của $\Delta SAB$)}\\AB\subset (ABCD).\end{cases}$
			Suy ra  $NP\parallel (ABCD).$
			
			Mặt khác, $MN,NP\subset (ABCD)$.
			
			Vậy $(MNP)\parallel (ABCD)$.
			\item Chứng minh $(OMN)\parallel (SBC)$.
			
			Ta có $MN\parallel AD$, ($MN$ là đường trung bình của $\Delta SAD$) và $AD\parallel BC$, (do $ABCD$ là hình bình hành) nên $MN\parallel BC$.
			
			Mà $BC\subset (SBC)$ nên $MN\parallel (SBC)$.
			
			Ta lại có $OM\parallel SC$, (do $OM$ là đường trung bình của $\Delta SAC$).
			
			Mà $SC\subset (SBC)$ nên $OM\parallel (SBC)$.
			
			Mặt khác $(MN,OM\subset (OMN)$.
			
			Vậy $(OMN)\parallel (SBC)$.
		\end{enumerate}
	}
\end{vd}

\begin{vd}
	Cho hình chóp $S.ABCD$ với đáy $ABCD$ là hình thang mà $AD\parallel BC$ và $AD=2BC$. Gọi $M$, $N$ lần lượt là trung điểm của $SA$ và $AD$. Chứng minh: $(BMN)\parallel (SCD)$.
	\loigiai{
		\immini
		{
			Vì $N$ là trung điểm của $AD$ nên $NA=ND=\dfrac{AD}{2}=BC$.\\
			Tứ giác $NBCD$ có $ND=BC$ và $ND\parallel BC$ nên $NBCD$ là hình bình hành, suy ra $NB\parallel CD\Rightarrow NB\parallel (SCD)$.\\
			Tam giác $SAD$ có $M$, $N$ lần lượt là trung điểm của $AS$ và $AD$ nên $MN$ là đường trung bình của $\triangle ADS$, suy ra $MN\parallel SD\Rightarrow MN\parallel (SCD)$.\\
			Từ $\heva{& MN\parallel (SCD),\text{ }MN\subset (BMN) \\ & BN\parallel (SCD), \text{ }BN\subset (BMN)}\Rightarrow (BMN)\parallel (SCD).$
		}
		{
			\begin{tikzpicture}[line join = round, line cap = round,>=stealth,font=\footnotesize,scale=0.6]
			\tkzDefPoints{0/0/A,8/0/D,1.5/-2/B,5.5/-2/C,6/5/S}
			\tkzDefMidPoint(A,D)\tkzGetPoint{N}
			\tkzDefMidPoint(A,S)\tkzGetPoint{M}
			\tkzDrawSegments[dashed](A,D B,N M,N)
			\tkzDrawSegments(S,A S,B S,C S,D A,B B,C C,D B,M)
			\tkzDrawPoints[fill=black](A,B,C,D,S,M,N)
			\tkzLabelPoints[above](S,M)
			\tkzLabelPoints[below](B,C,N)
			\tkzLabelPoints[left](A)
			\tkzLabelPoints[right](D)
			\end{tikzpicture}
		}
	}
\end{vd}

\begin{vd}
	Cho hai hình bình hành $ABCD$ và $ABEF$ có chung cạnh $AB$ và không đồng phẳng. Gọi $I$, $J$, $K$ lần lượt là trung điểm $AB$, $CD$, $EF$. Chứng minh
	\begin{listEX}[2]
		\item $(ADF)\parallel (BCE)$.
		\item $(DIK)\parallel (JBE)$.
	\end{listEX}
	\loigiai{	\immini{\begin{listEX}
				\item Ta có $AD\parallel BC$, suy ra $AD\parallel (BCE)$. Tương tự $AF\parallel (BCE)$.\\
				Khi đó $(ADF)\parallel (BCE)$.
				\item Trong hình bình hành $ABCD$ có $I$, $J$ lần lượt là trung điểm của $AB$ và $CD$ nên $BI=DJ$. Do đó $IBJD$ là hình bình hành. Suy ra $DI\parallel BJ$ nên $DI\parallel (JBE)$. \\
				Trong hình bình hành $ABEF$ có $I$, $K$ lần lượt là trung điểm của $AB$ và $EF$ nên $IK\parallel EF$, suy ra $IK\parallel (JBE)$. \\
				Vậy $(DIK)\parallel (JBE)$.
		\end{listEX}}{
			\begin{tikzpicture}[scale=1, font=\footnotesize, line join=round, line cap=round, >=stealth]
			\clip(-4.6,-1.8) rectangle (5.4,4.5);
			\tkzDefPoints{2/-1/C,-1/4/F,4/1/B,-2/1/A,-4/-1/D,5/4/E}
			\tkzDefMidPoint(A,C)
			\tkzGetPoint{O}
			\tkzDefMidPoint(B,A)
			\tkzGetPoint{I}
			\tkzDefMidPoint(C,D)
			\tkzGetPoint{J}
			\tkzDefMidPoint(E,F)
			\tkzGetPoint{K}
			%\tkzDefMidPoint(O,M) \tkzGetPoint{I}
			\tkzDrawSegments(D,C C,B B,E E,F D,F D,K J,E)
			\tkzDrawSegments[dashed](A,F A,B A,D D,I I,K B,J)
			\tkzDrawPoints(A,B,C,D,E,F,I,J,K)
			%	\tkzLabelPoint[above right](M){$M$}
			\tkzLabelPoint[above](K){$K$}
			\tkzLabelPoint[below](J){$J$}
			\tkzLabelPoint[above left](I){$I$}
			\tkzLabelPoint[above](F){$F$}
			\tkzLabelPoint[left](A){$A$}
			\tkzLabelPoint[right](B){$B$}
			\tkzLabelPoint[above](E){$E$}
			%\tkzLabelPoint[below](O){$O$}
			\tkzLabelPoint[right](C){$C$}
			\tkzLabelPoint[left](D){$D$}
			\end{tikzpicture}}	
	}
\end{vd}

\begin{vd}
	Cho hình lăng trụ $ABC.A'B'C'$. Gọi $I$, $J$, $K$ lần lượt là trọng tâm các tam giác $ABC$, $ACC'$, $A'B'C'$. Chứng minh rằng $(IJK)\parallel (BCC'B')$ và $(A'JK)\parallel (AIB')$.
	\loigiai{
		\begin{center}
			\begin{tikzpicture}[>=stealth=0.3, line join=round, line cap = round,scale=0.5]
			\tkzDefPoints{0/0/A, 2/-3/B, 8/0/C,1/7/A'}
			\tkzDefPointBy[translation= from A to A'](B)\tkzGetPoint{B'}
			\tkzDefPointBy[translation= from A to A'](C)\tkzGetPoint{C'}
			\tkzDefMidPoint(B,C)\tkzGetPoint{M}
			\tkzDefMidPoint(C,C')\tkzGetPoint{N}
			\tkzDefMidPoint(B',C')\tkzGetPoint{P}
			\tkzDefPointBy[homothety=center A ratio 2/3](M)\tkzGetPoint{I}
			\tkzDefPointBy[homothety=center A ratio 2/3](N)\tkzGetPoint{J}
			\tkzDefPointBy[homothety=center A' ratio 2/3](P)\tkzGetPoint{K}
			\tkzInterLL(A',J)(A,C)\tkzGetPoint{X}
			\tkzDrawSegments[dashed](A,C A,M A,C' A,N I,J I,K J,K A',J I,B' A',X)
			\tkzDrawSegments(A,B B,C A,A' B,B' C,C' A',B' B',C' A',C' A',P M,P A,B' M,N C,P B',M)
			%Gán nhãn
			\tkzDrawPoints[fill=black](A,B,C,A',B', C',I,M,J,N,P,K,X)
			\tkzLabelPoints[left](A)
			\tkzLabelPoints[right=3pt](C,M,N)
			\tkzLabelPoints[below](B,I,J)
			\tkzLabelPoints[above](A',C',P,K)
			\tkzLabelPoints[left=4pt](B')
			\end{tikzpicture}
		\end{center}
		\begin{enumerate}
			\item  Gọi $M$, $N$, $P$ lần lượt là trung điểm của $BC$,  $CC'$ và $B'C'$. Theo tính chất của trọng tâm tam giác ta có
			$$\dfrac{AI}{AM}=\dfrac{AJ}{AN}\Rightarrow IJ\parallel MN.$$
			Tứ giác $AMPA'$ là hình bình hành và có $\dfrac{AI}{AM}=\dfrac{AK}{AP} =\dfrac{2}{3}\Rightarrow IK \parallel MP$.\\
			Vậy $(IJK)\parallel (BCC'B')$.
			\item Chú ý rằng mặt phẳng $(AIB')$ chính là mặt phẳng $(AMB')$. Mặt phẳng $(A'JK)$ chính là mặt phẳng $(A'CP)$.\\
			Vì $AM \parallel A'P$, $MB'\parallel CP$ (do tứ giác $B'MCP$ là hình bình hành). Vậy ta có $(A'JK)\parallel (AIB')$.
		\end{enumerate}
	}
\end{vd}

\begin{vd}
	Cho hai hình vuông $ABCD$ và $ABEF$ ở trong hai mặt phẳng phân biệt. Trên các đường chéo $AC$ và $BF$ lần lượt lấy các điểm $M$, $N$ sao cho $AM=BN$. Các đường thẳng song song với $AB$ vẽ từ $M$, $N$ lần lượt cắt $AD$ và $AF$ tại $M'$ và $N'$. 
	\begin{tasks}(2)
		\task Chứng minh rằng $(ADF)\parallel (BCE)$.
		\task Chứng minh rằng $(CDF)\parallel (MM'N'N)$. 
	\end{tasks}
	\loigiai{
		\begin{center}
			\begin{tikzpicture}[>=stealth=0.3, line join=round, line cap = round,scale=0.6]
			\tkzDefPoints{0/0/A, -2/-2/D,6/0/B, 4/-2/C, 2.5/3/E,-3.5/3/F}
			\tkzDefPointBy[homothety=center A ratio 1/3](C)\tkzGetPoint{M}
			\tkzDefPointBy[homothety=center B ratio 1/3](F)\tkzGetPoint{N}
			\tkzDefLine[parallel=through M](A,B)\tkzGetPoint{c}
			\tkzInterLL(M,c)(A,D)\tkzGetPoint{M'}
			\tkzDefLine[parallel=through N](A,B)\tkzGetPoint{d}
			\tkzInterLL(N,d)(A,F)\tkzGetPoint{N'}
			% Phần này trở xuống để gán nhãn
			\tkzDrawPoints[fill=black](A,B,C,D,E,F,M',M,N,N')
			\tkzDrawSegments(D,C B,E F,D E,C E,F C,B)
			\tkzDrawSegments[dashed](A,B A,D A,F A,C B,F M,M' N,N'  M',N' M,N)
			\tkzLabelPoints[left](A,M',N')
			\tkzLabelPoints[below left](M)
			\tkzLabelPoints[below](B,C,D)
			\tkzLabelPoints[above](E,F,N)
			\end{tikzpicture}
		\end{center}
		\begin{enumerate}[a)]
			\item Ta có 
			$\heva{&AD \parallel BC\\&AF \parallel BE\\&AD \cap AF =A}\Rightarrow (ADE)\parallel (BCF)$. 
			\item Ta có 
			\[ MM'\parallel CD \Rightarrow \dfrac{AM}{AC}=\dfrac{AM'}{AD} \tag \label{1} \]
			Ta cũng có 
			\[ NN'\parallel AB\Rightarrow \dfrac{BN}{BF}=\dfrac{AN'}{AF} \tag \label{2} \]
			Mà từ giả thiết ta có 
			\[ \dfrac{AM}{AC}= \dfrac{BN}{BF}\Rightarrow \dfrac{AM'}{AD}= \dfrac{AN'}{AF} \tag \label{3} \]
			Từ $(3)$ suy ra $M'N'\parallel DF$. Ta cũng có $MM'\parallel NN'\parallel DC \parallel FE$. \\
			Vậy  $(CDF)\parallel (MM'N'N)$. 
		\end{enumerate}
	}
\end{vd}

\begin{dang}{Chứng minh đường thẳng song song với mặt phẳng}
	Để chứng minh $a$ song song $(P)$, ta thường sử dụng một trong hai cách sau
	\begin{enumerate}[\iconMT]
		\item \indam{Cách 1: } (\textit{Đã xét ở bài học trước}) Ta cần chứng tỏ các ý sau:
		\begin{itemize}
			\item [$\bullet$] $a$ không nằm trên $(P)$;
			\item [$\bullet$] $a$ song song với một đường thẳng $b$ nằm trong $(P)$. Suy ra $a\parallel (P)$ hay $$\heva{&a\not\subset (P)\\&a\parallel b \\&b\subset (P)\\}\Rightarrow a\parallel (P)$$
		\end{itemize}
		\item \indam{Cách 2:} Ta chứng minh đường thẳng $a$ nằm trong mặt
		phẳng $(Q)$ và $(Q)\parallel (P)$ thì $a \parallel (P)$.
	\end{enumerate}
\end{dang}

\begin{vd}%[1H2B4-2]
	Cho hình chóp $S.ABCD$ có đáy $ABCD$ là hình bình hành. Gọi $G_1$, $G_2$, $G_3$ lần lượt là trọng tâm các tam giác $SAB$, $ABC$, $SBD$. Gọi $M$ là một điểm thuộc đường thẳng $G_2 G_3$. Chứng minh $G_1M \parallel (SBC)$.
	\loigiai{
		\begin{center}
			\begin{tikzpicture}[scale=0.6]
			\tkzDefPoints{3/3/A, 0/0/B, 8/0/C, 11/3/D, 3/10/S}
			\tkzDefMidPoint(B,D)\tkzGetPoint{O}
			\tkzDefMidPoint(B,A)\tkzGetPoint{N}
			\tkzDefMidPoint(S,A)\tkzGetPoint{I}
			\path[name path=sn] (S)--(N);
			\path[name path=bi] (B)--(I);
			\path[name intersections={of=sn and bi,by=G_1}];
			\path[name path=bd] (B)--(D);
			\path[name path=cn] (C)--(N);
			\path[name intersections={of=bd and cn,by=G_2}];
			\path[name path=so] (S)--(O);
			\path[name path=ci] (C)--(I);
			\path[name intersections={of=so and ci,by=G_3}];
			\coordinate (M) at ($(G_2)!0.2!(G_3)$);
			\draw (S)--(B)--(C)--(D)--(S) (S)--(C);
			\draw[dashed] (S)--(A)--(B) (A)--(D) (B)--(D) (A)--(C) (S)--(O) (S)--(N) (C)--(N);
			\draw[dashed] (G_1)--(G_2)--(G_3)--(G_1);
			\fill (A) circle (2pt) node[above right]{$A$};
			\fill (B) circle (2pt) node[below]{$B$};
			\fill (C) circle (2pt) node[below]{$C$};
			\fill (D) circle (2pt) node[right]{$D$};
			\fill (S) circle (2pt) node[above]{$S$};
			\fill (O) circle (2pt) node[above right]{$O$};
			\fill (G_1) circle (2pt) node[left]{$G_1$};
			\fill (G_2) circle (2pt) node[below]{$G_2$};
			\fill (G_3) circle (2pt) node[right]{$G_3$};
			\fill (N) circle (2pt) node[left]{$N$};
			\fill (M) circle (2pt) node[right]{$M$};
			\end{tikzpicture}
		\end{center}
		Gọi $O$ là tâm hình bình hành $ABCD$ và $N$ là trung điểm $AB$, suy ra $G_1 \in SN$, $G_2 \in CM$, $G_3 \in SO$.
		
		Do $G_1$, $G_2$ lần lượt là trọng tâm tam giác $SAB$, $ABC$ nên ta có: $$\begin{cases}
		\dfrac{NG_1}{MS}=\dfrac{1}{3} \bigskip \\
		\dfrac{NG_2}{MC}=\dfrac{1}{3}
		\end{cases}$$
		$\Rightarrow$ $G_1G_2 \parallel SC$ (Định lý Ta-lét trong $\Delta NSC$)
		
		$\Rightarrow$ $G_1G_2 \parallel (SBC)$.
		
		Do $G_2$, $G_3$ lần lượt là trọng tâm tam giác $ABC$, $SBD$ nên ta có:
		$$\begin{cases}
		\dfrac{OG_2}{OB}=\dfrac{1}{3} \bigskip \\
		\dfrac{OG_3}{OS}=\dfrac{1}{3}
		\end{cases}$$
		$\Rightarrow$ $G_2G_3 \parallel SB$ (Định lý Ta-lét trong $\Delta SOB$).
		
		$\Rightarrow$ $G_2G_3 \parallel (SBC)$.
		
		Ta đã có: $\begin{cases} G_1G_2 \parallel (SBC) \\ G_2G_3 \parallel (SBC) \end{cases}$ $\Rightarrow$ $(G_1G_2G_3) \parallel (SBC)$ 
		
		Mà $G_1M \subset (G_1G_2G_3)$ $\Rightarrow$ $G_1M \parallel (SBC)$.
	}
\end{vd}

\begin{vd}%[1H2B4-2]
	Cho hình chóp $S.ABCD$ có đáy là hình bình hành tâm $O$. Gọi $M$, $N$ lần lượt là trung điểm của $SA$ và $CD$.
	\begin{enumerate}
		\item Chứng minh hai mặt phẳng $(OMN)$ và $(SBC)$ song song với nhau.
		\item Gọi $I$ là trung điểm của $SD$, $J$ là một điểm trên $(ABCD)$ và cách đều $AB$, $CD$. Chứng minh $IJ$ song song với $(SAB)$.
		\end{enumerate}
	\loigiai{
		\begin{enumerate}
			\begin{minipage}{0.5\textwidth}
				\item Chứng minh $(OMN)\parallel (SBC)$.
				
				Do $ON$, $OM$ theo thứ tự là đường trung bình của các tam giác $BCD$ và $SAC$ nên $OM \parallel BC$, $ON \parallel SC$.
				
				Hơn nữa, $ON$, $OM$ không chứa trong $(SBC)$. Do đó $ON \parallel (SBC)$, $OM \parallel (SBC)$.
				
				Mặt khác, $OM \cap ON= O$ nên $(OMN)\parallel (SBC)$. 
				\item Chứng minh $IJ\parallel (SAB)$.
				
				Trong mặt phẳng $(ABCD)$, $O$ và $J$ cách đều hai đường thẳng song song $AB$ và $CD$ nên $OJ\parallel AB \parallel CD$. Hơn nữa, $OJ$ không chứa trong $(SAB)$. Do đó, $OJ\parallel (SAB)$.
			\end{minipage}
			\begin{minipage}{0.5\textwidth}
				\begin{tikzpicture}[scale=0.8]
				\tkzDefPoints{1.7/3/A, 0/0/B, 7/0/C, 1.2/7/S}
				\coordinate (D) at ($(C)-(B)+(A)$);
				\tkzInterLL(A,C)(B,D)\tkzGetPoint{O}
				\tkzDefMidPoint(S,A)\tkzGetPoint{M}
				\tkzDefMidPoint(C,D)\tkzGetPoint{N}
				\tkzDefMidPoint(S,D)\tkzGetPoint{I}
				%\coordinate (E) at ($1/3*(B)+2/3*(S)$);
				%\coordinate (F) at ($1/3*(C)+2/3*(D)$);
				\coordinate (J) at ($1.5*(O)-0.5*(3.5,0)$);
				%\tkzDefPointBy[homothety=center N ratio 0.25](M)\tkzGetPoint{H}
				\tkzDrawSegments(S,B B,C C,D S,C S,D)
				\tkzDrawSegments[dashed](S,A A,B A,C B,D A,D O,I I,J M,N O,M O,N O,J)
				\tkzDrawPoints(A,B,C,D,O,S,M,N,I,J)
				\tkzLabelPoints[above right](M)
				\tkzLabelPoints[below left](B)
				%\tkzLabelPoints[below right](C,P)
				\tkzLabelPoints[right](D,N,J)
				\tkzLabelPoints[below](O,C,A)
				\tkzLabelPoints[above](S,I)
				\end{tikzpicture}
			\end{minipage}
			Mặt khác, $OI$ là đường trung bình trong tam giác $SBD$ nên $OI \parallel SB$. Do đó, $OJ\parallel (SAB)$.
			
			Mặt phẳng $(OIJ)$ chứa hai đường thẳng cắt nhau và cùng song song với $(SAB)$ nên $(OIJ)\parallel (SAB)$. Hơn nữa, $IJ\subset (OIJ)$. Vì vậy, $IJ\parallel (SAB)$.
		\end{enumerate}}
\end{vd}


\begin{dang}{Định lý Thales}
	\textbf{Định lí Thales}: Ba mặt phẳng đôi một song song chắn trên hai cát tuyến bất kì những đoạn thẳng tương ứng tỉ lệ.
\end{dang}
\begin{vd}
	Cho ba mặt phẳng $(P),(Q)  ,(R) $  đôi một song song với nhau. Đường thẳng $a$   cắt các mặt phẳng $(P),(Q)  ,(R) $   lần lượt tại  $A, B, C$ sao cho $\dfrac{AB}{BC}=\dfrac{2}{3}$  và đường thẳng $b$   cắt các mặt phẳng $(P),(Q)  ,(R) $  lần lượt tại $A', B', C'$ . Tính tỉ số $\dfrac{A'B'}{B'C'}$  .
	\loigiai{
	Vì ba mặt phẳng  $(P),(Q)  ,(R) $  đôi một song song với nhau, áp dụng định lý Ta – lét trong không gian, ta có
			$$\dfrac{AB}{BC}=\dfrac{A'B'}{B'C'}=\dfrac{2}{3}.$$
}
\end{vd}
\begin{vd}
	Cho ba mặt phẳng  $(P),(Q)  ,(R) $  đôi một song song với nhau. Đường thẳng $a$  cắt các mặt phẳng $(P),(Q)  ,(R) $   lần lượt tại   $A, B, C$ sao cho $\dfrac{AB}{BC}=\dfrac{1}{3}$    và đường thẳng $b$  cắt các mặt phẳng   $(P),(Q)  ,(R) $   lần lượt tại $D, E, F$ .Tính tỉ số $\dfrac{ED}{DF}$  .
	\loigiai{
		Vì ba mặt phẳng  $(P),(Q)  ,(R) $  đôi một song song với nhau, áp dụng định lý Ta – lét trong không gian, ta có $$\dfrac{AB}{BC}=\dfrac{DE}{EF}=\dfrac{1}{3}$$
		Suy ra 
		$$EF=3DE\Rightarrow DF=EF+DE=4DE\Rightarrow \dfrac{DE}{DF}=\dfrac{1}{4}.$$
}
\end{vd}
\begin{vd}
	Cho hình tứ diện $S.ABC$ . Trên cạnh $SA$  lấy các điểm $A_1, A_2$  sao cho $2AA_1=2A_1A_2=A_2S$  . Gọi $(P)$   và $(Q)$  là hai mặt phẳng song song với mặt phẳng $(ABC)$  và lần lượt đi qua $A_1$, $A_2$ . Mặt phẳng $(P)$  cắt các cạnh $SB$, $SC$ lần lượt tại $B_1$, $C_1$ . Mặt phẳng $(Q)$  cắt các cạnh $SB$, $SC$  lần lượt tại $B_2$, $C_2$ . Chứng minh $2BB_1= 2B_1B_2=B_2S$ và $2CC_1=2C_1C_2=C_2S$ .
	\loigiai{
	\immini{
		Theo giả thiết thì $A_2A_1=A_1A$ và $A_2S=2A_2A_1$.\\
		Vì mặt phẳng $(P)$  qua $A_1$   song song với mặt phẳng $(ABC)$  nên 
		
		$$\heva{&(P) \cap (SAB)=A_1B_1, \text{ với } A_1B_1 \parallel AB\\& (P) \cap (SBC)=B_1C_1, \text{ với } B_1C_1 \parallel BC}$$
		
		Vì mặt phẳng $(Q)$  qua $A_2$  song song với mặt phẳng $(ABC)$  nên		
		$$\heva{&(Q) \cap (SAB)=A_2B_2, \text{ với } A_2B_2 \parallel AB\\& (Q) \cap (SBC)=B_2C_2, \text{ với } B_2C_2 \parallel BC}$$
	}{
\begin{tikzpicture}[scale=0.7, font=\footnotesize,>=stealth]
	\path
	%	Vẽ mp
	(0,0) coordinate (A)
	(1.5,-2) coordinate (B)
	(4,0) coordinate (C)
	(1,4) coordinate (S)
	(2.3,-0.7) coordinate (I)
	($(A)!0.5!(S)$)coordinate (A_2)
	($(A_2)!0.5!(A)$)coordinate (A_1)
	%
	($(B)!0.5!(S)$)coordinate (B_2)
	($(B_2)!0.5!(B)$)coordinate (B_1)
	%
	($(C)!0.5!(S)$)coordinate (C_2)
	($(C_2)!0.5!(C)$)coordinate (C_1)
	;
	\draw (S)--(B)--(C)--(S)--(A)--(B) (A_1)--(B_1)--(C_1) (A_2)--(B_2)--(C_2);
	\draw[dashed] (A)--(C) (A_1)--(C_1) (A_2)--(C_2);
	\foreach \x/\g in {A/-180,B/-90,C/0,S/90,A_1/170,A_2/160,B_1/-30,B_2/60,C_1/0,C_2/30}\draw[fill=black] (\x) circle (.05) +(\g:.5)node{\footnotesize$\x$};
\end{tikzpicture}}
	Các mặt phẳng $(ABC)$, $(A_1B_1C_1)$ và $(A_2B_2C_2)$ đôi một song song nhau nên theo định lí Ta let ta có 
	\begin{itemize}
		\item [$\bullet$] $\dfrac{A_2A_1}{A_1A}=\dfrac{B_2B_1}{B_1B}=\dfrac{C_2C_1}{C_1C}=1 \Rightarrow B_2B1=B_1B \text{ và } C_2C1=C_1C$ \quad (1);
		\item [$\bullet$] $\dfrac{SA_2}{A_2A_1}=\dfrac{SB_2}{B_2B_1}=\dfrac{SC_2}{C_2C_1}=2 \Rightarrow SB_2=2B_2B_1 \text{ và }  SC_2=2C_2C_1$ \quad (2)
	\end{itemize}

	Từ (1) và (2) suy ra
	Nên  $2BB_1= 2B_1B_2=B_2S$ và $2CC_1=2C_1C_2=C_2S$.
}
\end{vd}
\begin{vd}
	Một kệ để đồ bằng gỗ có mâm tầng dưới $(ABCD)$ và mâm tầng trên $(EFGH)$ song song với nhau. Bác thợ mộc đo được $AE=80$ cm, $CG=90$ cm và muốn đóng thêm một mâm tầng giữa $(IJKL)$ song song với hai mâm tầng trên và dưới sao cho khoảng cách $EI=36$ cm (tham khảo hình vẽ). Hãy giúp bác thợ mộc tính độ dài $GK$ để đặt mâm tầng giữa cho kệ để đồ đúng vị trí.
	%Hình vẽ 
	\loigiai{
	Theo định lý Thales ta có $\dfrac{EI}{GK}=\dfrac{AE}{CG}=\dfrac{80}{90}=\dfrac{8}{9}$. Suy ra $GK = 40,5$ cm.
}
\end{vd}
\begin{vd}
	Cho hình chóp $S.ABC$ có $SA=9, SB=12, SC=15$. Trên cạnh $SA$ lấy các điểm $M, N$ sao cho $SM=4, MN=3, NA=2$. Vẽ hai mặt phẳng song song với $(ABC)$ lần lượt đi qua $M, N$ , cắt $SB$ theo thứ tự $M', N'$ và cắt $SC$ theo thứ tự $M'', N''$. Tính độ dài các đoạn thẳng $SM', M'N', M''N'', N''C$.
\end{vd}

\begin{dang}{Hình hộp, hình lăng trụ}
\end{dang}
\begin{vd}
Cho hình hộp $ABCD.A'B'C'D'$ và một mặt phẳng $(\alpha)$ cắt các mặt của hình hộp theo các giao tuyến $MN, NP, PQ, QR, RS, SM$ như hình vẽ. Chứng minh các cặp cạnh đối của lục giác $MNPQRS$ song song nhau.
%Hình vẽ
\end{vd}
\begin{vd}
	Cho hình lăng trụ tứ giác $ABCD.A'B'C'D'$  với đáy là hình thang $AB\parallel CD$   . Một mặt phẳng song song với mặt phẳng $(AA'B'B)$   cắt các cạnh $AD, BC, B'C', A'D'$  lần lượt tại $E, F, M, H$ . Hỏi hình tạo bởi các điểm $E, F, M, H, D, D', C', C$  là hình gì?
	\loigiai{
}
\end{vd}
\begin{vd}
 Cho lăng trụ tam giác $A B C \cdot A' B' C'$. Gọi $M, N, P$ lần lượt là các điểm trên cạnh $A A', B B', C C'$ sao cho: $\dfrac{A M}{M A'}=\dfrac{B N}{N B'}=\dfrac{C P}{P C'}=\dfrac{1}{2}$. Hỏi hình tạo bởi các điểm $M, N, P, A', B', C'$ là hình gì?
 \loigiai{
}
\end{vd}


\subsection{BÀI TẬP TỰ LUYỆN}

\begin{bt}
	Cho hình chóp $S.ABCD$ có đáy $ABCD$ là hình bình hành tâm $O$. Gọi $M$, $N$ lần lượt là trung điểm của $SA$ và $CD$.	Chứng minh hai mặt phẳng $(MNO)$ và $(SBC)$ song song.
	\loigiai{
		\begin{center}
			\begin{tikzpicture}
				\tkzInit[xmin=-0.5,ymin=-0.5,xmax=7.5,ymax=6.5]
				\tkzDefPoints{0/0/B}
				\tkzDefShiftPoint[B](0:5){C}
				\tkzDefShiftPoint[B](50:2.5){A}
				\tkzDefShiftPoint[C](50:2.5){D}   
				\tkzDefShiftPoint[A](80:4){S} 
				\tkzInterLL(A,C)(B,D) \tkzGetPoint{O}
				\tkzDefMidPoint(S,A)\tkzGetPoint{M}
				\tkzDefMidPoint(C,D)\tkzGetPoint{N}
				\tkzDrawPolygon(S,B,C)
				\tkzDrawSegments[](C,D S,D)
				\tkzDrawSegments[dashed](A,B A,D S,A B,D A,C M,N N,O M,O)
				\tkzDrawPoints[fill=black](A,B,C,D,S,O,M,N)
				\tkzLabelPoints[above](S)
				\tkzLabelPoints[below left](B)
				\tkzLabelPoints[below right](C)
				\tkzLabelPoints[below](O)
				\tkzLabelPoints[right](D,N)
				\tkzLabelPoints[above left](A)
				\tkzLabelPoints[above right](M)
			\end{tikzpicture}
		\end{center}
		Ta có $M$ là trung điểm $SA$, $O$ là trung điểm $AC$\\
		$\Rightarrow MO$ là đường trung bình $\triangle SAC$\\
		$\Rightarrow MO \parallel SC$.\\
		Tương tự $ON \parallel BC$.\\
		Do đó $(OMN) \parallel (SBC)$.	
	}
\end{bt}

\begin{bt}%[Trần Thị Thu Hằng]%[1H2K4]
	Cho hình chóp $S.ABCD$, đáy $ABCD$ là hình thang có $AB \parallel CD$ và $AB=2CD$, $I$ là giao điểm của $AC$ và $BD$. Gọi $M$ là trung điểm của $SD$, $E$ là trung điểm đoạn $CM$ và $G$ là điểm đối xứng của $E$ qua $M$, $SE$ cắt $CD$ tại $K$. Chứng minh $(IKE) \parallel (ADG)$.
	\loigiai{
		\immini{Do $CE=ME=MG$ nên \begin{align}\label{2.1}
				CE=\dfrac{1}{3} CG. \tag{1}
			\end{align}
			Mặt khác $$\begin{cases} \widehat{BAI}&=\widehat{DCI},\text{ (so le trong)}, \\ \widehat{AIB}&=\widehat{CID},\text{ (đối đỉnh)}. \end{cases}$$
			Do đó $\triangle ABI \backsim \triangle CDI$, (g-g).
			Khi đó \begin{align}\label{2.2}
				\dfrac{CI}{IA}=\dfrac{CD}{AB}=\dfrac{1}{2}\hspace{5pt}\text{hay}\hspace{5pt} \dfrac{CI}{CA}=\dfrac{1}{3}.\tag{2}
			\end{align}
			Từ \eqref{2.1} và \eqref{2.2} suy ra 
			\begin{align}\label{*}
				EI \parallel GA.\tag{$\star$}
		\end{align}}{
			\begin{tikzpicture}[smooth]
				\tkzDefPoints{0/0/C, 3/0/D, -1/2.5/B, 5/2.5/A, 1.5/7/S}
				\tkzInterLL(A,C)(B,D)\tkzGetPoint{I}
				\tkzDefMidPoint(S,D)\tkzGetPoint{M}
				\tkzDefMidPoint(C,M)\tkzGetPoint{E}
				\tkzDefPointBy[symmetry=center M](E)\tkzGetPoint{G}
				\tkzInterLL(S,E)(C,D)\tkzGetPoint{K}
				\tkzInterLL(S,A)(G,C)\tkzGetPoint{N}
				\tkzDrawPoints(A,B,C,S,D,I,K,E,M,G)
				\tkzDrawSegments(S,B S,C S,D S,N B,C C,D A,D G,C S,K G,D G,A)
				\tkzDrawSegments[dashed](A,B N,A A,C B,D I,E I,K)
				\tkzLabelPoints[below left](C)
				\tkzLabelPoints[below right](D)
				\tkzLabelPoints[below](K)
				\tkzLabelPoints[left](B,M,E)
				\tkzLabelPoints[right](A)
				\tkzLabelPoints[above](S,G,I)
			\end{tikzpicture}
	}
Hơn nữa, tứ giác $SGDE$ có $SM=MD$ và $EM=MG$, nên tứ giác $SGDE$ là hình bình hành. Do đó 
\begin{align}\label{**}
	SE \parallel GD\hspace{5pt}\text{hay}\hspace{5pt} EK \parallel GD.\tag{$\star\star$}
\end{align}
Từ \eqref{*} và \eqref{**} suy ra $(IEK) \parallel (ADG)$.}
	
\end{bt}


\begin{bt}%[1H2K4-2]
	Cho tứ diện $ ABCD $. Gọi $ G_1 $, $ G_2 $, $ G_3 $ lần lượt là trọng tâm của các tam giác $ ABC $, $ ACD $, $ ADB $.	 Chứng minh $ (G_1G_2G_3)\parallel (BCD) $.
	
	\loigiai{
		\begin{center}
			\begin{tikzpicture}[scale=0.75, font=\footnotesize, line join=round, line cap=round, >=stealth]
				\tkzDefPoints{4/9/A, 0/3/B, 5/1/C, 7/3/D}
				\tkzDefMidPoint(B,C)\tkzGetPoint{M}
				\tkzDefMidPoint(D,C)\tkzGetPoint{N}
				\tkzDefMidPoint(D,B)\tkzGetPoint{L}
				\tkzCentroid(A,B,C)\tkzGetPoint{G1}
				\tkzCentroid(A,D,C)\tkzGetPoint{G2}
				\tkzCentroid(A,B,D)\tkzGetPoint{G3}
				\tkzDefLine[parallel=through G1](C,B) \tkzGetPoint{X}
				\tkzInterLL(G1,X)(A,B)\tkzGetPoint{E}
				\tkzInterLL(G1,X)(C,A)\tkzGetPoint{F}
				\tkzInterLL(G2,F)(D,A)\tkzGetPoint{G}
				\tkzDrawSegments(A,B B,C C,D D,A A,C E,F F,G A,M A,N)
				\tkzDrawSegments[dashed](B,D L,M M,N N,L A,L E,G)
				\tkzLabelPoints[right](G,D,N,F)
				\tkzLabelPoints[left](E,B)
				\tkzLabelPoints[above](A)
				\tkzLabelPoints[below](C,L,M)
				\node[below left] at (G1) {$ G_1 $};
				\node[above left] at (G2) {$ G_2 $};
				\node[above right] at (G3) {$ G_3 $};
				\tkzDrawPoints[fill=black](A,B,C,D,M,N,L,E,F,G,G1,G2,G3)
			\end{tikzpicture}
		\end{center}
	 Chứng minh $ (G_1G_2G_3)\parallel (BCD) $\\
			Gọi $ M $, $ N $, $ L $ lần lượt là trung điểm các cạnh $ BC $, $ CD $ và $ BD $. Trong tam giác $ AMN $, ta có
			$$ \dfrac{AG_1}{AM}=\dfrac{AG_2}{AN}=\dfrac{G_1G_2}{MN}=\dfrac{2}{3} (\text{tính chất trọng tâm}) $$
			Theo định lý Ta-lét đảo, suy ra $ G_1G_2\parallel MN $.\\
			Chứng minh tương tự, ta cũng có $ G_2G_3\parallel NL $ và $ G_3G_1\parallel LM $.\\
			Từ đó suy ra $$ \heva{&G_1G_2\parallel MN, G_2G_3\parallel NL\\&MN, NL\subset(BCD)\\&G_1G_2,G_2G_3\subset(G_1G_2G_3).} $$
			$ \Rightarrow (G_1G_2G_3)\parallel (BCD) $.\\
			}
\end{bt}

\begin{bt}%[1H2K4-2]
	Cho hình chóp $SABC$ có $G$ là trọng tâm tam giác $ABC$. Trên đoạn $SA$ lấy hai điểm $M$, $N$ sao cho $SM=MN=NA$.
	\begin{listEX}[1]
		\item Chứng minh rằng $GM\parallel (SBC)$. 
		\item Gọi $D$ là điểm đối xứng với $A$ qua $G$. Chứng minh rằng $(MCD)\parallel (NBG)$.
		\end{listEX} 
	\loigiai{
		\begin{center}
			\begin{tikzpicture}[>=stealth=0.5, line join=round, line cap = round,scale=0.7]
				\tkzDefPoints{0/0/A,1.5/-2/B,8/0/C,1.5/6/S}
				\tkzDefMidPoint(B,C)\tkzGetPoint{E}
				\tkzDefPointBy[homothety=center A ratio 2/3](E)\tkzGetPoint{G}
				\tkzDefPointBy[homothety=center A ratio 2/3](S)\tkzGetPoint{M}
				\tkzDefPointBy[homothety=center A ratio 1/3](S)\tkzGetPoint{N}
				\tkzDefPointBy[homothety=center A ratio 2](G)\tkzGetPoint{D}
				\tkzInterLL(D,M)(S,E)\tkzGetPoint{H}
				\tkzInterLL(D,H)(C,E)\tkzGetPoint{X}
				\tkzDrawSegments(A,B B,E S,A S,B S,C S,E D,E C,D B,N D,H)
				\tkzDrawSegments[dashed](A,C A,E G,M M,C B,G G,N H,M E,X E,C)
				%Gán nhãn
				\tkzDrawPoints[fill=black](A,B,C,S,E,G,M,N,H,D)
				\tkzLabelPoints[left](A,M,N)
				\tkzLabelPoints[below](B,G,E)
				\tkzLabelPoints[right](C,D,H)
				\tkzLabelPoints[above](S)
			\end{tikzpicture}
		\end{center}
		\begin{enumerate}
			\item Gọi $E$ là trung điểm của $BC$. Khi đó ta có $\dfrac{AG}{AE}=\dfrac{AM}{AS}=\dfrac{2}{3}\Rightarrow GM \parallel SE$. Vậy $GM \parallel (SBC)$.
			\item Từ giả thiết ta suy ra $G,N$ lần lượt là trung điểm của $AD$ và $AM$. Do đó $NG \parallel MD \quad (1)$ \hfill $(1)$\\
			Từ giác $BDCG$ có $E$ là trung điểm của hai đường chéo nên đó là hình bình hành. Suy ra $BG\parallel CD$ \hfill $(2)$\\
			Từ $(1)$ và $(2)$ suy ra $(MCD)\parallel (NBG)$.
		\end{enumerate}
	}
\end{bt}
\begin{bt}
	Cho hình hộp $ABCD.A'B'C'D'$  . Một mặt phẳng song song với mặt đáy $(ABCD)$  của hình hộp và cắt các cạnh $AA', BB', CC', DD'$  lần lượt tại $M, N, M', N'$ . Chứng minh rằng $ABCD.MNM'N'$  là hình hộp.
	\loigiai{
	}
\end{bt}
% \subsection{BÀI TẬP TRẮC NGHIỆM}
\Opensolutionfile{ans}[ans/1H4.B4]
\setcounter{ex}{0}

\begin{ex}%[1H2B4]
	Cho đường thẳng $d$ song song với mặt phẳng $(\alpha  )$. Có bao nhiêu mặt phẳng đi qua $d$ và song song với $(\alpha  )$?
	\choice
	{\True $1$}
	{$0$}
	{$2$}
	{Vô số}
	\loigiai{}
\end{ex}

\begin{ex}%[1H2B4]
	Trong các điều kiện sau, điều kiện nào kết luận mặt phẳng $(\alpha  )$ song song với mặt phẳng $(\beta )$?
	\choice{$(\alpha )\parallel (\gamma )$ và $(\beta )\parallel (\gamma )$ (với $(\gamma )$ là mặt phẳng nào đó)}
	{$(\alpha )\parallel a$ và $(\alpha )\parallel b$ với $a$, $b$ là hai đường thẳng phân biệt thuộc $(\beta )$}
	{$(\alpha )\parallel a$ và $(\alpha )\parallel b$ với $a$, $b$ là hai đường thẳng phân biệt cùng song song với $(\beta )$}
	{\True $(\alpha )\parallel a$ và $(\alpha )\parallel b$ với $a$, $b$ là hai đường thẳng cắt nhau thuộc $(\beta )$} 
\end{ex}

\begin{ex}%[1H2B4]
	Cho các mệnh đề sau:
	\begin{listEX}[1]
		\item [\ding{172}] Hai mặt phẳng phân biệt cùng song song với một đường thẳng thì chúng song song với nhau.
		\item [\ding{173}] Hai mặt phẳng cùng song song với một mặt phẳng thứ ba thì chúng song song với  nhau.
		\item [\ding{174}] Bất kì đường thẳng nào cắt một trong hai mặt phẳng song song thì nó cũng cắt mặt phẳng còn lại.
	\end{listEX}
	Số mệnh đề \textbf{sai} là
	\choice{$0$}
	{$1$}
	{\True $2$}
	{$3$}
	\loigiai{
		Mệnh đề đúng là $(3)$. Mệnh đề $(2)$ \textbf{sai} vì hai mặt phẳng đó có thể trùng nhau.
	}
\end{ex}

\begin{ex}%[1H2K4-6]
	Trong các mệnh đề sau. Mệnh đề \textbf{sai} là
	\choice
	{Hai mặt phẳng song song với nhau thì mọi đường thẳng nằm trong mặt phẳng này đều song song
		với mặt phẳng kia}
	{\True Hai mặt phẳng cùng song song với một mặt phẳng thì song song với nhau}
	{Một mặt phẳng cắt hai mặt phẳng song song cho trước theo hai giao tuyến thì hai giao tuyến song
		song với nhau}
	{Hai mặt phẳng song song thì không có điểm chung}
	\loigiai{
		Hai mặt phẳng \textit{phân biệt} cùng song song với một mặt phẳng thì song song với nhau.
	}
\end{ex}


\begin{ex}%[1H2B4-1]
	Cho mặt phẳng $(R)$ cắt hai mặt phẳng song song $(P)$ và $(Q)$ theo hai giao tuyến $a$ và $b$. Mệnh đề nào sau đây đúng?
	\choice
	{$a$ và $b$ vuông góc nhau}
	{$a$ và $b$ chéo nhau}
	{$a$ và $b$ cắt nhau}
	{\True $a$ và $b$ song song}
	\loigiai{
		\begin{center}
			\begin{tikzpicture}[>=stealth,scale=.5]
				\clip (-2,-3) rectangle (10,8.5);
				\tkzDefPoints{0/0/B, 6/0/C, 9/2/D, 3/2/A}
				\coordinate (A') at ($(A)+(0,3)$);
				\coordinate (B') at ($(B)+(0,3)$);
				\coordinate (C') at ($(C)+(0,3)$);
				\coordinate (D') at ($(D)+(0,3)$);
				\coordinate (M) at ($(A)!0.55!(D)$);
				\coordinate (N) at ($(B)!0.55!(C)$);
				\coordinate (P) at ($(B')!0.55!(C')$);
				\coordinate (Q) at ($(A')!0.55!(D')$);
				\coordinate (M') at ($(M)-(0,3)$);
				\coordinate (N') at ($(N)-(0,3)$);
				\coordinate (P') at ($(P)+(0,3)$);
				\coordinate (Q') at ($(Q)+(0,3)$);
				\tkzInterLL(P',N')(A',D')
				\tkzGetPoint{K}
				\tkzInterLL(P',N')(A,D)
				\tkzGetPoint{K'}
				\tkzInterLL(Q',M')(A',D')
				\tkzGetPoint{L}
				\tkzInterLL(Q',M')(C',D')
				\tkzGetPoint{L1}
				\tkzInterLL(Q',M')(C,D)
				\tkzGetPoint{L2}
				\tkzInterLL(Q',M')(A,D)
				\tkzGetPoint{L'}
				\tkzDrawSegments(A,B B,C C,D D,L' A,K')
				\tkzDrawSegments[dashed](L',K')
				\tkzDrawSegments(A',B' B',C' C',D' D',L A',K)
				\tkzDrawSegments[dashed](K,L)
				\tkzDrawSegments(M',N' N',P' P',Q' Q',L M',L2 L1,L')
				\tkzDrawSegments[dashed](L,L1 L',L2)
				\tkzDrawSegments(M,N P,Q)
				\tkzLabelSegment[left](M,N){$a$}
				\tkzLabelSegment[left](P,Q){$b$}
				%\tkzLabelSegment[blue](B,N){$P$}
				%\tkzLabelSegment[red](B',P){$Q$}
				%\tkzLabelSegment[black](Q,P'){$R$}
				\tkzLabelAngle[pos=1.25,rotate=30](A',B',C'){\footnotesize $Q$}
				\tkzMarkAngle[size =1.6,opacity=1](C',B',A');
				\tkzLabelAngle[pos=1.25,rotate=30](A,B,C){\footnotesize $P$}
				\tkzMarkAngle[size =1.6,opacity=1](C,B,A);
				\tkzLabelAngle[pos=1.15,rotate=30](M',Q',P'){\footnotesize $R$}
				\tkzMarkAngle[size =1.6,opacity=1](P',Q',M');
			\end{tikzpicture}
		\end{center}
	}
\end{ex}

\begin{ex}%[1H2B4-1]
	Cho đường thẳng $a$ thuộc mặt phẳng $(P)$ và đường thẳng $b$ thuộc mặt phẳng $(Q)$. Mệnh đề nào sau đây đúng?
	\choice
	{\True $(P)\parallel (Q)\Rightarrow a\parallel (Q)$ và $b\parallel (P)$}
	{$a$ và $b$ chéo nhau}
	{$(P)\parallel (Q)\Rightarrow a\parallel b$}
	{$a\parallel b\Rightarrow (P)\parallel (Q)$}
	\loigiai
	{
		$(P)\parallel (Q)$ suy ra $(P)$ và $(Q)$ không có điểm chung. Mặt khác $a\in (P)$ nên $a$ và $(Q)$ cũng không có điểm chung. Suy ra $a\parallel (Q)$. Tương tự ta cũng có $b\parallel (P)$.
	}
\end{ex}

\begin{ex} Hình lăng trụ tam giác có tất cả bao nhiêu cạnh?
	\choice
	{$6$}
	{\True $9$}
	{$12$}
	{$3$}
	\loigiai{
		\begin{itemize}
			\item [$\bullet$] Lăng trụ tam giác, có hai đáy. Mỗi mặt đáy có ba cạnh, suy ra có 6 cạnh
			\item [$\bullet$] Mặt khác, chúng có 3 cạnh bên.
		\end{itemize}
		Vậy, có tất cả là 9 cạnh.
	}
\end{ex}

\begin{ex}%[1H2B4]
	Đặc điểm nào sau đây là đúng với hình lăng trụ?
	\choice{Đáy của hình lăng trụ là hình bình hành}
	{Hình lăng trụ có tất cả các mặt song song với nhau}
	{\True Hình lăng trụ có tất cả các mặt bên là hình bình hành}
	{Hình lăng trụ có tất cả các mặt là hình bình hành}
	\loigiai{
	}
\end{ex}

\begin{ex}%[1H2B4-2]
	Cho hình hộp $ABCD.A'B'C'D'$. Mặt phẳng $(AB'D')$ song song với mặt phẳng nào sau đây?
	\choice
	{$(BCA')$}
	{$(BDA')$}
	{\True $(BDC')$}
	{$(A'C'C)$}
	\loigiai{
		\immini{
			Mặt phẳng $(AB'D')$ song song với mặt phẳng $(BDC')$.\\
			Thật vậy, ta có $AB'\parallel DC'$ và $AD'\parallel BC'$, có điều cần chứng minh.
		}{
			\begin{tikzpicture}
				\tkzInit[xmin=-1.5,ymin=-2,xmax=5,ymax=4]
				\tkzClip
				\tkzDefPoints{0/0/B,1/-1/A,3/0/C,1/2/A'}
				\tkzDefPointBy[ translation = from B to C](A)\tkzGetPoint{D}
				\tkzDefPointBy[ translation = from A to A'](B)\tkzGetPoint{B'}
				\tkzDefPointBy[ translation = from A to A'](C)\tkzGetPoint{C'}
				\tkzDefPointBy[ translation = from A to A'](D)\tkzGetPoint{D'}
				\tkzDrawSegments(B,A A,D A',B' B',C' C',D' D',A' A,A' B,B' D,D' A,B' A,D' B',D')
				\tkzDrawSegments[dashed](C,C' C,D C,B C',B B,D D,C')
				\tkzLabelPoints[right](D,C,D',C')
				\tkzLabelPoints[left](B,B',A)
				\tkzLabelPoints[above](A')
				\tkzDrawPoints(A,B,C,D,A',B',C',D')
			\end{tikzpicture}
		}
	}
\end{ex}

\begin{ex}%[1H2B4]
	Cho hai hình bình hành $ABCD$ và $ABEF$ không thuộc cùng một mặt phẳng, có cạnh chung $AB$. Kết quả nào sau đây đúng?
	\choice{ $BC \parallel (AEF)$}
	{$FD \parallel (BEF)$}
	{$(CEF)\parallel (ABD)$}
	{\True $(AFD)\parallel (BCE)$}
\end{ex}

\begin{ex}%[1H2B4] 
	Cho hình chóp $S.ABCD$ có đáy là hình thang $(AB \parallel CD)$ và $AB=2CD$. Gọi $I, J$ lần lượt là trung điểm $SB$ và $AB$. Mặt phẳng nào song song với mặt phẳng $(SAD)$?
	\choice{$(SJC)$}
	{$(ICB)$}
	{$(IJB)$}
	{\True $(IJC)$}
\end{ex}

\begin{ex}%[1H2B4] 
	Trong mặt phẳng $(P)$ cho hình bình hành $ABCD$, qua $A$, $B$, $C$, $D$  lần lượt vẽ bốn đường thẳng $a$, $b$, $c$, $d$  đôi một song song với nhau và không nằm trên $(P)$. Mặt phẳng song song với mặt phẳng $(b,c)$ là
	\choice{$(a,b)$}
	{$(a,c)$}
	{\True $(a,d)$}
	{$(d,b)$}
\end{ex}

\begin{ex}%[1H2Y4-2]
\immini[thm]{Cho hình hộp $ABCD.A'B'C'D'$. Mệnh đề nào sau đây là \textbf{sai}?
	\choice
	{$(ABCD)\parallel (A'B'C'D')$}
	{$(ABB'A')\parallel (CDD'C')$}
	{$(AA'D'D)\parallel (BCC'B')$}
	{\True $(BDD'B')\parallel (ACC'A')$}}{
	\begin{tikzpicture}[line width=0.8pt, scale=0.5]
		\path
		coordinate (B) at (0,0)
		coordinate (C) at (7,0)
		coordinate (D) at (8.5,2.5)
		coordinate (A) at (1.5,2.5)
		coordinate (B') at (0,5)
		coordinate (C') at (7,5)
		coordinate (D') at (8.5,7.5)
		coordinate (A') at (1.5,7.5)
		;
		\draw [dashed] (D)--(A)--(A') (A)--(B);
		\draw (A')--(B')--(C')--(D')--(A') (C)--(C') (D)--(D') (C)--(D) (C)--(B)--(B');
		\draw (A') node[above] {$A'$};
		\draw (D') node[above] {$D'$};
		\draw (C') node[right] {$C'$};
		\draw (B') node[left]{$B'$};
		\draw (A) node[left] {$A$};
		\draw (D) node[right] {$D$};
		\draw (C) node[right] {$C$};
		\draw (B) node[left] {$B$};
\end{tikzpicture}}
	\loigiai{
		\immini{Ta thấy $\heva{&(ABCD)\parallel (A'B'C'D')\\&(AA'D'D)\parallel (BCC'B')\\&(ABB'A')\parallel (CDD'C')}$ luôn đúng.\\
			và hai mặt phẳng $(BDD'B')$, $(ACC'A')$ là cắt nhau. }{
			\begin{tikzpicture}[line width=0.8pt, scale=0.5]
				\path
				coordinate (B) at (0,0)
				coordinate (C) at (7,0)
				coordinate (D) at (8.5,2.5)
				coordinate (A) at (1.5,2.5)
				coordinate (B') at (0,5)
				coordinate (C') at (7,5)
				coordinate (D') at (8.5,7.5)
				coordinate (A') at (1.5,7.5)
				;
				\draw [dashed] (D)--(A)--(A') (A)--(B);
				\draw (A')--(B')--(C')--(D')--(A') (C)--(C') (D)--(D') (C)--(D) (C)--(B)--(B');
				\draw (A') node[above] {$A'$};
				\draw (D') node[above] {$D'$};
				\draw (C') node[right] {$C'$};
				\draw (B') node[left]{$B'$};
				\draw (A) node[left] {$A$};
				\draw (D) node[right] {$D$};
				\draw (C) node[right] {$C$};
				\draw (B) node[left] {$B$};
		\end{tikzpicture}}
	}
\end{ex}

\begin{ex}%[1H2B4-2]
	Cho hình chóp $S.ABCD$ có đáy là một hình bình hành. Gọi $A'$, $B'$, $C'$, $D'$ lần lượt là trung điểm của các cạnh $SA,$ $SB,$ $SC,$ $SD.$ Tìm mệnh đề đúng trong các mệnh đề sau.
	\choice
	{$A'C' \parallel BD$}
	{$A'B' \parallel (SAD)$}
	{\True $(A'C'D') \parallel (ABC)$}
	{$A'B' \parallel (SBD)$}
	\loigiai{
		\immini{
			Ta có $A'C'\parallel AC \Rightarrow (A'C'D') \parallel (ABC).$
		}{\begin{tikzpicture}[scale=0.6,every node/.style={scale=0.6}]%hình chóp S.ABCD
				\tkzDefPoints{0/0/A, -2/-2/B, 3/-2/C, -1/4/S}
				\coordinate (D) at ($(A)+(C)-(B)$);
				\tkzDrawSegments[dashed](S,A A,B A,D)
				\tkzDrawPolygon(S,C,D)
				\tkzDefMidPoint(S,A) \tkzGetPoint{A'}
				\tkzDefMidPoint(S,B) \tkzGetPoint{B'}
				\tkzDefMidPoint(S,C) \tkzGetPoint{C'}\\
				\tkzDefMidPoint(S,D) \tkzGetPoint{D'}
				\tkzDrawSegments(S,B B,C)
				\tkzDrawPoints[fill=black](A,B,A',B',C,D,C',D',S)
				\tkzLabelPoints[left](A,B,A',B')
				\tkzLabelPoints[right](C,D,C',D')
				\tkzLabelPoints[above](S)
		\end{tikzpicture}}
	}
\end{ex}

\begin{ex}%[1H2B4-2]
	Cho hình chóp $S.ABCD$, có đáy $ABCD$ là hình bình hành tâm $O$. Gọi $M,N$ lần lượt là trung điểm $SA,SD$. Mặt phẳng $\left(OMN\right)$ song song với mặt phẳng nào sau đây?
	\choice
	{$\left(ABCD\right)$}
	{$\left(SCD\right)$}
	{\True $\left(SBC\right)$}
	{$\left(SAB\right)$}
	\loigiai{
		\immini{
			Vì $ABCD$ là hình bình hành nên $O$ là trung điểm $AC,BD$.\\
			Do đó $MO \parallel SC\Rightarrow MO\parallel \left(SBC\right)$\\
			Và $NO \parallel SB \Rightarrow NO\parallel \left(SBC\right)$\\
			Suy ra $\left(OMN\right)\parallel \left(SBC\right)$.}{\begin{tikzpicture}[line cap=round,line join=round, >=stealth,scale=.7]
				\def \xa{-2}
				\def \xb{-1}
				\def \y{4}
				\def \z{4}
				\coordinate (A) at (0,0);
				\coordinate (B) at ($(A)+(\xa,\xb)$);
				\coordinate (D) at ($(A)+(\y,0)$);
				\coordinate (C) at ($ (B)+(D)-(A) $);
				\coordinate (K) at ($ (A)!1/2!(B) $);
				\coordinate (S) at ($ (K)+(0,\z) $);
				\coordinate (M) at ($ (S)!1/2!(A) $);
				\coordinate (N) at ($ (S)!1/2!(D) $);
				\coordinate (O) at ($ (B)!1/2!(D) $);
				\draw [dashed] (B)--(A)--(D)--(B) (C)--(A)--(S) (O)--(M)--(N)--(O);
				\draw (S)--(B)--(C)--(D)--(S)--(C);
				\tkzDrawPoints(S,A,B,C,D,M,N)
				\tkzLabelPoint[right](D){$ D $}
				\tkzLabelPoints[below right](C)
				\tkzLabelPoints[above](S)
				\tkzLabelPoints[above left](M)
				\tkzLabelPoints[below](O)
				\tkzLabelPoints[above right](N)
				\tkzLabelPoints[above left](A)
				\tkzLabelPoint[below left](B){$ B $}
			\end{tikzpicture}
	}}
\end{ex}

% \centerline{---HẾT---}
\Closesolutionfile{ans}


%%Bài 14. Phép chiếu song song
% \setcounter{section}{13}
\setcounter{dang}{0}
\section{PHÉP CHIẾU PHẲNG SONG SONG}
\subsection{KIẾN THỨC CẦN NHỚ}
\subsubsection{ĐỊNH NGHĨA}
Cho mặt phẳng $(\alpha)$ và đường thẳng $\Delta$ cắt $(\alpha)$. Với mỗi điểm $M$ trong không gian ta xác định điểm $M'$ như sau:
\immini{\begin{itemize}
		\item Nếu $M$ thuộc $\Delta$ thì $M'$ là giao điểm của $\Delta$ và $(\alpha)$.
		\item Nếu $M$ không thuộc $\Delta$ thì $M'$ là giao điểm của $(\alpha)$ và đường thẳng qua $M$ song song $\Delta$.
		\item Điểm $M'$ gọi là hình chiếu song song của $M$ trên $(\alpha)$ theo phương $\Delta$.
		\item Phép đặt tương ứng mối điểm $M$ với hình chiếu $M'$ của nó được gọi là \indamm{phép chiếu song song} lên $(\alpha)$ theo phương $\Delta$.
		\item Mặt phẳng $(\alpha)$ gọi là mặt phẳng chiếu; phương $\Delta$ gọi là \indamm{phương chiếu.}
	\end{itemize}}{
\begin{tikzpicture}[scale=0.9]
	\tkzInit[xmin=-3.5,xmax=5,ymin=-3, ymax=4] \tkzClip[space=.1]
	\tkzDefPoints{-3/0/A, 3/0/B, 4/2/C, -2/2/D, 0/1/Q, -1/2/P, 0/4/M, 3/1/x}
	%\tkzDrawCircle(O,A)
	\tkzDefLine[parallel=through M,K=1](P,Q)\tkzGetPoint{N}
	\tkzInterLL(M,N)(Q,x) \tkzGetPoint{M'}
	\tkzInterLL(A,D)(P,Q) \tkzGetPoint{y}
	\tkzInterLL(A,B)(P,Q) \tkzGetPoint{y'}
	\tkzInterLL(M,M')(B,C) \tkzGetPoint{t}
	\tkzDefLine[parallel=through y',K=1](P,Q)\tkzGetPoint{z}
	\tkzDefLine[parallel=through t,K=1](P,Q)\tkzGetPoint{t'}
	%\tkzInterLC(C,M)(O,A) \tkzGetPoints{E}{C}
	%\tkzInterLC(A,B)(A,C) \tkzGetPoints{y}{P}
	%\tkzInterLC(C,P)(O,A) \tkzGetPoints{C}{Q}
	%\tkzTangent[at=Q](O) \tkzGetPoint{x}
	\tkzDrawSegments(A,B B,C C,D D,A P,Q M,N P,y N,M' y',z t,t')
	\tkzDrawSegments[dashed](Q,y' M',t)
	%\tkzDrawLine[add = 1 and 1](Q,x)
	\tkzDrawPoints[fill=white](M,M')
	\tkzLabelSegment[above right](P,y){$\Delta$}
	%\tkzLabelPoints[above](C)
	%\tkzLabelPoint[left](M)
	\tkzLabelPoints[below left](M,M')
	\tkzMarkAngle[size=0.6cm,opacity=.5](B,A,D)
	\tkzLabelAngle[pos =0.4](D,A,B){$\alpha$}
	%\tkzLabelPoints[below left](O,P)
\end{tikzpicture}}
\subsubsection{TÍNH CHẤT}
Phép chiếu song song có các tính chất sau:
\begin{itemize}
	\item[\ding{172}] Biến ba điểm thẳng hàng thành ba điểm thẳng hàng.
	\item[\ding{173}] Biến đường thẳng  thành đường thẳng , biến tia thành tia, đoạn thẳng thành đoạn thẳng.
	\item[\ding{174}] Biến hai đường thẳng song song thành hai đường thẳng song song  hoặc trùng nhau.
	\item[\ding{175}] Giữ nguyên tỉ số độ dài của hai đoạn thẳng cùng nằm trên một đường thẳng hoặc nằm trên hai đường thẳng song song.
\end{itemize}

\subsubsection{HÌNH BIỂU DIỄN CỦA MỘT HÌNH KHÔNG GIAN}
\begin{itemize}
	\item[\ding{172}] Hình biểu diễn của hình trong không gian là hình chiếu song song của hình đó trên một mặt phẳng theo một phương chiếu nào đó hoặc hình đồng dạng với hình chiếu đó.
	\item[\ding{173}] Hình biểu diễn của một hình không gian ( trong trường hợp hình phẳng nằm trong mặt phẳng không song song với phương chiếu) có các tính chất sau:
		\begin{itemize}
			\item Hình biểu diễn của một tam giác là một tam giác.
			\item Hình biểu diễn của hình chữ nhật, hình vuông, hình thoi, hình bình hành là hình bình hành.
			\item Hình biểu diễn của hình thang $ABCD$ với $AB\parallel CD$ là một hình thang $A'B'C'D'$ với $A'B'\parallel C'D'$ thoả mãn $\dfrac{AB}{CD}=\dfrac{A'B'}{C'D'}$.
			\item Hình biểu diễn của hình tròn là hình elip.
		\end{itemize}
\end{itemize}

% \subsection{PHÂN LOẠI, PHƯƠNG PHÁP GIẢI TOÁN}
\begin{dang}{Xác định ảnh của một hình qua phép chiếu song song}
\end{dang}
\begin{vd}
	Cho hình hộp $ABCD.A'B'C'D'$.
	\begin{itemize}
		\item [a)] Xác định ảnh của các điểm $A', B', C', D'$ qua phép chiếu song song lên mặt phẳng $(ABCD)$ theo phương $AA'$.
		\item [b)] Xác định ảnh của tam giác $A'C'D'$ qua phép chiếu song song lên mặt phẳng $(ABCD)$ theo phương $A'B$.
	\end{itemize}
	\loigiai{
		\begin{itemize}
			\item 	Vì các cạnh $AA',BB', CC', DD'$ song song nhau nên $A, B, C, D$ là hình chiếu của $A', B', C', D'$.
			\item 
		\end{itemize}
	
	}
\end{vd}
\begin{vd}
	Phép chiếu song song biến hình bình hành $ABCD$ thành hình bình hành $A'B'C'D'$. Chứng minh rằng phép chiếu đó biến tâm của hình bình hành $ABCD$ thành tâm của hình bình hành $A'B'C'D'$.
	\loigiai{
		Gọi $O$ là tâm hình bình hành $ABCD$, suy ra $O$ là trung điểm của $AC$.
		\\Phép chiếu song song biến $O$ thành $O'$.\\
		Ta có $A, O, C $ thẳng hàng theo thứ tự đó và $\dfrac{OA}{OC}=1$ nên ba điểm $A', O', C'$ thẳng hàng theo thứ tự đó và $\dfrac{O'A'}{O'C'}=1$. Suy ra $O'$ là trung điểm của $A'C'$.\\
		Vậy $O'$ là tâm của hình bình hành $A'B'C'D'$.
	}
\end{vd}
\begin{vd}
	Phép chiếu song song biến tam giác $ABC$  thành tam giác $A'B'C'$ . Chứng minh rằng phép chiếu đó biến đường trung bình của tam giác $ABC$  thành đường trung bình của tam giác $A'B'C'$  .
	\loigiai{
	Gọi $M ,N$   lần lượt là trung điểm $AB ,AC$   nên  $MN$ là đường trung bình của tam giác $ABC$.\\
	Phép chiếu song song biến $M$  thành $M'$ , biến $N$  thành $N'$ .\\
	Ta có ba điểm  $A, M , B$  thẳng hàng theo thứ tự đó và $\dfrac{AM}{AB}=\dfrac{1}{2}$  nên ba điểm $A' ,M'  , B'$  thằng hàng theo thứ tự đó và $\dfrac{A'M'}{A'B'}=\dfrac{1}{2}$ . Suy ra $M'$   là trung điểm $A'B'$ .\\
	Tương tự $N'$  là trung điểm $A'C'$ .\\
	Vậy $M'N'$  là đường trung bình của tam giác $A'B'C'$ .
	
}
\end{vd}
\begin{dang}{Vẽ hình biểu diễn của một số hình khối đơn giản}
\end{dang}
\begin{vd}
	Vẽ hình biểu diễn của các hình sau
\begin{itemize}
	\item [a)] Hình lục giác đều.
	\item [b)] Hình vuông nội tiếp trong hình tròn.
\end{itemize}
\end{vd}
\begin{vd}
	Vẽ hình biểu diễn của hình chóp $S.ABCD$ có đáy là hình thang $ABCD$ với $AB$ song song $CD$; $AB=2$ cm, $CD=6$ cm.
	\loigiai{
	\immini{Vì hình chóp $S.ABCD$ có $AB \parallel CD$ và $AB=2$ cm, $CD=6$ cm nên hình biễu diễn của hình chóp cũng có đáy $AB \parallel CD$ và $\dfrac{AB}{CD}=\dfrac{2}{6}=\dfrac{1}{3}$.\\
	Do đó, ta vẽ hình thang $ABCD$ có $AB \parallel CD$ và $AB=\dfrac{1}{3} CD$, vẽ điểm $S$ và nối $SA$, $SB$, $SC$, $SD$.}{
	\begin{tikzpicture}[scale=0.8, font=\footnotesize,>=stealth]
		\path
		%	Vẽ mp
		(0,0) coordinate (A)
		(2,0) coordinate (B)
		(5,1.5) coordinate (C)
		(-1,1.5) coordinate (D)
		(1,3) coordinate (S)
		;
		\draw[line width=0.1pt,gray!20!white] (-1.5,-0.5) grid (5.6,3.6);
		\draw (A)--(B)--(C)--(S)--(D)--(A)--(S) (C)--(S)--(B);
		\draw[dashed] (D)--(C);
		\foreach \x/\g in {A/-90,B/-90,C/0,D/180,S/90}\draw[fill=black] (\x) circle (.05) +(\g:.5)node{\footnotesize$\x$};
	\end{tikzpicture}
}
}

\end{vd}
\begin{vd}
	Vẽ hình biểu diễn của các hình sau
	\begin{itemize}
		\item [a)] Hình lăng trụ có đáy là tam giác đều.
		\item [b)] Hình lăng trụ có đáy là lục giác đều.
		\item [c)] Hình hộp.
		\item [d)] Hình chóp tam giác $S.ABC$ đặt trên một hình lăng trụ tam giác $ABC.A'B'C'$.
	\end{itemize}
\end{vd}

% 
\subsection{BÀI TẬP TRẮC NGHIỆM}

\Opensolutionfile{ans}[ans/1H4.B5]
\setcounter{ex}{0}
\begin{ex}
	Cho hình hộp $ABCD.A'B'C'D'$  . Gọi $M, M'$   lần lượt là trung điểm của các cạnh $BC, B'C'$  Hình chiếu của $\Delta B'DM$  qua phép chiếu song song trên $(A'B'C'D')$  theo phương chiếu $AA'$   là
	\choice
	{$\Delta B'A'M'$}
		{$\Delta C'D'M'$}
			{$\Delta DMM'$}
				{\True$\Delta B'D'M'$}
\end{ex}
\begin{ex}
		Cho hình hộp $ABCD.A'B'C'D'$  . Gọi $M, M'$   lần lượt là trung điểm của các cạnh $BC, B'C'$  Hình chiếu của $\Delta D'CM$  qua phép chiếu song song trên $(A'B'C'D')$  theo phương chiếu $BB'$   là
	\choice
	{$\Delta B'CM'$}
	{\True$\Delta C'D'M'$}
	{$\Delta DMM'$}
	{$\Delta B'D'M'$}
\end{ex}
\begin{ex}
	Cho hình hộp $ABCD.A'B'C'D'$  . Gọi $M, M'$   lần lượt là trung điểm của các cạnh $AD, A'D'$; $N, N'$ lần lườ là trung điểm của các cạnh $CD, C'D'$; $P$ là trung điểm của $DD'$.  Hình chiếu của $\Delta MNP$  qua phép chiếu song song trên $(A'B'C'D')$  theo phương chiếu $BB'$   là
	\choice
	{$\Delta B'N'M'$}
	{\True$\Delta D'M'N'$}
	{$\Delta PM'N'$}
	{$\Delta PD'M'$}
\end{ex}
\begin{ex}
	Trong các mệnh đề sau, có bao nhiêu mệnh đề đúng?
	\begin{itemize}
		\item [a)] Một đường thẳng có thể song song với hình chiếu của nó.
		\item [b)] Một đường thẳng có thể trùng với hình chiếu của nó.
		\item [c)] Hình chiếu song song của hai đường thẳng chéo nhau có thể song song với nhau.
		\item [d)]Hình chiếu song song của hai đường thẳng chéo nhau có thể trung nhau.
	\end{itemize}
\choice
{$1$}
{$2$}
{$3$}
{$4$}
\end{ex}
\begin{ex}
	Trong các mệnh đề sau, có bao nhiêu mệnh đề đúng?
	\begin{itemize}
		\item [a)] Phép chiếu song song biến đoạn thẳng thành đoạn thẳng.
		\item [b)] Phép chiếu song song biến hai đường thẳng song song thành hai đường thẳng cắt nhau.
		\item [c)] Phép chiếu song song biến tam giác đều thành tam giác cân.
		\item [d)] Phép chiếu song song biến hình vuông thành hình bình hành.
	\end{itemize}
	\choice
	{$1$}
	{$2$}
	{$3$}
	{$4$}
\end{ex}

\begin{ex}%[1H2K5]
	Hình chiếu của tứ diện $ABCD$ lên một mặt phẳng $(P)$ theo phương chiếu $AB$ ($AB$ không song song với $(P)$ là
	\choice{\True hình tam giác}
	{hình tứ giác}
	{đoạn thẳng}
	{hình thang}
\end{ex}

\begin{ex}%[1H2K5]
	Hình nào dưới đây \textbf{không phải} là hình biểu diễn của một tứ diện?
	\def\dotEX{}
	\choice{
		\begin{tikzpicture}[scale=0.3]
			\draw (3.,6.)-- (0.,0.);
			\draw (0.,0.)-- (10.,0.);
			\draw (3.,6.)-- (10.,0.);
			\draw [dashed](3.,6.)-- (4.,1.);
			\draw [dashed](4.,1.)-- (0.,0.);
			\draw [dashed](4.,1.)-- (10.,0.);
		\end{tikzpicture}	
	}{
		\begin{tikzpicture}[scale=0.3]
			\draw (3.,6.)-- (0.,0.);
			\draw (0.,0.)-- (10.,0.);
			\draw (3.,6.)-- (10.,0.);
			\draw (3.,6.)-- (4.,1.);
			\draw (4.,1.)-- (0.,0.);
			\draw (4.,1.)-- (10.,0.);
		\end{tikzpicture}	
	}{
		\begin{tikzpicture}[scale=0.3]
			\draw (3.,6.)-- (0.,0.);
			\draw (0.,0.)-- (10.,0.);
			\draw (3.,6.)-- (10.,0.);
			\draw (3.,6.)-- (4.,0.);
		\end{tikzpicture}	
	}{
		\True
		\begin{tikzpicture}[scale=0.2]
			\draw (4.,8.)-- (0.,0.);
			\draw (4.,8.)-- (12.,0.);
			\draw (0.,0.)-- (3.,-2.);
			\draw (3.,-2.)-- (12.,0.);
			\draw (4.,8.)-- (3.,-2.);
			\draw (0.,0.)-- (12.,0.);
		\end{tikzpicture}	
	}
\end{ex}



\begin{ex}%[Võ Đông Phước]%[1D1G2]
	Cho hình lăng trụ tam giác $ABC.A'B'C'$. Gọi $M, M'$ lần lượt là trung điểm của các cạnh $BC, B'C'$ và $I$ là giao điểm của đường thẳng $A'M$ và $(AB'C')$. Tìm hình chiếu song song của $I$ trên $(A'B'C')$ theo phương $BB'$.
	\choice{\True Trung điểm của đoạn thẳng $A'M'$}
	{Trọng tâm của tam giác $A'B'C'$}
	{Điểm $A'$}
	{Điểm $M'$}
\end{ex}
%%%%%%%%%%%%%%%%%%%%%%%%%%%%%%%%%%%%%%%%
\begin{ex}%[Võ Đông Phước]%[1D1G2]
	Cho tứ diện $ABCD$. Gọi $M, N$ lần lượt là trung điểm của các cạnh $AC, BC$, trên cạnh $BD$ lấy điểm $P$ sao cho $BP=2PD$. Mặt phẳng $(MNP)$ cắt mặt phẳng $(ACD)$ theo giao tuyến $d$. Tìm hình chiếu song song của đường thẳng $d$ trên $(BCD)$ theo phương $AD$.
	\choice{Đường thẳng $DN$}
	{\True Đường thẳng $CD$}
	{Đường thẳng $BD$}
	{Điểm $M$}
\end{ex}

\begin{ex}%[1H2G5]
	Cho tứ diện $ABCD$ và $M$ là điểm bất kì thuộc miền trong của tam giác $BCD$. Gọi $B'$, $C'$, $D'$ lần lượt là hình chiếu song song của $M$ theo các phương $AB$, $AC$, $AD$ lên các mặt $(ACD)$, $(ABD)$, $(ABC)$. Tính $\dfrac{MB'}{AB}+\dfrac{MC'}{AC}+\dfrac{MD'}{AD}$.
	\choice{\True $1$}
	{$\dfrac{1}{9}$}
	{$\dfrac{1}{3}$}
	{$3$}
	\loigiai{
		\immini{
			Trong tam giác $ABI$ ta có $\dfrac{MB'}{AB}=\dfrac{MI}{BI}$.\\
			Tương tự ta cũng có $\dfrac{MC'}{AC}=\dfrac{MJ}{CJ}$ và $\dfrac{MD'}{AD}=\dfrac{MK}{DK}$.\\
			Dễ thấy rằng $\dfrac{S_{MBD}}{S_{CBD}}=\dfrac{MJ}{CJ}$, $\dfrac{S_{MCD}}{S_{BCD}}=\dfrac{MI}{BI}$, $\dfrac{S_{MBC}}{S_{DBC}}=\dfrac{MK}{DK}$.\\
			Cộng các đẳng thức với nhau vế theo vế ta được $$\dfrac{MB'}{AB}+\dfrac{MC'}{AC}+\dfrac{MD'}{AD}=1$$
		}{
			\begin{tikzpicture}[scale=0.5]
				\draw [dashed] (2.,5.)-- (10.,5.);
				\draw [line width=1.2pt] (2.,5.)-- (4.,2.);
				\draw [line width=1.2pt] (4.,2.)-- (10.,5.);
				\draw [line width=1.2pt] (3.,11.)-- (2.,5.);
				\draw [line width=1.2pt] (3.,11.)-- (4.,2.);
				\draw [line width=1.2pt] (3.,11.)-- (10.,5.);
				\draw [line width=1.2pt] (3.,11.)-- (7.,3.5);
				\draw [dashed] (2.,5.)-- (7,3.5);
				\draw [dashed] (10.,5.)-- (3.,3.5);
				\draw [dashed] (4.,2.)-- (6.,5.);
				\draw [dashed] (5.4,4)-- (5.7,6);
				\begin{scriptsize}
					\draw (2.14,5.33) node[left] {$B$};
					\draw (10.14,5.33) node {$D$};
					\draw (4.14,2.1) node [below]{$C$};
					\draw (3.14,11.33) node {$A$};
					\draw (7.3,3.83) node [below right]{$I$};
					\draw (3,3.81) node [below left]{$K$};
					\draw (6.14,5.33) node [right]{$J$};
					\draw (5.5,4.1) node [below]{$M$};
					\draw (5.82,6.31) node [above]{$B'$};
				\end{scriptsize}
			\end{tikzpicture}
		}
		
	}	
	
\end{ex}
\centerline{---HẾT---}
\Closesolutionfile{ans}

%Chương V
%%Bài 15. Giới hạn dãy số
\chap{Giới hạn. Hàm số liên tục}
\section{Giới hạn dãy số}
\subsection{Tóm tắt lý thuyết}
\subsubsection{Dãy số có giới hạn $0$}
	\begin{dn}
		Ta nói dãy số $\left(u_{n}\right)$ có giới hạn là $0$ khi $n$ dần tới dương vô cực, nếu $\left|u_{n}\right|$ có thể nhỏ hơn một số dương bé tuỳ ý, kể từ một số hạng nào đó trở đi, kí hiệu $\lim \limits_{n \rightarrow+\infty} u_{n}=0$ hay $u_{n} \rightarrow 0$ khi $n \rightarrow+\infty$.
	\end{dn}
	Từ định nghĩa dãy số có giới hạn $0$, ta có các kết quả sau:
	\begin{itemize}
		\item $\lim \limits_{n \rightarrow+\infty} \dfrac{1}{n^{k}}=0$ với $k$ là một số nguyên dương;	
		\item $\lim \limits_{n \rightarrow+\infty} q^{n}=0$ nếu $|q|<1$;	
		\item Nếu $\left|u_{n}\right| \leq v_{n}$ với mọi $n \geq 1$ và $\lim \limits_{n \rightarrow+\infty} v_{n}=0$ thì $\lim \limits_{n \rightarrow+\infty} u_{n}=0$.	
	\end{itemize}
	\subsubsection{Dãy số có giới hạn hữu hạn}
	\begin{dn}
		Ta nói dãy số $\left(u_{n}\right)$ có giới hạn là số thực a khi $n$ dần tới dương vô cực nếu $$\lim \limits_{n \rightarrow+\infty}\left(u_{n}-a\right)=0,$$ kí hiệu $\lim \limits_{n \rightarrow+\infty} u_{n}=a$ hay $u_{n} \rightarrow a$ khi $n \rightarrow+\infty$. 
	\end{dn}
	\begin{itemize}
		\item Nếu $u_{n}=c$ (c là hằng số) thì $\lim \limits_{n \rightarrow+\infty} u_{n}=c$. 
		\item $\lim \limits_{n \rightarrow+\infty} u_{n}=a$ khi và chỉ khi $\lim \limits_{n \rightarrow+\infty}\left(u_{n}-a\right)=0$.
	\end{itemize}
	\subsubsection{Các quy tắc tính giới hạn}
	\begin{tc} 
		\begin{enumEX}[a)]{1}
			\item Nếu $\lim \limits_{n \rightarrow+\infty} u_{n}=a$ và $\lim \limits_{n \rightarrow+\infty} v_{n}=b$ thì 
			\begin{enumEX}[-)]{2}
				\item 	$\lim \limits_{n \rightarrow+\infty}\left(u_{n}+v_{n}\right)=a+b$.
				\item   $\lim \limits_{n \rightarrow+\infty}\left(u_{n}-v_{n}\right)=a-b$. 
				\item $\lim \limits_{n \rightarrow+\infty}\left(u_{n} \cdot v_{n}\right)=a \cdot b$.
				\item $\lim \limits_{n \rightarrow+\infty} \dfrac{u_{n}}{v_{n}}=\dfrac{a}{b}$ (nếu $b \neq 0$).
			\end{enumEX}
			\item Nếu $u_{n} \geq 0$ với mọi $n$ và $\lim \limits_{n \rightarrow+\infty} u_{n}=a$ thì $
			a \geq 0 \text { và } \lim \limits_{n \rightarrow+\infty} \sqrt{u_{n}}=\sqrt{a}$.
		\end{enumEX}
		
	\end{tc}
\subsection{Các dạng toán thường gặp}
\begin{dang}{Phương pháp đặt thừa số chung (lim hữu hạn)}
	
\end{dang}
\subsubsection{Ví dụ minh hoạ}
\begin{vd}%[1C3Y1-2]%[Anh Duy]%Ví dụ 1.
	Tìm giới hạn sau $\lim\dfrac{2n^3-2n+3}{1-4n^3}$.
	\loigiai{
		\[\lim\dfrac{2n^3-2n+3}{1-4n^3}=\lim\dfrac{2-\dfrac{2}{n^2}+\dfrac{3}{n^3}}{\dfrac{1}{n^3}-4}=-\dfrac{1}{2}.\]}
\end{vd}
\begin{vd}%[1C3Y1-2]%[Anh Duy]%Ví dụ 2.
	Tìm giới hạn sau $\lim\dfrac{\sqrt{n^4+2n+2}}{n^2+1}$.
	\loigiai{
		\[\lim\dfrac{\sqrt{n^4+2n+2}}{n^2+1} = \lim\dfrac{\sqrt{1+\dfrac{2}{n^3}+\dfrac{2}{n^4}}}{1+\dfrac{1}{n^2}}=1.\]}
\end{vd}
\begin{vd}%[1C3Y1-2]%[Anh Duy]%Ví dụ 3.
	Tìm giới hạn sau $\lim\dfrac{3^{n+1}-4^n}{4^{n-1}+3}$.
	\loigiai{
		\[\lim\dfrac{3^{n+1}-4^n}{4^{n-1}+3} = \lim\dfrac{9\cdot 3^{n-1}-4\cdot 4^{n-1}}{4^{n-1}+3}=\lim\dfrac{9\cdot\left(\dfrac{3}{4}\right)^{n-1}-4}{1+3\cdot\left(\dfrac{1}{4}\right)^{n-1}}=-4.\]}
\end{vd}

\begin{vd}%[1C3B1-2]%[Anh Duy]%Ví dụ 4.
	Tìm giới hạn sau $\lim\dfrac{1+2+2^2+\cdots +2^n}{1+3+3^2+\cdots +3^n}$.
	\loigiai{
		\[\lim\dfrac{1+2+2^2+\cdots +2^n}{1+3+3^2+\cdots +3^n} = \lim\dfrac{\dfrac{1-2^{n+1}}{-1}}{\dfrac{1-3^{n+1}}{-2}} = \lim\dfrac{\left(1-2^{n+1}\right)\cdot 2}{1-3^{n+1}} = \lim\dfrac{\left(\left(\dfrac{1}{3}\right)^{n+1}-\left(\dfrac{2}{3}\right)^{n+1}\right)\cdot 2}{\left(\dfrac{1}{3}\right)^{n+1}-1}=0.\]}
\end{vd}
% \subsubsection{Bài tập rèn luyện}
% % \subsubsection{Bài tập tự luận}
% \begin{bt}%[1C3B1-2]%[Anh Duy]
% 	Tìm các giới hạn sau
% 	\begin{enumEX}[a)]{2}
% 		\item[a)] $\lim \limits_{n \rightarrow+\infty} \dfrac{n^{2}+n+1}{2 n^{2}+1}$.
% 		\item[b)] $\lim \limits_{n \rightarrow+\infty}\left(\sqrt{n^{2}+2 n}-n\right)$.
% 	\end{enumEX}
% 	\loigiai{
% 		\begin{enumEX}[a)]{1}
% 			\item $\lim \limits_{n \rightarrow+\infty} \dfrac{n^{2}+n+1}{2 n^{2}+1}=\lim \limits_{n \rightarrow+\infty} \dfrac{1+\dfrac{1}{n}+\dfrac{1}{n^2}}{2+\dfrac{1}{n^2}}=\dfrac{\lim \limits_{n \rightarrow+\infty} \left(1+\dfrac{1}{n}+\dfrac{1}{n^2}\right)}{\lim \limits_{n \rightarrow+\infty} \left(2+\dfrac{1}{n^2}\right)}=\dfrac{1}{2}$.
% 			\item $\lim \limits_{n \rightarrow+\infty}\left(\sqrt{n^{2}+2 n}-n\right)=\lim \limits_{n \rightarrow+\infty}\dfrac{n^2+2n-n^2}{\sqrt{n^{2}+2 n}+n}=\lim \limits_{n \rightarrow+\infty} \dfrac{2}{\sqrt{1+\dfrac{2}{n}}+1}=\dfrac{2}{\lim \limits_{n \rightarrow+\infty} \left(\sqrt{1+\dfrac{2}{n}}+1\right)}=1$.
% 		\end{enumEX}
% 	}
% \end{bt}

% \begin{bt}%[1C3B1-2]%[Anh Duy]
% 	Cho hai dãy số không âm $\left(u_{n}\right)$ và $\left(v_{n}\right)$ với $\lim \limits_{n \rightarrow+\infty} u_{n}=2$ và $\lim \limits_{n \rightarrow+\infty} v_{n}=3$. Tìm các giới hạn sau
% 	\begin{enumEX}[a)]{2}
% 		\item[a)] $\lim \limits_{n \rightarrow+\infty} \dfrac{u_{n}^{2}}{v_{n}-u_{n}}$;
% 		\item[b)] $\lim \limits_{n \rightarrow+\infty} \sqrt{u_{n}+2 v_{n}}$.
% 	\end{enumEX}
% 	\loigiai{
% 		\begin{enumEX}[a)]{1}
% 			\item $\lim \limits_{n \rightarrow+\infty} \dfrac{u_{n}^{2}}{v_{n}-u_{n}} = \dfrac{\lim \limits_{n \rightarrow+\infty} u_{n}^{2}}{\lim \limits_{n \rightarrow+\infty} v_{n}-\lim \limits_{n \rightarrow+\infty} u_{n}} = \dfrac{\left(\lim \limits_{n \rightarrow+\infty} u_{n}\right)^{2}}{\lim \limits_{n \rightarrow+\infty} v_{n}-\lim \limits_{n \rightarrow+\infty} u_{n}}=\dfrac{2^2}{3-2}=4$ ;
% 			\item $\lim \limits_{n \rightarrow+\infty} \sqrt{u_{n}+2 v_{n}} =  \sqrt{\lim \limits_{n \rightarrow+\infty}  u_{n}+\lim \limits_{n \rightarrow+\infty} 2 v_{n}} =\sqrt{\lim \limits_{n \rightarrow+\infty}  u_{n}+2\lim \limits_{n \rightarrow+\infty}  v_{n}} =\sqrt{2+2\cdot 3}=2\sqrt{2}$.
% 		\end{enumEX}
% 	}
% \end{bt}
% \begin{bt}%[1C3B1-2]%[Anh Duy]
% 	Tính các giới hạn sau:
% 	\begin{enumEX}{2}
% 		\item $\lim \limits_{n \to +\infty}\dfrac{2^n+3\cdot 4^n}{4^n-5\cdot 3^n}$.
% 		\item $T=\lim\dfrac{3\cdot7^n+2\cdot 4^n}{4\cdot 5^n+7^n}$.
% 	\end{enumEX}
% 	\loigiai{
% 		\begin{enumEX}{1}
% 			\item $\lim \limits_{n \to +\infty}\dfrac{2^n+3\cdot 4^n}{4^n-5\cdot 3^n}=\lim \limits_{n \to +\infty}\dfrac{\left(\dfrac {1} {2}\right)^n+3}{1-5\left(\dfrac {3} {4}\right)^n}=3$.
% 			\item Ta có $T=\lim\dfrac{3\cdot7^n+2\cdot 4^n}{4\cdot 5^n+7^n}=\lim\dfrac{3+2\cdot\left(\dfrac{4}{7}\right)^n}{4\cdot\left(\dfrac{5}{7}\right)^n+1}=3$.
% 		\end{enumEX}
% 	}
% \end{bt}
\subsubsection{Câu hỏi trắc nghiệm}
\Opensolutionfile{ans}[ans/ans-1K5-1-Dang1]
\begin{ex}%[1C3Y1-2]%[Anh Duy]%Câu 1.
	Tính giới hạn $I=\lim\dfrac{2n+2023}{3n+2024}$. 
	\choice
	{\True $I=\dfrac{2}{3}$}
	{$I=\dfrac{3}{2}$}
	{$I=\dfrac{2023}{2024}$}
	{$I=1$}
	\loigiai{
		Ta có $I=\lim\dfrac{2n+2023}{3n+2024} =\lim\dfrac{2+\dfrac{2017}{n}}{3+\dfrac{2018}{n}} =\dfrac{2}{3}$.}
\end{ex}
\begin{ex}%[1C3Y1-1]%[Anh Duy]%Câu 2.
	Phát biểu nào sau đây là \textbf{sai}?
	\choice
	{$\lim \limits_{n \to +\infty}u_n=c$ ($u_n=c$ là hằng số)}
	{\True $\lim \limits_{n \to +\infty}q^n=0 \;(|q|>1)$}
	{$\lim\dfrac{1}{n}=0$}
	{$\lim\dfrac{1}{n^k}=0 \; (k>1)$}
	\loigiai{
		Theo định nghĩa giới hạn hữu hạn của dãy số thì $\lim \limits_{n \to +\infty}q^n=0 \; (|q|<1)$.}
\end{ex}
\begin{ex}%[1C3Y1-2]%[Anh Duy]%Câu 3.
	Giá trị của $\lim\dfrac{2-n}{n+1}$ bằng
	\choice
	{$1$}
	{$2$}
	{\True $-1$}
	{$0$}
	\loigiai{
		Ta có $\lim\dfrac{2-n}{n+1} =\lim\dfrac{\dfrac{2}{n}-1}{1+\dfrac{1}{n}} =\dfrac{0-1}{1+0} =-1$.}
\end{ex}
\begin{ex}%[1C3Y1-2]%[Anh Duy]%Câu 4.
	Tính giới hạn $\lim\dfrac{4n+2024}{2n+1}$. 
	\choice
	{$\dfrac{1}{2}$}
	{$4$}
	{\True $2$}
	{$2024$}
	\loigiai{
		Ta có $\lim\dfrac{4n+2024}{2n+1}=\lim\dfrac{4+\dfrac{2024}{n}}{2+\dfrac{1}{n}}=2$.}
\end{ex}
\begin{ex}%[1C3Y1-2]%[Anh Duy]%Câu 5.
	$\lim\dfrac{2n^2-3}{n^6+5n^5}$ bằng 
	\choice
	{$2$}
	{\True $0$}
	{$\dfrac{-3}{5}$}
	{$-3$}
	\loigiai{
		Ta có $\lim\dfrac{2n^2-3}{n^6+5n^5} =\lim\dfrac{\dfrac{2}{n^4}-\dfrac{3}{n^6}}{1+\dfrac{5}{n}} =0$.}
\end{ex}
\begin{ex}%[1C3Y1-2]%[Anh Duy]%Câu 6.
	Tính $\lim\dfrac{2n+1}{1+n}$ được kết quả là
	\choice
	{\True $2$}
	{$0$}
	{$\dfrac{1}{2}$}
	{$1$}
	\loigiai{
		Ta có $\lim\dfrac{2n+1}{1+n}=\lim\dfrac{n\left(2+\dfrac{1}{n}\right)}{n\left(\dfrac{1}{n}+1\right)}=\lim\dfrac{2+\dfrac{1}{n}}{\dfrac{1}{n}+1}=\dfrac{2+0}{0+1}=2$.}
\end{ex}

\begin{ex}%[1C3Y1-2]%[Anh Duy]%Câu 7.
	Dãy số nào sau đây có giới hạn khác $0$?
	\choice
	{$\dfrac{1}{n}$}
	{$\dfrac{1}{\sqrt{n}}$}
	{\True $\dfrac{n+1}{n}$}
	{$\dfrac{\sin n}{\sqrt{n}}$}
	\loigiai{
		Có $\lim\dfrac{n+1}{n}=\lim \limits_{n \to +\infty}1+\lim\dfrac{1}{n}=1$.}
\end{ex}

\begin{ex}%[1C3B1-2]%[Anh Duy]%Câu 8.
	Giới hạn $\lim\dfrac{\sqrt{n}}{2n^2+3}$ có kết quả là 
	\choice
	{$2$}
	{\True $0$}
	{$+\infty$}
	{$4$}
	\loigiai{
		$\lim\dfrac{\sqrt{n}}{2n^2+3}=\lim\dfrac{\sqrt{\dfrac{1}{n^3}}}{2+\dfrac{3}{n^2}}=\dfrac{0}{2+0}=0$.}
\end{ex}
\begin{ex}%[1C3Y1-1]%[Anh Duy]%Câu 9.
	Dãy số $(u_n)$ với $u_n=\dfrac{1}{2n}$, chọn $M=\dfrac{1}{100}$, để $\dfrac{1}{2n}<\dfrac{1}{100}$ thì $n$ phải lấy từ số hạng thứ bao nhiêu trở đi?
	\choice
	{\True $51$}
	{$49$}
	{$48$}
	{$50$}
	\loigiai{Ta có $\dfrac{1}{2n}<\dfrac{1}{100}\Leftrightarrow 2n>100\Leftrightarrow n>50$.\\
		Vậy $n$ phải lấy từ số hạng thứ $51$ trở đi.}
\end{ex}
\begin{ex}%[1C3B1-2]%[Anh Duy]%Câu 10.
	Giới hạn $\lim\dfrac{3^n+2^n}{4^n}$ có kết quả là 
	\choice
	{\True $0$}
	{$\dfrac{5}{4}$}
	{$\dfrac{3}{4}$}
	{$+\infty$}
	\loigiai{
		Ta có $\lim\dfrac{3^n+2^n}{4^n}=\lim\dfrac{\left(\dfrac{3}{4}\right)^n+\left(\dfrac{2}{4}\right)^n}{1}=0$.}
\end{ex}
\begin{ex}%[1C3B1-2]%[Anh Duy]%Câu 11.
	Tính giới hạn $\lim\left[\dfrac{1}{1\cdot 2}+\dfrac{1}{2\cdot 3}+\dfrac{1}{3\cdot 4}+\cdots +\dfrac{1}{n(n+1)}\right]$. 
	\choice
	{$0$}
	{$2$}
	{\True $1$}
	{$\dfrac{3}{2}$}
	\loigiai{
		Ta có $\dfrac{1}{1\cdot 2}+\dfrac{1}{2\cdot 3}+\dfrac{1}{3\cdot 4}+\cdots +\dfrac{1}{n(n+1)} =\dfrac{1}{1}-\dfrac{1}{2}+\dfrac{1}{2}-\dfrac{1}{3}+\cdots+\dfrac{1}{n-1}-\dfrac{1}{n}+\dfrac{1}{n}-\dfrac{1}{n+1} =1-\dfrac{1}{n+1}$.\\
		Vậy $\lim\left[\dfrac{1}{1\cdot 2}+\dfrac{1}{2\cdot 3}+\dfrac{1}{3\cdot 4}+\cdots +\dfrac{1}{n(n+1)}\right] =\lim\left(1-\dfrac{1}{n+1}\right)=1$.}
\end{ex}

\begin{ex}%[1C3K1-2]%[Anh Duy]%Câu 12.
	Tính $\lim\sqrt{\dfrac{1^2+2^2+3^2+\cdots +n^2}{2n(n+7)(6n+5)}}$. 
	\choice
	{\True $\dfrac{1}{6}$}
	{$\dfrac{1}{2\sqrt{6}}$}
	{$\dfrac{1}{2}$}
	{$+\infty$}
	\loigiai{
		Ta có $1^2+2^2+3^2+\cdots +n^2=\dfrac{n(n+1)(2n+1)}{6}$.\\
		Khi đó $\lim\sqrt{\dfrac{1^2+2^2+3^3+\cdots +n^2}{2n(n+7)(6n+5)}}=\lim\sqrt{\dfrac{n(n+1)(2n+1)}{12n(n+7)(6n+5)}} =\lim\sqrt{\dfrac{\left(1+\dfrac{1}{n}\right)\left(2+\dfrac{1}{n}\right)}{12\left(1+\dfrac{7}{n}\right)\left(6+\dfrac{5}{n}\right)}} =\dfrac{1}{6}$.}
\end{ex}
\begin{ex}%[1C3K1-2]%[Anh Duy]%Câu 13.
	Giới hạn $\lim\dfrac{(2n-1)(3-n)^2}{(4n-5)^3}$ có kết quả bằng 
	\choice
	{$0$}
	{\True $\dfrac{1}{32}$}
	{$\dfrac{3}{2}$}
	{$\dfrac{1}{2}$}
	\loigiai{
		$\lim\dfrac{(2n-1)(3-n)^2}{(4n-5)^3}=\lim\dfrac{\left(2-\dfrac{1}{n}\right)\left(\dfrac{3}{n}-1\right)^2}{\left(4-\dfrac{5}{n}\right)^3}=\dfrac{2}{4^3}=\dfrac{1}{32}$.}
\end{ex}
\begin{ex}%[1C3K1-2]%[Anh Duy]%Câu 14.
	Tìm $L=\lim\left(\dfrac{1}{1}+\dfrac{1}{1+2}+\cdots +\dfrac{1}{1+2+\cdots +n}\right)$.
	\choice
	{$L=\dfrac{5}{2}$}
	{$L=+\infty$}
	{\True $L=2$}
	{$L=\dfrac{3}{2}$}
	\loigiai{
		Ta có $1+2+3+\cdots +k$ là tổng của cấp số cộng có $u_1=1$, $d=1$ nên $1+2+3+\cdots +k=\dfrac{(1+k)k}{2}$. Khi đó
		\[\dfrac{1}{1+2+\cdots +k}=\dfrac{2}{k(k+1)} =\dfrac{2}{k}-\dfrac{2}{k+1},\; \forall k\in\mathbb{N}^*.\]
		Suy ra	\[L=\lim\left(\dfrac{2}{1}-\dfrac{2}{2}+\dfrac{2}{2}-\dfrac{2}{3}+\dfrac{2}{3}-\dfrac{2}{4}+\cdots +\dfrac{2}{n}-\dfrac{2}{n+1}\right) =\lim\left(\dfrac{2}{1}-\dfrac{2}{n+1}\right) =2.\]}
\end{ex}
%--------------------------------------------------------------------------------------------------

\begin{ex}%[1C3G1-2]%[Anh Duy]%Câu 15.
	Đặt $f(n)=\left(n^2+n+1\right)^2+1$.
	Xét dãy số $(u_n)$ sao cho $u_n=\dfrac{f(1)\cdot f(3)\cdot f(5)\cdots f(2n-1)}{f(2)\cdot f(4)\cdot f(6)\cdots f(2n)}$. Tính $\lim \limits_{n \to +\infty}n\sqrt{u_n}$. 
	\choice
	{$\lim \limits_{n \to +\infty}n\sqrt{u_n}=\sqrt{2}$}
	{$\lim \limits_{n \to +\infty}n\sqrt{u_n}=\dfrac{1}{\sqrt{3}}$}
	{$\lim \limits_{n \to +\infty}n\sqrt{u_n}=\sqrt{3}$}
	{\True $\lim \limits_{n \to +\infty}n\sqrt{u_n}=\dfrac{1}{\sqrt{2}}$}
	\loigiai{
		Xét $g(n)=\dfrac{f(2n-1)}{f(2n)}\Rightarrow g(n)=\dfrac{\left(4n^2-2n+1\right)^2+1}{\left(4n^2+2n+1\right)^2+1}$.\\
		$g(n)=\dfrac{\left(4n^2+1\right)^2-4n\left(4n^2+1\right)+\left(4n^2+1\right)}{\left(4n^2+1\right)^2+4n\left(4n^2+1\right)+\left(4n^2+1\right)}=\dfrac{4n^2+1-4n+1}{4n^2+1+4n+1}=\dfrac{(2n-1)^2+1}{(2n+1)^2+1}$ \\
		$ \Rightarrow u_n=\dfrac{2}{10}\cdot\dfrac{10}{26}\cdot\dfrac{26}{50}\cdots\cdot\dfrac{(2n-3)^2+1}{(2n-1)^2+1}\cdot\dfrac{(2n-1)^2+1}{(2n+1)^2+1}=\dfrac{2}{(2n+1)^2+1} $ \\
		$ \Rightarrow\lim \limits_{n \to +\infty}n\sqrt{u_n}=\lim\sqrt{\dfrac{2n^2}{4n^2+4n+2}}=\dfrac{1}{\sqrt{2}} $.}
\end{ex}
\begin{ex}%[1C3G1-2]%[Anh Duy]%Câu 16.
	Có bao nhiêu giá trị nguyên của tham số $a$ thuộc khoảng $(0;2024)$ để có
	\[ \lim\sqrt{\dfrac{9^n+3^{n+1}}{5^n+9^{n+a}}}\leq\dfrac{1}{2187}\,? \]
	\choice
	{\True $2017$}
	{$2016$}
	{$2023$}
	{$2024$}
	\loigiai{
		Do $\dfrac{9^n+3^{n+1}}{5^n+9^{n+a}}>0$ với $\forall n$ nên $\lim\sqrt{\dfrac{9^n+3^{n+1}}{5^n+9^{n+a}}}=\sqrt{\lim\dfrac{9^n+3^{n+1}}{5^n+9^{n+a}}} =\sqrt{\lim\dfrac{1+3\cdot\left(\dfrac{1}{3}\right)^n}{\left(\dfrac{5}{9}\right)^n+9^a}} =\sqrt{\dfrac{1}{9^a}} =\dfrac{1}{3^a}$.\\
		Theo đề bài ta có $\lim\sqrt{\dfrac{9^n+3^{n+1}}{5^n+9^{n+a}}}\leq\dfrac{1}{2187}\Leftrightarrow\dfrac{1}{3^a}\leq\dfrac{1}{2187}\Leftrightarrow a\geq 7$.\\
		Do $a$ là số nguyên thuộc khoảng $(0;2024)$ nên có $a\in\left\{7;8;9;\ldots;2023\right\}\Rightarrow$ có $2017$ giá trị của $a$.}
\end{ex}
\Closesolutionfile{ans}
% \begin{indapan}{10}
% 	{ans/ans-1K5-1-Dang1}
% \end{indapan}
\begin{dang}{Phương pháp lượng liên hợp (lim hữu hạn)}
	Nếu giới hạn của dãy số ở dạng vô định thì ta sử dụng các phép biến đổi để đưa về dạng cơ bản. \\
	Một số phép biến đổi liên hợp: \\
	\begin{align*}
		f(n) - g(n) &= \dfrac{(f(n))^2 - (g(n))^2}{f(n) + g(n)} \\
		\sqrt{f(n)} - \sqrt{g(n)} &= \dfrac{f(n) - g(n)}{\sqrt{f(n)} + \sqrt{g(n)}} \\
		\sqrt{f(n)} - g(n) &= \dfrac{f(n) - (g(n))^2}{\sqrt{f(n)} + g(n)} \\
		\sqrt[3]{f(n)} - \sqrt[3]{g(n)} &= \dfrac{f(n) - g(n)}{\sqrt[3]{(f(n))^2} + \sqrt[3]{f(n)g(n)} + \sqrt[3]{(g(n))^2}}
	\end{align*}
	
\end{dang}
\subsubsection{Ví dụ minh hoạ}
\begin{vd}%[Dự án soạn đề cương toán 11 - KNTT, Minh Trí]%[1K5BE-3]
	Tính giới hạn $I = \lim \limits_{n \to +\infty}\left(\sqrt{n^2 - 2n + 3} - n\right)$. 
	\loigiai{
		Ta có 
		\begin{align*}
			I &= \lim \limits_{n \to +\infty}\left(\sqrt{n^2 - 2n + 3} - n\right) \\
			&= \lim \limits_{n \to +\infty}\dfrac{n^2 - 2n + 3 - n^2}{\sqrt{n^2 - 2n + 3} + n} \\
			&= \lim \limits_{n \to +\infty}\dfrac{- 2n + 3}{\sqrt{n^2 - 2n + 3} + n} \\
			&= \lim \limits_{n \to +\infty}\dfrac{- 2 + \dfrac{3}{n}}{\sqrt{1 - \dfrac{2}{n} + \dfrac{3}{n^2}} + 1} \\
			&= \dfrac{- 2}{\sqrt{1} + 1} = - 1
		\end{align*}
	}
\end{vd} 
\begin{vd}%[Dự án soạn đề cương toán 11 - KNTT, Minh Trí]%[1K5BE-3]
	Tính giới hạn $I = \lim \limits_{n \to +\infty}\left(\sqrt{n^2 + 7} - \sqrt{n^2 + 5}\right)$. 
	\loigiai{
		Ta có 
		\begin{align*}
			I &= \lim \limits_{n \to +\infty}\left(\sqrt{n^2 + 7} - \sqrt{n^2 + 5}\right) \\
			&= \lim \limits_{n \to +\infty}\dfrac{n^2 + 7 - (n^2 + 5)}{\sqrt{n^2 + 7} + \sqrt{n^2 + 5}} \\
			&= \lim \limits_{n \to +\infty}\dfrac{2}{\sqrt{n^2 + 7} + \sqrt{n^2 + 5}} \\
			&= 0
		\end{align*}
	}
\end{vd}
\begin{vd}%[Dự án soạn đề cương toán 11 - KNTT, Minh Trí]%[1K5KE-3]
	Tính giới hạn $I = \lim \limits_{n \to +\infty}\left(\sqrt{n^2 + 2n} - \sqrt{n^2 - 2n}\right)$. 
	\loigiai{
		Ta có 
		\begin{align*}
			I &= \lim \limits_{n \to +\infty}\left(\sqrt{n^2 + 2n} - \sqrt{n^2 - 2n}\right) \\
			&= \lim \limits_{n \to +\infty}\dfrac{n^2 + 2n - (n^2 - 2n)}{\sqrt{n^2 + 2n} + \sqrt{n^2 - 2n}} \\
			&= \lim \limits_{n \to +\infty}\dfrac{4n}{\sqrt{n^2 + 2n} + \sqrt{n^2 - 2n}} \\
			&= \lim \limits_{n \to +\infty}\dfrac{4}{\sqrt{1 + \dfrac{2}{n}} + \sqrt{1 - \dfrac{2}{n}}} \\
			&= \dfrac{4}{\sqrt{1} + \sqrt{1}} = 2
		\end{align*}
	}
\end{vd}

\begin{vd}%[Dự án soạn đề cương toán 11 - KNTT, Minh Trí]%[1K5KE-3]
	Tính giới hạn $I = \lim \limits_{n \to +\infty}\left(\sqrt{2n^2 - n + 1} - \sqrt{2n^2 - 3n + 2}\right)$. 
	\loigiai{
		Ta có 
		\begin{align*}
			I &= \lim \limits_{n \to +\infty}\left(\sqrt{2n^2 - n + 1} - \sqrt{2n^2 - 3n + 2}\right) \\
			&= \lim \limits_{n \to +\infty}\dfrac{2n^2 - n + 1 - (2n^2 - 3n + 2)}{\sqrt{2n^2 - n + 1} + \sqrt{2n^2 - 3n + 2}} \\
			&= \lim \limits_{n \to +\infty}\dfrac{2n - 1}{\sqrt{2n^2 - n + 1} + \sqrt{2n^2 - 3n + 2}} \\
			&= \lim \limits_{n \to +\infty}\dfrac{2 - \dfrac{1}{n}}{\sqrt{2 - \dfrac{1}{n} + \dfrac{1}{n^2}} + \sqrt{2 - \dfrac{3}{n} + \dfrac{2}{n^2}}} \\
			&= \dfrac{2}{\sqrt{2} + \sqrt{2}} = \dfrac{1}{\sqrt{2}}
		\end{align*}
	}
\end{vd} 
\begin{vd}%[Dự án soạn đề cương toán 11 - KNTT, Minh Trí]%[1K5GE-3]
	Tính giới hạn $I = \lim \limits_{n \to +\infty}\left(n - \sqrt[3]{n^3 + 3n^2 + 1}\right)$.
	\loigiai{
		Ta có 
		\begin{align*} 
			I &= \lim \limits_{n \to +\infty}\left(n - \sqrt[3]{n^3 + 3n^2 + 1}\right) \\
			&= \lim \limits_{n \to +\infty}\dfrac{n^3 - (n^3 + 3n^2 + 1)}{n^2 + \sqrt[3]{n^3 + 3n^2 + 1} + \sqrt[3]{\left(n^3 + 3n^2 + 1\right)^2}} \\
			&= \lim \limits_{n \to +\infty}\dfrac{- 3n^2 - 1}{n^2 + \sqrt[3]{n^3 + 3n^2 + 1} + \sqrt[3]{\left(n^3 + 3n^2 + 1\right)^2}} \\
			&= \lim \limits_{n \to +\infty}\dfrac{- 3 - \dfrac{1}{n^2}}{1 + \sqrt[3]{1 + \dfrac{3}{n} + \dfrac{1}{n^3}} + \sqrt[3]{\left(1 + \dfrac{3}{n} + \dfrac{1}{n^3}\right)^2}} \\ 
			&= \dfrac{- 3}{1 + \sqrt[3]{1} + \sqrt[3]{1}} = - 1
		\end{align*}
	}
\end{vd}
% \subsubsection{Bài tập rèn luyện}
% % \subsubsection{Bài tập tự luận}
% \begin{bt}%[Dự án soạn đề cương toán 11 - KNTT, Minh Trí]%[1K5KE-3]
% 	Tính giới hạn  $I = \lim \limits_{n \to +\infty}\left(\sqrt[3]{n^3 - 2n} - n\right)$. 
% 	\loigiai{
% 		Ta có
% 		\begin{align*} 
% 			I &= \lim \limits_{n \to +\infty}\left(\sqrt[3]{n^3 - 2n} - n\right) \\
% 			&= \lim \limits_{n \to +\infty}\dfrac{n^3 - 2n - n^3}{\sqrt[3]{\left(n^3 - 2n\right)^2} + n \sqrt[3]{n^3 - 2n} + n^2} \\ 
% 			&= \lim \limits_{n \to +\infty}\dfrac{-2n}{\sqrt[3]{\left(n^3 - 2n\right)^2} + n \sqrt[3]{n^3 - 2n} + n^2} \\
% 			&= \lim \limits_{n \to +\infty}\dfrac{\dfrac{-2}{n}}{\sqrt[3]{\left(1 - \dfrac{2}{n^2}\right)^2} + \sqrt[3]{1 - \dfrac{2}{n^2}} + 1} \\
% 			&= 0 
% 		\end{align*}
% 	}
% \end{bt}
% \begin{bt}%[Dự án soạn đề cương toán 11 - KNTT, Minh Trí]%[1K5KE-3]
% 	Tính giới hạn $I = \lim \limits_{n \to +\infty}\left(\sqrt{4n^2 + 5n} - 2n\right)$.
% 	\loigiai{
% 		Ta có
% 		\begin{align*} 
% 			I &= \lim \limits_{n \to +\infty}\left(\sqrt{4n^2 + 5n} - 2n\right) \\
% 			&= \lim \limits_{n \to +\infty}\dfrac{4n^2 + 5n - 4n^2}{\sqrt{4n^2 + 5n} - 2n} \\ 
% 			&= \lim \limits_{n \to +\infty}\dfrac{5n}{\sqrt{4n^2 + 5n} + 2n} \\
% 			&= \lim \limits_{n \to +\infty}\dfrac{5}{\sqrt{4 + \dfrac{5}{n}} + 2} \\
% 			&= \dfrac{5}{\sqrt{4} + 2} = \dfrac{5}{4}
% 		\end{align*}
% 	}
% \end{bt}
% \begin{bt}%[Dự án soạn đề cương toán 11 - KNTT, Minh Trí]%[1K5KE-3]
% 	Tính giới hạn $I = \lim \limits_{n \to +\infty}\left(3n - \sqrt{9n^2 + 1}\right)$.
% 	\loigiai{
% 		Ta có
% 		\begin{align*} 
% 			I &= \lim \limits_{n \to +\infty}\left(3n - \sqrt{9n^2 + 1}\right) \\
% 			&= \lim \limits_{n \to +\infty}\dfrac{9n^2 - (9n^2 + 1)}{3n + \sqrt{9n^2 + 1}} \\ 
% 			&= \lim \limits_{n \to +\infty}\dfrac{- 1}{3n + \sqrt{9n^2 + 1}} \\
% 			&= 0
% 		\end{align*}
% 	}
% \end{bt}  
% \begin{bt}%[Dự án soạn đề cương toán 11 - KNTT, Minh Trí]%[1K5KE-3]
% 	Tính giới hạn $I = \lim \limits_{n \to +\infty}\left(3n - 5 - \sqrt{9n^2 + 1}\right)$.
% 	\loigiai{
% 		Ta có
% 		\begin{align*} 
% 			I &= \lim \limits_{n \to +\infty}\left(3n - \sqrt{9n^2 + 1}\right) \\
% 			&= \lim \limits_{n \to +\infty}\dfrac{(3n - 5)^2 - (9n^2 + 1)}{3n - 5 + \sqrt{9n^2 + 1}} \\ 
% 			&= \lim \limits_{n \to +\infty}\dfrac{- 30n + 24}{3n - 5 + \sqrt{9n^2 + 1}} \\
% 			&= \lim \limits_{n \to +\infty}\dfrac{- 30 + \dfrac{24}{n}}{3 - \dfrac{5}{n} + \sqrt{\dfrac{1}{n^2}}} \\
% 			&= \dfrac{- 30}{3 + \sqrt{9}} = - 5
% 		\end{align*}
% 	}
% \end{bt}  
% \begin{bt}%[Dự án soạn đề cương toán 11 - KNTT, Minh Trí]%[1K5KE-3]
% 	Tính giới hạn $I = \lim \limits_{n \to +\infty}\left(\sqrt[3]{n + 2} - \sqrt[3]{n}\right)$.
% 	\loigiai{
% 		Ta có
% 		\begin{align*} 
% 			I &= \lim \limits_{n \to +\infty}\left(\sqrt[3]{n + 2} - \sqrt[3]{n}\right) \\
% 			&= \lim \limits_{n \to +\infty}\dfrac{n + 2 - n}{\sqrt[3]{(n + 2)^2} + \sqrt[3]{n(n + 2)} + \sqrt[3]{n^2}} \\ 
% 			&= \lim \limits_{n \to +\infty}\dfrac{2}{\sqrt[3]{(n + 2)^2} + \sqrt[3]{n(n + 2)} + \sqrt[3]{n^2}} \\  
% 			&= 0
% 		\end{align*}
% 	}
% \end{bt} 
% \begin{bt}%[Dự án soạn đề cương toán 11 - KNTT, Minh Trí]%[1K5KE-3]
% 	Tính giới hạn $I = \lim \limits_{n \to +\infty}\dfrac{\sqrt{4n^2 + 2n} - n + 1}{\sqrt{9n^2 + n} - 2n}$.
% 	\loigiai{
% 		Ta có
% 		\begin{align*} 
% 			I &= \lim \limits_{n \to +\infty}\dfrac{\sqrt{4n^2 + 2n} - n + 1}{\sqrt{9n^2 + n} - 2n} \\
% 			&= \lim \limits_{n \to +\infty}\dfrac{\left[4n^2 + 2n - (n - 1)^2\right]\left(\sqrt{9n^2 + n} + 2n\right)}{\left(\sqrt{4n^2 + 2n} + n - 1\right)(9n^2 + n - 4n^2)} \\ 
% 			&= \lim \limits_{n \to +\infty}\dfrac{(3n^2 + 4n - 1)\left(\sqrt{9n^2 + n} + 2n\right)}{\left(\sqrt{4n^2 + 2n} + n - 1)(5n^2 + n\right)} \\
% 			&= \lim \limits_{n \to +\infty}\dfrac{\left(3 + \dfrac{4}{n} - \dfrac{1}{n^2}\right)\left(\sqrt{9 + \dfrac{1}{n}} + 2\right)}{\left(\sqrt{4 + \dfrac{2}{n}} + 1 - \dfrac{1}{n}\right)\left(5 + \dfrac{1}{n}\right)} \\
% 			&= \dfrac{3\left(\sqrt{9} + 2\right)}{5\left(\sqrt{4} + 1\right)} = 1
% 		\end{align*}
% 	}
% \end{bt}  
% \begin{bt}%[Dự án soạn đề cương toán 11 - KNTT, Minh Trí]%[1K5KE-3]
% 	Tính giới hạn $I = \lim \limits_{n \to +\infty}\left(\sqrt[3]{8n^3 + 1} - \sqrt{4n^2 - n + 5}\right)$.
% 	\loigiai{
% 		Ta có
% 		\begin{align*} 
% 			I &= \lim \limits_{n \to +\infty}\left(\sqrt[3]{8n^3 + 1} - \sqrt{4n^2 - n + 5}\right) \\
% 			&= \lim \limits_{n \to +\infty}\left(\sqrt[3]{8n^3 + 1} - 2n\right) + \lim \limits_{n \to +\infty}\left(2n - \sqrt{4n^2 - n + 5}\right) \\ 
% 			&= \lim \limits_{n \to +\infty}\dfrac{8n^3 + 1 - 8n^3}{\sqrt[3]{(8n^3 + 1)^2} + 2n\sqrt[3]{8n^3 + 1} + 4n^2} + \lim \limits_{n \to +\infty}\dfrac{4n^2 - (4n^2 - n + 5)}{2n + \sqrt{4n^2 - n + 5}} \\
% 			&= \lim \limits_{n \to +\infty}\dfrac{1}{\sqrt[3]{(8n^3 + 1)^2} + 2n\sqrt[3]{8n^3 + 1} + 4n^2} + \lim \limits_{n \to +\infty}\dfrac{n - 5}{2n + \sqrt{4n^2 - n + 5}} \\
% 			&= 0 + \lim \limits_{n \to +\infty}\dfrac{1 - \dfrac{5}{n}}{2 + \sqrt{4 - \dfrac{1}{n} + \dfrac{5}{n^2}}} \\
% 			&= \dfrac{1}{2 + \sqrt{4}} = \dfrac{1}{4}
% 		\end{align*}
% 	}
% \end{bt}  
% \begin{bt}%[Dự án soạn đề cương toán 11 - KNTT, Minh Trí]%[1K5KE-3]
% 	Tính giới hạn $I = \lim \limits_{n \to +\infty}\dfrac{\sqrt{3n^2 + 1} - \sqrt{n - 1}}{n}$.
% 	\loigiai{
% 		Ta có
% 		\begin{align*} 
% 			I &= \lim \limits_{n \to +\infty}\dfrac{\sqrt{3n^2 + 1} - \sqrt{n - 1}}{n} \\
% 			&= \lim \limits_{n \to +\infty}\dfrac{3n^2 + 1 - (n - 1)}{n \left(\sqrt{3n^2 + 1} + \sqrt{n - 1}\right)} \\ 
% 			&= \lim \limits_{n \to +\infty}\dfrac{3n^2 - n + 2}{n \left(\sqrt{3n^2 + 1} + \sqrt{n - 1}\right)} \\
% 			&= \lim \limits_{n \to +\infty}\dfrac{3 - \dfrac{1}{n} + \dfrac{2}{n^2}}{\sqrt{3 + \dfrac{1}{n^2}} + \sqrt{\dfrac{1}{n} - \dfrac{1}{n^2}}} \\
% 			&= \sqrt{3}
% 		\end{align*}
% 	}
% \end{bt}  
\subsubsection{Câu hỏi trắc nghiệm}
\Opensolutionfile{ans}[ans/ans-1K5-1-Dang2]
\begin{ex}%[Dự án soạn đề cương toán 11 - KNTT, Minh Trí]%[1K5TE-3]
	Tính giới hạn $I = \lim \limits_{n \to +\infty}\left(\sqrt{n^2 + 2n + 3} - n\right)$
	\choice
	{\True $1$}
	{$0$}
	{$2$}
	{$3$}
	\loigiai{
		Ta có $I = \lim \limits_{n \to +\infty}\left(\sqrt{n^2 + 2n + 3} - n\right) = \lim \limits_{n \to +\infty}\dfrac{2n + 3}{\sqrt{n^2 + 2n + 3} + n} = \lim \limits_{n \to +\infty}\dfrac{2 + \dfrac{3}{n}}{\sqrt{1 + \dfrac{2}{n} + \dfrac{3}{n^2}} + 1} = 1$
	}
\end{ex}
\begin{ex}%[Dự án soạn đề cương toán 11 - KNTT, Minh Trí]%[1K5KE-3]
	Tính giới hạn $I = \lim \limits_{n \to +\infty}\left(\sqrt{n^2 + 1} - \sqrt{n^2 - 2}\right)$
	\choice
	{$3$}
	{\True $0$}
	{$\sqrt{3}$}
	{$\dfrac{3}{2}$}
	\loigiai{
		Ta có $I = \lim \limits_{n \to +\infty}\left(\sqrt{n^2 + 1} - \sqrt{n^2 - 2}\right) = \lim \limits_{n \to +\infty}\dfrac{3}{\sqrt{n^2 + 1} + \sqrt{n^2 - 2}} = 0$
	}
\end{ex}

\begin{ex}%[Dự án soạn đề cương toán 11 - KNTT, Minh Trí]%[1K5BE-3]
	Tính giới hạn $I = \lim \limits_{n \to +\infty}\left(n - \sqrt{n^2 + 2n - 3}\right)$
	\choice
	{$0$}
	{$-2$}
	{\True $-1$}
	{$2$}
	\loigiai{
		Ta có $I = \lim \limits_{n \to +\infty}\left(n - \sqrt{n^2 + 2n - 3}\right) = \lim \limits_{n \to +\infty}\dfrac{- 2n + 3}{n + \sqrt{n^2 + 2n - 3}} = \lim \limits_{n \to +\infty}\dfrac{- 2 + \dfrac{3}{n}}{1 + \sqrt{1 + \dfrac{2}{n} - \dfrac{3}{n^2}}} = - 1$
	}
\end{ex}
\begin{ex}%[Dự án soạn đề cương toán 11 - KNTT, Minh Trí]%[1K5BE-3]
	Tính giới hạn $I = \lim \limits_{n \to +\infty}\left(\sqrt{n^2 - n + 1} - n\right)$
	\choice
	{$\dfrac{1}{2}$}
	{$1$}
	{\True $- \dfrac{1}{2}$}
	{$0$}
	\loigiai{
		Ta có $I = \lim \limits_{n \to +\infty}\left(\sqrt{n^2 - n + 1} - n\right) = \lim \limits_{n \to +\infty}\dfrac{- n + 1}{\sqrt{n^2 - n + 1} + n} = \lim \limits_{n \to +\infty}\dfrac{- 1 + \dfrac{1}{n}}{\sqrt{1 - \dfrac{1}{n} + \dfrac{1}{n^2}} + 1} = - \dfrac{1}{2}$
	}
\end{ex}
\begin{ex}%[Dự án soạn đề cương toán 11 - KNTT, Minh Trí]%[1K5BE-3]
	Tính giới hạn $I = \lim \limits_{n \to +\infty}\left(\sqrt[3]{n^3 - n^2} - n\right)$
	\choice
	{\True $- \dfrac{1}{3}$}
	{$\dfrac{1}{3}$}
	{$1$}
	{$0$}
	\loigiai{
		Ta có $I = \lim \limits_{n \to +\infty}\left(\sqrt[3]{n^3 - n^2} - n\right)) = \lim \limits_{n \to +\infty}\dfrac{- n^2}{\sqrt[3]{(n^3 - n^2)^2} + n\sqrt[3]{(n^3 - n^2)} + n^2} = \lim \limits_{n \to +\infty}\dfrac{- 1}{\sqrt[3]{\left(1 - \dfrac{1}{n}\right)^2} + \sqrt[3]{1 - \dfrac{1}{n}} + 1} = - \dfrac{1}{3}$
	}
\end{ex}

\begin{ex}%[Dự án soạn đề cương toán 11 - KNTT, Minh Trí]%[1K5BE-3]
	Tính giới hạn $I = \lim \limits_{n \to +\infty}\left(\sqrt{2n^2 + 2n - 1} - \sqrt{2n^2 + n}\right)$
	\choice
	{$0$}
	{$\dfrac{1}{4}$}
	{$\dfrac{1}{2}$}
	{\True $\dfrac{1}{2\sqrt{2}}$}
	\loigiai{
		Ta có $I = \lim \limits_{n \to +\infty}\left(\sqrt{2n^2 + 2n - 1} - \sqrt{2n^2 + n}\right) = \lim \limits_{n \to +\infty}\dfrac{n}{\sqrt{2n^2 + 2n - 1} + \sqrt{2n^2 + n}} = \lim \limits_{n \to +\infty}\dfrac{1}{\sqrt{2 + \dfrac{2}{n} - \dfrac{1}{n^2}} + \sqrt{2 + \dfrac{2}{n}}} = \dfrac{1}{2\sqrt{2}}$
	}
\end{ex}

\begin{ex}%[Dự án soạn đề cương toán 11 - KNTT, Minh Trí]%[1K5BE-3]
	Tính giới hạn $I = \lim \limits_{n \to +\infty}\left(\sqrt[3]{n^3 + 2} - \sqrt[3]{n^3 + 1}\right)$
	\choice
	{\True $0$}
	{$-\dfrac{1}{3}$}
	{$1$}
	{$\dfrac{1}{3}$}
	\loigiai{
		Ta có $I = \lim \limits_{n \to +\infty}\left(\sqrt[3]{n^3 + 2} - \sqrt[3]{n^3 + 1}\right) = \lim \limits_{n \to +\infty}\dfrac{1}{\sqrt[3]{(n^3 + 2)^2} + \sqrt[3]{(n^3 + 2)(n^3 + 1)} + \sqrt[3]{(n^3 + 1)^2}} = 0$
	}
\end{ex}
\begin{ex}%[Dự án soạn đề cương toán 11 - KNTT, Minh Trí]%[1K5BE-3]
	Tính giới hạn $I = \lim \limits_{n \to +\infty}n\left(\sqrt{n^2 + n + 1} - \sqrt{n^2 + n - 8}\right)$
	\choice
	{$0$}
	{$\infty$}
	{$2$}
	{\True $\dfrac{9}{2}$}
	\loigiai{
		Ta có $I = \lim \limits_{n \to +\infty}n\left(\sqrt{n^2 + n + 1} - \sqrt{n^2 + n - 8}\right) = \lim \limits_{n \to +\infty}\dfrac{9n}{\sqrt{n^2 + n + 1} + \sqrt{n^2 + n - 8}}\\ = \lim \limits_{n \to +\infty}\dfrac{9}{\sqrt{1 + \dfrac{1}{n} + \dfrac{1}{n^2}} + \sqrt{1 + \dfrac{1}{n} - \dfrac{8}{n^2}}} = \dfrac{9}{2}$ 
	}
\end{ex}
\Closesolutionfile{ans}
% \begin{indapan}{10}
% 	{ans/ans-1K5-1-Dang2}
% \end{indapan}
\begin{dang}{Giới hạn vô cực}
	Ta nói dãy $\{u_n\}$ có giới hạn là $+ \infty$ khi $n \rightarrow + \infty$, nếu $u_n$ có thể lớn hơn một số dương bất kì, kể từ một số hạng nào đó trở đi. \\
	Kí hiệu: $\lim \limits_{n \to +\infty}u_n = + \infty$ hay $u_n \rightarrow + \infty$ khi $n \rightarrow + \infty$. \\ 
	Dãy số $\{u_n\}$ có giới hạn là $- \infty$ khi $n \rightarrow + \infty$, nếu $\lim \limits_{n \to +\infty}- u_n = + \infty$. \\
	Kí hiệu: $\lim \limits_{n \to +\infty}u_n = - \infty$ hay $u_n \rightarrow - \infty$ khi $n \rightarrow + \infty$. \\ 
	\textbf{Một số giới hạn đặc biệt và định lí về giới hạn dãy số} \\
	\textit{Giới hạn đặc biệt}: \\
	$\displaystyle \lim_{n \rightarrow + \infty} \sqrt{n} = + \infty$ \\
	$\displaystyle \lim_{n \rightarrow + \infty} n^k = + \infty$ với $k$ là số nguyên dương. \\
	$\displaystyle \lim_{n \rightarrow + \infty} q^n = + \infty$ nếu $q > 1$ \\
	\textit{Định lý}: \\
	Nếu $\lim \limits_{n \to +\infty}u_n = a > 0$ và $\lim \limits_{n \to +\infty}v_n = 0$ với $v_n > 0$ thì $\lim \limits_{n \to +\infty}\dfrac{u_n}{v_n} = + \infty$. \\
	Nếu $\lim \limits_{n \to +\infty}u_n = + \infty$ và $\lim \limits_{n \to +\infty}v_n = a > 0$ thì $\lim \limits_{n \to +\infty}u_nv_n = + \infty$.
\end{dang}
\subsubsection{Ví dụ minh hoạ}
\begin{vd}%[Dự án soạn đề cương toán 11 - KNTT, Minh Trí]%[1K5YE-4]
	Tìm giới hạn 
	\begin{enumEX}[a)]{2}
		\item[a)] $\lim \limits_{n \to +\infty}(n^3 + n^2 + n + 1)$.
		\item[b)] $\lim \limits_{n \to +\infty}\left(n^2 - n\sqrt{n} + 1\right)$.
	\end{enumEX}
	\loigiai{
		\begin{enumEX}[a)]{1}
			\item $\lim \limits_{n \to +\infty}(n^3 + n^2 + n + 1) = \lim \limits_{n \to +\infty}n^3\left(1 + \dfrac{1}{n} + \dfrac{1}{n^2} + \dfrac{1}{n^3}\right) = + \infty$.
			\item $\lim \limits_{n \to +\infty}\left(n^2 - n\sqrt{n} + 1\right) = \lim \limits_{n \to +\infty}n^2\left(1 - \dfrac{1}{\sqrt{n}} + \dfrac{1}{n^2}\right) = + \infty.$
		\end{enumEX}
	}
\end{vd}
\begin{vd}%[Dự án soạn đề cương toán 11 - KNTT, Minh Trí]%[1K5BE-4]
	Tìm giới hạn
	\begin{enumEX}[a)]{3}
		\item[a)] $\lim \limits_{n \to +\infty}\dfrac{n^5 + n^4 - n - 2}{4n^3 + 6n^2 + 9}$.
		\item[b)] $\lim \limits_{n \to +\infty}\dfrac{\sqrt[3]{n^6 - 7n^3 - 5n + 8}}{n + 12}$.
		\item[c)] $\lim \limits_{n \to +\infty}\left(n + \sqrt{n^2 - n + 1}\right)$.
	\end{enumEX}
	\loigiai{
		\begin{enumEX}[a)]{1}
			\item $\lim \limits_{n \to +\infty}\dfrac{n^5 + n^4 - n - 2}{4n^3 + 6n^2 + 9} = \lim \limits_{n \to +\infty}\dfrac{n^2 + n - \dfrac{1}{n^2} - \dfrac{2}{n^3}}{4 + \dfrac{6}{n} + \dfrac{9}{n^3}} = \lim \limits_{n \to +\infty}\dfrac{n^2 + n}{4} = + \infty$.
			\item $\lim \limits_{n \to +\infty}\dfrac{\sqrt[3]{n^6 - 7n^3 - 5n + 8}}{n + 12} = \lim \limits_{n \to +\infty}\dfrac{n^2\sqrt[3]{1 - \dfrac{7}{n^3} - \dfrac{5}{n^5} + \dfrac{8}{n^6}}}{n + 12} = \lim \limits_{n \to +\infty}\dfrac{n\sqrt[3]{1 - \dfrac{7}{n^3} - \dfrac{5}{n^5} + \dfrac{8}{n^6}}}{1 + \dfrac{12}{n}} = + \infty$.
			\item $\lim \limits_{n \to +\infty}\left(n + \sqrt{n^2 - n + 1}\right) = n\left(1 + \sqrt{1 - \dfrac{1}{n} + \dfrac{1}{n^2}}\right) = \lim \limits_{n \to +\infty}2n = + \infty$
		\end{enumEX}
	}
\end{vd}
\begin{vd}%[Dự án soạn đề cương toán 11 - KNTT, Minh Trí]%[1K5KE-4]
	Tìm giới hạn
	\begin{enumEX}[a)]{3}
		\item[a)] $\lim \limits_{n \to +\infty}\dfrac{1^3 + 2^3 + ... + n^3}{n^2 + 3n\sqrt{n} + 2}$.
		\item[b)] $\lim \limits_{n \to +\infty}\left(n + \sqrt[3]{n^3 - 2n + 1}\right)$.
		\item[c)] $\lim \limits_{n \to +\infty}\dfrac{n^3 - 3n}{2n + 15}$.
	\end{enumEX}
	\loigiai{
		\begin{enumEX}[a)]{1}
			\item $\lim \limits_{n \to +\infty}\dfrac{1^3 + 2^3 + ... + n^3}{n^2 + 3n\sqrt{n} + 2} = \lim \limits_{n \to +\infty}\dfrac{\dfrac{1}{4}n^2(n + 1)^2}{n^2 + 3n\sqrt{n} + 2} = \lim \limits_{n \to +\infty}\dfrac{\dfrac{1}{4}(n + 1)^2}{1 + \dfrac{3}{\sqrt{n}} + \dfrac{2}{n^2}} = \lim \limits_{n \to +\infty}\dfrac{1}{4}(n + 1)^2 = + \infty$.
			\item $\lim \limits_{n \to +\infty}\left(n + \sqrt[3]{n^3 - 2n + 1}\right) = n\left(1 + \sqrt[3]{1 - \dfrac{2}{n^2} + \dfrac{1}{n^3}}\right) = \lim \limits_{n \to +\infty}2n = + \infty$.
			\item $\lim \limits_{n \to +\infty}\dfrac{n^3 - 3n}{2n + 15} = \lim \limits_{n \to +\infty}\dfrac{n^2 - 3}{2 + \dfrac{15}{n}} = + \infty$
		\end{enumEX}
	}
\end{vd}
% \subsubsection{Bài tập rèn luyện} 
% \subsubsection{Bài tập tự luận}
% \begin{bt}%[Dự án soạn đề cương toán 11 - KNTT, Minh Trí]%[1K5BE-4]
% 	Tìm giới hạn
% 	\begin{enumEX}[a)]{2}
% 		\item[a)] $\lim \limits_{n \to +\infty}\sqrt{5n^2 - 8n + 7}$.
% 		\item[b)] $\lim \limits_{n \to +\infty}\sqrt{n^3 - 5n + 6}$.
% 	\end{enumEX}
% 	\loigiai{
% 		\begin{enumEX}[a)]{1}
% 			\item $\lim \limits_{n \to +\infty}\sqrt{5n^2 - 8n + 7} = \lim \limits_{n \to +\infty}n\sqrt{5 - \dfrac{8}{n} + \dfrac{7}{n^2}} = + \infty$.
% 			\item $\lim \limits_{n \to +\infty}\sqrt{n^3 - 5n + 6} = \lim \limits_{n \to +\infty}n\sqrt{n} \sqrt{1 - \dfrac{5}{n^2} + \dfrac{6}{n^3}} = + \infty$.
% 		\end{enumEX}
% 	}
% \end{bt}
% \begin{bt}%[Dự án soạn đề cương toán 11 - KNTT, Minh Trí]%[1K5KE-4]
% 	Tìm giới hạn
% 	\begin{enumEX}[a)]{2}
% 		\item[a)] $\lim \limits_{n \to +\infty}\dfrac{\sqrt{5n^4 - 8n^2 + 10}}{4n + 5}$.
% 		\item[b)] $\lim \limits_{n \to +\infty}\dfrac{n^2 - 15n + 11}{\sqrt{n^2 - 8n + 7}}$.
% 	\end{enumEX}
% 	\loigiai{
% 		\begin{enumEX}[a)]{1}
% 			\item $\lim \limits_{n \to +\infty}\dfrac{\sqrt{5n^4 - 8n^2 + 10}}{4n + 5} = \lim \limits_{n \to +\infty}\dfrac{n^2\sqrt{5 - \dfrac{8}{n^2} + \dfrac{10}{n^4}}}{4n + 5} = \dfrac{n\sqrt{5 - \dfrac{8}{n^2} + \dfrac{10}{n^4}}}{4 + \dfrac{5}{n}} = \lim \limits_{n \to +\infty}\dfrac{n\sqrt{5}}{4} = + \infty$.
% 			\item $\lim \limits_{n \to +\infty}\dfrac{n^2 - 15n + 11}{\sqrt{n^2 - 8n + 7}} = \lim \limits_{n \to +\infty}\dfrac{n - 15 + \dfrac{11}{n}}{\sqrt{1 - \dfrac{8}{n} + \dfrac{7}{n^2}}} = + \infty$.
% 		\end{enumEX}
% 	}
% \end{bt}
% \begin{bt}%[Dự án soạn đề cương toán 11 - KNTT, Minh Trí]%[1K5BE-4]
% 	Tìm $\lim \limits_{n \to +\infty}\left(\dfrac{1}{n^2}+\dfrac{2}{n^2}+\ldots+\dfrac{n}{n^2}\right)$.
% 	\loigiai{
% 		$$
% 		\lim \limits_{n \to +\infty}\left(\dfrac{1}{n^2}+\dfrac{2}{n^2}+\ldots+\dfrac{n}{n^2}\right)=\lim \limits_{n \to +\infty}\left(\dfrac{1+2+\ldots+n}{n^2}\right)=\lim \limits_{n \to +\infty}\left(\dfrac{n(n+1)}{2 n^2}\right)=\lim \limits_{n \to +\infty}\left(\dfrac{1+\dfrac{1}{n}}{2}\right)=\dfrac{1}{2} .
% 		$$}
% \end{bt}
% \begin{bt}%[Dự án soạn đề cương toán 11 - KNTT, Minh Trí]%[1K5GE-4]
% 	Tính giới hạn: $\lim \limits_{n \to +\infty}\left[\left(1-\dfrac{1}{2^2}\right)\left(1-\dfrac{1}{3^2}\right) \ldots\left(1-\dfrac{1}{n^2}\right)\right]$.\\
% 	Xét dãy số $\left(u_n\right)$, với $u_n=\left(1-\dfrac{1}{2^2}\right)\left(1-\dfrac{1}{3^2}\right) \ldots\left(1-\dfrac{1}{n^2}\right), n \geq 2, n \in \mathbb{N}$.
% 	\loigiai{
% 		Ta có:
% 		$$
% 		\begin{aligned}
% 			& u_2=1-\dfrac{1}{2^2}=\dfrac{3}{4}=\dfrac{2+1}{2 \cdot 2} \\
% 			& u_3=\left(1-\dfrac{1}{2^2}\right) \cdot\left(1-\dfrac{1}{3^2}\right)=\dfrac{3}{4} \cdot \dfrac{8}{9}=\dfrac{4}{6}=\dfrac{3+1}{2 \cdot 3} ; \\
% 			& u_4=\left(1-\dfrac{1}{2^2}\right) \cdot\left(1-\dfrac{1}{3^2}\right)\left(1-\dfrac{1}{4^2}\right)=\dfrac{3}{4} \cdot \dfrac{8}{9} \cdot \dfrac{15}{16}=\dfrac{5}{8}=\dfrac{4+1}{2 \cdot 4} \\
% 			& \ldots \ldots . \\
% 			& u_n=\dfrac{n+1}{2 n} .
% 		\end{aligned}
% 		$$
% 		Dễ dàng chứng minh bằng phương pháp qui nạp để khẳng định $u_n=\dfrac{n+1}{2 n}, \forall n \geq 2$
% 		\\ Khi đó $\lim \limits_{n \to +\infty}\left[\left(1-\dfrac{1}{2^2}\right)\left(1-\dfrac{1}{3^2}\right) \ldots\left(1-\dfrac{1}{n^2}\right)\right]=\lim \limits_{n \to +\infty}\dfrac{n+1}{2 n}=\dfrac{1}{2}$.}
% \end{bt}
% \begin{bt}%[Dự án soạn đề cương toán 11 - KNTT, Minh Trí]%[1K5BE-4]
% 	Cho dãy số $\left(u_n\right), n \in \mathbb{N}^*$, thỏa mãn điều kiện $\left\{\begin{array}{c}u_1=3 \\ u_{n+1}=-\dfrac{u_n}{5}\end{array}\right.$. Gọi $S=u_1+u_2+u_3+\ldots+u_n$ là tồng $n$ số hạng đầu tiên của dãy số đã cho. Khi đó lim $S_n$ bằng
% 	\loigiai{
% 		Ta có $\dfrac{u_{n+1}}{u_n}=\dfrac{-\dfrac{u_n}{5}}{u_n}=-\dfrac{1}{5}$ do đó dãy $\left(u_n\right), n \in \mathbb{N}^*$ là một cấp số nhân lùi vô hạn có $u_1=3, d=-\dfrac{1}{5}$.
% 		Suy ra $\lim \limits_{n \to +\infty}S_n=\dfrac{u_1}{1-q}=\dfrac{3}{1+\dfrac{1}{5}}=\dfrac{5}{2}$.}
% \end{bt}
% \begin{bt}%[Dự án soạn đề cương toán 11 - KNTT, Minh Trí]%[1K5TE-4]
% 	Trong một lần Đoàn trường Lê Văn Hưu tổ chức chơi bóng chuyền hơi, bạn Nam thả một quả bóng chuyền hơi từ tầng ba, độ cao $8 m$ so với mặt đất và thấy rằng mỗi lần chạm đất thì quả bóng lại nảy lên một độ cao bằng ba phần tư độ cao lần rơi trước. Biết quả bóng chuyển động vuông góc với mặt đất. Khi đó tổng quãng đường quả bóng đã bay từ lúc thả bóng đến khi quả bóng không nảy nữa bằng bao nhiêu ?
% 	\loigiai{
% 		Lần đầu rơi xuống, quảng đường quả bóng đã bay đến lúc chạm đất là $8 m$.\\
% 		Sau đó quả bóng nảy lên và rơi xuống chạm đất lần thứ 2 thì quảng đường quả bóng đã bay là $8+2.8 \cdot \dfrac{3}{4}$.\\
% 		Tương tự, khi quả bóng nảy lên và rơi xuống chạm đất lần thứ $n$ thì quảng đường quả bóng đã bay là $8+2\cdot 8 \cdot \dfrac{3}{4}+\ldots \ldots +2.8 \cdot\left(\dfrac{3}{4}\right)^{n-1}=8+\dfrac{1-\left(\dfrac{3}{4}\right)^n}{1-\dfrac{3}{4}}=8+48\left(1-\left(\dfrac{3}{4}\right)^{n-1}\right)$.\\
% 		Quảng đường quả bóng đã bay từ lúc thả đến lúc không máy nữa bằng: $\lim \limits_{n \to +\infty}\left[8+48\left(1-\left(\dfrac{3}{4}\right)^{n-1}\right)\right]=8+48=56$.}
% \end{bt}
% \begin{bt}%[Dự án soạn đề cương toán 11 - KNTT, Minh Trí]%[1K5GE-4]
% 	Cho hình vuông $ABCD$ có cạnh bằng $a$. Người ta dựng hình vuông $A_1B_1C_1D_1$ có cạnh bằng $\dfrac{1}{2}$ đường chéo của hình vuông $ABCD$; dựng hình vuông $A_2 B_2 C_2 D_2$ có cạnh bằng $\dfrac{1}{2}$ đường chéo của hình vuông $A_1B_1C_1D_1$ và cứ tiếp tục như vậy (tham khảo hình vẽ).
% 	Giả sử cách dựng trên có thể tiến ra vô hạn. Nếu tổng diện tích $S$ của tất cả các hình vuông $ABCD$, $A_1B_1C_1D_1$, $A_2B_2C_2D_2$, $\ldots$ bằng $8$ thì $a$ bằng bao nhiêu? 
% 	\begin{center}
% 		\begin{tikzpicture}[scale=1, font=\footnotesize, line join=round, line cap=round, >=stealth]
% 			\coordinate (A) at (0,0);
% 			\coordinate (B) at (6,0);
% 			\coordinate (D) at (0,6);
% 			\coordinate (C) at ($(B)+(D)-(A)$);
% 			\draw (A)--(B)--(C)--(D)--(A);
			
% 			\coordinate (A_1) at ($(A)!1/2!(B)$);
% 			\coordinate (B_1) at ($(C)!1/2!(B)$);
% 			\coordinate (C_1) at ($(C)!1/2!(D)$);
% 			\coordinate (D_1) at ($(A)!1/2!(D)$);
% 			\draw (A_1)--(B_1)--(C_1)--(D_1)--(A_1);
% 			\foreach \x in {2,3,4,5,6,7}
% 			{ \pgfmathsetmacro{\j}{\x-1}
% 				\coordinate (A_\x) at ($(A_\j)!1/2!(D_\j)$);
% 				\coordinate (B_\x) at ($(A_\j)!1/2!(B_\j)$);
% 				\coordinate (C_\x) at ($(C_\j)!1/2!(B_\j)$);
% 				\coordinate (D_\x) at ($(C_\j)!1/2!(D_\j)$);
% 				\draw (A_\x)--(B_\x)--(C_\x)--(D_\x)--(A_\x);
% 			}
% 			\foreach \x/\y in {A/225,B/-45,C/45,D/135,A_1/-90,B_1/0,C_1/90,D_1/180}{\fill (\x) circle (1pt) ($(\x)+(\y:0.3cm)$) node{$\x$};}
% 			\def \goc{-90};
			
% 			\foreach \t in {2,3,4,5}
% 			{\pgfmathsetmacro{\goca}{\goc - \t*45+45}
% 				\pgfmathsetmacro{\gocb}{\goca + 90}
% 				\pgfmathsetmacro{\gocc}{\gocb + 90}
% 				\pgfmathsetmacro{\gocd}{\gocc + 90}
% 				\foreach \x/\y in {A_\t/\goca,B_\t/\gocb,C_\t/\gocc,D_\t/\gocd}{\fill (\x) circle (1pt) ($(\x)+(\y:0.3cm)$) node{$\x$};}		
% 			}
			
% 		\end{tikzpicture}
% 	\end{center}
% 	\loigiai{
% 		$$
% 		\begin{aligned}
% 			& \text { Ta có } S_{A B C D}=a^2 ; S_{A_1 B_1 C_1 D_1}=\left(\dfrac{a \sqrt{2}}{2}\right)^2=\dfrac{a^2}{2} ; S_{A_2 B_2 C_2 D_2}=\left(\dfrac{a}{2}\right)^2=\dfrac{a^2}{4}=\dfrac{a^2}{2^2} \\
% 			& S=S_{A B C D}+S_{A_1 B_1 C_1 D_1}+S_{A_2 B_2 C_2 D_2}+\ldots=a^2+\dfrac{a^2}{2}+\dfrac{a^2}{2^2}+\ldots=a^2\left(1+\dfrac{1}{2}+\dfrac{1}{2^2}+\ldots\right)=a^2 \cdot \dfrac{1}{1-\dfrac{1}{2}}=2 a^2
% 		\end{aligned}
% 		$$
% 	}
% \end{bt}
% \begin{bt}%[Dự án soạn đề cương toán 11 - KNTT, Minh Trí]%[1K5GE-4]
% 	Cho hình vuông $C_1$ có cạnh bằng $a$. Người ta chia mỗi cạnh của hình vuông thành bốn phần bằng nhau và nối các điểm chia một cách thích hợp để có hình vuông $C_2$ (tham khảo hình vẽ).
% 	Từ hình vuông $C_2$ lại tiếp tục làm như trên ta nhận được dãy các hình vuông $C_1, C_2, C_3, \ldots, C_n, \ldots$.Gọi $S_i$ là diện tích của hình vuông $C_i(i \in\{1 ; 2 ; 3 ; \ldots\})$. Tính tổng $S=S_1+S_2+S_3+\ldots+S_n+\ldots$\\
% 	\begin{center}
% 		\begin{tikzpicture}[scale=1, font=\footnotesize, line join=round, line cap=round, >=stealth]
% 			\coordinate (A1) at (0,0);
% 			\coordinate (B1) at (4,0);
% 			\coordinate (D1) at (0,4);
% 			\coordinate (C1) at ($(B1)+(D1)-(A1)$);
% 			\draw (A1)--(B1)--(C1)--(D1)--(A1);
			
% 			\foreach \x in {2,3,4}
% 			{ \pgfmathsetmacro{\j}{\x-1}
% 				\coordinate (A\x) at ($(A\j)!1/4!(B\j)$);
% 				\coordinate (B\x) at ($(B\j)!1/4!(C\j)$);
% 				\coordinate (C\x) at ($(C\j)!1/4!(D\j)$);
% 				\coordinate (D\x) at ($(D\j)!1/4!(A\j)$);
% 				\draw (A\x)--(B\x)--(C\x)--(D\x)--(A\x);
% 			}
% 		\end{tikzpicture}
% 	\end{center}
% 	\loigiai{
% 		Ta có $S_1=a^2, S_2=\dfrac{5}{8} a^2, S_3=\dfrac{25}{64} a^2, \ldots$
% 		Nên $S=S_1+S_2+S_3+\ldots+S_n+\ldots$ là tổng của cấp số nhân lùi vô hạn với $\left\{\begin{array}{l}u_1=a^2 \\ q=\dfrac{5}{8}\end{array}\right.$.
% 		Khi đó $S=\dfrac{u_1}{1-q}=\dfrac{a^2}{1-\dfrac{5}{8}}=\dfrac{8}{3} a^2$.}
% \end{bt}
% \subsubsection{Câu hỏi trắc nghiệm}
% \Opensolutionfile{ans}[ans/ans-1K5-1-Dang3]
% \begin{ex}%[Dự án soạn đề cương toán 11 - KNTT, Minh Trí]%[1K5KE-4]
% 	Giá trị của giới hạn $\lim \limits_{n \to +\infty}\left(1+\dfrac{1}{2}+\dfrac{1}{2^2}+\ldots+\dfrac{1}{2^n}\right)$ là
% 	\choice
% 	{$1$}
% 	{\True $2$} 
% 	{$\dfrac{1}{2}$}
% 	{$\dfrac{3}{2}$}
% 	\loigiai{
% 		Ta có: $\lim \limits_{n \to +\infty}\left(1+\dfrac{1}{2}+\dfrac{1}{2^2}+\ldots+\dfrac{1}{2^n}\right)=\dfrac{1}{1-\dfrac{1}{2}}=2$.}
% \end{ex}
% \begin{ex}%[Dự án soạn đề cương toán 11 - KNTT, Minh Trí]%[1K5BE-4]
% 	Tính giới hạn $I=\lim \limits_{n \to +\infty}\dfrac{5\cdot4^{n+1}+3^{n+2}}{2^{2 n+1}+1}$.
% 	\choice
% 	{$I=+\infty$}
% 	{\True $I=10$}
% 	{$I=0$}
% 	{$I=20$}
% 	\loigiai{
% 		Ta có $I=\lim \limits_{n \to +\infty}\dfrac{5.4^{n+1}+3^{n+2}}{2^{2 n+1}+1}=\lim \limits_{n \to +\infty}\dfrac{20.4^n+9.3^n}{2 \cdot 4^n+1}=\lim \limits_{n \to +\infty}\dfrac{20+9 \cdot\left(\dfrac{3}{4}\right)^n}{2+\left(\dfrac{1}{4}\right)^n}=\dfrac{20}{2}=10$.}
% \end{ex}

% \begin{ex}%[Dự án soạn đề cương toán 11 - KNTT, Minh Trí]%[1K5BE-4]
% 	Tính tồng $S=1+\dfrac{1}{2}+\dfrac{1}{4}+\dfrac{1}{8}+\cdots+\dfrac{1}{2^n}+\cdots$
% 	\choice
% 	{\True $2$}
% 	{$3$}
% 	{$1$}
% 	{$\dfrac{1}{2}$}
% 	\loigiai{
% 		$S=1+\dfrac{1}{2}+\dfrac{1}{4}+\dfrac{1}{8}+\ldots+\dfrac{1}{2^n}+\ldots=\dfrac{u_1}{1-q}=\dfrac{1}{1-\dfrac{1}{2}}=2$.}
% \end{ex}
% \begin{ex}%[Dự án soạn đề cương toán 11 - KNTT, Minh Trí]%[1K5BE-4]
% 	Tính $\lim \limits_{n \to +\infty}\dfrac{3 n^3-2}{1-2 n^3}$ được kết quả là
% 	\choice
% 	{$\dfrac{3}{2}$}
% 	{\True $-\dfrac{3}{2}$}
% 	{$\dfrac{1}{2}$}
% 	{$\dfrac{-1}{2}$}
% 	\loigiai{
% 		Ta có: $\lim \limits_{n \to +\infty}\dfrac{3 n^3-2}{1-2 n^3}=\lim \limits_{n \to +\infty}\dfrac{n^3\left(3-\dfrac{2}{n^3}\right)}{n^3\left(\dfrac{1}{n^3}-2\right)}=\lim \limits_{n \to +\infty}\dfrac{3-\dfrac{2}{n^3}}{\dfrac{1}{n^3}-2}=\dfrac{3-0}{0-2}=-\dfrac{3}{2}$.}
	
% \end{ex}
% \begin{ex}%[Dự án soạn đề cương toán 11 - KNTT, Minh Trí]%[1K5GE-4]
% 	Cho các số $a, b, c \in R ; b+c=5 ; \lim\limits_{x \rightarrow+\infty}\left(\sqrt{a x^2+b x}-c x\right)=2$. Tính $P=a+2 b+c$
% 	\choice
% 	{$P=12$}
% 	{$P=15$}
% 	{\True $P=10$}
% 	{$P=5$}
% 	\loigiai{
% 		Ta có: Biện luận \\
% 		+ Điều kiện cần để tồn tại giới hạn đã cho là $a>0$\\
% 		+ Nếu $c \leq 0 \Rightarrow \lim\limits_{x \rightarrow+\infty}\left(\sqrt{a x^2+b x}-c x\right)=+\infty$ (loại) \\
% 		+ Nếu $c>0$\\
% 		$2=\lim\limits_{x \rightarrow+\infty}\left(\sqrt{a x^2+b x}-c x\right)=\lim\limits_{x \rightarrow+\infty} \dfrac{\left(\sqrt{a x^2+b x}-c x\right)\left(\sqrt{a x^2+b x}+c x\right)}{\sqrt{a x^2+b x}+c x}=\lim\limits_{x \rightarrow+\infty} \dfrac{\left(a-c^2\right) x^2+b x}{\sqrt{a x^2+b x}+c x}$ là hữu hạn nên: $a-c^2=0 \Leftrightarrow a=c^2$ (1)\\
% 		Khi đó: $2=\lim\limits_{x \rightarrow+\infty} \dfrac{b x}{\sqrt{a x^2+b x}+c x}=\lim\limits_{x \rightarrow+\infty} \dfrac{b}{\sqrt{a+\dfrac{b}{x}}+c}=\dfrac{b}{\sqrt{a}+c} \Leftrightarrow 2(\sqrt{a}+c)=b$\\
% 		Từ ta có hệ: $\left\{\begin{array}{l}a=c^2 \\ 2(\sqrt{a}+c)=b \\ b+c=5 \\ a, c>0\end{array} \Leftrightarrow\left\{\begin{array}{l}a=c^2 \\ 4 c=b \\ b+c=5 \\ a, c>0\end{array} \Leftrightarrow\left\{\begin{array}{l}a=1 \\ b=4 \\ c=1\end{array} \Rightarrow P=a+2 b+c=10\right.\right.\right.$
% 	}
% \end{ex} 
% \begin{ex}%[Dự án soạn đề cương toán 11 - KNTT, Minh Trí]%[1K5KE-4]
% 	Tính $I=\lim\limits_{x \rightarrow+\infty} \dfrac{\sqrt{x^2+x+1}-x}{3}=\dfrac{a}{b} ; a, b \in \mathbb{N}$ và $\dfrac{a}{b}$ là phân số tối giản. Khi đó $2 a-b$ bằng kết quả nào sau đây?
% 	\choice
% 	{$4$}
% 	{\True $-4$}
% 	{$-5$}
% 	{$5$}
% 	\loigiai{
% 		Ta có, $\lim\limits_{x \rightarrow+\infty} \dfrac{\sqrt{x^2+x+1}-x}{3}=\lim\limits_{x \rightarrow+\infty} \dfrac{\left(\sqrt{x^2+x+1}-x\right)\left(\sqrt{x^2+x+1}+x\right)}{3\left(\sqrt{x^2+x+1}+x\right)}$
% 		$$
% 		=\lim\limits_{x \rightarrow+\infty} \dfrac{\left(x^2+x+1\right)-x^2}{3\left(\sqrt{x^2+x+1}+x\right)}=\lim\limits_{x \rightarrow+\infty} \dfrac{x+1}{3\left(\sqrt{x^2+x+1}+x\right)}=\lim\limits_{x \rightarrow+\infty} \dfrac{1+\dfrac{1}{x}}{3\left(\sqrt{1+\dfrac{1}{x}+\dfrac{1}{x^2}}+1\right)}=\dfrac{1}{6}
% 		$$
% 		Khi đó, $a=1 ; b=6$. Vậy $2 a-b=-4$}
% \end{ex}
% \begin{ex}%[Dự án soạn đề cương toán 11 - KNTT, Minh Trí]%[1K5GE-4]
% 	Biết lim $\dfrac{\sqrt{n^2-4 n}-\sqrt{4 n^2+1}}{\sqrt{3 n^2+1}-n}=\dfrac{6-\sqrt{3}}{2}-\dfrac{a}{b}$, trong đó $\dfrac{a}{b}$ là phân số tối giản, $a$ và $b$ là các số nguyên dương. Chọn khẳng định đúng trong các khẳng định sau:
% 	\choice
% 	{ $a=b$}
% 	{ $a+b=7$}
% 	{\True $a+b=14$}
% 	{$\dfrac{b}{a}=\dfrac{7}{2}$}
% 	\loigiai{
		
% 		$$
% 		\begin{aligned}
% 			& \lim \limits_{n \to +\infty}\dfrac{\sqrt{n^2-4 n}-\sqrt{4 n^2+1}}{\sqrt{3 n^2+1}-n}=\lim \limits_{n \to +\infty}\dfrac{\sqrt{1-\dfrac{4}{n}}-\sqrt{4+\dfrac{1}{n^2}}}{\sqrt{3+\dfrac{1}{n^2}}-1}=\dfrac{-1-\sqrt{3}}{2}=\dfrac{6-\sqrt{3}}{2}-\dfrac{7}{2} . \\
% 			& \text { Suy ra } \dfrac{a}{b}=\dfrac{7}{2} \Rightarrow a=7 ; b=2 \Rightarrow a . b=14 .
% 		\end{aligned}
% 		$$
% 	}
% \end{ex}
% \begin{ex}%[Dự án soạn đề cương toán 11 - KNTT, Minh Trí]%[1K5KE-4]
% 	Tìm giới hạn $I=\lim\limits_{x \rightarrow+\infty}\left(x+1-\sqrt{x^2-x+2}\right)$.
% 	\choice
% 	{$I=\dfrac{46}{31}$}
% 	{$I=\dfrac{17}{11}$}
% 	{\True $I=\dfrac{3}{2}$}
% 	{$I=\dfrac{1}{2}$}
% 	\loigiai{
% 		$$
% 		\text { Ta có } I=\lim\limits_{x \rightarrow+\infty}\left(x+1-\sqrt{x^2-x+2}\right)=\lim\limits_{x \rightarrow+\infty} \dfrac{3 x-1}{x+1+\sqrt{x^2-x+2}}=\lim\limits_{x \rightarrow+\infty} \dfrac{3-\dfrac{1}{x}}{1+\dfrac{1}{x}+\sqrt{1-\dfrac{1}{x}+\dfrac{2}{x^2}}}=\dfrac{3}{2} \text {. }
% 		$$
% 	}
% \end{ex}
% \begin{ex}%[Dự án soạn đề cương toán 11 - KNTT, Minh Trí]%[1K5BE-4]
% 	Cho $a$ là một số thực khác $0$ thỏa mãn $\lim\limits_{x \rightarrow a} \dfrac{x^4-a}{x-a}=4$.
% 	Khi đó $a$ bằng
% 	\choice
% 	{$4$}
% 	{$-1$}
% 	{\True $1$}
% 	{$-4$}
% 	\loigiai{
% 		Ta có
% 		$$
% 		\lim\limits_{x \rightarrow a} \dfrac{x^4-a}{x-a}=\lim\limits_{x \rightarrow a} \dfrac{(x-a)(x+a)\left(x^2+a^2\right)}{x-a}=\lim\limits_{x \rightarrow a}\left[(x+a)\left(x^2+a^2\right)\right]=4 a^3
% 		$$
% 		Mà theo giả thiết $\lim\limits_{x \rightarrow a} \dfrac{x^4-a}{x-a}=4$. Do đó $4 a^3=4 \Leftrightarrow a=1$.}
% \end{ex}
% \begin{ex}%[Dự án soạn đề cương toán 11 - KNTT, Minh Trí]%[1K5KE-4]
% 	Cho $a, b, c$ là các số thực khác 0 . Tìm hệ thức liên hệ giữa $a, b, c$ để $\lim\limits_{x \rightarrow-\infty} \dfrac{a x-b \sqrt{9 x^2+2}}{c x+1}=5$.
% 	\choice
% 	{$\dfrac{a-3 b}{c}=5$}
% 	{$\dfrac{a+3 b}{c}=-5$}
% 	{$\dfrac{a-3 b}{c}=-5$}
% 	{\True $\dfrac{a+3 b}{c}=5$}
% 	\loigiai{
% 		Ta có: $\lim\limits_{x \rightarrow-\infty} \dfrac{a x-b \sqrt{9 x^2+2}}{c x+1}=5 \Leftrightarrow \lim\limits_{x \rightarrow-\infty} \dfrac{a x-b|x| \sqrt{9+\dfrac{2}{x^2}}}{c x+1}=5 \Leftrightarrow \lim\limits_{x \rightarrow-\infty} \dfrac{a+b \sqrt{9+\dfrac{2}{x^2}}}{c+\dfrac{1}{x}}=5$ $\Leftrightarrow \dfrac{a+b \sqrt{9+0}}{c+0}=5 \Leftrightarrow \dfrac{a+3 b}{c}=5$.}
% \end{ex}
% \begin{ex}%[Dự án soạn đề cương toán 11 - KNTT, Minh Trí]%[1K5KE-4]
% 	Tính $\lim \limits_{n \to +\infty}\left(\dfrac{1}{n^2+4}+\dfrac{2}{n^2+4}+\dfrac{3}{n^2+4}+\ldots+\dfrac{2n+4}{n^2+4}\right)$.
% 	\choice
% 	{$\dfrac{1}{2}$}
% 	{$0$}
% 	{$1$}
% 	{\True $2$}
% 	\loigiai{
% 		Ta có $\lim \limits_{n \to +\infty}\left(\dfrac{1}{n^2+4}+\dfrac{2}{n^2+4}+\dfrac{3}{n^2+4}+\ldots+\dfrac{2 n+4}{n^2+4}\right)$
% 		$$
% 		\begin{aligned}
% 			& =\lim \limits_{n \to +\infty}\left(\dfrac{1+2+3+\ldots+2 n+4}{n^2+4}\right) \\
% 			& =\lim \limits_{n \to +\infty}\dfrac{(1+2 n+4)(2 n+4)}{2\left(n^2+4\right)} \\
% 			& =\lim \limits_{n \to +\infty}\dfrac{(2 n+5)(2 n+4)}{2\left(n^2+4\right)} \\
% 			& =\lim \limits_{n \to +\infty}\dfrac{\left(2+\dfrac{5}{n}\right)\left(2+\dfrac{4}{n}\right)}{2\left(1+\dfrac{4}{n^2}\right)}=2
% 		\end{aligned}
% 		$$
% 	}
% \end{ex}

% \begin{ex}%[Dự án soạn đề cương toán 11 - KNTT, Minh Trí]%[1K5GE-4]
% 	Gọi $S_1$ là diện tích tam giác đều $A_1 B_1 C_1$ cạnh bằng $a$. Gọi $S_2$ là diện tích tam giác $A_2 B_2 C_2$ vói các đỉnh trung điểm các cạnh $A_1 B_1, B_1 C_1, A_1 C_1$, gọi $S_3$ là diện tích tam giác $A_3 B_3 C_3$ với các định trung điểm các cạnh $A_2 B_2, B_2 C_2, A_2 C_2, \ldots$ và gọi $S_n$ là diện tích tam giác $A_n B_n C_n$ với các đính trung điểm các cạnh $A_{n-1} B_{n-1}, B_{n-1} C_{n-1}, A_{n-1} C_{n-1}$. Khi $n$ tiến về dương vô cực tính tổng $S=S_1+S_2+S_3+\ldots+S_n+\ldots$
% 	\begin{center}
% 		\begin{tikzpicture}[scale=1, font=\footnotesize, line join=round, line cap=round, >=stealth]
% 			\coordinate (A_1) at (90:4);
% 			\coordinate (B_1) at (210:4);
% 			\coordinate (C_1) at (-30:4);
% 			\draw (A_1)--(B_1)--(C_1)--(A_1);
% 			\def \goc{-90};
% 			\foreach \t in {2,3,4}
% 			{ \pgfmathsetmacro{\j}{\t-1}
% 				\coordinate (A_\t) at ($(C_\j)!1/2!(B_\j)$);
% 				\coordinate (B_\t) at ($(A_\j)!1/2!(C_\j)$);
% 				\coordinate (C_\t) at ($(A_\j)!1/2!(B_\j)$);
% 				\draw (A_\t)--(B_\t)--(C_\t)--(A_\t);
% 			}
% 			\foreach \t in {1,2,3,4}
% 			{\pgfmathsetmacro{\goca}{\goc + \t*180}
% 				\pgfmathsetmacro{\gocb}{\goca + 120}
% 				\pgfmathsetmacro{\gocc}{\gocb + 120}
% 				\foreach \x/\y in {A_\t/\goca,B_\t/\gocb,C_\t/\gocc}{\fill (\x) circle (1pt) ($(\x)+(\y:0.3cm)$) node{$\x$};}
% 			}
% 		\end{tikzpicture}
% 	\end{center}
% 	\choice
% 	{$S=\dfrac{4 \sqrt{3} a^2}{3}$}
% 	{$S=\dfrac{\sqrt{3} a^2}{4}$}
% 	{\True $S=\dfrac{\sqrt{3} a^2}{3}$}
% 	{$S=\dfrac{\sqrt{3} a^2}{2}$}	
% 	\loigiai{
% 		Ta có: $S_1=\dfrac{\sqrt{3}}{4}(a)^2, S_2=\dfrac{\sqrt{3}}{4}\left(\dfrac{a}{2}\right)^2=\dfrac{\sqrt{3} a^2}{16}, S_3=\dfrac{\sqrt{3}}{4}\left(\dfrac{a}{4}\right)^2=\dfrac{\sqrt{3} a^2}{64}, \ldots S_n=\dfrac{\sqrt{3}}{4}\left(\dfrac{a}{2^{x-1}}\right)^2=\dfrac{\sqrt{3} a^2}{4^{x-1}}$. \\
% 		Khi $n \rightarrow+\infty \Rightarrow \dfrac{1}{4^{n-1}} \rightarrow 0 \Rightarrow S_n \rightarrow 0$. Lúc đó: \\ $S=S_1+S_2+S_3+\ldots+S_n+\ldots$ là tồng cấp số nhân lùi
% 		vô hạn với $S_1=\dfrac{\sqrt{3}}{4}(a)^2$ và công bội $q=\dfrac{1}{4}$. Vậy tổng diện tích các hình là $S=S_1 \cdot \dfrac{1}{1-q}=\dfrac{\sqrt{3}}{4}(a)^2 \cdot \dfrac{4}{3}=\dfrac{\sqrt{3} a^2}{3}$.}
% \end{ex}
% \begin{ex}%[Dự án soạn đề cương toán 11 - KNTT, Minh Trí]%[1K5TE-4]
% 	Một quả bóng tenis được thả từ độ cao $81(m)$. Mỗi lần chạm đất, quả bóng lại nảy lên hai phần ba độ cao của lần rơi trước. Tính tổng các khoảng cách rơi và nảy của quả bóng từ lúc thả bóng cho đến lúc bóng không nảy nữa.
% 	\choice
% 	{$243(m)$}
% 	{\True $405(\mathrm{~m})$}
% 	{$486(\mathrm{~m})$}
% 	{$524(m)$}
% 	\loigiai{
% 		Đặt $h_1=81(m)$. Sau lần chạm đất đầu tiên, quả bóng nảy lên một độ cao $h_2=\dfrac{2}{3} h_1$. Tiếp đó, bóng roi từ độ cao $h_2$, chạm đất và nảy lên độ cao $h_3=\dfrac{2}{3} h_2$ rồi roi từ độ cao $h_3$ và cứ tiếp tụ như vậy. Sau lần chạm đất thứ $n$ từ độ cao $h_{n'}$ quả bóng nảy lên $h_{n+1}=\dfrac{2}{3} h_{n'}, \ldots$ \\
% 		Vậy tổng các khoảng cách rơi và nảy của quả bóng từ lúc thả bóng cho đến lúc bóng không nảy nữa là $d=\left(h_1+h_2+\ldots+h_n+\ldots\right)+\left(h_2+\ldots+h_n+\ldots\right) \Rightarrow d$ là tổng của hai cấp số nhân lùi vo hạn có số hạng đầu, theo thứ tự là $h_1, h_2$ và có cùng công bội $q=\dfrac{2}{3}$. Suy ra: $d=\dfrac{h_1}{1-\dfrac{2}{3}}+\dfrac{h_2}{1-\dfrac{2}{3}}=405(m)$.}
% \end{ex}
% \begin{ex}%[Dự án soạn đề cương toán 11 - KNTT, Minh Trí]%[1K5TE-4]
% 	Để trang trí cho một tấm bìa hình vuông có cạnh bằng $1$m, bạn $\mathrm{A}$ quyết định vẽ các hình vuông lên tấm bìa bằng cách: hình vuông thứ nhất có các đỉnh là trung điểm của các cạnh tấm bìa, hình vuông thứ hai có các đỉnh là trung điểm của các cạnh hình vuông thứ nhất,hình vuông thứ ba có các đỉnh là trung điểm của các cạnh hình vuông thứ hai,... Giả sử quy trình vẽ hình vuông của bạn $A$ có thể tiến ra vô hạn. Tính độ dài $L$ các nét vẽ hình vuông của bạn $A$.
% 	\begin{center}
% 		\begin{tikzpicture}[scale=0.8, font=\footnotesize, line join=round, line cap=round, >=stealth]
% 			\coordinate (A) at (0,0);
% 			\coordinate (B) at (6,0);
% 			\coordinate (D) at (0,6);
% 			\coordinate (C) at ($(B)+(D)-(A)$);
% 			\draw (A)--(B)--(C)--(D)--(A);
			
% 			\coordinate (A_1) at ($(A)!1/2!(B)$);
% 			\coordinate (B_1) at ($(C)!1/2!(B)$);
% 			\coordinate (C_1) at ($(C)!1/2!(D)$);
% 			\coordinate (D_1) at ($(A)!1/2!(D)$);
% 			\draw (A_1)--(B_1)--(C_1)--(D_1)--(A_1);
% 			\foreach \x in {2,3,4,5,6,7}
% 			{ \pgfmathsetmacro{\j}{\x-1}
% 				\coordinate (A_\x) at ($(A_\j)!1/2!(D_\j)$);
% 				\coordinate (B_\x) at ($(A_\j)!1/2!(B_\j)$);
% 				\coordinate (C_\x) at ($(C_\j)!1/2!(B_\j)$);
% 				\coordinate (D_\x) at ($(C_\j)!1/2!(D_\j)$);
% 				\draw (A_\x)--(B_\x)--(C_\x)--(D_\x)--(A_\x);
% 			}			
% 		\end{tikzpicture}
% 	\end{center}
% 	\choice
% 	{$1+\sqrt{2}$}
% 	{$2+\sqrt{2}$}
% 	{\True $4+4 \sqrt{2}$}
% 	{$8+4 \sqrt{2}$}
% 	\loigiai{
% 		Hình vuông thứ nhất có cạnh là $\dfrac{1}{2} \cdot \sqrt{2}=\dfrac{\sqrt{2}}{2}$ nên có chu vi $S_1=2 \sqrt{2}$,\\
% 		Hình vuông thứ hai có cạnh là $\dfrac{\sqrt{2}}{4} \cdot \sqrt{2}=\dfrac{1}{2}$ nên có chu vi $S_2=2$,\\
% 		Hình vuông thứ ba có cạnh là $\dfrac{1}{4} \cdot \sqrt{2}=\dfrac{\sqrt{2}}{4}$ nên có chu vi $S_3=\sqrt{2}$,\\
% 		Hình vuông thứ $n$ có cạnh là $\left(\dfrac{\sqrt{2}}{2}\right)^n$ nên có chu vi $S_n=4 .\left(\dfrac{\sqrt{2}}{2}\right)^n, \ldots$\\
% 		Khi đó độ dài các nét vẽ cạnh hình vuông là $S=S_1+S_2+\ldots+S_n+\ldots=\dfrac{2 \sqrt{2}}{1-\dfrac{\sqrt{2}}{2}}=4+4 \sqrt{2}$.}
% \end{ex}
% \begin{ex}%[Dự án soạn đề cương toán 11 - KNTT, Minh Trí]%[1K5GE-4]
% 	Tính giới hạn của dãy số $u_n=\dfrac{1}{2 \sqrt{1}+\sqrt{2}}+\dfrac{1}{3 \sqrt{2}+2 \sqrt{3}}+\ldots+\dfrac{1}{(n+1) \sqrt{n}+n \sqrt{n+1}}$ :
% 	\choice
% 	{$+\infty$}
% 	{$-\infty$}
% 	{$0$}
% 	{\True $1$}
% 	\loigiai{
% 		Ta có $: \dfrac{1}{(k+1) \sqrt{k}+k \sqrt{k+1}}=\dfrac{1}{\sqrt{k}}-\dfrac{1}{\sqrt{k+1}}$
% 		Suy ra $u_n=1-\dfrac{1}{\sqrt{n+1}} \Rightarrow \lim \limits_{n \to +\infty}u_n=1$}
% \end{ex}

% \begin{ex}%[Dự án soạn đề cương toán 11 - KNTT, Minh Trí]%[1K5GE-4]
% 	Với $n$ là số tự nhiên lớn hơn $2$ , đặt $S_n=\dfrac{1}{\mathrm{C}_3^3}+\dfrac{1}{\mathrm{C}_4^3}+\dfrac{1}{\mathrm{C}_5^3}+\ldots+\dfrac{1}{\mathrm{C}_n^3}$. Tính $\lim \limits_{n \to +\infty}S_n$
% 	\choice
% 	{$1$} 
% 	{$\dfrac{3}{2}$}
% 	{$3$}
% 	{$\dfrac{1}{3}$}
% 	\loigiai{
% 		$$
% 		\begin{aligned}
% 			& S_n=\dfrac{1}{\mathrm{C}_3^3}+\dfrac{1}{\mathrm{C}_4^3}+\dfrac{1}{\mathrm{C}_5^3}+\ldots+\dfrac{1}{\mathrm{C}_n^3}=\dfrac{3!}{1\cdot2\cdot3}+\dfrac{3 !}{2\cdot3\cdot 4}+\dfrac{3!}{3\cdot 4\cdot 5}+\ldots+\dfrac{3!}{n(n-1)(n-2)} \\
% 			& =6\left[\dfrac{1}{2}\left(-\dfrac{1}{3.2}+\dfrac{1}{2.1}-\dfrac{1}{4.3}+\dfrac{1}{3\cdot 2}+\ldots-\dfrac{1}{n(n-1)}+\dfrac{1}{(n-1)(n-2)}\right)\right]=3\left(\dfrac{1}{2\cdot 1}-\dfrac{1}{n(n-1)}\right)
% 		\end{aligned}
% 		$$
% 		Vậy $\lim \limits_{n \to +\infty}S_n=\lim \limits_{n \to +\infty}3\left(\dfrac{1}{2}-\dfrac{1}{n(n-1)}\right)=\dfrac{3}{2}$}
% \end{ex} 
% \begin{ex}%[Dự án soạn đề cương toán 11 - KNTT, Minh Trí]%[1K5GE-4]
% 	Cho $f(n)=\left(n^2+n+1\right)^2+1$. Xét dãy số $\left(u_n\right)$ sao cho $u_n=\dfrac{f(1) \cdot f(3) \cdot f(5) \ldots f(2 n-1)}{f(2) \cdot f(4) \cdot f(6) \ldots f(2 n)}$. Tính $\lim \limits_{n \to +\infty}n \sqrt{u_n}$
% 	\choice
% 	{\True $\lim \limits_{n \to +\infty}n \sqrt{u_n}=\dfrac{1}{\sqrt{2}}$}
% 	{$\lim \limits_{n \to +\infty}n \sqrt{u_n}=\sqrt{2}$}
% 	{$\lim \limits_{n \to +\infty}n \sqrt{u_n}=\sqrt{3}$}
% 	{$\lim \limits_{n \to +\infty}n \sqrt{u_n}=\dfrac{1}{\sqrt{3}}$}
% 	\loigiai{
% 		$g(n)=\dfrac{f(2 n-1)}{f(2 n)} \Rightarrow g(n)=\dfrac{\left(4 n^2-2 n+1\right)^2+1}{\left(4 n^2+2 n+1\right)^2+1}$ \\ 
% 		Đặt $\left\{\begin{array}{l}a=4 n^2+1 \\ b=2 n\end{array} \Rightarrow\left\{\begin{array}{l}a=b^2+1 \\ a-2 b=(2 n-1)^2 \\ a+2 b=(2 n+1)^2\end{array}\right.\right.$.\\ 
% 		Suy ra $g(n)=\dfrac{(a-b)^2+1}{(a+b)^2+1}=\dfrac{a^2-2 a b+b^2+1}{a^2+2 a b+b^2+1}=\dfrac{a^2-2 a b+a}{a^2+2 a b+b}=\dfrac{a-2 b+1}{a+2 b+1}=\dfrac{(2 n-1)^2+1}{(2 n+1)^2+1}$. \\
% 		$\Rightarrow u_n=g(1) g(2) \ldots \ldots g(n)=\dfrac{2}{10} \cdot \dfrac{10}{26} \ldots \ldots \dfrac{(2 n-1)^2+1}{(2 n+1)^2+1}=\dfrac{2}{(2 n+1)^2+1}$.
% 		$\lim \limits_{n \to +\infty}n \sqrt{u_n}=\lim \limits_{n \to +\infty}n \cdot \sqrt{\dfrac{2}{(2 n+1)^2+1}}=\dfrac{1}{\sqrt{2}}$}
% \end{ex}
\Closesolutionfile{ans}
% \begin{indapan}{10}
% 	{ans/ans-1K5-1-Dang3}
% \end{indapan}
\begin{dang}{Tính tổng của dãy cấp số nhân lùi vô hạn}
	\begin{dn}
		Cấp số nhân vô hạn $u_1, u_1q,...,u_1q^{n-1},...$ có công bội $q$ thỏa mãn $|q|<1$ được gọi là cấp số nhân lùi vô hạn. 
		Tổng của cấp số nhân lùi vô hạn đã cho là $$S=u_1+u_1q+u_1q^2+...=\dfrac{u_1}{1-q}.$$
	\end{dn}
\end{dang}
\subsubsection{Ví dụ minh hoạ}
\begin{vd}%[1C3B1-5]
	Cho cấp số nhân $(u_n)$, với $u_1=1$ và công bội $q=\dfrac{1}{2}$.
	\begin{enumEX}{1}
		\item So sánh $\left|q\right|$ với $1$.
		\item Tính $S_n=u_1+u_2+\cdots+u_n$ từ đó hãy tính $\lim \limits_{n \to +\infty}S_n$.
	\end{enumEX}
	\loigiai{
		\begin{enumerate}
			\item Ta có $\left|q\right|=\left|\dfrac{1}{2}\right|=\dfrac{1}{2}<1$.
			\item Ta có $S_n=\dfrac{u_1\left(1-q^n\right)}{1-q}=\dfrac{1\cdot\left[1-\left(\dfrac{1}{2}\right)^n\right]}{1-\dfrac{1}{2}}=2\cdot\left(1-\dfrac{1}{2^n}\right)=2-\dfrac{1}{2^{n-1}}$.\\
			Khi đó $\lim \limits_{n \to +\infty}S_n=\lim \limits_{n \to +\infty}\left(2-\dfrac{1}{2^{n-1}}\right)=2$.
		\end{enumerate}
	}
\end{vd}
\begin{vd}%[1C3Y1-5]
	Tính tổng $T=1+\dfrac{1}{3}+\dfrac{1}{3^2}+\ldots+\dfrac{1}{3^n}+\ldots$
	\loigiai{
		Các số hạn của tổng lập thành câp số nhân $(u_n)$, có $u_1=1$, $q=\dfrac{1}{3}$ nên\\ $$T=1+\dfrac{1}{3}+\dfrac{1}{3^2}+\ldots+\dfrac{1}{3^n}+\ldots=\dfrac{1}{1-\dfrac{1}{3}}=\dfrac{2}{3}\cdot$$
	}
\end{vd}
\begin{vd}
	Tính tổng $S=1-\dfrac{1}{2}+\dfrac{1}{4}-\dfrac{1}{8}+\ldots+\left(-\dfrac{1}{2}\right)^{n-1}+\ldots$.
	\loigiai{
		Đây là tổng của cấp số nhân lùi vô hạng với $u_{1}=1$ và $q=-\dfrac{1}{2}$. Do đó
		$
		S=\dfrac{u_{1}}{1-q}=\dfrac{1}{1-\left(-\dfrac{1}{2}\right)}=\dfrac{2}{3}.
		$	
	}
\end{vd}

\begin{vd}%[1C3B1-5]
	Biểu diễn số thập phân vô hạn tuần hoàn $2{,}222 \ldots$ dưới dạng phân số.
	\loigiai{
		Ta có $2{,}222 \ldots=2+0{,}2+0{,}02+0{,}002+\ldots=2+2 \cdot 10^{-1}+2 \cdot 10^{-2}+2 \cdot 10^{-3}+\ldots$.\\
		Đây là tổng của cấp số nhân lùi vô hạn với $u_{1}=2, q=10^{-1}$ nên
		$$
		2{,}222 \ldots=\dfrac{u_{1}}{1-q}=\dfrac{2}{1-\dfrac{1}{10}}=\dfrac{20}{9}.
		$$
	}
\end{vd}
\begin{vd}%[1C3B1-5]
	Biểu diễn số thập phân vô hạn tuần hoàn $0,(3)$ dưới dạng phân số.
	\loigiai{
		Ta có  $0,(3)=\dfrac{3}{10}+\dfrac{3}{10^2}+\ldots+\dfrac{3}{10^n}+\ldots=\dfrac{\dfrac{3}{10}}{1-\dfrac{1}{10}}=\dfrac{1}{3}\cdot$
	}
\end{vd}
% \subsubsection{Bài tập rèn luyện} 
% % \subsubsection{Bài tập tự luận}
% \begin{bt}%[1C3B1-5]
% 	\begin{enumEX}{1}
% 		\item[] 
% 		\item Tính tổng cấp số nhân lùi vô hạn $(u_n)$ với $u_1=\dfrac{2}{3}, q=-\dfrac{1}{4}$.
% 		\item Biểu diễn số thập phân vô hạn tuần hoàn $1,(6)$ dưới dạng phân số.
% 	\end{enumEX}
% 	\loigiai{
% 		\begin{enumerate}
% 			\item Ta có  $S=\dfrac{u_1}{1-q}=\dfrac{\dfrac{2}{3}}{1-\left(-\dfrac{1}{4}\right)}=\dfrac{8}{15}$.
% 			\item Ta có  $1,(6)=1+0,(6)=1+\dfrac{6}{10}+\dfrac{6}{10^2}+\cdots+\dfrac{6}{10^n}+\cdots=1+\dfrac{\dfrac{6}{10}}{1-\dfrac{1}{10}}=\dfrac{5}{3}$.
% 		\end{enumerate}
% 	}
% \end{bt}
% \begin{bt}%[1T3B1-6]
% 	Tính tổng của các cấp số nhân lùi vô hạn sau
% 	\begin{listEX}[2]
% 		\item $-\dfrac{1}{2}+\dfrac{1}{4}-\dfrac{1}{8}+\cdots+\left(-\dfrac{1}{2}\right)^n+\cdots$.
% 		\item $\dfrac{1}{4}+\dfrac{1}{16}+\dfrac{1}{64}+\cdots+\left(\dfrac{1}{4}\right)^n+\cdots$.
% 	\end{listEX}
% 	\loigiai{
% 		\begin{enumerate}
% 			\item Tổng trên là tổng của cấp số nhân lùi vô hạn có số hạng đầu $u_1=-\dfrac{1}{2}$ và công bội $q=-\dfrac{1}{2}$ nên 
% 			\[-\dfrac{1}{2}+\dfrac{1}{4}-\dfrac{1}{8}+\cdots+\left(-\dfrac{1}{2}\right)^n+\cdots=\dfrac{-\dfrac{1}{2}}{1-\left(-\dfrac{1}{2}\right)}=-\dfrac{1}{3}. \]
% 			\item Tổng trên là tổng của cấp số nhân lùi vô hạn có số hạng đầu $u_1=\dfrac{1}{4}$ và công bội $q=\dfrac{1}{4}$ nên 
% 			\[\dfrac{1}{4}+\dfrac{1}{16}+\dfrac{1}{64}+\cdots+\left(\dfrac{1}{4}\right)^n+\cdots=\dfrac{\dfrac{1}{4}}{1-\dfrac{1}{4}}=\dfrac{1}{3}. \]
% 		\end{enumerate}
% 	}
% \end{bt}

% \begin{bt}%[1T3B1-5]
% 	Tính tổng của cấp số nhân lùi vô hạn: $1-\dfrac{1}{4}+\dfrac{1}{16}-\dfrac{1}{64}+\cdots+\left(-\dfrac{1}{4}\right)^n+\cdots$.
% 	\loigiai{
% 		Tổng trên là tổng của cấp số nhân lùi vô hạn có số hạng đầu $u_1=1$ và công bội $q=-\dfrac{1}{4}$ nên
% 		\[ 1-\dfrac{1}{4}+\dfrac{1}{16}-\dfrac{1}{64}+\cdots+\left(-\dfrac{1}{4}\right)^n+\cdots=\dfrac{1}{1-\left(-\dfrac{1}{4}\right)}=\dfrac{4}{5}.\]
% 	}
% \end{bt}

% \begin{bt}%[1T3B1-5]
% 	Biết rằng có thể coi số thập phân vô hạn tuần hoàn $0{,}666 \ldots$ là tổng của một cấp số nhân lùi vô hạn:
% 	\[ 0{,}666 \ldots=0{,}6+0{,}06+0{,}006+\cdots=0{,}6+0{,}6 \cdot \dfrac{1}{10}+0{,}6 \cdot \dfrac{1}{10^2}+\cdots.
% 	\]
% 	Hãy viết $0{,}666 \ldots$ dưới dạng phân số.
% 	\loigiai{
% 		Số $0{,}666 \ldots$ là tổng của cấp số nhân lùi vô hạn có số hạng đầu bằng $0{,}6$ và công bội bằng $\dfrac{1}{10}$.\\
% 		Do đó $0{,}666\ldots=\dfrac{0{,}6}{1-\dfrac{1}{10}}=\dfrac{6}{9}=\dfrac{2}{3}$.
% 	}
% \end{bt}

% \begin{bt}%[1T3B1-5]
% 	Tính tổng của cấp số nhân lùi vô hạn: $1+\dfrac{1}{3}+\left(\dfrac{1}{3}\right)^2+\cdots+\left(\dfrac{1}{3}\right)^n+\cdots$.	
% 	\loigiai{
% 		Tổng trên là tổng của cấp số nhân lùi vô hạn có số hạng đầu $u_1=1$ và công bội $q=\dfrac{1}{3}$ nên
% 		\[ 1+\dfrac{1}{3}+\left(\dfrac{1}{3}\right)^2+\cdots+\left(\dfrac{1}{3}\right)^n+\cdots=\dfrac{1}{1-\dfrac{1}{3}}= \dfrac{3}{2}.\]
% 	}
% \end{bt}
\subsubsection{Câu hỏi trắc nghiệm}
\Opensolutionfile{ans}[ans/ans-1K5-1-Dang4]
\begin{ex}%[1D4B1-5]
	Cho cấp số nhân $u_1,u_2,\ldots$ với công bội $q$ thỏa điều kiện $|q|<1$. Lúc đó, ta nói cấp số nhân đã cho là lùi vô hạn. Tổng của cấp số nhân đã cho là $S=u_1+u_2+u_3+\cdots +u_n+\cdots$ bằng 
	\choice
	{$\dfrac{u_1}{q-1}$}
	{$\dfrac{u_1\left(q^n-1\right)}{q-1}$}
	{$\dfrac{u_1}{1+q}$}
	{\True $\dfrac{u_1}{1-q}$}
	\loigiai{
		Theo định nghĩa cấp số nhân lùi vô hạn ta chứng minh được.\\
		$S=u_1+u_2+u_3+\cdots +u_n+\cdots =u_1+u_1q^1+u_1q^2+\cdots +u_1q^{n-1}+\cdots =\dfrac{u_1}{1-q}$.}
\end{ex}
\begin{ex}%[1D4B1-5]
	Gọi $S=\dfrac{1}{3}-\dfrac{1}{9}+\cdots +\dfrac{(-1)^{n+1}}{3^n}$. Khi đó, $\lim \limits_{n \to +\infty}S$ bằng 
	\choice
	{$\dfrac{3}{4}$}
	{\True $\dfrac{1}{4}$}
	{$\dfrac{1}{2}$}
	{$1$}
	\loigiai{
		Ta có 
		\allowdisplaybreaks
		\begin{eqnarray*}		
			&&S=\dfrac{1}{3}-\dfrac{1}{9}+\cdots +\dfrac{(-1)^{n+1}}{3^n} \\
			&\Leftrightarrow& S=-\left(-\dfrac{1}{3}+\dfrac{1}{9}+\cdots +\dfrac{(-1)^n}{3^n}\right) \\
			&\Leftrightarrow& S=\dfrac{1}{3}\cdot\dfrac{1-\left(\dfrac{-1}{3}\right)^n}{1-\dfrac{-1}{3}}\\
			&\Leftrightarrow& S=\dfrac{1}{4}\cdot\left(1-\left(\dfrac{-1}{3}\right)^n\right).
		\end{eqnarray*}
		Suy ra $\lim \limits_{n \to +\infty}S=\lim \limits_{n \to +\infty}\dfrac{1}{4}\cdot\left(1-\left(\dfrac{-1}{3}\right)^n\right)=\dfrac{1}{4}$.	
	}
\end{ex}
\begin{ex}%[1D4B1-5]
	Tổng $S=\dfrac{1}{3}+\dfrac{1}{3^2}+\cdots +\dfrac{1}{3^n}+\cdots$ có giá trị là 
	\choice
	{$\dfrac{1}{3}$}
	{\True $\dfrac{1}{2}$}
	{$\dfrac{1}{9}$}
	{$\dfrac{1}{4}$}
	\loigiai{
		Ta có $S=\dfrac{1}{3}+\dfrac{1}{3^2}+\cdots +\dfrac{1}{3^n}+\cdots=\dfrac{1}{3}\cdot\dfrac{1}{1-\frac{1}{3}}=\dfrac{1}{2}$. 
	}
\end{ex}
\begin{ex}%[1D4B1-5]
	Tính $S=9+3+1+\dfrac{1}{3}+\dfrac{1}{9}+\cdots +\dfrac{1}{3^{n-3}}+\cdots$. Kết quả là 
	\choice
	{\True $\dfrac{27}{2}$}
	{$14$}
	{$16$}
	{$15$}
	\loigiai{
		Ta có $S=9+3+1+\dfrac{1}{3}+\dfrac{1}{9}+\cdots +\dfrac{1}{3^{n-3}}+\cdots=13+\dfrac{1}{3}\cdot\dfrac{1}{1-\frac{1}{3}}=13+\dfrac{1}{2}=\dfrac{27}{2}$. 
	}
\end{ex}
\begin{ex}%[1D4B1-5]
	Tổng các cấp số nhân vô hạn: $1,-\dfrac{1}{2},\dfrac{1}{4},-\dfrac{1}{8},\ldots,\dfrac{(-1)^{n+1}}{2^{n-1}},\ldots$ là
	\choice
	{$\dfrac{3}{2}$}
	{\True $\dfrac{2}{3}$}
	{$-\dfrac{2}{3}$}
	{$2$}
	\loigiai{
		Ta có $S=1-\dfrac{1}{2}+\dfrac{1}{4}-\dfrac{1}{8}+\cdots +\dfrac{(-1)^{n+1}}{2^{n-1}}+\cdots=1-\dfrac{1}{2}\cdot\dfrac{1}{1+\frac{1}{2}}=1-\dfrac{1}{3}=\dfrac{2}{3}$.
	}
\end{ex}
\begin{ex}%[1D4B1-5]
	Gọi $S=1+\dfrac{2}{3}+\dfrac{4}{9}+\cdots +\dfrac{2^n}{3^n}+\cdots$. Giá trị của $S$ bằng
	\choice
	{\True $3$}
	{$5$}
	{$6$}
	{$4$}
	\loigiai{
		Ta có $S=1+\dfrac{2}{3}+\dfrac{4}{9}+\cdots +\dfrac{2^n}{3^n}+\cdots=1+\dfrac{2}{3}\cdot\dfrac{1}{1-\frac{2}{3}}=1+2=3$.
	}
\end{ex}
\begin{ex}%[1D4B1-5]%
	Số thập phân vô hạn tuần hoàn $0{,}233333\ldots$ biểu diễn dưới dạng số là 
	\choice
	{$\dfrac{1}{23}$}
	{$\dfrac{2333}{10000}$}
	{$\dfrac{23333}{10^5}$}
	{\True $\dfrac{7}{30}$}
	\loigiai{
		$0{,}233333\ldots=0{,}2+3\left(\dfrac{1}{10^2}+\dfrac{1}{10^3}+\ldots\right)=0{,}2+3\cdot\dfrac{1}{100}\cdot\dfrac{1}{1-\frac{1}{10}}=\dfrac{1}{5}+\dfrac{1}{30}=\dfrac{7}{30}$.
	}
\end{ex}
\begin{ex}%[1D4B1-5]%
	Số thập phân vô hạn tuần hoàn $0{,}212121\ldots$ biểu diễn dưới dạng phân số là 
	\choice
	{$\dfrac{2121}{10^4}$}
	{$\dfrac{1}{21}$}
	{\True $\dfrac{7}{33}$}
	{$\dfrac{212121}{10^6}$}
	\loigiai{
		$0{,}212121\ldots=21\left(\dfrac{1}{10^2}+\dfrac{1}{10^4}+\ldots\right)=21\cdot\dfrac{1}{10^2}\cdot\dfrac{1}{1-\frac{1}{100}}=\dfrac{7}{33}$.
	}
\end{ex}

\begin{ex}%[1D4B1-5]
	Số thập phân vô hạn tuần hoàn $0{,}271414\ldots$ được biểu diễn bằng phân số: 
	\choice
	{$\dfrac{2714}{9900}$}
	{$\dfrac{2617}{9900}$}
	{$\dfrac{2786}{9900}$}
	{\True $\dfrac{2687}{9900}$}
	\loigiai{
		$0{,}271414\cdots=0{,}27+14\left(\dfrac{1}{10^4}+\dfrac{1}{10^6}\ldots\right)=0{,}27+14\cdot\dfrac{1}{10^4}\cdot\dfrac{1}{1-\frac{1}{100}}=\dfrac{27}{100}+\dfrac{7}{4950}=\dfrac{2687}{9900}$.
	}
\end{ex}

\begin{ex}%[1D4B1-5]
	Tổng của cấp số nhân lùi vô hạn: $-\dfrac{1}{2},\dfrac{1}{4},\dfrac{1}{8},\ldots,\dfrac{(-1)^n}{2^n},\ldots$ là 
	\choice
	{\True $-\dfrac{1}{3}$}
	{$-\dfrac{1}{4}$}
	{$-1$}
	{$\dfrac{1}{2}$}
	\loigiai{
		Từ $-\dfrac{1}{2},\dfrac{1}{4},\dfrac{1}{8},\ldots,\dfrac{(-1)^n}{2^n},\ldots$ có $u_1=-\dfrac{1}{2}$ và $q=-\dfrac{1}{2}$.\\
		Có $S=-\dfrac{1}{2}+\dfrac{1}{4}+\dfrac{1}{8}+\cdots +\dfrac{(-1)^n}{2^n}+\cdots =\dfrac{\left(-\frac{1}{2}\right)}{1-\left(-\frac{1}{2}\right)}=-\dfrac{1}{3}$.}
\end{ex}
\begin{ex}%[1D4B1-5]
	Số thập phân vô hạn tuần hoàn $0{,}511111\cdots$ được biểu diễn bởi phân số
	\choice
	{$\dfrac{47}{90}$}
	{\True $\dfrac{46}{90}$}
	{$\dfrac{6}{11}$}
	{$\dfrac{43}{90}$}
	\loigiai{
		Ta có\\
		$0{,}511111\cdots=0{,}5+\dfrac{1}{10^2}+\dfrac{1}{10^3}\ldots=\dfrac{1}{2}+\dfrac{1}{10^2}\cdot\dfrac{1}{1-\frac{1}{10}}=\dfrac{1}{2}+\dfrac{1}{90}=\dfrac{23}{45}$.
	}
\end{ex}
\begin{ex}%[1D4B1-5]
	Tổng của cấp số nhân vô hạn $\dfrac{1}{2}$, $-\dfrac{1}{4}$, $\dfrac{1}{8},\ldots\dfrac{(-1)^{n+1}}{2^n},\ldots$ là
	\choice
	{$-\dfrac{2}{3}$}
	{$1$}
	{$-\dfrac{1}{3}$}
	{\True $\dfrac{1}{3}$}
	\loigiai{
		Ta có $S=\dfrac{1}{2}-\dfrac{1}{4}+\dfrac{1}{8}+\cdots +\dfrac{(-1)^{n+1}}{2^n}+\cdots =\dfrac{1}{2}\cdot\dfrac{1}{1+\frac{1}{2}}=\dfrac{1}{3}$.}
\end{ex}
\begin{ex}%[1D4B1-5]
	Tổng của cấp số nhân lùi vô hạn $\dfrac{1}{2},-\dfrac{1}{6},\dfrac{1}{18},\ldots,\dfrac{(-1)^{n+1}}{2\cdot 3^{n-1}},\ldots$ là
	\choice
	{$\dfrac{3}{4}$}
	{$\dfrac{8}{3}$}
	{$\dfrac{2}{3}$}
	{\True $\dfrac{3}{8}$}
	\loigiai{
		Cấp số nhân có $u_1=\dfrac{1}{2}, q=-\dfrac{1}{3}$. Do đó tổng cần tìm là
		$$S=\dfrac{u_1}{1-q}=\dfrac{\dfrac{1}{2}}{1+\frac{1}{3}}=\dfrac{1}{2}\cdot\dfrac{3}{4}=\dfrac{3}{8}.$$}
\end{ex}
\begin{ex}%[1D4B1-5]
	Tổng của cấp số nhân lùi vô hạn $\dfrac{1}{3},-\dfrac{1}{9},\dfrac{1}{27},\ldots\cdot,\dfrac{(-1)^{n+1}}{3^n},\ldots$ là
	\choice
	{$4$}
	{$\dfrac{1}{2}$}
	{$\dfrac{3}{4}$}
	{\True $\dfrac{1}{4}$}
	\loigiai{
		Cấp số nhân có $u_1=\dfrac{1}{3}, q=-\dfrac{1}{3}$. Do đó tổng cần tìm là\\
		$$S=\dfrac{u_1}{1-q}=\dfrac{\dfrac{1}{3}}{1+\frac{1}{3}}=\dfrac{1}{3}\cdot\dfrac{3}{4}=\dfrac{1}{4}.$$}
\end{ex}
\begin{ex}%[1D4B1-5]
	Số thập phân vô hạn tuần hoàn $0{,}17232323\ldots$ được biểu diễn bởi phân số?
	\choice
	{$\dfrac{1706}{9900}$}
	{$\dfrac{153}{990}$}
	{$\dfrac{164}{990}$}
	{\True $\dfrac{853}{4950}$}
	\loigiai{
		$0{,}17232323\ldots=0{,}17+23\left(\dfrac{1}{10^4}+\dfrac{1}{10^6}+\ldots\right)=\dfrac{17}{100}+23\cdot\dfrac{1}{10^4}\cdot\dfrac{1}{1-\frac{1}{100}}=\dfrac{17}{100}+\dfrac{23}{9900}=\dfrac{853}{4950}$.
	}
\end{ex}
\Closesolutionfile{ans}
% \begin{indapan}{10}
% 	{ans/ans-1K5-1-Dang4}
% \end{indapan}

\begin{dang}{Toán thực tế, liên môn liên quan đến giới hạn dãy số}
	$$S=u_1+u_1q+u_1q^2+...=\dfrac{u_1}{1-q}.$$
\end{dang}
\subsubsection{Ví dụ minh hoạ}
\begin{vd}%[1C3K1-6]
	\immini{Từ hình vuông có độ dài cạnh bằng $1$, người ta nối các trung điểm của cạnh hình vuông để tạo ra hình vuông mới như hình bên. Tiếp tục quá trình này đến vô hạn.
		\begin{enumEX}{1}
			\item Tính diện tích $S_n$ của hình vuông được tạo thành từ bước thứ $n$.
			\item Tính tổng diện tích của tất cả các hình vuông được tạo thành.
	\end{enumEX}}	
	{
		\begin{tikzpicture}[scale=0.6, font=\footnotesize,>=stealth]
			\def\canhAD{6};
			\coordinate (D) at (0,0);
			\coordinate (A) at ($(D)+(90:\canhAD)$);
			\coordinate (C) at ($(D)+(0:\canhAD)$);
			\coordinate (B) at ($(A)+(0:\canhAD)$);
			\coordinate (E) at ($(A)!0.5!(B)$);
			\coordinate (F) at ($(B)!0.5!(C)$);
			\coordinate (G) at ($(C)!0.5!(D)$);
			\coordinate (H) at ($(D)!0.5!(A)$);
			\coordinate (K) at ($(E)!0.5!(F)$);
			\coordinate (L) at ($(F)!0.5!(G)$);
			\coordinate (M) at ($(G)!0.5!(H)$);
			\coordinate (J) at ($(H)!0.5!(E)$);
			\coordinate (I) at ($(E)!0.5!(G)$);
			\coordinate (X) at ($(J)!0.5!(K)$);
			\coordinate (Y) at ($(K)!0.5!(L)$);
			\coordinate (T) at ($(L)!0.5!(M)$);
			\coordinate (Q) at ($(M)!0.5!(J)$);
			\draw(A) rectangle (C)(E)--(F)--(G)--(H)--cycle (J)--(K)--(L)--(M)--cycle (X)--(Y)--(T)--(Q)--cycle;
			\draw [>=stealth,|<->|] ([shift=({0,-0.4})]D)--([shift=({0,-0.4})]C) node [midway,below]{$1$};
			
			%			\foreach \x/\y in {A/90,D/-90,C/-90,B/90,E/90,G/-90,H/180,F/0,K/90,L/-90,M/-90,J/90,I/-110}{\fill (\x) circle(1pt) ($(\x)+(\y:0.3cm)$) node{$\x$};}
		\end{tikzpicture}
	}
	
	\loigiai{
		\begin{enumerate}
			\item Từ giả thiết suy ra diện tích hình vuông sau bằng $\dfrac{1}{2}$ diện tích hình vuông trước.\\ 
			Khi đó diện tích của các hình vuông tạo thành một cấp số nhân lùi vô hạn với số hạng đầu $S_1=1$ và công bội $q=\dfrac{1}{2}$.\\
			Diện tích $S_n$ của hình vuông được tạo thành từ bước thứ $n$ là $S_n=S_1\cdot q^{n-1}=\left(\dfrac{1}{2}\right)^{n-1}$.
			\item Tổng diện tích của tất cả các hình vuông được tạo thành là:\\
			$$S=\dfrac{u_1}{1-q}=\dfrac{1}{1-\dfrac{1}{2}}=2.$$
		\end{enumerate}
	}
\end{vd}
\begin{vd}%[1C3K1-6]
	Có $1$ kg chất phóng xạ độc hại. Biết rằng, cứ sau một khoảng thời gian $T=24000$ năm thì một nửa số chất phóng xạ này bị phân rã thành chất khác không độc hại đối với sức khỏe của con người ($T$ được gọi là \textit{chu kì bán rã}).
	\begin{flushright}
		(\textit{Nguồn: Đại số và giải tích 11, NXB GD Việt Nam, 2021})
	\end{flushright}
	Gọi $u_n$ là khối lượng chất phóng xạ còn lại sau chu kì thứ $n$.
	\begin{enumerate}
		\item  Tìm số hạng tổng quát $u_n$ của dãy số $(u_n)$.
		\item  Chứng minh rằng $(u_n)$ có giới hạn là $0$.
		\item  Từ kết quả câu $2$, chứng tỏ rằng sau một số năm nào đó khối lượng phóng xạ đã cho ban đầu không còn độc hại với con người, biết rằng chất phóng xạ này sẽ không độc hại nữa nếu khối lượng chất phóng xạ còn lại bé hơn $10^{-6}$ g.
	\end{enumerate}
	\loigiai{
		\begin{enumerate}
			\item  Khối lượng chất phóng xạ còn lại sau chu kì bán rã thứ $1$ là $u_1=\dfrac{1}{2}\cdot 1=1$ kg.\\
			Khối lượng chất phóng xạ còn lại sau chu kì bán rã thứ $2$ là $u_2=\dfrac{1}{2}\cdot u_1=\dfrac{1}{2}\cdot \dfrac{1}{2}=\dfrac{1}{2^2}$ kg.\\
			Khối lượng chất phóng xạ còn lại sau chu kì bán rã thứ $3$ là $u_3=\dfrac{1}{2}\cdot u_2=\dfrac{1}{2}\cdot \dfrac{1}{4}=\dfrac{1}{2^3}$ kg.\\
			Khối lượng chất phóng xạ còn lại sau chu kì bán rã thứ $n$ là $u_n=\dfrac{1}{2^n}$ kg.\\
			\item $\lim \limits_{n \to +\infty}u_n=\lim \limits_{n \to +\infty}\dfrac{1}{2^n}=\lim\left(\dfrac{1}{2}\right)^n=0$.
			\item Chất phóng xạ sẽ không độc hại nữa nếu khối lượng chất phóng xạ còn lại bé hơn $10^{-6}~\mathrm{g}=10^{-9}$ kg
			$$\Leftrightarrow u_n<10^{-9}\Leftrightarrow\dfrac{1}{2^n}<10^{-9}\Leftrightarrow 2^n>10^9\Leftrightarrow n\geq 30.$$
			Vậy sau ít nhất $30$ chu kì bằng $30\cdot 24000=720000$ năm thì khối lượng phóng xạ đã cho ban đầu không còn độc hại với con người nữa.
		\end{enumerate}
	}
\end{vd}
\begin{vd}%[1C3G1-6]
	Gọi $C$ là nữa đường tròn đường kính $AB=2R$.\\
	$C_1$ là đường gồm hai nửa đường tròn đường kính $\dfrac{AB}{2}$,\\
	$C_2$ là đường gồm bốn nửa đường tròn đường kính $\dfrac{AB}{4},\cdots$\\
	$C_n$ là đường gồm $2^n$ nửa đường tròn đường kính $\dfrac{AB}{2^n},\cdots$
	\immini{	Gọi $p_n$ là độ dài của $C_n$, $S_n$ là diện tích hình phẳng giới hạn bởi $C_n$ và đoạn thẳng $AB$.
		\begin{enumEX}{1}
			\item Tính $p_n$, $S_n$.
			\item Tính giới hạn của các dãy số $(p_n)$ và $(S_n)$.
		\end{enumEX}
	}{
		\begin{tikzpicture}[declare function={r=3;}]
			\path
			(0,0) coordinate (O)
			(0:r) coordinate (A)
			(180:r) coordinate (B)
			($(A)!.5!(O)$) coordinate (M)
			($(B)!.5!(O)$) coordinate (N)
			($(B)!.5!(N)$) coordinate (I)
			($(N)!.5!(O)$) coordinate (J)
			($(O)!.5!(M)$) coordinate (K)
			($(M)!.5!(A)$) coordinate (H)
			($(B)!.5!(I)$) coordinate (Q);
			\draw (A) arc(0:180:r) (A) arc(0:180:r/2) (O) arc(0:180:r/2) (A) arc(0:180:r/4) (A) arc(0:180:r/4)(M) arc(0:180:r/4) (O) arc(0:180:r/4) (N) arc(0:180:r/4)
			(A) arc(0:180:r/8) (H) arc(0:180:r/8) (M) arc(0:180:r/8) (K) arc(0:180:r/8) (O) arc(0:180:r/8) (J) arc(0:180:r/8) (N) arc(0:180:r/8) (I) arc(0:180:r/8);
			\draw (A)--(B);
			\path 
			($(O)+(90:0.9*r)$) node{$C$}
			($(N)+(90:0.4*r)$) node[scale=0.8]{$C_1$}
			($(I)+(90:0.18*r)$) node[scale=0.7]{$C_2$}
			($(Q)+(90:0.06*r)$) node[scale=0.6]{$C_3$};
			\foreach \x/\y in {A/-90, B/-90}{\fill (\x) circle(1pt) ($(\x)+(\y:0.3cm)$) node{$\x$};}
		\end{tikzpicture}
	}
	
	\loigiai{
		\begin{enumerate}
			\item Ta có  
			\begin{eqnarray*}
				p_n&=&2^n \cdot \pi r=2^n\cdot\pi \cdot \dfrac{AB}{2\cdot2^n}\\
				&=&\dfrac{\pi AB}{2}\\
				&=&\dfrac{\pi \cdot 2R}{2}\\
				&=&\pi R.
			\end{eqnarray*}
			\begin{eqnarray*}
				S_n&=&2^n \cdot \dfrac{1}{2}\pi r^2\\
				&=&2^n \cdot \dfrac{1}{2}\pi \left(\dfrac{AB}{2\cdot2^n}\right)^2\\
				&=&2^n \cdot \dfrac{1}{2}\pi \left(\dfrac{2R}{2\cdot2^n}\right)^2\\
				&=&2^n \cdot \dfrac{1}{2}\pi \dfrac{R^2}{(2^n)^2}\\
				&=&\dfrac{\pi R^2}{2^{n+1}}.	
			\end{eqnarray*}			
			\item $\lim \limits_{n \to +\infty}p_n=\lim \limits_{n \to +\infty}\left(\pi R\right) = \pi R$.\\
			$\lim \limits_{n \to +\infty}S_n=\lim \limits_{n \to +\infty}\dfrac{\pi R^2}{2^{n+1}}=0$ (Vì $\lim \limits_{n \to +\infty}\left(\pi R^2\right)=\pi R^2$ và $\lim \limits_{n \to +\infty}2^{n+1}=+\infty$).
		\end{enumerate}
	}
\end{vd}
\begin{vd}%[1D4K1-6]
	\immini{Từ độ cao $55,8 \mathrm{~m}$ của tháp nghiêng Pisa nước Ý, người ta thả một quả bóng cao su chạm xuống đất hình bên dưới. Giả sử mỗi lần chạm đất quả bóng lại nảy lên độ cao bằng $\dfrac{1}{10}$ độ cao mà quả bóng đạt được trước đó. Gọi $S_n$ là tổng độ dài quãng đường di chuyển của quả bóng tính từ lúc thả ban đầu cho đến khi quả bóng đó chạm đất $n$ lần. Tính $\lim \limits_{n \to +\infty}S_n$.}{
		
		\definecolor{lightcornflowerblue}{rgb}{0.6, 0.81, 0.93}
		\definecolor{cadmiumgreen}{rgb}{0.0, 0.42, 0.24}
		\definecolor{trueblue}{rgb}{0.0, 0.45, 0.81}
		\definecolor{tumbleweed}{rgb}{0.87, 0.67, 0.53}%màu cát
		\begin{tikzpicture}[line join=round, line cap=round,scale=1,transform shape]
			\clip (-4,-3.5) rectangle (4,3.5);
			\tikzset{thap/.pic={%Xây từ dưới lên
					\def\T{ 
						(-.88,-1.78)%1
						..controls +(40:.17) and +(120:.1) ..  (.686,-1.91)--(.68,-1.94)
						..controls +(120:.1) and +(40:.12) ..  (-.87,-1.81)
						--cycle
						
						(-.8,-1.2)%2
						..controls +(40:.14) and +(120:.14) ..  (.72,-1.34)--(.7,-1.36)
						..controls +(120:.1) and +(40:.12) ..  (-.78,-1.23)
						--cycle
						
						(-.74,-.65)%3
						..controls +(40:.22) and +(120:.17) ..  (.74,-.78)--(.72,-.8)
						..controls +(120:.15) and +(40:.15) ..  (-.72,-.67)
						--cycle
						
						(-.68,-.13)%4
						..controls +(40:.34) and +(110:.2) ..  (.8,-.26)--(.78,-.28)
						..controls +(110:.17) and +(40:.27) ..  (-.66,-.15)
						--cycle
						
						(-.65,.36)%5
						..controls +(40:.44) and +(120:.3) ..  (.85,.24)--(.83,.22)
						..controls +(130:.35) and +(40:.35) ..  (-.63,.34)
						--cycle
						
						(-.58,.88)%6
						..controls +(40:.44) and +(120:.3) ..  (.87,.78)--(.85,.76)
						..controls +(130:.35) and +(40:.35) ..  (-.56,.86)
						--cycle
						
						(-.52,1.35)%7
						..controls +(120:.32) and +(110:.47) ..  (.93,1.26)--(.9,1.25)
						..controls +(110:.43) and +(110:.27) ..  (-.49,1.35)
						--cycle
						
						(-.31,2)%8
						..controls +(120:.47) and +(55:.46) ..  (.76,1.94)--(.74,1.94)
						..controls +(75:.31) and +(115:.28) ..  (-.28,2)
						--cycle
						
						(-.48,1.5)
						..controls +(40:.33) and +(110:.2) ..  (.86,1.4)
						(-.476,1.52)
						..controls +(40:.33) and +(110:.2) ..  (.86,1.42)
						(-.472,1.54)
						..controls +(40:.33) and +(110:.2) ..  (.86,1.44)
						(-.468,1.56)
						..controls +(40:.33) and +(110:.2) ..  (.86,1.46)
						
						(-.28,2.2)
						..controls +(40:.27) and +(140:.2) ..  (.74,2.16)
						(-.276,2.22)
						..controls +(40:.27) and +(140:.2) ..  (.74,2.18)
						(-.272,2.24)
						..controls +(40:.27) and +(140:.2) ..  (.74,2.2)
						(-.268,2.26)
						..controls +(40:.27) and +(140:.2) ..  (.74,2.22)
						;}
					\draw \T;
					%\fill[tumbleweed] \T;
			}}
			
			\tikzset{rao/.pic={%Rào
					\def\R{ 
						(-.48,1.5)
						..controls +(40:.33) and +(110:.2) ..  (.86,1.4)
						(-.476,1.52)
						..controls +(40:.33) and +(110:.2) ..  (.86,1.42)
						(-.472,1.54)
						..controls +(40:.33) and +(110:.2) ..  (.86,1.44)
						(-.468,1.56)
						..controls +(40:.33) and +(110:.2) ..  (.86,1.46)
						
						(-.28,2.2)
						..controls +(40:.27) and +(140:.2) ..  (.74,2.16)
						(-.276,2.22)
						..controls +(40:.27) and +(140:.2) ..  (.74,2.18)
						(-.272,2.24)
						..controls +(40:.27) and +(140:.2) ..  (.74,2.2)
						(-.268,2.26)
						..controls +(40:.27) and +(140:.2) ..  (.74,2.22)
						;}
					\draw \R;
					%\fill[tumbleweed] \R;
			}}
			
			\tikzset{co/.pic={%Cờ
					\def\C{ 
						(0.1,2.3)--(.1,2.9)
						(.1,2.82)
						..controls +(-40:.2) and +(140:.22) ..  (.2,2.6)
						..controls +(-60:0) and +(100:.1) ..(.16,2.5)
						..controls +(140:.1) and +(-30:0) ..  (.1,2.53)--cycle
						;}
					\draw \C;
					\fill[red] \C;
			}}
			
			\tikzset{cong0/.pic={%Đường cong trệt
					\def\D0{ 
						(.68,-1.96)
						..controls +(120:.1) and +(40:.12) ..  (-.87,-1.81)--
						(-.87,-1.94)
						..controls +(80:.12) and +(95:0.12) ..  (-.68,-2.06)
						..controls +(80:.12) and +(85:0.3) ..  (-.4,-2.12)
						..controls +(80:.25) and +(95:0.25) ..  (-.02,-2.12)
						..controls +(80:.25) and +(95:0.25) ..  (.27,-2.14)
						..controls +(80:.12) and +(95:0.12) ..  (.5,-2.14)
						..controls +(70:.12) and +(95:0.02) ..  (.66,-2.1)--cycle
						;}
					\draw \D0;
					\fill[tumbleweed] \D0;
			}}
			
			\tikzset{cong1/.pic={%Đường cong 1
					\def\D1{ 
						(.7,-1.36)
						..controls +(120:.1) and +(40:.12) ..  (-.78,-1.23)--
						(-.78,-1.35)
						..controls +(80:.12) and +(95:0.12) ..  (-.72,-1.35)
						..controls +(80:.12) and +(95:0.12) ..  (-.64,-1.35)
						..controls +(80:.12) and +(95:0.12) ..  (-.55,-1.35)
						..controls +(80:.12) and +(95:0.12) ..  (-.44,-1.36)
						..controls +(80:.12) and +(95:0.12) ..  (-.3,-1.37)
						..controls +(80:.12) and +(95:0.12) ..  (-.14,-1.38)
						..controls +(80:.12) and +(95:0.12) ..  (0.02,-1.39)
						..controls +(80:.12) and +(95:0.12) ..  (0.18,-1.4)
						..controls +(80:.12) and +(95:0.12) ..  (0.32,-1.42)
						..controls +(70:.12) and +(95:0.12) ..  (0.42,-1.44)
						..controls +(70:.12) and +(95:0.12) ..  (0.52,-1.46)
						..controls +(70:.12) and +(95:0.12) ..  (0.6,-1.47)
						..controls +(70:.12) and +(95:0.12) ..  (0.68,-1.5)--cycle
						;}
					\draw \D1;
					\fill[tumbleweed] \D1;
			}}
			
			\tikzset{cong2/.pic={%Đường cong 2
					\def\D2{ 
						(.72,-.8)
						..controls +(120:.15) and +(40:.15) ..  (-.72,-.67)--
						(-.72,-.79)
						..controls +(80:.12) and +(95:0.12) ..  (-.66,-.79)
						..controls +(80:.12) and +(95:0.12) ..  (-.58,-.79)
						..controls +(80:.12) and +(95:0.12) ..  (-.48,-.79)
						..controls +(80:.12) and +(95:0.12) ..  (-.38,-.8)
						..controls +(80:.12) and +(95:0.12) ..  (-.23,-.8)
						..controls +(80:.12) and +(95:0.12) ..  (-0.08,-.81)
						..controls +(80:.12) and +(95:0.12) ..  (0.08,-.82)
						..controls +(80:.12) and +(95:0.12) ..  (0.22,-.84)
						..controls +(80:.12) and +(95:0.12) ..  (0.36,-.85)
						..controls +(70:.12) and +(95:0.12) ..  (0.48,-.87)
						..controls +(70:.12) and +(95:0.12) ..  (0.58,-.89)
						..controls +(70:.12) and +(95:0.12) ..  (0.66,-.9)
						..controls +(70:.12) and +(95:0.12) ..  (0.71,-.92)--cycle
						;}
					\draw \D2;
					\fill[tumbleweed] \D2;
			}}
			
			\tikzset{cong3/.pic={%Đường cong 3
					\def\D3{ 
						(.78,-.28)
						..controls +(110:.17) and +(40:.27) ..  (-.66,-.15)--
						(-.66,-.26)
						..controls +(80:.12) and +(95:0.12) ..  (-.6,-.25)
						..controls +(80:.12) and +(95:0.12) ..  (-.52,-.24)
						..controls +(80:.12) and +(95:0.12) ..  (-.44,-.23)
						..controls +(80:.12) and +(95:0.12) ..  (-.32,-.23)
						..controls +(80:.12) and +(95:0.12) ..  (-.18,-.23)
						..controls +(80:.12) and +(95:0.12) ..  (-.04,-.24)
						..controls +(80:.12) and +(95:0.12) ..  (0.13,-.25)
						..controls +(80:.12) and +(95:0.12) ..  (0.26,-.26)
						..controls +(70:.12) and +(95:0.12) ..  (0.42,-.29)
						..controls +(70:.12) and +(95:0.12) ..  (0.54,-.31)
						..controls +(70:.12) and +(95:0.12) ..  (0.63,-.33)
						..controls +(70:.12) and +(95:0.12) ..  (0.7,-.35)
						..controls +(70:.12) and +(95:0.12) ..  (0.765,-.38)--cycle
						;}
					\draw \D3;
					\fill[tumbleweed] \D3;
			}}
			
			\tikzset{cong4/.pic={%Đường cong 4
					\def\D4{ 
						(.83,.22)
						..controls +(130:.35) and +(40:.35) ..  (-.63,.34)--
						(-.63,.23)
						..controls +(80:.12) and +(95:0.12) ..  (-.57,.25)
						..controls +(80:.12) and +(95:0.12) ..  (-.49,.26)
						..controls +(80:.12) and +(95:0.12) ..  (-.4,.28)
						..controls +(80:.12) and +(95:0.12) ..  (-.27,.3)
						..controls +(80:.12) and +(95:0.12) ..  (-.15,.3)
						..controls +(80:.12) and +(95:0.12) ..  (0.02,.3)
						..controls +(80:.12) and +(95:0.12) ..  (0.16,.29)
						..controls +(80:.12) and +(95:0.12) ..  (0.31,.27)
						..controls +(70:.12) and +(95:0.12) ..  (0.45,.25)
						..controls +(70:.12) and +(95:0.12) ..  (0.58,.23)
						..controls +(70:.12) and +(95:0.12) ..  (0.67,.2)
						..controls +(70:.12) and +(95:0.12) ..  (0.75,.17)
						..controls +(70:.12) and +(95:0.12) ..  (0.8,.14)--cycle
						;}
					\draw \D4;
					\fill[tumbleweed] \D4;
			}}
			
			\tikzset{cong5/.pic={%Đường cong 5
					\def\D5{ 
						(.85,.76)
						..controls +(130:.35) and +(40:.35) ..  (-.56,.86)--
						(-.56,.74)
						..controls +(80:.12) and +(95:0.12) ..  (-.5,.77)
						..controls +(80:.12) and +(95:0.12) ..  (-.44,.79)
						..controls +(80:.12) and +(95:0.12) ..  (-.34,.81)
						..controls +(80:.12) and +(95:0.12) ..  (-.22,.83)
						..controls +(80:.12) and +(95:0.12) ..  (-0.08,.85)
						..controls +(80:.12) and +(95:0.12) ..  (0.06,.85)
						..controls +(80:.12) and +(95:0.12) ..  (0.22,.83)
						..controls +(70:.12) and +(95:0.12) ..  (0.36,.81)
						..controls +(70:.12) and +(95:0.12) ..  (0.5,.78)
						..controls +(70:.12) and +(95:0.12) ..  (0.62,.74)
						..controls +(70:.12) and +(95:0.12) ..  (0.73,.72)
						..controls +(70:.12) and +(95:0.12) ..  (0.82,.68)--cycle
						;}
					\draw \D5;
					\fill[tumbleweed] \D5;
			}}
			
			\tikzset{cong6/.pic={%Đường cong 6
					\def\D6{ 
						(.9,1.25)
						..controls +(110:.43) and +(110:.27) ..  (-.49,1.35)--
						(-.49,1.3)
						..controls +(80:.12) and +(95:0.12) ..  (-.45,1.33)
						..controls +(80:.12) and +(95:0.12) ..  (-.38,1.33)
						..controls +(80:.12) and +(95:0.12) ..  (-.28,1.35)
						..controls +(80:.12) and +(95:0.12) ..  (-.16,1.37)
						..controls +(80:.12) and +(95:0.12) ..  (-0.04,1.38)
						..controls +(80:.12) and +(95:0.12) ..  (0.12,1.38)
						..controls +(80:.12) and +(95:0.12) ..  (0.26,1.38)
						..controls +(70:.12) and +(95:0.12) ..  (0.42,1.37)
						..controls +(70:.12) and +(95:0.12) ..  (0.55,1.34)
						..controls +(70:.12) and +(95:0.12) ..  (0.67,1.3)
						..controls +(70:.12) and +(95:0.12) ..  (0.78,1.25)
						..controls +(70:.12) and +(95:0.12) ..  (0.84,1.21)
						..controls +(70:.12) and +(95:0.12) ..  (0.88,1.16)--cycle
						;}
					\draw \D6;
					\fill[tumbleweed] \D6;
			}}
			
			\tikzset{cong7/.pic={%Đường cong 7
					\def\D7{ 
						(.74,1.94)
						..controls +(75:.31) and +(115:.28) ..  (-.28,2)--
						(-.28,1.94)
						..controls +(80:.12) and +(95:0.12) ..  (-.24,1.98)
						..controls +(80:.12) and +(95:0.12) ..  (-.14,2.04)
						..controls +(80:.12) and +(95:0.12) ..  (0.02,2.08)
						..controls +(80:.12) and +(95:0.12) ..  (0.2,2.08)
						..controls +(80:.12) and +(95:0.12) ..  (0.4,2.05)
						..controls +(80:.12) and +(95:0.12) ..  (0.56,2)
						..controls +(80:.12) and +(95:0.12) ..  (0.66,1.96)
						..controls +(70:.12) and +(95:0.12) ..  (0.74,1.9)--cycle
						;}
					\draw \D7;
					\fill[tumbleweed] \D7;
			}}
			
			\tikzset{cua/.pic={%Cửa
					\def\W{ 
						(-.7,-2.85)--(-.63,-2.25)
						..controls +(70:.1) and +(100:.1) ..  (-.46,-2.25)--(-.52,-2.85)
						--cycle
						;}
					\draw \W;
					\fill[tumbleweed] \W;
			}}
			
			\tikzset{vien/.pic={%viền ngoài
					\def\V{ 
						(.9,1.25)
						..controls +(110:.43) and +(110:.27) ..  (-.49,1.35)--(-.98,-2.85)--(.59,-2.85)--cycle
						
						(.74,1.94)
						..controls +(75:.31) and +(115:.28) ..  (-.28,2)--(-.32,1.54)
						..controls +(115:.2) and +(75:.13) ..  (.72,1.48)--cycle
						;}
					\draw \V;
					\fill[gray!50!] \V;
			}}
			
			\tikzset{soc/.pic={%Thanh sọc
					\def\S{ 
						(.82,1.36)--(.54,-1.85)--(.58,-1.85)--(.86,1.36)
						--cycle
						;}
					\draw \S;
					\fill[tumbleweed!40!] \S;
			}}
			
			\tikzset{soc2/.pic={%Thanh sọc lớn
					\def\S{ 
						(.54,-1.95)--(.44,-2.85)--(.48,-2.85)--(.58,-1.95)
						--cycle
						;}
					\draw \S;
					\fill[tumbleweed!40!] \S;
			}}
			
			\tikzset{soc3/.pic={%Thanh sọc
					\def\S{ 
						(.54,2)--(.57,2)--(.54,1.5)--(.51,1.5)
						--cycle
						(.66,2)--(.69,2)--(.66,1.5)--(.63,1.5)
						--cycle
						(-.26,2)--(-.23,2)--(-.26,1.5)--(-.29,1.5)
						--cycle
						;}
					\draw \S;
					\fill[tumbleweed!40!] \S;
			}}
			
			\tikzset{thoi/.pic={%Hình thoi
					\def\T{ 
						(0.15,-1.98)--(.24,-2.12)--(0.13,-2.22)--(0.04,-2.1)
						--cycle
						;}
					\draw \T;
					\fill[tumbleweed!40!] \T;
			}}
			
			\tikzset{cay/.pic={%Cây
					\def\T{ 
						(-1.73,-2.85)
						..controls +(40:.03) and +(40:.01) ..  (-1.74,-2.7)
						..controls +(140:.04) and +(60:.02) ..  (-1.78,-2.6)
						..controls +(140:.04) and +(60:.02) ..  (-1.8,-2.55)
						..controls +(100:.03) and +(70:.02) ..  (-1.82,-2.5)
						..controls +(140:.04) and +(60:.02) ..  (-1.815,-2.4)
						..controls +(140:.04) and +(60:.02) ..  (-1.818,-2.3)
						..controls +(85:.03) and +(-30:.03) ..  (-1.8,-2.2)
						..controls +(80:.04) and +(50:.02) ..  (-1.78,-2)
						..controls +(60:.04) and +(120:.02) ..  (-1.76,-1.9)
						..controls +(80:.03) and +(-100:.02) ..  (-1.74,-1.8)
						..controls +(80:.01) and +(-100:.02) ..  (-1.72,-1.9)
						..controls +(60:.02) and +(-120:.02) ..  (-1.7,-2)
						..controls +(-80:.02) and +(110:.03) ..  (-1.66,-2.2)
						..controls +(-100:.02) and +(60:.03) ..  (-1.64,-2.3)
						..controls +(-30:.02) and +(70:.02) ..  (-1.6,-2.44)
						..controls +(-100:.02) and +(70:.02) ..  (-1.63,-2.52)
						..controls +(-60:.02) and +(70:.01) ..  (-1.66,-2.6)
						..controls +(60:.01) and +(70:.02) ..  (-1.7,-2.7)
						..controls +(-120:.02) and +(120:.03) ..  (-1.68,-2.85)
						;}
					\draw \T;
					\fill[cadmiumgreen!90!] \T;
			}}
			\fill[cadmiumgreen!80!] (-4,-3.5) rectangle (8,-2.2);
			\fill[lightcornflowerblue] (-4,3.5) rectangle (8,-2.2);
			\path 
			(0,0)pic[scale=1]{cay}(.35,0)pic[scale=.9]{cay}(1.05,0.6)pic[scale=1.2]{cay}(-.5,-1)pic[scale=.6]{cay}
			(3,0)pic[scale=1]{cay}(2.8,0)pic[scale=1]{cay}
			(0,0)pic[scale=1]{vien}
			(0,0)pic[scale=1]{soc} (0,0)pic[scale=1]{soc}(-.15,0)pic[scale=1]{soc}(-.3,0)pic[scale=1]{soc}(-.45,0)pic[scale=1]{soc}(-.6,0)pic[scale=1]{soc}(-.75,0)pic[scale=1]{soc}(-.9,0)pic[scale=1]{soc}(-1.02,0)pic[scale=1]{soc}(-1.12,0)pic[scale=1]{soc}(-1.22,0)pic[scale=1]{soc}(-1.32,0)pic[scale=1]{soc}
			
			(0,0)pic[scale=1]{soc3}
			(0,0)pic[scale=1]{thap}(0,0)pic[scale=1]{co}
			(.4,3.6)pic[scale=.7,rotate=3]{cua}
			(.8,3.6)pic[scale=1.2,rotate=7,yscale=.6]{cua}
			(.3,3.4)pic[scale=1.1,rotate=7,yscale=.6]{cua}
			(0,-.04)pic[scale=1]{thoi}(0.25,-.05)pic[scale=1]{thoi}(-0.35,0)pic[scale=1]{thoi}(-.67,0)pic[scale=1]{thoi}(-.9,0)pic[scale=1]{thoi}
			(-1.22,0)pic[scale=1]{soc2}(-1.36,0)pic[scale=1]{soc2}
			(-.94,0)pic[scale=1]{soc2}(-.56,0)pic[scale=1]{soc2}
			(-.56,0)pic[scale=1]{soc2}(.08,0)pic[scale=1]{soc2}(-.04,0)pic[scale=1]{soc2}(-.28,0)pic[scale=1]{soc2}
			(0,0)pic[scale=1]{cong0} (0,0.02)pic[scale=1]{cong0}%
			(0,0)pic[scale=1]{cong1} (0,0.02)pic[scale=1]{cong1}%
			(0,0)pic[scale=1]{cong2} (0,0.02)pic[scale=1]{cong2}%
			(0,0)pic[scale=1]{cong3}(0,0.02)pic[scale=1]{cong3}
			(0,0)pic[scale=1]{cong4}(0,0.02)pic[scale=1]{cong4}
			(0,0)pic[scale=1]{cong5}(0,0.02)pic[scale=1]{cong5}
			(0,0)pic[scale=1]{cong6}(0,0.02)pic[scale=1]{cong6}
			(0,0)pic[scale=1]{cong7}(0,0.02)pic[scale=1]{cong7}
			(0,0)pic[scale=1]{cua}
			
			(0,0)pic[scale=1]{rao}
			;		
		\end{tikzpicture}
	}
	\loigiai{Mỗi khi chạm đất quả bóng lại nảy lên một độ cao bằng $\dfrac{1}{10}$ độ cao của lần rơi ngay trước đó và sau đó lại rơi xuống từ độ cao thứ hai này. Do đó, độ dài hành trình của quả bóng kể từ thời điểm rơi ban đầu đến:\\    
		Thời điểm chạm đất lần thứ nhất là $d_1=55{,}8$.\\
		Thời điềm chạm đất lần thứ hai là $d_2=55{,}8+2\cdot \dfrac{55{,}8}{10}$.\\
		Thời điểm chạm đất lần thứ ba là $d_3=55{,}8+2 \cdot\dfrac{55{,}8}{10}+2\cdot \dfrac{55{,}8}{10^2}$.\\
		Thời điểm chạm đất lần thứ tư là $d_4=55{,}8+2 \cdot\dfrac{55{,}8}{10}+2\cdot \dfrac{55,8}{10^2}+2\cdot \dfrac{55{,}8}{10^3}$.\\
		$\ldots$\\
		Thời điểm chạm đất lần thứ $n~(n>1)$ là
		$$d_n=55{,}8+2\cdot55{,}8+2\cdot \frac{55{,}8}{10^2}+2\cdot \frac{55{,}8}{10^3}+\ldots+2\cdot \frac{55{,}8}{10^{n-1}}.$$
		Do đó, quãng đường mà quả bóng đi được kể từ thời điềm rơi đến khi nằm yên trên mặt đất là:
		$$ d=55{,}8+2.55{,}8+2\cdot \frac{55{,}8}{10^2}+2\cdot \frac{55{,}8}{10^3}+\ldots+2\cdot \frac{55{,}8}{10^{n-1}}+\ldots=\lim \limits_{n \to +\infty}d_n.$$
		Vì $2\cdot \dfrac{55{,}8}{10} ; 2\cdot \dfrac{55{,}8}{10^2} ; 2\cdot \dfrac{55{,}8}{10^3}; \ldots ; 2\cdot \dfrac{55{,}8}{10^{n-1}}; \ldots$ là một cấp số nhân lùi vô hạn với công bội $q=\dfrac{1}{10}$ nên ta có:
		$$ 2 \cdot\dfrac{55,8}{10}+2\cdot \dfrac{55{,}8}{10^2}+2\cdot \dfrac{55{,}8}{10^3}+\ldots+2\cdot \dfrac{55{,}8}{10^{n-1}}+\ldots=\dfrac{2\cdot \dfrac{55{,}8}{10}}{1-\dfrac{1}{10}}=12{,}4.$$
		Vậy $d=55{,}8+12{,}4=68{,}2$ m.
	}
\end{vd}
\begin{vd}%[0D1Y1-1]
	Cho một tam giác đều $A B C$ cạnh $a$. Tam giác $A_1 B_1 C_1$ có các đỉnh là trung điểm các cạnh của tam giác $A B C$, tam giác $A_2 B_2 C_2$ có các đỉnh là trung điểm các cạnh của tam giác $A_1 B_1 C_1, \ldots$, tam giác $A_{n+1} B_{n+1} C_{n+1}$ có các đỉnh là trung điểm các cạnh của tam giác $A_n B_n C_n, \ldots$ Gọi $p_1, p_2, \ldots, p_n, \ldots$ và $S_1, S_2, \ldots, S_n, \ldots$ theo thứ tự là chu vi và diện tích của các tam giác $A_1 B_1 C_1, A_2 B_2 C_2, \ldots, A_n B_n C_n, \ldots$.
	\begin{listEX}
		\item[a)] Tìm giới hạn của các dãy số $\left(p_n\right)$ và $\left(S_n\right)$.
		\item[b)] Tìm các tổng $p_1+p_2+\ldots+p_n+\ldots$ và $S_1+S_2+\ldots+S_n+\ldots$.
	\end{listEX}
	\loigiai{
		\begin{listEX}
			\item[a)] Ta có $p_1, p_2, \ldots, p_n, \ldots$ lần lượt là chu vi của các tam giác $A_1 B_1 C_1, A_2 B_2 C_2, \ldots, A_n B_n C_n, \ldots$
			$$\begin{aligned}
				& p_1=3 a \\
				& p_2=3 \cdot \frac{1}{2} a \\
				& \ldots \\
				& p_n=3 \cdot \frac{1}{2^{n-1}} a 
			\end{aligned} $$
			suy ra $\lim \limits_{n \to +\infty}p_n=\lim \limits_{n \to +\infty}3 \cdot \dfrac{1}{2^{n-1}} a=0$.
			$$ \begin{aligned}
				& S_1=\frac{a^2 \sqrt{3}}{4} \\
				& S_2=\frac{1}{4} \frac{a^2 \sqrt{3}}{4} \\
				& \ldots \\
				& S_n=\frac{1}{4^{n-1}} \cdot \frac{a^2 \sqrt{3}}{4}
			\end{aligned}$$
			suy ra $\lim \limits_{n \to +\infty}S_n=\lim \limits_{n \to +\infty}\dfrac{1}{4^{n-1}} \cdot \dfrac{a^2 \sqrt{3}}{4}=0$.
			\item[b)] Dựa vào dữ kiện đề bài suy ra tổng $\left(p_n\right)$ là tổng của cấp số nhân lùi vô hạn với công bội $q=\dfrac{1}{2}$ và
			$ p_1+p_2+\ldots+p_n+\ldots=\lim \limits_{n \to +\infty}\left(p_n\right)=\dfrac{p_1}{1-q}=\dfrac{3 a}{1-\frac{1}{2}}=6a.$\\
			Dựa vào dữ kiện đề bài suy ra tổng $\left(S_n\right)$ là tổng của cấp số nhân lùi vô hạn với công bội $q=\dfrac{1}{4}$ và
			$S_1+S_2+\ldots+S_n+\ldots=\lim \limits_{n \to +\infty}\left(S_n\right)=\dfrac{S_1}{1-q}=\dfrac{\frac{a^2 \sqrt{3}}{4}}{1-\frac{1}{4}}=\dfrac{a^2 \sqrt{3}}{12}$.
		\end{listEX}
	}
\end{vd}
% \subsubsection{Bài tập rèn luyện} 
% % \subsubsection{Bài tập tự luận}
% \begin{bt}%[1T3K1-5]
% 	Từ tờ giấy, cắt một hình tròn bán kính $R$ (cm) như Hình $3a$. Tiếp theo, cắt hai hình~tròn
% 	\immini{
% 		bán kính $\dfrac{R}{2}$ rồi chồng lên hình tròn đầu tiên như Hình $3b$. Tiếp theo, cắt bốn hình tròn bán  kính $\dfrac{R}{4}$ 
% 		rồi chồng lên các hình trước như Hình $3c$. Cứ thế tiếp tục mãi. Tính tổng diện tích của các hình tròn.
% 	}{
% 		\begin{tikzpicture}[>=stealth,line join=round,line cap=round,font=\footnotesize,scale=1,declare function={r=1.5;}]
% 			\begin{scope}
% 				\draw[fill=blue] (0,0)circle(r);
% 				\path (0,-r) node[below]{$a)$};
% 			\end{scope}
% 			\begin{scope}[xshift={3.5cm}]
% 				\draw[fill=blue] (0,0)circle(r);
% 				\draw[fill=yellow] (r/2,0)circle(r/2);
% 				\draw[fill=yellow] (-r/2,0)circle(r/2);
% 				\path (0,-r) node[below]{$b)$};
% 			\end{scope}
% 			\begin{scope}[xshift={7cm}]
% 				\draw[fill=blue] (0,0)circle(r);
% 				\draw[fill=yellow] (r/2,0)circle(r/2);
% 				\draw[fill=yellow] (-r/2,0)circle(r/2);
% 				\foreach \i in {0,1,2,3}
% 				\draw[fill=green,shift={(r/2*\i,0)}] (-3*r/4,0)circle(r/4);
% 				\path (0,-r) node[below]{$c)$};
% 			\end{scope}
% 			\path (current bounding box.south) node[below]{Hình $3$};
% 		\end{tikzpicture}
% 	}
% 	\loigiai{
% 		Diện tích của các hình tròn trong các lần cắt là
% 		\begin{enumerate}
% 			\item Lần thứ 1: $S_1=\pi R^2$.
% 			\item  Lần thứ 2: $S_2=2\cdot \pi \left(\dfrac{R}{2}\right)^2= \dfrac{\pi R^2}{2}$.
% 			\item  Lần thứ 3: $S_2=4\cdot \pi \left(\dfrac{R}{4}\right)^2= \dfrac{\pi R^2}{2^2}$.	
% 			\item Lần thứ $n$: $S_n= \dfrac{\pi R^2}{2^{n-1}}$.
% 		\end{enumerate}
% 		Do đó  diện tích các hình tròn lập thành một cấp số nhân lùi vô hạn có số hạng đầu $S_1=\pi R^2$ và công bội $q=\dfrac{1}{2}$ nên tổng diện tích các hình tròn là 
% 		\[ S_1+S_2+\cdots=\dfrac{\pi R^2}{1-\dfrac{1}{2}}=2\pi R^2. \]
% 	}
% \end{bt}
% \begin{bt}%[1T3K1-5]
% 	\immini{
% 		Từ hình vuông đầu tiên có cạnh bằng $1$ (đơn vị độ dài), nối các trung điểm của bốn cạnh để có hình vuông thứ hai. Tiếp tục nối các trung điểm của bốn cạnh của hình vuông thứ hai để được hình vuông thứ ba. Cứ tiếp tục làm như thế, nhận được một dãy hình vuông (xem Hình $5$).
% 	}{\hspace*{.5cm}
% 		\begin{tikzpicture}[>=stealth,line join=round,line cap=round,font=\footnotesize,scale=1,declare function={a=3;}]
% 			\foreach \i in {1,2,...,7}{
% 				\pgfmathsetmacro\r{a*(sin(45))^(\i-1)}
% 				\pgfmathsetmacro{\j}{int(mod(\i,2))}
% 				\ifnum \j=1
% 				\draw (-\r/2,-\r/2) rectangle (\r/2,\r/2);
% 				\else
% 				\draw[rotate=45] (-\r/2,-\r/2) rectangle (\r/2,\r/2);
% 				\fi
% 			}
% 			\path (current bounding box.south) node[below]{Hình $5$};
% 		\end{tikzpicture}\hspace*{1cm}
% 	}
% 	\begin{enumerate}
% 		\item Kí hiệu $a_n$ là diện tích của hình vuông thứ $n$ và $S_n$ là tổng diện tích của $n$ hình vuông đầu tiên. Viết công thức tính $a_n$, $S_n$ ($n=1,2,3, \ldots$) và tìm $\lim \limits_{n \to +\infty}S_n$ (giới hạn này nếu có được gọi là tổng diện tích của các hình vuông).
% 		\item Kí hiệu $p_n$ là chu vi của hình vuông thứ $n$ và $Q_n$ là tổng chu vi của $n$ hình vuông đầu tiên. Viết công thức tính $p_n$ và $Q_n$ $(n=1,2,3, \ldots)$ và tìm $\lim \limits_{n \to +\infty}Q_n$ (giới hạn này nếu có được gọi là tổng chu vi của các hình vuông).
% 	\end{enumerate}
% 	\loigiai{
% 		\begin{enumerate}
% 			\item Ta có hình vuông thứ nhất có cạnh bằng $1$,
% 			hình vuông thứ hai có cạnh bằng $\dfrac{\sqrt{2}}{2}$.\\
% 			Hình vuông thứ ba có cạnh bằng $\dfrac{1}{2}$.\\
% 			Suy ra	hình vuông thứ $n$ có cạnh bằng $\left(\dfrac{\sqrt{2}}{2}\right)^{n-1}$.\\
% 			Diện tích của hình vuông thứ $n$ là $a_n=\left(\dfrac{\sqrt{2}}{2}\right)^{n-1}\cdot \left(\dfrac{\sqrt{2}}{2}\right)^{n-1}=\left(\dfrac{1}{2}\right)^{n-1}$.\\
% 			Tổng diện tích của $n$ hình vuông đầu tiên là tổng của cấp số nhân có  số hạng đầu $a_1=1$ và công bội $q=\dfrac{1}{2}$ nên
% 			\[ S_n=\dfrac{a_1\left(1-q^n\right)}{1-q}=\dfrac{1-\left(\dfrac{1}{2}\right)^n}{1-\dfrac{1}{2}}=2\left[1-\left(\dfrac{1}{2}\right)^n\right].\]
% 			$\lim \limits_{n \to +\infty}S_n=\lim \limits_{n \to +\infty}2\left[1-\left(\dfrac{1}{2}\right)^n\right]=2 \left[\lim \limits_{n \to +\infty}1-\lim \limits_{n \to +\infty}\left(\dfrac{1}{2}\right)^n\right]=2 $.
% 			\item Hình vuông thứ nhất có chu vi bằng $4$, hình vuông thứ $2$ có chu vi là $2\sqrt{2}$, hình vuông thứ $3$ có chu vi là $2$.\\
% 			Suy ra hình vuông thứ $n$ có chu vi bằng $p_n=4\cdot \left(\dfrac{\sqrt{2}}{2}\right)^{n-1}$.
% 			Tổng chu vi của $n$ hình vuông đầu tiên là tổng của cấp số nhân có  số hạng đầu $p_1=4$ và công bội $q=\dfrac{\sqrt{2}}{2}$ nên 
% 			\[ Q_n=\dfrac{p_1\left(1-q^n\right)}{1-q}=\dfrac{4\left(1-\left(\dfrac{\sqrt{2}}{2}\right)^n\right)}{1-\dfrac{\sqrt{2}}{2}}=\left(8+4\sqrt{2}\right)\left[1-\left(\dfrac{\sqrt{2}}{2}\right)^n\right].\]
% 			$\lim \limits_{n \to +\infty}Q_n=\left(8+4\sqrt{2}\right)\lim \limits_{n \to +\infty}\left[1-\left(\dfrac{\sqrt{2}}{2}\right)^n\right]=\left(8+4\sqrt{2}\right)\left(1-0\right)=8+4\sqrt{2}$.
% 		\end{enumerate}	
% 	}
% \end{bt}

% \begin{bt}%[1T3K1-4]
% 	Xét quá trình tạo ra hình có chu vi vô cực và diện tích bằng $0$ như sau:\\ Bắt đầu bằng một hình vuông $H_0$ cạnh bằng 1 đơn vị độ dài (xem Hình $6a$). Chia hình vuông $H_0$ thành chín hình vuông bằng nhau, bỏ đi bốn hình vuông, nhận được hình $H_1$ (xem Hình $6b$). Tiếp theo, chia mỗi hình vuông của $H_1$ thành chín hình vuông, rồi bỏ đi bốn hình vuông, nhận được hình $H_2$ (xem Hình $6c$). Tiếp tục quá trình này, ta nhận được một dãy hình $H_n$ $(n=1,2,3,\ldots)$.	
% 	\\[1mm]
% 	\centerline{
% 		\begin{tikzpicture}% Muốn vẽ hình Hn thì dùng \hv{n}
% 			\def\a{2}
% 			\pgfmathsetmacro\sh{2*\a *sqrt(2)/3}
% 			\def\hv#1{
% 				\ifnum#1>0
% 				\draw[white,fill=white] 
% 				(-\a/3,\a/3) rectangle (\a/3,\a)
% 				(-\a/3,-\a/3) rectangle (-\a,\a/3)
% 				(-\a/3,-\a/3) rectangle (\a/3,-\a)
% 				(\a/3,-\a/3) rectangle (\a,\a/3)
% 				;
% 				\pgfmathtruncatemacro{\k}{#1-1}
% 				\begin{scope}[scale=1/3]\hv{\k}\end{scope}
% 				\begin{scope}[shift={(45:\sh)},scale=1/3]\hv{\k}\end{scope}
% 				\begin{scope}[shift={(135:\sh)},scale=1/3]\hv{\k}\end{scope}
% 				\begin{scope}[shift={(225:\sh)},scale=1/3]\hv{\k}\end{scope}
% 				\begin{scope}[shift={(315:\sh)},scale=1/3]\hv{\k}\end{scope}
% 				\fi
% 			}
% 			\begin{scope}
% 				\fill[green] (-\a,-\a) rectangle (\a,\a);
% 				\hv{0}
% 				\path (0,-\a)node[below]{$H_0$}
% 				node[below=.5cm]{$a)$};
% 			\end{scope}
% 			\begin{scope}[xshift=4.5cm]
% 				\fill[green] (-\a,-\a) rectangle (\a,\a);
% 				\hv{1}
% 				\path (0,-\a)node[below]{$H_1$}
% 				node[below=.5cm]{$b)$};
% 			\end{scope}
% 			\begin{scope}[xshift=9cm]
% 				\fill[green] (-\a,-\a) rectangle (\a,\a);
% 				\hv{2}
% 				\path (0,-\a)node[below]{$H_2$}
% 				node[below=.5cm]{$c)$};
% 			\end{scope}
% 			\begin{scope}[xshift=13.5cm]
% 				\fill[green] (-\a,-\a) rectangle (\a,\a);
% 				\hv{3}
% 				\path (0,-\a)node[below]{$H_3$}
% 				node[below=.5cm]{$d)$};
% 			\end{scope}
% 			\path (current bounding box.south) node[below]{Hình $6$};
% 		\end{tikzpicture}
% 	}
% 	Ta có: $H_1$ có $5$ hình vuông, mỗi hình vuông có cạnh bằng $\dfrac{1}{3}$;\\
% 	{\color{white}{Ta có: }}$H_2$ có $5\cdot5=5^2$ hình vuông, mỗi hình vuông có cạnh bằng $\dfrac{1}{3} \cdot \dfrac{1}{3}=\dfrac{1}{3^2}; \ldots$.\\
% 	Từ đó, nhận được $H_n$ có $5^n$ hình vuông, mỗi hình vuông có cạnh bằng $\dfrac{1}{3^n}$.
% 	\begin{enumerate}
% 		\item Tính diện tích $S_n$ của $H_n$ và tính $\lim \limits_{n \to +\infty}S_n$.
% 		\item Tính chu vi $p_n$ của $H_n$ và tính $\lim \limits_{n \to +\infty}p_n$.
% 	\end{enumerate}
% 	(Quá trình trên tạo nên một hình, gọi là một fractal, được coi là có diện tích $\lim \limits_{n \to +\infty}S_n$ và chu vi $\lim \limits_{n \to +\infty}p_n$).
% 	\loigiai{
% 		\begin{enumerate}
% 			\item Hình vuông $H_1$ có diện tích $S_1=5\cdot \left(\dfrac{1}{3}\right)^2=\dfrac{5}{9}$.\\
% 			Hình vuông $H_2$ có diện tích $S_2=5^2\cdot \left(\dfrac{1}{3^2}\right)^2=\left(\dfrac{5}{9}\right)^2$.\\
% 			Hình vuông $H_n$ có diện tích $S_n=5^n\cdot \left(\dfrac{1}{3^n}\right)^2=\left(\dfrac{5}{9}\right)^n$.\\
% 			$\lim \limits_{n \to +\infty}S_n=\lim \limits_{n \to +\infty}\left(\dfrac{5}{9}\right)^n=0$.
% 			\item Hình vuông $H_1$ có chu vi $p_1=5\cdot 4\cdot  \dfrac{1}{3}=4\cdot \dfrac{5}{3}$.\\
% 			Hình vuông $H_2$ có chu vi $p_2=5^2\cdot4\cdot \dfrac{1}{3^2}=4\cdot \left(\dfrac{5}{3}\right)^2$.\\
% 			Hình vuông $H_n$ có diện tích $p_n=5^n\cdot4\cdot  \dfrac{1}{3^n}=4\cdot \left(\dfrac{5}{3}\right)^n$.\\
% 			$\lim \limits_{n \to +\infty}p_n=\lim \limits_{n \to +\infty}4\cdot \left(\dfrac{5}{3}\right)^n=+\infty$.
% 		\end{enumerate}
% 	}
% \end{bt}
% \begin{bt}%[1T3K1-5]
% 	\immini{Cho tam giác đều có cạnh bằng $a$, gọi là tam giác $H_1$. Nối các trung điểm của $H_1$ để tạo thành tam giác $\mathrm{H}_2$. Tiếp theo, nối các trung điểm của $\mathrm{H}_2$ để tạo thành tam giác $\mathrm{H}_3$ (Hình bên). Cứ tiếp tục như vậy, nhận được dãy tam giác $H_1, H_2, H_3, \ldots$\\
% 		Tính tổng chu vi và tổng diện tích các tam giác của dãy.}{
% 		\begin{tikzpicture}[scale=1, font=\footnotesize, line join=round, line cap=round, >=stealth]
% 			(0,0) coordinate (A)	
% 			(4,0) coordinate (B)	
% 			(2,2.8284) coordinate (C)
% 			\draw (A)--(B)--(C)--cycle;
% 			\coordinate (I) at ($(A)!0.5!(B)$);
% 			\coordinate (J) at ($(B)!0.5!(C)$);
% 			\coordinate (K) at ($(C)!0.5!(A)$);
% 			\draw (I)--(J)--(K)--cycle;
% 			\coordinate (H) at ($(I)!0.5!(J)$);
% 			\coordinate (G) at ($(J)!0.5!(K)$);
% 			\coordinate (F) at ($(K)!0.5!(I)$);
% 			\draw (H)--(G)--(F)--cycle;
% 			\coordinate (X) at ($(H)!0.5!(G)$);
% 			\coordinate (Y) at ($(G)!0.5!(F)$);
% 			\coordinate (Z) at ($(F)!0.5!(H)$);
% 			\draw (X)--(Y)--(Z)--cycle;
% 			\draw[dashed] (2,2.8284)--(3.3,1.45)node[above right]{$a$}--(4,0);
% 		\end{tikzpicture}
% 	}
% 	\loigiai{
% 		Gọi $S_i$ và $C_i$ ($i=1,2,\ldots$) lần lượt là diện tích và chu vi của tam giác $H_i$, $i=1,2,\ldots $.\\
% 		Khi đó ta có
% 		\begin{itemize}
% 			\item[$\bullet$] $S_1=\dfrac{a^2\sqrt{3}}{4};S_2=\left(\dfrac{a}{2}\right)^2\dfrac{\sqrt{3}}{4}=\dfrac{a^2\sqrt{3}}{16}=\dfrac{S_1}{4};S_3=\dfrac{1}{4}S_2,\ldots $.\\
% 			Do đó $(S_n)$ là một cấp số nhân lùi vô hạn với $S_1=\dfrac{a^2\sqrt{3}}{4}$ và $q_1=\dfrac{1}{4}$.\\
% 			Tổng diện tích $S=S_1+S_2+\cdots =\dfrac{S_1}{1-q_1}=\dfrac{a^2\sqrt{3}}{3}$.
% 			\item[$\bullet$] $C_1=3a; C_2=\dfrac{3a}{2}=\dfrac{1}{2}C_1,\ldots$\\
% 			Do đó $(C_n)$ là một cấp số nhân lùi vô hạn với $C_1=3a;q_2=\dfrac{1}{2}$.\\
% 			Tổng chu vi là $C=C_1+C_2+\cdots =\dfrac{C_1}{1-q_2}=6a$.
% 		\end{itemize}
% 	}	
% \end{bt}
% \begin{bt}%[1D4K1-6]
% 	Một thấu kính hội tụ có tiêu cự là $f$. Gọi $d$ và $d'$ lần lượt là khoảng cách từ một vật thật $A B$ và từ ảnh $A' B'$ của nó tới quang tâm $O$ của thấu kính như hình vẽ bên dưới. Công thức thấu kính là $\dfrac{1}{d}+\dfrac{1}{d}=\dfrac{1}{f}$.
% 	\begin{center}
% 		\begin{tikzpicture}[scale=0.6, font=\footnotesize, line join=round, line cap=round, >=stealth]		
% 			%\draw[domain=-3.65:3.65, blue,] plot({3/10*(\x)^2},\x);
% 			%\draw[gray!10] (-1,-1) grid (5,5);
% 			\draw [red] (-10,0)--(9,0);
% 			\draw [dashed] (0,-3)--(0,2.5);
% 			\coordinate (A) at (-10,0); 
% 			\coordinate (A') at (7,0); 
% 			\coordinate (B) at (-10,1); 
% 			\coordinate (L) at ( 0,1);
% 			\coordinate (O) at (0,0); 
% 			\coordinate (B') at (7,-1); 
% 			\coordinate (F') at (5,0); 
% 			\coordinate (F) at (-5,0); 
% 			\coordinate (H) at (-5,1); 
% 			\coordinate (M) at (5,2); 
% 			\coordinate (M') at (-5,2); 
% 			\coordinate (S) at (-10,-3); 
% 			\coordinate (K) at (7,-3);
% 			\coordinate (K') at (-5.5,0.55); 
% 			\coordinate (N) at (5,-0.74); 
% 			\coordinate (G) at (2,0.45); 
% 			\coordinate (Q) at (-2.5,0.9); 
% 			\coordinate (Q') at (2.5,0.9);
% 			\coordinate (J) at (-5,-2.9); 
% 			\coordinate (J') at (5,-2.9);
% 			\coordinate (X) at (-8.5,0);
% 			\coordinate (V) at (0,-3); \coordinate (V') at (0,2);
% 			%\draw  (F)--(D) (G)--(C) (F')--(D') (G')--(C') ;
% 			\draw (B)--(L)--(B');
% 			\draw (B)--(O)--(B');\draw[red,->] (A)--(X);
% 			\draw[blue,->] (B)--(H); \draw[blue,->] (B)--(K');
% 			\draw[blue,->] (A)--(B); \draw[blue,->] (O)--(N);
% 			\draw[blue,->] (L)--(G); \draw[red,->] (A')--(B');
% 			\draw[dashed] (F)--(M')--(M)--(F');
% 			\draw[dashed] (A)--(S)--(K)--(B');
% 			\draw[dashed,<->] (S)--(V);\draw[dashed,<->] (V)--(K) ;
% 			\draw[dashed,<->] (M')--(V');\draw[dashed,<->] (V')--(M) ;
% 			%\draw[red,->] (E)--(H3); 
% 			%\draw[->] (C)--(K3) ;
% 			%\draw[red,->] (E)--(H4); 
% 			%\draw[->] (C')--(K4) ;
% 			%\draw[red,->] (E)--(H2); 
% 			%\draw[->] (D')--(K2) ;
			
% 			%\draw[red] (E)--(C) (E)--(D) (E)--(C') (E)--(D');
% 			\draw[fill=black] (A') circle (1pt) node [above] {$A'$};
% 			\draw (Q) node [above] {$f$}; \draw (Q') node [above] {$f$}; \draw (J) node [above] {$d$}; \draw (J') node [above] {$d'$};
% 			\draw[fill=black] (B') circle (1pt) node [right] {$B'$};
% 			\draw[fill=black] (F') circle (1pt) node [above right] {$F'$};
% 			\draw[fill=black] (F) circle (1pt) node [below] {$F$};
% 			\draw[fill=black] (A) circle (1pt) node [left] {$A$};
% 			\draw[fill=black] (B) circle (1pt) node [left] {$B$};
% 			%\draw[fill=black] (E) circle (1pt) node [below right] {$S$};
% 			\draw[fill=black] (O) circle (1pt) node [below left] {$O$};
% 		\end{tikzpicture}
% 	\end{center}
% 	\begin{listEX}
% 		\item[a)] Tìm biểu thức xác định hàm số $d'=\varphi(d)$.
% 		\item[b)] Tìm $\underset{d \rightarrow f^{+}}\lim \limits_{n \to +\infty}\varphi(d), \underset{d \rightarrow f^{-}}\lim \limits_{n \to +\infty}\varphi(d)$ và $\underset{d \rightarrow f}\lim \limits_{n \to +\infty}\varphi(d)$. Giải thích ý nghĩa của các kết quả tìm được.
% 	\end{listEX}
% 	\loigiai{
% 		\begin{listEX}
% 			\item[a)] Ta có $$\dfrac{1}{d}+\dfrac{1}{d'}=\dfrac{1}{f} \Leftrightarrow d'=\dfrac{d f}{d-f}.$$
% 			Vậy $\varphi(d)=\dfrac{d f}{d-f}$.		
% 			\item[b)] Vì $\underset{d \rightarrow f^{+}}\lim \limits_{n \to +\infty}df=f^2; \underset{d \rightarrow f^{+}}\lim \limits_{n \to +\infty}(d-f)=0 ; d \rightarrow f^{+} \Rightarrow d-f>0 $ nên $ \underset{d \rightarrow f^{+}}\lim \limits_{n \to +\infty} \dfrac{d f}{d-f}=+\infty$.\\
% 			Vậy $\underset{d \rightarrow f^{+}}\lim \limits_{n \to +\infty}\varphi(d)=\underset{d \rightarrow f^{+}}\lim \limits_{n \to +\infty} \dfrac{d f}{d-f}=+\infty$.\\
% 			\textbf{Ý nghĩa}: Khi đặt vật nằm ngoài tiêu cự và tiến dần đến tiêu điểm thì cho ảnh thật ngược chiều với vật ở vô cùng.\\
% 			Vì $\underset{d \rightarrow f^{-}}\lim \limits_{n \to +\infty}df=f^2; \underset{d \rightarrow f^{+}}\lim \limits_{n \to +\infty}(d-f)=0 ; d \rightarrow f^{-} \Rightarrow d-f<0 $ nên $ \underset{d \rightarrow f^{+}}\lim \limits_{n \to +\infty} \dfrac{d f}{d-f}=-\infty$.\\
% 			Vậy $\underset{d \rightarrow f^{+}}\lim \limits_{n \to +\infty}\varphi(d)=\underset{d \rightarrow f^{+}}\lim \limits_{n \to +\infty} \dfrac{d f}{d-f}=-\infty$.\\
% 			\textbf{Ý nghĩa}: Khi đặt vật nằm trong tiêu cự và tiến dần đến tiêu điểm thì cho ảnh ảo cùng chiều với vật và nằm ở vô cùng.\\
% 			Vì không tồn tại $\underset{d \rightarrow f^{+}}\lim \limits_{n \to +\infty}\varphi(d)$ và $\underset{d \rightarrow f^{-}}\lim \limits_{n \to +\infty}\varphi(d)$ nên không tồn tại $\underset{d \rightarrow f}\lim \limits_{n \to +\infty}\varphi(d)$.
% 		\end{listEX}
% 	}
% \end{bt}
\subsubsection{Câu hỏi trắc nghiệm}
\Opensolutionfile{ans}[ans/ans-1K5-1-Dang5]
\begin{ex}%[1C3K1-6]
	Có $1$ kg chất phóng xạ độc hại. Biết rằng, cứ sau một khoảng thời gian $T=24000$ năm thì một nửa số chất phóng xạ này bị phân rã thành chất khác không độc hại đối với sức khỏe của con người ($T$ được gọi là \textit{chu kì bán rã}). Gọi $u_n$ là khối lượng chất phóng xạ còn lại sau chu kì thứ $n$.
	Sau ít nhất bao nhiêu chu kì bán rã thì khối lượng phóng xạ đã cho ban đầu không còn độc hại với con người, biết rằng chất phóng xạ này sẽ không độc hại nữa nếu khối lượng chất phóng xạ còn lại bé hơn $10^{-6}$ g.
	\choice
	{$24$}
	{\True $30$}
	{$100$}
	{$15$}
	\loigiai{
		\item Chất phóng xạ sẽ không độc hại nữa nếu khối lượng chất phóng xạ còn lại bé hơn $10^{-6}~\mathrm{g}=10^{-9}$ kg
		$$\Leftrightarrow u_n<10^{-9}\Leftrightarrow\dfrac{1}{2^n}<10^{-9}\Leftrightarrow 2^n>10^9\Leftrightarrow n\geq 30.$$
		Vậy sau ít nhất $30$ chu kì thì khối lượng phóng xạ đã cho ban đầu không còn độc hại với con người nữa.
	}
\end{ex}
\begin{ex}%[1C3K1-6]
	Có $1$ kg chất phóng xạ độc hại. Biết rằng, cứ sau một khoảng thời gian $T=24000$ năm thì một nửa số chất phóng xạ này bị phân rã thành chất khác không độc hại đối với sức khỏe của con người ($T$ được gọi là \textit{chu kì bán rã}). Gọi $u_n$ là khối lượng chất phóng xạ còn lại sau chu kì thứ $n$.
	Sau ít nhất bao nhiêu năm thì khối lượng phóng xạ đã cho ban đầu không còn độc hại với con người, biết rằng chất phóng xạ này sẽ không độc hại nữa nếu khối lượng chất phóng xạ còn lại bé hơn $10^{-6}$ g.
	\choice
	{$30$}
	{$2400$}
	{\True $720000$}
	{$10000$}
	\loigiai{
		\item Chất phóng xạ sẽ không độc hại nữa nếu khối lượng chất phóng xạ còn lại bé hơn $10^{-6}~\mathrm{g}=10^{-9}$ kg
		$$\Leftrightarrow u_n<10^{-9}\Leftrightarrow\dfrac{1}{2^n}<10^{-9}\Leftrightarrow 2^n>10^9\Leftrightarrow n\geq 30.$$
		Vậy sau ít nhất $30$ chu kì bằng $30\cdot 24000=720000$ năm thì khối lượng phóng xạ đã cho ban đầu không còn độc hại với con người nữa.
	}
\end{ex}
\begin{ex}%[1C3K1-6]
	Từ hình vuông có độ dài cạnh bằng $1$, người ta nối các trung điểm của cạnh hình vuông để tạo ra hình vuông mới như hình bên. Tiếp tục quá trình này đến vô hạn. Tính tổng diện tích của tất cả các hình vuông được tạo thành.
	\choice
	{$1$}
	{\True $2$}
	{$3$}
	{$4$}	
	\loigiai{
		Từ giả thiết suy ra diện tích hình vuông sau bằng $\dfrac{1}{2}$ diện tích hình vuông trước.\\ 
		Khi đó diện tích của các hình vuông tạo thành một cấp số nhân lùi vô hạn với số hạng đầu $S_1=1$ và công bội $q=\dfrac{1}{2}$.\\
		Diện tích $S_n$ của hình vuông được tạo thành từ bước thứ $n$ là $S_n=S_1\cdot q^{n-1}=\left(\dfrac{1}{2}\right)^{n-1}$.\\
		Tổng diện tích của tất cả các hình vuông được tạo thành là:\\
		$$S=\dfrac{u_1}{1-q}=\dfrac{1}{1-\dfrac{1}{2}}=2.$$		
	}
\end{ex}
\begin{ex}%[1T3K1-5]
	Từ hình vuông đầu tiên có cạnh bằng $1$ (đơn vị độ dài), nối các trung điểm của bốn cạnh để có hình vuông thứ hai. Tiếp tục nối các trung điểm của bốn cạnh của hình vuông thứ hai để được hình vuông thứ ba. Cứ tiếp tục làm như thế, nhận được một dãy hình vuông. Tính tổng chu vi của dãy các hình vuông trên. 
	\choice
	{$8+\sqrt{2}$}
	{$2+\sqrt{2}$}
	{\True $8+4\sqrt{2}$}
	{$4+4\sqrt{2}$}
	\loigiai{
		Hình vuông thứ nhất có chu vi bằng $4$, hình vuông thứ $2$ có chu vi là $2\sqrt{2}$, hình vuông thứ $3$ có chu vi là $2$.\\
		Suy ra hình vuông thứ $n$ có chu vi bằng $p_n=4\cdot \left(\dfrac{\sqrt{2}}{2}\right)^{n-1}$.
		Tổng chu vi của $n$ hình vuông đầu tiên là tổng của cấp số nhân có  số hạng đầu $p_1=4$ và công bội $q=\dfrac{\sqrt{2}}{2}$ nên 
		\[ Q_n=\dfrac{p_1\left(1-q^n\right)}{1-q}=\dfrac{4\left(1-\left(\dfrac{\sqrt{2}}{2}\right)^n\right)}{1-\dfrac{\sqrt{2}}{2}}=\left(8+4\sqrt{2}\right)\left[1-\left(\dfrac{\sqrt{2}}{2}\right)^n\right].\]
		$\lim \limits_{n \to +\infty}Q_n=\left(8+4\sqrt{2}\right)\lim \limits_{n \to +\infty}\left[1-\left(\dfrac{\sqrt{2}}{2}\right)^n\right]=\left(8+4\sqrt{2}\right)\left(1-0\right)=8+4\sqrt{2}$.
	}
\end{ex}
\begin{ex}%[1D4K1-6]
	Từ độ cao $55,8 \mathrm{~m}$ của tháp nghiêng Pisa nước Ý, người ta thả một quả bóng cao su chạm xuống đất hình bên dưới. Giả sử mỗi lần chạm đất quả bóng lại nảy lên độ cao bằng $\dfrac{1}{10}$ độ cao mà quả bóng đạt được trước đó. Gọi $S_n$ là tổng độ dài quãng đường di chuyển của quả bóng tính từ lúc thả ban đầu cho đến khi quả bóng đó chạm đất $n$ lần. Tính $\lim \limits_{n \to +\infty}S_n$.
	\choice
	{$58{,}8$}
	{$67{,}2$}
	{$68$}
	{\True $68{,}2$}
	\loigiai{Mỗi khi chạm đất quả bóng lại nảy lên một độ cao bằng $\dfrac{1}{10}$ độ cao của lần rơi ngay trước đó và sau đó lại rơi xuống từ độ cao thứ hai này. Do đó, độ dài hành trình của quả bóng kể từ thời điểm rơi ban đầu đến:\\    
		Thời điểm chạm đất lần thứ nhất là $d_1=55{,}8$.\\
		Thời điềm chạm đất lần thứ hai là $d_2=55{,}8+2\cdot \dfrac{55{,}8}{10}$.\\
		Thời điểm chạm đất lần thứ ba là $d_3=55{,}8+2 \cdot\dfrac{55{,}8}{10}+2\cdot \dfrac{55{,}8}{10^2}$.\\
		Thời điểm chạm đất lần thứ tư là $d_4=55{,}8+2 \cdot\dfrac{55{,}8}{10}+2\cdot \dfrac{55,8}{10^2}+2\cdot \dfrac{55{,}8}{10^3}$.\\
		$\ldots$\\
		Thời điểm chạm đất lần thứ $n~(n>1)$ là
		$$d_n=55{,}8+2\cdot55{,}8+2\cdot \frac{55{,}8}{10^2}+2\cdot \frac{55{,}8}{10^3}+\ldots+2\cdot \frac{55{,}8}{10^{n-1}}.$$
		Do đó, quãng đường mà quả bóng đi được kể từ thời điềm rơi đến khi nằm yên trên mặt đất là:
		$$ d=55{,}8+2.55{,}8+2\cdot \frac{55{,}8}{10^2}+2\cdot \frac{55{,}8}{10^3}+\ldots+2\cdot \frac{55{,}8}{10^{n-1}}+\ldots=\lim \limits_{n \to +\infty}d_n.$$
		Vì $2\cdot \dfrac{55{,}8}{10} ; 2\cdot \dfrac{55{,}8}{10^2} ; 2\cdot \dfrac{55{,}8}{10^3}; \ldots ; 2\cdot \dfrac{55{,}8}{10^{n-1}}; \ldots$ là một cấp số nhân lùi vô hạn với công bội $q=\dfrac{1}{10}$ nên ta có:
		$$ 2 \cdot\dfrac{55,8}{10}+2\cdot \dfrac{55{,}8}{10^2}+2\cdot \dfrac{55{,}8}{10^3}+\ldots+2\cdot \dfrac{55{,}8}{10^{n-1}}+\ldots=\dfrac{2\cdot \dfrac{55{,}8}{10}}{1-\dfrac{1}{10}}=12{,}4.$$
		Vậy $d=55{,}8+12{,}4=68{,}2$ m.
	}
\end{ex}
\Closesolutionfile{ans}
% \begin{indapan}{10}
% 	{ans/ans-1K5-1-Dang5}
% \end{indapan}
\begin{dang}{Nguyên lý kẹp}
	Để tìm giới hạn của dãy số theo nguyên lý kẹp ta cần nhớ:
	\begin{itemize}
		\item Cho hai dãy số $(u_n)$ và $(v_n)$. Nếu $|u_n| \leq v_n$ với mọi $n$ và $\lim \limits_{n \to +\infty}v_n = 0$ thì $\lim \limits_{n \to +\infty}u_n =0$.
		\item Cho ba dãy số $(u_n)$, $(v_n)$ và $(w_n)$. Nếu $ u_n \leq v_n \leq w_n$ với mọi $n$ và $\lim \limits_{n \to +\infty}u_n = \lim \limits_{n \to +\infty}w_n = L$ thì $\lim \limits_{n \to +\infty}v_n = L$.
	\end{itemize}
\end{dang}
\subsubsection{Ví dụ minh hoạ}
%bai1
\begin{vd}%[1D4B1-2]
	Chứng minh rằng các dãy số với số hạng tổng quát sau đây có giới hạn $0$.
	\begin{enumEX}[a)]{2}
		\item $u_n=\dfrac{(-1)^n}{3n+2}$.
		\item $u_n=\dfrac{n\sin 2n}{n^3+2}$.
		\item $u_n=\dfrac{(-1)^n\cos n}{\sqrt{n}}$.
		\item $u_n=\dfrac{3\sin n-4\cos n}{2n^2+1}$.
	\end{enumEX}
	\loigiai{
		\begin{enumerate}[a)]
			\item Ta có: $0 \leqslant \left|u_n\right|= \dfrac{1}{3n+2}<\dfrac{1}{3n}<\dfrac{1}{n}$, $\forall n\in \mathbb{N^*}$.\\
			Mà $\lim \limits_{n \to +\infty}\dfrac{1}{n}=0$ nên suy ra $\lim \limits_{n \to +\infty}\dfrac{(-1)^n}{3n+2}=0$.
			\item Ta có $0\leqslant \left|u_n\right|=\dfrac{n\left|\sin 2n\right|}{n^3+2}\leqslant \dfrac{n}{n^3+2}<\dfrac{n}{n^3}\leqslant \dfrac{1}{n^2}$, $\forall n\in \mathbb{N^*}$.\\
			Mà $\lim \limits_{n \to +\infty}\dfrac{1}{n^2}=0$ nên suy ra $\lim \limits_{n \to +\infty}\dfrac{n\sin 2n}{n^3+2}=0$.
			\item Ta có $0\leqslant \left|\dfrac{(-1)^n\cos n}{\sqrt{n}}\right|\leqslant \dfrac{1}{\sqrt{n}}$, $\forall n\in \mathbb{N^*}$.\\
			Mà $\lim \limits_{n \to +\infty}\dfrac{1}{\sqrt{n}}=0$ nên suy ra $\lim \limits_{n \to +\infty}\dfrac{(-1)^n\cos n}{\sqrt{n}}=0$.
			\item Theo bất đẳng thức Bunhiacopxki, ta có\\
			$\left|3\sin n-4\cos n\right|\leqslant \sqrt{(3^2+4^2)\left(\sin^2n+\cos^2n\right)}=5$.\\
			Do đó $0\leqslant \left|\dfrac{3\sin n-4\cos n}{2n^2+1}\right|\leqslant \dfrac{5}{2n^2+1}<\dfrac{5}{2n^2}$, $\forall n\in \mathbb{N^*}$.\\
			Mà $\lim \limits_{n \to +\infty}\dfrac{5}{2n^2+1}=0$ nên suy ra $\lim \limits_{n \to +\infty}\dfrac{3\sin n-4\cos n}{2n^2+1}=0$.
		\end{enumerate}
	}
\end{vd}
%bai2
\begin{vd}%[1D4K1-2]
	Chứng minh rằng các dãy số với số hạng tổng quát sau đây có giới hạn $0$. 
	\begin{enumEX}{2}
		\item $u_n=\sqrt{n^3+2}-\sqrt{n^3+1}$.
		\item $u_n=\dfrac{3^n\sin 2n+4^n}{2^n+4\cdot 5^n}$.
		\item $u_n=\dfrac{n+\sin 2n}{n^2+n}$.
		\item $\dfrac{n+\cos \dfrac{n\pi}{5}}{n\sqrt{n}+\sqrt{n}}$.
	\end{enumEX}
	\loigiai{
		\begin{enumerate}
			\item Ta có $u_n=\sqrt{n^3+2}-\sqrt{n^3+1}=\dfrac{n^3+2-(n^3+1)}{\sqrt{n^3+2}+\sqrt{n^3+1}}=\dfrac{1}{\sqrt{n^3+2}+\sqrt{n^3+1}}$.\\
			Do đó $0\leqslant \left|u_n\right|=\dfrac{1}{\sqrt{n^3+2}+\sqrt{n^3+1}}<\dfrac{2}{\sqrt{n^2}+\sqrt{n^2}}=\dfrac{2}{2n}=\dfrac{1}{n}$, $\forall n\in \mathbb{N^*}$.\\
			Mà $\lim \limits_{n \to +\infty}\dfrac{1}{n}=0$ nên suy ra $\lim \limits_{n \to +\infty}\left(\sqrt{n^3+2}-\sqrt{n^3+1}\right)=0$.
			\item Ta có $0\leqslant \left|\dfrac{3^n\sin 2n+4^n}{2^n+4.5^n}\right|\leqslant \dfrac{3^n\left|\sin 2n\right|+4^n}{2^n+4\cdot 5^n}\leqslant \dfrac{3^n+4^n}{2^n+4\cdot 5^n}=\dfrac{{\left(\dfrac{3}{5}\right)}^n+{\left(\dfrac{4}{5}\right)}^n}{{\left(\dfrac{2}{5}\right)}^n+4}$, $\forall n\in \mathbb{N^*}$.\\
			Mà $\lim \limits_{n \to +\infty}\left[{\left(\dfrac{3}{5}\right)}^n+{\left(\dfrac{4}{5}\right)}^n\right]=\lim \limits_{n \to +\infty}{\left(\dfrac{3}{5}\right)}^n+\lim \limits_{n \to +\infty}{\left(\dfrac{4}{5}\right)}^n=0$ và $$\lim \limits_{n \to +\infty}\left[{\left(\dfrac{2}{5}\right)}^n+4\right]=\lim \limits_{n \to +\infty}{\left(\dfrac{2}{5}\right)}^n+4=0+4=4$$ nên $\lim \limits_{n \to +\infty}\dfrac{3^n\sin 2n+4^n}{2^n+4\cdot 5^n}=0$.
			\item Ta có $0\leqslant \left|\dfrac{n+\sin 2n}{n^2+n}\right|\leqslant \dfrac{n+\left|\sin 2n\right|}{n^2+n}\leqslant \dfrac{n+1}{n^2+n}=\dfrac{1}{n}$, $\forall n\in \mathbb{N^*}$. \\
			Mà $\lim \limits_{n \to +\infty}\dfrac{1}{n}=0$ nên suy ra $\lim \limits_{n \to +\infty}\dfrac{n+\sin 2n}{n^2+n}=0$.
			\item Ta có $0\leqslant \left|\dfrac{n+\cos \dfrac{n\pi}{5}}{n\sqrt{n}+\sqrt{n}}\right|\leqslant \dfrac{n+\left|\cos \dfrac{n\pi}{5}\right|}{n\sqrt{n}+\sqrt{n}}\leqslant \dfrac{n+1}{n\sqrt{n}+\sqrt{n}}=\dfrac{1}{\sqrt{n}}$, $\forall n\in \mathbb{N^*}$. \\
			Mà $\lim \limits_{n \to +\infty}\dfrac{1}{\sqrt{n}}=0$ nên suy ra $\lim \limits_{n \to +\infty}\dfrac{n+\cos \dfrac{n\pi}{5}}{n\sqrt{n}+\sqrt{n}}=0$.
		\end{enumerate}
	}
\end{vd}
%bai3
\begin{vd}%[1D4B1-1]
	Cho dãy số $(u_n)$ với $u_n=\dfrac{n}{3^n}$.
	\begin{enumerate}[a)]
		\item Chứng minh rằng $\dfrac{u_{n+1}}{u_n}\leqslant \dfrac{2}{3}$ với mọi $n\in \mathbb{N^*}$.
		\item Bằng phương pháp quy nạp chứng minh rằng $0<u_n<{\left(\dfrac{2}{3}\right)}^n$ với mọi $n\in \mathbb{N^*}$.
		\item Dãy $(u_n)$ có giới hạn $0$.
	\end{enumerate}
	\loigiai{
		\begin{enumerate}[a)]
			\item Với mọi $n\in \mathbb{N}$, ta có $\dfrac{u_{n+1}}{u_n}=\dfrac{\dfrac{n+1}{3^{n+1}}}{\dfrac{n}{3^n}}=\dfrac{n+1}{n}\cdot \dfrac{1}{3}$. \\
			Mặt khác, $n+1\leqslant n+n\leqslant 2n$. Suy ra $\dfrac{n+1}{n}\leqslant 2$. \\
			Do đó $\dfrac{u_{n+1}}{u_n}\leqslant \dfrac{2}{3}$ với mọi $n\in \mathbb{N^*}$.
			\item Rõ ràng với mọi $n\in \mathbb{N^*}$, ta có $u_n>0$. Do đó ta chỉ cần chứng minh $u_n<{\left(\dfrac{2}{3}\right)}^n$.
			\begin{itemize}
				\item Với $n=1$, ta có $u_1=\dfrac{1}{3^1}=\dfrac{1}{3}<{\left(\dfrac{2}{3}\right)}^1$. Nghĩa là mệnh đề đúng với $n=1$. \\
				\item Giả sử mệnh đề đúng với $n=k\geqslant 1$, tức là $u_k<{\left(\dfrac{2}{3}\right)}^k$. \\
				\item Bây giờ ta cần chứng minh mệnh đề đúng với $n=k+1$, tức là cần chứng minh $u_{k+1}<{\left(\dfrac{2}{3}\right)}^{k+1}$. \\
				Theo chứng minh câu a) ta có $\dfrac{u_{k+1}}{u_k}\leqslant \dfrac{2}{3}$ suy ra $u_{k+1}\leqslant \dfrac{2}{3}\cdot u_k<\dfrac{2}{3}\cdot {\left(\dfrac{2}{3}\right)}^k={\left(\dfrac{2}{3}\right)}^{k+1}$ hay $u_{k+1}<{\left(\dfrac{2}{3}\right)}^{k+1}$. \\
				Nghĩa là mệnh đề cũng đúng với $n=k+1$. Vậy $0<u_n<{\left(\dfrac{2}{3}\right)}^n$ với mọi $n\in \mathbb{N^*}$.
			\end{itemize}
			\item Theo câu b), ta có $0<u_n<{\left(\dfrac{2}{3}\right)}^n$. Mà $\lim \limits_{n \to +\infty}{\left(\dfrac{2}{3}\right)}^n=0$. Do đó $\lim \limits_{n \to +\infty}u_n=0$.
		\end{enumerate}
	}
\end{vd}
%bai4
\begin{vd}%[1D4B1-2]
	Chứng minh rằng
	\begin{enumEX}[a)]{2}
		\item $\lim \limits_{n \to +\infty}\left(\dfrac{-n^3}{n^3+1}\right)=-1$.
		\item $\lim \limits_{n \to +\infty}\left(\dfrac{n^2+3n+2}{2n^2+n}\right)=\dfrac{1}{2}$.
	\end{enumEX}
	\loigiai{
		\begin{enumerate}
			\item Ta có $\lim \limits_{n \to +\infty}\left(\dfrac{-n^3}{n^3+1}-(-1)\right)=\lim \limits_{n \to +\infty}\left(\dfrac{1}{n^3+1}\right)$. \\
			Vì $0\leqslant \left|\dfrac{1}{n^3+1}\right|<\dfrac{1}{n^3}$, $\forall n\in \mathbb{N^*}$. Mà $\lim \limits_{n \to +\infty}\dfrac{1}{n^3}=0$ nên suy ra $\lim \limits_{n \to +\infty}\left(\dfrac{1}{n^3+1}\right)=0$. \\
			Do đó $\lim \limits_{n \to +\infty}\left(\dfrac{-n^3}{n^3+1}\right)=-1$.
			\item Ta có $\lim \limits_{n \to +\infty}\left(\dfrac{n^2+3n+2}{2n^2+n}-\dfrac{1}{2}\right)=\lim \limits_{n \to +\infty}\dfrac{5n+4}{2(2n^2+n)}$. \\
			Vì $0<\left|\dfrac{5n+4}{2(2n^2+n)}\right|<\dfrac{5n+5}{2n(n+1)}=\dfrac{5}{2}\cdot \dfrac{1}{n}$, $\forall n\in \mathbb{N^*}$. Mà $\lim \limits_{n \to +\infty}\left(\dfrac{5}{2}\cdot \dfrac{1}{n}\right)=\dfrac{5}{2}\cdot \lim \limits_{n \to +\infty}\dfrac{1}{n}=0$ nên suy ra $\lim \limits_{n \to +\infty}\dfrac{5n+4}{2(2n^2+n)}=0$. \\
			Do đó $\lim \limits_{n \to +\infty}\left(\dfrac{n^2+3n+2}{2n^2+n}\right)=\dfrac{1}{2}$.
		\end{enumerate}
	}
\end{vd}
%bai5
\begin{vd}%[1D4K1-2]
	Chứng minh rằng
	\begin{enumEX}[a)]{2}
		\item $\lim \limits_{n \to +\infty}\left(\dfrac{3\cdot 3^n-\sin 3n}{3^n}\right)=3$.
		\item $\lim \limits_{n \to +\infty}\left(\sqrt{n^2+n}-n\right)=\dfrac{1}{2}$.
	\end{enumEX}
	\loigiai{
		\begin{enumerate}[a)]
			\item Ta có $\lim \limits_{n \to +\infty}\left(\dfrac{3.3^n-\sin 3n}{3^n}-3\right)=\lim \limits_{n \to +\infty}\left(\dfrac{-\sin 3n}{3^n}\right)$. \\
			Vì $0\leqslant \left|\dfrac{-\sin 3n}{3^n}\right|=\dfrac{\left|-\sin 3n\right|}{3^n}\leqslant \dfrac{1}{3^n}={\left(\dfrac{1}{3}\right)}^n$, $\forall n\in \mathbb{N^*}$. Mà $\lim \limits_{n \to +\infty}{\left(\dfrac{1}{3}\right)}^n=0$ nên suy ra $\lim \limits_{n \to +\infty}\left(\dfrac{-\sin 3n}{3^n}\right)=0$. \\
			Do đó $\lim \limits_{n \to +\infty}\left(\dfrac{3.3^n-\sin 3n}{3^n}\right)=3$.
			\item Ta có $\lim \limits_{n \to +\infty}\left(\sqrt{n^2+n}-n-\dfrac{1}{2}\right)=\lim \limits_{n \to +\infty}\dfrac{2\sqrt{n^2+n}-(2n+1)}{2}=\lim \limits_{n \to +\infty}\dfrac{-1}{2\left(2\sqrt{n^2+n}+(2n+1)\right)}$. \\
			Vì $0\leqslant \left|\dfrac{-1}{2\left(2\sqrt{n^2+n}+(2n+1)\right)}\right| \leqslant \dfrac{1}{2\left(2\sqrt{n^2+n}+(2n+1)\right)}\leqslant \dfrac{1}{2\left(2\sqrt{n^2}+2n\right)}=\dfrac{1}{8}\cdot \dfrac{1}{n}$, $\forall n\in \mathbb{N^*}$. \\
			Mà $\lim \limits_{n \to +\infty}\dfrac{1}{8}\cdot \dfrac{1}{n}=\dfrac{1}{8}\lim \limits_{n \to +\infty}\dfrac{1}{n}=0$ nên suy ra $\lim \limits_{n \to +\infty}\left(\sqrt{n^2+n}-n-\dfrac{1}{2}\right)$. \\
			Do đó $\lim \limits_{n \to +\infty}\left(\sqrt{n^2+n}-n\right)=\dfrac{1}{2}$.
		\end{enumerate}
	}
\end{vd}
%Bài 6
\begin{vd}%[1D4K1-5]
	Tìm các giới hạn sau
	\begin{enumEX}[a)]{1}
		\item $\lim \limits_{n \to +\infty}\left(\dfrac{1}{\sqrt{4n^2+1}}+\dfrac{1}{\sqrt{4n^2+2}}+\cdots +\dfrac{1}{\sqrt{4n^2+n}}\right)$.
		\item $\lim \limits_{n \to +\infty}\dfrac{{1\cdot 3\cdot 5\cdot 7}\cdots (2n-1)}{{2\cdot 4\cdot 6}\cdots (2n)}$.
	\end{enumEX}
	\loigiai{
		\begin{enumerate}[a)]
			\item Ta có
			\begin{align*}
				\dfrac{1}{\sqrt{4n^2}}+\dfrac{1}{\sqrt{4n^2}}+\cdots +\dfrac{1}{\sqrt{4n^2}} &\leqslant \dfrac{1}{\sqrt{4n^2+1}}+\dfrac{1}{\sqrt{4n^2+2}}+\cdots +\dfrac{1}{\sqrt{4n^2+n}}\\
				&\leqslant \dfrac{1}{\sqrt{4n^2+n}}+\dfrac{1}{\sqrt{4n^2+n}}+\cdots +\dfrac{1}{\sqrt{4n^2+n}}.
			\end{align*}
			hay
			\begin{center}
				$\dfrac{n}{\sqrt{4n^2}}\leqslant \dfrac{1}{\sqrt{4n^2+1}}+\dfrac{1}{\sqrt{4n^2+2}}+\cdots +\dfrac{1}{\sqrt{4n^2+n}}\leqslant \dfrac{n}{\sqrt{4n^2+n}}$ với mọi $n\in \mathbb{N^*}$.
			\end{center}
			Mà $\lim \limits_{n \to +\infty}\dfrac{n}{\sqrt{4n^2}}=\lim \limits_{n \to +\infty}\dfrac{1}{2}=\dfrac{1}{2}$; $\lim \limits_{n \to +\infty}\dfrac{n}{\sqrt{4n^2+n}}=\lim \limits_{n \to +\infty}\dfrac{1}{\sqrt{4+\dfrac{1}{n}}}=\dfrac{1}{\sqrt{4+0}}=\dfrac{1}{2}$. \\
			Do đó $\lim \limits_{n \to +\infty}\left(\dfrac{1}{\sqrt{4n^2+1}}+\dfrac{1}{\sqrt{4n^2+2}}+\cdots +\dfrac{1}{\sqrt{4n^2+n}}\right)=\dfrac{1}{2}$.
			\item Ta có $u_n=\dfrac{{1\cdot 3\cdot 5\cdot 7}\cdots (2n-1)}{{2\cdot 4\cdot 6}\cdots (2n)}$, suy ra $$u_n^2=\dfrac{1^2\cdot 3^2\cdot 5^2\cdot 7^2\cdots (2n-1)^2}{2^2\cdot 4^2\cdot 6^2\cdots (2n)^2}=\dfrac{1\cdot 3}{2^2}\cdot \dfrac{3\cdot 5}{4^2}\cdots \dfrac{(2n-1)(2n+1)}{(2n)^2}\cdot \dfrac{1}{2n+1}<\dfrac{1}{2n+1}.$$
			(do $\dfrac{1\cdot 3}{2^2}\cdot \dfrac{3\cdot 5}{4^2}\cdots \dfrac{(2n-1)(2n+1)}{(2n)^2}<\dfrac{2^2}{2^2}\cdot \dfrac{4^2}{4^2}\cdots \dfrac{(2n)^2}{(2n)^2}=1$ ) \\
			Vậy ta có $0<u_n<\dfrac{1}{\sqrt{2n+1}}$, $\forall n\in \mathbb{N^*}$. Mà $\lim \limits_{n \to +\infty}\dfrac{1}{\sqrt{2n+1}}=0$ nên suy ra $$\lim \limits_{n \to +\infty}\dfrac{{1\cdot 3\cdot 5\cdot 7}\cdots (2n-1)}{{2\cdot 4\cdot 6}\cdots (2n)}=0.$$
		\end{enumerate}
	}
\end{vd}
% \subsubsection{Bài tập rèn luyện}
% \subsubsection{Câu hỏi trắc nghiệm}
% \Opensolutionfile{ans}[ans/ans-1K5-1-Dang6]

% %%==========Câu 1
% \begin{ex}%[1D4B1-2]
% 	Giới hạn $\displaystyle\lim\dfrac{\sin n+1}{n}$ bằng
% 	\choice
% 	{$+\infty$}
% 	{$1$}
% 	{$-\infty$}
% 	{\True $0$}
% 	\loigiai{
% 		Với mọi $n>0$ thì $|\sin n+1|\leq 2$. Do đó, với mọi $n>0$, ta có
% 		$$0\leq \left|\dfrac{\sin n+1}{n}\right|\leq \dfrac{2}{n}.$$
% 		Từ đó $$0\leq \lim\limits\left|\dfrac{\sin n+1}{n}\right|\leq \lim\limits\dfrac{2}{n}=0\Rightarrow \lim\limits\left|\dfrac{\sin n+1}{n}\right|=0\Rightarrow \lim\limits\dfrac{\sin n+1}{n}=0.$$
% 	}
% \end{ex}
% %%==========Câu 2
% \begin{ex}%[1D4B1-2]
% 	Giới hạn $\displaystyle\lim\dfrac{\sin n+1}{n}$ bằng
% 	\choice
% 	{$+\infty$}
% 	{$1$}
% 	{$-\infty$}
% 	{\True $0$}
% 	\loigiai{
% 		Với mọi $n>0$ thì $|\sin n+1|\leq 2$. Do đó, với mọi $n>0$, ta có
% 		$$0\leq \left|\dfrac{\sin n+1}{n}\right|\leq \dfrac{2}{n}.$$
% 		Từ đó $$0\leq \lim\limits\left|\dfrac{\sin n+1}{n}\right|\leq \lim\limits\dfrac{2}{n}=0\Rightarrow \lim\limits\left|\dfrac{\sin n+1}{n}\right|=0\Rightarrow \lim\limits\dfrac{\sin n+1}{n}=0.$$
% 	}
% \end{ex}
% %%==========Câu 3
% \begin{ex}%[1D4B1-2]
% 	Giới hạn $\lim \limits_{n \to +\infty}\dfrac{\cos n}{n}$ bằng
% 	\choice
% 	{$1$}
% 	{\True $0$}
% 	{$-1$}
% 	{$+\infty$}
% 	\loigiai{
% 		Ta có: $\left| \dfrac{\cos n}{n} \right|\le \dfrac{1}{n}$ và $\lim \limits_{n \to +\infty}\dfrac{1}{n}=0$ nên $\lim \limits_{n \to +\infty}\dfrac{\cos n}{n}=0$.
% 	}
% \end{ex}
% %%==========Câu 4
% \begin{ex}%[1D4B1-2]
% 	Tính $\lim \limits_{n \to +\infty}\dfrac{\sin n}{n^3+1}$.
% 	\choice
% 	{$1$}
% 	{\True $0$}
% 	{$-\infty$}
% 	{$+\infty$}
% 	\loigiai{
% 		Ta có
% 		$ \left|\dfrac{\sin n}{n^3+1}\right|\le \dfrac{1}{n^3+1}$ mà
% 		$\lim \limits_{n \to +\infty}\dfrac{1}{n^3+1}=\lim \limits_{n \to +\infty}\dfrac{1}{n^3\left(1+\dfrac{1}{n^3}\right)}=0$.\\
% 		Vậy $\lim \limits_{n \to +\infty}\dfrac{\sin n}{n^3+1}=0$
% 	}
% \end{ex}
% %%==========Câu 5
% \begin{ex}%[1D4B1-2]
% 	Tính $\lim \limits_{n \to +\infty}\dfrac{\sin 2024n}{n}$.
% 	\choice
% 	{\True $0$}
% 	{$1$}
% 	{$+ \infty$}
% 	{$2024$}
% 	\loigiai{
% 		Ta có $-1 \leqslant \sin 2024n \leqslant 1  \Leftrightarrow - \dfrac{1}{n} \leqslant \dfrac{\sin 2024n}{n} \leqslant \dfrac{1}{n}$.\\
% 		Vì $\lim \limits_{n \to +\infty}\left( - \dfrac{1}{n} \right) = \lim \limits_{n \to +\infty} \dfrac{1}{n} = 0$ nên $\lim \limits_{n \to +\infty}\dfrac{\sin 2024n}{n} = 0$.
% 	}
% \end{ex}
% %%==========Câu 6
% \begin{ex}%[1D4K1-2]
% 	Tính $I=\lim \limits_{n \to +\infty}\left(\dfrac{1}{\sqrt{n^2+n+1}}+\dfrac{1}{\sqrt{n^2+n+2}}+...+\dfrac{1}{\sqrt{n^2+2n}}\right)$.
% 	\choice
% 	{$I=+\infty$}
% 	{$I=3$}
% 	{$I=2$}
% 	{\True $I=1$}
% 	\loigiai{
% 		Ta có $\dfrac{1}{\sqrt{n^2+2n}}<\dfrac{1}{\sqrt{n^2+n+k}}<\dfrac{1}{\sqrt{n^2+n+1}},\forall k=2,3,...,n-1$.\\
% 		$\Rightarrow \dfrac{n}{\sqrt{n^2+2n}}<\dfrac{1}{\sqrt{n^2+n+1}}+\dfrac{1}{\sqrt{n^2+n+2}}+\cdots+\dfrac{1}{\sqrt{n^2+2n}}<\dfrac{n}{\sqrt{n^2+n+1}}$.\\
% 		Mà $\lim \limits_{n \to +\infty}\dfrac{n}{\sqrt{n^2+2n}}=\lim \limits_{n \to +\infty}\dfrac{1}{\sqrt{1+\dfrac{2}{n}}}=1$; $\lim \limits_{n \to +\infty}\dfrac{n}{\sqrt{n^2+n+1}}=\lim \limits_{n \to +\infty}\dfrac{1}{\sqrt{1+\dfrac{1}{n}+\dfrac{1}{n^2}}}=1$.\\
% 		Vậy $I=1$.
% 	}
% \end{ex}
% %%==========Câu 7
% \begin{ex}%[1D4K1-2]
% 	Tính $T=\lim\dfrac{n\sin n-3n^2}{n^2}$.
% 	\choice
% 	{$T=+\infty$}
% 	{$T=-\infty$}
% 	{$T=1$}
% 	{\True $T=-3$}
% 	\loigiai{
% 		Ta có $T=\lim\dfrac{n\sin n-3n^2}{n^2}=\lim\left(\dfrac{\sin n}{n}-3\right)$.\\
% 		Do $-1\le \sin n\le 1$ suy ra $-\dfrac{1}{n}\le \dfrac{\sin n}{n}\le \dfrac{1}{n}$, mà $\lim\left(-\dfrac{1}{n}\right)=0$ và $\lim\dfrac{1}{n}=0$ nên $\lim\dfrac{\sin n}{n}=0$.\\
% 		Từ đó suy ra $T=\lim\left(\dfrac{\sin n}{n}-3\right)=-3$.
% 	}
% \end{ex}
% %%==========Câu 8
% \begin{ex}%[1D4K1-2]
% 	Tính giá trị của $I=\lim\dfrac{n^3+n\cdot\sin^2n}{10000n^3-n+2}$.
% 	\choice
% 	{\True $I=0{,}0001$}
% 	{$I=\dfrac{1}{1000}$}
% 	{$I=0$}
% 	{$I=0{,}00001$}
% 	\loigiai{Ta có: $I=\lim\dfrac{n^3+n\cdot\sin^2n}{10000n^3-n+2}=\lim\dfrac{1+\dfrac{\sin^2 n}{n^2}}{10000-\dfrac{1}{n^2}+\dfrac{2}{n^3}}=\dfrac{1}{10000}=0{,}0001$.\\
% 		Chú ý rằng: $0\leq\dfrac{\sin^2n}{n^2}\leq\dfrac{1}{n^2}$. Mà $\lim\dfrac{1}{n^2}=0\Rightarrow\lim\dfrac{\sin^2n}{n^2}=0$.}
% \end{ex}
% %%==========Câu 9
% \begin{ex}%[1D4G1-2]
% 	Tính $I=\lim \limits_{n \to +\infty}\left(\dfrac{1}{2}+\dfrac{3}{2^2}+\dfrac{5}{2^3}+\cdots +\dfrac{2n-1}{2^n}\right)$.
% 	\choice
% 	{\True $I=3$}
% 	{$I=0$}
% 	{$I=\dfrac{1}{2}$}
% 	{$I=+\infty $}
% 	\loigiai{
% 		Đặt $S_n=\dfrac{1}{2}+\dfrac{3}{2^2}+\dfrac{5}{2^3}+\cdots +\dfrac{2n-1}{2^n}$. \\
% 		Khi đó $\dfrac{1}{2}S_n=\dfrac{1}{2^2}+\dfrac{3}{2^3}+\dfrac{5}{2^4}+\cdots +\dfrac{2n-3}{2^n}+\dfrac{2n-1}{2^{n+1}}$. \\
% 		Trừ vế theo vế ta được \\
% 		$S_n-\dfrac{1}{2}S_n=\dfrac{1}{2}+\left(\dfrac{2}{2^2}+\dfrac{2}{2^3}+\cdots +\dfrac{2}{2^n}\right)-\dfrac{2n-1}{2^{n+1}}$. \\
% 		Từ đó $S_n=1+\left(1+\dfrac{1}{2}+\dfrac{1}{4}+\cdots +\dfrac{1}{2^{n-2}}\right)-\dfrac{2n-1}{2^n}=1+2\left[1-{\left(\dfrac{1}{2}\right)}^{n-1}\right]-\dfrac{2n-1}{2^n}$. \\
% 		Với mọi $n\geqslant 4$ ta có $2^n\geqslant n^2$. Thật vậy,
% 		\begin{itemize}
% 			\item Ta có $2^4 \geqslant 4^2$.
% 			\item Nếu $2^k \geqslant k^2~(k \geqslant 4)$ thì $2^{k+1}=2\cdot 2^k \geqslant 2\cdot k^2>k^2+(2k+1)=(k+1)^2$ (do $k \geqslant 4$).
% 		\end{itemize}
% 		Từ đó $0<\dfrac{2n-1}{2^n}\leqslant \dfrac{2n-1}{n^2}$. Mà $\lim \limits_{n \to +\infty}\dfrac{2n-1}{n^2}=0$ nên $\lim \limits_{n \to +\infty}\dfrac{2n-1}{2^n}=0$. \\
% 		Vậy $I=\lim \limits_{n \to +\infty}S_n=\lim \limits_{n \to +\infty}\left[1+2\left[1-{\left(\dfrac{1}{2}\right)}^{n-1}\right]-\dfrac{2n-1}{2^n}\right]=1+2=3$.}
% \end{ex}
% %%==========Câu 10
% \begin{ex}%[1D4G1-2]
% 	Cho dãy số $(u_n)$ được xác định bởi $\heva{& u_1=3 \\ & 2(n+1)u_{n+1}=nu_n+n+2}$. Tính $\lim \limits_{n \to +\infty}u_n$.
% 	\choice
% 	{\True $\lim \limits_{n \to +\infty}u_n=1$}
% 	{$\lim \limits_{n \to +\infty}u_n=4$}
% 	{$\lim \limits_{n \to +\infty}u_n=3$}
% 	{$\lim \limits_{n \to +\infty}u_n=0$}
% 	\loigiai{
% 		Ta chứng minh $1\le u_{n+1}\le 1+\dfrac{1}{2n}$, $\forall n\ge 1$. Thật vậy
% 		\begin{itemize}
% 			\item $u_{n+1}\ge 1$, $\forall n\ge 1$ $(1)$.\\
% 			Với $n=1\Rightarrow u_2=\dfrac{3}{2}\ge 1\Rightarrow (1)$ đúng với $n=1$.\\
% 			Giả sử $(1)$ đúng với $n=k\ge 1$, tức là $u_{k+1}\ge 1$.\\
% 			Ta cần chứng minh $(1)$ đúng với $n=k+1$, tức là chứng minh $u_{k+2}\ge 1$. Thật vậy\\
% 			$u_{k+2}=\dfrac{(k+1) u_{k+1}+1}{2(k+2)}+\dfrac{1}{2}\ge \dfrac{k+2}{2(k+2)}+\dfrac{1}{2}=1$.
% 			\item $u_{n+1}\le 1+\dfrac{1}{2n}$, $\forall n\ge 1$ $(2)$.\\
% 			Với $n=1\Rightarrow u_2=\dfrac{3}{2}\le 1+\dfrac{3}{2}\Rightarrow (2)$ đúng với $n=1$.\\
% 			Giả sử $(2)$ đúng với $n=k\ge 1$, tức là $u_{k+1}\le 1+\dfrac{1}{2k}$.\\
% 			Ta cần chứng minh $(2)$ đúng với $n=k+1$, tức là chứng minh $u_{k+2}\le 1+\dfrac{1}{2(k+1)}$. Thật vậy
% 			$$u_{k+2}=\dfrac{(k+1)u_{k+1}+1}{2(k+2)}+\dfrac{1}{2}\le \dfrac{(k+1)\left(1+\dfrac{1}{2(k+1)}\right)}{2(k+2)}+\dfrac{1}{2}\le 1+\dfrac{1}{4(k+2)}\le 1+\dfrac{1}{2(k+1)}.$$
% 			Vậy $\lim \limits_{n \to +\infty}u_n=1$.
% 		\end{itemize}
% 	}
% \end{ex}
% %%==========Câu 11
% \begin{ex}%[1D4G1-2]
% 	Cho dãy số $\left( u_n\right) $ thỏa mãn $\heva{&u_1=\dfrac{1}{3}\\&u_{n+1}=\dfrac{(n+1)u_n}{3n},\,\, \forall n\geq 1}$. Có bao nhiêu số nguyên dương $n$ thỏa mãn $u_n<\dfrac{1}{2020}.$
% 	\choice
% 	{$0$}
% 	{$9$}
% 	{\True vô số}
% 	{$5$}
% 	\loigiai
% 	{
% 		\begin{itemize}
% 			\item Đặt $v_n=\dfrac{u_n}{n}$, ta có $v_{n+1}=\dfrac{v_n}{3}$.
% 			\item Do đó $v_n$ là cấp số nhân với công bội là $\dfrac{1}{3}$, mà $v_1=\dfrac{u_1}{1}=\dfrac{1}{3}$ nên
% 			$v_n=v_1\cdot\left(\dfrac{1}{3}\right)^{n-1}=\left(\dfrac{1}{3}\right)^n$.
% 			\item Từ đó $u_n=n\left(\dfrac{1}{3}\right)^n=\dfrac{n}{3^n}$.
% 			\item Bằng quy nạp, ta chứng minh được $3^n>n^2$, $\forall\, n\ge 1$. Khi đó $|u_n|=\dfrac{n}{3^n}<\dfrac{n}{n^2}=\dfrac{1}{n}$.\\
% 			Mà $\lim\dfrac{1}{n}=0\Rightarrow\lim \limits_{n \to +\infty}u_n=0$. Suy ra có vô số $n$ để $u_n<\dfrac{1}{2020}$.
% 		\end{itemize}
% 	}
% \end{ex}
% %%==========Câu 12
% \begin{ex}%[1D4K1-2]
% 	Tìm giới hạn của $(u_n)$ với $u_n = \dfrac{1}{\sqrt{n^2 + 1}} + \dfrac{1}{\sqrt{n^2 + }} +  \cdots + \dfrac{1}{\sqrt{n^2 + n}}$.
% 	\choice
% 	{\True $1$}
% 	{$0$}
% 	{$+\infty$}
% 	{$-\infty$}
% 	\loigiai{Với mỗi số nguyên $k$ mà $1 \leq k \leq n$, ta có $\dfrac{1}{\sqrt{n^2 + n}} \leq \dfrac{1}{\sqrt{n^2 + k}} \leq \dfrac{1}{\sqrt{n^2 + 1}}$.\\
% 		Do đó $\dfrac{n}{\sqrt{n^2 + n}} \leq u_n \leq \dfrac{n}{\sqrt{n^2 + 1}}$ với mọi $n$.\\
% 		Mặt khác $\lim \limits_{n \to +\infty}\dfrac{n}{\sqrt{n^2 + n}} = \lim \limits_{n \to +\infty}\dfrac{n}{\sqrt{n^2 + 1}} = 1$.\\
% 		Do đó $\lim \limits_{n \to +\infty}u_n = 1$.
% 	}
% \end{ex}
% %Câu 13
% \begin{ex}%[1D4K1-2]
% 	Tìm giới hạn của $(u_n)$ với $u_n = \dfrac{1}{\sqrt{n^2 + 1}} + \dfrac{1}{\sqrt{n^2 + }} +  \cdots + \dfrac{1}{\sqrt{n^2 + n}}$.
% 	\choice
% 	{\True $1$}
% 	{$0$}
% 	{$+\infty$}
% 	{$-\infty$}
% 	\loigiai{Với mỗi số nguyên $k$ mà $1 \leq k \leq n$, ta có $\dfrac{1}{\sqrt{n^2 + n}} \leq \dfrac{1}{\sqrt{n^2 + k}} \leq \dfrac{1}{\sqrt{n^2 + 1}}$.\\
% 		Do đó $\dfrac{n}{\sqrt{n^2 + n}} \leq u_n \leq \dfrac{n}{\sqrt{n^2 + 1}}$ với mọi $n$.\\
% 		Mặt khác $\lim \limits_{n \to +\infty}\dfrac{n}{\sqrt{n^2 + n}} = \lim \limits_{n \to +\infty}\dfrac{n}{\sqrt{n^2 + 1}} = 1$.\\
% 		Do đó $\lim \limits_{n \to +\infty}u_n = 1$.
% 	}
% \end{ex}
% %Câu 14
% \begin{ex}%[1D4K1-2]
% 	Kết quả đúng của $\lim \limits_{n \to +\infty}\left(5 - \dfrac{n\cos 2n}{n^2 + 1} \right)$ là
% 	\choice
% 	{$4$}
% 	{\True $5$}
% 	{$-4$}
% 	{$\dfrac{1}{4}$}
% 	\loigiai{
% 		Với mọi $n \in \mathbb{N}$ ta có $-\dfrac{n}{n^2 + 1} \le \dfrac{n\cos 2n}{n^2 + 1} \le \dfrac{n}{n^2 + 1}$.\\
% 		Ta có $\lim \limits_{n \to +\infty}\left(-\dfrac{n}{n^2 + 1}\right) = \lim \limits_{n \to +\infty}\dfrac{-\dfrac{1}{n}}{1 + \dfrac{1}{n^2}} = 0$; $\lim \limits_{n \to +\infty}\dfrac{n}{n^2 + 1} = \lim \limits_{n \to +\infty}\dfrac{\dfrac{1}{n}}{1 + \dfrac{1}{n^2}} = 0$.\\
% 		Suy ra $\lim \limits_{n \to +\infty}\left(\dfrac{n\cos 2n}{n^2 + 1}\right) = 0 \Rightarrow \lim \limits_{n \to +\infty}\left(5 - \dfrac{n\cos 2n}{n^2 + 1}\right) = 5$.
% 	}
% \end{ex}
% %Câu 15
% \begin{ex}%[1D4K1-2]
% 	Kết quả của $\lim \limits_{n \to +\infty}\left(n^2\sin \dfrac{n\pi }{5} - 2n^3\right)$ bằng
% 	\choice
% 	{\True $-\infty$}
% 	{$0$}
% 	{$+\infty$}
% 	{$-2$}
% 	\loigiai{
% 		Ta có $\lim \limits_{n \to +\infty}\left(n^2\sin \dfrac{n\pi}{5} - 2n^3\right) = \lim \limits_{n \to +\infty}n^3\left(\dfrac{1}{n}\sin \dfrac{n\pi}{5} - 2\right) = -\infty $.\\
% 		Vì $\sin \dfrac{n\pi}{5} \le 1 \Rightarrow \dfrac{1}{n} \sin \dfrac{n\pi }{5} \le \dfrac{1}{n}$.\\
% 		Mà $\lim \limits_{n \to +\infty}\dfrac{1}{n} = 0$ nên $\lim \limits_{n \to +\infty}\left(\dfrac{1}{n}\sin \dfrac{n\pi}{5} - 2\right) = -2$.\\
% 		Mặt khác $\lim \limits_{n \to +\infty}n^3 = +\infty$.\\
% 		Vậy $\lim \limits_{n \to +\infty}\left(n^2\sin \dfrac{n\pi }{5} - 2n^3\right) = -\infty$.
% 	}
% \end{ex}
% %Câu 16
% \begin{ex}%[1D4K1-2]
% 	Tính $I = \lim \limits_{n \to +\infty}\left( \dfrac{1}{\sqrt{n^2 + n + 1}} + \dfrac{1}{\sqrt {n^2 + n + 2}} + \cdots + \dfrac{1}{\sqrt{n^2 + 2n}}\right)$.
% 	\choice
% 	{$I = +\infty$}
% 	{$I = 3$}
% 	{$I = 2$}
% 	{\True $I = 1$}
% 	\loigiai{
% 		Ta có $\dfrac{1}{\sqrt{n^2 + 2n}} < \dfrac{1}{\sqrt{n^2 + n + k}} < \dfrac{1}{\sqrt{n^2 + n + 1}},\forall k = 2{,}3,\cdots ,n-1$.\\
% 		Suy ra $\dfrac{n}{\sqrt{n^2 + 2n}} < \dfrac{1}{\sqrt{n^2 + n + 1}} + \dfrac{1}{\sqrt{n^2 + n + 2}} + \cdots + \dfrac{1}{\sqrt{n^2 + 2n}} < \dfrac{n}{\sqrt{n^2 + n + 1}}$.\\
% 		Mà $\lim \limits_{n \to +\infty}\dfrac{n}{\sqrt{n^2 + 2n}} = \lim \limits_{n \to +\infty}\dfrac{1}{\sqrt {1 + \dfrac{2}{n}}} = 1$; $ \lim \limits_{n \to +\infty}\dfrac{n}{\sqrt{n^2 + n + 1}} = \lim \limits_{n \to +\infty}\dfrac{1}{\sqrt {1 + \dfrac{1}{n} + \dfrac{1}{n^2}}} = 1$.\\
% 		Vậy $I = 1$.
% 	}
% \end{ex}	
% \Closesolutionfile{ans}
% \begin{indapan}{10}
% 	{ans/ans-1K5-1-Dang6}
% \end{indapan}
%%Bài 16. Giới hạn hàm số
\section{Giới hạn của hàm số}
\setcounter{dang}{0}
\subsection{Tóm tắt lý thuyết}
\begin{tomtat}
	\subsubsection{Giới hạn hữu hạn của hàm số tại một điểm}
	
	\begin{dn}
		Cho điểm $ x_0 $ thuộc khoảng $ K $ và hàm số $ y=f(x) $ xác định trên $ K $ hoặc $ K\setminus\{x_0\} $.\\
		Ta nói hàm số $ y=f(x) $ \textbf{\textit{có giới hạn hữu hạn}} là số $ L $ khi $ x $ dần tới $ x_0 $ nếu với dãy số $ (x_n) $ bất kì, $ x_n\in K\setminus\{x_0\} $ và $ x_n \to x_0 $ thì $ f(x_n)\to L $, kí hiệu $ \lim \limits_{x \to x_0} f(x) =L$ hay $ f(x)\to L $ khi $ x\to x_0 $.
	\end{dn}
	
	\begin{note} 
		$ \lim \limits_{x \to x_0} x=x_0 $; \quad $ \lim \limits_{x \to x_0} c=c $ ($ c $ là hằng số).
	\end{note} 
	\subsubsection{Các phép toán về giới hạn hữu hạn của hàm số}
	\begin{enumerate}
		\item Cho $ \lim \limits_{x \to x_0} f(x) =L$ và $ \lim \limits_{x \to x_0} g(x)=M $. Khi đó:
		\begin{itemize}
			\item $ \lim \limits_{x \to x_0} [f(x)+g(x)]=L+M $;
			\item $ \lim \limits_{x \to x_0} [f(x)-g(x)]=L-M $;
			\item $ \lim \limits_{x \to x_0} [f(x)\cdot g(x)]=L\cdot M $;
			\item $ \lim \limits_{x \to x_0} \dfrac{f(x)}{g(x)}=\dfrac{L}{M} $ (với $ M\neq 0 $).			
		\end{itemize}
		\item Nếu $ f(x)\geq 0 $ và $ \lim \limits_{x \to x_0} f(x) =L$ thì $ L\geq 0 $ và $ \lim \limits_{x \to x_0} \sqrt{f(x)}=\sqrt{L} $.\\
		(Dấu của $ f(x) $ được xét trên khoảng tìm giới hạn, $ x\neq x_0 $).
	\end{enumerate}
	\begin{note} 
		\indent
		\begin{enumerate}
			\item $ \lim \limits_{x \to x_0} x^k=x_0^k $, $ k $ là số nguyên dương;
			\item $ \lim \limits_{x \to x_0} [cf(x)]=c\lim \limits_{x \to x_0} f(x) $ ($ c\in \mathbb{R} $, nếu tồn tại $ \lim \limits_{x \to x_0} f(x) \in \mathbb{R}$).
		\end{enumerate}
	\end{note} 
	
	\subsubsection{Giới hạn một phía}
	\begin{dn}
		\
		\begin{itemize}
			\item Cho hàm số $ y=f(x) $ xác định trên khoảng $ (x_0;b) $.\\
			Ta nói hàm số $ y=f(x) $ \textbf{\textit{có giới hạn bên phải}} là số $ L $ khi $ x $ dần tới $ x_0 $ nếu với dãy số $ (x_n) $ bất kì, $ x_0<x_n<b $ và $ x_n\to x_0 $ thì $ f(x_n)\to L $, kí hiệu $ \lim \limits_{x \to x_0^+} f(x) =L$.
			\item Cho hàm số $ y=f(x) $ xác định trên khoảng $ (a;x_0) $.\\
			Ta nói hàm số $ y=f(x) $ \textbf{\textit{có giới hạn bên trái}} là số $ L $ khi $ x $ dần tới $ x_0 $ nếu với dãy số $ (x_n) $ bất kì, $ a<x_n<x_0 $ và $ x_n\to x_0 $ thì $ f(x_n)\to L $, kí hiệu $ \lim \limits_{x \to x_0^-} f(x) =L$.
		\end{itemize}
	\end{dn}
	\begin{note} 
		\begin{enumerate}
			\item Ta thừa nhận các kết quả sau:
			\begin{itemize}
				\item $ \lim \limits_{x \to x_0^+} f(x)=L$ và $ \lim \limits_{x \to x_0^-} f(x)=L $ khi và chỉ khi $ \lim \limits_{x \to x_0} f(x) =L$;
				\item Nếu $ \lim \limits_{x \to x_0^+} f(x)\neq \lim \limits_{x \to x_0^-} f(x)$ thì không tồn tại $ \lim \limits_{x \to x_0} f(x) $.
			\end{itemize}
			\item Các phép toán về giới hạn hữu hạn của hàm số ở Mục 2 vẫn đúng khi ta thay $ x\to x_0 $ bằng $ x\to x_0^+ $ hoặc $ x\to x_0^- $.
		\end{enumerate}
	\end{note}
	
	\subsubsection{Giới hạn hữu hạn của hàm số tại vô cực}
	\begin{dn}
		\
		\begin{itemize}
			\item Cho hàm số $ y=f(x) $ xác định trên khoảng $ (a;+\infty) $.\\
			Ta nói hàm số $ y=f(x) $ \textbf{\textit{có giới hạn hữu hạn}} là số $ L $ khi $ x \to +\infty  $ nếu với dãy số $ (x_n) $ bất kì, $ x_n>a $ và $ x_n\to +\infty $ thì $ f(x_n)\to L $, kí hiệu $ \lim \limits_{x \to +\infty} f(x) =L$ hay $ f(x) \to L $ khi $ x\to +\infty $.
			\item Cho hàm số $ y=f(x) $ xác định trên khoảng $ (-\infty;a) $.\\
			Ta nói hàm số $ y=f(x) $ \textbf{\textit{có giới hạn hữu hạn}} là số $ L $ khi $ x \to -\infty $ nếu với dãy số $ (x_n) $ bất kì, $ x_n<a $ và $ x_n\to -\infty $ thì $ f(x_n)\to L $, kí hiệu $ \lim \limits_{x \to -\infty} f(x) =L$ hay $ f(x)\to L $ khi $ x\to -\infty $.
		\end{itemize}
	\end{dn}
	\begin{note}
		\begin{enumerate}
			\item Với $ c $ là hằng số và $ k $ là  số nguyên dương, ta luôn có:
			$$\lim \limits_{x \to \pm \infty} c=c\quad \text{và} \quad \lim \limits_{x \to \pm \infty} \dfrac{c}{x^k}=0.$$
			\item Các phép toán trên giới hạn hàm số ở Mục 2 vẫn đúng khi thay $ x\to x_0 $ bằng $ x\to +\infty $ hoặc $ x\to -\infty $.
		\end{enumerate}
	\end{note}
	
	\subsubsection{Giới hạn vô cực của hàm số tại một điểm}
	
	\begin{dn}
		Cho hàm số $ y=f(x) $ xác định trên khoảng $ (x_0;b) $.
		\begin{itemize}
			\item 
			Ta nói hàm số $ y=f(x) $ \textbf{\textit{có giới hạn bên phải}} là $ +\infty $ khi $ x $ dần tới $ x_0 $ về bên phải nếu với dãy số $ (x_n) $ bất kì, $ x_0<x_n<b $ và $ x_n\to x_0 $ thì $ f(x_n)\to +\infty $, kí hiệu $ \lim \limits_{x \to x_0^+} f(x) =+\infty$ hay $ f(x)\to +\infty $ khi $ x \to x_0^+ $.
			\item Ta nói hàm số $ y=f(x) $ \textbf{\textit{có giới hạn bên phải}} là  $ -\infty $ khi $ x $ dần tới $ x_0 $ về bên phải nếu với dãy số $ (x_n) $ bất kì, $ x_0<x_n<b $ và $ x_n\to x_0 $ thì $ f(x_n)\to -\infty $, kí hiệu $ \lim \limits_{x \to x_0^+} f(x) =-\infty$ hay $ f(x)\to -\infty $ khi $ x\to x_0^+ $.
		\end{itemize}
	\end{dn}
	
	\begin{note}
		\begin{enumerate}
			\item Các giới hạn $ \lim \limits_{x \to x_0^-} f(x)=+\infty $, $ \lim \limits_{x \to x_0^-} f(x)=-\infty $, $ \lim \limits_{x \to +\infty} f(x)=+\infty $, $ \lim \limits_{x \to +\infty} f(x)=-\infty $, $ \lim \limits_{x \to -\infty} f(x)=+\infty $, $ \lim \limits_{x \to -\infty} f(x)=-\infty $ được định nghĩa như trên.
			\item Ta có các giới hạn thường dùng như sau:
			\begin{itemize}
				\item $ \lim \limits_{x \to a^+} \dfrac{1}{x-a}=+\infty $ và $ \lim \limits_{x \to a^-}\dfrac{1}{x-a}=-\infty $ ($ a\in \mathbb{R} $);
				\item $ \lim \limits_{x \to +\infty}x^k=+\infty $ với $ k $ nguyên dương;
				\item $ \lim \limits_{x \to -\infty}x^k=+\infty $ với $ k $ là số chẵn;
				\item $ \lim \limits_{x \to -\infty}x^k=-\infty $ với $ k $ là số lẻ.
			\end{itemize}
			\item Các phép toán trên giới hạn hàm số của Mục 2 chỉ áp dụng được khi tất cả các hàm số được xét có giới hạn hữu hạn. Với giới hạn vô cực, ta có một số quy tắc sau đây.\\
			Nếu $ \lim \limits_{x \to x_0^+} f(x)=L\neq 0 $ và $ \lim \limits_{x \to x_0^+} g(x)=+\infty $ (hoặc $ \lim \limits_{x \to x_0^+} g(x)=-\infty $) thì $ \lim \limits_{x \to x_0^+}[f(x)\cdot g(x)] $ được tính theo quy tắc cho bởi bảng sau:
			\begin{center}
				\begin{tabular}{|c|c|c|}
					\hline 
					$ \lim \limits_{x \to x_0^+} f(x) $	& $  \lim \limits_{x \to x_0^+} g(x) $&$ \lim \limits_{x \to x_0^+} [f(x)\cdot g(x)] $  \\ 
					\hline 
					$ L>0 $	&$ +\infty $  &$ +\infty $  \\ 
					\hline 
					$ L>0 $	&$ -\infty $ &$ -\infty $  \\ 
					\hline 
					$ L<0 $&  $ +\infty $&  $ -\infty $\\ 
					\hline 
					$ L<0 $& $ -\infty $ &$ +\infty $  \\ 
					\hline 
				\end{tabular} 
			\end{center}
			Các quy tắc trên vẫn đúng khi thay $ x_0^+ $ thành $ x_0^- $ (hoặc $ +\infty $, $ -\infty $).
		\end{enumerate}
	\end{note}
\end{tomtat}

\subsection{Các dạng toán thường gặp}
\begin{dang}{Thay số trực tiếp}
\end{dang}
\subsubsection{Ví dụ minh hoạ}
\begin{vd}%[1K5YF-2]
	Tính các giới hạn sau
	\begin{listEX}[2]
		\item $ \lim \limits_{x \to 1} (x^2-4x+2)$;
		\item $ \lim \limits_{x \to 2} \dfrac{3x-2}{2x+1}$.
	\end{listEX}
	\loigiai{
		\begin{enumerate}[a)]
			\item $ \lim \limits_{x \to 1} (x^2-4x+2)=\lim \limits_{x \to 1} x^2-\lim \limits_{x \to 1} (4x)+\lim \limits_{x \to 1} 2=1^2-4\lim \limits_{x \to 1} x+2=1-4\cdot 1+2=-1$;
			\item $ \lim \limits_{x \to 2} \dfrac{3x-2}{2x+1}=\dfrac{\lim \limits_{x \to 2} (3x-2)}{\lim \limits_{x \to 2} (2x+1)}=\dfrac{3\lim \limits_{x \to 2} x-2}{2\lim \limits_{x \to 2} x+1}=\dfrac{3 \cdot 2-2}{2\cdot 2+1}=\dfrac{4}{5}$.
		\end{enumerate}
	}
\end{vd}

\begin{vd}%[1K5YF-2]
	Tìm các giới hạn sau
	\begin{enumEX}[a)]{2}
		\item $\lim\limits_{x\to 3} \sqrt{\dfrac{x^2}{x^3-x-6}}$.
		\item $\lim\limits_{x\to -2} \sqrt[3]{\dfrac{2x^4+3x+2}{x^2-x+2}}$.
	\end{enumEX}
	\loigiai{
		\begin{enumerate}[a)]
			\item $\lim\limits_{x\to 3} \sqrt{\dfrac{x^2}{x^3-x-6}}$; do $\lim\limits_{x\to 3} \dfrac{x^2}{x^3-x-6}=\dfrac{3^2}{3^3-3-6}=\dfrac{1}{2}>0$ \\
			$ \Rightarrow \lim\limits_{x\to 3} \sqrt{\dfrac{x^2}{x^3-x-6}}=\sqrt{\dfrac{1}{2}}=\dfrac{\sqrt{2}}{2}$.
			\item $\lim\limits_{x\to -2} \sqrt[3]{\dfrac{2x^4+3x+2}{x^2-x+2}}$; do $\lim\limits_{x\to -2} \dfrac{2x^4+3x+2}{x^2-x+2}=\dfrac{7}{2} \Rightarrow \lim\limits_{x\to -2} \sqrt[3]{\dfrac{2x^4+3x+2}{x^2-x+2}}=\sqrt[3]{\dfrac{7}{2}}=\dfrac{\sqrt[3]{28}}{2}$. 
		\end{enumerate}
	}
\end{vd}


\begin{vd}%[1K5YF-2]
	Cho $f(x)$ là một đa thức thỏa mãn $\displaystyle\lim\limits_{x\to 1}\dfrac{f(x)-16}{x-1}=24$. Tính giới hạn sau $$\displaystyle\lim\limits_{x\to 1}\dfrac{f(x)-16}{\left({x-1}\right)\left(\sqrt{2f(x)+4}+6\right)}.$$
	\loigiai{
		Vì $\displaystyle\lim\limits_{x\to 1}\dfrac{f(x)-16}{x-1}=24$ nên $f(1)=16.$ Khi đó
		$$ \lim\limits_{x\to 1}\dfrac{f(x)-16}{\left({x-1}\right)\left(\sqrt{2f(x)+4}+6\right)} =\frac{1}{12}\cdot \lim\limits_{x\to 1}\dfrac{f(x)-16}{x-1}=2.$$
	}
\end{vd}
% \subsubsection{Bài tập rèn luyện}
% % \subsubsection{Bài tập tự luận}
% \begin{bt}%[1K5YF-2]
% 	Tính các giới hạn sau:
% 	\begin{listEX}[2]
% 		\item $ \lim \limits_{x \to -2} (x^2+5x-2)$;
% 		\item $ \lim \limits_{x \to 1} \dfrac{x^2-1}{x-1}$.
% 	\end{listEX}
% 	\loigiai{
% 		\begin{enumerate}
% 			\item $ \lim \limits_{x \to -2} (x^2+5x-2)=(\lim \limits_{x \to -2} x)^2+\lim \limits_{x \to -2} (5x)-\lim \limits_{x \to -2} 2=(-2)^2+5\cdot (-2)-2=-8$.
% 			\item $ \lim \limits_{x \to 1} \dfrac{x^2-1}{x-1}=\lim \limits_{x \to 1} \dfrac{(x-1)(x+1)}{(x-1)}=\lim \limits_{x \to 1} (x+1)=\lim \limits_{x \to 1} x+1=1+1=2$.
% 		\end{enumerate}
% 	}
% \end{bt}

% \begin{bt}%[1K5YF-2]
% 	Tính các giới hạn sau
% 	\begin{enumEX}[a)]{2}
% 		\item $\lim\limits_{x\to -1} (3x^2-2x+1)$.
% 		\item $\lim\limits_{x\to 2} \dfrac{(x^3-3x)(x+1)}{x^2+3}$.
% 	\end{enumEX}
% 	\loigiai{
% 		\begin{enumerate}[a)]
% 			\item $\lim\limits_{x\to -1} (3x^2-2x+1)=3\lim\limits_{x\to -1} x^2-2\lim\limits_{x\to -1} x+\lim\limits_{x\to -1} 1=3{(1)}^2-2\cdot 1+1=2$.
% 			\item Do $\lim\limits_{x\to 2} (x^2+3)=2^2+3=7\ne 0$ và\\ $$\lim\limits_{x\to 2} (x^3-3x)(x+1)=\lim\limits_{x\to 2} (x^3-3x)\cdot \lim\limits_{x\to 2} (x+1)=(2^3-3\cdot 2)\cdot (2+1)=6$$
% 			Nên $\lim\limits_{x\to 2} \dfrac{(x^3-3x)(x+1)}{x^2+3}=\dfrac{6}{7}$.
% 		\end{enumerate}
% 	}
% \end{bt}

% \begin{bt}%[1K5YF-2]
% 	Tìm các giới hạn sau
% 	\begin{enumEX}[a)]{2}
% 		\item $\lim\limits_{x\to 2} \sqrt{\dfrac{2}{x^2-x+3}}$.
% 		\item $\lim\limits_{x\to -3} \sqrt[3]{\dfrac{-5}{x^2+x-12}}$.
% 	\end{enumEX}
% 	\loigiai{
% 		\begin{enumerate}[a)]
% 			\item $\lim\limits_{x\to 2} \sqrt{\dfrac{x}{x^2-x+3}}$; do $\lim\limits_{x\to 2} \dfrac{2}{x^2-x+3}=\dfrac{2}{2^2-2+3}=\dfrac{2}{5}>0$ \\
% 			$ \Rightarrow \lim\limits_{x\to 2} \sqrt{\dfrac{2}{x^2-x+3}}=\sqrt{\dfrac{2}{5}}=\dfrac{\sqrt{10}}{5}$.
% 			\item $\lim\limits_{x\to -3} \sqrt[3]{\dfrac{-5}{x^2+x-12}}$; do $\lim\limits_{x\to -3} \dfrac{-5}{x^2+x-12}=\dfrac{5}{6} \Rightarrow \lim\limits_{x\to -3} \sqrt[3]{\dfrac{-5}{x^2+x-12}}=\sqrt[3]{\dfrac{5}{6}}=\dfrac{\sqrt[3]{180}}{6}$. 
% 		\end{enumerate}
% 	}
% \end{bt} 

% \begin{bt}%[1K5YF-2]
% 	Cho $f(x)=x-1$ và $g(x)=x^{3}$. Tính các giới hạn sau:
% 	\begin{enumerate}
% 		\item $\lim \limits_{x \rightarrow 1}[3 f(x)-g(x)]$.
% 		\item $\lim \limits_{x \rightarrow 1} \dfrac{[f(x)]^{2}}{g(x)}$.
% 	\end{enumerate}
% 	\loigiai{Ta có $\lim \limits_{x \rightarrow 1} f(x)=\lim \limits_{x \rightarrow 1}(x-1)=\lim \limits_{x \rightarrow 1} x-\lim \limits_{x \rightarrow 1} 1=1-1=0$. Mặt khác, ta thấy $\lim \limits_{x \rightarrow 1} g(x)=\lim \limits_{x \rightarrow 1} x^{3}=1$.
% 		\begin{enumerate}
% 			\item Ta có
% 			$$
% 			\lim \limits_{x \rightarrow 1}[3 f(x)-g(x)]=\lim \limits_{x \rightarrow 1}[3 f(x)]-\lim \limits_{x \rightarrow 1} g(x)=\lim \limits_{x \rightarrow 1} 3 \cdot \lim \limits_{x \rightarrow 1} f(x)-\lim \limits_{x \rightarrow 1} g(x)=3 \cdot 0-1=-1 .
% 			$$
% 			\item  Ta có
% 			$$
% 			\lim \limits_{x \rightarrow 1} \dfrac{[f(x)]^{2}}{g(x)}=\dfrac{\lim \limits_{x \rightarrow 1}[f(x)]^{2}}{\lim \limits_{x \rightarrow 1} g(x)}=\dfrac{\lim \limits_{x \rightarrow 1} f(x) \cdot \lim \limits_{x \rightarrow 1} f(x)}{\lim \limits_{x \rightarrow 1} g(x)}=\dfrac{0}{1}=0.
% 			$$
% 	\end{enumerate}}
% 	\subsubsection{Bài tập trắc nghiệm}
% \end{bt}
\subsubsection{Câu hỏi trắc nghiệm}
\Opensolutionfile{ans}[ans/ans-1K5-2-Dang1]
\begin{ex}%[1T3Y2-1]
	Biết $\lim\limits_{x \to +\infty}f(x)=m$, $\lim\limits_{x \to +\infty}g(x)=n$. Tính $\lim\limits_{x \to +\infty}\left[f(x)+g(x)\right]$.
	\choice
	{\True $m+n$}
	{$m-n$}
	{$mn$}
	{$\dfrac{m}{n}$}
	\loigiai
	{
		Ta có $\lim\limits_{x \to +\infty}\left[f(x)+g(x)\right] = \lim\limits_{x \to +\infty}f(x) + \lim\limits_{x \to +\infty}g(x) = m+n$.
	}
\end{ex}


\begin{ex}%[1K5YF-2]
	Khẳng định nào sau đây là đúng?
	\choice
	{$\lim\limits_{x\to x_0} \sqrt[3]{f(x)+g(x)} =\sqrt[3]{\lim\limits_{x \to x_0} f(x)}+\sqrt[3]{\lim\limits_{x \to x_0} g(x)}$}
	{$\lim\limits_{x\to x_0} \sqrt[3]{f(x)+g(x)} =\lim\limits_{x \to x_0}\sqrt[3]{f(x)}+\lim\limits_{x \to x_0}\sqrt[3]{g(x)}$}
	{\True $\lim\limits_{x\to x_0} \sqrt[3]{f(x)+g(x)} =\sqrt[3]{\lim\limits_{x \to x_0} [f(x)+g(x)]}$}
	{$\lim\limits_{x\to x_0} \sqrt[3]{f(x)+g(x)} =\lim\limits_{x \to x_0}\left[\sqrt[3]{f(x)}+\sqrt[3]{g(x)}\right]$}
	\loigiai{
		Theo định lý về giới hạn của thì $\lim\limits_{x\to x_0} \sqrt[3]{f(x)+g(x)} =\sqrt[3]{\lim\limits_{x \to x_0} [f(x)+g(x)]}$.
	}
\end{ex}


\begin{ex}%[1K5YF-2]
	Cho các giới hạn $\lim \limits_{x\rightarrow x_0}f(x)=2$, $\lim \limits_{x\rightarrow x_0}g(x)=3$. Tính $M=\lim \limits_{x\rightarrow x_0}[3f(x)-4g(x)]$.
	\choice
	{$M=5$}
	{$M=2$}
	{\True $M=-6$}
	{$M=3$}
	\loigiai{
		Ta có $M=\lim \limits_{x\rightarrow x_0}[3f(x)-4g(x)]=3\lim \limits_{x\rightarrow x_0}f(x)-4\lim \limits_{x\rightarrow x_0}g(x)=6-12=-6.$
	}
\end{ex}


\begin{ex}%[1K5YF-2]
	Biết $\lim\limits_{x \to +\infty}f(x)=m$, $\lim\limits_{x \to +\infty}g(x)=n$. Tính $\lim\limits_{x \to +\infty}\left[f(x)-g(x)\right]$.
	\choice
	{ $m+n$}
	{\True $m-n$}
	{$mn$}
	{$\dfrac{m}{n}$}
	\loigiai
	{
		Ta có $\lim\limits_{x \to +\infty}\left[f(x)-g(x)\right] = \lim\limits_{x \to +\infty}f(x) - \lim\limits_{x \to +\infty}g(x) = m-n$.
	}
\end{ex}


\begin{ex}%[1K5YF-2]
	Cho $\lim\limits_{x \to a} f(x)=-\infty$, kết quả của $\lim\limits_{x \to a} [-3\cdot f(x)]$ bằng
	\choice
	{\True $+ \infty$}
	{$0$}
	{$3$}
	{$-\infty$}
	\loigiai{
		Có $\lim\limits_{x \to a} [-3\cdot f(x)]=-3\cdot\lim\limits_{x \to a}f(x)=+ \infty$.
		
	}
\end{ex}


\begin{ex}%[1K5YF-2]
	Cho $k\in \mathbb{Z}$, kết quả của $\lim\limits_{x \to -\infty} x^{2k+1}$ bằng
	\choice
	{$0$}
	{\True $-\infty$}
	{$+\infty$}
	{$5$}
	\loigiai{
		Theo tính chất của giới hạn hàm số, ta có $\lim\limits_{x \to -\infty} x^{2k+1}=-\infty$.
		
	}
\end{ex}


\begin{ex}%[1K5YF-2]
	Cho $\displaystyle\lim\limits_{x \to 3} f(x) =-2$. Giá trị $\displaystyle\lim\limits_{x \to 3} \left[f(x)+4x-1\right]$ bằng
	\choice
	{$5$}
	{$6$}
	{$-11$}
	{\True $9$}
	\loigiai{
		$\displaystyle\lim\limits_{x \to 3} \left[f(x)+4x-1\right] = \lim\limits_{x \to 3} f(x) + \lim\limits_{x \to 3} 4x -1 = -2 + 4 \cdot 3 -1 =9$.
	}
\end{ex}


\begin{ex}%[1K5YF-2]
	Cho $\lim\limits_{x\to 2}f(x)=3$. Giá trị của $\lim\limits_{x \to 2}\left[f(x)+x\right]$ bằng
	\choice
	{\True $5$}
	{$6$}
	{$1$}
	{$4$}
	\loigiai
	{
		Ta có $\lim\limits_{x \to 2}\left[f(x)+x\right] = \lim\limits_{x \to 2}f(x) + \lim\limits_{x \to 2}x = 3+2=5$.
	}
\end{ex}


\begin{ex}%[1K5YF-2]
	Với $k$ là số nguyên dương. Kết quả của giới hạn $\lim\limits_{x\to+\infty}x^{2k}$ là
	\choice
	{\True $+\infty$}
	{$0$}
	{$-\infty$}
	{$1$}
	\loigiai{
		Với $k$ nguyên dương thì $\lim\limits_{x\to +\infty}x^k = +\infty \Rightarrow \lim\limits_{x\to+\infty}x^{2k}=+\infty$.
	}
\end{ex}


\begin{ex}%[1K5YF-2]
	Cho $c$ là hằng số, $k$ là số nguyên dương. Khẳng định nào sau đây \textbf{sai}?	
	\choice
	{\True $\lim\limits_{ x \rightarrow +\infty}c=+\infty$}
	{$\lim\limits_{ x \rightarrow +\infty}\dfrac{c}{x^k}=0$}
	{$\lim\limits_{ x \rightarrow x_0}c=c$}
	{$\lim\limits_{ x \rightarrow x_0}x=x_0$}
	\loigiai{Theo định lý về giới hạn, khẳng định sai là $\lim\limits_{ x \rightarrow +\infty}c=+\infty $.
		
	}
\end{ex}


\begin{ex}%[1K5YF-2]
	Cho $\lim\limits_{x \to +\infty} f(x)=a$,$\lim\limits_{x \to +\infty} g(x)=b$. Hỏi mệnh đề nào sau đây là mệnh đề $\textbf{sai}$?
	\choice
	{$\lim\limits_{x \to +\infty} \left[f(x)\cdot g(x)\right]=ab$}
	{$\lim\limits_{x \to +\infty} \left[f(x)- g(x)\right]=a-b$}
	{$\lim\limits_{x \to +\infty} \left[f(x)+ g(x)\right]=a+b$}
	{\True $\lim\limits_{x \to +\infty} \dfrac{f(x)}{g(x)}=\dfrac{a}{b}$}
	\loigiai{
		Khi $\lim\limits_{x \to +\infty} g(x)=b=0$ thì $\lim\limits_{x \to +\infty} \dfrac{f(x)}{g(x)}=\dfrac{a}{b}$ không đúng.
	}
\end{ex}


\begin{ex}%[1K5YF-2]
	Với $k$ là số nguyên dương thì $\lim\limits_{x \to -\infty} \dfrac{1}{x^k}$ bằng
	\choice
	{$+\infty$}
	{$-\infty$}
	{$x$}
	{\True $0$}
	\loigiai{
		Vì $\left\{\begin{aligned}
			&\lim\limits_{x \to -\infty} 1=1\\
			&\lim\limits_{x \to -\infty} x^k=\pm \infty\\
		\end{aligned}\right. $ nên $\lim\limits_{x \to -\infty} \dfrac{1}{x^k}=0$.
	}
\end{ex}


\begin{ex}%[1K5YF-2]
	Tính $I=\lim\limits_{x \to 2}\left(x^2+x-6\right)$.
	\choice
	{\True $0$}
	{$1$}
	{$2$}
	{$3$}
	\loigiai{
		\begin{eqnarray*}
			\lim\limits_{x \to 2}\left(x^2+x-6\right)&=&\lim\limits_{x \to 2} x^2+\lim\limits_{x \to 2} x-\lim\limits_{x \to 2} 6\\
			&=&4+2-6\\
			&=&0.
		\end{eqnarray*}	
	}
\end{ex}


\begin{ex}%[1K5YF-2]
	Tính $I=\lim\limits_{x \to 1} \dfrac{x^2+2 x+3}{2 x-1}$.	
	\choice
	{ $4$}
	{$5$}
	{\True $6$}
	{$7$}
	\loigiai{
		\begin{eqnarray*}
			\lim\limits_{x \to 1} \dfrac{x^2+2 x+3}{2 x-1}&=&\dfrac{\lim\limits_{x \to 1}\left(x^2+2 x+3\right)}{\lim\limits_{x \to 1}(2 x-1)}\\
			&=&\dfrac{\lim\limits_{x \to 1} x^2+\lim\limits_{x \to 1}(2 x)+\lim\limits_{x \to 1} 3}{\lim\limits_{x \to 1}(2 x)-\lim\limits_{x \to 1} 1}\\
			&=&\dfrac{1+2+3}{2-1}\\
			&=&6.
		\end{eqnarray*}		
	}
\end{ex}
\begin{ex}%[1K5YF-2]
	Tính  $I=\lim\limits_{x \to 0} \dfrac{\left|x\right|}{x}$. 
	\choice
	{$1$}
	{$-1$}
	{\True Không tồn tại }
	{$0$}
	\loigiai{
		Ta có :\\
		$\lim\limits_{x \to 0^{+}} \dfrac{\left|x\right|}{x}=\lim\limits_{x \to 0^{+}} \dfrac{x}{x}=\lim\limits_{x \to 0^{+}} 1=1$.\\
		$\lim\limits_{x \to 0^{-}} \dfrac{\left|x\right|}{x}=\lim\limits_{x \to 0^{-}} \dfrac{(-x)}{x}=\lim\limits_{x \to 0^{-}} (-1)=-1$.\\
		Vậy không tồn tại $\lim\limits_{x \to 0} \dfrac{\left|x\right|}{x}$. 
		
	}
\end{ex}
\Closesolutionfile{ans}
% \begin{indapan}{10}
% 	{ans/ans-1K5-2-Dang1}
% \end{indapan}
\begin{dang}{Phương pháp đặt thừa số chung - kết quả hữu hạn}
	\begin{itemize}
		\item Nếu tam thức bậc hai $ax^2+bx+c$ có hai nghiệm $x_1$, $x_2$ thì $ ax^2+bx+c=a(x-x_1)(x-x_2)$.
		\item $a^n-b^n=(a-b)\left(a^{n-1}+a^{n-2}b+\cdots +ab^{n-2}+b^{n-1}\right)$.
		\item $\lim\limits_{x\to \pm \infty } c=c;\lim\limits_{x\to \pm \infty } \dfrac {c}{x^k}=0$ với $c$ là hằng số và $ k\in \mathbb{N}$.
		\item $a\sqrt{b}=\heva{&\sqrt{a^2b}\quad a\ge 0\\&-\sqrt{a^2b}\quad a<0.}$
	\end{itemize}
\end{dang}
\subsubsection{Ví dụ minh hoạ}
%VD1
\begin{vd}[NB]%[DCHT Toán 11 - KNTT -Nguyễn Văn Hiệp]%[1K5YF-3]
	Tính giới hạn $\lim\limits_{x\to 3} \dfrac {x^2-9}{x-3}$.\dapso{$I=6$.}
	\loigiai{Ta có $\lim\limits_{x\to 3} \dfrac {x^2-9}{x-3}=\lim\limits_{x\to 3} \dfrac {(x-3)(x+3)}{x-3}=\lim\limits_{x\to 3} (x+3)=6$.}
\end{vd}
%VD2
\begin{vd}[TH]%[DCHT Toán 11 - KNTT -Nguyễn Văn Hiệp]%[1K5BF-3]
	Tính giới hạn $I=\lim\limits_{x\to 2} \dfrac {x^2-5x+6}{x-2}$.
	\dapso{$I=-1$.}
	\loigiai{
		$I=\lim\limits_{x\to 2} \dfrac {x^2-5x+6}{x-2}=\lim\limits_{x\to 2} \dfrac {(x-2)(x-3)}{x-2}=\lim\limits_{x\to 2} (x-3)=-1$.
	}
\end{vd}
%VD3
\begin{vd}[TH]%[DCHT Toán 11 - KNTT -Nguyễn Văn Hiệp]%[1K5BF-3]
	Tính giới hạn $\lim\limits_{x\to +\infty } \dfrac {x^4+7}{x^4+1}$.
	\dapso{$1$.}
	\loigiai{Ta có
		$\lim\limits_{x\to +\infty} \dfrac {x^4+7}{x^4+1}=\lim\limits_{x\to +\infty } \dfrac {x^4\left(1+\dfrac {7}{x^4}\right)}{x^4\left(1+\dfrac {1}{x^4}\right)}=\lim\limits_{x\to +\infty } \dfrac {1+\dfrac {7}{x^4}}{1+\dfrac {1}{x^4}}=1$.
	}
\end{vd}
%VD4
\begin{vd}[TH]%[DCHT Toán 11 - KNTT -Nguyễn Văn Hiệp]%[1K5BF-3]
	Tìm giới hạn $\lim\limits_{x\to +\infty } \sqrt {\dfrac {x^2+1}{2x^4+x^2-3}}$.
	\dapso{$0$.}
	\loigiai{
		Ta có $\lim\limits_{x\to +\infty } \sqrt {\dfrac {x^2+1}{2x^4+x^2-3}}=\lim\limits_{x\to +\infty} \dfrac{x}{x^2}\cdot \sqrt {\dfrac {\dfrac {1}{x^2}+\dfrac {1}{x^4}}{2+\dfrac {1}{x^2}-\dfrac {3}{x^4}}}=\lim\limits_{x\to +\infty} \dfrac{1}{x}\cdot \sqrt {\dfrac {\dfrac {1}{x^2}+\dfrac {1}{x^4}}{2+\dfrac {1}{x^2}-\dfrac {3}{x^4}}}=0$.}
\end{vd}
%VD5
\begin{vd}[TH]%[DCHT Toán 11 - KNTT -Nguyễn Văn Hiệp]%[1K5BF-3]
	Tính giới hạn $\lim\limits_{x\to 1} \left(\dfrac {1}{1-x}-\dfrac {3}{1-x^3} \right)$. 
	\dapso{$-1$.}
	\loigiai{$\lim\limits_{x\to 1} \left(\dfrac {1}{1-x}-\dfrac {3}{1-x^3} \right)=\lim\limits_{x\to 1} \dfrac {1+x+x^2-3}{1-x^3}=\lim\limits_{x\to 1} \dfrac {(x-1 )(x+2)}{(1-x)\left( 1+x+x^2 \right)}=\lim\limits_{x\to 1} \dfrac {-(x+2)}{1+x+x^2}=-1$. 
	}
\end{vd}
%VD6
\begin{vd}[VDT]%[DCHT Toán 11 - KNTT -Nguyễn Văn Hiệp]%[1K5KF-3]
	Cho $m,n$ là các số thực khác $0$. Nếu giới hạn $\lim\limits_{x\to -5} \dfrac {x^2+mx+n}{x+5}=3$, hãy tìm $mn$.
	\dapso{$mn=520$.}
	\loigiai{
		Vì $\lim\limits_{x\to -5} \dfrac {x^2+mx+n}{x-5}=3$ nên $ x=-5$ là nghiệm của phương trình $ x^2+mx+n=0$.\\
		$\Rightarrow -5m+n+25=0\Leftrightarrow n=5m-25$.\\
		Khi đó 
		\allowdisplaybreaks
		$\begin{aligned}[t]
			\lim\limits_{x\to -5} \dfrac {x^2+mx+n}{x-1}&=\lim\limits_{x\to -5} \dfrac {x^2+mx+5m-25}{x+5}\\
			&=\lim\limits_{x\to -5} \dfrac {(x+5)(x-5+m)}{x+5}\\
			&=\lim\limits_{x\to -5} (x-5+m)=m-10.
		\end{aligned}$\\
		Ta có $m-10=3\Leftrightarrow m=13\Rightarrow n=40$.\\
		Vậy $mn=13\cdot 40=520$.
	}
\end{vd}
%VD7
\begin{vd}[VDT]%[DCHT Toán 11 - KNTT -Nguyễn Văn Hiệp]%[1K5KF-3]
	Tìm số thực $a$ thỏa mãn $\lim\limits_{x\to +\infty} \dfrac {a\sqrt {2x^2+3}+2024}{2x+2023}=\dfrac {1}{2}$.\dapso{$a=\dfrac {\sqrt {2}}{2}$.}
	\loigiai{Ta có $\lim\limits_{x\to +\infty } \dfrac {a\sqrt {2x^2+3}+2024}{2x+2023}=\dfrac {1}{2}\Leftrightarrow \lim\limits_{x\to +\infty } \dfrac {a\sqrt {2+\dfrac {3}{x^2}}+\dfrac {2024}{x}}{2+\dfrac {2023}{x}}=\dfrac {1}{2}\Leftrightarrow \dfrac {a\sqrt {2}}{2}=\dfrac {1}{2}\Leftrightarrow a=\dfrac {\sqrt {2}}{2}$.}
\end{vd}
% \subsubsection{Bài tập rèn luyện}
% % \subsubsection{Bài tập tự luận}
% %BT1
% \begin{bt}[NB]%[DCHT Toán 11 - KNTT -Nguyễn Văn Hiệp]%[1K5YF-3]
% 	Tính $\lim\limits_{x\to 2} \dfrac {x^2-4}{x-2}$.
% 	\dapso{$4$}
% 	\loigiai{$\lim\limits_{x\to 2} \dfrac {x^2-4}{x-2}=\lim\limits_{x\to 2} \dfrac {(x-2)(x+2)}{x-2}=\lim\limits_{x\to 2} (x+2)=2+2=4$.}
% \end{bt}
% %BT2
% \begin{bt}[NB]%[DCHT Toán 11 - KNTT -Nguyễn Văn Hiệp]%[1K5YF-3]
% 	Tính $\lim\limits_{x\to 5} \dfrac {x^2-12x+35}{25-5x}$.
% 	\dapso{$\dfrac{2}{5}$.}
% 	\loigiai{Ta có $\lim\limits_{x\to 5} \dfrac {x^2-12x+35}{25-5x}=\lim\limits_{x\to 5} \dfrac {(x-7)(x-5)}{5(5-x)}=\lim\limits_{x\to 5} \dfrac {7-x}{5}=\dfrac {2}{5}$.\\
% 		Vậy $\lim\limits_{x\to 5} \dfrac {x^2-12x+35}{25-5x}=\dfrac {2}{5}$.
% 	}
% \end{bt}
% %BT3
% \begin{bt}[TH]%[DCHT Toán 11 - KNTT -Nguyễn Văn Hiệp]%[1K5BF-3]
% 	Tính giới hạn $I=\lim\limits_{x\to 2} \dfrac {x^3-8}{x^2-4}$.
% 	\dapso{$I=3$.}
% 	\loigiai{
% 		Ta có $I=\lim\limits_{x\to 2} \dfrac {x^3-8}{x^2-4}=\lim\limits_{x\to 2} \dfrac {(x-2)\left(x^2+2x+4\right)}{(x-2)(x+2)}=\lim\limits_{x\to 2} \dfrac {x^2+2x+4}{x+2}=\dfrac {12}{4}=3$.
% 	}
% \end{bt}
% %BT4
% \begin{bt}[TH]%[DCHT Toán 11 - KNTT -Nguyễn Văn Hiệp]%[1K5BF-3]
% 	Tìm giới hạn $A=\lim\limits_{x\to 2} \dfrac {x^4-5x^2+4}{x^3-8}$.
% 	\dapso{$A=1$.}
% 	\loigiai{Ta có 
% 		\allowdisplaybreaks
% 		\begin{eqnarray*}
% 			A&=&\lim\limits_{x\to 2} \dfrac {x^4-5x^2+4}{x^3-8}=\lim\limits_{x\to 2} \dfrac {\left(x^2-1\right) \left(x^2-4\right)}{x^3-2^3}\\
% 			&=&\lim\limits_{x\to 2} \dfrac {\left(x^2-1\right)\left(x-2 \right)\left(x+2\right)}{\left(x-2\right)\left(x^2+2x+4 \right)}\\
% 			&=&\lim\limits_{x\to 2} \dfrac {\left(x^2-1\right)(x+2)}{x^2+2x+4}\\
% 			&=&1.
% 	\end{eqnarray*}}
% \end{bt}
% %BT5
% \begin{bt}[TH]%[DCHT Toán 11 - KNTT -Nguyễn Văn Hiệp]%[1K5BF-3]
% 	Tìm giới hạn $\lim\limits_{x\to -\infty }\dfrac {1+3x}{\sqrt {2x^2+3}}$.
% 	\dapso{$-\dfrac {3\sqrt {2}}{2}$.}
% 	\loigiai{
% 		Ta có $\lim\limits_{x\to -\infty } \dfrac {1+3x}{\sqrt {2x^2+3}}=\lim\limits_{x\to -\infty } \dfrac {x\cdot \left(\dfrac {1}{x}+3\right)}{-x\cdot \left(\sqrt {2+\dfrac {3}{x}}\right)}=\lim\limits_{x\to -\infty } \dfrac {\dfrac {1}{x}+3}{-\sqrt {2+\dfrac {3}{x}}}=-\dfrac {3\sqrt {2}}{2}$.
% 	}
% \end{bt}
% %BT6
% \begin{bt}[TH]%[DCHT Toán 11 - KNTT -Nguyễn Văn Hiệp]%[1K5BF-3]
% 	Tìm giới hạn $\lim\limits_{x\to +\infty } \dfrac {2x-\sqrt {3x^2+2}}{5x+\sqrt {x^2+2}}$.
% 	\dapso{$\dfrac {2-\sqrt {3}}{6}$.}
% 	\loigiai{
% 		Ta có $\lim\limits_{x\to +\infty } \dfrac {2x-\sqrt {3x^2+2}}{5x+\sqrt {x^2+2}}=\lim\limits_{x\to +\infty }\dfrac{x}{x}\cdot \dfrac {2-\sqrt {3+\dfrac {2}{x^2}}}{5+\sqrt {1+\dfrac {2}{x^2}}}=\lim\limits_{x\to +\infty } \dfrac {2-\sqrt {3+\dfrac {2}{x^2}}}{5+\sqrt {1+\dfrac {2}{x^2}}}=\dfrac {2-\sqrt {3}}{6}$.
% 	}
% \end{bt}
% %BT7
% \begin{bt}[VDT]%[DCHT Toán 11 - KNTT -Nguyễn Văn Hiệp]%[1K5KF-3]
% 	Giá trị của $\lim\limits_{x\to 1} \dfrac {x^{2024}+x-2}{x^{2023}+x-2}$ bằng $\dfrac {a}{b}$, với $\dfrac {a}{b}$ là phân số tối giản. Tính giá trị của $a^2-b^2$.
% 	\dapso{$4049$.}
% 	\loigiai{Ta có
% 		\allowdisplaybreaks
% 		\begin{eqnarray*}
% 			&&\lim\limits_{x\to 1} \dfrac {x^{2024}+x-2}{x^{2023}+x-2}=\lim\limits_{x\to 1} \dfrac {x^{2024}-1+x-1}{x^{2023}-1+x-1}\\
% 			&=&\lim\limits_{x\to 1} \dfrac {(x-1)(x^{2023}+x^{2022}\cdots +x+1)+x-1}{(x-1)(x^{2022}+x^{2021}+\cdots+x+1)+x-1}\\
% 			&=&\lim\limits_{x\to 1} \dfrac {x^{2023}+x^{2022}\cdots +x+2}{x^{2022}+x^{2021}+\cdots+x+2}\\
% 			&=&\dfrac {1+1+\cdots +1+2}{1+1+\cdots +1+2}=\dfrac {2025}{2024}.
% 		\end{eqnarray*}
% 		Vậy $a^2-b^2=2025^2-2024^2=4049$.}
% \end{bt}
% %BT8
% \begin{bt}[VDT]%[DCHT Toán 11 - KNTT -Nguyễn Văn Hiệp]%[1K5KF-3]
% 	Cho giới hạn $\lim\limits_{x\to 3} \dfrac {x^2+ax+b}{x-3}=3$. Tìm $a$, $b$.
% 	\dapso{$a=-3$, $b=0$.}
% 	\loigiai{
% 		Để $\lim\limits_{x\to 3} \dfrac {x^2+ax+b}{x-3}=3$ thì ta phải có $x^2+ax+b=(x-3)(x-m)$.\\
% 		Khi đó $3-m=3\Leftrightarrow m=0$. Vậy $x^2+ax+b=(x-3 )x=x^2-3x$.\\
% 		Suy ra $a=-3$ và $b=0$.
% 	}
% \end{bt}
% %BT9
% \begin{bt}[VDT]%[DCHT Toán 11 - KNTT -Nguyễn Văn Hiệp]%[1K5KF-3]
% 	Tìm $m$ để $\lim\limits_{x\to -\infty } \dfrac {\sqrt {4x^2+x+1}+4}{mx-2}=\dfrac {1}{2}$.\dapso{$m=-4$.}
% 	\loigiai{Ta có $\lim\limits_{x\to -\infty } \dfrac {\sqrt {4x^2+x+1}+4}{mx-2}=\lim\limits_{x\to -\infty }\dfrac{-x}{x}\cdot  \dfrac {\sqrt {4+\dfrac {1}{x}+\dfrac {1}{x^2}}-\dfrac {4}{x}}{m-\dfrac {2}{x}}=\lim\limits_{x\to -\infty } \dfrac {-\sqrt {4+\dfrac {1}{x}+\dfrac {1}{x^2}}+\dfrac {4}{x}}{m-\dfrac {2}{x}}=-\dfrac {2}{m}$.\\
% 		Theo bài ra ta có $-\dfrac {2}{m}=\dfrac {1}{2}\Leftrightarrow m=-4$.
% 	}
% \end{bt}
% %BT10
% \begin{bt}[VDC]%[DCHT Toán 11 - KNTT -Nguyễn Văn Hiệp]%[1K5GF-3]
% 	Tính giới hạn $\lim\limits_{x\to 1} \left( \dfrac {m}{1-x^m}-\dfrac {n}{1-x^n} \right)$, $m,n\in \mathbb{N^*}$.
% 	\dapso{ $\dfrac{m-n}{2}$.}
% 	\loigiai{
% 		\allowdisplaybreaks
% 		\begin{eqnarray*}
% 			\lim\limits_{x\to 1} \left(\dfrac {m}{1-x^m}-\dfrac {n}{1-x^n} \right)&=&\lim\limits_{x\to 1} \left[ \left( \dfrac {m}{1-x^m}-\dfrac {1}{1-x} \right)-\left( \dfrac {n}{1-x^n}-\dfrac {1}{1-x} \right)\right]\\
% 			&=&\lim\limits_{x\to 1} \left( \dfrac {m}{1-x^m}-\dfrac {1}{1-x} \right)-\lim\limits_{x\to 1} \left( \dfrac {n}{1-x^n}-\dfrac {1}{1-x} \right)=A-B.
% 		\end{eqnarray*}
% 		\allowdisplaybreaks
% 		\begin{eqnarray*}
% 			A&=&\lim\limits_{x\to 1} \left( \dfrac {m}{1-x^m}-\dfrac {1}{1-x} \right)\\
% 			&=&\lim\limits_{x\to 1} \dfrac {m-\left(1+x+x^2+\cdots +x^{m-1}\right)}{1-x^m}\\
% 			&=&\lim\limits_{x\to 1} \dfrac {(1-x)+\left(1-x^2\right)+\cdots +\left(1-x^{m-1}\right)}{1-x^m}\\
% 			&=&\lim\limits_{x\to 1} \dfrac {(1-x)\left[1+(1+x)+\cdots +\left( 1+x+\cdots +x^{m-2} \right) \right]}{(1-x)\left(1+x+\cdots +x^{m-1}\right)}\\
% 			&=&\lim\limits_{x\to 1} \dfrac {1+(1+x)+\cdots +\left(1+x+\cdots +x^{m-2} \right)}{1+x+\cdots +x^{m-1}}\\
% 			&=&\lim\limits_{x\to 1} \dfrac {1+2+\cdots +m-1}{m}\\
% 			&=&\dfrac {m-1}{2}.
% 		\end{eqnarray*}
% 		Tương tự, ta tính được $B=\dfrac {n-1}{2}$.\\
% 		Vậy $\lim\limits_{x\to 1} \left( \dfrac {m}{1-x^m}-\dfrac {n}{1-x^n} \right)=A-B=\dfrac {m-n}{2}$.
% 	}
% \end{bt}
\subsubsection{Câu hỏi trắc nghiệm}
\Opensolutionfile{ans}[ans/ans-1K5-2-Dang2]

\begin{ex}%[DCHT Toán 11 - KNTT -Nguyễn Văn Hiệp]%[1K5BF-3]
	Tìm $\lim\limits_{x\to 3} \dfrac {9-x^2}{x^2-4x+3}$. Kết quả là
	\choice
	{\True $-3$}
	{$4$}
	{$-4$}
	{$3$}
	\loigiai
	{Ta có $\lim\limits_{x\to 3} \dfrac {9-x^2}{x^2-4x+3}=\lim\limits_{x\to 3} \dfrac {(3-x)(3+x)}{(x-3)(x-1)}=\lim\limits_{x\to 3} \dfrac {-(x+3)}{x-1}=-3$.
	}
\end{ex}
%Cau2
\begin{ex}%[DCHT Toán 11 - KNTT -Nguyễn Văn Hiệp]%[1K5YF-3]
	Tìm $\lim\limits_{x\to 4} \dfrac {x^2-16}{x-4}$. Kết quả là
	\choice
	{$7$}
	{\True $8$}
	{$5$}
	{$6$}
	\loigiai
	{Ta có $\lim\limits_{x\to 4} \dfrac {x^2-16}{x-4}=\lim\limits_{x\to 4} \dfrac {(x-4)(x+4)}{x-4}=\lim\limits_{x\to 4} (x+4)=8$.
	}
\end{ex}
%Cau3
\begin{ex}%[DCHT Toán 11 - KNTT -Nguyễn Văn Hiệp]%[1K5YF-3]
	Tính giới hạn $A=\lim\limits_{x\to 1} \dfrac {x^3-1}{x-1}$.
	\choice
	{$A=-\infty $}
	{$A=0$}
	{\True $A=3$}
	{$A=+\infty $}
	\loigiai{Ta có $A=\lim\limits_{x\to 1} \dfrac {x^3-1}{x-1}=\lim\limits_{x\to 1} \dfrac {( x-1 )\left(x^2+x+1 \right)}{x-1}=\lim\limits_{x\to 1} \left(x^2+x+1 \right)=3$.}
\end{ex}
%Cau4
\begin{ex}%[DCHT Toán 11 - KNTT -Nguyễn Văn Hiệp]%[1K5BF-3]
	Chọn kết quả đúng trong các kết quả sau của $\lim\limits_{x\to -1} \dfrac {x^2+2x+1}{2x+2}$ là
	\choice
	{$-\infty $}
	{\True $0$}
	{$\dfrac {1}{2}$}
	{$+\infty $}
	\loigiai
	{$\lim\limits_{x\to -1} \dfrac {x^2+2x+1}{2x+2}=\lim\limits_{x\to -1} \dfrac {(x+1)^2}{2(x+1)}=\lim\limits_{x\to -1} \dfrac {x+1}{2}=0$.
	}
\end{ex}
%Cau5
\begin{ex}%[DCHT Toán 11 - KNTT -Nguyễn Văn Hiệp]%[1K5BF-3]
	Giới hạn $\lim\limits_{x\to 4} \dfrac {x^2+2x-15}{x-3}$ bằng
	\choice
	{$\dfrac {1}{8}$}
	{\True $9$}
	{$+\infty $}
	{$8$}
	\loigiai{$\lim\limits_{x\to 4} \dfrac {x^2+2x-15}{x-3}=\lim\limits_{x\to 4} \dfrac {( x-3 )( x+5 )}{x-3}=9 $.}
\end{ex}
%Cau6
\begin{ex}%[DCHT Toán 11 - KNTT -Nguyễn Văn Hiệp]%[1K5BF-3]
	Tính giới hạn $I=\lim\limits_{x\to -\infty } \dfrac {3x-2}{2x+1}$.
	\choice
	{$I=-2$}
	{$I=-\dfrac {3}{2}$}
	{$I=2$}
	{\True $I=\dfrac {3}{2}$}
	\loigiai{Ta có $I=\lim\limits_{x\to -\infty } \dfrac {3x-2}{2x+1}=\lim\limits_{x\to -\infty } \dfrac {x\left(3-\dfrac {2}{x}\right)}{x\left(2+\dfrac {1}{x}\right)}=\lim\limits_{x\to -\infty } \dfrac {3-\dfrac {2}{x}}{2+\dfrac {1}{x}}=\dfrac {3}{2}$.}
\end{ex}
%Cau7
\begin{ex}%[DCHT Toán 11 - KNTT -Nguyễn Văn Hiệp]%[1K5BF-3]
	$\lim\limits_{x\to +\infty } \dfrac {x-2}{x+3}$ bằng
	\choice
	{$-\dfrac {2}{3}$}
	{\True $1$}
	{$2$}
	{$-3$}
	\loigiai{Ta có $\lim\limits_{x\to +\infty} \dfrac {x-2}{x+3}=\lim\limits_{x\to +\infty } \dfrac {x\left(1-\dfrac {2}{x}\right)}{x\left(1+\dfrac {3}{x}\right)}=\lim\limits_{x\to +\infty } \dfrac {1-\dfrac {2}{x}}{1+\dfrac {3}{x}}=\dfrac {1}{1}=1$.}
\end{ex}
%Cau8
\begin{ex}%[DCHT Toán 11 - KNTT -Nguyễn Văn Hiệp]%[1K5BF-3]
	Giới hạn $\lim\limits_{x\to 1} \dfrac {x^2-3x+2}{x^3-x^2+x-1}$ bằng
	\choice
	{$-2$}
	{$-1$}
	{\True $-\dfrac {1}{2}$}
	{$\dfrac {1}{2}$}
	\loigiai{Ta có $\lim\limits_{x\to 1} \dfrac {x^2-3x+2}{x^3-x^2+x-1}$$=\lim\limits_{x\to 1} \dfrac {( x-1 )( x-2 )}{( x-1 )( x^2+1 )}$$=\lim\limits_{x\to 1} \dfrac {x-2}{x^2+1}$$=-\dfrac {1}{2}$.}
\end{ex}
%Cau9
\begin{ex}%[DCHT Toán 11 - KNTT -Nguyễn Văn Hiệp]%[1K5BF-3]
	Giới hạn $T=\lim\limits_{x\to 1} \dfrac {x^4-3x+2}{x^3+2x-3}$ bằng
	\choice
	{$\dfrac {2}{9}$}
	{$\dfrac {2}{5}$}
	{\True $\dfrac {1}{5}$}
	{$+\infty $}
	\loigiai
	{$T=\lim\limits_{x\to 1} \dfrac {x^4-3x+2}{x^3+2x-3}=\lim\limits_{x\to 1} \dfrac {(x-1)\left(x^3+x^2+x-2\right)}{(x-1)\left(x^2+x+3 \right)}=\lim\limits_{x\to 1} \dfrac {x^3+x^2+x-2}{x^2+x+3}=\dfrac {1^3+1^2+1-2}{1^2+1+3}=\dfrac {1}{5}$.
	}
\end{ex}
%Cau10
\begin{ex}%[DCHT Toán 11 - KNTT -Nguyễn Văn Hiệp]%[1K5BF-3]
	Giới hạn $\lim\limits_{x\to 2} \dfrac {x^2-5x+6}{x^3-x^2-x-2}$ bằng
	\choice
	{$0$}
	{\True $-\dfrac {1}{7}$}
	{$-7$}
	{$+\infty $}
	\loigiai{Ta có $\lim\limits_{x\to 2} \dfrac {x^2-5x+6}{x^3-x^2-x-2}=\lim\limits_{x\to 2} \dfrac {(x-2)(x-3 )}{(x-2)\left(x^2+x+1\right)}=\lim\limits_{x\to 2} \dfrac {x-3}{x^2+x+1}=\dfrac {-1}{7}$.}
\end{ex}
%Cau11
\begin{ex}%[DCHT Toán 11 - KNTT -Nguyễn Văn Hiệp]%[1K5BF-3]
	Tìm $\lim\limits_{x\to 1} \dfrac {x^4-3x^2+2}{x^3+2x-3}$.
	\choice
	{$-\dfrac {5}{2}$}
	{\True $-\dfrac {2}{5}$}
	{$\dfrac {1}{5}$}
	{$+\infty $}
	\loigiai{$\lim\limits_{x\to 1} \dfrac {x^4-3x^2+2}{x^3+2x-3}=\lim\limits_{x\to 1} \dfrac {(x-1)(x+1)\left(x^2-2\right)}{(x-1)\left(x^2+x+3 \right)}=\lim\limits_{x\to 1} \dfrac {(x+1)\left(x^2-2 \right)}{x^2+x+3}=-\dfrac {2}{5}$.}
\end{ex}
%Câu 12
\begin{ex}%[DCHT Toán 11 - KNTT -Nguyễn Văn Hiệp]%[1K5BF-3]
	Tính $\lim\limits_{x\to 2} \left( \dfrac {1}{x^2-3x+2}+\dfrac {1}{x^2-5x+6} \right)$.
	\choice
	{$2 $}
	{$+\infty$}
	{\True $-2$}
	{$0$}
	\loigiai
	{\allowdisplaybreaks
		$\begin{aligned}[t]
			&\lim\limits_{x\to 2} \left( \dfrac {1}{x^2-3x+2}+\dfrac {1}{x^2-5x+6} \right)=\lim\limits_{x\to 2} \dfrac {2x^2-8x+8}{\left(x^2-3x+2 \right) \left(x^2-5x+6\right)}\\
			=&\lim\limits_{x\to 2} \dfrac {2(x-2)^2}{(x-1)(x-2)(x-2)(x-3)}=\lim\limits_{x\to 2} \dfrac {2}{(x-1)(x-3)}=-2.
		\end{aligned}$
	}
\end{ex}
%Cau 13
\begin{ex}%[DCHT Toán 11 - KNTT -Nguyễn Văn Hiệp]%[1K5BF-3]
	Giới hạn $\lim \limits_{ x \to + \infty} \dfrac {\sqrt {x ^2  + 2} - 2} {x - 2}$ bằng
	\choice
	{$- \infty$}
	{\True $1$}
	{$+\infty$}
	{$-1$}
	\loigiai{$\lim\limits_{x\to +\infty } \dfrac {\sqrt {x^2+2}-2}{x-2}=\lim\limits_{x\to +\infty } \dfrac {x\sqrt {1+\dfrac {2}{x^2}}-2}{x-2}=\lim\limits_{x\to +\infty } \dfrac {\sqrt {1+\dfrac {2}{x^2}}-\dfrac {2}{x}}{1-\dfrac {2}{x}}=1$.}
\end{ex}
%Cau 14
\begin{ex}%[DCHT Toán 11 - KNTT -Nguyễn Văn Hiệp]%[1K5BF-3]
	Cho hàm số $ f(x)=\dfrac {(4x+1)^3(2x+1)^4}{(3+2x)^7}$. Tính $\lim\limits_{x\to -\infty}f(x)$.
	\choice
	{$2$}
	{\True $8$}
	{$4$}
	{$0$}
	\loigiai{$\lim\limits_{x\to -\infty} f(x)=\lim\limits_{x\to -\infty } \dfrac {(4x+1)^3(2x+1)^4}{(3+2x)^7}=\lim\limits_{x\to -\infty} \dfrac {\left(4+\dfrac {1}{x} \right)^3\left( 2+\dfrac {1}{x} \right)^4}{\left( \dfrac {3}{x}+2 \right)^7}=2^3=8$.}
\end{ex}
%Cau 15
\begin{ex}%[DCHT Toán 11 - KNTT -Nguyễn Văn Hiệp]%[1K5KF-3]
	Biết $\lim\limits_{x\to 3} \dfrac {x^2+bx+c}{x-3}=8$, $(b,c\in \mathbb{R})$. Tính $P=b+c$.
	\choice
	{\True $P=-13$}
	{$P=-11$}
	{$P=5$}
	{$P=-12 $}
	\loigiai{Vì $\lim\limits_{x\to 3} \dfrac {x^2+bx+c}{x-3}=8$ là hữu hạn nên tam thức $ x^2+bx+c$ có nghiệm $x=3$.\\
		$\Rightarrow 3b+c+9=0\Leftrightarrow c=-9-3b$.\\
		Khi đó
		$\begin{aligned}[t]
			\lim\limits_{x\to 3} \dfrac {x^2+bx+c}{x-3}&=\lim\limits_{x\to 3} \dfrac {x^2+bx-9-3b}{x-3}=\lim\limits_{x\to 3} \dfrac {(x-3)(x+3+b)}{x-3} \\
			&=\lim\limits_{x\to 3} (x+3+b)=8\Leftrightarrow 6+b=8\Leftrightarrow b=2\Rightarrow c=-15.
		\end{aligned}$\\
		Vậy $P=b+c=-13$.}
\end{ex}
%Cau 16
\begin{ex}%[DCHT Toán 11 - KNTT -Nguyễn Văn Hiệp]%[1K5KF-3]
	Cho $a,b$ là số nguyên và $\lim\limits_{x\to 1} \dfrac {ax^2+bx-5}{x-1}=7$. Tính $a^2+b^2+a+b$.
	\choice
	{\True $18$}
	{$1$}
	{$15$}
	{$5$}
	\loigiai{Vì $\lim\limits_{x\to 1} \dfrac {ax^2+bx-5}{x-1}=7$ hữu hạn nên $ x=1$ phải là nghiệm của phương trình $ ax^2+bx-5=0$ suy ra $ a+b-5=0\Rightarrow b=5-a$.\\
		Khi đó $\lim\limits_{x\to 1} \dfrac {ax^2+\left( 5-a \right)x-5}{x-1}=\lim\limits_{x\to 1} \dfrac {\left( x-1 \right)( ax+5 )}{x-1}=a+5=7\Rightarrow a=2$ nên $b=3$.\\
		Suy ra $a^2+b^2+a+b=18$.}
\end{ex}
%Cau 17
\begin{ex}%[DCHT Toán 11 - KNTT -Nguyễn Văn Hiệp]%[1K5KF-3]
	Biết rằng $\lim\limits_{x\to +\infty } \left( \dfrac {x^2+1}{x-2}+ax-b \right)=-5$. Tính tổng $a+b$.
	\choice
	{\True $6$}
	{$7$}
	{$8$}
	{$5$}
	\loigiai{
		\allowdisplaybreaks
		$\begin{aligned}[t]
			&\lim\limits_{x\to +\infty } \left(\dfrac {x^2+1}{x-2}+ax-b\right)=\lim\limits_{x\to +\infty}\left( \dfrac {(a+1)x^2-(2a+b)x+2b+1}{x-2}\right)=-5\\
			\Leftrightarrow& \heva{&a+1=0 \\ &2a+b=5}\Leftrightarrow \heva{&a=-1\\&b=7.}
		\end{aligned}$\\
		Vậy $ a+b=6$.}
\end{ex}
%Cau 18
\begin{ex}%[DCHT Toán 11 - KNTT -Nguyễn Văn Hiệp]%[1K5KF-3]
	Cho hai số thực $ a$ và $ b$ thỏa mãn $\lim\limits_{x\to +\infty} \left( \dfrac {4x^2-3x+1}{x+2}-ax-b\right)=0$. Khi đó $a+b$ bằng
	\choice
	{$-4$}
	{$4$}
	{$7$}
	{\True $-7$}
	\loigiai{$\lim\limits_{x\to +\infty}\left(\dfrac{4x^2-3x+1}{x+2}-ax-b \right)=0\Leftrightarrow \lim\limits_{x\to +\infty } \left( \left( 4-a \right)x-b-11+\dfrac {23}{x+2} \right)=0$.\\
		$\Rightarrow \heva{&4-a=0\\&-11-b=0 }\Leftrightarrow \heva{&a=4\\&b=-11}\Rightarrow a+b=-7$.}
\end{ex}
%Cau 19
\begin{ex}%[DCHT Toán 11 - KNTT -Nguyễn Văn Hiệp]%[1K5KF-3]
	Cho $\lim\limits_{x\to 1} \dfrac {f(x)+1}{x-1}=-1$. Tính $\lim\limits_{x\to 1} \dfrac {\left(x^2+x\right)f(x)+2}{x-1}$.
	\choice
	{$I=5$}
	{$I=-4$}
	{$I=4$}
	{\True $I=-5$}
	\loigiai{$\lim\limits_{x\to 1} \dfrac {\left(x^2+x \right)f(x)+2}{x-1}=\lim\limits_{x\to 1} \dfrac {\left(x^2+x \right)\left(f(x)+1 \right)-x^2-x+2}{x-1}=\lim\limits_{x\to 1} \left( \dfrac {\left(x^2+x\right)(f(x)+1)}{x-1}-x-2\right)=-5$.}
\end{ex}
%Cau 20
\begin{ex}%[DCHT Toán 11 - KNTT -Nguyễn Văn Hiệp]%[1K5GF-3]
	Gọi $A$ là giới hạn của hàm số $f(x)=\dfrac {x+x^2+x^3+\cdots +x^{50}-50}{x-1}$ khi $x$ tiến đến $1$. Tính giá trị của $A$.
	\choice
	{$A$ không tồn tại}
	{$A=1725$}
	{$A=1527$}
	{\True $A=1275$}
	\loigiai
	{Ta có \allowdisplaybreaks
		\begin{eqnarray*}
			\lim\limits_{x\to 1} f(x)&=&\lim\limits_{x\to 1} \dfrac {x+x^2+x^3+\cdots +x^{50}-50}{x-1}\\
			&=&\lim\limits_{x\to 1} \left[ 1+(x+1)+\left(x^2+x+1\right)+\cdots +\left(x^{49}+x^{48}+\cdots +1\right) \right]\\
			&=&1+2+3+\cdots +50=25(1+50)=1275.
		\end{eqnarray*}
		Vậy $A=\lim\limits_{x\to 1} f(x)=1275$.}
\end{ex}
\Closesolutionfile{ans}
% \begin{indapan}{10}
% 	{ans/ans-1K5-2-Dang2}
% \end{indapan}
\begin{dang}{Phương pháp đặt thừa số chung - kết quả vô cực}
	Để tìm giới hạn của hàm số ta cần nhớ
	\begin{itemize}
		\item $\lim\limits_{x\to +\infty} x^k=+\infty$; $\lim\limits_{x\to -\infty} x^k=\heva{& +\infty,k=2n\\& -\infty,k=2n+1.}$
		\item $\lim\limits_{x\to \pm \infty}c=c$; $\lim\limits_{x\to \pm \infty} \dfrac {c}{x^k}=0$; $\lim\limits_{x\to 0} \dfrac {1}{x}=\infty$.
	\end{itemize}
\end{dang}
\subsubsection{Ví dụ minh hoạ}
%VD1
\begin{vd}[NB]%[DCHT Toán 11 - KNTT -Nguyễn Văn Hiệp]%[1K5YF-4]
	Tính $\lim\limits_{x\to +\infty} x^3$.\dapso{$+\infty$.}
	\loigiai{Ta có $\lim\limits_{x\to +\infty} x^3=+\infty$.}
\end{vd}
%VD2
\begin{vd}[TH]%[DCHT Toán 11 - KNTT -Nguyễn Văn Hiệp]%[1K5BF-4]
	Tính $\lim\limits_{x\to -\infty }\left(x^3+3x+1\right)$.
	\dapso{$-\infty$}
	\loigiai{Ta có $\lim\limits_{x\to -\infty } \left( x^3+3x+1 \right)=\lim\limits_{x\to -\infty } \left[ x^3\left( 1+\dfrac {3}{x^2}+\dfrac {1}{x^3} \right) \right]=-\infty $.\\
		Vì $\lim\limits_{x\to -\infty } x^3=-\infty$; $\lim\limits_{x\to -\infty } \left( 1+\dfrac {3}{x^2}+\dfrac {1}{x^3} \right)=1>0$.}
\end{vd}
%VD3
\begin{vd}[TH]%[DCHT Toán 11 - KNTT -Nguyễn Văn Hiệp]%[1K5BF-4]
	Tính $\lim\limits_{x\to -\infty}\left(-4x^5-3x^3+x+1\right)$.
	\dapso{$+\infty$.}
	\loigiai{
		Ta có $\lim\limits_{x\to -\infty } \left( -4x^5-3x^3+x+1 \right)=\lim\limits_{x\to -\infty } x^5\left( -4-\dfrac {3}{x^2}+\dfrac {1}{x^4}+\dfrac {1}{x^5} \right)=+\infty $.\\
		Vì $\heva{&\lim\limits_{x\to -\infty } \left( -4-\dfrac {3}{x^2}+\dfrac {1}{x^4}+\dfrac {1}{x^5} \right)=-4<0 \\& \lim\limits_{x\to -\infty } x^5=-\infty.}$
	}
\end{vd}
%VD4
\begin{vd}[TH]%[DCHT Toán 11 - KNTT -Nguyễn Văn Hiệp]%[1K5BF-4]
	Tính giới hạn $\lim\limits_{x\to -3} \dfrac {x+2}{(x+3)^2}$. 
	\dapso{$-\infty$.}
	\loigiai{Ta có $\lim\limits_{x\to -3} \dfrac {x+2}{(x+3)^2}=-\infty $.\\
		Vì $\lim\limits_{x\to -3} (x+2)=-3+2=-1<0$, $\lim\limits_{x\to -3} (x+3)^2=0$ và $(x+3)^2>0$ khi $ x\ne -3$.}
\end{vd}
%VD5
\begin{vd}[VDT]%[DCHT Toán 11 - KNTT -Nguyễn Văn Hiệp]%[1K5KF-4]
	Tìm tất cả các giá trị nguyên của tham số $m$ để $I=\lim\limits_{x\to +\infty}\left[(m^2-1)x^3+2x\right]=-\infty$.
	\dapso{$m=0$}
	\loigiai{
		Ta có $\lim\limits_{x\to +\infty } \left[ \left(m^2-1 \right)x^3+2x \right]=\lim\limits_{x\to +\infty} x^3\left[ m^2-1+\dfrac {2}{x^2} \right]$.\\
		Vì $\lim\limits_{x\to +\infty} x^3=+\infty $ nên $I=-\infty \Leftrightarrow \lim\limits_{x\to +\infty } \left[ m^2-1+\dfrac {2}{x^2} \right]<0\Leftrightarrow m^2-1<0\Leftrightarrow -1<m<1$.\\
		Do $ m\in \mathbb{Z}$ nên $m=0$.}
\end{vd}
% \subsubsection{Bài tập rèn luyện}
% % \centerline{\fcolorbox{red}{yellow!50}{\bf {BÀI TẬP TỰ LUẬN}}}
% %bt1
% \begin{bt}[NB]%[DCHT Toán 11 - KNTT -Nguyễn Văn Hiệp]%[1K5YF-4]
% 	Tính $\lim\limits_{x\to -\infty} x^2$.\dapso{$+\infty$.}
% 	\loigiai{Ta có $\lim\limits_{x\to -\infty} x^2=+\infty$.}
% \end{bt}
% %bt2
% \begin{bt}[NB]%[DCHT Toán 11 - KNTT -Nguyễn Văn Hiệp]%[1K5YF-4]
% 	Tính $\lim\limits_{x\to -\infty} \left(-x^4-\dfrac{1}{x}\right)$.\dapso{$-\infty$.}
% 	\loigiai{Ta có $\lim\limits_{x\to -\infty} -x^4=-\infty$ và $\lim\limits_{x\to -\infty} \dfrac{1}{x}=0$. Suy ra $\lim\limits_{x\to -\infty} \left(-x^4-\dfrac{1}{x}\right)=-\infty$.}
% \end{bt}
% %bt3
% \begin{bt}[TH]%[DCHT Toán 11 - KNTT -Nguyễn Văn Hiệp]%[1K5BF-4]
% 	Tính giới hạn $\lim\limits_{x\to +\infty }\left(-x^3+5x^2+2x+1\right)$.
% 	\dapso{$-\infty$.}
% 	\loigiai{
% 		Ta có$\lim\limits_{x\to +\infty} \left(-x^3+5x^2+2x+1 \right)=\lim\limits_{x\to +\infty } \left[ x^3\left(-1+\dfrac {5}{x}+\dfrac {2}{x^2}+\dfrac {1}{x^3} \right) \right]$.\\
% 		Do $\lim\limits_{x\to +\infty}x^3=+\infty$; $\lim\limits_{x\to +\infty}\left(-1+\dfrac {5}{x}+\dfrac {2}{x^2}+\dfrac {1}{x^3} \right)=-1<0$ nên $\lim\limits_{x\to +\infty} \left(-x^3+5x^2+2x+1\right)=-\infty$.}
% \end{bt}
% %bt4
% \begin{bt}[TH]%[DCHT Toán 11 - KNTT -Nguyễn Văn Hiệp]%[1K5BF-4]
% 	Tính $\lim\limits_{x\to +\infty} \dfrac {3x^2-x}{x+1}$.
% 	\dapso{$+\infty$.}
% 	\loigiai{
% 		Ta có $\lim\limits_{x\to +\infty} \dfrac {3x^2-x}{x+1}=\lim\limits_{x\to +\infty} \dfrac{x^2}{x}\cdot \left(\dfrac{3-\dfrac{1}{x}}{1+\dfrac{1}{x}} \right)=\lim\limits_{x\to +\infty} x\cdot \left(\dfrac {3-\dfrac {1}{x}}{1+\dfrac {1}{x}} \right)=+\infty $.\\
% 		Vì $\lim\limits_{x\to +\infty} x=+\infty $ và $\lim\limits_{x\to +\infty} \dfrac {3-\dfrac {1}{x}}{1+\dfrac {1}{x}}=3$.}
% \end{bt}
% %bt5
% \begin{bt}[TH]%[DCHT Toán 11 - KNTT -Nguyễn Văn Hiệp]%[1K5BF-4]
% 	Giá trị của giới hạn $\lim\limits_{x\to +\infty} \left(\sqrt {1+2x^2}-x \right)$ là bao nhiêu?
% 	\dapso{$+\infty$.}
% 	\loigiai{Ta có $\lim\limits_{x\to +\infty} \left(\sqrt {1+2x^2}-x\right)=\lim\limits_{x\to +\infty } x\left(\sqrt {\dfrac {1}{x^2}+2}-1 \right)=+\infty$.\\
% 		Vì $\lim\limits_{x\to +\infty}x=+\infty$; $\lim\limits_{x\to +\infty}\left(\sqrt {\dfrac {1}{x^2}+2}-1 \right)=\sqrt {2}-1>0$.
% 	}
% \end{bt}
% %bt6
% \begin{bt}[TH]%[DCHT Toán 11 - KNTT -Nguyễn Văn Hiệp]%[1K5BF-4]
% 	Tính $\lim\limits_{x\to 3} \left( \dfrac {1}{x}-\dfrac {1}{3} \right)\dfrac {1}{(x-3)^3}$.
% 	\dapso{$-\infty$}
% 	\loigiai{$\lim\limits_{x\to 3} \left(\dfrac {1}{x}-\dfrac {1}{3} \right)\dfrac {1}{(x-3)^3}=\lim\limits_{x\to 3} \dfrac {3-x}{3x}\cdot \dfrac {1}{(x-3)^3}=\lim\limits_{x\to 3} \dfrac {-1}{3x(x-3)^2}=-\infty$.}
% \end{bt}
% %bt7
% \begin{bt}[VDT]%[DCHT Toán 11 - KNTT -Nguyễn Văn Hiệp]%[1K5KF-4]
% 	Có bao nhiêu giá trị $ m$ nguyên thuộc đoạn $[-20;20]$ để $\lim\limits_{x\to +\infty } \left( \sqrt {4x^2-3x+2}+mx-1 \right)=-\infty$?
% 	\dapso{$18$.}
% 	\loigiai{Ta có
% 		\allowdisplaybreaks
% 		$\begin{aligned}[t]
% 			\lim\limits_{x\to +\infty} \left(\sqrt {4x^2-3x+2}+mx-1 \right)&=\lim\limits_{x\to +\infty} \left(x\sqrt {4-\dfrac {3}{x}+\dfrac {2}{x^2}}+mx-1\right)\\
% 			&=\lim\limits_{x\to +\infty} x\left(\sqrt {4-\dfrac {3}{x}+\dfrac {2}{x^2}}+m-\dfrac {1}{x}\right).
% 		\end{aligned}$\\
% 		Mà $\lim\limits_{x\to +\infty } x=+\infty $ và $\lim\limits_{x\to +\infty } \left(\sqrt {4-\dfrac {3}{x}+\dfrac {2}{x^2}}+m-\dfrac {1}{x} \right)=2+m$ nên $\lim\limits_{x\to +\infty } \left( \sqrt {4x^2-3x+2}+mx-1 \right)=-\infty $ khi $2+m<0\Leftrightarrow m<-2$.\\
% 		Do $ m$ nguyên thuộc đoạn $[-20;20]$ nên $m\in \{ -20;-19;-18;\ldots;-3 \}$.\\
% 		Vậy có $18$ giá trị $ m$ nguyên thuộc đoạn $[-20;20]$ thỏa bài toán.}
% \end{bt}

\subsubsection{Câu hỏi trắc nghiệm}
\Opensolutionfile{ans}[ans/ans-1K5-2-Dang3]
%Câu 1
\begin{ex}%[DCHT Toán 11 - KNTT -Nguyễn Văn Hiệp]%[1K5YF-4]
	Giá trị của $\lim\limits_{x\to -\infty}(-x^3)$ bằng
	\choice
	{\True $+\infty$}
	{$-\infty$}
	{$1$}
	{$-1$}
	\loigiai{Ta có $\lim\limits_{x\to -\infty}(-x^3)=+\infty $.}
\end{ex}
%Cau2
\begin{ex}%[DCHT Toán 11 - KNTT -Nguyễn Văn Hiệp]%[1K5BF-4]
	Giới hạn$\lim\limits_{x\to -\infty} \left(3x^3+5x^2-9\sqrt {3}x-2022 \right)$ bằng
	\choice
	{\True $-\infty $}
	{$3$}
	{$-3$}
	{$+\infty $}
	\loigiai{Ta có $\lim\limits_{x\to -\infty } \left( 3x^3+5x^2-9\sqrt {3}x-2022 \right)=\lim\limits_{x\to -\infty } x^3\left( 3+5\cdot \dfrac {1}{x}-9\sqrt {3}\cdot \dfrac {1}{x^2}-2022\cdot \dfrac {1}{x^3} \right)=-\infty$.}
\end{ex}
%Cau3
\begin{ex}%[DCHT Toán 11 - KNTT -Nguyễn Văn Hiệp]%[1K5BF-4]
	Tính $\lim\limits_{x\to -\infty } ( x^3+3x-3 )$.
	\choice
	{$2$}
	{$1$}
	{\True $-\infty $}
	{$+\infty $}
	\loigiai{Ta có $\lim\limits_{x\to -\infty} \left( x^3+3x-3 \right)=\lim\limits_{x\to -\infty } \left[ x^3\left( 1+\dfrac {3}{x^2}-\dfrac {3}{x^3} \right) \right]=-\infty $.\\
		Vì $\lim\limits_{x\to -\infty}x^3=-\infty$; $\lim\limits_{x\to -\infty}\left(1+\dfrac {3}{x^2}-\dfrac {3}{x^3}\right)=1>0$.}
\end{ex}
%Cau4
\begin{ex}%[DCHT Toán 11 - KNTT -Nguyễn Văn Hiệp]%[1K5BF-4]
	Với $ k$ là số nguyên dương chẵn. Kết quả của $\lim\limits_{x\to -\infty } \left(-3x^k \right)$ là
	\choice
	{$0$}
	{\True $-\infty $}
	{$-3x_0^k$}
	{$+\infty $}
	\loigiai{Ta có $\lim\limits_{x\to -\infty } x^k=+\infty $ khi $ k$ là số nguyên dương chẵn.\\
		Suy ra $\lim\limits_{x\to -\infty } \left(-3x^k\right)=-\infty $.}
\end{ex}
%Cau5
\begin{ex}%[DCHT Toán 11 - KNTT -Nguyễn Văn Hiệp]%[1K5BF-4]
	Cho hai hàm số $ f(x)$, $g(x)$ thỏa mãn $\lim\limits_{x\to 1} f(x)=2$ và $\lim\limits_{x\to 1} g(x)=+\infty $. Giá trị của $\lim\limits_{x\to 1} [ f(x)\cdot g(x)]$ bằng
	\choice
	{\True $+\infty $}
	{$-\infty $}
	{$2$}
	{$-2$}
	\loigiai{Theo quy tắc giới hạn vô cực ta có $\lim\limits_{x\to 1} f(x)=2>0$ và $\lim\limits_{x\to 1} g( x )=+\infty $ thì $\lim\limits_{x\to 1} [f(x)\cdot g(x)]=+\infty $.}
\end{ex}
%Cau6
\begin{ex}%[DCHT Toán 11 - KNTT -Nguyễn Văn Hiệp]%[1K5BF-4]
	Giới hạn $\lim\limits_{x\to -2} \dfrac {x+1}{(x+2)^2}$ bằng
	\choice
	{\True $-\infty $}
	{$\dfrac {3}{16}$}
	{$0$}
	{$+\infty $}
	\loigiai{Ta có $\lim\limits_{x\to -2} \dfrac {x+1}{(x+2)^2}=-\infty$.\\
		Vì $\lim\limits_{x\to -2} (x+1)=-2+1=-1<0$, $\lim\limits_{x\to -2} (x+2)^2=0$ và $(x+2)^2>0$ khi $ x\ne -2$.}
\end{ex}
%Cau7
\begin{ex}%[DCHT Toán 11 - KNTT -Nguyễn Văn Hiệp]%[1K5BF-4]
	$\lim\limits_{x\to -\infty }\left(-3x^3+2x\right)$ bằng
	\choice
	{$-\infty$}
	{\True $+\infty$}
	{$1$}
	{$-1$}
	\loigiai{Ta có $\lim\limits_{x\to -\infty }\left( -3x^3+2x \right)=\lim\limits_{x\to -\infty}x^3\left(-3+\dfrac {2}{x^2} \right)=+\infty $.\\
		Vì $\lim\limits_{x\to -\infty } x^3=-\infty $ và $\lim\limits_{x\to -\infty } \left( -3+\dfrac {2}{x^2} \right)=-3<0$.}
\end{ex}
%Cau8
\begin{ex}%[DCHT Toán 11 - KNTT -Nguyễn Văn Hiệp]%[1K5BF-4]
	Tìm $L=\lim\limits_{x\to -1} \dfrac {2x^2+x-3}{(x+1)^2}$.
	\choice
	{$L=+\infty $}	
	{$L=2$}
	{Không tồn tại $\lim\limits_{x\to -1} \dfrac {2x^2+x-3}{(x+1)^2}$}
	{\True $L=-\infty $}
	\loigiai{$\lim\limits_{x\to -1} \dfrac {2x^2+x-3}{(x+1)^2}=-\infty $ vì $\heva{& \lim\limits_{x\to -1} (2x^2+x-3)=-2 \\ & \lim\limits_{x\to -1} (x+1)^2=0 \\ & x\to -1\Rightarrow (x+1)^2>0.}$}
\end{ex}
%Cau9
\begin{ex}%[DCHT Toán 11 - KNTT -Nguyễn Văn Hiệp]%[1K5BF-4]
	$\lim\limits_{x\to +\infty } \left(-2x^3-2x\right)$ bằng
	\choice
	{\True $-\infty$}
	{$+\infty$}
	{$2$}
	{$-2$}
	\loigiai{Ta có $\lim\limits_{x\to +\infty} \left(-2x^3-2x\right)=\lim\limits_{x\to +\infty}x^3\left(-2-\dfrac {2}{x^2}\right)$.\\
		Mà $\lim\limits_{x\to +\infty}x^3=+\infty$; $\lim\limits_{x\to +\infty}(-2-\dfrac {2}{x^2})=-2<0$ nên $\lim\limits_{x\to +\infty}x^3\left(-2-\dfrac {2}{x^2}\right)=-\infty$.\\
		Vậy $\lim\limits_{x\to +\infty }\left(-2x^3-2x\right)=-\infty$.}
\end{ex}
%Cau10
\begin{ex}%[DCHT Toán 11 - KNTT -Nguyễn Văn Hiệp]%[1K5BF-4]
	Tính giới hạn $\lim\limits_{x\to -\infty } \dfrac {x^2+1}{x-2}$.
	\choice
	{$1$}
	{$-\dfrac {1}{2}$}
	{$+\infty $}
	{\True $-\infty $}
	\loigiai{$\lim\limits_{x\to -\infty} \dfrac {x^2+1}{x-2}=\lim\limits_{x\to -\infty } \dfrac {x^2}{x}\cdot \dfrac {1+\dfrac {1}{x^2}}{1-\dfrac {2}{x}}=\lim\limits_{x\to -\infty } x\cdot \dfrac {1+\dfrac {1}{x^2}}{1-\dfrac {2}{x}}$.\\
		Do $\lim\limits_{x\to -\infty } x=-\infty $ và $\lim\limits_{x\to -\infty } \dfrac {1+\dfrac {1}{x^2}}{1-\dfrac {2}{x}}=1$ nên $\lim\limits_{x\to -\infty } \dfrac {x^2+1}{x-2}=-\infty $.}
\end{ex}
%Câu 11
\begin{ex}%[DCHT Toán 11 - KNTT -Nguyễn Văn Hiệp]%[1K5KF-4]
	Trong các mệnh đề sau, mệnh đề nào đúng?
	\choice
	{\True $\lim\limits_{x\to -\infty } \dfrac {\sqrt {x^4-x}}{1-2x}=+\infty $}
	{$\lim\limits_{x\to -\infty } \dfrac {\sqrt {x^4-x}}{1-2x}=1$}
	{$\lim\limits_{x\to -\infty } \dfrac {\sqrt {x^4-x}}{1-2x}=-\infty $}
	{$\lim\limits_{x\to -\infty } \dfrac {\sqrt {x^4-x}}{1-2x}=0$}
	\loigiai{
		Vì $\lim\limits_{x\to -\infty } \dfrac {\sqrt {x^4-x}}{1-2x}=\lim\limits_{x\to -\infty} \dfrac {x^2 \cdot \sqrt {1-\dfrac {1}{x^3}}}{x\cdot \left( \dfrac {1}{x}-2 \right)}=\lim\limits_{x\to -\infty } x\cdot \dfrac {\sqrt {1-\dfrac {1}{x^3}}}{\dfrac {1}{x}-2}=+\infty $. }
\end{ex}
%Cau 12
\begin{ex}%[DCHT Toán 11 - KNTT -Nguyễn Văn Hiệp]%[1K5KF-4]
	Biết $\lim\limits_{x\to -2} f(x)=-1$. Khi đó $\lim\limits_{x\to -2} \dfrac {f(x)}{(x+2)^4}$ bằng
	\choice
	{$-1$}
	{$+\infty $}
	{\True $-\infty $}
	{$0$}
	\loigiai{Ta có $\lim\limits_{x\to -2} f(x)=-1<0$; $\lim\limits_{x\to -2} (x+2)^4=0$ và $\forall x\ne -2$ thì $(x+2)^4>0$.\\
		Suy ra $\lim\limits_{x\to -2} \dfrac {f(x)}{(x+2)^4}=-\infty $.}
\end{ex}
%Cau 13
\begin{ex}%[DCHT Toán 11 - KNTT -Nguyễn Văn Hiệp]%[1K5KF-4]
	Biết $\lim\limits_{x\to 1} f(x)=-2$. Khi đó $\lim\limits_{x\to 1} \dfrac {f(x)}{(x-1)^2}$ bằng
	\choice
	{\True $-\infty $}
	{$0$}
	{$+\infty$}
	{$-2$}
	\loigiai{Có $\lim\limits_{x\to 1} f(x)=-2<0$, $\lim\limits_{x\to 1} (x-1)^2=0$ và $(x-1)^2>0$, $\forall x\ne 1$ nên $\lim\limits_{x\to 1} \dfrac {f(x)}{(x-1)^2}=-\infty $.}
\end{ex}
%Cau 14
\begin{ex}%[DCHT Toán 11 - KNTT -Nguyễn Văn Hiệp]%[1K5KF-4]
	Có bao nhiêu giá trị nguyên của tham số $m$ thuộc $[-5;5]$ để $L=\lim\limits_{x\to +\infty}\left[x-2(m^2-4)x^3\right]=-\infty$?
	\choice
	{$5$}
	{$10$}
	{$3$}
	{\True $6$}
	\loigiai{Ta có $\lim\limits_{x\to +\infty }\left[ x-2\left( m^2-4 \right)x^3 \right]=\lim\limits_{x\to +\infty } x^3\left[\dfrac {1}{x^2}-2\left( m^2-4 \right) \right]$.\\
		Ta có
		\allowdisplaybreaks
		\begin{eqnarray*}
			&&\lim\limits_{x\to +\infty}x^3=+\infty \Rightarrow L=-\infty \Leftrightarrow \lim\limits_{x\to +\infty} \left[\dfrac {1}{x^2}-2\left(m^2-4\right) \right]<0\\
			&\Leftrightarrow& -2(m^2-4)<0\Leftrightarrow m^2-4>0\Leftrightarrow \hoac{&m>2\\& m<-2.}
		\end{eqnarray*}
		Lại có $m$ thuộc đoạn $[-5;5]$ nên các giá trị nguyên thỏa mãn bài toán của $m$ là $\{-5;-4;-3;3;4;5\}$.\\
		Vậy có $6$ số nguyên thỏa mãn bài toán.}
\end{ex}
%Câu 15
\begin{ex}%[DCHT Toán 11 - KNTT -Nguyễn Văn Hiệp]%[1K5GF-4]
	Có bao nhiêu giá trị $m$ nguyên thuộc đoạn $[-20;20 ]$ để $\lim\limits_{x\to -\infty } \left( \sqrt {4x^2-3x+2}+mx-1 \right)=-\infty$?
	\choice
	{$21$}
	{$22$}
	{\True $18$}
	{$41$}
	\loigiai{Ta có $\lim\limits_{x\to -\infty } \left( \sqrt {4x^2-3x+2}+mx-1 \right)=\lim\limits_{x\to -\infty } x\left( -\sqrt {4-\dfrac {3}{x}+\dfrac {2}{x^2}}+m-\dfrac {1}{x} \right)$.\\
		Có $\lim\limits_{x\to -\infty } x=-\infty $ và $\lim\limits_{x\to -\infty } \left( -\sqrt {4-\dfrac {3}{x}+\dfrac {2}{x^2}}+m-\dfrac {1}{x} \right)=m-2$\\
		Để $\lim\limits_{x\to -\infty } \left( \sqrt {4x^2-3x+2}+mx-1 \right)=-\infty $ suy ra $ m-2>0\Leftrightarrow m>2$.\\
		Với $ m\in \mathbb{Z}$ và $ m\in [ -20;20 ]$ có $ m\in \{ 3;4;5;\ldots ;20 \}$ thỏa mãn yêu cầu bài toán.\\
		Kết luận: Vậy có $18$ giá trị nguyên của $ m$ thỏa mãn yêu cầu bài toán.}
\end{ex}
\Closesolutionfile{ans}
% \begin{indapan}{10}
% 	{ans/ans-1K5-2-Dang3}
% \end{indapan}
\begin{dang}{Phương pháp lượng liên hợp kết quả hữu hạn}
	Nội dung và phương pháp giải
\end{dang}
\subsubsection{Ví dụ minh hoạ}
\Opensolutionfile{ans}[ans/ans-1K5-2-Dang4]
\setcounter{vd}{0}
\begin{vd}%[1K5BF-5]
	Cho $P = \lim \limits_{x \to 2} \dfrac{\sqrt{x+2}-2}{x-2}$. Tính $P$.
	\choice
	{\True $P = \dfrac{1}{4}$}
	{$P = \dfrac{1}{2}$}
	{$P = 1$}
	{$P = 0$}
	\loigiai{
		Ta có: $\lim \limits_{x \to 2} \dfrac{\sqrt{x+2}-2}{x-2} = \lim \limits_{x \to 2} \dfrac{x-2}{(x-2) \left(\sqrt{x + 2}+2 \right)} = \lim \limits_{x \to 2} \dfrac{1}{\sqrt{x + 2}+2} = \dfrac{1}{4}$. \\
		Vậy $P = \dfrac{1}{4}$.
	}
\end{vd}
\begin{vd}%[1K5BF-5]
	Cho $m$ là hằng số. Tính $\lim\limits_{x\to 1}\dfrac{\sqrt{x+3}-2}{x^2+mx-x-m}$.
	\choice
	{$\dfrac{1}{m}$}
	{$1$}
	{$\dfrac{1}{4}$}
	{\True $\dfrac{1}{4(m+1)}$}
	\loigiai{
		$\lim\limits_{x\to 1}\dfrac{\sqrt{x+3}-2}{x^2+mx-x-m}=\lim\limits_{x\to 1}\dfrac{x-1}{(x-1)(x+m)\left(\sqrt{x+3}+2\right)}=\dfrac{1}{4(m+1)}$.
	}
\end{vd}
\begin{vd}%[1K5BF-5]
	Biết $\lim\limits_{x\to -\infty}\left(\sqrt{x^2+1}+x+1\right)=a$. Tính $2a+1.$
	\choice
	{$-1$}
	{$-3$}
	{$0$}
	{\True $3$}
	\loigiai{
		{\allowdisplaybreaks
			\begin{eqnarray*}
				\lim\limits_{x\to -\infty}\left(\sqrt{x^2+1}+x+1\right)
				&=&\lim\limits_{x\to -\infty}\dfrac{-2x}{\sqrt{x^2+1}-(x+1)}\\
				&=&\lim\limits_{x\to -\infty}\dfrac{-2}{-\sqrt{1+\dfrac{1}{x^2}}-\left(1+\dfrac{1}{x}\right)}\\
				&=&1\\
				&\Rightarrow& a=1.	
			\end{eqnarray*}
			Vậy $2a+1=3$.
		}		
	}
\end{vd}


\begin{vd}%[1K5KF-5]
	Biết $\lim\limits_{x\rightarrow +\infty} \left(\sqrt{4x^2-3x+1}-(ax+b)\right) = 0.$ Tính giá trị biểu thức $T=a-4b$.
	\choice
	{$T=-2$}
	{\True $T=5$}
	{$T=-1$}
	{$T=3$}
	\loigiai{
		Từ giả thiết, đường thẳng $y=ax+b$ là tiệm cận xiên của đồ thị hàm số $y=\sqrt{4x^2-3x+1}$, khi $x\to +\infty.$ Từ đó,
		\begin{eqnarray*}\begin{array}{ccl} a&=&\lim\limits_{x\rightarrow +\infty}\dfrac{\sqrt{4x^2-3x+1}}x=2,\\
				b&=&\lim\limits_{x\rightarrow +\infty}\left(\sqrt{4x^2-3x+1}-2x\right)\\
				&=&\lim\limits_{x\rightarrow +\infty}\dfrac{-3x+1}{\sqrt{4x^2-3x+1}+2x}\\
				&=&\lim\limits_{x\rightarrow +\infty}\dfrac{-3+\frac1x}{\sqrt{4-\frac3x+\frac1{x^2}}+2}=-\dfrac34.
			\end{array}
		\end{eqnarray*}
		Suy ra $a-4b=5.$}
\end{vd}
\begin{vd}%[1K5GF-5]
	Cho $f(x)$ là hàm đa thức thỏa $\lim\limits_{x \to 2}\dfrac{f(x)+1}{x-2}=a$ và tồn tại $\lim\limits_{x \to 2}\dfrac{\sqrt{f(x)+2x+1}-x}{x^2-4}=T$. Chọn đẳng thức đúng
	\choice
	{$T=\dfrac{a+2}{16}$}
	{$T=\dfrac{a+2}{8}$}
	{$T=\dfrac{a-2}{8}$}
	{\True $T=\dfrac{a-2}{16}$}
	\loigiai{
		Vì $f(x)$ là đa thức và $\lim\limits_{x \to 2}\dfrac{f(x)+1}{x-2}=a$ nên suy ra $f(x)+1=(x-2)g(x), g(2)=a$.\\
		Do đó
		\begin{eqnarray*}
			&T&=\lim\limits_{x \to 2}\dfrac{\sqrt{(x-2)g(x)+2x}-x}{x^2-4}\\
			& &=\lim\limits_{x \to 2}\dfrac{(x-2)g(x)+2x-x^2}{(x-2)(x+2)\left[\sqrt{(x-2)g(x)+2x}+x\right]}\\
			& &=\lim\limits_{x \to 2}\dfrac{g(x)-x}{(x+2)\left[\sqrt{(x-2)g(x)+2x}+x\right]} \\
			& &=\dfrac{a-2}{16}.
		\end{eqnarray*}
	}
\end{vd}
% \subsubsection{Bài tập rèn luyện}
\subsubsection{Câu hỏi trắc nghiệm}
\Opensolutionfile{ans}[ans/ans-1K5-2-Dang4]
%\setcounter{ex}{0}
\begin{ex}%[1K5BF-5]
	Biết $\displaystyle\lim_{x\rightarrow +\infty}\left(\sqrt{x^2+ax-1}-x\right)=5$. Khi đó giá trị của tham số $a$ là
	\choice
	{\True $10$}
	{$-6$}
	{$6$}
	{$-10$}
	\loigiai{
		$\displaystyle\lim_{x\rightarrow +\infty}\left(\sqrt{x^2+ax-1}-x\right)=\lim_{x\rightarrow +\infty}\dfrac{x^2+ax-1-x^2}{\sqrt{x^2+ax-1}+x}=\lim_{x\rightarrow +\infty}\dfrac{a-\dfrac{1}{x}}{\sqrt{1+\dfrac{a}{x}-\dfrac{1}{x^2}}+1}=\dfrac{a}{2}=5$.
		\\ Suy ra $a=10$. 
	}
\end{ex}
\begin{ex}%[1K5KF-5]
	Có tất cả bao nhiêu giá trị nguyên của tham số m để $\displaystyle\lim\limits_{x\to +\infty}(\sqrt{{x^2+m^2x}}-x)=\dfrac{1}{2}$?
	\choice
	{$0$}
	{$1$}
	{\True $2$}
	{$4$}
	\loigiai{
		Ta có \begin{align*}
			\displaystyle \lim \limits_{x\to +\infty}\left (\sqrt{x^2+m^2x}-x\right ) =&\displaystyle \lim \limits_{x\to +\infty}\dfrac{m^2x}{\sqrt{x^2+m^2x}+x}\\
			=& \displaystyle \lim \limits_{x\to +\infty}\dfrac{m^2}{\sqrt{1+\frac{m^2}{x}} +1}\\
			=&\dfrac{m^2}{2}.  
		\end{align*}	
		Do đó $\displaystyle\lim\limits_{x\to +\infty}(\sqrt{{x^2+m^2x}}-x)=\dfrac{1}{2}\Leftrightarrow \dfrac{m^2}{2}=\dfrac{1}{2}\Leftrightarrow m=\pm 1.$
		Vậy có hai giá trị nguyên của tham số $m$ thỏa mãn yêu cầu của bài toán.
	}
\end{ex}
\begin{ex}%[1K5BF-5]
	Tính $L=\lim\limits_{x \to - \infty} \left( \sqrt{x^2 -7x+1}- \sqrt{x^2-3x+2}\right)$.
	\choice
	{$L= + \infty$}
	{$L= - \infty$}
	{\True $L= 2$}
	{$L= -2$}
	\loigiai{
		\begin{align*}
			L &= \lim\limits_{x \to - \infty} \dfrac{-4x-1}{\sqrt{x^2 - 7x +1}+\sqrt{x^2-3x+2}}\\
			&=\lim\limits_{x \to - \infty} \dfrac{-4x-1}{-x \sqrt{1-\dfrac{7}{x}+\dfrac{1}{x^2}}-x \sqrt{1-\dfrac{3}{x}+\dfrac{2}{x^2}}}\\
			&= \lim\limits_{x \to - \infty} \dfrac{-4-\dfrac{1}{x}}{- \sqrt{1-\dfrac{7}{x}+\dfrac{1}{x^2}}-\sqrt{1-\dfrac{3}{x}+\dfrac{2}{x^2}}}\\
			&=2.
		\end{align*}
	}
\end{ex}


\begin{ex}%[1K5BF-5]
	Giá trị của $\lim\limits_{x \to 1} \dfrac{\sqrt {2x + 1}  - \sqrt {x + 2} }{x - 1}$  là 
	\choice
	{$- \dfrac{ \sqrt{3} }{5}$}
	{$-\dfrac{\sqrt{3} }{6}$}
	{\True $\dfrac{\sqrt{3}}{6}$}
	{$\dfrac{\sqrt{3}}{5}$}
	\loigiai{
		Ta có $$\lim\limits_{x \to 1} \dfrac{\sqrt{2x+1}+\sqrt{x+2}}{x-1}=\lim\limits_{x \to 1}\dfrac{x-1}{(x-1)\left(\sqrt{2x+1}-\sqrt{x+2} \right)}=\lim\limits_{x \to 1}\dfrac{1}{\sqrt{2x+1}+\sqrt{x+2}}=\dfrac{1}{2\sqrt{3}}.$$
	}
\end{ex}

\begin{ex}%[1K5BF-5]
	Cho $\displaystyle\lim_{x \to 0}\dfrac{1-\sqrt[3]{1-x}}{x}=\dfrac{m}{n}$, trong đó $m,n$ là các số nguyên và $\dfrac{m}{n}$ tối giản.\\ Tính $A=2m-n$.
	\choice
	{$A=1$}
	{\True $A=-1$}
	{$A=0$}
	{$A=-2$}
	\loigiai
	{Ta có $\displaystyle\lim_{x \to 0}\dfrac{1-\sqrt[3]{1-x}}{x}=\displaystyle\lim_{x \to 0}\dfrac{x}{x\cdot\left(1+\sqrt[3]{1-x}+\sqrt[3]{(1-x)^2}\right)}=\dfrac{1}{3}.$\\
		Vậy $A=2m-n=2\cdot1-3=-1.$}
\end{ex}
\begin{ex}%[1K5BF-5]
	Giới hạn $\displaystyle\lim\limits_{x\to 3}\dfrac{x+1-\sqrt{5x+1}}{x-\sqrt{4x-3}}=\dfrac{a}{b}$, với $a,b\in\mathbb{Z},b>0$ và $\dfrac{a}{b}$ là phân số tối giản. Giá trị của $a-b$ là
	\choice
	{$\dfrac{1}{9}$}
	{$-1$}
	{$\dfrac{9}{8}$}
	{\True $1$}
	\loigiai{
		Ta có
		\begin{eqnarray*}
			\displaystyle\lim\limits_{x\to 3}\dfrac{x+1-\sqrt{5x+1}}{x-\sqrt{4x-3}}&=& \lim\limits_{x\to 3}\dfrac{\dfrac{(x+1)^2-(5x+1)}{x+1+\sqrt{5x+1}}}{\dfrac{x^2-(4x-3)}{x+\sqrt{4x-3}}} \\
			&=& \lim\limits_{x\to 3}\dfrac{\left(x+\sqrt{4x-3}\right)(x-3)x}{\left(x+1+\sqrt{5x+1}\right)(x-3)(x-1)} \\
			&=& \lim\limits_{x\to 3}\dfrac{x\left(x+\sqrt{4x-3}\right)}{\left(x+1+\sqrt{5x+1}\right)(x-1)}=\dfrac{9}{8}.
		\end{eqnarray*}
		Vậy $a=9,b=8$, suy ra $a-b=1$.
	}
\end{ex}
\begin{ex}%[1K5BF-5]
	Cho $\underset{x\to 4}{\mathop{\lim}}\,\dfrac{\sqrt{3x+4}-4}{x-4}=\dfrac{a}{b}$, với $\dfrac{a}{b}$ là phân số tối giản. Tính $2a+b^2$.
	\choice
	{$22$}
	{$66$}
	{$14$}
	{\True $70$}
	\loigiai{
		Có $\underset{x\to 4}{\mathop{\lim}}\,\dfrac{\sqrt{3x+4}-4}{x-4}=\underset{x\to 4}{\mathop{\lim}}\,\dfrac{3\left( x-4 \right)}{\left( x-4 \right)\left( \sqrt{3x+4}+4 \right)}=\underset{x\to 4}{\mathop{\lim}}\,\dfrac{3}{\sqrt{3x+4}+4}=\dfrac{3}{8}$.\\
		$\Rightarrow 2a+b^2=6+64=70$.}
\end{ex}

\begin{ex}%[1K5BF-5]
	Tính $\lim\limits_{x \rightarrow + \infty} \left(\sqrt{x^2 + 3x + 2} - x\right)$.
	\choice
	{$- \dfrac{3}{2}$}
	{\True $\dfrac{3}{2}$}
	{$\dfrac{7}{2}$}
	{$- \dfrac{7}{2}$}
	\loigiai{
		\begin{eqnarray*}
			\lim\limits_{x \rightarrow + \infty} \left(\sqrt{x^2 + 3x + 2} - x\right) & = &\lim\limits_{x \rightarrow + \infty} \dfrac{3x + 2}{\sqrt{x^2 + 3x + 2} + x}\\
			& = &\lim\limits_{x \rightarrow + \infty} \dfrac{x\left(3 + \dfrac{2}{x}\right)}{|x|\sqrt{1 + \dfrac{3}{x} + \dfrac{2}{x^2}} + x}\\
			& = &\lim\limits_{x \rightarrow + \infty} \dfrac{x\left(3 + \dfrac{2}{x}\right)}{x\left(\sqrt{1 + \dfrac{3}{x} + \dfrac{2}{x^2}} + 1\right)}\\
			& = &\lim\limits_{x \rightarrow + \infty} \dfrac{3 + \dfrac{2}{x}}{\sqrt{1 + \dfrac{3}{x} + \dfrac{2}{x^2}} + 1} = \dfrac{3}{2}.
		\end{eqnarray*}
	}
\end{ex}
\begin{ex}%[1K5BF-5]
	Tìm giới hạn $M=\underset{x\to -\infty}{\lim}\left(\sqrt{x^2-4x}-\sqrt{x^2-x}\right)$.
	\choice
	{$M=-\dfrac{1}{2}$}
	{\True $M=\dfrac{3}{2}$}
	{$M=-\dfrac{3}{2}$}
	{$M=\dfrac{1}{2}$}
	\loigiai{
		Ta có
		\begin{align*}
			M&=\underset{x\to -\infty}{\lim}\left(\sqrt{x^2-4x}-\sqrt{x^2-x}\right)\\
			&=\underset{x\to -\infty}{\lim}\dfrac{-3x}{\sqrt{x^2-4x}+\sqrt{x^2-x}}\\
			&=\underset{x\to -\infty}{\lim}\dfrac{-3x}{|x|\left(\sqrt{1-\dfrac{4}{x}}+\sqrt{1-\dfrac{1}{x}}\right)}\\
			&=\underset{x\to -\infty}{\lim}\dfrac{3}{\sqrt{1-\dfrac{4}{x}}+\sqrt{1-\dfrac{1}{x}}}=\dfrac{3}{2}.
		\end{align*}
	}
\end{ex}





\begin{ex}%[1K5BF-5]
	Biết rằng $\lim\limits_{x\to -\infty} \left(\sqrt{2x^2-3x+1}+x\sqrt{2}\right)=\dfrac{a}{b}\sqrt{2}$,\quad ($a,b\in\mathbb{Z}, b>0, \dfrac{a}{b}$ tối giản). Tổng $a+b$ có giá trị là
	\choice
	{$5$}
	{$4$}
	{\True $7$}
	{$1$}
	\loigiai{
		Ta có
		\begin{eqnarray*}
			\lim\limits_{x\to -\infty} \left(\sqrt{2x^2-3x+1}+x\sqrt{2}\right) & = & \lim\limits_{x\to -\infty} \dfrac{2x^2-3x+1-2x^2}{\sqrt{2x^2-3x+1}-x\sqrt{2}}\\
			&=& \lim\limits_{x\to -\infty} \dfrac{-3x+1}{\sqrt{2x^2-3x+1}-x\sqrt{2}}\\
			&=&\lim\limits_{x\to -\infty} \dfrac{-3+\dfrac{1}{x}}{-\sqrt{2-\dfrac{3}{x}+\dfrac{1}{x^2}}-\sqrt{2}}\\
			&=& \dfrac{3}{4}\cdot \sqrt{2}.
		\end{eqnarray*}
		Vậy $a=3, b=4$. Tổng $a+b=7$.
	}
\end{ex}




\begin{ex}%[1K5BF-5]
	Tìm giới hạn $I=\underset{x\to +\infty}{\mathop{\lim}}\,\left( x+1-\sqrt{x^2-x+2} \right)$.
	\choice
	{$I=\dfrac{1}{2}$}
	{$I=\dfrac{46}{31}$}
	{$I=\dfrac{17}{11}$}
	{\True $I=\dfrac{3}{2}$}
	\loigiai{
		Ta có: $I=\underset{x\to +\infty}{\mathop{\lim}}\,\left( x+1-\sqrt{x^2-x+2} \right)=\underset{x\to +\infty}{\mathop{\lim}}\,\left( \dfrac{x^2-x^2+x-2}{x+\sqrt{x^2-x+2}}+1 \right)\\
		=\underset{x\to +\infty}{\mathop{\lim}}\,\left( \dfrac{x-2}{x+\sqrt{x^2-x+2}}+1 \right)=\underset{x\to +\infty}{\mathop{\lim}}\,\left( \dfrac{1-\dfrac{2}{x}}{1+\sqrt{1-\dfrac{1}{x}+\dfrac{2}{x^2}}}+1 \right)=\dfrac{3}{2}$.}
\end{ex}

\begin{ex}%[1K5KF-5]
	Cho $f(x)$ là một đa thức thỏa mãn $\lim \limits_{x\to 2} \dfrac{f(x)-15}{x-2}=3$. Tính \[\lim \limits_{x\to 2} \dfrac{f(x)-15}{(x^2-4)\left( \sqrt{2f(x)+6}+3\right)}.\]
	\choice
	{$\dfrac{1}{10}$}
	{$\dfrac{1}{6}$}
	{\True $\dfrac{1}{12}$}
	{$\dfrac{1}{8}$}
	\loigiai{
		Do $\lim \limits_{x\to 2} \dfrac{f(x)-15}{x-2}=3$ và $\lim \limits_{x\to 2} (x-2) = 0$ nên $\lim \limits_{x\to 2} \left( f(x) - 15 \right) = 0 \Rightarrow f(2) = 15$.\\
		Ta có
		\begin{eqnarray*} 
			\lim \limits_{x\to 2} \dfrac{f(x)-15}{(x^2-4)\left( \sqrt{2f(x)+6}+3\right)}
			& = & \lim \limits_{x\to 2} \left[ \dfrac{f(x)-15}{x-2} \cdot \dfrac{1}{(x+2)\left( \sqrt{2f(x)+6}+3\right)}\right] \\
			& = & 3 \cdot \dfrac{1}{4\cdot (\sqrt{2\cdot 15 +6}+3)}=\dfrac{1}{12}.
		\end{eqnarray*}
	}
\end{ex}
\begin{ex}%[1K5GF-5]
	Biết rằng $b>0$, $a+b=5$ và $\lim\limits_{x\to 0}\dfrac{\sqrt[3]{ax+1}-\sqrt{1-bx}}{x}=2$. Khẳng định nào dưới đây là \textbf{sai}?
	\choice
	{$a^2+b^2>10$}
	{\True  $a^2-b^2>6$}
	{ $a-b\geq 0$}
	{$1\le a\le 3$}
	\loigiai{ Ta có
		$$\lim\limits_{x\to 0}\dfrac{\sqrt[3]{ax+1}-\sqrt{1-bx}}{x}=\lim\limits_{x\to 0}\dfrac{\sqrt[3]{ax+1}-1}{x}-\lim\limits_{x\to 0}\dfrac{\sqrt{1-bx}-1}{x}.$$
		Với $L_1=\lim\limits_{x\to 0}\dfrac{\sqrt[3]{ax+1}-1}{x}=\lim\limits_{x\to 0}\dfrac{ax}{x\left(\sqrt[3]{(ax+1)^2}+\sqrt[3]{ax+1}+1\right)}=\dfrac{a}{3}$.\\
		Với $\lim\limits_{x\to 0}\dfrac{\sqrt{1-bx}-1}{x}=\lim\limits_{x\to 0}\dfrac{-bx}{x\left(\sqrt{1-bx}+1\right)}=-\dfrac{b}{2}$.\\
		Từ giả thiết bài toán ta có
		$$L=L_1-L_2=2\Leftrightarrow \dfrac{a}{3}+\dfrac{b}{2}=2\Leftrightarrow 2a+3b=12.$$
		Ta có hệ phương trình $\heva{&a+b=5\\&2a+3b=12}\Leftrightarrow \heva{&a=3\\&b=2.}$\\
		Kiểm tra trực tiếp từng đáp án ta thấy $a^2-b^2>6$ là sai.
	}
\end{ex}
\Closesolutionfile{ans}
% \begin{indapan}{10}
% 	{ans/ans-1K5-2-Dang4}
% \end{indapan}
\begin{dang}{Toán thực tế, liên môn về hàm số liên tục}
\end{dang}
\subsubsection{Ví dụ minh hoạ}
\begin{vd}%[1K5BF-6]
	Tính $\underset{x\to +\infty}{\mathop{\lim}}\,\left( \sqrt{x^2+3x\sqrt{x}}-x+1 \right)$.
	\choice
	{\True $+\infty$}
	{$4$}
	{$-\infty$}
	{$\dfrac{1}{2}$}
	\loigiai{
		Có: $\underset{x\to +\infty}{\mathop{\lim}}\,\left( \sqrt{x^2+3x\sqrt{x}}-x+1 \right)=\underset{x\to +\infty}{\mathop{\lim}}\,\left( \dfrac{x^2+3x\sqrt{x}-x^2}{\sqrt{x^2+3x\sqrt{x}}+x}+1 \right)$\\
		$=\underset{x\to +\infty}{\mathop{\lim}}\,\left( \dfrac{3x\sqrt{x}}{\sqrt{x^2+3x\sqrt{x}}+x}+1 \right)$
		$=\underset{x\to +\infty}{\mathop{\lim}}\,\left(\sqrt{x} \dfrac{3}{\sqrt{1+3\sqrt{x}}+1}+1 \right)
		=+\infty$.}
\end{vd}

\begin{vd}%[1K5BF-6]
	Giới hạn hàm số $\underset{x\to -\infty}{\mathop{\lim}}\,\left(\sqrt{x^2-x\sqrt{|x|}+3}+x\right)$ bằng
	\choice
	{$0$}
	{$\dfrac{1}{2}$}
	{\True $+\infty$}
	{$-\infty $}
	\loigiai{
		Ta có
		$\underset{x\to -\infty}{\mathop{\lim}}\,\left(\sqrt{x^2-x\sqrt{|x|}+3}+x\right)$
		$=\underset{x\to -\infty}{\mathop{\lim}}\,\dfrac{\left(\sqrt{x^2-x\sqrt{|x|}+3}+x\right)\left(\sqrt{x^2-x\sqrt{|x|}+3}-x\right)}{\sqrt{x^2-x\sqrt{|x|}+3}-x}$\\
		$=\underset{x\to -\infty}{\mathop{\lim}}\,\dfrac{-x\sqrt{|x|}+3}{\sqrt{x^2-x\sqrt{|x|}+3}-x}$
		$=\underset{x\to -\infty}{\mathop{\lim}}\,\sqrt{|x|}\dfrac{-1+\dfrac{3}{x\sqrt{|x|}}}{-\sqrt{1-\sqrt{|x|}+\dfrac{3}{x^2}}-1}=+\infty$.}
\end{vd}
\begin{vd}%[1K5BF-6]
	Tìm giới hạn $I=\displaystyle\lim\limits_{x\to-\infty}\left(\sqrt{x^4+4x^3+1}-x^2\right)$
	\choice
	{$I=-4$}
	{$I=1$}
	{\True $I=-2$}
	{$I=-1$}
	\loigiai{
		Ta có \begin{eqnarray*}
			I&=&\displaystyle\lim\limits_{x\to-\infty}\left(\sqrt{x^4+4x^3+1}-x^2\right)=\displaystyle\lim\limits_{x\to-\infty}\dfrac{\left(\sqrt{x^4+4x^3+1}+x^2\right)\left(\sqrt{x^4+4x^3+1}-x^2\right)}{\sqrt{x^4+4x^3+1}+x^2}\\
			&=& \displaystyle\lim\limits_{x\to-\infty}\dfrac{4x^3+1}{\sqrt{x^4+4x^3+1}+x^2}=\displaystyle\lim\limits_{x\to-\infty}\dfrac{4x^3+1}{\sqrt{x^4+4x^3+1}+x^2}\\
			&=&\displaystyle\lim\limits_{x\to-\infty}x\dfrac{4+\dfrac{1}{x^3}}{\sqrt{1+\dfrac{4}{x}+\dfrac{1}{x^4}}+1}=-\infty.
		\end{eqnarray*}
	}
\end{vd}
\begin{vd}%[1K5BF-6]
	Tính $L=\lim\limits_{x \to - \infty} \left( \sqrt{x^2 -7x\sqrt{|x|}+1}- \sqrt{x^2-3x\sqrt{|x|}+2}\right)$.
	\choice
	{\True $L= + \infty$}
	{$L= -\infty$}
	{\True $L= 2$}
	{$L= -2$}
	\loigiai{
		\begin{align*}
			L &= \lim\limits_{x \to - \infty} \dfrac{-4x\sqrt{|x|}-1}{\sqrt{x^2 - 7x\sqrt{|x|} +1}+\sqrt{x^2-3x\sqrt{|x|}+2}}\\
			&=\lim\limits_{x \to - \infty} \dfrac{-4x\sqrt{|x|}-1}{-x \sqrt{1-\dfrac{7}{\sqrt{|x|}}+\dfrac{1}{x^2}}-x \sqrt{1-\dfrac{3}{\sqrt{|x|}}+\dfrac{2}{x^2}}}\\
			&= \lim\limits_{x \to - \infty}\sqrt{|x|} \dfrac{-4-\dfrac{1}{x\sqrt{|x|}}}{- \sqrt{1-\dfrac{7}{\sqrt{|x|}}+\dfrac{1}{x^2}}-\sqrt{1-\dfrac{3}{\sqrt{|x|}}+\dfrac{2}{x^2}}}\\
			&=+\infty.
		\end{align*}
	}
\end{vd}
\begin{vd}%[1K5BF-6]
	Tìm tham số m để $\displaystyle\lim\limits_{x\to +\infty}(\sqrt{{x^3+mx^2}}-x\sqrt{x})=-\infty$.
	\choice
	{$m=0$}
	{$m>0$}
	{\True $m<0$}
	{$m=2$}
	\loigiai{
		Ta có \begin{align*}
			\displaystyle \lim \limits_{x\to +\infty}\left (\sqrt{x^3+mx^2}-x\sqrt{x}\right ) =&\displaystyle \lim \limits_{x\to +\infty}\dfrac{mx^2}{\sqrt{x^3+mx^2}+x\sqrt{x}}\\
			=& \displaystyle \lim \limits_{x\to +\infty}\sqrt{x}\dfrac{m}{\sqrt{1+\dfrac{m}{x}} +1}.  
		\end{align*}	
		Do đó $\displaystyle\lim\limits_{x\to +\infty}(\sqrt{{x^3+mx^2}}-x\sqrt{x})=-\infty\Leftrightarrow m<0$.
	}
\end{vd}
% \subsubsection{Bài tập rèn luyện}
\subsubsection{Câu hỏi trắc nghiệm}
\Opensolutionfile{ans}[ans/ans-1K5-2-Dang5]
\begin{ex}%[1K5BF-6]
	Tính $\underset{x\to +\infty}{\mathop{\lim}}\,\left( \sqrt{x^2+3x\sqrt{x}}-x \right)$.
	\choice
	{\True $+\infty$}
	{$4$}
	{$-\infty$}
	{$\dfrac{1}{2}$}
	\loigiai{
		Ta có $\underset{x\to +\infty}{\mathop{\lim}}\,\left( \sqrt{x^2+3x\sqrt{x}}-x \right)=\underset{x\to +\infty}{\mathop{\lim}}\, \dfrac{x^2+3x\sqrt{x}-x^2}{\sqrt{x^2+3x\sqrt{x}}+x}$\\
		$=\underset{x\to +\infty}{\mathop{\lim}}\, \dfrac{3x\sqrt{x}}{\sqrt{x^2+3x\sqrt{x}}+x} $
		$=\underset{x\to +\infty}{\mathop{\lim}}\,\sqrt{x} \dfrac{3}{\sqrt{1+\dfrac{3}{x}}+1} 
		=+\infty$.}
\end{ex}
\begin{ex}%[1K5KF-6]
	$\lim \limits_{x\to -\infty} \left(\sqrt{4x^2+3x+1}+mx\right)=+\infty$ nếu
	\choice
	{\True $m<2$}
	{$m>2$}
	{$m\ge 2$}
	{$m\le 2$}
	\loigiai{
		Ta có 
		\[ \lim \limits_{x\to -\infty} \left( \sqrt{4x^2+3x+1}+mx\right) 
		= \lim \limits_{x\to -\infty}\left[ x\cdot \left( -\sqrt{4+\dfrac{3}{x}+\dfrac{1}{x^2}}+m\right) \right].\]
		Do $\lim \limits_{x\to -\infty} x = -\infty$ và $\lim \limits_{x\to -\infty} \left( -\sqrt{4+\dfrac{3}{x}+\dfrac{1}{x^2}}+m\right) = -2+m$ nên để \\ $\lim \limits_{x\to -\infty} \left( \sqrt{4x^2+3x+1}+mx\right) = +\infty$ thì $-2+m<0 \Leftrightarrow m<2$.
	}
\end{ex}


\begin{ex}%[1K5KF-6]
	Biết $\displaystyle \lim\limits_{x\to +\infty}\dfrac{(2-a)x-3}{x-\sqrt{{x^2+1}}}=+\infty $ (với $a$ là tham số). Giá trị nhỏ nhất của $P=a^2-2a+4$ là
	\choice
	{$3 $}
	{\True $4 $}
	{$5 $}
	{$1 $}
	\loigiai{
		Ta có $\displaystyle \lim\limits_{x\to +\infty}\dfrac{(2-a)x-3}{x-\sqrt{{x^2+1}}}=\displaystyle \lim\limits_{x\to +\infty}\dfrac{((2-a)x-3)\left(x+\sqrt{x^2+1}\right)}{x^2-x^2-1}=\displaystyle \lim\limits_{x\to +\infty} ((a-2)x+3)\left(x+\sqrt{x^2+1}\right)$
		\begin{itemize}
			\item Với $a=2$, $$\displaystyle \lim\limits_{x\to +\infty} ((a-2)x+3)\left(x+\sqrt{x^2+1}\right)=\displaystyle \lim\limits_{x\to +\infty} 3(x+\sqrt{x^2+1})=+\infty.$$
			\item Với $a>2$ suy ra $a-2>0$, nên $$\displaystyle \lim\limits_{x\to +\infty} ((a-2)x+3)\left(x+\sqrt{x^2+1}\right)=\displaystyle \lim\limits_{x\to +\infty} x\left((a-2)+\dfrac{3}{x}\right)\left(1+\sqrt{1+\dfrac{1}{x^2}}\right)=+\infty.$$
			\item Với $a<2$ suy ra $a-2<0$, nên $$\displaystyle \lim\limits_{x\to +\infty} ((a-2)x+3)\left(x+\sqrt{x^2+1}\right)=\displaystyle \lim\limits_{x\to +\infty} x\left((a-2)+\dfrac{3}{x}\right)\left(1+\sqrt{1+\dfrac{1}{x^2}}\right)=-\infty.$$
		\end{itemize}
		Vậy $a\ge 2$ thì $\displaystyle \lim\limits_{x\to +\infty}\dfrac{(2-a)x-3}{x-\sqrt{{x^2+1}}}=+\infty $.
		Xét hàm số $f(a)=a^2-2a+4$ với $a\ge 2$. Ta có bảng biến thiên sau
		\begin{center}
			\begin{tikzpicture}
				\tkzTabInit[nocadre=false, lgt=1.5, espcl=3]{$a$ /1,$f(a)$ /2}{$2$,$+\infty$}
				%\tkzTabLine{,-,$0$,+,}
				\tkzTabVar{-/ $4$,+/$+\infty $}
			\end{tikzpicture}
		\end{center}
		Suy ra $\min P=4$.
	}
\end{ex}
\begin{ex}%[1K5KF-6]
	Tìm số các số nguyên $m$ thỏa mãn $\displaystyle\lim\limits_{x\rightarrow +\infty} \left(3 \sqrt{mx^2+2x+1}-mx \right)= +\infty$.
	\choice
	{$4$}
	{$10$}
	{$3$}
	{\True $9$}
	\loigiai{
		\begin{itemize}
			\item Với $m=0$ thì $\displaystyle\lim\limits_{x\rightarrow +\infty} \left(3 \sqrt{2x+1} \right)= +\infty$ (thỏa yêu cầu bài toán).
			\item Với $m \neq 0$ ta có 
			$\displaystyle\lim\limits_{x\rightarrow +\infty} \left(3 \sqrt{mx^2+2x+1}-mx \right)= \lim\limits_{x\rightarrow +\infty} x\left(3 \sqrt{m+\dfrac{2}{x}+\dfrac{1}{x^2} }-m \right)= +\infty
			$.\\
			Suy ra 
			$\displaystyle\lim\limits_{x\rightarrow +\infty} \left(3 \sqrt{m+\dfrac{2}{x}+\dfrac{1}{x^2} }-m \right)= 3\sqrt{m}-m>0$, $(m>0)$.\\
			Ta có $3\sqrt{m}-m>0 \Leftrightarrow 3\sqrt{m}>m \Leftrightarrow m^2-9m<0 \Leftrightarrow 0<m<9$.	Do đó $m \in \{1;2;\ldots ;8\}$.
		\end{itemize}
		Vậy $m \in \{0;1;2;\ldots ;8\}$. 
	}
\end{ex}

\begin{ex}%[1K5KF-6]
	Giới hạn $\lim\limits_{x\to+\infty}\left(\sqrt{x^2-3x+1}+x\right)$ bằng
	\choice
	{\True $+\infty$}
	{$-\infty$}
	{$0$}
	{$2$}
	\loigiai{Ta có
		\begin{eqnarray*}
			&&\lim\limits_{x\to+\infty}\left(\sqrt{x^2-3x+1}+x\right)=\lim\limits_{x\to+\infty}\left(x\sqrt{1-\dfrac{3}{x}+\dfrac{1}{x^2}}+x\right)\\
			&=&\lim\limits_{x\to+\infty}\left(x\left(\sqrt{1-\dfrac{3}{x}+\dfrac{1}{x^2}}+1\right)\right)=+\infty.
		\end{eqnarray*}
		Vì $\lim\limits_{x\to+\infty}x=+\infty$ và $\lim\limits_{x\to+\infty}\left(\sqrt{1-\dfrac{3}{x}+\dfrac{1}{x^2}}+1\right)=2$.}
\end{ex}
\begin{ex}%[1K5KF-6]
	Biết $\displaystyle\lim_{x\rightarrow +\infty}\left(\sqrt{x^2+ax\sqrt{|x|}-1}-x\right)=-\infty$. Khi đó giá trị của tham số $a$ là
	\choice
	{\True $a<0$}
	{$a>0$}
	{$a=6$}
	{$a=10$}
	\loigiai{
		$\displaystyle\lim_{x\rightarrow +\infty}\left(\sqrt{x^2+ax\sqrt{|x|}-1}-x\right)=\lim_{x\rightarrow +\infty}\dfrac{x^2+ax\sqrt{|x|}-1-x^2}{\sqrt{x^2+ax\sqrt{|x|}-1}+x}=\lim_{x\rightarrow +\infty}\sqrt{|x|}\dfrac{a-\dfrac{1}{x\sqrt{|x|}}}{\sqrt{1+\dfrac{a}{\sqrt{|x|}}-\dfrac{1}{x^2}}+1}$.\\
		Để  $\displaystyle\lim_{x\rightarrow +\infty}\left(\sqrt{x^2+ax\sqrt{|x|}-1}-x\right)=-\infty\Leftrightarrow a<0$. 
	}
\end{ex}
\begin{ex}%[1K5KF-6]
	Tính $\displaystyle\lim_{x\rightarrow -\infty}\left(\sqrt{7x^2+2x\sqrt{|x|}}+x\sqrt{7}\right)$.
	\choice
	{$0$}
	{$-\dfrac{5\sqrt{7}}{14}$}
	{\True$-\infty$}
	{$+\infty$}
	\loigiai{$\displaystyle\lim_{x\rightarrow -\infty}\left(\sqrt{7x^2+2x\sqrt{|x|}}+x\sqrt{7}\right)=\lim_{x\rightarrow -\infty}\dfrac{2x\sqrt{|x|}}{\sqrt{7x^2+2x\sqrt{|x|}}-x\sqrt{7}}=\lim_{x\rightarrow -\infty}\dfrac{2x\sqrt{|x|}}{-x\sqrt{7+\dfrac2x}-x\sqrt{7}}\\=\lim_{x\rightarrow -\infty}\sqrt{|x|}\dfrac{2}{-\sqrt{7+\dfrac2x}-\sqrt{7}}=-\infty$.}
\end{ex}
\begin{ex}%[1K5KF-6]
	Cho số thực $a$ thỏa mãn $\lim\limits_{x\to -\infty}\left(\sqrt{x^4+5ax^3-1}-x^2\right)=-\infty$. Tìm số thực $a$.
	\choice
	{$(a<0$}
	{$a\in(-10;-5)$}
	{\True $a>0$}
	{$a\in(-3;-1)$}
	\loigiai{
		
		\begin{align*}
			\lim\limits_{x\to -\infty}\left(\sqrt{x^4+5ax^3-1}-x^2\right)
			&=\lim\limits_{x\to -\infty}\dfrac{x^4+5ax^3-1-x^4}{\sqrt{x^4+5ax^3-1}+x^2}\\
			&=\lim\limits_{x\to -\infty}x\dfrac{5a-\dfrac{1}{x^3}}{\sqrt{1+\dfrac{5a}{x}-\dfrac{1}{x^4}}+1}.
		\end{align*}
		Để $\lim\limits_{x\to -\infty}\left(\sqrt{x^4+5ax^3-1}-x^2\right)=-\infty\Leftrightarrow a>0.$
	}
\end{ex}
\begin{ex}%[1K5KF-6]
	Tính $\displaystyle \lim_{x\to+\infty}\left(3x^2+1-\sqrt{9x^4-6x^3+1}\right)$.
	\choice
	{$\dfrac{1}{4}$}
	{$\dfrac{1}{2}$}
	{\True $+\infty$}
	{$-\infty$}
	\loigiai{Vì $x\to+\infty$ nên ta có\\
		$$\begin{aligned}
			&\displaystyle \lim_{x\to+\infty}\left(3x^2+1-\sqrt{9x^4-6x^3+1}\right)\\
			=&\displaystyle \lim_{x\to+\infty} \dfrac{\left(3x^2+1-\sqrt{9x^4-6x^3+1}\right)\left(3x^2+1+\sqrt{9x^4-6x^3+1}\right)}{3x^2+1+\sqrt{9x^4-6x^3+1}}\\
			=&\displaystyle \lim_{x\to+\infty} \dfrac{\left(3x^2+1\right)^2-\left(9x^4-6x^3+1\right)}{3x^2+1+\sqrt{9x^4-6x^2+1}}\\
			=&\displaystyle \lim_{x\to+\infty} \dfrac{x^3\left(6+\dfrac{6}{x}\right)}{x^2\left(3+\dfrac{1}{x^2}+\sqrt{9-6\cdot\dfrac{1}{x^2}+\dfrac{1}{x^4}}\right)}\\
			=&\displaystyle \lim_{x\to+\infty} x\dfrac{6+\dfrac{6}{x}}{3+\dfrac{1}{x}+\sqrt{9-6\cdot\dfrac{1}{x}+\dfrac{1}{x^2}}}\\
			=&+\infty.
		\end{aligned}$$
	} 
\end{ex}
\begin{ex}%[1K5KF-6]
	Tính $\displaystyle \lim_{x\to+\infty}\left(2x-\sqrt{x^2-x+1}\right)$.
	\choice
	{\True $+\infty $}
	{$-\infty $}
	{$\dfrac{1}{2}$}
	{$-\dfrac{1}{2}$}
	\loigiai{Vì ${x\to +\infty}$ nên $\displaystyle \lim_{x\to+\infty}\left(2x-\sqrt{x^2-x+1}\right)=\displaystyle \lim_{x\to+\infty}x\cdot \displaystyle \lim_{x\to+\infty}\left(2-\sqrt{1-\dfrac{1}{x}+\dfrac{1}{x^2}}\right)=+\infty$.
	} 
\end{ex}
\begin{ex}%[1K5KF-6]
	Tìm giới hạn $M=\underset{x\to -\infty}{\lim}\left(\sqrt{x^2-4x\sqrt{|x|}}-\sqrt{x^2-x\sqrt{|x|}}\right)$.
	\choice
	{$M=-\infty$}
	{\True $M=+\infty$}
	{$M=-\dfrac{3}{2}$}
	{$M=\dfrac{1}{2}$}
	\loigiai{
		Ta có
		\begin{align*}
			M&=\underset{x\to -\infty}{\lim}\left(\sqrt{x^2-4x\sqrt{|x|}}-\sqrt{x^2-x\sqrt{|x|}}\right)\\
			&=\underset{x\to -\infty}{\lim}\dfrac{-3x\sqrt{|x|}}{\sqrt{x^2-4x\sqrt{|x|}}+\sqrt{x^2-x\sqrt{|x|}}}\\
			&=\underset{x\to -\infty}{\lim}\sqrt{|x|}\dfrac{-3}{-\sqrt{1-\dfrac{4}{\sqrt{|x|}}}-\sqrt{1-\dfrac{1}{\sqrt{|x|}}}}\\
			&=+\infty.
		\end{align*}
	}
\end{ex}


\begin{ex}%[1K5KF-6]
	Giá trị của $\lim \limits_{x \to  - \infty } \left( \sqrt {x^2 + 5}  - x \right)$ là
	\choice
	{\True $+\infty $}
	{$-\infty $}
	{$1$}
	{$0$}
	\loigiai{
		Ta có $$\lim\limits_{x \to -\infty} \left(\sqrt{x^2+5}-x\right)=\lim\limits_{x \to -\infty} x\left(-\sqrt{1+\dfrac{5}{x^2}}-1 \right)=+\infty.$$
	}
\end{ex}
\begin{ex}%[1K5KF-6]
	Giá trị của $\lim \limits_{x \to  + \infty } \left( \sqrt {x^2 + 5x\sqrt{x}}  - x \right)$ là
	\choice
	{\True $+\infty $}
	{$-\infty $}
	{$1$}
	{$0$}
	\loigiai{
		Ta có $$\lim\limits_{x \to +\infty} \left(\sqrt{x^2+5x\sqrt{x}}-x\right)=\lim\limits_{x \to +\infty} \dfrac{5x\sqrt{x}}{\sqrt{x^2+5x\sqrt{x}}+x}=\lim\limits_{x \to +\infty}\sqrt{x}\dfrac{5}{\sqrt{1+\dfrac{5}{\sqrt{x}}}+1}=+\infty.$$
	}
\end{ex}
\Closesolutionfile{ans}
% \begin{indapan}{10}
% 	{ans/ans-1K5-2-Dang5}
% \end{indapan}

\begin{dang}{Giới hạn một bên}	
\end{dang}
\subsubsection{Ví dụ}
\begin{vd}%[DCHT Toán 11 - KNTT- Phạm Tuấn]%[1K5BF-7] 
	Tính giới hạn $\lim\limits _{x \rightarrow 2^{-}} \dfrac{x^2-3 x+2}{\sqrt{2-x}}$.
	\dapso{$\lim\limits _{x \rightarrow 2^{-}} \dfrac{x^2-3 x+2}{\sqrt{2-x}} =0$}
	\loigiai{
		Ta có 
		\[
		\lim\limits _{x \rightarrow 2^{-}} \dfrac{x^2-3 x+2}{\sqrt{2-x}} = \lim\limits _{x \rightarrow 2^{-}} \dfrac{(2-x)(1-x)}{\sqrt{2-x}} = \lim\limits _{x \rightarrow 2^{-}} (1-x)\sqrt{2-x}  = 0.
		\]
	}
\end{vd}

\begin{vd}%[DCHT Toán 11 - KNTT- Phạm Tuấn]%[1K5BF-7] 
	Tính giới hạn $\lim\limits _{x \rightarrow (-1)^{+}} \dfrac{x^3+1}{x^3+2x^2+x}$. 
	\dapso{$\lim\limits _{x \rightarrow (-1)^{+}} \dfrac{x^3+1}{x^3+2x^2+x}= -\infty$}
	\loigiai{
		Ta có 
		\[
		\lim\limits _{x \rightarrow (-1)^{+}} \dfrac{x^3+1}{x^3+2x^2+x} = \lim\limits _{x \rightarrow (-1)^{+}} \dfrac{(x+1)(x^2-x+1)}{x(x+1)^2} = \lim\limits _{x \rightarrow (-1)^{+}} \dfrac{x^2-x+1}{x(x+1)}.
		\]
		Khi $x \to (-1)^+$ thì $\heva{&x+1 \to 0\\&x+1 >0\\& \dfrac{x^2-x+1}{x} \to -3}$ suy ra $\lim\limits _{x \rightarrow (-1)^{+}} \dfrac{x^2-x+1}{x(x+1)} = -\infty.$ \\
		Vậy $\lim\limits _{x \rightarrow (-1)^{+}} \dfrac{x^3+1}{x^3+2x^2+x}= -\infty$. 
	}
\end{vd}

\begin{vd}%[DCHT Toán 11 - KNTT- Phạm Tuấn]%[1K5BF-7] 
	Cho hàm số $f(x) = \heva{&\sqrt{9-x^2} && \text{ khi } -3 \leq x < 3\\& 1 && \text{ khi } x=3\\& \sqrt{x^2-9} && \text{ khi } x>3.}$ \\
	Hàm số $f(x)$ có giới hạn khi $x \to 3$ hay không?
	\dapso{$\lim\limits _{x \rightarrow 3} f(x) =0$}
	\loigiai{
		Ta có $\lim\limits _{x \rightarrow 3^{-}} f(x) = \lim\limits _{x \rightarrow 3^{-}} \sqrt{9-x^2} =0$; $\lim\limits _{x \rightarrow 3^{+}} f(x) = \lim\limits _{x \rightarrow 3^{+}} \sqrt{x^2-9} =0$. \\
		Suy ra $\lim\limits _{x \rightarrow 3^{-}} f(x) = \lim\limits _{x \rightarrow 3^{+}} f(x) =0$. \\
		Vậy $\lim\limits _{x \rightarrow 3} f(x) =0$.
	}
\end{vd}

\begin{vd}%[DCHT Toán 11 - KNTT- Phạm Tuấn]%[1K5BF-7] 
	Ta gọi phần nguyên của số thực $x$ là số nguyên lớn nhất không lớn hơn $x$ và kí hiệu nó là $[x]$. 
	Ví dụ $[5]=5 $; $[3,12]=3 $; $[-2{,}725]=-3$. \\
	Tìm $\lim\limits _{x \rightarrow 1^{-}} [x]$ và  $\lim\limits _{x \rightarrow 1^{+}} [x]$. Giới hạn $\lim\limits _{x \rightarrow 1} [x]$ có tồn tại hay không?
	\dapso{$\lim\limits _{x \rightarrow 1^{-}} [x] =0$; $\lim\limits _{x \rightarrow 1^{+}} [x] =1$}
	\loigiai{
		Ta có $\lim\limits _{x \rightarrow 1^{-}} [x] =0$; $\lim\limits _{x \rightarrow 1^{+}} [x] =1$. \\
		Suy ra $\lim\limits _{x \rightarrow 1^{+}} [x] \neq  \lim\limits _{x \rightarrow 1^{-}} [x]$. \\
		Vậy giới hạn $\lim\limits _{x \rightarrow 1} [x]$ không tồn tại.
	}
\end{vd}

\begin{vd}%[DCHT Toán 11 - KNTT- Phạm Tuấn]%[1K5BF-7] 
	Cho hàm số $f(x) = \heva{& \dfrac{x-\sqrt{2x}}{4-x^2} && \text{ khi } x < 2\\& x^2-x+m && \text{ khi } x \geq  2}$  ($m$ là tham số). \\
	Tìm $m$ để hàm số $f(x)$ có giới hạn khi $x \to 2$.
	\dapso{$m= -\dfrac{17}{8}$}
	\loigiai{
		Ta có 
		\begin{align*}
			&\lim\limits _{x \rightarrow 2^{-}}  f(x) = \lim\limits _{x \rightarrow 2^{-}} \dfrac{x-\sqrt{2x}}{4-x^2} = \lim\limits _{x \rightarrow 2^{-}}  \dfrac{x(x-2)}{-(x-2)(x+2)(x+\sqrt{2x})} = -\dfrac{1}{8}; \\
			& \lim\limits _{x \rightarrow 2^{+}}  f(x)  = \lim\limits _{x \rightarrow 2^{+}}  (x^2-x+m) = 2+m.
		\end{align*}
		Hàm số $f(x)$ có giới hạn khi $x \to 2$ khi và chỉ khi 
		$$\lim\limits _{x \rightarrow 2^{-}}  f(x)  = \lim\limits _{x \rightarrow 2^{+}}  f(x) \Leftrightarrow -\dfrac{1}{8}=2+m \Leftrightarrow m= -\dfrac{17}{8}. $$
	}
\end{vd}
% \subsubsection{Bài tập rèn luyện}
% % \centerline{\fcolorbox{red}{yellow!50}{\bf {BÀI TẬP TỰ LUẬN}}}
% \begin{bt}%[DCHT Toán 11 - KNTT- Phạm Tuấn]%[1K5BF-7] 
% 	Tính giới hạn $\lim\limits _{x \rightarrow 1^{-}} \dfrac{-x^2-x+2}{x^2-3x^2+3x-1}$. 
% 	\dapso{$\lim\limits _{x \rightarrow 1^{-}} \dfrac{-x^2-x+2}{x^2-3x^2+3x-1} = -\infty$}
% 	\loigiai{
% 		Ta có $\lim\limits _{x \rightarrow 1^{-}} \dfrac{-x^2-x+2}{x^2-3x^2+3x-1} = \lim\limits _{x \rightarrow 1^{-}}  \dfrac{-(x-1)(x+2)}{(x-1)^3} = \lim\limits _{x \rightarrow 1^{-}}   \dfrac{-x-2}{(x-1)^2}$.  \\
% 		Khi $x \to 1^-$ thì $\heva{& (x-1)^2  \to 0\\& (x-1)^2 >0\\& -x-2\to -3}$ suy ra $\lim \limits _{x \rightarrow 1^{-}}   \dfrac{-x-2}{(x-1)^2}= -\infty.$\\
% 		Vậy $\lim\limits _{x \rightarrow 1^{-}} \dfrac{-x^2-x+2}{x^2-3x^2+3x-1} = -\infty$. 
% 	}
% \end{bt}

% \begin{bt}%[DCHT Toán 11 - KNTT- Phạm Tuấn]%[1K5BF-7] 
% 	Cho hàm số $f(x)=\heva{& \dfrac{x^2-1}{1-x} \,&\text{ khi }x < 1\\& x^3-2x^2+3\,&\text{ khi }x \geq  1}$. Tính $\lim\limits_{x\to 1^-}f(x)$ và $\lim\limits_{x\to 1^+}f(x)$.
% 	\dapso{$\lim\limits_{x\to 1^-}f(x)=-2$; $\lim\limits_{x\to 1^+}f(x)=2$}
% 	\loigiai{
% 		Ta có $\lim\limits_{x\to 1^-}f(x) = \lim\limits_{x\to 1^-}  \dfrac{x^2-1}{1-x} =   \lim\limits_{x\to 1^-}  -(x+1) = -2$; 
% 		$\lim\limits_{x\to 1^+} f(x) = \lim\limits_{x\to 1^+} (x^3-2x^2+3) = 2 $.
% 	}
% \end{bt}

% \begin{bt}%[DCHT Toán 11 - KNTT- Phạm Tuấn]%[1K5BF-7] 
% 	Tính giới hạn $\lim\limits _{x \rightarrow 2^{-}} \dfrac{|x^2-3x+2|}{x^2-4}$. 
% 	\dapso{$\lim\limits _{x \rightarrow 2^{-}} \dfrac{|x^2-3x+2|}{x^2-4} =  -\dfrac{1}{4}$}
% 	\loigiai{
% 		Khi $x \to 2^-$ thì $x^2-3x+2 <0$ nên 
% 		\[
% 		\lim\limits _{x \rightarrow 2^{-}} \dfrac{|x^2-3x+2|}{x^2-4} = \lim\limits _{x \rightarrow 2^{-}} \dfrac{-x^2+3x-2}{x^2-4} = \lim\limits _{x \rightarrow 2^{-}} \dfrac{1-x}{x+2} = -\dfrac{1}{4}.
% 		\]
% 	}
% \end{bt}

% \begin{bt}%[DCHT Toán 11 - KNTT- Phạm Tuấn]%[1K5BF-7] 
% 	Cho hàm số $f(x) = \heva{&\dfrac{1-\sqrt{x}}{x^2-2x+1} \text{ khi  } x >1\\& \dfrac{2x}{x^3-2x+1}  \text{ khi  } x <1}$. Tính $\lim\limits _{x \rightarrow 1} f(x)$.
% 	\dapso{$\lim\limits _{x \rightarrow 1} f(x)=-\infty$}
% 	\loigiai{
% 		Xét $\lim\limits _{x \rightarrow 1^{+}}  f(x) = \lim\limits _{x \rightarrow 1^{+}} \dfrac{1-\sqrt{x}}{x^2-2x+1} =\lim\limits _{x \rightarrow 1^{+}} \dfrac{1-x}{(x-1)^2(\sqrt{x}+1)} = \lim\limits _{x \rightarrow 1^{+}} \dfrac{1}{(1-x)(\sqrt{x}+1)}$. \\
% 		Khi $x \to 1^+$ thì $\heva{&1-x <0\\&1-x \to 0\\&\sqrt{x}+1 \to 2}$, suy ra $\lim\limits _{x \rightarrow 1^{+}}  f(x) = -\infty$. \\
% 		Xét $\lim\limits _{x \rightarrow 1^{-}}  f(x) = \lim\limits _{x \rightarrow 1^{-}}  \dfrac{2x}{x^3-2x+1} = \lim\limits _{x \rightarrow 1^{-}} \dfrac{2x}{(x-1)(x^2+x-1)}$. \\
% 		Khi $x \to 1^-$ thì $\heva{&x-1 <0\\&x-1 \to 0\\&x^2+x-1 \to 1}$, suy ra $\lim\limits _{x \rightarrow 1^{-}}  f(x) = -\infty$. \\
% 		Suy ra $\lim\limits _{x \rightarrow 1^{+}}  f(x) =\lim\limits _{x \rightarrow 1^{-}}  f(x) = -\infty$. Vậy $\lim\limits _{x \rightarrow 1} f(x)=-\infty$.
% 	}
% \end{bt}

% \begin{bt}%[DCHT Toán 11 - KNTT- Phạm Tuấn]%[1K5BF-7] 
% 	Cho hàm số $f(x) = |x^2-2x-3|$. Tính các giới hạn $\lim\limits _{x \rightarrow 0^{-}} \dfrac{f(x+3)- f(3)}{x}$ và $\lim\limits _{x \rightarrow 0^{+}} \dfrac{f(x+3)- f(3)}{x}$. 
% 	\dapso{$\lim\limits _{x \rightarrow 0^{-}} \dfrac{f(x+3)- f(3)}{x}=-4$;  $\lim\limits _{x \rightarrow 0^{+}} \dfrac{f(x+3)- f(3)}{x}=4$}
% 	\loigiai{
% 		Ta có $\lim\limits _{x \rightarrow 0^{-}} \dfrac{f(x+3)- f(3)}{x} = \lim\limits _{x \rightarrow 0^{-}} \dfrac{|(x+3)^2-2(x+3)-3|-0}{x} = \lim\limits _{x \rightarrow 0^{-}} \dfrac{|x(x+4)|}{x}$. \\
% 		Khi $x \to 0^-$ thì $x<0$, suy ra  $\lim\limits _{x \rightarrow 0^{-}} \dfrac{|x(x+4)|}{x} = \lim\limits _{x \rightarrow 0^{-}} -(x+4) = -4$. \\
% 		Ta có $\lim\limits _{x \rightarrow 0^{+}} \dfrac{f(x+3)- f(3)}{x} = \lim\limits _{x \rightarrow 0^{+}} \dfrac{|(x+3)^2-2(x+3)-3|-0}{x} = \lim\limits _{x \rightarrow 0^{+}} \dfrac{|x(x+4)|}{x}$. \\
% 		Khi $x \to 0^+$ thì $x>0$, suy ra  $\lim\limits _{x \rightarrow 0^{+}} \dfrac{|x(x+4)|}{x} = \lim\limits _{x \rightarrow 0^{+}} (x+4) = 4$.
% 	}
% \end{bt}

% \begin{bt}%[DCHT Toán 11 - KNTT- Phạm Tuấn]%[1K5KF-7] 
% 	Tìm $m$ để hàm số $f(x) = \heva{&\sin \dfrac{1}{2x} && \text{ khi } x <0\\& x^2+m && \text{ khi } x \geq 0}$ có giới hạn khi $x\to 0$.
% 	\dapso{Không tồn tại $m$}
% 	\loigiai{
% 		Ta có $\lim\limits _{x \rightarrow 0^{+}} f(x) = \lim\limits _{x \rightarrow 0^{+}} (x^2+m)=m$. \\
% 		Xét $\lim\limits _{x \rightarrow 0^{-}} f(x) = \lim\limits _{x \rightarrow 0^{+}} \sin \dfrac{1}{2x}$. \\
% 		Chọn dãy số $x_n = -\dfrac{2}{n\pi }$. Dễ thấy $x_n<0$ và $\lim \limits_{n \to +\infty}x_n =0$. \\
% 		Ta có $\lim \limits_{n \to +\infty}\sin \dfrac{1}{2x} = \lim \limits_{n \to +\infty}\sin (-n\pi ) =0$. \\
% 		Chọn dãy số $x_n = -\dfrac{2}{\frac{\pi}{2}+ n2\pi} $. Dễ thấy $x_n<0$ và $\lim \limits_{n \to +\infty}x_n =0$. \\
% 		Ta có $\lim \limits_{n \to +\infty}\sin \dfrac{1}{2x} = \lim \limits_{n \to +\infty}\sin (-\frac{\pi}{2}- n2\pi ) =-1$.  \\
% 		Suy ra $\lim\limits _{x \rightarrow 0^{-}} f(x)$ không tồn tại. \\
% 		Vậy không tồn tại $m$ để $f(x)$ có giới hạn khi $x\to 0$.
% 	}
% \end{bt}

% \begin{bt}%[DCHT Toán 11 - KNTT- Phạm Tuấn]%[1K5BF-7] 
% 	Cho hàm số $f(x)=\heva{& \dfrac{1}{x-1} - \dfrac{3}{x^3-1} \,&\text{ nếu }x > 1\\& mx+2\,&\text{ nếu }x \geq  1}$.  \\
% 	Với giá trị nào của tham số $m$ thì hàm số $f(x)$ có giới hạn khi $x \rightarrow 1$? Tìm giới hạn này.
% 	\dapso{$m=-1$; $\lim\limits _{x \rightarrow 1} f(x)=1$}
% 	\loigiai{
% 		Ta có
% 		\begin{align*}
% 			\lim\limits  _{x \rightarrow 1^{+}} f(x) &=\lim\limits  _{x \rightarrow 1^{+}}\left(\frac{1}{x-1}-\frac{3}{x^3-1}\right)=\lim\limits  _{x \rightarrow 1^{+}} \frac{x^2+x-2}{(x-1)\left(x^2+x+1\right)} \\
% 			&=\lim\limits  _{x \rightarrow 1^{+}} \frac{(x-1)(x+2)}{(x-1)\left(x^2+x+1\right)}=\lim\limits  _{x \rightarrow 1^{+}} \frac{x+2}{x^2+x+1}=1 .
% 		\end{align*}
% 		$\lim\limits _{x \rightarrow 1^{-}} f(x)=\lim\limits _{x \rightarrow 1^{-}}(m x+2)=m+2$. \\
% 		$f(x)$ có giới hạn khi $x \rightarrow 1 \Leftrightarrow m+2=1 \Leftrightarrow m=-1$. Khi đó $\lim\limits _{x \rightarrow 1} f(x)=1$.
% 	}
% \end{bt}

% \begin{bt}%[DCHT Toán 11 - KNTT- Phạm Tuấn]%[1K5BF-7] 
% 	Cho hàm số $f(x) = \heva{&x\cos \dfrac{1}{x} && \text{ khi } x <0\\& \sin x^2 + m  && \text{ khi } x \geq 0.}$ \\
% 	Tìm $m$ để hàm số $f(x)$ có giới hạn khi $x \to 0$.
% 	\dapso{$m=0$}
% 	\loigiai{
% 		Xét $\lim\limits _{x \rightarrow 0^{-}} f(x) = \lim\limits _{x \rightarrow 0^{-}}  x\cos \dfrac{1}{x}$. \\
% 		Ta có $0\leq  |x\cos \dfrac{1}{x}| \leq |x|$ và $\lim\limits _{x \rightarrow 0^{-}} |x| =0$. Suy ra $\lim\limits _{x \rightarrow 0^{-}}  x\cos \dfrac{1}{x} =0$. \\
% 		Ta lại có $\lim\limits _{x \rightarrow 0^{+}} f(x) = \lim\limits _{x \rightarrow 0^{-}}  (\sin x^2 + m) = m$. \\
% 		$f(x)$ có giới hạn khi $x \rightarrow 0$ khi và chỉ khi 
% 		\[
% 		\lim\limits _{x \rightarrow 0^{-}}  f(x) = \lim\limits _{x \rightarrow 0^{+}}  f(x)  \Leftrightarrow m=0.
% 		\]
% 	}
% \end{bt}
\subsubsection{Câu hỏi trắc nghiệm}
\Opensolutionfile{ans}[ans/ans-1K5-2-Dang6]
\begin{ex}%[DCHT Toán 11 - KNTT- Phạm Tuấn]%[1K5BF-7]
	Tính giới hạn $\lim\limits_{x \to(-2)^{-}} \dfrac{3+2 x}{x+2}$.
	\choice
	{$-\infty$}
	{$2$}
	{\True $+\infty$}
	{$\dfrac{3}{2}$}
	\loigiai{
		Khi $x \to (-2)^{-}$ thì $\heva{& 3+3x\to -1\\&x+2\to 0 \\&x+2<0.}$ \\
		Suy ra  $\lim\limits_{x \to(-2)^{-}} \dfrac{3+2 x}{x+2}=+\infty$.
	}
\end{ex}

\begin{ex}%[DCHT Toán 11 - KNTT- Phạm Tuấn]%[1K5BF-7]
	Cho hàm số $f(x)=\heva{&2x^2-2\,&\text{ khi }x\ge 6\\&x-2\,&\text{ khi }x<6}$. Tính $\lim\limits_{x\to 6^-}f(x)$ bằng
	\choice
	{$2$}
	{$5$}
	{$1$}
	{\True $4$}
	\loigiai{
		Ta có $\lim\limits_{x\to 6^-}f(x)=\lim\limits_{x\to 6^-}(x-2)=6-2=4$.
	}
\end{ex}


\begin{ex}%[DCHT Toán 11 - KNTT- Phạm Tuấn]%[1K5BF-7]
	$\displaystyle \lim \limits_{x \rightarrow 5^+} \dfrac{|10-2x|}{x^2-6x+5}$ là
	\choice
	{\True $\dfrac{1}{2}$}
	{$0$}
	{$+\infty$}
	{$- \dfrac{1}{2}$}
	\loigiai{
		Ta có $\displaystyle \lim \limits_{x \rightarrow 5^+} \dfrac{|10-2x|}{x^2-6x+5} 
		= \lim \limits_{x \rightarrow 5^+} \dfrac{2x-10}{(x-1)(x-5)} 
		= \lim \limits_{x \rightarrow 5^+} \dfrac{2}{x-1} = \dfrac{1}{2}$.
	}
\end{ex}

\begin{ex}%[DCHT Toán 11 - KNTT- Phạm Tuấn]%[1K5BF-7]
	Tính $\lim\limits _{x \to 2^{+}}\dfrac{|2-x|}{x^{2}-x-2}$. 
	\choice
	{$+\infty$}
	{$0$}
	{$-\dfrac{1}{3}$}
	{\True $\dfrac{1}{3}$}
	\loigiai{
		Vì $x \to 2^+$ nên $x>2$. Do đó $\lim\limits _{x \to 2^{+}}\dfrac{|2-x|}{x^{2}-x-2} = \lim\limits _{x \to 2^{+}}\dfrac{x-2}{(x-2)(x+1)} = \lim\limits _{x \to 2^{+}}\dfrac{1}{x+1}=\dfrac{1}{3}$.
	}
\end{ex}

\begin{ex}%[DCHT Toán 11 - KNTT- Phạm Tuấn]%[1K5BF-7]
	Trong các giới hạn sau, giới hạn nào không tồn tại?
	\choice
	{$\lim\limits_{x \to \infty} \dfrac{2x + 1}{x^2 + 1}$}
	{$\lim\limits_{x \to 0} \dfrac{x}{\sqrt{x} + 1}$}
	{$\lim\limits_{x \to 1}\dfrac{x}{(x + 1)^2}$}
	{\True $\lim\limits_{x \to 0} \dfrac{1}{x}$}	
	\loigiai{
		$\lim\limits_{x \to 0^+} = \dfrac{1}{x} = +\infty; \lim\limits_{x \to 0^-} = \dfrac{1}{x} = -\infty $ nên giới hạn không tồn tại.
	}
\end{ex}

\begin{ex}%[DCHT Toán 11 - KNTT- Phạm Tuấn]%[1K5BF-7]
	Gọi $a$ là số thực để hàm số $f(x)=\heva{ & x^2+ax+2 & & \text{khi} \ x>2 \\ & 2x^2-x+1 & & \text{khi} \ x\leqslant2}$ có giới hạn khi $x\to2$. Hãy chọn hệ thức đúng.
	\choice
	{$2a^2+3a+1=0$}
	{$a^2-3a+2=0$}
	{\True $4a^2-1=0$}
	{$a^2-4=0$}
	\loigiai
	{
		Ta có 
		\[\heva{&\lim\limits_{x\to2^+}f(x) = \lim\limits_{x\to2^+}\left(x^2+ax+2\right) = 2a+6 \\&\lim\limits_{x\to2^-}f(x) = \lim\limits_{x\to2^-} \left(2x^2 - x  + 1\right)=7.}\]
		Để hàm số có giới hạn khi $x\to 2$ thì \[\lim\limits_{x\to2^+}f(x) = \lim\limits_{x\to2^-}f(x) \Leftrightarrow 2a + 6 = 7 \Leftrightarrow a = \dfrac{1}{2}.\]
		Khi đó $4a^2 - 1=0$ là hệ thức đúng.
	}
\end{ex}

\begin{ex}%[DCHT Toán 11 - KNTT- Phạm Tuấn]%[1K5BF-7]
	Cho hàm số $f(x)=\heva{&\dfrac{x^3-3 x^2+2}{x-1}\,\, \text { nếu } \,\, x>1 \\& a x+3 \,\, \text { nếu } \,\,x \leq 1}$. Tìm $a$ để $\lim\limits_{x \to 1} f(x)$ tồn tại.
	\choice
	{$a=6$}
	{$a=1$}
	{$a=0$}
	{\True $a=-6$}
	\loigiai{
		$\lim\limits_{x \to 1^+} f(x)=\lim\limits_{x \to 1^+} \dfrac{x^3-3 x^2+2}{x-1}=\lim\limits_{x \to 1+} \dfrac{(x-1)(x^2-2x-2)}{x-1}=\lim\limits_{x \to 1^+}(x^2-2x-2)=-3$.\\
		$\lim\limits_{x \to 1^-} f(x)=\lim\limits_{x \to 1^-} (ax+3)=3+a$.\\
		Giới hạn $\lim\limits_{x \to 1} f(x)$ tồn tại  khi $3+a=-3\Rightarrow a=-6$. 	
	} 
\end{ex}

\begin{ex}%[DCHT Toán 11 - KNTT- Phạm Tuấn]%[1K5BF-7]
	Cho hàm số $f(x)=\heva{&\dfrac{\left|x-1\right|}{x-1}&\text{ khi }& x<1\\&x+2+a &\text{ khi } &1\leq x\leq 3\\ & \dfrac{x^2-81}{\sqrt{x}-3} &\text{ khi }& x>3}$. Tìm tất cả giá trị của tham số $a$ để hàm số có giới hạn tại $x=3$.
	\choice
	{$a=12\left(3+\sqrt{3}\right)$}
	{$a=12\left(3-\sqrt{3}\right)$}
	{\True $a=12\left(3+\sqrt{3}\right)-5$}
	{$a=12\left(3-\sqrt{3}\right)+5$}
	\loigiai{Với $1\leq x\leq 3$ thì $f(x)=x+2+a$ nên $\lim\limits_{x \to 3^-}f(x)=\lim\limits_{x \to 3^-} \left(x+2-a\right)=5+a$.\\
		Với $x>3$ thì $f(x)=\dfrac{x^2-81}{\sqrt{x}-3}=\left(\sqrt{x}+3\right)\left(x+9\right)$ nên $\lim\limits_{x \to 3^+} f(x)=\lim\limits_{x \to 3^+}\left(\sqrt{x}+3\right)\left(x+9\right) =12\left(3+\sqrt{3}\right)$.\\
		Do đó, để hàm số có giới hạn tại $x=3$ thì $\lim\limits_{x \to 3^-}f(x)=\lim\limits_{x \to 3^+}f(x) \Leftrightarrow a=12\left(3+\sqrt{3}\right)-5$.
	}
\end{ex}
\Closesolutionfile{ans}
% \begin{indapan}{10}
% 	{ans/ans-1K5-2-Dang6}
% \end{indapan}

\begin{dang}{Toán thực tế, liên môn về giới hạn hàm số}
\end{dang}
\subsubsection{Ví dụ}
\begin{vd}%[DCHT Toán 11 - KNTT- Phạm Tuấn]%[1K5YF-8]
	Chiều dài một loài động vật nhỏ được tính theo công công thức $h(t)=\dfrac{300}{1+9 \cdot  (0{,}8)^t}$ mm, trong đó $t$ số ngày sau khi sinh của loài động vật đó. Tính chiều dài cuối cùng của nó (chiều dài khi $t \to +\infty$).
	\dapso{$300$ mm}
	\loigiai{
		Ta có $\displaystyle \lim \limits_{t \to +\infty } \dfrac{300}{1+9 \cdot  (0{,}8)^t} = 300$. \\
		Vậy chiều dài cuối cùng của loài động vật  là $300$ mm. 
	}
\end{vd}


\begin{vd}%[DCHT Toán 11 - KNTT- Phạm Tuấn]%[1K5BF-8] 
	Theo thuyết tương đối, khối lượng $m$ của một hạt phụ thuộc vào vận tốc $v$ của nó, theo công thức
	$$
	m=\frac{m_0}{\sqrt{1-\dfrac{v^2}{c^2}}}
	$$
	trong đó $m_0$ là khối lượng khi hạt đứng yên và $c$ là tốc độ ánh sáng. Tìm giới hạn của khối lượng khi $v$ tiến đến $c^{-}$.
	\dapso{$\lim\limits _{v \to c^-} \frac{m_0}{\sqrt{1-\dfrac{v^2}{c^2}}} = +\infty$}
	\loigiai{
		Với $m_0 =0$ thì $\displaystyle \lim_{v \to c^-} =0$. \\
		Với $m_0 \neq 0$. \\
		Khi $c \to c^-$ thì $\heva{&\sqrt{1-\dfrac{v^2}{c^2}} \to 0\\&\sqrt{1-\dfrac{v^2}{c^2}} >0}$ suy ra $\displaystyle \lim_{v \to c^-} \frac{m_0}{\sqrt{1-\dfrac{v^2}{c^2}}} = +\infty$. \\
		Vậy nếu một hạt có khối lượng nghỉ khác $0$ thì khối lượng của hạt sẽ lớn vô cùng khi vận tốc tiến gần vận tốc ánh sáng.
	}
\end{vd}

\begin{vd}%[DCHT Toán 11 - KNTT- Phạm Tuấn]%[1K5BF-8] 
	Một chất điểm chuyển động thẳng với phương trình $s(t)$. Khi đó vận tốc tức thời tại thời điểm $t_0$ được định nghĩa là $\displaystyle \lim \limits_{\Delta t} \dfrac{s(t_0+ \Delta t) - s(t_0)}{\Delta t}$. Tính vận tốc tức thời của chất điểm với phương trình chuyển động $s(t) = 3t^2-2t+3$ ($s(t)$ có đơn vị là m, $t$ đơn vị là giây), tại thời điểm $t=4$ giây. 
	\dapso{$v=22  \mathrm{~m/s}$}
	\loigiai{
		Vận tốc tức thời của chất điểm tại thời điểm $t=4$ giây là
		\begin{align*}
			\lim \limits_{\Delta t \to 0} \dfrac{s(4+\Delta t) - s(4)}{\Delta t} &=  \lim \limits_{\Delta t \to 0} \dfrac{3(4+\Delta t)^2-2(4+\Delta t)+3 - 43}{\Delta t} \\
			& = \lim \limits_{\Delta t \to 0} \dfrac{3 (\Delta t)^2+ 22 \Delta t}{\Delta t} = 22  \mathrm{~m/s}.
		\end{align*}
	}
\end{vd}

\begin{vd}%[DCHT Toán 11 - KNTT- Phạm Tuấn]%[1K5BF-8] 
	Số lượng đơn vị hàng tồn kho trong một công ty  được cho bởi
	$$
	N(t)=200\left(3 \left [\frac{t+3}{3}  \right ]-t\right)
	$$
	trong đó $t$ là thời gian tính bằng ngày, $[x]$ là số nguyên lớn nhất không vượt quá $x$ (ví dụ $[-1{,}5]=-2$, $[8{,}8] = 8$).
	\begin{enumerate}
		\item Tính $\displaystyle \lim_{t \to 55^+} N(t)$.
		\item  Tính $\displaystyle \lim_{t \to 201^-} N(t)$.
	\end{enumerate}
	\dapso{$\displaystyle \lim_{t \to 55^+} N(t) = 400$; $\displaystyle \lim_{t \to 201^-} N(t) =0$}
	\loigiai{
		\begin{enumerate}
			\item 
			Khi $t \to 55^+$, ta có $\left [\dfrac{t+3}{3}  \right ] = 19$. \\
			Suy ra  $\displaystyle \lim_{t \to 55^+} N(t) =\lim_{t \to 55^+}  200\left(3 \left [\frac{t+3}{3}  \right ]-t\right) =200(3 \cdot 19 - 55) = 400$.
			\item  
			Khi $t \to 201^-$, ta có $\left [\dfrac{t+3}{3}  \right ] = 201$. \\
			Suy ra  $\displaystyle \lim_{t \to 201^-} N(t) = \lim_{t \to 201^-}  200\left(3 \left [\frac{t+3}{3}  \right ]-t\right)= 200 \cdot 0 = 0$.
		\end{enumerate}
	}
\end{vd}


\begin{vd}%[DCHT Toán 11 - KNTT- Phạm Tuấn]%[1K5BF-8] 
	Một chất điểm chuyển động thẳng với vận tốc $v(t)$. Khi đó gia tốc tức thời tại thời điểm $t_0$ được định nghĩa là $\displaystyle \lim \limits_{\Delta t \to 0} \dfrac{v(t_0+ \Delta t) - v(t_0)}{\Delta t}$. Một chất điểm chuyển động với vận tốc $v(t) = 0{,}1t^2-0{,}4t+1$ (m/s), tính gia tốc tức thời tại thời điểm $t=8$ giây. 
	\dapso{$1{,}2$ $\mathrm{m/s^2}$}
	\loigiai{
		Gia tốc tức thời của chất điểm tại thời điểm $t=8$ giây là
		\begin{align*}
			\lim \limits_{\Delta t \to 0} \dfrac{v(8+\Delta t) - v(8)}{\Delta t} &=  \lim \limits_{\Delta t \to 0} \dfrac{0{,}1(8+\Delta t)^2-0{,}4(8+\Delta t)+1- 4{,}2}{\Delta t} \\
			& = \lim \limits_{\Delta t \to 0} \dfrac{0{,}1 (\Delta t)^2 + 1{,}2 \Delta t}{\Delta t}   \\
			& = \lim \limits_{\Delta t \to 0} (0{,}1\Delta t + 1{,}2)  \\
			& = 1{,}2.
		\end{align*}
		Vậy gia tốc tức thời của chất điểm tại thời điểm $t=8$ giây là $1{,}2$ ($\mathrm{m/s^2}$).
	}
\end{vd}

\begin{vd}%[DCHT Toán 11 - KNTT- Phạm Tuấn]%[1K5BF-8] 
	Một người lái xe từ thành phố $A$ đến thành phố $B$ với vận tốc trung bình  là $x$ km/h. Trên chuyến trở về, vận tốc trung bình là $y$ km/h. Vận tốc trung bình của cả đi và về là $60$ km/h. (Giả sử người lái xe đi trên cùng  một con đường trên cả chuyến đi và về).
	\begin{enumerate}
		\item Chứng minh rằng $y= \dfrac{30x}{x-30}$.
		\item Tìm giới hạn của $y$ khi $x \rightarrow 30^{+}$.
	\end{enumerate}
	\dapso{$\displaystyle \lim\limits_{x\to 30^+}  y  = +\infty$}
	\loigiai{
		\begin{enumerate}
			\item  
			Gọi  khoảng cách giữa $A$ và $B$ là $s$ km. \\
			Thời gian  chuyến đi là $\dfrac{s}{x}$, thời gian  chuyến trở về là $\dfrac{s}{y}$. \\
			Suy ra 
			$$\dfrac{2s}{60} = \dfrac{s}{x} + \dfrac{s}{y} \Leftrightarrow \dfrac{1}{y} = \dfrac{1}{30} - \dfrac{1}{x} \Leftrightarrow y= \dfrac{30x}{x-30}.$$
			\item  Ta có $\displaystyle \lim\limits_{x\to 30^+}  y = \lim\limits_{x\to 30^+}  \dfrac{30x}{x-30} = +\infty$. 
		\end{enumerate}
	}
\end{vd}


\begin{vd}%[DCHT Toán 11 - KNTT- Phạm Tuấn]%[1K5BF-8] 
	Một hình elip với bán trục lớn $a$ và bán trục nhỏ $b$ thì diện tích được tính theo công thức $S=\pi ab$. Tính giới hạn diện tích của elip khi tiêu cự gần tới $0$.
	\dapso{$\pi a^2$}
	\loigiai{
		Ta có $S= \pi ab = \pi a \sqrt{a^2-c^2}$. \\
		Vậy $\lim\limits_{c \to 0} S= \lim\limits_{c \to 0} \pi a \sqrt{a^2-c^2} = \pi a^2$. \\
		Ta thấy khi $c\to 0$, thì giới hạn diện tích của elip là diện tích hình tròn bán kính $R=a$.
	}
\end{vd}

\begin{vd}%[DCHT Toán 11 - KNTT- Phạm Tuấn]%[1K5BF-8]  
	Các nhà vật lý  thấy rằng thuyết tương đối hẹp của Einstein quy về cơ học Newton khi $c \rightarrow +\infty$, trong đó $c$ là tốc độ ánh sáng. Điều này được minh họa bởi ví dụ: Một hòn đá được ném thẳng đứng từ mặt đất để nó quay trở lại trái đất một giây sau đó. Sử dụng các định luật Newton, chúng ta thấy rằng chiều cao tối đa của hòn đá là $h=\dfrac{g}{8}$ mét ($g = 9{,}8 \mathrm{m/ s ^2}$). Theo thuyết tương đối hẹp, khối lượng của hòn đá phụ thuộc vào vận tốc của nó chia cho $c$, và có chiều cao cực đại là 
	\[
	h(c)=c \sqrt{\dfrac{c^2}{g^2}+\dfrac{1}{4}}- \dfrac{c^2}{g}.
	\]
	Tính $\lim\limits _{c \rightarrow +\infty} h(c)$.
	\dapso{$\lim\limits _{c \rightarrow +\infty} h(c)= \dfrac{g}{8}$}
	\loigiai{
		Ta có 
		\[
		\lim\limits _{c \rightarrow +\infty} c \sqrt{\dfrac{c^2}{g^2}+\dfrac{1}{4}}- \dfrac{c^2}{g} = \lim\limits _{c \rightarrow +\infty} \dfrac{c\left (\dfrac{c^2}{g^2}+\dfrac{1}{4} - \dfrac{c^2}{g^2}\right )}{\sqrt{\dfrac{c^2}{g^2}+\dfrac{1}{4}} + \dfrac{c}{g}} = \lim\limits _{c \rightarrow +\infty} \dfrac{\dfrac{1}{4}}{\sqrt{\dfrac{1}{g^2} + \dfrac{1}{4c^2}}+ \dfrac{1}{g}} = \dfrac{g}{8}.
		\]
	}
\end{vd}

\subsubsection{Bài tập rèn luyện}
% \centerline{\fcolorbox{red}{yellow!50}{\bf {BÀI TẬP TỰ LUẬN}}}
\begin{bt}%[DCHT Toán 11 - KNTT- Phạm Tuấn]%[1K5YF-8] 
	Thế Lennard-Jones có dạng $$U(r) = \dfrac{B}{r^{12}} - \dfrac{A}{r^6}$$ trong đó $A$, $B$ là các hằng số và $r$ là khoảng cách giữa các hạt. 
	Tính $\lim\limits _{r \rightarrow +\infty} U(r)$.
	\dapso{$\lim\limits _{r \rightarrow +\infty} U(r) =0$}
	\loigiai{
		Ta có 
		$$\lim\limits _{r \rightarrow +\infty} U(r) =\lim\limits _{r \rightarrow +\infty} \left (\dfrac{B}{r^{12}} - \dfrac{A}{r^6} \right )  =0.$$
	}
\end{bt}

\begin{bt}%[DCHT Toán 11 - KNTT- Phạm Tuấn]%[1K5YF-8] 
	Trong thuyết tương đối, chiều dài của một vật thể đối với người quan sát phụ thuộc vào tốc độ mà vật thể đang chuyển động đối với người quan sát. Nếu người quan sát đo chiều dài của vật thể là $L_0$ khi đứng yên, thì ở tốc độ $v$ chiều dài  là
	$$
	L=L_0 \sqrt{1-\frac{v^2}{c^2}}
	$$
	trong đó $c$ là tốc độ ánh sáng trong chân không. Tìm $\displaystyle \lim \limits_{n \to +\infty}_{v \rightarrow c^{-}} L$. 
	\dapso{$\displaystyle \lim \limits_{n \to +\infty}_{v \rightarrow c^{-}} L =0$}
	\loigiai{
		Ta có $\displaystyle \lim \limits_{n \to +\infty}_{v \rightarrow c^{-}} L = \lim \limits_{n \to +\infty}_{v \rightarrow c^{-}} L_0 \sqrt{1-\frac{v^2}{c^2}} =  \lim \limits_{n \to +\infty}_{v \rightarrow c^{-}} L_0 \sqrt{1-\frac{c^2}{c^2}} =0$. 
	}
\end{bt}

\begin{bt}%[DCHT Toán 11 - KNTT- Phạm Tuấn]%[1K5BF-8] 
	Trong kỹ thuật ứng dụng, chúng ta thường xuyên ghi nhận được các hàm số mà giá trị của nó thay đổi đột ngột tại một thời điểm $t$ xác định. Ví dụ:  Sự thay đổi điện áp của một mạch điện tại thời điểm t khi đóng hoặc ngắt mạch. Thông thường, giá trị t = 0 luôn được chọn là thời điểm bắt đầu cho việc đóng hoặc ngắt điện áp. Quá trình đóng, ngắt mạch trên có thể mô tả bằng mô hình toán học bởi hàm Heaviside
	\[
	u(t) = \heva{&0 && \text{ nếu } t <0\\& 1 && \text{ nếu } t \geq 0.}
	\]
	Có tồn tại giới hạn $\displaystyle \lim \limits_{t\to 0} u(t)$ hay không?
	\dapso{$\displaystyle \lim \limits_{t\to 0} u(t)$ không tồn tại}
	\loigiai{
		Ta có $\displaystyle \lim \limits_{t\to 0^+} u(t) = \lim \limits_{t\to 0^+} 1 =1$; $\displaystyle \lim \limits_{t\to 0^-} u(t) = \lim \limits_{t\to 0^-} 0 =0$. \\
		Vậy giới hạn $\displaystyle \lim \limits_{t\to 0} u(t)$ không tồn tại.
	}
\end{bt}

\begin{bt}%[DCHT Toán 11 - KNTT- Phạm Tuấn]%[1K5BF-8] 
	Trong một cuộc thi các môn thể thao trên tuyết, người ta muốn thiết kế một đường trượt bằng băng cho nội dung đổ dốc tốc độ đường dài.
	\begin{center}
		\begin{tikzpicture}[scale=0.9, font=\footnotesize, line join=round, line cap=round, >=stealth]
			\draw[->] (0,0)--(10,0) node[below]{$x$} ;
			\draw[->] (0,0)--(0,4) node[left]{$y$} ;
			\foreach \x in {5,10,15,20,25,30,35,40,45}
			\draw[shift={({\x/5},0)},color=black] (0,0) -- (0pt,-2pt) node[below] {$\x$};
			\draw (0,3) node[left]{$15$} (0,0) node[below left]{$O$};
			\clip (0,0) rectangle (9,4) ;
			\draw[thick,smooth,samples=100,domain=0:9] plot(\x,{9/(2*(\x)+3)}) ;
		\end{tikzpicture}
	\end{center}
	Vận động viên sẽ xuất phát từ vị trí $(0 ; 15)$ cao $15$ m so với mặt đất (trục $Ox$). Đường trượt phải thoả mãn yêu cầu là càng ra xa thì càng gần mặt đất để tiết kiệm lượng tuyết nhân tạo. Một nhà thiết kế đề nghị sử dụng đường cong là đồ thị hàm số $y=f(x)=\dfrac{150}{x+10}$, với $x \geq 0$. Hãy kiểm tra xem hàm số $y=f(x)$ có thoả mãn các điều kiện dưới đây hay không:
	\begin{enumerate}
		\item    Có đồ thị qua điểm $(0; 15)$;
		\item    Giảm trên $[0 ;+\infty)$;
		\item    Càng ra xa ($x$ càng lớn), đồ thị của hàm số càng gần trục $O x$ với khoảng cách nhỏ tuỳ ý.
	\end{enumerate}
	\dapso{Đồ thị qua điểm $(0; 15)$; Hàm số giảm trên $[0 ;+\infty)$; $\lim\limits _{x \rightarrow +\infty} f(x) = 0$}
	\loigiai{
		\begin{enumerate}
			\item    Ta có $f(0)= \dfrac{150}{10}$ nên đồ thị hàm số $f(x)$ đi qua điểm $(0; 15)$.
			\item    Chọn bất kì $x_1,x_2 \in [0;+\infty]$ và $x_1 \ne x_2$. \\
			Ta có $\dfrac{f(x_2)-f(x_1)}{x_2-x_1} =  \dfrac{\dfrac{150}{x_2+10} - \dfrac{150}{x_1+10}}{x_2-x_1} = \dfrac{x_1-x_2}{(x_2-x_1)(x_1+10)(x_2+10)} = -\dfrac{1}{(x_1+10)(x_2+10)} <0$. \\
			Suy ra hàm số nghịch biến trên  $[0 ;+\infty)$ hay hàm số giảm trên $[0 ;+\infty)$.
			\item   Ta có $\lim\limits _{x \rightarrow +\infty}  f(x) = \lim\limits _{x \rightarrow +\infty} \dfrac{150}{x+10} = 0$. \\
			Vậy khi $x$ càng lớn, đồ thị của hàm số càng gần trục $O x$ với khoảng cách nhỏ tuỳ ý.
		\end{enumerate}
	}
\end{bt}

\begin{bt}%[DCHT Toán 11 - KNTT- Phạm Tuấn]%[1K5YF-8] 
	Chiều dài một loài động vật nhỏ được tính theo công công thức $h(t)=\dfrac{100}{2+3 \cdot  (0{,}4)^t}$ mm, trong đó $t$ số ngày sau khi sinh của loài động vật đó. Tính chiều dài  cuối cùng của nó (chiều dài khi $t \to +\infty$).
	\dapso{$100$ mm}
	\loigiai{
		Ta có $\displaystyle \lim \limits_{t \to +\infty } h(t)=\dfrac{100}{2+3 \cdot  (0{,}4)^t} = 100$. \\
		Vậy chiều dài của loài động vật khi trưởng thành là $100$ mm. 
	}
\end{bt}

\begin{bt}%[DCHT Toán 11 - KNTT- Phạm Tuấn]%[1K5BF-8] 
	Một chất điểm chuyển động thẳng với phương trình $s(t)$. Khi đó vận tốc tức thời tại thời điểm $t_0$ được định nghĩa là $\displaystyle \lim \limits_{\Delta t} \dfrac{s(t_0+ \Delta t) - s(t_0)}{\Delta t}$. Tính vận tốc tức thời của chất điểm với phương trình chuyển động $s(t) = 4t^2-3t+1$ ($s(t)$ có đơn vị là m, $t$ đơn vị là giây), tại thời điểm $t=8$ giây. 
	\dapso{$61  \mathrm{~m/s}$}
	\loigiai{
		Vận tốc tức thời của chất điểm tại thời điểm $t=8$ giây là
		\begin{align*}
			\lim \limits_{\Delta t \to 0} \dfrac{s(8+\Delta t) - s(8)}{\Delta t} &=  \lim \limits_{\Delta t \to 0} \dfrac{4(8+\Delta t)^2-3(8+\Delta t)+1 - 233}{\Delta t} \\
			& = \lim \limits_{\Delta t \to 0} \dfrac{4 (\Delta t)^2+ 61 \Delta t}{\Delta t} = 61  \mathrm{~m/s}.
		\end{align*}
	}
\end{bt}

\begin{bt}%[DCHT Toán 11 - KNTT- Phạm Tuấn]%[1K5YF-8] 
	Bỏ qua lực cản của không khí, độ cao tối đa mà tên lửa đạt được khi phóng với vận tốc ban đầu $v_0$ là $h=\dfrac{v_0^2 R}{19{,}6 R-v_0^2}$, trong đó $R$ là bán kính của trái đất. Tính  $\displaystyle \lim \limits_{R \to +\infty} h$. 
	\dapso{$\lim \limits_{R \to +\infty} h =  \dfrac{v_0^2}{19{,}6}$}
	\loigiai{
		Ta có 
		\begin{align*}
			\lim \limits_{R \to +\infty} h &= \displaystyle \lim \limits_{R \to +\infty} \dfrac{v_0^2 R}{19{,}6 R-v_0^2} \\
			&= \lim \limits_{R \to +\infty} \dfrac{v_0^2}{19{,}6 - \dfrac{v_0^2}{R}} = \dfrac{v_0^2}{19{,}6}.
		\end{align*}
	}
\end{bt}


\begin{bt}%[DCHT Toán 11 - KNTT- Phạm Tuấn]%[1K5BF-8] 
	Một hình elip với bán trục lớn $a$ và bán trục nhỏ $b$ thì diện tích được tính theo công thức $S=\pi ab$. Cho elip có bán trục nhỏ bằng $30$ cm, tính giới hạn diện tích của elip khi tiêu cự gần tới $0$.
	\dapso{$900\pi \mathrm{~cm^2}$}
	\loigiai{
		Ta có $S= \pi ab = \pi b\sqrt{b^2+c^2}$. \\
		Vậy $\lim\limits_{c \to 0} S= \lim\limits_{c \to 0} \pi b\sqrt{b^2+c^2} = \pi b^2 = 900\pi \mathrm{~cm^2}$. 
	}
\end{bt}





\begin{bt}%[DCHT Toán 11 - KNTT- Phạm Tuấn]%[1K5BF-8] 
	Số lượng đơn vị hàng tồn kho trong một công ty nhỏ được cho bởi
	$$
	N(t)=25\left(2 \left [\frac{t+2}{2}  \right ]-t\right)
	$$
	trong đó $t$ là thời gian tính bằng tháng, $[x]$ là số nguyên lớn nhất không vượt quá $x$ (ví dụ $[2{,}4]=2$, $[-2{,}7] = -3$).
	\begin{enumerate}
		\item Tính $\lim\limits_{t \to 8^+} N(t)$.
		\item  Tính $\lim\limits_{t \to 16^-} N(t)$.
	\end{enumerate}
	\dapso{$\lim\limits _{t \to 8^+} N(t)=50$; $\lim\limits _{t \to 16^-} N(t) =0$}
	\loigiai{
		\begin{enumerate}
			\item 
			Khi $t \to 8^+$, ta có $\left [\dfrac{t+2}{2}  \right ] = 5$. \\
			Suy ra  $\lim\limits _{t \to 8^+} N(t) =\lim_{t \to 8^+}  25\left(2 \left [\frac{t+2}{2}  \right ]-t\right) = 50$.
			\item  
			Khi $t \to 16^-$, ta có $\left [\dfrac{t+2}{2}  \right ] = 8$. \\
			Suy ra  $\lim\limits _{t \to 16^-} N(t) = \lim_{t \to 16^-} 25\left(2 \left [\frac{t+2}{2}  \right ]-t\right) = 0$.
		\end{enumerate}
	}
\end{bt}


\begin{bt}%[DCHT Toán 11 - KNTT- Phạm Tuấn]%[1K5BF-8] 
	Định luật Boyle được phát biểu:  ``Đối với một lượng khí ở nhiệt độ không đổi, áp suất $P$ tỷ lệ nghịch với thể tích $V$''. Tìm giới hạn của $P$ là $V \rightarrow 0^{+}$.
	\dapso{$\lim \limits_{V \to 0^+} P = +\infty$}
	\loigiai{
		Ta có $P= \dfrac{k}{V}$ với $k$ là số thực dương không đổi. \\
		Khi đó $\lim \limits_{V \to 0^+} P= \dfrac{k}{V} = +\infty$.
	}
\end{bt}

\begin{bt}%[DCHT Toán 11 - KNTT- Phạm Tuấn]%[1K5BF-8] 
	Một vật khối lượng $m$ (không đổi) bắt đầu chuyển động với vận tốc $v_0=0$, được gia tốc bởi một lực $F$ không đổi trong $t$ giây. Theo định luật Newton về chuyển động, vật tốc của vật là $v_N = \dfrac{Ft}{m}$. Theo thuyết tương đối Einstein, vật có vận tốc $v_E = \dfrac{Fct}{\sqrt{m^2c^2+F^2t^2}}$, với $c$ là vận tốc ánh sáng. Tính $\displaystyle \lim \limits_{t \to +\infty} v_N$ và $\displaystyle \lim \limits_{t \to +\infty} v_E$.
	\dapso{$\lim \limits_{t \to +\infty} v_N  = +\infty$; $\lim \limits_{t \to +\infty} v_E  = \dfrac{c}{F}$}
	\loigiai{
		Ta có
		\begin{align*}
			& \lim_{t \to +\infty} v_N = \lim_{t \to +\infty}\dfrac{Ft}{m} = +\infty; \\
			& \lim_{t \to +\infty} v_E = \lim_{t \to +\infty}\dfrac{Fct}{\sqrt{m^2c^2+F^2t^2}} =  \lim_{t \to +\infty} \dfrac{Fc}{\sqrt{\dfrac{m^2c^2}{t^2}}+F^2} = \dfrac{c}{F}.
		\end{align*}
	}
\end{bt}


\begin{bt}%[DCHT Toán 11 - KNTT- Phạm Tuấn]%[1K5BF-8] 
	\immini{
		Gọi $S$ là diện tích hình phẳng giới hạn bởi đường tròn bán kính $10$ và tam giác vuông (hình vẽ bên).
		\begin{enumerate}
			\item Đặt $S= f(\varphi)$, với $f(\varphi)$ là hàm số của $\varphi$ (Đơn vị rad). Tìm công thức của $f(\varphi)$ với $0< \varphi < \dfrac{\pi}{2}$.
			\item Tính giới hạn của $f(\varphi)$ khi $\varphi \to \dfrac{\pi}{2}^-$.
		\end{enumerate}
	}
	{
		\begin{tikzpicture}[scale=1, font=\footnotesize, line join=round, line cap=round, >=stealth]
			\path
			(0,0) coordinate (A)
			(4,0) coordinate (B)
			(0,2.5) coordinate (C)
			($(B)!{4/sqrt(4^2+2.5^2)}!(C)$)  coordinate (D)
			;
			\fill[cyan!30] (A) arc (180:{180-atan(2.5/4)}:4) -- (C)--cycle;
			\draw (A) arc (180:{180-atan(2.5/4)}:4)  ;
			\draw (A)--(B)--(C)--(A);
			\draw pic["$\varphi$", draw=black, angle eccentricity=1.3, angle radius=0.7cm]{angle=C--B--A} ;
			\foreach \x/\g in {A/-120,B/-30,C/90,D/50} 
			\fill[black](\x) circle (1pt)+(\g:2.5mm) node{$\x$};
		\end{tikzpicture}
	}
	\dapso{$S=8(\tan \varphi - \varphi)$; $\lim \limits_{\varphi \to \tfrac{\pi}{2}^-}  f(\varphi) =+\infty $}
	\loigiai{
		\begin{enumerate}
			\item Diện tích tam giác $ABC$ là $S_{ABC} = \dfrac{1}{2} AB \cdot AC = \dfrac{1}{2} \cdot 4 \cdot 4 \tan \varphi = 8  \tan \varphi$. \\
			Diện tích hình quạt $ABD$ là $S_q = \dfrac{4^2 \varphi }{2} = 8 \varphi$. \\
			Diện tích hình phẳng $S$ là $S=f(\varphi) = S_{ABC} - S_q = 8(\tan \varphi - \varphi)$.
			\item 
			Khi $\varphi \to \tfrac{\pi}{2}^-$ thì $\heva{&\cos \varphi \to 0\\& \cos \varphi >0.}$\\
			Suy ra $\displaystyle \lim \limits_{\varphi \to \tfrac{\pi}{2}^-}  f(\varphi) = \lim \limits_{\varphi \to \tfrac{\pi}{2}^-}  8\left (\dfrac{\sin \varphi}{\cos \varphi} - \varphi \right )= +\infty$.
		\end{enumerate}
	}
\end{bt}

\begin{bt}%[DCHT Toán 11 - KNTT- Phạm Tuấn]%[1K5BF-8] 
	Trên một chuyến đi dài $d$ km đến một thành phố khác, vận tốc trung bình của một tài xế xe tải là $x$ km/h. Trên chuyến trở về, vận tốc trung bình là $y$ km/h. Vận tốc trung bình của cả đi và về là 50 km/h.
	\begin{enumerate}
		\item Chứng minh rằng $y= \dfrac{25x}{x-25}$.
		\item Tìm giới hạn của $y$ khi $x \rightarrow 25^{+}$ và giải thích ý nghĩa của nó.
	\end{enumerate}
	\dapso{$\displaystyle \lim\limits_{x\to 25^+}  y = +\infty$}
	\loigiai{
		\begin{enumerate}
			\item  Thời gian  chuyến đi là $\dfrac{d}{x}$, thời gian  chuyến trở về là $\dfrac{d}{y}$. \\
			Suy ra 
			$$\dfrac{2d}{50} = \dfrac{d}{x} + \dfrac{d}{y} \Leftrightarrow \dfrac{1}{y} = \dfrac{1}{25} - \dfrac{1}{x} \Leftrightarrow y= \dfrac{25x}{x-25}.$$
			\item  Ta có $\displaystyle \lim\limits_{x\to 25^+}  y = \lim\limits_{x\to 25^+}  \dfrac{25x}{x-25} = +\infty$. \\
			Khi vận tốc trung bình chuyến đi bằng 25 km/h,  thì  vận tốc trung bình chuyến của cả chuyến đi và về không thể là 50 km/h.
		\end{enumerate}
	}
\end{bt}

\begin{bt}%[DCHT Toán 11 - KNTT- Phạm Tuấn]%[1K5BF-8] 
	Một chất điểm chuyển động thẳng với vận tốc $v(t)$. Khi đó gia tốc tức thời tại thời điểm $t_0$ được định nghĩa là $\displaystyle \lim \limits_{\Delta t \to 0} \dfrac{v(t_0+ \Delta t) - v(t_0)}{\Delta t}$. Một chất điểm chuyển động với vận tốc $v(t) = 5 \sin \left (4\pi t\right )$ (m/s), tính gia tốc tức thời tại thời điểm $t=5$ giây.  (Biết $\displaystyle \lim_{x \to 0} \dfrac{\sin x}{x} =1$). 
	\dapso{$20\pi$ $\mathrm{m/s^2}$}
	\loigiai{
		Gia tốc tức thời của chất điểm tại thời điểm $t=5$ giây là
		\begin{align*}
			\lim \limits_{\Delta t \to 0} \dfrac{v(5+\Delta t) - s(5)}{\Delta t} &=  \lim \limits_{\Delta t \to 0} \dfrac{5\sin (20\pi+4\pi\Delta t) - 5\sin (20\pi)}{\Delta t} \\
			& = \lim \limits_{\Delta t \to 0} \dfrac{5\sin (4\pi\Delta t)}{\Delta t}   \\
			& = \lim \limits_{\Delta t \to 0} \dfrac{20\pi\sin (4\pi\Delta t)}{4\pi\Delta t}   \\
			& = 20\pi.
		\end{align*}
		Vậy gia tốc tức thời của chất điểm tại thời điểm $t=5$ giây là $20\pi$ ($\mathrm{m/s^2}$).
	}
\end{bt}


\begin{bt}%[DCHT Toán 11 - KNTT- Phạm Tuấn]%[1K5BF-8] 
	Một bể chứa $5000$ lít nước tinh khiết. Nước muối chứa $30$ g muối trên một lít nước được bơm vào bể với tốc độ $25$ lít/phút. Gọi nồng độ của muối sau $t$ phút (tính bằng gam trên lít) là $C(t)$. Tính $\displaystyle \lim \limits_{t \to +\infty} C(t)$.  Giải thích ý nghĩa của giới hạn này.
	\dapso{$\lim \limits_{t \to +\infty} C(t) =30$}
	\loigiai{
		Số lít nước muối được bơm vào bể sau $t$ phút là $25t$ lít. \\
		Số g muối có trong $25t$ lít nước muối là $30 \cdot 25 t = 750t$ gam. \\
		Nồng độ của muối trong bể sau $t$ phút  là
		\[
		\dfrac{750t}{25t + 5000} = \dfrac{30t}{t+200} \text{~gam/lít}.
		\]
		Ta có $\displaystyle \lim \limits_{t \to +\infty} C(t) = \lim \limits_{t \to +\infty} \dfrac{30t}{t+200} =  30  \text{~gam/lít}$. \\
		Khi thời gian tiến tới vô hạn thì nồng độ của muối trong bể bằng nồng độ của nước muối bơm vào bể.
	}
\end{bt}



\begin{bt}%[DCHT Toán 11 - KNTT- Phạm Tuấn]%[1K5BF-8] 
	Một thấu kính hội tụ có tiêu cự $f=30 \mathrm{~cm}$. Trong Vật lí, ta biết rằng nếu đặt vật thật $A B$ cách quang tâm của thấu kính một khoảng $d>30$ ($\mathrm{cm}$) thì được ảnh thật $A' B'$ cách quang tâm của thấu kính một khoảng $d'$ (cm) (Hình vẽ dưới). Ngược lại, nếu $0<d<30$, ta có ảnh ảo. Công thức của thấu kính là $\dfrac{1}{d}+\dfrac{1}{d'}=\dfrac{1}{30}$.
	\begin{center}
		\begin{tikzpicture}[scale=1, font=\footnotesize, line join=round, line cap=round, >=stealth]
			\path
			(0,0) coordinate (O)
			(-4,0) coordinate (A)
			(8,0) coordinate (A')
			(-4,1) coordinate (B)
			(8,-2) coordinate (B')
			(0,1) coordinate (B1)
			({-8/3},1.4) coordinate (P)
			(0,1.4) coordinate (Q)
			({8/3},1.4) coordinate (R)
			(-4,-1.4) coordinate (X)
			(0,-1.4) coordinate (Y)
			(8,-1.4) coordinate (Z)
			(intersection of O--A' and B1--B')  coordinate (F')
			($(F')!2!(O)$) coordinate (F)
			;
			\draw (-5,0)--(9,0) ;
			\draw[<->,>=triangle 45,very thick] (0,-2.2)--(0,2.2);  
			\draw[->,>=triangle 45,very thick] (A)--(B);  
			\draw[->,>=triangle 45,very thick] (A')--(B');  
			\draw[<->,dashed] (P)--(Q) ;
			\draw[<->,dashed] (Q)--(R) ;
			\draw[<->,dashed] (X)--(Y) ;
			\draw[<->,dashed] (Y)--(Z) ;
			\draw[dashed] (F)--(P) (F')--(R) (A)--(X) (A')--(Z);
			\draw[fill] ($(P)!0.5!(Q)$) node[above]{$30$} ($(Q)!0.5!(R)$) node[above]{$30$}
			($(X)!0.5!(Y)$) node[above]{$d$} ($(Y)!0.4!(Z)$) node[above]{$d'$} (F) circle(1.2pt) (F') circle(1.2pt);
			\draw (B)--(B') (B)--(B1)--(B');
			\foreach \x/\g in{A/-50,B/90,A'/90,B'/-90,O/-130,F/-90,F'/-90}
			\fill[black](\x)  ($(\x)+(\g:3mm)$) node{$\x$}; 
		\end{tikzpicture}
	\end{center} 
	\begin{enumerate}
		\item Từ công thức của thấu kính, hãy tìm biểu thức xác định hàm số $d'=h(d)$.
		\item Tìm các giới hạn $\lim\limits  _{d \rightarrow 30^{+}} h(d) ; \lim\limits _{d \rightarrow 30^{-}} h(d)$ và $\lim\limits _{d \rightarrow+\infty} h(d)$. Sử dụng các kết quả này để giải thích ý nghĩa đã biết trong Vật lí.
	\end{enumerate}
	\dapso{$d'=h(d) = \dfrac{30d}{d-30}$; $\lim\limits  _{d \rightarrow 30^{+}} h(d) = +\infty$; $\lim\limits  _{d \rightarrow 30^{-}} h(d) = -\infty$}
	\loigiai{
		\begin{enumerate}
			\item Ta có $\dfrac{1}{d}+\dfrac{1}{d'}=\dfrac{1}{30} \Leftrightarrow \dfrac{1}{d'} = \dfrac{1}{30} - \dfrac{1}{d} = \dfrac{d-30}{30d} \Leftrightarrow d'= \dfrac{30d}{d-30}$. \\
			Vậy $d'=h(d) = \dfrac{30d}{d-30}$.
			\item 
			Khi $d\to 30^+$, ta có $d-30 \to 0$, $d-30 >0$ và $30d \to 900$.\\
			Suy ra $\lim\limits _{d \rightarrow 30^{+}} h(d) = \lim\limits _{d \rightarrow 30^{+}} \dfrac{30d}{d-30} = +\infty$. \\
			Khi $d\to 30^-$, ta có $d-30 \to 0$, $d-30 <0$ và $30d \to 900$.\\
			Suy ra $\lim\limits _{d \rightarrow 30^{-}} h(d) = \lim\limits _{d \rightarrow 30^{-}} \dfrac{30d}{d-30} = -\infty$. \\
			Ta có $\lim\limits _{d \rightarrow +\infty} h(d) =  \lim\limits _{d \rightarrow +\infty} \dfrac{30d}{d-30} = \lim\limits _{d \rightarrow +\infty}  \dfrac{30}{1-\dfrac{30}{d}} =30$. \\
			Vậy 
			\begin{itemize}
				\item  Khi vị trí của vật tiến gần  tiêu điểm $F$ ($d >f$) thì vị trí ảnh thật của vật dần ra xa vô cực. 
				\item  Khi vị trí của vật tiến gần  tiêu điểm $F$ ($d <f$) thì vị trí ảnh ảo của vật dần ra xa vô cực. 
				\item  Khi vị trí của vật tiến ra xa vô cực thì  ảnh thật của vật dần tới tiêu điểm. 
			\end{itemize}
		\end{enumerate}
	}
\end{bt}



%%Bài 17. Hàm số liên tục
\section{Hàm số liên tục}
\setcounter{dang}{0}
\subsection{Tóm tắt lý thuyết}
\begin{tomtat}
	\subsubsection{Hàm số liên tục tại một điểm}
	\begin{itemize}
		\item Cho hàm số $y=f(x)$ xác định trên khoảng $(a;b)$ chứa điểm $x_0$. Hàm số $f(x)$ được gọi là \textcolor{red}{liên tục tại điểm} $x_0$ nếu $\lim\limits_{x\to{x_0}}f(x)=f(x_0)$.
		\item Hàm số $f(x)$ không liên tục tại $x_0$ được gọi là \textcolor{red}{gián đoạn} tại điểm đó.
		\begin{note}
			Hàm số $y=f(x)$ liên tục tại $x_0$ khi và chỉ khi $\lim\limits_{x\to{x_0}^+}f(x)=\lim\limits_{x\to{x_0}^-}f(x)=f(x_0)$.
		\end{note}
	\end{itemize}
	\subsubsection{Hàm số liên tục trên một khoảng}
	\begin{itemize}
		\item Hàm số $y=f(x)$ được gọi là \textcolor{red}{liên tục trên khoảng} $(a;b)$ nếu nó liên tục tại mọi điểm thuộc khoảng này.
		\item Hàm số $y=f(x)$ được gọi là \textcolor{red}{liên tục trên đoạn} $[a;b]$ nếu nó liên tục trên khoảng $(a;b)$ và $\lim\limits_{x\to{a}^+}f(x)=f(a)$, $\lim\limits_{x\to{b}^-}f(x)=f(b)$.
		\item Các khái niệm hàm số liên tục trên nửa khoảng như $(a;b]$, $[a;+\infty)\ldots$ được định nghĩa theo cách tương tự.
		\item Đồ thị của hàm số liên tục trên một khoảng là một \textcolor{red}{đường liền nét} trên khoảng đó.
	\end{itemize}
	\subsubsection{Tính chất 1}
	\begin{itemize}
		\item Hàm số đa thức và các hàm số $y=\sin x$, $y=\cos x$ liên tục trên $\mathrm{R}$.
		\item Các hàm số $y=\tan x$, $y=\cot x$, $y=\sqrt x$ và các hàm phân thức hữu tỉ (thương của hai đa thức) liên tục trên mỗi khoảng xác định của chúng.
	\end{itemize}
	\subsubsection{Tính chất 2}
	Giả sử hai hàm số $y=f(x)$ và $y=g(x)$ liên tục tại điểm $x_0$. Khi đó:
	\begin{itemize}
		\item Các hàm số $y=f(x)+g(x)$, $y=f(x)-g(x)$ và $y=f(x)\cdot g(x)$ liên tục tại $x_0$.
		\item Hàm số $\dfrac{f(x)}{g(x)}$ liên tục tại $x_0$ nếu $g(x_0)\ne 0$.
	\end{itemize}
\end{tomtat}
\subsection{Các dạng toán thường gặp}
\begin{dang}{Dựa vào đồ thị xét tính liên tục của hàm số tại một điểm, một khoảng.}
	Để xét tính liên tục của hàm số khi biết đồ thị, ta cần nhớ:
	\begin{itemize}
		\item Đồ thị của hàm số liên tục trên một khoảng là một đường liền nét trên khoảng đó.
		\item Hàm số $y=f(x)$ liên tục tại $x_0$ khi và chỉ khi $\lim\limits_{x\to{x_0}^+}f(x)=\lim\limits_{x\to{x_0}^-}f(x)=f(x_0)$.
	\end{itemize}
\end{dang}
\subsubsection{Ví dụ mẫu}
\begin{vd}%[DCHT Toán 11 - KNTT -Vũ Hồng Toàn]%[1K5YG-2]
	\immini{
		Cho hàm số $y=f(x)$ có đồ thị như hình vẽ bên.\\ Xét tính liên tục của hàm số $y=f(x)$ trên khoảng $(0;2)$.	
	}
	{
		\begin{tikzpicture}[scale=.75, font=\footnotesize, line join=round, line cap=round,>=stealth]
			\def\xmin{-.5} \def\xmax{2.5}
			\def\ymin{-0.5} \def\ymax{2.5}
			\draw[->] (\xmin,0)--(0,0)node [below right]{$O$}-- (\xmax,0) node [below]{$x$};
			\draw[->] (0,\ymin)--(0,\ymax) node [left]{$y$};
			\draw[ thick] (0,0)--(1,2)--(2,2);
			\draw[dash pattern=on 2pt off 1.5pt] (1,0)node[below]{$1$}--(1,2) (2,0)node[below]{$2$}--(2,2)  (1,2)node[above ]{$y=f(x)$};
			\foreach \x/\y in{0/0,1/0,2/0,1/2,2/2} \fill(\x,\y)circle(.03);
		\end{tikzpicture}		
	}
	% \dapso{Hàm số liên tục trên khoảng $(0;2)$.}
	\loigiai{
		Đồ thị hàm số là một đường liền nét trên khoảng $(0;2)$ nên hàm số đã cho liên tục trên khoảng $(0;2)$.}
\end{vd}
\begin{vd}%[DCHT Toán 11 - KNTT -Vũ Hồng Toàn]%[1K5YG-2]
	\immini{
		Cho hàm số $y=f(x)$ có đồ thị như hình vẽ bên.\\ Xét tính liên tục của hàm số $y=f(x)$ trên khoảng $(-2;2)$.	
	}
	{
		\begin{tikzpicture}[scale=.75, font=\footnotesize, line join=round, line cap=round,>=stealth]
			\def\xmin{-2.5} \def\xmax{2.5}
			\def\ymin{-0.5} \def\ymax{2.5}
			\draw[->] (\xmin,0)--(0,0)node [below right]{$O$}-- (\xmax,0) node [below]{$x$};
			\draw[->] (0,\ymin)--(0,\ymax) node [left]{$y$};
			\draw[ thick] plot[domain=-2:2](\x,{abs(\x)});
			\draw[dash pattern=on 2pt off 1.5pt] (-2,0)node[below]{$-2$}--(-2,2) (2,0)node[below]{$2$}--(2,2)  (.5,1)node[above ]{$y=f(x)$};
			\foreach \x/\y in{0/0,2/0,-2/2,2/2} \fill(\x,\y)circle(.03);
		\end{tikzpicture}		
	}
	% \dapso{Hàm số liên tục trên khoảng $(-2;2)$.}
	\loigiai{
		Đồ thị hàm số là một đường liền nét trên khoảng $(-2;2)$ nên hàm số đã cho liên tục trên khoảng $(-2;2)$.}
\end{vd}

\begin{vd}%[DCHT Toán 11 - KNTT -Vũ Hồng Toàn]%[1K5BG-2]
	\immini{
		Cho hàm số $y=f(x)$ có đồ thị như hình vẽ bên.\\ Xét tính liên tục của hàm số $y=f(x)$ trên khoảng $(0;2)$.	
	}
	{
		\begin{tikzpicture}[scale=.75, font=\footnotesize, line join=round, line cap=round,>=stealth]
			\def\xmin{-.5} \def\xmax{2.5}
			\def\ymin{-0.5} \def\ymax{3}
			\draw[->] (\xmin,0)--(0,0)node [below right]{$O$}-- (\xmax,0) node [below]{$x$};
			\draw[->] (0,\ymin)--(0,\ymax) node [left]{$y$};
			\draw[ thick] (0,0)--(1,1) (1,2)--(2,2);
			\draw[dash pattern=on 2pt off 1.5pt] (1,0)node[below]{$1$}--(1,1) (2,0)node[below]{$2$}--(2,2) (0,1)node[left]{$1$} (0,2)node[left]{$2$}
			(1,2)node[above ]{$y=f(x)$}
			;
			\foreach \x/\y in{0/0,1/0,2/0,1/2,2/2,1/1,0/1,0/2} \fill(\x,\y)circle(.03);
		\end{tikzpicture}		
	}
	
	% \dapso{Hàm số đã cho liên tục trên các khoảng $(0,1)$, $(1,2)$ và gián đoạn tại $x=1$.}
	\loigiai{
		\begin{itemize}
			\item Đồ thị hàm số là các đường liền nét trên các khoảng $(0;1)$, $(1;2)$ do đó hàm số liên tục trên các khoảng này.
			\item Đồ thị hàm số không liền nét tại điểm $x=1$ do đó hàm số đã cho gián đoạn tại điểm này.
		\end{itemize}
	}
\end{vd}
\begin{vd}%[DCHT Toán 11 - KNTT -Vũ Hồng Toàn]%[1K5BG-2]
	\immini{
		Cho hàm số $y=f(x)$ có đồ thị như hình vẽ bên.\\ Xét tính liên tục của hàm số $y=f(x)$ trên khoảng $(0;2)$.	
	}
	{
		\begin{tikzpicture}[scale=.75, font=\footnotesize, line join=round, line cap=round,>=stealth]
			\def\xmin{-.5} \def\xmax{2.5}
			\def\ymin{-0.5} \def\ymax{2.8}
			\draw[->] (\xmin,0)--(0,0)node [below right]{$O$}-- (\xmax,0) node [below]{$x$};
			\draw[->] (0,\ymin)--(0,\ymax) node [left]{$y$} ;
			\draw[ thick] (1,1)--(2,1) (0,2)--(1,1.2);
			\draw[dash pattern=on 2pt off 1.5pt] (1,0)node[below]{$1$}--(1,1) (2,0)node[below]{$2$}--(2,1) (0,1)node[left]{$1$} (0,2)node[left]{$2$} (0,1)--(1,1)
			(1,2)node[above ]{$y=f(x)$}
			;
			\foreach \x/\y in{0/0,1/0,2/0,2/1,0/1,0/2} \fill(\x,\y)circle(.03);
		\end{tikzpicture}		
	}
	% \dapso{Hàm số đã cho liên tục trên các khoảng $(0,1)$, $(1,2)$ và gián đoạn tại $x=1$.}
	\loigiai{
		\begin{itemize}
			\item Đồ thị hàm số là các đường liền nét trên các khoảng $(0;1)$, $(1;2)$ do đó hàm số liên tục trên các khoảng này.
			\item Ta có $\lim\limits_{x\to{1}^-}f(x)>f(1)=1$ và $\lim\limits_{x\to{1}^+}f(x)=f(1)=1$.\\ Do đó $\lim\limits_{x\to{1}^-}f(x)\ne \lim\limits_{x\to{1}^+}f(x)$.\\
			Vậy hàm số đã cho gián đoạn tại $x=1$.
		\end{itemize}
	}
\end{vd}

\begin{vd}%[DCHT Toán 11 - KNTT -Vũ Hồng Toàn]%[1K5KG-2]
	\immini{
		Cho hàm số $y=f(x)$ có tập xác định $\mathscr{D}=\mathbb{R}\setminus \{0\}$ và có đồ thị như hình bên. Xét tính liên tục của hàm số $y=f(x)$ trên $\mathscr{D}$.	
	}
	{
		\begin{tikzpicture}[scale=.75, font=\footnotesize, line join=round, line cap=round,>=stealth]
			\def\xmin{-2.5} \def\xmax{2.8}
			\def\ymin{-0.5} \def\ymax{2.8}
			%\draw[color=gray!50,dashed] (\xmin,\ymin) grid (\xmax,\ymax);
			\draw[->] (\xmin,0)--(0,0)node [below right]{$O$}-- (\xmax,0) node [below]{$x$};
			\draw[->] (0,\ymin)--(0,\ymax) node [left]{$y$};
			\begin{scope}
				\clip (\xmin,\ymin) rectangle (\xmax,\ymax-.15);
				\draw[samples=200,smooth,variable=\x,thick, teal] plot[domain=\xmin:-0.5] (\x,{1/(\x)^2});
				\draw[samples=200,smooth,variable=\x,thick, teal] plot[domain=0.5:\xmax] (\x,{1/(\x)^2});
			\end{scope}
			\draw[ thick] (1.2,1)node[right]{$y=f(x)$};
			\foreach \x/\y in{0/0} \fill(\x,\y)circle(.03);
		\end{tikzpicture}	
	}
	% \dapso{Hàm số đã cho liên tục trên các khoảng $(-\infty;0)$ và $(0;+\infty)$. Gián đoạn tại điểm $x=0$.}
	\loigiai{
		Vì hàm số đã cho có tập xác định $\mathscr{D}=\mathbb{R}\setminus \{0\}$ nên
		\begin{itemize}
			\item $f(x)$ xác định trên khoảng $(-\infty;0)$ nên liên tục trên khoảng này.
			\item $f(x)$ xác định trên khoảng $(0;+\infty)$ nên liên tục trên khoảng này.
			\item $f(x)$ không xác định tại điểm $x=0$ nên gián đoạn tại điểm này.
		\end{itemize}	
	}
\end{vd}

% \subsubsection{Bài tập rèn luyện}
% % \centerline{\fcolorbox{teal}{yellow!50}{\bf {BÀI TẬP TỰ LUẬN }}}
% \begin{bt}%[DCHT Toán 11 - KNTT -Vũ Hồng Toàn]%[1K5YG-2]
% 	\immini{
% 		Cho hàm số $y=f(x)$ xác định trên $\mathscr{D}=\mathbb{R}$ và có đồ thị như hình vẽ bên.\\ Xét tính liên tục của hàm số $y=f(x)$ trên $\mathscr{D}$.	
% 	}
% 	{
% 		\begin{tikzpicture}[scale=.75, font=\footnotesize, line join=round, line cap=round,>=stealth]
% 			\def\a{1} \def\b{-2} \def\c{0} %\def\d{1} % Hệ số
% 			\def\xmin{-2} \def\xmax{2}
% 			\def\ymin{-1.5} \def\ymax{1.5}
% 			\draw[->] (\xmin,0)--(0,0)node [above right]{$O$}-- (\xmax,0) node [below]{$x$};
% 			\draw[->] (0,\ymin)--(0,\ymax) node [left]{$y$};
% 			\clip (\xmin+0.1,\ymin+0.1) rectangle (\xmax-0.1,\ymax-0.1);
% 			\draw[smooth,samples=100,thick ] plot(\x,{\a*(\x)^4+\b*(\x)^2+\c});
% 			\draw[ thick] (-.7,1)node[below]{$y=f(x)$};
% 		\end{tikzpicture}	
% 	}	
% 	% \dapso{Hàm số đã cho liên tục trên $\mathbb{R}$.}
% 	\loigiai{
% 		Do đồ thị hàm số là một đường liền nét trên $\mathscr{D}$ nên hàm số $y=f(x)$ liên tục trên $\mathscr{D}=\mathbb{R}$.}
% \end{bt}
% \begin{bt}%[DCHT Toán 11 - KNTT -Vũ Hồng Toàn]%[1K5BG-2]
% 	\immini{
% 		Cho hàm số $y=f(x)$ xác định trên $\mathscr{D}=\mathbb{R}$ và có đồ thị như hình vẽ bên. Xét tính liên tục của hàm số $y=f(x)$ trên $\mathscr{D}$.	
% 	}
% 	{
% 		\begin{tikzpicture}[scale=.75, font=\normalsize, line join=round, line cap=round,>=stealth]
% 			\def\a{2} %\def\b{-2} \def\c{0} %\def\d{1} % Hệ số
% 			\def\xmin{-2.5} \def\xmax{2.5}
% 			\def\ymin{-0.5} \def\ymax{3}
% 			%\draw[color=gray!50,dashed] (\xmin,\ymin) grid (\xmax,\ymax);
% 			\draw[->] (\xmin,0)--(0,0)node [below right]{$O$}-- (\xmax,0) node [below]{$x$};
% 			\draw[->] (0,\ymin)--(0,\ymax) node [left]{$y$};
% 			\clip (\xmin+0.1,\ymin+0.1) rectangle (\xmax-0.1,\ymax-0.1);
% 			\draw[smooth,samples=100,thick ] plot(\x,{\a^(\x)}) (.5,1) node [right]{$y=f(x)$};
% 		\end{tikzpicture}	
% 	}
% 	% \dapso{Hàm số đã cho liên tục trên $\mathbb{R}$.}
% 	\loigiai{
% 		Do đồ thị hàm số là một đường liền nét trên $\mathscr{D}$ nên hàm số $y=f(x)$ liên tục trên $\mathscr{D}=\mathbb{R}$.}
% \end{bt}
% \begin{bt}%[DCHT Toán 11 - KNTT -Vũ Hồng Toàn]%[1K5KG-2]
% 	\immini{
% 		Cho hàm số $y=f(x)$ xác định trên $\mathscr{D}=\mathbb{R}$ và có đồ thị như hình vẽ bên. Xét tính liên tục của hàm số $y=f(x)$ trên $\mathscr{D}$.	
% 	}
% 	{
% 		\begin{tikzpicture}[scale=1, font=\normalsize, line join=round, line cap=round,>=stealth]
% 			\def\xmin{-2} \def\xmax{3.5}
% 			\def\ymin{-.5} \def\ymax{2.5}
% 			\draw[->] (\xmin,0)--(0,0)node [below right]{$O$}-- (\xmax,0) node [below]{$x$};
% 			\draw[->] (0,\ymin)--(0,\ymax) node [left]{$y$};
% 			\clip (\xmin+0.1,\ymin+0.1) rectangle (\xmax-0.1,\ymax-0.1);
% 			\draw[smooth,samples=100,thick ] plot[domain=-1.5:1.5](\x,{(\x)^2})--(3,-.25) 
% 			(-1,1) node [left]{$y=f(x)$}; 
% 			\draw[dashed] (1.5,2.25)--(1.5,0) node [below]{$2$} ;
% 		\end{tikzpicture}	
% 	}
	
% 	% \dapso{Hàm số đã cho liên tục trên $\mathbb{R}$.}
% 	\loigiai{
% 		Do đồ thị hàm số là một đường liền nét trên $\mathscr{D}$ nên hàm số $y=f(x)$ liên tục trên $\mathscr{D}=\mathbb{R}$.}
% \end{bt}
% \begin{bt}%[DCHT Toán 11 - KNTT -Vũ Hồng Toàn]%[1K5KG-2]
% 	\immini{
% 		Cho hàm số $y=f(x)$ có đồ thị như hình vẽ bên.\\ Xét tính liên tục của hàm số $y=f(x)$ tại điểm $x_0=2$.	
% 	}
% 	{
% 		\begin{tikzpicture}[scale=1, font=\normalsize, line join=round, line cap=round,>=stealth]
% 			\def\xmin{-2} \def\xmax{3.5}
% 			\def\ymin{-.5} \def\ymax{2.5}
% 			\draw[->] (\xmin,0)--(0,0)node [below right]{$O$}-- (\xmax,0) node [below]{$x$};
% 			\draw[->] (0,\ymin)--(0,\ymax) node [left]{$y$};
% 			\clip (\xmin+0.1,\ymin+0.1) rectangle (\xmax-0.1,\ymax-0.1);
% 			\draw[smooth,samples=100,thick ] plot[domain=-1.5:1.5](\x,{(\x)^2}) (1.5,1.5)--(3,-.25) 
% 			(-1,1) node [left]{$y=f(x)$}; 
% 			\draw[dashed] (1.5,2.25)--(1.5,0) node [below]{$2$} (1.5,1.5)--(0,1.5)node [left]{$2$};
% 			\foreach \x/\y in{0/0,1.5/0,1.5/1.5,0/1.5,1.5/2.25} \fill(\x,\y)circle(.03);
% 		\end{tikzpicture}	
% 	}
% 	% \dapso{Hàm số đã cho gián đoạn tại $x_0=2$. }
% 	\loigiai{
% 		Ta có $\lim\limits_{x\to{2}^-}f(x)>2$ và $\lim\limits_{x\to{2}^+}f(x)=2$. Do đó $\lim\limits_{x\to{2}^-}f(x)\ne \lim\limits_{x\to{2}^+}f(x)$. Vậy hàm số đã cho gián đoạn tại $x_0=2$.
% 	}
% \end{bt}

% \begin{bt}%[DCHT Toán 11 - KNTT -Vũ Hồng Toàn]%[1K5KG-2]
% 	\immini{
% 		Cho hàm số $y=f(x)$ có đồ thị như hình vẽ bên.\\ Xét tính liên tục của hàm số $y=f(x)$ tại điểm $x_0=1$.	
% 	}
% 	{
% 		\begin{tikzpicture}[scale=.5, font=\footnotesize, line join=round, line cap=round,>=stealth]
% 			\def\a{0} \def\b{1} \def\c{1} \def\d{-1}
% 			\pgfmathsetmacro\tcd{int(round(-\d/\c))} \pgfmathsetmacro\tcn{int(round(\a/\c))} 
% 			\pgfmathsetmacro\xmin{\tcd-2.5} \pgfmathsetmacro\xmax{\tcd+2.5}
% 			\pgfmathsetmacro\ymin{\tcn-2.5} \pgfmathsetmacro\ymax{\tcn+2.5}
% 			\draw[->] (\xmin,0)--(0,0)node [below right]{$O$}-- (\xmax,0) node [below]{$x$};
% 			\draw[->] (0,\ymin)--(0,\ymax) node [left]{$y$};
% 			\draw[dash pattern=on 2pt off 1.5pt] (\tcd,\ymin)--(\tcd,\ymax) (\tcd,0)node [below right]{$\tcd$};
% 			\begin{scope}
% 				\clip (\xmin,\ymin) rectangle (\xmax-.1,\ymax-.1);
% 				\draw[samples=200,smooth,variable=\x,thick, teal] plot[domain=\xmin:\tcd-0.3] (\x,{(\a*(\x)+\b)/(\c*(\x)+\d)});
% 				\draw[samples=200,smooth,variable=\x,thick, teal] plot[domain=\tcd+.3:\xmax] (\x,{(\a*(\x)+\b)/(\c*(\x)+\d)});
% 			\end{scope}
% 			\draw[ thick] (1.2,-1.5)node[right]{$y=f(x)$};
% 			\foreach \x/\y in{0/0,1/0} \fill(\x,\y)circle(.03);
% 		\end{tikzpicture}	
% 	}
% 	% \dapso{Hàm số đã cho gián đoạn tại $x_0=1$. }
% 	\loigiai{
% 		Ta có $\lim\limits_{x\to{1}^-}f(x)=-\infty$ và $\lim\limits_{x\to{1}^+}f(x)=+\infty$. Do đó $\lim\limits_{x\to{1}^-}f(x)\ne \lim\limits_{x\to{1}^+}f(x)$.\\ Vậy hàm số đã cho gián đoạn tại $x_0=1$.
% 	}
% \end{bt}

\subsubsection{Câu hỏi trắc nghiệm}
\Opensolutionfile{ans}[ans/ans-1K5-3-Dang1]

\begin{ex}%[DCHT Toán 11 - KNTT -Vũ Hồng Toàn]%[1K5YG-2]
	\immini{
		Cho đồ thị hàm số $y=f(x)$ có đồ thị như hình vẽ bên. Chọn mệnh đề đúng trong các mệnh đề sau:
		\choice[1]
		{Hàm số $y=f(x)$ liên tục trên $\mathbb{R}$}
		{\True Hàm số $y=f(x)$ liên tục trên khoảng $(0,+\infty)$}
		{Hàm số $y=f(x)$ liên tục tại điểm $x_0=0$}
		{$\lim\limits_{x\to{0}^+}f(x)=+\infty$}	
	}
	{
		\begin{tikzpicture}[scale=.75, font=\footnotesize, line join=round, line cap=round,>=stealth]
			\def\xmin{-.5} \def\xmax{3.5}
			\def\ymin{-2.5} \def\ymax{1.5}
			\draw[->] (\xmin,0)--(0,0)node [below left]{$O$}-- (\xmax,0) node [below]{$x$};
			\draw[->] (0,\ymin)--(0,\ymax) node [left]{$y$};
			\clip (\xmin+0.1,\ymin+0.1) rectangle (\xmax-0.1,\ymax-0.1);
			\draw[smooth,samples=100,thick ] plot[domain=.1:3](\x,{ln(\x)}) 
			(1,-1) node [right]{$y=f(x)$}; 
		\end{tikzpicture}	
	}	
	\loigiai
	{
		Đồ thị hàm số là một đường liền nét trên khoảng $(0,+\infty)$ nên liên tục trên khoảng này.
	}
\end{ex}
%Cau2
\begin{ex}%[DCHT Toán 11 - KNTT -Vũ Hồng Toàn]%[1K5YG-2]
	\immini{
		Cho đồ thị hàm số $y=f(x)$ có đồ thị như hình vẽ bên. Chọn mệnh đề \textbf{sai} trong các mệnh đề sau:
		\choice[1]
		{\True $\lim\limits_{x\to{+\infty}}f(x)=-\infty$}
		{Hàm số $y=f(x)$ liên tục trên $\mathbb{R}$}
		{$\lim\limits_{x\to{+\infty}}f(x)=+\infty$}
		{$\lim\limits_{x\to{-\infty}}f(x)=-\infty$}	
	}
	{
		\begin{tikzpicture}[scale=.75, font=\normalsize, line join=round, line cap=round,>=stealth]
			\def\a{1} \def\b{0} \def\c{-3} \def\d{1}
			\def\f(#1){\a*(#1)^3+\b*(#1)^2+\c*(#1)+\d}
			\pgfmathsetmacro\tdx{int(round(-\b/(3*\a)))} \pgfmathsetmacro\tdy{int(round(\f(\tdx))}
			\pgfmathsetmacro\xmin{\tdx-2.5} \pgfmathsetmacro\xmax{\tdx+2.5}
			\pgfmathsetmacro\ymin{\tdy-2.5} \pgfmathsetmacro\ymax{\tdy+2.5}
			%\draw[color=gray!50,dashed] (\xmin,\ymin) grid (\xmax,\ymax);
			\draw[->] (\xmin,0)--(0,0)node [above left]{$O$}-- (\xmax,0) node [below]{$x$};
			\draw[->] (0,\ymin)--(0,\ymax) node [left]{$y$};
			\begin{scope}
				\clip (\xmin,\ymin) rectangle (\xmax-.1,\ymax-.1);
				\draw[samples=100,smooth,thick, teal] plot(\x,{\f(\x)});
			\end{scope}
			\draw[ thick] (-2,-1)node[right]{$y=f(x)$};
			\foreach \x/\y in{0/0,1/0} \fill(\x,\y)circle(.03);
		\end{tikzpicture} 	
	}
	\loigiai
	{
		Quan sát đồ thị hàm số, ta có
		\begin{itemize}
			\item Hàm số $y=f(x)$ liên tục trên $\mathbb{R}$.
			\item $\lim\limits_{x\to{+\infty}}f(x)=+\infty$.
			\item $\lim\limits_{x\to{-\infty}}f(x)=-\infty$.
		\end{itemize}
	}
\end{ex}
%Cau3
\begin{ex}%[DCHT Toán 11 - KNTT -Vũ Hồng Toàn]%[1K5YG-2]
	\immini{
		Cho đồ thị hàm số $y=f(x)$ có đồ thị như hình vẽ bên. Chọn mệnh đề đúng trong các mệnh đề sau:
		\choice
		{$\lim\limits_{x\to{+\infty}}f(x)=-\infty$}
		{\True Hàm số $y=f(x)$ liên tục trên khoảng $(0;+\infty)$}
		{Hàm số $y=f(x)$ liên tục tại điểm $x_0=0$}
		{Hàm số $y=f(x)$ liên tục trên $\mathbb{R}$}	
	}
	{
		\begin{tikzpicture}[scale=1, font=\normalsize, line join=round, line cap=round,>=stealth]
			\def\xmin{-.5} \def\xmax{3.5}
			\def\ymin{-.5} \def\ymax{2.5}
			%\draw[color=gray!50,dashed] (\xmin,\ymin) grid (\xmax,\ymax);
			\draw[->] (\xmin,0)--(0,0)node [below right]{$O$}-- (\xmax,0) node [below]{$x$};
			\draw[->] (0,\ymin)--(0,\ymax) node [left]{$y$};
			\clip (\xmin+0.1,\ymin+0.1) rectangle (\xmax-0.1,\ymax-0.1);
			\draw[smooth,samples=100,thick ] plot[domain=0:3.5](\x,{sqrt(\x)}) 
			(1,.8) node [right]{$y=f(x)$}; 
		\end{tikzpicture}	
	}
	\loigiai
	{
		Quan sát đồ thị hàm số, thấy đồ thị hàm số là một đường liền nét trên khoảng $(0;+\infty)$ do đó hàm số đã cho liên tục trên khoảng này.
	}
\end{ex}
%Cau4
\begin{ex}%[DCHT Toán 11 - KNTT -Vũ Hồng Toàn]%[1K5YG-2]
	\immini{
		Cho đồ thị hàm số $y=f(x)$ có đồ thị như hình vẽ bên. Chọn mệnh đề \textbf{sai} trong các mệnh đề sau:
		\choice[1]
		{Hàm số gián đoạn tại điểm $x_0=0$}
		{Hàm số liên tục trên khoảng $(-\infty;0)$}
		{\True Hàm số $y=f(x)$ liên tục trên $\mathbb{R}$}
		{Hàm số liên tục trên khoảng $(0;+\infty)$}	
	}
	{
		\begin{tikzpicture}[scale=.75, font=\footnotesize, line join=round, line cap=round,>=stealth]
			\def\xmin{-2.5} \def\xmax{2.8}
			\def\ymin{-1.5} \def\ymax{2.8}
			%\draw[color=gray!50,dashed] (\xmin,\ymin) grid (\xmax,\ymax);
			\draw[->] (\xmin,0)--(0,0)node [below right]{$O$}-- (\xmax,0) node [below]{$x$};
			\draw[->] (0,\ymin)--(0,\ymax) node [left]{$y$};
			\begin{scope}
				\clip (\xmin,\ymin) rectangle (\xmax,\ymax-.15);
				\draw[samples=200,smooth,variable=\x,thick, teal] plot[domain=\xmin:-0.5] (\x,{1/(\x)^2-1});
				\draw[samples=200,smooth,variable=\x,thick, teal] plot[domain=0.5:\xmax] (\x,{1/(\x)^2-1});
			\end{scope}
			\draw[dashed](\xmin,-1)--(\xmax,-1) (0,-1)node[below right]{$-1$};
			\draw[ thick] (1.2,1)node[right]{$y=f(x)$};
			\foreach \x/\y in{0/0} \fill(\x,\y)circle(.03);
		\end{tikzpicture}	
	}
	
	\loigiai
	{
		Quan sát đồ thị hàm số, ta có
		\begin{itemize}
			\item Hàm số gián đoạn tại điểm $x_0=0$.
			\item Hàm số liên tục trên khoảng $(-\infty;0)$.
			\item Hàm số liên tục trên khoảng $(0;+\infty)$.
		\end{itemize}
	}
\end{ex}
%Cau5
\begin{ex}%[DCHT Toán 11 - KNTT -Vũ Hồng Toàn]%[1K5BG-2]
	\immini{
		Cho đồ thị hàm số $y=f(x)$ có đồ thị như hình vẽ bên. Chọn mệnh đề \textbf{sai} trong các mệnh đề sau:
		\choice[1]
		{Hàm số gián đoạn tại điểm $x_0=1$}
		{Hàm số liên tục trên khoảng $(-\infty;1)$}
		{\True Hàm số $y=f(x)$ liên tục trên $\mathbb{R}$}
		{Hàm số liên tục trên khoảng $(1;+\infty)$}		
	}
	{
		\begin{tikzpicture}[scale=.75, font=\footnotesize, line join=round, line cap=round,>=stealth]
			\def\a{1} \def\b{-2} \def\c{1} \def\d{-1}
			\pgfmathsetmacro\tcd{int(round(-\d/\c))} \pgfmathsetmacro\tcn{int(round(\a/\c))} 
			\pgfmathsetmacro\xmin{\tcd-2.5} \pgfmathsetmacro\xmax{\tcd+2.5}
			\pgfmathsetmacro\ymin{\tcn-2.5} \pgfmathsetmacro\ymax{\tcn+2.5}
			%\draw[color=gray!50,dashed] (\xmin,\ymin) grid (\xmax,\ymax);
			\draw[->] (\xmin,0)--(0,0)node [below right]{$O$}-- (\xmax,0) node [below]{$x$};
			\draw[->] (0,\ymin)--(0,\ymax) node [left]{$y$};
			\draw[dash pattern=on 2pt off 1.5pt] (\tcd,\ymin)--(\tcd,\ymax) (\tcd,0)node [below right]{$\tcd$}
			(\xmin,\tcn)--(\xmax,\tcn) (0,\tcn)node [below right]{$\tcn$};
			\begin{scope}
				\clip (\xmin,\ymin) rectangle (\xmax-.1,\ymax-.1);
				\draw[samples=200,smooth,variable=\x,thick, teal] plot[domain=\xmin:\tcd-0.3] (\x,{(\a*(\x)+\b)/(\c*(\x)+\d)});
				\draw[samples=200,smooth,variable=\x,thick, teal] plot[domain=\tcd+.3:\xmax] (\x,{(\a*(\x)+\b)/(\c*(\x)+\d)});
			\end{scope}
			\draw[ thick] (1.5,-1)node[right]{$y=f(x)$};
			\foreach \x/\y in{0/0,1/0} \fill(\x,\y)circle(.03);
		\end{tikzpicture}	
	}
	\loigiai
	{
		Quan sát đồ thị hàm số, ta có
		\begin{itemize}
			\item Hàm số gián đoạn tại điểm $x_0=1$.
			\item Hàm số liên tục trên khoảng $(-\infty;1)$.
			\item Hàm số liên tục trên khoảng $(1;+\infty)$.
		\end{itemize}
	}
\end{ex}
%Cau6
\begin{ex}%[DCHT Toán 11 - KNTT -Vũ Hồng Toàn]%[1K5BG-2]
	\immini{
		Cho đồ thị hàm số $y=f(x)$ có đồ thị như hình vẽ bên. Chọn mệnh đề đúng trong các mệnh đề sau:
		\choice[1]
		{Hàm số $y=f(x)$ liên tục trên $\mathbb{R}$}
		{$\lim\limits_{x\to{+\infty}}f(x)=-\infty$}
		{Hàm số liên tục tại điểm $x_0=-2$}
		{\True Hàm số gián đoạn tại điểm $x_0=-2$}	
	}
	{
		\begin{tikzpicture}[scale=1, font=\normalsize, line join=round, line cap=round,>=stealth]
			\def\xmin{-3.3} \def\xmax{2}
			\def\ymin{-.5} \def\ymax{2.8}
			\draw[->] (\xmin,0)--(0,0)node [below right]{$O$}-- (\xmax,0) node [below]{$x$};
			\draw[->] (0,\ymin)--(0,\ymax) node [left]{$y$} ;
			\clip (\xmin+0.1,\ymin+0.1) rectangle (\xmax-0.1,\ymax-0.1);
			\draw[smooth,samples=100,thick ] plot[domain=-1.5:2](\x,{(\x)^2})  
			(-1,1) node [right]{$y=f(x)$}; 
			\draw[thick ] (-3,-.25)--(-1.5,2);
			\draw[dashed] (-1.5,0)node [below]{$-2$}|-(0,2.25) node [right]{$2$} ;
		\end{tikzpicture}	
	} 
	
	\loigiai
	{
		Quan sát đồ thị hàm số, ta có	
		\begin{itemize}
			\item Đồ thị hàm số là các đường liền nét trên các khoảng $(-\infty;-2)$, $(-2;+\infty)$ do đó hàm số liên tục trên các khoảng này.
			\item $\lim\limits_{x\to{+\infty}}f(x)=+\infty$
			\item Ta có $\lim\limits_{x\to{(-2)}^-}f(x)<f(-2)=2$ và $\lim\limits_{x\to{(-2)}^+}f(x)=f(-2)=2$. Do đó $\lim\limits_{x\to{(-2)}^-}f(x)\ne \lim\limits_{x\to{(-2)}^+}f(x)$.\\
			Vậy hàm số đã cho gián đoạn tại $x_0=-2$.
		\end{itemize}
	}
\end{ex}
%Cau7
\begin{ex}%[DCHT Toán 11 - KNTT -Vũ Hồng Toàn]%[1K5BG-2]
	\immini{
		Cho đồ thị hàm số $y=f(x)$ có đồ thị như hình vẽ bên. Chọn mệnh đề đúng trong các mệnh đề sau:
		\choice[1]
		{\True $\lim\limits_{x\to{1}^+}f(x)=+\infty$}
		{Hàm số $y=f(x)$ liên tục trên $\mathbb{R}$}
		{$\lim\limits_{x\to{1}^+}f(x)=-\infty$}
		{$\lim\limits_{x\to{1}^-}f(x)=+\infty$}	
	}
	{
		\begin{tikzpicture}[scale=.75, font=\footnotesize, line join=round, line cap=round,>=stealth]
			\def\a{-1} \def\b{2} \def\c{1} \def\d{-1}
			\pgfmathsetmacro\tcd{int(round(-\d/\c))} \pgfmathsetmacro\tcn{int(round(\a/\c))} 
			\pgfmathsetmacro\xmin{\tcd-2.5} \pgfmathsetmacro\xmax{\tcd+2.5}
			\pgfmathsetmacro\ymin{\tcn-2.5} \pgfmathsetmacro\ymax{\tcn+2.5}
			%\draw[color=gray!50,dashed] (\xmin,\ymin) grid (\xmax,\ymax);
			\draw[->] (\xmin,0)--(0,0)node [below right]{$O$}-- (\xmax,0) node [below]{$x$};
			\draw[->] (0,\ymin)--(0,\ymax) node [left]{$y$};
			\draw[dash pattern=on 2pt off 1.5pt] (\tcd,\ymin)--(\tcd,\ymax) (\tcd,0)node [below right]{$\tcd$}
			(\xmin,\tcn)--(\xmax,\tcn) (0,\tcn)node [below left]{$\tcn$};
			\begin{scope}
				\clip (\xmin,\ymin) rectangle (\xmax-.1,\ymax-.1);
				\draw[samples=200,smooth,variable=\x,thick, teal] plot[domain=\xmin:\tcd-0.3] (\x,{(\a*(\x)+\b)/(\c*(\x)+\d)});
				\draw[samples=200,smooth,variable=\x,thick, teal] plot[domain=\tcd+.3:\xmax] (\x,{(\a*(\x)+\b)/(\c*(\x)+\d)});
			\end{scope}
			\draw[ thick] (1.5,1)node[right]{$y=f(x)$};
			\foreach \x/\y in{0/0,1/0} \fill(\x,\y)circle(.03);
		\end{tikzpicture}	
	}
	
	\loigiai
	{
		Quan sát đồ thị hàm số, ta có	
		\begin{itemize}
			\item Đồ thị hàm số là các đường liền nét trên các khoảng $(-\infty;1)$, $(1;+\infty)$ do đó hàm số liên tục trên các khoảng này.
			\item $\lim\limits_{x\to{+\infty}}f(x)=-1$.
			\item Ta có $\lim\limits_{x\to{1}^-}f(x)=-\infty$ và $\lim\limits_{x\to{1}^+}f(x)=+\infty$. Do đó $\lim\limits_{x\to{1}^-}f(x)\ne \lim\limits_{x\to{1}^+}f(x)$.\\
			Vậy hàm số đã cho gián đoạn tại $x_0=1$.
		\end{itemize}
	}
\end{ex}
%Cau8
\begin{ex}%[DCHT Toán 11 - KNTT -Vũ Hồng Toàn]%[1K5KG-2]
	\immini{
		Cho đồ thị hàm số $y=f(x)$ có đồ thị như hình vẽ bên. Chọn mệnh đề đúng trong các mệnh đề sau:
		\choice[1]
		{Hàm số $y=f(x)$ gián đoạn tại $x_0=0$}
		{$\lim\limits_{x\to{+\infty}}f(x)=-1$}
		{$\lim\limits_{x\to{-\infty}}f(x)=-1$}
		{\True Hàm số $y=f(x)$ liên tục trên $\mathbb{R}$}	
	}
	{
		\begin{tikzpicture}[scale=1, font=\normalsize, line join=round, line cap=round,>=stealth]
			%	\def\a{2} %\def\b{-2} \def\c{0} %\def\d{1} % Hệ số
			\def\xmin{-1.5} \def\xmax{2.5}
			\def\ymin{-1.3} \def\ymax{2.5}
			%\draw[color=gray!50,dashed] (\xmin,\ymin) grid (\xmax,\ymax);
			\draw[->] (\xmin,0)--(0,0)node [below right]{$O$}-- (\xmax,0) node [below]{$x$};
			\draw[->] (0,\ymin)--(0,\ymax) node [left]{$y$};
			\clip (\xmin+0.1,\ymin+0.1) rectangle (\xmax-0.1,\ymax-0.1);
			\draw[thick ] (-1.5,-1.2)--(-.5,1)--(0,0)--(1,2)--(2,-1.2) 
			(1.7,-1) node [left]{$y=f(x)$}; 
		\end{tikzpicture}	
	}
	
	\loigiai
	{
		Quan sát đồ thị hàm số, ta có	
		\begin{itemize}
			\item Đồ thị hàm số là các đường liền nét trên khoảng $(-\infty;+\infty)$ do đó hàm số liên tục trên $\mathbb{R}$.
			\item $\lim\limits_{x\to{+\infty}}f(x)=-\infty$.
			\item $\lim\limits_{x\to{-\infty}}f(x)=-\infty$.
		\end{itemize}
	}
\end{ex}
%Cau9
\begin{ex}%[DCHT Toán 11 - KNTT -Vũ Hồng Toàn]%[1K5KG-2]
	\immini{
		Cho đồ thị hàm số $y=f(x)$ có đồ thị như hình vẽ bên. Chọn mệnh đề đúng trong các mệnh đề sau:
		\choice[1]
		{Hàm số $y=f(x)$ liên tục trên $\mathbb{R}$}
		{Hàm số $y=f(x)$ liên tục tại $x_0=-3$}
		{$\lim\limits_{x\to{-\infty}}f(x)=-\infty$}
		{\True Hàm số $y=f(x)$ gián đoạn tại $x_0=3$}	
	}
	{
		\begin{tikzpicture}[scale=.75, font=\normalsize, line join=round, line cap=round,>=stealth]
			\def\xmin{-5.5} \def\xmax{5.8}
			\def\ymin{-2.5} \def\ymax{2.8}
			%\draw[color=gray!50,dashed] (\xmin,\ymin) grid (\xmax,\ymax);
			\draw[->] (\xmin,0)--(0,0)node [above right]{$O$}-- (\xmax,0) node [below]{$x$};
			\draw[->] (0,\ymin)--(0,\ymax) node [left]{$y$};
			\draw[dash pattern=on 2pt off 1.5pt] (-3,\ymin)--(-3,\ymax) (3,\ymin)--(3,\ymax)
			(-3,0)node [below right]{$-3$} (3,0)node [below right]{$3$}
			;
			\begin{scope}
				\clip (\xmin,\ymin) rectangle (\xmax,\ymax-.15);
				\draw[samples=200,smooth,variable=\x,thick, teal] plot[domain=\xmin:-3.01] (\x,{(\x+2)/((\x)^2-9)});
				\draw[samples=200,smooth,variable=\x,thick, teal] plot[domain=-2.99:2.99] (\x,{(\x+2)/((\x)^2-9)});
				\draw[samples=200,smooth,variable=\x,thick, teal] plot[domain=3.01:\xmax] (\x,{(\x+2)/((\x)^2-9)});
			\end{scope}
			\draw[ thick] (-2.8,1)node[right]{$y=f(x)$};
			\foreach \x/\y in{0/0,-3/0,3/0} \fill(\x,\y)circle(.05);
		\end{tikzpicture} 	
	}
	\loigiai
	{
		Quan sát đồ thị hàm số, ta có	
		\begin{itemize}
			\item Đồ thị hàm số là các đường liền nét trên khoảng $(-\infty;-3)$, $(-3;3)$, $(3;+\infty) $ do đó hàm số liên tục trên các khoảng này.
			\item $\lim\limits_{x\to{-\infty}}f(x)=0$.
			\item Hàm số $y=f(x)$ gián đoạn tại các điểm $x_0=3$ và $x_0=-3$.
		\end{itemize}
	}
\end{ex}
%Cau10
\begin{ex}%[DCHT Toán 11 - KNTT -Vũ Hồng Toàn]%[1K5GG-2]
	\immini{
		Cho đồ thị hàm số $y=f(x)$ có đồ thị như hình vẽ bên. Chọn mệnh đề đúng trong các mệnh đề sau:
		\choice[1]
		{Hàm số $y=f(x)$ liên tục trên $\mathbb{R}$}
		{Hàm số $y=f(x)$ liên tục tại điểm $x_0=\dfrac{\pi}{2}$}
		{Hàm số $y=f(x)$ liên tục tại điểm $x_0=-\dfrac{\pi}{2}$}
		{\True Hàm số $y=f(x)$ gián đoạn tại điểm $x_0=\dfrac{\pi}{2}$}
	}
	{
		\begin{tikzpicture}[line join=round, line cap=round,thick,>=stealth,scale=.75 ]
			\draw[->](-1.6*pi,0)--(1.8*pi,0) node[below]{$x$};
			\draw[->](0,-3)--(0,3) node[left]{$y$};
			\draw (0,0) node[above left]{$O$};
			\foreach \i in {-1,0,1}{
				\pgfmathsetmacro{\start}{\i*pi-1.25}
				\pgfmathsetmacro{\left}{(\i-0.5)*pi}
				\pgfmathsetmacro{\end}{\i*pi+1.25}
				\draw[dashed, thin](\left,-3)--(\left,3);
				\draw[domain=\start:\end,samples=100,smooth ] plot(\x,{tan(\x r)}) (-3.7,-1.5) node[right]{$y=f(x)$};
			}
			\draw[dashed,thin] (1.5*pi,-3)--(1.5*pi,3);
			\draw (-pi/2,0) node [below left]{$-\pi/2$} (pi/2,0) node [below left]{$\pi/2$};
		\end{tikzpicture}	
	}
	\loigiai
	{
		Quan sát đồ thị hàm số, ta có	
		\begin{itemize}
			\item Đồ thị hàm số là các đường liền nét trên khoảng $(-\dfrac{\pi}{2};0)$, $(0;\dfrac{\pi}{2}) $ do đó hàm số liên tục trên các khoảng này.
			\item Hàm số $y=f(x)$ gián đoạn tại các điểm $x_0=\dfrac{\pi}{2}$ và $x_0=-\dfrac{\pi}{2}$.
		\end{itemize}
	}
\end{ex}
\Closesolutionfile{ans}
% \begin{indapan}{10}
% 	{ans/ans-1K5-3-Dang1}
% \end{indapan}

\begin{dang}{Hàm số liên tục tại một điểm}
	Để kiểm tra tính liên tục của hàm số $y=f(x)$ tại điểm $x=x_0$ ta cần làm như sau:
	\begin{itemize}
		\item Bước 1: Tính $\lim \limits_{x \to x_0} f(x)$.
		\item Bước 2: Tính $f(x_0)$.
		Nếu $\lim \limits_{x \to x_0} f(x) = f(x_0)$ thì kết luận hàm số $f(x)$ liên tục tại $x=x_0$.
		Nếu $\lim \limits_{x \to x_0} f(x) \ne f(x_0)$ thì kết luận hàm số $f(x)$ liên tục tại $x=x_0$.
	\end{itemize}
\end{dang}
\subsubsection{Ví dụ mẫu}
\begin{vd}%[DCHT Toán 11 - KNTT -Hứa Chí Ninh]%[1K5BG-3]
	Xét tính liên tục của hàm số $f(x)=\heva{&\dfrac{x^2-3x+2}{x-2}&\text{khi }x\neq 2\\&4x-7&\text{khi }x=2}$ tại điểm $x_0=2$.
	\loigiai{
		Ta có $f(x_0)=f(2)=4\cdot 2-7=1$.\\
		$\lim\limits_{x\rightarrow2}f(x)=\lim\limits_{x\rightarrow2}\dfrac{x^2-3x+2}{x-2}=\lim\limits_{x\rightarrow2}\dfrac{(x-1)(x-2)}{x-2}=\lim\limits_{x\rightarrow2}(x-1)=1$.\\
		Suy ra $f(2)=\lim\limits_{x\rightarrow 2}f(x)$ nên hàm số $f(x)$ liên tục tại điểm $x_0=2$.
	}
\end{vd}

\begin{vd}%[DCHT Toán 11 - KNTT -Hứa Chí Ninh]%[1K5BG-3]
	Xét tính liên tục của hàm số $f(x)=\begin{cases}
		\dfrac{1-\sqrt{2x-3}}{2-x}&\text{ nếu }\:x\ne2\\
		1&\text{ nếu }\:x=2
	\end{cases}$
	tại điểm $x_0=2$.
	\loigiai{Ta có
		\begin{itemize}
			\item $f(2)=1$.
			\item $\lim\limits_{x\to 2}f(x)=\lim\limits_{x\to 2}\dfrac{1-\sqrt{2x-3}}{2-x}=\lim\limits_{x\to 2}\dfrac{1-(2x-3)}{(2-x)(1+\sqrt{2x-3})}=\lim\limits_{x\to 2}\dfrac{2(2-x)}{(2-x)(1+\sqrt{2x-3})}\\
			=\lim\limits_{x\to 2}\dfrac{2}{1+\sqrt{2x-3}}=1=f(2)$
		\end{itemize}
		Vậy hàm số $f(x)$ liên tục tại $x_0=2$.}
\end{vd}

\begin{vd}%[DCHT Toán 11 - KNTT -Hứa Chí Ninh]%[1K5BG-3]
	Biết rằng $\lim\limits_{x\to 0} \dfrac{\sin x}{x}=1$. Hàm số $f\left( x \right)=\heva{&\dfrac{\tan x}{x} & \text{ khi }x\ne 0 \\&0 & \text{ khi }x=0}$. Xét tính liên tục của $y=f(x)$ tại $x=0$?
	 % \dapso{ $f\left( x \right)$ không liên tục tại $x=0$.}
	\loigiai{	Tập xác định $\mathscr{D}=\mathbb{R}\setminus \left\{ \dfrac{\pi }{2}+k\pi |k\in \mathbb{Z} \right\}$.\\
		Ta có $\lim\limits_{x\to 0} f\left( x \right)=\lim\limits_{x\to 0} \dfrac{\tan x}{x}=\lim\limits_{x\to 0} \dfrac{\sin x}{x}\cdot\dfrac{1}{\cos x}=1\cdot\dfrac{1}{\cos 0}=1\ne f\left( 0 \right)\Rightarrow $ $f\left( x \right)$ không liên tục tại $x=0$.}
\end{vd}
\begin{vd}%[DCHT Toán 11 - KNTT -Hứa Chí Ninh]%[1K5BG-3]
	Hàm số $f\left( x \right)=\left\{ \begin{array}{*{35}{l}}
		3 & \text{khi}\,\,x=-1 \\
		\dfrac{x^4+x}{x^2+x} & \text{khi}\,\,x\ne -1,\,\,x\ne 0 \\
		1 & \text{khi}\,\,x=0 \\
	\end{array} \right.$. Xét tính liên tục của hàm số tại $x=-1, x=0.$
	% \dapso{Hàm số liên tục tại $x=-1, x=0.$}
	\loigiai{
		Hàm số $y=f\left( x \right)$ có tập xác định $\mathscr{D}=\mathbb{R}$.\\
		Dễ thấy hàm số $y=f\left( x \right)$ liên tục trên mỗi khoảng $\left( -\infty ;-1 \right),\left( -1;0 \right)$ và $\left( 0;+\infty \right)$.\\
		(i) Xét tại $x=-1$, ta có\\
		$\lim\limits_{x\to -1} f\left( x \right)=\lim\limits_{x\to -1} \dfrac{x^4+x}{x^2+x}=\lim\limits_{x\to -1} \dfrac{x\left( x+1 \right)\left( {x^2}-x+1 \right)}{x\left( x+1 \right)}=\lim\limits_{x\to -1} \left( {x^2}-x+1 \right)=3=f\left( -1 \right).$ Vậy hàm số $y=f\left( x \right)$ liên tục tại $x=-1$.\\
		(ii) Xét tại $x=0$, ta có\\
		$\lim\limits_{x\to 0} f\left( x \right)=\lim\limits_{x\to 0} \dfrac{x^4+x}{x^2+x}=\lim\limits_{x\to 0} \dfrac{x\left( x+1 \right)\left( {x^2}-x+1 \right)}{x\left( x+1 \right)}=\lim\limits_{x\to 0} \left( {x^2}-x+1 \right)=1=f\left( 0 \right).$ Vậy hàm số $y=f\left( x \right)$ liên tục tại $x=0$.
	}
\end{vd}

\begin{vd}%[DCHT Toán 11 - KNTT -Hứa Chí Ninh]%[1K5BG-3]
	Xét tính liên tục của hàm số $f(x)=\begin{cases}
		x^2+1&\text{ nếu }\:x>0\\
		x&\text{ nếu }\:x\leq0
	\end{cases}$ tại điểm $x_0=0$.
	\loigiai{Ta có:
		\begin{itemize}
			\item $f(0)=0$.
			\item $\lim\limits_{x\to0^{+}}f(x)=\lim\limits_{x\to0^{+}}(x^2+1)=1$.
			\item $\lim\limits_{x\to0^{-}}f(x)=\lim\limits_{x\to0^{-}}x=0$.
		\end{itemize}
		Ta có: $f(0)=\lim\limits_{x\to0^{-}}f(x)\ne\lim\limits_{x\to0^{+}}f(x)$. Vậy hàm số $f(x)$ gián đoạn tại điểm $x=0$.
	}
\end{vd}

\begin{vd}%[DCHT Toán 11 - KNTT -Hứa Chí Ninh]%[1K5BG-3]
	Xét tính liên tục của hàm số $f\left( x \right)=\left\{ \begin{array}{*{35}{l}}
		1-\cos x & \text{khi }x\le 0 \\
		\sqrt{x+1} & \text{khi }x>0 \\
	\end{array} \right.$ tại $x=0?$
	% \dapso{Hàm số $y=f\left( x \right)$ gián đoạn tại $x=0$.}
	\loigiai{
		Hàm số xác định với mọi $x\in \mathbb{R}$.\\
		Ta có $f\left( x \right)$ liên tục trên $\left( -\infty ;0 \right)$ và $\left( 0;+\infty \right).$\\
		Mặt khác $\left\{ \begin{aligned}
			& f\left( 0 \right)=1 \\
			& \lim\limits_{x\to {0^{-}}} f\left( x \right)=\lim\limits_{x\to {0^{-}}} \left( 1-\cos x \right)=1-\cos 0=0 \\
			& \lim\limits_{x\to {0^{+}}} f\left( x \right)=\lim\limits_{x\to {0^{+}}} \sqrt{x+1}=\sqrt{0+1}=1 \\
		\end{aligned} \right.\Rightarrow f\left( x \right)$ gián đoạn tại $x=0$.
	}
\end{vd}

\subsubsection{Bài tập rèn luyện}
% \centerline{\fcolorbox{red}{yellow!50}{\bf {BÀI TẬP TỰ LUẬN}}}
\begin{bt}%[DCHT Toán 11 - KNTT -Hứa Chí Ninh]%[1K5BG-3]
	Xét tính liên tục của hàm số $f(x)=\begin{cases}
		\dfrac{x^2-6x+5}{x^2-1}&\text{ nếu }\: x\ne 1\\
		-2&\text{ nếu }\: x=1
	\end{cases}$ tại điểm $x_0=1$.
	\loigiai{Ta có: $f(1)=-2$.\\
		$\lim\limits_{x\to1}f(x)=\lim\limits_{x\to1}\dfrac{x^2-6x+5}{x^2-1}=\lim\limits_{x\to1}\dfrac{(x-5)(x-1)}{(x-1)(x+1)}=\lim\limits_{x\to1}\dfrac{x-5}{x+1}=-2=f(1)$.\\
		Vậy hàm số $f(x)$ liên tục tại $x=1$.}
\end{bt}
\begin{bt}%[DCHT Toán 11 - KNTT -Hứa Chí Ninh]%[1K5BG-3]
	Xét tính liên tục của hàm số $f(x)=\begin{cases}
		\dfrac{\sqrt{x}-2}{\sqrt{x+5}-3}&\text{ nếu }\:x\ne4\\
		-\dfrac{3}{2}&\text{ nếu }\:x=4
	\end{cases}$ tại điểm $x_0=4$.
	\loigiai{
		Ta có:
		\begin{itemize}
			\item $f(4)=-\dfrac{3}{2}$.
			\item $\lim\limits_{x\to 4}f(x)=\lim\limits_{x\to 4}\dfrac{\sqrt{x}-2}{\sqrt{x+5}-3}=\lim\limits_{x\to 4}\dfrac{(\sqrt{x+5}+3)(x-4)}{(x+5-9)(\sqrt{x}+2)}$\\
			$=\lim\limits_{x\to 4}\dfrac{\sqrt{x+5}+3}{\sqrt{x}+2}=\dfrac{6}{4}=\dfrac{3}{2}\ne f(4)$. 
		\end{itemize}
		Vậy hàm số $f(x)$ gián đoạn tại điểm $x=4$.
	}
\end{bt}

\begin{bt}%[DCHT Toán 11 - KNTT -Hứa Chí Ninh]%[1K5BG-3]
	Cho hàm số $f\left( x \right)=\left\{ \begin{aligned}
		& \dfrac{x^2}{x}\text{ khi }x<1,x\ne 0 \\
		& 0\text{ khi }x=0 \\
		& \sqrt{x}\text{ khi }x\ge 1 \\
	\end{aligned} \right..$ Xét tính liên tục của hàm số $f\left( x \right)$ tại $x=0, x=1?$
	% \dapso{Hàm số $y=f\left( x \right)$ liên tục tại $x=0$ và $x=1$.}
	\loigiai{
		Hàm số $y=f\left( x \right)$ có tập xác định $\mathscr{D}=\mathbb{R}$.\\
		Dễ thấy hàm số $y=f\left( x \right)$ liên tục trên mỗi khoảng $\left( -\infty ;0 \right),\left( 0;1 \right)$ và $\left( 1;+\infty \right)$.\\
		Ta có $\left\{ \begin{aligned}
			& f\left( 0 \right)=0 \\
			& \lim\limits_{x\to {0^{-}}} f\left( x \right)=\lim\limits_{x\to {0^{-}}} \dfrac{x^2}{x}=\lim\limits_{x\to {0^{-}}} x=0 \\
			& \lim\limits_{x\to {0^{+}}} f\left( x \right)=\lim\limits_{x\to {0^{+}}} \dfrac{x^2}{x}=\lim\limits_{x\to {0^{+}}} x=0 \\
		\end{aligned} \right.\Rightarrow f\left( x \right)$ liên tục tại $x=0.$\\
		Ta có $\left\{ \begin{aligned}
			& f\left( 1 \right)=1 \\
			& \lim\limits_{x\to {1^{-}}} f\left( x \right)=\lim\limits_{x\to {1^{-}}} \dfrac{x^2}{x}=\lim\limits_{x\to {1^{-}}} x=1 \\
			& \lim\limits_{x\to {1^{+}}} f\left( x \right)=\lim\limits_{x\to {1^{+}}} \sqrt{x}=1 \\
		\end{aligned} \right.\Rightarrow f\left( x \right)$ liên tục tại $x=1.$
	}
\end{bt}

\begin{bt}%[DCHT Toán 11 - KNTT -Hứa Chí Ninh]%[1K5BG-3]
	Cho hàm số $y=\left\{ \begin{aligned}
		& \dfrac{1-x^3}{1-x},\text{khi  }x<1 \\
		& 1\text{  },\text{khi  }x\ge 1 \\
	\end{aligned} \right.$. Xét tính liên tục phải của hàm số tại $x=1$?
	% \dapso{Hàm số liên tục phải tại $x=1.$}
	\loigiai{
		Ta có   $y\left( 1 \right)=1$.\\
		Ta có   $\lim\limits_{x\to {1^{+}}} y=1$; $\lim\limits_{x\to {1^{-}}} y=\lim\limits_{x\to {1^{-}}} \dfrac{1-x^3}{1-x}=\lim\limits_{x\to {1^{-}}} \dfrac{\left( 1-x \right)\left( 1+x+x^2 \right)}{1-x}=\lim\limits_{x\to {1^{-}}} \left( 1+x+x^2 \right)=4$.\\
		Nhận thấy $\lim\limits_{x\to {1^{+}}} y=y\left( 1 \right)$, suy ra $y$ liên tục phải tại $x=1$.}
\end{bt}
\begin{bt}%[DCHT Toán 11 - KNTT -Hứa Chí Ninh]%[1K5BG-3]
	
	Cho hàm số $y=\left\{ \begin{aligned}
		& \dfrac{x^2-7x+12}{x-3} & \text{ khi } x\ne 3 \\
		& -1 &\text{ khi }x=3 \\
	\end{aligned} \right.$. Xét tính liên tục của hàm số tại $x=3$.
	% \dapso{Hàm số đã cho có đạo hàm tại $x=3$}
	\loigiai{
		$\lim\limits_{x\to 3} \dfrac{x^2-7x+12}{x-3}=\lim\limits_{x\to 3} \left( x-4 \right)=-1=y\left( 3 \right)$ nên hàm số liên tục tại $x_0=3$.}
\end{bt}
\begin{bt}%[DCHT Toán 11 - KNTT -Hứa Chí Ninh]%[1K5KG-3]
	Cho hàm số $f\left( x \right)=\left\{ \begin{aligned}
		& \dfrac{x-2}{\sqrt{x+2}-2}&\text{ khi }x\ne 2 \\
		& 4&\text{ khi }x=2 \\
	\end{aligned} \right.$. Xét tính liên tục của hàm số tại $x=2$?
	% \dapso{Hàm số liên tục tại $x=2.$}
	\loigiai{
		Tập xác định  $\mathscr{D}=\mathbb{R}$.\\
		$\lim\limits_{x\to 2} f\left( x \right)=\lim\limits_{x\to 2} \dfrac{x-2}{\sqrt{x+2}-2}=\lim\limits_{x\to 2} \dfrac{\left( x-2 \right)\left( \sqrt{x+2}+2 \right)}{x-2}=\lim\limits_{x\to 2} \left( \sqrt{x+2}+2 \right)=4$.\\
		$f\left( 2 \right)=4$
		$\Rightarrow \lim\limits_{x\to 2} f\left( x \right)=f\left( 2 \right)$.
		Vậy hàm số liên tục tại $x=2$.}
\end{bt}
\begin{bt}%[DCHT Toán 11 - KNTT -Hứa Chí Ninh]%[1K5KG-3]% 
	Cho hàm số $f\left( x \right)=\left\{ \begin{aligned}
		& \dfrac{1-\cos x}{x^2}& \text{ khi } x\ne 0 \\
		& 1 &\text{ khi } x=0 \\
	\end{aligned} \right.\,\,$. Xét tính liên tục của hàm số tại $x=0?$
	% \dapso{Hàm số gián đoạn tại $x=0.$}
	\loigiai{
		Hàm số xác định trên $\mathbb{R}$.\\
		Ta có $f\left( 0 \right)=1$ và $\lim\limits_{x\to 0} f\left( x \right)=\lim\limits_{x\to 0} \dfrac{1-\cos x}{x^2}=\lim\limits_{x\to 0} \dfrac{2{\sin ^2}\dfrac{x}{2}}{4\cdot{{\left( \dfrac{x}{2} \right)}^2}}=\dfrac{1}{2}.$\\
		Vì $f\left( 0 \right)\ne \lim\limits_{x\to 0} f\left( x \right)$ nên $f\left( x \right)$ gián đoạn tại $x=0$. Do đó $f\left( x \right)$không có đạo hàm tại $x=0$.\\
		Vì $\forall x\ne 0$, $f\left( x \right)=\dfrac{1-\cos x}{x^2}\ge 0$ nên $f\left( \sqrt{2} \right)>0.$ Vậy $f\left( x \right)$ gián đoạn tại $x=0$.}
\end{bt}
\begin{bt}%[DCHT Toán 11 - KNTT -Hứa Chí Ninh]%[1K5BG-4]
	Cho hàm số $f\left( x \right)=\heva{& -x\cos x &\text{ khi } x<0 \\& \dfrac{x^2}{1+x}&\text{ khi } 0\le x<1 \\& {x^3}&\text{ khi } x\ge 1}$. Xét tính liên tục của hàm số tại $x=0?$
	% \dapso{Hàm số gián đoạn tại $x=1.$}
	\loigiai{
		$\bullet$ $f\left( x \right)$ liên tục tại $x\ne 0$ và $x\ne 1$.\\
		$\bullet$ Tại $x=0$\\
		$\lim\limits_{x\to {0^{-}}} f\left( x \right)=\lim\limits_{x\to {0^{-}}} \left( -x\cos x \right)=0$, $\lim\limits_{x\to {0^{+}}} f\left( x \right)=\lim\limits_{x\to {0^{+}}} \dfrac{x^2}{1+x}=0$, $f\left( 0 \right)=0$.\\
		Suy ra $\lim\limits_{x\to {0^{-}}} f\left( x \right)=\lim\limits_{x\to {0^{+}}} f\left( x \right)=f\left( 0 \right)$. Hàm số liên tục tại $x=0$.\\
		$\bullet$ Tại $x=1$\\
		$\lim\limits_{x\to {1^{-}}} f\left( x \right)=\lim\limits_{x\to {1^{-}}} \dfrac{x^2}{1+x}=\dfrac{1}{2}$, $\lim\limits_{x\to {1^{+}}} f\left( x \right)=\lim\limits_{x\to {1^{+}}} x^3=1$.\\
		Suy ra $\lim\limits_{x\to {1^{-}}} f\left( x \right)\ne \lim\limits_{x\to {1^{+}}} f\left( x \right)$. Hàm số gián đoạn tại $x=1$.}
\end{bt}
\subsubsection{Câu hỏi trắc nghiệm}
\Opensolutionfile{ans}[ans/ans-1K5-3-Dang2]
\begin{ex}%[DCHT Toán 11 - KNTT -Hứa Chí Ninh]%[1K5BG-1]
	Cho hàm số $y=f(x)$ xác định trên khoảng $K$ và $x_0\in K$. Hàm số $y=f(x)$ liên tục tại $x_0$ nếu
	\choice
	{$\lim \limits_{x\rightarrow x_0^+} f(x) = f(x_0)$}
	{$\lim \limits_{x\rightarrow x_0^-} f(x) = f(x_0)$ }
	{\True $\lim \limits_{x\rightarrow x_0} f(x)= f(x_0)$ }
	{$\lim \limits_{x\rightarrow x_0} f(x) \neq  f(x_0)$ }
	\loigiai{
		Theo định nghĩa: Cho hàm số $y=f(x)$ xác định trên khoảng $K$ và $x_0\in K$. Hàm số $y=f(x)$ liên tục tại $x_0$ nếu $\lim \limits_{x\rightarrow x_0} f(x)= f(x_0)$.
	}
\end{ex}

\begin{ex}%[DCHT Toán 11 - KNTT -Hứa Chí Ninh]%[1K5BG-3]
	Cho hàm số $f(x) = \dfrac{x^2 + 3x - 4}{x + 4}$ với $x \neq -4$. Để hàm số $f(x)$ liên tục tại $x = -4$ thì ta cần bổ sung giá trị $f(-4)$ bằng bao nhiêu?
	\choice
	{$5$}
	{\True $-5$}
	{$3$}
	{$0$}
	\loigiai{
		$f(-4) = \lim\limits_{x \to -4}\dfrac{x^2 + 3x - 4}{x + 4} = \lim\limits_{x \to -4}\dfrac{(x - 1)(x + 4)}{x + 4} = \lim\limits_{x \to -4}(x - 1) = -5$.
	}
\end{ex}
\begin{ex}%[DCHT Toán 11 - KNTT -Hứa Chí Ninh]%[1K5BG-3]	
	Cho hàm số $f\left( x \right)=\dfrac{2x-1}{x^3-x}$. Kết luận nào sau đây đúng?
	\choice
	{Hàm số liên tục tại $x=-1$}
	{Hàm số liên tục tại $x=0$}
	{Hàm số liên tục tại $x=1$}
	{\True Hàm số liên tục tại $x=\dfrac{1}{2}$}
	\loigiai{
		Tại $x=\dfrac{1}{2}$, ta có   $\lim\limits_{x\to \frac{1}{2}} f\left( x \right)=\lim\limits_{x\to \frac{1}{2}} \dfrac{2x-1}{x^3-1}=0=f\left( \dfrac{1}{2} \right)$. Vậy hàm số liên tục tại $x=2$.}
\end{ex}
\begin{ex}%[DCHT Toán 11 - KNTT -Hứa Chí Ninh]%[1K5BG-3]
	Hàm số nào sau đây liên tục tại $x=1$$\colon$ 
	\choice
	{$f\left( x \right)=\dfrac{x^2+x+1}{x-1}$}
	{$f\left( x \right)=\dfrac{x^2-x-2}{x^2-1}$}
	{\True $f\left( x \right)=\dfrac{x^2+x+1}{x}$}
	{$f\left( x \right)=\dfrac{x+1}{x-1}$}
	\loigiai{
		$\bullet$ $f\left( x \right)=\dfrac{x^2+x+1}{x-1}$\\
		$\lim\limits_{x\to {1^{+}}} f\left( x \right)=+\infty $ suy ra $f\left( x \right)$ không liên tục tại $x=1$.\\
		$\bullet$ $f\left( x \right)=\dfrac{x^2-x-2}{x^2-1}$\\
		$\lim\limits_{x\to {1^{+}}} f\left( x \right)=\lim\limits_{x\to {1^{+}}} \dfrac{x-2}{x-1}=-\infty $ suy ra $f\left( x \right)$ không liên tục tại $x=1$.\\
		$\bullet$ $f\left( x \right)=\dfrac{x^2+x+1}{x}$\\
		$\lim\limits_{x\to 1} f\left( x \right)=\lim\limits_{x\to 1} \dfrac{x^2+x+1}{x}=3=f\left( 1 \right)$ suy ra $f\left( x \right)$ liên tục tại $x=1$.\\
		$\bullet$ $f\left( x \right)=\dfrac{x+1}{x-1}$\\
		$\lim\limits_{x\to {1^{+}}} f\left( x \right)=\lim\limits_{x\to {1^{+}}} \dfrac{x+1}{x-1}=+\infty $ suy ra $f\left( x \right)$ không liên tục tại $x=1$.}
\end{ex}
\begin{ex}%[DCHT Toán 11 - KNTT -Hứa Chí Ninh]%[1K5BG-3]
	Hàm số nào dưới đây gián đoạn tại điểm $x_0=-1$.
	\choice
	{$y=\left( x+1 \right)\left( {x^2}+2 \right)$}
	{\True $y=\dfrac{2x-1}{x+1}$}
	{$y=\dfrac{x}{x-1}$}
	{$y=\dfrac{x+1}{x^2+1}$}
	\loigiai{
		Ta có $y=\dfrac{2x-1}{x+1}$ không xác định tại $x_0=-1$ nên gián đoạn tại $x_0=-1$.}
\end{ex}
\begin{ex}%[DCHT Toán 11 - KNTT -Hứa Chí Ninh]%[1K5BG-3]
	Hàm số nào sau đây gián đoạn tại $x=2$?
	\choice
	{\True $y=\dfrac{3x-4}{x-2}$}
	{$y=\sin x$}
	{$y=x^4-2x^2+1$}
	{$y=\tan x$}
	\loigiai{
		Ta có   $y=\dfrac{3x-4}{x-2}$ có tập xác định $\mathscr{D}=\mathbb{R}\setminus \left\{ 2 \right\}$, do đó gián đoạn tại $x=2$.}
\end{ex}
\begin{ex}%[DCHT Toán 11 - KNTT -Hứa Chí Ninh]%[1K5BG-3]
	Hàm số $y=\dfrac{x}{x+1}$ gián đoạn tại điểm $x_0$ bằng?
	\choice
	{ $x_0=2018$}
	{ $x_0=1$}
	{ $x_0=0$}
	{\True $x_0=-1$}
	\loigiai{
		Vì hàm số $y=\dfrac{x}{x+1}$ có tập xác định $\mathscr{D}=\mathbb{R}\setminus \left\{ -1 \right\}$ nên hàm số gián đoạn tại điểm $x_0=-1$.}
\end{ex}
\begin{ex}%[DCHT Toán 11 - KNTT -Hứa Chí Ninh]%[1K5BG-3]
	Hàm số nào dưới đây gián đoạn tại điểm $x=1$?
	\choice
	{$y=\dfrac{x-1}{x^2+x+1}$}
	{$y=\dfrac{x^2-x+1}{x+1}$}
	{$y=(x-1)(x^2+x+1)$}
	{\True $y=\dfrac{x^2+2}{x-1}$}
	\loigiai
	{
		Ta có $\lim\limits_{x\to1^+}\dfrac{x^2+2}{x-1}=+\infty$ và $\lim\limits_{x\to1^-}\dfrac{x^2+2}{x-1}=-\infty$ nên hàm số $y=\dfrac{x^2+2}{x-1}$ gián đoạn tại điểm $x=1$.
	}
\end{ex}
\begin{ex}%[DCHT Toán 11 - KNTT -Hứa Chí Ninh]%[1K5BG-3]
	Cho hàm số $y=\dfrac{x-3}{x^2-1}$. Mệnh đề nào sau đây đúng?
	\choice
	{\True Hàm số không liên tục tại các điểm $x=\pm 1$}
	{Hàm số liên tục tại mọi $x\in \mathbb{R}$}
	{Hàm số liên tục tại các điểm $x=-1$}
	{Hàm số liên tục tại các điểm $x=1$}
	\loigiai{
		Hàm số $y=\dfrac{x-3}{x^2-1}$ có tập xác định $\mathbb{R}\setminus \left\{ \pm 1 \right\}$. Do đó hàm số không liên tục tại các điểm $x=\pm 1$.}
\end{ex}
\begin{ex}%[DCHT Toán 11 - KNTT -Hứa Chí Ninh]%[1K5KG-3] 
	Cho hàm số $f(x)=\heva{&\dfrac{1-\cos x}{x^2}&\text{ khi } x\ne 0\\&1&\text{ khi } x=0}$.
	Khẳng định nào đúng trong các khẳng định sau?
	\choice
	{Hàm số gián đoạn tại $x=\sqrt{2}$}
	{$f(\sqrt{2})<0$}
	{$f(x)$ liên tục tại $x=0$}	
	{\True $f(x)$ gián đoạn tại $x=0$}
	\loigiai{
		Hàm số xác định trên $\mathbb{R}$.\\
		Ta có $f(0)=1$ và $\lim \limits_{x\to 0} f(x)=\lim \limits_{x\to 0} \dfrac{1-\cos x}{x^2}=\lim \limits_{x\to 0} \dfrac{2\sin ^2\dfrac{x}{2}}{4\cdot \left(\dfrac{x}{2}\right)^2}=\dfrac{1}{2}$.\\
		Vì $f(0)\ne \lim \limits_{x\to 0} f(x)$ nên $f(x)$ gián đoạn tại $x=0$.}
\end{ex}
\Closesolutionfile{ans}
% \begin{indapan}{10}
% 	{ans/ans-1K5-3-Dang2}
% \end{indapan}
\begin{dang}{Hàm số liên tục trên khoảng, đoạn}
	\begin{itemize}
		\item Hàm số $y=f(x)$ được gọi là liên tục trên một khoảng nếu nó liên tục tại mọi điểm của khoảng đó.
		\item Hàm số $y=f(x)$ được gọi là liên tục trên đoạn $[a,b]$ nếu nó liên tục trên khoảng $(a,b)$ và $$\mathop {\lim \limits{n \to +\infty}}\limits_{x \to {a^ + }} f\left( x \right) = f\left( a \right),{\rm{   }}\mathop {\lim \limits{n \to +\infty}}\limits_{x \to {b^ - }} f\left( x \right) = f\left( b \right).$$
		\item Đồ thị của hàm số liên tục trên một khoảng là một đường liền nét trên khoảng đó.
	\end{itemize}
\end{dang}
\subsubsection{Ví dụ mẫu}
\begin{vd}%[DCHT Toán 11 - KNTT -Hứa Chí Ninh]%[1K5KG-4]
	Xét tính liên tục của hàm số sau trên tập xác định của chúng.
	\begin{itemize}
		\item[a)] $f(x)=\heva{&\dfrac{x^2-x-2}{x+1}&\text{ khi } x\ne -1\\ &-3&\text{ khi } x=-1}.$
		\item[b)] $f(x)=\heva{&\dfrac{2x+1}{(x-1)^2}&\text{ khi } x\ne 1\\ &3 &\text{ khi } x=1}.$
	\end{itemize}
	\loigiai{
		\begin{enumerate}
			\item \begin{itemize}
				\item Tập xác định của hàm số là $\mathscr{D}=\mathbb{R}$.
				\item Khi $x \ne -1$, $f(x)=\dfrac{x^2-x-2}{x+1}$ là hàm phân thức hữu tỉ nên liên tục trên $(-\infty;-1)\cup(-1;+\infty)$.
				\item Tại điểm $x=-1$, ta có $f(-1)=-3$.\\
				$\lim\limits_{x\to -1}f(x)=\lim\limits_{x\to -1}\dfrac{x^2-x-2}{x+1}=\lim\limits_{x\to -1}(x-2)=-3=f(-1).$\\
				Do đó hàm số liên tục tại $x=-1$.
				\item Vậy hàm số liên tục trên $\mathbb{R}$.
			\end{itemize}
			\item \begin{itemize}
				\item Tập xác định của hàm số là $\mathscr{D}=\mathbb{R}$.\\
				\item Khi $x \ne 1$, $f(x)=\dfrac{2x+1}{(x-1)^2}$ là hàm phân thức hữu tỉ nên liên tục trên $(-\infty;1)\cup(1;+\infty)$.\\
				\item Tại điểm $x=1$, ta có $f(1)=3$.\\
				$\lim\limits_{x\to 1}f(x)=\lim\limits_{x\to 1}\dfrac{2x+1}{(x-1)^2}=+\infty\ne f(-1).$\\
				Do đó hàm số gián đoạn tại $x=1$.
				\item Vậy hàm số liên tục trên $\mathbb{R}\setminus\{1\}$.
			\end{itemize}
	\end{enumerate}}
\end{vd}

\begin{vd}%[DCHT Toán 11 - KNTT -Hứa Chí Ninh]%[1K5KG-4]
	Xét tính liên tục của hàm số sau trên tập xác định của chúng.
	\begin{itemize}
		\item[a)] $f(x)=\heva{&x^2+3x&\text{ khi } &x\ge 2\\ &6x+1&\text{ khi } &x<2.}$
		\item[b)] $f(x)=\heva{&x^2-3x+5&\text{ khi } &x> 1\\ &3 &\text{ khi } &x=1\\&2x+1 &\text{ khi }&x<1.}$
	\end{itemize}
	\loigiai{
		\begin{enumerate}
			\item \begin{itemize}
				\item Tập xác định của hàm số là $\mathscr{D}=\mathbb{R}$.
				\item Khi $x>2$, $f(x)=x^2+3x$ là hàm đa thức nên liên tục trên $(2;+\infty)$.
				\item Khi $x<2$, $f(x)=6x+1$ là hàm đa thức nên liên tục trên $(-\infty;2)$.
				\item Tại điểm $x=2$, ta có $f(2)=10$.\\
				$\lim\limits_{x\to 2^+}f(x)=\lim\limits_{x\to 2^+}(x^2+3x)=10$ và $\lim\limits_{x\to 2^-}f(x)=\lim\limits_{x\to 2^-}(6x+1)=13$.\\
				Vì không tồn tại $\lim\limits_{x\to 2}f(x)$ nên hàm số gián đoạn tại $x=2$.
				\item Vậy hàm số liên tục trên $\mathbb{R}\setminus\{2\}$.
			\end{itemize}
			\item \begin{itemize}
				\item Tập xác định của hàm số là $\mathscr{D}=\mathbb{R}$.
				\item Khi $x>1$, $f(x)=x^2-3x+5$ là hàm đa thức nên liên tục trên $(1;+\infty)$.
				\item Khi $x<1$, $f(x)=2x+1$ là hàm đa thức nên liên tục trên $(-\infty;1)$.
				\item Tại điểm $x=1$, ta có $f(1)=3$.\\
				$\lim\limits_{x\to 1^+}f(x)=\lim\limits_{x\to 1^+}(x^2-3x+5)=3$ và $\lim\limits_{x\to 1^-}f(x)=\lim\limits_{x\to 1^-}(2x+1)=3$.\\
				Vì $\lim\limits_{x\to 1}f(x)=f(1)$ nên hàm số liên tục tại $x=1$.
				\item Vậy hàm số liên tục trên $\mathbb{R}$.
			\end{itemize}
	\end{enumerate}}
\end{vd}

\subsubsection{Bài tập rèn luyện}
% \centerline{\fcolorbox{red}{yellow!50}{\bf {BÀI TẬP TỰ LUẬN}}}
\begin{bt}%[DCHT Toán 11 - KNTT -Hứa Chí Ninh]%[1K5KG-4]
	Xét tính liên tục của hàm số $f(x)=\heva{&\dfrac{x-2}{x^2-4}&\text { khi } x \neq 2 \\& 1&\text { khi } x=2}$ trên tập xác định.
	\loigiai{
		Tập xác định của hàm số là $\mathscr{D}=\mathbb{R}$.\\
		Khi $x \neq 2, f(x)=\dfrac{x-2}{x^2-4}$ là hàm phân thức hữu tỉ nên liên tục trên $(-\infty ; 2) \cup(2 ;+\infty)$.\\
		Tại điểm $x=2$, ta có $f(2)=1$.\\
		$\lim \limits_{x \to 2} f(x)=\lim \limits_{x \to 2} \dfrac{x-2}{x^2-4}=\lim \limits_{x \to 2} \dfrac{1}{x+2}=\dfrac{1}{4} \neq f(2)$.\\
		Do đó hàm số gián đoạn tại $x=2$.\\
		Vậy hàm số liên tục trên $(-\infty; 2)$ và $(2;+\infty)$.}
\end{bt}
\begin{bt}%[DCHT Toán 11 - KNTT -Hứa Chí Ninh]%[1K5KG-4]
	Xét tính liên tục của hàm số $f(x)=\heva{&\dfrac{x^3-1}{x-1}&\text { khi } x \neq 1 \\& 3&\text { khi } x=1}$ trên tập xác định.
	\loigiai{
		Tập xác định của hàm số là $\mathscr{D}=\mathbb{R}$.\\
		Khi $x \neq 1, f(x)=\dfrac{x^3-1}{x-1}$ là hàm phân thức hữu tỉ nên liên tục trên $(-\infty ; 1) \cup(1 ;+\infty)$.\\
		Tại điểm $x=1$, ta có $f(1)=3$.\\
		$\lim \limits_{x \to 1} f(x)=\lim \limits_{x \to 1} \dfrac{x^3-1}{x-1}=\lim \limits_{x \to 1}(x^2+x+1)=3=f(1)$.\\
		Do đó hàm số liên tục tại $x=1$.\\
		Vậy hàm số liên tục trên $\mathbb{R}$.}
\end{bt}
\begin{bt}%[DCHT Toán 11 - KNTT -Hứa Chí Ninh]%[1K5KG-4]
	Xét tính liên tục của hàm số $f(x)=\heva{&x^2&\text { khi } x \geq-2 \\& 2-x&\text { khi } x<-2}$ trên tập xác định.
	\loigiai{
		Tập xác định của hàm số là $\mathscr{D}=\mathbb{R}$.\\
		Khi $x>-2, f(x)=x^2$ là hàm đa thức nên liên tục trên $(-2 ;+\infty)$.\\
		Khi $x<-2, f(x)=2-x$ là hàm đa thức nên liên tục trên $(-\infty ;-2) $.\\
		Tại điểm $x=-2$, ta có $f(-2)=4$.\\
		$\lim \limits_{x \to(-2)^{+}} f(x)=\lim \limits_{x \to(-2)^{+}} x^2=4 \text { và } \lim \limits_{x \to(-2)^{-}} f(x)=\lim \limits_{x \to(-2)^{-}}(2-x)=4$.
		Vì $\lim \limits_{x \to(-2)^{+}} f(x)=\lim \limits_{x \to(-2)^{-}} f(x)=f(-2)$ nên hàm số liên tục tại $x=2$.\\
		Vậy hàm số liên tục trên $\mathbb{R}$.}
\end{bt}
\begin{bt}%[DCHT Toán 11 - KNTT -Hứa Chí Ninh]%[1K5KG-4]
	Xét tính liên tục của hàm số $f(x)=\heva{&3x-2 &\text { khi } x>-1 \\& 1 &\text { khi } x=-1 \\& x^2-6 &\text { khi } x<-1}$ trên tập xác định.
	\loigiai{
		Tập xác định của hàm số là $\mathscr{D}=\mathbb{R}$.\\
		Khi $x>-1, f(x)=3 x-2$ là hàm đa thức nên liên tục trên $(-1 ;+\infty)$.\\
		Khi $x<-1, f(x)=x^2-6$ là hàm đa thức nên liên tục trên $(-\infty ;-1)$.\\
		Tại điểm $x=-1$, ta có $f(-1)=1$.\\
		$\lim \limits_{x \to(-1)^{+}} f(x)=\lim \limits_{x \to(-1)^{+}}(3 x-2)=-5 \text { và } \lim \limits_{x \to(-1)^{-}} f(x)=\lim \limits_{x \to(-1)^{-}}(x^2-6)=3$.\\
		Vì không tồn tại $\lim \limits_{x \to 1} f(x)$ nên hàm số gián đoạn tại $x=-1 $.
		Vậy hàm số liên tục trên $\mathbb{R} \backslash\{-1\}$.}
\end{bt}

\begin{bt}%[DCHT Toán 11 - KNTT -Hứa Chí Ninh]%[1K5KG-4]
	Cho hàm số $y=f(x)=\heva{&\dfrac{1-x}{\sqrt{2-x}-1}& \text { khi } x<1 \\& 2x & \text { khi } x \geq 1}$. Xét sự liên tục của hàm số trên tập xác định.
	\loigiai{
		\begin{itemize}
			\item Hàm số xác định và liên tục trên $(-\infty ; 1)$ và $(1 ;+\infty)$.
			\item Xét tính liên tục tại $x=1$
			$$\begin{aligned}
				& f(1)=2\cdot 1=2.\\
				& \lim \limits_{x \to 1} f(x)=\lim \limits_{x \to 1} \dfrac{1-x}{\sqrt{2-x}-1} =\lim \limits_{x \to 1} \dfrac{(1-x)(\sqrt{2-x}+1)}{2-x-1} =\lim \limits_{x \to 1}(\sqrt{2-x}+1)=2.
			\end{aligned}$$
			Ta thấy $\lim \limits_{x \to 1} f(x)=f(1)$ nên hàm số liên tục tại $\mathrm{x}=1$.
		\end{itemize}
		Vậy hàm số liên tục trên $\mathbb{R}$.}
\end{bt}
\begin{bt}%[DCHT Toán 11 - KNTT -Hứa Chí Ninh]%[1K5KG-4]
	Tìm số điểm gián đoạn của hàm số $f\left( x \right)=\left\{ \begin{array}{*{35}{l}}
		0,5 & \text{khi}\,\,x=-1 \\
		\dfrac{x\left( x+1 \right)}{x^2-1} & \text{khi}\,\,x\ne -1,\,\,x\ne 1 \\
		1 & \text{khi}\,\,x=1 \\
	\end{array} \right.$?
	% \dapso{Hàm số $y=f\left( x \right)$ gián đoạn tại $x=1$.}
	\loigiai{
		Hàm số $y=f\left( x \right)$ có tập xác định $\mathscr{D}=\mathbb{R}$.\\
		Hàm số $f\left( x \right)=\dfrac{x\left( x+1 \right)}{x^2-1}$ liên tục trên mỗi khoảng $\left( -\infty ;-1 \right)$, $\left( -1;1 \right)$ và $\left( 1;+\infty \right)$.\\
		(i) Xét tại $x=-1$, ta có $\lim\limits_{x\to -1} f\left( x \right)=\lim\limits_{x\to -1} \dfrac{x\left( x+1 \right)}{x^2-1}=\lim\limits_{x\to -1} \dfrac{x}{x-1}=\dfrac{1}{2}=f\left( -1 \right)\Rightarrow $ Hàm số liên tục tại $x=-1.$\\
		(ii) Xét tại $x=1$, ta có $\left\{ \begin{aligned}
			& \lim\limits_{x\to {1^{+}}} f\left( x \right)=\lim\limits_{x\to {1^{+}}} \dfrac{x\left( x+1 \right)}{x^2-1}=\lim\limits_{x\to {1^{+}}} \dfrac{x}{x-1}=+\infty \\
			& \lim\limits_{x\to {1^{-}}} f\left( x \right)=\lim\limits_{x\to {1^{-}}} \dfrac{x\left( x+1 \right)}{x^2-1}=\lim\limits_{x\to {1^{-}}} \dfrac{x}{x-1}=-\infty \\
		\end{aligned} \right.\Rightarrow $Hàm số $y=f\left( x \right)$ gián đoạn tại $x=1$.
	}
\end{bt}

\begin{bt}%[DCHT Toán 11 - KNTT -Hứa Chí Ninh]%[1K5KG-4]
	Tìm điểm gián đoạn của hàm số $h\left( x \right)=\left\{ \begin{aligned}
		& 2x&\text{ khi }x<0 \\
		& {x^2}+1&\text{ khi }0\le x\le 2 \\
		& 3x-1&\text{ khi }x>2 \\
	\end{aligned} \right.$?
	% \dapso{Hàm số gián đoạn tại điểm $x=0$}
	\loigiai{
		Hàm số $y=h\left( x \right)$ có tập xác định $\mathscr{D}=\mathbb{R}$.\\
		Dễ thấy hàm số $y=h\left( x \right)$ liên tục trên mỗi khoảng $\left( -\infty ;0 \right),\left( 0;2 \right)$ và $\left( 2;+\infty \right)$.\\
		Ta có $\left\{ \begin{aligned}
			& h\left( 0 \right)=1 \\
			& \lim\limits_{x\to {0^{-}}} h\left( x \right)=\lim\limits_{x\to {0^{-}}} 2x=0 \\
		\end{aligned} \right.\Rightarrow f\left( x \right)$ không liên tục tại $x=0$.\\
		Ta có $\left\{ \begin{aligned}
			& h\left( 2 \right)=5 \\
			& \lim\limits_{x\to {2^{-}}} h\left( x \right)=\lim\limits_{x\to {2^{-}}} \left( {x^2}+1 \right)=5 \\
			& \lim\limits_{x\to {2^{+}}} h\left( x \right)=\lim\limits_{x\to {2^{+}}} \left( 3x-1 \right)=5 \\
		\end{aligned} \right.\Rightarrow f\left( x \right)$ liên tục tại $x=2$.}
\end{bt}
\begin{bt}%[DCHT Toán 11 - KNTT -Hứa Chí Ninh]%[1K5KG-4]
	Cho hàm số $f\left( x \right)=\left\{ \begin{aligned}
		& -x\cos x&\text{ khi}\,\,x<0 \\
		& \dfrac{x^2}{1+x}&\text{ khi}\,\,0\le x<1 \\
		& {x^3}&\text{ khi}\,\,x\ge 1 \\
	\end{aligned} \right..$ Hàm số $f\left( x \right)$ gián đoạn tại điểm nào?
	% \dapso{Hàm số gián đoạn tại điểm $x=1.$}
	\loigiai{
		Hàm số $y=f\left( x \right)$ có tập xác định $\mathscr{D}=\mathbb{R}$.\\
		Dễ thấy $f\left( x \right)$ liên tục trên mỗi khoảng $\left( -\infty ;0 \right),\left( 0;1 \right)$ và $\left( 1;+\infty \right)$.\\
		Ta có $\left\{ \begin{aligned}
			& f\left( 0 \right)=0 \\
			& \lim\limits_{x\to {0^{-}}} f\left( x \right)=\lim\limits_{x\to {0^{-}}} \left( -x\cos x \right)=0 \\
			& \lim\limits_{x\to {0^{+}}} f\left( x \right)=\lim\limits_{x\to {0^{+}}} \dfrac{x^2}{1+x}=0 \\
		\end{aligned} \right. \Rightarrow f\left( x \right)$ liên tục tại $x=0$.\\
		Ta có $\left\{ \begin{aligned}
			& f\left( 1 \right)=1 \\
			& \lim\limits_{x\to {1^{-}}} f\left( x \right)=\lim\limits_{x\to {1^{-}}} \dfrac{x^2}{1+x}=\dfrac{1}{2} \\
			& \lim\limits_{x\to {1^{+}}}{\mathop{\,\,\lim \limits{n \to +\infty}}}\,f\left( x \right)=\,\lim\limits_{x\to {1^{+}}}{\mathop{\lim \limits{n \to +\infty}{x^3}=1}}\, \\
		\end{aligned} \right. \Rightarrow f\left( x \right)$ không liên tục tại $x=1$.\\
	}
\end{bt}
\subsubsection{Câu hỏi trắc nghiệm}
\Opensolutionfile{ans}[ans/ans-1K5-3-Dang3]
\begin{ex}%[DCHT Toán 11 - KNTT -Hứa Chí Ninh]%[1K5BG-4]
	Cho hàm số $y=f\left( x \right)$ liên tục trên $(a;b)$. Điều kiện cần và đủ để hàm số liên tục trên $\left[ a;b \right]$ là
	\choice
	{$\lim\limits_{x\to {a^{+}}} f\left( x \right)=f\left( a \right)$ và $\lim\limits_{x\to {b^{+}}} f\left( x \right)=f\left( b \right)$}
	{$\lim\limits_{x\to {a^{-}}} f\left( x \right)=f\left( a \right)$ và $\lim\limits_{x\to {b^{-}}} f\left( x \right)=f\left( b \right)$}
	{\True $\lim\limits_{x\to {a^{+}}} f\left( x \right)=f\left( a \right)$ và $\lim\limits_{x\to {b^{-}}} f\left( x \right)=f\left( b \right)$}
	{$\lim\limits_{x\to {a^{-}}} f\left( x \right)=f\left( a \right)$ và $\lim\limits_{x\to {b^{+}}} f\left( x \right)=f\left( b \right)$}
	\loigiai{
		Theo định nghĩa hàm số liên tục trên đoạn $\left[ a;b \right]$. Chọn  $\lim\limits_{x\to {a^{+}}} f\left( x \right)=f\left( a \right)$ và $\lim\limits_{x\to {b^{-}}} f\left( x \right)=f\left( b \right)$.}
\end{ex}

\begin{ex}%[DCHT Toán 11 - KNTT -Hứa Chí Ninh]%[1K5KG-4]
	Cho hàm số $y=\left\{ \begin{aligned}
		& \dfrac{1-x^3}{1-x}&\text{khi  }x<1 \\
		& 1\text{  }&\text{khi  }x\ge 1 \\
	\end{aligned} \right.$. Hãy chọn kết luận đúng
	\choice
	{\True $y$ liên tục phải tại $x=1$}
	{$y$ liên tục tại $x=1$}
	{$y$ liên tục trái tại $x=1$}
	{$y$ liên tục trên $\mathbb{R}$}
	\loigiai{
		Ta có   $y\left( 1 \right)=1$.\\
		Ta có   $\lim\limits_{x\to {1^{+}}} y=1$; $\lim\limits_{x\to {1^{-}}} y=\lim\limits_{x\to {1^{-}}} \dfrac{1-x^3}{1-x}=\lim\limits_{x\to {1^{-}}} \dfrac{\left( 1-x \right)\left( 1+x+x^2 \right)}{1-x}=\lim\limits_{x\to {1^{-}}} \left( 1+x+x^2 \right)=4$.\\
		Nhận thấy $\lim\limits_{x\to {1^{+}}} y=y\left( 1 \right)$, suy ra $y$ liên tục phải tại $x=1$.}
\end{ex}
\begin{ex}%[DCHT Toán 11 - KNTT -Hứa Chí Ninh]%[1K5BG-4]% 
	
	Trong các hàm số sau, hàm số nào liên tục trên $\mathbb{R}$?
	\choice
	{\True $y=x^3-x$}
	{$y=\cot x$}
	{$y=\dfrac{2x-1}{x-1}$}
	{$y=\sqrt{x^2-1}$}
	\loigiai{
		Vì $y=x^3-x$ là đa thức nên liên tục trên $\mathbb{R}$.}
\end{ex}
\begin{ex}%[DCHT Toán 11 - KNTT -Hứa Chí Ninh]%[1K5BG-4]
	Trong các hàm số sau, hàm số nào liên tục trên $\mathbb{R}$?
	\choice
	{$f\left( x \right)=\tan x+5$}
	{$f\left( x \right)=\dfrac{x^2+3}{5-x}$}
	{$f\left( x \right)=\sqrt{x-6}$}
	{\True $f\left( x \right)=\dfrac{x+5}{x^2+4}$}
	\loigiai{
		Hàm số $f\left( x \right)=\dfrac{x+5}{x^2+4}$ là hàm phân thức hữu tỉ và có tập xác định là $\mathscr{D}=\mathbb{R}$ do đó hàm số $f\left( x \right)=\dfrac{x+5}{x^2+4}$ liên tục trên $\mathbb{R}$.}
\end{ex}
\begin{ex}%[DCHT Toán 11 - KNTT -Hứa Chí Ninh]%[1K5BG-4]
	Cho hàm số $y=\left\{ \begin{aligned}
		& -x^2+x+3&\text{  khi}\,\,x\ge 2 \\
		& 5x+2&\text{      khi}\,\,x<2 \\
	\end{aligned} \right.$. Chọn mệnh đề sai trong các mệnh đề sau$\colon$ 
	\choice
	{Hàm số liên tục tại $x_0=1$}
	{\True Hàm số liên tục trên $\mathbb{R}$}
	{Hàm số liên tục trên các khoảng $\left( -\infty ;\,2 \right),\,\,\left( 2;\,+\infty \right)$}
	{Hàm số gián đoạn tại $x_0=2$}
	\loigiai{
		+ Với $x>2$, ta có $f\left( x \right)=-x^2+x+3$ là hàm đa thức $\Rightarrow $ hàm số $f\left( x \right)$ liên tục trên khoảng $\left( 2;\,+\infty \right)$.\\
		+ Với $x<2$, ta có $f\left( x \right)=5x+2$ là hàm đa thức $\Rightarrow $ hàm số $f\left( x \right)$ liên tục trên khoảng $\left( -\infty ;\,2 \right)$.\\
		+ Tại $x=2$.\\
		$\lim\limits_{x\to {2^{+}}} f\left( x \right)=\lim\limits_{x\to {2^{+}}} \left( -x^2+x+3 \right)=1$
		$\lim\limits_{x\to {2^{^{-}}}} f\left( x \right)=\lim\limits_{x\to {2^{-}}} \left( 5x+2 \right)=12$
		$\Rightarrow \lim\limits_{x\to {2^{+}}} f\left( x \right)\ne \lim\limits_{x\to {2^{-}}} f\left( x \right)$. Do đó không tồn tại $\lim\limits_{x\to 2} f\left( x \right).$ Vậy hàm số gián đoạn tại $x_0=2$ hay
		Hàm số không liên tục trên $\mathbb{R}$.}
\end{ex}
\begin{ex}%[DCHT Toán 11 - KNTT -Hứa Chí Ninh]%[1K5BG-4]
	Hàm số nào sau đây liên tục trên $\mathbb{R}$?
	\choice
	{$f\left( x \right)=\sqrt{x}$}
	{\True $f\left( x \right)=x^4-4x^2$}
	{$f\left( x \right)=\sqrt{\dfrac{x^4-4x^2}{x+1}}$}
	{$f\left( x \right)=\dfrac{x^4-4x^2}{x+1}$}
	\loigiai{
		Vì hàm số $f\left( x \right)=x^4-4x^2$ có dạng đa thức với   $\mathscr{D}=\mathbb{R}$ nên hàm số này liên tục trên $\mathbb{R}$}
\end{ex}
\begin{ex}%[DCHT Toán 11 - KNTT -Hứa Chí Ninh]%[1K5KG-4]
	Cho bốn hàm số $f_1(x)=\sqrt{x-1}$; $f_2(x)=x$; $f_3(x)=\tan x$; $f_4(x)=\heva{&\dfrac{x^2-1}{x-1} &\text{ khi } x\ne 1\\&2&\text{ khi }  x=1} $. Hỏi trong bốn hàm số trên có bao nhiêu hàm số liên tục trên $\mathbb{R}$?
	\choice
	{$1$}
	{\True $2$}
	{$3$}
	{$4$}
	\loigiai{
		+ Hàm số $f_1(x)=\sqrt{x-1}$ và $f_3(x)=\tan x$ không có tập xác định là $\mathbb{R}$ nên hàm số không liên tục trên $\mathbb{R}$.\\
		+ Hàm số $f_2(x)=x$ liên tục trên $\mathbb{R}$.\\
		+ Hàm số $f_4(x)=\left\{\begin{aligned}
			&\dfrac{x^2-1}{x-1}\ \text{khi } x\ne 1\\
			&2\ \text{khi } x=1\\
		\end{aligned}\right. $ có tập xác định là $\mathbb{R}$ và hàm số liên tục trên các khoảng $(-\infty ;1)$ và $(1;+\infty )$. Ta cần xét tính liên tục của hàm số $y=f_4(x)$ tại $x=1$.\\
		Ta có $f_4(1)=2$ và $\lim \limits_{x\to 1} f_4(x) =\lim \limits_{x\to 1} \dfrac{x^2-1}{x-1} =\lim \limits_{x\to 1} (x+1) =2 =f_4(1)$ nên hàm số liên tục tại $x=1$. Do đó, hàm số $y=f_4(x)$ liên tục trên $\mathbb{R}$. Vậy trong bốn hàm số trên có $2$ hàm số liên tục trên $\mathbb{R}$.}
\end{ex}
\begin{ex}%[DCHT Toán 11 - KNTT -Hứa Chí Ninh]%[1K5BG-4]
	Hàm số nào dưới đây liên tục trên $\mathbb{R}$?
	\choice
	{\True $y=x^3+3x-2$}
	{$y=\sqrt{x^2-1}$}
	{$y=\dfrac{x+1}{x-1}$}
	{$y=x+\tan x$}
	\loigiai{
		Hàm số $y=x^3+3x-2$ có tập xác định là $\mathbb{R}$ nên liên tục trên $\mathbb{R}$.\\
		Hàm số $y=\sqrt{x^2-1}$ có tập xác định là $\left(-\infty ;-1\right]\cup \left[1;+\infty \right)$ nên liên tục trên $\left(-\infty ;-1\right]\cup \left[1;+\infty \right)$.\\
		Hàm số $y=\dfrac{x+1}{x-1}$ liên tục trên $\mathbb{R}\setminus \{1\}$.\\
		Hàm số $y=x+\tan x$ liên tục trên $\mathbb{R}\setminus \left\{\dfrac{\pi }{2}+k\pi ,k\in \mathbb{Z}\right\}$.}
\end{ex}
\begin{ex}%[DCHT Toán 11 - KNTT -Hứa Chí Ninh]%[1K5BG-4]
	Hàm số $f(x)=\heva{&3&\text{ khi } x=-1\\&\dfrac{x^4+x}{x^2+x} &\text{ khi }  x\ne -1,x\ne 0\\&1&\text{ khi } x=0}$ liên tục tại
	\choice
	{mọi điểm trừ $x=0,x=-1$}
	{\True mọi điểm $x\in \mathbb{R}$}
	{mọi điểm trừ $x=-1$}
	{mọi điểm trừ $x=0$}
	\loigiai{
		Dễ thấy hàm số $f(x)$ liên tục tại mọi điểm $x\ne -1,x\ne 0$.\\
		+) Tại $x=0$:\\
		Ta có $\lim \limits_{x\to 0} f(x)=\lim \limits_{x\to 0} \dfrac{x^4+x}{x^2+x} =\lim \limits_{x\to 0} \dfrac{(x^3+1)}{(x+1)}=1$ và $f(0)=1$, suy ra hàm số liên tục tại $x=0$.\\
		+) Tại $x=-1$:\\
		Ta có $\lim \limits_{x\to -1} f(x)=\lim \limits_{x\to -1} \dfrac{x^4+x}{x^2+x} =\lim \limits_{x\to -1} \dfrac{(x+1)(x^2-x+1)}{(x+1)}=3 =f(-1)$, suy ra hàm số liên tục tại $x=-1$.\\
		Vậy hàm số liên tục trên $\mathbb{R}$.}
\end{ex}
\begin{ex}%[DCHT Toán 11 - KNTT -Hứa Chí Ninh]%[1K5BG-4]
	Số điểm gián đoạn của hàm số $f(x)=\left\{\begin{aligned}
		&0,5&\text{ khi }  x=-1\\
		&\dfrac{x(x+1)}{x^2-1}&\text{ khi }  x\ne \pm 1\\
		&1&\text{ khi } x=1\\
	\end{aligned}\right. $ là
	\choice
	{$0$}
	{\True $1$}
	{$2$}
	{$3$}
	\loigiai{
		Hàm số liên tục trên các khoảng $(-\infty ;-1)$, $(-1;1)$ và $(1;+\infty )$\\
		Ta có $f(-1)=0,5$, $f(1)=1$\\
		+ $\lim \limits_{x\to -1} f(x) =\lim \limits_{x\to -1} \dfrac{x(x+1)}{x^2-1} =\lim \limits_{x\to -1} \dfrac{x}{x-1} =\dfrac{-1}{-1-1} =0,5 =f(-1)$ nên hàm số liên tục tại $x=-1$\\
		+ $\lim \limits_{x\to 1} f(x)=\lim \limits_{x\to 1} \dfrac{x(x+1)}{x^2-1}=\lim \limits_{x\to 1} \dfrac{x}{x-1}$\\
		Vì $\lim \limits_{x\to 1^{+}} \dfrac{x}{x-1}=+\infty $, $\lim \limits_{x\to 1^{-}} \dfrac{x}{x-1}=-\infty $ nên không tồn tại giới hạn $\lim \limits_{x\to 1} f(x)$\\
		Suy ra hàm số gián đoạn tại $x=1$\\
		Vậy hàm số có một điểm gián đoạn.}
\end{ex}
\begin{ex}%[DCHT Toán 11 - KNTT -Hứa Chí Ninh]%[1K5BG-4]
	Cho hàm số $f(x)=\heva{&\dfrac{x^5+x^2}{x^3+x^2}&\text{ khi } x\ne 0;x\ne -1\\
		&3&\text{ khi } x=-1\\
		&1&\text{ khi } x=0}$. Khi đó
	\choice
	{\True Hàm số liên tục tại mỗi điểm $x\in \mathbb{R}$}
	{Hàm số liên tục tại mỗi điểm trừ $x=0$}
	{Hàm số liên tục tại mỗi điểm trừ $x=-1$}
	{Hàm số liên tục tại mỗi điểm trừ $x=-1;x=0$}
	\loigiai{
		Tập xác định: $D=\mathbb{R}$\\
		Xét $x\ne 0,x\ne -1$:\\
		$f(x)=\dfrac{x^5+x^2}{x^3+x^2}=\dfrac{x^2(x^3+1)}{x^2(x+1)} =x^2-x+1 \Rightarrow $ hàm đa thức liên tục khi $x\ne 0,x\ne -1$.\\
		Xét: $x=-1$:\\
		$f(-1)=3$, $\lim \limits_{x\to -1} f(x)=(-1)^2-(-1)+1=3 =f(-1) \Rightarrow $ hàm số liên tục khi $x=-1$.\\
		Xét: $x=0$:\\
		$f(0)=1$, $\lim \limits_{x\to 0} f(x)=0^2-0+1=1 \Rightarrow $ hàm số liên tục khi $x=0$.\\
		Vậy hàm số liên tục trên $\mathbb{R}$.}
\end{ex}
\Closesolutionfile{ans}
% \begin{indapan}{10}
% 	{ans/ans-1K5-3-Dang3}
% \end{indapan}
\begin{dang}{Bài toán chứa tham số}
\end{dang}
\subsubsection{Ví dụ mẫu}
\begin{vd}%[DCHT Toán 11 - KNTT -Hứa Chí Ninh]%[TH]%[1K5BG-5]
	Tìm $m$ để hàm số $f\left( x \right)=\left\{ \begin{aligned}
		& \dfrac{x^2-16}{x-4} \text{ khi } x>4 \\
		& mx+1 \text{ khi } x\le 4 \\
	\end{aligned} \right.\,$ liên tục tại điểm $x=4$.
	% \dapso{ $m=\dfrac{7}{4}$.}
	\loigiai{
		Ta có $\lim\limits_{x\to {4^{-}}} f\left( x \right)=f\left( 4 \right)=4m+1$; $\lim\limits_{x\to {4^{+}}} f\left( x \right)=\lim\limits_{x\to {4^{+}}} \dfrac{x^2-16}{x-4}=\lim\limits_{x\to {4^{+}}} \,\left( x+4 \right)=8$.\\
		Hàm số liên tục tại điểm $x=4\Leftrightarrow \lim\limits_{x\to {4^{-}}} f\left( x \right)=\lim\limits_{x\to {4^{+}}} f\left( x \right)=f\left( 4 \right)\Leftrightarrow 4m+1=8\Leftrightarrow m=\dfrac{7}{4}$.}
\end{vd}
\begin{vd}%[DCHT Toán 11 - KNTT -Hứa Chí Ninh]%[TH]%[1K5BG-5]
	Tìm $m$ để hàm số $f(x)=\left\{ \begin{aligned}
		& \dfrac{x^2-x-2}{x+1} \text{ khi } x>-1 \\
		& mx-2m^2 \text{ khi } x\le -1 \\
	\end{aligned} \right.$ liên tục tại $x=-1.$
	% \dapso{ $m\in \left\{ 1;-\dfrac{3}{2} \right\}$.}
	\loigiai{
		Tập xác định $\mathscr{D}=\mathbb{R}$.\\
		$\bullet$ $f(-1)=-m-2m^2$\\
		$\bullet$ $\lim\limits_{x\to -{1^{-}}} f(x)=\lim\limits_{x\to -{1^{-}}} (mx-2m^2)=-m-2m^2$.\\
		$\bullet$ $\lim\limits_{x\to -{1^{+}}} f(x)=\lim\limits_{x\to -{1^{+}}} \dfrac{x^2-x-2}{x+1}=\lim\limits_{x\to -{1^{+}}} \dfrac{(x+1)(x-2)}{x+1}=\lim\limits_{x\to -{1^{+}}} (x-2)=-3.$\\
		Hàm số liên tục tại $x=-1$ khi và chỉ khi$\lim\limits_{x\to -{1^{-}}} f(x)=\lim\limits_{x\to -{1^{+}}} f(x)=f(-1)$\\
		$\Leftrightarrow -m-2m^2=-3\Leftrightarrow 2m^2+m-3=0\Leftrightarrow \left[ \begin{aligned}
			& m=1 \\
			& m=-\dfrac{3}{2} \\
		\end{aligned} \right..$\\
		Vậy các giá trị của m là $m\in \left\{ 1;-\dfrac{3}{2} \right\}.$}
\end{vd}
\begin{vd}%[DCHT Toán 11 - KNTT -Hứa Chí Ninh]%[TH]%[1K5BG-5]	
	Tìm giá trị của tham số $m$ để hàm số $f\left( x \right)=\left\{ \begin{aligned}
		& \dfrac{x^2+3x+2}{x^2-1}\,\,\,\,\,\text{khi}\,\,\,\,\,\,x\,<\,-1 \\
		& mx+2\,\,\,\,\,\,\,\,\,\,\,\,\,\,\,\text{khi}\,\,\,\,\,\,x\,\ge \,-1 \\
	\end{aligned} \right.$ liên tục tại $x=-1$?
	% \dapso{$m=\dfrac{5}{2}$.}
	\loigiai{
		Ta có  \\
		$\bullet$ $f\left( -1 \right)=-m+2$.\\
		$\bullet$ $\lim\limits_{x\to {{\left( -1 \right)}^{+}}} f\left( x \right)=-m+2$.\\
		$\bullet$$\lim\limits_{x\to {{\left( -1 \right)}^{-}}} f\left( x \right)=\lim\limits_{x\to {{\left( -1 \right)}^{-}}} \dfrac{x^2+3x+2}{x^2-1}=\lim\limits_{x\to {{\left( -1 \right)}^{-}}} \dfrac{\left( x+1 \right)\left( x+2 \right)}{\left( x-1 \right)\left( x+1 \right)}=\lim\limits_{x\to {{\left( -1 \right)}^{-}}} \dfrac{x+2}{x-1}=\dfrac{-1}{2}$.\\
		Hàm số liên tục tại $x=-1\Leftrightarrow f\left( -1 \right)=\lim\limits_{x\to {{\left( -1 \right)}^{+}}} f\left( x \right)=\lim\limits_{x\to {{\left( -1 \right)}^{-}}} f\left( x \right)\Leftrightarrow -m+2=\dfrac{-1}{2}\Leftrightarrow m=\dfrac{5}{2}$.}
\end{vd}
\begin{vd}%[DCHT Toán 11 - KNTT -Hứa Chí Ninh]%[TH]%[1K5BG-5]% 
	Cho hàm số $f\left( x \right)=\left\{ \begin{aligned}
		& \dfrac{x^2-3x+2}{\sqrt{x+2}-2}\,\,\,\,\,\,\,\,\,khi\,\,x>2 \\
		& {m^2}x-4m+6\,\,\,\,khi\,\,\,x\le 2 \\
	\end{aligned} \right.$, $m$ là tham số. Với giá trị nào của $m$ thì hàm số đã cho liên tục tại $x=2$?
	% \dapso{$m=1.$}
	\loigiai{
		Ta có\\
		$\lim\limits_{x\to {2^{+}}} f(x)=\lim\limits_{x\to {2^{+}}} \dfrac{x^2-3x+2}{\sqrt{x+2}-2}=\lim\limits_{x\to {2^{+}}} \dfrac{\left( x-2 \right)\left( x-1 \right)\left( \sqrt{x+2}+2 \right)}{x-2}=\lim\limits_{x\to {2^{+}}} \left( x-1 \right)\left( \sqrt{x+2}+2 \right)=4$.\\
		$\lim\limits_{x\to {2^{-}}} f(x)=\lim\limits_{x\to {2^{-}}} \left( {m^2}x-4m+6 \right)=2m^2-4m+6$.\\
		$f(2)=2m^2-4m+6$.\\
		Để hàm số liên tục tại $x=2$ thì $\lim\limits_{x\to {2^{+}}} f(x)=\lim\limits_{x\to {2^{-}}} f(x)=f(2)\Leftrightarrow 2m^2-4m+6=4\Leftrightarrow 2m^2-4m+2=0\Leftrightarrow m=1$.\\
		Vậy có một giá trị của $m$ thỏa mãn hàm số đã cho liên tục tại $x=2$.}
\end{vd}
\begin{vd}%[DCHT Toán 11 - KNTT -Hứa Chí Ninh]%[VD]%[1K5KG-5]
	Tìm giá trị của $m$ để hàm số $f\left( x \right)=\left\{ \begin{array}{*{35}{l}}
		\dfrac{\sqrt{1-x}-\sqrt{1+x}}{x} & \text{khi} & x<0 \\
		m+\dfrac{1-x}{1+x} & \text{khi} & x\ge 0 \\
	\end{array} \right.$ liên tục tại $x=0$?
	% \dapso{$m=-2$.}
	\loigiai{
		Ta có\\
		$\lim\limits_{x\to {0^{+}}} f\left( x \right)=\lim\limits_{x\to {0^{+}}} \left( m+\dfrac{1-x}{1+x} \right)=m+1$.
		$\lim\limits_{x\to {0^{-}}} f\left( x \right)=\lim\limits_{x\to {0^{-}}} \left( \dfrac{\sqrt{1-x}-\sqrt{1+x}}{x} \right)=\lim\limits_{x\to {0^{-}}} \dfrac{-2x}{x\left( \sqrt{1-x}+\sqrt{1+x} \right)}=\lim\limits_{x\to {0^{-}}} \dfrac{-2}{\left( \sqrt{1-x}+\sqrt{1+x} \right)}=-1$.\\
		$f\left( 0 \right)=m+1$.\\
		Để hàm liên tục tại $x=0$ thì $\lim\limits_{x\to {0^{+}}} f\left( x \right)=\lim\limits_{x\to {0^{-}}} f\left( x \right)=f\left( 0 \right)\Leftrightarrow m+1=-1\Rightarrow m=-2$.}
\end{vd}
\subsubsection{Bài tập rèn luyện}
% \centerline{\fcolorbox{red}{yellow!50}{\bf {BÀI TẬP TỰ LUẬN}}}
\begin{bt}%[DCHT Toán 11 - KNTT -Hứa Chí Ninh]%[1K5BG-3]
	Cho hàm số $f\left( x \right)=\left\{ \begin{aligned}
		& \dfrac{x^2-1}{x-1}&\text{ khi}\,\,x<3,\,\,x\ne 1 \\
		& 4&\text{ khi}\,\,x=1 \\
		& \sqrt{x+1}&\text{ khi}\,\,x\ge 3 \\
	\end{aligned} \right.$. Xét tính liên tục của hàm số $f\left( x \right)$?
	
	% \dapso{Hàm số $f(x)$ gián đoạn tại 2 điểm $x=1$ và $x=3$.}
	\loigiai{
		Hàm số $y=f\left( x \right)$ có tập xác định $\mathscr{D}=\mathbb{R}$.\\
		Dễ thấy hàm số $y=f\left( x \right)$ liên tục trên mỗi khoảng $\left( -\infty ;1 \right),\left( 1;3 \right)$ và $\left( 3;+\infty \right)$.\\
		Ta có $\left\{ \begin{aligned}
			& f\left( 1 \right)=4 \\
			& \lim\limits_{x\to 1} f\left( x \right)=\lim\limits_{x\to 1} \dfrac{x^2-1}{x-1}=\lim\limits_{x\to 1} \left( x+1 \right)=2 \\
		\end{aligned} \right.\Rightarrow f\left( x \right)$ gián đoạn tại $x=1.$\\
		Ta có $\left\{ \begin{aligned}
			& f\left( 3 \right)=2 \\
			& \lim\limits_{x\to {3^{-}}} f\left( x \right)=\lim\limits_{x\to {3^{-}}} \dfrac{x^2-1}{x-1}=\lim\limits_{x\to {3^{-}}} \left( x+1 \right)=4 \\
		\end{aligned} \right.\Rightarrow f\left( x \right)$ gián đoạn tại $x=3$.}
\end{bt}
\begin{bt}%[DCHT Toán 11 - KNTT -Hứa Chí Ninh]%[1K5KG-5]
	Cho hàm số $f\left( x \right)=\left\{ \begin{aligned}
		& \dfrac{\sqrt{x+3}-2}{x-1} \text{ khi }\left( x>1 \right) \\
		& {m^2}+m+\dfrac{1}{4}\text{  } \text{ khi }\left( x\le 1 \right) \\
	\end{aligned} \right.$. Tìm tất cả các giá trị của tham số thực $m$ để hàm số $f\left( x \right)$ liên tục tại $x=1$?
	% \dapso{$m\in \left\{ 0;-1 \right\}$.}
	\loigiai{
		Ta có $\lim\limits_{x\to {1^{+}}} f\left( x \right)=\lim\limits_{x\to {1^{+}}} \dfrac{\sqrt{x+3}-2}{x-1}=\lim\limits_{x\to {1^{+}}} \dfrac{1}{\sqrt{x+3}+2}=\dfrac{1}{4}$; $f\left( 1 \right)=\lim\limits_{x\to {1^{-}}} f\left( x \right)=m^2+m+\dfrac{1}{4}$.\\
		Để hàm số $f\left( x \right)$ liên tục tại $x=1$ thì $m^2+m+\dfrac{1}{4}=\dfrac{1}{4}\Leftrightarrow \left[ \begin{aligned}
			& m=-1 \\
			& m=0 \\
		\end{aligned} \right.$.}
\end{bt}
\begin{bt}%[DCHT Toán 11 - KNTT -Hứa Chí Ninh]%[1K5KG-5]	
	Tìm $a$ để hàm số  $f\left( x \right)=\left\{ \begin{aligned}
		& 2x+a\,\,\,\,\,\,\,\,\,\,\,\,\,\,\,\,\,\,\,\,\,\,\,\text{ khi }\,\,\,x\le 1 \\
		& \dfrac{x^3-x^2+2x-2}{x-1}\,\,\,\text{ khi }\,\,\,x>1 \\
	\end{aligned} \right.$ liên tục trên $\mathbb{R}$?
	% \dapso{$a=1.$}
	\loigiai{
		Khi $x<1$ thì $f\left( x \right)=2x+a$ là hàm đa thức nên liên tục trên khoảng $\left( -\infty ;\,1 \right)$.\\
		Khi $x>1$ thì $f\left( x \right)=\dfrac{x^3-x^2+2x-2}{x-1}$ là hàm phân thức hữu tỉ xác định trên khoảng $\left( 1;\,+\infty \right)$ nên liên tục trên khoảng $\left( 1;\,+\infty \right)$.\\
		Xét tính liên tục của hàm số tại điểm $x=1$, ta có  \\
		+ $f\left( 1 \right)=2+a$.\\
		+ $\lim\limits_{x\,\to \,{1^{-}}} f\left( x \right)=\lim\limits_{x\,\to \,{1^{-}}} \left( 2x+a \right)=2+a$.\\
		+ $\lim\limits_{x\,\to \,{1^{+}}} f\left( x \right)=\lim\limits_{x\,\to \,{1^{+}}} \dfrac{x^3-x^2+2x-2}{x-1}=\lim\limits_{x\,\to \,{1^{+}}} \dfrac{\left( x-1 \right)\left( {x^2}+2 \right)}{x-1}=\lim\limits_{x\,\to \,{1^{+}}} \left( {x^2}+2 \right)=3$.\\
		Hàm số $f\left( x \right)$ liên tục trên $\mathbb{R}$ $\Leftrightarrow $ hàm số $f\left( x \right)$ liên tục tại $x=1$.\\
		$\lim\limits_{x\,\to \,{1^{-}}} f\left( x \right)=\lim\limits_{x\,\to \,{1^{+}}} f\left( x \right)=f\left( 1 \right)$ $\rightarrow$ $2a+1=3$ $\rightarrow$ $a=1$.}
\end{bt}
\begin{bt}%[DCHT Toán 11 - KNTT -Hứa Chí Ninh]%[TH]%[1K5KG-5]	
	Tìm $m$ để hàm số $f(x)=\left\{ \begin{aligned}
		& \dfrac{x^2+4x+3}{x+1} \text{ khi } x>-1 \\
		& mx+2 \text{ khi } x\le -1 \\
	\end{aligned} \right.$ liên tục tại điểm $x=-1$.
	% \dapso{$m=2$.}
	\loigiai{
		Ta có   $\lim\limits_{x\to {{\left( -1 \right)}^{+}}} f\left( x \right)=\lim\limits_{x\to {{\left( -1 \right)}^{+}}} \dfrac{x^2+4x+3}{x+1}=\lim\limits_{x\to {{\left( -1 \right)}^{+}}} \dfrac{\left( x+1 \right)\left( x+3 \right)}{x+1}=\lim\limits_{x\to {{\left( -1 \right)}^{+}}} \left( x+3 \right)=2$.\\
		$\lim\limits_{x\to {{\left( -1 \right)}^{-}}} f\left( x \right)=\lim\limits_{x\to {{\left( -1 \right)}^{-}}} \left( mx+2 \right)=-m+2$.\\
		$f\left( -1 \right)=-m+2$.\\
		Để hàm số đã cho liên tục tại điểm $x=-1$ thì $\lim\limits_{x\to {{\left( -1 \right)}^{+}}} f\left( x \right)=\lim\limits_{x\to {{\left( -1 \right)}^{-}}} f\left( x \right)=f\left( -1 \right)\Leftrightarrow 2=-m+2\Leftrightarrow m=0$.}
\end{bt}
\begin{bt}%[DCHT Toán 11 - KNTT -Hứa Chí Ninh]%[VD]%[1K5BG-4]	
	Cho hàm số $f\left( x \right)=\left\{ \begin{matrix}
		\sin \pi x & \text{khi}\,\,\left| x \right|\le 1 \\
		x+1\ & \text{khi}\,\ \left| x \right|>1 \\
	\end{matrix} \right.$. Tìm các khoảng liên tục của hàm số?
	% \dapan{ Hàm số liên tục trên các khoảng $\left( -\infty ;1 \right)$ và $\left( 1;+\infty \right)$.}
	\loigiai{
		Ta có   $\lim\limits_{x\to {1^{+}}} \left( x+1 \right)=2$ và $\lim\limits_{x\to {1^{-}}} \sin \pi x=0\Rightarrow \lim\limits_{x\to {1^{+}}} f\left( x \right)\ne \lim\limits_{x\to {1^{-}}} f\left( x \right)$ do đó hàm số gián đoạn tại $x=1$.\\
		Tương tự $\lim\limits_{x\to {{\left( -1 \right)}^{-}}} \left( x+1 \right)=0$ và $\lim\limits_{x\to {{\left( -1 \right)}^{+}}} \sin \pi x=0$\\
		$\Rightarrow \lim\limits_{x\to {{\left( -1 \right)}^{+}}} f\left( x \right)=\lim\limits_{x\to {{\left( -1 \right)}^{-}}} f\left( x \right)=\lim\limits_{x\to -1} f\left( x \right)=f\left( -1 \right)$ do đó hàm số liên tục tại $x=-1$.\\
		Với $x\ne \pm 1$ thì hàm số liên tục trên tập xác định.\\
		Vậy hàm số đã cho liên tục trên các khoảng $\left( -\infty ;1 \right)$ và $\left( 1;+\infty \right)$.}
\end{bt}
\begin{bt}%[DCHT Toán 11 - KNTT -Hứa Chí Ninh]%[TH]%[1K5KG-4]	
	Cho hàm số $f\left( x \right)=\left\{ \begin{aligned}
		& \sin x\,\,\,\,\text{nếu}\,\cos x\ge 0 \\
		& 1+\cos x\,\,\,\,\text{nếu}\,\cos x<0 \\
	\end{aligned} \right..$ Hỏi hàm số $f$ có tất cả bao nhiêu điểm gián đoạn trên khoảng $\left( 0;2018 \right)$?
	% \dapso{$642$.}
	\loigiai{
		Vì $f$ là hàm lượng giác nên hàm số $f$ gián đoạn khi và chỉ khi hàm số $f$ gián đoạn tại $x$ làm cho $\cos x=0$ $\Leftrightarrow x=\dfrac{\pi }{2}+k\pi \,\left( k\in \mathbb{Z} \right)\in \left( 0;2018 \right)$ $\Leftrightarrow 0<\dfrac{\pi }{2}+k\pi <2018\Leftrightarrow 0<\dfrac{1}{2}+k<\dfrac{2018}{\pi }$ $\Leftrightarrow -\dfrac{1}{2}<k<\dfrac{2018}{\pi }-\dfrac{1}{2}\Leftrightarrow 0\le k\le 641$.}
\end{bt}
\begin{bt}%[DCHT Toán 11 - KNTT -Hứa Chí Ninh]%[TH]%[1K5KG-5]	
	Tìm $m$ để hàm số $y=f\left( x \right)=\left\{ \begin{aligned}
		& {x^2}+2\sqrt{x-2} \text{ khi } x\ge 2 \\
		& 5x-5m+m^2 \text{ khi } x<2 \\
	\end{aligned} \right.$ liên tục trên $\mathbb{R}$?
	% \dapso{$m=2; m=3$.}
	\loigiai{
		Tập xác định  $\mathbb{R}\,.$\\
		+ Xét trên $\left( 2;\,+\infty \right)$ khi đó $f\left( x \right)=x^2+2\sqrt{x-2}$.\\
		$\forall {x_0}\,\in \left( 2;\,+\infty \right), \,\lim\limits_{x\to {x_0}} \left( {x_0}^2\,+2\sqrt{x_0-2} \right)=x_0^2\,+2\sqrt{x_0-2}=f\left( {x_0} \right)\,\Rightarrow $hàm số liên tục trên $\left( 2;\,+\infty \right)$.\\
		+ Xét trên $\left( -\infty ;\,2 \right)$ khi đó $f\left( x \right)=5x-5m+m^2$ là hàm đa thức liên tục trên $\mathbb{R}\,\Rightarrow $ hàm số liên tục trên $\left( -\infty ;\,2 \right)$.\\
		+ Xét tại $x_0=2$, ta có  $f\left( 2 \right)=4$.\\
		$\lim\limits_{x\to {2^{+}}} f\left( x \right)=\lim\limits_{x\to {2^{+}}} \left( {x^2}+2\sqrt{x-2} \right)=4;\,\lim\limits_{x\to {2^{-}}} f\left( x \right)=\lim\limits_{x\to {2^{-}}} \left( 5x-5m+m^2 \right)=m^2-5m+10$.
		Để hàm số đã cho liên tục trên $\mathbb{R}$ thì nó phải liên tục tại $x_0=2$.\\
		$\Leftrightarrow \lim\limits_{x\to {2^{+}}} f\left( x \right)=\,\lim\limits_{x\to {2^{-}}} f\left( x \right)=f\left( 2 \right)\Leftrightarrow {m^2}-5m+10=4\,\Leftrightarrow {m^2}-5m+6=0\,\Leftrightarrow \left[ \begin{aligned}
			& m=2 \\
			& m=3 \\
		\end{aligned} \right.$.}
\end{bt}
\begin{bt}%[DCHT Toán 11 - KNTT -Hứa Chí Ninh]%[VD]%[1K5KG-5]	
	Cho hàm số $f\left( x \right)=\left\{ \begin{aligned}
		& \,3x+a-1 \text{ khi } x\le 0 \\
		& \dfrac{\sqrt{1+2x}-1}{x} \text{ khi } x>0 \\
	\end{aligned} \right.$. Tìm tất cả giá trị thực của a để hàm số đã cho liên tục trên $\mathbb{R}$.
	% \dapso{$a=2$.}
	\loigiai{
		Hàm số liên tục tại mọi điểm $x\ne 0$ với mọi a.\\
		Với $x=0$, ta có $f\left( 0 \right)=\,a\,-1.$\\
		$\lim\limits_{x\to {0^{-}}} f\left( x \right)\,=\,\lim\limits_{x\to {0^{-}}} \left( 3x+a-1 \right)\,=\,a-1$.\\
		$\lim\limits_{x\to {0^{+}}} f\left( x \right)\,=\,\lim\limits_{x\to {0^{+}}} \dfrac{\sqrt{1+2x}-1}{x}\,=\,\lim\limits_{x\to {0^{+}}} \dfrac{2x}{x\left( \sqrt{1+2x}+1 \right)}=\lim\limits_{x\to {0^{+}}} \dfrac{2}{\sqrt{1+2x}+1}=1$.\\
		Hàm số liên tục trên $\mathbb{R}$ khi và chỉ khi hàm số liên tục tại $x=0\Leftrightarrow a-1\,=\,1\Leftrightarrow a=2$.}
\end{bt}
\begin{bt}%[DCHT Toán 11 - KNTT -Hứa Chí Ninh]%[TH]%[1K5BG-5]	
	Có bao nhiêu giá trị thực của tham số $m$ để hàm số $f\left( x \right)=\left\{ \begin{array}{*{35}{l}}
		m^2{x^2} & \text{khi }x\le 2 \\
		\left( 1-m \right)x & \text{khi }x>2 \\
	\end{array} \right.$ liên tục trên $\mathbb{R}$?
	% \dapso{$2$.}
	\loigiai{
		Ta có hàm số luôn liên tục với $\forall x\ne 2$.\\
		Tại $x=2$, ta có $\lim\limits_{x\to {2^{+}}} f\left( x \right)=\lim\limits_{x\to {2^{-}}} \left( 1-m \right)x=\left( 1-m \right)2$;\\
		$\lim\limits_{x\to {2^{-}}} f\left( x \right)=\lim\limits_{x\to {2^{-}}} \left( {m^2}{x^2} \right)=4m^2$; $f\left( 2 \right)=4m^2$.\\
		Hàm số liên tục tại $x=2$ khi và chỉ khi\\
		$\lim\limits_{x\to {2^{+}}} f\left( x \right)=\lim\limits_{x\to {2^{+}}} f\left( x \right)=f\left( 2 \right)\Leftrightarrow 4m^2=\left( 1-m \right)2\Leftrightarrow 4m^2+2m-2=0.\left( 1 \right)$\\
		Phương trình luôn có hai nghiệm thực phân biệt. Vậy có hai giá trị của $m$.\\
	}
\end{bt}
\begin{bt}%[DCHT Toán 11 - KNTT -Hứa Chí Ninh]%[TH]%[1K5BG-5]	
	Tìm $P$ để hàm số $y=\left\{ \begin{aligned}
		& \dfrac{x^2-4x+3}{x-1} \text{ khi } x>1 \\
		& 6Px-3 \text{ khi } x\le 1\,\,\, \\
	\end{aligned} \right.$ liên tục trên $\mathbb{R}$.
	% \dapso{$P=\dfrac{1}{6}.$}
	\loigiai{
		Hàm số $y=f\left( x \right)$ liên tục trên $\mathbb{R}\Rightarrow $ $y=f\left( x \right)$ liên tục tại $x=1$
		$\Rightarrow $ $\lim\limits_{x\to {1^{+}}} f\left( x \right)=\lim\limits_{x\to {1^{-}}} f\left( x \right)=f\left( 1 \right)$.\\
		$\lim\limits_{x\to {1^{+}}} f\left( x \right)=\lim\limits_{x\to {1^{+}}} \dfrac{x^2-4x+3}{x-1}=\lim\limits_{x\to {1^{+}}} \left( x-3 \right)=-2$.\\
		$\lim\limits_{x\to {1^{-}}} f\left( x \right)=\lim\limits_{x\to {1^{-}}} \left( 6Px-3 \right)=6P-3$.\\
		$f\left( 1 \right)=6P-3$.\\
		Do đó $\lim\limits_{x\to {1^{+}}} f\left( x \right)=\lim\limits_{x\to {1^{-}}} f\left( x \right)=f\left( 1 \right)$ $\Leftrightarrow 6P-3=-2\Leftrightarrow P=\dfrac{1}{6}$.}
\end{bt}
\begin{bt}%[DCHT Toán 11 - KNTT -Hứa Chí Ninh]%[TH]%[1K5BG-5]	
	Hàm số ${f(x)=\left\{ \begin{aligned}
			& ax+b+1 \text{ khi } x>0 \\
			& a\cos x+b\sin x \text{ khi } x\le 0 \\
		\end{aligned} \right.}$ liên tục trên ${\mathbb{R}}$ khi và chỉ khi
	% \dapso{ ${a-b=1}$.}
	\loigiai{
		Khi $x<0$ thì $f\left( x \right)=a\cos x+b\sin x$ liên tục với $x<0$.\\
		Khi $x>0$ thì $f\left( x \right)=ax+b+1$ liên tục với mọi $x>0$.\\
		Tại $x=0$ ta có $f\left( 0 \right)=a$.\\
		${\lim\limits_{x\to {0^{+}}} f\left( x \right)}{=\lim\limits_{x\to {0^{+}}} \left( ax+b+1 \right)}{=b+1}$.\\
		${\lim\limits_{x\to {0^{-}}} f\left( x \right)}{=\lim\limits_{x\to {0^{-}}} \left( a\cos x+b\sin x \right)}{=a}$.\\
		Để hàm số liên tục tại ${x=0}$ thì ${\lim\limits_{x\to {0^{+}}} f\left( x \right)}{=\lim\limits_{x\to {0^{-}}} f\left( x \right)}=f\left( 0 \right)\Leftrightarrow a=b+1\Leftrightarrow a-b=1$.}
\end{bt}
\begin{bt}%[DCHT Toán 11 - KNTT -Hứa Chí Ninh]%[TH]%[1K5BG-5]	
	Cho hàm số $y=\left\{ \begin{aligned}
		& 3x+1 \text{ khi } x\ge -1 \\
		& x+m \text{ khi } x<-1 \\
	\end{aligned} \right.$, $m$ là tham số. Tìm $m$ để hàm số liên tục trên $\mathbb{R}$.
	% \dapso{ $m=-1$}
	\loigiai{
		Ta có hàm số liên tục trên các khoảng $\left( -\infty ;\,-1 \right)$ và $\left( -1;\,+\infty \right)$.\\
		Xét tính liên tục của hàm số tại $x=-1$.\\
		Có $y\left( -1 \right)=-2=\lim\limits_{x\to -{1^{+}}} y$ và $\lim\limits_{x\to -{1^{-}}} y=-1+m$.\\
		Để hàm số liên tục trên $\mathbb{R}$ thì $y\left( -1 \right)=\lim\limits_{x\to -{1^{+}}} y=\lim\limits_{x\to -{1^{-}}} y\Leftrightarrow -2=-1+m\Leftrightarrow m=-1$.}
\end{bt}
% \centerline{\fcolorbox{red}{yellow!50}{\bf {CÂU HỎI TRẮC NGHIỆM }}}
% \Opensolutionfile{ans}[ans/ans-1K5-3-Dang4]
% \begin{ex}%[DCHT Toán 11 - KNTT -Hứa Chí Ninh]%[1K5BG-5]
% 	Để hàm số $y=\left\{ \begin{aligned}
% 		& {x^2}+3x+2\begin{matrix}
% 			{} & \text{khi}\begin{matrix}
% 				{} & x\le -1 \\
% 			\end{matrix} \\
% 		\end{matrix} \\
% 		& 4x+a\begin{matrix}
% 			{} & {} & \,\,\text{khi}\begin{matrix}
% 				{} & x>-1 \\
% 			\end{matrix} \\
% 		\end{matrix} \\
% 	\end{aligned} \right.$ liên tục tại điểm $x=-1$ thì giá trị của $a$ là
% 	\choice
% 	{$-4$}
% 	{\True $4$}
% 	{$1$}
% 	{$-1$}
% 	\loigiai{
% 		Hàm số liên tục tại $x=-1$ khi và chỉ khi $\lim\limits_{x\to -{1^{+}}} y=\lim\limits_{x\to -{1^{-}}} y=y\left( -1 \right).$\\
% 		$\Leftrightarrow \lim\limits_{x\to -{1^{+}}} \left( 4x+a \right)=\lim\limits_{x\to -{1^{-}}} \left( {x^2}+3x+2 \right)=y\left( -1 \right)$ $\Leftrightarrow a-4=0\Leftrightarrow a=4$.}
% \end{ex}
% \begin{ex}%[DCHT Toán 11 - KNTT -Hứa Chí Ninh]%[1K5BG-5]	
% 	Tìm giá trị thực của tham số $m$ để hàm số $f\left( x \right)=\left\{ \begin{aligned}
% 		& \dfrac{x^3-x^2+2x-2}{x-1} \text{ khi } x\ne 1 \\
% 		& 3x+m  \text{ khi } x=1 \\
% 	\end{aligned} \right.$ liên tục tại $x=1$.
% 	\choice
% 	{\True $m=0$}
% 	{$m=6$}
% 	{$m=4$}
% 	{$m=2$}
% 	\loigiai{
% 		Ta có   $f\left( 1 \right)=m+3$.\\
% 		$\lim\limits_{x\to 1} f\left( x \right)=\lim\limits_{x\to 1} \dfrac{x^3-x^2+2x-2}{x-1}=\lim\limits_{x\to 1} \dfrac{\left( x-1 \right)\left( {x^2}+2 \right)}{x-1}=\lim\limits_{x\to 1} \left( {x^2}+2 \right)=3$.\\
% 		Để hàm số $f\left( x \right)$ liên tục tại $x=1$ thì $\lim\limits_{x\to 1} f\left( x \right)=f\left( 1 \right)\Leftrightarrow 3=m+3\Leftrightarrow m=0$.}
% \end{ex}
% \begin{ex}%[DCHT Toán 11 - KNTT -Hứa Chí Ninh]%[1K5GG-5]	
% 	Cho hàm số $f\left( x \right)=\left\{ \begin{aligned}
% 		& \dfrac{{x^{2016}}+x-2}{\sqrt{2018\text{x}+1}-\sqrt{x+2018}} \text{ khi } x\ne 1 \\
% 		& k\,\,\,\,\,\,\,\,\,\,\,\,\,\,\,\,\,\,\,\,\,\,\,\,\,\,\,\,\,\,\,\,\,\,\,\,\,\,\,\,\,\,\,\,\,\,\,\,\, \text{ khi } x=1 \\
% 	\end{aligned} \right.$. Tìm $k$ để hàm số $f\left( x \right)$ liên tục tại $x=1$.
% 	\choice
% 	{\True $k=2\sqrt{2019}$}
% 	{$k=\dfrac{2017.\sqrt{2018}}{2}$}
% 	{$k=1$}
% 	{$k=\dfrac{20016}{2017}\sqrt{2019}$}
% 	\loigiai{
% 		Ta có   $\lim\limits_{x\to 1} \dfrac{{x^{2016}}+x-2}{\sqrt{2018\text{x}+1}-\sqrt{x+2018}}=\lim\limits_{x\to 1} \dfrac{\left( {x^{2016}}-1+x-1 \right)\left( \sqrt{2018\text{x}+1}+\sqrt{x+2018} \right)}{2017\text{x}-2017}$\\
% 		$=\lim\limits_{x\to 1} \dfrac{\left( x-1 \right)\left( {x^{2015}}+{x^{2014}}+...+x+1+1 \right)\left( \sqrt{2018\text{x}+1}+\sqrt{x+2018} \right)}{2017\left( \text{x}-1 \right)}=2\sqrt{2019}$.\\
% 		Để hàm số liên tục tại $x=1$ $\Leftrightarrow \lim\limits_{x\to 1} f\left( x \right)=f\left( 1 \right)$ $\Leftrightarrow k=2\sqrt{2019}$.}
% \end{ex}
% \begin{ex}%[DCHT Toán 11 - KNTT -Hứa Chí Ninh]%[1K5KG-5]	
% 	Cho hàm số $f\left( x \right)=\left\{ \begin{aligned}
% 		& \dfrac{\sqrt{x}-1}{x-1} \text{ khi } x\ne 1 \\
% 		& a \text{ khi } x=1 \\
% 	\end{aligned} \right.$. Tìm $a$ để hàm số liên tục tại $x_0=1$.
% 	\choice
% 	{$a=0$}
% 	{$a=-\dfrac{1}{2}$}
% 	{\True $a=\dfrac{1}{2}$}
% 	{$a=1$}
% 	\loigiai{
% 		Ta có $\lim\limits_{x\to 1} f\left( x \right)=\lim\limits_{x\to 1} \dfrac{\sqrt{x}-1}{x-1}=\lim\limits_{x\to 1} \dfrac{\sqrt{x}-1}{\left( \sqrt{x}-1 \right)\left( \sqrt{x}+1 \right)}=\lim\limits_{x\to 1} \dfrac{1}{\sqrt{x}+1}=\dfrac{1}{2}$.\\
% 		Để hàm số liên tục tại $x_0=1$ khi $\lim\limits_{x\to 1} f\left( x \right)=f\left( 1 \right)\Leftrightarrow a=\dfrac{1}{2}$.}
% \end{ex}
% \begin{ex}%[DCHT Toán 11 - KNTT -Hứa Chí Ninh]%[1K5BG-5]	
% 	Biết hàm số $f\left( x \right)=\left\{ \begin{aligned}
% 		& 3x+b \text{ khi } x\le -1 \\
% 		& x+a \text{ khi } x>-1 \\
% 	\end{aligned} \right.$ liên tục tại $x=-1$. Mệnh đề nào dưới đây đúng?
% 	\choice
% 	{\True $a=b-2$}
% 	{$a=-2-b$}
% 	{$a=2-b$}
% 	{$a=b+2$}
% 	\loigiai{
% 		$\lim\limits_{x\,\to \,-{1^{-}}} f\left( x \right)=f\left( -1 \right)=b-3$; $\lim\limits_{x\,\to \,-{1^{+}}} f\left( x \right)=a-1$. Để hàm số liên tục tại $x=-1$ thì $b-3=a-1\Leftrightarrow a=b-2$.}
% \end{ex}
% \begin{ex}%[DCHT Toán 11 - KNTT -Hứa Chí Ninh]%[1K5KG-5]	
% 	Cho hàm số $f\left( x \right)=\left\{ \begin{aligned}
% 		& \dfrac{3-x}{\sqrt{x+1}-2}\text{ khi }x\ne 3 \\
% 		& m\text{        khi $x=3$} \\
% 	\end{aligned} \right.$. Hàm số đã cho liên tục tại $x=3$ khi $m$ bằng bao nhiêu?
% 	\choice
% 	{$-1$}
% 	{$1$}
% 	{$4$}
% 	{\True $-4$}
% 	\loigiai{
% 		$f\left( 3 \right)=m$
% 		$\lim\limits_{x\to 3} f\left( x \right)=\lim\limits_{x\to 3} \dfrac{3-x}{\sqrt{x+1}-2}=\lim\limits_{x\to 3} \dfrac{\left( 3-x \right)\left( \sqrt{x+1}+2 \right)}{x-3}$ $=\lim\limits_{x\to 3} \left( -\sqrt{x+1}-2 \right)=-4.$\\
% 		Để hàm số liên tục tại $x=3$ thì $\lim\limits_{x\to 3} f\left( x \right)=f\left( 3 \right)$.\\
% 		Suy ra $m=-4$.}
% \end{ex}

% \begin{ex}%[DCHT Toán 11 - KNTT -Hứa Chí Ninh]%[1K5BG-5]	
% 	Biết hàm số $f\left( x \right)=\left\{ \begin{matrix}
% 		a{x^2}+bx-5 & \text{khi} & x\le 1 \\
% 		2ax-3b & \text{khi} & x>1 \\
% 	\end{matrix} \right.$ liên tục tại $x=1$ Tính giá trị của biểu thức $P=a-4b$.
% 	\choice
% 	{$P=-4$}
% 	{\True $P=-5$}
% 	{$P=5$}
% 	{$P=4$}
% 	\loigiai{
% 		Ta có $\lim\limits_{x\to {1^{-}}} f\left( x \right)=\lim\limits_{x\to {1^{-}}} \left( a{x^2}+bx-5 \right)=a+b-5=f\left( 1 \right)$.\\
% 		$\lim\limits_{x\to {1^{+}}} f\left( x \right)=\lim\limits_{x\to {1^{+}}} \left( 2ax-3b \right)=2a-3b$.
% 		Vì hàm số liên tục tại $x=1$ nên $a+b-5=2a-3b\Rightarrow a-4b=-5$.}
% \end{ex}
% \begin{ex}%[DCHT Toán 11 - KNTT -Hứa Chí Ninh]%[1K5BG-5]
% 	Tìm $m$ để hàm số $f(x)=\left\{ \begin{aligned}
% 		& {{\dfrac{x^2-x}{x-1}}} \text{ khi } x\ne 1 \\
% 		& m-1 \text { khi } \mathop x=1 \\
% 	\end{aligned} \right.$ liên tục tại $x=1$?
% 	\choice
% 	{$m=0$}
% 	{$m=-1$}
% 	{$m=1$}
% 	{\True $m=2$}
% 	\loigiai{
% 		Tập xác định $\mathscr{D}=R$\\
% 		Ta có $\lim\limits_{x\to 1} f(x)=\lim\limits_{x\to 1} \dfrac{x^2-x}{x-1}=\lim\limits_{x\to 1} x=1$ và $f(1)=m-1$.\\
% 		Hàm số liên tục tại $x=1\Leftrightarrow m-1=1\Leftrightarrow m=2$.}
% \end{ex}
% \begin{ex}%[DCHT Toán 11 - KNTT -Hứa Chí Ninh]%[1K5BG-5]	
% 	Có bao nhiêu số tự nhiên $m$ để hàm số $f\left( x \right)=\left\{ \begin{aligned}
% 		& \dfrac{x^2-3x+2}{x-1} \text{ khi } x\ne 1 \\
% 		& {m^2}+m-1 \text{ khi } x=1 \\
% 	\end{aligned} \right.$ liên tục tại điểm $x=1$?
% 	\choice
% 	{$0$}
% 	{$3$}
% 	{$2$}
% 	{\True $1$}
% 	\loigiai{
% 		$\lim\limits_{x\to 1} \dfrac{x^2-3x+2}{x-1}=\lim\limits_{x\to 1} \dfrac{\left( x-1 \right)\left( x-2 \right)}{x-1}=\lim\limits_{x\to 1} \left( x-2 \right)=-1$.\\
% 		Vì hàm số $f\left( x \right)$ liên tục tại điểm $x=1$ nên $\lim\limits_{x\to 1} f\left( x \right)=f\left( 1 \right)$\\
% 		$\Leftrightarrow {m^2}+m-1=-1$
% 		$\Leftrightarrow {m^2}+m=0\Leftrightarrow \left[ \begin{aligned}
% 			& m=0 \text{ (Thoả mãn)} \\
% 			& m=-1 \text{ (Loại)}. \\
% 		\end{aligned} \right.$}
% \end{ex}
% \begin{ex}%[DCHT Toán 11 - KNTT -Hứa Chí Ninh]%[1K5KG-5]	
% 	Tìm $a$ để hàm số $f\left( x \right)=\left\{ \begin{aligned}
% 		& \dfrac{\sqrt{x+2}-2}{x-2}\,\,\,\,\,\,\,\text{khi }x\ne 2 \\
% 		& 2x+a\,\,\,\,\,\,\,\,\,\,\,\,\,\,\,\,\,\text{khi}\,x=2 \\
% 	\end{aligned} \right.$ liên tục tại $x=2$?
% 	\choice
% 	{$\dfrac{15}{4}$}
% 	{\True $-\dfrac{15}{4}$}
% 	{$\dfrac{1}{4}$}
% 	{$1$}
% 	\loigiai{
% 		Ta có $f\left( 2 \right)=4+a$.\\
% 		Ta tính được $\lim\limits_{x\to 2} f\left( x \right)=\lim\limits_{x\to 2} \dfrac{x+2-4}{\left( x-2 \right)\left( \sqrt{x+2}+2 \right)}=\lim\limits_{x\to 2} \dfrac{1}{\sqrt{x+2}+2}=\dfrac{1}{4}$.\\
% 		Hàm số đã cho liên tục tại $x=2$ khi và chỉ khi $f\left( 2 \right)=\lim\limits_{x\to 2} f\left( x \right)\Leftrightarrow 4+a=\dfrac{1}{4}\Leftrightarrow a=-\dfrac{15}{4}$.\\
% 		Vậy hàm số liên tục tại $x=2$ khi $a=-\dfrac{15}{4}$.}
% \end{ex}
% \begin{ex}%[DCHT Toán 11 - KNTT -Hứa Chí Ninh]%[1K5KG-5]	
% 	Cho hàm số $f\left( x \right)=\left\{ \begin{aligned}
% 		& \dfrac{x^2-3x+2}{\sqrt{x+2}-2} \text{ khi } x>2 \\
% 		& {m^2}x-4m+6 \text{ khi } x\le 2 \\
% 	\end{aligned} \right.$, $m$ là tham số. Có bao nhiêu giá trị của $m$ để hàm số đã cho liên tục tại $x=2$?
% 	\choice
% 	{$3$}
% 	{$0$}
% 	{$2$}
% 	{\True $1$}
% 	\loigiai{
% 		Ta có\\
% 		$\lim\limits_{x\to {2^{+}}} f(x)=\lim\limits_{x\to {2^{+}}} \dfrac{x^2-3x+2}{\sqrt{x+2}-2}=\lim\limits_{x\to {2^{+}}} \dfrac{\left( x-2 \right)\left( x-1 \right)\left( \sqrt{x+2}+2 \right)}{x-2}=\lim\limits_{x\to {2^{+}}} \left( x-1 \right)\left( \sqrt{x+2}+2 \right)=4$.\\
% 		$\lim\limits_{x\to {2^{-}}} f(x)=\lim\limits_{x\to {2^{-}}} \left( {m^2}x-4m+6 \right)=2m^2-4m+6$.\\
% 		$f(2)=2m^2-4m+6$.\\
% 		Để hàm số liên tục tại $x=2$ thì $\lim\limits_{x\to {2^{+}}} f(x)=\lim\limits_{x\to {2^{-}}} f(x)=f(2)\Leftrightarrow 2m^2-4m+6=4\Leftrightarrow 2m^2-4m+2=0\Leftrightarrow m=1$.\\
% 		Vậy có một giá trị của $m$ thỏa mãn hàm số đã cho liên tục tại $x=2$.}
% \end{ex}
% \begin{ex}%[DCHT Toán 11 - KNTT -Hứa Chí Ninh]%[1K5KG-5]	
% 	Cho hàm số $f\left( x \right)=\left\{ \begin{aligned}
% 		& \dfrac{\sqrt{3x^2+2x-1}-2}{x^2-1},\ x\ne 1 \\
% 		& 4-m,\ \quad \quad \quad \quad \quad x=1 \\
% 	\end{aligned} \right.$. Hàm số $f\left( x \right)$ liên tục tại $x_0=1$ khi
% 	\choice
% 	{\True $m=3$}
% 	{$m=-3$}
% 	{$m=7$}
% 	{$m=-7$}
% 	\loigiai{
% 		Tập xác định $\mathscr{D}=\mathbb{R}$, $x_0=1\in \mathbb{R}$.\\
% 		Ta có $f\left( 1 \right)=4-m$.\\
% 		$\lim\limits_{x\to 1} f\left( x \right)=\lim\limits_{x\to 1} \dfrac{\sqrt{3x^2+2x-1}-2}{\left( x+1 \right)\left( x-1 \right)}$ $=\lim\limits_{x\to 1} \dfrac{\left( x-1 \right)\left( 3x+5 \right)}{\left( x+1 \right)\left( x-1 \right)\left( \sqrt{3x^2+2x-1}+2 \right)}$\\
% 		$=\lim\limits_{x\to 1} \dfrac{3x+5}{\left( x+1 \right)\left( \sqrt{3x^2+2x-1}+2 \right)}=1$.\\
% 		Hàm số $f\left( x \right)$ liên tục tại $x_0=1$ khi và chỉ khi $\lim\limits_{x\to 1} \left( x \right)=f\left( 1 \right)\Leftrightarrow 4-m=1\Leftrightarrow m=3$.}
% \end{ex}
% \begin{ex}%[DCHT Toán 11 - KNTT -Hứa Chí Ninh]%[1K5KG-5]	
% 	Tìm giá trị của tham số $m$ để hàm số $f\left( x \right)=\left\{ \begin{aligned}
% 		& \dfrac{x^2+3x+2}{x^2-1}\,\,\,\,\,\text{khi}\,\,\,\,\,\,x\,<\,-1 \\
% 		& mx+2\,\,\,\,\,\,\,\,\,\,\,\,\,\,\,\text{khi}\,\,\,\,\,\,x\,\ge \,-1 \\
% 	\end{aligned} \right.$ liên tục tại $x=-1$.
% 	\choice
% 	{$m=\dfrac{-3}{2}$}
% 	{$m=\dfrac{-5}{2}$}
% 	{$m=\dfrac{3}{2}$}
% 	{\True $m=\dfrac{5}{2}$}
% 	\loigiai{
% 		Ta có  \\
% 		$\bullet$ $f\left( -1 \right)=-m+2$.\\
% 		$\bullet$ $\lim\limits_{x\to {{\left( -1 \right)}^{+}}} f\left( x \right)=-m+2$.\\
% 		$\bullet$$\lim\limits_{x\to {{\left( -1 \right)}^{-}}} f\left( x \right)=\lim\limits_{x\to {{\left( -1 \right)}^{-}}} \dfrac{x^2+3x+2}{x^2-1}=\lim\limits_{x\to {{\left( -1 \right)}^{-}}} \dfrac{\left( x+1 \right)\left( x+2 \right)}{\left( x-1 \right)\left( x+1 \right)}=\lim\limits_{x\to {{\left( -1 \right)}^{-}}} \dfrac{x+2}{x-1}=\dfrac{-1}{2}$.\\
% 		Hàm số liên tục tại $x=-1\Leftrightarrow f\left( -1 \right)=\lim\limits_{x\to {{\left( -1 \right)}^{+}}} f\left( x \right)=\lim\limits_{x\to {{\left( -1 \right)}^{-}}} f\left( x \right)\Leftrightarrow -m+2=\dfrac{-1}{2}\Leftrightarrow m=\dfrac{5}{2}$.}
% \end{ex}
% \begin{ex}%[DCHT Toán 11 - KNTT -Hứa Chí Ninh]%[1K5KG-5]	
% 	Cho hàm số $f(x)=\left\{ \begin{matrix}
% 		\dfrac{\sqrt{x^2+4}-2}{x^2}\ \ \ \,\,\text{khi }\ x\ne 0 \\
% 		2a-\dfrac{5}{4}\ \ \ \ \ \ \ \ \ \ \ \,\text{khi }\ x=0 \\
% 	\end{matrix} \right.$. Tìm giá trị thực của tham số $a$ để hàm số $f(x)$ liên tục tại $x=0$.
% 	\choice
% 	{$a=-\dfrac{3}{4}$}
% 	{$a=\dfrac{4}{3}$}
% 	{$a=-\dfrac{4}{3}$}
% 	{\True $a=\dfrac{3}{4}$}
% 	\loigiai{
% 		Tập xác định $\mathscr{D}=\mathbb{R}$.\\
% 		$\lim\limits_{x\to 0} f(x)=\lim\limits_{x\to 0} \dfrac{\sqrt{x^2+4}-2}{x^2}=\lim\limits_{x\to 0} \dfrac{\left( \sqrt{x^2+4}-2 \right)\left( \sqrt{x^2+4}+2 \right)}{x^2\left( \sqrt{x^2+4}+2 \right)}$
% 		$=\lim\limits_{x\to 0} \dfrac{x^2+4-4}{x^2(\sqrt{x^2+4}+2)}=\lim\limits_{x\to 0} \dfrac{1}{\sqrt{x^2+4}+2}=\dfrac{1}{4}$.\\
% 		$f(0)=2a-\dfrac{5}{4}$.\\
% 		Hàm số $f(x)$ liên tục tại $x=0\Leftrightarrow \lim\limits_{x\to 0} f(x)=f(0)\Leftrightarrow 2a-\dfrac{5}{4}=\dfrac{1}{4}\Leftrightarrow a=\dfrac{3}{4}$.\\
% 		Vậy $a=\dfrac{3}{4}$.}
% \end{ex}
% \begin{ex}%[DCHT Toán 11 - KNTT -Hứa Chí Ninh]%[1K5BG-5]	
% 	Cho hàm số $f\left( x \right)=\left\{\begin{aligned}
% 		& {x^2}-2x+3\text{  khi }x\ne 1 \\
% 		& 3x+m-1\text{   khi }x=1 \\
% 	\end{aligned} \right.$. Tìm $m$ để hàm số liên tục tại $x_0=1$.
% 	\choice
% 	{$m=1$}
% 	{$m=3$}
% 	{\True $m=0$}
% 	{$m=2$}
% 	\loigiai{
% 		Tập xác định $\mathscr{D}=\mathbb{R}$.\\
% 		Ta có $f\left( 1 \right)=2+m$.\\
% 		$\lim\limits_{x\to 1} f\left( x \right)=\lim\limits_{x\to 1}{\mathop{ }}\left( {x^2}-2x+3 \right)=2$.\\
% 		Hàm số liên tục tại $x_0=1\Leftrightarrow \lim\limits_{x\to 1} f\left( x \right)=f\left( 1 \right)\Leftrightarrow 2=m+2\Leftrightarrow m=0$.}
% \end{ex}
% \begin{ex}%[DCHT Toán 11 - KNTT -Hứa Chí Ninh]%[1K5BG-5]
% 	Cho hàm số $f(x)=\left\{ \begin{aligned}
% 		& \dfrac{x^2-3x+2}{x-2} \text{ khi } x\ne 2 \\
% 		& a \text{ khi } x=2 \\
% 	\end{aligned} \right.$. Hàm số liên tục tại $x=2$ khi $a$ bằng
% 	\choice
% 	{\True $1$}
% 	{$0$}
% 	{$2$}
% 	{$-1$}
% 	\loigiai{
% 		Hàm số liên tục tại $x=2$ khi và chỉ khi $\lim\limits_{x\to 2} f(x)=f(2)$.\\
% 		Ta có $f(2)=a,\lim\limits_{x\to 2} f(x)=\lim\limits_{x\to 2} \dfrac{x^2-3x+2}{x-2}=\lim\limits_{x\to 2} (x-1)=1$. Do đó $a=1$.}
% \end{ex}
% \begin{ex}%[DCHT Toán 11 - KNTT -Hứa Chí Ninh]%[1K5KG-5]
% 	Tìm tất cả các giá trị thực của $m$ để hàm số $f(x)=\left\{ \begin{aligned}
% 		& \dfrac{\sqrt{x+1}-1}{x} \text{ khi }  x>0 \\
% 		& \sqrt{x^2+1}-m \text{ khi } \le 0 \\
% 	\end{aligned} \right.$ liên tục trên $\mathbb{R}$.
% 	\choice
% 	{$m=\dfrac{3}{2}$}
% 	{\True $m=\dfrac{1}{2}$}
% 	{$m=-2$}
% 	{$m=-\dfrac{1}{2}$}
% 	\loigiai{
% 		Khi $x>0$, ta có   $f(x)=\dfrac{\sqrt{x+1}-1}{x}$ liên tục trên khoảng $\left( 0;+\infty \right)$.\\
% 		Khi $x<0$, ta có   $f(x)=\sqrt{x^2+1}-m$ liên tục trên khoảng $\left( -\infty ;0 \right)$.\\
% 		Hàm số liên tục trên $\mathbb{R}$ khi và chỉ khi hàm số liên tục tại $x=0$.\\
% 		Ta có   $\lim\limits_{x\to {0^{+}}} f(x)=\lim\limits_{x\to {0^{+}}} \dfrac{\sqrt{x+1}-1}{x}=\lim\limits_{x\to {0^{+}}} \dfrac{1}{\sqrt{x+1}+1}=\dfrac{1}{2}$.\\
% 		$\lim\limits_{x\to {0^{-}}} f(x)=\lim\limits_{x\to {0^{-}}} \left( \sqrt{x^2+1}-m \right)=1-m=f\left( 0 \right)$.
% 		Do đó hàm số liên tục tại $x=0$ khi và chỉ khi $\dfrac{1}{2}=1-m\Leftrightarrow m=\dfrac{1}{2}$.}
% \end{ex}
% \begin{ex}%[DCHT Toán 11 - KNTT -Hứa Chí Ninh]%[1K5BG-5]	
% 	Tìm tất cả các giá trị thực của tham số $m$ để hàm số $f\left( x \right)=\left\{ \begin{array}{*{35}{l}}
% 		\dfrac{x^2-16}{x-4} & \text{khi} & x>4 \\
% 		mx+1 & \text{khi} & x\le 4 \\
% 	\end{array} \right.$ liên tục trên $\mathbb{R}$.
% 	\choice
% 	{$m=8$ hoặc $m=-\dfrac{7}{4}$}
% 	{\True $m=\dfrac{7}{4}$}
% 	{$m=-\dfrac{7}{4}$}
% 	{$m=-8$ hoặc $m=\dfrac{7}{4}$}
% 	\loigiai{
% 		*) Với $x>4$ thì $f\left( x \right)=\dfrac{x^2-16}{x-4}$ là hàm phân thức nên liên tục trên tập xác định của nó $\Rightarrow f\left( x \right)$ liên tục trên $\left( 4;+\infty \right)$.\\
% 		*) Với $x<4$ thì $f\left( x \right)=mx+1$ là hàm đa thức nên liên tục trên $\mathbb{R}\Rightarrow f\left( x \right)$ liên tục trên $\left( -\infty ;4 \right)$.\\
% 		Do vậy hàm số $f\left( x \right)$ đã liên tục trên các khoảng $\left( 4;+\infty \right)$, $\left( -\infty ;4 \right)$.\\
% 		Suy ra hàm số $f\left( x \right)$ liên tục trên $\mathbb{R}$ khi và chỉ khi $f\left( x \right)$ liên tục tại $x=4$. Do đó\\
% 		$ \lim\limits_{x\to {4^{+}}} f\left( x \right)=\lim\limits_{x\to {4^{-}}} f\left( x \right)=f\left( 4 \right)\Leftrightarrow \lim\limits_{x\to {4^{+}}} \dfrac{x^2-16}{x-4}=\lim\limits_{x\to {4^{-}}} \left( mx+1 \right)=4m+1\Leftrightarrow \lim\limits_{x\to {4^{+}}} \left( x+4 \right)=4m+1$
% 		$\Leftrightarrow 4m+1=8\Leftrightarrow m=\dfrac{7}{4}$.}
% \end{ex}
% \begin{ex}%[DCHT Toán 11 - KNTT -Hứa Chí Ninh]%[1K5KG-5]	
% 	Nếu hàm số $f\left( x \right)=\left\{ \begin{aligned}
% 		& {x^2}+ax+b\,\,\text{khi}\,\,x<-5\,\, \\
% 		& x+17\,\,\,\,\,\,\,\,\,\,\,\,\text{khi}\,-5\le x\le 10 \\
% 		& ax+b+10\,\,\,\text{khi}\,x>10 \\
% 	\end{aligned} \right.$ liên tục trên $\mathbb{R}$ thì $a+b$ bằng
% 	\choice
% 	{\True $-1$}
% 	{$0$}
% 	{$1$}
% 	{$2$}
% 	\loigiai{
% 		Với $x<-5$ ta có $f\left( x \right)=x^2+ax+b$, là hàm đa thức nên liên tục trên $\left( -\infty ;-5 \right)$.\\
% 		Với $-5<x<10$ ta có $f\left( x \right)=x+7$, là hàm đa thức nên liên tục trên $\left( -5;10 \right)$.\\
% 		Với $x>10$ ta có $f\left( x \right)=ax+b+10$, là hàm đa thức nên liên tục trên $\left( 10;+\infty \right)$.\\
% 		Để hàm số liên tục trên $\mathbb{R}$ thì hàm số phải liên tục tại $x=-5$ và $x=10$.\\
% 		Ta có  \\
% 		$f\left( -5 \right)=12$;$f\left( 10 \right)=17$.\\
% 		$\lim\limits_{x\to -{5^{-}}} f\left( x \right)=\lim\limits_{x\to -{5^{-}}} \left( {x^2}+ax+b \right)$ $=-5a+b+25$.\\
% 		$\lim\limits_{x\to -{5^{+}}} f\left( x \right)=\lim\limits_{x\to -{5^{+}}} \left( x+17 \right)=12$.
% 		$\lim\limits_{x\to {{10}^{-}}} f\left( x \right)=\lim\limits_{x\to {{10}^{-}}} \left( x+17 \right)=27$.
% 		$\lim\limits_{x\to {{10}^{+}}} f\left( x \right)=\lim\limits_{x\to {{10}^{+}}} \left( ax+b+10 \right)=10a+b+10$.
% 		Hàm số liên tục tại $x=-5$ và $x=10$ khi\\
% 		$\left\{ \begin{aligned}
% 			& 5a+b+25=12 \\
% 			& 10a+b+10=27 \\
% 		\end{aligned} \right.\Leftrightarrow \left\{ \begin{aligned}
% 			& -5a+b=-13 \\
% 			& 10a+b=17 \\
% 		\end{aligned} \right.\Leftrightarrow \left\{ \begin{aligned}
% 			& a=2 \\
% 			& b=-3 \\
% 		\end{aligned} \right.\Rightarrow a+b=-1$.}
% \end{ex}
\Closesolutionfile{ans}
% \begin{indapan}{10}
% 	{ans/ans-1K5-3-Dang4}
% \end{indapan}
% \begin{dang}{Toán thực tế, liên môn về hàm số liên tục}
% \end{dang}
% \subsubsection{Ví dụ mẫu}
% \begin{vd}%[DCHT Toán 11 - KNTT- Phạm Tuấn]%[1K5BG-6]
% 	Trong kỹ thuật ứng dụng, chúng ta thường xuyên ghi nhận được các hàm số mà giá trị của nó thay đổi đột ngột tại một thời điểm $t$ xác định. Ví dụ:  Sự thay đổi điện áp của một mạch điện tại thời điểm t khi đóng hoặc ngắt mạch. Thông thường, giá trị $t = 0$ luôn được chọn là thời điểm bắt đầu cho việc đóng hoặc ngắt điện áp. Quá trình đóng, ngắt mạch trên có thể mô tả bằng mô hình toán học bởi hàm Heaviside
% 	\[
% 	u(t) = \heva{&0 && \text{ nếu } t <0\\& 1 && \text{ nếu } t \geq 0.}
% 	\]
% 	Hàm Heaviside có liên tục tại $t=0$ hay không?
% 	% \dapso{Không liên tục tại $t=0$}
% 	\loigiai{
% 		Ta có $\displaystyle \lim \limits_{t\to 0^+} u(t) = \lim \limits_{t\to 0^+} 1 =1$; $\displaystyle \lim \limits_{t\to 0^-} u(t) = \lim \limits_{t\to 0^-} 0 =0$. Do đó $\displaystyle \lim \limits_{t\to 0^+} u(t) \ne \displaystyle \lim \limits_{t\to 0^-} u(t)$. \\
% 		Vậy hàm Heaviside không liên tục tại $t=0$.
% 	}
% \end{vd}

% \begin{vd}%[DCHT Toán 11 - KNTT- Phạm Tuấn]%[1K5BG-6]
% 	Thuế thu nhập của một tiểu bang tại Hoa Kỳ được xác định bởi hàm số
% 	\[
% 	T(x) = \heva{& 0  && \text{ nếu } x \leq 0 \\&0{,}14x  && \text{ nếu } 0<x \leq 10000\\&c+0{,}21x  && \text{ nếu } x \geq 10000.}
% 	\]
% 	Trong đó $x$ là thu nhập tính bằng USD. Tìm $c$ để hàm số đã cho liên tục trên $\mathbb{R}$. 
% 	% \dapso{$c=-700$}
% 	\loigiai{
% 		Dễ thấy hàm số $T(x)$ liên tục khi $x \ne 0$  và $x \ne 10000$. \\
% 		Ta có $\displaystyle \lim\limits _{x \rightarrow 0^{-}} T(x) = \lim\limits _{x \rightarrow 0^{+}} T(x) =0$ nên hàm số liên tục tại $x=0$. \\
% 		Hàm số đã cho liên tục tại $x=10000$ khi và chỉ khi 
% 		\begin{align*}
% 			&\lim\limits _{x \rightarrow 10000^{-}} T(x) = \lim\limits _{x \rightarrow 10000^{+}} T(x) = T(10000) \\
% 			\Leftrightarrow~ & \lim\limits _{x \rightarrow 10000^{-}} 0{,}14x = \lim\limits _{x \rightarrow 10000^{+}} (c+0{,}21x) = c+2100 \\
% 			\Leftrightarrow~ & 1400 = c+2100 \Leftrightarrow c=-700.
% 		\end{align*}
% 	}
% \end{vd}

% \begin{vd}%[DCHT Toán 11 - KNTT- Phạm Tuấn]%[1K5BG-6]
% 	Thuế thu nhập của một tiểu bang tại Hoa Kỳ được xác định bởi hàm số
% 	\[
% 	T(x) = \heva{& 0  && \text{ nếu } x \leq 0 \\&a+0{,}12x  && \text{ nếu } 0<x \leq 20000\\&b+0{,}16(x-20000)  && \text{ nếu } x> 20000.}
% 	\]
% 	Trong đó $x$ là thu nhập tính bằng USD. Tìm $a,b$ để hàm số đã cho liên tục trên $\mathbb{R}$. 
% 	% \dapso{$a=0$; $b=2400$}
% 	\loigiai{
% 		Dễ thấy hàm số $T(x)$ liên tục khi $x \ne 0$  và $x \ne 20000$. \\
% 		Hàm số đã cho liên tục tại $x=0$ khi và chỉ khi 
% 		\begin{align*}
% 			&\lim\limits _{x \rightarrow 0^{-}} T(x) = \lim\limits _{x \rightarrow 0^{+}} T(x) = T(0) \\
% 			\Leftrightarrow~ & \lim\limits _{x \rightarrow 0^{-}} (a+0{,}12x)  = 0 \\
% 			\Leftrightarrow~ & a = 0 .
% 		\end{align*}
% 		Hàm số đã cho liên tục tại $x=20000$ khi và chỉ khi 
% 		\begin{align*}
% 			&\lim\limits _{x \rightarrow 20000^{-}} T(x) = \lim\limits _{x \rightarrow 20000^{+}} T(x) = T(20000) \\
% 			\Leftrightarrow~ & \lim\limits _{x \rightarrow 20000^{-}} 0{,}12x = \lim\limits _{x \rightarrow 20000^{+}} [b+0{,}16(x-20000)] = 2400 \\
% 			\Leftrightarrow~ & b=2400.
% 		\end{align*}
% 	}
% \end{vd}





% \begin{vd}%[DCHT Toán 11 - KNTT- Phạm Tuấn]%[1K5BG-6]
% 	Số lượng đơn vị hàng tồn kho trong một công ty nhỏ được cho bởi
% 	$$
% 	N(t)=80\left(2 \left [\frac{t+2}{2}  \right ]-t\right)
% 	$$
% 	trong đó $t$ là thời gian tính bằng tháng. $[x]$ là số nguyên lớn nhất không vượt quá $x$ (ví dụ $[2{,}4]=2$, $[-2{,}7] = -3$).
% 	Hàm số $N(t)$ có liên tục tại $t=10$ hay không?
% 	% \dapso{Hàm số $N(t)$ không liên tục tại $t=10$}
% 	\loigiai{
% 		Khi $t \to 10^+$, ta có $\left [\dfrac{t+2}{2}  \right ] = 6$. \\
% 		Suy ra  $\displaystyle \lim\limits _{t \to 10^+} N(t) =\lim_{t \to 10^+}  80\left(2 \left [\dfrac{t+2}{2}  \right ]-t\right) = 160$. \\
% 		Khi $t \to 10^-$, ta có $\left [\dfrac{t+2}{2}  \right ] = 5$. \\
% 		Suy ra  $\displaystyle \lim\limits _{t \to 10^-} N(t) = \lim_{t \to 10^-} 80\left(2 \left [\dfrac{t+2}{2}  \right ]-t\right) = 0$. \\
% 		Vậy hàm số $N(t)$ không liên tục tại $t=10$.
% 	}
% \end{vd}


% \begin{vd}%[DCHT Toán 11 - KNTT- Phạm Tuấn]%[1K5BG-6]
% 	Giả sử giá điện sinh hoạt trong mỗi tháng dành cho các hộ gia đình được cho bởi bảng sau
% 	\begin{center}
% 		\begin{tabular}{|c|c|}
% 			\hline
% 			Mức kWh điện tiêu thụ  & Giá bán điện (VNĐ/ kWh) \\
% 			\hline
% 			Mức 1: từ $0$ đến $100$ kWh  & $1600$ \\
% 			\hline
% 			Mức 2: từ trên $100$ đến $300$ kWh  & $2000$ \\
% 			\hline
% 			Mức 3: trên $300$ kWh  & $3000$ \\
% 			\hline
% 		\end{tabular}
% 	\end{center}
% 	\begin{enumerate}
% 		\item Thiết lập hàm số $f(x)$ liên hệ giữa $x$ (kWh) điện tiêu thụ và số tiền $f(x)$ tương ứng phải trả.
% 		\item  Hàm số $f(x)$  có liên tục tại $x=100$ hay không?
% 	\end{enumerate}
% 	% \dapso{Hàm số $f(x)$ liên tục tại $x=100$}
% 	\loigiai{
% 		Từ giả thiết ta có
% 		\begin{eqnarray*}
% 			&& f(x)=\heva{&1600x &(0<x\leq 100)&\\&1600\cdot 100+2000\cdot (x-100) &(100<x\leq 300)&\\&1600\cdot 100+2000\cdot 200+3000\cdot(x-300)&(x>300)&\\} \\ 
% 			&\Leftrightarrow& f(x)=\heva{&1600x &(0<x\leq 100)&\\&2000x-40000 &(100<x\leq 300)&\\&3000x-340000 &(x>300)&.}
% 		\end{eqnarray*}
% 		Ta có 
% 		\begin{align*}
% 			&\lim\limits _{x \rightarrow 100^{-}} f(x) =  \lim\limits _{x \rightarrow 100^{-}} 1600x = 160 000 \\
% 			& \lim\limits _{x \rightarrow 100^{+}} f(x) =  \lim\limits _{x \rightarrow 100^{+}} (2000x-40000) = 160 000. 
% 		\end{align*}
% 		Suy ra  $\displaystyle \lim\limits _{x \rightarrow 100^{-}} f(x) = \lim\limits _{x \rightarrow 100^{+}} f(x) = f(100) =160 000 $.\\
% 		Vậy hàm số $f(x)$ liên tục tại $x=100$.
% 	}
% \end{vd}



% \subsubsection{Bài tập rèn luyện}
% % \centerline{\fcolorbox{red}{yellow!50}{\bf {BÀI TẬP TỰ LUẬN}}}
% \begin{bt}%[DCHT Toán 11 - KNTT -Vũ Hồng Toàn]%[1K5YG-6]
% 	Từ ngày 04/05/2023 giá điện sinh hoạt $Q(x)$ (đồng)/tháng được chia thành 6 bậc. Trong đó $x$ (KWh)  của 3 bậc đầu được tính như sau:
% 	$$Q(x)=\heva{&1728 x&\text{ khi }& 0<x\le 50\\&1786 x&\text{ khi }& 51\le x\le 100\\&2074 x&\text{ khi }& 101\le x\le 200.}$$
% 	Hỏi
% 	\begin{enumerate}
% 		\item $Q(x)$ có liên tục tại $x_0=50$ không?
% 		\item Một gia đình dùng hết $150$ kWh/tháng. Tính tiền điện phải trả cho tháng đó?
% 	\end{enumerate}
	
% 	% \dapso{a) $Q(x)$ không liên tục tại $x_0=50$; b) $279.400$ đồng.}
% 	\loigiai{
% 		\begin{enumerate}
% 			\item  Hàm số $Q(x)$ không liên tục tại $x_0=50$ vì $\displaystyle \lim\limits _{x \rightarrow 50^{+}} Q(x) $ không tồn tại .
% 			\item  Ta có $1728\cdot 50+ 1786\cdot 50+2074\cdot 50=279.400$.\\
% 			Vậy tiền điện phải trả cho tháng đó là $279.400$ đồng.
% 		\end{enumerate}
% 	}
% \end{bt}

% \begin{bt}%[DCHT Toán 11 - KNTT -Vũ Hồng Toàn]%[1K5YG-6]
% 	Tại một xưởng sản xuất bột đá thạch anh, giá bán (tính theo nghìn đồng) của $x$ (kg) bột đá thạch anh được tính theo công thức sau $P(x)=\heva{&4{,}5x &\text{ khi } &0<x\le 400\\&4x+k &\text{ khi } &x> 400.}$\\
% 	Khi đó
% 	\begin{enumerate}		
% 		\item  Với $k = 0$, xét tính liên tục của hàm số $P(x)$ trên $(0;+\infty)$.\\
% 		\item  Với giá trị nào của $k$ thì hàm số $P(x)$ liên tục trên $(0;+\infty)$.
% 	\end{enumerate}
% 	% \dapso{a) $P(x)$ liên tục trên $(0;+\infty)$; b) $k=200$.}
% 	\loigiai{
% 		\begin{enumerate}
% 			\item  Với $x_0 \in(0 ; 400)$ khi đó $P(x)=4{,}5 x$.\\
% 			Suy ra $\lim\limits_{x \rightarrow x_{0}} P(x)=\lim\limits_{x \rightarrow x_{0}}(4{,}5 \mathrm{x})=4{,}5 \mathrm{x}_0=P\left(\mathrm{x}_0\right)$.\\
% 			Do đó $P(x)$ liên tục trên $(0 ; 400)$.
% 			Tại $x_0=400$, ta có
% 			\begin{eqnarray*}
% 				&& \lim\limits_{x \rightarrow 400^-} P(x)=\lim \limits{n \to +\infty}_{x \rightarrow 400^{-}}(4{,}5 x)=4{,}5 \cdot 400=1800. \\
% 				&& \lim \limits{n \to +\infty}_{x \rightarrow 400^{+}} P(x)=\lim \limits{n \to +\infty}_{x \rightarrow 400^{+}}(4 x)=4 \cdot 400=1600.
% 			\end{eqnarray*}
% 			Suy ra $\lim\limits_{x \rightarrow 400^-} P(x) \neq \lim\limits_{x \rightarrow 400^+} P(x)$.\\
% 			Do đó không tồn tại $\lim\limits_{x \rightarrow 400} P(x)$.\\
% 			Vì vậy hàm số không liên tục tại $x=400$.\\
% 			Với $x_0 \in(400 ;+\infty)$ khi đó $P(x)=4x$.\\
% 			Suy ra $\lim\limits_{x \rightarrow x_{0}} P(x)=\lim\limits_{x \rightarrow x_{0}}(4 x)=4 x_0=P\left(x_0\right)$.\\
% 			Do đó $P(x)$ liên tục trên $(400 ;+\infty)$.
% 			Vậy hàm số liên tục trên $(0 ; 400)$ và $(400 ;+\infty)$.
% 			\item  Để hàm số $P(x)$ liên tục trên $(0 ;+\infty)$ thì $P(x)$ phải liên tục trên $x_0=400$.\\
% 			Do đó $\lim\limits_{x \rightarrow 400^-} P(x)=\lim\limits_{x \rightarrow 400^+} P(x) \Leftrightarrow 1800=4\cdot 400+k \Leftrightarrow k=200$.\\
% 			Vậy với $k=200$ thì hàm số liên tục trên $(0 ;+\infty)$.
% 		\end{enumerate}
% 	}
% \end{bt}

% \begin{bt}%[DCHT Toán 11 - KNTT -Vũ Hồng Toàn]%[1K5YG-6]
% 	\immini{
% 		Cho hàm số $y=f(x)$ có đồ thị như hình vẽ bên. Xét tính liên tục của hàm số $y=f(x)$ trên tập xác định của nó.
% 	}{
% 		\begin{tikzpicture}[scale=.75, font=\footnotesize, line join=round, line cap=round,>=stealth]
% 			\def\a{-1} \def\b{2} \def\c{1} \def\d{-1}
% 			\pgfmathsetmacro\tcd{int(round(-\d/\c))} \pgfmathsetmacro\tcn{int(round(\a/\c))} 
% 			\pgfmathsetmacro\xmin{\tcd-2.5} \pgfmathsetmacro\xmax{\tcd+2.5}
% 			\pgfmathsetmacro\ymin{\tcn-2.5} \pgfmathsetmacro\ymax{\tcn+2.5}
% 			%\draw[color=gray!50,dashed] (\xmin,\ymin) grid (\xmax,\ymax);
% 			\draw[->] (\xmin,0)--(0,0)node [below right]{$O$}-- (\xmax,0) node [below]{$x$};
% 			\draw[->] (0,\ymin)--(0,\ymax) node [left]{$y$};
% 			\draw[dash pattern=on 2pt off 1.5pt] (\tcd,\ymin)--(\tcd,\ymax) (\tcd,0)node [below right]{$\tcd$}
% 			(\xmin,\tcn)--(\xmax,\tcn) (0,\tcn)node [below left]{$\tcn$};
% 			\begin{scope}
% 				\clip (\xmin,\ymin) rectangle (\xmax-.1,\ymax-.1);
% 				\draw[samples=200,smooth,variable=\x,thick, teal] plot[domain=\xmin:\tcd-0.3] (\x,{(\a*(\x)+\b)/(\c*(\x)+\d)});
% 				\draw[samples=200,smooth,variable=\x,thick, teal] plot[domain=\tcd+.3:\xmax] (\x,{(\a*(\x)+\b)/(\c*(\x)+\d)});
% 			\end{scope}
% 			\draw[ thick] (1.5,1)node[right]{$y=f(x)$};
% 			\foreach \x/\y in{0/0,1/0} \fill(\x,\y)circle(.03);
% 		\end{tikzpicture}	
% 	}
% 	% \dapso{Hàm số liên tục trên các khoảng $(-\infty;1)$, $(1; +\infty)$ và gián đoạn tại $x_0=1$.}
% 	\loigiai
% 	{
% 		Tập xác định $\mathscr{D}=\mathbb{R}\setminus \{1\}$.	
% 		\begin{itemize}
% 			\item Đồ thị hàm số là các đường liền nét trên các khoảng $(-\infty;1)$, $(1;+\infty)$ do đó hàm số liên tục trên các khoảng này.
% 			\item Ta có $\lim\limits_{x\to{1}^-}f(x)=-\infty$ và $\lim\limits_{x\to{1}^+}f(x)=+\infty$. Do đó $\lim\limits_{x\to{1}^-}f(x)\ne \lim\limits_{x\to{1}^+}f(x)$.\\
% 			Vậy hàm số đã cho gián đoạn tại $x_0=1$.
% 		\end{itemize}
% 	}
% \end{bt}

% \begin{bt}%[1K5KG-6]
% 	Một bảng giá cước taxi được cho như sau:
	
% 	\begin{tabular}{|c|c|c|}
% 		\hline 
% 		Giá mở cửa ($0,5$ km đầu)& Giá cước các km tiếp theo đến $30$ km& Giá cước từ km thứ $31$\\
% 		\hline
% 		$10\ 000$ đồng & $13\ 500$ đồng & $11\ 000$ đồng\\
% 		\hline
% 	\end{tabular}
% 	\begin{enumEX}{1}
% 		\item Viết công thức hàm số mô tả số tiền khách phải trả theo quãng đường di chuyển.
% 		\item Xét tính liên tục của hàm số ở câu a.
% 	\end{enumEX}
% 	% \dapso{Hàm số liên tục trên $(0;+\infty)$}
% 	\loigiai{
% 		a) Gọi $x$ là quãng đường di chuyển, $f(x)$ là giá tiền tính theo quãng đường. 
% 		\begin{itemize}
% 			\item $0\le x\le 0{,}5$, ta có $f(x)=10000$ đồng.
% 			\item $0{,}5<x \le 30$, $f(x)= 10000+13500(x-0{,}5)$ đồng.
% 			\item  $x>30$, $f(x)= 408250+11000(x-30)$ đồng.
% 		\end{itemize}
% 		Vậy $f(x)=\heva{&10000&\text{nếu }& 0\le x\le 0{,}5\\ &10000+13500(x-0{,}5)&\text{nếu }& 0{,}5<x\le 30\\
% 			&408250+11000(x-30)&\text{nếu }& x>30. } $ 
% 		\\
% 		b) Hàm số $f(x)$ liên tục trên các khoảng $(0; 0{,}5)$, $(0{,}5; 30)$ và $(30;+\infty)$. \\
% 		Tại $x=0{,}5$, ta có $f(0{,}5)=10000$, $\lim\limits_{x\to {0{,}5}^+}{f(x)}=10000$, $\lim\limits_{x\to {0{,}5}^-}{f(x)}=10000$.\\
% 		Vì $f(0,5)=\lim\limits_{x\to {0{,}5}^+}{f(x)}=\lim\limits_{x\to {0{,}5}^-}{f(x)}$, do đó $f(x)$ liên tục tại $x=0,5$.\\
% 		Tại $x=30$, ta có $f(30)=408250$, $\lim\limits_{x\to 30^-}{f(x)}=408250$, $\lim\limits_{x\to 30^+}{f(x)}=408250$.\\
% 		Vì $f(30)=\lim\limits_{x\to 30^-}{f(x)}=\lim\limits_{x\to 30^+}{f(x)}$, do đó$f(x)$ liên tục tại $x=30$. \\
% 		Vậy $f(x)$ liên tục trên khoảng $(0;+\infty)$.
% 	}
% \end{bt}

% \begin{bt}%[1T3B3-6]
% 	Một bãi đậu xe ô-tô đưa ra giá $C(x)$ (đồng) khi thời gian đậu xe là $x$ (giờ) như sau: $$C(x)=\heva{&60.000&\quad\text{khi }&0<x\le2\\&100.000&\quad\text{khi }&2<x\le4\\&200.000&\quad\text{khi }&4<x\le24.}$$
% 	Xét tính liên tục của hàm số $C(x)$.
% 	% \dapso{Hàm số $C(x)$ liên tục trên từng khoảng $(0;2)$, $(2;4)$, $(4;6)$}
% 	\loigiai{\begin{itemize}
% 			\item Hàm số $C(x)$ là hàm hằng trên từng khoảng $(0;2)$, $(2;4)$, $(4;6)$ nên liên tục trên từng khoảng đó.
% 			\item Ta có $\heva{&\lim\limits_{x\to2^-}C(x)=60.000\\&\lim\limits_{x\to2^+}C(x)=100.000}\Rightarrow$ không tồn tại $\lim\limits_{x\to2}C(x)$, vậy $C(x)$ không liên tục tại $x_0=2$.
% 			\item Ta có $\heva{&\lim\limits_{x\to4^-}C(x)=100.000\\&\lim\limits_{x\to4^+}C(x)=200.000}\Rightarrow$ không tồn tại $\lim\limits_{x\to4}C(x)$, vậy $C(x)$ không liên tục tại $x_0=4$.
% 		\end{itemize}
% 		Vậy hàm số $C(x)$ liên tục trên từng khoảng $(0;2)$, $(2;4)$, $(4;6)$.}
% \end{bt}
% \subsubsection{Câu hỏi trắc nghiệm}
% \Opensolutionfile{ans}[ans/ans-1K5-3-Dang5]
% \begin{ex}%[DCHT Toán 11 - KNTT -Đỗ Minh Phúc]%[1K5BG-6]
% 	\immini{Hình bên cạnh biểu thị độ cao $ h $ (m) của một quả bóng được đá lên thời gian $ t $ (s), trong đó $ h(t)= -2t^{2}+8t$. Kết luận nào sau đây là đúng?
% 		\choice
% 		{\True Hàm số $h(t)$ liên tục trên  $(0;4)$}
% 		{Hàm số $h(t)$ liên tục trên  $(0;8)$}
% 		{Hàm số $h(t)$ liên tục trên  $(-1;4)$}
% 		{$ \displaystyle\lim\limits_{t\to 2}\left(-2t^{2}+8t\right)=2$}
% 	}
% 	{		\begin{tikzpicture}[>=stealth,x=1cm,y=1cm,scale=0.5,font=\tiny]
% 			\def\a{-2}
% 			\def\b{8}
% 			\def\c{0}
% 			\draw[->] (-2,0) -- (6,0) node[below] {\scriptsize $t(s)$};
% 			\draw[->] (0,-2) -- (0,9) node[left] {\scriptsize $h(m)$};
% 			\draw (0,0)node[below left]{\scriptsize $O$};
% 			\fill (0,4) node[left]{$4$};
% 			\fill (0,8) node[left]{$8$};
% 			\fill (2,0) node[below]{$2$};
% 			\draw[dashed] (2,0)--(2,8)--(0,8);
% 			\pgfmathsetmacro\xdinh{-(\b)/2*(\a)}
% 			\pgfmathsetmacro\ydinh{(4*(\a)*(\c)-(\b)^2)/(4*(\a))}
% 			\clip (-2,-2)rectangle(6,9);
% 			\draw[red,thick,samples=150,smooth,domain=0:4] plot(\x,{\a*(\x)^2+(\b)*\x+(\c)});
% 			\foreach \x in {2,4} \draw (\x,0) circle (1pt);
% 			\foreach \y in {4,8} \draw (0,\y) circle (1pt);
% 	\end{tikzpicture}}
% 	\loigiai{
% 		Đồ thị hàm số cắt trục hoành tại $t=0$ và $t=4$.  \\
% 		Từ hình vẽ ta thấy tập xác định của $h(t)$ là $[0;4]$. Suy ra  hàm số $h(t)$ liên tục trên $(0;4)$.
% 	}
% \end{ex}

% \begin{ex}%[DCHT Toán 11 - KNTT -Đỗ Minh Phúc]%[1K5BG-6]
% 	Lực hấp dẫn do Trái Đất tác dụng lên một đơn vị khối lượng ở khoảng cách $r$ tính từ tâm của nó là $F(r)=\heva{&\dfrac{GMr}{R^3}&\quad\text{khi }&0<r\le R\\&\dfrac{GM}{r^2}&\quad\text{khi }&r\ge R}$, trong đó $M$ là khối lượng, $R$ là bán kính của Trái Đất, $G$ là hằng số hấp dẫn. Kết luận nào sau đây là đúng?
% 	\choice
% 	{Hàm số $F(r)$ liên tục trên $\mathbb{R}$}
% 	{Hàm số $F(r)$ liên tục tại điểm $r=0$}
% 	{Hàm số $F(r)$ liên tục trên $(-\infty;0)$}
% 	{\True Hàm số $F(r)$ liên tục trên $(0;+\infty)$}
% 	\loigiai{\begin{itemize}
% 			\item Với mọi $r\in(0;R)$, hàm số $F(r)=\dfrac{GMr}{R^3}$ luôn xác định nên liên tục tại đó.
% 			\item Với mọi $r\in(R;+\infty)$, hàm số $F(r)=\dfrac{GM}{r^2}$ luôn xác định nên liên tục tại đó.
% 			\item Ta có $\heva{&\lim\limits_{r\to R^-}F(r)=\lim\limits_{r\to R^-}\dfrac{GMr}{R^3}=\dfrac{GM}{R^2}\\&\lim\limits_{x\to R^+}F(r)=\lim\limits_{x\to R^+}\dfrac{GM}{r^2}=\dfrac{GM}{R^2}\\&F(R)=\dfrac{GM}{R^2}}$ nên hàm số $F(r)$ liên tục tại $r=R$.
% 		\end{itemize}
% 		Vậy hàm số $F(r)$ liên tục trên $(0;+\infty)$.}
% \end{ex}

% \begin{ex}%[DCHT Toán 11 - KNTT -Đỗ Minh Phúc]%[1K5BG-6]
% 	Trong một phòng thí nghiệm, nhiệt độ trong tủ sấy được điều khiển tăng từ $10^{\circ} \mathrm{C}$, mỗi phút tăng $2^{\circ} \mathrm{C}$ trong $60$ phút, sau đó giảm mỗi phút $3^{\circ} \mathrm{C}$ trong $40$ phút. Hàm số biểu thị nhiệt độ (tính theo $^{\circ} \mathrm{C}$ ) trong tủ theo thời gian $t$ (tính theo phút) có dạng
% 	\[T(t)= \heva{&10+2 t & \text { khi } 0 \leq t \leq 60 \\ &k-3 t & \text { khi } 60<t \leq 100}\ (k \text{ là hằng số}).\]
% 	Biết rằng, $T(t)$ là hàm liên tục trên tập xác định. Tìm giá trị của $k$.
% 	\choice
% 	{$k=0$}
% 	{$k=60$}
% 	{\True $k=310$}
% 	{$k=100$}
% 	\loigiai{
% 		Vì $T(t)$ là hàm liên tục trên tập xác định nên ta có hàm $T(t)$ liên tục tại $t=60$\\
% 		$\Leftrightarrow \lim\limits_{t\to 60^-} T(t)=\lim\limits_{t\to 60^+} T(t)=T(60)$\\
% 		$\Leftrightarrow 10+2\cdot 60=k-3\cdot 60\Leftrightarrow k=310$.\\
% 		Vậy $k=310$.
% 	}
% \end{ex}

% \begin{ex}%[DCHT Toán 11 - KNTT -Đỗ Minh Phúc]%[1K5BG-6]
% 	Một bảng giá cước taxi được cho như sau:
% 	\begin{center}
% 		\begin{tabular}{|c|c|c|}
% 			\hline 
% 			Giá mở cửa ($0,5$ km đầu)& Giá cước các km tiếp theo đến $30$ km& Giá cước từ km thứ $31$\\
% 			\hline
% 			$10\ 000$ đồng & $13\ 500$ đồng & $11\ 000$ đồng\\
% 			\hline
% 		\end{tabular}
% 	\end{center}
% 	Gọi $x$ là quãng đường di chuyển, $f(x)$ là giá tiền tính theo quãng đường có công thức như sau:
% 	\[f(x)=\heva{&10000&\text{nếu }& 0\le x\le 0{,}5\\ &10000+13500(x-0{,}5)&\text{nếu }& 0{,}5<x\le 30\\
% 		&408250+11000(x-30)&\text{nếu }& x>30. }\]
% 	Kết luận nào sau đây là đúng?
% 	\choice
% 	{Hàm số $f(x)$ liên tục trên  $\mathbb{R}$}
% 	{\True Hàm số $f(x)$ liên tục trên  $(0;+\infty)$}
% 	{Hàm số $f(x)$ liên tục trên  $(-\infty;0)$}
% 	{Hàm số $f(x)$ liên tục trên  $(0;30)$}
% 	\loigiai{
% 		Hàm số $f(x)$ liên tục trên các khoảng $(0; 0{,}5)$, $(0{,}5; 30)$ và $(30;+\infty)$. \\
% 		Tại $x=0{,}5$, ta có $f(0{,}5)=10000$, $\lim\limits_{x\to {0{,}5}^+}{f(x)}=10000$, $\lim\limits_{x\to {0{,}5}^-}{f(x)}=10000$.\\
% 		Vì $f(0,5)=\lim\limits_{x\to {0{,}5}^+}{f(x)}=\lim\limits_{x\to {0{,}5}^-}{f(x)}$, do đó $f(x)$ liên tục tại $x=0,5$.\\
% 		Tại $x=30$, ta có $f(30)=408250$, $\lim\limits_{x\to 30^-}{f(x)}=408250$, $\lim\limits_{x\to 30^+}{f(x)}=408250$.\\
% 		Vì $f(30)=\lim\limits_{x\to 30^-}{f(x)}=\lim\limits_{x\to 30^+}{f(x)}$, do đó$f(x)$ liên tục tại $x=30$. \\
% 		Vậy $f(x)$ liên tục trên khoảng $(0;+\infty)$.
% 	}
% \end{ex}
% \Closesolutionfile{ans}
% \begin{indapan}{10}
% 	{ans/ans-1K5-3-Dang5}
% \end{indapan}
\begin{dang}{Chứng minh phương trình có nghiệm}
	Để chứng minh minh một phương trình có nghiệm, ta thường tiến hành
	\begin{itemize}
		\item Đặt $f(x)$ là vế trái của phương trình (ứng với vế phải bằng $0$).
		\item Lập luận hàm số $f(x)$ liên tục trên $\mathbb{R}$ hoặc trên một đoạn con của $\mathbb{R}$ liên quan tới bài toán.
		\item Chỉ ra tồn tại các số $a$, $b$ ($a<b$) với $a$, $b$ thuộc đoạn con đang xét mà $f(a)\cdot f(b)<0$. Dựa vào tính chất của hàm số liên tục ta suy ra phương trình $f(x)=0$ có nghiệm thuộc khoảng $(a;b)$.		
		\begin{note}
			Nếu bài toán yêu cầu chứng minh phương trình có $k$ nghiệm thì cần lập luận $k$ đoạn con như trên.\\
			Nhiều trường hợp việc chỉ ra các số $a$, $b$ gặp khó khăn, ta có thể khai thác $\lim \limits _{x\to +\infty}f(x)$ hoặc $\lim \limits _{x\to -\infty}f(x)$ để có cơ sở lập luận. 
		\end{note}
	\end{itemize}
\end{dang}
\subsubsection{Ví dụ mẫu}
\begin{vd}%%[NB]%[DCHT Toán 11 - KNTT -Nguyễn Thành Nhân] %[1K5YG-7]
	Chứng minh rằng phương trình $x^5+4x^3-x^2-1=0$ có ít nhất một nghiệm thuộc khoảng $(0;1)$.
	\loigiai{
		Đặt $f(x) = x^5+4x^3-x^2-1$. Khi đó $f(x)$ liên tục trên $\mathbb{R}$ nên cũng liên tục trên đoạn $[0;1]$.\\
		Ta có $f(0) = -1$ và $f(1) = 3$ nên $f(0) \cdot f(1) < 0$.\\
		Do đó phương trình $f(x) =0$ có ít nhất một nghiệm thuộc $(0;1)$.
	}
\end{vd}
\begin{vd}%[TH]%[DCHT Toán 11 - KNTT -Nguyễn Thành Nhân] %[1K5BG-7]
	Chứng minh rằng phương trình $x^5-5x^3+4x-1=0$ có đúng $5$ nghiệm phân biệt.
	\loigiai{
		Đặt $f(x) = x^5-5x^3+4x-1$. Khi đó $f(x)$ liên tục trên $\mathbb{R}$.\\
		Ta có $f(-2)=-1<0$; $f\left(-\dfrac{3}{2}\right)=\dfrac{173}{32}>0$; $f(0)=-1<0$; $f\left(\dfrac{1}{2}\right)=\dfrac{18}{32}>0$; $f(1)=-1<0$; $f(3)=119>0$. \\
		Dựa vào tính chất liên tục của hàm số $f(x)$ trên $\mathbb{R}$, suy ra trên mỗi khoảng $\left(-2;-\dfrac{3}{2}\right)$; $\left(-\dfrac{3}{2};0\right)$; $\left(0;\dfrac{1}{2}\right)$; $\left(\dfrac{1}{2};1\right)$; $(1;3)$ có ít nhất một nghiệm. \\
		Do đó phương trình $f(x)=0$ có ít nhất $5$ nghiệm phân biệt. Vì $f(x)$ là phương trình bậc $5$ nên có tối đa $5$ nghiệm.\\
		Vậy phương trình $f(x)=0$ có đúng $5$ nghiệm phân biệt.
	}
\end{vd}

\begin{vd}%[TH]%[DCHT Toán 11 - KNTT -Nguyễn Thành Nhân] %[1K5BG-7]
	Chứng minh rằng phương trình $x^3-2mx^2-x+m=0$ luôn có nghiệm với mọi $m$ ($m$ là tham số).
	\loigiai{
		Xét hàm số $f(x)=x^3-2mx^2-x+m$. Khi đó $f(x)$ liên tục trên $\mathbb{R}$.\\
		Ta có $f(0)=m$ và $f(1)=-m$ nên $f(0)\cdot f(1)=-m^2\le 0$ với mọi $m$ nên phương trình $f(x)=0$ luôn có nghiệm thuộc đoạn $[0;1]$ với mọi $m$.
	}
\end{vd}

\begin{vd}%[TH]%[DCHT Toán 11 - KNTT -Nguyễn Thành Nhân] %[1K5BG-7]
	Chứng minh rằng phương trình $\left(1-m^2\right)(x+1)^3+x^2-x-3$ luôn có nghiệm với mọi giá trị của tham số $m$. 
	\loigiai{
		Đặt $f(x)= \left(1-m^2\right)(x+1)^3+x^2-x-3$ thì $f(x)$ liên tục trên $\mathbb{R}$. Ta có
		\[f(0)=-m^2-2<0,\,\forall m.\]
		và 
		\[f(-2)=m^2+2>0,\,\,\forall m.\]
		Vì $f(-2)\cdot f(0)<0$ nên phương trình $f(x)=0$ luôn có ít nhất một nghiệm thuộc khoảng $(-2;0)$ với mọi $m$.
	}
\end{vd}
\begin{vd}%[TH]%[DCHT Toán 11 - KNTT -Nguyễn Thành Nhân] %[1K5BG-7]
	Chứng minh rằng phương trình $m(x-8)^3(x-9)^4+2x-17=0$ luôn có nghiệm với mọi giá trị của $m$.
	\loigiai{
		Xét hàm số $f(x)=	m(x-8)^3(x-9)^4+2x-17$. Hàm số đã cho liên tục trên $\mathbb{R}$.\\
		Ta có $f(8)=-1, f(9)=1$. Vậy $f(8)\cdot f(9)<0$. Điều này suy ra phương trình có ít nhất một nghiệm trên $(8;9)$.
	}
\end{vd}
\begin{vd}%[VD]%[DCHT Toán 11 - KNTT -Nguyễn Thành Nhân]%[1K5KG-7]
	Chứng minh rằng phương trình $m(x+1)^2(x-2)^3+(x+2)(x-3)=0$ luôn có nghiệm với mọi tham số $m$.
	\loigiai{
		Xét hàm số $f(x)=m(x+1)^2(x-2)^3+(x+2)(x-3)$ xác định và liên tục trên $[-2;3]$.\\
		Ta có $f(-2)=-64m$, $f(3)	
		=16m$, $f(-2) \cdot f(3)=-1024m^2 \le 0$.
		\begin{itemize}
			\item Với $m=0$ suy ra $f(-2)=f(3)=0$ suy ra phương trình $f(x)=0$ có hai nghiệm $x=-2$ và $x=3$.
			\item Với $m \ne 0$ suy ra $f(-2) \cdot f(3) <0$, suy ra tồn tại $x_0 \in (-2;3)$ sao cho $f(x_0)=0$.
		\end{itemize}
		Do đó phương trình $f(x)=0$ luôn có nghiệm.\\
		Vậy phương trình ban đầu luôn có nghiệm.
	} 
\end{vd}
\begin{vd}%[VDC]%[DCHT Toán 11 - KNTT -Nguyễn Thành Nhân]%[1K5GG-7]
	Với mọi giá trị thực của tham số $m$, chứng minh phương trình $\left(m^2+1\right)x^3-2m^2x^2-4x+m^2+1=0$ luôn có ba nghiệm thực.
	\loigiai
	{
		Đặt $f(x)=\left(m^2+1\right)x^3-2m^2x^2-4x+m^2+1$.\\
		Hàm số $f(x)=\left(m^2+1\right)x^3-2m^2x^2-4x+m^2+1$ là một hàm số đa thức nên nó liên tục trên $\mathbb{R}$. Suy ra, nó cũng liên tục trên mỗi đoạn $[-3;0]$, $[0;1]$, $[1;2]$.
		Ta có
		\begin{itemize}
			\item $f(-3) =-27m^2-27-18m^2+12+m^2+1=-44m^2-14<0$, với mọi $m\in\mathbb{R}$.
			\item $f(0)=m^2+1>0$, với mọi $m\in\mathbb{R}$.
			\item $f(1)=m^2+1-2m^2-4+m^2+1=-2<0$.
			\item $f(2)=8m^2+8-8m^2-8+m^2+1=m^2+1>0$, với mọi $m\in\mathbb{R}$.
		\end{itemize}
		Vì $f(-3)\cdot f(0)<0$ nên phương trình đã cho có ít nhất một nghiệm thuộc khoảng $(-3;0)$.\\
		Vì $f(0)\cdot f(1)<0$ nên phương trình đã cho có ít nhất một nghiệm thuộc khoảng $(0;1)$.\\
		Vì $f(1)\cdot f(2)<0$ nên phương trình đã cho có ít nhất một nghiệm thuộc khoảng $(1;2)$.\\
		Phương trình $\left(m^2+1\right)x^3-2m^2x^2-4x+m^2+1=0$ là một phương trình bậc ba $(\text{vì } m^2+1\neq 0,\forall m\in\mathbb{R})$.\\
		Vậy phương trình $\left(m^2+1\right)x^3-2m^2x^2-4x+m^2+1=0$ luôn có ba nghiệm thực.
	}
\end{vd}
\begin{vd}%[VD]%[DCHT Toán 11 - KNTT -Nguyễn Thành Nhân]%[1K5KG-7]
	Chứng minh rằng phương trình sau luôn có nghiệm với mọi giá trị của tham số $m\ge -1$
	\[ (m-1)x^6+\left(m^2-\sqrt{4m+4}\right)x^3+6x-3=0. \]
	\loigiai{
		Đặt $f(x)=(m-1)x^6+\left(m^2-\sqrt{4m+4}\right)x^3+6x-3$. Khi đó $f(x)$ liên tục trên đoạn $[0;1]$. Ta có 
		\begin{align*}
			f(1)&=(m-1)+\left(m^2-\sqrt{4m+4}\right)+6-3\\
			&=m^2+(m+1)-2\sqrt{m+1}+1\\
			&=m^2+\left(\sqrt{m+1}-1\right)^2.
		\end{align*}
		\begin{itemize}
			\item Nếu $m=\sqrt{m+1}-1=0$ hay $m=0$ thì $f(1)=0$.
			\item Nếu $m\ne 0$ thì $f(1)>0$, mà $f(0)=-3<0$ nên $f(x)$ có một nghiệm trong khoảng $(0;1)$.
		\end{itemize}
		Vậy phương trình $f(x)=0$ luôn có nghiệm với mọi $m\ge -1$. 
	}
\end{vd}
\begin{vd}%[VDC]%[DCHT Toán 11 - KNTT -Nguyễn Thành Nhân]%[1K5GG-7]
	Với mọi giá trị thực của tham số $m,$ chứng minh phương trình $x^5+x^2-\left(m^2+2\right)x-1=0$ luôn có ít nhất ba nghiệm thực.
	\loigiai{
		Xét hàm số $ f(x)=x^5+x^2-\left(m^2+2\right)x-1$ liên tục trên $\mathbb R$.\\
		Ta có $ f(0)=-1 < 0$, $f(-1)=m^2+1 > 0$.\\
		Mặt khác, vì $\lim\limits_{x\to-\infty}f(x)=-\infty$ nên tồn tại $a <-1$ sao cho $f(a) < 0$.\\
		Vì $\lim\limits_{x\to+\infty}f(x)=+\infty$ nên tồn tại $b > 0$ sao cho $f(b) > 0$.\\
		Khi đó
		\begin{itemize}
			\item $ f(a)\cdot f\left(-1\right) < 0$ suy ra phương trình $ f(x)=0$ có ít nhất $ 1$ nghiệm thuộc $\left(a;-1\right)$,
			\item $ f\left(-1\right)\cdot f(0) < 0$ suy ra phương trình $ f(x)=0$ có ít nhất $ 1$ nghiệm thuộc $\left(-1;0\right)$,
			\item $ f(0)\cdot f(b) < 0$ suy ra phương trình $ f(x)=0$ có ít nhất $ 1$ nghiệm thuộc $\left(0;b\right)$.
		\end{itemize}
		Vậy phương trình đã cho có ít nhất $ 3$ nghiệm.
	}
\end{vd}
\begin{vd}%[VDC]%[DCHT Toán 11 - KNTT -Nguyễn Thành Nhân]%[1K5GG-7]
	Cho $a$, $b$ là hai số thực thỏa mãn $9 a+24 b>128$. Chứng minh phương trình $a x^2+b x-2=0$ có ít nhất một nghiệm thuộc khoảng $(0 ; 1)$.	
	\loigiai{
		Xét hàm số $ f(x)=x^2+b x-2$ liên tục trên $\mathbb R$.\\		
		Ta có $f(0)=-2<0$.\\
		Ta có $f\left(\dfrac{1}{2}\right)=\dfrac{a}{4}+\dfrac{b}{2}-2$, $f\left(1\right)=a+b-2$.\\
		Khi đó $2f(1)+4f\left(\dfrac{1}{2}\right)=3a+4b-12>\dfrac{128}{3}-12=\dfrac{92}{3}>0$. Suy ra một trong hai số $f(1)$  hoặc $f\left(\dfrac{1}{2}\right)$ là số dương.\\
		Do đó $\hoac{&f(0)\cdot f\left(\dfrac{1}{2}\right)<0\\&f(0)\cdot f(1)<0.}$\\
		Khi đó phương trình $f(x)=0$ có ít nhất một nghiệm thuộc khoảng $(0;1)$.
	}
\end{vd}
\begin{vd}%[VDC]%[DCHT Toán 11 - KNTT -Nguyễn Thành Nhân]%[1K5GG-7]
	Cho phương trình $ax^2+bx+c=0$ với $5a+3b+3c=0$. Chứng minh rằng phương trình luôn có nghiệm.	
	\loigiai{Do $5a+3b+3c=0$ nên $b=-\dfrac{5}{3}a-c$.\\
		Xét hàm số  $f(x)=ax^2+bx+c$ trên $\left[ 0;\dfrac{5}{3}\right] $.\\
		Ta có $f(0)=c$, $f\left( \dfrac{5}{3} \right)=\dfrac{25}{9}\cdot a+\dfrac{5}{3}\cdot b+c =\dfrac{25}{9}\cdot a+\dfrac{5}{3}\left(-\dfrac{5}{3}a-c \right)+c=-\dfrac{2}{3}c  $.
		\begin{itemize}
			\item Nếu $c=0$ thì $f(0)=f\left(\dfrac{5}{3} \right)=0 $, phương trình đã cho có hai nghiệm là $x=0$, $x=\dfrac{5}{3}$.
			\item Nếu $c\ne 0$ thì $f(0)\cdot f\left(\dfrac{5}{3} \right)=-\dfrac{2}{3}c^2<0 $. Vì $f(x)$ là hàm đa thức nên liên tục trên $\mathbb{R}$.\\
			Do đó, nó liên tục trên $\left[ 0;\dfrac{5}{3}\right] $.\\
			Từ đó suy ra phương trình $f(x)=0$  có ít nhất một nghiệm trên $\left( 0;\dfrac{5}{3}\right)$.
		\end{itemize}
		Vậy phương trình đã cho luôn có nghiệm.
	}
\end{vd}
\subsubsection{Bài tập rèn luyện}
% \centerline{\fcolorbox{red}{yellow!50}{\bf {BÀI TẬP TỰ LUẬN }}}
\begin{bt}%%%[NB]%[DCHT Toán 11 - KNTT -Tên GV] %[1K5YG-7]
	Chứng minh rằng phương trình $x^5+4x^3-x^2-1=0$ có ít nhất một nghiệm thuộc khoảng $(0;1)$.
	\loigiai{
		Đặt $f(x) = x^5+4x^3-x^2-1$ liên tục trên $[0;1]$.\\
		Ta có $f(0) = -1$ và $f(1) = 3$ nên $f(0) \cdot f(1) < 0$.\\
		Do đó phương trình $f(x) =0$ có ít nhất một nghiệm thuộc $(0;1)$.
	}
\end{bt}
\begin{bt}%%%[NB]%[DCHT Toán 11 - KNTT -Tên GV] %[1K5BG-7]
	Chứng minh rằng phương trình $ 2x^4-3x^3-5=0 $ có ít nhất một nghiệm.
	\loigiai{
		Đặt $f(x)= 2x^4-3x^3-5 $, $ f(x) $ là hàm đa thức nên liên tục trên $ \mathbb{R} $.\\
		Do đó $ f(x) $ liên tục trên đoạn $ [1;2] $.\\
		Ta có $ \heva{&f(1)=-6\\&f(2)=3} $ suy ra $ f(1)\cdot f(2)=-18<0 $.\\
		Nên phương trình $ f(x)=0 $ có ít nhất một nghiệm nằm trong khoảng $ (1;2) $.\\
		Vậy phương trình đã cho có ít nhất một nghiệm.
	}
\end{bt}
\begin{bt}%[TH]%[DCHT Toán 11 - KNTT -Nguyễn Thành Nhân]%[1K5BG-7]
	Chứng minh rằng phương trình $-3x^5+8x^2-1=0$ có ít nhất một nghiệm.
	\loigiai{
		Xét hàm số $f(x)=-3x^5+8x^2-1$ liên tục trên $\mathbb{R}$ nên liên tục trên đoạn $[0;1]$. Ta có $f(0)\cdot f(1)=-1\cdot 4=-4<0$.\\
		Vậy có ít nhất một số $x_0=c\in \left(0;1\right)$ để $f(x_0)=0$, hay nói cách khác phương trình $f(x)=0\Leftrightarrow -3x^5+8x^2-1=0$ có nghiệm.
	}
\end{bt}
\begin{bt}%[TH]%[DCHT Toán 11 - KNTT -Nguyễn Thành Nhân] %[1K5BG-7]
	Chứng minh phương trình $\left(m^2-2m+3\right)x^4-2x-4=0$ luôn có nghiệm âm với mọi giá trị thực của tham số $m$.
	\loigiai
	{Hàm số $f(x)=\left(m^2-2m+3\right)x^4-2x-4$ liên tục trên $\mathbb{R}$.\\
		Ta có $f(0)=-4$.\\
		Vì $m^2-2m+3=(m-1)^2+2>0,\ \forall m$ nên $$\lim\limits_{x\to -\infty} f(x)=\lim\limits_{x\to -\infty} x^4\left(m^2-2m+3-\dfrac{2}{x^3}-\dfrac{4}{x^4}\right)=+\infty.$$
		Do đó, tồn tại $a<0$ sao cho $f(a)>0$.\\
		Vì hàm số $f(x)$ liên tục trên $\mathbb{R}$ nên nó cũng liên tục trên $[a; 0]$. Hơn nữa $f(a)\cdot f(0)<0$ nên phương trình $f(x)=0$ luôn có ít nhất một nghiệm thuộc khoảng $(a; 0)$.\\
		Vậy phương trình đã cho luôn có nghiệm âm với mọi giá trị thực của tham số $m$.
	}
\end{bt}

\begin{bt}%[TH]%[DCHT Toán 11 - KNTT -Tên GV] %[1K5BG-7]
	Chứng minh phương trình $x^4 + x^3 + mx^2 + x\left(2m - 1\right) + m\sin \left(\pi x\right)=1$ có nghiệm với mọi $m$. 
	\loigiai{
		$x^4 + x^3 + mx^2 + x\left(2m - 1\right) + m\sin \left(\pi x\right)=1\Leftrightarrow x^4 + x^3 - x - 1 + m\left(x^2 + \sin \pi x + 2x\right)=0$.\\
		Đặt $f(x)=x^4 + x^3 - x - 1 + m\left(x^2 + \sin \pi x + 2x\right)$.\\
		Ta có $f(x)$ liên tục trên $\mathbb{R}$.\\
		$f\left(0\right)= - 1$ và $f\left(- 2\right)=9\Rightarrow f(0)\cdot f(-2)<0$.\\
		Suy ra phương trình đã cho luôn có ít nhất một nghiệm thuộc khoảng $ (-2;0) $.
	}
\end{bt}

\begin{bt}%[TH]%[DCHT Toán 11 - KNTT -Tên GV] %[1K5BG-7]
	Chứng minh phương trình $x^4+mx^2+(3m-1)x-5+2m=0$ luôn có ít nhất một nghiệm với mọi số thực $m$.
	\loigiai{
		Đặt $f(x)=x^4+mx^2+(3m-1)x-5+2m$.\\
		Vì $f(x)$ là hàm đa thức nên $f(x)$ liên tục trên $\mathbb{R}$.\\
		Lại có $f(-2)=13, f(-1)=-3$	$\Rightarrow f(-3) \cdot f(-1)<0$, do đó phương trình $f(x)=0$ có ít nhất một nghiệm trên $(-3;-1)$.\\
		Do đó phương trình luôn có nghiệm với mọi $m$.
	}
\end{bt}

\begin{bt}%[TH]%[DCHT Toán 11 - KNTT -Tên GV] %[1K5BG-7]
	Chứng minh rằng phương trình $m(x-8)^3(x-9)^4+2x-17=0$ luôn có nghiệm với mọi giá trị của $m$.
	\loigiai{
		Xét hàm số $f(x)=	m(x-8)^3(x-9)^4+2x-17$. Hàm số đã cho liên tục trên $\mathbb{R}$.\\
		Ta có $f(8)=-1, f(9)=1$. Vậy $f(8)\cdot f(9)<0$. Điều này suy ra phương trình có ít nhất một nghiệm trên $(8;9)$.
	}
\end{bt}
\begin{bt}%[TH]%[DCHT Toán 11 - KNTT -Tên GV] %[1K5BG-7]
	Chứng minh phương trình $\left(m^2-2m+3\right)x^4-2x-4=0$ luôn có nghiệm âm với mọi giá trị thực của tham số $m$.
	\loigiai
	{Hàm số $f(x)=\left(m^2-2m+3\right)x^4-2x-4$ có tập xác định là $\mathscr{D}=\mathbb{R}$.\\
		Ta có $f(0)=-4$.\\
		Vì $m^2-2m+3=(m-1)^2+2>0,\ \forall m$ nên $$\lim\limits_{x\to -\infty} f(x)=\lim\limits_{x\to -\infty} x^4\left(m^2-2m+3-\dfrac{2}{x^3}-\dfrac{4}{x^4}\right)=+\infty.$$
		Do đó, tồn tại $a<0$ sao cho $f(a)>0$.\\
		Vì hàm số $f(x)$ liên tục trên $\mathbb{R}$ nên nó cũng liên tục trên $[a; 0]$. Hơn nữa $f(a)\cdot f(0)<0$ nên phương trình $f(x)=0$ luôn có ít nhất một nghiệm thuộc khoảng $(a; 0)$.\\
		Vậy phương trình đã cho luôn có nghiệm âm với mọi giá trị thực của tham số $m$.
	}
\end{bt}
\begin{bt}%[TH]%[DCHT Toán 11 - KNTT -Tên GV] %[1K5BG-7]
	Chứng minh rằng phương trình $m\cdot\sin 2x+x^2\cdot\cos x+\left(m^2+1\right)\cdot\cos 2x=0$ luôn có nghiệm thuộc khoảng $\left(0 ;\dfrac{\pi}{2}\right)$ với mọi tham số $m$.
	\loigiai{
		Đặt $f(x)=m\cdot\sin 2x+x^2\cdot\cos x+\left(m^2+1\right)\cdot\cos 2x$.
		\begin{itemize}
			\item $f(x)$ liên tục trên $\mathbb{R}$.
			\item Ta có $f(0)=m^2+1>0\ \forall m$ và $ f\left(\dfrac{\pi}{2}\right)=-m^2-1<0\ \forall m $.
		\end{itemize}
		Suy ra phương trình $f(x)=0$ có ít nhất một nghiệm thuộc khoảng $\left(0 ;\dfrac{\pi}{2}\right)$.
	}
\end{bt}
\begin{bt}%[TH]%[DCHT Toán 11 - KNTT -Tên GV] %[1K5BG-7]
	Chứng minh phương trình $x^3+m x^2-4m x=19-3m$ có nghiệm với mọi $m$.
	\loigiai{
		Ta có $x^3+m x^2-4m x=19-3m \Leftrightarrow x^3+m x^2-4m x-19+3m=0$.\\
		Đặt $f\left(x\right)=x^3+m x^2-4m x-19+3m$.\\
		$f\left(x\right)$ liên tục trên $\mathbb{R}$.\\
		$f(1)=-18$, 
		$f(3)=8$,\\
		$\Rightarrow f(1)f(3)<0$\\
		$\Rightarrow$ phương trình có nghiệm với mọi $m$.
	}
\end{bt}
% \begin{bt}%[DCHT Toán 11 - KNTT -Nguyễn Thành Nhân]%[1K5BG-7]
% 	Tìm các giá trị nguyên của tham số $m$ để phương trình sau vô nghiệm 
% 	\[
% 	(m^2-1)(x-1)^{2020} =2019\sqrt{4-x}. 
% 	\]
% 	\loigiai{
% 		Điều kiện xác định $x \leq 4$. 
% 		\begin{itemize}
% 			\item Dễ thấy với $m^2 -1 <0 \Leftrightarrow -1 <m <1$, phương trình vô nghiệm. 
% 			\item Với $m^2 -1 =0 \Leftrightarrow m = \pm 1$, phương trình có nghiệm $x=4$. 
% 			\item Với $m^2 -1 >0$, do $x=1$ không là nghiệm nên phương trình tương đương với 
% 			\[
% 			m^2 -1 = \dfrac{2019\sqrt{4-x}}{(x-1)^{2020}}. \tag{1}
% 			\]
% 			Xét hàm số $f(x) = \dfrac{2019\sqrt{4-x}}{(x-1)^{2020}}$. \\
% 			Ta có $\displaystyle \lim_{x \to 1^+} f(x) = +\infty$, do đó tồn tại $ 1<\alpha<4$ sao cho $f(\alpha)  > m^2-1$. \\
% 			Dễ thấy hàm số $f(x)$ liên tục trên đoạn $[\alpha ; 4]$ và có $0 = f(0) < m^2-1 < f(\alpha)$, do đó theo định lý giá trị trung gian tồn tại $x_0 \in (\alpha ;4)$ sao cho $f(x_0) = m^2-1$. Điều đó có nghĩa là phương trình (1) có nghiệm $x=x_0$.\\
% 			Vậy $m=0$ là số nguyên duy nhất để phương trình đã cho vô nghiệm. 
% 		\end{itemize}
% 	}
% \end{bt}
% \begin{bt}%[DCHT Toán 11 - KNTT -Nguyễn Thành Nhân]%%[1K5BG-7]
% 	Chứng minh phương trình $(1-m^2)(x+1)^3+x^2-x-3=0$ có nghiệm với mọi $m$.
% 	\loigiai{
% 		Đặt $f(x)=(1-m^2)(x+1)^3+x^2-x-3$, $f(x)$ là hàm đa thức nên xác định và liên tục trên $\mathbb{R}$. Suy ra $f(x)$ liên tục trên đoạn $[-2;-1]$.\\
% 		Ta có $\heva{&f(-1)=-1<0\\&f(-2)=(1-m^2)(-1)^3+(-2)^2-(-2)-3=m^2+2>0,\,\forall m\in\mathbb{R}.}$\\
% 		Vì $f(-2)\cdot f(-1)<0$, $\forall m\in\mathbb{R}$ nên phương trình $f(x)=0$ có ít nhất một nghiệm trong khoảng $(-2;-1)$, $\forall m\in\mathbb{R}$.\\
% 		Vậy phương trình $(1-m^2)(x+1)^3+x^2-x-3=0$ có nghiệm với mọi $m$.
% 	}
% \end{bt}

% \begin{bt}%[DCHT Toán 11 - KNTT -Nguyễn Thành Nhân]%[1K5BG-7]
% 	Chứng minh rằng phương trình $x^5+4x^3-x^2-1=0$ có ít nhất một nghiệm thuộc khoảng $(0;1)$.
% 	\loigiai{
% 		Đặt $f(x) = x^5+4x^3-x^2-1$ liên tục trên $[0;1]$.\\
% 		Ta có $f(0) = -1$ và $f(1) = 3$ nên $f(0) \cdot f(1) < 0$.\\
% 		Do đó phương trình $f(x) =0$ có ít nhất một nghiệm thuộc $(0;1)$.
% 	}
% \end{bt}

% \begin{bt}%[DCHT Toán 11 - KNTT -Nguyễn Thành Nhân]%[1K5BG-7]
% 	Chứng minh rằng phương trình $ 2x^4-3x^3-5=0 $ có ít nhất một nghiệm.
% 	\loigiai{
% 		Đặt $f(x)= 2x^4-3x^3-5 $, $ f(x) $ là hàm đa thức nên liên tục trên $ \mathbb{R} $.\\
% 		Do đó $ f(x) $ liên tục trên đoạn $ [1;2] $.\\
% 		Ta có $ \heva{&f(1)=-6\\&f(2)=3} $ suy ra $ f(1)\cdot f(2)=-18<0 $.\\
% 		Nên phương trình $ f(x)=0 $ có ít nhất một nghiệm nằm trong khoảng $ (1;2) $.\\
% 		Vậy phương trình đã cho có ít nhất một nghiệm.
% 	}
% \end{bt}
% \begin{bt}%[DCHT Toán 11 - KNTT -Nguyễn Thành Nhân]%[1K5BG-7]
% 	Chứng minh rằng phương trình $2x^3 - 5x + 1 = 0$ có đúng ba nghiệm.
% 	\loigiai{
% 		Đặt $f(x) = 2x^3 - 5x + 1$. Tập xác định của hàm số là $\mathscr D = \mathbb{R}$.\\
% 		Ta có $f(-2) = -5$, $f(0) = 1$, $f(1) = -2$, $f(2) = 7$.\\
% 		Vì $f(x)$ liên tục trên $\mathbb{R}$ nên $f(x)$ liên tục trên $[-2; 0]$ và $f(-2)\cdot f(0) = -5 \cdot 1 = -5 < 0$.\\
% 		Do đó phương trình $f(x) = 0$ có ít nhất một nghiệm $x_1 \in (-2; 0)$. \hfill (1)\\
% 		Vì $f(x)$ liên tục trên $\mathbb{R}$ nên $f(x)$ liên tục trên $[0; 1]$ và $f(0)\cdot f(1) = 1 \cdot (-2) = -2 < 0$.\\
% 		Do đó phương trình $f(x) = 0$ có ít nhất một nghiệm $x_2 \in (0; 1)$. \hfill (2)\\
% 		Vì $f(x)$ liên tục trên $\mathbb{R}$ nên $f(x)$ liên tục trên $[1; 2]$ và $f(1)\cdot f(2) = -2 \cdot 7 = -14 < 0$.\\
% 		Do đó phương trình $f(x) = 0$ có ít nhất một nghiệm $x_3 \in (1; 2)$. \hfill (3)\\
% 		Do các khoảng $(-1; 0)$; $(0; 1)$; $(1; 2)$ không giao nhau
% 		Từ (1), (2) và (3), suy ra phương trình $2x^3 - 5x + 1 = 0$ có ít nhất ba nghiệm.\\
% 		Mà phương trình bậc ba có không quá $3$ nghiệm nên phương trình $2x^3 - 5x + 1 = 0$ có đúng ba nghiệm phân biệt. 
% 	}
% \end{bt}
% \begin{bt}%[VD]%[DCHT Toán 11 - KNTT -Tên GV] %[1K5KG-7]
% 	Chứng minh phương trình $2x^3-3x^2-1=0$ có nghiệm $x_0\in \left(\sqrt[3]{4};2\right)$.
% 	\loigiai{
% 		Đặt $f(x)=2x^3-3x^2-1$ thì $f(x)$ liên tục trên $\mathbb{R}$.\\
% 		Ta có $f(1)=-2<0$; $f(2)=3>0$. Suy ra phương trình $f(x)=0$ có nghiệm $x_0\in (1;2)$.\\
% 		Lại áp dụng bất đẳng thức Cauchy, ta có
% 		\[2x_0^3=3x_0^2+1=x_0^2+x_0^2+x_0^2+1\geq 4\sqrt[4]{x_0^6}=4\sqrt{x_0^3}.\]
% 		Suy ra $x_0\geq \sqrt[3]{4}$. Nhưng tại $x_0=\sqrt[3]{4}$ thì dấu đẳng thức không xảy ra, suy ra $x_0> \sqrt[3]{4}$.\\
% 		Vậy, phương trình đã cho có nghiệm $x_0\in \left(\sqrt[3]{4};2\right)$.	
% 	}
% \end{bt}
% \begin{bt}%[VD]%[DCHT Toán 11 - KNTT -Nguyễn Thành Nhân]%[1K5KG-7]
% 	Chứng minh phương trình $x^7-3x^6+x^4+x^3-(m^2+3)x+2=0$ có ít nhất một nghiệm dương với mọi tham số $m \in \mathbb{R}$.
% 	\loigiai{ Đặt $f(x)=x^7-3x^6+x^4+x^3-(m^2+3)x+2$, khi đó hàm số $f(x)$ liên tục trên $\mathbb{R}$.\\
% 		Ta có $f(0)=2$, $f(1)=-m^2-1$. 
% 		Suy ra $f(0)\cdot f(1)=2\left(-m^2-1\right)<0$ với mọi $m \in \mathbb{R}$.\\
% 		Vậy phương trình đã cho có ít nhất một nghiệm dương thuộc khoảng $(0;1)$ với mọi $m \in \mathbb{R}$.
% 	}
% \end{bt}
% \begin{bt}%[VD]%[DCHT Toán 11 - KNTT -Nguyễn Thành Nhân]%[1K5KG-7]
% 	Cho phương trình $x^4-x^3-(m^2+5)x^2+2(m^2+2)x+4=0$. Chứng minh rằng với mọi số nguyên $m$ thì phương trình sau luôn có đúng $4$ nghiệm phân biệt.
% 	\loigiai{
% 		Ta có 
% 		\allowdisplaybreaks
% 		\begin{eqnarray*}
% 			&&x^4-x^3-(m^2+5)x^2+2(m^2+2)x+4=0\\
% 			&\Leftrightarrow&(x-2)\left(x^3+x^2-(m^2+3)x-2\right)=0\\
% 			&\Leftrightarrow&\hoac{&x-2=0\\&x^3+x^2-(m^2+3)x-2=0.}	
% 		\end{eqnarray*}	
% 		Phương trình ban đầu có đúng $4$ nghiệm khi và chỉ khi phương trình $x^3+x^2-(m^2+3)x-2=0$ có đúng $3$ nghiệm phân biệt khác $2$.\\
% 		Xét $f(x)=x^3+x^2-(m^2+3)x-2$, ta có
% 		\begin{itemize}
% 			\item $\lim\limits_{x\to -\infty}f(x)=-\infty$ nên $\exists a<-1\colon f(a)<0$.
% 			\item $f(-1)=m^2+1>0$.
% 			\item $f(0)=-2<0$.
% 			\item $f\left(m^2+3\right)=\left(m^2+3\right)^3-2>0$.
% 		\end{itemize}
% 		Hàm số $f(x)$ là hàm số đa thức nên liên tục trên tập xác định $\mathbb{R}$, do đó $f(x)$ cũng liên tục trên các đoạn $[a;-1]$, $[-1;0]$ và $\left[0;m^2+3\right]$.\\
% 		Do $f(a) \cdot f(-1)<0$, $f(-1) \cdot f(0)<0$, $f(0) \cdot f\left(m^{2}+3\right)<0$ và $f(x)=0$ là phương trình bậc 3 nên có đúng 3 nghiệm thuộc các khoảng $\left(a ;-1\right)$, $(-1 ; 0)$, $\left(0 ; m^{2}+3\right)$.\\
% 		Mặt khác $f(2)\ne 0\Leftrightarrow m\ne \pm \sqrt{2}\notin \mathbb{Z}$.\\
% 		Vậy phương trình $x^4-x^3-(m^2+5)x^2+2(m^2+2)x+4=0$ luôn có đúng $4$ nghiệm phân biệt với mọi số nguyên $m$.
% 	}
% \end{bt}

% \begin{bt}%[VD]%[DCHT Toán 11 - KNTT -Nguyễn Thành Nhân]%[1K5KG-7]
% 	Với $m>2$, chứng minh rằng phương trình $x^3-2mx^2+2=0$ có ba nghiệm phân biệt. 
% 	\loigiai{
% 		Đặt $f(x)=x^3-2mx^2+2$ thì $f(x)$ là hàm đa thức nên liên tục trên $\mathbb{R}$.\\
% 		Ta có \\
% 		$f(-1)=1-2m<0$ do $m>2$, $f(0)=2>0$, $f(1)=3-2m<0$ do $m>2$. \\
% 		Vì $f(-1)\cdot f(0)<0$ nên phương trình có ít nhất một nghiệm thuộc khoảng $(-1;0)$.\\
% 		Vì $f(0)\cdot f(1)<0$ nên phương trình có ít nhất một nghiệm thuộc khoảng $(0;1)$.\\
% 		Vì $\lim\limits_{x\to +\infty}f(x)=\lim\limits_{x\to +\infty}x^3\left(1-\dfrac{2m}{x}+\dfrac{2}{x^3}\right)=+\infty$ nên tồn tại một số $a>1$ để $f(a)>0$.\\
% 		Vì $f(1)\cdot f(a)<0$ nên phương trình có ít nhất một nghiệm thuộc khoảng $(1;a)$.\\
% 		Từ đó suy ra phương trình có ít nhất ba nghiệm phân biệt. Mặt khác $f(x)=0$ là phương trình bậc $3$ nên có không quá ba nghiệm.\\
% 		Vậy phương trình $f(x)=0$ có đúng ba nghiệm phân biệt. 
% 	}
% \end{bt}
% \begin{bt}%[VD]%[DCHT Toán 11 - KNTT -Tên GV] %[1K5KG-7]
% 	Chứng minh phương trình $\left(1-m\right)x^5+9mx^2-16x-m=0$ có ít nhất hai nghiệm thực phân biệt với mọi giá trị thực của tham số $m$.
% 	\loigiai
% 	{Đặt $f(x)=\left(1-m\right)x^5+9mx^2-16x-m$ thì $f(x)$ liên tục trên $\mathbb{R}$.\\ 
% 		Ta biến đổi $f(x)=\left(-x^5+9x^2-1\right)m+x^5-16x$. \\
% 		Ta tính giá trị của $f(x)$ tại các giá trị của $x$ thỏa $x^5-16x=0$, tức là tính tại $0$ và $\pm 2$.\\
% 		Ta có $f(-2)=3m$; $f(0)=-m$; $f(2)=3m$.\\
% 		Nếu $m=0$ thì phương trình tương đương $$x^5-16x=0\Leftrightarrow \hoac{&x=0\\&x=\pm2},$$ nên phương trình có nghiệm khi $m=0$.\\
% 		Với $m\ne 0$ thì $f(-2)\cdot f(0)=-3m^2<0$; $f(0)\cdot f(2)=-3m^2<0$. \\
% 		Theo tính chất của hàm số liên tục, trên mỗi khoảng $(-2;0)$ và $(0;2)$, phương trình có ít nhất một nghiệm.\\
% 		Vậy phương trình đã cho có ít nhất hai nghiệm phân biệt.
% 	}
% \end{bt}
\begin{bt}%[VD]%[DCHT Toán 11 - KNTT -Tên GV] %[1K5KG-7]
	Cho $f(x)$ là hàm số liên tục trên đoạn $\left[a;b\right]$ sao cho với mọi $x\in \left[a;b\right]$ thì $a\le f(x)\le b$. Chứng minh rằng phương trình $f(x)=x$ có ít nhất một nghiệm thuộc đoạn $\left[a;b\right]$.
	\loigiai
	{Xét hàm số $h(x)=f(x)-x$, khi đó $h(x)$ liên tục trên đoạn $\left[a;b\right]$.\\
		Vì $a\le f(x)\le b$ với mọi $x\in \left[a;b\right]$ nên ta có $$h(a)=f(a)-a\geq 0;\,\,h(b)=f(b)-b\le 0.$$
		Suy ra $h(a)\cdot h(b)\le 0$. Do đó xảy ra một trong hai khả năng
		\begin{itemize}
			\item[•] Nếu $h(a)\cdot h(b)=0$ thì $\hoac{&h(a)=0\\&h(b)=0}$, do đó phương trình $h(x)=0$ có nghiệm $x=a$ hoặc $x=b$.\\
			Dẫn đến phương trình $f(x)=x$ có nghiệm $x=a$ hoặc $x=b$.
			\item[•] Nếu $h(a)\cdot h(b)<0$ thì do tính liên tục của $h(x)$ nên phương trình $h(x)=0$ có nghiệm thuộc khoảng $(a;b)$.\\
			Dẫn đến phương trình $f(x)=x$ cũng có nghiệm thuộc khoảng $(a;b)$.
		\end{itemize}
	}
\end{bt}
\begin{bt}%[VDC]%[DCHT Toán 11 - KNTT -Nguyễn Thành Nhân]%[1K5GG-7]
	Cho hai số $a$ và $b$ dương, $c\ne 0$ và $m$, $n$ là hai số thực tùy ý. Chứng minh phương trình $\dfrac{a}{x-m}+\dfrac{b}{x-n}=c$ luôn có nghiệm thực.
	\loigiai{
		Điều kiện xác định của phương trình $x\neq m$ và $x\neq n$.\\
		Khi $m=n$, ta có $$\dfrac{a}{x-m}+\dfrac{b}{x-n}=c\Leftrightarrow\dfrac{a}{x-m}+\dfrac{b}{x-m}=c\Leftrightarrow a+b=c\left(x-m\right)\Leftrightarrow x=m+\dfrac{a+b}{c}.$$
		Như vậy, khi $m=n$, phương trình đã cho luôn có nghiệm.\\
		Khi $m\ne n$, ta có
		$$\dfrac{a}{x-m}+\dfrac{b}{x-n}=c\Leftrightarrow a(x-n)+b(x-m)-c(x-m)(x-n)=0.$$
		Xét hàm số $f(x)=a(x-n)+b(x-m)-c(x-m)(x-n)$.\\
		Ta có hàm số liên tục trên $\mathbb{R}$.\\
		Dễ thấy $f(m)=a(m-n)$, $f(n)=b(n-m)=-b(m-n)$.\\
		Do đó $f(m)\cdot f(n)=-ab(m-n)^2< 0,\forall a > 0$, $b > 0$, $m\neq n$ .\\
		Do đó phương trình đã cho có ít nhất một nghiệm thuộc $\left(m;,n\right)$ nếu $m < n$, hoặc ít nhất một nghiệm thuộc $\left(n;m\right)$ nếu $m > n$.\\
		Vậy phương trình đã cho luôn có nghiệm thực.
	}
\end{bt}
\begin{bt}%[VDC]%[DCHT Toán 11 - KNTT -Tên GV] %[1K5GG-7]
	Cho $a$, $b$, $c$ là ba số thực tùy ý. Chứng minh rằng phương trình
	\[ab(x-a)(x-b)+bc(x-b)(x-c)+ca(x-c)(x-a)=0\] luôn có nghiệm.
	\loigiai
	{Đặt $f(x)=ab(x-a)(x-b)+bc(x-b)(x-c)+ca(x-c)(x-a)$ thì $f(x)$ liên tục trên $\mathbb{R}$.\\ 
		Ta có 
		\begin{eqnarray*}
			f(a)\cdot f(b)\cdot f(c)\cdot f(0)&=&\left[bc(a-b)(a-c)+ca(b-c)(b-a)+ab(c-a)(c-b)\right]\\
			&=&-a^2 b^2c^2\cdot (a-b)^2(b-c)^2(c-a)^2\cdot \left(a^2b^2+b^2c^2+c^2a^2\right)\\
			&\le & 0.\quad (1)
		\end{eqnarray*}
		Xảy ra một trong hai khả năng
		\begin{itemize}
			\item[•] Nếu $f(a)\cdot f(b)\cdot f(c)\cdot f(0)=0$ thì phương trình $f(x)=0$ có ít nhất một nghiệm là một trong các số $a$, $b$, $c$, $0$.
			\item[•] Nếu $f(a)\cdot f(b)\cdot f(c)\cdot f(0)<0$ thì từ $(1)$ suy ra $a$, $b$, $c$ khác $0$.\\
			Vì tích bốn số nhỏ hơn $0$ nên phải tồn tại hai số trong các số $f(a)$, $ f(b)$, $f(c)$, $f(0)$ trái dấu. Dù là hai số nào cũng đều dẫn đến phương trình $f(x)=0$ có nghiệm.
		\end{itemize}
	}
\end{bt}

\begin{bt}%[VDC]%[DCHT Toán 11 - KNTT -Tên GV] %[1K5GG-7]
	Cho phương trình $x^3-3x^2+\left(2m-2\right)x+m-3=0$. Tìm tất cả các giá trị của tham số $m$ để phương trình có ba nghiệm phân biệt $x_1$, $x_2$, $x_3$ thỏa mãn $x_1<-1<x_2<x_3$.
	\loigiai
	{ Ta giải bài toán bằng điều kiện cần và điều kiện đủ.
		\begin{itemize}
			\item[•] \textit{Điều kiện cần}.\\
			Đặt $f(x)=x^3-3x^2+\left(2m-2\right)x+m-3$ thì $f(x)$ liên tục trên $\mathbb{R}$.\\
			Giả sử phương trình có ba nghiệm phân biệt $x_1$, $x_2$, $x_3$ thỏa mãn $x_1<-1<x_2<x_3$. \\
			Ta có $f(x)=(x-x_1)\cdot (x-x_2)\cdot (x-x_3)$.\\
			Từ giả thiết $x_1<-1<x_2$ suy ra 
			\[f(-1)>0\Leftrightarrow -m-5>0\Rightarrow m<-5.\]
			Đó là điều kiện cần của bài toán. Ta chứng minh đó cũng là điều kiện đủ.
			\item[•] \textit{Điều kiện đủ}.\\
			Với $m<-5$. Ta có $f(-1)=-m-5>0$.\\
			Lai có $\lim \limits _{x\to -\infty}f(x)=-\infty$ nên tồn tại $a<-1$ để $f(a)<0$.\\
			Ta có $f(0)=m-3<0$. \\
			Lại có có $\lim \limits _{x\to +\infty}f(x)=+\infty$ nên tồn tại $b>0$ để $f(b)>0$.\\
			Vì $f(a)\cdot f(-1)<0$; $f(-1)\cdot f(0)<0$; $f(0)\cdot f(b)<0$ nên theo tính chất của hàm số liên tục, phương trình $f(x)=0$ có các nghiệm $x_1\in (a;-1)$, $x_2\in (-1;0)$; $x_3\in (0;b)$.\\
			Do đó phương trình $f(x)=0$ có ba nghiệm phân biệt $x_1$, $x_2$, $x_3$ thỏa mãn $x_1<-1<x_2<x_3$.
		\end{itemize}
		Vậy $m<-5$ là điều kiện cần và đủ của bài toán.
	}
\end{bt}
\begin{bt}%[VDC]%[DCHT Toán 11 - KNTT -Tên GV] %[1K5GG-7]
	Cho $2a+6b+19c=0$. Chứng minh rằng phương trình $ax^2+bx+c=0$ có nghiệm $\break x_0\in \left[0;\dfrac{1}{3}\right]$.
	\loigiai
	{ Đặt $f(x)=ax^2+bx+c$ thì $f(x)$ liên tục trên $\mathbb{R}$.\\
		Xét các giá trị sau $$f(0)=c;\,\,18\cdot f\left(\dfrac{1}{3}\right)=2a+6b+18c=(2a+6b+19c)-c=-c.$$
		Do đó $18\cdot f(0)\cdot f\left(\dfrac{1}{3}\right)=-c^2\le 0$. Xét hai khả năng
		\begin{itemize}
			\item[•] Nếu $c=0$ thì $18\cdot f(0)\cdot f\left(\dfrac{1}{3}\right)\Leftrightarrow \hoac{&f(0)=0\\&f\left(\dfrac{1}{3}\right)=0}$, suy ra phương trình $f(x)=0$ có nghiệm $x=0$ hoặc $x=\dfrac{1}{3}$.
			\item[•] Nếu $c\ne 0$ thì $18\cdot f(0)\cdot f\left(\dfrac{1}{3}\right)=-c^2<0$ nên phương trình luôn có nghiệm thuộc khoảng $\left(0;\dfrac{1}{3}\right)$.
		\end{itemize} 
		Vậy phương trình luôn có nghiệm thuộc đoạn $\left[0;\dfrac{1}{3}\right]$.
	}
\end{bt}
\begin{bt}%[VDC]%[DCHT Toán 11 - KNTT -Nguyễn Thành Nhân]%[1K5GG-7]
	Cho phương trình $a\cos{2x}+b\cos{x}+c=0$, với $a$, $b$, $c$ là các số thực thỏa mãn $3b+7c=3a$. Chứng minh phương trình đã cho luôn có nghiệm.
	\loigiai{Đặt $t=\cos{x}$, điều kiện $|t|\le 1$.\\
		Khi đó, ta có $\cos 2x=2\cos^2x-1=2t^2-1$. Phương trình đã cho trở thành 
		\[a(2t^2-1)+bt+c=0 \Leftrightarrow 2at^2+bt+(c-a)=0. \qquad (1)\]
		Phương trình đã cho có nghiệm khi và chỉ khi phương trình $(1)$ có nghiệm $t \in [-1;1]$.\\
		Xét $f(t)=2at^2+bt+(c-a)$, liên tục trên $\mathbb{R}$.\\
		Ta có $f(0)=c-a$, và
		\[f\left(\dfrac{2}{3}\right)=\dfrac{2}{3}\left(\dfrac{4}{3}a+b\right)+(c-a)=\dfrac{2}{3}\left(\dfrac{4a+3b}{3}\right)+(c-a).\]
		Theo giả thiết, ta có 
		\[3b+7c=3a \Rightarrow 3b=3a-7c \Rightarrow 4a+3b=7(a-c).\]
		Do đó $f\left(\dfrac{2}{3}\right)=\dfrac{2}{3}\cdot\dfrac{7(a-c)}{3}+(c-a)=\dfrac{5}{9}(a-c)$.\\
		Suy ra $f(0)f\left(\dfrac{2}{3}\right)=-\dfrac{5}{9}(c-a)^2 \le 0$.\\
		Vì $f$ liên tục trên $\mathbb{R}$ nên $f$ liên tục trên $\left[0;\dfrac{2}{3}\right]$. Do đó tồn tại $t_0 \in \left[0;\dfrac{2}{3}\right]$ sao cho $f(t_0)=0$.\\
		Suy ra phương trình $(1)$ có nghiệm thuộc $\left[0;\dfrac{2}{3}\right]$, cũng thuộc $[-1;1]$.\\
		Vậy, phương trình $a\cos{2x}+b\cos{x}+c=0$ luôn có nghiệm.}
\end{bt}

\begin{bt}%[VDC]%[DCHT Toán 11 - KNTT -Nguyễn Thành Nhân]%[1K5GG-7]
	Giả sử hai hàm số $y=f(x)$ và $y=f(x+1)$ đều liên tục trên đoạn $[0;2]$ và $f(0)=f(2)$. Chứng minh phương trình $f(x)-f(x+1)=0$ luôn có nghiệm thuộc đoạn $[0;1]$.
	\loigiai{
		Xét hàm số $g(x)=f(x)-f(x+1)$ trên đoạn $[0;1]$.\\
		Vì $y=f(x)$ và $y=f(x+1)$ đều liên tục trên đoạn $[0;2]$ nên hàm số $g(x)$ liên tục trên đoạn $[0;1]$.\\
		Ta có $\heva{&g(0)=f(0)-f(1)\\&g(1)=f(1)-f(2)=f(1)-f(0).}$\\
		Ta xét các trường hợp sau
		\begin{itemize}
			\item Nếu $f(0)-f(1)=0\Rightarrow \heva{&g(0)=0\\&g(1)=0}$. Suy ra phương trình $g(x)=0$ có nghiệm $x=0$, $x=1$. \quad $(1)$
			\item Nếu $f(0)-f(1)\ne 0$ thì $g(0)\cdot g(1)=[f(0)-f(1)]\cdot [f(1)-f(0)]<0$. Suy ra phương trình $g(x)=0$ luôn có ít nhất một nghiệm thuộc đoạn $[0;1]$. \quad $(2)$
		\end{itemize}
		Từ $(1)$ và $(2)$, suy ra phương trình $f(x)-f(x+1)=0$ luôn có nghiệm thuộc đoạn $[0;1]$.
	}
\end{bt}
% \subsubsection{Câu hỏi trắc nghiệm}
% \Opensolutionfile{ans}[ans/ans-1K5-3-Dang6]
% \begin{ex}%[DCHT Toán 11 - KNTT -Nguyễn Thành Nhân]%[1K5YG-7]
% 	Cho hàm số $y=f(x)$ liên tục trên đoạn $[a;b]$. Mệnh đề nào sau đây đúng?
% 	\choice
% 	{Nếu $f(a)\cdot f(b)>0$ thì phương trình $f(x)=0$ không có nghiệm thuộc $(a ; b)$}
% 	{\True Nếu $f(a)\cdot f(b)<0$ thì phương trình $f(x)=0$ có nghiệm thuộc $(a ; b)$}
% 	{Nếu $f(a)\cdot f(b)<0$ thì phương trình $f(x)=0$ không có nghiệm thuộc $(a ; b)$}
% 	{Nếu $f(a)\cdot f(b)>0$ thì phương trình $f(x)=0$ có nghiệm thuộc $(a ; b)$}
% 	\loigiai{
% 		"Nếu $f(a)\cdot f(b)<0$ thì phương trình $f(x)=0$ có nghiệm thuộc $(a ; b)$" là mệnh đề đúng.	
% 	} 
% \end{ex}

% \begin{ex}%[DCHT Toán 11 - KNTT -Nguyễn Thành Nhân]%[1K5YG-7]
% 	Phương trình $x^5+2x^3+16=0$ có nghiệm thuộc khoảng nào sau đây?
% 	\choice
% 	{$(0;1)$}
% 	{$(-10;-2)$}
% 	{$(-1;0)$}
% 	{\True $(-2;-1)$}
% 	\loigiai{
% 		Hàm số $x^5+2 x^3+16=0$ liên tục trên $\mathbb{R}$.\\
% 		Do $f(-2)\cdot f(-1)=-32\cdot 13 <0$ nên phương trình $x^5+2 x^3+16=0$ có nghiệm thuộc khoảng $(-2;-1)$. 	
% 	} 
% \end{ex}

% \begin{ex}%[DCHT Toán 11 - KNTT -Nguyễn Thành Nhân]%[1K5YG-7]
% 	Phương trình $ 3x^5+5x^3+10=0 $ có nghiệm thuộc khoảng nào sau đây?
% 	\choice
% 	{$ (0;1) $}
% 	{$ (-1;0) $}
% 	{$ (-10;-2) $}
% 	{\True $ (-2;-1) $}
% 	\loigiai{
% 		Đặt $ f(x)=3x^5+5x^3+10 $.\\
% 		Tập xác định của hàm số là $ \mathscr{D}=\mathbb{R} \Rightarrow f(x)$	liên tục trên $ \mathbb{R} $.\\
% 		Suy ra $ f(x) $ liên tục trên $ [-2;-1] $.\\
% 		Ta có $ f(-2)=-126 $; $ f(-1)=2\Rightarrow f(-2)\cdot f(-10)=-252<0 $.\\
% 		Phương trình $ 3x^5+5x^3+10=0 $ có nghiệm thuộc khoảng $ (-2;-1) $.
% 	}
% \end{ex}

% \begin{ex}%[DCHT Toán 11 - KNTT -Nguyễn Thành Nhân]%[1K5BG-7]
% 	Cho hàm số $f(x)$ liên tục trên đoạn $[-1 ; 4]$, biết $f(-1)=2$, $f(4)=7$. Có thể nói gì về số nghiệm của phương trình $f(x)=5$ trên đoạn $[-1 ; 4]$.	
% 	\choice
% 	{Có hai nghiệm phân biệt }
% 	{Có đúng một nghiệm}
% 	{\True Có ít nhất một nghiệm}
% 	{Vô nghiệm}
% 	\loigiai{
% 		Xét hàm số $g(x)=f(x)-5$ liên tục trên 	đoạn $[-1 ; 4]$ có $g(-1)=f(-1)-5=2-5=-3<0$ và $g(4)=f(4)-5=7-5=2>0$.\\
% 		Suy ra $g(-1)\cdot g(4)<0$.\\
% 		Do đó phương trình $g(x)=0$ có ít nhất một nghiệm thuộc $(-1;4)$ hay phương trình $f(x)=5$ trên đoạn $[-1 ; 4]$ 	Có ít nhất một nghiệm.
% 	}
% \end{ex}

% \begin{ex}%[DCHT Toán 11 - KNTT -Nguyễn Thành Nhân]%[1K5BG-7]
% 	Cho hàm số $y=f(x)$ liên tục trên $[2022;2023]$, biết rằng $f(2022)=-2023$, $f(2023)=2022$. Kết luận nào sau đây chắc chắn đúng?
% 	\choice
% 	{\True Phương trình $f(x)=0$ có nghiệm}
% 	{Phương trình $f(x)=0$ vô nghiệm}
% 	{Phương trình $f(x)=0$ có $3$ nghiệm}
% 	{Phương trình $f(x)=0$ có $2$ nghiệm}
% 	\loigiai{Ta có $f(x)$ liên tục trên $[2022;2023]$ và $f(2022)\cdot f(2023)<0$ nên $\exists x_0\in(2022;2023)\colon f(x_0)=0$.\\
% 		Vậy mệnh đề chắc chắc đúng là ``Phương trình $f(x)=0$ có nghiệm''.}
% \end{ex}

% \begin{ex}%[DCHT Toán 11 - KNTT -Nguyễn Thành Nhân]%[1K5BG-7]
% 	Cho các câu
% 	\begin{enumerate}
% 		\item Nếu hàm số $y=f(x)$ liên tục trên $(a;b)$ và $f(a) \cdot f(b)<0$ thì tồn tại $x_0 \in (a;b)$ sao cho $f\left(x_0 \right) = 0$.
% 		\item Nếu $y=f(x)$ liên tục trên $[a;b]$ và $f(a)\cdot f(b) <0$ thì phương trình $f(x) = 0$ có nghiệm thuộc khoảng $(a;b)$.
% 		\item Nếu hàm số $y=f(x)$ liên tục, đơn điệu trên $[a;b]$ và $f(a) \cdot f(b) <0$ thì phương trình $f\left( x \right)=0$ có nghiệm duy nhất thuộc $(a;b)$.
% 	\end{enumerate}
% 	Trong ba câu trên
% 	\choice
% 	{\True có đúng một câu {\bf sai}}
% 	{cả ba câu đều đúng}
% 	{có đúng hai câu {\bf sai}}
% 	{cả ba câu đều {\bf sai}}
% 	\loigiai{
% 		Trong ba câu trên có hai câu đúng
% 		\begin{itemize}
% 			\item Nếu $y=f(x)$ liên tục trên $[a;b]$ và $f(a)\cdot f(b) <0$ thì phương trình $f(x) = 0$ có nghiệm thuộc khoảng $(a;b)$.
% 			\item Nếu hàm số $y=f(x)$ liên tục, đơn điệu trên $[a;b]$ và $f(a) \cdot f(b) <0$ thì phương trình $f\left( x \right)=0$ có nghiệm duy nhất thuộc $(a;b)$.
% 		\end{itemize}
% 		và một câu sai
% 		\begin{itemize}
% 			\item Nếu hàm số $y=f(x)$ liên tục trên $(a;b)$ và $f(a) \cdot f(b)<0$ thì tồn tại $x_0 \in (a;b)$ sao cho $f\left(x_0 \right) = 0$.
% 		\end{itemize}
% 	}
% \end{ex}

% \begin{ex}%[DCHT Toán 11 - KNTT -Nguyễn Thành Nhân]%[1K5BG-7]
% 	Cho hàm số $f(x)$ xác định trên $[a;b]$. Trong các mệnh đề sau, mệnh đề nào đúng?
% 	\choice
% 	{Nếu phương trình $f(x)=0$ có nghiệm trong khoảng $(a;b)$ thì hàm số $f(x)$ phải liên tục trên $(a;b)$}
% 	{Nếu hàm số $f(x)$ liên tục trên $[a;b]$ và $f(a)\cdot f(b) >0$ thì phương trình $f(x)=0$ không có nghiệm trong khoảng $(a;b)$}
% 	{Nếu $f(a) \cdot f(b)<0$ thì phương trình $f(x)=0$ có ít nhất một nghiệm trong khoảng $(a;b)$}
% 	{\True Nếu hàm số $f(x)$ liên tục, tăng trên $[a;b]$ và $f(a) \cdot f(b)>0$ thì phương trình $f(x) =0$ không có nghiệm trong khoảng $(a;b)$}
% 	\loigiai{
% 		Mệnh đề đúng là ``Nếu hàm số $f(x)$ liên tục, tăng trên $[a;b]$ và $f(a) \cdot f(b)>0$ thì phương trình $f(x) =0$ không có nghiệm trong khoảng $(a;b)$.''
% 	}
% \end{ex}

% \begin{ex}%[DCHT Toán 11 - KNTT -Nguyễn Thành Nhân]%[1K5BG-7]
% 	Chứng minh rằng phương trình $x^3-x+3=0$ có ít nhất một nghiệm. Một bạn học sinh trình bày lời giải như sau:
% 	\begin{enumerate}[Bước 1.]
% 		\item Xét hàm số $y=f(x)=x^3-3x+3$ liên tục trên $\Bbb{R}$.
% 		\item Ta có $f(0)=3$ và $f(-2)=-3$.
% 		\item Suy ra $f(0)\cdot f(-2)>0$.
% 		\item Vậy phương trình đã cho có ít nhất một nghiệm. Hãy tìm bước giải {\bf sai} của bạn học sinh trên?
% 	\end{enumerate}
% 	\choice
% 	{\True Bước $3$}
% 	{Bước $4$ và Bước $3$}
% 	{Bước $1$}
% 	{Bước $1$ và Bước $3$}
% 	\loigiai{
% 		Học sinh sai ở Bước $3$ vì $f(0)\cdot f(-2)<0$.
% 	}
% \end{ex}

% \begin{ex}%[DCHT Toán 11 - KNTT -Nguyễn Thành Nhân]%[1K5BG-7]
% 	Cho hàm số $f(x)=3x^4+3x-2$. Khẳng định nào sau đây là {\bf sai}?
% 	\choice
% 	{\True Phương trình $f(x)=0$ vô nghiệm}
% 	{Phương trình $f(x)=0$ có nghiệm trong khoảng $(0;1)$}
% 	{Phương trình $f(x)=0$ có 1 it nhất một nghiệm trong khoảng $(-1;1)$}
% 	{Phương trình $f(x)=0$ có nghiệm trên $\mathbb{R}$}
% 	\loigiai{
% 		Hàm số $f(x)=3x^4+3x-2$ liên tục trên $\mathscr{D}=\mathbb{R}$.\\
% 		Ta có $\heva{&f(0)=-2\\&f(1)=4}\Rightarrow f(0)\cdot f(1)=-8<0$ nên phương trình $f(x)=0$ có ít nhất $1$ nghiệm thuộc khoảng $(0;1)$.\\
% 		Suy ra phương trình $f(x)=0$ có ít nhất $1$ nghiệm thuộc khoảng $(-1;1)$ cũng như có nghiệm trên $\mathbb{R}$.\\
% 		Vậy khẳng định {\bf sai} là \lq\lq Phương trình $f(x)=0$ vô nghiệm \rq\rq.
% 	}
% \end{ex}

% \begin{ex}%[DCHT Toán 11 - KNTT -Nguyễn Thành Nhân]%[1K5BG-7]
% 	Cho hàm số $y=f(x)$ liên tục trên đoạn $[1;5]$ và $f(1)=2$, $f(5)=10$. Khẳng định nào sau đây \textbf{đúng}?
% 	\choice
% 	{Phương trình $f(x)=6$ vô nghiệm}
% 	{\True Phương trình $f(x)=7$ có ít nhất một nghiệm trên khoảng $(1;5)$}
% 	{Phương trình $f(x)=2$ có hai nghiệm $x=1$, $x=5$}
% 	{Phương trình $f(x)=7$ vô nghiệm}
% 	\loigiai{
% 		\immini{Đồ thị hàm số $f(x)$ là một đường liền nét trên khoảng $(1;5)$.\\
% 			Đường thẳng $y=7$ song song với trục $Ox$ và sẽ cắt đồ thị hàm $y=f(x)$ tại ít nhất một điểm có hoành độ thuộc khoảng $(1;5)$. Do đó phương trình $f(x)=7$ có ít nhất một nghiệm thuộc khoảng $(1;5)$.\\
% 			Ta có hình vẽ minh hoạ như hình vẽ bên.}{\begin{tikzpicture}[scale=0.3, font=\footnotesize, line join=round, line cap=round, >=stealth]
% 				\tikzset{label style/.style={font=\footnotesize}}
% 				%%Nhập giới hạn đồ thị và hàm số cần vẽ
% 				\def \xmin{-1}
% 				\def \xmax{6}
% 				\def \ymin{-1}
% 				\def \ymax{11}
% 				%\def \hamso{sin(\x r)}
% 				%\def \tiemcanxien{\x+1}
% 				%%Tự động
% 				\draw[->] (\xmin,0)--(\xmax,0) node[below right] {$x$};
% 				\draw[->] (0,\ymin)--(0,\ymax) node[above] {$y$};
% 				\draw (0,0) node [below left] {$O$};
% 				%%Vẽ các điểm trên 2 hệ trục
% 				\foreach \x in {1,5}
% 				\draw[thin] (\x,1pt)--(\x,-1pt) node [below] {$\x$};
% 				\foreach \y in {2,10}
% 				\draw[thin] (1pt,\y)--(-1pt,\y) node [left] {$\y$};
% 				\draw (0,7) node[above left]{$y=7$};
% 				%%Vẽ thêm mấy cái râu ria
% 				\draw[name path=duongthang](-1,7)--(5.5,7);
% 				\draw[name path=duongcong1] (1,2) .. controls (2,-1) and (2.5,15) .. (3,7);
% 				\draw[name path=duongcong2] (3,7) .. controls (3.5,-3) and (4.5,9) .. (5,10);
% 				\clip (\xmin+0.01,\ymin+0.01) rectangle (\xmax-0.01,\ymax-0.01);
% 				\draw[dashed,thin](1,0)--(1,2)--(0,2);
% 				\draw[dashed,thin](5,0)--(5,10)--(0,10);
% 	\end{tikzpicture}}}
% \end{ex}

% \begin{ex}%[DCHT Toán 11 - KNTT -Nguyễn Thành Nhân]%[1K5BG-7]
% 	Cho phương trình $882x^5-441x^4-116x^3+58x^2+2x-1=0$. Mệnh đề nào sau đây \textbf{sai}?
% 	\choice
% 	{Phương trình có nghiệm trong khoảng $(0;1)$}
% 	{Phương trình có nghiệm trong khoảng $(-1;0)$}
% 	{Phương trình có $5$ nghiệm phân biệt}
% 	{\True Phương trình có đúng $4$ nghiệm}
% 	\loigiai{Hàm số $f(x)=882x^5-441x^4-116x^3+58x^2+2x-1$ liên tục trên $\mathbb{R}$ nên nó liên tục trên $\left(-1;1\right)$.
% 		\begin{itemize}
% 			\item $\heva{&f(-0{,}4)=-5{,}417\\&f(-0{,}3)=1{,}0366}\Rightarrow f(-0{,}4)\cdot f(-0{,}3)<0$ nên phương trình có ít nhất một nghiệm trên khoảng $\left(-0{,}4; -0{,}3\right)$.
% 			\item $\heva{&f(-0{,}2)=-0{,}8601\\&f(-0{,}1)=-0{,}556}\Rightarrow f(-0{,}2)\cdot f(-0{,}1)<0$ nên phương trình có ít nhất một nghiệm trên khoảng $\left(-0{,}2; -0{,}1\right)$.
% 			\item $\heva{&f(0{,}1)=-0{,}371\\&f(0{,}2)=0{,}3686}\Rightarrow f(0{,}1)\cdot f(0{,}2)<0$ nên phương trình có ít nhất một nghiệm trên khoảng $\left(0{,}1; 0{,}2\right)$.
% 			\item $\heva{&f(0{,}3)=0{,}2591\\&f(0{,}4)=-0{,}601}\Rightarrow f(0{,}3)\cdot f(0{,}4)<0$ nên phương trình có ít nhất một nghiệm trên khoảng $\left(0{,}3; 0{,}4\right)$.
% 			\item $\heva{&f(0{,}4)=-0{,}601\\&f(0{,}6)=7{,}4547}\Rightarrow f(0{,}4)\cdot f(0{,}6)<0$ nên phương trình có ít nhất một nghiệm trên khoảng $\left(0{,}4; 0{,}6\right)$.
% 		\end{itemize}
% 		Phương trình đã cho là phương trình bậc $5$ nên nó có đúng $5$ nghiệm.	
% 	}
% \end{ex}

% \begin{ex}%[DCHT Toán 11 - KNTT -Nguyễn Thành Nhân]%[1K5BG-7]
% 	Phương trình $x^3-3x^2+5x+1=0$ có ít nhất một nghiệm thuộc khoảng nào sau đây?
% 	\choice
% 	{$(0;1)$}
% 	{$(2;3)$}
% 	{$(-2;-1)$}
% 	{\True $(-1;0)$}
% 	\loigiai{Xét hàm số $f(x)=x^3-3x^2+5x+1$ là hàm đa thức nên liên tục trên $\mathbb{R}$.\\
% 		Ta có $f(x)$ liên tục trên $[-1;0]$ và $f(-1)\cdot f(0)=(-8)\cdot1=-8<0$.\\
% 		Từ đó ta suy ra tồn tại $x_0\in(-1;0)$ sao cho $f\left(x_0\right)=0$, hay phương trình $f(x)=0$ có ít nhất một nghiệm thuộc khoảng $(-1;0)$.}
% \end{ex}

% \begin{ex}%[DCHT Toán 11 - KNTT -Nguyễn Thành Nhân]%[1K5BG-7]
% 	Hàm số $f(x)$ liên tục trên đoạn $[2 ; 4]$ và $f(2) \cdot f(4)<0$. Khẳng định nào sau đây đúng?
% 	\choice
% 	{\True Phương trình $f(x)=0$ có nghiệm}
% 	{Phương trình $f(x)=0$ vô số nghiệm}
% 	{Phương trình $f(x)=0$ có ít nhất $6$ nghiệm}
% 	{Phương trình $f(x)=0$ vô nghiệm}
% 	\loigiai{
% 		Hàm số $f(x)$ liên tục trên đoạn $[2 ; 4]$ và $f(2) \cdot f(4)<0$ nên phương trình $f(x)=0$ có ít nhất một nghiệm thuộc $(2;4)$.
% 	}
% \end{ex}

% \begin{ex}%[DCHT Toán 11 - KNTT -Nguyễn Thành Nhân]%[1K5BG-7]
% 	Cho phương trình $2x^3-10x-7=0\quad (1)$. Mệnh đề nào sau đây là mệnh đề \textbf{sai}?
% 	\choice
% 	{\True Phương trình $(1)$ không có nghiệm trên khoảng $(0;+\infty)$}
% 	{Hàm số $f(x)=2x^3-10x-7$ liên tục trên $\mathbb{R}$}
% 	{Phương trình $(1)$ có ít nhất hai nghiệm trên khoảng $(-1;3)$}
% 	{Phương trình $(1)$ có ít nhất một nghiệm trên khoảng $(0;3)$}
% 	\loigiai{
% 		Hàm số $f(x)=2x^3-10x-7$ là hàm số đa thức nên nó liên tục trên $\mathbb{R}$. Vì thế, nó cũng liên tục trên mỗi đoạn $[0;3]$, $[-1;3]$.\\
% 		Vì $f(-1)=1$ và $f(0)=-7$ nên $f(-1)\cdot f(0)<0$. Vì thế, phương trình $2x^3-10-7=0$ có ít nhất một nghiệm thuộc khoảng $(-1;0)$.\\
% 		Vì $f(0)=-7$ và $f(3)=17$ nên $f(0)\cdot f(3)<0$. Vì thế, phương trình $2x^3-10-7=0$ có ít nhất một nghiệm thuộc khoảng $(0;3)$.\\
% 		Do đó, phương trình $2x^3-10x-7=0$ có ít nhất hai nghiệm thuộc khoảng $(-1;3)$.
% 	}
% \end{ex}

% \begin{ex}%[DCHT Toán 11 - KNTT -Nguyễn Thành Nhân]%[1K5BG-7]
% 	Phương trình $2x^3-6x+1=0$ có bao nhiêu nghiệm phân biệt thuộc $(-2;2)$?
% 	\choice 
% 	{\True $3$}
% 	{$1$}
% 	{$0$}
% 	{$2$}
% 	\loigiai{
% 		Xét hàm số $f(x)=2x^3-6x+1$ liên tục trên đoạn $[-2;2]$.\\
% 		Ta có
% 		\[f(-2)=-3,\quad f(0)=1,\quad f(1)=-3,\quad f(2)=5.\]
% 		Suy ra $f(-2)f(0)<0$, $f(0)f(1)<0$, $f(0)f(2)<0$.\\ 
% 		Từ đó suy ra phương trình đã cho có ít nhất $3$ nghiệm phân biệt thuộc khoảng $(-2;2)$. Mặt khác, do phương trình đã cho là phương trình bậc ba nên nó có đúng $3$ nghiệm phân biệt thuộc khoảng $(-2;2)$.
% 	}
% \end{ex}

% \begin{ex}%[DCHT Toán 11 - KNTT -Nguyễn Thành Nhân]%[1K5BG-7]
% 	Cho hàm số $f(x)=-4x^3+4 x-1$. Mệnh đề nào sau đây \textbf{sai}?
% 	\choice
% 	{Hàm số đã cho liên tục trên $\mathbb{R}$}
% 	{\True Phương trình $f(x)=0$ không có nghiệm trên khoảng $(-\infty ; 1)$}
% 	{Phương trình $f(x)=0$ có nghiệm trên khoảng $(-2 ; 0)$}
% 	{Phương trình $f(x)=0$ có ít nhất hai nghiệm trên khoảng $\left(-3 ; \dfrac{1}{2}\right)$}
% 	\loigiai{
% 		\begin{itemize}
% 			\item Hàm $f$ là hàm đa thức nên liên tục trên $\mathbb{R}$. Nên mệnh đề \lq\lq Hàm số đã cho liên tục trên $\mathbb{R}$\rq\rq\; đúng.
% 			\item Ta có $\heva{&f(-1)=-1<0 \\& f(-2)=23>0} \Rightarrow f(x)=0$ có nghiệm $x_1$ trên $(-2 ; -1)$, mà
% 			$(-2 ;-1) \subset(-2 ; 0) \subset(-\infty ; 1)$.\\
% 			Nên mệnh đề \lq\lq Phương trình $f(x)=0$ không có nghiệm trên khoảng $(-\infty ; 1)$\rq\rq\; sai và mệnh đề \lq\lq Phương trình $f(x)=0$ có nghiệm trên khoảng $(-2 ; 0)$\rq\rq\; đúng.
% 			\item Ta có $\heva{&f(-1)=-1<0 \\& f(-2)=23>0} \Rightarrow f(x)=0$ có nghiệm $x_1$ trên $(-2 ; -1)$, và
% 			$\heva{&f(-1)=-1 \\& f\left(\dfrac{1}{2}\right)=\dfrac{1}{2}} \Rightarrow f(-1) \cdot f\left(\dfrac{1}{2}\right)<0$\\
% 			$\Rightarrow f(x)=0 \text { có nghiệm } x_2 \text { trên }\left(-1 ; \dfrac{1}{2}\right)$.\\
% 			Nên mệnh đề \lq\lq Phương trình $f(x)=0$ có ít nhất hai nghiệm trên khoảng $\left(-3 ; \dfrac{1}{2}\right)$\rq\rq\; đúng.
% 		\end{itemize}
% 	}
% \end{ex}

% \begin{ex}%[DCHT Toán 11 - KNTT -Nguyễn Thành Nhân]%[1K5BG-7]
% 	Tìm khẳng định \textbf{sai} trong các khẳng định sau.
% 	\choice
% 	{\True Hàm số $f(x)=\dfrac{1}{x}$ có $f(-1)\cdot f(1)<0$ nên phương trình $f(x)=0$ có nghiệm trong $(-1;1)$}
% 	{Phương trình $-x^3-2x^2-3x+4=0$ có ít nhất một nghiệm}
% 	{Hàm số $f(x)=-x^3-2x^2-3x+4$ liên tục trên $\mathbb{R}$}
% 	{Hàm số $f(x)=-x^3-2x^2-3x+4$ có $f(0)\cdot f(1)<0$ nên phương trình $f(x)=0$ có ít nhất một nghiệm trong khoảng $(0;1)$}
% 	\loigiai{
% 		Hàm số $f(x)=\dfrac{1}{x}$ không xác định tại $x=0$ nên không liên tục trên $[0;1]$, do đó khẳng định \lq\lq$f(-1)\cdot f(1)<0$ nên phương trình $f(x)=0$ có nghiệm trong $(-1;1)$\rq\rq\, là sai.\\
% 		Hàm số $f(x)=-x^3-2x^2-3x+4$ liên tục trên $\mathbb{R}$ nên liên tục trên $[0;1]$, có $f(0)\cdot f(1)=-8<0$ nên phương trình $f(x)=0$ có ít nhất một nghiệm trên $(0;1)$, suy ra phương trình $f(x)=0$ có ít nhất một nghiệm.}
% \end{ex}
% \begin{ex}%[DCHT Toán 11 - KNTT -Nguyễn Thành Nhân]%[1K5BG-7]
% 	Cho phương trình $2x^3+x^2-1=0$ $(1)$. Mệnh đề nào dưới đây đúng?
% 	\choice
% 	{Phương trình $(1)$ vô nghiệm trên khoảng $(0 ; 2)$}
% 	{\True Phương trình $(1)$ có ít nhất một nghiệm trên khoảng $(0 ; 2)$}
% 	{Phương trình $(1)$ có đúng 4 nghiệm trên khoảng $(0 ; 2)$}
% 	{Phương trình $(1)$ có vô số nghiệm trên khoảng $(0 ; 2)$}
% 	\loigiai{
% 		Đặt $f(x)=2x^3+x^2-1=0$ là hàm liên tục trên $[0;2]$.\\
% 		Có $f(0)=-1$, $f(2)=19$ nên $f(0)\cdot f(2)<0$, vậy phương trình  $(1)$ có ít nhất một nghiệm trên khoảng $(0 ; 2)$.
% 	}
% \end{ex}

% \begin{ex}%[DCHT Toán 11 - KNTT -Nguyễn Thành Nhân]%[1K5KG-7]
% 	Xét phương trình sau trên tập số thực $x^3+x=a \quad$(1). Chọn khẳng định đúng trong các khẳng định dưới đây?
% 	\choice
% 	{Phương trình $(1)$ chỉ có nghiệm khi $x>a$}
% 	{Phương trình $(1)$ vô nghiệm khi $x \geq a$}
% 	{Phương trình $(1)$ chỉ có nghiệm khi $x \geq a$}
% 	{\True Phương trình $(1)$ có nghiệm $\forall a \in \mathbb{R}$}
% 	\loigiai{
% 		Xét $f(x)=x^3+x-a$ liên tục trên $\mathbb{R}$.\\
% 		Với mọi $a$ ta có $\lim \limits_{x \to -\infty} f(x)=-\infty$ và $\lim \limits_{x \to +\infty} f(x)=+\infty$.\\
% 		Do đó tồn tại các số $m$, $n$ sao cho $f(m)>0$ và $f(n)<0$, tức là $f(m)\cdot f(n)<0$.\\
% 		Vậy $f(x)$ luôn có nghiệm với mọi $a\in\mathbb{R}$.
% 	}
% \end{ex}

% \begin{ex}%[DCHT Toán 11 - KNTT -Nguyễn Thành Nhân]%[1K5KG-7]
% 	Cho phương trình $-4x^3+4x-1=0$. Tìm khẳng định \textbf{sai} trong các khẳng định sau.
% 	\choice
% 	{Phương trình đã cho có ba nghiệm phân biệt}
% 	{\True Phương trình đã cho chỉ có một nghiệm trong khoảng $(0;1)$}
% 	{Phương trình đã cho có ít nhất một nghiệm trong khoảng $(-2;0)$}
% 	{Phương trình đã cho có ít nhất một nghiệm trong khoảng $\left(-\dfrac{1}{2};\dfrac{1}{2}\right)$}
% 	\loigiai{
% 		Đặt $f(x)=-4x^3+4x-1$, $f(x)$ là hàm đa thức nên xác định và liên tục trên $\mathbb{R}$.\\
% 		Ta có $f(-2)=23$, $f(0)=-1$, $f\left(\dfrac{1}{2}\right)=\dfrac{1}{2}$, $f(1)=-1$, $f\left(-\dfrac{1}{2}\right)=-\dfrac{5}{2}$.\\
% 		Suy ra $\heva{&f(-2)\cdot f(0)<0\\&f(0)\cdot f\left(\dfrac{1}{2}\right)<0\\&f\left(\dfrac{1}{2}\right)\cdot f(1)<0.}$\\
% 		Do đó phương trình $f(x)=0$ có ít nhất một nghiệm trong khoảng $(-2;0)$, có ít nhất một nghiệm trong khoảng $\left(0;\dfrac{1}{2}\right)$, có ít nhất một nghiệm trong khoảng $\left(\dfrac{1}{2};1\right)$.\\
% 		Khi đó phương trình $f(x)=0$ có ít nhất hai nghiệm trong khoảng $(0;1)$ và phương trình $f(x)=0$ có $3$ nghiệm phân biệt.\\
% 		Mặt khác $f\left(-\dfrac{1}{2}\right)\cdot f\left(\dfrac{1}{2}\right)<0$ nên phương trình $f(x)=0$ có ít nhất một nghiệm trong khoảng $\left(-\dfrac{1}{2};\dfrac{1}{2}\right)$.\\
% 		Vậy khẳng định ``Phương trình đã cho chỉ có một nghiệm trong khoảng $(0;1)$'' là sai.
% 	}
% \end{ex}

% \begin{ex}%[DCHT Toán 11 - KNTT -Nguyễn Thành Nhân]%[1K5KG-7]
% 	Phương trình nào sau đây có nghiệm trong khoảng $(0;1)$?
% 	\choice
% 	{$(x-1)^5-x^9-2=0$}
% 	{$3x^4-4x^2+5=0$}
% 	{\True $3x^{2019}-8x+4=0$}
% 	{$2x^2-3x+4=0$}
% 	\loigiai{
% 		\begin{itemize}
% 			\item Hàm số $f(x)=3x^{2019}-8x+4$ liên tục trên đoạn $[0;1]$ và có $f(0)\cdot f(1)=4\cdot(-1)<0$ nên phương trình $f(x)=0$ có nghiệm trên khoảng $(0;1)$.
% 			\item Hàm số $g(x)=3x^4-4x^2+5=3\left(x^2-\dfrac{2}{3}\right)^2+\dfrac{11}{3} \geq \dfrac{11}{3}>0,\forall x$ nên phương trình $g(x)=0$ vô nghiệm.
% 			\item Hàm số $h(x)=2x^2-3x+4=2\left(x-\dfrac{3}{4}\right)^2+\dfrac{23}{8}>0,\forall x$ nên phương trình $h(x)=0$ vô nghiệm.
% 			\item Hàm số $p(x)=(x-1)^5-x^9-2$ trên $(0;1)$ có $-1<(x-1)^5<0 ;-1<-x^9<0$
% 			$ \Rightarrow p(x)<0,\forall x \in (0;1)$ nên trên $(0;1)$ phương trình $p(x)=0$ vô nghiệm.
% 		\end{itemize}		
% 	}
% \end{ex}
% \begin{ex}%[DCHT Toán 11 - KNTT -Nguyễn Thành Nhân]%[1K5KG-7]
% 	Cho phương trình $\left(3m^2-m-2\right)x^{2024}\cdot \left(x^{2023}+1\right)+2 x-1=0$. Tìm tất cả các giá trị của $m$ để phương trình có nghiệm.
% 	\choice
% 	{$m\in\mathbb{R}\setminus\left\{1;-\dfrac{2}{3}\right\}$}
% 	{\True $\forall m\in\mathbb{R}$}
% 	{$m=1; m=-\dfrac{2}{3}$}
% 	{$\hoac{&m<0\\& m>1}$}
% 	\loigiai{
% 		Xét hàm số $y=\left(3m^2-m-2\right)x^{2024}\cdot \left(x^{2023}+1\right)+2 x-1$.
% 		\begin{itemize}
% 			\item \textbf{Trường hợp $1$:} $3m^2-m-2>0$, ta có $\lim\limits_{x\rightarrow +\infty} y=+\infty$, $\lim\limits_{x\rightarrow -\infty} y=-\infty$ suy ra phương trình $y=0$ có nghiệm.
% 			\item \textbf{Trường hợp $2$:} $3m^2-m-2<0$, ta có $\lim\limits_{x\rightarrow +\infty} y=-\infty$, $\lim\limits_{x\rightarrow -\infty} y=+\infty$ suy ra phương trình $y=0$ có nghiệm.
% 			\item \textbf{Trường hợp $3$}: $3m^2-m-2=0$, khi đó phương trình đã cho có nghiệm $x=\dfrac{1}{2}$.
% 		\end{itemize}
% 		Vậy phương trình đã cho luôn có nghiệm $\forall m\in\mathbb{R}$.
% 	}
% \end{ex}
% \begin{ex}%[DCHT Toán 11 - KNTT -Nguyễn Thành Nhân]%[1K5KG-7]
% 	Cho phương trình $x^4-3x^3+x-\dfrac{1}{8}=0\quad(1)$. Chọn khẳng định đúng.
% 	\choice
% 	{Phương trình $(1)$ có đúng ba nghiệm trên khoảng $(-1;3)$}
% 	{\True Phương trình $(1)$ có đúng bốn nghiệm trên khoảng $(-1;3)$}
% 	{Phương trình $(1)$ có đúng hai nghiệm trên khoảng $(-1;3)$}
% 	{Phương trình $(1)$ có đúng một nghiệm trên khoảng $(-1;3)$}
% 	\loigiai{
% 		Hàm số $f(x)=x^4-3x^3+x-\dfrac{1}{8}$ liên tục trên đoạn $[-1;3]$ và có $f(-1)=\dfrac{23}{8}$;$f\left(-\dfrac{1}{2}\right)=-\dfrac{3}{16}$; $f\left(\dfrac{1}{2}\right)=\dfrac{1}{16}$;  $f\left(1\right)=-\dfrac{9}{8}$; $f\left(3\right)=\dfrac{23}{8}$.\\
% 		Suy ra trên mỗi khoảng $\left(-1;-\dfrac{1}{2}\right)$; $\left(-\dfrac{1}{2};\dfrac{1}{2}\right)$; $\left(\dfrac{1}{2};1\right)$ ; $\left(1;3\right)$ phương trình có ít nhất một nghiệm.\\ Mặt khác phương trình là bậc $4$ nên có không quá $4$ nghiệm.\\
% 		Vậy phương trình $(1)$ có đúng bốn nghiệm trên khoảng $(-1;3)$.
% 	}
% \end{ex}
% \begin{ex}%[DCHT Toán 11 - KNTT -Nguyễn Thành Nhân]%[1K5KG-7]
% 	Cho phương trình $x^3+ax^2+bx+c=0$\qquad $(1)$ trong đó $a$, $b$, $c$ là các tham số thực. Chọn khẳng định đúng trong các khẳng định sau.
% 	\choice
% 	{\True Phương trình $(1)$ có ít nhất một nghiệm với mọi $a$, $b$, $c$}
% 	{Phương trình $(1)$ có ít nhất hai nghiệm với mọi $a$, $b$, $c$}
% 	{Phương trình $(1)$ có ít nhất ba nghiệm với mọi $a$, $b$, $c$}
% 	{Phương trình $(1)$ vô nghiệm với mọi $a$, $b$, $c$}
% 	\loigiai{
% 		Xét hàm số $f(x)=x^3+ax^2+bx+c$ liên tục trên $\mathbb{R}$.\\
% 		Ta có $\lim\limits_{x\to -\infty} f(x)=-\infty$ và $\lim\limits_{x\to +\infty} f(x)=+\infty$.\\
% 		Do đó phương trinh $f(x)=0$ có ít nhất một nghiệm với mọi $a$, $b$, $c$.
% 	}
% \end{ex}

% \begin{ex}%[DCHT Toán 11 - KNTT -Nguyễn Thành Nhân]%[1K5KG-7]
% 	Khẳng định nào trong các khẳng định sau đây là \textbf{sai}?
% 	\choice
% 	{Phương trình $x^4+mx^2-2mx-2=0$ luôn có nghiệm với mọi giá trị của tham số $m$}
% 	{\True Phương trình $3x^6-3x^3+5x-2=0$ \textbf{không} có nghiệm thuộc khoảng $(-2;2)$}
% 	{Phương trình $x^3-3x+1=0$ có ba nghiệm phân biệt}
% 	{Phương trình $m(x-1)^2(x-2)+2x-3=0$ luôn có nghiệm với mọi giá trị của tham số $m$}
% 	\loigiai{
% 		\begin{itemize}
% 			\item Hàm số $f(x)=x^4+mx^2-2mx-2$ liên tục trên đoạn $[0;2]$, có $f(0)\cdot f(2) = -2\cdot 14 = -28<0$ nên phương trình $f(x)=0$ có ít nhất một nghiệm thuộc khoảng $(0;2)$ với mọi $m$ hay phương trình $x^4+mx^2-2mx-2=0$ luôn có nghiệm với mọi giá trị của tham số $m$.
% 			\item Hàm số $g(x)=3x^6-3x^3+5x-2$ liên tục trên đoạn $[0;1]$, có $g(0)\cdot g(1) = -2\cdot 3 =-6<0$ nên phương trình $g(x)=0$ có ít nhất một nghiệm thuộc khoảng $(0;1)$ hay phương trình $g(x)=0$ có nghiệm thuộc khoảng $(-2;2)$.
% 			\item Hàm số $h(x)=x^3-3x+1$ liên tục trên các đoạn $[-2;0]$, $[0;1]$, $[1;2]$ và có $h(-2)=-1$, $h(0)=1$, $h(1)=-1$, $h(2)=3$.\\
% 			Do $h(-2)\cdot h(0)<0$, $h(0)\cdot h(1)<0$, $h(1)\cdot h(2)<0$ nên phương trình $h(x)=0$ có ba nghiệm $x_1\in (-2;0)$, $x_2\in (0;1)$, $x_3\in (1;2)$ là ba nghiệm phân biệt.\\
% 			Mặt khác phương trình $h(x)=0$ là phương trình bậc ba nên có không quá ba nghiệm.\\
% 			Vậy phương trình $h(x)=0$ có ba nghiệm phân biệt.
% 			\item Hàm số $p(x)=m(x-1)^2(x-2)+2x-3$ liên tục trên đoạn $[1;2]$, có $p(1)\cdot p(2) = (-1)\cdot 1 = -1<0$ nên phương trình $p(x)=0$ luôn có nghiệm thuộc khoảng $(1;2)$ với mọi $m$ hay phương trình $m(x-1)^2(x-2)+2x-3=0$ luôn có nghiệm với mọi giá trị của tham số $m$.
% 		\end{itemize}
% 	}
% \end{ex}

% \begin{ex}%[DCHT Toán 11 - KNTT -Nguyễn Thành Nhân]%[1K5KG-7]
% 	Cho phương trình $x^4-4x^3+1=0$. Tìm mệnh đề \textbf{sai} trong các mệnh đề sau
% 	\choice
% 	{Phương trình có đúng một nghiệm $x>3$}
% 	{\True Phương trình vô nghiệm trên khoảng $(0;1)$}
% 	{Phương trình có ít nhất hai nghiệm}
% 	{Phương trình vô nghiệm trên khoảng $(-1;0)$}
% 	\loigiai{
% 		Xét hàm số $f(x)=x^4-4x^3+1$.\\
% 		Hàm số liên tục trên $\mathbb{R}$.\\
% 		Ta có $f(0)=1>0$ và $f(1)=-2<0$.\\
% 		Do $f(0)\cdot f(1)=-2<0$ và hàm số liên tục trên đoạn $[0;1]$ nên phương trình $f(x)=0$ có ít nhất một nghiệm $x_0\in (0;1)$.\\
% 		Vậy mệnh đề \lq\lq Phương trình vô nghiệm trên khoảng $(0;1)$\rq\rq\, là sai.	
% 	}
% \end{ex}

% \begin{ex}%[DCHT Toán 11 - KNTT -Nguyễn Thành Nhân]%[1K5KG-7]
% 	Xét phương trình sau trên tập số thực $x^{2023}+x=a \quad(1)$. Chọn khẳng định đúng trong các khẳng định dưới đây. 
% 	\choice
% 	{Phương trình $(1)$ chỉ có nghiệm khi $a>0$}
% 	{Phương trình $(1)$ chỉ có nghiệm khi $a<0$}
% 	{Phương trình $(1)$ vô nghiệm khi $a\geq 0$}
% 	{\True Phương trình $(1)$ có nghiệm $\forall a\in \mathbb{R}$}
% 	\loigiai
% 	{Phương trình $x^{2023}+x=a \Leftrightarrow x^{2023}+x-a=0 \quad(2)$.\\
% 		Xét hàm số $f(x)=x^{2023}+x-1$ trên $\mathbb{R}$, ta có
% 		\begin{itemize}
% 			\item $\lim\limits_{x\to -\infty}f(x)=\lim\limits_{x\to -\infty}\left(x^{2023}+x-a\right)=-\infty$ nên tồn tại số $\alpha<0$ để $f(\alpha)<0$.
% 			\item $\lim\limits_{x\to +\infty}f(x)=\lim\limits_{x\to +\infty}\left(x^{2023}+x-a\right)=+\infty$ nên tồn tại số $\beta>0$ để $f(\beta)>0$.
% 		\end{itemize}
% 		Hàm số $f(x)$ liên tục trên $[\alpha;\beta]$, có $f(\alpha)\cdot f(\beta)<0$ nên luôn tồn tại $c\in(\alpha;\beta)$ sao cho $f(c)=0$.\\
% 		Vậy phương trình $f(x)=0$ luôn có nghiệm với mọi $a\in\mathbb{R}$.}
% \end{ex}

% \begin{ex}%[DCHT Toán 11 - KNTT -Nguyễn Thành Nhân]%[1K5KG-7]
% 	Tìm tất cả các giá trị của tham số $m$ sao cho phương trình $(m^2-5m+3)x^5-2x^2+1=0$ có ít nhất một nghiệm thuộc khoảng $(-1;0)$.
% 	\choice
% 	{$m\in\left(-\infty;1\right)\cup\left(4;+\infty\right)$}
% 	{$m\in\mathbb{R}$}
% 	{\True $m\in\left(1;4\right)$}
% 	{$m\in\left(-\infty;1 \right]\cup\left[4;+\infty \right) $}
% 	\loigiai
% 	{
% 		Đặt $f(x)=(m^2-5m+3)x^5-2x^2+1$.\\
% 		Phương trình $(m^2-5m+3)x^5-2x^2+1=0$ có ít nhất một nghiệm thuộc khoảng $(-1;0)$
% 		\begin{eqnarray*}
% 			\Leftrightarrow f(-1)\cdot f(0)<0\Leftrightarrow (-m^2+5m-4)\cdot 1<0\Leftrightarrow 1<m<4.		
% 		\end{eqnarray*}
% 		Vậy $1<m<4$ thỏa yêu cầu bài toán.
% 	}
% \end{ex}


% \begin{ex}%[DCHT Toán 11 - KNTT -Nguyễn Thành Nhân]%[1K5KG-7]
% 	Cho phương trình $2x^4-5x^2+x+1=0$\quad$(1)$. Tìm mệnh đề đúng trong các mệnh đề sau:
% 	\choice
% 	{Phương trình $(1)$ không có nghiệm trong khoảng $(-2;0)$}
% 	{\True Phương trình $(1)$ có ít nhất $2$ nghiệm trong khoảng $(0;2)$}
% 	{Phương trình $(1)$ không có nghiệm trong khoảng $(-1;1)$}
% 	{Phương trình $(1)$ chỉ có $1$ nghiệm trong khoảng $(-2;1)$}
% 	\loigiai{
% 		Xét hàm số $f(x)=2x^4-5x^2+x+1$. Đây là hàm đa thức nên liên tục trên $\mathbb{R}$, do đó nó liên tục trên các đoạn $\left[0;1\right]$ và $\left[1;2\right]$.\\
% 		Ta có:
% 		$f(0)=1$; $f(1)=-1$; $f(2)=15$.\\
% 		$f(0)\cdot f(1)<0$ nên phương trình $f(x)=0$ có ít nhất một nghiệm $x_1\in(0;1)$\hfill$(1)$\\
% 		$f(1)\cdot f(2)<0$ nên phương trình $f(x)=0$ có ít nhất một nghiệm $x_2\in(1;2)$\hfill$(2)$\\
% 		Từ $(1)$ và $(2)$ suy ra phương trình $f(x)=0$ có ít nhất hai nghiệm trong khoảng $(0;2)$.
% 	}
% \end{ex}
% \begin{ex}%[DCHT Toán 11 - KNTT -Nguyễn Thành Nhân]%[1K5GG-7]
% 	Tập tất cả các giá trị của tham số thực $m$ để phương trình\\ $\left(2m^2-5m+2\right)\left(x-1\right)^{2021}\left(x^{2020}-2\right)+2x+3=0$ có nghiệm là
% 	\choice
% 	{$m\in \mathbb{R}\setminus \left\lbrace \dfrac{1}{2};2\right\rbrace$}
% 	{$m\in \left\lbrace \dfrac{1}{2};2\right\rbrace$}
% 	{$m\in \left(-\infty;\dfrac{1}{2}\right)\cup(2;+\infty)$}
% 	{\True $m\in \mathbb{R}$}
% 	\loigiai{
% 		Đặt $f(x)=\left(2m^2-5m+2\right)\left(x-1\right)^{2021}\left(x^{2020}-2\right)+2x+3$.
% 		\begin{itemize}
% 			\item Với $2m^2-5m+2=0\Leftrightarrow\hoac{&m=\dfrac{1}{2}\\&m=2}$ phương trình trên trở thành $2x+3=0\Leftrightarrow x=\dfrac{-3}{2}$.\\
% 			Vậy với $m\in \left\lbrace \dfrac{1}{2};2\right\rbrace$ phương trình có nghiệm $\qquad(*)$
% 			\item Với $2m^2-5m+2\ne0\Leftrightarrow m\in \mathbb{R}\setminus \left\lbrace \dfrac{1}{2};2\right\rbrace$, $f(x)$ là đa thức bậc lẻ có TXĐ $\mathscr{D}=\mathbb{R}$.\\
% 			Khi đó $f(x)$ liên tục trên $\mathbb{R}$.		
% 			\begin{itemize}
% 				\item Với $m\in\left(-\infty;\dfrac{1}{2}\right)\cap\left(2;+\infty\right)\Rightarrow 2m^2-5m+2>0$\\
% 				$\lim\limits_{x\rightarrow -\infty}f(x)=-\infty \Rightarrow \exists x_1<0\colon f(x_1)<0\qquad(1)$\\
% 				$\lim\limits_{x\rightarrow +\infty}f(x)=+\infty \Rightarrow \exists x_2>0\colon f(x_2)>0\qquad(2)$\\
% 				Từ $(1)$ và $(2)$ suy ra $f(x)$ luôn có ít nhất $1$ nghiệm $x_0\in\left(-\infty;+\infty\right)\qquad(**)$.
% 				\item Với $m\in\left(\dfrac{1}{2};2\right)\Rightarrow 2m^2-5m+2<0$\\
% 				$\lim\limits_{x\rightarrow -\infty}f(x)=+\infty \Rightarrow \exists x_3<0\colon f(x_3)>0\qquad(3)$\\
% 				$\lim\limits_{x\rightarrow +\infty}f(x)=-\infty \Rightarrow \exists x_4>0\colon f(x_4)<0\qquad(4)$\\
% 				Từ $(3)$ và $(4)$ suy ra $f(x)$ luôn có ít nhất $1$ nghiệm $x_0\in\left(-\infty;+\infty\right)\qquad(***)$.
% 			\end{itemize}
% 		\end{itemize}	
% 		Từ $(*),\;(**)$ và $(***)$ suy ra phương trình luôn có nghiệm với $m\in\mathbb{R}$.
% 	}
% \end{ex}
\Closesolutionfile{ans}
% \begin{indapan}{10}
% 	{ans/ans-1K5-3-Dang6}
% \end{indapan}

% %--------Lời giải chi tiết
\FULLWIDTH \hienLG \anDA 
% \setcounter{deso}{0}
% \chap{LỜI GIẢI CHI TIẾT}
% \setcounter{section}{0} \setcounter{ex}{0} \setcounter{dang}{0}
%Chương I
%%Bài 1. GTLG
% 
\section{Giá trị lượng giác của một góc lượng giác}
\subsection{Tóm tắt lý thuyết}
\begin{tomtat}
	\subsubsection{Khái niệm góc lượng giác và số đo của góc lượng giác}
	Trong mặt phẳng, cho hai tia $Ou$, $Ov$. Xét tia $Om$ cùng nằm trong mặt phẳng này. Nếu tia $Om$ quay quanh điểm $O$, theo một chiều nhất định từ $Ou$ đến $Ov$, thì ta nói nó quét một góc lượng giác với tia đầu  $Ou$, tia cuối $Ov$ và kí hiệu là ($Ou$, $Ov$).\\
	Mỗi góc lượng giác gốc $O$ được xác định bởi tia đầu $Ou$, tia cuối $Ov$ và số đo của nó.
	\begin{center}
		\begin{minipage}[H]{0.3\textwidth}
			\begin{tikzpicture}[scale=.7]	
				\draw (0,0) -- (4,0)node[below] {$u$};
				\draw[red] (0,0) -- (45:4)node[below right] {$v$};
				\draw[dashed,green!50!black] (0,0) -- (20:4)node[below right] {$m$};
				\draw[-stealth,red] (0:1) arc (0:45:1);
				\draw[-stealth] (2.75,1) arc (0:45:1);
				\path (30:3.5) node[below=-2pt]{$+$};
				\path (0:0) node[below left]{$O$};
			\end{tikzpicture}
		\end{minipage}
		\begin{minipage}[H]{0.3\textwidth}
			\begin{tikzpicture} [scale=.7]	
				\draw (0,0) -- (4,0)node[below] {$u$};
				\draw[red] (0,0) -- (45:4)node[below right] {$v$};
				\draw[dashed,green!50!black] (0,0) -- (75:3.5)node[below right] {$m$};
				\draw[red,-stealth,smooth,samples=100] plot[domain =0:2.25*pi]({.5*(1.1)^(\x) *cos(\x r)},{.5*(1.1)^(\x) *sin(\x r)});
				\draw[-stealth] (0.5,2) arc (75:110:2);
				\path (90:2.5) node[below=-2pt]{$+$};
				\path (0:0) node[below left]{$O$};
			\end{tikzpicture}
		\end{minipage}
		\begin{minipage}[H]{0.3\textwidth}
			\begin{tikzpicture}[scale=.7]		
				\draw (0,0) -- (4,0)node[below] {$u$};
				\draw[red] (0,0) -- (45:4)node[below right] {$v$};
				\draw[dashed,green!50!black] (0,0) -- (120:3)node[above right] {$m$};
				\draw[-stealth,red] (0:.8) arc (0:-315:.8);
				\draw[-stealth] (-1,1.7) arc (120:60:1);
				\path (105:2.4) node[below=-2pt]{$-$};
				\path (0:0) node[below left]{$O$};
			\end{tikzpicture}
		\end{minipage}
	\end{center}
	\subsubsection{Hệ thức Chasles}
	\immini{Hệ thức Chasles: Với ba tia $Ou$, $Ov$, $Ow$ bất kì, ta có 	
	$$
		\text{sđ}(Ou, Ov)+\text{sđ}(Ov,Ow)=\text{sđ}(Ou,Ow)+k 360^{\circ}(k \in \mathbb{Z}). 
		$$}
	{\begin{tikzpicture}[scale=0.77, font=\footnotesize, line join=round, line cap=round, >=stealth]		
			\draw (0,0) -- (4,0)node[below] {$u$};
			\draw[red] (0,0) -- (75:3.5)node[below right] {$w$};
			\draw[green!50!black] (0,0) -- (30:4)node[below right] {$v$};
			\draw[-stealth,red] (0:1) arc (0:-285:1);
			\draw[-stealth] (0:.9) arc (0:30:.9);
			\draw[-stealth] (.86,.5) arc (30:75:1);
			\path (0:0) node[below left]{$O$};
	\end{tikzpicture}}
	% Nhận xét. Từ hệ thức Chasles, ta suy ra:
	% Với ba tia tuỳ ý $Ox$, $Ou$, $Ov$ ta có
	% $$
	% \text{sđ}(Ou, Ov)=\text{sđ}(Ox, Ov)-\text{sđ}(Ox,Ou)+k360^{\circ}(k \in \mathbb{Z}). 
	% $$
	% Hệ thức này đóng vai trò quan trọng trong việc tính toán số đo của góc lượng giác.
	\subsubsection{Đơn vị đo góc và cung tròn}
	\textbf{Đơn vị độ}: Góc $1^{\circ}$ bằng $\dfrac{1}{180}$ góc bẹt.\\
	Đơn vị độ được chia thành những đơn vị nhỏ hơn: $1^{\circ}=60'; 1'=60"$.\\
	% Đối với các góc lượng giác, khi mà số vòng quay trong chuyển động tương ứng từ tia đầu đến tia cuối là khá lớn thì số đo của chúng tính bằng độ sẽ trở nên cồng kềnh. Do đó, trong khoa học và kĩ thuật, bên cạnh việc đo bằng độ, người ta còn sử dụng đơn vị đo góc bằng rađian.\\
	\immini{\textbf{Đơn vị rađian}: Cho đường tròn $(O)$ tâm $O$, bán kính $R$ và một cung $AB$ trên $(O)$.
		Ta nói cung tròn $AB$ có số đo bằng 1 rađian nếu độ dài của nó đúng bằng bán kính $R$.
		Khi đó ta cũng nói rằng góc $AOB$ có số đo bằng 1 rađian và viết: $\overset\frown{AOB}=1$ rad.}
	{\begin{tikzpicture}[scale=0.77, font=\footnotesize, line join=round, line cap=round, >=stealth]		
			\draw[green!50!black] (30:2) arc (30:-270:2);
			\draw[red] (30:2) arc (30:90:2);
			\draw (90:2)node[above]{$B$}--(0:0)--(30:2)node[above right]{$A$};
			\path (0:0) node[below left]{$O$};
			\draw(60:2) node[above right]{1 rad};
	\end{tikzpicture}}
	\textbf{Quan hệ giữa độ và rađian:}
	$$
	1 \text{ góc bẹt }=180^\circ = 1 \mathrm{rad} \Leftrightarrow  1^\circ=\dfrac{\pi}{180} \mathrm{rad} \quad \text { và }\quad 1\,  \mathrm{rad}=\left(\dfrac{180}{\pi}\right)^\circ.
	$$
	\begin{note}
		Khi viết số đo của một góc theo đơn vị rađian, người ta thường không viết chữ rad sau số đo. Chẳng hạn góc $\dfrac{\pi}{2}$ được hiểu là góc $\dfrac{\pi}{2}$ rad.
	\end{note}
	\begin{note}
		Dưới đây là bảng tương ứng giữa số đo bằng độ và số đo bằng rađian của các góc đặc biệt trong phạm vi từ $0^{\circ}$ đến $180^{\circ}$.
	\end{note}
	\begin{center}
		\renewcommand{\arraystretch}{2}
		\begin{tabular}{|l|c|c|c|c|c|c|c|c|c|}
			\hline Độ & $0^{\circ}$ & $30^{\circ}$ & $45^{\circ}$ & $60^{\circ}$ & $90^{\circ}$ & $120^{\circ}$ & $135^{\circ}$ & $150^{\circ}$ & $180^{\circ}$ \\
			\hline Rađian & 0 & $\dfrac{\pi}{6}$ & $\dfrac{\pi}{4}$ & $\dfrac{\pi}{3}$ & $\dfrac{\pi}{2}$ & $\dfrac{2 \pi}{3}$ & $\dfrac{3 \pi}{4}$ & $\dfrac{5 \pi}{6}$ & $\pi$ \\
			\hline
		\end{tabular}
	\end{center}
	\subsubsection{Độ dài cung tròn}
	Một cung của đường tròn bán kính $R$ và có số đo $\alpha$ rad thì có độ dài $l=R \alpha$.
	\subsubsection{Đường tròn lượng giác}
	\immini{\begin{itemize}
			\item Đường tròn lượng giác là đường tròn có tâm tại gốc toạ độ, bán kính bằng $1$, được định hướng và lấy điểm $A(1 ; 0)$ làm điểm gốc của đường tròn.
			\item Điểm trên đường tròn lượng giác biểu diễn góc lượng giác có số đo $\alpha$ là điểm $M$ trên đường tròn lượng giác sao cho sđ$(OA, OM)=\alpha$.
	\end{itemize}
	\begin{note}
		Góc $\alpha$ và $\beta$ có chung điểm biểu diễn khi \fbox{$\alpha - \beta = k2\pi$} (chẵn lần $\pi$)
		\end{note}}
	{
		\begin{tikzpicture}[line join = round, line cap = round, >=stealth, font=\footnotesize, scale=0.6]
			\tikzset{label style/.style={font=\footnotesize}}
			\path (0,0) coordinate (O)
			(3,0) coordinate (A)
			(0,3) coordinate (B)
			(0,-3) coordinate (B')
			(-3,0) coordinate (A')
			(0:0)++(150:3) coordinate (M)
			($(O)!(M)!(A')$) coordinate (H)
			($(O)!(M)!(B)$) coordinate (K)
			;
			\draw[->] (-4,0) -- (4,0) node[above,blue]{$x$};
			\draw[->] (0,-4) -- (0.,4) node[left,blue]{$y$};
			\draw[orange] (O) circle (3cm);
			\draw[rotate=0,->,green!50!black] (0.5,0) arc (0:150:0.5cm);
			\draw (0.35,0.25) node[above,blue] {$\alpha$};
			\draw[dashed] (H)--(M)--(K);
			\draw[green!50!black] (M)--(O);
			\foreach \p/\r in {A/-45,M/150,H/-90,O/-150,A'/-135,B'/-45,B/45,K/0}
			\fill (\p) circle (1pt) node[shift={(\r:3mm)},blue]{$\p$};
		\end{tikzpicture}
	}
	\subsubsection{Các giá trị lượng giác của góc lượng giác}
	\immini{Gọi $M(x;y)$ là điểm biểu diễn của góc lượng giác $\alpha$ trên đường tròn lượng giác. Khi đó, ta có:
		\begin{itemize}
			\item $\cos\alpha=x.$
			\item $\sin\alpha=y.$
			\item $\tan\alpha=\dfrac{\sin\alpha}{\cos\alpha}=\dfrac{y}{x} ~(x\neq0).$
		\item $\cot\alpha=\dfrac{\cos\alpha}{\sin\alpha}=\dfrac{x}{y} ~(y\neq0).$
	\end{itemize}}
	{\begin{tikzpicture}[line join = round, line cap = round, >=stealth, font=\footnotesize, scale=0.6]
			\tikzset{label style/.style={font=\footnotesize}}
			\path (0,0) coordinate (O)
			(3,0) coordinate (A)
			(0,3) coordinate (B)
			(0,-3) coordinate (B')
			(-3,0) coordinate (A')
			(0:0)++(40:3) coordinate (M)
			($(O)!(M)!(A')$) coordinate (H)
			($(O)!(M)!(B)$) coordinate (K)
			;
			\draw[->] (-4,0) -- (4,0) node[above,blue]{$x$};
			\draw[->] (0,-4) -- (0.,4) node[left,blue]{$y$};
			\draw[orange] (O) circle (3cm);
			\draw[rotate=0,->,green!50!black] (0.7,0) arc (0:40:0.7cm);
			\draw (1,0) node[above,blue] {$\alpha$};
			\draw[dashed] (H)--(M)--(K);
			\draw[green!50!black] (M)--(O);
			\draw[blue,fill=black] (0,2) node[left]{$\sin\alpha$}(2,0) circle(1pt) node[below]{$\cos\alpha$}(3,2.3) node{$M(x;y)$};
			\foreach \p/\r in {A/-45,O/-135,A'/-135,B'/-45,B/45}
			\fill (\p) circle (1pt) node[shift={(\r:3mm)},blue]{$\p$};
	\end{tikzpicture}}
	\begin{note}
		a) Ta còn gọi trục tung là trục sin, trục hoành là trục côsin.\\
		b) Từ định nghĩa ta suy ra:
		\begin{itemize}
			\item $\sin\alpha$, $\cos\alpha$ xác định với mọi giá trị của $\alpha$ và ta có:
			$$-1\leq \sin\alpha\leq 1; \quad -1\leq \cos\alpha\leq 1; \quad \sin(\alpha+k2\pi)=\sin\alpha;\quad \cos(\alpha+k2\pi)=\cos\alpha\,\, (k\in\mathbb{Z}).$$
			\item $\tan\alpha$ xác định khi $\alpha\neq\dfrac{\pi}{2}+k\pi\,\,  (k\in\mathbb{Z})$.
			\item $\cot\alpha$ xác định khi $\alpha\neq k\pi\,\,  (k\in\mathbb{Z})$.
			\item Dấu của các giá trị lượng giác của một góc lượng giác phụ thuộc vào vị trí điểm biểu diễn $M$ trên đường tròn lượng giác.
		\end{itemize}
	\end{note}
	\begin{minipage}[h]{0.6\textwidth}
		\begin{tabular}{c|c|c|c|c|}
			\cline{2-5}
			& \multicolumn{4}{c|}{Góc phần tư} \\ \hline
			\multicolumn{1}{|c|}{Giá trị lượng giác} & I     & II     & III     & IV    \\ \hline
			\multicolumn{1}{|c|}{$\sin \alpha$}     &   $+$    &  $ +$      &    $-$    &   $-$  \\ \hline
			\multicolumn{1}{|c|}{$\cos \alpha$}     &   $+$    &  $ -$      &    $-$    &   $+$  \\ \hline
			\multicolumn{1}{|c|}{$\tan \alpha$}     &   $+$    &  $ -$      &    $+$    &   $-$  \\ \hline
			\multicolumn{1}{|c|}{$\cot \alpha$}     &   $+$    &  $ -$      &    $+$    &   $-$  \\ \hline
		\end{tabular}
	\end{minipage}
	\begin{minipage}[h]{0.6\textwidth}
		\begin{tikzpicture}[line join = round, line cap = round, >=stealth, font=\footnotesize, scale=0.6]
			\tikzset{label style/.style={font=\footnotesize}}
			\path (0,0) coordinate (O)
			(3,0) coordinate (A)
			(0,3) coordinate (B)
			(0,-3) coordinate (B')
			(-3,0) coordinate (A')
			(0:0)++(-60:3) coordinate (M)
			($(O)!(M)!(A')$) coordinate (H)
			($(O)!(M)!(B)$) coordinate (K)
			;
			\draw[->] (-4,0) -- (4,0) node[above,blue]{$x$};
			\draw[->] (0,-4) -- (0.,4) node[left,blue]{$y$};
			\draw[orange] (O) circle (3cm);
			\draw[rotate=0,->,green!50!black] (0.5,0) arc (0:-60:0.5cm);
			\draw (0.75,-0.35) node[blue] {$\alpha$};
			\draw[dashed] (H)--(M)--(K);
			\draw[green!50!black] (M)--(O);
			\draw[blue] (2.5,2.5) node{$I$}(-2.5,2.5) node{$II$}(-2.5,-2.5) node{$III$}(2.5,-2.5) node{$IV$};
			\foreach \p/\r in {A/-45,M/-60,H/90,O/-150,A'/-135,B'/-45,B/45,K/180}
			\fill (\p) circle (1pt) node[shift={(\r:3mm)},blue]{$\p$};
		\end{tikzpicture}
	\end{minipage}
	% \subsubsection{Giá trị lượng giác của các góc đặc biệt}
	% \begin{center}
	% 	\renewcommand{\arraystretch}{2}
	% 	\begin{tabular}{|c|c|c|c|c|c|}
	% 		\hline
	% 		\multirow{2}{*}{Góc $\alpha$} & $0$              & $\dfrac{\pi}{6}$  & $\dfrac{\pi}{4}$  & $\dfrac{\pi}{3}$  & $\dfrac{\pi}{2}$              \\ \cline{2-6} 
	% 		& $0^\circ$              & $30^\circ$  & $45^\circ$  & $60^\circ$  & $90^\circ$             \\ \hline
	% 		$\sin\alpha$                  & $0$             & $\dfrac{1}{2}$ & $\dfrac{\sqrt{2}}{2}$ & $\dfrac{\sqrt{3}}{2}$ & 1              \\ \hline
	% 		$\cos\alpha$                  & $1$             & $\dfrac{\sqrt{3}}{2}$ & $\dfrac{\sqrt{2}}{2}$ & $\dfrac{1}{2}$ & 0              \\ \hline
	% 		$\tan\alpha$                 & $0$             & $\dfrac{1}{\sqrt{3}}$ & 1 & $\sqrt{3}$ & Không xác định \\ \hline
	% 		$\cot\alpha$                  & Không xác định & $\sqrt{3}$ & 1 & $\dfrac{1}{\sqrt{3}}$ & 0              \\ \hline
	% 	\end{tabular}
	% \end{center}
	\subsubsection{Các công thức lượng giác cơ bản}
	Đối với các giá trị lượng giác, ta có các hệ thức cơ bản sau
	\begin{enumEX}[$\bullet$]{2}
		\item $\sin^2 \alpha  + \cos^2 \alpha =1$
		\item $ 1+ \tan^2 \alpha= \dfrac{1}{\cos^2 \alpha}$ $\left(\alpha \neq \dfrac{\pi}{2}+k\pi , k\in \mathbb{Z}\right)$
		\item $ 1+ \cot^2 \alpha= \dfrac{1}{\sin^2 \alpha}$ $\left(\alpha \neq k\pi , k\in \mathbb{Z}\right)$
		\item $\tan \alpha \cdot \cot \alpha =1 $ $\left(\alpha \neq \dfrac{k\pi}{2}, k\in \mathbb{Z}\right)$
	\end{enumEX}
	\newpage
	\subsubsection{Giá trị lượng giác của các góc có liên quan đặc biệt}
	\begin{enumerate}
		\item Góc đối nhau ($\alpha$ và $-\alpha$)
		\immini{\begin{itemize}
				\item $\cos (-\alpha)=\cos \alpha$
				\item $\sin (-\alpha) =-\sin \alpha$
				\item $\tan (-\alpha) =-\tan \alpha$
				\item $\cot (-\alpha) =-\cot \alpha$
		\end{itemize}}
		{\vspace*{-1cm}\begin{tikzpicture}[line join = round, line cap = round, >=stealth, font=\footnotesize, scale=0.5]
				\tikzset{label style/.style={font=\footnotesize}}
				\path (0,0) coordinate (O)
				(3,0) coordinate (A)
				(0:0)++(120:3) coordinate (M)
				(0:0)++(-120:3) coordinate (N)
				(0,4) coordinate (C)
				(0,-4) coordinate (D)
				($(O)!(M)!(C)$) coordinate (E)
				($(O)!(N)!(D)$) coordinate (F)
				;
				\draw[->] (-4,0) -- (4,0) node[above,blue]{$x$};
				\draw[->] (0,-4) -- (0.,4) node[left,blue]{$y$};
				\draw[orange] (O) circle (3cm);
				\draw[rotate=0,->,red] (0.5,0) arc (0:120:0.5cm);
				\draw[rotate=0,->,green!50!black] (0.6,0) arc (0:-120:0.6cm);
				\draw (0,0) node[above right=2pt,blue] {$\alpha$} (0,-0) node[below right=2pt,blue]{$-\alpha$};
				\draw[dashed] (E)--(M)--(N)--(F);
				\draw[green!50!black] (M)--(O);
				\draw[red] (N)--(O);
				\foreach \p/\r in {A/-45,M/120,N/-120,O/-150}
				\fill (\p) circle (1pt) node[shift={(\r:3mm)},blue]{$\p$};
		\end{tikzpicture}}
		\item Góc bù nhau ($\alpha$ và $\pi-\alpha$)
		\immini{\begin{itemize}
				\item $\sin (\pi -\alpha)=\sin \alpha$
				\item $\cos (\pi -\alpha) =-\cos \alpha$
				\item $\tan (\pi -\alpha) =-\tan \alpha$
				\item $\cot (\pi -\alpha) =-\cot \alpha$
		\end{itemize}}
		{\vspace*{-0.5cm}\begin{tikzpicture}[line join = round, line cap = round, >=stealth, font=\footnotesize, scale=0.5]
				\tikzset{label style/.style={font=\footnotesize}}
				\path (0,0) coordinate (O)
				(3,0) coordinate (A)
				(0:0)++(30:3) coordinate (M)
				(0:0)++(150:3) coordinate (N)
				(4,0) coordinate (C)
				(-4,0) coordinate (D)
				($(O)!(M)!(C)$) coordinate (E)
				($(O)!(N)!(D)$) coordinate (F)
				;
				\draw[->] (-4,0) -- (4,0) node[above,blue]{$x$};
				\draw[->] (0,-4) -- (0.,4) node[left,blue]{$y$};
				\draw[orange] (O) circle (3cm);
				\draw[rotate=0,->,red] (0.5,0) arc (0:150:0.5cm);
				\draw[rotate=0,->,green!50!black] (1.6,0) arc (0:30:1.6cm);
				\draw (2,0) node[above,blue] {$\alpha$} (0.3,1.5) node[below,blue]{$\pi-\alpha$};
				\draw[dashed] (E)--(M)--(N)--(F);
				\draw[red] (O)--(N);
				\draw[green!50!black] (O)--(M);
				\foreach \p/\r in {A/-45,M/30,N/150,O/-130}
				\fill (\p) circle (1pt) node[shift={(\r:3mm)},blue]{$\p$};
		\end{tikzpicture}}
		\item Góc phụ nhau ($\alpha$ và $\dfrac{\pi}{2}-\alpha$)
		\immini{\begin{itemize}
				\item $\sin \left( \dfrac{\pi}{2}-\alpha\right)=\cos \alpha$
				\item $\cos \left( \dfrac{\pi}{2}-\alpha\right)=\sin \alpha$
				\item $\tan \left( \dfrac{\pi}{2}-\alpha\right)=\cot \alpha$
				\item $\cot \left( \dfrac{\pi}{2}-\alpha\right)=\tan \alpha$
		\end{itemize}}
		{\vspace*{-0.5cm}\begin{tikzpicture}[line join = round, line cap = round, >=stealth, font=\footnotesize, scale=0.5]
				\tikzset{label style/.style={font=\footnotesize}}
				\path (0,0) coordinate (O)
				(3,0) coordinate (A)
				(0:0)++(20:3) coordinate (M)
				(0:0)++(70:3) coordinate (N)
				(0,4) coordinate (C)
				(4,0) coordinate (D)
				($(O)!(M)!(C)$) coordinate (E)
				($(O)!(N)!(D)$) coordinate (F)
				(2.82,0) coordinate (G)
				(0,2.82) coordinate (H)
				;
				\draw[->] (-4,0) -- (4,0) node[above,blue]{$x$};
				\draw[->] (0,-4) -- (0.,4) node[left,blue]{$y$};
				\draw[orange] (O) circle (3cm);
				\draw[rotate=0,->,red] (0.7,0) arc (0:70:0.7cm);
				\draw[rotate=0,->,green!50!black] (1.6,0) arc (0:20:1.6cm);
				\draw (2,0) node[above,blue] {$\alpha$} (1,-.2) node[below,blue]{$\frac{\pi}{2}-\alpha$};
				\draw[dashed] (E)--(M) (F)--(N) (G)--(M) (H)--(N);
				\draw[dashed] (-3,-3)--(3,3);
				\draw[->] (0.8,-.5)--(0.5,0.45);
				\draw[red] (O)--(N);
				\draw[green!50!black] (O)--(M);
				\foreach \p/\r in {A/-45,M/20,N/70,O/-220}
				\fill (\p) circle (1pt) node[shift={(\r:3mm)},blue]{$\p$};
		\end{tikzpicture}}
		\item Góc hơn kém $\pi$ ($\alpha$ và $\pi+\alpha$)
		\immini{\begin{itemize}
				\item $\sin (\pi +\alpha)=-\sin \alpha$
				\item $\cos (\pi +\alpha)=-\cos \alpha$
				\item $\tan (\pi +\alpha)=\tan \alpha$
				\item $\cot (\pi +\alpha)=\cot \alpha$
		\end{itemize}}
		{\vspace*{-0.5cm}\begin{tikzpicture}[line join = round, line cap = round, >=stealth, font=\footnotesize, scale=0.5]
				\tikzset{label style/.style={font=\footnotesize}}
				\path (0,0) coordinate (O)
				(3,0) coordinate (A)
				(0:0)++(60:3) coordinate (M)
				(0:0)++(240:3) coordinate (N)
				;
				\draw[->] (-4,0) -- (4,0) node[above,blue]{$x$};
				\draw[->] (0,-4) -- (0.,4) node[left,blue]{$y$};
				\draw[orange] (O) circle (3cm);
				\draw[rotate=0,->,red] (1.7,0) arc (0:240:1.7cm);
				\draw[rotate=0,->,green!50!black] (0.6,0) arc (0:60:0.6cm);
				\draw (1,0) node[above,blue] {$\alpha$};
				\draw (-1.2,1) node[below,blue,rotate=60]{$\pi+\alpha$};
				\draw[red] (O)--(N);
				\draw[green!50!black] (O)--(M);
				\foreach \p/\r in {A/-45,M/60,N/240,O/-150}
				\fill (\p) circle (1pt) node[shift={(\r:3mm)},blue]{$\p$};
		\end{tikzpicture}}
	\end{enumerate}
\end{tomtat}

% \foreach \i in {1,2,...,7} {\input{data/11KNTT/data1/1K1-2-\i.tex}}
%%Bài 2. CTLG
% %\chapter{Hàm số  lượng giác và phương trình lượng giác}
\setcounter{section}{1}
\section{Công thức lượng giác}
\subsection{Tóm tắt lý thuyết}
\begin{tomtat}
% 	\begin{center}
% 		\begin{tikzpicture}[scale = 2.5]
% 			\path (0,0) coordinate (O) (1.5,0) coordinate (x) (0,1.5) coordinate (y);
% 			\draw[thick,->] (-1.5,0)--(x);
% 			\draw[thick,->] (0,-1.5)--(y);
% 			\draw (O) circle (1);
% 			\path ($(O)+(55:1)$) coordinate (M) 
% 			($(O)+(30:1)$) coordinate (N);
% 			\path ($(O)!(M)!(x)$) coordinate (x_M)
% 			($(O)!(M)!(y)$) coordinate (y_M)
% 			($(O)!(N)!(x)$) coordinate (x_N)
% 			($(O)!(N)!(y)$) coordinate (y_N);
% 			\draw[dashed] (x_M)--(M)--(y_M) (x_N)--(N)--(y_N);
% 			\foreach \x/\g in {O/-135,x/-90,y/180,x_M/-90,x_N/-90,y_M/180,y_N/180}
% 			\fill ($(\g:1mm)+(\x)$) node {$\x$};
% 			\fill 	(M) circle (0.5pt)
% 			($(15:4mm)+(M)$) node {$M\left(x_M,y_M\right)$};
% 			\fill (N) circle (0.5pt)
% 			($(15:4mm)+(N)$) node {$N\left(x_N,y_N\right)$};
% 	\draw (M)--(O)--(N);		
% 	\draw pic[draw,,angle radius=6mm,->,red]{angle=x--O--M};
% 	\fill[red] (45:3mm) node {$\alpha$};
% 	\draw pic[red,draw,,angle radius=10mm,->]{angle=x--O--N};
% 	\fill[red] (15:5mm) node {$\beta$};
% 		\end{tikzpicture}
% 	\end{center}
% Trong mặt phẳng $Oxy$ cho hai điểm $M,N$ trên đường tròn lượng giác.\\ Đặt $\alpha = \text{sđ} (Ox,OM), \beta = \text{sđ} (Ox,ON)$, ta có $M(\cos \alpha,\sin \alpha)$ và $N(\cos \beta, \sin \beta)$. Khi đó ta tính được $\overrightarrow{OM}.\overrightarrow{ON}$ bằng hai cách
% \begin{align*}
% 	\overrightarrow{OM}.\overrightarrow{ON}&=\left|\overrightarrow{OM}\right|.\left|\overrightarrow{ON}\right|.\cos \left(\overrightarrow{OM},\overrightarrow{ON}\right) = \cos (\alpha-\beta),\\
% 	\overrightarrow{OM}.\overrightarrow{ON} &= x_Mx_N+y_My_N= \cos \alpha \cos\beta +\sin\alpha\sin\beta.
% \end{align*}
% Từ đó dẫn tới công thức
% \begin{align*}
% 	\cos (\alpha-\beta) = \cos \alpha \cos \beta + \sin \alpha\sin \beta \tag{$\star$}
% \end{align*}
% Tất cả các công thức trong bài học được xây dựng dựa trên công thức $(\star)$.\\
% Trong suốt bài học, khi không nói gì thêm, chỉ xét các góc lượng giác mà trong đó giá trị lượng giác được để cập có nghĩa. 
	\subsubsection{Công thức cộng}
	\begin{khung4}{Công thức cộng}
	\begin{tasks}[style=itemize](2)
		\task $\cos (a-b) = \cos a \cos b + \sin a\sin b$.
		\task $\cos (a+b) = \cos a \cos b - \sin a\sin b$.
		\task $\sin (a-b) = \sin a \cos b - \sin b \cos a$.
		\task $\sin (a+b) = \sin a \cos b + \sin b \cos a$.
		\task $\tan (a-b) = \dfrac{\tan a - \tan b}{1+\tan a \tan b}$.
		\task $\tan (a+b) = \dfrac{\tan a + \tan b}{1-\tan a \tan b}$.
	\end{tasks}
	\end{khung4}
	\subsubsection{Công thức nhân đôi}
	Công thức nhân đôi được xây dựng bằng cách thay $b=a$ trong công thức cộng.
	\begin{khung4}{Công thức nhân đôi}
		\begin{tasks}[style=itemize]
			\task $\sin 2a = 2\sin a \cos a$.
			\task $\cos 2a = \cos^2a-\sin^2a = 2\cos^2a-1 = 1-2\sin^2a$.
			\task $\tan 2a = \dfrac{2\tan a}{1-\tan^2a}$.
		\end{tasks}
		\end{khung4}
\begin{note}
	Từ công thức nhân đôi, ta có công thức hạ bậc:
\end{note}
	\begin{khung4}{Công thức hạ bậc}
		\begin{tasks}[style=itemize](3)
			\task $\sin^2a= \dfrac{1-\cos 2a}{2}$.
		\task $\cos^2a = \dfrac{1+\cos 2a}{2}$.
		\task $\tan^2a=\dfrac{1-\cos2a}{1+\cos 2a}$.
		\end{tasks}
		\end{khung4}

\begin{note}
	Áp dụng công thức cộng cho $3a = a +2a$, ta có công thức nhân ba:
\end{note}
	\begin{khung4}{Công thức nhân ba}
		\begin{tasks}[style=itemize](2)
			\task $\sin3a= 3\sin a -4\sin^3a$.
		\task $\cos3a= 4\cos^3a-3\cos a$.
		\task $\tan3a = \dfrac{3\tan a - \tan^3 a}{1-3\tan^2a}$.
		\end{tasks}
		\end{khung4}

	\subsubsection{Công thức biến đổi tích thành tổng}
	\begin{khung4}{Công thức tích thành tổng}
		\begin{tasks}[style=itemize]
			\task $\cos a \cos b = \dfrac{1}{2}\left[\cos (a-b) + \cos (a+b)\right]$.
		\task $\sin a \sin b = \dfrac{1}{2}\left[\cos (a-b)-\cos(a+b)\right]$.
		\task $\sin a \cos b = \dfrac{1}{2}\left[\sin (a-b)+\sin (a+b)\right]$.
		\end{tasks}
		\end{khung4}
	\subsubsection{Công thức biến đổi tổng thành tích}
	Công thức biến đổi tổng thành tích được xây dựng bằng cách $a=\dfrac{a+b}{2}, b = \dfrac{a-b}{2}$ trong công thức biến đổi tích thành tổng.
	\begin{khung4}{Công thức tổng thành tích}
		\begin{tasks}[style=itemize](2)
			\task $\cos a+ \cos b = 2\cos\dfrac{a+b}{2}\cos \dfrac{a-b}{2}$.
		\task $\cos a- \cos b = -2\sin\dfrac{a+b}{2}\sin \dfrac{a-b}{2}$.
		\task $\sin a+ \sin b = 2\sin\dfrac{a+b}{2}\cos \dfrac{a-b}{2}$.
		\task $\sin a -\sin b = 2\cos\dfrac{a+b}{2}\sin \dfrac{a-b}{2}$.
		\end{tasks}
		\end{khung4}
\end{tomtat}


% \foreach \i in {1,2,...,5} {\input{data/11KNTT/data1/1K1-2-\i.tex}}
%%Bài 3. HSLG
% \setcounter{section}{2}
\section{Hàm số lượng giác}
\subsection{Tóm tắt lý thuyết}
\begin{tomtat}
	% \subsubsection{Định nghĩa hàm số lượng giác}
	% \begin{dn}
	% 	\begin{itemize}
	% 		\item Hàm số sin $y=\sin x$ có tập xác định là $\mathbb{R}$.
	% 		\item Hàm số cos $y=\cos x$ có tập xác định là $\mathbb{R}$.
	% 		\item Hàm số tan  $y=\tan x$ có tập xác định là $\mathbb{R} \setminus\left\{\dfrac{\pi}{2}+k \pi \Big| k \in \mathbb{Z} \right\}$.
	% 		\item  Hàm số cot $y=\cot x$ có tập xác định là $\mathbb{R} \setminus \left\{k \pi \Big| k \in \mathbb{Z} \right\}$.
	% 	\end{itemize}
	% \end{dn}
	\subsubsection{Hàm số chẵn, hàm số lẻ}
	\begin{dn}
	Cho hàm số $y=f(x)$ có tập xác định là $\mathscr{D}$.
	\begin{itemize}
		\item Hàm số $f(x)$ được gọi là \textbf{hàm số chẵn} nếu $\forall x \in \mathscr{D}$ thì $-x \in \mathscr{D}$ và $f(-x)=f(x)$. Đồ thị của một hàm số chẵn nhận trục tung là trục đối xứng.
		\item Hàm số $f(x)$ được gọi là \textbf{hàm số lẻ} nếu $\forall x \in \mathscr{D}$ thì $-x \in \mathscr{D}$ và $f(-x)=-f(x)$. Đồ thị của một hàm số lẻ nhận gốc toạ độ là tâm đối xứng.
	\end{itemize}
	\end{dn}
	\subsubsection{Hàm số tuần hoàn}
	\begin{dn}
		Hàm số $y=f(x)$ có tập xác định $\mathscr{D}$ được gọi là \textbf{hàm số tuần hoàn} nếu tồn tại số $T \neq 0$ sao cho với mọi $x \in \mathscr{D}$ ta có:
		\begin{enumerate}[i)]
			\item $x+T \in \mathscr{D}$ và $x-T \in \mathscr{D}$;
			\item $f(x+T)=f(x)$.
		\end{enumerate}
		Số $T$ dương nhỏ nhất thỏa mãn các điều kiện trên (nếu có) được gọi là \textbf{chu kì} của hàm số tuần hoàn đó.
	\end{dn}
	\begin{nx}
		\
		\begin{itemize}
			\item  Các hàm số $y=\sin x$ và $y=\cos x$ tuần hoàn với chu kì $2 \pi$. Các hàm số $y=\tan x$ và $y=\cot x$ tuần hoàn với chu kì $\pi$.
		\end{itemize}
	\end{nx}
	\begin{note} 
		Tổng quát, người ta chứng minh được các hàm số $y=A \sin \omega x$ và $y=A \cos \omega x$ $(\omega>0)$ là những hàm số tuần hoàn với chu kì \fbox{$T=\dfrac{2 \pi}{\omega}$}.
	\end{note}
	\subsubsection{Đồ thị và tính chất của hàm số $y=\sin x$}
	\begin{tc}
		Hàm số $y=\sin x$:
		\begin{itemize}
			\item   Có tập xác định là $\mathbb{R}$ và tập giá trị là $[-1 ; 1]$;
			\item   Là hàm số lẻ và tuần hoàn với chu kì $2 \pi$;
			\item    Đồng biến trên mỗi khoảng $\left(-\dfrac{\pi}{2}+k 2 \pi ; \dfrac{\pi}{2}+k 2 \pi\right)$ và nghịch biến trên mỗi khoảng \\
			$\left(\dfrac{\pi}{2}+k 2 \pi ; \dfrac{3 \pi}{2}+k 2 \pi\right)$, $k \in \mathbb{Z}$;
			\item    Có đồ thị đối xứng qua gốc toạ độ và gọi là một \textbf{đường hình sin}.
		\end{itemize}
	\begin{center}
		\begin{tikzpicture}[>=stealth,scale=0.7,transform shape] 
			\path
			({-2.5*pi},0) coordinate (X1)
			({-2*pi},0) coordinate (X2)
			({-1.5*pi},0) coordinate (X3)
			({-pi},0) coordinate (X4)
			({-0.5*pi},0) coordinate (X5)
			(0,0) coordinate (O)
			({0.5*pi},0) coordinate (X6)
			({pi},0) coordinate (X7)
			({1.5*pi},0) coordinate (X8)
			({2*pi},0) coordinate (X9)
			({2.5*pi},0) coordinate (X10)
			({-pi},-2) coordinate (A)
			({pi},-2) coordinate (B)
			;
			\draw[->] (-9.5,0) -- (9.5,0) node[below] {\small $x$};
			\draw[->] (0,-1.5) -- (0,1.8) node[right] {\small $y$};
			\draw [dotted] (X3)--({-1.5*pi},1)--({2.5*pi},1)--({2.5*pi},0)  ({0.5*pi},1)--({0.5*pi},0)
			(X1)--({-2.5*pi},-1)--({1.5*pi},-1)--({1.5*pi},0)  ({-0.5*pi},-1)--({-0.5*pi},0)
			({-pi},0) -- (A) ({pi},0) -- (B);
			\foreach \x/\g/\z in {X1/90/-\tfrac{5\pi}{2},X2/140/-2\pi,X3/-90/-\tfrac{3\pi}{2},X4/-135/-\pi,X5/90/-\tfrac{\pi}{2},X6/-90/\tfrac{\pi}{2},X7/60/\pi,X8/90/\tfrac{3\pi}{2},X9/-40/2\pi,X10/-90/\tfrac{5\pi}{2}} 
			\fill[black] (\x) circle(1pt) +(\g:5mm) node {$\z$};
			\draw [<->] ({-pi},-1.7)--({pi},-1.7) ; 
			\draw (0,0) node[below right]{$O$} (0,-1.7) node[below]{$T=2\pi$}
			(0,1) node[above right]{$1$} (0,-1) node[below right]{$-1$};
			\clip (-9.5,-1.4) rectangle (9.5,1.6) ;
			\draw[thick,samples=100,domain=-9.3:9.3] plot(\x,{sin((\x)*180/pi)});
			
		\end{tikzpicture}
	\end{center}
	\end{tc}
	\subsubsection{Đồ thị và tính chất của hàm số $y=\cos x$}
	\begin{tc}
		Hàm số $y=\cos x$:
		\begin{itemize}
			\item    Có tập xác định là $\mathbb{R}$ và tập giá trị là $[-1 ; 1]$;
			\item    Là hàm số chẵn và tuần hoàn với chu kì $2 \pi$;
			\item    Đồng biến trên mỗi khoảng $(-\pi+k 2 \pi ; k 2 \pi)$ và nghịch biến trên mỗi khoảng $(k 2 \pi ; \pi+k 2 \pi), k \in \mathbb{Z}$;
			\item    Có đồ thị là một đường hình sin đối xứng qua trục tung.
		\end{itemize}
	\begin{center}
		\begin{tikzpicture}[>=stealth,scale=0.77,transform shape] 
			\path
			({-2.5*pi},0) coordinate (X1)
			({-2*pi},0) coordinate (X2)
			({-1.5*pi},0) coordinate (X3)
			({-pi},0) coordinate (X4)
			({-0.5*pi},0) coordinate (X5)
			(0,0) coordinate (O)
			({0.5*pi},0) coordinate (X6)
			({pi},0) coordinate (X7)
			({1.5*pi},0) coordinate (X8)
			({2*pi},0) coordinate (X9)
			({2.5*pi},0) coordinate (X10)
			({-pi},-2) coordinate (A)
			({pi},-2) coordinate (B)
			;
			\draw[->] (-9.5,0) -- (9.5,0) node[below] {\small $x$};
			\draw[->] (0,-1.5) -- (0,1.8) node[right] {\small $y$};
			\draw [dotted] (X2)--({-2*pi},1)--({2*pi},1)--({2*pi},0) (X4)--({-pi},-1)--({pi},-1)--({pi},0) 
			({-pi},0) -- (A) ({pi},0) -- (B);
			\foreach \x/\g/\z in {X1/125/-\tfrac{5\pi}{2},X2/-90/-2\pi,X3/-120/-\tfrac{3\pi}{2},X4/90/-\pi,X5/110/-\tfrac{\pi}{2},X6/-120/\tfrac{\pi}{2},X7/90/\pi,X8/120/\tfrac{3\pi}{2},X9/-90/2\pi,X10/-120/\tfrac{5\pi}{2}} 
			\fill[black] (\x) circle(1pt) +(\g:5mm) node {$\z$};
			\draw [<->] ({-pi},-1.7)--({pi},-1.7) ; 
			\draw (0,0) node[below right]{$O$} (0,-1.7) node[below]{$T=2\pi$}
			(0,1) node[above right]{$1$} (0,-1) node[below right]{$-1$}
			;
			\clip (-9.5,-1.4) rectangle (9.5,1.6) ;
			\draw[thick,samples=100,domain=-9.3:9.3] plot(\x,{cos((\x)*180/pi)});
			
		\end{tikzpicture}
	\end{center}
	\end{tc}
	\subsubsection{Đồ thị và tính chất của hàm số $y=\tan x$}
	\begin{tc}
		Hàm số $y=\tan x$:
		\begin{itemize}
			\item    Có tập xác định là $\mathbb{R} \setminus\left\{\dfrac{\pi}{2}+k \pi \Big| k \in \mathbb{Z} \right\}$ và tập giá trị là $\mathbb{R}$;
			\item    Là hàm số lẻ và tuần hoàn với chu kì $\pi$;
			\item    Đồng biến trên mỗi khoảng $\left(-\dfrac{\pi}{2}+k \pi ; \dfrac{\pi}{2}+k \pi\right)$, $k \in \mathbb{Z}$;
			\item    Có đồ thị đối xứng qua gốc toạ độ.
		\end{itemize}
	\begin{center}
		\begin{tikzpicture}[>=stealth,scale=0.77,transform shape] 
			\path
			({-1.5*pi},0) coordinate (X1)
			({-pi},0) coordinate (X2)
			({-0.5*pi},0) coordinate (X3)
			({0.5*pi},0) coordinate (X4)
			({pi},0) coordinate (X5)
			({1.5*pi},0) coordinate (X6)
			;
			\draw[->] (-6.5,0) -- (6.5,0) node[below] {\small $x$};
			\draw[->] (0,-3.5) -- (0,3.5) node[right] {\small $y$};
			\draw [dashed] ({-3*pi/2},3.5)--({-3*pi/2},-3.5) ({-pi/2},3.5)--({-pi/2},-3.5) ({pi/2},3.5)--({pi/2},-3.5) ({3*pi/2},3.5)--({3*pi/2},-3.5) ;
			\foreach \x/\g/\z in {X1/-150/-\tfrac{3\pi}{2},X2/-40/-\pi,X3/-40/-\tfrac{\pi}{2},X4/-40/\tfrac{\pi}{2},X5/-400/\pi,X6/-40/\tfrac{3\pi}{2}} 
			\fill[black] (\x) circle(1pt) +(\g:5mm) node {$\z$};
			\draw (0,0) node[below right]{$O$};
			\clip (-6.5,-3.5) rectangle (6.5,3.5) ;
			\draw[thick,samples=100,domain={-pi/2+0.2}:{pi/2-0.2}] plot(\x,{tan((\x)*180/pi)});
			\draw[thick,samples=100,domain={pi/2+0.2}:{3*pi/2-0.2}] plot(\x,{tan((\x)*180/pi)});
			\draw[thick,samples=100,domain={-3*pi/2+0.2}:{-pi/2-0.2}] plot(\x,{tan((\x)*180/pi)});
		\end{tikzpicture}
	\end{center}
	\end{tc}
	\subsubsection{Đồ thị và tính chất của hàm số $y=\cot x$}
	\begin{tc}
		Hàm số $y=\cot x$:
		\begin{itemize}
			\item    Có tập xác định là $\mathbb{R} \setminus\{k \pi \mid k \in \mathbb{Z} \}$ và tập giá trị là $\mathbb{R}$; 
			\item    Là hàm số lẻ và tuần hoàn với chu kì $\pi$;
			\item    Nghịch biến trên mỗi khoảng $(k \pi ; \pi+k \pi), k \in \mathbb{Z}$;
			\item    Có đồ thị đối xứng qua gốc toạ độ.
		\end{itemize}
	\begin{center}
		\begin{tikzpicture}[>=stealth,scale=0.77,transform shape] 
			\path
			({-2*pi},0) coordinate (X1)
			({-1.5*pi},0) coordinate (X2)
			({-pi},0) coordinate (X3)
			({-0.5*pi},0) coordinate (X4)
			({0.5*pi},0) coordinate (X5)
			({pi},0) coordinate (X6)
			({1.5*pi},0) coordinate (X7)
			({2*pi},0) coordinate (X8)
			;
			\draw[->] (-7.5,0) -- (7.5,0) node[below] {\small $x$};
			\draw[->] (0,-3.5) -- (0,3.5) node[left] {\small $y$};
			\draw [dashed] ({-2*pi},3.5)--({-2*pi},-3.5) ({-pi},3.5)--({-pi},-3.5) ({pi},3.5)--({pi},-3.5) ({2*pi},3.5)--({2*pi},-3.5) ;
			\foreach \x/\g/\z in {X1/-150/-2\pi,X2/-130/-\tfrac{3\pi}{2},X3/-140/-\pi,X4/-100/-\tfrac{\pi}{2},X5/-100/\tfrac{\pi}{2},X6/-120/\pi,X7/-100/\tfrac{3\pi}{2}, X8/-120/2\pi}
			\fill[black] (\x) circle(1pt) +(\g:5mm) node {$\z$};
			\draw (0,0) node[below left]{$O$};
			\clip (-6.5,-3.5) rectangle (6.5,3.5) ;
			\draw[thick,samples=100,domain={0.2}:{pi-0.2}] plot(\x,{cot((\x)*180/pi)});
			\draw[thick,samples=100,domain={pi+0.2}:{2*pi-0.2}] plot(\x,{cot((\x)*180/pi)});
			\draw[thick,samples=100,domain={-0.2}:{-pi+0.2}] plot(\x,{cot((\x)*180/pi)});
			\draw[thick,samples=100,domain={-pi-0.2}:{-2*pi+0.2}] plot(\x,{cot((\x)*180/pi)});
		\end{tikzpicture}
	\end{center}
	\end{tc}
\end{tomtat}
\subsection{Các dạng toán thường gặp}

% \foreach \i in {1,2,...,5} {\input{data/11KNTT/data1/1K1-3-\i.tex}}
%%Bài 4. PTLG
% 
\setcounter{section}{3}
\section{Phương trình lượng giác cơ bản}
\subsection{Tóm tắt lý thuyết}
\begin{tomtat}
\subsubsection{Phương trình $\sin x=m$}
\begin{itemize}
	\item Với $|m|>1$ thì phương trình $\sin x=m$ vô nghiệm.
	\item Với $|m|\leq 1$, sẽ tồn tại duy nhất $\alpha \in \left[-\dfrac{\pi}{2}; \dfrac{\pi}{2}\right]$ thỏa mãn $\sin\alpha=m$. Khi đó
	\begin{center}
		$\sin x=m\Leftrightarrow\sin x=\sin\alpha\Leftrightarrow\hoac{&x=\alpha+k2\pi\\&x=\pi-\alpha+k2\pi}$ ($k\in \mathbb{Z}$).
	\end{center}
\item Nếu số đo của góc $\alpha$ được đo bằng đơn vị độ thì
\begin{center}
	$\sin x=\sin\alpha^\circ\Leftrightarrow\hoac{&x=\alpha^\circ+k360^\circ\\&x=180^\circ-\alpha^\circ+k360^\circ}$ ($k\in\mathbb{Z}$).
\end{center}
\item Tổng quát,
\begin{center}
	$\sin f(x)=\sin g(x)\Leftrightarrow\hoac{&f(x)=g(x)+k2\pi\\&f(x)=\pi - g(x)+k2\pi}$ ($k\in\mathbb{Z}$).
\end{center}
\item Một số trường hợp đặt biệt:
\begin{enumEX}[\faCheckCircleO]{1}
	\item $\sin x=0\Leftrightarrow x=k\pi$, $k\in\mathbb{Z}$.
	\item $\sin x=1\Leftrightarrow x=\dfrac{\pi}{2}+k2\pi$, $k\in\mathbb{Z}$.
	\item $\sin x=-1\Leftrightarrow x=-\dfrac{\pi}{2}+k2\pi$, $k\in \mathbb{Z}$.
\end{enumEX}
\end{itemize}
	\subsubsection{Phương trình $\cos x=m$}
\begin{itemize}
	\item Với $|m|>1$ thì phương trình $\cos x=m$ vô nghiệm.
	\item Với $|m|\leq 1$, sẽ tồn tại duy nhất $\alpha\in\left[0; \pi\right]$ thỏa mãn $\cos x=m$. Khi đó
	\begin{center}
		$\cos x=m\Leftrightarrow\cos x=\cos \alpha\Leftrightarrow\hoac{&x=\alpha+k2\pi\\&x=-\alpha+k2\pi}$ ($k\in\mathbb{Z}$).
	\end{center}
\item Nếu số đo của góc $\alpha$ được đo bằng đơn vị độ thì
\begin{center}
	$\cos x=\cos\alpha\Leftrightarrow\hoac{&x=\alpha^\circ+k360^\circ\\&x=-\alpha^\circ+k360^\circ}$ ($k\in\mathbb{Z}$).
\end{center}
\item Tổng quát,
\begin{center}
	$\cos f(x)=\cos g(x)\Leftrightarrow\hoac{&f(x)=g(x)+k2\pi\\&f(x)=-g(x)+k2\pi}$ ($k\in \mathbb{Z}$)
\end{center}
\item Một số trường hợp đặc biệt:
\begin{enumEX}[\faCheckCircleO]{1}
\item $\cos x=0\Leftrightarrow x=\dfrac{\pi}{2}+k\pi$, $k\in\mathbb{Z}$.
\item $\cos x=1\Leftrightarrow x=k2\pi$, $k\in\mathbb{Z}$.
\item $\cos x=-1\Leftrightarrow x=\pi+k2\pi$, $k\in\mathbb{Z}$.
\end{enumEX}
\end{itemize}
\subsubsection{Phương trình $\tan x=m$}
\begin{itemize}
	\item Với mọi $m\in\mathbb{R}$, tồn tại duy nhất $\alpha\in\left(-\dfrac{\pi}{2};\dfrac{\pi}{2}\right)$ thỏa mãn $\tan \alpha=m$. Khi đó
	\begin{center}
		$\tan x=m\Leftrightarrow\tan x=\tan \alpha\Leftrightarrow x=\alpha+k\pi$ ($k\in\mathbb{Z}$).
	\end{center}
\item Nếu số đo của góc $\alpha$ được đo bằng đơn vị độ thì
\begin{center}
	$\tan x=\tan\alpha^\circ\Leftrightarrow x=\alpha^\circ+k180^\circ$, $k\in \mathbb{Z}$
\end{center}
\item Tổng quát,
\begin{center}
	$\tan f(x)=\tan g(x)\Leftrightarrow f(x)=g(x)+k\pi$, $k\in\mathbb{Z}$.
\end{center}
\end{itemize}
\subsubsection{Phương trình $\cot x=m$}
\begin{itemize}
	\item Với mọi $m\in\mathbb{R}$, tồn tại duy nhất $\alpha\in\left(0;\pi\right)$ thỏa mãn $\cot\alpha=m$. Khi đó
	\begin{center}
		$\cot x=m\Leftrightarrow\cot x=\cot\alpha\Leftrightarrow x=\alpha+k\pi$ $k\in\mathbb{Z}$.
	\end{center}
\item Nếu số đo của góc $\alpha$ được đo bằng đơn vị độ thì
\begin{center}
	$\cot x=\cot\alpha^\circ\Leftrightarrow x=\alpha^\circ+k180^\circ$, $k\in\mathbb{Z}$.
\end{center}
\item Tổng quát,
\begin{center}
	$\cot f(x)=\cot g(x)\Leftrightarrow f(x)=g(x)+k\pi$, $k\in\mathbb{Z}$.
\end{center}
\end{itemize}
\end{tomtat}


% \foreach \i in {2,3,4,5,6,7,11} {\input{data/11KNTT/data1/1K1-4-\i.tex}}
%%Ôn tập chương I
% \newpage
% \section*{LÝ THUYẾT GÓC LƯỢNG GIÁC - GIÁ TRỊ - HÀM SỐ LƯỢNG GIÁC}
\boldmath
\subsection{GTLG Góc lượng giác}
\begin{itemize}
	\item \textbf{Đổi đơn vị đo}: \fbox{1 vòng = $360^\circ = 2\pi\ rad$}, \fbox{$180^\circ = \pi rad$}
	      \begin{center}
		      \renewcommand{\arraystretch}{2}
		      \begin{tabular}{|l|c|c|c|c|c|c|c|c|c|}
			      \hline Độ     & $0^{\circ}$ & $30^{\circ}$     & $45^{\circ}$     & $60^{\circ}$     & $90^{\circ}$     & $120^{\circ}$      & $135^{\circ}$      & $150^{\circ}$      & $180^{\circ}$ \\
			      \hline Rađian & 0           & $\dfrac{\pi}{6}$ & $\dfrac{\pi}{4}$ & $\dfrac{\pi}{3}$ & $\dfrac{\pi}{2}$ & $\dfrac{2 \pi}{3}$ & $\dfrac{3 \pi}{4}$ & $\dfrac{5 \pi}{6}$ & $\pi$         \\
			      \hline
		      \end{tabular}
	      \end{center}
	      \immini{
	\item \textbf{Độ dài cung tròn} bán kính $R$ số đo $\alpha$ rad là \fbox{$l=R\alpha$}.
	\item \textbf{Điểm biểu diễn góc lượng giác} $\alpha$ lên đường tròn lượng giác là $M$. Khi đó $M$ cũng biểu diễn các góc lượng giác $\alpha+k2\pi$.\\
	      Góc $\alpha$ và $\beta$ có chung điểm biểu diễn khi \fbox{$\alpha - \beta = k2\pi$} (chẵn lần $\pi$)}
	      {
	      \begin{tikzpicture}[line join = round, line cap = round, >=stealth, font=\footnotesize, scale=0.4]
		      \tikzset{label style/.style={font=\footnotesize}}
		      \path (0,0) coordinate (O)
		      (3,0) coordinate (A)
		      (0,3) coordinate (B)
		      (0,-3) coordinate (B')
		      (-3,0) coordinate (A')
		      (0:0)++(150:3) coordinate (M)
		      ($(O)!(M)!(A')$) coordinate (H)
		      ($(O)!(M)!(B)$) coordinate (K)
		      ;
		      \draw[->] (-4,0) -- (4,0) node[above,blue]{$x$};
		      \draw[->] (0,-4) -- (0.,4) node[left,blue]{$y$};
		      \draw[orange] (O) circle (3cm);
		      \draw[rotate=0,->,green!50!black] (0.5,0) arc (0:150:0.5cm);
		      \draw (0.35,0.25) node[above,blue] {$\alpha$};
		      \draw[dashed] (H)--(M)--(K);
		      \draw[green!50!black] (M)--(O);
		      \foreach \p/\r in {A/-45,M/150,O/-150,A'/-135,B'/-45,B/45}
		      \fill (\p) circle (1pt) node[shift={(\r:3mm)},blue]{$\p$};
	      \end{tikzpicture}
	      }
\end{itemize}
\begin{multicols}{2}
	\begin{khung4}{Định nghĩa GTLG}
		\immini{\begin{itemize}
				\item $\cos \alpha =x$
				\item $\sin \alpha = y$
				\item $\tan \alpha =\dfrac{\sin\alpha}{\cos\alpha}=\dfrac{y}{x}$
				\item $\cot\alpha=\dfrac{\cos\alpha}{\sin\alpha}=\dfrac{x}{y}$
			\end{itemize}}
		{\begin{tikzpicture}[line join = round, line cap = round, >=stealth, font=\footnotesize, scale=0.4]
				\tikzset{label style/.style={font=\footnotesize}}
				\path (0,0) coordinate (O)
				(3,0) coordinate (A)
				(0,3) coordinate (B)
				(0,-3) coordinate (B')
				(-3,0) coordinate (A')
				(0:0)++(40:3) coordinate (M)
				($(O)!(M)!(A')$) coordinate (H)
				($(O)!(M)!(B)$) coordinate (K)
				;
				\draw[->] (-4,0) -- (4,0) node[above,blue]{$x$};
				\draw[->] (0,-4) -- (0.,4) node[left,blue]{$y$};
				\draw[orange] (O) circle (3cm);
				\draw[rotate=0,->,green!50!black] (0.7,0) arc (0:40:0.7cm);
				\draw (1,0) node[above,blue] {$\alpha$};
				\draw[dashed] (H)--(M)--(K);
				\draw[green!50!black] (M)--(O);
				\draw[blue,fill=black] (0,2) node[left]{$\sin\alpha$}(2,0) circle(1pt) node[below]{$\cos\alpha$}(3,2.3) node{$M(x;y)$};
				\foreach \p/\r in {A/-45,O/-135,A'/-135,B'/-45,B/45}
				\fill (\p) circle (1pt) node[shift={(\r:3mm)},blue]{$\p$};
			\end{tikzpicture}}
	\end{khung4}
	\begin{khung4}{Các công thức lượng giác cơ bản}
		\begin{itemize}
			\item $\sin^2 \alpha  + \cos^2 \alpha =1$
			\item $ 1+ \tan^2 \alpha= \dfrac{1}{\cos^2 \alpha}$ $\left(\alpha \neq \dfrac{\pi}{2}+k\pi , k\in \mathbb{Z}\right)$
			\item $ 1+ \cot^2 \alpha= \dfrac{1}{\sin^2 \alpha}$ $\left(\alpha \neq k\pi , k\in \mathbb{Z}\right)$
			\item $\tan \alpha \cdot \cot \alpha =1 $ $\left(\alpha \neq \dfrac{k\pi}{2}, k\in \mathbb{Z}\right)$
		\end{itemize}
	\end{khung4}
\end{multicols}
\textit{\underline{Chú ý}: $\tan\alpha$ xác định khi $\alpha\neq\dfrac{\pi}{2}+k\pi\,\,  (k\in\mathbb{Z})$ và $\cot\alpha$ xác định khi $\alpha\neq k\pi\,\,  (k\in\mathbb{Z})$.}

\begin{multicols}{2}
	\begin{khung4}{$\cos$ đối}
		\immini{
			\begin{itemize}
				\item $\cos (-\alpha)=\cos \alpha$
				\item $\sin (-\alpha) =-\sin \alpha$
				\item $\tan (-\alpha) =-\tan \alpha$
				\item $\cot (-\alpha) =-\cot \alpha$
			\end{itemize}
		}{\begin{tikzpicture}[line join = round, line cap = round, >=stealth, font=\footnotesize, scale=0.4]
				\tikzset{label style/.style={font=\footnotesize}}
				\path (0,0) coordinate (O)
				(3,0) coordinate (A)
				(0:0)++(120:3) coordinate (M)
				(0:0)++(-120:3) coordinate (N)
				(0,4) coordinate (C)
				(0,-4) coordinate (D)
				($(O)!(M)!(C)$) coordinate (E)
				($(O)!(N)!(D)$) coordinate (F)
				;
				\draw[->] (-4,0) -- (4,0) node[above,blue]{$x$};
				\draw[->] (0,-4) -- (0.,4) node[left,blue]{$y$};
				\draw[orange] (O) circle (3cm);
				\draw[rotate=0,->,red] (0.5,0) arc (0:120:0.5cm);
				\draw[rotate=0,->,green!50!black] (0.6,0) arc (0:-120:0.6cm);
				\draw (0,0) node[above right=2pt,blue] {$\alpha$} (0,-0) node[below right=2pt,blue]{$-\alpha$};
				\draw[dashed] (E)--(M)--(N)--(F);
				\draw[green!50!black] (M)--(O);
				\draw[red] (N)--(O);
				\foreach \p/\r in {A/-45,M/120,N/-120,O/-150}
				\fill (\p) circle (1pt) node[shift={(\r:3mm)},blue]{$\p$};
			\end{tikzpicture}}
	\end{khung4}
	\begin{khung4}{phụ chéo}
		\immini{\begin{itemize}
				\item $\sin \left( \dfrac{\pi}{2}-\alpha\right)=\cos \alpha$
				\item $\cos \left( \dfrac{\pi}{2}-\alpha\right)=\sin \alpha$
				\item $\tan \left( \dfrac{\pi}{2}-\alpha\right)=\cot \alpha$
				\item $\cot \left( \dfrac{\pi}{2}-\alpha\right)=\tan \alpha$
			\end{itemize}}
		{\begin{tikzpicture}[line join = round, line cap = round, >=stealth, font=\footnotesize, scale=0.4]
				\tikzset{label style/.style={font=\footnotesize}}
				\path (0,0) coordinate (O)
				(3,0) coordinate (A)
				(0:0)++(20:3) coordinate (M)
				(0:0)++(70:3) coordinate (N)
				(0,4) coordinate (C)
				(4,0) coordinate (D)
				($(O)!(M)!(C)$) coordinate (E)
				($(O)!(N)!(D)$) coordinate (F)
				(2.82,0) coordinate (G)
				(0,2.82) coordinate (H)
				;
				\draw[->] (-4,0) -- (4,0) node[above,blue]{$x$};
				\draw[->] (0,-4) -- (0.,4) node[left,blue]{$y$};
				\draw[orange] (O) circle (3cm);
				\draw[rotate=0,->,red] (0.7,0) arc (0:70:0.7cm);
				\draw[rotate=0,->,green!50!black] (1.6,0) arc (0:20:1.6cm);
				\draw (2,0) node[above,blue] {$\alpha$} (-2,3.5) node[blue]{$\frac{\pi}{2}-\alpha$};
				\draw[dashed] (E)--(M) (F)--(N) (G)--(M) (H)--(N);
				\draw[dashed] (-3,-3)--(3,3);
				\draw[->] (-2,3)--(0.5,2);
				\draw[red] (O)--(N);
				\draw[green!50!black] (O)--(M);
				\foreach \p/\r in {A/-45,M/20,N/70,O/-220}
				\fill (\p) circle (1pt) node[shift={(\r:3mm)},blue]{$\p$};
			\end{tikzpicture}}
	\end{khung4}
	\begin{khung4}{$\sin$ bù}
		\immini{\begin{itemize}
				\item $\sin (\pi -\alpha)=\sin \alpha$
				\item $\cos (\pi -\alpha) =-\cos \alpha$
				\item $\tan (\pi -\alpha) =-\tan \alpha$
				\item $\cot (\pi -\alpha) =-\cot \alpha$
			\end{itemize}}
		{\begin{tikzpicture}[line join = round, line cap = round, >=stealth, font=\footnotesize, scale=0.4]
				\tikzset{label style/.style={font=\footnotesize}}
				\path (0,0) coordinate (O)
				(3,0) coordinate (A)
				(0:0)++(30:3) coordinate (M)
				(0:0)++(150:3) coordinate (N)
				(4,0) coordinate (C)
				(-4,0) coordinate (D)
				($(O)!(M)!(C)$) coordinate (E)
				($(O)!(N)!(D)$) coordinate (F)
				;
				\draw[->] (-4,0) -- (4,0) node[above,blue]{$x$};
				\draw[->] (0,-4) -- (0.,4) node[left,blue]{$y$};
				\draw[orange] (O) circle (3cm);
				\draw[rotate=0,->,red] (0.5,0) arc (0:150:0.5cm);
				\draw[rotate=0,->,green!50!black] (1.6,0) arc (0:30:1.6cm);
				\draw (2,0) node[above,blue] {$\alpha$} (0.3,1.5) node[below,blue]{$\pi-\alpha$};
				\draw[dashed] (E)--(M)--(N)--(F);
				\draw[red] (O)--(N);
				\draw[green!50!black] (O)--(M);
				\foreach \p/\r in {A/-45,M/30,N/150,O/-130}
				\fill (\p) circle (1pt) node[shift={(\r:3mm)},blue]{$\p$};
			\end{tikzpicture}}
	\end{khung4}
	\begin{khung4}{$\pm \pi\ \tan, \cot$}
		\immini{\begin{itemize}
				\item $\sin (\pi +\alpha)=-\sin \alpha$
				\item $\cos (\pi +\alpha)=-\cos \alpha$
				\item $\tan (\pi +\alpha)=\tan \alpha$
				\item $\cot (\pi +\alpha)=\cot \alpha$
			\end{itemize}}
		{\begin{tikzpicture}[line join = round, line cap = round, >=stealth, font=\footnotesize, scale=0.4]
				\tikzset{label style/.style={font=\footnotesize}}
				\path (0,0) coordinate (O)
				(3,0) coordinate (A)
				(0:0)++(60:3) coordinate (M)
				(0:0)++(240:3) coordinate (N)
				;
				\draw[->] (-4,0) -- (4,0) node[above,blue]{$x$};
				\draw[->] (0,-4) -- (0.,4) node[left,blue]{$y$};
				\draw[orange] (O) circle (3cm);
				\draw[rotate=0,->,red] (1.7,0) arc (0:240:1.7cm);
				\draw[rotate=0,->,green!50!black] (0.6,0) arc (0:60:0.6cm);
				\draw (1,0) node[above,blue] {$\alpha$};
				\draw (-1.2,1) node[below,blue,rotate=60]{$\pi+\alpha$};
				\draw[red] (O)--(N);
				\draw[green!50!black] (O)--(M);
				\foreach \p/\r in {A/-45,M/60,N/240,O/-150}
				\fill (\p) circle (1pt) node[shift={(\r:3mm)},blue]{$\p$};
			\end{tikzpicture}}
	\end{khung4}
\end{multicols}
\newpage

\subsection{Công thức lượng giác}
\subsubsection{Công thức cộng}
\begin{khung4}{Công thức cộng}
	\begin{multicols}{2}
	\begin{itemize}
		\item $\cos (a-b) = \cos a \cos b + \sin a\sin b$.
		\item $\cos (a+b) = \cos a \cos b - \sin a\sin b$.
		\item $\sin (a-b) = \sin a \cos b - \sin b \cos a$.
		\item $\sin (a+b) = \sin a \cos b + \sin b \cos a$.
		\item $\tan (a-b) = \dfrac{\tan a - \tan b}{1+\tan a \tan b}$.
		\item $\tan (a+b) = \dfrac{\tan a + \tan b}{1-\tan a \tan b}$.
	\end{itemize}
\end{multicols}
\end{khung4}
\begin{khung4}{Trường hợp đặc biệt}
	\begin{itemize}
		\item $\sin x + \cos x = \sqrt{2}\sin\left(x+\dfrac{\pi}{4}\right) = \sqrt{2}\cos \left(x-\dfrac{\pi}{4}\right)$.
		\item $\sqrt{3}\sin x + \cos x = 2\sin \left(x+\dfrac{\pi}{6}\right) = 2\cos \left(x-\dfrac{\pi}{3}\right)$.
		\item $\sin x + \sqrt{3}\cos x = 2\sin \left(x+ \dfrac{\pi}{3}\right) = 2\cos \left(x-\dfrac{\pi}{6}\right)$.
	\end{itemize}
\end{khung4}
\subsubsection{Công thức nhân đôi}
\begin{multicols}{2}
	\begin{khung4}{Công thức nhân đôi}
		\begin{itemize}
			\item $\sin 2a = 2\sin a \cos a$.
			\item $\cos 2a = \cos^2a-\sin^2a = 2\cos^2a-1 = 1-2\sin^2a$.
			\item $\tan 2a = \dfrac{2\tan a}{1-\tan^2a}$.
		\end{itemize}
	\end{khung4}
	\begin{khung4}{Công thức hạ bậc}
		\begin{itemize}
			\item $\sin^2a= \dfrac{1-\cos 2a}{2}$.
			\item $\cos^2a = \dfrac{1+\cos 2a}{2}$.
			\item $\tan^2a=\dfrac{1-\cos2a}{1+\cos 2a}$.
		\end{itemize}
	\end{khung4}
\end{multicols}

\begin{note}
	Áp dụng công thức cộng cho $3a = a +2a$, ta có công thức nhân ba:
\end{note}
\begin{khung4}{Công thức nhân ba}
	\begin{multicols}{2}
	\begin{itemize}
		\item $\sin3a= 3\sin a -4\sin^3a$.		
		\item $\cos3a= 4\cos^3a-3\cos a$.
		\item $\tan3a = \dfrac{3\tan a - \tan^3 a}{1-3\tan^2a}$.
	\end{itemize}
\end{multicols}
\end{khung4}

\subsubsection{Công thức biến đổi tích thành tổng}
\begin{khung4}{Công thức tích thành tổng}
	\begin{multicols}{2}
	\begin{itemize}
		\item $\cos a \cos b = \dfrac{1}{2}\left[\cos (a-b) + \cos (a+b)\right]$.
		\item $\sin a \sin b = \dfrac{1}{2}\left[\cos (a-b)-\cos(a+b)\right]$.
		\item $\sin a \cos b = \dfrac{1}{2}\left[\sin (a-b)+\sin (a+b)\right]$.
	\end{itemize}
	\end{multicols}
\end{khung4}
\subsubsection{Công thức biến đổi tổng thành tích}
Công thức biến đổi tổng thành tích được xây dựng bằng cách $a=\dfrac{a+b}{2}, b = \dfrac{a-b}{2}$ trong công thức biến đổi tích thành tổng.
\begin{khung4}{Công thức tổng thành tích}
	\begin{multicols}{2}
	\begin{itemize}
		\item $\cos a+ \cos b = 2\cos\dfrac{a+b}{2}\cos \dfrac{a-b}{2}$.
		\item $\cos a- \cos b = -2\sin\dfrac{a+b}{2}\sin \dfrac{a-b}{2}$.
		\item $\sin a+ \sin b = 2\sin\dfrac{a+b}{2}\cos \dfrac{a-b}{2}$.
		\item $\sin a -\sin b = 2\cos\dfrac{a+b}{2}\sin \dfrac{a-b}{2}$.
	\end{itemize}
\end{multicols}
\end{khung4}
\newpage

\subsection{Hàm số lượng giác}
\begin{khung4}{Hàm số chẵn, hàm số lẻ}
	\begin{multicols}{2}
		\begin{itemize}
			\item Hàm số $f(x)$ được gọi là \textbf{hàm số chẵn} nếu $\forall x \in \mathscr{D}$ thì $-x \in \mathscr{D}$ và $f(-x)=f(x)$. Đồ thị của một \textbf{hàm số chẵn} nhận \textbf{trục tung} là trục đối xứng.
			\item Hàm số $f(x)$ được gọi là \textbf{hàm số lẻ} nếu $\forall x \in \mathscr{D}$ thì $-x \in \mathscr{D}$ và $f(-x)=-f(x)$. Đồ thị của một \textbf{hàm số lẻ} nhận \textbf{gốc toạ độ} là tâm đối xứng.
		\end{itemize}
	\end{multicols}
	Các hàm số $y=\sin x$, $y=\tan x$, $y=\cot x$ là hàm số \textit{lẻ}, hàm số $y=\cos x$ là hàm số \textit{chẵn}.
\end{khung4}
\begin{khung4}{Hàm số tuần hoàn}
	\begin{dn}
		Hàm số $y=f(x)$ có tập xác định $\mathscr{D}$ được gọi là \textbf{hàm số tuần hoàn} nếu tồn tại số $T \neq 0$ sao cho với mọi $x \in \mathscr{D}$ ta có:
		\begin{itemize}
			\item $x+T \in \mathscr{D}$ và $x-T \in \mathscr{D}$;
			\item $f(x+T)=f(x)$.
		\end{itemize}
		Số $T$ dương nhỏ nhất thỏa mãn các điều kiện trên (nếu có) được gọi là \textbf{chu kì} của hàm số tuần hoàn đó.
	\end{dn}
	Các hàm số $y=A \sin \omega x$ và $y=A \cos \omega x$ $(\omega>0)$ là những hàm số tuần hoàn với chu kì $T=\dfrac{2 \pi}{\omega}$.\\
	Các hàm số $y=A \tan \omega x$ và $y=A \cot \omega x$ $(\omega>0)$ là những hàm số tuần hoàn với chu kì $T=\dfrac{\pi}{\omega}$.
\end{khung4}
\begin{center}
	\begin{tikzpicture}[line join = round, line cap = round, >=stealth, font=\small, thick, scale=.7]
		\path (0,0) coordinate (O)
		(3,0) coordinate (A)
		(0,3) coordinate (B)
		(0,-3) coordinate (B')
		(-3,0) coordinate (A');
		\draw[->] (-4,0) -- (4,0) node[above,blue]{$\cos$};
		\draw[->] (0,-4) -- (0.,3.5) node[left,blue]{$\sin$};
		\draw[orange] (O) circle (3cm);
		\draw[->,green!50!black] (40:3.5) arc (40:50:3.5cm) node[midway,above right] {(+)};
		\draw[->,green!50!black](5,0)node[align=center,rotate=90,below]{$\sin x$ đồng biến\\ trên $\left(-\dfrac{\pi}{2}+k2\pi;\dfrac{\pi}{2}+k2\pi\right)$}
		(-30:5) arc (-30:30:5cm) ;
		\draw[->,red!50!black](0,5)node[align=center,above]{$\cos x$ nghịch biến\\ trên $\left(k2\pi;\pi+k2\pi\right)$}
		(60:5) arc (60:120:5cm) ;
		\draw[->,red!50!black](-5,0)node[align=center,rotate=-90,below]{$\sin x$ nghịch biến\\ trên $\left(\dfrac{\pi}{2}+k2\pi;\dfrac{3\pi}{2}+k2\pi\right)$}
		(150:5) arc (150:210:5cm) ;

		\draw[->,green!50!black](0,-5)node[align=center,below]{$\cos x$ đồng biến\\ trên $\left(-\pi+k2\pi;k2\pi\right)$}
		(-120:5) arc (-120:-60:5cm) ;

		\foreach \p/\r in {A/-45,O/-150,A'/-135,B'/-45,B/45}
		\fill (\p) circle (1pt) node[shift={(\r:3mm)},blue]{$\p$};
	\end{tikzpicture}
\end{center}

\subsection{Phương trình lượng giác}
\begin{khung4}{Phương trình $\sin x = a$.}
	\begin{enumerate}[\faCheckSquareO]
		\item Trường hợp $a>1$ hoặc $a<-1$ phương trình vô nghiệm.
		\item Trường hợp $a \in \{-1;0;1\}$.
		      \begin{multicols}{3}
			      \begin{tikzpicture}[smooth,samples=300,scale=0.8,>=stealth]
				      \draw[->] (-1.5,0)--(1.5,0) node[below]{\footnotesize$cos$};
				      \draw[->] (0,-1.5)--(0,1.8) node[right]{\footnotesize$sin$};
				      \draw (0,0) node[below left]{\footnotesize$O$};
				      \tkzDefPoints{0/0/I}
				      \draw[orange] (I) circle(1cm);
				      \draw[fill,blue] (0,1) circle(2.5pt) node[above left] {$B$};
				      \node[below] at (0,-1.5) {\fbox{$\sin x=1 \Leftrightarrow x=\tfrac{\pi}{2}+k2\pi$}};
			      \end{tikzpicture}
			      \hspace{-0.4cm}
			      \begin{tikzpicture}[smooth,samples=300,scale=0.8,>=stealth]
				      \draw[->] (-1.5,0)--(1.5,0) node[below]{\footnotesize$cos$};
				      \draw[->] (0,-1.5)--(0,1.8) node[right]{\footnotesize$sin$};
				      \draw (0,0) node[below left]{\footnotesize$O$};
				      \tkzDefPoints{0/0/I}
				      \draw[orange] (I) circle(1cm);
				      \draw[fill,blue] (0,-1) circle(2.5pt)node[below left] {$B'$};
				      \node[below] at (0,-1.5) {\fbox{$\sin x=-1 \Leftrightarrow x=-\tfrac{\pi}{2}+k2\pi$}};
			      \end{tikzpicture}
			      \hspace{-0.4cm}
			      \begin{tikzpicture}[smooth,samples=300,scale=0.8,>=stealth]
				      \draw[->] (-1.5,0)--(1.5,0) node[below]{\footnotesize$cos$};
				      \draw[->] (0,-1.5)--(0,1.8) node[right]{\footnotesize$sin$};
				      \draw (0,0) node[below left]{\footnotesize$O$};
				      \tkzDefPoints{0/0/I}
				      \draw[orange] (I) circle(1cm);
				      \draw[fill,blue] (1,0) circle(2.5pt) (-1,0) circle(2.5pt) node[above right] at (1,0) {$A$};
				      \node[above left,blue] at (-1,0) {$A'$};
				      \node[below] at (0,-1.5) {\fbox{$\sin x=0 \Leftrightarrow x=k\pi$}};
			      \end{tikzpicture}
		      \end{multicols}
		\item Trường hợp $a \in \left\{\pm \dfrac{1}{2};\pm \dfrac{\sqrt{2}}{2};\pm \dfrac{\sqrt{3}}{2}\right\}$ hoặc $a \in (-1;1)$. Ta bấm máy \shiftk \sink để đổi tìm góc $\alpha$ hoặc $\beta^\circ$.
			      \immini{
				      \begin{listEX}[1]
					      \item [\ding{172}] Công thức theo đơn vị rad:
					      $\sin x = \sin \alpha \Leftrightarrow \hoac{&x=\alpha+k2\pi\\&x=\pi-\alpha+k2\pi}, \,k \in \mathbb{Z}$
					      \item [\ding{173}] Công thức theo đơn vị độ: $\sin x = \sin \beta^\circ \Leftrightarrow \hoac{&x=\beta^\circ+k360^\circ\\&x=180^\circ-\beta^\circ+k360^\circ}, \,k \in \mathbb{Z}$
				      \end{listEX}
			      }{\begin{tikzpicture}[smooth,samples=300,scale=1,>=stealth]
					      \draw[->] (-1.5,0)--(1.5,0);
					      \draw[->] (0,-1.3)--(0,1.5) node[right]{\footnotesize$sin$};
					      \draw (0,0) node[below left]{\footnotesize$O$};
					      \tkzDefPoints{0/0/I}
					      \draw[orange] (I) circle(1cm);
					      \coordinate (M) at ($(I)+(50:1cm)$);
					      \coordinate (N) at ($(I)+(130:1cm)$);
					      \tkzDrawPoints[size=3,fill=blue](M,N)
					      \tkzDrawSegments(I,M I,N)
					      \tkzDrawSegments[dashed](M,N)
					      \tkzLabelPoints[right,blue](M)
					      \tkzLabelPoints[left,blue](N)
					      \node[below right] at (0,0.7) {$a$};
				      \end{tikzpicture}
			      }
	\end{enumerate}
\end{khung4}
\begin{khung4}{Phương trình $\cos x = a$.}
\begin{enumerate}[\faCheckSquareO]
	\item Trường hợp $a>1$ hoặc $a<-1$ phương trình vô nghiệm.
	\item Trường hợp $a \in \{-1;0;1\}$.
	      \begin{multicols}{3}
		      \begin{tikzpicture}[smooth,samples=300,scale=0.8,>=stealth]
			      \draw[->] (-1.5,0)--(1.8,0) node[below]{\footnotesize$cos$};
			      \draw[->] (0,-1.5)--(0,1.8) node[right]{\footnotesize$sin$};
			      \draw (0,0) node[below left]{\footnotesize$O$};
			      \tkzDefPoints{0/0/I}
			      \draw[orange] (I) circle(1cm);
			      \draw[fill,blue] (1,0) circle(2.5pt)node[above right]  {$A$};
			      \node[below] at (0,-1.5) {\fbox{$\cos x=1 \Leftrightarrow x=k2\pi$}};
		      \end{tikzpicture}
		      \begin{tikzpicture}[smooth,samples=300,scale=0.8,>=stealth]
			      \draw[->] (-1.5,0)--(1.5,0) node[below]{\footnotesize$cos$};
			      \draw[->] (0,-1.8)--(0,1.5) node[right]{\footnotesize$sin$};
			      \draw (0,0) node[below left]{\footnotesize$O$};
			      \tkzDefPoints{0/0/I}
			      \draw[orange] (I) circle(1cm);
			      \draw[fill,blue] (-1,0) circle(2.5pt)node[below left] {$A'$};
			      \node[below] at (0,-1.5) {\fbox{$\cos x=-1 \Leftrightarrow x=\pi+k2\pi$}};
		      \end{tikzpicture}
		      \begin{tikzpicture}[smooth,samples=300,scale=0.8,>=stealth]
			      \draw[->] (-1.5,0)--(1.8,0) node[below]{\footnotesize$cos$};
			      \draw[->] (0,-1.5)--(0,1.5);
			      \draw (0,0) node[below left]{\footnotesize$O$};
			      \tkzDefPoints{0/0/I}
			      \draw[orange] (I) circle(1cm);
			      \draw[fill,blue] (0,1) circle(2.5pt) (0,-1) circle(2.5pt) node[above right] at (0,1) {$B$};
			      \node[below left,blue] at (0,-1) {$B'$};
			      \node[below] at (0,-1.5) {\fbox{$\cos x=0 \Leftrightarrow x=\frac{\pi}{2}+k\pi$}};
		      \end{tikzpicture}
	      \end{multicols}
	\item Trường hợp $a \in \left\{\pm \dfrac{1}{2};\pm \dfrac{\sqrt{2}}{2};\pm \dfrac{\sqrt{3}}{2}\right\}$ hoặc $a \in (-1;1)$. Ta bấm máy \shiftk \cosk để tìm góc $\alpha$ hoặc $\beta^\circ$ tương ứng.
	      \immini{
		      \begin{listEX}[1]
			      \item [\ding{172}] Công thức theo đơn vị rad:
			      $\cos x = \cos \alpha \Leftrightarrow \hoac{&x=\alpha+k2\pi\\&x=-\alpha+k2\pi}, \,k \in \mathbb{Z}$
			      \item [\ding{173}] Công thức theo đơn vị độ: $\cos x = \cos \beta^\circ \Leftrightarrow \hoac{&x=\beta^\circ+k360^\circ\\&x=-\beta^\circ+k360^\circ}, \,k \in \mathbb{Z}$
		      \end{listEX}
	      }{\begin{tikzpicture}[smooth,samples=300,scale=1,>=stealth]
			      \draw[->] (-1.5,0)--(1.7,0)node[above]{\footnotesize$cos$};
			      \draw[->] (0,-1.3)--(0,1.5);
			      \draw (0,0) node[below left]{\footnotesize$O$};
			      \tkzDefPoints{0/0/I}
			      \draw[orange] (I) circle(1cm);
			      \coordinate (M) at ($(I)+(50:1cm)$);
			      \coordinate (N) at ($(I)+(-50:1cm)$);
			      \tkzDrawPoints[size=3,fill=blue](M,N)
			      \tkzDrawSegments(I,M I,N)
			      \tkzDrawSegments[dashed](M,N)
			      \tkzLabelPoints[above,blue](M)
			      \tkzLabelPoints[below,blue](N)
			      \node[below right] at (0.7,0) {$a$};
		      \end{tikzpicture}}
\end{enumerate}
\end{khung4}
\begin{khung4}{Phương trình $\tan x = a$ và $\cot x = b$.}
\begin{enumerate}[\faCheckSquareO]
	\item Trường hợp $a \in \left\{0;\pm \dfrac{\sqrt{3}}{3};\pm 1; \pm \sqrt{3}\right\}$ hoặc $a$ bất kì. Ta bấm máy \shiftk \tank để tìm góc $\alpha$ hoặc $\beta^\circ$ tương ứng.
		      \immini{
			      \begin{listEX}[1]
				      \item [\ding{172}] Công thức theo đơn vị rad:
				      $$\tan x = \tan \alpha \Leftrightarrow x=\alpha+k\pi, \,k \in \mathbb{Z}$$
				      \item [\ding{173}] Công thức theo đơn vị độ:
				      $$\tan x = \tan \beta^\circ \Leftrightarrow x=\beta^\circ +k180^\circ, \,k \in \mathbb{Z}$$
			      \end{listEX}
		      }{\begin{tikzpicture}[smooth,samples=300,scale=1,>=stealth]
				      \draw[->] (-1.5,0)--(1.5,0);
				      \draw[->] (0,-1.3)--(0,1.5);
				      \draw (0,0) node[below right]{\footnotesize$O$};
				      \tkzDefPoints{0/0/I, 1/0.9/A}
				      \draw[orange] (I) circle(1cm);
				      \draw[->] (1,-1.3)--(1,1.5)node[right]{\footnotesize$tan$};
				      \tkzInterLC[R](I,A)(I,1cm)\tkzGetPoints{M}{N}
				      \tkzDrawPoints[size=3,fill=blue](I,M,N,A)
				      \tkzDrawSegments(A,N)
				      \tkzLabelPoints[below,font=\footnotesize,blue](M)
				      \tkzLabelPoints[above,font=\footnotesize,blue](N)
				      \node[right] at (1,0.9) {$a$};
			      \end{tikzpicture}}
\end{enumerate}

\textbf{$\bigstar$ Phương trình $\cot x = b$.}
	$b \in \left\{\pm \dfrac{\sqrt{3}}{3};\pm 1; \pm \sqrt{3}\right\}$ hoặc $b$ bất kì. Ta bấm máy \shiftk \tank \fbox{$\tfrac{1}{b}$} để tìm góc $\alpha$ hoặc $\beta^\circ$ tương ứng. Riêng $b=0$ thì $\alpha=\dfrac{\pi}{2}$. Công thức nghiệm tương tự phương trình $\tan x =a$
\end{khung4}
% 
% \section{BÀI TẬP ÔN TẬP CHƯƠNG I}
\Opensolutionfile{ans}[ans/ans-1K1-4-OTC]
\begin{ex}%[Câu 1]%[1K1Y1-1]
	Đổi $225^\circ$ sang rađian.
	\choice
	{$\dfrac{4\pi}{5}$}
	{$\dfrac{6\pi}{5}$}
	{$\dfrac{3\pi}{7}$}
	{\True $\dfrac{5\pi}{4}$}
	%<MyLT2>
	\loigiai{
		Ta có $225^\circ = \dfrac{225}{180}\pi=\dfrac{5\pi}{4}$ (rađian).
	}
\end{ex}
\begin{ex}%[Câu 2]%[1K1Y1-3]
	Một đường tròn có bán kính $R=10$ cm. Độ dài cung $40^\circ$ trên đường tròn gần bằng
	\choice
	{$11$ cm}
	{$13$ cm}
	{\True $7$ cm}
	{$9$ cm}
	\loigiai{
		Ta có $40^\circ = 40\cdot \dfrac{\pi}{180} =\dfrac{2\pi}{9}$ rađian.\\
		Độ dài cung $l=\dfrac{2\pi}{9}\cdot 10=\dfrac{20\pi}{9}\approx 7$ cm.
	}
\end{ex}
\begin{ex}%[Câu 3]%[1K1B1-4]
	Bánh xe của người đi xe đạp quay được $2$ vòng trong $6$ giây. Hỏi trong $1$ giây, bánh xe quay được bao nhiêu độ?
	\choice
	{$60^\circ$}
	{$72^\circ$}
	{$240^\circ$}
	{\True $120^\circ$}
	\loigiai{
		Trong $6$ giây, bánh xe quay được $2\cdot 360^\circ=720^\circ$.\\
		Trong $1$ giây, bánh xe quay được $720^\circ\colon 6=120^\circ$.
	}
\end{ex}
\begin{ex}%[Câu 4]%[1K1Y1-6]
	Cho góc $\alpha$ thỏa mãn $90^\circ <\alpha <180^\circ$. Khẳng định nào sau đây đúng?
	\choice
	{$\cos\alpha>0$}
	{\True $\sin\alpha>0$}
	{$\tan\alpha>0$}
	{$\cot\alpha>0$}
	\loigiai
	{Vì $90^\circ <\alpha <180^\circ$ nên $\sin\alpha>0$, $\cos\alpha<0$, $\tan\alpha<0$ và $\cot\alpha<0$.}
\end{ex}
\begin{ex}%[Câu 5]%[1K1B1-6]
	Cho $\sin \alpha =\dfrac{1}{3}$ và $\dfrac{\pi}{2}<\alpha<\pi$. Khi đó $\cos \alpha$ có giá trị là
	\choice
	{$\cos \alpha =-\dfrac{2}{3}$}
	{$\cos \alpha =\dfrac{2\sqrt{2}}{3}$}
	{$\cos \alpha =\dfrac{8}{9}$}
	{\True $\cos \alpha =-\dfrac{2\sqrt{2}}{3}$}
	\loigiai{
		Ta có $\cos^2 \alpha =1-\sin^2 \alpha =1-\left(\dfrac{1}{3}\right)^2=\dfrac{8}{9}$.\\
		Vì $\dfrac{\pi}{2}<\alpha<\pi$ nên $\cos \alpha <0$.\\
		Do đó $\cos \alpha =-\dfrac{2\sqrt{2}}{3}$.
	}
\end{ex}
\begin{ex}%[Câu 6]%[1K1B1-7]
	Cho $A$, $B$, $C$ là ba góc của tam giác $ABC$. Trong các khẳng định sau, khẳng định nào \textbf{sai}?
	\choice
	{$\sin (B+C)=\sin A$}
	{$\cos (B+C)=-\cos A$}
	{\True $\tan (B+C)=\tan A$}
	{$\cot (B+C)=-\cot A$}
	\loigiai{
		Ta có $B+C=180^\circ-A$.\\Suy ra $\tan(B+C)=\tan (180^\circ-A)=-\tan A$.
	}
\end{ex}
\begin{ex}%[Câu 7]%[1K1B1-8]
	Tính giá trị biểu thức $P=\cos ^2\dfrac{\pi}{8}+\cos ^2\dfrac{{3\pi}}{8}+\cos ^2\dfrac{{5\pi}}{8}+\cos ^2\dfrac{{7\pi}}{8}.$
	\choice
	{$P=-1$}
	{$P=0$}
	{$P=1$}
	{\True $P=2$}
	\loigiai{Ta có $\cos ^2\dfrac{7\pi}{8}=\cos ^2\dfrac{\pi}{8}$ và $\cos ^2\dfrac{5\pi}{8}=\cos ^2\dfrac{3\pi}{8}$
		$\Rightarrow P=2\left({{\cos}^2\dfrac{\pi}{8}+{\cos}^2\dfrac{{3\pi}}{8}}\right)$.\\
		Vì $\dfrac{\pi}{8}+\dfrac{{3\pi}}{8}=\dfrac{\pi}{2}\Rightarrow \cos \dfrac{\pi}{8}=\sin \dfrac{{3\pi}}{8}\Rightarrow \cos ^2\dfrac{\pi}{8}=\sin ^2\dfrac{{3\pi}}{8}.$\\
		Do đó $ P=2 \left({{\sin}^2\dfrac{{3\pi}}{8}+{\cos}^2\dfrac{{3\pi}}{8}}\right)=2\cdot1=2.$
	}
\end{ex}
\begin{ex}%[Câu 8]%[1K1B1-8]
	Cho $\sin a + \cos a = - \dfrac 54$, khi đó giá trị của $\sin a \cos a$ bằng
	\choice
	{$1$}
	{$\dfrac{5}{4}$}
	{$\dfrac{3}{16}$}
	{\True $\dfrac{9}{32}$}
	\loigiai{
		$\sin a\cos a = \dfrac{(\sin a + \cos a)^2 -1}{2} = \dfrac{9}{32}$.
	}
\end{ex}
\begin{ex}%[Câu 9]%[1K1Y1-8]
	Cho $\tan x=\dfrac{1}{2}$. Tính $\tan \left(x+\dfrac{\pi}{4}\right)$.
	\choice
	{$2$}
	{$\dfrac{3}{2}$}
	{$6$}
	{\True $3$}
	\loigiai
	{
		Ta có $\tan \left(x+\dfrac{\pi}{4}\right)=\dfrac{\tan x+\tan \dfrac{\pi}{4}}{1-\tan x\cdot \tan \dfrac{\pi}{4}}=\dfrac{\dfrac{1}{2}+1}{1-\dfrac{1}{2}} = 3$.
	}
\end{ex}
\begin{ex}%[Câu 10]%[1K1Y1-3]
	Biểu diễn các góc lượng giác $\alpha=-\dfrac{5\pi}{6}$, $\beta=\dfrac{\pi}{3}$, $\gamma=\dfrac{25\pi}{3}$, $\delta=\dfrac{17\pi}{6}$ trên đường tròn lượng giác. Các góc nào có điểm biểu diễn trùng nhau?
	\choice
	{\True $\beta$ và $\gamma$}
	{$\alpha$, $\beta$, $\gamma$}
	{$\beta$, $\gamma$, $\delta$}
	{$\alpha$ và $\beta$}
	\loigiai{
		Ta có $\beta+8\pi=\dfrac{\pi}{3}+8\pi=\dfrac{25\pi}{3}=\gamma$.\\
		Do đó, $\beta$ và $\gamma$ có điểm biểu diễn trùng nhau trên đường tròn lượng giác.
	}
\end{ex}
\begin{ex}%[Câu 11]%[1K1Y1-7]
	Trong các khẳng định sau, khẳng định  nào là \textbf{sai}?
	\choice
	{$\sin(\pi-\alpha)=\sin\alpha$}
	{\True $\cos(\pi-\alpha)=\cos \alpha$}
	{$\sin(\pi+\alpha)=-\sin\alpha$}
	{$\cos(\pi+\alpha)=-\cos \alpha$}
	\loigiai{
		Ta có $\cos(\pi-\alpha)=-\cos \alpha$ nên $\cos(\pi-\alpha)=\cos \alpha$ là khẳng định \textbf{sai}.
	}
\end{ex}
\begin{ex}%[Câu 12]%[1K1Y1-2]
	Góc lượng giác nào tương ứng với chuyển động quay $3\dfrac{1}{5}$ vòng ngược chiều kim đồng hồ?
	\choice
	{$\dfrac{16 \pi}{5}$}
	{$\left(\dfrac{16}{5}\right)^\circ$}
	{\True $1152^\circ$}
	{$1152 \pi$}
	\loigiai{
		Chuyển động quay ngược chiều kim đồng hồ là quay theo chiều dương; góc tương ứng là
		$$3\dfrac{1}{5}\cdot 2\pi=\dfrac{32\pi}{5}, \text{ tương ứng với } 1152^\circ.$$
	}
\end{ex}

\begin{ex}%[Câu 13]%[1K1Y2-1]
	Trong các khẳng định sau, khẳng định nào \textbf{sai}?
	\choice
	{\True $\cos (a-b)=\cos a\cos b-\sin a\sin b$}
	{$\sin (a-b)=\sin a\cos b-\cos a\sin b$}
	{$\cos (a+b)=\cos a\cos b-\sin a\sin b$}
	{$\sin (a+b)=\sin a\cos b+\cos a\sin b$}
	\loigiai{
		Ta có $\cos (a-b)=\cos a\cos b+\sin a\sin b$ nên $\cos (a-b)=\cos a\cos b-\sin a\sin b$ là khẳng định \textbf{sai}.
	}
\end{ex}
\begin{ex}%[Câu 14]%[1K1Y1-7]
	Trong trường hợp nào dưới đây $\cos \alpha=\cos \beta$ và $\sin \alpha=-\sin \beta$?
	\choice
	{\True $\beta=-\alpha$}
	{$\beta=\pi-\alpha$}
	{$\beta=\pi+\alpha$}
	{$\beta=\dfrac{\pi}{2}+\alpha$}
	\loigiai{
		Trong trường hợp hai cung đối nhau thì các giá trị $\cos$ của chúng bằng nhau, các giá trị $\sin$ của chúng đối nhau.
	}
\end{ex}

\begin{ex}%[Câu 15]%[1K1B2-2]
	Nếu $\cos a=\dfrac{1}{4}$ thì $\cos 2 a$ bằng
	\choice
	{$\dfrac{7}{8}$}
	{\True $-\dfrac{7}{8}$}
	{$\dfrac{15}{16}$}
	{$-\dfrac{15}{16}$}
	\loigiai{
		Ta có  $\cos 2 a =2 \cos^2 a-1=2 \cdot \left( \dfrac{1}{4} \right)^2-1=-\dfrac{7}{8}$.
	}
\end{ex}
\begin{ex}%[Câu 16]%[1K1K2-2]
	Nếu $\tan (a+b)=3, \tan (a-b)=-3$ thì $\tan 2 a$ bằng
	\choice
	{\True $0$}
	{$\dfrac{3}{5}$}
	{$1$}
	{$-\dfrac{3}{4}$}
	\loigiai{
		Ta có $\tan (a+b)=3 \Leftrightarrow \tan a+ \tan b= 3- 3 \tan a \tan b$ \quad (1)\\
		và 
		$\tan (a-b)=-3 \Leftrightarrow \tan a- \tan b= -3- 3 \tan a \tan b$. \quad (2) \\
		Lấy vế trừ vế của (1) và (2) ta được $2\tan b=6\Leftrightarrow \tan b =3$.\\
		Thay $\tan b =3$ vào (1) ta được $\tan a= 0$.\\
		Khi đó $\tan 2 a = \dfrac{2 \tan a}{1- \tan^2 a}=0$.
	}
\end{ex}

\begin{ex}%[Câu 17]%[1K1K2-3]
	Nếu $\cos a=\dfrac{3}{5}$ và $\cos b=-\dfrac{4}{5}$ thì $\cos (a+b) \cos (a-b)$ bằng
	\choice
	{\True $0$}
	{$2$}
	{$4$}
	{$5$}
	\loigiai{
		Do  $\cos a=\dfrac{3}{5}$ và $\cos b=-\dfrac{4}{5}$ nên  $\cos 2a=-\dfrac{7}{25}$ và $\cos 2b=\dfrac{7}{25}$.\\
		Ta có $2 \cos (a+b) \cos (a-b)= \cos 2a +\cos 2b=-\dfrac{7}{25}+ \dfrac{7}{25}=0$.\\
		Do đó $\cos (a+b) \cos (a-b)=0$.
	}
\end{ex}




\begin{ex}%[Câu 18]%[1K1B2-3]
	Rút gọn biểu thức $M=\cos(a+b)\cos(a-b)-\sin (a+b)\sin(a-b)$, ta được
	\choice
	{$M=\sin 4a$}
	{$M=1-2\cos^2a$}
	{\True $M=1-2\sin^2a$}
	{$M=\cos 4a$}
	\loigiai{
		Ta có
		\allowdisplaybreaks
		\begin{eqnarray*}
			M&=&\cos(a+b)\cos(a-b)-\sin (a+b)\sin(a-b)\\
			&=&\dfrac{1}{2}\left(\cos2a+\cos 2b\right)+\dfrac{1}{2}\left(\cos2a-\cos 2b\right)\\
			&=&\cos 2a\\
			&=&1-2\sin^2a.
		\end{eqnarray*}
	}
\end{ex}
\begin{ex}%[Câu 19]%[1K1B2-2]
	Nếu $\sin x +\cos x = \dfrac{1}{2}$ thì $\sin 2x$ bằng
	\choice
	{$\dfrac{3}{4}$}
	{$\dfrac{3}{8}$}
	{$\dfrac{\sqrt{2}}{2}$}
	{\True $-\dfrac{3}{4}$}
	\loigiai{
		Ta có $\sin 2x=\left( {\sin x+\cos x} \right)^2-\left( {\sin^2 x+\cos ^2x} \right)=\left( {\dfrac{1}{2}} \right)^2-1=-\dfrac{3}{4}$.}
\end{ex}
\begin{ex}%[Câu 20]%[1K1Y2-3]
	Mệnh đề nào dưới đây đúng?
	\choice
	{\True $\cos3x\cdot\cos5x=\dfrac{1}{2}(\cos8x+\cos2x)$}
	{$\cos3x\cdot\cos5x=\dfrac{1}{2}(\cos8x-\cos2x)$}
	{$\cos3x\cdot\cos5x=\dfrac{1}{2}(\cos2x-\cos8x)$}
	{$\cos3x\cdot\cos5x=\dfrac{1}{2}(\sin8x+\sin2x)$}
	\loigiai{Ta có $\cos3x\cdot\cos5x=\dfrac{1}{2}[\cos(3x+5x)+\cos(3x-5x)]=\dfrac{1}{2}(\cos8x+\cos2x)$.}
\end{ex}
\begin{ex}%[Câu 21]%[1K1B2-4]
	Giả sử $3\sin ^4x-\cos ^4x=\dfrac{1}{2}$ thì $\sin ^4x+3\cos ^4x$ có giá trị bằng
	\choice
	{$2$}
	{\True $1$}
	{$4$}
	{$3$}
	\loigiai{
		\begin{eqnarray*}
			3\sin ^4x-\cos ^4x=\dfrac{1}{2}\Leftrightarrow 6\sin ^4x-2\cos ^4x=1&\Leftrightarrow& 6\sin ^4x-2\left(1-\sin^2\alpha\right)^2=1\\
			&\Leftrightarrow& 4\sin ^4x-4\sin ^2\alpha-3=0\\
			&\Leftrightarrow& \left(2\sin^2\alpha+3\right)\left(2\sin^2\alpha-1\right)=0\\
			&\Rightarrow&{\sin ^2}\alpha=\dfrac{1}{2}.
		\end{eqnarray*}
		Ta có $\sin ^4x+3\cos ^4x$ $=\sin ^4\alpha+3\left(1-\sin^2\alpha\right)^2$ $=\dfrac{1}{4}+3\left(1-\dfrac{1}{2}\right)^2=1$.}
\end{ex}
\begin{ex}%[Câu 22]%[1K1B3-2]
	Hàm số $y=\sin x$ đồng biến trên khoảng
	\choice
	{$(0 ; \pi)$}
	{$\left(-\dfrac{3 \pi}{2} ;-\dfrac{\pi}{2}\right)$}
	{\True $\left(-\dfrac{ \pi}{2} ;\dfrac{\pi}{2}\right)$}
	{$(-\pi ; 0)$}
	\loigiai{ 
		Do hàm số $y=\sin x$ đồng biến trên mỗi khoảng $\left( -\dfrac{\pi}{2}+k2 \pi;\dfrac{\pi}{2}+k2 \pi \right) $ nên ứng với $k=0$, ta có hàm số $y=\sin x$ đồng biến trên khoảng $\left(-\dfrac{ \pi}{2} ;\dfrac{\pi}{2}\right)$.
	}
\end{ex}

\begin{ex}%[Câu 23]%[1K1B3-2]
	Hàm số nghịch biến trên khoảng $(\pi ; 2 \pi)$ là
	\choice
	{$y=\sin x$}
	{$y=\cos x$}
	{$y=\tan x$}
	{\True $y=\cot x$}
	\loigiai{
		Do hàm số $y=\cot x$ nghịch biến trên mỗi khoảng $\left( k \pi; \pi +k \pi \right) $ nên  ứng với $k=1$, ta có hàm số $y=\cot x$ nghịch biến trên khoảng $(\pi ; 2 \pi)$.
	}
\end{ex}
\begin{ex}%[Câu 24]%[1K1B3-1]
	Tập xác định của hàm số $y=\dfrac{\cos x}{\sin x-1}$ là
	\choice
	{$\mathbb{R}\setminus \left\{k2\pi| k\in\mathbb{Z}\right\}$}
	{\True $\mathbb{R}\setminus \left\{\dfrac{\pi}{2}+k2\pi| k\in\mathbb{Z}\right\}$}
	{$\mathbb{R}\setminus \left\{\dfrac{\pi}{2}+k\pi| k\in\mathbb{Z}\right\}$}
	{$\mathbb{R}\setminus \left\{k\pi| k\in\mathbb{Z}\right\}$}
	\loigiai{
		Hàm số xác định khi và chỉ khi $\sin x-1\ne 0\Leftrightarrow\sin x\ne 1\Leftrightarrow x\ne \dfrac{\pi}{2}+k2\pi$ với $k\in \mathbb{Z}$.\\
		Vậy tập xác định của hàm số là $\mathbb{R}\setminus \left\{\dfrac{\pi}{2}+k2\pi| k\in\mathbb{Z}\right\}$.
	}
\end{ex}
\begin{ex}%[Câu 25]%[1K1Y3-1]
	Khẳng định nào sau đây là \textbf{sai}?
	\choice
	{Hàm số $y=\cos x$ có tập xác định là $\mathbb{R}$}
	{Hàm số $y=\cos x$ có tập giá trị là $[-1;1]$}
	{\True Hàm số $y=\cos x$ là hàm số lẻ}
	{Hàm số $y=\cos x$ tuần hoàn với chu kì $2\pi$}
	\loigiai{
		Hàm số $y=\cos x$ là hàm số chẵn.
	}
\end{ex}
\begin{ex}%[Câu 26]%[1K1Y3-4]
	Trong các hàm số sau đây, hàm số nào là hàm tuần hoàn?
	\choice
	{$y=\tan x+x$}
	{$y=x^2+1$}
	{\True $y=\cot x$}
	{$y=\dfrac{\sin x}{x}$}
	\loigiai{
		Hàm số $y=\cot x$ là hàm số tuần hoàn với chu kỳ $T=\pi$.
	}
\end{ex}
\begin{ex}%[Câu 27]%[1K1Y3-3]
	Khẳng định nào sau đây đúng?
	\choice
	{Hàm số $y=\sin x$ là hàm số chẵn}
	{\True Hàm số $y=\cos x$ là hàm số chẵn}
	{Hàm số $y=\tan x$ là hàm số chẵn}
	{Hàm số $y=\cot x$ là hàm số chẵn}
	\loigiai{
		Hàm số $y=\cos x$ là hàm số chẵn;  các hàm số còn lại là hàm số lẻ.
	}
\end{ex}
\begin{ex}%[Câu 28]%[1K1B3-3]
	Khẳng định nào sau đây là đúng?
	\choice
	{Hàm số $y=\cos x$ là hàm số lẻ}
	{\True Hàm số $y=\tan 2x- \sin x$ là hàm số lẻ}
	{Hàm số $y=\sin x$ là hàm số chẵn}
	{Hàm số $y=\tan x \cdot \sin x$ là hàm số lẻ}
	\loigiai{
		Xét hàm số $y=f\left( {x} \right)=\tan 2x-\sin x$.
		\\Hàm số xác định khi $\cos 2x \ne 0 \Leftrightarrow x \ne \dfrac{\pi}{4}+k\dfrac{\pi}{2}$, $\left( {k \in \mathbb{Z}} \right)$.
		\\Tập xác định $\mathscr{D}=\mathbb{R} \setminus \left\{ {\dfrac{\pi}{4}+k\dfrac{\pi}{2},k \in \mathbb{Z}} \right\}$.
		\\ Với mọi $x \in \mathscr{D}$ thì $-x \in \mathscr{D}$ và $f\left( {-x} \right)=\tan \left( {-2x} \right)- \sin \left( {-x} \right)=-\tan 2x + \sin x=-f\left( {x} \right)$.\\ Do đó hàm số $y=\tan 2x-\sin x$ là hàm số lẻ.}
\end{ex}
\begin{ex}%[Câu 29]%[1K1B3-1]
	Tập xác định của hàm số $y=\dfrac{\cot x}{\cos x-1}$ là
	\choice
	{$\mathbb{R}\setminus\left\{\dfrac{k\pi}{2}, k\in\mathbb{Z}\right\}$}
	{$\mathbb{R}\setminus\left\{\dfrac{k}{2}+k\pi, k\in\mathbb{Z}\right\}$}
	{\True $\mathbb{R}\setminus\left\{k\pi, k\in\mathbb{Z}\right\}$}
	{$\mathbb{R}\setminus\left\{k2\pi, k\in\mathbb{Z}\right\}$}
	\loigiai{
		Hàm số xác định khi và chỉ khi $\heva{&\sin x\ne 0\\ &\cos x\ne 1}\Leftrightarrow\heva{&x\ne k\pi\\ &x\ne l2\pi}\ (k,l\in\mathbb{Z})\Leftrightarrow x\ne k\pi, k\in\mathbb{Z}$.\\
		Vậy, tập xác định của hàm số $y=\dfrac{\cot x}{\cos x-1}$ là $\mathbb{R}\setminus\left\{k\pi, k\in\mathbb{Z}\right\}$.
	}
\end{ex}
\begin{ex}%[Câu 30]%[1K1K3-2]
	Cho đồ thị hàm số $y=\sin x$ như hình vẽ sau
	\begin{center}
		
		\begin{tikzpicture}[>=stealth,scale=0.7]
			\draw [->] (-11,0)--(0,0)
			node[below right]{$O$}--(11,0)node[below]{$x$}; % Hệ trục tọa độ
			\draw[->] (0,-1.5) --(0,2) node[left]{$y$};
			\draw[dashed] (-5*pi/2,0)node[above]{$-\tfrac{5\pi}{2}$}--(-5*pi/2,-1)--(3*pi/2,-1)--(3*pi/2,0)node[above]{$\tfrac{3\pi}{2}$};
			\draw[dashed] (-3*pi/2,0)node[below]{$-\tfrac{3\pi}{2}$} --(-3*pi/2,1)--(5*pi/2,1)--(5*pi/2,0)node[below]{$\tfrac{5\pi}{2}$};
			\draw[dashed] (-pi/2,0)node[above]{$-\tfrac{\pi}{2}$}--(-pi/2,-1);
			\draw[dashed] (pi/2,0)node[below]{$\tfrac{\pi}{2}$}--(pi/2,1);
			\draw(-2.9*pi,0) node[below left]{$-3\pi$}(-1.9*pi,0) node[below]{$-2\pi$}(-1.1*pi,0) node[below]{$-\pi$}(3*pi,0) node[below]{$3\pi$}(2*pi,0) node[below]{$2\pi$}(pi,0) node[above]{$\pi$}(0,-1.6)node[below]{$2\pi$}(0,1)node[above right]{$1$};
			\draw[dashed] (-pi,0)--(-pi,-1.6)(pi,0)--(pi,-1.6);
			\draw[<->](-pi,-1.6)--(pi,-1.6);
			\draw [domain=-3.5*pi:3.4*pi,samples=100] plot (\x, {sin(\x r)});
		\end{tikzpicture}
	\end{center}
	Mệnh đề nào dưới đây \textbf{sai}?
	\choice
	{Hàm số $y=\sin x$ tăng trên khoảng $\left(-\dfrac{\pi}{2};\dfrac{\pi}{2}\right)$}
	{Hàm số $y=\sin x$ giảm trên khoảng $\left(\dfrac{\pi}{2};\dfrac{3\pi}{2}\right)$}
	{Hàm số $y=\sin x$ giảm trên khoảng $\left(-\dfrac{3\pi}{2};-\pi \right)$}
	{\True Hàm số $y=\sin x$ tăng trên khoảng $\left(0;\pi \right)$}
	\loigiai{
		\begin{itemize}
			\item Hàm số $y=\sin x$ tăng trên $\left(0;\dfrac{\pi}{2}\right)$ và giảm trên $\left(\dfrac{\pi}{2};\pi \right)$.
			\item Vậy trên khoảng $\left(0;\pi \right)$, hàm số $y=\sin x$ vừa tăng vừa giảm nên khẳng định hàm số $y=\sin x$ tăng trên khoảng $\left(0;\pi \right)$ là khẳng định \textbf{sai}.
	\end{itemize}}
\end{ex}
\begin{ex}%[Câu 31]%[1K1B3-2]
	Chọn khẳng định đúng trong các khẳng định sau
	\choice
	{Hàm số $y = \tan x$ tuần hoàn với chu kì $2\pi$}
	{Hàm số $y = \cos x$ tuần hoàn với chu kì $\pi$}
	{\True Hàm số $y = \sin x$ đồng biến trên khoảng $\left(0; \dfrac{\pi}{2}\right)$}
	{Hàm số $y = \cot x$ nghịch biến trên $\mathbb{R}$}
	\loigiai{Ta xét $y = \sin x$ suy ra  $y'  = \cos x$. Dễ thấy $\cos x > 0\ ,\  \forall x\in \left(0; \dfrac{\pi}{2}\right)$. Do đó hàm số $y = \sin x$ đồng biến trên khoảng $\left(0; \dfrac{\pi}{2}\right)$.
	}
\end{ex}
\begin{ex}%[Câu 32]%[1K1K3-6]
	Đồ thị của hàm số $y=\sin x$ và $y=\cos x$ cắt nhau tại bao nhiêu điểm có hoành độ thuộc đoạn $\left[-2\pi;\dfrac{5\pi}{2}\right]$?
	\choice
	{\True $5$}
	{$6$}
	{$4$}
	{$7$}
	\loigiai{
		Xét phương trình hoành độ giao điểm của hai đồ thị hàm số $\sin x=\cos x$.\\
		Nếu $\cos x=0$ thì $\sin x=0$ nên vô lý.\\
		Do đó, $\cos x\ne 0$. Ta có
		\allowdisplaybreaks
		\begin{eqnarray*}
			\sin x=\cos x&\Leftrightarrow&\tan x=1\\
			&\Leftrightarrow&x=\dfrac{\pi}{4}+k\pi ,\quad \left(k\in\mathbb{Z}\right).
		\end{eqnarray*}
		Ta lại có
		\allowdisplaybreaks
		\begin{eqnarray*}
			-2\pi \le x\le \dfrac{5\pi}{2}&\Leftrightarrow& -2\pi \le \dfrac{\pi}{4}+k\pi\le \dfrac{5\pi}{2}\\
			&\Leftrightarrow& -2 \le \dfrac{1}{4}+k\le \dfrac{5}{2}\\
			&\Leftrightarrow& \dfrac{-9}{4} \le k\le \dfrac{9}{4}.
		\end{eqnarray*}
		Do $k\in\mathbb{Z}$ nên $k\in\left\{-2;-1;0;1;2\right\}$.\\
		Vậy hai đồ thị hàm số cắt nhau tại $5$ điểm có hoành độ thuộc đoạn $\left[-2\pi;\dfrac{5\pi}{2}\right]$.
	}
\end{ex}
\begin{ex}%[Câu 33]%[1K1B3-5]
	Tìm tập giá trị của hàm số $y=2\cos3x +1$.
	\choice
	{$[-3;1]$}
	{$[-3;-1]$}
	{\True $[-1;3]$}
	{$[1;3]$}
	\loigiai{
		$\forall x\in \mathbb{R}$ ta có
		\begin{eqnarray*}
			&& -1\leq\cos3x\leq1 \\
			&\Leftrightarrow& -2\leq2\cos3x\leq2 \\
			&\Leftrightarrow& -1\leq2\cos3x+1\leq3.
		\end{eqnarray*}
	}
\end{ex}
\begin{ex}%[Câu 34]%[1K1B3-6]
	Đường cong trong hình bên là đồ thị trên đoạn $\left[-\pi ;\pi\right]$ của một hàm số trong bốn hàm số được liệt kê ở bốn phương án $\textbf{A, B, C, D}$ dưới đây. Hỏi đó là hàm số nào?
	\begin{center}
		\definecolor{x}{rgb}{0.75,0.75,0.75}
		\begin{tikzpicture}[scale=1, line join=round, line cap=round,>=stealth]
			\draw[->] (-4,0.) -- (4,0.)node [above] { $x$};
			\draw[shift={(-3.14,0)}] node[below left] {\footnotesize $-\pi$};
			\draw[shift={(-1.57,0)}] node[above left] {\footnotesize $-\dfrac{\pi}{2}$};
			\draw[shift={(1.57,0)}] node[below left] {\footnotesize $\dfrac{\pi}{2}$};
			\draw[shift={(3.14,0)}] node[below left] {\footnotesize $\pi$};
			\draw[->] (0.,-1.3) -- (0.,1.5)node [right] { $y$};
			\draw (0,1) node[above left] {\footnotesize $1$};
			\draw (0,-1) node[above left] {\footnotesize $-1$};
			\draw (0pt,-10pt) node[right] {\footnotesize $O$};
			\clip(-4.2,-1.3) rectangle (4.2,1.5);
			\draw[line width=1.2pt,smooth,samples=100,domain=-3.14:3.14] plot(\x,{sin(((\x))*180/pi)});
			\draw [dashed] (-1.57,0)--(-1.57,-1)--(0,-1)(1.57,0)--(1.57,1)--(0,1);
		\end{tikzpicture}
	\end{center}
	\choice
	{\True $y=\sin x$}
	{$y=\cos x$}
	{$y=\tan x$}
	{$y=\cot x$}
	\loigiai{
		Đồ thị hàm số đi qua các điểm $(0;0),(\pi;0), \left(\dfrac{\pi}{2};1\right)$ và nhận $O$ làm tâm đối xứng.
	}
\end{ex}
\begin{ex}%[Câu 35]%[1K1Y4-3]
	Phương trình $\cot x=-1$ có nghiệm là
	\choice
	{\True $-\dfrac{\pi}{4}+k \pi(k \in \mathbb{Z})$}
	{$\dfrac{\pi}{4}+k \pi(k \in \mathbb{Z})$}
	{$\dfrac{\pi}{4}+k 2 \pi(k \in \mathbb{Z})$}
	{$-\dfrac{\pi}{4}+k 2 \pi(k \in \mathbb{Z})$}
	\loigiai{
		Ta có $\cot x=-1 \Leftrightarrow \cot x=\cot \left(-\dfrac{\pi}{4}\right) \Leftrightarrow x =-\dfrac{\pi}{4}+k \pi(k \in \mathbb{Z}) $.
	}
\end{ex}
\begin{ex}%[Câu 36]%[1K1Y4-3]
	Trong các phép biến đổi sau, phép biến đổi nào \textbf{sai}?
	\choice
	{$\sin x=1\Leftrightarrow x=\dfrac{\pi}{2}+k2\pi,(k\in \mathbb{Z})$}
	{$\tan x=1\Leftrightarrow x=\dfrac{\pi}{4}+k\pi,(k\in \mathbb{Z})$}
	{$\cos x=\dfrac{1}{2}\Leftrightarrow \hoac{
			& x=\dfrac{\pi}{3}+k2\pi,(k\in \mathbb{Z}) \\
			& x=-\dfrac{\pi}{3}+k2\pi,(k\in \mathbb{Z})}$}
	{\True $\sin x=0\Leftrightarrow x=k2\pi,(k\in \mathbb{Z})$}
	\loigiai{
		Ta có $\sin x=0\Leftrightarrow x=k\pi,(k\in \mathbb{Z})$, nên đáp án $\sin x=0\Leftrightarrow x=k2\pi,(k\in \mathbb{Z})$ sai.}
\end{ex}
\begin{ex}%[Câu 37]%[1K1B4-5]
	Nghiệm của phương trình $\sin x\cdot \cos x=\dfrac{1}{2}$ là
	\choice
	{$x=k2\pi$; $k\in \mathbb{Z}$}
	{$x=\dfrac{k\pi}{4}$; $k\in \mathbb{Z}$}
	{\True $x=\dfrac{\pi}{4}+k\pi$; $k\in \mathbb{Z}$}
	{$x=k\pi$; $k\in \mathbb{Z}$}
	\loigiai{
		Ta có $\sin x\cdot \cos x=\dfrac{1}{2}\Leftrightarrow \sin 2x=1\Leftrightarrow 2x=\dfrac{\pi}{2}+k2\pi\Leftrightarrow x=\dfrac{\pi}{4}+k\pi$ với $k\in\mathbb{Z}$.
	}
\end{ex}
\begin{ex}%[Câu 38]%[1K1Y4-3]
	Họ nghiệm của phương trình $\sin2x=1$ là
	\choice
	{$x=\dfrac{\pi}{2}+k\pi,\,k\in\mathbb{Z}$}
	{$x=\dfrac{\pi}{2}+k2\pi,\,k\in\mathbb{Z}$}
	{\True $x=\dfrac{\pi}{4}+k\pi,\,k\in\mathbb{Z}$}
	{$x=\dfrac{\pi}{4}+\dfrac{k\pi}{2},\,k\in\mathbb{Z}$}
	\loigiai{
		Ta có $\sin2x=1\Leftrightarrow 2x=\dfrac{\pi}{2}+k2\pi\Leftrightarrow x=\dfrac{\pi}{4}+k\pi,\, k\in\mathbb{Z}$.
	}
\end{ex}
\begin{ex}%[Câu 39]%[1K1B4-5]
	Phương trình $\sin 2x \cos x = \sin 7x \cos 4x$ có các họ nghiệm là
	\choice
	{$x=\dfrac{k2\pi}{5};x=\dfrac{\pi}{12}+\dfrac{k\pi}{6} (k \in \Bbb{Z})$}
	{$x=\dfrac{k\pi}{5};x=\dfrac{\pi}{12}+\dfrac{k\pi}{3} (k \in \Bbb{Z})$}
	{\True $x=\dfrac{k\pi}{5};x=\dfrac{\pi}{12}+\dfrac{k\pi}{6} (k \in \Bbb{Z})$}
	{$x=\dfrac{k2\pi}{5};x=\dfrac{\pi}{12}+\dfrac{k\pi}{3} (k \in \Bbb{Z})$}
	\loigiai{
		Ta có \begin{eqnarray*}
			\sin 2x \cos x = \sin 7x \cos 4x &\Leftrightarrow & \dfrac{1}{2}(\sin 3x+\sin x)=\dfrac{1}{2}(\sin 11x+\sin 3x)\\
			&\Leftrightarrow & \sin 11x = \sin x\\
			&\Leftrightarrow & \hoac{&x=\dfrac{k\pi}{5}\\&x=\dfrac{\pi}{12}+\dfrac{k\pi}{3} }(k \in \Bbb{Z}).
		\end{eqnarray*}
	}
\end{ex}
\begin{ex}%[Câu 40]%[1K1K4-3]
	Số nghiệm của phương trình $\cos x=0$ trên đoạn $[0 ; 10 \pi]$ là
	\choice
	{$5$}
	{$9$}
	{\True $10$}
	{$11$}
	\loigiai{
		Ta có $\cos x=0 \Leftrightarrow x =\dfrac{\pi}{2}+ k \pi (k \in \mathbb{Z})$.\\
		Do $0 \leq x \leq 10 \pi \Leftrightarrow 0 \leq \dfrac{\pi}{2}+ k \pi \leq 10 \Leftrightarrow -\dfrac{1}{2} \leq k \leq \dfrac{19}{2}\Leftrightarrow 0\leq k \leq 9( k \in  \mathbb{Z} )$.\\
		Do đó phương trình $\cos x=0$ có $10$ nghiệm.
	}
\end{ex}

\begin{ex}%[Câu 41]%[1K1B4-3]
	Số nghiệm của phương trình $\sin x=0$ trên đoạn $[0 ; 10 \pi]$ là
	\choice
	{$10$}
	{$6$}
	{$5$}
	{\True $11$}
	\loigiai{
		Ta có $\sin x=0 \Leftrightarrow x = k \pi (k \in \mathbb{Z})$.\\
		Do $0 \leq x \leq 10 \pi \Leftrightarrow 0 \leq k\leq 10$.\\
		Do đó phương trình $\sin x=0$ có $11$ nghiệm.
	}
\end{ex}

\begin{ex}%[Câu 42]%[1K1B4-3]
	Số nghiệm của phương trình $\sin \left(x+\dfrac{\pi}{4}\right)=\dfrac{\sqrt{2}}{2}$ trên đoạn $[0; \pi]$ là
	\choice
	{$4$}
	{$1$}
	{\True $2$}
	{$3$}
	\loigiai{
		Ta có $\sin \left(x+\dfrac{\pi}{4}\right)=\dfrac{\sqrt{2}}{2} \Leftrightarrow \sin \left(x+\dfrac{\pi}{4}\right)=\sin \left( \dfrac{\pi}{4}\right) \Leftrightarrow  \hoac{&x= k 2 \pi\\&x=\dfrac{\pi}{2}+ k 2 \pi} (k \in \mathbb{Z})$.\\
		Do $x \in [0 ; \pi]$ nên $x=0$ hoặc $x=\dfrac{\pi}{2}$.
	}
\end{ex}
\begin{ex}%[Câu 43]%[1K1B4-5]
	Phương trình $ \sin{2x}+3\cos x=0 $ có bao nhiêu nghiệm trong khoảng $ (0;\pi)$?
	\choice
	{$ 0 $}
	{\True $ 1 $}
	{$ 2 $}
	{$ 3 $}
	\loigiai
	{
		Ta có $ \sin{2x}+3\cos x=0 \Leftrightarrow \hoac{& \cos x=0 \\ &\sin x=-\dfrac{3}{2}}\Leftrightarrow \cos x=0 \Leftrightarrow x= \dfrac{\pi}{2}+k\pi$. Do $ x \in (0;\pi) $ nên có một nghiệm là $ x=\dfrac{\pi}{2}$.
	}
\end{ex}

\begin{ex}%[Câu 44]%[1K1B1-9]
	Một bánh xe có $72$ răng. Số đo góc mà bánh xe đã quay được khi di chuyển $10$ răng là
	\choice
	{$40^\circ	$}
	{\True $50^\circ$}
	{$60^\circ$}
	{$30^\circ$}
	\loigiai{
		1 bánh răng tương ứng với $\dfrac{360^\circ}{72}=5^\circ$$\Rightarrow 10$ bánh răng là $50^\circ$.}
\end{ex}

\begin{ex}%[Câu 45]%[1K1K1-9]
	\immini{Người ta muốn làm một cánh diều hình quạt có bán kính là $a$, độ dài cung tròn là $b$ và có chu vi là $80$ cm (như hình vẽ). Khi diện tích cánh diều đạt giá trị lớn nhất, tổng $a+b$ bằng
		\choice
		{$50$ cm}
		{$40$ cm}
		{$70$ cm}
		{\True $60$ cm}}{
		\begin{tikzpicture}
			\draw (0,3) arc (150:210:3);
			\coordinate [label=below:$A$] (A) at (0,0);
			\coordinate [label=right:$O$] (O) at (30:3);
			\coordinate [label=above:$C$] (C) at (0,3);
			\foreach \point in {O,A,C} \fill[black] (\point) circle (1pt);
			\draw (O)--(C) (O)--(A);
		\end{tikzpicture}}
	\loigiai{
		Gọi $\varphi$ (rad) là số đo cung của hình quạt. Khi đó $\varphi =\dfrac{b}{a}$.\\
		Chu vi cánh diều bằng $b+2a=80$.\\
		Diện tích cánh diều bằng $S=\dfrac{\varphi a^2}{2}=\dfrac{ab}{2}=\dfrac{1}{4}(b \cdot 2a) \le \dfrac{1}{4} \cdot \left(\dfrac{b+2a}{2}\right)^2=400$.\\
		Dấu bằng xảy ra khi và chỉ khi $\heva{&b=2a \\& b+2a=80}\Leftrightarrow \heva{&b=40 \\& a=20.}$\\
		Do vậy $a+b=60$ cm.}
\end{ex}
\begin{ex}%[Câu 46]%[1K1K1-9]
	\immini{
		Khi một tia sáng truyền từ không khí vào mặt nước thì một phần tia sáng bị phản xạ trên bề mặt, phần còn lại bị khúc xạ như hình bên. Góc tới $i$ liên hệ với góc khúc xạ $r$ bởi Định luật khúc xạ ánh sáng
		$$\dfrac{\sin i}{\sin r}=\dfrac{n_2}{n_1}.$$
		Ở đây, $n_1$ và $n_2$ tương ứng với chiết suất của môi trường $1$ (không khí) và môi trường $2$ (nước). Cho biết góc tới $i=50^\circ$ và  chiết suất của không khí bằng $1$ còn chiết suất của nước là $1{,}33$. Khi đó  góc khúc xạ gần với kết quả nào sau đây.
		\choice
		{\True$35{,}17^\circ$}
		{$55{,}47^\circ$}
		{$31{,}42^\circ$}
		{$12{,}35^\circ$}
	}
	{
		\begin{tikzpicture}[>=stealth,line join=round,line cap=round,font=\footnotesize,scale=.7]
			\path
			(0,0)coordinate(I)++(90:3)coordinate(N)++(-90:6)coordinate(N')
			(I)++(0:3)coordinate(B)++(180:6)coordinate(A)
			(I)++(30:3)coordinate(S')
			(I)++(150:3)coordinate(S)
			(I)++(-55:3)coordinate(R)
			;
			\fill[cyan!20!](-3,-3)rectangle(3,0)
			;
			\draw (A)--(B)
			;
			\draw[dashed](N)--(N')
			;
			\draw[->,midway](S)--(I)
			;
			\draw[->](I)--(S')
			;
			\draw[->](I)--(R)
			;
			\foreach \p/\r in {N/180,N'/180,S/160,S'/90,R/0,I/-135}
			\fill (\p) node[shift={(\r:3mm)}]{$\p$}
			;
			\draw pic[angle radius=3mm,draw=red,fill=green!50,angle eccentricity=1.5] {angle = N--I--S}
			;
			\draw pic[angle radius=4mm,draw=orange,fill=orange!50,angle eccentricity=1.5] {angle = S'--I--N}
			;
			\draw pic[angle radius=4mm,draw=blue,fill=blue!50,angle eccentricity=1.5] {angle = N'--I--R}
			;
			\draw (-2.5,.5)circle(7pt)node{$1$}
			(-2.5,-.5)circle(8pt)node{$2$}
			;	
		\end{tikzpicture}
	}
	\loigiai{
		Ta có $\dfrac{\sin i}{\sin r}=\dfrac{n_2}{n_1}\Leftrightarrow \dfrac{\sin 50^\circ}{\sin r}=\dfrac{1{,}33}{1}\Leftrightarrow \sin r=\dfrac{\sin 50^\circ}{1{,}33}\Rightarrow r\approx 35{,}17^\circ$.
	}
\end{ex}
\begin{ex}%[Câu 47]%[1K1G2-4]
	Giả sử $a, b, c$ lần lượt là ba cạnh đối diện với ba góc $A, B, C$ của tam giác $ABC$ thỏa điều kiện $2\cos\dfrac{B}{2}\cos\dfrac{C}{2}=\dfrac{1}{2}+\dfrac{b+c}{a}\sin\dfrac{A}{2}$. Tính góc $A$ của tam giác $ABC$.
	\choice
	{$30^\circ$}
	{$45^\circ$}
	{\True $60^\circ$}
	{$90^\circ$}
	\loigiai
	{\noindent Đặt $2\cos\dfrac{B}{2}\cos\dfrac{C}{2}=\dfrac{1}{2}+\dfrac{b+c}{a}\sin\dfrac{A}{2}\;(\star)$. Ta có
		\begin{align*}
			(\star)&\Leftrightarrow  2\cos\dfrac{B}{2}\cos\dfrac{C}{2}=\dfrac{1}{2}+\dfrac{\sin B+\sin C}{\sin A}\sin\dfrac{A}{2}\\
			&\Leftrightarrow  \cos\dfrac{B+C}{2}+\cos\dfrac{B-C}{2}=\dfrac{1}{2}+\dfrac{2\sin\dfrac{B+C}{2}\cos\dfrac{B-C}{2}}{2\sin\dfrac{A}{2}\cos\dfrac{A}{2}}\sin\dfrac{A}{2}\\
			&\Leftrightarrow  \sin\dfrac{A}{2}+\cos\dfrac{B-C}{2}=\dfrac{1}{2}+\cos\dfrac{B-C}{2}\;\left(\text{vì}\;\sin\dfrac{A}{2}>0, \cos\dfrac{A}{2}=\sin\dfrac{B+C}{2}\right)\\
			&\Leftrightarrow  \sin\dfrac{A}{2}=\dfrac{1}{2}\Leftrightarrow  A=\dfrac{\pi}{3}.
		\end{align*}
	}
\end{ex}
\begin{ex}%[Câu 48]%[1K1G4-5]
	Phương trình $2\sqrt{3}\sin\left(x-\dfrac{\pi}{8}\right)\cos\left(x-\dfrac{\pi}{8}\right)+2\cos^2\left(x-\dfrac{\pi}{8}\right) = \sqrt{3}+1$ có nghiệm là
	\choice
	{\True $x=\dfrac{5\pi}{24}+k\pi$, $x=\dfrac{3\pi}{8}+k\pi$ với $k\in\mathbb{Z}$}
	{$x=\dfrac{5\pi}{12}+k\pi$, $x=\dfrac{3\pi}{4}+k\pi$ với $k\in\mathbb{Z}$}
	{$x=\dfrac{5\pi}{4}+k\pi$, $x=\dfrac{5\pi}{16}+k\pi$ với $k\in\mathbb{Z}$}
	{$x=\dfrac{5\pi}{8}+k\pi$, $x=\dfrac{7\pi}{24}+k\pi$ với $k\in\mathbb{Z}$}
	\loigiai
	{
		Ta có
		\allowdisplaybreaks
		\begin{eqnarray*}
			&& 2\sqrt{3}\sin\left(x-\dfrac{\pi}{8}\right)\cos\left(x-\dfrac{\pi}{8}\right)+2\cos^2\left(x-\dfrac{\pi}{8}\right) = \sqrt{3}+1\\
			&\Leftrightarrow & \sqrt{3}\sin\left(2x-\dfrac{\pi}{4}\right)+1+\cos\left(2x-\dfrac{\pi}{4}\right) = \sqrt{3}+1\\
			&\Leftrightarrow & \sqrt{3}\sin\left(2x-\dfrac{\pi}{4}\right)+\cos\left(2x-\dfrac{\pi}{4}\right) = \sqrt{3}\\
			&\Leftrightarrow & \dfrac{\sqrt{3}}{2}\sin\left(2x-\dfrac{\pi}{4}\right)+\dfrac{1}{2}\cos\left(2x-\dfrac{\pi}{4}\right) = \dfrac{\sqrt{3}}{2}\\
			&\Leftrightarrow & \sin\left(2x-\dfrac{\pi}{12}\right) = \dfrac{\sqrt{3}}{2}\\
			&\Leftrightarrow & \left[\begin{aligned}&2x-\dfrac{\pi}{12}=\dfrac{\pi}{3}+k2\pi,k\in\mathbb{Z} \\&2x-\dfrac{\pi}{12}=\dfrac{2\pi}{3}+k2\pi,k\in\mathbb{Z}\end{aligned}\right.\\
			&\Leftrightarrow & \left[\begin{aligned}&x=\dfrac{5\pi}{24}+k\pi,k\in\mathbb{Z} \\&x=\dfrac{3\pi}{8}+k\pi,k\in\mathbb{Z}.\end{aligned}\right.
		\end{eqnarray*}
		Vậy phương trình đã cho có nghiệm $x=\dfrac{5\pi}{24}+k\pi$, $x=\dfrac{3\pi}{8}+k\pi$ với $k\in\mathbb{Z}$.
	}
\end{ex}
\begin{ex}%[Câu 49]%[1K1G4-5]
	Nghiệm dương nhỏ nhất của phương trình $\sin x+\sin 2x=\cos x+2\cos^2 x$ là
	\choice{$\dfrac{\pi}{6}$}{$\dfrac{\pi}{3}$}{$2\dfrac{\pi}{3}$}{\True$\dfrac{\pi}{4}$}
	\loigiai{\begin{eqnarray*}
			& &\sin x+\sin 2x=\cos x+2\cos^2 x\\
			&\Leftrightarrow & \sin x + 2\sin x\cos x = \cos x\left(2\cos x + 1\right)\\
			&\Leftrightarrow & \sin x\left(2\cos x + 1\right) = \cos x\left(2\cos x + 1\right)\\
			&\Leftrightarrow & \hoac{&\cos x = - \dfrac{1}{2}\\&\sin x = \cos x}\\
			&\Leftrightarrow & \hoac{&x = \pm \dfrac{2\pi}{3} + k2\pi\\&x = \dfrac{\pi}{4} + k\pi} \quad \left(k \in \mathbb{Z}\right).
		\end{eqnarray*}
		Khi đó nghiệm dương nhỏ nhất của phương trình là $x = \dfrac{\pi}{4}$.}
\end{ex}
\begin{ex}%[Câu 50]%[1K1G4-5]
	Số nghiệm của phương trình $ \dfrac{2\sin x-1}{2\sin^2x+\sin x-1}=2 $ trong khoảng $ \left(\dfrac{\pi}{2}; \dfrac{7\pi}{2}\right) $ là
	\choice
	{$ 5 $}
	{$ 2 $}
	{$ 4 $}
	{\True $ 3 $}
	\loigiai{
		Điều kiện $ 2\sin^2 x+\sin x-1\neq 0\Leftrightarrow\heva{& \sin x\neq -1\\ & \sin x\neq \dfrac{1}{2}} $.\\
		Khi đó phương trình đã cho tương đương với \begin{eqnarray*}
			& &2\sin x-1=4\sin^2 x+2\sin x-2\\
			& \Leftrightarrow & 4\sin^2 x=1\\
			&\Leftrightarrow &\hoac{& \sin x=\dfrac{1}{2}\ (\text{không thỏa mãn điều kiện})\\
				&\sin x=-\dfrac{1}{2}\ (\text{thỏa mãn điều kiện})}\\
			&\Leftrightarrow & \hoac{& x=-\dfrac{\pi}{6}+k2\pi\\ & x=\dfrac{7\pi}{6}+k2\pi},\ k\in\mathbb{Z}.
		\end{eqnarray*}
		\begin{itemize}
			\item Trường hợp $ x=-\dfrac{\pi}{6}+k2\pi $. Khi đó, $\begin{aligned}[t]
				x\in \left(\dfrac{\pi}{2}; \dfrac{7\pi}{2}\right)&\Leftrightarrow \dfrac{\pi}{2}< -\dfrac{\pi}{6}+k2\pi<\dfrac{7\pi}{2}\\
				&\Leftrightarrow \dfrac{2\pi}{3}<k2\pi<\dfrac{11\pi}{3}\\
				&\Leftrightarrow \dfrac{1}{3}<k<\dfrac{11}{6}\\
				&\Leftrightarrow k=1 \ (\text{vì}\; k\in\mathbb{Z}).
			\end{aligned} $
			\item Trường hợp $ x=\dfrac{7\pi}{6}+k2\pi $. Khi đó, $\begin{aligned}[t]
				x\in \left(\dfrac{\pi}{2}; \dfrac{7\pi}{2}\right)&\Leftrightarrow \dfrac{\pi}{2}< \dfrac{7\pi}{6}+k2\pi<\dfrac{7\pi}{2}\\
				&\Leftrightarrow -\dfrac{\pi}{3}<k2\pi<\dfrac{7\pi}{3}\\
				&\Leftrightarrow -\dfrac{1}{6}<k<\dfrac{7}{6}\\
				&\Leftrightarrow k\in\{0; 1\} \ (\text{vì}\; k\in\mathbb{Z}).
			\end{aligned} $
		\end{itemize}
		Vậy phương trình đã cho có tất cả 3 nghiệm thuộc khoảng $ \left(\dfrac{\pi}{2}; \dfrac{7\pi}{2}\right) $.
	}
\end{ex}
\Closesolutionfile{ans}
% \setcounter{deso}{0}
\begin{name}
	{\tenchude}
	{ĐỀ ÔN TẬP CHƯƠNG I}
	{LỚP TOÁN THẦY PHÁT}
	{\thoigian}
\end{name}
\TN
%Câu 1
\begin{ex}
	Cho góc lượng giác $\alpha $. Mệnh đề nào sau đây đúng?
	\choice
	{$\sin \left(-\alpha\right)=\sin \alpha $}
	{$\cos \left(-\alpha\right)=-\cos \alpha $}
	{$\tan \left(-\alpha\right)=\tan \alpha $}
	{\True $\cot \left(-\alpha\right)=-\cot \alpha $}
	\loigiai{
		Dựa vào tính chất của hai góc đối nhau nên $\cot \left(-\alpha\right)=-\cot \alpha $
	}
\end{ex}
%Câu 2
\begin{ex}
	Giá trị $\cos 75^\circ $ là :
	\choice
	{$\dfrac{\sqrt{6}+\sqrt{2}}{4}$}
	{$\dfrac{\sqrt{6}-\sqrt{2}}{2}$}
	{\True $\dfrac{\sqrt{6}-\sqrt{2}}{4}$}
	{$\dfrac{\sqrt{6}+\sqrt{2}}{2}$}
	\loigiai{
		Ta có $\cos 75^\circ =\cos \left(30^\circ+45^\circ \right)=\cos 30^\circ \cos 45^\circ -\sin 30^\circ \sin 45^\circ=\dfrac{\sqrt{6}-\sqrt{2}}{4}$
	}
\end{ex}
%Câu 3
\begin{ex}
	Cho $\sin \alpha =\dfrac{5}{13}$ với $\dfrac{\pi }{2}<\alpha <\pi $. Mệnh đề nào sau đây đúng?
	\choice
	{$\cos \alpha =\dfrac{12}{13}$}
	{$\cos \alpha =\dfrac{8}{13}$}
	{$\cos \alpha =-\dfrac{8}{13}$}
	{\True $\cos \alpha =-\dfrac{12}{13}$}
	\loigiai{
	Ta có $\cos \alpha =\pm \sqrt{1-{{\sin }^2}\alpha }=\pm \dfrac{12}{13}$. Do $\dfrac{\pi }{2}<\alpha <\pi $ nên $\cos \alpha =-\dfrac{12}{13}$
	}
\end{ex}
%Câu 4
\begin{ex}
	Cho các góc $\alpha $, $\beta $ thỏa mãn $\alpha ,\beta \in \left(\dfrac{\pi }{2};\pi\right)$ và $\sin \alpha =\dfrac{1}{3}$, $\cos \beta =-\dfrac{2}{3}$. Tính $\sin \left(\alpha +\beta\right)$.
	\choice
	{\True $\sin \left(\alpha +\beta\right)=-\dfrac{2+2\sqrt{10}}{9}$}
	{$\sin \left(\alpha +\beta\right)=\dfrac{2\sqrt{10}-2}{9}$}
	{$\sin \left(\alpha +\beta\right)=\dfrac{\sqrt{5}-4\sqrt{2}}{9}$}
	{$\sin \left(\alpha +\beta\right)=\dfrac{\sqrt{5}+4\sqrt{2}}{9}$}
	\loigiai{
	Do $\alpha ,\beta \in \left(\dfrac{\pi }{2};\pi\right)$ nên có: $\heva{& \cos \alpha <0 \\& \sin \beta >0}$.\\
	Ta có $\cos \alpha =-\ \sqrt{1-{{\sin }^2}\alpha }=-\ \sqrt{1-\dfrac{1}{9}}=-\ \dfrac{2\sqrt{2}}{3}$ và $\sin \beta =\sqrt{1-{{\cos }^2}\beta }=\sqrt{1-\dfrac{4}{9}}=\dfrac{\sqrt{5}}{3}$.\\
	Suy ra $\sin \left(\alpha +\beta\right)=\sin \alpha \cdot \cos \beta +\cos \alpha \cdot \sin \beta =\dfrac{1}{3} \cdot \left(-\dfrac{2}{3}\right)+\left(-\dfrac{2\sqrt{2}}{3}\right) \cdot \dfrac{\sqrt{5}}{3}=-\ \dfrac{2+2\sqrt{10}}{9}$.\\
	Vậy $\sin \left(\alpha +\beta\right)=-\ \dfrac{2+2\sqrt{10}}{9}$
	}
\end{ex}
%Câu 5
\begin{ex}
	Biết $\sin \alpha +\text{cos}\alpha =m$. Tính $P=\text{cos}\left(\alpha -\dfrac{\pi }{4}\right)$ theo $m$.
	\choice
	{$P=2m$}
	{$P=\dfrac{m}{2}$}
	{\True $P=\dfrac{m}{\sqrt{2}}$}
	{$P=m\sqrt{2}$}
	\loigiai{
	Ta có $P=\text{cos}\left(\alpha -\dfrac{\pi }{4}\right)=\text{cos}\alpha \cdot \cos \dfrac{\pi }{4}+\sin \alpha \sin \dfrac{\pi }{4}=\dfrac{1}{\sqrt{2}}\text{cos}\alpha +\dfrac{1}{\sqrt{2}}\sin \alpha $\\
	$\Rightarrow P=\dfrac{1}{\sqrt{2}}\left(\sin \alpha +\text{cos}\alpha\right)=\dfrac{m}{\sqrt{2}}$
	}
\end{ex}
%Câu 6
\begin{ex}
	Cho $x=\tan \alpha $. Tính $\sin 2\alpha $ theo $x$.
	\choice
	{$2x\sqrt{1+x^2}$}
	{$\dfrac{1-x^2}{1+x^2}$}
	{$\dfrac{2x}{1-x^2}$}
	{\True $\dfrac{2x}{1+x^2}$}
	\loigiai{
	Ta có $\sin 2\alpha =2\sin \alpha \cdot \cos\alpha =2\dfrac{\sin \alpha }{\cos\alpha }\cdot \cos^2\alpha =2\tan \alpha \cdot \dfrac{1}{1+{{\tan }^2}\alpha }=\dfrac{2x}{1+x^2}$
	}
\end{ex}
%Câu 7
\begin{ex}
	Tập xác định của hàm số $y=\cot x$ là
	\choice
	{$D=\mathbb{R}$}
	{$D=\mathbb{R}\backslash \left\{ k\dfrac{\pi }{2}\left| k\in \mathbb{Z} \right. \right\}$}
	{$D=\mathbb{R}\backslash \left\{ \pi +k\dfrac{\pi }{2}\left| k\in \mathbb{Z} \right. \right\}$}
	{\True $D=\mathbb{R}\backslash \left\{ k\pi \left| k\in \mathbb{Z} \right. \right\}$}
	\loigiai{
		Điều kiện: $\sin x\ne 0\Leftrightarrow x\ne k\pi \left(k\in \mathbb{Z}\right)$.\\
		Do đó, tập xác định của hàm số $y=\cot x$ là $D=\mathbb{R}\backslash \left\{ k\pi \left| k\in \mathbb{Z} \right. \right\}$
	}
\end{ex}
%Câu 8
\begin{ex}
	Trên khoảng $\left(-\pi ;\pi\right)$, hàm số $y=\sin x$ nghịch biến trên khoảng nào sau đây?
	\choice
	{$\left(-\pi ;0\right)$}
	{$\left(-\dfrac{\pi }{2};\dfrac{\pi }{2}\right)$}
	{$\left(0;\pi\right)$}
	{\True $\left(\dfrac{\pi }{2};\pi\right)$}
	\loigiai{
		Hàm số $y=\sin x$ nghịch biến trong khoảng $\left(\dfrac{\pi }{2};\pi\right)$
	}
\end{ex}
%Câu 9
\begin{ex}
	Hàm số $y={{\sin }^2}2x-{{\cos }^2}2x$ tuần hoàn với chu kỳ bằng
	\choice
	{$2\pi $}
	{$\pi $}
	{\True $\dfrac{\pi }{2}$}
	{$\dfrac{\pi }{4}$}
	\loigiai{
	Ta có $y={{\sin }^2}2x-{{\cos }^2}2x=-\cos 4x$. Vậy hàm số đã cho tuần hoàn với chu kỳ $\dfrac{2\pi }{4}=\dfrac{\pi }{2}$
	}
\end{ex}
%Câu 10
\begin{ex}
	Nghiệm của phương trình $2\sin x+1=0$ là
	\choice
	{$x=\dfrac{\pm \pi }{6}+k2\pi ,k\in \mathbb{Z}$}
	{$x=\dfrac{\pi }{6}+k2\pi ,k\in \mathbb{Z}$}
	{$x=\dfrac{7\pi }{6}+k2\pi ,k\in \mathbb{Z}$}
	{\True $\hoac{& x=\dfrac{-\pi }{6}+k2\pi \\& x=\dfrac{7\pi }{6}+k2\pi},k\in \mathbb{Z}$}
	\loigiai{
		Ta có: $2\sin x+1=0\Leftrightarrow \sin x=\dfrac{-1}{2}\Leftrightarrow \hoac{& x=\dfrac{-\pi }{6}+k2\pi \\& x=\dfrac{7\pi }{6}+k2\pi},k\in \mathbb{Z}$
	}
\end{ex}
%Câu 11
\begin{ex}
	Phương trình nào dưới đây vô nghiệm.
	\choice
	{$\cos x=\dfrac{1}{2}$}
	{\True $\sin x-\cos x=2$}
	{$\sin (5x+1)=1$}
	{$\sin x+\sqrt{3}\cos x=1$}
	\loigiai{
		Chú ý\\
		- $\left| \sin \alpha \right|\le 1,\forall \alpha \in \mathbb{R}$ và $\left| \cos \alpha \right|\le 1,\forall \alpha \in \mathbb{R}$ nên các phương trình ở đáp án A, C có nghiệm.\\
		- Phương trình $a\sin x+b\cos x=c$ có nghiệm khi $a^2+b^2\ge c^2$, ta kiểm tra được phương trình đáp án B vô nghiệm, đáp án D có nghiệm
	}
\end{ex}
%Câu 12
\begin{ex}
	Cho phương trình $2\tan x-3=\dfrac{-2}{\tan x+1}$. Gọi $S$ là tập hợp các nghiệm của phương trình thuộc khoảng $\left(0;\dfrac{\pi }{2}\right)$. Tổng các phần tử của $S$ là
	\choice
	{$0$}
	{$\dfrac{\pi }{3}$}
	{\True $\dfrac{\pi }{4}$}
	{$1$}
	\loigiai{
		Điều kiện : $\cos x\ne 0,\tan x\ne -1$.\\
		Vì $x\in \left(0;\dfrac{\pi }{2}\right)\Rightarrow \tan x>0$.\\
		Phương trình ban đầu tương đương\\
		$\begin{aligned}
				& \Leftrightarrow \left(2\tan x-3\right)\left(\tan x+1\right)=-2\Leftrightarrow 2{{\tan }^2}x-\tan x-3=-2 \\& \Leftrightarrow 2{{\tan }^2}x-\tan x-1=0 \end{aligned}$\\
		$\Leftrightarrow \hoac{& \tan x=1\begin{matrix}\\
					{} & (TM) \\\\
				\end{matrix} \\& \tan x=\dfrac{-1}{2}(L)}$\\
		+ Với $\tan x=1\Leftrightarrow x=\dfrac{\pi }{4}+k\pi ,k\in \mathbb{Z}$. Vì $x\in \left(0;\dfrac{\pi }{2}\right)$ nên $x=\dfrac{\pi }{4}$.\\
		Vậy $S=\left\{ \dfrac{\pi }{4} \right\}$ và tổng các phần tử của $S$ là $\dfrac{\pi }{4}$
	}
\end{ex}
\TNTF
%Câu 13
\begin{ex}
	Xét tính đúng sai của các mệnh đề sau:
	\choiceTF
	{${{\sin }^2}x=\dfrac{1+\sin 2x}{2}$}
	{\True Nếu $\cos \alpha =\dfrac{1}{3}$ thì $\cos 2\alpha =-\dfrac{7}{9}$}
	{\True Nếu $\sin x=\dfrac{3}{4}$ với $x\in \left(0;\dfrac{\pi }{2}\right)$ thì $\sin 2x=\dfrac{3\sqrt{7}}{8}$}
	{\True Cho $\cos \alpha =\dfrac{2}{3}$ với $\alpha \in \left(-\dfrac{\pi }{2};0\right)$ biết $\tan \left(\alpha +\dfrac{\pi }{4}\right)=a+b\sqrt{c}$, $c$ là số nguyên tố $\left(a,b,c\in \mathbb{Z},c\ge 0\right)$ Khi đó $a+b+c=0$}
	\loigiai{
	a) ${{\sin }^2}x=\dfrac{1-\cos 2x}{2}$\\
	b) $\cos 2\alpha =2{{\cos }^2}\alpha -1=2{{\left(\dfrac{1}{3}\right)}^2}-1=\dfrac{-7}{9}$\\
	c) Ta có ${{\cos }^2}x=1-{{\sin }^2}x=1-{{\left(\dfrac{3}{4}\right)}^2}=\dfrac{7}{16}$.\\
	Vì $x\in \left(0;\dfrac{\pi }{2}\right)$ nên $\cos x>0\Rightarrow \cos x=\dfrac{\sqrt{7}}{4}$ suy ra $\sin 2x=2\sin x \cdot \cos x=2\cdot \dfrac{\sqrt{7}}{4}\cdot \dfrac{3}{4}=\dfrac{3\sqrt{7}}{8}$\\
	d) Ta có ${{\tan }^2}\alpha =\dfrac{1}{{{\cos }^2}\alpha }-1=\dfrac{1}{{{\left(\dfrac{2}{3}\right)}^2}}-1=\dfrac{5}{4}$\\
	Vì $\alpha \in \left(-\dfrac{\pi }{2};0\right)$ nên $\tan \alpha <0\Rightarrow \tan \alpha =\dfrac{-\sqrt{5}}{2}$\\
	$\tan \left(\alpha +\dfrac{\pi }{4}\right)=\dfrac{\tan \alpha +\tan \dfrac{\pi }{4}}{1-\tan \alpha \cdot \tan \dfrac{\pi }{4}}=\dfrac{\dfrac{-\sqrt{5}}{2}+1}{1-\left(\dfrac{-\sqrt{5}}{2}\right) \cdot 1}=-9+4\sqrt{5}$\\
	Vậy $a=-9,b=4,c=5$ nên mệnh đề đúng
	}
\end{ex}
%Câu 14
\begin{ex}
	Biết $\cos x=\dfrac{1}{3}$ và $-\dfrac{\pi }{2}<x<0$. Khi đó: Các mệnh đề sau đúng hay sai?
	\choiceTF
	{\True $\sin \left(\dfrac{\pi }{2}-x\right)>0$}
    {$\sin 2x=\dfrac{4\sqrt{2}}{9}$}
	{\True $\cos \left(x+\dfrac{4\pi }{3}\right)=-\dfrac{1+3\sqrt{6}}{6}$}
	{\True $\sin x+\sin 3x=-\dfrac{8\sqrt{2}}{27}$}
	\loigiai{
	a) Ta có $\sin \left(\dfrac{\pi }{2}-x\right)=\cos x=\dfrac{1}{3}>0$\\
	b) Ta có ${{\sin }^2}x=1-{{\cos }^2}x=1-{{\left(\dfrac{1}{3}\right)}^2}=\dfrac{8}{9}\Rightarrow \sin x=\pm \dfrac{2\sqrt{2}}{3}$.\\
	Vì $-\dfrac{\pi }{2}<x<0$ nên $\sin x=-\dfrac{2\sqrt{2}}{3}$.\\
	Áp dụng công thức nhân đôi ta có: $\sin 2x=2\sin x\cos x=2 \cdot \left(-\dfrac{2\sqrt{2}}{3}\right) \cdot \dfrac{1}{3}=-\dfrac{4\sqrt{2}}{9}$\\
	c) $\cos \left(x+\dfrac{4\pi }{3}\right)=\cos x \cdot \cos \dfrac{4\pi }{3}-\sin x \cdot \sin \dfrac{4\pi }{3}=\dfrac{1}{3} \cdot \left(-\dfrac{1}{2}\right)-\left(-\dfrac{2\sqrt{2}}{2}\right) \cdot \left(-\dfrac{\sqrt{3}}{2}\right)=-\dfrac{1+3\sqrt{6}}{6}$\\
	d) Áp dụng công thức ta có:\\
	$\sin x+\sin 3x=2\sin 2x \cdot \cos x=2 \cdot \left(-\dfrac{4\sqrt{2}}{9}\right) \cdot \dfrac{1}{3}=-\dfrac{8\sqrt{2}}{27}$
	}
\end{ex}
%Câu 15
\begin{ex}
	Cho hàm số $f(x)=-2\sin \left(2x-\dfrac{\pi }{2}\right)+2025$. Các mệnh đề sau đúng hay sai?
	\choiceTF
	{\True Hàm số $f(x)$ có tập xác định là $\mathbb{R}$}
	{Hàm số $f(x)$ tuần hoàn với chu kì $T=2\pi $}
	{Hàm số $f(x)$ không chẵn, không lẻ}
	{\True Hàm số $f(x)$ đạt giá trị lớn nhất tại $x=k\pi ,k\in \mathbb{Z}$}
	\loigiai{
		a). Vì tập xác định của hàm $\sin $ là $\mathbb{R}$ nên hàm số $f(x)$ có tập xác định là $\mathbb{R}$.\\
		b). Ta có $-2\sin \left(2x-\dfrac{\pi }{2}\right)+2025=2\sin \left(\dfrac{\pi }{2}-2x\right)+2025=2\cos 2x+2025$.\\
		Do đó $f(x)=2\cos 2x+2025$ nên hàm số $f(x)$ tuần hoàn với chu kì $T=\dfrac{2\pi }{2}=\pi $.\\
		c) Ta có $\forall x\in \mathbb{R},-x\in \mathbb{R}$ và $f(-x)=2\cos (-2x)+2025=2\cos 2x+2025=f(x)$ nên hàm số $f(x)$ là hàm số chẵn.\\
		d) Ta có $-2\le 2\cos 2x\le 2,\forall x\in \mathbb{R}$ hay $2023\le 2\cos 2x+2025\le 2027,\forall x\in \mathbb{R}$.\\
		Do đó $f(x)=2027\Leftrightarrow \cos 2x=1\Leftrightarrow x=k\pi ,k\in \mathbb{Z}$.\\
		Vậy hàm số $f(x)$ đạt giá trị lớn nhất tại $x=k\pi ,k\in \mathbb{Z}$
	}
\end{ex}
%Câu 16
\begin{ex}
	Cho hàm số $f(x)=\dfrac{1}{{{\cos }^2}x}+\dfrac{1}{{{\sin }^2}x}$. Xét tính đúng sai của các mệnh đề sau
	\choiceTF
	{\True Hàm số đã cho là hàm số tuần hoàn}
	{\True Hàm số đã cho là hàm số chẵn}
	{Tập xác định của hàm số là $D=\mathbb{R}\backslash \left\{ \dfrac{\pi }{2}+k\pi ,k\in \mathbb{Z} \right\}$}
	{\True Giá trị nhỏ nhất của hàm số là 4}
	\loigiai{
		a) Hàm số tuần hoàn do hai hàm $y=\operatorname{sinx}$ và $y=\cos x$ cùng tuần hoàn với chu kì $2\pi $.\\
		b) Ta có $f(-x)=\dfrac{1}{{{\cos }^2}(-x)}+\dfrac{1}{{{\sin }^2}(-x)}=\dfrac{1}{{{\left(\operatorname{cosx}\right)}^2}}+\dfrac{1}{{{\left(-\sin x\right)}^2}}=\dfrac{1}{{{\cos }^2}x}+\dfrac{1}{{{\sin }^2}x}=f(x)$.\\
		Do đó hàm số đã cho là hàm số chẵn\\
		c) Hàm số xác định khi $\heva{& \operatorname{sinx}\ne 0 \\& \operatorname{cosx}\ne 0}\Leftrightarrow \sin 2x\ne 0\Leftrightarrow 2x\ne k\pi \Leftrightarrow x\ne \dfrac{k\pi }{2},k\in \mathbb{Z}$\\
		Tập xác định của hàm số là $D=\mathbb{R}\backslash \left\{ \dfrac{k\pi }{2},k\in \mathbb{Z} \right\}$.\\
		d) Khi $x\ne \dfrac{k\pi }{2},k\in \mathbb{Z}$ ta có\\
		$f(x)=\dfrac{1}{{{\cos }^2}x}+\dfrac{1}{{{\sin }^2}x}\ge 2\sqrt{\dfrac{1}{{{\cos }^2}x} \cdot \dfrac{1}{{{\sin }^2}x}}=2\sqrt{\dfrac{4}{{{\sin }^2}2x}}=\dfrac{4}{\left| \sin 2x \right|}\ge \dfrac{4}{1}=4$.\\
		Nên giá trị nhỏ nhất của hàm số là 4
	}
\end{ex}
\TNSA
%Câu 19
\begin{ex}
    Tìm tập giá trị của các hàm số $y=\sqrt{2+\cos x}-5$ là đoạn $[a;b]$. Giá trị $a+b$ (làm tròn đến hàng phần chục) là
    \shortans{-7,3}
    \loigiai{
    Vì $\cos x\ge -1\Leftrightarrow 2+\cos x\ge 1>0,\forall x\in \mathbb{R}$ nên tập xác định của hàm số là $D=\mathbb{R}$.\\
    $\forall x\in \mathbb{R}$, ta có:
    \begin{eqnarray*}
        & & -1\le \cos x\le 1 \\
        & \Leftrightarrow & 1\le 2+\cos x\le 3 \\
        & \Leftrightarrow & 1\le \sqrt{2+\cos x}\le \sqrt{3} \\
        & \Leftrightarrow & -4\le \sqrt{2+\cos x}\,-5\le \sqrt{3}-5
    \end{eqnarray*}
    Vậy tập giá trị của hàm số là $T=\left[-4;\sqrt{3}-5\right]$. Suy ra $a+b \approx -7,3$.
    }    
\end{ex}
%Câu 18
\begin{ex}
	Tổng số giờ ban ngày của ngày thứ $x$ trong một năm không nhuận được tính bởi công thức
	$g(x)=3\sin (0,0172x-1,376)+12$.
	Trong đó $x$ đại diện cho ngày trong năm, $1\le x\le 365$. Ngày $\overline{ab}$ tháng $\overline{cd}$ có số giờ ban ngày dài nhất. Số $\overline{abcd}$ bằng
	\shortans{2006}
	\loigiai{
		Ta có $-1\le \sin (0,0172x-1,376)\le 1$\\
		$-3\le 3\sin (0,0172x-1,376)\le 3$\\
		$9\le 3\sin (0,0172x-1,376)+12\le 15$\\
		Suy ra $9\le g(x)\le 15$\\
		Do đó, số giờ ban ngày dài nhất trong một ngày là 15 giờ.\\
		Ta có phương trình $3\sin (0,0172x-1,376)+12=15$\\
		$\sin (0,0172x-1,376)=1$\\
		$x\approx 171{,}3$\\
		Vậy vào khoảng ngày thứ 171 trong năm (ngày 20 tháng 6) thì số giờ ban ngày dài nhất
	}
\end{ex}
%Câu 19
\begin{ex}
	Hai thành phố có cùng kinh độ. Vĩ tuyến của thành phố A là $10^\circ $ Bắc và vĩ tuyến của thành phố B là $40^\circ $ Bắc. Giả sử bán kính trái đất là 3960 dặm. Tìm khoảng cách giữa hai thành phố (làm tròn đến chữ số hàng đơn vị)
	\shortans{2073}
	\loigiai{
		Khoảng cách từ điểm trên đường xích đạo đến thành phố B ở cùng kinh độ là $3960 \cdot \dfrac{40}{180} \cdot \pi =880\pi $ (dặm)\\
		Khoảng cách từ điểm trên đường xích đạo đến thành phố A ở cùng kinh độ là $3960 \cdot \dfrac{10}{180}\pi =220\pi $ (dặm)\\
		Khoảng cách giữa hai thành phố A và B là $880\pi -220\pi =660\pi \approx 2073$ (dặm)
	}
\end{ex}
%Câu 20
\begin{ex}
	Giả sử vận tốc $v$ (tính bằng lít/ giây) của luồng khí trong một chu kì hô hấp (tức là thời gian từ lúc bắt đầu của một nhịp thở đến khi bắt đầu của nhịp thở tiếp theo) của một người nào đó ở trạng thái nghỉ ngơi được cho bởi công thức $v=0{,}85\sin \dfrac{\pi t}{3}$, trong đó $t$ là thời gian (tính bằng giây). Biết rằng quá trình hít vào xảy ra khi $v>0$ và quá trình thở ra xảy ra khi $v<0$. Trong khoảng thời gian từ 5 đến 10 giây, khoảng thời điểm sau $a$ giây đến trước $b$ giây thì người đó hít vào. Tính $\sqrt{a+b}$ (làm tròn đến hàng phần trăm).
	\shortans{3,87}
	\loigiai{
		+) Vì quá trình hít vào xảy ra khi $v>0$ nên ta có\\
		$0{,}85\sin \dfrac{\pi t}{3}>0\Leftrightarrow \sin \dfrac{\pi t}{3}>0\Leftrightarrow \dfrac{\pi t}{3}\in \left(k2\pi ;\pi +k2\pi\right)(k\in \mathbb{Z})$\\
		$\Leftrightarrow t\in \left(6k;3+6k\right)\,\left(k\in \mathbb{Z}\right)$\\
		+) Vì $t\in [5;10]$ nên $k=1$ suy ra $t\in \left(6;9\right)$.\\
		Trong khoảng thời gian từ 5 đến 10 giây, khoảng thời điểm sau $6$ giây đến trước $9$ giây thì người đó hít vào nên $\sqrt{a+b}=\sqrt{15}\approx 3,87$.
	}
\end{ex}
%Câu 21
\begin{ex}
	Nghiệm phương trình lượng giác $\sqrt{3}\sin x-\cos x=0$ có dạng $x=\dfrac{\pi }{a}+k \cdot b\pi $ ($a$, $b$, $k\in \mathbb{Z}$, $a\ne 0$). Tính $(a+b)^4$.
	\shortans{2401}
	\loigiai{
		Phương trình tương đương\\
		$\dfrac{\sqrt{3}}{2}\sin x-\dfrac{1}{2}\cos x=0$\\
		$\Leftrightarrow \sin x\cos \dfrac{\pi }{6}-\cos x\sin \dfrac{\pi }{6}=0$\\
		$\Leftrightarrow \sin \left(x-\dfrac{\pi }{6}\right)=0$\\
		$\Leftrightarrow x-\dfrac{\pi }{6}=k\pi $ ($k\in \mathbb{Z}$)\\
		$\Leftrightarrow x=\dfrac{\pi }{6}+k\pi $.\\
		Phương trình có nghiệm là: $x=\dfrac{\pi }{6}+k\pi $ ($k\in \mathbb{Z}$).\\
		Suy ra $a=6$; $b=1$. Vậy $(a+b)^4=7^4=2401$.
	}
\end{ex}
%Câu 22
\begin{ex}
	Một vật $M$ được gắn vào đầu lò xo và dao động quanh vị trí cân bằng, toạ độ $x$ (đơn vị: cm) tại thời điểm $t$ (giây) được tính bởi công thức $x=8{,}6\sin \left(8t+\dfrac{\pi }{2}\right)$. Có $n$ thời điểm trong khoảng 2 giây đầu tiên thì $s=4{,}3$ cm. Giá trị $\sqrt[3]{n}$ (làm tròn đến hàng phần trăm)
	\shortans{1,71}
	\loigiai{
	Khi $x=4{,}3$ thì $8{,}6\sin \left(8t+\dfrac{\pi }{2}\right)=4{,}3\Rightarrow \sin \left(8t+\dfrac{\pi }{2}\right)=\dfrac{1}{2}$\\
	$\Leftrightarrow \hoac{&8t+\dfrac{\pi }{2}=\dfrac{\pi }{6}+k2\pi  \\
			&8t+\dfrac{\pi }{2}=\dfrac{5\pi }{6}+l2\pi}(k,l\in \mathbb{Z})
            \Leftrightarrow \hoac{& t=-\dfrac{\pi }{24}+k\dfrac{\pi }{4} \\
            & t=\dfrac{\pi }{24}+l\dfrac{\pi }{4} }(k,l\in \mathbb{Z})$.\\
	Vì $t\in (0;2)$ nên $\heva{& 0<-\dfrac{\pi }{24}+k\dfrac{\pi }{4}<2 \\ & 0<\dfrac{\pi }{24}+l\dfrac{\pi }{4}<2} \Leftrightarrow \heva{& \dfrac{1}{6}<k<\dfrac{8}{\pi }+\dfrac{1}{6}  \\ & -\dfrac{1}{6}<l<\dfrac{8}{\pi }-\dfrac{1}{6}}$\\
	Mà $k,l\in \mathbb{Z}$ nên $k\in \left\{ 1;2 \right\}$; $l\in \left\{ 0;1;2 \right\}$.\\
	Vậy có $5$ thời điểm thỏa mãn đề bài nên $\sqrt[3]{n}\approx 1,71$
	}
\end{ex}

% \begin{name}
	{\tenchude}
	{ĐỀ ÔN TẬP CHƯƠNG I}
	{LỚP TOÁN THẦY PHÁT}
	{\thoigian}
\end{name}
\TN
\setcounter{ex}{0}
\Opensolutionfile{ans}[ans/ans-TN-C1-De1]
\TN
%Câu 1
\begin{ex}
	Rút gọn biểu thức $M=\cos 2x \cdot \cos x+\sin 2x \cdot \sin x$ ta được kết quả là:
	\choice
	{\True $M=\cos x$}
	{$M=\cos 3x$}
	{$M=\sin x$}
	{$M=\sin 3x$}
	\loigiai{
		Ta có: $M=\cos 2x \cdot \cos x+\sin 2x \cdot \sin x=\cos (2x-x)=\cos x$
	}
\end{ex}
%Câu 2
\begin{ex}
	Đẳng thức nào không đúng với mọi $x$?
	\choice
	{$\cos^2 3x=\dfrac{1+\cos 6x}{2}$}
	{$\cos 2x=1-2\sin^2x$}
	{$\sin 2x=2\sin x\cos x$}
	{\True $\sin^2 2x=\dfrac{1+\cos 4x}{2}$}
	\loigiai{
		Ta có $\sin^2 2x=\dfrac{1-\cos 4x}{2}$
	}
\end{ex}
%Câu 3
\begin{ex}
	Góc có số đo $\dfrac{\pi }{24}$ đổi sang độ bằng
	\choice
	{$7^\circ $}
	{\True $7^\circ 3{0}'$}
	{$8^\circ $}
	{$8^\circ 3{0}'$}
	\loigiai{
		Ta có: $\dfrac{\pi }{24}=\dfrac{180^\circ }{24}=7^\circ 30'$
	}
\end{ex}
%Câu 4
\begin{ex}
	Một đường tròn có đường kính là $50$ (cm). Độ dài của cung tròn trên đường tròn có số đo là $\dfrac{\pi }{4}$ bằng (làm tròn đến hàng đơn vị)
	\choice
	{$40$ (cm)}
	{$39$ (cm)}
	{$19$ (cm)}
	{\True $20$ (cm)}
	\loigiai{
		Độ dài của cung tròn $l=\alpha \cdot R=\dfrac{\pi }{4} \cdot 25=\dfrac{25}{4}\pi \approx 20$ (cm)
	}
\end{ex}
%Câu 5
\begin{ex}
	Chọn phát biểu đúng:
	\choice
	{Các hàm số $y=\sin x$, $y=\cos x$, $y=\cot x$ đều là hàm số chẵn}
	{Các hàm số $y=\sin x$, $y=\cos x$, $y=\cot x$ đều là hàm số lẻ}
	{Các hàm số $y=\sin x$, $y=\cot x$, $y=\tan x$ đều là hàm số chẵn}
	{\True Các hàm số $y=\sin x$, $y=\cot x$, $y=\tan x$ đều là hàm số lẻ}
	\loigiai{
		Hàm số $y=\cos x$ là hàm số chẵn, hàm số $y=\sin x$, $y=\cot x$, $y=\tan x$ là các hàm số lẻ
	}
\end{ex}
%Câu 6
\begin{ex}
	Nếu $\sin x+\cos x=\dfrac{1}{2}$ thì $\sin 2x$ bằng
	\choice
	{$\dfrac{3}{4}$}
	{$\dfrac{3}{8}$}
	{$\dfrac{\sqrt{2}}{2}$}
	{\True $\dfrac{-3}{4}$}
	\loigiai{
	Do $\sin x+\cos x=\dfrac{1}{2}\Rightarrow \dfrac{1}{4}={{\left(\sin x+\cos x\right)}^2}={{\left(\sin x\right)}^2}+{{\left(\text{cosx}\right)}^2}+2\sin x \cdot \cos x$\\
	$\Rightarrow \dfrac{1}{4}=1+\sin 2x\Rightarrow \sin 2x =\dfrac{-3}{4}$
	}
\end{ex}
%Câu 7
\begin{ex}
	Một con lắc lò xo sau khi được kéo xuống dưới vị trí cân bằng $4$ cm và thả ra thì nó dao động điều hòa với phương trình: $y=-4\cos 8t$ (cm). Biên độ $A$ cm và chu kỳ $T$ của dao động là
	\choice
	{\True $A=4$ cm, $T=\dfrac{\pi }{4}$}
	{$A=4$ cm, $T=\dfrac{\pi }{2}$}
	{$A=8$ cm, $T=\dfrac{\pi }{4}$}
	{$A=4$ cm, $T=2\pi $}
	\loigiai{
		Biên độ của dao động là: $A=|-4|=4$ (cm).\\
		Chu kỳ của dao động là:$T=\dfrac{2\pi }{|8|}=\dfrac{\pi }{4}$
	}
\end{ex}
%Câu 8
\begin{ex}
	Hãy tìm tập tất cả các giá trị của $m$ để phương trình $\left| \sin x \right|=m$ có nghiệm?
	\choice
	{$-1\le m\le 1$}
	{$-1\le m\le 0$}
	{$-1<m<0$}
	{\True $0\le m\le 1$}
	\loigiai{
		Vì $0<= |\sin x|<=1, \forall x \in \mathbb{R}$ nên phương trình $\left| \sin x \right|=m$ có nghiệm khi và chỉ khi $0\le m\le 1$.
	}
\end{ex}
%Câu 9
\begin{ex}
	Nghiệm của phương trình $2\sin \left(4x-\dfrac{\pi }{3}\right)-1=0$ là:
	\choice
	{$x=\pi +k2\pi ;x=k\dfrac{\pi }{2}\ (k \in \mathbb{Z})$}
	{$x=\dfrac{\pi }{8}+k\dfrac{\pi }{2};x=\dfrac{7\pi }{24}+k\dfrac{\pi }{2}\ (k \in \mathbb{Z})$}
	{$x=k2\pi ;x=\dfrac{\pi }{2}+k2\pi\ (k \in \mathbb{Z})$}
	{$x=k\pi ;x=\pi +k2\pi\ (k \in \mathbb{Z})$}
	\loigiai{
		$2\sin \left(4x-\dfrac{\pi }{3}\right)-1=0\Leftrightarrow \sin \left(4x-\dfrac{\pi }{3}\right)=\dfrac{1}{2}\Leftrightarrow \hoac{& 4x-\dfrac{\pi }{3}=\dfrac{\pi }{6}+k2\pi \\& 4x-\dfrac{\pi }{3}=\pi -\dfrac{\pi }{6}+k2\pi}\Leftrightarrow \hoac{& x=\dfrac{\pi }{8}+k\dfrac{\pi }{2} \\& x=\dfrac{7\pi }{24}+k\dfrac{\pi }{2}}\left(k\in \mathbb{Z}\right)$
	}
\end{ex}
%Câu 10
\begin{ex}
	Biết $\sin \left(\alpha +\dfrac{3\pi }{2}\right)+\cos \left(\alpha +\dfrac{3\pi }{2}\right)=\sqrt{2}$. Tính $\sin \left(\alpha +\pi\right)-2\cos \left(\alpha -\pi\right)$.
	\choice
	{$\dfrac{3}{\sqrt{2}}$}
	{\True $-\dfrac{3}{\sqrt{2}}$}
	{$-\dfrac{1}{\sqrt{2}}$}
	{$\dfrac{1}{\sqrt{2}}$}
	\loigiai{
	Ta có $\sin \left(\alpha +\dfrac{3\pi }{2}\right)=\sin \left(\alpha +2\pi -\dfrac{\pi }{2}\right)=\sin \left(\alpha -\dfrac{\pi }{2}\right)=-\sin \left(\dfrac{\pi }{2}-\alpha\right)=-\cos \alpha $.\\
	$\cos \left(\alpha +\dfrac{3\pi }{2}\right)=\cos \left(\alpha +2\pi -\dfrac{\pi }{2}\right)=\cos \left(\alpha -\dfrac{\pi }{2}\right)=\cos \left(\dfrac{\pi }{2}-\alpha\right)=\sin \alpha $.\\
	Suy ra $\sin \alpha -\cos \alpha =\sqrt{2}\Rightarrow \sin \alpha =\cos \alpha +\sqrt{2}$.\\
	Vì ${{\sin }^2}\alpha +{{\cos }^2}\alpha =1\Rightarrow 2{{\cos }^2}\alpha +2\sqrt{2}\cos \alpha +2=1$\\
	$\Leftrightarrow 2{{\cos }^2}\alpha +2\sqrt{2}\cos \alpha +1=0\Leftrightarrow \cos \alpha =-\dfrac{1}{\sqrt{2}}\Rightarrow \sin \alpha =\dfrac{1}{\sqrt{2}}$.\\
	Do đó $\sin \left(\alpha +\pi\right)-2\cos \left(\alpha -\pi\right)=-\sin \alpha +2\cos \alpha =-\dfrac{3}{\sqrt{2}}$
	}
\end{ex}
%Câu 11
\begin{ex}
	Hằng ngày mực nước của con kênh lên xuống theo thủy triều. Độ sâu $h$(mét) của mực nước trong kênh được tính tại thời điểm $t$ (giờ) trong một ngày bởi công thức $h=3\cos \left(\dfrac{\pi t}{7=8}+\dfrac{\pi }{4}\right)+12$. Mực nước của kênh cao nhất khi:
	\choice
	{$t=13$(giờ)}
	{\True $t=14$(giờ)}
	{$t=15$(giờ)}
	{$t=16$(giờ)}
	\loigiai{
		Mực nước của kênh cao nhất khi $h$ lớn nhất\\
		$\Leftrightarrow \cos \left(\dfrac{\pi t}{8}+\dfrac{\pi }{4}\right)=1\Leftrightarrow \dfrac{\pi t}{8}+\dfrac{\pi }{4}=k2\pi $ với $0<t\le 24$ và $k\in \mathbb{Z}$.\\
		Lần lượt thay các đáp án, ta được đáp án B thỏa mãn.\\
		Vì với $t=14$ thì $\dfrac{\pi t}{8}+\dfrac{\pi }{4}=2\pi $ (đúng với $k=1\in \mathbb{Z}$)
	}
\end{ex}
%Câu 12
\begin{ex}
	Số giờ có ánh sáng mặt trời của một thành phố A ở vĩ độ ${{40}^{\text{o}}}$ bắc trong ngày thứ t của một năm không nhuận được cho bởi hàm số $d(t)=3\sin \left[\dfrac{\pi }{180}(t-80)\right]+12$ với $t\in \mathbb{Z}$ và $0<t\le 365$. Vào ngày nào trong năm thì thành phố A có nhiều giờ có ánh sáng mặt trời nhất?
	\choice
	{\True 170}
	{171}
	{172}
	{173}
	\loigiai{
		Ta có $d(t)=3\sin \left[\dfrac{\pi }{180}(t-80)\right]+12\le 3 \cdot 1+12=15$.\\
		Vậy thành phố A có nhiều giờ có ánh sáng mặt trời nhất khi $\sin \left[\dfrac{\pi }{180}(t-80)\right]=1\Leftrightarrow \dfrac{\pi }{180}(t-80)=\dfrac{\pi }{2}+k2\pi \Leftrightarrow t=170+360k (k\in \mathbb{Z})$.\\
		Vì $0<t\le 365$ nên $0<170+360k\le 365\Leftrightarrow -\dfrac{17}{36}<k\le \dfrac{39}{72}\Rightarrow k=0\Rightarrow t=170$.
	}
\end{ex}

\TNTF
%Câu 13
\begin{ex}
	Cho phương trình $\sin x=a$ (1).
	\choiceTF
	{\True Nếu $a>1$ thì phương trình (1) vô nghiệm}
	{Nếu $a=1$ thì phương trình (1) có nghiệm $\alpha =\dfrac{\pi }{2}+k\pi ,\left(k\in \mathbb{Z}\right)$}
	{\True Nếu $-1\le a\le 1$ thì phương trình (1) có nghiệm $\hoac{& x=\alpha +k2\pi \\ & x=\pi -\alpha +k2\pi} \left(k\in \mathbb{Z}\right)$}
	{Phương trình (1) luôn có hai điểm biểu diễn nghiệm trên đường tròn lượng giác}
	\loigiai{
		Nếu $a=1\Rightarrow \sin \alpha =1\Leftrightarrow \alpha =\dfrac{\pi }{2}+k2\pi ,\left(k\in \mathbb{Z}\right)$
	}
\end{ex}
%Câu 14
\begin{ex}
	Các mệnh đề sau đúng hay sai?
	\choiceTF
	{\True Hàm số $y=\sin \sqrt{x+4}$ có tập xác định là $D=\left[-4;+\infty\right)$}
	{Hàm số $y=\cot \left(\dfrac{\pi }{2}+x\right)$ có tập xác định là $D=\mathbb{R}$}
	{\True Hàm số $y=\sqrt{3-2\cos x}$ có tập xác định là $D=\mathbb{R}$}
	{Hàm số $y=\dfrac{1-3\cos x}{\sin x}$ có tập xác định là $D=\mathbb{R}\backslash \left\{ k\dfrac{\pi }{2},k\in \mathbb{Z} \right\}$}
	\loigiai{
	a) Hàm số xác định khi và chỉ khi $x+4\ge 0\Leftrightarrow x\ge -4$.\\
	Vậy tập xác định của hàm số là $D=\left[-4;+\infty\right)$.\\
	b) Hàm số xác định khi và chỉ khi $\sin \left(x+\dfrac{\pi }{2}\right)\ne 0\Leftrightarrow x+\dfrac{\pi }{2}\ne k\pi \Leftrightarrow x\ne -\dfrac{\pi }{2}+k\pi ;k\in \mathbb{Z}$.\\
	Vậy tập xác định của hàm số là $D=\mathbb{R}\backslash \left\{ -\dfrac{\pi }{2}+k\pi ;k\in \mathbb{Z} \right\}$.\\
	c) Hàm số xác định khi $3-2\cos x\ge 0\Leftrightarrow \cos x\le \dfrac{3}{2}$ (đúng $\forall x\in \mathbb{R}$), vì $-1\le \cos x\le 1,\forall x\in \mathbb{R}$.\\
	Vậy tập xác định của hàm là $D=\mathbb{R}$.\\
	d) Hàm số xác định khi và chỉ khi $\sin x\ne 0\Leftrightarrow x\ne k\pi \left(k\in \mathbb{Z}\right)$.\\
	Vậy tập xác định của hàm số là $D=\mathbb{R}\backslash \left\{ k\pi ,k\in \mathbb{Z} \right\}$
	}
\end{ex}
%Câu 15
\begin{ex}
	Hằng ngày mực nước của con kênh lên xuống theo thủy triều. Độ sâu $h$ (mét) của mực nước trong kênh tính theo thời gian $t$ (giờ) được cho bởi công thức $h(t)=3\cos \left(\dfrac{\pi t}{6}+\dfrac{\pi }{4}\right)+14$.
	\choiceTF
	{Công thức tuần hoàn với chu kì $T=2\pi $}
	{\True Chiều sâu của mực nước thấp nhất là $11 \text{m}$}
	{Chiều sâu của mực nước cao nhất là $14 \text{m}$}
	{\True Thời gian để mực nước cao nhất là $t=9$}
	\loigiai{
		a) Công thức có dạng $y=\cos (ax+b)$ tuần hoàn với chu kì $T=\dfrac{2\pi }{|a|}$ nên chu kì cần tìm là $T=\dfrac{2\pi }{\left| \dfrac{\pi }{6} \right|}=12$.\\
		b) Ta có $\forall t\colon -1\le \cos \left(\dfrac{\pi t}{6}+\dfrac{\pi }{4}\right)\le 1\Leftrightarrow -3\le 3\cos \left(\dfrac{\pi t}{6}+\dfrac{\pi }{4}\right)\le 3\Leftrightarrow 11\le 3\cos \left(\dfrac{\pi t}{6}+\dfrac{\pi }{4}\right)+14\le 17\Leftrightarrow 11\le h\le 17$. Vậy chiều sâu của mực nước thấp nhất là $11 \text{m}$.\\
		c) Ta có $\forall t\colon -1\le \cos \left(\dfrac{\pi t}{6}+\dfrac{\pi }{4}\right)\le 1\Leftrightarrow -3\le 3\cos \left(\dfrac{\pi t}{6}+\dfrac{\pi }{4}\right)\le 3\Leftrightarrow 11\le 3\cos \left(\dfrac{\pi t}{6}+\dfrac{\pi }{4}\right)+14\le 17\Leftrightarrow 11\le h\le 17$. Chiều sâu của mực nước cao nhất là $17 \text{m}$.\\
		d) Ta có $\forall t\colon -1\le \cos \left(\dfrac{\pi t}{6}+\dfrac{\pi }{4}\right)\le 1\Leftrightarrow -3\le 3\cos \left(\dfrac{\pi t}{6}+\dfrac{\pi }{4}\right)\le 3\Leftrightarrow 11\le 3\cos \left(\dfrac{\pi t}{6}+\dfrac{\pi }{4}\right)+14\le 17\Leftrightarrow 11\le h\le 17$. Chiều sâu của mực nước cao nhất là $17 \text{m}$.\\
		Max $h=17\Leftrightarrow \cos \left(\dfrac{\pi t}{6}+\dfrac{\pi }{4}\right)=1\Leftrightarrow \dfrac{\pi t}{6}+\dfrac{\pi }{4}=k2\pi \Leftrightarrow t=-3+12k,k\in \mathbb{Z}$.\\
		Vì thời gian không âm và $k\in \mathbb{Z}$ nên ta chọn $t=1$. Vậy thời gian ngắn nhất $t=-3+12=9$
	}
\end{ex}
%Câu 16
\begin{ex}
	Cho phương trình $\left(2\cos x-1\right)\left(\sin 2x-m\right)=0$ (1).
	\choiceTF
	{\True $x=\dfrac{7\pi }{3}$ là một nghiệm của phương trình $(1)$}
	{Khi $m=2$ thì phương trình $(1)\Leftrightarrow \hoac{& x=\pm\dfrac{\pi }{3}+k2\pi \\& x=\dfrac{\pi }{2}+l2\pi} (k,l \in \mathbb{Z})$}
	{\True Khi $m=1$ thì tập nghiệm của phương trình $(1)$ có tất cả 4 điểm biểu diễn trên đường tròn lượng giác}
	{Chỉ tìm được một giá trị của $m$ để phương trình $(1)$ có đúng hai nghiệm thuộc $\left(-\dfrac{\pi }{4};\dfrac{3\pi }{4}\right]$}
			\loigiai{
			Ta có $\left(2\cos x-1\right)\left(\sin 2x-m\right)=0\Leftrightarrow \hoac{& \cos x=\dfrac{1}{2} \\& \sin 2x=m}\Leftrightarrow \hoac{& x=\dfrac{\pi }{3}+k2\pi \\& x=-\dfrac{\pi }{3}+k2\pi \\& \sin 2x=m}$\\
			a) Thay $x=\dfrac{7\pi }{3}$ phương trình $(1)$ ta thấy thỏa mãn nên $x=\dfrac{7\pi }{3}$ là một nghiệm của phương trình $(1)$.\\
			b) Khi $m=2$ thì phương trình $(1)\Leftrightarrow \hoac{& x=\dfrac{\pi }{3}+k2\pi \\& x=-\dfrac{\pi }{3}+k2\pi} (k \in \mathbb{Z})$\\
			c) Khi $m=1$ phương trình $(1)\Leftrightarrow \hoac{& x=\dfrac{\pi }{3}+k2\pi \\& x=-\dfrac{\pi }{3}+k2\pi \\& \sin 2x=1}\Leftrightarrow \hoac{& x=\dfrac{\pi }{3}+k2\pi \\& x=-\dfrac{\pi }{3}+k2\pi \\& x=\dfrac{\pi }{4}+l\pi}$.\\
			Do đó tập nghiệm của phương trình $(1)$ có tất cả $4$ điểm biểu diễn trên đường tròn lượng giác.\\
			d) Do phương trình $(2)$ có một nghiệm $x=\dfrac{\pi }{3}$ thuộc $\left(-\dfrac{\pi }{4};\dfrac{3\pi }{4}\right]$.\\
			Do đó để phương trình $(1)$ có đúng hai nghiệm thuộc $\left(-\dfrac{\pi }{4};\dfrac{3\pi }{4}\right]$ thì phương trình $\sin 2x=m$ có 1 nghiệm thuộc $\left(-\dfrac{\pi }{4};\dfrac{3\pi }{4}\right]$ khác $\dfrac{\pi }{3}$ (*)\\
			Ta có $x\in \left(-\dfrac{\pi }{4};\dfrac{3\pi }{4}\right]\Rightarrow 2x\in \left(-\dfrac{\pi }{2};\dfrac{3\pi }{2}\right]$ hay $2x\in \left[0;2\pi\right]$\\
			Từ (*) suy ra $m=1$ hoặc $m=-1$\\
	}
\end{ex}

\TNSA
%Câu 17
\begin{ex}
	Cho góc $\alpha $ thỏa mãn $\sin \alpha =\dfrac{1}{5}$. Khi đó giá trị biểu thức $P={{\cos }^2}2x+{{\cos }^2}x$ bằng $\dfrac{a}{b}$. Tính $a+b$. Biết rằng phân số $\dfrac{a}{b}$ là phân số tối giản
	\shortans{1754}
	\loigiai{
		Biến đổi biểu thức $P$ rồi thay giá trị $\sin \alpha =\dfrac{1}{5}$ vào $P$, ta được:\\
		$\begin{aligned}
				& P={{\cos }^2}2x+{{\cos }^2}x \\& \text{ }={{\left(1-2{{\sin }^2}\alpha\right)}^2}+\left(1-{{\sin }^2}\alpha\right)={{\left(1-2 \cdot {{\left(\dfrac{1}{5}\right)}^2}\right)}^2}+\left(1-{{\left(\dfrac{1}{5}\right)}^2}\right)=\dfrac{1129}{625} \end{aligned}$\\
		$\Rightarrow \heva{& a=1129 \\& b=625}\Rightarrow a+b=1754$
	}
\end{ex}
%Câu 18
\begin{ex}
	Số điểm chung của đồ thị hàm số $y=\sin x$ và $y=\cos x$ trên $\left[ -\dfrac{\pi }{2};\dfrac{3\pi }{2} \right]$ là $n$. Giá trị $\sqrt{n}$ (làm tròn đến hàng phần trăm) bằng
	\shortans{1,41}
	\loigiai{
		Số điểm chung của đồ thị hàm số $y=\sin x$ và $y=\cos x$ trên $\left[ -\dfrac{\pi }{2};\dfrac{3\pi }{2} \right]$ bằng số nghiệm phương trình $\sin x = \cos x$ trên $\left[ -\dfrac{\pi }{2};\dfrac{3\pi }{2} \right]$.\\
		Ta có $\sin x = \cos x \Leftrightarrow \sin x - \cos x =0 \Leftrightarrow \sin \left(x-\dfrac{\pi}{4} \right)=0 \Leftrightarrow x-\dfrac{\pi}{4}=k \pi \Leftrightarrow x= \dfrac{\pi}{4} +k\pi \ (k \in \mathbb{Z})$.\\
		$x \in \left[ -\dfrac{\pi }{2};\dfrac{3\pi }{2} \right]$ nên $x \in \left\{ \dfrac{\pi}{4}; \dfrac{5\pi}{4}\right\}$.\\
		Vậy $n=2$ nên $\sqrt{n} \approx 1,41$.
	}
\end{ex}
%Câu 19
\begin{ex}
	Biết có $n$ giá trị nguyên của tham số $m$ để phương trình $\cos x=m$ có nghiệm. Giá trị $\sqrt{n}$ (làm tròn đến hàng phần trăm) bằng
	\shortans{1,73}
	\loigiai{
		$\cos x=m$ có nghiệm $\Leftrightarrow -1\le m\le 1$. Mà $m\in \mathbb{Z}\Rightarrow m\in \left\{ -1;0;1 \right\}$. Vậy $\sqrt{n}\approx 1{,}73$
	}
\end{ex}
%Câu 20
\begin{ex}
	Biết $x=x_0$ là nghiệm duy nhất của phương trình $2\sin \left(x-\dfrac{\pi }{6}\right)+2=0$ trên khoảng $\left(0;2\pi\right)$. Giá trị $x_0$ (làm tròn đến hàng phần trăm) bằng
	\shortans{5,24}
	\loigiai{
		Ta có: $2\sin \left(x-\dfrac{\pi }{6}\right)+2=0\Leftrightarrow \sin \left(x-\dfrac{\pi }{6}\right)=-1\Leftrightarrow x=-\dfrac{\pi }{3}+k2\pi ,k\in \mathbb{Z}$\\
		Do $x\in \left(0;2\pi\right)$ nên $0<-\dfrac{\pi }{3}+k2\pi <2\pi \Leftrightarrow \dfrac{1}{6}<k<\dfrac{7}{6}\Leftrightarrow k=1$.\\
		Vậy phương trình có một nghiệm $x=\dfrac{5\pi }{3}\approx 5{,}24$
	}
\end{ex}
%Câu 21
\begin{ex}
	Gọi $M$ và $m$ lần lượt là giá trị lớn nhất và giá trị nhỏ nhất của hàm số $y=\sin x+\sqrt{3}\cos x+\sqrt{2}$. Tính $M^2m$ (làm tròn đến hàng phần trăm)
	\shortans{6,83}
	\loigiai{
		Ta có $y=\sin x+\sqrt{3}\cos x+\sqrt{2}=2\left(\dfrac{1}{2}\sin x+\dfrac{\sqrt{3}}{2}\cos x\right)+\sqrt{2}=2\sin \left(x+\dfrac{\pi }{3}\right)+\sqrt{2}$.\\
		Suy ra $M=2+\sqrt{2}$, $m=-2+\sqrt{2}$. Nên $M^2m\approx 6{,}83$
	}
\end{ex}
%Câu 22
\begin{ex}
	Mùa xuân ở Hội Lim (tỉnh Bắc Ninh) thường có trò chơi đu. Khi người chơi đu nhún đều, cây đu sẽ đưa người chơi đu dao động qua lại vị trí cân bằng. Nghiên cứu trò chơi này, người ta thấy khoảng cách $h$ (mét) được tính từ vị trí chân người chơi đu đến vị trí cân bằng được biểu diễn bởi hệ thức $h=|d|$ với $d=3\cos \left[\dfrac{\pi }{3}(2t-1)\right]$ ($t\ge 0$ và được tính bằng giây), trong đó ta quy ước $d>0$ khi vị trí cân bằng ở về phía sau lưng người chơi đu và $d<0$ trong trường hợp ngược lại.
	Biết $t_1$, $t_2$ lần lượt là thời điểm đầu tiên người đu ở vị trí phía sau lưng và vị trí phía trước vị trí cân bằng $1{,}5$ mét. Giá trị $t_1+t_2^2$ (làm tròn đến hàng phần trăm) bằng
	\shortans{3,25}
	\loigiai{
	Người chơi cách vị trí cân bằng 1 mét khi $3\cos \left[\dfrac{\pi }{3}(2t-1)\right]=\pm 1{,}5$\\
	$\Leftrightarrow \cos^2\left[\dfrac{\pi }{3}(2t-1)\right]=\dfrac{1}{4}\Leftrightarrow \cos \left[\dfrac{2\pi }{3}(2t-1)\right]=-\dfrac{1}{2}$ $\Leftrightarrow \hoac{& \dfrac{2\pi }{3}(2t-1)=\dfrac{2\pi }{3}+k2\pi \\& \dfrac{2\pi }{3}(2t-1)=-\dfrac{2\pi }{3}+k2\pi} \left(k\in \mathbb{Z}\right)
		\Leftrightarrow \hoac{& t=1+\dfrac{3k}{2} \\& t=\dfrac{3k}{2}}\left(k\in \mathbb{Z}\right)$.\\
	Vì $t>0$ nên $t_1=1$ và $t_2=1{,}5$. Vậy $t_1+t_2^2=3{,}25$
	}
\end{ex}

\Closesolutionfile{ans}

\indapan{10}{ans/ans-SA-C1-De1}
% \section*{BT ÔN TẬP CHƯƠNG 1}
\setcounter{ex}{0}\setcounter{bt}{0}
\Opensolutionfile{ans}[ans/ans1C3-CD-1]
\noindent\textbf{I. PHẦN TRẮC NGHIỆM:}
\begin{ex}%[1T1B2-3]
        Tính tổng $S=\sin^25^{\circ}+\sin^210^{\circ}+\sin^215^{\circ}+ \cdots +\sin^285^{\circ}$.
        \choice
        {$S=\dfrac{19}{2}$}
        {\True $S=\dfrac{17}{2}$}
        {$S=8$}
        {$S=9$}
        \loigiai{
            \begin{align*}
                S&=\sin^25^{\circ}+\sin^210^{\circ}+\sin^215^{\circ}+ \cdots +\sin^285^{\circ}\\
                &=\left(\sin^25^{\circ}+\sin^285^{\circ}\right)+\left(\sin^210^{\circ}+\sin^280^{\circ}\right)+ \cdots +\left(\sin^240^{\circ}+\sin^250^{\circ}\right)+\sin^245^{\circ} \\
                &=8+\dfrac{1}{2}=\dfrac{17}{2}.
        \end{align*}}
    \end{ex}

\begin{ex}%[1C1Y1-2]
Cho góc lượng giác với tia đầu và tia cuối như trong hình. Tên của góc lượng giác là
    \begin{center}
        \begin{tikzpicture}[scale=1, font=\footnotesize, line join=round, line cap=round, >=stealth]
            \begin{axis}[
                axis line style={draw=none},
                axis lines=middle,
                axis equal image,
                enlargelimits,
                xtick=\empty,
                ytick=\empty,
                data cs=polar,
                samples=200,
                thick,
                line cap=round,
                line join=round,
                >=stealth
                ]
                \addplot [smooth, domain=0:390,->] {1+x/5000};
                \addplot [smooth, domain=420:450,->] {1.5+x/5000} node[above,midway]{$+$};
                \addplot [mark=none] (0.475,0) node [below left] {$O$};
                \addplot [mark=none] coordinates {(0,0) (390,3+390/5000)} node[below]{$y$};
                \addplot [mark=none] coordinates {(0,0) (0,3+0/5000)} node[below]{$x$};
                \addplot [mark=none,dashed] coordinates {(0,0) (420,3+420/5000)} node[right]{$m$};
            \end{axis}
        \end{tikzpicture}
    \end{center}
\choice
{\True $(Ox,Oy)$}
{$(Oy,Ox)$}
{$(Om,Oy)$}
{$(Om,Ox)$}
    \loigiai{
        Trong hình, góc lượng giác là $(Ox,Oy)$ với tia đầu $Ox$ và tia cuối $Oy$.
    }
    \end{ex}

\begin{ex}%[1T1B2-2]
        Cho $\tan a=\dfrac{2}{3}$, $5\pi <a<\dfrac{11\pi}{2}$. Khi đó $\cos \left(a+\dfrac{\pi}{3}\right)$ bằng
        \choice
        {$\dfrac{2\sqrt{3}+3}{2\sqrt{13}}$}
        {\True $\dfrac{2\sqrt{3}-3}{2\sqrt{13}}$}
        {$\dfrac{-2\sqrt{3}+3}{2\sqrt{13}}$}
        {$\dfrac{-2\sqrt{3}-3}{2\sqrt{13}}$}
        \loigiai{
            Ta có $\cos^2a=\dfrac{1}{1+\tan^2 a}=\dfrac{1}{1+\dfrac{4}{9}}=\dfrac{9}{13}$.\\
            Vì $5\pi <a<\dfrac{11\pi}{2}$ nên $\cos a<0$ và $\sin a<0$.\\
             Do đó, $\cos a=-\dfrac{3\sqrt{13}}{13}$ và $\sin a=-\dfrac{2\sqrt{13}}{13}$.\\
            Vậy $\cos \left(a+\dfrac{\pi}{3}\right)=\dfrac{1}{2}\cos a-\dfrac{\sqrt{3}}{2}\sin a=\dfrac{-3+2\sqrt{3}}{2\sqrt{13}}$.}
    \end{ex}

\begin{ex}%[Tex hóa SGK CD-CT,T12/22, TVN-006]%[1K1Y1-8]
    Trong các khẳng định sau, khẳng định  nào là \textbf{sai}?
    \choice
    {$\sin(\pi-\alpha)=\sin\alpha$}
    {\True $\cos(\pi-\alpha)=\cos \alpha$}
    {$\sin(\pi+\alpha)=-\sin\alpha$}
    {$\cos(\pi+\alpha)=-\cos \alpha$}
    \loigiai{
        Ta có $\cos(\pi-\alpha)=-\cos \alpha$ nên $\cos(\pi-\alpha)=\cos \alpha$ là khẳng định \textbf{sai}.
    }
\end{ex}

\begin{ex}%[1C1B1-2]
    Cho góc lượng giác gốc $O$ có tia đầu $Ou$, tia cuối $Ov$ và có số đo $\dfrac{2\pi}{3}$. Cho góc lượng giác $(O'u',O'v')$ có tia đầu $O'u'\equiv Ou$, tia cuối $O'v'\equiv Ov$. Viết công thức biểu thị số đo góc lượng giác $(O'u',O'v')$.
\choice
{$(O'u',Ov')=\dfrac{\pi}{3}+k2\pi\ (k\in \mathbb{Z})$}
{$(O'u',Ov')=\dfrac{4\pi}{3}+k2\pi\ (k\in \mathbb{Z})$}
{\True $(O'u',Ov')=\dfrac{2\pi}{3}+k2\pi\ (k\in \mathbb{Z})$}
{$(O'u',Ov')=-\dfrac{\pi}{3}+k2\pi\ (k\in \mathbb{Z})$}
    \loigiai{
        Ta có $(O'u',Ov')=(Ou,Ov)+k2\pi=\dfrac{2\pi}{3}+k2\pi\ (k\in \mathbb{Z})$.
    }
\end{ex}

\begin{ex}%[Tex hóa SGK CD-CT,T12/22, TVN-006]%[1K1B2-3]
    Rút gọn biểu thức $M=\cos(a+b)\cos(a-b)-\sin (a+b)\sin(a-b)$, ta được
    \choice
    {$M=\sin 4a$}
    {$M=1-2\cos^2a$}
    {\True $M=1-2\sin^2a$}
    {$M=\cos 4a$}
    \loigiai{
        Ta có
        \allowdisplaybreaks
        \begin{eqnarray*}
            M&=&\cos(a+b)\cos(a-b)-\sin (a+b)\sin(a-b)\\
            &=&\dfrac{1}{2}\left(\cos2a+\cos 2b\right)+\dfrac{1}{2}\left(\cos2a-\cos 2b\right)\\
            &=&\cos 2a\\
            &=&1-2\sin^2a.
        \end{eqnarray*}
    }
\end{ex}

\begin{ex}%[1T1B5-3]
        Tập nghiệm của phương trình $3\cos\left(3x-\dfrac{\pi}{3}\right)=0$ là
        \choice
        {$\left\{\dfrac{\pi}{2}+k\pi, k \in \mathbb{Z}\right\}$}
        {$\left\{\dfrac{5\pi}{6}+k 2\pi, k \in \mathbb{Z}\right\}$}
        {$\left\{\dfrac{5\pi}{18}+\dfrac{k 2\pi}{3}, k \in \mathbb{Z}\right\}$}
        {\True $\left\{\dfrac{5\pi}{18}+\dfrac{k\pi}{3}, k \in \mathbb{Z}\right\}$}
        \loigiai{
            $3\cos\left(3x-\dfrac{\pi}{3}\right)=0\Leftrightarrow 3x-\dfrac{\pi}{3}=\dfrac{\pi}{2}+k\pi\Leftrightarrow x=\dfrac{5\pi}{18}+\dfrac{k\pi}{3}, k\in\mathbb{Z}$.
            Tập nghiệm phương trình $S=\left\{\dfrac{5\pi}{18}+\dfrac{k\pi}{3}, k\in\mathbb{Z}\right\}$.
        }
    \end{ex}

\begin{ex}%[1T1B6-3]
        Phương trình $\sqrt{3}\sin x+\cos x=1$ tương đương với phương trình nào sau đây?
        \choice
        {$\cos \left( x+\dfrac{\pi}{6}\right) =\dfrac{1}{2}$}
        {$\sin \left( x+\dfrac{\pi}{3}\right) =\dfrac{1}{2}$}
        {\True $\cos \left( x-\dfrac{\pi}{3}\right) =\dfrac{1}{2}$}
        {$\sin \left( x-\dfrac{\pi}{6}\right) =\dfrac{1}{2}$}
        \loigiai{
            Chia hai vế của phương trình cho $2$, ta được
            \begin{eqnarray*}
                &\sqrt{3}\sin x+\cos x=1&\Leftrightarrow\dfrac{\sqrt{3}}{2}\sin x+\dfrac{1}{2}\cos x=\dfrac{1}{2}\\
                &&\Leftrightarrow\sin\dfrac{\pi}{3}\sin x+\cos\dfrac{\pi}{3}\cos x=\dfrac{1}{2}\\
                &&\Leftrightarrow\cos\left( x-\dfrac{\pi}{3}\right) =\dfrac{1}{2}.
            \end{eqnarray*}
        }
    \end{ex}

\begin{ex}%[1T1Y4-1]
        Tìm điều kiện xác định của hàm số  $y=\cot x$.
        \choice
        {$x \neq \dfrac{\pi}{4}+k \pi, k \in \mathbb{Z}$}
        { $x \neq k 2 \pi, k \in \mathbb{Z}$}
        {\True $x \neq k \pi, k \in \mathbb{Z}$}
        {$x\neq \dfrac{\pi}{2}+k \pi, k \in \mathbb{Z}$}
        \loigiai{
            Hàm số $y=\cot x$ xác định khi và chỉ khi $\sin x \ne 0 \Leftrightarrow x\neq k \pi, k \in \mathbb{Z} .$}
    \end{ex}

\begin{ex}%[1T1B4-2]
        Hàm số nào sau đây đồng biến trên khoảng $(0;\pi)$?
        \choice
        {\True $y=x^2$}
        {$y=\cos x$}
        {$y=\sin x$}
        {$y=\tan x$}
        \loigiai{
            Hàm số $y = x^2$ đồng biến khi $x > 0 \Rightarrow$ hàm số đồng biên trên khoảng $\left(0;\pi\right)$.}
    \end{ex}

\begin{ex}%[1C1B1-2]
    Cho góc lượng giác gốc $O$ có tia đầu $Ou$, tia cuối $Ov$ và có số đo $-\dfrac{5\pi}{6}$. Cho góc lượng giác $(O'u',O'v')$ có tia đầu $O'u'\equiv Ou$, tia cuối $O'v'\equiv Ov$. Viết công thức biểu thị số đo góc lượng giác $(O'u',O'v')$.
    \choice
    {$(O'u',Ov')=\dfrac{\pi}{6}+k2\pi\ (k\in \mathbb{Z})$}
    {$(O'u',Ov')=\dfrac{4\pi}{3}+k2\pi\ (k\in \mathbb{Z})$}
    {$(O'u',Ov')=-\dfrac{\pi}{6}+k2\pi\ (k\in \mathbb{Z})$}
    {\True $(O'u',Ov')=-\dfrac{5\pi}{6}+k2\pi\ (k\in \mathbb{Z})$}
    \loigiai{
        Ta có $(O'u',Ov')=(Ou,Ov)+k2\pi=-\dfrac{5\pi}{6}+k2\pi\ (k\in \mathbb{Z})$.
    }
\end{ex}

\begin{ex}%[1T1B4-6]
        Hình bên dưới là đồ thị của hàm số nào dưới đây?
        \begin{center}
            \begin{tikzpicture}[>=stealth,line join=round,line cap=round,font=\footnotesize,scale=0.7]
                \def\a{3.141592654}
                \draw[color=gray,dash pattern=on 1pt off 1pt,xstep=3.14cm,ystep=1.0cm] (-9.424,-3) grid (9.424,3);
                \draw[->] (-9.424,0) -- (9.7,0)node[below]{\scriptsize $x$};
                \draw[->] (0,-3.5) -- (0,3.5) node[left] {\scriptsize $y$};
                \draw (0,0)node[below left]{\scriptsize $O$};
                \clip (-9.58,-3.5)rectangle(9.58,3.5);
                \draw[samples=300,smooth,domain=-9.424:9.424] plot(\x,{-3*cos(\x*180/pi)});
                \path(-2*\a,0)node[shift={(-150:12pt)}]{$-2\pi$}
                (-\a,0)node[shift={(-150:12pt)}]{$-\pi$}
                (\a,0)node[shift={(-135:10pt)}]{$\pi$}
                (2*\a,0)node[shift={(-140:10pt)}]{$2\pi$}
                (0,3)node[shift={(-140:10pt)}]{$3$}
                (0,-3)node[shift={(-140:10pt)}]{$-3$};
                \foreach \x/\y in{-2*\a/0,-\a/0,\a/0,2*\a/0,-3*\a/3,3*\a/3,-2*\a/-3,2*\a/-3,\a/3,-\a/3,0/3,0/-3,0/0}\draw(\x,\y) circle (1pt);
            \end{tikzpicture}
        \end{center}
        \choice
        {\True $y=-3\cos x$}
        {$y=-2-\cos x$}
        {$y=2+|\cos x|$}
        {$y=\cos x-4$}
        \loigiai{
            \begin{itemize}
                \item $y(0)=-3\Rightarrow $ loại $y=\cos x-4$ và $y=2+|\cos x|$.
                \item $y(\pi)=3\Rightarrow $ loại $y=-2-\cos x$.
            \end{itemize}
        }
    \end{ex}

\begin{ex}%[1T1Y4-1]
        Điều kiện xác định của hàm số $y=\cot x$ là
        \choice
        { $x \ne \dfrac{\pi}{8}+k\dfrac{\pi}{2}$}
        {$x \ne \dfrac{\pi}{2}+k\pi$}
        {\True $x \ne k\pi$}
        {$x \ne \dfrac{\pi}{4}+k\pi$}
        \loigiai{
            Hàm số xác định khi và chỉ khi $\sin x \ne 0 \Leftrightarrow x \ne k\pi$, $k \in \mathbb{Z}$.
        }
    \end{ex}

\begin{ex}%[1T1B4-5]
        Cho hàm số $y=\sin^2x-\sin x+2$. Gọi $M,N$ lần lượt là GTLN và GTNN của hàm số đã cho. Khi đó $M+N$ bằng
        \choice
        {$k=-\dfrac{1}{2}$}
        {\True $\dfrac{23}{4}$}
        {$\dfrac{15}{4}$}
        {$6$}
        \loigiai{
            Ta có $y=\sin^2x-\sin x+2=\left(\sin x-\dfrac{1}{2}\right)^2+\dfrac{7}{4}$. \\
            Vì $-1 \leq \sin x \leq 1,\,\forall x \in \mathbb{R}$ nên $-\dfrac{3}{2} \leq \sin x-\dfrac{1}{2} \leq \dfrac{1}{2},\,\forall x \in \mathbb{R}$.\\
            Suy ra $0 \leq \left(\sin x-\dfrac{1}{2}\right)^2 \leq \dfrac{9}{4},\,\forall x \in \mathbb{R}$.\\
            Suy ra $\dfrac{7}{4} \leq \left(\sin x-\dfrac{1}{2}\right)^2+\dfrac{7}{4} \leq 4,\,\forall x \in \mathbb{R}$.\\
            Suy ra $\dfrac{7}{4} \leq y \leq 4,\,\forall x \in \mathbb{R}$.\\
            Vậy $M+N=\dfrac{7}{4}+4=\dfrac{23}{4}$.}
    \end{ex}

\begin{ex}%[Tex hóa SGK CD-CT,T12/22, TVN-006]%[1K1Y3-4]
    Trong các hàm số sau đây, hàm số nào là hàm tuần hoàn?
    \choice
    {$y=\tan x+x$}
    {$y=x^2+1$}
    {\True $y=\cot x$}
    {$y=\dfrac{\sin x}{x}$}
    \loigiai{
        Hàm số $y=\cot x$ là hàm số tuần hoàn với chu kỳ $T=\pi$.
    }
\end{ex}

\begin{ex}%[1T1Y1-1]
        Góc $18^\circ$ có số đo bằng rađian là bao nhiêu?
        \choice
        {$\pi$}
        {$\dfrac{\pi}{360}$}
        {\True $\dfrac{\pi}{10}$}
        {$\dfrac{\pi}{18}$}
        \loigiai{
            Ta có $18^\circ=\dfrac{\pi}{10}$ rad.
        }
    \end{ex}

\begin{ex}%[Tex hóa SGK CD-CT,T12/22, TVN-006]%[1K1Y1-4]
    Biểu diễn các góc lượng giác $\alpha=-\dfrac{5\pi}{6}$, $\beta=\dfrac{\pi}{3}$, $\gamma=\dfrac{25\pi}{3}$, $\delta=\dfrac{17\pi}{6}$ trên đường tròn lượng giác. Các góc nào có điểm biểu diễn trùng nhau?
    \choice
    {\True $\beta$ và $\gamma$}
    {$\alpha$, $\beta$, $\gamma$}
    {$\beta$, $\gamma$, $\delta$}
    {$\alpha$ và $\beta$}
    \loigiai{
        Ta có $\beta+8\pi=\dfrac{\pi}{3}+8\pi=\dfrac{25\pi}{3}=\gamma$.\\
        Do đó, $\beta$ và $\gamma$ có điểm biểu diễn trùng nhau trên đường tròn lượng giác.
    }
\end{ex}

\begin{ex}%[1C1B1-2]
    Cho góc lượng giác $(Ou,Ov)$ có số đo là $\dfrac{3\pi}{4}$, góc lượng giác $(Ou,Ow)$ có số đo là $\dfrac{5\pi}{4}$. Số đo của góc lượng giác $(Ov,Ow)$ là
\choice
{\True $(Ov,Ow)=\dfrac{\pi}{2}+k2\pi\ (k\in \mathbb{Z})$}
{$(Ov,Ow)=2\pi+k2\pi\ (k\in \mathbb{Z})$}
{$(Ov,Ow)=-\dfrac{\pi}{2}+k2\pi\ (k\in \mathbb{Z})$}
{$(Ov,Ow)=-\dfrac{\pi}{6}+k2\pi\ (k\in \mathbb{Z})$}
    \loigiai{
        Theo hệ thức Chasles, ta có
        \begin{eqnarray*}
            (Ov,Ow)&=&(Ou,Ow)-(Ou,Ov)+k2\pi\\
            &=&\dfrac{5\pi}{4}-\dfrac{3\pi}{4}+k2\pi\\
            &=&\dfrac{\pi}{2}+k2\pi\ (k\in \mathbb{Z}).
        \end{eqnarray*}
    }
\end{ex}

\begin{ex}%[1C1B1-2]
    Cho góc lượng giác gốc $O$ có tia đầu $Ou$, tia cuối $Ov$ và có số đo $45^\circ$. Cho góc lượng giác $(O'u',O'v')$ có tia đầu $O'u'\equiv Ou$, tia cuối $O'v'\equiv Ov$. Công thức biểu thị số đo góc lượng giác $(O'u',O'v')$ là
\choice
{$(O'u',Ov')=-45^\circ+k360^\circ\ (k\in \mathbb{Z})$}
{\True $(O'u',Ov')=45^\circ+k360^\circ\ (k\in \mathbb{Z})$}
{$(O'u',Ov')=135^\circ+k360^\circ\ (k\in \mathbb{Z})$}
{$(O'u',Ov')=-135^\circ+k360^\circ\ (k\in \mathbb{Z})$}
    \loigiai{
        Ta có $(O'u',Ov')=(Ou,Ov)+k360^\circ=45^\circ+k360^\circ\ (k\in \mathbb{Z})$.
    }
\end{ex}

\begin{ex}%[1T1B4-5]
        Hàm số $y=3-5\sin x$ có giá trị lớn nhất bằng
        \choice
        {$6$}
        {$2$}
        {\True $8$}
        {$4$}
        \loigiai{
            Ta có
            $$-1\le \sin x\le 1 \Leftrightarrow 5\ge-5\sin x\ge-5\Leftrightarrow  8\ge 3-5\sin x\ge -2\Rightarrow -2\le y\le 8.$$
            Suy ra giá trị lớn nhất của hàm số là $8$, đạt được khi $x=\dfrac{\pi}{2}+k2\pi,k\in\mathbb{Z}$.
        }
    \end{ex}

\begin{ex}%[1T1B2-4]
        Rút gọn biểu thức $M=\sin(\pi-a)+\tan\left(\dfrac{\pi}{2}-a\right)+\sin(-a)+\cot(\pi+a)$ được
        \choice
        {$M=2\cos a$}
        {$M=2\tan a$}
        {\True $M=2\cot a$}
        {$M=0$}
        \loigiai{
            Ta có $M=\sin a+\cot a-\sin a+\cot a=2\cot a$.
        }
    \end{ex}

\begin{ex}%[1T1Y4-6]
        Đồ thị hàm số $y=\cos x$ đi qua điểm nào sau đây?
        \choice
        {$P(-1;\pi)$}
        {$M(\pi;1)$}
        {$Q(3\pi; 1)$}
        {\True $N(0;1)$}
        \loigiai{
            Điểm $N(0;1)$ thuộc đồ thị hàm số.
        }
    \end{ex}

\begin{ex}%[1T1B4-1]
        Tập xác định của hàm số $y=2017\tan^{2018} \left( 2x+\dfrac{\pi}{3}\right)$ là
        \choice
        {\True $\mathscr{D}=\mathbb{R}\setminus\left\lbrace\dfrac{\pi}{12}+k\dfrac{\pi}{2}, k\in\mathbb{Z} \right\rbrace $}
        {$\mathscr{D}=\mathbb{R}\setminus\left\lbrace\dfrac{\pi}{2}+k\dfrac{\pi}{2}, k\in\mathbb{Z} \right\rbrace $}
        {$\mathscr{D}=\mathbb{R}\setminus\left\lbrace\dfrac{\pi}{2}+k\dfrac{\pi}{2}, k\in\mathbb{Z} \right\rbrace $}
        {$\mathscr{D}=\mathbb{R}\setminus\left\lbrace\dfrac{\pi}{2}+k\dfrac{\pi}{2}, k\in\mathbb{Z} \right\rbrace $}
        \loigiai{
            Hàm số xác định khi $2x+\dfrac{\pi}{3}\ne \dfrac{\pi}{2}+k\pi\Leftrightarrow x\ne\dfrac{\pi}{12}+k\dfrac{\pi}{2}, k\in\mathbb{Z}.$
        }
        \end{ex}

\begin{ex}%[1K1Y1-7]
        Tìm khẳng định đúng (với điều kiện các hệ thức đã xác định).
        \choice
        {$\cos \left(\pi -\alpha \right)=\cos \alpha$}
        {\True $\cos \left(-\alpha \right)=\cos \alpha$}
        {$\sin \left(\pi -\alpha \right)=-\sin \alpha$}
        {$\sin \left(-\alpha \right)=\sin \alpha$}
        \loigiai{
            Ta có
            \begin{itemize}
                \item $\sin \left(-\alpha \right)=-\sin \alpha$.
                \item $\cos \left(\pi -\alpha \right)=-\cos \alpha$.
                \item $\cos \left(-\alpha \right)=\cos \alpha$.
                \item $\sin \left(\pi -\alpha \right)=\sin \alpha$.
            \end{itemize}
        }
    \end{ex}


\noindent\textbf{II. PHẦN TỰ LUẬN:}
\begin{ex}%[Cánh Diều]%[1C1B4-3]
    Giải các phương trình
    \begin{multicols}{3}
        \begin{enumerate}[a)]
            \item $\sin x=-\dfrac{1}{2}$;
            \item $\sin x=\dfrac{\sqrt{2}}{2}$;
            \item $\sin3x=\sin2x$;
            \item $\sin x=\cos3x$;
            \item $\cos x=\dfrac{\sqrt{3}}{2}$;
            \item $\cos x=-\dfrac{\sqrt{2}}{2}$;
            \item $\cos x=-\dfrac{1}{2}$;
            \item $\cos3x=\cos\left(x+\dfrac{\pi}{3}\right)$;
            \item $\tan x=\dfrac{1}{\sqrt{3}}$;
            \item $\tan x=-1$;
            \item $\cot2x=-\sqrt{3}$.
        \end{enumerate}
    \end{multicols}
\loigiai{
\begin{enumerate}[a)]
    \item Do $\sin\left(-\dfrac{\pi}{6}\right)=-\dfrac{1}{2}$ nên $$\sin x=\sin\left(-\dfrac{\pi}{6}\right)\Leftrightarrow\hoac{&x=-\dfrac{\pi}{6}+k2\pi\\&x=\pi-\left(-\dfrac{\pi}{6}\right)+k2\pi}\Leftrightarrow\hoac{&x=-\dfrac{\pi}{6}+k2\pi\\&x=\dfrac{7\pi}{6}+k2\pi}\,(k\in\mathbb{Z}).$$
    \item Do $\sin\dfrac{\pi}{4}=\dfrac{\sqrt{2}}{2}$ nên $$\sin x=\sin\dfrac{\pi}{4}\Leftrightarrow\hoac{&x=\dfrac{\pi}{4}+k2\pi\\&x=\pi-\dfrac{\pi}{4}+k2\pi}\Leftrightarrow\hoac{&x=\dfrac{\pi}{4}+k2\pi\\&x=\dfrac{3\pi}{4}+k2\pi}\,(k\in\mathbb{Z}).$$
    \item $\sin3x=\sin2x\Leftrightarrow\hoac{&3x=2x+k2\pi\\&3x=\pi-2x+k2\pi}\Leftrightarrow\hoac{&x=k2\pi\\&x=\dfrac{\pi}{5}+k\dfrac{2\pi}{5}}\,(k\in\mathbb{Z})$.
    \item $\sin x=\cos3x\Leftrightarrow\sin x=\sin\left(\dfrac{\pi}{2}-3x\right)\Leftrightarrow\hoac{&x=\dfrac{\pi}{2}-3x+k2\pi\\&x=\pi-\left(\dfrac{\pi}{2}-3x\right)+k2\pi}\Leftrightarrow\hoac{&x=\dfrac{\pi}{8}+k\dfrac{\pi}{2}\\&x=-\dfrac{\pi}{4}+k\pi}\,(k\in\mathbb{Z})$.
    \item $\cos x=\dfrac{\sqrt{3}}{2}\Leftrightarrow\cos x=\cos\dfrac{\pi}{6}\Leftrightarrow x=\pm\dfrac{\pi}{6}+k2\pi,\,(k\in\mathbb{Z})$;
    \item $\cos x=-\dfrac{\sqrt{2}}{2}\Leftrightarrow \cos x=\cos\dfrac{3\pi}{4}\Leftrightarrow x=\pm\dfrac{3\pi}{4}+k2\pi,\,(k\in\mathbb{Z})$;
    \item $\cos x=-\dfrac{1}{2}\Leftrightarrow\cos x=\cos\dfrac{2\pi}{3}\Leftrightarrow x=\pm\dfrac{2\pi}{3}+k2\pi,\,(k\in\mathbb{Z})$;
    \item $\cos3x=\cos\left(x+\dfrac{\pi}{3}\right)\Leftrightarrow\hoac{&3x=x+\dfrac{\pi}{3}+k2\pi\\&3x=-x-\dfrac{\pi}{3}+k2\pi}\Leftrightarrow\hoac{&x=\dfrac{\pi}{6}+k\pi\\&x=-\dfrac{\pi}{12}+\dfrac{k\pi}{2}},\,(k\in\mathbb{Z})$;
    \item $\tan x=\dfrac{1}{\sqrt{3}}\Leftrightarrow\tan x=\tan\dfrac{\pi}{6}\Leftrightarrow x=\dfrac{\pi}{6}+k\pi,\,(k\in\mathbb{Z})$;
    \item $\tan x=-1\Leftrightarrow\tan x=\tan\left(-\dfrac{\pi}{4}\right)\Leftrightarrow x=-\dfrac{\pi}{4}+k\pi,\,(k\in\mathbb{Z})$;
    \item $\cot2x=-\sqrt{3}\Leftrightarrow\cot2x=\cot\left(-\dfrac{\pi}{6}\right)\Leftrightarrow x=-\dfrac{\pi}{6}+k\pi,\,(k\in\mathbb{Z})$.
\end{enumerate}}
\end{ex}

\begin{ex}%[1C1B4-3]
    Giải phương trình:
    \begin{listEX}[3]
        \item $\sin \left(2x-\dfrac{\pi}{3}\right)=-\dfrac{\sqrt{3}}{2}$;
        \item $\sin \left(3x+\dfrac{\pi}{4}\right)=-\dfrac{1}{2}$;
        \item $\cos \left(\dfrac{x}{2}+\dfrac{\pi}{4}\right) =\dfrac{\sqrt{3}}{2}$;
        \item $2\cos 3x+5=3$;
        \item $3\tan x=-\sqrt{3}$;
        \item $\cot x-3=\sqrt{3}\left(1-\cot x\right)$.
    \end{listEX}
    \loigiai{
        \begin{enumerate}[a)]
            \item Ta có 
            \begin{eqnarray*}
                &&\sin \left(2x-\dfrac{\pi}{3}\right)=-\dfrac{\sqrt{3}}{2}\\
                &\Leftrightarrow& \sin \left(2x-\dfrac{\pi}{3}\right) =\sin \left(-\dfrac{\pi}{3}\right)\\
                &\Leftrightarrow& \hoac{
                    &2x-\dfrac{\pi}{3} =-\dfrac{\pi}{3}+k2\pi\\
                    &2x-\dfrac{\pi}{3} = \pi+\dfrac{\pi}{3}+k2\pi}\\
                &\Leftrightarrow&
                \hoac{&2x=k2\pi\\&2x=\dfrac{5\pi}{3} +k2\pi}\\
                &\Leftrightarrow&
                \hoac{&x=k\pi\\ &x=\dfrac{5\pi}{6}+k\pi} (k\in \mathbb{Z}).
            \end{eqnarray*}         
            \item Ta có 
            \begin{eqnarray*}
                &&\sin \left(3x+\dfrac{\pi}{4}\right)=-\dfrac{1}{2} \\
                &\Leftrightarrow& \sin \left(3x+\dfrac{\pi}{4}\right) =\sin \left(-\dfrac{\pi}{6}\right)\\  
                &\Leftrightarrow& \hoac{
                    &3x+\dfrac{\pi}{4} = -\dfrac{\pi}{6}+k2\pi\\
                    &3x+\dfrac{\pi}{4}=\pi -\left(-\dfrac{\pi}{6}\right)+k2\pi} \\
                &\Leftrightarrow&
                \hoac{
                    &3x= -\dfrac{5\pi}{12}+k2\pi\\
                    &3x=\dfrac{11\pi}{12}+k2\pi} \\
                &\Leftrightarrow&
                \hoac{&x=-\dfrac{5}{36}+\dfrac{k2\pi}{3}\\
                    &x=\dfrac{11\pi}{36}+\dfrac{k2\pi}{3}} (k\in \mathbb{Z}).
            \end{eqnarray*}                     
            \item Ta có 
            \begin{eqnarray*}
                &&\cos \left(\dfrac{x}{2}+\dfrac{\pi}{4}\right) =\dfrac{\sqrt{3}}{2} \\ &\Leftrightarrow& \cos \left(\dfrac{x}{2}+\dfrac{\pi}{4}\right)=\cos \dfrac{\pi}{6}\\
                &\Leftrightarrow& \hoac{
                    &\dfrac{x}{2}+\dfrac{\pi}{4} = \dfrac{\pi}{6}+k2\pi\\
                    &\dfrac{x}{2}+\dfrac{\pi}{4}= -\dfrac{\pi}{6}+k2\pi}\\
                &\Leftrightarrow&
                \hoac{
                    &\dfrac{x}{2}=-\dfrac{\pi}{12}+k2\pi\\
                    &\dfrac{x}{2}=-\dfrac{5\pi}{12}+k2\pi}\\
                &\Leftrightarrow&
                \hoac{
                    &x=-\dfrac{\pi}{6}+k4\pi\\
                    &x=-\dfrac{5\pi}{6}+k4\pi} (k\in \mathbb{Z}).
            \end{eqnarray*} 
            \item Ta có $2\cos 3x+5=3 \Leftrightarrow \cos 3x =-1 \Leftrightarrow 3x=\pi+k2\pi \Leftrightarrow x = \dfrac{\pi}{3}+\dfrac{k2\pi}{3}\,\,(k\in \mathbb{Z})$.
            \item Ta có $3\tan x=-\sqrt{3} \Leftrightarrow \tan x =-\dfrac{\sqrt{3}}{3} \Leftrightarrow 
            \tan x=\tan \left(-\dfrac{\pi}{6}\right) \Leftrightarrow x=-\dfrac{\pi}{6}+k\pi\,\, (k\in \mathbb{Z}).$
            \item Ta có 
            \begin{eqnarray*}
                &&\cot x-3=\sqrt{3}\left(1-\cot x\right)\\
                &\Leftrightarrow& \cot x-3 =\sqrt{3}-\sqrt{3}\cot x\\
                &\Leftrightarrow& (1+\sqrt{3})\cot x=\sqrt{3}(1+\sqrt{3})\\
                &\Leftrightarrow& \cot x=\sqrt{3}\\
                &\Leftrightarrow& \cot x=\cot \dfrac{\pi}{6}\\
                &\Leftrightarrow& x=\dfrac{\pi}{6}+k\pi\,\, (k\in \mathbb{Z}).
            \end{eqnarray*} 
        \end{enumerate}
    }
\end{ex}

\begin{ex}%[1C1B4-3]
    Giải phương trình:
    \begin{listEX}[3]
        \item $\sin \left(2x+\dfrac{\pi}{4}\right)=\sin x$;
        \item $\sin 2x=\cos 3x$;
        \item $\cos^2 2x =\cos^2 \left(x+\dfrac{\pi}{6}\right)$.
    \end{listEX}
    \loigiai{
        \begin{enumerate}[a)]
            \item Ta có 
            \[\sin \left(2x+\dfrac{\pi}{4}\right)=\sin x
            \Leftrightarrow 
            \hoac{&2x+\dfrac{\pi}{4}=x+k2\pi\\&2x+\dfrac{\pi}{4}=\pi-x+k2\pi} 
            \Leftrightarrow \hoac{&x=-\dfrac{\pi}{4}+k2\pi\\&3x=-\dfrac{\pi}{4}+k2\pi} \Leftrightarrow \hoac{&x=-\dfrac{\pi}{4}+k2\pi\\&x=-\dfrac{\pi}{12}+\dfrac{k2\pi}{3}
            },\, (k\in \mathbb{Z}).\]
            \item Ta có 
            \begin{eqnarray*}
                \sin 2x=\cos 3x &\Leftrightarrow& \cos 3x =\cos \left(\dfrac{\pi}{2}-2x\right)\\
                &\Leftrightarrow& \hoac{
                    &3x=\dfrac{\pi}{2}-2x+k2\pi\\
                    &3x=\pi-\left(\dfrac{\pi}{2}-2x\right) +k2\pi}\\
                &\Leftrightarrow&
                \hoac{&5x=\dfrac{\pi}{2}+k2\pi\\&x=\dfrac{\pi}{2}+k2\pi}\\
                &\Leftrightarrow&
                \hoac{&x=\dfrac{\pi}{12}+\dfrac{k2\pi}{5}\\
                    &x=\dfrac{\pi}{2}+k2\pi}\, (k\in \mathbb{Z}).
            \end{eqnarray*} 
            \item Ta có $\cos^2 2x =\cos^2 \left(x+\dfrac{\pi}{6}\right) \Leftrightarrow 
            \hoac{
                &\cos 2x=\cos \left(x+\dfrac{\pi}{6}\right) &(1)\\
                &\cos 2x=-\cos \left(x+\dfrac{\pi}{6}\right). &(2)
            }$\\
            +) $(1) \Leftrightarrow \hoac{
                &2x=x+\dfrac{\pi}{6}+k2\pi\\
                &2x=-\left(x+\dfrac{\pi}{6}\right)+k2\pi
            } \Leftrightarrow
            \hoac{
                &x=\dfrac{\pi}{6}+k2\pi\\
                &3x=-\dfrac{\pi}{6}+k2\pi
            }
            \Leftrightarrow
            \hoac{
                &x=\dfrac{\pi}{6}+k2\pi\\
                &x=-\dfrac{\pi}{18}+\dfrac{k2\pi}{3}
            }(k\in \mathbb{Z})$.\\
            +) $(2) \Leftrightarrow 
            \cos 2x=\cos\left[\pi- \left(x+\dfrac{\pi}{6}\right) \right]
            \Leftrightarrow
            \hoac{
                &2x=\pi- \left(x+\dfrac{\pi}{6}\right)+k2\pi\\
                &2x=-\left[\pi- \left(x+\dfrac{\pi}{6}\right)\right]+k2\pi
            } $
            \[\Leftrightarrow
            \hoac{
                &3x=\dfrac{5\pi}{6}+k2\pi\\
                &x=-\dfrac{5\pi}{6}+k2\pi
            } \Leftrightarrow
            \hoac{
                &x=\dfrac{5\pi}{18}+\dfrac{k2\pi}{3}\\
                &x=-\dfrac{5\pi}{6}+k2\pi
            } \,(k\in\mathbb{Z}).\]
        \end{enumerate}
    }
\end{ex}

\begin{ex}Giải các phương trình sau
    \begin{listEX}[2]
        \item $2\sin x+\sqrt{2}=0$;
        \item $\sin2x-\cos x+2\sin x=1$;
        \item $3\sin ^2 x-5\sin x+2=0$;
        \item $\sqrt{3}\tan^2 x-2\tan x+\sqrt{3}=0$;
        \item $2\cos^2 2x-5\cos 2x+2=0$;
        \item $\sin^2\dfrac{x}{2}+\sin\dfrac{x}{2}-2=0$.
    \end{listEX}
    \loigiai{
        \begin{enumerate}[a)]
            \item $2\sin x+\sqrt{2}=0\Leftrightarrow\sin x=-\dfrac{\sqrt{2}}{2}\Leftrightarrow\sin x=\sin\left(-\dfrac{\pi}{4}\right)\Leftrightarrow\hoac{&x=-\dfrac{\pi}{4}+k2\pi\\&x=\dfrac{5\pi}{4}+k2\pi}\,(k\in\mathbb{Z})$;
            \item $\sin2x-\cos x+2\sin x=1\Leftrightarrow2\sin x\cos x-\cos x+2\sin x-1=0\Leftrightarrow(2\sin x-1)(\cos x+1)=0\\\Leftrightarrow\hoac{&\sin x=\dfrac{1}{2}\\&\cos x=-1}\Leftrightarrow\hoac{&\sin x=\sin\dfrac{\pi}{6}\\&x=(2k+1)\pi}\Leftrightarrow\hoac{&x=\dfrac{\pi}{6}+k2\pi\\&x=\dfrac{5\pi}{6}+k2\pi\\&x=(2k+1)\pi}\,(k\in\mathbb{Z})$;
            \item Đặt $t=\cos x$, $-1\le t\le 1$, phương trình đã cho trở thành $3t^2-5t+2=0$, ta được $t=1$ hoặc $t=\dfrac{2}{3}$.\\
            Với $t=1$ ta có $\cos x=1 \Leftrightarrow x= k2\pi,\, (k\in\mathbb{Z})$.\\
            Với $t=\dfrac{2}{3}$ ta có $\cos x=\dfrac{2}{3}=\cos\alpha\Leftrightarrow x=\pm \alpha +k2\pi,\, (k\in\mathbb{Z})$.\\
            Vậy tập nghiệm của phương trình đã cho là $S=\left\{ k2\pi, \pm \alpha+k2\pi, k\in\mathbb{Z} \right\}$.
            \item Đặt $t=\tan x$, phương trình đã cho trở thành $\sqrt{3} t^2-2t+\sqrt{3}=0$. Phương trình này vô nghiệm. Vậy phương trình đã cho vô nghiệm.
            \item $2\cos^2 2x-5\cos 2x+2=0\Leftrightarrow\hoac{&\cos2x=2\\&\cos2x=\dfrac{1}{2}}\Leftrightarrow\cos2x=\cos\dfrac{\pi}{3}\Leftrightarrow x=\pm\dfrac{\pi}{6}+k\pi,\, (k\in\mathbb{Z})$;
            \item $\sin^2\dfrac{x}{2}+\sin\dfrac{x}{2}-2=0\Leftrightarrow\hoac{&\sin\dfrac{x}{2}=1\\&\sin\dfrac{x}{2}=-2}\Leftrightarrow\dfrac{x}{2}=\dfrac{\pi}{2}+k2\pi\Leftrightarrow x=\pi+k4\pi,\, (k\in\mathbb{Z})$.
        \end{enumerate}
    }
\end{ex}

\begin{ex}%[1D1G1-5]
    Tìm giá trị lớn nhất và giá trị nhỏ nhất của hàm số  $y=2(\sin x+\cos x)+\sin 2 x+3$.
    \loigiai
    {Tập xác định $\mathscr{D}=\mathbb{R}$.\\
        Đặt $t=\sin x+ \cos x=\sqrt{2}\sin \left(x+\dfrac{\pi}{4}\right)$, $t\in \left[-\sqrt{2};\sqrt{2}\right]$.\\
        Ta có $t^2=\left(\sin x+ \cos x\right)^2=1+2\sin x\cos x=1+\sin 2x\Rightarrow \sin 2x =t^2-1$.\\
        Hàm số trở thành $y=g(t)=t^2+2t+2$. \\
        Bảng biến thiên của hàm số $y=g(t)$ trên đoạn $ \left[-\sqrt{2};\sqrt{2}\right]$
        \begin{center}
            \begin{tikzpicture}
                \tkzTabInit[lgt=1,espcl=3,deltacl=1]%nocadre,
                {$t$/1, $g(t)$ /2}
                {$-\sqrt{2}$ , $-1$ ,  $\sqrt{2}$}
                \tkzTabVar {+/$4-2\sqrt{2}$,-/$1$ ,+/ $4+2\sqrt{2}$}
            \end{tikzpicture}
        \end{center}
        Vậy $\max\limits_{x \in \mathbb{R}} y=4+2\sqrt{2}$ và $\min\limits_{x \in \mathbb{R}} y=1$.
    }
\end{ex}

\begin{ex}%[1D1K1-5]
    Tìm giá trị lớn nhất và giá trị nhỏ nhất của hàm số $y=\sqrt{3} \sin x-\cos x+5$.
    \loigiai
    {
        Tập xác định $\mathscr{D}=\mathbb{R}$.\\
        Biến đổi $y=\sqrt{3} \sin x-\cos x+5=2\left(\dfrac{\sqrt{3}}{2}\cdot\sin x-\dfrac{1}{2}\cdot\cos x\right)+5=2\sin\left(x-\dfrac{\pi}{6}\right)+5$.\\
        Với mọi $x\in \mathbb{R}$ ta có
        \allowdisplaybreaks
        \begin{eqnarray*}
            & & -1\leq \sin\left(x-\dfrac{\pi}{6}\right)\leq 1\\
            &\Leftrightarrow& -2\leq 2\sin\left(x-\dfrac{\pi}{6}\right)\leq 2\\
            &\Leftrightarrow&3\leq  2\sin\left(x-\dfrac{\pi}{6}\right)+5\leq 7.
        \end{eqnarray*}
        Vậy $\max\limits_{x \in \mathbb{R}} y=7$ khi $x=\dfrac{2\pi}{3}$ và $\min\limits_{x \in \mathbb{R}} y=3$ khi $x=-\dfrac{\pi}{3}$.
    }
\end{ex}
\Closesolutionfile{ans}

%Chương II
%%Bài 5. DS
% 
\setcounter{section}{4}
\setcounter{dang}{0}
\setcounter{ex}{0}
\setcounter{bt}{0}
\setcounter{vd}{0}
\section{Dãy số}
\subsection{Tóm tắt lý thuyết}
\begin{tomtat}
	\subsubsection{Định nghĩa dãy số} 
	\begin{itemize}
		\item Mỗi hàm $u$ xác định trên tập các số nguyên dương $\mathbb{N^{*}}$ được gọi là một dãy vô hạn (gọi tắt là dãy số), kí hiệu $u=u(n)$.
		\item 	Ta thường viết $u_n$ thay cho $u(n)$ và kí hiệu dãy số $u=u(n)$ bởi $(u_n)$, do đó dãy số $(u_n)$ được viết dưới dạng khai triển $u_1, u_2, u_3, \ldots, u_n, \ldots$\\
		Số $u_1$ gọi là số hạng đầu, $u_n$ gọi là số hạng thứ $n$ và gọi là số hạng tổng quát của dãy số.
		\item Nếu $\forall n \in \mathrm{N^*}, u_n=c$ thì $(u_n)$ được gọi là dãy số không đổi.
		\item Mỗi hàm $u$ xác định trên tập $\mathrm{M}=\left\{1;2;3;\ldots;m\right\}, \forall m \in \mathrm{N^*}$ được gọi là một dãy số hữu hạn.
		\item Dạng khai triển của dãy hữu hạn là $u_1, u_2, u_3, \ldots, u_m$.\\
		Số $u_1$ gọi là số hạng đầu, số $u_m$ gọi là số hạng cuối.
		
	\end{itemize}
	\subsubsection{Các cách cho một dãy số}
	Một dãy số có thể cho bằng:
	\begin{itemize}
		\item Liệt kê các số hạng (chỉ dùng cho các dãy hữu hạn và có ít số hạng);
		\item Công thức của số hạng tổng quát;
		\item Phương pháp mô tả;
		\item Phương pháp truy hồi.
	\end{itemize}
	\subsubsection{Dãy số tăng, dãy số giảm, dãy số bị chặn}
	\begin{itemize}
		\item Dãy số $(u_n)$ được gọi là dãy số tăng nếu ta có $u_{n+1}>u_n, \forall n \in \mathrm{N^*}$.
		\item Dãy số $(u_n)$ được gọi là dãy số giảm nếu ta có $u_{n+1}<u_n, \forall n \in \mathrm{N^*}$.
		\item Dãy số $(u_n)$ được gọi là bị chặn trên nếu tồn tại số $M$ sao cho $u_n \le M, \forall n \in \mathrm{N^*}$.
		\item Dãy số $(u_n)$ được gọi là bị chặn dưới nếu tồn tại số $m$ sao cho $u_n \ge m, \forall n \in \mathrm{N^*}$.
		\item Dãy số $(u_n)$ được gọi là bị chặn nếu nó vừa bị chặn trên vừa bị chặn dưới, tức là  tồn tại các số $m, M$ sao cho $m \le u_n \le M, \forall n \in \mathrm{N^*}$.
	\end{itemize}
\end{tomtat}

\subsection{Các dạng toán thường gặp}
\begin{dang}{Số hạng tổng quát, biểu diễn dãy số}
	Để tìm số hạng tổng quát của một dãy bất kỳ khi biết một vài số hạng đầu của dãy số ta làm như sau
	\begin{itemize}
		\item Phân tích các số hạng sau theo các số hạng đã biết theo một quy luật nào đó.
		\item Dự đoán số hạng tổng quát 
		\item Kiểm tra bằng cách thay lần lượt các giá trị $n\in \mathrm{N^*}$ vào công thức tổng quát (Chứng minh bằng phương pháp quy nạp).
	\end{itemize}
	Để biểu diễn một dãy số khi biết công thức tổng quát ta lần lượt thay $n\in \mathrm{N^*}$ vào công thức tổng quát để tìm các số hạng thứ nhất, thứ hai, $\ldots$
\end{dang}
\subsubsection{Ví dụ minh hoạ}
\begin{vd}[NB]%[DCHT Toán 11 - KNTT -Tên GV]%[1K2Y1-1]
	Xác định số hạng đầu và số hạng tổng quát của dãy số $(u_n)$ các số tự nhiên lẻ $1, 3, 5, 7, \ldots $
	\dapso{$u_n=2n-1$}	
	\loigiai{Dãy $(u_n)$ có số hạng đầu $u_1=1$ và số hạng tổng quát $u_n=2n-1$.}
\end{vd}
\begin{vd}[NB]%[DCHT Toán 11 - KNTT -Tên GV]%[1K2Y1-1]
	Xác định số hạng đầu và số hạng tổng quát của dãy số $(v_n)$ các số nguyên dương chia hết cho $5$: $5,10,15,20,\ldots$
	\dapso{$v_n=5n$}
	\loigiai{Dãy $(v_n)$ có số hạng đầu $v_1=5$ và số hạng tổng quát $v_n=5n$.}
\end{vd}
% \begin{vd}[NB]%[DCHT Toán 11 - KNTT -Tên GV]%[1K2Y1-1]
% 	Viết năm số hạng đầu và số hạng thứ $100$ của dãy số $(u_n)$ có số hạng tổng quát $u_n=3n-2$.
% 	\dapso{$u_{100}=298$}
% 	\loigiai{Năm số hạng đầu của dãy số là $1,4,7,10,13$.\\
% 		Số hạng thứ $100$ của dãy là $u_{100}=3\cdot100-2=298$.}
% \end{vd}
% \begin{vd}[NB]%[DCHT Toán 11 - KNTT -Tên GV]%[1K2Y1-1]
% 	Cho dãy số xác định bằng hệ thức truy hồi: $u_1=1, u_n=3u_{n-1}+2$ với $n\ge 2$. Viết ba số hạng đầu của dãy số này.
% 	\dapso{$u_1=1, u_2=5, u_3=17$}
% 	\loigiai{Ta có $u_1=1, u_2=3u_1+2=5, u_3=3u_2+2=17$.}
% \end{vd}
% \begin{vd}[NB]%[DCHT Toán 11 - KNTT -Tên GV]%[1K2Y1-1]
% 	Dãy số $(u_n)$ cho bởi hệ thức truy hồi: $u_1=1, u_n=n \cdot u_{n-1}$ với $n \ge 2$. Viết năm số hạng đầu của dãy số và dự đoán công thức tổng quát $u_n$.
% 	\dapso{$u_n=n!$}
% 	\loigiai{Năm số hạng đầu của dãy là
% 		$u_1=1, u_2=2\cdot u_1=2, u_3=3\cdot u_2=6, u_4=4 \cdot u_3= 24, u_5=5 \cdot u_4=124$.\\
% 		Số hạng tổng quát\\
% 		Ta có $ u_2=2\cdot 1, u_3=6=3\cdot 2\cdot 1, u_4=24=4\cdot 3\cdot 2\cdot 1, u_5=124= 5\cdot4\cdot3\cdot2\cdot1  $.\\
% 		Vậy số hạng tổng quát $u_n=n!$.}
% \end{vd}
\subsubsection{Bài tập tự luận}
 
\begin{bt}[NB]%[DCHT Toán 11 - KNTT -Tên GV]%[1K2Y1-1]
	Xét dãy số hữu hạn gồm các số tự nhiên lẻ nhỏ hơn 20, sắp xếp theo thứ tự từ bé đến lớn. Liệt kê tất cả các số hạng của dãy số này, tìm số hạng đầu và số hạng cuối của dãy. 
	\dapso{$u_1=1$, $u_{11}=19$}
	\loigiai{Các số hạng của dãy là $1,3,5,7,9,10,11,13,15,17,19$.\\
		Số hạng đầu của dãy là $u_1=1$.\\
		Số hạng cuối của dãy là $u_{11}=19$.}
\end{bt}
\begin{bt}[TH]%[DCHT Toán 11 - KNTT -Tên GV]%[1K2Y1-1]
	Xét dãy số gồm tất cả các số tự nhiên chia cho $5$ dư $1$. Xác định số hạng tổng quát của dãy số.
	\dapso{$u_n=5n+1$}
	\loigiai{Các số tự nhiên chia cho $5$ dư $1$ gồm các số sau:
		$6,11,16,21, \ldots $\\
		Số hạng tổng quát $u_n=5n+1$.}
\end{bt}
% \begin{bt}[NB]%[DCHT Toán 11 - KNTT -Tên GV] %[1K2Y1-1]
% 	Tìm năm số hạng đầu và số hạng thứ $100$ của dãy $(u_n)$ có số hạng tổng quát $u_n= \dfrac{(-1)^n}{n}$.
% 	\dapso{$u_1=-1, \dfrac{1}{2},-\dfrac{1}{3}, \dfrac{1}{4}, -\dfrac{1}{5} $, $u_{100}=\dfrac{1}{100}$}
% 	\loigiai{
% 		Năm số hạng đầu của dãy là $u_1=-1, \dfrac{1}{2},-\dfrac{1}{3}, \dfrac{1}{4}, -\dfrac{1}{5} $.\\
% 		Số hạng thứ $100$ là $u_{100}=\dfrac{1}{100}$.}
% \end{bt}
\begin{bt}[NB]%[DCHT Toán 11 - KNTT -Tên GV]%[1K2Y1-1]
	Viết năm số hạng đầu của dãy số gồm các số nguyên tố theo thứ tự tăng dần.
	\dapso{$2,3,5,7,11$}
	\loigiai {Năm số hạng đầu của dãy số trên là $2,3,5,7,11$.} 
\end{bt}
% \begin{bt}[NB]%[DCHT Toán 11 - KNTT-Tên GV]%[1K2Y1-1]
% 	Viết năm số hạng đầu của dãy $(u_n)$ với số hạng tổng quát là $u_n=n!$.
% 	\dapso{$1,2,6,24,120$}
% 	\loigiai{Năm số hạng đầu của dãy trên là $1,2,6,24,120$.}
% \end{bt}
\subsubsection{Câu hỏi trắc nghiệm}
\Opensolutionfile{ans}[ans/ans-1K2-1-Dang1]
%Cau1
\begin{ex}%[DCHT Toán 11 - KNTT -Tên GV]%[1K2B1-1]
	Cho dãy số có các số hạng đầu là $5,10,15,20,25, \ldots$ Số hạng tổng quát của dãy số này là
	\choice
	{$ u_n=5(n-1) $}
	{\True$ u_n=5n $}
	{$ u_n=5+n $}
	{$ u_n=5n+1 $}
	\loigiai
	{Ta có $5=5\cdot 1, 10=5 \cdot 2, 15 = 5\cdot 3, 20=5 \cdot 4, 25 = 5\cdot 5, \ldots$\\
		Vậy dãy trên có số hạng tổng quát là $u_n=5n$.
	}
\end{ex}
%Cau2
\begin{ex}%[DCHT Toán 11 - KNTT -Tên GV]%[1K2B1-1]
	Cho dãy số $(u_n)$ với $u_n=\dfrac{an^2}{n+1}$, $a$ là hằng số. $u_{n+1}$ là số hạng nào trong các số hạng sau
	\choice
	{\True $u_{n+1}=\dfrac{a(n+1)^2}{n+2} $}
	{$u_{n+1}=\dfrac{a(n+1)^2}{n+1}$}
	{$u_{n+1}=\dfrac{an^2+1}{n+1}$}
	{$u_{n+1}=\dfrac{an^2}{n+2} $}
	\loigiai
	{Ta có $u_{n+1}=\dfrac{a(n+1)^2}{n+1+1}=\dfrac{a(n+1)^2}{n+2}$.
	}
\end{ex}
%cau3
\begin{ex}%[DCHT Toán 11 - KNTT -Tên GV]%[1K2B1-1]
	Cho dãy số có các số hạng đầu là $8,15,22,29,36, \ldots$ Số hạng tổng quát của dãy số này là
	\choice
	{$ u_n=7n+7 $}
	{$ u_n=7n $}
	{\True $ u_n=7n+1 $}
	{$ u_n$ không viết được dưới dạng công thức }
	\loigiai
	{Ta có $8=7\cdot 1+1, 15=7 \cdot 2+1, 22 = 7\cdot 3+1, 29=7 \cdot 4+1, 36 = 7\cdot 5+1, \ldots$\\
		Vậy dãy trên có số hạng tổng quát là $u_n=7n+1$.
	}
\end{ex}
%Cau4
\begin{ex}%[DCHT Toán 11 - KNTT -Tên GV]%[1K2B1-1]
	Cho dãy số có các số hạng đầu là $0,\dfrac{1}{2},\dfrac{2}{3},\dfrac{3}{4},\dfrac{4}{5}, \ldots$ Số hạng tổng quát của dãy số này là
	\choice
	{$ u_n=\dfrac{n+1}{n}$}
	{\True $ u_n=\dfrac{n}{n+1} $}
	{$ u_n=\dfrac{n-1}{n}$}
	{$ u_n=\dfrac{n^2-n}{n+1}$  }
	\loigiai
	{Ta có $0=\dfrac{0}{0+1}, \dfrac{1}{2}=\dfrac{1}{1+1} ,\dfrac{2}{3} = \dfrac{2}{2+1}, \dfrac{3}{4}=\dfrac{3}{3+1}, \dfrac{4}{5} = \dfrac{4}{4+1}, \ldots$\\
		Vậy dãy trên có số hạng tổng quát là $u_n=\dfrac{n}{n+1}$.
	}
\end{ex}
%Cau5
\begin{ex}%[DCHT Toán 11 - KNTT -Tên GV]%[1K2B1-1]
	Cho dãy số $(u_n)$ với $u_1=1, u_{n+1}=u_n-1$. Số hạng tổng quát $u_n$ của dãy số là số hạng nào dưới đây?
	\choice
	{\True $ u_n=2-n$}
	{$ u_n$ không xác định}
	{$ u_n=1-n$}
	{$ u_n=-n$, với mọi $n$ }
	\loigiai
	{Ta có $u_1=1, u_2=0 ,u_3 = -1, u_4=-2,  \ldots$\\
		Dễ dàng dự đoán được số hạng tổng quát là $u_n=2-n$.
	}
\end{ex}
% %cau6
% \begin{ex}%[DCHT Toán 11 - KNTT -Tên GV]%[1K2B1-1]
% 	Cho dãy số $(u_n)$ với $u_n=\dfrac{2n^2-1}{n^2+3}, \forall n \in \mathrm{N*}$. Số hạng đầu tiên của dãy số là 
% 	\choice
% 	{$ u_1=-\dfrac{1}{3}$}
% 	{$ u_1=\dfrac{2}{3}$}
% 	{$ u_1=\dfrac{1}{3}$}
% 	{\True $ u_1=\dfrac{1}{4}$ }
% 	\loigiai
% 	{Ta có $u_1=\dfrac{2\cdot 1^2-1}{1^2+3}=\dfrac{1}{4}$.
% 	}
% \end{ex}
% %cau7
% \begin{ex}%[DCHT Toán 11 - KNTT -Tên GV]%[1K2B1-1]
% 	Cho dãy số $(u_n)$ với $u_1=-1, u_{n+1}=u_n+3$ với $n \ge 1$. Ba số hạng đầu tiên của dãy số lần lượt là 
% 	\choice
% 	{\True $-1, 2, 5$}
% 	{$ 1, 4, 7$}
% 	{$ 4,7,10$}
% 	{$-1,3,7$ }
% 	\loigiai
% 	{Ta có $u_1=-1, u_2=-1+3=2 ,u_3 = 2+3=5$.
% 	}
% \end{ex}
\Closesolutionfile{ans}
\begin{indapan}{10}
	{ans/ans-1K2-1-Dang1}
\end{indapan}

\begin{dang}{Tìm số hạng cụ thể của dãy số}
	Để tìm số hạng cụ thể của dãy số ta làm như sau
	\begin{itemize} 
		\item Với trường hợp dãy số đã cho biết công thức tổng quát của dãy số thì ta chỉ cần thay giá trị tương ứng của số hạng đó vào công thức tổng quát.
		\item  Với trường hợp dãy số cho bởi công thức truy hồi hoặc dưới dạng thì ta phải tìm lần lượt từ những số hạng đầu tiên cho đến số đứng trước số cần tìm trong dãy.
	\end{itemize}
\end{dang}
\subsubsection{Ví dụ minh hoạ}
\begin{vd}[NB]%[1K2Y5-2]
	Cho dãy số $(u_n),$ biết $u_n=(-1 )^n\cdot \dfrac{2^n}{n}$. Tìm số hạng $u_3$.
	\dapso{$u_3=-\dfrac{8}{3}$}
	\choice
	{\True $u_3=-\dfrac{8}{3}$}
	{$u_3=2$}
	{$u_3=-2$}
	{$u_3=\dfrac{8}{3}$}
	\loigiai{
		Ta có
		$$u_3=(-1)^3\cdot \dfrac{2^3}{3}=-\dfrac{8}{3}.$$}
\end{vd}
\begin{vd}[NB]%[DCHT Toán 11 - KNTT -Nguyễn Long]%[1K2Y5-2]
	Cho dãy số $(u_n)$, biết $u_n=\dfrac{2n^2-1}{n^2+3}$. Tìm số hạng $u_5$.
	\dapso{$u_5=\dfrac{7}{4}$}
	\choice
	{$u_5=\dfrac{1}{4}$}
	{\True $u_5=\dfrac{7}{4}$}
	{$u_5=\dfrac{17}{12}$}
	{$u_5=\dfrac{71}{39}$}
	\loigiai{
		Ta có $u_5=\dfrac{2\cdot 5^2-1}{5^2+3}=\dfrac{49}{28}=\dfrac{7}{4}$.}
	
\end{vd}
\begin{vd}[NB]%[1K2Y5-2]
	Cho dãy số $u_n$ bao gồm các số nguyên tố. Tìm số hạng thứ $5$ của dãy số.
	\dapso{$u_5=11$}
	\loigiai{ 
		Ta có
		$u_1=2,u_2=3,u_3=5,u_4=7,u_5=11$. \\
		Vậy số hạng thứ $5$ của dãy số là $11$.
	}
\end{vd}
\begin{vd}[NB]%[1K2Y5-2]
	Cho dãy số $(u_n) $ thỏa mãn $ \heva{& u_1 = 5 \\& u_{n+1} = u_n+n}$. Tìm số hạng thứ $5$ của dãy số.
	\dapso{$u_5=15$}
	\choice
	{$ 11 $}
	{\True $ 15 $}
	{$ 16 $}
	{$ 12 $}
	\loigiai{
		Ta có $ u_2=u_1+1=6$, $ u_3=u_2+2=8$, $ u_4=u_3+3=11$,  $ u_5=u_4+4=15$.
	}
\end{vd}

\begin{vd}[TH]%[VD 5 SGK KNTT]%[1K2B5-2]
	Cho dãy số xác định bằng hệ thức truy hồi
	$$
	u_1=1, u_n=3 u_{n-1}+2 \text { với } n \geq 2
	$$
	Viết ba số hạng đầu của dãy số này.
	\dapso{$u_5=17$}
	\loigiai{
		Ta có: $u_1=1, u_2=3 u_1+2=3 \cdot 1+2=5, u_3=3 u_2+2=3 \cdot 5+2=17$.
	}
\end{vd}

\begin{vd}[VD]%[1K2B5-2]
	Cho dãy số $\left(u_n\right)\colon\heva{&u_1=5 \\ &u_{n+1}=u_n+n}$. Số $20$ là số hạng thứ mấy trong dãy?
	\dapso{số hạng thứ $6$}
	\loigiai{
		Ta có $u_1=5, u_2=6, u_3=8, u_4=11, u_5=16, u_6=20$.\\
		Vậy số $20$ là số hạng thứ $6$.}
\end{vd}

\subsubsection{Bài tập tự luận}
 
\begin{bt}[NB]%[1K2Y5-2]
	Cho dãy số $u_n=\dfrac{1}{\sqrt{n}+1}$. Tìm số hạng $u_4$.	
	\dapso{$u_4=\dfrac{1}{3}$}
	\loigiai{ Ta có
		$u_4=\dfrac{1}{\sqrt{4}}+1=\dfrac{1}{3}.$		
	}
\end{bt}
%%%
\begin{bt}[NB]%[1K2Y5-2]
	Cho dãy số $(u_n)$ có số hạng tổng quát: $u_n=2 n+\sqrt{n^2+4}$. Tìm số hạng thứ $6$ của dãy số.
	\dapso{$u_6=12+2\sqrt{10}$}
	\loigiai{
		Ta có $u_6=12+2 \sqrt{10}$.
	}
\end{bt}
%%%
\begin{bt}[NB]%[1K2Y5-2]
	Cho dãy số $(u_n)$ xác định bởi: $\heva{&u_1=-1 ; u_2=3 \\&u_{n+1}=5 u_n-6 u_{n-1} \forall n \geq 2}.$ Tìm số hạng thứ $7$ của dãy.
	\dapso{$3261$}
	\loigiai{
		Ta có
		$$
		u_3=5 u_2-6 u_1=21 ;~ u_4=5 u_3-6 u_2=87 ;~ u_5=309 ;~ u_6=1023 ;~ u_7=3261
		$$
		Vậy số hạng thứ $7$ của dãy là $3261$.
	}
\end{bt}
%%%%%%%
\begin{bt}[NB]%[1K2Y5-2]
	Viết năm số hạng đầu của dãy số Fibonacci $\left(F_n\right)$ cho bởi hệ thức truy hồi
	$$
	\heva{
		&F_1=1, F_2=1 \\
		&F_n=F_{n-1}+F_{n-2}~(n \geq 3) .
	}
	$$
	\dapso{$F_3=2,~F_4=3,~F_5=5$}
	\loigiai{
		Ta có $F_3=2,~F_4=3,~F_5=5.$
	}
\end{bt}
%%%
\begin{bt}[NB]%[1K2T5-2]
	Người ta nuôi cấy $5$ con vi khuẩn E-coli trong môi trường nhân tạo. Cứ $30$ phút thì vi khuẩn E-coli sẽ nhân đôi 1 lần. Tính số lượng vi khuẩn thu được sau $1,2,3$ lần nhân đôi.
	\dapso{$u_2=10, u_3=20, u_4=40$}
	\loigiai{
		Đặt $u_1=5$, gọi số vi khuẩn sau $n$ lần phân chia là $u_{n+1}$, khi đó ta có dãy số $(u_n)$ thỏa mãn $$u_1=5, \; u_{n+1}=2u_n$$
		Ta có $u_2=10, u_3=20, u_4=40$.
	}	
\end{bt}
%%%%%
\begin{bt}[TH]%[1K2B5-2]
	Cho dãy số $(u_n)$ được xác định bởi $u_n=\dfrac{n^2+3n+7}{n+1}$.
	\begin{listEX}
		\item Viết năm số hạng đầu của dãy.
		\item Dãy số có bao nhiêu số hạng nhận giá trị nguyên.
	\end{listEX}
	\dapso{$u_1=\dfrac{11}{2}$; $u_2=\dfrac{17}{3}$; $u_3=\dfrac{25}{4}$; $u_4=7$; $u_5=\dfrac{47}{6}$. $u_4=7 $}
	\loigiai{		
		\begin{listEX}
			\item Ta có năm số hạng đầu của dãy
			$u_1=\dfrac{1^2+3.1+7}{1+1}=\dfrac{11}{2}$; $u_2=\dfrac{17}{3}$; $u_3=\dfrac{25}{4}$; $u_4=7$; $u_5=\dfrac{47}{6}$.
			\item Ta có: $u_n=n+2+\dfrac{5}{n+1}$, do đó $u_n$ nguyên khi và chỉ khi $ \dfrac{5}{n+1}$ nguyên hay $ n+1 $ là ước của 5. Điều đó xảy ra khi $ n+1=5\Leftrightarrow n=4 $. Vậy dãy số có duy nhất một số hạng nguyên là $u_4=7 $.
		\end{listEX}		
	}
\end{bt}
\begin{bt}[VD]%[1K2K5-2]
	Cho dãy số $\left(x_n\right)$ thỏa mãn điều kiện $x_1=1, x_{n+1}-x_n=\dfrac{1}{n(n+1)}, n=1,2,3, \ldots$. Số hạng $x_{2023}$ bằng
	\dapso{$x_{2023}=\dfrac{4045}{2023}$}
	\loigiai{
		Ta có
		$$
		\begin{aligned}
			x_{n+1}-x_n=\dfrac{1}{n(n+1)}=\dfrac{1}{n}-\dfrac{1}{n+1} & \Leftrightarrow \sum_{k=1}^{n-1}\left(x_{k+1}-x_k\right)=\sum_{k=1}^{n-1}\left(\dfrac{1}{k}-\dfrac{1}{k+1}\right) \\
			& \Leftrightarrow x_n-x_1=1-\dfrac{1}{n} \\
			& \Leftrightarrow x_n=\dfrac{2n-1}{n} .
		\end{aligned}
		$$
	}
\end{bt}
\begin{bt}[VDC]%[1K2G5-2]
	Cho dãy số $\left(u_n\right)$ biết $\heva{&u_1=99 \\&u_{n+1}=u_n-2 n-1, n \geq 1}$. Hỏi số $-861$ là số hạng thứ mấy?
	\dapso{$-861$ là số hạng thứ $31$}
	\loigiai{
		Ta có
		$$
		\begin{aligned}
			&u_n &=& &u_{n-1}-2 n+1 \\
			&u_{n-1} & = & &u_{n-2}-2 n+3 \\
			&\vdots &\vdots&  &\vdots \\
			&u_3 & = & &u_2-2 n+2 n-5 \\
			&u_2 & = & &u_1-2 n+2 n-3
		\end{aligned}
		$$
		Suy ra
		$$
		\begin{aligned}
			& u_n=u_1-2 n \cdot(n-1)+1+3+5+\cdots+(2 n-5)+(2 n-3) \\
			& u_n=99-2 n^2+2 n+\dfrac{n-1}{2}\cdot[2 \cdot 1+(n-2) \cdot 2]=100-n^2
		\end{aligned}
		$$
		Giả sử $u_n=-861 \Rightarrow n^2=961 \Rightarrow n=31$ (vì $n \in \mathbb{N}$).
		Vậy số $-861$ là số hạng thứ $31$ .}
\end{bt}
\subsubsection{Câu hỏi trắc nghiệm}
\Opensolutionfile{ans}[ans/ans-1K2-1-Dang2]
%Câu 1
\begin{ex}%[1K2Y5-2]
	Cho dãy số $({{u}_{n}} )$, biết ${{u}_{n}}=\dfrac{n}{{{3}^{n}}-1}$. Ba số hạng đầu tiên của dãy số đó lần lượt là những số nào dưới đây?
	\choice
	{$\dfrac{1}{2};\dfrac{1}{4};\dfrac{1}{16}$}
	{$\dfrac{1}{2};\dfrac{2}{3};\dfrac{3}{4}$}
	{ \True $\dfrac{1}{2};\dfrac{1}{4};\dfrac{3}{26}$}
	{$\dfrac{1}{2};\dfrac{1}{4};\dfrac{1}{8}$}
	\loigiai {
		Ta có
		${{u}_{1}}=\dfrac{1}{2};\,\,{{u}_{2}}=\dfrac{2}{{{3}^2}-1}=\dfrac{2}{8}=\dfrac{1}{4};\,\,{{u}_{3}}=\dfrac{3}{{{3}^3}-1}=\dfrac{3}{26}$.}
	
\end{ex}
%%%%%%%%%%
%Câu 2
\begin{ex}%[1K2Y5-2]
	Cho dãy số $({{u}_{n}} ),$ biết ${{u}_{n}}={{(-1 )}^{n}}\cdot 2n$. Mệnh đề nào sau đây {\bf sai}?
	\choice
	{${{u}_{3}}=-6$}
	{${{u}_{2}}=4$}
	{ \True ${{u}_{4}}=-8$}
	{${{u}_{1}}=-2$}
	\loigiai {
		Ta có\\
		${{u}_{1}}=-2\cdot 1=-2;\,\,{{u}_{2}}={{(-1 )}^2}\cdot 2\cdot 2=4,\,\,{{u}_{3}}={{(-1 )}^3}\cdot 2\cdot 3=-6;\,\,{{u}_{4}}={{(-1 )}^4}\cdot 2\cdot 4=8$.\\
		\textbf{Nhận xét:} Dễ thấy ${{u}_{n}}>0$ khi $n$ chẵn và ngược lại nên đáp án $u_4=-8$ sai.}
	
\end{ex}
%%%%%%%%%%
%Câu 3
\begin{ex}%[1K2Y5-2]
	Cho dãy số $({{u}_{n}} )$ xác định bởi $\heva{
		& {{u}_{1}}=2 \\
		& {{u}_{n+1}}=\dfrac{1}{3}({{u}_{n}}+1 ) \\
	}.$ Tìm số hạng ${{u}_{4}}$.
	\choice
	{${{u}_{4}}=\dfrac{2}{3}$}
	{${{u}_{4}}=1$}
	{${{u}_{4}}=\dfrac{14}{27}$}
	{ \True ${{u}_{4}}=\dfrac{5}{9}$}
	\loigiai {
		Ta có
		${{u}_{2}}=\dfrac{1}{3}({{u}_{1}}+1 )=\dfrac{1}{3}(2+1 )=1;\,\,{{u}_{3}}=\dfrac{1}{3}({{u}_{2}}+1 )=\dfrac{2}{3};\,\,{{u}_{4}}=\dfrac{1}{3}({{u}_{3}}+1 )=\dfrac{1}{3}\cdot\left(\dfrac{2}{3}+1\right)=\dfrac{5}{9}$. \\
	}
\end{ex}
%%%%%%%%%%
%Câu 4
\begin{ex}%[1K2Y5-2]
	Cho dãy số $({{u}_{n}} )$, biết $\heva{
		& {{u}_{1}}=-1 \\
		& {{u}_{n+1}}={{u}_{n}}+3 \\
	}$ với $n\ge 0$. Ba số hạng đầu tiên của dãy số đó là lần lượt là những số nào dưới đây?
	\choice
	{\True $-1;\,2;\,5$}
	{$-1;3;7$}
	{$1;\,4;\,7$}
	{$4;\,7;\,10$}
	\loigiai {
		Ta có ${{u}_{1}}=-1;\,\,{{u}_{2}}={{u}_{1}}+3=2;\,\,{{u}_{3}}={{u}_{2}}+3=5$. \\
		\textbf{Nhận xét.} (i) Dùng chức năng “lặp” của MTCT để tính:\\
		Nhập vào màn hình: $X=X+3$ \\
		Bấm CALC và cho $X=-1$ (ứng với ${{u}_{1}}=-1)$ \\
		Để tính ${{u}_{n}}$ cần bấm “=” ra kết quả liên tiếp $n-1$ lần. Ví dụ để tính ${{u}_{2}}$ ta bấm “=” ra kết quả lần đầu tiên, bấm “=” ra kết quả thứ hai chính là ${{u}_{3}},\ldots$\\
		(ii) Vì ${{u}_{1}}=-1$ nên loại các đáp án $u_1=1$, $u_1=4$.\\
		Còn lại các đáp án có $u_1=-1$; để biết đáp án nào ta chỉ cần kiểm tra ${{u}_{2}}$ (vì ${{u}_{2}}$ ở hai đáp án là khác nhau): ${{u}_{2}}={{u}_{1}}+3=2$.
	}
	
\end{ex}
%%%%%%%%%%
%Câu 5
\begin{ex}%[1K2B5-2]
	Cho dãy số $({{u}_{n}} ),$ biết ${{u}_{n}}=\dfrac{2n+5}{5n-4}$. Số $\dfrac{7}{12}$ là số hạng thứ mấy của dãy số?
	\choice
	{$9$}
	{$6$}
	{$10$}
	{\True $8$}
	\loigiai {
		Ta có
		$${{u}_{n}}=\dfrac{2n+5}{5n-4}=\dfrac{7}{12}\Leftrightarrow 24n+60=35n-28\Leftrightarrow 11n=88\Leftrightarrow n=8.$$}
	
\end{ex}
%%%%%%%%%%
%Câu 6
\begin{ex}%[1K2B5-2]
	Cho dãy $(u_n)$ xác định bởi $\heva{& u_1=3 \\& u_{n+1}=\dfrac{u_n}{2}+2}$. Mệnh đề nào sau đây {\bf sai}?
	\choice
	{\True ${{u}_{2}}=\dfrac{5}{2}$}
	{${{u}_{4}}=\dfrac{31}{8}$}
	{${{u}_{3}}=\dfrac{15}{4}$}
	{${{u}_{5}}=\dfrac{63}{16}$}
	\loigiai {
		Ta có $\heva{
			& {{u}_{2}}=\dfrac{{{u}_{1}}}{2}+2=\dfrac{3}{2}+2=\dfrac{7}{2};\,\,{{u}_{3}}=\dfrac{{{u}_{2}}}{2}+2=\dfrac{7}{4}+2=\dfrac{15}{4}. \\
			& {{u}_{4}}=\dfrac{{{u}_{3}}}{2}+2=\dfrac{15}{8}+2=\dfrac{31}{8};\,\,{{u}_{5}}=\dfrac{{{u}_{4}}}{2}+2=\dfrac{31}{16}+2=\dfrac{63}{16}. \\
		}$}
\end{ex}
%%%%%%%%%%
%Câu 7
\begin{ex}%[1K2B5-2]
	Cho dãy số $({{u}_{n}} ),$ với ${{u}_{n}}={{\left(\dfrac{n-1}{n+1} \right)}^{2n+3}}$. Tìm số hạng ${{u}_{n+1}}$.
	\choice
	{${{u}_{n+1}}={{\left(\dfrac{n-1}{n+1} \right)}^{2(n-1 )+3}}$}
	{${{u}_{n+1}}={{\left(\dfrac{n-1}{n+1} \right)}^{2(n+1 )+3}}$ }
	{\True ${{u}_{n+1}}={{\left(\dfrac{n}{n+2} \right)}^{2n+5}}$}
	{${{u}_{n+1}}={{\left(\dfrac{n}{n+2} \right)}^{2n+3}}$}
	\loigiai {
		${{u}_{n}}={{\left(\dfrac{n-1}{n+1} \right)}^{2n+3}}\Rightarrow {{u}_{n+1}}={{\left(\dfrac{(n+1 )-1}{(n+1 )+1} \right)}^{2(n+1 )+3}}={{\left(\dfrac{n}{n+2} \right)}^{2n+5}}$.}
	
\end{ex}
%%%%%%%%%%
%Câu 8
\begin{ex}%[1K2K5-2]
	Cho dãy số $({{a}_{n}} ),$ được xác định $\heva{
		& {{a}_{1}}=3 \\
		& {{a}_{n+1}}=\dfrac{1}{2}{{a}_{n}},~n\ge 1 \\
	}$. Mệnh đề nào sau đây {\bf sai}?
	\choice
	{${{a}_{1}}+{{a}_{2}}+{{a}_{3}}+{{a}_{4}}+{{a}_{5}}=\dfrac{93}{16}$}
	{${{a}_{10}}=\dfrac{3}{512}$}
	{ \True ${{a}_{n}}=\dfrac{3}{{{2}^{n}}}$}
	{${{a}_{n+1}}+{{a}_{n}}=\dfrac{9}{{{2}^{n}}}$}
	\loigiai {
		Ta có ${{a}_{1}}=3;\,{{a}_{2}}=\dfrac{{{u}_{1}}}{2};\,\,{{a}_{3}}=\dfrac{{{u}_{2}}}{2}=\dfrac{{{u}_{1}}}{{{2}^2}};\,\,{{a}_{4}}=\dfrac{{{u}_{3}}}{2}=\dfrac{{{u}_{1}}}{{{2}^3}},\ldots \\
		\Rightarrow {{u}_{n}}=\dfrac{{{u}_{1}}}{{{2}^{n-1}}}=\dfrac{3}{{{2}^{n-1}}}$ nên suy ra đáp án ${{a}_{n}}=\dfrac{3}{{{2}^{n}}}$ sai. \\
		Xét đáp án\\
		${{a}_{1}}+{{a}_{2}}+{{a}_{3}}+{{a}_{4}}+{{a}_{5}}=3\left(1+\dfrac{1}{2}+\dfrac{1}{{{2}^2}}+\dfrac{1}{{{2}^3}}+\dfrac{1}{{{2}^4}}\right)=3.\dfrac{1-{{(\dfrac{1}{2} )}^5}}{1-\dfrac{1}{2}}=\dfrac{93}{16}\Rightarrow $  đúng.\\
		Xét đáp án ${{a}_{10}}=\dfrac{3}{{{2}^{9}}}=\dfrac{3}{512}\Rightarrow $  đúng.\\
		Xét đáp án ${{a}_{n+1}}+{{a}_{n}}=\dfrac{3}{{{2}^{n}}}+\dfrac{3}{{{2}^{n-1}}}=\dfrac{3+3\cdot 2}{{{2}^{n}}}=\dfrac{9}{{{2}^{n}}}\Rightarrow $ đúng.}
	
\end{ex}
%%%%%%%%%%
%Câu 9
\begin{ex}%[1K2K5-2]
	Cho dãy số $(u_n)$ biết $\heva{&u_1=1\\&u_2=4\\&u_{n+2}=3u_{n+1}-2u_n}$ với mọi $n \ge 1$. Giá trị $u_{101}-u_{100}$ là 
	\choice
	{$3\cdot 2^{102} $}
	{$3\cdot 2^{101} $}
	{$3\cdot 2^{100} $}
	{\True $ 3\cdot 2^{99}$}
	\loigiai{
		Theo bài  ta có 
		\begin{eqnarray*}
			&u_{n+2}=3u_{n+1}-2u_n\\
			\Leftrightarrow \,& u_{n+2}=u_{n+1}+2(u_{n+1}-u_n)\\
			\Leftrightarrow \,& u_{n+2}-u_{n+1}=2(u_{n+1}-u_n).
		\end{eqnarray*}
		Với $n=99$ ta có 
		\begin{align*}
			u_{101}-u_{100}&=2(u_{100}-u_{99})\\
			&=2\cdot 2 (u_{99}-u_{98})\\
			&= \ldots\\
			&=2^{99}\cdot(u_2-u_1)=3\cdot2^{99}.
		\end{align*}
	}
\end{ex}
%%%%%%%%%%
%Câu 10
\begin{ex}%[1K2G5-2]
	Cho dãy số $\left(u_n\right)$ thoả mãn $u_1=\sqrt{2}$ và $u_{n+1}=\sqrt{2+u_n}$ với mọi $n\geq 1$. Tìm $u_{2023}$.
	\choice
	{$u_{2023}=\sqrt{2}\cos\dfrac{\pi}{2^{2022}}$}
	{\True $u_{2023}=\sqrt{2}\cos\dfrac{\pi}{2^{2024}}$}
	{$u_{2023}=\sqrt{2}\cos\dfrac{\pi}{2^{2023}}$}
	{$u_{2023}=2$}
	\loigiai{Ta chứng minh bằng phương pháp quy nạp số hạng tổng quát của dãy là $u_n=2\cos\dfrac{\pi}{2^{n+1}}$.\\
		Dễ thấy, với $n=1$ ta có $u_1=\sqrt{2}$ (đúng).\\
		Giả sử mệnh đề đúng với $n=k, \forall k\in \mathbb{N}^\ast$ nghĩa là $u_k=2\cos\dfrac{\pi}{2^{k+1}}$ ta phải chứng minh mệnh đề đúng với $n=k+1$ nghĩa là $u_{k+1}=2\cos\dfrac{\pi}{2^{k+2}}$.\\
		Thật vậy, $u_{k+1}=\sqrt{2+u_k}=\sqrt{2+2\cos\dfrac{\pi}{2^{k+1}}}=\sqrt{4\cos^2\dfrac{\pi}{2^{k+2}}}=2\cos\dfrac{\pi}{2^{k+2}}$.\\
		Áp dụng công thức tổng quát trên ta có $u_{2023}=\sqrt{2}\cos\dfrac{\pi}{2^{2024}}$.
	}
\end{ex}
%%%%%%%%%%
\Closesolutionfile{ans}
\begin{indapan}{10}
	{ans/ans-1K2-1-Dang2}
\end{indapan}
\begin{dang}{Xét tính tăng giảm của dãy số}
	\begin{enumerate}
		\item Phương pháp 1. Xét dấu của hiệu số $u_{n+1}-u_n$.
		\begin{enumerate}
			\item Nếu $u_{n+1}-u_n>0, \forall n \in \mathbb{N}^\ast$ thì $(u_n)$ là dãy số tăng.
			\item Nếu $u_{n+1}-u_n<0, \forall n \in \mathbb{N}^\ast$ thì $(u_n)$ là dãy số giảm.
		\end{enumerate}
		\item Phương pháp 2. Nếu $u_n>0, \forall n\in \mathbb{N}^\ast$ thì ta có thể so sánh thương $\dfrac{u_{n+1}}{u_n}$ với $1$.
		\begin{enumerate}
			\item Nếu $\dfrac{u_{n+1}}{u_n}>1$ thì $(u_n)$ là dãy số tăng.
			\item Nếu $\dfrac{u_{n+1}}{u_n}<1$ thì $(u_n)$ là dãy số giảm.
		\end{enumerate}
		Nếu $u_n<0, \forall n\in \mathbb{N}^\ast$ thì ta có thể so sánh thương $\dfrac{u_{n+1}}{u_n}$ với $1$.
		\begin{enumerate}
			\item Nếu $\dfrac{u_{n+1}}{u_n}<1$ thì $(u_n)$ là dãy số tăng.
			\item Nếu $\dfrac{u_{n+1}}{u_n}>1$ thì $(u_n)$ là dãy số giảm.
		\end{enumerate}
		\item Phương pháp 3. Nếu dãy số $(u_n)$ cho bởi hệ thức truy hồi thì thường dùng phương pháp quy nạp để chứng minh $u_{n+1}>u_n, \forall n \in \mathbb{N}^\ast$ (hoặc $u_{n+1}<u_n \forall n \in \mathbb{N}^\ast$).
	\end{enumerate}
\end{dang}
\subsubsection{Ví dụ minh hoạ}
\begin{vd}[NB]%[1K2Y5-3]
	Xét sự tăng giảm của dãy số $(u_n)$ với $u_n=(-1)^n$.
	\dapso{dãy không tăng không giảm}
	\loigiai{
		Ta có:\\ $u_1=(-1)^1=-1,\,
		u_2=(-1)^2=1,\,
		u_3=(-1)^3=-1.$\\
		Vậy $(u_n)$ là dãy không tăng không giảm.
	}
\end{vd}
\begin{vd}[NB]%[1K2Y5-3]
	Xét tính tăng giảm của dãy số sau $(u_n)$ với $u_n=\dfrac{2n+1}{n+1}$.
	\dapso{dãy số tăng}
	\loigiai
	{
		Ta có: $u_n=\dfrac{2n+1}{n+1}=2-\dfrac{1}{n+1}$.\\
		$u_{n+1}-u_n=\left(2-\dfrac{1}{n+1+1}\right)-\left(2-\dfrac{1}{n+1}\right)=\dfrac{1}{n+1}-\dfrac{1}{n+2}>0, \forall n \in \mathbb{N}^\ast$.\\
		Vậy dãy số $(u_n)$ là dãy số tăng.
	}
\end{vd}
\begin{vd}[TH]%[1K2B5-3]
	Xét tính tăng giảm của dãy số $(u_n)$ với $u_n=\sqrt{n}-\sqrt{n+2}$.
	\dapso{dãy số tăng}
	\loigiai{
		Ta có $u_n=\sqrt{n}-\sqrt{n+2}=\dfrac{-2}{\sqrt{n}+\sqrt{n+2}}$.\\
		Xét hiệu\\ 
		$\begin{aligned}
			u_{n+1}-u_n&=\dfrac{-2}{\sqrt{n+1}+\sqrt{n+3}}-\dfrac{-2}{\sqrt{n}+\sqrt{n+2}}\\
			&=\dfrac{2}{\sqrt{n}+\sqrt{n+2}}-\dfrac{2}{\sqrt{n+1}+\sqrt{n+3}}>0, \forall n\in \mathbb{N}^\ast.
		\end{aligned}$\\
		Vậy $(u_n)$ là dãy số tăng.
	}
\end{vd}

\begin{vd}[TH]%[1K2B5-3]
	Xét tính tăng giảm của dãy số $(u_n)$ với $u_n=\dfrac{n}{3^n}$.
	\dapso{dãy số giảm}
	\loigiai{
		Ta có $u_n=\dfrac{n}{3^n}>0, \forall n \in \mathbb{N}^\ast$.\\
		Xét thương $\dfrac{u_{n+1}}{u_n}=\dfrac{n+1}{3^{n+1}}:\dfrac{n}{3^n}=\dfrac{n+1}{3.n}<1, \forall  n \in \mathbb{N}^\ast$.\\
		Vậy $(u_n)$ là dãy số giảm.
	}
\end{vd}
\begin{vd}[VD]%[1K2K5-3]
	Xét tính tăng giảm của dãy số $(u_n)$ với $ \heva{& u_1=2\\
		& u_{n+1}=\dfrac{3u_n+1}{u_n+1}, n\in \mathbb{N}^\ast.}$
	\dapso{dãy số tăng}
	\loigiai{
		Giả sử $u_{n+1}>u_n , \forall n \in \mathbb{N}^\ast. \qquad (*)$\\
		Ta chứng minh $(*)$ bằng phương pháp quy nạp.
		\begin{itemize}
			\item Với $n=1, u_2=\dfrac{3.2+1}{2+1}=\dfrac{6}{3}=\dfrac{7}{3}>u_1=2.$
			\item Giả sử $(*)$ đúng khi $n=k, k\in \mathbb{N}^\ast$, tức là $u_{k+1}>u_k$.\\
			Ta sẽ chứng minh $(*)$ đúng với $n=k+1$, tức là
			$u_{k+2}>u_{k+1}$.\\
			Thật vậy\\ $u_{k+2}-u_{k+1}=\left(3-\dfrac{2}{u_{k+1}+1}\right)-\left(3-\dfrac{2}{u_k+1}\right)=\dfrac{2}{u_k+1}-\dfrac{2}{u_{k+1}+1}.$\\
			Theo giả thiết quy nạp ta có:\\ $u_{k+1}>u_k \Rightarrow u_{k+1}+1>u_k+1 \Rightarrow \dfrac{2}{u_k+1}>\dfrac{2}{u_{k+1}+1}$.\\
			Vậy $u_{k+2}-u_{k+1}>0$.\\
			Do đó, $(*)$ đúng với mọi số nguyên dương $n$.
		\end{itemize}
		Vậy $(u_n)$ là dãy số tăng.}
\end{vd}
\subsubsection{Bài tập tự luận}
 
\begin{bt}[NB]%[1K2Y5-3]
	Xét tính tăng giảm của dãy số $(u_n)$ với $u_n=\dfrac{\sqrt{2}}{3^n}$.
	\dapso{dãy số giảm}
	\loigiai{
		Ta có $u_n>0, \forall n \in \mathbb{N}^\ast$.\\
		Xét thương $$\dfrac{u_{n+1}}{u_n}=\dfrac{\sqrt{2}}{3^{n+1}}: \dfrac{\sqrt{2}}{\sqrt{3^2}}=\dfrac{3^n}{3^{n+1}}=\dfrac{1}{3}<1.$$
		Vậy $\left(u_n\right)$ là dãy số giảm.
	}
\end{bt}
\begin{bt}[NB]%[1K2Y5-3]
	Xét tính tăng giảm của dãy số $\left(u_n\right)$ với $u_n=\dfrac{1}{n(n+1)}$.
	\dapso{dãy số tăng}
	\loigiai{
		Ta có $u_n=\dfrac{1}{n(n+1)}=\dfrac{1}{n}-\dfrac{1}{n+1}$.
		Xét hiệu:
		$$
		\begin{aligned}
			u_{n+1}-u_n & =\left(\dfrac{1}{n}-\dfrac{1}{n+1}\right)-\left(\dfrac{1}{n+1}-\dfrac{1}{n+2}\right) \\
			& =\dfrac{1}{n}-\dfrac{1}{n+2}>0, \forall n \in \mathbb{N}^\ast
		\end{aligned}
		$$
		Vậy  $\left(u_n\right)$ là dãy số tăng.
	}
\end{bt}
\begin{bt}[TH]%[1K2B5-3]
	Xét tính tăng giảm của dãy số $\left(u_n\right)$ với $u_n=n+\cos ^2 n$.
	\dapso{dãy số tăng}
	\loigiai{
		Xét hiệu
		$$
		\begin{aligned}
			u_{n+1}-u_n & =\left(n+1+\cos ^2(n+1)\right)-\left(n+\cos ^2 n\right) \\
			& =1+\cos ^2(n+1)-\cos ^2 n \\
			& =\cos ^2(n+1)+\sin ^2 n>0, \forall n \in \mathbb{N}^\ast .
		\end{aligned}
		$$
		Vậy $\left(u_n\right)$ là dãy số tăng.
	}
\end{bt}
\begin{bt}[TH]%[1K2B5-3]
	Xét tính tăng giảm của dãy số $(u_n)$ với $u_n=\dfrac{1}{n+1}+\dfrac{1}{n+2}+\ldots+\dfrac{1}{2n}$.
	\dapso{dãy số giảm}
	\loigiai{
		Xét hiệu\\
		$\begin{aligned}
			u_{n+1}-u_n&=\left(\dfrac{1}{n+2}+\dfrac{1}{n+3}+\ldots+\dfrac{1}{2(n+1)}\right)-\left(\dfrac{1}{n+1}+\dfrac{1}{n+2}+\ldots+\dfrac{1}{2n}\right)\\
			&=\dfrac{1}{n+2}-\dfrac{1}{2n+1}-\dfrac{1}{2n+2}\\
			&=\dfrac{1}{2n+2}-\dfrac{1}{2n+1}<0, \forall n\in \mathbb{N}^\ast.
		\end{aligned}$\\
		Vậy $(u_n)$ là dãy số giảm.
	}
	
\end{bt}

\begin{bt}[TH]%[1K2B5-3]
	Xét tính tăng giảm của dãy số $\left(u_n\right)$ với $u_n=\dfrac{1}{n+1}+\dfrac{1}{n+2}+\ldots+\dfrac{1}{2 n}$.
	\dapso{dãy số giảm}
	\loigiai{
		Xét hiệu
		$$
		\begin{aligned}
			u_{n+1}-u_n & =\left(\dfrac{1}{n+2}+\dfrac{1}{n+3}+\ldots+\dfrac{1}{2(n+1)}\right)-\left(\dfrac{1}{n+1}+\dfrac{1}{n+2}+\ldots+\dfrac{1}{2 n}\right) \\
			& =\dfrac{1}{n+2}-\dfrac{1}{2 n+1}-\dfrac{1}{2 n+2} \\
			& =\dfrac{1}{2 n+2}-\dfrac{1}{2 n+1}<0, \forall n \in \mathbb{N}^\ast
		\end{aligned}
		$$
		Vậy $\left(u_n\right)$ là dãy số giảm.
	}
\end{bt}
\begin{bt}[VD]%[1K2K5-3]
	Xét tính tăng giảm của dãy số $\left(u_n\right)$ cho bởi
	$$
	\left(u_n\right)\colon\heva{
		&u_1=1 ; u_2=2 \\
		&u_{n+1}=\sqrt{u_n}+\sqrt{u_{n-1}} \forall n \geq 2
	}
	$$
	\dapso{dãy số tăng}
	\loigiai{
		Ta chứng minh dãy $\left(u_n\right)$ là dãy tăng bằng phương pháp quy nạp.\\
		Dễ thấy $u_1<u_2<u_3$.\\
		Giả sử $u_{k-1}<u_k ~\forall k \geq 2$, ta chứng minh $u_{k+1}>u_k$.\\
		Thật vậy ta có $u_{k+1}=\sqrt{u_k}+\sqrt{u_{k-1}}>\sqrt{u_{k-1}}+\sqrt{u_{n-2}}=u_k$.\\ Vậy $\left(u_n\right)$ là dãy tăng.
	}
	
\end{bt}
\begin{bt}[VD]%[1K2K5-3]
	Cho dãy số $\left(u_n\right)$ biết $u_n=\dfrac{b \cdot2 n^2+1}{n^2+3}$ và $b \in \mathbb{R}$. Hãy xác định $b$ để
	\begin{listEX}[2]
		\item $\left(u_n\right)$ là dãy số giảm.
		\item $\left(u_n\right)$ là dãy số tăng.
	\end{listEX}
	\dapso{$b<\dfrac{1}{6}$ dãy số giảm; $b>\dfrac{1}{6}$ dãy số tăng}
	\loigiai{
		Ta có
		$$
		u_n=2 b+\dfrac{1-6 b}{n^2+3}
		$$
		Xét hiệu $$u_{n+1}-u_n=\dfrac{1-6 b}{(n+1)^2+3}-\dfrac{1-6 b}{n^2+3}=(1-6 b) \cdot\left(\dfrac{1}{(n+1)^2+3}-\dfrac{1}{n^2+3}\right)=A_n.$$
		\begin{listEX}
			\item Để $\left(u_n\right)$ là dãy sỗ giảm thì $A_n<0, \forall n \in \mathbb{N}^\ast$.
			$$
			A_n<0 \Leftrightarrow 1-6 b>0 \Leftrightarrow b<\dfrac{1}{6}
			$$
			\item Để $\left(u_n\right)$ là dãy số tăng thì $A_n>0, \forall n \in \mathbb{N}^\ast$.
			$$
			A_n>0 \Leftrightarrow 1-6 b<0 \Leftrightarrow b>\dfrac{1}{6}.
			$$
		\end{listEX}
		
	}
\end{bt}
\begin{bt}[VDC]%[1K2G5-3]
	Xét tính tăng giảm của dãy số $\left(u_n\right)$ với $u_n=\sin n+\cos n$.
	\dapso{dãy khōng tăng, không giảm}
	\loigiai{
		Ta có: $u_n=\sin n+\cos n=\sqrt{2} \sin \left(n+\dfrac{\pi}{4}\right)$.
		Xét hiệu
		$$
		\begin{aligned}
			u_{n+1}-u_n&=\sqrt{2} \sin \left(n+1+\dfrac{\pi}{4}\right)-\sqrt{2} \sin \left(n+\dfrac{\pi}{4}\right) \\
			&=2 \sqrt{2} \cdot \cos \left(2 n+\dfrac{1}{2}+\dfrac{\pi}{4}\right) \cdot \sin \dfrac{1}{2}=A_n . \\
		\end{aligned}
		$$
		Với  $n=1, A_1>0$. Với  $n=100, A_{100}<100$ . \\
		Vậy $\left(u_n\right)$ là dãy khōng tăng, không giảm.
	}
\end{bt}
\subsubsection{Bài tập trắc nghiệm}
\Opensolutionfile{ans}[ans/ans-1K2-1-Dang3]
%Câu 1
\begin{ex}%[1K2Y5-3]
	Cho các dãy số sau. Dãy số nào là dãy số tăng?
	\choice
	{$1;1;1;1;1;1;\ldots $}
	{$1;\dfrac{1}{2};\dfrac{1}{4};\dfrac{1}{8};\dfrac{1}{16};\ldots $}
	{$1;-\dfrac{1}{2};\dfrac{1}{4};-\dfrac{1}{8};\dfrac{1}{16};\ldots $}
	{ \True $1;3;5;7;9;\ldots $}
	\loigiai {
		Xét đáp án $1;1;1;1;1;1;\ldots$ đây là dãy hằng nên không tăng không giảm.\\
		Xét đáp án $1;-\dfrac{1}{2};\dfrac{1}{4};-\dfrac{1}{8};\dfrac{1}{16};\ldots \Rightarrow {{u}_{1}}>{{u}_{2}}<{{u}_{3}}\Rightarrow $ loại.\\
		Xét đáp án $1;3;5;7;9;\ldots \Rightarrow {{u}_{n}}<{{u}_{n+1}},\,\,n\in {{\mathbb{N}}^{*}}\Rightarrow $ chọn.\\
		Xét đáp án $1;\dfrac{1}{2};\dfrac{1}{4};\dfrac{1}{8};\dfrac{1}{16};\ldots \Rightarrow {{u}_{1}}>{{u}_{2}}>{{u}_{3}}\ldots >{{u}_{n}}>\ldots \Rightarrow $ loại.}
	
\end{ex}
%%%%%%%%%%
%Câu 2
\begin{ex}%[1K2Y5-3]
	Với giá trị nào của $a$ thì dãy số $\left(u_n\right)$ với $u_n=\dfrac{a n-1}{n+2}, \forall n \geq 1$ là dãy số tăng?
	\choice
	{$a>2$}
	{$a<-2$}
	{\True $a>-\dfrac{1}{2}$}
	{$a<-\dfrac{1}{2}$}
	\loigiai{
		Ta có $u_n=a-\dfrac{1+2 a}{n+2}$.\\
		$u_{n+1}-u_n=(1+2 a)\left(\dfrac{1}{n+2}-\dfrac{1}{n+3}\right)$.\\
		Suy ra dãy số đã cho tăng khi $a>-\dfrac{1}{2}$.
	}
\end{ex}
%%%%%%%%%%%%
%Câu 3
\begin{ex}%[1K2Y5-3]
	Trong các dãy $\left(u_n\right)$ sau đây dãy nào là dãy số giảm ?
	\choice
	{$u_n=(-1)^n$}
	{$u_n=2^n$}
	{$u_n=3 n+1$}
	{\True $u_n=\dfrac{1}{3^n}$}
	\loigiai{
		Xét dãy số $\left(u_n\right)$ có $u_n=\dfrac{1}{3^n}$, ta thấy $u_n>0, \forall n \in \mathbb{N}^\ast$ và $\dfrac{u_{n+1}}{u_n}=\dfrac{\dfrac{1}{3^{n+1}}}{\dfrac{1}{3^n}}=\dfrac{1}{3}<1$ nên dãy số $\left(u_n\right)$ này là dãy số giảm.
	}
\end{ex}
%%%%%%%%%%
%Câu 4
\begin{ex}%[1K2B5-3]
	Trong các dãy số $({{u}_{n}} )$ cho bởi số hạng tổng quát ${{u}_{n}}$ sau, dãy số nào là dãy số tăng?
	\choice
	{${{u}_{n}}=\dfrac{1}{n}$}
	{${{u}_{n}}=\dfrac{1}{{{2}^{n}}}$}
	{${{u}_{n}}=\dfrac{n+5}{3n+1}$}
	{ \True ${{u}_{n}}=\dfrac{2n-1}{n+1}$ }
	\loigiai {
		Vì ${{2}^{n}};\,n$ là các dãy dương và tăng nên $\dfrac{1}{{{2}^{n}}};\,\,\dfrac{1}{n}$ là các dãy giảm, do đó loại các đáp án ${{u}_{n}}=\dfrac{1}{{{2}^{n}}}$ và ${{u}_{n}}=\dfrac{1}{n}$.\\
		Xét đáp án ${{u}_{n}}=\dfrac{n+5}{3n+1}\Rightarrow \heva{
			& {{u}_{1}}=\dfrac{3}{2} \\
			& {{u}_{2}}=\dfrac{7}{6} \\
		}\Rightarrow {{u}_{1}}>{{u}_{2}}\Rightarrow $ loại.\\
		Xét đáp án ${{u}_{n}}=\dfrac{2n-1}{n+1}=2-\dfrac{3}{n+1}\Rightarrow  {{u}_{n+1}}-{{u}_{n}}=3\left(\dfrac{1}{n+1}-\dfrac{1}{n+2}\right)>0\Rightarrow$ nhận.}
	
\end{ex}
%%%%%%%%%%
%%%%%%%%%%
%Câu 5
\begin{ex}%[1K2B5-3]
	Trong các dãy số $({{u}_{n}} )$ cho bởi số hạng tổng quát ${{u}_{n}}$ sau, dãy số nào là dãy số giảm?
	\choice
	{${{u}_{n}}={{n}^2}$}
	{${{u}_{n}}=\dfrac{3n-1}{n+1}$}
	{${{u}_{n}}=\sqrt{n+2}$}
	{ \True ${{u}_{n}}=\dfrac{1}{{{2}^{n}}}$}
	\loigiai {
		Vì ${{2}^{n}}$ là dãy dương và tăng nên $\dfrac{1}{{{2}^{n}}}$ là dãy giảm. \\
		Xét ${{u}_{n}}=\dfrac{3n-1}{n+1}\Rightarrow \heva{
			& {{u}_{1}}=1 \\
			& {{u}_{2}}=\dfrac{5}{3} \\
		}\Rightarrow {{u}_{1}}<{{u}_{2}},$ loại.\\
		Hoặc
		${{u}_{n+1}}-{{u}_{n}}=\dfrac{3n+2}{n+2}-\dfrac{3n-1}{n+1}=\dfrac{4}{(n+1 )(n+2 )}>0$ nên $({{u}_{n}} )$ là dãy tăng.\\
		Xét ${{u}_{n}}={{n}^2}\Rightarrow {{u}_{n+1}}-{{u}_{n}}={{(n+1 )}^2}-{{n}^2}=2n+1>0,$ loại.\\
		Xét ${{u}_{n}}=\sqrt{n+2}\Rightarrow {{u}_{n+1}}-{{u}_{n}}=\sqrt{n+3}-\sqrt{n+2}=\dfrac{1}{\sqrt{n+3}+\sqrt{n+2}}>0,$ loại.}
	
\end{ex}

%Câu 6

\begin{ex}%[1K2B5-3]
	Trong các dãy số $(u_{n})$ sau, hãy chọn dãy số tăng.
	\choice
	{\True $u_{n}=(-1)^{2n}(5^{n}+1)$, $n\in \mathbb N^*$}
	{$u_{n}=\dfrac{n}{n^{2}+1}$, $n\in \mathbb N^*$}
	{$u_{n}=(-1)^{n+1}\sin \dfrac{\pi}{n}$, $n\in \mathbb N^*$}
	{$u_{n}=\dfrac{1}{\sqrt{n+1}+n}$, $n\in \mathbb N^*$}
	\loigiai
	{
		Xét dãy số $(u_n)$ với $u_{n}=(-1)^{2n}(5^{n}+1)$, ta có
		\[u_{n+1}-u_n = (-1)^{2n+2}(5^{n+1}+1)-(-1)^{2n}(5^{n}+1) = 5^{n+1}+1-5^n-1 = 4\cdot 5^n>0, \forall n\in\mathbb{N}^\ast.\]
		Vậy dãy trên là dãy số tăng.\\
		Xét các dãy số còn lại
		\begin{itemize}
			\item Với $u_{n}=(-1)^{n+1}\sin \dfrac{\pi}{n}$ ta có $u_1=0$, $u_2=-1$ hay $u_1>u_2$. Vậy dãy số này không là dãy số tăng.
			\item Với $u_{n}=\dfrac{1}{\sqrt{n+1}+n}$ ta có $u_1=\sqrt{2}-1$, $u_2=2-\sqrt{3}$ hay $u_1>u_2$. Vậy dãy số này không là dãy số tăng.
			\item Với $u_{n}=\dfrac{n}{n^{2}+1}$ ta có $u_1=\dfrac{1}{2}$, $u_2=\dfrac{2}{5}$ hay $u_1>u_2$. Vậy dãy số này không là dãy số tăng.
		\end{itemize}
	}
\end{ex}
%%%%%%%%%%
%Câu 7
\begin{ex}%[1K2K5-3]
	Trong các dãy số $({{u}_{n}} )$ cho bởi số hạng tổng quát ${{u}_{n}}$ sau, dãy số nào là dãy số giảm?
	\choice
	{${{u}_{n}}=\dfrac{{{n}^2}+1}{n}$}
	{${{u}_{n}}={{(-1 )}^{n}}\cdot ({{2}^{n}}+1 )$}
	{\True ${{u}_{n}}=\sqrt{n}-\sqrt{n-1}\,$}
	{${{u}_{n}}=\sin n$}
	\loigiai {
		Xét ${{u}_{n}}=\sin n\Rightarrow  {{u}_{n+1}}-{{u}_{n}}=2\cos \left(n+\dfrac{1}{2} \right)\sin \dfrac{1}{2}$ có thể dương hoặc âm phụ thuộc $n$ nên đáp án sai. Hoặc dễ thấy $\sin n$ có dấu thay đổi trên ${{\mathbb{N}}^{*}}$ nên dãy $\sin n$ không tăng, không giảm.\\
		Xét ${{u}_{n}}=\dfrac{{{n}^2}+1}{n}=n+\dfrac{1}{n}\Rightarrow  {{u}_{n+1}}-{{u}_{n}}=1+\dfrac{1}{n+1}-\dfrac{1}{n}=\dfrac{{{n}^2}+n-1}{n(n+1 )}>0$ nên dãy đã cho tăng nên đáp án sai.\\
		Xét ${{u}_{n}}=\sqrt{n}-\sqrt{n-1}=\dfrac{1}{\sqrt{n}+\sqrt{n+1}},$ dãy $\sqrt{n}+\sqrt{n-1}>0$ là dãy tăng nên suy ra ${{u}_{n}}$ giảm. \\
		Xét ${{u}_{n}}={{(-1 )}^{n}}({{2}^{n}}+1 )$ là dãy thay dấu nên không tăng không giảm, nên đáp án đúng.\\
		Cách trắc nghiệm\\
		Xét ${{u}_{n}}=\sin n$ có dấu thay đổi trên ${{\mathbb{N}}^{*}}$ nên dãy này không tăng không giảm.\\
		Xét ${{u}_{n}}=\dfrac{{{n}^2}+1}{n}$, ta có $\heva{
			& n=1\to {{u}_{1}}=2 \\
			& n=2\to {{u}_{2}}=\dfrac{5}{2} \\
		}\Rightarrow {{u}_{1}}<{{u}_{2}}\Rightarrow {{u}_{n}}=\dfrac{{{n}^2}+1}{n}$ không giảm.\\
		Xét ${{u}_{n}}=\sqrt{n}-\sqrt{n-1}$, ta có $\heva{
			& n=1\to {{u}_{1}}=1 \\
			& n=2\to {{u}_{2}}=\sqrt{2}-1 \\
		}\Rightarrow {{u}_{1}}>{{u}_{2}}$ nên dự đoán dãy này giảm.\\
		Xét ${{u}_{n}}={{(-1 )}^{n}}({{2}^{n}}+1 )$ là dãy thay dấu nên không tăng không giảm.\\
		Cách CASIO.\\
		Các dãy $\sin n;\,\,{{(-1 )}^{n}}({{2}^{n}}+1 )$ có dấu thay đổi trên ${{\mathbb{N}}^{*}}$ nên các dãy này không tăng không giảm nên loại các đáp án này.\\
		Xét hai đáp án còn lại, ta chỉ cần kiểm tra một đáp án bằng chức năng $TABLE$.\\
		Chẳng hạn kiểm tra đáp án ${{u}_{n}}=\dfrac{{{n}^2}+1}{n}$, ta vào chức năng $TABLE$ nhập $F(X )=\dfrac{X^2+1}{X}$ với thiết lập $\text{Start}=1,\text{ End}=10,\text{ Step}=1$.\\
		Nếu thấy cột $F(X )$ các giá trị tăng thì loại ${{u}_{n}}=\dfrac{{{n}^2}+1}{n}$ nếu ngược lại nếu thấy cột $F(X )$ các giá trị giảm dần thị chọn ${{u}_{n}}=\dfrac{{{n}^2}+1}{n}$.}
	
\end{ex}
%%%%%%%%%%
%Câu 8
\begin{ex}%[1K2K5-3]
	Mệnh đề nào sau đây đúng?
	\choice
	{Dãy số ${{u}_{n}}=\dfrac{1}{n}-2$ là dãy tăng}
	{\True Dãy số ${{u}_{n}}=2n+\cos \dfrac{1}{n}$ là dãy tăng}
	{Dãu số ${{u}_{n}}=\dfrac{n-1}{n+1}$ là dãy giảm}
	{Dãy số ${{u}_{n}}={{(-1 )}^{n}}({{2}^{n}}+1 )$ là dãy giảm}
	\loigiai {
		Xét đáp án ${{u}_{n}}=\dfrac{1}{n}-2\Rightarrow {{u}_{n+1}}-{{u}_{n}}=\dfrac{1}{n+1}-\dfrac{1}{n}<0\Rightarrow $loại.\\
		Xét đáp án ${{u}_{n}}={{(-1 )}^{n}}({{2}^{n}}+1 )$ là dãy có dấu thay đổi nên không giảm nên loại.\\
		Xét đáp án ${{u}_{n}}=\dfrac{n-1}{n+1}=1-\dfrac{2}{n+1}\Rightarrow {{u}_{n+1}}-{{u}_{n}}=2\left(\dfrac{1}{n+1}-\dfrac{1}{n+2}\right)>0\Rightarrow $ loại.\\
		Xét đáp án ${{u}_{n}}=2n+\cos \dfrac{1}{n}\Rightarrow {{u}_{n+1}}-{{u}_{n}}=\left(2-\cos \dfrac{1}{n+1}\right)+\cos \dfrac{1}{n+2}>0$ chọn.}
	
\end{ex}
%%%%%%%%%%
%Câu 9
\begin{ex}%[1K2K5-3]
	Mệnh đề nào sau đây {\bf sai}?
	\haicot
	{Dãy số ${{u}_{n}}=\dfrac{1-n}{\sqrt{n}}$ là dãy giảm}
	{Dãy số ${{u}_{n}}=n+\sin ^2n$ là dãy tăng}
	{\True Dãy số ${{u}_{n}}={{\left(1+\dfrac{1}{n}\right)}^{n}}$ là dãy giảm}
	{Dãy số ${{u}_{n}}=2{{n}^2}-5$ là dãy tăng}
	\loigiai {
		Xét đáp án \\ ${{u}_{n}}=\dfrac{1-n}{\sqrt{n}}=\dfrac{1}{\sqrt{n}}-\sqrt{n}\Rightarrow {{u}_{n+1}}-{{u}_{n}}=\dfrac{1}{\sqrt{n+1}}-\dfrac{1}{\sqrt{n}}+\sqrt{n}-\sqrt{n+1}<0$ nên dãy $({{u}_{n}} )$ là dãy giảm nên đúng.\\
		Xét đáp án ${{u}_{n}}=2{{n}^2}-5$ là dãy tăng vì ${{n}^2}$ là dãy tăng nên đúng. \\
		Hoặc
		${{u}_{n+1}}-{{u}_{n}}=2(2n+1 )>0$ nên $({{u}_{n}} )$ là dãy tăng.\\
		Xét đáp án ${{u}_{n}}={{\left(1+\dfrac{1}{n}\right)}^{n}}={{\left(\dfrac{n+1}{n} \right)}^{n}}>0\Rightarrow\dfrac{{{u}_{n+1}}}{{{u}_{n}}}=\dfrac{n+2}{n+1}\cdot {{\left(\dfrac{n+2}{n}\right)}^{n}}>1\Rightarrow ({{u}_{n}} )$ là dãy tăng nên sai.\\
		Xét đáp án ${{u}_{n}}=n+\sin ^2n\Rightarrow {{u}_{n+1}}-{{u}_{n}}=(1-\sin ^2(n+1 ) )+\sin ^2n>0$.}
	
\end{ex}
%%%%%%%%%%
%Câu 10%VDC 
\begin{ex}%[Nguyễn Long]%[1K2G5-3]
	Cho dãy $(u_n)\colon\heva{&u_1=1\\&u_{n+1}=\dfrac{n}{2(n+1)}u_n+\dfrac{3(n+2)}{2(n+1)}},n \in \mathbb{N^*}$. Nhận xét nào sau đây đúng
	\choice
	{\True Dãy số $(u_n)$ là dãy số tăng}
	{Dãy số $(u_n)$ là dãy số giảm}
	{Dãy số $(u_n)$ là dãy số không tăng, không giảm}
	{Tất cả các đáp án còn lại đều sai}
	\loigiai{ Ta chứng minh quy nạp $u_n<3, \forall n \in N^*$.\\
		Giả sử mđ đúng với $\mathrm{n}=\mathrm{k}$ khi đó có:
		$$
		u_{k+1}=\dfrac{k}{2(k+1)} u_k+\dfrac{3(k+2)}{2(k+1)}<\dfrac{3 k}{2(k+2)}+\dfrac{3(k+2)}{2(k+1)}=3 .
		$$
		Vậy mệnh đề đúng với $\mathrm{n}=\mathrm{k}+1$.
		Từ đó ta có $$u_{n+1}-u_n=\dfrac{\left(3-u_n\right)(n+2)}{n+1}>0.$$
		Vậy dãy $\left(u_n\right)$ tăng }
\end{ex}
\Closesolutionfile{ans}
\begin{indapan}{10}
	{ans/ans-1K2-1-Dang3}
\end{indapan}

\begin{dang}{Xét tính bị chặn của dãy số}
	\begin{itemize}
		\item Để chứng minh dãy số $(u_n)$ bị chặn trên bởi $M$, ta chứng minh $u_n\le M$, $\forall n\in\mathbb{N}^\ast$.
		\item Để chứng minh dãy số $(u_n)$ bị chặn dưới bởi $m$, ta chứng minh $u_n\ge m$, $\forall n\in\mathbb{N}^\ast$.
		\item Để chứng minh dãy số bị chặn ta chứng minh nó bị chặn trên và bị chặn dưới.
		\begin{itemize}
			\item Nếu dãy số $(u_n)$ tăng thì bị chặn dưới bởi $u_1$.
			\item Nếu dãy số $(u_n)$ giảm thì bị chặn trên bởi $u_1$.
		\end{itemize}
	\end{itemize}
\end{dang}
\subsubsection{Ví dụ minh hoạ}

%ví dụ 1
\begin{vd}[NB]%[1K2Y5-4]%[Trương Đăng Khoa]
	Chứng minh rằng dãy số $(u_n)$ với $u_n=\dfrac{3n}{n^2+9}$ bị chặn trên bởi $\dfrac{1}{2}$.
	\dapso{dãy số đã cho bị chặn trên bởi $\dfrac{1}{2}$}
	\loigiai{
		Với mọi $n\ge 1$, ta có $\dfrac{3n}{n^2+9}\le\dfrac{1}{2}\Leftrightarrow n^2+9\le 6n\Leftrightarrow(n-3)^2\le 0$ (đúng).\\
		Vậy dãy số đã cho bị chặn trên bởi $\dfrac{1}{2}$.
	}
\end{vd}

%ví dụ 2
\begin{vd}[NB]%[1K2Y5-4]%[Trương Đăng Khoa]
	Chứng minh rằng dãy số $(u_n)$ xác đinh bởi $u_n=\dfrac{8n+3}{3n+5}$ là một dãy số bị chặn.
	\dapso{dãy số bị chặn}
	\loigiai{
		Ta có $u_n>0$, $\forall n\ge 1$. Suy ra dãy số bị chặn dưới.\\
		Mặt khác $u_n=\dfrac{8n+3}{3n+5}<\dfrac{8n+3}{3n}=\dfrac{8}{3}+\dfrac{1}{n}<\dfrac{8}{3}+1=\dfrac{11}{3}$. Do đó dãy số bị chặn trên bởi $\dfrac{11}{3}$.\\
		Vậy dãy số đã cho bị chặn.
	}
\end{vd}
%ví dụ 4
\begin{vd}[TH]%[1K2B5-4]%[Trương Đăng Khoa]
	Xét tính bị chặn của dãy số $\left(u_n\right)$ với $u_n=\dfrac{3n+1}{n+3}$.
	\dapso{dãy số bị chặn}
	\loigiai{
		Với $n\in \mathbb{N}^\ast$ ta có $u_n=\dfrac{3n+1}{n+3}>0$.\\
		Nên dãy $\left(u_n\right)$ bị chặn dưới bởi $0$.\\
		Mặt khác $u_n=\dfrac{3n+1}{n+3}=\dfrac{3n+9-8}{n+3}=3-\dfrac{8}{n+3}<3$, $\forall n\in\mathbb{N}^\ast$.\\
		Nên dãy $\left(u_n\right)$ bị chặn trên bởi $3$.\\
		Vậy dãy số $\left(u_n\right)$ bị chặn.
	}
\end{vd}
%ví dụ 3
\begin{vd}[VD]%[1K2K5-4]%[Trương Đăng Khoa]
	Cho dãy số $(u_n)$ xác định bởi $u_1=1$ và $u_{n+1}=\dfrac{u_n+2}{u_n+1}$, $\forall n\ge 1$. Chứng minh rằng dãy $(u_n)$ bị chặn trên bởi sô $\dfrac{3}{2}$ và bị chặn dưới bởi số $1$.	
	\loigiai{
		Ta chứng minh $1\le u_n\le\dfrac{3}{2},\forall n\ge 1$ bằng phương pháp quy nạp.
		\begin{itemize}
			\item Với $n=1$ ta có $1\le u_1\le\dfrac{3}{2}$.
			\item Giả sử $1\le u_n\le\dfrac{3}{2}$ với mọi $n=k\ge 1$, tức là $1\le u_k\le\dfrac{3}{2}$. Ta cần chứng minh $1\le u_{k+1}\le\dfrac{3}{2}$.
		\end{itemize}
		Thật vậy 
		$u_{k+1}=1+\dfrac{1}{u_k+1}$.\\
		Vì $u_k+1>0$ nên $u_{k+1}=1+\dfrac{1}{u_k+1}>1$.\\
		Vì $u_k+1\ge 2$ nên $u_{k+1}=1+\dfrac{1}{u_k+1}\le 1+\dfrac{1}{2}=\dfrac{3}{2}$.\\
		Vậy $1\le u_n\le\dfrac{3}{2}$, $\forall n\ge 1$ hay dãy $(u_n)$ bị chặn trên bởi số $\dfrac{3}{2}$ và bị chặn dưới bởi số $1$.
	}
\end{vd}

%ví dụ 5
\begin{vd}[VD]%[1K2K5-4]%[Trương Đăng Khoa]
	Xét tính bị chặn của dãy số $\left(u_n\right)$ với $u_n=\sin n+ \cos n$.
	\dapso{dãy số bị chặn}
	\loigiai{
		Ta có $\begin{aligned}[t]
			&\ \sin n+\cos n \\
			=&\ \sqrt{2}\left(\dfrac{1}{\sqrt{2}}\sin n+\dfrac{1}{\sqrt{2}}\cos n\right)\\
			=&\ \sqrt{2}\left(\sin n\cdot\cos \dfrac{\pi}{4}+\cos n\cdot\sin \dfrac{\pi}{4}\right)\\
			=&\ \sqrt{2}\sin \left(n+\dfrac{\pi}{4}\right).
		\end{aligned}$\\
		Vì $\begin{aligned}[t]
			&\ -1\leq \sqrt{2}\sin \left(n+\dfrac{\pi}{4}\right) \leq 1\\
			\Rightarrow&\ -\sqrt{2}\leq \sqrt{2} \sin \left(n+\dfrac{\pi}{4}\right)\leq \sqrt{2}\\
			\Rightarrow&\ -\sqrt{2}\leq \sin n+\cos n \leq \sqrt{2},\ \forall n\in\mathbb{N}^\ast\\
			\Rightarrow&\ -\sqrt{2}\leq u_n \leq \sqrt{2},\ \forall n\in\mathbb{N}^\ast.
		\end{aligned}$\\
		Vậy dãy số $\left(u_n\right)$ là dãy số bị chặn.}
\end{vd}
\subsubsection{Bài tập tự luận}
 
%Bài 1 
\begin{bt}[TH]%[1K2B5-4]%[Trương Đăng Khoa]
	Xét tính bị chặn của các dãy số sau 
		\begin{listEX}[3]
			\item $u_n=\dfrac{1}{2n^2-1}$.
			\item 
			$u_n=3\cdot\cos\dfrac{n x}{3}$. 
			\item  $u_n=2n^3+1$.
			\item  $u_n=\dfrac{n^2+2n}{n^2+n+1}$.
			\item  $u_n=n+\dfrac{1}{n}$.
		\end{listEX}
	\loigiai{
		\begin{enumerate}
			\item  $u_n=\dfrac{1}{2n^2-1}$.\\
			Ta có $2n^2-1\ge 1\Rightarrow u_n=\dfrac{1}{2n^2-1}\le 1$, $\forall n\ge 1$.\\
			Vậy dãy số bị chặn trên bởi $1$.\\
			\item $u_n=3\cdot\cos\dfrac{n x}{3}$ có $-1\le\cos\dfrac{n x}{3}\le 1\Rightarrow-3\le 3\cdot\cos\dfrac{n x}{3}\le 3$.\\
			Vậy dãy số bị chặn dưới bởi $-3$ và chặn trên bởi $3$.
			\item  $u_n=2n^3+1$ có $2n^3+1\ge 3$, $\forall n\ge 1$.\\
			Vậy dãy số bị chặn dưới bởi $3$.
			\item $u_n=\dfrac{n^2+2n}{n^2+n+1}$ có $u_n=\dfrac{n^2+2n}{n^2+n+1}=1+\dfrac{n-1}{n^2+n+1}\ge 1$, $\forall n\ge 1$.\\
			Vậy dãy số bị chặn dưới bởi $1$.
			\item  $u_n=n+\dfrac{1}{n}$ có $u_n=n+\dfrac{1}{n}\ge 2\sqrt{n\cdot\dfrac{1}{n}}=2$, $\forall n>0$.\\
			Vậy dãy số bị chặn bởi $2$.
		\end{enumerate}
	}
\end{bt}
%Bài 2
\begin{bt}[VD]%[1K2K5-4]%[Trương Đăng Khoa]
	Xét tính bị chặn của dãy số $(u_n)$ với:
	\begin{listEX}[3]	 
		\item $u_{n}=\dfrac{4}{n}-5$.
		\item $u_{n}=\dfrac{n+4}{n+2}$.
		\item $u_{n}=\dfrac{5}{n^2+1}+\dfrac{n+2}{n+1}+\cos n$.
	\end{listEX}
	\loigiai{
		\begin{enumerate} 
			\item $u_{n}=\dfrac{4}{n}-5$.\\
			Ta có $u_n=\dfrac{4}{n}-5 \le \dfrac{4}{1}-5=-1$, $\forall n \in \mathbb{N}^{*}$ suy ra dãy $(u_n)$ bị chặn trên bởi $-1$.\\
			Mặt khác $u_n=\dfrac{4}{n}-5 \ge -5 \,\, \forall n \in \mathbb{N}^{*}$ suy ra dãy $(u_n)$ bị chặn dưới bởi $-5$.\\
			Vậy dãy $(u_n)$ bị chặn.	
			\item $u_{n}=\dfrac{n+4}{n+2}$.\\
			Ta có $u_n=	\dfrac{n+4}{n+2}=1+\dfrac{2}{n+2}> 1$, $\forall n \in \mathbb{N}^{*}$ suy ra dãy $(u_n)$ bị chặn dưới bởi $1$.\\
			Mặt khác $u_n=\dfrac{n+4}{n+2}=1+\dfrac{2}{n+2} \le 1+\dfrac{2}{1+2}=\dfrac{3}{5}$, $\forall n \in \mathbb{N}^{*}$ suy ra dãy $(u_n)$ bị chặn trên bởi $\dfrac{3}{5}$.\\
			Vậy dãy $(u_n)$ bị chặn.
			\item $u_{n}=\dfrac{5}{n^2+1}+\dfrac{n+2}{n+1}+\cos n$.\\
			Ta có $u_n=	\dfrac{5}{n^2+1}+\dfrac{n+2}{n+1}+\cos n=\dfrac{5}{n^2+1}+1+\dfrac{1}{n+1}+\cos n<5$, $\forall n \in \mathbb{N}^{*}$.\\
			Suy ra dãy $(u_n)$ bị chặn trên bởi $5$.\\
			Mặt khác $u_n=	\dfrac{5}{n^2+1}+\dfrac{n+2}{n+1}+\cos n=\dfrac{5}{n^2+1}+1+\dfrac{1}{n+1}+\cos n>0$, $\forall n \in \mathbb{N}^{*}$.\\
			Suy ra dãy $(u_n)$ bị chặn trên bởi $0$.\\
			Vậy dãy $(u_n)$ bị chặn.
		\end{enumerate}			
	}		
\end{bt}
%Bài 3
\begin{bt}[VDC]%[1K2G5-4]%[Trương Đăng Khoa]
	Xét tính bị chặn của dãy số $u_n=\left(1+\dfrac{1}{n}\right)^n$, $n\in N^\ast$.
	\loigiai{
		Ta có $u_n=\left(1+\dfrac{1}{n}\right)^n>0$, $\forall n\in N^\ast$ nên $(u_n)$ bị chặn dưới $(1)$.\\
		Lại có $\begin{aligned}[t]
			u_n&\ =\left(1+\dfrac{1}{n}\right)^n=\displaystyle\sum\limits_{k=0}^n C_n^k\left(\dfrac{1}{n}\right)^k\\
			&\ =\displaystyle\sum\limits_{k=0}^n\left[\dfrac{n!}{k!\cdot(n-k)!\cdot n^k}\right]\\
			&\ =\displaystyle\sum\limits_{k=0}^n\left[\dfrac{1}{k!}\cdot\dfrac{(n-k+1)}{n}\cdot\dfrac{(n-k+2)}{n}\ldots\dfrac{(n-k+k)}{n}\right]\le\displaystyle\sum\limits_{k=0}^n\dfrac{1}{k!},\, n\in \mathbb{N}^\ast
		\end{aligned}$\\
		Mà $\begin{aligned}[t]
			\displaystyle\sum\limits_{k=0}^n\dfrac{1}{k!}&\ \le 1+1+\dfrac{1}{1\cdot2}+\dfrac{1}{2\cdot3}+\dfrac{1}{3\cdot4}+\ldots+\dfrac{1}{(n-1)\cdot n}\\
			&\ =2+\left(1-\dfrac{1}{2}\right)+\left(\dfrac{1}{2}-\dfrac{1}{3}\right)+\ldots+\left(\dfrac{1}{n-1}-\dfrac{1}{n}\right)\\
			&\ 
			=3-\dfrac{1}{n}<3,\, \forall n\in \mathbb{N}^\ast.
		\end{aligned}$\\
		Suy ra $u_n<3$, $\forall n\in \mathbb{N}^\ast$ nên dãy số $(u_n)$ bị chặn trên $(2)$.\\
		Từ $(1)$ và $(2)$  suy ra dãy số $(u_n)$ bị chặn.}
\end{bt}
%Bài 4
\begin{bt}[VD]%[1K2K5-4]%[Trương Đăng Khoa]
	Cho dãy số $(u_n)$ xác định bởi $u_1=0$ và $u_{n+1}=\dfrac{1}{2}u_n+4$, $ \forall n\geq 1$.
	\begin{enumerate}
		\item Chứng minh dãy $(u_n)$ bị chặn trên bởi số $8$.
		\item Chứng minh dãy $(u_n)$ tăng, từ đó suy ra dãy $(u_n)$ bị chặn.
	\end{enumerate}
	\loigiai{
		\begin{enumerate}
			\item Ta chứng minh $u_n\leq 8$ với mọi $n\geq 1$.
			\begin{itemize}
				\item Khi $n=1$, ta có $u_1=0 <8$.
				\item Giả sử $u_n\leq 8$ với $n=k\geq 1$, tức là $u_k\leq 8$.\\
				Ta cần chứng minh $u_{k+1}\leq 8$.\\
				Thật vậy, $u_{k+1}=\dfrac{1}{2}u_k+4\leq \dfrac{1}{2}\cdot 8+4\leq 8$.
			\end{itemize}
			Vậy $u_n\leq 8$ với mọi $n\geq 1$, hay $(u_n)$ bị chặn trên bởi $8$.
			\item Với mọi $n\geq 1$, ta có $u_{n+1}-u_n=4-\dfrac{1}{2}u_n$. Mà $u_n\leq 8$ nên $u_{n+1}-u_n\geq 0$.\\
			Suy ra $u_n$ là dãy số tăng. Do đó $(u_n)$ bị chặn dưới bởi $u_1=0$.\\
			Kết hợp với câu a, ta được dãy số $(u_n)$ bị chặn.
		\end{enumerate}
	}
\end{bt}
%Bài 5
\begin{bt}[VD]%[1K2K5-4]%[Trương Đăng Khoa]
	Trong các dãy số $(u_n)$ sau, dãy số nào bị chặn trên, bị chặn dưới và bị chặn?
	\begin{listEX}[3]
		\item $u_n=n^2+5$.
		\item $u_n=\dfrac{3n+1}{2n+5}$.
		\item $u_n=(-1)^n\cos \dfrac{\pi}{2n}$.
		\item $u_n=\dfrac{n^2+2n}{n^2+n+1}$.
		\item $u_n=\dfrac{n}{\sqrt{n^2+2n}+n}$.
	\end{listEX}
	\loigiai{
		\begin{enumerate}
			\item Dãy số bị chặn dưới bởi $6$, không bị chặn trên.
			\item Dãy $(u_n)$ bị chặn dưới bởi $0$. Vì $u_n<\dfrac{3n+1}{2n}=\dfrac{3}{2}+\dfrac{1}{2n}<\dfrac{3}{2}+1=\dfrac{5}{2}$ nên dãy số bị chặn trên bởi $\dfrac{5}{2}$. Vậy dãy số bị chặn.
			\item Ta có $|u_n|\leq 1$ nên dãy số bị chặn trên bởi 1, bị chặn dưới bởi $-1$.
			\item Dãy số bị chặn dưới bởi $0$. Vì $u_n<\dfrac{n^2+2n}{n^2}=1+\dfrac{2}{n}\leq 3$ nên dãy số bị chặn trên. Vậy dãy số bị chặn.
			\item Ta có $0<u_n\leq 1$ vậy dãy số bị chặn.
		\end{enumerate}
	}
\end{bt}

\subsubsection{Câu hỏi trắc nghiệm}

\Opensolutionfile{ans}[ans/ans-1K2-1-Dang4]
%Câu 1
\begin{ex}%[1K2K5-4]%[Trương Đăng Khoa]
	Cho dãy số $(u_n)$ xác định bởi $u_1=3$ và $u_{n+1}=\dfrac{u_n+1}{2}$, $\forall n\geq 1$. Mệnh đề nào sau đây là đúng?
	\choice
	{\True Dãy số bị chặn}
	{Dãy số bị chặn trên}
	{Dãy số bị chặn dưới}
	{Dãy số không bị chặn}
	\loigiai{Ta chứng minh $u_n>1, \forall n\geq 1$ bằng phương pháp quy nạp.\\
		Suy ra dãy số bị chặn dưới bởi $1$.\\
		Ta có
		$u_{n+1}-u_n=\dfrac{1-u_n}{2}<0$, $\forall n\geq 1$.\\
		Do đó dãy số này là dãy số giảm nên nó bị chặn trên bởi $u_1=3$.\\
		Vậy dãy số đã cho là dãy số bị chặn.		
	}
\end{ex}
%Câu 2
\begin{ex}%[1K2K5-4]%[Trương Đăng Khoa]
	Cho dãy số $(u_n)$ xác định bởi $u_1=\sqrt{2}$ và $u_{n+1}=\sqrt{2+u_n}$, $\forall n\geq 1$. Mệnh đề nào sau đây là đúng?
	\choice
	{Dãy số bị chặn trên}
	{Dãy số bị chặn dưới}
	{\True Dãy số bị chặn}
	{Dãy số không bị chặn}
	\loigiai{Vì $u_n\geq 0$, $\forall n\geq 1$ nên dãy số bị chặn dưới bởi $0$.\\
		Ta chứng minh $u_n\geq 2, \forall n\geq 1$. Suy ra dãy số bị chặn trên bởi $2$.\\
		Vậy dãy số đã cho là dãy số bị chặn.		
	}
\end{ex}
%Câu 3
\begin{ex}%[1K2K5-4]%[Trương Đăng Khoa]
	Xét tính bị chặn của  dãy số $(u_n)$ với $u_n=\dfrac{1}{1\cdot2}+\dfrac{1}{2\cdot3}+\ldots+\dfrac{1}{n\cdot(n+1)}$.
	\choice{Không bị chặn}{Bị chặn trên}{Bị chặn dưới}{\True Bị chặn}
	\loigiai{
		Ta có $u_n=1-\dfrac{1}{2}+\dfrac{1}{2}-\dfrac{1}{3}+\ldots+\dfrac{1}{n}-\dfrac{1}{n+1}=1-\dfrac{1}{n+1}$.\\
		Do đó $0\leq u_n \leq 1$, $\forall n\geq 1$.\\
		Vậy dãy số đã cho bị chặn.
	}
\end{ex}
%Câu 4
\begin{ex}%[1K2G5-4]%[Trương Đăng Khoa]
	Cho dãy số $(u_n)$ với $u_n=\dfrac{1}{1\cdot4}+\dfrac{1}{2\cdot5}+\ldots+\dfrac{1}{n\cdot(n+3)}$. Dãy số $\left(u_n\right)$ bị chặn dưới và chặn trên lần lượt bởi các số $m$ và $M$ nào dưới đây?
	\choice
	{$m=0$, $M=1$}
	{$m=1$, $M=\dfrac{1}{2}$}
	{$m=1$, $M=\dfrac{10}{19}$}
	{\True $m=0$, $M=\dfrac{11}{18}$}
	\loigiai{
		Rõ ràng $u_n>0$, $\forall n\in\mathbb{N}^\ast$ nên $(u_n)$ bị chặn dưới.\\
		Mặt khác $\dfrac{1}{k(k+3)}=\dfrac{1}{3}\left(\dfrac{1}{k}-\dfrac{1}{k+3}\right)$.\\
		Suy ra $\begin{aligned}[t]
			u_n&\  =\dfrac{1}{3}\bigg[\left(1-\dfrac{1}{4}\right)+\left(\dfrac{1}{2}-\dfrac{1}{5}\right)+\left(\dfrac{1}{3}-\dfrac{1}{6}\right)+\left(\dfrac{1}{4}-\dfrac{1}{7}\right)+\\
			&\  \ldots+\left(\dfrac{1}{n-3}-\dfrac{1}{n}\right)+\left(\dfrac{1}{n-2}-\dfrac{1}{n+1}\right)+\left(\dfrac{1}{n-1}-\dfrac{1}{n+2}\right)+\left(\dfrac{1}{n}-\dfrac{1}{n+3}\right)\bigg]\\
			&\ = \dfrac{1}{3}\left(1+\dfrac{1}{2}+\dfrac{1}{3}-\dfrac{1}{n+1}-\dfrac{1}{n+2}-\dfrac{1}{n+3}\right)<\dfrac{11}{18}, \, \forall n\in\mathbb{N}^\ast.
		\end{aligned}$\\
		Do đó $(u_n)$ bị chặn trên.\\
		Vậy $m=0$, $M=\dfrac{11}{18}$.
	}
\end{ex}
%Câu 5
\begin{ex}%[1K2G5-4]%[Trương Đăng Khoa]
	Cho dãy số $(u_n)$ biết $u_n=\dfrac{1\cdot 3\cdot 5\ldots(2n-1)}{2\cdot 4\cdot 6\cdot 2n}$. Dãy số $\left(u_n\right)$ bị chặn dưới và chặn trên lần lượt bởi các số $m$ và $M$. Tính giá trị biểu thức $m+M$?
	\choice{$\dfrac{1}{\sqrt{2}}$}{\True $\dfrac{1}{\sqrt{3}}$}{$\dfrac{1}{\sqrt{5}}$}{$\dfrac{1}{\sqrt{7}}$}
	\loigiai{
		Xét $ \dfrac{2 k-1}{2 k}<\dfrac{2 k-1}{\sqrt{4 k^2-1}}
		=\dfrac{\sqrt{(2 k-1)^2}}{\sqrt{(2 k-1)(2 k+1)}} =\dfrac{\sqrt{2 k-1}}{\sqrt{2 k+1}}$,  $\forall k \ge 1$.\\
		$\Rightarrow u_n<\dfrac{\sqrt{1}}{\sqrt{3}} \cdot\dfrac{\sqrt{3}}{\sqrt{5}} \cdot\dfrac{\sqrt{5}}{\sqrt{7}}\cdot \ldots \cdot \dfrac{\sqrt{2 n-1}}{\sqrt{2 n+1}}=\dfrac{1}{\sqrt{2 n+1}} \le\dfrac{1}{\sqrt{3}}$,  $\forall n \in\mathbb{N}^\ast$.\\
		$\Rightarrow 0<u_n<\dfrac{1}{\sqrt{3}}$, $\forall n \in\mathbb{N}^\ast$.\\
		Vậy $m+M=0+\dfrac{1}{\sqrt{3}}$.
	}
\end{ex}
%Câu 6
\begin{ex}%[1K2G5-4]%[Trương Đăng Khoa]
	Cho dãy số $(u_n)$, với $u_n=\dfrac{1}{2^2}+\dfrac{1}{3^2}+\ldots+\dfrac{1}{n^2}$, $\forall n=2;3;4;\ldots$. Khẳng định nào sau đây là đúng?
	\choice
	{\True Dãy số bị chặn}
	{Dãy số bị chặn trên}
	{Dãy số bị chặn dưới}
	{Dãy số không bị chặn}
	\loigiai{
		Ta có $u_n>0\Rightarrow(u_n)$ bị chặn dưới bởi $0$.\\
		Mặt khác $\dfrac{1}{k^2}<\dfrac{1}{(k-1) k}=\dfrac{1}{k-1}-\dfrac{1}{k}$, ($k\in\mathbb{N}^\ast$, $k\ge 2$) nên suy ra
		\begin{eqnarray*}
			u_n&<&\dfrac{1}{1 \cdot 2}+\dfrac{1}{2 \cdot 3}+\dfrac{1}{3 \cdot 4}+\cdots+\dfrac{1}{n(n+1)}\\
			&=&1-\dfrac{1}{2}+\dfrac{1}{2}-\dfrac{1}{3}+\dfrac{1}{2}-\dfrac{1}{4}+\cdots+\dfrac{1}{n}-\dfrac{1}{n+1}=1-\dfrac{1}{n+1}<1.
		\end{eqnarray*}
		Nên dãy $(u_n)$ bị chặn trên, do đó dãy $(u_n)$ bị chặn.
	}
\end{ex}
%Câu 7
\begin{ex}%[1K2G5-4]%[Trương Đăng Khoa]
	Cho dãy số $\left(u_n\right)$ và đặt $u_n= \displaystyle \sum_{k=1}^{n} a_k$ với $a_k=\dfrac{1}{4k^2-1}$. Mệnh đề nào sau đây là đúng?
	\choice{$0<u_n <1$}
	{$0\leq u_n\leq \dfrac{1}{2}$}
	{\True $0<u_n<\dfrac{1}{2}$}
	{$0\leq u_n\leq 1$}
	\loigiai{
		\begin{itemize}
			\item
			Ta có $a_k=\dfrac{1}{4k^2-1}=\dfrac{1}{(2k+1)(2k-1)}=\dfrac{1}{2}\cdot\dfrac{(2k+1)-(2k-1)}{(2k+1)(2k-1)}=\dfrac{1}{2}\cdot\left(\dfrac{1}{2k-1}-\dfrac{1}{2k+1}\right)$.\\
			\item Mặt khác $u_n=\displaystyle \sum_{k=1}^{n} a_k$.
			Do đó
			\begin{eqnarray*}
				&u_n&=\dfrac{1}{2}\cdot\left(\dfrac{1}{1}-\dfrac{1}{3}\right)+\dfrac{1}{2}\cdot\left(\dfrac{1}{3}-\dfrac{1}{5}\right)+\ldots + \dfrac{1}{2}\cdot \left(\dfrac{1}{2n-1}-\dfrac{1}{2n+1}\right)\\
				&&=\dfrac{1}{2}\left(\dfrac{1}{1}-\dfrac{1}{2n+1}\right)\\
				&&=\dfrac{1}{2}\cdot\dfrac{2n}{2n+1}=\dfrac{n}{2n+1}.
			\end{eqnarray*}
			\item 
			
			Với mọi $n \in \mathbb{N}^\ast$ thì $u_n>0$ nên dãy số $\left(u_n\right)$ bị chặn dưới.\\
			Ta lại có $u_n=\dfrac{1}{2}\cdot\left(1-\dfrac{1}{2n+1}\right)<\dfrac{1}{2}$.\\
			Vậy dãy số bị chặn.
		\end{itemize}
	}
\end{ex}
%Câu 8
\begin{ex}%[1K2G5-4]%[Trương Đăng Khoa]
	Cho dãy số $\left(u_n\right)$ và đặt $u_n= \displaystyle \sum_{k=1}^{n} a_k$ với $a_k=\dfrac{1}{k(k+4)}$.  Dãy số $\left(u_n\right)$ bị chặn dưới và chặn trên lần lượt bởi các số $m$ và $M$ nào sau đây?
	\choice
	{\True $m=0$, $M=\dfrac{25}{48}$}
	{$m=0$, $M=\dfrac{25}{12}$}
	{$m=1$, $M=\dfrac{1}{4}$}
	{$m=1$, $M=\dfrac{1}{2}$}
	\loigiai{
		Ta có $a_k=\dfrac{1}{k(k+4)}=\dfrac{1}{4}\cdot\dfrac{4}{k(k+4)}=\dfrac{1}{4}\cdot\dfrac{k+4-k}{k(k+4)}=\dfrac{1}{4}\cdot\left(\dfrac{1}{k}-\dfrac{1}{k+4}\right)$.\\
		Mặt khác $u_n=\displaystyle \sum_{k=1}^{n} a_k$.
		Do đó
		\begin{eqnarray*}
			&u_n&=\dfrac{1}{4}\cdot\left(\dfrac{1}{1}-\dfrac{1}{5}\right)+\dfrac{1}{4}.\left(\dfrac{1}{2}-\dfrac{1}{6}\right)+\ldots + \dfrac{1}{4}\cdot\left(\dfrac{1}{n}-\dfrac{1}{n+4}\right)\\
			&&=\dfrac{1}{4}\left(\dfrac{1}{1}+\dfrac{1}{2}+\dfrac{1}{3}+\dfrac{1}{4}-\dfrac{1}{n+1}-\dfrac{1}{n+2}-\dfrac{1}{n+3}-\dfrac{1}{n+4}\right)\\
			&&=\dfrac{1}{4}\left(\dfrac{25}{12}-\dfrac{1}{n+1}-\dfrac{1}{n+2}-\dfrac{1}{n+3}-\dfrac{1}{n+4}\right).
		\end{eqnarray*}
		Với mọi $n \in \mathbb{N}^\ast$ thì $u_n>0$ nên dãy số $\left(u_n\right)$ bị chặn dưới.\\
		Ta lại có $u_n=\dfrac{1}{4}\cdot\left(\dfrac{25}{12}-\dfrac{1}{n+1}-\dfrac{1}{n+2}-\dfrac{1}{n+3}-\dfrac{1}{n+4}\right)<\dfrac{1}{4}\cdot\dfrac{25}{12}=\dfrac{25}{48}$.\\
		Vậy $m=0$, $M=\dfrac{25}{48}$.
	}
\end{ex}
%Câu 9
\begin{ex}%[1K2G5-4]%[Trương Đăng Khoa]
	Xét tính bị chặn của dãy số $\left(u_n\right)$ và đặt $u_n=\displaystyle \sum_{k=1}^{n} a_k$ với $a_k=\dfrac{1}{k(k+1)}$.
	\choice{\True Bị chặn}{Bị chặn dưới}{Bị chặn trên}{Không bị chặn.}
	\loigiai{
		Ta có $a_k=\dfrac{1}{k(k+1)}=\dfrac{1}{k}-\dfrac{1}{k+1}$. Do đó\\
		$u_n=\displaystyle \sum_{k=1}^{n}a_k=\left(1-\dfrac{1}{2}\right)+\left(\dfrac{1}{2}-\dfrac{1}{3}\right)+\ldots+\left(\dfrac{1}{n-1}-\dfrac{1}{n}\right)+\left(\dfrac{1}{n}-\dfrac{1}{n+1}\right)=1-\dfrac{1}{n+1}=\dfrac{n}{n+1}$.\\
		Với mọi $n \in \mathbb{N}^*$ thì $u_n>0$ nên dãy số $\left(u_n\right)$ bị chặn dưới.\\
		Ta lại có $u_n=1-\dfrac{n}{n+1}<1$, $\forall n \in \mathbb{N}^\ast$ nên dãy số $\left(u_n\right)$ bị chặn trên.\\
		Vậy dãy số bị chặn.
	}
\end{ex}
%Câu 10
\begin{ex}%[1K2G5-4]%[Trương Đăng Khoa]
	Cho dãy số $(u_n)$, xác định bởi $\heva{&u_1=6\\&u_{n+1}=\sqrt{6+u_n},\, \forall n\in\mathbb{N}^\ast}$. Mệnh đề nào sau đây là đúng?
	\choice{
		$\sqrt{6}<u_n<2\sqrt{3}$	
	}
	{\True $\sqrt{6}\leq u_n\leq 2\sqrt{3}$}
	{$\sqrt{6}<u_n\leq 2\sqrt{3}$	}
	{$\sqrt{6}\geq u_n<2\sqrt{3}$	}
	\loigiai{
		Ta có 
		$\heva{&u_1=6\\
			&u_{n+1}=\sqrt{6+u_n}} \Rightarrow
		\heva{
			&u_1=6\\
			&u_{n+1} \ge 0 } \Rightarrow u_n \ge 0 \Rightarrow\heva{&u_1=6\\
			&u_{n+1}=\sqrt{6+u_n}\ge\sqrt{6}}
		\Rightarrow u_n \ge\sqrt{6}$.\\
		Ta chứng minh quy nạp $\heva{&u_n\le 2\sqrt{3}\\ &u_1\le 2\sqrt{3}\\ &u_k\le 2\sqrt{3}.}$\\
		$\Rightarrow u_{k+1}=\sqrt{6+u_{k+1}} \le\sqrt{6+2 \sqrt{3}}<\sqrt{6+6}=2 \sqrt{3}$.\\
		Vậy $\sqrt{6} \leq  u_n \leq 2\sqrt{3}$.
	}
\end{ex}
\Closesolutionfile{ans}
\begin{indapan}{10}
	{ans/ans-1K2-1-Dang3}
\end{indapan}

\begin{dang}{Toán thực tế về dãy số}
\end{dang}
\subsubsection{Ví dụ minh hoạ}
% \begin{vd}%[1T2B1-5]%[Trương Đăng Khoa]%Ví dụ 1
% 	Một chồng cột gỗ được xếp thành các lớp, hai lớp liên tiếp hơn kém nhau một cột gỗ.
% 	\begin{center}
% 		\begin{tikzpicture}[font=\footnotesize, line join=round, line cap=round, >=stealth,scale=0.8]
% 			\def\r{0.2}
% 			\def\n{25}
% 			\def\g{110}
% 			\fill[teal!50!green](-6*\r,-3*\r)rectangle(3.5*\n*\r,0.5*\n*\r);
% 			\fill[teal!50!green,opacity=0.25](3.5*\n*\r,3*\r)rectangle(-6*\r,2*\n*\r);
% 			\foreach \j in {0,...,12}{
% 				\pgfmathsetmacro{\m}{\n-\j}
% 				\foreach \i in{0,...,\m}{
% 					\fill[left color=orange, right color=teal!30,draw=brown](2*\i*\r,0)++(60:2*\j*\r)++(\g:\r)--++(\g-90:6)arc(\g:-60:\r)--++(\g-270:6)--cycle;
% 					\fill[orange!20!brown!40,draw=teal](2*\i*\r,0)++(60:2*\j*\r)circle(\r);
% 				}
% 			}
% 		\end{tikzpicture}
% 	\end{center}
% 	\begin{enumerate}
% 		\item  Gọi $u_1=25$ là số cột gỗ có ở hàng dưới cùng của chồng cột gỗ, $u_n$ là số cột gỗ có ở hàng thứ $n$ tính từ dưới lên trên. Xét tính tăng, giảm của dãy số này.
% 		\item  Gọi $v_1=14$ là số cột gỗ có ở hàng trên cùng của chồng cột gỗ, $v_n$ là số cột gỗ có ở hàng thứ $n$ tính từ trên xuống dưới. Xét tinh tăng, giảm của dãy số này.
% 	\end{enumerate}
% 	\loigiai{
% 		\begin{enumerate}
% 			\item Ta có $u_n=26-n>u_{n+1}=26-n-1=25-n$.\\
% 			Vậy dãy số $(u_n)$ là dãy số giảm.
% 			\item Ta có $v_n=13+n<v_{n+1}=13+n+1=14+n$.\\
% 			Vậy dãy số $(u_n)$ là dãy số tăng
% 		\end{enumerate}
% 	}
% \end{vd}

\begin{vd}%[1T2B1-5]%[Trương Đăng Khoa]%Ví dụ 2
	Trên lưới ô vuông, mỗi ô cạnh $1$ đơn vị, người ta vẽ $8$ hình vuông và tô màu khác nhau như hình vẽ. Tìm dãy số biểu diễn độ dài cạnh của $8$ hình vuông đó từ nhỏ đến lớn. Có nhận xét gì về dãy số trên?
	\begin{center}
		\begin{tikzpicture}[scale=0.8]
			\def\r{21}
			\def\hv(#1){
				\ifnum #1= 1\else
				\pgfmathsetmacro{\R}{250*rnd}
				\pgfmathsetmacro{\G}{250*rnd}
				\pgfmathsetmacro{\B}{250*rnd}
				\definecolor{mau}{RGB}{\R,\G,\B}
				\fill[mau!30](0,0)rectangle(\r,\r);
				\draw[red,line width=1pt] (0,0) arc(180:90:\r)(0,0)rectangle(\r,\r);
				\pgfmathtruncatemacro{\k}{#1-1}
				\begin{scope}[shift={(45:\r*sqrt(2))},rotate=-90,scale={(sqrt(5)-1)/2}]
					\hv(\k)
					\pgfmathsetmacro{\n}{int((1/(sqrt(5))*(((1+sqrt(5))/2)^(\k)-(1-(sqrt(5))/2)^(\k)+1)}
					\ifnum \k>1
					\path (\r/2,\r/2)node[scale=1.75]{\color{red}$\n$};
					\else
					\fi
				\end{scope}
				\fi
			}
			\begin{scope}[scale=0.35]
				\hv(9)
				\draw[teal](0,0)grid(34,21);
				\path(21/2,21/2)node[scale=2]{21};
			\end{scope}
		\end{tikzpicture}
	\end{center}
	\loigiai{
		\begin{multicols}{4}
			\begin{itemize}
				\item $u_1=1$.
				\item $u_2=1$.
				\item $u_3=2$.
				\item $u_4=3$.
				\item $u_5=5$.
				\item $u_6=8$.
				\item $u_7=13$.
				\item $u_8=21$.
			\end{itemize}
		\end{multicols}
		Ta có dãy số $\left(u_n\right)\colon\heva{& u_1=1\\ &u_2=1\\ &u_n=u_{n-1}-u_{n-2}.}$
	}
\end{vd}

\begin{vd}%[1C2K1-5]%[Trương Đăng Khoa]% Ví dụ 3
	Chị Mai gửi tiền tiết kiệm vào ngân hàng theo thể thức lãi kép như sau. Lần đầu chị gửi $100$ triệu đồng. Sau đó, cứ hết $1$ tháng chị lại gửi thêm vào ngân hàng $6$ triệu đồng. Biết lãi suất của ngân hàng là $0{,}5\%$ một tháng. Gọi $P_n$ (triệu đồng) là số tiền chị có trong ngân hàng sau $n$ tháng.
	\begin{enumerate}
		\item Tính số tiền chị có trong ngân hàng sau $1$ tháng.
		\item Tính số tiền chị có trong ngân hàng sau $3$ tháng.
		\item Dự đoán công thức của $P_n$ tính theo $n$.
	\end{enumerate}
	\loigiai{
		\begin{enumerate}
			\item Số tiền chị có trong ngân hàng sau $1$ tháng là $P_1=+100+100\cdot0{,}5\%+6=100{,}5+6$ (triệu đồng).
			\item Số tiền chị có trong ngân hàng sau 2 tháng là 
			\begin{eqnarray*}
				P_2&=&100{,}5+6+(100{,}5+6)\cdot 0{,}5\%+6\\
				&=&(100{,}5+6)(1+0{,}5\%)+6\\
				&=& 100{,}5(1+0{,}5\%)+6\cdot(1+0{,}5\%)+6\, (\text{triệu đồng}).
			\end{eqnarray*}
			Số tiền chị có trong ngân hàng sau $3$ tháng là
			\begin{eqnarray*}
				P_3&=&(100{,}5+6)(1+0{,}5 \%)+6+[(100{,}5+6)(1+0{,}5 \%)+6] \cdot 0{,}5 \%+6\\
				&=& 100{,}5 \cdot(1+0{,}5 \%)^2+6(1+0{,}5 \%)^2+6 \cdot(1+0{,}5 \%)+6 \text{(triệu đồng)}.
			\end{eqnarray*}
			\item Số tiền chị có trong ngân hàng sau $4$ tháng là
			\begin{eqnarray*}
				P_4&=&(100{,}5+6)(1+0{,}5 \%)^2+6 \cdot(1+0{,}5 \%)+6+\left[(100{,}5+6)(1+0{,}5 \%)^2+6 \cdot(1+0{,}5 \%)+6\right]\cdot 0{,}5 \%+6\\ 
				&=&100{,}5 \cdot(1+0{,}5 \%)^3+6 \cdot(1+0{,}5 \%)^3+6\cdot(1+0{,}5 \%)^2+6 \cdot(1+0{,}5 \%)+6\, (\text{triệu đồng}).
			\end{eqnarray*}
			Số tiền chị có trong ngân hàng sau $n$ tháng là
			$$P_n=100{,}5 \cdot(1+0{,}5 \%)^{n-1}+6\cdot(1+0{,}5 \%)^{n-1}+6\cdot(1+0{,}5 \%)^{n-2}+6 \cdot(1+0{,}5 \%)^{n-3}+\ldots+6$$ với mọi $n \in\mathbb{N}^\ast$.
		\end{enumerate}
	}
\end{vd}

\begin{vd}%[1K2K5-5]%[Trương Đăng Khoa]% Ví dụ 4
	Anh Thanh vừa được tuyển dụng vào một công ty công nghệ, được cam kết lương năm đầu sẽ là $200$ triệu đồng và lương mỗi năm tiếp theo sẽ được tăng thêm $25$ triệu đồng. Gọi $s_n$ (triệu đồng) là lương vào năm thứ $n$ mà anh Thanh làm việc cho công ty đó. Khi đó ta có
	$$s_1=200,\, s_n=s_{n-1}+25\, \text{với}\, n \ge 2.$$
	\begin{enumerate}
		\item Tính lương của anh Thanh vào năm thứ $5$ làm việc cho công ty.
		\item Chứng minh $(s_n)$ là dãy số tăng. Giải thích ý nghĩa thực tế của kết quả này. 
	\end{enumerate}
	\loigiai{
		\begin{enumerate}
			\item Ta có \begin{eqnarray*}
				s_2&=&s_1+25=200+25=225\\
				s_3&=&s_2+25=225+25=250\\
				s_4&=&s_3+25=250+25=275\\
				s_5&=&s_4+25=275+25=300. 
			\end{eqnarray*}
			Vậy lương của anh Thanh vào năm thứ $5$ làm việc cho công ty là $300$ triệu đồng.
			\item  Ta có $s_n=s_{n-1}+25\Leftrightarrow s_n-s_{n-1}=25>0$ với mọi $n\ge 2$, $n\in\mathbb{N}^\ast$.\\
			Tức là $s_n>s_{n-1}$ với mọi $n\ge 2$, $n\in\mathbb{N}^\ast$.\\
			Vậy $(s_n)$ là dãy số tăng.\\
			Điều này có nghĩa là mức lương hàng năm của anh Thanh tăng dần theo thời gian làm việc.
		\end{enumerate}
	}
\end{vd}

\begin{vd}%[1K2K5-5]%[Trương Đăng Khoa]%Ví dụ 5
	Ông An gửi tiết kiệm $100$ triệu đồng kì hạn $1$ tháng với lãi suất $6\%$ một năm theo hình thức tính lãi kép. Số tiền (triệu đồng) của ông An thu được sau $n$ tháng được cho bởi công thứC 
	$$A_n=100\left(1+\dfrac{0{,}06}{12}\right)^n.$$
	\begin{enumerate}
		\item Tìm số tiền ông An nhận được sau tháng thứ nhất, sau tháng thứ hai.
		\item Tìm số tiền ông An nhận được sau $1$ năm.
	\end{enumerate}
	\loigiai{
		\begin{enumerate}
			\item Số tiền ông An nhận được sau tháng thứ nhất là 
			$$A_1=100\left(1+\dfrac{0{,}06}{12}\right)^1=100{,}5\, \text{(triệu đồng)}.$$
			Số tiền ông An nhận được sau tháng thứ hai là 
			$$A_2=100\left(1+\dfrac{0{,}06}{12}\right)^2=101{,}0025\, \text{(triệu đồng)}.$$
			\item  Số tiền ông An nhận được sau $1$ năm ($12$ tháng) là 
			$$A_{12}=100\left(1+\dfrac{0{,}06}{12}\right)^{12} \approx 106{,}17\, \text{(triệu đồng)}.$$
		\end{enumerate}
	}
\end{vd}

\begin{vd}%[1K2G5-5]%[Trương Đăng Khoa]%Ví dụ 6
	Chị Hương vay trả góp một khoản tiền $100$ triệu đồng và đồng ý trả dần $2$ triệu đồng mỗi tháng với lãi suất $0{,}8\%$ số tiền còn lại của mỗi tháng.
	Gọi $A_n$, ($n\in\mathbb{N}$) là số tiền còn nợ (triệu đồng) của chị Hương sau $n$ tháng.
	\begin{enumerate}
		\item Tìm lần lượt $A_0$, $A_1$, $A_2$, $A_3$, $A_4$, $A_5$, $A_6$ đễ tính số tiền còn nợ của chị Hương sau $6$ tháng.
		\item  Dự đoán hệ thức truy hồi đối với dãy số $(A_n)$.
	\end{enumerate}
	\loigiai{
		\begin{enumerate}
			\item  Ta có $A_0=100$ (triệu đồng).
			\begin{itemize}
				\item Tiền lãi chị Hương phải trả sau $1$ tháng là $100\cdot 0{,}8\%=0{,}8$ (triệu đồng).\\
				Do đó, số tiền gốc chị Hương trả được sau $1$ tháng là $2-0{,}8=1{,}2$ (triệu đồng).\\
				Khi đó, số tiền còn nợ của chị Hương sau $1$ tháng là 
				$A_1=100-1{,}2=98{,}8$ (triệu đồng).
				\item  Tiền lãi chị Hương phải trả sau $2$ tháng là $98{,}8\cdot 0{,}8\%=0{,}7904$ (triệu đồng).\\
				Do đó, số tiền gốc chị Hương trả được sau $2$ tháng là $2-0{,}7904=1{,}2096$ (triệu đồng).\\
				Khi đó, số tiền còn nợ của chị Hương sau $2$ tháng là 
				$A_2=98{,}8-1{,}2096=97{,}5904$ (triệu đồng).
				\item Tiền lãi chị Hương phải trả sau $3$ tháng là $97{,}5904\cdot 0{,}8\%=0{,}7807232$ (triệu đồng).\\
				Do đó, số tiền gốc chị Hương trả được sau $3$ tháng là $2-0{,}7807232=1{,}2192768$ (triệu đồng).\\
				Khi đó, số tiền còn nợ của chị Hương sau $3$ tháng là 
				$A_3=97{,}5904-1{,}2192768=96{,}3711232$ (triệu đồng).
				\item Tiền lãi chị Hương phải trả sau $4$ tháng là $96{,}3711232\cdot 0{,}8\%\approx 0{,}77097$ (triệu đồng).\\
				Do đó, số tiền gốc chị Hương trả được sau $4$ tháng là $2-0{,}77097=1{,}22903$ (triệu đồng).\\
				Khi đó, số tiền còn nợ của chị Hương sau $4$ tháng là 
				$A_4=96{,}3711232-1{,}22903=95{,}1420932$ (triệu đồng).
				\item Tiền lãi chị Hương phải trả sau $5$ tháng là $95{,}1420932\cdot 0{,}8\%\approx 0{,}76114$ (triệu đồng).\\
				Do đó, số tiền gốc chị Hương trả được sau $5$ tháng là $2-0{,}76114=1{,}23886$ (triệu đồng).\\
				Khi đó, số tiền còn nợ của chị Hương sau $5$ tháng là $A_5=95{,}1420932-1{,}23886=93{,}9032332$ (triệu đồng).
				\item Tiền lãi chị Hương phải trả sau $6$ tháng là $93{,}9032332\cdot 0{,}8\%\approx 0{,}75123$ (triệu đồng).\\
				Do đó, số tiền gốc chị Hương trả được sau $6$ tháng là $2-0{,}75123=1{,}24877$ (triệu đồng).\\
				Khi đó, số tiền còn nợ của chị Hương sau $6$ tháng là $A_6=93{,}9032332-1{,}24877=92{,}6544632$ (triệu đồng).
			\end{itemize}
			\item Dự đoán hệ thức truy hồi đối với dãy số $(A_n)$ là 
			\[A_0=100,\, A_n=A_{n-1}-\left(2-A_{n-1} \cdot 0{,}8 \%\right)=1{,}008 A_{n-1}-2\]
		\end{enumerate}
	}
\end{vd}

%%Bài 6. CSC
% \def\tenchude{CẤP SỐ CỘNG}
\setcounter{section}{5}
\setcounter{dang}{0}
\setcounter{ex}{0}
\setcounter{bt}{0}
\setcounter{vd}{0}
\section{Cấp số cộng}
\subsection{Tóm tắt lý thuyết}
\begin{tomtat}
	\subsubsection{Định nghĩa}
	Dãy số là cấp số cộng nếu mỗi một số hạng (kể từ số hạng thứ hai) đều bằng tổng của số hạng đứng ngay trước nó với một số không đổi $ d $.\\
	Dãy số $ (u_n) $ là cấp số cộng $ \Leftrightarrow u_{n+1}=u_n+d $, $ \forall n \in \mathbb{N}^* $.\\
	$ d $ là số không đổi, gọi là \textbf{\textit{công sai}} của cấp số cộng.
	\subsubsection{Tính chất}
	Nếu $ (u_n) $ là cấp số cộng thì kể từ số hạng thứ hai (trừ số hạng cuối nếu là cấp số cộng hữu hạn) đều là trung bình cộng của hai số hạng đứng kề nó trong dãy. Tức là $$u_k=\dfrac{u_{k-1}+u_{k+1}}{2}, (\forall k\ge 2, k \in \mathbb{N}^*).$$
	\subsubsection{Số hạng tổng quát}
	Nếu cấp số cộng $ (u_n) $ có số hạng đầu $ u_1 $ và công sai $ d $ thì số hạng tổng quát $ u_n $ được xác định bởi công thức $$u_n=u_1+(n-1)d \text{ với $n\ge 2$}.$$
	\subsubsection{Tổng $ n $ số hạng đầu tiên}
	Cho cấp số cộng $ (u_n) $. Tổng $ n $ số hạng đầu tiên của cấp số cộng kí hiệu là $ S_n=u_1+u_2+\ldots+u_n $.\\
	Khi đó $ S_n $ được tính theo công thức $$ S_n=\dfrac{n(u_1+u_n)}{2}=\dfrac{n}{2}\left[ 2u_1+(n-1)d\right]. $$
\end{tomtat}
\subsection{Các dạng toán thường gặp}
\begin{dang}{Nhận diện cấp số cộng, công sai $ d $, số hạng tổng quát $ u_n $}
	% Dựa theo định nghĩa của cấp số cộng, để nhận diện $ (u_n) $ là cấp số cộng $ \Leftrightarrow u_{n+1}=u_n+d $.\\
	% Khi đó công sai $ d=u_{n+1}-u_{n} $, $ \forall n \in \mathbb{N}^* $.
\end{dang}
\subsubsection{Ví dụ minh hoạ}
\begin{vd}%[NB]%[DCHT Toán 11 - KNTT -Lê Hải Phụng] %[1K2Y6-1]
	Dãy số hữu hạn nào là một cấp số cộng? Vì sao?
	\begin{listEX}[2]
		\item  $-2$, $1$, $4$, $7$, $10$, $13$, $16$.
		\item  $ 1 $, $ -2 $, $ -4 $, $ -6 $, $ -8 $.
	\end{listEX}
	\dapso{ Dãy số 1 là một cấp số cộng, dãy số 2 không là một cấp số cộng.}
	\loigiai{
		\begin{enumerate}
			\item Ta thấy $ u_2=u_1+3 $  do $ 1=(-2)+3 $.\\
			Vì $ u_k=u_{k-1}+d,\ \forall k\geq2$ $\left(\ 1=\left(-2\right)+3;4=1+3;7=4+3;10=7+3;13=10+3;16=13+3\right) $ nên dãy số đã cho là cấp số cộng. 
			\item Ta thấy $ u_2=u_1+(-3) $  do $-2=1+(-3)$.\\
			Vì $ {u_3\neq u}_2+(-3) $ bởi $ \left(\ -4\neq-2+\left(-3\right)\right)\ $ nên dãy số đã cho không là cấp số cộng.
		\end{enumerate}
	}
\end{vd}
\begin{vd}%[TH]%[DCHT Toán 11 - KNTT -Lê Hải Phụng] %[1K2B6-1]
	Trong các dãy số dưới đây, dãy số nào là cấp số cộng? 
	\begin{listEX}[2]
		\item  Dãy số $\left({a_n}\right)$ với ${a_n}=4n-3$;
		\item  Dãy số $\left({c_n}\right)$ với ${c_n}={2018^n}$.
	\end{listEX}
	\dapso{Dãy số 1 là một cấp số cộng, dãy số 2 không là một cấp số cộng.}
	\loigiai{	
		\begin{enumerate}
			\item Ta có $a_{n+1}=4(n+1)-3=4n+1$ nên $a_{n+1}-a_n=(4n+1)-(4n-3)=4$,$\forall n\ge 1.$.\\
			Do đó $(a_n)$ là cấp số cộng.
			\item Ta có $c_{n+1}=2018^{n+1}$ nên $c_{n+1}-c_n=2018^{n+1}-2018^n=2017\cdot 2018^n$ (phụ thuộc vào giá trị của $n$).\\ 
			Suy ra $(c_n)$ không phải là một cấp số cộng.
		\end{enumerate}	
	}
\end{vd}
\begin{vd}%[NB]%[DCHT Toán 11 - KNTT -Lê Hải Phụng] %[1K2Y6-1]
	Cho cấp số cộng $(u_n)$  có công thức số hạng tổng quát $u_n=3n+1$, $n\in\mathbb{N}^\ast$ . Tìm số hạng đầu $u_1$ và công sai $d$?
	\dapso{$u_1=4 $, $d=3$.}
	\loigiai{
		Từ công thức số hạng tổng quát, ta có $ u_1=4 $, $u_2=7$ suy ra $d=u_2-u_1=3$.
	}
\end{vd}

\begin{vd}%[TH]%[DCHT Toán 11 - KNTT -Lê Hải Phụng] %[1K2B6-1]
	Cho cấp số cộng $(u_n)$ với $u_1=3$, $u_2=9$. Công sai của cấp số cộng đã cho bằng bao nhiêu?
	\dapso{$ d=6 $}
	\loigiai{
		Cấp số cộng $(u_n)$ có số hạng tổng quát là $u_n=u_1+(n-1)d$ với $n \ge 2$.\\
		Suy ra $u_2=u_1+d \Leftrightarrow 9=3+d \Leftrightarrow d=6$.\\
		Vậy công sai của cấp số cộng đã cho là $6$.
	}
\end{vd}
\begin{vd}%[VD]%[DCHT Toán 11 - KNTT -Lê Hải Phụng] %[1K2K6-1]
	Tính số hạng đầu $u_1$ và công sai $d$ của một cấp số cộng biết $u_4=10$ và $u_7=19$.
	\dapso{$ u_1=1 $, $ d=3 $.}
	\loigiai{Ta có $ \heva{& u_4=10 \\ & u_7=19} \Leftrightarrow \heva{& u_1+3d=10 \\ & u_1+6d=19} \Leftrightarrow \heva{& u_1=1 \\ & d=3.}$}
\end{vd}

\begin{vd}%[TH]%[Dự án DCHT-11-KNTT]%[Dao-V- Thuy]%[1K2B5-1]
	Xác định số hạng tổng quát của cấp số cộng $(u_n),$ biết $\heva{&u_7=8\\ &d=2.}$
	\dapso{$u_n=2n-6$}
	\loigiai{
		Ta có
		\begin{equation*}
			\heva{&u_7=8\\ &d=2} \Leftrightarrow \heva{&u_1+6d=8\\&d=2} \Leftrightarrow \heva{&u_1=-4\\ &d=2.}
		\end{equation*}
		Vậy công thức tổng quát của cấp số cộng
		\begin{center}
			$u_n=-4+(n-1)2 \Leftrightarrow u_n=2n-6 $ với $n \geq 2.$
		\end{center}	
	}
\end{vd}

% \begin{vd}%[TH]%[Dự án DCHT-11-KNTT]%[Dao-V- Thuy]%[1K2B5-1]
% 	Tìm số hạng đầu và công sai của cấp số cộng $(u_n)$, biết $\heva{&u_1+u_5-u_3=10\\ &u_1+u_6=17.}$
% 	\dapso{$u_1=16$, $d=-3$}
% 	\loigiai{
% 		Ta có
% 		\begin{align*}
% 			\heva{&u_1+u_5-u_3=10\\ &u_1+u_6=17} &\Leftrightarrow \heva{&u_1+u_1+4d-(u_1+2d)=10\\ &u_1+u_1+5d=17}\\ & \Leftrightarrow\heva{&u_1+2d=10 \\ &2u_1+5d=17} \Leftrightarrow \heva{&u_1=16 \\ &d=-3.}
% 		\end{align*}
% 		Vậy $u_1=16$, $d=-3$.
% 	}
% \end{vd}

\begin{vd}%[TH]%[Dự án DCHT-11-KNTT]%[Dao-V- Thuy]%[1K2B5-1]
	Cho cấp số cộng $(u_n)$ với $\heva{&u_1=-9\\ &u_{n-1}=u_n-5}$. Tìm số hạng tổng quát của cấp số cộng $(u_n)$.
	\dapso{$u_n= 5n-14$}
	\loigiai{
		Từ công thức $u_{n-1}=u_n-5 \Leftrightarrow u_n= u_{n-1}+5$, suy ra $d=5$.\\
		Vậy công thức tổng quát của cấp số cộng $(u_n)$ là $u_n=-9 + 5(n-1)= 5n-14$.
	}
\end{vd}

\begin{vd}%[TH]%[Dự án DCHT-11-KNTT]%[Dao-V- Thuy]%[1K2B5-1]
	Cho cấp số cộng $(u_n)$ có $u_{20}=-52$ và $u_{51}=-145$. Hãy tìm số hạng tổng quát của cấp số cộng đó.
	\dapso{$u_n= -3n+8$}
	\loigiai{
		Ta có
		\begin{eqnarray*}
			\heva{&u_{20}=-52 \\&u_{51}=-145} &\Leftrightarrow& \heva{&u_1+19d=-52\\ &u_1+50d=-145} \Leftrightarrow \heva{&u_1=5 \\ &d= -3.}
		\end{eqnarray*}
		Vậy số hạng tổng quát cần tìm là $u_n= u_1+ (n-1)d= 5+(n-1) \cdot (-3)= -3n+8$.
	}
\end{vd}

\begin{vd}%[VD]%[Dự án DCHT-11-KNTT]%[Dao-V- Thuy]%[1K2B5-1]
	Tìm số hạng đầu và công sai của cấp số cộng $(u_n)$, biết
		\begin{listEX}[2]
			\item $\heva{&u_9=5u_2\\ &u_{13}=2u_6+5.}$
			\item $\heva{&u_1-u_3+u_5=10\\ &u_1+u_6=7.}$
		\end{listEX}
	\dapso{$u_1=3$, $d=4$; $u_1=36$, $d=-13$}
	\loigiai
	{
		\begin{enumerate}
			\item Ta có
			\begin{eqnarray*}
				\heva{&u_9=5u_2\\ &u_{13}=2u_6+5} &\Leftrightarrow& \heva{&u_1+8d= 5 \left( u_1+d \right) \\ &u_1+12d = 2 \left( u_1+5d\right) + 5}\\
				&\Leftrightarrow& \heva{&-4u_1+3d=0\\ &-u_1+2d=5} \Leftrightarrow \heva{&u_1=3 \\ &d=4.}
			\end{eqnarray*}
			Vậy $u_1=3$, $d=4$.
			\item Ta có 
			\begin{eqnarray*}
				\heva{&u_1-u_3+u_5=10\\ &u_1+u_6=7} &\Leftrightarrow& \heva{&u_1-\left( u_1+2d\right) + \left( u_1+4d\right) = 10\\ &u_1+ \left( u_1+5d\right) = 7}\\
				&\Leftrightarrow& \heva{&u_1+2d=10\\ &2u_1+5d=7} \Leftrightarrow \heva{&u_1=36 \\ &d=-13.}
			\end{eqnarray*}
			Vậy $u_1=36$, $d=-13$.
		\end{enumerate}
	}
\end{vd}

\begin{vd}%[VD]%[Dự án DCHT-11-KNTT]%[Dao-V- Thuy]%[1K2K5-1]
	Tìm số hạng đầu và công sai của cấp số cộng $(u_n)$, biết
		\begin{listEX}[2]
			\item $\heva{&-u_3+u_7=8\\ &u_2u_7=75.}$
			\item $\heva{&u_5=4u_3\\ &u_2u_6=-11.}$
		\end{listEX}
	\dapso{$\heva{&u_1=3\\ &d=2} \text{ hoặc } \heva{&u_1=-17\\ &d=2}$; $\heva{&u_1=-4\\ &d=3}$ hoặc $\heva{&u_1=4\\ &d=-3}$}
	\loigiai{
		\begin{enumerate}
			\item Ta có
			\begin{eqnarray*}
				\heva{&-u_3+u_7=8\\ &u_2u_7=75} &\Leftrightarrow& \heva{&-\left( u_1+2d\right) + \left( u_1+6d\right) = 8\\ &\left( u_1+d\right) \left( u_1+6d\right) = 75}\\
				&\Leftrightarrow& \heva{&4d=8\\ &u_1^2+7u_1d+6d^2=75}\\
				&\Leftrightarrow& \heva{&d=2\\ &u_1^2+14u_1-51=0}\\
				&\Leftrightarrow& \heva{&u_1=3\\ &d=2} \text{ hoặc } \heva{&u_1=-17\\ &d=2.}
			\end{eqnarray*}
			Vậy $\heva{&u_1=3\\ &d=2} \text{ hoặc } \heva{&u_1=-17\\ &d=2.}$
			\item Ta có 
			\begin{eqnarray*}
				\heva{&u_5=4u_3\\ &u_2u_6=-11} &\Leftrightarrow& \heva{&u_1+4d=4 \left( u_1+2d\right)\\ &\left( u_1+d\right) \left( u_1+5d\right)= -11}\\
				&\Leftrightarrow& \heva{&3u_1+4d=0 &(1)\\ &u_1^2+6du_1+5d^2=-11 &(2)}
			\end{eqnarray*}
			Từ $(1)$ suy ra $3u_1=-4d$. Thay vào $(2)$ ta được
			\begin{eqnarray*}
				9u_1^2+54du_1+45d^2=-99 &\Leftrightarrow& 16d^2 -72d^2+45d^2=-99\\
				&\Leftrightarrow& -11d^2=-99 \Leftrightarrow \hoac{&d=3\\ &d=-3.}
			\end{eqnarray*}
			Với $d=3$, ta có $u_1=-4$.\\
			Với $d=-3$, ta có $u_1=4$.\\
			Vậy $\heva{&u_1=-4\\ &d=3}$ hoặc $\heva{&u_1=4\\ &d=-3.}$
		\end{enumerate}
	}
\end{vd}

\subsubsection{Bài tập tự luận}
 

\begin{bt}%[NB]%[DCHT Toán 11 - KNTT -Lê Hải Phụng] %[1K2Y6-1]
	Trong các dãy số sau, dãy số nào là một cấp số cộng?
	\begin{listEX}[1]
		\item $1$, $-3$, $-7$, $-11$, $-15$, $\ldots$;
		\item $1$, $-2$, $-4$, $-6$, $-8,$ $\ldots$.
		\item $ \dfrac{1}{2} $, $0$, $-\dfrac{1}{2}$, $-1$, $-\dfrac{3}{2}$, $\ldots$
	\end{listEX}
	\dapso{1) và 3) là cấp số cộng.}
	\loigiai{Ta lần lượt đi kiểm tra: $ u_2-u_1=u_3-u_2=u_4-u_3=\ldots $?\\
		Xét từng dãy số thì ta thấy 1) và 3) là cấp số cộng. 
	}
\end{bt}

\begin{bt}%[NB]%[DCHT Toán 11 - KNTT -Lê Hải Phụng] %[1K2Y6-1]
	Trong các dãy số sau, dãy nào là cấp số cộng. Tìm số hạng đầu và công sai của cấp số cộng đó.
	\begin{listEX}[2]
		\item Dãy số $ (u_n) $ với $ u_n=19n-5 $;
		\item Dãy số $ (u_n) $ với $ u_n=n^2+n+1 $. 
	\end{listEX}
	\dapso{Dãy số 1) $ (u_n) $ là một cấp số cộng với số hạng đầu là $ u_1=19\cdot1-5=14 $ và công sai $ d=19 $. Dãy số 2) không là một cấp số cộng.}
	\loigiai{
		\begin{enumerate}
			\item Dãy số $ (u_n) $ với $ u_n=19n-5 $.\\
			Ta có $ u_{n+1}-u_n=19(n+1)-5-(19n-5)=19 $. Vậy $ (u_n) $ là một cấp số cộng với số hạng đầu là $ u_1=19\cdot1-5=14 $ và công sai $ d=19 $.
			\item Dãy số $ (u_n) $ với $ u_n=n^2+n+1 $.\\
			Ta có $ u_{n+1}-u_n=(n+1)^2+(n+1)+1-(n^2+n+1)=2n+2$ phụ thuộc vào $ n $. Vậy $ (u_n) $ không là một cấp số cộng.
	\end{enumerate}}
\end{bt}

\begin{bt}%[TH]%[DCHT Toán 11 - KNTT -Lê Hải Phụng] %[1K2B6-1]
	Cho cấp số cộng $\left(u_n\right)$ với $u_1=3$, $u_2=9$. Công sai của cấp số cộng đã cho bằng bao nhiêu?
	\dapso{Công sai của cấp số cộng đã cho là 6.}
	\loigiai{Cấp số cộng $(u_n)$ có số hạng tổng quát là
		$u_n=u_1+\left(n-1\right)d$ với $n \ge 2$
		(số hạng đầu $u_1$ và công sai $d$)\\
		Suy ra $ u_2=u_1+d\Leftrightarrow9=3+d\Leftrightarrow d=6 $.\\
		Vậy công sai của cấp số cộng đã cho là 6.
	}
\end{bt}


\begin{bt}%[TH]%[Dự án DCHT-11-KNTT]%[Dao-V- Thuy]%[1K2B5-1]
	Xác định công thức tổng quát của cấp số cộng $(u_n)$, biết $\heva{&u_{11}=5\\ &d=-6.}$
	\loigiai{
		Ta có
		\begin{equation*}
			\heva{&u_{11}=5\\ &d=-6} \Leftrightarrow \heva{&u_1+10d=5\\ &d=-6} \Leftrightarrow \heva{&u_1=65\\ &d=-6.}
		\end{equation*}
		Vậy công thức tổng quát của cấp số cộng:
		\begin{center}
			$u_n=65+(n-1).(-6) \Leftrightarrow u_n=-6n+71$  với $n \geq 2.$
		\end{center}	
	}
\end{bt}

\begin{bt}%[TH]%[Dự án DCHT-11-KNTT]%[Dao-V- Thuy]%[1K2B5-1]
	Tìm số hạng đầu và công sai của cấp số cộng $(u_n),$ biết $\heva{&u_2+u_5-u_3=10\\ &u_4+u_6=26.}$
	\loigiai{
		Ta có
		\begin{align*}
			\heva{&u_2+u_5-u_3=10\\ &u_4+u_6=26} 
			&\Leftrightarrow \heva{&u_1+d+u_1+4d-(u_1+2d)=10\\ &u_1+3d+u_1+5d=26}\\ 
			& \Leftrightarrow\heva{&u_1+3d=10 \\ &2u_1+8d=26}
			\Leftrightarrow \heva{&u_1=1 \\&d=3.}
		\end{align*}
		Vậy $u_1=1$, $d=3$.
	}
\end{bt}

\begin{bt}%[TH]%[Dự án DCHT-11-KNTT]%[Dao-V- Thuy]%[1K2B5-1]
	Tìm số hạng đầu và công sai của cấp số cộng, biết
		\begin{listEX}[3]
			\item $\heva{&u_7 = 27\\&u_{15} = 59.}$
			\item $\heva{&u_9 = 5u_2\\&u_{13} = 2u_6 + 5.}$
			\item $\heva{&u_2 + u_4 - u_6 = -7\\&u_8 - u_7 = 2u_4.}$
			\item $\heva{&u_3 - u_7 = -8\\&u_2 \cdot u_7 = 75.}$
			\item $\heva{&u_6 + u_7 = 60\\&u_4^2 + u_{12}^2 = 1170.}$
		\end{listEX}
	\loigiai{
		\begin{enumerate}
			\item Ta có $\heva{&u_7 = 27\\&u_{15} = 59} \Leftrightarrow \heva{&u_1 + 6d = 27\\&u_1 + 14d = 59} \Leftrightarrow \heva{&u_1 = 3\\&d = 4.}$\\
			Vậy số hạng đầu của cấp số cộng là $u_1 = 3$, công sai là $d = 4$.
			\item Ta có $\heva{&u_9 = 5u_2\\&u_{13} = 2u_6 + 5} \Leftrightarrow \heva{&u_1 + 8d = 5u_1 + 5d\\&u_1 + 12d = 2u_1 + 10d + 5} \Leftrightarrow \heva{&4u_1 - 3d = 0\\&-u_1 + 2d = 5} \Leftrightarrow \heva{&u_1 = 3\\&d = 4.}$\\
			Vậy số hạng đầu của cấp số cộng là $u_1 = 3$, công sai là $d = 4$.
			\item Ta có $\heva{&u_2 + u_4 - u_6 = -7\\&u_8 - u_7 = 2u_4} \Leftrightarrow \heva{&u_1 + d + u_1 + 3d - u_1 - 5d = -7\\&u_1 + 7d - u_1 - 6d = 2u_1 + 6d} \Leftrightarrow \heva{&u_1 - d = -7\\&2u_1 + 5d = 0} \Leftrightarrow \heva{&u_1 = -5\\&d = 2.}$\\
			Vậy số hạng đầu của cấp số cộng là $u_1 = -5$, công sai là $d = 2$.
			\item Ta có $\heva{&u_3 - u_7 = -8\\&u_2 \cdot u_7 = 75} \Leftrightarrow \heva{&u_1 + 2d -u_1 - 6d = -8\\&(u_1 + d)(u_1 + 6d) = 75} \Leftrightarrow \heva{&d = 2\\&u_1^2 + 14u_1 - 51 = 0} \Leftrightarrow \heva{&d = 2\\&\hoac{&u_1 = 3\\&u_1 = -17.}}$\\
			Vậy số hạng đầu của cấp số cộng là $u_1 = 3$, công sai là $d = 2$ hoặc $u_1 = -17$, $d = 2$.
			\item Ta có $\heva{&u_6 + u_7 = 60\\&u_4^2 + u_{12}^2 = 1170} \Leftrightarrow \heva{&2u_6 + d = 60&(1)\\&(u_6 - 2d)^2 + (u_6 + 6d)^2 = 1170.&(2)}$\\
			Từ (1), suy ra $d = 60 - 2u_6$, thay vào (2), ta có
			$$(5u_6 - 120)^2 + (360 - 11u_6)^2 = 1170 \Leftrightarrow 146u_6^2 - 9120u_6 + 142830 = 0 \,\, (\text{vô nghiệm}).$$ 
			Vậy không tồn tại cấp số cộng thỏa yêu cầu bài toán.
		\end{enumerate}
	}
\end{bt}
% \begin{bt}%[TH]%[DCHT Toán 11 - KNTT -Lê Hải Phụng] %[1K2B6-1]
% 	Tìm số hạng đầu tiên, công sai của cấp số cộng sau $ \heva{&u_5=19\\&u_9=35.}$
	
% 	\dapso{Số hạng đầu tiên $ u_1=3 $, công sai $ d=4 $.}
% 	\loigiai{Áp dụng công thức $ u_n=u_1+(n-1)d $ ta có $\heva{&u_5=19\\&u_9=35} \Leftrightarrow \heva{&u_1+4d=19\\&u_1+8d=35} \Leftrightarrow \heva{&u_1=3\\&d=4.}$\\
% 	Vậy số hạng đầu tiên $ u_1=3 $, công sai $ d=4 $.}
% \end{bt}

\begin{bt}%[VD]%[1K2K6-1]
	Cho cấp số cộng $ (u_n) $ thỏa mãn $ \heva{&u_2+u_4-u_6=-7\\&u_8+u_7=2u_4} $. Xác định số hạng đầu $ u_1 $ và công sai $ d $ cấp số cộng.        
	
	% \dapso{$ u_1=5 $, $ d=2 $.}
	\loigiai{
		Ta có $ \heva{&u_2+u_4-u_6=-7\\&u_8+u_7=2u_4} \Leftrightarrow \heva{& u_1+d+(u_1+3d)-(u_1+5d)=-7 \\ & u_1+7d-(u_1+6d)=2(u_1+3d)} \Leftrightarrow \heva{& u_1-d=-7 \\ & 2u_1+5d=0} \Leftrightarrow \heva{&u_1=-5\\&d=2.}$}
\end{bt}

\begin{bt}%[VD]%[DCHT Toán 11 - KNTT -Lê Hải Phụng] %[1K2K6-1]
Cho cấp số cộng $ (u_n) $ thỏa mãn $ \heva{&u_2-u_3+u_5=10\\&u_4+u_6=26} $. Xác định số hạng đầu $ u_1 $ và công sai $ d $ cấp số cộng.         
\dapso{$ u_1=1 $, $ d=3 $.}
\loigiai{Ta có $ \heva{&u_2-u_3+u_5=10\\&u_4+u_6=26} \Leftrightarrow \heva{& u_1+d-(u_1+2d)+u_1+4d=10 \\ & u_1+3d+u_1+5d=26} \Leftrightarrow \heva{& u_1+3d=10 \\ & u_1+4d=13} \Leftrightarrow \heva{u_1=1\\d=3.}$}
\end{bt}

\begin{bt}%[VDC]%[DCHT Toán 11 - KNTT -Lê Hải Phụng] %[1K2G6-1]
Tính số hạng đầu $ u_1 $ và công sai $d$ của một cấp số cộng biết $ \heva{&u_1+u_2+u_3=27\\&u_1^2+u_2^2+u_3^2=275} $

\dapso{$ u_1=5 $, $ d=4 $ hoặc $ u_1=13 $, $ d=-4 $.}
\loigiai{Ta có $ \heva{&u_1+u_2+u_3=27\\&u_1^2+u_2^2+u_3^2=275} \Leftrightarrow \heva{&u_2-d+u_2+u_2+d=27\\&(u_2-d)^2+u_2^2+(u_2+d)^2=275}\Leftrightarrow \heva{&u_2=9\\&3u_2^2+2d^2=275.}$\\
Thay $ u_2=9 $ vào $ 3u_2^2+2d^2=275 $ ta được $ d=4 $ hay $ d=-4 $.\\
Vậy $ u_1=5 $, $ d=4 $ hoặc $ u_1=13 $, $ d=-4 $.}
\end{bt}
\subsubsection{Câu hỏi trắc nghiệm}
\Opensolutionfile{ans}[ans/ans-1K2-2-Dang1]

\begin{ex}%[DCHT Toán 11 - KNTT -Lê Hải Phụng] %[1K2Y6-1]
Trong các dãy số sau, dãy số nào là một cấp số cộng?
\choice
{\True $ 1 $; $ -3 $; $ -7 $; $ -11 $; $ -15 $; $ \ldots $}
{$ 1 $; $ -3 $; $ -6 $; $ -9 $; $ -12 $; $ \ldots $}
{$ 1 $; $ -2 $; $ -4 $; $ -6 $; $ -8 $; $ \ldots $}
{$ 1 $; $ -3 $; $ -5 $; $ -7 $; $ -9 $; $ \ldots $}
\loigiai
{
	Ta lần lượt tính khoảng cách $ d $ các phần tử, ta thấy dãy số đáp án A có $ d= -4$.
}
\end{ex}
%Cau2
\begin{ex}%[DCHT Toán 11 - KNTT -Lê Hải Phụng] %[1K2Y6-1]
Dãy số nào sau đây \textbf{không} phải là cấp số cộng?
\choice
{$ -\dfrac{2}{3} $; $ -\dfrac{1}{3} $; $ 0 $; $ \dfrac{1}{3} $; $ \dfrac{2}{3} $; $ 1 $; $ \dfrac{4}{3} $}
{$ 15\sqrt{2} $; $ 12\sqrt{2} $; $ 9\sqrt{2} $; $ 6\sqrt{2} $}
{\True $ \dfrac{4}{5} $; $ 1 $; $ \dfrac{7}{5} $; $ \dfrac{9}{5} $; $ \dfrac{11}{5} $}
{$ \dfrac{1}{\sqrt{3}} $; $ \dfrac{2\sqrt{3}}{3} $; $ \sqrt{3} $; $ \dfrac{4\sqrt{3}}{3} $; $ \dfrac{5}{\sqrt{3}} $}
\loigiai
{
	Ta lần lượt tính khoảng cách $ d $ các phần tử, ta thấy dãy số trừ đáp án C có khoảng cách các phần tử không bằng nhau.
}
\end{ex}
%Cau3
\begin{ex}%[DCHT Toán 11 - KNTT -Lê Hải Phụng] %[1K2Y6-1]
Cho cấp số cộng $ (u_n) $ với $ u_1=2 $ và $ u_2=6 $. Công sai của cấp số cộng đã cho là	
\choice
{\True $ 4 $}
{$ -4 $}
{$ 8 $}
{$ 3 $}
\loigiai
{
	Ta có $ u_2=6 \Leftrightarrow 6=u_1+d \Leftrightarrow d=4 $.
}
\end{ex}
%Cau4
\begin{ex}%[DCHT Toán 11 - KNTT -Lê Hải Phụng] %[1K2Y6-1]
Cho cấp số cộng $ (u_n) $ với $ u_1=-3 $ và $ u_6=27 $. Công sai $ d $ của cấp số cộng đã cho là	
\choice
{$ d=7 $}
{$ d=5 $}
{$ d=8 $}
{\True $ d=6 $}
\loigiai
{
	Ta có $ u_6=27 \Leftrightarrow 27=u_1+5d \Leftrightarrow d=6 $.
}
\end{ex}
%Cau5
\begin{ex}%[DCHT Toán 11 - KNTT -Lê Hải Phụng] %[1K2B6-1]
Cho cấp số cộng $ (u_n) $ với $ u_{17}=33 $ và $ u_{33}=65 $. Công sai của cấp số cộng đã cho là	
\choice
{$ 1 $}
{$ 3 $}
{$ -2 $}
{\True $ 2 $}
\loigiai
{
	Gọi $ u_1 $, $ d $ lần lượt là số hạng đầu và công sai của cấp số cộng $ (u_n) $.\\
	Khi đó, ta có $ u_{17}=u_1+16d $, $ u_{33}=u_1+32d $\\
	Suy ra $ u_{33}-u_{17}=65-33 \Leftrightarrow 16d=32 \Leftrightarrow d=2 $\\
	Vậy công sai bằng $ 2 $.
}
\end{ex}
%Cau6
\begin{ex}%[DCHT Toán 11 - KNTT -Tên GV] %[1K2B6-1]
Cho cấp số cộng có $ u_1=-3 $ và $ d=4 $. Chọn khẳng định đúng trong các khẳng định sau.
\choice
{$ u_5=15 $}
{$ u_4=8 $}
{\True $ u_3=5 $}
{$ u_2=2 $}
\loigiai
{
	Ta có $ u_3=u_1+2d=-3+2\cdot4=5 $.
}
\end{ex}
%Cau7
\begin{ex}%[DCHT Toán 11 - KNTT -Tên GV] %[1K2Y6-1]
Cho cấp số cộng có $ u_1=11 $ và công sai $ d=4 $. Hãy tính $ u_{99} $.
\choice
{$ 401 $}
{\True $ 403 $}
{$ 402 $}
{$ 404 $}
\loigiai
{
	Ta có $ u_{99}=u_1+98d=11+98\cdot4=403 $.
}
\end{ex}
%Cau8
\begin{ex}%[DCHT Toán 11 - KNTT -Tên GV] %[1K2B6-1]
Một cấp số cộng $ (u_n) $ có $ u_{13}=8 $ và $ d=-3 $. Tìm số hạng thứ ba của cấp số cộng $ (u_n) $.
\choice
{$ 50 $}
{$ 28 $}
{\True $ 38 $}
{$ 44 $}
\loigiai
{
	Ta có $ u_{13}=u_1+12d \Leftrightarrow 8=u_1+12\cdot(-3)\Rightarrow u_1=44 \Rightarrow u_{3}=u_1+2d=44-6=38$.
}
\end{ex}
%Cau9
\begin{ex}%[DCHT Toán 11 - KNTT -Tên GV] %[1K2Y6-1]
Cho cấp số cộng $(u_n) $ có số hạng đầu $ u_1=2 $ và công sai $ d=4 $. Hãy tính giá trị $ u_{2019} $ bằng
\choice
{\True $ 8074 $}
{$ 4074 $}
{$ 8078 $}
{$ 4078 $}
\loigiai
{
	Ta có $ u_{2019}=u_1+2018d=2+2018\cdot 4=8074 $.
}
\end{ex}
%Cau10
\begin{ex}%[DCHT Toán 11 - KNTT -Tên GV] %[1K2K6-1]
Cho cấp số cộng $ (u_n) $ có số hạng tổng quát là $ u_n=3n-2 $. Tìm công sai $ d $ của cấp số cộng.
\choice
{\True $ d=3 $}
{$ d=2 $}
{$ d=-2 $}
{$ d=-3 $}
\loigiai
{
	Ta có $ u_{n+1}-u_n=3(n+1)-2-3n+2=3 $. Suy ra công sai $ d=3 $.
}
\end{ex}

\begin{ex}%[Dự án DCHT-11-KNTT]%[Dao-V- Thuy]%[1K2Y5-1]
	Cho cấp số cộng $(u_n)$ có số hạng đầu $u_1$ và công sai $d$. Công thức tìm số hạng tổng quát $u_n$ là 
	\choice
	{\True $u_n=u_1+(n-1)d$}
	{$u_n=u_1+nd$}
	{$u_n=u_1+(n+1)d$}
	{$u_n=nu_1+d$}
	\loigiai{
		Ta có $u_n=u_1+(n-1)d$.
	}
\end{ex}

\begin{ex}%[Dự án DCHT-11-KNTT]%[Dao-V- Thuy]%[1K2Y5-1]
	Cho cấp số cộng $(u_n)$ có $u_1=-3$ và $d=\dfrac{1}{2}$. Khẳng định nào sau đây đúng?
	\choice
	{$u_n=-3+\dfrac{1}{2}(n+1 )$}
	{$u_n=-3+\dfrac{1}{2}n-1$}
	{\True $u_n=-3+\dfrac{1}{2}(n-1)$}
	{$u_n=-3+\dfrac{1}{4}(n-1 )$}
	\loigiai {
		Ta có $\heva{
			&u_1=-3 \\
			& d=\dfrac{1}{2} \\
		}\xrightarrow{CTTQ} u_n=u_1+(n-1 )d=-3+\dfrac{1}{2}(n-1 )$.}
\end{ex}

\begin{ex}%[Dự án DCHT-11-KNTT]%[Dao-V- Thuy]%[1K2Y5-1]
	Cho cấp số cộng $\left(u_n\right)$ xác định bởi $u_n=2n+1$. Xác định số hạng đầu $u_1$ và công sai $d$ của cấp số cộng.
	\choice
	{$u_1=3$, $d=1$}
	{$u_1=1$, $d=1$}
	{\True $u_1=3$, $d=2$}
	{$u_1=1$, $d=2$}
	\loigiai{
		Ta có $u_1=2\cdot 1+1=3$ và $u_2=2\cdot 2+1=5$, nên $d=u_2-u_1=2$.
	}
\end{ex}

\begin{ex}%[Dự án DCHT-11-KNTT]%[Dao-V- Thuy]%[1K2B5-1]
	Cho cấp số cộng $\left(u_n\right)$ có $u_4=-12$, $u_{14}=18$. Tìm số hạng đầu $u_1$ và công sai $d$ của cấp số cộng $\left(u_n\right)$. 
	\choice 
	{$u_1=-20$, $d=-3$}
	{$u_1=-22$, $d=3$ }
	{\True $u_1=-21$, $d=3$}
	{$u_1=-21$, $d=-3$}
	\loigiai{
		Ta có $$\heva{&u_4=u_1+(4-1)d\\&u_{14}=u_1+(14-1)d} \Leftrightarrow \heva{&-12=u_1+3d\\&18=u_1+13d}\Leftrightarrow \heva{&u_1=-12\\&d=3.}$$
	}
\end{ex}

\begin{ex}%[Dự án DCHT-11-KNTT]%[Dao-V- Thuy]%[1K2B5-1]
	Tìm số hạng đầu và công sai của cấp số cộng $(u_n)$ thỏa mãn $\heva{&u_1+u_9=12\\&u_4-3u_2=1.}$
	\choice
	{$u_1=\dfrac{1}{2}$; $d=\dfrac{13}{8}$}
	{$u_1=-1$; $d=\dfrac{13}{8}$}
	{\True $u_1=-\dfrac{1}{2}$; $d=\dfrac{13}{8}$}
	{$u_1=-1$; $d=2$}
	\loigiai{Ta có: $\heva{&u_1+u_9=12\\&u_4-3u_2=1}\Leftrightarrow\heva{&u_1+(u_1+8d)=12\\&(u_1+3d)-3(u_1+d)=1}\Leftrightarrow\heva{&2u_1+8d=12\\&-2u_1=1}\Leftrightarrow\heva{&d=\dfrac{13}{8}\\&u_1=-\dfrac{1}{2}}$}
\end{ex}

\begin{ex}%[Dự án DCHT-11-KNTT]%[Dao-V- Thuy]%[1K2B5-1]
	Cho cấp số cộng $(u_n)$ có $u_4=-12$ và $u_{14} =18$. Khi đó, số hạng đầu tiên $u_1$ và công sai $d$ của cấp số cộng $(u_n)$ lần lượt là
	\choice
	{$u_1=-20$, $d=-3$}
	{$u_1=-22$, $d=3$}
	{\True $u_1=-21$, $d=3$}
	{$u_1=-21$, $d=-3$}
	\loigiai{Ta có: $\heva{&u_4=-12\\&u_{14}=18}\Leftrightarrow\heva{&u_1+3d=-12\\&u_1+13d=18}\Leftrightarrow\heva{&u_1=-21\\&d=3.}$}
\end{ex}

\begin{ex}%[Dự án DCHT-11-KNTT]%[Dao-V- Thuy]%[1K2B5-1]
	Cho cấp số cộng $(u_n )$ có các số hạng đầu lần lượt là $5;\,9;\,13;\,17;\ldots $. Tìm số hạng tổng quát $u_n$ của cấp số cộng.
	\choice
	{$u_n=5n+1$}
	{$u_n=5n-1$}
	{\True $u_n=4n+1$}
	{$u_n=4n-1$}
	\loigiai{
		Cấp số cộng đã cho có $u_1=5$, $ d=u_2-u_1=4 $. Suy ra $u_n=u_1+(n-1 )d=5+4(n-1 )=4n+1$.
		}
\end{ex}

\begin{ex}%[Dự án DCHT-11-KNTT]%[Dao-V- Thuy]%[1K2B5-1]
	Cho cấp số cộng $(u_n)$ có $u_3=15$ và $d=-2$. Tìm $u_n$.
	\choice
	{\True $u_n=-2n+21$}
	{$u_n=-\dfrac{3}{2}n+12$}
	{$u_n=-3n-17$}
	{$u_n=\dfrac{3}{2}{{n}^2}-4$}
	\loigiai {
		Ta có $\heva{ & 15=u_3=u_1+2d \\& d=-2}
		\Leftrightarrow \heva{&u_1=19 \\& d=-2}
		\Rightarrow u_n=u_1+(n-1 )d=-2n+21$.
		}
\end{ex}

\begin{ex}%[Dự án DCHT-11-KNTT]%[Dao-V- Thuy]%[1K2B5-1]
	Trong các dãy số được cho dưới đây, dãy số nào {\bf không} phải là cấp số cộng?
	\choice
	{$u_n=-4n+9$}
	{$u_n=-2n+19$}
	{$u_n=-2n-21$}
	{\True $u_n=-2^n+15$}
	\loigiai {
		Dãy số $u_n=-2^n+15$ không có dạng $an+b$ nên có không phải là cấp số cộng.}
\end{ex}

\begin{ex}%[Dự án DCHT-11-KNTT]%[Dao-V- Thuy]%[1K2B5-1]
	Cho cấp số cộng $(u_n)$ có $u_4=-12$ và $u_{14}=18$. Tìm số hạng đầu tiên $u_1$ và công sai $d$ của cấp số cộng đã cho.
	\choice
	{\True $u_1=-21$; $d=3$}
	{$u_1=-20$; $d=-3$}
	{$u_1=-22$; $d=3$}
	{$u_1=-21$; $d=-3$}
	\loigiai {
		Ta có 
		$\heva{&u_4=-12\\ &u_{14}=18} \Leftrightarrow \heva{
			&u_1+3d=-12\\
			&u_1+13d=18 \\
		}\Leftrightarrow \heva{
			&u_1=-21 \\
			& d=3. \\
		}$}
\end{ex}

\begin{ex}%[Dự án DCHT-11-KNTT]%[Dao-V- Thuy]%[1K2K5-1]
	Cho cấp số cộng $(u_n)$ thoả mãn $\heva{&u_2-u_3+u_5=10\\ &u_3+u_4=17}$. Số hạng đầu tiên và công sai của cấp số cộng đó lần lượt là
	\choice
	{\True $1$ và $3$}
	{$-3$ và $4$}
	{$4$ và $-3$}
	{$-4$ và $-3$}
	\loigiai{
		$\heva{&u_2-u_3+u_5=10\\ &u_3+u_4=17}\Leftrightarrow\heva{&(u_1+d)-(u_1+2d)+(u_1+4d)=10\\&(u_1+2d)+(u_1+3d)=17}\Leftrightarrow\heva{&u_1+3d=10\\&2u_1+5d=17}\Leftrightarrow\heva{&u_1=1\\&d=3.}$}
\end{ex}

\begin{ex}%[Dự án DCHT-11-KNTT]%[Dao-V- Thuy]%[1K2K5-1]
	Cho cấp số cộng $(u_n)$ có công sai $d<0$, $u_{31}+u_{34}=11$ và $(u_{31})^2 + (u_{34})^2=101$. Số hạng tổng quát của $(u_n)$ là
	\choice
	{$u_{n}=86-3n$}
	{$u_{n}=92-3n$}
	{$u_{n}=95-3n$}
	{\True $u_{n}=103-3n$}
	\loigiai{Gọi cấp số cộng $(u_n)$ có công sai $d$.\\
		$(u_{31})^2 + (u_{34})^2=101 \Leftrightarrow \left( {u_{31}+u_{34}}\right)^2-2u_{31}.u_{34}=101$ $\Rightarrow u_{31}.u_{34}=10$.\\
		Do đó, ta có $\heva{&u_{31}+u_{34}=11\\ &u_{31}.u_{34}=10}$ $\Rightarrow \heva{&u_{31}=10 \\ &u_{34}=1}$(vì $d<0$)\\
		$u_{31}+u_{34}=11 \Rightarrow 2u_{31}+3d =11 \Rightarrow d=-3 \,\,\text{và}\,\, u_{1}=100$.\\
		Do đó: $u_{n}=103-3n$.}
\end{ex}
\Closesolutionfile{ans}
% \begin{indapan}{10}
% 	{ans/ans-1K2-2-Dang2}
% \end{indapan}
\begin{dang}{Tổng của $n$ số hạng đầu tiên của một cấp số cộng. Tính chất của cấp số cộng}
	Tổng của $n$ số hạng đầu tiên:	Đặt ${{S}_{n}}={{u}_{1}}+{{u}_{2}}+{{u}_{3}}+\cdots+{{u}_{n}}.$ Khi đó
	\begin{itemize}
		\item [$\bullet$] ${{S}_{n}}=\dfrac{n\left( {{u}_{1}}+{{u}_{n}} \right)}{2}=\dfrac{n\left( {{u}_{2}}+{{u}_{n-1}} \right)}{2}=\dfrac{n\left( {{u}_{3}}+{{u}_{n-2}} \right)}{2}=\cdots$
		\item [$\bullet$] Vì ${{u}_{n}}={{u}_{1}}+\left( n-1 \right)d$ nên công thức trên có thể viết lại là \fbox{${{S}_{n}}=\dfrac{n}{2}\left[2u_1 + \left(n-1\right)d \right]  .$}
	\end{itemize}
	Tính chất của cấp số cộng:
	\begin{itemize}
		\item [\ding{172}] Nếu $a$; $b$; $c$ theo thứ tự lập thành cấp số cộng thì $a+c=2b$.
		\item [\ding{173}] Lưu ý:
		\begin{itemize}
			\item [$\bullet$] Nếu cho ba số liên tiếp của một cấp số cộng, ta có thể xem ba số đó là $$a-d;\quad a; \quad a+d$$
			\item [$\bullet$] Nếu cho bốn số liên tiếp của một cấp số cộng, ta có thể xem ba số đó là $$a-3d;\quad a-d; \quad a+d; \quad a+3d.$$
		\end{itemize}
	\end{itemize}
\end{dang}
\viduminhhoa
\begin{vd}
	Cho một cấp số cộng $(u_n)$ có $u_3 + u_{28} = 100$. Hãy tính tổng của $30$ số hạng đầu tiên của cấp số cộng đó.\dapso{$1500$}
	\loigiai{Ta có $S_{30} = \dfrac{30(u_1 + u_{30})}{2} = \dfrac{30(u_1 + 2d + u_{30} - 2d)}{2} = \dfrac{30(u_3 + u_{28})}{2} = \dfrac{30 \cdot 100}{2} = 1500$.}
\end{vd}\dongcham{7}

\begin{vd}
	Cho một cấp số cộng $(u_n)$ có $S_6 = 18$ và $S_{10} = 110$. Tính $S_{20}$.	\dapso{$ 620 $.}
	\loigiai{
		Giả sử cấp số cộng $(u_n)$ có số hạng đầu là $u_1$ và công sai là $d$.\\
		Ta có $S_6 = 6u_1 + \dfrac{6 \cdot 5}{2}d \Leftrightarrow 6u_1 + 15d = 18$. \quad (1)\\
		$S_{10} = 10u_1 + \dfrac{10 \cdot 9}{2}d \Leftrightarrow 10u_1 + 45d = 110$. \quad (2)\\
		Từ (1) và (2), ta có hệ phương trình $\heva{&6u_1 + 15d = 18\\&10u_1 + 45d = 110} \Leftrightarrow \heva{&u_1 = -7\\&d = 4.}$\\
		Khi đó $S_{20} = 20u_1 + \dfrac{20 \cdot 19}{2}d = 20 \cdot (-7) + 190 \cdot 4 = 620$.
	}
\end{vd}\dongcham{8}


\begin{vd}
	Tìm số hạng đầu và công sai của cấp số cộng, biết
	\begin{tasks}(2)
		\task $\heva{&u_1^2 + u_2^2 + u_3^2 = 155\\&S_3 = 21.}$	\dapso{$u_1 = 9$, $d = -2$ hoặc $u_1 = 5$, $d = 2$.}
		\task $\heva{&S_3 = 12\\&S_5 = 35.}$	\dapso{$u_1 = 1$, $d = 3$.}
	\end{tasks}
	\loigiai{
		\begin{listEX}
			\item $\heva{&u_1^2 + u_2^2 + u_3^2 = 155\\&S_3 = 21} \Leftrightarrow \heva{&u_1^2 + (u_1 + d)^2 + (u_1 + 2d)^2 = 155 &(1)\\&3u_1 + 3d = 21.&(2)}$\\
			Từ (2), ta có $3u_1 + 3d = 21 \Rightarrow d = 7 - u_1$, thay vào (1)
			$$u_1^2 + 7^2 + (14 - u_1)^2 = 155 \Leftrightarrow 2u_1^2 - 28u_1 + 90 = 0 \Leftrightarrow \hoac{&u_1 = 9\\&u_1 = 5.}$$
			Với $u_1 = 9$ thì $d = -2$. Với $u_1 = 5$ thì $d = 2$.\\
			Vậy số hạng đầu của cấp số cộng là $u_1 = 9$, công sai là $d = -2$ hoặc $u_1 = 5$, $d = 2$.
			\item $\heva{&S_3 = 12\\&S_5 = 35} \Leftrightarrow \heva{&3u_1 + 3d = 12\\&5u_1 + 10d = 35} \Leftrightarrow \heva{&u_1 = 1\\&d = 3.}$\\
			Vậy số hạng đầu của cấp số cộng là $u_1 = 1$, công sai là $d = 3$.
	\end{listEX}}
\end{vd}\dongcham{12}

\begin{vd}
	Tìm số hạng tổng quát của cấp số cộng, biết 
	$\heva{&S_4 = 20\\&\dfrac{1}{u_1} + \dfrac{1}{u_2} + \dfrac{1}{u_3} + \dfrac{1}{u_4} = \dfrac{25}{24}}$ và cấp số cộng có công sai là một số nguyên âm.	\dapso{$ u_n=10-2n $.}
	\loigiai{
		$\heva{&S_4 = 20 &(1)\\&\dfrac{1}{u_1} + \dfrac{1}{u_2} + \dfrac{1}{u_3} + \dfrac{1}{u_4} = \dfrac{25}{24}&(2).}$\\
		Từ (1), suy ra $u_1 + u_4 = u_2 + u_3 = 10$ và $u_1 = 5 - \dfrac{3}{2}d$.\\
		Từ (2), ta có 
		\begin{eqnarray*}
			& &\dfrac{u_1 + u_4}{u_1 \cdot u_4} + \dfrac{u_2 + u_3}{u_2 \cdot u_3} = \dfrac{25}{24} \Leftrightarrow \dfrac{10}{u_1(u_1 + 3d)} + \dfrac{10}{(u_1 + d)(u_1 + 2d)} = \dfrac{25}{24}\\
			&\Leftrightarrow & \dfrac{10}{\left(5 - \dfrac{3}{2}d\right)\left(5 + \dfrac{3}{2}d\right)} + \dfrac{10}{\left(5 - \dfrac{1}{2}d\right)\left(5 + \dfrac{1}{2}d\right)} = \dfrac{25}{24} \Leftrightarrow \dfrac{10}{25 - \dfrac{9}{4}d^2} + \dfrac{10}{25 - \dfrac{1}{4}d^2} = \dfrac{25}{24}\\
			&\Leftrightarrow & 10\left(25 - \dfrac{9}{4}d^2 + 25 - \dfrac{1}{4}d^2\right) = \dfrac{25}{24}\left(25 - \dfrac{9}{4}d^2\right)\left(25 - \dfrac{1}{4}d^2\right)\\
			&\Leftrightarrow & \dfrac{75}{128}d^4 - \dfrac{1925}{48}d^2 + \dfrac{3625}{24} = 0 \Leftrightarrow \hoac{&d^2 = \dfrac{580}{9}\\&d^2 = 4} \Leftrightarrow \hoac{&d = \pm \dfrac{2\sqrt{145}}{3}\\&d = \pm 2.}
		\end{eqnarray*}
		Với $d = -2$ thì $u_1 = 8$. Suy ra $u_n=u_1+(n-1)d=10-2n$}
\end{vd}\dongcham{18}

\begin{vd}
	Tính các tổng sau
	\begin{tasks}(2)
		\task $S = 1 + 3 + 5 + \cdots + (2n - 1) + (2n + 1)$.\dapso{$S = (n + 1)^2$}
		\task $S = 100^2 - 99^2 + 98^2 - 97^2 + \cdots + 2^2 - 1^2$.\dapso{$S = 5050$}
	\end{tasks}
	\loigiai{
		\begin{enumEX}{1}
			\item $S = 1 + 3 + 5 + \cdots + (2n - 1) + (2n + 1)$.\\
			Xét cấp số cộng $(u_k)$, $k \in \mathbb{N}^*$ với số hạng đầu là $u_1 = 1$ và công sai là $d = 2$.\\
			Ta có $u_k = u_1 + (k - 1)d \Leftrightarrow 2n + 1 = 1 + 2(k - 1) \Leftrightarrow k = n + 1$.\\
			Vậy $S = \dfrac{k(u_1 + u_k)}{2} = \dfrac{(n + 1)(1 + 2n + 1)}{2} = (n + 1)^2$.
			\item $S = 100^2 - 99^2 + 98^2 - 97^2 + \cdots + 2^2 - 1^2 = 199 + 195 + \cdots + 3$.\\
			Xét cấp số cộng $(u_n)$ có số hạng đầu $u_1 = 199$ và công sai $d = u_2 - u_1 = 195 - 199 = -4$.\\
			Ta có $u_n = u_1 + (n - 1)d \Leftrightarrow 3 = 199 - 4(n - 1) \Leftrightarrow n = 50$.\\
			Khi đó $S = \dfrac{n(u_1 + u_{50})}{2} = \dfrac{50(199 + 3)}{2} = 5050$.
			
		\end{enumEX}
	}
\end{vd}\dongcham{18}

\begin{vd}
	Tìm ba số hạng liên tiếp của một cấp số cộng biết tổng của chúng bằng $27$ và tổng các bình phương của chúng là $293$.\dapso{$4$, $9$, $14$}
	\loigiai{
		Gọi ba số hạng liên tiếp của cấp số cộng là $x - d$, $x$, $x + d$ trong đó $d$ là công sai của cấp số cộng.\\
		Khi đó ta có $x - d + x + x + d = 27 \Leftrightarrow 3x = 27 \Leftrightarrow x = 9$.\\
		Mà $(x - d)^2 + x^2 + (x + d)^2 = 293 \Leftrightarrow (9 - d)^2 + 81 + (9 + d)^2 = 293 \Leftrightarrow 2d^2 -50 = 0 \Leftrightarrow \hoac{&d = 5\\&d = -5.}$\\	
		Với $d = 5$ thì ba số hạng của cấp số cộng là $4$, $9$, $14$.\\
		Với $d = -5$ thì ba số hạng của cấp số cộng là $14$, $9$, $4$.\\
		Vậy ba số hạng liên tiếp của cấp số cộng là $4$, $9$, $14$.
	}
\end{vd}\dongcham{14}

\begin{vd}
	Tìm bốn số hạng liên tiếp của một cấp số cộng, biết tổng của chúng bằng $10$ và tổng bình phương của chúng bằng $30$.\dapso{$1$, $2$, $3$, $4$}
	\loigiai{
		Gọi bốn số hạng liên tiếp của cấp số cộng là $x - 3d$, $x - d$, $x 
		+ d$, $x + 3d$ với $2d$ là công sai của cấp số cộng.\\
		Khi đó ta có $x - 3d + x - d + x + d + x + 3d = 10 \Leftrightarrow 4x = 10 \Leftrightarrow x = \dfrac{5}{2}$.\\
		Mặt khác $$(x - 3d)^2 + (x - d)^2 + (x + d)^2 + (x + 3d)^2 = 30 \Leftrightarrow 4x^2 + 20d^2 = 30 \Leftrightarrow d^2 = \dfrac{1}{4} \Leftrightarrow \hoac{&d = \dfrac{1}{2}\\&d = -\dfrac{1}{2}.}$$
		Với $x = \dfrac{5}{2}$ thì $d = \dfrac{1}{2}$, khi đó bốn số hạng liên tiếp của cấp số cộng là $1$, $2$, $3$, $4$.\\
		Với $x = \dfrac{5}{2}$ thì $d = -\dfrac{1}{2}$, khi đó bốn số hạng liên tiếp của cấp số cộng là $4$, $3$, $2$, $1$.\\
		Vậy bốn số hạng liên tiếp của cấp số cộng là $1$, $2$, $3$, $4$.
	}
\end{vd}\dongcham{14}

\begin{vd}
	Ba góc của một tam giác vuông lập thành một cấp số cộng. Tìm ba góc đó.
	\loigiai{Gọi ba góc của tam giác lần lượt là $A$, $B$, $C$.
		Khi đó ta có $A + B + C = 180^\circ$.\\
		Do ba góc $A$, $B$, $C$ của tam giác theo thứ tự lập thành một cấp số cộng nên $B-A=C-A \Leftrightarrow A + C = 2B$.\\
		Do đó $2B + B = 180^\circ \Rightarrow 3B = 180^\circ \Rightarrow B = 60^\circ$.\\
		Do tam giác $ABC$ vuông nên giả sử $C = 90^\circ$ khi đó công sai $d$ của cấp số cộng là $d = C - B = 30^\circ$.\\
		Vậy góc $A$ của tam giác là $A = 30^\circ$.}
\end{vd}\dongcham{10}

% \begin{vd}
% 	Cho $a$, $b$, $c$ là ba số hạng liên tiếp của một cấp số cộng. Chứng minh rằng
% 	\begin{tasks}(1)
% 		\task $a^2 + 2bc = c^2 + 2ab$.
% 		\task $2(a+b+c)^3 = 9 \left[ a^2(b+c) + b^2(a+c) + c^2(a+b) \right]$.
% 		\task  $b^2 + bc +c^2$, $a^2 + ac + c^2$, $a^2 + ab + b^2$ cũng là một cấp số cộng.
% 	\end{tasks}
% 	\loigiai{
% 		\begin{enumerate}[a)]
% 			\item Vì $a$, $b$, $c$ là ba số liên tiếp của một cấp số cộng nên $a + c = 2b \Rightarrow a = 2b -c$.\\
% 			Do đó
% 			$$a^2 +2bc = (2b-c)^2 + 2bc = 4b^2 - 2bc + c^2 = 2b(2b -c) + c^2 = 2ba + c^2 = c^2 + 2ab.$$
% 			Vậy $a^2 + 2bc = c^2 + 2ab$ (đpcm).
			
% 			\item Vì $a$, $b$, $c$ là ba số liên tiếp của một cấp số cộng nên $a + c = 2b \Rightarrow a = 2b -c$.\\
% 			Do đó
% 			\allowdisplaybreaks
% 			\begin{eqnarray*}
% 				& \mbox{VT}  & = 2(a+b+c)^3 = 2(3b)^3 = 54b^3\\
% 				& \mbox{VP}  & = 9\left[ a^2(b+c) + b^2(a+c) + c^2(a+b) \right] \\
% 				& & = 9\left[ (2b-c)^2(b+c) + b^2(2b-c+c) + c^2(2b-c+b) \right] \\ 
% 				& & = 9\left[ (4b^2 - 4bc+c^2)(b+c) + b^2(2b) + c^2(3b-c) \right] \\
% 				& & = 9\left[ 4b^3 - 4b^2c +bc^2 + 4b^2c - 4bc^2 + c^3 + 2b^3 + 3bc^2 - c^3 \right] \\
% 				& & = 9\cdot (6b^3) = 54b^3 = \mbox{ VT }.
% 			\end{eqnarray*}
% 			Vậy $2(a+b+c)^3 = 9 \left[ a^2(b+c) + b^2(a+c) + c^2(a+b) \right]$ (đpcm).
			
% 			\item Vì ba số $a$, $b$, $c$ theo thứ tự lập thành một cấp số cộng thì $a + c = 2b \Rightarrow a = 2b - c$.\\
% 			Xét
% 			\allowdisplaybreaks
% 			\begin{eqnarray*}
% 				& 2(a^2 + ac + c^2) - (a^2 + ab + b^2) & = a^2 + a(2c-b) + 2c^2 - b^2 \\
% 				& & = (2b-c)^2 + (2b-c)(2c-b) + 2c^2 - b^2 \\
% 				& & = b^2 + bc  +c^2\\
% 				&\Rightarrow (b^2 + bc  +c^2) + (a^2 + ab + b^2) &= 2(a^2 + ac + c^2).
% 			\end{eqnarray*}
% 			Vậy ba số: $b^2 + bc +c^2$, $a^2 + ac + c^2$, $a^2 + ab + b^2$ cũng là một cấp số cộng.
% 		\end{enumerate}
% 	}
% \end{vd}\dongcham{25}

\begin{vd}%[TH]%[Dự án DCHT-11-KNTT]%[Dao-V- Thuy]%[1K2B5-2]
	Xác định $4$ góc của một tứ giác lồi, biết rằng $4$ góc hợp thành cấp số cộng và góc lớn nhất bằng $5$ lần góc nhỏ nhất.
	\dapso{$36^\circ; \, 72^\circ; \, 108^\circ; \, 144^\circ$}
	\loigiai
	{
		Gọi số đo bốn góc cần tìm là $u_1$, $u_2$, $u_3$, $u_4$. Ta có
		\begin{eqnarray*}
			\heva{&u_1+u_2+u_3+u_4=360\\ &u_5=5u_1} \Leftrightarrow \heva{&4u_1+6d=360\\ &4d=4u_1} \Leftrightarrow \heva{&u_1=36\\ &d=36.}
		\end{eqnarray*}
		Vậy số đo bốn góc cần tìm là
		\[
		36^\circ; \, 72^\circ; \, 108^\circ; \, 144^\circ.
		\]
	}
\end{vd}

\subsubsection{Bài tập tự luận}
 

\begin{bt}%[TH]%[Dự án DCHT-11-KNTT]%[Dao-V- Thuy]%[1K2B5-2]
	Giữa các số $10$ và $64$ hãy đặt thêm $17$ số nữa để được một cấp số cộng.
	\dapso{$13; 16; 19; 22; 25; 28; 31; 34; 37; 40; 43; 46; 49; 52; 55; 58; 61$}
	\loigiai{
		Ta có
		\begin{equation*}
			\heva{&u_1=10\\ &u_{19}=64} \Leftrightarrow \heva{&u_1=10\\ &u_1+18d=64} \Leftrightarrow \heva{&u_1=10\\ &d=3.}
		\end{equation*}
		Vậy $17$ số đặt thêm giữa các số $10$ và $64$ để được một cấp số cộng là
		\begin{center}
			13; 16; 19; 22; 25; 28; 31; 34; 37; 40; 43; 46; 49; 52; 55; 58; 61.
		\end{center} 
	}
\end{bt}

\begin{bt}%[TH]%[Dự án DCHT-11-KNTT]%[Dao-V- Thuy]%[1K2B5-2]
	Tổng ba số hạng liên tiếp của một cấp số cộng bằng $2$ và tổng các bình phương của ba số đó bằng $\dfrac{14}{9}$. Xác định ba số đó và tính công sai của cấp số cộng.
	\dapso{$1;\dfrac{2}{3};\dfrac{1}{3}$ ứng với $d=-\dfrac{1}{3}$ hoặc $\dfrac{1}{3};\dfrac{2}{3};1$ ứng với $d=\dfrac{1}{3}$}
	\loigiai{
		Ta có hệ
		\begin{align*}
			&\quad \heva{&u_k+u_{k+1}+u_{k+2}=2\\ &u^2_k+u^2_{k+1}+u^2_{k+2}=\dfrac{14}{9}}
			\Leftrightarrow \heva{&u_k+u_k+d+u_k+2d=2\\ &u^2_k+\left(u_k+d\right)^2 +\left(u_k+2d\right)^2=\dfrac{14}{9}} \\
			&\Leftrightarrow \heva{&3u_k+3d=2\\ &3u^2_k+6u_kd+5d^2=\dfrac{14}{9}}
			\Leftrightarrow \heva{&u_k=1\\ &d=-\dfrac{1}{3}} \text{ hoặc } \heva{&u_k=\dfrac{1}{3}\\ &d=\dfrac{1}{3}.}
		\end{align*}
		Vậy ba số hạng liên tiếp của cấp số cộng thỏa yêu cầu bài toán $1;\dfrac{2}{3};\dfrac{1}{3}$ ứng với $d=-\dfrac{1}{3}$ hoặc $\dfrac{1}{3};\dfrac{2}{3};1$ ứng với $d=\dfrac{1}{3}.$
	}
\end{bt}

\begin{bt}%[TH]%[Dự án DCHT-11-KNTT]%[Dao-V- Thuy]%[1K2B5-2]
	Một cấp số cộng có $7$ số hạng với công sai $d$ dương và số hạng thứ tư bằng $11$. Hãy tìm các số hạng còn lại của cấp số cộng đó, biết hiệu của số hạng thứ ba và số hạng thứ năm bằng $6$.
	\dapso{$u_1=2$; $ u_2=5$; $u_4=11$; $u_6=17$; $u_7=20$}
	\loigiai{
		Gọi số hạng đầu của cấp số cộng là $u_1$, công sai $d$.
		Vì số hạng thứ tư của cấp số cộng bằng $11$ nên ta có $u_4=11$.\\
		Do $d$ dương nên $ u_5>u_3$.\\
		Vì hiệu của số hạng thứ ba và số hạng thứ năm bằng $6$ nên ta có $ u_5-u_3=6$.\\
		Ta có \begin{align*}
			\heva{&u_4=11\\&u_5-u_3=6 }
			\Leftrightarrow \heva{&u_1+3d=11\\&(u_1+4d)-(u_1+2d)=6 }
			\Leftrightarrow \heva{&u_1+3\cdot 3=11\\&d=3 }
			\Leftrightarrow \heva{&u_1=2\\&d=3.}
		\end{align*}
		Vậy các số  hạng còn lại của cấp số cộng là $u_1=2$; $ u_2=5$; $u_4=11$; $u_6=17$; $u_7=20$.
	}
\end{bt}

\begin{bt}%[VD]%[Dự án DCHT-11-KNTT]%[Dao-V- Thuy]%[1K2K5-2]
	Tìm bốn số hạng liên tiếp của một cấp số cộng, biết rằng:
	\begin{enumerate}
		\item Tổng của chúng bằng $10$ và tổng bình phương bằng $70$.
		\item Tổng của chúng bằng $22$ và tổng bình phương bằng $66$.
		\item  Tổng của chúng bằng $36$ và tổng bình phương bằng $504$.
		\item  Chúng có tổng bằng $20$ và tích của chúng bằng $384$.
		\item  Tổng của chúng bằng $ 20$, tổng nghịch đảo của chúng bằng $ \dfrac{25}{24}$ và các số này là những số nguyên.
		\item  Nó là số đo của một tứ giác lồi và góc lớn nhất gấp $5$ lần góc nhỏ nhất.
	\end{enumerate}
	\dapso{$-2$; $ 1$; $ 4$; $7$.} 
	\dapso{không tồn tại bốn số hạng liên tiếp của cấp số cộng thỏa mãn yêu cầu đề bài. $0$; $ 6$; $ 12$; $18$.}
	\dapso{$2$; $ 4$; $ 6$; $8$ hoặc $5-\sqrt{241}$; $ \dfrac{15-\sqrt{241}}{3}$; $ \dfrac{15+\sqrt{241}}{3}$; $5+\sqrt{241}$.} 
	\dapso{$30^\circ$; $70^\circ$; $ 110^\circ$; $150^\circ$.}
	\loigiai{
		\begin{enumerate}
			\item Gọi bốn số hạng liên tiếp của cấp số cộng là $x-3d$; $x-d$; $x+d$, $x+3d$ trong đó $2d$ là công sai.\\
			Theo đề bài ta có 
			\begin{align*}
				&\quad \heva{& (x-3d)+(x-d)+(x+d)+(x+3d)=10\\& (x-3d)^2+(x-d)^2+(x+d)^2+(x+3d)^2=70}
				\Leftrightarrow \heva{& 4x=10 \\& 4x^2+20d^2=70}\\
				&\Leftrightarrow  \heva{&x=\dfrac{5}{2}\\& 4\cdot \left( \dfrac{5}{2}\right) ^2+20d^2=70}
				\Leftrightarrow  \heva{&x=\dfrac{5}{2}\\& d^2=\dfrac{9}{4}}
				\Leftrightarrow  \heva{&x=\dfrac{5}{2}\\& d=\pm \dfrac{3}{2}.}
			\end{align*}
			Vậy bốn số hạng liên tiếp của cấp số cộng là $-2$; $ 1$; $ 4$; $7$.
			\item Gọi bốn số hạng liên tiếp của cấp số cộng là $x-3d$; $x-d$; $x+d$; $x+3d$ trong đó $2d$ là công sai.\\
			Theo đề bài ta có 
			\begin{align*}
				&\quad \heva{& (x-3d)+(x-d)+(x+d)+(x+3d)=22\\& (x-3d)^2+(x-d)^2+(x+d)^2+(x+3d)^2=66}
				\Leftrightarrow \heva{& 4x=22 \\& 4x^2+20d^2=66}\\
				&\Leftrightarrow  \heva{&x=\dfrac{11}{2}\\& 4\cdot \left( \dfrac{11}{2}\right) ^2+20d^2=66}
				\Leftrightarrow  \heva{&x=\dfrac{11}{2}\\& d^2=\dfrac{-11}{4} 
					\ (\text{loại}).}\\
			\end{align*}
			Vậy không tồn tại bốn số hạng liên tiếp của cấp số cộng thỏa mãn yêu cầu đề bài.
			\item Gọi bốn số hạng liên tiếp của cấp số cộng là $x-3d$; $x-d$; $x+d$; $x+3d$ trong đó $2d$ là công sai.\\
			Theo đề bài ta có 
			\begin{align*}
				&\quad \heva{& (x-3d)+(x-d)+(x+d)+(x+3d)=36\\& (x-3d)^2+(x-d)^2+(x+d)^2+(x+3d)^2=504}
				\Leftrightarrow \heva{& 4x=36 \\& 4x^2+20d^2=504}\\
				&\Leftrightarrow  \heva{&x=9\\& 4\cdot 9^2+20d^2=504}
				\Leftrightarrow  \heva{&x=9\\& d^2=9}
				\Leftrightarrow  \heva{&x=9\\& d=\pm 3.}
			\end{align*}
			Vậy  bốn  số hạng liên tiếp của cấp số cộng là $0$; $ 6$; $ 12$; $18$.
			\item Gọi bốn số hạng liên tiếp của cấp số cộng là $x-3d$; $x-d$; $x+d$; $x+3d$ trong đó $2d$ là công sai.\\
			Theo đề bài ta có 
			\begin{align*}
				&\quad \heva{& (x-3d)+(x-d)+(x+d)+(x+3d)=20\\& (x-3d)(x-d)(x+d)(x+3d)=384}
				\Leftrightarrow  \heva{&x=5\\& (x^2-d^2)(x^2-9d^2)=384}\\
				&\Leftrightarrow  \heva{&x=5\\& (25-d^2)(25-9d^2)=384}
				\Leftrightarrow  \heva{&x=5\\& 9d^4-250d^2+241=0}
				\Leftrightarrow  \heva{&x=5\\& \hoac{&d^2=1\\& d^2=\dfrac{241}{9}}}
				\Leftrightarrow  \heva{&x=5\\& \hoac{&d=\pm 1\\& d=\pm \dfrac{\sqrt{241}}{3}.}}
			\end{align*}
			Vậy bốn số hạng liên tiếp của cấp số cộng là $2$; $ 4$; $ 6$; $8$ hoặc $5-\sqrt{241}$; $ \dfrac{15-\sqrt{241}}{3}$; $ \dfrac{15+\sqrt{241}}{3}$; $5+\sqrt{241}$.
			\item Gọi bốn số hạng liên tiếp của cấp số cộng là $x-3d$; $x-d$; $x+d$; $x+3d$ trong đó $2d$ là công sai trong đó $ 2d \in \mathbb{Z}$.\\
			Theo đề bài ta có 
			\begin{align*}
				&\quad \heva{& (x-3d)+(x-d)+(x+d)+(x+3d)=20\\& \dfrac{1}{x-3d}+\dfrac{1}{x-d}+\dfrac{1}{x+d}+\dfrac{1}{x+3d}=\dfrac{25}{24}}
				\Leftrightarrow \heva{& 4x=20 \\& \dfrac{1}{5-3d}+\dfrac{1}{5-d}+\dfrac{1}{5+d}+\dfrac{1}{5+3d}=\dfrac{25}{24}}\\
				&\Leftrightarrow  \heva{&x=5\\& \dfrac{10}{25-9d^2}+\dfrac{10}{25-d^2}=\dfrac{25}{24}}
				\Leftrightarrow  \heva{&x=5\\& 9d^4-250d^2+241=0}\\
				&\Leftrightarrow  \heva{&x=5\\& \hoac{&d^2=1 \\& d^2=\dfrac{241}{9} }}
				\Leftrightarrow  \heva{&x=5\\& \hoac{&d= \pm 1 \ (\text{thỏa mãn})\\& d= \pm \dfrac{\sqrt{241}}{3} \,\,(\text{loại vì} \,  2d \in \mathbb{Z}).}}
			\end{align*}
			Vậy bốn  số hạng  nguyên liên tiếp của cấp số cộng là $2$; $ 4$; $ 6$; $8$.
			\item Gọi bốn số hạng liên tiếp của cấp số cộng xếp theo thứ tự tăng dần  là $x-3d$; $x-d$; $x+d$; $x+3d$ trong đó $2d>0$ là công sai.\\
			Theo đề bài ta có 
			\begin{align*}
				&\quad \heva{& (x-3d)+(x-d)+(x+d)+(x+3d)=360^\circ\\& x+3d=5(x-3d)}\\
				&\Leftrightarrow \heva{& 4x=360^\circ\\& 4x=18d}
				\Leftrightarrow  \heva{&x=90^\circ\\& 4 \cdot 90^\circ=18d}
				\Leftrightarrow  \heva{&x=90^\circ\\& d=20^\circ.}
			\end{align*}
			Vậy bốn  góc của tứ giác lồi lần lượt là  $30^\circ$; $70^\circ$; $ 110^\circ$; $150^\circ$.
		\end{enumerate}
	}
\end{bt}
% \subsubsection{Câu hỏi trắc nghiệm}
% \Opensolutionfile{ans}[ans/ans-1K2-2-Dang3]
% \begin{ex}%[Dự án DCHT-11-KNTT]%[Dao-V- Thuy]%[1K2Y5-2]
% 	Cấp số cộng $(u_n)$ có số hạng đầu $u_1=-5$ và công sai $d=3$. Tính $u_{15}$.
% 	\choice
% 	{$u_{15}=27$}
% 	{\True $u_{15}=37$}
% 	{$u_{15}=47$}
% 	{$u_{15}=57$}
% 	\loigiai{$u_{15}=u_1+14d=-5+14\times 3=37$.}
% \end{ex}

% \begin{ex}%[Dự án DCHT-11-KNTT]%[Dao-V- Thuy]%[1K2Y5-2]
% 	Cho cấp số cộng có các số hạng ban đầu là $1$; $5$; $9$; $13$; $\cdots$. Số hạng thứ $ 6 $ của cấp số cộng này là bao nhiêu?
% 	\choice{\True $ 21$}
% 	{$19 $}
% 	{$ 22$}
% 	{$ 20$}
% 	\loigiai{Ta có $u_1=1$, $d=5-1=4$ nên $u_6=1+5d=1+20=21$.
% 	}
% \end{ex}

% \begin{ex}%[Dự án DCHT-11-KNTT]%[Dao-V- Thuy]%[1K2Y5-2]
% 	Cho cấp số cộng $\left( u_n \right)$ có các số hạng lần lượt là $-4;\,1;\,6;\,x$. Tìm giá trị của $x$.
% 	\choice
% 	{$x=7$}
% 	{$x=10$}
% 	{\True $x=11$}
% 	{$x=12$}
% 	\loigiai{
% 		Dễ thấy $u_1=-4$, $d=5$ nên $u_4=-4+3\cdot 5=11$.
% 	}
% \end{ex}

% \begin{ex}%[Dự án DCHT-11-KNTT]%[Dao-V- Thuy]%[1K2B5-2]
% 	Cho cấp số cộng $(u_n)$ có $u_1=-5$ và $d=3$. Mệnh đề nào sau đây đúng?
% 	\choice
% 	{$u_{15}=34$}
% 	{$u_{15}=45$}
% 	{\True $u_{13}=31$}
% 	{$u_{10}=35$}
% 	\loigiai {
% 		$\heva{
% 			& u_1=-5 \\
% 			& d=3 \\
% 		}\Rightarrow u_n=3n-8\Rightarrow \heva{
% 			& u_{15}=37 \\
% 			& u_{13}=31 \\
% 			& u_{10}=22. \\
% 		}$}
% \end{ex}

% \begin{ex}%[Dự án DCHT-11-KNTT]%[Dao-V- Thuy]%[1K2B5-2]
% 	Cho cấp số cộng có số hạng đầu là $u_1=-\dfrac{1}{2}$, công sai $d=\dfrac{1}{2}$. Trong mỗi bộ gồm năm số hạng dưới đây, bộ năm số nào là các số hạng liên tiếp của dãy này?
% 	\choice
% 	{$-\dfrac{1}{2};\,0;\,1;\,\dfrac{1}{2};\,1$}
% 	{$-\dfrac{1}{2};\,0;\,\dfrac{1}{2};\,0;\,\dfrac{1}{2}$}
% 	{$\dfrac{1}{2};\,1;\,2;\,\dfrac{5}{2};\,\dfrac{7}{2}$}
% 	{\True $1;\,\dfrac{3}{2};\,2;\,\dfrac{5}{2};\,3$}
% 	\loigiai{
% 		Ta có $u_1=-\dfrac{1}{2}$; $u_2=0$; $u_3=\dfrac{1}{2}$, $u_4=1$; $u_5=\dfrac{3}{2}$; $u_6=2$; $u_7=\dfrac{5}{2}$; $u_8=3$.
% 	}
% \end{ex}

% \begin{ex}%[Dự án DCHT-11-KNTT]%[Dao-V- Thuy]%[1K2B5-2]
% 	Cho cấp số cộng $(u_n)$ có $u_7=\dfrac{19}{5}$ và công sai $d=\dfrac{2}{5}$. Tính $u_{10}$.
% 	\choice
% 	{$\dfrac{2}{5}$}
% 	{$\dfrac{19}{5}$}
% 	{\True $5$}
% 	{$\dfrac{27}{5}$}
% 	\loigiai{Ta có: $u_7=u_1+6d\Rightarrow u_1=u_7-6d=\dfrac{19}{5}-6\cdot \dfrac{2}{5}=\dfrac{7}{5}$.\\
% 		Suy ra $u_{10}=u_1+9d=\dfrac{7}{5}+9\cdot \dfrac{2}{5}=5$.}
% \end{ex}

% \begin{ex}%[Dự án DCHT-11-KNTT]%[Dao-V- Thuy]%[1K2B5-2]
% 	Cho cấp số cộng $(u_n)$ có số hạng đầu $u_1=-1$ và công sai $d=-3$. Số hạng thứ $20$ của cấp số cộng này là
% 	\choice
% 	{\True $u_{20}=-58$}
% 	{$u_{20}=60$}
% 	{$u_{20}=-72$}
% 	{$u_{20}=-61$}
% 	\loigiai{Số hạng thứ $20$ là: $u_{20}=u_1+19d=-1+19\cdot (-3)=-58$.}
% \end{ex}

% \begin{ex}%[Dự án DCHT-11-KNTT]%[Dao-V- Thuy]%[1K2B5-2]
% 	Cho cấp số cộng $(u_n)$ có $u_1=-5$ và $d=3$. Số $100$ là số hạng thứ mấy của cấp số cộng?
% 	\choice
% 	{Thứ $15$}
% 	{Thứ $20$}
% 	{Thứ $35$}
% 	{\True Thứ $36$}
% 	\loigiai {
% 		Ta có $\heva{
% 			&u_1=-5 \\
% 			& d=3 \\
% 		}$. Vì $u_n=100 \Rightarrow 100=u_n=u_1+(n-1 )d=3n-8\Leftrightarrow n=36$.
% 	}
% \end{ex}

% \begin{ex}%[Dự án DCHT-11-KNTT]%[Dao-V- Thuy]%[1K2B5-2]
% 	Cho cấp số cộng $(u_n)$ có $u_2=2001$ và $u_5=1995$. Khi đó $u_{1001}$ bằng
% 	\choice
% 	{$u_{1001}=4005$}
% 	{$u_{1001}=4003$}
% 	{\True $u_{1001}=3$}
% 	{$u_{1001}=1$}
% 	\loigiai {
% 		$\heva{
% 			& 2001=u_2=u_1+d \\
% 			& 1995=u_5=u_1+4d \\
% 		}\Leftrightarrow \heva{
% 			&u_1=2003 \\
% 			& d=-2 \\
% 		}\Rightarrow u_{1001}=u_1+1000d=3$.}
% \end{ex}

% \begin{ex}%[Dự án DCHT-11-KNTT]%[Dao-V- Thuy]%[1K2B5-2]
% 	Cho cấp số cộng $(u_n)$ biết $\heva{&u_1+u_3=7\\&u_2+u_4=12}$. Tính $u_{21}$.
% 	\choice
% 	{$u_{21}=1$}
% 	{\True $u_{21}=51$}
% 	{$u_{21}=31$}
% 	{$u_{21}=21$}
% 	\loigiai{
% 		Ta có $\heva{&u_1+u_3=7\\&u_2+u_4=12}\Leftrightarrow\heva{&u_1+u_1+2d=7\\&u_1+d+u_1+3d=12}\Leftrightarrow\heva{&2u_1+2d=7\\&2u_1+4d=12}\Leftrightarrow\heva{&u_1=1\\&d=\dfrac{5}{2}.}$\\
% 		Suy ra $u_{21}=u_1+20d=1+20\cdot\dfrac{5}{2}=1+50=51$.}
% \end{ex}

% \begin{ex}%[Dự án DCHT-11-KNTT]%[Dao-V- Thuy]%[1K2B5-2]
% 	Một cấp số cộng có $7$ số hạng. Biết rằng tổng của số hạng đầu và số hạng cuối bằng $30$, tổng của số hạng thứ ba và số hạng thứ sáu bằng $35$. Tìm số hạng thứ bảy của cấp số cộng đã cho.
% 	\choice
% 	{$u_7=25$}
% 	{\True $u_7=30$}
% 	{$u_7=35$}
% 	{$u_7=40$}
% 	\loigiai{
% 		Theo đề ta có: $\heva{&u_1+u_7=30\\&u_3+u_6=35}\Leftrightarrow \heva{&u_1+(u_1+6d)=30\\&(u_1+2d)+(u_1+5d)=35}\Leftrightarrow \heva{&2u_1+6d=30\\&2u_1+7d=35}\Leftrightarrow \heva{&u_1=0\\&d=5.}$\\
% 		Do đó $u_7=u_1+6d=0+6\cdot 5=30$.}
% \end{ex}

% \begin{ex}%[Dự án DCHT-11-KNTT]%[Dao-V- Thuy]%[1K2G5-2]
% 	Cho dãy số $(u_n)$ có xác định bởi $\heva{&u_1=-2,\\&u_{n+1}=\dfrac{u_n}{1-u_n}} \; (\text{với } n\in\mathbb{N}^*)$ và dãy số $(v_n)$ được xác định bởi $v_n=\dfrac{u_n+1}{u_n}$. Số hạng thứ $2023$ của dãy $(v_n)$ là
% 	\choice
% 	{$ -\dfrac{2023}{3}$}
% 	{$ -\dfrac{4046}{3}$}
% 	{\True$ -\dfrac{4043}{2}$}
% 	{$ -2023$}
% 	\loigiai{
% 		Ta có $v_{n+1}-v_n=\dfrac{u_{n+1}+1}{u_{n+1}}-\dfrac{u_n+1}{u_n}=\dfrac{\dfrac{u_n}{1-u_n}+1}{\dfrac{u_n}{1-u_n}}-\dfrac{u_n+1}{u_n}=\dfrac{1}{u_n}-\dfrac{u_n+1}{u_n}=-1$. Vậy $(v_n)$	là một CSC có công sai $d=-1$.
% 		\\Mặt khác, ta có $v_1=\dfrac{u_1+1}{u_1}=\dfrac{1}{2}$, do đó số hạng tổng quát $v_n=\dfrac{1}{2}+(n-1)(-1)=-n+\dfrac{3}{2}$. \\
% 		Do đó $v_{2023}=-2023+\dfrac{3}{2}=-\dfrac{4043}{2}$.
% 	}
% \end{ex}
% \Closesolutionfile{ans}
% \begin{indapan}{10}
% 	{ans/ans-1K2-2-Dang3}
% \end{indapan}

\begin{dang}{Các bài toán thực tế}
	Các bài toán thực tế về cấp số cộng có thể được giải bằng cách sử dụng công thức của cấp số cộng. Công thức của cấp số cộng là: $ u_n = u_1 + (n-1)d $. Trong đó:
	\begin{itemize}
		\item $ u_n $ là số hạng thứ $ n $ của cấp số cộng.
		\item $ u_1 $ là số hạng đầu tiên của cấp số cộng.
		\item $ d $ là công sai của cấp số cộng.
		\item Một số công thức thường gặp:
		\begin{enumEX}[\faCheckCircleO]{1}
			\item $u_n=\dfrac{u_{n-1}+u_{n+1}}{2}=u_1+(n-1)d$.
			\item $S_n=\dfrac{(u_1+u_n)\cdot n}{2}=\dfrac{2u_1+(n-1)d}{2}\cdot n$.
		\end{enumEX}			
	\end{itemize}
\end{dang}
\subsubsection{Ví dụ minh hoạ}
\begin{vd}%[NB]%[DCHT Toán 11 - KNTT - Nguyễn Hữu Đức] %[1K2B6-6]
	Một người có một khoản tiền gửi ngân hàng với lãi suất 10\% /năm theo hình thức lãi đơn. Nếu sau $ 5 $ năm người đó nhận được tổng số tiền là $ 550 $ triệu đồng thì số tiền gửi ban đầu của người đó là bao nhiêu?
	\dapso{$366{,}67$ triệu đồng.}
	\loigiai{
		Gọi $x$ là số tiền gửi ban đầu của người đó $ (x>0) $.\\
		Sau 5 năm, số tiền nhận được bằng số tiền gốc cộng với lãi suất:
		$$
		x + 0{,}1x \times 5 = 1{,}5x.
		$$
		Theo đề bài, tổng số tiền nhận được sau 5 năm là $550$ triệu đồng, do đó ta có phương trình:
		$$
		1{,}5x = 550.
		$$
		Giải phương trình ta có:
		$$
		x = \frac{550}{1{,}5} \approx 366{,}67.
		$$
		Vậy số tiền gửi ban đầu của người đó là $366{,}67$ triệu đồng.
	}
\end{vd}
\begin{vd}%[DCHT Toán 11 - KNTT - Nguyễn Hữu Đức] %[1K2B6-6]
	Bạn An muốn mua một món quà tặng mẹ nhân ngày mùng $8/3$. Bạn quyết định tiết kiệm từ ngày $1/2/2017$ đến hết ngày $6/3/2017$. Ngày đầu An có $5\,000$ đồng, kể từ ngày thứ hai số tiền An tiết kiệm được ngày sau cao hơn ngày trước mỗi ngày $1\,000$ đồng. Tính số tiền An tiết kiệm được để mua quà tặng mẹ.
	\dapso{$731\,000$ đồng.}
	\loigiai{
		Tính số ngày mà An tiết kiệm được từ ngày $1/2/2017$ đến hết ngày $6/3/2017$:\\
		Số ngày từ ngày $1/2/2017$ đến hết ngày $28/2/2017$ là $28$ ngày.\\
		Số ngày từ ngày $1/3/2017$ đến hết ngày $6/3/2017$ là $6$ ngày.\\
		Vậy An tiết kiệm được $28+6=34$ ngày.\\
		Gọi $u_n$ là số tiền An tiết kiệm được vào ngày thứ $n$ kể từ ngày $1/2/2017$.\\
		Theo đề ta có $u_1=5\,000$ đồng.\\
		Vì ngày sau An tiết kiệm được nhiều hơn ngày trước mỗi ngày $1\,000$ đồng nên $u_n=u_{n-1}+1\,000$, với $n\ge 2$.\\
		Vậy $(u_n)$ là một cấp số cộng với $u_1=5\,000$ và công sai $d=1\,000$.\\
		Tổng số tiền An tiết kiệm được trong $34$ ngày là:
		$$S_{34}=\dfrac{n}{2} \left(2u_1+33d\right)= \dfrac{34}{2} \left(2\cdot 5\,000+33\cdot 1\,000\right)=731\,000.$$
		Vậy số tiền An tiết kiệm được để mua quà tặng mẹ là $731\,000$ đồng.
	}
\end{vd}

\begin{vd}%[TH]%[DCHT Toán 11 - KNTT - Nguyễn Hữu Đức] %[1K2B6-6]
	[Cấp số nhân] Một hội đồng quản trị quyết định tăng lương cho nhân viên hàng năm theo tỷ lệ cố định. Ví dụ, lương của một nhân viên được tăng thêm $ 5 $\% so với năm trước. Hỏi nếu lương của một nhân viên là $ 10 $ triệu đồng/năm vào năm nay, thì lương của nhân viên đó sẽ là bao nhiêu vào năm thứ $ 5 $?
	\dapso{$12{,}1550625 $ triệu đồng/năm.}
	\loigiai{		
		Theo giả thiết, lương của nhân viên được tăng thêm $ 5 $ \% so với năm trước đó.
		\begin{itemize}
			\item Vậy lương của nhân viên vào năm thứ $ 2 $ sẽ là $ 10\cdot(1+0{,}05)=10{,}5 $ triệu đồng/năm.
			\item Tương tự, lương của nhân viên vào năm thứ $ 3 $ sẽ là $ 10{,}5 \cdot(1+0{,}05)=11{,}025 $ triệu đồng/năm.
			\item Lương của nhân viên vào năm thứ $ 4 $ sẽ là $ 11{,}025\cdot (1+0{,}05)=11{,}57625 $ triệu đồng/năm.
			\item Cuối cùng, lương của nhân viên vào năm thứ $ 5 $ sẽ là $ 11{,}57625\cdot(1+0{,}05)=12{,}1550625 $ triệu đồng/năm.
		\end{itemize}
		Vậy lương của nhân viên đó vào năm thứ $ 5 $ sẽ là $ 12{,}1550625 $ triệu đồng/năm.\\
		Chú ý: Lương của nhân viên đó vào năm thứ $ 5 $ sẽ là $ u_5=u_1+4d=10+4\cdot 10\cdot 0{,}05=12 $ triệu đồng chỉ đúng trong trường hợp lương của một nhân viên được tăng thêm $ 5 $\% so với năm đầu tiên.
	}
\end{vd}


\begin{vd}%[TH]%%[DCHT Toán 11 - KNTT - Nguyễn Hữu Đức] %[1K2B6-6]
	Hùng đang tiết kiệm để mua một cây guitar. Trong tuần đầu tiên, anh ta để dành  $ 42 $ đô la, và trong mỗi tuần tiết theo, anh ta đã thêm $ 8 $ đô la vào tài khoản tiết kiệm của mình. Cây guitar Hùng cần mua có giá $ 400 $ đô la. Hỏi vào tuần thứ bao nhiêu thì anh ấy có đủ tiền để mua cây guitar đó?
	\dapso{$ n=46 $.}
	\loigiai{
		Gọi $ n $ là số tuần anh ta đã thêm $ 8 $ đô la vào tài khoản tiết kiệm của mình.\\
		Số tiền anh ta tiết kiệm được sau $ n $ tuần đó là $ S=42+8n $. \\
		Theo bài ra $ S=42+8n\ge 400\Leftrightarrow n\ge 44,75\Rightarrow n=45 $.\\
		Vậy kể cả tuần đầu thì tuần thứ $ 46 $ anh ta có đủ tiền để mua cây guitar đó.
	}
\end{vd}

\begin{vd}%[DCHT Toán 11 - KNTT - Nguyễn Hữu Đức]%[1K2Y6-6]
	[Cấp số nhân] Hàng tháng ông An gửi vào ngân hàng một số tiền như nhau là $5\,000\,000$ đồng (vào ngày đầu mỗi tháng) với lãi suất $0{,}5\%$ một tháng, biết tiền lãi của tháng trước được nhập vào tiền gốc của tháng sau. Hỏi sau $36$ tháng ông An nhận được số tiền vốn và lãi là bao nhiêu? (làm tròn đến hàng đơn vị).
	\dapso{ $ 197\,663\,927 $ đồng.}
	\loigiai{
		Gọi $a$ là số tiền ông An gửi vào hàng tháng, $r$ là lãi suất trên một tháng và $P_n$ là số tiền vốn và lãi ông An nhận được sau $n$ tháng.
		\begin{itemize}
			\item Sau một tháng, ông An có số tiền là $P_1=a+ar=a(1+r)$.
			\item Đầu tháng thứ hai, ông An có số tiền là $P_1+a=a(1+r)+a$.
			\item Sau hai tháng, ông An có số tiền là $P_2=a(1+r)+a+\left[a(1+r)+a\right]r=a\left[(1+r)^2+(1+r)\right]$.
			\item Cuối tháng thứ $36$, ông An có số tiền là
			\begin{align*}
				P_{36}&=a\left[(1+r)^{36}+(1+r)^{35}+\ldots+(1+r)\right]\\
				&=a(1+r)\dfrac{(1+r)^{36}-1}{r}\\
				&=5000000\cdot (1+0{,}005)\cdot\dfrac{(1+0{,}005)^{36}-1}{0{,}005}\\
				&\approx 197\,663\,927 \quad \text{(đồng)}.
			\end{align*}
		\end{itemize}
	}
\end{vd}

\begin{vd}[VDT]%[DCHT Toán 11 - KNTT - Nguyễn Hữu Đức] %[1K2Y6-6]
	Một xưởng có đăng tuyển công nhân với đãi ngộ về lương như sau: Trong quý đầu tiên thì xưởng trả là $ 6 $ triệu đồng/quý và kể từ quý thứ $ 2 $ sẽ tăng lên $ 0{,}5 $ triệu cho $ 1 $ quý. Hỏi với đãi ngộ trên thì sau $ 5 $ năm làm việc tại xưởng, tổng số lương của công nhân đó là bao nhiêu?
	\dapso{ $ 215 $ triệu đồng.}
	\loigiai{
		Gọi $u_n$ (triệu đồng) là số lương của công nhân trong quý thứ $n$.\\
		Theo đề:\\
		Quý đầu: $ u_1 = 6 $ triệu.\\
		Các quý tiếp theo: $ u_{n+1} = u_{n} + 0,5 $ với $\forall n \ge 1$.\\
		Mức lương của công nhân mỗi quý là $ 1 $ số hạng của dãy số $ u_n $. Mặt khác, lương của quý sau hơn lương quý trước là $ 0,5 $ triệu nên dãy số $ u_n $ là một cấp số cộng với công sai $ d = 0{,}5 $.\\
		Ta biết $ 1 $ năm sẽ có $ 4 $ quý nên $ 5 $ năm sẽ có $ 5\cdot 4 = 20 $ quý. Theo yêu cầu của đề bài ta cần tính tổng của $ 20 $ số hạng đầu tiên của cấp số cộng ($ u_n $).\\
		Lương tháng quý $ 20 $ của công nhân: $ u_{20} = 6 + (20 - 1)\cdot 0{,}5 = 15{,}5 $ triệu đồng.\\
		Tổng số lương của công nhân nhận được sau $ 5 $ năm làm việc tại xưởng: $ S_{12}=20\cdot (6+15,5)2=215 $ (triệu đồng).
	}
\end{vd}

% \subsubsection{Bài tập tự luận}
 
%[DCHT Toán 11 - KNTT - Nguyễn Hữu Đức] %[1K2B6-6]
% \begin{bt}%[NB]%[DCHT Toán 11 - KNTT - Nguyễn Hữu Đức] %[1K2B6-6]
% 	Sinh nhật bạn của An vào ngày $ 01 $ tháng năm. An muốn mua một món quà sinh nhật cho bạn nên quyết định bỏ ống heo $ 100 $ đồng vào ngày $ 01 $ tháng $ 01 $ năm $ 2016 $, sau đó cứ liên tục ngày sau hơn ngày trước $ 100 $ đồng. Hỏi đến ngày sinh nhật của bạn, An đã tích lũy được bao nhiêu tiền? (thời gian bỏ ống heo tính từ ngày $ 01 $ tháng $ 01 $ năm $ 2016 $ đến ngày $ 30 $ tháng $ 04 $  năm $ 2016 $).\dapso{$ 738\,100 $ đồng.}
% 	\loigiai{
% 		Từ ngày $1$ tháng $1$ năm $2016$ đến ngày $30$ tháng $4$ năm $2016$ có tổng cộng $31+29+31+30=121$ ngày.\\
% 		Gọi $S$ là số tiền An tích lũy được vào ngày sinh nhật của bạn.\\
% 		Do An bỏ được $100$ đồng vào ngày đầu tiên nên số tiền An tích lũy được vào ngày thứ $n$ là 
% 		$$S= 100 + 100(n-1).$$
% 		Vậy tổng số tiền An tích lũy được là:
% 		$$
% 		S=100 + 200 + \cdots + 12\,100 = \frac{121(100 + 12\,100)}{2} = 738\,100
% 		.$$
% 		Vậy An đã tích lũy được $738\,100$ đồng vào ngày sinh nhật của bạn.}
% \end{bt}

% \begin{bt}%[TH]%[DCHT Toán 11 - KNTT - Nguyễn Hữu Đức] %[1K2Y6-6]
% 	Người ta trồng $ 3\,003 $ cây theo dạng một hình tam giác như sau: hàng thứ nhất trồng $ 1 $ cây, hàng thứ hai trồng $ 2 $ cây, hàng thứ ba trồng $ 3 $ cây... cứ tiếp tục trồng như thế cho đến khi hết số cây. Số hàng cây được trồng là bao nhiêu?
% 	\dapso{$ 77 $ hàng.}
% 	\loigiai{
% 		Tổng số cây trồng được là $1 + 2 + 3 + \cdots + n$, nghĩa là tổng của $n$ số tự nhiên đầu tiên. Ta cần tìm số $n$ để tổng này bằng $3003$.\\
% 		Ta có công thức tổng của $n$ số tự nhiên đầu tiên là:
% 		$$
% 		1 + 2 + 3 + \cdots + n = \frac{n(n+1)}{2}.
% 		$$
% 		Giải phương trình:
% 		$$
% 		\frac{n(n+1)}{2} = 3\,003.
% 		$$
% 		Ta có:
% 		$$
% 		n(n+1) = 6\,006
% 		\Rightarrow n=77.$$
% 		Vậy số hàng cây được trồng là $77$.
% 	}
% \end{bt}
% \begin{bt}%[TH]%[DCHT Toán 11 - KNTT - Nguyễn Hữu Đức] %[1K2B6-6]
% 	Một công ty định mức sản phẩm hàng tháng theo cấp số cộng. Ví dụ, sản lượng hàng tháng của một công ty được tăng thêm $10$ sản phẩm so với tháng trước. Nếu công ty sản xuất được $ 100 $ sản phẩm trong tháng này, hỏi công ty sẽ sản xuất được bao nhiêu sản phẩm trong tháng thứ $ 12 $?
% 	\dapso{$ 210 $ sản phẩm.}
% 	\loigiai{
% 		Công thức cấp số cộng được sử dụng để tính sản lượng hàng tháng của công ty. Nếu công ty sản xuất được $100$ sản phẩm trong tháng này và sản lượng hàng tháng được tăng thêm $10$ sản phẩm so với tháng trước, ta có thể sử dụng công thức sau để tính sản lượng hàng tháng của công ty trong tháng thứ $12$:
% 		$$
% 		a_n = a_1 + (n-1)d.
% 		$$
% 		Trong đó $a_1$ là sản lượng hàng tháng ban đầu, $d$ là công sai và $n$ là số tháng.\\
% 		Với bài toán này, ta có: $a_1 = 100$, $d = 10$, $n = 12$.\\
% 		Sản lượng hàng tháng của công ty trong tháng thứ $12$ là:
% 		$$
% 		a_{12} = a_1 + (n-1)d = 100 + (12-1) \times 10 = 210.
% 		$$
% 		Vậy công ty sẽ sản xuất được $210$ sản phẩm trong tháng thứ $12$.
% 	}
% \end{bt}

\subsubsection{Câu hỏi trắc nghiệm}
\Opensolutionfile{ans}[ans/ans-1K2-2-dang4]
\begin{ex}%[TH]%[DCHT Toán 11 - KNTT - Nguyễn Hữu Đức]%[1K2B6-6]
	Một công ty đang cần tuyển dụng thêm nhân viên. Công ty quyết định tăng số lượng nhân viên hàng tháng theo cấp số cộng. Nếu công ty đã có $ 20 $ nhân viên và quyết định tăng thêm $ 2 $ nhân viên hàng tháng, hỏi sau bao nhiêu tháng công ty sẽ có 50 nhân viên?
	\choice
	{$19$ tháng}
	{\True $16$ tháng}
	{$36$ tháng}
	{$26$ tháng}
	\loigiai{
		Để giải bài toán này, ta có thể sử dụng công thức cấp số cộng:
		$$
		a_n = a_1 + (n-1) \times d.
		$$
		Trong đó $a_1$ là số lượng nhân viên ban đầu, $d$ là số lượng nhân viên tăng hàng tháng và $n$ là số tháng.\\
		Ta cần tìm số tháng $n$ để công ty có được $50$ nhân viên. Thay các giá trị vào công thức cấp số cộng ta có:
		$$
		50 = 20 + (n-1) \times 2.
		$$
		Suy ra:
		$$
		n = \frac{50 - 20}{2} + 1 = 16
		.$$
		Vậy sau $16$ tháng kể từ khi công ty quyết định tăng số lượng nhân viên hàng tháng theo cấp số cộng, công ty sẽ có được $50$ nhân viên.
	}
\end{ex}
\begin{ex}%[VD]%[DCHT Toán 11 - KNTT - Nguyễn Hữu Đức] %[1K2B6-6]
	Một người đang tăng cường luyện tập thể thao hàng ngày. Anh ta quyết định tăng mức độ luyện tập theo cấp số cộng hàng tuần. Nếu anh ta bắt đầu với mức luyện tập $ 30 $ phút mỗi ngày và tăng thêm $ 5 $ phút mỗi ngày, hỏi anh ta sẽ luyện tập được bao lâu để đạt được mức luyện tập $ 60 $ phút mỗi ngày?
	\choice
	{$16$ ngày}
	{\True $6$ ngày}
	{$9$ ngày}
	{$7$ ngày}
	\loigiai{
		Gọi $n$ là số ngày liên tiếp mà người đó tăng mức độ luyện tập. Theo đó, mức độ luyện tập của người đó sau $n$ ngày là:
		$$
		30 + 5n\, \text{(phút)}.
		$$
		Vì để đạt được mức luyện tập $ 60 $ phút mỗi ngày nên:
		$$
		30 + 5n = 60.
		$$
		Từ đó suy ra:
		$$
		n = \frac{60-30}{5} = 6.
		$$
		Vậy người đó cần luyện tập liên tiếp trong $6$ ngày để đạt được mức luyện tập $60$ phút mỗi ngày.	
	}
\end{ex}
\begin{ex}%[VD]%[DCHT Toán 11 - KNTT - Nguyễn Hữu Đức] %[1K2Y6-6]
	Nếu một công ty công nghệ mới thành lập có số lượng người dùng ban đầu là $ 10\,000 $ và mỗi tháng tăng thêm cố định $ 5\,000 $ lượng người dùng, thì sau bao lâu có số lượng người dùng là $ 1 $ triệu. 
	\choice
	{\True $198$ tháng}
	{$197$ tháng}
	{$18$ tháng}
	{$98$ tháng}
	
	\loigiai{
		Ta cần tính số tháng $n$ theo công thức sau:
		$$10\,000 + 5\,000n = 1\,000\,000.$$
		$$\Rightarrow n = \frac{1\,000\,000 - 10\,000}{5\,000} = 198.$$
		Vậy sau khoảng $198$ tháng (khoảng $16$ năm và $6$ tháng), công ty sẽ đạt được $1$ triệu người dùng.
	}
\end{ex}
\begin{ex}%[VDC]%[DCHT Toán 11 - KNTT - Nguyễn Hữu Đức] %[1K2Y6-6]
	Một nhà đầu tư đang đầu tư vào một quỹ đầu tư với mức lợi nhuận cố định hàng năm. Nếu nhà đầu tư đầu tư vào quỹ đầu tư với số tiền ban đầu là $ 20 $ triệu đồng và mức lợi nhuận hàng năm là $ 10 $\%, hỏi số tiền nhà đầu tư sẽ nhận được sau $ 7 $ năm?
	\choice
	{\True $34$ triệu đồng}
	{$14$ triệu đồng}
	{$30$ triệu đồng}
	{$39$ triệu đồng}
	\loigiai{
		Với số tiền ban đầu là $ 20 $ triệu đồng và mức lợi nhuận hàng năm là $ 10 $\%, ta có thể tính được số tiền nhà đầu tư sẽ nhận được sau $ 1 $ năm, sau đó sử dụng cấp số cộng để tính số tiền nhà đầu tư sẽ nhận được sau $ 7 $ năm.
		
		Số tiền nhà đầu tư sẽ nhận được sau $ 1 $ năm là:
		
		$ 20 $ triệu đồng $\times$ $ 10 $\% = $ 2 $ triệu đồng
		
		Số tiền nhà đầu tư sẽ nhận được sau $ 7 $ năm là:
		
		$ 2 $ triệu đồng $\times$ $ 7 $ năm + $ 20 $ triệu đồng = $ 34 $ triệu đồng
		
		Vậy sau $ 7 $ năm, nhà đầu tư sẽ nhận được tổng cộng $ 34 $ triệu đồng.
	}
\end{ex}
\begin{ex}%[VDC]%[DCHT Toán 11 - KNTT - Nguyễn Hữu Đức] %[1K2Y6-6]
	Một công ty sản xuất bánh kẹo tăng sản lượng sản phẩm của mình lên mỗi tháng. Nếu sản lượng ban đầu là $ 1\,000 $ sản phẩm, một sản phẩm lợi nhuận $ 1 $ USD và tăng thêm $ 200 $ sản phẩm mỗi tháng, thì sau bao nhiêu tháng lợi nhuận công ty $ 1 $ triệu đô.
	\choice
	{$8\,000$ tháng}
	{$7\,000$ tháng}
	{$9\,000$ tháng}
	{\True $5\,000$ tháng}
	\loigiai{
		Để tính thời gian công ty đạt được lợi nhuận 1 triệu đô, chúng ta cần biết lợi nhuận của công ty đạt được bao nhiêu sau mỗi tháng.\\
		Giả sử sản lượng ban đầu là $1\,000$ sản phẩm một sản phẩm lợi nhuận $1$ USD và tăng thêm $200$ sản phẩm mỗi tháng. Ta có thể tính được lợi nhuận của công ty sau mỗi tháng như sau:
		\begin{itemize} 
			\item Tháng 1: $1\,000 \times 1 = 1000$ USD.
			\item Tháng 2: $(1\,000 + 200) \times 1 = 1200$ USD.
			\item Tháng 3: $(1\,000 + 2 \times 200) \times 1 = 1\,400$ USD.
			\item Tháng 4: $(1\,000 + 3 \times 200) \times 1 = 1\,600$ USD.
			\item Tháng $n$: $(1\,000 + (n - 1) \times 200) \times 1 = (n - 1) \times 200 + 1\,000$ USD.
		\end{itemize}
		Để tính thời gian để công ty đạt được lợi nhuận $1$ triệu đô, ta giải phương trình sau:
		
		$(n - 1) \times 200 + 1\,000 = 10^6$
		
		$\Rightarrow (n - 1) \times 200 = (10^6 - 1000)$
		
		$\Rightarrow n - 1 = \dfrac{10^6 - 1\,000}{200}$
		
		$\Rightarrow n = \dfrac{10^6 - 1\,000}{200} + 1$
		
		$\Rightarrow n = 5\,001$
		
		Vậy sau $5\,000$ tháng, công ty sẽ đạt được lợi nhuận $1$ triệu đô.
	}
\end{ex}
\begin{ex}%[VDC]%[DCHT Toán 11 - KNTT - Nguyễn Hữu Đức] %[1K2Y6-6]
	Một công ty tăng lương cho nhân viên hàng năm bằng cách thêm một số tiền cố định vào lương của họ. Ví dụ: Nếu lương ban đầu của một nhân viên là $ 10 $ triệu đồng và công ty tăng lương $ 2 $ triệu đồng mỗi năm, thì lương của nhân viên sẽ là bao nhiêu nếu làm cho công ty $ 19 $ năm?
	\choice
	{$ 16 $ triệu đồng}
	{$ 26 $ triệu đồng}
	{$ 28 $ triệu đồng}
	{\True $ 46 $ triệu đồng}
	\loigiai{
		Do tăng lương cho nhân viên hàng năm bằng cách thêm một số tiền cố định nên ta có thể sử dụng công thức tính số hạng thứ $ n $ của cấp số cộng
		$ a_n = a_1 + (n - 1)d $.\\
		Ở bài toán này, ta có:\\
		$ a_1 = 10 $ (triệu đồng) là lương ban đầu của nhân viên.\\
		$ d = 2 $ (triệu đồng) là công sai của cấp số cộng.\\
		$ n = 19  $ là số thứ tự của số hạng.\\
		Ta thay các giá trị này vào công thức trên để tính lương của nhân viên sau $ 19 $ năm:\\
		$ a_{19} = 10 + (19 - 1)2 \Rightarrow$
		$ a_{19} = 46 $ (triệu đồng).\\
		Vậy lương của nhân viên sau $ 19 $ năm làm việc cho công ty là $ 46 $ triệu đồng.
	}
\end{ex}

\begin{ex}%[VDC]%[DCHT Toán 11 - KNTT - Nguyễn Hữu Đức] %[1K2Y6-6]
	Tài sản thường bị khấu hao khiến chúng có tuổi thọ hữu ích giới hạn. Ví dụ, nếu một công ty mua một chiếc xe tải với giá $ 35\,000 $ đô la và nó bị khấu hao với tốc độ không đổi là $ 700 $ đô la mỗi tháng, thì sau bao lâu giá trị của nó còn $ 5\,000 $ đô la.
	\choice
	{$ x = 23 $ tháng}
	{\True  $ x = 43 $ tháng}
	{$ x = 41 $ tháng}
	{$ x = 40 $ tháng}
	\loigiai{
		\textit{Cách 1:} Thời gian để giá trị của chiếc xe tải trên được khấu hao xuống còn $5.000 $ đô la có thể được tính bằng cách sử dụng công thức sau:\\
		Giá trị khởi đầu của chiếc xe tải là $35\,000$
		Giá trị cuối cùng của chiếc xe tải là $5\,000$
		Tốc độ khấu hao tương ứng $700$/tháng\\
		Để tìm ra thời gian cần thiết để giá trị của chiếc xe tải giảm xuống còn $5.000$, ta cần tìm số tháng được khấu hao.\\
		Giả sử số tháng cần khấu hao là $ x $ tháng.\\
		Giá trị của chiếc xe tải sau $ x $ tháng khấu hao được tính bằng:\\ 
		$35\,000 - 700x = 5\,000$.\\
		Giải phương trình trên ta có: $ x \approx 43 $ tháng\\
		Vì vậy, sau $ 43 $ tháng, giá trị của chiếc xe tải sẽ giảm xuống còn $5\,000$.
		Ngoài ra ta có thể giải theo cấp số cộng như sau:\\
		\textit{Cách 2:} Ta có thể sử dụng cộng thức tính số hạng thứ $ n $ của cấp số cộng
		$ a_n = a_{1} + (n - 1)d $
		\begin{itemize}
			\item $ u_1 = 35\,000 $ (đô la) là giá trị ban đầu của xe tải.
			\item $ d = -700 $ (đô la) là công sai của cấp số cộng (âm vì giá trị xe tải giảm).
			\item $ a_n = 5\,000 $ (đô la) là giá trị cuối cùng của xe tải.
		\end{itemize}
		Ta thay các giá trị này vào công thức trên để tính số tháng mà xe tải bị khấu hao đến $ 5\,000 $ đô la:
		$$ 5\,000 = 35\,000 + (n - 1)(-700)\Rightarrow n = 43{,}857.$$
		Vậy sau khoảng $ 43{,}857 $ tháng, tức là khoảng $ 3 $ năm và $ 7 $ tháng, giá trị của xe tải sẽ còn khoảng $ 5\,000 $ đô la.
	}
\end{ex}
\begin{ex}%[VDC]%[DCHT Toán 11 - KNTT - Nguyễn Hữu Đức] %[1K2Y6-6]
	Các thiết bị điện tử như máy tính, điện thoại, hoặc máy ảnh thường bị khấu hao nhanh chóng do sự phát triển của công nghệ mới. Ví dụ, nếu một người mua một máy tính Macbook với giá $ 2\,000 $ đô la và nó bị khấu hao với tốc độ không đổi là $ 100 $ đô la mỗi tháng, thì giá trị của Macbook còn lại $ 1\,000 $ đô la sau bao nhiêu tháng?
	%\dapso{$ 11 $ tháng.}
	\choice
	{$ x = 12 $ tháng}
	{$ x = 43 $ tháng}
	{\True  $ x = 11 $ tháng}
	{$ x = 10 $ tháng}
	\loigiai{
		Để giải bài toán này, ta có thể sử dụng công thức tính số hạng thứ $ n $ của cấp số cộng $ a_n = a + (n - 1)d. $\\
		Ở bài toán này, ta có:\\
		$ a = 2\,000 $ (đô la) là giá trị ban đầu của máy tính Macbook.\\
		$ d = -100 $ (đô la) là công sai của cấp số cộng (âm vì giá trị máy tính giảm).\\
		$ a_n = 1\,000 $ (đô la) là giá trị cuối cùng của máy tính Macbook.\\
		Ta thay các giá trị này vào công thức trên để tính số tháng mà máy tính bị khấu hao đến $ 1\,000 $ đô la:
		$$ 1\,000 = 2\,000 + (n - 1)(-100)\Rightarrow n=11. $$
		Vậy sau $ 11 $ tháng, giá trị của máy tính Macbook sẽ còn $ 1\,000 $ đô la.
	}
\end{ex}
\begin{ex}%[VDC]%[DCHT Toán 11 - KNTT - Nguyễn Hữu Đức] %[1K2Y6-6]
	Ban đầu có 1m$^2$ bèo sinh sôi trên mặt hồ biết tốc độ sinh sôi ngày sau hơn ngày trước $ 0{,}5 $m$^2 $. Biết diện tích mặt hồ nước là $ 120 $m$^2 $ hỏi sau bao lâu bèo phủ đầy mặt hồ?
	%\dapso{$ 238 $ ngày}
	\choice
	{$ x = 120 $ tháng}
	{$ x = 143 $ tháng}
	{\True  $ x = 238 $ tháng}
	{$ x = 130 $ tháng}
	\loigiai{
		Giả sử sau $ x $ ngày, diện tích của bèo phủ đầy mặt hồ là $ S m^2 $.
		
		Theo đề bài, ta biết được rằng:
		\begin{itemize}
			\item Tốc độ sinh sôi của bèo là $ 0{,}5 $m$^2 $/ngày.
			\item Ban đầu, diện tích của bèo là  1 m$^2 $.
			\item Diện tích mặt hồ là  $ 120 $m$^2 $.
		\end{itemize}
		Vậy ta có phương trình sau đây:	$ S = 1 + 0{,}5x. $\\
		Điều kiện để bèo phủ đầy mặt hồ là $ S = 120 $.\\
		$ 1 + 0{,}5x = 120 $ hay	$ 0{,}5x = 119 $ $\Rightarrow x = 238 $ ngày.\\
		Vậy sau $ 238 $ ngày, bèo sẽ phủ đầy mặt hồ.
	}
\end{ex}
\begin{ex}%[VDC]%[DCHT Toán 11 - KNTT - Nguyễn Hữu Đức] %[1K2Y6-6]
	Nhà hát lớn Dạ Cỗ Vĩ Lan ở An Cư có hàng ghế đầu kí hiệu dãy A là $50$ chỗ hàng ghế, sau dãy B là $48$ chỗ và như thế hàng sau ít hơn hàng trước $ 2 $ ghế, biết hàng cuối cùng có $ 10 $ ghế. Tính tổng số dãy ghế và tổng số chỗ ngồi?
	%\dapso{ $21$ dãy và $ 630 $ chỗ.}
	\choice
	{\True  $21$ dãy và $ 630 $ chỗ}
	{$20$ dãy và $ 630 $ chỗ}
	{$11$ dãy và $ 630 $ chỗ}
	{$21$ dãy và $ 930 $ chỗ}
	\loigiai{
		Gọi $n$ là số dãy ghế. Theo đề bài, ta có:
		$$
		\begin{cases}
			S=50 + 48 + \cdots + 10 = \dfrac{50+10}{2}n \\
			S=\dfrac{2.50+(n-1)\cdot (-2)}{2}n
		\end{cases}
		$$
		Từ phương trình đầu tiên, ta có:
		$$
		S = 50 + 48 + \cdots + 10 = \frac{50+10}{2}n = 30n.
		$$
		Từ phương trình thứ hai, ta có:
		$$
		S = \frac{2\cdot 50+(n-1)\cdot(-2)}{2}n = (50 - n + 1)n = (51 - n)n.
		$$
		Do đó, ta có:
		$$
		30n = (51 - n)n
		\Rightarrow n=21.$$
		Vậy $n = 21$ dãy ghế và $ 30\cdot 21=630 $ ghế.
		
	}
\end{ex}
\begin{ex}%[VDC]%[DCHT Toán 11 - KNTT - Nguyễn Hữu Đức] %[1K2Y6-6]
	Người ta trồng  cây theo dạng một hình tam giác như sau: hàng thứ nhất trồng $ 1 $ cây, hàng thứ hai trồng $ 3 $ cây, hàng thứ ba trồng $ 5 $ cây,... cứ tiếp tục trồng như thế cho đến khi hết số cây là $ 6\,561 $. Số hàng cây được trồng là bao nhiêu?
	%\dapso{ $ 81 $ hàng.}
	\choice
	{\True  $ 81 $ hàng}
	{$ 16 $ hàng}
	{$ 100 $ hàng}
	{$ 89 $ hàng}
	\loigiai{
		Để giải bài toán này, ta cần tìm số hàng cây được trồng cho đến khi tổng số cây là $ 2023 $. 
		\begin{itemize}
			\item Hàng thứ nhất trồng $ 1 $ cây. 
			\item Hàng thứ hai trồng $ 3 $ cây ($ 1 $ cây $ + 2 $ cây).
			\item Hàng thứ ba trồng $ 5 $ cây ($ 1 $ cây $ + 2 $ cây $ + 2 $ cây).
			\item ...
		\end{itemize}
		Vậy ta thấy rằng số cây trồng trong hàng thứ $n$ là $(n-1)\cdot 2+1$. \\
		Số cây được trồng trong $n$ hàng đầu tiên là: 
		$$1 + 3 + 5 + ... + (2n-1) = n^2.$$ 
		Để tìm số hàng cây được trồng cho đến khi tổng số cây là $ 6561 $, ta giải phương trình sau:\\ 
		$n^2 = 6\,561.$ 
		Vậy số hàng cây được trồng là $ 81 $ hàng.
	}
\end{ex}
\begin{ex}%[VDC]%[DCHT Toán 11 - KNTT - Nguyễn Hữu Đức] %[1K2Y6-6]
	Người ta thả một $ 1 $ m$^2$ lá bèo vào một hồ nước. Kinh nghiệm cho thấy sau $ x $ giờ, bèo sẽ sinh sôi kín cả mặt hồ $ 500 $ m$^2 $. Biết rằng sau mỗi giờ, lượng lá bèo tăng thêm $ 0{,}5 $ m$^2 $ và tốc độ tăng không đổi tìm $ x $?
	%\dapso{ $999$ giờ.}
	\choice
	{$888$ giờ}
	{$777$ giờ}
	{\True  $999$ giờ}
	{$700$ giờ}
	\loigiai{
		Bài toán này có thể giải bằng cách sử dụng công thức tăng trưởng của bèo. Giả sử lượng lá bèo ban đầu là $ 1 $ m$^2$, sau mỗi giờ lượng lá bèo tăng thêm $ 0{,}5 $ m$^2$. Sau $x$ giờ, lượng lá bèo đã phủ kín mặt hồ $ 500 $ m$^2$. Ta có thể viết phương trình sau:
		$$1 + 0{,}5x = 500.$$
		Giải phương trình ta được:
		$$x = \frac{500-1}{0{,}5} \approx 999.$$
		Vậy sau khoảng $999$ giờ (khoảng 41 ngày), lượng lá bèo sẽ phủ kín mặt hồ $ 500 $ m$^2$.
	}
\end{ex}
\Closesolutionfile{ans}
% \begin{indapan}{10}
% 	{ans/ans-1K2-2-dang6}
% \end{indapan}
%%Bài 7. CSN
% \def\tenchude{CẤP SỐ NHÂN}
\setcounter{section}{6}
\setcounter{dang}{0}
\setcounter{ex}{0}
\setcounter{bt}{0}
\setcounter{vd}{0}
\section{Cấp số nhân}
\subsection{Tóm tắt lý thuyết}
\begin{tomtat}
	\subsubsection{Định nghĩa} 
	Cấp số nhân là một dãy số (hữu hạn hoặc vô hạn) mà trong đó, kể từ số hạng thứ hai, mỗi số hạng đều bằng tích một số đứng ngay trước nó với một số $ q $ không đổi, nghĩa là:
	$$ u_{n}=u_{n-1}\cdot q\,\,\text{với}\,\forall n\in \mathbf{N}{,}\,n\ge 2 $$
	Số $ q $ được gọi là công bội của cấp số nhân
	\subsubsection{Số hạng tổng quát của cấp số nhân}
	Nếu cấp số nhân $ (u_n) $ có số hạng đầu là $ u_1 $ và công bội $ q $ thì số hạng tổng quát $ u_n $ của nó được xác định bởi công thức:
	$$u_n = u_1 \cdot q^{n-1}\,,n\ge 2$$
	\subsubsection{Tổng của $ n $ số hạng đầu tiên của cấp số nhân}
	Giả sử $ (u_n) $ là cấp số nhân có công bội $ q\ne 1 $. Đặt $ S_n=u_1+u_2+\cdots +u_n, $ khi đó
	$$S_n = u_1\cdot\frac{1-q^n}{1-q}.$$
	\begin{note}
		Khi $ q=1 $ thì $ S_n=n\cdot u_1 $.
	\end{note}	
	\begin{itemize}
		\item Công bội của cấp số nhân: $q = \sqrt[n-1]{\frac{u_n}{u_1}}$.
		\item Số hạng đầu tiên của cấp số nhân: $u_1 = \frac{u_n}{q^{n-1}}$.
		\item $ a,b,c $ là ba số hạng liên tiếp cấp số nhân thì $ a\cdot c=b^2 $. 
	\end{itemize}	
\end{tomtat}
\subsection{Các dạng toán thường gặp}
\begin{dang}{Nhận diện cấp số nhân, công bội $ q $}
	Để nhận diện (chứng minh) mỗi dãy số là cấp số nhân, ta làm như sau:\\
	Chứng minh $ u_{n+1}=u_nq $, $ \forall n\in\mathbb{N}^* $ và $ q $ là một số không đổi.\\
	Nếu $ u_n\ne 0 $, $ \forall n\in\mathrm{N}^* $ thì ta lập tỉ số $ \dfrac{u_{n+1}}{u_n}=k $.
	\begin{itemize}
		\item Nếu $ k $ là hằng số thì $ (u_n) $ là cấp số nhân với công bội $ q=k $.
		\item Nếu $ k $ phụ thuộc vào $ n $ thì $ (u_n) $ không phải là cấp số nhân.
	\end{itemize}
	Để chứng minh dãy $ (u_n) $ không phải là một cấp số nhân. Khi đó, ta chỉ cần chỉ ra ba số hạng liên tiếp không tạo thành một cấp số nhân, chẳng hạn $ \dfrac{u_3}{u_2}\ne \dfrac{u_2}{u_1} $.\\
	Để chứng minh ba số $ a,b,c $ theo thứ tự đó lập được một cấp số nhân, thì ta chứng minh $ ac=b^2 $ hoặc $ |b|=\sqrt{ac} $.	
\end{dang}
\subsubsection{Ví dụ minh hoạ}
\begin{vd}%[NB]%[DCHT Toán 11 - KNTT -Tên Huỳnh Thanh Chí]%[1K2Y7-1]
	Dãy số $ 1;1;1;1;\ldots $ có phải là một cấp số nhân hay không?
	\dapso{Dãy số $ 1;1;1;1;\ldots $ là một cấp số nhân.}
	\loigiai{
	Dễ thấy $ \dfrac{u_2}{u_1}=\dfrac{u_3}{u_2}=\ldots=1 $ là một số không đổi.\\
	Do đó dãy số $ 1;1;1;1;\ldots $ là một cấp số nhân.
	}
\end{vd}
\begin{vd}%[TH]%[DCHT Toán 11 - KNTT -Tên Huỳnh Thanh Chí] %[ID6 chương trình mới]
	Dãy số $ u_n=3^n $ có phải là một cấp số nhân không? Nếu có, hãy tìm công bội của cấp số nhân đó.
	\dapso{$ (u_n) $ là cấp số nhân với công bội $ q=3 $.}
	\loigiai{
	Ta có $ \dfrac{u_{n+1}}{u_n}=\dfrac{3^{n+1}}{3^n}=\dfrac{3^n\cdot 3}{3^n}=3 $ là số không đổi nên $ (u_n) $ là cấp số nhân với công bội $ q=3 $.
	}
\end{vd}
\begin{vd}%[TH]%[DCHT Toán 11 - KNTT -Tên Huỳnh Thanh Chí]%[1K2B7-1]
	Dãy số $ \heva{& u_1=3\\ & u_{n+1}=\dfrac{9}{u_n}} $ có phải là một cấp số nhân không? Nếu có, hãy tìm công bội của cấp số nhân đó.
	\dapso{$ (u_n) $ là một cấp số nhân với công bội $ q=1 $.}
	\loigiai{
	Xét dãy số $ \heva{& u_1=3\\ & u_{n+1}=\dfrac{9}{u_n}} $ có
	$ \dfrac{u_{n+1}}{u_n}=\dfrac{9}{u_n}:\dfrac{9}{u_{n-1}}=\dfrac{u_{n-1}}{u_n}\Rightarrow u_{n+1}=u_{n-1},\forall n\ge 2 $.\\
	Do đó ta có $ \heva{& u_1=u_3=u_5=\ldots=u_{2n+1}=\ldots \quad (1)\\ & u_2=u_4=u_6=\ldots=u_{2n}=\ldots \quad (2).} $\\
	Theo đề bài ta có $ u_1=3 \Rightarrow u_2=\dfrac{9}{u_1}=3 $ (3).\\
	Từ $ (1), (2) $ và $ (3) $ suy ra $ u_1=u_2=u_3=u_4=\ldots=u_{2n}=u_{2n+1}=\ldots $.\\
	Do đó $ (u_n) $ là một cấp số nhân với công bội $ q=1 $.
	}
\end{vd}
\begin{vd}%[TH]%[DCHT Toán 11 - KNTT -Tên Huỳnh Thanh Chí]%[1K2B7-1]
	Cho $ (u_n) $ là cấp số nhân có công bội $ q\ne 0,u_1\ne 0 $. Chứng minh rằng dãy số $ (v_n) $ với $ v_n=u_nu_{2n} $ cũng là một cấp số nhân.
	\dapso{$ (v_n) $ là một cấp số nhân với công bội là $ q^3 $.}
	\loigiai{
		Ta có $ \dfrac{v_n}{v_{n-1}}=\dfrac{u_nu_{2n}}{u_{n-1}u_{2(n-1)}}=\dfrac{u_1q^{n-1}\cdot u_1q^{2n-1}}{u_1q^{n-2}\cdot u_1q^{2n-3}}=q^3 $. Do đó $ (v_n) $ là một cấp số nhân với công bội là $ q^3 $.
	}
\end{vd}
\begin{vd}[VDT]%[DCHT Toán 11 - KNTT -Tên Huỳnh Thanh Chí]%[1K2K7-1]
	Cho dãy số $ (u_n) $ được xác định bởi $ \heva{& u_1=2\\ & u_{n+1}=4u_n+9},\forall n\in\mathbb{N}^* $. Chứng minh rằng dãy số $ (v_n) $ xác định bởi $ v_n=u_n+3,\forall n\in\mathbb{N}^* $ là một cấp số nhân. Hãy xác định số hạng đầu và công bội của cấp số nhân đó.
	\dapso{$ (v_n) $ là cấp số nhân với công bội $ q=4 $ và số hạng đầu $ v_1=5 $.}
	\loigiai{
	Ta có $ v_n=u_n+3 $ (1) và $ v_{n+1}=u_{n+1}+3 $ (2).\\
	Theo đề ta có $ u_{n+1}=4u_n+9 \Rightarrow u_{n+1}+3=4(u_n+3) $ (3).\\
	Thay (1) và (2)	vào (3) ta được $ v_{n+1}=4v_n \Rightarrow \dfrac{v_{n+1}}{v_n}=4,\forall n\in\mathbb{N}^* $.\\
	Suy ra $ (v_n) $ là cấp số nhân với công bội $ q=4 $ và số hàng đầu $ v_1=u_1+3=2+3=5 $.
	}
\end{vd}
\subsubsection{Bài tập tự luận}
 
\begin{bt}%[NB]%[DCHT Toán 11 - KNTT -Tên Huỳnh Thanh Chí]%[1K2Y7-1]
	Dãy số $25$; $5$; $1$; $\dfrac{1}{5}$; $\ldots$ có phải là một cấp số nhân không? Nếu có hãy tìm công bội của cấp số nhân đó.
	\dapso{Dãy số $25$; $5$; $1$; $\dfrac{1}{5}$; $\ldots$ là một cấp số nhân với công bội $ q=\dfrac{1}{5} $.}
	\loigiai{
	Ta có $ \dfrac{u_2}{u_1}=\dfrac{u_3}{u_2}=\ldots=\dfrac{1}{5} $ là một số không đổi.\\
	Do đó dãy số $25$; $5$; $1$; $\dfrac{1}{5}$; $\ldots$ là một cấp số nhân với công bội $ q=\dfrac{1}{5} $.
	}
\end{bt}
\begin{bt}%[NB]%[DCHT Toán 11 - KNTT -Tên Huỳnh Thanh Chí]%[1K2B7-1]
	Dãy số $1$; $n$; $n^2$; $n^3$; $ n^4 $; $\ldots$ (với $ n>1 $) có phải là một cấp số nhân không? Nếu có hãy tìm công bội của cấp số nhân đó.
	\dapso{Dãy số $1$; $n$; $n^2$; $n^3$; $ n^4 $; $\ldots$ (với $ n>1 $) là một cấp số nhân với công bội $ q=n $.}
	\loigiai{
		Ta có $ \dfrac{u_2}{u_1}=\dfrac{u_3}{u_2}=\ldots=n $ (với $ n>1 $) là một số không đổi.\\
		Do đó dãy số $1$; $n$; $n^2$; $n^3$; $ n^4 $; $\ldots$ (với $ n>1 $) là một cấp số nhân với công bội $ q=n $.
	}
\end{bt}
\begin{bt}%[TH]%[DCHT Toán 11 - KNTT -Tên Huỳnh Thanh Chí]%[1K2B7-1]
	Cho dãy số $ (u_n) $ được xác định bởi $ \heva{& u_1=2\\ & u_{n+1}=u_n^2} $. Hỏi dãy số $ (u_n) $ có là một cấp số nhân hay không?
	\dapso{Dãy số $ (u_n) $ không là một cấp số nhân.}
	\loigiai{
	Ta có $ u_2=u_1^2=4,u_3=u_2^2=16,u_4=u_3^2=256 $.\\
	Suy ra $ \dfrac{u_2}{u_1}=2 $; $ \dfrac{u_3}{u_2}=4 $ và $ \dfrac{u_4}{u_3}=16 $. Vì $ \dfrac{u_2}{u_1}\ne \dfrac{u_3}{u_2}\ne \dfrac{u_4}{u_3} $ nên $ (u_n) $ không là một cấp số nhân	
	}
\end{bt}
\begin{bt}%[TH]%[DCHT Toán 11 - KNTT -Tên Huỳnh Thanh Chí]%[1K2B7-1]
	Cho dãy số $ (u_n) $, biết $ u_1=2 $ và $ u_{n+1}=\dfrac{1}{3}u_n $. Chứng minh $ (u_n) $ là một cấp số nhân và tìm số hạng $ u_3 $.
	\dapso{}
	\loigiai{
	Ta có $ u_{n+1}=\dfrac{1}{3}u_n\Rightarrow \dfrac{u_{n+1}}{u_n}=\dfrac{1}{3} $ là một số không đổi nên $ (u_n) $ là một cấp số nhân với công bội là $ q=\dfrac{1}{3} $.\\
	Do đó $ u_3=u_2\cdot q=u_1\cdot q^2=2\cdot \dfrac{1}{3^2}=\dfrac{2}{9} $.
	}
\end{bt}
\begin{bt}%[TH]%[DCHT Toán 11 - KNTT -Tên Huỳnh Thanh Chí]%[1K2B7-1]
	Cho $ (u_n) $ là cấp số nhân có công bội $ q\ne 0,u_1\ne 0 $. Chứng minh rằng dãy số $ (v_n) $ với $ v_n=\dfrac{u_nu_{2n+1}}{4} $ cũng là một cấp số nhân.
	\dapso{$ (v_n) $ là một cấp số nhân với công bội là $ q^3 $.}
	\loigiai{
		Ta có $ \dfrac{v_n}{v_{n-1}}=\dfrac{\dfrac{u_nu_{2n+1}}{4}}{\dfrac{u_{n-1}u_{2(n-1)+1}}{4}}=\dfrac{u_1q^{n-1}\cdot u_1q^{2n}}{u_1q^{n-2}\cdot u_1q^{2n-2}}=q^3 $. Do đó $ (v_n) $ là một cấp số nhân với công bội là $ q^3 $.}
\end{bt}
\begin{bt}%[VD]%[DCHT Toán 11 - KNTT -Tên Huỳnh Thanh Chí]%[1K2K7-1]
	Cho dãy số $ (u_n) $ được xác định bởi $ \heva{& u_1=3\\ & u_{n+1}=2u_n-2},\forall n\in\mathbb{N}^* $. Chứng minh rằng dãy số $ (v_n) $ xác định bởi $ v_n=2u_n-4,\forall n\in\mathbb{N}^* $ là một cấp số nhân. Hãy xác định số hạng đầu và công bội của cấp số nhân đó.
	\dapso{$ (v_n) $ là cấp số nhân với công bội $ q=2 $ và số hạng đầu $ v_1=2 $.}
	\loigiai{
	Ta có $ v_n=2u_n-4 $ (1) và $ v_{n+1}=2u_{n+1}-4 $ (2).\\
	Theo đề ta có $ u_{n+1}=2u_n-2 \Rightarrow 2u_{n+1}-4=2(2u_n-4) $ (3).\\
	Thay (1) và (2)	vào (3) ta được $ v_{n+1}=2v_n \Rightarrow \dfrac{v_{n+1}}{v_n}=2,\forall n\in\mathbb{N}^* $.\\
	Suy ra $ (v_n) $ là cấp số nhân với công bội $ q=2 $ và số hàng đầu $ v_1=2u_1-4=2\cdot 3-4=2 $.
	}
\end{bt}
\subsubsection{Câu hỏi trắc nghiệm}
\Opensolutionfile{ans}[ans/ans-1K2-3-dang1]
\begin{ex}%[DCHT Toán 11 - KNTT -Tên Huỳnh Thanh Chí]%[1K2Y7-1]
	Trong các dãy số sau, dãy số nào là một cấp số nhân?
	\choice
	{\True $128$; $-64$; $32$; $-16$; $8$; $\ldots$} 
	{$\sqrt{2}$; $2$; $4$; $4\sqrt{2}$; $\ldots$}
	{$5$; $6$; $7$; $8$; $\ldots$}
	{$15$; $5$; $1$; $\dfrac{1}{5}$; $\ldots$}
	\loigiai{
	Xét phương án $128$; $-64$; $32$; $-16$; $8$; $\ldots$. \\
	Có $ \dfrac{u_2}{u_1}=\dfrac{u_3}{u_2}=\ldots=-\dfrac{1}{2} $ là một số không đổi nên dãy số $128$; $-64$; $32$; $-16$; $8$; $\ldots$ là một cấp số nhân.
	}
\end{ex}
\begin{ex}%[DCHT Toán 11 - KNTT -Tên Huỳnh Thanh Chí]%[1K2Y7-1]
	Dãy số nào sau đây \textbf{không phải} là cấp số nhân?
	\choice
	{$1$; $-1$; $1$; $-1$; $\ldots$}
	{$3$; $3^2$; $3^3$; $3^4$; $\ldots$}
	{$a$; $a^3$; $a^5$; $a^7$; $\ldots$  $(a\not =0)$}
	{\True $\dfrac{1}{\pi}$; $\dfrac{1}{{\pi}^2}$; $\dfrac{1}{{\pi}^4}$; $\dfrac{1}{{\pi}^6}$; $\ldots$}
	\loigiai{
	Xét dãy $\dfrac{1}{\pi}$; $\dfrac{1}{{\pi}^2}$; $\dfrac{1}{{\pi}^4}$; $\dfrac{1}{{\pi}^6}$; $\ldots$ có $ \dfrac{u_2}{u_1}\ne \dfrac{u_3}{u_2} \left(\dfrac{1}{\pi}\ne\dfrac{1}{\pi^2} \right) $.\\
	Do đó dãy $\dfrac{1}{\pi}$; $\dfrac{1}{{\pi}^2}$; $\dfrac{1}{{\pi}^4}$; $\dfrac{1}{{\pi}^6}$; $\ldots$ không là một cấp số nhân.
	}
\end{ex}
\begin{ex}%[DCHT Toán 11 - KNTT -Tên Huỳnh Thanh Chí]%[1K2Y7-1]
	Dãy số $1$; $2$; $4$; $8$; $16$; $32$; $\ldots$ là một cấp số nhân với 
	\choice
	{Công bội là $1$ và số hạng đầu tiên là $2$}
	{\True Công bội là $2$ và số hạng đầu tiên là $1$}
	{Công bội là $2$ và số hạng đầu tiên là $2$}
	{Công bội là $1$ và số hạng đầu tiên là $1$}
	\loigiai{
	Ta có $ q=\dfrac{u_2}{u_1}=\dfrac{u_3}{u_2}=\ldots=2 $. \\
	Vậy dãy số đã cho là một cấp số nhân với công bội là $ q=2 $ và số hạng đầu tiên là $ u_1=1 $.
	}
\end{ex}
\begin{ex}%[DCHT Toán 11 - KNTT -Tên Huỳnh Thanh Chí]%[1K2Y7-1]
	Cho cấp số nhân $(u_n)$ với $u_1=-2$ và công bội $q=-5$. Viết bốn số hạng đầu tiên của cấp số nhân.
	\choice
	{$-2$; $10$; $50$; $-250$}
	{\True $-2$; $10$; $-50$; $250$}
	{$-2$; $-10$; $-50$; $-250$}
	{$-2$; $10$; $50$; $250$}
	\loigiai{
	Vì $ (u_n) $ là một cấp số nhân nên ta có $ u_{n+1}=u_nq $. \\
	Do đó $ u_2=u_1q=(-2)\cdot (-5)=10 $, $ u_3=u_2q=10\cdot (-5)=-50 $, $ u_4=u_3q=(-50)\cdot (-5)=250 $.\\
	Vậy bốn số hạng đầu tiên của cấp số nhân đó là $ -2; 10; -50; 250 $.
	}
\end{ex}
\begin{ex}%[DCHT Toán 11 - KNTT -Tên Huỳnh Thanh Chí]%[1K2B7-1]
	Một cấp số nhân có hai số hạng liên tiếp là $3$ và $12$. Số hạng tiếp theo của cấp số nhân là
	\choice
	{$15$}
	{$21$}
	{$36$}
	{\True $48$}
	\loigiai{
		Một cấp số nhân có hai số hạng liên tiếp là $3$ và $12$, do đó ta có $ q=\dfrac{u_{n+1}}{u_n}=\dfrac{12}{3}=4 $.\\
		Vậy số hạng tiếp theo của cấp số nhân đó là $ u_{n+2}=u_{n+1}q=12\cdot 4=48 $.
	}
\end{ex}
\begin{ex}%[DCHT Toán 11 - KNTT -Tên Huỳnh Thanh Chí]%[1K2B7-1]
	Cho cấp số nhân $(u_n)$ có số hạng tổng quát là $u_n=\dfrac{3}{2}\cdot 5^n$. Khi đó số hạng đầu $u_1$ và công bội $q$ là
	\choice
	{$u_1=\dfrac{3}{2}, q=\dfrac{1}{5}$} 
	{$u_1=\dfrac{3}{2}, q=5$}
	{$u_1=\dfrac{15}{2}, q=\dfrac{1}{5}$}
	{\True $u_1=\dfrac{15}{2}, q=5$}
	\loigiai{	
	Ta có $ u_1=\dfrac{3}{2}\cdot5^1=\dfrac{15}{2}$ và $ u_2=\dfrac{3}{2}\cdot 5^2=\dfrac{75}{2} $.\\
	Vì $ (u_n) $ là một cấp số nhân nên $ q=\dfrac{u_2}{u_1}=\dfrac{75}{2}:\dfrac{15}{2}=5 $.
	}
\end{ex}
\begin{ex}%[DCHT Toán 11 - KNTT -Tên Huỳnh Thanh Chí]%[1K2B7-1]
	Trong các dãy số $(u_n)$ cho bởi số hạng tổng quát $u_n$ sau, dãy số nào là một cấp số nhân?
	\choice
	{\True $u_n=\dfrac{1}{3^{n-2}}$}
	{$u_n=\dfrac{n}{3^n}$}
	{$u_n=(n+2)\cdot 3^n$}
	{$u_n=n^2$}
	\loigiai{
		\begin{itemize}
			\item Với $u_n=\dfrac{1}{3^{n-2}}$, ta có $ q=\dfrac{u_{n+1}}{u_n}=\dfrac{1}{3^{n-3}}:\dfrac{1}{3^{n-2}}=3 $ là một số không đổi.\\
			Vậy dãy số $ (u_n) $ có số hạng tổng quát $u_n=\dfrac{1}{3^{n-2}}$ là một cấp số nhân.
			\item Với $u_n=\dfrac{n}{3^n}$, ta có $ q=\dfrac{u_{n+1}}{u_n}=\dfrac{n+1}{3^{n+1}}:\dfrac{n}{3^n}=\dfrac{n+1}{3n} $ không phải là một số không đổi.\\
			Vậy dãy số $ (u_n) $ có số hạng tổng quát $u_n=\dfrac{n}{3^n}$ không là một cấp số nhân.
			\item Với $u_n=(n+2)\cdot 3^n$, ta có $ q=\dfrac{u_{n+1}}{u_n}=\dfrac{(n+3)\cdot 3^{n+1}}{(n+2)\cdot 3^n}=\dfrac{3(n+3)}{n+2} $ không phải là một số không đổi.\\
			Vậy dãy số $ (u_n) $ có số hạng tổng quát $u_n=(n+2)\cdot 3^n$ không là một cấp số nhân.
			\item Với $u_n=n^2$, ta có $ q=\dfrac{u_{n+1}}{u_n}=\dfrac{(n+1)^2}{n^2}=\left(1+\dfrac{1}{n}\right)^2 $ không là một số không đổi.\\
			Vậy dãy số $ (u_n) $ có số hạng tổng quát $u_n=n^2$ không là một cấp số nhân.
		\end{itemize}
	}
\end{ex}
\begin{ex}%[DCHT Toán 11 - KNTT -Tên Huỳnh Thanh Chí]%[1K2B7-1]
	Trong các dãy số $(u_n)$ cho bởi số hạng tổng quát $u_n$ sau, dãy số nào là một cấp số nhân?
	\choice
	{$u_n=7-3n$}
	{$u_n=7-3^n$}
	{$u_n=\dfrac{7}{3n}$}
	{\True $u_n=7\cdot 3^n$}
	\loigiai{
	\begin{itemize}
		\item Với $u_n=7-3n$, ta có $ q=\dfrac{u_{n+1}}{u_n}=\dfrac{7-3(n+1)}{7-3n}=\dfrac{4-3n}{7-3n} $ không phải là một số không đổi.\\
		Vậy dãy số $ (u_n) $ có số hạng tổng quát $u_n=7-3n$ không là một cấp số nhân.
		\item Với $u_n=7-3^n$, ta có $ q=\dfrac{u_{n+1}}{u_n}=\dfrac{7-3^{n+1}}{7-3^n}=\dfrac{7-3\cdot 3^n}{7-3^n}=1-\dfrac{2\cdot 3^n}{7-3^n} $ không phải là một số không đổi.\\
		Vậy dãy số $ (u_n) $ có số hạng tổng quát $u_n=7-3^n$ không là một cấp số nhân.
		\item Với $u_n=\dfrac{7}{3n}$, ta có $ q=\dfrac{u_{n+1}}{u_n}=\dfrac{7}{3(n+1)}:\dfrac{7}{3n}=\dfrac{n}{n+1} $ không phải là một số không đổi.\\
		Vậy dãy số $ (u_n) $ có số hạng tổng quát $u_n=\dfrac{7}{3n}$ không là một cấp số nhân.
		\item Với $u_n=7\cdot 3^n$, ta có $ q=\dfrac{u_{n+1}}{u_n}=\dfrac{7\cdot 3^{n+1}}{7\cdot 3^n}=3 $ là một số không đổi.\\
		Vậy dãy số $ (u_n) $ có số hạng tổng quát $u_n=7\cdot 3^n$ là một cấp số nhân.
	\end{itemize}
	}
\end{ex}
\begin{ex}%[DCHT Toán 11 - KNTT -Tên Huỳnh Thanh Chí]%[1K2B7-1]
	Mệnh đề nào sau đây \textbf{sai}?
	\choice
	{Dãy số có tất cả các số hạng bằng nhau là một cấp số nhân}
	{Dãy số có tất cả các số hạng bằng nhau là một cấp số cộng}
	{Một cấp số cộng có công sai dương là một dãy số tăng}
	{\True Một cấp số nhân có công bội $q>1$ là một dãy tăng}
	\loigiai{
	\begin{itemize}
		\item Dãy số có tất cả các số hạng bằng nhau là một cấp số nhân là mệnh đề đúng. \\
		Vì xét dãy số $ (u_n) $ là một cấp số nhân.
		Khi đó $ u_{n+1}=u_n\cdot q $ với $ u_n\ne 0,q=1 $ thì $ u_{n+1}=u_n $.
		\item Dãy số có tất cả các số hạng bằng nhau là một cấp số cộng là mệnh đề đúng.\\
		Vì $ u_{n+1}=u_n+d $, với $ d=0 $ thì $ u_{n+1}=u_n $.
		\item Một cấp số cộng có công sai dương là một dãy số tăng là mệnh đề đúng.\\
		Ta xét dãy $ (u_n) $ là một cấp số cộng có công sai $ d>0 $.\\
		Vì $ u_{n+1}=u_n+d \Rightarrow u_{n+1}-u_n=d>0 $. \\
		Do đó dãy $ (u_n) $ là dãy số tăng.
		\item Một cấp số nhân có công bội $q>1$ là một dãy tăng là mệnh đề \textbf{sai}.\\
		Ta xét dãy số $ (u_n) $ là một cấp số nhân có công bội $ q>1 $.\\
		Vì $ u_{n+1}=u_nq $ với $ u_n\ne 0,q>1 $. Khi đó $ u_{n+1}-u_n=u_nq-u_n=u_n(q-1) $.\\
		Nếu $ u_n<0 $ thì $ u_{n+1}-u_n=u_nq-u_n=u_n(q-1) <0 $. \\
		Do đó dãy $ (u_n) $ là dãy số giảm.
	\end{itemize}	
	}
\end{ex}
\begin{ex}%[DCHT Toán 11 - KNTT -Tên Huỳnh Thanh Chí]%[1K2K7-1]
	Cho dãy số $(u_n)$ được xác định bởi $ u_1=2,u_n=2u_{n-1}+3n-1 $. Công thức số hạng tổng quát của dãy số đã cho là biểu thức có dạng $ a2^n+bn+c $, với $ a,b,c\in\mathbb{Z},n\ge 2, n\in\mathbb{N} $. Khi đó tổng $ a+b+c $ có giá trị bằng
	\choice
	{$ -4 $}
	{$ 4 $}
	{\True $ -3 $}
	{$ 3 $}
	\loigiai{
	Ta có $ u_n=2u_{n-1}+3n-1 \Leftrightarrow u_n+3n+5=2\left[u_{n-1}+3(n-1)+5\right] $ với $ n\ge 2, n\in\mathbb{N} $.\\
	Đặt $ v_n=u_n+3n+5 $, ta có $ v_n=2v_{n-1} $ với $ n\ge 2, n\in\mathbb{N} $.\\
	Như vậy $ (v_n) $ là cấp số nhân với công bội $ q=2 $ và $ v_1=10 $.\\
	Do đó $ v_n=10\cdot 2^{n-1}=5\cdot 2^n $.\\
	Suy ra $ u_n+3n+5=5\cdot 2^n $ hay $ u_n=5\cdot 2^n-3n-5 $ với $ n\ge 2, n\in\mathbb{N} $.\\
	Vậy $ a=5,b=-3,c=-5 $, suy ra $ a+b+c=-3 $.
	}
\end{ex}
\Closesolutionfile{ans}
% \begin{indapan}{10}
% 	{ans/ans-1K2-3-dang1}
% \end{indapan}
\begin{dang}{Số hạng tổng quát của cấp số nhân}
	Dựa vào giả thuyết, ta lập một hệ phương trình chứa công bội $ q $ và số hạng đầu $ u_n $. Giải hệ phương trình này tìm được $ u_1 $ và $ q $.\\
	Nếu cấp số nhân $ (u_n) $ có số hạng đầu $ u_1 $ và công bội $ q $ thì số hạng tổng quát $ u_n $ được xác định bởi công thức $$ u_n=u_1\cdot q^{n-1} \text{ với } n\ge 2. $$
\end{dang}
\subsubsection{Ví dụ minh hoạ}
\begin{vd}%[NB]%[DCHT Toán 11 - KNTT -Tên Huỳnh Thanh Chí]%[1K2Y7-2]
	Tìm số hạng tổng quát của dãy số $ 2;4;8;16;32;\ldots $, biết dãy $ (u_n) $ là một cấp số nhân.  
	\dapso{$ u_n=2\cdot 2^{n-1} $.}
	\loigiai{
		Vì dãy số $ (u_n) $ là một cấp số nhân nên $ q=\dfrac{u_2}{u_1}=\dfrac{u_3}{u_2}=\ldots=2 $ và số hạng đầu $ u_1=2 $.\\
		Do đó dãy số $ 2;4;8;16;32;\ldots $ là một cấp số nhân có số hạng tổng quát là $ u_n=u_1q^{n-1}=2\cdot 2^{n-1} $.
	}
\end{vd}
\begin{vd}%[TH]%[DCHT Toán 11 - KNTT -Tên Huỳnh Thanh Chí]%[1K2B7-2]
	Tìm số hạng đầu, công bội và số hạng tổng quát của cấp số nhân, biết $ \heva{& u_1+u_5=51\\ & u_2+u_6=102.} $
	\dapso{$ u_1=3 $, $ q=2 $ và $ u_n=3\cdot 2^{n-1} $.}
	\loigiai{
	Vì $ (u_n) $ là một cấp số nhân nên $ u_n=u_1\cdot q^{n-1} $.\\
	Ta có $ \heva{& u_1+u_5=51\\ & u_2+u_6=102}\Leftrightarrow\heva{& u_1+u_1q^4=51\\ & u_1q+u_1q^5=102}\Leftrightarrow\heva{& u_1(1+q^4)=51 \qquad (1)\\ & u_1q(1+q^4)=102 \qquad (2).} $\\
	Chia từng vế của $ (2) $ cho $ (1) $ ta được $ \dfrac{u_1q(1+q^4)}{u_1(1+q^4)}=\dfrac{102}{51} \Leftrightarrow q=2 $.\\
	Suy ra $ u_1=\dfrac{51}{1+q^4}=\dfrac{51}{17}=3 $, $ u_n=u_1\cdot q^{n-1}=3\cdot 2^{n-1} $.\\
	Vậy $ u_1=3 $, $ q=2 $ và $ u_n=3\cdot 2^{n-1} $.
	}
\end{vd}
\begin{vd}%[TH]%[DCHT Toán 11 - KNTT -Tên Huỳnh Thanh Chí]%[1K2B7-2]
	Tìm số hạng đầu, công bội và số hạng tổng quát của cấp số nhân, biết $ \heva{& u_1+u_6=30\\ & u_2+u_7=120.} $
	\dapso{$ u_1=\dfrac{6}{205} $, $ q=4 $ và $ u_n=\dfrac{6}{205}\cdot 4^{n-1} $.}
	\loigiai{
		Vì $ (u_n) $ là một cấp số nhân nên $ u_n=u_1\cdot q^{n-1} $.\\
		Ta có $ \heva{& u_1+u_6=30\\ & u_2+u_7=120}\Leftrightarrow\heva{& u_1+u_1q^5=30\\ & u_1q+u_1q^6=102}\Leftrightarrow\heva{& u_1(1+q^5)=30 \qquad (1)\\ & u_1q(1+q^5)=120 \qquad (2).} $\\
		Chia từng vế của $ (2) $ cho $ (1) $ ta được $ \dfrac{u_1q(1+q^5)}{u_1(1+q^5)}=\dfrac{120}{30} \Leftrightarrow q=4 $.\\
		Suy ra $ u_1=\dfrac{30}{1+q^5}=\dfrac{30}{1+4^5}=\dfrac{6}{205} $, $ u_n=u_1\cdot q^{n-1}=\dfrac{6}{205}\cdot 4^{n-1} $.\\
		Vậy $ u_1=\dfrac{6}{205} $, $ q=4 $ và $ u_n=\dfrac{6}{205}\cdot 4^{n-1} $.
	}
\end{vd}
\begin{vd}%[TH]%[DCHT Toán 11 - KNTT -Tên Huỳnh Thanh Chí]%[1K2B7-2]
	Tìm số hạng đầu, công bội và số hạng tổng quát của cấp số nhân, biết $ \heva{& u_3=40\\ & u_6=160.} $
	\dapso{$ u_1=\dfrac{40}{9} $, $ q=3 $ và $ u_n=40\cdot 3^{n-3} $.}
	\loigiai{
	Vì $ (u_n) $ là một cấp số nhân nên $ u_n=u_1\cdot q^{n-1} $.\\	
	Ta có $ \heva{& u_3=40\\ & u_6=1080}\Leftrightarrow \heva{& u_1q^2=40 \qquad (1)\\ & u_1q^5=1080\qquad (2).} $\\
	Chia từng vế của $ (2) $ cho $ (1) $ ta được $ \dfrac{u_1q^5}{u_1q^2}=\dfrac{1080}{40} \Leftrightarrow q^3=27 \Leftrightarrow q=3 $.\\
	Suy ra $ u_1=\dfrac{40}{q^2}=\dfrac{40}{3^2}=\dfrac{40}{9} $, $ u_n=u_1\cdot q^{n-1}=\dfrac{40}{9}\cdot 3^{n-1}=40\cdot 3^{n-3} $.\\
	Vậy $ u_1=\dfrac{40}{9} $, $ q=3 $ và $ u_n=40\cdot 3^{n-3} $.
	}
\end{vd}
\begin{vd}[VDT]%[DCHT Toán 11 - KNTT -Tên Huỳnh Thanh Chí]%[1K2K7-2]
	Tìm số hạng đầu, công bội và số hạng tổng quát của cấp số nhân có công bội $ q\in \mathbb{Z},q\ne 0 $, biết $ \heva{& u_2+u_4=10\\ & u_1+u_3+u_5=-21.} $
	\dapso{$ u_1=-1 $, $ q=-2 $ và $ u_n=\dfrac{(-2)^n}{2} $.}
	\loigiai{
		Vì $ (u_n) $ là một cấp số nhân nên $ u_n=u_1\cdot q^{n-1} $ với $ q\in \mathbb{Z},q\ne 0 $.\\
		Ta có $ \heva{& u_2+u_4=10\\ & u_1+u_3+u_5=-21}\Leftrightarrow\heva{& u_1q+u_1q^3=10\\ & u_1q+u_1q^2+u_1q^4=-21}\Leftrightarrow\heva{& u_1(q+q^3)=10 \qquad (1)\\ & u_1(1+q^2+q^4)=-21 \qquad (2).} $\\
		Chia từng vế của $ (2) $ cho $ (1) $ ta được 
		\allowdisplaybreaks
		\begin{eqnarray*}
			 \dfrac{u_1(1+q^2+q^4)}{u_1(q+q^3)}=\dfrac{-21}{10} 
			 &\Leftrightarrow& 10q^4+21q^3+10q^2+21q+10=0 \\
			 &\Leftrightarrow& (q+2)(2q+1)(5q^2-2q+5)=0 \\
			 &\Leftrightarrow& \hoac{& q=-2 \text{ (thỏa mãn)}\\ & q=-\dfrac{1}{2} \text{ (loại)}.}
		\end{eqnarray*}
		Suy ra $ u_1=\dfrac{10}{q+q^3}=-1 $, $ u_n=u_1\cdot q^{n-1}=(-1)\cdot (-2)^{n-1}=-(-2)^{n-1}=-\dfrac{(-2)^n}{-2}=\dfrac{(-2)^n}{2} $.\\
		Vậy $ u_1=-1 $, $ q=-2 $ và $ u_n=\dfrac{(-2)^n}{2} $.
	}
\end{vd}
\subsubsection{Bài tập tự luận}
 
% \begin{bt}%[NB]%[DCHT Toán 11 - KNTT -Tên Huỳnh Thanh Chí]%[1K2Y7-2]
% 	Tìm số hạng thứ $ 100 $ của cấp số nhân $ 8;-4;2;-1;\ldots $
% 	\dapso{$ u_{100}=-\dfrac{1}{2^{96}} $.}
% 	\loigiai{
% 	Cấp số nhân này có số hạng đầu $ u_1=8 $ và công bội $ q=\dfrac{-4}{8}=-\dfrac{1}{2} $.\\
% 	Do đó số hạng tổng quát $ u_n=8\cdot \left(-\dfrac{1}{2}\right)^{n-1} $.\\
% 	Vậy $ u_{100}=8\cdot \left(-\dfrac{1}{2}\right)^{100-1}=8\cdot \left(-\dfrac{1}{2}\right)^{99}=-\dfrac{1}{2^{96}} $.
% 	}
% \end{bt}
\begin{bt}%[NB]%[DCHT Toán 11 - KNTT -Tên Huỳnh Thanh Chí]%[1K2B7-2]
	Tìm số hạng tổng quát của dãy số $ 3;12;48;192;\ldots $, biết dãy $ (u_n) $ là một cấp số nhân.  
	\dapso{$ u_n=3\cdot 4^{n-1} $.}
	\loigiai{
		Vì dãy số $ (u_n) $ là một cấp số nhân nên $ q=\dfrac{u_2}{u_1}=\dfrac{12}{3}=4 $ và số hạng đầu $ u_1=3 $.\\
		Do đó dãy số $ 3;12;48;192;\ldots $ là một cấp số nhân có số hạng tổng quát là $ u_n=u_1q^{n-1}=3\cdot 4^{n-1} $.}
\end{bt}
\begin{bt}%[TH]%[DCHT Toán 11 - KNTT -Tên Huỳnh Thanh Chí]%[1K2B7-2]
	Tìm số hạng tổng quát của cấp số nhân, biết $ \heva{& u_1+u_3=51\\ & u_2+u_4=153.} $
	\dapso{$ u_n=\dfrac{51}{10}\cdot 3^{n-1} $.}
	\loigiai{
		Vì $ (u_n) $ là một cấp số nhân nên $ u_n=u_1\cdot q^{n-1} $.\\
		Ta có $ \heva{& u_1+u_3=51\\ & u_2+u_4=153}\Leftrightarrow\heva{& u_1+u_1q^2=51\\ & u_1q+u_1q^3=153}\Leftrightarrow\heva{& u_1(1+q^2)=51 \qquad (1)\\ & u_1q(1+q^2)=153 \qquad (2).} $\\
		Chia từng vế của $ (2) $ cho $ (1) $ ta được $ \dfrac{u_1q(1+q^2)}{u_1(1+q^2)}=\dfrac{153}{51} \Leftrightarrow q=3 $.\\
		Suy ra $ u_1=\dfrac{51}{1+q^2}=\dfrac{51}{10} $, $ u_n=u_1\cdot q^{n-1}=\dfrac{51}{10}\cdot 3^{n-1} $.\\
		Vậy số hạng tổng quát $ u_n=\dfrac{51}{10}\cdot 3^{n-1} $.
	}
\end{bt}
\begin{bt}%[TH]%[DCHT Toán 11 - KNTT -Tên Huỳnh Thanh Chí]%[1K2B7-2]
	Tìm số hạng đầu, công bội và số hạng tổng quát của cấp số nhân, biết $ \heva{& u_3=15\\ & u_6=120.} $
	\dapso{$ u_1=\dfrac{15}{4} $, $ q=2 $ và $ u_n=15\cdot 3^{n-3} $.}
	\loigiai{
		Vì $ (u_n) $ là một cấp số nhân nên $ u_n=u_1\cdot q^{n-1} $.\\	
		Ta có $ \heva{& u_3=15\\ & u_6=120}\Leftrightarrow \heva{& u_1q^2=15 \qquad (1)\\ & u_1q^5=120\qquad (2).} $\\
		Chia từng vế của $ (2) $ cho $ (1) $ ta được $ \dfrac{u_1q^5}{u_1q^2}=\dfrac{120}{15} \Leftrightarrow q^3=8 \Leftrightarrow q=2 $.\\
		Suy ra $ u_1=\dfrac{15}{q^2}=\dfrac{15}{2^2}=\dfrac{15}{4} $, $ u_n=u_1\cdot q^{n-1}=\dfrac{15}{4}\cdot 2^{n-1}=15\cdot 2^{n-3} $.\\
		Vậy $ u_1=\dfrac{15}{4} $, $ q=2 $ và $ u_n=15\cdot 3^{n-3} $.
	}
\end{bt}
\begin{bt}%[TH]%[DCHT Toán 11 - KNTT -Tên Huỳnh Thanh Chí]%[1K2B7-2]
	Tìm số hạng tổng quát của cấp số nhân, biết $ \heva{& u_4=35\\ & u_8=560.} $
	\dapso{$ u_n=35\cdot 2^{n-4} $ với $ q=2 $ hoặc $ u_n=35\cdot (-2)^{n-4} $ với $ q=-2 $.}
	\loigiai{
		Vì $ (u_n) $ là một cấp số nhân nên $ u_n=u_1\cdot q^{n-1} $.\\	
		Ta có $ \heva{& u_4=35\\ & u_8=560}\Leftrightarrow \heva{& u_1q^3=35 \qquad (1)\\ & u_1q^7=560\qquad (2).} $\\
		Chia từng vế của $ (2) $ cho $ (1) $ ta được $ \dfrac{u_1q^7}{u_1q^3}=\dfrac{560}{35} \Leftrightarrow q^4=16 \Leftrightarrow \hoac{& q=2\\ & q=-2.} $\\
		Với $ q=2 $. Suy ra $ u_1=\dfrac{35}{q^3}=\dfrac{35}{8} $, $ u_n=u_1\cdot q^{n-1}=\dfrac{35}{8}\cdot 2^{n-1}=35\cdot 2^{n-4} $.\\
		Với $ q=-2 $. Suy ra $ u_1=\dfrac{35}{q^3}=-\dfrac{35}{8} $, $ u_n=u_1\cdot q^{n-1}=-\dfrac{35}{8}\cdot (-2)^{n-1}=35\cdot (-2)^{n-4} $.\\
		Vậy $ u_n=35\cdot 2^{n-4} $ với $ q=2 $ hoặc $ u_n=35\cdot (-2)^{n-4} $ với $ q=-2 $.
	}
\end{bt}

\subsubsection{Câu hỏi trắc nghiệm}
\Opensolutionfile{ans}[ans/ans-1K2-3-dang2]
\begin{ex}%[DCHT Toán 11 - KNTT -Tên Huỳnh Thanh Chí]%[1K2Y7-2]
	Cho cấp số nhân $(u_n)$ có số hạng đầu là $u_1\ne 0$ và công bội $q\ne 0$. Số hạng tổng quát của cấp số nhân bằng
	\choice
	{$ u_{n}=u_1+(n-1)q $}
	{\True $ u_{n}=u_1\cdot q^{n-1} $}
	{$ u_{n}=u_1\cdot q^n $}
	{$ u_{n}=u_1\cdot q^{n+1} $}
	\loigiai{
	Số hạng tổng quát của cấp số nhân là $ u_{n}=u_1\cdot q^{n-1} $.
	}
\end{ex}
\begin{ex}%[DCHT Toán 11 - KNTT -Tên Huỳnh Thanh Chí]%[1K2Y7-2]
	Cấp số nhân  $\left(u_{n}\right)$ có $u_{n}=\dfrac{3}{5}\cdot 2^{n}$. Số hạng đầu tiên và công bội $ q $ là
	\choice
	{$u_1=\dfrac{6}{5},q=3$}
	{$u_1=\dfrac{6}{5},q=-2$}
	{\True $u_1=\dfrac{6}{5},q=2$}
	{$u_1=\dfrac{6}{5},q=5$}
	\loigiai{
	Ta có $u_{n}=\dfrac{3}{5}\cdot 2^{n}=\dfrac{6}{5}\cdot 2^{n-1}$, suy ra $ u_1=\dfrac{6}{5} $ và $ q=2 $.
	}
\end{ex}
\begin{ex}%[DCHT Toán 11 - KNTT -Tên Huỳnh Thanh Chí]%[1K2Y7-2]
	Cho cấp số nhân $(u_n)$ có $u_1=-3$ và công bội $q=\dfrac{2}{3}$. Chọn mệnh đề đúng?
	\choice
	{$u_5=-\dfrac{27}{16}$}
	{$u_5=-\dfrac{16}{27}$}
	{\True $u_5=\dfrac{16}{27}$}
	{$u_5=\dfrac{27}{16}$}
	\loigiai{
	Số hạng tổng quát của cấp số nhân là $ u_{n}=u_1\cdot q^{n-1}=3\cdot\left(\dfrac{2}{3}\right)^{n-1} $.\\
	Vậy $ u_5=3\cdot \left(\dfrac{2}{3}\right)^{5-1}=\dfrac{16}{27} $.
	}
\end{ex}
\begin{ex}%[DCHT Toán 11 - KNTT -Tên Huỳnh Thanh Chí]%[1K2Y7-2]
	Dãy số có số hạng tổng quát $u_{n}=\dfrac{1}{\sqrt{3}}^{2n}$ là một cấp số nhân có công bội $ q $ bằng
	\choice
	{$ \dfrac{1}{\sqrt{3}} $}
	{$ \sqrt{3} $}
	{$ \dfrac{1}{9} $}
	{\True $ \dfrac{1}{3} $}
	\loigiai{
		Ta có $u_{n}=\dfrac{1}{\sqrt{3}}^{2n}=\left[\left(\dfrac{1}{\sqrt{3}}\right)^2\right]^n=\left(\dfrac{1}{3}\right)^n=\dfrac{1}{3}\cdot\left(\dfrac{1}{3}\right)^{n-1} $.\\
		Suy ra công bội của cấp số nhân $ q=\dfrac{1}{3} $.
	}
\end{ex}
\begin{ex}%[DCHT Toán 11 - KNTT -Tên Huỳnh Thanh Chí]%[1K2Y7-2]
	Cho cấp số nhân $(u_n)$ có $u_1=1, u_2=-2$. Mệnh đề nào sau đây đúng?
	\choice
	{\True $u_{2024}=-2^{2023}$}
	{$u_{2024}=2^{2023}$}
	{$u_{2024}=-2^{2024}$}
	{$u_{2024}=2^{2024}$}
	\loigiai{
	Số hạng tổng quát của cấp số nhân là $ u_{n}=u_1\cdot q^{n-1}=\left(-2\right)^{n-1} $.\\
	Vậy $ u_{2024}=\left(-2\right)^{2024-1}=(-2)^{2023}=-2^{2023} $.
	}
\end{ex}
\begin{ex}%[DCHT Toán 11 - KNTT -Tên Huỳnh Thanh Chí]%[1K2B7-2]
	Cho cấp số nhân có $\heva{& u_4-u_2=54\\ & u_5-u_3=108}$. Số hạng đầu tiên $u_1$ và công bội $ q $ của cấp số nhân là
	\choice
	{\True $ u_1=9 $ và $ q=2 $ }
	{$ u_1=9 $ và $ q=-2 $}
	{$ u_1=-9 $ và $ q=2 $}
	{$ u_1=-9 $ và $ q=-2 $}
	\loigiai{
	Ta có $\heva{& u_4-u_2=54\\ & u_5-u_3=108}\Leftrightarrow \heva{& u_1q^3-u_1q=54\\ & u_1q^4-u_1q^2=108}\Leftrightarrow\heva{& u_1q(q^2-1)-54 \quad (1)\\ & u_1q^2(q^2-1)=108 \quad (2).} $\\
	Chia từng vế của $ (2) $ cho $ (1) $ ta được $ \dfrac{u_1q^2(q^2-1)}{u_1q(q^2-1)}=\dfrac{108}{54} \Leftrightarrow q=2 $.\\
	Suy ra $ u_1=\dfrac{54}{q^3-q}=\dfrac{54}{2^3-2}=9 $.
	}
\end{ex}
\begin{ex}%[DCHT Toán 11 - KNTT -Tên Huỳnh Thanh Chí]%[1K2B7-2]
	Cho cấp số nhân $\left( u_n\right)$ biết $\heva{& u_1+ u_2+ u_3=31\\& u_1+ u_3=26 }$. Giá trị $u_1$ và $ q $ là
	\choice
	{$ u_1=2; q=5 $ hoặc $u_1=25; q=\dfrac{1}{5}$}
	{$ u_1=5; q=1 $ hoặc $u_1=25; q=\dfrac{1}{5}$}
	{$ u_1=25; q=5 $ hoặc $u_1=1; q=\dfrac{1}{5}$}
	{\True$ u_1=1; q=5 $ hoặc $u_1=25; q=\dfrac{1}{5}$}
	\loigiai{
	Vì $ (u_n) $ là một cấp số nhân nên $ u_n=u_1\cdot q^{n-1} $.\\	
	Ta có $ \heva{& u_1+ u_2+ u_3=31\\& u_1+ u_3=26 }\Leftrightarrow\heva{& u_2=5\\ & u_1+u_3=26}\Leftrightarrow\heva{& u_1q=5 \quad (1)\\ & u_1(1+q^2)=26\quad (2).} $\\
	Chia từng vế của $ (2) $ cho $ (1) $ ta được $ \dfrac{q^2+1}{q}=\dfrac{26}{5} \Leftrightarrow 5q^2-26q+5=0 \Leftrightarrow \hoac{& q=5\\ & q=\dfrac{1}{5}.} $\\
	Với $ q=5 $. Suy ra $ u_1=\dfrac{5}{q}=\dfrac{5}{5}=1 $.\\
	Với $ q=\dfrac{1}{5} $. Suy ra $ u_1=\dfrac{5}{q}=5:\dfrac{1}{5}=25 $.\\
	Vậy $ u_1=1 $ với $ q=5 $ hoặc $ u_1=25 $ với $ q=\dfrac{1}{5} $.
	}
\end{ex}
\begin{ex}%[DCHT Toán 11 - KNTT -Tên Huỳnh Thanh Chí]%[1K2B7-2]
	Số hạng đầu tiên và công bội của cấp số nhân thỏa mãn $\heva{& u_5+u_2=36\\ & u_6-u_4=48}$ (với $ q>0 $) là
	\choice
	{$ u_1=4,q=4 $}
	{$ u_1=2,q=4 $}
	{\True$ u_1=2,q=2 $}
	{$ u_1=4,q=2 $}
	\loigiai{
	Ta có $\heva{& u_5+u_2=36\\ & u_6-u_4=48}\Leftrightarrow\heva{& u_1q^4+u_1q=36\\ & u_1q^5-u_1q^3=48}\Leftrightarrow\heva{& u_1q(q^3+1)=36\quad (1)\\ & u_1q(q^4-q^2)=48\quad (2).} $\\
	 Chia từng vế của $ (2) $ cho $ (1) $ ta được $$ \dfrac{u_1q(q^4-q^2)}{u_1q(q^3+1)}=\dfrac{48}{36}\Leftrightarrow \dfrac{q^4-q^2}{q^3+1}=\dfrac{4}{3}\Leftrightarrow 3q^4-4q^3-3q^2-4=0\hoac{& q=2\\ &q=-1.} $$
	 Từ điều kiện $ q>0 $ suy ra công bội của cấp số nhân là $ q=2 $, do đó $ u_1=\dfrac{36}{q^4+q}=2 $.\\
	 Vậy $ u_1=2 $ và $ q=2 $.
	}
\end{ex}
\begin{ex}%[DCHT Toán 11 - KNTT -Tên Huỳnh Thanh Chí]%[1K2B7-2]
	Cho cấp số nhân $u_2=\dfrac{1}{4},u_5=16$. Công bội và số hạng đầu tiên của cấp số nhân là
	\choice
	{$q=\dfrac{1}{2};u_1=\dfrac{1}{2}$}
	{$q=\dfrac{-1}{2};u_1=\dfrac{-1}{2}$}
	{\True $q=4;u_1=\dfrac{1}{16}$}
	{$q=-4;u_1=\dfrac{-1}{16}$}
	\loigiai{
	Ta có $ u_2=u_1q=\dfrac{1}{4} $ (1) và $ u_5=u_1q^4=16 $ (2).\\
	Lấy $ (2) $ chia cho $ (1) $ vế theo vế ta được $ \dfrac{u_1q^4}{u_1q}=\dfrac{16}{\tfrac{1}{4}} \Leftrightarrow q^3=64 \Leftrightarrow q=4 $.\\
	Suy ra $ u_1=\dfrac{1}{4}:q=\dfrac{1}{4}:4=\dfrac{1}{16} $.\\
	Vậy $ u_1=\dfrac{1}{16},q=4 $.
	}
\end{ex}
\begin{ex}%[DCHT Toán 11 - KNTT -Tên Huỳnh Thanh Chí]%[1K2K7-2]
	Người ta thiết kế một cái tháp gồm $ 11 $ tầng. Diện tích mặt trên của mỗi tầng bằng nửa diện tích mặt trên của tầng ngay bên dưới và diện tích mặt trên của tầng 1 bằng nửa diện tích của đế tháp (có diện tích là $12\ 288$ m$ ^2 $). Diện tích mặt trên cùng (của tầng thứ $ 11 $) có giá trị nào sau đây?
	\choice
	{\True$ 6 $ m$ ^2 $}
	{$ 8 $ m$ ^2 $}
	{$ 10 $ m$ ^2 $}
	{$ 12 $ m$ ^2 $}
	\loigiai{
	Vì diện tích của mặt trên của mỗi tầng bằng nửa diện tích mặt trên của tầng ngay bên dưới và diện tích mặt trên của tầng 1 bằng nửa diện tích của đế tháp.\\
	Do đó diện tích của mỗi tầng tạo nên dãy số và dãy số đó là một cấp số nhân có công bội $ q=\dfrac{1}{2} $.\\
	Vậy số hạng tổng quát của cấp số nhân đó là $ u_n=12\ 288\cdot \left(\dfrac{1}{2}\right)^{n-1} $.\\
	Vì từ đế tháp đến tầng thứ 11 của tháp sẽ có 12 mặt nền, do đó diện tích của mặt của tầng thứ 11 là $ u_{12}=12\ 288\cdot\left(\dfrac{1}{2}\right)^{12-1}=6 $ m$ ^2 $.
	}
\end{ex}
\Closesolutionfile{ans}
% \begin{indapan}{10}
% 	{ans/ans-1K2-3-dang2}
% \end{indapan}
\begin{dang}{Tìm số hạng cụ thể của CSN}
	Ta chuyển các số hạng của CSN về số hạng đầu $u_1$ và công bội $q$. Sử dụng công thức $u_n=u_1\cdot q^{n-1}$. \\
	Chia hai phương trình vế theo vế ta thu được phương trình theo $q$. \\
	Giải tìm $q$ và $u_1$. Từ đó tìm được số hạng cần tìm thỏa ycbt.
\end{dang}
\subsubsection{Ví dụ minh hoạ}
\begin{vd}%[NB]%[1K2Y3-3]
	Cho $u_n$ là CSN thỏa $u_1=2$; $u_4=16$. Tìm số hạng thứ $5$ của CSN.
	\loigiai{
		Do $u_n$ là CSN nên ta có $u_4=u_1\cdot q^3 \Rightarrow q^3=\dfrac{u_4}{u_1}=8 \Rightarrow q=2$. \\
		Vậy $u_5=u_1\cdot q^4=2\cdot 2^4=32$.
	}
\end{vd}
\begin{vd}%[TH]%[1K2B3-3]
	Cho cấp số nhân $(u_n)$ có $\heva{&u_4+u_6=-540 \\ &u_3+u_5=180}$. Tính số hạng đầu $u_1$ và công bội $q$ của cấp Số nhân.
	\loigiai{
		Ta có $\heva{&u_4 + u_6=-540 \\ &u_3+u_5=180}
		\Leftrightarrow \heva{&u_1q^3(1+q^2)=-540 \\ &u_1q^2(1+q^2)=180}
		\Leftrightarrow \heva{&u_1=2 \\ &q=-3.}$ \\
		Vậy $\heva{&u_1=2 \\ &q=-3}$ là số hạng cần tìm.
	}
\end{vd}
\begin{vd}%[TH]%[1K2B3-3]
	Cho cấp số nhân có $u_1=-3$, $q=\dfrac{2}{3}$. Số $\dfrac{-96}{243}$ là số hạng thứ mấy của cấp số nhân?
	\loigiai{
		Giả sử số $\dfrac{-96}{243}$ là số hạng thứ $n$ của cấp số nhân.\\
		Ta có: $u_1\cdot q^{n-1}=\dfrac{-96}{243}\Leftrightarrow(-3)\left(\dfrac{2}{3}\right)^{n-1}=\dfrac{-96}{243}\Leftrightarrow n=6$.\\
		Vậy số $\dfrac{-96}{243}$ là số hạng thứ $6$ của cấp số nhân.}
\end{vd}
\begin{vd}%[TH]%[1K2B3-3]
	Cấp số nhân $\left(u_{n}\right)$ có số hạng tổng quát là $u_n=\dfrac{3}{5} \cdot 2^{n-1}, n \in \mathbb{N}^*$. Số hạng đầu tiên và công bội của cấp số nhân đó là
	\loigiai{
		Ta có $u_{1}=\dfrac{3}{5} \cdot 2^{1-1}=\dfrac{3}{5}$ và $u_{2}=\dfrac{3}{5} \cdot 2^{2-1}=\dfrac{6}{5} \Rightarrow q=\dfrac{u_{2}}{u_{1}}=2$.\\
		Vậy $u_{1}=\dfrac{3}{5}$ và $q=2$.
	}
\end{vd}

\subsubsection{Bài tập tự luận}
 
\begin{bt}%[TH]%[1K2B3-3]
	Cho cấp số nhân $(u_n)$ biết $\heva{&u_4-u_2=25 \\ &u_3-u_1=50.}$
	\begin{enumEX}{1}
		\item Tìm số hạng đầu và công bội của cấp số nhân $(u_n)$.
		\item Tìm số hạng thứ $8$ của cấp số nhân $(u_n)$.
	\end{enumEX}
	\dapso{$\heva{&q=\dfrac{1}{2} \\ &u_1=-200}$, $u_8=-\dfrac{25}{16}$}
	\loigiai{
		\begin{enumerate}
			\item Ta có $\heva{&u_4-u_2=25 \\ &u_3-u_1=50} 
			\Leftrightarrow \heva{&u_1(q^3-q)=25 \\ &u_1(q^2-q)=50} 
			\Rightarrow \heva{&q=\dfrac{1}{2} \\ &u_1=-200.}$
			\item Ta có $u_8=u_1\cdot q^7=-200\cdot \dfrac{1}{2^7}=-\dfrac{25}{16}$.
		\end{enumerate}
	}
\end{bt}
\begin{bt}%[TH]%[1K2B3-3]
	Tìm số hạng thứ $10$ của cấp số nhân $(u_n)$ biết $\heva{&u_4-u_2=72 \\ &u_5-u_3=144.}$	
	\dapso{$u_{10}=6144$}
	\loigiai{
		Ta có $\heva{&u_4-u_2=72 \\ &u_5-u_3=144}\Leftrightarrow \heva{&u_4-u_2=72 \\ &q(u_4-u_2)=144}
		\Rightarrow \heva{&q=2 \\ &u_1(q^3-q)=72} \Leftrightarrow \heva{&q=2 \\ &u_1=12.}$ \\
		Khi đó $u_{10}=u_1\cdot q^9=6144$.
	}
\end{bt}
\begin{bt}%[TH]%[1K2B3-3]
	Cho một cấp số nhân có $5$ số hạng biết $2$ số hạng đầu là số dương, tích số hạng đầu và số hạng thứ $3$ là $1$, tích số hạng thứ $3$ và số hạng cuối là $\dfrac{1}{16}$. Tìm cấp số nhân này.	
	\dapso{$2; 1; \dfrac{1}{2}; \dfrac{1}{4}; \dfrac{1}{8}$}
	\loigiai{
		Gọi $5$ số hạng cần tìm có dạng $\dfrac{x}{q^2}$; $\dfrac{x}{q}$; $x$; $xq$; $xq^2$.\\
		Theo đề ra ta có $\heva{&\dfrac{x}{q^2}\cdot x=1 \\ &x\cdot xq^2=\dfrac{1}{16}} 
		\Leftrightarrow \heva{&x=\dfrac{1}{2} \\ &q=\dfrac{1}{2}}$ (do hai số hạng đầu dương nên $q>0$). \\
		Vậy $5$ số hạng cần tìm là $2; 1; \dfrac{1}{2}; \dfrac{1}{4}; \dfrac{1}{8}$.
	}
\end{bt}
\begin{bt}%[TH]%[1K2B3-3]
	Tìm số hạng đầu và công bội của cấp số nhân $(u_n)$ biết $\heva{&u_2+u_5-u_4=10 \\ &u_3+u_6-u_5=20.}$
	\dapso{$\heva{&q=2 \\ &u_1=1}$}
	\loigiai{
		Ta có $\heva{&u_2+u_5-u_4=10 \\ &u_3+u_6-u_5=20} 
		\Leftrightarrow \heva{&u_1(q+q^4-q^3)=10 \\ &u_1(q^2+q^5-q^4)=20}
		\Leftrightarrow \heva{&q=2 \\ &u_1=1.}$
	}
\end{bt}
\begin{bt}%[TH]%[1K2B3-4]
	Tìm $5$ số lập thành một cấp số nhân có công bội bằng $\dfrac{1}{4}$ số thứ nhất và tổng $2$ số đầu là $\dfrac{5}{4}$.
	\dapso{$1; \dfrac{1}{4}; \dfrac{1}{16}; \dfrac{1}{64}; \dfrac{1}{128}$ hoặc $-5; -\dfrac{5}{4}; -\dfrac{5}{16}; -\dfrac{5}{64}; -\dfrac{1}{128}$}
	\loigiai{
		Theo đề, ta có $\heva{&q=\dfrac{1}{4}u_1 \\ &u_1+u_2=\dfrac{5}{4}}
		\Leftrightarrow \heva{&q=\dfrac{1}{4}u_1 \\ &u_1+u_1\cdot q=\dfrac{5}{4}}
		\Leftrightarrow \heva{&q=\dfrac{1}{4}u_1 \\ &u_1^2+4u_1-5=0}
		\Leftrightarrow \heva{&q=\dfrac{1}{4} \\ &u_1=1}$ hoặc $\heva{&q=-\dfrac{5}{4} \\ &u_1=-5.}$\\
		Vậy có hai CSN là $1; \dfrac{1}{4}; \dfrac{1}{16}; \dfrac{1}{64}; \dfrac{1}{128}$ và $-5; -\dfrac{5}{4}; -\dfrac{5}{16}; -\dfrac{5}{64}; -\dfrac{1}{128}$.
	}
\end{bt}
\begin{bt}%[TH]%[1K2B3-4]
	Tìm $3$ số lập thành một cấp số nhân có tổng là $63$ và tích là $1728$.
	\dapso{$3; 12; 48$}
	\loigiai{
		Gọi ba số cần tìm là $\dfrac{x}{q}; x; xq$. Theo đề ra, ta có $x^3=1728\Rightarrow x=12$. \\
		Mặt khác $\dfrac{x}{q}+x+xq=63\Leftrightarrow 12q+12+\dfrac{12}{q}=63
		\Leftrightarrow 12q^2-51q+12=0 \Leftrightarrow \hoac{&q=4 \\ &q=\dfrac{1}{4}\cdot}$ \\
		Vậy CSN cần tìm là $3; 12; 48$.
	}
\end{bt}
\subsubsection{Câu hỏi trắc nghiệm}
\Opensolutionfile{ans}[ans/ans-1K2-3-dang3]
\begin{ex}%[1K2B3-3]
	Cho cấp số nhân $(u_n)$ có $u_{20}=8u_{17}$. Công bội của cấp số nhân là
	\choice
	{\True $q=2$}
	{$q=-2$}
	{$q=4$}
	{$q=-4$}
	\loigiai{
		Ta có $u_{20}=8u_{17}\Rightarrow u_1\cdot q^{19}=8\cdot u_1\cdot q^{16}\Rightarrow  q=2$.
	}
\end{ex}
\begin{ex}%[1K2B3-3]
	Cho cấp số nhân $\left(u_n\right)$ có $10$ số hạng với công bội $q\neq 0$ và $u_1\neq 0$. Đẳng thức nào sau đây là đúng?
	\choice
	{$u_7=u_4\cdot q^6$}
	{\True $u_7=u_4\cdot q^3$}
	{$u_7=u_4\cdot q^4$}
	{$u_7=u_4\cdot q^5$}
	\loigiai
	{Ta có $u_7=u_1\cdot q^6=\left(u_1\cdot q^3\right)\cdot q^3=u_4\cdot q^3$.
	}
\end{ex}
\begin{ex}%[1K2B3-3]
	Cho cấp số nhân $(u_n)$ có số hạng đầu $u_1=2$ và công bội $q=3$. Giá trị $u_{2019}$ bằng
	\choice
	{$3\cdot2^{2019}$}
	{$2\cdot3^{2019}$}
	{$3\cdot2^{2018}$}
	{\True $2\cdot3^{2018}$}
	\loigiai{
		Áp dụng công thức của số hạng tổng quát $u_n=u_1\cdot q^{n-1}=2\cdot 3^{2018}$.}
\end{ex}
\begin{ex}%[1K2B3-3]
	Cho cấp số nhân $(u_n)$ với công bội $q < 0$ và $u_2=4$, $u_4=9$. Tìm $u_1$.
	\choice
	{$u_1=6$}
	{\True $u_1=-\dfrac{8}{3}$}
	{$u_1=-6$}
	{$u_1=\dfrac{8}{3}$}
	\loigiai{
		Vì $q<0$, $u_2>0$ nên $u_3<0$. Do đó $u_3=-\sqrt{u_2\cdot u_4}=-\sqrt{4\cdot 9}=-6$.\\
		Ta có $u_2^2=u_1\cdot u_3\Rightarrow u_1=\dfrac{u_2^2}{u_3}=\dfrac{4^2}{-6}=-\dfrac{8}{3}$.
	}
\end{ex}
\begin{ex}%[1K2B3-3]
	Cho cấp số nhân $\left(u_{n}\right)$ có $u_{2}=-6, u_{3}=3$. Công bội $q$ của cấp số nhân đã cho bằng
	\choice
	{$2 $}
	{$\dfrac{1}{2}$}
	{\True $-\dfrac{1}{2}$}
	{$-2$}
	\loigiai{
		Công bội của cấp số nhân đã cho là $$q=\dfrac{u_3}{u_2}=-\dfrac{1}{2}.$$}
\end{ex}
\begin{ex}%[1K2Y3-3]
	Cho cấp số nhân có $u_1=-3$, $q=\dfrac{2}{3}$. Tính $u_5$?
	\choice
	{$u_5=\dfrac{27}{16}$}
	{\True $u_5=\dfrac{-16}{27}$}
	{$u_5=\dfrac{-27}{16}$}
	{$u_5=\dfrac{16}{27}$}
	\loigiai{
		Ta có: $u_5=u_1\cdot q^4=(-3)\left(\dfrac{2}{3}\right)^4=-\dfrac{16}{27}$.}
\end{ex}
\begin{ex}%[1K2Y3-3]
	Cho cấp số nhân $(u_n)$ có $u_2=\dfrac{1}{4}$; $u_5=-16$. Tìm $q$ và số hạng đầu tiên của cấp số nhân?
	\choice
	{$q=\dfrac{1}{2};u_1=\dfrac{1}{2}$}
	{$q=-\dfrac{1}{2},u_1=-\dfrac{1}{2}$}
	{\True $q=-4,u_1=\dfrac{1}{16}$}
	{$q=-4,u_1=-\dfrac{1}{16}$}
	\loigiai{
		Ta có $\heva{&u_2=\dfrac{1}{4}\\&u_5=16}\Rightarrow \heva{&u_1\cdot q=\dfrac{1}{4}\\&u_1\cdot q^4=-16}\Rightarrow q^3=-64\Rightarrow q=-4 \Rightarrow u_1=\dfrac{1}{16}$.
	}
\end{ex}
\begin{ex}%[1K2B3-3]
	Cho cấp số nhân $(u_n)$, biết: $u_n=81,u_{n+1}=9$. Lựa chọn đáp án đúng.
	\choice
	{$q=-\dfrac{1}{9}$}
	{\True $q=\dfrac{1}{9}$}
	{$q=9$}
	{$q=-9$}
	\loigiai{
		Ta có  $q=\dfrac{u_{n+1}}{u_n}=\dfrac{9}{81}=\dfrac{1}{9}$.
	}
\end{ex}
\begin{ex}%[1K2Y3-3]
	Cho cấp số nhân $\left( {u_n} \right)$ với $u_1=2$ và công bội $q=3$. Số hạng $u_2$ bằng
	\choice
	{$8$}
	{\True $6$}
	{$12$}
	{$18$}
	\loigiai{
		Ta có $u_2=u_1\cdot q=2\cdot 3=6$.}
\end{ex}
\begin{ex}%[1K2Y3-3]
	Cho cấp số nhân $(u_n)$ với $u_1=2$ và $u_3=8$. Số hạng thứ hai của cấp số nhân đã cho bằng
	\choice
	{$u_2=4$}
	{$u_2=6$}
	{\True $u_2=\pm 4$}
	{$u_2=-4$}
	\loigiai{
		Ta có $u_1 \cdot u_3=u_2^2 \Leftrightarrow u^2_2=16 \Leftrightarrow \hoac{&u_2=4\\&u_2=-4.}$
	}
\end{ex}
\begin{ex}%[1K2B3-3]
	Cho cấp số nhân $(u_n)$ có $u_1=-1; q=\dfrac{-1}{10}$. Số $\dfrac{1}{10^{103}}$ là số hạng thứ bao nhiêu?
	\choice
	{số hạng thứ $103$}
	{số hạng thứ $105$}
	{\True số hạng thứ $104$}
	{Đáp án khác}
	\loigiai{
		Ta có $u_n=u_1\cdot q^{n-1}\Leftrightarrow	\dfrac{1}{10^{103}}=-1\cdot \left(\dfrac{-1}{10}\right)^{n-1} \Leftrightarrow \left(\dfrac{-1}{10}\right)^{n-1}=\left(\dfrac{-1}{10}\right)^{103}\Rightarrow n=104$.
	}
\end{ex}
\begin{ex}%[1K2Y3-3]
	Cho cấp số nhân $\left(u_n\right)$ có các số hạng lần lượt là $3$, $9$, $27$, $81$,\ldots Khi đó $u_n$ bằng
	\choice
	{$3+3^n$}
	{$3^{n-1}$}
	{$3^{n+1}$}
	{\True $3^n$}
	\loigiai
	{Cấp số nhân đã cho có $u_1=3$ và công bội $q=3$ nên $u_n=u_1\cdot q^{n-1}=3\cdot 3^{n-1}=3^n$.
	}
\end{ex}
\begin{ex}%[1K2K3-3]
	Cho cấp số nhân $(u_n)$ có $u_1=3$ và $15u_1-4u_2+u_3$ đạt giá trị nhỏ nhất. Tìm số hạng thứ $13$ của cấp số nhân đã cho.
	\choice
	{\True $u_{13}=12288$}
	{$u_{13}=3072$}
	{$u_{13}=24567$}
	{$u_{13}=49152$}
	\loigiai{
		Gọi $q$ là công bội của cấp số nhân $(u_n)$.\\
		Ta có $15u_1-4u_2+u_3=45-12q+3q^2=3(q-2)^2+33\geq 33$ $\forall q \in \mathbb{R}$.\\
		Suy ra $15u_1-4u_2+u_3$ đạt giá trị nhỏ nhất khi $q=2$.\\
		Khi đó $u_{13}=u_1q^{12}=12288$.}
\end{ex}
\begin{ex}%[1K2B3-3]
	Cho cấp số nhân $(u_n)$ biết $u_1+u_5=51$ và $u_2+u_6=102$. Hỏi số $12288$ là số hạng thứ mấy của cấp số nhân $(u_n)$?
	\choice
	{\True Số hạng thứ $13$}
	{Số hạng thứ $10$}
	{Số hạng thứ $11$}
	{Số hạng thứ $12$}
	\loigiai{
		Gọi $q$ là công bội của cấp số nhân đã cho. Theo đề bài, ta có\\
		\centerline{$\heva{&u_1+u_5=51\\&u_2+u_6=102}\Leftrightarrow\heva{&u_1\left(1+q^4\right)=51\\&u_1q\left(1+q^4\right)=102}\Rightarrow q=2\Rightarrow u_1=3\Rightarrow u_n=3\cdot 2^{n-1.}$}
		Mặt khác $u_n=12288\Leftrightarrow 3\cdot 2^{n-1}=12288\Leftrightarrow 2^{n-1}=2^{12}\Leftrightarrow n=13$.
	}
\end{ex}
% \begin{ex}%[1K2K3-3]
% 	Một tứ giác lồi có số đo các góc lập thành một cấp số nhân. Biết rằng số đo của góc nhỏ nhất bằng $\dfrac{1}{9}$ số đo của góc nhỏ thứ ba. Hãy tính số đo của các góc trong tứ giác đó.
% 	\choice
% 	{$5^{\circ}$, $15^{\circ}$, $45^{\circ}$, $225^{\circ}$}
% 	{\True $9^{\circ}$, $27^{\circ}$, $81^{\circ}$, $243^{\circ}$}
% 	{$7^{\circ}$, $21^{\circ}$, $63^{\circ}$, $269^{\circ}$}
% 	{$8^{\circ}$, $32^{\circ}$, $72^{\circ}$, $248^{\circ}$}
% 	\loigiai{
% 		Gọi các góc của tứ giác là $a$, $aq$, $aq^2$, $aq^3,$ trong đó $q>1$.\\
% 		Theo giả thiết, ta có $a=\dfrac{1}{9}aq^2$ nên $q=3$.\\
% 		Suy ra các góc của tứ giác là $a$, $3a$, $9a$, $27a$.\\
% 		Vì tổng các góc trong tứ giác bằng $360^{\circ}$ nên ta có $a+3a+9a+27a=360^{\circ}\Leftrightarrow a=9^{\circ}$.\\
% 		Vậy số đo các góc trong tứ giác lần lượt là $9^{\circ}$, $27^{\circ}$, $81^{\circ}$, $243^{\circ}$. }
% \end{ex}
\Closesolutionfile{ans}
% \begin{indapan}{10}
% 	{ans/ans-1K2-3-dang3}
% \end{indapan}
\begin{dang}{Tìm điều kiện để một dãy số lập thành CSN}
	Dãy số $a, b, c$ lập thành CSN khi $b^2=a\cdot c$. \\
	Dãy số $a, b, c, d$ lập thành CSN khi $\heva{&b^2=a\cdot c \\ &c^2=b\cdot d.}$
\end{dang}
\subsubsection{Ví dụ minh hoạ}
\begin{vd}%[NB]%[1K2B3-4]
	Cho dãy $3,x,12,y$. Tìm $x,y$ để dãy là CSN.
	\loigiai{
		Dãy là CSN khi $\heva{&x^2=3\cdot 12 \\ &12^2=x\cdot y}\Leftrightarrow 
		\heva{&x=6 \\ &y=24}$ hoặc $\heva{&x=-6 \\ &y=-24.}$
	}
\end{vd}
\begin{vd}%[TH]%[1K2B3-4]
	Cho dãy  $x-1, 2x, 4x+3$. Tìm $x$ để dãy là CSN. 
	\loigiai{
		Dãy là CSN khi $(2x)^2=(x-1)(4x+3) \Leftrightarrow x=-3$.
	} 
\end{vd}
\begin{vd}%[VD]%[1K2K3-4]
	Các số $x+6y$, $5x+2y$, $8x+y$ theo thứ tự đó lập thành một cấp số cộng, đồng thời, các số $x+\dfrac{5}{3}$, $y-1$, $2x-3y$ theo thứ tự đó lập thành một cấp số nhân. Hãy tìm $x$ và $y$.
	\loigiai{
		\begin{itemize}
			\item Ba số $x+6y$, $5x+2y$, $8x+y$ lập thành cấp số cộng nên $(x+6y)+(8x+y)=2(5x+2y)\Leftrightarrow x=3y$.
			\item Ba số $x+\dfrac{5}{3}$, $y-1$, $2x-3y$ lập thành cấp số nhân nên $\left(x+\dfrac{5}{3}\right)(2x-3y)=\left(y-1\right)^2$.
		\end{itemize}
		Thay $x=3y$ vào ta được $8y^2+7y-1=0\Leftrightarrow y=-1$ hoặc $y=\dfrac{1}{8}$.\\
		Với $y=-1$ thì $x=-3$; với $y=\dfrac{1}{8}$ thì $x=\dfrac{3}{8}$.}
\end{vd}
\begin{vd}%[VD]%[1K2K3-4]
	Tìm tất cả các giá trị của tham số $m$ để phương trình sau có ba nghiệm phân biệt lập thành một cấp số nhân $x^3-7x^2+2\left(m^2+6m\right)x-8=0$.
	\loigiai{
		+ \textbf{Điều kiện cần:} \\
		Giả sử phương trình đã cho có ba nghiệm phân biệt $x_1$,$x_2$,$x_3$ lập thành một cấp số nhân.\\
		Theo định lý Vi-ét, ta có $x_1x_2x_3=8$.\\
		Theo tính chất của cấp số nhân, ta có $x_1x_3=x_2^2$. Suy ra  $x_2^3=8\Leftrightarrow x_2=2$.\\
		Với nghiệm $x=2$, ta có $m^2+6m-7=0\Leftrightarrow\hoac{&m=1\\&m=-7.}$ \\
		+ \textbf{Điều kiện đủ:}\\
		Với $m=1$ hoặc $m=-7$ thì $m^2+6m=7$.\\
		Khi đó phương trình ban đầu trở thành $x^3-7x^2+14x-8=0$.\\
		Giải phương trình này, ta được các nghiệm là $1$,$2$,$4$. Hiển nhiên ba nghiệm này lập thành một cấp số nhân với công bội $q=2$.\\
		Vậy $m=1$ và $m=-7$ là các giá trị cần tìm.}
\end{vd}
\begin{vd}%[VD]%[1K2K3-4]
	Các số $x+6y$, $5x+2y$, $8x+y$ theo thứ tự đó lập thành một cấp số cộng; đồng thời các số $x-1$, $y+2$, $x-3y$ theo thứ tự đó lập thành một cấp số nhân. Tính $x^2+y^2$.
	\loigiai{
		Theo giả thiết ta có
		\[
		\heva{&(x+6y)+(8x+y)=2(5x+2y)\\&(x-1)(x-3y)=\left(y+2\right)^2}\Leftrightarrow\heva{&x=3y\\&(3y-1)(3y-3y)=\left(y+2\right)^2}\Leftrightarrow \heva{&x=3y\\&0=\left(y+2\right)^2}\Leftrightarrow \heva{&x=-6\\&y=-2.}
		\]
		Vậy $x^2+y^2=40$.}
\end{vd}
\subsubsection{Bài tập tự luận}
 
\begin{bt}%[TH]%[1K2B3-4]
	Xác định $x$ dương để $2x-3$; $x$; $2x+3$ lập thành cấp số nhân.
	\dapso{$x=\sqrt{3}$}
	\loigiai{
		Ba số $2x-3$; $x$; $2x+3$ lập thành cấp số nhân khi $x^2=(2x-3)(2x+3) \Leftrightarrow x=\pm \sqrt{3}$.\\
		Do $x>0$ nên chọn $x=\sqrt{3}$.
	}
\end{bt}
\begin{bt}%[TH]%[1K2B3-4]
	Cho cấp số nhân $x, 12, y, 192$. Tìm $x$ và $y$.
	\dapso{$\heva{&x=-3 \\ &y=-48}$}	
	\loigiai{
		Bốn số $x, 12, y, 192$ lập thành CSN khi $\heva{&xy=12^2 \\ &y^2=12\cdot 192}
		\Leftrightarrow \heva{&x=3 \\ &y=48}$ hoặc $\heva{&x=-3 \\ &y=-48.}$
	}
\end{bt}
\begin{bt}%[TH]%[1K2B3-4]
	Tìm $x$ để dãy số $1$, $x^2$, $6-x^2$ lập thành cấp số nhân.
	\dapso{$x=\pm \sqrt{2}$}
	\loigiai{
		Ta có $1, x^2, 6-x^2$ lập thành cấp số nhân $\Leftrightarrow x^4=6-x^2 \Leftrightarrow x= \pm \sqrt{2}$.
	}
\end{bt}
\begin{bt}%[TH]%[1K2B3-4]
	Viết $6$ số xen giữa hai số $-2$ và $256$ để được một cấp số nhân có $8$ số hạng. Tìm cấp số nhân này.
	\dapso{$-2; 4; -8; 16; -32; 64; -128; 256$}
	\loigiai{
		Theo đề ra, ta có $\heva{&u_1=-2 \\ &u_8=256} \Leftrightarrow \heva{&u_1=-2 \\ &u_1\cdot q^7=256}
		\Leftrightarrow \heva{&u_1=-2 \\ &q=-2.}$ \\
		Cấp số nhân cần tìm là $-2; 4; -8; 16; -32; 64; -128; 256$.
	}
\end{bt}
\begin{bt}%[VD]%%[1K2K3-4]
	Bốn góc của một tứ giác lồi lập thành một cấp số nhân, góc lớn nhất gấp $8$ lần góc nhỏ nhất. Tìm $4$ góc đó.
	\dapso{$24^\circ; 48^\circ; 96^\circ; 192^\circ$}	
	\loigiai{
		Giả sử $4$ góc của tứ giác là $A\leq B\leq C\leq D$. Suy ra $A+B+C+D=360^\circ$.\\
		Theo đề, ta có $D=8A \Leftrightarrow Aq^3=8A \Leftrightarrow q=2$. Khi đó, ta được
		$$A(1+q+q^2+q^3)=360^\circ \Rightarrow A=24^\circ.$$
		Vậy $4$ góc của tứ giác lần lượt là $24^\circ; 48^\circ; 96^\circ; 192^\circ$.
	}
\end{bt}
\begin{bt}%[VD]%%[1K2K3-4]
	Tìm tất cả các giá trị của tham số $m$ để phương trình sau có ba nghiệm phân biệt lập thành một cấp số nhân $x^3-7mx^2+2(m^2+6 m)x-64=0$.
	\dapso{$m-8$}
	\loigiai{
		+ Điều kiện cần: \\
		Giả sử phương trình đã cho có ba nghiệm phân biệt $x_1; x_2; x_3$ lập thành một cấp số nhân. \\
		Theo định lý Vi-ét, ta có $x_1 \cdot x_2 \cdot x_3=64$. \\
		Theo tính chất của cấp số nhân, ta có $x_1\cdot x_3=x_2^2$. Suy ra ta có $x_2^3=64 \Leftrightarrow x_2=4$. \\
		Thay $x=4$ vào phương trình đã cho ta được 
		$$4^3-7m\cdot 4^2+2(m^2+6m) \cdot 4-64=0
		\Leftrightarrow m^2-8m=0
		\Leftrightarrow \hoac{&m=0 \\ &m=8.}$$
		+ Điều kiện đủ: \\
		Với $m=0$ thay vào phương trình đã cho ta được: $x^3-64=0$ hay $x=4$
		(nghiệm kép-loại). \\
		Với $m=8$ thay vào phương trình đã cho nên ta có phương trình $x^3-56x^2+224x-64=0$. \\
		Phương trình này có $3$ nghiệm phân biệt lập thành cấp số nhân. \\
		Vậy $m=8$ là giá trị cần tìm.
	}
\end{bt}
\subsubsection{Câu hỏi trắc nghiệm}
\Opensolutionfile{ans}[ans/ans-1K2-3-dang4]
% \begin{ex}%[1K2K3-4]
% 	Bốn góc của một tứ giác tạo thành cấp số nhân và góc lớn nhất gấp $27$ lần góc nhỏ nhất. Tổng của góc lớn nhất và góc bé nhất bằng
% 	\choice
% 	{$56^{\circ}$}
% 	{$102^{\circ}$}
% 	{$168^{\circ}$}
% 	{\True $252^{\circ}$}
% 	\loigiai{
% 		Giả sử 4 góc $A$, $B$, $C$, $D$ (với $A<B<C<D$) theo thứ tự đó lập thành cấp số nhân thỏa yêu cầu với công bội $q$.\\
% 		Theo giả thiết ta có
% 		\[\heva{&A+B+C+D=360\\&D=27A}\Leftrightarrow\heva{&A\left(1+q+q^2+q^3\right)=360\\&Aq^3=27A}\Leftrightarrow\heva{&q=3\\&A=9.}\]
% 		Suy ra $D=A\cdot q^3=9\cdot 3^3=243$.\\
% 		Vậy tổng số đo góc lớn nhất và góc bé nhất là $A+D=252^\circ$.
% 	}
% \end{ex}
\begin{ex}%[1K2B3-4]
	Xác định $x$ để $3$ số $2x-1$; $x$; $2x+1$ theo thứ tự lập thành một cấp số nhân:
	\choice
	{$x=\pm\sqrt{3}$}
	{$x=\pm\dfrac{1}{3}$}
	{\True $x=\pm\dfrac{1}{\sqrt{3}}$}
	{Không có giá trị nào của $x$}
	\loigiai{
		Ba số $2x-1$; $x$; $2x+1$ theo thứ tự lập thành cấp số nhân\\
		$\Leftrightarrow (2x-1)(2x+1)=x^2 \Leftrightarrow 3x^2=1 \Leftrightarrow x=\pm\dfrac{1}{\sqrt{3}}$.
	}
\end{ex}
\begin{ex}%[1K2K3-4]
	Cho $4$ số nguyên dương, trong đó $3$ số đầu lập thành cấp số cộng, $3$ số cuối lập thành cấp số nhân. Biết tổng số đầu và cuối là $37$, tổng $2$ số hạng giữa là $36$. Hỏi số lớn nhất thuộc khoảng nào sau đây?
	\choice
	{$\left(26;29\right)$}
	{\True $\left(24;26\right)$}
	{$\left(30;33\right)$}
	{$\left(22;25\right)$}
	\loigiai{
		Giả sử $4$ số đó là $a$, $b$, $c$, $d$ $\left(a,b,c,d\in\mathbb{N}^{\ast}\right)$. \\
		Do $a$, $b$, $c$ lập thành cấp số cộng nên ta có $a+c=2b$ $(1)$.\\
		Do $b$, $c$, $d$ lập thành cấp số nhân nên ta có $b \cdot d=c^2$ $(\ast)$. \\
		Theo giả thiết ta có $\heva{ & a+d=37 & & (2) \\ & b+c=36. & & (3)}$ \\
		Từ $(1)$, $(2)$, $(3)$ ta có $\heva{ & a=-d+37 \\ & b=\dfrac{-d+73}{3} \\ & c=\dfrac{d+35}{3}.}$ \\
		Thay vào $(\ast)$ ta có $\dfrac{-d+73}{3}\cdot d=\left(\dfrac{d+35}{3}\right)^2 \Leftrightarrow 4d^2-149d+1225=0 \Leftrightarrow \hoac{ & d=25 \\ & d=\dfrac{49}{4} & & \text{(loại)}.}$\\
		Với $d=25$, ta có $a=12$, $b=16$, $c=20$. \\
		Vậy số lớn nhất là $25\in\left(24;26\right)$.
	}
\end{ex}
\begin{ex}%[1K2K3-4]
	Ba số $x$, $y$, $z$ theo thứ tự lập thành một cấp số nhân với công bội $q$ khác $1$ đồng thời các số $x$, $2y$, $3z$ theo thứ tự lập thành một cấp số cộng với công sai khác $0$. Tìm giá trị của $q$.
	\choice
	{$q=-\dfrac{1}{3}$}
	{$q=\dfrac{1}{9}$}
	{$q=-3$}
	{\True $q=\dfrac{1}{3}$}
	\loigiai{
		Theo giả thiết ta có
		\[\heva{&y=xq \\ & z=xq^2\\&x+3z=2(2y)}\Rightarrow x+3xq^2=4xq \Rightarrow x\left(3q^2-4q+1\right)=0\Leftrightarrow \hoac{&x=0\\&3q^2-4q+1=0.}\]
		Nếu $x=0\Rightarrow y=z=0\Rightarrow$ công sai của cấp số cộng $x$, $2y$, $3z$ bằng 0 (vô lí).\\
		Nếu $3q^2-4q+1=0\Leftrightarrow\hoac{&q=1\\&q=\dfrac{1}{3}}\Leftrightarrow q=\dfrac{1}{3}$ vì $(q\not=1)$.}
\end{ex}
\begin{ex}%[1D3Y4-4]
	Trong các dãy số $(u_n)$ cho bởi số hạng tổng quát $u_n$ sau, dãy số nào là một cấp số nhân?
	\choice
	{$u_n=\dfrac{1}{3^n}-1$}
	{$u_n=n+\dfrac{1}{3}$}
	{$u_n=n^2-\dfrac{1}{3}$}
	{\True $u_n=\dfrac{1}{3^{n-2}}$}
	\loigiai{
		Từ các đáp án trên, với dãy $(u_n)$ cho bởi $u_n=\dfrac{1}{3^{n-2}}$ là một cấp số nhân, vì
		$$T=\dfrac{u_{n+1}}{u_n}=\dfrac{3^{n-2}}{3^{n-1}}=\dfrac{1}{3} \text{ (không đổi).}$$
	}
\end{ex}
\begin{ex}%[1K2B3-4]
	Trong các mệnh đề dưới đây, mệnh đề nào là \textbf{sai}?
	\choice
	{Dãy số $\left(a_n\right)$, với $a_1=3$ và $a_{n+1}=\sqrt{a_n+6}$, $\forall n\geq 1,$ vừa là cấp số cộng vừa là cấp số nhân}
	{\True Dãy số $\left(d_n\right)$, với $d_1=-3$ và $d_{n+1}=2d_n^2-15$, $\forall n\geq 1,$ vừa là cấp số cộng vừa là cấp số nhân}
	{Dãy số $\left(b_n\right)$, với $b_1=1$ và $b_{n+1}\left(2b_n^2+1\right)=3$, $\forall n\geq 1,$ vừa là cấp số cộng vừa là cấp số nhân}
	{Dãy số $\left(c_n\right)$, với $c_1=2$ và $c_{n+1}=3c_n^2-10$, $\forall n\geq 1,$ vừa là cấp số cộng vừa là cấp số nhân}
	\loigiai{
		Kiểm tra từng phương án ta có
		\begin{itemize}
			\item Ta có $a_2=3$, $a_2=3$, \ldots Bằng phương pháp quy nạp toán học chúng ra chứng minh được rằng $a_n=3$, $\forall n\geq 1$. Do đó $\left(a_n\right)$ là dãy số không đổi. Suy ra nó vừa là cấp số cộng (công sai bằng $0$) vừa là cấp số nhân (công bội bằng $1$).
			\item Tương tự như phương án trên, chúng ta chỉ ra được $b_n=1$, $\forall n\geq 1$. Do đó $\left(b_n\right)$ là dãy số không đổi. Suy ra nó vừa là cấp số cộng (công sai bằng $0$) vừa là cấp số nhân (công bội bằng $1$).
			\item Tương tự như phương án trên, chúng ta chỉ ra được $c_n=2$, $\forall n\geq 1$. Do đó $\left(c_n\right)$ là dãy số không đổi. Suy ra nó vừa là cấp số cộng (công sai bằng $0$) vừa là cấp số nhân (công bội bằng $1$).
			\item Ta có $d_1=-3$, $d_2=3$, $d_3=3$. Ba số hạng này không lập thành cấp số cộng cũng không lập thành cấp số nhân nên dãy số $\left(d_n\right)$ không phải là cấp số cộng và cũng không là cấp số nhân.
	\end{itemize}}
\end{ex}
\begin{ex}%[1K2K3-4]
	Biết rằng tồn tại hai giá trị $m_1$ và $m_2$ để phương trình \[2x^3+2\left(m^2+2m-1\right)x^2-7\left(m^2+2m-2\right)x-54=0\] có ba nghiệm phân biệt lập thành một cấp số nhân. Tính giá trị của biểu thức $P=m_1^3+m_2^3$.
	\choice
	{$P=56$}
	{$P=8$}
	{$P=-8$}
	{\True $P=-56$}
	\loigiai{
		Theo định lý Vi-ét, ta có $x_1\cdot x_2\cdot x_3 = -\dfrac{d}{a}=-\dfrac{-54}{2}=27 \Leftrightarrow x_2^3=27 \Leftrightarrow x_2=3$.\\
		Điều kiện cần để phương trình đã cho có ba nghiệm phân biệt lập thành một cấp số nhân là $x=3$ phải là nghiệm của phương trình đã cho. Suy ra
		\[m^2+2m-8=0\Leftrightarrow \hoac{&m=2 \\&m=-4.}\]
		Vì giả thiết cho biết tồn tại đúng hai giá trị của tham số $m$ nên $m=2$ và $m=-4$ là các giá trị thỏa mãn.\\
		Vậy $P=2^3+\left(-4\right)^3=-56$.
	}
\end{ex}
\begin{ex}%[1K2K3-4]
	Cho bốn số $a$, $b$, $c$, $d$ biết rằng $a$, $b$, $c$ theo thứ tự đó lập thành một cấp số nhân với công bội $q>1$; còn $b$, $c$, $d$ theo thứ tự đó lập thành cấp số cộng. Tìm $q$, biết rằng $a+d=14$ và $b+c=12$.
	\choice
	{$q=\dfrac{20+\sqrt{73}}{24}$}
	{\True $q=\dfrac{19+\sqrt{73}}{24}$}
	{$q=\dfrac{21+\sqrt{73}}{24}$}
	{$q=\dfrac{18+\sqrt{73}}{24}$}
	\loigiai{
		Giả sử $a$, $b$, $c$ lập thành cấp số cộng công bội $q$. Khi đó theo giả thiết ta có
		\[\heva{&b=aq,\;c=aq^2\\&b+d=2c\\&a+d=14 \\&b+c=12}\Rightarrow\heva{&aq+d=2aq^2&\quad(1)\\&a+d=14&(2)\\& a\left(q+q^2\right)=12.&(3)}\]
		\begin{itemize}
			\item Nếu $q=0\Rightarrow b=c=d=0$. (Vô lí!)
			\item Nếu $q=-1\Rightarrow b=-a$; $c=a\Rightarrow b+c=0$. (Vô lí!)
		\end{itemize}
		Vậy $q\not=0$, $q\not=-1$, từ $(2)$ và $(3)$ ta có $d=14-a$ và $a=\dfrac{12}{q+q^2}$. Thay vào $(1)$ ta được
		\[\begin{aligned}\dfrac{12q}{q+q^2}+\dfrac{14q^2+14q-12}{q+q^2}=\dfrac{24q^3}{q+q^2}
			&\Leftrightarrow 12q^3-7q^2-13q+6=0\\
			&\Leftrightarrow(q+1)\left(12q^2-19q+6\right)=0\\
			&\Leftrightarrow \hoac{ & q=-1 \quad \text{(loại)} \\ & q=\dfrac{19+\sqrt{73}}{24} \\ & q=\dfrac{19-\sqrt{73}}{24}.}
		\end{aligned}\]
		Vì $q>1$ nên $q=\dfrac{19+\sqrt{73}}{24}$.
	}
\end{ex}
\begin{ex}%[1K2K3-4]
	Cho dãy số tăng $a$, $b$, $c$ $\left(c\in\mathbb{Z}\right)$ theo thứ tự lập thành cấp số nhân; đồng thời $a$, $b+8$, $c$ theo thứ tự lập thành cấp số cộng và $a$, $b+8$, $c+64$ theo thứ tự lập thành cấp số nhân. Tính giá trị biểu thức $P=a-b+2c$.
	\choice
	{$P=32$}
	{$P=\dfrac{92}{9}$}
	{\True $P=64$}
	{$P=\dfrac{184}{9}$}
	\loigiai{
		Theo giả thiết, ta có hệ phương trình \[\heva{&ac=b^2\\&a+c=2(b+8)\\&a(c+64)=(b+8)^2}\Leftrightarrow\heva{&ac=b^2&\quad(1)\\&a-2b=16-c&(2)\\&ac+64a=\left(b+8\right)^2.&\quad(3)}\]
		Thay $(1)$ vào $(3)$ ta được $b^2+64a=b^2+16b+64\Leftrightarrow 4a-b=4.\qquad(4)$\\
		Kết hợp $(2)$ với $(4)$ ta được \[\heva{&a-2b=16-c\\&4a-b=4}\Leftrightarrow\heva{&a=\dfrac{c-8}{7}\\&b=\dfrac{4c-60}{7}.} \qquad (5)\]
		Thay $(5)$ vào $(1)$ ta được
		\[7(c-8)c=(4c-60)^2\Leftrightarrow 9c^2-424c+3600=0\Leftrightarrow\hoac{&c=36\\&c=\dfrac{100}{9}}\Leftrightarrow c=36.\qquad\left(\text{Vì}\;c\in\mathbb{Z}\right)\]
		Với $c=36$ $\Rightarrow a=4$, $b=12\Rightarrow P=4-12+72=64$.}
\end{ex}
\begin{ex}%[1K2K3-4]
	Cho $3$ số $a$, $b$, $c$ theo thứ tự lập thành cấp số nhân với công bội khác $1$ . Biết cũng theo thứ tự đó chúng lần lượt là số thứ nhất, thứ tư và thứ tám của một cấp số cộng công sai là $d$, $(d \neq 0)$. Tính $\dfrac{a}{d}$.
	\choice
	{$\dfrac{4}{3}$}
	{\True $9$}
	{$\dfrac{4}{9}$}
	{$3$}
	\loigiai{
		Do $a, b, c$ theo thứ tự lần lượt là số thứ nhất, thứ tư và thứ tám của một cấp số cộng công sai là $d,(d \neq 0)$ nên $\left\{\begin{array}{l}b=a+3 d \\ c=a+7 d\end{array}\right.$.\\
		Hơn nữa $a, b, c$ theo thứ tự lập thành cấp số nhân với công bội khác 1 nên $a c=b^{2}$.\\
		Khi đó \begin{eqnarray*}
			a(a+7 d)=(a+3 d)^{2} &\Leftrightarrow& a^{2}+7 a d=a^{2}+6 a d+9 d^{2}\\
			&\Leftrightarrow& 9 d^{2}-a d=0 \Leftrightarrow 9 d=a \Leftrightarrow \dfrac{a}{d}=9.
		\end{eqnarray*}
		Vậy $\dfrac{a}{d}=9$.
	}
\end{ex}
\begin{ex}%[1K2B3-4]
	Cho dãy số $(u_n)$ là một cấp số nhân với $u_n\neq 0$, $n\in\mathbb{N}^*$. Dãy số nào sau đây không phải là cấp số nhân?
	\choice
	{\True $u_1+2$; $u_2+2$; $u_3+2$; $\ldots$}
	{$3u_1$; $3u_2$; $3u_3$; $\ldots$}
	{$\dfrac{1}{u_1}$; $\dfrac{1}{u_2}$; $\dfrac{1}{u_3}$; $\ldots$}
	{$u_1$; $u_3$; $u_5$; $\ldots$}
	\loigiai{
		Giả sử $(u_n)$ là một cấp số nhân với công bội $q$.\\
		Ta có $u_2=u_1 q$, $u_3=u_1 q^2$.\\
		Dễ thấy $\dfrac{u_2+2}{u_1+2}=\dfrac{u_1 q+2}{u_1+2}$ và $\dfrac{u_3+2}{u_2+2}=\dfrac{u_1 q^2+2}{u_1 q+2}$.\\
		Do $\dfrac{u_2+2}{u_1+2}\neq\dfrac{u_3+2}{u_2+2}$ $\Rightarrow$ dãy số $u_1+2$; $u_2+2$; $u_3+2$; $\ldots$ không phải là cấp số nhân.
	}
\end{ex}
\begin{ex}%[1K2B3-4]
	Xác định $x$ để $3$ số $x-2$; $x+1$; $3-x$ theo thứ tự lập thành một cấp số nhân
	\choice
	{$x=\pm 1$}
	{\True Không có giá trị nào của $x$}
	{$x=-3$}
	{$x=2$}
	\loigiai{
		Ba số $x-2$; $x+1$; $3-x$ theo thứ tự lập thành một cấp số nhân\\
		$\Leftrightarrow(x-2)(3-x)=(x+1)^2 \Leftrightarrow 2x^2-3x+7=0$ (Phương trình vô nghiệm).
	}
\end{ex}
\begin{ex}%[1D3Y4-4]
	Trong các dãy số $(u_n)$ cho bởi số hạng tổng quát $u_n$ sau, dãy số nào là một cấp số nhân?
	\choice
	{\True $u_n=7\cdot 3^n$}
	{$u_n=\dfrac{7}{3n}$}
	{$u_n=7-3^n$}
	{$u_n=7-3n$}
	\loigiai{
		Từ các đáp án trên, với dãy $(u_n)$ cho bởi $u_n=7\cdot 3^n$ là một cấp số nhân, vì
		$$T=\dfrac{u_{n+1}}{u_n}=\dfrac{7\cdot 3^{n+1}}{7\cdot 3^n}=3 \text{ (không đổi).}$$
	}
\end{ex}
\begin{ex}%[1K2K3-4]
	Số hạng thứ hai, số hạng đầu và số hạng thứ ba của một cấp số cộng với công sai khác $0$ theo thứ tự đó lập thành một cấp số nhân với công bội $q$. Tìm $q$.
	\choice
	{\True $q=-2$}
	{$q=-\dfrac{3}{2}$}
	{$q=\dfrac{3}{2}$}
	{$q=2$}
	\loigiai{
		Giả sử ba số hạng $a$; $b$; $c$ lập thành cấp số cộng thỏa mãn yêu cầu, khi đó $b$; $a$; $c$ theo thứ tự đó lập thành cấp số nhân công bội $q\neq 1$. Ta có\\
		\[\heva{&a+c=2b\\&a=bq; c=bq^2}\Rightarrow bq+bq^2=2b\Leftrightarrow\hoac{&b=0\\&q^2+q-2=0.}\]
		\begin{itemize}
			\item Nếu $b=0\Rightarrow a=b=c=0$ nên $a$; $b$; $c$ là cấp số cộng công sai $d=0$. (Vô lí!)
			\item Nếu $q^2+q-2=0$ thì $ q=1$ hoặc $q=-2$. Dễ thấy trường hợp $q = 1$ là không thỏa mãn, vì khi đó $a = b = c$. Do đó $q = -2$.
		\end{itemize}
	}
\end{ex}
\begin{ex}%[1K2K3-4]
	Ba số $x$, $y$, $z$ lập thành một cấp số cộng và có tổng bằng $21$. Nếu lần lượt thêm các số $2$, $3$, $9$ vào ba số đó (theo thứ tự của cấp số cộng) thì được ba số lập thành một cấp số nhân. Tính $F=x^2+y^2+z^2$.
	\choice
	{$F=389$ hoặc $F=395$}
	{$F=395$ hoặc $F=179$}
	{$F=441$ hoặc $F=357$}
	{\True $F=389$ hoặc $F=179$}
	\loigiai{
		Theo tính chất của cấp số cộng, ta có $x+z=2y$.\\
		Kết hợp với giả thiết $x+y+z=21$, ta suy ra $3y=21\Leftrightarrow y=7$.\\
		Gọi $d$ là công sai của cấp số cộng thì $x=y-d=7-d$ và $z=y+d=7+d$.\\
		Sau khi thêm các số $2$, $3$, $9$ vào ba số $x$, $y$, $z$ ta được ba số là $x+2$, $y+3$, $z+9$ hay $9-d$, $10$, $16+d$.\\
		Theo tính chất của cấp số nhân, ta có $(9-d)(16+d)=10^2\Leftrightarrow d^2+7d-44=0$.\\
		Giải phương trình ta được $d=-11$ hoặc $d=4$.\\
		Với $d=-11$, cấp số cộng $18$, $7$, $-4$. Lúc này $F=389$.\\
		Với $d=4$, cấp số cộng $3$, $7$, $11$. Lúc này $F=179$.}
\end{ex}
\Closesolutionfile{ans}
% \begin{indapan}{10}
% 	{ans/ans-1K2-3-dang4}
% \end{indapan}
\begin{dang}{Tính tổng của cấp số nhân}
	Phương pháp
	\begin{itemize}
		\item Xác định số hạng đầu $u_1$, công bội $q$.
		\item Áp dụng công thức tính tổng các số hạng của cấp số nhân. 
	\end{itemize}
\end{dang}
\subsubsection{Ví dụ minh hoạ}
\begin{vd}%[NB]%[DCHT Toán 11 - KNTT -Đỗ Chí Tâm] %[1K2Y7-5]
	Tính tổng 10 số hạng đầu tiên của cấp số nhân $(u_n)$, biết $u_1=-3$ và công bội $q=-2$. \dapso{$S_{10}=1023$}
	\loigiai{
		Ta có: $S_{10}=\dfrac{u_1\left(1-q^{10}\right)}{1-q}=1023$.}
\end{vd}

\begin{vd}%[TH]%[DCHT Toán 11 - KNTT -Đỗ Chí Tâm] %[1K2B7-5]
	Tính tổng $8$ số hạng đầu tiên của cấp số nhân $(u_n),$ biết $u_1=3$ và $u_2=6$. \dapso{$S_8=765$}
	\loigiai{
		Ta có: $u_2=u_1.q \Leftrightarrow 6=3.q \Leftrightarrow q=2$\\
		$\hspace*{1.2cm}  S_8=u_1\dfrac{1-q^8}{1-q}=3.\dfrac{1-2^8}{1-2}=765. $
	}
\end{vd}

\begin{vd}[thuộc chương giới hạn]%[TH]%[DCHT Toán 11 - KNTT -Đỗ Chí Tâm] %[1K2B7-5]
	Tính tổng vô hạn $S=1+\dfrac{1}{2}+\dfrac{1}{2^2}+...+\dfrac{1}{2^n}+...$ \dapso{$S=2$}
	\loigiai{
		Đây là tổng của cấp số nhân lùi vô hạn, với $u_1=1, q=\dfrac{1}{2}$. Khi đó
		$$S=\dfrac{u_1}{1-q}=\dfrac{1}{1-\dfrac{1}{2}}=2.$$
	}
\end{vd}

\begin{vd}%[VD]%[DCHT Toán 11 - KNTT -Đỗ Chí Tâm] %[1K2K7-5]
	Tính tổng $200$ số hạng đầu tiên của dãy số $(u_n)$ biết $\heva{&u_1=1\\&u_{n+1}=3u_n}$.
	\loigiai{
		Dễ thấy dãy đã cho là một cấp số nhân với công bội $q=3; u_1=1$.\\
		Từ đó $S_{200}=u_1\dfrac{q^{200}-1}{q-1}$ $=\dfrac{3^{200}-1}{2}$.
	}
\end{vd}

\begin{vd}%[VD]%[DCHT Toán 11 - KNTT -Đỗ Chí Tâm]%[1K2K7-5]
	Một cấp số nhân có số hạng đầu $u_1=3$, công bội $q=2$. Biết $S_n=765$, tìm $n$.
	% \dapso{$n=8$}
	\loigiai{ 
		Áp dụng công thức tính tổng của cấp số nhân ta có
		$S_n=765 \Leftrightarrow \dfrac{u_1( 1-q^n )}{1-q}=765 \Leftrightarrow \dfrac{3.( 1-2^n )}{1-2}=765 \Leftrightarrow 2^n=256=2^8 \Leftrightarrow n=8$.} 
\end{vd}

\subsubsection{Bài tập tự luận}
 
\begin{bt}%[DCHT Toán 11 - KNTT -Đỗ Chí Tâm] %[1K2Y7-5]
	Một cấp số nhân có số hạng đầu $u_1=3$ và công bội $q=2 $. Tính tổng $8$ số hạng đầu của cấp số nhân.
	\dapso{$765$}		
	\loigiai{
		Ta có $S_8=\dfrac{u_1\left(1-q^8\right)}{1-q}=\dfrac{3 \left(1-2^8\right)}{1-2}=765$.	
	}	
\end{bt}

% \begin{bt}%[DCHT Toán 11 - KNTT -Đỗ Chí Tâm]%[1K2Y7-5]
% 	Một cấp số nhân có số hạng đầu $u_1=1$ và công bội $q=3$. Tính $S_{10}.$
% 	\dapso{$29524$}	
% 	\loigiai{
% 		Ta có $S_{10}=\dfrac{u_1\left(1-q^{10}\right)}{1-q}=1.\dfrac{1-3^{10}}{1-3}=29524$.	
% 	}	
% \end{bt}

% \begin{bt}%[DCHT Toán 11 - KNTT -Đỗ Chí Tâm] %[1K2Y7-5]
% 	Một cấp số nhân $(u_n)$ có $u_1=4$ và công bội $q=2$. Tính $S_{20}$.
% 	\dapso{$4194300$}	
% 	\loigiai{
% 		Ta có $S_{20}=\dfrac{u_1\left(1-q^{20}\right)}{1-q}=4194300$.}
% \end{bt}

\begin{bt}[thuộc chương giới hạn]%[DCHT Toán 11 - KNTT -Đỗ Chí Tâm] %[1K2B7-5]	
	Tính tổng $S=1+\dfrac{1}{3}+\dfrac{1}{3^2}+\cdots+\dfrac{1}{3^n}+\cdots$.
	\dapso{$\dfrac{3}{2}$}			
	\loigiai{
		Đây là tổng của một cấp số nhân lùi vô hạn với $u_1=1, q=\dfrac{1}{3}$\\
		Suy ra   $S=\dfrac{u_1}{1-q}=\dfrac{1}{1-\dfrac{1}{3}}=\dfrac{3}{2}.$
	}
\end{bt}

\begin{bt}%[DCHT Toán 11 - KNTT -Đỗ Chí Tâm] %[1K2B7-5]	
	Cho cấp số nhân có $q=-3, S_6=730$. Tính $u_1$.
	\dapso{$4$}		
	\loigiai{
		$S_6=u_1.\dfrac{1-q^6}{1-q}\Rightarrow u_1=S_6\cdot\dfrac{1-q}{1-q^6}=730\cdot \dfrac{1-(-3)}{1-(-3)^6}=4$.
	}
\end{bt}

% \begin{bt}%[DCHT Toán 11 - KNTT -Đỗ Chí Tâm] %[1K2B7-5]
% 	Một cấp số nhân $(u_n)$ có $u_1=-5$, $u_2=10$.Tính tổng của $15$ số hạng đầu của cấp số nhân đó.
% 	\dapso{$-54615$}	
% 	\loigiai{
% 		Công bội của cấp số nhân đã cho là: $q=\dfrac{u_2}{u_1}=\dfrac{10}{-5}=-2$.\\
% 		Tổng của $15$ số hạng đầu của cấp số nhân đó là $S_{15}=-5\cdot \dfrac{1-(-2)^{15}}{1-(-2)}=-54615$.
% 	}
% \end{bt}

% \begin{bt}%[DCHT Toán 11 - KNTT -Đỗ Chí Tâm] %[1K2B7-5]
% 	Một cấp số nhân $(u_n)$ có $u_1=2$, $u_2=-2$.Tính tổng của $9$ số hạng đầu của cấp số nhân đó.
% 	\dapso{$2$}	
% 	\loigiai{
% 		Công bội của cấp số nhân đã cho là: $q=\dfrac{u_2}{u_1}=\dfrac{-2}{2}=-1$.\\
% 		Tổng của $9$ số hạng đầu của cấp số nhân đó là: $S_9=2\cdot \dfrac{1-(-1)^9}{1-(-1)}=2$.
% 	}
% \end{bt}

\begin{bt}%[DCHT Toán 11 - KNTT -Đỗ Chí Tâm] %[1K2K7-5]
	Một cấp số nhân $(u_n)$ có $u_3=8$, $u_5=32$ và công bội $q>0$. Tính tổng của $10$ số hạng đầu tiên của cấp số nhân.
	\dapso{$2046$}	
	\loigiai{
		$\heva{u_3&=8\\u_5&=32}\Leftrightarrow \heva{u_1.q^2&=8\\u_1.q^4&=32}\Rightarrow q^2=\dfrac{32}{8}=4\Rightarrow q=2, u_1=2$\\
		\hspace*{5.2cm}$\Rightarrow S_{10}=u_1.\dfrac{1-q^{10}}{1-q}=2.\dfrac{2-2^{10}}{1-2}=2046$.
	}
\end{bt}

\begin{bt}%[DCHT Toán 11 - KNTT -Đỗ Chí Tâm] %[1K2K7-5]	
	Tính tổng $S=2+6+18+...+13122$.
	\dapso{$19682$}
	\loigiai{
		Xét cấp số nhân có $u_1=2, q=3$. Khi đó $13122=u_1.q^{n-1}\Leftrightarrow 13122=2.3^{n-1}\Leftrightarrow n=9$\\
		Vậy $S=S_9=u_1\dfrac{1-q^9}{1-q}=2.\dfrac{1-3^9}{1-3}=19682$
	}
\end{bt}

\begin{bt}%[DCHT Toán 11 - KNTT -Đỗ Chí Tâm] %[1K2K7-5]	
	Tính tổng $S=1+2+4+8+\cdots+1024$.	
	\dapso{$2047$}
	\loigiai{
		Xét cấp số nhân có $u_1=1, q=2$. Khi đó $1024=u_1.q^{n-1}\Leftrightarrow 1024=1.2^{n-1}\Leftrightarrow n=11$.\\
		Vậy $S=S_{11}=u_1\dfrac{1-q^{11}}{1-q}=1.\dfrac{1-2^{11}}{1-2}=2047$.
	}
\end{bt}

\begin{bt}%[DCHT Toán 11 - KNTT -Đỗ Chí Tâm] %[1K2K7-5]	
	Một cấp số nhân có $u_1=1, q=3$, biết $S_n=3280$. Tìm $n$.	
	\dapso{$8$}
	\loigiai{
		$S_n=u_1\dfrac{1-q^n}{1-q}=1.\dfrac{1-3^n}{1-3}=3280\Rightarrow n=8$.
	}
\end{bt}

% \begin{bt}%[DCHT Toán 11 - KNTT -Đỗ Chí Tâm] %[1K2K7-5]	
% 	Một cấp số nhân $(u_n)$ có $u_4+u_6=-540, u_3+u_5=180$. Tính $S_5$.
% 	\dapso{$122$}
% 	\loigiai{
% 		$\heva{u_4+u_6&=-540\\u_3+u_5&=180}\Leftrightarrow \heva{u_1.q^3+u_1.q^5&=-540\\u_1.q^2+u_1.q^4&=180} \Leftrightarrow \heva{u_1q^3(1+q^2)&=-540\\u_1q^2(1+q^2)&=180}\Leftrightarrow \heva{u_1&=2\\ q&=-3}$.\\
% 		Vậy $S_5=u_1\dfrac{1-q^5}{1-q}=2.\dfrac{1-(-3)^5}{1+3}=122$.
% 	}
% \end{bt}

\begin{bt}%[DCHT Toán 11 - KNTT -Đỗ Chí Tâm] %[1K2K7-5]
	Bốn số hạng liên tiếp của một cấp số nhân, trong đó số hạng thứ hai nhỏ hơn số hạng thứ nhất $35$, còn số hạng thứ ba lớn hơn số hạng thứ tư $560$. Tìm tổng của bốn số hạng trên, biết công bội mang giá trị dương.
	\dapso{$-\dfrac{2975}{3}$}
	\loigiai{
		Theo đề ta có $\heva{u_1-u_2&=35\\u_3-u_4&=560}\Leftrightarrow \heva{&u_1-u_1q=35\\&u_1q^2-u_1q^3=560}\Leftrightarrow \heva{&u_1(1-q)=35\,\,\,\,\,\,\, (1)\\&u_1q^2(1-q)=560\,\,\,\, (2)}$\\
		Thay $(1)$ vào $(2)$ ta được $q^2=16\Leftrightarrow q=\pm 4$.\\
		Với $q=4$ thay vào $(1)$ ta được $u_1=-\dfrac{35}{3}$.	\\
		$S_4=u_1.\dfrac{1-q^4}{1-q}=-\dfrac{2975}{3}$.
	}	
\end{bt}

\begin{bt}[thuộc chương giới hạn]%[DCHT Toán 11 - KNTT -Đỗ Chí Tâm] %[1K2G7-5]
	Tổng của một cấp số nhân lùi vô hạn bằng $ \dfrac{1}{4} $, tổng ba số hạng đầu tiên của cấp số nhân đó bằng $ \dfrac{7}{27} $. Tổng của số hạng đầu và công bội của cấp số nhân đó bằng
	\dapso{$S=0$}	
	\loigiai{Gọi $ u_1 $ và $ q $ với ($ |q|<1 $) lần lượt là số hạng đầu và cộng bội của cấp số nhân lùi vô hạn. Theo giả thiết, ta có $$\begin{cases}
			\dfrac{u_1}{q-1}=\dfrac{1}{4}\\
			u_1+u_1q+u_1q^2=\dfrac{7}{27}
		\end{cases} \Leftrightarrow \begin{cases}
			\dfrac{u_1}{q-1}=\dfrac{1}{4}\\
			u_1 (1-q^3)=\dfrac{7}{27}(1-q)
		\end{cases} \Leftrightarrow \begin{cases}\dfrac{u_1}{1-q}=\dfrac{1}{4}\\ q^3=-\dfrac{1}{27} \end{cases}\Leftrightarrow
		\heva{& u_1=\dfrac{1}{3}\\& q=-\dfrac{1}{3}.}$$ 
		Vậy $ u_1+q=0 $.
	}
\end{bt}

\begin{bt}%[DCHT Toán 11 - KNTT -Đỗ Chí Tâm] %[1K2G7-5]
	Một du khách vào trường đua ngựa đặt cược, lần đầu đặt $20.000$ đồng, mỗi lần sau tiền đặt gấp đôi số tiền lần đặt trước. Người đó thua $10$ lần liên tiếp và thắng ở lần thứ $11$. Hỏi du khách trên thắng hay thua bao nhiêu tiền?
	\dapso{$20000$ đồng}
	\loigiai{
		Số tiền du khách đặt trong mỗi lần (kể từ lần đầu) là một cấp số nhân có $u_1=20.000$ và công bội $q=2$.\\
		Du khách thua trong 10 lần liên tiếp đầu tiên nên tổng số tiền thua là
		$$S_{10}=\dfrac{u_1(1-q^{10})}{1-q}=\dfrac{20000(1-2^{10})}{1-2}=20000(2^{10}-1) \text{(đồng).}$$ 
		Số tiền du khách thắng trong lần thứ 11 là $u_{11}=u_1q^{10}=20000.2^{10}$ (đồng).\\
		Ta có $u_{11}-S_{10}=20000>0$. Vậy du khách thắng $20000$ đồng.
	}
\end{bt}
\subsubsection{Câu hỏi trắc nghiệm}
\Opensolutionfile{ans}[ans/ans-1K2-3-Dang5]
\begin{ex}%[1K2Y7-5]
	Cho cấp số nhân $u_1,u_2,u_3,\ldots,u_n$ với công bội $q$ ($q\neq 0$, $q\neq 1$). Đặt \[S_n=u_1+u_2+u_3+\cdots +u_n.\] Khẳng định nào sau đây là đúng?
	\choice
	{\True $S_n = \dfrac{u_1\left(q^n-1\right)}{q-1}$}
	{$S_n = \dfrac{u_1\left(q^n+1\right)}{q+1}$}
	{$S_n = \dfrac{u_1\left(q^{n-1}-1\right)}{q+1}$}
	{$S_n = \dfrac{u_1\left(q^{n-1}-1\right)}{q-1}$}
	\loigiai{
		Ta có $S_n=u_1+u_2+u_3+\cdots +u_n = u_1\cdot \dfrac{1-q^n}{1-q} = \dfrac{u_1\left(q^n-1\right)}{q-1}$.
	}
\end{ex}
\begin{ex}%[1K2Y7-5]
	Cho cấp số nhân $(u_n)$ có số hạng đầu $u_1=12$ và công sai $q=\dfrac{3}{2}$. Tổng $5$ số hạng đầu của cấp số nhân bằng
	\choice
	{$\dfrac{93}{4}$}
	{$\dfrac{633}{2}$}
	{\True $\dfrac{633}{4}$}
	{$\dfrac{93}{2}$}
	\loigiai{
		Gọi $S_5$ là tổng $5$ số hạng đầu của cấp số nhân đã cho. Khi đó ta có $$S_5=u_1\cdot \dfrac{1-q^5}{1-q}=12\cdot \dfrac{1-\left(\dfrac{3}{2}\right)^5}{1-\dfrac{3}{2}}=\dfrac{633}{4}.$$
	}
\end{ex}
\begin{ex}%[1K2Y7-5]
	Cho cấp số nhân $(u_n)$ có số hạng đầu $u_1=3$, công bội $q=-2$. Tính tổng $10$ số hạng đầu tiên của cấp số nhân $(u_n)$.
	\choice
	{\True $-1023$}
	{$1023$}
	{$513$}
	{$-513$}
	\loigiai{
		Tổng của $10$ số hạng đầu bằng
		$$S_{10}=u_1\cdot\dfrac{q^{10}-1}{q-1}=3\cdot\dfrac{(-2)^{10}-1}{-2-1}=-1023.$$
	}
\end{ex}
\begin{ex}%[1K2Y7-5]
	Cho cấp số nhân $(u_n)$ có $u_2=-2$ và $u_5=54$. Tính tổng $1000$ số hạng đầu tiên của cấp số nhân đã cho.
	\choice
	{$S_{1000}=\dfrac{3^{1000}-1}{2}$}
	{\True $S_{1000}=\dfrac{1-3^{1000}}{6}$}
	{$S_{1000}=\dfrac{3^{1000}-1}{6}$}
	{$S_{1000}=\dfrac{1-3^{1000}}{4}$}
	\loigiai{
		Ta có $u_5=u_2\cdot q^3\Leftrightarrow q^3=\dfrac{u_5}{u_2}=\dfrac{54}{-2}=-27=(-3)^3\Rightarrow q=-3$ và $u_1=\dfrac{u_2}{q}=\dfrac{2}{3}$.\\
		Suy ra $S_{1000}=u_1\cdot\dfrac{1-q^n}{1-q}=\dfrac{2}{3}\cdot\dfrac{1-(-3)^{1000}}{1+3}=\dfrac{1-3^{1000}}{6}$.
	}
\end{ex}
\begin{ex}%[1K2Y7-5]
	Tính tổng tất cả các số hạng của một cấp số nhân, biết số hạng đầu bằng $18$, số hạng thứ hai bằng $54$ và số hạng cuối bằng $39366$.
	\choice
	{$19674$}
	{\True $59040$}
	{$177138$}
	{$~6552$}
	\loigiai{
		$u_1=18,u_2=54 \Rightarrow q=3$.\\
		$u_n=39366 \Leftrightarrow u_1 \cdot q^{n-1}=39366 \Leftrightarrow 18 \cdot 3^{n-1}=39366 \Leftrightarrow 3^{n-1}=3^7 \Leftrightarrow n=8$.\\
		Vậy $S_8=18 \cdot \dfrac{1-3^8}{1-3}=59040$.}
\end{ex}
\begin{ex}%[1K2B7-5]
	Dãy số $\left(u_n\right)$ xác định bởi $\heva{&	u_1=1\\&u_{n + 1}= \dfrac{1}{2}u_n}$ với $n \ge 1$. Tính tổng $S = u_1 + u_2 +\cdots + u_{10}$.
	\choice
	{$S = \dfrac {1023} {2048}$}
	{$S = \dfrac{5}{2}$}
	{\True $\dfrac {1023} {512}$}
	{$S = 2$}
	\loigiai{
		Ta có các số hạng của dãy số $\left(u_n\right)$ là  $1,\dfrac{1}{2},\dfrac{1}{4},\dfrac{1}{8},\dfrac {1}{16},\dfrac{1}{32},\ldots ,\dfrac{1} {2^n}$. Khi đó $\left(u_n\right)$ lập thành một cấp số nhân có $u_1 = 1$ và công bội $q = \dfrac{1}{2}$. \\
		Suy ra $S = u_1 + u_2 +\cdots + u_{10} 
		=1+\dfrac{1}{2} + \dfrac{1}{4} +\cdots + \dfrac{1}{2^9}
		=\dfrac{1\cdot\left[1-\left(\dfrac{1}{2}\right)^{10}\right]}{1 - \dfrac{1}{2}}
		=\dfrac {1023} {512}$.}
\end{ex}
\begin{ex}%[1K2B7-5]
	Cho cấp số nhân $({{u}_{n}} )$ có ${{u}_{1}}=-6$ và $q=-2$. Tổng $n$ số hạng đầu tiên của cấp số nhân đã cho bằng $2046$. Tìm $n$.
	\choice
	{$n=9$}
	{$n=12$}
	{$n=11$}
	{\True $n=10$}
	\loigiai {
		Ta có
		$2046={{S}_{n}}={{u}_{1}}\cdot \dfrac{1-{{q}^{n}}}{1-q}=-6\cdot \dfrac{1-{{(-2 )}^{n}}}{1-(-2 )}=2({{(-2 )}^{n}}-1 )\Rightarrow {{(-2 )}^{n}}=1024\Leftrightarrow n=10$.}
\end{ex}
\begin{ex}%[1K2B7-5]
	Tổng $100$ số hạng đầu của dãy số $\left(u_n\right)$ với $u_n=2 n-1$ là
	\choice
	{$199$}
	{$2^{100}-1$}
	{\True $10000$}
	{$9999$}
	\loigiai{
		Ta có $(u_n)$ là cấp số cộng công sai $d=2$ và $u_1=1$.\\
		Do đó $S_{n}=n\cdot u_1+\dfrac{n(n-1)}{2}\cdot d=100 \cdot 1 +\dfrac{100 \cdot 99 }{2} \cdot 2 = 10000$.
	}
\end{ex}
\begin{ex}%[1K2B7-5]
	Cho dãy số $(u_n)$ với $u_n = \left(\dfrac{1}{2}\right)^n+1, \forall n \in \mathbb{N}^*$. Tính $S_{2019}=u_1+u_2+u_3+ \cdots + u_{2019}$.
	\choice
	{$S_{2019}=2019+\dfrac{1}{2^{2019}}$}
	{$S_{2019}=\dfrac{4039}{2}$}
	{$S_{2019}=\dfrac{6057}{2}$}
	{\True $S_{2019}=2020-\dfrac{1}{2^{2019}}$}
	\loigiai{
		Ta có
		\allowdisplaybreaks
		\begin{eqnarray*}
			S_{2019} &=& u_1+u_2+u_3+ \cdots + u_{2019}\\
			&=& \left(\dfrac{1}{2}+1\right) + \left[\left(\dfrac{1}{2}\right)^2+1\right] + \left[\left(\dfrac{1}{2}\right)^3+1\right] + \cdots + \left[\left(\dfrac{1}{2}\right)^{2019}+1\right]\\
			&=& 2019 + \dfrac{1}{2} + \left(\dfrac{1}{2}\right)^2 + \left(\dfrac{1}{2}\right)^3 + \cdots + \left(\dfrac{1}{2}\right)^{2019}\\
			&=& 2019 + \dfrac{1}{2} \cdot \dfrac{1-\left(\dfrac{1}{2}\right)^{2019}}{1-\dfrac{1}{2}} = 2019 + 1 - \dfrac{1}{2^{2019}}\\
			&=& 2020 - \dfrac{1}{2^{2019}}.
		\end{eqnarray*}
	}
\end{ex}
\begin{ex}%[1K2K7-5]
	Cho $S=11+101+1001+\cdots +\underbrace{1000\ldots 01}_{(n-1)\text{ chữ số 0}}$. Khẳng định nào sau đây là đúng?
	\choice
	{$S=10\left(\dfrac{10^n-1}{9}\right)$}
	{$S=10\left(\dfrac{10^n-1}{9}\right)-n$}
	{\True $S=10\left(\dfrac{10^n-1}{9}\right)+n$}
	{$S=\left(\dfrac{10^n-1}{9}\right)+n$}
	\loigiai{
		Ta có
		\begin{align*}
			S&=(10+1)+(10^2+1)+(10^3+1)+\cdots +(10^n+1)\\
			&=\left(10+10^2+10^3+\cdots +10^n\right)+\underbrace{1+1+1+\cdots+1}_{n\text{ số } 1}\\
			&=10\left(\dfrac{10^n-1}{9}\right)+n.
		\end{align*}
	}
\end{ex}
\begin{ex}%[1K2K7-5]
	Gọi $S=1+11+111+\cdots+\underbrace{111\ldots1}_{(n\text{ số }1)}$ thì $S$ nhận giá trị nào sau đây?
	\choice
	{\True $S=\dfrac{1}{9}\left[10\left(\dfrac{10^n-1}{9}\right)-n \right]$}
	{$S=\dfrac{10^n-1}{81}$}
	{$S=10\left(\dfrac{10^n-1}{81}\right)-n$}
	{$S=10\left(\dfrac{10^n-1}{81} \right)$}
	\loigiai {
		Ta có
		$S=\dfrac{1}{9}(9+99+999+\cdots+\underbrace{99\ldots9}_{\text{n số }9} )=\dfrac{1}{9}\cdot \left[ 10\cdot \dfrac{1-{{10}^{n}}}{1-10}-n \right]$.}
\end{ex}
\begin{ex}%[1K2K7-5]
	Cho dãy số $(u_n)$ thỏa mãn $\heva{&u_1=1\\&u_n=2u_{n-1}+1, n\geq2}$. Tổng $S=u_1+u_2+ \cdots +u_{20}$ là
	\choice
	{$2^{21}-20$}
	{\True $2^{21}-22$}
	{$2^{20}$}
	{$2^{20}-20$}
	\loigiai{
		Dự đoán công thức số hạng tổng quát $u_n=2^n-1$ (Chứng minh bằng phương pháp quy nạp TH).\\
		$S=2^1+2^2+\cdots+2^{20}-20=2\cdot\dfrac{1-2^{20}}{1-2}-20=2^{21}-22$.
	}
\end{ex}
\begin{ex}%[1K2K7-5]
	Biết rằng $S=1+2\cdot 3+{{3\cdot 3}^2}+\cdots+{{11\cdot 3}^{10}}=a+\dfrac{{{21\cdot 3}^{b}}}{4}$. Tính $P=a+\dfrac{b}{4}$.
	\choice
	{\True $P=3$}
	{$P=4$}
	{$P=1$}
	{$P=2$}
	\loigiai {
		Từ giả thiết suy ra $3S=3+{{2\cdot 3}^2}+{{3\cdot 3}^3}+\cdots+{{11\cdot 3}^{11}}$.\\
		Do đó
		{\allowdisplaybreaks
			\begin{eqnarray*}
				-2S&=&S-3S=1+3+{{3}^2}+\cdots+{{3}^{10}}-{{10\cdot 3}^{11}}\\
				&=&\dfrac{1-{{3}^{11}}}{1-3}-{{11\cdot 3}^{11}}=-\dfrac{1}{2}-\dfrac{{{21\cdot 3}^{11}}}{2}\Rightarrow S=\dfrac{1}{4}+\dfrac{21}{4}\cdot{{3}^{11}}.
			\end{eqnarray*}
		}
		Vì $S=\dfrac{1}{4}+\dfrac{{{21\cdot 3}^{11}}}{4}=a+\dfrac{{{21\cdot 3}^{b}}}{4}\Rightarrow a=\dfrac{1}{4},\,\,b=11\Rightarrow P=\dfrac{1}{4}+\dfrac{11}{4}=3$.}
\end{ex}
\Closesolutionfile{ans}
% \begin{indapan}{10}
% 	{ans/ans-1K2-3-Dang5}
% \end{indapan}

\begin{dang}{Kết hợp cấp số cộng và cấp số nhân}
	Nhắc lại tính chất CSC, CSN
	\begin{itemize}
		\item $3$ số $a,b,c$ theo thứ tự lập thành CSC thì $a+c=2b$.
		\item $3$ số $a,b,c$ theo thứ tự lập thành CSN thì $a.c=b^2$.
	\end{itemize}
\end{dang}
\subsubsection{Ví dụ minh hoạ}
\begin{vd}%[TH]%[DCHT Toán 11 - KNTT -Đỗ Chí Tâm] %[1K2K7-6]
	Ba số $x, y, z$ theo thứ tự đó lập thành một CSN với công bội $q (q\ne 1)$, đồng thời các số $x, 2y, 3z$ theo thứ tự đó lập thành một CSC với công sai $d$ . Hãy tìm $q$?
	\loigiai{
		Ta có $x+3z=2.2y \Leftrightarrow x+3xq^2=2.2xq\Leftrightarrow 1+3q^2=4q \Leftrightarrow \hoac{q&=\dfrac{1}{3}\\q&=1 (L)}$
	}
\end{vd}

\begin{vd}%[TH]%[DCHT Toán 11 - KNTT -Đỗ Chí Tâm] %[1K2K7-6]
	Biết rằng $a, b, c$ là ba số hạng liên tiếp của một CSC và $a, c, b$ là ba số hạng liên tiếp của một CSN, đồng thời $a+b+c=30$. Tìm $a,b,c$.
	\loigiai{
		Theo đề ta có  $\heva{&a+c=2b\,\,\,\hspace*{0.8cm}(1)\\&ab=c^2\,\,\,\hspace*{1.2cm}(2)\\&a+b+c=30\,\,\, (3)}$\\
		Từ $(1)$ và $(3)$ ta được $3b=30\Leftrightarrow b=10$\\
		Thay $b=10$ vào $(1), (2)$ ta được $\heva{a+c=20\\10a=c^2}\Leftrightarrow \hoac{&c=10, a=10\,\, (L)\\ &c=-20, a=40 (N)}$\\
		Vậy $a=40, b=10, c=-20$
	}
\end{vd}

\begin{vd}%[VD]%[DCHT Toán 11 - KNTT -Đỗ Chí Tâm] %[1K2K7-6]
	Ba số $x, y, z$ theo thứ tự đó lập thành một CSN. Ba số $x, y-4 , z$ theo thứ tự đó lập thành CSN. Đồng thời các số $x, y-4 , z-9$ theo thứ tự đó lập thành CSC. Tìm $x,y,z$.
	
	\loigiai{
		Theo đề ta có  $\heva{&xz=y^2\,\,\,\hspace*{2.2cm}(1)\\&xz=(y-4)^2\,\,\,\hspace*{1.4cm}(2)\\&x+(z-9)=2(y-4)\,\,\, (3)}$\\
		Từ $(1)$ và $(2)$ ta có $y^2=(y-4)^2\Leftrightarrow y=2$\\
		Thế $y=2$ vào $(1)$ và $(3)$ ta được $\heva{xz=4\\x+z=5}\Rightarrow x=4, z=1$ hoặc $x=1, z=4$\\
		Vậy có 2 bộ $(x,y,z)$ thỏa yêu cầu bài toán là $(1,2,4)$ và $(4,2,1)$
	}
\end{vd}

\begin{vd}%[VD]%[DCHT Toán 11 - KNTT -Đỗ Chí Tâm] %[1K2K7-6]
	Cho $a,b,c$ là ba số hạng liên tiếp của một CSN và $a,b,c-4$ là ba số hạng liên tiếp của một CSC, đồng thời $a,b-1,c-5$ là ba số hạng liên tiếp của một CSN. Tìm $a,b,c$ biết $a,b,c$ là các số nguyên. 
	
	\loigiai{
		Theo đề ta có  $\heva{&ac=b^2\,\,\,\hspace*{1.9cm}(1)\\&a+c-4=2b\,\,\,\hspace*{0.8cm}(2)\\&a(c-5)=(b-1)^2\,\,\,\,(3)}$\\
		Thay $(1)$ vào $(3)$: $b^2-5a=b^2-2b+1\Leftrightarrow b=\dfrac{5a+1}{2}$\\
		Thay vào $(2)$ ta được $a+c-4=5a+1\Leftrightarrow c=4a+5$\\
		Thế $b, c$ theo $a$ vào $(1)$ ta được $9a^2-10a+1=0\Leftrightarrow a=1 \vee a=\dfrac{1}{9} (L)$
		Vậy $a=1, b=3, c=9$
	}
\end{vd}

\begin{vd}%[VDC]%[DCHT Toán 11 - KNTT -Đỗ Chí Tâm]%[1K2G7-6]
	Cho $4$ số nguyên dương, trong đó $3$ số đầu lập thành một CSC, $3$ số hạng sau thành lập CSN.
	Biết rằng tổng của số hạng đầu và số hạng cuối là $37$, tổng của hai số hạng giữa là $36$. Tìm tổng $4$ số đó
	\loigiai{
		Gọi 4 số cần tìm lần lượt là $a, b, c, d$\\
		$a, b, c$ là $3$ số hạng liên tiếp của CSC. Ta có $a+c=2b\,\,\, (1)$\\
		$b,c,d$ là $3$ số hạng liên tiếp của CSN. Ta có $bd=c^2\,\,\, (2)$\\
		Theo giả thuyết ta có $\heva{a+d&=37\,\,\, (3)\\b+c&=36\,\,\, (4)}$\\
		Từ $(4)\Rightarrow b=36-c$ thay vào $(1)$ ta được $a=72-3c$, thay $a$ vào $(3)$ ta được $d=-35+3c$\\
		Thế $b,d$ vào $(2)$ ta được $(36-c)(-35+3c)=c^2\Rightarrow c=20 \vee c=\dfrac{63}{4} (L)$\\
		Vậy $c=20, a=12, b=16, d=95\Rightarrow S=a+b+c+d=143$
	}
\end{vd}

\subsubsection{Bài tập tự luận}
 
\begin{bt}%[DCHT Toán 11 - KNTT -Đỗ Chí Tâm]%[1K2K7-6]
	Biết $x, y, x+4$ theo thứ tự lập thành cấp số cộng và $x+1, y+1, 2y+2$ theo thứ tự lập thành cấp số nhân với $x, y$ là số thực dương. Tính $x+y$.
	\dapso{$4$}
	\loigiai{
		Theo giả thiết ta có:\\
		$\heva{&x+(x+4)=2y\\&(x+1)(2y+2)=(y+1)^2}\Leftrightarrow \heva{&y=x+2\\&(x+1)(2x+6)=(x+3)^2}\Leftrightarrow \heva{&x=1\Rightarrow y=3\\&x=-3\Rightarrow y=-1}$\\
		Do đó $x+y=4$.
	}	
\end{bt}

\begin{bt}%[DCHT Toán 11 - KNTT -Đỗ Chí Tâm] %[1K2K7-6]
	Cho $3$ số $a, b, c$ theo thứ tự tạo thành một cấp số nhân với công bội khác $1$. Biết cũng theo thứ tự đó chúng lần lượt là số hạng thứ nhất, thứ tư và thứ tám của một cấp số cộng với công sai $d\ne 0$. Tính $\dfrac{a}{d}$.
	\dapso{$9$}
	\loigiai{
		$a, b, c$ lần lượt là số hạng thứ nhất, thứ tư, thứ tám của một CSC với công sai $d$\\
		ta có $\heva{b &=a+3d\\c&=a+7d}$.\\
		Mặt khác $a, b, c$ là $3$ số hạng liên tiếp của CSN nên\\ $a.c=b^2\Leftrightarrow a(a+7d)=(a+3d)^2\Leftrightarrow a^2+7ad=a^2+6ad+9d^2\Leftrightarrow 9d^2=ad\Leftrightarrow \dfrac{a}{d}=9$.
	}	
\end{bt}

\begin{bt}%[DCHT Toán 11 - KNTT -Đỗ Chí Tâm] %[1K2K7-6]
	Tìm tích các số dương $a$ và $b$ sao cho $a, a + 2b, 2a + b$ lập thành một cấp số cộng và $(b + 1)^2, ab + 5,(a + 1)^2$ lập thành một cấp số nhân.
	\dapso{$3$}
	\loigiai{
		Theo tính chất CSC ta có $a+(2a+b)=2(a+2b)\,\,\,\, (1)$\\
		Theo tính chất CSN ta có $(b+1)^2.(a+1)^2=(ab+5)^2\,\,\,\, (2)$\\
		Từ $(1)$ ta được $a=3b$, thay vào $(2)$ ta được $(b+1)^2(3b+1)^2=(3b^2+5)^2$\\
		$\Leftrightarrow \hoac{&(b+1)(3b+1)=(3b^2+5)\\& (b+1)(3b+1)=-(3b^2+5) \,\, (\text{Vô nghiêm})}\Leftrightarrow b=1, a=3\Rightarrow ab=3.$	
	}	
\end{bt}

\begin{bt}%[DCHT Toán 11 - KNTT -Đỗ Chí Tâm] %[1K2K7-6]
	$a,b,c\,(a\ne b\ne c)$ là ba số hạng liên tiếp của một cấp số cộng và $b,c,a$ là ba số hạng liên tiếp của một cấp số nhân, đồng thời $a.b.c=125$. Tìm $a,b,c$.	
	\dapso{$5$}
	\loigiai{
		$a,b, c$ là ba số hạng liên tiếp của cấp số cộng, nên có $a+c=2b$.\\
		$b,c,a$ là ba số hạng liên tiếp của một cấp số nhân, nên có $b.a=c^2$.\\
		Ta có hệ $\heva{&a+c=2b\,\,\,\, (1)\\&b.a=c^2\,\,\,\,\,\,\,\,(2) \\&a.b.c=125\,\,\,\, (3)}$\\
		Thay $(2)$ vào $(3)$ ta được $c^3=125\Rightarrow c=5$\\
		Thay $c=5$ vào $(1), (2)$ ta được hệ $\heva{a+5&=2b\\ab&=25}\Leftrightarrow \heva{&a=2b-5\\&2b^2-5b-25=0}\Leftrightarrow \hoac{&b=5\Rightarrow a=5\\&b=-\dfrac{5}{2}\Rightarrow a=-10}$\\
		Vậy $a=-10, b=-\dfrac{5}{2}, c=5$.	
	}	
\end{bt}

\begin{bt}%[DCHT Toán 11 - KNTT -Đỗ Chí Tâm] %[1K2K7-6]
	Một cấp số cộng và một cấp số nhân đều là các dãy tăng các số hạng thứ nhất của hai dãy số đều bằng $3$, các số hạng thứ hai bằng nhau. Tỉ số giữa các số hạng thứ ba của CSN và CSC là $\dfrac{9}{5}$. Tìm tích ba số hạng của cấp số cộng thỏa mãn tính chất trên.
	\dapso{$405$}
	\loigiai{
		Gọi $u_1, u_2, u_3$ là $3$ số hạng liên tiếp của CSC.\\
		Gọi $a_1, a_2, a_3$ là $3$ số hạng liên tiếp của CSN.\\
		Theo đề ta có hệ $\heva{&u_1=a_1=3\\&u_2=a_2\\&a_3=\dfrac{9}{5}u_3}\Leftrightarrow \heva{&u_1=a_1=3\\&3+d=3q\\&5(3q^2)=9(3+2d)}\Rightarrow q=3 \vee q=\dfrac{3}{5}$\\
		Chọn $q=3$ vì dãy tăng, khi đó $d=6$. \\
		Vậy $3$ số hạng của cấp số cộng là $3; 9; 15\Rightarrow 3\cdot9\cdot15=405$	
	}	
\end{bt}

\begin{bt}%[DCHT Toán 11 - KNTT -Đỗ Chí Tâm] %[1K2K7-6]
	Một CSC và CSN đều có số hạng đầu tiên là bằng 5, số hạng thứ hai của CSC lớn hơn số hạng thứ hai của CSN là 10, còn các số hạng thứ 3 của hai cấp số thì bằng nhau. Tìm tổng các số hạng của cấp số cộng biết công bội của cấp số nhân không âm.
	\dapso{$75$}
	\loigiai{
		Gọi $u_1, u_2, u_3$ là $3$ số hạng liên tiếp của CSC với công sai $d$.\\
		Gọi $a_1, a_2, a_3$ là $3$ số hạng liên tiếp của CSN với công bội $q$.\\
		Theo đề bài ta có: $\heva{&u_1=a_1=5\\&u_2-a_2=10\\&u_3=a_3}\Leftrightarrow \heva{&u_1=a_1=5\\&u_1+d-a_1q=10\\&u_1+2d=a_1q^2}\Leftrightarrow \heva{&u_1=a_1=5\\&d=5+5q\\&5+2d=5q^2}\Rightarrow q=3\vee q=-1 (L)$\\
		Với $q=3\Rightarrow d=20$. Vậy CSC là $5;25;45 \Rightarrow S=5+25+45=75$
	}	
\end{bt}

\begin{bt}%[DCHT Toán 11 - KNTT -Đỗ Chí Tâm] %[1K2G7-6]	
	Ba số khác nhau có tổng bằng $114$ có thể coi là ba số hạng liên tiếp của một CSN, hoặc coi là số hạng thứ nhất, thứ tư và thứ hai mươi lăm của một CSC. Tìm ba số đó.
	\dapso{$2; 14; 98$}
	\loigiai{
		Gọi $u_1, u_2, u_3$ là $3$ số hạng liên tiếp của CSN với công bội $q$.\\
		Theo đề $u_1=a_1, u_2=a_4, u_3=a_{25}$ với $a_1,a_4, a_{25}$ là 3 số hạng của CSC với công sai $d$.\\
		Ta có $\heva{&a_4=a_1+3d\\&a_{25}=a_1+24d}\Rightarrow 8a_4-a_{25}=7a_1\Leftrightarrow 8u_2-u_3=7u_1\Leftrightarrow 8u_1q-u_1q^2=7u_1$\\
		\hspace*{4.2cm}$\Leftrightarrow q^2-8q+7=0 \Leftrightarrow q=1 (L) \vee q=7 (N)$\\
		Theo đề ta cũng có $u_1+u_2+u_3=114\Leftrightarrow u_1+u_1q+u_1q^2=114\Rightarrow u_1=2$\\
		Vậy $3$ số cần tìm là $2; 14; 98$.
	}	
\end{bt}

\begin{bt}%[DCHT Toán 11 - KNTT -Đỗ Chí Tâm] %[1K2G7-6]
	Ba số khác nhau có tổng là $217$ có thể coi là các số hạng liên tiếp của một CSN hoặc là các số hạng thứ $2$ thứ $9$ và thứ $44$ của một CSC. Tìm 3 số đó. 
	\dapso{$7; 35; 175$}
	\loigiai{
		Gọi $u_1, u_2, u_3$ là $3$ số hạng liên tiếp của CSN với công bội $q$.\\
		Theo đề $u_1=a_2, u_2=a_9, u_3=a_{44}$ với $a_2,a_9, a_{44}$ là 3 số hạng của CSC với công sai $d$.\\
		Ta có $\heva{&a_9=a_2+7d\\&a_{44}=a_2+42d}\Rightarrow 6a_9-a_{44}=5a_2\Leftrightarrow 6u_2-u_3=5u_1\Leftrightarrow 6u_1q-u_1q^2=5u_1$\\
		\hspace*{4.2cm}$\Leftrightarrow q^2-6q+5=0 \Leftrightarrow q=1 (L) \vee q=5 (N)$\\
		Theo đề ta cũng có $u_1+u_2+u_3=217\Leftrightarrow u_1+u_1q+u_1q^2=217\Rightarrow u_1=7$\\
		Vậy $3$ số cần tìm là $7; 35; 175$.
	}	
\end{bt}

% \begin{bt}%[DCHT Toán 11 - KNTT -Đỗ Chí Tâm] %[1K2K7-6]
% 	$a,b,c$ là ba số hạng liên tiếp của một CSN và $a,b+2,c+9$ là ba số hạng liên tiếp của một CSC, đồng thời $a,b+2,c$ là ba số hạng liên tiếp của một CSN khác. Tìm $a$.
% 	\dapso{$m=\dfrac{-7\pm3\sqrt{5}}{2}$}
% 	\loigiai{
% 		Vì $a,b,c$ là ba số hạng liên tiếp của CSN, ta có $ac=b^2\,\,\,\, (1)$\\
% 		Vì $a,b+2,c+9$ là ba số hạng liên tiếp của CSC, ta có $a+(c+9)=2(b+2)\,\,\,\, (2)$\\
% 		Vì $a,b+2,c$ là ba số hạng liên tiếp của CSN, ta có $a.c=(b+2)^2\,\,\,\, (3)$\\
% 		Thế $(1)$ vào $(3)$, ta được $b^2=(b+2)^2\Leftrightarrow b=-1$\\
% 		Thay $b=-1$ vào $(1), (2)$, ta được $\heva{&ac=1\\&a+c=-7}\Leftrightarrow a=\dfrac{-7+3\sqrt{5}}{2}\vee a=\dfrac{-7-3\sqrt{5}}{2}$
% 	}	
% \end{bt}

% \begin{bt}%[DCHT Toán 11 - KNTT -Đỗ Chí Tâm] %[1K2G7-6]
% 	Một CSC và CSN có cùng các số hạng thứ $m+1$, thứ $n+1$, thứ $p+1$ và $3$ số hạng này là $3$ số dương $a, b, c$. Tính $T=a^{b-c}.b^{c-a}.c^{a-b}$.
% 	\dapso{$m=1$}
% 	\loigiai{
% 		$a=u_1+md=q^m.v_1$\\
% 		$b=u_1+nd=q^n.v_1$\\
% 		$c=u_1+pd=q^p.v_1$\\
% 		Suy ra $T=a^{b-c}.b^{c-a}.c^{a-b}=\left(q^mv_1 \right)^{(n-p)d}. \left(q^nv_1 \right)^{(p-m)d}. \left(q^pv_1 \right)^{(m-n)d}=1$
% 	}	
% \end{bt}

% \begin{bt}%[DCHT Toán 11 - KNTT -Đỗ Chí Tâm] %[1K2G7-6]
% 	Tìm $m$ dương để phương trình $x^3+(5-m)x^2+(6-5m)x-6m=0 \,\,\,(*)$ có $3$ nghiệm phân biệt lập thành cấp số nhân.
% 	\dapso{$m=\sqrt{6}$}
% 	\loigiai{
% 		$(*)\Leftrightarrow (x+2) \left(x^2+(3-m)x-3m \right)=0\Leftrightarrow x=-2 \vee x=-3 \vee x=m$.\\
% 		Để $(*)$ có 3 nghiệm phân biệt thì $m\ne -3$ và $m\ne -2$.\\
% 		Do $3$ nghiệm này lập thành cấp số nhân, ta sắp xếp các nghiệm này theo thứ tự tăng dần được các dãy số sau
% 		\begin{itemize}
% 			\item $-3;-2; m$ lập thành CSN $\Leftrightarrow$ $-3m=(-2)^2\Leftrightarrow m=-\dfrac{4}{3}$
% 			\item $-3; m; -2$ lập thành CSN $\Leftrightarrow -3(-2)=m^2\Leftrightarrow m=\pm \sqrt{6}$
% 			\item $m; -3; -2$ lập thành CSN $\Leftrightarrow m(-2)=(-3)^2\Leftrightarrow m=-\dfrac{9}{2}$
% 		\end{itemize}	
% 		So với điều kiện thì $m=\sqrt{6}$ thỏa yêu cầu bài toán.
% 	}	
% \end{bt}
% \begin{bt}%[DCHT Toán 11 - KNTT -Đỗ Chí Tâm] %[1K2G7-6]
% 	Tìm tham số $m$ để phương trình $x^3-(2m+1)x^2+2mx=0 \,\, (*)$ có $3$ nghiệm phân biệt lập thành một cấp số cộng, biết $m<0$.
% 	\dapso{$m=-\dfrac{1}{2}$}
% 	\loigiai{
% 		$(*)\Leftrightarrow x.\left(x^2-(2m+1)x+2m \right)=0\Leftrightarrow x=0 \vee x=1 \vee x=2m$.\\
% 		Để $(*)$ có 3 nghiệm phân biệt thì $m\ne 0$ và $m\ne \dfrac{1}{2}$.\\
% 		Do $3$ nghiệm này lập thành cấp số cộng, ta sắp xếp các nghiệm này theo thứ tự tăng dần được các dãy số sau
% 		\begin{itemize}
% 			\item $2m;0; 1$ lập thành CSC $\Leftrightarrow$ $2m+1=2.0\Leftrightarrow m=-\dfrac{1}{2}$
% 			\item $0; 2m; 1$ lập thành CSC $\Leftrightarrow 0+1=4m\Leftrightarrow m=\dfrac{1}{4}$
% 			\item $0; 1; 2m$ lập thành CSC $\Leftrightarrow 0+2m=2.1\Leftrightarrow m=1$
% 		\end{itemize}	
% 		Vậy $m=-\dfrac{1}{2}$ là giá trị cần tìm.
% 	}	
% \end{bt}
% \subsubsection{Câu hỏi trắc nghiệm}
% \Opensolutionfile{ans}[ans/ans-1K2-3-Dang6]
% \begin{ex}%[1K2K7-6]
% 	Các số $x+6y$, $5x+2y$, $8x+y$ theo thứ tự đó lập thành một cấp số cộng, đồng thời các số $x-1$, $y+2$, $x-3y$ theo thứ tự đó lập thành một cấp số nhân. Tính $x^2+y^2$.
% 	\choice
% 	{$x^2+y^2=25$}
% 	{\True $x^2+y^2=40$}
% 	{$x^2+y^2=100$}
% 	{$x^2+y^2=10$}
% 	\loigiai{
% 		Theo bài ra, ta có
% 		\[ \heva{& (x+6 y)+(8 x+y)=2(5 x+2 y) \\ & (y+2)^2=(x-1)(x-3y)} \Rightarrow \heva{& x=3y \\ & (y+2)^2=0}\Rightarrow \heva{& x=-6 \\ & y=-2}\Rightarrow x^2+y^2=40. \]
% 	}
% \end{ex}

% \begin{ex}%[1K2K7-6]
% 	Cho hai số dương $ a $ và $ b $ không vượt quá $ 10 $ sao cho $ a-b $; $ 2 $; $ b $ theo thứ tự tạo thành một cấp số cộng và $ a+b $; $ 3a-2b $; $ 5a $ theo thứ tự lập thành một cấp số nhân. Tính giá trị của $ S=a+b $.
% 	\choice
% 	{$ S=8 $}
% 	{$ S=20 $}
% 	{$ S=7 $}
% 	{\True $ S=5 $}
% 	\loigiai{
% 		Vì $ a-b $; $ 2 $; $ b $ theo thứ tự tạo thành một cấp số cộng nên $ 2-(a-b)=b-2 \Leftrightarrow a=4 $.\\
% 		Vì $ a+b $; $ 3a-2b $; $ 5a $ theo thứ tự lập thành một cấp số nhân nên $$ \dfrac{3a-2b}{a+b}=\dfrac{5a}{3a-2b}\Leftrightarrow \dfrac{12-2b}{4+b}=\dfrac{20}{12-2b}\Leftrightarrow \hoac{&b=1\\&b=16 \text{ (loại).}} $$
% 		Vậy $ S=a+b=4+1=5. $
% 	}
% \end{ex}
% \begin{ex}%[1K2K7-6]
% 	Số hạng thứ hai, số hạng đầu và số hạng thứ ba của một cấp số cộng với công sai khác 0 theo thứ tự đó lập thành một cấp số nhân với công bội $q$. Tìm $q$.
% 	\choice
% 	{$q=2$}
% 	{\True $q=-2$}
% 	{$q=\dfrac{3}{2}$}
% 	{$q=-\dfrac{3}{2}$}
% 	\loigiai {
% 		Giả sử ba số hạng $a;b;c$ lập thành cấp số cộng thỏa yêu cầu, khi đó $b;a;c$ theo thứ tự đó lập thành cấp số nhân công bội $q$. Ta có\\
% 		$\heva{
% 			& a+c=2b \\
% 			& a=bq;\,c=b{{q}^2} \\
% 		}\Rightarrow bq+b{{q}^2}=2b\Leftrightarrow \hoac{
% 			& b=0 \\
% 			& {{q}^2}+q-2=0. \\
% 		}$ \\
% 		Nếu $b=0\Rightarrow a=b=c=0$ nên $a;b;c$ là cấp số cộng công sai $d=0$ (vô lí).\\
% 		Nếu ${{q}^2}+q-2=0\Leftrightarrow q=1$ hoặc $q=-2$ Nếu $q=1\Rightarrow a=b=c$ (vô lí), do đó $q=-2$.}
% \end{ex}

% \begin{ex}%[1K2K7-6]
% 	Cho ba số $a$, $b$, $c$ theo thứ tự tạo thành cấp số nhân với công bội khác $1$. Biết cũng theo thứ tự đó chúng lần lượt là số hạng thứ nhất, thứ tư và thứ tám của một cấp số cộng công sai là $s\neq 0$. Tính $\dfrac{a}{s}$.
% 	\choice
% 	{$3$}
% 	{$\dfrac{4}{9}$}
% 	{\True $9$}
% 	{$\dfrac{4}{3}$}
% 	\loigiai{
% 		Rõ ràng $a\neq 0$. Vì $s$ là công sai cấp số cộng nên $a$, $a+3s$, $a+7s$ lập thành cấp số nhân, do đó
% 		$$a(a+7s)=(a+3s)^2 \Leftrightarrow 9s^2-as=0\Leftrightarrow \hoac{& s=0\quad\text{(loại)}\\ & a=9s}\Leftrightarrow \dfrac{a}{s}=9.$$
% 	}
% \end{ex}

% \begin{ex}%[1K2K7-6]
% 	Xét các số thực dương $a, b$ sao cho $-25$, $2a$, $3b$ là cấp số cộng và $2$, $a+2$, $b-3$ là cấp số nhân. Khi đó $a^2 + b^2-3ab$ bằng
% 	\choice
% 	{$76$}
% 	{$89$}
% 	{$31$}
% 	{\True $59$}
% 	\loigiai{
% 		$-25,\,2a,\,3b$ là cấp số cộng $ \Leftrightarrow 2.2a=-25+3b \Leftrightarrow b=\dfrac{1}{3}(4a+25)$.
% 		\begin{eqnarray*}
% 			& 2,a+2,b-3 \text{ là cấp số nhân}& \Leftrightarrow (a+2)^2=2(b-3)\\
% 			& & \Leftrightarrow (a+2)^2=2\left[\dfrac{1}{3}(4a+25)-3\right]\\
% 			& & \Leftrightarrow 3a^2+4a-20=0 \\
% 			& & \Leftrightarrow \hoac{&a=2 \\&a=-\dfrac{10}{3}\,(\text{loại}).}
% 		\end{eqnarray*}
% 		Suy ra $b=11 \Rightarrow a^2+b^2-3ab=59$.
% 	}
% \end{ex}
% \begin{ex}%[1K2K7-6]
% 	Cho ba số $x$; $5$; $2y$ theo thứ tự lập thành cấp số cộng và ba số $x$; $4$; $2y$ theo thứ tự lập thành cấp số nhân thì $|x-2y|$ bằng
% 	\choice
% 	{$10$}
% 	{$8$}
% 	{$9$}
% 	{\True $6$}
% 	\loigiai{
% 		Theo giả thiết ta có
% 		\begin{itemize}
% 			\item Do $x;5;2y$ theo thứ tự lập thành một cấp số cộng nên ta có $5=\dfrac{x+2y}{2}\quad\quad (1)$.
% 			\item Do $x;4;2y$ theo thứ tự lập thành một cấp số nhân nên ta có $x\cdot 2y=4^2\quad\quad (2)$.
% 		\end{itemize}
% 		Từ $(1)$ và $(2)$ ta có $\heva{&x+2y=10\\&x\cdot 2y =16}\Rightarrow \hoac{& x=2,2y=8\\&x=8,2y=2}\Rightarrow \left|x-2y\right|=6$.
% 	}
% \end{ex}

% \begin{ex}%[1K2K7-6]
% 	Cho dãy số tăng $a, b, c\,\,(c\in \mathbb{Z} )$ theo thứ tự lập thành cấp số nhân; đồng thời $a,b+8,c$ theo thứ tự lập thành cấp số cộng và $a, b+8, c+64$ theo thứ tự lập thành cấp số nhân. Tính giá trị biểu thức $P=a-b+2c$.
% 	\choice
% 	{$P=\dfrac{184}{9}$}
% 	{\True $P=64$}
% 	{$P=\dfrac{92}{9}$}
% 	{$P=32$}
% 	\loigiai{
% 		Ta có $\heva{
% 			& ac=b^2 \\
% 			& a+c=2(b+8 ) \\
% 			& a(c+64 )={{(b+8 )}^2} \\
% 		}\Leftrightarrow \heva{
% 			& ac=b^2\quad \quad(1 ) \\
% 			& a-2b=16-c\quad(2 ) \\
% 			& ac+64a=(b+8 )^2\quad(3 )\\
% 		}$.
% 		Thay $(1)$ vào $(3)$ ta được: $$b^2+64a=b^2+16b+64\Leftrightarrow 4a-b=4. \quad
% 		(4 )$$
% 		Kết hợp $(2)$ với $(4)$ ta được: $\heva{
% 			& a-2b=16-c \\
% 			& 4a-b=4 \\
% 		}\Leftrightarrow \heva{
% 			& a=\dfrac{c-8}{7} \\
% 			& b=\dfrac{4c-60}{7}. \\
% 		}\,\,\,\,\,(5 )$ \\
% 		Thay $(5)$ vào $(1)$ ta được:\\
% 		$7(c-8 )c={{(4c-60 )}^2}\Leftrightarrow 9{{c}^2}-424c+3600=0\Leftrightarrow \hoac{
% 			& c=36 \\
% 			& c=\dfrac{100}{9} \\
% 		}\Leftrightarrow c=36\,\,(c\in \mathbb{Z} )$. \\
% 		Với $c=36\Rightarrow a=4,\,\,b=12\Rightarrow P=4-12+72=64$.}
% \end{ex}

% \begin{ex}%[1K2K7-6]
% 	Cho bốn số $a$, $b$, $c$, $d$ theo thứ tự đó tạo thành cấp số nhân với công bội khác $1$. Biết tổng ba số hạng đầu bằng $\dfrac{148}{9}$, đồng thời theo thứ tự đó $a$, $b$, $c$ lần lượt là số hạng thứ nhất, thứ tư và thứ tám của một cấp số cộng. Tính giá trị của biểu thức $T=a-b+c-d$.
% 	\choice
% 	{$T=-\dfrac{101}{27}$}
% 	{$T=\dfrac{100}{27}$}
% 	{\True $T=-\dfrac{100}{27}$}
% 	{$T=\dfrac{101}{27}$}
% 	\loigiai{
% 		Gọi $s$ ($s\neq 0$) là công sai của cấp số cộng. Vì $a$, $b$, $c$ theo thứ tự lần lượt là số hạng thứ nhất, thứ tư và thứ tám của cấp số cộng nên $b=a+3s$ và $c=a+7s$.\\
% 		Mặt khác, $a$, $b$, $c$ theo thứ tự tạo thành cấp số nhân với công bội khác $1$ nên
% 		\[ac=b^2 \Leftrightarrow a(a+7s)=(a+3s)^2 \Leftrightarrow as=9s^2 \Leftrightarrow a=9s \,\,(\text{vì } s\neq 0).\]
% 		Suy ra $b=12s$, $c=16s$.\\
% 		Theo giả thiết
% 		\[a+b+c=\dfrac{148}{9} \Leftrightarrow 9s+12s+16s=\dfrac{148}{9} \Leftrightarrow 37s=\dfrac{148}{9} \Leftrightarrow s=\dfrac{4}{9}.\]
% 		Suy ra $a=4$, $b=\dfrac{16}{3}$, $c=\dfrac{64}{9}$. Từ đó ta tính được $d=4\cdot \left(\dfrac{4}{3}\right)^3 = \dfrac{256}{27}$.\\
% 		Vậy $T=a-b+c-d = 4-\dfrac{16}{3}+\dfrac{64}{9}-\dfrac{256}{27} = -\dfrac{100}{27}$.
% 	}
% \end{ex}
% \begin{ex}%[1K2K7-6]
% 	Cho $x$ và $y$ là các số nguyên thỏa mãn các số $x+6y$ ,$5x+2y$, $8x+y$ theo thứ tự lập thành cấp cộng và các số $x-\dfrac{5}{3}y$, $y-1$, $2x-3y$ theo thứ tự lập thành cấp số nhân. Tính tổng $S=2x+3y$.
% 	\choice
% 	{$9$}
% 	{$6$}
% 	{$-6$}
% 	{\True $-9$}
% 	\loigiai{
% 		Vì các số $x+6y$ ,$5x+2y$, $8x+y$ theo thứ tự lập thành cấp cộng nên ta có
% 		$$ (x+6y)+(8x+y)=2(5x+2y)\Leftrightarrow x=3y. $$
% 		Vì các số $x-\dfrac{5}{3}y$, $y-1$, $2x-3y$ theo thứ tự lập thành cấp số nhân nên ta có
% 		$$ \left(x-\dfrac{5}{3}y\right) (2x-3y)=(y-1)^2.  $$
% 		Thay $x=3y$ vào phương trình trên, ta được
% 		\begin{eqnarray*}
% 			& & \left(3y-\dfrac{5}{3}y\right) (6y-3y)=(y-1)^2\\
% 			&\Leftrightarrow & 4y^2=y^2-2y+1\\
% 			&\Leftrightarrow & \hoac{& y=-1\\& y=\dfrac{1}{3}}.
% 		\end{eqnarray*}
% 		Ta loại trường hợp $y=\dfrac{1}{3}$ vì $y$ là số nguyên. Suy ra $x=3y=3(-1)=-3$. Vậy $$S=2x+3y=2(-3)+3(-1)=-9.$$
% 	}
% \end{ex}
\Closesolutionfile{ans}
% \begin{indapan}{10}
% 	{ans/ans-1K2-3-Dang6}
% \end{indapan}
\begin{dang}{Bài toán thực tế}
	\textit{Bài toán lãi kép:} Một người gửi tiết kiệm vào ngân hàng một số tiền $A$ với lãi suất $r\%$ mỗi kì hạn. Số tiền lãi sẽ được nhập vào vốn ban đầu để tính lãi cho kì hạn tiếp theo. Hỏi sau $n$ kì hạn thì người đó có tất cả bao nhiêu tiền?\\
	\textit{Lời giải:} Gọi $u_n$ là số tiền người đó có sau $n$ kì hạn. Ta có:
	\begin{itemize}
		\item Số tiền người đó có sau kì hạn thứ nhất là: $u_1=A+A\cdot r\%=A\left(1+r\%\right)$.
		\item Số tiền người đó có sau $n$ kì hạn là: $u_n=u_{n-1}+u_{n-1}\cdot r\%=u_n\left(1+r\%\right)$.
	\end{itemize}
	Suy ra dãy số $(u_n)$ là một cấp số nhân với số hạng đầu $u_1=A\left(1+r\%\right)$ và công bội $q=1+r\%$.\\
	Vậy số tiền người đó có sau $n$ kì hạn là: \fbox{$u_n=A\left(1+r\%\right)^n$}.
\end{dang}
\subsubsection{Ví dụ minh hoạ}
\begin{vd}%[VD]%[DCHT Toán 11 - KNTT -Tên Huỳnh Thanh Chí]%[1K2K7-2]
	Trong một lọ nuôi cấy vi khuẩn, ban đầu có $ 5\ 000 $ con vi khuẩn và số lượng vi khuẩn tăng lên thêm $ 8\% $ mỗi giờ. Hỏi sau $ 5 $ giờ thì số lượng vi khuẩn là bao nhiêu?
	\dapso{}
	\loigiai{
	Ta có $ A=5\ 000 $ là số lượng vi khuẩn ban đầu, $ r=8\%=0{,}08 $ là tỉ lệ gia tăng vi khuẩn sau một giờ.
	\begin{itemize}
	\item Tại thời điểm sau $ 1 $ giờ: $ u_1=5000+5000\cdot 0{,}08= 5\ 000\cdot(1{,}08)$.
	\item Tại thời điểm sau $ n $ giờ: $ u_n=u_{n-1}+u_{n-1}\cdot0{,}08=u_{n-1}\cdot (1{,}08)$.
	\end{itemize}
	Do đó ta có thể nhận thấy rằng, số lượng vi khuẩn ở thời gian $ n $ giờ là một cấp số nhân có số hạng đầu $ u_1=5000.1,08 $ và công bội $ q=1{,}08 $.\\
	% Suy ra số hạng tổng quát $ u_n=5\ 000\cdot (1{,}08)^{n} $.\\
	Vậy số lượng vi khuẩn sau $ 5 $ giờ là $ u_5=5\ 000\cdot (1{,}08)^{5}\approx 7346 $ (vi khuẩn).
	}
\end{vd}
\begin{vd}%[VD]%[1K2K3-3]
	Người ta thiết kế một cái tháp gồm $10$ tầng theo cách: Diện tích bề mặt trên của mỗi tầng bằng nửa diện tích bề mặt trên của tầng ngay bên dưới và diện tích bề mặt của tầng 1 bằng nửa diện tích bề mặt đế tháp. Biết diện tích bề mặt đế tháp là $12\, 288$ m$^2$, tính diện tích bề mặt trên cùng của tháp.
	\loigiai{
		Gọi $S$ là diện tích mặt đế và $T_1, T_2, \ldots, T_{10}$ là diện tích bề mặt của tầng 1, tầng 2, \ldots, tầng 10.\\
		Khi đó, ta có
		\allowdisplaybreaks
		\begin{eqnarray*}
			&T_1&=\dfrac{1}{2}\cdot S;\\
			&T_n&=\dfrac{1}{2}\cdot T_{n-1}
		\end{eqnarray*}
		Suy ra $\left(T_n\right)$ là cấp số nhân có số hạng đầu $T_1=\dfrac{1}{2}\cdot 12288=6144$ m$^2$ và công bội $q=\dfrac{1}{2}$.\\
		Vậy diện tích bề mặt trên cùng của tháp là $T_{10}=\dfrac{1}{2^{10}}\cdot 12288=12$ m$^2$.
	}
\end{vd}
\begin{vd}%[TH] %[DCHT Toán 11 - KNTT - Dung Phuong] %[1K2B7-7]
	Dân số trung bình của Việt Nam năm $2020$ là $97{,}6$ triệu người, tỉ lệ tăng dân số là $1{,}14 \% /$năm.
	\begin{flushright}
		\textit{(Nguồn: Niên giám thống kê của Việt Nam năm 2020, NXB Thống kê, 2021)}
	\end{flushright}
	Giả sử tỉ lệ tăng dân số không đổi qua các năm.
	\begin{enumerate}
		\item Sau 1 năm, dân số của Việt Nam sẽ là bao nhiêu triệu người (làm tròn kết quả đến hàng phần mười)?
		\item Viết công thức tính dân số Việt Nam sau $n$ năm kể từ năm $2020$.
	\end{enumerate}
	\dapso{$\approx 98{,}7$ triệu người}
	\loigiai{
		\begin{enumerate}
			\item Sau 1 năm, dân số của Việt Nam sẽ là
			\allowdisplaybreaks
			\begin{eqnarray*}
				u_1&=&97{,}6+97{,}6 \cdot 0{,}0114=97{,}6 \cdot(1+0{,}0114)\\
				&=&97{,}6 \cdot 1{,}0114 \approx 98{,}7 (\text{triệu người}).
			\end{eqnarray*} 
			\item Gọi $u_n$ là dân số của Việt Nam sau $n$ năm.\\
			Do tỉ lệ tăng dân số hàng năm là $1{,}14 \%$ nên ta có
			\allowdisplaybreaks
			\begin{eqnarray*}
				u_n &=&u_{n-1}+u_{n-1} \cdot 0{,}0114=u_{n-1} \cdot(1+0{,}0114) \\
				&=&u_{n-1} \cdot 1{,}0114\ \text{với}\ n \geq 2.
			\end{eqnarray*}
			Do đó, $\left(u_n\right)$ là cấp số nhân có số hạng đầu $u_1=97{,}6\cdot 1{,}0114$, công bội $q=1{,}0114$.\\
			Vậy dân số của Việt Nam sau $n$ năm kể từ năm $2020$ là
			\[u_n=97{,}6 \cdot 1{,}0114 \cdot 1{,}0114^{n-1}=97{,}6 \cdot 1{,}0114^n\ (\text{triệu người}). \]
		\end{enumerate}
	}
\end{vd}

\begin{vd}%[TH] %[DCHT Toán 11 - KNTT - Dung Phuong] %[1K2B7-7]
	Bác Linh gửi vào ngân hàng $100$ triệu đồng tiền tiết kiệm với hình thức lãi kép, kì hạn 1 năm với lãi suất $6 \% /$năm. Viết công thức tính số tiền (cả gốc và lãi) mà bác Linh có được sau $n$ năm (giả sử lãi suất không thay đổi qua các năm).
	\dapso{$100 \cdot 1{,}06^{n-1}$ triệu đồng}
	\loigiai{
		Gọi $u_n$ là số tiền (cả gốc lẫn lãi) mà bác Linh có được sau $n$ năm.\\
		Do lãi suất 1 năm là $6\%$ nên ta có
		\allowdisplaybreaks
		\begin{eqnarray*}
			u_n &=&u_{n-1}+u_{n-1} \cdot 0{,}06=u_{n-1} \cdot(1+0{,}06) \\
			&=&u_{n-1} \cdot 1{,}06\ \text{với}\ n \geq 2.
		\end{eqnarray*}
		Do đó, $\left(u_n\right)$ là cấp số nhân có số hạng đầu $u_1=100\cdot 1{,}06$ (triệu đồng), công bội $q=1{,}06$.\\
		Vậy số tiền mà bác Linh có được sau $n$ năm là
		\[u_n=100 \cdot 1{,}06^{n}\ (\text{triệu đồng}). \]
	}
\end{vd}
\begin{vd}%[VD] %[DCHT Toán 11 - KNTT - Dung Phuong] %[1K2K7-7]
	Một hình vuông có cạnh $1$ đơn vị dài được chia thành chín hình vuông nhỏ hơn và hình vuông ở chính giữa được tô màu xanh như hình. Mỗi hình vuông nhỏ hơn lại được chia thành chín hình vuông con, và mỗi hình vuông con ở chính giữa lại được tô màu xanh. Nếu quá trình này được tiếp tục lặp lại năm lần, thì tổng diện tích các hình vuông được tô màu xanh là bao nhiêu?
	\dapso{$\dfrac{26281}{39366}$}
	\begin{center}
		\begin{tikzpicture}[>=stealth,thick,scale=0.7]
			\def\n{1}
			\def\a{3}
			\pgfmathsetmacro{\m}{int(3^(\n))}
			\def\hv#1{
				\ifnum#1>0
				\fill[blue!50] (-\a/3,-\a/3) rectangle (\a/3,\a/3);
				\pgfmathtruncatemacro{\k}{#1-1}
				\foreach \i in {0,...,3}{\begin{scope}[shift={(90*\i:2)},scale=1/3]\hv{\k}\end{scope}}
				\foreach \i in {0,...,3}{\begin{scope}[shift={(45+90*\i:{4/sqrt(2)})},scale=1/3]\hv{\k}\end{scope}}
				\fi
			}
			\draw(-\a,-\a) rectangle (\a,\a);
			\hv{\n}
			\foreach \i in {0,1,...,\m}{
				\draw[blue!50] 
				({-\a+2*\i *\a/\m},\a)--++(270:2*\a)
				(\a,{-\a+2*\i *\a/\m})--++(180:2*\a)
				;
			}
		\end{tikzpicture}
		\begin{tikzpicture}[>=stealth,thick,scale=0.7]
			\def\n{2}
			\def\a{3}
			\pgfmathsetmacro{\m}{int(3^(\n))}
			\def\hv#1{
				\ifnum#1>0
				\fill[blue!50] (-\a/3,-\a/3) rectangle (\a/3,\a/3);
				\pgfmathtruncatemacro{\k}{#1-1}
				\foreach \i in {0,...,3}{\begin{scope}[shift={(90*\i:2)},scale=1/3]\hv{\k}\end{scope}}
				\foreach \i in {0,...,3}{\begin{scope}[shift={(45+90*\i:{4/sqrt(2)})},scale=1/3]\hv{\k}\end{scope}}
				\fi
			}
			\draw(-\a,-\a) rectangle (\a,\a);
			\hv{\n}
			\foreach \i in {0,1,...,\m}{
				\draw[blue!50] 
				({-\a+2*\i *\a/\m},\a)--++(270:2*\a)
				(\a,{-\a+2*\i *\a/\m})--++(180:2*\a)
				;
			}
		\end{tikzpicture}
		\begin{tikzpicture}[>=stealth,thick,scale=0.7]
			\def\n{4}
			\def\a{3}
			\def\hv#1{
				\ifnum#1>0
				\fill[blue!50] (-\a/3,-\a/3) rectangle (\a/3,\a/3);
				\pgfmathtruncatemacro{\k}{#1-1}
				\foreach \i in {0,...,3}{\begin{scope}[shift={(90*\i:2)},scale=1/3]\hv{\k}\end{scope}}
				\foreach \i in {0,...,3}{\begin{scope}[shift={(45+90*\i:{4/sqrt(2)})},scale=1/3]\hv{\k}\end{scope}}
				\fi
			}
			\draw (-\a,-\a) rectangle (\a,\a);
			\hv{\n}
		\end{tikzpicture}
	\end{center}
	\loigiai{
		Lần phân chia thứ nhất, $1$ hình vuông thành $9$ hình vuông con, diện tích hình vuông tô màu xanh là $u_1=\dfrac{1}{9}$.\\
		Lần phân chia thứ hai, $8$ hình vuông thành $9$ hình vuông con, diện tích hình vuông tô màu xanh tăng thêm là $u_2=\dfrac{1}{9}\left(\dfrac{8}{9}\right)$.\\
		Lần phân chia thứ ba, $8^2$ hình vuông thành $9$ hình vuông con, diện tích hình vuông tô màu xanh tăng thêm là $u_3=\dfrac{1}{9}\left(\dfrac{8}{9}\right)^2$.\\
		Lần phân chia thứ tư, $8^3$ hình vuông thành $9$ hình vuông con, diện tích hình vuông tô màu xanh tăng thêm là $u_4=\dfrac{1}{9}\left(\dfrac{8}{9}\right)^3$.\\
		Lần phân chia thứ năm, $8^4$ hình vuông thành $9$ hình vuông con, diện tích hình vuông tô màu xanh tăng thêm là $u_5=\dfrac{1}{9}\left(\dfrac{8}{9}\right)^4$.\\
		Như vậy diện tích các hình vuông tăng thêm sau mỗi lần chia  tạo thành cấp số nhân có công bội là $q=\dfrac{8}{9}$, số hạng đầu là $u_1=\dfrac{1}{9}$.\\
		Do đó, tổng diện tích hình vuông tô màu xanh sau $5$ lần chia là\\
		\[u_1+u_2+u_3+u_4+u_5=\dfrac{1-q^5}{1-q}\cdot u_1=\dfrac{1-\left(\dfrac{8}{9}\right)^5}{1-\dfrac{8}{9}}\cdot \dfrac{1}{9}=\dfrac{26281}{39366}.\]	
		
	}
\end{vd}

\begin{vd}%[TH] %[DCHT Toán 11 - KNTT - Dung Phuong] %[1K2B7-7]
	Một khay nước có nhiệt độ $23^\circ$ được đặt vào ngăn đá của tủ lạnh. Biết sau mỗi giờ, nhiệt độ của nước giảm $20\%$. Tính nhiệt độ của khay nước đó sau $6$ giờ theo đơn vị độ $C$.
	\dapso{$ \approx 7,5^\circ $ gam}
	\loigiai{
		Nhiệt độ sau mỗi giờ của khay nước theo thứ tự lập thành cấp số nhân với $u_1=23.(1-20\%)$ và $q=(1-20\%)$.\\
		Ta có $u_6=u_1.q^5=23.(1-20\%)^6 \approx 7,5$.\\
		Nhiệt độ của khay nước sau $6$ giờ là $ \approx 6,0^\circ $.
	}
	
\end{vd}
\begin{vd}%[TH] %[DCHT Toán 11 - KNTT - Dung Phuong] %[1K2B7-7]
	Chu kì bán rã của nguyên tố phóng xạ poloni $210$ là $138$ ngày, nghĩa là sau $138$ ngày, khối lượng của nguyên tố đó chi còn một nửa (theo: https://vi.wikipedia.org/wiki/Poloni-210). Tính khối lượng còn lại của $20$ gam poloni $210$ sau:
	\begin{listEX}[2]
		\item [a)]  $690$ ngày;
		\item [b)] $7314$ ngày (khoảng $20$ năm).
	\end{listEX}
	\dapso{$\dfrac{20}{2^{53}}$ gam}
	\loigiai{
		\begin{listEX}[1]
			\item [a)] Ta có $\dfrac{690}{138}=5$ suy ra khối lượng còn lại sau 690 này là $\dfrac{20}{2^5}=0{,}625$ gam;
			\item [b)] Ta có $\dfrac{7314}{138}=53$ suy ra khối lượng còn lại sau 7314 này là $\dfrac{20}{2^{53}}$ gam.
		\end{listEX}
	}
\end{vd}	
\begin{vd}%[TH] %[DCHT Toán 11 - KNTT - Dung Phuong] %[1K2K7-7]
	Tế bào E.Coli trong điều kiện nuôi cấy thích hợp cứ $20$ phút lại phân đôi một lần. Hỏi sau $24$ giờ, tế bào ban đầu sẽ phân chia thành bao nhiếu tế bào?
	\dapso{$2^{72}$}
	\loigiai{
		Lần phân chia thứ nhất, $1$ tế bào thành $2$ tế bào, số tế bào lần $1$ phân chia là $u_1 = 2$.\\
		Lần phân chia thứ hai $2$, số tế bào lần $2$ phân chia là  $u_2=2\cdot 2 = u_1 \cdot 2$.\\
		Lần phân chia thứ $3$ có  $4$ tế bào phân chia, số tế bào lần $3$ phân chia là $u_3=2\cdot u_2$.\\
		Như vậy một tế bào phân đôi sẽ tạo thành cấp số nhân có công bội là $2$, số hạng đầu là $u_1=2$.\\
		Sau $n$ lần phân chia từ một tế bào phân được thành $u_n=2^{n-1}u_1$.\\
		Đổi $24$ giờ $=24 \cdot 60 =  72 \cdot 20$ (phút)  $\Rightarrow 24$ giờ gấp $72$ lần $20$ phút. \\
		Do đó, sau $24$ giờ số tế bào nhận được là $u_{72}=2^{71}\cdot 2 = 2^{72}$ (tế bào).
	}
\end{vd}

\subsubsection{Bài tập tự luận}
 


\begin{bt}%[TH] %[DCHT Toán 11 - KNTT - Dung Phuong] %[1K2B7-7]
	Một quốc gia có dân số năm 2011 là $P$ triệu người. Trong $10$ năm tiếp theo, mỗi năm dân số tăng $a \%$. Chứng minh rằng dân số các năm từ năm 2011 đến năm 2021 của quốc gia đó tạo thành cấp số nhân. Tìm công bội của cấp số nhân này.
	\loigiai{
		Coi ngày điều tra dân số năm 2011 và năm 2021 trùng nhau thì từ năm 2011 đến năm 2021 là 10 năm. Vậy dân số nước ta tính đến năm 2021 là 
		\[u_{10} = P\cdot \left(1+a\%\right)^{10}.\]
		Ta có \[u_{1} = P\cdot \left(1+a\%\right)^{1}.\]
		\[u_{2} = P\cdot \left(1+a\%\right)^{2}.\]
		Và công bội của cáp số nhân này là $\, \dfrac{u_2}{u_1} = q = \dfrac{P\cdot \left(1+a\%\right)^{2}}{P\cdot \left(1+a\%\right)^{1}} = 1+a\%.$
	}
\end{bt}
\begin{bt}%[TH] %[DCHT Toán 11 - KNTT - Dung Phuong] %[1K2B7-7]
	Vào năm 2020, dân số của một quốc gia là khoảng $97$ triệu người và tốc độ tăng trưởng dân số là $0{,}91 \%$. Nếu tốc độ tăng trưởng dân số này được giữ nguyên hằng năm, hãy ước tính dân số của quốc gia đó vào năm 2030.
	\dapso{$106{,}1973784$}
	\loigiai{
		Dân số năm 2021 tăng lên so với năm 2020 là $97 \cdot 0{,}91 \% $ triệu người.\\
		Dân số năm 2021 là 
		\begin{center}
			$97 + 97 \cdot 0{,}91 \% = 97\cdot (1+0{,}91 \%)$ triệu người.
		\end{center}
		Dân số năm 2022 tăng lên so với năm 2021 là $97\cdot (1+0{,}91 \%)\cdot 0{,}91 \% $ triệu người.\\
		Dân số năm 2022 là 
		\begin{center}
			$97\cdot (1+0{,}91 \%) + 97\cdot (1+0{,}91 \%) \cdot 0{,}91 \% = 97\cdot (1+0{,}91 \%)^2$ triệu người.
		\end{center}
		Dân số năm 2023 tăng lên so với năm 2021 là $97\cdot (1+0{,}91 \%)^2\cdot 0{,}91 \% $ triệu người.\\
		Dân số năm 2023 là
		\begin{center}
			$97\cdot (1+0{,}91 \%)^2 + 97\cdot (1+0{,}91 \%)^2\cdot 0{,}91 \% = 97\cdot (1+0{,}91 \%)^3$ triệu người.
		\end{center}
		Tương tự vậy ta có dân số năm 2030 là $97\cdot (1+0{,}91 \%)^{10} = 106{,}1973784$ triệu người.
	}
\end{bt}
\begin{bt}%[TH] %[DCHT Toán 11 - KNTT - Dung Phuong] %[1K2B7-7]
	Một tỉnh có $2$ triệu dân vào năm 2020 với tỉ lệ tăng dân số là $1$ \%/năm. Gọi $u_n$ là số dân của tỉnh đó sau $n$ năm. Giả sử tỉ lệ tăng dân số là không đổi.
	\begin{enumEX}[a)]{1} 
		\item Viết công thức tính số dân của tỉnh đó sau $n$ năm kể từ năm 2020.
		\item Tính số dân của tỉnh đó sau $10$ năm kể từ năm 2020.  
	\end{enumEX}
	\loigiai{
		\begin{enumEX}[a)]{1} 
			\item Với $u_n$ là số dân của tỉnh đó sau $n$ năm. \\
			Ta có $u_1=2 \cdot 1,01$ (triệu dân).\\
			$u_{n+1}=u_n+u_n\cdot 0{,}01 = 1{,}01u_n$. \\
			Do đó, $(u_n)$ là cấp số nhân với số hạng đầu $u_1=2 \cdot 1,01$ và công bội $q=1{,}01$. \\
			Vậy công thức tính số dân của tỉnh đó sau $n$ năm là $u_n=u_1q^{n-1}\Rightarrow u_n=2\cdot 1{,}01^{n}$.
			\item Số dân của tỉnh đó sau $10$ năm kể từ năm 2020 là $u_{10}=2\cdot 1{,}01^10 = 2{,}209$ (triệu dân).
		\end{enumEX}
	}
\end{bt}
\begin{bt}%[TH] %[DCHT Toán 11 - KNTT - Dung Phuong] %[1K2B7-7]
	Giả sử một thành phố có dân số năm 2022 là khoảng $2{,}1$ triệu người và tốc độ gia tăng dân số trung bình mỗi năm là $0{,}75 \%$.
	\begin{listEX}[1]
		\item [a)]  Dự đoán dân số của thành phố đó vào năm $2032$; \dapso{$\approx2262924$ (người)}
		\item [b)]  Nếu tốc độ gia tăng dân số vẫn giữ nguyên như trên thì uớc tính vào năm nào dân số của thành phố đó sẽ tăng gấp đôi so với năm $2022$?  \dapso{$2116$}	\end{listEX}
	
	\loigiai{
		\begin{listEX}
			\item [a)] 	Giả sử dân số năm $2022$ là $u_1=2{,}1\cdot 10^6$ thì dân số năm 
			$2023$ 
			là 
			$u_2=u_1+ 0{,}0075u_1=1{,}0075u_1$.\\
			Tương tự dân số năm $2024$ là $u_3=1{,}0075u_2$.\\
			Do đó dân số của thành phố qua các năm lập thành một cấp số nhân với 
			$u_1=2{,}1\cdot10^6$; $q=1{,}0075$.\\
			Vậy dân số năm $2032$ tương ứng với $u_{11}=u_1\cdot q^{10}=2,1\cdot 
			10^6\cdot1{,}0075^{10}\approx2262924$ (người).
			\item [b)] Giả sử đến năm thứ $n$ thì dân số gấp đôi năm $2022$. \\
			Suy ra 
			$u_n=2u_1 \Leftrightarrow q^{n-1}=2\Leftrightarrow  1{,}0075^{n-1}=2 
			\Leftrightarrow n \approx 93{,}7.$\\
			Vậy $94$ năm sau tức là năm $2116$ thì dân số thành phố sẽ gấp đôi năm $2022$.
	\end{listEX}}
\end{bt}
\begin{bt}%[TH] %[DCHT Toán 11 - KNTT - Dung Phuong] %[1K2B7-7]
	Giả sử anh Tuấn kí hợp đồng lao động trong $10$ năm với điều khoản về tiền lương như sau: Năm thứ nhất, tiền lương của anh Tuấn là $60$ triệu. Kể từ năm thứ hai trở đi, mỗi năm tiền lương của anh Tuấn được tăng lên $8 \%$. Tính tổng số tiền lương anh Tuấn lĩnh được trong $10$ năm đi làm (đơn vị: triệu đồng, làm tròn đến hàng phần nghìn).
	\dapso{$\approx 869{,}194$ triệu người}
	\loigiai{
		Gọi $u_n$ là số tiền lương (triệu đồng) anh Tuấn được lĩnh ở năm làm việc thứ $n$. Ta có: $u_1=60$;
		\[u_n=u_{n-1}+u_{n-1} \cdot 0{,}08=u_{n-1} \cdot(1+0{,}08)=u_{n-1} \cdot 1{,}08. \]
		Do đó, $\left(u_n\right)$ là cấp số nhân có số hạng đầu $u_1=60$, công bội $q=1{,}08$. Áp dụng công thức tính tổng $S_n$, ta có tổng số tiền lương anh Tuấn lĩnh được trong $10$ năm đi làm là
		\[S_{10}=\dfrac{60\cdot\left(1-1{,}08^{10}\right)}{1-1{,}08} \approx 869{,}194\ (\text{triệu người}). \]
	}
\end{bt}
\begin{bt}%[TH] %[DCHT Toán 11 - KNTT - Dung Phuong] %[1K2B7-7]
	Một công ty xây dựng mua một chiếc máy ủi với giá $3$ tỉ đồng. Cứ sau mỗi năm sử dụng, giá trị của chiếc máy ủi này lại giảm $20 \%$ so với giá trị của nó trong năm liền trước đó. Tìm giá trị còn lại của chiếc máy ủi đó sau $5$ năm sử dụng.
	\dapso{$983$ triệu đồng}
	\loigiai{
		Gọi $u_n$ là Giá trị của máy ủi sau $n$ sử dụng. \\
		Dãy số ($u_n$) là một cấp số nhân có $u_1=3.0,8$, $q=0{,}8$.\\
		Số hạng tổng quát của cấp số nhân này là $u_n=3\cdot 0{,}2^{n}$.\\
		Ta có $u_5=3\cdot 0{,}8^5=0{,}98304$.\\
		Tương ứng giá trị của chiếc máy ủi sau $5$ năm xấp xỉ $983$ triệu đồng.
	}
\end{bt}
\begin{bt}%[TH] %[DCHT Toán 11 - KNTT - Dung Phuong] %[1K2B7-7]
	Một gia đình mua một chiếc ô tô giá $800$ triệu đồng. Trung bình sau mỗi năm sử dụng, giá trị còn lại của ô tô giảm đi $4 \%$ (so với năm trước đó).
	\begin{enumEX}[a)]{1} 
		\item Viết công thức tính giá trị của ô tô sau $1$ năm, $2$ năm sử dụng.
		\item Viết công thức tính giá trị của ô tô sau $n$ năm sử dụng.
		\item Sau $10$ năm, giá trị của ô tô ước tính còn bao nhiêu triệu đồng? 
	\end{enumEX} 
	\dapso{$\approx 531{,}87$ triệu đồng}
	\loigiai{
		Gọi $u_n$ là giá trị còn lại của ô tô sau $n$ năm sử dụng. 
		\begin{enumEX}[a)]{1} 
			\item Giá trị của ô tô sau $1$ năm sử dụng là $u_1=800-800\cdot0{,}04=800\cdot0{,}96=768$ triệu đồng.\\
			Giá trị của ô tô sau $2$ năm sử dụng là $u_2=u_1-u_1\cdot0{,}04=u_1\cdot0{,}96=737{,}28$ triệu đồng.
			\item Ta có $u_n=u_{n-1}-u_{n-1}\cdot0{,}04=u_{n-1}\cdot0{,}96$. \\
			Do đó, $(u_n)$ là cấp số nhân với số hạng đầu $u_1=768$ và công bội $q=0{,}96$. \\
			Vậy sau $n$ năm sử dụng, giá trị còn lại của chiếc ô tô là $u_n=u_1q^{n-1}\Rightarrow u_n=768\cdot0{,}96^{n-1}$.
			\item Sau $10$ năm, ước tính giá trị của ô tô còn lại là $u_{10}=768\cdot0{,}96^9\approx 531{,}87$ triệu đồng.
		\end{enumEX} 
	}
\end{bt}
% \begin{bt} [VD] %[DCHT Toán 11 - KNTT - Dung Phuong] %[1K2K7-7]
% 	Ông An vay ngân hàng $1$ tỉ đồng với lãi suất $12\%/$năm. Ông đã trả nợ theo cách: Bắt đầu từ tháng thứ nhất sau khi vay, cuối tháng ông trả ngân hàng số tiền là $a$ (đồng) và đã trả hết nợ sau đúng $2$ năm kể từ ngày vay. Hỏi số tiền mỗi tháng mà ông An phải trả là bao nhiêu đồng (làm tròn kết quả đến hàng nghìn)?
% 	\dapso{$47073500$}
% 	\loigiai{
% 		Do lãi suất là $12\%$/năm tương đương với lãi là $1\%$/tháng.\\
% 		Sau $1$ tháng, ông An còn nợ là: $10^9.(1+1\%)-a=10^9.(1,01)-S_1$.\\
% 		Sau $2$ tháng, ông An còn nợ là: $10^9.(1.01)^2-a.(1.01)-a=10^9(1,01)^2-S_2$.\\
% 		Sau $3$ tháng, ông An còn nợ là: $10^9.(1.01)^3-a(1.01)^2-a(1.01)-a=10^9.(1.01)^3-S_3$.\\
% 		Sau $24$ tháng, ông An còn nợ là: $10^9.(1.01)^{24}-S_{24}=0$.\\
% 		Do đó $S_{24}=10^9.(1.01)^{24}  \Leftrightarrow a.\dfrac{1-(1.01)^{24}}{1-(1.01)}=10^9.(1.01)^{24} \Leftrightarrow a =\dfrac{10^9.(1.01)^{24}.0.01}{(1.01)^{24}-1}\approx 47073472,22$.\\
% 		Vậy mỗi tháng ông An phải trả $47073500$.
% 	}
% \end{bt}
\begin{bt}%[VD] %[DCHT Toán 11 - KNTT - Dung Phuong] %[1K2K7-7]
	\immini
	{
		Một người nhảy bungee (một trò chơi mạo hiểm mà người chơi nhảy từ một nơi có địa thế cao xuống với dây đai an toàn buộc xung quanh người) từ một cây cầu và căng một sợi dây dài $100$ m. Sau mỗi lần rơi xuống, nhờ sự đàn hồi của dây, người nhảy được kéo lên một quãng đường có độ dài bằng $75$\% so với lần rơi trước đó và lại bị rơi xuống đúng bằng quãng đường vừa được kéo lên. Tính tổng quãng đường người đó đi được sau $10$ lần kéo lên và lại rơi xuống. 
	}
	{
		\begin{tikzpicture}[xscale=.5, font=\small, line join=round, line cap=round, >=stealth,yscale=1]
			\def\a{-0.12} % Hệ số a phải khác 0
			\def\b{0.86}
			\def\c{0}
			\def\m{-0.12} % Hệ số a phải khác 0
			\def\n{0.86}
			\def\p{-0.3}
			\clip (-2,-2)rectangle(9,2);
			\fill[red!40] (-2,1.8)--(9,1.8)--(9,1.55)--(-2,1.55)--cycle;%Mặt phẳng của cây cầu
			\fill[green!50] (-2,1.55)--(0,0)--(1.5,-1)--(-1,-2)--(-2,-2)--cycle;
			\fill[green!50] (9,1.55)--(7,0)--(5.5,-1)--(8,-2)--(9,-2)--cycle;
			\fill[blue!40] (-1,-2)--(8,-2)--(5.5,-1)--(1.5,-1)--cycle;
			\draw[color=blue!50,line width=2pt,<->] (2,-.5)--(2,1.5)node[left,midway]{$100$ m};	
			\draw[color=blue!50,line width=2pt,<->] (5,-.5)--(5,1)node[midway,right]{$0{,}75\cdot100$ m};
			\draw(3.5,1.55)--(3.5,-.25)node[rotate=200]{\faChild};
			\draw[color=white,<->] (3.5,1.8)--(3.5,1.55); 
		\end{tikzpicture}
	}
	\dapso{$\approx666{,}2 \text{ m}$}
	\loigiai{
		Gọi $u_n$ là quãng đường người đó được kéo lên ở lần thứ $n$ được kéo lên và lại rơi xuống (đơn vị tính: mét). \\
		Ta có $u_1=0{,}75\cdot100=100\cdot1{,}5=75$ m và $u_n=0{,}75\cdot u_{n-1}$. \\
		Vậy $(u_n)$ là cấp số nhân với số hạng đầu $u_1=75$ và công bội $q=0{,}75$. \\
		Tổng quãng đường người đó đi được sau $10$ lần kéo lên và lại rơi xuống là 
		$$\begin{aligned}
			S&=100+2u_1+2u_2+\cdots+2u_{10}\\
			&=100+2S_{10}
			=100+2\cdot\dfrac{75\left(1-0{,}75^{10}\right)}{1-0{,}75}\\
			&\approx666{,}2 \text{ m}.
		\end{aligned}$$ 
	}
\end{bt}
\begin{bt} [TH] %[DCHT Toán 11 - KNTT - Dung Phuong] %[1K2B7-7]
	Một cái tháp có $11$ tầng. Diện tích của mặt sàn tầng $2$ bằng nửa diện tích của mặt đáy tháp và diện tích của mặt sàn mỗi tầng bằng nửa diện tích của mặt sàn mỗi tầng ngay bên dưới. Biết mặt đáy tháp có diện tích là $12 288m^2$. Tính diện tích của mặt sàn tầng trên cùng của tháp theo đơn vị mét vuông.
	\dapso{$12m^2$}
	\loigiai{ (Lưu ý: Một số nơi xem tầng 1 là tầng trệt. Nên bài toán này giống bài toán tháp 10 tầng ở phần trên)
		Do diện tích của mặt sàn tính từ tầng một lập thành một cấp số nhân với $u_2=\dfrac{1}{2}.12288=6144$ và $q=\dfrac{1}{2}$.\\
		Ta có $\heva{u_2&=6144 \\ q&=\dfrac{1}{2}}  \Leftrightarrow \heva{u_1&=12288 \\ q&=\dfrac{1}{2}}$.\\
		Ta có $u_{11}=u_1.q^{10}=12288.\dfrac{1}{2^{10}}=12m^2$.
		Vậy diện tích của mặt sàn tầng trên cùng là	$12m^2$.
	}
\end{bt}

\begin{bt}%[TH]%[DCHT Toán 11 - KNTT - Dung Phuong]%[1K2B7-7]
	\immini{Cho hình vuông $C_1$ có cạnh bằng $4$. Người ta chia mỗi cạnh hình vuông thành bốn phần bằng nhau và nối các điểm chia một cách thích hợp để có hình vuông $C_2$ . Từ hình vuông $C_2$ lại làm tiếp tục như trên để có hình vuông $C_3$. Cứ tiếp tục quá trình như trên, ta nhận được dãy các hình vuông $C_1, C_2, C_3, \ldots , C_n, \ldots$ Gọi $a_n$ là độ dài cạnh hình vuông $C_n$. Chứng minh rằng dãy số $\left(a_n\right)$ là cấp số nhân.}{
		\begin{tikzpicture}[scale=.8]
			\def\a{2}  %cạnh hình vuông
			\def\t{.7}  % tỷ lệ điểm cho vòng lặp tiếp
			\path 
			(-\a,-\a) coordinate (A1)
			(-\a,\a) coordinate (B1)
			(\a,\a) coordinate (C1)
			(\a,-\a) coordinate (D1);
			\draw (A1)--(B1)--(C1)--(D1)--cycle;
			\foreach \i[count=\j from 2] in {1,...,10}
			\draw
			(barycentric cs:A\i=\t,B\i=1-\t) coordinate (A\j)--
			(barycentric cs:B\i=\t,C\i=1-\t) coordinate (B\j)--
			(barycentric cs:C\i=\t,D\i=1-\t) coordinate (C\j)--
			(barycentric cs:D\i=\t,A\i=1-\t) coordinate (D\j)--cycle
			;	
			% \node at (0,-2.2) [below]{\textit{Hình 4}};
		\end{tikzpicture}	
	}
	\loigiai{
		\immini{Gọi cạnh một hình vuông thứ $n$, $n+1$ lần lượt là $a_n, a_{n+1}$.\\
			Do $MN=\sqrt{MB^2+BN^2}=\sqrt{\left(\dfrac{AB}{4}\right)^2+\left(\dfrac{3AB}{4}\right)^2 }=AB\cdot\dfrac{\sqrt{10}}{4}$.\\
			Nên ta có cạnh hình vuông thứ $n+1$ là:\\ $a_{n+1}=a_n.\dfrac{\sqrt{10}}{4}$.\\
			Vậy dãy số $\left(a_n\right)$ là cấp số nhân.	
		}{
			\begin{tikzpicture}[scale=0.8,>=stealth, font=\footnotesize, line join=round, line cap=round]
				\path
				(0,0) coordinate (A)
				(4,0) coordinate (B)
				(4,4) coordinate (C)
				(0,4) coordinate (D)
				($(A)!0.75!(B)$) coordinate (M)
				($(B)!0.75!(C)$) coordinate (N)
				($(C)!0.75!(D)$) coordinate (P)
				($(D)!0.75!(A)$) coordinate (Q)
				;
				\draw (A)--(B)--(C)--(D)--cycle (M)--(N)--(P)--(Q)--cycle;
				\node at ($(A)!0.5!(B)$)[below]{$a_n$};
				\node at ($(M)!0.5!(N)$)[left]{$a_{n+1}$};
				\foreach \p/\q in {A/180,B/0,C/0,D/180,M/-90,N/0,P/90,Q/180}
				\fill[black] (\p) circle (1.0pt) ($(\p)+(\q:2.5mm)$) node{$\p$};
		\end{tikzpicture}
	}
	}
\end{bt}

\begin{bt}%[TH] %[DCHT Toán 11 - KNTT - Dung Phuong] %[1K2B7-7]
	Một cây đàn organ có tần số âm thanh các phím liên tiếp tạo thành một cấp số nhân. Cho biết tần số phím La trung là $400$ Hz và tần số của phím La cao cao hơn $12$ phím là $800$ Hz (nguồn: https://vi.wikipedia.org/wikiOrgan). Tìm công bội của cấp số nhân nói trên (làm tròn kết quả đến hàng phần nghìn).
	\dapso{$q = \pm \sqrt[12]{2}$}
	\loigiai{
		Theo đề ta có $\heva{&u_1=400\\&u_{13}=800} \Leftrightarrow \heva{&u_1=400\\&u_1q^{12}=800} \Rightarrow q^{12} = 2 \Rightarrow q = \pm \sqrt[12]{2}$.
	}
\end{bt}


\begin{bt}%[VD]%[DCHT Toán 11 - KNTT - Dung Phuong] %[1K2K7-7] 
	Một loại thuốc được dùng mỗi ngày một lần. Lúc đầu nồng độ thuốc trong máu của bệnh nhân tăng nhanh, nhưng mỗi liều kế tiếp có tác dụng ít hơn liều trước đó. Lượng thuốc trong máu ở ngày thứ nhất là $50 \,\mathrm{mg}$, và mỗi ngày sau đó giảm chỉ còn một nửa so với ngày kề trước đó. Tính tổng lượng thuốc (tính bằng $\mathrm{mg}$) trong máu của bệnh nhân sau khi dùng thuốc $10$ ngày liên tiếp.
	\dapso{$99{,}902$ mg.}
	\loigiai{
		Gọi $u_n$ là giá trị của lượng thuốc trong máu của bệnh nhân trong ngày thứ $n$. \\
		Dãy số này là một cấp số nhân có $u_1=50$, $q=\dfrac{1}{2}$.\\
		Tổng của $n$ số hạng đầu tiên của cấp số nhân là $S_n=u_1\dfrac{1-q^n}{1-q}$.\\
		Theo bài toán, ta có $S_{10}=50 \cdot\dfrac{1-\left(\dfrac{1}{2}\right)^{10}}{1-\dfrac{1}{2}} \approx 99{,}902$.\\
		Vậy tổng lượng thuốc trong máu của bệnh nhân sau khi dùng thuốc $10$ ngày liên tiếp là $99{,}902$ mg.	
	}
\end{bt}
\subsubsection{Câu hỏi trắc nghiệm}
\Opensolutionfile{ans}[ans/ans-1K2-2-Dang7]
\begin{ex}%[1K2K7-7]
	\immini{
		Cho hình vuông có cạnh là $1$. Nối các trung điểm của hình vuông trên ta được một hình vuông có diện tích $S_1$, tiếp tục quá trình trên với các hình vuông với diện tích là $S_2$; $S_3$; $\ldots ;S_n;\ldots$. Tính tổng vô hạn $S_1+ S_2+ S_3+\cdots+S_n+\cdots$.
		\choice
		{$2$}
		{$\dfrac{1}{2}$}
		{\True $1$}
		{$\dfrac{3}{2}$}
	}
	{\hspace*{1 cm}
		\begin{tikzpicture}[scale=0.8,line cap=round,line join=round]
			\path
			(0,0) coordinate (A)
			(4,0) coordinate (B)
			(0,4) coordinate (D);			
			\coordinate (C) at ($(B)-(A)+(D)$);
			\coordinate (H) at ($(A)!0.5!(B)$);
			\coordinate (I) at ($(A)!0.5!(D)$);
			\coordinate (J) at ($(D)!0.5!(C)$);
			\coordinate (K) at ($(B)!0.5!(C)$);
			\coordinate (E) at ($(I)!0.5!(H)$);
			\coordinate (F) at ($(H)!0.5!(K)$);
			\coordinate (G) at ($(K)!0.5!(J)$);
			\coordinate (O) at ($(J)!0.5!(I)$);
			\coordinate (M) at ($(E)!0.5!(F)$);
			\coordinate (N) at ($(F)!0.5!(G)$);
			\coordinate (P) at ($(G)!0.5!(O)$);
			\coordinate (Q) at ($(O)!0.5!(E)$);
			\draw (A)--(B)--(C)--(D)--cycle (I)--(H)--(K)--(J)--cycle
			(E)--(F)--(G)--(O)--cycle (M)--(N)--(P)--(Q)--cycle;
			\foreach \p in {A,B,C,D,E,F,G,H,I,J,K,M,N,P,Q,O}
			\fill[black] (\p) circle (1.0pt);			
		\end{tikzpicture}
	}
	\loigiai{
		Ta có $S_1=\dfrac{1}{2}$, $S_2=\dfrac{1}{4}$, $S_3=\dfrac{1}{8},\cdots  S_n=\dfrac{1}{2^n},\ldots$ tạo thành $1$ cấp số nhân với công bội $q=\dfrac{1}{2}<1$. \\
		Vậy $S_1+ S_2+ S_3+\cdots+S_n+\cdots=\dfrac{\dfrac{1}{2}}{1-\dfrac{1}{2}}=1$.
	}
\end{ex}
\begin{ex}%[1K2K7-7]
	Cho $n$ là số nguyên dương và $n$ tam giác $A_1B_1C_1,A_2B_2C_2,\ldots,A_nB_nC_n$, trong đó các điểm lần ${A}_{i+1},{B}_{i+1},{C}_{i+1}$ lượt nằm trên các cạnh $B_iC_i,A_iC_i,A_iB_i(i=1,2,\ldots,n-1)$ sao cho ${A}_{i+1}C_i=3{A}_{i+1}B_i,{B}_{i+1}A_i=3{B}_{i+1}C_i,{C}_{i+1}B_i=3{C}_{i+1}A_i$. Gọi $S$ là tổng tất cả các diện tích của tam giác $A_1B_1C_1,A_2B_2C_2,\ldots,A_nB_nC_n$ biết rằng tam giác $A_1B_1C_1$ có diện tích bằng $\dfrac{9}{16}$. Tìm số nguyên dương sao cho $S=\dfrac{{16}^{29}-7^{29}}{{16}^{29}}$.
	\choice
	{$n=28$}
	{$n=2018$}
	{$n=30$}
	{\True $n=29$}
	\loigiai{
		Gọi $S_i(i=1,2,3,...,n)$ là diện tích của $\Delta A_iB_iC_i$. Ta có $\dfrac{S_{A_1B_2C_2}}{S_{A_1B_1C_1}}=\dfrac{A_1B_2}{A_1C_1}\cdot \dfrac{A_1C_2}{A_1B_1}=\dfrac{1}{4}\cdot \dfrac{3}{4}=\dfrac{3}{16}$. Tương tự, ta có $\dfrac{S_{A_2B_1C_2}}{S_{A_1B_1C_1}}=\dfrac{S_{A_2B_2C_1}}{S_{A_1B_1C_1}}=\dfrac{3}{16}$. Do đó $\dfrac{S_{A_2B_2C_2}}{S_{A_1B_1C_1}}=1-3\cdot \dfrac{3}{16}=\dfrac{7}{16}\Rightarrow S_2=\dfrac{7}{16}S_1$.\\
		Tương tự, ta có ${S}_{i+1}=\dfrac{7}{16}S_i,i=1,2,\ldots,n$.
		Khi đó $$S=S_1\left[1+\dfrac{7}{16}+\cdots+{\left(\dfrac{7}{16}\right)}^{n-1}\right]=\dfrac{9}{16}\cdot \dfrac{1-{\left(\dfrac{7}{16}\right)}^n}{1-\dfrac{7}{16}}=1-{\left(\dfrac{7}{16}\right)}^n.$$
		Theo giả thiết ta có $1-{\left(\dfrac{7}{16}\right)}^n=1-{\left(\dfrac{7}{16}\right)}^{29}\Leftrightarrow n=29$.}
\end{ex}
\begin{ex}%[1K2K7-7]
	Người ta thiết kế một cái tháp gồm $11$ tầng. Diện tích bề mặt trên của mỗi tầng bằng nửa diện của mặt trên tầng ngay bên dưới và diện tích tầng $1$ bằng nửa diện tích của đế tháp. Biết đế tháp có diện tích là $12288\, \mathrm{m}^2$. Tính diện tích mặt trên cùng.
	\choice
	{$12\, \mathrm{m}^2$}
	{\True $6\, \mathrm{m}^2$}
	{$10\, \mathrm{m}^2$}
	{$8\, \mathrm{m}^2$}
	\loigiai{
		Gọi $S_{i}$ là diện tích của tầng thứ $i$ với $i = 1,2,\ldots,11$.\\
		Do giả thiết suy ra $S_{i + 1} = \dfrac{1}{2}S_{i}$ với $i = 1,2,\ldots,10$.\\
		Do đó $\left\{S_{i}\right\}$ là một cấp số nhân với công bội $q = \dfrac{1}{2}$. Do đó  $S_{11} = \dfrac{1}{2^{10}}S_{1} = \dfrac{1}{2^{11}}\cdot 12288 = 6\left(\mathrm{m}^2\right)$.
	}
\end{ex}

\begin{ex}%[1K2K7-7]
	Cho tứ giác $ABCD$ có bốn góc tạo thành cấp số nhân có công bội $ q=2 $. Góc có số đo nhỏ nhất trong bốn góc đó là
	\choice
	{\True $ 24^\circ $}
	{$ 1^\circ $}
	{$ 12^\circ $}
	{$ 30^\circ $}
	\loigiai{
		Gọi số đo bốn góc của tứ giác $ ABCD $ là $ x $, $ 2x $, $ 4x $, $ 8x $.
		\\ Có $ x+2x+4x+8x=360 \Leftrightarrow 15x=360 \Leftrightarrow x=24 $.}
\end{ex}

\begin{ex}%[1K2K7-7]
	Một du khách vào chuồng đua ngựa đặt cược, lần đầu tiên đặt $20000$ đồng, mỗi lần sau tiền đặt gấp đôi lần tiền đặt cược trước. Người đó thua lần $9$ liên tiếp và thắng ở lần thứ $10$. Hỏi du khách đó thắng hay thua bao nhiêu tiền?
	\choice
	{\True Thắng $20000$ đồng}
	{Thua $40000$ đồng}
	{Hòa vốn}
	{Thua $20000$ đồng}
	\loigiai{
		Số tiền đặt cược lần thứ $n$ là $u_n=u_1\cdot 2^{n-1}$ với $u_1=20000$. \\
		Ta có: $u_{10}-\displaystyle\sum_{n=1}^9 u_1\cdot 2^{n-1}=20000\cdot 2^9-\displaystyle\sum_{n=1}^9 20000\cdot 2^{n-1}=20000$. \\
		Vậy du khách thắng $20000$ đồng.
	}
\end{ex}

% \begin{ex}%[1K2K7-7]
% 	Một người gửi tiết kiệm vào ngân hàng với lãi suất $7{,}5$ \%/năm. Biết rằng nếu không rút tiền ra khỏi ngân hàng thì cứ sau mỗi năm số tiền lãi sẽ được nhập vào vốn để tính lãi cho năm tiếp theo. Hỏi sau ít nhất bao nhiêu năm người đó thu được (cả số tiền gửi ban đầu và lãi) gấp đôi số tiền đã gửi, giả định trong khoảng thời gian này lãi suất không thay đổi và người đó không rút tiền ra?
% 	\choice
% 	{$12$ năm}
% 	{$11$ năm}
% 	{\True $10$ năm}
% 	{$9$ năm}
% 	\loigiai{
% 		Áp dụng công thức: $S_n=A(1+r)^n \Rightarrow n=\log_{(1+r)}\left(\dfrac{S_n}{A}\right) \Rightarrow n=\log_{\left(1+7{,}5\%\right)}(2)\approx 9{,}6$.}
% \end{ex}

\begin{ex}%[1K2K7-7]
	Cho tam giác $ ABC $ cân tại $ A $ có cạnh đáy $ BC $,  đường cao $ AH $ và cạnh bên $ AB $ theo thứ tự đó lập thành cấp số nhân công bội $ q $. Giá trị của $ q $ là
	\choice
	{$ q=\dfrac{1}{2}\sqrt{\sqrt{2}+1} $ }
	{$ q=\sqrt{2}+1 $ }
	{$ q=\sqrt{2(\sqrt{2}+1)} $}
	{\True $ q=\dfrac{1}{2}\sqrt{2(\sqrt{2}+1)} $ }
	\loigiai{
		Giả sử $ BC=u_1 $, $ AH=u_1\cdot q $ và $ AB=u_1\cdot q^2 $ với $ u_1> 0, q> 0 $.\\
		Do $ \triangle ABC $ cân tại $ A $ suy ra
		\begin{align*}
			AB^2=AH^2+\dfrac{BC^2}{4}\Leftrightarrow
			& u_1^2\cdot q^4=\dfrac{u_1^2}{4}+u_1^2\cdot q^2\\
			\Leftrightarrow & 4q^4-4q^2-1=0\\
			\Leftrightarrow & q^2=\dfrac{1\pm \sqrt{2}}{2}.
		\end{align*}
		Kết hợp với điều kiện bài toán ta có $ q=\sqrt{\dfrac{1+ \sqrt{2}}{2}}=\dfrac{1}{2}\sqrt{2(\sqrt{2}+1)} $.
	}
\end{ex}
\begin{ex}%[1K2K7-7]
	Giả sử một người đi làm được lĩnh lương khởi điểm là $2.000.000$ đồng/tháng. Cứ $3$ năm người ấy lại được tăng lương một lần với mức tăng bằng $7\%$ của tháng trước đó. Hỏi sau $36$ năm làm việc người ấy lĩnh được tất cả bao nhiêu tiền?
	\choice
	{\True $ 1.287.968.492 $ đồng}
	{$ 10.721.769.110 $ đồng}
	{$ 7{,}068289036\cdot 10^8 $ đồng}
	{$ 429.322.830{,}5 $ đồng}
	\loigiai{
		Ta có $36$ năm tương ứng với $12$ kỳ lương; mỗi kỳ lương có $36$ tháng và kỳ sau tăng $7\%$ so với kỳ trước. Do đó tổng số tiền mỗi kỳ lương là một cấp số nhân với $u_1=36\times 2=72$ (triệu đồng) và công bội $q=1{,}07$.\\
		Vậy tổng số tiền sau $36$ năm là $T=\dfrac{72\cdot \left[(1{,}07)^{12}-1\right]}{1{,}07-1}=1287{,}968492$ (triệu đồng).
	}
\end{ex}

\begin{ex}%[1K2K7-7]
	Từ độ cao $55{,}8$ (mét) của tháp nghiên Pisa nước Italia người ta thả một quả bóng cao su chạm xuống đất. Giả sử mỗi lần chạm đất bóng lại nảy lên độ cao bằng $\dfrac{1}{10}$ độ cao mà bóng đạt trước đó. Tổng độ dài hành trình (mét) của bóng được thả từ lúc ban đầu cho đến khi nó nằm yên trên mặt đất thuộc khoảng nào trong các khoảng sau đây?
	\choice
	{$(69;72)$}
	{$(60;63)$}
	{\True $(67;69)$}
	{$(64;66)$}
	\loigiai{
		Đặt $u_1=55{,}8$ (mét) là quãng đường bóng rơi khi thả xuống, $u_{n+1}=\dfrac{1}{10^{n}} u_1, n\ge 1$ là quãng đường bóng rơi sau lần nảy lên thứ $n$. \\
		Ta có $(u_n)$ là dãy cấp số nhân với $u_1=55{,}8$ và công bội $q=\dfrac{1}{10}$.\\
		Suy ra tổng quãng đường quả bóng rơi xuống là $\displaystyle \lim \limits_{n \rightarrow +\infty} u_1 \cdot \dfrac{1-q^n}{1-q}=\displaystyle \lim \limits_{n \rightarrow +\infty}55{,}8\cdot\dfrac{1-\left( \dfrac{1}{10}\right)^n}{1-\dfrac{1}{10}}=62 $.\\
		Ngoài ra ta còn phải tính tổng quãng đường mà bóng nảy lên. Ta có tổng quãng đường bóng nảy lên bằng tổng quãng đường rơi của bóng trừ đi quãng đường thả rơi xuống.\\
		Vậy tổng quãng đường hành trình của quả bóng là $62+62-55{,}8=68{,}2$ (mét).
	}
\end{ex}

\begin{ex}%[1K2K7-7]
	Một gia đình lập kế hoạch tiết kiệm như sau: Họ lập một sổ tiết kiệm tại một ngân hàng và cứ đầu mỗi tháng họ gửi
	vào sổ tiết kiệm đó $15$ triệu đồng. Giả sử lãi suất tiền gửi không đổi là $0{,}6$ \%/tháng và tiền gửi được tính lãi theo hình thức lãi
	kép. Hỏi sau $3$ năm gia đình đó tiết kiệm được số tiền gần nhất với con số nào dười đây?
	\choice
	{$543240000$ đồng}
	{$589269000$ đồng}
	{$669763000$ đồng}
	{\True $604359000$ đồng}
	\loigiai{
		Gọi $S_0$ triệu đồng là số tiền gia đình đó định kỳ gửi tiết kiệm vào đầu hằng tháng, $r$ là lãi suất tiền gửi hằng tháng. Ta có $S_0=15$ triệu đồng, $r=0{,}6$
		\%/tháng.\\
		Gọi $S_i$, $i=\overline{1,n}$ là số tiền trong sổ tiết kiệm cuối tháng thứ $i$.\\
		Ta có \begin{itemize}
			\item $S_1=S_0+S_0\cdot r=S_0(1+r)$,
			\item  $S_2=\left[ S_0+S_0(1+r)\right]+\left[ S_0+S_0(1+r)\right]r=S_0 (1+r)+S_0(1+r)^2$,
			\item  $\begin{aligned}[t]
				S_3=&\ \left[S_0+S_0(1+r)+S_0(1+r)^2 \right] +\left[S_0+S_0(1+r)+S_0(1+r)^2 \right]r\\
				=&\ S_0(1+r)+S_0(1+r)^2+S_0(1+r)^3,\end{aligned}$,
			\item \ldots
			\item$\begin{aligned}[t]S_n=&\ S_0(1+r)+S_0(1+r)^2+S_0(1+r)^3+\cdots +S_0(1+r)^n\\=&\ S_0\left[ (1+r)+(1+r)^2+(1+r)^3+\cdots+(1+r)^n\right]\\
				=&\  S_0(1+r)\cdot \dfrac{(1+r)^{n}-1}{(1+r)-1}=S_0(1+r)\cdot \dfrac{(1+r)^{n}-1}{r}.
			\end{aligned}$
		\end{itemize}
		Vậy sau $3$ năm, tức cuối tháng thứ $36$ thì gia đình tiết kiệm được số tiền là
		\[S_{36}=15\cdot 10^6(1+0{,}6\cdot 10^{-2})\cdot \dfrac{(1+0{,}6\cdot 10^{-2})^{36}-1}{0{,}6\cdot 10^{-2}}=604358538{,}2 \ \text{đồng}.\]
	}
\end{ex}
\Closesolutionfile{ans}
% \begin{indapan}{10}
% 	{ans/ans-1K2-2-Dang7}
% \end{indapan}

%%Ôn tập chương II
% \setcounter{dang}{0}
\setcounter{ex}{0}
\setcounter{bt}{0}
\setcounter{vd}{0}
\section*{Ôn tập chương 2}
\Opensolutionfile{ans}[ans/ans-1K2-Ontapchuong2]
\begin{ex}%[1K2Y5-2]
	Cho dãy số $\left(u_n\right)$, biết $u_n=\left(-1\right)^n.2n$. Mệnh đề nào sau đây sai?
	\choice
	{$u_1=-2$}
	{$u_2=4$}
	{$u_3=-6$}
	{\True $u_4=-8$}
	\loigiai{
		Thay trực tiếp vào kiểm tra, ta có
		\begin{eqnarray*}
			u_1&=&-2.1=-2\\
			u_2&=&(-1)^2.2.2=4\\
			u_3&=&(-1)^3.2.3=-6\\
			u_4&=&(-1)^4.2.4=8.
		\end{eqnarray*}
	}
\end{ex}
\begin{ex}%[1K2Y5-2]
	Cho dãy số $\left(u_n\right)$, biết $u_n=\left(-1\right)^n.\dfrac{2^n}{n}$. Tìm số hạng $u_3$.
	\choice
	{$u_3=\dfrac{8}{3}$}
	{$u_3=2$}
	{$u_3=-2$}
	{\True $u_3=-\dfrac{8}{3}$}
	\loigiai{
		Thay trực tiếp vào kiểm tra, ta có
		\begin{center}
			$u_3=(-1)^3.\dfrac{2^3}{3}=-\dfrac{8}{3}$.
		\end{center}
	}
\end{ex}
\begin{ex}%[1K2Y5-2]
	Cho dãy số $\left(u_n\right)$, biết $u_n=\dfrac{2n+5}{5n-4}$. Số $\dfrac{7}{12}$ là số hạng thứ mấy của dãy số?
	\choice
	{\True $8$}
	{$6$}
	{$9$}
	{$10$}
\end{ex}
\loigiai{
	Ta có
	\allowdisplaybreaks
	\begin{eqnarray*}
		&&u_n=\dfrac{2n+5}{5n-4}\\
		&\Leftrightarrow&\dfrac{7}{12}=	\dfrac{2n+5}{5n-4}\\
		&\Leftrightarrow&24n+60=35n-28\\
		&\Leftrightarrow&11n=88\\
		&\Leftrightarrow&n=8.
	\end{eqnarray*}
	Vậy số $\dfrac{7}{12}$ là số hạng thứ 8.
}
\begin{ex}%[1K2Y5-2]
	Cho dãy số $\left(u_n\right)$, biết $u_n=2^n$. Tìm số hạng $u_{n+1}$.
	\choice
	{\True $u_{n+1}=2^n.2$}
	{$u_{n+1}=2^n+1$}
	{$u_{n+1}=2\left(n+1\right)$}
	{$u_{n+1}=2^n+2$}
	\loigiai{
		Ta có
	}
	\loigiai{
		Thay $n$ bằng $n+1$ trong công thức $u_n$ ta được
		\allowdisplaybreaks
		\begin{eqnarray*}
			u_{n+1}&=&2^{n+1}\\
			& =&2.2^n.
		\end{eqnarray*}
	}
\end{ex}
\begin{ex}%[1K2B5-2]
	Cho dãy số $\left(u_n\right)$, biết $u_n=5^{n+1}$. Tìm số hạng $u_{n-1}$.
	\choice
	{$u_{n-1}=5^{n-1}$}
	{\True $u_{n-1}=5^{n}$}
	{$u_{n-1}=5.5^{n+1}$}
	{$u_{n-1}=5.5^{n-1}$}
	\loigiai{
		Thay $n$ bằng $n-1$ trong công thức $u_n$ ta được
		\allowdisplaybreaks
		\begin{eqnarray*}
			u_{n-1}& = &5^{n-1+1}\\
			& = &5^n.
		\end{eqnarray*}
	}
\end{ex}
\begin{ex}%[1K2Y5-1]
	Cho dãy số có các số hạng đầu là $-2;0;2;4;6;...$. Số hạng tổng quát của dãy số này là công thức nào dưới đây?
	\choice
	{$u_n=-2n$}
	{$u_n=n-2$}
	{$u_n=-2\left(n+1\right)$}
	{\True $u_n=2n-4$}
	\loigiai{
		Kiểm tra $u_1=-2$ ta loại các đáp án B và C. Tương tự kiểm tra $u_2=0$ ta loại đáp án A.
	}
\end{ex}
\begin{ex}%[1K2B5-1]
	Cho dãy số $\left(u_n\right)$, được xác định $\heva{&u_1=\dfrac{1}{2}\\&u_{n+1}=u_n-2}$. Số hạng tổng quát $u_n$ của dãy số là số hạng nào dưới đây?
	\choice
	{$u_n=\dfrac{1}{2}+2\left(n-1\right)$}
	{\True $u_n=\dfrac{1}{2}-2\left(n-1\right)$}
	{$u_n=\dfrac{1}{2}-2n$}
	{$u_n=\dfrac{1}{2}+2n$}
	\loigiai{
		Ta có
		\begin{center}
			$\heva{&u_1=\dfrac{1}{2}\\&u_{n+1}=u_n-2}\Rightarrow\heva{&u_1=\dfrac{1}{2}\\&u_2=-\dfrac{3}{2}\\&u_3=-\dfrac{7}{2}}$
		\end{center} 
		Ta thấy chỉ có đáp án B đều thoả mãn.
	}
\end{ex}
\begin{ex}%[1K2B5-1]
	Cho dãy số $\left(u_n\right)$, được xác định $\heva{&u_1=-2\\&u_{n+1}=-2-\dfrac{1}{u_n}}$. Số hạng tổng quát $u_n$ của dãy số là số hạng nào dưới đây?
	\choice
	{$u_n=\dfrac{-n+1}{n}$}
	{$u_n=\dfrac{n+1}{n}$}
	{\True $u_n=-\dfrac{n+1}{n}$}
	{$u_n=-\dfrac{n}{n+1}$}
	\loigiai{
		Ta có
		\begin{center}
			$\heva{&u_1=-2\\&u_{n+1}=-2-\dfrac{1}{u_n}}\Rightarrow\heva{&u_1=-2\\&u_2=-\dfrac{3}{2}}$
		\end{center}
		Ta thấy chỉ có đáp án C thoả mãn.
	}
\end{ex}
\begin{ex}%[1K2Y5-2]
	Cho cấp số cộng có số hạng đầu $u_1=-\dfrac{1}{2}$, công sai $d=\dfrac{1}{2}$. Năm số hạng liên tiếp đầu tiên của cấp số này là.
	\choice
	{$-\dfrac{1}{2};0;1;\dfrac{1}{2};1$}
	{$-\dfrac{1}{2};0;\dfrac{1}{2};0;\dfrac{1}{2}$}
	{$\dfrac{1}{2};1;\dfrac{3}{2};2;\dfrac{5}{2}$}
	{\True $-\dfrac{1}{2};0;\dfrac{1}{2};1;\dfrac{3}{2}$}
	\loigiai{
		Ta dùng công thức tổng quát $u_n=u_1+(n-1)d=-\dfrac{1}{2}+(n-1)\dfrac{1}{2}=-1+\dfrac{n}{2}$ để tính các số hạng của một cấp số cộng. Ta có
		\begin{center}
			$u_1=-\dfrac{1}{2},u_2=0,u_3=\dfrac{1}{2},u_4=1,u_5=\dfrac{3}{2}$.
		\end{center}
	}
\end{ex}
\begin{ex}%[1K2B6-3]
	Viết ba số hạng xen giữa các số $2$ và $22$ để được một cấp số cộng có năm số hạng.
	\choice
	{\True 
		$7;12;17$}
	{$6;10;14$}
	{$8;13;18$}
	{$6;12;18$}
	\loigiai{
		Giữa $2$ và $22$ có thêm ba số hạng nữa lập thành cấp số cộng, xem như ta có một cấp số cộng có năm số hạng với $u_1=22;u_5=22$, ta cần tìm $u_2,u_3,u_4$. Ta có
		\begin{eqnarray*}
			&&u_5=u_1+4d\\
			&\Leftrightarrow&d=\dfrac{u_5-u_1}{4}\\&\Leftrightarrow&d=5\\
			&\Rightarrow&\heva{&u_2=7\\&u_3=12\\&u_4=17}.
		\end{eqnarray*}
	}
\end{ex}
\begin{ex}%[1K2K6-3]
	Biết các số $C_n^1;C_n^2;C_n^3$ theo thứ tự lập thành một cấp số cộng với $n>3$. Tìm $n$.
	\choice
	{$n=5$}
	{\True $n=7$}
	{$n=9$}
	{$n=11$}
	\loigiai{
		Ba số $C_n^1;C_n^2;C_n^3$ theo thứ tự $u_1;u_2;u_3$ lập thành một cấp số cộng nên
		\begin{eqnarray*}
			&&u_1+u_3=2u_2\\
			&\Leftrightarrow&C_n^1+C_n^3=2C_n^2\\
			&\Leftrightarrow&n+\dfrac{(n-2)(n-1)n}{6}=2.\dfrac{(n-1)n}{2}\\
			&\Leftrightarrow&1+\dfrac{n^2-3n+2}{6}=n-1\\
			&\Leftrightarrow&n^2-9n+14\\
			&\Leftrightarrow&\hoac{&n=2\\&n=7}	
		\end{eqnarray*}
		Kết hợp với điều kiện $n>3$, do đó $n=7$ thoả mãn yêu cầu bài toán.
	}
\end{ex}
\begin{ex}%[1K2B6-2]
	Cho cấp số cộng $\left(u_n\right)$ có các số hạng đầu lần lượt là $5; 9; 13; 17;...$. Tìm số hạng tổng quát $u_n$ của cấp số cộng.
	\choice
	{$u_n=5n+1$}
	{$u_n=5n-1$}
	{\True $u_n=4n+1$}
	{$u_n=4n-1$}
	\loigiai{
		Các số $5; 9; 13; 17 th;...$ theo thứ tự đó lập thành cấp số cộng $\left(u_n\right)$ nên
		\begin{center}
			$\heva{&u_1=5\\&d=u_2-u_1=4}\Rightarrow u_n=u_1+(n-1)d=5+4(n-1)=4n+1$.
		\end{center}
	}
\end{ex}
\begin{ex}%[1K2K6-1]
	Cho cấp số cộng $\left(u_n\right)$ có $u_1=3$ và $d=\dfrac{1}{2}$. Khẳng định nào sau đây đúng?
	\choice
	{$u_n=-3+\dfrac{1}{2}(n+1)$}
	{$u_n=-3+\dfrac{1}{2}n-1$}
	{\True $u_n=-3+\dfrac{1}{2}(n-1)$}
	{$u_n=-3+\dfrac{1}{4}(n-1)$}
	\loigiai{
		Ta có
		\begin{center}
			$\heva{&u_1=-3\\&d=\dfrac{1}{2}}\Rightarrow u_n=u_1+(n-1)d=-3+\dfrac{1}{2}(n-1)$.
		\end{center}
	}
\end{ex}
\begin{ex}%[1K2K6-1]
	Trong các dãy số được cho dưới đây, dãy số nào là cấp số cộng?
	\choice
	{\True $u_7=7-3n$}
	{$u_7=7-3^n$}
	{$u_7=\dfrac{7}{3n}$}
	{$u_7=7.3^n$}
	\loigiai{
		Dãy $\left(u_n\right)$ là cấp số cộng khi và chỉ khi $u_n=an+b$ với $a,b$ là hằng số.
	}
\end{ex}
\begin{ex}%[1K2B6-3]
	Cho cấp số cộng $\left(u_n\right)$ có $u_1=-5$ và $d=3$. Mệnh đề nào sau đây đúng?
	\choice
	{$u_{15}=34$}
	{$u_{15}=45$}
	{\True $u_{13}=31$}
	{$u_{10}=35$}
	\loigiai{
		Ta có
		\begin{center}
			$\heva{&u_1=-5\\&d=3}\Rightarrow u_n=3n-8\Rightarrow\heva{&u_{15}=37\\&u_{13}=31\\&u_{10}=22}$.
		\end{center}
	}
\end{ex}
\begin{ex}%[1K2B6-3]
	Cho cấp số cộng $\left(u_n\right)$ có $d=-2$ và $S_8=72$. Tìm số hạng đầu tiên $u_1$.
	\choice
	{\True $u_1=16$}
	{$u_1=-16$}
	{$u_1=\dfrac{1}{16}$}
	{$u_1=-\dfrac{1}{16}$}
	\loigiai{
		Ta có $\heva{&d=-2\\&S_8=72}\Leftrightarrow\heva{&d=-2\\&8u_1+\dfrac{8.7}{2}d=72}\Rightarrow 8u_1+28.(-2)=72\Leftrightarrow u_1=16$.
	}
\end{ex}
\begin{ex}%[1K2K6-3]
	Một cấp số cộng có số hạng đầu là $1$, công sai là $4$, tổng của n số hạng đầu là $561$. Khi đó số
	hạng thứ $n$ của cấp số cộng đó là $u_n$ có giá trị là bao nhiêu?
	\choice
	{$u_n=57$}
	{$u_n=61$}
	{\True $u_n=65$}
	{$u_n=69$}
	\loigiai{
		Ta có $\heva{&u_1=1,d=4\\&S_n=561}\Leftrightarrow\heva{&u_=1,d=4\\&nu_1+\dfrac{n(n-1)}{2}d=561}\Rightarrow n+\dfrac{n^2-n}{2}.4=561\Leftrightarrow 2n^2-n-561=0\Leftrightarrow n=17$.\\
		Từ đây suy ra $u_{17}=u_1+16d=1+16.4=65$.
	}
\end{ex}
\begin{ex}%[1K2K6-5]
	Tổng $n$ số hạng đầu tiên của một cấp số cộng là $S_n=\dfrac{3n^2-19n}{4}$ với $n\in\mathbb{N}^*$. Tìm số hạng đầu
	tiên $u_1$ và công sai $d$ của cấp số cộng đã cho.
	\choice
	{$u_1=2,d=-\dfrac{1}{2}$}
	{\True $u_1=-4,d=\dfrac{3}{2}$}
	{$u_1=-\dfrac{3}{2},d=-2$}
	{$u_1=\dfrac{5}{2},d=\dfrac{1}{2}$}
	\loigiai{
		Ta có $\dfrac{3n^2-19n}{4}=\dfrac{3}{4}n^2-\dfrac{19n}{4}=S_n=nu_1+\dfrac{n^2-n}{2}d=\dfrac{d}{2}n^2+\left(u_1-\dfrac{d}{2}\right)n$.\\
		Đồng nhất hai vế của phương trình, ta có $\heva{&\dfrac{d}{2}=\dfrac{3}{4}\\&u_1-\dfrac{d}{2}=-\dfrac{19}{4}}\Leftrightarrow\heva{&u_1=-4\\&d=\dfrac{3}{2}}$.
	}
\end{ex}
\begin{ex}%[1K2K6-3]
	Cho cấp số cộng $\left(u_n\right)$ có $u_2=2001$ và $u_5=1995$. Khi đó $u_{1001}$ bằng.
	\choice
	{$u_{1001}=4005$}
	{$u_{1001}=4003$}
	{\True $u_{1001}=3$}
	{$u_{1001}=1$}
	\loigiai{
		Ta có $\heva{&u_2=2001\\&u_5=1995}\Leftrightarrow\heva{&u_1+d=2001\\&u_1+4d=1995}\Leftrightarrow \heva{&u_1=2003\\&d=-2}\Rightarrow u_{1001}=u_1+1000d=3$.
	}
\end{ex}
\begin{ex}%[1K2B6-1]
	Cho cấp số cộng $\left(u_n\right)$ biết $u_n=-1,u_{n+1}=8$. Tính công sai $d$ của cấp số cộng đó.
	\choice
	{$d=-9$}
	{$d=7$}
	{$d=-7$}
	{\True $d=9$}
	\loigiai{
		Ta có $d=u_{n+1}-u_n=8-(-1)=9$.
	}
\end{ex}
\begin{ex}%[1K2K6-5]
	Cho cấp số cộng $\left(u_n\right)$ thỏa mãn $u_2+u_{23}=60$. Tính tổng $S_24$ của $24$ số hạng đầu tiên của
	cấp số cộng đã cho.
	\choice
	{$S_{24}=60$}
	{$S_{24}=120$}
	{\True $S_{24}=720$}
	{$S_{24}=1440$}
	\loigiai{
		Ta có $u_2+u_{23}=60\Leftrightarrow u_1+d+u_1+22d=60\Leftrightarrow 2u_1+23d=60$.\\
		Khi đó $S_{24}=\dfrac{24}{2}\left(u_1+u_{24}\right)=12\left(u_1+u_1+23d\right)=12.60=720$.
	}
\end{ex}
\begin{ex}%[1K2K6-1]
	Một cấp số cộng có $6$ số hạng. Biết rằng tổng của số hạng đầu và số hạng cuối bằng $17$, tổng
	của số hạng thứ hai và số hạng thứ tư bằng $14$. Tìm công sai $d$ của câp số cộng đã cho.
	\choice
	{$d=2$}
	{\True $d=-3$}
	{$d=4$}
	{$d=5$}
	\loigiai{
		Ta có $\heva{&u_1+u_6=17\\&u_2+u_4=14}\Leftrightarrow\heva{&2u_1+5d=17\\&2u_1+6d=14}\Leftrightarrow\heva{&u_1=16\\&d=-3}$.
	}
\end{ex}
\begin{ex}%[1K2K6-1]
	Cho cấp số cộng $\left(u_n\right)$ thỏa mãn $\heva{&u_7-u_3=8\\&u_2u_7=75}$. Tìm công sai $d$ của cấp số cộng đã cho.
	\choice
	{$d=\dfrac{1}{2}$}
	{$d=\dfrac{1}{3}$}
	{\True $d=2$}
	{$d=3$}
	\loigiai{
		Ta có $\heva{&u_7-u_3=8\\&u_2u_7=75}\Leftrightarrow\heva{&u_1+6d-u_1-2d=8\\&(u_1+d)(u_1+6d)=75}\Leftrightarrow\heva{&d=2\\&(u_1+2)(u_1+12)=75}$.
	}
\end{ex}
\begin{ex}%[1K2K6-3]
	Ba góc của một tam giác vuông tạo thành cấp số cộng. Hai góc nhọn của tam giác có số đo
	(độ) là
	\choice
	{$20^\circ$ và $70^\circ$}
	{$45^\circ$ và $45^\circ$}
	{$20^\circ$ và $45^\circ$}
	{\True $30^\circ$ và $60^\circ$}
	\loigiai{
		Ba góc $A,B,C$ của một tam giác vuông theo thứ tự đó $(A<B<C)$ lập thánh cấp số cộng nên $C=90,C+A=2B$.\\
		Ta có $\heva{&A+B+C=180\\&A+C=2B\\&C=90}\Leftrightarrow\heva{&A=30\\&B=60\\&C+90}$.
	}
\end{ex}
\begin{ex}%[1K2K6-3]
	Một tam giác vuông có chu vi bằng $3$ và độ dài các cạnh lập thành một cấp số cộng. Độ dài các
	cạnh của tam giác đó là
	\choice
	{$\dfrac{1}{2};1;\dfrac{3}{2}$}
	{$\dfrac{1}{3};1;\dfrac{5}{3}$}
	{\True $\dfrac{3}{4};1;\dfrac{5}{4}$}
	{$\dfrac{1}{4};1;\dfrac{7}{4}$}
	\loigiai{
		Ba cạnh $a,b,c,(a<b<c)$ của một tam giác theo thứ tự đó lập thành một cấp số cộng.\\
		Ta có $\heva{&a^2+b^2=c^2\\&a+b+c=3\\&a+c=2b}\Leftrightarrow\heva{&a^2+b^2=c^2\\&3b=3\\&a+c=2b}\Leftrightarrow\heva{&a^2+b^2=c^2\\&b=1\\&a=2-c}$.\\
		Từ đây suy ra $a^2+b^2=c^2\Rightarrow (2-c)^2+1=c^2\Leftrightarrow c=\dfrac{5}{4}\Leftrightarrow\heva{&a=\dfrac{3}{4}\\&b=1\\&c=\dfrac{5}{4}}$.
	}
\end{ex}
\begin{ex}%[1K2K6-6]
	Một rạp hát có $30$ dãy ghế, dãy đầu tiên có $25$ ghế. Mỗi dãy sau có hơn dãy trước $3$ ghế. Hỏi rạp
	hát có tất cả bao nhiêu ghế?
	\choice
	{$1635$}
	{$1792$}
	{\True $2055$}
	{$3125$}
	\loigiai{
		Số ghế của mỗi dãy (bắt đầu từ dãy đầu tiên) theo thứ tự đó lập thành một cấp số cộng có $30$ số hạng có công sai $d=3$ và $u_1=25$.\\
		Tổng số ghế là $S_{30}=30u_1+\dfrac{30.29}{2}d=2055$.
	}
\end{ex}
\begin{ex}%[1K2K6-6]
	Người ta trồng $3003$ cây theo một hình tam giác như sau: hàng thứ nhất trồng $1$ cây, hàng thứ hai trồng $2$ cây, hàng thứ ba trồng $3$ cây,... .Hỏi có tất cả bao nhiêu hàng cây?
	\choice
	{$73$}
	{$75$}
	{\True $77$}
	{$79$}
	\loigiai{
		Số cây mỗi hàng (bắt đầu từ hàng thứ nhất) lập thành một cấp số cộng $(u_n)$ có $u_1=1,d=1$. Giả sử có $n$ hàng cây thì $u_1+u_2+...+u_n=S_n$.\\
		Ta có $S_n=1.n+\dfrac{n(n-1)}{2}.1=3003\Leftrightarrow n=77$.
	}
\end{ex}
\begin{ex}%[1K2G6-6]
	Một chiếc đồng hồ đánh chuông, kể từ thời điểm $0$ (giờ) thì sau mỗi giờ thì số tiếng chuông được đánh đúng bằng số giờ mà đồng hồ chỉ tại thời điểm đánh chuông. Hỏi một ngày đồng hồ đó đánh bao nhiêu tiếng chuông?
	\choice
	{$78$}
	{$156$}
	{\True $300$}
	{$48$}
	\loigiai{
		Kể từ lúc $1$ (giờ) đến $24$ (giờ) số tiếng chuông được đánh lập thành cấp số cộng có $24$ số hạng với $u_1=1$, công sai $d=1$. Vậy số tiếng chuông được đánh trong $1$ ngày là $S_{24}=1.24+\dfrac{24.23}{2}.1=300$.
	}
\end{ex}
\begin{ex}%[1K2G6-6]
	Trên một bàn cờ có nhiều ô vuông, người ta đặt $7$ hạt dẻ vào ô đầu tiên, sau đó đặt tiếp vào ô thứ
	hai số hạt nhiều hơn ô thứ nhất là $5$, tiếp tục đặt vào ô thứ ba số hạt nhiều hơn ô thứ hai là $5$,...
	và cứ thế tiếp tục đến ô thứ $n$. Biết rằng đặt hết số ô trên bàn cờ người ta phải sử dụng $25450$
	hạt. Hỏi bàn cờ đó có bao nhiêu ô vuông?
	\choice
	{$98$}
	{\True $100$}
	{$102$}
	{$104$}
	\loigiai{
		Số hạt dẻ trên mỗi ô (bắt đầu từ ô thứ nhất) theo thứ tự đó lập thành cấp số cộng $(u_n)$ có $u_1=7,d=5$. Gọi $n$ là số ô trên bàn cờ thì $u_1+u_2+...+u_n=S_n$.\\
		Ta có $S_n=25450\Leftrightarrow 7n+\dfrac{n(n-1)}{2}.7=25450\Leftrightarrow n=100$.
	}
\end{ex}
\begin{ex}%[1K2G6-6]
	Một gia đình cần khoan một cái giếng để lấy nước. Họ thuê một đội khoan giếng nước đến để khoan giếng nước. Biết giá của mét khoan đầu tiên là $80.000$ đồng, kể từ mét khoan thứ $2$ giá của mỗi mét khoan tăng thêm $5000$ đồng so với giá của mét khoan trước đó. Biết cần phải khoan sâu xuống $50$ mét mới có nước. Vậy hỏi phải trả bao nhiêu tiền để khoan cái giếng đó?
	\choice
	{$5.250.000$ đồng}
	{\True $10.125.000$ đồng}
	{$4.00.000$ đồng}
	{$4.245.000$ đồng}
	\loigiai{
		Giá tiền khoang mỗi mét (bắt đầu từ mét đầu tiên) lập thành cấp số cộng $(u_n)$ có $u_1=80000,d=5000$. Do cần khoang $50$ mét nên tổng số tiền cần trả là $S_{50}=80000.50+\dfrac{50.49}{2}.5000=10125000$.
	}
\end{ex}
\begin{ex}%[1K2Y7-3]
	Một cấp số nhân có hai số hạng liên tiếp là $16$ và $36$. Số hạng tiếp theo là
	\choice
	{$720$}
	{\True$81$}
	{$64$}
	{$56$}
	\loigiai{
		Ta có cấp số nhân $(u_n)$ có $\heva{&u_n=36\\&u_{n+1}=36}\Rightarrow q=\dfrac{u_{n+1}}{u_n}=\dfrac{9}{4}$. Từ đây suy ra $u_{n+2}=u_{n+1}.q=36.\dfrac{9}{4}=81$.
	}
\end{ex}
\begin{ex}%[1K2B7-3]
	Tìm x để các số $2;8;x;128$ theo thứ tự đó lập thành một cấp số nhân.
	\choice
	{$x=14$}
	{\True 
		$x=32$}
	{$x=64$}
	{$x=68$}
	\loigiai{
		Cấp số nhân$ 2;8;x;128$ theo thứ tự đó sẽ là $u_1,u_2,u_3,u_4$.\\
		Ta có $\heva{&\dfrac{u_2}{u_1}=\dfrac{u_3}{u_2}\\&\dfrac{u_3}{u_2}=\dfrac{u_4}{u_3}}\Leftrightarrow\heva{&\dfrac{8}{2}=\dfrac{x}{8}\\&\dfrac{128}{x}=\dfrac{x}{8}}\Leftrightarrow\heva{&x=32\\&x^2=1024}\Rightarrow x=32$.
	}
\end{ex}
\begin{ex}%[1K2K7-3]
	Tìm tất cả giá trị của $x$ để ba số $2x-11;x;2x+1$ theo thứ tự đó lập thành một cấp số nhân.
	\choice
	{\True $x=\pm\dfrac{1}{\sqrt{3}}$}
	{$x=\pm\dfrac{1}{3}$}
	{$x=\pm\sqrt{3}$}
	{$x=\pm3$}
	\loigiai{
		Cấp số nhân $2x-1;x;2x+1$, suy ra $(2x-1)(2x+1)=x^2\Leftrightarrow x=\pm \dfrac{1}{\sqrt{3}}$.
	}
\end{ex}
\begin{ex}%[1K2K7-3]
	Với giá trị $x,y$ nào dưới đây thì các số hạng lần lượt là $-2;x;-18;y$ theo thứ tự đó lập thành cấp số nhân?
	\choice
	{$\heva{&x=6\\&y=-54}$}
	{$\heva{&x=-10\\&y=-26}$}
	{\True $\heva{&x=-6\\&y=-54}$}
	{$\heva{&x=-6\\&y=54}$}
	\loigiai{
		Cấp số nhân $-2;x;-18;y$, suy ra $\heva{&\dfrac{x}{-2}=\dfrac{-18}{x}\\&\dfrac{-18}{x}=\dfrac{y}{-18}}\Leftrightarrow\heva{&x=\pm6\\& y=\pm 54}$. Vậy $(x,y)=(6;54)$ hoặc $(x;y)=(-6;-54)$.
	}
\end{ex}
\begin{ex}%[1K2K7-3]
	Hai số hạng đầu của của một cấp số nhân là $2x+1$ và $4x^2-1$. Số hạng thứ ba của cấp số nhân là.
	\choice
	{$2x-1$}
	{$2x+1$}
	{\True $8x^3-4x^2-2x+1$}
	{$8x^3+4x^2-2x-1$}
	\loigiai{
		Công bội của cấp số nhân là $q=\dfrac{4x^2-1}{2x+1}=2x-1$. Vậy số hạng thứ ba của cấp số nhân là $(4x^2-1)(2x-1)=8x^3-4x^2-2x+1$.
	}
\end{ex}
\begin{ex}%[1K2B7-1]
	Trong các dãy số $(u_n)$ cho bởi số hạng tổng quát nu sau, dãy số nào là một cấp số nhân
	\choice
	{\True 
		$u_n=\dfrac{1}{3^{n-2}}$}
	{$u_n=\dfrac{1}{3^{n}}-1$}
	{$u_n=n+\dfrac{1}{3}$}
	{$u_n=n^2-\dfrac{1}{3}$}
	\loigiai{
		Dãy $u_n=\dfrac{1}{3^{n-2}}=3\left(\dfrac{1}{3}\right)^{n-1}$ là cấp số nhân có $u_1=3,q=\dfrac{1}{3}$.
	}
\end{ex}
\begin{ex}%[1K2B7-1]
	Một cấp số nhân có $6$ số hạng, số hạng đầu bằng $2$ và số hạng thứ sáu bằng $486$. Tìm công bội $q$ của cấp số nhân đã cho.
	\choice
	{\True $q=3$}
	{$q=-3$}
	{$q=2$}
	{$q=-2$}
	\loigiai{
		Ta có $\heva{&u_1=2\\&u_6=486}\Rightarrow u_6=u_1q^5\Leftrightarrow 486=2.q^5\Leftrightarrow q=3$.
	}
\end{ex}
\begin{ex}%[1K2B7-1]
	Cho cấp số nhân $\left(u_n\right)$ có $u_1=-3$ và $q=\dfrac{2}{3}$ Mệnh đề nào sau đây đúng.
	\choice
	{$u_5=-\dfrac{27}{16}$}
	{\True $u_5=-\dfrac{16}{27}$}
	{$u_5=\dfrac{16}{27}$}
	{$u_5=\dfrac{27}{16}$}
	\loigiai{
		Ta có $\heva{&u_1=-3\\&q=\dfrac{2}{3}}\Rightarrow u_5=u_1.q^4=-3.\left(\dfrac{2}{3}\right)^4=-\dfrac{16}{27}$.
	}
\end{ex}
\begin{ex}%[1K2K7-3]
	Cho cấp số nhân $\left(u_n\right)$ có $u_1=3$ và $q=-2$. Số $192$ là số hạng thứ mấy của cấp số nhân đã cho.
	\choice
	{$5$}
	{$6$}
	{\True $7$}
	{Không là số hạng của cấp số đã cho}
	\loigiai{
		Ta có $u_n=u_1.q^{n-1}\Leftrightarrow 192=3.(-2)^{n-1}\Leftrightarrow n=7$.
	}
\end{ex}
\begin{ex}%[1K2K7-3]
	Một cấp số nhân có công bội bằng $3$ và số hạng đầu bằng $5$. Biết số hạng chính giữa là $32805$. Hỏi cấp số nhân đã cho có bao nhiêu số hạng?
	\choice
	{$18$}
	{\True $17$}
	{$16$}
	{$9$}
	\loigiai{
		Ta có $u_n=u_1.q^{n-1}\Leftrightarrow 32805=3.5^{n-1}\Leftrightarrow n=9$. Vậy $u_9$ là số hạng chính giữa của cấp số nhân, nên cấp số nhân đã cho có $17$ số hạng.
	}
\end{ex}
\begin{ex}%[1K2K7-5]
	Cho cấp số nhân $\left(u_n\right)$ có $u_1=-3$ và $q=-2$. Tính tổng $10$ số hạng đầu tiên của cấp số nhân đã cho.
	\choice
	{$S_{10}=-511$}
	{$S_{10}=-1025$}
	{$S_{10}=1025$}
	{\True $S_{10}=1023$}
	\loigiai{
		Ta có $\heva{&u_1=-3\\&q=-2}\Rightarrow S_{10}=u_1.\dfrac{q^{n}-1}{q-1}=(-3).\dfrac{(-2)^{10}-1}{-2-1}=1023$.
	}
\end{ex}
\begin{ex}%[1K2G7-5]
	Cho cấp số nhân có các số hạng lần lượt là $1;4;16;64;...$. Gọi $S_n$ là tổng của $n$ số hạng đầu tiên của cấp số nhân đó. Mệnh đề nào sau đây đúng?
	\choice
	{$S_n=4^{n-1}$}
	{$S_n=\dfrac{n\left(1+4^{n-1}\right)}{2}$}
	{\True $S_n=\dfrac{4^n-1}{3}$}
	{$S_n=\dfrac{4\left(4^n-1\right)}{3}$}
	\loigiai{
		Ta có $\heva{&u_1=-3\\&q=4}\Rightarrow S_n=u_1.\dfrac{q^{n}-1}{q-1}=\dfrac{4^n-1}{3}$.
	}
\end{ex}
\begin{ex}%[1K2G7-1]
	Số hạng thứ hai, số hạng đầu và số hạng thứ ba của một cấp số cộng với công sai khác $0$ theo thứ tự đó lập thành một cấp số nhân với công bội $q$. Tìm $q$.
	\choice
	{$q=2$}
	{\True $q=-2$}
	{$q=-\dfrac{3}{2}$}
	{$q=\dfrac{3}{2}$}
	\loigiai{
		Giả sử ba số hạng $a;b;c$ lập thành cấp số cộng thỏa yêu cầu, khi đó $b;a;c$ theo thứ tự đó lập thành cấp số nhân công bội $q$. Ta có $\heva{&a+c=2b\\&a=bq\\&c=bq^2}\Rightarrow bq+bq^2=2b\Leftrightarrow\heva{&b=0\\&q^2+q-2=0}$.\\
		Nếu $b=0\Rightarrow a=b=c=0$ nên $a;b;c$ là cấp số cộng công sai $d=0$ (vô lí).\\
		Nếu $q^2+q-2=0\Leftrightarrow\hoac{&q=1\\&q=-2}$. Nếu $q=1\Rightarrow a=b=c$ (vô lí), do đó $q=-2$.
	}
\end{ex}
\begin{ex}%[1K2G7-1]
	Cho bố số $a,b,c,d$ biết rằng $a,b,c$ theo thứ tự đó lập thành một cấp số nhân công bội $q>1$,
	còn $b,c,d$ theo thứ tự đó lập thành cấp số cộng. Tìm $q$ biết rằng $a+d=14$ và $b+c=12$.
	\choice
	{$q=\dfrac{18+\sqrt{73}}{24}$}
	{\True $q=\dfrac{19+\sqrt{73}}{24}$}
	{$q=\dfrac{20+\sqrt{73}}{24}$}
	{$q=\dfrac{21+\sqrt{73}}{24}$}
	\loigiai{
		Giả sử $a,b,c$ lập thành cấp số cộng công bội $q$. Khi đó theo giả thiết ta có\\
		$\heva{&b=aq\\&c=aq^2\\&b+d=2c\\&a+d=14\\&c+d=12}\Rightarrow\heva{&aq+d=aq^2,&(1)\\&a+d=14,&(2)\\&a\left(q+q^2\right)=12,&(3)}$.\\
		Nếu $q=0\Rightarrow b=c=d=0$ (vô lý).\\
		Nếu $q=-1\Rightarrow b=-a=-c\Rightarrow b+c=0$ (vô lý).\\
		Vậy $q\ne 0,q\ne -1$, từ $(2)$ và $(3)$, ta có $d=14-a$ và $a=\dfrac{12}{q+q^2}$, thay vào $(1)$, ta được\\
		$\dfrac{12q}{q+q^2}+\dfrac{14q^2+14q-12}{q+q^2}=\dfrac{24q^3}{q+q^2}\Leftrightarrow 12q^3-7q^2-13q+6=0\Leftrightarrow q=\dfrac{19\pm \sqrt{73}}{24}$.\\
		Mà $q>1$ nên $q=\dfrac{19+\sqrt{73}}{24}$.
	}
\end{ex}
\begin{ex}%[1K2G7-5]
	Gọi $S=1+11+111+\cdots+111\ldots1$ ($n$ số $1$) thì $S$ nhận giá trị nào sau đây?
	\choice
	{$S=\dfrac{10^n-1}{81}$}
	{$S=10\cdot\dfrac{10^n-1}{81}$}
	{$S=10\cdot\dfrac{10^n-1}{81}-1$}
	{\True $S=\dfrac{1}{9}\left[10\cdot\dfrac{10^n-1}{9}-1\right]$}
	\loigiai{
		Ta có $S=\dfrac{1}{9}\left(9+99+999+\cdots+999\ldots9\right)=\dfrac{1}{9}\left(10+100+1000+\cdots+100\ldots0-n\right)=\dfrac{1}{9}\left[10\cdot\dfrac{10^n-1}{9}-1\right]$.
	}
\end{ex}
\begin{ex}%[1K2G7-7]
	Biết rằng $S=1+2\cdot3+3\cdot3^2+\cdots+11.3^{10}=a+\dfrac{21\cdot3^b}{4}$. Tính $P=a+\dfrac{b}{4}$.
	\choice
	{$P=1$}
	{$P=2$}
	{\True $P=3$}
	{$P=4$}
	\loigiai{
		Từ giả thiết suy ra $3S=3+2\cdot3^2+3\cdot3^3+\cdots+11\cdot3^{11}$.\\
		Do đó $-2S=S-3S=1+3+3^2+3^3+\cdots+3^{10}-10.3^{11}=\dfrac{1-3^{11}}{1-3}-11\cdot3^{11}\Rightarrow S=\dfrac{1}{4}+\dfrac{21}{4}\cdot3^{11}$.\\
		Vậy $a=\dfrac{1}{4},b=11$, suy ra $P=3$.
	}
\end{ex}
\begin{ex}%[1K2K7-1]
	Một cấp số nhân có ba số hạng là $a,b,c$ (theo thứ tự đó) trong đó các số hạng đều khác $0$ và công bội $q\ne 0$. Mệnh đề nào sau đây là đúng.
	\choice
	{$\dfrac{1}{a^2}=\dfrac{1}{bc}$}
	{\True $\dfrac{1}{b^2}=\dfrac{1}{ac}$}
	{$\dfrac{1}{c^2}=\dfrac{1}{ba}$}
	{$\dfrac{1}{a}+\dfrac{1}{b}=\dfrac{2}{c}$}
	\loigiai{
		Ta có $ac=b^2\Rightarrow \dfrac{1}{b^2}=\dfrac{1}{ac}$
	}
\end{ex}
\begin{ex}%[1K2K7-3]
	Bốn góc của một tứ giác tạo thành cấp số nhân và góc lớn nhất gấp $27$ lần góc nhỏ nhất. Tổng của góc lớn nhất và góc bé nhất bằng.
	\choice
	{$56^\circ$}
	{$102^\circ$}
	{\True $252^\circ$}
	{$168^\circ$}
	\loigiai{
		Giả sử $4$ góc $A, B, C, D$ (với $A<B<C<D$) theo thứ tự đó lập thành cấp số nhân thỏa yêu cầu với công bội $q$.\\
		Ta có $\heva{&A+B+C+D=360\\&D=27A}\Leftrightarrow\heva{&A\left(1+q+q^2+q^3\right)=360\\&Aq^3=27A}\Leftrightarrow\heva{&q=3\\&A=9\\&D=243}\Rightarrow A+D=252$.
	}
\end{ex}
\begin{ex}%[1K2G7-7]
	Người ta thiết kế một cái tháp gồm $11$ tầng. Diện tích bề mặt trên của mỗi tầng bằng nữa diện tích của mặt trên của tầng ngay bên dưới và diện tích mặt trên của tầng $1$ bằng nửa diện tích của đế tháp (có diện tích là $12288m^2$). Tính diện tích mặt trên cùng.
	\choice
	{\True $6m^2$}
	{$8m^2$}
	{$10m^2$}
	{$12m^2$}
	\loigiai{
		Diện tích bề mặt của mỗi tầng (kể từ $1$) lập thành một cấp số nhân có công bội $q=\dfrac{1}{2}$ và
		$u_1=\dfrac{12288}{2}=6144$. Khi đó diện tích mặt trên cùng là $u_{11}=u_1\cdot q^{10}=6144\cdot\left(\dfrac{1}{2}\right)^{10}=6$.
	}
\end{ex}
\begin{ex}%[1K2G7-7]
	Một du khách vào chuồng đua ngựa đặt cược, lần đầu đặt $20000$ đồng, mỗi lần sau tiền đặt gấp
	đôi lần tiền đặt cọc trước. Người đó thua $9$ lần liên tiếp và thắng ở lần thứ $10$. Hỏi du khác trên thắng hay thua bao nhiêu?
	\choice
	{Hoà vốn}
	{Thua $20000$ đồng}
	{\True Thắng $20000$ đồng}
	{Thua $40000$ đồng}
	\loigiai{
		Số tiền du khác đặt trong mỗi lần (kể từ lần đầu) là một cấp số nhân có $u_1=20000$ và công bội $q=2$. Du khách thua trong $9$ lần đầu tiên nên tổng số tiền thua là $S_9=u_1.\dfrac{q^9-1}{q-1}=20000\cdot\dfrac{2^9-1}{2-1}=10220000$.\\
		Số tiền mà du khách thắng trong lần thứ $10$ là $u_{10}=u_1\cdot q^9=20000\cdot2^9=10240000$.\\
		Ta có $u_{10}-S_9=20000>0$ nên du khách thắng $20000$.
	}
\end{ex}
\Closesolutionfile{ans}
% \begin{indapan}{10}
% 	{ans/ans-1K2-Ontapchuong2}
% \end{indapan}

%Chương III
% \setcounter{chapter}{2}
\setcounter{subsubsection}{0}
\setcounter{ex}{0}
\setcounter{bt}{0}
\chap{Một số yếu tố thống kê và xác suất}
% \section{Các số đặc trưng đo xu thế trung tâm cho mẫu số liệu ghép nhóm}

\subsection{Tóm tắt lý thuyết}
\begin{tomtat}
	\subsubsection{Mẫu số liệu ghép nhóm}
		\begin{enumerate}
			\item \textbf{\textit{Mẫu số liệu ghép nhóm}} là mẫu số liệu cho dưới dạng bảng tần số ghép nhóm.
			\item Mỗi số liệu gồm một số giá trị của mẫu số liệu được ghép nhóm theo một tiêu chí xác định có dạng $\left[a;b\right)$, trong đó $a$ là \textit{đầu mút trái}, $b$ là \textit{đầu mút phải}. Độ dài nhóm là $b-a$.
			\item \textbf{\textit{Tần số tích luỹ}} của một nhóm là số số liệu trong mẫu số liệu có giá trị nhỏ hơn giá trị đầu mút phải của nhóm đó. Tần số tích luỹ của nhóm $1$, nhóm $2$, $\ldots$, nhóm $m$ kí hiệu lần lượt là $cf_1$, $cf_2$, $\ldots$, $cf_m$.
		\end{enumerate}
	\subsubsection{Số trung bình cộng (số trung bình)}
		\begin{enumerate}
			\item Trung điểm $x_i$ của nửa khoảng (tính bằng trung bình cộng của hai đầu mút) ứng với nhóm $i$ là \textit{giá trị đại diện} của nhóm đó.
			\item \textit{Số trung bình cộng} của mẫu số liệu ghép nhóm, kí hiệu $\overline{x}$, được tính theo công thức 
				$$\overline{x} = \dfrac{n_1x_1 + n_2x_2 + \ldots + n_mx_m}{n}.$$
		\end{enumerate}
	\subsubsection{Trung vị}
	Để tính trung vị của mẫu số liệu ghép nhóm, ta làm như sau:
\begin{itemize}
    \item \textbf{Bước 1:} Xác định nhóm chứa trung vị. Giả sử đó là nhóm thứ $p : [a_p; a_{p + 1})$.
    \item \textbf{Bước 2:} Trung vị là 
    $$M_e = a_p + \dfrac{\dfrac{n}{2} - (m_1 + \cdots + m_{p-1})}{m_p} \cdot ( a_{p + 1} - a_p)$$
    trong đó $n$ là cỡ mẫu, $m_p$ là tần số nhóm $p$. Với $p=1$ ta quy ước $m_1 + \cdots + m_{p-1} = 0$.
\end{itemize}
		\begin{note}
			Nhóm chứa trung vị là nhóm đầu tiên có tần số tích luỹ $cf_p=m_1 + \cdots + m_{p}$ lớn hơn hoặc bằng $\dfrac{n}{2}$
		\end{note}
	\subsubsection{Tứ phân vị}Để tính tứ phân vị thứ nhất $Q_1$ của mẫu số liệu ghép nhóm, trước hết ta xác định nhóm chứa $Q_1$, giả sử đó là nhóm thứ  $p:\left[a_p;a_{p+1} \right)$. Khi đó 
	$$Q_1=a_p+\dfrac{\dfrac{n}{4}-\left(m_1+\cdots+m_{p-1}\right)}{m_p}\cdot \left(a_{p+1}-a_p\right).$$
	trong đó, $n$ là cỡ mẫu, $m_p$ là tần số nhóm $p$. Với $p=1$, ta quy ước $m_1+\cdots+m_{p-1}=0$.\\
	Để tính tứ phân vị thứ ba $Q_3$ của mẫu số liệu ghép nhóm, trước hết ta xác định nhóm chứa $Q_3$, giả sử đó là nhóm thứ  $p:\left[a_p;a_{p+1} \right)$. Khi đó 
	$$Q_3=a_p+\dfrac{\dfrac{3n}{4}-\left(m_1+\cdots+m_{p-1}\right)}{m_p}\cdot \left(a_{p+1}-a_p\right).$$
	Trong đó $n$ là cỡ mẫu, $m_p$ là tần số nhóm $p$. Với $p=1$, ta quy ước $m_1+\cdots+m_{p-1}=0$.\\
	Tứ phân vị thứ hai $Q_2$ chính là trung vị $M_e$.\\
\subsubsection{Mốt của mẫu số liệu ghép nhóm}
Để tìm mốt của mẫu số liệu ghép nhóm, ta thực hiện theo các bước sau:
\begin{enumerate}
	\item [Bước 1.] Xác định nhóm có tần số lớn nhất (gọi là nhóm chứa mốt), giả sử là nhóm $j:\left[a_j;a_{j+1} \right)$.
	\item [Bước 2.] Mốt được xác định là $M_o=a_j+\dfrac{m_i-m_{j-1}}{\left(m_i-m_{j-1}\right)+\left(m_i-m_{j+1}\right)}\cdot h$.\\
\end{enumerate}
trong đó, $m_j$ là tần số nhóm $j$ (quy ước $m_0=m_{k+1}=0$) và $h$ là độ dài của nhóm.	

\end{tomtat}
%=================================================
\setcounter{subsubsection}{0}
\setcounter{ex}{0}
\setcounter{bt}{0}
\subsection{Các dạng toán thường gặp}
\begin{dang}{Mẫu số liệu ghép nhóm}
\end{dang}
\subsubsection{Ví dụ minh hoạ}
\begin{vd}%[Cánh Diều]%[1C5Y1-1]
	\immini{
		\textbf{Bảng bên} biểu diễn mẫu số liệu ghép nhóm được cho dưới dạng bảng tần số ghép nhóm. Hãy cho biết 
		\begin{enumerate}
			\item Mẫu số liệu có bao nhiêu số liệu; bao nhiêu nhóm?
			\item Tần số của mỗi nhóm.
		\end{enumerate}
	}{
		\begin{tabular}{|c|c|}
			\hline
			\textbf{Nhóm} & \textbf{Tần số}\\ 
			\hline
			$\left[0;5\right)$ & $11$\\
			\hline
			$\left[5;10\right)$ & $31$\\
			\hline
			$\left[10;15\right)$ & $45$\\
			\hline
			$\left[15;20\right)$ & $21$\\
			\hline
			$\left[20;26\right)$ & $12$\\
			\hline
			& $n = 120$ \\
			\hline
		\end{tabular}
	}
	\loigiai{
		\begin{enumerate}
			\item Mẫu số liệu gồm $120$ số liệu và $5$ nhóm.
			\item Tần số lần lượt của các nhóm $1$, $2$, $3$, $4$, $5$ lần lượt là $11$, $31$, $45$, $21$, $12$.
		\end{enumerate}	
	}
\end{vd}
\begin{vd}%[CTST]%[1T5B1-1]
	Một cửa hàng đã thống kê số ba lô bán được mỗi ngày trong tháng 9 với kết quả cho như sau: \begin{center}
		\begin{tabular}{lllllllllllllll}
			$12$ & $29$ & $12$ & $19$ & $15$ & $21$ & $19$ & $29$ & $28$ & $12$ & $15$ & $25$ & $16$ & $20$ & $29$\\
			$21$ & $12$ & $24$ & $14$ & $10$ & $12$ & $10$ & $23$ & $27$ & $28$ & $18$ & $16$ & $10$ & $20$ & $21$
		\end{tabular}
	\end{center}
	Hãy chia mẫu số liệu trên thành 5 nhóm, lập bảng tần số ghép nhóm, hiệu chỉnh bảng tần số ghép nhóm và xác định giá trị đại diện cho mỗi nhóm.
	\loigiai{
		Khoảng biến thiên của mẫu số liệu trên là $R=29-10=19$.\\
		Độ dài mỗi nhóm $L>\dfrac{R}{k}=\dfrac{19}{5}=3{,}8$.\\
		Ta chọn $L=4$ và chia dữ liệu thành các nhóm $[10; 14)$, $[14; 18)$, $[18; 22)$, $[22; 26)$, $[26; 30)$.\\
		Khi đó ta có bảng tần số ghép nhóm sau
		\begin{center}
			\begin{tabular}{|c|c|c|c|c|c|}
				\hline \textbf{Cân nặng} &{$[10; 14)$} &{$[14; 18)$} &{$[18; 22)$} &{$[22; 26)$} &{$[26; 30)$} \\
				\hline \textbf{Giá trị đại diện} & $12$ & $16$ & $20$ & $24$ & $28$ \\
				\hline \textbf{Số ba lô bán được} & $8$ & $5$ & $8$ & $3$ & $6$ \\
				\hline
			\end{tabular}
		\end{center}
	}
\end{vd}
\begin{vd}%[KNTT]%[Ngọc Hiếu]%[1K3B8-1]
	Bảng thống kê sau cho biết thời gian chạy (phút) của $30$ vận động viên (VĐV) trong một giải chạy Marathon.
	\begin{center}
		\begin{tabular}{|c|c|c|c|c|c|c|c|c|c|c|c|c|}
			\hline
			Thời gian&$129$&$130$&$133$&$134$&$135$&$136$&$138$&$141$&$142$&$143$&$144$&$145$\\
			\hline
			Số VĐV&$1$&$2$&$1$&$1$&$1$&$2$&$3$&$3$&$4$&$5$&$2$&$5$\\
			\hline
		\end{tabular}
	\end{center}
	Hãy chuyển mẫu số liệu trên sang mẫu số liệu ghép nhóm gồm sáu nhóm có độ dài bằng nhau và bằng $3$.
	\loigiai{
		Giá trị nhỏ nhất là $129$, giá trị lớn nhất là $145$ nên khoảng biến thiên là $145-129=16$. Tổng độ dài của sáu nhóm là $18$. Để cho đối xứng, ta chọn đầu mút trái của nhóm đầu tiên là $127{,}5$ và đầu mút phải của nhóm cuối cùng là $145{,}5$ ta được các nhóm là $[127{,}5;130{,}5),\; [130{,5};133{,5}],\ldots , [142{,}5;145{,}5]$. Đếm số giá trị thuộc mỗi nhóm, ta có mẫu số liệu ghép nhóm như sau
		\begin{center}
			\fontsize{9}{1pt}
			{\begin{tabular}{|c|c|c|c|c|c|c|}
					\hline
					Thời gian&$[125{,}5;130{,}5)$&$[130{,}5;133{,}5)$&$[133{,}5;136{,}5)$&$[136{,}5;139{,}5)$&$[139{,}5;142{,}5)$&$[142{,}5;145{,}5)$\\
					\hline
					Số VĐV&$3$&$1$&$4$&$3$&$7$&$12$\\
					\hline
			\end{tabular}}
		\end{center}
	}
\end{vd}
\begin{vd}%[Cánh Diều]%[1C5B1-1]
	Một trường trung học phổ thông chọn $36$ học sinh nam của khối $11$, do chiều cao của các bạn học sinh đó và thu được mẫu số liệu sau (đơn vị: centimét):
	$$
	\begin{array}{llllllllllll}
		160 & 161 & 161 & 162 & 162 & 162 & 163 & 163 & 163 & 164 & 164 & 164 \\
		164 & 165 & 165 & 165 & 165 & 165 & 166 & 166 & 166 & 166 & 167 & 167 \\
		168 & 168 & 168 & 168 & 169 & 169 & 170 & 171 & 171 & 172 & 172 & 174
	\end{array}
	$$
	Lập bảng tần số ghép nhóm bao gồm cả tần số tích luỹ cho mẫu số liệu trên có $5$ nhóm ứng với $5$ nửa khoảng:
	$$
	\left[160;163 \right),\ \left[163;169 \right),\ \left[166;169 \right),\ \left[169;172 \right),\ \left[172;175 \right).
	$$
	\loigiai{
		Bảng tần số ghép nhóm bao gồm cả tần số tích luỹ như sau:
		\begin{center}
			\begin{tabular}{|c|c|c|}
				\hline
				\textbf{Nhóm} & \textbf{Tần số} & \textbf{Tần số tích luỹ}\\ 
				\hline
				$\left[169;163\right)$ & $6$ & $6$\\
				\hline
				$\left[163;166\right)$ & $12$ & $18$\\
				\hline
				$\left[166;169\right)$ & $10$ & $28$\\
				\hline
				$\left[169;172\right)$ & $5$ & $33$\\
				\hline
				$\left[172;175\right)$ & $3$ & $36$\\
				\hline
				& $n = 36$ &\\
				\hline
			\end{tabular}
		\end{center}
	}
\end{vd}
\subsubsection{Bài tập rèn luyện}
\begin{bt}%[KNTT]%[1K3B8-1]
	Trong các mẫu số liệu sau, mẫu nào là mẫu số liệu ghép nhóm? Đọc và giải thích mẫu số liệu ghép nhóm đó.
	\begin{enumerate}
		\item Số tiền mà sinh viên chi cho thanh toán cước điện thoại trong tháng.
		\begin{center}
			\begin{tabular}{|c|c|c|c|c|c|}
				\hline
				Số tiền (nghìn đồng)&$[0;50)$&$[50;100)$&$[100;150)$&$[150;200)$&$[200;250)$\\
				\hline
				Số sinh viên&$5$&$12$&$23$&$17$&$3$\\
				\hline
			\end{tabular}
		\end{center}
		\item Thống kê nhiệt độ tại một điểm trong $40$ ngày, ta có bảng số liệu sau
		\begin{center}
			\begin{tabular}{|c|c|c|c|c|}
				\hline
				Nhiệt độ $(^\circ$ C)&$[19;22)$&$[22;25)$&$[25;28)$&$[28;31)$\\
				\hline
				Số ngày&$7$&$15$&$12$&$6$\\
				\hline
			\end{tabular}
		\end{center}
	\end{enumerate}
	\loigiai{
		Cả hai mẫu số liệu trên đều là mẫu số lớp ghép nhóm.
		\begin{enumerate}
			\item Có năm nhóm là
			\begin{itemize}
				\item Dưới $50$ nghìn đồng có $5$ sinh viên.
				\item Từ $50$ đến dưới $100$ nghìn đồng có $12$ sinh viên.
				\item Từ $100$ đến dưới $150$ nghìn đồng có $23$ sinh viên.
				\item Từ $150$ đến dưới $200$ nghìn đồng có $17$ sinh viên.
				\item Từ $200$ đến dưới $250$ nghìn đồng có $3$ sinh viên.
			\end{itemize}
			\item Có bốn nhóm là
			\begin{itemize}
				\item Từ $19^\circ$ C đến dưới $22^\circ$ C có $7$ ngày.
				\item Từ $22^\circ$ C đến dưới $25^\circ$ C có $15$ ngày.
				\item Từ $25^\circ$ C đến dưới $28^\circ$ C có $12$ ngày.
				\item Từ $128^\circ$ C đến dưới $31^\circ$ C có $6$ ngày.
			\end{itemize}
		\end{enumerate}
	}
\end{bt}	
\begin{bt}%[KNTT]%[1K3B8-1]
	Số sản phẩm một công nhân làm được trong một ngày được cho như sau:
	\begin{center}
		\begin{tabular}{c c c c c c c c c c c c c}
			$18$&$25$&$39$&$12$&$54$&$27$&$46$&$25$&$19$&$8$&$36$&$22$&\\
			$20$&$19$&$17$&$44$&$5$&$18$&$23$&$28$&$25$&$34$&$46$&$27$&$16$
		\end{tabular}
	\end{center}
	Hãy chuyển mẫu số liệu sang dạng ghép nhóm với sáu nhóm có độ dài bằng nhau.
	\loigiai{
		Khoảng biến thiên là $54-5=49$.\\
		Ta chia thành các nhóm sau $[4{,}5;13); [13;21{,}5);[21{,}5;30);\ldots ;[47;55{,}5)$.\\
		Đếm số giá trị của mỗi nhóm, ta có bảng ghép nhóm sau:
		\begin{center}
			\begin{tabular}{|c|c|c|c|c|c|c|}
				\hline
				Số sản phẩm &$[4{,}5;13)$&$[13;21{,}5)$&$[21{,}5;30)$&$[30;38{,}5)$&$[38{,}5;47)$&$[47;55{,}5)$\\
				\hline
				Số công nhân&$3$&$7$&$8$&$2$&$4$&$1$\\
				\hline
			\end{tabular}
		\end{center}
	}
\end{bt}
\begin{bt}%[KNTT]%[1K3B8-1]
	Thời gian ra sân (giờ) của một số cựu cầu thủ ở giải ngoại hạng Anh qua các thời kì được cho như sau:
	\begin{center}
		\begin{tabular}{c c c c c c c c}
			$653$ & $632$ & $609$ & $572$ & $565$ & $535$ & $516$ & $514$ \\
			$508$ & $505$ & $504$ & $504$ & $503$ & $499$ & $496$ & $492$ 
		\end{tabular}
	\end{center}
	Hãy chuyển mẫu số liệu trên sang dạng ghép nhóm với bảy nhóm có độ dài bằng nhau.
	\loigiai{
		Khoảng biến thiên là $653-492=161$.\\
		Ta chia thành các nhóm sau $[492;515); [515;538);[538;561);\ldots; [630;653]$.\\
		Đếm số giá trị của mỗi nhóm, ta có bảng ghép nhóm sau:
		\begin{center}
			\begin{tabular}{|c|c|c|c|c|c|c|c|}
				\hline
				Thời gian &$[492;515)$&$[515;538)$&$[538;561)$&$[561;584)$&$[584;607)$&$[607;630)$&$[630;653]$\\
				\hline
				Số cầu thủ &$9$&$2$&$0$&$2$&$0$&$1$&$2$\\
				\hline
			\end{tabular}
		\end{center}
	}
\end{bt}

%===================================
\setcounter{subsubsection}{0}
\setcounter{ex}{0}
\setcounter{bt}{0}
\begin{dang}{Số trung bình cộng (số trung bình)}
\end{dang}
\subsubsection{Ví dụ minh hoạ}
\begin{vd}%[Cánh Diều]%[1C5Y1-2]
	\immini
	{
		Một nhà thực vật học đo chiều dài của $74$ lá cây (đơn vị: milimét) và thu được bảng tần số như bảng bên. Tính chiều dài trung bình của $74$ lá cây trên theo đơn vị milimét (làm tròn kết quả đến hàng phần trăm).
	}
	{
		\begin{tabular}{|c|c|c|}
			\hline
			\textbf{Nhóm} & \textbf{Giá trị đại diện} & \textbf{Tần số}\\ 
			\hline
			$\left[5{,}45;5{,}85\right)$ & $5{,}65$ & $5$\\
			$\left[5{,}85;6{,}25\right)$ & $6{,}05$ & $9$\\
			$\left[6{,}25;6{,}65\right)$ & $6{,}45$ & $15$\\
			$\left[6{,}65;7{,}05\right)$ & $6{,}85$ & $19$\\
			$\left[7{,}05;7{,}45\right)$ & $7{,}25$ & $16$\\
			$\left[7{,}45;7{,}85\right)$ & $7{,}65$ & $8$\\
			$\left[7{,}85;8{,}25\right)$ & $8{,}05$ & $2$\\
			\hline
			&  & $n = 74$\\
			\hline
		\end{tabular}
	}
	\loigiai{
		Chiều dài trung bình của $74$ lá cây mà nhà thực vật học đo xấp xỉ là 
		\[
		\overline{x} = \dfrac{5\cdot 5{,}65 + 9 \cdot 6{,}05 + 15\cdot 6{,}45 + 19\cdot 6{,}85 + 16 \cdot 7{,}25 + 8\cdot 7{,}65 + 2\cdot 8{,}05}{74} \approx 6{,}80\ (\text{mm}).
		\]
	}
\end{vd}
\begin{vd}%[CTST]%[1T5B1-2]
	Kết quả khảo sát cân nặng của $25$ quả cam ở mỗi lô hàng $A$ và $B$ được cho ở bảng sau:
	\begin{center}
		\begin{tabular}{|c|c|c|c|c|c|}
			\hline \multicolumn{1}{|c|}{Cân nặng $(\mathrm{g})$} &{$[150; 155)$} &{$[155; 160)$} &{$[160; 165)$} &{$[165; 170)$} &{$[170; 175)$} \\
			\hline Số quả cam ở lô hàng $A$ & 2 & 6 & 12 & 4 & 1 \\
			\hline Số quả cam ở lô hàng $B$ & 1 & 3 & 7 & 10 & 4 \\
			\hline
		\end{tabular}
	\end{center}
	\begin{enumerate}
		\item Hãy ước lượng cân nặng trung bình của mỗi quả cam ở lô hàng $A$ và lô hàng $B$.
		\item Nếu so sánh theo số trung bình thì cam ở lô hàng nào nặng hơn?
	\end{enumerate}
	\loigiai{
		Ta có bảng thống kê số lượng cam theo giá trị đại diện:
		\begin{center}
			\begin{tabular}{|c|c|c|c|c|c|}
				\hline \multicolumn{1}{|c|}{Cân nặng $(\mathrm{g})$} &{$152{,}5$} &{$157{,}5$} &{$162{,}5$} &{$167{,}5$} &$172{,}5$\\
				\hline Số quả cam ở lô hàng $A$ & 2 & 6 & 12 & 4 & 1 \\
				\hline Số quả cam ở lô hàng $B$ & 1 & 3 & 7 & 10 & 4 \\
				\hline
			\end{tabular}
		\end{center}
		\begin{enumerate}
			\item Cân nặng trung bình của mỗi quả cam ở lô hàng $A$ xấp xỉ bằng
			\[(2\cdot 152{,}5+6\cdot 157{,}5+12\cdot 162{,}5+4\cdot 167{,}5+1\cdot 172{,}5): 25=161{,}7\ (\mathrm{g}). \]
			Cân nặng trung bình của mỗi quả cam ở lô hàng $B$ xấp xỉ bằng
			\[(1\cdot 152{,}5+3\cdot 157{,}5+7\cdot 162{,}5+10\cdot 167{,}5+4\cdot 172{,}5): 25=165{,}1\ (\mathrm{g}). \]
			\item Nếu so sánh theo số trung bình thì cam ở lô hàng $B$ nặng hơn cam ở lô hàng $A$.
		\end{enumerate}
	}
\end{vd}
\begin{vd}%[KNTT]%[1K3B9-1]
	Tìm cân nặng trung bình của học sinh lớp $11D$ cho trong Bảng $3.5$.
	\begin{center}
		\begin{tabular}{|c|c|c|c|c|c|c|}
			\hline
			Cân nặng	& $\left[40{,}5;45{,}5 \right)$ & $\left[45{,}5;50{,}5 \right)$ & $\left[50{,}5;55{,}5 \right)$ & $\left[55{,}5;60{,}5 \right)$ & $\left[60{,}5;65{,}5 \right)$ & $\left[65{,}5;70{,}5 \right)$ \\
			\hline
			Số học sinh&$10$	& $7$ & $16$ &$4$  & $2$ & $3$ \\
			\hline
		\end{tabular}

		Bảng $3.5$. Cân nặng của học sinh lớp $11D$.	
	\end{center}	
	\loigiai{
		Trong mỗi khoảng cân nặng, giá trị đại diện là trung bình cộng của hai giá trị đầu mút nên ta có bảng sau
		\begin{center}
			\begin{tabular}{|c|c|c|c|c|c|c|}
				\hline
				Cân nặng (kg)	& $43$ & $48$ & $53$ & $58$ & $63$ & $68$ \\
				\hline
				Số học sinh &$10$ & $7$ & $16$ &$4$  & $2$ & $3$ \\
				\hline
			\end{tabular}
		\end{center}	
		Tổng số học sinh là $n=42$. Cân nặng trung bình của học sinh lớp $11D$ là $$\overline{x}=\dfrac{10\cdot 43+7\cdot 48+16\cdot 53+4\cdot 58+2\cdot 63+3\cdot 68}{42}\approx51{,}81\,\mathrm{(kg)}.$$
	}
\end{vd}
\subsubsection{Bài tập rèn luyện}
\begin{bt}%[Cánh diều]%[1C5B1-5]
	Mẫu số liệu dưới đây ghi lại tốc độ của $40$ ô tô khi đi qua một trạm đo tốc độ (đơn vị: km/h)
	\[
	\begin{array}{cccccccccc}
		48{,}5 & 43 & 50 & 55 & 45 & 60 & 53 & 55,5 & 44 & 65 \\
		51 & 62,5 & 41 & 44,5 & 57 & 57 & 68 & 49 & 46{,}5 & 53{,}5 \\
		61 & 49{,}5 & 54 & 62 & 59 & 56 & 47 & 50 & 60 & 61 \\
		49{,}5 & 52{,}5 & 57 & 47 & 60 & 55 & 45 & 47,5 & 48 & 61{,}5
	\end{array}
	\]
	\begin{enumerate}
		\item Lập bảng tần số ghép nhóm cho mẫu số liệu trên có sáu nhóm ứng với sáu nửa khoảng:
		\[
		[40 ; 45),[45 ; 50),[50 ; 55),[55 ; 60),[60 ; 65),[65 ; 70).
		\]
		\item Xác định số trung bình cộng của mẫu số liệu ghép nhóm trên.
	\end{enumerate}
	\loigiai{
		\begin{enumerate}
			\item Ta có bảng tần số ghép nhóm của mẫu số liệu trên như sau:
			\begin{center}
				\begin{tabular}{|c|c|c|c|}
					\hline
					\textbf{Nhóm} & \textbf{Giá trị đại diện} & \textbf{Tần số} & \textbf{Tần số tích luỹ}\\ 
					\hline
					$\left[40;45\right)$ & $42{,}5$ & $4$ & $4$\\
					$\left[45;50\right)$ & $47{,}5$ & $11$ & $15$\\
					$\left[50;55\right)$ & $52{,}5$ & $7$ & $22$\\
					$\left[55;60\right)$ & $57{,}5$ & $8$ & $30$\\
					$\left[60;65\right)$ & $62{,}5$ & $8$ & $38$\\
					$\left[65;70\right)$ & $67{,}5$ & $2$ & $40$\\
					\hline
					&  & $n = 40$ &\\
					\hline
				\end{tabular}
			\end{center}
			\item Trung bình cộng của mẫu số liệu trên là
			\[
			\overline{x} = \dfrac{42{,}5 \cdot 4 + 47{,}5 \cdot 11 + 52{,}5 \cdot 7+ 57{,}5 \cdot 8+ 62{,}5 \cdot 8 + 67{,}5 \cdot 2}{40} = 53{,}875\text{ (km/h)}.
			\]
			\item Ta thấy: Nhóm $2$ ứng với nửa khoảng $\left[45;50\right)$ là nhóm có tần số lớn nhất với $u=45$, $g=5$, $n_2 = 11$. Nhóm $1$ có tần số $n_1 = 4$, nhóm $3$ có tần số $n_3 = 7$.
		\end{enumerate}
	}
\end{bt}
\begin{bt}%[KNTT]%[1K3B9-4]
	Tuổi thọ (năm) của 50 bình ắc quy ô tô được cho như sau:
	\begin{center}
		\begin{tabular}{|c|c|c|c|c|c|c|}
			\hline
			Tuổi thọ (năm)	& $\left[2;2{,}5 \right)$ & $\left[2{,}5;3 \right)$ & $\left[3;3{,}5 \right)$&$\left[3{,}5;4 \right)$&$\left[4;4{,}5 \right)$&$\left[4{,}5;5 \right)$  \\
			\hline
			Tần số &$4$	& $9$ & $14$ &$11$  & $7$&$5$ \\
			\hline
		\end{tabular}
	\end{center}
	Tính tuổi thọ trung bình của $50$ bình ắc quy ô tô này.
	\loigiai{
		Ta có bảng sau
		\begin{center}
			\begin{tabular}{|c|c|c|c|c|c|c|}
				\hline
				Tuổi thọ (năm)	& $2{,}25$ & $2{,}75$ & $3{,}25$&$3{,}75$&$4{,}25$&$4{,}75$  \\
				\hline
				Tần số &$4$	& $9$ & $14$ &$11$  & $7$&$5$\\
				\hline
			\end{tabular}	
		\end{center}
		Tuổi thọ trung bình của 50 bình ắc quy ô tô này là
		$$\overline{x}=\dfrac{2{,}25\cdot 4+2{,}75\cdot 9+3{,}25\cdot 14+3{,}75\cdot 11+4{,}25\cdot 7+4{,}75\cdot 5}{50}=3{,}48 \, \text{(năm)}.$$
	}
\end{bt}
\begin{bt}%[KNTT]%[1K3B9-4]
	\immini{
		Phỏng vấn một số học sinh lớp $11$ về thời gian (giờ) ngủ của một buổi tối, thu được bảng số liệu ở bên. So sánh thời gian ngủ trung bình của các bạn học sinh nam và nữ.
	}
	{
		\begin{tabular}{|c|c|c|}
			\hline
			Thời gian	& Số học sinh nam & Số học sinh nữ\\
			\hline
			$\left[4;5 \right)$	& $6$ & $4$ \\
			\hline
			$\left[5;6 \right)$	& $10$ & $8$ \\
			\hline
			$\left[6;7 \right)$	& $13$ & $10$ \\
			\hline
			$\left[7;8 \right)$	& $9$ & $11$ \\
			\hline
			$\left[8;9 \right)$	& $7$ & $8$ \\
			\hline
		\end{tabular}
	}
	\loigiai{
		Trong mỗi khoảng thời gian, giá trị đại diện là trung bình cộng của giá trị hai đầu mút nên ta có bảng sau:
		\begin{center}
			\begin{tabular}{|c|c|c|}
				\hline
				Thời gian	& Số học sinh nam & Số học sinh nữ\\
				\hline
				$4{,}5$	& $6$ & $4$ \\
				\hline
				$5{,}5$	& $10$ & $8$ \\
				\hline
				$6{,}5$	& $13$ & $10$ \\
				\hline
				$7{,}5$	& $9$ & $11$ \\
				\hline
				$8{,}5$	& $7$ & $8$ \\
				\hline
			\end{tabular}	
		\end{center}
		Tổng số học sinh nam là $n_1=6+10+13+9+7=45$.\\ Thời gian ngủ trung bình của học sinh nam là:
		$$\overline{x_1}=\dfrac{4{,}5\cdot 6+5{,}5\cdot10+6{,}5\cdot13+7{,}5\cdot9+8{,}5\cdot7}{45}=\dfrac{587}{90}\approx 6{,}52\,\, \text{(giờ)}.$$
		Tổng số học sinh nữ là $n_2=4+8+10+11+8=41$. Thời gian ngủ trung bình của học sinh nữ là:
		$$\overline{x_2}=\dfrac{4,5\cdot4+5,5\cdot8+6,5\cdot10+7,5\cdot11+8,5\cdot8}{41}=\dfrac{555}{82}\approx 6{,}77 \,\,\text{(giờ)}.$$
		Vì $\overline{x_2}>\overline{x_1}$ nên thời gian ngủ trung bình của các bạn học sinh nữ lớn hơn thời gian ngủ trung bình của các bạn nam.
	}
\end{bt}
\begin{bt}%[KNTT]%[1K3B9-4]
	Quãng đường (km) từ nhà đến nơi làm việc của 40 công nhân một nhà máy được ghi lại như sau:
	\begin{center}
		\begin{tabular}{cccccccccccccccccccc}
			$5$	& $3$ &$10$ & $20$ & $25$ & $11$ & $13$ & $7$ & $12$ & $31$\\
			$19$ &$10$  &$12$  & $17$ & $18$ & $11$ & $32$ & $17$ &$16$  &$2$ \\
			$7$	& $9$ &$7$ & $8$ & $3$ & $5$ & $12$ & $15$ & $18$ & $3$\\
			$12$ &$14$  &$2$  & $9$ & $6$ & $15$ & $15$ & $7$ &$6$  &$12$
		\end{tabular}
	\end{center}
	\begin{enumerate}
		\item [a)] Ghép nhóm dãy số liệu trên thành các khoảng có độ rộng bằng nhau, khoảng đầu tiên là $\left[0;5\right)$. Tìm giá trị đại diện cho mỗi nhóm.
		\item [b)] Tính số trung bình của mẫu số liệu không ghép nhóm và mẫu số liệu ghép nhóm. Giá trị nào chính xác hơn?
	\end{enumerate}
	\loigiai{
		\begin{enumerate}
			\item [a)] Giá trị nhỏ nhất của mẫu số liệu là $2$, giá trị lớn nhất là $32$, khoảng đầu tiên của mẫu số liệu ghép nhóm là $\left[0;5\right)$ nên ta ghép nhóm mẫu số liệu như sau
			\begin{center}
				\begin{tabular}{|c|c|c|c|c|c|c|c|}
					\hline
					Quãng đường		 & $\left[0;5\right)$ & $\left[5;10\right)$ & $\left[10;15\right)$ & $\left[15;20\right)$ & $\left[20;25\right)$& $\left[25;30\right)$& $\left[30;35\right)$\\
					\hline
					Số công nhân		& $5$ & $11$ & $11$ & $9$ & $1$ & $1$ & $2$ \\
					\hline
				\end{tabular}
			\end{center}
			Trong mỗi khoảng, giá trị đại điện là trung bình cộng của hai giá trị đầu mút nên ta có bảng sau
			\begin{center}
				\begin{tabular}{|c|c|c|c|c|c|c|c|}
					\hline
					Quãng đường		 & $2{,}5$ & $7{,}5$ & $12{,}5$ & $17{,}5$ & $22{,}5$& $27{,}5$& $32{,}5$\\
					\hline
					Số công nhân		& $5$ & $11$ & $11$ & $9$ & $1$ & $1$ & $2$ \\
					\hline
				\end{tabular}
			\end{center}
			\item [b)] Số trung bình của mẫu số liệu không ghép nhóm là
			$$\overline{x}=\dfrac{5+3+10+\cdots +12}{40}=11{,}9.$$
			Số trung bình của mẫu số liệu ghép nhóm là
			$$\overline{x}=\dfrac{5\cdot 2{,}5+11\cdot 7{,}5+11\cdot 12{,}5+9\cdot 17{,}5+1\cdot 22{,}5+1\cdot 27{,}5+2\cdot 32{,}5}{40}=12{,}625.$$
			Số trung bình của mẫu số liệu không ghép nhóm sẽ chính xác hơn số trung bình của mẫu số liệu ghép nhóm vì số trung bình của dữ liệu không ghép nhóm sử dụng chính xác các số liệu, còn số trung bình của dữ liệu ghép nhóm sử dụng giá trị đại diện của mỗi khoảng ghép nhóm.
		\end{enumerate}
	}
\end{bt}
\begin{bt}%[CTST]%[1T5B1-2]
	Anh Văn ghi lại cự li 30 lần ném lao của mình ở bảng sau (đơn vị: mét):
	\begin{center}
		\begin{tabular}{|c|c|c|c|c|c|c|c|c|c|}
			\hline $72{,}1$ & $72{,}9$ & $70{,}2$ & $70{,}9$ & $72{,}2$ & $71{,}5$ & $72{,}5$ & $69{,}3$ & $72{,}3$ & $69{,}7$ \\
			\hline $72{,}3$ & $71{,}5$ & $71{,}2$ & $69{,}8$ & $72{,}3$ & $71{,}1$ & $69{,}5$ & $72{,}2$ & $71{,}9$ & $73{,}1$ \\
			\hline $71{,}6$ & $71{,}3$ & $72{,}2$ & $71{,}8$ & $70{,}8$ & $72{,}2$ & $72{,}2$ & $72{,}9$ & $72{,}7$ & $70{,}7$ \\
			\hline
		\end{tabular}
	\end{center}
	\begin{enumerate}
		\item Tính cự li trung bình của mỗi lần ném.
		\item Tổng hợp lại kết quả ném của anh Văn vào bảng tần số ghép nhóm theo mẫu sau:
		\begin{center}
			\begin{tabular}{|c|c|c|c|c|c|}
				\hline Cự li $(\mathrm{m})$ &{$[69{,}2; 70)$} &{$[70; 70{,}8)$} &{$[70{,}8; 71{,}6)$} &{$[71{,}6; 72{,}4)$} &{$[72{,}4; 73{,}2)$} \\
				\hline Số lần & $?$ & $?$ & $?$ & $?$ & $?$ \\
				\hline
			\end{tabular}
		\end{center}
		\item Hãy ước lượng cự li trung bình mỗi lần ném từ bảng tần số ghép nhóm trên.
		\item Khả năng anh Văn ném được khoảng bao nhiêu mét là cao nhất?
	\end{enumerate}
	\loigiai{
		\begin{enumerate}
			\item Điểm tổng của mỗi đợt gồm 10 lần ném
			\begin{center}
				\begin{tabular}{|c|c|c|c|c|c|c|c|c|c|c|}
					\hline Điểm &Điểm &Điểm &Điểm &Điểm &Điểm &Điểm &Điểm &Điểm &Điểm &Tổng \\
					\hline $72{,}1$ & $72{,}9$ & $70{,}2$ & $70{,}9$ & $72{,}2$ & $71{,}5$ & $72{,}5$ & $69{,}3$ & $72{,}3$ & $69{,}7$ &$713{,}6$\\
					\hline $72{,}3$ & $71{,}5$ & $71{,}2$ & $69{,}8$ & $72{,}3$ & $71{,}1$ & $69{,}5$ & $72{,}2$ & $71{,}9$ & $73{,}1$ &$714{,}9$\\
					\hline $71{,}6$ & $71{,}3$ & $72{,}2$ & $71{,}8$ & $70{,}8$ & $72{,}2$ & $72{,}2$ & $72{,}9$ & $72{,}7$ & $70{,}7$ &$718{,}4$\\
					\hline
				\end{tabular}
			\end{center}
			Cự li trung bình của mỗi lần ném của anh Văn
			\[\overline{x}=\dfrac{713{,}6+714{,}9+718{,}4}{30}\approx71{,}56\ (\mathrm{m}). \]
			\item Bảng tần số ghép nhóm kết quả ném của anh Văn:
			\begin{center}
				\begin{tabular}{|c|c|c|c|c|c|}
					\hline Cự li $(\mathrm{m})$ &{$[69{,}2; 70)$} &{$[70; 70{,}8)$} &{$[70{,}8; 71{,}6)$} &{$[71{,}6; 72{,}4)$} &{$[72{,}4; 73{,}2)$} \\
					\hline Số lần & $4$ & $2$ & $7$ & $12$ & $5$ \\
					\hline
				\end{tabular}
			\end{center}
			\item Bảng tần số ghép nhóm kết quả ném của anh Văn (theo giá trị đại diện):
			\begin{center}
				\begin{tabular}{|c|c|c|c|c|c|}
					\hline Cự li $(\mathrm{m})$ &{$[69{,}2; 70)$} &{$[70; 70{,}8)$} &{$[70{,}8; 71{,}6)$} &{$[71{,}6; 72{,}4)$} &{$[72{,}4; 73{,}2)$} \\
					\hline Giá trị đại diện &$69{,}6$ &$70{,}4$ &$71{,}2$ &$72{,}0$ &$72{,}8$\\
					\hline Số lần & $4$ & $2$ & $7$ & $12$ & $5$ \\
					\hline
				\end{tabular}
			\end{center}
			Cự li trung bình mỗi lần ném của anh Văn qua bảng tần số ghép nhóm
			\[(69{,}6\cdot 4+70{,}4\cdot 2+71{,}2\cdot 7+72\cdot 12+72{,}8\cdot 5):30=71{,}52\ (\mathrm{m}).  \]
			\item Nhóm chứa mốt của mẫu số liệu trên là nhóm $[71{,}6; 72{,}4)$.\\
			Do đó $u_m=71{,}6$; $n_{m-1}=7$; $n_m=12$; $n_{m+1}=5$; $u_{m+1}-u_m=72{,}4-71{,}6=0{,}8$.\\
			Mốt của mẫu số liệu ghép nhóm là
			\[M_0=71{,}6+\dfrac{12-7}{(12-7)+(12-5)} \cdot 0{,}8=\dfrac{101}{14} \approx 71{,}93. \]
			Dựa vào kết quả trên thì khả năng anh Văn ném được cao nhất là khoảng $71{,}93$ mét.
		\end{enumerate}
	}
\end{bt}
\begin{bt}%[CTST]%[1T5B1-2]
	Người ta đếm số xe ô tô đi qua một trạm thu phí mỗi phút trong khoảng thời gian từ $9$ giờ đến $9$ giờ $30$ phút sáng. Kết quả được ghi lại ở bảng sau:
	\begin{center}
		\begin{tabular}{|c|c|c|c|c|c|c|c|c|c|c|c|c|c|c|}
			\hline $15$ & $16$ & $13$ & $21$ & $17$ & $23$ & $15$ & $21$ & $6$ & $11$ & $12$ & $23$ & $19$ & $25$ & $11$ \\
			\hline $25$ & $7$ & $29$ & $10$ & $28$ & $29$ & $24$ & $6$ & $11$ & $23$ & $11$ & $21$ & $9$ & $27$ & $15$ \\
			\hline
		\end{tabular}
	\end{center}
	\begin{enumerate}
		\item Tính số xe trung bình đi qua trạm thu phí trong mỗi phút.
		\item Tổng hợp lại số liệu trên vào bảng tần số ghép nhóm theo mẫu sau:
		\begin{center}
			\begin{tabular}{|c|c|c|c|c|c|}
				\hline Số xe &{$[6; 10]$} &{$[11; 15]$} &{$[16; 20]$} &{$[21; 25]$} &{$[26; 30]$} \\
				\hline Số lần & $?$ & $?$ & $?$ & $?$ & $?$ \\
				\hline
			\end{tabular}
		\end{center}
		\item Hãy ước lượng trung bình số xe đi qua trạm thu phí trong mỗi phút từ bảng tần số ghép nhóm trên.
	\end{enumerate}
	\loigiai{
		\begin{enumerate}
			\item 
%			Bảng tần số
%			\begin{center}
%				\begin{tabular}{|c|c|c|c|c|c|c|c|c|c|c|c|c|c|c|c|c|c|c|c|}
%					\hline Giá trị &$6$ & $7$ & $9$ & $10$ & $11$ & $12$ & $13$ & $15$ & $16$ & $17$ & $19$ & $21$ & $23$ & $24$ & $25$ & $27$ & $28$ & $29$ &\\
%					\hline Tần số &$2$ & $1$ & $1$ & $1$ & $4$ & $1$ & $1$ & $3$ & $1$ & $1$ & $1$ & $3$ & $3$ & $1$ & $2$ & $1$ & $1$ & $2$ &$N=30$\\
%					\hline
%				\end{tabular}
%			\end{center}
			Số xe trung bình đi qua trạm thu phí trong mỗi phút là
			\allowdisplaybreaks
			\begin{eqnarray*}
				\overline{x}&=&\dfrac{6\cdot 2+7+9+10+11\cdot 4+12+13+15\cdot 3}{30}\\
				&&+\dfrac{16+17+19+21\cdot 3+23\cdot 3+24+25\cdot 2+27+28+29\cdot 2}{30}\\
				&\approx& 17{,}43\ (\text{xe}).
			\end{eqnarray*}	
			\item Bảng tần số ghép nhóm
			\begin{center}
				\begin{tabular}{|c|c|c|c|c|c|}
					\hline Số xe &{$[6; 10]$} &{$[11; 15]$} &{$[16; 20]$} &{$[21; 25]$} &{$[26; 30]$} \\
					\hline Số lần & $5$ & $9$ & $3$ & $9$ & $4$ \\
					\hline
				\end{tabular}
			\end{center}
			\item Bảng tần số ghép nhóm (theo giá trị đại diện) được hiệu chỉnh lại như sau
			\begin{center}
				\begin{tabular}{|c|c|c|c|c|c|}
					\hline Số xe &{$[5{,}5; 10{,}5)$} &{$[10{,}5; 15{,}5)$} &{$[15{,}5; 20{,}5)$} &{$[20{,}5; 25{,}5)$} &{$[25{,}5; 30{,}5)$} \\
					\hline Giá trị đại diện &{$8$} &{$13$} &{$18$} &{$23$} &{$28$} \\
					\hline Số lần & $5$ & $9$ & $3$ & $9$ & $4$ \\
					\hline
				\end{tabular}
			\end{center}
			Số xe trung bình đi qua trạm qua bảng tần số ghép nhóm là
			\[\overline{x}=\dfrac{8\cdot 5+13\cdot 9+18\cdot 3+23\cdot 9+28\cdot 4}{30}\approx 17{,}67\ (\text{xe}). \]
		\end{enumerate}
	}
\end{bt}
\begin{bt}%[CTST]%[1T5B1-2]
	Một thư viện thống kê số lượng sách được mượn mỗi ngày trong ba tháng ở bảng sau:
	\begin{center}
		\begin{tabular}{|c|c|c|c|c|c|c|c|}
			\hline Số sách &{$[16; 20]$} &{$[21; 25]$} &{$[26; 30]$} &{$[31; 35]$} &{$[36; 40]$} &{$[41; 45]$} &{$[46; 50]$} \\
			\hline Số ngày & 3 & 6 & 15 & 27 & 22 & 14 & 5 \\
			\hline
		\end{tabular}
	\end{center}
	Hãy ước lượng số trung bình của mẫu số liệu ghép nhóm trên.
	\loigiai{
		Vì số lượng sách được mượn là số nguyên nên ta hiệu chỉnh bảng tần số ghép nhóm (theo giá trị đại diện) như sau
		\begin{center}
			{\footnotesize \begin{tabular}{|c|c|c|c|c|c|c|c|}
					\hline Số sách &{$[15{,}5; 20{,}5)$} &{$[20{,}5; 25{,}5)$} &{$[25{,}5; 30{,}5)$} &{$[30{,}5; 35{,}5)$} &{$[35{,}5; 40{,}5]$} &{$[40{,}5; 45{,}5)$} &{$[45{,}5; 50{,}5)$} \\
					\hline Giá trị đại diện &{$18$} &{$23$} &{$28$} &{$33$} &{$38$} &{$43$} &{$48$} \\
					\hline Số ngày & 3 & 6 & 15 & 27 & 22 & 14 & 5 \\
					\hline
			\end{tabular}}
		\end{center}
		Trung bình số lượng sách được mượn mỗi ngày trong 3 tháng của thư viện là
		\[\overline{x}=\dfrac{18\cdot 3+23\cdot 6+28\cdot 15+33\cdot 27+38\cdot 22+43\cdot 14+48\cdot 5}{92}\approx 34{,}58. \]
	}
\end{bt}
\begin{bt}%[CTST]%[1T5B1-2]
	Kết quả đo chiều cao của $200$ cây keo $3$ năm tuổi ở một nông trường được biểu diễn ở biểu đồ dưới đây.
	\begin{center}
		\begin{tikzpicture}[scale=1,font=\scriptsize]
			\def\hoanh{11.5};
			\def\tung{6.5};
			\def\mau{cyan};
			\foreach \x/\n in{1/20,3/35,5/60,7/55,9/30}{\draw[line width=16mm,\mau] (\x,0)--++(0,{\n/10});
				\draw[dashed] (\x,{\n/10})node[above]{$\n$}--(0,{\n/10}) node[left]{$\n$};}
			\foreach \x/\p in {1/[8{,}5;8{,}8),3/[8{,}8;9{,}1),5/[9{,}1;9{,}4),7/[9{,}4;9{,}7),9/[9{,}7;10{,}0)}{\node[below] at (\x,0){\scriptsize $\p$};}
			\draw[->] (0,0)--(\hoanh,0) node[below]{($m$)};
			\draw[->] (0,0)node[below left]{$O$}--(0,\tung) node[left]{(Số cây)};
			\path (current bounding box.north) node[above]		{\textbf{Chiều cao 200 cây keo 3 năm tuổi}};
		\end{tikzpicture}
	\end{center}
	Hãy ước lượng số trung bình của mẫu số liệu ghép nhóm trên.
	\loigiai{
		Bảng tần số ghép nhóm (theo giá trị đại diện)
		\begin{center}
			\begin{tabular}{|c|c|c|c|c|c|}
				\hline Chiều cao &$[8{,}5; 8{,}8)$ &{$[8{,}8; 9{,}1)$} &{$[9{,}1; 9{,}4)$} &{$[9{,}4; 9{,}7)$} &{$[9{,}7; 10{,}0)$} \\
				\hline Giá trị đại diện &$8{,}65$ &$8{,}95$ &$9{,}25$ &$9{,}55$ &$9{,}85$ \\
				\hline Số cây & $20$ & $35$ & $60$ & $55$ & $30$\\
				\hline
			\end{tabular}
		\end{center}
		Chiều cao trung bình của $200$ cây keo 3 năm tuổi là
		\[\overline{x}=\dfrac{8{,}65\cdot 20+8{,}95\cdot 35+9{,}25\cdot 60+9{,}55\cdot 55+9{,}85\cdot 30}{200}\approx 9{,}31. \]
	}
\end{bt}
\begin{bt}%[CTST]%[1T5K2-2]
	Kiểm tra điện lượng của một số viên pin tiểu do một hãng sản xuất thu được kết quả như sau:
	\begin{center}
		\begin{tabular}{|c|c|c|c|c|c|}
			\hline 
			\begin{tabular}{c}
				\textbf{Điện lượng} \\	\textbf{(nghìn mAh)}
			\end{tabular} 
			& $ \left[ 0{,}9 ; 0{,}95\right)  $ & $ \left[ 0{,}95 ; 1{,}0\right)  $ & $ \left[ 1{,}0 ; 1{,}05\right)  $ &$ \left[ 1{,}05 ; 1{,}1\right)  $  &  $ \left[ 1{,}1 ; 1{,}15\right)  $\\ 
			\hline 
			\textbf{Số viên pin}& $ 10 $ & $ 20 $ & $ 35 $ & $ 15 $ & $ 5 $ \\ 
			\hline 
		\end{tabular} 
	\end{center}
	Hãy ước lượng số trung bình của mẫu số liệu ghép nhóm trên.
	\loigiai{
		Tìm số trung bình của mẫu số liệu ghép nhóm.\\
		Ta có bảng thống kê điện lượng của pin theo giá trị đại diện là:
		\begin{center}
			\begin{tabular}{|c|c|c|c|c|c|}
				\hline 
				\begin{tabular}{c}
					\textbf{Điện lượng} \\	\textbf{(nghìn mAh)}
				\end{tabular} 
				& $ \left[ 0{,}9 ; 0{,}95\right)  $ & $ \left[ 0{,}95 ; 1{,}0\right)  $ & $ \left[ 1{,}0 ; 1{,}05\right)  $ &$ \left[ 1{,}05 ; 1{,}1\right)  $  &  $ \left[ 1{,}1 ; 1{,}15\right)  $\\ 
				\hline 
				\textbf{Giá trị đại diện}& $ 0{,}925 $ & $ 0{,}975 $ & $ 1{,}025 $ & $ 1{,}075 $ & $ 1{,}125 $ \\ 
				\hline
				\textbf{Số viên pin}& $ 10 $ & $ 20 $ & $ 35 $ & $ 15 $ & $ 5 $ \\ 
				\hline 
			\end{tabular} 
		\end{center}
		Số trung bình của mẫu số liệu ghép nhóm theo dõi điện lượng của một số viên pin xấp xỉ bằng $$\dfrac{0{,}925\cdot 10 + 0{,}975\cdot 20 +1{,}025 \cdot 35 +1{,}075 \cdot 15+1{,}125 \cdot 5}{10+20+35+15+5}\approx 1{,}016.$$
	}
\end{bt}

%===================================
\setcounter{subsubsection}{0}
\setcounter{ex}{0}
\setcounter{bt}{0}
\begin{dang}{Trung vị}
\end{dang}
\subsubsection{Ví dụ minh hoạ}
\begin{vd}%[Cánh Diều]%[1C5B1-3]
	\immini
	{
		Sau khi kiểm tra về số học sinh trong $100$ lớp học, người ta chia mẫu số liệu đó thành năm nhóm căn cứ vào số lượng học sinh của mỗi lớp (đơn vị: học sinh) và lập bảng tần số ghép nhóm bao gồm tần số tích luỹ như bảng bên. Tìm trung vị của mẫu số liệu đó.
	}
	{
		\begin{tabular}{|c|c|c|}
			\hline
			\textbf{Nhóm} & \textbf{Tần số} & \textbf{Tần số tích luỹ}\\ 
			\hline
			$\left[36;38\right)$ & $9$ & $9$\\
			$\left[38;40\right)$ & $15$ & $24$\\
			$\left[40;42\right)$ & $25$ & $49$\\
			$\left[42;44\right)$ & $30$ & $79$\\
			$\left[44;46\right)$ & $21$ & $100$\\
			\hline
			& $n = 100$ &\\
			\hline
		\end{tabular}
	}
	\loigiai{
		Số phần tử của mẫu là $n=100$. Ta có $\dfrac{n}{2} = \dfrac{100}{2} = 50$.\\
		Do $cf_3 = 49 < 50 < cf_4 = 79$ nên nhóm $4$ là nhóm đầu tiên có tần số tích luỹ lớn hơn hoặc bằng $50$.\\
		Xét nhóm $4$ là nhóm $\left[42;44\right)$ có $r=42$; $d=2$ và $n_4=30$ và nhóm $3$ là nhóm $\left[40;42\right)$ có $cf_3 = 49$.\\
		Khi đó trung vị của mẫu số liệu là 
		\[
		M_e = 42 + \dfrac{50 - 49}{30} \cdot 2 \approx 42\text{ (học sinh)}.
		\]
	}
\end{vd}
\begin{vd}%[KNTT]%[1K3B9-2]
	Thời gian (phút) truy cập internet mỗi buổi tối của một số học sinh được cho trong bảng sau:
	\begin{center}
		\begin{tabular}{|c|c|c|c|c|c|c|}
			\hline
			Thời gian (phút)	& $\left[9{,}5;12{,}5 \right)$ & $\left[12{,}5;15{,}5 \right)$ & $\left[15{,}5;18{,}5 \right)$ & $\left[18{,}5;21{,}5 \right)$ & $\left[21{,}5;24{,}5 \right)$ \\
			\hline
			Số học sinh&$3$	& $12$ & $15$ &$24$  & $2$  \\
			\hline
		\end{tabular}	
	\end{center}
	Tính trung vị của mẫu số liệu ghép nhóm này.
	\loigiai{
		Cỡ mẫu là $n=3+12+15+24+2=56$.\\
		Gọi $x_1,\,\ldots,\,x_{56}$ là thời gian vào internet của $56$ học sinh và giả sử dãy này đã được sắp xếp theo thứ tự tăng dần. Khi đó, trung vị là $\dfrac{x_{28}+x_{29}}{2}$. Do $2$ giá trị $x_{28},\,x_{29}$ thuộc nhóm $\left[15{,}5;18{,}5 \right)$ nên nhóm này chứa trung vị. Do đó, $p=3$; $a_3=15{,}5$; $m_3=15$; $m_1+m_2=3+12=15$; $a_4-a_3=3$ và ta có $$M_e=15{,}5+\dfrac{\dfrac{56}{2}-15}{15}\cdot 3=18{,}1.$$
	}
\end{vd}
\begin{vd}%[CTST]%[1T5B2-1]
	Kết quả khảo sát cân nặng của $ 25 $ quả bơ ở một lô hàng cho trong bảng sau:
	\begin{center}
		\begin{tabular}{|c|c|c|c|c|c|}
			\hline 
			\textbf{Cân nặng}\textbf{ (g)}	& $ \left[150 ; 155 \right) $ & $ \left[ 155 ; 160\right)  $ & $ \left[160 ; 165\right)  $ & $ \left[ 165 ; 170\right)  $ & $ \left[170 ; 175 \right)  $ \\ 
			\hline 
			\textbf{Số quả bơ}	& $ 1 $ & $ 7 $ & $ 12 $ & $ 3 $ & $ 2 $ \\ 
			\hline 
		\end{tabular} 
	\end{center}
	Hãy tìm trung vị của mẫu số liệu ghép nhóm trên.
	\loigiai{
		Gọi $ x_1; x_2; \ldots ; x_{25} $ là cân nặng của $ 25$ quả bơ xếp theo thứ tự không giảm.\\
		Do $ x_1\in \left[150 ; 155 \right) $; $ x_2, \ldots, x_8 \in \left[ 155 ; 160\right) $; $ x_9, \ldots, x_{20} \in \left[ 160 ; 165\right) $ nên trung vị của mẫu số liệu $ x_1; x_2; \ldots; x_{25} $ là $ x_{13}\in \left[ 160 ; 165\right)$.\\
		Ta xác định được $ n=25 $, $ n_m=12 $, $ C=1+7=8 $, $ u_m=160 $, $ u_{m+1}=165 $.\\
		Vậy trung vị của mẫu số liệu ghép nhóm là $$ M_e=160+\dfrac{\dfrac{25}{2}-8}{12}\cdot(165-160) =161{,}875.$$
	}
\end{vd}
\begin{vd}%[CTST]%[1T5K2-1]
	Trong tuần lễ bảo vệ môi trường, các học sinh khối $ 11 $ tiến hành thu nhặt vỏ chai nhựa để tái chế. Nhà trường thống kê kết quả thu nhặt vỏ chai của học sinh khối $ 11 $ ở bảng sau
	\begin{center}
		\begin{tabular}{|c|c|c|c|c|c|}
			\hline 
			\textbf{Số vỏ chai nhựa}	& $ \left[ 11 ; 15\right]  $ & $ \left[ 16 ; 29\right]  $ & $ \left[21 ; 25 \right]  $ & $ \left[ 26 ; 30\right]  $ & $ \left[31 ; 35 \right]  $ \\ 
			\hline 
			\textbf{Số học sinh}	& $ 53 $ & $ 82 $ & $ 48 $ & $ 39 $ & $ 18 $ \\ 
			\hline 
		\end{tabular} 
	\end{center}
	Hãy tìm trung vị của mẫu số liệu ghép nhóm trên.
	\loigiai{
		Do số vỏ chai là số nguyên nên ta hiệu chỉnh lại như sau:
		\begin{center}
			\begin{tabular}{|c|c|c|c|c|c|}
				\hline 
				\textbf{Số vỏ chai nhựa}	& $ \left[ 10{,}5 ; 15{,}5\right) $ & $ \left[ 15{,}5 ; 20{,}5\right) $ & $ \left[ 20{,}5 ; 25{,}5\right) $ & $ \left[ 25{,}5 ; 30{,}5\right) $ & $ \left[30{,}5 ; 35{,}5 \right)  $ \\ 
				\hline 
				\textbf{Số học sinh}& $ 53 $ & $ 82 $ & $ 48 $ & $ 39 $ & $ 18 $ \\ 
				\hline 
			\end{tabular} 
		\end{center}
		Số học sinh tham gia thu nhặt vỏ chai nhựa là $$ n=53+82+48+39+18=240.$$
		Gọi $ x_1; x_2; \ldots ; x_{240} $ lần lượt là số vỏ chai $ 240 $ học sinh khối $ 11 $ thu nhặt được xếp theo thứ tự không giảm.\\
		Do $ x_1, \ldots, x_{53}\in \left[10{,}5 ; 15{,}5 \right) $; $ x_{54}, \ldots, x_{135}\in \left[ 15{,}5 ; 20{,}5\right)$ nên trung vị của mẫu số liệu $ x_1; x_2; \ldots;x_{240} $ là $$ \dfrac{1}{2}\left( x_{120}+x_{121}\right)\in \left[ 15{,}5 ; 20{,}5\right).$$
		Ta xác định được $ n=240$; $ n_m=82 $; $ C=53 $; $ u_m=15{,}5 $; $ u_{m+1}=20{,}5 $.\\
		Trung vị của mẫu số liệu ghép nhóm là $$ M_e=15{,}5+\dfrac{\dfrac{240}{2}-53}{82}\cdot \left( 20{,}5-15{,}5\right)=\dfrac{803}{41}\approx 19{,}59. $$
	}
\end{vd}
\begin{vd}%[CTST]%[1T5K2-1]
	Trong một hội thao, thời gian chạy $200$m của một nhóm các vận động viên được ghi lại ở bảng sau
	\begin{center}
		\begin{tabular}{|c|c|c|c|c|c|}
			\hline 
			\textbf{Thời gian} \textbf{(giây)}& $ \left[21 ; 21{,}5 \right)  $ & $ \left[ 21{,}5 ; 22\right)  $ & $ \left[ 22 ; 22{,}5\right)  $ & $ \left[ 22{,}5 ; 23\right)  $ & $ \left[ 23 ; 23{,}5\right)  $ \\ 
			\hline 
			\textbf{Số vận động viên} & $ 5 $ & $ 12 $ & $ 32 $ & $ 45 $ & $ 30 $ \\ 
			\hline 
		\end{tabular} 
	\end{center}
	Dựa vào bảng số liệu trên, ban tổ chức muốn chọn ra khoảng $ 50 \% $ số vận động viên chạy nhanh nhất để tiếp tục thi vòng $ 2 $. Ban tổ chức nên chọn các vận động viên có thời gian chạy không quá bao nhiêu giây?
	\loigiai{
		Số vận động viên tham gia là $$n=5+12+32+45+30=124.$$
		Gọi $ x_1; x_2; \ldots ; x_{124} $ lần lượt là thời gian chạy $ 200 $ m của $ 124 $ vận động viên được xếp theo thứ tự không giảm.\\
		Do $ x_1, \ldots, x_5 \in \left[ 21 ; 21{,}5 \right)$, $ x_6, \ldots, x_{17} \in \left[ 21{,}5 ; 22\right) $, $ x_{18}, \ldots, x_{49} \in \left[22 ; 22{,}5\right) $, $ x_{50},\ldots, x_{94} \in \left[ 22{,}5 ; 23\right) $ nên trung vị của mẫu số liệu $ x_1; x_2; \ldots ;x_{124} $ là
		$$\dfrac{1}{2}\cdot \left( x_{62}+x_{63}\right) \in  \left[ 22{,}5 ; 23\right).$$
		Ta xác định được $ n=124 $; $ n_m=45$; $ C=5+12+32=49 $; $ u_m= 22{,}5$; $ u_{m+1}=23$.\\
		Trung vị của mẫu số liệu ghép nhóm là $$M_e=22{,}5 +\dfrac{\dfrac{124}{2}-49}{45}\cdot \left( 23-22{,}5\right)= \dfrac{1019}{45}\approx 22{,}64.$$
		Vậy ban tổ chức nên chọn các vận động viên  có thời gian chạy không quá $ 22{,}64$ (giây) được tiếp tục thi vòng hai.
	}
\end{vd}
\subsubsection{Bài tập rèn luyện}
\begin{bt}%[Cánh diều]%[1C5B1-5]
	\immini
	{
		Bảng bên cho ta bảng tần số ghép nhóm số liệu thống kê chiều cao của $40$ mẫu cây ở một vườn thực vật (đơn vị: centimét). Xác định trung vị của mẫu số liệu ghép nhóm trên.
	}
	{
		\begin{tabular}{|c|c|c|}
			\hline
			\textbf{Nhóm} & \textbf{Tần số} & \textbf{Tần số tích luỹ}\\ 
			\hline
			$\left[30;40\right)$ & $4$ & $4$\\
			$\left[40;50\right)$ & $10$ & $14$\\
			$\left[50;60\right)$ & $14$ & $28$\\
			$\left[60;70\right)$ & $6$ & $34$\\
			$\left[70;80\right)$ & $4$ & $38$\\
			$\left[80;90\right)$ & $2$ & $40$\\
			\hline
			& $n = 40$ &\\
			\hline
		\end{tabular}
	}
	\loigiai{
		Ta có $\dfrac{n}{2} = 20$, mà $14<20<28$ nên nhóm $3$ là nhóm đầu tiên có tần số tích luỹ lớn hơn hoặc bằng $20$. \\
		Xét nhóm $3$ là nhóm $\left[50;60\right)$ có $r=50$, $d=10$, $n_3=14$ và nhóm $2$ có $cf_2 = 14$.\\
		Khi đó, tứ phân vị thứ hai (cũng là trung vị) là
		\[
		M_e = 50 + \dfrac{20 - 14}{14} \cdot 10 = 54{,}3\text{ (cm)}.
		\]
	}
\end{bt}
\begin{bt}%[Cánh diều]%[1C5B1-5]
	Mẫu số liệu sau ghi lại cân nặng của $30$ bạn học sinh (đơn vị: kilôgam)
	\[
	\begin{array}{cccccccccc}
		17 & 40 & 39 & 40{,}5 & 42 & 51 & 41{,}5 & 39 & 41 & 30\\
		40 & 42 & 40{,}5 & 39{,}5 & 41 & 40{,}5 & 37 & 39{,}5 & 40 & 41\\
		38{,}5 & 39{,}5 & 40 & 41 & 39 & 40{,}5 & 40 & 38{,}5 & 39{,}5 & 41{,}5
	\end{array}
	\]
	\begin{enumerate}
		\item Lập bảng tần số ghép nhóm cho mẫu số liệu trên có tám nhóm ứng với tám nửa khoảng:
		\[
		[15 ; 20),[20 ; 25),[25 ; 30),[30 ; 35),[35 ; 40),[40 ; 45),[45 ; 50),[50 ; 55).
		\]
		\item Xác định trung vị của mẫu số liệu ghép nhóm trên.
	\end{enumerate}
	\loigiai{
		\begin{enumerate}
			\item Ta có bảng tần số ghép nhóm của mẫu số liệu trên như sau:
			\begin{center}
				\begin{tabular}{|c|c|c|c|}
					\hline
					\textbf{Nhóm} & \textbf{Giá trị đại diện} & \textbf{Tần số} & \textbf{Tần số tích luỹ}\\ 
					\hline
					$\left[15;20\right)$ & $17{,}5$ & $1$ & $1$\\
					$\left[20;25\right)$ & $22{,}5$ & $0$ & $1$\\
					$\left[25;30\right)$ & $27{,}5$ & $0$ & $1$\\
					$\left[30;35\right)$ & $32{,}5$ & $1$ & $2$\\
					$\left[35;40\right)$ & $37{,}5$ & $10$ & $12$\\
					$\left[40;45\right)$ & $42{,}5$ & $17$ & $29$\\
					$\left[45;50\right)$ & $47{,}5$ & $0$ & $29$\\
					$\left[50;55\right)$ & $52{,}5$ & $1$ & $30$\\
					\hline
					&  & $n = 30$ &\\
					\hline
				\end{tabular}
			\end{center}
			\item Ta có $\dfrac{n}{2} = 15$, mà $12<15<29$ nên nhóm $6$ là nhóm đầu tiên có tần số tích luỹ lớn hơn hoặc bằng $15$. \\
			Xét nhóm $6$ là nhóm $\left[40;45\right)$ có $r=40$, $d=5$, $n_6=17$ và nhóm $5$ có $cf_5 = 12$.\\
			Khi đó, trung vị là
			\[
			M_e = 40 + \dfrac{15-12}{17} \cdot 5 = 40{,}9\text{ (kg)}.
			\]
		\end{enumerate}
	}
\end{bt}
\begin{bt}%[CTST]%[1T5G2-2]
	Cân nặng của một số lợn con mới sinh thuộc hai giống $ A $ và $ B $ được cho ở biểu đồ dưới đây (đơn vị: kg).
	\begin{center}
		\begin{tikzpicture}[>=stealth,line join=round,line cap=round,font=\footnotesize,scale=0.85,line width=1pt]
			\draw[->] (0,0)--(0,5)node[left]{(\text{Số con})};
			\foreach \y in {1,2,3,4}
			\draw[shift={(0,\y)}] (0,0)--(-2pt,0) node[left]{\scriptsize ${\y}0$};
			%	\path (4.5,6) node {\normalsize{\textbf{Cân nặng của một số lợn con mới sinh}}};
			\path (4.5,5.5) node {
				$\begin{array}{c}
					\normalsize{\textbf{Cân nặng của một số}}\\
					\normalsize{\textbf{lợn con mới sinh}}
				\end{array}$
			};
			%% nhãn
			\path (2.5,-1.5) node[rectangle,fill=cyan,draw=none]{};
			\path (3.6,-1.5) node {\text{Giống $ A $}};
			\path (5,-1.5) node[rectangle,fill=orange,draw=none]{};
			\path (6.1,-1.5) node {\text{Giống $ B $}};
			% đường gióng
			\foreach \y in {1,2,3,4}{
				\draw[line width=0.2pt] (0,\y)--(8.4,\y);
			}
			%% cột
			\draw[fill=cyan,draw=none] (0,0)--(0,0.8)--(1,0.8)node[midway,above]{$ 8 $}--(1,0)--cycle;
			\draw[fill=orange,draw=none] (1,0)--(1,1.3)--(2,1.3)node[midway,above]{$ 13 $}--(2,0)--cycle;
			\draw[fill=cyan,draw=none] (2,0)--(2,2.8)--(3,2.8)node[midway,above]{$ 28 $}--(3,0)--cycle;
			\draw[fill=orange,draw=none] (3,0)--(3,1.4)--(4,1.4)node[midway,above]{$ 14 $}--(4,0)--cycle;
			\draw[fill=cyan,draw=none] (4,0)--(4,3.2)--(5,3.2)node[midway,above]{$ 32 $}--(5,0)--cycle;
			\draw[fill=orange,draw=none] (5,0)--(5,2.4)--(6,2.4)node[midway,above]{$ 24 $}--(6,0)--cycle;
			\draw[fill=cyan,draw=none] (6,0)--(6,1.7)--(7,1.7)node[midway,above]{$ 17 $}--(7,0)--cycle;
			\draw[fill=orange,draw=none] (7,0)--(7,1.4)--(8,1.4)node[midway,above]{$ 14 $}--(8,0)--cycle;
			%% miền
			\node [below] at (1,0){$ \left[1{,}0 ; 1{,}1 \right)$};
			\node [below] at (3,0){$ \left[1{,}1 ; 1{,}2 \right)$};
			\node [below] at (5,0){$ \left[1{,}2 ; 1{,}3 \right)$};
			\node [below] at (7,0){$ \left[1{,}3 ; 1{,}4 \right)$};
			\draw[->] (0,0)node [below left=-2pt]{$ O $}--(9,0)node[below]{(\text{kg})};
		\end{tikzpicture}
	\end{center}
	Hãy so sánh cân nặng của lợn con mới sinh giống $ A $ và giống $ B $ theo số trung bình và trung vị.
	\loigiai{
		Bảng tần số ghép nhóm thống kê cân nặng của lợn con mới sinh giống $ A $ và giống $ B $ như sau:
		\begin{center}
			\begin{tabular}{|c|c|c|c|c|}
				\hline 
				\textbf{Cân nặng (kg)}	& $ \left[1{,}0 ; 1{,}1 \right)$  &$ \left[1{,}1 ; 1{,}2 \right)$  &$ \left[1{,}2 ; 1{,}3 \right)$  & $ \left[1{,}3 ; 1{,}4 \right)$ \\ 
				\hline 
				\textbf{Giá trị đại diện (kg)}	& $1{,}05 $ & $ 1{,}15 $ & $ 1{,}25 $ & $ 1{,}35 $ \\ 
				\hline 
				\begin{tabular}{c}
					\textbf{Giống A}
					\\ 
					\textbf{(đơn vị: con)}
				\end{tabular} 	& $ 8 $ & $ 28 $ & $ 32 $ & $ 17 $ \\ 
				\hline 
				\begin{tabular}{c}
					\textbf{Giống B}
					\\ 
					\textbf{(đơn vị: con)}
				\end{tabular} 	& $ 13 $ & $ 14 $ & $ 24 $ & $ 14 $ \\ 
				\hline 
			\end{tabular} 
		\end{center}
		Cân nặng trung bình của lợn con mới sinh giống $ A $ là $$ \dfrac{1{,}05\cdot 8 + 1{,}15 \cdot 28 + 1{,}25 \cdot 32 + 1{,}35 \cdot 17}{8+28+32+17}=\dfrac{2071}{1700}\approx 1{,}218.$$
		Cân nặng trung bình của lợn con mới sinh giống $ B $ là $$ \dfrac{1{,}05\cdot 13 + 1{,}15 \cdot 14 + 1{,}25 \cdot 24 + 1{,}35 \cdot 14}{13+14+24+14}=\dfrac{121}{100} \approx 1{,}21.$$
		Suy ra cân nặng trung bình của lợn con mới sinh giống $ A $  lớn hơn cân nặng trung bình của lợn con mới sinh giống $ B $. \\
		Trung vị của mẫu số liệu ghép nhóm cân nặng của lợn con  giống $ A $ là $$M_e=1{,}2+\dfrac{\dfrac{85}{2}-(8+28)}{32}\cdot (1{,}3-1{,}2)=\dfrac{781}{640}\approx 1{,}22.$$
		Trung vị của mẫu số liệu ghép nhóm cân nặng của lợn con  giống  $ B $ là $$M_e=1{,}2+\dfrac{\dfrac{65}{2}-(13+14)}{24}\cdot (1{,}3-1{,}2)=\dfrac{587}{480}\approx 1{,}22.$$
		Suy ra trung vị của mẫu số liệu ghép nhóm cân nặng của của lợn con giống $ A $ bằng trung vị của mẫu số liệu ghép nhóm cân nặng của của lợn con giống $ B $.
	}
\end{bt}

\setcounter{subsubsection}{0}
\setcounter{ex}{0}
\setcounter{bt}{0}
\begin{dang}{Tứ phân vị}
	
\end{dang}
\subsubsection{Ví dụ minh hoạ}
\begin{vd}
	\immini
	{
		Bảng bên cho biết tần số ghép nhóm số liệu thống kê cân nặng của $40$ học sinh lớp $11A$ trong một trường trung học phổ thông (đơn vị: kilôgam). Xác định tứ phân vị của mẫu số liệu ghép nhóm.
	}
	{
		\begin{tabular}{|c|c|c|}
			\hline
			\textbf{Nhóm} & \textbf{Tần số} & \textbf{Tần số tích luỹ}\\ 
			\hline
			$\left[30;40\right)$ & $2$ & $2$\\
			$\left[40;50\right)$ & $10$ & $12$\\
			$\left[50;60\right)$ & $16$ & $28$\\
			$\left[60;70\right)$ & $8$ & $36$\\
			$\left[70;80\right)$ & $2$ & $38$\\
			$\left[80;90\right)$ & $2$ & $40$\\
			\hline
			& $n = 40$ &\\
			\hline
		\end{tabular}
	}
	\loigiai{
		Số phần tử của mẫu là $n=40$.
		\begin{itemize}
			\item Ta có $\dfrac{n}{4} = 10$, mà $2<10<12$ nên nhóm $2$ là nhóm đầu tiên có tần số tích luỹ lớn hơn hoặc bằng $10$. \\
			Xét nhóm $2$ là nhóm $\left[40;50\right)$ có $s=40$, $h=10$, $n_2=10$ và nhóm $1$ có $cf_1 = 2$.\\
			Khi đó, tứ phân vị thứ nhất là
			\[
			Q_1 = 40 + \dfrac{10-2}{10} \cdot 10 = 48\text{ (kg)}.
			\]
			\item Ta có $\dfrac{n}{2} = 20$, mà $12<20<28$ nên nhóm $3$ là nhóm đầu tiên có tần số tích luỹ lớn hơn hoặc bằng $20$. \\
			Xét nhóm $3$ là nhóm $\left[50;60\right)$ có $r=50$, $d=10$, $n_3=16$ và nhóm $2$ có $cf_2 = 12$.\\
			Khi đó, tứ phân vị thứ hai là
			\[
			Q_2 = 50 + \dfrac{20-12}{16} \cdot 10 = 55\text{ (kg)}.
			\]
			\item Ta có $\dfrac{3n}{4} = 30$, mà $28<30<36$ nên nhóm $4$ là nhóm đầu tiên có tần số tích luỹ lớn hơn hoặc bằng $30$. \\
			Xét nhóm $4$ là nhóm $\left[60;70\right)$ có $t=50$, $l=10$, $n_4=8$ và nhóm $3$ có $cf_3 = 28$.\\
			Khi đó, tứ phân vị thứ ba là
			\[
			Q_3 = 60 + \dfrac{30-28}{8} \cdot 10 = 62{,}5\text{ (kg)}.
			\]
		\end{itemize}
		Vậy tứ phân vị của mẫu số liệu trên là $48$, $55$ và $62{,}5$.
	}
\end{vd}
\subsubsection{Bài tập rèn luyện}
\begin{bt}%[1K3B9-3]
	Thời gian (phút) truy cập internet mỗi buổi tối của một số học sinh được cho trong bảng sau:
	\begin{center}
		\begin{tabular}{|c|c|c|c|c|c|c|}
			\hline
			Thời gian (phút)	& $\left[9{,}5;12{,}5 \right)$ & $\left[12{,}5;15{,}5 \right)$ & $\left[15{,}5;18{,}5 \right)$ & $\left[18{,}5;21{,}5 \right)$ & $\left[21{,}5;24{,}5 \right)$ \\
			\hline
			Số học sinh&$3$	& $12$ & $15$ &$24$  & $2$  \\
			\hline
		\end{tabular}	
	\end{center}
	Tìm tứ phân vị thứ nhất $Q_1$ và tứ phân vị thứ ba $Q_3$ của mẫu số liệu ghép nhóm.
	\loigiai{
		Cỡ mẫu là $n=3+12+15+24+2=56$.\\
		Tứ phân vị thứ nhất $Q_1$ là $\dfrac{x_{14}+x_{15}}{2}$. Do $2$ giá trị $x_{28},\,x_{29}$ thuộc nhóm $\left[12{,}5;15{,}5 \right)$ nên nhóm này chứa $Q_1$. Do đó, $p=2$; $a_2=12{,}5$; $m_2=12$; $m_1=3$; $a_3-a_2=3$ và ta có $$Q_1=12{,}5+\dfrac{\dfrac{56}{4}-3}{12}\cdot 3=15{,}25.$$
		Với tứ phân vị thứ ba $Q_3$ là $\dfrac{x_{42}+x_{43}}{2}$. Do $2$ giá trị $x_{42},\,x_{43}$ thuộc nhóm $\left[18{,}5;21{,}5 \right)$ nên nhóm này chứa $Q_3$. Do đó, $p=4$; $a_4=18{,}5$; $m_4=24$; $m_1+m_2+m_3=3+12+15=30$; $a_5-a_4=3$ và ta có $$Q_3=18{,}5+\dfrac{\dfrac{3\cdot 56}{4}-30}{24}\cdot 3=20.$$
	}
\end{bt}
\begin{bt}%[1K3B9-4]
	Điểm thi môn Toán (thang điểm 100, điểm được làm tròn đến 1) của 60 thí sinh được cho trong bảng sau:
	\begin{center}
		\begin{tabular}{|c|c|c|c|c|c|}
			\hline
			Điểm		& $0-9$ & $10-19$ & $20-29$ & $30-39$ & $40-49$ \\
			\hline
			Số thí sinh	& $1$ & $2$ & $4$ & $6$ & $15$ \\
			\hline
			Điểm	& $50-59$ & $60-69$ & $70-79$ & $80-89$ & $90-99$  \\
			\hline
			Số thí sinh	& $12$ & $10$ & $6$ & $3$ & $1$  \\
			\hline
		\end{tabular}
	\end{center}
	\begin{enumerate}
		\item [a)] Hiệu chỉnh để thu được mẫu số liệu ghép nhóm dạng Bảng $3.2$.
		\item [b)] Tìm các tứ phân vị và giải thích ý nghĩa của chúng.
	\end{enumerate}
	\loigiai{
		\begin{enumerate}
			\item [a)] Bảng số liệu ghép nhóm về điểm thi môn Toán của 60 thí sinh
			\begin{center}
				\begin{tabular}{|c|c|c|c|c|c|}
					\hline
					Điểm		& $\left[0;20\right)$ & $\left[20;40\right)$ & $\left[40;60\right)$ & $\left[60;80\right)$ & $\left[80;100\right)$ \\
					\hline
					Số thí sinh	& $3$ & $10$ & $27$ & $16$ & $4$ \\
					\hline
				\end{tabular}
			\end{center}
			
			\item [b)] Cỡ mẫu $n=60$. Gọi $x_1$, $x_2$,$\ldots$, $x_{60}$ là điểm thi môn Toán của 60 học sinh và giả sử dãy này đã được sắp xếp theo thứ tự tăng dần. Khi đó, trung vị là $\dfrac{x_{30}+x_{31}}{2}$.\\
			Do hai giá trị $x_{30}$, $x_{31}$ thuộc nhóm $\left[40;60\right)$ nên nhóm này chứa trung vị. Do đó, $p=3;a_3=40;m_3=27;m_1+m_2=13;a_4-a_3=20$ và ta có
			$$Q_2=M_e=40+\dfrac{\dfrac{60}{2}-13}{27}\cdot 20\approx 52{,}6.$$
			Tứ phân vị thứ nhất $Q_1=\dfrac{x_{15}+x_{16}}{2}$. Do hai giá trị $x_{15}$, $x_{16}$ thuộc nhóm $\left[40;60\right)$ nên nhóm này chứa $Q_1$. Do đó, $p=3;\,a_3=40;\,m_3=27;\,m_1+m_2=13;\,a_4-a_3=20$ và ta có
			$$Q_1=40+\dfrac{\dfrac{60}{4}-13}{27}\cdot 20\approx 41{,}5.$$
			Tứ phân vị thứ ba $Q_3=\dfrac{x_{45}+x_{46}}{2}$. Do hai giá trị $x_{45}$, $x_{46}$ thuộc nhóm $\left[60;80\right)$ nên nhóm này chứa $Q_3$. Do đó, $p=4;\,a_4=60;\,m_4=16;\,m_1+m_2+m_3=40;\,a_5-a_4=20$ và ta có
			$$Q_3=60+\dfrac{\dfrac{3\cdot 60}{4}-40}{16}\cdot 20\approx 66{,}3.$$
			Khoảng cách từ $Q_1$ đến $Q_2$ là $11{,}1$ còn khoảng cách từ $Q_2$ và $Q_3$ là $13{,}7$. Điều này cho thấy mẫu số liệu tập trung với mật độ cao hơn ở bên trái $Q_2$ và mật độ thấp hơn ở bên phải $Q_2$.
		\end{enumerate}	
	}
\end{bt}
\begin{bt}%[1K3B9-4]
	\immini{
		Phỏng vấn một số học sinh lớp $11$ về thời gian (giờ) ngủ của một buổi tối, thu được bảng số liệu ở bên.
		\begin{enumerate}
			\item [a)] So sánh thời gian ngủ trung bình của các bạn học sinh nam và nữ.
			\item [b)] Hãy cho biết $75\%$ học sinh khối $11$ ngủ ít nhất bao nhiêu giờ?
		\end{enumerate}
	}
	{
		\begin{tabular}{|c|c|c|}
			\hline
			Thời gian	& Số học sinh nam & Số học sinh nữ\\
			\hline
			$\left[4;5 \right)$	& $6$ & $4$ \\
			\hline
			$\left[5;6 \right)$	& $10$ & $8$ \\
			\hline
			$\left[6;7 \right)$	& $13$ & $10$ \\
			\hline
			$\left[7;8 \right)$	& $9$ & $11$ \\
			\hline
			$\left[8;9 \right)$	& $7$ & $8$ \\
			\hline
		\end{tabular}
	}
	\loigiai{
		\begin{enumerate}
			\item [a)] Trong mỗi khoảng thời gian, giá trị đại diện là trung bình cộng của giá trị hai đầu mút nên ta có bảng sau:
			\begin{center}
				\begin{tabular}{|c|c|c|}
					\hline
					Thời gian	& Số học sinh nam & Số học sinh nữ\\
					\hline
					$4{,}5$	& $6$ & $4$ \\
					\hline
					$5{,}5$	& $10$ & $8$ \\
					\hline
					$6{,}5$	& $13$ & $10$ \\
					\hline
					$7{,}5$	& $9$ & $11$ \\
					\hline
					$8{,}5$	& $7$ & $8$ \\
					\hline
				\end{tabular}	
			\end{center}
			Tổng số học sinh nam là $n_1=6+10+13+9+7=45$.\\ Thời gian ngủ trung bình của học sinh nam là:
			$$\overline{x_1}=\dfrac{4{,}5\cdot 6+5{,}5\cdot10+6{,}5\cdot13+7{,}5\cdot9+8{,}5\cdot7}{45}=\dfrac{587}{90}\approx 6{,}52\,\, \text{(giờ)}.$$
			Tổng số học sinh nữ là $n_2=4+8+10+11+8=41$. Thời gian ngủ trung bình của học sinh nữ là:
			$$\overline{x_2}=\dfrac{4,5\cdot4+5,5\cdot8+6,5\cdot10+7,5\cdot11+8,5\cdot8}{41}=\dfrac{555}{82}\approx 6{,}77 \,\,\text{(giờ)}.$$
			Vì $\overline{x_2}>\overline{x_1}$ nên thời gian ngủ trung bình của các bạn học sinh nữ lớn hơn thời gian ngủ trung bình của các bạn nam.
			\item [b)] Tổng số học sinh được điều tra là $n=n_1+n_2=45+41=86$.\\
			Giả sử $x_1;x_2;x_3;\cdot \cdot;x_{86}$ là dãy giá trị được sắp xếp theo thứ tự không giảm.\\
			Ta có bảng sau:
			\begin{center}
				\begin{tabular}{|c|c|c|}
					\hline
					Thời gian	& Số học sinh \\
					\hline
					$\left[4;5 \right)$	& $10$  \\
					\hline
					$\left[5;6 \right)$	& $18$  \\
					\hline
					$\left[6;7 \right)$	& $23$  \\
					\hline
					$\left[7;8 \right)$	& $20$  \\
					\hline
					$\left[8;9 \right)$	& $15$  \\
					\hline
				\end{tabular}
			\end{center}
			Tứ phân vị thứ nhất $Q_1$ là $x_{22}$. Do $x_{22}$ thuộc nhóm $\left[5;6\right)$ nên nhóm này chứa $Q_1$.\\ Do đó, $p=2;\,a_2=5;\,m_2=18;\,m_1=10;\,a_3-a_2=1$ và ta có
			$$Q_1=5+\dfrac{\dfrac{86}{4}-10}{18}\cdot 1=\dfrac{203}{36}\approx 5{,}64 \text{(giờ)}.$$
			Nghĩa là có $25\%$ học sinh khối $11$ ngủ ít hơn $5{,}64$ giờ.\\ Vậy $75\%$ học sinh khối $11$ ngủ ít nhất $5{,}64$ giờ.
		\end{enumerate}
	}
\end{bt}
\setcounter{subsubsection}{0}
\setcounter{ex}{0}
\setcounter{bt}{0}
\begin{dang}{Mốt}
	
\end{dang}
\subsubsection{Ví dụ minh hoạ}
\begin{vd}%[Tex hóa SGK CD, TVN-223]%[1C5B1-5]
	Kết quả kiểm tra môn Toán của lớp $11D$ như sau
	\[
	\begin{array}{cccccccccccccccccccc}
	5 & 6 & 7 & 5 & 6 & 9 & 10 & 8 & 5 & 5 & 4 & 5 & 4 & 5 & 7 & 4 & 5 & 8 & 9 & 10 \\
	5 & 3 & 5 & 6 & 5 & 7 & 5 & 8 & 4 & 9 & 5 & 6 & 5 & 6 & 8 & 8 & 7 & 9 & 7 & 9
	\end{array}
	\]
	\begin{enumerate}
		\item Lập bảng tần số ghép nhóm của mẫu số liệu trên có bốn nhóm ứng với bốn nửa khoảng $\left[3;5\right)$, $\left[5;7\right)$, $\left[7;9\right)$, $\left[9;11\right)$.
		\item Mốt của bảng số liệu ghép nhóm trên là bao nhiêu (làm tròn kết quả đến hàng phần mười)?
	\end{enumerate}
	\loigiai{
		\immini
		{
			\begin{enumerate}
				\item Bảng bên là bảng tần số ghép nhóm cho kết quả kiểm tra môn Toán của lớp $11D$.
				\item Ta thấy: Nhóm $2$ ứng với nửa khoảng $\left[5;7\right)$ là nhóm có tần số lớn nhất với $u=5$, $g=2$, $n_2 = 18$. Nhóm $1$ có tần số $n_1 = 5$, nhóm $3$ có tần số $n_3=10$.\\
				Khi đó, mốt của mẫu số liệu là 
				\[
				M_o = 5 + \left( \dfrac{18- 5}{2\cdot 18 - 5 - 10} \right) \cdot 2 \approx 6{,}2.
				\]
			\end{enumerate}
		}
		{
			\begin{tabular}{|c|c|}
				\hline
				\textbf{Nhóm} & \textbf{Tần số}\\ 
				\hline
				$\left[3;5\right)$ & $5$\\
				\hline
				$\left[5;7\right)$ & $18$\\
				\hline
				$\left[7;9\right)$ & $10$\\
				\hline
				$\left[9;11\right)$ & $7$\\
				\hline
				& $n = 40$ \\
				\hline
			\end{tabular}
		}
	}
\end{vd}
\subsubsection{Bài tập rèn luyện}
\begin{bt}%[Tex hóa SGK CD, TVN-223]%[1C5B1-5]
	Mẫu số liệu dưới đây ghi lại tốc độ của $40$ ô tô khi đi qua một trạm đo tốc độ (đơn vị: km/h):
	\[
	\begin{array}{cccccccccc}
	48{,}5 & 43 & 50 & 55 & 45 & 60 & 53 & 55,5 & 44 & 65 \\
	51 & 62,5 & 41 & 44,5 & 57 & 57 & 68 & 49 & 46{,}5 & 53{,}5 \\
	61 & 49{,}5 & 54 & 62 & 59 & 56 & 47 & 50 & 60 & 61 \\
	49{,}5 & 52{,}5 & 57 & 47 & 60 & 55 & 45 & 47,5 & 48 & 61{,}5
	\end{array}
	\]
	\begin{enumerate}
		\item Lập bảng tần số ghép nhóm cho mẫu số liệu trên có sáu nhóm ứng với sáu nửa khoảng:
		\[
		[40 ; 45),[45 ; 50),[50 ; 55),[55 ; 60),[60 ; 65),[65 ; 70).
		\]
		\item Mốt của mẫu số liệu ghép nhóm trên là bao nhiêu?
	\end{enumerate}
	\loigiai{
		\begin{enumerate}
			\item Ta có bảng tần số ghép nhóm của mẫu số liệu trên như sau:
			\begin{center}
				\begin{tabular}{|c|c|c|c|}
					\hline
					\textbf{Nhóm} & \textbf{Giá trị đại diện} & \textbf{Tần số} & \textbf{Tần số tích luỹ}\\ 
					\hline
					$\left[40;45\right)$ & $42{,}5$ & $4$ & $4$\\
					$\left[45;50\right)$ & $47{,}5$ & $11$ & $15$\\
					$\left[50;55\right)$ & $52{,}5$ & $7$ & $22$\\
					$\left[55;60\right)$ & $57{,}5$ & $8$ & $30$\\
					$\left[60;65\right)$ & $62{,}5$ & $8$ & $38$\\
					$\left[65;70\right)$ & $67{,}5$ & $2$ & $40$\\
					\hline
					&  & $n = 40$ &\\
					\hline
				\end{tabular}
			\end{center}
			
			\item Ta thấy: Nhóm $2$ ứng với nửa khoảng $\left[45;50\right)$ là nhóm có tần số lớn nhất với $u=45$, $g=5$, $n_2 = 11$. Nhóm $1$ có tần số $n_1 = 4$, nhóm $3$ có tần số $n_3 = 7$.\\
			Khi đó, mốt của mẫu số liệu là 
			\[
			M_o = 45 + \left( \dfrac{11 - 4}{2\cdot 11 - 4 - 7} \right) \cdot 5 \approx 48{,}2\text{ (km/h)}.
			\]
		\end{enumerate}
	}
\end{bt}
\begin{bt}%[Tex hóa SGK CD, TVN-223]%[1C5B1-5]
	Mẫu số liệu sau ghi lại cân nặng của $30$ bạn học sinh (đơn vị: kilôgam):
	\[
	\begin{array}{cccccccccc}
	17 & 40 & 39 & 40{,}5 & 42 & 51 & 41{,}5 & 39 & 41 & 30\\
	40 & 42 & 40{,}5 & 39{,}5 & 41 & 40{,}5 & 37 & 39{,}5 & 40 & 41\\
	38{,}5 & 39{,}5 & 40 & 41 & 39 & 40{,}5 & 40 & 38{,}5 & 39{,}5 & 41{,}5
	\end{array}
	\]
	\begin{enumerate}
		\item Lập bảng tần số ghép nhóm cho mẫu số liệu trên có tám nhóm ứng với tám nửa khoảng:
		\[
		[15 ; 20),[20 ; 25),[25 ; 30),[30 ; 35),[35 ; 40),[40 ; 45),[45 ; 50),[50 ; 55).
		\]
		
		\item Mốt của mẫu số liệu ghép nhóm trên là bao nhiêu?
	\end{enumerate}
	\loigiai{
		\begin{enumerate}
			\item Ta có bảng tần số ghép nhóm của mẫu số liệu trên như sau:
			\begin{center}
				\begin{tabular}{|c|c|c|c|}
					\hline
					\textbf{Nhóm} & \textbf{Giá trị đại diện} & \textbf{Tần số} & \textbf{Tần số tích luỹ}\\ 
					\hline
					$\left[15;20\right)$ & $17{,}5$ & $1$ & $1$\\
					$\left[20;25\right)$ & $22{,}5$ & $0$ & $1$\\
					$\left[25;30\right)$ & $27{,}5$ & $0$ & $1$\\
					$\left[30;35\right)$ & $32{,}5$ & $1$ & $2$\\
					$\left[35;40\right)$ & $37{,}5$ & $10$ & $12$\\
					$\left[40;45\right)$ & $42{,}5$ & $17$ & $29$\\
					$\left[45;50\right)$ & $47{,}5$ & $0$ & $29$\\
					$\left[50;55\right)$ & $52{,}5$ & $1$ & $30$\\
					\hline
					&  & $n = 30$ &\\
					\hline
				\end{tabular}
			\end{center}
		
			\item Ta thấy: Nhóm $6$ ứng với nửa khoảng $\left[40;45\right)$ là nhóm có tần số lớn nhất với $u=40$, $g=5$, $n_6 = 17$. Nhóm $5$ có tần số $n_5 = 10$, nhóm $7$ có tần số $n_7 = 0$.\\
			Khi đó, mốt của mẫu số liệu là 
			\[
			M_o = 40 + \left( \dfrac{17 - 10}{2\cdot 17 - 10 - 0} \right) \cdot 5 \approx 41{,}5\text{ (kg)}.
			\]
		\end{enumerate}
	}
\end{bt}
\begin{bt}%[Tex hóa SGK CD, TVN-223]%[1C5B1-5]
	\immini
	{
		Bảng bên cho ta bảng tần số ghép nhóm số liệu thống kê chiều cao của $40$ mẫu cây ở một vườn thực vật (đơn vị: centimét).
		
		
			 Mốt của mẫu số liệu ghép nhóm trên là bao nhiêu?
		
	}
	{
		\begin{tabular}{|c|c|c|}
			\hline
			\textbf{Nhóm} & \textbf{Tần số} & \textbf{Tần số tích luỹ}\\ 
			\hline
			$\left[30;40\right)$ & $4$ & $4$\\
			$\left[40;50\right)$ & $10$ & $14$\\
			$\left[50;60\right)$ & $14$ & $28$\\
			$\left[60;70\right)$ & $6$ & $34$\\
			$\left[70;80\right)$ & $4$ & $38$\\
			$\left[80;90\right)$ & $2$ & $40$\\
			\hline
			& $n = 40$ &\\
			\hline
		\end{tabular}
	}
	\loigiai{
	Ta thấy: Nhóm $3$ ứng với nửa khoảng $\left[50;60\right)$ là nhóm có tần số lớn nhất với $u=50$, $g=10$, $n_3 = 14$. Nhóm $2$ có tần số $n_2 = 10$, nhóm $4$ có tần số $n_4 = 6$.\\
			Khi đó, mốt của mẫu số liệu là 
			\[
			M_o = 50 + \left( \dfrac{14 - 10}{2\cdot 14 - 10 - 6} \right) \cdot 10 \approx 53{,}3\text{ (cm)}.
			\]
	
	}
\end{bt}
\subsection{Bài tập trắc nghiệm}
\Opensolutionfile{ans}[ans/ansOC3]
% \begin{ex}%[1C5Y1-1]
% 	Một cuộc khảo sát đã tiến hành xác định tuổi (theo năm) của $120$ chiếc ô-tô. Kết quả điều tra được cho trong bảng sau
% 	\begin{center}
% 		\begin{tabular}{ |c|c|c|c|c|c|c| }
% 			\hline
% 			Nhóm & $[0;4)$ & $[4;8)$ & $[8;12)$ & $[12;16)$ & $[16;20)$ &  \\
% 			\hline
% 			Tần số & $23$ & $25$ & $27$ & $26$ & $19$ & $n=120$ \\
% 			\hline
% 		\end{tabular}
% 	\end{center}
% 	Mẫu số liệu trên có bao nhiêu nhóm?
% 	\choice
% 	{$10$}
% 	{$11$}
% 	{\True $5$}
% 	{$7$}
% 	\loigiai{
% 		Từ bảng, ta thấy mẫu số liệu trên có $5$ nhóm.
% 	}
% \end{ex}
% \begin{ex}%[1C5Y1-1]
% 	Một cuộc khảo sát đã tiến hành xác định tuổi (theo năm) của $120$ chiếc ô-tô. Kết quả điều tra được cho trong bảng sau
% 	\begin{center}
% 		\begin{tabular}{ |c|c|c|c|c|c|c| }
% 			\hline
% 			Nhóm & $[0;4)$ & $[4;8)$ & $[8;12)$ & $[12;16)$ & $[16;20)$ &  \\
% 			\hline
% 			Tần số & $23$ & $25$ & $27$ & $26$ & $19$ & $n=120$ \\
% 			\hline
% 		\end{tabular}
% 	\end{center}
% 	Nhóm có tần số bằng $19$ là
% 	\choice
% 	{$[0;4)$}
% 	{$[8;12)$}
% 	{$[12;16)$}
% 	{\True $[16;20)$}
% 	\loigiai{
% 		Từ bảng, ta thấy nhóm có tần số bằng $19$ là $[16;20)$.
% 	}
% \end{ex}
\begin{ex}%[1C5Y1-1]
	Một cuộc khảo sát đã tiến hành xác định tuổi (theo năm) của $120$ chiếc ô-tô. Kết quả điều tra được cho trong bảng sau
	\begin{center}
		\begin{tabular}{ |c|c|c|c|c|c|c| }
			\hline
			Nhóm & $[0;4)$ & $[4;8)$ & $[8;12)$ & $[12;16)$ & $[16;20)$ &  \\
			\hline
			Tần số & $23$ & $25$ & $27$ & $26$ & $19$ & $n=120$ \\
			\hline
		\end{tabular}
	\end{center}
	Số ô-tô có độ tuổi dưới $12$ là
	\choice
	{\True $75$}
	{$27$}
	{$48$}
	{$26$}
	\loigiai{
		Từ bảng, ta thấy số ô-tô có độ tuổi dưới $12$ là $23+25+27=75$.
	}
\end{ex}
% \begin{ex}%[1C5Y1-1]
% 	Cho mẫu số liệu ghép nhóm sau
% 	\begin{center}
% 		\begin{tabular}{ |c|c|c|c|c|c|c|c| }
% 			\hline
% 			Thời gian & $[15;20)$ & $[20;25)$ & $[25;30)$ & $[30;35)$ & $[35;40)$ & $[40;45)$ & $[45;50)$ \\
% 			\hline
% 			Số nhân viên & $6$ & $14$ & $25$ & $37$ & $21$ & $13$ & $9$ \\
% 			\hline
% 		\end{tabular}
% 	\end{center}
% 	Tần số của nhóm $[15;20)$ là bao nhiêu?
% 	\choice
% 	{\True $6$}
% 	{$7$}
% 	{$14$}
% 	{$25$}
% 	\loigiai{
% 		Ta thấy tần số của nhóm $[15;20)$ là $6$.
% 	}
% \end{ex}
\begin{ex}%[1C5Y1-1]
	Khảo sát thời gian tập thể dục trong ngày của một số học sinh khối $11$ thu được mẫu số liệu ghép nhóm sau
	\begin{center}
		\begin{tabular}{ |c|c|c|c|c|c| }
			\hline
			Thời gian (phút)& $[0;20)$ & $[20;40)$ & $[40;60)$ & $[60;80)$ & $[80;100)$ \\
			\hline
			Số học sinh & $5$ & $9$ & $12$ & $10$ & $6$ \\
			\hline
		\end{tabular}
	\end{center}
	Giá trị đại diện của nhóm $[20;40)$ là
	\choice
	{$10$}
	{\True $30$}
	{$20$}
	{$40$}
	\loigiai{
		Giá trị đại diện của nhóm $[20;40)$ là $\dfrac{20+40}{2}=30$.
	}
\end{ex}
\begin{ex}%[1C5Y1-1]
	Doanh thu bán hàng trong $20$ ngày được lựa chọn ngẫu nhiên của một cửa hàng được ghi lại ở bảng sau (đơn vị: triệu đồng):
	\begin{center}
		\begin{tabular}{ |c|c|c|c|c|c| }
			\hline
			Doanh thu & $[5;7)$ & $[7;9)$ & $[9;11)$ & $[11;13)$ & $[13;15)$ \\
			\hline
			Số ngày & $2$ & $7$ & $7$ & $3$ & $1$ \\
			\hline
		\end{tabular}
	\end{center}
	Doanh thu bán hàng của cửa hàng trong ngày $A$ là $7$ triệu đồng thì được xếp vào nhóm nào?
	\choice
	{$[5;7)$}
	{\True $[7;9)$}
	{$[9;11)$}
	{$[13;15)$}
	\loigiai{
		Doanh thu bán hàng là $7$ triệu đồng thì được xếp vào nhóm $[7;9)$.
	}
\end{ex}
\begin{ex}%[1C5Y1-1]
	Doanh thu bán hàng trong $20$ ngày được lựa chọn ngẫu nhiên của một cửa hàng được ghi lại ở bảng sau (đơn vị: triệu đồng):
	\begin{center}
		\begin{tabular}{ |c|c|c|c|c|c| }
			\hline
			Doanh thu & $[5;7)$ & $[7;9)$ & $[9;11)$ & $[11;13)$ & $[13;15)$ \\
			\hline
			Số ngày & $2$ & $7$ & $7$ & $3$ & $1$ \\
			\hline
		\end{tabular}
	\end{center}
	Các nhóm có độ dài bằng
	\choice
	{\True $2$}
	{$3$}
	{$4$}
	{$5$}
	\loigiai{
		Các nhóm có độ dài bằng nhau, và bằng $2$.
	}
\end{ex}
\begin{ex}%[1C5B1-1]
	Cho bảng số liệu về khối lượng của $30$ củ khoai tây thu hoạch từ một thửa ruộng như hình bên dưới. Tần suất của lớp $[100;110)$ là bao nhiêu?
	\begin{center}
		\begin{tabular}{ |c|c|c|c|c|c| }
			\hline
			Lớp khối lượng (gam) & $[70;80)$ & $[80;90)$ & $[90;100)$ & $[100;110)$ & $[110;120]$ \\
			\hline
			Tần số & $3$ & $6$ & $12$ & $6$ & $3$ \\
			\hline
		\end{tabular}
	\end{center}
	\choice
	{\True $20\%$}
	{$10\%$}
	{$40\%$}
	{$90\%$}
	\loigiai{
		Tần suất ghép lớp $[100;110)$ là $\dfrac{6}{30}\cdot 100\%=20\%$.
	}
\end{ex}
\begin{ex}%[1C5B1-1]
	Cân nặng của $28$ học sinh nam lớp $11$ được cho ở bảng sau
	\begin{center}
		\begin{tabular}{ |c|c|c|c|c|c|c| }
			\hline
			Cân nặng & $[45;49)$ & $[49;53)$ & $[53;57)$ & $[57;61)$ & $[61;65)$\\
			\hline
			Số học sinh & $4$ & $5$ & $7$ & $7$ & $5$ \\
			\hline
		\end{tabular}
	\end{center}
	Tấn số tích lũy của nhóm $[49;53)$ là bao nhiêu?
	\choice
	{$5$}
	{$4$}
	{\True $9$}
	{$20$}
	\loigiai{
		Tần số tích lũy của nhóm $[49;53)$ là $4+5=9$.
	}
\end{ex}
% \begin{ex}%[1C5K1-1]
% 	Một thư viện thống kê sô người đến đọc sách vào buổi tối trong $30$ ngày của tháng vừa qua như sau
% 	\begin{center}
% 		\begin{tabular}{ cccccccccc }
% 			$26$ & $35$ & $68$ & $84$ & $33$ & $84$ & $62$ & $45$ & $57$ & $46$ \\
% 			$35$ & $29$ & $28$ & $50$ & $26$ & $34$ & $75$ & $74$ & $43$ & $49$ \\
% 			$54$ & $55$ & $83$ & $82$ & $81$ & $54$ & $27$ & $36$ & $41$ & $52$ \\
% 		\end{tabular}
% 	\end{center}
% 	Bạn An lập bảng tần số mẫu số liệu trên như sau
% 	\begin{center}
% 		\begin{tabular}{ |c|c|c|c|c|c|c|c| }
% 			\hline
% 			Nhóm & $[25;35)$ & $[35;45)$ & $[45;55)$ & $[55;65)$ & $[65;75]$ & $[75;85)$ & \\
% 			\hline
% 			Tần số & $7$ & $4$ & $7$ & $3$ & $3$ & $6$ & $n=30$ \\
% 			\hline
% 		\end{tabular}
% 	\end{center}
% 	Bạn Khuê lập bảng tần số mẫu số liệu trên như sau
% 	\begin{center}
% 		\begin{tabular}{ |c|c|c|c|c|c|c|c|c| }
% 			\hline
% 			Nhóm & $[23;31)$ & $[31;39)$ & $[39;47)$ & $[47;55)$ & $[55;63]$ & $[71;79)$ & $[79;87)$ & \\
% 			\hline
% 			Tần số & $5$ & $5$ & $4$ & $5$ & $3$ & $1$ & $2$ & $n=30$ \\
% 			\hline
% 		\end{tabular}
% 	\end{center}
% 	Hỏi bảng tần số của bạn nào đúng?
% 	\choice
% 	{Bảng tần số của bạn An}
% 	{\True Bảng tần số của bạn Khuê}
% 	{Cả hai bạn đều đúng}
% 	{Cả hai bạn đều sai}
% 	\loigiai{
% 		Từ các bảng tần số mẫu số liệu trên, ta thấy bảng tần số của bạn An sai ở hai nhóm $[35;45)$ (tần số đúng bằng $5$) và nhóm $[65;75)$ (tần số đúng bằng $2$).
% 	}
% \end{ex}
% \begin{ex}%[0D5Y3-1]
% 	Cho dãy số liệu thống kê: $21$, $23$, $24$, $25$, $22$, $20$. Số trung bình cộng của các số liệu thống kê đã cho là
% 	\choice
% 	{$23{,}5$}
% 	{$22$}
% 	{\True $22{,}5$}
% 	{$14$}
% 	\loigiai{
% 		Ta có $\overline{x}=\dfrac{21+23+24+25+22+20}{6}=22{,}5$.
% 	}
% \end{ex}
% \begin{ex}%[0D5Y3-1]
% 	Điều tra về số con của $30$ gia đình ở khu vực, kết quả thu được như sau
% 	\begin{center}
% 		\begin{tabular}{|c|c|c|c|c|c|c|}
% 			\hline 
% 			Giá trị (số con) & $0$ &$1$ & $2$ &$3$ & $4$ & Tổng \\ 
% 			\hline 
% 			Tần số & $1$ & $7$ & $15$ & $5$ & $2$ & $N=30$ \\ 
% 			\hline 
% 		\end{tabular}
% 	\end{center}
% 	Tìm số trung bình $\overline{x}$ của mẫu số liệu trên.
% 	\choice
% 	{\True $\overline{x}=2$}
% 	{$\overline{x}=1$}
% 	{$\overline{x}=1{,}5$}
% 	{$\overline{x}=3$}
% 	\loigiai{
% 		Ta có $\overline{x}=\dfrac{0\cdot 1+1\cdot 7+2\cdot 15+3\cdot 5+4\cdot 2}{30}=2$.
% 	}
% \end{ex}
% \begin{ex}%[0D5Y3-1]
% 	Điểm môn Toán của lớp $11A$ được cho trong bảng sau
% 	\begin{center}
% 		\begin{tabular}{|l|c|c|c|c|c|c|c|c|c|c|}
% 			\hline
% 			Điểm&$1$&$2$&$3$&$4$&$5$&$6$&$7$&$8$&$9$&$10$\\
% 			\hline
% 			Tần số&$2$&$1$&$4$&$3$&$9$&$7$&$5$&$5$&$3$&$1$\\
% 			\hline
% 		\end{tabular}
% 	\end{center}
% 	Điểm trung bình của các học sinh lớp $10A$ là bao nhiêu?
% 	\choice
% 	{$5$}
% 	{$5{,}5$}
% 	{$5{,}6$}
% 	{\True $5{,}7$}
% 	\loigiai{
% 		Điểm trung bình lớp $11A$ là
% 		$$\overline{x}= \dfrac{1 \cdot 2 + 2 \cdot 1 + \cdots + 10 \cdot 1}{40}= \dfrac{227}{40} \approx 5{,}7.$$
% 	}
% \end{ex}
% \begin{ex}%[0D5Y3-1]
% 	Kết quả điểm kiểm tra môn Toán của $40$ học sinh lớp $10A$ được trình bày ở bảng sau:
% 	\begin{center}
% 		\begin{tabular}{|c|c|c|c|c|c|c|c|c|}
% 			\hline 
% 			Điểm&$4$  &$5$  &$6$  &$7$  &$8$  &$9$  &$10$  &Cộng  \\ 
% 			\hline 
% 			Tần số&$2$  &$8$  &$7$  &$10$  &$8$  &$3$  &$2$  &$40$  \\ 
% 			\hline 
% 		\end{tabular} 
% 	\end{center}
% 	Tính số trung bình cộng của bảng trên. (làm tròn kết quả đến một chữ số thập phân).
% 	\choice
% 	{\True $6{,}8$}
% 	{$6{,}4$}
% 	{$7{,}0$}
% 	{$6{,}7$}
% 	\loigiai{
% 		Ta có $\overline{x}=\dfrac{4\cdot 2+ 5\cdot 8+ 6\cdot 7+ 7\cdot 10+ 8\cdot 8+ 9\cdot 3+ 10\cdot 2}{40}\approx 6{,}8$.
% 	}
% \end{ex}
\begin{ex}%[0D5B3-1]
	Điểm môn Toán của lớp $10$A được cho như bảng sau
	\begin{center}
		\begin{tabular}{|c|c|c|c|c|c|}
			\hline
			Điểm &$[0;2)$& $[2;4)$& $[4;6)$& $[6;8)$& $[8;10)$\\\hline
			Tần số& $3$& $5$& $12$& $12$& 8\\ \hline
		\end{tabular}
	\end{center}
	Điểm trung bình của các học sinh lớp $10$A là bao nhiêu?
	\choice
	{$5$}
	{\True $5{,}85$}
	{$5{,}65$}
	{$5{,}45$}	
	\loigiai{
		Điểm trung bình $\overline{x}=\dfrac{1\cdot 3+3\cdot 5+5\cdot 12+7\cdot 12+9\cdot 8}{40}=5{,}85$.		
	}
\end{ex}
\begin{ex}%[0D5B3-1]
	Cho bảng phân bố tần số ghép lớp
	\begin{center}
		\begin{tabular}{|l|c|c|c|c|}
			\hline
			{Lớp các giá trị $x$}&{[8; 10)}&{[10; 12)}&{[12; 14]}&{Cộng}\\
			\hline
			{Tần số $n_i$}&{15}&{30}&{55}&{100}\\
			\hline
		\end{tabular}	
	\end{center}
	Số trung bình của các giá trị trong bảng trên là
	\choice
	{$9$}
	{$13$}
	{$11$}
	{\True $11{,}8$}	
	\loigiai{
		Giá trị đại diện của lớp $\left[8; 10\right)$: $c_1=\dfrac{8 + 10}{2}=9$.\\ 
		Giá trị đại diện của lớp $\left[10; 12\right)$: $c_2=\dfrac{10 + 12}{2}=11$.\\ 
		Giá trị đại diện của lớp $\left[12; 14\right)$: $c_3=\dfrac{12 + 14}{2}=13$.\\ 
		Vậy số trung bình cộng $\overline{x}=\dfrac{9\cdot 15 + 11\cdot 30 + 13\cdot 55}{15 + 30 + 55}=\dfrac{59}{5}$.
	}
\end{ex}
\begin{ex}%[0D5B3-1]
	Kết quả khảo sát cân nặng của $25$ quả cam ở lô hàng $A$ được cho như sau
	\begin{center}
		\begin{tabular}{|l|c|c|c|c|c|}
			\hline
			Cân nặng (g) & $[150;155)$ & $[155;160)$ & $[160;165)$ & $[165;170)$ & $[170;175)$ \\
			\hline
			Số quả cam & $2$ & $6$ & $12$ & $4$ & $1$ \\
			\hline
		\end{tabular}	
	\end{center}
	Tính cân nặng trung bình của mỗi quả cam ở lô hàng $A$.
	\choice
	{\True $161{,}7$ (g)}
	{$161{,}7$ (kg)}
	{$155$ (g)}
	{$160$ (kg)}
	\loigiai{
		Ta có giá trị đại diện của các nhóm lần lượt là $152{,}5$; $157{,}5$; $162{,}5$; $167{,}5$; $172{,}5$.\\
		Vậy cân nặng trung bình của mỗi quả cam là
		$$\overline{x}=\dfrac{152{,}5\cdot 2+ 157{,}5\cdot 6+162{,}5\cdot 12+167{,}5\cdot 4+172{,}5\cdot 1}{25}=161{,}7 \text{(g).}$$
	}
\end{ex}
% \begin{ex}%[0D5K3-1]
% 	Ba nhóm học sinh gồm $10$ người, $15$ người, $25$ người. Khối lượng trung bình của mỗi nhóm lần lượt là $50$ kg, $38$ kg, $40$ kg. Khối lượng trung bình của cả ba nhóm học sinh là
% 	\choice
% 	{\True $41{,}4$ kg}
% 	{$42{,}4$ kg}
% 	{$26$ kg}
% 	{$37$ kg}
% 	\loigiai{
% 		Tổng khối lượng nhóm thứ nhất là $50\cdot 10=500$ (kg).\\
% 		Tổng khối lượng nhóm thứ hai là $38\cdot 15=570$ (kg).\\
% 		Tổng khối lượng nhóm thứ ba là $40\cdot 25=1000$ (kg).\\
% 		Tổng khối lượng cả ba nhóm là $500+570+1000=2070$ (kg).\\
% 		Tổng số người cả ba nhóm là $10+15+25=50$ (người).\\
% 		Khối lượng trung bình của cả ba nhóm học sinh là $\dfrac{2070}{50}=41{,}4$ (kg).
% 	}
% \end{ex}
\begin{ex}%[0D5K3-1]
	Sau một kì thi học sinh giỏi Toán, người ta thống kê kết quả (thang điểm $20$) và thu được bảng tần số sau
	\begin{center}
		\begin{tabular}{|l|c|c|c|c|}
			\hline
			Lớp điểm & $[6;10]$ & $[11;15]$ & $[16;20]$ & Cộng\\
			\hline
			Tần số & $22$ & $12$ & $6$ & $40$ \\
			\hline
		\end{tabular}
	\end{center}
	Nếu những học sinh chỉ cần đạt điểm trung bình của bảng điểm trên đều được nhận Giấy Khen của ban tổ chức, thì số học sinh được nhận Giấy Khen là bao nhiêu?
	\choice
	{$11$}
	{\True $18$}
	{$12$}
	{$6$}
	\loigiai{
		Ta lập lại bảng với thêm dòng giá trị đại diện
		\begin{center}
			\begin{tabular}{|l|c|c|c|c|}
				\hline
				Lớp điểm & $[6;10]$ & $[11;15]$ & $[16;20]$ & Cộng\\
				\hline
				Giá trị đại diện & $8$ & $13$ & $18$ & \\
				\hline
				Tần số & $22$ & $12$ & $6$ & $40$ \\
				\hline
			\end{tabular}
		\end{center}
		Điểm trung bình là $\overline{x}=\dfrac{22\cdot 8+12\cdot 13+6\cdot 18}{40}=11$.\\
		Vậy số học sinh được nhận thưởng là $12+6=18$ (học sinh).
	}
\end{ex}
% \begin{ex}%[0D5G3-1]
% 	Cho biết tình hình thu hoạch lúa vụ mùa năm $1980$ của ba hợp tác xã ở địa phương $V$ như sau
% 	\begin{center}
% 		\begin{tabular}{|l|c|c|c|}
% 			\hline
% 			Hợp tác xã & $A$ & $B$ & $C$ \\
% 			\hline
% 			Năng suất lúa (tạ/ha) & $40$ & $38$ & $36$ \\
% 			\hline
% 			Diện tích trồng lúa (ha) & $150$ & $130$ & $120$ \\
% 			\hline
% 		\end{tabular}	
% 	\end{center}
% 	Hãy tính năng suất lúa trung bình của vụ mùa năm $1980$ trong toàn bộ ba hợp tác xã kể trên.
% 	\choice
% 	{\True $38{,}15$ tạ/ha}
% 	{$38{,}05$ tạ/ha}
% 	{$38{,}10$ tạ/ha}
% 	{$38{,}20$ tạ/ha}
% 	\loigiai{
% 		Sản lượng lúa của hợp tác xã $A$ là $40\cdot 150=6000$ (tạ).\\
% 		Sản lượng lúa của hợp tác xã $B$ là $38\cdot 130=4940$ (tạ).\\
% 		Sản lượng lúa của hợp tác xã $C$ là $36\cdot 120=4320$ (tạ).\\
% 		Tổng sản lượng lúa của cả ba hợp tác xã là $6000+4940+4320=15260$ (tạ).\\
% 		Tổng diện tích trồng lúa của cả ba hợp tác xã là $150+130+120=400$ (ha).\\
% 		Vậy năng suất lúa trung bình của cả ba hợp tác xã là $\dfrac{15260}{400}=38{,}15$ tạ/ha.\\
% 		\textbf{\underline{Lưu ý}:} Không thể tính năng suất trung bình bằng cách $\dfrac{40+38+36}{3}=38$ (tạ/ha), vì khi chênh lệch diện tích càng lớn thì số trung bình càng không chính xác.
% 	}
% \end{ex}
% \begin{ex}%[0D5Y3-2]
% 	Tiền lương hàng tháng của $7$ nhân viên trong một công ty du lịch là $650$, $840$, $690$, $2500$, $720$, $670$, $3000$ (đơn vị: nghìn đồng). Tìm số trung vị của các số liệu thống kê đã cho.
% 	\choice
% 	{$690$}
% 	{$2500$}
% 	{\True $720$}
% 	{$670$}
% 	\loigiai{ 
% 		Sắp xếp thứ tự các số liệu thống kê, ta thu được dãy tăng các số liệu như sau: $650$, $670$, $690$, $720$, $840$, $2500$, $3000$ (nghìn đồng).\\
% 		Ta có số các số liệu thống kê là $n=7=2\cdot 3+1$ nên số trung vị là $M_e=x_4=720$.
% 	}
% \end{ex}
% \begin{ex}%[0D5Y3-2]
% 	Điểm học kì một của một học sinh được cho bởi bảng số liệu sau (Đơn vị: điểm)
% 	\begin{center}
% 		\begin{tabular}{|c|c|c|c|c|c|c|c|c|}
% 			\hline
% 			5& 6 &6&7& 7 &8 &8& 8,5&9\\
% 			\hline
% 		\end{tabular}
% 	\end{center}
% 	Số trung vị của bảng nói trên là
% 	\choice
% 	{$6$}
% 	{$9$}
% 	{\True $7$}
% 	{$8$}
% 	\loigiai{ Ta có $N=9$ là số lẻ. Số liệu thứ $\dfrac{N+1}{2} = 5$ là số trung vị.\\
% 		Do đó số trung vị là $M_e = 7$ (điểm).
% 	}
% \end{ex}
% \begin{ex}%[0D5Y3-2]
% 	Điều tra số học sinh giỏi khối $10$ của $15$ trường cấp ba trên địa bản tỉnh $A$, ta được bảng số liệu như sau
% 	\begin{center}
% 		\begin{tabular}{|c|c|c|c|c|c|c|c|c|c|c|c|c|c|c|}
% 			\hline
% 			22& 29 &29&29& 30 &31 &32& 32&33 &34 &34 &35 &35 &35 &36  \\
% 			\hline	
% 		\end{tabular}
% 	\end{center}
% 	Số trung vị của bảng nói trên là
% 	\choice
% 	{\True $8$}
% 	{$9$}
% 	{$6$}
% 	{$7$}
% 	\loigiai{ Ta có $N=15$ là số lẻ $\Rightarrow$ số liệu thứ $\dfrac{15+1}{2}=8$ là số trung vị.\\
% 		Vậy số trung vị là $M_e = 8$.
% 	}
% \end{ex}
% \begin{ex}%[0D5Y3-2]
% 	Cho mẫu số liệu thống kê $\{6;4;4;1;9;10;7\}$. Số liệu trung vị của mẫu số liệu thống kê trên là
% 	\choice
% 	{$1$}
% 	{\True $6$}
% 	{$4$}
% 	{$10$}
% 	\loigiai{
% 		Sắp xếp thành dãy không giảm: $1$, $4$, $4$, $6$, $7$, $9$, $10$.\\
% 		Từ dãy trên ta có số trung vị là số $6$ trong dãy trên.
% 	}
% \end{ex}
% \begin{ex}%[0D5B3-2]
% 	Điểm kiểm tra môn Toán của $10$ học sinh được cho như sau: $6$; $7$; $7$; $6$; $7$; $8$; $8$; $7$; $9$; $9$. Số trung vị của mẫu số liệu trên là	
% 	\choice
% 	{$6$}
% 	{\True $7$}
% 	{$8$}
% 	{$9$}
% 	\loigiai{
% 		Ta sắp xếp số liệu theo thứ tự không giảm như sau: $6$; $6$; $7$; $7$; $7$; $7$; $8$; $8$; $9$; $9$.\\
% 		Dãy số trên có tất cả $10$ giá trị, và $2$ giá trị chính giữa bằng $7$.\\
% 		Vậy số trung vị của mẫu số liệu trên là $\dfrac{7+7}{2}=7$.
% 	}
% \end{ex}
% \begin{ex}%[0D5B3-2]
% 	Một cửa hàng dép da đã thống kê cỡ dép của một số khách hàng nam cho kết quả như sau: $39$; $38$; $39$; $40$; $41$; $41$; $43$; $37$; $38$; $40$; $43$; $41$; $42$; $41$; $42$. Tìm trung vị của mẫu số liệu trên.
% 	\choice
% 	{$37$}
% 	{$39$}
% 	{\True $41$}
% 	{$43$}
% 	\loigiai{
% 		Ta sắp xếp số liệu theo thứ tự không giảm: $37$; $38$; $38$; $39$; $39$; $40$; $40$; $41$; $41$; $41$; $41$; $42$; $42$; $42$; $43$.\\
% 		Vì $n=15$ là số lẻ nên số trung vị là số chính giữa của dãy số liệu.\\
% 		Vậy trung vị là $M_e=41$.
% 	}
% \end{ex}
% \begin{ex}%[0D5B3-2]
% 	Điều tra số học sinh của $30$ lớp học, ta được bảng số liệu như sau
% 	\begin{center}
% 		\begin{tabular}{|c|c|c|c|c|c|c|c|c|c|c|c|c|c|c|}
% 			\hline
% 			35& 39 &39&40& 40 &41 &41& 41&41 &44 &44 &45 &45 &45 &46  \\
% 			\hline
% 			48 &48 &48&48& 49 &49 &49&49 &49 &49 &50 &50 &50 &50 &51  \\
% 			\hline	
% 		\end{tabular}
% 	\end{center}
% 	Số trung vị của bảng nói trên là
% 	\choice
% 	{$46$}
% 	{$49$}
% 	{\True $47$}
% 	{$48$}
% 	\loigiai{ Ta có $N=30$ là số chẵn. Số liệu thứ $15$ và $16$ lần lượt là $46$, $48$ là số trung vị.\\
% 		Vậy số trung vị là $M_e = \dfrac{46+48}{2}=47$ (học sinh).
% 	}
% \end{ex}
% \begin{ex}%[0D5B3-2]
% 	Cho bảng phân bố tần số
% 	\begin{center}
% 		\begin{tabular}{|l|c|c|c|c|c|c|}
% 			\hline
% 			Tuổi & $18$ & $19$ & $20$ & $21$ & $22$ & Cộng \\
% 			\hline
% 			Tần số & $10$ & $50$ & $70$ & $29$ & $10$ & $169$ \\
% 			\hline	
% 		\end{tabular}
% 	\end{center}
% 	Số trung vị của bảng phân bố tần số đã cho là
% 	\choice
% 	{$18$ tuổi}
% 	{\True $20$ tuổi}
% 	{$19$ tuổi}
% 	{$21$ tuổi}
% 	\loigiai{
% 		Sau khi sắp xếp các tuổi trên thành dãy không giảm, do có $169$ số nên số trung vị là số thứ $85$ trong dãy trên.\\
% 		Mà số thứ $85$ trong dãy là $20$. Vậy $M_e=20$.
% 	}
% \end{ex}
% \begin{ex}%[0D5B3-2]
% 	Để khảo sát kết quả thi tuyển sinh môn Toán trong kì thi tuyển sinh Đại học năm vừa qua của trường $A$, người điều tra chọn một mẫu gồm $100$ học sinh tham gia kì thi tuyển sinh đó. Điểm môn Toán (thang điểm $10$) của các học sinh này được cho ở bảng phân bố tần số sau đây
% 	\begin{center}
% 		\begin{tabular}{|l|c|c|c|c|c|c|c|c|c|c|c|c|c|}
% 			\hline
% 			Điểm & $0$ & $1$ & $2$ & $3$ & $4$ & $5$ & $6$ & $7$ & $8$ & $9$ & $10$ &  \\
% 			\hline
% 			Tần số & $1$ & $1$ & $3$ & $5$ & $8$ & $13$ & $19$ & $24$ & $14$ & $10$ & $2$ & $n=100$ \\
% 			\hline	
% 		\end{tabular}
% 	\end{center}
% 	Số trung vị của mẫu số liệu trên.
% 	\choice
% 	{\True $M_e=6{,}5$}
% 	{$M_e=7{,}5$}
% 	{$M_e=5{,}5$}
% 	{$M_e=6$}
% 	\loigiai{
% 		Do kích thước mẫu $n=100$ là một số chẵn nên số trung vị là trung bình cộng của hai giá trị đứng thứ $\dfrac{n}{20}=50$ và $\dfrac{n}{2}+1=51$.\\
% 		Do đó $M_e=\dfrac{6+7}{2}=6{,}5$.
% 	}
% \end{ex}
% \begin{ex}%[0D5B3-2]
% 	Số áo bán được trong một quý ở một cửa hàng bán áo sơ-mi nam được cho trong bảng sau
% 	\begin{center}
% 		\begin{tabular}{|l|c|c|c|c|c|c|c|c|}
% 			\hline
% 			Cỡ số & $36$ & $37$ & $38$ & $39$ & $40$ & $41$ & $42$ & Cộng \\
% 			\hline
% 			Số áo bán được & $13$ & $45$ & $126$ & $110$ & $126$ & $40$ & $5$ & $465$ \\
% 			\hline	
% 		\end{tabular}
% 	\end{center}
% 	Hãy tìm số trung vị của các số liệu thống kê trên.
% 	\choice
% 	{$37$}
% 	{$38$}
% 	{\True $39$}
% 	{$40$}
% 	\loigiai{
% 		Ta sắp xếp dãy số áo bán được theo dãy không giảm
% 		$$36, 36, 36, \ldots, 36, 37, 37, \ldots, 37, 38, 38, \ldots, 38, \ldots, 42, 42.$$
% 		Dãy trên gồm $465$ số nên số trung vị là số thứ $233$.\\
% 		Mà số thứ $233$ là số $39$. Vậy $M_e=233$.
% 	}
% \end{ex}
\begin{ex}
	Cho bảng tần số về cân nặng của 180 người dân trong một xã như sau: (đơn vị: kg)
	\begin{center}
	\begin{tabular}{|c|c|c|}
	\hline
	\textbf{Nhóm} & \textbf{Tần số} & \textbf{Tần số tích luỹ}\\ 
	\hline
	$\left[0;10\right)$ & $6$ & $6$\\
	$\left[10;20\right)$ & $15$ & $21$\\
	$\left[20;30\right)$ & $37$ & $58$\\
	$\left[30;40\right)$ & $48$ & $106$\\
	$\left[40;50\right)$ & $22$ & $128$\\
	$\left[50;60\right)$ & $29$ & $157$\\
	$\left[60;70\right)$ & $23$ & $180$\\

	\hline
	& $n = 180$ &\\
	\hline
\end{tabular}	
	\end{center}
	Tứ phân vị thứ nhất của mẫu số liệu trên là
	\choice
	{$56{,}486$ kg}
	{\True $26{,}486$ kg}
	{$25{,}496$ kg}
	{$36{,}486$ kg}
	\loigiai{
	Số phần tử của mẫu là $n=180$  và $\dfrac{n}{4}=\dfrac{\cdot 180}{4}=45$.\\ Ta có  $21<135<58$ nên nhóm $3$ là nhóm  đầu tiên có   tần số tích luỹ  lớn hơn hoặc bằng $45$.\\
	Xét nhóm $3$ là  nhóm $\left[20;30\right)$ có $s=20$, $h=10$, $n_3=37$, nhóm $2$ là nhóm có $cf_2=21$.\\
	Vậy $Q_1=20+\dfrac{45-21}{37}\cdot 10\approx 26{,}49$ (kg).
}
	\end{ex}

\begin{ex}
	Cho bảng tần số chiều cao của 46 học sinh nam của khối lớp $11$ như sau
	\begin{center}
		\begin{tabular}{|c|c|}
			\hline
			\textbf{Nhóm} & \textbf{Tần số} \\
			\hline
			$\left[155;160\right)$ & $3$ \\
			$\left[160;165\right)$ & $18$ \\
			$\left[165;170\right)$ & $10$ \\
			$\left[170;175\right)$ & $15$ \\
			
			\hline
			& $n = 46$ \\
			\hline
		\end{tabular}	
	\end{center}
	Xác định tứ phân vị thứ nhất của mẫu số liệu trên
\choice
{$161{,}36$}
{$161{,}63$}
{\True $162{,}36$}
{$162{,}63$}
\loigiai{
	Số phần tử của mẫu là $n=46$; $\dfrac{n}{4}=11{,}5$.\\
		Ta có $cf_1=3$, $cf_2=3+18=21$ và $6<11{,}5<21$ nên nhóm $2$ là nhóm đầu tiên có   tần số tích luỹ  lớn hơn hoặc bằng $11{,}5$.\\
		Xét nhóm $2$ là nhóm $\left[160;165\right)$, ta có $s=160$, $h=5$, $n_6=21$; nhóm $1$ có tần số tích luỹ bằng $6$.\\
		Vậy $Q_1=160+\dfrac{11{,}5-3}{18}\cdot 5=162{,}36$.
}
\end{ex}
\begin{ex}
	Cho bảng tần số chiều cao của 46 học sinh nam của khối lớp $11$ như sau
	\begin{center}
		\begin{tabular}{|c|c|}
			\hline
			\textbf{Nhóm} & \textbf{Tần số} \\
			\hline
			$\left[155;160\right)$ & $3$ \\
			$\left[160;165\right)$ & $18$ \\
			$\left[165;170\right)$ & $10$ \\
			$\left[170;175\right)$ & $15$ \\
			
			\hline
			& $n = 46$ \\
			\hline
		\end{tabular}	
	\end{center}
	Xác định tứ phân vị thứ ba của mẫu số liệu trên
	\choice
	{$162{,}36$}
	{$166{,}5$}
	{ $166$}
	{\True $171{,}16$}
	\loigiai{
		Số phần tử của mẫu là $n=46$; $\dfrac{3n}{4}=34{,}5$.\\
		Ta có $cf_3=3+18+10=31$, $cf_4=31+15=46$ và $31<34{,}5<46$ nên nhóm $4$ là nhóm đầu tiên có   tần số tích luỹ  lớn hơn hoặc bằng $34{,}5$.\\
		Xét nhóm $4$ là nhóm $\left[170;175\right)$, ta có $t=170$, $l=5$, $n_4=15$; nhóm $3$ có tần số tích luỹ bằng $31$.\\
		Vậy $Q_3=170+\dfrac{34{,}5-31}{15}\cdot 5=171{,}16$.
	}
\end{ex}
\begin{ex}
	\immini{Cho bảng tần số ghép nhóm số liệu thống kê  chiều cao  của $40$ mẫu cây ở  một vườn thực vật 	(đơn vị: centimét).\\
		Xác định tứ phân vị thứ hai của  số liệu ghép nhóm trên
\choice
{\True $56{,}43$}
{$56{,}34$}
{$46{,}43$}
{$36{,}43$}		

}{\begin{tabular}{|c|c|c|}
	\hline
	\textbf{Nhóm} & \textbf{Tần số} & \textbf{Tần số tích luỹ}\\ 
	\hline
	$\left[30;40\right)$ & $4$ & $4$\\
	$\left[40;50\right)$ & $10$ & $14$\\
	$\left[50;60\right)$ & $14$ & $28$\\
	$\left[60;70\right)$ & $6$ & $34$\\
	$\left[70;80\right)$ & $4$ & $38$\\
	$\left[80; 	90\right)$ & $2$ & $40$\\

	
	\hline
	& $n = 40$ &\\
	\hline
\end{tabular}}
\loigiai{
	Số phần tử của mẫu là $n=46$; $\dfrac{n}{2}=23$.\\
	Ta có $cf_2=14<23<cf_3=28$ nên nhóm $3$ là nhóm đầu tiên có tần số tích lũy lớn hơn hoặc bằng $23$.\\
	Xét nhóm $3$ là nhóm $[50;60)$ có $r=50$, $d=10$, $n_3=14$ và nhóm $2$ là nhóm $\left[40;50\right)$ có $cf_2=14$.\\
	Do đó $Q_2=50+\dfrac{23-14}{14}\cdot 10=56{,}43$.
}
\end{ex}
\begin{ex}
Một bảng xếp hạng đã tính điềm chuẩn hoá cho chỉ số nghiên cứu của một số trường đại học ở
Việt Nam và thu được kết quả sau
\begin{center}
\begin{tabular}{|c|c|c|c|c|c|c|}
	\hline
	\textbf{Điểm} & Dưới $20$  &  $[20;30)$&  $[30;40)$ &  $[40;60)$ &  $[60;80)$ &  $[80;100)$\\
	\hline
	Số điểm & $4$ & $19$&$6$&$2$&$3$&$1$\\

	
	
	\hline

\end{tabular}	
\end{center}
Xác định điểm ngưỡng đề đưa ra danh sách $25$\% trường đại học có chỉ số nghiên cứu tốt nhất Việt Nam.	
\choice
{$25{,}26$}
{\True $35{,}42$}
{$45{,}35$}
{$45{,}42$}	
\loigiai{
Điểm ngưỡng để đưa ra danh sách $25$\% trường đại học có chỉ số nghiên cứu tốt nhất Việt Nam là tứ phân
vị thứ ba.\\
Ta có  $n=35$ và $\dfrac{3n}{4}=26{,}25$.\\
Do $cf_2=4+19=23<26{,}25<cf_3=23+6=29$ nên nhóm $[30;40])$ là nhóm đầu tiên có tần số tích lũy lớn hơn hoặc bằng $23{,}25$.\\
Nhóm $[30;40])$ có $r=30$, $d=10$, $n_3=6$; nhóm $2$ có $cf_2=23$. Do đó
$$Q_3=30+\dfrac{26{,}25-23}{6}\cdot 10\approx 35{,}41.$$ 
Vậy để đưa ra danh sách $25$\% trường đại học có chỉ số nghiên cứu tốt nhất Việt Nam ta lấy các trường có
điểm chuẩn hóa trên $35{,}42$.
}
\end{ex}
% \begin{ex}%1%[1K3B9-3]
% 	Điểm thi môn Toán (thang điểm 100, điểm được làm tròn đến 1) của 60 thí sinh được cho trong bảng sau
% 	\begin{center}
% 		\begin{tabular}{|l|c|c|c|c|c|}
% 			\hline Điểm & $[0-9,5)$ & $[9,5-19,5)$ & $[19,5-29,5)$ & $[29,5-39,5)$ & $[39,5-49,5)$ \\
% 			\hline Số thí sinh & $1 $& $2$ & $4$ & $6$ & $15$ \\
% 			\hline Điểm & $[49,5-59,5)$ & $[59,5-69,5)$ & $[69,5-79,5)$ & $[79,5-89,5)$ & $[89,5-99,5)$ \\
% 			\hline Số thí & $12$ & $10$ & $6$ & $3$ & $1$ \\
% 			\hline
% 		\end{tabular}    
% 	\end{center}
% 	Tìm  tứ phân vị thứ hai của mẫu số liệu.
% 	\choice
% 	{\True $Q_2\approx 51,17$}
% 	{$Q_2\approx 51,67$}
% 	{$Q_2\approx 49,5$}
% 	{$Q_2\approx 41,3$}
% 	\loigiai{Cỡ mẫu là $n=60$.\\
% 		Tứ phân vị thứ nhất $Q_2$ là $\dfrac{x_{30}+x_{31}}{2}$. Do $x_{30}$, $x_{31}$ đều thuộc nhóm $[49,5 ; 59,5)$ nên nhóm này chứa $Q_2$. \\Do đó, $p=6 ; \;a_6=49,5 ;\; m_6=12 ; \;m_1+\ldots+m_5=28, \;a_7-a_6=10$ và ta có
% 		$$
% 		Q_2=49,5+\dfrac{\frac{60}{2}-28}{12}\cdot 10\approx51,17.
% 		$$
% 	}
% \end{ex}
% %2
% \begin{ex}%[1K3B9-3]
% 	Điểm thi môn Toán (thang điểm 100, điểm được làm tròn đến 1) của $60$ thí sinh được cho trong bảng sau
% 	\begin{center}
% 		\begin{tabular}{|l|c|c|c|c|c|}
% 			\hline Điểm & $[0-9,5)$ & $[9,5-19,5)$ & $[19,5-29,5)$ & $[29,5-39,5)$ & $[39,5-49,5)$ \\
% 			\hline Số thí sinh & $1 $& $2$ & $4$ & $6$ & $15$ \\
% 			\hline Điểm & $[49,5-59,5)$ & $[59,5-69,5)$ & $[69,5-79,5)$ & $[79,5-89,5)$ & $[89,5-99,5)$ \\
% 			\hline Số thí & $12$ & $10$ & $6$ & $3$ & $1$ \\
% 			\hline
% 		\end{tabular}    
% 	\end{center}
% 	Tìm  tứ phân vị thứ nhất của mẫu số liệu.
% 	\choice
% 	{\True $Q_1\approx 41,3$}
% 	{$Q_1\approx 51,67$}
% 	{$Q_1\approx 40,83$}
% 	{$Q_1\approx 51,17$}
% 	\loigiai{Cỡ mẫu là $n=60$.\\
% 		Tứ phân vị thứ nhất $Q_1$ là $\dfrac{x_{15}+x_{16}}{2}$. Do $x_{15}$, $x_{16}$ đều thuộc nhóm $[39,5-49,5)$ nên nhóm này chứa $Q_1$. \\Do đó, $p=5 ; \;a_5=39,5 ;\; m_5=15 ; \;m_1+\ldots+m_4=13, \;a_6-a_5=10$ và ta có
% 		$$
% 		Q_1=39,5+\dfrac{\frac{60}{4}-13}{15}\cdot 10\approx 40,83.
% 		$$
% 	}
% \end{ex}
%3
\begin{ex}%[1K3B9-3]
	Phỏng vấn một số học sinh khối 11 vể thời gian (giờ) ngủ của một buổi tối, thu được bảng số liệu như sau.
	\begin{center}
		\begin{tabular}{|l|c|c|c|c|c|}
			\hline Thời gian  (giờ)  &{$[4 ; 5)$}&{$[5 ; 6)$}&{$[6 ; 7)$}&{$[7 ; 8)$}&{$[8 ; 9)$}\\
			\hline Số học sinh & $6$ & $10$ & $13$ & $9$ & $7$ \\
			\hline
		\end{tabular}     
	\end{center}
	Hãy cho biết $75 \%$ học sinh khối 11 ngủ ít nhất bao nhiêu giờ?
	\choice
	{$7,675$}
	{\True $7,53$}
	{$8$}
	{ $7,9$}
	\loigiai{
		Cỡ mẫu là $n=45$.\\
		Gọi $x_1, \ldots, x_{45}$ là mẫu số liệu được sắp xếp theo thứ tự không giảm. Khi đó, trung vị là $x_{23}$. Do đó, tứ phân vị thứ ba $Q_3$ là $x_{34}$. Do $x_{34}$ đều thuộc nhóm $[7;8)$ nên nhóm này chứa $Q_3$. Do đó, $p=4 ; \;a_4=7 ;\; m_4=9 ; \;m_1+m_2+m_3=29 ; \;a_5-a_4=1$ và ta có
		$$
		Q_3=7+\dfrac{\frac{3 \cdot 45}{4}-29}{9}\cdot 1\approx7,53.
		$$ 
		Vậy $75\%$ học sinh khối 11 ngủ ít nhất $7,53$ giờ.
	}
\end{ex}
%4
% \begin{ex}%[1K3B9-3]
% 	Điểm thi môn Toán (thang điểm 100, điểm được làm tròn đến 1) của 60 thí sinh được cho trong bảng sau
% 	\begin{center}
% 		\begin{tabular}{|l|c|c|c|c|c|}
% 			\hline Điểm & $[0-9,5)$ & $[9,5-19,5)$ & $[19,5-29,5)$ & $[29,5-39,5)$ & $[39,5-49,5)$ \\
% 			\hline Số thí sinh & $1 $& $2$ & $4$ & $6$ & $15$ \\
% 			\hline Điểm & $[49,5-59,5)$ & $[59,5-69,5)$ & $[69,5-79,5)$ & $[79,5-89,5)$ & $[89,5-99,5)$ \\
% 			\hline Số thí & $12$ & $10$ & $6$ & $3$ & $1$ \\
% 			\hline
% 		\end{tabular}    
% 	\end{center}
% 	Tìm  tứ phân vị thứ ba của mẫu số liệu.
% 	\choice
% 	{$Q_3=41,3$}
% 	{$Q_3=51,67$}
% 	{$Q_3=45$}
% 	{\True $Q_3=65$}
% 	\loigiai{Cỡ mẫu là $n=60$.\\
% 		Với tứ phân vị thứ ba $Q_3$ là $\dfrac{x_{45}+x_{46}}{2}$. Do $x_{45}$, $x_{46}$ đều thuộc nhóm $[60 ; 70)$ nên nhóm này chứa $Q_3$. Do đó, $p=7 ; \;a_7=60 ;\; m_7=10 ; \;m_1+\ldots+m_6=40 ; \;a_8-a_7=10$ và ta có
% 		$$
% 		Q_3=59,5+\dfrac{\frac{3 \cdot 60}{4}-40}{10}\cdot 10=64,5.
% 		$$
% 	}
% \end{ex}
%5
\begin{ex}%[1K3B9-3]
	Một hãng xe ô tô thống kê lại số lần gặp sự cố về động cơ về động cơ của $100$ chiếc xe cùng loại sau 2 năm sử dụng đầu tiên ở dảng sau
	\begin{center}
		\begin{tabular}{|l|c|c|c|c|c|}
			\hline Số lần gặp sự cố  &{$[0,5;2,5)$}&{$[2,5;4,5)$}&{$[4,5;6,5)$}&{$[6,5 ; 8,5)$}&{$[8,5;10,5)$}\\
			\hline Số xe & $17$ & $33$ & $25$ & $20$ & $5$ \\
			\hline
		\end{tabular}     
	\end{center}
	Tìm tứ phân vị thứ nhất của mẫu số liệu.
	\choice
	{\True $Q_1\approx 4$}
	{$Q_1\approx 2,98$}
	{$Q_1\approx 2,5$}
	{$Q_1\approx 3,5$}
	\loigiai{
		Cỡ mẫu là $n=100$.\\
		Gọi $x_1, \ldots, x_{100}$ là mẫu số liệu được sắp xếp theo thứ tự không giảm. Khi đó, trung vị là $\dfrac{x_{50}+x_{51}}{2}$. 
		Do đó, tứ phân vị thứ nhất $Q_1$ là $\dfrac{x_{25}+x_{26}}{2}$. Do $x_{25}$, $x_{26}$ đều thuộc nhóm $[2,5;4,5)$ nên nhóm này chứa $Q_1$. \\Do đó, $p=2 ; \;a_2=2,5;\; m_2=33 ; \;m_1=17, \;a_3-a_2=2$ và ta có
		$$
		Q_1=2,5+\dfrac{\frac{100}{4}-17}{33}\cdot 2\approx 2,98.
		$$
	}    
\end{ex}
%6
% \begin{ex}%[1K3B9-3]
% 	Một hãng xe ô tô thống kê lại số lần gặp sự cố về động cơ về động cơ của $100$ chiếc xe cùng loại sau 2 năm sử dụng đầu tiên ở dảng sau
% 	\begin{center}
% 		\begin{tabular}{|l|c|c|c|c|c|}
% 			\hline Số lần gặp sự cố  &{$[0,5;2,5)$}&{$[2,5;4,5)$}&{$[4,5;6,5)$}&{$[6,5 ; 8,5)$}&{$[8,5;10,5)$}\\
% 			\hline Số xe & $17$ & $33$ & $25$ & $20$ & $5$ \\
% 			\hline
% 		\end{tabular}     
% 	\end{center}
% 	Tìm   tứ phân vị thứ hai của mẫu số liệu.
% 	\choice
% 	{\True $Q_2=4,5$}
% 	{$Q_2\approx 5,12$}
% 	{$Q_2\approx 4,89$}
% 	{$Q_2\approx 5,2$}
% 	\loigiai{
% 		Cỡ mẫu là $n=100$.\\
% 		Gọi $x_1, \ldots, x_{100}$ là mẫu số liệu được sắp xếp theo thứ tự không giảm. Khi đó, trung vị là $\dfrac{x_{50}+x_{51}}{2}$. Do $x_{50} \in [2,5;4,5)$, $x_{51} \in [4,5;6,5)$  nên tứ phân vị thứ hai của mẫu số liệu ghép nhóm là  $Q_2=4,5$. 
% 	}
% \end{ex}
%7
\begin{ex}%[1K3B9-3]
	Một hãng xe ô tô thống kê lại số lần gặp sự cố về động cơ về động cơ của $100$ chiếc xe cùng loại sau 2 năm sử dụng đầu tiên ở dảng sau
	\begin{center}
		\begin{tabular}{|l|c|c|c|c|c|}
			\hline Số lần gặp sự cố  &{$[0,5;2,5)$}&{$[2,5;4,5)$}&{$[4,5;6,5)$}&{$[6,5 ; 8,5)$}&{$[8,5;10,5)$}\\
			\hline Số xe & $17$ & $33$ & $25$ & $20$ & $5$ \\
			\hline
		\end{tabular}     
	\end{center}
	Tìm   tứ phân vị thứ ba của mẫu số liệu.  
	\choice
	{$Q_3=6,3$}
	{$Q_3=6,8$}
	{$Q_3=7,2$}
	{\True $Q_3=6,5$}
	\loigiai{ Cỡ mẫu là $n=100$.\\
		Với tứ phân vị thứ ba $Q_3$ là $\dfrac{x_{75}+x_{76}}{2}$. Do $x_{75} \in [4,5;6,5)$, $x_{76} \in [6,5 ; 8,5)$  nên tứ phân vị thứ ba của mẫu số liệu ghép nhóm là $Q_3=6,5$. 
		
	}
\end{ex}
%8
% \begin{ex}%[1K3B9-3]
% 	Lương tháng của một số nhân viên văn phòng được ghi lại như sau (đơn vị: triệu đồng)
% 	\begin{center}
% 		\begin{tabular}{|l|c|c|c|c|c|}
% 			\hline Lương tháng (triệu đồng)  &{$[6;8)$}&{$[8;10)$}&{$[10;12)$}&{$[12;14)$}\\
% 			\hline Số nhân viên & $3$ & $6$ & $8$ & $7$  \\
% 			\hline
% 		\end{tabular}     
% 	\end{center}  
% 	Tìm tứ phân vị thứ nhất của mẫu số liệu.
% 	\choice
% 	{\True $Q_1= 9$}
% 	{$Q_1= 8,5$}
% 	{$Q_1= 9,5$}
% 	{$Q_1= 8,2$}
% 	\loigiai{
% 		Cỡ mẫu là $n=24$.\\
% 		Gọi $x_1, \ldots, x_{24}$ là mẫu số liệu được sắp xếp theo thứ tự không giảm. Khi đó, trung vị là $\dfrac{x_{12}+x_{13}}{2}$. 
% 		Do đó, tứ phân vị thứ nhất $Q_1$ là $\dfrac{x_{6}+x_{7}}{2}$. Do $x_{6}$, $x_{7}$ đều thuộc nhóm $[8;10)$ nên nhóm này chứa $Q_1$. \\Do đó, $p=2 ; \;a_2=8;\; m_2=6 ; \;m_1=3, \;a_3-a_2=2$ và ta có
% 		$$
% 		Q_1=8+\dfrac{\frac{24}{4}-3}{6}\cdot 2=9.
% 		$$
% 	}    
% \end{ex}
% %9
% \begin{ex}%[1K3B9-3]
% 	Lương tháng của một số nhân viên văn phòng được ghi lại như sau (đơn vị: triệu đồng)
% 	\begin{center}
% 		\begin{tabular}{|l|c|c|c|c|c|}
% 			\hline Lương tháng (triệu đồng)  &{$[6;8)$}&{$[8;10)$}&{$[10;12)$}&{$[12;14)$}\\
% 			\hline Số nhân viên & $3$ & $6$ & $8$ & $7$  \\
% 			\hline
% 		\end{tabular}     
% 	\end{center}   
% 	Tìm   tứ phân vị thứ hai của mẫu số liệu.
% 	\choice
% 	{\True $Q_2=10,75$}
% 	{$Q_2= 10,5$}
% 	{$Q_2= 11$}
% 	{$Q_2=11,5$}
% 	\loigiai{
% 		Cỡ mẫu là $n=24$.\\
% 		Gọi $x_1, \ldots, x_{24}$ là mẫu số liệu được sắp xếp theo thứ tự không giảm. Khi đó, trung vị là $\dfrac{x_{12}+x_{13}}{2}$. 
% 		Do đó,  tứ phân vị thứ hai $Q_2$ là $\dfrac{x_{12}+x_{13}}{2}$. Do $x_{12}$, $x_{13}$ đều thuộc nhóm $[10;12)$ nên nhóm này chứa $Q_2$. \\Do đó, $p=3 ; \;a_3=10;\; m_3=8 ; \;m_1+m_2=9, \;a_4-a_3=2$ và ta có
% 		$$
% 		Q_2=10+\dfrac{\frac{24}{2}-9}{8}\cdot 2=10,75.
% 		$$
% 	}
% \end{ex}
% %10
% \begin{ex}%[1K3B9-3]
% 	Lương tháng của một số nhân viên văn phòng được ghi lại như sau (đơn vị: triệu đồng)
% 	\begin{center}
% 		\begin{tabular}{|l|c|c|c|c|c|}
% 			\hline Lương tháng (triệu đồng)  &{$[6;8)$}&{$[8;10)$}&{$[10;12)$}&{$[12;14)$}\\
% 			\hline Số nhân viên & $3$ & $6$ & $8$ & $7$  \\
% 			\hline
% 		\end{tabular}     
% 	\end{center}  
% 	Tìm   tứ phân vị thứ ba của mẫu số liệu.
% 	\choice 
% 	{$Q_3\approx 12,5$}
% 	{$Q_3\approx 13,2$}
% 	{$Q_3\approx 13,5$}
% 	{\True $Q_3\approx 12,3$}
% 	\loigiai{
% 		Cỡ mẫu là $n=24$.\\
% 		Gọi $x_1, \ldots, x_{24}$ là mẫu số liệu được sắp xếp theo thứ tự không giảm. Khi đó, trung vị là $\dfrac{x_{12}+x_{13}}{2}$. 
% 		Do đó,  tứ phân vị thứ ba $Q_3$ là $\dfrac{x_{18}+x_{19}}{2}$. Do $x_{18}$, $x_{19}$ đều thuộc nhóm $[12;14)$ nên nhóm này chứa $Q_3$. \\Do đó, $p=4 ; \;a_4=12;\; m_4=7 ; \;m_1+m_2+m_3=17, \;a_4-a_3=2$ và ta có
% 		$$
% 		Q_2=12+\dfrac{\frac{24\cdot 3}{4}-17}{7}\cdot 2\approx 12,3.
% 		$$
% 	}
% \end{ex}
%11
% \begin{ex}%[1K3B9-3]
% 	Số điểm một cầu thủ bóng rổ ghi được trong 20 trận đấu được cho ở bảng sau
% 	\begin{center}
% 		\begin{tabular}{|l|c|c|c|c|c|}
% 			\hline Điểm số  &{$[5,5;10,5)$}&{$[10,5;15,5)$}&{$[15,5;20,5)$}&{$[20,5;25,5)$}\\
% 			\hline Số trận & $3$ & $9$ & $2$ & $6$  \\
% 			\hline
% 		\end{tabular}     
% 	\end{center}  
% 	Tìm   tứ phân vị thứ ba của mẫu số liệu.
% 	\choice 
% 	{$Q_3\approx 23,5$}
% 	{$Q_3\approx 22,2$}
% 	{$Q_3\approx 21,6$}
% 	{\True $Q_3\approx 21,3$}
% 	\loigiai{
% 		Cỡ mẫu là $n=20$.\\
% 		Gọi $x_1, \ldots, x_{20}$ là mẫu số liệu được sắp xếp theo thứ tự không giảm. Khi đó, trung vị là $\dfrac{x_{10}+x_{11}}{2}$. 
% 		Do đó,  tứ phân vị thứ ba $Q_3$ là $\dfrac{x_{15}+x_{16}}{2}$. Do $x_{15}$, $x_{16}$ đều thuộc nhóm $[20,5;25,5)$ nên nhóm này chứa $Q_3$. \\Do đó, $p=4 ; \;a_4=20,5;\; m_4=6 ; \;m_1+m_2+m_3=14, \;a_5-a_4=25,5-20,5=5$ và ta có
% 		$$
% 		Q_2=20,5+\dfrac{\frac{20\cdot 3}{4}-14}{6}\cdot 5 \approx 21,3.
% 		$$
% 	}
% \end{ex}
% %12
% \begin{ex}%[1K3B9-3]
% 	Số điểm một cầu thủ bóng rổ ghi được trong 20 trận đấu được cho ở bảng sau
% 	\begin{center}
% 		\begin{tabular}{|l|c|c|c|c|c|}
% 			\hline Điểm số  &{$[5,5;10,5)$}&{$[10,5;15,5)$}&{$[15,5;20,5)$}&{$[20,5;25,5)$}\\
% 			\hline Số trận & $3$ & $9$ & $2$ & $6$  \\
% 			\hline
% 		\end{tabular}     
% 	\end{center} 
% 	Tìm tứ phân vị thứ nhất của mẫu số liệu.
% 	\choice
% 	{\True $Q_1\approx 11,6$}
% 	{$Q_1\approx 11,3$}
% 	{$Q_1\approx 21,6$}
% 	{$Q_1\approx 21,3$}
% 	\loigiai{
% 		Cỡ mẫu là $n=20$.\\
% 		Gọi $x_1, \ldots, x_{20}$ là mẫu số liệu được sắp xếp theo thứ tự không giảm. Khi đó, trung vị là $\dfrac{x_{10}+x_{11}}{2}$. 
% 		Do đó, tứ phân vị thứ nhất $Q_1$ là $\dfrac{x_{5}+x_{6}}{2}$. Do $x_{5}$, $x_{6}$ đều thuộc nhóm $[10,5;15,5)$ nên nhóm này chứa $Q_1$. \\Do đó, $p=2 ; \;a_2=10,5;\; m_2=9 ; \;m_1=3, \;a_3-a_2=5$ và ta có
% 		$$
% 		Q_1=10,5+\dfrac{\frac{20}{4}-3}{9}\cdot 5\approx 11,6.
% 		$$
% 	}
% \end{ex}
%13
\begin{ex}%[1K3B9-3]
	Số điểm một cầu thủ bóng rổ ghi được trong 20 trận đấu được cho ở bảng sau
	\begin{center}
		\begin{tabular}{|l|c|c|c|c|c|}
			\hline Điểm số  &{$[5,5;10,5)$}&{$[10,5;15,5)$}&{$[15,5;20,5)$}&{$[20,5;25,5)$}\\
			\hline Số trận & $3$ & $9$ & $2$ & $6$  \\
			\hline
		\end{tabular}     
	\end{center} 
	Tìm tứ phân vị thứ nhất của mẫu số liệu.
	\choice
	{\True $Q_1\approx 11,6$}
	{$Q_1\approx 14,4$}
	{$Q_1\approx 15,6$}
	{$Q_1\approx 21,3$}
	\loigiai{
		Cỡ mẫu là $n=20$.\\
		Gọi $x_1, \ldots, x_{20}$ là mẫu số liệu được sắp xếp theo thứ tự không giảm. Khi đó, trung vị là $\dfrac{x_{10}+x_{11}}{2}$. 
		Do đó, tứ phân vị thứ nhất $Q_2$ là $\dfrac{x_{10}+x_{11}}{2}$. Do $x_{10}$, $x_{11}$ đều thuộc nhóm $[10,5;15,5)$ nên nhóm này chứa $Q_2$. \\Do đó, $p=2 ; \;a_2=10,5;\; m_2=9 ; \;m_1=3, \;a_3-a_2=5$ và ta có
		$$
		Q_1=10,5+\dfrac{\frac{20}{2}-3}{9}\cdot 5\approx 14,4.
		$$   
	}
\end{ex}
%14
% \begin{ex}%[1K3B9-3]
% 	Một người thống kê lại thời gian thực hiện các cuộc gọi điện thoại của người đó trong một tuần cho trong bảng sau
% 	\begin{center}
% 		\begin{tabular}{|l|c|c|c|c|c|}
% 			\hline Số bệnh nhân &{$[0;60)$}&{$[60;120)$}&{$[120;180)$}&{$[180;240)$}&{$[240;300)$}\\
% 			\hline Số ngày & $8$ & $10$ & $7$ & $5$ & $2$ \\
% 			\hline
% 		\end{tabular}     
% 	\end{center}
% 	Tìm   tứ phân vị thứ ba của mẫu số liệu.  
% 	\choice
% 	{$Q_3\approx 175,28$}
% 	{$Q_3\approx 150,32 $}
% 	{$Q_3=175$}
% 	{\True $Q_3\approx 171,43$}
% 	\loigiai{ Cỡ mẫu là $n=32$.\\
% 		Gọi $x_1, \ldots, x_{32}$ là mẫu số liệu được sắp xếp theo thứ tự không giảm. Khi đó, trung vị là $\dfrac{x_{16}+x_{17}}{2}$.\\
% 		Do đó, tứ phân vị thứ ba $Q_3$ là $\dfrac{x_{24}+x_{25}}{2}$. Do $x_{24} $, $x_{25} \in [120;180)$   nên nhóm này chứa $Q_3$. \\Do đó, $p= 3; \;a_3=120 ;\; m_3=7 ; \;m_1+m_2=18 ; \;a_4-a_3=60$ và ta có
% 		$$
% 		Q_3=120+\dfrac{\frac{3 \cdot 32}{4}-18}{7}\cdot 60\approx 171,43.
% 		$$
% 	}
% \end{ex}
% %15
% \begin{ex}%[1K3B9-3]
% 	Một người thống kê lại thời gian thực hiện các cuộc gọi điện thoại của người đó trong một tuần cho trong bảng sau
% 	\begin{center}
% 		\begin{tabular}{|l|c|c|c|c|c|}
% 			\hline Số bệnh nhân &{$[0;60)$}&{$[60;120)$}&{$[120;180)$}&{$[180;240)$}&{$[240;300)$}\\
% 			\hline Số ngày & $8$ & $10$ & $7$ & $5$ & $2$ \\
% 			\hline
% 		\end{tabular}     
% 	\end{center}
% 	Tìm   tứ phân vị thứ hai của mẫu số liệu.  
% 	\choice
% 	{$Q_2\approx 80,25$}
% 	{$Q_2\approx 100,32$}
% 	{$Q_2=115$}
% 	{\True $Q_2=108$}
% 	\loigiai{ Cỡ mẫu là $n=32$.\\
% 		Gọi $x_1, \ldots, x_{32}$ là mẫu số liệu được sắp xếp theo thứ tự không giảm. Khi đó, trung vị là $\dfrac{x_{16}+x_{17}}{2}$.\\
% 		Do đó, tứ phân vị thứ hai $Q_2$ là $\dfrac{x_{16}+x_{17}}{2}$. Do $x_{16} $, $x_{17} \in [60;120)$   nên nhóm này chứa $Q_2$. \\Do đó, $p= 2; \;a_2=60 ;\; m_2=10 ; \;m_1=8 ; \;a_3-a_2=60$ và ta có
% 		$$
% 		Q_3=60+\dfrac{\frac{ 32}{2}-8}{10}\cdot 60=108.
% 		$$
% 	}    
% \end{ex}
% \begin{ex}
% 	Một công ty xây dựng khảo sát khách hàng xem họ có nhu cầu mua nhà ở mức giá nào. Kết quả khảo sát được ghi lại ở bảng sau
% 	\begin{center}
% 		\begin{tabular}{|c|c|c|c|c|c|}
% 			\hline \begin{tabular}{c}
% 				\textbf{Mức giá} \\
% 				\textbf{(triệu đồng/$\mathrm{m}^2)$}
% 			\end{tabular} &{$[10; 14)$} &{$[14; 18)$} &{$[18; 22)$} &{$[22; 26)$} &{$[26; 30)$} \\
% 			\hline \textbf{Số khách hàng} & 54 & 78 & 120 & 45 & 12 \\
% 			\hline
% 		\end{tabular}
% 	\end{center}
	
% 	 Công ty nên xây nhà ở mức giá nào để nhiều người có nhu cầu mua nhất?
% 	\choice
% 	{\True $19{,}4$ triệu đồng$/ \mathrm{m}^2$ }
% 	{$20{,}4$ triệu đồng$/ \mathrm{m}^2$ }
% 	{$19{,}6$ triệu đồng$/ \mathrm{m}^2$ }
% 	{$20{,}6$ triệu đồng$/ \mathrm{m}^2$ }	
% 	\loigiai{
% 		\ Nhóm chứa mốt của mẫu số liệu trên là nhóm $[18; 22)$.\\ Do đó $u_m=18$, $n_{m-1}=78$, $n_m=120$, $n_{m+1}=45$, $u_{m+1}-u_m=22-18=4$.\\
% 			Mốt của mẫu số liệu ghép nhóm là
% 			\[M_0=18+\dfrac{120-78}{(120-78)+(120-45)} \cdot 4=\dfrac{758}{39} \approx 19{,}4. \]
% 		 Dựa vào kết quả trên ta có thể dự đoán rằng nếu công ty xây nhà ở mức giá $19{,}4$ triệu đồng$/ \mathrm{m}^2$ thì sẽ có nhiều người có nhu cầu mua nhất.
		
% 	}
% \end{ex}
\begin{ex}%[1T5B1-3]
	Số cuộc gọi điện thoại một nguời thực hiện mỗi ngày trong $30$ ngày được lựa chọn ngẫu nhiên được thống kê trong bảng sau:
\begin{center}
	\begin{tabular}{|c|c|c|c|c|c|}
		\hline Số cuộc gọi &{$[3; 5]$} &{$[6; 8]$} &{$[9; 11]$} &{$[12; 14]$} &{$[15; 17]$} \\
		\hline Số ngày & 5 & 13 & 7 & 3 & 2 \\
		\hline
	\end{tabular}
\end{center}
 Tìm mốt của mẫu số liệu ghép nhóm trên.
 Hãy dự đoán xem khả năng người đó thực hiện bao nhiêu cuộc gọi mỗi ngày là cao nhất.
\choice
{$4$}
{$6$}
{$5$}
{\True $7$}	
\loigiai{
	Do số cuộc gọi là số nguyên nên ta hiệu chỉnh lại như sau:
	\begin{center}
		\begin{tabular}{|c|c|c|c|c|c|}
			\hline Số cuộc gọi &{$[2{,}5; 5{,}5)$} &{$[5{,}5; 8{,}5)$} &{$[8{,}5; 11{,}5)$} &{$[11{,}5; 14{,}5)$} &{$[14{,}5; 17{,}5)$} \\
			\hline Số ngày & 5 & 13 & 7 & 3 & 2 \\
			\hline
		\end{tabular}
	\end{center}	
		 Nhóm chứa mốt của mẫu số liệu trên là nhóm $[5{,}5; 8{,}5)$.\\
		Do đó $u_m=5{,}5$; $n_{m-1}=5$; $n_m=13$; $n_{m+1}=7$; $u_{m+1}-u_m=8{,}5-5{,}5=3$.\\
		Mốt của mẫu số liệu ghép nhóm là
		\[M_0=5{,}5+\dfrac{13-5}{(13-5)+(13-7)} \cdot 3=\dfrac{101}{14} \approx 7{,}2. \]
	 Dựa vào kết quả trên ta có thể dự đoán rằng khả năng người đó thực hiện $7$ cuộc gọi mỗi ngày là cao nhất.
	
}
\end{ex}
\begin{ex}
	Một thư viện thống kê số lượng sách được mượn mỗi ngày trong ba tháng ở bảng sau:
	\begin{center}
		\begin{tabular}{|c|c|c|c|c|c|c|c|}
			\hline Số sách &{$[16; 20]$} &{$[21; 25]$} &{$[26; 30]$} &{$[31; 35]$} &{$[36; 40]$} &{$[41; 45]$} &{$[46; 50]$} \\
			\hline Số ngày & 3 & 6 & 15 & 27 & 22 & 14 & 5 \\
			\hline
		\end{tabular}
	\end{center}
	Hãy ước lượng  mốt của mẫu số liệu ghép nhóm trên.
	\choice
	{$34{,}33$}
	{\True $34{,}03$}
	{$35{,}63$}
	{$34{,}13$}	
	\loigiai{
		Vì số lượng sách được mượn là số nguyên nên ta hiệu chỉnh bảng tần số ghép nhóm (theo giá trị đại diện) như sau
		\begin{center}
			{\footnotesize \begin{tabular}{|c|c|c|c|c|c|c|c|}
					\hline Số sách &{$[15{,}5; 20{,}5)$} &{$[20{,}5; 25{,}5)$} &{$[25{,}5; 30{,}5)$} &{$[30{,}5; 35{,}5)$} &{$[35{,}5; 40{,}5]$} &{$[40{,}5; 45{,}5)$} &{$[45{,}5; 50{,}5)$} \\
					\hline Giá trị đại diện &{$18$} &{$23$} &{$28$} &{$33$} &{$38$} &{$43$} &{$48$} \\
					\hline Số ngày & 3 & 6 & 15 & 27 & 22 & 14 & 5 \\
					\hline
			\end{tabular}}
		\end{center}
		Trung bình số lượng sách được mượn mỗi ngày trong 3 tháng của thư viện là
		\[\overline{x}=\dfrac{18\cdot 3+23\cdot 6+28\cdot 15+33\cdot 27+38\cdot 22+43\cdot 14+48\cdot 5}{92}\approx 34{,}58. \]
		Nhóm chứa mốt của mẫu số liệu trên là nhóm $[30{,}5; 35{,}5)$.\\
		Do đó $u_m=30{,}5$; $n_{m-1}=15$; $n_m=27$; $n_{m+1}=22$; $u_{m+1}-u_m=35{,}5-30{,}5=5$.\\
		Mốt của mẫu số liệu ghép nhóm là
		\[M_0=30{,}5+\dfrac{27-15}{(27-15)+(27-22)} \cdot 5\approx 34{,}03. \]
	}
\end{ex}
\begin{ex}
	Kết quả đo chiều cao của $200$ cây keo $3$ năm tuổi ở một nông trường được biểu diễn ở biểu đồ dưới đây.
	\begin{center}
		\begin{tikzpicture}[scale=1,font=\scriptsize]
		\def\hoanh{11.5};
		\def\tung{6.5};
		\def\mau{cyan};
		\foreach \x/\n in{1/20,3/35,5/60,7/55,9/30}{\draw[line width=16mm,\mau] (\x,0)--++(0,{\n/10});
			\draw[dashed] (\x,{\n/10})node[above]{$\n$}--(0,{\n/10}) node[left]{$\n$};}
		\foreach \x/\p in {1/[8{,}5;8{,}8),3/[8{,}8;9{,}1),5/[9{,}1;9{,}4),7/[9{,}4;9{,}7),9/[9{,}7;10{,}0)}{\node[below] at (\x,0){\scriptsize $\p$};}
		\draw[->] (0,0)--(\hoanh,0) node[below]{($m$)};
		\draw[->] (0,0)node[below left]{$O$}--(0,\tung) node[left]{(Số cây)};
		\path (current bounding box.north) node[above]		{\textbf{Chiều cao 200 cây keo 3 năm tuổi}};
		\end{tikzpicture}
	\end{center}
	Mốt của mẫu số liệu ghép nhóm trên là
	\choice
	{$9{,}35$}
	{$10{,}53$}
	{$10{,}35$}
	{$9{,}53$}	
	\loigiai{
		Bảng tần số ghép nhóm (theo giá trị đại diện)
		\begin{center}
			\begin{tabular}{|c|c|c|c|c|c|}
				\hline Chiều cao &$[8{,}5; 8{,}8)$ &{$[8{,}8; 9{,}1)$} &{$[9{,}1; 9{,}4)$} &{$[9{,}4; 9{,}7)$} &{$[9{,}7; 10{,}0)$} \\
				\hline Giá trị đại diện &$8{,}65$ &$8{,}95$ &$9{,}25$ &$9{,}55$ &$9{,}85$ \\
				\hline Số cây & $20$ & $35$ & $60$ & $55$ & $30$\\
				\hline
			\end{tabular}
		\end{center}
		Chiều cao trung bình của $200$ cây keo 3 năm tuổi là
		\[\overline{x}=\dfrac{8{,}65\cdot 20+8{,}95\cdot 35+9{,}25\cdot 60+9{,}55\cdot 55+9{,}85\cdot 30}{200}\approx 9{,}31. \]
		Nhóm chứa mốt của mẫu số liệu trên là nhóm $[9{,}1; 9{,}4)$.\\
		Do đó $u_m=9{,}1$; $n_{m-1}=35$; $n_m=60$; $n_{m+1}=55$; $u_{m+1}-u_m=9{,}4-9{,}1=0{,}3$.\\
		Mốt của mẫu số liệu ghép nhóm là
		\[M_0=9{,}1+\dfrac{60-35}{(60-35)+(60-55)} \cdot 0{,}3= 9{,}35. \]
	}
\end{ex}
\begin{ex}%[1K3B9-4]
	Bảng số liệu ghép nhóm sau cho biết chiều cao (cm) của $50$ học sinh lớp $11A$.
	\begin{center}
		\begin{tabular}{|c|c|c|c|c|c|c|}
			\hline
			Khoảng chiều cao (cm)	& $\left[145;150 \right)$ & $\left[150;155 \right)$ & $\left[155;160 \right)$ & $\left[160;165 \right)$&$\left[165;170 \right)$  \\
			\hline
			Số học sinh&$7$	& $14$ & $10$ &$10$  & $9$ \\
			\hline
		\end{tabular}
	\end{center}
Mốt của mẫu số liệu ghép nhóm là
	\choice
	{$154{,}20$}
	{\True $153{,}18$}
	{$155{,}12$}
	{$158{,}36$}	
	\loigiai{
		Tần số lớn nhất là $14$ nên nhóm chứa mốt là nhóm $\left[150;155 \right)$. Ta có $j=2$, $a_2=150$, $m_2=14$, $m_1=7$, $m_3=10$, $h=5$. Do đó $$M_o=150+\dfrac{14-7}{\left(14-7\right)+\left(14-10\right)\cdot 5}\approx 153{,}18.$$	
		Số học sinh có chiều cao khoảng $153{,}18$ là nhiều nhất.
	}
\end{ex}
\begin{ex}%[1K3Y9-4]
	Chọn khẳng định \textbf{sai}.
	\choice
	{ Mốt của mẫu số liệu không ghép nhóm là giá trị có khả năng xuất hiện cao nhất khi lấy mẫu}
	{Mốt của mẫu số liệu sau khi ghép nhóm xấp xỉ với mốt của mẫu số liệu không ghép nhóm}
	{\True Một mẫu số liệu ghép nhóm chỉ có một mốt}
	{Một mẫu số liệu ghép nhóm có thể có nhiều nhóm chứa mốt và nhiều mốt}
	\loigiai{
		Mốt của mẫu số liệu không ghép nhóm là giá trị có khả năng xuất hiện cao nhất khi lấy mẫu.\\ Mốt của mẫu số liệu sau khi ghép nhóm xấp xỉ với mốt của mẫu số liệu không ghép nhóm. \\
		Một mẫu số liệu ghép nhóm có thể có nhiều nhóm chứa mốt và nhiều mốt.\\
		Do đó khẳng định sai là: Một mẫu số liệu ghép nhóm chỉ có một mốt.
	}    
\end{ex}
% \begin{ex}%[1K3Y9-4]
% 	Người ta ghi lại tuổi thọ của một số con ong cho kết quả như sau:
% 	\begin{center}
% 		\begin{tabular}{|l|c|c|c|c|c|c|}
% 			\hline Tuồi thọ (ngày) &{$[0;20)$}&{$[20;40)$}&{$[40;60)$}&{$[60;80)$}&{$[80;100)$}\\
% 			\hline Số lượng & $5$ & $12$ & $23$ & $31$ & $29$  \\
% 			\hline
% 		\end{tabular}
% 	\end{center}
% 	Nhóm chứa mốt của mẫu số liệu này là
% 	\choice
% 	{ $[20;40)$}
% 	{$[40;60)$}
% 	{\True $[60;80)$}
% 	{$[80;100)$}
% 	\loigiai{
% 		Nhóm chứa mốt của mẫu số liệu này là $[60;80)$.
% 	}    
% \end{ex}
% %6
% \begin{ex}%[1K3B9-4]
% 	Người ta ghi lại tuổi thọ của một số con ong cho kết quả như sau:
% 	\begin{center}
% 		\begin{tabular}{|l|c|c|c|c|c|c|}
% 			\hline Tuồi thọ (ngày) &{$[0;20)$}&{$[20;40)$}&{$[40;60)$}&{$[60;80)$}&{$[80;100)$}\\
% 			\hline Số lượng & $5$ & $12$ & $23$ & $31$ & $29$  \\
% 			\hline
% 		\end{tabular}
% 	\end{center}
% 	Mốt của mẫu số liệu này là
% 	\choice
% 	{ $M_0=70$}
% 	{$M_0=60$}
% 	{\True $M_0=76$}
% 	{$M_0=31$}
% 	\loigiai{
% 		Tần số lớn nhất là $31$ nên nhóm chứa mốt là nhóm $[60;80)$. \\
% 		Ta có, $j=4, a_4=60, m_4=31$, $m_3=23, m_5=29, h=20$. Do đó
% 		$$
% 		M_0=60+\frac{31-23}{(31-29)+(14-11)}\cdot 20 =76.
% 		$$
% 	}   
% \end{ex}
% %7
% \begin{ex}%[1K3Y9-4]
% 	Doanh thu bán hàng 20 ngày được lựa chọn ngẫu nhiên của một cửa hàng được ghi lại ở bảng sau (đơn vị: triệu đồng)
% 	\begin{center}
% 		\begin{tabular}{|l|c|c|c|c|c|c|}
% 			\hline Doanh thu &{$[5;7)$}&{$[7;9)$}&{$[9;11)$}&{$[11;13)$}&{$[13;15)$}\\
% 			\hline Số ngày & $2$ & $9$ & $7$ & $3$ & $1$ \\
% 			\hline
% 		\end{tabular}
% 	\end{center}
% 	Nhóm chứa mốt của mẫu số liệu này là
% 	\choice
% 	{ $[9;11)$}
% 	{$[5;7)$}
% 	{\True $[7;9)$}
% 	{$[11;13)$}
% 	\loigiai{
% 		Tần số lớn nhất là $9$ nên nhóm chứa mốt là nhóm $[7;9)$. \\
% 	}   
% \end{ex}
% %8
% \begin{ex}%[1K3B9-4]
% 	Doanh thu bán hàng 20 ngày được lựa chọn ngẫu nhiên của một cửa hàng được ghi lại ở bảng sau (đơn vị: triệu đồng)
% 	\begin{center}
% 		\begin{tabular}{|l|c|c|c|c|c|c|}
% 			\hline Doanh thu &{$[5;7)$}&{$[7;9)$}&{$[9;11)$}&{$[11;13)$}&{$[13;15)$}\\
% 			\hline Số ngày & $2$ & $9$ & $7$ & $3$ & $1$ \\
% 			\hline
% 		\end{tabular}
% 	\end{center}
% 	Xác định mốt của mẫu số liệu.
% 	\choice
% 	{ $M_0\approx 8$}
% 	{$M_0\approx 8,5$}
% 	{\True $M_0\approx 8,56$}
% 	{$M_0\approx 9$}
% 	\loigiai{
% 		Tần số lớn nhất là $9$ nên nhóm chứa mốt là nhóm $[7;9)$. \\
% 		Ta có, $j=2, a_2=7, m_2=9$, $m_1=2, m_3=7, h=2$. Do đó
% 		$$
% 		M_0=7+\frac{9-2}{(9-2)+(9-7)}\cdot 2\approx 8,56.
% 		$$
% 	}    
% \end{ex}
% %9
% \begin{ex}%[1K3B9-4]
% 	Điểm kiểm tra môn Toán của lớp 12A được cho trong bảng sau
% 	\begin{center}
% 		\begin{tabular}{|l|c|c|c|c|c|c|c|c|}
% 			\hline Khoảng điểm &{$[6,5;7)$}&{$[7;7,5)$}&{$[7,5;8)$}&{$[8;8,5)$}&{$[8,5;9)$}&{$[9;9,5)$}&{$[9,5;10)$}\\
% 			\hline Tần số & $8$ & $10$ & $16$ & $24$& $13$ & $7$ & $4$ \\
% 			\hline
% 		\end{tabular}
% 	\end{center}
% 	Xác định mốt của mẫu số liệu ghép nhóm này.    
% 	\choice
% 	{ $M_0\approx 8,4$}
% 	{$M_0\approx 8,5$}
% 	{\True $M_0\approx 8,21$}
% 	{$M_0\approx 24$}
% 	\loigiai{
% 		Tần số lớn nhất là $24$ nên nhóm chứa mốt là nhóm $[8;8,5)$. \\
% 		Ta có, $j=4, a_4=8, m_4=24$, $m_3=16, m_5=13, h=0,5$. Do đó
% 		$$
% 		M_0=8+\frac{24-16}{(24-16)+(24-13)}\cdot 0,5\approx 8,21.
% 		$$
% 	}
% \end{ex}
% %10
% \begin{ex}%[1K3Y9-4]
% 	Điểm kiểm tra môn Toán của lớp 12A được cho trong bảng sau
% 	\begin{center}
% 		\begin{tabular}{|l|c|c|c|c|c|c|c|c|}
% 			\hline Khoảng điểm &{$[6,5;7)$}&{$[7;7,5)$}&{$[7,5;8)$}&{$[8;8,5)$}&{$[8,5;9)$}&{$[9;9,5)$}&{$[9,5;10)$}\\
% 			\hline Tần số & $8$ & $10$ & $16$ & $24$& $13$ & $7$ & $4$ \\
% 			\hline
% 		\end{tabular}
% 	\end{center}
% 	Nhóm chứa mốt của mẫu số liệu này là
% 	\choice
% 	{ $[7;7,5)$}
% 	{$[7,5;8)$}
% 	{\True $[8;8,5)$}
% 	{$[8,5;9)$}
% 	\loigiai{
% 		Tần số lớn nhất là $24$ nên nhóm chứa mốt là nhóm $[8,5;9)$. \\
% 	}       
% \end{ex}
% %11
% \begin{ex}%[1K3Y9-4]
% 	Để kiểm tra thời gian sử dụng pin của   chiếc điện thoại mới, bạn A thống kê thời gian sử dụng điện thoại của mình từ lúc sạc đầy cho đến khi hết pin ở bảng sau
% 	\begin{center}
% 		\begin{tabular}{|l|c|c|c|c|c|c|c|c|}
% 			\hline Thời gian sử dụng (giờ) &{$[7;9)$}&{$[9;11)$}&{$[11;13)$}&{$[13;15)$}&{$[15;17)$}\\
% 			\hline Số lần & $2$ & $5$ & $7$ & $6$& $3$  \\
% 			\hline
% 		\end{tabular}
% 	\end{center}
% 	Nhóm chứa mốt của mẫu số liệu này là
% 	\choice
% 	{ $[9;11)$}
% 	{\True $[11;13)$}
% 	{ $[13;15)$}
% 	{$[15;17)$}
% 	\loigiai{
% 		Tần số lớn nhất là $7$ nên nhóm chứa mốt là nhóm $[11;13)$. \\
% 	}        
% \end{ex}
% %12
% \begin{ex}%[1K3B9-4]
% 	Để kiểm tra thời gian sử dụng pin của   chiếc điện thoại mới, bạn A thống kê thời gian sử dụng điện thoại của mình từ lúc sạc đầy cho đến khi hết pin ở bảng sau
% 	\begin{center}
% 		\begin{tabular}{|l|c|c|c|c|c|c|c|c|}
% 			\hline Thời gian sử dụng (giờ) &{$[7;9)$}&{$[9;11)$}&{$[11;13)$}&{$[13;15)$}&{$[15;17)$}\\
% 			\hline Số lần & $2$ & $5$ & $7$ & $6$& $3$  \\
% 			\hline
% 		\end{tabular}
% 	\end{center}   
% 	Xác định mốt của mẫu số liệu ghép nhóm này.    
% 	\choice
% 	{ $M_0\approx 11,67$}
% 	{$M_0\approx 12$}
% 	{\True $M_0\approx 12,33$}
% 	{$M_0\approx 7$}
% 	\loigiai{
% 		Tần số lớn nhất là $7$ nên nhóm chứa mốt là nhóm $[11;13)$. \\
% 		Ta có, $j=3, a_3=11, m_3=7$, $m_2=5, m_4=6, h=2$. Do đó
% 		$$
% 		M_0=11+\frac{7-5}{(7-5)+(7-6)}\cdot 2\approx 12,33.
% 		$$
% 	}
% \end{ex}
% %13
% \begin{ex}%[1K3Y9-4]
% 	Tổng lượng mưa trong tháng 8 đo được tại một trạm quan trắc đặt tại Vũng Tàu từ năm 2002 đến năm 2020 được ghi lại như sau (đơn vị: mm)
% 	\begin{center}
% 		\begin{tabular}{|l|c|c|c|c|c|c|c|c|}
% 			\hline Tổng lượng mưa trong tháng 8 (mm) &{$[120;175)$}&{$[175;230)$}&{$[230;285)$}&{$[285;340)$}\\
% 			\hline Số năm & $10$ & $5$ & $3$ & $1$  \\
% 			\hline
% 		\end{tabular}
% 	\end{center}  
% 	Nhóm chứa mốt của mẫu số liệu này là
% 	\choice
% 	{ $[175;230)$}
% 	{ $[230;285)$}
% 	{\True $[120;175)$}
% 	{$[285;340)$}
% 	\loigiai{
% 		Tần số lớn nhất là $10$ nên nhóm chứa mốt là nhóm $[120;175)$. \\
% 	}        
% \end{ex}
% %14
% \begin{ex}%[1K3B9-4]
% 	Tổng lượng mưa trong tháng 8 đo được tại một trạm quan trắc đặt tại Vũng Tàu từ năm 2002 đến năm 2020 được ghi lại như sau (đơn vị: mm)
% 	\begin{center}
% 		\begin{tabular}{|l|c|c|c|c|c|c|c|c|}
% 			\hline Tổng lượng mưa trong tháng 8 (mm) &{$[120;175)$}&{$[175;230)$}&{$[230;285)$}&{$[285;340)$}\\
% 			\hline Số năm & $10$ & $5$ & $3$ & $1$  \\
% 			\hline
% 		\end{tabular}
% 	\end{center}  
% 	Xác định mốt của mẫu số liệu ghép nhóm này.    
% 	\choice
% 	{ $M_0\approx 172,25$}
% 	{$M_0\approx 146,125$}
% 	{\True $M_0\approx 156,67$}
% 	{$M_0\approx 10$}
% 	\loigiai{
% 		Tần số lớn nhất là $10$ nên nhóm chứa mốt là nhóm $[120;175)$. \\
% 		Ta có, $j=1, a_1=120, m_1=10$, $m_2=5, m_0=0, h=55$. Do đó
% 		$$
% 		M_0=120+\frac{10-0}{(10-0)+(10-5)}\cdot 55\approx 156.67.
% 		$$
% 	}
% \end{ex}
% %15
% \begin{ex}%[1K3B9-4]
% 	Một công ty xây dựng khảo sát khách hàng xem họ có nhu cầu mua nhà ở mức giá nào. Kết quả khảo sát được ghi lại ở bảng sau (đơn vị: triệu đồng/$\mathrm{m}^2$
% 	\begin{center}
% 		\begin{tabular}{|l|c|c|c|c|c|c|c|c|}
% 			\hline Mức giá &{$[10;14)$}&{$[14;18)$}&{$[18;22)$}&{$[22;26)$}&{$[26;30)$}\\
% 			\hline Số khách hàng & $54$ & $78$ & $120$ & $45$& $12$  \\
% 			\hline
% 		\end{tabular}
% 	\end{center}   
% 	Xác định mốt của mẫu số liệu ghép nhóm này.    
% 	\choice
% 	{ $M_0\approx 18$}
% 	{\True $M_0\approx 19,4$}
% 	{ $M_0\approx 20$}
% 	{$M_0\approx 120$}
% 	\loigiai{
% 		Tần số lớn nhất là $120$ nên nhóm chứa mốt là nhóm $[18;22)$. \\
% 		Ta có, $j=3, a_3=18, m_3=120$, $m_2=78, m_4=45, h=4$. Do đó
% 		$$
% 		M_0=18+\frac{120-78}{(120-78)+(120-45)}\cdot 4\approx 19,4.
% 		$$
% 	}
% \end{ex}
% \begin{ex}%[1K3K9-4]
% 	Một công ty xây dựng khảo sát khách hàng xem họ có nhu cầu mua nhà ở mức giá nào. Kết quả khảo sát được ghi lại ở bảng sau (đơn vị: triệu đồng/$\mathrm{m}^2$
% 	\begin{center}
% 		\begin{tabular}{|l|c|c|c|c|c|c|c|c|}
% 			\hline Mức giá &{$[10;14)$}&{$[14;18)$}&{$[18;22)$}&{$[22;26)$}&{$[26;30)$}\\
% 			\hline Số khách hàng & $54$ & $78$ & $120$ & $45$& $12$  \\
% 			\hline
% 		\end{tabular}
% 	\end{center}   
% 	Công ty nên xây nhà ở mức giá nào để nhiều người có nhu cầu mua nhất?    
% 	\choice
% 	{ $ 18$ triệu đồng/$\mathrm{m}^2$}
% 	{\True $19,4$ triệu đồng/$\mathrm{m}^2$}
% 	{ $20$ triệu đồng/$\mathrm{m}^2$}
% 	{$21$ triệu đồng/$\mathrm{m}^2$}
% 	\loigiai{
% 		Tần số lớn nhất là $120$ nên nhóm chứa mốt là nhóm $[18;22)$. \\
% 		Ta có, $j=3, a_3=18, m_3=120$, $m_2=78, m_4=45, h=4$. Do đó
% 		$$
% 		M_0=18+\frac{120-78}{(120-78)+(120-45)}\cdot 4\approx 19,4.
% 		$$
% 		Dựa vào kết quả trên ta dự đoán rằng nếu công ty xây nhà ở mức giá $19,4$ triệu đồng/$\mathrm{m}^2$ thì sẽ có nhiều người có nhu cầu mua nhất.
% 	}
% \end{ex}
% \begin{ex}%[1K3Y9-4]
% 	Số cuộc gọi điện thoại một người thực hiện mỗi ngày trong 30 ngày được lựa chọn ngẫu nhiên được thống kê trong bảng sau
% 	\begin{center}
% 		\begin{tabular}{|l|c|c|c|c|c|c|c|c|}
% 			\hline Số cuộc gọi &{$[2,5;5,5)$}&{$[5,5;8,5)$}&{$[8,5;11,5)$}&{$[11,5;14,5)$}&{$[14,5;17,5)$}\\
% 			\hline Số ngày & $5$ & $13$ & $7$ & $3$& $2$  \\
% 			\hline
% 		\end{tabular}
% 	\end{center}   
% 	Nhóm chứa mốt của mẫu số liệu này là
% 	\choice
% 	{ $[2,5;5,5)$}
% 	{\True $[5,5;8,5)$}
% 	{ $[8,5;11,5)$}
% 	{$[11,5;14,5)$}
% 	\loigiai{
% 		Tần số lớn nhất là $13$ nên nhóm chứa mốt là nhóm $[5,5;8,5)$. \\
% 	}
% \end{ex}
% \begin{ex}%[1K3K9-4]
% 	Số cuộc gọi điện thoại một người thực hiện mỗi ngày trong 30 ngày được lựa chọn ngẫu nhiên được thống kê trong bảng sau
% 	\begin{center}
% 		\begin{tabular}{|l|c|c|c|c|c|c|c|c|}
% 			\hline Số cuộc gọi &{$[2,5;5,5)$}&{$[5,5;8,5)$}&{$[8,5;11,5)$}&{$[11,5;14,5)$}&{$[14,5;17,5)$}\\
% 			\hline Số ngày & $5$ & $13$ & $7$ & $3$& $2$  \\
% 			\hline
% 		\end{tabular}
% 	\end{center}   
% 	Hãy dự đoán xem khả năng người đó thực hiện bao nhiêu cuộc gọi mỗi ngày là cao nhất?   
% 	\choice
% 	{ $5$}
% 	{\True $7$}
% 	{ $6$}
% 	{$8$}
% 	\loigiai{
% 		Tần số lớn nhất là $13$ nên nhóm chứa mốt là nhóm $[5,5;8,5)$. \\
% 		Ta có, $j=2, a_2=5,5, m_2=13$, $m_1=5, m_3=7, h=3$. Do đó
% 		$$
% 		M_0=5,5+\frac{13-5}{(13-5)+(13-7)}\cdot 3\approx 7,2.
% 		$$
% 		Do đó ta có thể dự đoán khả năng người đó thực hiện $7$ cuộc gọi mỗi ngày là cao nhất.
% 	}
% \end{ex}

\Closesolutionfile{ans}

%GK1
% \begin{name}
	{\tenchude}
	{TOÁN 11}
	{LỚP TOÁN THẦY PHÁT}
	{Thời gian: 90 phút - Không kể thời gian phát đề}
\end{name}
\setcounter{ex}{0}\setcounter{bt}{0}
\noindent{\bf\fontfamily{qag}\selectfont\color{violet}A. PHẦN TRẮC NGHIỆM}
\Opensolutionfile{ans}[ans/ans-1-GK1-KNTT-14-NH23-24]
\begin{ex}%[0H3B6-2]% ::Cau 1::
	Với góc $\alpha $ bất kì, đẳng thức nào sau đây là đúng?
	\choice
	{$\cos \left( \pi-\alpha \right)=\cos \alpha $}
	{\True $\cos \left( \pi-\alpha \right)=-\cos \alpha $}
	{$\sin \left( \pi-\alpha \right)=-\sin \alpha $}
	{$\tan \left( \pi-\alpha \right)=\tan \alpha $}
	\loigiai{
		Ta có $\cos \left( \pi-\alpha \right)=-\cos \alpha $, $\sin \left( \pi-\alpha \right)=\sin \alpha $, $\tan \left( \pi-\alpha \right)=-\tan \alpha $.\\
		Do đó ta chọn phương án $\cos \left( \pi-\alpha \right)=-\cos \alpha $.}
\end{ex}
\begin{ex}%[0H3B5-2]% ::Cau 2::
	Biết góc $\alpha $ thỏa mãn $\cos \alpha =\dfrac{2}{3}$. Hỏi $\alpha $ có thể nhận giá trị trong khoảng nào dưới đây?
	\choice
	{$\left( \dfrac{\pi}{2};\dfrac{2\pi}{3} \right)$}
	{$\left( \dfrac{8\pi}{3};\dfrac{17\pi}{6} \right)$}
	{\True $\left( \dfrac{\pi}{4};\dfrac{\pi}{3} \right)$}
	{$\left( -\pi;-\dfrac{2\pi}{3} \right)$}
	\loigiai{
		Vì $\cos \alpha =\dfrac{2}{3}$ nên $\alpha \in \left( -\dfrac{\pi}{2}+k2\pi,\dfrac{\pi}{2}+k2\pi \right)$ với $k\in \mathbb{Z}$.\\
		Với $k=0$ thì $\alpha\in \left(-\dfrac{\pi}{2};\dfrac{\pi}{2}\right)$. Vì $\left( \dfrac{\pi}{4};\dfrac{\pi}{3} \right)\subset \left(-\dfrac{\pi}{2};\dfrac{\pi}{2}\right)$.\\
		Do đó, ta chọn phương án $\left( \dfrac{\pi}{4};\dfrac{\pi}{3} \right)$.}
\end{ex}
\begin{ex}%[0H3Y5-2]% ::Cau 3::
	Cho góc $\alpha $ thỏa $-\dfrac{3\pi}{2}<\alpha <-\pi$. Tìm mệnh đề đúng trong các mệnh đề sau:
	\choice
	{$\cos \alpha >0$}
	{$\cot \alpha >0$}
	{\True $\sin \alpha >0$}
	{$\tan \alpha >0$}
	\loigiai{
		\begin{center}
			\begin{tikzpicture}[scale=1.2,font=\footnotesize,line join=round,line cap=round,>=stealth]
				\path 	
				(0,0) coordinate (O)
				(1,0) coordinate (A)
				(-1,0) coordinate (A')
				(0,1) coordinate (B)
				(0,-1) coordinate (B')
				;		
				\draw[-stealth] (-1.5,0)--(1.5,0)node[below]{$\cos x$};
				\draw[-stealth] (0,-1.5)--(0,1.5)node[left]{$\sin x$};
				\draw (O) node[below left]{$O$}  circle (1);
				\draw (1,1) node{I} (-1,1) node{II} (-1,-1) node{III} (1,-1)node{IV} ;
%				\foreach \x/\g in {A/30,B/45,A'/120,B'/-30}
%				\fill[black] 	(\x) circle (1pt)
%				($(\g:3mm)+(\x)$) node {$\x$};
			\end{tikzpicture}
		\end{center}
		Do $-\dfrac{3\pi}{2}<\alpha <-\pi$ nên điểm $M$ biểu diễn góc lượng giác có số đo $\alpha $ thuộc góc phần tư số II. \\
		Do đó $\sin \alpha >0$, $\cos \alpha <0$, $\tan \alpha <0$, $\cot \alpha <0$.}
\end{ex}
\begin{ex}%[0H3B5-2]% ::Cau 4::
	Cho $\cot \alpha =4\tan \alpha $ và $\alpha \in \left( \dfrac{\pi}{2};\pi \right)$. Khi đó $\sin \alpha $ bằng
	\choice
	{$-\dfrac{\sqrt{5}}{5}$}
	{$\dfrac{1}{2}$}
	{$\dfrac{2\sqrt{5}}{5}$}
	{\True $\dfrac{\sqrt{5}}{5}$}
	\loigiai{
		Ta có 
		\allowdisplaybreaks
		\begin{eqnarray*}
		&& \cot \alpha =4\tan \alpha \\
		&\Leftrightarrow& \dfrac{\cot \alpha }{\tan \alpha }=4\\
		&\Leftrightarrow& \cot^2 \alpha =4\Leftrightarrow 1+\cot^2 \alpha =5\\
		&\Leftrightarrow& \dfrac{1}{{\sin ^2}\alpha }=5\\
		&\Leftrightarrow& {\sin ^2}\alpha =\dfrac{1}{5}\\
		&\Leftrightarrow& \sin \alpha =\pm \dfrac{\sqrt{5}}{5}.
		\end{eqnarray*}
		Vì $\alpha \in \left( \dfrac{\pi}{2};\pi \right)$ nên $\sin \alpha =\dfrac{\sqrt{5}}{5}$.}
\end{ex}
\begin{ex}%[0H3Y6-2]% ::Cau 5::
	Khẳng định nào sau đây là \textbf{sai}?
	\choice
	{$\cos 2a=2\cos^2 a-1$}
	{$\cos 2a=\cos^2 a-{\sin ^2}a$}
	{$\sin 2a=2\sin a\cos a$}
	{\True $\cos 2a=2{\sin ^2}a-1$}
	\loigiai{
		Theo công thức nhân đôi ta có $\cos 2a=1-2{\sin ^2}a$.}
\end{ex}
\begin{ex}%[0H3Y6-2]% ::Cau 6::
	Khẳng định nào sau đây là \textbf{sai}?
	\choice
	{$\cos a+\cos b=2\cos \dfrac{a+b}{2}\cos \dfrac{a-b}{2}$}
	{\True $\cos a-\cos b=2\sin \dfrac{a+b}{2}\sin \dfrac{a-b}{2}$}
	{$\sin a+\sin b=2\sin \dfrac{a+b}{2}\cos \dfrac{a-b}{2}$}
	{$\sin a-\sin b=2\cos \dfrac{a+b}{2}\sin \dfrac{a-b}{2}$}
	\loigiai{
		Theo công thức biến tổng thành tích ta có $\cos a-\cos b=-2\sin \dfrac{a+b}{2}\cdot\sin \dfrac{a-b}{2}$.}
\end{ex}
\begin{ex}%[0H3B5-2]% ::Cau 7::
	Cho $\tan \alpha +\cot \alpha =m$. Tính giá trị của biểu thức $\tan^3 \alpha +\cot^3 \alpha $
	\choice
	{$m^3+3m$}
	{$3m^3+m$}
	{$3m^3-m$}
	{\True $m^3-3m$}
	\loigiai{
		Ta có
		\allowdisplaybreaks
		\begin{eqnarray*}
		\tan^3 \alpha +\cot^3 \alpha &=& \left( \tan \alpha +\cot \alpha \right)\left( \tan^2 \alpha -\tan \alpha \cdot\cot \alpha +\cot^2 \alpha \right)\\
		&=& \left( \tan \alpha +\cot \alpha \right)\left[ \left( \tan \alpha +\cot \alpha \right)^2-3\tan \alpha \cdot\cot \alpha \right]\\
		&=& \left( \tan \alpha +\cot \alpha \right)\cdot\left[ \left( \tan \alpha +\cot \alpha \right)^2-3 \right]\\
		&=& m\left( m^2-3 \right)=m^3-3m.
		\end{eqnarray*}
	}
\end{ex}
\begin{ex}%[1D1Y1-1]% ::Cau 8::
	Tìm tập xác định $\mathscr{D}$ của hàm số $y=\dfrac{2023}{\sin x}$.
	\choice
	{$\mathscr{D}=\mathbb{R}\setminus \left\{\dfrac{\pi}{2}+k\pi\,\middle|\, k\in \mathbb{Z} \right\}$}
	{$\mathscr{D}=\mathbb{R}\setminus \left\{\dfrac{k\pi}{2} \,\middle|\, k\in \mathbb{Z} \right\}$}
	{$\mathscr{D}=\mathbb{R}\setminus \left\{0 \right\}$}
	{\True $D=\mathbb{R}\setminus \left\{k\pi\,\middle|\, k\in \mathbb{Z} \right\}$}
	\loigiai{
		Điều kiện xác định $\sin x\ne 0\Leftrightarrow x\ne k\pi,k\in \mathbb{Z}$.\\
	Vậy $\mathscr{D}=\mathbb{R}\setminus \left\{k\pi\,\middle|\, k\in \mathbb{Z}  \right\}$.}
\end{ex}
\begin{ex}%[1D1Y1-3]% ::Cau 9::
	Cho hàm số $y=\tan x$. Khẳng định sau đây là \textbf{sai}?
	\choice
	{\True Hàm số đã cho là hàm số chẵn}
	{Tập xác định của hàm số đã cho là $\mathbb{R}\setminus \left\{\dfrac{\pi}{2}+k\pi\,\middle|\, k\in \mathbb{Z} \right\}$}
	{Hàm số đã cho đồng biến trên mỗi khoảng $\left( -\dfrac{\pi}{2}+k\pi;\dfrac{\pi}{2}+k\pi \right)$ với $k\in \mathbb{Z}$}
	{Hàm số đã cho tuần hoàn theo chu kì $\pi$}
	\loigiai{
	Hàm số $y=\tan x$ là hàm số lẻ.
	}
\end{ex}
\begin{ex}%[1D1Y1-6]% ::Cau 10::
	Trong các hàm số sau, hàm số nào có đồ thị như hình vẽ bên dưới?
	\begin{center}
		\begin{tikzpicture}[scale=1,font=\footnotesize,line join=round,line cap=round,>=stealth]		
			\draw[-stealth] (-6.7,0)--(6.7,0)node[below]{$x$};
			\draw[-stealth] (0,-1.2)--(0,2)node[left]{$y$};
			\draw (0,0) node[below right]{$O$};
			\draw[thick,smooth,samples=200] plot[domain=-6.6:6.6] (\x,{sin (\x*180/pi)});
			\draw[fill=black]  (-6.28,0) circle (1pt) node [above left] {$-2\pi$};
			\draw[fill=black]  (-4.71,0) circle (1pt) node [below] {$-\dfrac{3\pi}{2}$};
			\draw[fill=black]  (-3.14,0) circle (1pt) node [above] {$-\pi$};
			\draw[fill=black] (-1.57,0) circle (1pt) node [above] {$-\dfrac{\pi}{2}$};
			\draw[fill=black]  (1.57,0) circle (1pt) node [below] {$\dfrac{\pi}{2}$};
			\draw[fill=black]  (3.14,0) circle (1pt) node [above] {$\pi$};
			\draw[fill=black]  (4.71,0) circle (1pt) node [above] {$\dfrac{3\pi}{2}$};
			\draw[fill=black]  (6.28,0) circle (1pt) node [above] {$2\pi$};
			\draw (0,1) node[above right] {$1$} (0,-1) node[below right] {$-1$};
			\draw[dashed] (-6.7,-1)--(6.7,-1) (-6.7,1)--(6.7,1);
			\draw[dashed] (-1.57,0)--(-1.57,-1) (1.57,0)--(1.57,1) (4.71,0)--(4.71,-1) (-4.71,0)--(-4.71,1);
		\end{tikzpicture}
	\end{center}
	\choice
	{\True $y=\sin x$}
	{$y=\cos x$}
	{$y=\tan x$}
	{$y=\cot x$}
	\loigiai{
		Từ hình vẽ ta thấy hàm số có miền giá trị từ $-1$ đến $1$, tuần hoàn với chu kỳ $2\pi$ và nhận gốc tọa độ làm tâm đối xứng nên đây là đồ thị của hàm số $y=\sin x$.}
\end{ex}
\begin{ex}%[1D1B1-4]% ::Cau 11::
	Tìm chu kì $T$ của hàm số $y=2\cos^2 x+2023$.
	\choice
	{$T=3\pi$}
	{$T=2\pi$}
	{\True $T=\pi$}
	{$T=4\pi$}
	\loigiai{
		Ta có $y=2\cos^2 x+2023=\cos 2x+2024$.\\
		Vậy hàm số có chu kì $T=\pi$.}
\end{ex}
\begin{ex}%[1D1K1-1]% ::Cau 12::
	Tập xác định $\mathscr{D}$ của hàm số $y=\sqrt{4+\sin x}-\dfrac{1+x}{\tan^2 \left( x-\dfrac{\pi}{4} \right)-1}+3\tan \left( x+\dfrac{\pi}{4} \right)$ là
	\choice
	{$\mathscr{D}=\mathbb{R}\setminus \left\{\dfrac{\pi}{4}+k\dfrac{\pi}{2},k\in \mathbb{Z} \right\}$}
	{$\mathscr{D}=\mathbb{R}\setminus \left\{k\dfrac{\pi}{2},k\in \mathbb{Z} \right\}$}
	{\True $\mathscr{D}=\mathbb{R}\setminus \left\{k\dfrac{\pi}{4},k\in \mathbb{Z} \right\}$}
	{$\mathscr{D}=\mathbb{R}\setminus \left\{\dfrac{\pi}{4}+k\pi,k\in \mathbb{Z} \right\}$}
	\loigiai{
		Do $-1\le \sin x\le 1$ nên $4+\sin x>0$, $\,\forall x\in \mathbb{R}$.\\
		Hàm số xác định khi và chỉ khi\\
		$\heva{
			& x-\dfrac{\pi}{4}\ne \dfrac{\pi}{2}+m\pi\\
			& \tan^2 \left( x-\dfrac{\pi}{4} \right)\ne 1 \\
			& x+\dfrac{\pi}{4}\ne \dfrac{\pi}{2}+q\pi\\
		}\Leftrightarrow \heva{
			& x\ne \dfrac{3\pi}{4}+m\pi\\
			& x-\dfrac{\pi}{4}\ne \dfrac{\pi}{4}+n\pi\\
			& x-\dfrac{\pi}{4}\ne -\dfrac{\pi}{4}+p\pi\\
			& x\ne \dfrac{\pi}{4}+q\pi\\
		}\Leftrightarrow \heva{
			& x\ne \dfrac{3\pi}{4}+m\pi\\
			& x\ne \dfrac{\pi}{2}+n\pi\\
			& x\ne p\pi\\
			& x\ne \dfrac{\pi}{4}+q\pi\\
		}\Leftrightarrow x\ne k\dfrac{\pi}{4},\,\left( m,n,p,q,k\in \mathbb{Z} \right)$.\\
		Vậy tập xác định $\mathscr{D}=\mathbb{R}\setminus \left\{k\dfrac{\pi}{4},k\in \mathbb{Z} \right\}$.}
\end{ex}
\begin{ex}%[1D1B2-1]% ::Cau 13::
	Giải phương trình $\cos \left( x-\dfrac{\pi}{6} \right)=\dfrac{1}{2}$.
	\choice
	{$\hoac{
			& x=\dfrac{\pi}{3}+k2\pi\\
			& x=k2\pi\\
		}\,\left( k\in \mathbb{Z} \right)$}
	{$\hoac{
			& x=\dfrac{\pi}{2}+k\pi\\
			& x=-\dfrac{\pi}{6}+k\pi\\
		}\left( k\in \mathbb{Z} \right)$}
	{$\hoac{
			& x=\dfrac{\pi}{2}+k2\pi\\
			& x=\dfrac{\pi}{6}+k2\pi\\
		}\left( k\in \mathbb{Z} \right)$}
	{\True $\hoac{
			& x=\dfrac{\pi}{2}+k2\pi\\
			& x=-\dfrac{\pi}{6}+k2\pi\\
		}\left( k\in \mathbb{Z} \right)$}
	\loigiai{
		Ta có $\cos \left( x-\dfrac{\pi}{6} \right)=\dfrac{1}{2}\Leftrightarrow \hoac{
			& x-\dfrac{\pi}{6}=\dfrac{\pi}{3}+k2\pi\\
			& x-\dfrac{\pi}{6}=-\dfrac{\pi}{3}+k2\pi\\
		}\Leftrightarrow \hoac{
			& x=\dfrac{\pi}{2}+k2\pi\\
			& x=-\dfrac{\pi}{6}+k2\pi\\
		}\left( k\in \mathbb{Z} \right)$.}
\end{ex}
\begin{ex}%[1D1B2-1]% ::Cau 14::
	Phương trình $\sin 2x=-\dfrac{1}{2}$ có tập nghiệm là
	\choice
	{$\hoac{
			& x=\dfrac{7\pi}{12}+k\pi\\
			& x=-\dfrac{7\pi}{12}+k\pi\\
		}\left( k\in \mathbb{Z} \right)$}
	{$\hoac{
			& x=\dfrac{\pi}{12}+k2\pi\\
			& x=-\dfrac{7\pi}{12}+k2\pi\\
		}\left( k\in \mathbb{Z} \right)$}
	{$\hoac{
			& x=\dfrac{\pi}{12}+k\pi\\
			& x=\dfrac{7\pi}{12}+k\pi\\
		}\left( k\in \mathbb{Z} \right)$}
	{\True $\hoac{
			& x=-\dfrac{\pi}{12}+k\pi\\
			& x=\dfrac{7\pi}{12}+k\pi\\
		}\left( k\in \mathbb{Z} \right)$}
	\loigiai{
		Ta có $\sin 2x=-\dfrac{1}{2}\Leftrightarrow \hoac{
			& 2x=-\dfrac{\pi}{6}+k2\pi\\
			& 2x=\dfrac{7\pi}{6}+k2\pi\\
		}\Leftrightarrow \hoac{
			& x=-\dfrac{\pi}{12}+k\pi\\
			& x=\dfrac{7\pi}{12}+k\pi\\
		}\left( k\in \mathbb{Z} \right)$.}
\end{ex}
\begin{ex}%[1D1B2-1]% ::Cau 15::
	Phương trình $\sin \left( 2x-\dfrac{\pi}{4} \right)=\sin \left( x+\dfrac{3\pi}{4} \right)$ có tổng các nghiệm thuộc khoảng $\left( 0;\pi \right)$ bằng
	\choice
	{$\dfrac{7\pi}{2}$}
	{\True $\pi$}
	{$\dfrac{3\pi}{2}$}
	{$\dfrac{\pi}{4}$}
	\loigiai{
		Ta có $$\sin \left( 2x-\dfrac{\pi}{4} \right)=\sin \left( x+\dfrac{3\pi}{4} \right)\Leftrightarrow \hoac{
			& 2x-\dfrac{\pi}{4}=x+\dfrac{3\pi}{4}+k2\pi\\
			& 2x-\dfrac{\pi}{4}=\dfrac{\pi}{4}-x+l2\pi}
		\Leftrightarrow \hoac{
			& x=\pi+k2\pi\\
			& x=\dfrac{\pi}{6}+l\dfrac{2\pi}{3}},\left( k,l\in \mathbb{Z} \right).$$
		Họ nghiệm $x=\pi+k2\pi$ không có nghiệm nào thuộc khoảng $\left( 0;\pi \right)$.\\
		Với $x=\dfrac{\pi}{6}+l\dfrac{2\pi}{3}\in \left( 0;\pi \right)\Rightarrow 0<\dfrac{\pi}{6}+l\dfrac{2\pi}{3}<\pi\Leftrightarrow l\in \left\{0;1 \right\}$.\\
		Vậy phương trình có hai nghiệm thuộc khoảng $\left( 0;\pi \right)$ là $x=\dfrac{\pi}{6}$ và $x=\dfrac{5\pi}{6}$. \\
		Từ đó suy ra tổng các nghiệm thuộc khoảng $\left( 0;\pi \right)$ của phương trình này bằng $\pi$.}
\end{ex}
\begin{ex}
	Khai triển $\cos 4 \alpha$ theo $\cos \alpha$ ta được biểu thức $a\cos^4 \alpha +b\cos^2 \alpha +c$. Giá trị biểu thức $a-b+c$ bằng
	\choice
	{$13$}
	{\True $17$}
	{$-11$}
	{$-15$}
	\loigiai{
Ta có $\cos 4 \alpha =2\cos^2 2\alpha -1= 2 \left(2\cos^2 \alpha -1\right)^2 -1 = 8\cos^4 \alpha -8\cos^2 \alpha +1$.
}
\end{ex}
\begin{ex}%[1K1K4-3]%Câu 27
    Cho hai phương trình $\cos 3x-1=0$; $\cos 2x=-\dfrac{1}{2}$. Các giá trị nào dưới đây là nghiệm chung của hai phương trình đã cho?
    \choice
        {$x=\dfrac{\pi }{3}+k2\pi, k\in \mathbb{Z}$}
        {$x=k2\pi, k\in \mathbb{Z}$}
        {$x=\pm \dfrac{\pi }{3}+k2\pi, k\in \mathbb{Z}$}
        {\True $x=\pm \dfrac{2\pi }{3}+k2\pi, k\in \mathbb{Z}$}
    \loigiai{
        Ta có 
        \begin{itemize}
            \item $\cos 3x-1=0\Leftrightarrow \cos 3x=1
            \Leftrightarrow x=k\dfrac{2\pi }{3}, k\in \mathbb{Z}$.
            \item $\cos 2x=-\dfrac{1}{2}
            \Leftrightarrow 2x=\pm \dfrac{2\pi }{3}+k2\pi \Leftrightarrow x=\pm \dfrac{\pi }{3}+k\pi,k\in \mathbb{Z}.$
        \end{itemize}
        Biểu diễn các nghiệm trên đường tròn lượng giác ta có tập các nghiệm của hai phương trình là $x=\pm \dfrac{2\pi }{3}+k\pi, k\in \mathbb{Z}$.}
\end{ex}
\begin{ex}%[1K1B2-3]%Câu 21
    Biểu thức $\dfrac{\sin 10^\circ +\sin 20^\circ}{\cos 10^\circ +\cos 20^\circ}$ bằng $a\tan b$ với $a,b\in \mathbb{N}$ và $b\in [0;180]$. Tính $a+b$. 
    \choice
        {$88$}
        {$69$}
        {$29$}
        {\True $16$}
    \loigiai{
        $\dfrac{\sin 10^\circ+\sin 20^\circ}{\cos 10^\circ+\cos 20^\circ}=\dfrac{2\sin 15^\circ\cos 5^\circ}{2\cos 15^\circ\cos 5^\circ}=\tan 15^\circ$.}
\end{ex}
\begin{ex}%[1D3Y4-3]% ::Cau 16::
	Cho cấp số nhân $(u_n)$ có số hạng đầu $u_1=4$ và công bội $q=2$. Số hạng thứ $10$ của cấp số nhân đó là
	\choice
	{$u_{10}=2^{12}$}
	{\True $u_{10}=2^{11}$}
	{$u_{10}=2^{10}$}
	{$u_{10}=2^9$}
	\loigiai{
	Ta có $u_{10}=u_1\cdot q^{10-1}=4\cdot 2^9=2^2\cdot 2^9=2^{11}$.
	}
\end{ex}
\begin{ex}%[1D3B4-2]% ::Cau 17::
	Cho cấp số nhân $(u_n)$ với $u_1=-3;u_6=96$. Công bội của cấp số nhân đó là
	\choice
	{\True $q=-2$}
	{$q=-3$}
	{$q=2$}
	{$q=3$}
	\loigiai{
	Ta có $u_6=u_1q^5 \Rightarrow q^5=\dfrac{u_6}{u_1}=\dfrac{96}{-3}=-32$, suy ra $q=-2$.}
\end{ex}
\begin{ex}%::Cau 18::
	Công ty muốn ước lượng tỉ lệ các cỡ áo khi may cho học sinh lớp 11 đã đo chiều cao của 36 học sinh nam khối 11 của một trường và thu được mẫu số liệu sau (đơn vị là centimét):
	\begin{center}
		\begin{tabular}{lllllllllllll}
			160 & 161 & 161 & 162 & 162 & 162 & 163 & 163 & 163 & 164 & 164 & 164 & 164 \\ 
			165 & 165 & 165 & 165 & 165 & 166 & 166 & 166 & 166 & 167 & 167 & 168 & 168 \\ 
			168 & 168 & 169 & 169 & 170 & 171 & 171 & 172 & 172 & 174 & & & 
		\end{tabular}
	\end{center}
	Biết rằng học sinh có chiều cao thuộc $[160;167)$ sẽ mua cỡ áo M. Có bao nhiêu học sinh mua cỡ áo M?
	\choice
	{\True $22$}
	{$6$}
	{$15$}
	{$20$}
	\loigiai{
	Bảng tần số ghép nhóm
			\begin{center}
				\begin{tabular}{|c|c|c|c|c|c|}
					\hline Chiều cao $(\mathrm{cm})$ & {$[150 ; 160)$} & {$[160 ; 167)$} & {$[167 ; 170)$} & {$[170 ; 175)$} & {$[175 ; 180)$} \\
					\hline Số học sinh &0& 22 & 8 & 6 & 0 \\
					\hline
				\end{tabular}
			\end{center}
	}
\end{ex}
\begin{ex}%:Cau 19::
	Tìm tứ phân vị thứ nhất và thứ ba (làm tròn đến hàng phần chục) của mẫu số liệu sau 
	\begin{center}
		\begin{tabular}{|c|c|c|c|c|c|}
			\hline Chiều cao $(\mathrm{cm})$ & {$[150 ; 160)$} & {$[160 ; 167)$} & {$[167 ; 170)$} & {$[170 ; 175)$} & {$[175 ; 180)$} \\
			\hline Số học sinh &5& 17 & 8 & 6 & 0 \\
			\hline
		\end{tabular}
	\end{center}
	\choice
	{$Q_1 \approx 159{,}2$, $Q_3 \approx 169{,}8$}
	{\True $Q_1 \approx 161{,}6$, $Q_3 \approx 168{,}9$}
	{$Q_1 \approx 160{,}2$, $Q_3 \approx 170{,}3$}
	{$Q_1 \approx 163{,}6$, $Q_3 \approx 171{,}4$}
	\loigiai{
	\begin{itemize}
		\item $\dfrac{n}{4}=9$, $p=2$, $a_2=160$, $m_1=5$, $m_2=17$, $a_3-a_2=7$. $$Q_1=160 + \dfrac{9-5}{17} \cdot 7 \approx 161{,}6.$$
		\item $\dfrac{3n}{4}=27$, $p=3$, $a_3=167$, $m_1+m_2=22$, $m_3=8$, $a_4-a_3=3$. $$Q_3=167 + \dfrac{27-22}{8} \cdot 3 \approx 168{,}9$$
	\end{itemize}
	}
\end{ex}
\begin{ex}%[1D3Y2-2]% ::Cau 20::
	Cho dãy số $(u_n)$ với $u_n=\dfrac{n+1}{n-2}$. Tính $u_{20}$.
	\choice
	{\True $\dfrac{21}{18}$}
	{$\dfrac{18}{21}$}
	{$\dfrac{11}{8}$}
	{$\dfrac{8}{11}$}
	\loigiai{
		Ta có $u_{20}=\dfrac{20+1}{20-2}=\dfrac{21}{18}$.}
\end{ex}
\begin{ex}%[1D3B4-2]% ::Cau 21::
	Cho cấp số nhân có $u_1=2$ và $u_6=486$. Tìm công bội của cấp số nhân.
	\choice
	{$1$}
	{\True $3$}
	{$2$}
	{$4$}
	\loigiai{
		Ta có $u_6=u_1\cdot q^5\Rightarrow {q^5}=\dfrac{u_6}{u_1}=\dfrac{486}{2}=243=3^5\Rightarrow q=3$.}
\end{ex}
\begin{ex}%[0D5BC-1]% ::Cau 22::
	Cho mẫu số liệu ghép nhóm về thống kê chiều cao (mét) của $35$ cây bạch đàn trong rừng, ta có bảng số liệu sau:
	\begin{center}
		\begin{tabular}{|c|c|c|c|c|}
			\hline
		Khoảng chiều cao (m)	& $[6{,5};7)$ & $[7;7{,5})$ & $[7{,}5;8)$ & $[8;8{,}5)$ \\
			\hline
		Số cây	& $6$ & $15$ & $11$ & $3$ \\
			\hline
		\end{tabular}
	\end{center}
	Tính chiều cao trung bình của $35$ cây bạch đàn trên. (Kết quả làm tròn đến hàng phần nghìn).
	\choice
	{\True $7{,}407$ m}
	{$4{,}707$ m}
	{$7{,}704$ m}
	{$7{,}5$ m}
	\loigiai{
		Ta có giá trị đại diện các nhóm được cho dưới bảng sau:
		\begin{center}
			\begin{tabular}{|c|c|c|c|c|}
				\hline
				Khoảng chiều cao (m)	& $[6{,5};7)$ & $[7;7{,5})$ & $[7{,}5;8)$ & $[8;8{,}5)$ \\
				\hline
				Giá trị đại diện	& $6{,}75$	&$7{,}25$&$7{,}75$	&$8{,}25$	\\
				\hline
				Tần số (Số cây)	& $6$ & $15$ & $11$ & $3$ \\
				\hline
			\end{tabular}
		\end{center}
		Từ đó suy ra chiều cao trung bình của 35 cây bạch đàn là
		$$\overline{x}=\dfrac{6{,}75\cdot 6+7{,}25\cdot 15+7{,}75\cdot 11+8{,}25\cdot 3}{35}=7{,}047 \,\text{m}.$$}
\end{ex}
\begin{ex}%[0D5BC-3]% ::Cau 23::
	Tìm hiểu thời gian hoàn thành một bài tập (đơn vị: phút) của một số học sinh thu được kết quả sau:
	\begin{center}
		\begin{tabular}{|c|c|c|c|c|c|}
			\hline
		Thời gian (phút)	& $[0;4)$ & $[4;8)$ & $[8;12)$ & $[12;16)$ & $[16;20)$ \\
			\hline
		Số học sinh	& $2$ & $4$ & $7$ & $4$ & $3$ \\
			\hline
		\end{tabular}
	\end{center}
	Tứ phân vị thứ ba của mẫu số liệu ghép nhóm này là
	\choice
	{$Q_3=13$}
	{\True $Q_3=14$}
	{$Q_3=15$}
	{$Q_3=12$}
	\loigiai{
		Cỡ mẫu: $n=2+4+7+4+3=20$.\\
		Tứ phân vị thứ ba $Q_3$ là $\dfrac{x_{15}+x_{16}}{2}$. \\
		Do ${x_{15}},{x_{16}}$ đều thuộc nhóm $\left[ 12;16 \right)$ nên nhóm này chứa $Q_3$.\\
		Do đó $p=4$, $a_4=12$, $m_4=4$, $m_1+m_2+m_3=2+4+7=13$, $a_5-a_4=4$. \\
		Ta có $Q_3=12+\dfrac{\dfrac{3\cdot 20}{4}-13}{4}\cdot 4=14$.}
\end{ex}
\begin{ex}%[0D5YC-1]% ::Cau 24::
	Một nhóm 10 học sinh có điểm thi môn toán là: $5$; $6$; $7$; $5$; $8$; $8$; $10$; $9$; $7$; $8$. Tính điểm trung bình của nhóm học sinh trên.
	\choice
	{$8$}
	{\True $7{,}3$}
	{$8{,}3$}
	{$7{,}7$}
	\loigiai{
		Điểm trung bình là $\overline{x}=\dfrac{5\cdot 2+6+7\cdot 2+8\cdot 3+9+10}{10}=7{,}3$.}
\end{ex}
\begin{ex}%[0D5YC-2]% ::Cau 25::
	Thời gian xem ti vi trong tuần (đơn vị: giờ) của một số học sinh thu được kết quả như sau:
	\begin{center}
	\begin{tabular}{|c|c|c|c|c|c|}
		\hline
		Thời gian (giờ)	& $[0;4)$ & $[4;8)$ & $[8;12)$ & $[12;16)$ & $[16;20)$ \\
		\hline
		Số học sinh	& $6$ & $12$ & $4$ & $4$ & $2$ \\
		\hline
	\end{tabular}
	\end{center}
	Giá trị đại diện của nhóm $\left[ 12;16 \right)$ là
	\choice
	{$12$}
	{\True $14$}
	{$10$}
	{$16$}
	\loigiai{
	Giá trị đại diện của nhóm $\left[ 12;16 \right)$ là $\dfrac{12+16}{2}=14$.
	}
\end{ex}
\begin{ex}%[1D3Y2-2]% ::Cau 26::
	Cho dãy số $u_n$ biết với $u_n=\dfrac{1}{n+1}$, ba số hạng đầu tiên của dãy đó là
	\choice
	{\True $\dfrac{1}{2};\dfrac{1}{3};\dfrac{1}{4}$}
	{$1;\dfrac{1}{2};\dfrac{1}{3}$}
	{$\dfrac{1}{2};\dfrac{1}{4};\dfrac{1}{6}$}
	{$1;\dfrac{1}{3};\dfrac{1}{5}$}
	\loigiai{
	Ta có $u_n=\dfrac{1}{n+1}$, khi đó $u_1=\dfrac{1}{1+1}=\dfrac{1}{2}$, $u_2=\dfrac{1}{2+1}=\dfrac{1}{3}$, $u_3=\dfrac{1}{3+1}=\dfrac{1}{4}$.\\
	Ba số hạng đầu tiên của dãy đó là $\dfrac{1}{2};\dfrac{1}{3};\dfrac{1}{4}$.
	}
\end{ex}
\begin{ex}%[1D3B3-6]% ::Cau 27::
	Người ta trồng $465$ cây trong một khu vườn hình tam giác như sau: Hàng thứ nhất có $1$ cây, hàng thứ hai có $2$ cây, hàng thứ ba có $3$ cây. Số hàng cây trong khu vườn là:
	\choice
	{$31$}
	{\True $30$}
	{$29$}
	{$28$}
	\loigiai{
		Cách trồng $465$ cây trong một khu vườn hình tam giác như trên lập thành một cấp số cộng $\left( {u_n} \right)$ với số $u_n$ là số cây ở hàng thứ $n$ và $u_1=1$ và công sai $d=1$.\\
		Tổng số cây trồng được là
		$$S_n=465\Leftrightarrow \dfrac{n\left( n+1 \right)}{2}=465\Leftrightarrow {n^2}+n-930=0\Leftrightarrow \hoac{
			& n=30 \\
			& n=-31\,\left( \text{loại} \right).}$$
		Như vậy số hàng cây trong khu vườn là $30$.}
\end{ex}
\begin{ex}%[1D3B3-5]% ::Cau 28::
	Cho cấp số cộng $\left( u_n \right)$ có $u_{27}+u_2=83$. Khi đó tổng $28$ số hạng đầu tiên của cấp số cộng $\left( u_n \right)$ là
	\choice
	{\True $S_{28}=1162$}
	{$S_{28}=1612$}
	{$S_{28}=2611$}
	{$S_{28}=1261$}
	\loigiai{
		Gọi $d$ và $u_1$ lần lượt là công sai và số hạng đầu của cấp số cộng $\left( {u_n} \right)$\\
		Ta có $S_{28}=\dfrac{28\left( u_1+u_{28} \right)}{2}=\dfrac{28\left( u_2-d+u_{27}+d \right)}{2}=\dfrac{28\left( u_2+u_{27} \right)}{2}=\dfrac{28\cdot 83}{2}=1162$.}
\end{ex}
\begin{ex}%[1D3B3-2]% ::Cau 29::
	Cho cấp số cộng $\left( u_n \right)$ biết $u_{27}=-76$ và $u_{83}=-244$. Khi đó số hạng đầu $u_1$ của cấp số cộng đã cho bằng
	\choice
	{$-3$}
	{$5$}
	{$4$}
	{\True $2$}
	\loigiai{
		Gọi $d$ là công sai của cấp số cộng đã cho.\\
		Áp dụng công thức $u_n=u_1+\left( n-1 \right)d$, ta có 
		$$\heva{
			& u_{27}=-76 \\
			& u_{83}=-244}
		\Leftrightarrow \heva{
			& u_1+26d=-76 \\
			& u_1+82d=-244}
		\Leftrightarrow \heva{
			& u_1=2 \\
			& d=-3.}$$}
\end{ex}
\begin{ex}%[1D3K3-4]% ::Cau 30::
	Cho $a<b<c$ là ba số nguyên. Biết $a$, $b$, $c$ theo thứ tự tạo thành một cấp số cộng và $a$, $c$, $b$ theo thứ tự tạo thành một cấp số nhân. Tìm giá trị nhỏ nhất của $c$.
	\choice
	{$-2$}
	{\True $2$}
	{$-1$}
	{$4$}
	\loigiai{
		Ta có $\heva{& 2b=a+c \\& c^2=ab>0}$. Suy ra $$2c^2=a\left( a+c \right)\Rightarrow 2c^2-ac-a^2=0\Rightarrow \hoac{
			& c=a\,\left( \text{loại} \right) \\
			& c=-\dfrac{a}{2}\Rightarrow b=\dfrac{a}{4}=-\dfrac{c}{2}.}$$
		Suy ra $a$, $b$ trái dấu với $c$ $\Rightarrow \heva{
			& a<0 \\
			& c>0.}$\\
		Do $a$, $b$, $c$ nguyên nên $c$ chia hết cho $2$.\\
		Do đó $c$ nhỏ nhất bằng $2$ khi đó $a=-4$, $b=-1$.}
\end{ex}
\begin{ex}%[Dự án 1 - TLDH - TeamTeXHoa - Lê Quân]%[1K2B5-1] 
	Cho dãy số $(u_n)$ có $u_n=2\cdot 3^n$. Công thức truy hồi của dãy số $(u_n)$ là 
	\choice
	{$\heva{&u_1=6	\\&u_n=6u_{n-1},\, \forall n>1}$}
	{\True $\heva{&u_1=6	\\&u_n=3u_{n-1},\, \forall n>1}$}
	{$\heva{&u_1=3	\\&u_n=3u_{n-1},\, \forall n>1}$}
	{$\heva{&u_1=3	\\&u_n=6u_{n-1},\, \forall n>1}$}
	\loigiai{
		Ta có $u_n=2\cdot 3^n\Rightarrow \heva{&u_1=2\cdot 3^1=6	\\&u_{n+1}=2\cdot 3^{n+1}.}$\\
		$\Rightarrow u_{n+1}=2\cdot3\cdot 3^n =3u_n\Rightarrow u_n=3\cdot u_{n-1}$.\\
		Vậy $\heva{&u_1=6	\\&u_n=3u_{n-1},\, \forall n>1.}$
	}
\end{ex}
\begin{ex}%[Dự án 1 - TLDH - TeamTeXHoa - Sauluoi3105]%[1K2B5-4]
	Cho dãy số $\left(u_n\right)$, với $u_n=\dfrac{1}{1\cdot 4}+\dfrac{1}{2\cdot 5}+\ldots+\dfrac{1}{n(n+3)}, \forall n=1 ; 2 ; 3 \cdots$. Mệnh đề nào sau đây đúng?
	\choice
	{Dãy số $\left(u_n\right)$ bị chặn trên và không bị chặn dưới}
	{Dãy số $\left(u_n\right)$ bị chặn dưới và không bị chặn trên}
	{\True Dãy số $\left(u_n\right)$ bị chặn}
	{Dãy số $\left(u_n\right)$ không bị chặn}
	\loigiai{
		Ta có $u_n>0$ suy ra $\left(u_n\right)$ bị chặn dưới bởi $0$.\\ 
		Mặt khác $\dfrac{1}{k(k+3)}<\dfrac{1}{k(k+1)}=\dfrac{1}{k}-\dfrac{1}{k+1}\left(k \in \mathbb{N}^*\right)$ nên 
		$$u_n<\dfrac{1}{1\cdot 2}+\dfrac{1}{2\cdot 3}+\dfrac{1}{3\cdot 4}+\cdots+\dfrac{1}{n(n+1)}=1-\dfrac{1}{2}+\dfrac{1}{2}-\dfrac{1}{3}+\dfrac{1}{2}-\dfrac{1}{4}+\cdots+\dfrac{1}{n}-\dfrac{1}{n+1}=1-\dfrac{1}{n+1}<1.$$
Suy ra dãy $\left(u_n\right)$ bị chặn trên.\\
Vậy dãy $\left(u_n\right)$ bị chặn.
	}
\end{ex}
\Closesolutionfile{ans}
% \inputans{10}{ans/ans-1-GK1-KNTT-14-NH23-24}
\noindent{\bf\fontfamily{qag}\selectfont\color{violet}B. PHẦN TỰ LUẬN}
\setcounter{bt}{0}
\begin{bt}%[1D1B2-1]% ::Cau 31::
	% \begin{enumEX}{1}
		% \item 
		Cho $\alpha \in (-\dfrac{\pi}{2};0)$ và $\sin \alpha = -\dfrac13$. Tìm $\cos \alpha$, $\tan \alpha$, $\cot \alpha$. 
		% \item Giải phương trình $\cot \left( 2x-40^{\circ} \right)=-\sqrt{3}$.
	% \end{enumEX}
	\loigiai{
		% \begin{enumEX}{1}
		% \item 
		Ta có $\alpha \in (-\dfrac{\pi}{2};0)$ nên $\cos \alpha >0$. Suy ra $$\cos^2 \alpha = 1-\sin^2 \alpha = 1-\left(-\dfrac13\right)=\dfrac89 \Rightarrow \cos \alpha = \dfrac{2\sqrt{2}}{3}.$$ 
		$\tan \alpha = \dfrac{\sin \alpha}{\cos \alpha} = \dfrac{-\dfrac13}{\dfrac{2\sqrt{2}}{3}}=-\dfrac{\sqrt{2}}{4}$.\\
		$\cot \alpha = \dfrac{1}{\tan \alpha} = -2\sqrt{2}$.
		% \item Ta có $\cot \left( 2x-40^{\circ} \right)=-\sqrt{3}\Leftrightarrow 2x-40^{\circ} =-30^{\circ} +k180^{\circ} \Leftrightarrow x=5^{\circ} +k90^{\circ} ,\,\left( k\in \mathbb{Z} \right)$.
	% \end{enumEX}
	}
\end{bt}

\begin{bt}%[1D3B3-3]% ::Cau 34::
	Tìm tổng $15$ số hạng đầu tiên của cấp số cộng $\left( {u_n} \right)$, biết $\heva{& u_1+u_5-u_3=10 \\& u_1+u_6=17.}$
	\loigiai{
	Ta có
		$$\heva{
			& {u_1}+u_5-u_3=10 \\
			& {u_1}+u_6=17}
		\Leftrightarrow \heva{
			& {u_1}+u_1+4d-\left( {u_1}+2d \right)=10 \\
			& {u_1}+u_1+5d=17}
		\Leftrightarrow \heva{
			& {u_1}+2d=10 \\
			& 2u_1+5d=17}
		\Leftrightarrow \heva{
			& u_1=16 \\
			& d=-3.}$$
		Suy ra $S_{15}=\dfrac{15}{2}\left(2u_1+14d\right)=\dfrac{15}{2}[2 \cdot 16 + 14\cdot (-3)]=-150$.}
\end{bt}

\begin{bt}%[1D1K3-8]% ::Cau 36::
	Hàng ngày mực nước của một con kênh lên xuống theo thủy triều. Độ sâu $h$ (mét) của mực nước trong kênh tính theo thời gian $t$ (giờ) $\left( 0\le t\le 24 \right)$ được mô tả bởi công thức $h=A\cos \left( \dfrac{\pi t}{6}+1 \right)+B$, với $A, B$ là các số thực dương cho trước. Biết độ sâu của mực nước lớn nhất là $15$ mét khi thủy triều lên cao và khi thủy triều xuống thấp thì độ sâu của mực nước thấp nhất là $9$ mét. Tính thời điểm độ sâu của mực nước là $13{,}5$ mét (tính chính xác đến $\dfrac{1}{100}$ giờ).
	\loigiai{
		Với mọi $0\le t\le 24$, ta có
		\allowdisplaybreaks
		\begin{eqnarray*}
		&& -1\le \cos \left( \dfrac{\pi t}{6}+1 \right)\le 1 \\
		&\Leftrightarrow& -A+B\le A\cos \left( \dfrac{\pi t}{6}+1 \right)+B\le A+B.
		\end{eqnarray*}
		Độ sâu của mực nước lớn nhất bằng $A+B$ khi $\cos \left( \dfrac{\pi t}{6}+1 \right)=1$ và thấp nhất bằng $-A+B$ khi $\cos \left( \dfrac{\pi t}{6}+1 \right)=-1$.\\
		Ta có hệ $\heva{
			& A+B=15 \\
			& -A+B=9}
		\Leftrightarrow \heva{
			& B=12 \\
			& A=3.}$\\
		Ta được $h=3\cos \left( \dfrac{\pi t}{6}+1 \right)+12$.\\
		Theo đề, ta tìm thời điểm mà độ sâu 
		\allowdisplaybreaks
		\begin{eqnarray*}
		&& h=13{,}5\Leftrightarrow 3\cos \left( \dfrac{\pi t}{6}+1 \right)+12=13{,}5\Leftrightarrow \cos \left( \dfrac{\pi t}{6}+1 \right)=\dfrac{1}{2}\\
		&\Leftrightarrow& \hoac{
			& \dfrac{\pi t}{6}+1=\dfrac{\pi}{3}+k2\pi\\
			& \dfrac{\pi t}{6}+1=-\dfrac{\pi}{3}+k2\pi},\left( k\in \mathbb{Z} \right)
		\Leftrightarrow \heva{
			& t=\left( -1+\dfrac{\pi}{3} \right)\cdot \dfrac{6}{\pi}+12k \\
			& t=\left( -1-\dfrac{\pi}{3} \right)\cdot \dfrac{6}{\pi}+12k},\left( k\in \mathbb{Z} \right).
		\end{eqnarray*}
		Do $0\le t\le 24; k\in \mathbb{Z}$ nên $t=0{,}09$ (giờ); $t=12{,}09$ (giờ); $t=8{,}09$ (giờ); $t=20{,}09$ (giờ).}
\end{bt}

	\begin{bt}%[Trần Ngọc Thành, CTST-BG11]%[1T3T1-6]
		Cho hình vuông $H_0$ cạnh bằng 1 đơn vị độ dài. Chia hình vuông $H_0$ thành chín hình vuông bằng nhau, bỏ đi bốn hình vuông, nhận được hình $H_1$. Tiếp theo, chia mỗi hình vuông của $H_1$ thành chín hình vuông, rồi bỏ đi bốn hình vuông, nhận được hình $H_2$. Tiếp tục quá trình này, ta nhận được một dãy hình $H_n$ $(n=1,2,3,\ldots)$.	
		\\
		\centerline{
			\begin{tikzpicture}[scale=.8]% Muốn vẽ hình Hn thì dùng \hv{n}
				\def\a{2}
				\pgfmathsetmacro\sh{2*\a *sqrt(2)/3}
				\def\hv#1{
					\ifnum#1>0
					\draw[white,fill=white] 
					(-\a/3,\a/3) rectangle (\a/3,\a)
					(-\a/3,-\a/3) rectangle (-\a,\a/3)
					(-\a/3,-\a/3) rectangle (\a/3,-\a)
					(\a/3,-\a/3) rectangle (\a,\a/3)
					;
					\pgfmathtruncatemacro{\k}{#1-1}
					\begin{scope}[scale=1/3]\hv{\k}\end{scope}
					\begin{scope}[shift={(45:\sh)},scale=1/3]\hv{\k}\end{scope}
					\begin{scope}[shift={(135:\sh)},scale=1/3]\hv{\k}\end{scope}
					\begin{scope}[shift={(225:\sh)},scale=1/3]\hv{\k}\end{scope}
					\begin{scope}[shift={(315:\sh)},scale=1/3]\hv{\k}\end{scope}
					\fi
				}
				\begin{scope}
					\fill[gray] (-\a,-\a) rectangle (\a,\a);
					\hv{0}
					\path (0,-\a)node[below]{$H_0$};
				\end{scope}
				\begin{scope}[xshift=4.5cm]
					\fill[gray] (-\a,-\a) rectangle (\a,\a);
					\hv{1}
					\path (0,-\a)node[below]{$H_1$};
				\end{scope}
				\begin{scope}[yshift=-5cm]
					\fill[gray] (-\a,-\a) rectangle (\a,\a);
					\hv{2}
					\path (0,-\a)node[below]{$H_2$};
				\end{scope}
				\begin{scope}[xshift=4.5cm,yshift=-5cm]
					\fill[gray] (-\a,-\a) rectangle (\a,\a);
					\hv{3}
					\path (0,-\a)node[below]{$H_3$};
				\end{scope}
				% \path (current bounding box.south) node[below]{Hình $6$};
			\end{tikzpicture}
		}
		Tính tổng diện tích và tổng chu vi tất cả hình vuông được tô màu trong hình $H_5$.
		\loigiai{
			\begin{enumerate}
				\item Hình vuông $H_1$ có diện tích $S_1=5\cdot \left(\dfrac{1}{3}\right)^2=\dfrac{5}{9}$.\\
				Hình vuông $H_2$ có diện tích $S_2=5^2\cdot \left(\dfrac{1}{3^2}\right)^2=\left(\dfrac{5}{9}\right)^2$.\\
				Hình vuông $H_n$ có diện tích $S_n=5^n\cdot \left(\dfrac{1}{3^n}\right)^2=\left(\dfrac{5}{9}\right)^n$.\\
				
				\item Hình vuông $H_1$ có chu vi $C_1=5\cdot 4\cdot  \dfrac{1}{3}=4\cdot \dfrac{5}{3}$.\\
				Hình vuông $H_2$ có chu vi $C_2=5^2\cdot4\cdot \dfrac{1}{3^2}=4\cdot \left(\dfrac{5}{3}\right)^2$.\\
				Hình vuông $H_n$ có diện tích $C_n=5^n\cdot4\cdot  \dfrac{1}{3^n}=4\cdot \left(\dfrac{5}{3}\right)^n$.\\
				
			\end{enumerate}
			Vậy $S_5=\left(\dfrac{5}{9}\right)^5$ và $C_5=4\cdot \left(\dfrac{5}{3}\right)^5$.
		}
	\end{bt}

% \section*{ÔN TẬP KIỂM TRA GIỮA KÌ 1 - ĐỀ 02}
\setcounter{ex}{0}\setcounter{bt}{0}
\noindent{\bf\fontfamily{qag}\selectfont\color{violet}A. PHẦN TRẮC NGHIỆM}
\Opensolutionfile{ans}[ans/ans-1-GK1-KNTT-De15-NH23-24]

%%==========Câu 1
\begin{ex}%[1K1Y1-5] 
	$\sin\alpha>0$ khi điểm cuối của cung $\alpha$ trên đường tròn lượng giác thuộc các góc phần tư thứ
	\choice
	{I và III}
	{\True I và II}
	{II và IV}
	{I và IV}
	\loigiai{
		$\sin\alpha>0$ khi điểm cuối của cung $\alpha$ trên đường tròn lượng giác các góc phần tư thứ I và II.	
	}
\end{ex}
	\begin{ex}%[1K1Y1-8]
	Trong các khẳng định sau, khẳng định nào sai?
	\choice
	{$\tan\left(\pi-\alpha\right)=-\tan\alpha$}
	{\True $\tan\left(\pi+\alpha\right)=-\tan\alpha$}
	{$\tan\left(-\alpha\right)=-\tan\alpha$}
	{$\tan\left(\dfrac{\pi}{2}-\alpha\right)=\cot\alpha$}
	\loigiai{
		Vì $\tan\left(\pi+\alpha\right)=\tan\alpha$ nên khẳng định sai là $\tan\left(\pi+\alpha\right)=-\tan\alpha$.	
	}
\end{ex}
	\begin{ex}%[1K1Y1-5]
	Khi biểu diễn cung lượng giác $\alpha$ lên đường tròn lượng giác thì điểm cuối của cung $\alpha$ thuộc góc phần tư thứ ba của đường tròn lượng giác. Khẳng định nào sau đây là \textbf{đúng}?
	\choice
	{$\sin\alpha>0$}
	{$\cos\alpha>0$}
	{\True $\tan\alpha>0$}
	{$\cot\alpha<0$}
	\loigiai{
		Vì khi biểu diễn cung lượng giác $\alpha$ lên đường tròn lượng giác thì điểm cuối của cung $\alpha$ thuộc góc phần tư thứ ba của đường tròn lượng giác nên $\tan\alpha>0$.
	}
\end{ex}
\begin{ex}%[1K1B1-6]
	Cho góc $\alpha$ thỏa mãn $\sin\alpha=\dfrac{4}{5}$ và $\dfrac{\pi}{2}<\alpha<\pi$. Tính $\cos\alpha$.
	\choice
	{$\cos\alpha=\dfrac{3}{5}$}
	{\True $\cos\alpha=-\dfrac{3}{5}$}
	{$\cos\alpha=-\dfrac{1}{5}$}
	{$\cos\alpha=\dfrac{1}{5}$}
	\loigiai{
		Ta có $\cos^2\alpha=1-\sin^2\alpha=1-\left(\dfrac{4}{5}\right)^2 \Leftrightarrow \cos^2\alpha =\dfrac{9}{25} \Leftrightarrow \cos\alpha=\pm\dfrac{3}{5}$.\\
		Mà $\dfrac{\pi}{2}<\alpha<\pi$ nên $\cos\alpha<0$. Vậy $\cos\alpha=-\dfrac{3}{5}$.	
	}
\end{ex}
\begin{ex}%[1K1Y2-1]
	Trong các công thức sau, công thức nào đúng?
	\choice
	{\True $\sin\left(a-b\right)=\sin a\cdot\cos b-\sin b\cdot\cos a$}
	{$\cos\left(a-b\right)=\cos a\cdot\cos b-\sin a\cdot\sin b$}
	{$\sin\left(a+b\right)=\sin a\cdot\cos b-\sin b\cdot\cos a$}
	{$\cos\left(a+b\right)=\cos a\cdot\cos b$ $+$ $\sin a\cdot\sin b$}
	\loigiai{
		Công thức cộng $\sin\left(a-b\right)=\sin a\cdot\cos b-\sin b\cdot\cos a$.	
	}
\end{ex}
\begin{ex}%[1K1K2-1]
	Rút gọn biểu thức $\sin\left(a-17^\circ\right)\cos\left(a+13^\circ\right)-\sin\left(a+13^\circ\right)\cos\left(a-17^\circ\right)$, ta được
\choice
{$\sin2a$}
{$\cos2a$}
{\True $-\dfrac{1}{2}$}
{$\dfrac{1}{2}$}
\loigiai{ 
	Ta có $\sin\left(a-17^\circ\right)\cdot\cos\left(a+13^\circ\right)-\sin\left(a+13^\circ\right)\cdot\cos\left(a-17^\circ\right)=\sin\left[\left(a-17^\circ\right)-\left(a+13^\circ\right)\right]$\\
	$=\sin\left(-30^\circ\right)=-\dfrac{1}{2}$.	
	}
\end{ex}
\begin{ex}%[1K1K2-3]
	Với $\alpha$ là số thực bất kỳ, mệnh đề nào sau đây là mệnh đề đúng? 
	\choice
	{$\cos2\alpha+\cos4\alpha=2\cos2\alpha\cdot\cos6\alpha$}
	{$\sin2\alpha+\sin4\alpha=2\sin\alpha\cdot\cos3\alpha$}
	{$\cos2\alpha-\cos4\alpha=-2\sin3\alpha\cdot\sin\alpha$}
	{\True $\sin2\alpha-\sin4\alpha=-2\cos3\alpha\cdot\sin\alpha$}
	\loigiai{
		Ta có\\
		$\cos2\alpha+\cos4\alpha=2\cos\dfrac{2\alpha+4\alpha}{2}\cdot\cos\dfrac{2\alpha-4\alpha}{2}=2\cos3\alpha\cdot\cos\alpha$. Do đó $\cos2\alpha+\cos4\alpha=2\cos2\alpha\cdot\cos6\alpha$ sai.\\
		$\sin2\alpha+\sin4\alpha=2\sin\dfrac{2\alpha+4\alpha}{2}\cdot\cos\dfrac{2\alpha-4\alpha}{2}=2\sin\alpha.\cos\alpha$. Do đó $\sin2\alpha+\sin4\alpha=2\sin\alpha\cdot\cos3\alpha$ sai.\\
		$\cos2\alpha-\cos4\alpha=-2\sin\dfrac{2\alpha+4\alpha}{2}\cdot\sin\dfrac{2\alpha-4\alpha}{2}=2\sin3\alpha\cdot\sin\alpha$. Do đó $\cos2\alpha-\cos4\alpha=-2\sin3\alpha\cdot\sin\alpha$ sai.\\	
		$\sin2\alpha-\sin4\alpha=2\cos\dfrac{2\alpha+4\alpha}{2}\cdot\sin\dfrac{2\alpha-4\alpha}{2}=-2\cos3\alpha\cdot\cos\alpha$ là đáp án đúng.
	}
\end{ex}
\begin{ex}%[1K1B3-1]
	Tập xác định của hàm số $y=\tan x$ là 
	\choice
	{$\mathscr{D}=[-1;1]$}
	{$\mathscr{D}=\mathbb{R}\setminus\left\{k\pi,k\in\mathbb{Z}\right\}$}
	{\True $\mathscr{D}=\mathbb{R}\setminus\left\{\dfrac{\pi}{2}+k\pi,k\in\mathbb{Z}\right\}$}
	{$\mathscr{D}=\mathbb{R}$}
	\loigiai{
		Điều kiện xác định $\cos x\neq 0 \Leftrightarrow x\neq \dfrac{\pi}{2}+k\pi,k\in\mathbb{Z}$.\\
		Vậy tập xác định của hàm số là $\mathscr{D}=\mathbb{R}\setminus\left\{\dfrac{\pi}{2}+k\pi,k\in\mathbb{Z}\right\}$.
	}
\end{ex}
\begin{ex}%[1K1Y3-4]
	Hàm số $y=\sin x$ tuần hoàn với chu kỳ là 
	\choice
	{$\dfrac{\pi}{2}$}
	{$\dfrac{\pi}{3}$}
	{\True $2\pi$}
	{$\pi$}
	\loigiai{
		Hàm số $y=\sin x$ tuần hoàn với chu kỳ là $2\pi$.
	}
\end{ex}
\begin{ex}%[1K1Y3-3]
	Trong các hàm số sau $y=\sin x$, $y=\cos x$, $y=\tan x$, $y=\cot x$ hàm số nào là hàm số chẵn? 
	\choice
	{\True $y=\cos x$}
	{$y=\tan x$}
	{$y=\cot x$}
	{$y=\sin x$}
	\loigiai{
		Hàm số $y=\cos x$ là hàm số chẵn.
	}
\end{ex}
\begin{ex}%[1K1B3-1]
	Tìm tập xác định $\mathscr{D}$ của hàm số $y=\dfrac{3\tan x-5}{1-\sin^2 x}$.
	\choice
	{$\mathscr{D}=\mathbb{R}\setminus\left\{\dfrac{\pi}{2}+k2\pi,k\in\mathbb{Z}\right\}$}
	{\True $\mathscr{D}=\mathbb{R}\setminus\left\{\dfrac{\pi}{2}+k\pi,k\in\mathbb{Z}\right\}$}
	{$\mathscr{D}=\mathbb{R}\setminus\left\{\pi+k\pi,k\in\mathbb{Z}\right\}$}
	{$\mathscr{D}=\mathbb{R}$}
	\loigiai{
	Hàm số xác định khi và chỉ khi $\heva{&1-\sin^2 x\neq 0\\&\cos x\neq 0}$\\
	$\Leftrightarrow \heva{&\sin^2 x\neq 1\\&\cos x\neq 0} \Leftrightarrow \cos x\neq 0 \Leftrightarrow x\neq\dfrac{\pi}{2}+k\pi,k\in\mathbb{Z}$.\\
	Vậy tập xác định của hàm số là $\mathscr{D}=\mathbb{R}\setminus\left\{\dfrac{\pi}{2}+k\pi,k\in\mathbb{Z}\right\}$. 
	}
\end{ex}
\begin{ex}%[1K1K3-5]
	Gọi $M, m$, lần lượt là giá trị lớn nhất, giá trị nhỏ nhất của hàm số $y=\sqrt{3}\sin 2x - \cos 2x -1$. Giá trị của $M+m$ bằng
	\choice
	{$M+m=0$}
	{\True $M+m=-2$}
	{$M+m=1$}
	{$M+m=-1$}
	\loigiai{
		Ta có $y=\sqrt{3}\sin 2x - \cos 2x -1=2\left(\dfrac{\sqrt{3}}{2}\sin 2x -\dfrac{1}{2}\cos 2x\right)-1= 2\sin\left(2x-\dfrac{\pi}{6}\right)-1$\\
		Vì $\forall x\in\mathbb{R}, -1\leq \sin\left(2x-\dfrac{\pi}{6}\right)\leq1$ nên suy ra $-2 -1\leq 2\sin\left(2x-\dfrac{\pi}{6}\right)-1\leq2-1$.\\
		Do đó, $-3\leq y\leq1,\forall x\in\mathbb{R}$.\\
		Do đó giá trị lớn nhất và giá trị nhỏ nhất của hàm số $y=\sqrt{3}\sin 2x - \cos 2x -1$ lần lượt là $M=1,m=-3$.\\
		Khi $\sin\left(2x-\dfrac{\pi}{6}\right)=1 \Leftrightarrow 2x-\dfrac{\pi}{6}=\dfrac{\pi}{2}+k2\pi \Leftrightarrow x=\dfrac{\pi}{3}+k2\pi,k\in\mathbb{Z}$.\\
		$\sin\left(2x-\dfrac{\pi}{6}\right)=-1 \Leftrightarrow 2x-\dfrac{\pi}{6}=-\dfrac{\pi}{2}+k2\pi \Leftrightarrow x=-\dfrac{\pi}{6}+k\pi,k\in\mathbb{Z}$.\\
		Vậy $M+m=1-3=-2$.
	}
\end{ex}
\begin{ex}%[1K1Y4-2]
	Phương trình nào sau đây có nghiệm?
	\choice
	{$\sin2x=2$}
	{$\cos2x=-2$}
	{\True $\sin3x=\dfrac{2}{3}$}
	{$\cos x =\pi$}
	\loigiai{
		Do $\dfrac{2}{3}\in[-1;1]$ nên phương trình $\sin3x=\dfrac{2}{3}$ có nghiệm.
	}
\end{ex}
\begin{ex}%[1K1B4-5]
	Nghiệm của phương trình $\sin x=\dfrac{1}{2}$ là
	\choice
	{$x=\dfrac{\pi}{6}+k\pi$; $x=\dfrac{5\pi}{6}+k\pi,k\in\mathbb{Z}$}
	{$x=\dfrac{\pi}{6}+k\pi$; $x=\dfrac{-\pi}{6}+k\pi,k\in\mathbb{Z}$}
	{\True $x=\dfrac{\pi}{6}+k2\pi$; $x=\dfrac{5\pi}{6}+k2\pi,k\in\mathbb{Z}$}
	{$x=\dfrac{\pi}{6}+k2\pi$; $x=\dfrac{-\pi}{6}+k2\pi,k\in\mathbb{Z}$}
	\loigiai{
		Ta có $\sin x=\dfrac{1}{2} \Leftrightarrow \sin x=\sin\dfrac{\pi}{6} \Leftrightarrow \hoac{&x=\dfrac{\pi}{6}+k2\pi\\&x=\dfrac{5\pi}{6}+k2\pi}, k\in\mathbb{Z}$. 
	}
\end{ex}
\begin{ex}%[1K1K4-5]
	Tổng tất cả các nghiệm của phương trình $\sin\left(x+\dfrac{\pi}{4}\right)+\cos\left(x-\dfrac{3\pi}{4}\right)=0$ thuộc $\left(0;5\pi\right)$ bằng
	\choice
	{\True $10\pi$}
	{$7\pi$}
	{$6\pi$}
	{$9\pi$}
	\loigiai{
		Ta có\\
		$\sin\left(x+\dfrac{\pi}{4}\right)+\cos\left(x-\dfrac{3\pi}{4}\right)=0 \Leftrightarrow \sin x\cdot\cos\dfrac{\pi}{4}+\cos x\cdot\sin\dfrac{\pi}{4}+\cos x\cdot\cos\dfrac{3\pi}{4}+\sin x\cdot\sin\dfrac{3\pi}{4}=0$\\
		$\Leftrightarrow \sqrt{2}\sin x=0 \Leftrightarrow \sin x=0 \Leftrightarrow x=k\pi,k\in\mathbb{Z}$.\\
		Vì $x\in\left(0;5\pi\right) \Rightarrow 0<k\pi<5\pi \Leftrightarrow 0<k<5$.\\
		Vì $k\in\mathbb{Z} \Rightarrow k\in\left\{1;2;3;4\right\} \Rightarrow x\in\left\{\pi;2\pi;3\pi;4\pi\right\}$.\\
		Khi đó, tổng các nghiệm của phương trình là $S=\pi+2\pi+3\pi+4\pi=10\pi$.	
	}
\end{ex}
\begin{ex}%[1K2B5-2]
	Cho dãy số $\left(u_{n}\right)$ có số hạng tổng quát $u_{n}=n^2-3$. Số hạng thứ $10$ của dãy số là
	\choice
	{$7$}
	{\True $97$}
	{$100$}
	{$103$}
	\loigiai{
		Số hạng thứ $10$ của dãy số là $u_{10}=10^2-3=97$.	
	}
\end{ex}
\begin{ex}%[1K2B5-2]
	Cho dãy số $0;2;4;6;\ldots;304$. Hỏi dãy số trên có bao nhiêu số hạng?
	\choice
	{$304$}
	{$152$}
	{\True $153$}
	{$305$}
	\loigiai{
		Số các số hạng của dãy số là $\dfrac{304-0}{2}+1=153$.	
	}
\end{ex}
\begin{ex}%[1K2B6-1]
	Cho cấp số cộng $1;1;1;\ldots$. Công sai của cấp số cộng trên là
	\choice
	{\True $0$}
	{$1$}
	{$-1$}
	{$\varnothing$}
	\loigiai{
		Cấp số cộng có công sai $d=1-1=0$. Đây là dãy số không đổi.	
	}
\end{ex}
\begin{ex}%[1K2B6-3]
	Cho cấp số cộng $\left(u_{n}\right)$ với $u_{1}=-2$ và công sai $d=3$ thì số hạng $u_{5}$ bằng
	\choice
	{$7$}
	{\True $10$}
	{$5$}
	{$6$}
	\loigiai{
		Áp dụng công thức số hạng thứ $n$ của cấp số cộng $\left(u_{n}\right)$ là $u_{n}=u_{1}+\left(n-1\right)\cdot d$.\\
		Khi đó số hạng $u_{5}=u_{1}+\left(5-1\right)\cdot d=-2+4\cdot3=10$. Vậy $u_{5}=10$.	
	}
\end{ex}
\begin{ex}%[1K2G6-6]
	Vào năm $2023$, nhiệt độ trung bình của thành phố $A$ là khoảng $29{,}5^\circ C$. Giả sử do biến đổi khí hậu nên mỗi năm nhiệt độ trung bình của thành phố $A$ đều tăng thêm khoảng $0{,}1^\circ C$. Hãy ước tính kể từ năm nào thì nhiệt độ trung bình của thành phố $A$ đạt từ $35^\circ C$ trở lên.
	\choice
	{$2076$}
	{$2077$}
	{\True $2078$}
	{$2079$}
	\loigiai{
		Theo bài toán, nhiệt độ trung bình ở mỗi năm của thành phố $A$ lập thành cấp số cộng với công sai là $d=0{,}1\left(^\circ C\right)$ và $u_{1}=29{,}5^\circ C$ là nhiệt độ trung bình của thành phố $A$ vào năm $2023$.\\
		Giả sử số hạng thứ $n$ của cấp số cộng có giá trị lớn hơn hoặc bằng $35$.\\
		Tức là, $u_{n}\geq35^\circ C$ hay $u_{1}+\left(n-1\right)\cdot d\geq35 \Leftrightarrow 29{,}5+\left(n-1\right)\cdot0{,}1\geq35 \Leftrightarrow n\geq56$.\\
		Do đó, kể từ số hạng thứ $56$ trở đi thì chúng đều có giá trị lớn hơn hoặc bằng $35$.\\
		Ta có $u_{1}$ là nhiệt độ trung bình của thành phố $A$ vào năm $2023$.\\
		Nên $u_{55}$ là nhiệt độ trung bình của thành phố $A$ vào năm $\left(2023+56-1\right)=2078$.\\
		Vậy kể từ năm $2078$ thì nhiệt độ trung bình của thành phố $A$ đạt từ $35^\circ C$ trở lên. 	
	}
\end{ex}
\begin{ex}%[1K2K7-1]
	Cho cấp số nhân $\left(u_{n}\right)$ có công bội dương và $u_{2}=\dfrac{1}{5}$, $u_{4}=5$. Tính công bội $q$.
	\choice
	{\True $5$}
	{$25$}
	{$\dfrac{1}{5}$}
	{$125$}
	\loigiai{
		Ta có $\heva{&u_{2}=u_{1}\cdot q=\dfrac{1}{5}\\&u_{4}=u_{1}\cdot q^3=5} \Rightarrow \dfrac{u_{4}}{u_{2}}=q^2=25 \Leftrightarrow q=\pm5$.\\
		Mà cấp số nhân $\left(u_{n}\right)$ có công bội dương nên $q=5$.
	}
\end{ex}
\begin{ex}%[1K2B7-4]
	Tìm $x$ để các số $2;8;x;128$ theo thứ tự đó lập thành một cấp số nhân.
	\choice
	{$16$}
	{$64$}
	{$34$}
	{\True $32$}
	\loigiai{
	Các số $2;8;x;128$ theo thứ tự đó lập thành một cấp số nhân khi $x=\sqrt{8\cdot 128}=32$.
	}
\end{ex}
\begin{ex}%[1K2B7-1]
	Cho cấp số nhân $\left(u_{n}\right)$, biết $u_{1}=1$, $u_{4}=64$. Tính công bội $q$ của cấp số nhân.
	\choice
	{$21$}
	{$\pm4$}
	{\True $4$}
	{$2\sqrt{2}$}
	\loigiai{
		Theo công thức tổng quát của cấp số nhân $u_{4}=u_{1}\cdot q^3 \Leftrightarrow 64=1\cdot q^3 \Leftrightarrow q=4$.
	}
\end{ex}
\begin{ex}%[1K2B7-3]
	Cho cấp số nhân $\left(u_{n}\right)$ với $u_{1}=-1$, $q=\dfrac{-1}{10}$. Số $\dfrac{1}{10^{103}}$ là số hạng thứ mấy của $\left(u_{n}\right)$?
	\choice
	{số hạng thứ $103$}
	{\True số hạng thứ $104$}
	{số hạng thứ $105$}
	{Không là số hạng của cấp số đã cho}
	\loigiai{
		Ta có $u_{n}=u_{1}\cdot q^{n-1} \Rightarrow \dfrac{1}{10^{103}}=-1\cdot \left(\dfrac{-1}{10}\right)^{n-1} \Rightarrow n-1=103 \Rightarrow n=104$.
	}
\end{ex}
\begin{ex}%[1K3Y9-5]
	Mỗi nhóm số liệu ghép nhóm là tập hợp gồm
	\choice
	{Các giá trị của số liệu được ghép nhóm theo nhiều tiêu chí xác định}
	{Các giá trị của số liệu được ghép nhóm theo hai tiêu chí xác định}
	{\True Các giá trị của số liệu được ghép nhóm theo một tiêu chí xác định}
	{Các giá trị của số liệu được ghép nhóm theo ba tiêu chí xác định}
	\loigiai{
		Theo định nghĩa số liệu ghép nhóm: Các giá trị của số liệu được ghép nhóm theo một tiêu chí xác định.
	}
\end{ex}
\begin{ex}%[1K3Y8-1]
	Mẫu số liệu sau cho biết phân bố theo độ tuổi của dân số Việt Nam năm $2019$\\
	\begin{center}
		\begin{tabular}{|c|c|c|c|}
		\hline
		Độ tuổi& Dưới $15$ & Từ $15$ đến $65$ & Từ $65$ trở lên\\
		\hline
		Số người& $23 371 882$ & $65 420 451$ & $7 416 651$\\
		\hline
	\end{tabular}
	\end{center}
	Số dân Việt Nam năm $2019$ là
	\choice
	{$73837102$}
	{$72837102$}
	{$95208984$}
	{\True $96208984$}
	\loigiai{
		Số dân Việt Nam năm $2019$ là $23371882+65420451+7416651= 96208984$.	
	}
\end{ex}
\begin{ex}%[1K3Y9-4]
	Khảo sát thời gian tập thể dục của một số học sinh khối $11$ thu được mẫu số liệu ghép nhóm sau
	\begin{center}
		\begin{tabular}{|c|c|c|c|c|c|}
			\hline
			Thời gian $\left(\text{phút}\right)$& $\left[0;20\right)$ & $\left[20;40\right)$ & $\left[40;60\right)$& $\left[60;80\right)$ & $\left[80;100\right)$\\
			\hline
			Số học sinh & $5$ & $9$ & $12$ & $10$ & $6$\\
			\hline
		\end{tabular}
	\end{center}
	Nhóm chứa mốt của mẫu số liệu trên là
	\choice
	{$\left[20;40\right)$}
	{$\left[60;80\right)$}
	{\True $\left[40;60\right)$}
	{$\left[80;100\right)$}
	\loigiai{
		Mốt $M_0$ chứa trong nhóm $\left[40;60\right)$.		
	}
\end{ex}
\begin{ex}%[1K3K9-3]
	Khảo sát thời gian tập thể dục của một số học sinh khối $11$ thu được mẫu số liệu ghép nhóm sau
	\begin{center}
		\begin{tabular}{|c|c|c|c|c|c|}
			\hline
			Thời gian $\left(\text{phút}\right)$& $\left[0;20\right)$ & $\left[20;40\right)$ & $\left[40;60\right)$& $\left[60;80\right)$ & $\left[80;100\right)$\\
			\hline
			Số học sinh & $5$ & $9$ & $12$ & $10$ & $6$\\
			\hline
		\end{tabular}
	\end{center}
	Nhóm chứa tứ phân vị thứ ba của mẫu số liệu trên là
	\choice
	{$\left[20;40\right)$}
	{\True $\left[60;80\right)$}
	{$\left[40;60\right)$}
	{$\left[80;100\right)$}
	\loigiai{
		Ta có $n=42$ nên tứ phân vị thứ ba của mẫu số liệu trên là $Q_3=x_{33}$.\\
		Mà $x_{33}\in\left[60;80\right)$.\\
		Vậy nhóm chứa tứ phân vị thứ ba của mẫu số liệu trên là nhóm $\left[60;80\right)$.		
	}
\end{ex}
\begin{ex}%[1K3B9-1]
	Khi thống kê chiều cao của $40$ bạn lớp $11A$, ta thu được mẫu số liệu ghép nhóm được cho ở bảng sau (đơn vị: centimét).
	\begin{center}
		\begin{tabular}{|c|c|}
		\hline
		Nhóm& Tần số\\
		\hline
		$\left[155;160\right)$&$5$\\
		\hline
		$\left[160;165\right)$&$12$\\
		\hline
		$\left[165;170\right)$& $16$\\
		\hline
		$\left[170;175\right)$&$7$\\
		\hline
		& $n=40$\\
		\hline
	\end{tabular}
	\end{center}
	Số trung bình cộng bằng
	\choice
	{\True $165{,}6$}
	{$156{,}6$}
	{$155{,}6$}
	{$156{,}5$}
	\loigiai{
		Số trung bình cộng là $\bar{x}=\dfrac{5\cdot157{,}5+12\cdot162{,}5+16\cdot167{,}5+7\cdot172{,}5}{40}\approx165{,}6$.		
	}
\end{ex}
\begin{ex}%[1K3K9-3]
	Cho mẫu số liệu ghép nhóm thống kê thời gian sử dụng điện thoại trước khi ngủ (đơn vị: phút) của một người trong $120$ ngày như ở bảng sau. Xác định các số đặc trưng đo xu thế trung tâm cho mẫu số liệu đó (làm tròn các kết quả đến hàng phần mười).
	\begin{center}
		\begin{tabular}{|c|c|}
			\hline
			Nhóm & Tần số\\
			\hline
			$\left[0;4\right)$& $13$\\
			\hline
			$\left[4;8\right)$& $29$\\
			\hline
			$\left[8;12\right)$& $48$\\
			\hline
			$\left[12;16\right)$& $22$\\
			\hline
			$\left[16;20\right)$& $8$\\
			\hline
			& $n=120$\\
			\hline
		\end{tabular}
	\end{center}
	Giá trị các tứ phân vị thứ nhất, thứ hai và thứ ba lần lượt là
	\choice
	{$9{,}5;12;6{,}3$}
	{\True $6{,}3;9{,}5;12$}
	{$9{,}5;6{,}3;12$}
	{$12;6{,}3;9{,}5$}
	\loigiai{
		Bảng tần số ghép nhóm bao gồm cả tần số tích luỹ được cho như ở bảng
	\begin{center}
			\begin{tabular}{|c|c|c|}
			\hline
			Nhóm & Tần số & Tần số tích lũy\\
			\hline
			$\left[0;4\right)$& $13$ & $13$\\
			\hline
			$\left[4;8\right)$& $29$ & $42$\\
			\hline
			$\left[8;12\right)$& $48$ & $90$\\
			\hline
			$\left[12;16\right)$& $22$ & $112$\\
			\hline
			$\left[16;20\right)$& $8$ & $120$\\
			\hline
			& $n=120$ &\\
			\hline
		\end{tabular}
	\end{center}
		Ta có $\dfrac{n}{2}=60$, $\dfrac{n}{4}=30$, $\dfrac{3n}{4}=90$.\\
		Vì $42<60<90$ nên nhóm $3$ là nhóm đầu tiên có tần số tích luỹ lớn hơn hoặc bằng $60$.\\
		Suy ra trung vị là $M_e=8+\left(\dfrac{60-42}{48}\right)\cdot 4=9{,}5$.\\
		Tứ phân vị thứ hai là $Q_2=M_e=9{,}5$.\\
		Do $13<30<42$ nên nhóm $2$ là nhóm đầu tiên có tần số tích luỹ lớn hơn hoặc bằng $30$. Suy ra tứ phân vị thứ nhất là $Q_1=4+\left(\dfrac{30-13}{29}\right)\cdot 4\approx6{,}3$.\\
		Do $42<90\leq 90$ nên nhóm $3$ là nhóm đầu tiên có tần số tích luỹ lớn hơn hoặc bằng $90$. Suy ra tứ phân vị thứ ba là $Q_3=8+\left(\dfrac{90-42}{48}\right)\cdot 4=12$.		
	}
\end{ex}


\Closesolutionfile{ans}
% \inputans{10}{ans/ans-1-GK1-KNTT-De15-NH23-24}
\noindent{\bf\fontfamily{qag}\selectfont\color{violet}B. PHẦN TỰ LUẬN}
\setcounter{bt}{0}

%%==========Bài 1
\begin{bt}%[1K1K3-1]
	\begin{enumEX}{1}
		\item Tìm tập xác định của hàm số $y=\dfrac{\sqrt{1+\cos{2x}}}{1-\left(\sin x-\cos x\right)^2}$.
		\item Cho góc $\alpha \in (-\pi ; -\dfrac{\pi}{2})$ và $\tan \alpha = 3$. Tìm các GTLG của $\alpha$.
	\end{enumEX}
	\loigiai{
	Hàm số xác định khi và chỉ khi $\heva{&1+\cos{2x}\geq0\\&1-\left(\sin x-\cos x\right)^2\neq0} \Leftrightarrow \heva{&\cos{2x}\geq-1\\&\sin{2x}\neq0}$.\\
	$\cos{2x}\geq-1$ thỏa mãn $\forall x\in\mathbb{R}$.\\
	$\sin{2x}\neq0 \Leftrightarrow 2x\neq k\pi,k\in\mathbb{Z} \Leftrightarrow x\neq k\dfrac{\pi}{2},k\in\mathbb{Z}$.\\
	Vậy tập xác định của hàm số là $\mathscr{D}=\mathbb{R}\setminus\left\{k\dfrac{\pi}{2},k\in\mathbb{Z}\right\}$.
	}
\end{bt}

\begin{bt}%[1K2K6-5]
	Cho cấp số cộng $\left(u_n\right)$ có $u_5=-15$, $u_{20}=60$. Tính tổng $10$ số hạng đầu tiên của cấp số cộng đó.
	\loigiai{
		Gọi $u_1$, $d$ lần lượt là số hạng đầu và công sai của cấp số cộng.\\
		Ta có $\heva{&u_5=-15\\&u_{20}=60} \Leftrightarrow \heva{&u_1+4d=-15\\&u_1+19d=60} \Leftrightarrow \heva{&u_1=35\\&d=5}$.\\
		Vậy $S_{10}=\dfrac{10}{2}\cdot\left(2u_1+9d\right)=5\cdot\left[2\cdot\left(-35\right)+9\cdot5\right]=-125$.
	}
\end{bt}
\begin{bt}
	Số giờ có ánh sáng mặt trời của một thành phố A ở vĩ độ $40^\circ$ bắc trong ngày thứ $t$ của một năm không nhuận được cho bởi hàm số $d\left( t \right) = 3\sin \left[ {\dfrac{\pi }{{182}}\left( {t - 80} \right)} \right] + 12$  với $t \in \mathbb{Z}$ và $0<t \le 365$. Hãy cho biết ngày tháng nào có nhiều giờ có ánh sáng mặt trời nhất và ngày tháng nào có ít giờ có ánh sáng mặt trời nhất trong năm (không nhuận)?
\end{bt}
\begin{bt}%[1K2G7-6]
	Tìm $4$ số hạng đầu của một cấp số nhân biết tổng $3$ số hạng đầu bằng $\dfrac{148}{9}$, đồng thời theo thứ tự chúng là số hạng thứ $1$, thứ $4$, thứ $8$ của một cấp số cộng có công sai khác $0$.
	\loigiai{
		Gọi $4$ số hạng đầu của cấp số nhân đã cho là $u_1$, $u_2$, $u_3$, $u_4$; công bội của cấp số nhân là $ q $, công sai của cấp số cộng là $ d $ $\left(d\neq0\right)$.\\
		Tổng $3$ số hạng đầu của cấp số nhân bằng $\dfrac{148}{9}$ nên $u_1+u_2+u_3=\dfrac{148}{9} \Leftrightarrow u_1+u_1\cdot q+u_1\cdot q^2=\dfrac{148}{9}\left(1\right)$.\\
		Do $u_1$, $u_2$, $u_3$ theo thứ tự chúng là số hạng thứ $1$, thứ $4$, thứ $8$ của một cấp số cộng có công sai $d\neq0$ nên\\
		$\heva{&u_1\cdot q=u_1+3d\left(2\right)\\&u_1\cdot q^2=u_1+7d\left(3\right)}$.\\
		Nhân phương trình $\left(2\right)$ với $ 7 $ và nhân phương trình $\left(3\right)$ với $ 3 $, sau đó trừ hai phương trình theo vế ta được $u_1\left(3q^2-7q+4\right)=0\left(4\right)$.\\
		Từ phương trình $\left(1\right)$ ta có $u_1\neq0$. Khi đó $\left(3\right)\Leftrightarrow 3q^2-7q+4=0 \Leftrightarrow \hoac{&q=1\\&q=\dfrac{4}{3}}$.
		\begin{itemize}
			\item[+)] Với $q=1$, thay vào $\left(1\right)$ suy ra $u_1=u_2=u_3=\dfrac{148}{27}$ (loại do $u_1$, $_2$, $u_3$ theo thứ tự chúng là số
			hạng thứ $1$, thứ $4$, thứ $8$ của một cấp số cộng có công sai $d\neq0$).
			\item[+)] Với $q=\dfrac{4}{3}$, thay vào $\left(1\right)$ suy ra $u_1=4$, $u_2=\dfrac{16}{3}$, $u_3=\dfrac{64}{9}$, $u_4=\dfrac{256}{27}$.
		\end{itemize}
			Vậy $4$ số hạng đầu của cấp số nhân là $u_1=4$, $u_2=\dfrac{16}{3}$, $u_3=\dfrac{64}{9}$, $u_4=\dfrac{256}{27}$.
	}
\end{bt}
\begin{bt}%[Dự án TLDH2-Nhóm Latex, Kiều Ngân]%[2D2B5-6]%Câu 7.
	Một người mỗi tháng đều đặn gửi vào ngân hàng một khoản tiền $T$ theo hình thức lãi kép với lãi suất $0{,}6\%$ mỗi tháng. Biết sau $15$ tháng, người đó có số tiền là $100$ triệu đồng. Hỏi số tiền $T$ gần với số tiền nào nhất trong các số sau?
	\loigiai{
		Với số tiền $T$ gửi đều đặn mỗi tháng theo hình thức lãi kép với lãi suất $r\%$ mỗi tháng, ta có\\
		Sau một tháng, số tiền của người đó là $A_1=T(1+r)$ đồng.\\
		Sau hai tháng, số tiền của người đó là $A_2=[T(1+r)+T](1+r)=T\left[(1+r)^2+(1+r)\right]$ đồng.\\
		Sau ba tháng, số tiền của người đó là
		$$A_3=\left\{T\left[(1+r)^2+(1+r)\right]+T\right\}(1+r)=T\left[(1+r)^3+(1+r)^2+(1+r)\right]\text{ đồng}.$$
		\ldots \\
		Sau mười lăm tháng, số tiền của người đó là
		$$A_{15}=T\left[(1+r)^{15}+(1+r)^{14}+\cdots +(1+r)\right]=\dfrac{T}{r}(1+r)\left[(1+r)^{15}-1\right]\text{ đồng}.$$
		Khi đó $T=\dfrac{A_{15}\cdot r}{(1+r)\left[(1+r)^{15}-1\right]}=\dfrac{10^8\cdot 0{,}006}{1{,}006\cdot (1{,}006^{15}-1)}\approx 6.350.000$ đồng.}
\end{bt}

\Closesolutionfile{ans}
% \inputans{10}{ans/ans-0-GK1-CanhDieu-De1-NH23-24}
% \begin{name}
	{\tenchude}
	{TOÁN 11}
	{LỚP TOÁN THẦY PHÁT}
	{Thời gian: 90 phút - Không kể thời gian phát đề}
\end{name}
\setcounter{ex}{0}\setcounter{bt}{0}
\noindent{\bf\fontfamily{qag}\selectfont\color{violet}A. PHẦN TRẮC NGHIỆM}
\Opensolutionfile{ans}[ans/ans-1-GK1-KNTT-De17-NH23-24]
%%==========Câu 1
\begin{ex}%[1K1Y1-1]
Số đo bằng độ của cung lượng giác $\dfrac{\pi}{12}$ là
\choice
{$\left(\dfrac{75}{2}\right)^{\circ}$}
{$45^{\circ}$}
{$-345^{\circ}$}
{\True $15^{\circ}$}
\loigiai{
	Ta có $\dfrac{\pi}{12}=\dfrac{\pi}{12} \cdot\left(\dfrac{180}{\pi}\right)^o=15^{\circ}.$
}
	\end{ex}
%%%Câu 2
\begin{ex}%[1K1Y1-1] 
Đổi $80^{\circ}$ sang radian
	\choice
{\True $\dfrac{4 \pi}{9}$}
{$\dfrac{2 \pi}{9}$}
{$\dfrac{\pi}{9}$}
{$\dfrac{5 \pi}{9}$}
\loigiai{
 Ta có : $80^{\circ}=80 \cdot \dfrac{\pi}{180}=\dfrac{4 \pi}{9}$.
}
\end{ex}
%%%Câu 3
\begin{ex}%[1K1Y1-3] 
Cung tròn bán kính bằng $8$ cm có số đo $3$ rad có độ dài là
	\choice
	{$\dfrac{8}{3}$ cm}
	{$\dfrac{3}{11}$ cm}
	{$11$ cm}
	{\True $24$ cm}
	\loigiai{
		Độ dài cung tròn $l=R\alpha=8\cdot 3=24$ cm.
	}
\end{ex}
	
%%==========Câu 4
\begin{ex}%[1K1Y1-6]
	Cho góc $\alpha$ thỏa mãn $\sin \alpha=\dfrac{4}{5}$ và $\dfrac{\pi}{2}<\alpha<\pi$. Tính $P=\dfrac{1}{1+\tan^2\alpha}$
	\choice
	{$P=-\dfrac{3}{5}$}
	{$P=\dfrac{3}{5}$}
	{$P=-\dfrac{9}{25}$}
	{\True $P=\dfrac{9}{25}$}
	\loigiai{
Ta có $\cos^2 x=1-\sin^2x=1-\left(\dfrac{4}{5}\right)^2=\dfrac{9}{25}$ nên $1+\tan^2x=\dfrac{1}{\cos^2x}=\dfrac{25}{9}$.\\
Do đó $P=\dfrac{1}{1+\tan^2x}=\dfrac{9}{25}$.
}
\end{ex}
%%%%Câu 5
\begin{ex}%[1K1B1-6]
Cho $\sin a=\dfrac{1}{3}$. Tính $P=\dfrac{3 \cot a+2 \tan a}{\cot a+\tan a}$
\choice
{$P=\dfrac{9}{26}$}
{\True $P=\dfrac{26}{9}$}
{$P=-6$}
{$P=6$}
\loigiai{
Ta có 
\[
P=\dfrac{3 \cot a+2 \tan a}{\cot a+\tan a}=\dfrac{3 \cot ^2 a+2}{\cot ^2 a+1}=\dfrac{3\left(\dfrac{1}{\sin ^2 a}-1\right)+2}{\dfrac{1}{\sin ^2 a}-1+1}=\dfrac{3 \dfrac{1}{\sin ^2 a}-1}{\dfrac{1}{\sin ^2 a}}=\dfrac{26}{9}
\]
}
\end{ex}
%%%%Câu 6
\begin{ex}%[1K1Y2-1]
Trong các công thức sau, công thức nào đúng?
\choice
{$\cos (a-b)=\cos a \cdot \cos b-\sin a \cdot \sin b$}
{\True $\cos (a-b)=\cos a \cdot \cos b+\sin a \cdot \sin b$}
{$\sin (a+b)=\sin a \cdot \cos b-\cos a \cdot \sin b$}
{$\sin (a-b)=\sin a \cdot \cos b+\cos a \cdot \sin b$}
\loigiai{
}
\end{ex}
%%%%Câu 7
\begin{ex}%[1T1Y3-2]
Trong các công thức sau, công thức nào \textbf{sai}?
\choice
{$\cos 2 a=\cos ^2 a-\sin ^2 a$}
{\True $\cos 2 a=\cos ^2 a+\sin ^2 a$}
{$\cos 2 a=2 \cos ^2 a-1$}
{$\cos 2 a=1-2 \sin ^2 a$}
\loigiai{

}
\end{ex}
%%%%Câu 8
\begin{ex}%[1K1Y2-2]
Cho $\cos 2 \alpha=\dfrac{1}{2}$. Tính giá trị biểu thức $P=5 \sin ^2 \alpha-4 \cos ^2 \alpha$.
\choice
{$\dfrac{7}{4}$}
{$-\dfrac{5}{8}$}
{\True $-\dfrac{7}{4}$}
{$\dfrac{1}{8}$}
 \loigiai{
  	$$
\begin{aligned}
	P=5 \sin ^2 \alpha-4 \cos ^2 \alpha & =5\left(\frac{1-\cos 2 \alpha}{2}\right)-4\left(\frac{1+\cos 2 \alpha}{2}\right) \\
	& =\frac{1}{2}-\frac{9}{2} \cos 2 \alpha=\frac{1}{2}-\frac{9}{2} \cdot \frac{1}{2}=-\frac{7}{4} .
\end{aligned}
$$
}
\end{ex}
%%%%%câu 9
\begin{ex}%[1K1B2-1]
Cho hai góc $\alpha, \beta$ thỏa mãn $\sin \alpha=\dfrac{5}{13},\left(\dfrac{\pi}{2}<\alpha<\pi\right)$ và $\cos \beta=\dfrac{3}{5},\left(0<\beta<\dfrac{\pi}{2}\right)$. Tính giá trị đúng của $\cos (\alpha-\beta)$.
\choice
{$\dfrac{16}{65}$}
{\True $-\dfrac{16}{65}$}
{$\dfrac{18}{65}$}
{$-\dfrac{18}{65}$}
\loigiai{
Ta có\\
$\sin \alpha=\dfrac{5}{13},\left(\dfrac{\pi}{2}<\alpha<\pi\right)$ nên $\cos \alpha=-\sqrt{1-\sin ^2 \alpha}=-\sqrt{1-\left(\dfrac{5}{13}\right)^2}=-\dfrac{12}{13}$.\\
$\cos \beta=\dfrac{3}{5},\left(0<\beta<\dfrac{\pi}{2}\right)$ nên $\sin \beta=\sqrt{1-\cos ^2 \beta}=\sqrt{1-\left(\dfrac{3}{5}\right)^2}=\dfrac{4}{5}$.\\
Do đó $\cos (\alpha-\beta)=\cos \alpha \cos \beta+\sin \alpha \sin \beta=-\dfrac{12}{13} \cdot \dfrac{3}{5}+\dfrac{5}{13} \cdot \dfrac{4}{5}=-\dfrac{16}{65}$.
}
\end{ex}
%%%%Câu 10
\begin{ex}%[1K1K2-3]
Rút gọn biểu thức: $\dfrac{\sin a+\sin 3 a+\sin 5 a}{\cos a+\cos 3 a+\cos 5 a}$.
\choice
{\True $\tan 3a$}
{$\tan a$}
{$2 \tan 3 a$}
{$\cot 3 a$}
 \loigiai{
 	Ta có
	$$
	\begin{aligned}
		 \dfrac{\sin a+\sin 3 a+\sin 5 a}{\cos a+\cos 3 a+\cos 5 a}&=\dfrac{\sin a+\sin 5 a+\sin 3 a}{\cos a+\cos 5 a+\cos 3 a}=\dfrac{2 \sin 3 a \cdot \cos a+\sin 3 a}{2 \cos 3 a \cdot \cos a+\cos 3 a} \\
		&= \dfrac{\sin 3 a \cdot(2 \cos a+1)}{\cos 3 a \cdot(2 \cos a+1)}=\tan 3 a .
	\end{aligned}
	$$
}
\end{ex}
%%%%Câu 11
\begin{ex}%[1K1Y3-1]
Xét bốn mệnh đề sau:
\begin{enumerate}
	\item [(1)] Hàm số $y=\sin x$ có tập xác định là $\mathbb{R}$.
	\item [(2)] Hàm số $y=\cos x$ có tập xác định là $\mathbb{R}$.
	\item [(3)] Hàm số $y=\tan x$ có tập xác định là $D=\mathbb{R} \backslash\left\{\dfrac{\pi}{2}+k \pi \mid k \in \mathbb{Z}\right\}$.
	\item [(4)] Hàm số $y=\cot x$ có tập xác định là $D=\mathbb{R} \backslash\left\{k \dfrac{\pi}{2} \mid k \in \mathbb{Z}\right\}$.
\end{enumerate}
Số mệnh đề đúng là 
\choice
{\True $3$}
{$2$}
{$1$}
{$4$}
\loigiai{
Các mệnh đề đúng là:
\begin{enumerate}[1)]
	\item  Hàm số $y=\sin x$ có tập xác định là $\mathbb{R}$.
\item (2) Hàm số $y=\cos x$ có tập xác định là $\mathbb{R}$.
\item (3) Hàm số $y=\tan x$ có tập xác định là $D=\mathbb{R} \backslash\left\{\frac{\pi}{2}+k \pi \mid k \in \mathbb{Z}\right\}$.
\end{enumerate}
}
\end{ex}
%%%%Câu 12
\begin{ex}%[1K1Y3-4]
Chu kỳ tuần hoàn của hàm số $y=\tan x$ là
\choice
{$k \pi, (k \in \mathbb{Z})$}
{\True $\pi$}
{$\dfrac{\pi}{3}$}
{$3 \pi$}
\loigiai{
	Hàm số $y=\tan x$ tuần hoàn với chu kỳ là $\pi$.
}
\end{ex}
%%%Câu 13
\begin{ex}%[1K1B3-4]
Tìm chu kì của hàm số $f(x)=\sin \dfrac{x}{2}+2 \cos \dfrac{3 x}{2}$.
\choice
{$5 \pi$}
{$\dfrac{\pi}{2}$}
{\True $4 \pi$}
{$2 \pi$}
\loigiai{
Chu kỳ của $\sin \dfrac{x}{2}$ là $T_1=\dfrac{2 \pi}{\left|\dfrac{1}{2}\right|}=4 \pi$ và chu kỳ của $\cos \dfrac{3 x}{2}$ là $T_2=\dfrac{2 \pi}{\left|\dfrac{3}{2}\right|}=\dfrac{4 \pi}{3}$.\\
Chu kì của hàm ban đầu là bội chung nhỏ nhất của hai chu kì $T_1$ và $T_2$ vừa tìm được ở trên. Do đó chu kì của hàm ban đầu $T=4 \pi$.  
}
\end{ex}
%%%%Câu 14
\begin{ex}%[1K1B3-3]
Xét tính chẵn lẻ của hàm số $y=\dfrac{\sin 2 x}{2 \cos x-3}$ thì $y=f(x)$ là
\choice
{Hàm số chẵn}
{\True Hàm số lẻ}
{Không chẵn không lẻ}
{Vừa chẵn vừa lẻ}
 \loigiai{
 Tập xác định $\mathscr{D}=\mathbb{R}$.\\
 Ta có $\forall x \in \mathscr{D} \Rightarrow-x \in \mathscr{D}$.\\
 $f(-x)=\dfrac{\sin (-2 x)}{2 \cos (-x)-3}=\dfrac{-\sin 2 x}{2 \cos x-3}=-f(x)$.\\
 Vậy hàm số đã cho là hàm số lẻ.}
\end{ex}
%%%%Câu 15
\begin{ex}%[1K1B3-5]
Giá trị nhỏ nhất của hàm số $y=2 \cos ^2 x-\sin 2 x+5$.
\choice
{$\sqrt{2}$}
{$-\sqrt{2}$}
{\True $6-\sqrt{2}$}
{$6+\sqrt{2}$}
\loigiai{
Ta có $y=2 \cos ^2 x-\sin 2 x+5=\cos 2 x-\sin 2 x+6=\sqrt{2} \cos \left(2 x+\dfrac{\pi}{4}\right)+6$.\\
	Do $-\sqrt{2} \leq \sqrt{2} \cos \left(2 x+\dfrac{\pi}{4}\right) \leq \sqrt{2}$ nên $-\sqrt{2}+6 \leq \sqrt{2} \cos \left(2 x+\dfrac{\pi}{4}\right)+6 \leq \sqrt{2}+6$.\\
	Vậy giá trị nhỏ nhất của hàm số $y=2 \cos ^2 x-\sin 2 x+5$ là $6-\sqrt{2}$.
}
\end{ex}
%%%Câu 16
\begin{ex}%[1K1Y4-3]
Cho $x=\dfrac{\pi}{2}+k 2 \pi, k \in \mathbb{Z}$ là nghiệm của phương trình nào sau đây
\choice
{$\sin x=0$}
{\True $\sin x=1$}
{$\sin x=-1$}
{$\cos x=1$}
 \loigiai{
	$$
	\sin x=1 \Leftrightarrow x=\frac{\pi}{2}+k 2 \pi, k \in \mathbb{Z}.
	$$
}
\end{ex}
%%%Câu 17
\begin{ex}%[1K1Y4-3]
Trong các phương trình sau, phương trình nào vô nghiệm?
\choice 
{$\sin x=\dfrac{1}{2}$}
{\True $\sin x=\dfrac{5}{3}$}
{$\tan x=-2023$}
{$\cos x=\dfrac{3}{5}$}
 \loigiai{
Phương trình $\sin x=a$ có nghiệm khi $|a| \leq 1$.
}
\end{ex}
%%%%Vâu 18
\begin{ex}%[1K1Y4-3]
Phương trình $\sin x=m-1$ có nghiệm khi $m$ là
\choice
{$-1 \leq m \leq 1$}
{\True $0 \leq m \leq 2$}
{$m \leq 0$}
{$-1 \leq m \leq 0$}
\loigiai{
Phương trình $\sin x=m-1$ có nghiệm khi $-1 \leq m-1 \leq 1 \Leftrightarrow 0 \leq m \leq 2$
}
\end{ex}
%%%Câu 19
\begin{ex}%[1K1B4-3]
Phương trình $1+2 \sin x \cos x=0$ có nghiệm là
\choice
{$x=\dfrac{\pi}{2}+k 2 \pi$}
{\True $x=-\dfrac{\pi}{4}+k \pi$}
{$x=-\dfrac{\pi}{3}+k 2 \pi$}
{$x=-\dfrac{\pi}{3}+k \pi$}
 \loigiai{
Phương trình: $1+2 \sin x \cos x=0 \Leftrightarrow \sin 2 x=-1 \Leftrightarrow 2 x=-\dfrac{\pi}{2}+k 2 \pi \Leftrightarrow x=-\dfrac{\pi}{4}+k \pi$
}
\end{ex}
%%%Câu 20
\begin{ex}%[1K1Y4-A]
Phương trình $(2 \cos x+1)(\cos 2 x-\sqrt{3})=0$ có nghiệm là
\choice
{$x=\dfrac{\pi}{2}+k 2 \pi$}
{\True $x= \pm \dfrac{2 \pi}{3}+k 2 \pi$}
{$x= \pm \dfrac{\pi}{4}+k 2 \pi$}
{$x= \pm \dfrac{\pi}{6}+k 2 \pi$}
  \loigiai{
Phương trình 
\allowdisplaybreaks
\begin{eqnarray*}
(2 \cos x+1)(\cos 2 x-\sqrt{3})=0 \Leftrightarrow \hoac{&\cos x=-\dfrac{1}{2} \\ &\cos 2 x=\sqrt{3} \quad (\text{Vô Nghiệm})}
\end{eqnarray*}
Với $	\cos x=-\dfrac{1}{2} \Leftrightarrow x= \pm \dfrac{2 \pi}{3}+k 2 \pi
	$}
\end{ex}
%%%Câu 21
\begin{ex}%[1K1B4-A]
\textit{(Bảng số liệu sau dùng cho câu 21-24)}
Quãng đường (km) các cầu thủ (không tính thủ môn) chạy trong một trận bóng đá tại giải ngoại hạng Anh được cho trong bảng thống kê sau:
	\begin{center}
		\begin{tabular}{|c|c|c|c|c|c|}
			\hline Quãng đường & {$[2 ; 4)$} & {$[4 ; 6)$} & {$[6 ; 8)$} & {$[8 ; 10)$} & {$[10 ; 12)$} \\
			\hline Số cầu thủ & 2 & 5 & 6 & 9 & 3 \\
			\hline
		\end{tabular}
	\end{center}
	Tính quãng đường trung bình một cầu thủ chạy trong trận đấu này.
	\choice{$7{,}02$}{\True $7{,}48$}{$5{,}23$}{$8{,}36$}
		\loigiai{
			Tổng số cầu thủ là $n=2+5+6+9+3=25$.\\
			Quãng đường trung bình một cầu thủ chạy trong trận đấu này là
			$$\overline{x}=\dfrac{2\cdot 3+5\cdot 5+6\cdot 7+9\cdot 9+3\cdot 11}{25}=7{,}48~(\mathrm{km}).$$ 
		}
\end{ex}
%%%%Câu 22
\begin{ex}%[1K2Y5-2]
	Tìm trung vị của mẫu số liệu.
	\choice{\True $7{,}83$}{$7{,}48$}{$6{,}23$}{$3{,}56$}
		\loigiai{
			Cỡ mẫu $n = 2 + 5 + 6 + 9 + 3 = 25$.\\
			Gọi $x_1, x_2, \ldots, x_{25}$ là quãng đường chạy của $25$ cầu thủ và giả sử dãy này đã được sắp xếp theo thứ tự không giảm. Khi đó, trung vị là $x_{13}$, mà $x_{13}$ thuộc nhóm $[6; 8)$ nên nhóm này chứa trung vị. Do đó, trung vị là
			$$	M_e=6+\dfrac{\dfrac{25}{2}-(2+5)}{6}\cdot (8-6) \approx 7{,}83.	$$			
		}
\end{ex}
%%%Câu 23
\begin{ex}%[1K2Y5-2]
Tìm $a$ sao cho có $25 \%$ số cầu thủ tham gia trận đấu chạy ít nhất $a~(\mathrm{km})$.
\choice
{\True $9{,}28$}
{$7{,}48$}
{$12{,}23$}
{$13{,}56$}
		\loigiai{
			Số $a$ thỏa mãn có $25 \%$ số cầu thủ tham gia trận đấu chạy ít nhất $a$ (km).\\
			Do đó, $a$ chính là tứ phân vị thứ ba của mẫu số liệu trên.\\
			Cỡ mẫu $n=25$.\\
			Gọi $x_1, x_2 \ldots, x_{25}$ là quãng đường chạy của $25$ cầu thủ và giả sử dãy này đã được sắp xếp theo thứ tự không giảm. Khi đó tứ phân vị thứ ba là $\dfrac{x_{19}+x_{20}}{2}$.\\ Do $x_{19}, x_{20}$ đều thuộc nhóm $[8 ; 10)$ nên nhóm này chứa tứ phân vị thứ ba.\\ Do đó $a=Q_3=8+\dfrac{\dfrac{3.25}{4}-(2+5+6)}{9} \cdot(10-8) \approx 9{,}28$. 
		}
\end{ex}
%%%%Câu 24
\begin{ex}%[1K2B5-3]
Tính mốt của mẫu số liệu thu được.
\choice{$9{,}28$}{$7{,}48$}{\True $8{,}67$}{$13{,}56$}
		\loigiai{
			Tần số lớn nhất là $9$ nên nhóm chứa mốt là $[8; 10)$.\\
			Mốt là $M_o=8+\dfrac{(9-6)}{(9-6)+(9-3)} \cdot 2 \approx 8{,}67$.\\
}
\end{ex}
%%%%%Câu 25
\begin{ex}%[1K2B5-4]
Cho dãy số $\left(u_n\right)$ biết $u_n=\dfrac{4 n+5}{n+1}$. Mệnh đề nào sau đây đúng?
\choice
{Dãy số bị chặn trên}
{Dãy số bị chặn dưới}
{\True Dãy số bị chặn}
{Không bị chặn}
\loigiai{
Ta có $u_n=\dfrac{4 n+5}{n+1}>0, \forall n \in \mathbb{N}^*$. Khi đó
$$
u_n=\dfrac{4 n+5}{n+1}=\dfrac{4(n+1)+1}{n+1}=4+\dfrac{1}{n+1} \leq 4+\dfrac{1}{2}=\dfrac{9}{2} \Rightarrow u_n \leq \dfrac{9}{2}, \forall n \in \mathbb{N}^*
$$
Suy ra $0<u_n \leq \dfrac{9}{2}, \forall n \in \mathbb{N}^*$.\\
Vậy dãy số $\left(u_n\right)$ bị chặn.
}
\end{ex}
%%%%Câu 26
\begin{ex}%[1K2Y6-3]
Cho cấp số cộng có số hạng đầu $u_1=-\dfrac{1}{2}$, công sai $d=\dfrac{1}{2}$. Năm số hạng liên tiếp đầu tiên của cấp số này là
\choice
{$-\dfrac{1}{2}; 0; 1; \dfrac{1}{2}; 1$}
{$-\dfrac{1}{2}; 0; \dfrac{1}{2}; 0; \dfrac{1}{2}$}
{$\dfrac{1}{2}; 1; \dfrac{3}{2}; 2; \dfrac{5}{2}$}
{\True $-\dfrac{1}{2}; 0; \dfrac{1}{2}; 1; \dfrac{3}{2}$}
 \loigiai{
Ta có $u_1=-\dfrac{1}{2}$ và $d=\dfrac{1}{2}$ nên  \[\heva{&u_1=-\dfrac{1}{2} \\ &u_2=u_1+d=0 \\ &u_3-u_2+d=\dfrac{1}{2} \\ &u_4=u_3+d=1 \\ &u_5=u_4+d=\dfrac{3}{2}}.\]
Vậy $5$ số hạng cần tìm là $-\dfrac{1}{2}; 0; \dfrac{1}{2}; 1; \dfrac{3}{2}$.
}
\end{ex}
%%%%Câu 27
\begin{ex}%[1K2Y6-3]
Cho cấp số cộng $\left(u_n\right)$ biết $u_1=2$ và công sai $d=3$. Tính $u_2$ bằng
\choice
{\True $5$}
{$6$}
{$7$}
{$8$}
\loigiai{
	Ta có: $u_2=u_1+d=2+3=5$.
}
\end{ex}
%%%Câu 28
\begin{ex}%[1K2B6-1]
Cho cấp số cộng $\left(u_n\right)$ thỏa mãn $\heva{&u_4=10 \\& u_4+u_6=26}$ có công sai là
\choice
{$d=-3$}
{\True $d=3$}
{$d=5$}
{$d=6$}
\loigiai{
Ta có: $\heva{&u_4=10 \\& u_4+u_6=26} \Leftrightarrow\heva{&u_1+3 d=10 \\& 2 u_1+8 d=26} \Leftrightarrow\heva{&u_1=1 \\& d=3.}$
\\Vậy công sai $d=3$.
}
\end{ex}
%%%%Câu 29
\begin{ex}%[1K2B6-1]
Cho cấp số cộng $\left(u_n\right)$ biết $u_1=1$ và $u_4=10$. Công sai $d$ bằng
\choice
{$-2$}
{$1$}
{\True $3$}
{$-4$}
\loigiai{
Ta có $u_4=u_1+3 d=1+3 d=10 \Leftrightarrow d=3$.
}
\end{ex}
%%%%Câu 30
\begin{ex}%[1K2B6-5]
Tính tổng $K=15+20+25+\ldots+7510$.
\choice
{$5\,634\,750$ }
{$5\,643\,705$}
{$5\,643\,250$}
{\True $5\,643\,750$}
 \loigiai{	Ta thấy các số hạng của tổng $K$ tạo thành một cấp số cộng với số hạng đầu $u_1=15$ và công sai $d=5$.\\
	Giả sử tổng trên có $n$ số hạng thì 
	\[u_n=7510
	\Leftrightarrow u_1+(n-1) d=7\,510 \Leftrightarrow 15+(n-1) 5=7\,510 \Leftrightarrow n=1\,500.\]
Vậy $K=S_{1\,500}=\dfrac{1\,500(15+7\,510)}{2}=5\,643\,750$.}
\end{ex}
%%%%Câu 31
\begin{ex}%[1K2Y7-1]
Cho cấp số nhân $\left(u_n\right)$ biết $u_1=3$ và công bội $q=2$. Tính $u_4$ bằng
\choice
{$9$}
{$16$}
{\True $24$}
{$48$}
 \loigiai{
	Ta có $u_4=u_1 \cdot q^3=3 \cdot 2^3=24$.
}
\end{ex}
%%%%Câu 32
\begin{ex}%[1K2Y7-1]
Cho cấp số nhân $\left(u_n\right)$ biết $u_1=5$ và $u_2=-20$. Công bội $q$ bằng
\choice
{$-2$}
{$4$}
{$3$}
{\True $-4$}
\loigiai{	Ta có $q=\dfrac{u_2}{u_1}=\dfrac{-20}{5}=-4$.
}
\end{ex}
%%%%Câu 33
\begin{ex}%[1K2Y7-3]
Cho cấp số nhân: $1 ; 2 ; 4 ; 8 ; \ldots$ Số hạng thứ năm là
\choice
{$10$}
{\True $16$}
{$12$}
{$32$}
\loigiai{	Ta có công bội $q=\dfrac{u_2}{u_1}=\dfrac{2}{1}=2$.\\
	Suy ra $u_5=u_4 \cdot q=8.2=16$.
}
\end{ex}
%%%%Câu 34
\begin{ex}%[1K2B7-5]
Cho cấp số nhân $\left(u_n\right)$ biết $u_1=5$ và công bội $q=2$. Tính tổng của $10$ số hạng đầu tiên $S_{10}$
\choice
{$4225$}
{$4115$}
{$5225$}
{\True $5115$}
 \loigiai{
Ta có: $S_{10}=u_1 \cdot \dfrac{1-q^{10}}{1-q}=5 \cdot \dfrac{1-2^{10}}{1-2}=5115$.
}
\end{ex}
%%%%Câu 35
\begin{ex}%[1K2B7-1]
Cho cấp số nhân $\left(u_n\right)$ biết $\heva{&u_4+u_6=540 \\& u_1+u_3=20}$. Công bội $q$ bằng
\choice
{$6$}
{$27$}
{$2$}
{\True$3$}
 \loigiai{
	Ta có:
	\allowdisplaybreaks
	\begin{eqnarray*}
\heva{&u_4+u_6=540 \\& u_1+u_3=20} \Leftrightarrow\heva{&u_1 q^3 \cdot\left(1+q^2\right)=540 \\& u_1\left(1+q^2\right)=20} \Leftrightarrow\heva{&q^3=27 \\& u_1\left(1+q^2\right)=20} \Leftrightarrow\heva{&q=3 \\& u_1=2.}
\end{eqnarray*}
}
\end{ex}


\Closesolutionfile{ans}
% \inputans{10}{ans/ans-1-GK1-KNTT-De17-NH23-24}
\noindent{\bf\fontfamily{qag}\selectfont\color{violet}B. PHẦN TỰ LUẬN}
\setcounter{bt}{0}
%%==========Bài 1
 \begin{bt}%[1K1Y1-6]
Cho $\sin \alpha=\dfrac{3}{5}$ và $\dfrac{\pi}{2}<\alpha<\pi$. Tính giá trị của $\cos \alpha$ ?
\loigiai{
 Ta có 
 \[\sin ^2 \alpha+\cos ^2 \alpha=1 \Rightarrow \cos ^2 \alpha=1-\sin ^2 \alpha=1-\dfrac{9}{25}=\dfrac{16}{25} \Leftrightarrow\hoac{&\cos \alpha=\frac{4}{5} \\& \cos \alpha=-\dfrac{4}{5}.}\] 
 Vì $\dfrac{\pi}{2}<\alpha<\pi$ nên $\cos \alpha<0$, do đó $\cos \alpha=-\dfrac{4}{5}$.
} 
 \end{bt}
%%%%%Bài 37
\begin{bt}%[1K2B6-6]
Trong dịp nghỉ lễ 02/9 gia đình anh An cần thuê một xe Taxi để di chuyển từ thủ đô Hà Nội về thăm quê tại TP Bắc Giang. Biết giá của kilômét đầu tiên là $10.000$ đồng, kể từ kilômét thứ 2 giá của mỗi kilômét tăng thêm $500$ đồng so với giá của kilômét trước đó. Biết quãng đường Taxi di chuyển từ thủ đô Hà Nội về TP Bắc Giang là $80$ km. Hỏi gia đình anh An phải trả bao nhiêu tiền cho chuyến Taxi đó?
\loigiai{
Số tiền ở từng kilômet lập thành một cấp số cộng.\\
Số hạng đầu tiên là $u_1=10.000$, công sai $d=500$.\\
Cấp số cộng có $80$ số hạng.\\
Tổng số tiền là $S_{80}=80.10000+\dfrac{79.80 .500}{2}=2.380 .000$ đồng.
}
\end{bt}
%%%%Bài 38
\begin{bt}%[1D1T5-6]
	\immini{
		Một chiếc guồng nước có dạng hình tròn bán kính $2{,}5$ m; trục của nó đặt cách mặt nước $2$ m (hình bên). Khi guồng quay đều, khoảng cách $h$ (m) tính từ một chiếc gầu tại điểm $A$ trên guồng đến mặt nước là $h=|y|$ trong đó $$y=2+2{,}5\sin 2\pi\left(x-\dfrac{1}{4}\right)$$ với $x$ là thời gian quay của guồng ($x\ge 0$), tính bằng phút; ta quy ước rằng $y>0$ khi gầu ở trên mặt nước và $y<0$ khi gầu ở dưới mạt nước
		\begin{enumerate}
			\item Khi nào chiếc gầu ở vị trí cao nhất? Thấp nhất?
			\item Chiếc gầu cách mặt nước $2$ mét lần đầu tiên khi nào?
		\end{enumerate}
	}
	{
		\begin{tikzpicture}[>=stealth,line join=round,line cap=round,font=\footnotesize,scale=1]
			\def\r{2.5}
			\coordinate[label=above:$O$] (O) at (0,0);
			\draw[name path=(C)] (O) circle (\r);
			\coordinate (M) at (0,-2);
			\path 
			(O)++(45:\r)coordinate(A)node[above right]{$A$}
			(-3,-2)coordinate(B)
			(3,-2)coordinate(C)
			($(B)!(A)!(C)$)coordinate(H)
			;
			\draw[dashed] (A)--(H)node[midway,right]{$h$ m}
			(B)--(C)
			;
			\draw[<->](O)--(A)node[midway,left]{$2{,}5$ m};
			\draw[<->](O)--(M)node[midway,right]{$2$ m};
			\draw[->] (A) arc (45:60:2.5);
		\end{tikzpicture}
	}
	\loigiai 
	{
		\begin{enumerate}
			\item Với mọi $x\in \mathbb{R}$, ta có
			\allowdisplaybreaks
			\begin{eqnarray*}
				-1\le \sin 2\pi\left(x-\dfrac{1}{4}\right)\le 1&\Leftrightarrow&-2{,}5\le 2{,}5\sin 2\pi\left(x-\dfrac{1}{4}\right)\le 2{,}5\\
				&\Leftrightarrow&-0{,}5\le 2+2{,5}\sin 2\pi\left(x-\dfrac{1}{4}\right)\le 4{,}5.
			\end{eqnarray*}
			Suy ra, gầu ở vị trí cao nhất khi 
			$$\sin 2\pi\left(x-\dfrac{1}{4}\right)=1\Leftrightarrow 2x\left(x-\dfrac{1}{4}\right)=\dfrac{\pi}{2}+k2\pi\Leftrightarrow x=\dfrac{1}{2}+k, \,(k\in\mathbb{Z}).$$
			Vậy gầu ở vị trí cao nhất tại các thời điểm $\dfrac{1}{2}$, $\dfrac{3}{2}$, $\dfrac{5}{2}$,$\ldots$ phút.\\
			Tương tự, gầu ở vị trí thấp nhất khi 
			$$\sin 2\pi\left(x-\dfrac{1}{4}\right)=-1\Leftrightarrow 2x\left(x-\dfrac{1}{4}\right)=-\dfrac{\pi}{2}+k2\pi\Leftrightarrow xk, \,(k\in\mathbb{Z}).$$
			Vậy gầu ở vị trí thấp nhất tại các thời điểm $0$, $1$, $2$,$\ldots$ phút.
			\item Gầu các mặt nước $2$ m nên 
			\allowdisplaybreaks
			\begin{eqnarray*}
				2+2{,}5\sin 2\pi\left(x-\dfrac{1}{2}\right)=2&\Leftrightarrow&\sin 2\pi\left(x-\dfrac{1}{2}\right)=0\\
				&\Leftrightarrow&2\pi\left(x-\dfrac{1}{4}\right)=k\pi\\
				&\Leftrightarrow&x=\dfrac{1}{4}+\dfrac{k}{2},\,(k\in\mathbb{Z}).
			\end{eqnarray*}
			Vậy chiếc gầu cách mặt nước $2$ m lần đầu tiên tại thời điểm $x=\dfrac{1}{4}$ phút.
		\end{enumerate}		
	}
\end{bt}
%%%%Bài 39
\begin{bt}%[1K2B7-7]
		Từ tờ giấy, cắt một hình tròn bán kính $R$ (cm). Tiếp theo, cắt hai hình tròn bán kính $\dfrac{R}{2}$ rồi chồng lên hình tròn đầu tiên . Tiếp theo, cắt bốn hình tròn bán  kính $\dfrac{R}{4}$ 
		rồi chồng lên các hình trước. Cứ thế tiếp tục mãi. Tính tổng diện tích của các hình tròn.
	\begin{center}
		\begin{tikzpicture}[>=stealth,line join=round,line cap=round,font=\footnotesize,scale=1,declare function={r=1.5;}]
			\begin{scope}
				\draw (0,0)circle(r);
				\path (0,-r) node[below]{$a)$};
			\end{scope}
			\begin{scope}[xshift={3.5cm}]
				\draw(0,0)circle(r);
				\draw (r/2,0)circle(r/2);
				\draw(-r/2,0)circle(r/2);
				\path (0,-r) node[below]{$b)$};
			\end{scope}
			\begin{scope}[xshift={7cm}]
				\draw (0,0)circle(r);
				\draw (r/2,0)circle(r/2);
				\draw (-r/2,0)circle(r/2);
				\foreach \i in {0,1,2,3}
				\draw[shift={(r/2*\i,0)}] (-3*r/4,0)circle(r/4);
				\path (0,-r) node[below]{$c)$};
			\end{scope}
			% \path (current bounding box.south) node[below]{Hình $3$};
		\end{tikzpicture}
	\end{center}
	\loigiai{
		Diện tích của các hình tròn trong các lần cắt là
		\begin{enumerate}
			\item Lần thứ 1: $S_1=\pi R^2$.
			\item  Lần thứ 2: $S_2=2\cdot \pi \left(\dfrac{R}{2}\right)^2= \dfrac{\pi R^2}{2}$.
			\item  Lần thứ 3: $S_2=4\cdot \pi \left(\dfrac{R}{4}\right)^2= \dfrac{\pi R^2}{2^2}$.	
			\item Lần thứ $n$: $S_n= \dfrac{\pi R^2}{2^{n-1}}$.
		\end{enumerate}
		Do đó  diện tích các hình tròn lập thành một cấp số nhân lùi vô hạn có số hạng đầu $S_1=\pi R^2$ và công bội $q=\dfrac{1}{2}$ nên tổng diện tích các hình tròn là 
		\[ S_1+S_2+\cdots=\dfrac{\pi R^2}{1-\dfrac{1}{2}}=2\pi R^2. \]
	}
\end{bt}



%Chương IV
%%Bài 10. ĐT MP
% \setcounter{section}{9} \setcounter{dang}{0}
\section{ĐƯỜNG THẲNG VÀ MẶT PHẲNG TRONG KHÔNG GIAN}
\subsection{KIẾN THỨC CẦN NHỚ}
\subsubsection{CÁC KHÁI NIỆM MỞ ĐẦU}
\begin{enumerate}[\iconMT]
	\item \indam{Mặt phẳng:} Để biểu diễn mặt phẳng, người ta dùng hình bình hành hay một miền góc
	\begin{listEX}[2]
		\item [] \begin{tikzpicture}[scale=0.7, line join=round, line cap=round]
			\tkzDefPoints{0/0/A,4/0/B,5.5/2/C,1.5/2/D}
			\tkzDrawPolygon(A,B,C,D)
			\draw (A)--(B)--(C)--(D)--cycle;
			\tkzMarkAngles[size=0.7cm,arc=l](B,A,D)
			\tkzLabelAngles[pos=0.4,rotate=30](B,A,D){\footnotesize$P$}
			\node[right] at (-0.4,-1) {Kí hiệu $(P)$ hoặc mp$(P)$};
		\end{tikzpicture}
		\item [] \begin{tikzpicture}[scale=0.7, line join=round, line cap=round]
			\tkzDefPoints{0/0/A,4/0/B,1.5/2/D}
			\tkzDrawSegments(A,B A,D)
			\tkzMarkAngles[size=0.7cm,arc=l](B,A,D)
			\tkzLabelAngles[pos=0.4,rotate=30](B,A,D){\footnotesize$\alpha$}
			\node[right] at (-0.4,-1) {Kí hiệu $(\alpha)$ hoặc mp$(\alpha)$};
		\end{tikzpicture}
	\end{listEX}
	\item \indam{Điểm thuộc mặt phẳng:}
	Cho điểm $A$, $B$ và mặt phẳng $(\alpha)$.
		\begin{tcolorbox}[colframe=\maudl,colback=cyan!3!white,boxrule=0.5mm]
	\immini{
			\begin{itemize}
				\item [\ding{172}] Khi $A$ thuộc mặt phẳng $(\alpha)$, ta kí hiệu $A \in (\alpha)$.
				\item [\ding{173}] Khi $B$ không thuộc mặt phẳng $(\alpha)$, ta kí hiệu $B \notin (\alpha)$.
				\begin{note}
					Dấu hiệu nhận biết $A \in (\alpha)$ là điểm $A$ thuộc một đường thẳng nằm trong $(\alpha)$
				\end{note}
			\end{itemize}
			}{
		\begin{tikzpicture}[scale=0.7, font=\footnotesize,>=stealth]
			\path
			%	Vẽ mp
			(0,0) coordinate (M)
			(5,0) coordinate (N)
			(6,2) coordinate (P)
			(1,2) coordinate (Q)
			(2,3) coordinate (B)
			(3,1) coordinate (A)
			(3.5,0) coordinate (A')
			;
			\draw (M)--(N)--(P)--(Q)--cycle (1.5,4)--(A) (A')--(4,-1);
			\draw[dashed] (A)--(A');
			\foreach \x/\g in {B/0,A/60}\draw[fill=black] (\x) circle (.05) +(\g:.5)node{\footnotesize$\x$};
			\draw
			pic["$P$",draw,angle radius=6mm]{angle=N--M--Q};
			\end{tikzpicture}
		}
	\end{tcolorbox}
		\item \indam{Biểu diễn hình không gian lên một mặt phẳng:}
		\begin{tcolorbox}[colframe=\maudl,colback=cyan!3!white,boxrule=0.5mm]
			\begin{itemize}
				\item[\ding{172}] Dùng nét vẽ liền để biểu diễn cho những đường trông thấy và dùng nét đứt đoạn (- - - -) để biểu diễn cho những đường bị che khuất.
				\item[\ding{172}]  Quan hệ thuộc, song song được giữ nguyên, nghĩa là
				\begin{itemize}
					\item Nếu hình thực tế điểm $A$ thuộc đường thẳng $\Delta$ thì hình biểu diễn phải giữ nguyên quan hệ đó.
					\item Nếu hình thực tế hai đường thẳng song song thì hình biểu diễn phải giữ nguyên quan hệ đó.
				\end{itemize}
			\end{itemize}
		\end{tcolorbox}
	\indamm{Hình biểu diễn của các mô hình không gian thường gặp:}\
	\begin{center}
		\begin{tikzpicture}[scale=0.5, line join=round, line cap=round]
			\tkzDefPoints{0/0/B,1.3/-1.6/C,4.5/0/D,1/3.5/A}
			\tkzDrawPolygon(A,B,C,D)
			\tkzDrawSegments(A,C)
			\tkzDrawSegments[dashed](B,D)
			\tkzDrawPoints[fill=black,size=4](A,B,C,D)
			\tkzLabelPoints[above,font=\footnotesize](A)
			\tkzLabelPoints[below,font=\footnotesize](C)
			\tkzLabelPoints[left,font=\footnotesize](B)
			\tkzLabelPoints[right,font=\footnotesize](D)
			\node[below right] at (0,-2.4) {Hình tứ diện};
		\end{tikzpicture}
		\begin{tikzpicture}[scale=0.5, line join=round, line cap=round]
			\tkzDefPoints{0/0/A,-1.3/-1.6/B,2.5/-1.6/C}
			\coordinate (D) at ($(A)+(C)-(B)$);
			\coordinate (S) at ($(A)+(100:3)$);
			\tkzDrawPolygon(S,B,C,D)
			\tkzDrawSegments(S,C)
			\tkzDrawSegments[dashed](A,S A,B A,D)
			\tkzDrawPoints[fill=black,size=4](D,C,A,B,S)
			\tkzLabelPoints[above,font=\footnotesize](S)
			\tkzLabelPoints[below,font=\footnotesize](A,B,C)
			\tkzLabelPoints[right,font=\footnotesize](D)
			\node[below] at (0.6,-2.4) {Hình chóp tứ giác đáy hbh};
		\end{tikzpicture}
		\begin{tikzpicture}[scale=0.5, line join=round, line cap=round]
			\tkzDefPoints{0/0/A,-1.3/-1.1/B,2/-1.1/C}
			\coordinate (D) at ($(A)+(C)-(B)$);
			\coordinate (A') at ($(A)+(0,2.5)$);
			\tkzDefPointsBy[translation=from A to A'](B,C,D){B'}{C'}{D'}
			\tkzDrawPolygon(A',B',B,C,D,D')
			\tkzDrawSegments(B',C' C',D' C,C')
			\tkzDrawSegments[dashed](A,B A,D A,A')
			\tkzDrawPoints[fill=black,size=4](A,B,D,C,A',B',C',D')
			\tkzLabelPoints[above,font=\footnotesize](A',D')
			\tkzLabelPoints[below,font=\footnotesize](A,B,C)
			\tkzLabelPoints[left,font=\footnotesize](B')
			\tkzLabelPoints[right,font=\footnotesize](C',D)
			\node[below] at (0.9,-2.4) {Hình lập phương, hộp chữ nhật};
		\end{tikzpicture}
	\end{center}	
\end{enumerate}
\subsubsection{CÁC TÍNH CHẤT THỪA NHẬN}
Xét trong không gian, ta thừa nhận các tính chất sau:
\begin{enumerate}[\iconMT]
	\item \indam{Tính chất 1:} Có một và chỉ một đường thẳng đi qua hai điểm phân biệt.
	\item \indam{Tính chất 2:} Có một và chỉ một mặt phẳng đi qua ba điểm không thẳng hàng.
	\item \indam{Tính chất 3:} Tồn tại $4$ điểm không cùng thuộc một mặt phẳng.
	\begin{tcolorbox}[colframe=\maudl,colback=cyan!3!white,boxrule=0.5mm]
		Một mặt phẳng hoàn toàn xác định nếu biết ba điểm không thẳng hàng thuộc mặt phẳng đó. Ta kí hiệu mặt phẳng đi qua ba điểm không thẳng hàng $A$, $B$, $C$ là $(A B C)$. Nếu có nhiều điểm cùng thuộc một mặt phẳng thì ta nói những điểm đó đồng phẳng. Nếu \textit{không} có mặt phẳng nào chứa các điểm đó thì ta nói những điểm đó \textit{không đồng phẳng}.
	\end{tcolorbox}
	\item \indam{Tính chất 4:} Nếu một đường thẳng có hai điểm thuộc một mặt phẳng thì tất cả các điểm của đường thẳng đều thuộc mặt phẳng đó.\\
	\begin{tcolorbox}[colframe=\maudl,colback=cyan!3!white,boxrule=0.5mm]
		Cho đường thẳng $d$ và mặt phẳng $(\alpha)$.
		\begin{itemize}
			\item [\ding{172}] Khi $d$ nằm trong $(\alpha)$, ta kí hiệu $d \subset (\alpha)$ hoặc $(P) \supset d$ .\quad (không được viết  $d \in (\alpha)$ nhé!!!)
			\item [\ding{173}] Khi $d$ không nằm trong $(\alpha)$, ta kí hiệu $d \not\subset (\alpha)$.
			\begin{note}
				Dấu hiệu nhận biết $d \subset (\alpha)$  là trên $d$ có hai điểm phân biệt thuộc $(\alpha)$
			\end{note}
		\end{itemize}
	\end{tcolorbox}
	\item \indam{Tính chất 5:} Nếu hai mặt phẳng phân biệt có điểm chung thì các điểm chung của hai mặt phẳng là một đường thẳng đi qua điểm chung đó.
	\immini{
	\begin{tcolorbox}[colframe=\maudl,colback=cyan!3!white,boxrule=0.5mm]
		Đường thẳng chung $d$ (nếu có) của hai mặt phẳng phân biệt $(P)$ và $(Q)$ được gọi là giao tuyến của hai mặt phẳng đó và kí hiệu là $d=(P) \cap(Q)$.
	\end{tcolorbox}}{
\begin{tikzpicture}[declare function={a=2.5;},font=\footnotesize]
	\path 
	(0,0) coordinate (a)
	(0,-a) coordinate (b)
	(b)+(-20:a/1.5) coordinate (c)
	($(a)+(c)-(b)$) coordinate (d)
	(b)+(220:a/1.5) coordinate (e)
	($(a)+(e)-(b)$) coordinate (f)
	($(a)!.35!(b)$) coordinate (M)
	;
	\draw (a)--(b)--(c)--(d)--cycle
	(b)--(e)--(f)--(a)
	;
	\draw pic[draw, angle radius=5mm]{angle=e--f--a};
	\path (f)+(-28:8pt) node{$\alpha$};
	\draw pic[draw, angle radius=5mm]{angle=a--d--c};
	\path (d)+(225:8pt) node{$\beta$};
	\path ($(a)!.7!(b)$)node[right]{$d$};
	\draw[fill=black] (M)node[right]{$M$} circle (1pt);
\end{tikzpicture}}
\item \indam{Tính chất 6:} Trên mỗi mặt phẳng các kết quả đã biết trong hình học phẳng đều đúng.
\end{enumerate}

\subsubsection{CÁCH XÁC ĐỊNH MỘT MẶT PHẲNG}
Ba cách xác định một mặt phẳng
\begin{tcolorbox}[colframe=\maudl,colback=cyan!3!white,boxrule=0.5mm]
	\begin{itemize}
		\item Một mặt phẳng được xác định nếu biết nó đi qua ba điểm $A,B,C$ không thẳng hàng của mặt phẳng, kí hiệu $\left(ABC\right) $.
		\item Một mặt phẳng được xác định nếu biết nó đi qua một đường thẳng $d$ và một điểm $A$ không thuộc $d,$ kí hiệu $\left(A,d\right) $.
		\item Một mặt phẳng được xác định nếu biết nó đi qua hai đường thẳng $a,b$ cắt nhau, kí hiệu $\left(a,b\right) $.
	\end{itemize}
\end{tcolorbox}
\subsubsection{HÌNH CHÓP VÀ HÌNH TỨ DIỆN}
\begin{enumerate}[\iconMT]
	\item \indam{Hình chóp:}
	\begin{itemize}
		\item [\iconCH] \indamm{Định nghĩa:} Cho đa giác $A_1A_2\ldots A_n$ và cho điểm $S$ nằm ngoài mặt phẳng chứa đa giác đó. Nối $S$ với các đỉnh $A_1,A_2,\ldots ,A_n$ ta được $n$ miền đa giác $SA_1A_2,SA_2A_3,\ldots ,SA_{n-1}A_n $.
		Hình gồm $n$ tam giác đó và đa giác $A_1A_2A_3\ldots A_n$ được gọi là hình chóp $S.A_1A_2A_3\ldots A_n $.
		\immini{\item[\iconCH] \indamm{Các tên gọi:}
			\begin{itemize}
				\item Điểm $S$ gọi là đỉnh của hình chóp.
				\item Đa giác $A_1A_2\ldots A_n$ gọi là mặt đáy của hình chóp.
				\item Các đoạn thẳng $A_{1}A_{2},A_{2}A_{3},\ldots,A_{n-1}A_{n}$ gọi là các cạnh đáy của hình chóp.
				\item Các đoạn thẳng $SA_{1},SA_{2},\ldots,SA_{n}$ gọi là các cạnh bên của hình chóp. 
				\item Các miền tam giác $SA_1A_2,SA_2A_3,\ldots ,SA_{n-1}A_n$ gọi là các mặt bên của hình chóp.	 
		\end{itemize}}{
			\begin{tikzpicture}[scale=1.3, line join=round, line cap=round]
				\tkzDefPoints{0/0/A1,0.3/-1/A2,1/-1.6/A3,1.7/-1.5/A4,2.2/-1/A5,2.5/0.3/A6,1/2/S,-1.5/-2/P,-0.7/0.3/Q,3/-2/R}
				\tkzDrawPolygon(S,A1,A2,A3,A4,A5,A6)
				\tkzDrawSegments(S,A2 S,A3 S,A4 S,A5 P,Q P,R)
				\tkzDrawSegments[dashed](A1,A6)
				\tkzDrawPoints[fill=black](S,A1,A2,A3,A4,A5,A6)
				\tkzLabelPoints[above](S)
				\tkzLabelPoints[above right](P)
				\tkzLabelPoint[left](A1){$A_1$}
				\tkzLabelPoint[left](A2){$A_2$}
				\tkzLabelPoint[below](A3){$A_3$}
				\tkzLabelPoint[below](A4){$A_4$}
				\tkzLabelPoint[right](A5){$A_5$}
				\tkzLabelPoint[right](A6){$A_6$}
				\tkzMarkAngle[size=0.6cm,opacity=.4,draw=black,mksize=2](R,P,Q)
		\end{tikzpicture}}
	\end{itemize}
	\item \indam{Hình tứ diện:}
	\begin{itemize}
		\item[\iconCH] \indamm{Định nghĩa:} 
		Cho bốn điểm $A, B, C, D$ không đồng phẳng. Hình gồm bốn tam giác $ABC$, $ACD$, $ABD$, $BCD$ được gọi là hình tứ diện và được kí hiệu là $ABCD$.
		\item[\iconCH] \indamm{Chú ý:} 
	\immini{	\begin{itemize}
			\item Hai cạnh không có đỉnh chung gọi là hai cạnh đối diện, đỉnh không nằm trên một mặt được gọi là đỉnh đối diện với mặt đó.
			\item Hình chóp tam giác còn được gọi là hình tứ diện. 
			\item Hình tứ diện có bốn mặt là những tam giác đều hay có tất cả các cạnh bằng nhau được gọi là hình tứ diện đều. 
		\end{itemize}
	}{
	\begin{tikzpicture}[scale=0.6, line join=round, line cap=round]
		\tkzDefPoints{0/0/B,1.3/-1.6/C,4.5/0/D,1/3.5/A}
		\tkzDrawPolygon(A,B,C,D)
		\tkzDrawSegments(A,C)
		\tkzDrawSegments[dashed](B,D)
		\tkzDrawPoints[fill=black,size=4](A,B,C,D)
		\tkzLabelPoints[above,font=\footnotesize](A)
		\tkzLabelPoints[below,font=\footnotesize](C)
		\tkzLabelPoints[left,font=\footnotesize](B)
		\tkzLabelPoints[right,font=\footnotesize](D)
		\node[below right] at (0,-2.4) {Hình tứ diện};
\end{tikzpicture}}
	\end{itemize}
\end{enumerate}

% \subsection{PHÂN LOẠI, PHƯƠNG PHÁP GIẢI TOÁN}
\begin{dang}{Các quan hệ cơ bản}
	\begin{itemize}
		\item [\ding{172}] Chứng minh điểm $A$ thuộc $(\alpha)$: Ta chứng tỏ điểm $A$ thuộc đường thẳng $\Delta$ nằm trong $\alpha$, nghĩa là
		$$A \in \Delta, \Delta \subset (\alpha) \Rightarrow A \in (\alpha).$$
		\item [\ding{173}] Chứng minh đường thẳng $d$ nằm trong $(\alpha)$: Ta chứng tỏ $d$ có hai điểm phân biệt cùng thuộc $(\alpha)$, nghĩa là
		$$\heva{&A \in (\alpha), B \in (\alpha) \\& A,\,B \in d} \Rightarrow d \subset (\alpha).$$
		\item [\ding{174}] Chứng minh $A$ là điểm chung của hai mặt phẳng $(\alpha)$ và $(\beta)$:Ta thường sử dụng một trong hai cách sau
		$$\heva{&A \in (\alpha) \\& A \in (\beta)} \Rightarrow A \in (\alpha) \cap  (\beta)
		\text{ hoặc } 
		\heva{&d \subset (\alpha) \\& \Delta \subset (\beta)\\& d \cap \Delta = A} \Rightarrow A \in (\alpha) \cap  (\beta).$$
	\end{itemize}
\end{dang}
\begin{vd}
	Cho tam giác $ABC$ và điểm $S$ không thuộc mặt phẳng $\left(ABC\right)$. Lấy $D, E$ là các điểm lần lượt thuộc các cạnh $SA, SB \quad (D, E \text{ khác } S)$.
	\begin{itemize}
		\item [a)] Đường thẳng $DE$ có nằm trong mặt phẳng $\left(SAB\right)$ không?
		\item [b)] Giả sử $DE$ cắt $AB$ tại $F$. Chứng minh rằng $F$ là điểm chung của hai mặt phẳng $\left(SAB\right)$ và $\left(CDE\right)$.
	\end{itemize}
\end{vd}
\begin{vd}
	Cho hình chóp $S.ABCD$, gọi $O$ là giao điểm của $AC$ và $BD$. Lấy $M, N$ lần lượt thuộc các cạnh $SA, SC$.
	\begin{itemize}
		\item[a)] Chứng minh rằng đường thẳng $MN$ nằm trong mặt phẳng $\left(SAC\right)$.
		\item[b)] Chứng minh rằng $O$ là điểm chung của hai mặt phẳng $\left(SAC\right)$ và $\left(SBD\right)$.
	\end{itemize}
\end{vd}
\begin{vd}
	Cho hình tứ diện $ABCD$. Gọi $I$ là trung điểm cạnh $CD$. Gọi $M, N$ lần lượt là trọng tâm của các tam giác $BCD, CDA$.
	\begin{itemize}
		\item[a)] Chứng minh rằng các điểm $M, N$ thuộc mặt phẳng $\left(ABI\right)$.
		\item[b)] Gọi $G$ là giao điểm của $AM$ và $BN$. Chứng minh rằng $\dfrac{GM}{GA}=\dfrac{GN}{GB}=\dfrac{1}{3}$.
		\end{itemize}
\end{vd} 
\begin{dang}{Xác định giao tuyến của hai mặt phẳng}
	Cho hai mặt phẳng $(\alpha)$ và $(\beta)$ cắt nhau. Để xác định giao tuyến của chúng, ta đi tìm hai điểm chung phân biệt. Cụ thể, ta thường gặp một trong ba trường hợp sau:
\begin{itemize}
		\item [\ding{172}] Hai mặt phẳng $(\alpha)$ và $(\beta)$ có sẵn hai điểm chung phân biệt: Khi đó giao tuyến là đường thẳng qua hai điểm chung đó.
		\item [\ding{173}] Hai mặt phẳng  $(\alpha)$ và $(\beta)$ thấy trước một điểm chung $A$:
			\begin{itemize}
				\item [$\bullet$] $A$ là điểm chung thứ nhất hay $A \in (\alpha) \cap (\beta)$.
				\item [$\bullet$] Ta tìm điểm chung thứ 2: Trong $ (\alpha)$ tìm một đường thẳng $d_1$, trong $ (\beta)$ tìm một đường thẳng $d_2$ sao cho chúng có thể cắt nhau (đồng phẳng).			Gọi $B = d_1 \cap d_2$, suy ra $B \in (\alpha) \cap (\beta)$. Vậy $AB=(\alpha) \cap (\beta)$.
		\end{itemize}
		\item [\ding{174}] Hai mặt phẳng  $(\alpha)$ và $(\beta)$ chưa thấy điểm chung: Ta mở rộng mặt phẳng để tìm điểm chung tương tự như cách tìm điểm chung ở mục số \ding{173}.
	\end{itemize}
\end{dang}

\begin{vd}
	\immini{Cho tứ giác $ABCD$ sao cho các cạnh đối không song song với nhau. Lấy một điểm $S$ không thuộc mặt phẳng $\left(ABCD\right)$. Xác định giao tuyến của 
		\begin{itemize}
			\item [a)] Mặt phẳng $\left(SAC\right)$ và mặt phẳng $\left(SBD\right)$. 
			\item [b)] Mặt phẳng $\left(SAB\right)$ và mặt phẳng $\left(SCD\right)$.
			\item [c)] Mặt phẳng $\left(SAD\right)$ và mặt phẳng $\left(SBC\right)$.  
		\end{itemize}
	}{
		\begin{tikzpicture}[scale=0.6, font=\footnotesize, line join=round, line cap=round, >=stealth]
			\tkzDefPoints{0/0/A, 1/-1.5/D, 6/0/B, 4/-2.5/C, 2/3.5/S}
			\tkzLabelPoints[above](S)
			\tkzLabelPoints[above left](A)
			\tkzLabelPoints[below left](D)
			\tkzLabelPoints[right](B,C)
			\tkzDrawPoints[fill=black](A,B,C,D)
			\tkzDrawSegments[dashed](B,A)
			\tkzDrawSegments(S,A S,B S,C S,D A,D D,C C,B)
	\end{tikzpicture}}
	\loigiai{
		\begin{itemize}
			\immini{	
				\item[a)] Gọi $H$ là giao điểm của $AC$ với $BD$. 
				Khi đó 
				$$\heva{&H\in AC\\&H\in BD}\Rightarrow H\in \left(SAC\right)\cap\left(SBD\right)\quad (1).$$
				Dễ thấy 
				$S\in \left(SAC\right)\cap\left(SBD\right)\quad (2).$\\
				Từ $(1)$ và $(2)$ suy ra $SH=\left(SBD\right)\cap\left(SAC\right)$.
				\item[b)] Gọi $K$ là giao điểm của hai đường thẳng $CD$ và $AB$.\\
				Khi đó 
				$\heva{&K\in AB\\&K\in CD}\Rightarrow K\in \left(SAB\right)\cap\left(SCD\right)\quad (3)$.\\
				Dễ thấy 
				$S\in \left(SAB\right)\cap\left(SCD\right)\quad (4)$.\\
				Từ $(3)$ và $(4)$ suy ra $SK=\left(SAB\right)\cap\left(SCD\right)$.}{
				\begin{tikzpicture}[scale=0.6, font=\scriptsize, line join=round, line cap=round, >=stealth]
					\tkzDefPoints{0/0/A, 1/-1.5/D, 6/0/B, 4/-2.5/C, 2/4/S}
					\tkzInterLL(A,C)(B,D)\tkzGetPoint{H}
					\tkzInterLL(A,B)(C,D)\tkzGetPoint{K}
					\tkzInterLL(A,D)(B,C)\tkzGetPoint{L}
					\tkzLabelPoints[above](S)
					\tkzLabelPoints[above left](A)
					\tkzLabelPoints[below left](D)
					\tkzLabelPoints[below](H,K,L)
					\tkzLabelPoints[right](B,C)
					\tkzDrawPoints[fill=black](A,B,C,D,S,H,K,L)
					\tkzDrawSegments[dashed](B,K A,C B,D S,H S,A A,D C,D)
					\tkzDrawSegments(S,C S,B S,D D,K S,K B,C D,L C,L S,L)
			\end{tikzpicture}}
			\item[c)] Gọi $L$ là giao điểm của hai đường thẳng $AD$ và $BC$.\\
			Khi đó $\heva{&L\in AD\\&K\in BC}\Rightarrow L\in \left(SAD\right)\cap\left(SBC\right)\quad (5)$.
			Mặt khác $S\in \left(SAD\right)\cap\left(SBC\right)\quad (6)$.\\
			Từ $(5)$ và $(6)$ suy ra $SL=\left(SAD\right)\cap\left(SBC\right)$.
		\end{itemize}
	}
\end{vd}

\begin{vd}
	\immini{Cho tứ diện $ABCD$. Lấy các điểm $M$ thuộc cạnh $AB$, $N$  thuộc cạnh $AC$ sao cho $MN$ cắt $BC$. Gọi $I$ là điểm bên trong tam giác $BCD$. Tìm giao tuyến của mặt phẳng $\left(MNI\right)$ với các mặt phẳng $\left(ABC\right)$, $\left(BCD\right)$, $\left(ABD\right)$, $\left(ACD\right)$.
	}{
	\begin{tikzpicture}[scale=0.7, font=\footnotesize,>=stealth]
		\path
		%	Vẽ mp
		(0,0) coordinate (B)
		(5,0) coordinate (C)
		(1.5,-1.5) coordinate (D)
		(1,4) coordinate (A)
		(2.3,-0.7) coordinate (I)
		($(A)!0.6!(B)$)coordinate (M)
		($(C)!0.7!(A)$)coordinate (N)
		;
		\draw (B)--(A)--(D)--(C)--(A) (B)--(D);
		\draw[dashed] (M)--(N)--(I)--cycle (B)--(C);
		\foreach \x/\g in {B/180,A/90,C/0,D/-90,M/180,I/30,N/30}\draw[fill=black] (\x) circle (.05) +(\g:.5)node{\footnotesize$\x$};
\end{tikzpicture}}
	\loigiai{
		\begin{itemize}
			\immini{
				\item [a)] Dễ thấy $(MNI) \cap (ABC)=MN$.
				\item [b)] Tìm $(MNI) \cap (BCD)$.
				\begin{itemize}
					\item [$\bullet$] Gọi $H$ là giao điểm của $MN$ và $BC$. Suy ra 
					$$H\in \left(MNI\right)\cap\left(BCD\right)\quad (1).$$
					\item [$\bullet$]Do $I$ là điểm trong $\triangle BCD$ nên 
					$$I\in \left(MNI\right)\cap\left(BCD\right)\quad (2).$$
				\end{itemize}
				Từ $(1)$ và $(2)$ suy ra $IH=\left(MNI\right)\cap\left(BCD\right)$.}{
				\begin{tikzpicture}[scale=0.8, font=\footnotesize, line join=round, line cap=round, >=stealth]
					\tkzDefPoints{0/0/B, 2/-2/C, 6/0/D, 2/4/A, 3/-1/I}
					\tkzDefBarycentricPoint(A=1,B=2)
					\tkzGetPoint{M}
					\tkzDefBarycentricPoint(A=3,C=2)
					\tkzGetPoint{N}
					\tkzInterLL(M,N)(C,B)\tkzGetPoint{H}
					\tkzLabelPoints[left](H)
					\tkzLabelPoints[below left](B,I)
					\tkzLabelPoints[above left](M)
					\tkzLabelPoints[above](A)
					\tkzLabelPoints[below](C)
					\tkzLabelPoints[right](D,N)
					\tkzDrawPoints[fill=black](A,B,C,D,M,N,H,I)
					\tkzDrawSegments[dashed](B,D H,I N,I M,I)
					\tkzDrawSegments(A,C A,B A,D C,D C,H N,H)
			\end{tikzpicture}}
			\immini{\item [c)] Tìm $(MNI) \cap (ABD)$.
				\begin{itemize}
					\item [$\bullet$] Gọi $E=IH\cap BD$. Ta có
					$$\heva{&E\in BD\\&E\in IH}\Rightarrow E\in \left(MNI\right)\cap\left(ABD\right)\quad (3).$$
					\item [$\bullet$] Dễ thấy $M\in \left(ABD\right)\cap\left(MNI\right)\quad (4)$.
				\end{itemize}.\\
				Từ $(3)$ và $(4)$ suy ra $ME=\left(ABD\right)\cap\left(MNI\right)$.
			}{
				\begin{tikzpicture}[scale=0.8, font=\footnotesize, line join=round, line cap=round, >=stealth]
					\tkzDefPoints{0/0/B, 2/-2/C, 6/0/D, 2/4/A, 3/-1/I}
					\tkzDefBarycentricPoint(A=1,B=2)
					\tkzGetPoint{M}
					\tkzDefBarycentricPoint(A=3,C=2)
					\tkzGetPoint{N}
					\tkzInterLL(M,N)(C,B)\tkzGetPoint{H}
					\tkzInterLL(I,H)(B,D)\tkzGetPoint{E}
					\tkzInterLL(I,H)(C,D)\tkzGetPoint{F}
					\tkzLabelPoints[left](H)
					\tkzLabelPoints[below left](B,I)
					\tkzLabelPoints[above left](M)
					\tkzLabelPoints[above](A)
					\tkzLabelPoints[below](C,E)
					\tkzLabelPoints[right](D,N)
					\tkzLabelPoints[below right](F)
					\tkzDrawPoints[fill=black](A,B,C,D,M,N,H,I,E,F)
					\tkzDrawSegments[dashed](B,D H,F M,E M,I N,I)
					\tkzDrawSegments(A,C A,B A,D C,D C,H N,H N,F)
			\end{tikzpicture}}
			\item [d)] Tìm $(MNI) \cap (BCD)$.
			\begin{itemize}
				\item [$\bullet$] Gọi $F=IH\cap CD$. Ta có
				$$\heva{&F\in CD\\&F\in IH}\Rightarrow F\in \left(MNI\right)\cap\left(ACD\right)\quad (5).$$
				\item [$\bullet$] Mặt khác: $N\in AC$ nên $N\in \left(ACD\right)$. Suy ra
				$N\in \left(MNI\right)\cap\left(ACD\right)\quad (6)$.\\
				Từ $(5)$ và $(6)$ suy ra $NF=\left(ACD\right)\cap\left(MNI\right)$.
			\end{itemize}
		\end{itemize}
	}
\end{vd}

\begin{vd}
	Cho tứ diện $ABCD$. Gọi $I$, $J$ lần lượt là trung điểm các cạnh $AD$, $BC$.  
	\begin{itemize}
		\item [a)] Tìm giao tuyến của hai mặt phẳng $\left(IBC\right)$ và mặt phẳng $\left(JAD\right)$. 
		\item [b)] Lấy điểm $M$ thuộc cạnh $AB$, $N$ thuộc cạnh $AC$ sao cho $M$, $N$ không là trung điểm. Tìm giao tuyến của  mặt phẳng $\left(IBC\right)$ và mặt phẳng $\left(DMN\right)$. 
	\end{itemize}
	\loigiai{
		\immini{
			\begin{itemize}
				\item [a)] Do giả thiết $I\in AD$ nên $I\in \left(JAD\right)$. 
				Suy ra 
				$$I\in \left(BCI\right)\cap\left(ADJ\right)\quad (1).$$
				Tương tự, ta có $J\in \left(BCI\right)\cap\left(ADJ\right)\quad (2)$.\\
				Từ $(1)$ và $(2)$ suy ra $IJ=\left(BCI\right)\cap \left(ADJ\right)$.
				\item [b)] Gọi $E=DM\cap BI$. Khi đó 
				$$\heva{&E\in BI\\&E\in DM}\Rightarrow E\in \left(MND\right)\cap\left(IBC\right)\quad (3).$$
				Tương tự, gọi $F=DN\cap CI$ suy ra
				$$F\in \left(BCI\right)\cap\left(MND\right)\quad (4).$$
				Từ $(3)$ và $(4)$ suy ra $EF=\left(BCI\right)\cap \left(MND\right)$.
			\end{itemize}
		}{\begin{tikzpicture}[scale=0.8, font=\footnotesize, line join=round, line cap=round, >=stealth]
				\tkzDefPoints{0/0/B, 2/-2/C, 6/0/D, 2/4/A}
				\tkzDefMidPoint(A,D)\tkzGetPoint{I}
				\tkzDefMidPoint(B,C)\tkzGetPoint{J}
				\tkzDefBarycentricPoint(A=1,B=3)
				\tkzGetPoint{M}
				\tkzDefBarycentricPoint(A=3,C=2)
				\tkzGetPoint{N}
				\tkzInterLL(D,N)(C,I)\tkzGetPoint{F}
				\tkzInterLL(D,M)(B,I)\tkzGetPoint{E}
				\tkzLabelPoints[below left](B,J)
				\tkzLabelPoints[above left](M)
				\tkzLabelPoints[above](A)
				\tkzLabelPoints[below](C,E)
				\tkzLabelPoints[right](D,N)
				\tkzLabelPoints[above right](I,F)
				\tkzDrawPoints[fill=black](A,B,C,D,I,J,M,N,E,F)
				\tkzDrawSegments[dashed](B,D B,I D,J M,D I,J E,F)
				\tkzDrawSegments(A,C A,B A,D C,D B,C C,I A,J M,N N,D)
			\end{tikzpicture}
		}
	}
\end{vd}

\begin{vd}
	Cho tứ diện $ABCD$, $M$ là một điểm bên trong tam giác $ABD$, $N$ là một điểm bên trong tam giác $ACD$. Tìm giao tuyến của các cặp mặt phẳng sau
	\begin{enumEX}[]{2}
		\item $(AMN)$ và $(BCD)$.
		\item $(DMN)$ và $(ABC)$.
	\end{enumEX}
	\loigiai{
		\immini
		{
			\begin{enumEX}[]{1}
				\item Tìm $(AMN)\cap (BCD)$.\\
				Trong $(ABD)$, gọi $E=AM\cap BD$.\\
				Ta có $\heva{& E\in AM\subset (AMN) \\ & E\in BD\subset (BCD)}\Rightarrow E\in (AMN)\cap (BCD)$ $(1)$.\\
				Trong $(ACD)$, gọi $F=AN\cap CD$.\\
				Ta có $\heva{& F\in AN\subset (AMN) \\ & F\in CD\subset (BCD)}\Rightarrow F\in (AMN)\cap (BCD)$ $(2)$.\\
				Từ $(1)$ và $(2)$ suy ra $(AMN)\cap (BCD)=EF$.
				\item Tìm $(DMN)\cap (ABC)$.\\
				Trong $(ABD)$, gọi $P=DM\cap AB$.\\
				Ta có $\heva{& P\in DM\subset (DMN) \\ & P\in AB\subset (ABC)}\Rightarrow P\in (DMN)\cap (ABC)$ $(3)$.\\
				Trong $(ACD)$, gọi $Q=DN\cap AC$.\\
				Ta có $\heva{& Q\in DN\subset (DMN) \\ & Q\in AC\subset (ABC)}\Rightarrow Q\in (DMN)\cap (ABC)$ $(4)$.\\
				Từ $(3)$ và $(4)$ suy ra $(DMN)\cap (ABC)=PQ$.
			\end{enumEX}
		}
		{
			\begin{tikzpicture}
				[scale=1, font=\footnotesize, line join=round, line cap=round, >=stealth]
				\tkzDefPoints{0/0/B,3/-2/C,5/0/D,2/4.5/A}
				\coordinate (E) at ($(B)!1/3!(D)$);
				\coordinate (F) at ($(C)!2/3!(D)$);
				\coordinate (P) at ($(A)!1.3/3!(B)$);
				\coordinate (Q) at ($(A)!3/5!(C)$);
				\tkzInterLL(A,E)(P,D)\tkzGetPoint{M}
				\tkzInterLL(A,F)(Q,D)\tkzGetPoint{N}
				\tkzDrawPolygon(A,B,C,D)
				\tkzDrawSegments(A,C A,F P,Q Q,D)
				\tkzDrawSegments[dashed](B,D A,E E,F P,D)
				\tkzDrawPoints[fill=black](A,B,C,D,Q,P,M,N,E,F)
				\tkzLabelPoints[above](A)
				\tkzLabelPoints[below](C,E)
				\tkzLabelPoints[left](B,P,Q)
				\tkzLabelPoints[right](D,F)
				\tkzLabelPoints[right](M)
				\tkzLabelPoints[right](N)
			\end{tikzpicture}
		}
	}
\end{vd}

\begin{vd}
	\immini{Cho hình chóp $S.ABCD$ đáy là hình bình hành tâm $O$. Gọi $M$, $N$, $P$ lần lượt là trung điểm của cạnh $BC$, $CD$, $SA$. Tìm giao tuyến của 
		\begin{tasks}(1)
			\task $(MNP)$ và $(SAB)$.
			\task $(MNP)$ và $(SBC)$.
			\task $(MNP)$ và $(SAD)$.
			\task $(MNP)$ và $(SCD)$.
	\end{tasks}}{
		\begin{tikzpicture}
			[scale=1, font=\footnotesize, line join=round, line cap=round, >=stealth]
			\tkzDefPoints{0/0/A,-1.3/-1.6/B,2.5/-1.6/C}
			\coordinate (D) at ($(A)+(C)-(B)$);
			\coordinate (S) at ($(A)+(0.5,2.5)$);
			\coordinate (P) at ($(S)!0.5!(A)$);
			\coordinate (M) at ($(B)!0.5!(C)$);
			\coordinate (N) at ($(C)!0.5!(D)$);
			\tkzDrawPoints[fill=black](D,C,A,B,S,M,N,P)
			\tkzDrawPolygon(S,B,C,D)
			\tkzDrawPolygon[dashed](M,N,P)
			\tkzDrawSegments[dashed](A,B A,D A,S)
			\tkzDrawSegments(S,C)
			\tkzLabelPoints[above](S)
			\tkzLabelPoints[below](A,B,C,D,M,N)
			\tkzLabelPoints[above right](P)
	\end{tikzpicture}}
	\loigiai{
		\begin{itemize}
			\immini{\item[a)] Tìm $(MNP)\cap (SAB)$.
				\begin{itemize}
					\item [$\bullet$] Ta có $P\in (MNP)\cap (SAB)$ $(1)$.
					\item [$\bullet$] Gọi $F=MN \cap AB $ thì $\heva{& F\in MN\subset (MNP) \\ & F\in AB\subset (SAB)}$\\
					nên $F\in (MNP)\cap (SAB) \quad (2)$.
				\end{itemize}
				Từ $(1)$ và $(2)$ suy ra $(MNP)\cap (SAB)=PF$.
				\item[b)] Tìm $(MNP)\cap (SBC)$.
				\begin{itemize}
					\item [$\bullet$] Ta có $M\in (MNP)\cap (SBC)$ $(3)$.
					\item [$\bullet$] Gọi $K=PF \cap SB $ thì $\heva{& K\in PF\subset (MNP) \\ & K\in SB\subset (SBC)}$\\
					nên $K\in (MNP)\cap (SBC) \quad (4)$.
				\end{itemize}
				Từ $(3)$ và $(4)$ suy ra $(MNP)\cap (SBC)=MK$.}{
				\begin{tikzpicture}
					[scale=1, font=\footnotesize, line join=round, line cap=round, >=stealth]
					\tkzDefPoints{0/0/A,-1.3/-1.6/B,2.5/-1.6/C}
					\coordinate (D) at ($(A)+(C)-(B)$);
					\coordinate (S) at ($(A)+(0.5,2.5)$);
					\coordinate (P) at ($(S)!0.5!(A)$);
					\coordinate (M) at ($(B)!0.5!(C)$);
					\coordinate (N) at ($(C)!0.5!(D)$);
					\tkzInterLL(A,B)(M,N)\tkzGetPoint{F}
					\tkzInterLL(S,B)(F,P)\tkzGetPoint{K}
					\tkzDrawPoints[fill=black](D,C,A,B,S,M,N,P,F,K)
					\tkzDrawPolygon[dashed](M,N,P)
					\tkzDrawSegments[dashed](A,B A,D A,S K,P F,B B,K B,M)
					\tkzDrawSegments(S,C M,F F,K M,K S,K S,D C,M C,D)
					\tkzLabelPoints[above](S)
					\tkzLabelPoints[left](K)
					\tkzLabelPoints[below](A,B,C,D,M,N,F)
					\tkzLabelPoints[above right](P)
			\end{tikzpicture}	}
			\immini{\item[c)] Tìm $(MNP)\cap (SAD)$.
				\begin{itemize}
					\item [$\bullet$] Ta có $P\in (MNP)\cap (SAD)$ $(5)$.
					\item [$\bullet$] Gọi $E=MN \cap AD $, suy ra 
					$$ E\in (MNP)\cap (SAD) \quad (6).$$
				\end{itemize}
				Từ $(5)$ và $(6)$ suy ra $(MNP)\cap (SAD)=EP$.
				\item[d)] Tìm $(MNP)\cap (SCD)$.
				\begin{itemize}
					\item [$\bullet$] Ta có $N\in (MNP)\cap (SCD)$ $(7)$.
					\item [$\bullet$] Gọi $H=PE \cap SD$, suy ra
					$$H\in (MNP)\cap (SCD) \quad (8).$$
				\end{itemize}
				Từ $(7)$ và $(8)$ suy ra $(MNP)\cap (SCD)=HN$.
			}{
				\begin{tikzpicture}
					[scale=1, font=\footnotesize, line join=round, line cap=round, >=stealth]
					\tkzDefPoints{0/0/A,-1.3/-1.6/B,2.5/-1.6/C}
					\coordinate (D) at ($(A)+(C)-(B)$);
					\coordinate (S) at ($(A)+(0.5,2.5)$);
					\coordinate (P) at ($(S)!0.5!(A)$);
					\coordinate (M) at ($(B)!0.5!(C)$);
					\coordinate (N) at ($(C)!0.5!(D)$);
					\tkzInterLL(A,D)(M,N)\tkzGetPoint{E}
					\tkzInterLL(S,D)(E,P)\tkzGetPoint{H}
					\tkzDrawPoints[fill=black](D,C,A,B,S,M,N,P,E,H)
					\tkzDrawPolygon[dashed](M,N,P)
					\tkzDrawSegments[dashed](A,B A,D A,S H,P E,D D,H D,N)
					\tkzDrawSegments(S,C N,E E,H N,H S,B S,C S,H C,M B,C C,N)
					\tkzLabelPoints[above](S)
					\tkzLabelPoints[above right](H)
					\tkzLabelPoints[below](A,B,C,D,M,N,E)
					\tkzLabelPoints[above right](P)
			\end{tikzpicture}	}
		\end{itemize}
		
	}
\end{vd}

\begin{vd}
	Cho hình chóp $S.ABCD$ có đáy $ABCD$ là hình bình hành. Gọi $M,P$ lần lượt là trung điểm của $SA, BC$. $N$ là điểm trên cạnh $SB$ sao cho $BN=\dfrac{1}{4}BS$. Xác định giao tuyến của mặt phẳng $(MNP)$ với các mặt phẳng
	\begin{tasks}(3)
		\task $(ABCD)$.
		\task $(SAD)$.
		\task $(SCD)$.
	\end{tasks}
	\loigiai{
		\immini{a) Gọi $I$ là giao điểm của $MN$ và $AB$, khi đó ta có $\heva{&I\in MN\\&I\in AB}\Rightarrow \heva{&I\in (MNP)\\&I\in (ABCD)}.$\hfill (1)\\
			Hiển nhiên $\heva{&P\in (MNP)\\&P\in (ABCD)}.$\hfill (2)\\
			Từ (1) và (2) suy ra $PI$ là giao tuyến của các mặt phẳng $(MNP)$ và $(ABCD)$.\\
			b) Gọi $K$ là giao điểm của $IP$ với $AD$, khi đó $\heva{&K\in IP\\&K\in AD}\Rightarrow \heva{&K\in (MNP)\\&K\in (SAD)}.$\hfill (3)\\
			Hiển nhiên $\heva{&M\in (MNP)\\&M\in (ABCD)}.$\hfill (4)\\
			Từ (3) và (4) suy ra $MK$ là giao tuyến của các mặt phẳng $(MNP)$ và $(ABCD)$.\\
			c) Gọi $Q$ là giao điểm của $IP$ và $CD$, $R$ là giao điểm của $MK$ và $SD$. Khi đó ta chứng minh được $QR$ là giao tuyến của các mặt phẳng $(MNP)$ và $(SCD)$.
		}
		{
			\begin{tikzpicture}[scale=1]
				\tkzDefPoints{0/0/A, -2/-2/B, 4/0/D}
				\coordinate (C) at ($(B)+(D)-(A)$);
				\coordinate (S) at ($(A)+(0,5)$);
				\coordinate (M) at ($(A)!0.5!(S)$);
				\coordinate (P) at ($(B)!0.5!(C)$);
				\coordinate (Q) at ($(C)!0.5!(D)$);
				\tkzInterLL(P,Q)(A,B) \tkzGetPoint{I}
				\tkzInterLL(P,Q)(A,D) \tkzGetPoint{K}
				\tkzInterLL(M,I)(S,B) \tkzGetPoint{N}
				\tkzInterLL(M,K)(S,D) \tkzGetPoint{R}
				\tkzDrawSegments(S,N N,I I,P P,N P,C C,Q C,S Q,R Q,K K,R R,S)
				\tkzDrawSegments[dashed](S,A A,I B,P B,N A,K Q,P R,D M,N M,R Q,D)
				\tkzLabelPoints[above](S,R)
				\tkzLabelPoints[below](B,I,P,Q,C,K,D,A)
				\tkzLabelPoints[left](N)
				\tkzLabelPoints[above right](M)
			\end{tikzpicture}
		}
	}
\end{vd}

\begin{dang}{Tìm giao điểm của đường thẳng và mặt phẳng}
	\begin{note}
		Muốn tìm giao điểm của đường thẳng $d$ và mặt phẳng $(P)$ (phân biệt, không song song), ta tìm giao điểm của $d$ với một đường thẳng $a$ nằm trong $(P)$. Xét hai khả năng: 
	\end{note}
	\immini{
		\begin{itemize}
			\item [\ding{172}] Nếu đường thẳng $a$ dễ tìm, nghĩa là có sẵn $a \subset (P)$ và $a$ cắt được $d$. Khi đó
			\begin{itemize}
				\item [$\bullet$] Gọi $M=d \cap a$ thì $\heva{& M \in d\\& M \in a \subset (P)}$.
				\item [$\bullet$] Vậy $M = d \cap (P)$.
			\end{itemize}
			\item [\ding{173}] Nếu đường thẳng $a$ khó tìm, ta thực hiện các bước sau:
			\begin{itemize}
				\item [$\bullet$] Tìm một mặt phẳng $(Q)$ chứa đường thẳng $d$ và dễ tìm giao tuyến với $(P)$;
				\item [$\bullet$] Tìm $(Q) \cap (P)= a$.
				\item [$\bullet$] Tìm $M=d \cap a$, suy ra $M = d \cap (P)$.
			\end{itemize}
		\end{itemize}
	}{\vspace{1cm}
		\begin{tikzpicture}[line cap=round, line join=round,font=\footnotesize,>=stealth, scale=0.9]
			\tikzset{label style/.style={font=\footnotesize}}
			\tkzDefPoints{0/0/a, 4.5/0/b, 4/2/c, 4.2/2.3/b'}
			\tkzDefBarycentricPoint(a=1,c=1,b=-1)\tkzGetPoint{d}
			\tkzDefBarycentricPoint(a=0.75,b=0.25)\tkzGetPoint{a'}
			\tkzDefBarycentricPoint(d=0.75,c=0.25)\tkzGetPoint{d'}
			\tkzDefBarycentricPoint(d'=1,b'=1,a'=-1)\tkzGetPoint{c'}
			\tkzInterLL(a',b')(c,d)\tkzGetPoint{x}
			\tkzDefMidPoint(a',d')\tkzGetPoint{M}
			\tkzDefLine[parallel=through M](a',b')\tkzGetPoint{y}
			\tkzDrawPolygon(a',b',c',d')
			\tkzDrawSegments(a,b b,c d,a c,x d,d')
			\tkzDrawSegments[dashed](x,d')
			\tkzDrawSegments[add=0 and -0.3](M,y)
			\tkzLabelSegments[above, pos=0.7](M,y){$d$}
			\tkzDrawSegments[add=0.2 and -1, dashed](M,y)
			\tkzMarkAngle[size=0.5](c',b',a')
			\tkzLabelAngle[pos=-0.25](a',b',c'){$Q$}
			\tkzMarkAngle[size=0.5](b,a,d)
			\tkzLabelAngle[pos=0.25](b,a,d){$P$}
			\tkzLabelSegments[pos=0.8, right](a',d'){$a$}
			
			\tkzDrawPoints[fill=black](M)
			\tkzLabelPoints[below right](M)
		\end{tikzpicture}
	}
\end{dang}

\begin{vd}
	Cho tứ diện $ABCD$. Gọi $M,N$ lần lượt là trung điểm của $AC$ và $BC$. $K$ là điểm nằm trên $BD$ sao cho $KD<KB$. 
	\begin{tasks}
		\task Tìm giao điểm của $CD$ với mặt phẳng $(MNK)$.
		\task Tìm giao điểm của $AD$ với mặt phẳng $(MNK)$.
	\end{tasks}
	\loigiai{
		\begin{enumerate}[\faPencilSquareO]
			\immini{	\item Tìm giao điểm của $CD$ với mp$(MNK)$.
				\begin{note}
					Dễ thấy trong mặt phẳng $(MNK)$ có đường thẳng $NK$ có thể cắt được đường $CD$. Nên ta giải như sau:
				\end{note}
				\begin{itemize}
					\item [$\bullet$] Do $KD<KC$ nên $K$ không là trung điểm của $BD$, suy ra $NK$ cắt $CD$; 
					\item [$\bullet$] Gọi $I=CD \cap NK$, ta có
					$$\heva{&I \in CD\\& I \in NK, NK \subset (MNK)}\Rightarrow I=CD \cap (MNK).$$
				\end{itemize}
			}{
				\begin{tikzpicture}[line cap=round, line join=round,font=\footnotesize,>=stealth, scale=1.1]
					\tikzset{label style/.style={font=\footnotesize}}
					\tkzDefPoints{0/0/B, 4/0/D, 3/-1.5/C, 1/2.5/A}
					\tkzDefBarycentricPoint(B=0.2,D=0.8)\tkzGetPoint{K}
					\tkzDefMidPoint(A,C)\tkzGetPoint{M}
					\tkzDefMidPoint(B,C)\tkzGetPoint{N}
					\tkzInterLL(N,K)(C,D)\tkzGetPoint{I}
					\tkzInterLL(I,M)(A,D)\tkzGetPoint{H}
					\tkzDrawSegments[dashed](B,D M,K N,I)
					\tkzDrawSegments(A,B B,C C,I I,M A,D A,C M,N)
					
					\tkzDrawPoints[fill=black](A,B,C,D,M,N,K,H,I)
					\tkzLabelPoints[above](A)
					\tkzLabelPoints[left](B,M)
					\tkzLabelPoints[below](C,K)
					\tkzLabelPoints[below right](D)
					\tkzLabelPoints[above right](I,H)
					\tkzLabelPoints[below left](N)
			\end{tikzpicture}}
			\item Tìm giao điểm của $AD$ và $(MNK)$.\\
			Trong mặt phẳng $(ACD)$, gọi $H=AD \cap MI$. Ta có
			$$\heva{&H \in AD\\& H \in MI, MI \subset (MNK)}\Rightarrow H=AD \cap (MNK).$$		
	\end{enumerate}}
\end{vd}

\begin{vd}
	Cho tứ diện $ABCD$. trên cạnh $AC$ và $AD$ lấy hai điểm $M$, $N$ sao cho $AC=3AM$ và $AN=\dfrac{2}{3}AD$. Gọi $O$ là điểm bên trong tam giác $(BCD)$.
	\begin{tasks}(1)
		\task Tìm giao điểm của $BC$ với $(OMN)$.
		\task Tìm giao điểm của $BD$ với $(OMN)$.
	\end{tasks}
	\loigiai{
		
		\begin{enumerate}[a)]
			\immini{
				\item Tìm giao điểm của $BC$ với $(OMN)$.\\
				Xét $BC \subset (BCD)$. Ta đi tìm giao tuyến của $(BCD)$ với $(OMN)$.
				\begin{itemize}
					\item [$\bullet$] Gọi $I= CD \cap MN \Rightarrow I \in (BCD) \cap (OMN) \quad (1)$.\\
					\item [$\bullet$] Mặt khác $O \in (BCD) \cap (OMN) \quad (2)$.
				\end{itemize}
				Từ (1) và (2), suy ra $OI=(BCD) \cap (OMN)$.\\
				Trong $(BCD)$, gọi $P=OI \cap BC$. Ta có 
				$$\heva{&P \in BC\\&P \in OI, OI \subset (OMN)}\Rightarrow P=BC \cap (OMN).$$
			}{\vspace{0.6cm}
				\begin{tikzpicture}[line cap=round, line join=round,font=\footnotesize,>=stealth, scale=1]
					\tikzset{label style/.style={font=\footnotesize}}
					\tkzDefPoints{0/0/B, 4/0/D, 1.6/-1.2/C, 1/3/A}
					\tkzDefBarycentricPoint(A=0.35,D=0.65)\tkzGetPoint{N}
					\tkzDefBarycentricPoint(A=0.6,C=0.4)\tkzGetPoint{M}
					\tkzDefBarycentricPoint(B=0.35,C=0.3,D=0.35)\tkzGetPoint{O}
					\tkzInterLL(M,N)(C,D)\tkzGetPoint{I}
					\tkzInterLL(B,C)(O,I)\tkzGetPoint{P}
					\tkzInterLL(D,B)(O,I)\tkzGetPoint{Q}
					\tkzDrawSegments[dashed](B,D I,O O,M O,N O,P)
					\tkzDrawSegments(A,B B,C C,D A,C M,I A,D I,D P,M)
					\tkzLabelPoints[above](A,Q)
					\tkzLabelPoints[left](B,M)
					\tkzLabelPoints[below](C,O)
					\tkzLabelPoints[below right](D)
					\tkzLabelPoints[above right](N,I)
					\tkzLabelPoints[below left](P)
					\tkzDrawPoints[fill=black](A,B,C,D,M,N,P,Q,I,O)
				\end{tikzpicture}
			}
			\item Tìm giao điểm của $BD$ với $(OMN)$.\\
			Trong $(BCD)$, gọi $Q=OI \cap BD$. Ta có 
			$\heva{&Q \in BD\\&Q \in OI, OI \subset (OMN)}\Rightarrow Q=BD \cap (OMN).$
		\end{enumerate}	
	}
\end{vd}

\begin{vd}
	Cho hình chóp $S.ABCD$ có đáy là hình bình hành. Gọi $M$ là trung điểm của $SC$.
	\begin{itemize}
		\item [a)] Tìm giao điểm $I$ của đường thẳng $AM$ và mặt phẳng $\left(SBD\right)$. Chứng minh $IA=2IM$.
		\item [b)] Tìm giao điểm $E$ của đường thẳng $SD$ và mặt phẳng $\left(ABM\right)$.
		\item [c)] Gọi $N$ là một điểm tuỳ ý trên cạnh $AB$. Tìm giao điểm của đường thẳng $MN$ và mặt phẳng $\left(SBD\right)$.
	\end{itemize}
\end{vd}


\begin{vd}
	Cho tứ giác $ABCD$ và một điểm $S$ không thuộc mặt phẳng $(ABCD)$. Trên đoạn $AB$ lấy một điểm $M$, trên đoạn $SC$ lấy một điểm $N$ ($M,N$ không trùng với các đầu mút).
	\begin{tasks}(1)
		\task Tìm giao điểm của đường thẳng $AN$ với mặt phẳng $(SBD)$.
		\task Tìm giao điểm của đường thẳng $MN$ với mặt phẳng $(SBD)$.
	\end{tasks}
	\loigiai{
		\immini{\begin{enumerate}[a)]
				\item \textbf{Tìm giao điểm của đường thẳng $AN$ với mặt phẳng $(SBD)$.}
				\begin{itemize}
					\item Chọn mặt phẳng phụ $(SAC)\supset AN$. Ta tìm giao tuyến của $(SAC)$ và $(SBD)$.\\
					Trong $(ABCD)$ gọi $P=AC\cap BD$. Suy ra $(SAC)\cap(SBD)=SP$.
					\item Trong $(SAC)$ gọi $I=AN\cap SP$.\\
					$\heva{&I\in AN \\&I\in SP, SP\subset (SBD)}\Rightarrow I=AN\cap (SBD)$.
				\end{itemize}
				\item \textbf{Tìm giao điểm của đường thẳng $MN$ với mặt phẳng $(SBD)$.}
				\begin{itemize}
					\item Chọn mặt phẳng phụ $(SMC)\supset MN$. Ta tìm giao tuyến của $(SMC)$ và $(SBD)$.\\
					Trong $(ABCD)$ gọi $Q=MC\cap BD$. Suy ra $(SMC)\cap(SBD)=SQ$.
					\item Trong $(SMC)$ gọi $J=MN\cap SQ$.\\
					$\heva{&J\in MN \\&J\in SQ, SQ\subset (SBD)}\Rightarrow J=MN\cap (SBD)$.
				\end{itemize}
			\end{enumerate}
		}{
			\begin{tikzpicture}[line cap=round, line join=round,font=\footnotesize,>=stealth, scale=1]
				\tikzset{label style/.style={font=\footnotesize}}
				\tkzDefPoints{0/0/A, 4.5/0/D, 1.2/3/S, 0.9/-1.5/B, 3.5/-1.3/C}
				\tkzDefBarycentricPoint(A=0.3,B=0.7)\tkzGetPoint{M}
				\tkzDefBarycentricPoint(S=0.6,C=0.4)\tkzGetPoint{N}
				\tkzInterLL(A,C)(B,D)\tkzGetPoint{P}
				\tkzInterLL(S,P)(A,N)\tkzGetPoint{I}
				\tkzInterLL(C,M)(B,D)\tkzGetPoint{Q}
				\tkzInterLL(S,Q)(M,N)\tkzGetPoint{J}
				
				\tkzDrawSegments[dashed](S,P S,Q A,D A,C A,N C,M M,N B,D)
				\tkzDrawSegments(S,A S,B S,C S,D S,M A,B B,C C,D)
				
				\tkzLabelPoints[above](S)
				\tkzLabelPoints[left](A,J)
				\tkzLabelPoints[below left](B,M)
				\tkzLabelPoints[below right](C)
				\tkzLabelPoints[right](D)
				\tkzLabelPoints[above right](P,N)
				\tkzLabelPoints[above left](Q)
				\draw (I)+(-0.2,0.15) node{$I$};
				\tkzDrawPoints[fill=black](S,A,B,C,D,M,N,P,Q,I,J)
			\end{tikzpicture}
		} 
	}
\end{vd}

\begin{dang}{Chứng minh ba điểm thẳng hàng}
\begin{note}
	Muốn chứng minh ba điểm $A$, $B$, $C$ thẳng hàng, ta chứng minh ba điểm đó lần lượt thuộc hai mặt phẳng phân biệt $(\alpha)$ và $(\beta)$, nghĩa là chúng cùng nằm trên một đường giao tuyến.
\end{note}
	% \immini{	\begin{itemize}
	% 		\item [$\bullet$]  Ta có $A=a \cap b$, mà $a \subset (\alpha)$, $b \subset (\beta)$ nên
	% 		$$A \in (\alpha) \cap (\beta) \quad (1)$$
	% 		\item [$\bullet$] Tương tự ta cũng tìm xem $B$ và $C$ tương ứng là giao của cặp đường thẳng nào nằm trong $(\alpha)$ và $(\beta)$. Từ đó, suy ra
	% 		$$B \in (\alpha) \cap (\beta) \quad (2)$$
	% 		và 
	% 		$$C \in (\alpha) \cap (\beta) \quad(3)$$
	% 	\end{itemize}
	% }	
	% {\begin{tikzpicture}[scale=0.4, font=\footnotesize,line join=round, line cap=round,>=stealth]
	% 		\tikzset{label style/.style={font=\footnotesize}}
	% 		\tkzDefPoints{0/0/M,7/0/N,10/-2/P,3/-2/Q,6/2/X,3/5/Y, 2.5/3/E, 5.5/-1/F}
	% 		\tkzInterLL(M,N)(Q,X)\tkzGetPoint{T}
	% 		\coordinate (A) at ($(M)!0.15!(Q)$);
	% 		\coordinate (B) at ($(M)!0.45!(Q)$);
	% 		\coordinate (C) at ($(M)!0.84!(Q)$);
	% 		\tkzInterLL(A,F)(X,Q)\tkzGetPoint{I}
	% 		\tkzDrawPoints[size=5,fill=black](A,B,C)
	% 		\tkzLabelPoints[below](A,B,C)
	% 		\tkzLabelSegment(I,F){$b$}
	% 		\tkzLabelSegment(A,E){$a$}
	% 		\tkzLabelAngles[pos=0.6,rotate=30](M,Y,X){$\alpha$}
	% 		\tkzLabelAngles[pos=0.9,rotate=10](N,P,Q){$\beta$}
	% 		\tkzDrawPolygon(M,Q,X,Y)
	% 		\tkzDrawSegments(M,Q Q,P N,P T,N A,E I,F)
	% 		\tkzDrawSegments[dashed](M,T A,I)
	% 		\tkzMarkAngles[size=1cm,arc=l](M,Y,X)
	% 		\tkzMarkAngles[size=1.35cm,arc=l](N,P,Q)
	% 	\end{tikzpicture}\hspace{-2cm}}
	% 	Từ (1), (2), (3) suy ra $A$, $B$, $C$ cùng thuộc đường giao tuyến nên chúng thẳng hàng.
	
\end{dang}
\begin{vd}
	Cho tứ diện $ABCD$ có $G$ là trọng tâm tam giác $BCD$, Gọi $M$, $N$, $P$ lần lượt là trung điểm của $AB$, $BC$, $CD$.
	\begin{tasks}(1)
		\task Tìm giao tuyến của $(AND)$ và $(ABP)$.
		\task Gọi $I=AG\cap MP$, $J=CM\cap AN$. Chứng minh $D$, $I$, $J$ thẳng hàng.
	\end{tasks}
	\loigiai{
		\begin{center}
			\begin{tikzpicture}[scale=1, font=\footnotesize,line join=round, line cap=round,>=stealth]
				\tkzDefPoints{0/0/B,2/-2/C,6/0/D,2/4/A}
				\tkzDefMidPoint(A,B)\tkzGetPoint{M}
				\tkzDefMidPoint(B,C)\tkzGetPoint{N}
				\tkzDefMidPoint(C,D)\tkzGetPoint{P}
				\tkzInterLL(B,P)(D,N)\tkzGetPoint{G}
				\tkzInterLL(A,G)(M,P)\tkzGetPoint{I}
				\tkzInterLL(C,M)(A,N)\tkzGetPoint{J}
				\tkzDrawSegments(A,B B,C C,D D,A A,C A,N C,M A,P)
				\tkzDrawSegments[dashed](B,D B,P D,N M,P D,J A,G D,M)
				\tkzDrawPoints[fill=black](A,B,C,D,M,N,P,I,J,G)
				\tkzLabelPoints[left](M,J)
				\tkzLabelPoints[above right](I)
				\tkzLabelPoints[above](A)
				\tkzLabelPoints[below](D,C,B,N,P,G)
			\end{tikzpicture}
		\end{center}
		\begin{enumerate}[a)]
			\item Tìm giao tuyến của $(AND)$ và $(ABP)$.\\
			$A\in (ABP)\cap (ADN)$.\hfill $(1)$\\
			Ta có $G=BP\cap DN$, có $\heva{&G\in BP,\, BP\subset (ABP)\\&G\in DN,\, DN\subset (ADN)}\Rightarrow G\in (ABP)\cap (ADN)$. \hfill $(2)$\\
			Từ $(1)$ và $(2)$ ta có $AG=(ABP)\cap (ADN)$.
			\item Chứng minh $D$, $I$, $J$ thẳng hàng.\\
			$I=AG\cap MP$, $AG\subset (ADG)$, $MP\subset (DMN)\Rightarrow I\in(ADG)\cap (DMN)$.\hfill $(3)$\\
			$J=CM\cap AN$, $AN\subset (ADG)$, $CM\subset (DMN)\Rightarrow J\in(ADG)\cap (DMN)$.\hfill $(4)$\\
			$D\in (ADG)\cap (DMN)$. \hfill $(5)$\\
			Từ $(3)$, $(4)$, $(5)$ suy ra ba điểm $D$, $I$, $J$ thuộc giao tuyến của hai mặt phẳng $(ADG)$ và $(DMN)$.\\
			Vậy ba điểm $D$, $I$, $J$ thẳng hàng.
		\end{enumerate}	
	}
\end{vd}
\begin{vd}
	Cho hình chóp $S.ABCD$ có đáy là hình bình hành. Gọi $O$ là giao điểm của $AC$ và $BD$ ; $M, N$ lần lượt là trung điểm của $SB, SD$; $P$ thuộc đoạn $SC$ và không là trung điểm của $SC$.
	\begin{itemize}
		\item [a)] Tìm giao điểm $E$ của đường thẳng $SO$ và mặt phẳng $\left(MNP\right)$.
		\item [b)] Tìm giao điểm $Q$ của đường thẳng $SA$ và mặt phẳng $\left(MNP\right)$.
		\item [c)] Gọi $I, J, K$ lần lượt là giao điểm của $QM$ và $AB$, $QP$ và $AC$, $QN$ và $AD$. Chứng minh rằng $I, J, K$ thẳng hàng.
	\end{itemize}
\end{vd}

\begin{dang}{Vận dụng thực tiễn}
\end{dang}
\begin{vd}
	Giải thích tại sao ghế bốn chân có thể bị khập khiễng còn ghế ba chân thì không. 
	\loigiai{
	Dựa vào tính chất được thừa nhận của hình học không gian, có một và chỉ một mặt phẳng đi qua ba điểm không thẳng hàng cho trước. 
	Do đó qua bốn điểm có thể không cùng nằm trên một phẳng phẳng.
	
}
\end{vd}
\begin{vd}
	Giải thích tại sao chân máy ảnh có thể đặt ở hầu hết các loại hình mà vẫn đứng vững.
	\loigiai{
	Dựa vào tính chất được thừa nhận của hình học không gian, có một và chỉ một mặt phẳng đi qua ba điểm không thẳng hàng cho trước. 
	Do đó giá đỡ ba chân của máy ảnh khi đặt trên mặt đất không bị cập kênh. 
	
}
\end{vd}
\begin{vd}
	Hãy giải thích tại sao phần giao nhau giữa 2 vách tường nhà luôn là 1 đường thẳng 
	\loigiai{
Do mỗi vách tường nhà là 1 phần của mặt phẳng nên phần giao nhau là giao tuyến của của 2 mặt phẳng tức là 1 đường thẳng.

}
\end{vd}
\begin{vd}
	Hãy giải thích vì sao khi gấp đôi một tờ giấy thì nếp gấp luôn là 1 đường thẳng 
	\loigiai{
	Do khi gấp đôi tờ giấy thì mỗi phần của tờ giấy trở thành một phần của mặt phẳng khác nhau và nếp gấp là phần chung tức là giao tuyến của 2 mặt phẳng đó.
}
\end{vd}

\subsection{BÀI TẬP TỰ LUYỆN}


\begin{bt}
	Cho tứ diện $ABCD$. Trên $AB$, $AC$ lấy $2$ điểm $M$, $N$ sao cho $MN$ không song song $BC$. Gọi $O$ là một điểm trong tam giác $BCD$.
	\begin{tasks}(1)
		\task Tìm giao tuyến của $(OMN)$ và $(BCD)$.
		\task Tìm giao điểm của $DC$, $BD$ với $(OMN)$.
	%	\task Tìm thiết diện của $(OMN)$ với hình chóp.
	\end{tasks}
	\loigiai{
		\begin{center}
			\begin{tikzpicture}[scale=1, font=\footnotesize, line join=round, line cap=round, >=stealth]
				\tkzDefPoints{0/0/B,1.3/-1.6/C,4.5/0/D,1/3.5/A,2/-0.5/O}
				\coordinate (M) at ($(A)!2/3!(B)$);
				\coordinate (N) at ($(A)!0.4!(C)$);
				\tkzInterLL(M,N)(B,C)\tkzGetPoint{H}
				\tkzInterLL(B,D)(H,O)\tkzGetPoint{I}
				\tkzInterLL(C,D)(H,O)\tkzGetPoint{J}
				\tkzDrawSegments(C,D A,B A,D A,C H,N H,C N,J)
				\tkzDrawSegments[dashed](B,D H,J N,O M,I M,O)
				\tkzDrawPoints[fill=black](A,B,C,D,M,N,H,O,I,J)
				\tkzLabelPoints[above](A)
				\tkzLabelPoints[below](C,O,I,J)
				\tkzLabelPoints[left](B,M,H)
				\tkzLabelPoints[right](D,N)
			\end{tikzpicture}
		\end{center}
		\begin{enumerate}[a)]
			\item Tìm $(OMN)\cap (BCD)$.
			Trong $(ABC)$, gọi $H=MN\cap BC$.\\
			Ta có $\heva{& H\in MN\subset (MNO) \\ & H\in BC\subset (BCD)}\Rightarrow H\in (BCD)\cap (MNO)$ $(1)$.\\
			Mặt khác $O\in (BCD)\cap (MNO)$ $(2)$.\\
			Từ $(1)$ và $(2)$ suy ra $(BCD)\cap (MNO)=HO$.
			\item Tìm $DC\cap (OMN)$ và $BD\cap (OMN)$.\\
			Trong $(BCD)$, gọi $I=BD\cap HO$.\\
			Ta có $\heva{& I\in BD \\ & I\in HO\subset (MNO)}\Rightarrow I=BD\cap (MNO)$.\\
			Trong $(BCD)$, gọi $J=CD\cap HO$.\\
			Ta có $\heva{& J\in CD \\ & J\in HO\subset (MNO)}\Rightarrow J=CD\cap (MNO)$.
		%	\item Tìm thiết diện của $(OMN)$ và hình chóp.\\
		%	Ta có $\heva{& (ABC)\cap (MNO)=MN \\ & (ABD)\cap (MNO)=MI\\&(ACD)\cap (MNO)=NJ\\&(BCD)\cap (MNO)=IJ}.$ Vậy thiết diện cần tìm là tứ giác $MNJI$.
		\end{enumerate}
	}
\end{bt}

\begin{bt}
	Cho hình chóp $S.ABCD$. Gọi $O$ là giao điểm của $AC$ và $BD$. $M$, $N$, $P$ lần lượt là các điểm trên $SA$, $SB$, $SD$.
	\begin{tasks}(1)
		\task Tìm giao điểm $I$ của $SO$ với mặt phẳng $(MNP)$.
		\task Tìm giao điểm $Q$ của $SC$ với mặt phẳng $(MNP)$.
	\end{tasks}
	\loigiai{
		\begin{enumerate}[a)]	
			\item Tìm giao điểm $I$ của $SO$ với mặt phẳng $(MNP)$.
			\immini{Trong mặt phẳng $(SBD)$, gọi $I=SO\cap NP$, có
				$$\heva{&I\in SO\\&I\in NP\subset (MNP)}\Rightarrow I=SO\cap (MNP).$$
				\item Tìm giao điểm $Q$ của $SC$ với mặt phẳng $(MNP)$.\\
				$\bullet$ Chọn mặt phẳng phụ $(SAC)\supset SC$.\\
				$\bullet$ Tìm giao tuyến của $(SAC)$ và $(MNP)$.\\
				Ta có $\heva{&M\in(MNP)\\&M\in SA,\, SA\subset (SAC)}\Rightarrow M\in(MNP)\cap (SAC)$. \hfill $(1)$\\
				Và $\heva{&I\in SP,\, SP\subset(MNP)\\&I\in SO,\, SO\subset (SAC)}\Rightarrow I\in(MNP)\cap (SAC)$. \hfill $(2)$\\
			}
			{\begin{tikzpicture}[scale=1, font=\footnotesize,line join=round, line cap=round,>=stealth]
					\tkzDefPoints{-1/-2/B,-2/0/A,3/0/D,2/-1.5/C,0/4/S}
					\coordinate (M) at ($(A)!0.3!(S)$);
					\coordinate (N) at ($(B)!0.4!(S)$);
					\coordinate (P) at ($(D)!0.6!(S)$);
					\tkzInterLL(A,C)(B,D)\tkzGetPoint{O}
					\tkzInterLL(S,O)(N,P)\tkzGetPoint{I}
					\tkzInterLL(M,I)(S,C)\tkzGetPoint{Q}
					\tkzDrawSegments(A,B B,C C,D S,A S,B S,C S,D M,N N,Q Q,P)
					\tkzDrawSegments[dashed](A,D A,C B,D M,Q N,P M,P S,O)
					\tkzDrawPoints[fill=black](A,B,C,D,S,O,M,N,P,I,Q)
					\tkzLabelPoints[left](B,A,M,N)
					\tkzLabelPoints[right](C,D,P,Q)
					\tkzLabelPoints[above](S)
					\tkzLabelPoints[above left](I)
					\tkzLabelPoints[below](O)
			\end{tikzpicture}}
			Từ $(1)$ và $(2)$ có $(MNP)\cap (SAC)=MI$.\\
			$\bullet$ Trong mặt phẳng $(SAC)$ gọi $Q=SC\cap MI$, có $\heva{&Q\in SC\\&Q\in MI,\, MI\subset(MNP)}\Rightarrow Q=SC\cap (MNP)$.
			
		\end{enumerate}
	}
\end{bt}

\begin{bt}%[Dự án HHKG 11, 2018, TranTony]%[1H2B1-4]
	Cho hình chóp $S.ABCD$ có đáy $ABCD$ là hình thang với $AB$ song song với $CD$. O là giao điểm của hai đường chéo, $M$ thuộc $SB$.
	\begin{tasks}(1)
		\task Xác định giao tuyến của các cặp mặt phẳng: $(SAC)$ và $(SBD)$; $(SAD)$ và $(SBC)$.
		\task Tìm giao điểm $SO\cap (MCD)$; $SA\cap (MCD)$.
	\end{tasks}
	\loigiai{
		\centerline{\begin{tikzpicture}[scale=1,font=
				\footnotesize,line join=round,line cap=round, >=stealth]
				\tkzDefPoints{0/0/A,7/0/B,2/4/S,2/-3/H}
				\tkzDefPointBy[homothety=center A ratio 0.6](H)\tkzGetPoint{D}
				\tkzDefPointBy[homothety=center B ratio 0.6](H)\tkzGetPoint{C}
				\tkzDefPointBy[homothety=center S ratio 0.4](B)\tkzGetPoint{M}
				\tkzInterLL(A,C)(B,D)    \tkzGetPoint{O}
				\tkzInterLL(S,O)(D,M)    \tkzGetPoint{I}
				\tkzInterLL(S,A)(C,I)    \tkzGetPoint{J}
				\tkzDrawSegments[dashed](A,B A,C B,D S,O D,C D,M C,J)
				\tkzDrawSegments(S,A S,B S,D S,H S,C A,H B,H C,M)
				\tkzDrawPoints(A,B,C,S,O,H,D,M,I,J)
				\tkzLabelPoints[left](A,H,D,S,J)
				\tkzLabelPoints[below right](B,C)
				\tkzLabelPoints[right](M,I)
				\tkzLabelPoints[below](O)
		\end{tikzpicture}}\\
		\begin{enumerate}[a)]
			\item Xác định giao tuyến của $(SAC)$ và $(SBD)$.\\
			Ta có $S$ là điểm chung thứ nhất và $O$ là điểm chung thứ hai của hai mặt phẳng $(SAC)$ và $(SBD)$.\\
			Vậy $(SAC)\cap (SBD)=SO$.\\
			Xác định giao tuyến của $(SAD)$ và $(SBC)$.\\
			Ta có $S\in (SAD)\cap (SBC)$. \hfill (1)\\
			Trong mặt phẳng $(ABCD)$ gọi $H=AD\cap BC$, có $\heva{&H\in AD,AD\subset (SAD)\\& H\in BC, BC\subset (SBC)}\Rightarrow H\in (SAD)\cap (SBC)$. \hfill (2)\\
			Từ $(1)$ và $(2)$ suy ra $(SAD)\cap (SBC)=SH$.
			\item Tìm giao điểm $SO\cap (MCD)$; $SA\cap (MCD)$.\\
			Gọi $I=SO\cap DM$ (vì $SO, DM\subset (SBD)$).\\
			Ta có $\heva{&I\in SO\\&I\in DM ,DM\subset (MCD)}\Rightarrow I=SO\cap (MCD)$.\\
			Gọi $J=SA\cap CI$ (vì $SA, CI\subset (SAC)$).\\
			Ta có $\heva{&J\in SA\\& J\in CI, CI\subset (MCD)}\Rightarrow J=SA\cap (MCD)$.
		\end{enumerate}
	}
\end{bt}

\begin{bt}
	Cho hình chóp $S.ABCD$ có đáy $ABCD$ là hình bình hành tâm $O$. Gọi $M$, $N$ lần lượt là trung điểm của $AB$, $SC$.
	\begin{tasks}(2)
		\task Tìm $I=AN\cap (SBD)$.
		\task Tìm $K=MN\cap (SBD)$.
		\task Tính tỉ số $\dfrac{KM}{KN}$.
		\task Chứng minh $B, I, K$ thẳng hàng. Tính $\dfrac{IB}{IK}$.
	\end{tasks}
	\loigiai{
		\begin{center}
			\begin{tikzpicture}[scale=0.8,font=\footnotesize,line join=round,line cap=round, >=stealth]
				\tkzDefPoints{0/0/A,1/-2.7/B,8.6/-2.7/C, 4/4.8/S}
				\tkzDefPointBy[translation=from B to A](C)\tkzGetPoint{D}
				\tkzDefMidPoint(A,B)\tkzGetPoint{M}
				\tkzDefMidPoint(S,C)\tkzGetPoint{N}
				\tkzInterLL(A,C)(B,D)\tkzGetPoint{O}
				\tkzInterLL(S,O)(A,N)\tkzGetPoint{I}
				\tkzInterLL(M,N)(B,I)\tkzGetPoint{K}
				\tkzDrawSegments[dashed](A,D S,O A,C B,D M,N A,N B,I)
				\tkzDrawSegments(A,B B,C C,D S,A S,B S,C S,D B,N)
				\tkzDrawPoints(A,B,C,D,S,M,I,N,O,K)
				\tkzLabelPoints[left](A,B,M,S)
				\tkzLabelPoints[below](O)
				\tkzLabelPoints[above left](I)
				\tkzLabelPoints[below right](K,N)
				\tkzLabelPoints[right](C,D)
				
			\end{tikzpicture}
		\end{center}
		\begin{enumerate}[a)]
			\item Tìm $I=AN\cap (SBD)$.\\
			Trước hết ta tìm giao tuyến của mp$(SAC)$ và mp$(SBD)$. Ta có $S\in (SAC)\cap (SBD)$. \hfill (1)\\
			Có $\heva{&O\in AC, AC\subset (SAC)\\&O\in BD, BD\subset (SBD)}\Rightarrow O\in (SAC)\cap (SBD)$. \hfill (2)\\
			Từ $(1)$ và $(2)$ suy ra $SO=(SAC)\cap (SBD)$.\\
			Gọi $I=SO\cap AN$ (vì $SO, AN\subset (SAC)$). Suy ra $I=AN\cap (SBD)$.
			\item Tìm $K=MN\cap (SBD)$.\\
			Chọn mp$(ABN)$ chứa $MN$. Tìm giao tuyến của mp$(ABN)$ và mp$(SBD)$.\\ Có $\heva{&I\in SO, SO\subset (SBD)\\&I\in AN, AN\subset (ABN)}\Rightarrow I\in (ABN)\cap (SBD)$. \hfill (3)\\
			Có $B\in (ABN)\cap (SBD)$. \hfill (4)\\
			Từ $(3)$ và $(4)$ suy ra $BI=(ABN)\cap (SBD)$; $K=BI\cap MN$. Khi đó $K=MN\cap (SBD)$.
			\item Tính tỉ số $\dfrac{KM}{KN}$.
			\begin{center}
				\begin{tikzpicture}[scale=1,font=
					\footnotesize,line join=round,line cap=round, >=stealth]
					\tkzDefPoints{0/0/A,6/0/N,4/-2/B,2/0/Q}
					\tkzDefMidPoint(B,A)\tkzGetPoint{M}
					\tkzDefMidPoint(Q,N)\tkzGetPoint{I}
					\tkzInterLL(M,N)(B,I)\tkzGetPoint{K}
					\tkzDrawSegments(A,B B,N N,A Q,M M,N B,I)
					\tkzDrawPoints(A,N,I,K,Q)
					\tkzLabelPoints[left](A,M)
					\tkzLabelPoints[below right](K,B)
					\tkzLabelPoints[above](Q,I,N)
					\tkzMarkSegments[mark=||](I,N I,Q Q,A)
					\tkzMarkSegments[mark=|](M,A M,B)
				\end{tikzpicture}
			\end{center}
			Gọi $Q$ là trung điểm của $AI$. Ta có $AQ=QI=IN$ (vì $I$ là trọng tâm tam giác $SAC$). Có $MQ$ là đường trung bình của tam giác $ABI$. Suy ra $MQ\parallel BI$. Ta có $IK$ là đường trung bình tam giác $MNQ$. Vậy $K$ là trung điểm $MN$. Suy ra $\dfrac{KM}{KN}=1$.
			\item Chứng minh $B, I, K$ thẳng hàng. Tính tỉ số $\dfrac{IB}{IK}$.\\
			Theo cách tìm giao tuyến của câu 2 thì ba điểm $B$, $K$, $I$ thẳng hàng.\\
			Trong tam giác $ABI$, có $QM=\dfrac{1}{2}BI\Rightarrow IB=4IK\Leftrightarrow \dfrac{IB}{IK}=4$.
		\end{enumerate}
	}
\end{bt}

\begin{bt}
	Cho hình chóp $S.ABCD$ với đáy $ABCD$ là hình bình hành. Gọi $M$ là điểm bất kỳ thuộc $SB$, $N$ thuộc miền trong tam giác $S\Delta SCD$.
	\begin{tasks}(1)
		\task Tìm giao điểm của $MN$ và mặt phẳng $\left(ABCD\right)$
		\task Tìm $SC\cap \left(AMN\right)$ và $SD\cap \left(AMN\right)$
		\task Tìm $SA\cap \left(CMN\right)$
	\end{tasks}
	\loigiai{
		\begin{center}
			\begin{tikzpicture}[scale=0.7, line join=round, line cap=round,>=stealth]
				\tkzDefPoints{0/0/A,-1.2/-3/B,4.8/-3/C,6/0/D,0/5.2/S}
				\coordinate (M) at ($(S)!0.22!(B)$);
				%\coordinate (K) at ($(S)!0.3!(C)$);
				\coordinate (I) at ($(C)!0.6!(D)$);
				\coordinate (N) at ($(S)!0.4!(I)$);
				\coordinate (Q) at ($(S)!0.3!(A)$);
				\tkzInterLL(B,I)(M,N)\tkzGetPoint{H}
				\tkzInterLL(A,C)(B,I)\tkzGetPoint{O}
				\tkzInterLL(S,O)(C,Q)\tkzGetPoint{E}
				\tkzInterLL(A,E)(S,C)\tkzGetPoint{K}
				\tkzInterLL(K,N)(S,D)\tkzGetPoint{P}
				\tkzDrawPoints[fill=black](S,A,B,C,D,E,M,N,K,I,O,H,P,Q)
				\tkzLabelPoints[above](S)
				\tkzLabelPoints[left](A)
				\tkzLabelPoints[below left](B,M,E)
				\tkzLabelPoints[below right](K,N,P,C,I)
				\tkzLabelPoints[above  right](D,H)
				\tkzLabelPoints[right](Q)
				\tkzLabelPoints[below](O)
				\tkzDrawSegments(S,B S,C S,D B,C C,D S,I C,N N,H C,M H,I K,N N,P)
				\tkzDrawSegments[dashed](S,A A,B A,D A,C B,I M,N A,N S,O C,Q A,K)
			\end{tikzpicture}
		\end{center}
		\begin{enumerate}[a)]
			\item Tìm giao điểm của $MN$ và $\left(ABCD\right)$.\\
			Gọi $I=SN \cap CD$ (vì $SN,\,CD \subset \left(SCD\right)$). Chọn mặt phẳng $\left(SBI\right)$ chứa $MN$. Ta có $B$ và $I$ là hai điểm chung của hai mặt phẳng $\left(SBI\right)$ và $\left(ABCD\right)$. Vậy $\left(SBI\right)\cap \left(ABCD\right)=BI$.\\
			Gọi $H=MN \cap BI$ (vì $MN,\,BI\subset \left(SBI\right)$)
			Ta có $\heva{& H \in MN \\ & H \in BI,\, BI \subset \left(ABCD\right)}$ $\Rightarrow H=MN\cap \left(ABCD\right)$
			\item Tìm $SC\cap \left(MAN\right)$.\\
			Đầu tiên ta tìm giao tuyến của mặt phẳng $\left(SAC\right)$ và $\left(SBI\right)$. Gọi $O=AC \cap BI$ (vì $AC,\,BI \subset \left(ABCD\right)$).\\
			Ta có $S$ và $O$ là hai điểm chung của hai mặt phẳng $\left(SAC\right)$ và $\left(SBI\right)$.\\
			Vậy $SO=\left(SAC\right)\cap \left(SBI\right)$.\\
			Gọi $E=SO \cap MN$ (vì $SO,\,MN \subset \left(SBI\right)$). Chọn mặt phẳng $\left(SAC\right)$ chứa $SC$. Tìm giao tuyến của hai mặt phẳng $\left(SAC\right)$ và $\left(AMN\right)$
			\item
		\end{enumerate}
	}
\end{bt}

\begin{bt}%[Dự án HHKG 11, 2018, Chu Duc Minh]%[1H2K1-5]
	Cho tứ diện $ABCD$	. Gọi $M$ là trung điểm $AB$, $K$ là trọng tâm của tam giác $ACD$. 
	\begin{tasks}(1)
		\task Xác định giao tuyến của $(AKM)$ và $(BCD)$. 
		\task Tìm giao điểm $H$ của $MK$ và mp$(BCD)$. Chứng minh $K$ là trọng tâm của tam giác $ABH$. 
		\task Trên $BC$ lấy điểm $N$. Tìm giao điểm $P, Q$ của $CD$, $AD$ với mp$(MNK)$. 
	\end{tasks}
	\loigiai{
		\begin{center}
			\begin{tikzpicture}[scale=0.6, line join=round, line cap=round]
				\tkzDefPoints{0/0/B,1.3/-1.6/C,4.5/0/D,1/3.5/A}
				\coordinate (M) at ($1/2*(A)+1/2*(B)$);
				\coordinate (N) at ($1/3*(B)+2/3*(C)$);
				\coordinate (K) at ($1/3*(A)+1/3*(C)+1/3*(D)$);
				\coordinate (G) at ($1/2*(C)+1/2*(D)$);
				\tkzInterLL(M,K)(B,G)\tkzGetPoint{H}
				\tkzInterLL(M,N)(A,C)\tkzGetPoint{E}
				\tkzInterLL(E,K)(C,D)\tkzGetPoint{P}	
				\tkzInterLL(E,K)(A,D)\tkzGetPoint{Q}
				\tkzInterLL(M,Q)(N,P)\tkzGetPoint{F}
				\tkzDrawPolygon(A,B,C,D)
				\tkzDrawSegments(A,C A,G H,K H,G E,M E,A E,Q F,N F,M)
				\tkzDrawSegments[dashed](B,D M,K B,G N,P M,Q F,B)
				\tkzDrawPoints[fill=black,size=4](A,B,C,D,K,G,H,E)
				\tkzLabelPoints[above](A)
				\tkzLabelPoints[left](F, E)
				\tkzLabelPoints[above left](B,M)
				\tkzLabelPoints[below left](N)
				\tkzLabelPoints[below right](C,P,G)
				\tkzLabelPoints[right](D,K,Q,H)
			\end{tikzpicture}
		\end{center}
		\begin{enumerate}[a)]
			\item \textbf{Xác định giao tuyến của $(AKM)$ và $(BCD)$.} \\
			Gọi $G = AK \cap CD$  (vì $AK, CD \subset (ACD)$). \\
			Ta có $\heva{&G \in AK, AK \subset (AKM) \\ & G \in CD, CD \subset (BCD)}$\\
			$\Rightarrow G \in (AKM) \cap (BCD). \quad (1)$\\
			$B \in (ABG) \cap (BCD). \quad (2)$\\
			Từ $(1)$ và $(2)$ suy ra $(ABG) \cap (BCD) = BG$. 
			\item \textbf{Tìm giao điểm $H$ của $MK$ và} mp$(BCD)$. \\
			Trong mp$(ABG)$, gọi $H = MK \cap BG$, \\ có $\heva{& H \in MK \\& H \in BG, BG \subset (BCD)}$\\
			$\Rightarrow H = MK \cap (BCD)$.\\
			\textbf{Chứng minh $K$ là trọng tâm của tam giác $ABH$.}
			\begin{center}
				\begin{tikzpicture}[scale=1, line join=round, line cap=round]
					\tkzDefPoints{1/3/A,0/0/B,5/0/H}
					\coordinate (G) at ($1/2*(B)+1/2*(H)$);
					\coordinate (M) at ($1/2*(A)+1/2*(B)$);
					\coordinate (K) at ($1/3*(A)+1/3*(B)+1/3*(H)$);
					\coordinate (L) at ($2*(G)-(K)$);
					\tkzDrawPolygon(A,B,H)
					\tkzDrawSegments(A,L M,H L,B)
					\tkzDrawPoints[fill=black,size=4](A,B,H,G,L,M,K)
					\tkzLabelPoints[above](A)
					\tkzLabelPoints[left](B,M)
					\tkzLabelPoints[above right](K)
					\tkzLabelPoints[below right](H,L)
					\tkzLabelPoints[below left](G)
				\end{tikzpicture}
			\end{center}
			Vì $K$ là trọng tâm của tam giác $ACD$ nên $K$ chia đoạn $AG$ thành ba phần bằng nhau. \\
			Gọi $L$ là điểm đối xứng của $K$ qua $G$ thì $K$ là trung điểm của $AL$. \\ Trong $\triangle ABL$, $MK$ là đường trung bình của tam giác.\\
			Ta có $\triangle BGL = \triangle HGK$(g.c.g) $\Rightarrow BG= HG$.\\ Vậy $K$ là trọng tâm của tam giác $ABH$. 
			\item \textbf{Tìm giao điểm $P, Q$ của $CD, AD$ với} mp$(MNK)$. \\
			Trong mp$(ABC)$ gọi $E = MN \cap AC$. Trong mp$(ACD)$ đường thẳng $EK$ cắt $CD$ và $AD$ lần lượt tại $P, Q$, thì $P$ và $Q$ chính là giao điểm của $CD$ và $AD$ với mp$(MNK)$. 
		\end{enumerate}
	}
\end{bt}

\begin{bt}%[Dự án HHKG 11, 2018, Nguyễn Thành Tiến]%[1H2B1-1]%[1H2K1-3]
	Cho tứ giác $ABCD$ và $ S \not\in \left(ABCD\right)$. Gọi $I,\,J$ là hai điểm trên $AD$ và $SB$, $AD$ cắt $BC$ tại $O$ và $OJ$ cắt $SC$ tại $M$.
	\begin{tasks}(1)
		\task Tìm giao điểm $K=IJ \cap \left(SAC\right)$.
		\task Xác định giao điểm $L=DJ \cap \left(SAC\right)$.
		\task Chứng minh $A,\,K,\,L,\,M$ thẳng hàng.
	\end{tasks}
	\loigiai{
		\begin{center}
			\begin{tikzpicture}[scale=0.8, line join=round, line cap=round,>=stealth]
				\tkzDefPoints{0/0/A, 7/0/B,2/-2.1/D, 4.8/-1.6/C, 2/4/S}
				\coordinate (I) at ($(A)!0.3!(D)$);
				\coordinate (J) at ($(S)!0.33!(B)$);
				\tkzDrawPoints[fill=black](S,A,B,C,D,I,J)
				\tkzLabelPoints[above](S)
				\tkzLabelPoints[left](A)
				\tkzLabelPoints[right](B)
				\tkzLabelPoints[below right](C)
				\tkzLabelPoints[below left](D)
				\tkzLabelPoints[below left](I)
				\tkzLabelPoints[above right](J)
				\tkzDrawSegments(S,A S,B S,C S,D A,D B,C S,I)
				\tkzDrawSegments[dashed](A,C A,B B,D C,D I,J D,J I,B) 		
				\tkzInterLL(A,D)(B,C)\tkzGetPoint{O}
				\tkzInterLL(S,C)(O,J)\tkzGetPoint{M}
				\tkzInterLL (A,C)(B,I)\tkzGetPoint{E}
				\tkzInterLL(A,C)(B,D)\tkzGetPoint{F}
				\tkzInterLL(A,M)(I,J)\tkzGetPoint{K}
				\tkzInterLL(D,J)(S,F)\tkzGetPoint{L}
				\tkzDrawPoints[fill=black](O,E,M,F,K,L)
				\tkzDrawSegments[dashed](S,E A,M)
				\tkzLabelPoints[below](O)
				\tkzLabelPoints[right](M)
				\tkzLabelPoints[below](E)
				\tkzLabelPoints[below](F)
				\tkzLabelPoints[above left](L)
				\tkzDrawSegments(O,D O,C O,J)
			\end{tikzpicture}
		\end{center}	
		\begin{enumerate}[a)]
			\item Tìm giao điểm $K=IJ \cap \left(SAC\right)$.\\
			Chọn mặt phẳng phụ $\left(SIB\right)$ chứa $IJ$.\\
			Tìm giao tuyến của $\left(SIB\right)$ và $\left(SAC\right)$.\\
			có $S \in \left(SBI\right) \cap \left(SAC\right)$\hfill (1)\\
			Trong mặt phẳng $\left(ABCD\right)$ gọi $E=AC \cap BI$, ta có$\colon$\\
			$\heva{& E \in AC, \, AC \subset \left(SAC\right) \\ &E \in BI,\, BI \subset \left(SBI\right)}$ $\Rightarrow E=\left(SAC\right) \cap \left(SBI\right)$ \hfill (2)\\
			Từ $(1)$ và $(2)$ suy ra $SE=\left(SBI\right) \cap \left(SAC\right)$.\\
			Trong mặt phẳng $\left(SIB\right)$, gọi $K=IJ \cap SE$.\\
			Ta có $\heva{& K \in IJ \\ & K \in SE, SE \subset \left(SAC\right)}$ $\Rightarrow K=IJ \cap \left(SAC\right)$
			\item Xác định giao điểm $L=DJ \cap \left(SAC\right)$.\\
			Chọn mặt phẳng phụ $\left(SBD\right)$ chứa $DJ$. Tìm giao tuyến của $\left(SBD\right)$ với $\left(SAC\right)$.\\
			Ta có $S \in \left(SBD\right) \cap \left(SAC\right) $ \hfill (3)\\
			\noindent Trong mặt phẳng $\left(ABCD\right)$ gọi $F=AC \subset BD$. Suy ra $F$ là điểm chung thứ hai của hai mặt phẳng $\left(SBD\right)$ và $\left(SAC\right)$. \hfill (4)\\
			Từ $(3)$ và $(4)$ suy ra $SF=\left(SBD\right)\subset \left(SAC\right)$. Trong mặt phẳng $\left(SBD\right)$ gọi $L=DJ \cap SF $.\\
			Vậy
			$\heva{& L \in DJ \\ & L \in SF,\, SF \subset \left(SAC\right)}$ $\Rightarrow L=DJ \cap \left(SAC\right)$
			\item Chứng minh $A,\, K,\, L, \, M$ thẳng hàng.\\
			Ta có $A \in \left(SAC\right)\cap \left(AJO\right)$ \quad (3)\\
			và $\heva{& K \in IJ,\, IJ \subset \left(AJO\right) \\ & K \in SE,\, SE \subset \left(SAC\right)}$ $\Rightarrow K \in \left(SAC\right)\cap \left(AJO\right)$. \quad (4)\\
			có $\heva{& L \in DJ,\, DJ \subset \left(AJO\right) \\ & L \in SF,\, SF \subset \left(SAC\right)}$ $\Rightarrow L \in \left(SAC\right) \cap \left(AJO\right)$ \quad (5)\\
			có $\heva{& M \in JO,\, JO \subset \left(AJO\right) \\ & M \in SC,\, SC \subset \left(SAC\right)}$ $\Rightarrow M \in \left(SAC\right)\cap \left(AJO\right)$ \quad (6)\\
			Từ (3), (4), (5) và (6) suy ra bốn điểm $A,\,K,\, L,\,M$ cùng thuộc giao tuyến của hai mặt phẳng $\left(SAC\right)$ và $\left(AJO\right)$. Vậy $A,\,K,\,L,\,M$ thẳng hàng.
		\end{enumerate}
	}	
\end{bt}

\begin{bt}
	Cho hình chóp $S.ABCD$ có đáy $ABCD$ và hình bình hành. Gọi $G$ là trọng tâm của tam giác $SAD$, $M$ là trung điểm của $SB$. 
	\begin{tasks}(1)
		\task Tìm giao điểm $N$ của $MG$ và mặt phẳng $(ABCD)$. 
		\task Chứng minh ba điểm $C, D, N$ thẳng hàng và $D$ là trung điểm của $CN$. 
	\end{tasks}
	\loigiai{
		\begin{center}
			\begin{tikzpicture}[scale=1, line join=round, line cap=round]
				\tkzDefPoints{0/0/A,-1.7/-1.6/B,2.5/-1.6/C}
				\coordinate (D) at ($(A)+(C)-(B)$);
				\coordinate (S) at ($(A)+(-0.5,3)$);
				\coordinate (G) at ($1/3*(S)+1/3*(A)+1/3*(D)$);
				\coordinate (M) at ($1/2*(S)+1/2*(B)$);
				\coordinate (E) at ($1/2*(A)+1/2*(D)$);
				\coordinate (F) at ($1/2*(B)+1/2*(M)$);
				\tkzInterLL(B,E)(M,G)\tkzGetPoint{N}
				\tkzInterLL(B,N)(S,D)\tkzGetPoint{X}
				\tkzInterLL(M,N)(S,D)\tkzGetPoint{Y}
				
				\tkzDrawPolygon(S,B,C,D)
				\tkzDrawSegments(S,C S,N Y,N X,N N,D)
				\tkzDrawSegments[dashed](A,S A,B A,D S,E F,E B,X M,Y)
				\tkzDrawPoints[fill=black,size=4](D,C,A,B,S,M,G,N,E,F)
				\tkzLabelPoints[above](S)
				\tkzLabelPoints[left](A,M,F)
				\tkzLabelPoints[below](B,C,E)
				\tkzLabelPoints[right](D,N)
				\tkzLabelPoints[above right](G)
			\end{tikzpicture}
		\end{center}
		\begin{enumerate}[a)]
			\item Trong mặt phẳng chứa $MG$, gọi $N$ là giao điểm của $MG$ và $BE$. 
			Vì $BE$ thuộc mặt phẳng $(ABCD)$, nên $N$ thuộc $(ABCD)$. Vậy $N$ là giao điểm của $MG$ và mặt phẳng $(ABCD)$.
			\item Trong mặt phẳng $(SBN)$, kẻ $EF \parallel MN$ ($F$ thuộc $SB$). \\
			Trong tam giác $SEF$ có $MG \parallel EF$ nên $$\dfrac{SM}{MF} = \dfrac{SG}{GE} = 2 \Rightarrow SM = 2MF \Leftrightarrow BM = 2MF.$$
			Vậy $F$ là trung điểm của $BM$. 
		\end{enumerate}	
		Trong $\triangle  BMN$ có $EF \parallel MN$ nên  $\dfrac{BF}{FM} = \dfrac{BE}{EN} = 1 \Rightarrow BE = EN$. Vậy $E$ là trung điểm của $BN$. 
		
		Dễ dàng chứng minh $\triangle AEB = \triangle DEN$ (c.g.c) $\Rightarrow \widehat{ABE} = \widehat{END}$.\\ Hai góc này bằng nhau theo trường hợp so le trong nên $AB \parallel DN$,  mà $AB \parallel CD$ nên $C, D, N$ thẳng hàng. 
		
		$ED$ là đường trung bình của tam giác $NBC$ suy ra $D$ là trung điểm của $CN$. 
		
	}
\end{bt}

\begin{bt}
	Cho hình chóp $S.ABCD$, đáy $ABCD$ là hình bình hành tâm $O$. Gọi $M$ là trung điểm của $SC$. 
	\begin{tasks}(1)
		\task Xác định giao tuyến của $(ABM)$ và $(SCD)$. 
		\task Gọi $N$ là trung điểm của $BO$. Xác định giao điểm $I$ của $(AMN)$ với $SD$. Chứng minh $\dfrac{SI}{ID} = \dfrac{2}{3}$. 
		%Tìm thiết diện của hình chóp $S.ABCD$ cắt bởi mặt phẳng $(AMN)$. 
	\end{tasks}
	\loigiai{
		\begin{center}
			\begin{tikzpicture}[scale=1, line join=round, line cap=round]
				\tkzDefPoints{0/0/A,-3/-2.4/B,2.4/-2.4/C}
				\coordinate (D) at ($(A)+(C)-(B)$);
				\coordinate (O) at ($(A)!1/2!(C)$);
				\coordinate (S) at ($(A)+(0.5,3)$);
				\coordinate (M) at ($1/2*(S)+1/2*(C)$);
				\coordinate (N) at ($1/2*(B)+1/2*(O)$);
				\coordinate (H) at ($1/2*(S)+1/2*(D)$);
				\tkzInterLL(A,M)(S,O)\tkzGetPoint{K}
				\tkzInterLL(N,K)(S,D)\tkzGetPoint{I}
				\tkzInterLL(A,N)(B,C)\tkzGetPoint{L}
				\coordinate (P) at ($2/3*(I)+1/3*(D)$);
				\tkzFillPolygon[color=gray!50!](A,I,M,L)
				\tkzDrawPolygon(S,B,C,D)
				\tkzDrawSegments(S,C M,H M,I L,M)
				\tkzDrawSegments[dashed](A,S A,B A,D A,C B,D S,O A,M N,I A,I A,L)
				\tkzDrawPoints[fill=black,size=4](D,C,A,B,S,N,O,K,I,P,H,L)
				\tkzLabelPoints[above](S)
				\tkzLabelPoints[left](A)
				\tkzLabelPoints[below](B,C,L,O)
				\tkzLabelPoints[right](D,M)
				\tkzLabelPoints[above right](I,H)
				\tkzLabelPoints[left](K)
			\end{tikzpicture}
		\end{center}
		\begin{enumerate}[a)]
			\item Xác định giao tuyến của $(ABM)$ và $(SCD)$. \\
			Ta có $\heva{& M \in (ABM) \cap (SCD)\\ &AB \parallel CD\\ &AB \subset (ABM), CD \subset (SCD)}\Rightarrow (ABM) \cap (SCD) = MH$ ($MH \parallel AB \parallel CD$.)
			\item Xác định giao điểm $I$ của $(AMN)$  và $SD$
			Ta có $(SAC) \cap (SBD) = SO$. Gọi $K = AM \cap SO$  ($AM, SO \subset (SAC)$). \\
			\textbf{Tìm giao tuyến $(AMN)$ và $(SBD)$.}\\
			Ta có $\heva{&N \in (AMN)\\ &N \in BD, BD \subset (SBD)} \Rightarrow N \in (AMN) \cap (SBD). \quad (1)$\\
			$\heva{&K \in AM, AM \subset (AMN)\\ & K \in BD, BD \subset (SBD)} \Rightarrow K \in (AMN) \cap (SBD). \quad (2)$\\
			Từ $(1)$ và $(2)$ suy ra $(AMN) \cap (SBD) = NK$. 
			$NK$ cắt $SD$ tại điểm $I$, thì $I$ chính là giao điểm của $(AMN)$ và $SD$. \\
			\begin{center}
				\begin{center}
					\begin{tikzpicture}[scale=1, line join=round, line cap=round]
						\tkzDefPoints{0/0/B,1/4/S,5/0/D}
						\coordinate (N) at ($(B)!1/4!(D)$);
						\coordinate (O) at ($(B)!1/2!(D)$);
						\coordinate (I) at ($3/5*(S)+2/5*(D)$);
						\coordinate (P) at ($(I)!1/3!(D)$);
						\coordinate (K) at ($(S)!2/3!(O)$);
						\tkzDrawPolygon(S,B,D)
						\tkzDrawSegments(O,P N,I S,O)
						\tkzDrawPoints[fill=black,size=4](S,B,D,I,N,P,K)
						\tkzLabelPoints[above](S)
						\tkzLabelPoints[below](B,D,O,N)
						\tkzLabelPoints[above right](I,P)
						\tkzLabelPoints[left](K)
					\end{tikzpicture}
				\end{center}
			\end{center}
			Trong mặt phẳng $(SBD)$, từ $O$ dựng $OP \parallel NI (P \in SD)$. \\
			Trong $\triangle DNI$, có $OP \parallel DI$ nên có $\dfrac{DO}{ON} = \dfrac{DP}{PI} = \dfrac{2}{1} = 2 \Rightarrow DP = 2PI. \quad (3)$\\
			Trong $\triangle SOP$ có $KI \parallel OP$ nên có $\dfrac{SK}{KO} = \dfrac{SI}{PI} = \dfrac{2}{1} = 2 \Rightarrow SI = 2PI. \quad (4)$ 
			($K$) là trọng tâm của $\triangle SAC$. 
			Từ $(3)$ và $(4)$ suy ra $\dfrac{IS}{ID} = \dfrac{2}{3}$. 
			
		%	\textbf{Thiết diện của hình chóp bị cắt bởi mặt phẳng $(AMN)$.} \\
		%	Gọi $L$ là giao điểm của $AN$ và $BC$. Kết luận thiết diện là tứ giác $ALMI$. 
		\end{enumerate}
	}
\end{bt}


\begin{bt}
	Cho tứ diện $SABC$. Gọi $I$, $H$ lần lượt là trung điểm của $SA$, $AB$. Trên cạnh $SC$ lấy điểm $K$ sao cho $CK=3SK$.
	\begin{tasks}(1)
		\task Tìm giao điểm $F$ của $BC$ với mặt phẳng $(IHK)$. Tính tỉ số $\dfrac{FB}{FC}$.
		\task Gọi $M$ là trung điểm của đoạn thẳng $IH$. Tìm giao điểm của $KM$ và mặt phẳng $(ABC)$.
	\end{tasks}
	\loigiai{
		\begin{center}
			{\begin{tikzpicture}[scale=1, font=\footnotesize,line join=round, line cap=round,>=stealth]
					\tkzDefPoints{-2/0/A,0/-2.5/B,3/0/C,1/4/S}
					\coordinate (I) at ($(A)!0.5!(S)$);
					\coordinate (H) at ($(A)!0.5!(B)$);
					\coordinate (M) at ($(I)!0.5!(H)$);
					\coordinate (K) at ($(S)!1/4!(C)$);
					\coordinate (D) at ($(S)!0.5!(C)$);
					\coordinate (N) at ($(B)!0.5!(C)$);
					\tkzInterLL(A,C)(K,I)\tkzGetPoint{E}
					\tkzInterLL(E,H)(B,C)\tkzGetPoint{F}
					\tkzInterLL(K,M)(E,H)\tkzGetPoint{J}
					\tkzDrawSegments(H,B B,C S,I S,B S,C E,I K,F E,H I,H M,J)
					\tkzDrawSegments[dashed](A,H A,C A,N H,F H,K A,D M,K E,A A,I I,K)
					\tkzDrawPoints[fill=black](A,B,C,S,I,H,M,K,E,F,D,J,N)
					\tkzLabelPoints[left](A,B,E)
					\tkzLabelPoints[right](C,M,K,F,D,N)
					\tkzLabelPoints[above](S)
					\tkzLabelPoints[above left](I)
					\tkzLabelPoints[below](H,J)
					\tkzMarkSegments[mark=|](S,K K,D)
					\tkzMarkSegments[mark=||](A,H H,B)
					\tkzMarkSegments[mark=|||](A,I S,I)
			\end{tikzpicture}}
		\end{center}
		\begin{enumerate}[a)]
			\item Tìm giao điểm $F$ của $BC$ với mặt phẳng $(IHK)$. Tính tỉ số $\dfrac{FB}{FC}$.\\
			$\bullet$ Ta tìm giao tuyến của $(ABC)$ và $(IHK)$ trước.\\
			Gọi $E=AC\cap KI$ ($AC,\,KI\subset (SAC)$), ta có\\
			$\heva{&E\in AC,\, AC\subset (ABC)\\&E\in KI,\, KI\subset (IHK)}\Rightarrow E\in(ABC)\cap (IHK)$. \hfill $(1)$\\
			$\heva{&H\in (IHK)\\&H\in AB,\, AB\subset (ABC)}\Rightarrow H\in(ABC)\cap (IHK)$. \hfill $(2)$\\
			Từ $(1)$ và $(2)$ suy ra $EH=(ABC)\cap (IHK)$.\\
			$\bullet$ Gọi $F=EH\cap BC$ ($EH,\,BC\subset (ABC)$), có
			$$\heva{&F\in BC\\&F\in EH, \, EH\subset (IHK)}\Rightarrow F=BC\cap (IHK).$$
			Gọi $D$ là trung điểm của $SC$, ta có $IK$ là đường trung bình của $\triangle SAD$.\\
			Trong $\triangle CEK$ có $\dfrac{CA}{AE}=\dfrac{CD}{DK}=2\Rightarrow CA=2CK$.\\
			Trong mặt phẳng $(ABC)$ kẻ $AN\parallel EF$ ($N\in BC$).
			Ta có
			\begin{eqnarray*}
				&& HF\parallel AN\Rightarrow \dfrac{BH}{HA}=\dfrac{BF}{FN}=1\Rightarrow BF=FN.\\
				&& EF\parallel AN\Rightarrow \dfrac{CA}{AE}=\dfrac{CN}{NF}=2\Rightarrow CN=2NF.
			\end{eqnarray*}
			Do đó $\dfrac{FB}{FC}=\dfrac{FB}{FN+NC}=\dfrac{FB}{3FB}=\dfrac{1}{3}$.
			\item Tìm giao điểm của $KM$ và mặt phẳng $(ABC)$.\\
			Ta có $KM\subset (IHK)$. Gọi $J=KM\cap EH$ ($ EH,\, KM\subset (IHK)$).\\
			Ta có $\heva{&J\in KM\\&J\in EH,\, EH\subset (ABC)}\Rightarrow J=KM\cap (ABC)$.
		\end{enumerate}
	}
\end{bt}
% \subsection{BÀI TẬP TRẮC NGHIỆM}

\Opensolutionfile{ans}[ans/1H4.B1]
\setcounter{ex}{0}
\begin{ex}%[1H2Y1-1]
	Cho tứ giác $ABCD$. Có thể xác định được bao nhiêu mặt phẳng chứa tất cả các đỉnh của tứ giác $ABCD$?
	\choice
	{\True $1 $}
	{$3 $}
	{$0 $}
	{$2 $}
	\loigiai{
		$4$ điểm $A,B,C,D$ tạo thành $1$ tứ giác, khi đó $4$ điểm $A,B,C,D$ đã đồng phẳng và tạo thành $1$ mặt phẳng duy nhất là mặt phẳng $\left(ABCD\right)$.}
\end{ex}

\begin{ex}%[1H2Y1-1]%
	Hình chóp tam giác có số cạnh là
	\choice
	{\True $ 6 $}
	{$ 4 $}
	{$ 5 $}
	{$ 3 $}
	\loigiai{
		\immini{Xét hình chóp tam giác $ S.ABC $ có các cạnh là $ SA $, $ SB $, $ SC $, $ AB $, $ BC $ và $ CA $. Vậy hình chóp có số cạnh là $ 6 $.}
		{\begin{tikzpicture}[line join=round,line cap=round,line width=.6pt,font=\footnotesize,scale=0.6]
				\coordinate[label=left:$A$] (A) at (0,0);
				\coordinate[label=below left:$B$] (B) at (1,-1);
				\coordinate[label=right:$C$] (C) at (4,0);
				\coordinate[label=above left:$S$] (S) at (1.2,4);
				\draw (A)--(B)--(C)--(S)--cycle (S)--(B);
				\draw[dashed] (A)--(C);
				\fill (A)circle(2pt) (B)circle(2pt) (C)circle(2pt) (S)circle(2pt);
			\end{tikzpicture}
		}
	}
\end{ex}

\begin{ex}%[1H2B1]
	Hình chóp lục giác có bao nhiêu mặt?
	\choice
	{$10$}
	{$6$}
	{$8$}
	{\True $7$}
	\loigiai{
		Hình chóp có $7$ mặt trong đó có $6$ mặt bên và $1$ mặt đáy.}
\end{ex}


\begin{ex}%[1H2Y1-1]
	Các yếu tố nào sau đây xác định một mặt phẳng duy nhất?
	\choice
	{Một điểm và một đường thẳng}
	{\True Hai đường thẳng cắt nhau}
	{Bốn điểm phân biệt}
	{Ba điểm phân biệt}
	\loigiai{
		\begin{itemize}
			\item Mệnh đề \lq\lq Ba điểm phân biệt\rq\rq\  sai. Trong trường hợp $3$ điểm phân biệt thẳng hàng thì sẽ có vô số mặt phẳng chứa $3 $ điểm thẳng hàng đã cho.
			\item Mệnh đề \lq\lq Một điểm và một đường thẳng\rq\rq\ sai. Trong trường hợp điểm thuộc đường thẳng đã cho, khi đó ta chỉ có $1$ đường thẳng, có vô số mặt phẳng đi qua đường thẳng đó.
			\item  Mệnh đề \lq\lq Bốn điểm phân biệt\rq\rq\ sai. Trong trường hợp $4$ điểm phân biệt thẳng hàng thì có vô số mặt phẳng đi qua $4$ điểm đó hoặc trong trường hợp $4$ điểm mặt phẳng không đồng phẳng thì sẽ không tạo được mặt phẳng nào đi qua cả $4$ điểm.
	\end{itemize}}
\end{ex}

\begin{ex}%[1H2Y1]
	Khẳng định nào sau đây là \textbf{sai}?
	\choice
	{Nếu hai mặt phẳng phân biệt có một điểm chung thì chúng có một đường thẳng chung duy nhất}
	{Nếu hai mặt phẳng có một điểm chung thì chúng có vô số điểm chung khác nữa}
	{Nếu ba điểm phân biệt cùng thuộc hai mặt phẳng phân biệt thì chúng thẳng hàng}
	{\True  Nếu hai mặt phẳng có một điểm chung thì chúng có một đường thẳng chung duy nhất}
	\loigiai{
		Hai mặt phẳng có một điểm chung thì có thể trùng nhau, khi đó chúng có vô số đường thẳng chung.
	}
\end{ex}

\begin{ex}%[1H2Y1-1]
	Cho $5$ điểm $A,B,C,D,E$ trong đó không có $4$ điểm nào đồng phẳng. Hỏi có bao nhiêu mặt phẳng tạo bởi $3$ trong $5$ điểm đã cho?
	\choice
	{\True $10 $}
	{$14 $}
	{$12 $}
	{$8 $}
	\loigiai{
		Với $3$ điểm phân biệt không thẳng hàng, ta luôn tạo được $1$ mặt phẳng xác định.
		Ta có $\mathrm{C}_5^3$ cách chọn $3$ điểm trong $5$ điểm đã cho để tạo được $1$ mặt phẳng xác định. Vậy số mặt phẳng tạo được là $10$.}
\end{ex}

\begin{ex}%[1H2Y1-1]
	Trong các khẳng định sau, khẳng định nào đúng?
	\choice
	{Qua $3$ điểm phân biệt bất kì có duy nhất một mặt phẳng}
	{Qua $4$ điểm phân biệt bất kì có duy nhất một mặt phẳng}
	{Qua $2$ điểm phân biệt có duy nhất một mặt phẳng}
	{\True Qua $3$ điểm không thẳng hàng có duy nhất một mặt phẳng}
	\loigiai{
		\begin{itemize}
			\item Mệnh đề \lq\lq Qua $2$ điểm phân biệt có duy nhất một mặt phẳng\rq\rq\ sai. Vì qua $2 $ điểm phân biệt, tạo được $1$ đường thẳng, khi đó chưa đủ điều kiện để lập một mặt phẳng xác định. Có vô số mặt phẳng đi qua $2$ điểm đã cho.
			\item Mệnh đề \lq\lq Qua $3$ điểm phân biệt bất kì có duy nhất một mặt phẳng\rq\rq\ sai. Vì trong trường hợp $3$ điểm phân biệt thẳng hàng thì chỉ tạo được đường thẳng, khi đó có vô số mặt phẳng đi qua $3$ điểm phân biệt thẳng hàng.
			\item Mệnh đề \lq\lq Qua $4$ điểm phân biệt bất kì có duy nhất một mặt phẳng\rq\rq\ sai. Vì trong trường hợp $4$ điểm phân biệt thẳng hàng thì có vô số mặt phẳng đi qua $4$ điểm đó hoặc trong trường hợp $4$ điểm mặt phẳng không đồng phẳng thì sẽ tạo không tạo được mặt phẳng nào đi qua cả $4$ điểm.
	\end{itemize}}
\end{ex}

\begin{ex}%[Nguyễn Phúc Đức]%[1H2B1]
	Cho các hình vẽ sau: \\
	\begin{tabular}{cccc}
		\begin{tikzpicture}[scale=0.6,line width=1pt,line cap=round,line join=round,>=triangle 45,x=1.0cm,y=1.0cm]
			\clip(-3.48,0.6) rectangle (2.4,5.4);
			\draw (-1,5)-- (-3,2);
			\draw (-1,5)-- (0,1);
			\draw (-1,5)-- (2,2);
			\draw (-3,2)-- (0,1);
			\draw (0,1)-- (2,2);
			\draw [dash pattern=on 2pt off 2pt] (-3,2)-- (2,2);
			\fill [color=black] (-1,5) circle (1.5pt);
			\draw[color=black] (-0.8,5.2) node {$A$};
			\fill [color=black] (-3,2) circle (1.5pt);
			\draw[color=black] (-3.1,2.3) node {$B$};
			\fill [color=black] (0,1) circle (1.5pt);
			\draw[color=black] (-0.3,0.8) node {$C$};
			\fill [color=black] (2,2) circle (1.5pt);
			\draw[color=black] (2.2,2.3) node {$D$};
		\end{tikzpicture} & \begin{tikzpicture}[scale=0.6,line width=1pt,line cap=round,line join=round,>=triangle 45,x=1.0cm,y=1.0cm]
			\clip(-3.52,0.46) rectangle (2.56,5.48);
			\draw (-1,5)-- (-3,1);
			\draw (-1,5)-- (0,2);
			\draw (-1,5)-- (2,1);
			\draw (-3,1)-- (0,2);
			\draw (0,2)-- (2,1);
			\draw (-3,1)-- (2,1);
			\fill [color=black] (-1,5) circle (1.5pt);
			\draw[color=black] (-1.3,5.2) node {$A$};
			\fill [color=black] (-3,1) circle (1.5pt);
			\draw[color=black] (-3.14,1.26) node {$B$};
			\fill [color=black] (0,2) circle (1.5pt);
			\draw[color=black] (-0.04,1.6) node {$C$};
			\fill [color=black] (2,1) circle (1.5pt);
			\draw[color=black] (2.16,1.28) node {$D$};
		\end{tikzpicture} & \begin{tikzpicture}[scale=0.6,line width=1pt,line cap=round,line join=round,>=triangle 45,x=1.0cm,y=1.0cm]
			\clip(-3.42,0.58) rectangle (2.56,5.4);
			\draw (-1,5)-- (-3,1);
			\draw [dash pattern=on 2pt off 2pt] (-1,5)-- (0,2);
			\draw (-1,5)-- (2,1);
			\draw [dash pattern=on 2pt off 2pt] (-3,1)-- (0,2);
			\draw [dash pattern=on 2pt off 2pt] (0,2)-- (2,1);
			\draw (-3,1)-- (2,1);
			\fill [color=black] (-1,5) circle (1.5pt);
			\draw[color=black] (-1.3,5.2) node {$A$};
			\fill [color=black] (-3,1) circle (1.5pt);
			\draw[color=black] (-3.14,1.26) node {$B$};
			\fill [color=black] (0,2) circle (1.5pt);
			\draw[color=black] (-0.04,1.6) node {$C$};
			\fill [color=black] (2,1) circle (1.5pt);
			\draw[color=black] (2.16,1.28) node {$D$};
		\end{tikzpicture}& \begin{tikzpicture}[scale=0.6,line width=1pt,line cap=round,line join=round,>=triangle 45,x=1.0cm,y=1.0cm]
			\clip(-3.48,0.6) rectangle (2.4,5.4);
			\draw (-1,5)-- (-3,2);
			\draw (-1,5)-- (0,1);
			\draw (-1,5)-- (2,2);
			\draw (-3,2)-- (0,1);
			\draw (0,1)-- (2,2);
			\draw (-3,2)-- (2,2);
			\fill [color=black] (-1,5) circle (1.5pt);
			\draw[color=black] (-0.8,5.2) node {$A$};
			\fill [color=black] (-3,2) circle (1.5pt);
			\draw[color=black] (-3.1,2.3) node {$B$};
			\fill [color=black] (0,1) circle (1.5pt);
			\draw[color=black] (-0.3,0.8) node {$C$};
			\fill [color=black] (2,2) circle (1.5pt);
			\draw[color=black] (2.2,2.3) node {$D$};
		\end{tikzpicture} \\ 
		Hình $(1)$& Hình $(2)$ & Hình $(3)$ & Hình $(4)$
	\end{tabular} \\
	Trong các hình trên, những hình nào biểu diễn cho tứ diện?
	\choice{Hình (1) và hình (2)}
	{\True Hình (1), hình (2) và hình (3)}
	{Hình (1) và hình (3)}
	{Hình (1), hình (3) và hình (4)}
\end{ex}

\begin{ex}%[1H2B1-2]
\immini[thm]{Cho tứ diện $ABCD$. Gọi $M, N$ lần lượt là trung điểm của $AC,CD$. Giao tuyến của hai mặt phẳng $\left(MBD\right)$ và $\left(ABN\right)$ là
	\choice
	{\True đường thẳng $BG$ ($G$ là trọng tâm tam giác $ACD$)}
	{đường thẳng $AH$ ($H$ là trực tâm tam giác $ACD$)}
	{đường thẳng $MN $}
	{đường thẳng $AM $}
}{
	\begin{tikzpicture}[scale=0.6, line join=round, line cap=round]
	\tkzDefPoints{0/0/B,1.5/-1.8/C,4/0/D, 2/3/A}
	\tkzDefMidPoint(A,C)\tkzGetPoint{M}
	\tkzDefMidPoint(D,C)\tkzGetPoint{N}
	\tkzDrawPolygon(A,B,C,D)
	\tkzDrawSegments(A,C B,M D,M A,N)
	\tkzDrawSegments[dashed](B,D B,N)
	\tkzDrawPoints[fill=black](M,A,B,C,D,N)
	\tkzLabelPoints[above,font=\footnotesize](A)
	\tkzLabelPoints[below,font=\footnotesize](C,N)
	\tkzLabelPoints[left,font=\footnotesize](B)
	\tkzLabelPoints[right,font=\footnotesize](D)
	\tkzLabelPoints[above left,font=\footnotesize](M)
	\end{tikzpicture}}
	\loigiai{
		\immini{
			\begin{itemize}
				\item $B$ là điểm chung thứ nhất giữa hai mặt phẳng $\left(MBD\right)$ và $\left(ABN\right) $.
				\item Vì $M,N$ lần lượt là trung điểm của $AC,CD$ nên suy ra $AN,DM$ là hai trung tuyến của tam giác $ACD. $\\
				Gọi $G=AN\cap DM$
				$\Rightarrow \left\{\begin{aligned}& G\in AN\subset \left(ABN\right)\Rightarrow G\in \left(ABN\right) \\
				& G\in DM\subset \left(MBD\right)\Rightarrow G\in \left(MBD\right)
				\end{aligned}\right.$\\
				$\Rightarrow G$ là điểm chung thứ hai giữa hai mặt phẳng $\left( {MBD} \right)$ và $\left( {ABN} \right).$
			\end{itemize}
			Vậy $\left(ABN\right)\cap \left(MBD\right)=BG $.
		}
		{
			\begin{tikzpicture}[scale=1, line join=round, line cap=round]
			\tkzDefPoints{0/0/B,1.5/-1.8/C,4/0/D, 2/3/A}
			\tkzDefMidPoint(A,C)\tkzGetPoint{M}
			\tkzDefMidPoint(D,C)\tkzGetPoint{N}
			\tkzCentroid(D,A,C)\tkzGetPoint{G}
			\tkzDrawPolygon(A,B,C,D)
			\tkzDrawSegments(A,C B,M D,M A,N)
			\tkzDrawSegments[dashed](B,D B,G B,N)
			\tkzDrawPoints[fill=black](M,A,B,C,D,N,G)
			\tkzLabelPoints[above](A)
			\tkzLabelPoints[below](C,N)
			\tkzLabelPoints[left](B)
			\tkzLabelPoints[right](D)
			\tkzLabelPoints[above right](M,G)
			\end{tikzpicture}
		}
	}
\end{ex}

\begin{ex}%[1H2B1-2]
	\immini[thm]{Cho $4$ điểm không đồng phẳng $A,B,C,D$.  Gọi $I,K$ lần lượt là trung điểm của $AD$ và $BC$. Giao tuyến của $\left(IBC\right)$ và $\left(KAD\right)$ là
	\choice
	{\True $IK$}
	{$DK $}
	{$AK $}
	{$BC$}}{

\begin{tikzpicture}[scale=0.7, font=\footnotesize,>=stealth]
	\path
	%	Vẽ mp
	(0,0) coordinate (B)
	(5,0) coordinate (C)
	(1.5,-1.5) coordinate (D)
	(1,3.2) coordinate (A)
	($(A)!0.5!(D)$)coordinate (I)
	($(C)!0.5!(B)$)coordinate (K)
	;
	\draw (B)--(A)--(D)--(C)--(A) (B)--(D);
	\draw[dashed] (B)--(C);
	\foreach \x/\g in {B/180,A/90,C/0,D/-90,I/180,K/30}\draw[fill=black] (\x) circle (.05) +(\g:.5)node{\footnotesize$\x$};
\end{tikzpicture}}
	\loigiai{
		\immini{
			Điểm $K$ là trung điểm của $BC$ suy ra $K\in \left(IBC\right)\Rightarrow IK\subset \left(IBC\right) $.\\
			Điểm $I$ là trung điểm của $AD$ suy ra $I\in \left(KAD\right)\Rightarrow IK\subset \left(KAD\right) $.\\
			Vậy giao tuyến của hai mặt phẳng $\left(IBC\right)$ và $\left(KAD\right)$ là $IK$.
		}
		{
			\begin{tikzpicture}[scale=0.7, line join=round, line cap=round]
			\tkzDefPoints{0/0/B,1.5/-1.8/C,4/0/D, 2/2.5/A}
			\tkzDefMidPoint(C,B)\tkzGetPoint{K}
			\tkzDefMidPoint(A,D)\tkzGetPoint{I}
			\tkzDrawPolygon(A,B,C,D)
			\tkzDrawSegments(A,K C,I)
			\tkzDrawSegments[dashed](B,D B,I K,I K,D)
			\tkzDrawPoints[fill=black](K,A,B,C,D,I)
			\tkzLabelPoints[above](A)
			\tkzLabelPoints[above right](I)
			\tkzLabelPoints[below](C)
			\tkzLabelPoints[below left](K)
			\tkzLabelPoints[left](B)
			\tkzLabelPoints[right](D)
			\end{tikzpicture}
		}
	}
\end{ex}

\begin{ex}%[1H2B1-2]
	\immini[thm]{Cho tứ diện $ABCD$. Gọi $G$ là trọng tâm của tam giác $BCD$. Giao tuyến của mặt phẳng $\left(ACD\right)$ và $\left(GAB\right)$ là
	\choice
	{$AH$ ($H$ là hình chiếu của $B$ trên $CD) $}
	{$AM$ ($M$ là trung điểm của $AB) $}
	{$AK$ ($K$ là hình chiếu của $C$ trên $BD) $}
	{\True $AN$ ($N$ là trung điểm của $CD) $}
}{
	\begin{tikzpicture}[scale=0.7, line join=round, line cap=round]
	\tkzDefPoints{0/0/B,0.5/-1.5/C,4/0/D, 1.5/3/A}
	\tkzCentroid(D,B,C)\tkzGetPoint{G}
	\tkzDrawPolygon(A,B,C,D)
	\tkzDrawSegments(A,C)
	\tkzDrawSegments[dashed](B,D A,G B,G)
	\tkzDrawPoints[fill=black](G,A,B,C,D)
	\tkzLabelPoints[above](A)
	\tkzLabelPoints[below](C,G)
	\tkzLabelPoints[left](B)
	\tkzLabelPoints[right](D)
	\end{tikzpicture}}
	\loigiai{
		\immini{
			\begin{itemize}
				\item $A$ là điểm chung thứ nhất giữa hai mặt phẳng $\left(ACD\right)$ và $\left(GAB\right) $.
				\item Ta có $BG\cap CD=N$\\
				$ \Rightarrow \left\{\begin{aligned}& N\in BG\subset \left(ABG\right)\Rightarrow N\in \left(ABG\right) \\
				& N\in CD\subset \left(ACD\right)\Rightarrow N\in \left(ACD\right).
				\end{aligned}\right.$\\
				$\Rightarrow N$ là điểm chung thứ hai giữa hai mặt phẳng $\left(ACD\right)$ và $\left(GAB\right) $.
			\end{itemize}
			Vậy $\left( {ABG} \right) \cap \left( {ACD} \right) = AN.$
		}
		{
			\begin{tikzpicture}[scale=0.8, line join=round, line cap=round]
			\tkzDefPoints{0/0/B,0.5/-1.5/C,4/0/D, 1.5/3/A}
			\tkzDefMidPoint(D,C)\tkzGetPoint{N}
			\tkzCentroid(D,B,C)\tkzGetPoint{G}
			\tkzDrawPolygon(A,B,C,D)
			\tkzDrawSegments(A,C A,N)
			\tkzDrawSegments[dashed](B,D A,G B,N)
			\tkzDrawPoints[fill=black](G,A,B,C,D,N)
			\tkzLabelPoints[above](A)
			\tkzLabelPoints[below](C,G)
			\tkzLabelPoints[left](B)
			\tkzLabelPoints[right](D)
			\end{tikzpicture}
		}
	}
\end{ex}

\begin{ex}%[1H2B1-2]
\immini[thm]{Cho hình chóp $S.ABCD$ có đáy là hình thang $ABCD\left(AD\parallel BC\right) $. Gọi $M$ là trung điểm $CD $. Giao tuyến của hai mặt phẳng $\left(MSB\right)$ và $\left(SAC\right)$ là
	\choice
	{$SJ$ ($J$ là giao điểm của $AM$ và $BD$)}
	{\True $SI$ ($I$ là giao điểm của $AC$ và $BM$)}
	{$SO$ ($O$ là giao điểm của $AC$ và $BD$)}
	{$SP$ ($P$ là giao điểm của $AB$ và $CD$)}
}{
	\begin{tikzpicture}[scale=0.8, line join=round, line cap=round]
	\tkzDefPoints{0/0/A,0.5/-1.5/B,3/-1.5/C,4/0/D, 1/2/S}
	\tkzDefMidPoint(C,D)\tkzGetPoint{M}
	\tkzDrawPolygon(S,A,B,C,D)
	\tkzDrawSegments(S,C S,B S,M)
	\tkzDrawSegments[dashed](A,D A,C B,M)
	\tkzDrawPoints[fill=black](M,A,B,C,D,S)
	\tkzLabelPoints[above](S)
	\tkzLabelPoints[below](B,C)
	\tkzLabelPoints[left](A)
	\tkzLabelPoints[right](D,M)
	\end{tikzpicture}}
	\loigiai{
		\immini{
			\begin{itemize}
				\item  $S$ là điểm chung thứ nhất giữa hai mặt phẳng $\left(MSB\right)$ và $\left(SAC\right) $.\\
				\item  Ta có $\left\{\begin{aligned}& I\in BM\subset \left(SBM\right)\Rightarrow I\in \left(SBM\right) \\
				& I\in \left(AC\right)\in \left(SAC\right)\Rightarrow I\in \left(SAC\right)
				\end{aligned}\right.$\\
				$\Rightarrow I$ là điểm chung thứ hai giữa hai mặt phẳng $\left(SAC \right)$ và $\left(SAC\right).$
			\end{itemize}
			Vậy $\left(MSB\right)\cap \left(SAC\right)=SI $.
		}
		{
			\begin{tikzpicture}[scale=1, line join=round, line cap=round]
			\tkzDefPoints{0/0/A,0.5/-1.5/B,3/-1.5/C,4/0/D, 1/3/S}
			\tkzDefMidPoint(C,D)\tkzGetPoint{M}
			\tkzInterLL(A,C)(B,M)\tkzGetPoint{I}
			\tkzDrawPolygon(S,A,B,C,D)
			\tkzDrawSegments(S,C S,B S,M)
			\tkzDrawSegments[dashed](S,I A,D A,C B,M)
			\tkzDrawPoints[fill=black](M,A,B,C,D,I,S)
			\tkzLabelPoints[above](S)
			\tkzLabelPoints[above right](I)
			\tkzLabelPoints[below](B,C)
			\tkzLabelPoints[left](A)
			\tkzLabelPoints[right](D,M)
			\end{tikzpicture}
		}
	}
\end{ex}

\begin{ex}%[1H2B1-2]
	Cho hình chóp $S.ABCD$ có đáy là hình thang $ABCD\left(AB\parallel CD\right) $. Khẳng định nào sau đây \textbf{sai}?
	\choice
	{Hình chóp $S.ABCD$ có $4$ mặt bên}
	{Giao tuyến của hai mặt phẳng $\left(SAC\right)$ và $\left(SBD\right)$ là $SO$ ($O$ là giao điểm của $AC$ và $BD) $}
	{\True Giao tuyến của hai mặt phẳng $\left(SAB\right)$ và $\left(SAD\right)$ là đường trung bình của $ABCD$
	}
	{Giao tuyến của hai mặt phẳng $\left(SAD\right)$ và $\left(SBC\right)$ là $SI$ ($I$ là giao điểm của $AD$ và $BC) $}
	\loigiai{
		\immini{
			\begin{itemize}
				\item Hình chóp $S.ABCD$ có 4 mặt bên: $\left(SAB\right),\left(SBC\right),\left(SCD\right),\left(SAD\right) $.
				\item là điểm chung thứ nhất của hai mặt phẳng $\left(SAC\right)$ và $\left(SBD\right) $.
				$\left\{\begin{aligned}& O\in AC\subset \left(SAC\right)\Rightarrow O\in \left(SAC\right) \\
				& O\in BD\subset \left(SBD\right)\Rightarrow O\in \left(SBD\right)
				\end{aligned}\right.\Rightarrow O$ là điểm chung thứ hai của hai mặt phẳng $\left(SAC\right)$ và $\left(SBD\right).$\\
				$\Rightarrow \left(SAC\right)\cap \left(SBD\right)=SO $.
				\item Tương tự, ta có $\left(SAD\right)\cap \left(SBC\right)=SI $.
				\item $\left(SAB\right)\cap \left(SAD\right)=SA$ mà $SA$ không phải là đường trung bình của hình thang $ABCD $.
			\end{itemize}
			Vậy \lq\lq Giao tuyến của hai mặt phẳng $\left(SAB\right)$ và $\left(SAD\right)$ là đường trung bình của $ABCD$\rq\rq\ là mệnh đề sai.
		}
		{
			\begin{tikzpicture}[scale=1, line join=round, line cap=round]
			\tkzDefPoints{0/0/A,0.5/-1.5/D,4/0/B,2.4/-1.5/C, 1.5/3/S}
			\tkzInterLL(A,C)(B,D)\tkzGetPoint{O}
			\tkzInterLL(A,D)(B,C)\tkzGetPoint{I}
			\tkzDrawPolygon(S,A,I,B)
			\tkzDrawSegments(S,D S,C S,I)
			\tkzDrawSegments[dashed](S,O A,B D,C A,C B,D)
			\tkzDrawPoints[fill=black](S,A,B,C,D,O,I)
			\tkzLabelPoints[above](S)
			\tkzLabelPoints[below](O,I)
			\tkzLabelPoints[left](A,D)
			\tkzLabelPoints[right](C,B)
			\end{tikzpicture}
		}
	}
\end{ex}

\begin{ex}%[1H2B1-2]
\immini[thm]{Cho hình chóp $S.ABCD$ có đáy $ABCD$ là hình bình hành. Gọi $M, N$ lần lượt là trung điểm $AD$ và $BC$. Giao tuyến của hai mặt phẳng $\left(SMN\right)$ và $\left(SAC\right)$ là
	\choice
	{$SG$ ($G$ là trung điểm $AB$)}
	{$SD$}
	{\True $SO$ ($O$ là tâm hình bình hành $ABCD$)}
	{$SF$ ($F$ là trung điểm $CD$)}
}{
	\begin{tikzpicture}[scale=0.6, line join=round, line cap=round]
	\tkzDefPoints{0/0/A,-1.5/-1.8/B,4/0/D, 0.2/3/S}
	\coordinate (C) at ($(B)+(D)-(A)$);
	\tkzDefMidPoint(A,D)\tkzGetPoint{M}
	\tkzDefMidPoint(B,C)\tkzGetPoint{N}
	\tkzDrawPolygon(S,B,C,D)
	\tkzDrawSegments(S,C S,D S,N)
	\tkzDrawSegments[dashed](S,A A,B A,C M,N A,D S,M)
	\tkzDrawPoints[fill=black](M,A,B,C,D,N,S)
	\tkzLabelPoints[above](S)
	\tkzLabelPoints[above right](M)
	\tkzLabelPoints[below](B,C,N)
	\tkzLabelPoints[left](A)
	\tkzLabelPoints[right](D)
	\end{tikzpicture}}
	\loigiai{
		\immini{
			\begin{itemize}
				\item $S$ là điểm chung thứ nhất giữa hai mặt phẳng $\left(SMN\right)$ và $\left(SAC\right) $.
				\item Gọi $O=AC\cap BD$ là tâm của hình hình hành.
			\end{itemize}
			Trong mặt phẳng $\left(ABCD\right)$, gọi $T=AC\cap MN$\\
			$\Rightarrow \left\{\begin{aligned}& O\in AC\subset \left(SAC\right)\Rightarrow O\in \left(SAC\right) \\
			& O\in MN\subset \left(SMN\right)\Rightarrow O\in \left(SMN\right)
			\end{aligned}\right.$\\
			$\Rightarrow O$ là điểm chung thứ hai giữa hai mặt phẳng $\left(SMN\right)$ và $\left( {SAC} \right).$\\
			Vậy $\left(SMN\right)\cap \left(SAC\right)=SO $.
		}
		{
			\begin{tikzpicture}[scale=1, line join=round, line cap=round]
			\tkzDefPoints{0/0/A,-1.5/-1.8/B,4/0/D, 0.2/3/S}
			\coordinate (C) at ($(B)+(D)-(A)$);
			\tkzDefMidPoint(A,D)\tkzGetPoint{M}
			\tkzDefMidPoint(B,C)\tkzGetPoint{N}
			\tkzInterLL(A,C)(M,N)\tkzGetPoint{O}
			\tkzDrawPolygon(S,B,C,D)
			\tkzDrawSegments(S,C S,D S,N)
			\tkzDrawSegments[dashed](S,A A,B A,C M,N A,D S,O S,M)
			\tkzDrawPoints[fill=black](M,A,B,C,D,N,O,S)
			\tkzLabelPoints[above](S)
			\tkzLabelPoints[above right](M)
			\tkzLabelPoints[below](B,C,N,O)
			\tkzLabelPoints[left](A)
			\tkzLabelPoints[right](D)
			\end{tikzpicture}
		}
	}
\end{ex}

\begin{ex}%[1H2B1-2]
	\immini[thm]{Cho điểm $A$ không nằm trên mặt phẳng $\left(\alpha \right)$ chứa tam giác $BCD. $ Lấy $E,F$ là các điểm lần lượt nằm trên các cạnh $AB, AC$. Khi $EF$ và $BC$ cắt nhau tại $I$ thì $I$ không phải là điểm chung của hai mặt phẳng nào sau đây?
	\choice
	{$\left(BCD\right)$ và $\left(ABC\right) $}
	{\True $\left(BCD\right)$ và $\left(ABD\right) $}
	{$\left(BCD\right)$ và $\left(AEF\right) $}
	{$\left(BCD\right)$ và $\left(DEF\right) $}
}{
	\begin{tikzpicture}[scale=0.7, line join=round, line cap=round]
	\tkzDefPoints{0/0/B,1.5/-1.8/C,4/0/D, 2/3/A}
	\coordinate (E) at ($(A)!0.4!(B)$);
	\coordinate (F) at ($(A)!0.7!(C)$);
	\tkzDrawPolygon(A,B,C)
	\tkzDrawSegments(A,D E,F F,D C,D)
	\tkzDrawSegments[dashed](B,D E,D)
	\tkzDrawPoints[fill=black](E,A,B,C,D,F)
	\tkzLabelPoints[above](A)
	\tkzLabelPoints[below left](C)
	\tkzLabelPoints[left](B,E,F)
	\tkzLabelPoints[right](D)
	\end{tikzpicture}}
	\loigiai{
		\immini{
			Điểm $I$ là giao điểm của $EF$ và $BC$,\\
			mà $\left\{\begin{aligned}& EF\subset \left(DEF\right) \\
			& EF\subset \left(ABC\right) \\
			& EF\subset \left(AEF\right)
			\end{aligned}\right.\Rightarrow \left\{\begin{aligned}& I=\left(BCD\right)\cap \left(DEF\right) \\
			& I=\left(BCD\right)\cap \left(ABC\right) \\
			& I=\left(BCD\right)\cap \left(AEF\right)
			\end{aligned}\right. $.
		}
		{
			\begin{tikzpicture}[scale=0.8, line join=round, line cap=round]
			\tkzDefPoints{0/0/B,1.5/-1.8/C,4/0/D, 2/3/A}
			\coordinate (E) at ($(A)!0.4!(B)$);
			\coordinate (F) at ($(A)!0.7!(C)$);
			\tkzInterLL(E,F)(C,B)\tkzGetPoint{I}
			\tkzInterLL(E,F)(C,D)\tkzGetPoint{K}
			\tkzDrawPolygon(A,B,C)
			\tkzDrawSegments(A,D C,I E,I F,D K,D)
			\tkzDrawSegments[dashed](B,D E,D C,K)
			\tkzDrawPoints[fill=black](E,A,B,C,D,F)
			\tkzLabelPoints[above](A)
			\tkzLabelPoints[below](I)
			\tkzLabelPoints[below left](C)
			\tkzLabelPoints[left](B,E,F)
			\tkzLabelPoints[right](D)
			\end{tikzpicture}
		}
	}
\end{ex}

\begin{ex}%[1H2K1-5]
	\immini[thm]{Cho tứ diện $ABCD$. Gọi $G$ là trọng tâm tam giác $BCD$, $M$ là trung điểm $CD$, $I$ là điểm ở trên đoạn thẳng $AG,BI$ cắt mặt phẳng $\left(ACD\right)$ tại $J $. Khẳng định nào sau đây \textbf{sai}?
	\choice
	{\True $J$ là trung điểm của $AM $}
	{$AM=\left(ACD\right)\cap \left(ABG\right) $}
	{$A,J,M$ thẳng hàng}
	{$DJ=\left(ACD\right)\cap \left(BDJ\right) $}}{
\begin{tikzpicture}[scale=.8, line join=round, line cap=round]
	\tkzDefPoints{0/0/B,1.5/-1.5/C,4/0/D, 1.5/3/A}
	\tkzCentroid(D,B,C)\tkzGetPoint{G}
	\coordinate (I) at ($(A)!0.6!(G)$);
	\tkzDefMidPoint(D,C)\tkzGetPoint{M}
	\tkzInterLL(B,I)(M,A)\tkzGetPoint{J}
	\tkzDrawPolygon(A,B,C,D)
	\tkzDrawSegments(A,C A,M)
	\tkzDrawSegments[dashed](B,D B,M B,J A,G)
	\tkzDrawPoints[fill=black](A,B,C,M,D,G,I,J)
	\tkzLabelPoints[above](A)
	\tkzLabelPoints[above right](J)
	\tkzLabelPoints[below](C,G)
	\tkzLabelPoints[below right](M,I,D)
	\tkzLabelPoints[left](B)
\end{tikzpicture}}
	\loigiai{
		\immini{
			Ta có $A$ là điểm chung thứ nhất giữa hai mặt phẳng $\left(ACD\right)$ và $\left(GAB\right) $. \\
			Do $BG\cap CD=M\Rightarrow \left\{\begin{aligned}& M\in BG\subset \left(ABG\right)\Rightarrow M\in \left(ABG\right) \\
			& M\in CD\subset \left(ACD\right)\Rightarrow M\in \left(ACD\right)
			\end{aligned}\right.$\\
			$\Rightarrow M$ là điểm chung thứ hai giữa hai mặt phẳng $(ABG)$ và $(ACD)$\\
			$\Rightarrow \left(ABG\right)\cap \left(ACD\right)=AM$.\\
			Ta có $\left\{\begin{aligned}& BI\subset \left(ABG\right) \\
			& AM\subset \left(ABM\right) \\
			& \left(ABG\right)\equiv \left(ABM\right)
			\end{aligned}\right.\Rightarrow AM,BI$ đồng phẳng.\\
			$\Rightarrow J=BI\cap AM\Rightarrow A,J,M$ thẳng hàng.
		}
		{
			\begin{tikzpicture}[scale=1, line join=round, line cap=round]
			\tkzDefPoints{0/0/B,1.5/-1.5/C,4/0/D, 1.5/3/A}
			\tkzCentroid(D,B,C)\tkzGetPoint{G}
			\coordinate (I) at ($(A)!0.6!(G)$);
			\tkzDefMidPoint(D,C)\tkzGetPoint{M}
			\tkzInterLL(B,I)(M,A)\tkzGetPoint{J}
			\tkzDrawPolygon(A,B,C,D)
			\tkzDrawSegments(A,C A,M)
			\tkzDrawSegments[dashed](B,D B,M B,J A,G)
			\tkzDrawPoints[fill=black](A,B,C,M,D,G,I,J)
			\tkzLabelPoints[above](A)
			\tkzLabelPoints[above right](J)
			\tkzLabelPoints[below](C,G)
			\tkzLabelPoints[below right](M,I)
			\tkzLabelPoints[left](B)
			\end{tikzpicture}
		}
		Ta có $\left\{\begin{aligned}& DJ\subset \left(ACD\right) \\
		& DJ\subset \left(BDJ\right)
		\end{aligned}\right.\Rightarrow DJ=\left(ACD\right)\cap \left(BDJ\right)$.
		Điểm $I$ di động trên $AG$ nên $J$ có thể không phải là trung điểm của $AM$.
	}
\end{ex}

\begin{ex}%[1H2K1-3]
	\immini[thm]{Cho tứ diện $ABCD$. Gọi $E$ và $F$ lần lượt là trung điểm của $AB$ và $CD$; $G$ là trọng tâm tam giác $BCD$. Giao điểm của đường thẳng $EG$ và mặt phẳng $\left(ACD\right)$ là
	\choice
	{Giao điểm của đường thẳng $EG$ và $CD$}
	{Giao điểm của đường thẳng $EG$ và $AC$}
	{\True Giao điểm của đường thẳng $EG$ và $AF $}
	{Điểm $F $}}{
\begin{tikzpicture}[scale=.8, line join=round, line cap=round]
	\tkzDefPoints{0/0/B,2.6/-1.8/C,4/0/D, 2/3/A}
	\tkzDefMidPoint(A,B)\tkzGetPoint{E}
	\tkzDefMidPoint(D,C)\tkzGetPoint{F}
	\tkzCentroid(D,B,C)\tkzGetPoint{G}
	\tkzDrawPolygon(A,B,C,D)
	\tkzDrawSegments(A,C A,F)
	\tkzDrawSegments[dashed](B,D E,G B,F)
	\tkzDrawPoints[fill=black](A,B,C,D,E,F,G)
	\tkzLabelPoints[above](A)
	\tkzLabelPoints[below](C)
	\tkzLabelPoints[below left](G)
	\tkzLabelPoints[left](B,E)
	\tkzLabelPoints[right](D,F)
\end{tikzpicture}}
	\loigiai{
		\immini{
			Vì $G$ là trọng tâm tam giác $BCD$, $F$ là trung điểm của $CD$\\
			$\Rightarrow G\in \left(ABF\right) $.\\
			Ta có $E$ là trung điểm của $AB$\\
			$\Rightarrow E\in \left(ABF\right) $.\\
			Gọi $M$ là giao điểm của $EG$ và $AF$ mà $AF\subset \left(ACD\right)$ suy ra $M\in \left(ACD\right) $.\\
			Vậy giao điểm của $EG$ và $\left(ACD\right)$ là $M=EG\cap AF $.
		}
		{
			\begin{tikzpicture}[scale=1, line join=round, line cap=round]
			\tkzDefPoints{0/0/B,2.6/-1.8/C,4/0/D, 2/3/A}
			\tkzDefMidPoint(A,B)\tkzGetPoint{E}
			\tkzDefMidPoint(D,C)\tkzGetPoint{F}
			\tkzCentroid(D,B,C)\tkzGetPoint{G}
			\tkzInterLL(E,G)(C,D)\tkzGetPoint{I}
			\tkzInterLL(A,F)(E,G)\tkzGetPoint{M}
			\tkzDrawPolygon(A,B,C,D)
			\tkzDrawSegments(A,C A,M M,I)
			\tkzDrawSegments[dashed](B,D E,I B,F)
			\tkzDrawPoints[fill=black](M,A,B,C,D,E,F,G)
			\tkzLabelPoints[above](A)
			\tkzLabelPoints[below](C,M)
			\tkzLabelPoints[below left](G)
			\tkzLabelPoints[left](B,E)
			\tkzLabelPoints[right](D,F)
			\end{tikzpicture}
		}
	}
\end{ex}

\begin{ex}%[1H2K1-3]
	\immini[thm]{Cho tứ giác $ABCD$ có $AC$ và $BD$ giao nhau tại $O$ và một điểm $S$ không thuộc mặt phẳng $\left(ABCD\right)$. Trên đoạn $SC$ lấy một điểm $M$ không trùng với $S$ và $C$. Giao điểm của đường thẳng $SD$ với mặt phẳng $\left(ABM\right)$ là
	\choice
	{\True Giao điểm của $SD$ và $BK$ (với $K=SO\cap AM$)}
	{Giao điểm của $SD$ và $AB$}
	{Giao điểm của $SD$ và $MK$ (với $K=SO\cap AM$)}
	{Giao điểm của $SD$ và $AM$}}{
\begin{tikzpicture}[scale=.8, line join=round, line cap=round]
	\tkzDefPoints{0/0/A,0.5/-1.7/B,4/0/D, 1.2/3/S,3/-2/C}
	\coordinate (M) at ($(S)!0.45!(C)$);
	\tkzInterLL(A,C)(B,D)\tkzGetPoint{O}
	\tkzDrawPolygon(S,A,B,C,D)
	\tkzDrawSegments(S,C S,B B,M)
	\tkzDrawSegments[dashed](A,D A,C B,D A,M)
	\tkzDrawPoints[fill=black](M,A,B,C,D,O,S)
	\tkzLabelPoints[above](S)
	\tkzLabelPoints[above right](M)
	\tkzLabelPoints[below](B,C,O)
	\tkzLabelPoints[left](A)
\tkzLabelPoints[right](D)
\end{tikzpicture}}
	\loigiai{
		\immini{
			\begin{itemize}
				\item Chọn mặt phẳng phụ $\left(SBD\right)$ chứa $SD$.
				\item Tìm giao tuyến của hai mặt phẳng $\left(SBD\right)$ và $\left(ABM\right)$.
				Ta có $B$ là điểm chung thứ nhất của $\left(SBD\right)$ và $\left(ABM\right)$.
				\item Trong mặt phẳng $\left(ABCD\right)$, gọi $O=AC\cap BD$. Trong mặt phẳng $\left(SAC\right)$, gọi $K=AM\cap SO$. Ta có:
				\begin{itemize}
					\item $K\in SO$ mà $SO \subset (SBD)$ suy ra $K\in (SBD)$.
					\item $K\in AM$ mà $AM \subset (ABM)$ suy ra $K \in (AMB)$.
				\end{itemize}
				Suy ra $K$ là điểm chung thứ hai của $BCD$ và $\left(MNP\right)$.
				Do đó $ (SBD)\cap (ABM)=BK$.
				\item Trong mặt phẳng $(SBD)$, gọi $N=SD\cap BK$. Ta có: $N\in BK$, mà $BK\cap (ABM)$ suy ra $N\cap (ABM)$. Mặt khác $N\in SD$.
			\end{itemize}
			Vậy $N=SD\cap (ABM)$.
		}
		{
			\begin{tikzpicture}[scale=1, line join=round, line cap=round]
			\tkzDefPoints{0/0/A,0.5/-1.7/B,4/0/D, 1.2/3/S,3/-2/C}
			\coordinate (M) at ($(S)!0.45!(C)$);
			\tkzInterLL(A,C)(B,D)\tkzGetPoint{O}
			\tkzInterLL(A,M)(S,O)\tkzGetPoint{K}
			\tkzInterLL(S,D)(B,K)\tkzGetPoint{N}
			\tkzDrawPolygon(S,A,B,C,D)
			\tkzDrawSegments(S,C S,B)
			\tkzDrawSegments[dashed](A,D A,C B,D S,O A,M B,N)
			\tkzDrawPoints[fill=black](M,A,B,C,D,O,K,S,N)
			\tkzLabelPoints[above](S)
			\tkzLabelPoints[above right](M,N)
			\tkzLabelPoints[below](B,C,O)
			\tkzLabelPoints[left](A)
			\tkzLabelPoints[above left](K)
			\tkzLabelPoints[right](D)
			\end{tikzpicture}
		}
	}
\end{ex}


\begin{ex}%[1H2K1-5]
	\immini[thm]{Cho tứ diện $ABCD$. Gọi $M,N$ lần lượt là trung điểm của $AB$ và $CD$. Mặt phẳng $\left(\alpha \right)$ qua $MN$ cắt $AD, BC$ lần lượt tại $P$ và $Q$. Biết $MP$ cắt $NQ$ tại $I $. Ba điểm nào sau đây thẳng hàng?
	\choice
	{\True $I,B,D$}
	{$I,A,C $}
	{$I,C,D$ }
	{$I,A,B$}}{
\begin{tikzpicture}[scale=.8, line join=round, line cap=round]
	\tkzDefPoints{0/0/B,3/-1.8/C,4/0/D, 1.5/3/A}
	\tkzDefMidPoint(A,B)\tkzGetPoint{M}
	\tkzDefMidPoint(D,C)\tkzGetPoint{N}
	\coordinate (P) at ($(A)!0.7!(D)$);
	\tkzInterLL(M,P)(B,D)\tkzGetPoint{I}
	\tkzInterLL(B,C)(I,N)\tkzGetPoint{Q}
	\tkzDrawPolygon(A,B,C)
	\tkzDrawSegments(A,P C,N N,P M,Q)
	\tkzDrawSegments[dashed](B,D M,P N,Q N,D P,D)
	\tkzDrawPoints[fill=black](M,A,B,C,D,N,P,Q)
	\tkzLabelPoints[above](A)
	\tkzLabelPoints[above right](P,D)
	\tkzLabelPoints[below](C,Q)
	\tkzLabelPoints[below right](N)
	\tkzLabelPoints[left](B,M)
\end{tikzpicture}}
	\loigiai{
		\immini{
			Ta có $\left(ABD\right)\cap \left(BCD\right)=BD$.
			Lại có $\left\{\begin{aligned}& I\in MP\subset \left(ABD\right) \\
				& I\in NQ\subset \left(BCD\right)
			\end{aligned}\right.$\\
			$\Rightarrow I$ thuộc giao tuyến của $(ABC)$ và $\left( {BCD} \right)$ \\
			$\Rightarrow I\in BD\Rightarrow I,B,D$ thẳng hàng.
		}
		{
			\begin{tikzpicture}[scale=1, line join=round, line cap=round]
				\tkzDefPoints{0/0/B,3/-1.8/C,4/0/D, 1.5/3/A}
				\tkzDefMidPoint(A,B)\tkzGetPoint{M}
				\tkzDefMidPoint(D,C)\tkzGetPoint{N}
				\coordinate (P) at ($(A)!0.7!(D)$);
				\tkzInterLL(M,P)(B,D)\tkzGetPoint{I}
				\tkzInterLL(B,C)(I,N)\tkzGetPoint{Q}
				\tkzDrawPolygon(A,B,C)
				\tkzDrawSegments(A,P C,N N,P N,I P,I M,Q)
				\tkzDrawSegments[dashed](B,D M,P N,Q N,D P,D I,D M,N)
				\tkzDrawPoints[fill=black](M,A,B,C,D,N,P,Q,I)
				\tkzLabelPoints[above](A)
				\tkzLabelPoints[above right](P,D)
				\tkzLabelPoints[below](C,Q)
				\tkzLabelPoints[below right](N)
				\tkzLabelPoints[left](B,M)
				\tkzLabelPoints[right](I)
			\end{tikzpicture}
		}
	}
\end{ex}
\begin{ex}%[1H2K1-5]
	\immini[thm]{Cho tứ diện $SABC$. Gọi $L,M,N$ lần lượt là các điểm trên các cạnh $SA,SB$ và $AC$ sao cho $LM$ không song song với $AB, LN$ không song song với $SC$. Mặt phẳng $\left(LMN\right)$ cắt các cạnh $AB,BC,SC$ lần lượt tại $K,I,J$. Ba điểm nào sau đây thẳng hàng?
	\choice
	{$M,K,J $}
	{$N,I,J $}
	{$K,I,J $}
	{\True $M,I,J $}}{
\begin{tikzpicture}[scale=.8, line join=round, line cap=round]
	\tkzDefPoints{0/0/A,1.5/-1.5/B,4/0/C, 1.5/3/S}
	\coordinate (L) at ($(S)!0.4!(A)$);
	\coordinate (M) at ($(S)!0.75!(B)$);
	\coordinate (N) at ($(A)!0.8!(C)$);
	\tkzInterLL(M,L)(B,A)\tkzGetPoint{K}
	\tkzInterLL(S,C)(L,N)\tkzGetPoint{J}
	\tkzInterLL(B,C)(L,M)\tkzGetPoint{E}
	\tkzInterLL(B,C)(K,N)\tkzGetPoint{I}
	\tkzInterLL(B,C)(L,N)\tkzGetPoint{H}
	\tkzDrawPolygon(S,A,B)
	\tkzDrawSegments(L,K B,K K,I E,C S,J H,J)
	\tkzDrawSegments[dashed](A,C B,E M,N L,N I,N N,H)
	\tkzDrawPoints[fill=black](S,A,B,C,M,N,L,I,J,K)
	\tkzLabelPoints[above](S,N)
	\tkzLabelPoints[above right](C)
	\tkzLabelPoints[below](K,B,J)
	\tkzLabelPoints[below right](I)
	\tkzLabelPoints[left](A,M,L)
\end{tikzpicture}}
	\loigiai{
		\immini{
			Ta có
			\begin{itemize}
				\item $M\in SB$ suy $M$ là điểm chung của $\left(LMN\right)$ và $\left(SBC\right)$.
				\item $I$ là điểm chung của $\left(LMN\right)$ và $\left(SBC\right)$.
				\item $J$ là điểm chung của $\left(LMN\right)$ và $\left(SBC\right)$.
			\end{itemize}
			Vậy $M,I,J$ thẳng hàng vì cùng thuộc giao tuyến của $\left(LMN\right)$ và $\left(SBC\right)$.
		}
		{
			\begin{tikzpicture}[scale=1, line join=round, line cap=round]
				\tkzDefPoints{0/0/A,1.5/-1.5/B,4/0/C, 1.5/3/S}
				\coordinate (L) at ($(S)!0.4!(A)$);
				\coordinate (M) at ($(S)!0.75!(B)$);
				\coordinate (N) at ($(A)!0.8!(C)$);
				\tkzInterLL(M,L)(B,A)\tkzGetPoint{K}
				\tkzInterLL(S,C)(L,N)\tkzGetPoint{J}
				\tkzInterLL(B,C)(L,M)\tkzGetPoint{E}
				\tkzInterLL(B,C)(K,N)\tkzGetPoint{I}
				\tkzInterLL(B,C)(L,N)\tkzGetPoint{H}
				\tkzDrawPolygon(S,A,B)
				\tkzDrawSegments(L,K B,K K,I E,C S,J H,J)
				\tkzDrawSegments[dashed](A,C B,E M,N L,N I,N N,H)
				\tkzDrawPoints[fill=black](S,A,B,C,M,N,L,I,J,K)
				\tkzLabelPoints[above](S,N)
				\tkzLabelPoints[above right](C)
				\tkzLabelPoints[below](K,B,J)
				\tkzLabelPoints[below right](I)
				\tkzLabelPoints[left](A,M,L)
			\end{tikzpicture}
		}
	}
\end{ex}

\Closesolutionfile{ans}


%%Bài 11. Hai đt song song
% \setcounter{section}{10} \setcounter{dang}{0}
\newpage
\section{HAI ĐƯỜNG THẲNG SONG SONG}
\subsection{KIẾN THỨC CẦN NHỚ}
\subsubsection{VỊ TRÍ TƯƠNG ĐỐI CỦA HAI ĐƯỜNG THẲNG}
Trong không gian, cho hai đường thẳng $a$ và $b$.
\begin{enumerate}[\iconMT]
	\item \indam{Các trường hợp có thể xảy ra:}
	\begin{itemize}
		\item [$\bullet$] Nếu $a$ và $b$ đồng phẳng (cùng thuộc một mặt phẳng) thì chúng có các khả năng: cắt nhau; song song nhau hoặc trùng nhau.
		\item [$\bullet$] Nếu $a$ và $b$ không đồng phẳng (không tồn tại mặt phẳng chưa được cả $a$ và $b$) thì ta nói $a$ và $b$ chéo nhau.
	\end{itemize}
	\begin{tabular}{cccc}
		\begin{tikzpicture}[scale=0.5,font=\small]
			\tkzDefPoints{0/0/A, 5/0/B, 6/3/C}
			\coordinate (D) at ($(A)+(C)-(B)$);
			\tkzDrawPolygon(A,B,C,D)
			\tkzMarkAngle[size=.85](B,A,D)
			\draw (A) node[above right]{$\alpha$};
			\tkzDefPoints{1/2/E, 4.5/0.5/F, 0.8/1.5/G, 5/2.5/H}
			\draw (E)--(F) (G)--(H);
			\tkzInterLL(E,F)(G,H) \tkzGetPoint{M}
			\tkzDrawPoints[size=5,fill=black](M)
			\tkzLabelPoints[above](M)
			\draw (F) node[above]{$a$};
			\draw (H) node[below]{$b$};
		\end{tikzpicture}
		&\begin{tikzpicture}[scale=0.5,font=\small]
			\tkzDefPoints{0/0/A, 5/0/B, 6/3/C}
			\coordinate (D) at ($(A)+(C)-(B)$);
			\tkzDrawPolygon(A,B,C,D)
			\tkzMarkAngle[size=.85](B,A,D)
			\draw (A) node[above right]{$\alpha$};
			\tkzDefPoints{0.8/1.5/G, 5/2.5/H, 1/0.5/I}
			\coordinate (K) at ($(H)+(I)-(G)$);
			\draw (G)--(H) (I)--(K);
			\draw ($(G)!0.8!(H)$) node[above]{$a$};
			\draw ($(I)!0.8!(K)$) node[above]{$b$};
		\end{tikzpicture}
		&\begin{tikzpicture}[scale=0.5,font=\small]
			\tkzDefPoints{0/0/A, 5/0/B, 6/3/C}
			\coordinate (D) at ($(A)+(C)-(B)$);
			\tkzDrawPolygon(A,B,C,D)
			\tkzMarkAngle[size=.85](B,A,D)
			\draw (A) node[above right]{$\alpha$};
			\tkzDefPoints{0.5/0.8/G, 5/2.5/H}
			\draw (G)--(H);
			\draw ($(G)!0.2!(H)$) node[above]{$a$};
			\draw ($(G)!0.2!(H)$) node[below]{$b$};
		\end{tikzpicture}
		&\begin{tikzpicture}[scale=0.5,font=\small]
			\tkzDefPoints{0/0/A, 5/0/B, 6/3/C}
			\coordinate (D) at ($(A)+(C)-(B)$);
			\tkzDrawPolygon(A,B,C,D)
			\tkzMarkAngle[size=.85](B,A,D)
			\draw (A) node[above right]{$\alpha$};
			\tkzDefPoints{0.8/0.5/G, 4.5/1.5/H, 2/3.5/E, 2.5/-0.5/F}
			\tkzInterLL(A,B)(E,F) \tkzGetPoint{K}
			\coordinate (I) at ($(E)!0.4!(F)$);
			\draw (G)--(H) (E)--(I) (K)--(F);
			\draw[dashed] (I)--(K);
			\draw ($(E)!0.1!(F)$) node[above right]{$a$};
			\draw ($(G)!0.9!(H)$) node[above]{$b$};
			\draw [fill=black] (I) circle(1pt);
			\tkzLabelPoints[left](I)
		\end{tikzpicture}\\
		\small * $a$ cắt $b$ & \small * $a$ song song $b$ & \small * $a$ trùng $b$ & \small * $a$ chéo $b$\\
		\small * Kí hiệu $a \cap b = M$ & \small * Kí hiệu $a \parallel b $  & \small * Kí hiệu 	$a \equiv b$ & \small * $a$, $b$ không điểm chung
	\end{tabular}
	\item \indam{Chú ý:}
		Cho hai đường thẳng $a$ và $b$ phân biệt.
		\begin{itemize}
			\item [$\bullet$] Khi kiểm tra hai đường thẳng $a$ và $b$ \textbf{song song} hay \textbf{cắt nhau} thì trước tiên chúng phải đồng phẳng (cùng thuộc một mặt phẳng nào đó);
			\item [$\bullet$] Khi $a$ và $b$ không có điểm chung thì chúng có thể song song hoặc chéo nhau. Vấn đề này các bạn hay bị nhầm lẫn, cần chú ý. 
		\end{itemize}
\end{enumerate}

\subsubsection{CÁC ĐỊNH LÝ VÀ HỆ QUẢ CẦN NHỚ}
\begin{enumerate}[\iconMT]
	\item \indam{Định lý 1:} Trong không gian, qua một điểm không nằm trên đường thẳng cho trước, có một và chỉ một đường thẳng song song với đường thẳng đã cho.
	\item \indam{Định lý 2:} Hai đường thẳng phân biệt cùng song song với đường thẳng thứ ba thì song song với nhau.
\immini{	\item \indam{Định lý 3:} Nếu ba mặt phẳng phân biệt đôi một cắt nhau theo ba giao tuyến phân biệt thì ba giao tuyến đó hoặc đồng quy hoặc đôi một song song với nhau.}{
\begin{tikzpicture}[line cap=round,line join=round,x=1.0cm,y=1.0cm, font=\small]
	\begin{scope}[>=stealth,scale=0.5]
		\tikzset{label style/.style={font=\footnotesize}}
		\tkzDefPoints{0/0/A,0/5/B, 3/4/C,3/-1/D, -3/3/E, -3/-2/F, 0/4/I}
		\coordinate (J) at ($(A)!0.77!(F)$);
		\coordinate (M) at ($(A)!0.5!(D)$);
		\coordinate (c) at ($(I)!0.5!(B)$);
		\coordinate (a) at ($(I)!0.5!(J)$);
		\coordinate (b) at ($(I)!0.5!(M)$);
		\tkzDrawSegments[dashed](A,M A,J A,I)
		\tkzDrawSegments(F,J E,F E,B B,I B,C C,D D,M M,I M,J I,J)
		\tkzMarkAngles[size=0.8cm](F,E,B)
		\tkzLabelAngles[pos=0.5,rotate=30](B,E,F){$\alpha$}
		\tkzMarkAngles[size=0.8cm](B,C,D)
		\tkzLabelAngles[pos=0.5,rotate=320](D,C,B){$\beta$}
		\tkzMarkAngles[size=0.8cm](M,J,I)
		\tkzLabelAngles[pos=0.5,rotate=30](M,J,I){$\gamma$}
		%\tkzMarkAngles[size=0.5cm]
		\tkzLabelPoints[right](b,c)
		\tkzLabelPoints[left](a)
		
	\end{scope}
	\begin{scope}[xshift=5cm,>=stealth,scale=0.5]
		\tikzset{label style/.style={font=\footnotesize}}
		\tkzDefPoints{0/0/A,0/5/B, 3/4/C,3/-1/D, -3/3/E, -3/-2/F, 0/4/I}
		\coordinate (J) at ($(A)!0.6!(F)$);
		\coordinate (P) at ($(B)!0.6!(E)$);
		\coordinate (Q) at ($(B)!0.5!(C)$);
		\tkzInterLL(A,B)(Q,P)    \tkzGetPoint{I}
		\coordinate (M) at ($(A)!0.5!(D)$);
		\coordinate (c) at ($(I)!0.5!(B)$);
		\coordinate (a) at ($(I)!0.5!(J)$);
		\coordinate (b) at ($(I)!0.5!(M)$);
		\tkzDrawSegments[dashed](A,M A,J A,I)
		\tkzDrawSegments(F,J E,F E,B B,I B,C C,D D,M M,Q M,J J,P P,Q)
		\tkzMarkAngles[size=0.8cm](F,E,B)
		\tkzLabelAngles[pos=0.5,rotate=30](B,E,F){$\alpha$}
		\tkzMarkAngles[size=0.8cm](B,C,D)
		\tkzLabelAngles[pos=0.5,rotate=320](D,C,B){$\beta$}
		\tkzMarkAngles[size=0.8cm](M,J,I)
		\tkzLabelAngles[pos=0.5,rotate=30](M,J,P){$\gamma$}
		%\tkzMarkAngles[size=0.5cm]
		\tkzLabelPoints[right](b,c)
		\tkzLabelPoints[left](a)
	\end{scope} 
\end{tikzpicture}}
	\begin{note}
		\textbf{Hệ quả:} Nếu hai mặt phẳng phân biệt lần lượt chứa hai đường thẳng song song thì giao tuyến của chúng (nếu có) cũng song song với hai đường thẳng đó hoặc trùng với một trong hai đường thẳng đó.

	\end{note}
\end{enumerate}

	
% \subsection{PHÂN LOẠI, PHƯƠNG PHÁP GIẢI TOÁN}
\begin{dang} {Xét vị trí tương đối của hai đường thẳng}
	Cho hai đường thẳng $a$ và $b$ phân biệt. Xét vị trí tương đối của $a$ với $b$:
	\begin{itemize}
		\item Nếu $a$ và $b$ không đồng phẳng thì $a$ và $b$ chéo nhau.
		\item Nếu $a$ và $b$ đồng phẳng thì xét số điểm chung của $a$ và $b$. Nếu $a$ và $b$ không có điểm chung thì $a\parallel b$. Nếu $a$ và $b$ có một điểm chung thì $a$ và $b$ cắt nhau.
	\end{itemize}
\end{dang}
\begin{vd}
	Cho hình chóp $S.ABCD$ có đáy $ABCD$ là hình bình hành. Xét vị trí tương đối của các cặp đường thẳng sau
	\begin{listEX}[3]
		\item [a)] $AB$ và $CD$.
		\item [b)] $SA$ và $SC$.
		\item [c)] $SA$ và $BC$.
	\end{listEX}
\end{vd}
\begin{vd}
	Cho tứ diện $ABCD$ có $M, N$ lần lượt là trung điểm của $AB, AC$. Xét vị trí tương đối của các cặp đường thẳng sau
	\begin{listEX}[3]
		\item [a)] $MN$ và $BC$.
		\item [b)] $AN$ và $CD$.
		\item [c)] $MN$ và $CD$.
	\end{listEX}
\end{vd}
\begin{dang}{Chứng minh hai đường thẳng song song}
	\indamm{Phương pháp thường dùng:}
	\begin{itemize}
		\item [\ding{172}] Sử dụng các kết quả của hình học phẳng như:
		      \begin{itemize}
			      \item  Cặp cạnh đối hình bình hành thì song song nhau;...
			      \item  Đường trung bình của tam giác thì song song và bằng nửa cạnh đáy.
		      \end{itemize}
		\item [\ding{173}] Sử dụng tỉ lệ (Định lý Thales)
		      \begin{itemize}
			      \item Nếu $\dfrac{AE}{AB}=\dfrac{AF}{AC}$ thì $EF \parallel BC$.
			      \item Chú ý tỉ lệ trọng tâm:  $AG=\dfrac{2}{3}AM$.
		      \end{itemize}
	\end{itemize}
\end{dang}

\begin{vd}
	Cho tứ diện $ABCD$ có $I$, $J$ lần lượt là trọng tâm của tam giác $ABC$ và $ABD$. Chứng minh rằng $IJ \parallel CD$.
	\loigiai{\immini{Gọi $E$ là trung điểm $AB$. Ta có $\heva{&I\in CE\\&J\in DE}\Rightarrow IJ$ và $CD$ đồng phẳng.\\
			Vì $I$, $J$ lần lượt là trọng tâm của tam giác $ABC$ và $ABD$ nên $$\dfrac{EI}{EC}=\dfrac{EJ}{ED}=\dfrac{1}{3}.$$
			Theo định lí đảo Thales suy ra $IJ\parallel CD$ (đpcm).
		}{	\begin{tikzpicture}[scale=0.6,font=\footnotesize,line join=round,line cap=round,>=stealth]
				\tkzDefPoints{0/0/B, 3/4/A, 5/-3/C, 6/0/D}
				\coordinate (E) at ($(A)!0.5!(B)$);
				\coordinate (I) at ($(E)!1/3!(C)$);
				\coordinate (J) at ($(E)!1/3!(D)$);
				\tkzDrawPolygon(A,B,C,D)
				\tkzDrawSegments(A,C E,C)
				\tkzDrawSegments[dashed](E,D I,J B,D)
				\tkzDrawPoints[fill=black](A,B,C,D,E,I,J)
				\tkzLabelPoints[above](A,J)
				\tkzLabelPoints[right](D,I)
				\tkzLabelPoints[left](B,E)
				\tkzLabelPoints[below](C)
				\tkzMarkSegments[mark=||](E,A E,B)
			\end{tikzpicture}}}
\end{vd}
\begin{vd}
	Cho hình chóp $S.ABCD$ có đáy $ABCD$ là hình thang với $AB$ là đáy lớn và $AB=2CD$. Gọi $M, N$ lần lượt là trung điểm của các cạnh $SA$ và $SB$. Chứng minh rằng $NC\parallel MD$.
\end{vd}
\begin{vd}
	Cho tứ diện $ABCD$. Gọi $I, J$ lần lượt là trung điểm của các cạnh $BC, CD$. Trên cạnh $AC$ lấy điểm $K$. Gọi $M$ là giao điểm của $BK$ và $AI$, $N$ là giao điểm của $DK$ và $AJ$. Chứng minh rằng $MN\parallel BD$.
\end{vd}
\begin{vd}
	Cho tứ diện $ABCD$. Gọi $M$, $N$, $P$, $Q$, $R$, $S$ lần lượt là trung điểm của $AB$, $CD$, $BC$, $AD$, $AC$, $BD$.
	\begin{tasks}(1)
		\task Chứng minh $MPNQ$ là hình bình hành.
		\task Chứng minh ba đoạn thẳng $MN$, $PQ$, $RS$ cắt nhau tại trung điểm $G$ của mỗi đoạn.
	\end{tasks}
	\loigiai{
		\immini{
			\begin{enumerate}[a)]
				\item Vì $MP$ là đường trung bình của $\triangle ABC$ nên
				      $\heva{& MP \parallel AC\\& MP=\dfrac{1}{2}AC.} \qquad (1)$\\
				      Vì $NQ$ là đường trung bình của $\triangle ACD$ nên
				      $\heva{& NQ \parallel AC\\& NQ=\dfrac{1}{2}AC.} \qquad (2)$\\
				      Từ $(1)$ và $(2)$ suy ra $\heva{& MP \parallel NQ\\& MP=NQ.}$\\
				      Do đó, $MPNQ$ là hình bình hành.
				\item 	Do $MPNQ$ là hình bình hành nên $MN$, $PQ$ cắt nhau tại trung điểm $G$ của mỗi đoạn.\\
				      Mặt khá, chứng minh tương tự ta được $PSQR$ là hình bình hành nên $PQ$, $RS$ cắt nhau tại trung điểm $G$ của mỗi đoạn.\\
				      Vậy $MN$, $PQ$, $RS$ cắt nhau tại trung điểm $G$ của mỗi đoạn.
			\end{enumerate}
		}{	\begin{tikzpicture}[scale=0.6,font=\footnotesize,line join=round,line cap=round,>=stealth]
				\tikzset{label style/.style={font=\footnotesize}}
				\tkzDefPoints{0/0/B, 3/6/A, 1.3/-3/C, 8/0/D}
				\coordinate (M) at ($(A)!0.5!(B)$);
				\coordinate (N) at ($(C)!0.5!(D)$);
				\coordinate (P) at ($(B)!0.5!(C)$);
				\coordinate (Q) at ($(A)!0.5!(D)$);
				\coordinate (R) at ($(A)!0.5!(C)$);
				\coordinate (S) at ($(B)!0.5!(D)$);
				\coordinate (G) at ($(M)!0.5!(N)$);
				\tkzDrawPolygon(A,B,C,D)
				\tkzDrawSegments(A,C M,P Q,N P,R R,Q)
				\tkzDrawSegments[dashed](B,D R,S M,N P,Q M,Q P,N P,S S,Q)
				\tkzDrawPoints[fill=black](A,B,C,D,M,N,P,Q,R,S,G)
				\tkzLabelPoints[above](A,G)
				\tkzLabelPoints[right](D)
				\tkzLabelPoints[left](B,R)
				\tkzLabelPoints[below](C)
				\tkzLabelPoints[below right](S,N)
				\tkzLabelPoints[below left](P)
				\tkzLabelPoints[above right](Q)
				\tkzLabelPoints[above left](M)
			\end{tikzpicture}}}
\end{vd}

\begin{vd}
	Cho hình chóp $S.ABCD$ có đáy là hình bình hành. Gọi $M,N,P,Q$ lần lượt là trung điểm $BC,CD,SB,SD$.
	\begin{tasks}(1)
		\task Chứng minh rằng $MN\parallel PQ$.
		\task Gọi $I$ là trọng tâm của tam giác $ABC$, $J$ thuộc $SA$ sao cho $\dfrac{JS}{JA}=\dfrac{1}{2}$. Chứng minh $IJ\parallel SM$.
	\end{tasks}
	\loigiai{
		\begin{enumerate}[a)]
			\immini{	\item Chứng minh $MN\parallel PQ$.\\
			      Ta có: $MN\parallel BD$ ($MN$ là đường trung bình của $\Delta BCD$).\\
			      và $PQ\parallel BD$ ($PQ$ là đường trung bình của $\Delta SBD$).\\
			      Suy ra $MN\parallel PQ$
			\item Chứng minh $IJ\parallel SM$.\\
			      $\dfrac{AJ}{AS}=\dfrac{2}{3}$.\\
			      $ \dfrac{JS}{JA}=\dfrac{1}{2}$.\\
			      $\dfrac{AI}{AM}=\dfrac{2}{3}$ ($I$ là trọng tâm của $\Delta ABC$).\\
			      Suy ra $\dfrac{AJ}{AS}=\dfrac{AI}{AM}$.\\
			      Theo định lí Viet đảo ta có $IJ\parallel SM$.}{\begin{tikzpicture}
				      \tkzDefPoints{-1/4/S,-2/0/B,2/0/C} %Định nghĩa các toạ đô dịnh cơ sở
				      \tkzDefPointBy[rotation = center B angle 60](C)
				      \tkzGetPoint{A1}
				      \tkzDefPointsBy[homothety=center B ratio 0.5](A1){A}
				      \tkzDefPointBy[translation = from B to A](C)
				      \tkzGetPoint{D}
				      \tkzDefMidPoint(S,B)
				      \tkzGetPoint{P}
				      \tkzDefMidPoint(C,B)
				      \tkzGetPoint{M}
				      \tkzDefMidPoint(C,D)
				      \tkzGetPoint{N}
				      \tkzDefMidPoint(S,D)
				      \tkzGetPoint{Q}
				      \tkzDefPointsBy[homothety=center S ratio 0.3333](A){J}
				      \tkzInterLL(A,M)(B,D)
				      \tkzGetPoint{I}
				      \tkzLabelPoints[left](P)
				      \tkzLabelPoints[above](S)
				      \tkzLabelPoints[right](N,Q,J)
				      \tkzLabelPoints[above left](A,B)
				      \tkzLabelPoints[above](I)
				      %\tkzLabelPoints[right](A,B,C,D,N,M,P)
				      \tkzLabelPoints[right](D)
				      \tkzLabelPoints[below](C,M)
				      \tkzDrawSegments[dashed](A,B S,A A,D Q,P A,C D,B I,J A,M M,N)
				      \tkzDrawSegments(S,B S,C S,D S,M B,C C,D)
			      \end{tikzpicture}}
		\end{enumerate}
	}
\end{vd}

\begin{dang}{Xác định giao tuyến $d$ của hai mặt phẳng cắt nhau}
	\indamm{Ta thực hiện một trong hai cách sau đây:}
	\begin{itemize}
		\item [\iconCH] \indamm{Cách 1:}Tìm hai điểm chung phân biệt (đã xét ở bài học trước)
		\item [\iconCH] \indamm{Cách 2:} Tìm 1 điểm chung. Sau đó nếu hai mặt phẳng có cặp đường thẳng song song nhau thì giao tuyến $d$ sẽ đi qua điểm chung và song song (hoặc trùng) với một trong hai đường thẳng đó.
	\end{itemize}
\end{dang}

\begin{vd}
	Cho tứ diện $ABCD$. Trên $AB$, $AC$ lần lượt lấy $M$, $N$ sao cho $\dfrac{AM}{AB}=\dfrac{AN}{AC}$. Tìm giao tuyến của hai mặt phẳng $(DBC)$ và $(DMN)$.
	\loigiai{
		\begin{center}
			\begin{tikzpicture}[>=stealth,line join=round,line cap=round,font=\footnotesize,scale=1]
				\coordinate (A) at (2,4.5);
				\coordinate (B) at (0,0);
				\coordinate (C) at (6,0);
				\coordinate (D) at (3,-2);
				\coordinate (M) at ($(A)!.7!(B)$);
				\coordinate (N) at ($(A)!.7!(C)$);
				\coordinate (X) at ($(C)+(D)-(B)$);
				\coordinate (Y) at ($(D)!.5!(X)$);
				\coordinate (Z) at ($(D)!-0.3!(X)$);
				\draw (A)--(B)--(D)--(C)--(A)--(D)--(M) (D)--(N) (Y)--(Z);
				\draw[dashed] (M)--(N) (B)--(C);
				\foreach \p/\pos in {A/above,B/left,C/right,D/below,M/above left, N/above right}
				\fill (\p) circle(1pt)node[\pos]{\p};
				\draw (Y) node[above] {$x$};
			\end{tikzpicture}
		\end{center}
		Trong tam giác $ABC$, theo giả thiết $\dfrac{AM}{AB}=\dfrac{AN}{AC}$ suy ra $MN\parallel BC$. \\
		Ta có $\left\{\begin{aligned}
				 & D\in (DBC)\cap (DMN)            \\
				 & BC\subset (DBC),MN\subset (DMN) \\
				 & BC\parallel MN
			\end{aligned}\right. $ suy ra \\
		$(DBC)\cap (DMN)=Dx\parallel BC\parallel MN$.
	}
\end{vd}

\begin{vd}
	Cho tứ diện $ABCD$. Gọi $M$, $N$ lần lượt là trung điểm của $AD$ và $BD$; $G$ là trọng tâm tam giác $ABC$. Tìm giao tuyến của hai mặt phẳng $(ABC)$ và $(MNG)$.
	\loigiai{
		\begin{center}
			\begin{tikzpicture}[>=stealth,line join=round,line cap=round,font=\footnotesize,scale=1]
				\coordinate (A) at (4,4);
				\coordinate (B) at (0,0);
				\coordinate (C) at (6,0);
				\coordinate (D) at (5,-2);
				\coordinate (M) at ($(A)!.5!(D)$);
				\coordinate (N) at ($(B)!.5!(D)$);
				\coordinate (X) at ($(A)!.5!(B)$);
				\coordinate (G) at ($(C)!2/3!(X)$);
				\coordinate (P) at ($(C)!2/3!(A)$);
				\coordinate (Q) at ($(C)!2/3!(B)$);
				\draw (A)--(B)--(D)--(C)--(A)--(D) (M)--(N);
				\draw[dashed] (M)--(G)--(N) (P)--(Q) (C)--(X) (B)--(C);
				\foreach \p/\pos in {A/above, B/left, D/below, C/right, M/right, N/below, P/right, G/above, Q/above}
				\fill (\p) circle(1pt) node[\pos]{\p};
			\end{tikzpicture}
		\end{center}
		Do $M$, $N$ lần lượt là trung điểm của $AD$ và $BD$ nên $MN\parallel AB$. \\
		Ta có $\left\{\begin{aligned}
				 & G\in (ABC)\cap (MNG)            \\
				 & AB\subset (ABC),MN\subset (MNG) \\
				 & AB\parallel MN
			\end{aligned}\right. $ suy ra \\
		$(ABC)\cap (MNG)=PQ\parallel AB\parallel MN$, \\
		với $PQ$ qua $G$ và song song với $AB$.
	}
\end{vd}

\begin{vd}
	Cho hình chóp $S.ABCD$ có đáy $ABCD$ là hình bình hành. Điểm $M$ thuộc cạnh $SA$. Điểm $E$, $F$ lần lượt là trung điểm của $AB$ và $BC$.
	\begin{enumEX}{2}
		\item Tìm $(SAB) \cap (SCD)$.
		\item Tìm $(MBC) \cap (SAD)$.
		\item Tìm $(MEF) \cap (SAC)$.
		\item Tìm $AD \cap (MEF)$.
		\item Tìm $SD \cap (MEF)$.
		%	\item Tìm thiết diện của hình chóp cắt bởi $(MEF)$.
	\end{enumEX}
	\loigiai{
		\begin{enumerate}
			\immini{	\item $\heva{& S \in (SAB) \cap (SCD)\\& AB \subset (SAB),~ CD \subset (SCD)\\& AB \parallel CD}$\\
			      $\Rightarrow (SAB) \cap (SCD) =Sx ~\text{với}~Sx \parallel AB \parallel CD$.
			\item $\heva{& M \in (MBC) \cap (SAD)\\& BC \subset (MBC),~ AD \subset (SAD)\\& BC \parallel AD}$\\
			      $\Rightarrow (MBC) \cap (SAD) =My ~\text{với}~My \parallel BC \parallel AD$.
			\item $\heva{& M \in (MEF) \cap (SAC)\\& EF \subset (MEF),~ AC \subset (SAC)\\& EF \parallel AC}$\\
			      $\Rightarrow (MEF) \cap (SAC) =Mz ~\text{với}~Mz \parallel EF \parallel AC$.
			      }{	\begin{tikzpicture}[scale=0.7,font=\footnotesize,line join=round,line cap=round,>=stealth]
				      \tikzset{label style/.style={font=\footnotesize}}
				      \tkzDefPoints{0/0/A, -3/-3/B, 6/0/D}
				      \coordinate (C) at ($(B)+(D)-(A)$);
				      \coordinate (S) at ($(A)+(1,6)$);
				      \coordinate (M) at ($(S)!2/5!(A)$);
				      \coordinate (E) at ($(A)!1/2!(B)$);
				      \coordinate (F) at ($(B)!1/2!(C)$);
				      \tkzDefPointBy[translation=from A to B](S) \tkzGetPoint{x}
				      \coordinate (U) at ($(x)!5/4!(S)$);
				      \tkzDefPointBy[translation=from A to D](M) \tkzGetPoint{y}
				      \coordinate (V) at ($(y)!5/4!(M)$);
				      \tkzInterLL(y,V)(S,B)    \tkzGetPoint{Z}
				      \tkzInterLL(y,V)(S,D)    \tkzGetPoint{W}
				      \tkzDefPointBy[translation=from A to C](M) \tkzGetPoint{z}
				      \coordinate (T) at ($(z)!2/5!(M)$);
				      \tkzInterLL(z,M)(S,C)    \tkzGetPoint{K}
				      \tkzInterLL(E,F)(A,D)    \tkzGetPoint{I}
				      \tkzInterLL(E,F)(S,B)    \tkzGetPoint{J}
				      \tkzInterLL(A,D)(S,B)    \tkzGetPoint{L}
				      \tkzInterLL(I,M)(S,B)    \tkzGetPoint{a}
				      \tkzInterLL(I,M)(S,D)    \tkzGetPoint{N}
				      \tkzDrawPolygon(S,B,C,D)
				      \tkzDrawSegments(S,C x,U y,W V,Z K,T I,L I,J I,a K,N K,F)
				      \tkzDrawSegments[dashed](S,A A,B A,D B,M C,M Z,W M,E E,F M,F A,C M,K A,L E,J a,N)
				      \tkzDrawPoints[fill=black](S,A,B,C,D,E,F,M,I,N,K)
				      \tkzLabelPoints[above](S,x,y)
				      \tkzLabelPoints[right](D,K)
				      \tkzLabelPoints[left](I)
				      \tkzLabelPoints[above left](A,M)
				      \tkzLabelPoints[above right](N)
				      \tkzLabelPoints[above right,,xshift=-1cm,yshift=0.8cm](z)
				      \tkzLabelPoints[below](B,C,F)
				      \tkzLabelPoints[below,xshift=-0.1cm,yshift=-0.2cm](E)
			      \end{tikzpicture}}
			\item Trong $(ABCD)$, gọi $I=EF \cap AD$.\\
			      Mà $EF \subset (MEF)$ nên $AD \cap (MEF)=I$.
			\item Trong $(SAD)$, gọi $N=SD \cap IM$.\\
			      Mà $IM \subset (MEF)$ nên $SD \cap (MEF)=N$.
			      %	\item Thiết diện của hình chóp cắt bởi $(MEF)$ là ngũ giác $MNKFE$.
		\end{enumerate}
	}
\end{vd}

\subsection{BÀI TẬP TỰ LUYỆN}
\begin{bt}
	Cho hình chóp $S.ABCD$ có đáy $ABCD$ là hình bình hành tâm $O$. Gọi $M$, $N$ lần lượt là trung điểm của $SA$, $SD$. Chứng minh
	\begin{listEX}[2]
		\item $MN\parallel AD$ và $MN\parallel BC$;
		\item $MO\parallel SC$ và $NO\parallel SB$.
	\end{listEX}
	\loigiai{
		\begin{center}
			\begin{tikzpicture}[line join = round, line cap = round,>=stealth,font=\footnotesize,scale=1]
				\tkzDefPoints{0/0/A}
				\coordinate (B) at ($(A)+(5,0)$);
				\tkzDefShiftPoint[A](-150:3){D}
				\coordinate (C) at ($(B)+(D)-(A)$);
				\tkzInterLL(A,C)(B,D)    \tkzGetPoint{O}
				\coordinate (S) at ($(A)+(-0.2,4)$);
				\coordinate (M) at ($(S)!0.5!(A)$);
				\coordinate (N) at ($(S)!0.5!(D)$);
				\tkzDrawPolygon(S,B,C,D)
				\tkzDrawSegments(S,C)
				\tkzDrawSegments[dashed](A,S A,B A,D A,C B,D M,N O,M O,N)
				\tkzMarkSegments[mark=||,size=2pt](M,A M,S)
				\tkzMarkSegments[mark=|,size=2pt](N,S N,D)
				\tkzDrawPoints[fill=black](A,B,D,C,O,S,M,N)
				\tkzLabelPoints[above](S)
				\tkzLabelPoints[below](O)
				\tkzLabelPoints[left](A,D,N)
				\tkzLabelPoints[above left](M)
				\tkzLabelPoints[right](B)
				\tkzLabelPoints[below right](C)
			\end{tikzpicture}
		\end{center}
		\begin{enumerate}
			\item Xét tam giác $SAD$ có
			      \begin{itemize}
				      \item $M$ là trung điểm của $SA$ (giả thiết);
				      \item $N$ là trung điểm của $SD$ (giả thiết).
			      \end{itemize}
			      Suy ra $MN$ là đường trung bình của $\triangle SAD$. Do đó $MN\parallel AD$.\\
			      Ta có $\heva{&MN\parallel AD~ (\text{chứng minh trên})\\ &BC\parallel AD ~(ABCD \text{ là hình bình hành})}\Rightarrow MN\parallel BC$.
			\item Xét tam giác $ASC$ có
		\end{enumerate}
		\begin{itemize}
			\item $M$ là trung điểm của $SA$ (giả thiết);
			\item $O$ là trung điểm của $AC$ ($O$ là tâm của hình bình hành $ABCD$).
		\end{itemize}
		Suy ra $OM$ là đường trung bình của $\triangle SAC$. Do đó $MO\parallel SC$.\\
		Tương tự, $NO$ là đường trung bình của $\triangle SDB$ nên $NO\parallel SB$.
	}
\end{bt}

\begin{bt}
	Cho hình chóp $S.ABCD$ có đáy $ABCD$ là hình bình hành tâm $O$. Gọi $M$, $N$ lần lượt là trung điểm của $AB$, $AD$. Gọi $I$, $J$, $G$ lần lượt là trọng tâm của các tam giác $SAB$, $SAD$ và $AOD$. Chứng minh
	\begin{listEX}[2]
		\item $IJ\parallel MN$;
		\item $IJ\parallel BD$ và $GJ\parallel SO$.
	\end{listEX}
	\loigiai{
		\begin{center}
			\begin{tikzpicture}[line join = round, line cap = round,>=stealth,font=\footnotesize,scale=1]
				\tkzDefPoints{0/0/A}
				\coordinate (D) at ($(A)+(5,0)$);
				\tkzDefShiftPoint[A](30:2.5){B}
				\coordinate (C) at ($(B)+(D)-(A)$);
				\tkzInterLL(A,C)(B,D)    \tkzGetPoint{O}
				\coordinate (S) at ($(B)+(0.2,3)$);
				\coordinate (M) at ($(A)!0.5!(B)$);
				\coordinate (N) at ($(A)!0.5!(D)$);
				\tkzDefPointBy[homothety = center S ratio 2/3](M)    \tkzGetPoint{I}
				\tkzDefPointBy[homothety = center S ratio 2/3](N)    \tkzGetPoint{J}
				\tkzDefPointBy[homothety = center O ratio 2/3](N)    \tkzGetPoint{G}
				\tkzDrawPolygon(S,A,D,C)
				\tkzDrawSegments(S,D S,N)
				\tkzDrawSegments[dashed](O,S B,S B,A B,C B,D A,C S,M M,N O,N I,J J,G)
				\tkzMarkSegments[mark=||,pos=.5,size=2pt](N,A N,D)
				\tkzMarkSegments[mark=|,pos=.5,size=2pt](M,A M,B)
				\tkzDrawPoints[fill=black](A,B,D,C,O,S,I,J,G)
				\tkzLabelPoints[above](S)
				\tkzLabelPoints[below](N,B)
				\tkzLabelPoints[left](A,M,I)
				\tkzLabelPoints[above right](J)
				\tkzLabelPoints[right](C,G)
				\tkzLabelPoints[below right](D,O)
			\end{tikzpicture}
		\end{center}
		\begin{enumerate}
			\item Xét tam giác $SMN$ có
			      \begin{itemize}
				      \item $SI=\dfrac{2}{3}SM$ ($I$ là trọng tâm của $\triangle SAB$);
				      \item $SJ=\dfrac{2}{3}SN$ ($J$ là trọng tâm của $\triangle SAD$).
			      \end{itemize}
			      suy ra $IJ\parallel MN$ (định lý Ta-lét đảo).
			\item Vì $MN$ là đường trung bình của $\triangle ABD$ nên $MN\parallel BD$.\\
			      Mà $IJ\parallel MN$ (chứng minh trên) nên $IJ\parallel BD$.\\
			      Xét tam giác $SON$ có
			      \begin{itemize}
				      \item $NG=\dfrac{1}{3}NO$ ($G$ là trọng tâm của $\triangle AOD$);
				      \item $NJ=\dfrac{1}{3}SN$ ($J$ là trọng tâm của $\triangle SAD$).
			      \end{itemize}
			      suy ra $GJ\parallel SO$ (định lý Ta-lét đảo).
		\end{enumerate}
	}
\end{bt}

\begin{bt}
	Cho hình chóp $SABCD$ có đáy $ABCD$ là hình thang đáy lớn $AB$. Gọi $E$,
	$F$ lần lượt là trung điểm của $SA$ và $SB$.
	\begin{listEX}[3]
		\item Chứng minh $EF \parallel CD$.
		\item Tìm $I = AF \cap (SCD)$.
		\item Chứng minh $SI \parallel AB \parallel CD$.
	\end{listEX}
	\loigiai{
		\begin{center}
			\begin{tikzpicture}[scale=1, line join=round, line cap=round]
				\tkzDefPoints{0/0/A,4/0/B,2.5/-1.6/C,1/3/S}
				\tkzDefPointBy[translation=from B to A](C)\tkzGetPoint{D'}
				\tkzDefPointBy[homothety=center C ratio 0.9](D')\tkzGetPoint{D}
				\coordinate (E) at ($(S)!0.5!(A)$);
				\coordinate (F) at ($(S)!0.5!(B)$);
				\tkzDefPointBy[translation=from A to B](S)\tkzGetPoint{x}
				\tkzInterLL(A,F)(S,x)\tkzGetPoint{I}
				\tkzInterLL(S,C)(I,D)\tkzGetPoint{K}

				\tkzDrawPolygon(S,B,C,D)
				\tkzDrawSegments(S,C I,D F,I S,I)
				\tkzDrawSegments[dashed](A,S A,B A,D A,F E,F)
				\tkzDrawPoints[fill=black,size=4](D,C,A,B,S,E,F,I)

				\tkzLabelPoints[above](S,F)
				\tkzLabelPoints[below](A,B,C)
				\tkzLabelPoints[right](D,E)
				\tkzLabelPoints[left](I)
			\end{tikzpicture}
		\end{center}
		\begin{listEX}[]
			\item Ta có $EF$ là đường trung bình của tam giác $SAB$ nên $EF
				\parallel AB$\\
			mà $AB \parallel CD$ (hai đáy của hình thang)\\
			nên $EF \parallel CD$.
			\item Hai mặt phẳng $(SAB)$ và $(SCD)$ có $AB \parallel CD$ nên giao
			tuyến là đường thẳng $Sx \parallel AB \parallel CD$.\\
			Kéo dài $AF$ cắt $Sx$ tại $I$.\\
			Ta thấy $I$ là điểm chung của $AF$ và $(SCD)$.
			\item Theo ý \circled{\textbf{2}}.
		\end{listEX}
	}
\end{bt}

\begin{bt}
	Cho hình chóp $S.ABCD$ có đáy $ABCD$ là hình bình hành. Gọi $M$, $N$ lần lượt là trung điểm của $SA$, $SB$. Gọi $P$ là một điểm trên cạnh $BC$. Tìm giao tuyến của
	\begin{listEX}[3]
		\item $(SBC)$ và $(SAD)$;
		\item $(SAB)$ và $(SCD)$;
		\item $(MNP)$ và $(ABCD)$.
	\end{listEX}
	\loigiai{
		\begin{center}
			\begin{tikzpicture}[line join = round, line cap = round,>=stealth,font=\footnotesize,scale=1]
				\tkzDefPoints{0/0/A}
				\coordinate (D) at ($(A)+(5,0)$);
				\tkzDefShiftPoint[A](30:2.5){B}
				\coordinate (C) at ($(B)+(D)-(A)$);
				\tkzInterLL(A,C)(B,D)    \tkzGetPoint{O}
				\coordinate (S) at ($(A)+(0.5,4)$);
				\coordinate (M) at ($(S)!0.5!(A)$);
				\coordinate (N) at ($(S)!0.5!(B)$);
				\coordinate (P) at ($(B)!0.6!(C)$);
				\coordinate (Q) at ($(A)!0.6!(D)$);
				%
				\coordinate (E) at ($(A)+(S)-(B)$);
				\coordinate (y) at ($(S)+(B)-(A)$);
				\coordinate (F) at ($(E)!0.8!(S)$);
				%
				\draw (0,4)--(4,4) node at (4.2,4){$x$};
				\tkzDrawPolygon(S,A,D,C)
				\tkzDrawPolygon[fill=cyan,dashed](M,N,P)
				\tkzDrawSegments(S,D F,y)
				\tkzDrawSegments[dashed](B,S B,A B,C B,D A,C P,Q)
				\tkzMarkSegments[mark=||,pos=.5,size=2pt](N,S N,B)
				\tkzMarkSegments[mark=|,pos=.5,size=2pt](M,A M,S)
				\tkzDrawPoints[fill=black](A,B,D,C,O,S,M,N,Q,P)
				\tkzLabelPoints[above](S,O,y)
				\tkzLabelPoints[below](Q)
				\tkzLabelPoints[left](M,B)
				\tkzLabelPoints[above right](P,N)
				\tkzLabelPoints[right](C,D)
				\tkzLabelPoints[below left](A)
			\end{tikzpicture}
		\end{center}
		\begin{enumerate}
			\item Ta có
			      \begin{itemize}
				      \item $(SBC)\cap (ABCD)=BC$;
				      \item $(SAD)\cap (ABCD)=AD$;
				      \item $AD\parallel BC$ ($ABCD$ là hình bình hành).
			      \end{itemize}
			      Mà $S$ là điểm chung của 2 mặt phẳng $(SBC)$ và $(SAD)$ nên giao tuyến của 2 mặt phẳng $(SBC)$ và $(SAD)$ là đường thẳng $Sx\parallel BC\parallel AD$.
		\end{enumerate}
		\begin{enumerate}
			\setcounter{enumi}{1}
			\item Giao tuyến của hai mặt phẳng $(SAB)$ và $(SCD)$ là đường thẳng $Sy\parallel AB\parallel CD$.
			\item Vì $MN\parallel AB$ ($MN$ là đường trung bình của $\triangle SAB$) nên qua $P$ kẻ $PQ\parallel AB~(Q\in AD)$. Khi đó giao tuyến của hai mặt phẳng $(MNP)$ và $(ABCD)$ là đường thẳng $PQ$.
		\end{enumerate}
	}
\end{bt}
\begin{bt}
	Cho tứ diện $SABC$. Gọi $E$ và $F$ lần lượt là trung điểm của các cạnh $SB$ và $AB$, $G$ là một điểm trên cạnh $AC$. Tìm giao tuyến của các cặp mặt phẳng sau
	\begin{listEX}[2]
		\item $(SAC)$ và $(EFC)$;
		\item $(SAC)$ và $(EFG)$.
	\end{listEX}
	\loigiai{
		\begin{center}
			\begin{tikzpicture}[line join = round, line cap = round,>=stealth,font=\footnotesize,scale=1]
				\tkzDefPoints{0/0/A}
				\coordinate (C) at ($(A)+(5,0)$);
				\tkzDefShiftPoint[A](-30:3){B}
				\coordinate (S) at ($(A)+(1.5,4)$);
				\coordinate (E) at ($(S)!0.5!(B)$);
				\coordinate (F) at ($(A)!0.5!(B)$);
				\coordinate (G) at ($(A)!0.13!(C)$);
				\coordinate (H) at ($(S)!0.13!(C)$);
				\coordinate (x) at ($(S)+(C)-(A)$);
				\coordinate (D) at ($(A)+(C)-(S)$);
				\coordinate (K) at ($(D)!0.8!(C)$);
				\tkzDrawPolygon(S,A,B,C)
				\tkzDrawSegments(S,B K,x E,F E,C)
				\tkzDrawSegments[dashed](A,C G,H F,C G,E F,G)
				\tkzMarkSegments[mark=||,pos=.5,size=2pt](E,S E,B)
				\tkzMarkSegments[mark=|,pos=.5,size=2pt](F,A F,B)
				\tkzDrawPoints[fill=black](B,C,A,S,E,F,G,H)
				\tkzLabelPoints[above](S)
				\tkzLabelPoints[below](B)
				\tkzLabelPoints[left](A)
				\tkzLabelPoints[right](C,x)
				\tkzLabelPoints[above right](H,E)
				\tkzLabelPoints[below left](F)
				\tkzLabelPoints[above left](G)
			\end{tikzpicture}
		\end{center}
		\begin{enumerate}
			\item Ta có
			      \begin{itemize}
				      \item $(SAC)\cap (SAB)=SA$;
				      \item $(EFC)\cap (SAB)=EF$;
				      \item $SA\parallel EF$ ($EF$ là đường trung bình của $\triangle SAB$).
			      \end{itemize}
			      Do đó giao tuyến của 2 mặt phẳng $(SAC)$ và $(EFC)$ sẽ song song với $SA$ và $EF$.\\
			      Mà $C$ là điểm chung của 2 mặt phẳng $(SAC)$ và $(EFC)$ nên giao tuyến của chúng là đường thẳng $Cx\parallel SA\parallel EF$.

			\item Vì $EF\parallel SA$ ($EF$ là đường trung bình của $\triangle SAB$) nên qua $G$ kẻ $GH\parallel SA~(H\in SC)$. Khi đó giao tuyến của hai mặt phẳng $(SAC)$ và $(EFG)$ là đường thẳng $GH$.
		\end{enumerate}
	}
\end{bt}

\begin{bt}
	Cho hình chóp $S.ABCD$ có đáy $ABCD$ là hình bình hành. Gọi $G$ là trọng tâm tam giác $ABD$, $N$ là trung điểm $SG$. Tìm giao tuyến của hai mặt phẳng $(ABN)$ và $(SCD)$.
	\loigiai{
		\begin{center}
			\begin{tikzpicture}[>=stealth,line join=round,line cap=round,font=\footnotesize,scale=0.9]
				\coordinate (S) at (0,5);
				\coordinate (A) at (-3,-1.5);
				\coordinate (B) at (0,0);
				\coordinate (C) at (5,0);
				\coordinate (D) at ($(A)+(C)-(B)$);
				\coordinate (O) at (intersection of A--C and B--D);
				\coordinate (G) at ($(A)!2/3!(O)$);
				\coordinate (N) at ($(S)!.5!(G)$);
				\coordinate (P) at (intersection of A--N and S--C);
				\coordinate (X) at ($(P)+(D)-(C)$);
				\coordinate (Q) at (intersection of P--X and S--D);
				\draw (S)--(A)--(D)--(C)--(S)--(D) (P)--(Q);
				\draw[dashed] (S)--(B)--(A)--(O) (B)--(D) (S)--(G) (B)--(C) (A)--(P);
				\foreach \p/\pos in {S/above, A/below, D/below, C/right, B/above right, G/below, P/right, Q/left, N/left, O/below}
				\fill (\p) circle(1pt) node[\pos]{\p};
			\end{tikzpicture}
		\end{center}
		Trong mặt phẳng $(SAC)$, gọi $P=AN\cap SC$. Ta có \\
		$P\in AN$ mà $AN\subset (ABN)$ suy ra $P\in (ABN)$. \\
		$P\in SC$ mà $SC\subset (SCD)$ suy ra $P\in (SCD)$. \\
		Do đó $P\in (ABN)\cap (SCD)$. \\
		Ta có $\left\{\begin{aligned}
				 & P\in (ABN)\cap (SCD)            \\
				 & AB\subset (ABN),CD\subset (SCD) \\
				 & AB\parallel CD
			\end{aligned}\right. $ suy ra \\
		$(ABN)\cap (SCD)=PQ\parallel CD\parallel AB$.
	}
\end{bt}

\begin{bt}
	Cho tứ diện $ABCD$. Gọi $M, N$ theo thứ tự là trung điểm của $AB, BC$ và $Q$ là một điểm nằm trên cạnh $AD$ ($QA\neq QD$) và $P$ là giao điểm của $CD$ với mặt phẳng $(MNQ)$. Chứng minh rằng $PQ\parallel MN$ và $PQ\parallel AC$.
	\loigiai{
		\immini{
			Vì $QA\neq QD$ nên gọi $K=QM\cap BD$ suy ra $KN \cap CD=P$.\\
			Theo định lý về giao tuyến ba mặt phẳng\\
			Ta xét ba mặt phẳng $(ABC)$ $(ACD)$ và $(MNQ)$.\\
			Ta có: $\heva{&(ABC)\cap (ACD)=AC\\&(ABC)\cap (MNQ)=MN\\&(ACD)\cap (MNQ)=QP}$.\\
			Vậy $AC \parallel MN $ nên $AC\parallel QP\parallel NM$.}
		{\begin{tikzpicture}%[scale=1]
				%\tkzInit[xmin=-5,ymin=-1.5,xmax=3,ymax=2]
				%\tkzClip
				\tkzDefPoints{-3/0/B,2/0/D,-1/3/A} % Định nghĩa các toạ đô dịnh cơ sở
				\tkzDefPointsBy[homothety=center B ratio 0.6](D){B1}
				\tkzDefPointBy[rotation = center B angle -30](B1)
				\tkzGetPoint{C}
				\tkzDefMidPoint(A,B)
				\tkzGetPoint{M}
				\tkzDefMidPoint(C,B)
				\tkzGetPoint{N}
				\tkzDefPointsBy[homothety=center A ratio 0.3](D){Q}
				\tkzInterLL(M,Q)(B,D)
				\tkzGetPoint{K}
				\tkzInterLL(K,N)(C,D)
				\tkzGetPoint{P}
				\tkzLabelPoints[above](A)
				\tkzLabelPoints[above right](Q)
				\tkzLabelPoints[above left](K,B,M)
				\tkzLabelPoints[right](D,P)
				\tkzLabelPoints[below](C,N)
				\tkzDrawSegments[dashed](B,D M,Q M,B B,N B,K N,P)
				\tkzDrawSegments(A,M A,C A,D A,M Q,P C,D N,C M,K N,K M,N)
			\end{tikzpicture}}
	}
\end{bt}

\begin{bt}
	Cho hình chóp $ S.ABCD $ có đáy $ ABCD $ là hình thang với $ AD $ là đáy lớn và $ AD=2BC $. Gọi $ M $, $ N $, $ P $ lần lượt thuộc các đoạn $ SA $, $ AD $, $ BC $ sao cho $ MA=2MS $, $ NA=2ND $, $ PC=2PB $.
	\begin{itemize}
		\item[a)] Tìm giao tuyến của các cặp mặt phẳng sau: $ (SAD) $ và $ (SBC) $, $ (SAC) $ và $ (SBD) $.
		\item[b)] Xác định giao điểm $ Q $ của $ SB $ với $ (MNP) $.
		\item[c)] Gọi $ K $ là trung điểm của $ SD $. Chứng minh $ CK=(MQK)\cap (SCD) $.
	\end{itemize}
	\loigiai{
		\begin{center}
			\begin{tikzpicture}[line join=round, line cap=round,thick,scale=0.8]
				\tikzset{label style/.style={font=\footnotesize}}
				\coordinate (A) at (0,0);
				\coordinate (B) at (-0.8,-3);
				\coordinate (D) at (10,0);
				\tkzDefPointWith[colinear = at B,K=0.5](A,D)
				\tkzGetPoint{C}
				\coordinate (S) at ($(A)+(-0.5,6)$);
				\coordinate (N) at ($(A)!0.6667!(D)$);
				\coordinate (M) at ($(S)!0.3333!(A)$);
				\coordinate (P) at ($(B)!0.3333!(C)$);

				\tkzInterLL(A,C)(B,D) \tkzGetPoint{O}
				\tkzInterLL(N,P)(B,A) \tkzGetPoint{E}
				\tkzInterLL(E,M)(B,S) \tkzGetPoint{Q}
				\coordinate (K) at ($(S)!0.5!(D)$);
				\coordinate (K') at ($(S)!0.5!(A)$);
				\coordinate (t) at ($(S)+(K)-(K')$);
				\tkzInterLL(K,M)(t,S) \tkzGetPoint{F}
				\tkzDrawSegments(S,B S,C B,C C,D S,D S,t Q,E C,K P,E B,E F,C S,F)
				\tkzDrawSegments[dashed,thin](S,A A,B A,C A,D B,D S,O N,P M,Q M,N F,K Q,K K,K')
				\tkzDrawPoints[fill=black,size=3pt](S,A,B,C,D,M,N,P,O,E,Q,K,F,K')
				\tkzLabelPoints[above](S,N,K,t)
				\tkzLabelPoint[above](F){$ F\equiv F' $}
				\tkzLabelPoints[above left](A)
				\tkzLabelPoints[below](C,P,E)
				\tkzLabelPoints[right](D)
				\tkzLabelPoints[below left](Q)
				\tkzLabelPoints[below right](K',B)
				\tkzLabelPoints[above right](M,O)
			\end{tikzpicture}
		\end{center}
		\begin{listEX}
			\item Vì $ \heva{&S\in (SAD) \cap (SBC)\\&AD\subset (SAD) \text{ và } BC\subset (SBC) \\& AD\parallel BC} $\\ nên $ (SAD)\cap (SBC) = St \parallel AD \parallel BC $.\\
			Gọi $ O=AC\cap BD\Rightarrow \heva{&O\in AC\subset (SAC) \\ &O\in BD\subset (SBD)} $
			suy ra $ SO=(SAC)\cap (SBD) $.
			\item Gọi $ E=NP\cap AB $ và $ Q=EM\cap SB $.
			Vì $ \heva{&Q\in SB \\&Q\in ME\subset (MNP)} $ nên $ Q=SB\cap (MNP) $.
			\item Gọi $ F=MK\cap St $ và $ F'=QC\cap St $. Dựa vào các vị trí các điểm $ Q $, $ C $, $ M $ và $ K $ của giả thiết cho, dễ thấy $ F $ và $ F' $ cùng nằm về một phía so với mặt phẳng $ (SAB) $.\\
			Trong mặt phẳng $ (SF'BC) $, áp dụng định lý Thales (để ý rằng $ SF'\parallel BC $) ta có
			\[ \dfrac{QS}{QB}=\dfrac{BC}{SF'}=\dfrac{1}{2}. \tag {1}\]
			Gọi $ K' $ là trung điểm của $ SA $. suy ra $ \dfrac{MK'}{MS}=\dfrac{1}{2}. $\\
			Trong mặt phẳng $ (SFAD) $, áp dụng định lý Thales (để ý rằng $ SF\parallel KK' $) ta có
			\[ \dfrac{MK'}{MS}=\dfrac{KK'}{SF}=\dfrac{1}{2}. \tag {2}\]
			Từ (1), (2) và $ AD=2BC $ suy ra $ SF=SF' $. Do đó $ F\equiv F' $, suy ra bốn điểm $ Q $, $ C $, $ M $ và $ K $ đồng phẳng.\\
			Vậy $ CK=(MQK)\cap (SCD) $.
		\end{listEX}
	}
\end{bt}

\begin{bt}
	Cho hình chóp $ S.ABCD $ có $ O $ là tâm của hình bình hành $ ABCD $, điểm $ M $ thuộc cạnh $ SA $ sao cho $ SM=2MA $, $ N $ là trung điểm của $ AD $.
	\begin{itemize}
		\item[a)] Tìm giao tuyến của mặt phẳng $ (SAD) $ và $ (MBC) $.
		\item[b)] Tìm giao điểm $ I $ của $ SB $ và $ (CMN) $, giao điểm $ J $ của $ SA $ và $ (ICD) $.
		\item[c)] Chứng minh ba đường thẳng $ ID $, $ JC $, $ SO $ cắt nhau tại $ E $. Tính tỉ số $ \dfrac{SE}{SO} $.
	\end{itemize}
	\loigiai{
		\begin{center}
			\begin{tikzpicture}[line join=round, line cap=round,thick,scale=0.8]
				\tikzset{label style/.style={font=\footnotesize}}
				\coordinate (A) at (0,0);
				\coordinate (B) at (-2.5,-2);
				\coordinate (D) at (8,0);
				\coordinate (C) at ($(B)+(D)-(A)$);
				\coordinate (O) at ($(A)!0.5!(C)$);
				\coordinate (S) at ($(O)+(-1,7)$);
				\coordinate (M) at ($(S)!0.6667!(A)$);
				\coordinate (P) at ($(S)!0.6667!(D)$);
				\coordinate (N) at ($(A)!0.5!(D)$);
				\coordinate (S') at ($(S)+(A)-(D)$);
				\coordinate (t) at ($(S)+(P)-(M)$);
				\tkzInterLL(S,S')(M,N) \tkzGetPoint{F}
				\tkzInterLL(S,B)(C,F) \tkzGetPoint{I}
				\coordinate (I') at ($(I)+(D)-(C)$);
				\tkzInterLL(I,I')(S,A) \tkzGetPoint{J}
				\tkzInterLL(S,O)(I,D) \tkzGetPoint{E}
				\tkzDrawSegments(S,B S,C B,C C,F S,F C,D S,D P,C S,t)
				\tkzDrawSegments[dashed,thin](S,A A,B B,M A,C A,D S,O B,D M,P F,N C,N I,J I,D J,C C,M)
				\tkzDrawPoints[fill=black,size=3pt](S,A,B,C,D,O,M,N,P,F,I,J,E)
				\tkzLabelPoints[above](S,F,N,t)
				\tkzLabelPoints[above right](J,M,E)
				\tkzLabelPoints[left](A)
				\tkzLabelPoints[below left](I)
				\tkzLabelPoints[below](B,C,O)
				\tkzLabelPoints[right](D,P)
			\end{tikzpicture}
		\end{center}
		\begin{listEX}
			\item Vì $ \heva{& M\in (MBC) \cap (SAD) \\ & BC \subset (MBC) \text{ và } AD \subset (SAD) \\& BC\parallel AD}$\\ nên $ (SAD)\cap (MBC)=MP\parallel BC \parallel AD $ (với $ P\in SD $).
			\item Vì $ \heva{& S\in (SAD)\cap (SBC)\\& AD\subset (SAD) \text{ và } BC\subset (SBC) \\ &AD \parallel BC} $ \\ nên $ (SAD) \cap (SBC) =St \parallel AD \parallel BC $.\\
			Gọi $ F=MN\cap St $; $ I=CF\cap SB $.\\
			Vì $ \heva{&I\in SB \\ &I\in CF \subset (CMN)} $ nên $ I=SB\cap (CMN) $.\\
			Qua $ I $ kẻ đường thẳng song song với $ AB $  cắt $ SA $ tại $ J $.\\
			Vì $ \heva{&J\in SA \\ & J\in JI\subset (ICD) (\text{vì } IJ\parallel CD \Rightarrow (IJCD)\equiv (ICD))} $ nên $ J=SA\cap (ICD) $.
			\item Xét $ 3 $ mặt phẳng $ (SAC) $, $ (SBD) $ và $ (CDJI) $, ta có
			$$\heva{& SO=(SAC)\cap (SBD)\\& ID=(SBD)\cap (CDJI)\\& JC=(SAC)\cap (CDJI).}$$
			Do đó ba đường thẳng $ ID $, $ JC $, $ SO $ đồng quy. Gọi điểm đồng quy là $ E $.\\
			Trong mặt phẳng $ (SFAD) $, áp dụng định lý Thales (để ý rằng $ AN\parallel SF $) ta có
			\[ \dfrac{MA}{MS}=\dfrac{AN}{SF}=\dfrac{1}{2}. \]
			Suy ra $ SF=AD=BC $ và $ SFBC $ là hình bình hành.\\
			$ I=SB\cap CF $ nên $ I $ là trung điểm của $ SB $.\\
			$ \triangle SBD $ có $ DI $ và $ SO $ là trung tuyến nên $ E $ là trọng tâm của $ \triangle SBD $.\\
			Vậy $ \dfrac{SE}{SO}=\dfrac{2}{3} $.
		\end{listEX}
	}
\end{bt}
\begin{bt}
	Cho hình chóp $S.ABCD$ có đáy $ABCD$ là hình bình hành. Gọi $M, N, P, Q$ lần lượt là trung điểm của các cạnh $AB, BC, CD, DA$; gọi $I, J, K, L$ lần lượt là trung điểm của các đoạn thẳng $SM, SN, SP, SQ$.
	\begin{itemize}
		\item [a)] Chứng minh rằng bốn điểm $I, J, K, L$ đồng phẳng và tứ giác $IJKL$ là hình bình hành.
		\item [b)] Chứng minh rằng $IK\parallel BC$.
		\item [c)] Xác định giao tuyến của hai mặt phẳng $\left(IJKL\right)$ và $\left(SBC\right)$.
	\end{itemize}
\end{bt}


% \subsection{BÀI TẬP TRẮC NGHIỆM}
\Opensolutionfile{ans}[ans/1H4.B2]
\setcounter{ex}{0}

\begin{ex}%[Huỳnh Văn Quy]%[1H2Y2]
	Hai đường thẳng không có điểm chung thì
	\choice{chéo nhau}
	{song song}
	{cắt nhau}
	{\True chéo nhau hoặc song song}
\end{ex}

\begin{ex}%[Huỳnh Văn Quy]%[1H2Y2]
	Hai đường thẳng phân biệt không song song thì
	\choice{chéo nhau}
	{có điểm chung}
	{\True cắt nhau hoặc chéo nhau}
	{ không có điểm chung}
\end{ex}

\begin{ex}%[TLDH9-Lê Hồng Phi]%[1H2Y2-1]%
	Cho hai đường thẳng phân biệt không có điểm chung cùng nằm trong một mặt phẳng thì hai đường thẳng đó
	\choice
	{trùng nhau}
	{chéo nhau}
	{\True song song}
	{cắt nhau}
	\loigiai{
		Hai đường thẳng phân biệt không có điểm chung cùng nằm trong một mặt phẳng thì hai đường thẳng đó song song.}
\end{ex}

\begin{ex}%[Đề HKI Lớp 11, Lạc Long Quân, Khánh Hòa 2017-2018]%[Lê Hồng Phi - DA 11HK1-18]%[1H2Y2-1]%
	Chọn khẳng định {\bf sai}
	\choice
	{Hai đường thẳng phân biệt cùng song song với đường thẳng thứ ba thì song song với nhau}
	{Nếu hai đường thẳng chéo nhau thì chúng không đồng phẳng}
	{\True Hai đường thẳng song song thì không đồng phẳng và không có điểm chung}
	{Hai đường thẳng cắt nhau thì đồng phẳng và có một điểm chung}
	\loigiai{
		Khẳng định sai là \lq\lq  Hai đường thẳng song song thì không đồng phẳng và không có điểm chung\rq\rq.
	}
\end{ex}

\begin{ex}%[Ân Thọ]%[1H2B2]
	Cho đường thẳng $a$ cắt mặt phẳng $(P)$ tại điểm $A$. Mệnh đề nào sau đây đúng?
	\choice
	{Mọi đường thẳng nằm trong $(P)$ đều chéo với $a$}
	{Mọi đường thẳng nằm trong $(P)$ đều cắt $a$}
	{\True Mọi đường thẳng nằm trong $(P)$ hoặc chéo với $a$, hoặc cắt $a$}
	{Mọi đường thẳng nằm trong $(P)$ đều không cắt $a$}
\end{ex}

\begin{ex}%[Ân Thọ]%[1H2B2]
	% \immini[thm]{
		Cho tứ diện $ABCD$. Gọi $M$ và $N$ là hai điểm phân biệt nằm trên đường thẳng $AB$, $M'$ và $N'$ là hai điểm phân biệt nằm trên đường thẳng $CD$. Các mệnh đề sau đây, mệnh đề nào đúng?	
		\choice
		{Hai đường thẳng $MM'$ và $NN'$ có thể cắt nhau}
		{Hai đường thẳng $MM'$ và $NN'$ có thể song song với nhau}
		{Hai đường thẳng $MM'$ và $NN'$ hoặc cắt nhau hoặc song song với nhau}
		{\True Hai đường thẳng $MM'$ và $NN'$ chéo nhau}
	% }{\begin{tikzpicture}[line cap=round,line join=round,>=stealth,x=0.75cm,y=0.75cm]
	% 	\draw (0.0,-0.0)-- (2.0,-2.0);
	% 	\draw (2.0,-2.0)-- (5.0,0.0);
	% 	\draw (5.0,0.0)-- (3.0,3.0);
	% 	\draw (3.0,3.0)-- (0.0,-0.0);
	% 	\draw (3.0,3.0)-- (2.0,-2.0);
	% 	%\draw [dash pattern=on 2pt off 2pt] (1.5,1.5)-- (3.5,-1.0);
	% 	\draw [dash pattern=on 2pt off 2pt] (0.0,-0.0)-- (5.0,0.0);
	% 	\begin{scriptsize}
	% 	\draw [fill=black] (0.0,-0.0) circle (1pt);
	% 	\draw[color=black] (-0.06,0.2) node {$B$};
	% 	\draw [fill=black] (2.0,-2.0) circle (1pt);
	% 	\draw[color=black] (1.7,-1.95) node {$C$};
	% 	\draw [fill=black] (5.0,0.0) circle (1pt);
	% 	\draw[color=black] (5.07,0.15) node {$D$};
	% 	\draw [fill=black] (3.0,3.0) circle (1pt);
	% 	\draw[color=black] (3.2,3.0) node {$A$};
	% 	\end{scriptsize}
	% 	\end{tikzpicture}
	% }
\end{ex}

\begin{ex}%[Nguyễn Tiến Thùy]%[1H2B2]
	Cho tứ diện $ABCD$, lấy $M, N$ lần lượt là trung điểm của $CD, AB$. Khi đó, xác định vị trí tương đối giữa hai đường thẳng $BC$ và $MN$.
	\choice
	{\True Chéo nhau}
	{Có hai điểm chung}	
	{Song song}	
	{Cắt nhau}
\end{ex}

\begin{ex}%[Nguyễn Tiến Thùy]%[1H2B2]
	Cho tứ diện $MNPQ$. Mệnh đề nào trong các mệnh đề dưới đây là đúng?
	\choice
	{$MN\parallel PQ$}
	{$MN$ cắt $PQ$}
	{$MN$ và $PQ$ đồng phẳng}	
	{\True $MN$ và $PQ$ chéo nhau}
\end{ex}

\begin{ex}%[Nguyễn Tiến Thùy]%[1H2B2]
	Cho hình chóp $S.ABCD$, đáy $ABCD$ là hình bình hành. Điểm $M$ thuộc cạnh $SC$ sao cho $SM=3MC$, $N$ là giao điểm của $SD$ và $(MAB)$. Khi đó tứ giác $ABMN$ là hình gì?
	\choice
	{Tứ giác không có cặp cạnh nào song song}
	{Hình vuông}
	{\True Hình thang}
	{Hình bình hành}
\end{ex}

\begin{ex}%[Nguyễn Tiến Thùy]%[1H2B2]
	Cho hình chóp $S.ABCD$ có đáy $ABCD$ là hình thang $AB\parallel CD$. Gọi $d$ là giao tuyến của hai mặt phẳng $(ASB)$ và $(SCD)$. Khẳng định nào sau đây là đúng?
	\choice
	{\True $d\parallel AB$}
	{$d$ cắt $AB$}
	{$d$ cắt $AD$}	
	{$d$ cắt $CD$}
\end{ex}

\begin{ex}%[Huỳnh Văn Quy]%[1H2B2]
	Cho hình chóp $S.ABCD$. Gọi $G$, $E$ lần lượt là trọng tâm các tam giác $SAD$ và $SCD$. Lấy $M$, $N$ lần lượt là trung điểm $AB$, $BC$ . Khi đó ta có:
	\choice
	{$GE$ và $MN$ trùng nhau}	
	{$GE$ và $MN$ chéo nhau}
	{\True $GE$ và $MN$ song song với nhau}
	{$GE$ cắt $BC$}
	\loigiai{Gọi $I$ là trung điểm của $SD$. Xét tam giác $IAC$ có: $\dfrac{IG}{IA}=\dfrac{IE}{IC}=\dfrac{1}{3}$.
		\immini{Theo định lí đảo của định lí Thalet, ta có $GE||AC$\quad (1).\\
			Mặt khác, ta có $MN$ là đường trung bình của $\triangle ABC$ nên suy ra $MN||AC$\quad (2).\\
			Từ (1) và (2) ta có $MN||GE$}
		{\begin{tikzpicture}[scale=1.5,line cap=round,line join=round,>=stealth,x=1.0cm,y=1.0cm]
			\draw [line width=0.8pt,dash pattern=on 1pt off 1pt] (0.539,2.203)-- (2.201,2.151);
			\draw [line width=0.8pt] (0.539,2.203)-- (0.983,1.253);
			\draw [line width=0.8pt] (0.983,1.253)-- (1.757,1.439);
			\draw [line width=0.8pt] (1.757,1.439)-- (2.201,2.151);
			\draw [line width=0.8pt] (1.241,3.339)-- (0.539,2.203);
			\draw [line width=0.8pt] (1.241,3.339)-- (2.201,2.151);
			\draw [line width=0.8pt] (1.241,3.339)-- (1.757,1.439);
			\draw [line width=0.8pt] (1.241,3.339)-- (0.983,1.253);
			\draw [line width=0.8pt,dash pattern=on 1pt off 1pt] (0.539,2.203)-- (1.757,1.439);
			\draw [line width=0.8pt,dash pattern=on 1pt off 1pt] (0.761,1.728)-- (1.370,1.346);
			\draw [line width=0.8pt] (1.721,2.745)-- (1.757,1.439);
			\draw [line width=0.8pt] (1.979,1.795)-- (1.241,3.339);
			\draw [line width=0.8pt,dash pattern=on 1pt off 1pt] (1.370,2.177)-- (1.241,3.339);
			\draw [line width=0.8pt,dash pattern=on 1pt off 1pt] (1.327,2.565)-- (1.733,2.310);
			\draw [line width=0.8pt,dash pattern=on 1pt off 1pt] (0.539,2.203)-- (1.721,2.745);
			\begin{scriptsize}
			\draw [fill=black] (0.539,2.203) circle (1.0pt);
			\draw[color=black] (0.404,2.239) node {$A$};
			\draw [fill=black] (2.201,2.151) circle (1.0pt);
			\draw[color=black] (2.273,2.281) node {$D$};
			\draw [fill=black] (0.983,1.253) circle (1.0pt);
			\draw[color=black] (0.879,1.114) node {$B$};
			\draw [fill=black] (1.757,1.439) circle (1.0pt);
			\draw[color=black] (1.829,1.403) node {$C$};
			\draw [fill=black] (1.241,3.339) circle (1.0pt);
			\draw[color=black] (1.313,3.468) node {$S$};
			\draw [fill=black] (1.721,2.745) circle (1.0pt);
			\draw[color=black] (1.798,2.869) node {$I$};
			\draw [fill=black] (1.370,1.346) circle (1.0pt);
			\draw[color=black] (1.406,1.238) node {$N$};
			\draw [fill=black] (0.761,1.728) circle (1.0pt);
			\draw[color=black] (0.559,1.671) node {$M$};
			\draw [fill=black] (1.733,2.310) circle (1.0pt);
			\draw[color=black] (1.809,2.435) node {$E$};
			\draw [fill=black] (1.327,2.565) circle (1.0pt);
			\draw[color=black] (1.189,2.663) node {$G$};
			\end{scriptsize}
			\end{tikzpicture}}}
\end{ex}

\begin{ex}%[Ân Thọ]%[1H2B2]
	\immini[thm]{Cho tứ diện $ABCD$ có $P, Q$ lần lượt là trọng tâm tam giác $ABC$ và $BCD$. Xác định giao tuyến của mặt phẳng $(ABQ)$ và mặt phẳng $(CDP)$.
		\choice
		{\True Giao tuyến là đường thẳng đi qua trung điểm hai cạnh $AB$ và $CD$}
		{Giao tuyến là đường thẳng đi qua trung điểm hai cạnh $AB$ và $AD$}
		{Giao tuyến là đường thẳng $PQ$}
		{Giao tuyến là đường thẳng $QA$}}
	{\begin{tikzpicture}[line cap=round,line join=round,>=stealth,x=0.75cm,y=0.75cm]
		\draw (0.0,-0.0)-- (2.0,-2.0);
		\draw (2.0,-2.0)-- (5.0,0.0);
		\draw (5.0,0.0)-- (3.0,3.0);
		\draw (3.0,3.0)-- (0.0,-0.0);
		\draw (3.0,3.0)-- (2.0,-2.0);
		%\draw [dash pattern=on 2pt off 2pt] (1.5,1.5)-- (3.5,-1.0);
		\draw [dash pattern=on 2pt off 2pt] (0.0,-0.0)-- (5.0,0.0);
		\begin{scriptsize}
		\draw [fill=black] (0.0,-0.0) circle (1pt);
		\draw[color=black] (-0.06,0.2) node {$B$};
		\draw [fill=black] (2.0,-2.0) circle (1pt);
		\draw[color=black] (1.7,-1.95) node {$C$};
		\draw [fill=black] (5.0,0.0) circle (1pt);
		\draw[color=black] (5.07,0.15) node {$D$};
		\draw [fill=black] (3.0,3.0) circle (1pt);
		\draw[color=black] (3.2,3.0) node {$A$};
		\end{scriptsize}
		\end{tikzpicture}
	}
\end{ex}

\begin{ex}%[1H2B2-2]
	Cho tứ diện $ABCD.$ Gọi $I$ và $J$ theo thứ tự là trung điểm của $AD$ và $AC,G$ là trọng tâm tam giác $BCD.$ Giao tuyến của hai mặt phẳng $\left(GIJ\right)$ và $\left(BCD\right)$ là đường thẳng
	\choice
	{qua $J$ và song song với $BD$}
	{qua $G$ và song song với $BC$}
	{qua $I$ và song song với $AB$}
	{\True qua $G$ và song song với $CD$}
	\loigiai{
		\immini{
			Ta có $\heva{
				& \left(GIJ\right)\cap \left(BCD\right)=G \\
				& IJ\subset \left(GIJ\right),\ CD\subset \left(BCD\right) \\
				& IJ\parallel CD}$\\
			$\Rightarrow \left(GIJ\right)\cap \left(BCD\right)=Gx\parallel IJ\parallel CD.$
		}
		{
			\begin{tikzpicture}[scale=0.7, line join=round, line cap=round]
			\tkzDefPoints{0/0/C,1.5/-1.8/B,4/0/D,2/3/A}
			\tkzDefMidPoint(A,C)\tkzGetPoint{J}
			\tkzDefMidPoint(D,A)\tkzGetPoint{I}
			\tkzCentroid(D,B,C)\tkzGetPoint{G}
			\tkzDefLine[parallel=through G](D,C)\tkzGetPoint{g}
			\tkzInterLL(g,G)(C,B)\tkzGetPoint{E}
			\tkzInterLL(g,G)(D,B)\tkzGetPoint{F}
			\tkzDrawPolygon(A,C,B,D)
			\tkzDrawSegments(A,B)
			\tkzDrawSegments[dashed](C,D I,J E,F)
			\tkzDrawPoints[fill=black](I,A,B,C,D,J,G)
			\tkzLabelPoints[above](A)
			\tkzLabelPoints[below](B,G)
			\tkzLabelPoints[left](C,J)
			\tkzLabelPoints[right](D,I)
			\end{tikzpicture}
		}
	}
\end{ex}

\begin{ex}%[1H2B2-2]
	Cho hình chóp $S.ABCD$ có đáy là hình thang với các cạnh đáy là $AB$ và $CD.$ Gọi $\left(ACI\right)$ lần lượt là trung điểm của $AD$ và $BC$ và $G$ là trọng tâm của tam giác $SAB.$ Giao tuyến của $\left(SAB\right)$ và $\left(IJG\right)$ là
	\choice
	{đường thẳng qua $S$ và song song với $AB$}
	{\True đường thẳng qua $G$ và song song với $DC$}
	{$SC$}
	{đường thẳng qua $G$ và cắt $BC$}
	\loigiai{
		\immini{
			Ta có: $I,J$ lần lượt là trung điểm của $AD$ và $BC$
			$\Rightarrow IJ$ là đường trung bình của hình thang $ABCD\Rightarrow IJ\parallel AB\parallel CD.$ \\
			Gọi $d=\left(SAB\right)\cap \left(IJG\right)$
			Ta có: $G$ là điểm chung giữa hai mặt phẳng $\left(SAB\right)$ và $\left(IJG\right)$ \\
			Mặt khác: $\heva{
				& \left(SAB\right)\supset AB;\left(IJG\right)\supset IJ \\
				& AB\parallel IJ \\}$ \\
			$\Rightarrow $Giao tuyến $d$ của $\left(SAB\right)$ và $\left(IJG\right)$ là đường thẳng qua $G$ và song song với $AB$ và $IJ.$
		}
		{
			\begin{tikzpicture}[scale=1, line join=round, line cap=round]
			\tkzDefPoints{0/0/A,0.5/-2/D,3.5/-2/C,5/0/B,2/3/S}
			\tkzDefMidPoint(B,C)\tkzGetPoint{J}
			\tkzDefMidPoint(D,A)\tkzGetPoint{I}
			\tkzCentroid(S,B,A)\tkzGetPoint{G}
			\tkzDefLine[parallel=through G](A,B)\tkzGetPoint{g}
			\tkzInterLL(g,G)(S,B)\tkzGetPoint{Q}
			\tkzInterLL(g,G)(S,A)\tkzGetPoint{P}
			\tkzDrawPolygon(S,A,D,C,B)
			\tkzDrawSegments(S,D S,C)
			\tkzDrawSegments[dashed](A,B I,J P,Q I,G J,G)
			\tkzDrawPoints[fill=black](I,A,B,C,D,J,G,S,P,Q)
			\tkzLabelPoints[above](S)
			\tkzLabelPoints[above left](G)
			\tkzLabelPoints[below](C,D)
			\tkzLabelPoints[left](A,I,P)
			\tkzLabelPoints[right](B,J,Q)
			\end{tikzpicture}
		}
	}
\end{ex}

\begin{ex}%[1H2B2-2]
	Cho hình chóp $S.ABCD$ có đáy $ABCD$ là hình bình hành. Gọi $I,J,E,F$ lần lượt là trung điểm $SA,SB,SC,SD.$ Trong các đường thẳng sau, đường thẳng nào không song song với $IJ?$
	\choice
	{$DC$}
	{$AB$}
	{\True $AD$}
	{$EF$}
	\loigiai{
		\immini{
			Ta có $IJ\parallel AB$ (tính chất đường trung bình trong tam giác $SAB$) và $EF\parallel CD$ (tính chất đường trung bình trong tam giác $SCD$).\\
			Mà $CD\parallel AB$ (đáy là hình bình hành)\\ $\Rightarrow CD\parallel AB\parallel EF\parallel IJ.$
		}
		{
			\begin{tikzpicture}[scale=0.7, line join=round, line cap=round]
			\tkzDefPoints{0/0/A,-1.5/-1.8/B,4/0/D, 0.2/3/S}
			\coordinate (C) at ($(B)+(D)-(A)$);
			\tkzDefMidPoint(A,S)\tkzGetPoint{I}
			\tkzDefMidPoint(B,S)\tkzGetPoint{J}
			\tkzDefMidPoint(C,S)\tkzGetPoint{E}
			\tkzDefMidPoint(D,S)\tkzGetPoint{F}
			\tkzDrawPolygon(S,B,C,D)
			\tkzDrawSegments(S,C E,F)
			\tkzDrawSegments[dashed](S,A A,B A,C B,D A,D I,J)
			\tkzDrawPoints[fill=black](S,A,B,C,D,I,J,E,F)
			\tkzLabelPoints[above](S)
			\tkzLabelPoints[above right](F)
			\tkzLabelPoints[below](B,C)
			\tkzLabelPoints[left](A,J)
			\tkzLabelPoints[right](D,I)
			\end{tikzpicture}
		}
	}
\end{ex}

\begin{ex}%[1H2B2-2]
	Cho hình chóp $S.ABCD$ có đáy $ABCD$ là hình bình hành. Gọi $d$ là giao tuyến của hai mặt phẳng $\left(SAD\right)$ và $\left(SBC\right)$. Khẳng định nào sau đây đúng?
	\choice
	{$d$ qua $S$ và song song với $DC$}
	{$d$ qua $S$ và song song với $BD$}
	{\True $d$ qua $S$ và song song với $BC$}
	{$d$ qua $S$ và song song với $AB$}
	\loigiai{
		\immini{
			Ta có $\heva{
				& \left(SAD\right)\cap \left(SBC\right)=S \\
				& AD\subset \left(SAD\right),BC\subset \left(SBC\right) \\
				& AD\parallel BC \\}$\\
			$\Rightarrow \left(SAD\right)\cap \left(SBC\right)=Sx\parallel AD\parallel BC$ (với $d\equiv Sx$).
		}
		{
			\begin{tikzpicture}[scale=0.6, line join=round, line cap=round]
			\tkzDefPoints{0/0/A,-1.5/-1.8/B,4/0/D, 0.2/3/S}
			\coordinate (C) at ($(B)+(D)-(A)$);
			\tkzDefLine[parallel=through S](B,C)\tkzGetPoint{K}
			\tkzDrawLines[add = .2 and .2,color=black](S,K)
			\tkzDrawPolygon(S,B,C,D)
			\tkzDrawSegments(S,C)
			\tkzDrawSegments[dashed](S,A A,B A,D)
			\tkzDrawPoints[fill=black](S,A,B,C,D)
			\tkzLabelPoints[above](S)
			\tkzLabelPoints[below](B,C)
			\tkzLabelPoints[left](A)
			\tkzLabelPoints[right](D)
			\end{tikzpicture}
		}
	}
\end{ex}

\begin{ex}%[1H2B2-1]
	Cho tứ diện $ABCD.$ Gọi $M,N$ là hai điểm phân biệt cùng thuộc đường thẳng $AB$. $P,Q$ là hai điểm phân biệt cùng thuộc đường thẳng $CD.$ Xét vị trí tương đối của hai đường thẳng $MP,NQ.$
	\choice
	{$MP\parallel NQ$}
	{$MP$ cắt $NQ$}
	{$MP$ trùng $NQ$}
	{\True $MP,NQ$ chéo nhau}
	\loigiai{
		\immini{
			Xét mặt phẳng $\left(ABP\right).$ \\
			Ta có: $M,N$ thuộc $AB\Rightarrow M,N$ thuộc mặt phẳng $\left(ABP\right).$ \\
			Mặt khác: $CD\cap \left(ABP\right)=P.$ \\
			Mà: $Q\in CD\Rightarrow Q\notin \left(ABP\right)$\\
			$\Rightarrow M,N,P,Q$ không đồng phẳng.
		}
		{
			\begin{tikzpicture}[scale=0.7, line join=round, line cap=round]
			\tkzDefPoints{0/0/B,1.5/-1.8/C,4/0/D,2/3/A}
			\coordinate (M) at ($(A)!0.3!(B)$);
			\coordinate (N) at ($(A)!0.6!(B)$);
			\coordinate (P) at ($(C)!0.3!(D)$);
			\coordinate (Q) at ($(C)!0.6!(D)$);
			\tkzDrawPolygon(A,B,C,D)
			\tkzDrawSegments(A,P A,C)
			\tkzDrawSegments[dashed](B,D B,P N,Q M,P)
			\tkzDrawPoints[fill=black](M,A,B,C,D,N,P,Q)
			\tkzLabelPoints[above](A)
			\tkzLabelPoints[below](C)
			\tkzLabelPoints[left](B)
			\tkzLabelPoints[right](D)
			\tkzLabelPoints[above left](M,N)
			\tkzLabelPoints[below right](P,Q)
			\end{tikzpicture}
		}
	}
\end{ex}


\begin{ex}%[1H2K2-2]
	Cho hình chóp $S.ABCD$ có đáy $ABCD$ là hình thang với đáy lớn $AB$ đáy nhỏ $CD.$ Gọi $M,N$ lần lượt là trung điểm của $SA$ và $SB.$ Gọi $P$ là giao điểm của $SC$ và $\left(AND\right).$ Gọi $I$ là giao điểm của $AN$ và $DP.$ Hỏi tứ giác $SABI$ là hình gì?
	\choice
	{\True Hình bình hành}
	{Hình thoi}
	{Hình vuông}
	{Hình chữ nhật}
	\loigiai{
		\immini{
			Gọi $E=AD\cap BC, P=NE\cap SC$. Suy ra $P=SC\cap \left(AND\right)$.\\
			Ta có $S$ là điểm chung thứ nhất của hai mặt phẳng $\left(SAB\right)$ và $\left(SCD\right)$; $I=DP\cap AN\Rightarrow I$ là điểm chung thứ hai của hai mặt phẳng $\left(SAB\right)$ và $\left(SCD\right).$\\
			Suy ra $SI=\left(SAB\right)\cap \left(SCD\right)$.\\
			Mà $AB\parallel CD\Rightarrow SI\parallel AB\parallel CD.$
			Vì $MN$ là đường trung bình của tam giác $SAB$ và chứng minh được cũng là đường trung bình của tam giác $SAI$ nên suy ra $SI=AB$.\\
			Vậy $SABI$ là hình bình hành.
		}{
			\begin{tikzpicture}
			\tkzDefPoints{0/0/A,4/0/B,3/-1.5/C,1.5/-1.5/D,1.5/3/S}
			\coordinate (N) at ($(S)!0.5!(A)$);
			\coordinate (M) at ($(S)!0.5!(B)$);
			\coordinate (I) at ($2*(M)-(A)$);
			\tkzInterLL(A,D)(B,C) \tkzGetPoint{E}
			\tkzInterLL(S,C)(D,I) \tkzGetPoint{P}
			\tkzInterLL(S,B)(D,I) \tkzGetPoint{m}
			\draw (S)--(A)--(E)--(B) (S)--(C) (S)--(D)--(I)--(S) (B)--(m) (E)--(P);
			\draw[dashed] (I)--(A)--(B) (C)--(D) (N)--(M)--(P) (S)--(m);
			\tkzDrawPoints(A,B,C,D,E,M,N,P,S,I)
			\tkzLabelPoints[above](S,I,M)
			\tkzLabelPoints[right](B,C,P)
			\tkzLabelPoints[left](A,N)
			\tkzLabelPoints[below left=-0.1](D,E)
			\end{tikzpicture}
		}
	}
\end{ex}

\begin{ex}%[1H2B2-4]%
	Cho hình chóp $ S.ABCD $  có đáy $ABCD$ là hình bình hành. Gọi $ M, N, P, Q $  lần lượt là trung điểm của các cạnh $ SA, SB, SC, SD $. Gọi $I$ là một điểm trên cạnh $B $. Thiết diện của hình chóp với mặt phẳng $(IMN)$ là hình gì?
	\choice
	{Tam giác $MNQ$}
	{Tam giác $MNI$}
	{\True Hình thang $MNIJ$}
	{Hình bình hành $MNIJ$}
	\loigiai{
		\immini{Ta có $ MN\parallel AB $; $ MN=\dfrac{1}{2}AB$ và $PQ \parallel CD; PQ=\dfrac{1}{2}CD$. Từ đó, suy ra $MN=PQ$ và $ MN\parallel PQ$.\\
			Vậy $ MNPQ$  là hình bình hành.\\
			Ta có $\heva{&I\in (IMN)\cap (ABCD)\\&AB\subset (ABCD)\\&MN\subset (IMN)\\&AB\parallel MN}$.
			$\Rightarrow (IMN)\cap (ABCD)=IJ\parallel AB \parallel MN \text{ với } J\in AD$.
			Thiết diện của hình chóp cắt bởi mặt phẳng $ (IMN) $ là hình thang $ MNIJ $. }
		{\begin{tikzpicture}[line join=round,line cap=round,line width=.6pt,font=\footnotesize,scale=0.8]
				\coordinate[label=below left:$B$] (B) at (-0.5,-1);
				\coordinate[label=left:$A$] (A) at (1,.8);
				\coordinate[label=below right:$C$] (C) at (3.5,-1);
				\coordinate[label=above right:$D$] (D) at ($(C)-(B)+(A)$);
				\coordinate[label=above left:$S$] (S) at (1.7,4);
				\draw (S)--(B)--(C)--(D)--(S)--cycle (S)--(C);
				\draw[dashed] (A)--(D) (S)--(A)--(B);
				%\draw ($ (A)!5pt!(D)$)--($(A)!2!($($(A)!5pt!(D)$)!.5!($(A)!5pt!(S)$)$)$)--($(A)!5pt!(S)$);
				\fill (A)circle(1.5pt) (B)circle(1.5pt) (C)circle(1.5pt) (D)circle(1.5pt) (S)circle(1.5pt);
				\coordinate[label=above right:$M$] (M) at ($(S)!0.5!(A)$);
				\coordinate[label=above left:$N$] (N) at ($(S)!0.5!(B)$);
				\coordinate[label=below left:$P$] (P) at ($(S)!0.5!(C)$);
				\coordinate[label=above:$Q$] (Q) at ($(S)!0.5!(D)$);
				\draw(N)--(P)--(Q); \draw[dashed](Q)--(M)--(N);
				\fill(M) circle (1.5pt) (N)circle (1.5pt) (P) circle (1.5pt) (Q) circle (1.5pt);
				\coordinate[label=below:$I$] (I) at ($(B)!0.75!(C)$);
				\coordinate[label=below:$J$] (J) at ($(A)!0.75!(D)$);
				\fill(I) circle (1.5pt) (J) circle (1.5pt) ;
				\draw(N)--(I); \draw[dashed](M)--(J)--(I);
	\end{tikzpicture}}}
\end{ex}

\begin{ex}%[1H2B2-4]%
	Cho hình chóp $S.ABCD$ có đáy $ABCD$ là hình bình hành. Gọi $I$ là trung điểm $SA$. Thiết diện của hình chóp $S.ABCD$ cắt bởi mặt phẳng $(IBC)$ là
	\choice
	{Tam giác $IBC$}
	{Tứ giác $IBCD$}
	{Hình thang $IGBC$ ($G$ là trung điểm $SB$)}
	{\True Hình thang $IBCJ$ ($J$ là trung điểm $SD$)}
	\loigiai{
		\immini{Ta có $\heva{
				& (IBC)\cap (SAD)=I \\
				& BC\subset (IBC), AD\subset (SAD) \\
				& BC\parallel AD}$\\
			Suy ra $(IBC)\cap (SAD)=Ix\parallel BC\parallel AD$.\\
			Trong mặt phẳng $(SAD)$ ta có $Ix\parallel AD$.\\
			Gọi $Ix\cap SD=J\Rightarrow $$IJ\parallel BC$.
			Vậy thiết diện của hình chóp $S.ABCD$ cắt bởi mặt phẳng $(IBC)$ là hình thang $IBCJ$.}
		{\begin{tikzpicture}[line cap=round,line join=round, >=stealth,scale=0.8]
				\tkzDefPoints{0/0/A,-2/-2/B,2/-2/C,4/0/D,-0.2/3/S}
				\tkzDefMidPoint(S,A)    \tkzGetPoint{I}
				\tkzDefMidPoint(S,D)    \tkzGetPoint{J}
				\tkzFillPolygon[green!50,opacity=0.25](B,C,J,I)
				\draw [dashed] (S)--(A)--(D)--(B)--(A)--(C) (B)--(I)--(J);
				\draw (S)--(B)--(C)--(D)--(S)--(C)--(J);
				\tkzDrawPoints[fill=black](A,B,C,D,S,I,J)
				\tkzLabelPoints[right](C,D,J)
				\tkzLabelPoints[left](A,B,I)
				\tkzLabelPoints[above](S)
		\end{tikzpicture}}
	}
\end{ex}

% \centerline{---HẾT---}
\Closesolutionfile{ans}
%%Bài 12. ĐT song song MP
% \setcounter{section}{11}
\setcounter{dang}{0}
\section{ĐƯỜNG THẲNG VÀ MẶT PHẲNG SONG SONG}
\subsection{KIẾN THỨC CẦN NHỚ}
\subsubsection{VỊ TRÍ TƯƠNG ĐỐI CỦA ĐƯỜNG THẲNG VÀ MẶT PHẲNG}
Cho đường thẳng $a$ và mặt phẳng $(P) $. Căn cứ vào số điểm chung của đường thẳng và mặt phẳng, ta có ba trường hợp sau:\\
\begin{tabular}{ccc}
	\begin{tikzpicture}[scale=.6]
		\tkzDefPoints{0/0/A', 4/0/B', 1/2/D', 0/2.5/M}
		\coordinate (C') at ($(B')+(D')-(A')$);
		\tkzDefPointBy[translation = from A' to B'](M) \tkzGetPoint{N}
		\tkzLabelSegment[pos=.3](M,N){ $a$}
		%\tkzDrawSegments[dashed](A,A' A',B' A',C')
		\tkzDrawPolygon(A',B',C',D')
		\tkzDrawSegments(M,N)
		%\tkzLabelPoints[right](B)
		%\tkzLabelPoints[above](z,y)
		\tkzMarkAngles[size=1](B',A',D')
		\tkzLabelAngle[pos=0.6](D',A',B'){\footnotesize $P$ }
		%\tkzDrawPoints(M,N,A)
	\end{tikzpicture}
	&\begin{tikzpicture}[scale=.6]
		\tkzDefPoints{0/0/A', 4/0/B', 1/2/D', 2/1/A, 0/2.5/B}
		\coordinate (C') at ($(B')+(D')-(A')$);
		\tkzLabelSegment[pos=.3](A,B){ $a$}
		\tkzInterLL(A,B)(A',B')\tkzGetPoint{I}
		\tkzInterLL(A,B)(C',B')\tkzGetPoint{J}
		\tkzDrawPolygon(A',B',C',D')
		\tkzDrawSegments(A,B I,J)
		\tkzDrawSegments[dashed](A,I)
		%\tkzLabelPoints[right](B)
		\tkzLabelPoints[above right](A)
		\tkzMarkAngles[size=1](B',A',D')
		\tkzLabelAngle[pos=0.6](D',A',B'){\footnotesize $P$ }
		\tkzDrawPoints(A)
	\end{tikzpicture}
	&\begin{tikzpicture}[scale=.6]
		\tkzDefPoints{0/0/A', 4/0/B', 1/2/D', 1/1/A, 4/1/B}
		\coordinate (C') at ($(B')+(D')-(A')$);
		\tkzLabelSegment[pos=.6](A,B){$a$}
		%\tkzDrawSegments[dashed](A,A' A',B' A',C')
		\tkzDrawPolygon(A',B',C',D')
		\tkzDrawSegments(A,B)
		%\tkzLabelPoints[right](B)
		\tkzLabelPoints[above](A,B)
		\tkzMarkAngles[size=1](B',A',D')
		\tkzLabelAngle[pos=0.6](D',A',B'){\footnotesize $P$ }
	\end{tikzpicture}\\
	$a \parallel (P)$ & $a \cap (P)=A $ & $a \subset (P)$
\end{tabular}

\subsubsection{CÁC ĐỊNH LÝ VÀ HỆ QUẢ CẦN NHỚ}
\begin{enumerate}[\iconMT]
	\immini{\item \indam{Định lý 1:} Nếu đường thẳng $a$ không nằm trong mặt phẳng $(P)$ và song song với một đường thẳng nào đó trong $(P)$ thì $a$ song song với $(P)$, hay
	$$a\not\subset (P) \text{ và } \heva{&a\parallel d \\&d\subset (P)\\}\Rightarrow a\parallel (P)$$}{\hspace{1cm}
	\begin{tikzpicture}[scale=0.57]
		\tkzDefPoints{0/0/A', 4/0/B', 1/2/D', 1/1/A, 4/1/B, 0/2.5/M}
		\coordinate (C') at ($(B')+(D')-(A')$);
		\tkzDefPointBy[translation = from A' to B'](M) \tkzGetPoint{N}
		\tkzLabelSegment[pos=.3](M,N){ $a$}
		\tkzLabelSegment[pos=.3](A,B){ $d$}
		%\tkzDrawSegments[dashed](A,A' A',B' A',C')
		\tkzDrawPolygon(A',B',C',D')
		\tkzDrawSegments(A,B M,N)
		%\tkzLabelPoints[right](B)
		%        \tkzLabelPoints[above](A,B)
		\tkzMarkAngles[size=0.8](B',A',D')
		\tkzLabelAngle[pos=0.5](D',A',B'){\footnotesize $P$ }
	\end{tikzpicture}
}
	\immini{
		\item \indam{Định lý 2:}Cho hai đường thẳng chéo nhau. Có duy nhất một mặt phẳng chứa đường thẳng này và song song với đường thẳng kia.
	}{	\begin{tikzpicture}[>=stealth,scale=0.45, line join=round, line cap=round]
	\tikzset{label style/.style={font=\footnotesize}}
	\tkzDefPoints{0/0/A, 6/0/B, 5/-3/C, -1/-3/D, 1/-1/E, 5/-1/b', 4/-2.5/a, 2/-1/M, 1/-0.5/Q, 1/1/L, 5/1/b}	
	\tkzDrawSegments(A,B B,C C,D D,A E,b' a,Q L,b)		
	\tkzLabelPoints[above](M,b',b,a)	
	\tkzMarkAngles[size=0.8cm](C,D,A)
	\tkzLabelAngles[pos=0.5,rotate=30](C,D,A){\scriptsize$\alpha$}
\end{tikzpicture}}
	\immini{\item \indam{Định lý 3:} Nếu đường thẳng $a$ song song với mặt phẳng $(\alpha)$. Nếu mặt phẳng $(\beta)$ chứa $a$ và cắt $(\alpha)$ theo giao tuyến $b$ thì $b$ song song với $a$.}{
	\begin{tikzpicture}[>=stealth,scale=0.46, line join=round, line cap=round]
		\tikzset{label style/.style={font=\footnotesize}}
		\tkzDefPoints{0/0/A,6/0/B, 4/-2/C,-2/-2/D,1/0/I, 4/0/J, 5/-1/L, 2/-1/M, 2/2/H, -1/2/K, 0.5/1/E, 2.5/1/F, 2/1/a, 3/-1/b}	
		\tkzDrawSegments(A,I J,B B,C C,D D,A M,K L,H H,K E,F)
		\tkzDrawSegments[thick,blue](M,L)
		\tkzDrawSegments[dashed](I,J)
		\tkzLabelPoints[above](a,b)	
		\tkzMarkAngles[size=0.8cm](C,D,A)
		\tkzLabelAngles[pos=0.5,rotate=30](C,D,A){\scriptsize $\alpha$}
		\tkzMarkAngles[size=1cm](M,K,H)
		\tkzLabelAngles[pos=0.6,rotate=340](M,K,H){\scriptsize $\beta$}
\end{tikzpicture}}
\immini{
	\begin{note}
		Nếu hai mặt phẳng phân biệt cùng song song với một đường thẳng thì giao tuyến của chúng (nếu có) cũng song song với đường thẳng đó.
	\end{note}
}{	\begin{tikzpicture}[>=stealth,scale=0.46, line join=round, line cap=round]
		\tikzset{label style/.style={font=\footnotesize}}
		\tkzDefPoints{0/0/A, 4/0/B, 4/5/C, 0/5/D, 2/4/E, 2/0/I, 2/-1/F, 5/4/M, 5/-1/N, 0/2/d', 5/2/d}	
		\tkzDrawSegments(A,D  A,F  F,E  E,D  I,B  B,C  C,D  M,N)
		\tkzDrawSegments[dashed](A,I)
		\tkzLabelPoints[right](d)	
		\tkzLabelPoints[left](d')
		\tkzMarkAngles[size=0.8cm](E,F,A)
		\tkzLabelAngles[pos=0.5,rotate=30](E,F,A){\scriptsize $\alpha$}
		\tkzMarkAngles[size=0.8cm](C,B,A)
		\tkzLabelAngles[pos=0.5,rotate=30](C,B,A){\scriptsize $\beta$}
\end{tikzpicture}}	
\end{enumerate}

		
	

% \subsection{PHÂN LOẠI, PHƯƠNG PHÁP GIẢI TOÁN}
\begin{dang}{Chứng minh đường thẳng song song với mặt phẳng}
	% \begin{enumerate}[\iconMT]
		\indam{Phương pháp giải:} Để chứng minh đường thẳng $a$ song song với mặt phẳng $(P)$, ta cần chứng tỏ các ý sau đây
		\immini{
			\begin{itemize}
				\item [$\bullet$] $a$ không nằm trên $(P)$;
				\item [$\bullet$] $a$ song song với một đường thẳng $b$ nằm trong $(P)$. Suy ra $a\parallel (P)$.
				\item [] Tóm lại \quad \quad \fbox{$\heva{&a\not\subset (P)\\&a\parallel b \\&b\subset (P)\\}\Rightarrow a\parallel (P)$}
			\end{itemize}
		}{\vspace{1cm}
			\begin{tikzpicture}[scale=0.7]
				\tikzset{label style/.style={font=\footnotesize}}
				\tkzDefPoints{0/0/A', 4/0/B', 1/2/D', 1/1/A, 4/1/B, 0/2.5/M}
				\coordinate (C') at ($(B')+(D')-(A')$);
				\tkzDefPointBy[translation = from A' to B'](M) \tkzGetPoint{N}
				\tkzLabelSegment[pos=.3](M,N){ $a$}
				\tkzLabelSegment[pos=.3](A,B){ $d$}
				%\tkzDrawSegments[dashed](A,A' A',B' A',C')
				\tkzDrawPolygon(A',B',C',D')
				\tkzDrawSegments(A,B M,N)
				%\tkzLabelPoints[right](B)
				%        \tkzLabelPoints[above](A,B)
				\tkzMarkAngles[size=0.8](B',A',D')
				\tkzLabelAngle[pos=0.5](D',A',B'){\footnotesize $P$ }
		\end{tikzpicture}}
		% \item \indam{Chú ý:} Việc chứng minh $a \parallel b$, ta thường đi đến việc xét các yếu tố song song đã biết trong hình học phẳng như
		% 	\immini{\begin{listEX}[1]
		% 			\item [\ding{172}] Cặp cạnh đối của hình bình hành;
		% 			\item [\ding{173}] Đường trung bình trong tam giác;
		% 			\item [\ding{174}] Tỉ lệ $\dfrac{AM}{AB}=\dfrac{AN}{AC}\Rightarrow MN \parallel BC$ (hình bên). Đặc biệt cần chú ý tỉ lệ trọng tâm của tam giác.
		% 		\end{listEX}
		% 	}{
		% 		\begin{tikzpicture}[scale=0.8, line join=round, line cap=round]
		% 			\tkzDefPoints{0/0/B,5/0/C,1.5/2/A}
		% 			\coordinate (M) at ($(A)!0.3!(B)$);
		% 			\coordinate (N) at ($(A)!0.3!(C)$);
		% 			\tkzDrawPoints[size=5,fill=black](A,B,C,M,N)
		% 			\tkzDrawSegments(M,N)
		% 			\tkzDrawPolygon(A,B,C)
		% 			\tkzLabelPoints[below,font=\footnotesize](B,C)
		% 			\tkzLabelPoints[above,font=\footnotesize](A)
		% 			\tkzLabelPoints[left,font=\footnotesize](M)
		% 			\tkzLabelPoints[right,font=\footnotesize](N)
		% 	\end{tikzpicture}}
		% \end{enumerate}
\end{dang}

\begin{vd}
Cho tứ diện $ABCD$. Gọi $M$ và $N$ lần lượt là trọng tâm của các tam giác $ACD$ và $BCD$. Chứng minh rằng $MN$ song song với các mặt phẳng $(ABC)$ và $(ABD)$.
	\loigiai{
		\immini
		{Gọi $P$, $Q$ lần lượt là trung điểm của $BC$ và $CD$.\\
			Khi đó, ta có $\dfrac{QM}{MA}=\dfrac{QN}{NB}=\dfrac{1}{3}\Rightarrow MN \parallel  AB$.\\
			Vì $\heva{&MN \not\subset (ABC)\\&AB\subset (ABC)\\&MN\parallel AB}$ nên $MN \parallel (ABC)$.	\\
			Tương tự, ta có $\heva{&MN \not\subset (ABD)\\&AB\subset (ABD)\\&MN\parallel AB}$ nên $MN \parallel (ABD)$.	}
		{\begin{tikzpicture}[scale=1, fill=black]
			\coordinate (A) at (1,4);
			\coordinate (B) at (-1,0);
			\coordinate (D) at (4,-1);
			\coordinate (C) at (1,-2);
			\coordinate (P) at ($(B)!0.5!(C)$);
			\coordinate (Q) at ($(D)!0.5!(C)$);
			\coordinate (M) at ($(A)!0.666!(Q)$);
			\coordinate (N) at ($(B)!0.666!(Q)$);
			\tkzLabelPoints[above](A)
			\tkzLabelPoints[left](B,P)
			\tkzLabelPoints[below=0.1](N)
			\tkzLabelPoints[right](D,M)
			\tkzLabelPoints[below](C,Q)
			\tkzDrawSegments(A,B A,C A,D B,C C,D A,Q)
			\tkzDrawSegments[dashed](B,D B,Q M,N)
			\end{tikzpicture}}
	}
\end{vd}
\begin{vd}
	Cho tứ diện $ABCD$. Gọi $G$ là trọng tâm tam giác $ABD$, điểm $I$ nằm trên cạnh $BC$ sao cho $BI=2IC$. Chứng minh rằng $IG$ song song $\left(ACD\right)$.
\end{vd}
\begin{vd}
	Cho hình chóp $S{.}ABCD$ có đáy $ABCD$ là hình bình hành. Lấy $M$ nằm trên cạnh $AD$ sao cho $AD=3AM$. Gọi $G, N$ lần lượt là trọng tâm của tam giác $SAB$ và $ABC$.
	\begin{itemize}
		\item [a)] Tìm giao tuyến của $\left(SAB\right)$ và $\left(SCD\right)$.
		\item [b)] Chứng minh $MN$ song song $\left(SCD\right)$ và $NG$ song song $\left(SAC\right)$.
	\end{itemize}
\end{vd}
\begin{vd}
	Cho hình chóp $S{.}ABCD$ có đáy $ABCD$ là hình bình hành. Gọi $M$, $N$ lần lượt là trung điểm của các cạnh $AB$ và $CD$.
	\begin{listEX}[1]
		\item Chứng minh $MN$ song song với các mặt phẳng $(SBC)$ và $(SAD)$.
		\item Gọi $E$ là trung điểm của $SA$. Chứng minh $SB$ và $SC$ đều song song với mặt phẳng $(MNE)$.
	\end{listEX}
	\loigiai{
		\immini
		{	\begin{listEX}[1]
				\item Từ giả thiết, ta suy ra $MN\parallel BC$ và $MN\parallel AD$.\\
				Vì $\heva{&MN \not\subset (SBC)\\&BC\subset (SBC)\\&MN\parallel BC}$ nên $MN \parallel (SBC)$.	\\
				Tương tự, ta có  $\heva{&MN \not\subset (SAD)\\&AD\subset (SAD)\\&MN\parallel AD}$ nên $MN \parallel (SAD)$.	
				\item Từ giả thiết, ta có $\dfrac{AE}{AS}=\dfrac{AM}{AB}=\dfrac{1}{2}\Rightarrow ME \parallel SB$.\\
				Vì $\heva{&SB \not\subset (MNE)\\&ME\subset (MNE)\\&ME\parallel SB}$ nên $SB \parallel (MNE)$.	\\
				Tương tự, gọi $O$ là tâm của hình bình hành.\\
				Khi đó $\dfrac{AO}{AC}=\dfrac{AE}{AS}=\dfrac{1}{2}\Rightarrow EO \parallel SC$.\\
				Vì $\heva{&SC \not\subset (MNE)\\&EO\subset (MNE)\\&EO\parallel SC}$ nên $SC \parallel (MNE)$.	
		\end{listEX}	}
		{   \begin{tikzpicture}[scale=1,fill=black]
			\coordinate (S) at (1,5);
			\coordinate (A) at (0,0);
			\coordinate (B) at (-2,-2);
			\coordinate (C) at (3,-2);
			\coordinate (D) at (5,0);
			\coordinate (M) at ($(A)!0.5!(B)$);
			\coordinate (O) at ($(A)!0.5!(C)$);
			\coordinate (N) at ($(C)!0.5!(D)$);
			\coordinate (E) at ($(A)!0.5!(S)$);
			\tkzLabelPoints[above](S)
			\tkzLabelPoints[below](O)
			\tkzLabelPoints[right](C,D,N,E)
			\tkzLabelPoints[left](A,B,M)
			\tkzDrawSegments(S,B S,C S,D B,C C,D)
			\tkzDrawSegments[dashed](S,A  A,B A,D M,N M,E E,O)
			\end{tikzpicture}}
	}
\end{vd}

\begin{vd}
Cho hình chóp $S.ABCD$ có đáy $ABCD$ là hình chữ nhật. Gọi $G$ là trọng tâm tam giác $SAD$ và $E$ là điểm trên cạnh $DC$ sao cho $DC=3DE$, $I$ là trung điểm $AD$.
	\begin{listEX}[1]
		\item Chứng minh $OI$ song song với các mặt phẳng $(SAB)$ và $(SCD)$.
		\item Tìm giao điểm $P$ của $IE$ và $(SBC)$. Chứng minh $GE\parallel (SBC)$.
	\end{listEX}
	\loigiai{
		\immini
		{
			\begin{listEX}[1]
				\item Ta có $\heva{&OI\parallel AB \\& AB\subset(SAB)\\& OI\not\subset (SAB)}\Rightarrow OI \parallel (SAB)$.\\
				Tương tự, $\heva{&OI\parallel CD \\& CD\subset(SCD)\\& OI\not\subset (SCD)}\Rightarrow OI \parallel (SCD)$.
				\item Vì $\dfrac{DI}{DA}=\dfrac{1}{2} \ne \dfrac{1}{3}=\dfrac{DE}{DC}$ nên $IE$ không song song với $ AC$. Trong hình chữ nhật $ABCD$, gọi $P = IE \cap BC$ $\Rightarrow P = IE \cap (SBC)$.\\
				Gọi $K$ là trung điểm của $BC$,  $G'$ là trọng tâm tam giác $SBC$. \\
				Khi đó $\dfrac{SG'}{SK}=\dfrac{SG}{SI}=\dfrac{G'G}{KI}=\dfrac{2}{3}$,suy ra $G'G\parallel KI\parallel CE$ và $\Rightarrow G'G = \dfrac{2}{3}KI=\dfrac{2}{3}CD = CE$. Do dó tứ giác $G'GEC$ là hình bình hành, suy ra $CG'\parallel CE$ $\Rightarrow CG\parallel (SBC)$.
			\end{listEX}
		}
		{\begin{tikzpicture}[scale=1,fill=black]
			\coordinate (A) at (-1,0);
			\coordinate (B) at (-2,-2);
			\coordinate (C) at (3,-2);
			\coordinate (D) at (4,0);
			\coordinate (O) at ($(A)!0.5!(C)$);
			\coordinate (S) at (-0.5,5);
			\coordinate (I) at ($(D)!0.5!(A)$);
			\coordinate (K) at ($(B)!0.5!(C)$);
			\coordinate (G) at ($(S)!0.666!(I)$);
			\coordinate (G') at ($(S)!0.666!(K)$);
			\coordinate (E) at ($(D)!0.33!(C)$);
			%\coordinate (N) at ($(S)!0.5!(D)$);
			\tkzLabelPoints[above](S)
			\tkzLabelPoints[left](A,G,G')
			\tkzLabelPoints[right](D,E)
			\tkzLabelPoints[below=0.1](B,C,O,I,K)
			\tkzDrawSegments(S,B S,C S,D B,C C,D S,K)
			\tkzDrawSegments[dashed](S,A A,B A,D A,C B,D S,I K,I G,E C,G' G,G')
			\end{tikzpicture}}
	}
\end{vd}

\begin{dang}{Tìm giao tuyến của hai mặt phẳng cắt nhau}
	Ngoài các phương pháp đã học ở bài trước, ta có thêm 2 cách nữa là áp dụng định lí 3 ở trên.
\end{dang}

\begin{vd}
\immini{Cho tứ diện $ABCD$ có $G$ là trọng tâm $\triangle ABC$, $M\in CD$ với $MC = 2MD$. 
	\begin{listEX}[1]
		\item Chứng minh $MG$ song song với $(ABD)$.
		\item Tìm giao tuyến của $(ABD)$ với $(BGM)$.
		\end{listEX}
	
	}{
\begin{tikzpicture}[scale=0.7]
\coordinate (B) at (-1,0);
\coordinate (D) at (4,0);
\coordinate (C) at (0,-2);
\coordinate (A) at (1,3);
\coordinate (H) at ($(B)!0.5!(C)$);
\coordinate (G) at ($(A)!0.666!(H)$);
\coordinate (M) at ($(D)!0.333!(C)$);
\tkzLabelPoints[above,font=\footnotesize](A)
\tkzLabelPoints[right,font=\footnotesize](D)
\tkzLabelPoints[left,font=\footnotesize](B,G)
\tkzLabelPoints[below,font=\footnotesize](C,M)
\tkzDrawSegments(A,B A,C C,D D,A A,H A,M B,C)
\tkzDrawSegments[dashed](B,D G,M)
\end{tikzpicture}}
	\loigiai{
		\immini
		{\begin{listEX}[1]
				\item Gọi $N$ là trung điểm của $AB$. Trong tam giác $CDN$, ta có
				$\dfrac{CM}{CD}=\dfrac{CG}{CN}=\dfrac{2}{3}\Rightarrow GM \parallel ND$. Vì $ND \subset (ABD)$, $GM \not\subset (ABD)$ nên $GM\parallel (ABD)$.
				\item Vì $\heva{&GM \parallel (ABD)\\&B \in (ABD) \cap (BGM)}\Rightarrow (ABD) \cap (BGM) = Bx \parallel GM\parallel ND$.
				\end{listEX}
		}
		{\begin{tikzpicture}
			\coordinate (B) at (-1,0);
			\coordinate (D) at (4,0);
			\coordinate (C) at (0,-2);
			\coordinate (A) at (1,4);
			\coordinate (N) at ($(A)!0.5!(B)$);
			\coordinate (H) at ($(B)!0.5!(C)$);
			\coordinate (G) at ($(A)!0.666!(H)$);
			\coordinate (M) at ($(D)!0.333!(C)$);
			\tkzDefPointBy[translation= from G to M](B)
			\tkzGetPoint{x}
			\tkzDefPointBy[translation= from G to M](A)
			\tkzGetPoint{y}
			\tkzLabelPoints[above](A,y)
			\tkzLabelPoints[right](D,M,x)
			\tkzLabelPoints[left](B,N,G)
			\tkzLabelPoints[below](C)
			\tkzDrawSegments(A,B A,C C,D D,A A,H C,N A,M B,C A,y)
			\tkzDrawSegments[dashed](B,D G,M B,M N,D B,x)
			\end{tikzpicture}}
	}
\end{vd}

\begin{vd}
\immini{Cho hình chóp $S.ABCD$ có đáy $ABCD$ là hình bình hành. Gọi $I$, $K$ lần lượt là trung điểm của $BC$ và  $CD$.
	\begin{enumerate}
		\item Tìm giao tuyến của $(SIK)$ và $(SAC)$, $(SIK)$ và $(SBD)$.
		\item Gọi $M$ là trung điểm của $SB$. Chứng minh $SD \parallel (ACM)$.
		\item Tìm giao điểm $F$ của $DM$ và $(SIK)$. Tính tỉ số $\dfrac{MF}{MD}$.
	\end{enumerate}
}{
	\begin{tikzpicture}[scale=0.7, line join=round, line cap=round]
	\tkzDefPoints{0/0/A,-1.7/-1.6/B,2.5/-1.6/C}
	\coordinate (D) at ($(A)+(C)-(B)$);
	\coordinate (S) at ($(A)+(0.7,3)$);
	\tkzDrawPolygon(S,B,C,D)
	\tkzDrawSegments(S,C)
	\tkzDrawSegments[dashed](A,S A,B A,D)
	\tkzDrawPoints[fill=black,size=4](D,C,A,B,S)
	\tkzLabelPoints[above](S)
	\tkzLabelPoints[below](A,B,C)
	\tkzLabelPoints[right](D)
	\end{tikzpicture}}
	\loigiai{
		\begin{center}
			\begin{tikzpicture}[scale=1]
			\tkzDefPoints{0/0/A, -2/-2/B, 3/-2/C}
			\coordinate (D) at ($(A)+(C)-(B)$);
			\coordinate (O) at ($(A)!.5!(C)$);
			\coordinate (H) at ($(B)!.2!(O)$);
			\coordinate (S) at ($(H)+(0,5)$);
			\coordinate (I) at ($(B)!.5!(C)$);
			\coordinate (K) at ($(C)!.5!(D)$);
			\coordinate (M) at ($(S)!.5!(B)$);
			\tkzInterLL(A,C)(I,K)\tkzGetPoint{E}
			\tkzDefPointBy[translation= from B to D](S)
			\tkzGetPoint{x}
			\tkzInterLL(D,M)(S,x)\tkzGetPoint{F}
			\tkzDrawSegments[dashed](M,O S,A A,B A,D A,C B,D I,K S,E A,M D,M)
			\tkzDrawPolygon(S,C,D)
			\tkzDrawSegments(S,B B,C S,I C,K S,K C,M F,x M,F)
			\tkzLabelPoints[above right](A,M)
			\tkzLabelPoints[right](C,D)
			\tkzLabelPoints[above](S,K,x,F)
			\tkzLabelPoints[left](B)
			\tkzLabelPoints[below](O,I,E)
			\tkzDrawPoints[fill=black](A,B,C,D,S,E,O,K,I,M,F)
			\end{tikzpicture}
		\end{center}
		\begin{enumerate}
			\item \begin{itemize}
				\item Ta có $S \in (SIK) \cap (SAC).$\\
				Trong mp$(ABCD)$, gọi $E=IK \cap AC \Rightarrow \heva{&E\in IK \subset (SIK)\\& E\in AC \subset (SAC)} \Rightarrow E \in (SIK) \cap (SAC).$\\
				Suy ra $SE=(SIK) \cap (SAC).$\\
				\item Ta có $\heva{&S \in (SIK) \cap (SBD)\\ &BD \in (SBD), IK \in (SIK), BD \parallel IK}\Rightarrow (SIK) \cap (SBD)=Sx$, (với $Sx \parallel BD \parallel IK).$
			\end{itemize}
			\item Trong mp$(ABCD)$, gọi $O=AC \cap BD$, ta có $SD \parallel MO$. Mà $MO \subset (ACM)$, suy ra $SD \parallel (ACM)$.
			\item \begin{itemize}
				\item Trong mp$(SBD)$, gọi $F=Sx \cap DM \Rightarrow \heva{&S \in DM\\& S\in Sx \subset (SIK)} \Rightarrow F= DM \cap (SIK)$.
				\item Ta có $SF \parallel BD \Rightarrow \dfrac{MF}{MD}=\dfrac{MS}{MB}=1$.
			\end{itemize}
		\end{enumerate}
	}
\end{vd}

\begin{vd}
	Cho hình chóp $S.ABCD$ có đáy là hình thang, đáy lớn $AD$. Gọi $I$ là trung điểm của $SB$. Gọi $(P)$ là mặt phẳng qua $I$, song song với $SD$ và $AC.$ Tìm giao tuyến của $(P)$ với các mặt $(SBD)$ và $(ABCD)$.
	\loigiai{
		\immini{
			\begin{enumerate}[a)]
				\item Ta có: $\heva{
					& I\in (P)\cap (SBD)\\
					& (P) \parallel SD\\
					& SD\subset (SBD)
				}$\\
				$\Rightarrow (P)\cap (SBD)=Ix$ trong đó $Ix\parallel SD$.\\
				Gọi $Ix\cap BD=K \Rightarrow (P)\cap (SBD)=IK.$
				\item Ta có: $\heva{
					& K\in (P)\cap (ABCD)\\
					& (P) \parallel AC\\
					& AC\subset (ABCD)
				}$\\
				$\Rightarrow (P)\cap (SBD)=Ky$ trong đó $Ky\parallel AC$.\\
				Gọi $Ky\cap AD=E, Ky\cap CD=F$\\
				$ \Rightarrow (P)\cap (SBD)=EF.$
			\end{enumerate}
			
		}{
			\begin{tikzpicture}[scale=0.8]
				\tkzInit[xmin=-3,xmax=7,ymin=-3,ymax=5]
				\tkzClip
				\tkzSetUpPoint[fill=black,size=4]
				\tkzDefPoints{0/0/A,1/-2/B,5/-2/C,7/0/D,2/4/S}
				\tkzDefMidPoint(S,B) \tkzGetPoint{I}
				\tkzDefMidPoint(B,D) \tkzGetPoint{K}
				\tkzDefPointWith[linear,K=0.79](D,A)
				\tkzGetPoint{E}
				\tkzDefPointWith[linear,K=0.79](D,C)
				\tkzGetPoint{F}
				\tkzDrawPoints(A,B,C,D,S,I,K,E,F)
				\tkzDrawSegments[dashed](A,D B,D I,K A,C E,F)
				\tkzDrawSegments(A,B B,C C,D S,A S,B S,C S,D)
				\tkzLabelPoint[left](I){\footnotesize $I$}
				\tkzLabelPoint[above](K){\footnotesize $K$}
				\tkzLabelPoint[below](E){\footnotesize $E$}
				\tkzLabelPoint[right](F){\footnotesize $F$}
				\tkzLabelPoint[left](A){\footnotesize $A$}
				\tkzLabelPoint[below](B){\footnotesize $B$}
				\tkzLabelPoint[below](C){\footnotesize $C$}
				\tkzLabelPoint[right](D){\footnotesize $D$}
				\tkzLabelPoint[above](S){\footnotesize $S$}
		\end{tikzpicture}}
	}
\end{vd}

\begin{vd}
\immini{Cho tứ diện $ABCD$. Gọi $M$, $I$ lần lượt là trung điểm của $BC$, $AC$. Mặt phẳng $(P)$ đi qua điểm $M$, song song với $BI$ và $SC$. Xác định trên hình vẽ các giao điểm $H, K, N$ của $(P)$ với các cạnh $AC$, $SA$, $SB$. Tứ giác $MNKH$ là hình gì?
	
}{
	\begin{tikzpicture}[scale=0.6,fill=black]
	\coordinate (A) at (-1,0);
	\coordinate (B) at (0,-2);
	\coordinate (C) at (4,0);
	\coordinate (S) at (1,3);
	\coordinate (M) at ($(B)!0.5!(C)$);
	\coordinate (I) at ($(C)!0.5!(A)$);
	\tkzLabelPoints[right,font=\footnotesize](C)
	\tkzLabelPoints[below,font=\footnotesize](B,M)
	\tkzLabelPoints[above=0.1,font=\footnotesize](S,I)
	\tkzLabelPoints[left,font=\footnotesize](A)
	\tkzDrawSegments(S,A S,B S,C A,B B,C)
	\tkzDrawSegments[dashed](A,C B,I)
	\tkzDrawPoints[size=5,fill=black](A,B,C,S,I,M)
	
	\end{tikzpicture}}
	\loigiai{
		\immini
		{
			Vì $\heva{&(P) \parallel SC\\& M\in (P)\cap (SBC)}\Rightarrow (P) \cap (SBC) = MN \parallel SC$, $N\in SB$ $\qquad (1)$\\
			Tương tự, $\heva{&(P)\parallel BI\\& M\in (P) \cap (ABC)}\Rightarrow (P) \cap (ABC) =MH \parallel BI$, $H\in AC$ $\qquad (2)$
			
			Mặt khác, $\heva{&(P)\parallel (SC)\\& N\in (P) cap (SAC)}\Rightarrow (P) \cap (SAC) = HK \parallel SC$, $K\in SA$ $(3)$
			Từ $(1)$, $(2)$ và $(3)$ ta có thiết diện của $(P)$ với tư diện $ABCD$ là tứ giác $MNKH$.
		}
		{\begin{tikzpicture}[scale=1,fill=black]
			\coordinate (A) at (-1,0);
			\coordinate (B) at (0,-2);
			\coordinate (C) at (4,0);
			\coordinate (S) at (1,4);
			\coordinate (M) at ($(B)!0.5!(C)$);
			\coordinate (I) at ($(C)!0.5!(A)$);
			\coordinate (N) at ($(S)!0.5!(B)$);
			\coordinate (H) at ($(I)!0.5!(C)$);
			\coordinate (Q) at ($(S)!0.5!(A)$);
			\coordinate (K) at ($(S)!0.5!(Q)$);
			\tkzLabelPoints[right](C)
			\tkzLabelPoints[below](B, M)
			\tkzLabelPoints[above=0.1](I,S,H)
			\tkzLabelPoints[left](A,K,N)
			\tkzDrawSegments(S,A S,B S,C M,N K,N A,B B,C)
			\tkzDrawSegments[dashed](A,C M,H H,K B,I)
			\end{tikzpicture}}
	}
\end{vd}

\begin{vd}%[1H2K3-4]
\immini{Cho hình chóp $S.ABCD$. Gọi $M$, $N$ thuộc cạnh $AB$, $CD$. Gọi $(\alpha)$ là mặt phẳng qua $MN$ và song song với $SA$. Tìm giao tuyến của $(\alpha)$ với các mặt của hình chóp.
	
}{
	\begin{tikzpicture}[line join = round, line cap = round,>=stealth,
	font=\footnotesize,scale=.7]
	\tkzDefPoints{0/0/A,1.5/-3/B,4.5/-2.5/C}
	\coordinate (D) at ($(A)+(7,0)$);
	\coordinate (S) at ($(A)+(3,3)$);
	\coordinate (M) at ($(A)!3/5!(B)$);
	\coordinate (N) at ($(C)!2/5!(D)$);
	\tkzDrawSegments(S,C S,B S,D C,D B,C A,B S,A)
	\tkzDrawSegments[dashed](A,D A,C M,N)
	\tkzDrawPoints[fill=black](A,B,D,C,S,M,N)
	\tkzLabelPoints[above](S)
	\tkzLabelPoints[right](D,N)
	\tkzLabelPoints[left](M)
	\tkzLabelPoints[below](B,C)
	\tkzLabelPoints[above left](A)
	\end{tikzpicture}}
	\loigiai{
				\immini{
				Ta có $\heva{&M \in (\alpha) \cap (SAB)\\&SA \parallel (\alpha)\\&SA \subset (SAB)} \Rightarrow (\alpha) \cap (SAB)= MP$, với $MP \parallel SA$.\\
				Trong mặt phẳng $(ABCD)$, gọi $R= MN \cap AC$.\\
				Ta có $\heva{&R \in (\alpha) \cap (SAC)\\&SA \parallel (\alpha)\\&SA \subset (SAC)} \Rightarrow (\alpha) \cap (SAC)= RQ$, với $RQ \parallel SA$.\\
				Ta có $(\alpha) \cap (SCD)= QN$. Vậy thiết diện là tứ giác $MNQP$.
						}{
				\begin{tikzpicture}[line join = round, line cap = round,>=stealth,
				font=\footnotesize,scale=.7]
				\tkzDefPoints{0/0/A,1.5/-3/B,4.5/-2.5/C}
				\coordinate (D) at ($(A)+(7,0)$);
				\coordinate (S) at ($(A)+(3,5)$);
				\coordinate (M) at ($(A)!3/5!(B)$);
				\coordinate (N) at ($(C)!2/5!(D)$);
				\tkzInterLL(A,C)(M,N) \tkzGetPoint{R}
				\coordinate (K') at ($(M)+(S)-(A)$);
				\tkzInterLL(M,K')(S,B) \tkzGetPoint{P}
				\coordinate (H') at ($(S)+(N)-(D)$);
				\tkzInterLL(N,H')(S,C) \tkzGetPoint{Q}
				\tkzDrawSegments(S,C S,B S,D C,D B,C A,B S,A M,P P,Q Q,N)
				\tkzDrawSegments[dashed](A,D A,C M,N R,Q)
				\tkzDrawPoints[fill=black](A,B,D,C,S,M,N,R,P,Q)
				\tkzLabelPoints[above](S)
				\tkzLabelPoints[right](D,N,Q)
				\tkzLabelPoints[left](M,P)
				\tkzLabelPoints[below](B,C,R)
				\tkzLabelPoints[above left](A)
				
				\end{tikzpicture}}
		
	}
\end{vd}

\begin{vd}
	\immini{Cho hình chóp $S.ABCD$ có đáy là hình bình hành, $O$ là giao điểm của $AC$ và $BD$, $M$ là trung điểm của $SA$. 
	\begin{enumerate}
		\item Chứng minh $OM \parallel (SCD)$.
		\item Gọi $(\alpha)$ là mặt phẳng đi qua $M$, đồng thời song song với $SC$ và $AD$. Tìm giao tuyến của mặt phẳng $(\alpha)$ với các mặt của hình chóp $S.ABCD$. Hình tạo bởi các giao tuyến là hình gì?
	\end{enumerate}
}{
	\begin{tikzpicture}[line join = round, line cap = round,>=stealth,
	font=\footnotesize,scale=.9]
	\tkzDefPoints{0/0/A}
	\coordinate (D) at ($(A)+(5,0)$);
	\tkzDefShiftPoint[A](-140:2.7){B}
	\coordinate (C) at ($(B)+(D)-(A)$);
	\tkzInterLL(A,C)(B,D) \tkzGetPoint{O}
	\coordinate (S) at ($(A)+(1,3)$);
	\coordinate (M) at ($(A)!.5!(S)$);
	\tkzDrawPolygon(S,B,C,D)
	\tkzDrawSegments(S,C)
	\tkzDrawSegments[dashed](A,S A,B A,D A,C B,D)
	\tkzDrawPoints[fill=black](A,B,D,C,O,S,M)
	\tkzLabelPoints[above](S)
	\tkzLabelPoints[below](O,C)
	\tkzLabelPoints[left](A,B)
	\tkzLabelPoints[right](D)
	\tkzLabelPoints[right](M)
	\end{tikzpicture}}
	\loigiai{
		\begin{enumerate}
			\item 
			\immini{Ta có $M, O$ là trung điểm của $SA$ và $AC$, suy ra $MO \parallel SC$.\\
				Mà $SC \subset (SCD) \Rightarrow OM \parallel (SCD)$.
				\item Vì $MO \parallel SC \Rightarrow O \in (\alpha)$.\\
				Ta có $\heva{&O \in (\alpha) \cap (ABCD)\\&AD \parallel (\alpha)\\&AD \subset (ABCD)} \Rightarrow (\alpha) \cap (ABCD)= PQ$.\\
				Với $PQ \parallel AD, O \in PQ, Q \in AB, P \in CD$.\\
				Lại có $\heva{&P \in (\alpha) \cap (SCD)\\&SC \parallel (\alpha)\\&SC \subset (SCD)} \Rightarrow (\alpha) \cap (SCD)= PN$, với $PN \parallel SC$.\\
				Có $(\alpha) \cap (SAD)= MN, (\alpha) \cap (SAB)= MQ$.\\
				Nhận thấy $P, Q$ là trung điểm của $CD$ và $AB$. Suy ra $N$ là trung điểm của $SD$.\\
				Suy ra $MN \parallel PQ$. Vậy thiết diện là hình thang $MNPQ$.
				
			}{
				\begin{tikzpicture}[line join = round, line cap = round,>=stealth,
				font=\footnotesize,scale=.7]
				\tkzDefPoints{0/0/A}
				\coordinate (D) at ($(A)+(5,0)$);
				\tkzDefShiftPoint[A](-140:2.7){B}
				\coordinate (C) at ($(B)+(D)-(A)$);
				\tkzInterLL(A,C)(B,D) \tkzGetPoint{O}
				\coordinate (S) at ($(O)+(0,6)$);
				\coordinate (M) at ($(A)!.5!(S)$);
				\coordinate (N) at ($(S)!.5!(D)$);
				\coordinate (Q) at ($(A)!.5!(B)$);
				\coordinate (P) at ($(C)!.5!(D)$);
				\tkzDrawPolygon(S,B,C,D)
				\tkzDrawSegments(S,C N,P)
				\tkzDrawSegments[dashed](A,S A,B A,D A,C B,D Q,M M,N M,O P,Q M,Q)
				\tkzDrawPoints[fill=black](A,B,D,C,O,S,M,N,P,Q)
				\tkzLabelPoints[above](S)
				\tkzLabelPoints[below](O,C)
				\tkzLabelPoints[left](A,B,Q)
				\tkzLabelPoints[right](D,N,P)
				\tkzLabelPoints[above right](M)
				\end{tikzpicture}
				
			}
		\end{enumerate}
	}
\end{vd}
\begin{vd}
	Cho tứ diện $ABCD$ và điểm $M$ thuộc cạnh $AB$. Gọi $\left(\alpha\right)$ là mặt phẳng đi qua $M$, song song với đường thẳng $BC$ và $AD$. Gọi $N, P, Q$ lần lượt là giao điểm của $\left(\alpha\right)$ với các cạnh $AC, CD$ và $DB$ .
	\begin{itemize}
		\item [a)] Chứng minh $MNPQ$ là hình bình hành.
		\item [b)] Trong trường hợp nào thì $MNPQ$ là hình thoi.
	\end{itemize}
\end{vd}

\subsection{BÀI TẬP TỰ LUYỆN}

\begin{bt}
	Cho tứ diện $ABCD$ có $G$ là trọng tâm tam giác $ABD$. Trên đoạn $BC$ lấy điểm $M$ sao cho $MB=2MC$. Chứng minh rằng đường thẳng $MG$ song song với mặt phẳng $(ACD)$.
	\loigiai{
		\begin{center}
			\begin{tikzpicture}
				\tkzDefPoints{0/0/B, 7/0/D, 2/-3/C, 4/5/A}
				\tkzDefMidPoint(A,D) \tkzGetPoint{N}		 
				\coordinate (M) at ($(B)!.666666!(C)$);
				\tkzCentroid(A,B,D) \tkzGetPoint{G}
				\tkzDrawPoints(M,N,G)
				\tkzDrawSegments(A,B B,C C,D D,A C,N A,C)
				\tkzDrawSegments[dashed](B,D M,G B,N)
				\tkzLabelPoints[below left](M)
				\tkzLabelPoints[above](A,G)
				\tkzLabelPoints[left](B)
				\tkzLabelPoints[right](N,D)
				\tkzLabelPoints[below](C)
			\end{tikzpicture}
		\end{center}
		Gọi $N$ là trung điểm của $AD$. Ta có: $\dfrac{BG}{BN}=\dfrac{2}{3}$ (Vì $G$ là trọng tâm tam giác $ABD$).\\
		Theo giả thiết, ta có: $MB=2MC \Rightarrow \dfrac{BM}{BC}=\dfrac{2}{3}$.\\
		Tam giác $BCN$ có $\dfrac{BG}{BN}=\dfrac{BM}{BC}=\dfrac{2}{3}$ $\Rightarrow MG \parallel CN$.\\
		Mà $MG \not\subset (ACD)$, $CN \subset (ACD) \Rightarrow MG \parallel (ACD)$.\\
	}
\end{bt}
\begin{bt}
	Cho hình chóp $S.ABCD$ có đáy $ABCD$ là hình bình hành tâm $O$. Gọi $M$, $N$, $P$ lần lượt là trung điểm của các cạnh $SD$, $CD$, $BC$.
	\begin{enumerate}
		\item Chứng minh đường thẳng $OM$ song song với các mặt phẳng $(SAB)$, $(SBC)$.
		\item Chứng minh đường thẳng $SP$ song song với mặt phẳng $(OMN)$.
	\end{enumerate}
	\loigiai{
		\begin{center}
			\begin{tikzpicture}
				\tkzDefPoints{0/0/D, 3/3/A, 6/0/C, 9/3/B, 2/8/S}
				\tkzDefMidPoint(S,D) \tkzGetPoint{M}
				\tkzDefMidPoint(C,D) \tkzGetPoint{N} 
				\tkzDefMidPoint(B,C) \tkzGetPoint{P}
				\tkzInterLL(A,C)(B,D) \tkzGetPoint{O}
				\tkzInterLL(O,N)(D,P) \tkzGetPoint{I}
				\tkzDrawPoints(M,N,P,O,I)
				\tkzDrawSegments(B,C C,D S,B S,C S,D S,P M,N)
				\tkzDrawSegments[dashed](A,B A,D A,C B,D S,A O,M O,N D,P M,I)
				\tkzLabelPoints[below left](D)
				\tkzLabelPoints[above](S,O)
				\tkzLabelPoints[above right](A)
				\tkzLabelPoints[left](M)
				\tkzLabelPoints[right](B,P)
				\tkzLabelPoints[below](N)
				\tkzLabelPoints[below right](C,I)
			\end{tikzpicture}
		\end{center}
		\begin{enumerate}
			\item Tam giác $SBD$ có $OB=OD$ và $MS=MD$ nên $OM$ là đường trung bình của tam giác $SBD$ $\Rightarrow OM \parallel SB$.\\
			Mà $OM$ không chứa trong các mặt phẳng $(SAB)$ và $(SBC)$ nên $OM \parallel (SAB)$ và $OM \parallel (SBC)$.
			\item Trong mặt phẳng $(ABCD)$, gọi $I$ là giao điểm của $ON$ và $DP$.\\
			Tam giác $BCD$ có $OB=OD$ và $NC=ND$ nên $ON$ là đường trung bình của tam giác $BCD$ $\Rightarrow I$ là trung điểm của $DP$.\\
			Tam giác $SDP$ có $MS=MD$ và $IP=ID$ nên $IM$ là đường trung bình của tam giác $SDP$ $\Rightarrow IM \parallel SP$.\\
			Mà $SP \not\subset (OMN)$, $IM \subset (OMN) \Rightarrow SP \parallel (OMN)$.
		\end{enumerate}
	}
\end{bt}


\begin{bt}
	Cho hình chóp $S.ABCD$ có đáy là hình thang đáy lớn $AB$, với $AB = 2CD$. Gọi $O$ là giao điểm của $AC$ và $BD$, $I$ là trung điểm của $SA$, $G$ là trọng tâm của tam giác $SBC$ và $E$ là một điểm trên cạnh $SD$ sao cho $3SE = 2SD$. Chứng minh:
	\begin{listEX}[3]
		\item $DI\parallel (SBC)$.
		\item $GO\parallel (SCD)$.
		\item $SB\parallel (ACE)$.
	\end{listEX}
	\loigiai{
		\begin{center}
			\begin{tikzpicture}[scale=1,fill=black]
				\coordinate (S) at (0,5);
				\coordinate (A) at (-1,0);
				\coordinate (B) at (5,0);
				\coordinate (D) at (0,-2);
				\coordinate (C) at (3,-2);
				\coordinate (I) at ($(S)!0.5!(A)$);
				\coordinate (N) at ($(S)!0.5!(B)$);
				\coordinate (M) at ($(B)!0.5!(C)$);
				\coordinate (P) at ($(S)!0.5!(C)$);
				\coordinate (E) at ($(S)!0.666!(D)$);
				\coordinate (G) at ($(S)!0.666!(M)$);
				\tkzInterLL(A,C)(B,D) \tkzGetPoint{O}
				\tkzLabelPoints[above](S)
				\tkzLabelPoints[below](C,D,O)
				\tkzLabelPoints[left](A,I,P)
				\tkzLabelPoints[right](B,E,G,M,N)
				\tkzDrawSegments(S,A S,D S,D S,B B,C C,D D,A S,C D,I S,M A,E N,C C,E B,P D,P)
				\tkzDrawSegments[dashed](A,C B,D A,B I,N G,O O,E)
			\end{tikzpicture}
		\end{center}
		\begin{listEX}[1]
			\item Gọi $N$ là trung điểm $SB$, khi đó $IN\parallel AB$ và $IN =\dfrac{1}{2}AB$. Suy ra $IN\parallel CD$, $IN =DC$ suy ra tứ giác $INCD$ là hình bình hành, do đó $ID\parallel NC$. Vậy $ID \parallel (SBC)$.
			\item $GO \parallel (SCD)$\\
			Gọi $P$ là trung điểm của $SC$, khi đó $GO\parallel PD$, suy ra $GO\parallel (SCD)$.
			\item Ta có $EO \parallel SB$, suy ra $SB \parallel (ACE)$.
		\end{listEX}
	}
\end{bt}

\begin{bt}
	Cho tứ diện $ABCD$. Gọi $I,\,J$ lần lượt là trung điểm của $AB$ và $CD$, $M$ là một điểm trên đoạn $IJ$. Gọi $(P)$ là mặt phẳng qua $M$ và song song với $AB$ và $CD$.
	\begin{listEX}
		\item Tìm giao tuyến của mặt phẳng $(P)$ và $(ICD)$.
		\item Xác định giao tuyến của mặt phẳng $(P)$ với các mặt của tứ diện. Hình tạo bởi các giao tuyến là hình gì?
	\end{listEX}
	\loigiai{
		\begin{center}
			\begin{tikzpicture}[line join=round, line cap=round,>=stealth,scale=1,font=\footnotesize]
				\tkzInit[ymin=-2.5,ymax=5,xmin=-2.5,xmax=5.5]
				\tkzClip
				\tkzDefPoints{0/0/B,5/0/D,2/-2/C,1/4/A}
				\tkzDefMidPoint(B,A)\tkzGetPoint{I}
				\tkzDefMidPoint(C,D)\tkzGetPoint{J};
				\coordinate (M) at ($(I)!0.6!(J)$);
				\tkzDefLine[parallel=through M](C,D) \tkzGetPoint{m}
				\tkzInterLL(M,m)(I,D) \tkzGetPoint{F}
				\tkzInterLL(M,m)(I,C) \tkzGetPoint{E}
				\tkzDefLine[parallel=through F](A,B) \tkzGetPoint{f}
				\tkzInterLL(F,f)(B,D) \tkzGetPoint{G}
				\tkzInterLL(F,f)(A,D) \tkzGetPoint{P}
				\tkzDefLine[parallel=through E](A,B) \tkzGetPoint{e}
				\tkzInterLL(E,e)(B,C) \tkzGetPoint{H}
				\tkzInterLL(E,e)(A,C) \tkzGetPoint{Q}
				\tkzDrawPoints[fill=black](A,B,M,C,D,I,J,E,F,G,H,P,Q)
				\tkzLabelPoints(C,J)
				\tkzLabelPoints[above](A,M)
				\tkzLabelPoints[above left](I)	
				\tkzLabelPoints[above right](F)	
				\tkzLabelPoints[below left](H)
				\tkzLabelPoints[below right](G)	
				%	\tkzLabelPoints[above right](M)
				\tkzLabelPoints[right](D,P)
				\tkzLabelPoints[left](B,E,Q)
				\tkzDrawSegments(A,B B,C C,A A,D C,D I,C Q,H P,Q)
				\tkzDrawSegments[dashed](B,D I,J I,D E,F G,H P,G)
			\end{tikzpicture}
		\end{center}
		\begin{listEX}
			\item 
			Gọi $\Delta_1=(P)\cap(ICD)$, ta có\\
			$\heva{&M\in (P)\\&M\in IJ,\,IJ \subset (ICD)}\Rightarrow M \in \Delta_1$.\\
			$\heva{&(P)\parallel CD\\& CD\subset (ICD)\\&(P)\cap (ICD)=\Delta_1}\Rightarrow \Delta_1 \parallel CD$.\\
			Vậy $\Delta_1$ là đường thẳng qua $M$ và song song với $CD$.\\
			Gọi $E=\Delta_1 \cap IC, F=\Delta_1 \cap TD$, ta được $(P)\cap (ICD)=EF$.
			\item 	
			Gọi $\Delta_2=(P)\cap(ABD)$, ta có\\
			$\heva{&F\in (P)\\&F\in ID,\,ID \subset (ABD)}\Rightarrow F \in \Delta_2$.\\
			$\heva{&(P)\parallel AB\\& AB\subset (ABD)\\&(P)\cap (ABD)=\Delta_2}\Rightarrow \Delta_2 \parallel AB$.\\
			Vậy $\Delta_2$ là đường thẳng qua $F$ và song song với $AB$.\\
			Gọi $G=\Delta_2 \cap BD, P=\Delta_2 \cap AD$, ta được $(P)\cap (ICD)=GP$.\\
			Gọi $\Delta_3=(P)\cap(ABC)$, ta có
			$$\heva{&E\in (P)\\&E\in IC,\,IC \subset (ABC)}\Rightarrow E \in \Delta_3.$$
			Ta có $$\heva{&(P)\parallel AB\\& AB\subset (ABC)\\&(P)\cap (ABC)=\Delta_3}\Rightarrow \Delta_3 \parallel AB.$$
			Vậy $\Delta_3$ là đường thẳng qua $E$ và song song với $AB$.\\
			Gọi $H=\Delta_3 \cap BC, Q=\Delta_3 \cap AC$, ta được $(P)\cap (ABC)=HQ$.\\
			Giao tuyến của $(P)$ với các mặt phẳng $(BCD),\,(ABD),\,(ACD),\,(ABC)$ lần lượt là $GH,\,GP,\,PQ,\,QH$. Do đó thiết diện của tứ diện với mặt phẳng $(P)$ là tứ giác $HGPQ$.\\
			Ta có
			$$\heva{&(P)\parallel CD\\& CD\subset (ACD)\\&(P)\cap (ACD)=PQ}\Rightarrow PQ \parallel CD$$
			và
			$$\heva{&(P)\parallel CD\\& CD\subset (BCD)\\&(P)\cap (BCD)=HG}\Rightarrow HG \parallel CD.$$
			Ta có $\heva{&HG\parallel PQ\,(\text{cùng song song với} \,CD)\\&HQ\parallel PG\,(\text{cùng song song với}\, AB )}\Rightarrow $ tứ giác $HGPQ$ là hình bình hành.
		\end{listEX}
	}
\end{bt}
\begin{bt}
	Cho hình chóp $S.ABCD$ có đáy $ABCD$ là hình bình hành tâm $O$. Gọi $K$ và $J$ lần lượt là trọng tâm của các tam giác $ABC$ và $SBC$.
	\begin{listEX}
		\item Chứng minh $KJ \parallel (SAB)$.
		\item Gọi $(P)$ là mặt phẳng chứa $KJ$ và song song với $AD$. Xác định giao tuyến của mặt phẳng $(P)$ với các mặt của hình chóp. Hình tạo bởi các giao tuyến là hình gì?
	\end{listEX}
	\loigiai{
		\begin{center}
			\begin{tikzpicture}[>=stealth, line join=round, line cap = round]
				\tkzInit[ymin=-2.5,ymax=5.5,xmin=-2.5,xmax=5.5]
				\tkzClip
				\tkzDefPoints{0/0/A, 5/0/D, -2/-2/B, 1/5/S}
				\tkzDefPointBy[translation=from A to D](B)\tkzGetPoint{C}
				\tkzInterLL(A,C)(B,D) \tkzGetPoint{O}
				\tkzDefMidPoint(B,C)\tkzGetPoint{H}
				\tkzCentroid(A,B,C)\tkzGetPoint{K}
				\tkzCentroid(S,B,C)\tkzGetPoint{J}
				\tkzDefLine[parallel=through J](C,B) \tkzGetPoint{j}
				\tkzInterLL(J,j)(B,S) \tkzGetPoint{M} \tkzInterLL(J,j)(C,S) \tkzGetPoint{N}
				\tkzDefLine[parallel=through K](C,B) \tkzGetPoint{k}
				\tkzInterLL(K,k)(B,A) \tkzGetPoint{E} \tkzInterLL(K,k)(C,D) \tkzGetPoint{F}
				\tkzDrawPoints[fill=black](A,B,C,D,O,J,K,M,N,E,F,H)
				\tkzDrawSegments(B,C C,D S,B S,C S,D M,N N,F S,H)
				\tkzDrawSegments[dashed](A,B S,A A,C B,D A,D E,F M,E A,H K,J)
				\tkzLabelPoints[left](A,M,E)
				\tkzLabelPoints[below](B,H)
				\tkzLabelPoints[right](C,D,N,F)
				\tkzLabelPoints[above](S,O)
				\tkzLabelPoints[above right](J)
				\tkzLabelPoints[above left](K)
			\end{tikzpicture}
		\end{center}
		\begin{listEX}	
			\item Gọi $H$ là trung điểm $BC$, theo tính chất trọng tâm ta có $\dfrac{HK}{HA}=\dfrac{HJ}{HS}=\dfrac{1}{3}\Rightarrow KJ\parallel SA$ (Định lý Ta-lét đảo).
			Ta có $\heva{&KJ\parallel SA\\&SA\subset (SAB)\\&KJ \not\subset (SAB)}\Rightarrow KJ\parallel (SAB)$.
			\item 
			Gọi $\Delta_1=(P)\cap(ABCD)$, ta có\\
			$\heva{&K\in KJ,\,KJ\subset (P)\\&K\in (ABCD)}\Rightarrow K \in \Delta_1$.\\
			$\heva{&(P)\parallel AD\\& AD\subset (ABCD)\\&(P)\cap (ABCD)=\Delta_1}\Rightarrow \Delta_1 \parallel AD$.\\
			Vậy $\Delta_1$ là đường thẳng qua $K$ và song song với $AD$.\\
			Gọi $E=\Delta_1 \cap AB, F=\Delta_1 \cap CD$, ta được$(P)\cap (ABCD)=EF$.\\
			
			Gọi $\Delta_2=(P)\cap(SBC)$, ta có
			$$\heva{&J\in KJ,\,KJ\subset (P)\\&J\in (SBC)}\Rightarrow K \in \Delta_2.$$
			Và	
			$$\heva{&(P)\parallel AD\parallel BC\\& BC\subset (ABCD)\\&(P)\cap (ABCD)=\Delta_2}\Rightarrow \Delta_2 \parallel BC.$$
			Vậy $\Delta_2$ là đường thẳng qua $J$ và song song với $BC$.\\
			Gọi $M=\Delta_2 \cap SB, N=\Delta_2 \cap SD$, ta được $(P)\cap (SBC)=MN$.\\
			Ta có giao tuyến của $(P)$ với các mặt phẳng $(ABCD),\,(SCD),\,(SBC),\,(SAB)$ lần lượt là $EF,\,FN,\,NM,\,NE$, do đó thiết diện của hình chóp cắt bởi mặt phẳng $(P)$ là tứ giác $MNFE$.
		\end{listEX}
	}
\end{bt}

\begin{bt}
	Cho tứ diện $ABCD$. Lấy điểm $M$ trên cạnh $AB$ sau cho $AM=2MB$. Gọi $G$ là trọng tâm $\triangle BCD$ và $I$ là trung điểm $CD$, $H$ là điểm đối xứng của $G$ qua $I$.
	\begin{enumerate}
		\item Chứng minh $GD \parallel (MCH)$.
		\item Tìm giao điểm $K$ của $MG$ với $(ACD)$. Tính tỉ số $\dfrac{GK}{GM}$.
	\end{enumerate}
	\loigiai{
		\begin{center}
			\begin{tikzpicture}[scale=.8]
				\tkzDefPoints{0/0/B, 4/-2/C, 6/0/D}
				\coordinate (h) at ($(C)!.7!(D)$);   
				\coordinate (H) at ($(B)!.4!(h)$);
				\coordinate (A) at ($(H)+(0,5)$);
				\coordinate (M) at ($(A)!2/3!(B)$);
				\coordinate (I) at ($(C)!.5!(D)$);
				\tkzCentroid(D,B,C)\tkzGetPoint{G}
				\tkzDefPointBy[symmetry=center I](G)\tkzGetPoint{H}
				\tkzInterLL(A,I)(M,H)\tkzGetPoint{h}
				\tkzInterLL(A,I)(M,G)\tkzGetPoint{K}
				\tkzInterLL(A,I)(C,H)\tkzGetPoint{k}
				\tkzInterLL(C,H)(M,G)\tkzGetPoint{i}
				\tkzInterLL(M,H)(C,D)\tkzGetPoint{o}
				\coordinate (E) at ($(A)!.5!(K)$);
				\tkzDrawSegments[dashed](B,D M,I B,I I,H M,h G,D M,i h,k C,o G,E)
				\tkzDrawSegments(A,B A,C A,D B,C o,D A,h M,C C,H h,H k,K i,K D,H)
				\tkzLabelPoints[left](B,M)
				\tkzLabelPoints[right](D,K,E)
				\tkzLabelPoints[below](C,I,G,H)
				\tkzLabelPoints[above](A)
				\tkzDrawPoints[fill=black](A,B,C,D,M,I,G,H,K,E)
			\end{tikzpicture}
		\end{center}
		\begin{enumerate}
			\item Ta có $IC = ID$ và $IG = IH$ nên $GDHC$ là hình bình hành.\\
			Do đó $GD \parallel CH$\\
			mà $CH \subset (MCH)$ nên $GD \parallel (MCH)$.
			\item Trong mp$(ABI)$, gọi $K=AI \cap MG$, ta có $\heva{&K \in AI \subset (ACD)\\&K \in MG}$\\
			$\Rightarrow K=MG \cap (ACD).$\\
			Trong mp$(ABI)$, kẻ $GE \parallel AB$, $(E \in AI)$.\\
			Xét tam giác $ABI$, có $GE \parallel AB$, suy ra $\dfrac{GE}{AB}=\dfrac{IG}{IB}=\dfrac{1}{3} \Rightarrow \dfrac{GE}{AM}=\dfrac{1}{2}$.\\
			Xét tam giác $AKM$, có $GE \parallel AM$, suy ra $\dfrac{KG}{KM}=\dfrac{GE}{AM}=\dfrac{1}{2}\Rightarrow \dfrac{GK}{GM}=1.$
		\end{enumerate}
	}
\end{bt}


\begin{bt}%[Trần Nhân Kiệt]%[1H2G3]
	Cho hình chóp $S.ABCD$ có đáy là hình bình hành tâm $O, M$ là trung điểm của $SA$. Gọi $(P)$ là mặt phẳng qua $O$, song song với $BM$ và $SD.$ Tìm giao tuyến của $(P)$ và $(SAD).$
	\loigiai{
		\immini{
			\begin{itemize}
				\item Tìm giao tuyến của $(P)$ và $(SBD).$\\
				Ta có: $\heva{
					& O\in (P)\cap (SBD)\\
					& (P) \parallel SD\\
					& SD\subset (SBD)
				}$\\
				$\Rightarrow (P)\cap (SBD)=Ox$
				trong đó $Ox\parallel SD$.\\
				Gọi $Ox\cap SB=N \Rightarrow (P)\cap (SBD)=ON.$
				\item Tìm giao tuyến của $(P)$ và $(SAB).$\\
				Ta có: $\heva{
					& N\in (P)\cap (SAB)\\
					& (P) \parallel BM\\
					& BM\subset (SAB)
				}$\\
				$\Rightarrow (P)\cap (SAB)=Ny$
				trong đó $Ny\parallel BM$.\\
				Gọi $Ny\cap SA=E \Rightarrow (P)\cap (SAB)=NE.$
				\item Tìm giao tuyến của $(P)$ và $(SAD).$\\
				Ta có: $\heva{
					& E\in (P)\cap (SAD)\\
					& (P) \parallel SD\\
					& SD\subset (SAD)
				}$\\
				$\Rightarrow (P)\cap (SAD)=Ez$
				trong đó $Ez\parallel SD$.\\
				Gọi $Ez\cap AD=F \Rightarrow (P)\cap (SAD)=EF.$
			\end{itemize}	
		}{\begin{tikzpicture}[scale=0.6]\\
				\tkzInit[xmin=-3,xmax=8.5,ymin=-3,ymax=7]
				\tkzClip
				\tkzSetUpPoint[fill=black,size=4]
				\tkzDefPoints{0/0/A,-2.5/-2/B,5/-2/C,0.5/6/S}
				\tkzDefPointWith[colinear = at A](B,C)
				\tkzGetPoint{D}
				\tkzDefMidPoint(A,C)
				\tkzGetPoint{O}
				\tkzDefMidPoint(S,A)
				\tkzGetPoint{M}
				\tkzDefPointWith[linear,K=0.25](S,A)
				\tkzGetPoint{E}
				\tkzDefMidPoint(S,B)
				\tkzGetPoint{N}
				\tkzDefPointWith[linear,K=0.75](A,D)
				\tkzGetPoint{F}
				\tkzDrawPoints(A,B,C,D,S,M,N,E,F)
				\tkzDrawSegments[dashed](A,D A,B S,A B,D A,C E,F B,M N,E N,O)
				\tkzDrawSegments(B,C C,D S,B S,C S,D)
				\tkzLabelPoint[above right](F){\footnotesize $F$}
				\tkzLabelPoint[right](M){\footnotesize $M$}
				\tkzLabelPoint[left](N){\footnotesize $N$}
				\tkzLabelPoint[left](A){\footnotesize $A$}
				\tkzLabelPoint[below](B){\footnotesize $B$}
				\tkzLabelPoint[below](C){\footnotesize $C$}
				\tkzLabelPoint[right](D){\footnotesize $D$}
				\tkzLabelPoint[right](E){\footnotesize $E$}
				\tkzLabelPoint[above](S){\footnotesize $S$}
				\tkzLabelPoint[above](O){\footnotesize $O$}
		\end{tikzpicture}}
	}
\end{bt}


% \subsection{BÀI TẬP TRẮC NGHIỆM}
\Opensolutionfile{ans}[ans/1H4.B3]
\setcounter{ex}{0}

\begin{ex} Trong không gian cho mặt phẳng $(\alpha)$ và $A$ không thuộc $(\alpha)$. Qua điểm $A$ có thể dựng được bao nhiêu đường thẳng song song với $(\alpha)$?
	\choice
	{Duy nhất}
	{\True Vô số }
	{$2$}
	{$4$}
	\loigiai{}
\end{ex}
\begin{ex}%[1H2B3-1]
	Trong không gian cho đường thẳng $\Delta$ và điểm $O$ không nằm trong $\Delta$. Qua điểm $O$ cho trước, có bao nhiêu mặt phẳng song song với đường thẳng $\Delta$?
	\choice
	{\True Vô số}
	{ $3$}
	{ $1$}
	{ $2$}
	\loigiai{
		Gọi $d$ là đường thẳng qua $O$ và song song với $\Delta$. Khi đó có vô số mặt phẳng chứa $d$ và không chứa $\Delta$. Vậy có vô số mặt phẳng qua $O$ và song song với $\Delta$. }
\end{ex}

\begin{ex}%[1H2B3-1]
	Có bao nhiêu mặt phẳng song song với cả hai đường thẳng chéo nhau?
	\choice
	{\True Vô số}
	{$1 $}
	{$2 $}
	{$3 $}
	\loigiai{}
\end{ex}

\begin{ex}%[1H2B3-1]
	Cho hai đường thẳng phân biệt $a$, $b$ và mặt phẳng $\left(\alpha \right)$. Giả sử $a\parallel \left(\alpha \right), b\subset \left(\alpha \right)$. Khi đó
	\choice
	{$a\parallel b $}
	{$a,b$ chéo nhau}
	{$a,b$ cắt nhau}
	{\True $a\parallel b$ hoặc $a,b$ chéo nhau}
	\loigiai{
		\begin{center}
			\begin{tikzpicture}[scale=1]
				\tkzDefPoints{0/0/A', 4/0/B', 1/2/D', 1/1/A, 4/1/B, 0/2.5/M}
				\coordinate (C') at ($(B')+(D')-(A')$);
				\tkzDefPointBy[translation = from A' to B'](M) \tkzGetPoint{N}
				\tkzLabelSegment[pos=.3](M,N){ $a$}
				\tkzLabelSegment[pos=.3](A,B){ $b$}
				%\tkzDrawSegments[dashed](A,A' A',B' A',C')
				\tkzDrawPolygon(A',B',C',D')
				\tkzDrawSegments(A,B M,N)
				%\tkzLabelPoints[right](B)
				%        \tkzLabelPoints[above](A,B)
				\tkzMarkAngles[size=.7](B',A',D')
				\tkzLabelAngle[pos=0.4](D',A',B'){\footnotesize $\alpha$ }
			\end{tikzpicture}
			\hspace{2cm}
			\begin{tikzpicture}[scale=1]
				\tkzDefPoints{0/0/A', 4/0/B', 1/2/D', 1/1/A, 4/1/B, 0/2.5/M, 1/.5/P, 4/1.75/Q}
				\coordinate (C') at ($(B')+(D')-(A')$);
				\tkzDefPointBy[translation = from A' to B'](M) \tkzGetPoint{N}
				\tkzLabelSegment[pos=.3](M,N){ $a$}
				\tkzLabelSegment[pos=.9](A,B){ $c$}
				\tkzLabelSegment[pos=.1, below](P,Q){ $b$}
				%\tkzDrawSegments[dashed](A,A' A',B' A',C')
				\tkzDrawPolygon(A',B',C',D')
				\tkzDrawSegments(A,B M,N P,Q)
				%\tkzLabelPoints[right](B)
				%        \tkzLabelPoints[above](A,B)
				\tkzMarkAngles[size=.7](B',A',D')
				\tkzLabelAngle[pos=0.4](D',A',B'){\footnotesize $\alpha$ }
			\end{tikzpicture}
		\end{center}
		Vì $a\parallel \left(\alpha \right)$ nên tồn tại đường thẳng $c\subset \left(\alpha \right)$ thỏa mãn $a\parallel c $. Suy ra $b,c$ đồng phẳng và xảy ra các trường hợp sau:
		\begin{itemize}
			\item  Nếu $b$ song song hoặc trùng với $c$ thì $a\parallel b$.
			\item  Nếu $b$ cắt $c$ thì $b$ cắt $\left(\beta \right)\equiv \left({a,c}\right)$ nên $a,b$ không đồng phẳng. Do đó $a,b$ chéo nhau.
		\end{itemize}
	}
\end{ex}

\begin{ex}%[1H2B3-1]
	Cho hai đường thẳng phân biệt $a$, $b$ và mặt phẳng $\left(\alpha \right)$. Giả sử $a\parallel b$ và  $b\parallel \left(\alpha \right)$. Khẳng định nào sau đây là khẳng định đúng?
	\choice
	{$a\parallel \left(\alpha \right) $}
	{$a\subset \left(\alpha \right) $}
	{\True $a\parallel \left(\alpha \right)$ hoặc $a\subset \left(\alpha \right) $}
	{$a$ cắt $\left(\alpha \right) $}
	\loigiai{
	}
\end{ex}

\begin{ex}%[1H2B3-1]
	Cho đường thẳng $a$ nằm trong mặt phẳng $\left(\alpha \right)$ và đường thẳng $b$ không thuộc $\left(\alpha \right)$. Mệnh đề nào sau đây đúng?
	\choice
	{Nếu $b\parallel \left(\alpha \right)$ thì $b\parallel a $}
	{\True Nếu $b\parallel a$ thì $b\parallel \left(\alpha \right) $}
	{Nếu $b$ cắt $\left(\alpha \right)$ và $\left(\beta \right)$ chứa $b$ thì giao tuyến của $\left(\alpha \right)$ và $\left(\beta \right)$ là đường thẳng cắt cả $a$ và $b $.
	}
	{Nếu $b$ cắt $\left(\alpha \right)$ thì $b$ cắt $a $}
	\loigiai{
		\begin{itemize}
			\item  A sai. Nếu $b\parallel \left(\alpha \right)$ thì $b\parallel a$ hoặc $a,b$ chéo nhau.
			\item B sai. Nếu $b$ cắt $\left(\alpha \right)$ thì $b$ cắt $a$ hoặc $a,b$ chéo nhau.
			\item  D sai. Nếu $b$ cắt $\left(\alpha \right)$ và $\left(\beta \right)$ chứa $b$ thì giao tuyến của $\left(\alpha \right)$ và $\left(\beta \right)$ là đường thẳng cắt $a$ hoặc song song với $a$.
		\end{itemize}
	}
\end{ex}


\begin{ex}%[1H2B3-1]
	Cho hai đường thẳng chéo nhau $a$ và $b$. Khẳng định nào sau đây \textbf{sai}?
	\choice
	{\True Có duy nhất một mặt phẳng song song với $a$ và $b $}
	{Có vô số đường thẳng song song với $a$ và cắt $b $}
	{Có duy nhất một mặt phẳng qua $a$ và song song với $b $}
	{Có duy nhất một mặt phẳng qua điểm $M$, song song với $a$ và $b$ (với $M$ là điểm cho trước)}
	\loigiai{
		Có có vô số mặt phẳng song song với 2 đường thẳng chéo nhau.
		}
\end{ex}


\begin{ex}%[1H2B3-1]
	Cho $d\parallel \left(\alpha \right)$, mặt phẳng $\left(\beta \right)$ qua $d$ cắt $\left(\alpha \right)$ theo giao tuyến $d'$. Khẳng định nào sau đây là đúng?
	\choice
	{$d$ cắt $d'$}
	{\True $d\parallel d' $}
	{$d$ và $d'$ chéo nhau}
	{$d\equiv d' $}
	\loigiai{
		Ta có $d'=\left(\alpha \right)\cap \left(\beta \right)$. Do $d$ và $d'$ cùng thuộc $\left(\beta \right)$ nên $d$ cắt $d'$ hoặc $d\parallel d'$.
		Nếu $d$ cắt $d'$. Khi đó, $d$ cắt $\left(\alpha \right)$ (mâu thuẫn với giả thiết).
		Vậy $d\parallel d'$.}
\end{ex}

\begin{ex}%[1H2Y3-2]
	Cho hình chóp tứ giác $S.ABCD$. Gọi $M$ và $N$ lần lượt là trung điểm của $SA$ và $SC$. Khẳng định nào sau đây đúng?
	\choice  
	{\True $MN \parallel (ABCD)$}
	{$MN \parallel (SAB)$} 
	{$MN \parallel (SCD)$}
	{$MN \parallel (SBC)$}
	\loigiai{
		\immini{Xét tam giác $SAC$ có $M, N$ lần lượt là trung điểm của $SA, SC$.\\ Suy ra $MN \parallel AC$ nên $MN \parallel (ABCD)$.}{\begin{tikzpicture}[scale=0.8, line join=round, line cap=round,thick]
				\tikzset{label style/.style={font=\footnotesize}}
				\coordinate (S) at (1,2.5);
				\coordinate (A) at (0,0);
				\coordinate (B) at (1,-0.9);
				\coordinate (D) at (3.5,0);
				\coordinate (C) at (2.5,-1.3);
				\coordinate (M) at ($(S)!0.5!(A)$);
				\coordinate (N) at ($(S)!0.5!(C)$);
				%\draw[name path=MN,dashed] (M)--(N);
				\tkzDrawSegments(S,A S,B S,C S,D A,B B,C C,D)
				\tkzDrawSegments[dashed](A,D M,N A,C)
				\tkzDrawPoints[fill=black,size=2pt](S,A,B,C,D,M,N)
				\tkzLabelPoints[above](S)
				\tkzLabelPoints[left](A,M)
				\tkzLabelPoints[below](B,C)
				\tkzLabelPoints[right](D,N)
		\end{tikzpicture}}
		
		
		
	}
\end{ex}

\begin{ex}%[1H2B3-2]
	Cho hình chóp $S.ABCD$ có đáy $ABCD$ là hình bình hành, $M$ và $N$ là hai điểm trên $SA,SB$ sao cho $\dfrac{SM}{SA}=\dfrac{SN}{SB}=\dfrac{1}{3} $. Vị trí tương đối giữa $MN$ và $\left({ABCD}\right)$ là
	\choice
	{$MN$ và $\left({ABCD}\right)$ chéo nhau}
	{\True $MN$ song song $\left({ABCD}\right)$}
	{$MN$ nằm trong $\left({ABCD}\right) $}
	{$MN$ cắt $\left({ABCD}\right) $}
	\loigiai{
		Theo định lí Talet, ta có $\dfrac{SM}{SA}=\dfrac{SN}{SB}$ suy ra $MN$ song song với $AB. $ \\
		Mà $AB$ nằm trong mặt phẳng $\left({ABCD}\right)$ suy ra $MN \parallel \left({ABCD}\right) $.}
\end{ex}

\begin{ex}%[1H2Y3-3]
	\immini[thm]{Cho hình chóp $S.ABCD$ có đáy $ABCD$ là hình bình hành. Tìm giao tuyến của hai mặt phẳng $(SAD)$ và $(SBC)$.
		\choice
		{Là đường thẳng đi qua đỉnh $S$ và song song với đường thẳng $BD$}
		{Là đường thẳng đi qua đỉnh $S$ và tâm $O$ của đáy}
		{\True Là đường thẳng đi qua đỉnh $S$ và song song với đường thẳng $BC$}
		{Là đường thẳng đi qua đỉnh $S$ và song song với đường thẳng $AB$}
	}
	{\begin{tikzpicture}[scale=0.3,>=stealth]
			\tkzDefPoints{0/0/A, 8/0/B,-3/-3/D, 2/6/S}
			\tkzDefPointBy[translation = from A to B](D)\tkzGetPoint{C}
			\tkzDrawPoints(A,B,C,D,S)
			\tkzLabelPoints[below](A,B,C,D)
			\tkzLabelPoints[above](S)
			\tkzDrawSegments (S,B S,C S,D D,C C,B)
			\tkzDrawSegments[dashed](D,A A,B S,A)
		\end{tikzpicture}
	}
	\loigiai{
		\immini{Do hai mặt phẳng $(SAD)$ và $(SBC)$ có chung điểm $S$ và có hai đường thẳng $AD$, $BC$ song song với nhau nên giao tuyến của hai mặt phẳng $(SAD)$ và $(SBC)$ là đường thẳng đi qua đỉnh $S$ và song song với đường thẳng $BC$.
		}{
			\begin{tikzpicture}[scale=0.3,>=stealth]
				\tkzDefPoints{0/0/A, 8/0/B,-3/-3/D, 2/6/S}
				\tkzDefPointBy[translation = from A to B](D)\tkzGetPoint{C}
				\tkzDefPointBy[translation = from A to D](S)\tkzGetPoint{s}
				\tkzDefPointBy[homothety = center S ratio -0.7](s)\tkzGetPoint{s1}
				\tkzDrawPoints(A,B,C,D,S)
				\tkzLabelPoints[below](A,B,C,D)
				\tkzLabelPoints[above](S)
				\tkzDrawSegments (S,B S,C S,D D,C C,B s,s1)
				\tkzDrawSegments[dashed](D,A A,B S,A)
			\end{tikzpicture}
	}}
\end{ex}

\begin{ex}%[Hà Lê]%[1H2B3]
	Cho tứ diện $ABCD$ có $I$, $J$ lần lượt là trung điểm của $BC$, $BD$. Giao tuyến của mặt phẳng $(AIJ)$ và $(ACD)$ là
	\haicot
	{đường thẳng $d$ đi qua $A$ và song song với $BC$}
	{đường thẳng $d$ đi qua $A$ và song song với $BD$}
	{\True đường thẳng $d$ đi qua $A$ và song song với $CD$}
	{đường thẳng $AB$}
\end{ex}

\begin{ex}%[1H2B3-2]
	Cho tứ diện $ABCD$. Gọi $G$ là trọng tâm của tam giác $ABD,\ Q$ thuộc cạnh $AB$ sao cho $AQ=2QB$, $P$ là trung điểm của $AB $, $M$ là trung điểm của $BD$. Khẳng định nào sau đây đúng?
	\choice
	{$Q \in \left({CDP}\right) $}
	{$QG$ cắt $\left({BCD}\right) $}
	{$MP \parallel \left({BCD}\right) $}
	{\True $GQ \parallel \left({BCD}\right) $}
	\loigiai{
		\immini{
			Vì $G$ là trọng tâm tam giác $ABD$ $\Rightarrow \dfrac{AG}{AM}=\dfrac{2}{3} $.\\
			Điểm $Q\in AB$ sao cho $AQ=2QB\Leftrightarrow \dfrac{AQ}{AB}=\dfrac{2}{3} $. Suy ra $\dfrac{AG}{AM}=\dfrac{AQ}{AB}\xrightarrow{}GQ \parallel BD. $\\
			Mặt khác $BD$ nằm trong mặt phẳng $\left({BCD}\right)$ suy ra $GQ \parallel \left({BCD}\right) $.
		}{
			\begin{tikzpicture}[scale=1]
			\tkzDefPoints{1/3/A, 0/0/B, 2/-2/C, 5/0/D}
			\tkzCentroid(A,B,D)\tkzGetPoint{G}
			\coordinate (M) at ($(B)!.5!(D)$);
			\coordinate (Q) at ($(A)!.67!(B)$);
			\coordinate (P) at ($(A)!.5!(B)$);
			\tkzDrawSegments[dashed](A,M B,D Q,G P,D)
			\tkzDrawPolygon(A,B,C,D)
			\tkzDrawSegments(A,C)
			\tkzLabelPoints[left](B,P,Q)
			\tkzLabelPoints[right](D)
			\tkzLabelPoints[above](A)
			\tkzLabelPoints[below](M,C)
			\tkzLabelPoints[above right](G)
			\tkzDrawPoints(A,B,C,D,M,P,Q,G)
			\end{tikzpicture}
		}
	}
\end{ex}


\begin{ex}%[1H2K3-2]
	Cho hai hình bình hành $ABCD$ và $ABEF$ không cùng nằm trong một mặt phẳng. Gọi $O$, $O_1$ lần lượt là tâm của $ABCD$, $ABEF $; $M$ là trung điểm của $CD. $ Khẳng định nào sau đây \textbf{sai}?
	\choice
	{$OO_1 \parallel \left({BEC}\right) $}
	{$OO_1 \parallel \left({EFM}\right) $}
	{\True $MO_1$ cắt $\left({BEC}\right) $}
	{$OO_1 \parallel \left({AFD}\right) $}
	\loigiai{
		\immini{
			Xét tam giác $ACE$ có $O,O_1$ lần lượt là trung điểm của $AC,AE $.\\
			Suy ra $OO_1$ là đường trung bình trong tam giác $ACE\Rightarrow OO_1 \parallel EC. $\\
			Tương tự, $OO_1$ là đường trung bình của tam giác $BFD$ nên $OO_1 \parallel FD. $\\
			Vậy $OO_1 \parallel \left({BEC}\right), OO_1 \parallel \left({AFD}\right)$ và $OO_1 \parallel \left({EFC}\right)$. Chú ý rằng: $\left({EFC}\right)\equiv \left({EFM}\right) $.
		}{
			\begin{tikzpicture}[scale=1]
				\tkzDefPoints{0/0/A, 4/0/B, 1/2/D, 1.5/-2.5/F}
				\coordinate (C) at ($(D)+(B)-(A)$);
				\coordinate (E) at ($(F)+(B)-(A)$);
				\coordinate (M) at ($(C)!0.5!(D)$);
				\tkzInterLL(A,C)(B,D)\tkzGetPoint{O}
				\tkzInterLL(A,E)(B,F)\tkzGetPoint{O_1}
				\tkzDrawSegments[dashed](B,D A,C A,E B,F O,O_1 A,B B,E B,C)
				\tkzDrawPolygon(D,C,E,F)
				\tkzDrawSegments(A,D A,F)
				\tkzLabelPoints[left](A)
				\tkzLabelPoints[right](B)
				\tkzLabelPoints[below](E,F,O_1)
				\tkzLabelPoints[above](C,D,O,M)
				\tkzDrawPoints(A,B,C,D,O,E,F,O_1,M)
			\end{tikzpicture}
		}
	}
\end{ex}

\begin{ex}%[HK1 THPT Quốc Thái, An Giang 2018]%[Lê Nguyễn Viết Tường-DA 11HK1-18]%[1H2K3-4]%
	Cho hình chóp $S.ABCD$ có đáy là hình bình hành. Thiết diện của hình chóp khi cắt bởi mặt phẳng đi qua trung điểm $M$ của cạnh $AB$ và song song với $BD,SA$ là hình gì?
	\choice
	{Ngũ giác}
	{Hình thang}
	{Tam giác}
	{\True Hình bình hành}
	\loigiai{
		\immini{
			Gọi $(\alpha)$ là mặt phẳng đi qua $M$ và song song với $BD,SA$.
			Ta có
			$BD\parallel (\alpha),BD\subset (ABCD),(\alpha)\cap (ABCD)=Mx$
			$\Rightarrow Mx\parallel BD\Rightarrow Mx$ cắt $AD$ tại $N$ trong $(ABCD)$.
			$SA\parallel (\alpha),SA\subset (SAD),(\alpha)\cap (SAD)=Ny$
			$\Rightarrow Ny\parallel SA\Rightarrow Ny$ cắt $SD$ tại $P$ trong $(SAD)$.
			$SA\parallel (\alpha),SA\subset (SAB),(\alpha)\cap (SAB)=Mt$
			$\Rightarrow Mt\parallel SA\Rightarrow Mt$ cắt $SB$ tại $Q$ trong $(SAB)$.
			Vậy thiết diện là hình bình hành $MNPQ$.
		}{
			\begin{tikzpicture}[line join = round, line cap = round]
				\tkzDefPoints{0/0/D,1.5/1.6/A,5.5/1.6/B,4/0/C,2.5/6/S}
				\tkzDefMidPoint(A,B)\tkzGetPoint{M}
				\tkzDefMidPoint(A,D)\tkzGetPoint{N}
				\tkzDefMidPoint(S,D)\tkzGetPoint{P}
				\tkzDefMidPoint(S,B)\tkzGetPoint{Q}
				\tkzLabelPoints[](M,N)
				\tkzLabelPoints[above](S)
				\tkzLabelPoints[above right](A)
				\tkzLabelPoints[right](B,C,Q)
				\tkzLabelPoints[left](D,P)
				\tkzDrawPoints[fill=black](S,A,B,C,D,M,N,P,Q)
				\tkzDrawSegments[](S,B S,C S,D B,C C,D)
				\tkzDrawSegments[dashed](S,A A,B A,D M,N N,P B,D M,Q Q,P)
			\end{tikzpicture}
		}
	}
\end{ex}
% \centerline{---HẾT---}
\Closesolutionfile{ans}


%%Bài 13. Hai MP song song
% \setcounter{section}{12}
\setcounter{dang}{0}
\section{HAI MẶT PHẲNG SONG SONG}
\subsection{KIẾN THỨC CẦN NHỚ}

\subsubsection{VỊ TRÍ TƯƠNG ĐỐI CỦA HAI MẶT PHẲNG}
Cho hai mặt phẳng $(P)$ và $(Q)$. Các trường hợp có thể xảy ra:
\begin{itemize}
	\item [\iconMT] \indam{Trường hợp 1:} $(P)$ và $(Q)$ trùng nhau.
	\item [\iconMT] \indam{Trường hợp 2:} $(P)$ và $(Q)$ có một điểm chung. Khi đó chúng sẽ có điểm chung khác nữa. Tập hợp tất cả các điểm chung đó gọi là  giao tuyến của hai mặt phẳng $(P)$ và $(Q)$ (\textbf{Hình 1}).
	\item [\iconMT] \indam{Trường hợp 3:} $(P)$ và $(Q)$ không có điểm chung. Khi đó ta nói $(P)$ song song $(Q)$ (\textbf{Hình 2}).
	\begin{itemize}
		\item Kí hiệu $(P)\parallel (Q)$;
		\item Khi $(P)\parallel (Q)$ và $a \subset (P)$ thì $a \parallel (Q)$.
	\end{itemize}
\end{itemize}
\hspace*{1cm}
\begin{tikzpicture}[line join = round, line cap = round,>=stealth,font=\footnotesize,scale=0.7]
\tkzDefPoints{0/0/A,5/0/B,6.5/1.5/C}
\coordinate (D) at ($(A)+(C)-(B)$);
\coordinate (E) at ($(A)!0.5!(B)$);
\coordinate (F) at ($(D)!0.5!(C)$);
\coordinate (M) at ($(E)+(-0.5,2)$);
\coordinate (N) at ($(F)+(-0.5,2)$);
\coordinate (Q) at ($(M)!2!(E)$);
\coordinate (P) at ($(N)!2!(F)$);
\tkzInterLL(M,Q)(C,D)    \tkzGetPoint{G}
\tkzInterLL(N,P)(A,B)    \tkzGetPoint{H}
\draw ($(A)+(0.6,0.2)$) node {$P$};
\draw ($(N)+(-0.1,-0.5)$) node {$Q$};
\tkzDrawSegments(A,B B,C C,F G,D N,F H,P D,A M,N P,Q Q,M E,F)
\tkzDrawSegments[dashed](G,F F,H)
\tkzMarkAngles[size=1cm,arc=l](B,A,D)
\tkzMarkAngles[size=1cm,arc=l](M,N,P)
\draw (3.25,-2.5) node {\textbf{Hình 1.}};
\draw (3.5,1) node[below] {$d$};
\draw (3.25,-3.5) node {$(P)$, $(Q)$ cắt nhau: $(P)\cap (Q)=d$};
\end{tikzpicture}
\hspace*{2cm}
\begin{tikzpicture}[line join = round, line cap = round,>=stealth,font=\footnotesize,scale=0.7]
\begin{scope}[shift={(8,0)}]
\tkzDefPoints{0/0/A,5/0/B,6.5/2/C}
\coordinate (D) at ($(A)+(C)-(B)$);
\draw ($(A)+(0.5,0.3)$) node {$P$};
\tkzDrawSegments(A,B B,C C,D D,A)
\tkzMarkAngles[size=1cm,arc=l](B,A,D)
\end{scope}
\begin{scope}[shift={(8,-3)}]
\tkzDefPoints{0/0/A,5/0/B,6.5/2/C}
\coordinate (D) at ($(A)+(C)-(B)$);
\draw ($(A)+(0.5,0.3)$) node {$Q$};
\draw (2,4)--(5,4.5)node[below]{$a$};
\tkzDrawSegments(A,B B,C C,D D,A)
\tkzMarkAngles[size=1cm,arc=l](B,A,D)
\draw (3.25,-1) node {\textbf{Hình 2.}};
\draw (3.25,-2) node {$(P)$, $(Q)$ không có điểm chung: $(P)\parallel (Q)$};
\end{scope}
\end{tikzpicture}
\subsubsection{CÁC ĐỊNH LÝ CƠ BẢN}
\begin{enumerate}[\iconMT]
	\item \indam{Định lý 1:} \immini{
		Nếu mặt phẳng $(\alpha)$ chứa hai đường thẳng cắt nhau $a$, $b$ và $a$, $b$ cùng song song với mặt phẳng $(\beta)$ thì $(\alpha)$ song song với $(\beta)$.
	}
	{
		\begin{tikzpicture}[line join = round, line cap = round,>=stealth,font=\footnotesize,scale=0.7]
			\begin{scope}[shift={(0,0)}]
				\tkzDefPoints{0/0/A,5/0/B,6.5/1.5/C}
				\coordinate (D) at ($(A)+(C)-(B)$);
				\coordinate (a) at ($(A)+(1.5,0.3)$);
				\coordinate (b) at ($(A)+(4.5,0.3)$);
				\coordinate (c) at ($(A)+(5.5,1.3)$);
				\coordinate (d) at ($(A)+(1.7,1.3)$);
				\tkzInterLL(a,c)(b,d)    \tkzGetPoint{M}
				\draw ($(A)+(0.5,0.2)$) node {$\alpha$};
				\tkzDrawSegments(A,B B,C C,D D,A a,c b,d)
				\tkzMarkAngles[size=0.8cm,arc=l](B,A,D)
				\tkzLabelPoints[above](M,a,b)
				\tkzDrawPoints[fill=black](M)
			\end{scope}
			\begin{scope}[shift={(0,-2)}]
				\tkzDefPoints{0/0/A,5/0/B,6.5/1.5/C}
				\coordinate (D) at ($(A)+(C)-(B)$);
				\draw ($(A)+(0.5,0.2)$) node {$\beta$};
				\tkzDrawSegments(A,B B,C C,D D,A)
				\tkzMarkAngles[size=0.8cm,arc=l](B,A,D)
			\end{scope}
		\end{tikzpicture}
	}

	\begin{note}
		\begin{itemize}
			\item Muốn chứng minh hai mặt phẳng song song, ta phải chứng minh có hai đường thẳng cắt nhau thuộc mặt phẳng này lần lượt song song với mặt phẳng kia.
			\item Muốn chứng minh đường thẳng $a\parallel
			(Q)$, ta chứng minh đường thẳng $a$ nằm trong mặt
			phẳng $(P)$ và $(P)\parallel (Q)$.
		\end{itemize}
	\end{note}
	\item \indam{Định lý 2:} \immini{
		Qua một điểm nằm ngoài một mặt phẳng cho trước có một và chỉ một mặt phẳng song song với mặt phẳng đã cho.
	}
	{
		\begin{tikzpicture}[line join = round, line cap = round,>=stealth,font=\footnotesize,scale=0.7]
			\begin{scope}[shift={(0,0)}]
				\tkzDefPoints{0/0/A,5/0/B,6.5/1.5/C}
				\coordinate (D) at ($(A)+(C)-(B)$);
				\coordinate (a) at ($(A)+(3.5,0.8)$);
				\draw ($(A)+(0.5,0.2)$) node {$\alpha$};
				\draw ($(a)+(0,-0.1)$) node[above] {$A$};
				\tkzDrawSegments(A,B B,C C,D D,A)
				\tkzMarkAngles[size=0.8cm,arc=l](B,A,D)
				\tkzDrawPoints[fill=black](a)
			\end{scope}
			\begin{scope}[shift={(0,-2)}]
				\tkzDefPoints{0/0/A,5/0/B,6.5/1.5/C}
				\coordinate (D) at ($(A)+(C)-(B)$);
				\draw ($(A)+(0.5,0.2)$) node {$\beta$};
				\tkzDrawSegments(A,B B,C C,D D,A)
				\tkzMarkAngles[size=0.8cm,arc=l](B,A,D)
			\end{scope}
		\end{tikzpicture}
	}
\end{enumerate}
\immini{
\iconMT\,\indam{Định lý 3:}Cho hai mặt phẳng song song. Nếu một mặt phẳng cắt mặt phẳng này thì cũng cắt mặt phẳng kia và hai giao tuyến song song với nhau.}
{\begin{tikzpicture}[font=\footnotesize]
	\coordinate (A) at (0,0);
	\coordinate (B) at (3.1,0);
	\coordinate (C) at (4,1.2);
	\coordinate (A1) at (0,1.7);
	\coordinate (M) at (1,3.2);
	\coordinate (N) at (2,-1.5);
	\coordinate (D) at ($(C)-(B)+(A)$);
	\coordinate (B1) at ($(B)-(A)+(A1)$);
	\coordinate (C1) at ($(C)-(B)+(B1)$);
	\coordinate (D1) at ($(C1)-(B1)+(A1)$);
	\coordinate (Q) at ($(D1)-(A1)+(M)$);
	\coordinate (P) at ($(N)-(M)+(Q)$);
	\path[name path=mn] (M)--(N); 
	\path[name path=pq] (P)--(Q);
	\path[name path=ab] (A)--(B); 
	\path[name path=cd] (C)--(D);
	\path[name path=a1b1] (A1)--(B1); 
	\path[name path=c1d1] (C1)--(D1);
	\path[name intersections={of=mn and ab,by=E}];
	\path[name intersections={of=mn and a1b1,by=F}];
	\path[name intersections={of=mn and cd,by=E1}];
	\path[name intersections={of=mn and c1d1,by=F1}];
	\path[name intersections={of=pq and ab,by=G1}];
	\path[name intersections={of=pq and cd,by=G}];
	\path[name intersections={of=pq and a1b1,by=H1}];
	\path[name intersections={of=pq and c1d1,by=H}];
	\draw (E1)--(D)--(A)--(B)--(C)--(G)--(E) (F1)--(D1)--(A1)--(B1)--(C1)--(H)--(F) (H)--(Q)--(M)--(N)--(P)--(G1) (G)--(H1);
	\draw[dashed] (H1)--(H)--(F1) (G1)--(G)--(E1);
	\draw ($(E)!.6!(G)$) node[left]{$a$} ($(F)!.6!(H)$) node[left]{$b$};
	\draw pic[draw,"$\beta$"]{angle=B--A--D} pic[draw,"$\alpha$"]{angle=B1--A1--D1} pic[draw,"$\gamma$", angle eccentricity=0.7]{angle=M--Q--P};
\end{tikzpicture}}
\immini{\iconMT\,\indam{Định lý 4:} (\textit{Định lí Thales}) Ba mặt phẳng đôi một song song chắn trên hai cát tuyến bất kì những đoạn thẳng tương ứng tỉ lệ.}
{\begin{tikzpicture}[line join = round, line cap = round,>=stealth,font=\footnotesize,scale=0.7]
	\tkzDefPoints{0/0/A,5/0/B,6.5/1.5/C,3.5/2/d,4/2/d',1/-4.9/e,4.8/-4.9/e'}
	\coordinate (D) at ($(A)+(C)-(B)$);
	\coordinate (E) at ($(A)+(0,-2.2)$);
	\coordinate (F) at ($(E)+(5,0)$);
	\coordinate (G) at ($(E)+(6.5,1.5)$);
	\coordinate (H) at ($(E)+(G)-(F)$);
	\coordinate (M) at ($(A)+(0,-4.4)$);
	\coordinate (N) at ($(M)+(5,0)$);
	\coordinate (P) at ($(M)+(6.5,1.5)$);
	\coordinate (Q) at ($(M)+(P)-(N)$);
	\draw ($(A)+(0.5,0.2)$) node {$\alpha$};
	\draw ($(E)+(0.5,0.2)$) node {$\beta$};
	\draw ($(M)+(0.5,0.2)$) node {$\gamma$};
	\tkzInterLL(d,e)(A,B)    \tkzGetPoint{m}
	\tkzInterLL(d',e')(A,B)    \tkzGetPoint{n}
	\tkzInterLL(d,e)(E,F)    \tkzGetPoint{p}
	\tkzInterLL(d',e')(E,F)    \tkzGetPoint{q}
	\tkzInterLL(d,e)(M,N)    \tkzGetPoint{r}
	\tkzInterLL(d',e')(M,N)    \tkzGetPoint{s}
	\coordinate (a) at ($(d)!0.2!(e)$);
	\coordinate (b) at ($(d)!0.5!(e)$);
	\coordinate (c) at ($(d)!0.8!(e)$);
	\coordinate (a') at ($(d')!0.15!(e')$);
	\coordinate (b') at ($(d')!0.5!(e')$);
	\coordinate (c') at ($(d')!0.85!(e')$);
	\tkzDrawSegments(A,B B,C C,D D,A E,F F,G G,H H,E M,N N,P P,Q Q,M d,a m,b p,c r,e d',a' n,b' q,c' s,e' a,a' b,b' c,c')
	\draw (a) node[left] {$A$};
	\draw (a') node[right] {$A'$};
	\draw (b) node[left] {$B$};
	\draw (b') node[right] {$B'$};
	\draw (c) node[left] {$C$};
	\draw (c') node[right] {$C'$};
	\tkzDrawSegments[dashed](a,m b,p c,r a',n b',q c',s)
	\tkzMarkAngles[size=0.7cm,arc=l](B,A,D F,E,H N,M,Q)
	\tkzDrawPoints[fill=black](a,a',b,b',c,c')
\end{tikzpicture}}

\subsubsection{HÌNH LĂNG TRỤ VÀ HÌNH HỘP}
\begin{enumerate}[\iconMT]
	\immini{\item \indam{Định nghĩa:}
	Cho hai mặt phẳng $(\alpha)\parallel(\alpha')$. Trong $(\alpha)$ cho đa giác lồi $A_1A_2\ldots A_n$. Qua các điểm $A_1,A_2,\ldots,A_n$ ta dựng các đường song song với nhau và cắt $(\alpha')$ tại $A'_1,A'_2,\ldots,A'_n$.
	
	Hình tạo thành bởi hai đa giác $A_1A_2\ldots A_n$, $A'_1A'_2\ldots A'_n$ cùng với các hình bình hành $A_1A_2A'_2A'_1$, $A_2A_3A'_3A'_2$, \ldots, $A_nA_1A'_1A'_n$ được gọi là \textit{hình lăng trụ} và được ký hiệu bởi $A_1A_2\ldots A_n.A'_1A'_2\ldots A'_n$.
	
	\begin{itemize}
		\item Hai đa giác $A_1A_2\ldots A_n$, $A'_1A'_2\ldots A'_n$ được gọi là hai \textit{mặt đáy} (bằng nhau) của hình lăng trụ.
		\item Các đoạn thẳng $A_1A'_1$, $A_2A'_2$,\ldots, $A_nA'_n$ gọi là các \textit{cạnh bên} của hình lăng trụ.
		\item Các hình bình hành $A_1A_2A'_2A'_1$, $A_2A_3A'_3A'_2$,\ldots, $A_nA_1A'_1A'_n$ gọi là các \textit{mặt bên} của hình lăng trụ.
		\item Các đỉnh của hai đa giác đáy gọi là các \textit{đỉnh} của hình lăng trụ.
	\end{itemize}
}{
	\begin{tikzpicture}[font=\footnotesize]
	\coordinate (A) at (0,0);
	\coordinate (B) at (5.5,0);
	\coordinate (C) at (6.5,2);
	\coordinate (A1) at (0,4.5);
	\coordinate (M1) at (1,7);
	\coordinate (N1) at (2,-1);
	\coordinate (D) at ($(C)-(B)+(A)$);
	\coordinate (B1) at ($(B)-(A)+(A1)$);
	\coordinate (C1) at ($(C)-(B)+(B1)$);
	\coordinate (D1) at ($(C1)-(B1)+(A1)$);
	\coordinate (A11) at ($(M1)!.78!(N1)$);
	\coordinate[shift={(1,-0.3)}] (M2) at (1,7);
	\coordinate[shift={(3,-0.2)}] (M3) at (1,7);
	\coordinate[shift={(3.5,0.5)}] (M4) at (1,7);
	\coordinate[shift={(1.5,0.7)}] (M5) at (1,7);
	\coordinate[shift={(1,-0.3)}] (N2) at (2,-1);
	\coordinate[shift={(3,-0.2)}] (N3) at (2,-1);
	\coordinate[shift={(3.5,0.5)}] (N4) at (2,-1);
	\coordinate[shift={(1.5,0.7)}] (N5) at (2,-1);
	\coordinate[shift={(1,-0.3)}] (B11) at ($(M1)!.78!(N1)$);
	\coordinate[shift={(3,-0.2)}] (C11) at ($(M1)!.78!(N1)$);
	\coordinate[shift={(3.5,0.5)}] (D11) at ($(M1)!.78!(N1)$);
	\coordinate[shift={(1.5,0.7)}] (E11) at ($(M1)!.78!(N1)$);
	\coordinate (A22) at ($(M1)!.22!(N1)$);
	\coordinate[shift={(1,-0.3)}] (B22) at ($(M1)!.22!(N1)$);
	\coordinate[shift={(3,-0.2)}] (C22) at ($(M1)!.22!(N1)$);
	\coordinate[shift={(3.5,0.5)}] (D22) at ($(M1)!.22!(N1)$);
	\coordinate[shift={(1.5,0.7)}] (E22) at ($(M1)!.22!(N1)$);
	\path[name path=k1] (M1)--(N1); 
	\path[name path=k2] (M2)--(N2);
	\path[name path=k3] (M3)--(N3);
	\path[name path=k4] (M4)--(N4);
	\path[name path=k5] (M5)--(N5);
	\path[name path=ab] (A)--(B); 
	\path[name path=cd] (C)--(D);
	\path[name path=a1b1] (A1)--(B1);
	\path[name intersections={of=k1 and ab,by=E1}];
	\path[name intersections={of=k2 and ab,by=E2}];
	\path[name intersections={of=k3 and ab,by=E3}];
	\path[name intersections={of=k4 and ab,by=E4}];
	\path[name intersections={of=k5 and ab,by=E5}];
	\path[name intersections={of=k1 and cd,by=F1}];
	\path[name intersections={of=k4 and cd,by=F4}];
	\path[name intersections={of=k1 and a1b1,by=G1}];
	\path[name intersections={of=k2 and a1b1,by=G2}];
	\path[name intersections={of=k3 and a1b1,by=G3}];
	\path[name intersections={of=k4 and a1b1,by=G4}];
	\path[name intersections={of=k5 and a1b1,by=G5}];
	\draw (M1)--(A22) (M2)--(B22) (M3)--(C22) (M4)--(D22) (M5)--(E22);
	\draw (F1)--(D)--(A)--(B)--(C)--(F4) (A1)--(B1)--(C1)--(D1)--cycle (A11)--(G1) (B11)--(G2) (C11)--(G3) (D11)--(G4) (E1)--(N1) (E2)--(N2) (E3)--(N3) (E4)--(N4) (E5)--(N5) (A11)--(B11)--(C11)--(D11) (A22)--(B22)--(C22)--(D22)--(E22)--cycle;
	\draw[dashed] (A11)--(E1) (B11)--(E2) (C11)--(E3) (D11)--(E4) (E11)--(E5) (F1)--(F4) (A22)--(G1) (B22)--(G2) (C22)--(G3) (D22)--(G4) (E22)--(E11) (A11)--(E11)--(D11);
	\draw (A11) node[left] {$A_1$} (A22) node[left] {$A_1'$} (B11) node[below left] {$A_2$} (C11) node[below left] {$A_3$} (D11) node[right] {$A_4$} (E11) node[above left] {$A_5$} (B22) node[above right] {$A_2'$} (C22) node[below left] {$A_3'$} (D22) node[right] {$A_4'$} (E22) node[above left] {$A_5'$};
	\tkzMarkAngles[size=0.7cm,arc=l](B,A,D B1,A1,D1)
	\tkzLabelAngles[pos=0.5](B,A,D){$\alpha$}
	\tkzLabelAngles[pos=0.5](B1,A1,D1){$\alpha'$}
	\end{tikzpicture}}
	\item \indam{Tính chất:}
		\begin{itemize}
			\item Các cạnh bên của hình lăng trụ thì song song và bằng nhau.
			\item Các mặt bên của hình lăng trụ đều là hình bình hành.
			\item Hai đáy của hình lăng trụ là hai đa giác bằng nhau.
	\end{itemize}
	\begin{tcolorbox}[colframe=cyan,colback=red!3!white,boxrule=0.5mm]
		Hình lăng trụ có đáy là hình bình hành gọi là \textit{hình hộp}. 
		\begin{itemize}
			\item Các mặt của hình hộp là hình bình hành.
			\item Hai mặt phẳng lần lượt chứa hai mặt đối diện của hình hộp thì song song nhau.
		\end{itemize}
	\end{tcolorbox}
	\item \indam{Minh họa vài mô hình thường gặp:}\\
	\begin{tabular}{llll}
		\begin{tikzpicture}[line cap=round,line join=round, scale=.6]%[Hoàng Anh]
			\tkzDefPoints{0/0/A, 1.5/1/B, 3.5/-0.5/C, -0.5/3.5/z}
			\coordinate (D) at ($(A)+(z)$);
			\coordinate (E) at ($(B)+(D)-(A)$);%Vẽ hình bình hành ABED
			\coordinate (F) at ($(C)+(E)-(B)$);
			\tkzDrawPolygon(E,F,D) %Vẽ đa giác EFD
			\tkzDrawSegments(D,A C,F A,C) %Vẽ các đoạn thẳng AD, CF, AC
			\tkzDrawSegments[dashed](A,B B,C B,E)%Vẽ nét đứt các đoạn thẳng AB, BC, BE
		\end{tikzpicture}
		&
		\begin{tikzpicture}[line cap=round,line join=round,  scale=0.5]
			%-------------- Đáy ABCD
			\tkzDefPoints{0/0/A, -0.5/-1/B, 3/0/D, 1/-1/C}
			
			%-------------- Đáy A'B'C'D'
			\tkzDefPointBy[rotation = center A angle 100](D) \tkzGetPoint{A'} %Phép quay tâm A, góc quay 90 độ, biến D thành A'
			\coordinate (B') at ($(B)+(A')-(A)$);
			\coordinate (D') at ($(D)+(A')-(A)$);
			\coordinate (C') at ($(C)+(B')-(B)$);
			%---------------
			\tkzDrawSegments[dashed](A,B A,D A,A')
			\tkzDrawPolygon(A',B',C',D')
			\tkzDrawPolygon(B,C,C',B')
			\tkzDrawSegments(D,D' C,D)
		\end{tikzpicture}
		&
		\begin{tikzpicture}[line cap=round,line join=round,scale=0.8]
			%-------------- Đáy ABCD
			\tkzDefPoints{0/0/A, -0.5/-1/B, 2/0/D}
			\coordinate (C) at ($(B)+(D)-(A)$);
			%-------------- Đáy A'B'C'D'
			\tkzDefPointBy[rotation = center A angle 100](D) \tkzGetPoint{A'} %Phép quay tâm A, góc quay 90 độ, biến D thành A'
			\coordinate (B') at ($(B)+(A')-(A)$);
			\coordinate (D') at ($(D)+(A')-(A)$);
			\coordinate (C') at ($(B')+(D')-(A')$);
			%---------------
			\tkzDrawSegments[dashed](A,B A,D A,A')
			\tkzDrawPolygon(A',B',C',D')
			\tkzDrawPolygon(B,C,C',B')
			\tkzDrawSegments(D,D' C,D)
		\end{tikzpicture}
		&
		\begin{tikzpicture}[line cap=round, line join=round, scale=.45]
			\tkzDefPoints{1/5/A, 3.5/5/B,  6/5.5/C, 5/7.5/D, 2/7.5/E, 0/0/A'}
			\tkzDefPointsBy[translation= from A to A'](B,C,D,E){B',C',D',E'}
			\tkzFillPolygon[white](A,B,B',A')
			\tkzFillPolygon[white](C,B,B',C')
			\tkzDrawPolygon(A,B,C,D,E)
			\tkzDrawSegments[dashed](E,E' D,D' A',E' E',D' D',C')
			\tkzDrawSegments(A,A' B,B' C,C' A',B' B',C')
		\end{tikzpicture}\\
		\small * Lăng trụ tam giác & \small * Lăng trụ tứ giác  & \small * Hình hộp  & \small * Lăng trụ ngũ giác 
	\end{tabular}
\end{enumerate}
% \subsection{PHÂN LOẠI, PHƯƠNG PHÁP GIẢI TOÁN}
\begin{dang}{Chứng minh hai mặt phẳng song song}
\begin{enumerate}[\iconMT]
	\item \indam{Phương pháp:} 
	\immini{Chứng minh trên mặt phẳng này có hai đường thẳng cắt nhau cùng song song với mặt phẳng còn lại.
		$$\heva{&a \text{ cắt } b\\&a \subset (\alpha),\,b\subset (\alpha)\\&a\parallel(\beta),\, b\parallel (\beta)}\quad\Rightarrow (\alpha) \parallel (\beta).$$

}{
\begin{tikzpicture}[scale=0.4]
\tkzDefPoints{0/0/A, 9/0/B, 12/2.5/C}
\coordinate (A1) at ($(A)+(0,3)$);
\coordinate (B1) at ($(B)+(0,3)$);
\coordinate (C1) at ($(C)+(0,3)$);
\coordinate (D) at ($(A)-(B)+(C)$);
\coordinate (D1) at ($(A1)-(B1)+(C1)$);
\tkzDefPoints{2/3.3/a1, 9/5/a2, 3/5/b1, 8/3.3/b2}
\tkzDrawSegments(a1,a2 b1,b2 A,B B,C C,D D,A A1,B1 B1,C1 C1,D1 D1,A1)
\tkzDefPoint[label=above right:$a$](2.5,3.5){a}
\tkzDefPoint[label=right:$b$](4,5){b}
\tkzMarkAngles[size=1.6cm,arc=l](B,A,D)
\tkzMarkAngles[size=1.6cm,arc=l](B1,A1,D1)
\tkzLabelAngles[pos=1,rotate=30](B1,A1,D1){\scriptsize$\alpha$}
\tkzLabelAngles[pos=1,rotate=30](B,A,D){\scriptsize$\beta$}
%	\draw pic[draw,"$\beta$",  angle radius=1cm]{angle=B--A--D} pic[draw,"$\alpha$", angle radius=1cm]{angle=B1--A1--D1};
\end{tikzpicture}}
	\item \indam{Chú ý:} Hai mặt phẳng phân biệt cùng song song với mặt phẳng thứ ba thì song song nhau.
\end{enumerate}
\end{dang}

\begin{vd}
	Cho hình chóp $S.ABCD$ có đáy là hình bình hành tâm $O$. Gọi $M$, $N$, $P$ lần lượt là trung điểm của $SA$, $SD$ và $SB$.
	\begin{enumerate}
		\item Chứng minh rằng $(MNP)\parallel (ABCD)$.
		\item Chứng minh rằng $(OMN)\parallel (SBC)$.
	\end{enumerate}
	\loigiai{
		\begin{center}
			\begin{tikzpicture}[scale=0.8]
				\tkzDefPoints{-2/3/A, 0/0/B, 6/0/C, 3.4/7/S}
				\coordinate (D) at ($(C)-(B)+(A)$);
				\tkzInterLL(A,C)(B,D)\tkzGetPoint{O}
				\tkzDefMidPoint(S,A)\tkzGetPoint{M}
				\tkzDefMidPoint(S,D)\tkzGetPoint{N}
				\tkzDefMidPoint(S,B)\tkzGetPoint{P}
				%\tkzDefPointBy[homothety=center N ratio 0.25](M)\tkzGetPoint{H}
				\tkzDrawSegments(A,B B,C S,A S,B S,C M,P)
				\tkzDrawSegments[dashed](A,D C,D A,C N,P B,D S,D M,N O,M O,N)
				\tkzDrawPoints(A,B,C,D,O,S,M,N,P)
				\tkzLabelPoints[above left](A,M)
				\tkzLabelPoints[below left](B)
				\tkzLabelPoints[below right](C,P)
				\tkzLabelPoints[right](D,N,D)
				\tkzLabelPoints[below](O)
				\tkzLabelPoints[above](S)
			\end{tikzpicture}
		\end{center}
		\begin{enumerate}
			\item Chứng minh $(MNP)\parallel (ABCD)$.
			
			Ta có
			
			$\begin{cases}MN\parallel AD,\text{ (do $MN$ là đường trung bình của $\Delta SAD$)}\\AD\subset (ABCD).\end{cases}$
			Suy ra  $MN\parallel (ABCD).$
			
			Ta lại có
			
			$\begin{cases}NP\parallel AB,\text{ (do $NP$ là đường trung bình của $\Delta SAB$)}\\AB\subset (ABCD).\end{cases}$
			Suy ra  $NP\parallel (ABCD).$
			
			Mặt khác, $MN,NP\subset (ABCD)$.
			
			Vậy $(MNP)\parallel (ABCD)$.
			\item Chứng minh $(OMN)\parallel (SBC)$.
			
			Ta có $MN\parallel AD$, ($MN$ là đường trung bình của $\Delta SAD$) và $AD\parallel BC$, (do $ABCD$ là hình bình hành) nên $MN\parallel BC$.
			
			Mà $BC\subset (SBC)$ nên $MN\parallel (SBC)$.
			
			Ta lại có $OM\parallel SC$, (do $OM$ là đường trung bình của $\Delta SAC$).
			
			Mà $SC\subset (SBC)$ nên $OM\parallel (SBC)$.
			
			Mặt khác $(MN,OM\subset (OMN)$.
			
			Vậy $(OMN)\parallel (SBC)$.
		\end{enumerate}
	}
\end{vd}

\begin{vd}
	Cho hình chóp $S.ABCD$ với đáy $ABCD$ là hình thang mà $AD\parallel BC$ và $AD=2BC$. Gọi $M$, $N$ lần lượt là trung điểm của $SA$ và $AD$. Chứng minh: $(BMN)\parallel (SCD)$.
	\loigiai{
		\immini
		{
			Vì $N$ là trung điểm của $AD$ nên $NA=ND=\dfrac{AD}{2}=BC$.\\
			Tứ giác $NBCD$ có $ND=BC$ và $ND\parallel BC$ nên $NBCD$ là hình bình hành, suy ra $NB\parallel CD\Rightarrow NB\parallel (SCD)$.\\
			Tam giác $SAD$ có $M$, $N$ lần lượt là trung điểm của $AS$ và $AD$ nên $MN$ là đường trung bình của $\triangle ADS$, suy ra $MN\parallel SD\Rightarrow MN\parallel (SCD)$.\\
			Từ $\heva{& MN\parallel (SCD),\text{ }MN\subset (BMN) \\ & BN\parallel (SCD), \text{ }BN\subset (BMN)}\Rightarrow (BMN)\parallel (SCD).$
		}
		{
			\begin{tikzpicture}[line join = round, line cap = round,>=stealth,font=\footnotesize,scale=0.6]
			\tkzDefPoints{0/0/A,8/0/D,1.5/-2/B,5.5/-2/C,6/5/S}
			\tkzDefMidPoint(A,D)\tkzGetPoint{N}
			\tkzDefMidPoint(A,S)\tkzGetPoint{M}
			\tkzDrawSegments[dashed](A,D B,N M,N)
			\tkzDrawSegments(S,A S,B S,C S,D A,B B,C C,D B,M)
			\tkzDrawPoints[fill=black](A,B,C,D,S,M,N)
			\tkzLabelPoints[above](S,M)
			\tkzLabelPoints[below](B,C,N)
			\tkzLabelPoints[left](A)
			\tkzLabelPoints[right](D)
			\end{tikzpicture}
		}
	}
\end{vd}

\begin{vd}
	Cho hai hình bình hành $ABCD$ và $ABEF$ có chung cạnh $AB$ và không đồng phẳng. Gọi $I$, $J$, $K$ lần lượt là trung điểm $AB$, $CD$, $EF$. Chứng minh
	\begin{listEX}[2]
		\item $(ADF)\parallel (BCE)$.
		\item $(DIK)\parallel (JBE)$.
	\end{listEX}
	\loigiai{	\immini{\begin{listEX}
				\item Ta có $AD\parallel BC$, suy ra $AD\parallel (BCE)$. Tương tự $AF\parallel (BCE)$.\\
				Khi đó $(ADF)\parallel (BCE)$.
				\item Trong hình bình hành $ABCD$ có $I$, $J$ lần lượt là trung điểm của $AB$ và $CD$ nên $BI=DJ$. Do đó $IBJD$ là hình bình hành. Suy ra $DI\parallel BJ$ nên $DI\parallel (JBE)$. \\
				Trong hình bình hành $ABEF$ có $I$, $K$ lần lượt là trung điểm của $AB$ và $EF$ nên $IK\parallel EF$, suy ra $IK\parallel (JBE)$. \\
				Vậy $(DIK)\parallel (JBE)$.
		\end{listEX}}{
			\begin{tikzpicture}[scale=1, font=\footnotesize, line join=round, line cap=round, >=stealth]
			\clip(-4.6,-1.8) rectangle (5.4,4.5);
			\tkzDefPoints{2/-1/C,-1/4/F,4/1/B,-2/1/A,-4/-1/D,5/4/E}
			\tkzDefMidPoint(A,C)
			\tkzGetPoint{O}
			\tkzDefMidPoint(B,A)
			\tkzGetPoint{I}
			\tkzDefMidPoint(C,D)
			\tkzGetPoint{J}
			\tkzDefMidPoint(E,F)
			\tkzGetPoint{K}
			%\tkzDefMidPoint(O,M) \tkzGetPoint{I}
			\tkzDrawSegments(D,C C,B B,E E,F D,F D,K J,E)
			\tkzDrawSegments[dashed](A,F A,B A,D D,I I,K B,J)
			\tkzDrawPoints(A,B,C,D,E,F,I,J,K)
			%	\tkzLabelPoint[above right](M){$M$}
			\tkzLabelPoint[above](K){$K$}
			\tkzLabelPoint[below](J){$J$}
			\tkzLabelPoint[above left](I){$I$}
			\tkzLabelPoint[above](F){$F$}
			\tkzLabelPoint[left](A){$A$}
			\tkzLabelPoint[right](B){$B$}
			\tkzLabelPoint[above](E){$E$}
			%\tkzLabelPoint[below](O){$O$}
			\tkzLabelPoint[right](C){$C$}
			\tkzLabelPoint[left](D){$D$}
			\end{tikzpicture}}	
	}
\end{vd}

\begin{vd}
	Cho hình lăng trụ $ABC.A'B'C'$. Gọi $I$, $J$, $K$ lần lượt là trọng tâm các tam giác $ABC$, $ACC'$, $A'B'C'$. Chứng minh rằng $(IJK)\parallel (BCC'B')$ và $(A'JK)\parallel (AIB')$.
	\loigiai{
		\begin{center}
			\begin{tikzpicture}[>=stealth=0.3, line join=round, line cap = round,scale=0.5]
			\tkzDefPoints{0/0/A, 2/-3/B, 8/0/C,1/7/A'}
			\tkzDefPointBy[translation= from A to A'](B)\tkzGetPoint{B'}
			\tkzDefPointBy[translation= from A to A'](C)\tkzGetPoint{C'}
			\tkzDefMidPoint(B,C)\tkzGetPoint{M}
			\tkzDefMidPoint(C,C')\tkzGetPoint{N}
			\tkzDefMidPoint(B',C')\tkzGetPoint{P}
			\tkzDefPointBy[homothety=center A ratio 2/3](M)\tkzGetPoint{I}
			\tkzDefPointBy[homothety=center A ratio 2/3](N)\tkzGetPoint{J}
			\tkzDefPointBy[homothety=center A' ratio 2/3](P)\tkzGetPoint{K}
			\tkzInterLL(A',J)(A,C)\tkzGetPoint{X}
			\tkzDrawSegments[dashed](A,C A,M A,C' A,N I,J I,K J,K A',J I,B' A',X)
			\tkzDrawSegments(A,B B,C A,A' B,B' C,C' A',B' B',C' A',C' A',P M,P A,B' M,N C,P B',M)
			%Gán nhãn
			\tkzDrawPoints[fill=black](A,B,C,A',B', C',I,M,J,N,P,K,X)
			\tkzLabelPoints[left](A)
			\tkzLabelPoints[right=3pt](C,M,N)
			\tkzLabelPoints[below](B,I,J)
			\tkzLabelPoints[above](A',C',P,K)
			\tkzLabelPoints[left=4pt](B')
			\end{tikzpicture}
		\end{center}
		\begin{enumerate}
			\item  Gọi $M$, $N$, $P$ lần lượt là trung điểm của $BC$,  $CC'$ và $B'C'$. Theo tính chất của trọng tâm tam giác ta có
			$$\dfrac{AI}{AM}=\dfrac{AJ}{AN}\Rightarrow IJ\parallel MN.$$
			Tứ giác $AMPA'$ là hình bình hành và có $\dfrac{AI}{AM}=\dfrac{AK}{AP} =\dfrac{2}{3}\Rightarrow IK \parallel MP$.\\
			Vậy $(IJK)\parallel (BCC'B')$.
			\item Chú ý rằng mặt phẳng $(AIB')$ chính là mặt phẳng $(AMB')$. Mặt phẳng $(A'JK)$ chính là mặt phẳng $(A'CP)$.\\
			Vì $AM \parallel A'P$, $MB'\parallel CP$ (do tứ giác $B'MCP$ là hình bình hành). Vậy ta có $(A'JK)\parallel (AIB')$.
		\end{enumerate}
	}
\end{vd}

\begin{vd}
	Cho hai hình vuông $ABCD$ và $ABEF$ ở trong hai mặt phẳng phân biệt. Trên các đường chéo $AC$ và $BF$ lần lượt lấy các điểm $M$, $N$ sao cho $AM=BN$. Các đường thẳng song song với $AB$ vẽ từ $M$, $N$ lần lượt cắt $AD$ và $AF$ tại $M'$ và $N'$. 
	\begin{tasks}(2)
		\task Chứng minh rằng $(ADF)\parallel (BCE)$.
		\task Chứng minh rằng $(CDF)\parallel (MM'N'N)$. 
	\end{tasks}
	\loigiai{
		\begin{center}
			\begin{tikzpicture}[>=stealth=0.3, line join=round, line cap = round,scale=0.6]
			\tkzDefPoints{0/0/A, -2/-2/D,6/0/B, 4/-2/C, 2.5/3/E,-3.5/3/F}
			\tkzDefPointBy[homothety=center A ratio 1/3](C)\tkzGetPoint{M}
			\tkzDefPointBy[homothety=center B ratio 1/3](F)\tkzGetPoint{N}
			\tkzDefLine[parallel=through M](A,B)\tkzGetPoint{c}
			\tkzInterLL(M,c)(A,D)\tkzGetPoint{M'}
			\tkzDefLine[parallel=through N](A,B)\tkzGetPoint{d}
			\tkzInterLL(N,d)(A,F)\tkzGetPoint{N'}
			% Phần này trở xuống để gán nhãn
			\tkzDrawPoints[fill=black](A,B,C,D,E,F,M',M,N,N')
			\tkzDrawSegments(D,C B,E F,D E,C E,F C,B)
			\tkzDrawSegments[dashed](A,B A,D A,F A,C B,F M,M' N,N'  M',N' M,N)
			\tkzLabelPoints[left](A,M',N')
			\tkzLabelPoints[below left](M)
			\tkzLabelPoints[below](B,C,D)
			\tkzLabelPoints[above](E,F,N)
			\end{tikzpicture}
		\end{center}
		\begin{enumerate}[a)]
			\item Ta có 
			$\heva{&AD \parallel BC\\&AF \parallel BE\\&AD \cap AF =A}\Rightarrow (ADE)\parallel (BCF)$. 
			\item Ta có 
			\[ MM'\parallel CD \Rightarrow \dfrac{AM}{AC}=\dfrac{AM'}{AD} \tag \label{1} \]
			Ta cũng có 
			\[ NN'\parallel AB\Rightarrow \dfrac{BN}{BF}=\dfrac{AN'}{AF} \tag \label{2} \]
			Mà từ giả thiết ta có 
			\[ \dfrac{AM}{AC}= \dfrac{BN}{BF}\Rightarrow \dfrac{AM'}{AD}= \dfrac{AN'}{AF} \tag \label{3} \]
			Từ $(3)$ suy ra $M'N'\parallel DF$. Ta cũng có $MM'\parallel NN'\parallel DC \parallel FE$. \\
			Vậy  $(CDF)\parallel (MM'N'N)$. 
		\end{enumerate}
	}
\end{vd}

\begin{dang}{Chứng minh đường thẳng song song với mặt phẳng}
	Để chứng minh $a$ song song $(P)$, ta thường sử dụng một trong hai cách sau
	\begin{enumerate}[\iconMT]
		\item \indam{Cách 1: } (\textit{Đã xét ở bài học trước}) Ta cần chứng tỏ các ý sau:
		\begin{itemize}
			\item [$\bullet$] $a$ không nằm trên $(P)$;
			\item [$\bullet$] $a$ song song với một đường thẳng $b$ nằm trong $(P)$. Suy ra $a\parallel (P)$ hay $$\heva{&a\not\subset (P)\\&a\parallel b \\&b\subset (P)\\}\Rightarrow a\parallel (P)$$
		\end{itemize}
		\item \indam{Cách 2:} Ta chứng minh đường thẳng $a$ nằm trong mặt
		phẳng $(Q)$ và $(Q)\parallel (P)$ thì $a \parallel (P)$.
	\end{enumerate}
\end{dang}

\begin{vd}%[1H2B4-2]
	Cho hình chóp $S.ABCD$ có đáy $ABCD$ là hình bình hành. Gọi $G_1$, $G_2$, $G_3$ lần lượt là trọng tâm các tam giác $SAB$, $ABC$, $SBD$. Gọi $M$ là một điểm thuộc đường thẳng $G_2 G_3$. Chứng minh $G_1M \parallel (SBC)$.
	\loigiai{
		\begin{center}
			\begin{tikzpicture}[scale=0.6]
			\tkzDefPoints{3/3/A, 0/0/B, 8/0/C, 11/3/D, 3/10/S}
			\tkzDefMidPoint(B,D)\tkzGetPoint{O}
			\tkzDefMidPoint(B,A)\tkzGetPoint{N}
			\tkzDefMidPoint(S,A)\tkzGetPoint{I}
			\path[name path=sn] (S)--(N);
			\path[name path=bi] (B)--(I);
			\path[name intersections={of=sn and bi,by=G_1}];
			\path[name path=bd] (B)--(D);
			\path[name path=cn] (C)--(N);
			\path[name intersections={of=bd and cn,by=G_2}];
			\path[name path=so] (S)--(O);
			\path[name path=ci] (C)--(I);
			\path[name intersections={of=so and ci,by=G_3}];
			\coordinate (M) at ($(G_2)!0.2!(G_3)$);
			\draw (S)--(B)--(C)--(D)--(S) (S)--(C);
			\draw[dashed] (S)--(A)--(B) (A)--(D) (B)--(D) (A)--(C) (S)--(O) (S)--(N) (C)--(N);
			\draw[dashed] (G_1)--(G_2)--(G_3)--(G_1);
			\fill (A) circle (2pt) node[above right]{$A$};
			\fill (B) circle (2pt) node[below]{$B$};
			\fill (C) circle (2pt) node[below]{$C$};
			\fill (D) circle (2pt) node[right]{$D$};
			\fill (S) circle (2pt) node[above]{$S$};
			\fill (O) circle (2pt) node[above right]{$O$};
			\fill (G_1) circle (2pt) node[left]{$G_1$};
			\fill (G_2) circle (2pt) node[below]{$G_2$};
			\fill (G_3) circle (2pt) node[right]{$G_3$};
			\fill (N) circle (2pt) node[left]{$N$};
			\fill (M) circle (2pt) node[right]{$M$};
			\end{tikzpicture}
		\end{center}
		Gọi $O$ là tâm hình bình hành $ABCD$ và $N$ là trung điểm $AB$, suy ra $G_1 \in SN$, $G_2 \in CM$, $G_3 \in SO$.
		
		Do $G_1$, $G_2$ lần lượt là trọng tâm tam giác $SAB$, $ABC$ nên ta có: $$\begin{cases}
		\dfrac{NG_1}{MS}=\dfrac{1}{3} \bigskip \\
		\dfrac{NG_2}{MC}=\dfrac{1}{3}
		\end{cases}$$
		$\Rightarrow$ $G_1G_2 \parallel SC$ (Định lý Ta-lét trong $\Delta NSC$)
		
		$\Rightarrow$ $G_1G_2 \parallel (SBC)$.
		
		Do $G_2$, $G_3$ lần lượt là trọng tâm tam giác $ABC$, $SBD$ nên ta có:
		$$\begin{cases}
		\dfrac{OG_2}{OB}=\dfrac{1}{3} \bigskip \\
		\dfrac{OG_3}{OS}=\dfrac{1}{3}
		\end{cases}$$
		$\Rightarrow$ $G_2G_3 \parallel SB$ (Định lý Ta-lét trong $\Delta SOB$).
		
		$\Rightarrow$ $G_2G_3 \parallel (SBC)$.
		
		Ta đã có: $\begin{cases} G_1G_2 \parallel (SBC) \\ G_2G_3 \parallel (SBC) \end{cases}$ $\Rightarrow$ $(G_1G_2G_3) \parallel (SBC)$ 
		
		Mà $G_1M \subset (G_1G_2G_3)$ $\Rightarrow$ $G_1M \parallel (SBC)$.
	}
\end{vd}

\begin{vd}%[1H2B4-2]
	Cho hình chóp $S.ABCD$ có đáy là hình bình hành tâm $O$. Gọi $M$, $N$ lần lượt là trung điểm của $SA$ và $CD$.
	\begin{enumerate}
		\item Chứng minh hai mặt phẳng $(OMN)$ và $(SBC)$ song song với nhau.
		\item Gọi $I$ là trung điểm của $SD$, $J$ là một điểm trên $(ABCD)$ và cách đều $AB$, $CD$. Chứng minh $IJ$ song song với $(SAB)$.
		\end{enumerate}
	\loigiai{
		\begin{enumerate}
			\begin{minipage}{0.5\textwidth}
				\item Chứng minh $(OMN)\parallel (SBC)$.
				
				Do $ON$, $OM$ theo thứ tự là đường trung bình của các tam giác $BCD$ và $SAC$ nên $OM \parallel BC$, $ON \parallel SC$.
				
				Hơn nữa, $ON$, $OM$ không chứa trong $(SBC)$. Do đó $ON \parallel (SBC)$, $OM \parallel (SBC)$.
				
				Mặt khác, $OM \cap ON= O$ nên $(OMN)\parallel (SBC)$. 
				\item Chứng minh $IJ\parallel (SAB)$.
				
				Trong mặt phẳng $(ABCD)$, $O$ và $J$ cách đều hai đường thẳng song song $AB$ và $CD$ nên $OJ\parallel AB \parallel CD$. Hơn nữa, $OJ$ không chứa trong $(SAB)$. Do đó, $OJ\parallel (SAB)$.
			\end{minipage}
			\begin{minipage}{0.5\textwidth}
				\begin{tikzpicture}[scale=0.8]
				\tkzDefPoints{1.7/3/A, 0/0/B, 7/0/C, 1.2/7/S}
				\coordinate (D) at ($(C)-(B)+(A)$);
				\tkzInterLL(A,C)(B,D)\tkzGetPoint{O}
				\tkzDefMidPoint(S,A)\tkzGetPoint{M}
				\tkzDefMidPoint(C,D)\tkzGetPoint{N}
				\tkzDefMidPoint(S,D)\tkzGetPoint{I}
				%\coordinate (E) at ($1/3*(B)+2/3*(S)$);
				%\coordinate (F) at ($1/3*(C)+2/3*(D)$);
				\coordinate (J) at ($1.5*(O)-0.5*(3.5,0)$);
				%\tkzDefPointBy[homothety=center N ratio 0.25](M)\tkzGetPoint{H}
				\tkzDrawSegments(S,B B,C C,D S,C S,D)
				\tkzDrawSegments[dashed](S,A A,B A,C B,D A,D O,I I,J M,N O,M O,N O,J)
				\tkzDrawPoints(A,B,C,D,O,S,M,N,I,J)
				\tkzLabelPoints[above right](M)
				\tkzLabelPoints[below left](B)
				%\tkzLabelPoints[below right](C,P)
				\tkzLabelPoints[right](D,N,J)
				\tkzLabelPoints[below](O,C,A)
				\tkzLabelPoints[above](S,I)
				\end{tikzpicture}
			\end{minipage}
			Mặt khác, $OI$ là đường trung bình trong tam giác $SBD$ nên $OI \parallel SB$. Do đó, $OJ\parallel (SAB)$.
			
			Mặt phẳng $(OIJ)$ chứa hai đường thẳng cắt nhau và cùng song song với $(SAB)$ nên $(OIJ)\parallel (SAB)$. Hơn nữa, $IJ\subset (OIJ)$. Vì vậy, $IJ\parallel (SAB)$.
		\end{enumerate}}
\end{vd}


\begin{dang}{Định lý Thales}
	\textbf{Định lí Thales}: Ba mặt phẳng đôi một song song chắn trên hai cát tuyến bất kì những đoạn thẳng tương ứng tỉ lệ.
\end{dang}
\begin{vd}
	Cho ba mặt phẳng $(P),(Q)  ,(R) $  đôi một song song với nhau. Đường thẳng $a$   cắt các mặt phẳng $(P),(Q)  ,(R) $   lần lượt tại  $A, B, C$ sao cho $\dfrac{AB}{BC}=\dfrac{2}{3}$  và đường thẳng $b$   cắt các mặt phẳng $(P),(Q)  ,(R) $  lần lượt tại $A', B', C'$ . Tính tỉ số $\dfrac{A'B'}{B'C'}$  .
	\loigiai{
	Vì ba mặt phẳng  $(P),(Q)  ,(R) $  đôi một song song với nhau, áp dụng định lý Ta – lét trong không gian, ta có
			$$\dfrac{AB}{BC}=\dfrac{A'B'}{B'C'}=\dfrac{2}{3}.$$
}
\end{vd}
\begin{vd}
	Cho ba mặt phẳng  $(P),(Q)  ,(R) $  đôi một song song với nhau. Đường thẳng $a$  cắt các mặt phẳng $(P),(Q)  ,(R) $   lần lượt tại   $A, B, C$ sao cho $\dfrac{AB}{BC}=\dfrac{1}{3}$    và đường thẳng $b$  cắt các mặt phẳng   $(P),(Q)  ,(R) $   lần lượt tại $D, E, F$ .Tính tỉ số $\dfrac{ED}{DF}$  .
	\loigiai{
		Vì ba mặt phẳng  $(P),(Q)  ,(R) $  đôi một song song với nhau, áp dụng định lý Ta – lét trong không gian, ta có $$\dfrac{AB}{BC}=\dfrac{DE}{EF}=\dfrac{1}{3}$$
		Suy ra 
		$$EF=3DE\Rightarrow DF=EF+DE=4DE\Rightarrow \dfrac{DE}{DF}=\dfrac{1}{4}.$$
}
\end{vd}
\begin{vd}
	Cho hình tứ diện $S.ABC$ . Trên cạnh $SA$  lấy các điểm $A_1, A_2$  sao cho $2AA_1=2A_1A_2=A_2S$  . Gọi $(P)$   và $(Q)$  là hai mặt phẳng song song với mặt phẳng $(ABC)$  và lần lượt đi qua $A_1$, $A_2$ . Mặt phẳng $(P)$  cắt các cạnh $SB$, $SC$ lần lượt tại $B_1$, $C_1$ . Mặt phẳng $(Q)$  cắt các cạnh $SB$, $SC$  lần lượt tại $B_2$, $C_2$ . Chứng minh $2BB_1= 2B_1B_2=B_2S$ và $2CC_1=2C_1C_2=C_2S$ .
	\loigiai{
	\immini{
		Theo giả thiết thì $A_2A_1=A_1A$ và $A_2S=2A_2A_1$.\\
		Vì mặt phẳng $(P)$  qua $A_1$   song song với mặt phẳng $(ABC)$  nên 
		
		$$\heva{&(P) \cap (SAB)=A_1B_1, \text{ với } A_1B_1 \parallel AB\\& (P) \cap (SBC)=B_1C_1, \text{ với } B_1C_1 \parallel BC}$$
		
		Vì mặt phẳng $(Q)$  qua $A_2$  song song với mặt phẳng $(ABC)$  nên		
		$$\heva{&(Q) \cap (SAB)=A_2B_2, \text{ với } A_2B_2 \parallel AB\\& (Q) \cap (SBC)=B_2C_2, \text{ với } B_2C_2 \parallel BC}$$
	}{
\begin{tikzpicture}[scale=0.7, font=\footnotesize,>=stealth]
	\path
	%	Vẽ mp
	(0,0) coordinate (A)
	(1.5,-2) coordinate (B)
	(4,0) coordinate (C)
	(1,4) coordinate (S)
	(2.3,-0.7) coordinate (I)
	($(A)!0.5!(S)$)coordinate (A_2)
	($(A_2)!0.5!(A)$)coordinate (A_1)
	%
	($(B)!0.5!(S)$)coordinate (B_2)
	($(B_2)!0.5!(B)$)coordinate (B_1)
	%
	($(C)!0.5!(S)$)coordinate (C_2)
	($(C_2)!0.5!(C)$)coordinate (C_1)
	;
	\draw (S)--(B)--(C)--(S)--(A)--(B) (A_1)--(B_1)--(C_1) (A_2)--(B_2)--(C_2);
	\draw[dashed] (A)--(C) (A_1)--(C_1) (A_2)--(C_2);
	\foreach \x/\g in {A/-180,B/-90,C/0,S/90,A_1/170,A_2/160,B_1/-30,B_2/60,C_1/0,C_2/30}\draw[fill=black] (\x) circle (.05) +(\g:.5)node{\footnotesize$\x$};
\end{tikzpicture}}
	Các mặt phẳng $(ABC)$, $(A_1B_1C_1)$ và $(A_2B_2C_2)$ đôi một song song nhau nên theo định lí Ta let ta có 
	\begin{itemize}
		\item [$\bullet$] $\dfrac{A_2A_1}{A_1A}=\dfrac{B_2B_1}{B_1B}=\dfrac{C_2C_1}{C_1C}=1 \Rightarrow B_2B1=B_1B \text{ và } C_2C1=C_1C$ \quad (1);
		\item [$\bullet$] $\dfrac{SA_2}{A_2A_1}=\dfrac{SB_2}{B_2B_1}=\dfrac{SC_2}{C_2C_1}=2 \Rightarrow SB_2=2B_2B_1 \text{ và }  SC_2=2C_2C_1$ \quad (2)
	\end{itemize}

	Từ (1) và (2) suy ra
	Nên  $2BB_1= 2B_1B_2=B_2S$ và $2CC_1=2C_1C_2=C_2S$.
}
\end{vd}
\begin{vd}
	Một kệ để đồ bằng gỗ có mâm tầng dưới $(ABCD)$ và mâm tầng trên $(EFGH)$ song song với nhau. Bác thợ mộc đo được $AE=80$ cm, $CG=90$ cm và muốn đóng thêm một mâm tầng giữa $(IJKL)$ song song với hai mâm tầng trên và dưới sao cho khoảng cách $EI=36$ cm (tham khảo hình vẽ). Hãy giúp bác thợ mộc tính độ dài $GK$ để đặt mâm tầng giữa cho kệ để đồ đúng vị trí.
	%Hình vẽ 
	\loigiai{
	Theo định lý Thales ta có $\dfrac{EI}{GK}=\dfrac{AE}{CG}=\dfrac{80}{90}=\dfrac{8}{9}$. Suy ra $GK = 40,5$ cm.
}
\end{vd}
\begin{vd}
	Cho hình chóp $S.ABC$ có $SA=9, SB=12, SC=15$. Trên cạnh $SA$ lấy các điểm $M, N$ sao cho $SM=4, MN=3, NA=2$. Vẽ hai mặt phẳng song song với $(ABC)$ lần lượt đi qua $M, N$ , cắt $SB$ theo thứ tự $M', N'$ và cắt $SC$ theo thứ tự $M'', N''$. Tính độ dài các đoạn thẳng $SM', M'N', M''N'', N''C$.
\end{vd}

\begin{dang}{Hình hộp, hình lăng trụ}
\end{dang}
\begin{vd}
Cho hình hộp $ABCD.A'B'C'D'$ và một mặt phẳng $(\alpha)$ cắt các mặt của hình hộp theo các giao tuyến $MN, NP, PQ, QR, RS, SM$ như hình vẽ. Chứng minh các cặp cạnh đối của lục giác $MNPQRS$ song song nhau.
%Hình vẽ
\end{vd}
\begin{vd}
	Cho hình lăng trụ tứ giác $ABCD.A'B'C'D'$  với đáy là hình thang $AB\parallel CD$   . Một mặt phẳng song song với mặt phẳng $(AA'B'B)$   cắt các cạnh $AD, BC, B'C', A'D'$  lần lượt tại $E, F, M, H$ . Hỏi hình tạo bởi các điểm $E, F, M, H, D, D', C', C$  là hình gì?
	\loigiai{
}
\end{vd}
\begin{vd}
 Cho lăng trụ tam giác $A B C \cdot A' B' C'$. Gọi $M, N, P$ lần lượt là các điểm trên cạnh $A A', B B', C C'$ sao cho: $\dfrac{A M}{M A'}=\dfrac{B N}{N B'}=\dfrac{C P}{P C'}=\dfrac{1}{2}$. Hỏi hình tạo bởi các điểm $M, N, P, A', B', C'$ là hình gì?
 \loigiai{
}
\end{vd}


\subsection{BÀI TẬP TỰ LUYỆN}

\begin{bt}
	Cho hình chóp $S.ABCD$ có đáy $ABCD$ là hình bình hành tâm $O$. Gọi $M$, $N$ lần lượt là trung điểm của $SA$ và $CD$.	Chứng minh hai mặt phẳng $(MNO)$ và $(SBC)$ song song.
	\loigiai{
		\begin{center}
			\begin{tikzpicture}
				\tkzInit[xmin=-0.5,ymin=-0.5,xmax=7.5,ymax=6.5]
				\tkzDefPoints{0/0/B}
				\tkzDefShiftPoint[B](0:5){C}
				\tkzDefShiftPoint[B](50:2.5){A}
				\tkzDefShiftPoint[C](50:2.5){D}   
				\tkzDefShiftPoint[A](80:4){S} 
				\tkzInterLL(A,C)(B,D) \tkzGetPoint{O}
				\tkzDefMidPoint(S,A)\tkzGetPoint{M}
				\tkzDefMidPoint(C,D)\tkzGetPoint{N}
				\tkzDrawPolygon(S,B,C)
				\tkzDrawSegments[](C,D S,D)
				\tkzDrawSegments[dashed](A,B A,D S,A B,D A,C M,N N,O M,O)
				\tkzDrawPoints[fill=black](A,B,C,D,S,O,M,N)
				\tkzLabelPoints[above](S)
				\tkzLabelPoints[below left](B)
				\tkzLabelPoints[below right](C)
				\tkzLabelPoints[below](O)
				\tkzLabelPoints[right](D,N)
				\tkzLabelPoints[above left](A)
				\tkzLabelPoints[above right](M)
			\end{tikzpicture}
		\end{center}
		Ta có $M$ là trung điểm $SA$, $O$ là trung điểm $AC$\\
		$\Rightarrow MO$ là đường trung bình $\triangle SAC$\\
		$\Rightarrow MO \parallel SC$.\\
		Tương tự $ON \parallel BC$.\\
		Do đó $(OMN) \parallel (SBC)$.	
	}
\end{bt}

\begin{bt}%[Trần Thị Thu Hằng]%[1H2K4]
	Cho hình chóp $S.ABCD$, đáy $ABCD$ là hình thang có $AB \parallel CD$ và $AB=2CD$, $I$ là giao điểm của $AC$ và $BD$. Gọi $M$ là trung điểm của $SD$, $E$ là trung điểm đoạn $CM$ và $G$ là điểm đối xứng của $E$ qua $M$, $SE$ cắt $CD$ tại $K$. Chứng minh $(IKE) \parallel (ADG)$.
	\loigiai{
		\immini{Do $CE=ME=MG$ nên \begin{align}\label{2.1}
				CE=\dfrac{1}{3} CG. \tag{1}
			\end{align}
			Mặt khác $$\begin{cases} \widehat{BAI}&=\widehat{DCI},\text{ (so le trong)}, \\ \widehat{AIB}&=\widehat{CID},\text{ (đối đỉnh)}. \end{cases}$$
			Do đó $\triangle ABI \backsim \triangle CDI$, (g-g).
			Khi đó \begin{align}\label{2.2}
				\dfrac{CI}{IA}=\dfrac{CD}{AB}=\dfrac{1}{2}\hspace{5pt}\text{hay}\hspace{5pt} \dfrac{CI}{CA}=\dfrac{1}{3}.\tag{2}
			\end{align}
			Từ \eqref{2.1} và \eqref{2.2} suy ra 
			\begin{align}\label{*}
				EI \parallel GA.\tag{$\star$}
		\end{align}}{
			\begin{tikzpicture}[smooth]
				\tkzDefPoints{0/0/C, 3/0/D, -1/2.5/B, 5/2.5/A, 1.5/7/S}
				\tkzInterLL(A,C)(B,D)\tkzGetPoint{I}
				\tkzDefMidPoint(S,D)\tkzGetPoint{M}
				\tkzDefMidPoint(C,M)\tkzGetPoint{E}
				\tkzDefPointBy[symmetry=center M](E)\tkzGetPoint{G}
				\tkzInterLL(S,E)(C,D)\tkzGetPoint{K}
				\tkzInterLL(S,A)(G,C)\tkzGetPoint{N}
				\tkzDrawPoints(A,B,C,S,D,I,K,E,M,G)
				\tkzDrawSegments(S,B S,C S,D S,N B,C C,D A,D G,C S,K G,D G,A)
				\tkzDrawSegments[dashed](A,B N,A A,C B,D I,E I,K)
				\tkzLabelPoints[below left](C)
				\tkzLabelPoints[below right](D)
				\tkzLabelPoints[below](K)
				\tkzLabelPoints[left](B,M,E)
				\tkzLabelPoints[right](A)
				\tkzLabelPoints[above](S,G,I)
			\end{tikzpicture}
	}
Hơn nữa, tứ giác $SGDE$ có $SM=MD$ và $EM=MG$, nên tứ giác $SGDE$ là hình bình hành. Do đó 
\begin{align}\label{**}
	SE \parallel GD\hspace{5pt}\text{hay}\hspace{5pt} EK \parallel GD.\tag{$\star\star$}
\end{align}
Từ \eqref{*} và \eqref{**} suy ra $(IEK) \parallel (ADG)$.}
	
\end{bt}


\begin{bt}%[1H2K4-2]
	Cho tứ diện $ ABCD $. Gọi $ G_1 $, $ G_2 $, $ G_3 $ lần lượt là trọng tâm của các tam giác $ ABC $, $ ACD $, $ ADB $.	 Chứng minh $ (G_1G_2G_3)\parallel (BCD) $.
	
	\loigiai{
		\begin{center}
			\begin{tikzpicture}[scale=0.75, font=\footnotesize, line join=round, line cap=round, >=stealth]
				\tkzDefPoints{4/9/A, 0/3/B, 5/1/C, 7/3/D}
				\tkzDefMidPoint(B,C)\tkzGetPoint{M}
				\tkzDefMidPoint(D,C)\tkzGetPoint{N}
				\tkzDefMidPoint(D,B)\tkzGetPoint{L}
				\tkzCentroid(A,B,C)\tkzGetPoint{G1}
				\tkzCentroid(A,D,C)\tkzGetPoint{G2}
				\tkzCentroid(A,B,D)\tkzGetPoint{G3}
				\tkzDefLine[parallel=through G1](C,B) \tkzGetPoint{X}
				\tkzInterLL(G1,X)(A,B)\tkzGetPoint{E}
				\tkzInterLL(G1,X)(C,A)\tkzGetPoint{F}
				\tkzInterLL(G2,F)(D,A)\tkzGetPoint{G}
				\tkzDrawSegments(A,B B,C C,D D,A A,C E,F F,G A,M A,N)
				\tkzDrawSegments[dashed](B,D L,M M,N N,L A,L E,G)
				\tkzLabelPoints[right](G,D,N,F)
				\tkzLabelPoints[left](E,B)
				\tkzLabelPoints[above](A)
				\tkzLabelPoints[below](C,L,M)
				\node[below left] at (G1) {$ G_1 $};
				\node[above left] at (G2) {$ G_2 $};
				\node[above right] at (G3) {$ G_3 $};
				\tkzDrawPoints[fill=black](A,B,C,D,M,N,L,E,F,G,G1,G2,G3)
			\end{tikzpicture}
		\end{center}
	 Chứng minh $ (G_1G_2G_3)\parallel (BCD) $\\
			Gọi $ M $, $ N $, $ L $ lần lượt là trung điểm các cạnh $ BC $, $ CD $ và $ BD $. Trong tam giác $ AMN $, ta có
			$$ \dfrac{AG_1}{AM}=\dfrac{AG_2}{AN}=\dfrac{G_1G_2}{MN}=\dfrac{2}{3} (\text{tính chất trọng tâm}) $$
			Theo định lý Ta-lét đảo, suy ra $ G_1G_2\parallel MN $.\\
			Chứng minh tương tự, ta cũng có $ G_2G_3\parallel NL $ và $ G_3G_1\parallel LM $.\\
			Từ đó suy ra $$ \heva{&G_1G_2\parallel MN, G_2G_3\parallel NL\\&MN, NL\subset(BCD)\\&G_1G_2,G_2G_3\subset(G_1G_2G_3).} $$
			$ \Rightarrow (G_1G_2G_3)\parallel (BCD) $.\\
			}
\end{bt}

\begin{bt}%[1H2K4-2]
	Cho hình chóp $SABC$ có $G$ là trọng tâm tam giác $ABC$. Trên đoạn $SA$ lấy hai điểm $M$, $N$ sao cho $SM=MN=NA$.
	\begin{listEX}[1]
		\item Chứng minh rằng $GM\parallel (SBC)$. 
		\item Gọi $D$ là điểm đối xứng với $A$ qua $G$. Chứng minh rằng $(MCD)\parallel (NBG)$.
		\end{listEX} 
	\loigiai{
		\begin{center}
			\begin{tikzpicture}[>=stealth=0.5, line join=round, line cap = round,scale=0.7]
				\tkzDefPoints{0/0/A,1.5/-2/B,8/0/C,1.5/6/S}
				\tkzDefMidPoint(B,C)\tkzGetPoint{E}
				\tkzDefPointBy[homothety=center A ratio 2/3](E)\tkzGetPoint{G}
				\tkzDefPointBy[homothety=center A ratio 2/3](S)\tkzGetPoint{M}
				\tkzDefPointBy[homothety=center A ratio 1/3](S)\tkzGetPoint{N}
				\tkzDefPointBy[homothety=center A ratio 2](G)\tkzGetPoint{D}
				\tkzInterLL(D,M)(S,E)\tkzGetPoint{H}
				\tkzInterLL(D,H)(C,E)\tkzGetPoint{X}
				\tkzDrawSegments(A,B B,E S,A S,B S,C S,E D,E C,D B,N D,H)
				\tkzDrawSegments[dashed](A,C A,E G,M M,C B,G G,N H,M E,X E,C)
				%Gán nhãn
				\tkzDrawPoints[fill=black](A,B,C,S,E,G,M,N,H,D)
				\tkzLabelPoints[left](A,M,N)
				\tkzLabelPoints[below](B,G,E)
				\tkzLabelPoints[right](C,D,H)
				\tkzLabelPoints[above](S)
			\end{tikzpicture}
		\end{center}
		\begin{enumerate}
			\item Gọi $E$ là trung điểm của $BC$. Khi đó ta có $\dfrac{AG}{AE}=\dfrac{AM}{AS}=\dfrac{2}{3}\Rightarrow GM \parallel SE$. Vậy $GM \parallel (SBC)$.
			\item Từ giả thiết ta suy ra $G,N$ lần lượt là trung điểm của $AD$ và $AM$. Do đó $NG \parallel MD \quad (1)$ \hfill $(1)$\\
			Từ giác $BDCG$ có $E$ là trung điểm của hai đường chéo nên đó là hình bình hành. Suy ra $BG\parallel CD$ \hfill $(2)$\\
			Từ $(1)$ và $(2)$ suy ra $(MCD)\parallel (NBG)$.
		\end{enumerate}
	}
\end{bt}
\begin{bt}
	Cho hình hộp $ABCD.A'B'C'D'$  . Một mặt phẳng song song với mặt đáy $(ABCD)$  của hình hộp và cắt các cạnh $AA', BB', CC', DD'$  lần lượt tại $M, N, M', N'$ . Chứng minh rằng $ABCD.MNM'N'$  là hình hộp.
	\loigiai{
	}
\end{bt}
% \subsection{BÀI TẬP TRẮC NGHIỆM}
\Opensolutionfile{ans}[ans/1H4.B4]
\setcounter{ex}{0}

\begin{ex}%[1H2B4]
	Cho đường thẳng $d$ song song với mặt phẳng $(\alpha  )$. Có bao nhiêu mặt phẳng đi qua $d$ và song song với $(\alpha  )$?
	\choice
	{\True $1$}
	{$0$}
	{$2$}
	{Vô số}
	\loigiai{}
\end{ex}

\begin{ex}%[1H2B4]
	Trong các điều kiện sau, điều kiện nào kết luận mặt phẳng $(\alpha  )$ song song với mặt phẳng $(\beta )$?
	\choice{$(\alpha )\parallel (\gamma )$ và $(\beta )\parallel (\gamma )$ (với $(\gamma )$ là mặt phẳng nào đó)}
	{$(\alpha )\parallel a$ và $(\alpha )\parallel b$ với $a$, $b$ là hai đường thẳng phân biệt thuộc $(\beta )$}
	{$(\alpha )\parallel a$ và $(\alpha )\parallel b$ với $a$, $b$ là hai đường thẳng phân biệt cùng song song với $(\beta )$}
	{\True $(\alpha )\parallel a$ và $(\alpha )\parallel b$ với $a$, $b$ là hai đường thẳng cắt nhau thuộc $(\beta )$} 
\end{ex}

\begin{ex}%[1H2B4]
	Cho các mệnh đề sau:
	\begin{listEX}[1]
		\item [\ding{172}] Hai mặt phẳng phân biệt cùng song song với một đường thẳng thì chúng song song với nhau.
		\item [\ding{173}] Hai mặt phẳng cùng song song với một mặt phẳng thứ ba thì chúng song song với  nhau.
		\item [\ding{174}] Bất kì đường thẳng nào cắt một trong hai mặt phẳng song song thì nó cũng cắt mặt phẳng còn lại.
	\end{listEX}
	Số mệnh đề \textbf{sai} là
	\choice{$0$}
	{$1$}
	{\True $2$}
	{$3$}
	\loigiai{
		Mệnh đề đúng là $(3)$. Mệnh đề $(2)$ \textbf{sai} vì hai mặt phẳng đó có thể trùng nhau.
	}
\end{ex}

\begin{ex}%[1H2K4-6]
	Trong các mệnh đề sau. Mệnh đề \textbf{sai} là
	\choice
	{Hai mặt phẳng song song với nhau thì mọi đường thẳng nằm trong mặt phẳng này đều song song
		với mặt phẳng kia}
	{\True Hai mặt phẳng cùng song song với một mặt phẳng thì song song với nhau}
	{Một mặt phẳng cắt hai mặt phẳng song song cho trước theo hai giao tuyến thì hai giao tuyến song
		song với nhau}
	{Hai mặt phẳng song song thì không có điểm chung}
	\loigiai{
		Hai mặt phẳng \textit{phân biệt} cùng song song với một mặt phẳng thì song song với nhau.
	}
\end{ex}


\begin{ex}%[1H2B4-1]
	Cho mặt phẳng $(R)$ cắt hai mặt phẳng song song $(P)$ và $(Q)$ theo hai giao tuyến $a$ và $b$. Mệnh đề nào sau đây đúng?
	\choice
	{$a$ và $b$ vuông góc nhau}
	{$a$ và $b$ chéo nhau}
	{$a$ và $b$ cắt nhau}
	{\True $a$ và $b$ song song}
	\loigiai{
		\begin{center}
			\begin{tikzpicture}[>=stealth,scale=.5]
				\clip (-2,-3) rectangle (10,8.5);
				\tkzDefPoints{0/0/B, 6/0/C, 9/2/D, 3/2/A}
				\coordinate (A') at ($(A)+(0,3)$);
				\coordinate (B') at ($(B)+(0,3)$);
				\coordinate (C') at ($(C)+(0,3)$);
				\coordinate (D') at ($(D)+(0,3)$);
				\coordinate (M) at ($(A)!0.55!(D)$);
				\coordinate (N) at ($(B)!0.55!(C)$);
				\coordinate (P) at ($(B')!0.55!(C')$);
				\coordinate (Q) at ($(A')!0.55!(D')$);
				\coordinate (M') at ($(M)-(0,3)$);
				\coordinate (N') at ($(N)-(0,3)$);
				\coordinate (P') at ($(P)+(0,3)$);
				\coordinate (Q') at ($(Q)+(0,3)$);
				\tkzInterLL(P',N')(A',D')
				\tkzGetPoint{K}
				\tkzInterLL(P',N')(A,D)
				\tkzGetPoint{K'}
				\tkzInterLL(Q',M')(A',D')
				\tkzGetPoint{L}
				\tkzInterLL(Q',M')(C',D')
				\tkzGetPoint{L1}
				\tkzInterLL(Q',M')(C,D)
				\tkzGetPoint{L2}
				\tkzInterLL(Q',M')(A,D)
				\tkzGetPoint{L'}
				\tkzDrawSegments(A,B B,C C,D D,L' A,K')
				\tkzDrawSegments[dashed](L',K')
				\tkzDrawSegments(A',B' B',C' C',D' D',L A',K)
				\tkzDrawSegments[dashed](K,L)
				\tkzDrawSegments(M',N' N',P' P',Q' Q',L M',L2 L1,L')
				\tkzDrawSegments[dashed](L,L1 L',L2)
				\tkzDrawSegments(M,N P,Q)
				\tkzLabelSegment[left](M,N){$a$}
				\tkzLabelSegment[left](P,Q){$b$}
				%\tkzLabelSegment[blue](B,N){$P$}
				%\tkzLabelSegment[red](B',P){$Q$}
				%\tkzLabelSegment[black](Q,P'){$R$}
				\tkzLabelAngle[pos=1.25,rotate=30](A',B',C'){\footnotesize $Q$}
				\tkzMarkAngle[size =1.6,opacity=1](C',B',A');
				\tkzLabelAngle[pos=1.25,rotate=30](A,B,C){\footnotesize $P$}
				\tkzMarkAngle[size =1.6,opacity=1](C,B,A);
				\tkzLabelAngle[pos=1.15,rotate=30](M',Q',P'){\footnotesize $R$}
				\tkzMarkAngle[size =1.6,opacity=1](P',Q',M');
			\end{tikzpicture}
		\end{center}
	}
\end{ex}

\begin{ex}%[1H2B4-1]
	Cho đường thẳng $a$ thuộc mặt phẳng $(P)$ và đường thẳng $b$ thuộc mặt phẳng $(Q)$. Mệnh đề nào sau đây đúng?
	\choice
	{\True $(P)\parallel (Q)\Rightarrow a\parallel (Q)$ và $b\parallel (P)$}
	{$a$ và $b$ chéo nhau}
	{$(P)\parallel (Q)\Rightarrow a\parallel b$}
	{$a\parallel b\Rightarrow (P)\parallel (Q)$}
	\loigiai
	{
		$(P)\parallel (Q)$ suy ra $(P)$ và $(Q)$ không có điểm chung. Mặt khác $a\in (P)$ nên $a$ và $(Q)$ cũng không có điểm chung. Suy ra $a\parallel (Q)$. Tương tự ta cũng có $b\parallel (P)$.
	}
\end{ex}

\begin{ex} Hình lăng trụ tam giác có tất cả bao nhiêu cạnh?
	\choice
	{$6$}
	{\True $9$}
	{$12$}
	{$3$}
	\loigiai{
		\begin{itemize}
			\item [$\bullet$] Lăng trụ tam giác, có hai đáy. Mỗi mặt đáy có ba cạnh, suy ra có 6 cạnh
			\item [$\bullet$] Mặt khác, chúng có 3 cạnh bên.
		\end{itemize}
		Vậy, có tất cả là 9 cạnh.
	}
\end{ex}

\begin{ex}%[1H2B4]
	Đặc điểm nào sau đây là đúng với hình lăng trụ?
	\choice{Đáy của hình lăng trụ là hình bình hành}
	{Hình lăng trụ có tất cả các mặt song song với nhau}
	{\True Hình lăng trụ có tất cả các mặt bên là hình bình hành}
	{Hình lăng trụ có tất cả các mặt là hình bình hành}
	\loigiai{
	}
\end{ex}

\begin{ex}%[1H2B4-2]
	Cho hình hộp $ABCD.A'B'C'D'$. Mặt phẳng $(AB'D')$ song song với mặt phẳng nào sau đây?
	\choice
	{$(BCA')$}
	{$(BDA')$}
	{\True $(BDC')$}
	{$(A'C'C)$}
	\loigiai{
		\immini{
			Mặt phẳng $(AB'D')$ song song với mặt phẳng $(BDC')$.\\
			Thật vậy, ta có $AB'\parallel DC'$ và $AD'\parallel BC'$, có điều cần chứng minh.
		}{
			\begin{tikzpicture}
				\tkzInit[xmin=-1.5,ymin=-2,xmax=5,ymax=4]
				\tkzClip
				\tkzDefPoints{0/0/B,1/-1/A,3/0/C,1/2/A'}
				\tkzDefPointBy[ translation = from B to C](A)\tkzGetPoint{D}
				\tkzDefPointBy[ translation = from A to A'](B)\tkzGetPoint{B'}
				\tkzDefPointBy[ translation = from A to A'](C)\tkzGetPoint{C'}
				\tkzDefPointBy[ translation = from A to A'](D)\tkzGetPoint{D'}
				\tkzDrawSegments(B,A A,D A',B' B',C' C',D' D',A' A,A' B,B' D,D' A,B' A,D' B',D')
				\tkzDrawSegments[dashed](C,C' C,D C,B C',B B,D D,C')
				\tkzLabelPoints[right](D,C,D',C')
				\tkzLabelPoints[left](B,B',A)
				\tkzLabelPoints[above](A')
				\tkzDrawPoints(A,B,C,D,A',B',C',D')
			\end{tikzpicture}
		}
	}
\end{ex}

\begin{ex}%[1H2B4]
	Cho hai hình bình hành $ABCD$ và $ABEF$ không thuộc cùng một mặt phẳng, có cạnh chung $AB$. Kết quả nào sau đây đúng?
	\choice{ $BC \parallel (AEF)$}
	{$FD \parallel (BEF)$}
	{$(CEF)\parallel (ABD)$}
	{\True $(AFD)\parallel (BCE)$}
\end{ex}

\begin{ex}%[1H2B4] 
	Cho hình chóp $S.ABCD$ có đáy là hình thang $(AB \parallel CD)$ và $AB=2CD$. Gọi $I, J$ lần lượt là trung điểm $SB$ và $AB$. Mặt phẳng nào song song với mặt phẳng $(SAD)$?
	\choice{$(SJC)$}
	{$(ICB)$}
	{$(IJB)$}
	{\True $(IJC)$}
\end{ex}

\begin{ex}%[1H2B4] 
	Trong mặt phẳng $(P)$ cho hình bình hành $ABCD$, qua $A$, $B$, $C$, $D$  lần lượt vẽ bốn đường thẳng $a$, $b$, $c$, $d$  đôi một song song với nhau và không nằm trên $(P)$. Mặt phẳng song song với mặt phẳng $(b,c)$ là
	\choice{$(a,b)$}
	{$(a,c)$}
	{\True $(a,d)$}
	{$(d,b)$}
\end{ex}

\begin{ex}%[1H2Y4-2]
\immini[thm]{Cho hình hộp $ABCD.A'B'C'D'$. Mệnh đề nào sau đây là \textbf{sai}?
	\choice
	{$(ABCD)\parallel (A'B'C'D')$}
	{$(ABB'A')\parallel (CDD'C')$}
	{$(AA'D'D)\parallel (BCC'B')$}
	{\True $(BDD'B')\parallel (ACC'A')$}}{
	\begin{tikzpicture}[line width=0.8pt, scale=0.5]
		\path
		coordinate (B) at (0,0)
		coordinate (C) at (7,0)
		coordinate (D) at (8.5,2.5)
		coordinate (A) at (1.5,2.5)
		coordinate (B') at (0,5)
		coordinate (C') at (7,5)
		coordinate (D') at (8.5,7.5)
		coordinate (A') at (1.5,7.5)
		;
		\draw [dashed] (D)--(A)--(A') (A)--(B);
		\draw (A')--(B')--(C')--(D')--(A') (C)--(C') (D)--(D') (C)--(D) (C)--(B)--(B');
		\draw (A') node[above] {$A'$};
		\draw (D') node[above] {$D'$};
		\draw (C') node[right] {$C'$};
		\draw (B') node[left]{$B'$};
		\draw (A) node[left] {$A$};
		\draw (D) node[right] {$D$};
		\draw (C) node[right] {$C$};
		\draw (B) node[left] {$B$};
\end{tikzpicture}}
	\loigiai{
		\immini{Ta thấy $\heva{&(ABCD)\parallel (A'B'C'D')\\&(AA'D'D)\parallel (BCC'B')\\&(ABB'A')\parallel (CDD'C')}$ luôn đúng.\\
			và hai mặt phẳng $(BDD'B')$, $(ACC'A')$ là cắt nhau. }{
			\begin{tikzpicture}[line width=0.8pt, scale=0.5]
				\path
				coordinate (B) at (0,0)
				coordinate (C) at (7,0)
				coordinate (D) at (8.5,2.5)
				coordinate (A) at (1.5,2.5)
				coordinate (B') at (0,5)
				coordinate (C') at (7,5)
				coordinate (D') at (8.5,7.5)
				coordinate (A') at (1.5,7.5)
				;
				\draw [dashed] (D)--(A)--(A') (A)--(B);
				\draw (A')--(B')--(C')--(D')--(A') (C)--(C') (D)--(D') (C)--(D) (C)--(B)--(B');
				\draw (A') node[above] {$A'$};
				\draw (D') node[above] {$D'$};
				\draw (C') node[right] {$C'$};
				\draw (B') node[left]{$B'$};
				\draw (A) node[left] {$A$};
				\draw (D) node[right] {$D$};
				\draw (C) node[right] {$C$};
				\draw (B) node[left] {$B$};
		\end{tikzpicture}}
	}
\end{ex}

\begin{ex}%[1H2B4-2]
	Cho hình chóp $S.ABCD$ có đáy là một hình bình hành. Gọi $A'$, $B'$, $C'$, $D'$ lần lượt là trung điểm của các cạnh $SA,$ $SB,$ $SC,$ $SD.$ Tìm mệnh đề đúng trong các mệnh đề sau.
	\choice
	{$A'C' \parallel BD$}
	{$A'B' \parallel (SAD)$}
	{\True $(A'C'D') \parallel (ABC)$}
	{$A'B' \parallel (SBD)$}
	\loigiai{
		\immini{
			Ta có $A'C'\parallel AC \Rightarrow (A'C'D') \parallel (ABC).$
		}{\begin{tikzpicture}[scale=0.6,every node/.style={scale=0.6}]%hình chóp S.ABCD
				\tkzDefPoints{0/0/A, -2/-2/B, 3/-2/C, -1/4/S}
				\coordinate (D) at ($(A)+(C)-(B)$);
				\tkzDrawSegments[dashed](S,A A,B A,D)
				\tkzDrawPolygon(S,C,D)
				\tkzDefMidPoint(S,A) \tkzGetPoint{A'}
				\tkzDefMidPoint(S,B) \tkzGetPoint{B'}
				\tkzDefMidPoint(S,C) \tkzGetPoint{C'}\\
				\tkzDefMidPoint(S,D) \tkzGetPoint{D'}
				\tkzDrawSegments(S,B B,C)
				\tkzDrawPoints[fill=black](A,B,A',B',C,D,C',D',S)
				\tkzLabelPoints[left](A,B,A',B')
				\tkzLabelPoints[right](C,D,C',D')
				\tkzLabelPoints[above](S)
		\end{tikzpicture}}
	}
\end{ex}

\begin{ex}%[1H2B4-2]
	Cho hình chóp $S.ABCD$, có đáy $ABCD$ là hình bình hành tâm $O$. Gọi $M,N$ lần lượt là trung điểm $SA,SD$. Mặt phẳng $\left(OMN\right)$ song song với mặt phẳng nào sau đây?
	\choice
	{$\left(ABCD\right)$}
	{$\left(SCD\right)$}
	{\True $\left(SBC\right)$}
	{$\left(SAB\right)$}
	\loigiai{
		\immini{
			Vì $ABCD$ là hình bình hành nên $O$ là trung điểm $AC,BD$.\\
			Do đó $MO \parallel SC\Rightarrow MO\parallel \left(SBC\right)$\\
			Và $NO \parallel SB \Rightarrow NO\parallel \left(SBC\right)$\\
			Suy ra $\left(OMN\right)\parallel \left(SBC\right)$.}{\begin{tikzpicture}[line cap=round,line join=round, >=stealth,scale=.7]
				\def \xa{-2}
				\def \xb{-1}
				\def \y{4}
				\def \z{4}
				\coordinate (A) at (0,0);
				\coordinate (B) at ($(A)+(\xa,\xb)$);
				\coordinate (D) at ($(A)+(\y,0)$);
				\coordinate (C) at ($ (B)+(D)-(A) $);
				\coordinate (K) at ($ (A)!1/2!(B) $);
				\coordinate (S) at ($ (K)+(0,\z) $);
				\coordinate (M) at ($ (S)!1/2!(A) $);
				\coordinate (N) at ($ (S)!1/2!(D) $);
				\coordinate (O) at ($ (B)!1/2!(D) $);
				\draw [dashed] (B)--(A)--(D)--(B) (C)--(A)--(S) (O)--(M)--(N)--(O);
				\draw (S)--(B)--(C)--(D)--(S)--(C);
				\tkzDrawPoints(S,A,B,C,D,M,N)
				\tkzLabelPoint[right](D){$ D $}
				\tkzLabelPoints[below right](C)
				\tkzLabelPoints[above](S)
				\tkzLabelPoints[above left](M)
				\tkzLabelPoints[below](O)
				\tkzLabelPoints[above right](N)
				\tkzLabelPoints[above left](A)
				\tkzLabelPoint[below left](B){$ B $}
			\end{tikzpicture}
	}}
\end{ex}

% \centerline{---HẾT---}
\Closesolutionfile{ans}


%%Bài 14. Phép chiếu song song
% \setcounter{section}{13}
\setcounter{dang}{0}
\section{PHÉP CHIẾU PHẲNG SONG SONG}
\subsection{KIẾN THỨC CẦN NHỚ}
\subsubsection{ĐỊNH NGHĨA}
Cho mặt phẳng $(\alpha)$ và đường thẳng $\Delta$ cắt $(\alpha)$. Với mỗi điểm $M$ trong không gian ta xác định điểm $M'$ như sau:
\immini{\begin{itemize}
		\item Nếu $M$ thuộc $\Delta$ thì $M'$ là giao điểm của $\Delta$ và $(\alpha)$.
		\item Nếu $M$ không thuộc $\Delta$ thì $M'$ là giao điểm của $(\alpha)$ và đường thẳng qua $M$ song song $\Delta$.
		\item Điểm $M'$ gọi là hình chiếu song song của $M$ trên $(\alpha)$ theo phương $\Delta$.
		\item Phép đặt tương ứng mối điểm $M$ với hình chiếu $M'$ của nó được gọi là \indamm{phép chiếu song song} lên $(\alpha)$ theo phương $\Delta$.
		\item Mặt phẳng $(\alpha)$ gọi là mặt phẳng chiếu; phương $\Delta$ gọi là \indamm{phương chiếu.}
	\end{itemize}}{
\begin{tikzpicture}[scale=0.9]
	\tkzInit[xmin=-3.5,xmax=5,ymin=-3, ymax=4] \tkzClip[space=.1]
	\tkzDefPoints{-3/0/A, 3/0/B, 4/2/C, -2/2/D, 0/1/Q, -1/2/P, 0/4/M, 3/1/x}
	%\tkzDrawCircle(O,A)
	\tkzDefLine[parallel=through M,K=1](P,Q)\tkzGetPoint{N}
	\tkzInterLL(M,N)(Q,x) \tkzGetPoint{M'}
	\tkzInterLL(A,D)(P,Q) \tkzGetPoint{y}
	\tkzInterLL(A,B)(P,Q) \tkzGetPoint{y'}
	\tkzInterLL(M,M')(B,C) \tkzGetPoint{t}
	\tkzDefLine[parallel=through y',K=1](P,Q)\tkzGetPoint{z}
	\tkzDefLine[parallel=through t,K=1](P,Q)\tkzGetPoint{t'}
	%\tkzInterLC(C,M)(O,A) \tkzGetPoints{E}{C}
	%\tkzInterLC(A,B)(A,C) \tkzGetPoints{y}{P}
	%\tkzInterLC(C,P)(O,A) \tkzGetPoints{C}{Q}
	%\tkzTangent[at=Q](O) \tkzGetPoint{x}
	\tkzDrawSegments(A,B B,C C,D D,A P,Q M,N P,y N,M' y',z t,t')
	\tkzDrawSegments[dashed](Q,y' M',t)
	%\tkzDrawLine[add = 1 and 1](Q,x)
	\tkzDrawPoints[fill=white](M,M')
	\tkzLabelSegment[above right](P,y){$\Delta$}
	%\tkzLabelPoints[above](C)
	%\tkzLabelPoint[left](M)
	\tkzLabelPoints[below left](M,M')
	\tkzMarkAngle[size=0.6cm,opacity=.5](B,A,D)
	\tkzLabelAngle[pos =0.4](D,A,B){$\alpha$}
	%\tkzLabelPoints[below left](O,P)
\end{tikzpicture}}
\subsubsection{TÍNH CHẤT}
Phép chiếu song song có các tính chất sau:
\begin{itemize}
	\item[\ding{172}] Biến ba điểm thẳng hàng thành ba điểm thẳng hàng.
	\item[\ding{173}] Biến đường thẳng  thành đường thẳng , biến tia thành tia, đoạn thẳng thành đoạn thẳng.
	\item[\ding{174}] Biến hai đường thẳng song song thành hai đường thẳng song song  hoặc trùng nhau.
	\item[\ding{175}] Giữ nguyên tỉ số độ dài của hai đoạn thẳng cùng nằm trên một đường thẳng hoặc nằm trên hai đường thẳng song song.
\end{itemize}

\subsubsection{HÌNH BIỂU DIỄN CỦA MỘT HÌNH KHÔNG GIAN}
\begin{itemize}
	\item[\ding{172}] Hình biểu diễn của hình trong không gian là hình chiếu song song của hình đó trên một mặt phẳng theo một phương chiếu nào đó hoặc hình đồng dạng với hình chiếu đó.
	\item[\ding{173}] Hình biểu diễn của một hình không gian ( trong trường hợp hình phẳng nằm trong mặt phẳng không song song với phương chiếu) có các tính chất sau:
		\begin{itemize}
			\item Hình biểu diễn của một tam giác là một tam giác.
			\item Hình biểu diễn của hình chữ nhật, hình vuông, hình thoi, hình bình hành là hình bình hành.
			\item Hình biểu diễn của hình thang $ABCD$ với $AB\parallel CD$ là một hình thang $A'B'C'D'$ với $A'B'\parallel C'D'$ thoả mãn $\dfrac{AB}{CD}=\dfrac{A'B'}{C'D'}$.
			\item Hình biểu diễn của hình tròn là hình elip.
		\end{itemize}
\end{itemize}

% \subsection{PHÂN LOẠI, PHƯƠNG PHÁP GIẢI TOÁN}
\begin{dang}{Xác định ảnh của một hình qua phép chiếu song song}
\end{dang}
\begin{vd}
	Cho hình hộp $ABCD.A'B'C'D'$.
	\begin{itemize}
		\item [a)] Xác định ảnh của các điểm $A', B', C', D'$ qua phép chiếu song song lên mặt phẳng $(ABCD)$ theo phương $AA'$.
		\item [b)] Xác định ảnh của tam giác $A'C'D'$ qua phép chiếu song song lên mặt phẳng $(ABCD)$ theo phương $A'B$.
	\end{itemize}
	\loigiai{
		\begin{itemize}
			\item 	Vì các cạnh $AA',BB', CC', DD'$ song song nhau nên $A, B, C, D$ là hình chiếu của $A', B', C', D'$.
			\item 
		\end{itemize}
	
	}
\end{vd}
\begin{vd}
	Phép chiếu song song biến hình bình hành $ABCD$ thành hình bình hành $A'B'C'D'$. Chứng minh rằng phép chiếu đó biến tâm của hình bình hành $ABCD$ thành tâm của hình bình hành $A'B'C'D'$.
	\loigiai{
		Gọi $O$ là tâm hình bình hành $ABCD$, suy ra $O$ là trung điểm của $AC$.
		\\Phép chiếu song song biến $O$ thành $O'$.\\
		Ta có $A, O, C $ thẳng hàng theo thứ tự đó và $\dfrac{OA}{OC}=1$ nên ba điểm $A', O', C'$ thẳng hàng theo thứ tự đó và $\dfrac{O'A'}{O'C'}=1$. Suy ra $O'$ là trung điểm của $A'C'$.\\
		Vậy $O'$ là tâm của hình bình hành $A'B'C'D'$.
	}
\end{vd}
\begin{vd}
	Phép chiếu song song biến tam giác $ABC$  thành tam giác $A'B'C'$ . Chứng minh rằng phép chiếu đó biến đường trung bình của tam giác $ABC$  thành đường trung bình của tam giác $A'B'C'$  .
	\loigiai{
	Gọi $M ,N$   lần lượt là trung điểm $AB ,AC$   nên  $MN$ là đường trung bình của tam giác $ABC$.\\
	Phép chiếu song song biến $M$  thành $M'$ , biến $N$  thành $N'$ .\\
	Ta có ba điểm  $A, M , B$  thẳng hàng theo thứ tự đó và $\dfrac{AM}{AB}=\dfrac{1}{2}$  nên ba điểm $A' ,M'  , B'$  thằng hàng theo thứ tự đó và $\dfrac{A'M'}{A'B'}=\dfrac{1}{2}$ . Suy ra $M'$   là trung điểm $A'B'$ .\\
	Tương tự $N'$  là trung điểm $A'C'$ .\\
	Vậy $M'N'$  là đường trung bình của tam giác $A'B'C'$ .
	
}
\end{vd}
\begin{dang}{Vẽ hình biểu diễn của một số hình khối đơn giản}
\end{dang}
\begin{vd}
	Vẽ hình biểu diễn của các hình sau
\begin{itemize}
	\item [a)] Hình lục giác đều.
	\item [b)] Hình vuông nội tiếp trong hình tròn.
\end{itemize}
\end{vd}
\begin{vd}
	Vẽ hình biểu diễn của hình chóp $S.ABCD$ có đáy là hình thang $ABCD$ với $AB$ song song $CD$; $AB=2$ cm, $CD=6$ cm.
	\loigiai{
	\immini{Vì hình chóp $S.ABCD$ có $AB \parallel CD$ và $AB=2$ cm, $CD=6$ cm nên hình biễu diễn của hình chóp cũng có đáy $AB \parallel CD$ và $\dfrac{AB}{CD}=\dfrac{2}{6}=\dfrac{1}{3}$.\\
	Do đó, ta vẽ hình thang $ABCD$ có $AB \parallel CD$ và $AB=\dfrac{1}{3} CD$, vẽ điểm $S$ và nối $SA$, $SB$, $SC$, $SD$.}{
	\begin{tikzpicture}[scale=0.8, font=\footnotesize,>=stealth]
		\path
		%	Vẽ mp
		(0,0) coordinate (A)
		(2,0) coordinate (B)
		(5,1.5) coordinate (C)
		(-1,1.5) coordinate (D)
		(1,3) coordinate (S)
		;
		\draw[line width=0.1pt,gray!20!white] (-1.5,-0.5) grid (5.6,3.6);
		\draw (A)--(B)--(C)--(S)--(D)--(A)--(S) (C)--(S)--(B);
		\draw[dashed] (D)--(C);
		\foreach \x/\g in {A/-90,B/-90,C/0,D/180,S/90}\draw[fill=black] (\x) circle (.05) +(\g:.5)node{\footnotesize$\x$};
	\end{tikzpicture}
}
}

\end{vd}
\begin{vd}
	Vẽ hình biểu diễn của các hình sau
	\begin{itemize}
		\item [a)] Hình lăng trụ có đáy là tam giác đều.
		\item [b)] Hình lăng trụ có đáy là lục giác đều.
		\item [c)] Hình hộp.
		\item [d)] Hình chóp tam giác $S.ABC$ đặt trên một hình lăng trụ tam giác $ABC.A'B'C'$.
	\end{itemize}
\end{vd}

% 
\subsection{BÀI TẬP TRẮC NGHIỆM}

\Opensolutionfile{ans}[ans/1H4.B5]
\setcounter{ex}{0}
\begin{ex}
	Cho hình hộp $ABCD.A'B'C'D'$  . Gọi $M, M'$   lần lượt là trung điểm của các cạnh $BC, B'C'$  Hình chiếu của $\Delta B'DM$  qua phép chiếu song song trên $(A'B'C'D')$  theo phương chiếu $AA'$   là
	\choice
	{$\Delta B'A'M'$}
		{$\Delta C'D'M'$}
			{$\Delta DMM'$}
				{\True$\Delta B'D'M'$}
\end{ex}
\begin{ex}
		Cho hình hộp $ABCD.A'B'C'D'$  . Gọi $M, M'$   lần lượt là trung điểm của các cạnh $BC, B'C'$  Hình chiếu của $\Delta D'CM$  qua phép chiếu song song trên $(A'B'C'D')$  theo phương chiếu $BB'$   là
	\choice
	{$\Delta B'CM'$}
	{\True$\Delta C'D'M'$}
	{$\Delta DMM'$}
	{$\Delta B'D'M'$}
\end{ex}
\begin{ex}
	Cho hình hộp $ABCD.A'B'C'D'$  . Gọi $M, M'$   lần lượt là trung điểm của các cạnh $AD, A'D'$; $N, N'$ lần lườ là trung điểm của các cạnh $CD, C'D'$; $P$ là trung điểm của $DD'$.  Hình chiếu của $\Delta MNP$  qua phép chiếu song song trên $(A'B'C'D')$  theo phương chiếu $BB'$   là
	\choice
	{$\Delta B'N'M'$}
	{\True$\Delta D'M'N'$}
	{$\Delta PM'N'$}
	{$\Delta PD'M'$}
\end{ex}
\begin{ex}
	Trong các mệnh đề sau, có bao nhiêu mệnh đề đúng?
	\begin{itemize}
		\item [a)] Một đường thẳng có thể song song với hình chiếu của nó.
		\item [b)] Một đường thẳng có thể trùng với hình chiếu của nó.
		\item [c)] Hình chiếu song song của hai đường thẳng chéo nhau có thể song song với nhau.
		\item [d)]Hình chiếu song song của hai đường thẳng chéo nhau có thể trung nhau.
	\end{itemize}
\choice
{$1$}
{$2$}
{$3$}
{$4$}
\end{ex}
\begin{ex}
	Trong các mệnh đề sau, có bao nhiêu mệnh đề đúng?
	\begin{itemize}
		\item [a)] Phép chiếu song song biến đoạn thẳng thành đoạn thẳng.
		\item [b)] Phép chiếu song song biến hai đường thẳng song song thành hai đường thẳng cắt nhau.
		\item [c)] Phép chiếu song song biến tam giác đều thành tam giác cân.
		\item [d)] Phép chiếu song song biến hình vuông thành hình bình hành.
	\end{itemize}
	\choice
	{$1$}
	{$2$}
	{$3$}
	{$4$}
\end{ex}

\begin{ex}%[1H2K5]
	Hình chiếu của tứ diện $ABCD$ lên một mặt phẳng $(P)$ theo phương chiếu $AB$ ($AB$ không song song với $(P)$ là
	\choice{\True hình tam giác}
	{hình tứ giác}
	{đoạn thẳng}
	{hình thang}
\end{ex}

\begin{ex}%[1H2K5]
	Hình nào dưới đây \textbf{không phải} là hình biểu diễn của một tứ diện?
	\def\dotEX{}
	\choice{
		\begin{tikzpicture}[scale=0.3]
			\draw (3.,6.)-- (0.,0.);
			\draw (0.,0.)-- (10.,0.);
			\draw (3.,6.)-- (10.,0.);
			\draw [dashed](3.,6.)-- (4.,1.);
			\draw [dashed](4.,1.)-- (0.,0.);
			\draw [dashed](4.,1.)-- (10.,0.);
		\end{tikzpicture}	
	}{
		\begin{tikzpicture}[scale=0.3]
			\draw (3.,6.)-- (0.,0.);
			\draw (0.,0.)-- (10.,0.);
			\draw (3.,6.)-- (10.,0.);
			\draw (3.,6.)-- (4.,1.);
			\draw (4.,1.)-- (0.,0.);
			\draw (4.,1.)-- (10.,0.);
		\end{tikzpicture}	
	}{
		\begin{tikzpicture}[scale=0.3]
			\draw (3.,6.)-- (0.,0.);
			\draw (0.,0.)-- (10.,0.);
			\draw (3.,6.)-- (10.,0.);
			\draw (3.,6.)-- (4.,0.);
		\end{tikzpicture}	
	}{
		\True
		\begin{tikzpicture}[scale=0.2]
			\draw (4.,8.)-- (0.,0.);
			\draw (4.,8.)-- (12.,0.);
			\draw (0.,0.)-- (3.,-2.);
			\draw (3.,-2.)-- (12.,0.);
			\draw (4.,8.)-- (3.,-2.);
			\draw (0.,0.)-- (12.,0.);
		\end{tikzpicture}	
	}
\end{ex}



\begin{ex}%[Võ Đông Phước]%[1D1G2]
	Cho hình lăng trụ tam giác $ABC.A'B'C'$. Gọi $M, M'$ lần lượt là trung điểm của các cạnh $BC, B'C'$ và $I$ là giao điểm của đường thẳng $A'M$ và $(AB'C')$. Tìm hình chiếu song song của $I$ trên $(A'B'C')$ theo phương $BB'$.
	\choice{\True Trung điểm của đoạn thẳng $A'M'$}
	{Trọng tâm của tam giác $A'B'C'$}
	{Điểm $A'$}
	{Điểm $M'$}
\end{ex}
%%%%%%%%%%%%%%%%%%%%%%%%%%%%%%%%%%%%%%%%
\begin{ex}%[Võ Đông Phước]%[1D1G2]
	Cho tứ diện $ABCD$. Gọi $M, N$ lần lượt là trung điểm của các cạnh $AC, BC$, trên cạnh $BD$ lấy điểm $P$ sao cho $BP=2PD$. Mặt phẳng $(MNP)$ cắt mặt phẳng $(ACD)$ theo giao tuyến $d$. Tìm hình chiếu song song của đường thẳng $d$ trên $(BCD)$ theo phương $AD$.
	\choice{Đường thẳng $DN$}
	{\True Đường thẳng $CD$}
	{Đường thẳng $BD$}
	{Điểm $M$}
\end{ex}

\begin{ex}%[1H2G5]
	Cho tứ diện $ABCD$ và $M$ là điểm bất kì thuộc miền trong của tam giác $BCD$. Gọi $B'$, $C'$, $D'$ lần lượt là hình chiếu song song của $M$ theo các phương $AB$, $AC$, $AD$ lên các mặt $(ACD)$, $(ABD)$, $(ABC)$. Tính $\dfrac{MB'}{AB}+\dfrac{MC'}{AC}+\dfrac{MD'}{AD}$.
	\choice{\True $1$}
	{$\dfrac{1}{9}$}
	{$\dfrac{1}{3}$}
	{$3$}
	\loigiai{
		\immini{
			Trong tam giác $ABI$ ta có $\dfrac{MB'}{AB}=\dfrac{MI}{BI}$.\\
			Tương tự ta cũng có $\dfrac{MC'}{AC}=\dfrac{MJ}{CJ}$ và $\dfrac{MD'}{AD}=\dfrac{MK}{DK}$.\\
			Dễ thấy rằng $\dfrac{S_{MBD}}{S_{CBD}}=\dfrac{MJ}{CJ}$, $\dfrac{S_{MCD}}{S_{BCD}}=\dfrac{MI}{BI}$, $\dfrac{S_{MBC}}{S_{DBC}}=\dfrac{MK}{DK}$.\\
			Cộng các đẳng thức với nhau vế theo vế ta được $$\dfrac{MB'}{AB}+\dfrac{MC'}{AC}+\dfrac{MD'}{AD}=1$$
		}{
			\begin{tikzpicture}[scale=0.5]
				\draw [dashed] (2.,5.)-- (10.,5.);
				\draw [line width=1.2pt] (2.,5.)-- (4.,2.);
				\draw [line width=1.2pt] (4.,2.)-- (10.,5.);
				\draw [line width=1.2pt] (3.,11.)-- (2.,5.);
				\draw [line width=1.2pt] (3.,11.)-- (4.,2.);
				\draw [line width=1.2pt] (3.,11.)-- (10.,5.);
				\draw [line width=1.2pt] (3.,11.)-- (7.,3.5);
				\draw [dashed] (2.,5.)-- (7,3.5);
				\draw [dashed] (10.,5.)-- (3.,3.5);
				\draw [dashed] (4.,2.)-- (6.,5.);
				\draw [dashed] (5.4,4)-- (5.7,6);
				\begin{scriptsize}
					\draw (2.14,5.33) node[left] {$B$};
					\draw (10.14,5.33) node {$D$};
					\draw (4.14,2.1) node [below]{$C$};
					\draw (3.14,11.33) node {$A$};
					\draw (7.3,3.83) node [below right]{$I$};
					\draw (3,3.81) node [below left]{$K$};
					\draw (6.14,5.33) node [right]{$J$};
					\draw (5.5,4.1) node [below]{$M$};
					\draw (5.82,6.31) node [above]{$B'$};
				\end{scriptsize}
			\end{tikzpicture}
		}
		
	}	
	
\end{ex}
\centerline{---HẾT---}
\Closesolutionfile{ans}

%Chương V
%%Bài 15. Giới hạn dãy số
\chap{Giới hạn. Hàm số liên tục}
\section{Giới hạn dãy số}
\subsection{Tóm tắt lý thuyết}
\subsubsection{Dãy số có giới hạn $0$}
	\begin{dn}
		Ta nói dãy số $\left(u_{n}\right)$ có giới hạn là $0$ khi $n$ dần tới dương vô cực, nếu $\left|u_{n}\right|$ có thể nhỏ hơn một số dương bé tuỳ ý, kể từ một số hạng nào đó trở đi, kí hiệu $\lim \limits_{n \rightarrow+\infty} u_{n}=0$ hay $u_{n} \rightarrow 0$ khi $n \rightarrow+\infty$.
	\end{dn}
	Từ định nghĩa dãy số có giới hạn $0$, ta có các kết quả sau:
	\begin{itemize}
		\item $\lim \limits_{n \rightarrow+\infty} \dfrac{1}{n^{k}}=0$ với $k$ là một số nguyên dương;	
		\item $\lim \limits_{n \rightarrow+\infty} q^{n}=0$ nếu $|q|<1$;	
		\item Nếu $\left|u_{n}\right| \leq v_{n}$ với mọi $n \geq 1$ và $\lim \limits_{n \rightarrow+\infty} v_{n}=0$ thì $\lim \limits_{n \rightarrow+\infty} u_{n}=0$.	
	\end{itemize}
	\subsubsection{Dãy số có giới hạn hữu hạn}
	\begin{dn}
		Ta nói dãy số $\left(u_{n}\right)$ có giới hạn là số thực a khi $n$ dần tới dương vô cực nếu $$\lim \limits_{n \rightarrow+\infty}\left(u_{n}-a\right)=0,$$ kí hiệu $\lim \limits_{n \rightarrow+\infty} u_{n}=a$ hay $u_{n} \rightarrow a$ khi $n \rightarrow+\infty$. 
	\end{dn}
	\begin{itemize}
		\item Nếu $u_{n}=c$ (c là hằng số) thì $\lim \limits_{n \rightarrow+\infty} u_{n}=c$. 
		\item $\lim \limits_{n \rightarrow+\infty} u_{n}=a$ khi và chỉ khi $\lim \limits_{n \rightarrow+\infty}\left(u_{n}-a\right)=0$.
	\end{itemize}
	\subsubsection{Các quy tắc tính giới hạn}
	\begin{tc} 
		\begin{enumEX}[a)]{1}
			\item Nếu $\lim \limits_{n \rightarrow+\infty} u_{n}=a$ và $\lim \limits_{n \rightarrow+\infty} v_{n}=b$ thì 
			\begin{enumEX}[-)]{2}
				\item 	$\lim \limits_{n \rightarrow+\infty}\left(u_{n}+v_{n}\right)=a+b$.
				\item   $\lim \limits_{n \rightarrow+\infty}\left(u_{n}-v_{n}\right)=a-b$. 
				\item $\lim \limits_{n \rightarrow+\infty}\left(u_{n} \cdot v_{n}\right)=a \cdot b$.
				\item $\lim \limits_{n \rightarrow+\infty} \dfrac{u_{n}}{v_{n}}=\dfrac{a}{b}$ (nếu $b \neq 0$).
			\end{enumEX}
			\item Nếu $u_{n} \geq 0$ với mọi $n$ và $\lim \limits_{n \rightarrow+\infty} u_{n}=a$ thì $
			a \geq 0 \text { và } \lim \limits_{n \rightarrow+\infty} \sqrt{u_{n}}=\sqrt{a}$.
		\end{enumEX}
		
	\end{tc}
\subsection{Các dạng toán thường gặp}
\begin{dang}{Phương pháp đặt thừa số chung (lim hữu hạn)}
	
\end{dang}
\subsubsection{Ví dụ minh hoạ}
\begin{vd}%[1C3Y1-2]%[Anh Duy]%Ví dụ 1.
	Tìm giới hạn sau $\lim\dfrac{2n^3-2n+3}{1-4n^3}$.
	\loigiai{
		\[\lim\dfrac{2n^3-2n+3}{1-4n^3}=\lim\dfrac{2-\dfrac{2}{n^2}+\dfrac{3}{n^3}}{\dfrac{1}{n^3}-4}=-\dfrac{1}{2}.\]}
\end{vd}
\begin{vd}%[1C3Y1-2]%[Anh Duy]%Ví dụ 2.
	Tìm giới hạn sau $\lim\dfrac{\sqrt{n^4+2n+2}}{n^2+1}$.
	\loigiai{
		\[\lim\dfrac{\sqrt{n^4+2n+2}}{n^2+1} = \lim\dfrac{\sqrt{1+\dfrac{2}{n^3}+\dfrac{2}{n^4}}}{1+\dfrac{1}{n^2}}=1.\]}
\end{vd}
\begin{vd}%[1C3Y1-2]%[Anh Duy]%Ví dụ 3.
	Tìm giới hạn sau $\lim\dfrac{3^{n+1}-4^n}{4^{n-1}+3}$.
	\loigiai{
		\[\lim\dfrac{3^{n+1}-4^n}{4^{n-1}+3} = \lim\dfrac{9\cdot 3^{n-1}-4\cdot 4^{n-1}}{4^{n-1}+3}=\lim\dfrac{9\cdot\left(\dfrac{3}{4}\right)^{n-1}-4}{1+3\cdot\left(\dfrac{1}{4}\right)^{n-1}}=-4.\]}
\end{vd}

\begin{vd}%[1C3B1-2]%[Anh Duy]%Ví dụ 4.
	Tìm giới hạn sau $\lim\dfrac{1+2+2^2+\cdots +2^n}{1+3+3^2+\cdots +3^n}$.
	\loigiai{
		\[\lim\dfrac{1+2+2^2+\cdots +2^n}{1+3+3^2+\cdots +3^n} = \lim\dfrac{\dfrac{1-2^{n+1}}{-1}}{\dfrac{1-3^{n+1}}{-2}} = \lim\dfrac{\left(1-2^{n+1}\right)\cdot 2}{1-3^{n+1}} = \lim\dfrac{\left(\left(\dfrac{1}{3}\right)^{n+1}-\left(\dfrac{2}{3}\right)^{n+1}\right)\cdot 2}{\left(\dfrac{1}{3}\right)^{n+1}-1}=0.\]}
\end{vd}
% \subsubsection{Bài tập rèn luyện}
% % \subsubsection{Bài tập tự luận}
% \begin{bt}%[1C3B1-2]%[Anh Duy]
% 	Tìm các giới hạn sau
% 	\begin{enumEX}[a)]{2}
% 		\item[a)] $\lim \limits_{n \rightarrow+\infty} \dfrac{n^{2}+n+1}{2 n^{2}+1}$.
% 		\item[b)] $\lim \limits_{n \rightarrow+\infty}\left(\sqrt{n^{2}+2 n}-n\right)$.
% 	\end{enumEX}
% 	\loigiai{
% 		\begin{enumEX}[a)]{1}
% 			\item $\lim \limits_{n \rightarrow+\infty} \dfrac{n^{2}+n+1}{2 n^{2}+1}=\lim \limits_{n \rightarrow+\infty} \dfrac{1+\dfrac{1}{n}+\dfrac{1}{n^2}}{2+\dfrac{1}{n^2}}=\dfrac{\lim \limits_{n \rightarrow+\infty} \left(1+\dfrac{1}{n}+\dfrac{1}{n^2}\right)}{\lim \limits_{n \rightarrow+\infty} \left(2+\dfrac{1}{n^2}\right)}=\dfrac{1}{2}$.
% 			\item $\lim \limits_{n \rightarrow+\infty}\left(\sqrt{n^{2}+2 n}-n\right)=\lim \limits_{n \rightarrow+\infty}\dfrac{n^2+2n-n^2}{\sqrt{n^{2}+2 n}+n}=\lim \limits_{n \rightarrow+\infty} \dfrac{2}{\sqrt{1+\dfrac{2}{n}}+1}=\dfrac{2}{\lim \limits_{n \rightarrow+\infty} \left(\sqrt{1+\dfrac{2}{n}}+1\right)}=1$.
% 		\end{enumEX}
% 	}
% \end{bt}

% \begin{bt}%[1C3B1-2]%[Anh Duy]
% 	Cho hai dãy số không âm $\left(u_{n}\right)$ và $\left(v_{n}\right)$ với $\lim \limits_{n \rightarrow+\infty} u_{n}=2$ và $\lim \limits_{n \rightarrow+\infty} v_{n}=3$. Tìm các giới hạn sau
% 	\begin{enumEX}[a)]{2}
% 		\item[a)] $\lim \limits_{n \rightarrow+\infty} \dfrac{u_{n}^{2}}{v_{n}-u_{n}}$;
% 		\item[b)] $\lim \limits_{n \rightarrow+\infty} \sqrt{u_{n}+2 v_{n}}$.
% 	\end{enumEX}
% 	\loigiai{
% 		\begin{enumEX}[a)]{1}
% 			\item $\lim \limits_{n \rightarrow+\infty} \dfrac{u_{n}^{2}}{v_{n}-u_{n}} = \dfrac{\lim \limits_{n \rightarrow+\infty} u_{n}^{2}}{\lim \limits_{n \rightarrow+\infty} v_{n}-\lim \limits_{n \rightarrow+\infty} u_{n}} = \dfrac{\left(\lim \limits_{n \rightarrow+\infty} u_{n}\right)^{2}}{\lim \limits_{n \rightarrow+\infty} v_{n}-\lim \limits_{n \rightarrow+\infty} u_{n}}=\dfrac{2^2}{3-2}=4$ ;
% 			\item $\lim \limits_{n \rightarrow+\infty} \sqrt{u_{n}+2 v_{n}} =  \sqrt{\lim \limits_{n \rightarrow+\infty}  u_{n}+\lim \limits_{n \rightarrow+\infty} 2 v_{n}} =\sqrt{\lim \limits_{n \rightarrow+\infty}  u_{n}+2\lim \limits_{n \rightarrow+\infty}  v_{n}} =\sqrt{2+2\cdot 3}=2\sqrt{2}$.
% 		\end{enumEX}
% 	}
% \end{bt}
% \begin{bt}%[1C3B1-2]%[Anh Duy]
% 	Tính các giới hạn sau:
% 	\begin{enumEX}{2}
% 		\item $\lim \limits_{n \to +\infty}\dfrac{2^n+3\cdot 4^n}{4^n-5\cdot 3^n}$.
% 		\item $T=\lim\dfrac{3\cdot7^n+2\cdot 4^n}{4\cdot 5^n+7^n}$.
% 	\end{enumEX}
% 	\loigiai{
% 		\begin{enumEX}{1}
% 			\item $\lim \limits_{n \to +\infty}\dfrac{2^n+3\cdot 4^n}{4^n-5\cdot 3^n}=\lim \limits_{n \to +\infty}\dfrac{\left(\dfrac {1} {2}\right)^n+3}{1-5\left(\dfrac {3} {4}\right)^n}=3$.
% 			\item Ta có $T=\lim\dfrac{3\cdot7^n+2\cdot 4^n}{4\cdot 5^n+7^n}=\lim\dfrac{3+2\cdot\left(\dfrac{4}{7}\right)^n}{4\cdot\left(\dfrac{5}{7}\right)^n+1}=3$.
% 		\end{enumEX}
% 	}
% \end{bt}
\subsubsection{Câu hỏi trắc nghiệm}
\Opensolutionfile{ans}[ans/ans-1K5-1-Dang1]
\begin{ex}%[1C3Y1-2]%[Anh Duy]%Câu 1.
	Tính giới hạn $I=\lim\dfrac{2n+2023}{3n+2024}$. 
	\choice
	{\True $I=\dfrac{2}{3}$}
	{$I=\dfrac{3}{2}$}
	{$I=\dfrac{2023}{2024}$}
	{$I=1$}
	\loigiai{
		Ta có $I=\lim\dfrac{2n+2023}{3n+2024} =\lim\dfrac{2+\dfrac{2017}{n}}{3+\dfrac{2018}{n}} =\dfrac{2}{3}$.}
\end{ex}
\begin{ex}%[1C3Y1-1]%[Anh Duy]%Câu 2.
	Phát biểu nào sau đây là \textbf{sai}?
	\choice
	{$\lim \limits_{n \to +\infty}u_n=c$ ($u_n=c$ là hằng số)}
	{\True $\lim \limits_{n \to +\infty}q^n=0 \;(|q|>1)$}
	{$\lim\dfrac{1}{n}=0$}
	{$\lim\dfrac{1}{n^k}=0 \; (k>1)$}
	\loigiai{
		Theo định nghĩa giới hạn hữu hạn của dãy số thì $\lim \limits_{n \to +\infty}q^n=0 \; (|q|<1)$.}
\end{ex}
\begin{ex}%[1C3Y1-2]%[Anh Duy]%Câu 3.
	Giá trị của $\lim\dfrac{2-n}{n+1}$ bằng
	\choice
	{$1$}
	{$2$}
	{\True $-1$}
	{$0$}
	\loigiai{
		Ta có $\lim\dfrac{2-n}{n+1} =\lim\dfrac{\dfrac{2}{n}-1}{1+\dfrac{1}{n}} =\dfrac{0-1}{1+0} =-1$.}
\end{ex}
\begin{ex}%[1C3Y1-2]%[Anh Duy]%Câu 4.
	Tính giới hạn $\lim\dfrac{4n+2024}{2n+1}$. 
	\choice
	{$\dfrac{1}{2}$}
	{$4$}
	{\True $2$}
	{$2024$}
	\loigiai{
		Ta có $\lim\dfrac{4n+2024}{2n+1}=\lim\dfrac{4+\dfrac{2024}{n}}{2+\dfrac{1}{n}}=2$.}
\end{ex}
\begin{ex}%[1C3Y1-2]%[Anh Duy]%Câu 5.
	$\lim\dfrac{2n^2-3}{n^6+5n^5}$ bằng 
	\choice
	{$2$}
	{\True $0$}
	{$\dfrac{-3}{5}$}
	{$-3$}
	\loigiai{
		Ta có $\lim\dfrac{2n^2-3}{n^6+5n^5} =\lim\dfrac{\dfrac{2}{n^4}-\dfrac{3}{n^6}}{1+\dfrac{5}{n}} =0$.}
\end{ex}
\begin{ex}%[1C3Y1-2]%[Anh Duy]%Câu 6.
	Tính $\lim\dfrac{2n+1}{1+n}$ được kết quả là
	\choice
	{\True $2$}
	{$0$}
	{$\dfrac{1}{2}$}
	{$1$}
	\loigiai{
		Ta có $\lim\dfrac{2n+1}{1+n}=\lim\dfrac{n\left(2+\dfrac{1}{n}\right)}{n\left(\dfrac{1}{n}+1\right)}=\lim\dfrac{2+\dfrac{1}{n}}{\dfrac{1}{n}+1}=\dfrac{2+0}{0+1}=2$.}
\end{ex}

\begin{ex}%[1C3Y1-2]%[Anh Duy]%Câu 7.
	Dãy số nào sau đây có giới hạn khác $0$?
	\choice
	{$\dfrac{1}{n}$}
	{$\dfrac{1}{\sqrt{n}}$}
	{\True $\dfrac{n+1}{n}$}
	{$\dfrac{\sin n}{\sqrt{n}}$}
	\loigiai{
		Có $\lim\dfrac{n+1}{n}=\lim \limits_{n \to +\infty}1+\lim\dfrac{1}{n}=1$.}
\end{ex}

\begin{ex}%[1C3B1-2]%[Anh Duy]%Câu 8.
	Giới hạn $\lim\dfrac{\sqrt{n}}{2n^2+3}$ có kết quả là 
	\choice
	{$2$}
	{\True $0$}
	{$+\infty$}
	{$4$}
	\loigiai{
		$\lim\dfrac{\sqrt{n}}{2n^2+3}=\lim\dfrac{\sqrt{\dfrac{1}{n^3}}}{2+\dfrac{3}{n^2}}=\dfrac{0}{2+0}=0$.}
\end{ex}
\begin{ex}%[1C3Y1-1]%[Anh Duy]%Câu 9.
	Dãy số $(u_n)$ với $u_n=\dfrac{1}{2n}$, chọn $M=\dfrac{1}{100}$, để $\dfrac{1}{2n}<\dfrac{1}{100}$ thì $n$ phải lấy từ số hạng thứ bao nhiêu trở đi?
	\choice
	{\True $51$}
	{$49$}
	{$48$}
	{$50$}
	\loigiai{Ta có $\dfrac{1}{2n}<\dfrac{1}{100}\Leftrightarrow 2n>100\Leftrightarrow n>50$.\\
		Vậy $n$ phải lấy từ số hạng thứ $51$ trở đi.}
\end{ex}
\begin{ex}%[1C3B1-2]%[Anh Duy]%Câu 10.
	Giới hạn $\lim\dfrac{3^n+2^n}{4^n}$ có kết quả là 
	\choice
	{\True $0$}
	{$\dfrac{5}{4}$}
	{$\dfrac{3}{4}$}
	{$+\infty$}
	\loigiai{
		Ta có $\lim\dfrac{3^n+2^n}{4^n}=\lim\dfrac{\left(\dfrac{3}{4}\right)^n+\left(\dfrac{2}{4}\right)^n}{1}=0$.}
\end{ex}
\begin{ex}%[1C3B1-2]%[Anh Duy]%Câu 11.
	Tính giới hạn $\lim\left[\dfrac{1}{1\cdot 2}+\dfrac{1}{2\cdot 3}+\dfrac{1}{3\cdot 4}+\cdots +\dfrac{1}{n(n+1)}\right]$. 
	\choice
	{$0$}
	{$2$}
	{\True $1$}
	{$\dfrac{3}{2}$}
	\loigiai{
		Ta có $\dfrac{1}{1\cdot 2}+\dfrac{1}{2\cdot 3}+\dfrac{1}{3\cdot 4}+\cdots +\dfrac{1}{n(n+1)} =\dfrac{1}{1}-\dfrac{1}{2}+\dfrac{1}{2}-\dfrac{1}{3}+\cdots+\dfrac{1}{n-1}-\dfrac{1}{n}+\dfrac{1}{n}-\dfrac{1}{n+1} =1-\dfrac{1}{n+1}$.\\
		Vậy $\lim\left[\dfrac{1}{1\cdot 2}+\dfrac{1}{2\cdot 3}+\dfrac{1}{3\cdot 4}+\cdots +\dfrac{1}{n(n+1)}\right] =\lim\left(1-\dfrac{1}{n+1}\right)=1$.}
\end{ex}

\begin{ex}%[1C3K1-2]%[Anh Duy]%Câu 12.
	Tính $\lim\sqrt{\dfrac{1^2+2^2+3^2+\cdots +n^2}{2n(n+7)(6n+5)}}$. 
	\choice
	{\True $\dfrac{1}{6}$}
	{$\dfrac{1}{2\sqrt{6}}$}
	{$\dfrac{1}{2}$}
	{$+\infty$}
	\loigiai{
		Ta có $1^2+2^2+3^2+\cdots +n^2=\dfrac{n(n+1)(2n+1)}{6}$.\\
		Khi đó $\lim\sqrt{\dfrac{1^2+2^2+3^3+\cdots +n^2}{2n(n+7)(6n+5)}}=\lim\sqrt{\dfrac{n(n+1)(2n+1)}{12n(n+7)(6n+5)}} =\lim\sqrt{\dfrac{\left(1+\dfrac{1}{n}\right)\left(2+\dfrac{1}{n}\right)}{12\left(1+\dfrac{7}{n}\right)\left(6+\dfrac{5}{n}\right)}} =\dfrac{1}{6}$.}
\end{ex}
\begin{ex}%[1C3K1-2]%[Anh Duy]%Câu 13.
	Giới hạn $\lim\dfrac{(2n-1)(3-n)^2}{(4n-5)^3}$ có kết quả bằng 
	\choice
	{$0$}
	{\True $\dfrac{1}{32}$}
	{$\dfrac{3}{2}$}
	{$\dfrac{1}{2}$}
	\loigiai{
		$\lim\dfrac{(2n-1)(3-n)^2}{(4n-5)^3}=\lim\dfrac{\left(2-\dfrac{1}{n}\right)\left(\dfrac{3}{n}-1\right)^2}{\left(4-\dfrac{5}{n}\right)^3}=\dfrac{2}{4^3}=\dfrac{1}{32}$.}
\end{ex}
\begin{ex}%[1C3K1-2]%[Anh Duy]%Câu 14.
	Tìm $L=\lim\left(\dfrac{1}{1}+\dfrac{1}{1+2}+\cdots +\dfrac{1}{1+2+\cdots +n}\right)$.
	\choice
	{$L=\dfrac{5}{2}$}
	{$L=+\infty$}
	{\True $L=2$}
	{$L=\dfrac{3}{2}$}
	\loigiai{
		Ta có $1+2+3+\cdots +k$ là tổng của cấp số cộng có $u_1=1$, $d=1$ nên $1+2+3+\cdots +k=\dfrac{(1+k)k}{2}$. Khi đó
		\[\dfrac{1}{1+2+\cdots +k}=\dfrac{2}{k(k+1)} =\dfrac{2}{k}-\dfrac{2}{k+1},\; \forall k\in\mathbb{N}^*.\]
		Suy ra	\[L=\lim\left(\dfrac{2}{1}-\dfrac{2}{2}+\dfrac{2}{2}-\dfrac{2}{3}+\dfrac{2}{3}-\dfrac{2}{4}+\cdots +\dfrac{2}{n}-\dfrac{2}{n+1}\right) =\lim\left(\dfrac{2}{1}-\dfrac{2}{n+1}\right) =2.\]}
\end{ex}
%--------------------------------------------------------------------------------------------------

\begin{ex}%[1C3G1-2]%[Anh Duy]%Câu 15.
	Đặt $f(n)=\left(n^2+n+1\right)^2+1$.
	Xét dãy số $(u_n)$ sao cho $u_n=\dfrac{f(1)\cdot f(3)\cdot f(5)\cdots f(2n-1)}{f(2)\cdot f(4)\cdot f(6)\cdots f(2n)}$. Tính $\lim \limits_{n \to +\infty}n\sqrt{u_n}$. 
	\choice
	{$\lim \limits_{n \to +\infty}n\sqrt{u_n}=\sqrt{2}$}
	{$\lim \limits_{n \to +\infty}n\sqrt{u_n}=\dfrac{1}{\sqrt{3}}$}
	{$\lim \limits_{n \to +\infty}n\sqrt{u_n}=\sqrt{3}$}
	{\True $\lim \limits_{n \to +\infty}n\sqrt{u_n}=\dfrac{1}{\sqrt{2}}$}
	\loigiai{
		Xét $g(n)=\dfrac{f(2n-1)}{f(2n)}\Rightarrow g(n)=\dfrac{\left(4n^2-2n+1\right)^2+1}{\left(4n^2+2n+1\right)^2+1}$.\\
		$g(n)=\dfrac{\left(4n^2+1\right)^2-4n\left(4n^2+1\right)+\left(4n^2+1\right)}{\left(4n^2+1\right)^2+4n\left(4n^2+1\right)+\left(4n^2+1\right)}=\dfrac{4n^2+1-4n+1}{4n^2+1+4n+1}=\dfrac{(2n-1)^2+1}{(2n+1)^2+1}$ \\
		$ \Rightarrow u_n=\dfrac{2}{10}\cdot\dfrac{10}{26}\cdot\dfrac{26}{50}\cdots\cdot\dfrac{(2n-3)^2+1}{(2n-1)^2+1}\cdot\dfrac{(2n-1)^2+1}{(2n+1)^2+1}=\dfrac{2}{(2n+1)^2+1} $ \\
		$ \Rightarrow\lim \limits_{n \to +\infty}n\sqrt{u_n}=\lim\sqrt{\dfrac{2n^2}{4n^2+4n+2}}=\dfrac{1}{\sqrt{2}} $.}
\end{ex}
\begin{ex}%[1C3G1-2]%[Anh Duy]%Câu 16.
	Có bao nhiêu giá trị nguyên của tham số $a$ thuộc khoảng $(0;2024)$ để có
	\[ \lim\sqrt{\dfrac{9^n+3^{n+1}}{5^n+9^{n+a}}}\leq\dfrac{1}{2187}\,? \]
	\choice
	{\True $2017$}
	{$2016$}
	{$2023$}
	{$2024$}
	\loigiai{
		Do $\dfrac{9^n+3^{n+1}}{5^n+9^{n+a}}>0$ với $\forall n$ nên $\lim\sqrt{\dfrac{9^n+3^{n+1}}{5^n+9^{n+a}}}=\sqrt{\lim\dfrac{9^n+3^{n+1}}{5^n+9^{n+a}}} =\sqrt{\lim\dfrac{1+3\cdot\left(\dfrac{1}{3}\right)^n}{\left(\dfrac{5}{9}\right)^n+9^a}} =\sqrt{\dfrac{1}{9^a}} =\dfrac{1}{3^a}$.\\
		Theo đề bài ta có $\lim\sqrt{\dfrac{9^n+3^{n+1}}{5^n+9^{n+a}}}\leq\dfrac{1}{2187}\Leftrightarrow\dfrac{1}{3^a}\leq\dfrac{1}{2187}\Leftrightarrow a\geq 7$.\\
		Do $a$ là số nguyên thuộc khoảng $(0;2024)$ nên có $a\in\left\{7;8;9;\ldots;2023\right\}\Rightarrow$ có $2017$ giá trị của $a$.}
\end{ex}
\Closesolutionfile{ans}
% \begin{indapan}{10}
% 	{ans/ans-1K5-1-Dang1}
% \end{indapan}
\begin{dang}{Phương pháp lượng liên hợp (lim hữu hạn)}
	Nếu giới hạn của dãy số ở dạng vô định thì ta sử dụng các phép biến đổi để đưa về dạng cơ bản. \\
	Một số phép biến đổi liên hợp: \\
	\begin{align*}
		f(n) - g(n) &= \dfrac{(f(n))^2 - (g(n))^2}{f(n) + g(n)} \\
		\sqrt{f(n)} - \sqrt{g(n)} &= \dfrac{f(n) - g(n)}{\sqrt{f(n)} + \sqrt{g(n)}} \\
		\sqrt{f(n)} - g(n) &= \dfrac{f(n) - (g(n))^2}{\sqrt{f(n)} + g(n)} \\
		\sqrt[3]{f(n)} - \sqrt[3]{g(n)} &= \dfrac{f(n) - g(n)}{\sqrt[3]{(f(n))^2} + \sqrt[3]{f(n)g(n)} + \sqrt[3]{(g(n))^2}}
	\end{align*}
	
\end{dang}
\subsubsection{Ví dụ minh hoạ}
\begin{vd}%[Dự án soạn đề cương toán 11 - KNTT, Minh Trí]%[1K5BE-3]
	Tính giới hạn $I = \lim \limits_{n \to +\infty}\left(\sqrt{n^2 - 2n + 3} - n\right)$. 
	\loigiai{
		Ta có 
		\begin{align*}
			I &= \lim \limits_{n \to +\infty}\left(\sqrt{n^2 - 2n + 3} - n\right) \\
			&= \lim \limits_{n \to +\infty}\dfrac{n^2 - 2n + 3 - n^2}{\sqrt{n^2 - 2n + 3} + n} \\
			&= \lim \limits_{n \to +\infty}\dfrac{- 2n + 3}{\sqrt{n^2 - 2n + 3} + n} \\
			&= \lim \limits_{n \to +\infty}\dfrac{- 2 + \dfrac{3}{n}}{\sqrt{1 - \dfrac{2}{n} + \dfrac{3}{n^2}} + 1} \\
			&= \dfrac{- 2}{\sqrt{1} + 1} = - 1
		\end{align*}
	}
\end{vd} 
\begin{vd}%[Dự án soạn đề cương toán 11 - KNTT, Minh Trí]%[1K5BE-3]
	Tính giới hạn $I = \lim \limits_{n \to +\infty}\left(\sqrt{n^2 + 7} - \sqrt{n^2 + 5}\right)$. 
	\loigiai{
		Ta có 
		\begin{align*}
			I &= \lim \limits_{n \to +\infty}\left(\sqrt{n^2 + 7} - \sqrt{n^2 + 5}\right) \\
			&= \lim \limits_{n \to +\infty}\dfrac{n^2 + 7 - (n^2 + 5)}{\sqrt{n^2 + 7} + \sqrt{n^2 + 5}} \\
			&= \lim \limits_{n \to +\infty}\dfrac{2}{\sqrt{n^2 + 7} + \sqrt{n^2 + 5}} \\
			&= 0
		\end{align*}
	}
\end{vd}
\begin{vd}%[Dự án soạn đề cương toán 11 - KNTT, Minh Trí]%[1K5KE-3]
	Tính giới hạn $I = \lim \limits_{n \to +\infty}\left(\sqrt{n^2 + 2n} - \sqrt{n^2 - 2n}\right)$. 
	\loigiai{
		Ta có 
		\begin{align*}
			I &= \lim \limits_{n \to +\infty}\left(\sqrt{n^2 + 2n} - \sqrt{n^2 - 2n}\right) \\
			&= \lim \limits_{n \to +\infty}\dfrac{n^2 + 2n - (n^2 - 2n)}{\sqrt{n^2 + 2n} + \sqrt{n^2 - 2n}} \\
			&= \lim \limits_{n \to +\infty}\dfrac{4n}{\sqrt{n^2 + 2n} + \sqrt{n^2 - 2n}} \\
			&= \lim \limits_{n \to +\infty}\dfrac{4}{\sqrt{1 + \dfrac{2}{n}} + \sqrt{1 - \dfrac{2}{n}}} \\
			&= \dfrac{4}{\sqrt{1} + \sqrt{1}} = 2
		\end{align*}
	}
\end{vd}

\begin{vd}%[Dự án soạn đề cương toán 11 - KNTT, Minh Trí]%[1K5KE-3]
	Tính giới hạn $I = \lim \limits_{n \to +\infty}\left(\sqrt{2n^2 - n + 1} - \sqrt{2n^2 - 3n + 2}\right)$. 
	\loigiai{
		Ta có 
		\begin{align*}
			I &= \lim \limits_{n \to +\infty}\left(\sqrt{2n^2 - n + 1} - \sqrt{2n^2 - 3n + 2}\right) \\
			&= \lim \limits_{n \to +\infty}\dfrac{2n^2 - n + 1 - (2n^2 - 3n + 2)}{\sqrt{2n^2 - n + 1} + \sqrt{2n^2 - 3n + 2}} \\
			&= \lim \limits_{n \to +\infty}\dfrac{2n - 1}{\sqrt{2n^2 - n + 1} + \sqrt{2n^2 - 3n + 2}} \\
			&= \lim \limits_{n \to +\infty}\dfrac{2 - \dfrac{1}{n}}{\sqrt{2 - \dfrac{1}{n} + \dfrac{1}{n^2}} + \sqrt{2 - \dfrac{3}{n} + \dfrac{2}{n^2}}} \\
			&= \dfrac{2}{\sqrt{2} + \sqrt{2}} = \dfrac{1}{\sqrt{2}}
		\end{align*}
	}
\end{vd} 
\begin{vd}%[Dự án soạn đề cương toán 11 - KNTT, Minh Trí]%[1K5GE-3]
	Tính giới hạn $I = \lim \limits_{n \to +\infty}\left(n - \sqrt[3]{n^3 + 3n^2 + 1}\right)$.
	\loigiai{
		Ta có 
		\begin{align*} 
			I &= \lim \limits_{n \to +\infty}\left(n - \sqrt[3]{n^3 + 3n^2 + 1}\right) \\
			&= \lim \limits_{n \to +\infty}\dfrac{n^3 - (n^3 + 3n^2 + 1)}{n^2 + \sqrt[3]{n^3 + 3n^2 + 1} + \sqrt[3]{\left(n^3 + 3n^2 + 1\right)^2}} \\
			&= \lim \limits_{n \to +\infty}\dfrac{- 3n^2 - 1}{n^2 + \sqrt[3]{n^3 + 3n^2 + 1} + \sqrt[3]{\left(n^3 + 3n^2 + 1\right)^2}} \\
			&= \lim \limits_{n \to +\infty}\dfrac{- 3 - \dfrac{1}{n^2}}{1 + \sqrt[3]{1 + \dfrac{3}{n} + \dfrac{1}{n^3}} + \sqrt[3]{\left(1 + \dfrac{3}{n} + \dfrac{1}{n^3}\right)^2}} \\ 
			&= \dfrac{- 3}{1 + \sqrt[3]{1} + \sqrt[3]{1}} = - 1
		\end{align*}
	}
\end{vd}
% \subsubsection{Bài tập rèn luyện}
% % \subsubsection{Bài tập tự luận}
% \begin{bt}%[Dự án soạn đề cương toán 11 - KNTT, Minh Trí]%[1K5KE-3]
% 	Tính giới hạn  $I = \lim \limits_{n \to +\infty}\left(\sqrt[3]{n^3 - 2n} - n\right)$. 
% 	\loigiai{
% 		Ta có
% 		\begin{align*} 
% 			I &= \lim \limits_{n \to +\infty}\left(\sqrt[3]{n^3 - 2n} - n\right) \\
% 			&= \lim \limits_{n \to +\infty}\dfrac{n^3 - 2n - n^3}{\sqrt[3]{\left(n^3 - 2n\right)^2} + n \sqrt[3]{n^3 - 2n} + n^2} \\ 
% 			&= \lim \limits_{n \to +\infty}\dfrac{-2n}{\sqrt[3]{\left(n^3 - 2n\right)^2} + n \sqrt[3]{n^3 - 2n} + n^2} \\
% 			&= \lim \limits_{n \to +\infty}\dfrac{\dfrac{-2}{n}}{\sqrt[3]{\left(1 - \dfrac{2}{n^2}\right)^2} + \sqrt[3]{1 - \dfrac{2}{n^2}} + 1} \\
% 			&= 0 
% 		\end{align*}
% 	}
% \end{bt}
% \begin{bt}%[Dự án soạn đề cương toán 11 - KNTT, Minh Trí]%[1K5KE-3]
% 	Tính giới hạn $I = \lim \limits_{n \to +\infty}\left(\sqrt{4n^2 + 5n} - 2n\right)$.
% 	\loigiai{
% 		Ta có
% 		\begin{align*} 
% 			I &= \lim \limits_{n \to +\infty}\left(\sqrt{4n^2 + 5n} - 2n\right) \\
% 			&= \lim \limits_{n \to +\infty}\dfrac{4n^2 + 5n - 4n^2}{\sqrt{4n^2 + 5n} - 2n} \\ 
% 			&= \lim \limits_{n \to +\infty}\dfrac{5n}{\sqrt{4n^2 + 5n} + 2n} \\
% 			&= \lim \limits_{n \to +\infty}\dfrac{5}{\sqrt{4 + \dfrac{5}{n}} + 2} \\
% 			&= \dfrac{5}{\sqrt{4} + 2} = \dfrac{5}{4}
% 		\end{align*}
% 	}
% \end{bt}
% \begin{bt}%[Dự án soạn đề cương toán 11 - KNTT, Minh Trí]%[1K5KE-3]
% 	Tính giới hạn $I = \lim \limits_{n \to +\infty}\left(3n - \sqrt{9n^2 + 1}\right)$.
% 	\loigiai{
% 		Ta có
% 		\begin{align*} 
% 			I &= \lim \limits_{n \to +\infty}\left(3n - \sqrt{9n^2 + 1}\right) \\
% 			&= \lim \limits_{n \to +\infty}\dfrac{9n^2 - (9n^2 + 1)}{3n + \sqrt{9n^2 + 1}} \\ 
% 			&= \lim \limits_{n \to +\infty}\dfrac{- 1}{3n + \sqrt{9n^2 + 1}} \\
% 			&= 0
% 		\end{align*}
% 	}
% \end{bt}  
% \begin{bt}%[Dự án soạn đề cương toán 11 - KNTT, Minh Trí]%[1K5KE-3]
% 	Tính giới hạn $I = \lim \limits_{n \to +\infty}\left(3n - 5 - \sqrt{9n^2 + 1}\right)$.
% 	\loigiai{
% 		Ta có
% 		\begin{align*} 
% 			I &= \lim \limits_{n \to +\infty}\left(3n - \sqrt{9n^2 + 1}\right) \\
% 			&= \lim \limits_{n \to +\infty}\dfrac{(3n - 5)^2 - (9n^2 + 1)}{3n - 5 + \sqrt{9n^2 + 1}} \\ 
% 			&= \lim \limits_{n \to +\infty}\dfrac{- 30n + 24}{3n - 5 + \sqrt{9n^2 + 1}} \\
% 			&= \lim \limits_{n \to +\infty}\dfrac{- 30 + \dfrac{24}{n}}{3 - \dfrac{5}{n} + \sqrt{\dfrac{1}{n^2}}} \\
% 			&= \dfrac{- 30}{3 + \sqrt{9}} = - 5
% 		\end{align*}
% 	}
% \end{bt}  
% \begin{bt}%[Dự án soạn đề cương toán 11 - KNTT, Minh Trí]%[1K5KE-3]
% 	Tính giới hạn $I = \lim \limits_{n \to +\infty}\left(\sqrt[3]{n + 2} - \sqrt[3]{n}\right)$.
% 	\loigiai{
% 		Ta có
% 		\begin{align*} 
% 			I &= \lim \limits_{n \to +\infty}\left(\sqrt[3]{n + 2} - \sqrt[3]{n}\right) \\
% 			&= \lim \limits_{n \to +\infty}\dfrac{n + 2 - n}{\sqrt[3]{(n + 2)^2} + \sqrt[3]{n(n + 2)} + \sqrt[3]{n^2}} \\ 
% 			&= \lim \limits_{n \to +\infty}\dfrac{2}{\sqrt[3]{(n + 2)^2} + \sqrt[3]{n(n + 2)} + \sqrt[3]{n^2}} \\  
% 			&= 0
% 		\end{align*}
% 	}
% \end{bt} 
% \begin{bt}%[Dự án soạn đề cương toán 11 - KNTT, Minh Trí]%[1K5KE-3]
% 	Tính giới hạn $I = \lim \limits_{n \to +\infty}\dfrac{\sqrt{4n^2 + 2n} - n + 1}{\sqrt{9n^2 + n} - 2n}$.
% 	\loigiai{
% 		Ta có
% 		\begin{align*} 
% 			I &= \lim \limits_{n \to +\infty}\dfrac{\sqrt{4n^2 + 2n} - n + 1}{\sqrt{9n^2 + n} - 2n} \\
% 			&= \lim \limits_{n \to +\infty}\dfrac{\left[4n^2 + 2n - (n - 1)^2\right]\left(\sqrt{9n^2 + n} + 2n\right)}{\left(\sqrt{4n^2 + 2n} + n - 1\right)(9n^2 + n - 4n^2)} \\ 
% 			&= \lim \limits_{n \to +\infty}\dfrac{(3n^2 + 4n - 1)\left(\sqrt{9n^2 + n} + 2n\right)}{\left(\sqrt{4n^2 + 2n} + n - 1)(5n^2 + n\right)} \\
% 			&= \lim \limits_{n \to +\infty}\dfrac{\left(3 + \dfrac{4}{n} - \dfrac{1}{n^2}\right)\left(\sqrt{9 + \dfrac{1}{n}} + 2\right)}{\left(\sqrt{4 + \dfrac{2}{n}} + 1 - \dfrac{1}{n}\right)\left(5 + \dfrac{1}{n}\right)} \\
% 			&= \dfrac{3\left(\sqrt{9} + 2\right)}{5\left(\sqrt{4} + 1\right)} = 1
% 		\end{align*}
% 	}
% \end{bt}  
% \begin{bt}%[Dự án soạn đề cương toán 11 - KNTT, Minh Trí]%[1K5KE-3]
% 	Tính giới hạn $I = \lim \limits_{n \to +\infty}\left(\sqrt[3]{8n^3 + 1} - \sqrt{4n^2 - n + 5}\right)$.
% 	\loigiai{
% 		Ta có
% 		\begin{align*} 
% 			I &= \lim \limits_{n \to +\infty}\left(\sqrt[3]{8n^3 + 1} - \sqrt{4n^2 - n + 5}\right) \\
% 			&= \lim \limits_{n \to +\infty}\left(\sqrt[3]{8n^3 + 1} - 2n\right) + \lim \limits_{n \to +\infty}\left(2n - \sqrt{4n^2 - n + 5}\right) \\ 
% 			&= \lim \limits_{n \to +\infty}\dfrac{8n^3 + 1 - 8n^3}{\sqrt[3]{(8n^3 + 1)^2} + 2n\sqrt[3]{8n^3 + 1} + 4n^2} + \lim \limits_{n \to +\infty}\dfrac{4n^2 - (4n^2 - n + 5)}{2n + \sqrt{4n^2 - n + 5}} \\
% 			&= \lim \limits_{n \to +\infty}\dfrac{1}{\sqrt[3]{(8n^3 + 1)^2} + 2n\sqrt[3]{8n^3 + 1} + 4n^2} + \lim \limits_{n \to +\infty}\dfrac{n - 5}{2n + \sqrt{4n^2 - n + 5}} \\
% 			&= 0 + \lim \limits_{n \to +\infty}\dfrac{1 - \dfrac{5}{n}}{2 + \sqrt{4 - \dfrac{1}{n} + \dfrac{5}{n^2}}} \\
% 			&= \dfrac{1}{2 + \sqrt{4}} = \dfrac{1}{4}
% 		\end{align*}
% 	}
% \end{bt}  
% \begin{bt}%[Dự án soạn đề cương toán 11 - KNTT, Minh Trí]%[1K5KE-3]
% 	Tính giới hạn $I = \lim \limits_{n \to +\infty}\dfrac{\sqrt{3n^2 + 1} - \sqrt{n - 1}}{n}$.
% 	\loigiai{
% 		Ta có
% 		\begin{align*} 
% 			I &= \lim \limits_{n \to +\infty}\dfrac{\sqrt{3n^2 + 1} - \sqrt{n - 1}}{n} \\
% 			&= \lim \limits_{n \to +\infty}\dfrac{3n^2 + 1 - (n - 1)}{n \left(\sqrt{3n^2 + 1} + \sqrt{n - 1}\right)} \\ 
% 			&= \lim \limits_{n \to +\infty}\dfrac{3n^2 - n + 2}{n \left(\sqrt{3n^2 + 1} + \sqrt{n - 1}\right)} \\
% 			&= \lim \limits_{n \to +\infty}\dfrac{3 - \dfrac{1}{n} + \dfrac{2}{n^2}}{\sqrt{3 + \dfrac{1}{n^2}} + \sqrt{\dfrac{1}{n} - \dfrac{1}{n^2}}} \\
% 			&= \sqrt{3}
% 		\end{align*}
% 	}
% \end{bt}  
\subsubsection{Câu hỏi trắc nghiệm}
\Opensolutionfile{ans}[ans/ans-1K5-1-Dang2]
\begin{ex}%[Dự án soạn đề cương toán 11 - KNTT, Minh Trí]%[1K5TE-3]
	Tính giới hạn $I = \lim \limits_{n \to +\infty}\left(\sqrt{n^2 + 2n + 3} - n\right)$
	\choice
	{\True $1$}
	{$0$}
	{$2$}
	{$3$}
	\loigiai{
		Ta có $I = \lim \limits_{n \to +\infty}\left(\sqrt{n^2 + 2n + 3} - n\right) = \lim \limits_{n \to +\infty}\dfrac{2n + 3}{\sqrt{n^2 + 2n + 3} + n} = \lim \limits_{n \to +\infty}\dfrac{2 + \dfrac{3}{n}}{\sqrt{1 + \dfrac{2}{n} + \dfrac{3}{n^2}} + 1} = 1$
	}
\end{ex}
\begin{ex}%[Dự án soạn đề cương toán 11 - KNTT, Minh Trí]%[1K5KE-3]
	Tính giới hạn $I = \lim \limits_{n \to +\infty}\left(\sqrt{n^2 + 1} - \sqrt{n^2 - 2}\right)$
	\choice
	{$3$}
	{\True $0$}
	{$\sqrt{3}$}
	{$\dfrac{3}{2}$}
	\loigiai{
		Ta có $I = \lim \limits_{n \to +\infty}\left(\sqrt{n^2 + 1} - \sqrt{n^2 - 2}\right) = \lim \limits_{n \to +\infty}\dfrac{3}{\sqrt{n^2 + 1} + \sqrt{n^2 - 2}} = 0$
	}
\end{ex}

\begin{ex}%[Dự án soạn đề cương toán 11 - KNTT, Minh Trí]%[1K5BE-3]
	Tính giới hạn $I = \lim \limits_{n \to +\infty}\left(n - \sqrt{n^2 + 2n - 3}\right)$
	\choice
	{$0$}
	{$-2$}
	{\True $-1$}
	{$2$}
	\loigiai{
		Ta có $I = \lim \limits_{n \to +\infty}\left(n - \sqrt{n^2 + 2n - 3}\right) = \lim \limits_{n \to +\infty}\dfrac{- 2n + 3}{n + \sqrt{n^2 + 2n - 3}} = \lim \limits_{n \to +\infty}\dfrac{- 2 + \dfrac{3}{n}}{1 + \sqrt{1 + \dfrac{2}{n} - \dfrac{3}{n^2}}} = - 1$
	}
\end{ex}
\begin{ex}%[Dự án soạn đề cương toán 11 - KNTT, Minh Trí]%[1K5BE-3]
	Tính giới hạn $I = \lim \limits_{n \to +\infty}\left(\sqrt{n^2 - n + 1} - n\right)$
	\choice
	{$\dfrac{1}{2}$}
	{$1$}
	{\True $- \dfrac{1}{2}$}
	{$0$}
	\loigiai{
		Ta có $I = \lim \limits_{n \to +\infty}\left(\sqrt{n^2 - n + 1} - n\right) = \lim \limits_{n \to +\infty}\dfrac{- n + 1}{\sqrt{n^2 - n + 1} + n} = \lim \limits_{n \to +\infty}\dfrac{- 1 + \dfrac{1}{n}}{\sqrt{1 - \dfrac{1}{n} + \dfrac{1}{n^2}} + 1} = - \dfrac{1}{2}$
	}
\end{ex}
\begin{ex}%[Dự án soạn đề cương toán 11 - KNTT, Minh Trí]%[1K5BE-3]
	Tính giới hạn $I = \lim \limits_{n \to +\infty}\left(\sqrt[3]{n^3 - n^2} - n\right)$
	\choice
	{\True $- \dfrac{1}{3}$}
	{$\dfrac{1}{3}$}
	{$1$}
	{$0$}
	\loigiai{
		Ta có $I = \lim \limits_{n \to +\infty}\left(\sqrt[3]{n^3 - n^2} - n\right)) = \lim \limits_{n \to +\infty}\dfrac{- n^2}{\sqrt[3]{(n^3 - n^2)^2} + n\sqrt[3]{(n^3 - n^2)} + n^2} = \lim \limits_{n \to +\infty}\dfrac{- 1}{\sqrt[3]{\left(1 - \dfrac{1}{n}\right)^2} + \sqrt[3]{1 - \dfrac{1}{n}} + 1} = - \dfrac{1}{3}$
	}
\end{ex}

\begin{ex}%[Dự án soạn đề cương toán 11 - KNTT, Minh Trí]%[1K5BE-3]
	Tính giới hạn $I = \lim \limits_{n \to +\infty}\left(\sqrt{2n^2 + 2n - 1} - \sqrt{2n^2 + n}\right)$
	\choice
	{$0$}
	{$\dfrac{1}{4}$}
	{$\dfrac{1}{2}$}
	{\True $\dfrac{1}{2\sqrt{2}}$}
	\loigiai{
		Ta có $I = \lim \limits_{n \to +\infty}\left(\sqrt{2n^2 + 2n - 1} - \sqrt{2n^2 + n}\right) = \lim \limits_{n \to +\infty}\dfrac{n}{\sqrt{2n^2 + 2n - 1} + \sqrt{2n^2 + n}} = \lim \limits_{n \to +\infty}\dfrac{1}{\sqrt{2 + \dfrac{2}{n} - \dfrac{1}{n^2}} + \sqrt{2 + \dfrac{2}{n}}} = \dfrac{1}{2\sqrt{2}}$
	}
\end{ex}

\begin{ex}%[Dự án soạn đề cương toán 11 - KNTT, Minh Trí]%[1K5BE-3]
	Tính giới hạn $I = \lim \limits_{n \to +\infty}\left(\sqrt[3]{n^3 + 2} - \sqrt[3]{n^3 + 1}\right)$
	\choice
	{\True $0$}
	{$-\dfrac{1}{3}$}
	{$1$}
	{$\dfrac{1}{3}$}
	\loigiai{
		Ta có $I = \lim \limits_{n \to +\infty}\left(\sqrt[3]{n^3 + 2} - \sqrt[3]{n^3 + 1}\right) = \lim \limits_{n \to +\infty}\dfrac{1}{\sqrt[3]{(n^3 + 2)^2} + \sqrt[3]{(n^3 + 2)(n^3 + 1)} + \sqrt[3]{(n^3 + 1)^2}} = 0$
	}
\end{ex}
\begin{ex}%[Dự án soạn đề cương toán 11 - KNTT, Minh Trí]%[1K5BE-3]
	Tính giới hạn $I = \lim \limits_{n \to +\infty}n\left(\sqrt{n^2 + n + 1} - \sqrt{n^2 + n - 8}\right)$
	\choice
	{$0$}
	{$\infty$}
	{$2$}
	{\True $\dfrac{9}{2}$}
	\loigiai{
		Ta có $I = \lim \limits_{n \to +\infty}n\left(\sqrt{n^2 + n + 1} - \sqrt{n^2 + n - 8}\right) = \lim \limits_{n \to +\infty}\dfrac{9n}{\sqrt{n^2 + n + 1} + \sqrt{n^2 + n - 8}}\\ = \lim \limits_{n \to +\infty}\dfrac{9}{\sqrt{1 + \dfrac{1}{n} + \dfrac{1}{n^2}} + \sqrt{1 + \dfrac{1}{n} - \dfrac{8}{n^2}}} = \dfrac{9}{2}$ 
	}
\end{ex}
\Closesolutionfile{ans}
% \begin{indapan}{10}
% 	{ans/ans-1K5-1-Dang2}
% \end{indapan}
\begin{dang}{Giới hạn vô cực}
	Ta nói dãy $\{u_n\}$ có giới hạn là $+ \infty$ khi $n \rightarrow + \infty$, nếu $u_n$ có thể lớn hơn một số dương bất kì, kể từ một số hạng nào đó trở đi. \\
	Kí hiệu: $\lim \limits_{n \to +\infty}u_n = + \infty$ hay $u_n \rightarrow + \infty$ khi $n \rightarrow + \infty$. \\ 
	Dãy số $\{u_n\}$ có giới hạn là $- \infty$ khi $n \rightarrow + \infty$, nếu $\lim \limits_{n \to +\infty}- u_n = + \infty$. \\
	Kí hiệu: $\lim \limits_{n \to +\infty}u_n = - \infty$ hay $u_n \rightarrow - \infty$ khi $n \rightarrow + \infty$. \\ 
	\textbf{Một số giới hạn đặc biệt và định lí về giới hạn dãy số} \\
	\textit{Giới hạn đặc biệt}: \\
	$\displaystyle \lim_{n \rightarrow + \infty} \sqrt{n} = + \infty$ \\
	$\displaystyle \lim_{n \rightarrow + \infty} n^k = + \infty$ với $k$ là số nguyên dương. \\
	$\displaystyle \lim_{n \rightarrow + \infty} q^n = + \infty$ nếu $q > 1$ \\
	\textit{Định lý}: \\
	Nếu $\lim \limits_{n \to +\infty}u_n = a > 0$ và $\lim \limits_{n \to +\infty}v_n = 0$ với $v_n > 0$ thì $\lim \limits_{n \to +\infty}\dfrac{u_n}{v_n} = + \infty$. \\
	Nếu $\lim \limits_{n \to +\infty}u_n = + \infty$ và $\lim \limits_{n \to +\infty}v_n = a > 0$ thì $\lim \limits_{n \to +\infty}u_nv_n = + \infty$.
\end{dang}
\subsubsection{Ví dụ minh hoạ}
\begin{vd}%[Dự án soạn đề cương toán 11 - KNTT, Minh Trí]%[1K5YE-4]
	Tìm giới hạn 
	\begin{enumEX}[a)]{2}
		\item[a)] $\lim \limits_{n \to +\infty}(n^3 + n^2 + n + 1)$.
		\item[b)] $\lim \limits_{n \to +\infty}\left(n^2 - n\sqrt{n} + 1\right)$.
	\end{enumEX}
	\loigiai{
		\begin{enumEX}[a)]{1}
			\item $\lim \limits_{n \to +\infty}(n^3 + n^2 + n + 1) = \lim \limits_{n \to +\infty}n^3\left(1 + \dfrac{1}{n} + \dfrac{1}{n^2} + \dfrac{1}{n^3}\right) = + \infty$.
			\item $\lim \limits_{n \to +\infty}\left(n^2 - n\sqrt{n} + 1\right) = \lim \limits_{n \to +\infty}n^2\left(1 - \dfrac{1}{\sqrt{n}} + \dfrac{1}{n^2}\right) = + \infty.$
		\end{enumEX}
	}
\end{vd}
\begin{vd}%[Dự án soạn đề cương toán 11 - KNTT, Minh Trí]%[1K5BE-4]
	Tìm giới hạn
	\begin{enumEX}[a)]{3}
		\item[a)] $\lim \limits_{n \to +\infty}\dfrac{n^5 + n^4 - n - 2}{4n^3 + 6n^2 + 9}$.
		\item[b)] $\lim \limits_{n \to +\infty}\dfrac{\sqrt[3]{n^6 - 7n^3 - 5n + 8}}{n + 12}$.
		\item[c)] $\lim \limits_{n \to +\infty}\left(n + \sqrt{n^2 - n + 1}\right)$.
	\end{enumEX}
	\loigiai{
		\begin{enumEX}[a)]{1}
			\item $\lim \limits_{n \to +\infty}\dfrac{n^5 + n^4 - n - 2}{4n^3 + 6n^2 + 9} = \lim \limits_{n \to +\infty}\dfrac{n^2 + n - \dfrac{1}{n^2} - \dfrac{2}{n^3}}{4 + \dfrac{6}{n} + \dfrac{9}{n^3}} = \lim \limits_{n \to +\infty}\dfrac{n^2 + n}{4} = + \infty$.
			\item $\lim \limits_{n \to +\infty}\dfrac{\sqrt[3]{n^6 - 7n^3 - 5n + 8}}{n + 12} = \lim \limits_{n \to +\infty}\dfrac{n^2\sqrt[3]{1 - \dfrac{7}{n^3} - \dfrac{5}{n^5} + \dfrac{8}{n^6}}}{n + 12} = \lim \limits_{n \to +\infty}\dfrac{n\sqrt[3]{1 - \dfrac{7}{n^3} - \dfrac{5}{n^5} + \dfrac{8}{n^6}}}{1 + \dfrac{12}{n}} = + \infty$.
			\item $\lim \limits_{n \to +\infty}\left(n + \sqrt{n^2 - n + 1}\right) = n\left(1 + \sqrt{1 - \dfrac{1}{n} + \dfrac{1}{n^2}}\right) = \lim \limits_{n \to +\infty}2n = + \infty$
		\end{enumEX}
	}
\end{vd}
\begin{vd}%[Dự án soạn đề cương toán 11 - KNTT, Minh Trí]%[1K5KE-4]
	Tìm giới hạn
	\begin{enumEX}[a)]{3}
		\item[a)] $\lim \limits_{n \to +\infty}\dfrac{1^3 + 2^3 + ... + n^3}{n^2 + 3n\sqrt{n} + 2}$.
		\item[b)] $\lim \limits_{n \to +\infty}\left(n + \sqrt[3]{n^3 - 2n + 1}\right)$.
		\item[c)] $\lim \limits_{n \to +\infty}\dfrac{n^3 - 3n}{2n + 15}$.
	\end{enumEX}
	\loigiai{
		\begin{enumEX}[a)]{1}
			\item $\lim \limits_{n \to +\infty}\dfrac{1^3 + 2^3 + ... + n^3}{n^2 + 3n\sqrt{n} + 2} = \lim \limits_{n \to +\infty}\dfrac{\dfrac{1}{4}n^2(n + 1)^2}{n^2 + 3n\sqrt{n} + 2} = \lim \limits_{n \to +\infty}\dfrac{\dfrac{1}{4}(n + 1)^2}{1 + \dfrac{3}{\sqrt{n}} + \dfrac{2}{n^2}} = \lim \limits_{n \to +\infty}\dfrac{1}{4}(n + 1)^2 = + \infty$.
			\item $\lim \limits_{n \to +\infty}\left(n + \sqrt[3]{n^3 - 2n + 1}\right) = n\left(1 + \sqrt[3]{1 - \dfrac{2}{n^2} + \dfrac{1}{n^3}}\right) = \lim \limits_{n \to +\infty}2n = + \infty$.
			\item $\lim \limits_{n \to +\infty}\dfrac{n^3 - 3n}{2n + 15} = \lim \limits_{n \to +\infty}\dfrac{n^2 - 3}{2 + \dfrac{15}{n}} = + \infty$
		\end{enumEX}
	}
\end{vd}
% \subsubsection{Bài tập rèn luyện} 
% \subsubsection{Bài tập tự luận}
% \begin{bt}%[Dự án soạn đề cương toán 11 - KNTT, Minh Trí]%[1K5BE-4]
% 	Tìm giới hạn
% 	\begin{enumEX}[a)]{2}
% 		\item[a)] $\lim \limits_{n \to +\infty}\sqrt{5n^2 - 8n + 7}$.
% 		\item[b)] $\lim \limits_{n \to +\infty}\sqrt{n^3 - 5n + 6}$.
% 	\end{enumEX}
% 	\loigiai{
% 		\begin{enumEX}[a)]{1}
% 			\item $\lim \limits_{n \to +\infty}\sqrt{5n^2 - 8n + 7} = \lim \limits_{n \to +\infty}n\sqrt{5 - \dfrac{8}{n} + \dfrac{7}{n^2}} = + \infty$.
% 			\item $\lim \limits_{n \to +\infty}\sqrt{n^3 - 5n + 6} = \lim \limits_{n \to +\infty}n\sqrt{n} \sqrt{1 - \dfrac{5}{n^2} + \dfrac{6}{n^3}} = + \infty$.
% 		\end{enumEX}
% 	}
% \end{bt}
% \begin{bt}%[Dự án soạn đề cương toán 11 - KNTT, Minh Trí]%[1K5KE-4]
% 	Tìm giới hạn
% 	\begin{enumEX}[a)]{2}
% 		\item[a)] $\lim \limits_{n \to +\infty}\dfrac{\sqrt{5n^4 - 8n^2 + 10}}{4n + 5}$.
% 		\item[b)] $\lim \limits_{n \to +\infty}\dfrac{n^2 - 15n + 11}{\sqrt{n^2 - 8n + 7}}$.
% 	\end{enumEX}
% 	\loigiai{
% 		\begin{enumEX}[a)]{1}
% 			\item $\lim \limits_{n \to +\infty}\dfrac{\sqrt{5n^4 - 8n^2 + 10}}{4n + 5} = \lim \limits_{n \to +\infty}\dfrac{n^2\sqrt{5 - \dfrac{8}{n^2} + \dfrac{10}{n^4}}}{4n + 5} = \dfrac{n\sqrt{5 - \dfrac{8}{n^2} + \dfrac{10}{n^4}}}{4 + \dfrac{5}{n}} = \lim \limits_{n \to +\infty}\dfrac{n\sqrt{5}}{4} = + \infty$.
% 			\item $\lim \limits_{n \to +\infty}\dfrac{n^2 - 15n + 11}{\sqrt{n^2 - 8n + 7}} = \lim \limits_{n \to +\infty}\dfrac{n - 15 + \dfrac{11}{n}}{\sqrt{1 - \dfrac{8}{n} + \dfrac{7}{n^2}}} = + \infty$.
% 		\end{enumEX}
% 	}
% \end{bt}
% \begin{bt}%[Dự án soạn đề cương toán 11 - KNTT, Minh Trí]%[1K5BE-4]
% 	Tìm $\lim \limits_{n \to +\infty}\left(\dfrac{1}{n^2}+\dfrac{2}{n^2}+\ldots+\dfrac{n}{n^2}\right)$.
% 	\loigiai{
% 		$$
% 		\lim \limits_{n \to +\infty}\left(\dfrac{1}{n^2}+\dfrac{2}{n^2}+\ldots+\dfrac{n}{n^2}\right)=\lim \limits_{n \to +\infty}\left(\dfrac{1+2+\ldots+n}{n^2}\right)=\lim \limits_{n \to +\infty}\left(\dfrac{n(n+1)}{2 n^2}\right)=\lim \limits_{n \to +\infty}\left(\dfrac{1+\dfrac{1}{n}}{2}\right)=\dfrac{1}{2} .
% 		$$}
% \end{bt}
% \begin{bt}%[Dự án soạn đề cương toán 11 - KNTT, Minh Trí]%[1K5GE-4]
% 	Tính giới hạn: $\lim \limits_{n \to +\infty}\left[\left(1-\dfrac{1}{2^2}\right)\left(1-\dfrac{1}{3^2}\right) \ldots\left(1-\dfrac{1}{n^2}\right)\right]$.\\
% 	Xét dãy số $\left(u_n\right)$, với $u_n=\left(1-\dfrac{1}{2^2}\right)\left(1-\dfrac{1}{3^2}\right) \ldots\left(1-\dfrac{1}{n^2}\right), n \geq 2, n \in \mathbb{N}$.
% 	\loigiai{
% 		Ta có:
% 		$$
% 		\begin{aligned}
% 			& u_2=1-\dfrac{1}{2^2}=\dfrac{3}{4}=\dfrac{2+1}{2 \cdot 2} \\
% 			& u_3=\left(1-\dfrac{1}{2^2}\right) \cdot\left(1-\dfrac{1}{3^2}\right)=\dfrac{3}{4} \cdot \dfrac{8}{9}=\dfrac{4}{6}=\dfrac{3+1}{2 \cdot 3} ; \\
% 			& u_4=\left(1-\dfrac{1}{2^2}\right) \cdot\left(1-\dfrac{1}{3^2}\right)\left(1-\dfrac{1}{4^2}\right)=\dfrac{3}{4} \cdot \dfrac{8}{9} \cdot \dfrac{15}{16}=\dfrac{5}{8}=\dfrac{4+1}{2 \cdot 4} \\
% 			& \ldots \ldots . \\
% 			& u_n=\dfrac{n+1}{2 n} .
% 		\end{aligned}
% 		$$
% 		Dễ dàng chứng minh bằng phương pháp qui nạp để khẳng định $u_n=\dfrac{n+1}{2 n}, \forall n \geq 2$
% 		\\ Khi đó $\lim \limits_{n \to +\infty}\left[\left(1-\dfrac{1}{2^2}\right)\left(1-\dfrac{1}{3^2}\right) \ldots\left(1-\dfrac{1}{n^2}\right)\right]=\lim \limits_{n \to +\infty}\dfrac{n+1}{2 n}=\dfrac{1}{2}$.}
% \end{bt}
% \begin{bt}%[Dự án soạn đề cương toán 11 - KNTT, Minh Trí]%[1K5BE-4]
% 	Cho dãy số $\left(u_n\right), n \in \mathbb{N}^*$, thỏa mãn điều kiện $\left\{\begin{array}{c}u_1=3 \\ u_{n+1}=-\dfrac{u_n}{5}\end{array}\right.$. Gọi $S=u_1+u_2+u_3+\ldots+u_n$ là tồng $n$ số hạng đầu tiên của dãy số đã cho. Khi đó lim $S_n$ bằng
% 	\loigiai{
% 		Ta có $\dfrac{u_{n+1}}{u_n}=\dfrac{-\dfrac{u_n}{5}}{u_n}=-\dfrac{1}{5}$ do đó dãy $\left(u_n\right), n \in \mathbb{N}^*$ là một cấp số nhân lùi vô hạn có $u_1=3, d=-\dfrac{1}{5}$.
% 		Suy ra $\lim \limits_{n \to +\infty}S_n=\dfrac{u_1}{1-q}=\dfrac{3}{1+\dfrac{1}{5}}=\dfrac{5}{2}$.}
% \end{bt}
% \begin{bt}%[Dự án soạn đề cương toán 11 - KNTT, Minh Trí]%[1K5TE-4]
% 	Trong một lần Đoàn trường Lê Văn Hưu tổ chức chơi bóng chuyền hơi, bạn Nam thả một quả bóng chuyền hơi từ tầng ba, độ cao $8 m$ so với mặt đất và thấy rằng mỗi lần chạm đất thì quả bóng lại nảy lên một độ cao bằng ba phần tư độ cao lần rơi trước. Biết quả bóng chuyển động vuông góc với mặt đất. Khi đó tổng quãng đường quả bóng đã bay từ lúc thả bóng đến khi quả bóng không nảy nữa bằng bao nhiêu ?
% 	\loigiai{
% 		Lần đầu rơi xuống, quảng đường quả bóng đã bay đến lúc chạm đất là $8 m$.\\
% 		Sau đó quả bóng nảy lên và rơi xuống chạm đất lần thứ 2 thì quảng đường quả bóng đã bay là $8+2.8 \cdot \dfrac{3}{4}$.\\
% 		Tương tự, khi quả bóng nảy lên và rơi xuống chạm đất lần thứ $n$ thì quảng đường quả bóng đã bay là $8+2\cdot 8 \cdot \dfrac{3}{4}+\ldots \ldots +2.8 \cdot\left(\dfrac{3}{4}\right)^{n-1}=8+\dfrac{1-\left(\dfrac{3}{4}\right)^n}{1-\dfrac{3}{4}}=8+48\left(1-\left(\dfrac{3}{4}\right)^{n-1}\right)$.\\
% 		Quảng đường quả bóng đã bay từ lúc thả đến lúc không máy nữa bằng: $\lim \limits_{n \to +\infty}\left[8+48\left(1-\left(\dfrac{3}{4}\right)^{n-1}\right)\right]=8+48=56$.}
% \end{bt}
% \begin{bt}%[Dự án soạn đề cương toán 11 - KNTT, Minh Trí]%[1K5GE-4]
% 	Cho hình vuông $ABCD$ có cạnh bằng $a$. Người ta dựng hình vuông $A_1B_1C_1D_1$ có cạnh bằng $\dfrac{1}{2}$ đường chéo của hình vuông $ABCD$; dựng hình vuông $A_2 B_2 C_2 D_2$ có cạnh bằng $\dfrac{1}{2}$ đường chéo của hình vuông $A_1B_1C_1D_1$ và cứ tiếp tục như vậy (tham khảo hình vẽ).
% 	Giả sử cách dựng trên có thể tiến ra vô hạn. Nếu tổng diện tích $S$ của tất cả các hình vuông $ABCD$, $A_1B_1C_1D_1$, $A_2B_2C_2D_2$, $\ldots$ bằng $8$ thì $a$ bằng bao nhiêu? 
% 	\begin{center}
% 		\begin{tikzpicture}[scale=1, font=\footnotesize, line join=round, line cap=round, >=stealth]
% 			\coordinate (A) at (0,0);
% 			\coordinate (B) at (6,0);
% 			\coordinate (D) at (0,6);
% 			\coordinate (C) at ($(B)+(D)-(A)$);
% 			\draw (A)--(B)--(C)--(D)--(A);
			
% 			\coordinate (A_1) at ($(A)!1/2!(B)$);
% 			\coordinate (B_1) at ($(C)!1/2!(B)$);
% 			\coordinate (C_1) at ($(C)!1/2!(D)$);
% 			\coordinate (D_1) at ($(A)!1/2!(D)$);
% 			\draw (A_1)--(B_1)--(C_1)--(D_1)--(A_1);
% 			\foreach \x in {2,3,4,5,6,7}
% 			{ \pgfmathsetmacro{\j}{\x-1}
% 				\coordinate (A_\x) at ($(A_\j)!1/2!(D_\j)$);
% 				\coordinate (B_\x) at ($(A_\j)!1/2!(B_\j)$);
% 				\coordinate (C_\x) at ($(C_\j)!1/2!(B_\j)$);
% 				\coordinate (D_\x) at ($(C_\j)!1/2!(D_\j)$);
% 				\draw (A_\x)--(B_\x)--(C_\x)--(D_\x)--(A_\x);
% 			}
% 			\foreach \x/\y in {A/225,B/-45,C/45,D/135,A_1/-90,B_1/0,C_1/90,D_1/180}{\fill (\x) circle (1pt) ($(\x)+(\y:0.3cm)$) node{$\x$};}
% 			\def \goc{-90};
			
% 			\foreach \t in {2,3,4,5}
% 			{\pgfmathsetmacro{\goca}{\goc - \t*45+45}
% 				\pgfmathsetmacro{\gocb}{\goca + 90}
% 				\pgfmathsetmacro{\gocc}{\gocb + 90}
% 				\pgfmathsetmacro{\gocd}{\gocc + 90}
% 				\foreach \x/\y in {A_\t/\goca,B_\t/\gocb,C_\t/\gocc,D_\t/\gocd}{\fill (\x) circle (1pt) ($(\x)+(\y:0.3cm)$) node{$\x$};}		
% 			}
			
% 		\end{tikzpicture}
% 	\end{center}
% 	\loigiai{
% 		$$
% 		\begin{aligned}
% 			& \text { Ta có } S_{A B C D}=a^2 ; S_{A_1 B_1 C_1 D_1}=\left(\dfrac{a \sqrt{2}}{2}\right)^2=\dfrac{a^2}{2} ; S_{A_2 B_2 C_2 D_2}=\left(\dfrac{a}{2}\right)^2=\dfrac{a^2}{4}=\dfrac{a^2}{2^2} \\
% 			& S=S_{A B C D}+S_{A_1 B_1 C_1 D_1}+S_{A_2 B_2 C_2 D_2}+\ldots=a^2+\dfrac{a^2}{2}+\dfrac{a^2}{2^2}+\ldots=a^2\left(1+\dfrac{1}{2}+\dfrac{1}{2^2}+\ldots\right)=a^2 \cdot \dfrac{1}{1-\dfrac{1}{2}}=2 a^2
% 		\end{aligned}
% 		$$
% 	}
% \end{bt}
% \begin{bt}%[Dự án soạn đề cương toán 11 - KNTT, Minh Trí]%[1K5GE-4]
% 	Cho hình vuông $C_1$ có cạnh bằng $a$. Người ta chia mỗi cạnh của hình vuông thành bốn phần bằng nhau và nối các điểm chia một cách thích hợp để có hình vuông $C_2$ (tham khảo hình vẽ).
% 	Từ hình vuông $C_2$ lại tiếp tục làm như trên ta nhận được dãy các hình vuông $C_1, C_2, C_3, \ldots, C_n, \ldots$.Gọi $S_i$ là diện tích của hình vuông $C_i(i \in\{1 ; 2 ; 3 ; \ldots\})$. Tính tổng $S=S_1+S_2+S_3+\ldots+S_n+\ldots$\\
% 	\begin{center}
% 		\begin{tikzpicture}[scale=1, font=\footnotesize, line join=round, line cap=round, >=stealth]
% 			\coordinate (A1) at (0,0);
% 			\coordinate (B1) at (4,0);
% 			\coordinate (D1) at (0,4);
% 			\coordinate (C1) at ($(B1)+(D1)-(A1)$);
% 			\draw (A1)--(B1)--(C1)--(D1)--(A1);
			
% 			\foreach \x in {2,3,4}
% 			{ \pgfmathsetmacro{\j}{\x-1}
% 				\coordinate (A\x) at ($(A\j)!1/4!(B\j)$);
% 				\coordinate (B\x) at ($(B\j)!1/4!(C\j)$);
% 				\coordinate (C\x) at ($(C\j)!1/4!(D\j)$);
% 				\coordinate (D\x) at ($(D\j)!1/4!(A\j)$);
% 				\draw (A\x)--(B\x)--(C\x)--(D\x)--(A\x);
% 			}
% 		\end{tikzpicture}
% 	\end{center}
% 	\loigiai{
% 		Ta có $S_1=a^2, S_2=\dfrac{5}{8} a^2, S_3=\dfrac{25}{64} a^2, \ldots$
% 		Nên $S=S_1+S_2+S_3+\ldots+S_n+\ldots$ là tổng của cấp số nhân lùi vô hạn với $\left\{\begin{array}{l}u_1=a^2 \\ q=\dfrac{5}{8}\end{array}\right.$.
% 		Khi đó $S=\dfrac{u_1}{1-q}=\dfrac{a^2}{1-\dfrac{5}{8}}=\dfrac{8}{3} a^2$.}
% \end{bt}
% \subsubsection{Câu hỏi trắc nghiệm}
% \Opensolutionfile{ans}[ans/ans-1K5-1-Dang3]
% \begin{ex}%[Dự án soạn đề cương toán 11 - KNTT, Minh Trí]%[1K5KE-4]
% 	Giá trị của giới hạn $\lim \limits_{n \to +\infty}\left(1+\dfrac{1}{2}+\dfrac{1}{2^2}+\ldots+\dfrac{1}{2^n}\right)$ là
% 	\choice
% 	{$1$}
% 	{\True $2$} 
% 	{$\dfrac{1}{2}$}
% 	{$\dfrac{3}{2}$}
% 	\loigiai{
% 		Ta có: $\lim \limits_{n \to +\infty}\left(1+\dfrac{1}{2}+\dfrac{1}{2^2}+\ldots+\dfrac{1}{2^n}\right)=\dfrac{1}{1-\dfrac{1}{2}}=2$.}
% \end{ex}
% \begin{ex}%[Dự án soạn đề cương toán 11 - KNTT, Minh Trí]%[1K5BE-4]
% 	Tính giới hạn $I=\lim \limits_{n \to +\infty}\dfrac{5\cdot4^{n+1}+3^{n+2}}{2^{2 n+1}+1}$.
% 	\choice
% 	{$I=+\infty$}
% 	{\True $I=10$}
% 	{$I=0$}
% 	{$I=20$}
% 	\loigiai{
% 		Ta có $I=\lim \limits_{n \to +\infty}\dfrac{5.4^{n+1}+3^{n+2}}{2^{2 n+1}+1}=\lim \limits_{n \to +\infty}\dfrac{20.4^n+9.3^n}{2 \cdot 4^n+1}=\lim \limits_{n \to +\infty}\dfrac{20+9 \cdot\left(\dfrac{3}{4}\right)^n}{2+\left(\dfrac{1}{4}\right)^n}=\dfrac{20}{2}=10$.}
% \end{ex}

% \begin{ex}%[Dự án soạn đề cương toán 11 - KNTT, Minh Trí]%[1K5BE-4]
% 	Tính tồng $S=1+\dfrac{1}{2}+\dfrac{1}{4}+\dfrac{1}{8}+\cdots+\dfrac{1}{2^n}+\cdots$
% 	\choice
% 	{\True $2$}
% 	{$3$}
% 	{$1$}
% 	{$\dfrac{1}{2}$}
% 	\loigiai{
% 		$S=1+\dfrac{1}{2}+\dfrac{1}{4}+\dfrac{1}{8}+\ldots+\dfrac{1}{2^n}+\ldots=\dfrac{u_1}{1-q}=\dfrac{1}{1-\dfrac{1}{2}}=2$.}
% \end{ex}
% \begin{ex}%[Dự án soạn đề cương toán 11 - KNTT, Minh Trí]%[1K5BE-4]
% 	Tính $\lim \limits_{n \to +\infty}\dfrac{3 n^3-2}{1-2 n^3}$ được kết quả là
% 	\choice
% 	{$\dfrac{3}{2}$}
% 	{\True $-\dfrac{3}{2}$}
% 	{$\dfrac{1}{2}$}
% 	{$\dfrac{-1}{2}$}
% 	\loigiai{
% 		Ta có: $\lim \limits_{n \to +\infty}\dfrac{3 n^3-2}{1-2 n^3}=\lim \limits_{n \to +\infty}\dfrac{n^3\left(3-\dfrac{2}{n^3}\right)}{n^3\left(\dfrac{1}{n^3}-2\right)}=\lim \limits_{n \to +\infty}\dfrac{3-\dfrac{2}{n^3}}{\dfrac{1}{n^3}-2}=\dfrac{3-0}{0-2}=-\dfrac{3}{2}$.}
	
% \end{ex}
% \begin{ex}%[Dự án soạn đề cương toán 11 - KNTT, Minh Trí]%[1K5GE-4]
% 	Cho các số $a, b, c \in R ; b+c=5 ; \lim\limits_{x \rightarrow+\infty}\left(\sqrt{a x^2+b x}-c x\right)=2$. Tính $P=a+2 b+c$
% 	\choice
% 	{$P=12$}
% 	{$P=15$}
% 	{\True $P=10$}
% 	{$P=5$}
% 	\loigiai{
% 		Ta có: Biện luận \\
% 		+ Điều kiện cần để tồn tại giới hạn đã cho là $a>0$\\
% 		+ Nếu $c \leq 0 \Rightarrow \lim\limits_{x \rightarrow+\infty}\left(\sqrt{a x^2+b x}-c x\right)=+\infty$ (loại) \\
% 		+ Nếu $c>0$\\
% 		$2=\lim\limits_{x \rightarrow+\infty}\left(\sqrt{a x^2+b x}-c x\right)=\lim\limits_{x \rightarrow+\infty} \dfrac{\left(\sqrt{a x^2+b x}-c x\right)\left(\sqrt{a x^2+b x}+c x\right)}{\sqrt{a x^2+b x}+c x}=\lim\limits_{x \rightarrow+\infty} \dfrac{\left(a-c^2\right) x^2+b x}{\sqrt{a x^2+b x}+c x}$ là hữu hạn nên: $a-c^2=0 \Leftrightarrow a=c^2$ (1)\\
% 		Khi đó: $2=\lim\limits_{x \rightarrow+\infty} \dfrac{b x}{\sqrt{a x^2+b x}+c x}=\lim\limits_{x \rightarrow+\infty} \dfrac{b}{\sqrt{a+\dfrac{b}{x}}+c}=\dfrac{b}{\sqrt{a}+c} \Leftrightarrow 2(\sqrt{a}+c)=b$\\
% 		Từ ta có hệ: $\left\{\begin{array}{l}a=c^2 \\ 2(\sqrt{a}+c)=b \\ b+c=5 \\ a, c>0\end{array} \Leftrightarrow\left\{\begin{array}{l}a=c^2 \\ 4 c=b \\ b+c=5 \\ a, c>0\end{array} \Leftrightarrow\left\{\begin{array}{l}a=1 \\ b=4 \\ c=1\end{array} \Rightarrow P=a+2 b+c=10\right.\right.\right.$
% 	}
% \end{ex} 
% \begin{ex}%[Dự án soạn đề cương toán 11 - KNTT, Minh Trí]%[1K5KE-4]
% 	Tính $I=\lim\limits_{x \rightarrow+\infty} \dfrac{\sqrt{x^2+x+1}-x}{3}=\dfrac{a}{b} ; a, b \in \mathbb{N}$ và $\dfrac{a}{b}$ là phân số tối giản. Khi đó $2 a-b$ bằng kết quả nào sau đây?
% 	\choice
% 	{$4$}
% 	{\True $-4$}
% 	{$-5$}
% 	{$5$}
% 	\loigiai{
% 		Ta có, $\lim\limits_{x \rightarrow+\infty} \dfrac{\sqrt{x^2+x+1}-x}{3}=\lim\limits_{x \rightarrow+\infty} \dfrac{\left(\sqrt{x^2+x+1}-x\right)\left(\sqrt{x^2+x+1}+x\right)}{3\left(\sqrt{x^2+x+1}+x\right)}$
% 		$$
% 		=\lim\limits_{x \rightarrow+\infty} \dfrac{\left(x^2+x+1\right)-x^2}{3\left(\sqrt{x^2+x+1}+x\right)}=\lim\limits_{x \rightarrow+\infty} \dfrac{x+1}{3\left(\sqrt{x^2+x+1}+x\right)}=\lim\limits_{x \rightarrow+\infty} \dfrac{1+\dfrac{1}{x}}{3\left(\sqrt{1+\dfrac{1}{x}+\dfrac{1}{x^2}}+1\right)}=\dfrac{1}{6}
% 		$$
% 		Khi đó, $a=1 ; b=6$. Vậy $2 a-b=-4$}
% \end{ex}
% \begin{ex}%[Dự án soạn đề cương toán 11 - KNTT, Minh Trí]%[1K5GE-4]
% 	Biết lim $\dfrac{\sqrt{n^2-4 n}-\sqrt{4 n^2+1}}{\sqrt{3 n^2+1}-n}=\dfrac{6-\sqrt{3}}{2}-\dfrac{a}{b}$, trong đó $\dfrac{a}{b}$ là phân số tối giản, $a$ và $b$ là các số nguyên dương. Chọn khẳng định đúng trong các khẳng định sau:
% 	\choice
% 	{ $a=b$}
% 	{ $a+b=7$}
% 	{\True $a+b=14$}
% 	{$\dfrac{b}{a}=\dfrac{7}{2}$}
% 	\loigiai{
		
% 		$$
% 		\begin{aligned}
% 			& \lim \limits_{n \to +\infty}\dfrac{\sqrt{n^2-4 n}-\sqrt{4 n^2+1}}{\sqrt{3 n^2+1}-n}=\lim \limits_{n \to +\infty}\dfrac{\sqrt{1-\dfrac{4}{n}}-\sqrt{4+\dfrac{1}{n^2}}}{\sqrt{3+\dfrac{1}{n^2}}-1}=\dfrac{-1-\sqrt{3}}{2}=\dfrac{6-\sqrt{3}}{2}-\dfrac{7}{2} . \\
% 			& \text { Suy ra } \dfrac{a}{b}=\dfrac{7}{2} \Rightarrow a=7 ; b=2 \Rightarrow a . b=14 .
% 		\end{aligned}
% 		$$
% 	}
% \end{ex}
% \begin{ex}%[Dự án soạn đề cương toán 11 - KNTT, Minh Trí]%[1K5KE-4]
% 	Tìm giới hạn $I=\lim\limits_{x \rightarrow+\infty}\left(x+1-\sqrt{x^2-x+2}\right)$.
% 	\choice
% 	{$I=\dfrac{46}{31}$}
% 	{$I=\dfrac{17}{11}$}
% 	{\True $I=\dfrac{3}{2}$}
% 	{$I=\dfrac{1}{2}$}
% 	\loigiai{
% 		$$
% 		\text { Ta có } I=\lim\limits_{x \rightarrow+\infty}\left(x+1-\sqrt{x^2-x+2}\right)=\lim\limits_{x \rightarrow+\infty} \dfrac{3 x-1}{x+1+\sqrt{x^2-x+2}}=\lim\limits_{x \rightarrow+\infty} \dfrac{3-\dfrac{1}{x}}{1+\dfrac{1}{x}+\sqrt{1-\dfrac{1}{x}+\dfrac{2}{x^2}}}=\dfrac{3}{2} \text {. }
% 		$$
% 	}
% \end{ex}
% \begin{ex}%[Dự án soạn đề cương toán 11 - KNTT, Minh Trí]%[1K5BE-4]
% 	Cho $a$ là một số thực khác $0$ thỏa mãn $\lim\limits_{x \rightarrow a} \dfrac{x^4-a}{x-a}=4$.
% 	Khi đó $a$ bằng
% 	\choice
% 	{$4$}
% 	{$-1$}
% 	{\True $1$}
% 	{$-4$}
% 	\loigiai{
% 		Ta có
% 		$$
% 		\lim\limits_{x \rightarrow a} \dfrac{x^4-a}{x-a}=\lim\limits_{x \rightarrow a} \dfrac{(x-a)(x+a)\left(x^2+a^2\right)}{x-a}=\lim\limits_{x \rightarrow a}\left[(x+a)\left(x^2+a^2\right)\right]=4 a^3
% 		$$
% 		Mà theo giả thiết $\lim\limits_{x \rightarrow a} \dfrac{x^4-a}{x-a}=4$. Do đó $4 a^3=4 \Leftrightarrow a=1$.}
% \end{ex}
% \begin{ex}%[Dự án soạn đề cương toán 11 - KNTT, Minh Trí]%[1K5KE-4]
% 	Cho $a, b, c$ là các số thực khác 0 . Tìm hệ thức liên hệ giữa $a, b, c$ để $\lim\limits_{x \rightarrow-\infty} \dfrac{a x-b \sqrt{9 x^2+2}}{c x+1}=5$.
% 	\choice
% 	{$\dfrac{a-3 b}{c}=5$}
% 	{$\dfrac{a+3 b}{c}=-5$}
% 	{$\dfrac{a-3 b}{c}=-5$}
% 	{\True $\dfrac{a+3 b}{c}=5$}
% 	\loigiai{
% 		Ta có: $\lim\limits_{x \rightarrow-\infty} \dfrac{a x-b \sqrt{9 x^2+2}}{c x+1}=5 \Leftrightarrow \lim\limits_{x \rightarrow-\infty} \dfrac{a x-b|x| \sqrt{9+\dfrac{2}{x^2}}}{c x+1}=5 \Leftrightarrow \lim\limits_{x \rightarrow-\infty} \dfrac{a+b \sqrt{9+\dfrac{2}{x^2}}}{c+\dfrac{1}{x}}=5$ $\Leftrightarrow \dfrac{a+b \sqrt{9+0}}{c+0}=5 \Leftrightarrow \dfrac{a+3 b}{c}=5$.}
% \end{ex}
% \begin{ex}%[Dự án soạn đề cương toán 11 - KNTT, Minh Trí]%[1K5KE-4]
% 	Tính $\lim \limits_{n \to +\infty}\left(\dfrac{1}{n^2+4}+\dfrac{2}{n^2+4}+\dfrac{3}{n^2+4}+\ldots+\dfrac{2n+4}{n^2+4}\right)$.
% 	\choice
% 	{$\dfrac{1}{2}$}
% 	{$0$}
% 	{$1$}
% 	{\True $2$}
% 	\loigiai{
% 		Ta có $\lim \limits_{n \to +\infty}\left(\dfrac{1}{n^2+4}+\dfrac{2}{n^2+4}+\dfrac{3}{n^2+4}+\ldots+\dfrac{2 n+4}{n^2+4}\right)$
% 		$$
% 		\begin{aligned}
% 			& =\lim \limits_{n \to +\infty}\left(\dfrac{1+2+3+\ldots+2 n+4}{n^2+4}\right) \\
% 			& =\lim \limits_{n \to +\infty}\dfrac{(1+2 n+4)(2 n+4)}{2\left(n^2+4\right)} \\
% 			& =\lim \limits_{n \to +\infty}\dfrac{(2 n+5)(2 n+4)}{2\left(n^2+4\right)} \\
% 			& =\lim \limits_{n \to +\infty}\dfrac{\left(2+\dfrac{5}{n}\right)\left(2+\dfrac{4}{n}\right)}{2\left(1+\dfrac{4}{n^2}\right)}=2
% 		\end{aligned}
% 		$$
% 	}
% \end{ex}

% \begin{ex}%[Dự án soạn đề cương toán 11 - KNTT, Minh Trí]%[1K5GE-4]
% 	Gọi $S_1$ là diện tích tam giác đều $A_1 B_1 C_1$ cạnh bằng $a$. Gọi $S_2$ là diện tích tam giác $A_2 B_2 C_2$ vói các đỉnh trung điểm các cạnh $A_1 B_1, B_1 C_1, A_1 C_1$, gọi $S_3$ là diện tích tam giác $A_3 B_3 C_3$ với các định trung điểm các cạnh $A_2 B_2, B_2 C_2, A_2 C_2, \ldots$ và gọi $S_n$ là diện tích tam giác $A_n B_n C_n$ với các đính trung điểm các cạnh $A_{n-1} B_{n-1}, B_{n-1} C_{n-1}, A_{n-1} C_{n-1}$. Khi $n$ tiến về dương vô cực tính tổng $S=S_1+S_2+S_3+\ldots+S_n+\ldots$
% 	\begin{center}
% 		\begin{tikzpicture}[scale=1, font=\footnotesize, line join=round, line cap=round, >=stealth]
% 			\coordinate (A_1) at (90:4);
% 			\coordinate (B_1) at (210:4);
% 			\coordinate (C_1) at (-30:4);
% 			\draw (A_1)--(B_1)--(C_1)--(A_1);
% 			\def \goc{-90};
% 			\foreach \t in {2,3,4}
% 			{ \pgfmathsetmacro{\j}{\t-1}
% 				\coordinate (A_\t) at ($(C_\j)!1/2!(B_\j)$);
% 				\coordinate (B_\t) at ($(A_\j)!1/2!(C_\j)$);
% 				\coordinate (C_\t) at ($(A_\j)!1/2!(B_\j)$);
% 				\draw (A_\t)--(B_\t)--(C_\t)--(A_\t);
% 			}
% 			\foreach \t in {1,2,3,4}
% 			{\pgfmathsetmacro{\goca}{\goc + \t*180}
% 				\pgfmathsetmacro{\gocb}{\goca + 120}
% 				\pgfmathsetmacro{\gocc}{\gocb + 120}
% 				\foreach \x/\y in {A_\t/\goca,B_\t/\gocb,C_\t/\gocc}{\fill (\x) circle (1pt) ($(\x)+(\y:0.3cm)$) node{$\x$};}
% 			}
% 		\end{tikzpicture}
% 	\end{center}
% 	\choice
% 	{$S=\dfrac{4 \sqrt{3} a^2}{3}$}
% 	{$S=\dfrac{\sqrt{3} a^2}{4}$}
% 	{\True $S=\dfrac{\sqrt{3} a^2}{3}$}
% 	{$S=\dfrac{\sqrt{3} a^2}{2}$}	
% 	\loigiai{
% 		Ta có: $S_1=\dfrac{\sqrt{3}}{4}(a)^2, S_2=\dfrac{\sqrt{3}}{4}\left(\dfrac{a}{2}\right)^2=\dfrac{\sqrt{3} a^2}{16}, S_3=\dfrac{\sqrt{3}}{4}\left(\dfrac{a}{4}\right)^2=\dfrac{\sqrt{3} a^2}{64}, \ldots S_n=\dfrac{\sqrt{3}}{4}\left(\dfrac{a}{2^{x-1}}\right)^2=\dfrac{\sqrt{3} a^2}{4^{x-1}}$. \\
% 		Khi $n \rightarrow+\infty \Rightarrow \dfrac{1}{4^{n-1}} \rightarrow 0 \Rightarrow S_n \rightarrow 0$. Lúc đó: \\ $S=S_1+S_2+S_3+\ldots+S_n+\ldots$ là tồng cấp số nhân lùi
% 		vô hạn với $S_1=\dfrac{\sqrt{3}}{4}(a)^2$ và công bội $q=\dfrac{1}{4}$. Vậy tổng diện tích các hình là $S=S_1 \cdot \dfrac{1}{1-q}=\dfrac{\sqrt{3}}{4}(a)^2 \cdot \dfrac{4}{3}=\dfrac{\sqrt{3} a^2}{3}$.}
% \end{ex}
% \begin{ex}%[Dự án soạn đề cương toán 11 - KNTT, Minh Trí]%[1K5TE-4]
% 	Một quả bóng tenis được thả từ độ cao $81(m)$. Mỗi lần chạm đất, quả bóng lại nảy lên hai phần ba độ cao của lần rơi trước. Tính tổng các khoảng cách rơi và nảy của quả bóng từ lúc thả bóng cho đến lúc bóng không nảy nữa.
% 	\choice
% 	{$243(m)$}
% 	{\True $405(\mathrm{~m})$}
% 	{$486(\mathrm{~m})$}
% 	{$524(m)$}
% 	\loigiai{
% 		Đặt $h_1=81(m)$. Sau lần chạm đất đầu tiên, quả bóng nảy lên một độ cao $h_2=\dfrac{2}{3} h_1$. Tiếp đó, bóng roi từ độ cao $h_2$, chạm đất và nảy lên độ cao $h_3=\dfrac{2}{3} h_2$ rồi roi từ độ cao $h_3$ và cứ tiếp tụ như vậy. Sau lần chạm đất thứ $n$ từ độ cao $h_{n'}$ quả bóng nảy lên $h_{n+1}=\dfrac{2}{3} h_{n'}, \ldots$ \\
% 		Vậy tổng các khoảng cách rơi và nảy của quả bóng từ lúc thả bóng cho đến lúc bóng không nảy nữa là $d=\left(h_1+h_2+\ldots+h_n+\ldots\right)+\left(h_2+\ldots+h_n+\ldots\right) \Rightarrow d$ là tổng của hai cấp số nhân lùi vo hạn có số hạng đầu, theo thứ tự là $h_1, h_2$ và có cùng công bội $q=\dfrac{2}{3}$. Suy ra: $d=\dfrac{h_1}{1-\dfrac{2}{3}}+\dfrac{h_2}{1-\dfrac{2}{3}}=405(m)$.}
% \end{ex}
% \begin{ex}%[Dự án soạn đề cương toán 11 - KNTT, Minh Trí]%[1K5TE-4]
% 	Để trang trí cho một tấm bìa hình vuông có cạnh bằng $1$m, bạn $\mathrm{A}$ quyết định vẽ các hình vuông lên tấm bìa bằng cách: hình vuông thứ nhất có các đỉnh là trung điểm của các cạnh tấm bìa, hình vuông thứ hai có các đỉnh là trung điểm của các cạnh hình vuông thứ nhất,hình vuông thứ ba có các đỉnh là trung điểm của các cạnh hình vuông thứ hai,... Giả sử quy trình vẽ hình vuông của bạn $A$ có thể tiến ra vô hạn. Tính độ dài $L$ các nét vẽ hình vuông của bạn $A$.
% 	\begin{center}
% 		\begin{tikzpicture}[scale=0.8, font=\footnotesize, line join=round, line cap=round, >=stealth]
% 			\coordinate (A) at (0,0);
% 			\coordinate (B) at (6,0);
% 			\coordinate (D) at (0,6);
% 			\coordinate (C) at ($(B)+(D)-(A)$);
% 			\draw (A)--(B)--(C)--(D)--(A);
			
% 			\coordinate (A_1) at ($(A)!1/2!(B)$);
% 			\coordinate (B_1) at ($(C)!1/2!(B)$);
% 			\coordinate (C_1) at ($(C)!1/2!(D)$);
% 			\coordinate (D_1) at ($(A)!1/2!(D)$);
% 			\draw (A_1)--(B_1)--(C_1)--(D_1)--(A_1);
% 			\foreach \x in {2,3,4,5,6,7}
% 			{ \pgfmathsetmacro{\j}{\x-1}
% 				\coordinate (A_\x) at ($(A_\j)!1/2!(D_\j)$);
% 				\coordinate (B_\x) at ($(A_\j)!1/2!(B_\j)$);
% 				\coordinate (C_\x) at ($(C_\j)!1/2!(B_\j)$);
% 				\coordinate (D_\x) at ($(C_\j)!1/2!(D_\j)$);
% 				\draw (A_\x)--(B_\x)--(C_\x)--(D_\x)--(A_\x);
% 			}			
% 		\end{tikzpicture}
% 	\end{center}
% 	\choice
% 	{$1+\sqrt{2}$}
% 	{$2+\sqrt{2}$}
% 	{\True $4+4 \sqrt{2}$}
% 	{$8+4 \sqrt{2}$}
% 	\loigiai{
% 		Hình vuông thứ nhất có cạnh là $\dfrac{1}{2} \cdot \sqrt{2}=\dfrac{\sqrt{2}}{2}$ nên có chu vi $S_1=2 \sqrt{2}$,\\
% 		Hình vuông thứ hai có cạnh là $\dfrac{\sqrt{2}}{4} \cdot \sqrt{2}=\dfrac{1}{2}$ nên có chu vi $S_2=2$,\\
% 		Hình vuông thứ ba có cạnh là $\dfrac{1}{4} \cdot \sqrt{2}=\dfrac{\sqrt{2}}{4}$ nên có chu vi $S_3=\sqrt{2}$,\\
% 		Hình vuông thứ $n$ có cạnh là $\left(\dfrac{\sqrt{2}}{2}\right)^n$ nên có chu vi $S_n=4 .\left(\dfrac{\sqrt{2}}{2}\right)^n, \ldots$\\
% 		Khi đó độ dài các nét vẽ cạnh hình vuông là $S=S_1+S_2+\ldots+S_n+\ldots=\dfrac{2 \sqrt{2}}{1-\dfrac{\sqrt{2}}{2}}=4+4 \sqrt{2}$.}
% \end{ex}
% \begin{ex}%[Dự án soạn đề cương toán 11 - KNTT, Minh Trí]%[1K5GE-4]
% 	Tính giới hạn của dãy số $u_n=\dfrac{1}{2 \sqrt{1}+\sqrt{2}}+\dfrac{1}{3 \sqrt{2}+2 \sqrt{3}}+\ldots+\dfrac{1}{(n+1) \sqrt{n}+n \sqrt{n+1}}$ :
% 	\choice
% 	{$+\infty$}
% 	{$-\infty$}
% 	{$0$}
% 	{\True $1$}
% 	\loigiai{
% 		Ta có $: \dfrac{1}{(k+1) \sqrt{k}+k \sqrt{k+1}}=\dfrac{1}{\sqrt{k}}-\dfrac{1}{\sqrt{k+1}}$
% 		Suy ra $u_n=1-\dfrac{1}{\sqrt{n+1}} \Rightarrow \lim \limits_{n \to +\infty}u_n=1$}
% \end{ex}

% \begin{ex}%[Dự án soạn đề cương toán 11 - KNTT, Minh Trí]%[1K5GE-4]
% 	Với $n$ là số tự nhiên lớn hơn $2$ , đặt $S_n=\dfrac{1}{\mathrm{C}_3^3}+\dfrac{1}{\mathrm{C}_4^3}+\dfrac{1}{\mathrm{C}_5^3}+\ldots+\dfrac{1}{\mathrm{C}_n^3}$. Tính $\lim \limits_{n \to +\infty}S_n$
% 	\choice
% 	{$1$} 
% 	{$\dfrac{3}{2}$}
% 	{$3$}
% 	{$\dfrac{1}{3}$}
% 	\loigiai{
% 		$$
% 		\begin{aligned}
% 			& S_n=\dfrac{1}{\mathrm{C}_3^3}+\dfrac{1}{\mathrm{C}_4^3}+\dfrac{1}{\mathrm{C}_5^3}+\ldots+\dfrac{1}{\mathrm{C}_n^3}=\dfrac{3!}{1\cdot2\cdot3}+\dfrac{3 !}{2\cdot3\cdot 4}+\dfrac{3!}{3\cdot 4\cdot 5}+\ldots+\dfrac{3!}{n(n-1)(n-2)} \\
% 			& =6\left[\dfrac{1}{2}\left(-\dfrac{1}{3.2}+\dfrac{1}{2.1}-\dfrac{1}{4.3}+\dfrac{1}{3\cdot 2}+\ldots-\dfrac{1}{n(n-1)}+\dfrac{1}{(n-1)(n-2)}\right)\right]=3\left(\dfrac{1}{2\cdot 1}-\dfrac{1}{n(n-1)}\right)
% 		\end{aligned}
% 		$$
% 		Vậy $\lim \limits_{n \to +\infty}S_n=\lim \limits_{n \to +\infty}3\left(\dfrac{1}{2}-\dfrac{1}{n(n-1)}\right)=\dfrac{3}{2}$}
% \end{ex} 
% \begin{ex}%[Dự án soạn đề cương toán 11 - KNTT, Minh Trí]%[1K5GE-4]
% 	Cho $f(n)=\left(n^2+n+1\right)^2+1$. Xét dãy số $\left(u_n\right)$ sao cho $u_n=\dfrac{f(1) \cdot f(3) \cdot f(5) \ldots f(2 n-1)}{f(2) \cdot f(4) \cdot f(6) \ldots f(2 n)}$. Tính $\lim \limits_{n \to +\infty}n \sqrt{u_n}$
% 	\choice
% 	{\True $\lim \limits_{n \to +\infty}n \sqrt{u_n}=\dfrac{1}{\sqrt{2}}$}
% 	{$\lim \limits_{n \to +\infty}n \sqrt{u_n}=\sqrt{2}$}
% 	{$\lim \limits_{n \to +\infty}n \sqrt{u_n}=\sqrt{3}$}
% 	{$\lim \limits_{n \to +\infty}n \sqrt{u_n}=\dfrac{1}{\sqrt{3}}$}
% 	\loigiai{
% 		$g(n)=\dfrac{f(2 n-1)}{f(2 n)} \Rightarrow g(n)=\dfrac{\left(4 n^2-2 n+1\right)^2+1}{\left(4 n^2+2 n+1\right)^2+1}$ \\ 
% 		Đặt $\left\{\begin{array}{l}a=4 n^2+1 \\ b=2 n\end{array} \Rightarrow\left\{\begin{array}{l}a=b^2+1 \\ a-2 b=(2 n-1)^2 \\ a+2 b=(2 n+1)^2\end{array}\right.\right.$.\\ 
% 		Suy ra $g(n)=\dfrac{(a-b)^2+1}{(a+b)^2+1}=\dfrac{a^2-2 a b+b^2+1}{a^2+2 a b+b^2+1}=\dfrac{a^2-2 a b+a}{a^2+2 a b+b}=\dfrac{a-2 b+1}{a+2 b+1}=\dfrac{(2 n-1)^2+1}{(2 n+1)^2+1}$. \\
% 		$\Rightarrow u_n=g(1) g(2) \ldots \ldots g(n)=\dfrac{2}{10} \cdot \dfrac{10}{26} \ldots \ldots \dfrac{(2 n-1)^2+1}{(2 n+1)^2+1}=\dfrac{2}{(2 n+1)^2+1}$.
% 		$\lim \limits_{n \to +\infty}n \sqrt{u_n}=\lim \limits_{n \to +\infty}n \cdot \sqrt{\dfrac{2}{(2 n+1)^2+1}}=\dfrac{1}{\sqrt{2}}$}
% \end{ex}
\Closesolutionfile{ans}
% \begin{indapan}{10}
% 	{ans/ans-1K5-1-Dang3}
% \end{indapan}
\begin{dang}{Tính tổng của dãy cấp số nhân lùi vô hạn}
	\begin{dn}
		Cấp số nhân vô hạn $u_1, u_1q,...,u_1q^{n-1},...$ có công bội $q$ thỏa mãn $|q|<1$ được gọi là cấp số nhân lùi vô hạn. 
		Tổng của cấp số nhân lùi vô hạn đã cho là $$S=u_1+u_1q+u_1q^2+...=\dfrac{u_1}{1-q}.$$
	\end{dn}
\end{dang}
\subsubsection{Ví dụ minh hoạ}
\begin{vd}%[1C3B1-5]
	Cho cấp số nhân $(u_n)$, với $u_1=1$ và công bội $q=\dfrac{1}{2}$.
	\begin{enumEX}{1}
		\item So sánh $\left|q\right|$ với $1$.
		\item Tính $S_n=u_1+u_2+\cdots+u_n$ từ đó hãy tính $\lim \limits_{n \to +\infty}S_n$.
	\end{enumEX}
	\loigiai{
		\begin{enumerate}
			\item Ta có $\left|q\right|=\left|\dfrac{1}{2}\right|=\dfrac{1}{2}<1$.
			\item Ta có $S_n=\dfrac{u_1\left(1-q^n\right)}{1-q}=\dfrac{1\cdot\left[1-\left(\dfrac{1}{2}\right)^n\right]}{1-\dfrac{1}{2}}=2\cdot\left(1-\dfrac{1}{2^n}\right)=2-\dfrac{1}{2^{n-1}}$.\\
			Khi đó $\lim \limits_{n \to +\infty}S_n=\lim \limits_{n \to +\infty}\left(2-\dfrac{1}{2^{n-1}}\right)=2$.
		\end{enumerate}
	}
\end{vd}
\begin{vd}%[1C3Y1-5]
	Tính tổng $T=1+\dfrac{1}{3}+\dfrac{1}{3^2}+\ldots+\dfrac{1}{3^n}+\ldots$
	\loigiai{
		Các số hạn của tổng lập thành câp số nhân $(u_n)$, có $u_1=1$, $q=\dfrac{1}{3}$ nên\\ $$T=1+\dfrac{1}{3}+\dfrac{1}{3^2}+\ldots+\dfrac{1}{3^n}+\ldots=\dfrac{1}{1-\dfrac{1}{3}}=\dfrac{2}{3}\cdot$$
	}
\end{vd}
\begin{vd}
	Tính tổng $S=1-\dfrac{1}{2}+\dfrac{1}{4}-\dfrac{1}{8}+\ldots+\left(-\dfrac{1}{2}\right)^{n-1}+\ldots$.
	\loigiai{
		Đây là tổng của cấp số nhân lùi vô hạng với $u_{1}=1$ và $q=-\dfrac{1}{2}$. Do đó
		$
		S=\dfrac{u_{1}}{1-q}=\dfrac{1}{1-\left(-\dfrac{1}{2}\right)}=\dfrac{2}{3}.
		$	
	}
\end{vd}

\begin{vd}%[1C3B1-5]
	Biểu diễn số thập phân vô hạn tuần hoàn $2{,}222 \ldots$ dưới dạng phân số.
	\loigiai{
		Ta có $2{,}222 \ldots=2+0{,}2+0{,}02+0{,}002+\ldots=2+2 \cdot 10^{-1}+2 \cdot 10^{-2}+2 \cdot 10^{-3}+\ldots$.\\
		Đây là tổng của cấp số nhân lùi vô hạn với $u_{1}=2, q=10^{-1}$ nên
		$$
		2{,}222 \ldots=\dfrac{u_{1}}{1-q}=\dfrac{2}{1-\dfrac{1}{10}}=\dfrac{20}{9}.
		$$
	}
\end{vd}
\begin{vd}%[1C3B1-5]
	Biểu diễn số thập phân vô hạn tuần hoàn $0,(3)$ dưới dạng phân số.
	\loigiai{
		Ta có  $0,(3)=\dfrac{3}{10}+\dfrac{3}{10^2}+\ldots+\dfrac{3}{10^n}+\ldots=\dfrac{\dfrac{3}{10}}{1-\dfrac{1}{10}}=\dfrac{1}{3}\cdot$
	}
\end{vd}
% \subsubsection{Bài tập rèn luyện} 
% % \subsubsection{Bài tập tự luận}
% \begin{bt}%[1C3B1-5]
% 	\begin{enumEX}{1}
% 		\item[] 
% 		\item Tính tổng cấp số nhân lùi vô hạn $(u_n)$ với $u_1=\dfrac{2}{3}, q=-\dfrac{1}{4}$.
% 		\item Biểu diễn số thập phân vô hạn tuần hoàn $1,(6)$ dưới dạng phân số.
% 	\end{enumEX}
% 	\loigiai{
% 		\begin{enumerate}
% 			\item Ta có  $S=\dfrac{u_1}{1-q}=\dfrac{\dfrac{2}{3}}{1-\left(-\dfrac{1}{4}\right)}=\dfrac{8}{15}$.
% 			\item Ta có  $1,(6)=1+0,(6)=1+\dfrac{6}{10}+\dfrac{6}{10^2}+\cdots+\dfrac{6}{10^n}+\cdots=1+\dfrac{\dfrac{6}{10}}{1-\dfrac{1}{10}}=\dfrac{5}{3}$.
% 		\end{enumerate}
% 	}
% \end{bt}
% \begin{bt}%[1T3B1-6]
% 	Tính tổng của các cấp số nhân lùi vô hạn sau
% 	\begin{listEX}[2]
% 		\item $-\dfrac{1}{2}+\dfrac{1}{4}-\dfrac{1}{8}+\cdots+\left(-\dfrac{1}{2}\right)^n+\cdots$.
% 		\item $\dfrac{1}{4}+\dfrac{1}{16}+\dfrac{1}{64}+\cdots+\left(\dfrac{1}{4}\right)^n+\cdots$.
% 	\end{listEX}
% 	\loigiai{
% 		\begin{enumerate}
% 			\item Tổng trên là tổng của cấp số nhân lùi vô hạn có số hạng đầu $u_1=-\dfrac{1}{2}$ và công bội $q=-\dfrac{1}{2}$ nên 
% 			\[-\dfrac{1}{2}+\dfrac{1}{4}-\dfrac{1}{8}+\cdots+\left(-\dfrac{1}{2}\right)^n+\cdots=\dfrac{-\dfrac{1}{2}}{1-\left(-\dfrac{1}{2}\right)}=-\dfrac{1}{3}. \]
% 			\item Tổng trên là tổng của cấp số nhân lùi vô hạn có số hạng đầu $u_1=\dfrac{1}{4}$ và công bội $q=\dfrac{1}{4}$ nên 
% 			\[\dfrac{1}{4}+\dfrac{1}{16}+\dfrac{1}{64}+\cdots+\left(\dfrac{1}{4}\right)^n+\cdots=\dfrac{\dfrac{1}{4}}{1-\dfrac{1}{4}}=\dfrac{1}{3}. \]
% 		\end{enumerate}
% 	}
% \end{bt}

% \begin{bt}%[1T3B1-5]
% 	Tính tổng của cấp số nhân lùi vô hạn: $1-\dfrac{1}{4}+\dfrac{1}{16}-\dfrac{1}{64}+\cdots+\left(-\dfrac{1}{4}\right)^n+\cdots$.
% 	\loigiai{
% 		Tổng trên là tổng của cấp số nhân lùi vô hạn có số hạng đầu $u_1=1$ và công bội $q=-\dfrac{1}{4}$ nên
% 		\[ 1-\dfrac{1}{4}+\dfrac{1}{16}-\dfrac{1}{64}+\cdots+\left(-\dfrac{1}{4}\right)^n+\cdots=\dfrac{1}{1-\left(-\dfrac{1}{4}\right)}=\dfrac{4}{5}.\]
% 	}
% \end{bt}

% \begin{bt}%[1T3B1-5]
% 	Biết rằng có thể coi số thập phân vô hạn tuần hoàn $0{,}666 \ldots$ là tổng của một cấp số nhân lùi vô hạn:
% 	\[ 0{,}666 \ldots=0{,}6+0{,}06+0{,}006+\cdots=0{,}6+0{,}6 \cdot \dfrac{1}{10}+0{,}6 \cdot \dfrac{1}{10^2}+\cdots.
% 	\]
% 	Hãy viết $0{,}666 \ldots$ dưới dạng phân số.
% 	\loigiai{
% 		Số $0{,}666 \ldots$ là tổng của cấp số nhân lùi vô hạn có số hạng đầu bằng $0{,}6$ và công bội bằng $\dfrac{1}{10}$.\\
% 		Do đó $0{,}666\ldots=\dfrac{0{,}6}{1-\dfrac{1}{10}}=\dfrac{6}{9}=\dfrac{2}{3}$.
% 	}
% \end{bt}

% \begin{bt}%[1T3B1-5]
% 	Tính tổng của cấp số nhân lùi vô hạn: $1+\dfrac{1}{3}+\left(\dfrac{1}{3}\right)^2+\cdots+\left(\dfrac{1}{3}\right)^n+\cdots$.	
% 	\loigiai{
% 		Tổng trên là tổng của cấp số nhân lùi vô hạn có số hạng đầu $u_1=1$ và công bội $q=\dfrac{1}{3}$ nên
% 		\[ 1+\dfrac{1}{3}+\left(\dfrac{1}{3}\right)^2+\cdots+\left(\dfrac{1}{3}\right)^n+\cdots=\dfrac{1}{1-\dfrac{1}{3}}= \dfrac{3}{2}.\]
% 	}
% \end{bt}
\subsubsection{Câu hỏi trắc nghiệm}
\Opensolutionfile{ans}[ans/ans-1K5-1-Dang4]
\begin{ex}%[1D4B1-5]
	Cho cấp số nhân $u_1,u_2,\ldots$ với công bội $q$ thỏa điều kiện $|q|<1$. Lúc đó, ta nói cấp số nhân đã cho là lùi vô hạn. Tổng của cấp số nhân đã cho là $S=u_1+u_2+u_3+\cdots +u_n+\cdots$ bằng 
	\choice
	{$\dfrac{u_1}{q-1}$}
	{$\dfrac{u_1\left(q^n-1\right)}{q-1}$}
	{$\dfrac{u_1}{1+q}$}
	{\True $\dfrac{u_1}{1-q}$}
	\loigiai{
		Theo định nghĩa cấp số nhân lùi vô hạn ta chứng minh được.\\
		$S=u_1+u_2+u_3+\cdots +u_n+\cdots =u_1+u_1q^1+u_1q^2+\cdots +u_1q^{n-1}+\cdots =\dfrac{u_1}{1-q}$.}
\end{ex}
\begin{ex}%[1D4B1-5]
	Gọi $S=\dfrac{1}{3}-\dfrac{1}{9}+\cdots +\dfrac{(-1)^{n+1}}{3^n}$. Khi đó, $\lim \limits_{n \to +\infty}S$ bằng 
	\choice
	{$\dfrac{3}{4}$}
	{\True $\dfrac{1}{4}$}
	{$\dfrac{1}{2}$}
	{$1$}
	\loigiai{
		Ta có 
		\allowdisplaybreaks
		\begin{eqnarray*}		
			&&S=\dfrac{1}{3}-\dfrac{1}{9}+\cdots +\dfrac{(-1)^{n+1}}{3^n} \\
			&\Leftrightarrow& S=-\left(-\dfrac{1}{3}+\dfrac{1}{9}+\cdots +\dfrac{(-1)^n}{3^n}\right) \\
			&\Leftrightarrow& S=\dfrac{1}{3}\cdot\dfrac{1-\left(\dfrac{-1}{3}\right)^n}{1-\dfrac{-1}{3}}\\
			&\Leftrightarrow& S=\dfrac{1}{4}\cdot\left(1-\left(\dfrac{-1}{3}\right)^n\right).
		\end{eqnarray*}
		Suy ra $\lim \limits_{n \to +\infty}S=\lim \limits_{n \to +\infty}\dfrac{1}{4}\cdot\left(1-\left(\dfrac{-1}{3}\right)^n\right)=\dfrac{1}{4}$.	
	}
\end{ex}
\begin{ex}%[1D4B1-5]
	Tổng $S=\dfrac{1}{3}+\dfrac{1}{3^2}+\cdots +\dfrac{1}{3^n}+\cdots$ có giá trị là 
	\choice
	{$\dfrac{1}{3}$}
	{\True $\dfrac{1}{2}$}
	{$\dfrac{1}{9}$}
	{$\dfrac{1}{4}$}
	\loigiai{
		Ta có $S=\dfrac{1}{3}+\dfrac{1}{3^2}+\cdots +\dfrac{1}{3^n}+\cdots=\dfrac{1}{3}\cdot\dfrac{1}{1-\frac{1}{3}}=\dfrac{1}{2}$. 
	}
\end{ex}
\begin{ex}%[1D4B1-5]
	Tính $S=9+3+1+\dfrac{1}{3}+\dfrac{1}{9}+\cdots +\dfrac{1}{3^{n-3}}+\cdots$. Kết quả là 
	\choice
	{\True $\dfrac{27}{2}$}
	{$14$}
	{$16$}
	{$15$}
	\loigiai{
		Ta có $S=9+3+1+\dfrac{1}{3}+\dfrac{1}{9}+\cdots +\dfrac{1}{3^{n-3}}+\cdots=13+\dfrac{1}{3}\cdot\dfrac{1}{1-\frac{1}{3}}=13+\dfrac{1}{2}=\dfrac{27}{2}$. 
	}
\end{ex}
\begin{ex}%[1D4B1-5]
	Tổng các cấp số nhân vô hạn: $1,-\dfrac{1}{2},\dfrac{1}{4},-\dfrac{1}{8},\ldots,\dfrac{(-1)^{n+1}}{2^{n-1}},\ldots$ là
	\choice
	{$\dfrac{3}{2}$}
	{\True $\dfrac{2}{3}$}
	{$-\dfrac{2}{3}$}
	{$2$}
	\loigiai{
		Ta có $S=1-\dfrac{1}{2}+\dfrac{1}{4}-\dfrac{1}{8}+\cdots +\dfrac{(-1)^{n+1}}{2^{n-1}}+\cdots=1-\dfrac{1}{2}\cdot\dfrac{1}{1+\frac{1}{2}}=1-\dfrac{1}{3}=\dfrac{2}{3}$.
	}
\end{ex}
\begin{ex}%[1D4B1-5]
	Gọi $S=1+\dfrac{2}{3}+\dfrac{4}{9}+\cdots +\dfrac{2^n}{3^n}+\cdots$. Giá trị của $S$ bằng
	\choice
	{\True $3$}
	{$5$}
	{$6$}
	{$4$}
	\loigiai{
		Ta có $S=1+\dfrac{2}{3}+\dfrac{4}{9}+\cdots +\dfrac{2^n}{3^n}+\cdots=1+\dfrac{2}{3}\cdot\dfrac{1}{1-\frac{2}{3}}=1+2=3$.
	}
\end{ex}
\begin{ex}%[1D4B1-5]%
	Số thập phân vô hạn tuần hoàn $0{,}233333\ldots$ biểu diễn dưới dạng số là 
	\choice
	{$\dfrac{1}{23}$}
	{$\dfrac{2333}{10000}$}
	{$\dfrac{23333}{10^5}$}
	{\True $\dfrac{7}{30}$}
	\loigiai{
		$0{,}233333\ldots=0{,}2+3\left(\dfrac{1}{10^2}+\dfrac{1}{10^3}+\ldots\right)=0{,}2+3\cdot\dfrac{1}{100}\cdot\dfrac{1}{1-\frac{1}{10}}=\dfrac{1}{5}+\dfrac{1}{30}=\dfrac{7}{30}$.
	}
\end{ex}
\begin{ex}%[1D4B1-5]%
	Số thập phân vô hạn tuần hoàn $0{,}212121\ldots$ biểu diễn dưới dạng phân số là 
	\choice
	{$\dfrac{2121}{10^4}$}
	{$\dfrac{1}{21}$}
	{\True $\dfrac{7}{33}$}
	{$\dfrac{212121}{10^6}$}
	\loigiai{
		$0{,}212121\ldots=21\left(\dfrac{1}{10^2}+\dfrac{1}{10^4}+\ldots\right)=21\cdot\dfrac{1}{10^2}\cdot\dfrac{1}{1-\frac{1}{100}}=\dfrac{7}{33}$.
	}
\end{ex}

\begin{ex}%[1D4B1-5]
	Số thập phân vô hạn tuần hoàn $0{,}271414\ldots$ được biểu diễn bằng phân số: 
	\choice
	{$\dfrac{2714}{9900}$}
	{$\dfrac{2617}{9900}$}
	{$\dfrac{2786}{9900}$}
	{\True $\dfrac{2687}{9900}$}
	\loigiai{
		$0{,}271414\cdots=0{,}27+14\left(\dfrac{1}{10^4}+\dfrac{1}{10^6}\ldots\right)=0{,}27+14\cdot\dfrac{1}{10^4}\cdot\dfrac{1}{1-\frac{1}{100}}=\dfrac{27}{100}+\dfrac{7}{4950}=\dfrac{2687}{9900}$.
	}
\end{ex}

\begin{ex}%[1D4B1-5]
	Tổng của cấp số nhân lùi vô hạn: $-\dfrac{1}{2},\dfrac{1}{4},\dfrac{1}{8},\ldots,\dfrac{(-1)^n}{2^n},\ldots$ là 
	\choice
	{\True $-\dfrac{1}{3}$}
	{$-\dfrac{1}{4}$}
	{$-1$}
	{$\dfrac{1}{2}$}
	\loigiai{
		Từ $-\dfrac{1}{2},\dfrac{1}{4},\dfrac{1}{8},\ldots,\dfrac{(-1)^n}{2^n},\ldots$ có $u_1=-\dfrac{1}{2}$ và $q=-\dfrac{1}{2}$.\\
		Có $S=-\dfrac{1}{2}+\dfrac{1}{4}+\dfrac{1}{8}+\cdots +\dfrac{(-1)^n}{2^n}+\cdots =\dfrac{\left(-\frac{1}{2}\right)}{1-\left(-\frac{1}{2}\right)}=-\dfrac{1}{3}$.}
\end{ex}
\begin{ex}%[1D4B1-5]
	Số thập phân vô hạn tuần hoàn $0{,}511111\cdots$ được biểu diễn bởi phân số
	\choice
	{$\dfrac{47}{90}$}
	{\True $\dfrac{46}{90}$}
	{$\dfrac{6}{11}$}
	{$\dfrac{43}{90}$}
	\loigiai{
		Ta có\\
		$0{,}511111\cdots=0{,}5+\dfrac{1}{10^2}+\dfrac{1}{10^3}\ldots=\dfrac{1}{2}+\dfrac{1}{10^2}\cdot\dfrac{1}{1-\frac{1}{10}}=\dfrac{1}{2}+\dfrac{1}{90}=\dfrac{23}{45}$.
	}
\end{ex}
\begin{ex}%[1D4B1-5]
	Tổng của cấp số nhân vô hạn $\dfrac{1}{2}$, $-\dfrac{1}{4}$, $\dfrac{1}{8},\ldots\dfrac{(-1)^{n+1}}{2^n},\ldots$ là
	\choice
	{$-\dfrac{2}{3}$}
	{$1$}
	{$-\dfrac{1}{3}$}
	{\True $\dfrac{1}{3}$}
	\loigiai{
		Ta có $S=\dfrac{1}{2}-\dfrac{1}{4}+\dfrac{1}{8}+\cdots +\dfrac{(-1)^{n+1}}{2^n}+\cdots =\dfrac{1}{2}\cdot\dfrac{1}{1+\frac{1}{2}}=\dfrac{1}{3}$.}
\end{ex}
\begin{ex}%[1D4B1-5]
	Tổng của cấp số nhân lùi vô hạn $\dfrac{1}{2},-\dfrac{1}{6},\dfrac{1}{18},\ldots,\dfrac{(-1)^{n+1}}{2\cdot 3^{n-1}},\ldots$ là
	\choice
	{$\dfrac{3}{4}$}
	{$\dfrac{8}{3}$}
	{$\dfrac{2}{3}$}
	{\True $\dfrac{3}{8}$}
	\loigiai{
		Cấp số nhân có $u_1=\dfrac{1}{2}, q=-\dfrac{1}{3}$. Do đó tổng cần tìm là
		$$S=\dfrac{u_1}{1-q}=\dfrac{\dfrac{1}{2}}{1+\frac{1}{3}}=\dfrac{1}{2}\cdot\dfrac{3}{4}=\dfrac{3}{8}.$$}
\end{ex}
\begin{ex}%[1D4B1-5]
	Tổng của cấp số nhân lùi vô hạn $\dfrac{1}{3},-\dfrac{1}{9},\dfrac{1}{27},\ldots\cdot,\dfrac{(-1)^{n+1}}{3^n},\ldots$ là
	\choice
	{$4$}
	{$\dfrac{1}{2}$}
	{$\dfrac{3}{4}$}
	{\True $\dfrac{1}{4}$}
	\loigiai{
		Cấp số nhân có $u_1=\dfrac{1}{3}, q=-\dfrac{1}{3}$. Do đó tổng cần tìm là\\
		$$S=\dfrac{u_1}{1-q}=\dfrac{\dfrac{1}{3}}{1+\frac{1}{3}}=\dfrac{1}{3}\cdot\dfrac{3}{4}=\dfrac{1}{4}.$$}
\end{ex}
\begin{ex}%[1D4B1-5]
	Số thập phân vô hạn tuần hoàn $0{,}17232323\ldots$ được biểu diễn bởi phân số?
	\choice
	{$\dfrac{1706}{9900}$}
	{$\dfrac{153}{990}$}
	{$\dfrac{164}{990}$}
	{\True $\dfrac{853}{4950}$}
	\loigiai{
		$0{,}17232323\ldots=0{,}17+23\left(\dfrac{1}{10^4}+\dfrac{1}{10^6}+\ldots\right)=\dfrac{17}{100}+23\cdot\dfrac{1}{10^4}\cdot\dfrac{1}{1-\frac{1}{100}}=\dfrac{17}{100}+\dfrac{23}{9900}=\dfrac{853}{4950}$.
	}
\end{ex}
\Closesolutionfile{ans}
% \begin{indapan}{10}
% 	{ans/ans-1K5-1-Dang4}
% \end{indapan}

\begin{dang}{Toán thực tế, liên môn liên quan đến giới hạn dãy số}
	$$S=u_1+u_1q+u_1q^2+...=\dfrac{u_1}{1-q}.$$
\end{dang}
\subsubsection{Ví dụ minh hoạ}
\begin{vd}%[1C3K1-6]
	\immini{Từ hình vuông có độ dài cạnh bằng $1$, người ta nối các trung điểm của cạnh hình vuông để tạo ra hình vuông mới như hình bên. Tiếp tục quá trình này đến vô hạn.
		\begin{enumEX}{1}
			\item Tính diện tích $S_n$ của hình vuông được tạo thành từ bước thứ $n$.
			\item Tính tổng diện tích của tất cả các hình vuông được tạo thành.
	\end{enumEX}}	
	{
		\begin{tikzpicture}[scale=0.6, font=\footnotesize,>=stealth]
			\def\canhAD{6};
			\coordinate (D) at (0,0);
			\coordinate (A) at ($(D)+(90:\canhAD)$);
			\coordinate (C) at ($(D)+(0:\canhAD)$);
			\coordinate (B) at ($(A)+(0:\canhAD)$);
			\coordinate (E) at ($(A)!0.5!(B)$);
			\coordinate (F) at ($(B)!0.5!(C)$);
			\coordinate (G) at ($(C)!0.5!(D)$);
			\coordinate (H) at ($(D)!0.5!(A)$);
			\coordinate (K) at ($(E)!0.5!(F)$);
			\coordinate (L) at ($(F)!0.5!(G)$);
			\coordinate (M) at ($(G)!0.5!(H)$);
			\coordinate (J) at ($(H)!0.5!(E)$);
			\coordinate (I) at ($(E)!0.5!(G)$);
			\coordinate (X) at ($(J)!0.5!(K)$);
			\coordinate (Y) at ($(K)!0.5!(L)$);
			\coordinate (T) at ($(L)!0.5!(M)$);
			\coordinate (Q) at ($(M)!0.5!(J)$);
			\draw(A) rectangle (C)(E)--(F)--(G)--(H)--cycle (J)--(K)--(L)--(M)--cycle (X)--(Y)--(T)--(Q)--cycle;
			\draw [>=stealth,|<->|] ([shift=({0,-0.4})]D)--([shift=({0,-0.4})]C) node [midway,below]{$1$};
			
			%			\foreach \x/\y in {A/90,D/-90,C/-90,B/90,E/90,G/-90,H/180,F/0,K/90,L/-90,M/-90,J/90,I/-110}{\fill (\x) circle(1pt) ($(\x)+(\y:0.3cm)$) node{$\x$};}
		\end{tikzpicture}
	}
	
	\loigiai{
		\begin{enumerate}
			\item Từ giả thiết suy ra diện tích hình vuông sau bằng $\dfrac{1}{2}$ diện tích hình vuông trước.\\ 
			Khi đó diện tích của các hình vuông tạo thành một cấp số nhân lùi vô hạn với số hạng đầu $S_1=1$ và công bội $q=\dfrac{1}{2}$.\\
			Diện tích $S_n$ của hình vuông được tạo thành từ bước thứ $n$ là $S_n=S_1\cdot q^{n-1}=\left(\dfrac{1}{2}\right)^{n-1}$.
			\item Tổng diện tích của tất cả các hình vuông được tạo thành là:\\
			$$S=\dfrac{u_1}{1-q}=\dfrac{1}{1-\dfrac{1}{2}}=2.$$
		\end{enumerate}
	}
\end{vd}
\begin{vd}%[1C3K1-6]
	Có $1$ kg chất phóng xạ độc hại. Biết rằng, cứ sau một khoảng thời gian $T=24000$ năm thì một nửa số chất phóng xạ này bị phân rã thành chất khác không độc hại đối với sức khỏe của con người ($T$ được gọi là \textit{chu kì bán rã}).
	\begin{flushright}
		(\textit{Nguồn: Đại số và giải tích 11, NXB GD Việt Nam, 2021})
	\end{flushright}
	Gọi $u_n$ là khối lượng chất phóng xạ còn lại sau chu kì thứ $n$.
	\begin{enumerate}
		\item  Tìm số hạng tổng quát $u_n$ của dãy số $(u_n)$.
		\item  Chứng minh rằng $(u_n)$ có giới hạn là $0$.
		\item  Từ kết quả câu $2$, chứng tỏ rằng sau một số năm nào đó khối lượng phóng xạ đã cho ban đầu không còn độc hại với con người, biết rằng chất phóng xạ này sẽ không độc hại nữa nếu khối lượng chất phóng xạ còn lại bé hơn $10^{-6}$ g.
	\end{enumerate}
	\loigiai{
		\begin{enumerate}
			\item  Khối lượng chất phóng xạ còn lại sau chu kì bán rã thứ $1$ là $u_1=\dfrac{1}{2}\cdot 1=1$ kg.\\
			Khối lượng chất phóng xạ còn lại sau chu kì bán rã thứ $2$ là $u_2=\dfrac{1}{2}\cdot u_1=\dfrac{1}{2}\cdot \dfrac{1}{2}=\dfrac{1}{2^2}$ kg.\\
			Khối lượng chất phóng xạ còn lại sau chu kì bán rã thứ $3$ là $u_3=\dfrac{1}{2}\cdot u_2=\dfrac{1}{2}\cdot \dfrac{1}{4}=\dfrac{1}{2^3}$ kg.\\
			Khối lượng chất phóng xạ còn lại sau chu kì bán rã thứ $n$ là $u_n=\dfrac{1}{2^n}$ kg.\\
			\item $\lim \limits_{n \to +\infty}u_n=\lim \limits_{n \to +\infty}\dfrac{1}{2^n}=\lim\left(\dfrac{1}{2}\right)^n=0$.
			\item Chất phóng xạ sẽ không độc hại nữa nếu khối lượng chất phóng xạ còn lại bé hơn $10^{-6}~\mathrm{g}=10^{-9}$ kg
			$$\Leftrightarrow u_n<10^{-9}\Leftrightarrow\dfrac{1}{2^n}<10^{-9}\Leftrightarrow 2^n>10^9\Leftrightarrow n\geq 30.$$
			Vậy sau ít nhất $30$ chu kì bằng $30\cdot 24000=720000$ năm thì khối lượng phóng xạ đã cho ban đầu không còn độc hại với con người nữa.
		\end{enumerate}
	}
\end{vd}
\begin{vd}%[1C3G1-6]
	Gọi $C$ là nữa đường tròn đường kính $AB=2R$.\\
	$C_1$ là đường gồm hai nửa đường tròn đường kính $\dfrac{AB}{2}$,\\
	$C_2$ là đường gồm bốn nửa đường tròn đường kính $\dfrac{AB}{4},\cdots$\\
	$C_n$ là đường gồm $2^n$ nửa đường tròn đường kính $\dfrac{AB}{2^n},\cdots$
	\immini{	Gọi $p_n$ là độ dài của $C_n$, $S_n$ là diện tích hình phẳng giới hạn bởi $C_n$ và đoạn thẳng $AB$.
		\begin{enumEX}{1}
			\item Tính $p_n$, $S_n$.
			\item Tính giới hạn của các dãy số $(p_n)$ và $(S_n)$.
		\end{enumEX}
	}{
		\begin{tikzpicture}[declare function={r=3;}]
			\path
			(0,0) coordinate (O)
			(0:r) coordinate (A)
			(180:r) coordinate (B)
			($(A)!.5!(O)$) coordinate (M)
			($(B)!.5!(O)$) coordinate (N)
			($(B)!.5!(N)$) coordinate (I)
			($(N)!.5!(O)$) coordinate (J)
			($(O)!.5!(M)$) coordinate (K)
			($(M)!.5!(A)$) coordinate (H)
			($(B)!.5!(I)$) coordinate (Q);
			\draw (A) arc(0:180:r) (A) arc(0:180:r/2) (O) arc(0:180:r/2) (A) arc(0:180:r/4) (A) arc(0:180:r/4)(M) arc(0:180:r/4) (O) arc(0:180:r/4) (N) arc(0:180:r/4)
			(A) arc(0:180:r/8) (H) arc(0:180:r/8) (M) arc(0:180:r/8) (K) arc(0:180:r/8) (O) arc(0:180:r/8) (J) arc(0:180:r/8) (N) arc(0:180:r/8) (I) arc(0:180:r/8);
			\draw (A)--(B);
			\path 
			($(O)+(90:0.9*r)$) node{$C$}
			($(N)+(90:0.4*r)$) node[scale=0.8]{$C_1$}
			($(I)+(90:0.18*r)$) node[scale=0.7]{$C_2$}
			($(Q)+(90:0.06*r)$) node[scale=0.6]{$C_3$};
			\foreach \x/\y in {A/-90, B/-90}{\fill (\x) circle(1pt) ($(\x)+(\y:0.3cm)$) node{$\x$};}
		\end{tikzpicture}
	}
	
	\loigiai{
		\begin{enumerate}
			\item Ta có  
			\begin{eqnarray*}
				p_n&=&2^n \cdot \pi r=2^n\cdot\pi \cdot \dfrac{AB}{2\cdot2^n}\\
				&=&\dfrac{\pi AB}{2}\\
				&=&\dfrac{\pi \cdot 2R}{2}\\
				&=&\pi R.
			\end{eqnarray*}
			\begin{eqnarray*}
				S_n&=&2^n \cdot \dfrac{1}{2}\pi r^2\\
				&=&2^n \cdot \dfrac{1}{2}\pi \left(\dfrac{AB}{2\cdot2^n}\right)^2\\
				&=&2^n \cdot \dfrac{1}{2}\pi \left(\dfrac{2R}{2\cdot2^n}\right)^2\\
				&=&2^n \cdot \dfrac{1}{2}\pi \dfrac{R^2}{(2^n)^2}\\
				&=&\dfrac{\pi R^2}{2^{n+1}}.	
			\end{eqnarray*}			
			\item $\lim \limits_{n \to +\infty}p_n=\lim \limits_{n \to +\infty}\left(\pi R\right) = \pi R$.\\
			$\lim \limits_{n \to +\infty}S_n=\lim \limits_{n \to +\infty}\dfrac{\pi R^2}{2^{n+1}}=0$ (Vì $\lim \limits_{n \to +\infty}\left(\pi R^2\right)=\pi R^2$ và $\lim \limits_{n \to +\infty}2^{n+1}=+\infty$).
		\end{enumerate}
	}
\end{vd}
\begin{vd}%[1D4K1-6]
	\immini{Từ độ cao $55,8 \mathrm{~m}$ của tháp nghiêng Pisa nước Ý, người ta thả một quả bóng cao su chạm xuống đất hình bên dưới. Giả sử mỗi lần chạm đất quả bóng lại nảy lên độ cao bằng $\dfrac{1}{10}$ độ cao mà quả bóng đạt được trước đó. Gọi $S_n$ là tổng độ dài quãng đường di chuyển của quả bóng tính từ lúc thả ban đầu cho đến khi quả bóng đó chạm đất $n$ lần. Tính $\lim \limits_{n \to +\infty}S_n$.}{
		
		\definecolor{lightcornflowerblue}{rgb}{0.6, 0.81, 0.93}
		\definecolor{cadmiumgreen}{rgb}{0.0, 0.42, 0.24}
		\definecolor{trueblue}{rgb}{0.0, 0.45, 0.81}
		\definecolor{tumbleweed}{rgb}{0.87, 0.67, 0.53}%màu cát
		\begin{tikzpicture}[line join=round, line cap=round,scale=1,transform shape]
			\clip (-4,-3.5) rectangle (4,3.5);
			\tikzset{thap/.pic={%Xây từ dưới lên
					\def\T{ 
						(-.88,-1.78)%1
						..controls +(40:.17) and +(120:.1) ..  (.686,-1.91)--(.68,-1.94)
						..controls +(120:.1) and +(40:.12) ..  (-.87,-1.81)
						--cycle
						
						(-.8,-1.2)%2
						..controls +(40:.14) and +(120:.14) ..  (.72,-1.34)--(.7,-1.36)
						..controls +(120:.1) and +(40:.12) ..  (-.78,-1.23)
						--cycle
						
						(-.74,-.65)%3
						..controls +(40:.22) and +(120:.17) ..  (.74,-.78)--(.72,-.8)
						..controls +(120:.15) and +(40:.15) ..  (-.72,-.67)
						--cycle
						
						(-.68,-.13)%4
						..controls +(40:.34) and +(110:.2) ..  (.8,-.26)--(.78,-.28)
						..controls +(110:.17) and +(40:.27) ..  (-.66,-.15)
						--cycle
						
						(-.65,.36)%5
						..controls +(40:.44) and +(120:.3) ..  (.85,.24)--(.83,.22)
						..controls +(130:.35) and +(40:.35) ..  (-.63,.34)
						--cycle
						
						(-.58,.88)%6
						..controls +(40:.44) and +(120:.3) ..  (.87,.78)--(.85,.76)
						..controls +(130:.35) and +(40:.35) ..  (-.56,.86)
						--cycle
						
						(-.52,1.35)%7
						..controls +(120:.32) and +(110:.47) ..  (.93,1.26)--(.9,1.25)
						..controls +(110:.43) and +(110:.27) ..  (-.49,1.35)
						--cycle
						
						(-.31,2)%8
						..controls +(120:.47) and +(55:.46) ..  (.76,1.94)--(.74,1.94)
						..controls +(75:.31) and +(115:.28) ..  (-.28,2)
						--cycle
						
						(-.48,1.5)
						..controls +(40:.33) and +(110:.2) ..  (.86,1.4)
						(-.476,1.52)
						..controls +(40:.33) and +(110:.2) ..  (.86,1.42)
						(-.472,1.54)
						..controls +(40:.33) and +(110:.2) ..  (.86,1.44)
						(-.468,1.56)
						..controls +(40:.33) and +(110:.2) ..  (.86,1.46)
						
						(-.28,2.2)
						..controls +(40:.27) and +(140:.2) ..  (.74,2.16)
						(-.276,2.22)
						..controls +(40:.27) and +(140:.2) ..  (.74,2.18)
						(-.272,2.24)
						..controls +(40:.27) and +(140:.2) ..  (.74,2.2)
						(-.268,2.26)
						..controls +(40:.27) and +(140:.2) ..  (.74,2.22)
						;}
					\draw \T;
					%\fill[tumbleweed] \T;
			}}
			
			\tikzset{rao/.pic={%Rào
					\def\R{ 
						(-.48,1.5)
						..controls +(40:.33) and +(110:.2) ..  (.86,1.4)
						(-.476,1.52)
						..controls +(40:.33) and +(110:.2) ..  (.86,1.42)
						(-.472,1.54)
						..controls +(40:.33) and +(110:.2) ..  (.86,1.44)
						(-.468,1.56)
						..controls +(40:.33) and +(110:.2) ..  (.86,1.46)
						
						(-.28,2.2)
						..controls +(40:.27) and +(140:.2) ..  (.74,2.16)
						(-.276,2.22)
						..controls +(40:.27) and +(140:.2) ..  (.74,2.18)
						(-.272,2.24)
						..controls +(40:.27) and +(140:.2) ..  (.74,2.2)
						(-.268,2.26)
						..controls +(40:.27) and +(140:.2) ..  (.74,2.22)
						;}
					\draw \R;
					%\fill[tumbleweed] \R;
			}}
			
			\tikzset{co/.pic={%Cờ
					\def\C{ 
						(0.1,2.3)--(.1,2.9)
						(.1,2.82)
						..controls +(-40:.2) and +(140:.22) ..  (.2,2.6)
						..controls +(-60:0) and +(100:.1) ..(.16,2.5)
						..controls +(140:.1) and +(-30:0) ..  (.1,2.53)--cycle
						;}
					\draw \C;
					\fill[red] \C;
			}}
			
			\tikzset{cong0/.pic={%Đường cong trệt
					\def\D0{ 
						(.68,-1.96)
						..controls +(120:.1) and +(40:.12) ..  (-.87,-1.81)--
						(-.87,-1.94)
						..controls +(80:.12) and +(95:0.12) ..  (-.68,-2.06)
						..controls +(80:.12) and +(85:0.3) ..  (-.4,-2.12)
						..controls +(80:.25) and +(95:0.25) ..  (-.02,-2.12)
						..controls +(80:.25) and +(95:0.25) ..  (.27,-2.14)
						..controls +(80:.12) and +(95:0.12) ..  (.5,-2.14)
						..controls +(70:.12) and +(95:0.02) ..  (.66,-2.1)--cycle
						;}
					\draw \D0;
					\fill[tumbleweed] \D0;
			}}
			
			\tikzset{cong1/.pic={%Đường cong 1
					\def\D1{ 
						(.7,-1.36)
						..controls +(120:.1) and +(40:.12) ..  (-.78,-1.23)--
						(-.78,-1.35)
						..controls +(80:.12) and +(95:0.12) ..  (-.72,-1.35)
						..controls +(80:.12) and +(95:0.12) ..  (-.64,-1.35)
						..controls +(80:.12) and +(95:0.12) ..  (-.55,-1.35)
						..controls +(80:.12) and +(95:0.12) ..  (-.44,-1.36)
						..controls +(80:.12) and +(95:0.12) ..  (-.3,-1.37)
						..controls +(80:.12) and +(95:0.12) ..  (-.14,-1.38)
						..controls +(80:.12) and +(95:0.12) ..  (0.02,-1.39)
						..controls +(80:.12) and +(95:0.12) ..  (0.18,-1.4)
						..controls +(80:.12) and +(95:0.12) ..  (0.32,-1.42)
						..controls +(70:.12) and +(95:0.12) ..  (0.42,-1.44)
						..controls +(70:.12) and +(95:0.12) ..  (0.52,-1.46)
						..controls +(70:.12) and +(95:0.12) ..  (0.6,-1.47)
						..controls +(70:.12) and +(95:0.12) ..  (0.68,-1.5)--cycle
						;}
					\draw \D1;
					\fill[tumbleweed] \D1;
			}}
			
			\tikzset{cong2/.pic={%Đường cong 2
					\def\D2{ 
						(.72,-.8)
						..controls +(120:.15) and +(40:.15) ..  (-.72,-.67)--
						(-.72,-.79)
						..controls +(80:.12) and +(95:0.12) ..  (-.66,-.79)
						..controls +(80:.12) and +(95:0.12) ..  (-.58,-.79)
						..controls +(80:.12) and +(95:0.12) ..  (-.48,-.79)
						..controls +(80:.12) and +(95:0.12) ..  (-.38,-.8)
						..controls +(80:.12) and +(95:0.12) ..  (-.23,-.8)
						..controls +(80:.12) and +(95:0.12) ..  (-0.08,-.81)
						..controls +(80:.12) and +(95:0.12) ..  (0.08,-.82)
						..controls +(80:.12) and +(95:0.12) ..  (0.22,-.84)
						..controls +(80:.12) and +(95:0.12) ..  (0.36,-.85)
						..controls +(70:.12) and +(95:0.12) ..  (0.48,-.87)
						..controls +(70:.12) and +(95:0.12) ..  (0.58,-.89)
						..controls +(70:.12) and +(95:0.12) ..  (0.66,-.9)
						..controls +(70:.12) and +(95:0.12) ..  (0.71,-.92)--cycle
						;}
					\draw \D2;
					\fill[tumbleweed] \D2;
			}}
			
			\tikzset{cong3/.pic={%Đường cong 3
					\def\D3{ 
						(.78,-.28)
						..controls +(110:.17) and +(40:.27) ..  (-.66,-.15)--
						(-.66,-.26)
						..controls +(80:.12) and +(95:0.12) ..  (-.6,-.25)
						..controls +(80:.12) and +(95:0.12) ..  (-.52,-.24)
						..controls +(80:.12) and +(95:0.12) ..  (-.44,-.23)
						..controls +(80:.12) and +(95:0.12) ..  (-.32,-.23)
						..controls +(80:.12) and +(95:0.12) ..  (-.18,-.23)
						..controls +(80:.12) and +(95:0.12) ..  (-.04,-.24)
						..controls +(80:.12) and +(95:0.12) ..  (0.13,-.25)
						..controls +(80:.12) and +(95:0.12) ..  (0.26,-.26)
						..controls +(70:.12) and +(95:0.12) ..  (0.42,-.29)
						..controls +(70:.12) and +(95:0.12) ..  (0.54,-.31)
						..controls +(70:.12) and +(95:0.12) ..  (0.63,-.33)
						..controls +(70:.12) and +(95:0.12) ..  (0.7,-.35)
						..controls +(70:.12) and +(95:0.12) ..  (0.765,-.38)--cycle
						;}
					\draw \D3;
					\fill[tumbleweed] \D3;
			}}
			
			\tikzset{cong4/.pic={%Đường cong 4
					\def\D4{ 
						(.83,.22)
						..controls +(130:.35) and +(40:.35) ..  (-.63,.34)--
						(-.63,.23)
						..controls +(80:.12) and +(95:0.12) ..  (-.57,.25)
						..controls +(80:.12) and +(95:0.12) ..  (-.49,.26)
						..controls +(80:.12) and +(95:0.12) ..  (-.4,.28)
						..controls +(80:.12) and +(95:0.12) ..  (-.27,.3)
						..controls +(80:.12) and +(95:0.12) ..  (-.15,.3)
						..controls +(80:.12) and +(95:0.12) ..  (0.02,.3)
						..controls +(80:.12) and +(95:0.12) ..  (0.16,.29)
						..controls +(80:.12) and +(95:0.12) ..  (0.31,.27)
						..controls +(70:.12) and +(95:0.12) ..  (0.45,.25)
						..controls +(70:.12) and +(95:0.12) ..  (0.58,.23)
						..controls +(70:.12) and +(95:0.12) ..  (0.67,.2)
						..controls +(70:.12) and +(95:0.12) ..  (0.75,.17)
						..controls +(70:.12) and +(95:0.12) ..  (0.8,.14)--cycle
						;}
					\draw \D4;
					\fill[tumbleweed] \D4;
			}}
			
			\tikzset{cong5/.pic={%Đường cong 5
					\def\D5{ 
						(.85,.76)
						..controls +(130:.35) and +(40:.35) ..  (-.56,.86)--
						(-.56,.74)
						..controls +(80:.12) and +(95:0.12) ..  (-.5,.77)
						..controls +(80:.12) and +(95:0.12) ..  (-.44,.79)
						..controls +(80:.12) and +(95:0.12) ..  (-.34,.81)
						..controls +(80:.12) and +(95:0.12) ..  (-.22,.83)
						..controls +(80:.12) and +(95:0.12) ..  (-0.08,.85)
						..controls +(80:.12) and +(95:0.12) ..  (0.06,.85)
						..controls +(80:.12) and +(95:0.12) ..  (0.22,.83)
						..controls +(70:.12) and +(95:0.12) ..  (0.36,.81)
						..controls +(70:.12) and +(95:0.12) ..  (0.5,.78)
						..controls +(70:.12) and +(95:0.12) ..  (0.62,.74)
						..controls +(70:.12) and +(95:0.12) ..  (0.73,.72)
						..controls +(70:.12) and +(95:0.12) ..  (0.82,.68)--cycle
						;}
					\draw \D5;
					\fill[tumbleweed] \D5;
			}}
			
			\tikzset{cong6/.pic={%Đường cong 6
					\def\D6{ 
						(.9,1.25)
						..controls +(110:.43) and +(110:.27) ..  (-.49,1.35)--
						(-.49,1.3)
						..controls +(80:.12) and +(95:0.12) ..  (-.45,1.33)
						..controls +(80:.12) and +(95:0.12) ..  (-.38,1.33)
						..controls +(80:.12) and +(95:0.12) ..  (-.28,1.35)
						..controls +(80:.12) and +(95:0.12) ..  (-.16,1.37)
						..controls +(80:.12) and +(95:0.12) ..  (-0.04,1.38)
						..controls +(80:.12) and +(95:0.12) ..  (0.12,1.38)
						..controls +(80:.12) and +(95:0.12) ..  (0.26,1.38)
						..controls +(70:.12) and +(95:0.12) ..  (0.42,1.37)
						..controls +(70:.12) and +(95:0.12) ..  (0.55,1.34)
						..controls +(70:.12) and +(95:0.12) ..  (0.67,1.3)
						..controls +(70:.12) and +(95:0.12) ..  (0.78,1.25)
						..controls +(70:.12) and +(95:0.12) ..  (0.84,1.21)
						..controls +(70:.12) and +(95:0.12) ..  (0.88,1.16)--cycle
						;}
					\draw \D6;
					\fill[tumbleweed] \D6;
			}}
			
			\tikzset{cong7/.pic={%Đường cong 7
					\def\D7{ 
						(.74,1.94)
						..controls +(75:.31) and +(115:.28) ..  (-.28,2)--
						(-.28,1.94)
						..controls +(80:.12) and +(95:0.12) ..  (-.24,1.98)
						..controls +(80:.12) and +(95:0.12) ..  (-.14,2.04)
						..controls +(80:.12) and +(95:0.12) ..  (0.02,2.08)
						..controls +(80:.12) and +(95:0.12) ..  (0.2,2.08)
						..controls +(80:.12) and +(95:0.12) ..  (0.4,2.05)
						..controls +(80:.12) and +(95:0.12) ..  (0.56,2)
						..controls +(80:.12) and +(95:0.12) ..  (0.66,1.96)
						..controls +(70:.12) and +(95:0.12) ..  (0.74,1.9)--cycle
						;}
					\draw \D7;
					\fill[tumbleweed] \D7;
			}}
			
			\tikzset{cua/.pic={%Cửa
					\def\W{ 
						(-.7,-2.85)--(-.63,-2.25)
						..controls +(70:.1) and +(100:.1) ..  (-.46,-2.25)--(-.52,-2.85)
						--cycle
						;}
					\draw \W;
					\fill[tumbleweed] \W;
			}}
			
			\tikzset{vien/.pic={%viền ngoài
					\def\V{ 
						(.9,1.25)
						..controls +(110:.43) and +(110:.27) ..  (-.49,1.35)--(-.98,-2.85)--(.59,-2.85)--cycle
						
						(.74,1.94)
						..controls +(75:.31) and +(115:.28) ..  (-.28,2)--(-.32,1.54)
						..controls +(115:.2) and +(75:.13) ..  (.72,1.48)--cycle
						;}
					\draw \V;
					\fill[gray!50!] \V;
			}}
			
			\tikzset{soc/.pic={%Thanh sọc
					\def\S{ 
						(.82,1.36)--(.54,-1.85)--(.58,-1.85)--(.86,1.36)
						--cycle
						;}
					\draw \S;
					\fill[tumbleweed!40!] \S;
			}}
			
			\tikzset{soc2/.pic={%Thanh sọc lớn
					\def\S{ 
						(.54,-1.95)--(.44,-2.85)--(.48,-2.85)--(.58,-1.95)
						--cycle
						;}
					\draw \S;
					\fill[tumbleweed!40!] \S;
			}}
			
			\tikzset{soc3/.pic={%Thanh sọc
					\def\S{ 
						(.54,2)--(.57,2)--(.54,1.5)--(.51,1.5)
						--cycle
						(.66,2)--(.69,2)--(.66,1.5)--(.63,1.5)
						--cycle
						(-.26,2)--(-.23,2)--(-.26,1.5)--(-.29,1.5)
						--cycle
						;}
					\draw \S;
					\fill[tumbleweed!40!] \S;
			}}
			
			\tikzset{thoi/.pic={%Hình thoi
					\def\T{ 
						(0.15,-1.98)--(.24,-2.12)--(0.13,-2.22)--(0.04,-2.1)
						--cycle
						;}
					\draw \T;
					\fill[tumbleweed!40!] \T;
			}}
			
			\tikzset{cay/.pic={%Cây
					\def\T{ 
						(-1.73,-2.85)
						..controls +(40:.03) and +(40:.01) ..  (-1.74,-2.7)
						..controls +(140:.04) and +(60:.02) ..  (-1.78,-2.6)
						..controls +(140:.04) and +(60:.02) ..  (-1.8,-2.55)
						..controls +(100:.03) and +(70:.02) ..  (-1.82,-2.5)
						..controls +(140:.04) and +(60:.02) ..  (-1.815,-2.4)
						..controls +(140:.04) and +(60:.02) ..  (-1.818,-2.3)
						..controls +(85:.03) and +(-30:.03) ..  (-1.8,-2.2)
						..controls +(80:.04) and +(50:.02) ..  (-1.78,-2)
						..controls +(60:.04) and +(120:.02) ..  (-1.76,-1.9)
						..controls +(80:.03) and +(-100:.02) ..  (-1.74,-1.8)
						..controls +(80:.01) and +(-100:.02) ..  (-1.72,-1.9)
						..controls +(60:.02) and +(-120:.02) ..  (-1.7,-2)
						..controls +(-80:.02) and +(110:.03) ..  (-1.66,-2.2)
						..controls +(-100:.02) and +(60:.03) ..  (-1.64,-2.3)
						..controls +(-30:.02) and +(70:.02) ..  (-1.6,-2.44)
						..controls +(-100:.02) and +(70:.02) ..  (-1.63,-2.52)
						..controls +(-60:.02) and +(70:.01) ..  (-1.66,-2.6)
						..controls +(60:.01) and +(70:.02) ..  (-1.7,-2.7)
						..controls +(-120:.02) and +(120:.03) ..  (-1.68,-2.85)
						;}
					\draw \T;
					\fill[cadmiumgreen!90!] \T;
			}}
			\fill[cadmiumgreen!80!] (-4,-3.5) rectangle (8,-2.2);
			\fill[lightcornflowerblue] (-4,3.5) rectangle (8,-2.2);
			\path 
			(0,0)pic[scale=1]{cay}(.35,0)pic[scale=.9]{cay}(1.05,0.6)pic[scale=1.2]{cay}(-.5,-1)pic[scale=.6]{cay}
			(3,0)pic[scale=1]{cay}(2.8,0)pic[scale=1]{cay}
			(0,0)pic[scale=1]{vien}
			(0,0)pic[scale=1]{soc} (0,0)pic[scale=1]{soc}(-.15,0)pic[scale=1]{soc}(-.3,0)pic[scale=1]{soc}(-.45,0)pic[scale=1]{soc}(-.6,0)pic[scale=1]{soc}(-.75,0)pic[scale=1]{soc}(-.9,0)pic[scale=1]{soc}(-1.02,0)pic[scale=1]{soc}(-1.12,0)pic[scale=1]{soc}(-1.22,0)pic[scale=1]{soc}(-1.32,0)pic[scale=1]{soc}
			
			(0,0)pic[scale=1]{soc3}
			(0,0)pic[scale=1]{thap}(0,0)pic[scale=1]{co}
			(.4,3.6)pic[scale=.7,rotate=3]{cua}
			(.8,3.6)pic[scale=1.2,rotate=7,yscale=.6]{cua}
			(.3,3.4)pic[scale=1.1,rotate=7,yscale=.6]{cua}
			(0,-.04)pic[scale=1]{thoi}(0.25,-.05)pic[scale=1]{thoi}(-0.35,0)pic[scale=1]{thoi}(-.67,0)pic[scale=1]{thoi}(-.9,0)pic[scale=1]{thoi}
			(-1.22,0)pic[scale=1]{soc2}(-1.36,0)pic[scale=1]{soc2}
			(-.94,0)pic[scale=1]{soc2}(-.56,0)pic[scale=1]{soc2}
			(-.56,0)pic[scale=1]{soc2}(.08,0)pic[scale=1]{soc2}(-.04,0)pic[scale=1]{soc2}(-.28,0)pic[scale=1]{soc2}
			(0,0)pic[scale=1]{cong0} (0,0.02)pic[scale=1]{cong0}%
			(0,0)pic[scale=1]{cong1} (0,0.02)pic[scale=1]{cong1}%
			(0,0)pic[scale=1]{cong2} (0,0.02)pic[scale=1]{cong2}%
			(0,0)pic[scale=1]{cong3}(0,0.02)pic[scale=1]{cong3}
			(0,0)pic[scale=1]{cong4}(0,0.02)pic[scale=1]{cong4}
			(0,0)pic[scale=1]{cong5}(0,0.02)pic[scale=1]{cong5}
			(0,0)pic[scale=1]{cong6}(0,0.02)pic[scale=1]{cong6}
			(0,0)pic[scale=1]{cong7}(0,0.02)pic[scale=1]{cong7}
			(0,0)pic[scale=1]{cua}
			
			(0,0)pic[scale=1]{rao}
			;		
		\end{tikzpicture}
	}
	\loigiai{Mỗi khi chạm đất quả bóng lại nảy lên một độ cao bằng $\dfrac{1}{10}$ độ cao của lần rơi ngay trước đó và sau đó lại rơi xuống từ độ cao thứ hai này. Do đó, độ dài hành trình của quả bóng kể từ thời điểm rơi ban đầu đến:\\    
		Thời điểm chạm đất lần thứ nhất là $d_1=55{,}8$.\\
		Thời điềm chạm đất lần thứ hai là $d_2=55{,}8+2\cdot \dfrac{55{,}8}{10}$.\\
		Thời điểm chạm đất lần thứ ba là $d_3=55{,}8+2 \cdot\dfrac{55{,}8}{10}+2\cdot \dfrac{55{,}8}{10^2}$.\\
		Thời điểm chạm đất lần thứ tư là $d_4=55{,}8+2 \cdot\dfrac{55{,}8}{10}+2\cdot \dfrac{55,8}{10^2}+2\cdot \dfrac{55{,}8}{10^3}$.\\
		$\ldots$\\
		Thời điểm chạm đất lần thứ $n~(n>1)$ là
		$$d_n=55{,}8+2\cdot55{,}8+2\cdot \frac{55{,}8}{10^2}+2\cdot \frac{55{,}8}{10^3}+\ldots+2\cdot \frac{55{,}8}{10^{n-1}}.$$
		Do đó, quãng đường mà quả bóng đi được kể từ thời điềm rơi đến khi nằm yên trên mặt đất là:
		$$ d=55{,}8+2.55{,}8+2\cdot \frac{55{,}8}{10^2}+2\cdot \frac{55{,}8}{10^3}+\ldots+2\cdot \frac{55{,}8}{10^{n-1}}+\ldots=\lim \limits_{n \to +\infty}d_n.$$
		Vì $2\cdot \dfrac{55{,}8}{10} ; 2\cdot \dfrac{55{,}8}{10^2} ; 2\cdot \dfrac{55{,}8}{10^3}; \ldots ; 2\cdot \dfrac{55{,}8}{10^{n-1}}; \ldots$ là một cấp số nhân lùi vô hạn với công bội $q=\dfrac{1}{10}$ nên ta có:
		$$ 2 \cdot\dfrac{55,8}{10}+2\cdot \dfrac{55{,}8}{10^2}+2\cdot \dfrac{55{,}8}{10^3}+\ldots+2\cdot \dfrac{55{,}8}{10^{n-1}}+\ldots=\dfrac{2\cdot \dfrac{55{,}8}{10}}{1-\dfrac{1}{10}}=12{,}4.$$
		Vậy $d=55{,}8+12{,}4=68{,}2$ m.
	}
\end{vd}
\begin{vd}%[0D1Y1-1]
	Cho một tam giác đều $A B C$ cạnh $a$. Tam giác $A_1 B_1 C_1$ có các đỉnh là trung điểm các cạnh của tam giác $A B C$, tam giác $A_2 B_2 C_2$ có các đỉnh là trung điểm các cạnh của tam giác $A_1 B_1 C_1, \ldots$, tam giác $A_{n+1} B_{n+1} C_{n+1}$ có các đỉnh là trung điểm các cạnh của tam giác $A_n B_n C_n, \ldots$ Gọi $p_1, p_2, \ldots, p_n, \ldots$ và $S_1, S_2, \ldots, S_n, \ldots$ theo thứ tự là chu vi và diện tích của các tam giác $A_1 B_1 C_1, A_2 B_2 C_2, \ldots, A_n B_n C_n, \ldots$.
	\begin{listEX}
		\item[a)] Tìm giới hạn của các dãy số $\left(p_n\right)$ và $\left(S_n\right)$.
		\item[b)] Tìm các tổng $p_1+p_2+\ldots+p_n+\ldots$ và $S_1+S_2+\ldots+S_n+\ldots$.
	\end{listEX}
	\loigiai{
		\begin{listEX}
			\item[a)] Ta có $p_1, p_2, \ldots, p_n, \ldots$ lần lượt là chu vi của các tam giác $A_1 B_1 C_1, A_2 B_2 C_2, \ldots, A_n B_n C_n, \ldots$
			$$\begin{aligned}
				& p_1=3 a \\
				& p_2=3 \cdot \frac{1}{2} a \\
				& \ldots \\
				& p_n=3 \cdot \frac{1}{2^{n-1}} a 
			\end{aligned} $$
			suy ra $\lim \limits_{n \to +\infty}p_n=\lim \limits_{n \to +\infty}3 \cdot \dfrac{1}{2^{n-1}} a=0$.
			$$ \begin{aligned}
				& S_1=\frac{a^2 \sqrt{3}}{4} \\
				& S_2=\frac{1}{4} \frac{a^2 \sqrt{3}}{4} \\
				& \ldots \\
				& S_n=\frac{1}{4^{n-1}} \cdot \frac{a^2 \sqrt{3}}{4}
			\end{aligned}$$
			suy ra $\lim \limits_{n \to +\infty}S_n=\lim \limits_{n \to +\infty}\dfrac{1}{4^{n-1}} \cdot \dfrac{a^2 \sqrt{3}}{4}=0$.
			\item[b)] Dựa vào dữ kiện đề bài suy ra tổng $\left(p_n\right)$ là tổng của cấp số nhân lùi vô hạn với công bội $q=\dfrac{1}{2}$ và
			$ p_1+p_2+\ldots+p_n+\ldots=\lim \limits_{n \to +\infty}\left(p_n\right)=\dfrac{p_1}{1-q}=\dfrac{3 a}{1-\frac{1}{2}}=6a.$\\
			Dựa vào dữ kiện đề bài suy ra tổng $\left(S_n\right)$ là tổng của cấp số nhân lùi vô hạn với công bội $q=\dfrac{1}{4}$ và
			$S_1+S_2+\ldots+S_n+\ldots=\lim \limits_{n \to +\infty}\left(S_n\right)=\dfrac{S_1}{1-q}=\dfrac{\frac{a^2 \sqrt{3}}{4}}{1-\frac{1}{4}}=\dfrac{a^2 \sqrt{3}}{12}$.
		\end{listEX}
	}
\end{vd}
% \subsubsection{Bài tập rèn luyện} 
% % \subsubsection{Bài tập tự luận}
% \begin{bt}%[1T3K1-5]
% 	Từ tờ giấy, cắt một hình tròn bán kính $R$ (cm) như Hình $3a$. Tiếp theo, cắt hai hình~tròn
% 	\immini{
% 		bán kính $\dfrac{R}{2}$ rồi chồng lên hình tròn đầu tiên như Hình $3b$. Tiếp theo, cắt bốn hình tròn bán  kính $\dfrac{R}{4}$ 
% 		rồi chồng lên các hình trước như Hình $3c$. Cứ thế tiếp tục mãi. Tính tổng diện tích của các hình tròn.
% 	}{
% 		\begin{tikzpicture}[>=stealth,line join=round,line cap=round,font=\footnotesize,scale=1,declare function={r=1.5;}]
% 			\begin{scope}
% 				\draw[fill=blue] (0,0)circle(r);
% 				\path (0,-r) node[below]{$a)$};
% 			\end{scope}
% 			\begin{scope}[xshift={3.5cm}]
% 				\draw[fill=blue] (0,0)circle(r);
% 				\draw[fill=yellow] (r/2,0)circle(r/2);
% 				\draw[fill=yellow] (-r/2,0)circle(r/2);
% 				\path (0,-r) node[below]{$b)$};
% 			\end{scope}
% 			\begin{scope}[xshift={7cm}]
% 				\draw[fill=blue] (0,0)circle(r);
% 				\draw[fill=yellow] (r/2,0)circle(r/2);
% 				\draw[fill=yellow] (-r/2,0)circle(r/2);
% 				\foreach \i in {0,1,2,3}
% 				\draw[fill=green,shift={(r/2*\i,0)}] (-3*r/4,0)circle(r/4);
% 				\path (0,-r) node[below]{$c)$};
% 			\end{scope}
% 			\path (current bounding box.south) node[below]{Hình $3$};
% 		\end{tikzpicture}
% 	}
% 	\loigiai{
% 		Diện tích của các hình tròn trong các lần cắt là
% 		\begin{enumerate}
% 			\item Lần thứ 1: $S_1=\pi R^2$.
% 			\item  Lần thứ 2: $S_2=2\cdot \pi \left(\dfrac{R}{2}\right)^2= \dfrac{\pi R^2}{2}$.
% 			\item  Lần thứ 3: $S_2=4\cdot \pi \left(\dfrac{R}{4}\right)^2= \dfrac{\pi R^2}{2^2}$.	
% 			\item Lần thứ $n$: $S_n= \dfrac{\pi R^2}{2^{n-1}}$.
% 		\end{enumerate}
% 		Do đó  diện tích các hình tròn lập thành một cấp số nhân lùi vô hạn có số hạng đầu $S_1=\pi R^2$ và công bội $q=\dfrac{1}{2}$ nên tổng diện tích các hình tròn là 
% 		\[ S_1+S_2+\cdots=\dfrac{\pi R^2}{1-\dfrac{1}{2}}=2\pi R^2. \]
% 	}
% \end{bt}
% \begin{bt}%[1T3K1-5]
% 	\immini{
% 		Từ hình vuông đầu tiên có cạnh bằng $1$ (đơn vị độ dài), nối các trung điểm của bốn cạnh để có hình vuông thứ hai. Tiếp tục nối các trung điểm của bốn cạnh của hình vuông thứ hai để được hình vuông thứ ba. Cứ tiếp tục làm như thế, nhận được một dãy hình vuông (xem Hình $5$).
% 	}{\hspace*{.5cm}
% 		\begin{tikzpicture}[>=stealth,line join=round,line cap=round,font=\footnotesize,scale=1,declare function={a=3;}]
% 			\foreach \i in {1,2,...,7}{
% 				\pgfmathsetmacro\r{a*(sin(45))^(\i-1)}
% 				\pgfmathsetmacro{\j}{int(mod(\i,2))}
% 				\ifnum \j=1
% 				\draw (-\r/2,-\r/2) rectangle (\r/2,\r/2);
% 				\else
% 				\draw[rotate=45] (-\r/2,-\r/2) rectangle (\r/2,\r/2);
% 				\fi
% 			}
% 			\path (current bounding box.south) node[below]{Hình $5$};
% 		\end{tikzpicture}\hspace*{1cm}
% 	}
% 	\begin{enumerate}
% 		\item Kí hiệu $a_n$ là diện tích của hình vuông thứ $n$ và $S_n$ là tổng diện tích của $n$ hình vuông đầu tiên. Viết công thức tính $a_n$, $S_n$ ($n=1,2,3, \ldots$) và tìm $\lim \limits_{n \to +\infty}S_n$ (giới hạn này nếu có được gọi là tổng diện tích của các hình vuông).
% 		\item Kí hiệu $p_n$ là chu vi của hình vuông thứ $n$ và $Q_n$ là tổng chu vi của $n$ hình vuông đầu tiên. Viết công thức tính $p_n$ và $Q_n$ $(n=1,2,3, \ldots)$ và tìm $\lim \limits_{n \to +\infty}Q_n$ (giới hạn này nếu có được gọi là tổng chu vi của các hình vuông).
% 	\end{enumerate}
% 	\loigiai{
% 		\begin{enumerate}
% 			\item Ta có hình vuông thứ nhất có cạnh bằng $1$,
% 			hình vuông thứ hai có cạnh bằng $\dfrac{\sqrt{2}}{2}$.\\
% 			Hình vuông thứ ba có cạnh bằng $\dfrac{1}{2}$.\\
% 			Suy ra	hình vuông thứ $n$ có cạnh bằng $\left(\dfrac{\sqrt{2}}{2}\right)^{n-1}$.\\
% 			Diện tích của hình vuông thứ $n$ là $a_n=\left(\dfrac{\sqrt{2}}{2}\right)^{n-1}\cdot \left(\dfrac{\sqrt{2}}{2}\right)^{n-1}=\left(\dfrac{1}{2}\right)^{n-1}$.\\
% 			Tổng diện tích của $n$ hình vuông đầu tiên là tổng của cấp số nhân có  số hạng đầu $a_1=1$ và công bội $q=\dfrac{1}{2}$ nên
% 			\[ S_n=\dfrac{a_1\left(1-q^n\right)}{1-q}=\dfrac{1-\left(\dfrac{1}{2}\right)^n}{1-\dfrac{1}{2}}=2\left[1-\left(\dfrac{1}{2}\right)^n\right].\]
% 			$\lim \limits_{n \to +\infty}S_n=\lim \limits_{n \to +\infty}2\left[1-\left(\dfrac{1}{2}\right)^n\right]=2 \left[\lim \limits_{n \to +\infty}1-\lim \limits_{n \to +\infty}\left(\dfrac{1}{2}\right)^n\right]=2 $.
% 			\item Hình vuông thứ nhất có chu vi bằng $4$, hình vuông thứ $2$ có chu vi là $2\sqrt{2}$, hình vuông thứ $3$ có chu vi là $2$.\\
% 			Suy ra hình vuông thứ $n$ có chu vi bằng $p_n=4\cdot \left(\dfrac{\sqrt{2}}{2}\right)^{n-1}$.
% 			Tổng chu vi của $n$ hình vuông đầu tiên là tổng của cấp số nhân có  số hạng đầu $p_1=4$ và công bội $q=\dfrac{\sqrt{2}}{2}$ nên 
% 			\[ Q_n=\dfrac{p_1\left(1-q^n\right)}{1-q}=\dfrac{4\left(1-\left(\dfrac{\sqrt{2}}{2}\right)^n\right)}{1-\dfrac{\sqrt{2}}{2}}=\left(8+4\sqrt{2}\right)\left[1-\left(\dfrac{\sqrt{2}}{2}\right)^n\right].\]
% 			$\lim \limits_{n \to +\infty}Q_n=\left(8+4\sqrt{2}\right)\lim \limits_{n \to +\infty}\left[1-\left(\dfrac{\sqrt{2}}{2}\right)^n\right]=\left(8+4\sqrt{2}\right)\left(1-0\right)=8+4\sqrt{2}$.
% 		\end{enumerate}	
% 	}
% \end{bt}

% \begin{bt}%[1T3K1-4]
% 	Xét quá trình tạo ra hình có chu vi vô cực và diện tích bằng $0$ như sau:\\ Bắt đầu bằng một hình vuông $H_0$ cạnh bằng 1 đơn vị độ dài (xem Hình $6a$). Chia hình vuông $H_0$ thành chín hình vuông bằng nhau, bỏ đi bốn hình vuông, nhận được hình $H_1$ (xem Hình $6b$). Tiếp theo, chia mỗi hình vuông của $H_1$ thành chín hình vuông, rồi bỏ đi bốn hình vuông, nhận được hình $H_2$ (xem Hình $6c$). Tiếp tục quá trình này, ta nhận được một dãy hình $H_n$ $(n=1,2,3,\ldots)$.	
% 	\\[1mm]
% 	\centerline{
% 		\begin{tikzpicture}% Muốn vẽ hình Hn thì dùng \hv{n}
% 			\def\a{2}
% 			\pgfmathsetmacro\sh{2*\a *sqrt(2)/3}
% 			\def\hv#1{
% 				\ifnum#1>0
% 				\draw[white,fill=white] 
% 				(-\a/3,\a/3) rectangle (\a/3,\a)
% 				(-\a/3,-\a/3) rectangle (-\a,\a/3)
% 				(-\a/3,-\a/3) rectangle (\a/3,-\a)
% 				(\a/3,-\a/3) rectangle (\a,\a/3)
% 				;
% 				\pgfmathtruncatemacro{\k}{#1-1}
% 				\begin{scope}[scale=1/3]\hv{\k}\end{scope}
% 				\begin{scope}[shift={(45:\sh)},scale=1/3]\hv{\k}\end{scope}
% 				\begin{scope}[shift={(135:\sh)},scale=1/3]\hv{\k}\end{scope}
% 				\begin{scope}[shift={(225:\sh)},scale=1/3]\hv{\k}\end{scope}
% 				\begin{scope}[shift={(315:\sh)},scale=1/3]\hv{\k}\end{scope}
% 				\fi
% 			}
% 			\begin{scope}
% 				\fill[green] (-\a,-\a) rectangle (\a,\a);
% 				\hv{0}
% 				\path (0,-\a)node[below]{$H_0$}
% 				node[below=.5cm]{$a)$};
% 			\end{scope}
% 			\begin{scope}[xshift=4.5cm]
% 				\fill[green] (-\a,-\a) rectangle (\a,\a);
% 				\hv{1}
% 				\path (0,-\a)node[below]{$H_1$}
% 				node[below=.5cm]{$b)$};
% 			\end{scope}
% 			\begin{scope}[xshift=9cm]
% 				\fill[green] (-\a,-\a) rectangle (\a,\a);
% 				\hv{2}
% 				\path (0,-\a)node[below]{$H_2$}
% 				node[below=.5cm]{$c)$};
% 			\end{scope}
% 			\begin{scope}[xshift=13.5cm]
% 				\fill[green] (-\a,-\a) rectangle (\a,\a);
% 				\hv{3}
% 				\path (0,-\a)node[below]{$H_3$}
% 				node[below=.5cm]{$d)$};
% 			\end{scope}
% 			\path (current bounding box.south) node[below]{Hình $6$};
% 		\end{tikzpicture}
% 	}
% 	Ta có: $H_1$ có $5$ hình vuông, mỗi hình vuông có cạnh bằng $\dfrac{1}{3}$;\\
% 	{\color{white}{Ta có: }}$H_2$ có $5\cdot5=5^2$ hình vuông, mỗi hình vuông có cạnh bằng $\dfrac{1}{3} \cdot \dfrac{1}{3}=\dfrac{1}{3^2}; \ldots$.\\
% 	Từ đó, nhận được $H_n$ có $5^n$ hình vuông, mỗi hình vuông có cạnh bằng $\dfrac{1}{3^n}$.
% 	\begin{enumerate}
% 		\item Tính diện tích $S_n$ của $H_n$ và tính $\lim \limits_{n \to +\infty}S_n$.
% 		\item Tính chu vi $p_n$ của $H_n$ và tính $\lim \limits_{n \to +\infty}p_n$.
% 	\end{enumerate}
% 	(Quá trình trên tạo nên một hình, gọi là một fractal, được coi là có diện tích $\lim \limits_{n \to +\infty}S_n$ và chu vi $\lim \limits_{n \to +\infty}p_n$).
% 	\loigiai{
% 		\begin{enumerate}
% 			\item Hình vuông $H_1$ có diện tích $S_1=5\cdot \left(\dfrac{1}{3}\right)^2=\dfrac{5}{9}$.\\
% 			Hình vuông $H_2$ có diện tích $S_2=5^2\cdot \left(\dfrac{1}{3^2}\right)^2=\left(\dfrac{5}{9}\right)^2$.\\
% 			Hình vuông $H_n$ có diện tích $S_n=5^n\cdot \left(\dfrac{1}{3^n}\right)^2=\left(\dfrac{5}{9}\right)^n$.\\
% 			$\lim \limits_{n \to +\infty}S_n=\lim \limits_{n \to +\infty}\left(\dfrac{5}{9}\right)^n=0$.
% 			\item Hình vuông $H_1$ có chu vi $p_1=5\cdot 4\cdot  \dfrac{1}{3}=4\cdot \dfrac{5}{3}$.\\
% 			Hình vuông $H_2$ có chu vi $p_2=5^2\cdot4\cdot \dfrac{1}{3^2}=4\cdot \left(\dfrac{5}{3}\right)^2$.\\
% 			Hình vuông $H_n$ có diện tích $p_n=5^n\cdot4\cdot  \dfrac{1}{3^n}=4\cdot \left(\dfrac{5}{3}\right)^n$.\\
% 			$\lim \limits_{n \to +\infty}p_n=\lim \limits_{n \to +\infty}4\cdot \left(\dfrac{5}{3}\right)^n=+\infty$.
% 		\end{enumerate}
% 	}
% \end{bt}
% \begin{bt}%[1T3K1-5]
% 	\immini{Cho tam giác đều có cạnh bằng $a$, gọi là tam giác $H_1$. Nối các trung điểm của $H_1$ để tạo thành tam giác $\mathrm{H}_2$. Tiếp theo, nối các trung điểm của $\mathrm{H}_2$ để tạo thành tam giác $\mathrm{H}_3$ (Hình bên). Cứ tiếp tục như vậy, nhận được dãy tam giác $H_1, H_2, H_3, \ldots$\\
% 		Tính tổng chu vi và tổng diện tích các tam giác của dãy.}{
% 		\begin{tikzpicture}[scale=1, font=\footnotesize, line join=round, line cap=round, >=stealth]
% 			(0,0) coordinate (A)	
% 			(4,0) coordinate (B)	
% 			(2,2.8284) coordinate (C)
% 			\draw (A)--(B)--(C)--cycle;
% 			\coordinate (I) at ($(A)!0.5!(B)$);
% 			\coordinate (J) at ($(B)!0.5!(C)$);
% 			\coordinate (K) at ($(C)!0.5!(A)$);
% 			\draw (I)--(J)--(K)--cycle;
% 			\coordinate (H) at ($(I)!0.5!(J)$);
% 			\coordinate (G) at ($(J)!0.5!(K)$);
% 			\coordinate (F) at ($(K)!0.5!(I)$);
% 			\draw (H)--(G)--(F)--cycle;
% 			\coordinate (X) at ($(H)!0.5!(G)$);
% 			\coordinate (Y) at ($(G)!0.5!(F)$);
% 			\coordinate (Z) at ($(F)!0.5!(H)$);
% 			\draw (X)--(Y)--(Z)--cycle;
% 			\draw[dashed] (2,2.8284)--(3.3,1.45)node[above right]{$a$}--(4,0);
% 		\end{tikzpicture}
% 	}
% 	\loigiai{
% 		Gọi $S_i$ và $C_i$ ($i=1,2,\ldots$) lần lượt là diện tích và chu vi của tam giác $H_i$, $i=1,2,\ldots $.\\
% 		Khi đó ta có
% 		\begin{itemize}
% 			\item[$\bullet$] $S_1=\dfrac{a^2\sqrt{3}}{4};S_2=\left(\dfrac{a}{2}\right)^2\dfrac{\sqrt{3}}{4}=\dfrac{a^2\sqrt{3}}{16}=\dfrac{S_1}{4};S_3=\dfrac{1}{4}S_2,\ldots $.\\
% 			Do đó $(S_n)$ là một cấp số nhân lùi vô hạn với $S_1=\dfrac{a^2\sqrt{3}}{4}$ và $q_1=\dfrac{1}{4}$.\\
% 			Tổng diện tích $S=S_1+S_2+\cdots =\dfrac{S_1}{1-q_1}=\dfrac{a^2\sqrt{3}}{3}$.
% 			\item[$\bullet$] $C_1=3a; C_2=\dfrac{3a}{2}=\dfrac{1}{2}C_1,\ldots$\\
% 			Do đó $(C_n)$ là một cấp số nhân lùi vô hạn với $C_1=3a;q_2=\dfrac{1}{2}$.\\
% 			Tổng chu vi là $C=C_1+C_2+\cdots =\dfrac{C_1}{1-q_2}=6a$.
% 		\end{itemize}
% 	}	
% \end{bt}
% \begin{bt}%[1D4K1-6]
% 	Một thấu kính hội tụ có tiêu cự là $f$. Gọi $d$ và $d'$ lần lượt là khoảng cách từ một vật thật $A B$ và từ ảnh $A' B'$ của nó tới quang tâm $O$ của thấu kính như hình vẽ bên dưới. Công thức thấu kính là $\dfrac{1}{d}+\dfrac{1}{d}=\dfrac{1}{f}$.
% 	\begin{center}
% 		\begin{tikzpicture}[scale=0.6, font=\footnotesize, line join=round, line cap=round, >=stealth]		
% 			%\draw[domain=-3.65:3.65, blue,] plot({3/10*(\x)^2},\x);
% 			%\draw[gray!10] (-1,-1) grid (5,5);
% 			\draw [red] (-10,0)--(9,0);
% 			\draw [dashed] (0,-3)--(0,2.5);
% 			\coordinate (A) at (-10,0); 
% 			\coordinate (A') at (7,0); 
% 			\coordinate (B) at (-10,1); 
% 			\coordinate (L) at ( 0,1);
% 			\coordinate (O) at (0,0); 
% 			\coordinate (B') at (7,-1); 
% 			\coordinate (F') at (5,0); 
% 			\coordinate (F) at (-5,0); 
% 			\coordinate (H) at (-5,1); 
% 			\coordinate (M) at (5,2); 
% 			\coordinate (M') at (-5,2); 
% 			\coordinate (S) at (-10,-3); 
% 			\coordinate (K) at (7,-3);
% 			\coordinate (K') at (-5.5,0.55); 
% 			\coordinate (N) at (5,-0.74); 
% 			\coordinate (G) at (2,0.45); 
% 			\coordinate (Q) at (-2.5,0.9); 
% 			\coordinate (Q') at (2.5,0.9);
% 			\coordinate (J) at (-5,-2.9); 
% 			\coordinate (J') at (5,-2.9);
% 			\coordinate (X) at (-8.5,0);
% 			\coordinate (V) at (0,-3); \coordinate (V') at (0,2);
% 			%\draw  (F)--(D) (G)--(C) (F')--(D') (G')--(C') ;
% 			\draw (B)--(L)--(B');
% 			\draw (B)--(O)--(B');\draw[red,->] (A)--(X);
% 			\draw[blue,->] (B)--(H); \draw[blue,->] (B)--(K');
% 			\draw[blue,->] (A)--(B); \draw[blue,->] (O)--(N);
% 			\draw[blue,->] (L)--(G); \draw[red,->] (A')--(B');
% 			\draw[dashed] (F)--(M')--(M)--(F');
% 			\draw[dashed] (A)--(S)--(K)--(B');
% 			\draw[dashed,<->] (S)--(V);\draw[dashed,<->] (V)--(K) ;
% 			\draw[dashed,<->] (M')--(V');\draw[dashed,<->] (V')--(M) ;
% 			%\draw[red,->] (E)--(H3); 
% 			%\draw[->] (C)--(K3) ;
% 			%\draw[red,->] (E)--(H4); 
% 			%\draw[->] (C')--(K4) ;
% 			%\draw[red,->] (E)--(H2); 
% 			%\draw[->] (D')--(K2) ;
			
% 			%\draw[red] (E)--(C) (E)--(D) (E)--(C') (E)--(D');
% 			\draw[fill=black] (A') circle (1pt) node [above] {$A'$};
% 			\draw (Q) node [above] {$f$}; \draw (Q') node [above] {$f$}; \draw (J) node [above] {$d$}; \draw (J') node [above] {$d'$};
% 			\draw[fill=black] (B') circle (1pt) node [right] {$B'$};
% 			\draw[fill=black] (F') circle (1pt) node [above right] {$F'$};
% 			\draw[fill=black] (F) circle (1pt) node [below] {$F$};
% 			\draw[fill=black] (A) circle (1pt) node [left] {$A$};
% 			\draw[fill=black] (B) circle (1pt) node [left] {$B$};
% 			%\draw[fill=black] (E) circle (1pt) node [below right] {$S$};
% 			\draw[fill=black] (O) circle (1pt) node [below left] {$O$};
% 		\end{tikzpicture}
% 	\end{center}
% 	\begin{listEX}
% 		\item[a)] Tìm biểu thức xác định hàm số $d'=\varphi(d)$.
% 		\item[b)] Tìm $\underset{d \rightarrow f^{+}}\lim \limits_{n \to +\infty}\varphi(d), \underset{d \rightarrow f^{-}}\lim \limits_{n \to +\infty}\varphi(d)$ và $\underset{d \rightarrow f}\lim \limits_{n \to +\infty}\varphi(d)$. Giải thích ý nghĩa của các kết quả tìm được.
% 	\end{listEX}
% 	\loigiai{
% 		\begin{listEX}
% 			\item[a)] Ta có $$\dfrac{1}{d}+\dfrac{1}{d'}=\dfrac{1}{f} \Leftrightarrow d'=\dfrac{d f}{d-f}.$$
% 			Vậy $\varphi(d)=\dfrac{d f}{d-f}$.		
% 			\item[b)] Vì $\underset{d \rightarrow f^{+}}\lim \limits_{n \to +\infty}df=f^2; \underset{d \rightarrow f^{+}}\lim \limits_{n \to +\infty}(d-f)=0 ; d \rightarrow f^{+} \Rightarrow d-f>0 $ nên $ \underset{d \rightarrow f^{+}}\lim \limits_{n \to +\infty} \dfrac{d f}{d-f}=+\infty$.\\
% 			Vậy $\underset{d \rightarrow f^{+}}\lim \limits_{n \to +\infty}\varphi(d)=\underset{d \rightarrow f^{+}}\lim \limits_{n \to +\infty} \dfrac{d f}{d-f}=+\infty$.\\
% 			\textbf{Ý nghĩa}: Khi đặt vật nằm ngoài tiêu cự và tiến dần đến tiêu điểm thì cho ảnh thật ngược chiều với vật ở vô cùng.\\
% 			Vì $\underset{d \rightarrow f^{-}}\lim \limits_{n \to +\infty}df=f^2; \underset{d \rightarrow f^{+}}\lim \limits_{n \to +\infty}(d-f)=0 ; d \rightarrow f^{-} \Rightarrow d-f<0 $ nên $ \underset{d \rightarrow f^{+}}\lim \limits_{n \to +\infty} \dfrac{d f}{d-f}=-\infty$.\\
% 			Vậy $\underset{d \rightarrow f^{+}}\lim \limits_{n \to +\infty}\varphi(d)=\underset{d \rightarrow f^{+}}\lim \limits_{n \to +\infty} \dfrac{d f}{d-f}=-\infty$.\\
% 			\textbf{Ý nghĩa}: Khi đặt vật nằm trong tiêu cự và tiến dần đến tiêu điểm thì cho ảnh ảo cùng chiều với vật và nằm ở vô cùng.\\
% 			Vì không tồn tại $\underset{d \rightarrow f^{+}}\lim \limits_{n \to +\infty}\varphi(d)$ và $\underset{d \rightarrow f^{-}}\lim \limits_{n \to +\infty}\varphi(d)$ nên không tồn tại $\underset{d \rightarrow f}\lim \limits_{n \to +\infty}\varphi(d)$.
% 		\end{listEX}
% 	}
% \end{bt}
\subsubsection{Câu hỏi trắc nghiệm}
\Opensolutionfile{ans}[ans/ans-1K5-1-Dang5]
\begin{ex}%[1C3K1-6]
	Có $1$ kg chất phóng xạ độc hại. Biết rằng, cứ sau một khoảng thời gian $T=24000$ năm thì một nửa số chất phóng xạ này bị phân rã thành chất khác không độc hại đối với sức khỏe của con người ($T$ được gọi là \textit{chu kì bán rã}). Gọi $u_n$ là khối lượng chất phóng xạ còn lại sau chu kì thứ $n$.
	Sau ít nhất bao nhiêu chu kì bán rã thì khối lượng phóng xạ đã cho ban đầu không còn độc hại với con người, biết rằng chất phóng xạ này sẽ không độc hại nữa nếu khối lượng chất phóng xạ còn lại bé hơn $10^{-6}$ g.
	\choice
	{$24$}
	{\True $30$}
	{$100$}
	{$15$}
	\loigiai{
		\item Chất phóng xạ sẽ không độc hại nữa nếu khối lượng chất phóng xạ còn lại bé hơn $10^{-6}~\mathrm{g}=10^{-9}$ kg
		$$\Leftrightarrow u_n<10^{-9}\Leftrightarrow\dfrac{1}{2^n}<10^{-9}\Leftrightarrow 2^n>10^9\Leftrightarrow n\geq 30.$$
		Vậy sau ít nhất $30$ chu kì thì khối lượng phóng xạ đã cho ban đầu không còn độc hại với con người nữa.
	}
\end{ex}
\begin{ex}%[1C3K1-6]
	Có $1$ kg chất phóng xạ độc hại. Biết rằng, cứ sau một khoảng thời gian $T=24000$ năm thì một nửa số chất phóng xạ này bị phân rã thành chất khác không độc hại đối với sức khỏe của con người ($T$ được gọi là \textit{chu kì bán rã}). Gọi $u_n$ là khối lượng chất phóng xạ còn lại sau chu kì thứ $n$.
	Sau ít nhất bao nhiêu năm thì khối lượng phóng xạ đã cho ban đầu không còn độc hại với con người, biết rằng chất phóng xạ này sẽ không độc hại nữa nếu khối lượng chất phóng xạ còn lại bé hơn $10^{-6}$ g.
	\choice
	{$30$}
	{$2400$}
	{\True $720000$}
	{$10000$}
	\loigiai{
		\item Chất phóng xạ sẽ không độc hại nữa nếu khối lượng chất phóng xạ còn lại bé hơn $10^{-6}~\mathrm{g}=10^{-9}$ kg
		$$\Leftrightarrow u_n<10^{-9}\Leftrightarrow\dfrac{1}{2^n}<10^{-9}\Leftrightarrow 2^n>10^9\Leftrightarrow n\geq 30.$$
		Vậy sau ít nhất $30$ chu kì bằng $30\cdot 24000=720000$ năm thì khối lượng phóng xạ đã cho ban đầu không còn độc hại với con người nữa.
	}
\end{ex}
\begin{ex}%[1C3K1-6]
	Từ hình vuông có độ dài cạnh bằng $1$, người ta nối các trung điểm của cạnh hình vuông để tạo ra hình vuông mới như hình bên. Tiếp tục quá trình này đến vô hạn. Tính tổng diện tích của tất cả các hình vuông được tạo thành.
	\choice
	{$1$}
	{\True $2$}
	{$3$}
	{$4$}	
	\loigiai{
		Từ giả thiết suy ra diện tích hình vuông sau bằng $\dfrac{1}{2}$ diện tích hình vuông trước.\\ 
		Khi đó diện tích của các hình vuông tạo thành một cấp số nhân lùi vô hạn với số hạng đầu $S_1=1$ và công bội $q=\dfrac{1}{2}$.\\
		Diện tích $S_n$ của hình vuông được tạo thành từ bước thứ $n$ là $S_n=S_1\cdot q^{n-1}=\left(\dfrac{1}{2}\right)^{n-1}$.\\
		Tổng diện tích của tất cả các hình vuông được tạo thành là:\\
		$$S=\dfrac{u_1}{1-q}=\dfrac{1}{1-\dfrac{1}{2}}=2.$$		
	}
\end{ex}
\begin{ex}%[1T3K1-5]
	Từ hình vuông đầu tiên có cạnh bằng $1$ (đơn vị độ dài), nối các trung điểm của bốn cạnh để có hình vuông thứ hai. Tiếp tục nối các trung điểm của bốn cạnh của hình vuông thứ hai để được hình vuông thứ ba. Cứ tiếp tục làm như thế, nhận được một dãy hình vuông. Tính tổng chu vi của dãy các hình vuông trên. 
	\choice
	{$8+\sqrt{2}$}
	{$2+\sqrt{2}$}
	{\True $8+4\sqrt{2}$}
	{$4+4\sqrt{2}$}
	\loigiai{
		Hình vuông thứ nhất có chu vi bằng $4$, hình vuông thứ $2$ có chu vi là $2\sqrt{2}$, hình vuông thứ $3$ có chu vi là $2$.\\
		Suy ra hình vuông thứ $n$ có chu vi bằng $p_n=4\cdot \left(\dfrac{\sqrt{2}}{2}\right)^{n-1}$.
		Tổng chu vi của $n$ hình vuông đầu tiên là tổng của cấp số nhân có  số hạng đầu $p_1=4$ và công bội $q=\dfrac{\sqrt{2}}{2}$ nên 
		\[ Q_n=\dfrac{p_1\left(1-q^n\right)}{1-q}=\dfrac{4\left(1-\left(\dfrac{\sqrt{2}}{2}\right)^n\right)}{1-\dfrac{\sqrt{2}}{2}}=\left(8+4\sqrt{2}\right)\left[1-\left(\dfrac{\sqrt{2}}{2}\right)^n\right].\]
		$\lim \limits_{n \to +\infty}Q_n=\left(8+4\sqrt{2}\right)\lim \limits_{n \to +\infty}\left[1-\left(\dfrac{\sqrt{2}}{2}\right)^n\right]=\left(8+4\sqrt{2}\right)\left(1-0\right)=8+4\sqrt{2}$.
	}
\end{ex}
\begin{ex}%[1D4K1-6]
	Từ độ cao $55,8 \mathrm{~m}$ của tháp nghiêng Pisa nước Ý, người ta thả một quả bóng cao su chạm xuống đất hình bên dưới. Giả sử mỗi lần chạm đất quả bóng lại nảy lên độ cao bằng $\dfrac{1}{10}$ độ cao mà quả bóng đạt được trước đó. Gọi $S_n$ là tổng độ dài quãng đường di chuyển của quả bóng tính từ lúc thả ban đầu cho đến khi quả bóng đó chạm đất $n$ lần. Tính $\lim \limits_{n \to +\infty}S_n$.
	\choice
	{$58{,}8$}
	{$67{,}2$}
	{$68$}
	{\True $68{,}2$}
	\loigiai{Mỗi khi chạm đất quả bóng lại nảy lên một độ cao bằng $\dfrac{1}{10}$ độ cao của lần rơi ngay trước đó và sau đó lại rơi xuống từ độ cao thứ hai này. Do đó, độ dài hành trình của quả bóng kể từ thời điểm rơi ban đầu đến:\\    
		Thời điểm chạm đất lần thứ nhất là $d_1=55{,}8$.\\
		Thời điềm chạm đất lần thứ hai là $d_2=55{,}8+2\cdot \dfrac{55{,}8}{10}$.\\
		Thời điểm chạm đất lần thứ ba là $d_3=55{,}8+2 \cdot\dfrac{55{,}8}{10}+2\cdot \dfrac{55{,}8}{10^2}$.\\
		Thời điểm chạm đất lần thứ tư là $d_4=55{,}8+2 \cdot\dfrac{55{,}8}{10}+2\cdot \dfrac{55,8}{10^2}+2\cdot \dfrac{55{,}8}{10^3}$.\\
		$\ldots$\\
		Thời điểm chạm đất lần thứ $n~(n>1)$ là
		$$d_n=55{,}8+2\cdot55{,}8+2\cdot \frac{55{,}8}{10^2}+2\cdot \frac{55{,}8}{10^3}+\ldots+2\cdot \frac{55{,}8}{10^{n-1}}.$$
		Do đó, quãng đường mà quả bóng đi được kể từ thời điềm rơi đến khi nằm yên trên mặt đất là:
		$$ d=55{,}8+2.55{,}8+2\cdot \frac{55{,}8}{10^2}+2\cdot \frac{55{,}8}{10^3}+\ldots+2\cdot \frac{55{,}8}{10^{n-1}}+\ldots=\lim \limits_{n \to +\infty}d_n.$$
		Vì $2\cdot \dfrac{55{,}8}{10} ; 2\cdot \dfrac{55{,}8}{10^2} ; 2\cdot \dfrac{55{,}8}{10^3}; \ldots ; 2\cdot \dfrac{55{,}8}{10^{n-1}}; \ldots$ là một cấp số nhân lùi vô hạn với công bội $q=\dfrac{1}{10}$ nên ta có:
		$$ 2 \cdot\dfrac{55,8}{10}+2\cdot \dfrac{55{,}8}{10^2}+2\cdot \dfrac{55{,}8}{10^3}+\ldots+2\cdot \dfrac{55{,}8}{10^{n-1}}+\ldots=\dfrac{2\cdot \dfrac{55{,}8}{10}}{1-\dfrac{1}{10}}=12{,}4.$$
		Vậy $d=55{,}8+12{,}4=68{,}2$ m.
	}
\end{ex}
\Closesolutionfile{ans}
% \begin{indapan}{10}
% 	{ans/ans-1K5-1-Dang5}
% \end{indapan}
\begin{dang}{Nguyên lý kẹp}
	Để tìm giới hạn của dãy số theo nguyên lý kẹp ta cần nhớ:
	\begin{itemize}
		\item Cho hai dãy số $(u_n)$ và $(v_n)$. Nếu $|u_n| \leq v_n$ với mọi $n$ và $\lim \limits_{n \to +\infty}v_n = 0$ thì $\lim \limits_{n \to +\infty}u_n =0$.
		\item Cho ba dãy số $(u_n)$, $(v_n)$ và $(w_n)$. Nếu $ u_n \leq v_n \leq w_n$ với mọi $n$ và $\lim \limits_{n \to +\infty}u_n = \lim \limits_{n \to +\infty}w_n = L$ thì $\lim \limits_{n \to +\infty}v_n = L$.
	\end{itemize}
\end{dang}
\subsubsection{Ví dụ minh hoạ}
%bai1
\begin{vd}%[1D4B1-2]
	Chứng minh rằng các dãy số với số hạng tổng quát sau đây có giới hạn $0$.
	\begin{enumEX}[a)]{2}
		\item $u_n=\dfrac{(-1)^n}{3n+2}$.
		\item $u_n=\dfrac{n\sin 2n}{n^3+2}$.
		\item $u_n=\dfrac{(-1)^n\cos n}{\sqrt{n}}$.
		\item $u_n=\dfrac{3\sin n-4\cos n}{2n^2+1}$.
	\end{enumEX}
	\loigiai{
		\begin{enumerate}[a)]
			\item Ta có: $0 \leqslant \left|u_n\right|= \dfrac{1}{3n+2}<\dfrac{1}{3n}<\dfrac{1}{n}$, $\forall n\in \mathbb{N^*}$.\\
			Mà $\lim \limits_{n \to +\infty}\dfrac{1}{n}=0$ nên suy ra $\lim \limits_{n \to +\infty}\dfrac{(-1)^n}{3n+2}=0$.
			\item Ta có $0\leqslant \left|u_n\right|=\dfrac{n\left|\sin 2n\right|}{n^3+2}\leqslant \dfrac{n}{n^3+2}<\dfrac{n}{n^3}\leqslant \dfrac{1}{n^2}$, $\forall n\in \mathbb{N^*}$.\\
			Mà $\lim \limits_{n \to +\infty}\dfrac{1}{n^2}=0$ nên suy ra $\lim \limits_{n \to +\infty}\dfrac{n\sin 2n}{n^3+2}=0$.
			\item Ta có $0\leqslant \left|\dfrac{(-1)^n\cos n}{\sqrt{n}}\right|\leqslant \dfrac{1}{\sqrt{n}}$, $\forall n\in \mathbb{N^*}$.\\
			Mà $\lim \limits_{n \to +\infty}\dfrac{1}{\sqrt{n}}=0$ nên suy ra $\lim \limits_{n \to +\infty}\dfrac{(-1)^n\cos n}{\sqrt{n}}=0$.
			\item Theo bất đẳng thức Bunhiacopxki, ta có\\
			$\left|3\sin n-4\cos n\right|\leqslant \sqrt{(3^2+4^2)\left(\sin^2n+\cos^2n\right)}=5$.\\
			Do đó $0\leqslant \left|\dfrac{3\sin n-4\cos n}{2n^2+1}\right|\leqslant \dfrac{5}{2n^2+1}<\dfrac{5}{2n^2}$, $\forall n\in \mathbb{N^*}$.\\
			Mà $\lim \limits_{n \to +\infty}\dfrac{5}{2n^2+1}=0$ nên suy ra $\lim \limits_{n \to +\infty}\dfrac{3\sin n-4\cos n}{2n^2+1}=0$.
		\end{enumerate}
	}
\end{vd}
%bai2
\begin{vd}%[1D4K1-2]
	Chứng minh rằng các dãy số với số hạng tổng quát sau đây có giới hạn $0$. 
	\begin{enumEX}{2}
		\item $u_n=\sqrt{n^3+2}-\sqrt{n^3+1}$.
		\item $u_n=\dfrac{3^n\sin 2n+4^n}{2^n+4\cdot 5^n}$.
		\item $u_n=\dfrac{n+\sin 2n}{n^2+n}$.
		\item $\dfrac{n+\cos \dfrac{n\pi}{5}}{n\sqrt{n}+\sqrt{n}}$.
	\end{enumEX}
	\loigiai{
		\begin{enumerate}
			\item Ta có $u_n=\sqrt{n^3+2}-\sqrt{n^3+1}=\dfrac{n^3+2-(n^3+1)}{\sqrt{n^3+2}+\sqrt{n^3+1}}=\dfrac{1}{\sqrt{n^3+2}+\sqrt{n^3+1}}$.\\
			Do đó $0\leqslant \left|u_n\right|=\dfrac{1}{\sqrt{n^3+2}+\sqrt{n^3+1}}<\dfrac{2}{\sqrt{n^2}+\sqrt{n^2}}=\dfrac{2}{2n}=\dfrac{1}{n}$, $\forall n\in \mathbb{N^*}$.\\
			Mà $\lim \limits_{n \to +\infty}\dfrac{1}{n}=0$ nên suy ra $\lim \limits_{n \to +\infty}\left(\sqrt{n^3+2}-\sqrt{n^3+1}\right)=0$.
			\item Ta có $0\leqslant \left|\dfrac{3^n\sin 2n+4^n}{2^n+4.5^n}\right|\leqslant \dfrac{3^n\left|\sin 2n\right|+4^n}{2^n+4\cdot 5^n}\leqslant \dfrac{3^n+4^n}{2^n+4\cdot 5^n}=\dfrac{{\left(\dfrac{3}{5}\right)}^n+{\left(\dfrac{4}{5}\right)}^n}{{\left(\dfrac{2}{5}\right)}^n+4}$, $\forall n\in \mathbb{N^*}$.\\
			Mà $\lim \limits_{n \to +\infty}\left[{\left(\dfrac{3}{5}\right)}^n+{\left(\dfrac{4}{5}\right)}^n\right]=\lim \limits_{n \to +\infty}{\left(\dfrac{3}{5}\right)}^n+\lim \limits_{n \to +\infty}{\left(\dfrac{4}{5}\right)}^n=0$ và $$\lim \limits_{n \to +\infty}\left[{\left(\dfrac{2}{5}\right)}^n+4\right]=\lim \limits_{n \to +\infty}{\left(\dfrac{2}{5}\right)}^n+4=0+4=4$$ nên $\lim \limits_{n \to +\infty}\dfrac{3^n\sin 2n+4^n}{2^n+4\cdot 5^n}=0$.
			\item Ta có $0\leqslant \left|\dfrac{n+\sin 2n}{n^2+n}\right|\leqslant \dfrac{n+\left|\sin 2n\right|}{n^2+n}\leqslant \dfrac{n+1}{n^2+n}=\dfrac{1}{n}$, $\forall n\in \mathbb{N^*}$. \\
			Mà $\lim \limits_{n \to +\infty}\dfrac{1}{n}=0$ nên suy ra $\lim \limits_{n \to +\infty}\dfrac{n+\sin 2n}{n^2+n}=0$.
			\item Ta có $0\leqslant \left|\dfrac{n+\cos \dfrac{n\pi}{5}}{n\sqrt{n}+\sqrt{n}}\right|\leqslant \dfrac{n+\left|\cos \dfrac{n\pi}{5}\right|}{n\sqrt{n}+\sqrt{n}}\leqslant \dfrac{n+1}{n\sqrt{n}+\sqrt{n}}=\dfrac{1}{\sqrt{n}}$, $\forall n\in \mathbb{N^*}$. \\
			Mà $\lim \limits_{n \to +\infty}\dfrac{1}{\sqrt{n}}=0$ nên suy ra $\lim \limits_{n \to +\infty}\dfrac{n+\cos \dfrac{n\pi}{5}}{n\sqrt{n}+\sqrt{n}}=0$.
		\end{enumerate}
	}
\end{vd}
%bai3
\begin{vd}%[1D4B1-1]
	Cho dãy số $(u_n)$ với $u_n=\dfrac{n}{3^n}$.
	\begin{enumerate}[a)]
		\item Chứng minh rằng $\dfrac{u_{n+1}}{u_n}\leqslant \dfrac{2}{3}$ với mọi $n\in \mathbb{N^*}$.
		\item Bằng phương pháp quy nạp chứng minh rằng $0<u_n<{\left(\dfrac{2}{3}\right)}^n$ với mọi $n\in \mathbb{N^*}$.
		\item Dãy $(u_n)$ có giới hạn $0$.
	\end{enumerate}
	\loigiai{
		\begin{enumerate}[a)]
			\item Với mọi $n\in \mathbb{N}$, ta có $\dfrac{u_{n+1}}{u_n}=\dfrac{\dfrac{n+1}{3^{n+1}}}{\dfrac{n}{3^n}}=\dfrac{n+1}{n}\cdot \dfrac{1}{3}$. \\
			Mặt khác, $n+1\leqslant n+n\leqslant 2n$. Suy ra $\dfrac{n+1}{n}\leqslant 2$. \\
			Do đó $\dfrac{u_{n+1}}{u_n}\leqslant \dfrac{2}{3}$ với mọi $n\in \mathbb{N^*}$.
			\item Rõ ràng với mọi $n\in \mathbb{N^*}$, ta có $u_n>0$. Do đó ta chỉ cần chứng minh $u_n<{\left(\dfrac{2}{3}\right)}^n$.
			\begin{itemize}
				\item Với $n=1$, ta có $u_1=\dfrac{1}{3^1}=\dfrac{1}{3}<{\left(\dfrac{2}{3}\right)}^1$. Nghĩa là mệnh đề đúng với $n=1$. \\
				\item Giả sử mệnh đề đúng với $n=k\geqslant 1$, tức là $u_k<{\left(\dfrac{2}{3}\right)}^k$. \\
				\item Bây giờ ta cần chứng minh mệnh đề đúng với $n=k+1$, tức là cần chứng minh $u_{k+1}<{\left(\dfrac{2}{3}\right)}^{k+1}$. \\
				Theo chứng minh câu a) ta có $\dfrac{u_{k+1}}{u_k}\leqslant \dfrac{2}{3}$ suy ra $u_{k+1}\leqslant \dfrac{2}{3}\cdot u_k<\dfrac{2}{3}\cdot {\left(\dfrac{2}{3}\right)}^k={\left(\dfrac{2}{3}\right)}^{k+1}$ hay $u_{k+1}<{\left(\dfrac{2}{3}\right)}^{k+1}$. \\
				Nghĩa là mệnh đề cũng đúng với $n=k+1$. Vậy $0<u_n<{\left(\dfrac{2}{3}\right)}^n$ với mọi $n\in \mathbb{N^*}$.
			\end{itemize}
			\item Theo câu b), ta có $0<u_n<{\left(\dfrac{2}{3}\right)}^n$. Mà $\lim \limits_{n \to +\infty}{\left(\dfrac{2}{3}\right)}^n=0$. Do đó $\lim \limits_{n \to +\infty}u_n=0$.
		\end{enumerate}
	}
\end{vd}
%bai4
\begin{vd}%[1D4B1-2]
	Chứng minh rằng
	\begin{enumEX}[a)]{2}
		\item $\lim \limits_{n \to +\infty}\left(\dfrac{-n^3}{n^3+1}\right)=-1$.
		\item $\lim \limits_{n \to +\infty}\left(\dfrac{n^2+3n+2}{2n^2+n}\right)=\dfrac{1}{2}$.
	\end{enumEX}
	\loigiai{
		\begin{enumerate}
			\item Ta có $\lim \limits_{n \to +\infty}\left(\dfrac{-n^3}{n^3+1}-(-1)\right)=\lim \limits_{n \to +\infty}\left(\dfrac{1}{n^3+1}\right)$. \\
			Vì $0\leqslant \left|\dfrac{1}{n^3+1}\right|<\dfrac{1}{n^3}$, $\forall n\in \mathbb{N^*}$. Mà $\lim \limits_{n \to +\infty}\dfrac{1}{n^3}=0$ nên suy ra $\lim \limits_{n \to +\infty}\left(\dfrac{1}{n^3+1}\right)=0$. \\
			Do đó $\lim \limits_{n \to +\infty}\left(\dfrac{-n^3}{n^3+1}\right)=-1$.
			\item Ta có $\lim \limits_{n \to +\infty}\left(\dfrac{n^2+3n+2}{2n^2+n}-\dfrac{1}{2}\right)=\lim \limits_{n \to +\infty}\dfrac{5n+4}{2(2n^2+n)}$. \\
			Vì $0<\left|\dfrac{5n+4}{2(2n^2+n)}\right|<\dfrac{5n+5}{2n(n+1)}=\dfrac{5}{2}\cdot \dfrac{1}{n}$, $\forall n\in \mathbb{N^*}$. Mà $\lim \limits_{n \to +\infty}\left(\dfrac{5}{2}\cdot \dfrac{1}{n}\right)=\dfrac{5}{2}\cdot \lim \limits_{n \to +\infty}\dfrac{1}{n}=0$ nên suy ra $\lim \limits_{n \to +\infty}\dfrac{5n+4}{2(2n^2+n)}=0$. \\
			Do đó $\lim \limits_{n \to +\infty}\left(\dfrac{n^2+3n+2}{2n^2+n}\right)=\dfrac{1}{2}$.
		\end{enumerate}
	}
\end{vd}
%bai5
\begin{vd}%[1D4K1-2]
	Chứng minh rằng
	\begin{enumEX}[a)]{2}
		\item $\lim \limits_{n \to +\infty}\left(\dfrac{3\cdot 3^n-\sin 3n}{3^n}\right)=3$.
		\item $\lim \limits_{n \to +\infty}\left(\sqrt{n^2+n}-n\right)=\dfrac{1}{2}$.
	\end{enumEX}
	\loigiai{
		\begin{enumerate}[a)]
			\item Ta có $\lim \limits_{n \to +\infty}\left(\dfrac{3.3^n-\sin 3n}{3^n}-3\right)=\lim \limits_{n \to +\infty}\left(\dfrac{-\sin 3n}{3^n}\right)$. \\
			Vì $0\leqslant \left|\dfrac{-\sin 3n}{3^n}\right|=\dfrac{\left|-\sin 3n\right|}{3^n}\leqslant \dfrac{1}{3^n}={\left(\dfrac{1}{3}\right)}^n$, $\forall n\in \mathbb{N^*}$. Mà $\lim \limits_{n \to +\infty}{\left(\dfrac{1}{3}\right)}^n=0$ nên suy ra $\lim \limits_{n \to +\infty}\left(\dfrac{-\sin 3n}{3^n}\right)=0$. \\
			Do đó $\lim \limits_{n \to +\infty}\left(\dfrac{3.3^n-\sin 3n}{3^n}\right)=3$.
			\item Ta có $\lim \limits_{n \to +\infty}\left(\sqrt{n^2+n}-n-\dfrac{1}{2}\right)=\lim \limits_{n \to +\infty}\dfrac{2\sqrt{n^2+n}-(2n+1)}{2}=\lim \limits_{n \to +\infty}\dfrac{-1}{2\left(2\sqrt{n^2+n}+(2n+1)\right)}$. \\
			Vì $0\leqslant \left|\dfrac{-1}{2\left(2\sqrt{n^2+n}+(2n+1)\right)}\right| \leqslant \dfrac{1}{2\left(2\sqrt{n^2+n}+(2n+1)\right)}\leqslant \dfrac{1}{2\left(2\sqrt{n^2}+2n\right)}=\dfrac{1}{8}\cdot \dfrac{1}{n}$, $\forall n\in \mathbb{N^*}$. \\
			Mà $\lim \limits_{n \to +\infty}\dfrac{1}{8}\cdot \dfrac{1}{n}=\dfrac{1}{8}\lim \limits_{n \to +\infty}\dfrac{1}{n}=0$ nên suy ra $\lim \limits_{n \to +\infty}\left(\sqrt{n^2+n}-n-\dfrac{1}{2}\right)$. \\
			Do đó $\lim \limits_{n \to +\infty}\left(\sqrt{n^2+n}-n\right)=\dfrac{1}{2}$.
		\end{enumerate}
	}
\end{vd}
%Bài 6
\begin{vd}%[1D4K1-5]
	Tìm các giới hạn sau
	\begin{enumEX}[a)]{1}
		\item $\lim \limits_{n \to +\infty}\left(\dfrac{1}{\sqrt{4n^2+1}}+\dfrac{1}{\sqrt{4n^2+2}}+\cdots +\dfrac{1}{\sqrt{4n^2+n}}\right)$.
		\item $\lim \limits_{n \to +\infty}\dfrac{{1\cdot 3\cdot 5\cdot 7}\cdots (2n-1)}{{2\cdot 4\cdot 6}\cdots (2n)}$.
	\end{enumEX}
	\loigiai{
		\begin{enumerate}[a)]
			\item Ta có
			\begin{align*}
				\dfrac{1}{\sqrt{4n^2}}+\dfrac{1}{\sqrt{4n^2}}+\cdots +\dfrac{1}{\sqrt{4n^2}} &\leqslant \dfrac{1}{\sqrt{4n^2+1}}+\dfrac{1}{\sqrt{4n^2+2}}+\cdots +\dfrac{1}{\sqrt{4n^2+n}}\\
				&\leqslant \dfrac{1}{\sqrt{4n^2+n}}+\dfrac{1}{\sqrt{4n^2+n}}+\cdots +\dfrac{1}{\sqrt{4n^2+n}}.
			\end{align*}
			hay
			\begin{center}
				$\dfrac{n}{\sqrt{4n^2}}\leqslant \dfrac{1}{\sqrt{4n^2+1}}+\dfrac{1}{\sqrt{4n^2+2}}+\cdots +\dfrac{1}{\sqrt{4n^2+n}}\leqslant \dfrac{n}{\sqrt{4n^2+n}}$ với mọi $n\in \mathbb{N^*}$.
			\end{center}
			Mà $\lim \limits_{n \to +\infty}\dfrac{n}{\sqrt{4n^2}}=\lim \limits_{n \to +\infty}\dfrac{1}{2}=\dfrac{1}{2}$; $\lim \limits_{n \to +\infty}\dfrac{n}{\sqrt{4n^2+n}}=\lim \limits_{n \to +\infty}\dfrac{1}{\sqrt{4+\dfrac{1}{n}}}=\dfrac{1}{\sqrt{4+0}}=\dfrac{1}{2}$. \\
			Do đó $\lim \limits_{n \to +\infty}\left(\dfrac{1}{\sqrt{4n^2+1}}+\dfrac{1}{\sqrt{4n^2+2}}+\cdots +\dfrac{1}{\sqrt{4n^2+n}}\right)=\dfrac{1}{2}$.
			\item Ta có $u_n=\dfrac{{1\cdot 3\cdot 5\cdot 7}\cdots (2n-1)}{{2\cdot 4\cdot 6}\cdots (2n)}$, suy ra $$u_n^2=\dfrac{1^2\cdot 3^2\cdot 5^2\cdot 7^2\cdots (2n-1)^2}{2^2\cdot 4^2\cdot 6^2\cdots (2n)^2}=\dfrac{1\cdot 3}{2^2}\cdot \dfrac{3\cdot 5}{4^2}\cdots \dfrac{(2n-1)(2n+1)}{(2n)^2}\cdot \dfrac{1}{2n+1}<\dfrac{1}{2n+1}.$$
			(do $\dfrac{1\cdot 3}{2^2}\cdot \dfrac{3\cdot 5}{4^2}\cdots \dfrac{(2n-1)(2n+1)}{(2n)^2}<\dfrac{2^2}{2^2}\cdot \dfrac{4^2}{4^2}\cdots \dfrac{(2n)^2}{(2n)^2}=1$ ) \\
			Vậy ta có $0<u_n<\dfrac{1}{\sqrt{2n+1}}$, $\forall n\in \mathbb{N^*}$. Mà $\lim \limits_{n \to +\infty}\dfrac{1}{\sqrt{2n+1}}=0$ nên suy ra $$\lim \limits_{n \to +\infty}\dfrac{{1\cdot 3\cdot 5\cdot 7}\cdots (2n-1)}{{2\cdot 4\cdot 6}\cdots (2n)}=0.$$
		\end{enumerate}
	}
\end{vd}
% \subsubsection{Bài tập rèn luyện}
% \subsubsection{Câu hỏi trắc nghiệm}
% \Opensolutionfile{ans}[ans/ans-1K5-1-Dang6]

% %%==========Câu 1
% \begin{ex}%[1D4B1-2]
% 	Giới hạn $\displaystyle\lim\dfrac{\sin n+1}{n}$ bằng
% 	\choice
% 	{$+\infty$}
% 	{$1$}
% 	{$-\infty$}
% 	{\True $0$}
% 	\loigiai{
% 		Với mọi $n>0$ thì $|\sin n+1|\leq 2$. Do đó, với mọi $n>0$, ta có
% 		$$0\leq \left|\dfrac{\sin n+1}{n}\right|\leq \dfrac{2}{n}.$$
% 		Từ đó $$0\leq \lim\limits\left|\dfrac{\sin n+1}{n}\right|\leq \lim\limits\dfrac{2}{n}=0\Rightarrow \lim\limits\left|\dfrac{\sin n+1}{n}\right|=0\Rightarrow \lim\limits\dfrac{\sin n+1}{n}=0.$$
% 	}
% \end{ex}
% %%==========Câu 2
% \begin{ex}%[1D4B1-2]
% 	Giới hạn $\displaystyle\lim\dfrac{\sin n+1}{n}$ bằng
% 	\choice
% 	{$+\infty$}
% 	{$1$}
% 	{$-\infty$}
% 	{\True $0$}
% 	\loigiai{
% 		Với mọi $n>0$ thì $|\sin n+1|\leq 2$. Do đó, với mọi $n>0$, ta có
% 		$$0\leq \left|\dfrac{\sin n+1}{n}\right|\leq \dfrac{2}{n}.$$
% 		Từ đó $$0\leq \lim\limits\left|\dfrac{\sin n+1}{n}\right|\leq \lim\limits\dfrac{2}{n}=0\Rightarrow \lim\limits\left|\dfrac{\sin n+1}{n}\right|=0\Rightarrow \lim\limits\dfrac{\sin n+1}{n}=0.$$
% 	}
% \end{ex}
% %%==========Câu 3
% \begin{ex}%[1D4B1-2]
% 	Giới hạn $\lim \limits_{n \to +\infty}\dfrac{\cos n}{n}$ bằng
% 	\choice
% 	{$1$}
% 	{\True $0$}
% 	{$-1$}
% 	{$+\infty$}
% 	\loigiai{
% 		Ta có: $\left| \dfrac{\cos n}{n} \right|\le \dfrac{1}{n}$ và $\lim \limits_{n \to +\infty}\dfrac{1}{n}=0$ nên $\lim \limits_{n \to +\infty}\dfrac{\cos n}{n}=0$.
% 	}
% \end{ex}
% %%==========Câu 4
% \begin{ex}%[1D4B1-2]
% 	Tính $\lim \limits_{n \to +\infty}\dfrac{\sin n}{n^3+1}$.
% 	\choice
% 	{$1$}
% 	{\True $0$}
% 	{$-\infty$}
% 	{$+\infty$}
% 	\loigiai{
% 		Ta có
% 		$ \left|\dfrac{\sin n}{n^3+1}\right|\le \dfrac{1}{n^3+1}$ mà
% 		$\lim \limits_{n \to +\infty}\dfrac{1}{n^3+1}=\lim \limits_{n \to +\infty}\dfrac{1}{n^3\left(1+\dfrac{1}{n^3}\right)}=0$.\\
% 		Vậy $\lim \limits_{n \to +\infty}\dfrac{\sin n}{n^3+1}=0$
% 	}
% \end{ex}
% %%==========Câu 5
% \begin{ex}%[1D4B1-2]
% 	Tính $\lim \limits_{n \to +\infty}\dfrac{\sin 2024n}{n}$.
% 	\choice
% 	{\True $0$}
% 	{$1$}
% 	{$+ \infty$}
% 	{$2024$}
% 	\loigiai{
% 		Ta có $-1 \leqslant \sin 2024n \leqslant 1  \Leftrightarrow - \dfrac{1}{n} \leqslant \dfrac{\sin 2024n}{n} \leqslant \dfrac{1}{n}$.\\
% 		Vì $\lim \limits_{n \to +\infty}\left( - \dfrac{1}{n} \right) = \lim \limits_{n \to +\infty} \dfrac{1}{n} = 0$ nên $\lim \limits_{n \to +\infty}\dfrac{\sin 2024n}{n} = 0$.
% 	}
% \end{ex}
% %%==========Câu 6
% \begin{ex}%[1D4K1-2]
% 	Tính $I=\lim \limits_{n \to +\infty}\left(\dfrac{1}{\sqrt{n^2+n+1}}+\dfrac{1}{\sqrt{n^2+n+2}}+...+\dfrac{1}{\sqrt{n^2+2n}}\right)$.
% 	\choice
% 	{$I=+\infty$}
% 	{$I=3$}
% 	{$I=2$}
% 	{\True $I=1$}
% 	\loigiai{
% 		Ta có $\dfrac{1}{\sqrt{n^2+2n}}<\dfrac{1}{\sqrt{n^2+n+k}}<\dfrac{1}{\sqrt{n^2+n+1}},\forall k=2,3,...,n-1$.\\
% 		$\Rightarrow \dfrac{n}{\sqrt{n^2+2n}}<\dfrac{1}{\sqrt{n^2+n+1}}+\dfrac{1}{\sqrt{n^2+n+2}}+\cdots+\dfrac{1}{\sqrt{n^2+2n}}<\dfrac{n}{\sqrt{n^2+n+1}}$.\\
% 		Mà $\lim \limits_{n \to +\infty}\dfrac{n}{\sqrt{n^2+2n}}=\lim \limits_{n \to +\infty}\dfrac{1}{\sqrt{1+\dfrac{2}{n}}}=1$; $\lim \limits_{n \to +\infty}\dfrac{n}{\sqrt{n^2+n+1}}=\lim \limits_{n \to +\infty}\dfrac{1}{\sqrt{1+\dfrac{1}{n}+\dfrac{1}{n^2}}}=1$.\\
% 		Vậy $I=1$.
% 	}
% \end{ex}
% %%==========Câu 7
% \begin{ex}%[1D4K1-2]
% 	Tính $T=\lim\dfrac{n\sin n-3n^2}{n^2}$.
% 	\choice
% 	{$T=+\infty$}
% 	{$T=-\infty$}
% 	{$T=1$}
% 	{\True $T=-3$}
% 	\loigiai{
% 		Ta có $T=\lim\dfrac{n\sin n-3n^2}{n^2}=\lim\left(\dfrac{\sin n}{n}-3\right)$.\\
% 		Do $-1\le \sin n\le 1$ suy ra $-\dfrac{1}{n}\le \dfrac{\sin n}{n}\le \dfrac{1}{n}$, mà $\lim\left(-\dfrac{1}{n}\right)=0$ và $\lim\dfrac{1}{n}=0$ nên $\lim\dfrac{\sin n}{n}=0$.\\
% 		Từ đó suy ra $T=\lim\left(\dfrac{\sin n}{n}-3\right)=-3$.
% 	}
% \end{ex}
% %%==========Câu 8
% \begin{ex}%[1D4K1-2]
% 	Tính giá trị của $I=\lim\dfrac{n^3+n\cdot\sin^2n}{10000n^3-n+2}$.
% 	\choice
% 	{\True $I=0{,}0001$}
% 	{$I=\dfrac{1}{1000}$}
% 	{$I=0$}
% 	{$I=0{,}00001$}
% 	\loigiai{Ta có: $I=\lim\dfrac{n^3+n\cdot\sin^2n}{10000n^3-n+2}=\lim\dfrac{1+\dfrac{\sin^2 n}{n^2}}{10000-\dfrac{1}{n^2}+\dfrac{2}{n^3}}=\dfrac{1}{10000}=0{,}0001$.\\
% 		Chú ý rằng: $0\leq\dfrac{\sin^2n}{n^2}\leq\dfrac{1}{n^2}$. Mà $\lim\dfrac{1}{n^2}=0\Rightarrow\lim\dfrac{\sin^2n}{n^2}=0$.}
% \end{ex}
% %%==========Câu 9
% \begin{ex}%[1D4G1-2]
% 	Tính $I=\lim \limits_{n \to +\infty}\left(\dfrac{1}{2}+\dfrac{3}{2^2}+\dfrac{5}{2^3}+\cdots +\dfrac{2n-1}{2^n}\right)$.
% 	\choice
% 	{\True $I=3$}
% 	{$I=0$}
% 	{$I=\dfrac{1}{2}$}
% 	{$I=+\infty $}
% 	\loigiai{
% 		Đặt $S_n=\dfrac{1}{2}+\dfrac{3}{2^2}+\dfrac{5}{2^3}+\cdots +\dfrac{2n-1}{2^n}$. \\
% 		Khi đó $\dfrac{1}{2}S_n=\dfrac{1}{2^2}+\dfrac{3}{2^3}+\dfrac{5}{2^4}+\cdots +\dfrac{2n-3}{2^n}+\dfrac{2n-1}{2^{n+1}}$. \\
% 		Trừ vế theo vế ta được \\
% 		$S_n-\dfrac{1}{2}S_n=\dfrac{1}{2}+\left(\dfrac{2}{2^2}+\dfrac{2}{2^3}+\cdots +\dfrac{2}{2^n}\right)-\dfrac{2n-1}{2^{n+1}}$. \\
% 		Từ đó $S_n=1+\left(1+\dfrac{1}{2}+\dfrac{1}{4}+\cdots +\dfrac{1}{2^{n-2}}\right)-\dfrac{2n-1}{2^n}=1+2\left[1-{\left(\dfrac{1}{2}\right)}^{n-1}\right]-\dfrac{2n-1}{2^n}$. \\
% 		Với mọi $n\geqslant 4$ ta có $2^n\geqslant n^2$. Thật vậy,
% 		\begin{itemize}
% 			\item Ta có $2^4 \geqslant 4^2$.
% 			\item Nếu $2^k \geqslant k^2~(k \geqslant 4)$ thì $2^{k+1}=2\cdot 2^k \geqslant 2\cdot k^2>k^2+(2k+1)=(k+1)^2$ (do $k \geqslant 4$).
% 		\end{itemize}
% 		Từ đó $0<\dfrac{2n-1}{2^n}\leqslant \dfrac{2n-1}{n^2}$. Mà $\lim \limits_{n \to +\infty}\dfrac{2n-1}{n^2}=0$ nên $\lim \limits_{n \to +\infty}\dfrac{2n-1}{2^n}=0$. \\
% 		Vậy $I=\lim \limits_{n \to +\infty}S_n=\lim \limits_{n \to +\infty}\left[1+2\left[1-{\left(\dfrac{1}{2}\right)}^{n-1}\right]-\dfrac{2n-1}{2^n}\right]=1+2=3$.}
% \end{ex}
% %%==========Câu 10
% \begin{ex}%[1D4G1-2]
% 	Cho dãy số $(u_n)$ được xác định bởi $\heva{& u_1=3 \\ & 2(n+1)u_{n+1}=nu_n+n+2}$. Tính $\lim \limits_{n \to +\infty}u_n$.
% 	\choice
% 	{\True $\lim \limits_{n \to +\infty}u_n=1$}
% 	{$\lim \limits_{n \to +\infty}u_n=4$}
% 	{$\lim \limits_{n \to +\infty}u_n=3$}
% 	{$\lim \limits_{n \to +\infty}u_n=0$}
% 	\loigiai{
% 		Ta chứng minh $1\le u_{n+1}\le 1+\dfrac{1}{2n}$, $\forall n\ge 1$. Thật vậy
% 		\begin{itemize}
% 			\item $u_{n+1}\ge 1$, $\forall n\ge 1$ $(1)$.\\
% 			Với $n=1\Rightarrow u_2=\dfrac{3}{2}\ge 1\Rightarrow (1)$ đúng với $n=1$.\\
% 			Giả sử $(1)$ đúng với $n=k\ge 1$, tức là $u_{k+1}\ge 1$.\\
% 			Ta cần chứng minh $(1)$ đúng với $n=k+1$, tức là chứng minh $u_{k+2}\ge 1$. Thật vậy\\
% 			$u_{k+2}=\dfrac{(k+1) u_{k+1}+1}{2(k+2)}+\dfrac{1}{2}\ge \dfrac{k+2}{2(k+2)}+\dfrac{1}{2}=1$.
% 			\item $u_{n+1}\le 1+\dfrac{1}{2n}$, $\forall n\ge 1$ $(2)$.\\
% 			Với $n=1\Rightarrow u_2=\dfrac{3}{2}\le 1+\dfrac{3}{2}\Rightarrow (2)$ đúng với $n=1$.\\
% 			Giả sử $(2)$ đúng với $n=k\ge 1$, tức là $u_{k+1}\le 1+\dfrac{1}{2k}$.\\
% 			Ta cần chứng minh $(2)$ đúng với $n=k+1$, tức là chứng minh $u_{k+2}\le 1+\dfrac{1}{2(k+1)}$. Thật vậy
% 			$$u_{k+2}=\dfrac{(k+1)u_{k+1}+1}{2(k+2)}+\dfrac{1}{2}\le \dfrac{(k+1)\left(1+\dfrac{1}{2(k+1)}\right)}{2(k+2)}+\dfrac{1}{2}\le 1+\dfrac{1}{4(k+2)}\le 1+\dfrac{1}{2(k+1)}.$$
% 			Vậy $\lim \limits_{n \to +\infty}u_n=1$.
% 		\end{itemize}
% 	}
% \end{ex}
% %%==========Câu 11
% \begin{ex}%[1D4G1-2]
% 	Cho dãy số $\left( u_n\right) $ thỏa mãn $\heva{&u_1=\dfrac{1}{3}\\&u_{n+1}=\dfrac{(n+1)u_n}{3n},\,\, \forall n\geq 1}$. Có bao nhiêu số nguyên dương $n$ thỏa mãn $u_n<\dfrac{1}{2020}.$
% 	\choice
% 	{$0$}
% 	{$9$}
% 	{\True vô số}
% 	{$5$}
% 	\loigiai
% 	{
% 		\begin{itemize}
% 			\item Đặt $v_n=\dfrac{u_n}{n}$, ta có $v_{n+1}=\dfrac{v_n}{3}$.
% 			\item Do đó $v_n$ là cấp số nhân với công bội là $\dfrac{1}{3}$, mà $v_1=\dfrac{u_1}{1}=\dfrac{1}{3}$ nên
% 			$v_n=v_1\cdot\left(\dfrac{1}{3}\right)^{n-1}=\left(\dfrac{1}{3}\right)^n$.
% 			\item Từ đó $u_n=n\left(\dfrac{1}{3}\right)^n=\dfrac{n}{3^n}$.
% 			\item Bằng quy nạp, ta chứng minh được $3^n>n^2$, $\forall\, n\ge 1$. Khi đó $|u_n|=\dfrac{n}{3^n}<\dfrac{n}{n^2}=\dfrac{1}{n}$.\\
% 			Mà $\lim\dfrac{1}{n}=0\Rightarrow\lim \limits_{n \to +\infty}u_n=0$. Suy ra có vô số $n$ để $u_n<\dfrac{1}{2020}$.
% 		\end{itemize}
% 	}
% \end{ex}
% %%==========Câu 12
% \begin{ex}%[1D4K1-2]
% 	Tìm giới hạn của $(u_n)$ với $u_n = \dfrac{1}{\sqrt{n^2 + 1}} + \dfrac{1}{\sqrt{n^2 + }} +  \cdots + \dfrac{1}{\sqrt{n^2 + n}}$.
% 	\choice
% 	{\True $1$}
% 	{$0$}
% 	{$+\infty$}
% 	{$-\infty$}
% 	\loigiai{Với mỗi số nguyên $k$ mà $1 \leq k \leq n$, ta có $\dfrac{1}{\sqrt{n^2 + n}} \leq \dfrac{1}{\sqrt{n^2 + k}} \leq \dfrac{1}{\sqrt{n^2 + 1}}$.\\
% 		Do đó $\dfrac{n}{\sqrt{n^2 + n}} \leq u_n \leq \dfrac{n}{\sqrt{n^2 + 1}}$ với mọi $n$.\\
% 		Mặt khác $\lim \limits_{n \to +\infty}\dfrac{n}{\sqrt{n^2 + n}} = \lim \limits_{n \to +\infty}\dfrac{n}{\sqrt{n^2 + 1}} = 1$.\\
% 		Do đó $\lim \limits_{n \to +\infty}u_n = 1$.
% 	}
% \end{ex}
% %Câu 13
% \begin{ex}%[1D4K1-2]
% 	Tìm giới hạn của $(u_n)$ với $u_n = \dfrac{1}{\sqrt{n^2 + 1}} + \dfrac{1}{\sqrt{n^2 + }} +  \cdots + \dfrac{1}{\sqrt{n^2 + n}}$.
% 	\choice
% 	{\True $1$}
% 	{$0$}
% 	{$+\infty$}
% 	{$-\infty$}
% 	\loigiai{Với mỗi số nguyên $k$ mà $1 \leq k \leq n$, ta có $\dfrac{1}{\sqrt{n^2 + n}} \leq \dfrac{1}{\sqrt{n^2 + k}} \leq \dfrac{1}{\sqrt{n^2 + 1}}$.\\
% 		Do đó $\dfrac{n}{\sqrt{n^2 + n}} \leq u_n \leq \dfrac{n}{\sqrt{n^2 + 1}}$ với mọi $n$.\\
% 		Mặt khác $\lim \limits_{n \to +\infty}\dfrac{n}{\sqrt{n^2 + n}} = \lim \limits_{n \to +\infty}\dfrac{n}{\sqrt{n^2 + 1}} = 1$.\\
% 		Do đó $\lim \limits_{n \to +\infty}u_n = 1$.
% 	}
% \end{ex}
% %Câu 14
% \begin{ex}%[1D4K1-2]
% 	Kết quả đúng của $\lim \limits_{n \to +\infty}\left(5 - \dfrac{n\cos 2n}{n^2 + 1} \right)$ là
% 	\choice
% 	{$4$}
% 	{\True $5$}
% 	{$-4$}
% 	{$\dfrac{1}{4}$}
% 	\loigiai{
% 		Với mọi $n \in \mathbb{N}$ ta có $-\dfrac{n}{n^2 + 1} \le \dfrac{n\cos 2n}{n^2 + 1} \le \dfrac{n}{n^2 + 1}$.\\
% 		Ta có $\lim \limits_{n \to +\infty}\left(-\dfrac{n}{n^2 + 1}\right) = \lim \limits_{n \to +\infty}\dfrac{-\dfrac{1}{n}}{1 + \dfrac{1}{n^2}} = 0$; $\lim \limits_{n \to +\infty}\dfrac{n}{n^2 + 1} = \lim \limits_{n \to +\infty}\dfrac{\dfrac{1}{n}}{1 + \dfrac{1}{n^2}} = 0$.\\
% 		Suy ra $\lim \limits_{n \to +\infty}\left(\dfrac{n\cos 2n}{n^2 + 1}\right) = 0 \Rightarrow \lim \limits_{n \to +\infty}\left(5 - \dfrac{n\cos 2n}{n^2 + 1}\right) = 5$.
% 	}
% \end{ex}
% %Câu 15
% \begin{ex}%[1D4K1-2]
% 	Kết quả của $\lim \limits_{n \to +\infty}\left(n^2\sin \dfrac{n\pi }{5} - 2n^3\right)$ bằng
% 	\choice
% 	{\True $-\infty$}
% 	{$0$}
% 	{$+\infty$}
% 	{$-2$}
% 	\loigiai{
% 		Ta có $\lim \limits_{n \to +\infty}\left(n^2\sin \dfrac{n\pi}{5} - 2n^3\right) = \lim \limits_{n \to +\infty}n^3\left(\dfrac{1}{n}\sin \dfrac{n\pi}{5} - 2\right) = -\infty $.\\
% 		Vì $\sin \dfrac{n\pi}{5} \le 1 \Rightarrow \dfrac{1}{n} \sin \dfrac{n\pi }{5} \le \dfrac{1}{n}$.\\
% 		Mà $\lim \limits_{n \to +\infty}\dfrac{1}{n} = 0$ nên $\lim \limits_{n \to +\infty}\left(\dfrac{1}{n}\sin \dfrac{n\pi}{5} - 2\right) = -2$.\\
% 		Mặt khác $\lim \limits_{n \to +\infty}n^3 = +\infty$.\\
% 		Vậy $\lim \limits_{n \to +\infty}\left(n^2\sin \dfrac{n\pi }{5} - 2n^3\right) = -\infty$.
% 	}
% \end{ex}
% %Câu 16
% \begin{ex}%[1D4K1-2]
% 	Tính $I = \lim \limits_{n \to +\infty}\left( \dfrac{1}{\sqrt{n^2 + n + 1}} + \dfrac{1}{\sqrt {n^2 + n + 2}} + \cdots + \dfrac{1}{\sqrt{n^2 + 2n}}\right)$.
% 	\choice
% 	{$I = +\infty$}
% 	{$I = 3$}
% 	{$I = 2$}
% 	{\True $I = 1$}
% 	\loigiai{
% 		Ta có $\dfrac{1}{\sqrt{n^2 + 2n}} < \dfrac{1}{\sqrt{n^2 + n + k}} < \dfrac{1}{\sqrt{n^2 + n + 1}},\forall k = 2{,}3,\cdots ,n-1$.\\
% 		Suy ra $\dfrac{n}{\sqrt{n^2 + 2n}} < \dfrac{1}{\sqrt{n^2 + n + 1}} + \dfrac{1}{\sqrt{n^2 + n + 2}} + \cdots + \dfrac{1}{\sqrt{n^2 + 2n}} < \dfrac{n}{\sqrt{n^2 + n + 1}}$.\\
% 		Mà $\lim \limits_{n \to +\infty}\dfrac{n}{\sqrt{n^2 + 2n}} = \lim \limits_{n \to +\infty}\dfrac{1}{\sqrt {1 + \dfrac{2}{n}}} = 1$; $ \lim \limits_{n \to +\infty}\dfrac{n}{\sqrt{n^2 + n + 1}} = \lim \limits_{n \to +\infty}\dfrac{1}{\sqrt {1 + \dfrac{1}{n} + \dfrac{1}{n^2}}} = 1$.\\
% 		Vậy $I = 1$.
% 	}
% \end{ex}	
% \Closesolutionfile{ans}
% \begin{indapan}{10}
% 	{ans/ans-1K5-1-Dang6}
% \end{indapan}
%%Bài 16. Giới hạn hàm số
\section{Giới hạn của hàm số}
\setcounter{dang}{0}
\subsection{Tóm tắt lý thuyết}
\begin{tomtat}
	\subsubsection{Giới hạn hữu hạn của hàm số tại một điểm}
	
	\begin{dn}
		Cho điểm $ x_0 $ thuộc khoảng $ K $ và hàm số $ y=f(x) $ xác định trên $ K $ hoặc $ K\setminus\{x_0\} $.\\
		Ta nói hàm số $ y=f(x) $ \textbf{\textit{có giới hạn hữu hạn}} là số $ L $ khi $ x $ dần tới $ x_0 $ nếu với dãy số $ (x_n) $ bất kì, $ x_n\in K\setminus\{x_0\} $ và $ x_n \to x_0 $ thì $ f(x_n)\to L $, kí hiệu $ \lim \limits_{x \to x_0} f(x) =L$ hay $ f(x)\to L $ khi $ x\to x_0 $.
	\end{dn}
	
	\begin{note} 
		$ \lim \limits_{x \to x_0} x=x_0 $; \quad $ \lim \limits_{x \to x_0} c=c $ ($ c $ là hằng số).
	\end{note} 
	\subsubsection{Các phép toán về giới hạn hữu hạn của hàm số}
	\begin{enumerate}
		\item Cho $ \lim \limits_{x \to x_0} f(x) =L$ và $ \lim \limits_{x \to x_0} g(x)=M $. Khi đó:
		\begin{itemize}
			\item $ \lim \limits_{x \to x_0} [f(x)+g(x)]=L+M $;
			\item $ \lim \limits_{x \to x_0} [f(x)-g(x)]=L-M $;
			\item $ \lim \limits_{x \to x_0} [f(x)\cdot g(x)]=L\cdot M $;
			\item $ \lim \limits_{x \to x_0} \dfrac{f(x)}{g(x)}=\dfrac{L}{M} $ (với $ M\neq 0 $).			
		\end{itemize}
		\item Nếu $ f(x)\geq 0 $ và $ \lim \limits_{x \to x_0} f(x) =L$ thì $ L\geq 0 $ và $ \lim \limits_{x \to x_0} \sqrt{f(x)}=\sqrt{L} $.\\
		(Dấu của $ f(x) $ được xét trên khoảng tìm giới hạn, $ x\neq x_0 $).
	\end{enumerate}
	\begin{note} 
		\indent
		\begin{enumerate}
			\item $ \lim \limits_{x \to x_0} x^k=x_0^k $, $ k $ là số nguyên dương;
			\item $ \lim \limits_{x \to x_0} [cf(x)]=c\lim \limits_{x \to x_0} f(x) $ ($ c\in \mathbb{R} $, nếu tồn tại $ \lim \limits_{x \to x_0} f(x) \in \mathbb{R}$).
		\end{enumerate}
	\end{note} 
	
	\subsubsection{Giới hạn một phía}
	\begin{dn}
		\
		\begin{itemize}
			\item Cho hàm số $ y=f(x) $ xác định trên khoảng $ (x_0;b) $.\\
			Ta nói hàm số $ y=f(x) $ \textbf{\textit{có giới hạn bên phải}} là số $ L $ khi $ x $ dần tới $ x_0 $ nếu với dãy số $ (x_n) $ bất kì, $ x_0<x_n<b $ và $ x_n\to x_0 $ thì $ f(x_n)\to L $, kí hiệu $ \lim \limits_{x \to x_0^+} f(x) =L$.
			\item Cho hàm số $ y=f(x) $ xác định trên khoảng $ (a;x_0) $.\\
			Ta nói hàm số $ y=f(x) $ \textbf{\textit{có giới hạn bên trái}} là số $ L $ khi $ x $ dần tới $ x_0 $ nếu với dãy số $ (x_n) $ bất kì, $ a<x_n<x_0 $ và $ x_n\to x_0 $ thì $ f(x_n)\to L $, kí hiệu $ \lim \limits_{x \to x_0^-} f(x) =L$.
		\end{itemize}
	\end{dn}
	\begin{note} 
		\begin{enumerate}
			\item Ta thừa nhận các kết quả sau:
			\begin{itemize}
				\item $ \lim \limits_{x \to x_0^+} f(x)=L$ và $ \lim \limits_{x \to x_0^-} f(x)=L $ khi và chỉ khi $ \lim \limits_{x \to x_0} f(x) =L$;
				\item Nếu $ \lim \limits_{x \to x_0^+} f(x)\neq \lim \limits_{x \to x_0^-} f(x)$ thì không tồn tại $ \lim \limits_{x \to x_0} f(x) $.
			\end{itemize}
			\item Các phép toán về giới hạn hữu hạn của hàm số ở Mục 2 vẫn đúng khi ta thay $ x\to x_0 $ bằng $ x\to x_0^+ $ hoặc $ x\to x_0^- $.
		\end{enumerate}
	\end{note}
	
	\subsubsection{Giới hạn hữu hạn của hàm số tại vô cực}
	\begin{dn}
		\
		\begin{itemize}
			\item Cho hàm số $ y=f(x) $ xác định trên khoảng $ (a;+\infty) $.\\
			Ta nói hàm số $ y=f(x) $ \textbf{\textit{có giới hạn hữu hạn}} là số $ L $ khi $ x \to +\infty  $ nếu với dãy số $ (x_n) $ bất kì, $ x_n>a $ và $ x_n\to +\infty $ thì $ f(x_n)\to L $, kí hiệu $ \lim \limits_{x \to +\infty} f(x) =L$ hay $ f(x) \to L $ khi $ x\to +\infty $.
			\item Cho hàm số $ y=f(x) $ xác định trên khoảng $ (-\infty;a) $.\\
			Ta nói hàm số $ y=f(x) $ \textbf{\textit{có giới hạn hữu hạn}} là số $ L $ khi $ x \to -\infty $ nếu với dãy số $ (x_n) $ bất kì, $ x_n<a $ và $ x_n\to -\infty $ thì $ f(x_n)\to L $, kí hiệu $ \lim \limits_{x \to -\infty} f(x) =L$ hay $ f(x)\to L $ khi $ x\to -\infty $.
		\end{itemize}
	\end{dn}
	\begin{note}
		\begin{enumerate}
			\item Với $ c $ là hằng số và $ k $ là  số nguyên dương, ta luôn có:
			$$\lim \limits_{x \to \pm \infty} c=c\quad \text{và} \quad \lim \limits_{x \to \pm \infty} \dfrac{c}{x^k}=0.$$
			\item Các phép toán trên giới hạn hàm số ở Mục 2 vẫn đúng khi thay $ x\to x_0 $ bằng $ x\to +\infty $ hoặc $ x\to -\infty $.
		\end{enumerate}
	\end{note}
	
	\subsubsection{Giới hạn vô cực của hàm số tại một điểm}
	
	\begin{dn}
		Cho hàm số $ y=f(x) $ xác định trên khoảng $ (x_0;b) $.
		\begin{itemize}
			\item 
			Ta nói hàm số $ y=f(x) $ \textbf{\textit{có giới hạn bên phải}} là $ +\infty $ khi $ x $ dần tới $ x_0 $ về bên phải nếu với dãy số $ (x_n) $ bất kì, $ x_0<x_n<b $ và $ x_n\to x_0 $ thì $ f(x_n)\to +\infty $, kí hiệu $ \lim \limits_{x \to x_0^+} f(x) =+\infty$ hay $ f(x)\to +\infty $ khi $ x \to x_0^+ $.
			\item Ta nói hàm số $ y=f(x) $ \textbf{\textit{có giới hạn bên phải}} là  $ -\infty $ khi $ x $ dần tới $ x_0 $ về bên phải nếu với dãy số $ (x_n) $ bất kì, $ x_0<x_n<b $ và $ x_n\to x_0 $ thì $ f(x_n)\to -\infty $, kí hiệu $ \lim \limits_{x \to x_0^+} f(x) =-\infty$ hay $ f(x)\to -\infty $ khi $ x\to x_0^+ $.
		\end{itemize}
	\end{dn}
	
	\begin{note}
		\begin{enumerate}
			\item Các giới hạn $ \lim \limits_{x \to x_0^-} f(x)=+\infty $, $ \lim \limits_{x \to x_0^-} f(x)=-\infty $, $ \lim \limits_{x \to +\infty} f(x)=+\infty $, $ \lim \limits_{x \to +\infty} f(x)=-\infty $, $ \lim \limits_{x \to -\infty} f(x)=+\infty $, $ \lim \limits_{x \to -\infty} f(x)=-\infty $ được định nghĩa như trên.
			\item Ta có các giới hạn thường dùng như sau:
			\begin{itemize}
				\item $ \lim \limits_{x \to a^+} \dfrac{1}{x-a}=+\infty $ và $ \lim \limits_{x \to a^-}\dfrac{1}{x-a}=-\infty $ ($ a\in \mathbb{R} $);
				\item $ \lim \limits_{x \to +\infty}x^k=+\infty $ với $ k $ nguyên dương;
				\item $ \lim \limits_{x \to -\infty}x^k=+\infty $ với $ k $ là số chẵn;
				\item $ \lim \limits_{x \to -\infty}x^k=-\infty $ với $ k $ là số lẻ.
			\end{itemize}
			\item Các phép toán trên giới hạn hàm số của Mục 2 chỉ áp dụng được khi tất cả các hàm số được xét có giới hạn hữu hạn. Với giới hạn vô cực, ta có một số quy tắc sau đây.\\
			Nếu $ \lim \limits_{x \to x_0^+} f(x)=L\neq 0 $ và $ \lim \limits_{x \to x_0^+} g(x)=+\infty $ (hoặc $ \lim \limits_{x \to x_0^+} g(x)=-\infty $) thì $ \lim \limits_{x \to x_0^+}[f(x)\cdot g(x)] $ được tính theo quy tắc cho bởi bảng sau:
			\begin{center}
				\begin{tabular}{|c|c|c|}
					\hline 
					$ \lim \limits_{x \to x_0^+} f(x) $	& $  \lim \limits_{x \to x_0^+} g(x) $&$ \lim \limits_{x \to x_0^+} [f(x)\cdot g(x)] $  \\ 
					\hline 
					$ L>0 $	&$ +\infty $  &$ +\infty $  \\ 
					\hline 
					$ L>0 $	&$ -\infty $ &$ -\infty $  \\ 
					\hline 
					$ L<0 $&  $ +\infty $&  $ -\infty $\\ 
					\hline 
					$ L<0 $& $ -\infty $ &$ +\infty $  \\ 
					\hline 
				\end{tabular} 
			\end{center}
			Các quy tắc trên vẫn đúng khi thay $ x_0^+ $ thành $ x_0^- $ (hoặc $ +\infty $, $ -\infty $).
		\end{enumerate}
	\end{note}
\end{tomtat}

\subsection{Các dạng toán thường gặp}
\begin{dang}{Thay số trực tiếp}
\end{dang}
\subsubsection{Ví dụ minh hoạ}
\begin{vd}%[1K5YF-2]
	Tính các giới hạn sau
	\begin{listEX}[2]
		\item $ \lim \limits_{x \to 1} (x^2-4x+2)$;
		\item $ \lim \limits_{x \to 2} \dfrac{3x-2}{2x+1}$.
	\end{listEX}
	\loigiai{
		\begin{enumerate}[a)]
			\item $ \lim \limits_{x \to 1} (x^2-4x+2)=\lim \limits_{x \to 1} x^2-\lim \limits_{x \to 1} (4x)+\lim \limits_{x \to 1} 2=1^2-4\lim \limits_{x \to 1} x+2=1-4\cdot 1+2=-1$;
			\item $ \lim \limits_{x \to 2} \dfrac{3x-2}{2x+1}=\dfrac{\lim \limits_{x \to 2} (3x-2)}{\lim \limits_{x \to 2} (2x+1)}=\dfrac{3\lim \limits_{x \to 2} x-2}{2\lim \limits_{x \to 2} x+1}=\dfrac{3 \cdot 2-2}{2\cdot 2+1}=\dfrac{4}{5}$.
		\end{enumerate}
	}
\end{vd}

\begin{vd}%[1K5YF-2]
	Tìm các giới hạn sau
	\begin{enumEX}[a)]{2}
		\item $\lim\limits_{x\to 3} \sqrt{\dfrac{x^2}{x^3-x-6}}$.
		\item $\lim\limits_{x\to -2} \sqrt[3]{\dfrac{2x^4+3x+2}{x^2-x+2}}$.
	\end{enumEX}
	\loigiai{
		\begin{enumerate}[a)]
			\item $\lim\limits_{x\to 3} \sqrt{\dfrac{x^2}{x^3-x-6}}$; do $\lim\limits_{x\to 3} \dfrac{x^2}{x^3-x-6}=\dfrac{3^2}{3^3-3-6}=\dfrac{1}{2}>0$ \\
			$ \Rightarrow \lim\limits_{x\to 3} \sqrt{\dfrac{x^2}{x^3-x-6}}=\sqrt{\dfrac{1}{2}}=\dfrac{\sqrt{2}}{2}$.
			\item $\lim\limits_{x\to -2} \sqrt[3]{\dfrac{2x^4+3x+2}{x^2-x+2}}$; do $\lim\limits_{x\to -2} \dfrac{2x^4+3x+2}{x^2-x+2}=\dfrac{7}{2} \Rightarrow \lim\limits_{x\to -2} \sqrt[3]{\dfrac{2x^4+3x+2}{x^2-x+2}}=\sqrt[3]{\dfrac{7}{2}}=\dfrac{\sqrt[3]{28}}{2}$. 
		\end{enumerate}
	}
\end{vd}


\begin{vd}%[1K5YF-2]
	Cho $f(x)$ là một đa thức thỏa mãn $\displaystyle\lim\limits_{x\to 1}\dfrac{f(x)-16}{x-1}=24$. Tính giới hạn sau $$\displaystyle\lim\limits_{x\to 1}\dfrac{f(x)-16}{\left({x-1}\right)\left(\sqrt{2f(x)+4}+6\right)}.$$
	\loigiai{
		Vì $\displaystyle\lim\limits_{x\to 1}\dfrac{f(x)-16}{x-1}=24$ nên $f(1)=16.$ Khi đó
		$$ \lim\limits_{x\to 1}\dfrac{f(x)-16}{\left({x-1}\right)\left(\sqrt{2f(x)+4}+6\right)} =\frac{1}{12}\cdot \lim\limits_{x\to 1}\dfrac{f(x)-16}{x-1}=2.$$
	}
\end{vd}
% \subsubsection{Bài tập rèn luyện}
% % \subsubsection{Bài tập tự luận}
% \begin{bt}%[1K5YF-2]
% 	Tính các giới hạn sau:
% 	\begin{listEX}[2]
% 		\item $ \lim \limits_{x \to -2} (x^2+5x-2)$;
% 		\item $ \lim \limits_{x \to 1} \dfrac{x^2-1}{x-1}$.
% 	\end{listEX}
% 	\loigiai{
% 		\begin{enumerate}
% 			\item $ \lim \limits_{x \to -2} (x^2+5x-2)=(\lim \limits_{x \to -2} x)^2+\lim \limits_{x \to -2} (5x)-\lim \limits_{x \to -2} 2=(-2)^2+5\cdot (-2)-2=-8$.
% 			\item $ \lim \limits_{x \to 1} \dfrac{x^2-1}{x-1}=\lim \limits_{x \to 1} \dfrac{(x-1)(x+1)}{(x-1)}=\lim \limits_{x \to 1} (x+1)=\lim \limits_{x \to 1} x+1=1+1=2$.
% 		\end{enumerate}
% 	}
% \end{bt}

% \begin{bt}%[1K5YF-2]
% 	Tính các giới hạn sau
% 	\begin{enumEX}[a)]{2}
% 		\item $\lim\limits_{x\to -1} (3x^2-2x+1)$.
% 		\item $\lim\limits_{x\to 2} \dfrac{(x^3-3x)(x+1)}{x^2+3}$.
% 	\end{enumEX}
% 	\loigiai{
% 		\begin{enumerate}[a)]
% 			\item $\lim\limits_{x\to -1} (3x^2-2x+1)=3\lim\limits_{x\to -1} x^2-2\lim\limits_{x\to -1} x+\lim\limits_{x\to -1} 1=3{(1)}^2-2\cdot 1+1=2$.
% 			\item Do $\lim\limits_{x\to 2} (x^2+3)=2^2+3=7\ne 0$ và\\ $$\lim\limits_{x\to 2} (x^3-3x)(x+1)=\lim\limits_{x\to 2} (x^3-3x)\cdot \lim\limits_{x\to 2} (x+1)=(2^3-3\cdot 2)\cdot (2+1)=6$$
% 			Nên $\lim\limits_{x\to 2} \dfrac{(x^3-3x)(x+1)}{x^2+3}=\dfrac{6}{7}$.
% 		\end{enumerate}
% 	}
% \end{bt}

% \begin{bt}%[1K5YF-2]
% 	Tìm các giới hạn sau
% 	\begin{enumEX}[a)]{2}
% 		\item $\lim\limits_{x\to 2} \sqrt{\dfrac{2}{x^2-x+3}}$.
% 		\item $\lim\limits_{x\to -3} \sqrt[3]{\dfrac{-5}{x^2+x-12}}$.
% 	\end{enumEX}
% 	\loigiai{
% 		\begin{enumerate}[a)]
% 			\item $\lim\limits_{x\to 2} \sqrt{\dfrac{x}{x^2-x+3}}$; do $\lim\limits_{x\to 2} \dfrac{2}{x^2-x+3}=\dfrac{2}{2^2-2+3}=\dfrac{2}{5}>0$ \\
% 			$ \Rightarrow \lim\limits_{x\to 2} \sqrt{\dfrac{2}{x^2-x+3}}=\sqrt{\dfrac{2}{5}}=\dfrac{\sqrt{10}}{5}$.
% 			\item $\lim\limits_{x\to -3} \sqrt[3]{\dfrac{-5}{x^2+x-12}}$; do $\lim\limits_{x\to -3} \dfrac{-5}{x^2+x-12}=\dfrac{5}{6} \Rightarrow \lim\limits_{x\to -3} \sqrt[3]{\dfrac{-5}{x^2+x-12}}=\sqrt[3]{\dfrac{5}{6}}=\dfrac{\sqrt[3]{180}}{6}$. 
% 		\end{enumerate}
% 	}
% \end{bt} 

% \begin{bt}%[1K5YF-2]
% 	Cho $f(x)=x-1$ và $g(x)=x^{3}$. Tính các giới hạn sau:
% 	\begin{enumerate}
% 		\item $\lim \limits_{x \rightarrow 1}[3 f(x)-g(x)]$.
% 		\item $\lim \limits_{x \rightarrow 1} \dfrac{[f(x)]^{2}}{g(x)}$.
% 	\end{enumerate}
% 	\loigiai{Ta có $\lim \limits_{x \rightarrow 1} f(x)=\lim \limits_{x \rightarrow 1}(x-1)=\lim \limits_{x \rightarrow 1} x-\lim \limits_{x \rightarrow 1} 1=1-1=0$. Mặt khác, ta thấy $\lim \limits_{x \rightarrow 1} g(x)=\lim \limits_{x \rightarrow 1} x^{3}=1$.
% 		\begin{enumerate}
% 			\item Ta có
% 			$$
% 			\lim \limits_{x \rightarrow 1}[3 f(x)-g(x)]=\lim \limits_{x \rightarrow 1}[3 f(x)]-\lim \limits_{x \rightarrow 1} g(x)=\lim \limits_{x \rightarrow 1} 3 \cdot \lim \limits_{x \rightarrow 1} f(x)-\lim \limits_{x \rightarrow 1} g(x)=3 \cdot 0-1=-1 .
% 			$$
% 			\item  Ta có
% 			$$
% 			\lim \limits_{x \rightarrow 1} \dfrac{[f(x)]^{2}}{g(x)}=\dfrac{\lim \limits_{x \rightarrow 1}[f(x)]^{2}}{\lim \limits_{x \rightarrow 1} g(x)}=\dfrac{\lim \limits_{x \rightarrow 1} f(x) \cdot \lim \limits_{x \rightarrow 1} f(x)}{\lim \limits_{x \rightarrow 1} g(x)}=\dfrac{0}{1}=0.
% 			$$
% 	\end{enumerate}}
% 	\subsubsection{Bài tập trắc nghiệm}
% \end{bt}
\subsubsection{Câu hỏi trắc nghiệm}
\Opensolutionfile{ans}[ans/ans-1K5-2-Dang1]
\begin{ex}%[1T3Y2-1]
	Biết $\lim\limits_{x \to +\infty}f(x)=m$, $\lim\limits_{x \to +\infty}g(x)=n$. Tính $\lim\limits_{x \to +\infty}\left[f(x)+g(x)\right]$.
	\choice
	{\True $m+n$}
	{$m-n$}
	{$mn$}
	{$\dfrac{m}{n}$}
	\loigiai
	{
		Ta có $\lim\limits_{x \to +\infty}\left[f(x)+g(x)\right] = \lim\limits_{x \to +\infty}f(x) + \lim\limits_{x \to +\infty}g(x) = m+n$.
	}
\end{ex}


\begin{ex}%[1K5YF-2]
	Khẳng định nào sau đây là đúng?
	\choice
	{$\lim\limits_{x\to x_0} \sqrt[3]{f(x)+g(x)} =\sqrt[3]{\lim\limits_{x \to x_0} f(x)}+\sqrt[3]{\lim\limits_{x \to x_0} g(x)}$}
	{$\lim\limits_{x\to x_0} \sqrt[3]{f(x)+g(x)} =\lim\limits_{x \to x_0}\sqrt[3]{f(x)}+\lim\limits_{x \to x_0}\sqrt[3]{g(x)}$}
	{\True $\lim\limits_{x\to x_0} \sqrt[3]{f(x)+g(x)} =\sqrt[3]{\lim\limits_{x \to x_0} [f(x)+g(x)]}$}
	{$\lim\limits_{x\to x_0} \sqrt[3]{f(x)+g(x)} =\lim\limits_{x \to x_0}\left[\sqrt[3]{f(x)}+\sqrt[3]{g(x)}\right]$}
	\loigiai{
		Theo định lý về giới hạn của thì $\lim\limits_{x\to x_0} \sqrt[3]{f(x)+g(x)} =\sqrt[3]{\lim\limits_{x \to x_0} [f(x)+g(x)]}$.
	}
\end{ex}


\begin{ex}%[1K5YF-2]
	Cho các giới hạn $\lim \limits_{x\rightarrow x_0}f(x)=2$, $\lim \limits_{x\rightarrow x_0}g(x)=3$. Tính $M=\lim \limits_{x\rightarrow x_0}[3f(x)-4g(x)]$.
	\choice
	{$M=5$}
	{$M=2$}
	{\True $M=-6$}
	{$M=3$}
	\loigiai{
		Ta có $M=\lim \limits_{x\rightarrow x_0}[3f(x)-4g(x)]=3\lim \limits_{x\rightarrow x_0}f(x)-4\lim \limits_{x\rightarrow x_0}g(x)=6-12=-6.$
	}
\end{ex}


\begin{ex}%[1K5YF-2]
	Biết $\lim\limits_{x \to +\infty}f(x)=m$, $\lim\limits_{x \to +\infty}g(x)=n$. Tính $\lim\limits_{x \to +\infty}\left[f(x)-g(x)\right]$.
	\choice
	{ $m+n$}
	{\True $m-n$}
	{$mn$}
	{$\dfrac{m}{n}$}
	\loigiai
	{
		Ta có $\lim\limits_{x \to +\infty}\left[f(x)-g(x)\right] = \lim\limits_{x \to +\infty}f(x) - \lim\limits_{x \to +\infty}g(x) = m-n$.
	}
\end{ex}


\begin{ex}%[1K5YF-2]
	Cho $\lim\limits_{x \to a} f(x)=-\infty$, kết quả của $\lim\limits_{x \to a} [-3\cdot f(x)]$ bằng
	\choice
	{\True $+ \infty$}
	{$0$}
	{$3$}
	{$-\infty$}
	\loigiai{
		Có $\lim\limits_{x \to a} [-3\cdot f(x)]=-3\cdot\lim\limits_{x \to a}f(x)=+ \infty$.
		
	}
\end{ex}


\begin{ex}%[1K5YF-2]
	Cho $k\in \mathbb{Z}$, kết quả của $\lim\limits_{x \to -\infty} x^{2k+1}$ bằng
	\choice
	{$0$}
	{\True $-\infty$}
	{$+\infty$}
	{$5$}
	\loigiai{
		Theo tính chất của giới hạn hàm số, ta có $\lim\limits_{x \to -\infty} x^{2k+1}=-\infty$.
		
	}
\end{ex}


\begin{ex}%[1K5YF-2]
	Cho $\displaystyle\lim\limits_{x \to 3} f(x) =-2$. Giá trị $\displaystyle\lim\limits_{x \to 3} \left[f(x)+4x-1\right]$ bằng
	\choice
	{$5$}
	{$6$}
	{$-11$}
	{\True $9$}
	\loigiai{
		$\displaystyle\lim\limits_{x \to 3} \left[f(x)+4x-1\right] = \lim\limits_{x \to 3} f(x) + \lim\limits_{x \to 3} 4x -1 = -2 + 4 \cdot 3 -1 =9$.
	}
\end{ex}


\begin{ex}%[1K5YF-2]
	Cho $\lim\limits_{x\to 2}f(x)=3$. Giá trị của $\lim\limits_{x \to 2}\left[f(x)+x\right]$ bằng
	\choice
	{\True $5$}
	{$6$}
	{$1$}
	{$4$}
	\loigiai
	{
		Ta có $\lim\limits_{x \to 2}\left[f(x)+x\right] = \lim\limits_{x \to 2}f(x) + \lim\limits_{x \to 2}x = 3+2=5$.
	}
\end{ex}


\begin{ex}%[1K5YF-2]
	Với $k$ là số nguyên dương. Kết quả của giới hạn $\lim\limits_{x\to+\infty}x^{2k}$ là
	\choice
	{\True $+\infty$}
	{$0$}
	{$-\infty$}
	{$1$}
	\loigiai{
		Với $k$ nguyên dương thì $\lim\limits_{x\to +\infty}x^k = +\infty \Rightarrow \lim\limits_{x\to+\infty}x^{2k}=+\infty$.
	}
\end{ex}


\begin{ex}%[1K5YF-2]
	Cho $c$ là hằng số, $k$ là số nguyên dương. Khẳng định nào sau đây \textbf{sai}?	
	\choice
	{\True $\lim\limits_{ x \rightarrow +\infty}c=+\infty$}
	{$\lim\limits_{ x \rightarrow +\infty}\dfrac{c}{x^k}=0$}
	{$\lim\limits_{ x \rightarrow x_0}c=c$}
	{$\lim\limits_{ x \rightarrow x_0}x=x_0$}
	\loigiai{Theo định lý về giới hạn, khẳng định sai là $\lim\limits_{ x \rightarrow +\infty}c=+\infty $.
		
	}
\end{ex}


\begin{ex}%[1K5YF-2]
	Cho $\lim\limits_{x \to +\infty} f(x)=a$,$\lim\limits_{x \to +\infty} g(x)=b$. Hỏi mệnh đề nào sau đây là mệnh đề $\textbf{sai}$?
	\choice
	{$\lim\limits_{x \to +\infty} \left[f(x)\cdot g(x)\right]=ab$}
	{$\lim\limits_{x \to +\infty} \left[f(x)- g(x)\right]=a-b$}
	{$\lim\limits_{x \to +\infty} \left[f(x)+ g(x)\right]=a+b$}
	{\True $\lim\limits_{x \to +\infty} \dfrac{f(x)}{g(x)}=\dfrac{a}{b}$}
	\loigiai{
		Khi $\lim\limits_{x \to +\infty} g(x)=b=0$ thì $\lim\limits_{x \to +\infty} \dfrac{f(x)}{g(x)}=\dfrac{a}{b}$ không đúng.
	}
\end{ex}


\begin{ex}%[1K5YF-2]
	Với $k$ là số nguyên dương thì $\lim\limits_{x \to -\infty} \dfrac{1}{x^k}$ bằng
	\choice
	{$+\infty$}
	{$-\infty$}
	{$x$}
	{\True $0$}
	\loigiai{
		Vì $\left\{\begin{aligned}
			&\lim\limits_{x \to -\infty} 1=1\\
			&\lim\limits_{x \to -\infty} x^k=\pm \infty\\
		\end{aligned}\right. $ nên $\lim\limits_{x \to -\infty} \dfrac{1}{x^k}=0$.
	}
\end{ex}


\begin{ex}%[1K5YF-2]
	Tính $I=\lim\limits_{x \to 2}\left(x^2+x-6\right)$.
	\choice
	{\True $0$}
	{$1$}
	{$2$}
	{$3$}
	\loigiai{
		\begin{eqnarray*}
			\lim\limits_{x \to 2}\left(x^2+x-6\right)&=&\lim\limits_{x \to 2} x^2+\lim\limits_{x \to 2} x-\lim\limits_{x \to 2} 6\\
			&=&4+2-6\\
			&=&0.
		\end{eqnarray*}	
	}
\end{ex}


\begin{ex}%[1K5YF-2]
	Tính $I=\lim\limits_{x \to 1} \dfrac{x^2+2 x+3}{2 x-1}$.	
	\choice
	{ $4$}
	{$5$}
	{\True $6$}
	{$7$}
	\loigiai{
		\begin{eqnarray*}
			\lim\limits_{x \to 1} \dfrac{x^2+2 x+3}{2 x-1}&=&\dfrac{\lim\limits_{x \to 1}\left(x^2+2 x+3\right)}{\lim\limits_{x \to 1}(2 x-1)}\\
			&=&\dfrac{\lim\limits_{x \to 1} x^2+\lim\limits_{x \to 1}(2 x)+\lim\limits_{x \to 1} 3}{\lim\limits_{x \to 1}(2 x)-\lim\limits_{x \to 1} 1}\\
			&=&\dfrac{1+2+3}{2-1}\\
			&=&6.
		\end{eqnarray*}		
	}
\end{ex}
\begin{ex}%[1K5YF-2]
	Tính  $I=\lim\limits_{x \to 0} \dfrac{\left|x\right|}{x}$. 
	\choice
	{$1$}
	{$-1$}
	{\True Không tồn tại }
	{$0$}
	\loigiai{
		Ta có :\\
		$\lim\limits_{x \to 0^{+}} \dfrac{\left|x\right|}{x}=\lim\limits_{x \to 0^{+}} \dfrac{x}{x}=\lim\limits_{x \to 0^{+}} 1=1$.\\
		$\lim\limits_{x \to 0^{-}} \dfrac{\left|x\right|}{x}=\lim\limits_{x \to 0^{-}} \dfrac{(-x)}{x}=\lim\limits_{x \to 0^{-}} (-1)=-1$.\\
		Vậy không tồn tại $\lim\limits_{x \to 0} \dfrac{\left|x\right|}{x}$. 
		
	}
\end{ex}
\Closesolutionfile{ans}
% \begin{indapan}{10}
% 	{ans/ans-1K5-2-Dang1}
% \end{indapan}
\begin{dang}{Phương pháp đặt thừa số chung - kết quả hữu hạn}
	\begin{itemize}
		\item Nếu tam thức bậc hai $ax^2+bx+c$ có hai nghiệm $x_1$, $x_2$ thì $ ax^2+bx+c=a(x-x_1)(x-x_2)$.
		\item $a^n-b^n=(a-b)\left(a^{n-1}+a^{n-2}b+\cdots +ab^{n-2}+b^{n-1}\right)$.
		\item $\lim\limits_{x\to \pm \infty } c=c;\lim\limits_{x\to \pm \infty } \dfrac {c}{x^k}=0$ với $c$ là hằng số và $ k\in \mathbb{N}$.
		\item $a\sqrt{b}=\heva{&\sqrt{a^2b}\quad a\ge 0\\&-\sqrt{a^2b}\quad a<0.}$
	\end{itemize}
\end{dang}
\subsubsection{Ví dụ minh hoạ}
%VD1
\begin{vd}[NB]%[DCHT Toán 11 - KNTT -Nguyễn Văn Hiệp]%[1K5YF-3]
	Tính giới hạn $\lim\limits_{x\to 3} \dfrac {x^2-9}{x-3}$.\dapso{$I=6$.}
	\loigiai{Ta có $\lim\limits_{x\to 3} \dfrac {x^2-9}{x-3}=\lim\limits_{x\to 3} \dfrac {(x-3)(x+3)}{x-3}=\lim\limits_{x\to 3} (x+3)=6$.}
\end{vd}
%VD2
\begin{vd}[TH]%[DCHT Toán 11 - KNTT -Nguyễn Văn Hiệp]%[1K5BF-3]
	Tính giới hạn $I=\lim\limits_{x\to 2} \dfrac {x^2-5x+6}{x-2}$.
	\dapso{$I=-1$.}
	\loigiai{
		$I=\lim\limits_{x\to 2} \dfrac {x^2-5x+6}{x-2}=\lim\limits_{x\to 2} \dfrac {(x-2)(x-3)}{x-2}=\lim\limits_{x\to 2} (x-3)=-1$.
	}
\end{vd}
%VD3
\begin{vd}[TH]%[DCHT Toán 11 - KNTT -Nguyễn Văn Hiệp]%[1K5BF-3]
	Tính giới hạn $\lim\limits_{x\to +\infty } \dfrac {x^4+7}{x^4+1}$.
	\dapso{$1$.}
	\loigiai{Ta có
		$\lim\limits_{x\to +\infty} \dfrac {x^4+7}{x^4+1}=\lim\limits_{x\to +\infty } \dfrac {x^4\left(1+\dfrac {7}{x^4}\right)}{x^4\left(1+\dfrac {1}{x^4}\right)}=\lim\limits_{x\to +\infty } \dfrac {1+\dfrac {7}{x^4}}{1+\dfrac {1}{x^4}}=1$.
	}
\end{vd}
%VD4
\begin{vd}[TH]%[DCHT Toán 11 - KNTT -Nguyễn Văn Hiệp]%[1K5BF-3]
	Tìm giới hạn $\lim\limits_{x\to +\infty } \sqrt {\dfrac {x^2+1}{2x^4+x^2-3}}$.
	\dapso{$0$.}
	\loigiai{
		Ta có $\lim\limits_{x\to +\infty } \sqrt {\dfrac {x^2+1}{2x^4+x^2-3}}=\lim\limits_{x\to +\infty} \dfrac{x}{x^2}\cdot \sqrt {\dfrac {\dfrac {1}{x^2}+\dfrac {1}{x^4}}{2+\dfrac {1}{x^2}-\dfrac {3}{x^4}}}=\lim\limits_{x\to +\infty} \dfrac{1}{x}\cdot \sqrt {\dfrac {\dfrac {1}{x^2}+\dfrac {1}{x^4}}{2+\dfrac {1}{x^2}-\dfrac {3}{x^4}}}=0$.}
\end{vd}
%VD5
\begin{vd}[TH]%[DCHT Toán 11 - KNTT -Nguyễn Văn Hiệp]%[1K5BF-3]
	Tính giới hạn $\lim\limits_{x\to 1} \left(\dfrac {1}{1-x}-\dfrac {3}{1-x^3} \right)$. 
	\dapso{$-1$.}
	\loigiai{$\lim\limits_{x\to 1} \left(\dfrac {1}{1-x}-\dfrac {3}{1-x^3} \right)=\lim\limits_{x\to 1} \dfrac {1+x+x^2-3}{1-x^3}=\lim\limits_{x\to 1} \dfrac {(x-1 )(x+2)}{(1-x)\left( 1+x+x^2 \right)}=\lim\limits_{x\to 1} \dfrac {-(x+2)}{1+x+x^2}=-1$. 
	}
\end{vd}
%VD6
\begin{vd}[VDT]%[DCHT Toán 11 - KNTT -Nguyễn Văn Hiệp]%[1K5KF-3]
	Cho $m,n$ là các số thực khác $0$. Nếu giới hạn $\lim\limits_{x\to -5} \dfrac {x^2+mx+n}{x+5}=3$, hãy tìm $mn$.
	\dapso{$mn=520$.}
	\loigiai{
		Vì $\lim\limits_{x\to -5} \dfrac {x^2+mx+n}{x-5}=3$ nên $ x=-5$ là nghiệm của phương trình $ x^2+mx+n=0$.\\
		$\Rightarrow -5m+n+25=0\Leftrightarrow n=5m-25$.\\
		Khi đó 
		\allowdisplaybreaks
		$\begin{aligned}[t]
			\lim\limits_{x\to -5} \dfrac {x^2+mx+n}{x-1}&=\lim\limits_{x\to -5} \dfrac {x^2+mx+5m-25}{x+5}\\
			&=\lim\limits_{x\to -5} \dfrac {(x+5)(x-5+m)}{x+5}\\
			&=\lim\limits_{x\to -5} (x-5+m)=m-10.
		\end{aligned}$\\
		Ta có $m-10=3\Leftrightarrow m=13\Rightarrow n=40$.\\
		Vậy $mn=13\cdot 40=520$.
	}
\end{vd}
%VD7
\begin{vd}[VDT]%[DCHT Toán 11 - KNTT -Nguyễn Văn Hiệp]%[1K5KF-3]
	Tìm số thực $a$ thỏa mãn $\lim\limits_{x\to +\infty} \dfrac {a\sqrt {2x^2+3}+2024}{2x+2023}=\dfrac {1}{2}$.\dapso{$a=\dfrac {\sqrt {2}}{2}$.}
	\loigiai{Ta có $\lim\limits_{x\to +\infty } \dfrac {a\sqrt {2x^2+3}+2024}{2x+2023}=\dfrac {1}{2}\Leftrightarrow \lim\limits_{x\to +\infty } \dfrac {a\sqrt {2+\dfrac {3}{x^2}}+\dfrac {2024}{x}}{2+\dfrac {2023}{x}}=\dfrac {1}{2}\Leftrightarrow \dfrac {a\sqrt {2}}{2}=\dfrac {1}{2}\Leftrightarrow a=\dfrac {\sqrt {2}}{2}$.}
\end{vd}
% \subsubsection{Bài tập rèn luyện}
% % \subsubsection{Bài tập tự luận}
% %BT1
% \begin{bt}[NB]%[DCHT Toán 11 - KNTT -Nguyễn Văn Hiệp]%[1K5YF-3]
% 	Tính $\lim\limits_{x\to 2} \dfrac {x^2-4}{x-2}$.
% 	\dapso{$4$}
% 	\loigiai{$\lim\limits_{x\to 2} \dfrac {x^2-4}{x-2}=\lim\limits_{x\to 2} \dfrac {(x-2)(x+2)}{x-2}=\lim\limits_{x\to 2} (x+2)=2+2=4$.}
% \end{bt}
% %BT2
% \begin{bt}[NB]%[DCHT Toán 11 - KNTT -Nguyễn Văn Hiệp]%[1K5YF-3]
% 	Tính $\lim\limits_{x\to 5} \dfrac {x^2-12x+35}{25-5x}$.
% 	\dapso{$\dfrac{2}{5}$.}
% 	\loigiai{Ta có $\lim\limits_{x\to 5} \dfrac {x^2-12x+35}{25-5x}=\lim\limits_{x\to 5} \dfrac {(x-7)(x-5)}{5(5-x)}=\lim\limits_{x\to 5} \dfrac {7-x}{5}=\dfrac {2}{5}$.\\
% 		Vậy $\lim\limits_{x\to 5} \dfrac {x^2-12x+35}{25-5x}=\dfrac {2}{5}$.
% 	}
% \end{bt}
% %BT3
% \begin{bt}[TH]%[DCHT Toán 11 - KNTT -Nguyễn Văn Hiệp]%[1K5BF-3]
% 	Tính giới hạn $I=\lim\limits_{x\to 2} \dfrac {x^3-8}{x^2-4}$.
% 	\dapso{$I=3$.}
% 	\loigiai{
% 		Ta có $I=\lim\limits_{x\to 2} \dfrac {x^3-8}{x^2-4}=\lim\limits_{x\to 2} \dfrac {(x-2)\left(x^2+2x+4\right)}{(x-2)(x+2)}=\lim\limits_{x\to 2} \dfrac {x^2+2x+4}{x+2}=\dfrac {12}{4}=3$.
% 	}
% \end{bt}
% %BT4
% \begin{bt}[TH]%[DCHT Toán 11 - KNTT -Nguyễn Văn Hiệp]%[1K5BF-3]
% 	Tìm giới hạn $A=\lim\limits_{x\to 2} \dfrac {x^4-5x^2+4}{x^3-8}$.
% 	\dapso{$A=1$.}
% 	\loigiai{Ta có 
% 		\allowdisplaybreaks
% 		\begin{eqnarray*}
% 			A&=&\lim\limits_{x\to 2} \dfrac {x^4-5x^2+4}{x^3-8}=\lim\limits_{x\to 2} \dfrac {\left(x^2-1\right) \left(x^2-4\right)}{x^3-2^3}\\
% 			&=&\lim\limits_{x\to 2} \dfrac {\left(x^2-1\right)\left(x-2 \right)\left(x+2\right)}{\left(x-2\right)\left(x^2+2x+4 \right)}\\
% 			&=&\lim\limits_{x\to 2} \dfrac {\left(x^2-1\right)(x+2)}{x^2+2x+4}\\
% 			&=&1.
% 	\end{eqnarray*}}
% \end{bt}
% %BT5
% \begin{bt}[TH]%[DCHT Toán 11 - KNTT -Nguyễn Văn Hiệp]%[1K5BF-3]
% 	Tìm giới hạn $\lim\limits_{x\to -\infty }\dfrac {1+3x}{\sqrt {2x^2+3}}$.
% 	\dapso{$-\dfrac {3\sqrt {2}}{2}$.}
% 	\loigiai{
% 		Ta có $\lim\limits_{x\to -\infty } \dfrac {1+3x}{\sqrt {2x^2+3}}=\lim\limits_{x\to -\infty } \dfrac {x\cdot \left(\dfrac {1}{x}+3\right)}{-x\cdot \left(\sqrt {2+\dfrac {3}{x}}\right)}=\lim\limits_{x\to -\infty } \dfrac {\dfrac {1}{x}+3}{-\sqrt {2+\dfrac {3}{x}}}=-\dfrac {3\sqrt {2}}{2}$.
% 	}
% \end{bt}
% %BT6
% \begin{bt}[TH]%[DCHT Toán 11 - KNTT -Nguyễn Văn Hiệp]%[1K5BF-3]
% 	Tìm giới hạn $\lim\limits_{x\to +\infty } \dfrac {2x-\sqrt {3x^2+2}}{5x+\sqrt {x^2+2}}$.
% 	\dapso{$\dfrac {2-\sqrt {3}}{6}$.}
% 	\loigiai{
% 		Ta có $\lim\limits_{x\to +\infty } \dfrac {2x-\sqrt {3x^2+2}}{5x+\sqrt {x^2+2}}=\lim\limits_{x\to +\infty }\dfrac{x}{x}\cdot \dfrac {2-\sqrt {3+\dfrac {2}{x^2}}}{5+\sqrt {1+\dfrac {2}{x^2}}}=\lim\limits_{x\to +\infty } \dfrac {2-\sqrt {3+\dfrac {2}{x^2}}}{5+\sqrt {1+\dfrac {2}{x^2}}}=\dfrac {2-\sqrt {3}}{6}$.
% 	}
% \end{bt}
% %BT7
% \begin{bt}[VDT]%[DCHT Toán 11 - KNTT -Nguyễn Văn Hiệp]%[1K5KF-3]
% 	Giá trị của $\lim\limits_{x\to 1} \dfrac {x^{2024}+x-2}{x^{2023}+x-2}$ bằng $\dfrac {a}{b}$, với $\dfrac {a}{b}$ là phân số tối giản. Tính giá trị của $a^2-b^2$.
% 	\dapso{$4049$.}
% 	\loigiai{Ta có
% 		\allowdisplaybreaks
% 		\begin{eqnarray*}
% 			&&\lim\limits_{x\to 1} \dfrac {x^{2024}+x-2}{x^{2023}+x-2}=\lim\limits_{x\to 1} \dfrac {x^{2024}-1+x-1}{x^{2023}-1+x-1}\\
% 			&=&\lim\limits_{x\to 1} \dfrac {(x-1)(x^{2023}+x^{2022}\cdots +x+1)+x-1}{(x-1)(x^{2022}+x^{2021}+\cdots+x+1)+x-1}\\
% 			&=&\lim\limits_{x\to 1} \dfrac {x^{2023}+x^{2022}\cdots +x+2}{x^{2022}+x^{2021}+\cdots+x+2}\\
% 			&=&\dfrac {1+1+\cdots +1+2}{1+1+\cdots +1+2}=\dfrac {2025}{2024}.
% 		\end{eqnarray*}
% 		Vậy $a^2-b^2=2025^2-2024^2=4049$.}
% \end{bt}
% %BT8
% \begin{bt}[VDT]%[DCHT Toán 11 - KNTT -Nguyễn Văn Hiệp]%[1K5KF-3]
% 	Cho giới hạn $\lim\limits_{x\to 3} \dfrac {x^2+ax+b}{x-3}=3$. Tìm $a$, $b$.
% 	\dapso{$a=-3$, $b=0$.}
% 	\loigiai{
% 		Để $\lim\limits_{x\to 3} \dfrac {x^2+ax+b}{x-3}=3$ thì ta phải có $x^2+ax+b=(x-3)(x-m)$.\\
% 		Khi đó $3-m=3\Leftrightarrow m=0$. Vậy $x^2+ax+b=(x-3 )x=x^2-3x$.\\
% 		Suy ra $a=-3$ và $b=0$.
% 	}
% \end{bt}
% %BT9
% \begin{bt}[VDT]%[DCHT Toán 11 - KNTT -Nguyễn Văn Hiệp]%[1K5KF-3]
% 	Tìm $m$ để $\lim\limits_{x\to -\infty } \dfrac {\sqrt {4x^2+x+1}+4}{mx-2}=\dfrac {1}{2}$.\dapso{$m=-4$.}
% 	\loigiai{Ta có $\lim\limits_{x\to -\infty } \dfrac {\sqrt {4x^2+x+1}+4}{mx-2}=\lim\limits_{x\to -\infty }\dfrac{-x}{x}\cdot  \dfrac {\sqrt {4+\dfrac {1}{x}+\dfrac {1}{x^2}}-\dfrac {4}{x}}{m-\dfrac {2}{x}}=\lim\limits_{x\to -\infty } \dfrac {-\sqrt {4+\dfrac {1}{x}+\dfrac {1}{x^2}}+\dfrac {4}{x}}{m-\dfrac {2}{x}}=-\dfrac {2}{m}$.\\
% 		Theo bài ra ta có $-\dfrac {2}{m}=\dfrac {1}{2}\Leftrightarrow m=-4$.
% 	}
% \end{bt}
% %BT10
% \begin{bt}[VDC]%[DCHT Toán 11 - KNTT -Nguyễn Văn Hiệp]%[1K5GF-3]
% 	Tính giới hạn $\lim\limits_{x\to 1} \left( \dfrac {m}{1-x^m}-\dfrac {n}{1-x^n} \right)$, $m,n\in \mathbb{N^*}$.
% 	\dapso{ $\dfrac{m-n}{2}$.}
% 	\loigiai{
% 		\allowdisplaybreaks
% 		\begin{eqnarray*}
% 			\lim\limits_{x\to 1} \left(\dfrac {m}{1-x^m}-\dfrac {n}{1-x^n} \right)&=&\lim\limits_{x\to 1} \left[ \left( \dfrac {m}{1-x^m}-\dfrac {1}{1-x} \right)-\left( \dfrac {n}{1-x^n}-\dfrac {1}{1-x} \right)\right]\\
% 			&=&\lim\limits_{x\to 1} \left( \dfrac {m}{1-x^m}-\dfrac {1}{1-x} \right)-\lim\limits_{x\to 1} \left( \dfrac {n}{1-x^n}-\dfrac {1}{1-x} \right)=A-B.
% 		\end{eqnarray*}
% 		\allowdisplaybreaks
% 		\begin{eqnarray*}
% 			A&=&\lim\limits_{x\to 1} \left( \dfrac {m}{1-x^m}-\dfrac {1}{1-x} \right)\\
% 			&=&\lim\limits_{x\to 1} \dfrac {m-\left(1+x+x^2+\cdots +x^{m-1}\right)}{1-x^m}\\
% 			&=&\lim\limits_{x\to 1} \dfrac {(1-x)+\left(1-x^2\right)+\cdots +\left(1-x^{m-1}\right)}{1-x^m}\\
% 			&=&\lim\limits_{x\to 1} \dfrac {(1-x)\left[1+(1+x)+\cdots +\left( 1+x+\cdots +x^{m-2} \right) \right]}{(1-x)\left(1+x+\cdots +x^{m-1}\right)}\\
% 			&=&\lim\limits_{x\to 1} \dfrac {1+(1+x)+\cdots +\left(1+x+\cdots +x^{m-2} \right)}{1+x+\cdots +x^{m-1}}\\
% 			&=&\lim\limits_{x\to 1} \dfrac {1+2+\cdots +m-1}{m}\\
% 			&=&\dfrac {m-1}{2}.
% 		\end{eqnarray*}
% 		Tương tự, ta tính được $B=\dfrac {n-1}{2}$.\\
% 		Vậy $\lim\limits_{x\to 1} \left( \dfrac {m}{1-x^m}-\dfrac {n}{1-x^n} \right)=A-B=\dfrac {m-n}{2}$.
% 	}
% \end{bt}
\subsubsection{Câu hỏi trắc nghiệm}
\Opensolutionfile{ans}[ans/ans-1K5-2-Dang2]

\begin{ex}%[DCHT Toán 11 - KNTT -Nguyễn Văn Hiệp]%[1K5BF-3]
	Tìm $\lim\limits_{x\to 3} \dfrac {9-x^2}{x^2-4x+3}$. Kết quả là
	\choice
	{\True $-3$}
	{$4$}
	{$-4$}
	{$3$}
	\loigiai
	{Ta có $\lim\limits_{x\to 3} \dfrac {9-x^2}{x^2-4x+3}=\lim\limits_{x\to 3} \dfrac {(3-x)(3+x)}{(x-3)(x-1)}=\lim\limits_{x\to 3} \dfrac {-(x+3)}{x-1}=-3$.
	}
\end{ex}
%Cau2
\begin{ex}%[DCHT Toán 11 - KNTT -Nguyễn Văn Hiệp]%[1K5YF-3]
	Tìm $\lim\limits_{x\to 4} \dfrac {x^2-16}{x-4}$. Kết quả là
	\choice
	{$7$}
	{\True $8$}
	{$5$}
	{$6$}
	\loigiai
	{Ta có $\lim\limits_{x\to 4} \dfrac {x^2-16}{x-4}=\lim\limits_{x\to 4} \dfrac {(x-4)(x+4)}{x-4}=\lim\limits_{x\to 4} (x+4)=8$.
	}
\end{ex}
%Cau3
\begin{ex}%[DCHT Toán 11 - KNTT -Nguyễn Văn Hiệp]%[1K5YF-3]
	Tính giới hạn $A=\lim\limits_{x\to 1} \dfrac {x^3-1}{x-1}$.
	\choice
	{$A=-\infty $}
	{$A=0$}
	{\True $A=3$}
	{$A=+\infty $}
	\loigiai{Ta có $A=\lim\limits_{x\to 1} \dfrac {x^3-1}{x-1}=\lim\limits_{x\to 1} \dfrac {( x-1 )\left(x^2+x+1 \right)}{x-1}=\lim\limits_{x\to 1} \left(x^2+x+1 \right)=3$.}
\end{ex}
%Cau4
\begin{ex}%[DCHT Toán 11 - KNTT -Nguyễn Văn Hiệp]%[1K5BF-3]
	Chọn kết quả đúng trong các kết quả sau của $\lim\limits_{x\to -1} \dfrac {x^2+2x+1}{2x+2}$ là
	\choice
	{$-\infty $}
	{\True $0$}
	{$\dfrac {1}{2}$}
	{$+\infty $}
	\loigiai
	{$\lim\limits_{x\to -1} \dfrac {x^2+2x+1}{2x+2}=\lim\limits_{x\to -1} \dfrac {(x+1)^2}{2(x+1)}=\lim\limits_{x\to -1} \dfrac {x+1}{2}=0$.
	}
\end{ex}
%Cau5
\begin{ex}%[DCHT Toán 11 - KNTT -Nguyễn Văn Hiệp]%[1K5BF-3]
	Giới hạn $\lim\limits_{x\to 4} \dfrac {x^2+2x-15}{x-3}$ bằng
	\choice
	{$\dfrac {1}{8}$}
	{\True $9$}
	{$+\infty $}
	{$8$}
	\loigiai{$\lim\limits_{x\to 4} \dfrac {x^2+2x-15}{x-3}=\lim\limits_{x\to 4} \dfrac {( x-3 )( x+5 )}{x-3}=9 $.}
\end{ex}
%Cau6
\begin{ex}%[DCHT Toán 11 - KNTT -Nguyễn Văn Hiệp]%[1K5BF-3]
	Tính giới hạn $I=\lim\limits_{x\to -\infty } \dfrac {3x-2}{2x+1}$.
	\choice
	{$I=-2$}
	{$I=-\dfrac {3}{2}$}
	{$I=2$}
	{\True $I=\dfrac {3}{2}$}
	\loigiai{Ta có $I=\lim\limits_{x\to -\infty } \dfrac {3x-2}{2x+1}=\lim\limits_{x\to -\infty } \dfrac {x\left(3-\dfrac {2}{x}\right)}{x\left(2+\dfrac {1}{x}\right)}=\lim\limits_{x\to -\infty } \dfrac {3-\dfrac {2}{x}}{2+\dfrac {1}{x}}=\dfrac {3}{2}$.}
\end{ex}
%Cau7
\begin{ex}%[DCHT Toán 11 - KNTT -Nguyễn Văn Hiệp]%[1K5BF-3]
	$\lim\limits_{x\to +\infty } \dfrac {x-2}{x+3}$ bằng
	\choice
	{$-\dfrac {2}{3}$}
	{\True $1$}
	{$2$}
	{$-3$}
	\loigiai{Ta có $\lim\limits_{x\to +\infty} \dfrac {x-2}{x+3}=\lim\limits_{x\to +\infty } \dfrac {x\left(1-\dfrac {2}{x}\right)}{x\left(1+\dfrac {3}{x}\right)}=\lim\limits_{x\to +\infty } \dfrac {1-\dfrac {2}{x}}{1+\dfrac {3}{x}}=\dfrac {1}{1}=1$.}
\end{ex}
%Cau8
\begin{ex}%[DCHT Toán 11 - KNTT -Nguyễn Văn Hiệp]%[1K5BF-3]
	Giới hạn $\lim\limits_{x\to 1} \dfrac {x^2-3x+2}{x^3-x^2+x-1}$ bằng
	\choice
	{$-2$}
	{$-1$}
	{\True $-\dfrac {1}{2}$}
	{$\dfrac {1}{2}$}
	\loigiai{Ta có $\lim\limits_{x\to 1} \dfrac {x^2-3x+2}{x^3-x^2+x-1}$$=\lim\limits_{x\to 1} \dfrac {( x-1 )( x-2 )}{( x-1 )( x^2+1 )}$$=\lim\limits_{x\to 1} \dfrac {x-2}{x^2+1}$$=-\dfrac {1}{2}$.}
\end{ex}
%Cau9
\begin{ex}%[DCHT Toán 11 - KNTT -Nguyễn Văn Hiệp]%[1K5BF-3]
	Giới hạn $T=\lim\limits_{x\to 1} \dfrac {x^4-3x+2}{x^3+2x-3}$ bằng
	\choice
	{$\dfrac {2}{9}$}
	{$\dfrac {2}{5}$}
	{\True $\dfrac {1}{5}$}
	{$+\infty $}
	\loigiai
	{$T=\lim\limits_{x\to 1} \dfrac {x^4-3x+2}{x^3+2x-3}=\lim\limits_{x\to 1} \dfrac {(x-1)\left(x^3+x^2+x-2\right)}{(x-1)\left(x^2+x+3 \right)}=\lim\limits_{x\to 1} \dfrac {x^3+x^2+x-2}{x^2+x+3}=\dfrac {1^3+1^2+1-2}{1^2+1+3}=\dfrac {1}{5}$.
	}
\end{ex}
%Cau10
\begin{ex}%[DCHT Toán 11 - KNTT -Nguyễn Văn Hiệp]%[1K5BF-3]
	Giới hạn $\lim\limits_{x\to 2} \dfrac {x^2-5x+6}{x^3-x^2-x-2}$ bằng
	\choice
	{$0$}
	{\True $-\dfrac {1}{7}$}
	{$-7$}
	{$+\infty $}
	\loigiai{Ta có $\lim\limits_{x\to 2} \dfrac {x^2-5x+6}{x^3-x^2-x-2}=\lim\limits_{x\to 2} \dfrac {(x-2)(x-3 )}{(x-2)\left(x^2+x+1\right)}=\lim\limits_{x\to 2} \dfrac {x-3}{x^2+x+1}=\dfrac {-1}{7}$.}
\end{ex}
%Cau11
\begin{ex}%[DCHT Toán 11 - KNTT -Nguyễn Văn Hiệp]%[1K5BF-3]
	Tìm $\lim\limits_{x\to 1} \dfrac {x^4-3x^2+2}{x^3+2x-3}$.
	\choice
	{$-\dfrac {5}{2}$}
	{\True $-\dfrac {2}{5}$}
	{$\dfrac {1}{5}$}
	{$+\infty $}
	\loigiai{$\lim\limits_{x\to 1} \dfrac {x^4-3x^2+2}{x^3+2x-3}=\lim\limits_{x\to 1} \dfrac {(x-1)(x+1)\left(x^2-2\right)}{(x-1)\left(x^2+x+3 \right)}=\lim\limits_{x\to 1} \dfrac {(x+1)\left(x^2-2 \right)}{x^2+x+3}=-\dfrac {2}{5}$.}
\end{ex}
%Câu 12
\begin{ex}%[DCHT Toán 11 - KNTT -Nguyễn Văn Hiệp]%[1K5BF-3]
	Tính $\lim\limits_{x\to 2} \left( \dfrac {1}{x^2-3x+2}+\dfrac {1}{x^2-5x+6} \right)$.
	\choice
	{$2 $}
	{$+\infty$}
	{\True $-2$}
	{$0$}
	\loigiai
	{\allowdisplaybreaks
		$\begin{aligned}[t]
			&\lim\limits_{x\to 2} \left( \dfrac {1}{x^2-3x+2}+\dfrac {1}{x^2-5x+6} \right)=\lim\limits_{x\to 2} \dfrac {2x^2-8x+8}{\left(x^2-3x+2 \right) \left(x^2-5x+6\right)}\\
			=&\lim\limits_{x\to 2} \dfrac {2(x-2)^2}{(x-1)(x-2)(x-2)(x-3)}=\lim\limits_{x\to 2} \dfrac {2}{(x-1)(x-3)}=-2.
		\end{aligned}$
	}
\end{ex}
%Cau 13
\begin{ex}%[DCHT Toán 11 - KNTT -Nguyễn Văn Hiệp]%[1K5BF-3]
	Giới hạn $\lim \limits_{ x \to + \infty} \dfrac {\sqrt {x ^2  + 2} - 2} {x - 2}$ bằng
	\choice
	{$- \infty$}
	{\True $1$}
	{$+\infty$}
	{$-1$}
	\loigiai{$\lim\limits_{x\to +\infty } \dfrac {\sqrt {x^2+2}-2}{x-2}=\lim\limits_{x\to +\infty } \dfrac {x\sqrt {1+\dfrac {2}{x^2}}-2}{x-2}=\lim\limits_{x\to +\infty } \dfrac {\sqrt {1+\dfrac {2}{x^2}}-\dfrac {2}{x}}{1-\dfrac {2}{x}}=1$.}
\end{ex}
%Cau 14
\begin{ex}%[DCHT Toán 11 - KNTT -Nguyễn Văn Hiệp]%[1K5BF-3]
	Cho hàm số $ f(x)=\dfrac {(4x+1)^3(2x+1)^4}{(3+2x)^7}$. Tính $\lim\limits_{x\to -\infty}f(x)$.
	\choice
	{$2$}
	{\True $8$}
	{$4$}
	{$0$}
	\loigiai{$\lim\limits_{x\to -\infty} f(x)=\lim\limits_{x\to -\infty } \dfrac {(4x+1)^3(2x+1)^4}{(3+2x)^7}=\lim\limits_{x\to -\infty} \dfrac {\left(4+\dfrac {1}{x} \right)^3\left( 2+\dfrac {1}{x} \right)^4}{\left( \dfrac {3}{x}+2 \right)^7}=2^3=8$.}
\end{ex}
%Cau 15
\begin{ex}%[DCHT Toán 11 - KNTT -Nguyễn Văn Hiệp]%[1K5KF-3]
	Biết $\lim\limits_{x\to 3} \dfrac {x^2+bx+c}{x-3}=8$, $(b,c\in \mathbb{R})$. Tính $P=b+c$.
	\choice
	{\True $P=-13$}
	{$P=-11$}
	{$P=5$}
	{$P=-12 $}
	\loigiai{Vì $\lim\limits_{x\to 3} \dfrac {x^2+bx+c}{x-3}=8$ là hữu hạn nên tam thức $ x^2+bx+c$ có nghiệm $x=3$.\\
		$\Rightarrow 3b+c+9=0\Leftrightarrow c=-9-3b$.\\
		Khi đó
		$\begin{aligned}[t]
			\lim\limits_{x\to 3} \dfrac {x^2+bx+c}{x-3}&=\lim\limits_{x\to 3} \dfrac {x^2+bx-9-3b}{x-3}=\lim\limits_{x\to 3} \dfrac {(x-3)(x+3+b)}{x-3} \\
			&=\lim\limits_{x\to 3} (x+3+b)=8\Leftrightarrow 6+b=8\Leftrightarrow b=2\Rightarrow c=-15.
		\end{aligned}$\\
		Vậy $P=b+c=-13$.}
\end{ex}
%Cau 16
\begin{ex}%[DCHT Toán 11 - KNTT -Nguyễn Văn Hiệp]%[1K5KF-3]
	Cho $a,b$ là số nguyên và $\lim\limits_{x\to 1} \dfrac {ax^2+bx-5}{x-1}=7$. Tính $a^2+b^2+a+b$.
	\choice
	{\True $18$}
	{$1$}
	{$15$}
	{$5$}
	\loigiai{Vì $\lim\limits_{x\to 1} \dfrac {ax^2+bx-5}{x-1}=7$ hữu hạn nên $ x=1$ phải là nghiệm của phương trình $ ax^2+bx-5=0$ suy ra $ a+b-5=0\Rightarrow b=5-a$.\\
		Khi đó $\lim\limits_{x\to 1} \dfrac {ax^2+\left( 5-a \right)x-5}{x-1}=\lim\limits_{x\to 1} \dfrac {\left( x-1 \right)( ax+5 )}{x-1}=a+5=7\Rightarrow a=2$ nên $b=3$.\\
		Suy ra $a^2+b^2+a+b=18$.}
\end{ex}
%Cau 17
\begin{ex}%[DCHT Toán 11 - KNTT -Nguyễn Văn Hiệp]%[1K5KF-3]
	Biết rằng $\lim\limits_{x\to +\infty } \left( \dfrac {x^2+1}{x-2}+ax-b \right)=-5$. Tính tổng $a+b$.
	\choice
	{\True $6$}
	{$7$}
	{$8$}
	{$5$}
	\loigiai{
		\allowdisplaybreaks
		$\begin{aligned}[t]
			&\lim\limits_{x\to +\infty } \left(\dfrac {x^2+1}{x-2}+ax-b\right)=\lim\limits_{x\to +\infty}\left( \dfrac {(a+1)x^2-(2a+b)x+2b+1}{x-2}\right)=-5\\
			\Leftrightarrow& \heva{&a+1=0 \\ &2a+b=5}\Leftrightarrow \heva{&a=-1\\&b=7.}
		\end{aligned}$\\
		Vậy $ a+b=6$.}
\end{ex}
%Cau 18
\begin{ex}%[DCHT Toán 11 - KNTT -Nguyễn Văn Hiệp]%[1K5KF-3]
	Cho hai số thực $ a$ và $ b$ thỏa mãn $\lim\limits_{x\to +\infty} \left( \dfrac {4x^2-3x+1}{x+2}-ax-b\right)=0$. Khi đó $a+b$ bằng
	\choice
	{$-4$}
	{$4$}
	{$7$}
	{\True $-7$}
	\loigiai{$\lim\limits_{x\to +\infty}\left(\dfrac{4x^2-3x+1}{x+2}-ax-b \right)=0\Leftrightarrow \lim\limits_{x\to +\infty } \left( \left( 4-a \right)x-b-11+\dfrac {23}{x+2} \right)=0$.\\
		$\Rightarrow \heva{&4-a=0\\&-11-b=0 }\Leftrightarrow \heva{&a=4\\&b=-11}\Rightarrow a+b=-7$.}
\end{ex}
%Cau 19
\begin{ex}%[DCHT Toán 11 - KNTT -Nguyễn Văn Hiệp]%[1K5KF-3]
	Cho $\lim\limits_{x\to 1} \dfrac {f(x)+1}{x-1}=-1$. Tính $\lim\limits_{x\to 1} \dfrac {\left(x^2+x\right)f(x)+2}{x-1}$.
	\choice
	{$I=5$}
	{$I=-4$}
	{$I=4$}
	{\True $I=-5$}
	\loigiai{$\lim\limits_{x\to 1} \dfrac {\left(x^2+x \right)f(x)+2}{x-1}=\lim\limits_{x\to 1} \dfrac {\left(x^2+x \right)\left(f(x)+1 \right)-x^2-x+2}{x-1}=\lim\limits_{x\to 1} \left( \dfrac {\left(x^2+x\right)(f(x)+1)}{x-1}-x-2\right)=-5$.}
\end{ex}
%Cau 20
\begin{ex}%[DCHT Toán 11 - KNTT -Nguyễn Văn Hiệp]%[1K5GF-3]
	Gọi $A$ là giới hạn của hàm số $f(x)=\dfrac {x+x^2+x^3+\cdots +x^{50}-50}{x-1}$ khi $x$ tiến đến $1$. Tính giá trị của $A$.
	\choice
	{$A$ không tồn tại}
	{$A=1725$}
	{$A=1527$}
	{\True $A=1275$}
	\loigiai
	{Ta có \allowdisplaybreaks
		\begin{eqnarray*}
			\lim\limits_{x\to 1} f(x)&=&\lim\limits_{x\to 1} \dfrac {x+x^2+x^3+\cdots +x^{50}-50}{x-1}\\
			&=&\lim\limits_{x\to 1} \left[ 1+(x+1)+\left(x^2+x+1\right)+\cdots +\left(x^{49}+x^{48}+\cdots +1\right) \right]\\
			&=&1+2+3+\cdots +50=25(1+50)=1275.
		\end{eqnarray*}
		Vậy $A=\lim\limits_{x\to 1} f(x)=1275$.}
\end{ex}
\Closesolutionfile{ans}
% \begin{indapan}{10}
% 	{ans/ans-1K5-2-Dang2}
% \end{indapan}
\begin{dang}{Phương pháp đặt thừa số chung - kết quả vô cực}
	Để tìm giới hạn của hàm số ta cần nhớ
	\begin{itemize}
		\item $\lim\limits_{x\to +\infty} x^k=+\infty$; $\lim\limits_{x\to -\infty} x^k=\heva{& +\infty,k=2n\\& -\infty,k=2n+1.}$
		\item $\lim\limits_{x\to \pm \infty}c=c$; $\lim\limits_{x\to \pm \infty} \dfrac {c}{x^k}=0$; $\lim\limits_{x\to 0} \dfrac {1}{x}=\infty$.
	\end{itemize}
\end{dang}
\subsubsection{Ví dụ minh hoạ}
%VD1
\begin{vd}[NB]%[DCHT Toán 11 - KNTT -Nguyễn Văn Hiệp]%[1K5YF-4]
	Tính $\lim\limits_{x\to +\infty} x^3$.\dapso{$+\infty$.}
	\loigiai{Ta có $\lim\limits_{x\to +\infty} x^3=+\infty$.}
\end{vd}
%VD2
\begin{vd}[TH]%[DCHT Toán 11 - KNTT -Nguyễn Văn Hiệp]%[1K5BF-4]
	Tính $\lim\limits_{x\to -\infty }\left(x^3+3x+1\right)$.
	\dapso{$-\infty$}
	\loigiai{Ta có $\lim\limits_{x\to -\infty } \left( x^3+3x+1 \right)=\lim\limits_{x\to -\infty } \left[ x^3\left( 1+\dfrac {3}{x^2}+\dfrac {1}{x^3} \right) \right]=-\infty $.\\
		Vì $\lim\limits_{x\to -\infty } x^3=-\infty$; $\lim\limits_{x\to -\infty } \left( 1+\dfrac {3}{x^2}+\dfrac {1}{x^3} \right)=1>0$.}
\end{vd}
%VD3
\begin{vd}[TH]%[DCHT Toán 11 - KNTT -Nguyễn Văn Hiệp]%[1K5BF-4]
	Tính $\lim\limits_{x\to -\infty}\left(-4x^5-3x^3+x+1\right)$.
	\dapso{$+\infty$.}
	\loigiai{
		Ta có $\lim\limits_{x\to -\infty } \left( -4x^5-3x^3+x+1 \right)=\lim\limits_{x\to -\infty } x^5\left( -4-\dfrac {3}{x^2}+\dfrac {1}{x^4}+\dfrac {1}{x^5} \right)=+\infty $.\\
		Vì $\heva{&\lim\limits_{x\to -\infty } \left( -4-\dfrac {3}{x^2}+\dfrac {1}{x^4}+\dfrac {1}{x^5} \right)=-4<0 \\& \lim\limits_{x\to -\infty } x^5=-\infty.}$
	}
\end{vd}
%VD4
\begin{vd}[TH]%[DCHT Toán 11 - KNTT -Nguyễn Văn Hiệp]%[1K5BF-4]
	Tính giới hạn $\lim\limits_{x\to -3} \dfrac {x+2}{(x+3)^2}$. 
	\dapso{$-\infty$.}
	\loigiai{Ta có $\lim\limits_{x\to -3} \dfrac {x+2}{(x+3)^2}=-\infty $.\\
		Vì $\lim\limits_{x\to -3} (x+2)=-3+2=-1<0$, $\lim\limits_{x\to -3} (x+3)^2=0$ và $(x+3)^2>0$ khi $ x\ne -3$.}
\end{vd}
%VD5
\begin{vd}[VDT]%[DCHT Toán 11 - KNTT -Nguyễn Văn Hiệp]%[1K5KF-4]
	Tìm tất cả các giá trị nguyên của tham số $m$ để $I=\lim\limits_{x\to +\infty}\left[(m^2-1)x^3+2x\right]=-\infty$.
	\dapso{$m=0$}
	\loigiai{
		Ta có $\lim\limits_{x\to +\infty } \left[ \left(m^2-1 \right)x^3+2x \right]=\lim\limits_{x\to +\infty} x^3\left[ m^2-1+\dfrac {2}{x^2} \right]$.\\
		Vì $\lim\limits_{x\to +\infty} x^3=+\infty $ nên $I=-\infty \Leftrightarrow \lim\limits_{x\to +\infty } \left[ m^2-1+\dfrac {2}{x^2} \right]<0\Leftrightarrow m^2-1<0\Leftrightarrow -1<m<1$.\\
		Do $ m\in \mathbb{Z}$ nên $m=0$.}
\end{vd}
% \subsubsection{Bài tập rèn luyện}
% % \centerline{\fcolorbox{red}{yellow!50}{\bf {BÀI TẬP TỰ LUẬN}}}
% %bt1
% \begin{bt}[NB]%[DCHT Toán 11 - KNTT -Nguyễn Văn Hiệp]%[1K5YF-4]
% 	Tính $\lim\limits_{x\to -\infty} x^2$.\dapso{$+\infty$.}
% 	\loigiai{Ta có $\lim\limits_{x\to -\infty} x^2=+\infty$.}
% \end{bt}
% %bt2
% \begin{bt}[NB]%[DCHT Toán 11 - KNTT -Nguyễn Văn Hiệp]%[1K5YF-4]
% 	Tính $\lim\limits_{x\to -\infty} \left(-x^4-\dfrac{1}{x}\right)$.\dapso{$-\infty$.}
% 	\loigiai{Ta có $\lim\limits_{x\to -\infty} -x^4=-\infty$ và $\lim\limits_{x\to -\infty} \dfrac{1}{x}=0$. Suy ra $\lim\limits_{x\to -\infty} \left(-x^4-\dfrac{1}{x}\right)=-\infty$.}
% \end{bt}
% %bt3
% \begin{bt}[TH]%[DCHT Toán 11 - KNTT -Nguyễn Văn Hiệp]%[1K5BF-4]
% 	Tính giới hạn $\lim\limits_{x\to +\infty }\left(-x^3+5x^2+2x+1\right)$.
% 	\dapso{$-\infty$.}
% 	\loigiai{
% 		Ta có$\lim\limits_{x\to +\infty} \left(-x^3+5x^2+2x+1 \right)=\lim\limits_{x\to +\infty } \left[ x^3\left(-1+\dfrac {5}{x}+\dfrac {2}{x^2}+\dfrac {1}{x^3} \right) \right]$.\\
% 		Do $\lim\limits_{x\to +\infty}x^3=+\infty$; $\lim\limits_{x\to +\infty}\left(-1+\dfrac {5}{x}+\dfrac {2}{x^2}+\dfrac {1}{x^3} \right)=-1<0$ nên $\lim\limits_{x\to +\infty} \left(-x^3+5x^2+2x+1\right)=-\infty$.}
% \end{bt}
% %bt4
% \begin{bt}[TH]%[DCHT Toán 11 - KNTT -Nguyễn Văn Hiệp]%[1K5BF-4]
% 	Tính $\lim\limits_{x\to +\infty} \dfrac {3x^2-x}{x+1}$.
% 	\dapso{$+\infty$.}
% 	\loigiai{
% 		Ta có $\lim\limits_{x\to +\infty} \dfrac {3x^2-x}{x+1}=\lim\limits_{x\to +\infty} \dfrac{x^2}{x}\cdot \left(\dfrac{3-\dfrac{1}{x}}{1+\dfrac{1}{x}} \right)=\lim\limits_{x\to +\infty} x\cdot \left(\dfrac {3-\dfrac {1}{x}}{1+\dfrac {1}{x}} \right)=+\infty $.\\
% 		Vì $\lim\limits_{x\to +\infty} x=+\infty $ và $\lim\limits_{x\to +\infty} \dfrac {3-\dfrac {1}{x}}{1+\dfrac {1}{x}}=3$.}
% \end{bt}
% %bt5
% \begin{bt}[TH]%[DCHT Toán 11 - KNTT -Nguyễn Văn Hiệp]%[1K5BF-4]
% 	Giá trị của giới hạn $\lim\limits_{x\to +\infty} \left(\sqrt {1+2x^2}-x \right)$ là bao nhiêu?
% 	\dapso{$+\infty$.}
% 	\loigiai{Ta có $\lim\limits_{x\to +\infty} \left(\sqrt {1+2x^2}-x\right)=\lim\limits_{x\to +\infty } x\left(\sqrt {\dfrac {1}{x^2}+2}-1 \right)=+\infty$.\\
% 		Vì $\lim\limits_{x\to +\infty}x=+\infty$; $\lim\limits_{x\to +\infty}\left(\sqrt {\dfrac {1}{x^2}+2}-1 \right)=\sqrt {2}-1>0$.
% 	}
% \end{bt}
% %bt6
% \begin{bt}[TH]%[DCHT Toán 11 - KNTT -Nguyễn Văn Hiệp]%[1K5BF-4]
% 	Tính $\lim\limits_{x\to 3} \left( \dfrac {1}{x}-\dfrac {1}{3} \right)\dfrac {1}{(x-3)^3}$.
% 	\dapso{$-\infty$}
% 	\loigiai{$\lim\limits_{x\to 3} \left(\dfrac {1}{x}-\dfrac {1}{3} \right)\dfrac {1}{(x-3)^3}=\lim\limits_{x\to 3} \dfrac {3-x}{3x}\cdot \dfrac {1}{(x-3)^3}=\lim\limits_{x\to 3} \dfrac {-1}{3x(x-3)^2}=-\infty$.}
% \end{bt}
% %bt7
% \begin{bt}[VDT]%[DCHT Toán 11 - KNTT -Nguyễn Văn Hiệp]%[1K5KF-4]
% 	Có bao nhiêu giá trị $ m$ nguyên thuộc đoạn $[-20;20]$ để $\lim\limits_{x\to +\infty } \left( \sqrt {4x^2-3x+2}+mx-1 \right)=-\infty$?
% 	\dapso{$18$.}
% 	\loigiai{Ta có
% 		\allowdisplaybreaks
% 		$\begin{aligned}[t]
% 			\lim\limits_{x\to +\infty} \left(\sqrt {4x^2-3x+2}+mx-1 \right)&=\lim\limits_{x\to +\infty} \left(x\sqrt {4-\dfrac {3}{x}+\dfrac {2}{x^2}}+mx-1\right)\\
% 			&=\lim\limits_{x\to +\infty} x\left(\sqrt {4-\dfrac {3}{x}+\dfrac {2}{x^2}}+m-\dfrac {1}{x}\right).
% 		\end{aligned}$\\
% 		Mà $\lim\limits_{x\to +\infty } x=+\infty $ và $\lim\limits_{x\to +\infty } \left(\sqrt {4-\dfrac {3}{x}+\dfrac {2}{x^2}}+m-\dfrac {1}{x} \right)=2+m$ nên $\lim\limits_{x\to +\infty } \left( \sqrt {4x^2-3x+2}+mx-1 \right)=-\infty $ khi $2+m<0\Leftrightarrow m<-2$.\\
% 		Do $ m$ nguyên thuộc đoạn $[-20;20]$ nên $m\in \{ -20;-19;-18;\ldots;-3 \}$.\\
% 		Vậy có $18$ giá trị $ m$ nguyên thuộc đoạn $[-20;20]$ thỏa bài toán.}
% \end{bt}

\subsubsection{Câu hỏi trắc nghiệm}
\Opensolutionfile{ans}[ans/ans-1K5-2-Dang3]
%Câu 1
\begin{ex}%[DCHT Toán 11 - KNTT -Nguyễn Văn Hiệp]%[1K5YF-4]
	Giá trị của $\lim\limits_{x\to -\infty}(-x^3)$ bằng
	\choice
	{\True $+\infty$}
	{$-\infty$}
	{$1$}
	{$-1$}
	\loigiai{Ta có $\lim\limits_{x\to -\infty}(-x^3)=+\infty $.}
\end{ex}
%Cau2
\begin{ex}%[DCHT Toán 11 - KNTT -Nguyễn Văn Hiệp]%[1K5BF-4]
	Giới hạn$\lim\limits_{x\to -\infty} \left(3x^3+5x^2-9\sqrt {3}x-2022 \right)$ bằng
	\choice
	{\True $-\infty $}
	{$3$}
	{$-3$}
	{$+\infty $}
	\loigiai{Ta có $\lim\limits_{x\to -\infty } \left( 3x^3+5x^2-9\sqrt {3}x-2022 \right)=\lim\limits_{x\to -\infty } x^3\left( 3+5\cdot \dfrac {1}{x}-9\sqrt {3}\cdot \dfrac {1}{x^2}-2022\cdot \dfrac {1}{x^3} \right)=-\infty$.}
\end{ex}
%Cau3
\begin{ex}%[DCHT Toán 11 - KNTT -Nguyễn Văn Hiệp]%[1K5BF-4]
	Tính $\lim\limits_{x\to -\infty } ( x^3+3x-3 )$.
	\choice
	{$2$}
	{$1$}
	{\True $-\infty $}
	{$+\infty $}
	\loigiai{Ta có $\lim\limits_{x\to -\infty} \left( x^3+3x-3 \right)=\lim\limits_{x\to -\infty } \left[ x^3\left( 1+\dfrac {3}{x^2}-\dfrac {3}{x^3} \right) \right]=-\infty $.\\
		Vì $\lim\limits_{x\to -\infty}x^3=-\infty$; $\lim\limits_{x\to -\infty}\left(1+\dfrac {3}{x^2}-\dfrac {3}{x^3}\right)=1>0$.}
\end{ex}
%Cau4
\begin{ex}%[DCHT Toán 11 - KNTT -Nguyễn Văn Hiệp]%[1K5BF-4]
	Với $ k$ là số nguyên dương chẵn. Kết quả của $\lim\limits_{x\to -\infty } \left(-3x^k \right)$ là
	\choice
	{$0$}
	{\True $-\infty $}
	{$-3x_0^k$}
	{$+\infty $}
	\loigiai{Ta có $\lim\limits_{x\to -\infty } x^k=+\infty $ khi $ k$ là số nguyên dương chẵn.\\
		Suy ra $\lim\limits_{x\to -\infty } \left(-3x^k\right)=-\infty $.}
\end{ex}
%Cau5
\begin{ex}%[DCHT Toán 11 - KNTT -Nguyễn Văn Hiệp]%[1K5BF-4]
	Cho hai hàm số $ f(x)$, $g(x)$ thỏa mãn $\lim\limits_{x\to 1} f(x)=2$ và $\lim\limits_{x\to 1} g(x)=+\infty $. Giá trị của $\lim\limits_{x\to 1} [ f(x)\cdot g(x)]$ bằng
	\choice
	{\True $+\infty $}
	{$-\infty $}
	{$2$}
	{$-2$}
	\loigiai{Theo quy tắc giới hạn vô cực ta có $\lim\limits_{x\to 1} f(x)=2>0$ và $\lim\limits_{x\to 1} g( x )=+\infty $ thì $\lim\limits_{x\to 1} [f(x)\cdot g(x)]=+\infty $.}
\end{ex}
%Cau6
\begin{ex}%[DCHT Toán 11 - KNTT -Nguyễn Văn Hiệp]%[1K5BF-4]
	Giới hạn $\lim\limits_{x\to -2} \dfrac {x+1}{(x+2)^2}$ bằng
	\choice
	{\True $-\infty $}
	{$\dfrac {3}{16}$}
	{$0$}
	{$+\infty $}
	\loigiai{Ta có $\lim\limits_{x\to -2} \dfrac {x+1}{(x+2)^2}=-\infty$.\\
		Vì $\lim\limits_{x\to -2} (x+1)=-2+1=-1<0$, $\lim\limits_{x\to -2} (x+2)^2=0$ và $(x+2)^2>0$ khi $ x\ne -2$.}
\end{ex}
%Cau7
\begin{ex}%[DCHT Toán 11 - KNTT -Nguyễn Văn Hiệp]%[1K5BF-4]
	$\lim\limits_{x\to -\infty }\left(-3x^3+2x\right)$ bằng
	\choice
	{$-\infty$}
	{\True $+\infty$}
	{$1$}
	{$-1$}
	\loigiai{Ta có $\lim\limits_{x\to -\infty }\left( -3x^3+2x \right)=\lim\limits_{x\to -\infty}x^3\left(-3+\dfrac {2}{x^2} \right)=+\infty $.\\
		Vì $\lim\limits_{x\to -\infty } x^3=-\infty $ và $\lim\limits_{x\to -\infty } \left( -3+\dfrac {2}{x^2} \right)=-3<0$.}
\end{ex}
%Cau8
\begin{ex}%[DCHT Toán 11 - KNTT -Nguyễn Văn Hiệp]%[1K5BF-4]
	Tìm $L=\lim\limits_{x\to -1} \dfrac {2x^2+x-3}{(x+1)^2}$.
	\choice
	{$L=+\infty $}	
	{$L=2$}
	{Không tồn tại $\lim\limits_{x\to -1} \dfrac {2x^2+x-3}{(x+1)^2}$}
	{\True $L=-\infty $}
	\loigiai{$\lim\limits_{x\to -1} \dfrac {2x^2+x-3}{(x+1)^2}=-\infty $ vì $\heva{& \lim\limits_{x\to -1} (2x^2+x-3)=-2 \\ & \lim\limits_{x\to -1} (x+1)^2=0 \\ & x\to -1\Rightarrow (x+1)^2>0.}$}
\end{ex}
%Cau9
\begin{ex}%[DCHT Toán 11 - KNTT -Nguyễn Văn Hiệp]%[1K5BF-4]
	$\lim\limits_{x\to +\infty } \left(-2x^3-2x\right)$ bằng
	\choice
	{\True $-\infty$}
	{$+\infty$}
	{$2$}
	{$-2$}
	\loigiai{Ta có $\lim\limits_{x\to +\infty} \left(-2x^3-2x\right)=\lim\limits_{x\to +\infty}x^3\left(-2-\dfrac {2}{x^2}\right)$.\\
		Mà $\lim\limits_{x\to +\infty}x^3=+\infty$; $\lim\limits_{x\to +\infty}(-2-\dfrac {2}{x^2})=-2<0$ nên $\lim\limits_{x\to +\infty}x^3\left(-2-\dfrac {2}{x^2}\right)=-\infty$.\\
		Vậy $\lim\limits_{x\to +\infty }\left(-2x^3-2x\right)=-\infty$.}
\end{ex}
%Cau10
\begin{ex}%[DCHT Toán 11 - KNTT -Nguyễn Văn Hiệp]%[1K5BF-4]
	Tính giới hạn $\lim\limits_{x\to -\infty } \dfrac {x^2+1}{x-2}$.
	\choice
	{$1$}
	{$-\dfrac {1}{2}$}
	{$+\infty $}
	{\True $-\infty $}
	\loigiai{$\lim\limits_{x\to -\infty} \dfrac {x^2+1}{x-2}=\lim\limits_{x\to -\infty } \dfrac {x^2}{x}\cdot \dfrac {1+\dfrac {1}{x^2}}{1-\dfrac {2}{x}}=\lim\limits_{x\to -\infty } x\cdot \dfrac {1+\dfrac {1}{x^2}}{1-\dfrac {2}{x}}$.\\
		Do $\lim\limits_{x\to -\infty } x=-\infty $ và $\lim\limits_{x\to -\infty } \dfrac {1+\dfrac {1}{x^2}}{1-\dfrac {2}{x}}=1$ nên $\lim\limits_{x\to -\infty } \dfrac {x^2+1}{x-2}=-\infty $.}
\end{ex}
%Câu 11
\begin{ex}%[DCHT Toán 11 - KNTT -Nguyễn Văn Hiệp]%[1K5KF-4]
	Trong các mệnh đề sau, mệnh đề nào đúng?
	\choice
	{\True $\lim\limits_{x\to -\infty } \dfrac {\sqrt {x^4-x}}{1-2x}=+\infty $}
	{$\lim\limits_{x\to -\infty } \dfrac {\sqrt {x^4-x}}{1-2x}=1$}
	{$\lim\limits_{x\to -\infty } \dfrac {\sqrt {x^4-x}}{1-2x}=-\infty $}
	{$\lim\limits_{x\to -\infty } \dfrac {\sqrt {x^4-x}}{1-2x}=0$}
	\loigiai{
		Vì $\lim\limits_{x\to -\infty } \dfrac {\sqrt {x^4-x}}{1-2x}=\lim\limits_{x\to -\infty} \dfrac {x^2 \cdot \sqrt {1-\dfrac {1}{x^3}}}{x\cdot \left( \dfrac {1}{x}-2 \right)}=\lim\limits_{x\to -\infty } x\cdot \dfrac {\sqrt {1-\dfrac {1}{x^3}}}{\dfrac {1}{x}-2}=+\infty $. }
\end{ex}
%Cau 12
\begin{ex}%[DCHT Toán 11 - KNTT -Nguyễn Văn Hiệp]%[1K5KF-4]
	Biết $\lim\limits_{x\to -2} f(x)=-1$. Khi đó $\lim\limits_{x\to -2} \dfrac {f(x)}{(x+2)^4}$ bằng
	\choice
	{$-1$}
	{$+\infty $}
	{\True $-\infty $}
	{$0$}
	\loigiai{Ta có $\lim\limits_{x\to -2} f(x)=-1<0$; $\lim\limits_{x\to -2} (x+2)^4=0$ và $\forall x\ne -2$ thì $(x+2)^4>0$.\\
		Suy ra $\lim\limits_{x\to -2} \dfrac {f(x)}{(x+2)^4}=-\infty $.}
\end{ex}
%Cau 13
\begin{ex}%[DCHT Toán 11 - KNTT -Nguyễn Văn Hiệp]%[1K5KF-4]
	Biết $\lim\limits_{x\to 1} f(x)=-2$. Khi đó $\lim\limits_{x\to 1} \dfrac {f(x)}{(x-1)^2}$ bằng
	\choice
	{\True $-\infty $}
	{$0$}
	{$+\infty$}
	{$-2$}
	\loigiai{Có $\lim\limits_{x\to 1} f(x)=-2<0$, $\lim\limits_{x\to 1} (x-1)^2=0$ và $(x-1)^2>0$, $\forall x\ne 1$ nên $\lim\limits_{x\to 1} \dfrac {f(x)}{(x-1)^2}=-\infty $.}
\end{ex}
%Cau 14
\begin{ex}%[DCHT Toán 11 - KNTT -Nguyễn Văn Hiệp]%[1K5KF-4]
	Có bao nhiêu giá trị nguyên của tham số $m$ thuộc $[-5;5]$ để $L=\lim\limits_{x\to +\infty}\left[x-2(m^2-4)x^3\right]=-\infty$?
	\choice
	{$5$}
	{$10$}
	{$3$}
	{\True $6$}
	\loigiai{Ta có $\lim\limits_{x\to +\infty }\left[ x-2\left( m^2-4 \right)x^3 \right]=\lim\limits_{x\to +\infty } x^3\left[\dfrac {1}{x^2}-2\left( m^2-4 \right) \right]$.\\
		Ta có
		\allowdisplaybreaks
		\begin{eqnarray*}
			&&\lim\limits_{x\to +\infty}x^3=+\infty \Rightarrow L=-\infty \Leftrightarrow \lim\limits_{x\to +\infty} \left[\dfrac {1}{x^2}-2\left(m^2-4\right) \right]<0\\
			&\Leftrightarrow& -2(m^2-4)<0\Leftrightarrow m^2-4>0\Leftrightarrow \hoac{&m>2\\& m<-2.}
		\end{eqnarray*}
		Lại có $m$ thuộc đoạn $[-5;5]$ nên các giá trị nguyên thỏa mãn bài toán của $m$ là $\{-5;-4;-3;3;4;5\}$.\\
		Vậy có $6$ số nguyên thỏa mãn bài toán.}
\end{ex}
%Câu 15
\begin{ex}%[DCHT Toán 11 - KNTT -Nguyễn Văn Hiệp]%[1K5GF-4]
	Có bao nhiêu giá trị $m$ nguyên thuộc đoạn $[-20;20 ]$ để $\lim\limits_{x\to -\infty } \left( \sqrt {4x^2-3x+2}+mx-1 \right)=-\infty$?
	\choice
	{$21$}
	{$22$}
	{\True $18$}
	{$41$}
	\loigiai{Ta có $\lim\limits_{x\to -\infty } \left( \sqrt {4x^2-3x+2}+mx-1 \right)=\lim\limits_{x\to -\infty } x\left( -\sqrt {4-\dfrac {3}{x}+\dfrac {2}{x^2}}+m-\dfrac {1}{x} \right)$.\\
		Có $\lim\limits_{x\to -\infty } x=-\infty $ và $\lim\limits_{x\to -\infty } \left( -\sqrt {4-\dfrac {3}{x}+\dfrac {2}{x^2}}+m-\dfrac {1}{x} \right)=m-2$\\
		Để $\lim\limits_{x\to -\infty } \left( \sqrt {4x^2-3x+2}+mx-1 \right)=-\infty $ suy ra $ m-2>0\Leftrightarrow m>2$.\\
		Với $ m\in \mathbb{Z}$ và $ m\in [ -20;20 ]$ có $ m\in \{ 3;4;5;\ldots ;20 \}$ thỏa mãn yêu cầu bài toán.\\
		Kết luận: Vậy có $18$ giá trị nguyên của $ m$ thỏa mãn yêu cầu bài toán.}
\end{ex}
\Closesolutionfile{ans}
% \begin{indapan}{10}
% 	{ans/ans-1K5-2-Dang3}
% \end{indapan}
\begin{dang}{Phương pháp lượng liên hợp kết quả hữu hạn}
	Nội dung và phương pháp giải
\end{dang}
\subsubsection{Ví dụ minh hoạ}
\Opensolutionfile{ans}[ans/ans-1K5-2-Dang4]
\setcounter{vd}{0}
\begin{vd}%[1K5BF-5]
	Cho $P = \lim \limits_{x \to 2} \dfrac{\sqrt{x+2}-2}{x-2}$. Tính $P$.
	\choice
	{\True $P = \dfrac{1}{4}$}
	{$P = \dfrac{1}{2}$}
	{$P = 1$}
	{$P = 0$}
	\loigiai{
		Ta có: $\lim \limits_{x \to 2} \dfrac{\sqrt{x+2}-2}{x-2} = \lim \limits_{x \to 2} \dfrac{x-2}{(x-2) \left(\sqrt{x + 2}+2 \right)} = \lim \limits_{x \to 2} \dfrac{1}{\sqrt{x + 2}+2} = \dfrac{1}{4}$. \\
		Vậy $P = \dfrac{1}{4}$.
	}
\end{vd}
\begin{vd}%[1K5BF-5]
	Cho $m$ là hằng số. Tính $\lim\limits_{x\to 1}\dfrac{\sqrt{x+3}-2}{x^2+mx-x-m}$.
	\choice
	{$\dfrac{1}{m}$}
	{$1$}
	{$\dfrac{1}{4}$}
	{\True $\dfrac{1}{4(m+1)}$}
	\loigiai{
		$\lim\limits_{x\to 1}\dfrac{\sqrt{x+3}-2}{x^2+mx-x-m}=\lim\limits_{x\to 1}\dfrac{x-1}{(x-1)(x+m)\left(\sqrt{x+3}+2\right)}=\dfrac{1}{4(m+1)}$.
	}
\end{vd}
\begin{vd}%[1K5BF-5]
	Biết $\lim\limits_{x\to -\infty}\left(\sqrt{x^2+1}+x+1\right)=a$. Tính $2a+1.$
	\choice
	{$-1$}
	{$-3$}
	{$0$}
	{\True $3$}
	\loigiai{
		{\allowdisplaybreaks
			\begin{eqnarray*}
				\lim\limits_{x\to -\infty}\left(\sqrt{x^2+1}+x+1\right)
				&=&\lim\limits_{x\to -\infty}\dfrac{-2x}{\sqrt{x^2+1}-(x+1)}\\
				&=&\lim\limits_{x\to -\infty}\dfrac{-2}{-\sqrt{1+\dfrac{1}{x^2}}-\left(1+\dfrac{1}{x}\right)}\\
				&=&1\\
				&\Rightarrow& a=1.	
			\end{eqnarray*}
			Vậy $2a+1=3$.
		}		
	}
\end{vd}


\begin{vd}%[1K5KF-5]
	Biết $\lim\limits_{x\rightarrow +\infty} \left(\sqrt{4x^2-3x+1}-(ax+b)\right) = 0.$ Tính giá trị biểu thức $T=a-4b$.
	\choice
	{$T=-2$}
	{\True $T=5$}
	{$T=-1$}
	{$T=3$}
	\loigiai{
		Từ giả thiết, đường thẳng $y=ax+b$ là tiệm cận xiên của đồ thị hàm số $y=\sqrt{4x^2-3x+1}$, khi $x\to +\infty.$ Từ đó,
		\begin{eqnarray*}\begin{array}{ccl} a&=&\lim\limits_{x\rightarrow +\infty}\dfrac{\sqrt{4x^2-3x+1}}x=2,\\
				b&=&\lim\limits_{x\rightarrow +\infty}\left(\sqrt{4x^2-3x+1}-2x\right)\\
				&=&\lim\limits_{x\rightarrow +\infty}\dfrac{-3x+1}{\sqrt{4x^2-3x+1}+2x}\\
				&=&\lim\limits_{x\rightarrow +\infty}\dfrac{-3+\frac1x}{\sqrt{4-\frac3x+\frac1{x^2}}+2}=-\dfrac34.
			\end{array}
		\end{eqnarray*}
		Suy ra $a-4b=5.$}
\end{vd}
\begin{vd}%[1K5GF-5]
	Cho $f(x)$ là hàm đa thức thỏa $\lim\limits_{x \to 2}\dfrac{f(x)+1}{x-2}=a$ và tồn tại $\lim\limits_{x \to 2}\dfrac{\sqrt{f(x)+2x+1}-x}{x^2-4}=T$. Chọn đẳng thức đúng
	\choice
	{$T=\dfrac{a+2}{16}$}
	{$T=\dfrac{a+2}{8}$}
	{$T=\dfrac{a-2}{8}$}
	{\True $T=\dfrac{a-2}{16}$}
	\loigiai{
		Vì $f(x)$ là đa thức và $\lim\limits_{x \to 2}\dfrac{f(x)+1}{x-2}=a$ nên suy ra $f(x)+1=(x-2)g(x), g(2)=a$.\\
		Do đó
		\begin{eqnarray*}
			&T&=\lim\limits_{x \to 2}\dfrac{\sqrt{(x-2)g(x)+2x}-x}{x^2-4}\\
			& &=\lim\limits_{x \to 2}\dfrac{(x-2)g(x)+2x-x^2}{(x-2)(x+2)\left[\sqrt{(x-2)g(x)+2x}+x\right]}\\
			& &=\lim\limits_{x \to 2}\dfrac{g(x)-x}{(x+2)\left[\sqrt{(x-2)g(x)+2x}+x\right]} \\
			& &=\dfrac{a-2}{16}.
		\end{eqnarray*}
	}
\end{vd}
% \subsubsection{Bài tập rèn luyện}
\subsubsection{Câu hỏi trắc nghiệm}
\Opensolutionfile{ans}[ans/ans-1K5-2-Dang4]
%\setcounter{ex}{0}
\begin{ex}%[1K5BF-5]
	Biết $\displaystyle\lim_{x\rightarrow +\infty}\left(\sqrt{x^2+ax-1}-x\right)=5$. Khi đó giá trị của tham số $a$ là
	\choice
	{\True $10$}
	{$-6$}
	{$6$}
	{$-10$}
	\loigiai{
		$\displaystyle\lim_{x\rightarrow +\infty}\left(\sqrt{x^2+ax-1}-x\right)=\lim_{x\rightarrow +\infty}\dfrac{x^2+ax-1-x^2}{\sqrt{x^2+ax-1}+x}=\lim_{x\rightarrow +\infty}\dfrac{a-\dfrac{1}{x}}{\sqrt{1+\dfrac{a}{x}-\dfrac{1}{x^2}}+1}=\dfrac{a}{2}=5$.
		\\ Suy ra $a=10$. 
	}
\end{ex}
\begin{ex}%[1K5KF-5]
	Có tất cả bao nhiêu giá trị nguyên của tham số m để $\displaystyle\lim\limits_{x\to +\infty}(\sqrt{{x^2+m^2x}}-x)=\dfrac{1}{2}$?
	\choice
	{$0$}
	{$1$}
	{\True $2$}
	{$4$}
	\loigiai{
		Ta có \begin{align*}
			\displaystyle \lim \limits_{x\to +\infty}\left (\sqrt{x^2+m^2x}-x\right ) =&\displaystyle \lim \limits_{x\to +\infty}\dfrac{m^2x}{\sqrt{x^2+m^2x}+x}\\
			=& \displaystyle \lim \limits_{x\to +\infty}\dfrac{m^2}{\sqrt{1+\frac{m^2}{x}} +1}\\
			=&\dfrac{m^2}{2}.  
		\end{align*}	
		Do đó $\displaystyle\lim\limits_{x\to +\infty}(\sqrt{{x^2+m^2x}}-x)=\dfrac{1}{2}\Leftrightarrow \dfrac{m^2}{2}=\dfrac{1}{2}\Leftrightarrow m=\pm 1.$
		Vậy có hai giá trị nguyên của tham số $m$ thỏa mãn yêu cầu của bài toán.
	}
\end{ex}
\begin{ex}%[1K5BF-5]
	Tính $L=\lim\limits_{x \to - \infty} \left( \sqrt{x^2 -7x+1}- \sqrt{x^2-3x+2}\right)$.
	\choice
	{$L= + \infty$}
	{$L= - \infty$}
	{\True $L= 2$}
	{$L= -2$}
	\loigiai{
		\begin{align*}
			L &= \lim\limits_{x \to - \infty} \dfrac{-4x-1}{\sqrt{x^2 - 7x +1}+\sqrt{x^2-3x+2}}\\
			&=\lim\limits_{x \to - \infty} \dfrac{-4x-1}{-x \sqrt{1-\dfrac{7}{x}+\dfrac{1}{x^2}}-x \sqrt{1-\dfrac{3}{x}+\dfrac{2}{x^2}}}\\
			&= \lim\limits_{x \to - \infty} \dfrac{-4-\dfrac{1}{x}}{- \sqrt{1-\dfrac{7}{x}+\dfrac{1}{x^2}}-\sqrt{1-\dfrac{3}{x}+\dfrac{2}{x^2}}}\\
			&=2.
		\end{align*}
	}
\end{ex}


\begin{ex}%[1K5BF-5]
	Giá trị của $\lim\limits_{x \to 1} \dfrac{\sqrt {2x + 1}  - \sqrt {x + 2} }{x - 1}$  là 
	\choice
	{$- \dfrac{ \sqrt{3} }{5}$}
	{$-\dfrac{\sqrt{3} }{6}$}
	{\True $\dfrac{\sqrt{3}}{6}$}
	{$\dfrac{\sqrt{3}}{5}$}
	\loigiai{
		Ta có $$\lim\limits_{x \to 1} \dfrac{\sqrt{2x+1}+\sqrt{x+2}}{x-1}=\lim\limits_{x \to 1}\dfrac{x-1}{(x-1)\left(\sqrt{2x+1}-\sqrt{x+2} \right)}=\lim\limits_{x \to 1}\dfrac{1}{\sqrt{2x+1}+\sqrt{x+2}}=\dfrac{1}{2\sqrt{3}}.$$
	}
\end{ex}

\begin{ex}%[1K5BF-5]
	Cho $\displaystyle\lim_{x \to 0}\dfrac{1-\sqrt[3]{1-x}}{x}=\dfrac{m}{n}$, trong đó $m,n$ là các số nguyên và $\dfrac{m}{n}$ tối giản.\\ Tính $A=2m-n$.
	\choice
	{$A=1$}
	{\True $A=-1$}
	{$A=0$}
	{$A=-2$}
	\loigiai
	{Ta có $\displaystyle\lim_{x \to 0}\dfrac{1-\sqrt[3]{1-x}}{x}=\displaystyle\lim_{x \to 0}\dfrac{x}{x\cdot\left(1+\sqrt[3]{1-x}+\sqrt[3]{(1-x)^2}\right)}=\dfrac{1}{3}.$\\
		Vậy $A=2m-n=2\cdot1-3=-1.$}
\end{ex}
\begin{ex}%[1K5BF-5]
	Giới hạn $\displaystyle\lim\limits_{x\to 3}\dfrac{x+1-\sqrt{5x+1}}{x-\sqrt{4x-3}}=\dfrac{a}{b}$, với $a,b\in\mathbb{Z},b>0$ và $\dfrac{a}{b}$ là phân số tối giản. Giá trị của $a-b$ là
	\choice
	{$\dfrac{1}{9}$}
	{$-1$}
	{$\dfrac{9}{8}$}
	{\True $1$}
	\loigiai{
		Ta có
		\begin{eqnarray*}
			\displaystyle\lim\limits_{x\to 3}\dfrac{x+1-\sqrt{5x+1}}{x-\sqrt{4x-3}}&=& \lim\limits_{x\to 3}\dfrac{\dfrac{(x+1)^2-(5x+1)}{x+1+\sqrt{5x+1}}}{\dfrac{x^2-(4x-3)}{x+\sqrt{4x-3}}} \\
			&=& \lim\limits_{x\to 3}\dfrac{\left(x+\sqrt{4x-3}\right)(x-3)x}{\left(x+1+\sqrt{5x+1}\right)(x-3)(x-1)} \\
			&=& \lim\limits_{x\to 3}\dfrac{x\left(x+\sqrt{4x-3}\right)}{\left(x+1+\sqrt{5x+1}\right)(x-1)}=\dfrac{9}{8}.
		\end{eqnarray*}
		Vậy $a=9,b=8$, suy ra $a-b=1$.
	}
\end{ex}
\begin{ex}%[1K5BF-5]
	Cho $\underset{x\to 4}{\mathop{\lim}}\,\dfrac{\sqrt{3x+4}-4}{x-4}=\dfrac{a}{b}$, với $\dfrac{a}{b}$ là phân số tối giản. Tính $2a+b^2$.
	\choice
	{$22$}
	{$66$}
	{$14$}
	{\True $70$}
	\loigiai{
		Có $\underset{x\to 4}{\mathop{\lim}}\,\dfrac{\sqrt{3x+4}-4}{x-4}=\underset{x\to 4}{\mathop{\lim}}\,\dfrac{3\left( x-4 \right)}{\left( x-4 \right)\left( \sqrt{3x+4}+4 \right)}=\underset{x\to 4}{\mathop{\lim}}\,\dfrac{3}{\sqrt{3x+4}+4}=\dfrac{3}{8}$.\\
		$\Rightarrow 2a+b^2=6+64=70$.}
\end{ex}

\begin{ex}%[1K5BF-5]
	Tính $\lim\limits_{x \rightarrow + \infty} \left(\sqrt{x^2 + 3x + 2} - x\right)$.
	\choice
	{$- \dfrac{3}{2}$}
	{\True $\dfrac{3}{2}$}
	{$\dfrac{7}{2}$}
	{$- \dfrac{7}{2}$}
	\loigiai{
		\begin{eqnarray*}
			\lim\limits_{x \rightarrow + \infty} \left(\sqrt{x^2 + 3x + 2} - x\right) & = &\lim\limits_{x \rightarrow + \infty} \dfrac{3x + 2}{\sqrt{x^2 + 3x + 2} + x}\\
			& = &\lim\limits_{x \rightarrow + \infty} \dfrac{x\left(3 + \dfrac{2}{x}\right)}{|x|\sqrt{1 + \dfrac{3}{x} + \dfrac{2}{x^2}} + x}\\
			& = &\lim\limits_{x \rightarrow + \infty} \dfrac{x\left(3 + \dfrac{2}{x}\right)}{x\left(\sqrt{1 + \dfrac{3}{x} + \dfrac{2}{x^2}} + 1\right)}\\
			& = &\lim\limits_{x \rightarrow + \infty} \dfrac{3 + \dfrac{2}{x}}{\sqrt{1 + \dfrac{3}{x} + \dfrac{2}{x^2}} + 1} = \dfrac{3}{2}.
		\end{eqnarray*}
	}
\end{ex}
\begin{ex}%[1K5BF-5]
	Tìm giới hạn $M=\underset{x\to -\infty}{\lim}\left(\sqrt{x^2-4x}-\sqrt{x^2-x}\right)$.
	\choice
	{$M=-\dfrac{1}{2}$}
	{\True $M=\dfrac{3}{2}$}
	{$M=-\dfrac{3}{2}$}
	{$M=\dfrac{1}{2}$}
	\loigiai{
		Ta có
		\begin{align*}
			M&=\underset{x\to -\infty}{\lim}\left(\sqrt{x^2-4x}-\sqrt{x^2-x}\right)\\
			&=\underset{x\to -\infty}{\lim}\dfrac{-3x}{\sqrt{x^2-4x}+\sqrt{x^2-x}}\\
			&=\underset{x\to -\infty}{\lim}\dfrac{-3x}{|x|\left(\sqrt{1-\dfrac{4}{x}}+\sqrt{1-\dfrac{1}{x}}\right)}\\
			&=\underset{x\to -\infty}{\lim}\dfrac{3}{\sqrt{1-\dfrac{4}{x}}+\sqrt{1-\dfrac{1}{x}}}=\dfrac{3}{2}.
		\end{align*}
	}
\end{ex}





\begin{ex}%[1K5BF-5]
	Biết rằng $\lim\limits_{x\to -\infty} \left(\sqrt{2x^2-3x+1}+x\sqrt{2}\right)=\dfrac{a}{b}\sqrt{2}$,\quad ($a,b\in\mathbb{Z}, b>0, \dfrac{a}{b}$ tối giản). Tổng $a+b$ có giá trị là
	\choice
	{$5$}
	{$4$}
	{\True $7$}
	{$1$}
	\loigiai{
		Ta có
		\begin{eqnarray*}
			\lim\limits_{x\to -\infty} \left(\sqrt{2x^2-3x+1}+x\sqrt{2}\right) & = & \lim\limits_{x\to -\infty} \dfrac{2x^2-3x+1-2x^2}{\sqrt{2x^2-3x+1}-x\sqrt{2}}\\
			&=& \lim\limits_{x\to -\infty} \dfrac{-3x+1}{\sqrt{2x^2-3x+1}-x\sqrt{2}}\\
			&=&\lim\limits_{x\to -\infty} \dfrac{-3+\dfrac{1}{x}}{-\sqrt{2-\dfrac{3}{x}+\dfrac{1}{x^2}}-\sqrt{2}}\\
			&=& \dfrac{3}{4}\cdot \sqrt{2}.
		\end{eqnarray*}
		Vậy $a=3, b=4$. Tổng $a+b=7$.
	}
\end{ex}




\begin{ex}%[1K5BF-5]
	Tìm giới hạn $I=\underset{x\to +\infty}{\mathop{\lim}}\,\left( x+1-\sqrt{x^2-x+2} \right)$.
	\choice
	{$I=\dfrac{1}{2}$}
	{$I=\dfrac{46}{31}$}
	{$I=\dfrac{17}{11}$}
	{\True $I=\dfrac{3}{2}$}
	\loigiai{
		Ta có: $I=\underset{x\to +\infty}{\mathop{\lim}}\,\left( x+1-\sqrt{x^2-x+2} \right)=\underset{x\to +\infty}{\mathop{\lim}}\,\left( \dfrac{x^2-x^2+x-2}{x+\sqrt{x^2-x+2}}+1 \right)\\
		=\underset{x\to +\infty}{\mathop{\lim}}\,\left( \dfrac{x-2}{x+\sqrt{x^2-x+2}}+1 \right)=\underset{x\to +\infty}{\mathop{\lim}}\,\left( \dfrac{1-\dfrac{2}{x}}{1+\sqrt{1-\dfrac{1}{x}+\dfrac{2}{x^2}}}+1 \right)=\dfrac{3}{2}$.}
\end{ex}

\begin{ex}%[1K5KF-5]
	Cho $f(x)$ là một đa thức thỏa mãn $\lim \limits_{x\to 2} \dfrac{f(x)-15}{x-2}=3$. Tính \[\lim \limits_{x\to 2} \dfrac{f(x)-15}{(x^2-4)\left( \sqrt{2f(x)+6}+3\right)}.\]
	\choice
	{$\dfrac{1}{10}$}
	{$\dfrac{1}{6}$}
	{\True $\dfrac{1}{12}$}
	{$\dfrac{1}{8}$}
	\loigiai{
		Do $\lim \limits_{x\to 2} \dfrac{f(x)-15}{x-2}=3$ và $\lim \limits_{x\to 2} (x-2) = 0$ nên $\lim \limits_{x\to 2} \left( f(x) - 15 \right) = 0 \Rightarrow f(2) = 15$.\\
		Ta có
		\begin{eqnarray*} 
			\lim \limits_{x\to 2} \dfrac{f(x)-15}{(x^2-4)\left( \sqrt{2f(x)+6}+3\right)}
			& = & \lim \limits_{x\to 2} \left[ \dfrac{f(x)-15}{x-2} \cdot \dfrac{1}{(x+2)\left( \sqrt{2f(x)+6}+3\right)}\right] \\
			& = & 3 \cdot \dfrac{1}{4\cdot (\sqrt{2\cdot 15 +6}+3)}=\dfrac{1}{12}.
		\end{eqnarray*}
	}
\end{ex}
\begin{ex}%[1K5GF-5]
	Biết rằng $b>0$, $a+b=5$ và $\lim\limits_{x\to 0}\dfrac{\sqrt[3]{ax+1}-\sqrt{1-bx}}{x}=2$. Khẳng định nào dưới đây là \textbf{sai}?
	\choice
	{$a^2+b^2>10$}
	{\True  $a^2-b^2>6$}
	{ $a-b\geq 0$}
	{$1\le a\le 3$}
	\loigiai{ Ta có
		$$\lim\limits_{x\to 0}\dfrac{\sqrt[3]{ax+1}-\sqrt{1-bx}}{x}=\lim\limits_{x\to 0}\dfrac{\sqrt[3]{ax+1}-1}{x}-\lim\limits_{x\to 0}\dfrac{\sqrt{1-bx}-1}{x}.$$
		Với $L_1=\lim\limits_{x\to 0}\dfrac{\sqrt[3]{ax+1}-1}{x}=\lim\limits_{x\to 0}\dfrac{ax}{x\left(\sqrt[3]{(ax+1)^2}+\sqrt[3]{ax+1}+1\right)}=\dfrac{a}{3}$.\\
		Với $\lim\limits_{x\to 0}\dfrac{\sqrt{1-bx}-1}{x}=\lim\limits_{x\to 0}\dfrac{-bx}{x\left(\sqrt{1-bx}+1\right)}=-\dfrac{b}{2}$.\\
		Từ giả thiết bài toán ta có
		$$L=L_1-L_2=2\Leftrightarrow \dfrac{a}{3}+\dfrac{b}{2}=2\Leftrightarrow 2a+3b=12.$$
		Ta có hệ phương trình $\heva{&a+b=5\\&2a+3b=12}\Leftrightarrow \heva{&a=3\\&b=2.}$\\
		Kiểm tra trực tiếp từng đáp án ta thấy $a^2-b^2>6$ là sai.
	}
\end{ex}
\Closesolutionfile{ans}
% \begin{indapan}{10}
% 	{ans/ans-1K5-2-Dang4}
% \end{indapan}
\begin{dang}{Toán thực tế, liên môn về hàm số liên tục}
\end{dang}
\subsubsection{Ví dụ minh hoạ}
\begin{vd}%[1K5BF-6]
	Tính $\underset{x\to +\infty}{\mathop{\lim}}\,\left( \sqrt{x^2+3x\sqrt{x}}-x+1 \right)$.
	\choice
	{\True $+\infty$}
	{$4$}
	{$-\infty$}
	{$\dfrac{1}{2}$}
	\loigiai{
		Có: $\underset{x\to +\infty}{\mathop{\lim}}\,\left( \sqrt{x^2+3x\sqrt{x}}-x+1 \right)=\underset{x\to +\infty}{\mathop{\lim}}\,\left( \dfrac{x^2+3x\sqrt{x}-x^2}{\sqrt{x^2+3x\sqrt{x}}+x}+1 \right)$\\
		$=\underset{x\to +\infty}{\mathop{\lim}}\,\left( \dfrac{3x\sqrt{x}}{\sqrt{x^2+3x\sqrt{x}}+x}+1 \right)$
		$=\underset{x\to +\infty}{\mathop{\lim}}\,\left(\sqrt{x} \dfrac{3}{\sqrt{1+3\sqrt{x}}+1}+1 \right)
		=+\infty$.}
\end{vd}

\begin{vd}%[1K5BF-6]
	Giới hạn hàm số $\underset{x\to -\infty}{\mathop{\lim}}\,\left(\sqrt{x^2-x\sqrt{|x|}+3}+x\right)$ bằng
	\choice
	{$0$}
	{$\dfrac{1}{2}$}
	{\True $+\infty$}
	{$-\infty $}
	\loigiai{
		Ta có
		$\underset{x\to -\infty}{\mathop{\lim}}\,\left(\sqrt{x^2-x\sqrt{|x|}+3}+x\right)$
		$=\underset{x\to -\infty}{\mathop{\lim}}\,\dfrac{\left(\sqrt{x^2-x\sqrt{|x|}+3}+x\right)\left(\sqrt{x^2-x\sqrt{|x|}+3}-x\right)}{\sqrt{x^2-x\sqrt{|x|}+3}-x}$\\
		$=\underset{x\to -\infty}{\mathop{\lim}}\,\dfrac{-x\sqrt{|x|}+3}{\sqrt{x^2-x\sqrt{|x|}+3}-x}$
		$=\underset{x\to -\infty}{\mathop{\lim}}\,\sqrt{|x|}\dfrac{-1+\dfrac{3}{x\sqrt{|x|}}}{-\sqrt{1-\sqrt{|x|}+\dfrac{3}{x^2}}-1}=+\infty$.}
\end{vd}
\begin{vd}%[1K5BF-6]
	Tìm giới hạn $I=\displaystyle\lim\limits_{x\to-\infty}\left(\sqrt{x^4+4x^3+1}-x^2\right)$
	\choice
	{$I=-4$}
	{$I=1$}
	{\True $I=-2$}
	{$I=-1$}
	\loigiai{
		Ta có \begin{eqnarray*}
			I&=&\displaystyle\lim\limits_{x\to-\infty}\left(\sqrt{x^4+4x^3+1}-x^2\right)=\displaystyle\lim\limits_{x\to-\infty}\dfrac{\left(\sqrt{x^4+4x^3+1}+x^2\right)\left(\sqrt{x^4+4x^3+1}-x^2\right)}{\sqrt{x^4+4x^3+1}+x^2}\\
			&=& \displaystyle\lim\limits_{x\to-\infty}\dfrac{4x^3+1}{\sqrt{x^4+4x^3+1}+x^2}=\displaystyle\lim\limits_{x\to-\infty}\dfrac{4x^3+1}{\sqrt{x^4+4x^3+1}+x^2}\\
			&=&\displaystyle\lim\limits_{x\to-\infty}x\dfrac{4+\dfrac{1}{x^3}}{\sqrt{1+\dfrac{4}{x}+\dfrac{1}{x^4}}+1}=-\infty.
		\end{eqnarray*}
	}
\end{vd}
\begin{vd}%[1K5BF-6]
	Tính $L=\lim\limits_{x \to - \infty} \left( \sqrt{x^2 -7x\sqrt{|x|}+1}- \sqrt{x^2-3x\sqrt{|x|}+2}\right)$.
	\choice
	{\True $L= + \infty$}
	{$L= -\infty$}
	{\True $L= 2$}
	{$L= -2$}
	\loigiai{
		\begin{align*}
			L &= \lim\limits_{x \to - \infty} \dfrac{-4x\sqrt{|x|}-1}{\sqrt{x^2 - 7x\sqrt{|x|} +1}+\sqrt{x^2-3x\sqrt{|x|}+2}}\\
			&=\lim\limits_{x \to - \infty} \dfrac{-4x\sqrt{|x|}-1}{-x \sqrt{1-\dfrac{7}{\sqrt{|x|}}+\dfrac{1}{x^2}}-x \sqrt{1-\dfrac{3}{\sqrt{|x|}}+\dfrac{2}{x^2}}}\\
			&= \lim\limits_{x \to - \infty}\sqrt{|x|} \dfrac{-4-\dfrac{1}{x\sqrt{|x|}}}{- \sqrt{1-\dfrac{7}{\sqrt{|x|}}+\dfrac{1}{x^2}}-\sqrt{1-\dfrac{3}{\sqrt{|x|}}+\dfrac{2}{x^2}}}\\
			&=+\infty.
		\end{align*}
	}
\end{vd}
\begin{vd}%[1K5BF-6]
	Tìm tham số m để $\displaystyle\lim\limits_{x\to +\infty}(\sqrt{{x^3+mx^2}}-x\sqrt{x})=-\infty$.
	\choice
	{$m=0$}
	{$m>0$}
	{\True $m<0$}
	{$m=2$}
	\loigiai{
		Ta có \begin{align*}
			\displaystyle \lim \limits_{x\to +\infty}\left (\sqrt{x^3+mx^2}-x\sqrt{x}\right ) =&\displaystyle \lim \limits_{x\to +\infty}\dfrac{mx^2}{\sqrt{x^3+mx^2}+x\sqrt{x}}\\
			=& \displaystyle \lim \limits_{x\to +\infty}\sqrt{x}\dfrac{m}{\sqrt{1+\dfrac{m}{x}} +1}.  
		\end{align*}	
		Do đó $\displaystyle\lim\limits_{x\to +\infty}(\sqrt{{x^3+mx^2}}-x\sqrt{x})=-\infty\Leftrightarrow m<0$.
	}
\end{vd}
% \subsubsection{Bài tập rèn luyện}
\subsubsection{Câu hỏi trắc nghiệm}
\Opensolutionfile{ans}[ans/ans-1K5-2-Dang5]
\begin{ex}%[1K5BF-6]
	Tính $\underset{x\to +\infty}{\mathop{\lim}}\,\left( \sqrt{x^2+3x\sqrt{x}}-x \right)$.
	\choice
	{\True $+\infty$}
	{$4$}
	{$-\infty$}
	{$\dfrac{1}{2}$}
	\loigiai{
		Ta có $\underset{x\to +\infty}{\mathop{\lim}}\,\left( \sqrt{x^2+3x\sqrt{x}}-x \right)=\underset{x\to +\infty}{\mathop{\lim}}\, \dfrac{x^2+3x\sqrt{x}-x^2}{\sqrt{x^2+3x\sqrt{x}}+x}$\\
		$=\underset{x\to +\infty}{\mathop{\lim}}\, \dfrac{3x\sqrt{x}}{\sqrt{x^2+3x\sqrt{x}}+x} $
		$=\underset{x\to +\infty}{\mathop{\lim}}\,\sqrt{x} \dfrac{3}{\sqrt{1+\dfrac{3}{x}}+1} 
		=+\infty$.}
\end{ex}
\begin{ex}%[1K5KF-6]
	$\lim \limits_{x\to -\infty} \left(\sqrt{4x^2+3x+1}+mx\right)=+\infty$ nếu
	\choice
	{\True $m<2$}
	{$m>2$}
	{$m\ge 2$}
	{$m\le 2$}
	\loigiai{
		Ta có 
		\[ \lim \limits_{x\to -\infty} \left( \sqrt{4x^2+3x+1}+mx\right) 
		= \lim \limits_{x\to -\infty}\left[ x\cdot \left( -\sqrt{4+\dfrac{3}{x}+\dfrac{1}{x^2}}+m\right) \right].\]
		Do $\lim \limits_{x\to -\infty} x = -\infty$ và $\lim \limits_{x\to -\infty} \left( -\sqrt{4+\dfrac{3}{x}+\dfrac{1}{x^2}}+m\right) = -2+m$ nên để \\ $\lim \limits_{x\to -\infty} \left( \sqrt{4x^2+3x+1}+mx\right) = +\infty$ thì $-2+m<0 \Leftrightarrow m<2$.
	}
\end{ex}


\begin{ex}%[1K5KF-6]
	Biết $\displaystyle \lim\limits_{x\to +\infty}\dfrac{(2-a)x-3}{x-\sqrt{{x^2+1}}}=+\infty $ (với $a$ là tham số). Giá trị nhỏ nhất của $P=a^2-2a+4$ là
	\choice
	{$3 $}
	{\True $4 $}
	{$5 $}
	{$1 $}
	\loigiai{
		Ta có $\displaystyle \lim\limits_{x\to +\infty}\dfrac{(2-a)x-3}{x-\sqrt{{x^2+1}}}=\displaystyle \lim\limits_{x\to +\infty}\dfrac{((2-a)x-3)\left(x+\sqrt{x^2+1}\right)}{x^2-x^2-1}=\displaystyle \lim\limits_{x\to +\infty} ((a-2)x+3)\left(x+\sqrt{x^2+1}\right)$
		\begin{itemize}
			\item Với $a=2$, $$\displaystyle \lim\limits_{x\to +\infty} ((a-2)x+3)\left(x+\sqrt{x^2+1}\right)=\displaystyle \lim\limits_{x\to +\infty} 3(x+\sqrt{x^2+1})=+\infty.$$
			\item Với $a>2$ suy ra $a-2>0$, nên $$\displaystyle \lim\limits_{x\to +\infty} ((a-2)x+3)\left(x+\sqrt{x^2+1}\right)=\displaystyle \lim\limits_{x\to +\infty} x\left((a-2)+\dfrac{3}{x}\right)\left(1+\sqrt{1+\dfrac{1}{x^2}}\right)=+\infty.$$
			\item Với $a<2$ suy ra $a-2<0$, nên $$\displaystyle \lim\limits_{x\to +\infty} ((a-2)x+3)\left(x+\sqrt{x^2+1}\right)=\displaystyle \lim\limits_{x\to +\infty} x\left((a-2)+\dfrac{3}{x}\right)\left(1+\sqrt{1+\dfrac{1}{x^2}}\right)=-\infty.$$
		\end{itemize}
		Vậy $a\ge 2$ thì $\displaystyle \lim\limits_{x\to +\infty}\dfrac{(2-a)x-3}{x-\sqrt{{x^2+1}}}=+\infty $.
		Xét hàm số $f(a)=a^2-2a+4$ với $a\ge 2$. Ta có bảng biến thiên sau
		\begin{center}
			\begin{tikzpicture}
				\tkzTabInit[nocadre=false, lgt=1.5, espcl=3]{$a$ /1,$f(a)$ /2}{$2$,$+\infty$}
				%\tkzTabLine{,-,$0$,+,}
				\tkzTabVar{-/ $4$,+/$+\infty $}
			\end{tikzpicture}
		\end{center}
		Suy ra $\min P=4$.
	}
\end{ex}
\begin{ex}%[1K5KF-6]
	Tìm số các số nguyên $m$ thỏa mãn $\displaystyle\lim\limits_{x\rightarrow +\infty} \left(3 \sqrt{mx^2+2x+1}-mx \right)= +\infty$.
	\choice
	{$4$}
	{$10$}
	{$3$}
	{\True $9$}
	\loigiai{
		\begin{itemize}
			\item Với $m=0$ thì $\displaystyle\lim\limits_{x\rightarrow +\infty} \left(3 \sqrt{2x+1} \right)= +\infty$ (thỏa yêu cầu bài toán).
			\item Với $m \neq 0$ ta có 
			$\displaystyle\lim\limits_{x\rightarrow +\infty} \left(3 \sqrt{mx^2+2x+1}-mx \right)= \lim\limits_{x\rightarrow +\infty} x\left(3 \sqrt{m+\dfrac{2}{x}+\dfrac{1}{x^2} }-m \right)= +\infty
			$.\\
			Suy ra 
			$\displaystyle\lim\limits_{x\rightarrow +\infty} \left(3 \sqrt{m+\dfrac{2}{x}+\dfrac{1}{x^2} }-m \right)= 3\sqrt{m}-m>0$, $(m>0)$.\\
			Ta có $3\sqrt{m}-m>0 \Leftrightarrow 3\sqrt{m}>m \Leftrightarrow m^2-9m<0 \Leftrightarrow 0<m<9$.	Do đó $m \in \{1;2;\ldots ;8\}$.
		\end{itemize}
		Vậy $m \in \{0;1;2;\ldots ;8\}$. 
	}
\end{ex}

\begin{ex}%[1K5KF-6]
	Giới hạn $\lim\limits_{x\to+\infty}\left(\sqrt{x^2-3x+1}+x\right)$ bằng
	\choice
	{\True $+\infty$}
	{$-\infty$}
	{$0$}
	{$2$}
	\loigiai{Ta có
		\begin{eqnarray*}
			&&\lim\limits_{x\to+\infty}\left(\sqrt{x^2-3x+1}+x\right)=\lim\limits_{x\to+\infty}\left(x\sqrt{1-\dfrac{3}{x}+\dfrac{1}{x^2}}+x\right)\\
			&=&\lim\limits_{x\to+\infty}\left(x\left(\sqrt{1-\dfrac{3}{x}+\dfrac{1}{x^2}}+1\right)\right)=+\infty.
		\end{eqnarray*}
		Vì $\lim\limits_{x\to+\infty}x=+\infty$ và $\lim\limits_{x\to+\infty}\left(\sqrt{1-\dfrac{3}{x}+\dfrac{1}{x^2}}+1\right)=2$.}
\end{ex}
\begin{ex}%[1K5KF-6]
	Biết $\displaystyle\lim_{x\rightarrow +\infty}\left(\sqrt{x^2+ax\sqrt{|x|}-1}-x\right)=-\infty$. Khi đó giá trị của tham số $a$ là
	\choice
	{\True $a<0$}
	{$a>0$}
	{$a=6$}
	{$a=10$}
	\loigiai{
		$\displaystyle\lim_{x\rightarrow +\infty}\left(\sqrt{x^2+ax\sqrt{|x|}-1}-x\right)=\lim_{x\rightarrow +\infty}\dfrac{x^2+ax\sqrt{|x|}-1-x^2}{\sqrt{x^2+ax\sqrt{|x|}-1}+x}=\lim_{x\rightarrow +\infty}\sqrt{|x|}\dfrac{a-\dfrac{1}{x\sqrt{|x|}}}{\sqrt{1+\dfrac{a}{\sqrt{|x|}}-\dfrac{1}{x^2}}+1}$.\\
		Để  $\displaystyle\lim_{x\rightarrow +\infty}\left(\sqrt{x^2+ax\sqrt{|x|}-1}-x\right)=-\infty\Leftrightarrow a<0$. 
	}
\end{ex}
\begin{ex}%[1K5KF-6]
	Tính $\displaystyle\lim_{x\rightarrow -\infty}\left(\sqrt{7x^2+2x\sqrt{|x|}}+x\sqrt{7}\right)$.
	\choice
	{$0$}
	{$-\dfrac{5\sqrt{7}}{14}$}
	{\True$-\infty$}
	{$+\infty$}
	\loigiai{$\displaystyle\lim_{x\rightarrow -\infty}\left(\sqrt{7x^2+2x\sqrt{|x|}}+x\sqrt{7}\right)=\lim_{x\rightarrow -\infty}\dfrac{2x\sqrt{|x|}}{\sqrt{7x^2+2x\sqrt{|x|}}-x\sqrt{7}}=\lim_{x\rightarrow -\infty}\dfrac{2x\sqrt{|x|}}{-x\sqrt{7+\dfrac2x}-x\sqrt{7}}\\=\lim_{x\rightarrow -\infty}\sqrt{|x|}\dfrac{2}{-\sqrt{7+\dfrac2x}-\sqrt{7}}=-\infty$.}
\end{ex}
\begin{ex}%[1K5KF-6]
	Cho số thực $a$ thỏa mãn $\lim\limits_{x\to -\infty}\left(\sqrt{x^4+5ax^3-1}-x^2\right)=-\infty$. Tìm số thực $a$.
	\choice
	{$(a<0$}
	{$a\in(-10;-5)$}
	{\True $a>0$}
	{$a\in(-3;-1)$}
	\loigiai{
		
		\begin{align*}
			\lim\limits_{x\to -\infty}\left(\sqrt{x^4+5ax^3-1}-x^2\right)
			&=\lim\limits_{x\to -\infty}\dfrac{x^4+5ax^3-1-x^4}{\sqrt{x^4+5ax^3-1}+x^2}\\
			&=\lim\limits_{x\to -\infty}x\dfrac{5a-\dfrac{1}{x^3}}{\sqrt{1+\dfrac{5a}{x}-\dfrac{1}{x^4}}+1}.
		\end{align*}
		Để $\lim\limits_{x\to -\infty}\left(\sqrt{x^4+5ax^3-1}-x^2\right)=-\infty\Leftrightarrow a>0.$
	}
\end{ex}
\begin{ex}%[1K5KF-6]
	Tính $\displaystyle \lim_{x\to+\infty}\left(3x^2+1-\sqrt{9x^4-6x^3+1}\right)$.
	\choice
	{$\dfrac{1}{4}$}
	{$\dfrac{1}{2}$}
	{\True $+\infty$}
	{$-\infty$}
	\loigiai{Vì $x\to+\infty$ nên ta có\\
		$$\begin{aligned}
			&\displaystyle \lim_{x\to+\infty}\left(3x^2+1-\sqrt{9x^4-6x^3+1}\right)\\
			=&\displaystyle \lim_{x\to+\infty} \dfrac{\left(3x^2+1-\sqrt{9x^4-6x^3+1}\right)\left(3x^2+1+\sqrt{9x^4-6x^3+1}\right)}{3x^2+1+\sqrt{9x^4-6x^3+1}}\\
			=&\displaystyle \lim_{x\to+\infty} \dfrac{\left(3x^2+1\right)^2-\left(9x^4-6x^3+1\right)}{3x^2+1+\sqrt{9x^4-6x^2+1}}\\
			=&\displaystyle \lim_{x\to+\infty} \dfrac{x^3\left(6+\dfrac{6}{x}\right)}{x^2\left(3+\dfrac{1}{x^2}+\sqrt{9-6\cdot\dfrac{1}{x^2}+\dfrac{1}{x^4}}\right)}\\
			=&\displaystyle \lim_{x\to+\infty} x\dfrac{6+\dfrac{6}{x}}{3+\dfrac{1}{x}+\sqrt{9-6\cdot\dfrac{1}{x}+\dfrac{1}{x^2}}}\\
			=&+\infty.
		\end{aligned}$$
	} 
\end{ex}
\begin{ex}%[1K5KF-6]
	Tính $\displaystyle \lim_{x\to+\infty}\left(2x-\sqrt{x^2-x+1}\right)$.
	\choice
	{\True $+\infty $}
	{$-\infty $}
	{$\dfrac{1}{2}$}
	{$-\dfrac{1}{2}$}
	\loigiai{Vì ${x\to +\infty}$ nên $\displaystyle \lim_{x\to+\infty}\left(2x-\sqrt{x^2-x+1}\right)=\displaystyle \lim_{x\to+\infty}x\cdot \displaystyle \lim_{x\to+\infty}\left(2-\sqrt{1-\dfrac{1}{x}+\dfrac{1}{x^2}}\right)=+\infty$.
	} 
\end{ex}
\begin{ex}%[1K5KF-6]
	Tìm giới hạn $M=\underset{x\to -\infty}{\lim}\left(\sqrt{x^2-4x\sqrt{|x|}}-\sqrt{x^2-x\sqrt{|x|}}\right)$.
	\choice
	{$M=-\infty$}
	{\True $M=+\infty$}
	{$M=-\dfrac{3}{2}$}
	{$M=\dfrac{1}{2}$}
	\loigiai{
		Ta có
		\begin{align*}
			M&=\underset{x\to -\infty}{\lim}\left(\sqrt{x^2-4x\sqrt{|x|}}-\sqrt{x^2-x\sqrt{|x|}}\right)\\
			&=\underset{x\to -\infty}{\lim}\dfrac{-3x\sqrt{|x|}}{\sqrt{x^2-4x\sqrt{|x|}}+\sqrt{x^2-x\sqrt{|x|}}}\\
			&=\underset{x\to -\infty}{\lim}\sqrt{|x|}\dfrac{-3}{-\sqrt{1-\dfrac{4}{\sqrt{|x|}}}-\sqrt{1-\dfrac{1}{\sqrt{|x|}}}}\\
			&=+\infty.
		\end{align*}
	}
\end{ex}


\begin{ex}%[1K5KF-6]
	Giá trị của $\lim \limits_{x \to  - \infty } \left( \sqrt {x^2 + 5}  - x \right)$ là
	\choice
	{\True $+\infty $}
	{$-\infty $}
	{$1$}
	{$0$}
	\loigiai{
		Ta có $$\lim\limits_{x \to -\infty} \left(\sqrt{x^2+5}-x\right)=\lim\limits_{x \to -\infty} x\left(-\sqrt{1+\dfrac{5}{x^2}}-1 \right)=+\infty.$$
	}
\end{ex}
\begin{ex}%[1K5KF-6]
	Giá trị của $\lim \limits_{x \to  + \infty } \left( \sqrt {x^2 + 5x\sqrt{x}}  - x \right)$ là
	\choice
	{\True $+\infty $}
	{$-\infty $}
	{$1$}
	{$0$}
	\loigiai{
		Ta có $$\lim\limits_{x \to +\infty} \left(\sqrt{x^2+5x\sqrt{x}}-x\right)=\lim\limits_{x \to +\infty} \dfrac{5x\sqrt{x}}{\sqrt{x^2+5x\sqrt{x}}+x}=\lim\limits_{x \to +\infty}\sqrt{x}\dfrac{5}{\sqrt{1+\dfrac{5}{\sqrt{x}}}+1}=+\infty.$$
	}
\end{ex}
\Closesolutionfile{ans}
% \begin{indapan}{10}
% 	{ans/ans-1K5-2-Dang5}
% \end{indapan}

\begin{dang}{Giới hạn một bên}	
\end{dang}
\subsubsection{Ví dụ}
\begin{vd}%[DCHT Toán 11 - KNTT- Phạm Tuấn]%[1K5BF-7] 
	Tính giới hạn $\lim\limits _{x \rightarrow 2^{-}} \dfrac{x^2-3 x+2}{\sqrt{2-x}}$.
	\dapso{$\lim\limits _{x \rightarrow 2^{-}} \dfrac{x^2-3 x+2}{\sqrt{2-x}} =0$}
	\loigiai{
		Ta có 
		\[
		\lim\limits _{x \rightarrow 2^{-}} \dfrac{x^2-3 x+2}{\sqrt{2-x}} = \lim\limits _{x \rightarrow 2^{-}} \dfrac{(2-x)(1-x)}{\sqrt{2-x}} = \lim\limits _{x \rightarrow 2^{-}} (1-x)\sqrt{2-x}  = 0.
		\]
	}
\end{vd}

\begin{vd}%[DCHT Toán 11 - KNTT- Phạm Tuấn]%[1K5BF-7] 
	Tính giới hạn $\lim\limits _{x \rightarrow (-1)^{+}} \dfrac{x^3+1}{x^3+2x^2+x}$. 
	\dapso{$\lim\limits _{x \rightarrow (-1)^{+}} \dfrac{x^3+1}{x^3+2x^2+x}= -\infty$}
	\loigiai{
		Ta có 
		\[
		\lim\limits _{x \rightarrow (-1)^{+}} \dfrac{x^3+1}{x^3+2x^2+x} = \lim\limits _{x \rightarrow (-1)^{+}} \dfrac{(x+1)(x^2-x+1)}{x(x+1)^2} = \lim\limits _{x \rightarrow (-1)^{+}} \dfrac{x^2-x+1}{x(x+1)}.
		\]
		Khi $x \to (-1)^+$ thì $\heva{&x+1 \to 0\\&x+1 >0\\& \dfrac{x^2-x+1}{x} \to -3}$ suy ra $\lim\limits _{x \rightarrow (-1)^{+}} \dfrac{x^2-x+1}{x(x+1)} = -\infty.$ \\
		Vậy $\lim\limits _{x \rightarrow (-1)^{+}} \dfrac{x^3+1}{x^3+2x^2+x}= -\infty$. 
	}
\end{vd}

\begin{vd}%[DCHT Toán 11 - KNTT- Phạm Tuấn]%[1K5BF-7] 
	Cho hàm số $f(x) = \heva{&\sqrt{9-x^2} && \text{ khi } -3 \leq x < 3\\& 1 && \text{ khi } x=3\\& \sqrt{x^2-9} && \text{ khi } x>3.}$ \\
	Hàm số $f(x)$ có giới hạn khi $x \to 3$ hay không?
	\dapso{$\lim\limits _{x \rightarrow 3} f(x) =0$}
	\loigiai{
		Ta có $\lim\limits _{x \rightarrow 3^{-}} f(x) = \lim\limits _{x \rightarrow 3^{-}} \sqrt{9-x^2} =0$; $\lim\limits _{x \rightarrow 3^{+}} f(x) = \lim\limits _{x \rightarrow 3^{+}} \sqrt{x^2-9} =0$. \\
		Suy ra $\lim\limits _{x \rightarrow 3^{-}} f(x) = \lim\limits _{x \rightarrow 3^{+}} f(x) =0$. \\
		Vậy $\lim\limits _{x \rightarrow 3} f(x) =0$.
	}
\end{vd}

\begin{vd}%[DCHT Toán 11 - KNTT- Phạm Tuấn]%[1K5BF-7] 
	Ta gọi phần nguyên của số thực $x$ là số nguyên lớn nhất không lớn hơn $x$ và kí hiệu nó là $[x]$. 
	Ví dụ $[5]=5 $; $[3,12]=3 $; $[-2{,}725]=-3$. \\
	Tìm $\lim\limits _{x \rightarrow 1^{-}} [x]$ và  $\lim\limits _{x \rightarrow 1^{+}} [x]$. Giới hạn $\lim\limits _{x \rightarrow 1} [x]$ có tồn tại hay không?
	\dapso{$\lim\limits _{x \rightarrow 1^{-}} [x] =0$; $\lim\limits _{x \rightarrow 1^{+}} [x] =1$}
	\loigiai{
		Ta có $\lim\limits _{x \rightarrow 1^{-}} [x] =0$; $\lim\limits _{x \rightarrow 1^{+}} [x] =1$. \\
		Suy ra $\lim\limits _{x \rightarrow 1^{+}} [x] \neq  \lim\limits _{x \rightarrow 1^{-}} [x]$. \\
		Vậy giới hạn $\lim\limits _{x \rightarrow 1} [x]$ không tồn tại.
	}
\end{vd}

\begin{vd}%[DCHT Toán 11 - KNTT- Phạm Tuấn]%[1K5BF-7] 
	Cho hàm số $f(x) = \heva{& \dfrac{x-\sqrt{2x}}{4-x^2} && \text{ khi } x < 2\\& x^2-x+m && \text{ khi } x \geq  2}$  ($m$ là tham số). \\
	Tìm $m$ để hàm số $f(x)$ có giới hạn khi $x \to 2$.
	\dapso{$m= -\dfrac{17}{8}$}
	\loigiai{
		Ta có 
		\begin{align*}
			&\lim\limits _{x \rightarrow 2^{-}}  f(x) = \lim\limits _{x \rightarrow 2^{-}} \dfrac{x-\sqrt{2x}}{4-x^2} = \lim\limits _{x \rightarrow 2^{-}}  \dfrac{x(x-2)}{-(x-2)(x+2)(x+\sqrt{2x})} = -\dfrac{1}{8}; \\
			& \lim\limits _{x \rightarrow 2^{+}}  f(x)  = \lim\limits _{x \rightarrow 2^{+}}  (x^2-x+m) = 2+m.
		\end{align*}
		Hàm số $f(x)$ có giới hạn khi $x \to 2$ khi và chỉ khi 
		$$\lim\limits _{x \rightarrow 2^{-}}  f(x)  = \lim\limits _{x \rightarrow 2^{+}}  f(x) \Leftrightarrow -\dfrac{1}{8}=2+m \Leftrightarrow m= -\dfrac{17}{8}. $$
	}
\end{vd}
% \subsubsection{Bài tập rèn luyện}
% % \centerline{\fcolorbox{red}{yellow!50}{\bf {BÀI TẬP TỰ LUẬN}}}
% \begin{bt}%[DCHT Toán 11 - KNTT- Phạm Tuấn]%[1K5BF-7] 
% 	Tính giới hạn $\lim\limits _{x \rightarrow 1^{-}} \dfrac{-x^2-x+2}{x^2-3x^2+3x-1}$. 
% 	\dapso{$\lim\limits _{x \rightarrow 1^{-}} \dfrac{-x^2-x+2}{x^2-3x^2+3x-1} = -\infty$}
% 	\loigiai{
% 		Ta có $\lim\limits _{x \rightarrow 1^{-}} \dfrac{-x^2-x+2}{x^2-3x^2+3x-1} = \lim\limits _{x \rightarrow 1^{-}}  \dfrac{-(x-1)(x+2)}{(x-1)^3} = \lim\limits _{x \rightarrow 1^{-}}   \dfrac{-x-2}{(x-1)^2}$.  \\
% 		Khi $x \to 1^-$ thì $\heva{& (x-1)^2  \to 0\\& (x-1)^2 >0\\& -x-2\to -3}$ suy ra $\lim \limits _{x \rightarrow 1^{-}}   \dfrac{-x-2}{(x-1)^2}= -\infty.$\\
% 		Vậy $\lim\limits _{x \rightarrow 1^{-}} \dfrac{-x^2-x+2}{x^2-3x^2+3x-1} = -\infty$. 
% 	}
% \end{bt}

% \begin{bt}%[DCHT Toán 11 - KNTT- Phạm Tuấn]%[1K5BF-7] 
% 	Cho hàm số $f(x)=\heva{& \dfrac{x^2-1}{1-x} \,&\text{ khi }x < 1\\& x^3-2x^2+3\,&\text{ khi }x \geq  1}$. Tính $\lim\limits_{x\to 1^-}f(x)$ và $\lim\limits_{x\to 1^+}f(x)$.
% 	\dapso{$\lim\limits_{x\to 1^-}f(x)=-2$; $\lim\limits_{x\to 1^+}f(x)=2$}
% 	\loigiai{
% 		Ta có $\lim\limits_{x\to 1^-}f(x) = \lim\limits_{x\to 1^-}  \dfrac{x^2-1}{1-x} =   \lim\limits_{x\to 1^-}  -(x+1) = -2$; 
% 		$\lim\limits_{x\to 1^+} f(x) = \lim\limits_{x\to 1^+} (x^3-2x^2+3) = 2 $.
% 	}
% \end{bt}

% \begin{bt}%[DCHT Toán 11 - KNTT- Phạm Tuấn]%[1K5BF-7] 
% 	Tính giới hạn $\lim\limits _{x \rightarrow 2^{-}} \dfrac{|x^2-3x+2|}{x^2-4}$. 
% 	\dapso{$\lim\limits _{x \rightarrow 2^{-}} \dfrac{|x^2-3x+2|}{x^2-4} =  -\dfrac{1}{4}$}
% 	\loigiai{
% 		Khi $x \to 2^-$ thì $x^2-3x+2 <0$ nên 
% 		\[
% 		\lim\limits _{x \rightarrow 2^{-}} \dfrac{|x^2-3x+2|}{x^2-4} = \lim\limits _{x \rightarrow 2^{-}} \dfrac{-x^2+3x-2}{x^2-4} = \lim\limits _{x \rightarrow 2^{-}} \dfrac{1-x}{x+2} = -\dfrac{1}{4}.
% 		\]
% 	}
% \end{bt}

% \begin{bt}%[DCHT Toán 11 - KNTT- Phạm Tuấn]%[1K5BF-7] 
% 	Cho hàm số $f(x) = \heva{&\dfrac{1-\sqrt{x}}{x^2-2x+1} \text{ khi  } x >1\\& \dfrac{2x}{x^3-2x+1}  \text{ khi  } x <1}$. Tính $\lim\limits _{x \rightarrow 1} f(x)$.
% 	\dapso{$\lim\limits _{x \rightarrow 1} f(x)=-\infty$}
% 	\loigiai{
% 		Xét $\lim\limits _{x \rightarrow 1^{+}}  f(x) = \lim\limits _{x \rightarrow 1^{+}} \dfrac{1-\sqrt{x}}{x^2-2x+1} =\lim\limits _{x \rightarrow 1^{+}} \dfrac{1-x}{(x-1)^2(\sqrt{x}+1)} = \lim\limits _{x \rightarrow 1^{+}} \dfrac{1}{(1-x)(\sqrt{x}+1)}$. \\
% 		Khi $x \to 1^+$ thì $\heva{&1-x <0\\&1-x \to 0\\&\sqrt{x}+1 \to 2}$, suy ra $\lim\limits _{x \rightarrow 1^{+}}  f(x) = -\infty$. \\
% 		Xét $\lim\limits _{x \rightarrow 1^{-}}  f(x) = \lim\limits _{x \rightarrow 1^{-}}  \dfrac{2x}{x^3-2x+1} = \lim\limits _{x \rightarrow 1^{-}} \dfrac{2x}{(x-1)(x^2+x-1)}$. \\
% 		Khi $x \to 1^-$ thì $\heva{&x-1 <0\\&x-1 \to 0\\&x^2+x-1 \to 1}$, suy ra $\lim\limits _{x \rightarrow 1^{-}}  f(x) = -\infty$. \\
% 		Suy ra $\lim\limits _{x \rightarrow 1^{+}}  f(x) =\lim\limits _{x \rightarrow 1^{-}}  f(x) = -\infty$. Vậy $\lim\limits _{x \rightarrow 1} f(x)=-\infty$.
% 	}
% \end{bt}

% \begin{bt}%[DCHT Toán 11 - KNTT- Phạm Tuấn]%[1K5BF-7] 
% 	Cho hàm số $f(x) = |x^2-2x-3|$. Tính các giới hạn $\lim\limits _{x \rightarrow 0^{-}} \dfrac{f(x+3)- f(3)}{x}$ và $\lim\limits _{x \rightarrow 0^{+}} \dfrac{f(x+3)- f(3)}{x}$. 
% 	\dapso{$\lim\limits _{x \rightarrow 0^{-}} \dfrac{f(x+3)- f(3)}{x}=-4$;  $\lim\limits _{x \rightarrow 0^{+}} \dfrac{f(x+3)- f(3)}{x}=4$}
% 	\loigiai{
% 		Ta có $\lim\limits _{x \rightarrow 0^{-}} \dfrac{f(x+3)- f(3)}{x} = \lim\limits _{x \rightarrow 0^{-}} \dfrac{|(x+3)^2-2(x+3)-3|-0}{x} = \lim\limits _{x \rightarrow 0^{-}} \dfrac{|x(x+4)|}{x}$. \\
% 		Khi $x \to 0^-$ thì $x<0$, suy ra  $\lim\limits _{x \rightarrow 0^{-}} \dfrac{|x(x+4)|}{x} = \lim\limits _{x \rightarrow 0^{-}} -(x+4) = -4$. \\
% 		Ta có $\lim\limits _{x \rightarrow 0^{+}} \dfrac{f(x+3)- f(3)}{x} = \lim\limits _{x \rightarrow 0^{+}} \dfrac{|(x+3)^2-2(x+3)-3|-0}{x} = \lim\limits _{x \rightarrow 0^{+}} \dfrac{|x(x+4)|}{x}$. \\
% 		Khi $x \to 0^+$ thì $x>0$, suy ra  $\lim\limits _{x \rightarrow 0^{+}} \dfrac{|x(x+4)|}{x} = \lim\limits _{x \rightarrow 0^{+}} (x+4) = 4$.
% 	}
% \end{bt}

% \begin{bt}%[DCHT Toán 11 - KNTT- Phạm Tuấn]%[1K5KF-7] 
% 	Tìm $m$ để hàm số $f(x) = \heva{&\sin \dfrac{1}{2x} && \text{ khi } x <0\\& x^2+m && \text{ khi } x \geq 0}$ có giới hạn khi $x\to 0$.
% 	\dapso{Không tồn tại $m$}
% 	\loigiai{
% 		Ta có $\lim\limits _{x \rightarrow 0^{+}} f(x) = \lim\limits _{x \rightarrow 0^{+}} (x^2+m)=m$. \\
% 		Xét $\lim\limits _{x \rightarrow 0^{-}} f(x) = \lim\limits _{x \rightarrow 0^{+}} \sin \dfrac{1}{2x}$. \\
% 		Chọn dãy số $x_n = -\dfrac{2}{n\pi }$. Dễ thấy $x_n<0$ và $\lim \limits_{n \to +\infty}x_n =0$. \\
% 		Ta có $\lim \limits_{n \to +\infty}\sin \dfrac{1}{2x} = \lim \limits_{n \to +\infty}\sin (-n\pi ) =0$. \\
% 		Chọn dãy số $x_n = -\dfrac{2}{\frac{\pi}{2}+ n2\pi} $. Dễ thấy $x_n<0$ và $\lim \limits_{n \to +\infty}x_n =0$. \\
% 		Ta có $\lim \limits_{n \to +\infty}\sin \dfrac{1}{2x} = \lim \limits_{n \to +\infty}\sin (-\frac{\pi}{2}- n2\pi ) =-1$.  \\
% 		Suy ra $\lim\limits _{x \rightarrow 0^{-}} f(x)$ không tồn tại. \\
% 		Vậy không tồn tại $m$ để $f(x)$ có giới hạn khi $x\to 0$.
% 	}
% \end{bt}

% \begin{bt}%[DCHT Toán 11 - KNTT- Phạm Tuấn]%[1K5BF-7] 
% 	Cho hàm số $f(x)=\heva{& \dfrac{1}{x-1} - \dfrac{3}{x^3-1} \,&\text{ nếu }x > 1\\& mx+2\,&\text{ nếu }x \geq  1}$.  \\
% 	Với giá trị nào của tham số $m$ thì hàm số $f(x)$ có giới hạn khi $x \rightarrow 1$? Tìm giới hạn này.
% 	\dapso{$m=-1$; $\lim\limits _{x \rightarrow 1} f(x)=1$}
% 	\loigiai{
% 		Ta có
% 		\begin{align*}
% 			\lim\limits  _{x \rightarrow 1^{+}} f(x) &=\lim\limits  _{x \rightarrow 1^{+}}\left(\frac{1}{x-1}-\frac{3}{x^3-1}\right)=\lim\limits  _{x \rightarrow 1^{+}} \frac{x^2+x-2}{(x-1)\left(x^2+x+1\right)} \\
% 			&=\lim\limits  _{x \rightarrow 1^{+}} \frac{(x-1)(x+2)}{(x-1)\left(x^2+x+1\right)}=\lim\limits  _{x \rightarrow 1^{+}} \frac{x+2}{x^2+x+1}=1 .
% 		\end{align*}
% 		$\lim\limits _{x \rightarrow 1^{-}} f(x)=\lim\limits _{x \rightarrow 1^{-}}(m x+2)=m+2$. \\
% 		$f(x)$ có giới hạn khi $x \rightarrow 1 \Leftrightarrow m+2=1 \Leftrightarrow m=-1$. Khi đó $\lim\limits _{x \rightarrow 1} f(x)=1$.
% 	}
% \end{bt}

% \begin{bt}%[DCHT Toán 11 - KNTT- Phạm Tuấn]%[1K5BF-7] 
% 	Cho hàm số $f(x) = \heva{&x\cos \dfrac{1}{x} && \text{ khi } x <0\\& \sin x^2 + m  && \text{ khi } x \geq 0.}$ \\
% 	Tìm $m$ để hàm số $f(x)$ có giới hạn khi $x \to 0$.
% 	\dapso{$m=0$}
% 	\loigiai{
% 		Xét $\lim\limits _{x \rightarrow 0^{-}} f(x) = \lim\limits _{x \rightarrow 0^{-}}  x\cos \dfrac{1}{x}$. \\
% 		Ta có $0\leq  |x\cos \dfrac{1}{x}| \leq |x|$ và $\lim\limits _{x \rightarrow 0^{-}} |x| =0$. Suy ra $\lim\limits _{x \rightarrow 0^{-}}  x\cos \dfrac{1}{x} =0$. \\
% 		Ta lại có $\lim\limits _{x \rightarrow 0^{+}} f(x) = \lim\limits _{x \rightarrow 0^{-}}  (\sin x^2 + m) = m$. \\
% 		$f(x)$ có giới hạn khi $x \rightarrow 0$ khi và chỉ khi 
% 		\[
% 		\lim\limits _{x \rightarrow 0^{-}}  f(x) = \lim\limits _{x \rightarrow 0^{+}}  f(x)  \Leftrightarrow m=0.
% 		\]
% 	}
% \end{bt}
\subsubsection{Câu hỏi trắc nghiệm}
\Opensolutionfile{ans}[ans/ans-1K5-2-Dang6]
\begin{ex}%[DCHT Toán 11 - KNTT- Phạm Tuấn]%[1K5BF-7]
	Tính giới hạn $\lim\limits_{x \to(-2)^{-}} \dfrac{3+2 x}{x+2}$.
	\choice
	{$-\infty$}
	{$2$}
	{\True $+\infty$}
	{$\dfrac{3}{2}$}
	\loigiai{
		Khi $x \to (-2)^{-}$ thì $\heva{& 3+3x\to -1\\&x+2\to 0 \\&x+2<0.}$ \\
		Suy ra  $\lim\limits_{x \to(-2)^{-}} \dfrac{3+2 x}{x+2}=+\infty$.
	}
\end{ex}

\begin{ex}%[DCHT Toán 11 - KNTT- Phạm Tuấn]%[1K5BF-7]
	Cho hàm số $f(x)=\heva{&2x^2-2\,&\text{ khi }x\ge 6\\&x-2\,&\text{ khi }x<6}$. Tính $\lim\limits_{x\to 6^-}f(x)$ bằng
	\choice
	{$2$}
	{$5$}
	{$1$}
	{\True $4$}
	\loigiai{
		Ta có $\lim\limits_{x\to 6^-}f(x)=\lim\limits_{x\to 6^-}(x-2)=6-2=4$.
	}
\end{ex}


\begin{ex}%[DCHT Toán 11 - KNTT- Phạm Tuấn]%[1K5BF-7]
	$\displaystyle \lim \limits_{x \rightarrow 5^+} \dfrac{|10-2x|}{x^2-6x+5}$ là
	\choice
	{\True $\dfrac{1}{2}$}
	{$0$}
	{$+\infty$}
	{$- \dfrac{1}{2}$}
	\loigiai{
		Ta có $\displaystyle \lim \limits_{x \rightarrow 5^+} \dfrac{|10-2x|}{x^2-6x+5} 
		= \lim \limits_{x \rightarrow 5^+} \dfrac{2x-10}{(x-1)(x-5)} 
		= \lim \limits_{x \rightarrow 5^+} \dfrac{2}{x-1} = \dfrac{1}{2}$.
	}
\end{ex}

\begin{ex}%[DCHT Toán 11 - KNTT- Phạm Tuấn]%[1K5BF-7]
	Tính $\lim\limits _{x \to 2^{+}}\dfrac{|2-x|}{x^{2}-x-2}$. 
	\choice
	{$+\infty$}
	{$0$}
	{$-\dfrac{1}{3}$}
	{\True $\dfrac{1}{3}$}
	\loigiai{
		Vì $x \to 2^+$ nên $x>2$. Do đó $\lim\limits _{x \to 2^{+}}\dfrac{|2-x|}{x^{2}-x-2} = \lim\limits _{x \to 2^{+}}\dfrac{x-2}{(x-2)(x+1)} = \lim\limits _{x \to 2^{+}}\dfrac{1}{x+1}=\dfrac{1}{3}$.
	}
\end{ex}

\begin{ex}%[DCHT Toán 11 - KNTT- Phạm Tuấn]%[1K5BF-7]
	Trong các giới hạn sau, giới hạn nào không tồn tại?
	\choice
	{$\lim\limits_{x \to \infty} \dfrac{2x + 1}{x^2 + 1}$}
	{$\lim\limits_{x \to 0} \dfrac{x}{\sqrt{x} + 1}$}
	{$\lim\limits_{x \to 1}\dfrac{x}{(x + 1)^2}$}
	{\True $\lim\limits_{x \to 0} \dfrac{1}{x}$}	
	\loigiai{
		$\lim\limits_{x \to 0^+} = \dfrac{1}{x} = +\infty; \lim\limits_{x \to 0^-} = \dfrac{1}{x} = -\infty $ nên giới hạn không tồn tại.
	}
\end{ex}

\begin{ex}%[DCHT Toán 11 - KNTT- Phạm Tuấn]%[1K5BF-7]
	Gọi $a$ là số thực để hàm số $f(x)=\heva{ & x^2+ax+2 & & \text{khi} \ x>2 \\ & 2x^2-x+1 & & \text{khi} \ x\leqslant2}$ có giới hạn khi $x\to2$. Hãy chọn hệ thức đúng.
	\choice
	{$2a^2+3a+1=0$}
	{$a^2-3a+2=0$}
	{\True $4a^2-1=0$}
	{$a^2-4=0$}
	\loigiai
	{
		Ta có 
		\[\heva{&\lim\limits_{x\to2^+}f(x) = \lim\limits_{x\to2^+}\left(x^2+ax+2\right) = 2a+6 \\&\lim\limits_{x\to2^-}f(x) = \lim\limits_{x\to2^-} \left(2x^2 - x  + 1\right)=7.}\]
		Để hàm số có giới hạn khi $x\to 2$ thì \[\lim\limits_{x\to2^+}f(x) = \lim\limits_{x\to2^-}f(x) \Leftrightarrow 2a + 6 = 7 \Leftrightarrow a = \dfrac{1}{2}.\]
		Khi đó $4a^2 - 1=0$ là hệ thức đúng.
	}
\end{ex}

\begin{ex}%[DCHT Toán 11 - KNTT- Phạm Tuấn]%[1K5BF-7]
	Cho hàm số $f(x)=\heva{&\dfrac{x^3-3 x^2+2}{x-1}\,\, \text { nếu } \,\, x>1 \\& a x+3 \,\, \text { nếu } \,\,x \leq 1}$. Tìm $a$ để $\lim\limits_{x \to 1} f(x)$ tồn tại.
	\choice
	{$a=6$}
	{$a=1$}
	{$a=0$}
	{\True $a=-6$}
	\loigiai{
		$\lim\limits_{x \to 1^+} f(x)=\lim\limits_{x \to 1^+} \dfrac{x^3-3 x^2+2}{x-1}=\lim\limits_{x \to 1+} \dfrac{(x-1)(x^2-2x-2)}{x-1}=\lim\limits_{x \to 1^+}(x^2-2x-2)=-3$.\\
		$\lim\limits_{x \to 1^-} f(x)=\lim\limits_{x \to 1^-} (ax+3)=3+a$.\\
		Giới hạn $\lim\limits_{x \to 1} f(x)$ tồn tại  khi $3+a=-3\Rightarrow a=-6$. 	
	} 
\end{ex}

\begin{ex}%[DCHT Toán 11 - KNTT- Phạm Tuấn]%[1K5BF-7]
	Cho hàm số $f(x)=\heva{&\dfrac{\left|x-1\right|}{x-1}&\text{ khi }& x<1\\&x+2+a &\text{ khi } &1\leq x\leq 3\\ & \dfrac{x^2-81}{\sqrt{x}-3} &\text{ khi }& x>3}$. Tìm tất cả giá trị của tham số $a$ để hàm số có giới hạn tại $x=3$.
	\choice
	{$a=12\left(3+\sqrt{3}\right)$}
	{$a=12\left(3-\sqrt{3}\right)$}
	{\True $a=12\left(3+\sqrt{3}\right)-5$}
	{$a=12\left(3-\sqrt{3}\right)+5$}
	\loigiai{Với $1\leq x\leq 3$ thì $f(x)=x+2+a$ nên $\lim\limits_{x \to 3^-}f(x)=\lim\limits_{x \to 3^-} \left(x+2-a\right)=5+a$.\\
		Với $x>3$ thì $f(x)=\dfrac{x^2-81}{\sqrt{x}-3}=\left(\sqrt{x}+3\right)\left(x+9\right)$ nên $\lim\limits_{x \to 3^+} f(x)=\lim\limits_{x \to 3^+}\left(\sqrt{x}+3\right)\left(x+9\right) =12\left(3+\sqrt{3}\right)$.\\
		Do đó, để hàm số có giới hạn tại $x=3$ thì $\lim\limits_{x \to 3^-}f(x)=\lim\limits_{x \to 3^+}f(x) \Leftrightarrow a=12\left(3+\sqrt{3}\right)-5$.
	}
\end{ex}
\Closesolutionfile{ans}
% \begin{indapan}{10}
% 	{ans/ans-1K5-2-Dang6}
% \end{indapan}

\begin{dang}{Toán thực tế, liên môn về giới hạn hàm số}
\end{dang}
\subsubsection{Ví dụ}
\begin{vd}%[DCHT Toán 11 - KNTT- Phạm Tuấn]%[1K5YF-8]
	Chiều dài một loài động vật nhỏ được tính theo công công thức $h(t)=\dfrac{300}{1+9 \cdot  (0{,}8)^t}$ mm, trong đó $t$ số ngày sau khi sinh của loài động vật đó. Tính chiều dài cuối cùng của nó (chiều dài khi $t \to +\infty$).
	\dapso{$300$ mm}
	\loigiai{
		Ta có $\displaystyle \lim \limits_{t \to +\infty } \dfrac{300}{1+9 \cdot  (0{,}8)^t} = 300$. \\
		Vậy chiều dài cuối cùng của loài động vật  là $300$ mm. 
	}
\end{vd}


\begin{vd}%[DCHT Toán 11 - KNTT- Phạm Tuấn]%[1K5BF-8] 
	Theo thuyết tương đối, khối lượng $m$ của một hạt phụ thuộc vào vận tốc $v$ của nó, theo công thức
	$$
	m=\frac{m_0}{\sqrt{1-\dfrac{v^2}{c^2}}}
	$$
	trong đó $m_0$ là khối lượng khi hạt đứng yên và $c$ là tốc độ ánh sáng. Tìm giới hạn của khối lượng khi $v$ tiến đến $c^{-}$.
	\dapso{$\lim\limits _{v \to c^-} \frac{m_0}{\sqrt{1-\dfrac{v^2}{c^2}}} = +\infty$}
	\loigiai{
		Với $m_0 =0$ thì $\displaystyle \lim_{v \to c^-} =0$. \\
		Với $m_0 \neq 0$. \\
		Khi $c \to c^-$ thì $\heva{&\sqrt{1-\dfrac{v^2}{c^2}} \to 0\\&\sqrt{1-\dfrac{v^2}{c^2}} >0}$ suy ra $\displaystyle \lim_{v \to c^-} \frac{m_0}{\sqrt{1-\dfrac{v^2}{c^2}}} = +\infty$. \\
		Vậy nếu một hạt có khối lượng nghỉ khác $0$ thì khối lượng của hạt sẽ lớn vô cùng khi vận tốc tiến gần vận tốc ánh sáng.
	}
\end{vd}

\begin{vd}%[DCHT Toán 11 - KNTT- Phạm Tuấn]%[1K5BF-8] 
	Một chất điểm chuyển động thẳng với phương trình $s(t)$. Khi đó vận tốc tức thời tại thời điểm $t_0$ được định nghĩa là $\displaystyle \lim \limits_{\Delta t} \dfrac{s(t_0+ \Delta t) - s(t_0)}{\Delta t}$. Tính vận tốc tức thời của chất điểm với phương trình chuyển động $s(t) = 3t^2-2t+3$ ($s(t)$ có đơn vị là m, $t$ đơn vị là giây), tại thời điểm $t=4$ giây. 
	\dapso{$v=22  \mathrm{~m/s}$}
	\loigiai{
		Vận tốc tức thời của chất điểm tại thời điểm $t=4$ giây là
		\begin{align*}
			\lim \limits_{\Delta t \to 0} \dfrac{s(4+\Delta t) - s(4)}{\Delta t} &=  \lim \limits_{\Delta t \to 0} \dfrac{3(4+\Delta t)^2-2(4+\Delta t)+3 - 43}{\Delta t} \\
			& = \lim \limits_{\Delta t \to 0} \dfrac{3 (\Delta t)^2+ 22 \Delta t}{\Delta t} = 22  \mathrm{~m/s}.
		\end{align*}
	}
\end{vd}

\begin{vd}%[DCHT Toán 11 - KNTT- Phạm Tuấn]%[1K5BF-8] 
	Số lượng đơn vị hàng tồn kho trong một công ty  được cho bởi
	$$
	N(t)=200\left(3 \left [\frac{t+3}{3}  \right ]-t\right)
	$$
	trong đó $t$ là thời gian tính bằng ngày, $[x]$ là số nguyên lớn nhất không vượt quá $x$ (ví dụ $[-1{,}5]=-2$, $[8{,}8] = 8$).
	\begin{enumerate}
		\item Tính $\displaystyle \lim_{t \to 55^+} N(t)$.
		\item  Tính $\displaystyle \lim_{t \to 201^-} N(t)$.
	\end{enumerate}
	\dapso{$\displaystyle \lim_{t \to 55^+} N(t) = 400$; $\displaystyle \lim_{t \to 201^-} N(t) =0$}
	\loigiai{
		\begin{enumerate}
			\item 
			Khi $t \to 55^+$, ta có $\left [\dfrac{t+3}{3}  \right ] = 19$. \\
			Suy ra  $\displaystyle \lim_{t \to 55^+} N(t) =\lim_{t \to 55^+}  200\left(3 \left [\frac{t+3}{3}  \right ]-t\right) =200(3 \cdot 19 - 55) = 400$.
			\item  
			Khi $t \to 201^-$, ta có $\left [\dfrac{t+3}{3}  \right ] = 201$. \\
			Suy ra  $\displaystyle \lim_{t \to 201^-} N(t) = \lim_{t \to 201^-}  200\left(3 \left [\frac{t+3}{3}  \right ]-t\right)= 200 \cdot 0 = 0$.
		\end{enumerate}
	}
\end{vd}


\begin{vd}%[DCHT Toán 11 - KNTT- Phạm Tuấn]%[1K5BF-8] 
	Một chất điểm chuyển động thẳng với vận tốc $v(t)$. Khi đó gia tốc tức thời tại thời điểm $t_0$ được định nghĩa là $\displaystyle \lim \limits_{\Delta t \to 0} \dfrac{v(t_0+ \Delta t) - v(t_0)}{\Delta t}$. Một chất điểm chuyển động với vận tốc $v(t) = 0{,}1t^2-0{,}4t+1$ (m/s), tính gia tốc tức thời tại thời điểm $t=8$ giây. 
	\dapso{$1{,}2$ $\mathrm{m/s^2}$}
	\loigiai{
		Gia tốc tức thời của chất điểm tại thời điểm $t=8$ giây là
		\begin{align*}
			\lim \limits_{\Delta t \to 0} \dfrac{v(8+\Delta t) - v(8)}{\Delta t} &=  \lim \limits_{\Delta t \to 0} \dfrac{0{,}1(8+\Delta t)^2-0{,}4(8+\Delta t)+1- 4{,}2}{\Delta t} \\
			& = \lim \limits_{\Delta t \to 0} \dfrac{0{,}1 (\Delta t)^2 + 1{,}2 \Delta t}{\Delta t}   \\
			& = \lim \limits_{\Delta t \to 0} (0{,}1\Delta t + 1{,}2)  \\
			& = 1{,}2.
		\end{align*}
		Vậy gia tốc tức thời của chất điểm tại thời điểm $t=8$ giây là $1{,}2$ ($\mathrm{m/s^2}$).
	}
\end{vd}

\begin{vd}%[DCHT Toán 11 - KNTT- Phạm Tuấn]%[1K5BF-8] 
	Một người lái xe từ thành phố $A$ đến thành phố $B$ với vận tốc trung bình  là $x$ km/h. Trên chuyến trở về, vận tốc trung bình là $y$ km/h. Vận tốc trung bình của cả đi và về là $60$ km/h. (Giả sử người lái xe đi trên cùng  một con đường trên cả chuyến đi và về).
	\begin{enumerate}
		\item Chứng minh rằng $y= \dfrac{30x}{x-30}$.
		\item Tìm giới hạn của $y$ khi $x \rightarrow 30^{+}$.
	\end{enumerate}
	\dapso{$\displaystyle \lim\limits_{x\to 30^+}  y  = +\infty$}
	\loigiai{
		\begin{enumerate}
			\item  
			Gọi  khoảng cách giữa $A$ và $B$ là $s$ km. \\
			Thời gian  chuyến đi là $\dfrac{s}{x}$, thời gian  chuyến trở về là $\dfrac{s}{y}$. \\
			Suy ra 
			$$\dfrac{2s}{60} = \dfrac{s}{x} + \dfrac{s}{y} \Leftrightarrow \dfrac{1}{y} = \dfrac{1}{30} - \dfrac{1}{x} \Leftrightarrow y= \dfrac{30x}{x-30}.$$
			\item  Ta có $\displaystyle \lim\limits_{x\to 30^+}  y = \lim\limits_{x\to 30^+}  \dfrac{30x}{x-30} = +\infty$. 
		\end{enumerate}
	}
\end{vd}


\begin{vd}%[DCHT Toán 11 - KNTT- Phạm Tuấn]%[1K5BF-8] 
	Một hình elip với bán trục lớn $a$ và bán trục nhỏ $b$ thì diện tích được tính theo công thức $S=\pi ab$. Tính giới hạn diện tích của elip khi tiêu cự gần tới $0$.
	\dapso{$\pi a^2$}
	\loigiai{
		Ta có $S= \pi ab = \pi a \sqrt{a^2-c^2}$. \\
		Vậy $\lim\limits_{c \to 0} S= \lim\limits_{c \to 0} \pi a \sqrt{a^2-c^2} = \pi a^2$. \\
		Ta thấy khi $c\to 0$, thì giới hạn diện tích của elip là diện tích hình tròn bán kính $R=a$.
	}
\end{vd}

\begin{vd}%[DCHT Toán 11 - KNTT- Phạm Tuấn]%[1K5BF-8]  
	Các nhà vật lý  thấy rằng thuyết tương đối hẹp của Einstein quy về cơ học Newton khi $c \rightarrow +\infty$, trong đó $c$ là tốc độ ánh sáng. Điều này được minh họa bởi ví dụ: Một hòn đá được ném thẳng đứng từ mặt đất để nó quay trở lại trái đất một giây sau đó. Sử dụng các định luật Newton, chúng ta thấy rằng chiều cao tối đa của hòn đá là $h=\dfrac{g}{8}$ mét ($g = 9{,}8 \mathrm{m/ s ^2}$). Theo thuyết tương đối hẹp, khối lượng của hòn đá phụ thuộc vào vận tốc của nó chia cho $c$, và có chiều cao cực đại là 
	\[
	h(c)=c \sqrt{\dfrac{c^2}{g^2}+\dfrac{1}{4}}- \dfrac{c^2}{g}.
	\]
	Tính $\lim\limits _{c \rightarrow +\infty} h(c)$.
	\dapso{$\lim\limits _{c \rightarrow +\infty} h(c)= \dfrac{g}{8}$}
	\loigiai{
		Ta có 
		\[
		\lim\limits _{c \rightarrow +\infty} c \sqrt{\dfrac{c^2}{g^2}+\dfrac{1}{4}}- \dfrac{c^2}{g} = \lim\limits _{c \rightarrow +\infty} \dfrac{c\left (\dfrac{c^2}{g^2}+\dfrac{1}{4} - \dfrac{c^2}{g^2}\right )}{\sqrt{\dfrac{c^2}{g^2}+\dfrac{1}{4}} + \dfrac{c}{g}} = \lim\limits _{c \rightarrow +\infty} \dfrac{\dfrac{1}{4}}{\sqrt{\dfrac{1}{g^2} + \dfrac{1}{4c^2}}+ \dfrac{1}{g}} = \dfrac{g}{8}.
		\]
	}
\end{vd}

\subsubsection{Bài tập rèn luyện}
% \centerline{\fcolorbox{red}{yellow!50}{\bf {BÀI TẬP TỰ LUẬN}}}
\begin{bt}%[DCHT Toán 11 - KNTT- Phạm Tuấn]%[1K5YF-8] 
	Thế Lennard-Jones có dạng $$U(r) = \dfrac{B}{r^{12}} - \dfrac{A}{r^6}$$ trong đó $A$, $B$ là các hằng số và $r$ là khoảng cách giữa các hạt. 
	Tính $\lim\limits _{r \rightarrow +\infty} U(r)$.
	\dapso{$\lim\limits _{r \rightarrow +\infty} U(r) =0$}
	\loigiai{
		Ta có 
		$$\lim\limits _{r \rightarrow +\infty} U(r) =\lim\limits _{r \rightarrow +\infty} \left (\dfrac{B}{r^{12}} - \dfrac{A}{r^6} \right )  =0.$$
	}
\end{bt}

\begin{bt}%[DCHT Toán 11 - KNTT- Phạm Tuấn]%[1K5YF-8] 
	Trong thuyết tương đối, chiều dài của một vật thể đối với người quan sát phụ thuộc vào tốc độ mà vật thể đang chuyển động đối với người quan sát. Nếu người quan sát đo chiều dài của vật thể là $L_0$ khi đứng yên, thì ở tốc độ $v$ chiều dài  là
	$$
	L=L_0 \sqrt{1-\frac{v^2}{c^2}}
	$$
	trong đó $c$ là tốc độ ánh sáng trong chân không. Tìm $\displaystyle \lim \limits_{n \to +\infty}_{v \rightarrow c^{-}} L$. 
	\dapso{$\displaystyle \lim \limits_{n \to +\infty}_{v \rightarrow c^{-}} L =0$}
	\loigiai{
		Ta có $\displaystyle \lim \limits_{n \to +\infty}_{v \rightarrow c^{-}} L = \lim \limits_{n \to +\infty}_{v \rightarrow c^{-}} L_0 \sqrt{1-\frac{v^2}{c^2}} =  \lim \limits_{n \to +\infty}_{v \rightarrow c^{-}} L_0 \sqrt{1-\frac{c^2}{c^2}} =0$. 
	}
\end{bt}

\begin{bt}%[DCHT Toán 11 - KNTT- Phạm Tuấn]%[1K5BF-8] 
	Trong kỹ thuật ứng dụng, chúng ta thường xuyên ghi nhận được các hàm số mà giá trị của nó thay đổi đột ngột tại một thời điểm $t$ xác định. Ví dụ:  Sự thay đổi điện áp của một mạch điện tại thời điểm t khi đóng hoặc ngắt mạch. Thông thường, giá trị t = 0 luôn được chọn là thời điểm bắt đầu cho việc đóng hoặc ngắt điện áp. Quá trình đóng, ngắt mạch trên có thể mô tả bằng mô hình toán học bởi hàm Heaviside
	\[
	u(t) = \heva{&0 && \text{ nếu } t <0\\& 1 && \text{ nếu } t \geq 0.}
	\]
	Có tồn tại giới hạn $\displaystyle \lim \limits_{t\to 0} u(t)$ hay không?
	\dapso{$\displaystyle \lim \limits_{t\to 0} u(t)$ không tồn tại}
	\loigiai{
		Ta có $\displaystyle \lim \limits_{t\to 0^+} u(t) = \lim \limits_{t\to 0^+} 1 =1$; $\displaystyle \lim \limits_{t\to 0^-} u(t) = \lim \limits_{t\to 0^-} 0 =0$. \\
		Vậy giới hạn $\displaystyle \lim \limits_{t\to 0} u(t)$ không tồn tại.
	}
\end{bt}

\begin{bt}%[DCHT Toán 11 - KNTT- Phạm Tuấn]%[1K5BF-8] 
	Trong một cuộc thi các môn thể thao trên tuyết, người ta muốn thiết kế một đường trượt bằng băng cho nội dung đổ dốc tốc độ đường dài.
	\begin{center}
		\begin{tikzpicture}[scale=0.9, font=\footnotesize, line join=round, line cap=round, >=stealth]
			\draw[->] (0,0)--(10,0) node[below]{$x$} ;
			\draw[->] (0,0)--(0,4) node[left]{$y$} ;
			\foreach \x in {5,10,15,20,25,30,35,40,45}
			\draw[shift={({\x/5},0)},color=black] (0,0) -- (0pt,-2pt) node[below] {$\x$};
			\draw (0,3) node[left]{$15$} (0,0) node[below left]{$O$};
			\clip (0,0) rectangle (9,4) ;
			\draw[thick,smooth,samples=100,domain=0:9] plot(\x,{9/(2*(\x)+3)}) ;
		\end{tikzpicture}
	\end{center}
	Vận động viên sẽ xuất phát từ vị trí $(0 ; 15)$ cao $15$ m so với mặt đất (trục $Ox$). Đường trượt phải thoả mãn yêu cầu là càng ra xa thì càng gần mặt đất để tiết kiệm lượng tuyết nhân tạo. Một nhà thiết kế đề nghị sử dụng đường cong là đồ thị hàm số $y=f(x)=\dfrac{150}{x+10}$, với $x \geq 0$. Hãy kiểm tra xem hàm số $y=f(x)$ có thoả mãn các điều kiện dưới đây hay không:
	\begin{enumerate}
		\item    Có đồ thị qua điểm $(0; 15)$;
		\item    Giảm trên $[0 ;+\infty)$;
		\item    Càng ra xa ($x$ càng lớn), đồ thị của hàm số càng gần trục $O x$ với khoảng cách nhỏ tuỳ ý.
	\end{enumerate}
	\dapso{Đồ thị qua điểm $(0; 15)$; Hàm số giảm trên $[0 ;+\infty)$; $\lim\limits _{x \rightarrow +\infty} f(x) = 0$}
	\loigiai{
		\begin{enumerate}
			\item    Ta có $f(0)= \dfrac{150}{10}$ nên đồ thị hàm số $f(x)$ đi qua điểm $(0; 15)$.
			\item    Chọn bất kì $x_1,x_2 \in [0;+\infty]$ và $x_1 \ne x_2$. \\
			Ta có $\dfrac{f(x_2)-f(x_1)}{x_2-x_1} =  \dfrac{\dfrac{150}{x_2+10} - \dfrac{150}{x_1+10}}{x_2-x_1} = \dfrac{x_1-x_2}{(x_2-x_1)(x_1+10)(x_2+10)} = -\dfrac{1}{(x_1+10)(x_2+10)} <0$. \\
			Suy ra hàm số nghịch biến trên  $[0 ;+\infty)$ hay hàm số giảm trên $[0 ;+\infty)$.
			\item   Ta có $\lim\limits _{x \rightarrow +\infty}  f(x) = \lim\limits _{x \rightarrow +\infty} \dfrac{150}{x+10} = 0$. \\
			Vậy khi $x$ càng lớn, đồ thị của hàm số càng gần trục $O x$ với khoảng cách nhỏ tuỳ ý.
		\end{enumerate}
	}
\end{bt}

\begin{bt}%[DCHT Toán 11 - KNTT- Phạm Tuấn]%[1K5YF-8] 
	Chiều dài một loài động vật nhỏ được tính theo công công thức $h(t)=\dfrac{100}{2+3 \cdot  (0{,}4)^t}$ mm, trong đó $t$ số ngày sau khi sinh của loài động vật đó. Tính chiều dài  cuối cùng của nó (chiều dài khi $t \to +\infty$).
	\dapso{$100$ mm}
	\loigiai{
		Ta có $\displaystyle \lim \limits_{t \to +\infty } h(t)=\dfrac{100}{2+3 \cdot  (0{,}4)^t} = 100$. \\
		Vậy chiều dài của loài động vật khi trưởng thành là $100$ mm. 
	}
\end{bt}

\begin{bt}%[DCHT Toán 11 - KNTT- Phạm Tuấn]%[1K5BF-8] 
	Một chất điểm chuyển động thẳng với phương trình $s(t)$. Khi đó vận tốc tức thời tại thời điểm $t_0$ được định nghĩa là $\displaystyle \lim \limits_{\Delta t} \dfrac{s(t_0+ \Delta t) - s(t_0)}{\Delta t}$. Tính vận tốc tức thời của chất điểm với phương trình chuyển động $s(t) = 4t^2-3t+1$ ($s(t)$ có đơn vị là m, $t$ đơn vị là giây), tại thời điểm $t=8$ giây. 
	\dapso{$61  \mathrm{~m/s}$}
	\loigiai{
		Vận tốc tức thời của chất điểm tại thời điểm $t=8$ giây là
		\begin{align*}
			\lim \limits_{\Delta t \to 0} \dfrac{s(8+\Delta t) - s(8)}{\Delta t} &=  \lim \limits_{\Delta t \to 0} \dfrac{4(8+\Delta t)^2-3(8+\Delta t)+1 - 233}{\Delta t} \\
			& = \lim \limits_{\Delta t \to 0} \dfrac{4 (\Delta t)^2+ 61 \Delta t}{\Delta t} = 61  \mathrm{~m/s}.
		\end{align*}
	}
\end{bt}

\begin{bt}%[DCHT Toán 11 - KNTT- Phạm Tuấn]%[1K5YF-8] 
	Bỏ qua lực cản của không khí, độ cao tối đa mà tên lửa đạt được khi phóng với vận tốc ban đầu $v_0$ là $h=\dfrac{v_0^2 R}{19{,}6 R-v_0^2}$, trong đó $R$ là bán kính của trái đất. Tính  $\displaystyle \lim \limits_{R \to +\infty} h$. 
	\dapso{$\lim \limits_{R \to +\infty} h =  \dfrac{v_0^2}{19{,}6}$}
	\loigiai{
		Ta có 
		\begin{align*}
			\lim \limits_{R \to +\infty} h &= \displaystyle \lim \limits_{R \to +\infty} \dfrac{v_0^2 R}{19{,}6 R-v_0^2} \\
			&= \lim \limits_{R \to +\infty} \dfrac{v_0^2}{19{,}6 - \dfrac{v_0^2}{R}} = \dfrac{v_0^2}{19{,}6}.
		\end{align*}
	}
\end{bt}


\begin{bt}%[DCHT Toán 11 - KNTT- Phạm Tuấn]%[1K5BF-8] 
	Một hình elip với bán trục lớn $a$ và bán trục nhỏ $b$ thì diện tích được tính theo công thức $S=\pi ab$. Cho elip có bán trục nhỏ bằng $30$ cm, tính giới hạn diện tích của elip khi tiêu cự gần tới $0$.
	\dapso{$900\pi \mathrm{~cm^2}$}
	\loigiai{
		Ta có $S= \pi ab = \pi b\sqrt{b^2+c^2}$. \\
		Vậy $\lim\limits_{c \to 0} S= \lim\limits_{c \to 0} \pi b\sqrt{b^2+c^2} = \pi b^2 = 900\pi \mathrm{~cm^2}$. 
	}
\end{bt}





\begin{bt}%[DCHT Toán 11 - KNTT- Phạm Tuấn]%[1K5BF-8] 
	Số lượng đơn vị hàng tồn kho trong một công ty nhỏ được cho bởi
	$$
	N(t)=25\left(2 \left [\frac{t+2}{2}  \right ]-t\right)
	$$
	trong đó $t$ là thời gian tính bằng tháng, $[x]$ là số nguyên lớn nhất không vượt quá $x$ (ví dụ $[2{,}4]=2$, $[-2{,}7] = -3$).
	\begin{enumerate}
		\item Tính $\lim\limits_{t \to 8^+} N(t)$.
		\item  Tính $\lim\limits_{t \to 16^-} N(t)$.
	\end{enumerate}
	\dapso{$\lim\limits _{t \to 8^+} N(t)=50$; $\lim\limits _{t \to 16^-} N(t) =0$}
	\loigiai{
		\begin{enumerate}
			\item 
			Khi $t \to 8^+$, ta có $\left [\dfrac{t+2}{2}  \right ] = 5$. \\
			Suy ra  $\lim\limits _{t \to 8^+} N(t) =\lim_{t \to 8^+}  25\left(2 \left [\frac{t+2}{2}  \right ]-t\right) = 50$.
			\item  
			Khi $t \to 16^-$, ta có $\left [\dfrac{t+2}{2}  \right ] = 8$. \\
			Suy ra  $\lim\limits _{t \to 16^-} N(t) = \lim_{t \to 16^-} 25\left(2 \left [\frac{t+2}{2}  \right ]-t\right) = 0$.
		\end{enumerate}
	}
\end{bt}


\begin{bt}%[DCHT Toán 11 - KNTT- Phạm Tuấn]%[1K5BF-8] 
	Định luật Boyle được phát biểu:  ``Đối với một lượng khí ở nhiệt độ không đổi, áp suất $P$ tỷ lệ nghịch với thể tích $V$''. Tìm giới hạn của $P$ là $V \rightarrow 0^{+}$.
	\dapso{$\lim \limits_{V \to 0^+} P = +\infty$}
	\loigiai{
		Ta có $P= \dfrac{k}{V}$ với $k$ là số thực dương không đổi. \\
		Khi đó $\lim \limits_{V \to 0^+} P= \dfrac{k}{V} = +\infty$.
	}
\end{bt}

\begin{bt}%[DCHT Toán 11 - KNTT- Phạm Tuấn]%[1K5BF-8] 
	Một vật khối lượng $m$ (không đổi) bắt đầu chuyển động với vận tốc $v_0=0$, được gia tốc bởi một lực $F$ không đổi trong $t$ giây. Theo định luật Newton về chuyển động, vật tốc của vật là $v_N = \dfrac{Ft}{m}$. Theo thuyết tương đối Einstein, vật có vận tốc $v_E = \dfrac{Fct}{\sqrt{m^2c^2+F^2t^2}}$, với $c$ là vận tốc ánh sáng. Tính $\displaystyle \lim \limits_{t \to +\infty} v_N$ và $\displaystyle \lim \limits_{t \to +\infty} v_E$.
	\dapso{$\lim \limits_{t \to +\infty} v_N  = +\infty$; $\lim \limits_{t \to +\infty} v_E  = \dfrac{c}{F}$}
	\loigiai{
		Ta có
		\begin{align*}
			& \lim_{t \to +\infty} v_N = \lim_{t \to +\infty}\dfrac{Ft}{m} = +\infty; \\
			& \lim_{t \to +\infty} v_E = \lim_{t \to +\infty}\dfrac{Fct}{\sqrt{m^2c^2+F^2t^2}} =  \lim_{t \to +\infty} \dfrac{Fc}{\sqrt{\dfrac{m^2c^2}{t^2}}+F^2} = \dfrac{c}{F}.
		\end{align*}
	}
\end{bt}


\begin{bt}%[DCHT Toán 11 - KNTT- Phạm Tuấn]%[1K5BF-8] 
	\immini{
		Gọi $S$ là diện tích hình phẳng giới hạn bởi đường tròn bán kính $10$ và tam giác vuông (hình vẽ bên).
		\begin{enumerate}
			\item Đặt $S= f(\varphi)$, với $f(\varphi)$ là hàm số của $\varphi$ (Đơn vị rad). Tìm công thức của $f(\varphi)$ với $0< \varphi < \dfrac{\pi}{2}$.
			\item Tính giới hạn của $f(\varphi)$ khi $\varphi \to \dfrac{\pi}{2}^-$.
		\end{enumerate}
	}
	{
		\begin{tikzpicture}[scale=1, font=\footnotesize, line join=round, line cap=round, >=stealth]
			\path
			(0,0) coordinate (A)
			(4,0) coordinate (B)
			(0,2.5) coordinate (C)
			($(B)!{4/sqrt(4^2+2.5^2)}!(C)$)  coordinate (D)
			;
			\fill[cyan!30] (A) arc (180:{180-atan(2.5/4)}:4) -- (C)--cycle;
			\draw (A) arc (180:{180-atan(2.5/4)}:4)  ;
			\draw (A)--(B)--(C)--(A);
			\draw pic["$\varphi$", draw=black, angle eccentricity=1.3, angle radius=0.7cm]{angle=C--B--A} ;
			\foreach \x/\g in {A/-120,B/-30,C/90,D/50} 
			\fill[black](\x) circle (1pt)+(\g:2.5mm) node{$\x$};
		\end{tikzpicture}
	}
	\dapso{$S=8(\tan \varphi - \varphi)$; $\lim \limits_{\varphi \to \tfrac{\pi}{2}^-}  f(\varphi) =+\infty $}
	\loigiai{
		\begin{enumerate}
			\item Diện tích tam giác $ABC$ là $S_{ABC} = \dfrac{1}{2} AB \cdot AC = \dfrac{1}{2} \cdot 4 \cdot 4 \tan \varphi = 8  \tan \varphi$. \\
			Diện tích hình quạt $ABD$ là $S_q = \dfrac{4^2 \varphi }{2} = 8 \varphi$. \\
			Diện tích hình phẳng $S$ là $S=f(\varphi) = S_{ABC} - S_q = 8(\tan \varphi - \varphi)$.
			\item 
			Khi $\varphi \to \tfrac{\pi}{2}^-$ thì $\heva{&\cos \varphi \to 0\\& \cos \varphi >0.}$\\
			Suy ra $\displaystyle \lim \limits_{\varphi \to \tfrac{\pi}{2}^-}  f(\varphi) = \lim \limits_{\varphi \to \tfrac{\pi}{2}^-}  8\left (\dfrac{\sin \varphi}{\cos \varphi} - \varphi \right )= +\infty$.
		\end{enumerate}
	}
\end{bt}

\begin{bt}%[DCHT Toán 11 - KNTT- Phạm Tuấn]%[1K5BF-8] 
	Trên một chuyến đi dài $d$ km đến một thành phố khác, vận tốc trung bình của một tài xế xe tải là $x$ km/h. Trên chuyến trở về, vận tốc trung bình là $y$ km/h. Vận tốc trung bình của cả đi và về là 50 km/h.
	\begin{enumerate}
		\item Chứng minh rằng $y= \dfrac{25x}{x-25}$.
		\item Tìm giới hạn của $y$ khi $x \rightarrow 25^{+}$ và giải thích ý nghĩa của nó.
	\end{enumerate}
	\dapso{$\displaystyle \lim\limits_{x\to 25^+}  y = +\infty$}
	\loigiai{
		\begin{enumerate}
			\item  Thời gian  chuyến đi là $\dfrac{d}{x}$, thời gian  chuyến trở về là $\dfrac{d}{y}$. \\
			Suy ra 
			$$\dfrac{2d}{50} = \dfrac{d}{x} + \dfrac{d}{y} \Leftrightarrow \dfrac{1}{y} = \dfrac{1}{25} - \dfrac{1}{x} \Leftrightarrow y= \dfrac{25x}{x-25}.$$
			\item  Ta có $\displaystyle \lim\limits_{x\to 25^+}  y = \lim\limits_{x\to 25^+}  \dfrac{25x}{x-25} = +\infty$. \\
			Khi vận tốc trung bình chuyến đi bằng 25 km/h,  thì  vận tốc trung bình chuyến của cả chuyến đi và về không thể là 50 km/h.
		\end{enumerate}
	}
\end{bt}

\begin{bt}%[DCHT Toán 11 - KNTT- Phạm Tuấn]%[1K5BF-8] 
	Một chất điểm chuyển động thẳng với vận tốc $v(t)$. Khi đó gia tốc tức thời tại thời điểm $t_0$ được định nghĩa là $\displaystyle \lim \limits_{\Delta t \to 0} \dfrac{v(t_0+ \Delta t) - v(t_0)}{\Delta t}$. Một chất điểm chuyển động với vận tốc $v(t) = 5 \sin \left (4\pi t\right )$ (m/s), tính gia tốc tức thời tại thời điểm $t=5$ giây.  (Biết $\displaystyle \lim_{x \to 0} \dfrac{\sin x}{x} =1$). 
	\dapso{$20\pi$ $\mathrm{m/s^2}$}
	\loigiai{
		Gia tốc tức thời của chất điểm tại thời điểm $t=5$ giây là
		\begin{align*}
			\lim \limits_{\Delta t \to 0} \dfrac{v(5+\Delta t) - s(5)}{\Delta t} &=  \lim \limits_{\Delta t \to 0} \dfrac{5\sin (20\pi+4\pi\Delta t) - 5\sin (20\pi)}{\Delta t} \\
			& = \lim \limits_{\Delta t \to 0} \dfrac{5\sin (4\pi\Delta t)}{\Delta t}   \\
			& = \lim \limits_{\Delta t \to 0} \dfrac{20\pi\sin (4\pi\Delta t)}{4\pi\Delta t}   \\
			& = 20\pi.
		\end{align*}
		Vậy gia tốc tức thời của chất điểm tại thời điểm $t=5$ giây là $20\pi$ ($\mathrm{m/s^2}$).
	}
\end{bt}


\begin{bt}%[DCHT Toán 11 - KNTT- Phạm Tuấn]%[1K5BF-8] 
	Một bể chứa $5000$ lít nước tinh khiết. Nước muối chứa $30$ g muối trên một lít nước được bơm vào bể với tốc độ $25$ lít/phút. Gọi nồng độ của muối sau $t$ phút (tính bằng gam trên lít) là $C(t)$. Tính $\displaystyle \lim \limits_{t \to +\infty} C(t)$.  Giải thích ý nghĩa của giới hạn này.
	\dapso{$\lim \limits_{t \to +\infty} C(t) =30$}
	\loigiai{
		Số lít nước muối được bơm vào bể sau $t$ phút là $25t$ lít. \\
		Số g muối có trong $25t$ lít nước muối là $30 \cdot 25 t = 750t$ gam. \\
		Nồng độ của muối trong bể sau $t$ phút  là
		\[
		\dfrac{750t}{25t + 5000} = \dfrac{30t}{t+200} \text{~gam/lít}.
		\]
		Ta có $\displaystyle \lim \limits_{t \to +\infty} C(t) = \lim \limits_{t \to +\infty} \dfrac{30t}{t+200} =  30  \text{~gam/lít}$. \\
		Khi thời gian tiến tới vô hạn thì nồng độ của muối trong bể bằng nồng độ của nước muối bơm vào bể.
	}
\end{bt}



\begin{bt}%[DCHT Toán 11 - KNTT- Phạm Tuấn]%[1K5BF-8] 
	Một thấu kính hội tụ có tiêu cự $f=30 \mathrm{~cm}$. Trong Vật lí, ta biết rằng nếu đặt vật thật $A B$ cách quang tâm của thấu kính một khoảng $d>30$ ($\mathrm{cm}$) thì được ảnh thật $A' B'$ cách quang tâm của thấu kính một khoảng $d'$ (cm) (Hình vẽ dưới). Ngược lại, nếu $0<d<30$, ta có ảnh ảo. Công thức của thấu kính là $\dfrac{1}{d}+\dfrac{1}{d'}=\dfrac{1}{30}$.
	\begin{center}
		\begin{tikzpicture}[scale=1, font=\footnotesize, line join=round, line cap=round, >=stealth]
			\path
			(0,0) coordinate (O)
			(-4,0) coordinate (A)
			(8,0) coordinate (A')
			(-4,1) coordinate (B)
			(8,-2) coordinate (B')
			(0,1) coordinate (B1)
			({-8/3},1.4) coordinate (P)
			(0,1.4) coordinate (Q)
			({8/3},1.4) coordinate (R)
			(-4,-1.4) coordinate (X)
			(0,-1.4) coordinate (Y)
			(8,-1.4) coordinate (Z)
			(intersection of O--A' and B1--B')  coordinate (F')
			($(F')!2!(O)$) coordinate (F)
			;
			\draw (-5,0)--(9,0) ;
			\draw[<->,>=triangle 45,very thick] (0,-2.2)--(0,2.2);  
			\draw[->,>=triangle 45,very thick] (A)--(B);  
			\draw[->,>=triangle 45,very thick] (A')--(B');  
			\draw[<->,dashed] (P)--(Q) ;
			\draw[<->,dashed] (Q)--(R) ;
			\draw[<->,dashed] (X)--(Y) ;
			\draw[<->,dashed] (Y)--(Z) ;
			\draw[dashed] (F)--(P) (F')--(R) (A)--(X) (A')--(Z);
			\draw[fill] ($(P)!0.5!(Q)$) node[above]{$30$} ($(Q)!0.5!(R)$) node[above]{$30$}
			($(X)!0.5!(Y)$) node[above]{$d$} ($(Y)!0.4!(Z)$) node[above]{$d'$} (F) circle(1.2pt) (F') circle(1.2pt);
			\draw (B)--(B') (B)--(B1)--(B');
			\foreach \x/\g in{A/-50,B/90,A'/90,B'/-90,O/-130,F/-90,F'/-90}
			\fill[black](\x)  ($(\x)+(\g:3mm)$) node{$\x$}; 
		\end{tikzpicture}
	\end{center} 
	\begin{enumerate}
		\item Từ công thức của thấu kính, hãy tìm biểu thức xác định hàm số $d'=h(d)$.
		\item Tìm các giới hạn $\lim\limits  _{d \rightarrow 30^{+}} h(d) ; \lim\limits _{d \rightarrow 30^{-}} h(d)$ và $\lim\limits _{d \rightarrow+\infty} h(d)$. Sử dụng các kết quả này để giải thích ý nghĩa đã biết trong Vật lí.
	\end{enumerate}
	\dapso{$d'=h(d) = \dfrac{30d}{d-30}$; $\lim\limits  _{d \rightarrow 30^{+}} h(d) = +\infty$; $\lim\limits  _{d \rightarrow 30^{-}} h(d) = -\infty$}
	\loigiai{
		\begin{enumerate}
			\item Ta có $\dfrac{1}{d}+\dfrac{1}{d'}=\dfrac{1}{30} \Leftrightarrow \dfrac{1}{d'} = \dfrac{1}{30} - \dfrac{1}{d} = \dfrac{d-30}{30d} \Leftrightarrow d'= \dfrac{30d}{d-30}$. \\
			Vậy $d'=h(d) = \dfrac{30d}{d-30}$.
			\item 
			Khi $d\to 30^+$, ta có $d-30 \to 0$, $d-30 >0$ và $30d \to 900$.\\
			Suy ra $\lim\limits _{d \rightarrow 30^{+}} h(d) = \lim\limits _{d \rightarrow 30^{+}} \dfrac{30d}{d-30} = +\infty$. \\
			Khi $d\to 30^-$, ta có $d-30 \to 0$, $d-30 <0$ và $30d \to 900$.\\
			Suy ra $\lim\limits _{d \rightarrow 30^{-}} h(d) = \lim\limits _{d \rightarrow 30^{-}} \dfrac{30d}{d-30} = -\infty$. \\
			Ta có $\lim\limits _{d \rightarrow +\infty} h(d) =  \lim\limits _{d \rightarrow +\infty} \dfrac{30d}{d-30} = \lim\limits _{d \rightarrow +\infty}  \dfrac{30}{1-\dfrac{30}{d}} =30$. \\
			Vậy 
			\begin{itemize}
				\item  Khi vị trí của vật tiến gần  tiêu điểm $F$ ($d >f$) thì vị trí ảnh thật của vật dần ra xa vô cực. 
				\item  Khi vị trí của vật tiến gần  tiêu điểm $F$ ($d <f$) thì vị trí ảnh ảo của vật dần ra xa vô cực. 
				\item  Khi vị trí của vật tiến ra xa vô cực thì  ảnh thật của vật dần tới tiêu điểm. 
			\end{itemize}
		\end{enumerate}
	}
\end{bt}



%%Bài 17. Hàm số liên tục
\section{Hàm số liên tục}
\setcounter{dang}{0}
\subsection{Tóm tắt lý thuyết}
\begin{tomtat}
	\subsubsection{Hàm số liên tục tại một điểm}
	\begin{itemize}
		\item Cho hàm số $y=f(x)$ xác định trên khoảng $(a;b)$ chứa điểm $x_0$. Hàm số $f(x)$ được gọi là \textcolor{red}{liên tục tại điểm} $x_0$ nếu $\lim\limits_{x\to{x_0}}f(x)=f(x_0)$.
		\item Hàm số $f(x)$ không liên tục tại $x_0$ được gọi là \textcolor{red}{gián đoạn} tại điểm đó.
		\begin{note}
			Hàm số $y=f(x)$ liên tục tại $x_0$ khi và chỉ khi $\lim\limits_{x\to{x_0}^+}f(x)=\lim\limits_{x\to{x_0}^-}f(x)=f(x_0)$.
		\end{note}
	\end{itemize}
	\subsubsection{Hàm số liên tục trên một khoảng}
	\begin{itemize}
		\item Hàm số $y=f(x)$ được gọi là \textcolor{red}{liên tục trên khoảng} $(a;b)$ nếu nó liên tục tại mọi điểm thuộc khoảng này.
		\item Hàm số $y=f(x)$ được gọi là \textcolor{red}{liên tục trên đoạn} $[a;b]$ nếu nó liên tục trên khoảng $(a;b)$ và $\lim\limits_{x\to{a}^+}f(x)=f(a)$, $\lim\limits_{x\to{b}^-}f(x)=f(b)$.
		\item Các khái niệm hàm số liên tục trên nửa khoảng như $(a;b]$, $[a;+\infty)\ldots$ được định nghĩa theo cách tương tự.
		\item Đồ thị của hàm số liên tục trên một khoảng là một \textcolor{red}{đường liền nét} trên khoảng đó.
	\end{itemize}
	\subsubsection{Tính chất 1}
	\begin{itemize}
		\item Hàm số đa thức và các hàm số $y=\sin x$, $y=\cos x$ liên tục trên $\mathrm{R}$.
		\item Các hàm số $y=\tan x$, $y=\cot x$, $y=\sqrt x$ và các hàm phân thức hữu tỉ (thương của hai đa thức) liên tục trên mỗi khoảng xác định của chúng.
	\end{itemize}
	\subsubsection{Tính chất 2}
	Giả sử hai hàm số $y=f(x)$ và $y=g(x)$ liên tục tại điểm $x_0$. Khi đó:
	\begin{itemize}
		\item Các hàm số $y=f(x)+g(x)$, $y=f(x)-g(x)$ và $y=f(x)\cdot g(x)$ liên tục tại $x_0$.
		\item Hàm số $\dfrac{f(x)}{g(x)}$ liên tục tại $x_0$ nếu $g(x_0)\ne 0$.
	\end{itemize}
\end{tomtat}
\subsection{Các dạng toán thường gặp}
\begin{dang}{Dựa vào đồ thị xét tính liên tục của hàm số tại một điểm, một khoảng.}
	Để xét tính liên tục của hàm số khi biết đồ thị, ta cần nhớ:
	\begin{itemize}
		\item Đồ thị của hàm số liên tục trên một khoảng là một đường liền nét trên khoảng đó.
		\item Hàm số $y=f(x)$ liên tục tại $x_0$ khi và chỉ khi $\lim\limits_{x\to{x_0}^+}f(x)=\lim\limits_{x\to{x_0}^-}f(x)=f(x_0)$.
	\end{itemize}
\end{dang}
\subsubsection{Ví dụ mẫu}
\begin{vd}%[DCHT Toán 11 - KNTT -Vũ Hồng Toàn]%[1K5YG-2]
	\immini{
		Cho hàm số $y=f(x)$ có đồ thị như hình vẽ bên.\\ Xét tính liên tục của hàm số $y=f(x)$ trên khoảng $(0;2)$.	
	}
	{
		\begin{tikzpicture}[scale=.75, font=\footnotesize, line join=round, line cap=round,>=stealth]
			\def\xmin{-.5} \def\xmax{2.5}
			\def\ymin{-0.5} \def\ymax{2.5}
			\draw[->] (\xmin,0)--(0,0)node [below right]{$O$}-- (\xmax,0) node [below]{$x$};
			\draw[->] (0,\ymin)--(0,\ymax) node [left]{$y$};
			\draw[ thick] (0,0)--(1,2)--(2,2);
			\draw[dash pattern=on 2pt off 1.5pt] (1,0)node[below]{$1$}--(1,2) (2,0)node[below]{$2$}--(2,2)  (1,2)node[above ]{$y=f(x)$};
			\foreach \x/\y in{0/0,1/0,2/0,1/2,2/2} \fill(\x,\y)circle(.03);
		\end{tikzpicture}		
	}
	% \dapso{Hàm số liên tục trên khoảng $(0;2)$.}
	\loigiai{
		Đồ thị hàm số là một đường liền nét trên khoảng $(0;2)$ nên hàm số đã cho liên tục trên khoảng $(0;2)$.}
\end{vd}
\begin{vd}%[DCHT Toán 11 - KNTT -Vũ Hồng Toàn]%[1K5YG-2]
	\immini{
		Cho hàm số $y=f(x)$ có đồ thị như hình vẽ bên.\\ Xét tính liên tục của hàm số $y=f(x)$ trên khoảng $(-2;2)$.	
	}
	{
		\begin{tikzpicture}[scale=.75, font=\footnotesize, line join=round, line cap=round,>=stealth]
			\def\xmin{-2.5} \def\xmax{2.5}
			\def\ymin{-0.5} \def\ymax{2.5}
			\draw[->] (\xmin,0)--(0,0)node [below right]{$O$}-- (\xmax,0) node [below]{$x$};
			\draw[->] (0,\ymin)--(0,\ymax) node [left]{$y$};
			\draw[ thick] plot[domain=-2:2](\x,{abs(\x)});
			\draw[dash pattern=on 2pt off 1.5pt] (-2,0)node[below]{$-2$}--(-2,2) (2,0)node[below]{$2$}--(2,2)  (.5,1)node[above ]{$y=f(x)$};
			\foreach \x/\y in{0/0,2/0,-2/2,2/2} \fill(\x,\y)circle(.03);
		\end{tikzpicture}		
	}
	% \dapso{Hàm số liên tục trên khoảng $(-2;2)$.}
	\loigiai{
		Đồ thị hàm số là một đường liền nét trên khoảng $(-2;2)$ nên hàm số đã cho liên tục trên khoảng $(-2;2)$.}
\end{vd}

\begin{vd}%[DCHT Toán 11 - KNTT -Vũ Hồng Toàn]%[1K5BG-2]
	\immini{
		Cho hàm số $y=f(x)$ có đồ thị như hình vẽ bên.\\ Xét tính liên tục của hàm số $y=f(x)$ trên khoảng $(0;2)$.	
	}
	{
		\begin{tikzpicture}[scale=.75, font=\footnotesize, line join=round, line cap=round,>=stealth]
			\def\xmin{-.5} \def\xmax{2.5}
			\def\ymin{-0.5} \def\ymax{3}
			\draw[->] (\xmin,0)--(0,0)node [below right]{$O$}-- (\xmax,0) node [below]{$x$};
			\draw[->] (0,\ymin)--(0,\ymax) node [left]{$y$};
			\draw[ thick] (0,0)--(1,1) (1,2)--(2,2);
			\draw[dash pattern=on 2pt off 1.5pt] (1,0)node[below]{$1$}--(1,1) (2,0)node[below]{$2$}--(2,2) (0,1)node[left]{$1$} (0,2)node[left]{$2$}
			(1,2)node[above ]{$y=f(x)$}
			;
			\foreach \x/\y in{0/0,1/0,2/0,1/2,2/2,1/1,0/1,0/2} \fill(\x,\y)circle(.03);
		\end{tikzpicture}		
	}
	
	% \dapso{Hàm số đã cho liên tục trên các khoảng $(0,1)$, $(1,2)$ và gián đoạn tại $x=1$.}
	\loigiai{
		\begin{itemize}
			\item Đồ thị hàm số là các đường liền nét trên các khoảng $(0;1)$, $(1;2)$ do đó hàm số liên tục trên các khoảng này.
			\item Đồ thị hàm số không liền nét tại điểm $x=1$ do đó hàm số đã cho gián đoạn tại điểm này.
		\end{itemize}
	}
\end{vd}
\begin{vd}%[DCHT Toán 11 - KNTT -Vũ Hồng Toàn]%[1K5BG-2]
	\immini{
		Cho hàm số $y=f(x)$ có đồ thị như hình vẽ bên.\\ Xét tính liên tục của hàm số $y=f(x)$ trên khoảng $(0;2)$.	
	}
	{
		\begin{tikzpicture}[scale=.75, font=\footnotesize, line join=round, line cap=round,>=stealth]
			\def\xmin{-.5} \def\xmax{2.5}
			\def\ymin{-0.5} \def\ymax{2.8}
			\draw[->] (\xmin,0)--(0,0)node [below right]{$O$}-- (\xmax,0) node [below]{$x$};
			\draw[->] (0,\ymin)--(0,\ymax) node [left]{$y$} ;
			\draw[ thick] (1,1)--(2,1) (0,2)--(1,1.2);
			\draw[dash pattern=on 2pt off 1.5pt] (1,0)node[below]{$1$}--(1,1) (2,0)node[below]{$2$}--(2,1) (0,1)node[left]{$1$} (0,2)node[left]{$2$} (0,1)--(1,1)
			(1,2)node[above ]{$y=f(x)$}
			;
			\foreach \x/\y in{0/0,1/0,2/0,2/1,0/1,0/2} \fill(\x,\y)circle(.03);
		\end{tikzpicture}		
	}
	% \dapso{Hàm số đã cho liên tục trên các khoảng $(0,1)$, $(1,2)$ và gián đoạn tại $x=1$.}
	\loigiai{
		\begin{itemize}
			\item Đồ thị hàm số là các đường liền nét trên các khoảng $(0;1)$, $(1;2)$ do đó hàm số liên tục trên các khoảng này.
			\item Ta có $\lim\limits_{x\to{1}^-}f(x)>f(1)=1$ và $\lim\limits_{x\to{1}^+}f(x)=f(1)=1$.\\ Do đó $\lim\limits_{x\to{1}^-}f(x)\ne \lim\limits_{x\to{1}^+}f(x)$.\\
			Vậy hàm số đã cho gián đoạn tại $x=1$.
		\end{itemize}
	}
\end{vd}

\begin{vd}%[DCHT Toán 11 - KNTT -Vũ Hồng Toàn]%[1K5KG-2]
	\immini{
		Cho hàm số $y=f(x)$ có tập xác định $\mathscr{D}=\mathbb{R}\setminus \{0\}$ và có đồ thị như hình bên. Xét tính liên tục của hàm số $y=f(x)$ trên $\mathscr{D}$.	
	}
	{
		\begin{tikzpicture}[scale=.75, font=\footnotesize, line join=round, line cap=round,>=stealth]
			\def\xmin{-2.5} \def\xmax{2.8}
			\def\ymin{-0.5} \def\ymax{2.8}
			%\draw[color=gray!50,dashed] (\xmin,\ymin) grid (\xmax,\ymax);
			\draw[->] (\xmin,0)--(0,0)node [below right]{$O$}-- (\xmax,0) node [below]{$x$};
			\draw[->] (0,\ymin)--(0,\ymax) node [left]{$y$};
			\begin{scope}
				\clip (\xmin,\ymin) rectangle (\xmax,\ymax-.15);
				\draw[samples=200,smooth,variable=\x,thick, teal] plot[domain=\xmin:-0.5] (\x,{1/(\x)^2});
				\draw[samples=200,smooth,variable=\x,thick, teal] plot[domain=0.5:\xmax] (\x,{1/(\x)^2});
			\end{scope}
			\draw[ thick] (1.2,1)node[right]{$y=f(x)$};
			\foreach \x/\y in{0/0} \fill(\x,\y)circle(.03);
		\end{tikzpicture}	
	}
	% \dapso{Hàm số đã cho liên tục trên các khoảng $(-\infty;0)$ và $(0;+\infty)$. Gián đoạn tại điểm $x=0$.}
	\loigiai{
		Vì hàm số đã cho có tập xác định $\mathscr{D}=\mathbb{R}\setminus \{0\}$ nên
		\begin{itemize}
			\item $f(x)$ xác định trên khoảng $(-\infty;0)$ nên liên tục trên khoảng này.
			\item $f(x)$ xác định trên khoảng $(0;+\infty)$ nên liên tục trên khoảng này.
			\item $f(x)$ không xác định tại điểm $x=0$ nên gián đoạn tại điểm này.
		\end{itemize}	
	}
\end{vd}

% \subsubsection{Bài tập rèn luyện}
% % \centerline{\fcolorbox{teal}{yellow!50}{\bf {BÀI TẬP TỰ LUẬN }}}
% \begin{bt}%[DCHT Toán 11 - KNTT -Vũ Hồng Toàn]%[1K5YG-2]
% 	\immini{
% 		Cho hàm số $y=f(x)$ xác định trên $\mathscr{D}=\mathbb{R}$ và có đồ thị như hình vẽ bên.\\ Xét tính liên tục của hàm số $y=f(x)$ trên $\mathscr{D}$.	
% 	}
% 	{
% 		\begin{tikzpicture}[scale=.75, font=\footnotesize, line join=round, line cap=round,>=stealth]
% 			\def\a{1} \def\b{-2} \def\c{0} %\def\d{1} % Hệ số
% 			\def\xmin{-2} \def\xmax{2}
% 			\def\ymin{-1.5} \def\ymax{1.5}
% 			\draw[->] (\xmin,0)--(0,0)node [above right]{$O$}-- (\xmax,0) node [below]{$x$};
% 			\draw[->] (0,\ymin)--(0,\ymax) node [left]{$y$};
% 			\clip (\xmin+0.1,\ymin+0.1) rectangle (\xmax-0.1,\ymax-0.1);
% 			\draw[smooth,samples=100,thick ] plot(\x,{\a*(\x)^4+\b*(\x)^2+\c});
% 			\draw[ thick] (-.7,1)node[below]{$y=f(x)$};
% 		\end{tikzpicture}	
% 	}	
% 	% \dapso{Hàm số đã cho liên tục trên $\mathbb{R}$.}
% 	\loigiai{
% 		Do đồ thị hàm số là một đường liền nét trên $\mathscr{D}$ nên hàm số $y=f(x)$ liên tục trên $\mathscr{D}=\mathbb{R}$.}
% \end{bt}
% \begin{bt}%[DCHT Toán 11 - KNTT -Vũ Hồng Toàn]%[1K5BG-2]
% 	\immini{
% 		Cho hàm số $y=f(x)$ xác định trên $\mathscr{D}=\mathbb{R}$ và có đồ thị như hình vẽ bên. Xét tính liên tục của hàm số $y=f(x)$ trên $\mathscr{D}$.	
% 	}
% 	{
% 		\begin{tikzpicture}[scale=.75, font=\normalsize, line join=round, line cap=round,>=stealth]
% 			\def\a{2} %\def\b{-2} \def\c{0} %\def\d{1} % Hệ số
% 			\def\xmin{-2.5} \def\xmax{2.5}
% 			\def\ymin{-0.5} \def\ymax{3}
% 			%\draw[color=gray!50,dashed] (\xmin,\ymin) grid (\xmax,\ymax);
% 			\draw[->] (\xmin,0)--(0,0)node [below right]{$O$}-- (\xmax,0) node [below]{$x$};
% 			\draw[->] (0,\ymin)--(0,\ymax) node [left]{$y$};
% 			\clip (\xmin+0.1,\ymin+0.1) rectangle (\xmax-0.1,\ymax-0.1);
% 			\draw[smooth,samples=100,thick ] plot(\x,{\a^(\x)}) (.5,1) node [right]{$y=f(x)$};
% 		\end{tikzpicture}	
% 	}
% 	% \dapso{Hàm số đã cho liên tục trên $\mathbb{R}$.}
% 	\loigiai{
% 		Do đồ thị hàm số là một đường liền nét trên $\mathscr{D}$ nên hàm số $y=f(x)$ liên tục trên $\mathscr{D}=\mathbb{R}$.}
% \end{bt}
% \begin{bt}%[DCHT Toán 11 - KNTT -Vũ Hồng Toàn]%[1K5KG-2]
% 	\immini{
% 		Cho hàm số $y=f(x)$ xác định trên $\mathscr{D}=\mathbb{R}$ và có đồ thị như hình vẽ bên. Xét tính liên tục của hàm số $y=f(x)$ trên $\mathscr{D}$.	
% 	}
% 	{
% 		\begin{tikzpicture}[scale=1, font=\normalsize, line join=round, line cap=round,>=stealth]
% 			\def\xmin{-2} \def\xmax{3.5}
% 			\def\ymin{-.5} \def\ymax{2.5}
% 			\draw[->] (\xmin,0)--(0,0)node [below right]{$O$}-- (\xmax,0) node [below]{$x$};
% 			\draw[->] (0,\ymin)--(0,\ymax) node [left]{$y$};
% 			\clip (\xmin+0.1,\ymin+0.1) rectangle (\xmax-0.1,\ymax-0.1);
% 			\draw[smooth,samples=100,thick ] plot[domain=-1.5:1.5](\x,{(\x)^2})--(3,-.25) 
% 			(-1,1) node [left]{$y=f(x)$}; 
% 			\draw[dashed] (1.5,2.25)--(1.5,0) node [below]{$2$} ;
% 		\end{tikzpicture}	
% 	}
	
% 	% \dapso{Hàm số đã cho liên tục trên $\mathbb{R}$.}
% 	\loigiai{
% 		Do đồ thị hàm số là một đường liền nét trên $\mathscr{D}$ nên hàm số $y=f(x)$ liên tục trên $\mathscr{D}=\mathbb{R}$.}
% \end{bt}
% \begin{bt}%[DCHT Toán 11 - KNTT -Vũ Hồng Toàn]%[1K5KG-2]
% 	\immini{
% 		Cho hàm số $y=f(x)$ có đồ thị như hình vẽ bên.\\ Xét tính liên tục của hàm số $y=f(x)$ tại điểm $x_0=2$.	
% 	}
% 	{
% 		\begin{tikzpicture}[scale=1, font=\normalsize, line join=round, line cap=round,>=stealth]
% 			\def\xmin{-2} \def\xmax{3.5}
% 			\def\ymin{-.5} \def\ymax{2.5}
% 			\draw[->] (\xmin,0)--(0,0)node [below right]{$O$}-- (\xmax,0) node [below]{$x$};
% 			\draw[->] (0,\ymin)--(0,\ymax) node [left]{$y$};
% 			\clip (\xmin+0.1,\ymin+0.1) rectangle (\xmax-0.1,\ymax-0.1);
% 			\draw[smooth,samples=100,thick ] plot[domain=-1.5:1.5](\x,{(\x)^2}) (1.5,1.5)--(3,-.25) 
% 			(-1,1) node [left]{$y=f(x)$}; 
% 			\draw[dashed] (1.5,2.25)--(1.5,0) node [below]{$2$} (1.5,1.5)--(0,1.5)node [left]{$2$};
% 			\foreach \x/\y in{0/0,1.5/0,1.5/1.5,0/1.5,1.5/2.25} \fill(\x,\y)circle(.03);
% 		\end{tikzpicture}	
% 	}
% 	% \dapso{Hàm số đã cho gián đoạn tại $x_0=2$. }
% 	\loigiai{
% 		Ta có $\lim\limits_{x\to{2}^-}f(x)>2$ và $\lim\limits_{x\to{2}^+}f(x)=2$. Do đó $\lim\limits_{x\to{2}^-}f(x)\ne \lim\limits_{x\to{2}^+}f(x)$. Vậy hàm số đã cho gián đoạn tại $x_0=2$.
% 	}
% \end{bt}

% \begin{bt}%[DCHT Toán 11 - KNTT -Vũ Hồng Toàn]%[1K5KG-2]
% 	\immini{
% 		Cho hàm số $y=f(x)$ có đồ thị như hình vẽ bên.\\ Xét tính liên tục của hàm số $y=f(x)$ tại điểm $x_0=1$.	
% 	}
% 	{
% 		\begin{tikzpicture}[scale=.5, font=\footnotesize, line join=round, line cap=round,>=stealth]
% 			\def\a{0} \def\b{1} \def\c{1} \def\d{-1}
% 			\pgfmathsetmacro\tcd{int(round(-\d/\c))} \pgfmathsetmacro\tcn{int(round(\a/\c))} 
% 			\pgfmathsetmacro\xmin{\tcd-2.5} \pgfmathsetmacro\xmax{\tcd+2.5}
% 			\pgfmathsetmacro\ymin{\tcn-2.5} \pgfmathsetmacro\ymax{\tcn+2.5}
% 			\draw[->] (\xmin,0)--(0,0)node [below right]{$O$}-- (\xmax,0) node [below]{$x$};
% 			\draw[->] (0,\ymin)--(0,\ymax) node [left]{$y$};
% 			\draw[dash pattern=on 2pt off 1.5pt] (\tcd,\ymin)--(\tcd,\ymax) (\tcd,0)node [below right]{$\tcd$};
% 			\begin{scope}
% 				\clip (\xmin,\ymin) rectangle (\xmax-.1,\ymax-.1);
% 				\draw[samples=200,smooth,variable=\x,thick, teal] plot[domain=\xmin:\tcd-0.3] (\x,{(\a*(\x)+\b)/(\c*(\x)+\d)});
% 				\draw[samples=200,smooth,variable=\x,thick, teal] plot[domain=\tcd+.3:\xmax] (\x,{(\a*(\x)+\b)/(\c*(\x)+\d)});
% 			\end{scope}
% 			\draw[ thick] (1.2,-1.5)node[right]{$y=f(x)$};
% 			\foreach \x/\y in{0/0,1/0} \fill(\x,\y)circle(.03);
% 		\end{tikzpicture}	
% 	}
% 	% \dapso{Hàm số đã cho gián đoạn tại $x_0=1$. }
% 	\loigiai{
% 		Ta có $\lim\limits_{x\to{1}^-}f(x)=-\infty$ và $\lim\limits_{x\to{1}^+}f(x)=+\infty$. Do đó $\lim\limits_{x\to{1}^-}f(x)\ne \lim\limits_{x\to{1}^+}f(x)$.\\ Vậy hàm số đã cho gián đoạn tại $x_0=1$.
% 	}
% \end{bt}

\subsubsection{Câu hỏi trắc nghiệm}
\Opensolutionfile{ans}[ans/ans-1K5-3-Dang1]

\begin{ex}%[DCHT Toán 11 - KNTT -Vũ Hồng Toàn]%[1K5YG-2]
	\immini{
		Cho đồ thị hàm số $y=f(x)$ có đồ thị như hình vẽ bên. Chọn mệnh đề đúng trong các mệnh đề sau:
		\choice[1]
		{Hàm số $y=f(x)$ liên tục trên $\mathbb{R}$}
		{\True Hàm số $y=f(x)$ liên tục trên khoảng $(0,+\infty)$}
		{Hàm số $y=f(x)$ liên tục tại điểm $x_0=0$}
		{$\lim\limits_{x\to{0}^+}f(x)=+\infty$}	
	}
	{
		\begin{tikzpicture}[scale=.75, font=\footnotesize, line join=round, line cap=round,>=stealth]
			\def\xmin{-.5} \def\xmax{3.5}
			\def\ymin{-2.5} \def\ymax{1.5}
			\draw[->] (\xmin,0)--(0,0)node [below left]{$O$}-- (\xmax,0) node [below]{$x$};
			\draw[->] (0,\ymin)--(0,\ymax) node [left]{$y$};
			\clip (\xmin+0.1,\ymin+0.1) rectangle (\xmax-0.1,\ymax-0.1);
			\draw[smooth,samples=100,thick ] plot[domain=.1:3](\x,{ln(\x)}) 
			(1,-1) node [right]{$y=f(x)$}; 
		\end{tikzpicture}	
	}	
	\loigiai
	{
		Đồ thị hàm số là một đường liền nét trên khoảng $(0,+\infty)$ nên liên tục trên khoảng này.
	}
\end{ex}
%Cau2
\begin{ex}%[DCHT Toán 11 - KNTT -Vũ Hồng Toàn]%[1K5YG-2]
	\immini{
		Cho đồ thị hàm số $y=f(x)$ có đồ thị như hình vẽ bên. Chọn mệnh đề \textbf{sai} trong các mệnh đề sau:
		\choice[1]
		{\True $\lim\limits_{x\to{+\infty}}f(x)=-\infty$}
		{Hàm số $y=f(x)$ liên tục trên $\mathbb{R}$}
		{$\lim\limits_{x\to{+\infty}}f(x)=+\infty$}
		{$\lim\limits_{x\to{-\infty}}f(x)=-\infty$}	
	}
	{
		\begin{tikzpicture}[scale=.75, font=\normalsize, line join=round, line cap=round,>=stealth]
			\def\a{1} \def\b{0} \def\c{-3} \def\d{1}
			\def\f(#1){\a*(#1)^3+\b*(#1)^2+\c*(#1)+\d}
			\pgfmathsetmacro\tdx{int(round(-\b/(3*\a)))} \pgfmathsetmacro\tdy{int(round(\f(\tdx))}
			\pgfmathsetmacro\xmin{\tdx-2.5} \pgfmathsetmacro\xmax{\tdx+2.5}
			\pgfmathsetmacro\ymin{\tdy-2.5} \pgfmathsetmacro\ymax{\tdy+2.5}
			%\draw[color=gray!50,dashed] (\xmin,\ymin) grid (\xmax,\ymax);
			\draw[->] (\xmin,0)--(0,0)node [above left]{$O$}-- (\xmax,0) node [below]{$x$};
			\draw[->] (0,\ymin)--(0,\ymax) node [left]{$y$};
			\begin{scope}
				\clip (\xmin,\ymin) rectangle (\xmax-.1,\ymax-.1);
				\draw[samples=100,smooth,thick, teal] plot(\x,{\f(\x)});
			\end{scope}
			\draw[ thick] (-2,-1)node[right]{$y=f(x)$};
			\foreach \x/\y in{0/0,1/0} \fill(\x,\y)circle(.03);
		\end{tikzpicture} 	
	}
	\loigiai
	{
		Quan sát đồ thị hàm số, ta có
		\begin{itemize}
			\item Hàm số $y=f(x)$ liên tục trên $\mathbb{R}$.
			\item $\lim\limits_{x\to{+\infty}}f(x)=+\infty$.
			\item $\lim\limits_{x\to{-\infty}}f(x)=-\infty$.
		\end{itemize}
	}
\end{ex}
%Cau3
\begin{ex}%[DCHT Toán 11 - KNTT -Vũ Hồng Toàn]%[1K5YG-2]
	\immini{
		Cho đồ thị hàm số $y=f(x)$ có đồ thị như hình vẽ bên. Chọn mệnh đề đúng trong các mệnh đề sau:
		\choice
		{$\lim\limits_{x\to{+\infty}}f(x)=-\infty$}
		{\True Hàm số $y=f(x)$ liên tục trên khoảng $(0;+\infty)$}
		{Hàm số $y=f(x)$ liên tục tại điểm $x_0=0$}
		{Hàm số $y=f(x)$ liên tục trên $\mathbb{R}$}	
	}
	{
		\begin{tikzpicture}[scale=1, font=\normalsize, line join=round, line cap=round,>=stealth]
			\def\xmin{-.5} \def\xmax{3.5}
			\def\ymin{-.5} \def\ymax{2.5}
			%\draw[color=gray!50,dashed] (\xmin,\ymin) grid (\xmax,\ymax);
			\draw[->] (\xmin,0)--(0,0)node [below right]{$O$}-- (\xmax,0) node [below]{$x$};
			\draw[->] (0,\ymin)--(0,\ymax) node [left]{$y$};
			\clip (\xmin+0.1,\ymin+0.1) rectangle (\xmax-0.1,\ymax-0.1);
			\draw[smooth,samples=100,thick ] plot[domain=0:3.5](\x,{sqrt(\x)}) 
			(1,.8) node [right]{$y=f(x)$}; 
		\end{tikzpicture}	
	}
	\loigiai
	{
		Quan sát đồ thị hàm số, thấy đồ thị hàm số là một đường liền nét trên khoảng $(0;+\infty)$ do đó hàm số đã cho liên tục trên khoảng này.
	}
\end{ex}
%Cau4
\begin{ex}%[DCHT Toán 11 - KNTT -Vũ Hồng Toàn]%[1K5YG-2]
	\immini{
		Cho đồ thị hàm số $y=f(x)$ có đồ thị như hình vẽ bên. Chọn mệnh đề \textbf{sai} trong các mệnh đề sau:
		\choice[1]
		{Hàm số gián đoạn tại điểm $x_0=0$}
		{Hàm số liên tục trên khoảng $(-\infty;0)$}
		{\True Hàm số $y=f(x)$ liên tục trên $\mathbb{R}$}
		{Hàm số liên tục trên khoảng $(0;+\infty)$}	
	}
	{
		\begin{tikzpicture}[scale=.75, font=\footnotesize, line join=round, line cap=round,>=stealth]
			\def\xmin{-2.5} \def\xmax{2.8}
			\def\ymin{-1.5} \def\ymax{2.8}
			%\draw[color=gray!50,dashed] (\xmin,\ymin) grid (\xmax,\ymax);
			\draw[->] (\xmin,0)--(0,0)node [below right]{$O$}-- (\xmax,0) node [below]{$x$};
			\draw[->] (0,\ymin)--(0,\ymax) node [left]{$y$};
			\begin{scope}
				\clip (\xmin,\ymin) rectangle (\xmax,\ymax-.15);
				\draw[samples=200,smooth,variable=\x,thick, teal] plot[domain=\xmin:-0.5] (\x,{1/(\x)^2-1});
				\draw[samples=200,smooth,variable=\x,thick, teal] plot[domain=0.5:\xmax] (\x,{1/(\x)^2-1});
			\end{scope}
			\draw[dashed](\xmin,-1)--(\xmax,-1) (0,-1)node[below right]{$-1$};
			\draw[ thick] (1.2,1)node[right]{$y=f(x)$};
			\foreach \x/\y in{0/0} \fill(\x,\y)circle(.03);
		\end{tikzpicture}	
	}
	
	\loigiai
	{
		Quan sát đồ thị hàm số, ta có
		\begin{itemize}
			\item Hàm số gián đoạn tại điểm $x_0=0$.
			\item Hàm số liên tục trên khoảng $(-\infty;0)$.
			\item Hàm số liên tục trên khoảng $(0;+\infty)$.
		\end{itemize}
	}
\end{ex}
%Cau5
\begin{ex}%[DCHT Toán 11 - KNTT -Vũ Hồng Toàn]%[1K5BG-2]
	\immini{
		Cho đồ thị hàm số $y=f(x)$ có đồ thị như hình vẽ bên. Chọn mệnh đề \textbf{sai} trong các mệnh đề sau:
		\choice[1]
		{Hàm số gián đoạn tại điểm $x_0=1$}
		{Hàm số liên tục trên khoảng $(-\infty;1)$}
		{\True Hàm số $y=f(x)$ liên tục trên $\mathbb{R}$}
		{Hàm số liên tục trên khoảng $(1;+\infty)$}		
	}
	{
		\begin{tikzpicture}[scale=.75, font=\footnotesize, line join=round, line cap=round,>=stealth]
			\def\a{1} \def\b{-2} \def\c{1} \def\d{-1}
			\pgfmathsetmacro\tcd{int(round(-\d/\c))} \pgfmathsetmacro\tcn{int(round(\a/\c))} 
			\pgfmathsetmacro\xmin{\tcd-2.5} \pgfmathsetmacro\xmax{\tcd+2.5}
			\pgfmathsetmacro\ymin{\tcn-2.5} \pgfmathsetmacro\ymax{\tcn+2.5}
			%\draw[color=gray!50,dashed] (\xmin,\ymin) grid (\xmax,\ymax);
			\draw[->] (\xmin,0)--(0,0)node [below right]{$O$}-- (\xmax,0) node [below]{$x$};
			\draw[->] (0,\ymin)--(0,\ymax) node [left]{$y$};
			\draw[dash pattern=on 2pt off 1.5pt] (\tcd,\ymin)--(\tcd,\ymax) (\tcd,0)node [below right]{$\tcd$}
			(\xmin,\tcn)--(\xmax,\tcn) (0,\tcn)node [below right]{$\tcn$};
			\begin{scope}
				\clip (\xmin,\ymin) rectangle (\xmax-.1,\ymax-.1);
				\draw[samples=200,smooth,variable=\x,thick, teal] plot[domain=\xmin:\tcd-0.3] (\x,{(\a*(\x)+\b)/(\c*(\x)+\d)});
				\draw[samples=200,smooth,variable=\x,thick, teal] plot[domain=\tcd+.3:\xmax] (\x,{(\a*(\x)+\b)/(\c*(\x)+\d)});
			\end{scope}
			\draw[ thick] (1.5,-1)node[right]{$y=f(x)$};
			\foreach \x/\y in{0/0,1/0} \fill(\x,\y)circle(.03);
		\end{tikzpicture}	
	}
	\loigiai
	{
		Quan sát đồ thị hàm số, ta có
		\begin{itemize}
			\item Hàm số gián đoạn tại điểm $x_0=1$.
			\item Hàm số liên tục trên khoảng $(-\infty;1)$.
			\item Hàm số liên tục trên khoảng $(1;+\infty)$.
		\end{itemize}
	}
\end{ex}
%Cau6
\begin{ex}%[DCHT Toán 11 - KNTT -Vũ Hồng Toàn]%[1K5BG-2]
	\immini{
		Cho đồ thị hàm số $y=f(x)$ có đồ thị như hình vẽ bên. Chọn mệnh đề đúng trong các mệnh đề sau:
		\choice[1]
		{Hàm số $y=f(x)$ liên tục trên $\mathbb{R}$}
		{$\lim\limits_{x\to{+\infty}}f(x)=-\infty$}
		{Hàm số liên tục tại điểm $x_0=-2$}
		{\True Hàm số gián đoạn tại điểm $x_0=-2$}	
	}
	{
		\begin{tikzpicture}[scale=1, font=\normalsize, line join=round, line cap=round,>=stealth]
			\def\xmin{-3.3} \def\xmax{2}
			\def\ymin{-.5} \def\ymax{2.8}
			\draw[->] (\xmin,0)--(0,0)node [below right]{$O$}-- (\xmax,0) node [below]{$x$};
			\draw[->] (0,\ymin)--(0,\ymax) node [left]{$y$} ;
			\clip (\xmin+0.1,\ymin+0.1) rectangle (\xmax-0.1,\ymax-0.1);
			\draw[smooth,samples=100,thick ] plot[domain=-1.5:2](\x,{(\x)^2})  
			(-1,1) node [right]{$y=f(x)$}; 
			\draw[thick ] (-3,-.25)--(-1.5,2);
			\draw[dashed] (-1.5,0)node [below]{$-2$}|-(0,2.25) node [right]{$2$} ;
		\end{tikzpicture}	
	} 
	
	\loigiai
	{
		Quan sát đồ thị hàm số, ta có	
		\begin{itemize}
			\item Đồ thị hàm số là các đường liền nét trên các khoảng $(-\infty;-2)$, $(-2;+\infty)$ do đó hàm số liên tục trên các khoảng này.
			\item $\lim\limits_{x\to{+\infty}}f(x)=+\infty$
			\item Ta có $\lim\limits_{x\to{(-2)}^-}f(x)<f(-2)=2$ và $\lim\limits_{x\to{(-2)}^+}f(x)=f(-2)=2$. Do đó $\lim\limits_{x\to{(-2)}^-}f(x)\ne \lim\limits_{x\to{(-2)}^+}f(x)$.\\
			Vậy hàm số đã cho gián đoạn tại $x_0=-2$.
		\end{itemize}
	}
\end{ex}
%Cau7
\begin{ex}%[DCHT Toán 11 - KNTT -Vũ Hồng Toàn]%[1K5BG-2]
	\immini{
		Cho đồ thị hàm số $y=f(x)$ có đồ thị như hình vẽ bên. Chọn mệnh đề đúng trong các mệnh đề sau:
		\choice[1]
		{\True $\lim\limits_{x\to{1}^+}f(x)=+\infty$}
		{Hàm số $y=f(x)$ liên tục trên $\mathbb{R}$}
		{$\lim\limits_{x\to{1}^+}f(x)=-\infty$}
		{$\lim\limits_{x\to{1}^-}f(x)=+\infty$}	
	}
	{
		\begin{tikzpicture}[scale=.75, font=\footnotesize, line join=round, line cap=round,>=stealth]
			\def\a{-1} \def\b{2} \def\c{1} \def\d{-1}
			\pgfmathsetmacro\tcd{int(round(-\d/\c))} \pgfmathsetmacro\tcn{int(round(\a/\c))} 
			\pgfmathsetmacro\xmin{\tcd-2.5} \pgfmathsetmacro\xmax{\tcd+2.5}
			\pgfmathsetmacro\ymin{\tcn-2.5} \pgfmathsetmacro\ymax{\tcn+2.5}
			%\draw[color=gray!50,dashed] (\xmin,\ymin) grid (\xmax,\ymax);
			\draw[->] (\xmin,0)--(0,0)node [below right]{$O$}-- (\xmax,0) node [below]{$x$};
			\draw[->] (0,\ymin)--(0,\ymax) node [left]{$y$};
			\draw[dash pattern=on 2pt off 1.5pt] (\tcd,\ymin)--(\tcd,\ymax) (\tcd,0)node [below right]{$\tcd$}
			(\xmin,\tcn)--(\xmax,\tcn) (0,\tcn)node [below left]{$\tcn$};
			\begin{scope}
				\clip (\xmin,\ymin) rectangle (\xmax-.1,\ymax-.1);
				\draw[samples=200,smooth,variable=\x,thick, teal] plot[domain=\xmin:\tcd-0.3] (\x,{(\a*(\x)+\b)/(\c*(\x)+\d)});
				\draw[samples=200,smooth,variable=\x,thick, teal] plot[domain=\tcd+.3:\xmax] (\x,{(\a*(\x)+\b)/(\c*(\x)+\d)});
			\end{scope}
			\draw[ thick] (1.5,1)node[right]{$y=f(x)$};
			\foreach \x/\y in{0/0,1/0} \fill(\x,\y)circle(.03);
		\end{tikzpicture}	
	}
	
	\loigiai
	{
		Quan sát đồ thị hàm số, ta có	
		\begin{itemize}
			\item Đồ thị hàm số là các đường liền nét trên các khoảng $(-\infty;1)$, $(1;+\infty)$ do đó hàm số liên tục trên các khoảng này.
			\item $\lim\limits_{x\to{+\infty}}f(x)=-1$.
			\item Ta có $\lim\limits_{x\to{1}^-}f(x)=-\infty$ và $\lim\limits_{x\to{1}^+}f(x)=+\infty$. Do đó $\lim\limits_{x\to{1}^-}f(x)\ne \lim\limits_{x\to{1}^+}f(x)$.\\
			Vậy hàm số đã cho gián đoạn tại $x_0=1$.
		\end{itemize}
	}
\end{ex}
%Cau8
\begin{ex}%[DCHT Toán 11 - KNTT -Vũ Hồng Toàn]%[1K5KG-2]
	\immini{
		Cho đồ thị hàm số $y=f(x)$ có đồ thị như hình vẽ bên. Chọn mệnh đề đúng trong các mệnh đề sau:
		\choice[1]
		{Hàm số $y=f(x)$ gián đoạn tại $x_0=0$}
		{$\lim\limits_{x\to{+\infty}}f(x)=-1$}
		{$\lim\limits_{x\to{-\infty}}f(x)=-1$}
		{\True Hàm số $y=f(x)$ liên tục trên $\mathbb{R}$}	
	}
	{
		\begin{tikzpicture}[scale=1, font=\normalsize, line join=round, line cap=round,>=stealth]
			%	\def\a{2} %\def\b{-2} \def\c{0} %\def\d{1} % Hệ số
			\def\xmin{-1.5} \def\xmax{2.5}
			\def\ymin{-1.3} \def\ymax{2.5}
			%\draw[color=gray!50,dashed] (\xmin,\ymin) grid (\xmax,\ymax);
			\draw[->] (\xmin,0)--(0,0)node [below right]{$O$}-- (\xmax,0) node [below]{$x$};
			\draw[->] (0,\ymin)--(0,\ymax) node [left]{$y$};
			\clip (\xmin+0.1,\ymin+0.1) rectangle (\xmax-0.1,\ymax-0.1);
			\draw[thick ] (-1.5,-1.2)--(-.5,1)--(0,0)--(1,2)--(2,-1.2) 
			(1.7,-1) node [left]{$y=f(x)$}; 
		\end{tikzpicture}	
	}
	
	\loigiai
	{
		Quan sát đồ thị hàm số, ta có	
		\begin{itemize}
			\item Đồ thị hàm số là các đường liền nét trên khoảng $(-\infty;+\infty)$ do đó hàm số liên tục trên $\mathbb{R}$.
			\item $\lim\limits_{x\to{+\infty}}f(x)=-\infty$.
			\item $\lim\limits_{x\to{-\infty}}f(x)=-\infty$.
		\end{itemize}
	}
\end{ex}
%Cau9
\begin{ex}%[DCHT Toán 11 - KNTT -Vũ Hồng Toàn]%[1K5KG-2]
	\immini{
		Cho đồ thị hàm số $y=f(x)$ có đồ thị như hình vẽ bên. Chọn mệnh đề đúng trong các mệnh đề sau:
		\choice[1]
		{Hàm số $y=f(x)$ liên tục trên $\mathbb{R}$}
		{Hàm số $y=f(x)$ liên tục tại $x_0=-3$}
		{$\lim\limits_{x\to{-\infty}}f(x)=-\infty$}
		{\True Hàm số $y=f(x)$ gián đoạn tại $x_0=3$}	
	}
	{
		\begin{tikzpicture}[scale=.75, font=\normalsize, line join=round, line cap=round,>=stealth]
			\def\xmin{-5.5} \def\xmax{5.8}
			\def\ymin{-2.5} \def\ymax{2.8}
			%\draw[color=gray!50,dashed] (\xmin,\ymin) grid (\xmax,\ymax);
			\draw[->] (\xmin,0)--(0,0)node [above right]{$O$}-- (\xmax,0) node [below]{$x$};
			\draw[->] (0,\ymin)--(0,\ymax) node [left]{$y$};
			\draw[dash pattern=on 2pt off 1.5pt] (-3,\ymin)--(-3,\ymax) (3,\ymin)--(3,\ymax)
			(-3,0)node [below right]{$-3$} (3,0)node [below right]{$3$}
			;
			\begin{scope}
				\clip (\xmin,\ymin) rectangle (\xmax,\ymax-.15);
				\draw[samples=200,smooth,variable=\x,thick, teal] plot[domain=\xmin:-3.01] (\x,{(\x+2)/((\x)^2-9)});
				\draw[samples=200,smooth,variable=\x,thick, teal] plot[domain=-2.99:2.99] (\x,{(\x+2)/((\x)^2-9)});
				\draw[samples=200,smooth,variable=\x,thick, teal] plot[domain=3.01:\xmax] (\x,{(\x+2)/((\x)^2-9)});
			\end{scope}
			\draw[ thick] (-2.8,1)node[right]{$y=f(x)$};
			\foreach \x/\y in{0/0,-3/0,3/0} \fill(\x,\y)circle(.05);
		\end{tikzpicture} 	
	}
	\loigiai
	{
		Quan sát đồ thị hàm số, ta có	
		\begin{itemize}
			\item Đồ thị hàm số là các đường liền nét trên khoảng $(-\infty;-3)$, $(-3;3)$, $(3;+\infty) $ do đó hàm số liên tục trên các khoảng này.
			\item $\lim\limits_{x\to{-\infty}}f(x)=0$.
			\item Hàm số $y=f(x)$ gián đoạn tại các điểm $x_0=3$ và $x_0=-3$.
		\end{itemize}
	}
\end{ex}
%Cau10
\begin{ex}%[DCHT Toán 11 - KNTT -Vũ Hồng Toàn]%[1K5GG-2]
	\immini{
		Cho đồ thị hàm số $y=f(x)$ có đồ thị như hình vẽ bên. Chọn mệnh đề đúng trong các mệnh đề sau:
		\choice[1]
		{Hàm số $y=f(x)$ liên tục trên $\mathbb{R}$}
		{Hàm số $y=f(x)$ liên tục tại điểm $x_0=\dfrac{\pi}{2}$}
		{Hàm số $y=f(x)$ liên tục tại điểm $x_0=-\dfrac{\pi}{2}$}
		{\True Hàm số $y=f(x)$ gián đoạn tại điểm $x_0=\dfrac{\pi}{2}$}
	}
	{
		\begin{tikzpicture}[line join=round, line cap=round,thick,>=stealth,scale=.75 ]
			\draw[->](-1.6*pi,0)--(1.8*pi,0) node[below]{$x$};
			\draw[->](0,-3)--(0,3) node[left]{$y$};
			\draw (0,0) node[above left]{$O$};
			\foreach \i in {-1,0,1}{
				\pgfmathsetmacro{\start}{\i*pi-1.25}
				\pgfmathsetmacro{\left}{(\i-0.5)*pi}
				\pgfmathsetmacro{\end}{\i*pi+1.25}
				\draw[dashed, thin](\left,-3)--(\left,3);
				\draw[domain=\start:\end,samples=100,smooth ] plot(\x,{tan(\x r)}) (-3.7,-1.5) node[right]{$y=f(x)$};
			}
			\draw[dashed,thin] (1.5*pi,-3)--(1.5*pi,3);
			\draw (-pi/2,0) node [below left]{$-\pi/2$} (pi/2,0) node [below left]{$\pi/2$};
		\end{tikzpicture}	
	}
	\loigiai
	{
		Quan sát đồ thị hàm số, ta có	
		\begin{itemize}
			\item Đồ thị hàm số là các đường liền nét trên khoảng $(-\dfrac{\pi}{2};0)$, $(0;\dfrac{\pi}{2}) $ do đó hàm số liên tục trên các khoảng này.
			\item Hàm số $y=f(x)$ gián đoạn tại các điểm $x_0=\dfrac{\pi}{2}$ và $x_0=-\dfrac{\pi}{2}$.
		\end{itemize}
	}
\end{ex}
\Closesolutionfile{ans}
% \begin{indapan}{10}
% 	{ans/ans-1K5-3-Dang1}
% \end{indapan}

\begin{dang}{Hàm số liên tục tại một điểm}
	Để kiểm tra tính liên tục của hàm số $y=f(x)$ tại điểm $x=x_0$ ta cần làm như sau:
	\begin{itemize}
		\item Bước 1: Tính $\lim \limits_{x \to x_0} f(x)$.
		\item Bước 2: Tính $f(x_0)$.
		Nếu $\lim \limits_{x \to x_0} f(x) = f(x_0)$ thì kết luận hàm số $f(x)$ liên tục tại $x=x_0$.
		Nếu $\lim \limits_{x \to x_0} f(x) \ne f(x_0)$ thì kết luận hàm số $f(x)$ liên tục tại $x=x_0$.
	\end{itemize}
\end{dang}
\subsubsection{Ví dụ mẫu}
\begin{vd}%[DCHT Toán 11 - KNTT -Hứa Chí Ninh]%[1K5BG-3]
	Xét tính liên tục của hàm số $f(x)=\heva{&\dfrac{x^2-3x+2}{x-2}&\text{khi }x\neq 2\\&4x-7&\text{khi }x=2}$ tại điểm $x_0=2$.
	\loigiai{
		Ta có $f(x_0)=f(2)=4\cdot 2-7=1$.\\
		$\lim\limits_{x\rightarrow2}f(x)=\lim\limits_{x\rightarrow2}\dfrac{x^2-3x+2}{x-2}=\lim\limits_{x\rightarrow2}\dfrac{(x-1)(x-2)}{x-2}=\lim\limits_{x\rightarrow2}(x-1)=1$.\\
		Suy ra $f(2)=\lim\limits_{x\rightarrow 2}f(x)$ nên hàm số $f(x)$ liên tục tại điểm $x_0=2$.
	}
\end{vd}

\begin{vd}%[DCHT Toán 11 - KNTT -Hứa Chí Ninh]%[1K5BG-3]
	Xét tính liên tục của hàm số $f(x)=\begin{cases}
		\dfrac{1-\sqrt{2x-3}}{2-x}&\text{ nếu }\:x\ne2\\
		1&\text{ nếu }\:x=2
	\end{cases}$
	tại điểm $x_0=2$.
	\loigiai{Ta có
		\begin{itemize}
			\item $f(2)=1$.
			\item $\lim\limits_{x\to 2}f(x)=\lim\limits_{x\to 2}\dfrac{1-\sqrt{2x-3}}{2-x}=\lim\limits_{x\to 2}\dfrac{1-(2x-3)}{(2-x)(1+\sqrt{2x-3})}=\lim\limits_{x\to 2}\dfrac{2(2-x)}{(2-x)(1+\sqrt{2x-3})}\\
			=\lim\limits_{x\to 2}\dfrac{2}{1+\sqrt{2x-3}}=1=f(2)$
		\end{itemize}
		Vậy hàm số $f(x)$ liên tục tại $x_0=2$.}
\end{vd}

\begin{vd}%[DCHT Toán 11 - KNTT -Hứa Chí Ninh]%[1K5BG-3]
	Biết rằng $\lim\limits_{x\to 0} \dfrac{\sin x}{x}=1$. Hàm số $f\left( x \right)=\heva{&\dfrac{\tan x}{x} & \text{ khi }x\ne 0 \\&0 & \text{ khi }x=0}$. Xét tính liên tục của $y=f(x)$ tại $x=0$?
	 % \dapso{ $f\left( x \right)$ không liên tục tại $x=0$.}
	\loigiai{	Tập xác định $\mathscr{D}=\mathbb{R}\setminus \left\{ \dfrac{\pi }{2}+k\pi |k\in \mathbb{Z} \right\}$.\\
		Ta có $\lim\limits_{x\to 0} f\left( x \right)=\lim\limits_{x\to 0} \dfrac{\tan x}{x}=\lim\limits_{x\to 0} \dfrac{\sin x}{x}\cdot\dfrac{1}{\cos x}=1\cdot\dfrac{1}{\cos 0}=1\ne f\left( 0 \right)\Rightarrow $ $f\left( x \right)$ không liên tục tại $x=0$.}
\end{vd}
\begin{vd}%[DCHT Toán 11 - KNTT -Hứa Chí Ninh]%[1K5BG-3]
	Hàm số $f\left( x \right)=\left\{ \begin{array}{*{35}{l}}
		3 & \text{khi}\,\,x=-1 \\
		\dfrac{x^4+x}{x^2+x} & \text{khi}\,\,x\ne -1,\,\,x\ne 0 \\
		1 & \text{khi}\,\,x=0 \\
	\end{array} \right.$. Xét tính liên tục của hàm số tại $x=-1, x=0.$
	% \dapso{Hàm số liên tục tại $x=-1, x=0.$}
	\loigiai{
		Hàm số $y=f\left( x \right)$ có tập xác định $\mathscr{D}=\mathbb{R}$.\\
		Dễ thấy hàm số $y=f\left( x \right)$ liên tục trên mỗi khoảng $\left( -\infty ;-1 \right),\left( -1;0 \right)$ và $\left( 0;+\infty \right)$.\\
		(i) Xét tại $x=-1$, ta có\\
		$\lim\limits_{x\to -1} f\left( x \right)=\lim\limits_{x\to -1} \dfrac{x^4+x}{x^2+x}=\lim\limits_{x\to -1} \dfrac{x\left( x+1 \right)\left( {x^2}-x+1 \right)}{x\left( x+1 \right)}=\lim\limits_{x\to -1} \left( {x^2}-x+1 \right)=3=f\left( -1 \right).$ Vậy hàm số $y=f\left( x \right)$ liên tục tại $x=-1$.\\
		(ii) Xét tại $x=0$, ta có\\
		$\lim\limits_{x\to 0} f\left( x \right)=\lim\limits_{x\to 0} \dfrac{x^4+x}{x^2+x}=\lim\limits_{x\to 0} \dfrac{x\left( x+1 \right)\left( {x^2}-x+1 \right)}{x\left( x+1 \right)}=\lim\limits_{x\to 0} \left( {x^2}-x+1 \right)=1=f\left( 0 \right).$ Vậy hàm số $y=f\left( x \right)$ liên tục tại $x=0$.
	}
\end{vd}

\begin{vd}%[DCHT Toán 11 - KNTT -Hứa Chí Ninh]%[1K5BG-3]
	Xét tính liên tục của hàm số $f(x)=\begin{cases}
		x^2+1&\text{ nếu }\:x>0\\
		x&\text{ nếu }\:x\leq0
	\end{cases}$ tại điểm $x_0=0$.
	\loigiai{Ta có:
		\begin{itemize}
			\item $f(0)=0$.
			\item $\lim\limits_{x\to0^{+}}f(x)=\lim\limits_{x\to0^{+}}(x^2+1)=1$.
			\item $\lim\limits_{x\to0^{-}}f(x)=\lim\limits_{x\to0^{-}}x=0$.
		\end{itemize}
		Ta có: $f(0)=\lim\limits_{x\to0^{-}}f(x)\ne\lim\limits_{x\to0^{+}}f(x)$. Vậy hàm số $f(x)$ gián đoạn tại điểm $x=0$.
	}
\end{vd}

\begin{vd}%[DCHT Toán 11 - KNTT -Hứa Chí Ninh]%[1K5BG-3]
	Xét tính liên tục của hàm số $f\left( x \right)=\left\{ \begin{array}{*{35}{l}}
		1-\cos x & \text{khi }x\le 0 \\
		\sqrt{x+1} & \text{khi }x>0 \\
	\end{array} \right.$ tại $x=0?$
	% \dapso{Hàm số $y=f\left( x \right)$ gián đoạn tại $x=0$.}
	\loigiai{
		Hàm số xác định với mọi $x\in \mathbb{R}$.\\
		Ta có $f\left( x \right)$ liên tục trên $\left( -\infty ;0 \right)$ và $\left( 0;+\infty \right).$\\
		Mặt khác $\left\{ \begin{aligned}
			& f\left( 0 \right)=1 \\
			& \lim\limits_{x\to {0^{-}}} f\left( x \right)=\lim\limits_{x\to {0^{-}}} \left( 1-\cos x \right)=1-\cos 0=0 \\
			& \lim\limits_{x\to {0^{+}}} f\left( x \right)=\lim\limits_{x\to {0^{+}}} \sqrt{x+1}=\sqrt{0+1}=1 \\
		\end{aligned} \right.\Rightarrow f\left( x \right)$ gián đoạn tại $x=0$.
	}
\end{vd}

\subsubsection{Bài tập rèn luyện}
% \centerline{\fcolorbox{red}{yellow!50}{\bf {BÀI TẬP TỰ LUẬN}}}
\begin{bt}%[DCHT Toán 11 - KNTT -Hứa Chí Ninh]%[1K5BG-3]
	Xét tính liên tục của hàm số $f(x)=\begin{cases}
		\dfrac{x^2-6x+5}{x^2-1}&\text{ nếu }\: x\ne 1\\
		-2&\text{ nếu }\: x=1
	\end{cases}$ tại điểm $x_0=1$.
	\loigiai{Ta có: $f(1)=-2$.\\
		$\lim\limits_{x\to1}f(x)=\lim\limits_{x\to1}\dfrac{x^2-6x+5}{x^2-1}=\lim\limits_{x\to1}\dfrac{(x-5)(x-1)}{(x-1)(x+1)}=\lim\limits_{x\to1}\dfrac{x-5}{x+1}=-2=f(1)$.\\
		Vậy hàm số $f(x)$ liên tục tại $x=1$.}
\end{bt}
\begin{bt}%[DCHT Toán 11 - KNTT -Hứa Chí Ninh]%[1K5BG-3]
	Xét tính liên tục của hàm số $f(x)=\begin{cases}
		\dfrac{\sqrt{x}-2}{\sqrt{x+5}-3}&\text{ nếu }\:x\ne4\\
		-\dfrac{3}{2}&\text{ nếu }\:x=4
	\end{cases}$ tại điểm $x_0=4$.
	\loigiai{
		Ta có:
		\begin{itemize}
			\item $f(4)=-\dfrac{3}{2}$.
			\item $\lim\limits_{x\to 4}f(x)=\lim\limits_{x\to 4}\dfrac{\sqrt{x}-2}{\sqrt{x+5}-3}=\lim\limits_{x\to 4}\dfrac{(\sqrt{x+5}+3)(x-4)}{(x+5-9)(\sqrt{x}+2)}$\\
			$=\lim\limits_{x\to 4}\dfrac{\sqrt{x+5}+3}{\sqrt{x}+2}=\dfrac{6}{4}=\dfrac{3}{2}\ne f(4)$. 
		\end{itemize}
		Vậy hàm số $f(x)$ gián đoạn tại điểm $x=4$.
	}
\end{bt}

\begin{bt}%[DCHT Toán 11 - KNTT -Hứa Chí Ninh]%[1K5BG-3]
	Cho hàm số $f\left( x \right)=\left\{ \begin{aligned}
		& \dfrac{x^2}{x}\text{ khi }x<1,x\ne 0 \\
		& 0\text{ khi }x=0 \\
		& \sqrt{x}\text{ khi }x\ge 1 \\
	\end{aligned} \right..$ Xét tính liên tục của hàm số $f\left( x \right)$ tại $x=0, x=1?$
	% \dapso{Hàm số $y=f\left( x \right)$ liên tục tại $x=0$ và $x=1$.}
	\loigiai{
		Hàm số $y=f\left( x \right)$ có tập xác định $\mathscr{D}=\mathbb{R}$.\\
		Dễ thấy hàm số $y=f\left( x \right)$ liên tục trên mỗi khoảng $\left( -\infty ;0 \right),\left( 0;1 \right)$ và $\left( 1;+\infty \right)$.\\
		Ta có $\left\{ \begin{aligned}
			& f\left( 0 \right)=0 \\
			& \lim\limits_{x\to {0^{-}}} f\left( x \right)=\lim\limits_{x\to {0^{-}}} \dfrac{x^2}{x}=\lim\limits_{x\to {0^{-}}} x=0 \\
			& \lim\limits_{x\to {0^{+}}} f\left( x \right)=\lim\limits_{x\to {0^{+}}} \dfrac{x^2}{x}=\lim\limits_{x\to {0^{+}}} x=0 \\
		\end{aligned} \right.\Rightarrow f\left( x \right)$ liên tục tại $x=0.$\\
		Ta có $\left\{ \begin{aligned}
			& f\left( 1 \right)=1 \\
			& \lim\limits_{x\to {1^{-}}} f\left( x \right)=\lim\limits_{x\to {1^{-}}} \dfrac{x^2}{x}=\lim\limits_{x\to {1^{-}}} x=1 \\
			& \lim\limits_{x\to {1^{+}}} f\left( x \right)=\lim\limits_{x\to {1^{+}}} \sqrt{x}=1 \\
		\end{aligned} \right.\Rightarrow f\left( x \right)$ liên tục tại $x=1.$
	}
\end{bt}

\begin{bt}%[DCHT Toán 11 - KNTT -Hứa Chí Ninh]%[1K5BG-3]
	Cho hàm số $y=\left\{ \begin{aligned}
		& \dfrac{1-x^3}{1-x},\text{khi  }x<1 \\
		& 1\text{  },\text{khi  }x\ge 1 \\
	\end{aligned} \right.$. Xét tính liên tục phải của hàm số tại $x=1$?
	% \dapso{Hàm số liên tục phải tại $x=1.$}
	\loigiai{
		Ta có   $y\left( 1 \right)=1$.\\
		Ta có   $\lim\limits_{x\to {1^{+}}} y=1$; $\lim\limits_{x\to {1^{-}}} y=\lim\limits_{x\to {1^{-}}} \dfrac{1-x^3}{1-x}=\lim\limits_{x\to {1^{-}}} \dfrac{\left( 1-x \right)\left( 1+x+x^2 \right)}{1-x}=\lim\limits_{x\to {1^{-}}} \left( 1+x+x^2 \right)=4$.\\
		Nhận thấy $\lim\limits_{x\to {1^{+}}} y=y\left( 1 \right)$, suy ra $y$ liên tục phải tại $x=1$.}
\end{bt}
\begin{bt}%[DCHT Toán 11 - KNTT -Hứa Chí Ninh]%[1K5BG-3]
	
	Cho hàm số $y=\left\{ \begin{aligned}
		& \dfrac{x^2-7x+12}{x-3} & \text{ khi } x\ne 3 \\
		& -1 &\text{ khi }x=3 \\
	\end{aligned} \right.$. Xét tính liên tục của hàm số tại $x=3$.
	% \dapso{Hàm số đã cho có đạo hàm tại $x=3$}
	\loigiai{
		$\lim\limits_{x\to 3} \dfrac{x^2-7x+12}{x-3}=\lim\limits_{x\to 3} \left( x-4 \right)=-1=y\left( 3 \right)$ nên hàm số liên tục tại $x_0=3$.}
\end{bt}
\begin{bt}%[DCHT Toán 11 - KNTT -Hứa Chí Ninh]%[1K5KG-3]
	Cho hàm số $f\left( x \right)=\left\{ \begin{aligned}
		& \dfrac{x-2}{\sqrt{x+2}-2}&\text{ khi }x\ne 2 \\
		& 4&\text{ khi }x=2 \\
	\end{aligned} \right.$. Xét tính liên tục của hàm số tại $x=2$?
	% \dapso{Hàm số liên tục tại $x=2.$}
	\loigiai{
		Tập xác định  $\mathscr{D}=\mathbb{R}$.\\
		$\lim\limits_{x\to 2} f\left( x \right)=\lim\limits_{x\to 2} \dfrac{x-2}{\sqrt{x+2}-2}=\lim\limits_{x\to 2} \dfrac{\left( x-2 \right)\left( \sqrt{x+2}+2 \right)}{x-2}=\lim\limits_{x\to 2} \left( \sqrt{x+2}+2 \right)=4$.\\
		$f\left( 2 \right)=4$
		$\Rightarrow \lim\limits_{x\to 2} f\left( x \right)=f\left( 2 \right)$.
		Vậy hàm số liên tục tại $x=2$.}
\end{bt}
\begin{bt}%[DCHT Toán 11 - KNTT -Hứa Chí Ninh]%[1K5KG-3]% 
	Cho hàm số $f\left( x \right)=\left\{ \begin{aligned}
		& \dfrac{1-\cos x}{x^2}& \text{ khi } x\ne 0 \\
		& 1 &\text{ khi } x=0 \\
	\end{aligned} \right.\,\,$. Xét tính liên tục của hàm số tại $x=0?$
	% \dapso{Hàm số gián đoạn tại $x=0.$}
	\loigiai{
		Hàm số xác định trên $\mathbb{R}$.\\
		Ta có $f\left( 0 \right)=1$ và $\lim\limits_{x\to 0} f\left( x \right)=\lim\limits_{x\to 0} \dfrac{1-\cos x}{x^2}=\lim\limits_{x\to 0} \dfrac{2{\sin ^2}\dfrac{x}{2}}{4\cdot{{\left( \dfrac{x}{2} \right)}^2}}=\dfrac{1}{2}.$\\
		Vì $f\left( 0 \right)\ne \lim\limits_{x\to 0} f\left( x \right)$ nên $f\left( x \right)$ gián đoạn tại $x=0$. Do đó $f\left( x \right)$không có đạo hàm tại $x=0$.\\
		Vì $\forall x\ne 0$, $f\left( x \right)=\dfrac{1-\cos x}{x^2}\ge 0$ nên $f\left( \sqrt{2} \right)>0.$ Vậy $f\left( x \right)$ gián đoạn tại $x=0$.}
\end{bt}
\begin{bt}%[DCHT Toán 11 - KNTT -Hứa Chí Ninh]%[1K5BG-4]
	Cho hàm số $f\left( x \right)=\heva{& -x\cos x &\text{ khi } x<0 \\& \dfrac{x^2}{1+x}&\text{ khi } 0\le x<1 \\& {x^3}&\text{ khi } x\ge 1}$. Xét tính liên tục của hàm số tại $x=0?$
	% \dapso{Hàm số gián đoạn tại $x=1.$}
	\loigiai{
		$\bullet$ $f\left( x \right)$ liên tục tại $x\ne 0$ và $x\ne 1$.\\
		$\bullet$ Tại $x=0$\\
		$\lim\limits_{x\to {0^{-}}} f\left( x \right)=\lim\limits_{x\to {0^{-}}} \left( -x\cos x \right)=0$, $\lim\limits_{x\to {0^{+}}} f\left( x \right)=\lim\limits_{x\to {0^{+}}} \dfrac{x^2}{1+x}=0$, $f\left( 0 \right)=0$.\\
		Suy ra $\lim\limits_{x\to {0^{-}}} f\left( x \right)=\lim\limits_{x\to {0^{+}}} f\left( x \right)=f\left( 0 \right)$. Hàm số liên tục tại $x=0$.\\
		$\bullet$ Tại $x=1$\\
		$\lim\limits_{x\to {1^{-}}} f\left( x \right)=\lim\limits_{x\to {1^{-}}} \dfrac{x^2}{1+x}=\dfrac{1}{2}$, $\lim\limits_{x\to {1^{+}}} f\left( x \right)=\lim\limits_{x\to {1^{+}}} x^3=1$.\\
		Suy ra $\lim\limits_{x\to {1^{-}}} f\left( x \right)\ne \lim\limits_{x\to {1^{+}}} f\left( x \right)$. Hàm số gián đoạn tại $x=1$.}
\end{bt}
\subsubsection{Câu hỏi trắc nghiệm}
\Opensolutionfile{ans}[ans/ans-1K5-3-Dang2]
\begin{ex}%[DCHT Toán 11 - KNTT -Hứa Chí Ninh]%[1K5BG-1]
	Cho hàm số $y=f(x)$ xác định trên khoảng $K$ và $x_0\in K$. Hàm số $y=f(x)$ liên tục tại $x_0$ nếu
	\choice
	{$\lim \limits_{x\rightarrow x_0^+} f(x) = f(x_0)$}
	{$\lim \limits_{x\rightarrow x_0^-} f(x) = f(x_0)$ }
	{\True $\lim \limits_{x\rightarrow x_0} f(x)= f(x_0)$ }
	{$\lim \limits_{x\rightarrow x_0} f(x) \neq  f(x_0)$ }
	\loigiai{
		Theo định nghĩa: Cho hàm số $y=f(x)$ xác định trên khoảng $K$ và $x_0\in K$. Hàm số $y=f(x)$ liên tục tại $x_0$ nếu $\lim \limits_{x\rightarrow x_0} f(x)= f(x_0)$.
	}
\end{ex}

\begin{ex}%[DCHT Toán 11 - KNTT -Hứa Chí Ninh]%[1K5BG-3]
	Cho hàm số $f(x) = \dfrac{x^2 + 3x - 4}{x + 4}$ với $x \neq -4$. Để hàm số $f(x)$ liên tục tại $x = -4$ thì ta cần bổ sung giá trị $f(-4)$ bằng bao nhiêu?
	\choice
	{$5$}
	{\True $-5$}
	{$3$}
	{$0$}
	\loigiai{
		$f(-4) = \lim\limits_{x \to -4}\dfrac{x^2 + 3x - 4}{x + 4} = \lim\limits_{x \to -4}\dfrac{(x - 1)(x + 4)}{x + 4} = \lim\limits_{x \to -4}(x - 1) = -5$.
	}
\end{ex}
\begin{ex}%[DCHT Toán 11 - KNTT -Hứa Chí Ninh]%[1K5BG-3]	
	Cho hàm số $f\left( x \right)=\dfrac{2x-1}{x^3-x}$. Kết luận nào sau đây đúng?
	\choice
	{Hàm số liên tục tại $x=-1$}
	{Hàm số liên tục tại $x=0$}
	{Hàm số liên tục tại $x=1$}
	{\True Hàm số liên tục tại $x=\dfrac{1}{2}$}
	\loigiai{
		Tại $x=\dfrac{1}{2}$, ta có   $\lim\limits_{x\to \frac{1}{2}} f\left( x \right)=\lim\limits_{x\to \frac{1}{2}} \dfrac{2x-1}{x^3-1}=0=f\left( \dfrac{1}{2} \right)$. Vậy hàm số liên tục tại $x=2$.}
\end{ex}
\begin{ex}%[DCHT Toán 11 - KNTT -Hứa Chí Ninh]%[1K5BG-3]
	Hàm số nào sau đây liên tục tại $x=1$$\colon$ 
	\choice
	{$f\left( x \right)=\dfrac{x^2+x+1}{x-1}$}
	{$f\left( x \right)=\dfrac{x^2-x-2}{x^2-1}$}
	{\True $f\left( x \right)=\dfrac{x^2+x+1}{x}$}
	{$f\left( x \right)=\dfrac{x+1}{x-1}$}
	\loigiai{
		$\bullet$ $f\left( x \right)=\dfrac{x^2+x+1}{x-1}$\\
		$\lim\limits_{x\to {1^{+}}} f\left( x \right)=+\infty $ suy ra $f\left( x \right)$ không liên tục tại $x=1$.\\
		$\bullet$ $f\left( x \right)=\dfrac{x^2-x-2}{x^2-1}$\\
		$\lim\limits_{x\to {1^{+}}} f\left( x \right)=\lim\limits_{x\to {1^{+}}} \dfrac{x-2}{x-1}=-\infty $ suy ra $f\left( x \right)$ không liên tục tại $x=1$.\\
		$\bullet$ $f\left( x \right)=\dfrac{x^2+x+1}{x}$\\
		$\lim\limits_{x\to 1} f\left( x \right)=\lim\limits_{x\to 1} \dfrac{x^2+x+1}{x}=3=f\left( 1 \right)$ suy ra $f\left( x \right)$ liên tục tại $x=1$.\\
		$\bullet$ $f\left( x \right)=\dfrac{x+1}{x-1}$\\
		$\lim\limits_{x\to {1^{+}}} f\left( x \right)=\lim\limits_{x\to {1^{+}}} \dfrac{x+1}{x-1}=+\infty $ suy ra $f\left( x \right)$ không liên tục tại $x=1$.}
\end{ex}
\begin{ex}%[DCHT Toán 11 - KNTT -Hứa Chí Ninh]%[1K5BG-3]
	Hàm số nào dưới đây gián đoạn tại điểm $x_0=-1$.
	\choice
	{$y=\left( x+1 \right)\left( {x^2}+2 \right)$}
	{\True $y=\dfrac{2x-1}{x+1}$}
	{$y=\dfrac{x}{x-1}$}
	{$y=\dfrac{x+1}{x^2+1}$}
	\loigiai{
		Ta có $y=\dfrac{2x-1}{x+1}$ không xác định tại $x_0=-1$ nên gián đoạn tại $x_0=-1$.}
\end{ex}
\begin{ex}%[DCHT Toán 11 - KNTT -Hứa Chí Ninh]%[1K5BG-3]
	Hàm số nào sau đây gián đoạn tại $x=2$?
	\choice
	{\True $y=\dfrac{3x-4}{x-2}$}
	{$y=\sin x$}
	{$y=x^4-2x^2+1$}
	{$y=\tan x$}
	\loigiai{
		Ta có   $y=\dfrac{3x-4}{x-2}$ có tập xác định $\mathscr{D}=\mathbb{R}\setminus \left\{ 2 \right\}$, do đó gián đoạn tại $x=2$.}
\end{ex}
\begin{ex}%[DCHT Toán 11 - KNTT -Hứa Chí Ninh]%[1K5BG-3]
	Hàm số $y=\dfrac{x}{x+1}$ gián đoạn tại điểm $x_0$ bằng?
	\choice
	{ $x_0=2018$}
	{ $x_0=1$}
	{ $x_0=0$}
	{\True $x_0=-1$}
	\loigiai{
		Vì hàm số $y=\dfrac{x}{x+1}$ có tập xác định $\mathscr{D}=\mathbb{R}\setminus \left\{ -1 \right\}$ nên hàm số gián đoạn tại điểm $x_0=-1$.}
\end{ex}
\begin{ex}%[DCHT Toán 11 - KNTT -Hứa Chí Ninh]%[1K5BG-3]
	Hàm số nào dưới đây gián đoạn tại điểm $x=1$?
	\choice
	{$y=\dfrac{x-1}{x^2+x+1}$}
	{$y=\dfrac{x^2-x+1}{x+1}$}
	{$y=(x-1)(x^2+x+1)$}
	{\True $y=\dfrac{x^2+2}{x-1}$}
	\loigiai
	{
		Ta có $\lim\limits_{x\to1^+}\dfrac{x^2+2}{x-1}=+\infty$ và $\lim\limits_{x\to1^-}\dfrac{x^2+2}{x-1}=-\infty$ nên hàm số $y=\dfrac{x^2+2}{x-1}$ gián đoạn tại điểm $x=1$.
	}
\end{ex}
\begin{ex}%[DCHT Toán 11 - KNTT -Hứa Chí Ninh]%[1K5BG-3]
	Cho hàm số $y=\dfrac{x-3}{x^2-1}$. Mệnh đề nào sau đây đúng?
	\choice
	{\True Hàm số không liên tục tại các điểm $x=\pm 1$}
	{Hàm số liên tục tại mọi $x\in \mathbb{R}$}
	{Hàm số liên tục tại các điểm $x=-1$}
	{Hàm số liên tục tại các điểm $x=1$}
	\loigiai{
		Hàm số $y=\dfrac{x-3}{x^2-1}$ có tập xác định $\mathbb{R}\setminus \left\{ \pm 1 \right\}$. Do đó hàm số không liên tục tại các điểm $x=\pm 1$.}
\end{ex}
\begin{ex}%[DCHT Toán 11 - KNTT -Hứa Chí Ninh]%[1K5KG-3] 
	Cho hàm số $f(x)=\heva{&\dfrac{1-\cos x}{x^2}&\text{ khi } x\ne 0\\&1&\text{ khi } x=0}$.
	Khẳng định nào đúng trong các khẳng định sau?
	\choice
	{Hàm số gián đoạn tại $x=\sqrt{2}$}
	{$f(\sqrt{2})<0$}
	{$f(x)$ liên tục tại $x=0$}	
	{\True $f(x)$ gián đoạn tại $x=0$}
	\loigiai{
		Hàm số xác định trên $\mathbb{R}$.\\
		Ta có $f(0)=1$ và $\lim \limits_{x\to 0} f(x)=\lim \limits_{x\to 0} \dfrac{1-\cos x}{x^2}=\lim \limits_{x\to 0} \dfrac{2\sin ^2\dfrac{x}{2}}{4\cdot \left(\dfrac{x}{2}\right)^2}=\dfrac{1}{2}$.\\
		Vì $f(0)\ne \lim \limits_{x\to 0} f(x)$ nên $f(x)$ gián đoạn tại $x=0$.}
\end{ex}
\Closesolutionfile{ans}
% \begin{indapan}{10}
% 	{ans/ans-1K5-3-Dang2}
% \end{indapan}
\begin{dang}{Hàm số liên tục trên khoảng, đoạn}
	\begin{itemize}
		\item Hàm số $y=f(x)$ được gọi là liên tục trên một khoảng nếu nó liên tục tại mọi điểm của khoảng đó.
		\item Hàm số $y=f(x)$ được gọi là liên tục trên đoạn $[a,b]$ nếu nó liên tục trên khoảng $(a,b)$ và $$\mathop {\lim \limits{n \to +\infty}}\limits_{x \to {a^ + }} f\left( x \right) = f\left( a \right),{\rm{   }}\mathop {\lim \limits{n \to +\infty}}\limits_{x \to {b^ - }} f\left( x \right) = f\left( b \right).$$
		\item Đồ thị của hàm số liên tục trên một khoảng là một đường liền nét trên khoảng đó.
	\end{itemize}
\end{dang}
\subsubsection{Ví dụ mẫu}
\begin{vd}%[DCHT Toán 11 - KNTT -Hứa Chí Ninh]%[1K5KG-4]
	Xét tính liên tục của hàm số sau trên tập xác định của chúng.
	\begin{itemize}
		\item[a)] $f(x)=\heva{&\dfrac{x^2-x-2}{x+1}&\text{ khi } x\ne -1\\ &-3&\text{ khi } x=-1}.$
		\item[b)] $f(x)=\heva{&\dfrac{2x+1}{(x-1)^2}&\text{ khi } x\ne 1\\ &3 &\text{ khi } x=1}.$
	\end{itemize}
	\loigiai{
		\begin{enumerate}
			\item \begin{itemize}
				\item Tập xác định của hàm số là $\mathscr{D}=\mathbb{R}$.
				\item Khi $x \ne -1$, $f(x)=\dfrac{x^2-x-2}{x+1}$ là hàm phân thức hữu tỉ nên liên tục trên $(-\infty;-1)\cup(-1;+\infty)$.
				\item Tại điểm $x=-1$, ta có $f(-1)=-3$.\\
				$\lim\limits_{x\to -1}f(x)=\lim\limits_{x\to -1}\dfrac{x^2-x-2}{x+1}=\lim\limits_{x\to -1}(x-2)=-3=f(-1).$\\
				Do đó hàm số liên tục tại $x=-1$.
				\item Vậy hàm số liên tục trên $\mathbb{R}$.
			\end{itemize}
			\item \begin{itemize}
				\item Tập xác định của hàm số là $\mathscr{D}=\mathbb{R}$.\\
				\item Khi $x \ne 1$, $f(x)=\dfrac{2x+1}{(x-1)^2}$ là hàm phân thức hữu tỉ nên liên tục trên $(-\infty;1)\cup(1;+\infty)$.\\
				\item Tại điểm $x=1$, ta có $f(1)=3$.\\
				$\lim\limits_{x\to 1}f(x)=\lim\limits_{x\to 1}\dfrac{2x+1}{(x-1)^2}=+\infty\ne f(-1).$\\
				Do đó hàm số gián đoạn tại $x=1$.
				\item Vậy hàm số liên tục trên $\mathbb{R}\setminus\{1\}$.
			\end{itemize}
	\end{enumerate}}
\end{vd}

\begin{vd}%[DCHT Toán 11 - KNTT -Hứa Chí Ninh]%[1K5KG-4]
	Xét tính liên tục của hàm số sau trên tập xác định của chúng.
	\begin{itemize}
		\item[a)] $f(x)=\heva{&x^2+3x&\text{ khi } &x\ge 2\\ &6x+1&\text{ khi } &x<2.}$
		\item[b)] $f(x)=\heva{&x^2-3x+5&\text{ khi } &x> 1\\ &3 &\text{ khi } &x=1\\&2x+1 &\text{ khi }&x<1.}$
	\end{itemize}
	\loigiai{
		\begin{enumerate}
			\item \begin{itemize}
				\item Tập xác định của hàm số là $\mathscr{D}=\mathbb{R}$.
				\item Khi $x>2$, $f(x)=x^2+3x$ là hàm đa thức nên liên tục trên $(2;+\infty)$.
				\item Khi $x<2$, $f(x)=6x+1$ là hàm đa thức nên liên tục trên $(-\infty;2)$.
				\item Tại điểm $x=2$, ta có $f(2)=10$.\\
				$\lim\limits_{x\to 2^+}f(x)=\lim\limits_{x\to 2^+}(x^2+3x)=10$ và $\lim\limits_{x\to 2^-}f(x)=\lim\limits_{x\to 2^-}(6x+1)=13$.\\
				Vì không tồn tại $\lim\limits_{x\to 2}f(x)$ nên hàm số gián đoạn tại $x=2$.
				\item Vậy hàm số liên tục trên $\mathbb{R}\setminus\{2\}$.
			\end{itemize}
			\item \begin{itemize}
				\item Tập xác định của hàm số là $\mathscr{D}=\mathbb{R}$.
				\item Khi $x>1$, $f(x)=x^2-3x+5$ là hàm đa thức nên liên tục trên $(1;+\infty)$.
				\item Khi $x<1$, $f(x)=2x+1$ là hàm đa thức nên liên tục trên $(-\infty;1)$.
				\item Tại điểm $x=1$, ta có $f(1)=3$.\\
				$\lim\limits_{x\to 1^+}f(x)=\lim\limits_{x\to 1^+}(x^2-3x+5)=3$ và $\lim\limits_{x\to 1^-}f(x)=\lim\limits_{x\to 1^-}(2x+1)=3$.\\
				Vì $\lim\limits_{x\to 1}f(x)=f(1)$ nên hàm số liên tục tại $x=1$.
				\item Vậy hàm số liên tục trên $\mathbb{R}$.
			\end{itemize}
	\end{enumerate}}
\end{vd}

\subsubsection{Bài tập rèn luyện}
% \centerline{\fcolorbox{red}{yellow!50}{\bf {BÀI TẬP TỰ LUẬN}}}
\begin{bt}%[DCHT Toán 11 - KNTT -Hứa Chí Ninh]%[1K5KG-4]
	Xét tính liên tục của hàm số $f(x)=\heva{&\dfrac{x-2}{x^2-4}&\text { khi } x \neq 2 \\& 1&\text { khi } x=2}$ trên tập xác định.
	\loigiai{
		Tập xác định của hàm số là $\mathscr{D}=\mathbb{R}$.\\
		Khi $x \neq 2, f(x)=\dfrac{x-2}{x^2-4}$ là hàm phân thức hữu tỉ nên liên tục trên $(-\infty ; 2) \cup(2 ;+\infty)$.\\
		Tại điểm $x=2$, ta có $f(2)=1$.\\
		$\lim \limits_{x \to 2} f(x)=\lim \limits_{x \to 2} \dfrac{x-2}{x^2-4}=\lim \limits_{x \to 2} \dfrac{1}{x+2}=\dfrac{1}{4} \neq f(2)$.\\
		Do đó hàm số gián đoạn tại $x=2$.\\
		Vậy hàm số liên tục trên $(-\infty; 2)$ và $(2;+\infty)$.}
\end{bt}
\begin{bt}%[DCHT Toán 11 - KNTT -Hứa Chí Ninh]%[1K5KG-4]
	Xét tính liên tục của hàm số $f(x)=\heva{&\dfrac{x^3-1}{x-1}&\text { khi } x \neq 1 \\& 3&\text { khi } x=1}$ trên tập xác định.
	\loigiai{
		Tập xác định của hàm số là $\mathscr{D}=\mathbb{R}$.\\
		Khi $x \neq 1, f(x)=\dfrac{x^3-1}{x-1}$ là hàm phân thức hữu tỉ nên liên tục trên $(-\infty ; 1) \cup(1 ;+\infty)$.\\
		Tại điểm $x=1$, ta có $f(1)=3$.\\
		$\lim \limits_{x \to 1} f(x)=\lim \limits_{x \to 1} \dfrac{x^3-1}{x-1}=\lim \limits_{x \to 1}(x^2+x+1)=3=f(1)$.\\
		Do đó hàm số liên tục tại $x=1$.\\
		Vậy hàm số liên tục trên $\mathbb{R}$.}
\end{bt}
\begin{bt}%[DCHT Toán 11 - KNTT -Hứa Chí Ninh]%[1K5KG-4]
	Xét tính liên tục của hàm số $f(x)=\heva{&x^2&\text { khi } x \geq-2 \\& 2-x&\text { khi } x<-2}$ trên tập xác định.
	\loigiai{
		Tập xác định của hàm số là $\mathscr{D}=\mathbb{R}$.\\
		Khi $x>-2, f(x)=x^2$ là hàm đa thức nên liên tục trên $(-2 ;+\infty)$.\\
		Khi $x<-2, f(x)=2-x$ là hàm đa thức nên liên tục trên $(-\infty ;-2) $.\\
		Tại điểm $x=-2$, ta có $f(-2)=4$.\\
		$\lim \limits_{x \to(-2)^{+}} f(x)=\lim \limits_{x \to(-2)^{+}} x^2=4 \text { và } \lim \limits_{x \to(-2)^{-}} f(x)=\lim \limits_{x \to(-2)^{-}}(2-x)=4$.
		Vì $\lim \limits_{x \to(-2)^{+}} f(x)=\lim \limits_{x \to(-2)^{-}} f(x)=f(-2)$ nên hàm số liên tục tại $x=2$.\\
		Vậy hàm số liên tục trên $\mathbb{R}$.}
\end{bt}
\begin{bt}%[DCHT Toán 11 - KNTT -Hứa Chí Ninh]%[1K5KG-4]
	Xét tính liên tục của hàm số $f(x)=\heva{&3x-2 &\text { khi } x>-1 \\& 1 &\text { khi } x=-1 \\& x^2-6 &\text { khi } x<-1}$ trên tập xác định.
	\loigiai{
		Tập xác định của hàm số là $\mathscr{D}=\mathbb{R}$.\\
		Khi $x>-1, f(x)=3 x-2$ là hàm đa thức nên liên tục trên $(-1 ;+\infty)$.\\
		Khi $x<-1, f(x)=x^2-6$ là hàm đa thức nên liên tục trên $(-\infty ;-1)$.\\
		Tại điểm $x=-1$, ta có $f(-1)=1$.\\
		$\lim \limits_{x \to(-1)^{+}} f(x)=\lim \limits_{x \to(-1)^{+}}(3 x-2)=-5 \text { và } \lim \limits_{x \to(-1)^{-}} f(x)=\lim \limits_{x \to(-1)^{-}}(x^2-6)=3$.\\
		Vì không tồn tại $\lim \limits_{x \to 1} f(x)$ nên hàm số gián đoạn tại $x=-1 $.
		Vậy hàm số liên tục trên $\mathbb{R} \backslash\{-1\}$.}
\end{bt}

\begin{bt}%[DCHT Toán 11 - KNTT -Hứa Chí Ninh]%[1K5KG-4]
	Cho hàm số $y=f(x)=\heva{&\dfrac{1-x}{\sqrt{2-x}-1}& \text { khi } x<1 \\& 2x & \text { khi } x \geq 1}$. Xét sự liên tục của hàm số trên tập xác định.
	\loigiai{
		\begin{itemize}
			\item Hàm số xác định và liên tục trên $(-\infty ; 1)$ và $(1 ;+\infty)$.
			\item Xét tính liên tục tại $x=1$
			$$\begin{aligned}
				& f(1)=2\cdot 1=2.\\
				& \lim \limits_{x \to 1} f(x)=\lim \limits_{x \to 1} \dfrac{1-x}{\sqrt{2-x}-1} =\lim \limits_{x \to 1} \dfrac{(1-x)(\sqrt{2-x}+1)}{2-x-1} =\lim \limits_{x \to 1}(\sqrt{2-x}+1)=2.
			\end{aligned}$$
			Ta thấy $\lim \limits_{x \to 1} f(x)=f(1)$ nên hàm số liên tục tại $\mathrm{x}=1$.
		\end{itemize}
		Vậy hàm số liên tục trên $\mathbb{R}$.}
\end{bt}
\begin{bt}%[DCHT Toán 11 - KNTT -Hứa Chí Ninh]%[1K5KG-4]
	Tìm số điểm gián đoạn của hàm số $f\left( x \right)=\left\{ \begin{array}{*{35}{l}}
		0,5 & \text{khi}\,\,x=-1 \\
		\dfrac{x\left( x+1 \right)}{x^2-1} & \text{khi}\,\,x\ne -1,\,\,x\ne 1 \\
		1 & \text{khi}\,\,x=1 \\
	\end{array} \right.$?
	% \dapso{Hàm số $y=f\left( x \right)$ gián đoạn tại $x=1$.}
	\loigiai{
		Hàm số $y=f\left( x \right)$ có tập xác định $\mathscr{D}=\mathbb{R}$.\\
		Hàm số $f\left( x \right)=\dfrac{x\left( x+1 \right)}{x^2-1}$ liên tục trên mỗi khoảng $\left( -\infty ;-1 \right)$, $\left( -1;1 \right)$ và $\left( 1;+\infty \right)$.\\
		(i) Xét tại $x=-1$, ta có $\lim\limits_{x\to -1} f\left( x \right)=\lim\limits_{x\to -1} \dfrac{x\left( x+1 \right)}{x^2-1}=\lim\limits_{x\to -1} \dfrac{x}{x-1}=\dfrac{1}{2}=f\left( -1 \right)\Rightarrow $ Hàm số liên tục tại $x=-1.$\\
		(ii) Xét tại $x=1$, ta có $\left\{ \begin{aligned}
			& \lim\limits_{x\to {1^{+}}} f\left( x \right)=\lim\limits_{x\to {1^{+}}} \dfrac{x\left( x+1 \right)}{x^2-1}=\lim\limits_{x\to {1^{+}}} \dfrac{x}{x-1}=+\infty \\
			& \lim\limits_{x\to {1^{-}}} f\left( x \right)=\lim\limits_{x\to {1^{-}}} \dfrac{x\left( x+1 \right)}{x^2-1}=\lim\limits_{x\to {1^{-}}} \dfrac{x}{x-1}=-\infty \\
		\end{aligned} \right.\Rightarrow $Hàm số $y=f\left( x \right)$ gián đoạn tại $x=1$.
	}
\end{bt}

\begin{bt}%[DCHT Toán 11 - KNTT -Hứa Chí Ninh]%[1K5KG-4]
	Tìm điểm gián đoạn của hàm số $h\left( x \right)=\left\{ \begin{aligned}
		& 2x&\text{ khi }x<0 \\
		& {x^2}+1&\text{ khi }0\le x\le 2 \\
		& 3x-1&\text{ khi }x>2 \\
	\end{aligned} \right.$?
	% \dapso{Hàm số gián đoạn tại điểm $x=0$}
	\loigiai{
		Hàm số $y=h\left( x \right)$ có tập xác định $\mathscr{D}=\mathbb{R}$.\\
		Dễ thấy hàm số $y=h\left( x \right)$ liên tục trên mỗi khoảng $\left( -\infty ;0 \right),\left( 0;2 \right)$ và $\left( 2;+\infty \right)$.\\
		Ta có $\left\{ \begin{aligned}
			& h\left( 0 \right)=1 \\
			& \lim\limits_{x\to {0^{-}}} h\left( x \right)=\lim\limits_{x\to {0^{-}}} 2x=0 \\
		\end{aligned} \right.\Rightarrow f\left( x \right)$ không liên tục tại $x=0$.\\
		Ta có $\left\{ \begin{aligned}
			& h\left( 2 \right)=5 \\
			& \lim\limits_{x\to {2^{-}}} h\left( x \right)=\lim\limits_{x\to {2^{-}}} \left( {x^2}+1 \right)=5 \\
			& \lim\limits_{x\to {2^{+}}} h\left( x \right)=\lim\limits_{x\to {2^{+}}} \left( 3x-1 \right)=5 \\
		\end{aligned} \right.\Rightarrow f\left( x \right)$ liên tục tại $x=2$.}
\end{bt}
\begin{bt}%[DCHT Toán 11 - KNTT -Hứa Chí Ninh]%[1K5KG-4]
	Cho hàm số $f\left( x \right)=\left\{ \begin{aligned}
		& -x\cos x&\text{ khi}\,\,x<0 \\
		& \dfrac{x^2}{1+x}&\text{ khi}\,\,0\le x<1 \\
		& {x^3}&\text{ khi}\,\,x\ge 1 \\
	\end{aligned} \right..$ Hàm số $f\left( x \right)$ gián đoạn tại điểm nào?
	% \dapso{Hàm số gián đoạn tại điểm $x=1.$}
	\loigiai{
		Hàm số $y=f\left( x \right)$ có tập xác định $\mathscr{D}=\mathbb{R}$.\\
		Dễ thấy $f\left( x \right)$ liên tục trên mỗi khoảng $\left( -\infty ;0 \right),\left( 0;1 \right)$ và $\left( 1;+\infty \right)$.\\
		Ta có $\left\{ \begin{aligned}
			& f\left( 0 \right)=0 \\
			& \lim\limits_{x\to {0^{-}}} f\left( x \right)=\lim\limits_{x\to {0^{-}}} \left( -x\cos x \right)=0 \\
			& \lim\limits_{x\to {0^{+}}} f\left( x \right)=\lim\limits_{x\to {0^{+}}} \dfrac{x^2}{1+x}=0 \\
		\end{aligned} \right. \Rightarrow f\left( x \right)$ liên tục tại $x=0$.\\
		Ta có $\left\{ \begin{aligned}
			& f\left( 1 \right)=1 \\
			& \lim\limits_{x\to {1^{-}}} f\left( x \right)=\lim\limits_{x\to {1^{-}}} \dfrac{x^2}{1+x}=\dfrac{1}{2} \\
			& \lim\limits_{x\to {1^{+}}}{\mathop{\,\,\lim \limits{n \to +\infty}}}\,f\left( x \right)=\,\lim\limits_{x\to {1^{+}}}{\mathop{\lim \limits{n \to +\infty}{x^3}=1}}\, \\
		\end{aligned} \right. \Rightarrow f\left( x \right)$ không liên tục tại $x=1$.\\
	}
\end{bt}
\subsubsection{Câu hỏi trắc nghiệm}
\Opensolutionfile{ans}[ans/ans-1K5-3-Dang3]
\begin{ex}%[DCHT Toán 11 - KNTT -Hứa Chí Ninh]%[1K5BG-4]
	Cho hàm số $y=f\left( x \right)$ liên tục trên $(a;b)$. Điều kiện cần và đủ để hàm số liên tục trên $\left[ a;b \right]$ là
	\choice
	{$\lim\limits_{x\to {a^{+}}} f\left( x \right)=f\left( a \right)$ và $\lim\limits_{x\to {b^{+}}} f\left( x \right)=f\left( b \right)$}
	{$\lim\limits_{x\to {a^{-}}} f\left( x \right)=f\left( a \right)$ và $\lim\limits_{x\to {b^{-}}} f\left( x \right)=f\left( b \right)$}
	{\True $\lim\limits_{x\to {a^{+}}} f\left( x \right)=f\left( a \right)$ và $\lim\limits_{x\to {b^{-}}} f\left( x \right)=f\left( b \right)$}
	{$\lim\limits_{x\to {a^{-}}} f\left( x \right)=f\left( a \right)$ và $\lim\limits_{x\to {b^{+}}} f\left( x \right)=f\left( b \right)$}
	\loigiai{
		Theo định nghĩa hàm số liên tục trên đoạn $\left[ a;b \right]$. Chọn  $\lim\limits_{x\to {a^{+}}} f\left( x \right)=f\left( a \right)$ và $\lim\limits_{x\to {b^{-}}} f\left( x \right)=f\left( b \right)$.}
\end{ex}

\begin{ex}%[DCHT Toán 11 - KNTT -Hứa Chí Ninh]%[1K5KG-4]
	Cho hàm số $y=\left\{ \begin{aligned}
		& \dfrac{1-x^3}{1-x}&\text{khi  }x<1 \\
		& 1\text{  }&\text{khi  }x\ge 1 \\
	\end{aligned} \right.$. Hãy chọn kết luận đúng
	\choice
	{\True $y$ liên tục phải tại $x=1$}
	{$y$ liên tục tại $x=1$}
	{$y$ liên tục trái tại $x=1$}
	{$y$ liên tục trên $\mathbb{R}$}
	\loigiai{
		Ta có   $y\left( 1 \right)=1$.\\
		Ta có   $\lim\limits_{x\to {1^{+}}} y=1$; $\lim\limits_{x\to {1^{-}}} y=\lim\limits_{x\to {1^{-}}} \dfrac{1-x^3}{1-x}=\lim\limits_{x\to {1^{-}}} \dfrac{\left( 1-x \right)\left( 1+x+x^2 \right)}{1-x}=\lim\limits_{x\to {1^{-}}} \left( 1+x+x^2 \right)=4$.\\
		Nhận thấy $\lim\limits_{x\to {1^{+}}} y=y\left( 1 \right)$, suy ra $y$ liên tục phải tại $x=1$.}
\end{ex}
\begin{ex}%[DCHT Toán 11 - KNTT -Hứa Chí Ninh]%[1K5BG-4]% 
	
	Trong các hàm số sau, hàm số nào liên tục trên $\mathbb{R}$?
	\choice
	{\True $y=x^3-x$}
	{$y=\cot x$}
	{$y=\dfrac{2x-1}{x-1}$}
	{$y=\sqrt{x^2-1}$}
	\loigiai{
		Vì $y=x^3-x$ là đa thức nên liên tục trên $\mathbb{R}$.}
\end{ex}
\begin{ex}%[DCHT Toán 11 - KNTT -Hứa Chí Ninh]%[1K5BG-4]
	Trong các hàm số sau, hàm số nào liên tục trên $\mathbb{R}$?
	\choice
	{$f\left( x \right)=\tan x+5$}
	{$f\left( x \right)=\dfrac{x^2+3}{5-x}$}
	{$f\left( x \right)=\sqrt{x-6}$}
	{\True $f\left( x \right)=\dfrac{x+5}{x^2+4}$}
	\loigiai{
		Hàm số $f\left( x \right)=\dfrac{x+5}{x^2+4}$ là hàm phân thức hữu tỉ và có tập xác định là $\mathscr{D}=\mathbb{R}$ do đó hàm số $f\left( x \right)=\dfrac{x+5}{x^2+4}$ liên tục trên $\mathbb{R}$.}
\end{ex}
\begin{ex}%[DCHT Toán 11 - KNTT -Hứa Chí Ninh]%[1K5BG-4]
	Cho hàm số $y=\left\{ \begin{aligned}
		& -x^2+x+3&\text{  khi}\,\,x\ge 2 \\
		& 5x+2&\text{      khi}\,\,x<2 \\
	\end{aligned} \right.$. Chọn mệnh đề sai trong các mệnh đề sau$\colon$ 
	\choice
	{Hàm số liên tục tại $x_0=1$}
	{\True Hàm số liên tục trên $\mathbb{R}$}
	{Hàm số liên tục trên các khoảng $\left( -\infty ;\,2 \right),\,\,\left( 2;\,+\infty \right)$}
	{Hàm số gián đoạn tại $x_0=2$}
	\loigiai{
		+ Với $x>2$, ta có $f\left( x \right)=-x^2+x+3$ là hàm đa thức $\Rightarrow $ hàm số $f\left( x \right)$ liên tục trên khoảng $\left( 2;\,+\infty \right)$.\\
		+ Với $x<2$, ta có $f\left( x \right)=5x+2$ là hàm đa thức $\Rightarrow $ hàm số $f\left( x \right)$ liên tục trên khoảng $\left( -\infty ;\,2 \right)$.\\
		+ Tại $x=2$.\\
		$\lim\limits_{x\to {2^{+}}} f\left( x \right)=\lim\limits_{x\to {2^{+}}} \left( -x^2+x+3 \right)=1$
		$\lim\limits_{x\to {2^{^{-}}}} f\left( x \right)=\lim\limits_{x\to {2^{-}}} \left( 5x+2 \right)=12$
		$\Rightarrow \lim\limits_{x\to {2^{+}}} f\left( x \right)\ne \lim\limits_{x\to {2^{-}}} f\left( x \right)$. Do đó không tồn tại $\lim\limits_{x\to 2} f\left( x \right).$ Vậy hàm số gián đoạn tại $x_0=2$ hay
		Hàm số không liên tục trên $\mathbb{R}$.}
\end{ex}
\begin{ex}%[DCHT Toán 11 - KNTT -Hứa Chí Ninh]%[1K5BG-4]
	Hàm số nào sau đây liên tục trên $\mathbb{R}$?
	\choice
	{$f\left( x \right)=\sqrt{x}$}
	{\True $f\left( x \right)=x^4-4x^2$}
	{$f\left( x \right)=\sqrt{\dfrac{x^4-4x^2}{x+1}}$}
	{$f\left( x \right)=\dfrac{x^4-4x^2}{x+1}$}
	\loigiai{
		Vì hàm số $f\left( x \right)=x^4-4x^2$ có dạng đa thức với   $\mathscr{D}=\mathbb{R}$ nên hàm số này liên tục trên $\mathbb{R}$}
\end{ex}
\begin{ex}%[DCHT Toán 11 - KNTT -Hứa Chí Ninh]%[1K5KG-4]
	Cho bốn hàm số $f_1(x)=\sqrt{x-1}$; $f_2(x)=x$; $f_3(x)=\tan x$; $f_4(x)=\heva{&\dfrac{x^2-1}{x-1} &\text{ khi } x\ne 1\\&2&\text{ khi }  x=1} $. Hỏi trong bốn hàm số trên có bao nhiêu hàm số liên tục trên $\mathbb{R}$?
	\choice
	{$1$}
	{\True $2$}
	{$3$}
	{$4$}
	\loigiai{
		+ Hàm số $f_1(x)=\sqrt{x-1}$ và $f_3(x)=\tan x$ không có tập xác định là $\mathbb{R}$ nên hàm số không liên tục trên $\mathbb{R}$.\\
		+ Hàm số $f_2(x)=x$ liên tục trên $\mathbb{R}$.\\
		+ Hàm số $f_4(x)=\left\{\begin{aligned}
			&\dfrac{x^2-1}{x-1}\ \text{khi } x\ne 1\\
			&2\ \text{khi } x=1\\
		\end{aligned}\right. $ có tập xác định là $\mathbb{R}$ và hàm số liên tục trên các khoảng $(-\infty ;1)$ và $(1;+\infty )$. Ta cần xét tính liên tục của hàm số $y=f_4(x)$ tại $x=1$.\\
		Ta có $f_4(1)=2$ và $\lim \limits_{x\to 1} f_4(x) =\lim \limits_{x\to 1} \dfrac{x^2-1}{x-1} =\lim \limits_{x\to 1} (x+1) =2 =f_4(1)$ nên hàm số liên tục tại $x=1$. Do đó, hàm số $y=f_4(x)$ liên tục trên $\mathbb{R}$. Vậy trong bốn hàm số trên có $2$ hàm số liên tục trên $\mathbb{R}$.}
\end{ex}
\begin{ex}%[DCHT Toán 11 - KNTT -Hứa Chí Ninh]%[1K5BG-4]
	Hàm số nào dưới đây liên tục trên $\mathbb{R}$?
	\choice
	{\True $y=x^3+3x-2$}
	{$y=\sqrt{x^2-1}$}
	{$y=\dfrac{x+1}{x-1}$}
	{$y=x+\tan x$}
	\loigiai{
		Hàm số $y=x^3+3x-2$ có tập xác định là $\mathbb{R}$ nên liên tục trên $\mathbb{R}$.\\
		Hàm số $y=\sqrt{x^2-1}$ có tập xác định là $\left(-\infty ;-1\right]\cup \left[1;+\infty \right)$ nên liên tục trên $\left(-\infty ;-1\right]\cup \left[1;+\infty \right)$.\\
		Hàm số $y=\dfrac{x+1}{x-1}$ liên tục trên $\mathbb{R}\setminus \{1\}$.\\
		Hàm số $y=x+\tan x$ liên tục trên $\mathbb{R}\setminus \left\{\dfrac{\pi }{2}+k\pi ,k\in \mathbb{Z}\right\}$.}
\end{ex}
\begin{ex}%[DCHT Toán 11 - KNTT -Hứa Chí Ninh]%[1K5BG-4]
	Hàm số $f(x)=\heva{&3&\text{ khi } x=-1\\&\dfrac{x^4+x}{x^2+x} &\text{ khi }  x\ne -1,x\ne 0\\&1&\text{ khi } x=0}$ liên tục tại
	\choice
	{mọi điểm trừ $x=0,x=-1$}
	{\True mọi điểm $x\in \mathbb{R}$}
	{mọi điểm trừ $x=-1$}
	{mọi điểm trừ $x=0$}
	\loigiai{
		Dễ thấy hàm số $f(x)$ liên tục tại mọi điểm $x\ne -1,x\ne 0$.\\
		+) Tại $x=0$:\\
		Ta có $\lim \limits_{x\to 0} f(x)=\lim \limits_{x\to 0} \dfrac{x^4+x}{x^2+x} =\lim \limits_{x\to 0} \dfrac{(x^3+1)}{(x+1)}=1$ và $f(0)=1$, suy ra hàm số liên tục tại $x=0$.\\
		+) Tại $x=-1$:\\
		Ta có $\lim \limits_{x\to -1} f(x)=\lim \limits_{x\to -1} \dfrac{x^4+x}{x^2+x} =\lim \limits_{x\to -1} \dfrac{(x+1)(x^2-x+1)}{(x+1)}=3 =f(-1)$, suy ra hàm số liên tục tại $x=-1$.\\
		Vậy hàm số liên tục trên $\mathbb{R}$.}
\end{ex}
\begin{ex}%[DCHT Toán 11 - KNTT -Hứa Chí Ninh]%[1K5BG-4]
	Số điểm gián đoạn của hàm số $f(x)=\left\{\begin{aligned}
		&0,5&\text{ khi }  x=-1\\
		&\dfrac{x(x+1)}{x^2-1}&\text{ khi }  x\ne \pm 1\\
		&1&\text{ khi } x=1\\
	\end{aligned}\right. $ là
	\choice
	{$0$}
	{\True $1$}
	{$2$}
	{$3$}
	\loigiai{
		Hàm số liên tục trên các khoảng $(-\infty ;-1)$, $(-1;1)$ và $(1;+\infty )$\\
		Ta có $f(-1)=0,5$, $f(1)=1$\\
		+ $\lim \limits_{x\to -1} f(x) =\lim \limits_{x\to -1} \dfrac{x(x+1)}{x^2-1} =\lim \limits_{x\to -1} \dfrac{x}{x-1} =\dfrac{-1}{-1-1} =0,5 =f(-1)$ nên hàm số liên tục tại $x=-1$\\
		+ $\lim \limits_{x\to 1} f(x)=\lim \limits_{x\to 1} \dfrac{x(x+1)}{x^2-1}=\lim \limits_{x\to 1} \dfrac{x}{x-1}$\\
		Vì $\lim \limits_{x\to 1^{+}} \dfrac{x}{x-1}=+\infty $, $\lim \limits_{x\to 1^{-}} \dfrac{x}{x-1}=-\infty $ nên không tồn tại giới hạn $\lim \limits_{x\to 1} f(x)$\\
		Suy ra hàm số gián đoạn tại $x=1$\\
		Vậy hàm số có một điểm gián đoạn.}
\end{ex}
\begin{ex}%[DCHT Toán 11 - KNTT -Hứa Chí Ninh]%[1K5BG-4]
	Cho hàm số $f(x)=\heva{&\dfrac{x^5+x^2}{x^3+x^2}&\text{ khi } x\ne 0;x\ne -1\\
		&3&\text{ khi } x=-1\\
		&1&\text{ khi } x=0}$. Khi đó
	\choice
	{\True Hàm số liên tục tại mỗi điểm $x\in \mathbb{R}$}
	{Hàm số liên tục tại mỗi điểm trừ $x=0$}
	{Hàm số liên tục tại mỗi điểm trừ $x=-1$}
	{Hàm số liên tục tại mỗi điểm trừ $x=-1;x=0$}
	\loigiai{
		Tập xác định: $D=\mathbb{R}$\\
		Xét $x\ne 0,x\ne -1$:\\
		$f(x)=\dfrac{x^5+x^2}{x^3+x^2}=\dfrac{x^2(x^3+1)}{x^2(x+1)} =x^2-x+1 \Rightarrow $ hàm đa thức liên tục khi $x\ne 0,x\ne -1$.\\
		Xét: $x=-1$:\\
		$f(-1)=3$, $\lim \limits_{x\to -1} f(x)=(-1)^2-(-1)+1=3 =f(-1) \Rightarrow $ hàm số liên tục khi $x=-1$.\\
		Xét: $x=0$:\\
		$f(0)=1$, $\lim \limits_{x\to 0} f(x)=0^2-0+1=1 \Rightarrow $ hàm số liên tục khi $x=0$.\\
		Vậy hàm số liên tục trên $\mathbb{R}$.}
\end{ex}
\Closesolutionfile{ans}
% \begin{indapan}{10}
% 	{ans/ans-1K5-3-Dang3}
% \end{indapan}
\begin{dang}{Bài toán chứa tham số}
\end{dang}
\subsubsection{Ví dụ mẫu}
\begin{vd}%[DCHT Toán 11 - KNTT -Hứa Chí Ninh]%[TH]%[1K5BG-5]
	Tìm $m$ để hàm số $f\left( x \right)=\left\{ \begin{aligned}
		& \dfrac{x^2-16}{x-4} \text{ khi } x>4 \\
		& mx+1 \text{ khi } x\le 4 \\
	\end{aligned} \right.\,$ liên tục tại điểm $x=4$.
	% \dapso{ $m=\dfrac{7}{4}$.}
	\loigiai{
		Ta có $\lim\limits_{x\to {4^{-}}} f\left( x \right)=f\left( 4 \right)=4m+1$; $\lim\limits_{x\to {4^{+}}} f\left( x \right)=\lim\limits_{x\to {4^{+}}} \dfrac{x^2-16}{x-4}=\lim\limits_{x\to {4^{+}}} \,\left( x+4 \right)=8$.\\
		Hàm số liên tục tại điểm $x=4\Leftrightarrow \lim\limits_{x\to {4^{-}}} f\left( x \right)=\lim\limits_{x\to {4^{+}}} f\left( x \right)=f\left( 4 \right)\Leftrightarrow 4m+1=8\Leftrightarrow m=\dfrac{7}{4}$.}
\end{vd}
\begin{vd}%[DCHT Toán 11 - KNTT -Hứa Chí Ninh]%[TH]%[1K5BG-5]
	Tìm $m$ để hàm số $f(x)=\left\{ \begin{aligned}
		& \dfrac{x^2-x-2}{x+1} \text{ khi } x>-1 \\
		& mx-2m^2 \text{ khi } x\le -1 \\
	\end{aligned} \right.$ liên tục tại $x=-1.$
	% \dapso{ $m\in \left\{ 1;-\dfrac{3}{2} \right\}$.}
	\loigiai{
		Tập xác định $\mathscr{D}=\mathbb{R}$.\\
		$\bullet$ $f(-1)=-m-2m^2$\\
		$\bullet$ $\lim\limits_{x\to -{1^{-}}} f(x)=\lim\limits_{x\to -{1^{-}}} (mx-2m^2)=-m-2m^2$.\\
		$\bullet$ $\lim\limits_{x\to -{1^{+}}} f(x)=\lim\limits_{x\to -{1^{+}}} \dfrac{x^2-x-2}{x+1}=\lim\limits_{x\to -{1^{+}}} \dfrac{(x+1)(x-2)}{x+1}=\lim\limits_{x\to -{1^{+}}} (x-2)=-3.$\\
		Hàm số liên tục tại $x=-1$ khi và chỉ khi$\lim\limits_{x\to -{1^{-}}} f(x)=\lim\limits_{x\to -{1^{+}}} f(x)=f(-1)$\\
		$\Leftrightarrow -m-2m^2=-3\Leftrightarrow 2m^2+m-3=0\Leftrightarrow \left[ \begin{aligned}
			& m=1 \\
			& m=-\dfrac{3}{2} \\
		\end{aligned} \right..$\\
		Vậy các giá trị của m là $m\in \left\{ 1;-\dfrac{3}{2} \right\}.$}
\end{vd}
\begin{vd}%[DCHT Toán 11 - KNTT -Hứa Chí Ninh]%[TH]%[1K5BG-5]	
	Tìm giá trị của tham số $m$ để hàm số $f\left( x \right)=\left\{ \begin{aligned}
		& \dfrac{x^2+3x+2}{x^2-1}\,\,\,\,\,\text{khi}\,\,\,\,\,\,x\,<\,-1 \\
		& mx+2\,\,\,\,\,\,\,\,\,\,\,\,\,\,\,\text{khi}\,\,\,\,\,\,x\,\ge \,-1 \\
	\end{aligned} \right.$ liên tục tại $x=-1$?
	% \dapso{$m=\dfrac{5}{2}$.}
	\loigiai{
		Ta có  \\
		$\bullet$ $f\left( -1 \right)=-m+2$.\\
		$\bullet$ $\lim\limits_{x\to {{\left( -1 \right)}^{+}}} f\left( x \right)=-m+2$.\\
		$\bullet$$\lim\limits_{x\to {{\left( -1 \right)}^{-}}} f\left( x \right)=\lim\limits_{x\to {{\left( -1 \right)}^{-}}} \dfrac{x^2+3x+2}{x^2-1}=\lim\limits_{x\to {{\left( -1 \right)}^{-}}} \dfrac{\left( x+1 \right)\left( x+2 \right)}{\left( x-1 \right)\left( x+1 \right)}=\lim\limits_{x\to {{\left( -1 \right)}^{-}}} \dfrac{x+2}{x-1}=\dfrac{-1}{2}$.\\
		Hàm số liên tục tại $x=-1\Leftrightarrow f\left( -1 \right)=\lim\limits_{x\to {{\left( -1 \right)}^{+}}} f\left( x \right)=\lim\limits_{x\to {{\left( -1 \right)}^{-}}} f\left( x \right)\Leftrightarrow -m+2=\dfrac{-1}{2}\Leftrightarrow m=\dfrac{5}{2}$.}
\end{vd}
\begin{vd}%[DCHT Toán 11 - KNTT -Hứa Chí Ninh]%[TH]%[1K5BG-5]% 
	Cho hàm số $f\left( x \right)=\left\{ \begin{aligned}
		& \dfrac{x^2-3x+2}{\sqrt{x+2}-2}\,\,\,\,\,\,\,\,\,khi\,\,x>2 \\
		& {m^2}x-4m+6\,\,\,\,khi\,\,\,x\le 2 \\
	\end{aligned} \right.$, $m$ là tham số. Với giá trị nào của $m$ thì hàm số đã cho liên tục tại $x=2$?
	% \dapso{$m=1.$}
	\loigiai{
		Ta có\\
		$\lim\limits_{x\to {2^{+}}} f(x)=\lim\limits_{x\to {2^{+}}} \dfrac{x^2-3x+2}{\sqrt{x+2}-2}=\lim\limits_{x\to {2^{+}}} \dfrac{\left( x-2 \right)\left( x-1 \right)\left( \sqrt{x+2}+2 \right)}{x-2}=\lim\limits_{x\to {2^{+}}} \left( x-1 \right)\left( \sqrt{x+2}+2 \right)=4$.\\
		$\lim\limits_{x\to {2^{-}}} f(x)=\lim\limits_{x\to {2^{-}}} \left( {m^2}x-4m+6 \right)=2m^2-4m+6$.\\
		$f(2)=2m^2-4m+6$.\\
		Để hàm số liên tục tại $x=2$ thì $\lim\limits_{x\to {2^{+}}} f(x)=\lim\limits_{x\to {2^{-}}} f(x)=f(2)\Leftrightarrow 2m^2-4m+6=4\Leftrightarrow 2m^2-4m+2=0\Leftrightarrow m=1$.\\
		Vậy có một giá trị của $m$ thỏa mãn hàm số đã cho liên tục tại $x=2$.}
\end{vd}
\begin{vd}%[DCHT Toán 11 - KNTT -Hứa Chí Ninh]%[VD]%[1K5KG-5]
	Tìm giá trị của $m$ để hàm số $f\left( x \right)=\left\{ \begin{array}{*{35}{l}}
		\dfrac{\sqrt{1-x}-\sqrt{1+x}}{x} & \text{khi} & x<0 \\
		m+\dfrac{1-x}{1+x} & \text{khi} & x\ge 0 \\
	\end{array} \right.$ liên tục tại $x=0$?
	% \dapso{$m=-2$.}
	\loigiai{
		Ta có\\
		$\lim\limits_{x\to {0^{+}}} f\left( x \right)=\lim\limits_{x\to {0^{+}}} \left( m+\dfrac{1-x}{1+x} \right)=m+1$.
		$\lim\limits_{x\to {0^{-}}} f\left( x \right)=\lim\limits_{x\to {0^{-}}} \left( \dfrac{\sqrt{1-x}-\sqrt{1+x}}{x} \right)=\lim\limits_{x\to {0^{-}}} \dfrac{-2x}{x\left( \sqrt{1-x}+\sqrt{1+x} \right)}=\lim\limits_{x\to {0^{-}}} \dfrac{-2}{\left( \sqrt{1-x}+\sqrt{1+x} \right)}=-1$.\\
		$f\left( 0 \right)=m+1$.\\
		Để hàm liên tục tại $x=0$ thì $\lim\limits_{x\to {0^{+}}} f\left( x \right)=\lim\limits_{x\to {0^{-}}} f\left( x \right)=f\left( 0 \right)\Leftrightarrow m+1=-1\Rightarrow m=-2$.}
\end{vd}
\subsubsection{Bài tập rèn luyện}
% \centerline{\fcolorbox{red}{yellow!50}{\bf {BÀI TẬP TỰ LUẬN}}}
\begin{bt}%[DCHT Toán 11 - KNTT -Hứa Chí Ninh]%[1K5BG-3]
	Cho hàm số $f\left( x \right)=\left\{ \begin{aligned}
		& \dfrac{x^2-1}{x-1}&\text{ khi}\,\,x<3,\,\,x\ne 1 \\
		& 4&\text{ khi}\,\,x=1 \\
		& \sqrt{x+1}&\text{ khi}\,\,x\ge 3 \\
	\end{aligned} \right.$. Xét tính liên tục của hàm số $f\left( x \right)$?
	
	% \dapso{Hàm số $f(x)$ gián đoạn tại 2 điểm $x=1$ và $x=3$.}
	\loigiai{
		Hàm số $y=f\left( x \right)$ có tập xác định $\mathscr{D}=\mathbb{R}$.\\
		Dễ thấy hàm số $y=f\left( x \right)$ liên tục trên mỗi khoảng $\left( -\infty ;1 \right),\left( 1;3 \right)$ và $\left( 3;+\infty \right)$.\\
		Ta có $\left\{ \begin{aligned}
			& f\left( 1 \right)=4 \\
			& \lim\limits_{x\to 1} f\left( x \right)=\lim\limits_{x\to 1} \dfrac{x^2-1}{x-1}=\lim\limits_{x\to 1} \left( x+1 \right)=2 \\
		\end{aligned} \right.\Rightarrow f\left( x \right)$ gián đoạn tại $x=1.$\\
		Ta có $\left\{ \begin{aligned}
			& f\left( 3 \right)=2 \\
			& \lim\limits_{x\to {3^{-}}} f\left( x \right)=\lim\limits_{x\to {3^{-}}} \dfrac{x^2-1}{x-1}=\lim\limits_{x\to {3^{-}}} \left( x+1 \right)=4 \\
		\end{aligned} \right.\Rightarrow f\left( x \right)$ gián đoạn tại $x=3$.}
\end{bt}
\begin{bt}%[DCHT Toán 11 - KNTT -Hứa Chí Ninh]%[1K5KG-5]
	Cho hàm số $f\left( x \right)=\left\{ \begin{aligned}
		& \dfrac{\sqrt{x+3}-2}{x-1} \text{ khi }\left( x>1 \right) \\
		& {m^2}+m+\dfrac{1}{4}\text{  } \text{ khi }\left( x\le 1 \right) \\
	\end{aligned} \right.$. Tìm tất cả các giá trị của tham số thực $m$ để hàm số $f\left( x \right)$ liên tục tại $x=1$?
	% \dapso{$m\in \left\{ 0;-1 \right\}$.}
	\loigiai{
		Ta có $\lim\limits_{x\to {1^{+}}} f\left( x \right)=\lim\limits_{x\to {1^{+}}} \dfrac{\sqrt{x+3}-2}{x-1}=\lim\limits_{x\to {1^{+}}} \dfrac{1}{\sqrt{x+3}+2}=\dfrac{1}{4}$; $f\left( 1 \right)=\lim\limits_{x\to {1^{-}}} f\left( x \right)=m^2+m+\dfrac{1}{4}$.\\
		Để hàm số $f\left( x \right)$ liên tục tại $x=1$ thì $m^2+m+\dfrac{1}{4}=\dfrac{1}{4}\Leftrightarrow \left[ \begin{aligned}
			& m=-1 \\
			& m=0 \\
		\end{aligned} \right.$.}
\end{bt}
\begin{bt}%[DCHT Toán 11 - KNTT -Hứa Chí Ninh]%[1K5KG-5]	
	Tìm $a$ để hàm số  $f\left( x \right)=\left\{ \begin{aligned}
		& 2x+a\,\,\,\,\,\,\,\,\,\,\,\,\,\,\,\,\,\,\,\,\,\,\,\text{ khi }\,\,\,x\le 1 \\
		& \dfrac{x^3-x^2+2x-2}{x-1}\,\,\,\text{ khi }\,\,\,x>1 \\
	\end{aligned} \right.$ liên tục trên $\mathbb{R}$?
	% \dapso{$a=1.$}
	\loigiai{
		Khi $x<1$ thì $f\left( x \right)=2x+a$ là hàm đa thức nên liên tục trên khoảng $\left( -\infty ;\,1 \right)$.\\
		Khi $x>1$ thì $f\left( x \right)=\dfrac{x^3-x^2+2x-2}{x-1}$ là hàm phân thức hữu tỉ xác định trên khoảng $\left( 1;\,+\infty \right)$ nên liên tục trên khoảng $\left( 1;\,+\infty \right)$.\\
		Xét tính liên tục của hàm số tại điểm $x=1$, ta có  \\
		+ $f\left( 1 \right)=2+a$.\\
		+ $\lim\limits_{x\,\to \,{1^{-}}} f\left( x \right)=\lim\limits_{x\,\to \,{1^{-}}} \left( 2x+a \right)=2+a$.\\
		+ $\lim\limits_{x\,\to \,{1^{+}}} f\left( x \right)=\lim\limits_{x\,\to \,{1^{+}}} \dfrac{x^3-x^2+2x-2}{x-1}=\lim\limits_{x\,\to \,{1^{+}}} \dfrac{\left( x-1 \right)\left( {x^2}+2 \right)}{x-1}=\lim\limits_{x\,\to \,{1^{+}}} \left( {x^2}+2 \right)=3$.\\
		Hàm số $f\left( x \right)$ liên tục trên $\mathbb{R}$ $\Leftrightarrow $ hàm số $f\left( x \right)$ liên tục tại $x=1$.\\
		$\lim\limits_{x\,\to \,{1^{-}}} f\left( x \right)=\lim\limits_{x\,\to \,{1^{+}}} f\left( x \right)=f\left( 1 \right)$ $\rightarrow$ $2a+1=3$ $\rightarrow$ $a=1$.}
\end{bt}
\begin{bt}%[DCHT Toán 11 - KNTT -Hứa Chí Ninh]%[TH]%[1K5KG-5]	
	Tìm $m$ để hàm số $f(x)=\left\{ \begin{aligned}
		& \dfrac{x^2+4x+3}{x+1} \text{ khi } x>-1 \\
		& mx+2 \text{ khi } x\le -1 \\
	\end{aligned} \right.$ liên tục tại điểm $x=-1$.
	% \dapso{$m=2$.}
	\loigiai{
		Ta có   $\lim\limits_{x\to {{\left( -1 \right)}^{+}}} f\left( x \right)=\lim\limits_{x\to {{\left( -1 \right)}^{+}}} \dfrac{x^2+4x+3}{x+1}=\lim\limits_{x\to {{\left( -1 \right)}^{+}}} \dfrac{\left( x+1 \right)\left( x+3 \right)}{x+1}=\lim\limits_{x\to {{\left( -1 \right)}^{+}}} \left( x+3 \right)=2$.\\
		$\lim\limits_{x\to {{\left( -1 \right)}^{-}}} f\left( x \right)=\lim\limits_{x\to {{\left( -1 \right)}^{-}}} \left( mx+2 \right)=-m+2$.\\
		$f\left( -1 \right)=-m+2$.\\
		Để hàm số đã cho liên tục tại điểm $x=-1$ thì $\lim\limits_{x\to {{\left( -1 \right)}^{+}}} f\left( x \right)=\lim\limits_{x\to {{\left( -1 \right)}^{-}}} f\left( x \right)=f\left( -1 \right)\Leftrightarrow 2=-m+2\Leftrightarrow m=0$.}
\end{bt}
\begin{bt}%[DCHT Toán 11 - KNTT -Hứa Chí Ninh]%[VD]%[1K5BG-4]	
	Cho hàm số $f\left( x \right)=\left\{ \begin{matrix}
		\sin \pi x & \text{khi}\,\,\left| x \right|\le 1 \\
		x+1\ & \text{khi}\,\ \left| x \right|>1 \\
	\end{matrix} \right.$. Tìm các khoảng liên tục của hàm số?
	% \dapan{ Hàm số liên tục trên các khoảng $\left( -\infty ;1 \right)$ và $\left( 1;+\infty \right)$.}
	\loigiai{
		Ta có   $\lim\limits_{x\to {1^{+}}} \left( x+1 \right)=2$ và $\lim\limits_{x\to {1^{-}}} \sin \pi x=0\Rightarrow \lim\limits_{x\to {1^{+}}} f\left( x \right)\ne \lim\limits_{x\to {1^{-}}} f\left( x \right)$ do đó hàm số gián đoạn tại $x=1$.\\
		Tương tự $\lim\limits_{x\to {{\left( -1 \right)}^{-}}} \left( x+1 \right)=0$ và $\lim\limits_{x\to {{\left( -1 \right)}^{+}}} \sin \pi x=0$\\
		$\Rightarrow \lim\limits_{x\to {{\left( -1 \right)}^{+}}} f\left( x \right)=\lim\limits_{x\to {{\left( -1 \right)}^{-}}} f\left( x \right)=\lim\limits_{x\to -1} f\left( x \right)=f\left( -1 \right)$ do đó hàm số liên tục tại $x=-1$.\\
		Với $x\ne \pm 1$ thì hàm số liên tục trên tập xác định.\\
		Vậy hàm số đã cho liên tục trên các khoảng $\left( -\infty ;1 \right)$ và $\left( 1;+\infty \right)$.}
\end{bt}
\begin{bt}%[DCHT Toán 11 - KNTT -Hứa Chí Ninh]%[TH]%[1K5KG-4]	
	Cho hàm số $f\left( x \right)=\left\{ \begin{aligned}
		& \sin x\,\,\,\,\text{nếu}\,\cos x\ge 0 \\
		& 1+\cos x\,\,\,\,\text{nếu}\,\cos x<0 \\
	\end{aligned} \right..$ Hỏi hàm số $f$ có tất cả bao nhiêu điểm gián đoạn trên khoảng $\left( 0;2018 \right)$?
	% \dapso{$642$.}
	\loigiai{
		Vì $f$ là hàm lượng giác nên hàm số $f$ gián đoạn khi và chỉ khi hàm số $f$ gián đoạn tại $x$ làm cho $\cos x=0$ $\Leftrightarrow x=\dfrac{\pi }{2}+k\pi \,\left( k\in \mathbb{Z} \right)\in \left( 0;2018 \right)$ $\Leftrightarrow 0<\dfrac{\pi }{2}+k\pi <2018\Leftrightarrow 0<\dfrac{1}{2}+k<\dfrac{2018}{\pi }$ $\Leftrightarrow -\dfrac{1}{2}<k<\dfrac{2018}{\pi }-\dfrac{1}{2}\Leftrightarrow 0\le k\le 641$.}
\end{bt}
\begin{bt}%[DCHT Toán 11 - KNTT -Hứa Chí Ninh]%[TH]%[1K5KG-5]	
	Tìm $m$ để hàm số $y=f\left( x \right)=\left\{ \begin{aligned}
		& {x^2}+2\sqrt{x-2} \text{ khi } x\ge 2 \\
		& 5x-5m+m^2 \text{ khi } x<2 \\
	\end{aligned} \right.$ liên tục trên $\mathbb{R}$?
	% \dapso{$m=2; m=3$.}
	\loigiai{
		Tập xác định  $\mathbb{R}\,.$\\
		+ Xét trên $\left( 2;\,+\infty \right)$ khi đó $f\left( x \right)=x^2+2\sqrt{x-2}$.\\
		$\forall {x_0}\,\in \left( 2;\,+\infty \right), \,\lim\limits_{x\to {x_0}} \left( {x_0}^2\,+2\sqrt{x_0-2} \right)=x_0^2\,+2\sqrt{x_0-2}=f\left( {x_0} \right)\,\Rightarrow $hàm số liên tục trên $\left( 2;\,+\infty \right)$.\\
		+ Xét trên $\left( -\infty ;\,2 \right)$ khi đó $f\left( x \right)=5x-5m+m^2$ là hàm đa thức liên tục trên $\mathbb{R}\,\Rightarrow $ hàm số liên tục trên $\left( -\infty ;\,2 \right)$.\\
		+ Xét tại $x_0=2$, ta có  $f\left( 2 \right)=4$.\\
		$\lim\limits_{x\to {2^{+}}} f\left( x \right)=\lim\limits_{x\to {2^{+}}} \left( {x^2}+2\sqrt{x-2} \right)=4;\,\lim\limits_{x\to {2^{-}}} f\left( x \right)=\lim\limits_{x\to {2^{-}}} \left( 5x-5m+m^2 \right)=m^2-5m+10$.
		Để hàm số đã cho liên tục trên $\mathbb{R}$ thì nó phải liên tục tại $x_0=2$.\\
		$\Leftrightarrow \lim\limits_{x\to {2^{+}}} f\left( x \right)=\,\lim\limits_{x\to {2^{-}}} f\left( x \right)=f\left( 2 \right)\Leftrightarrow {m^2}-5m+10=4\,\Leftrightarrow {m^2}-5m+6=0\,\Leftrightarrow \left[ \begin{aligned}
			& m=2 \\
			& m=3 \\
		\end{aligned} \right.$.}
\end{bt}
\begin{bt}%[DCHT Toán 11 - KNTT -Hứa Chí Ninh]%[VD]%[1K5KG-5]	
	Cho hàm số $f\left( x \right)=\left\{ \begin{aligned}
		& \,3x+a-1 \text{ khi } x\le 0 \\
		& \dfrac{\sqrt{1+2x}-1}{x} \text{ khi } x>0 \\
	\end{aligned} \right.$. Tìm tất cả giá trị thực của a để hàm số đã cho liên tục trên $\mathbb{R}$.
	% \dapso{$a=2$.}
	\loigiai{
		Hàm số liên tục tại mọi điểm $x\ne 0$ với mọi a.\\
		Với $x=0$, ta có $f\left( 0 \right)=\,a\,-1.$\\
		$\lim\limits_{x\to {0^{-}}} f\left( x \right)\,=\,\lim\limits_{x\to {0^{-}}} \left( 3x+a-1 \right)\,=\,a-1$.\\
		$\lim\limits_{x\to {0^{+}}} f\left( x \right)\,=\,\lim\limits_{x\to {0^{+}}} \dfrac{\sqrt{1+2x}-1}{x}\,=\,\lim\limits_{x\to {0^{+}}} \dfrac{2x}{x\left( \sqrt{1+2x}+1 \right)}=\lim\limits_{x\to {0^{+}}} \dfrac{2}{\sqrt{1+2x}+1}=1$.\\
		Hàm số liên tục trên $\mathbb{R}$ khi và chỉ khi hàm số liên tục tại $x=0\Leftrightarrow a-1\,=\,1\Leftrightarrow a=2$.}
\end{bt}
\begin{bt}%[DCHT Toán 11 - KNTT -Hứa Chí Ninh]%[TH]%[1K5BG-5]	
	Có bao nhiêu giá trị thực của tham số $m$ để hàm số $f\left( x \right)=\left\{ \begin{array}{*{35}{l}}
		m^2{x^2} & \text{khi }x\le 2 \\
		\left( 1-m \right)x & \text{khi }x>2 \\
	\end{array} \right.$ liên tục trên $\mathbb{R}$?
	% \dapso{$2$.}
	\loigiai{
		Ta có hàm số luôn liên tục với $\forall x\ne 2$.\\
		Tại $x=2$, ta có $\lim\limits_{x\to {2^{+}}} f\left( x \right)=\lim\limits_{x\to {2^{-}}} \left( 1-m \right)x=\left( 1-m \right)2$;\\
		$\lim\limits_{x\to {2^{-}}} f\left( x \right)=\lim\limits_{x\to {2^{-}}} \left( {m^2}{x^2} \right)=4m^2$; $f\left( 2 \right)=4m^2$.\\
		Hàm số liên tục tại $x=2$ khi và chỉ khi\\
		$\lim\limits_{x\to {2^{+}}} f\left( x \right)=\lim\limits_{x\to {2^{+}}} f\left( x \right)=f\left( 2 \right)\Leftrightarrow 4m^2=\left( 1-m \right)2\Leftrightarrow 4m^2+2m-2=0.\left( 1 \right)$\\
		Phương trình luôn có hai nghiệm thực phân biệt. Vậy có hai giá trị của $m$.\\
	}
\end{bt}
\begin{bt}%[DCHT Toán 11 - KNTT -Hứa Chí Ninh]%[TH]%[1K5BG-5]	
	Tìm $P$ để hàm số $y=\left\{ \begin{aligned}
		& \dfrac{x^2-4x+3}{x-1} \text{ khi } x>1 \\
		& 6Px-3 \text{ khi } x\le 1\,\,\, \\
	\end{aligned} \right.$ liên tục trên $\mathbb{R}$.
	% \dapso{$P=\dfrac{1}{6}.$}
	\loigiai{
		Hàm số $y=f\left( x \right)$ liên tục trên $\mathbb{R}\Rightarrow $ $y=f\left( x \right)$ liên tục tại $x=1$
		$\Rightarrow $ $\lim\limits_{x\to {1^{+}}} f\left( x \right)=\lim\limits_{x\to {1^{-}}} f\left( x \right)=f\left( 1 \right)$.\\
		$\lim\limits_{x\to {1^{+}}} f\left( x \right)=\lim\limits_{x\to {1^{+}}} \dfrac{x^2-4x+3}{x-1}=\lim\limits_{x\to {1^{+}}} \left( x-3 \right)=-2$.\\
		$\lim\limits_{x\to {1^{-}}} f\left( x \right)=\lim\limits_{x\to {1^{-}}} \left( 6Px-3 \right)=6P-3$.\\
		$f\left( 1 \right)=6P-3$.\\
		Do đó $\lim\limits_{x\to {1^{+}}} f\left( x \right)=\lim\limits_{x\to {1^{-}}} f\left( x \right)=f\left( 1 \right)$ $\Leftrightarrow 6P-3=-2\Leftrightarrow P=\dfrac{1}{6}$.}
\end{bt}
\begin{bt}%[DCHT Toán 11 - KNTT -Hứa Chí Ninh]%[TH]%[1K5BG-5]	
	Hàm số ${f(x)=\left\{ \begin{aligned}
			& ax+b+1 \text{ khi } x>0 \\
			& a\cos x+b\sin x \text{ khi } x\le 0 \\
		\end{aligned} \right.}$ liên tục trên ${\mathbb{R}}$ khi và chỉ khi
	% \dapso{ ${a-b=1}$.}
	\loigiai{
		Khi $x<0$ thì $f\left( x \right)=a\cos x+b\sin x$ liên tục với $x<0$.\\
		Khi $x>0$ thì $f\left( x \right)=ax+b+1$ liên tục với mọi $x>0$.\\
		Tại $x=0$ ta có $f\left( 0 \right)=a$.\\
		${\lim\limits_{x\to {0^{+}}} f\left( x \right)}{=\lim\limits_{x\to {0^{+}}} \left( ax+b+1 \right)}{=b+1}$.\\
		${\lim\limits_{x\to {0^{-}}} f\left( x \right)}{=\lim\limits_{x\to {0^{-}}} \left( a\cos x+b\sin x \right)}{=a}$.\\
		Để hàm số liên tục tại ${x=0}$ thì ${\lim\limits_{x\to {0^{+}}} f\left( x \right)}{=\lim\limits_{x\to {0^{-}}} f\left( x \right)}=f\left( 0 \right)\Leftrightarrow a=b+1\Leftrightarrow a-b=1$.}
\end{bt}
\begin{bt}%[DCHT Toán 11 - KNTT -Hứa Chí Ninh]%[TH]%[1K5BG-5]	
	Cho hàm số $y=\left\{ \begin{aligned}
		& 3x+1 \text{ khi } x\ge -1 \\
		& x+m \text{ khi } x<-1 \\
	\end{aligned} \right.$, $m$ là tham số. Tìm $m$ để hàm số liên tục trên $\mathbb{R}$.
	% \dapso{ $m=-1$}
	\loigiai{
		Ta có hàm số liên tục trên các khoảng $\left( -\infty ;\,-1 \right)$ và $\left( -1;\,+\infty \right)$.\\
		Xét tính liên tục của hàm số tại $x=-1$.\\
		Có $y\left( -1 \right)=-2=\lim\limits_{x\to -{1^{+}}} y$ và $\lim\limits_{x\to -{1^{-}}} y=-1+m$.\\
		Để hàm số liên tục trên $\mathbb{R}$ thì $y\left( -1 \right)=\lim\limits_{x\to -{1^{+}}} y=\lim\limits_{x\to -{1^{-}}} y\Leftrightarrow -2=-1+m\Leftrightarrow m=-1$.}
\end{bt}
% \centerline{\fcolorbox{red}{yellow!50}{\bf {CÂU HỎI TRẮC NGHIỆM }}}
% \Opensolutionfile{ans}[ans/ans-1K5-3-Dang4]
% \begin{ex}%[DCHT Toán 11 - KNTT -Hứa Chí Ninh]%[1K5BG-5]
% 	Để hàm số $y=\left\{ \begin{aligned}
% 		& {x^2}+3x+2\begin{matrix}
% 			{} & \text{khi}\begin{matrix}
% 				{} & x\le -1 \\
% 			\end{matrix} \\
% 		\end{matrix} \\
% 		& 4x+a\begin{matrix}
% 			{} & {} & \,\,\text{khi}\begin{matrix}
% 				{} & x>-1 \\
% 			\end{matrix} \\
% 		\end{matrix} \\
% 	\end{aligned} \right.$ liên tục tại điểm $x=-1$ thì giá trị của $a$ là
% 	\choice
% 	{$-4$}
% 	{\True $4$}
% 	{$1$}
% 	{$-1$}
% 	\loigiai{
% 		Hàm số liên tục tại $x=-1$ khi và chỉ khi $\lim\limits_{x\to -{1^{+}}} y=\lim\limits_{x\to -{1^{-}}} y=y\left( -1 \right).$\\
% 		$\Leftrightarrow \lim\limits_{x\to -{1^{+}}} \left( 4x+a \right)=\lim\limits_{x\to -{1^{-}}} \left( {x^2}+3x+2 \right)=y\left( -1 \right)$ $\Leftrightarrow a-4=0\Leftrightarrow a=4$.}
% \end{ex}
% \begin{ex}%[DCHT Toán 11 - KNTT -Hứa Chí Ninh]%[1K5BG-5]	
% 	Tìm giá trị thực của tham số $m$ để hàm số $f\left( x \right)=\left\{ \begin{aligned}
% 		& \dfrac{x^3-x^2+2x-2}{x-1} \text{ khi } x\ne 1 \\
% 		& 3x+m  \text{ khi } x=1 \\
% 	\end{aligned} \right.$ liên tục tại $x=1$.
% 	\choice
% 	{\True $m=0$}
% 	{$m=6$}
% 	{$m=4$}
% 	{$m=2$}
% 	\loigiai{
% 		Ta có   $f\left( 1 \right)=m+3$.\\
% 		$\lim\limits_{x\to 1} f\left( x \right)=\lim\limits_{x\to 1} \dfrac{x^3-x^2+2x-2}{x-1}=\lim\limits_{x\to 1} \dfrac{\left( x-1 \right)\left( {x^2}+2 \right)}{x-1}=\lim\limits_{x\to 1} \left( {x^2}+2 \right)=3$.\\
% 		Để hàm số $f\left( x \right)$ liên tục tại $x=1$ thì $\lim\limits_{x\to 1} f\left( x \right)=f\left( 1 \right)\Leftrightarrow 3=m+3\Leftrightarrow m=0$.}
% \end{ex}
% \begin{ex}%[DCHT Toán 11 - KNTT -Hứa Chí Ninh]%[1K5GG-5]	
% 	Cho hàm số $f\left( x \right)=\left\{ \begin{aligned}
% 		& \dfrac{{x^{2016}}+x-2}{\sqrt{2018\text{x}+1}-\sqrt{x+2018}} \text{ khi } x\ne 1 \\
% 		& k\,\,\,\,\,\,\,\,\,\,\,\,\,\,\,\,\,\,\,\,\,\,\,\,\,\,\,\,\,\,\,\,\,\,\,\,\,\,\,\,\,\,\,\,\,\,\,\,\, \text{ khi } x=1 \\
% 	\end{aligned} \right.$. Tìm $k$ để hàm số $f\left( x \right)$ liên tục tại $x=1$.
% 	\choice
% 	{\True $k=2\sqrt{2019}$}
% 	{$k=\dfrac{2017.\sqrt{2018}}{2}$}
% 	{$k=1$}
% 	{$k=\dfrac{20016}{2017}\sqrt{2019}$}
% 	\loigiai{
% 		Ta có   $\lim\limits_{x\to 1} \dfrac{{x^{2016}}+x-2}{\sqrt{2018\text{x}+1}-\sqrt{x+2018}}=\lim\limits_{x\to 1} \dfrac{\left( {x^{2016}}-1+x-1 \right)\left( \sqrt{2018\text{x}+1}+\sqrt{x+2018} \right)}{2017\text{x}-2017}$\\
% 		$=\lim\limits_{x\to 1} \dfrac{\left( x-1 \right)\left( {x^{2015}}+{x^{2014}}+...+x+1+1 \right)\left( \sqrt{2018\text{x}+1}+\sqrt{x+2018} \right)}{2017\left( \text{x}-1 \right)}=2\sqrt{2019}$.\\
% 		Để hàm số liên tục tại $x=1$ $\Leftrightarrow \lim\limits_{x\to 1} f\left( x \right)=f\left( 1 \right)$ $\Leftrightarrow k=2\sqrt{2019}$.}
% \end{ex}
% \begin{ex}%[DCHT Toán 11 - KNTT -Hứa Chí Ninh]%[1K5KG-5]	
% 	Cho hàm số $f\left( x \right)=\left\{ \begin{aligned}
% 		& \dfrac{\sqrt{x}-1}{x-1} \text{ khi } x\ne 1 \\
% 		& a \text{ khi } x=1 \\
% 	\end{aligned} \right.$. Tìm $a$ để hàm số liên tục tại $x_0=1$.
% 	\choice
% 	{$a=0$}
% 	{$a=-\dfrac{1}{2}$}
% 	{\True $a=\dfrac{1}{2}$}
% 	{$a=1$}
% 	\loigiai{
% 		Ta có $\lim\limits_{x\to 1} f\left( x \right)=\lim\limits_{x\to 1} \dfrac{\sqrt{x}-1}{x-1}=\lim\limits_{x\to 1} \dfrac{\sqrt{x}-1}{\left( \sqrt{x}-1 \right)\left( \sqrt{x}+1 \right)}=\lim\limits_{x\to 1} \dfrac{1}{\sqrt{x}+1}=\dfrac{1}{2}$.\\
% 		Để hàm số liên tục tại $x_0=1$ khi $\lim\limits_{x\to 1} f\left( x \right)=f\left( 1 \right)\Leftrightarrow a=\dfrac{1}{2}$.}
% \end{ex}
% \begin{ex}%[DCHT Toán 11 - KNTT -Hứa Chí Ninh]%[1K5BG-5]	
% 	Biết hàm số $f\left( x \right)=\left\{ \begin{aligned}
% 		& 3x+b \text{ khi } x\le -1 \\
% 		& x+a \text{ khi } x>-1 \\
% 	\end{aligned} \right.$ liên tục tại $x=-1$. Mệnh đề nào dưới đây đúng?
% 	\choice
% 	{\True $a=b-2$}
% 	{$a=-2-b$}
% 	{$a=2-b$}
% 	{$a=b+2$}
% 	\loigiai{
% 		$\lim\limits_{x\,\to \,-{1^{-}}} f\left( x \right)=f\left( -1 \right)=b-3$; $\lim\limits_{x\,\to \,-{1^{+}}} f\left( x \right)=a-1$. Để hàm số liên tục tại $x=-1$ thì $b-3=a-1\Leftrightarrow a=b-2$.}
% \end{ex}
% \begin{ex}%[DCHT Toán 11 - KNTT -Hứa Chí Ninh]%[1K5KG-5]	
% 	Cho hàm số $f\left( x \right)=\left\{ \begin{aligned}
% 		& \dfrac{3-x}{\sqrt{x+1}-2}\text{ khi }x\ne 3 \\
% 		& m\text{        khi $x=3$} \\
% 	\end{aligned} \right.$. Hàm số đã cho liên tục tại $x=3$ khi $m$ bằng bao nhiêu?
% 	\choice
% 	{$-1$}
% 	{$1$}
% 	{$4$}
% 	{\True $-4$}
% 	\loigiai{
% 		$f\left( 3 \right)=m$
% 		$\lim\limits_{x\to 3} f\left( x \right)=\lim\limits_{x\to 3} \dfrac{3-x}{\sqrt{x+1}-2}=\lim\limits_{x\to 3} \dfrac{\left( 3-x \right)\left( \sqrt{x+1}+2 \right)}{x-3}$ $=\lim\limits_{x\to 3} \left( -\sqrt{x+1}-2 \right)=-4.$\\
% 		Để hàm số liên tục tại $x=3$ thì $\lim\limits_{x\to 3} f\left( x \right)=f\left( 3 \right)$.\\
% 		Suy ra $m=-4$.}
% \end{ex}

% \begin{ex}%[DCHT Toán 11 - KNTT -Hứa Chí Ninh]%[1K5BG-5]	
% 	Biết hàm số $f\left( x \right)=\left\{ \begin{matrix}
% 		a{x^2}+bx-5 & \text{khi} & x\le 1 \\
% 		2ax-3b & \text{khi} & x>1 \\
% 	\end{matrix} \right.$ liên tục tại $x=1$ Tính giá trị của biểu thức $P=a-4b$.
% 	\choice
% 	{$P=-4$}
% 	{\True $P=-5$}
% 	{$P=5$}
% 	{$P=4$}
% 	\loigiai{
% 		Ta có $\lim\limits_{x\to {1^{-}}} f\left( x \right)=\lim\limits_{x\to {1^{-}}} \left( a{x^2}+bx-5 \right)=a+b-5=f\left( 1 \right)$.\\
% 		$\lim\limits_{x\to {1^{+}}} f\left( x \right)=\lim\limits_{x\to {1^{+}}} \left( 2ax-3b \right)=2a-3b$.
% 		Vì hàm số liên tục tại $x=1$ nên $a+b-5=2a-3b\Rightarrow a-4b=-5$.}
% \end{ex}
% \begin{ex}%[DCHT Toán 11 - KNTT -Hứa Chí Ninh]%[1K5BG-5]
% 	Tìm $m$ để hàm số $f(x)=\left\{ \begin{aligned}
% 		& {{\dfrac{x^2-x}{x-1}}} \text{ khi } x\ne 1 \\
% 		& m-1 \text { khi } \mathop x=1 \\
% 	\end{aligned} \right.$ liên tục tại $x=1$?
% 	\choice
% 	{$m=0$}
% 	{$m=-1$}
% 	{$m=1$}
% 	{\True $m=2$}
% 	\loigiai{
% 		Tập xác định $\mathscr{D}=R$\\
% 		Ta có $\lim\limits_{x\to 1} f(x)=\lim\limits_{x\to 1} \dfrac{x^2-x}{x-1}=\lim\limits_{x\to 1} x=1$ và $f(1)=m-1$.\\
% 		Hàm số liên tục tại $x=1\Leftrightarrow m-1=1\Leftrightarrow m=2$.}
% \end{ex}
% \begin{ex}%[DCHT Toán 11 - KNTT -Hứa Chí Ninh]%[1K5BG-5]	
% 	Có bao nhiêu số tự nhiên $m$ để hàm số $f\left( x \right)=\left\{ \begin{aligned}
% 		& \dfrac{x^2-3x+2}{x-1} \text{ khi } x\ne 1 \\
% 		& {m^2}+m-1 \text{ khi } x=1 \\
% 	\end{aligned} \right.$ liên tục tại điểm $x=1$?
% 	\choice
% 	{$0$}
% 	{$3$}
% 	{$2$}
% 	{\True $1$}
% 	\loigiai{
% 		$\lim\limits_{x\to 1} \dfrac{x^2-3x+2}{x-1}=\lim\limits_{x\to 1} \dfrac{\left( x-1 \right)\left( x-2 \right)}{x-1}=\lim\limits_{x\to 1} \left( x-2 \right)=-1$.\\
% 		Vì hàm số $f\left( x \right)$ liên tục tại điểm $x=1$ nên $\lim\limits_{x\to 1} f\left( x \right)=f\left( 1 \right)$\\
% 		$\Leftrightarrow {m^2}+m-1=-1$
% 		$\Leftrightarrow {m^2}+m=0\Leftrightarrow \left[ \begin{aligned}
% 			& m=0 \text{ (Thoả mãn)} \\
% 			& m=-1 \text{ (Loại)}. \\
% 		\end{aligned} \right.$}
% \end{ex}
% \begin{ex}%[DCHT Toán 11 - KNTT -Hứa Chí Ninh]%[1K5KG-5]	
% 	Tìm $a$ để hàm số $f\left( x \right)=\left\{ \begin{aligned}
% 		& \dfrac{\sqrt{x+2}-2}{x-2}\,\,\,\,\,\,\,\text{khi }x\ne 2 \\
% 		& 2x+a\,\,\,\,\,\,\,\,\,\,\,\,\,\,\,\,\,\text{khi}\,x=2 \\
% 	\end{aligned} \right.$ liên tục tại $x=2$?
% 	\choice
% 	{$\dfrac{15}{4}$}
% 	{\True $-\dfrac{15}{4}$}
% 	{$\dfrac{1}{4}$}
% 	{$1$}
% 	\loigiai{
% 		Ta có $f\left( 2 \right)=4+a$.\\
% 		Ta tính được $\lim\limits_{x\to 2} f\left( x \right)=\lim\limits_{x\to 2} \dfrac{x+2-4}{\left( x-2 \right)\left( \sqrt{x+2}+2 \right)}=\lim\limits_{x\to 2} \dfrac{1}{\sqrt{x+2}+2}=\dfrac{1}{4}$.\\
% 		Hàm số đã cho liên tục tại $x=2$ khi và chỉ khi $f\left( 2 \right)=\lim\limits_{x\to 2} f\left( x \right)\Leftrightarrow 4+a=\dfrac{1}{4}\Leftrightarrow a=-\dfrac{15}{4}$.\\
% 		Vậy hàm số liên tục tại $x=2$ khi $a=-\dfrac{15}{4}$.}
% \end{ex}
% \begin{ex}%[DCHT Toán 11 - KNTT -Hứa Chí Ninh]%[1K5KG-5]	
% 	Cho hàm số $f\left( x \right)=\left\{ \begin{aligned}
% 		& \dfrac{x^2-3x+2}{\sqrt{x+2}-2} \text{ khi } x>2 \\
% 		& {m^2}x-4m+6 \text{ khi } x\le 2 \\
% 	\end{aligned} \right.$, $m$ là tham số. Có bao nhiêu giá trị của $m$ để hàm số đã cho liên tục tại $x=2$?
% 	\choice
% 	{$3$}
% 	{$0$}
% 	{$2$}
% 	{\True $1$}
% 	\loigiai{
% 		Ta có\\
% 		$\lim\limits_{x\to {2^{+}}} f(x)=\lim\limits_{x\to {2^{+}}} \dfrac{x^2-3x+2}{\sqrt{x+2}-2}=\lim\limits_{x\to {2^{+}}} \dfrac{\left( x-2 \right)\left( x-1 \right)\left( \sqrt{x+2}+2 \right)}{x-2}=\lim\limits_{x\to {2^{+}}} \left( x-1 \right)\left( \sqrt{x+2}+2 \right)=4$.\\
% 		$\lim\limits_{x\to {2^{-}}} f(x)=\lim\limits_{x\to {2^{-}}} \left( {m^2}x-4m+6 \right)=2m^2-4m+6$.\\
% 		$f(2)=2m^2-4m+6$.\\
% 		Để hàm số liên tục tại $x=2$ thì $\lim\limits_{x\to {2^{+}}} f(x)=\lim\limits_{x\to {2^{-}}} f(x)=f(2)\Leftrightarrow 2m^2-4m+6=4\Leftrightarrow 2m^2-4m+2=0\Leftrightarrow m=1$.\\
% 		Vậy có một giá trị của $m$ thỏa mãn hàm số đã cho liên tục tại $x=2$.}
% \end{ex}
% \begin{ex}%[DCHT Toán 11 - KNTT -Hứa Chí Ninh]%[1K5KG-5]	
% 	Cho hàm số $f\left( x \right)=\left\{ \begin{aligned}
% 		& \dfrac{\sqrt{3x^2+2x-1}-2}{x^2-1},\ x\ne 1 \\
% 		& 4-m,\ \quad \quad \quad \quad \quad x=1 \\
% 	\end{aligned} \right.$. Hàm số $f\left( x \right)$ liên tục tại $x_0=1$ khi
% 	\choice
% 	{\True $m=3$}
% 	{$m=-3$}
% 	{$m=7$}
% 	{$m=-7$}
% 	\loigiai{
% 		Tập xác định $\mathscr{D}=\mathbb{R}$, $x_0=1\in \mathbb{R}$.\\
% 		Ta có $f\left( 1 \right)=4-m$.\\
% 		$\lim\limits_{x\to 1} f\left( x \right)=\lim\limits_{x\to 1} \dfrac{\sqrt{3x^2+2x-1}-2}{\left( x+1 \right)\left( x-1 \right)}$ $=\lim\limits_{x\to 1} \dfrac{\left( x-1 \right)\left( 3x+5 \right)}{\left( x+1 \right)\left( x-1 \right)\left( \sqrt{3x^2+2x-1}+2 \right)}$\\
% 		$=\lim\limits_{x\to 1} \dfrac{3x+5}{\left( x+1 \right)\left( \sqrt{3x^2+2x-1}+2 \right)}=1$.\\
% 		Hàm số $f\left( x \right)$ liên tục tại $x_0=1$ khi và chỉ khi $\lim\limits_{x\to 1} \left( x \right)=f\left( 1 \right)\Leftrightarrow 4-m=1\Leftrightarrow m=3$.}
% \end{ex}
% \begin{ex}%[DCHT Toán 11 - KNTT -Hứa Chí Ninh]%[1K5KG-5]	
% 	Tìm giá trị của tham số $m$ để hàm số $f\left( x \right)=\left\{ \begin{aligned}
% 		& \dfrac{x^2+3x+2}{x^2-1}\,\,\,\,\,\text{khi}\,\,\,\,\,\,x\,<\,-1 \\
% 		& mx+2\,\,\,\,\,\,\,\,\,\,\,\,\,\,\,\text{khi}\,\,\,\,\,\,x\,\ge \,-1 \\
% 	\end{aligned} \right.$ liên tục tại $x=-1$.
% 	\choice
% 	{$m=\dfrac{-3}{2}$}
% 	{$m=\dfrac{-5}{2}$}
% 	{$m=\dfrac{3}{2}$}
% 	{\True $m=\dfrac{5}{2}$}
% 	\loigiai{
% 		Ta có  \\
% 		$\bullet$ $f\left( -1 \right)=-m+2$.\\
% 		$\bullet$ $\lim\limits_{x\to {{\left( -1 \right)}^{+}}} f\left( x \right)=-m+2$.\\
% 		$\bullet$$\lim\limits_{x\to {{\left( -1 \right)}^{-}}} f\left( x \right)=\lim\limits_{x\to {{\left( -1 \right)}^{-}}} \dfrac{x^2+3x+2}{x^2-1}=\lim\limits_{x\to {{\left( -1 \right)}^{-}}} \dfrac{\left( x+1 \right)\left( x+2 \right)}{\left( x-1 \right)\left( x+1 \right)}=\lim\limits_{x\to {{\left( -1 \right)}^{-}}} \dfrac{x+2}{x-1}=\dfrac{-1}{2}$.\\
% 		Hàm số liên tục tại $x=-1\Leftrightarrow f\left( -1 \right)=\lim\limits_{x\to {{\left( -1 \right)}^{+}}} f\left( x \right)=\lim\limits_{x\to {{\left( -1 \right)}^{-}}} f\left( x \right)\Leftrightarrow -m+2=\dfrac{-1}{2}\Leftrightarrow m=\dfrac{5}{2}$.}
% \end{ex}
% \begin{ex}%[DCHT Toán 11 - KNTT -Hứa Chí Ninh]%[1K5KG-5]	
% 	Cho hàm số $f(x)=\left\{ \begin{matrix}
% 		\dfrac{\sqrt{x^2+4}-2}{x^2}\ \ \ \,\,\text{khi }\ x\ne 0 \\
% 		2a-\dfrac{5}{4}\ \ \ \ \ \ \ \ \ \ \ \,\text{khi }\ x=0 \\
% 	\end{matrix} \right.$. Tìm giá trị thực của tham số $a$ để hàm số $f(x)$ liên tục tại $x=0$.
% 	\choice
% 	{$a=-\dfrac{3}{4}$}
% 	{$a=\dfrac{4}{3}$}
% 	{$a=-\dfrac{4}{3}$}
% 	{\True $a=\dfrac{3}{4}$}
% 	\loigiai{
% 		Tập xác định $\mathscr{D}=\mathbb{R}$.\\
% 		$\lim\limits_{x\to 0} f(x)=\lim\limits_{x\to 0} \dfrac{\sqrt{x^2+4}-2}{x^2}=\lim\limits_{x\to 0} \dfrac{\left( \sqrt{x^2+4}-2 \right)\left( \sqrt{x^2+4}+2 \right)}{x^2\left( \sqrt{x^2+4}+2 \right)}$
% 		$=\lim\limits_{x\to 0} \dfrac{x^2+4-4}{x^2(\sqrt{x^2+4}+2)}=\lim\limits_{x\to 0} \dfrac{1}{\sqrt{x^2+4}+2}=\dfrac{1}{4}$.\\
% 		$f(0)=2a-\dfrac{5}{4}$.\\
% 		Hàm số $f(x)$ liên tục tại $x=0\Leftrightarrow \lim\limits_{x\to 0} f(x)=f(0)\Leftrightarrow 2a-\dfrac{5}{4}=\dfrac{1}{4}\Leftrightarrow a=\dfrac{3}{4}$.\\
% 		Vậy $a=\dfrac{3}{4}$.}
% \end{ex}
% \begin{ex}%[DCHT Toán 11 - KNTT -Hứa Chí Ninh]%[1K5BG-5]	
% 	Cho hàm số $f\left( x \right)=\left\{\begin{aligned}
% 		& {x^2}-2x+3\text{  khi }x\ne 1 \\
% 		& 3x+m-1\text{   khi }x=1 \\
% 	\end{aligned} \right.$. Tìm $m$ để hàm số liên tục tại $x_0=1$.
% 	\choice
% 	{$m=1$}
% 	{$m=3$}
% 	{\True $m=0$}
% 	{$m=2$}
% 	\loigiai{
% 		Tập xác định $\mathscr{D}=\mathbb{R}$.\\
% 		Ta có $f\left( 1 \right)=2+m$.\\
% 		$\lim\limits_{x\to 1} f\left( x \right)=\lim\limits_{x\to 1}{\mathop{ }}\left( {x^2}-2x+3 \right)=2$.\\
% 		Hàm số liên tục tại $x_0=1\Leftrightarrow \lim\limits_{x\to 1} f\left( x \right)=f\left( 1 \right)\Leftrightarrow 2=m+2\Leftrightarrow m=0$.}
% \end{ex}
% \begin{ex}%[DCHT Toán 11 - KNTT -Hứa Chí Ninh]%[1K5BG-5]
% 	Cho hàm số $f(x)=\left\{ \begin{aligned}
% 		& \dfrac{x^2-3x+2}{x-2} \text{ khi } x\ne 2 \\
% 		& a \text{ khi } x=2 \\
% 	\end{aligned} \right.$. Hàm số liên tục tại $x=2$ khi $a$ bằng
% 	\choice
% 	{\True $1$}
% 	{$0$}
% 	{$2$}
% 	{$-1$}
% 	\loigiai{
% 		Hàm số liên tục tại $x=2$ khi và chỉ khi $\lim\limits_{x\to 2} f(x)=f(2)$.\\
% 		Ta có $f(2)=a,\lim\limits_{x\to 2} f(x)=\lim\limits_{x\to 2} \dfrac{x^2-3x+2}{x-2}=\lim\limits_{x\to 2} (x-1)=1$. Do đó $a=1$.}
% \end{ex}
% \begin{ex}%[DCHT Toán 11 - KNTT -Hứa Chí Ninh]%[1K5KG-5]
% 	Tìm tất cả các giá trị thực của $m$ để hàm số $f(x)=\left\{ \begin{aligned}
% 		& \dfrac{\sqrt{x+1}-1}{x} \text{ khi }  x>0 \\
% 		& \sqrt{x^2+1}-m \text{ khi } \le 0 \\
% 	\end{aligned} \right.$ liên tục trên $\mathbb{R}$.
% 	\choice
% 	{$m=\dfrac{3}{2}$}
% 	{\True $m=\dfrac{1}{2}$}
% 	{$m=-2$}
% 	{$m=-\dfrac{1}{2}$}
% 	\loigiai{
% 		Khi $x>0$, ta có   $f(x)=\dfrac{\sqrt{x+1}-1}{x}$ liên tục trên khoảng $\left( 0;+\infty \right)$.\\
% 		Khi $x<0$, ta có   $f(x)=\sqrt{x^2+1}-m$ liên tục trên khoảng $\left( -\infty ;0 \right)$.\\
% 		Hàm số liên tục trên $\mathbb{R}$ khi và chỉ khi hàm số liên tục tại $x=0$.\\
% 		Ta có   $\lim\limits_{x\to {0^{+}}} f(x)=\lim\limits_{x\to {0^{+}}} \dfrac{\sqrt{x+1}-1}{x}=\lim\limits_{x\to {0^{+}}} \dfrac{1}{\sqrt{x+1}+1}=\dfrac{1}{2}$.\\
% 		$\lim\limits_{x\to {0^{-}}} f(x)=\lim\limits_{x\to {0^{-}}} \left( \sqrt{x^2+1}-m \right)=1-m=f\left( 0 \right)$.
% 		Do đó hàm số liên tục tại $x=0$ khi và chỉ khi $\dfrac{1}{2}=1-m\Leftrightarrow m=\dfrac{1}{2}$.}
% \end{ex}
% \begin{ex}%[DCHT Toán 11 - KNTT -Hứa Chí Ninh]%[1K5BG-5]	
% 	Tìm tất cả các giá trị thực của tham số $m$ để hàm số $f\left( x \right)=\left\{ \begin{array}{*{35}{l}}
% 		\dfrac{x^2-16}{x-4} & \text{khi} & x>4 \\
% 		mx+1 & \text{khi} & x\le 4 \\
% 	\end{array} \right.$ liên tục trên $\mathbb{R}$.
% 	\choice
% 	{$m=8$ hoặc $m=-\dfrac{7}{4}$}
% 	{\True $m=\dfrac{7}{4}$}
% 	{$m=-\dfrac{7}{4}$}
% 	{$m=-8$ hoặc $m=\dfrac{7}{4}$}
% 	\loigiai{
% 		*) Với $x>4$ thì $f\left( x \right)=\dfrac{x^2-16}{x-4}$ là hàm phân thức nên liên tục trên tập xác định của nó $\Rightarrow f\left( x \right)$ liên tục trên $\left( 4;+\infty \right)$.\\
% 		*) Với $x<4$ thì $f\left( x \right)=mx+1$ là hàm đa thức nên liên tục trên $\mathbb{R}\Rightarrow f\left( x \right)$ liên tục trên $\left( -\infty ;4 \right)$.\\
% 		Do vậy hàm số $f\left( x \right)$ đã liên tục trên các khoảng $\left( 4;+\infty \right)$, $\left( -\infty ;4 \right)$.\\
% 		Suy ra hàm số $f\left( x \right)$ liên tục trên $\mathbb{R}$ khi và chỉ khi $f\left( x \right)$ liên tục tại $x=4$. Do đó\\
% 		$ \lim\limits_{x\to {4^{+}}} f\left( x \right)=\lim\limits_{x\to {4^{-}}} f\left( x \right)=f\left( 4 \right)\Leftrightarrow \lim\limits_{x\to {4^{+}}} \dfrac{x^2-16}{x-4}=\lim\limits_{x\to {4^{-}}} \left( mx+1 \right)=4m+1\Leftrightarrow \lim\limits_{x\to {4^{+}}} \left( x+4 \right)=4m+1$
% 		$\Leftrightarrow 4m+1=8\Leftrightarrow m=\dfrac{7}{4}$.}
% \end{ex}
% \begin{ex}%[DCHT Toán 11 - KNTT -Hứa Chí Ninh]%[1K5KG-5]	
% 	Nếu hàm số $f\left( x \right)=\left\{ \begin{aligned}
% 		& {x^2}+ax+b\,\,\text{khi}\,\,x<-5\,\, \\
% 		& x+17\,\,\,\,\,\,\,\,\,\,\,\,\text{khi}\,-5\le x\le 10 \\
% 		& ax+b+10\,\,\,\text{khi}\,x>10 \\
% 	\end{aligned} \right.$ liên tục trên $\mathbb{R}$ thì $a+b$ bằng
% 	\choice
% 	{\True $-1$}
% 	{$0$}
% 	{$1$}
% 	{$2$}
% 	\loigiai{
% 		Với $x<-5$ ta có $f\left( x \right)=x^2+ax+b$, là hàm đa thức nên liên tục trên $\left( -\infty ;-5 \right)$.\\
% 		Với $-5<x<10$ ta có $f\left( x \right)=x+7$, là hàm đa thức nên liên tục trên $\left( -5;10 \right)$.\\
% 		Với $x>10$ ta có $f\left( x \right)=ax+b+10$, là hàm đa thức nên liên tục trên $\left( 10;+\infty \right)$.\\
% 		Để hàm số liên tục trên $\mathbb{R}$ thì hàm số phải liên tục tại $x=-5$ và $x=10$.\\
% 		Ta có  \\
% 		$f\left( -5 \right)=12$;$f\left( 10 \right)=17$.\\
% 		$\lim\limits_{x\to -{5^{-}}} f\left( x \right)=\lim\limits_{x\to -{5^{-}}} \left( {x^2}+ax+b \right)$ $=-5a+b+25$.\\
% 		$\lim\limits_{x\to -{5^{+}}} f\left( x \right)=\lim\limits_{x\to -{5^{+}}} \left( x+17 \right)=12$.
% 		$\lim\limits_{x\to {{10}^{-}}} f\left( x \right)=\lim\limits_{x\to {{10}^{-}}} \left( x+17 \right)=27$.
% 		$\lim\limits_{x\to {{10}^{+}}} f\left( x \right)=\lim\limits_{x\to {{10}^{+}}} \left( ax+b+10 \right)=10a+b+10$.
% 		Hàm số liên tục tại $x=-5$ và $x=10$ khi\\
% 		$\left\{ \begin{aligned}
% 			& 5a+b+25=12 \\
% 			& 10a+b+10=27 \\
% 		\end{aligned} \right.\Leftrightarrow \left\{ \begin{aligned}
% 			& -5a+b=-13 \\
% 			& 10a+b=17 \\
% 		\end{aligned} \right.\Leftrightarrow \left\{ \begin{aligned}
% 			& a=2 \\
% 			& b=-3 \\
% 		\end{aligned} \right.\Rightarrow a+b=-1$.}
% \end{ex}
\Closesolutionfile{ans}
% \begin{indapan}{10}
% 	{ans/ans-1K5-3-Dang4}
% \end{indapan}
% \begin{dang}{Toán thực tế, liên môn về hàm số liên tục}
% \end{dang}
% \subsubsection{Ví dụ mẫu}
% \begin{vd}%[DCHT Toán 11 - KNTT- Phạm Tuấn]%[1K5BG-6]
% 	Trong kỹ thuật ứng dụng, chúng ta thường xuyên ghi nhận được các hàm số mà giá trị của nó thay đổi đột ngột tại một thời điểm $t$ xác định. Ví dụ:  Sự thay đổi điện áp của một mạch điện tại thời điểm t khi đóng hoặc ngắt mạch. Thông thường, giá trị $t = 0$ luôn được chọn là thời điểm bắt đầu cho việc đóng hoặc ngắt điện áp. Quá trình đóng, ngắt mạch trên có thể mô tả bằng mô hình toán học bởi hàm Heaviside
% 	\[
% 	u(t) = \heva{&0 && \text{ nếu } t <0\\& 1 && \text{ nếu } t \geq 0.}
% 	\]
% 	Hàm Heaviside có liên tục tại $t=0$ hay không?
% 	% \dapso{Không liên tục tại $t=0$}
% 	\loigiai{
% 		Ta có $\displaystyle \lim \limits_{t\to 0^+} u(t) = \lim \limits_{t\to 0^+} 1 =1$; $\displaystyle \lim \limits_{t\to 0^-} u(t) = \lim \limits_{t\to 0^-} 0 =0$. Do đó $\displaystyle \lim \limits_{t\to 0^+} u(t) \ne \displaystyle \lim \limits_{t\to 0^-} u(t)$. \\
% 		Vậy hàm Heaviside không liên tục tại $t=0$.
% 	}
% \end{vd}

% \begin{vd}%[DCHT Toán 11 - KNTT- Phạm Tuấn]%[1K5BG-6]
% 	Thuế thu nhập của một tiểu bang tại Hoa Kỳ được xác định bởi hàm số
% 	\[
% 	T(x) = \heva{& 0  && \text{ nếu } x \leq 0 \\&0{,}14x  && \text{ nếu } 0<x \leq 10000\\&c+0{,}21x  && \text{ nếu } x \geq 10000.}
% 	\]
% 	Trong đó $x$ là thu nhập tính bằng USD. Tìm $c$ để hàm số đã cho liên tục trên $\mathbb{R}$. 
% 	% \dapso{$c=-700$}
% 	\loigiai{
% 		Dễ thấy hàm số $T(x)$ liên tục khi $x \ne 0$  và $x \ne 10000$. \\
% 		Ta có $\displaystyle \lim\limits _{x \rightarrow 0^{-}} T(x) = \lim\limits _{x \rightarrow 0^{+}} T(x) =0$ nên hàm số liên tục tại $x=0$. \\
% 		Hàm số đã cho liên tục tại $x=10000$ khi và chỉ khi 
% 		\begin{align*}
% 			&\lim\limits _{x \rightarrow 10000^{-}} T(x) = \lim\limits _{x \rightarrow 10000^{+}} T(x) = T(10000) \\
% 			\Leftrightarrow~ & \lim\limits _{x \rightarrow 10000^{-}} 0{,}14x = \lim\limits _{x \rightarrow 10000^{+}} (c+0{,}21x) = c+2100 \\
% 			\Leftrightarrow~ & 1400 = c+2100 \Leftrightarrow c=-700.
% 		\end{align*}
% 	}
% \end{vd}

% \begin{vd}%[DCHT Toán 11 - KNTT- Phạm Tuấn]%[1K5BG-6]
% 	Thuế thu nhập của một tiểu bang tại Hoa Kỳ được xác định bởi hàm số
% 	\[
% 	T(x) = \heva{& 0  && \text{ nếu } x \leq 0 \\&a+0{,}12x  && \text{ nếu } 0<x \leq 20000\\&b+0{,}16(x-20000)  && \text{ nếu } x> 20000.}
% 	\]
% 	Trong đó $x$ là thu nhập tính bằng USD. Tìm $a,b$ để hàm số đã cho liên tục trên $\mathbb{R}$. 
% 	% \dapso{$a=0$; $b=2400$}
% 	\loigiai{
% 		Dễ thấy hàm số $T(x)$ liên tục khi $x \ne 0$  và $x \ne 20000$. \\
% 		Hàm số đã cho liên tục tại $x=0$ khi và chỉ khi 
% 		\begin{align*}
% 			&\lim\limits _{x \rightarrow 0^{-}} T(x) = \lim\limits _{x \rightarrow 0^{+}} T(x) = T(0) \\
% 			\Leftrightarrow~ & \lim\limits _{x \rightarrow 0^{-}} (a+0{,}12x)  = 0 \\
% 			\Leftrightarrow~ & a = 0 .
% 		\end{align*}
% 		Hàm số đã cho liên tục tại $x=20000$ khi và chỉ khi 
% 		\begin{align*}
% 			&\lim\limits _{x \rightarrow 20000^{-}} T(x) = \lim\limits _{x \rightarrow 20000^{+}} T(x) = T(20000) \\
% 			\Leftrightarrow~ & \lim\limits _{x \rightarrow 20000^{-}} 0{,}12x = \lim\limits _{x \rightarrow 20000^{+}} [b+0{,}16(x-20000)] = 2400 \\
% 			\Leftrightarrow~ & b=2400.
% 		\end{align*}
% 	}
% \end{vd}





% \begin{vd}%[DCHT Toán 11 - KNTT- Phạm Tuấn]%[1K5BG-6]
% 	Số lượng đơn vị hàng tồn kho trong một công ty nhỏ được cho bởi
% 	$$
% 	N(t)=80\left(2 \left [\frac{t+2}{2}  \right ]-t\right)
% 	$$
% 	trong đó $t$ là thời gian tính bằng tháng. $[x]$ là số nguyên lớn nhất không vượt quá $x$ (ví dụ $[2{,}4]=2$, $[-2{,}7] = -3$).
% 	Hàm số $N(t)$ có liên tục tại $t=10$ hay không?
% 	% \dapso{Hàm số $N(t)$ không liên tục tại $t=10$}
% 	\loigiai{
% 		Khi $t \to 10^+$, ta có $\left [\dfrac{t+2}{2}  \right ] = 6$. \\
% 		Suy ra  $\displaystyle \lim\limits _{t \to 10^+} N(t) =\lim_{t \to 10^+}  80\left(2 \left [\dfrac{t+2}{2}  \right ]-t\right) = 160$. \\
% 		Khi $t \to 10^-$, ta có $\left [\dfrac{t+2}{2}  \right ] = 5$. \\
% 		Suy ra  $\displaystyle \lim\limits _{t \to 10^-} N(t) = \lim_{t \to 10^-} 80\left(2 \left [\dfrac{t+2}{2}  \right ]-t\right) = 0$. \\
% 		Vậy hàm số $N(t)$ không liên tục tại $t=10$.
% 	}
% \end{vd}


% \begin{vd}%[DCHT Toán 11 - KNTT- Phạm Tuấn]%[1K5BG-6]
% 	Giả sử giá điện sinh hoạt trong mỗi tháng dành cho các hộ gia đình được cho bởi bảng sau
% 	\begin{center}
% 		\begin{tabular}{|c|c|}
% 			\hline
% 			Mức kWh điện tiêu thụ  & Giá bán điện (VNĐ/ kWh) \\
% 			\hline
% 			Mức 1: từ $0$ đến $100$ kWh  & $1600$ \\
% 			\hline
% 			Mức 2: từ trên $100$ đến $300$ kWh  & $2000$ \\
% 			\hline
% 			Mức 3: trên $300$ kWh  & $3000$ \\
% 			\hline
% 		\end{tabular}
% 	\end{center}
% 	\begin{enumerate}
% 		\item Thiết lập hàm số $f(x)$ liên hệ giữa $x$ (kWh) điện tiêu thụ và số tiền $f(x)$ tương ứng phải trả.
% 		\item  Hàm số $f(x)$  có liên tục tại $x=100$ hay không?
% 	\end{enumerate}
% 	% \dapso{Hàm số $f(x)$ liên tục tại $x=100$}
% 	\loigiai{
% 		Từ giả thiết ta có
% 		\begin{eqnarray*}
% 			&& f(x)=\heva{&1600x &(0<x\leq 100)&\\&1600\cdot 100+2000\cdot (x-100) &(100<x\leq 300)&\\&1600\cdot 100+2000\cdot 200+3000\cdot(x-300)&(x>300)&\\} \\ 
% 			&\Leftrightarrow& f(x)=\heva{&1600x &(0<x\leq 100)&\\&2000x-40000 &(100<x\leq 300)&\\&3000x-340000 &(x>300)&.}
% 		\end{eqnarray*}
% 		Ta có 
% 		\begin{align*}
% 			&\lim\limits _{x \rightarrow 100^{-}} f(x) =  \lim\limits _{x \rightarrow 100^{-}} 1600x = 160 000 \\
% 			& \lim\limits _{x \rightarrow 100^{+}} f(x) =  \lim\limits _{x \rightarrow 100^{+}} (2000x-40000) = 160 000. 
% 		\end{align*}
% 		Suy ra  $\displaystyle \lim\limits _{x \rightarrow 100^{-}} f(x) = \lim\limits _{x \rightarrow 100^{+}} f(x) = f(100) =160 000 $.\\
% 		Vậy hàm số $f(x)$ liên tục tại $x=100$.
% 	}
% \end{vd}



% \subsubsection{Bài tập rèn luyện}
% % \centerline{\fcolorbox{red}{yellow!50}{\bf {BÀI TẬP TỰ LUẬN}}}
% \begin{bt}%[DCHT Toán 11 - KNTT -Vũ Hồng Toàn]%[1K5YG-6]
% 	Từ ngày 04/05/2023 giá điện sinh hoạt $Q(x)$ (đồng)/tháng được chia thành 6 bậc. Trong đó $x$ (KWh)  của 3 bậc đầu được tính như sau:
% 	$$Q(x)=\heva{&1728 x&\text{ khi }& 0<x\le 50\\&1786 x&\text{ khi }& 51\le x\le 100\\&2074 x&\text{ khi }& 101\le x\le 200.}$$
% 	Hỏi
% 	\begin{enumerate}
% 		\item $Q(x)$ có liên tục tại $x_0=50$ không?
% 		\item Một gia đình dùng hết $150$ kWh/tháng. Tính tiền điện phải trả cho tháng đó?
% 	\end{enumerate}
	
% 	% \dapso{a) $Q(x)$ không liên tục tại $x_0=50$; b) $279.400$ đồng.}
% 	\loigiai{
% 		\begin{enumerate}
% 			\item  Hàm số $Q(x)$ không liên tục tại $x_0=50$ vì $\displaystyle \lim\limits _{x \rightarrow 50^{+}} Q(x) $ không tồn tại .
% 			\item  Ta có $1728\cdot 50+ 1786\cdot 50+2074\cdot 50=279.400$.\\
% 			Vậy tiền điện phải trả cho tháng đó là $279.400$ đồng.
% 		\end{enumerate}
% 	}
% \end{bt}

% \begin{bt}%[DCHT Toán 11 - KNTT -Vũ Hồng Toàn]%[1K5YG-6]
% 	Tại một xưởng sản xuất bột đá thạch anh, giá bán (tính theo nghìn đồng) của $x$ (kg) bột đá thạch anh được tính theo công thức sau $P(x)=\heva{&4{,}5x &\text{ khi } &0<x\le 400\\&4x+k &\text{ khi } &x> 400.}$\\
% 	Khi đó
% 	\begin{enumerate}		
% 		\item  Với $k = 0$, xét tính liên tục của hàm số $P(x)$ trên $(0;+\infty)$.\\
% 		\item  Với giá trị nào của $k$ thì hàm số $P(x)$ liên tục trên $(0;+\infty)$.
% 	\end{enumerate}
% 	% \dapso{a) $P(x)$ liên tục trên $(0;+\infty)$; b) $k=200$.}
% 	\loigiai{
% 		\begin{enumerate}
% 			\item  Với $x_0 \in(0 ; 400)$ khi đó $P(x)=4{,}5 x$.\\
% 			Suy ra $\lim\limits_{x \rightarrow x_{0}} P(x)=\lim\limits_{x \rightarrow x_{0}}(4{,}5 \mathrm{x})=4{,}5 \mathrm{x}_0=P\left(\mathrm{x}_0\right)$.\\
% 			Do đó $P(x)$ liên tục trên $(0 ; 400)$.
% 			Tại $x_0=400$, ta có
% 			\begin{eqnarray*}
% 				&& \lim\limits_{x \rightarrow 400^-} P(x)=\lim \limits{n \to +\infty}_{x \rightarrow 400^{-}}(4{,}5 x)=4{,}5 \cdot 400=1800. \\
% 				&& \lim \limits{n \to +\infty}_{x \rightarrow 400^{+}} P(x)=\lim \limits{n \to +\infty}_{x \rightarrow 400^{+}}(4 x)=4 \cdot 400=1600.
% 			\end{eqnarray*}
% 			Suy ra $\lim\limits_{x \rightarrow 400^-} P(x) \neq \lim\limits_{x \rightarrow 400^+} P(x)$.\\
% 			Do đó không tồn tại $\lim\limits_{x \rightarrow 400} P(x)$.\\
% 			Vì vậy hàm số không liên tục tại $x=400$.\\
% 			Với $x_0 \in(400 ;+\infty)$ khi đó $P(x)=4x$.\\
% 			Suy ra $\lim\limits_{x \rightarrow x_{0}} P(x)=\lim\limits_{x \rightarrow x_{0}}(4 x)=4 x_0=P\left(x_0\right)$.\\
% 			Do đó $P(x)$ liên tục trên $(400 ;+\infty)$.
% 			Vậy hàm số liên tục trên $(0 ; 400)$ và $(400 ;+\infty)$.
% 			\item  Để hàm số $P(x)$ liên tục trên $(0 ;+\infty)$ thì $P(x)$ phải liên tục trên $x_0=400$.\\
% 			Do đó $\lim\limits_{x \rightarrow 400^-} P(x)=\lim\limits_{x \rightarrow 400^+} P(x) \Leftrightarrow 1800=4\cdot 400+k \Leftrightarrow k=200$.\\
% 			Vậy với $k=200$ thì hàm số liên tục trên $(0 ;+\infty)$.
% 		\end{enumerate}
% 	}
% \end{bt}

% \begin{bt}%[DCHT Toán 11 - KNTT -Vũ Hồng Toàn]%[1K5YG-6]
% 	\immini{
% 		Cho hàm số $y=f(x)$ có đồ thị như hình vẽ bên. Xét tính liên tục của hàm số $y=f(x)$ trên tập xác định của nó.
% 	}{
% 		\begin{tikzpicture}[scale=.75, font=\footnotesize, line join=round, line cap=round,>=stealth]
% 			\def\a{-1} \def\b{2} \def\c{1} \def\d{-1}
% 			\pgfmathsetmacro\tcd{int(round(-\d/\c))} \pgfmathsetmacro\tcn{int(round(\a/\c))} 
% 			\pgfmathsetmacro\xmin{\tcd-2.5} \pgfmathsetmacro\xmax{\tcd+2.5}
% 			\pgfmathsetmacro\ymin{\tcn-2.5} \pgfmathsetmacro\ymax{\tcn+2.5}
% 			%\draw[color=gray!50,dashed] (\xmin,\ymin) grid (\xmax,\ymax);
% 			\draw[->] (\xmin,0)--(0,0)node [below right]{$O$}-- (\xmax,0) node [below]{$x$};
% 			\draw[->] (0,\ymin)--(0,\ymax) node [left]{$y$};
% 			\draw[dash pattern=on 2pt off 1.5pt] (\tcd,\ymin)--(\tcd,\ymax) (\tcd,0)node [below right]{$\tcd$}
% 			(\xmin,\tcn)--(\xmax,\tcn) (0,\tcn)node [below left]{$\tcn$};
% 			\begin{scope}
% 				\clip (\xmin,\ymin) rectangle (\xmax-.1,\ymax-.1);
% 				\draw[samples=200,smooth,variable=\x,thick, teal] plot[domain=\xmin:\tcd-0.3] (\x,{(\a*(\x)+\b)/(\c*(\x)+\d)});
% 				\draw[samples=200,smooth,variable=\x,thick, teal] plot[domain=\tcd+.3:\xmax] (\x,{(\a*(\x)+\b)/(\c*(\x)+\d)});
% 			\end{scope}
% 			\draw[ thick] (1.5,1)node[right]{$y=f(x)$};
% 			\foreach \x/\y in{0/0,1/0} \fill(\x,\y)circle(.03);
% 		\end{tikzpicture}	
% 	}
% 	% \dapso{Hàm số liên tục trên các khoảng $(-\infty;1)$, $(1; +\infty)$ và gián đoạn tại $x_0=1$.}
% 	\loigiai
% 	{
% 		Tập xác định $\mathscr{D}=\mathbb{R}\setminus \{1\}$.	
% 		\begin{itemize}
% 			\item Đồ thị hàm số là các đường liền nét trên các khoảng $(-\infty;1)$, $(1;+\infty)$ do đó hàm số liên tục trên các khoảng này.
% 			\item Ta có $\lim\limits_{x\to{1}^-}f(x)=-\infty$ và $\lim\limits_{x\to{1}^+}f(x)=+\infty$. Do đó $\lim\limits_{x\to{1}^-}f(x)\ne \lim\limits_{x\to{1}^+}f(x)$.\\
% 			Vậy hàm số đã cho gián đoạn tại $x_0=1$.
% 		\end{itemize}
% 	}
% \end{bt}

% \begin{bt}%[1K5KG-6]
% 	Một bảng giá cước taxi được cho như sau:
	
% 	\begin{tabular}{|c|c|c|}
% 		\hline 
% 		Giá mở cửa ($0,5$ km đầu)& Giá cước các km tiếp theo đến $30$ km& Giá cước từ km thứ $31$\\
% 		\hline
% 		$10\ 000$ đồng & $13\ 500$ đồng & $11\ 000$ đồng\\
% 		\hline
% 	\end{tabular}
% 	\begin{enumEX}{1}
% 		\item Viết công thức hàm số mô tả số tiền khách phải trả theo quãng đường di chuyển.
% 		\item Xét tính liên tục của hàm số ở câu a.
% 	\end{enumEX}
% 	% \dapso{Hàm số liên tục trên $(0;+\infty)$}
% 	\loigiai{
% 		a) Gọi $x$ là quãng đường di chuyển, $f(x)$ là giá tiền tính theo quãng đường. 
% 		\begin{itemize}
% 			\item $0\le x\le 0{,}5$, ta có $f(x)=10000$ đồng.
% 			\item $0{,}5<x \le 30$, $f(x)= 10000+13500(x-0{,}5)$ đồng.
% 			\item  $x>30$, $f(x)= 408250+11000(x-30)$ đồng.
% 		\end{itemize}
% 		Vậy $f(x)=\heva{&10000&\text{nếu }& 0\le x\le 0{,}5\\ &10000+13500(x-0{,}5)&\text{nếu }& 0{,}5<x\le 30\\
% 			&408250+11000(x-30)&\text{nếu }& x>30. } $ 
% 		\\
% 		b) Hàm số $f(x)$ liên tục trên các khoảng $(0; 0{,}5)$, $(0{,}5; 30)$ và $(30;+\infty)$. \\
% 		Tại $x=0{,}5$, ta có $f(0{,}5)=10000$, $\lim\limits_{x\to {0{,}5}^+}{f(x)}=10000$, $\lim\limits_{x\to {0{,}5}^-}{f(x)}=10000$.\\
% 		Vì $f(0,5)=\lim\limits_{x\to {0{,}5}^+}{f(x)}=\lim\limits_{x\to {0{,}5}^-}{f(x)}$, do đó $f(x)$ liên tục tại $x=0,5$.\\
% 		Tại $x=30$, ta có $f(30)=408250$, $\lim\limits_{x\to 30^-}{f(x)}=408250$, $\lim\limits_{x\to 30^+}{f(x)}=408250$.\\
% 		Vì $f(30)=\lim\limits_{x\to 30^-}{f(x)}=\lim\limits_{x\to 30^+}{f(x)}$, do đó$f(x)$ liên tục tại $x=30$. \\
% 		Vậy $f(x)$ liên tục trên khoảng $(0;+\infty)$.
% 	}
% \end{bt}

% \begin{bt}%[1T3B3-6]
% 	Một bãi đậu xe ô-tô đưa ra giá $C(x)$ (đồng) khi thời gian đậu xe là $x$ (giờ) như sau: $$C(x)=\heva{&60.000&\quad\text{khi }&0<x\le2\\&100.000&\quad\text{khi }&2<x\le4\\&200.000&\quad\text{khi }&4<x\le24.}$$
% 	Xét tính liên tục của hàm số $C(x)$.
% 	% \dapso{Hàm số $C(x)$ liên tục trên từng khoảng $(0;2)$, $(2;4)$, $(4;6)$}
% 	\loigiai{\begin{itemize}
% 			\item Hàm số $C(x)$ là hàm hằng trên từng khoảng $(0;2)$, $(2;4)$, $(4;6)$ nên liên tục trên từng khoảng đó.
% 			\item Ta có $\heva{&\lim\limits_{x\to2^-}C(x)=60.000\\&\lim\limits_{x\to2^+}C(x)=100.000}\Rightarrow$ không tồn tại $\lim\limits_{x\to2}C(x)$, vậy $C(x)$ không liên tục tại $x_0=2$.
% 			\item Ta có $\heva{&\lim\limits_{x\to4^-}C(x)=100.000\\&\lim\limits_{x\to4^+}C(x)=200.000}\Rightarrow$ không tồn tại $\lim\limits_{x\to4}C(x)$, vậy $C(x)$ không liên tục tại $x_0=4$.
% 		\end{itemize}
% 		Vậy hàm số $C(x)$ liên tục trên từng khoảng $(0;2)$, $(2;4)$, $(4;6)$.}
% \end{bt}
% \subsubsection{Câu hỏi trắc nghiệm}
% \Opensolutionfile{ans}[ans/ans-1K5-3-Dang5]
% \begin{ex}%[DCHT Toán 11 - KNTT -Đỗ Minh Phúc]%[1K5BG-6]
% 	\immini{Hình bên cạnh biểu thị độ cao $ h $ (m) của một quả bóng được đá lên thời gian $ t $ (s), trong đó $ h(t)= -2t^{2}+8t$. Kết luận nào sau đây là đúng?
% 		\choice
% 		{\True Hàm số $h(t)$ liên tục trên  $(0;4)$}
% 		{Hàm số $h(t)$ liên tục trên  $(0;8)$}
% 		{Hàm số $h(t)$ liên tục trên  $(-1;4)$}
% 		{$ \displaystyle\lim\limits_{t\to 2}\left(-2t^{2}+8t\right)=2$}
% 	}
% 	{		\begin{tikzpicture}[>=stealth,x=1cm,y=1cm,scale=0.5,font=\tiny]
% 			\def\a{-2}
% 			\def\b{8}
% 			\def\c{0}
% 			\draw[->] (-2,0) -- (6,0) node[below] {\scriptsize $t(s)$};
% 			\draw[->] (0,-2) -- (0,9) node[left] {\scriptsize $h(m)$};
% 			\draw (0,0)node[below left]{\scriptsize $O$};
% 			\fill (0,4) node[left]{$4$};
% 			\fill (0,8) node[left]{$8$};
% 			\fill (2,0) node[below]{$2$};
% 			\draw[dashed] (2,0)--(2,8)--(0,8);
% 			\pgfmathsetmacro\xdinh{-(\b)/2*(\a)}
% 			\pgfmathsetmacro\ydinh{(4*(\a)*(\c)-(\b)^2)/(4*(\a))}
% 			\clip (-2,-2)rectangle(6,9);
% 			\draw[red,thick,samples=150,smooth,domain=0:4] plot(\x,{\a*(\x)^2+(\b)*\x+(\c)});
% 			\foreach \x in {2,4} \draw (\x,0) circle (1pt);
% 			\foreach \y in {4,8} \draw (0,\y) circle (1pt);
% 	\end{tikzpicture}}
% 	\loigiai{
% 		Đồ thị hàm số cắt trục hoành tại $t=0$ và $t=4$.  \\
% 		Từ hình vẽ ta thấy tập xác định của $h(t)$ là $[0;4]$. Suy ra  hàm số $h(t)$ liên tục trên $(0;4)$.
% 	}
% \end{ex}

% \begin{ex}%[DCHT Toán 11 - KNTT -Đỗ Minh Phúc]%[1K5BG-6]
% 	Lực hấp dẫn do Trái Đất tác dụng lên một đơn vị khối lượng ở khoảng cách $r$ tính từ tâm của nó là $F(r)=\heva{&\dfrac{GMr}{R^3}&\quad\text{khi }&0<r\le R\\&\dfrac{GM}{r^2}&\quad\text{khi }&r\ge R}$, trong đó $M$ là khối lượng, $R$ là bán kính của Trái Đất, $G$ là hằng số hấp dẫn. Kết luận nào sau đây là đúng?
% 	\choice
% 	{Hàm số $F(r)$ liên tục trên $\mathbb{R}$}
% 	{Hàm số $F(r)$ liên tục tại điểm $r=0$}
% 	{Hàm số $F(r)$ liên tục trên $(-\infty;0)$}
% 	{\True Hàm số $F(r)$ liên tục trên $(0;+\infty)$}
% 	\loigiai{\begin{itemize}
% 			\item Với mọi $r\in(0;R)$, hàm số $F(r)=\dfrac{GMr}{R^3}$ luôn xác định nên liên tục tại đó.
% 			\item Với mọi $r\in(R;+\infty)$, hàm số $F(r)=\dfrac{GM}{r^2}$ luôn xác định nên liên tục tại đó.
% 			\item Ta có $\heva{&\lim\limits_{r\to R^-}F(r)=\lim\limits_{r\to R^-}\dfrac{GMr}{R^3}=\dfrac{GM}{R^2}\\&\lim\limits_{x\to R^+}F(r)=\lim\limits_{x\to R^+}\dfrac{GM}{r^2}=\dfrac{GM}{R^2}\\&F(R)=\dfrac{GM}{R^2}}$ nên hàm số $F(r)$ liên tục tại $r=R$.
% 		\end{itemize}
% 		Vậy hàm số $F(r)$ liên tục trên $(0;+\infty)$.}
% \end{ex}

% \begin{ex}%[DCHT Toán 11 - KNTT -Đỗ Minh Phúc]%[1K5BG-6]
% 	Trong một phòng thí nghiệm, nhiệt độ trong tủ sấy được điều khiển tăng từ $10^{\circ} \mathrm{C}$, mỗi phút tăng $2^{\circ} \mathrm{C}$ trong $60$ phút, sau đó giảm mỗi phút $3^{\circ} \mathrm{C}$ trong $40$ phút. Hàm số biểu thị nhiệt độ (tính theo $^{\circ} \mathrm{C}$ ) trong tủ theo thời gian $t$ (tính theo phút) có dạng
% 	\[T(t)= \heva{&10+2 t & \text { khi } 0 \leq t \leq 60 \\ &k-3 t & \text { khi } 60<t \leq 100}\ (k \text{ là hằng số}).\]
% 	Biết rằng, $T(t)$ là hàm liên tục trên tập xác định. Tìm giá trị của $k$.
% 	\choice
% 	{$k=0$}
% 	{$k=60$}
% 	{\True $k=310$}
% 	{$k=100$}
% 	\loigiai{
% 		Vì $T(t)$ là hàm liên tục trên tập xác định nên ta có hàm $T(t)$ liên tục tại $t=60$\\
% 		$\Leftrightarrow \lim\limits_{t\to 60^-} T(t)=\lim\limits_{t\to 60^+} T(t)=T(60)$\\
% 		$\Leftrightarrow 10+2\cdot 60=k-3\cdot 60\Leftrightarrow k=310$.\\
% 		Vậy $k=310$.
% 	}
% \end{ex}

% \begin{ex}%[DCHT Toán 11 - KNTT -Đỗ Minh Phúc]%[1K5BG-6]
% 	Một bảng giá cước taxi được cho như sau:
% 	\begin{center}
% 		\begin{tabular}{|c|c|c|}
% 			\hline 
% 			Giá mở cửa ($0,5$ km đầu)& Giá cước các km tiếp theo đến $30$ km& Giá cước từ km thứ $31$\\
% 			\hline
% 			$10\ 000$ đồng & $13\ 500$ đồng & $11\ 000$ đồng\\
% 			\hline
% 		\end{tabular}
% 	\end{center}
% 	Gọi $x$ là quãng đường di chuyển, $f(x)$ là giá tiền tính theo quãng đường có công thức như sau:
% 	\[f(x)=\heva{&10000&\text{nếu }& 0\le x\le 0{,}5\\ &10000+13500(x-0{,}5)&\text{nếu }& 0{,}5<x\le 30\\
% 		&408250+11000(x-30)&\text{nếu }& x>30. }\]
% 	Kết luận nào sau đây là đúng?
% 	\choice
% 	{Hàm số $f(x)$ liên tục trên  $\mathbb{R}$}
% 	{\True Hàm số $f(x)$ liên tục trên  $(0;+\infty)$}
% 	{Hàm số $f(x)$ liên tục trên  $(-\infty;0)$}
% 	{Hàm số $f(x)$ liên tục trên  $(0;30)$}
% 	\loigiai{
% 		Hàm số $f(x)$ liên tục trên các khoảng $(0; 0{,}5)$, $(0{,}5; 30)$ và $(30;+\infty)$. \\
% 		Tại $x=0{,}5$, ta có $f(0{,}5)=10000$, $\lim\limits_{x\to {0{,}5}^+}{f(x)}=10000$, $\lim\limits_{x\to {0{,}5}^-}{f(x)}=10000$.\\
% 		Vì $f(0,5)=\lim\limits_{x\to {0{,}5}^+}{f(x)}=\lim\limits_{x\to {0{,}5}^-}{f(x)}$, do đó $f(x)$ liên tục tại $x=0,5$.\\
% 		Tại $x=30$, ta có $f(30)=408250$, $\lim\limits_{x\to 30^-}{f(x)}=408250$, $\lim\limits_{x\to 30^+}{f(x)}=408250$.\\
% 		Vì $f(30)=\lim\limits_{x\to 30^-}{f(x)}=\lim\limits_{x\to 30^+}{f(x)}$, do đó$f(x)$ liên tục tại $x=30$. \\
% 		Vậy $f(x)$ liên tục trên khoảng $(0;+\infty)$.
% 	}
% \end{ex}
% \Closesolutionfile{ans}
% \begin{indapan}{10}
% 	{ans/ans-1K5-3-Dang5}
% \end{indapan}
\begin{dang}{Chứng minh phương trình có nghiệm}
	Để chứng minh minh một phương trình có nghiệm, ta thường tiến hành
	\begin{itemize}
		\item Đặt $f(x)$ là vế trái của phương trình (ứng với vế phải bằng $0$).
		\item Lập luận hàm số $f(x)$ liên tục trên $\mathbb{R}$ hoặc trên một đoạn con của $\mathbb{R}$ liên quan tới bài toán.
		\item Chỉ ra tồn tại các số $a$, $b$ ($a<b$) với $a$, $b$ thuộc đoạn con đang xét mà $f(a)\cdot f(b)<0$. Dựa vào tính chất của hàm số liên tục ta suy ra phương trình $f(x)=0$ có nghiệm thuộc khoảng $(a;b)$.		
		\begin{note}
			Nếu bài toán yêu cầu chứng minh phương trình có $k$ nghiệm thì cần lập luận $k$ đoạn con như trên.\\
			Nhiều trường hợp việc chỉ ra các số $a$, $b$ gặp khó khăn, ta có thể khai thác $\lim \limits _{x\to +\infty}f(x)$ hoặc $\lim \limits _{x\to -\infty}f(x)$ để có cơ sở lập luận. 
		\end{note}
	\end{itemize}
\end{dang}
\subsubsection{Ví dụ mẫu}
\begin{vd}%%[NB]%[DCHT Toán 11 - KNTT -Nguyễn Thành Nhân] %[1K5YG-7]
	Chứng minh rằng phương trình $x^5+4x^3-x^2-1=0$ có ít nhất một nghiệm thuộc khoảng $(0;1)$.
	\loigiai{
		Đặt $f(x) = x^5+4x^3-x^2-1$. Khi đó $f(x)$ liên tục trên $\mathbb{R}$ nên cũng liên tục trên đoạn $[0;1]$.\\
		Ta có $f(0) = -1$ và $f(1) = 3$ nên $f(0) \cdot f(1) < 0$.\\
		Do đó phương trình $f(x) =0$ có ít nhất một nghiệm thuộc $(0;1)$.
	}
\end{vd}
\begin{vd}%[TH]%[DCHT Toán 11 - KNTT -Nguyễn Thành Nhân] %[1K5BG-7]
	Chứng minh rằng phương trình $x^5-5x^3+4x-1=0$ có đúng $5$ nghiệm phân biệt.
	\loigiai{
		Đặt $f(x) = x^5-5x^3+4x-1$. Khi đó $f(x)$ liên tục trên $\mathbb{R}$.\\
		Ta có $f(-2)=-1<0$; $f\left(-\dfrac{3}{2}\right)=\dfrac{173}{32}>0$; $f(0)=-1<0$; $f\left(\dfrac{1}{2}\right)=\dfrac{18}{32}>0$; $f(1)=-1<0$; $f(3)=119>0$. \\
		Dựa vào tính chất liên tục của hàm số $f(x)$ trên $\mathbb{R}$, suy ra trên mỗi khoảng $\left(-2;-\dfrac{3}{2}\right)$; $\left(-\dfrac{3}{2};0\right)$; $\left(0;\dfrac{1}{2}\right)$; $\left(\dfrac{1}{2};1\right)$; $(1;3)$ có ít nhất một nghiệm. \\
		Do đó phương trình $f(x)=0$ có ít nhất $5$ nghiệm phân biệt. Vì $f(x)$ là phương trình bậc $5$ nên có tối đa $5$ nghiệm.\\
		Vậy phương trình $f(x)=0$ có đúng $5$ nghiệm phân biệt.
	}
\end{vd}

\begin{vd}%[TH]%[DCHT Toán 11 - KNTT -Nguyễn Thành Nhân] %[1K5BG-7]
	Chứng minh rằng phương trình $x^3-2mx^2-x+m=0$ luôn có nghiệm với mọi $m$ ($m$ là tham số).
	\loigiai{
		Xét hàm số $f(x)=x^3-2mx^2-x+m$. Khi đó $f(x)$ liên tục trên $\mathbb{R}$.\\
		Ta có $f(0)=m$ và $f(1)=-m$ nên $f(0)\cdot f(1)=-m^2\le 0$ với mọi $m$ nên phương trình $f(x)=0$ luôn có nghiệm thuộc đoạn $[0;1]$ với mọi $m$.
	}
\end{vd}

\begin{vd}%[TH]%[DCHT Toán 11 - KNTT -Nguyễn Thành Nhân] %[1K5BG-7]
	Chứng minh rằng phương trình $\left(1-m^2\right)(x+1)^3+x^2-x-3$ luôn có nghiệm với mọi giá trị của tham số $m$. 
	\loigiai{
		Đặt $f(x)= \left(1-m^2\right)(x+1)^3+x^2-x-3$ thì $f(x)$ liên tục trên $\mathbb{R}$. Ta có
		\[f(0)=-m^2-2<0,\,\forall m.\]
		và 
		\[f(-2)=m^2+2>0,\,\,\forall m.\]
		Vì $f(-2)\cdot f(0)<0$ nên phương trình $f(x)=0$ luôn có ít nhất một nghiệm thuộc khoảng $(-2;0)$ với mọi $m$.
	}
\end{vd}
\begin{vd}%[TH]%[DCHT Toán 11 - KNTT -Nguyễn Thành Nhân] %[1K5BG-7]
	Chứng minh rằng phương trình $m(x-8)^3(x-9)^4+2x-17=0$ luôn có nghiệm với mọi giá trị của $m$.
	\loigiai{
		Xét hàm số $f(x)=	m(x-8)^3(x-9)^4+2x-17$. Hàm số đã cho liên tục trên $\mathbb{R}$.\\
		Ta có $f(8)=-1, f(9)=1$. Vậy $f(8)\cdot f(9)<0$. Điều này suy ra phương trình có ít nhất một nghiệm trên $(8;9)$.
	}
\end{vd}
\begin{vd}%[VD]%[DCHT Toán 11 - KNTT -Nguyễn Thành Nhân]%[1K5KG-7]
	Chứng minh rằng phương trình $m(x+1)^2(x-2)^3+(x+2)(x-3)=0$ luôn có nghiệm với mọi tham số $m$.
	\loigiai{
		Xét hàm số $f(x)=m(x+1)^2(x-2)^3+(x+2)(x-3)$ xác định và liên tục trên $[-2;3]$.\\
		Ta có $f(-2)=-64m$, $f(3)	
		=16m$, $f(-2) \cdot f(3)=-1024m^2 \le 0$.
		\begin{itemize}
			\item Với $m=0$ suy ra $f(-2)=f(3)=0$ suy ra phương trình $f(x)=0$ có hai nghiệm $x=-2$ và $x=3$.
			\item Với $m \ne 0$ suy ra $f(-2) \cdot f(3) <0$, suy ra tồn tại $x_0 \in (-2;3)$ sao cho $f(x_0)=0$.
		\end{itemize}
		Do đó phương trình $f(x)=0$ luôn có nghiệm.\\
		Vậy phương trình ban đầu luôn có nghiệm.
	} 
\end{vd}
\begin{vd}%[VDC]%[DCHT Toán 11 - KNTT -Nguyễn Thành Nhân]%[1K5GG-7]
	Với mọi giá trị thực của tham số $m$, chứng minh phương trình $\left(m^2+1\right)x^3-2m^2x^2-4x+m^2+1=0$ luôn có ba nghiệm thực.
	\loigiai
	{
		Đặt $f(x)=\left(m^2+1\right)x^3-2m^2x^2-4x+m^2+1$.\\
		Hàm số $f(x)=\left(m^2+1\right)x^3-2m^2x^2-4x+m^2+1$ là một hàm số đa thức nên nó liên tục trên $\mathbb{R}$. Suy ra, nó cũng liên tục trên mỗi đoạn $[-3;0]$, $[0;1]$, $[1;2]$.
		Ta có
		\begin{itemize}
			\item $f(-3) =-27m^2-27-18m^2+12+m^2+1=-44m^2-14<0$, với mọi $m\in\mathbb{R}$.
			\item $f(0)=m^2+1>0$, với mọi $m\in\mathbb{R}$.
			\item $f(1)=m^2+1-2m^2-4+m^2+1=-2<0$.
			\item $f(2)=8m^2+8-8m^2-8+m^2+1=m^2+1>0$, với mọi $m\in\mathbb{R}$.
		\end{itemize}
		Vì $f(-3)\cdot f(0)<0$ nên phương trình đã cho có ít nhất một nghiệm thuộc khoảng $(-3;0)$.\\
		Vì $f(0)\cdot f(1)<0$ nên phương trình đã cho có ít nhất một nghiệm thuộc khoảng $(0;1)$.\\
		Vì $f(1)\cdot f(2)<0$ nên phương trình đã cho có ít nhất một nghiệm thuộc khoảng $(1;2)$.\\
		Phương trình $\left(m^2+1\right)x^3-2m^2x^2-4x+m^2+1=0$ là một phương trình bậc ba $(\text{vì } m^2+1\neq 0,\forall m\in\mathbb{R})$.\\
		Vậy phương trình $\left(m^2+1\right)x^3-2m^2x^2-4x+m^2+1=0$ luôn có ba nghiệm thực.
	}
\end{vd}
\begin{vd}%[VD]%[DCHT Toán 11 - KNTT -Nguyễn Thành Nhân]%[1K5KG-7]
	Chứng minh rằng phương trình sau luôn có nghiệm với mọi giá trị của tham số $m\ge -1$
	\[ (m-1)x^6+\left(m^2-\sqrt{4m+4}\right)x^3+6x-3=0. \]
	\loigiai{
		Đặt $f(x)=(m-1)x^6+\left(m^2-\sqrt{4m+4}\right)x^3+6x-3$. Khi đó $f(x)$ liên tục trên đoạn $[0;1]$. Ta có 
		\begin{align*}
			f(1)&=(m-1)+\left(m^2-\sqrt{4m+4}\right)+6-3\\
			&=m^2+(m+1)-2\sqrt{m+1}+1\\
			&=m^2+\left(\sqrt{m+1}-1\right)^2.
		\end{align*}
		\begin{itemize}
			\item Nếu $m=\sqrt{m+1}-1=0$ hay $m=0$ thì $f(1)=0$.
			\item Nếu $m\ne 0$ thì $f(1)>0$, mà $f(0)=-3<0$ nên $f(x)$ có một nghiệm trong khoảng $(0;1)$.
		\end{itemize}
		Vậy phương trình $f(x)=0$ luôn có nghiệm với mọi $m\ge -1$. 
	}
\end{vd}
\begin{vd}%[VDC]%[DCHT Toán 11 - KNTT -Nguyễn Thành Nhân]%[1K5GG-7]
	Với mọi giá trị thực của tham số $m,$ chứng minh phương trình $x^5+x^2-\left(m^2+2\right)x-1=0$ luôn có ít nhất ba nghiệm thực.
	\loigiai{
		Xét hàm số $ f(x)=x^5+x^2-\left(m^2+2\right)x-1$ liên tục trên $\mathbb R$.\\
		Ta có $ f(0)=-1 < 0$, $f(-1)=m^2+1 > 0$.\\
		Mặt khác, vì $\lim\limits_{x\to-\infty}f(x)=-\infty$ nên tồn tại $a <-1$ sao cho $f(a) < 0$.\\
		Vì $\lim\limits_{x\to+\infty}f(x)=+\infty$ nên tồn tại $b > 0$ sao cho $f(b) > 0$.\\
		Khi đó
		\begin{itemize}
			\item $ f(a)\cdot f\left(-1\right) < 0$ suy ra phương trình $ f(x)=0$ có ít nhất $ 1$ nghiệm thuộc $\left(a;-1\right)$,
			\item $ f\left(-1\right)\cdot f(0) < 0$ suy ra phương trình $ f(x)=0$ có ít nhất $ 1$ nghiệm thuộc $\left(-1;0\right)$,
			\item $ f(0)\cdot f(b) < 0$ suy ra phương trình $ f(x)=0$ có ít nhất $ 1$ nghiệm thuộc $\left(0;b\right)$.
		\end{itemize}
		Vậy phương trình đã cho có ít nhất $ 3$ nghiệm.
	}
\end{vd}
\begin{vd}%[VDC]%[DCHT Toán 11 - KNTT -Nguyễn Thành Nhân]%[1K5GG-7]
	Cho $a$, $b$ là hai số thực thỏa mãn $9 a+24 b>128$. Chứng minh phương trình $a x^2+b x-2=0$ có ít nhất một nghiệm thuộc khoảng $(0 ; 1)$.	
	\loigiai{
		Xét hàm số $ f(x)=x^2+b x-2$ liên tục trên $\mathbb R$.\\		
		Ta có $f(0)=-2<0$.\\
		Ta có $f\left(\dfrac{1}{2}\right)=\dfrac{a}{4}+\dfrac{b}{2}-2$, $f\left(1\right)=a+b-2$.\\
		Khi đó $2f(1)+4f\left(\dfrac{1}{2}\right)=3a+4b-12>\dfrac{128}{3}-12=\dfrac{92}{3}>0$. Suy ra một trong hai số $f(1)$  hoặc $f\left(\dfrac{1}{2}\right)$ là số dương.\\
		Do đó $\hoac{&f(0)\cdot f\left(\dfrac{1}{2}\right)<0\\&f(0)\cdot f(1)<0.}$\\
		Khi đó phương trình $f(x)=0$ có ít nhất một nghiệm thuộc khoảng $(0;1)$.
	}
\end{vd}
\begin{vd}%[VDC]%[DCHT Toán 11 - KNTT -Nguyễn Thành Nhân]%[1K5GG-7]
	Cho phương trình $ax^2+bx+c=0$ với $5a+3b+3c=0$. Chứng minh rằng phương trình luôn có nghiệm.	
	\loigiai{Do $5a+3b+3c=0$ nên $b=-\dfrac{5}{3}a-c$.\\
		Xét hàm số  $f(x)=ax^2+bx+c$ trên $\left[ 0;\dfrac{5}{3}\right] $.\\
		Ta có $f(0)=c$, $f\left( \dfrac{5}{3} \right)=\dfrac{25}{9}\cdot a+\dfrac{5}{3}\cdot b+c =\dfrac{25}{9}\cdot a+\dfrac{5}{3}\left(-\dfrac{5}{3}a-c \right)+c=-\dfrac{2}{3}c  $.
		\begin{itemize}
			\item Nếu $c=0$ thì $f(0)=f\left(\dfrac{5}{3} \right)=0 $, phương trình đã cho có hai nghiệm là $x=0$, $x=\dfrac{5}{3}$.
			\item Nếu $c\ne 0$ thì $f(0)\cdot f\left(\dfrac{5}{3} \right)=-\dfrac{2}{3}c^2<0 $. Vì $f(x)$ là hàm đa thức nên liên tục trên $\mathbb{R}$.\\
			Do đó, nó liên tục trên $\left[ 0;\dfrac{5}{3}\right] $.\\
			Từ đó suy ra phương trình $f(x)=0$  có ít nhất một nghiệm trên $\left( 0;\dfrac{5}{3}\right)$.
		\end{itemize}
		Vậy phương trình đã cho luôn có nghiệm.
	}
\end{vd}
\subsubsection{Bài tập rèn luyện}
% \centerline{\fcolorbox{red}{yellow!50}{\bf {BÀI TẬP TỰ LUẬN }}}
\begin{bt}%%%[NB]%[DCHT Toán 11 - KNTT -Tên GV] %[1K5YG-7]
	Chứng minh rằng phương trình $x^5+4x^3-x^2-1=0$ có ít nhất một nghiệm thuộc khoảng $(0;1)$.
	\loigiai{
		Đặt $f(x) = x^5+4x^3-x^2-1$ liên tục trên $[0;1]$.\\
		Ta có $f(0) = -1$ và $f(1) = 3$ nên $f(0) \cdot f(1) < 0$.\\
		Do đó phương trình $f(x) =0$ có ít nhất một nghiệm thuộc $(0;1)$.
	}
\end{bt}
\begin{bt}%%%[NB]%[DCHT Toán 11 - KNTT -Tên GV] %[1K5BG-7]
	Chứng minh rằng phương trình $ 2x^4-3x^3-5=0 $ có ít nhất một nghiệm.
	\loigiai{
		Đặt $f(x)= 2x^4-3x^3-5 $, $ f(x) $ là hàm đa thức nên liên tục trên $ \mathbb{R} $.\\
		Do đó $ f(x) $ liên tục trên đoạn $ [1;2] $.\\
		Ta có $ \heva{&f(1)=-6\\&f(2)=3} $ suy ra $ f(1)\cdot f(2)=-18<0 $.\\
		Nên phương trình $ f(x)=0 $ có ít nhất một nghiệm nằm trong khoảng $ (1;2) $.\\
		Vậy phương trình đã cho có ít nhất một nghiệm.
	}
\end{bt}
\begin{bt}%[TH]%[DCHT Toán 11 - KNTT -Nguyễn Thành Nhân]%[1K5BG-7]
	Chứng minh rằng phương trình $-3x^5+8x^2-1=0$ có ít nhất một nghiệm.
	\loigiai{
		Xét hàm số $f(x)=-3x^5+8x^2-1$ liên tục trên $\mathbb{R}$ nên liên tục trên đoạn $[0;1]$. Ta có $f(0)\cdot f(1)=-1\cdot 4=-4<0$.\\
		Vậy có ít nhất một số $x_0=c\in \left(0;1\right)$ để $f(x_0)=0$, hay nói cách khác phương trình $f(x)=0\Leftrightarrow -3x^5+8x^2-1=0$ có nghiệm.
	}
\end{bt}
\begin{bt}%[TH]%[DCHT Toán 11 - KNTT -Nguyễn Thành Nhân] %[1K5BG-7]
	Chứng minh phương trình $\left(m^2-2m+3\right)x^4-2x-4=0$ luôn có nghiệm âm với mọi giá trị thực của tham số $m$.
	\loigiai
	{Hàm số $f(x)=\left(m^2-2m+3\right)x^4-2x-4$ liên tục trên $\mathbb{R}$.\\
		Ta có $f(0)=-4$.\\
		Vì $m^2-2m+3=(m-1)^2+2>0,\ \forall m$ nên $$\lim\limits_{x\to -\infty} f(x)=\lim\limits_{x\to -\infty} x^4\left(m^2-2m+3-\dfrac{2}{x^3}-\dfrac{4}{x^4}\right)=+\infty.$$
		Do đó, tồn tại $a<0$ sao cho $f(a)>0$.\\
		Vì hàm số $f(x)$ liên tục trên $\mathbb{R}$ nên nó cũng liên tục trên $[a; 0]$. Hơn nữa $f(a)\cdot f(0)<0$ nên phương trình $f(x)=0$ luôn có ít nhất một nghiệm thuộc khoảng $(a; 0)$.\\
		Vậy phương trình đã cho luôn có nghiệm âm với mọi giá trị thực của tham số $m$.
	}
\end{bt}

\begin{bt}%[TH]%[DCHT Toán 11 - KNTT -Tên GV] %[1K5BG-7]
	Chứng minh phương trình $x^4 + x^3 + mx^2 + x\left(2m - 1\right) + m\sin \left(\pi x\right)=1$ có nghiệm với mọi $m$. 
	\loigiai{
		$x^4 + x^3 + mx^2 + x\left(2m - 1\right) + m\sin \left(\pi x\right)=1\Leftrightarrow x^4 + x^3 - x - 1 + m\left(x^2 + \sin \pi x + 2x\right)=0$.\\
		Đặt $f(x)=x^4 + x^3 - x - 1 + m\left(x^2 + \sin \pi x + 2x\right)$.\\
		Ta có $f(x)$ liên tục trên $\mathbb{R}$.\\
		$f\left(0\right)= - 1$ và $f\left(- 2\right)=9\Rightarrow f(0)\cdot f(-2)<0$.\\
		Suy ra phương trình đã cho luôn có ít nhất một nghiệm thuộc khoảng $ (-2;0) $.
	}
\end{bt}

\begin{bt}%[TH]%[DCHT Toán 11 - KNTT -Tên GV] %[1K5BG-7]
	Chứng minh phương trình $x^4+mx^2+(3m-1)x-5+2m=0$ luôn có ít nhất một nghiệm với mọi số thực $m$.
	\loigiai{
		Đặt $f(x)=x^4+mx^2+(3m-1)x-5+2m$.\\
		Vì $f(x)$ là hàm đa thức nên $f(x)$ liên tục trên $\mathbb{R}$.\\
		Lại có $f(-2)=13, f(-1)=-3$	$\Rightarrow f(-3) \cdot f(-1)<0$, do đó phương trình $f(x)=0$ có ít nhất một nghiệm trên $(-3;-1)$.\\
		Do đó phương trình luôn có nghiệm với mọi $m$.
	}
\end{bt}

\begin{bt}%[TH]%[DCHT Toán 11 - KNTT -Tên GV] %[1K5BG-7]
	Chứng minh rằng phương trình $m(x-8)^3(x-9)^4+2x-17=0$ luôn có nghiệm với mọi giá trị của $m$.
	\loigiai{
		Xét hàm số $f(x)=	m(x-8)^3(x-9)^4+2x-17$. Hàm số đã cho liên tục trên $\mathbb{R}$.\\
		Ta có $f(8)=-1, f(9)=1$. Vậy $f(8)\cdot f(9)<0$. Điều này suy ra phương trình có ít nhất một nghiệm trên $(8;9)$.
	}
\end{bt}
\begin{bt}%[TH]%[DCHT Toán 11 - KNTT -Tên GV] %[1K5BG-7]
	Chứng minh phương trình $\left(m^2-2m+3\right)x^4-2x-4=0$ luôn có nghiệm âm với mọi giá trị thực của tham số $m$.
	\loigiai
	{Hàm số $f(x)=\left(m^2-2m+3\right)x^4-2x-4$ có tập xác định là $\mathscr{D}=\mathbb{R}$.\\
		Ta có $f(0)=-4$.\\
		Vì $m^2-2m+3=(m-1)^2+2>0,\ \forall m$ nên $$\lim\limits_{x\to -\infty} f(x)=\lim\limits_{x\to -\infty} x^4\left(m^2-2m+3-\dfrac{2}{x^3}-\dfrac{4}{x^4}\right)=+\infty.$$
		Do đó, tồn tại $a<0$ sao cho $f(a)>0$.\\
		Vì hàm số $f(x)$ liên tục trên $\mathbb{R}$ nên nó cũng liên tục trên $[a; 0]$. Hơn nữa $f(a)\cdot f(0)<0$ nên phương trình $f(x)=0$ luôn có ít nhất một nghiệm thuộc khoảng $(a; 0)$.\\
		Vậy phương trình đã cho luôn có nghiệm âm với mọi giá trị thực của tham số $m$.
	}
\end{bt}
\begin{bt}%[TH]%[DCHT Toán 11 - KNTT -Tên GV] %[1K5BG-7]
	Chứng minh rằng phương trình $m\cdot\sin 2x+x^2\cdot\cos x+\left(m^2+1\right)\cdot\cos 2x=0$ luôn có nghiệm thuộc khoảng $\left(0 ;\dfrac{\pi}{2}\right)$ với mọi tham số $m$.
	\loigiai{
		Đặt $f(x)=m\cdot\sin 2x+x^2\cdot\cos x+\left(m^2+1\right)\cdot\cos 2x$.
		\begin{itemize}
			\item $f(x)$ liên tục trên $\mathbb{R}$.
			\item Ta có $f(0)=m^2+1>0\ \forall m$ và $ f\left(\dfrac{\pi}{2}\right)=-m^2-1<0\ \forall m $.
		\end{itemize}
		Suy ra phương trình $f(x)=0$ có ít nhất một nghiệm thuộc khoảng $\left(0 ;\dfrac{\pi}{2}\right)$.
	}
\end{bt}
\begin{bt}%[TH]%[DCHT Toán 11 - KNTT -Tên GV] %[1K5BG-7]
	Chứng minh phương trình $x^3+m x^2-4m x=19-3m$ có nghiệm với mọi $m$.
	\loigiai{
		Ta có $x^3+m x^2-4m x=19-3m \Leftrightarrow x^3+m x^2-4m x-19+3m=0$.\\
		Đặt $f\left(x\right)=x^3+m x^2-4m x-19+3m$.\\
		$f\left(x\right)$ liên tục trên $\mathbb{R}$.\\
		$f(1)=-18$, 
		$f(3)=8$,\\
		$\Rightarrow f(1)f(3)<0$\\
		$\Rightarrow$ phương trình có nghiệm với mọi $m$.
	}
\end{bt}
% \begin{bt}%[DCHT Toán 11 - KNTT -Nguyễn Thành Nhân]%[1K5BG-7]
% 	Tìm các giá trị nguyên của tham số $m$ để phương trình sau vô nghiệm 
% 	\[
% 	(m^2-1)(x-1)^{2020} =2019\sqrt{4-x}. 
% 	\]
% 	\loigiai{
% 		Điều kiện xác định $x \leq 4$. 
% 		\begin{itemize}
% 			\item Dễ thấy với $m^2 -1 <0 \Leftrightarrow -1 <m <1$, phương trình vô nghiệm. 
% 			\item Với $m^2 -1 =0 \Leftrightarrow m = \pm 1$, phương trình có nghiệm $x=4$. 
% 			\item Với $m^2 -1 >0$, do $x=1$ không là nghiệm nên phương trình tương đương với 
% 			\[
% 			m^2 -1 = \dfrac{2019\sqrt{4-x}}{(x-1)^{2020}}. \tag{1}
% 			\]
% 			Xét hàm số $f(x) = \dfrac{2019\sqrt{4-x}}{(x-1)^{2020}}$. \\
% 			Ta có $\displaystyle \lim_{x \to 1^+} f(x) = +\infty$, do đó tồn tại $ 1<\alpha<4$ sao cho $f(\alpha)  > m^2-1$. \\
% 			Dễ thấy hàm số $f(x)$ liên tục trên đoạn $[\alpha ; 4]$ và có $0 = f(0) < m^2-1 < f(\alpha)$, do đó theo định lý giá trị trung gian tồn tại $x_0 \in (\alpha ;4)$ sao cho $f(x_0) = m^2-1$. Điều đó có nghĩa là phương trình (1) có nghiệm $x=x_0$.\\
% 			Vậy $m=0$ là số nguyên duy nhất để phương trình đã cho vô nghiệm. 
% 		\end{itemize}
% 	}
% \end{bt}
% \begin{bt}%[DCHT Toán 11 - KNTT -Nguyễn Thành Nhân]%%[1K5BG-7]
% 	Chứng minh phương trình $(1-m^2)(x+1)^3+x^2-x-3=0$ có nghiệm với mọi $m$.
% 	\loigiai{
% 		Đặt $f(x)=(1-m^2)(x+1)^3+x^2-x-3$, $f(x)$ là hàm đa thức nên xác định và liên tục trên $\mathbb{R}$. Suy ra $f(x)$ liên tục trên đoạn $[-2;-1]$.\\
% 		Ta có $\heva{&f(-1)=-1<0\\&f(-2)=(1-m^2)(-1)^3+(-2)^2-(-2)-3=m^2+2>0,\,\forall m\in\mathbb{R}.}$\\
% 		Vì $f(-2)\cdot f(-1)<0$, $\forall m\in\mathbb{R}$ nên phương trình $f(x)=0$ có ít nhất một nghiệm trong khoảng $(-2;-1)$, $\forall m\in\mathbb{R}$.\\
% 		Vậy phương trình $(1-m^2)(x+1)^3+x^2-x-3=0$ có nghiệm với mọi $m$.
% 	}
% \end{bt}

% \begin{bt}%[DCHT Toán 11 - KNTT -Nguyễn Thành Nhân]%[1K5BG-7]
% 	Chứng minh rằng phương trình $x^5+4x^3-x^2-1=0$ có ít nhất một nghiệm thuộc khoảng $(0;1)$.
% 	\loigiai{
% 		Đặt $f(x) = x^5+4x^3-x^2-1$ liên tục trên $[0;1]$.\\
% 		Ta có $f(0) = -1$ và $f(1) = 3$ nên $f(0) \cdot f(1) < 0$.\\
% 		Do đó phương trình $f(x) =0$ có ít nhất một nghiệm thuộc $(0;1)$.
% 	}
% \end{bt}

% \begin{bt}%[DCHT Toán 11 - KNTT -Nguyễn Thành Nhân]%[1K5BG-7]
% 	Chứng minh rằng phương trình $ 2x^4-3x^3-5=0 $ có ít nhất một nghiệm.
% 	\loigiai{
% 		Đặt $f(x)= 2x^4-3x^3-5 $, $ f(x) $ là hàm đa thức nên liên tục trên $ \mathbb{R} $.\\
% 		Do đó $ f(x) $ liên tục trên đoạn $ [1;2] $.\\
% 		Ta có $ \heva{&f(1)=-6\\&f(2)=3} $ suy ra $ f(1)\cdot f(2)=-18<0 $.\\
% 		Nên phương trình $ f(x)=0 $ có ít nhất một nghiệm nằm trong khoảng $ (1;2) $.\\
% 		Vậy phương trình đã cho có ít nhất một nghiệm.
% 	}
% \end{bt}
% \begin{bt}%[DCHT Toán 11 - KNTT -Nguyễn Thành Nhân]%[1K5BG-7]
% 	Chứng minh rằng phương trình $2x^3 - 5x + 1 = 0$ có đúng ba nghiệm.
% 	\loigiai{
% 		Đặt $f(x) = 2x^3 - 5x + 1$. Tập xác định của hàm số là $\mathscr D = \mathbb{R}$.\\
% 		Ta có $f(-2) = -5$, $f(0) = 1$, $f(1) = -2$, $f(2) = 7$.\\
% 		Vì $f(x)$ liên tục trên $\mathbb{R}$ nên $f(x)$ liên tục trên $[-2; 0]$ và $f(-2)\cdot f(0) = -5 \cdot 1 = -5 < 0$.\\
% 		Do đó phương trình $f(x) = 0$ có ít nhất một nghiệm $x_1 \in (-2; 0)$. \hfill (1)\\
% 		Vì $f(x)$ liên tục trên $\mathbb{R}$ nên $f(x)$ liên tục trên $[0; 1]$ và $f(0)\cdot f(1) = 1 \cdot (-2) = -2 < 0$.\\
% 		Do đó phương trình $f(x) = 0$ có ít nhất một nghiệm $x_2 \in (0; 1)$. \hfill (2)\\
% 		Vì $f(x)$ liên tục trên $\mathbb{R}$ nên $f(x)$ liên tục trên $[1; 2]$ và $f(1)\cdot f(2) = -2 \cdot 7 = -14 < 0$.\\
% 		Do đó phương trình $f(x) = 0$ có ít nhất một nghiệm $x_3 \in (1; 2)$. \hfill (3)\\
% 		Do các khoảng $(-1; 0)$; $(0; 1)$; $(1; 2)$ không giao nhau
% 		Từ (1), (2) và (3), suy ra phương trình $2x^3 - 5x + 1 = 0$ có ít nhất ba nghiệm.\\
% 		Mà phương trình bậc ba có không quá $3$ nghiệm nên phương trình $2x^3 - 5x + 1 = 0$ có đúng ba nghiệm phân biệt. 
% 	}
% \end{bt}
% \begin{bt}%[VD]%[DCHT Toán 11 - KNTT -Tên GV] %[1K5KG-7]
% 	Chứng minh phương trình $2x^3-3x^2-1=0$ có nghiệm $x_0\in \left(\sqrt[3]{4};2\right)$.
% 	\loigiai{
% 		Đặt $f(x)=2x^3-3x^2-1$ thì $f(x)$ liên tục trên $\mathbb{R}$.\\
% 		Ta có $f(1)=-2<0$; $f(2)=3>0$. Suy ra phương trình $f(x)=0$ có nghiệm $x_0\in (1;2)$.\\
% 		Lại áp dụng bất đẳng thức Cauchy, ta có
% 		\[2x_0^3=3x_0^2+1=x_0^2+x_0^2+x_0^2+1\geq 4\sqrt[4]{x_0^6}=4\sqrt{x_0^3}.\]
% 		Suy ra $x_0\geq \sqrt[3]{4}$. Nhưng tại $x_0=\sqrt[3]{4}$ thì dấu đẳng thức không xảy ra, suy ra $x_0> \sqrt[3]{4}$.\\
% 		Vậy, phương trình đã cho có nghiệm $x_0\in \left(\sqrt[3]{4};2\right)$.	
% 	}
% \end{bt}
% \begin{bt}%[VD]%[DCHT Toán 11 - KNTT -Nguyễn Thành Nhân]%[1K5KG-7]
% 	Chứng minh phương trình $x^7-3x^6+x^4+x^3-(m^2+3)x+2=0$ có ít nhất một nghiệm dương với mọi tham số $m \in \mathbb{R}$.
% 	\loigiai{ Đặt $f(x)=x^7-3x^6+x^4+x^3-(m^2+3)x+2$, khi đó hàm số $f(x)$ liên tục trên $\mathbb{R}$.\\
% 		Ta có $f(0)=2$, $f(1)=-m^2-1$. 
% 		Suy ra $f(0)\cdot f(1)=2\left(-m^2-1\right)<0$ với mọi $m \in \mathbb{R}$.\\
% 		Vậy phương trình đã cho có ít nhất một nghiệm dương thuộc khoảng $(0;1)$ với mọi $m \in \mathbb{R}$.
% 	}
% \end{bt}
% \begin{bt}%[VD]%[DCHT Toán 11 - KNTT -Nguyễn Thành Nhân]%[1K5KG-7]
% 	Cho phương trình $x^4-x^3-(m^2+5)x^2+2(m^2+2)x+4=0$. Chứng minh rằng với mọi số nguyên $m$ thì phương trình sau luôn có đúng $4$ nghiệm phân biệt.
% 	\loigiai{
% 		Ta có 
% 		\allowdisplaybreaks
% 		\begin{eqnarray*}
% 			&&x^4-x^3-(m^2+5)x^2+2(m^2+2)x+4=0\\
% 			&\Leftrightarrow&(x-2)\left(x^3+x^2-(m^2+3)x-2\right)=0\\
% 			&\Leftrightarrow&\hoac{&x-2=0\\&x^3+x^2-(m^2+3)x-2=0.}	
% 		\end{eqnarray*}	
% 		Phương trình ban đầu có đúng $4$ nghiệm khi và chỉ khi phương trình $x^3+x^2-(m^2+3)x-2=0$ có đúng $3$ nghiệm phân biệt khác $2$.\\
% 		Xét $f(x)=x^3+x^2-(m^2+3)x-2$, ta có
% 		\begin{itemize}
% 			\item $\lim\limits_{x\to -\infty}f(x)=-\infty$ nên $\exists a<-1\colon f(a)<0$.
% 			\item $f(-1)=m^2+1>0$.
% 			\item $f(0)=-2<0$.
% 			\item $f\left(m^2+3\right)=\left(m^2+3\right)^3-2>0$.
% 		\end{itemize}
% 		Hàm số $f(x)$ là hàm số đa thức nên liên tục trên tập xác định $\mathbb{R}$, do đó $f(x)$ cũng liên tục trên các đoạn $[a;-1]$, $[-1;0]$ và $\left[0;m^2+3\right]$.\\
% 		Do $f(a) \cdot f(-1)<0$, $f(-1) \cdot f(0)<0$, $f(0) \cdot f\left(m^{2}+3\right)<0$ và $f(x)=0$ là phương trình bậc 3 nên có đúng 3 nghiệm thuộc các khoảng $\left(a ;-1\right)$, $(-1 ; 0)$, $\left(0 ; m^{2}+3\right)$.\\
% 		Mặt khác $f(2)\ne 0\Leftrightarrow m\ne \pm \sqrt{2}\notin \mathbb{Z}$.\\
% 		Vậy phương trình $x^4-x^3-(m^2+5)x^2+2(m^2+2)x+4=0$ luôn có đúng $4$ nghiệm phân biệt với mọi số nguyên $m$.
% 	}
% \end{bt}

% \begin{bt}%[VD]%[DCHT Toán 11 - KNTT -Nguyễn Thành Nhân]%[1K5KG-7]
% 	Với $m>2$, chứng minh rằng phương trình $x^3-2mx^2+2=0$ có ba nghiệm phân biệt. 
% 	\loigiai{
% 		Đặt $f(x)=x^3-2mx^2+2$ thì $f(x)$ là hàm đa thức nên liên tục trên $\mathbb{R}$.\\
% 		Ta có \\
% 		$f(-1)=1-2m<0$ do $m>2$, $f(0)=2>0$, $f(1)=3-2m<0$ do $m>2$. \\
% 		Vì $f(-1)\cdot f(0)<0$ nên phương trình có ít nhất một nghiệm thuộc khoảng $(-1;0)$.\\
% 		Vì $f(0)\cdot f(1)<0$ nên phương trình có ít nhất một nghiệm thuộc khoảng $(0;1)$.\\
% 		Vì $\lim\limits_{x\to +\infty}f(x)=\lim\limits_{x\to +\infty}x^3\left(1-\dfrac{2m}{x}+\dfrac{2}{x^3}\right)=+\infty$ nên tồn tại một số $a>1$ để $f(a)>0$.\\
% 		Vì $f(1)\cdot f(a)<0$ nên phương trình có ít nhất một nghiệm thuộc khoảng $(1;a)$.\\
% 		Từ đó suy ra phương trình có ít nhất ba nghiệm phân biệt. Mặt khác $f(x)=0$ là phương trình bậc $3$ nên có không quá ba nghiệm.\\
% 		Vậy phương trình $f(x)=0$ có đúng ba nghiệm phân biệt. 
% 	}
% \end{bt}
% \begin{bt}%[VD]%[DCHT Toán 11 - KNTT -Tên GV] %[1K5KG-7]
% 	Chứng minh phương trình $\left(1-m\right)x^5+9mx^2-16x-m=0$ có ít nhất hai nghiệm thực phân biệt với mọi giá trị thực của tham số $m$.
% 	\loigiai
% 	{Đặt $f(x)=\left(1-m\right)x^5+9mx^2-16x-m$ thì $f(x)$ liên tục trên $\mathbb{R}$.\\ 
% 		Ta biến đổi $f(x)=\left(-x^5+9x^2-1\right)m+x^5-16x$. \\
% 		Ta tính giá trị của $f(x)$ tại các giá trị của $x$ thỏa $x^5-16x=0$, tức là tính tại $0$ và $\pm 2$.\\
% 		Ta có $f(-2)=3m$; $f(0)=-m$; $f(2)=3m$.\\
% 		Nếu $m=0$ thì phương trình tương đương $$x^5-16x=0\Leftrightarrow \hoac{&x=0\\&x=\pm2},$$ nên phương trình có nghiệm khi $m=0$.\\
% 		Với $m\ne 0$ thì $f(-2)\cdot f(0)=-3m^2<0$; $f(0)\cdot f(2)=-3m^2<0$. \\
% 		Theo tính chất của hàm số liên tục, trên mỗi khoảng $(-2;0)$ và $(0;2)$, phương trình có ít nhất một nghiệm.\\
% 		Vậy phương trình đã cho có ít nhất hai nghiệm phân biệt.
% 	}
% \end{bt}
\begin{bt}%[VD]%[DCHT Toán 11 - KNTT -Tên GV] %[1K5KG-7]
	Cho $f(x)$ là hàm số liên tục trên đoạn $\left[a;b\right]$ sao cho với mọi $x\in \left[a;b\right]$ thì $a\le f(x)\le b$. Chứng minh rằng phương trình $f(x)=x$ có ít nhất một nghiệm thuộc đoạn $\left[a;b\right]$.
	\loigiai
	{Xét hàm số $h(x)=f(x)-x$, khi đó $h(x)$ liên tục trên đoạn $\left[a;b\right]$.\\
		Vì $a\le f(x)\le b$ với mọi $x\in \left[a;b\right]$ nên ta có $$h(a)=f(a)-a\geq 0;\,\,h(b)=f(b)-b\le 0.$$
		Suy ra $h(a)\cdot h(b)\le 0$. Do đó xảy ra một trong hai khả năng
		\begin{itemize}
			\item[•] Nếu $h(a)\cdot h(b)=0$ thì $\hoac{&h(a)=0\\&h(b)=0}$, do đó phương trình $h(x)=0$ có nghiệm $x=a$ hoặc $x=b$.\\
			Dẫn đến phương trình $f(x)=x$ có nghiệm $x=a$ hoặc $x=b$.
			\item[•] Nếu $h(a)\cdot h(b)<0$ thì do tính liên tục của $h(x)$ nên phương trình $h(x)=0$ có nghiệm thuộc khoảng $(a;b)$.\\
			Dẫn đến phương trình $f(x)=x$ cũng có nghiệm thuộc khoảng $(a;b)$.
		\end{itemize}
	}
\end{bt}
\begin{bt}%[VDC]%[DCHT Toán 11 - KNTT -Nguyễn Thành Nhân]%[1K5GG-7]
	Cho hai số $a$ và $b$ dương, $c\ne 0$ và $m$, $n$ là hai số thực tùy ý. Chứng minh phương trình $\dfrac{a}{x-m}+\dfrac{b}{x-n}=c$ luôn có nghiệm thực.
	\loigiai{
		Điều kiện xác định của phương trình $x\neq m$ và $x\neq n$.\\
		Khi $m=n$, ta có $$\dfrac{a}{x-m}+\dfrac{b}{x-n}=c\Leftrightarrow\dfrac{a}{x-m}+\dfrac{b}{x-m}=c\Leftrightarrow a+b=c\left(x-m\right)\Leftrightarrow x=m+\dfrac{a+b}{c}.$$
		Như vậy, khi $m=n$, phương trình đã cho luôn có nghiệm.\\
		Khi $m\ne n$, ta có
		$$\dfrac{a}{x-m}+\dfrac{b}{x-n}=c\Leftrightarrow a(x-n)+b(x-m)-c(x-m)(x-n)=0.$$
		Xét hàm số $f(x)=a(x-n)+b(x-m)-c(x-m)(x-n)$.\\
		Ta có hàm số liên tục trên $\mathbb{R}$.\\
		Dễ thấy $f(m)=a(m-n)$, $f(n)=b(n-m)=-b(m-n)$.\\
		Do đó $f(m)\cdot f(n)=-ab(m-n)^2< 0,\forall a > 0$, $b > 0$, $m\neq n$ .\\
		Do đó phương trình đã cho có ít nhất một nghiệm thuộc $\left(m;,n\right)$ nếu $m < n$, hoặc ít nhất một nghiệm thuộc $\left(n;m\right)$ nếu $m > n$.\\
		Vậy phương trình đã cho luôn có nghiệm thực.
	}
\end{bt}
\begin{bt}%[VDC]%[DCHT Toán 11 - KNTT -Tên GV] %[1K5GG-7]
	Cho $a$, $b$, $c$ là ba số thực tùy ý. Chứng minh rằng phương trình
	\[ab(x-a)(x-b)+bc(x-b)(x-c)+ca(x-c)(x-a)=0\] luôn có nghiệm.
	\loigiai
	{Đặt $f(x)=ab(x-a)(x-b)+bc(x-b)(x-c)+ca(x-c)(x-a)$ thì $f(x)$ liên tục trên $\mathbb{R}$.\\ 
		Ta có 
		\begin{eqnarray*}
			f(a)\cdot f(b)\cdot f(c)\cdot f(0)&=&\left[bc(a-b)(a-c)+ca(b-c)(b-a)+ab(c-a)(c-b)\right]\\
			&=&-a^2 b^2c^2\cdot (a-b)^2(b-c)^2(c-a)^2\cdot \left(a^2b^2+b^2c^2+c^2a^2\right)\\
			&\le & 0.\quad (1)
		\end{eqnarray*}
		Xảy ra một trong hai khả năng
		\begin{itemize}
			\item[•] Nếu $f(a)\cdot f(b)\cdot f(c)\cdot f(0)=0$ thì phương trình $f(x)=0$ có ít nhất một nghiệm là một trong các số $a$, $b$, $c$, $0$.
			\item[•] Nếu $f(a)\cdot f(b)\cdot f(c)\cdot f(0)<0$ thì từ $(1)$ suy ra $a$, $b$, $c$ khác $0$.\\
			Vì tích bốn số nhỏ hơn $0$ nên phải tồn tại hai số trong các số $f(a)$, $ f(b)$, $f(c)$, $f(0)$ trái dấu. Dù là hai số nào cũng đều dẫn đến phương trình $f(x)=0$ có nghiệm.
		\end{itemize}
	}
\end{bt}

\begin{bt}%[VDC]%[DCHT Toán 11 - KNTT -Tên GV] %[1K5GG-7]
	Cho phương trình $x^3-3x^2+\left(2m-2\right)x+m-3=0$. Tìm tất cả các giá trị của tham số $m$ để phương trình có ba nghiệm phân biệt $x_1$, $x_2$, $x_3$ thỏa mãn $x_1<-1<x_2<x_3$.
	\loigiai
	{ Ta giải bài toán bằng điều kiện cần và điều kiện đủ.
		\begin{itemize}
			\item[•] \textit{Điều kiện cần}.\\
			Đặt $f(x)=x^3-3x^2+\left(2m-2\right)x+m-3$ thì $f(x)$ liên tục trên $\mathbb{R}$.\\
			Giả sử phương trình có ba nghiệm phân biệt $x_1$, $x_2$, $x_3$ thỏa mãn $x_1<-1<x_2<x_3$. \\
			Ta có $f(x)=(x-x_1)\cdot (x-x_2)\cdot (x-x_3)$.\\
			Từ giả thiết $x_1<-1<x_2$ suy ra 
			\[f(-1)>0\Leftrightarrow -m-5>0\Rightarrow m<-5.\]
			Đó là điều kiện cần của bài toán. Ta chứng minh đó cũng là điều kiện đủ.
			\item[•] \textit{Điều kiện đủ}.\\
			Với $m<-5$. Ta có $f(-1)=-m-5>0$.\\
			Lai có $\lim \limits _{x\to -\infty}f(x)=-\infty$ nên tồn tại $a<-1$ để $f(a)<0$.\\
			Ta có $f(0)=m-3<0$. \\
			Lại có có $\lim \limits _{x\to +\infty}f(x)=+\infty$ nên tồn tại $b>0$ để $f(b)>0$.\\
			Vì $f(a)\cdot f(-1)<0$; $f(-1)\cdot f(0)<0$; $f(0)\cdot f(b)<0$ nên theo tính chất của hàm số liên tục, phương trình $f(x)=0$ có các nghiệm $x_1\in (a;-1)$, $x_2\in (-1;0)$; $x_3\in (0;b)$.\\
			Do đó phương trình $f(x)=0$ có ba nghiệm phân biệt $x_1$, $x_2$, $x_3$ thỏa mãn $x_1<-1<x_2<x_3$.
		\end{itemize}
		Vậy $m<-5$ là điều kiện cần và đủ của bài toán.
	}
\end{bt}
\begin{bt}%[VDC]%[DCHT Toán 11 - KNTT -Tên GV] %[1K5GG-7]
	Cho $2a+6b+19c=0$. Chứng minh rằng phương trình $ax^2+bx+c=0$ có nghiệm $\break x_0\in \left[0;\dfrac{1}{3}\right]$.
	\loigiai
	{ Đặt $f(x)=ax^2+bx+c$ thì $f(x)$ liên tục trên $\mathbb{R}$.\\
		Xét các giá trị sau $$f(0)=c;\,\,18\cdot f\left(\dfrac{1}{3}\right)=2a+6b+18c=(2a+6b+19c)-c=-c.$$
		Do đó $18\cdot f(0)\cdot f\left(\dfrac{1}{3}\right)=-c^2\le 0$. Xét hai khả năng
		\begin{itemize}
			\item[•] Nếu $c=0$ thì $18\cdot f(0)\cdot f\left(\dfrac{1}{3}\right)\Leftrightarrow \hoac{&f(0)=0\\&f\left(\dfrac{1}{3}\right)=0}$, suy ra phương trình $f(x)=0$ có nghiệm $x=0$ hoặc $x=\dfrac{1}{3}$.
			\item[•] Nếu $c\ne 0$ thì $18\cdot f(0)\cdot f\left(\dfrac{1}{3}\right)=-c^2<0$ nên phương trình luôn có nghiệm thuộc khoảng $\left(0;\dfrac{1}{3}\right)$.
		\end{itemize} 
		Vậy phương trình luôn có nghiệm thuộc đoạn $\left[0;\dfrac{1}{3}\right]$.
	}
\end{bt}
\begin{bt}%[VDC]%[DCHT Toán 11 - KNTT -Nguyễn Thành Nhân]%[1K5GG-7]
	Cho phương trình $a\cos{2x}+b\cos{x}+c=0$, với $a$, $b$, $c$ là các số thực thỏa mãn $3b+7c=3a$. Chứng minh phương trình đã cho luôn có nghiệm.
	\loigiai{Đặt $t=\cos{x}$, điều kiện $|t|\le 1$.\\
		Khi đó, ta có $\cos 2x=2\cos^2x-1=2t^2-1$. Phương trình đã cho trở thành 
		\[a(2t^2-1)+bt+c=0 \Leftrightarrow 2at^2+bt+(c-a)=0. \qquad (1)\]
		Phương trình đã cho có nghiệm khi và chỉ khi phương trình $(1)$ có nghiệm $t \in [-1;1]$.\\
		Xét $f(t)=2at^2+bt+(c-a)$, liên tục trên $\mathbb{R}$.\\
		Ta có $f(0)=c-a$, và
		\[f\left(\dfrac{2}{3}\right)=\dfrac{2}{3}\left(\dfrac{4}{3}a+b\right)+(c-a)=\dfrac{2}{3}\left(\dfrac{4a+3b}{3}\right)+(c-a).\]
		Theo giả thiết, ta có 
		\[3b+7c=3a \Rightarrow 3b=3a-7c \Rightarrow 4a+3b=7(a-c).\]
		Do đó $f\left(\dfrac{2}{3}\right)=\dfrac{2}{3}\cdot\dfrac{7(a-c)}{3}+(c-a)=\dfrac{5}{9}(a-c)$.\\
		Suy ra $f(0)f\left(\dfrac{2}{3}\right)=-\dfrac{5}{9}(c-a)^2 \le 0$.\\
		Vì $f$ liên tục trên $\mathbb{R}$ nên $f$ liên tục trên $\left[0;\dfrac{2}{3}\right]$. Do đó tồn tại $t_0 \in \left[0;\dfrac{2}{3}\right]$ sao cho $f(t_0)=0$.\\
		Suy ra phương trình $(1)$ có nghiệm thuộc $\left[0;\dfrac{2}{3}\right]$, cũng thuộc $[-1;1]$.\\
		Vậy, phương trình $a\cos{2x}+b\cos{x}+c=0$ luôn có nghiệm.}
\end{bt}

\begin{bt}%[VDC]%[DCHT Toán 11 - KNTT -Nguyễn Thành Nhân]%[1K5GG-7]
	Giả sử hai hàm số $y=f(x)$ và $y=f(x+1)$ đều liên tục trên đoạn $[0;2]$ và $f(0)=f(2)$. Chứng minh phương trình $f(x)-f(x+1)=0$ luôn có nghiệm thuộc đoạn $[0;1]$.
	\loigiai{
		Xét hàm số $g(x)=f(x)-f(x+1)$ trên đoạn $[0;1]$.\\
		Vì $y=f(x)$ và $y=f(x+1)$ đều liên tục trên đoạn $[0;2]$ nên hàm số $g(x)$ liên tục trên đoạn $[0;1]$.\\
		Ta có $\heva{&g(0)=f(0)-f(1)\\&g(1)=f(1)-f(2)=f(1)-f(0).}$\\
		Ta xét các trường hợp sau
		\begin{itemize}
			\item Nếu $f(0)-f(1)=0\Rightarrow \heva{&g(0)=0\\&g(1)=0}$. Suy ra phương trình $g(x)=0$ có nghiệm $x=0$, $x=1$. \quad $(1)$
			\item Nếu $f(0)-f(1)\ne 0$ thì $g(0)\cdot g(1)=[f(0)-f(1)]\cdot [f(1)-f(0)]<0$. Suy ra phương trình $g(x)=0$ luôn có ít nhất một nghiệm thuộc đoạn $[0;1]$. \quad $(2)$
		\end{itemize}
		Từ $(1)$ và $(2)$, suy ra phương trình $f(x)-f(x+1)=0$ luôn có nghiệm thuộc đoạn $[0;1]$.
	}
\end{bt}
% \subsubsection{Câu hỏi trắc nghiệm}
% \Opensolutionfile{ans}[ans/ans-1K5-3-Dang6]
% \begin{ex}%[DCHT Toán 11 - KNTT -Nguyễn Thành Nhân]%[1K5YG-7]
% 	Cho hàm số $y=f(x)$ liên tục trên đoạn $[a;b]$. Mệnh đề nào sau đây đúng?
% 	\choice
% 	{Nếu $f(a)\cdot f(b)>0$ thì phương trình $f(x)=0$ không có nghiệm thuộc $(a ; b)$}
% 	{\True Nếu $f(a)\cdot f(b)<0$ thì phương trình $f(x)=0$ có nghiệm thuộc $(a ; b)$}
% 	{Nếu $f(a)\cdot f(b)<0$ thì phương trình $f(x)=0$ không có nghiệm thuộc $(a ; b)$}
% 	{Nếu $f(a)\cdot f(b)>0$ thì phương trình $f(x)=0$ có nghiệm thuộc $(a ; b)$}
% 	\loigiai{
% 		"Nếu $f(a)\cdot f(b)<0$ thì phương trình $f(x)=0$ có nghiệm thuộc $(a ; b)$" là mệnh đề đúng.	
% 	} 
% \end{ex}

% \begin{ex}%[DCHT Toán 11 - KNTT -Nguyễn Thành Nhân]%[1K5YG-7]
% 	Phương trình $x^5+2x^3+16=0$ có nghiệm thuộc khoảng nào sau đây?
% 	\choice
% 	{$(0;1)$}
% 	{$(-10;-2)$}
% 	{$(-1;0)$}
% 	{\True $(-2;-1)$}
% 	\loigiai{
% 		Hàm số $x^5+2 x^3+16=0$ liên tục trên $\mathbb{R}$.\\
% 		Do $f(-2)\cdot f(-1)=-32\cdot 13 <0$ nên phương trình $x^5+2 x^3+16=0$ có nghiệm thuộc khoảng $(-2;-1)$. 	
% 	} 
% \end{ex}

% \begin{ex}%[DCHT Toán 11 - KNTT -Nguyễn Thành Nhân]%[1K5YG-7]
% 	Phương trình $ 3x^5+5x^3+10=0 $ có nghiệm thuộc khoảng nào sau đây?
% 	\choice
% 	{$ (0;1) $}
% 	{$ (-1;0) $}
% 	{$ (-10;-2) $}
% 	{\True $ (-2;-1) $}
% 	\loigiai{
% 		Đặt $ f(x)=3x^5+5x^3+10 $.\\
% 		Tập xác định của hàm số là $ \mathscr{D}=\mathbb{R} \Rightarrow f(x)$	liên tục trên $ \mathbb{R} $.\\
% 		Suy ra $ f(x) $ liên tục trên $ [-2;-1] $.\\
% 		Ta có $ f(-2)=-126 $; $ f(-1)=2\Rightarrow f(-2)\cdot f(-10)=-252<0 $.\\
% 		Phương trình $ 3x^5+5x^3+10=0 $ có nghiệm thuộc khoảng $ (-2;-1) $.
% 	}
% \end{ex}

% \begin{ex}%[DCHT Toán 11 - KNTT -Nguyễn Thành Nhân]%[1K5BG-7]
% 	Cho hàm số $f(x)$ liên tục trên đoạn $[-1 ; 4]$, biết $f(-1)=2$, $f(4)=7$. Có thể nói gì về số nghiệm của phương trình $f(x)=5$ trên đoạn $[-1 ; 4]$.	
% 	\choice
% 	{Có hai nghiệm phân biệt }
% 	{Có đúng một nghiệm}
% 	{\True Có ít nhất một nghiệm}
% 	{Vô nghiệm}
% 	\loigiai{
% 		Xét hàm số $g(x)=f(x)-5$ liên tục trên 	đoạn $[-1 ; 4]$ có $g(-1)=f(-1)-5=2-5=-3<0$ và $g(4)=f(4)-5=7-5=2>0$.\\
% 		Suy ra $g(-1)\cdot g(4)<0$.\\
% 		Do đó phương trình $g(x)=0$ có ít nhất một nghiệm thuộc $(-1;4)$ hay phương trình $f(x)=5$ trên đoạn $[-1 ; 4]$ 	Có ít nhất một nghiệm.
% 	}
% \end{ex}

% \begin{ex}%[DCHT Toán 11 - KNTT -Nguyễn Thành Nhân]%[1K5BG-7]
% 	Cho hàm số $y=f(x)$ liên tục trên $[2022;2023]$, biết rằng $f(2022)=-2023$, $f(2023)=2022$. Kết luận nào sau đây chắc chắn đúng?
% 	\choice
% 	{\True Phương trình $f(x)=0$ có nghiệm}
% 	{Phương trình $f(x)=0$ vô nghiệm}
% 	{Phương trình $f(x)=0$ có $3$ nghiệm}
% 	{Phương trình $f(x)=0$ có $2$ nghiệm}
% 	\loigiai{Ta có $f(x)$ liên tục trên $[2022;2023]$ và $f(2022)\cdot f(2023)<0$ nên $\exists x_0\in(2022;2023)\colon f(x_0)=0$.\\
% 		Vậy mệnh đề chắc chắc đúng là ``Phương trình $f(x)=0$ có nghiệm''.}
% \end{ex}

% \begin{ex}%[DCHT Toán 11 - KNTT -Nguyễn Thành Nhân]%[1K5BG-7]
% 	Cho các câu
% 	\begin{enumerate}
% 		\item Nếu hàm số $y=f(x)$ liên tục trên $(a;b)$ và $f(a) \cdot f(b)<0$ thì tồn tại $x_0 \in (a;b)$ sao cho $f\left(x_0 \right) = 0$.
% 		\item Nếu $y=f(x)$ liên tục trên $[a;b]$ và $f(a)\cdot f(b) <0$ thì phương trình $f(x) = 0$ có nghiệm thuộc khoảng $(a;b)$.
% 		\item Nếu hàm số $y=f(x)$ liên tục, đơn điệu trên $[a;b]$ và $f(a) \cdot f(b) <0$ thì phương trình $f\left( x \right)=0$ có nghiệm duy nhất thuộc $(a;b)$.
% 	\end{enumerate}
% 	Trong ba câu trên
% 	\choice
% 	{\True có đúng một câu {\bf sai}}
% 	{cả ba câu đều đúng}
% 	{có đúng hai câu {\bf sai}}
% 	{cả ba câu đều {\bf sai}}
% 	\loigiai{
% 		Trong ba câu trên có hai câu đúng
% 		\begin{itemize}
% 			\item Nếu $y=f(x)$ liên tục trên $[a;b]$ và $f(a)\cdot f(b) <0$ thì phương trình $f(x) = 0$ có nghiệm thuộc khoảng $(a;b)$.
% 			\item Nếu hàm số $y=f(x)$ liên tục, đơn điệu trên $[a;b]$ và $f(a) \cdot f(b) <0$ thì phương trình $f\left( x \right)=0$ có nghiệm duy nhất thuộc $(a;b)$.
% 		\end{itemize}
% 		và một câu sai
% 		\begin{itemize}
% 			\item Nếu hàm số $y=f(x)$ liên tục trên $(a;b)$ và $f(a) \cdot f(b)<0$ thì tồn tại $x_0 \in (a;b)$ sao cho $f\left(x_0 \right) = 0$.
% 		\end{itemize}
% 	}
% \end{ex}

% \begin{ex}%[DCHT Toán 11 - KNTT -Nguyễn Thành Nhân]%[1K5BG-7]
% 	Cho hàm số $f(x)$ xác định trên $[a;b]$. Trong các mệnh đề sau, mệnh đề nào đúng?
% 	\choice
% 	{Nếu phương trình $f(x)=0$ có nghiệm trong khoảng $(a;b)$ thì hàm số $f(x)$ phải liên tục trên $(a;b)$}
% 	{Nếu hàm số $f(x)$ liên tục trên $[a;b]$ và $f(a)\cdot f(b) >0$ thì phương trình $f(x)=0$ không có nghiệm trong khoảng $(a;b)$}
% 	{Nếu $f(a) \cdot f(b)<0$ thì phương trình $f(x)=0$ có ít nhất một nghiệm trong khoảng $(a;b)$}
% 	{\True Nếu hàm số $f(x)$ liên tục, tăng trên $[a;b]$ và $f(a) \cdot f(b)>0$ thì phương trình $f(x) =0$ không có nghiệm trong khoảng $(a;b)$}
% 	\loigiai{
% 		Mệnh đề đúng là ``Nếu hàm số $f(x)$ liên tục, tăng trên $[a;b]$ và $f(a) \cdot f(b)>0$ thì phương trình $f(x) =0$ không có nghiệm trong khoảng $(a;b)$.''
% 	}
% \end{ex}

% \begin{ex}%[DCHT Toán 11 - KNTT -Nguyễn Thành Nhân]%[1K5BG-7]
% 	Chứng minh rằng phương trình $x^3-x+3=0$ có ít nhất một nghiệm. Một bạn học sinh trình bày lời giải như sau:
% 	\begin{enumerate}[Bước 1.]
% 		\item Xét hàm số $y=f(x)=x^3-3x+3$ liên tục trên $\Bbb{R}$.
% 		\item Ta có $f(0)=3$ và $f(-2)=-3$.
% 		\item Suy ra $f(0)\cdot f(-2)>0$.
% 		\item Vậy phương trình đã cho có ít nhất một nghiệm. Hãy tìm bước giải {\bf sai} của bạn học sinh trên?
% 	\end{enumerate}
% 	\choice
% 	{\True Bước $3$}
% 	{Bước $4$ và Bước $3$}
% 	{Bước $1$}
% 	{Bước $1$ và Bước $3$}
% 	\loigiai{
% 		Học sinh sai ở Bước $3$ vì $f(0)\cdot f(-2)<0$.
% 	}
% \end{ex}

% \begin{ex}%[DCHT Toán 11 - KNTT -Nguyễn Thành Nhân]%[1K5BG-7]
% 	Cho hàm số $f(x)=3x^4+3x-2$. Khẳng định nào sau đây là {\bf sai}?
% 	\choice
% 	{\True Phương trình $f(x)=0$ vô nghiệm}
% 	{Phương trình $f(x)=0$ có nghiệm trong khoảng $(0;1)$}
% 	{Phương trình $f(x)=0$ có 1 it nhất một nghiệm trong khoảng $(-1;1)$}
% 	{Phương trình $f(x)=0$ có nghiệm trên $\mathbb{R}$}
% 	\loigiai{
% 		Hàm số $f(x)=3x^4+3x-2$ liên tục trên $\mathscr{D}=\mathbb{R}$.\\
% 		Ta có $\heva{&f(0)=-2\\&f(1)=4}\Rightarrow f(0)\cdot f(1)=-8<0$ nên phương trình $f(x)=0$ có ít nhất $1$ nghiệm thuộc khoảng $(0;1)$.\\
% 		Suy ra phương trình $f(x)=0$ có ít nhất $1$ nghiệm thuộc khoảng $(-1;1)$ cũng như có nghiệm trên $\mathbb{R}$.\\
% 		Vậy khẳng định {\bf sai} là \lq\lq Phương trình $f(x)=0$ vô nghiệm \rq\rq.
% 	}
% \end{ex}

% \begin{ex}%[DCHT Toán 11 - KNTT -Nguyễn Thành Nhân]%[1K5BG-7]
% 	Cho hàm số $y=f(x)$ liên tục trên đoạn $[1;5]$ và $f(1)=2$, $f(5)=10$. Khẳng định nào sau đây \textbf{đúng}?
% 	\choice
% 	{Phương trình $f(x)=6$ vô nghiệm}
% 	{\True Phương trình $f(x)=7$ có ít nhất một nghiệm trên khoảng $(1;5)$}
% 	{Phương trình $f(x)=2$ có hai nghiệm $x=1$, $x=5$}
% 	{Phương trình $f(x)=7$ vô nghiệm}
% 	\loigiai{
% 		\immini{Đồ thị hàm số $f(x)$ là một đường liền nét trên khoảng $(1;5)$.\\
% 			Đường thẳng $y=7$ song song với trục $Ox$ và sẽ cắt đồ thị hàm $y=f(x)$ tại ít nhất một điểm có hoành độ thuộc khoảng $(1;5)$. Do đó phương trình $f(x)=7$ có ít nhất một nghiệm thuộc khoảng $(1;5)$.\\
% 			Ta có hình vẽ minh hoạ như hình vẽ bên.}{\begin{tikzpicture}[scale=0.3, font=\footnotesize, line join=round, line cap=round, >=stealth]
% 				\tikzset{label style/.style={font=\footnotesize}}
% 				%%Nhập giới hạn đồ thị và hàm số cần vẽ
% 				\def \xmin{-1}
% 				\def \xmax{6}
% 				\def \ymin{-1}
% 				\def \ymax{11}
% 				%\def \hamso{sin(\x r)}
% 				%\def \tiemcanxien{\x+1}
% 				%%Tự động
% 				\draw[->] (\xmin,0)--(\xmax,0) node[below right] {$x$};
% 				\draw[->] (0,\ymin)--(0,\ymax) node[above] {$y$};
% 				\draw (0,0) node [below left] {$O$};
% 				%%Vẽ các điểm trên 2 hệ trục
% 				\foreach \x in {1,5}
% 				\draw[thin] (\x,1pt)--(\x,-1pt) node [below] {$\x$};
% 				\foreach \y in {2,10}
% 				\draw[thin] (1pt,\y)--(-1pt,\y) node [left] {$\y$};
% 				\draw (0,7) node[above left]{$y=7$};
% 				%%Vẽ thêm mấy cái râu ria
% 				\draw[name path=duongthang](-1,7)--(5.5,7);
% 				\draw[name path=duongcong1] (1,2) .. controls (2,-1) and (2.5,15) .. (3,7);
% 				\draw[name path=duongcong2] (3,7) .. controls (3.5,-3) and (4.5,9) .. (5,10);
% 				\clip (\xmin+0.01,\ymin+0.01) rectangle (\xmax-0.01,\ymax-0.01);
% 				\draw[dashed,thin](1,0)--(1,2)--(0,2);
% 				\draw[dashed,thin](5,0)--(5,10)--(0,10);
% 	\end{tikzpicture}}}
% \end{ex}

% \begin{ex}%[DCHT Toán 11 - KNTT -Nguyễn Thành Nhân]%[1K5BG-7]
% 	Cho phương trình $882x^5-441x^4-116x^3+58x^2+2x-1=0$. Mệnh đề nào sau đây \textbf{sai}?
% 	\choice
% 	{Phương trình có nghiệm trong khoảng $(0;1)$}
% 	{Phương trình có nghiệm trong khoảng $(-1;0)$}
% 	{Phương trình có $5$ nghiệm phân biệt}
% 	{\True Phương trình có đúng $4$ nghiệm}
% 	\loigiai{Hàm số $f(x)=882x^5-441x^4-116x^3+58x^2+2x-1$ liên tục trên $\mathbb{R}$ nên nó liên tục trên $\left(-1;1\right)$.
% 		\begin{itemize}
% 			\item $\heva{&f(-0{,}4)=-5{,}417\\&f(-0{,}3)=1{,}0366}\Rightarrow f(-0{,}4)\cdot f(-0{,}3)<0$ nên phương trình có ít nhất một nghiệm trên khoảng $\left(-0{,}4; -0{,}3\right)$.
% 			\item $\heva{&f(-0{,}2)=-0{,}8601\\&f(-0{,}1)=-0{,}556}\Rightarrow f(-0{,}2)\cdot f(-0{,}1)<0$ nên phương trình có ít nhất một nghiệm trên khoảng $\left(-0{,}2; -0{,}1\right)$.
% 			\item $\heva{&f(0{,}1)=-0{,}371\\&f(0{,}2)=0{,}3686}\Rightarrow f(0{,}1)\cdot f(0{,}2)<0$ nên phương trình có ít nhất một nghiệm trên khoảng $\left(0{,}1; 0{,}2\right)$.
% 			\item $\heva{&f(0{,}3)=0{,}2591\\&f(0{,}4)=-0{,}601}\Rightarrow f(0{,}3)\cdot f(0{,}4)<0$ nên phương trình có ít nhất một nghiệm trên khoảng $\left(0{,}3; 0{,}4\right)$.
% 			\item $\heva{&f(0{,}4)=-0{,}601\\&f(0{,}6)=7{,}4547}\Rightarrow f(0{,}4)\cdot f(0{,}6)<0$ nên phương trình có ít nhất một nghiệm trên khoảng $\left(0{,}4; 0{,}6\right)$.
% 		\end{itemize}
% 		Phương trình đã cho là phương trình bậc $5$ nên nó có đúng $5$ nghiệm.	
% 	}
% \end{ex}

% \begin{ex}%[DCHT Toán 11 - KNTT -Nguyễn Thành Nhân]%[1K5BG-7]
% 	Phương trình $x^3-3x^2+5x+1=0$ có ít nhất một nghiệm thuộc khoảng nào sau đây?
% 	\choice
% 	{$(0;1)$}
% 	{$(2;3)$}
% 	{$(-2;-1)$}
% 	{\True $(-1;0)$}
% 	\loigiai{Xét hàm số $f(x)=x^3-3x^2+5x+1$ là hàm đa thức nên liên tục trên $\mathbb{R}$.\\
% 		Ta có $f(x)$ liên tục trên $[-1;0]$ và $f(-1)\cdot f(0)=(-8)\cdot1=-8<0$.\\
% 		Từ đó ta suy ra tồn tại $x_0\in(-1;0)$ sao cho $f\left(x_0\right)=0$, hay phương trình $f(x)=0$ có ít nhất một nghiệm thuộc khoảng $(-1;0)$.}
% \end{ex}

% \begin{ex}%[DCHT Toán 11 - KNTT -Nguyễn Thành Nhân]%[1K5BG-7]
% 	Hàm số $f(x)$ liên tục trên đoạn $[2 ; 4]$ và $f(2) \cdot f(4)<0$. Khẳng định nào sau đây đúng?
% 	\choice
% 	{\True Phương trình $f(x)=0$ có nghiệm}
% 	{Phương trình $f(x)=0$ vô số nghiệm}
% 	{Phương trình $f(x)=0$ có ít nhất $6$ nghiệm}
% 	{Phương trình $f(x)=0$ vô nghiệm}
% 	\loigiai{
% 		Hàm số $f(x)$ liên tục trên đoạn $[2 ; 4]$ và $f(2) \cdot f(4)<0$ nên phương trình $f(x)=0$ có ít nhất một nghiệm thuộc $(2;4)$.
% 	}
% \end{ex}

% \begin{ex}%[DCHT Toán 11 - KNTT -Nguyễn Thành Nhân]%[1K5BG-7]
% 	Cho phương trình $2x^3-10x-7=0\quad (1)$. Mệnh đề nào sau đây là mệnh đề \textbf{sai}?
% 	\choice
% 	{\True Phương trình $(1)$ không có nghiệm trên khoảng $(0;+\infty)$}
% 	{Hàm số $f(x)=2x^3-10x-7$ liên tục trên $\mathbb{R}$}
% 	{Phương trình $(1)$ có ít nhất hai nghiệm trên khoảng $(-1;3)$}
% 	{Phương trình $(1)$ có ít nhất một nghiệm trên khoảng $(0;3)$}
% 	\loigiai{
% 		Hàm số $f(x)=2x^3-10x-7$ là hàm số đa thức nên nó liên tục trên $\mathbb{R}$. Vì thế, nó cũng liên tục trên mỗi đoạn $[0;3]$, $[-1;3]$.\\
% 		Vì $f(-1)=1$ và $f(0)=-7$ nên $f(-1)\cdot f(0)<0$. Vì thế, phương trình $2x^3-10-7=0$ có ít nhất một nghiệm thuộc khoảng $(-1;0)$.\\
% 		Vì $f(0)=-7$ và $f(3)=17$ nên $f(0)\cdot f(3)<0$. Vì thế, phương trình $2x^3-10-7=0$ có ít nhất một nghiệm thuộc khoảng $(0;3)$.\\
% 		Do đó, phương trình $2x^3-10x-7=0$ có ít nhất hai nghiệm thuộc khoảng $(-1;3)$.
% 	}
% \end{ex}

% \begin{ex}%[DCHT Toán 11 - KNTT -Nguyễn Thành Nhân]%[1K5BG-7]
% 	Phương trình $2x^3-6x+1=0$ có bao nhiêu nghiệm phân biệt thuộc $(-2;2)$?
% 	\choice 
% 	{\True $3$}
% 	{$1$}
% 	{$0$}
% 	{$2$}
% 	\loigiai{
% 		Xét hàm số $f(x)=2x^3-6x+1$ liên tục trên đoạn $[-2;2]$.\\
% 		Ta có
% 		\[f(-2)=-3,\quad f(0)=1,\quad f(1)=-3,\quad f(2)=5.\]
% 		Suy ra $f(-2)f(0)<0$, $f(0)f(1)<0$, $f(0)f(2)<0$.\\ 
% 		Từ đó suy ra phương trình đã cho có ít nhất $3$ nghiệm phân biệt thuộc khoảng $(-2;2)$. Mặt khác, do phương trình đã cho là phương trình bậc ba nên nó có đúng $3$ nghiệm phân biệt thuộc khoảng $(-2;2)$.
% 	}
% \end{ex}

% \begin{ex}%[DCHT Toán 11 - KNTT -Nguyễn Thành Nhân]%[1K5BG-7]
% 	Cho hàm số $f(x)=-4x^3+4 x-1$. Mệnh đề nào sau đây \textbf{sai}?
% 	\choice
% 	{Hàm số đã cho liên tục trên $\mathbb{R}$}
% 	{\True Phương trình $f(x)=0$ không có nghiệm trên khoảng $(-\infty ; 1)$}
% 	{Phương trình $f(x)=0$ có nghiệm trên khoảng $(-2 ; 0)$}
% 	{Phương trình $f(x)=0$ có ít nhất hai nghiệm trên khoảng $\left(-3 ; \dfrac{1}{2}\right)$}
% 	\loigiai{
% 		\begin{itemize}
% 			\item Hàm $f$ là hàm đa thức nên liên tục trên $\mathbb{R}$. Nên mệnh đề \lq\lq Hàm số đã cho liên tục trên $\mathbb{R}$\rq\rq\; đúng.
% 			\item Ta có $\heva{&f(-1)=-1<0 \\& f(-2)=23>0} \Rightarrow f(x)=0$ có nghiệm $x_1$ trên $(-2 ; -1)$, mà
% 			$(-2 ;-1) \subset(-2 ; 0) \subset(-\infty ; 1)$.\\
% 			Nên mệnh đề \lq\lq Phương trình $f(x)=0$ không có nghiệm trên khoảng $(-\infty ; 1)$\rq\rq\; sai và mệnh đề \lq\lq Phương trình $f(x)=0$ có nghiệm trên khoảng $(-2 ; 0)$\rq\rq\; đúng.
% 			\item Ta có $\heva{&f(-1)=-1<0 \\& f(-2)=23>0} \Rightarrow f(x)=0$ có nghiệm $x_1$ trên $(-2 ; -1)$, và
% 			$\heva{&f(-1)=-1 \\& f\left(\dfrac{1}{2}\right)=\dfrac{1}{2}} \Rightarrow f(-1) \cdot f\left(\dfrac{1}{2}\right)<0$\\
% 			$\Rightarrow f(x)=0 \text { có nghiệm } x_2 \text { trên }\left(-1 ; \dfrac{1}{2}\right)$.\\
% 			Nên mệnh đề \lq\lq Phương trình $f(x)=0$ có ít nhất hai nghiệm trên khoảng $\left(-3 ; \dfrac{1}{2}\right)$\rq\rq\; đúng.
% 		\end{itemize}
% 	}
% \end{ex}

% \begin{ex}%[DCHT Toán 11 - KNTT -Nguyễn Thành Nhân]%[1K5BG-7]
% 	Tìm khẳng định \textbf{sai} trong các khẳng định sau.
% 	\choice
% 	{\True Hàm số $f(x)=\dfrac{1}{x}$ có $f(-1)\cdot f(1)<0$ nên phương trình $f(x)=0$ có nghiệm trong $(-1;1)$}
% 	{Phương trình $-x^3-2x^2-3x+4=0$ có ít nhất một nghiệm}
% 	{Hàm số $f(x)=-x^3-2x^2-3x+4$ liên tục trên $\mathbb{R}$}
% 	{Hàm số $f(x)=-x^3-2x^2-3x+4$ có $f(0)\cdot f(1)<0$ nên phương trình $f(x)=0$ có ít nhất một nghiệm trong khoảng $(0;1)$}
% 	\loigiai{
% 		Hàm số $f(x)=\dfrac{1}{x}$ không xác định tại $x=0$ nên không liên tục trên $[0;1]$, do đó khẳng định \lq\lq$f(-1)\cdot f(1)<0$ nên phương trình $f(x)=0$ có nghiệm trong $(-1;1)$\rq\rq\, là sai.\\
% 		Hàm số $f(x)=-x^3-2x^2-3x+4$ liên tục trên $\mathbb{R}$ nên liên tục trên $[0;1]$, có $f(0)\cdot f(1)=-8<0$ nên phương trình $f(x)=0$ có ít nhất một nghiệm trên $(0;1)$, suy ra phương trình $f(x)=0$ có ít nhất một nghiệm.}
% \end{ex}
% \begin{ex}%[DCHT Toán 11 - KNTT -Nguyễn Thành Nhân]%[1K5BG-7]
% 	Cho phương trình $2x^3+x^2-1=0$ $(1)$. Mệnh đề nào dưới đây đúng?
% 	\choice
% 	{Phương trình $(1)$ vô nghiệm trên khoảng $(0 ; 2)$}
% 	{\True Phương trình $(1)$ có ít nhất một nghiệm trên khoảng $(0 ; 2)$}
% 	{Phương trình $(1)$ có đúng 4 nghiệm trên khoảng $(0 ; 2)$}
% 	{Phương trình $(1)$ có vô số nghiệm trên khoảng $(0 ; 2)$}
% 	\loigiai{
% 		Đặt $f(x)=2x^3+x^2-1=0$ là hàm liên tục trên $[0;2]$.\\
% 		Có $f(0)=-1$, $f(2)=19$ nên $f(0)\cdot f(2)<0$, vậy phương trình  $(1)$ có ít nhất một nghiệm trên khoảng $(0 ; 2)$.
% 	}
% \end{ex}

% \begin{ex}%[DCHT Toán 11 - KNTT -Nguyễn Thành Nhân]%[1K5KG-7]
% 	Xét phương trình sau trên tập số thực $x^3+x=a \quad$(1). Chọn khẳng định đúng trong các khẳng định dưới đây?
% 	\choice
% 	{Phương trình $(1)$ chỉ có nghiệm khi $x>a$}
% 	{Phương trình $(1)$ vô nghiệm khi $x \geq a$}
% 	{Phương trình $(1)$ chỉ có nghiệm khi $x \geq a$}
% 	{\True Phương trình $(1)$ có nghiệm $\forall a \in \mathbb{R}$}
% 	\loigiai{
% 		Xét $f(x)=x^3+x-a$ liên tục trên $\mathbb{R}$.\\
% 		Với mọi $a$ ta có $\lim \limits_{x \to -\infty} f(x)=-\infty$ và $\lim \limits_{x \to +\infty} f(x)=+\infty$.\\
% 		Do đó tồn tại các số $m$, $n$ sao cho $f(m)>0$ và $f(n)<0$, tức là $f(m)\cdot f(n)<0$.\\
% 		Vậy $f(x)$ luôn có nghiệm với mọi $a\in\mathbb{R}$.
% 	}
% \end{ex}

% \begin{ex}%[DCHT Toán 11 - KNTT -Nguyễn Thành Nhân]%[1K5KG-7]
% 	Cho phương trình $-4x^3+4x-1=0$. Tìm khẳng định \textbf{sai} trong các khẳng định sau.
% 	\choice
% 	{Phương trình đã cho có ba nghiệm phân biệt}
% 	{\True Phương trình đã cho chỉ có một nghiệm trong khoảng $(0;1)$}
% 	{Phương trình đã cho có ít nhất một nghiệm trong khoảng $(-2;0)$}
% 	{Phương trình đã cho có ít nhất một nghiệm trong khoảng $\left(-\dfrac{1}{2};\dfrac{1}{2}\right)$}
% 	\loigiai{
% 		Đặt $f(x)=-4x^3+4x-1$, $f(x)$ là hàm đa thức nên xác định và liên tục trên $\mathbb{R}$.\\
% 		Ta có $f(-2)=23$, $f(0)=-1$, $f\left(\dfrac{1}{2}\right)=\dfrac{1}{2}$, $f(1)=-1$, $f\left(-\dfrac{1}{2}\right)=-\dfrac{5}{2}$.\\
% 		Suy ra $\heva{&f(-2)\cdot f(0)<0\\&f(0)\cdot f\left(\dfrac{1}{2}\right)<0\\&f\left(\dfrac{1}{2}\right)\cdot f(1)<0.}$\\
% 		Do đó phương trình $f(x)=0$ có ít nhất một nghiệm trong khoảng $(-2;0)$, có ít nhất một nghiệm trong khoảng $\left(0;\dfrac{1}{2}\right)$, có ít nhất một nghiệm trong khoảng $\left(\dfrac{1}{2};1\right)$.\\
% 		Khi đó phương trình $f(x)=0$ có ít nhất hai nghiệm trong khoảng $(0;1)$ và phương trình $f(x)=0$ có $3$ nghiệm phân biệt.\\
% 		Mặt khác $f\left(-\dfrac{1}{2}\right)\cdot f\left(\dfrac{1}{2}\right)<0$ nên phương trình $f(x)=0$ có ít nhất một nghiệm trong khoảng $\left(-\dfrac{1}{2};\dfrac{1}{2}\right)$.\\
% 		Vậy khẳng định ``Phương trình đã cho chỉ có một nghiệm trong khoảng $(0;1)$'' là sai.
% 	}
% \end{ex}

% \begin{ex}%[DCHT Toán 11 - KNTT -Nguyễn Thành Nhân]%[1K5KG-7]
% 	Phương trình nào sau đây có nghiệm trong khoảng $(0;1)$?
% 	\choice
% 	{$(x-1)^5-x^9-2=0$}
% 	{$3x^4-4x^2+5=0$}
% 	{\True $3x^{2019}-8x+4=0$}
% 	{$2x^2-3x+4=0$}
% 	\loigiai{
% 		\begin{itemize}
% 			\item Hàm số $f(x)=3x^{2019}-8x+4$ liên tục trên đoạn $[0;1]$ và có $f(0)\cdot f(1)=4\cdot(-1)<0$ nên phương trình $f(x)=0$ có nghiệm trên khoảng $(0;1)$.
% 			\item Hàm số $g(x)=3x^4-4x^2+5=3\left(x^2-\dfrac{2}{3}\right)^2+\dfrac{11}{3} \geq \dfrac{11}{3}>0,\forall x$ nên phương trình $g(x)=0$ vô nghiệm.
% 			\item Hàm số $h(x)=2x^2-3x+4=2\left(x-\dfrac{3}{4}\right)^2+\dfrac{23}{8}>0,\forall x$ nên phương trình $h(x)=0$ vô nghiệm.
% 			\item Hàm số $p(x)=(x-1)^5-x^9-2$ trên $(0;1)$ có $-1<(x-1)^5<0 ;-1<-x^9<0$
% 			$ \Rightarrow p(x)<0,\forall x \in (0;1)$ nên trên $(0;1)$ phương trình $p(x)=0$ vô nghiệm.
% 		\end{itemize}		
% 	}
% \end{ex}
% \begin{ex}%[DCHT Toán 11 - KNTT -Nguyễn Thành Nhân]%[1K5KG-7]
% 	Cho phương trình $\left(3m^2-m-2\right)x^{2024}\cdot \left(x^{2023}+1\right)+2 x-1=0$. Tìm tất cả các giá trị của $m$ để phương trình có nghiệm.
% 	\choice
% 	{$m\in\mathbb{R}\setminus\left\{1;-\dfrac{2}{3}\right\}$}
% 	{\True $\forall m\in\mathbb{R}$}
% 	{$m=1; m=-\dfrac{2}{3}$}
% 	{$\hoac{&m<0\\& m>1}$}
% 	\loigiai{
% 		Xét hàm số $y=\left(3m^2-m-2\right)x^{2024}\cdot \left(x^{2023}+1\right)+2 x-1$.
% 		\begin{itemize}
% 			\item \textbf{Trường hợp $1$:} $3m^2-m-2>0$, ta có $\lim\limits_{x\rightarrow +\infty} y=+\infty$, $\lim\limits_{x\rightarrow -\infty} y=-\infty$ suy ra phương trình $y=0$ có nghiệm.
% 			\item \textbf{Trường hợp $2$:} $3m^2-m-2<0$, ta có $\lim\limits_{x\rightarrow +\infty} y=-\infty$, $\lim\limits_{x\rightarrow -\infty} y=+\infty$ suy ra phương trình $y=0$ có nghiệm.
% 			\item \textbf{Trường hợp $3$}: $3m^2-m-2=0$, khi đó phương trình đã cho có nghiệm $x=\dfrac{1}{2}$.
% 		\end{itemize}
% 		Vậy phương trình đã cho luôn có nghiệm $\forall m\in\mathbb{R}$.
% 	}
% \end{ex}
% \begin{ex}%[DCHT Toán 11 - KNTT -Nguyễn Thành Nhân]%[1K5KG-7]
% 	Cho phương trình $x^4-3x^3+x-\dfrac{1}{8}=0\quad(1)$. Chọn khẳng định đúng.
% 	\choice
% 	{Phương trình $(1)$ có đúng ba nghiệm trên khoảng $(-1;3)$}
% 	{\True Phương trình $(1)$ có đúng bốn nghiệm trên khoảng $(-1;3)$}
% 	{Phương trình $(1)$ có đúng hai nghiệm trên khoảng $(-1;3)$}
% 	{Phương trình $(1)$ có đúng một nghiệm trên khoảng $(-1;3)$}
% 	\loigiai{
% 		Hàm số $f(x)=x^4-3x^3+x-\dfrac{1}{8}$ liên tục trên đoạn $[-1;3]$ và có $f(-1)=\dfrac{23}{8}$;$f\left(-\dfrac{1}{2}\right)=-\dfrac{3}{16}$; $f\left(\dfrac{1}{2}\right)=\dfrac{1}{16}$;  $f\left(1\right)=-\dfrac{9}{8}$; $f\left(3\right)=\dfrac{23}{8}$.\\
% 		Suy ra trên mỗi khoảng $\left(-1;-\dfrac{1}{2}\right)$; $\left(-\dfrac{1}{2};\dfrac{1}{2}\right)$; $\left(\dfrac{1}{2};1\right)$ ; $\left(1;3\right)$ phương trình có ít nhất một nghiệm.\\ Mặt khác phương trình là bậc $4$ nên có không quá $4$ nghiệm.\\
% 		Vậy phương trình $(1)$ có đúng bốn nghiệm trên khoảng $(-1;3)$.
% 	}
% \end{ex}
% \begin{ex}%[DCHT Toán 11 - KNTT -Nguyễn Thành Nhân]%[1K5KG-7]
% 	Cho phương trình $x^3+ax^2+bx+c=0$\qquad $(1)$ trong đó $a$, $b$, $c$ là các tham số thực. Chọn khẳng định đúng trong các khẳng định sau.
% 	\choice
% 	{\True Phương trình $(1)$ có ít nhất một nghiệm với mọi $a$, $b$, $c$}
% 	{Phương trình $(1)$ có ít nhất hai nghiệm với mọi $a$, $b$, $c$}
% 	{Phương trình $(1)$ có ít nhất ba nghiệm với mọi $a$, $b$, $c$}
% 	{Phương trình $(1)$ vô nghiệm với mọi $a$, $b$, $c$}
% 	\loigiai{
% 		Xét hàm số $f(x)=x^3+ax^2+bx+c$ liên tục trên $\mathbb{R}$.\\
% 		Ta có $\lim\limits_{x\to -\infty} f(x)=-\infty$ và $\lim\limits_{x\to +\infty} f(x)=+\infty$.\\
% 		Do đó phương trinh $f(x)=0$ có ít nhất một nghiệm với mọi $a$, $b$, $c$.
% 	}
% \end{ex}

% \begin{ex}%[DCHT Toán 11 - KNTT -Nguyễn Thành Nhân]%[1K5KG-7]
% 	Khẳng định nào trong các khẳng định sau đây là \textbf{sai}?
% 	\choice
% 	{Phương trình $x^4+mx^2-2mx-2=0$ luôn có nghiệm với mọi giá trị của tham số $m$}
% 	{\True Phương trình $3x^6-3x^3+5x-2=0$ \textbf{không} có nghiệm thuộc khoảng $(-2;2)$}
% 	{Phương trình $x^3-3x+1=0$ có ba nghiệm phân biệt}
% 	{Phương trình $m(x-1)^2(x-2)+2x-3=0$ luôn có nghiệm với mọi giá trị của tham số $m$}
% 	\loigiai{
% 		\begin{itemize}
% 			\item Hàm số $f(x)=x^4+mx^2-2mx-2$ liên tục trên đoạn $[0;2]$, có $f(0)\cdot f(2) = -2\cdot 14 = -28<0$ nên phương trình $f(x)=0$ có ít nhất một nghiệm thuộc khoảng $(0;2)$ với mọi $m$ hay phương trình $x^4+mx^2-2mx-2=0$ luôn có nghiệm với mọi giá trị của tham số $m$.
% 			\item Hàm số $g(x)=3x^6-3x^3+5x-2$ liên tục trên đoạn $[0;1]$, có $g(0)\cdot g(1) = -2\cdot 3 =-6<0$ nên phương trình $g(x)=0$ có ít nhất một nghiệm thuộc khoảng $(0;1)$ hay phương trình $g(x)=0$ có nghiệm thuộc khoảng $(-2;2)$.
% 			\item Hàm số $h(x)=x^3-3x+1$ liên tục trên các đoạn $[-2;0]$, $[0;1]$, $[1;2]$ và có $h(-2)=-1$, $h(0)=1$, $h(1)=-1$, $h(2)=3$.\\
% 			Do $h(-2)\cdot h(0)<0$, $h(0)\cdot h(1)<0$, $h(1)\cdot h(2)<0$ nên phương trình $h(x)=0$ có ba nghiệm $x_1\in (-2;0)$, $x_2\in (0;1)$, $x_3\in (1;2)$ là ba nghiệm phân biệt.\\
% 			Mặt khác phương trình $h(x)=0$ là phương trình bậc ba nên có không quá ba nghiệm.\\
% 			Vậy phương trình $h(x)=0$ có ba nghiệm phân biệt.
% 			\item Hàm số $p(x)=m(x-1)^2(x-2)+2x-3$ liên tục trên đoạn $[1;2]$, có $p(1)\cdot p(2) = (-1)\cdot 1 = -1<0$ nên phương trình $p(x)=0$ luôn có nghiệm thuộc khoảng $(1;2)$ với mọi $m$ hay phương trình $m(x-1)^2(x-2)+2x-3=0$ luôn có nghiệm với mọi giá trị của tham số $m$.
% 		\end{itemize}
% 	}
% \end{ex}

% \begin{ex}%[DCHT Toán 11 - KNTT -Nguyễn Thành Nhân]%[1K5KG-7]
% 	Cho phương trình $x^4-4x^3+1=0$. Tìm mệnh đề \textbf{sai} trong các mệnh đề sau
% 	\choice
% 	{Phương trình có đúng một nghiệm $x>3$}
% 	{\True Phương trình vô nghiệm trên khoảng $(0;1)$}
% 	{Phương trình có ít nhất hai nghiệm}
% 	{Phương trình vô nghiệm trên khoảng $(-1;0)$}
% 	\loigiai{
% 		Xét hàm số $f(x)=x^4-4x^3+1$.\\
% 		Hàm số liên tục trên $\mathbb{R}$.\\
% 		Ta có $f(0)=1>0$ và $f(1)=-2<0$.\\
% 		Do $f(0)\cdot f(1)=-2<0$ và hàm số liên tục trên đoạn $[0;1]$ nên phương trình $f(x)=0$ có ít nhất một nghiệm $x_0\in (0;1)$.\\
% 		Vậy mệnh đề \lq\lq Phương trình vô nghiệm trên khoảng $(0;1)$\rq\rq\, là sai.	
% 	}
% \end{ex}

% \begin{ex}%[DCHT Toán 11 - KNTT -Nguyễn Thành Nhân]%[1K5KG-7]
% 	Xét phương trình sau trên tập số thực $x^{2023}+x=a \quad(1)$. Chọn khẳng định đúng trong các khẳng định dưới đây. 
% 	\choice
% 	{Phương trình $(1)$ chỉ có nghiệm khi $a>0$}
% 	{Phương trình $(1)$ chỉ có nghiệm khi $a<0$}
% 	{Phương trình $(1)$ vô nghiệm khi $a\geq 0$}
% 	{\True Phương trình $(1)$ có nghiệm $\forall a\in \mathbb{R}$}
% 	\loigiai
% 	{Phương trình $x^{2023}+x=a \Leftrightarrow x^{2023}+x-a=0 \quad(2)$.\\
% 		Xét hàm số $f(x)=x^{2023}+x-1$ trên $\mathbb{R}$, ta có
% 		\begin{itemize}
% 			\item $\lim\limits_{x\to -\infty}f(x)=\lim\limits_{x\to -\infty}\left(x^{2023}+x-a\right)=-\infty$ nên tồn tại số $\alpha<0$ để $f(\alpha)<0$.
% 			\item $\lim\limits_{x\to +\infty}f(x)=\lim\limits_{x\to +\infty}\left(x^{2023}+x-a\right)=+\infty$ nên tồn tại số $\beta>0$ để $f(\beta)>0$.
% 		\end{itemize}
% 		Hàm số $f(x)$ liên tục trên $[\alpha;\beta]$, có $f(\alpha)\cdot f(\beta)<0$ nên luôn tồn tại $c\in(\alpha;\beta)$ sao cho $f(c)=0$.\\
% 		Vậy phương trình $f(x)=0$ luôn có nghiệm với mọi $a\in\mathbb{R}$.}
% \end{ex}

% \begin{ex}%[DCHT Toán 11 - KNTT -Nguyễn Thành Nhân]%[1K5KG-7]
% 	Tìm tất cả các giá trị của tham số $m$ sao cho phương trình $(m^2-5m+3)x^5-2x^2+1=0$ có ít nhất một nghiệm thuộc khoảng $(-1;0)$.
% 	\choice
% 	{$m\in\left(-\infty;1\right)\cup\left(4;+\infty\right)$}
% 	{$m\in\mathbb{R}$}
% 	{\True $m\in\left(1;4\right)$}
% 	{$m\in\left(-\infty;1 \right]\cup\left[4;+\infty \right) $}
% 	\loigiai
% 	{
% 		Đặt $f(x)=(m^2-5m+3)x^5-2x^2+1$.\\
% 		Phương trình $(m^2-5m+3)x^5-2x^2+1=0$ có ít nhất một nghiệm thuộc khoảng $(-1;0)$
% 		\begin{eqnarray*}
% 			\Leftrightarrow f(-1)\cdot f(0)<0\Leftrightarrow (-m^2+5m-4)\cdot 1<0\Leftrightarrow 1<m<4.		
% 		\end{eqnarray*}
% 		Vậy $1<m<4$ thỏa yêu cầu bài toán.
% 	}
% \end{ex}


% \begin{ex}%[DCHT Toán 11 - KNTT -Nguyễn Thành Nhân]%[1K5KG-7]
% 	Cho phương trình $2x^4-5x^2+x+1=0$\quad$(1)$. Tìm mệnh đề đúng trong các mệnh đề sau:
% 	\choice
% 	{Phương trình $(1)$ không có nghiệm trong khoảng $(-2;0)$}
% 	{\True Phương trình $(1)$ có ít nhất $2$ nghiệm trong khoảng $(0;2)$}
% 	{Phương trình $(1)$ không có nghiệm trong khoảng $(-1;1)$}
% 	{Phương trình $(1)$ chỉ có $1$ nghiệm trong khoảng $(-2;1)$}
% 	\loigiai{
% 		Xét hàm số $f(x)=2x^4-5x^2+x+1$. Đây là hàm đa thức nên liên tục trên $\mathbb{R}$, do đó nó liên tục trên các đoạn $\left[0;1\right]$ và $\left[1;2\right]$.\\
% 		Ta có:
% 		$f(0)=1$; $f(1)=-1$; $f(2)=15$.\\
% 		$f(0)\cdot f(1)<0$ nên phương trình $f(x)=0$ có ít nhất một nghiệm $x_1\in(0;1)$\hfill$(1)$\\
% 		$f(1)\cdot f(2)<0$ nên phương trình $f(x)=0$ có ít nhất một nghiệm $x_2\in(1;2)$\hfill$(2)$\\
% 		Từ $(1)$ và $(2)$ suy ra phương trình $f(x)=0$ có ít nhất hai nghiệm trong khoảng $(0;2)$.
% 	}
% \end{ex}
% \begin{ex}%[DCHT Toán 11 - KNTT -Nguyễn Thành Nhân]%[1K5GG-7]
% 	Tập tất cả các giá trị của tham số thực $m$ để phương trình\\ $\left(2m^2-5m+2\right)\left(x-1\right)^{2021}\left(x^{2020}-2\right)+2x+3=0$ có nghiệm là
% 	\choice
% 	{$m\in \mathbb{R}\setminus \left\lbrace \dfrac{1}{2};2\right\rbrace$}
% 	{$m\in \left\lbrace \dfrac{1}{2};2\right\rbrace$}
% 	{$m\in \left(-\infty;\dfrac{1}{2}\right)\cup(2;+\infty)$}
% 	{\True $m\in \mathbb{R}$}
% 	\loigiai{
% 		Đặt $f(x)=\left(2m^2-5m+2\right)\left(x-1\right)^{2021}\left(x^{2020}-2\right)+2x+3$.
% 		\begin{itemize}
% 			\item Với $2m^2-5m+2=0\Leftrightarrow\hoac{&m=\dfrac{1}{2}\\&m=2}$ phương trình trên trở thành $2x+3=0\Leftrightarrow x=\dfrac{-3}{2}$.\\
% 			Vậy với $m\in \left\lbrace \dfrac{1}{2};2\right\rbrace$ phương trình có nghiệm $\qquad(*)$
% 			\item Với $2m^2-5m+2\ne0\Leftrightarrow m\in \mathbb{R}\setminus \left\lbrace \dfrac{1}{2};2\right\rbrace$, $f(x)$ là đa thức bậc lẻ có TXĐ $\mathscr{D}=\mathbb{R}$.\\
% 			Khi đó $f(x)$ liên tục trên $\mathbb{R}$.		
% 			\begin{itemize}
% 				\item Với $m\in\left(-\infty;\dfrac{1}{2}\right)\cap\left(2;+\infty\right)\Rightarrow 2m^2-5m+2>0$\\
% 				$\lim\limits_{x\rightarrow -\infty}f(x)=-\infty \Rightarrow \exists x_1<0\colon f(x_1)<0\qquad(1)$\\
% 				$\lim\limits_{x\rightarrow +\infty}f(x)=+\infty \Rightarrow \exists x_2>0\colon f(x_2)>0\qquad(2)$\\
% 				Từ $(1)$ và $(2)$ suy ra $f(x)$ luôn có ít nhất $1$ nghiệm $x_0\in\left(-\infty;+\infty\right)\qquad(**)$.
% 				\item Với $m\in\left(\dfrac{1}{2};2\right)\Rightarrow 2m^2-5m+2<0$\\
% 				$\lim\limits_{x\rightarrow -\infty}f(x)=+\infty \Rightarrow \exists x_3<0\colon f(x_3)>0\qquad(3)$\\
% 				$\lim\limits_{x\rightarrow +\infty}f(x)=-\infty \Rightarrow \exists x_4>0\colon f(x_4)<0\qquad(4)$\\
% 				Từ $(3)$ và $(4)$ suy ra $f(x)$ luôn có ít nhất $1$ nghiệm $x_0\in\left(-\infty;+\infty\right)\qquad(***)$.
% 			\end{itemize}
% 		\end{itemize}	
% 		Từ $(*),\;(**)$ và $(***)$ suy ra phương trình luôn có nghiệm với $m\in\mathbb{R}$.
% 	}
% \end{ex}
\Closesolutionfile{ans}
% \begin{indapan}{10}
% 	{ans/ans-1K5-3-Dang6}
% \end{indapan}

% ---------Mục lục chính
\FULLWIDTH
% \def\tenchude{DÃY SỐ}
\tableofcontents 

% \vfill

\begin{center}
	\includegraphics[width=5cm]{QRcode/11D3.png}
\end{center}
 %lệnh in mục lục chính

\end{document}